<!DOCTYPE html PUBLIC "-//W3C//DTD XHTML 1.0 Transitional//EN" "http://www.w3.org/TR/xhtml1/DTD/xhtml1-transitional.dtd">
<html xmlns="http://www.w3.org/1999/xhtml">
<head>
<meta http-equiv="Content-Type" content="text/html; charset=utf-8" />
<meta http-equiv="X-UA-Compatible" content="IE=Edge,chrome=1">
<title>資治通鑒_142-資治通鑑卷一百四十一_142-資治通鑑卷一百四十一</title>
<meta name="Keywords" content="資治通鑒_142-資治通鑑卷一百四十一_142-資治通鑑卷一百四十一">
<meta name="Description" content="資治通鑒_142-資治通鑑卷一百四十一_142-資治通鑑卷一百四十一">
<meta http-equiv="Cache-Control" content="no-transform" />
<meta http-equiv="Cache-Control" content="no-siteapp" />
<link href="/img/style.css" rel="stylesheet" type="text/css" />
<script src="/img/m.js?2020"></script> 
</head>
<body>
 <div class="ClassNavi">
<a  href="/24shi/">二十四史</a> | <a href="/SiKuQuanShu/">四库全书</a> | <a href="http://www.guoxuedashi.com/gjtsjc/"><font  color="#FF0000">古今图书集成</font></a> | <a href="/renwu/">历史人物</a> | <a href="/ShuoWenJieZi/"><font  color="#FF0000">说文解字</a></font> | <a href="/chengyu/">成语词典</a> | <a  target="_blank"  href="http://www.guoxuedashi.com/jgwhj/"><font  color="#FF0000">甲骨文合集</font></a> | <a href="/yzjwjc/"><font  color="#FF0000">殷周金文集成</font></a> | <a href="/xiangxingzi/"><font color="#0000FF">象形字典</font></a> | <a href="/13jing/"><font  color="#FF0000">十三经索引</font></a> | <a href="/zixing/"><font  color="#FF0000">字体转换器</font></a> | <a href="/zidian/xz/"><font color="#0000FF">篆书识别</font></a> | <a href="/jinfanyi/">近义反义词</a> | <a href="/duilian/">对联大全</a> | <a href="/jiapu/"><font  color="#0000FF">家谱族谱查询</font></a> | <a href="http://www.guoxuemi.com/hafo/" target="_blank" ><font color="#FF0000">哈佛古籍</font></a> 
</div>

 <!-- 头部导航开始 -->
<div class="w1180 head clearfix">
  <div class="head_logo l"><a title="国学大师官网" href="http://www.guoxuedashi.com" target="_blank"></a></div>
  <div class="head_sr l">
  <div id="head1">
  
  <a href="http://www.guoxuedashi.com/zidian/bujian/" target="_blank" ><img src="http://www.guoxuedashi.com/img/top1.gif" width="88" height="60" border="0" title="部件查字,支持20万汉字"></a>


<a href="http://www.guoxuedashi.com/help/yingpan.php" target="_blank"><img src="http://www.guoxuedashi.com/img/top230.gif" width="600" height="62" border="0" ></a>


  </div>
  <div id="head3"><a href="javascript:" onClick="javascript:window.external.AddFavorite(window.location.href,document.title);">添加收藏</a>
  <br><a href="/help/setie.php">搜索引擎</a>
  <br><a href="/help/zanzhu.php">赞助本站</a></div>
  <div id="head2">
 <a href="http://www.guoxuemi.com/" target="_blank"><img src="http://www.guoxuedashi.com/img/guoxuemi.gif" width="95" height="62" border="0" style="margin-left:2px;" title="国学迷"></a>
  

  </div>
</div>
  <div class="clear"></div>
  <div class="head_nav">
  <p><a href="/">首页</a> | <a href="/ShuKu/">国学书库</a> | <a href="/guji/">影印古籍</a> | <a href="/shici/">诗词宝典</a> | <a   href="/SiKuQuanShu/gxjx.php">精选</a> <b>|</b> <a href="/zidian/">汉语字典</a> | <a href="/hydcd/">汉语词典</a> | <a href="http://www.guoxuedashi.com/zidian/bujian/"><font  color="#CC0066">部件查字</font></a> | <a href="http://www.sfds.cn/"><font  color="#CC0066">书法大师</font></a> | <a href="/jgwhj/">甲骨文</a> <b>|</b> <a href="/b/4/"><font  color="#CC0066">解密</font></a> | <a href="/renwu/">历史人物</a> | <a href="/diangu/">历史典故</a> | <a href="/xingshi/">姓氏</a> | <a href="/minzu/">民族</a> <b>|</b> <a href="/mz/"><font  color="#CC0066">世界名著</font></a> | <a href="/download/">软件下载</a>
</p>
<p><a href="/b/"><font  color="#CC0066">历史</font></a> | <a href="http://skqs.guoxuedashi.com/" target="_blank">四库全书</a> |  <a href="http://www.guoxuedashi.com/search/" target="_blank"><font  color="#CC0066">全文检索</font></a> | <a href="http://www.guoxuedashi.com/shumu/">古籍书目</a> | <a   href="/24shi/">正史</a> <b>|</b> <a href="/chengyu/">成语词典</a> | <a href="/kangxi/" title="康熙字典">康熙字典</a> | <a href="/ShuoWenJieZi/">说文解字</a> | <a href="/zixing/yanbian/">字形演变</a> | <a href="/yzjwjc/">金 文</a> <b>|</b>  <a href="/shijian/nian-hao/">年号</a> | <a href="/diming/">历史地名</a> | <a href="/shijian/">历史事件</a> | <a href="/guanzhi/">官职</a> | <a href="/lishi/">知识</a> <b>|</b> <a href="/zhongyi/">中医中药</a> | <a href="http://www.guoxuedashi.com/forum/">留言反馈</a>
</p>
  </div>
</div>
<!-- 头部导航END --> 
<!-- 内容区开始 --> 
<div class="w1180 clearfix">
  <div class="info l">
   
<div class="clearfix" style="background:#f5faff;">
<script src='http://www.guoxuedashi.com/img/headersou.js'></script>

</div>
  <div class="info_tree"><a href="http://www.guoxuedashi.com">首页</a> > <a href="/SiKuQuanShu/fanti/">四库全书</a>
 > <h1>资治通鉴</h1> <!--         下载:【右键另存为】即可 --></div>
  <div class="info_content zj clearfix">
  
<div class="info_txt clearfix" id="show">
<center style="font-size:24px;">142-資治通鑑卷一百四十一</center>
    資治通鑑卷一百四十一 宋 司馬光 撰<br />
<br />
  胡三省 音註<br />
<br />
  齊紀七【起彊圉赤奮若盡著雍攝提格凡二年】<br />
<br />
  高宗明皇帝下<br />
<br />
  建武四年春正月大赦 【考異曰齊帝紀云庚午大赦按長歷是月己丑朔無庚午故不日】 丙申魏立皇子恪為太子魏主宴於清徽堂語及太子恂李冲謝曰臣忝師傅不能輔導帝曰朕尚不能化其惡師傅何謝也 乙巳魏主北巡 初尚書令王晏為世祖所寵任【事見一百三十六卷世祖永明七年】及上謀廢鬱林王晏即欣然推奉【事見一百三十九卷元年】鬱林王已廢上與晏宴於東府語及時事晏抵掌曰公常言晏怯今定何如上即位晏自謂佐命新朝常非薄世祖故事既居朝端【尚書令位居朝臣之右朝直遥翻】事多專决内外要職並用所親每與上爭用人上雖以事際須晏【事際謂舉事之際須者倚其為用】而心惡之【惡烏路翻】嘗料簡世祖中詔【料音聊】得與晏手敕三百餘紙皆論國家事又得晏啟諫世祖以上領選事【見一百三十七卷永明八年選須絹翻】以此愈猜薄之始安王遥光勸上誅晏上曰晏於我有功且未有罪遥光曰晏尚不能為武帝安能為陛下乎【為于偽翻】上默然上遣腹心陳世範等出塗巷採聽異言晏輕淺無防意望開府數呼相工自視【數所角翻相息亮翻】云當大貴與賓客語好屏人請間【好呼到翻屏必郢翻】上聞之疑晏欲反遂有誅晏之意奉朝請鮮于文粲密探上旨【朝直遥翻探吐南翻】告晏有異志世範又啟上云晏謀因四年南郊與世祖故主帥於道中竊【帥所類翻】會虎犯郊壇上愈懼未郊一日有敕停行先報晏及徐孝嗣孝嗣奉旨而晏陳郊祀事大必宜自力上益信世範之言丙辰召晏於華林省誅之【省在華林園因名 考異曰晏傳云元會畢乃召晏誅之本紀丙辰晏伏誅丙辰正月二十八日也按郊禮必在正月既云未郊一日敕停則誅晏必非元會之日也本傳盖言元會禮後耳】并北中郎司馬蕭毅臺隊主劉明達【明達盖世祖時主帥】及晏子德元德和下詔云晏與毅明達以河東王鉉識用微弱謀奉以為主使守虛器晏弟詡為廣州刺使上遣南中郎司馬蕭季敞襲殺之季敞上之從祖弟也【從才用翻下晏從同】蕭毅奢豪好弓馬為上所忌故因事陷之【毅高帝從子新吳侯景先之子也好呼到翻】河東王鉉先以年少才弱故未為上所殺鉉朝見常鞠躬俯僂【少詩照翻朝直遥翻見賢遍翻僂力主翻鞠躬曲身也俯低頭僂曲背】不敢平行直視至是年稍長【長知兩翻】遂坐晏事免官禁不得與外人交通鬱林王之將廢也晏從弟御史中丞思遠謂晏曰兄荷世祖厚恩【荷下可翻】今一旦贊人如此事彼或可以權計相須未知兄將來何以自立若及此引决猶可保全門戶不失後名【欲使之死鬱林之難也】晏曰方噉粥未暇此事及拜驃騎將軍【帝初即位進宴為驃騎大將軍噉徒濫翻又徒覽翻驃匹妙翻騎奇寄翻】集會子弟謂思遠兄思徵曰隆昌之末阿戎勸吾自裁若從其語豈有今日思遠遽應曰如阿戎所見今猶未晩也【晉宋間人多謂從弟為阿戎至唐猶然如杜甫於從弟杜位宅守歲詩云守歲阿戎家是也】思遠知上外待晏厚而内已疑異乘間謂晏曰時事稍異兄亦覺不【間古莧翻不讀曰否】凡人多拙於自謀而巧於謀人晏不應思遠退晏方歎曰世乃有勸人死者旬日而晏敗上聞思遠言故不之罪仍遷侍中晏外弟尉氏阮孝緒亦知晏必敗【尉氏縣漢屬陳留江左僑置於今六合縣界屬秦郡阮氏本尉氏人此時未必居秦郡界外弟妻弟也】晏屢至其門逃匿不見嘗食醬美問知得於晏家吐而覆之【既吐其所食者又覆其所餘者】及晏敗人為之懼【為于偽翻】孝緒曰親而不黨何懼之有卒免於罪【卒子恤翻】 二月壬戌魏主至太原 甲子以左僕射徐孝嗣為尚書令【代王晏也】征虜將軍蕭季敞為廣州刺史【代晏弟詡也】 癸酉魏主至平城引見穆泰陸叡之黨問之【見賢遍翻】無一人稱枉者時人皆服任城王澄之明穆泰及其親黨皆伏誅賜陸叡死於獄宥其妻子徙遼西為民 【考異曰齊書魏虜傳云偽征北將軍恒州刺史鉅鹿孤賀鹿渾守桑乾宏從叔平陽王安夀戍懷柵在桑乾西北渾非宏任用中國人與偽定州刺史馮翊公自鄰安樂公主拓拔阿幹兒謀立安夀分據河北期久不遂安夀懼告宏殺渾等數百人任安夀如故與魏書名姓全不同今從魏書】初魏主遷都變易舊俗并州刺史新興公丕皆所不樂【樂音洛】帝以其宗室耆舊亦不之逼但誘示大理令其不生同異而已【示以事理之大歸而已不反覆告語之誘音酉】及朝臣皆變衣冠朱衣滿坐【朝直遥翻坐徂卧翻】而丕獨胡服於其間晚乃稍加冠帶而不能修飾容儀帝亦不強也【強其兩翻】太子恂自平城將遷洛陽元隆與穆泰等密謀留恂因舉兵斷關規據陘北【陘北即恒朔二州之地關即雁門之東陘西陘二關也斷丁管翻陘音刑】丕在并州隆等以其謀告之丕外慮不成口雖折難【折之列翻難乃旦翻】心頗然之及事覺丕從帝至平城帝每推問泰等常令丕坐觀有司奏元業元隆元超罪當族丕應從坐帝以丕嘗受詔許以不死聽免死為民留其後妻二子與居於太原殺隆超同產乙升【同產同母兄弟】餘子徙敦煌【敦徒門翻】初丕叡與僕射李冲領軍于烈俱受不死之詔叡既誅帝賜冲烈詔曰叡反逆之志自負幽冥違誓在彼不關朕也反逆既異餘犯雖欲矜恕如何可得然猶不忘前言聽自死别府【不就恒州刺史府賜死而死於獄故曰别府】免其孥戮【書甘誓曰予則孥戮汝孔安國注曰孥子也免其孥戮謂叡妻子免死徙遼西也孥音奴】元丕二子一弟首為賊端連坐應死特恕為民朕本期始終而彼自棄絶違心乖念一何可悲故此别示想無致怪謀反之外皎如白日耳冲烈皆上表謝<br />
<br />
  臣光曰夫爵禄廢置殺生予奪人君所以馭臣之大柄也【此周禮所謂八柄馭羣臣者也予讀曰與】是故先王之制雖有親故賢能功貴勤賓苟有其罪不直赦也必議於槐棘之下【此周禮所謂八議也槐棘公卿之位王制獄成大司寇聽之於棘木之下】可赦則赦可宥則宥可刑則刑可殺則殺輕重視情寛猛隨時故君得以施恩而不失其威臣得以免罪而不敢自恃及魏則不然勳貴之臣往往豫許之以不死彼驕而觸罪又從而殺之是以不信之令誘之使陷於死地也【誘音酉】刑政之失無此為大焉<br />
<br />
  是時代鄉舊族多與泰等連謀唯于烈無所染涉帝由是益重之帝以北方酋長及侍子畏暑【酋自秋翻長知兩翻】聽秋朝洛陽春還部落時人謂之鴈臣【以鴈避寒而南來望暖而北還也朝直遥翻】 三月己酉魏主南至離石【離石漢縣屬西河郡隋為離石郡唐為石州】叛胡請降詔宥之【降戶江翻】夏四月庚申至龍門遣使祀夏禹【水經註龍門上口在漢河東北屈縣西所謂孟門也龍門下口在河東皮氏縣西北大禹所鑿故於此祀焉】癸亥至蒲坂祀虞舜【皇甫謐云舜都蒲坂故又於此祀焉坂音反】辛未至長安 魏太子恂既廢頗自悔過御史中尉李彪密表恂復與左右謀逆【復扶又翻】魏主使中書侍郎邢巒與咸陽王禧奉詔齎椒酒詣河陽賜恂死【椒味辛大熱有毒其合口者尤甚漢桓思后之議李咸擣椒自隨帝煮椒二斛以殺高武諸子孫皆是物也】歛以麤棺常服瘞於河陽【歛力贍翻瘞一計翻】 癸未魏大將軍宋明王劉昶卒於彭城葬以殊禮【謚法思慮果遠曰明謂昶遠慮果於違難而歸魏也】 五月己丑魏主東還【還從宣翻又如字】汎渭入河壬辰遣使祀周文王於豐武王於鎬【亦於故都祀之也周之豐鎬漢時悉在上林苑中使疏吏翻】六月庚申還洛陽 壬戌魏冀定瀛相濟五州兵二十萬【魏太宗泰常八年置濟州於濟北碻磝城領濟北平原東平南清河郡相息亮翻濟子禮翻】將入寇 魏穆泰之反也中書監魏郡公穆羆與之通謀赦後事削官爵為民羆弟司空亮以府事付司馬慕容契上表自劾【劾戶槩翻又戶得翻】魏主優詔不許亮固請不己癸亥聽亮遜位 丁卯魏分六師以定行留 秋七月魏立昭儀馮氏為皇后后欲母養太子恪恪母高氏自代如洛陽暴卒於共縣【馮昭儀既譛廢其妹又潜殺太子之母其心盖梟獍也以魏主之明而使之正位椒房他日不死於其手者幸耳共縣自漢以來屬河内郡晉及後魏屬汲郡唐衛州共城縣即其地共音恭】 戊辰魏以穆亮為征北大將軍開府儀同三司冀州刺史八月丙辰魏詔中外戒嚴【將南伐也】 壬戌魏立皇子愉<br />
<br />
  為京兆王懌為清河王懷為廣平王 追尊景皇所生王氏為恭太后【帝即位尊始安貞王曰景皇稱皇不稱帝用漢制也】 甲戌魏講武於華林園庚辰軍洛陽使吏部尚書任城王澄居守【任城王澄至是始為真吏部尚書守式又翻】以御史中丞李彪兼度支尚書【中丞當作中尉度徒洛翻】與僕射李冲參治留臺事【治直之翻】假彭城王勰中軍大將軍【勰音協】勰辭曰親疎並用古之道也臣獨何人頻煩寵授昔陳思求而不允【曹魏文帝時陳思王植上表求自試以攻吳蜀帝不許】愚臣不請而得何否泰之相遠也【天地交曰泰天地不交曰否陳思於魏文上下之情不通故曰否勰則君臣兄弟之情無間故曰泰否皮鄙翻】魏主大笑執勰手曰二曹以才名相忌【二曹謂魏文帝陳思王也】吾與汝以道德相親上遣軍主直閤將軍胡松助北襄城太守成公期戍赭陽【赭音者】軍主鮑舉助西汝南北義陽二郡太守黃瑶起戍舞陰【蕭子顯齊志西汝南屬雍州北義陽屬雍州寧蠻府自宋未有雙頭郡太守率治一處舞陰縣自漢以來屬南陽郡為西汝南北義陽二郡治所】 魏以氐帥楊靈珍為南梁州刺史靈珍舉州來降【魏置梁州於仇池置南梁州於武興帥所類翻降戶江翻下同】送其母及子於南鄭以為質【質音致】遣其弟婆羅阿卜珍將步騎萬餘襲魏武興王楊集始【將即亮翻騎奇寄翻】殺其二弟集同集衆集始窘急請降九月丁酉魏主以河南尹李崇為都督隴右諸軍事將兵數萬討之 初魏遷洛陽荆州刺史薛真度勸魏主先取樊鄧【此時魏荆州猶治魯陽樊鄧逼近洛陽欲先取之以廣封畧】真度引兵寇南陽太守房伯玉擊敗之【此謂去年沙堨之敗也擊敗補邁翻】魏主怒以南陽小郡志必㓕之遂引兵向襄陽彭城王勰等三十六軍前後相繼衆號百萬吹脣沸地【吹脣者以齒齧脣作氣吹之其聲如鷹隼其下者以指夾脣吹之然後有聲謂之嘯指】辛丑魏主留諸將攻赭陽自引兵南下癸卯至宛夜襲其郛克之【宛於元翻郛芳無翻城之外郭曰郛】房伯玉嬰内城拒守魏主遣中書舍人孫延景 【考異曰齊書作公孫雲今從魏書】謂伯玉曰我今蕩壹六合非如曏時冬來春去不有所克終不還北卿此城當我六龍之首【易曰時乘六龍以御天人君之象也】無容不先攻取遠期一年近止一月封侯梟首事在俯仰宜善圖之【梟堅堯翻】且卿有三罪今令卿知卿先事武帝蒙殊常之寵不能建忠致命而盡節於其讐罪一也【明帝夷㓕武帝子孫故謂之讐】頃年薛真度來卿傷我偏師罪二也今鸞輅親臨不面縛麾下罪三也伯玉遣軍副樂稚柔對曰承欲攻圍期於必克卑微常人得抗大威真可謂獲其死所外臣蒙武帝採拔豈敢忘恩但嗣君失德主上光紹大宗【言帝自小宗入為高帝第三子以紹大宗】非唯副億兆之深望抑亦兼武皇之遺敕是以區區盡節不敢失墜往者北師深入寇擾邊民輒厲將士以修職業【將即亮翻】反已而言不應垂責宛城東南隅溝上有橋魏主引兵過之伯玉使勇士數人衣班衣戴虎頭㡌【人衣如既翻虎頭㡌者㡌為虎頭形】伏於竇下突出擊之魏主人馬俱驚召善射者原靈度射之【度射而亦翻】應弦而斃乃得免 李崇槎山分道出氐不意表裏襲之【槎士下翻逆斫木也】羣氐皆棄楊靈珍散歸靈珍之衆减大半崇進據赤土【魏收志南秦州武階郡有赤土縣五代志武都郡覆津縣後魏置武階郡】靈珍遣從弟建屯龍門自帥精勇一萬屯鷲峽【按魏收志東益州武興郡有石門縣五代志武都將利縣舊曰石門又仇池山下有飛龍峽以氐酋楊飛龍據仇池得名又今龍州江油縣東二十里有龍門山又江油縣東百里有石門戍武興今為興州龍州去興州甚遠楊建所屯者必非江油之龍門也水經注仇池東北有龍門戍此其是歟鷲峽當在龍門西南從才用翻帥讀曰率下同鷲音就】龍門之北數十里中伐樹塞路鷲峽之口聚礌石臨崖下之以拒魏兵【塞悉則翻礌盧對翻埤蒼曰推石自高而下也漢書李陵傳乘隅下壘石師古曰言放石以投人因山隅曲而下也壘音盧對翻與此礌音同】崇命統軍慕容拒帥衆五千從它路入夜襲龍門破之崇自攻鷲峽靈珍連戰敗走俘其妻子遂克武興梁州刺史陰廣宗參軍鄭猷等將兵救靈珍崇進擊大破之斬楊婆羅阿卜珍生擒猷等靈珍奔還漢中魏主聞之喜曰使朕無西顧之憂者李崇也以崇為都督梁秦二州諸軍事梁州刺史以安集其地丁未魏主發南陽留太尉咸陽王禧等攻之己酉魏<br />
<br />
  主至新野新野太守劉思忌拒守冬十月丁巳魏軍攻之不克築長圍守之遣人謂城中曰房伯玉已降汝何為獨取糜碎思忌遣人對曰城中兵食猶多未暇從汝小虜語也魏右軍府長史韓顯宗將别軍屯赭陽【右軍府右軍將軍府也將即亮翻下同】成公期遣胡松引蠻兵攻其營【胡松時助戍赭陽】顯宗力戰破之斬其裨將高法援顯宗至新野魏主謂曰卿破賊斬將殊益軍勢朕方攻堅城何為不作露布【五代史志曰後魏每攻戰克捷欲天下聞知迺書帛建於竿上名曰露布魏主謂顯宗若露布上聞行在所則增益魏軍之勝勢可以揺城中堅守之心】對曰頃聞鎮南將軍王肅獲賊二三人驢馬數匹皆為露布臣在東觀私常哂之【韓顯宗對策甲科除著作郎故云在東觀觀古玩翻哂矢引翻笑不壞顔曰哂】近雖仰憑威靈得摧醜虜兵寡力弱擒斬不多脱復高曳長縑虚張功烈尤而效之其罪彌大【左傳曰尤而效之罪又甚焉尤責也過也甚之之辭也復扶又翻】臣所以不敢為之解上而已【丁度集韻解居隘翻聞上也上時掌翻自下而閒於上謂之上】魏主益賢之上詔徐州刺史裴叔業引兵救雍州【雍於用翻下同】叔業啟稱北人不樂遠行唯樂鈔掠【樂音洛】若侵虜境則司雍之寇自然分矣上從之叔業引兵攻虹城【此即漢沛郡之虹縣城也師古曰虹音貢南北兵爭其地在下邳夏丘縣界唐復為虹縣屬泗州虹今讀如絳】獲男女四千餘人甲戌遣太子中庶子蕭衍右軍司馬張稷救雍州十一月甲午前軍將軍韓秀方等十五將降於魏【將即亮翻降戶江翻】丁酉魏敗齊兵於沔北【敗補邁翻】將軍王伏保等為魏所獲 丙辰以楊靈珍為北秦州刺史 【考異曰齊氐傳作北梁州今從齊書】仇池公武都王 新野人張䐗帥萬餘家據柵拒魏【䐗與豬同陟魚翻帥讀曰率下同】十二月庚申魏人攻拔之雍州刺史曹虎與房伯玉不協故緩救之頓軍樊城【考異曰齊魏虜傳云均口今從虎傳 余謂曹虎之頓軍樊城不特因與房伯玉不協而然亦由畏魏兵之】<br />
<br />
  【彊而不敢進也】丁丑詔遣度支尚書崔慧景救雍州假慧景節帥衆二萬騎千匹向襄陽雍州衆軍並受節度【度徒洛翻雍於用翻騎奇寄翻】庚午魏主南臨沔水【沔彌兖翻】戊寅還新野將軍王曇紛以萬餘人攻魏南青州黃郭戍【魏收志東魏孝靜帝武定七年置義塘郡治黃郭城又按五代志海州懷仁縣梁置南北二青州東魏廢州立義塘郡及懷仁縣曇徒含翻】魏戌主崔僧淵破之舉軍皆没將軍魯康祚趙公政將兵萬人侵魏太倉口【據傅永傳太倉口在魏豫州界是時魏置豫州於汝南新息縣廣陵城與齊義陽隔淮對壘則太倉口當在淮北岸以魏人積倉粟於此而有是名也】魏豫州刺史王肅使長史清河傅永將甲士三千擊之康祚等軍於淮南永軍於淮北相去十餘里永曰南人好夜斫營【好呼到翻下好學同】必於渡淮之所置火以記淺乃夜分兵為二部伏於營外又以瓠貯火【瓠戶悞翻匏也貯丁呂翻盛也】密使人過淮南岸於深處置之戒曰見火起則亦然之【然與燃同】是夜康祚等果引兵斫永營伏兵夾擊之康祚等走趣淮水【趣七喻翻】火既競起不知所從溺死及斬首數千級【溺奴狄翻】生擒公政獲康祚之尸以歸豫州刺史裴叔業侵魏楚王戍【裴叔業盖自徐州遷為豫州水經注鮦陽縣有葛陵城城東北有楚武王冢民謂之楚王瑟城魏盖於此置戍因謂之楚王戍】肅復令永擊之【復扶又翻】永將心腹一人馳詣楚王戍令填外塹【塹七艷翻】夜伏戰士千人於城外曉而叔業等至城東部分將置長圍【分扶問翻】永伏兵擊其後軍破之叔業留將佐守營自將精兵數千救之【將即亮翻】永登門樓望叔業南行數里即開門奮擊大破之獲叔業傘扇鼓幕甲仗萬餘叔業進退失據遂走左右欲追之永曰吾弱卒不滿三千彼精甲猶盛非力屈而敗自墮吾計中耳既不測我之虚實足使喪膽【喪息浪翻】俘此足矣何更追之魏主遣謁者就拜永安遠將軍汝南太守封貝丘縣男【守式又翻】永有勇力好學能文魏主常歎曰上馬能擊賊下馬作露板唯傅脩期耳【言永有武幹又有文才也傳永字脩期】 曲江公遥欣好武事【好呼到翻】上以諸子尚幼内親則仗遥欣兄弟外親則倚后弟西中郎長史彭城劉暄内弟太子詹事江袥【帝母景皇后袥之姑也故曰内弟】故以始安王遥光為揚州刺史居中用事遥欣為都督荆雍等七州諸軍事荆州刺史鎮據西面而遥欣在江陵多招材勇厚自封殖上甚惡之【惡烏路翻】遥欣侮南郡太守劉季連季連密表遥欣有異迹【包藏禍心者謂之異志形見於事者謂之異迹】上乃以季連為益州刺史【為後劉季連據益州張本】使據遙欣上流以制之季連思考之子也【思考劉遵考之弟】 是歲高昌王馬儒遣司馬王體玄入貢於魏請兵迎接求舉國内徙魏主遣明威將軍韓安保迎之割伊吾之地五百里以居儒衆儒遣左長史顧禮右長史金城麴嘉將步騎一千五百迎安保而安保不至【將即亮翻騎奇寄翻】禮嘉還高昌安保亦還伊吾安保遣其屬朝興安等使高昌【朝姓也漢有鼂錯史記作朝錯朝直遥翻使疏吏翻】儒復遣顧禮將世子義舒迎安保【復扶又翻下同】至白棘城去高昌百六十里高昌舊人戀土不願東遷相與殺儒【魏太和五年馬儒始王高昌至是為國人所殺】立麴嘉為王【麴氏得高昌始此嘉字靈鳳金城榆中人】復臣於柔然安保獨與顧禮馬義舒還洛陽<br />
<br />
  永泰元年【是年四月始改元】春正月癸未朔大赦 加中軍大將軍徐孝嗣開府儀同三司孝嗣固辭 魏統軍李佐攻新野丁亥拔之縛劉思忌問之曰今欲降未【降戶江翻】思忌曰寧為南鬼不為北臣【史言劉思忌忠於所事】乃殺之於是沔北大震【沔彌兖翻】戊子湖陽戍主蔡道福【湖陽縣故蓼國漢屬南陽郡晉宋省齊於此置戍湖陽既入魏置西淮安郡唐為湖陽縣屬唐州】辛卯赭陽戍主成公期壬辰舞陰戍主黃瑶起南鄉太守席謙相繼南遁瑶起為魏所獲魏主以賜王肅肅臠而食之【黃瑶起殺王肅父奐見一百三十八卷世祖永明十一年】乙巳命太尉陳顯達救雍州【雍於用翻】 上有疾以近親寡弱忌高武子孫時高武子孫猶有十王【十王下所殺者是也】每朔朢入朝【朝直遥翻】上還後宫輒歎息曰我及司徒諸子皆不長【意呼遥光為司徒也考之遥光傳時未拜司徒詳考齊史帝弟安陸昭王緬先帝卒建武元年贈 徒此盖指言緬諸子】高武子孫日益長大【長皆音丁丈翻今知兩翻】上欲盡除高武之族以微言問陳顯達對曰此等豈足介慮以問揚州刺史始安王遥光遥光以為當以次施行遥光有足疾【遥光生而有躄疾】上常令乘輿自望賢門入【望賢門華林園門也本名鳳莊門以遥光父諱鳳改焉】每與上屏人久語畢上索香火嗚咽流涕明日必有所誅【左右以此覘知之屏必郢翻索山客翻】會上疾暴甚絶而復蘇【復扶又翻】遥光遂行其策丁未殺河東王鉉臨賀王子岳西陽王子文永陽王子峻南康王子琳衡陽王子珉湘東王子建南郡王子夏桂陽王昭粲巴陵王昭秀於是太祖世祖及世宗諸子皆盡矣【鉉太祖子子岳至子夏皆世祖子昭粲昭秀世宗子夏戶雅翻】鉉等已死乃使公卿奏其罪狀請誅之下詔不許再奏然後許之【難將一人手掩盡天下目齊明帝之詔類如此】南康侍讀濟陽江泌哭子琳淚盡繼之以血【濟子禮翻泌薄必翻又兵媚翻】親視殯葬畢乃去 庚戌魏主如南陽二月癸丑詔左衛將軍蕭惠休等救夀陽【是時魏不攻夀陽疑夀字悞】甲子魏人拔宛北城房伯玉面縛出降【宛於元翻降戶江翻】伯玉從父弟思安為魏中統軍數為伯玉泣請魏主乃赦之【宋泰始三年房法夀降魏故房氏羣從多仕於魏而思安得為伯王請從才用翻數為上所角翻下於偽翻】庚午魏主如新野辛巳以彭城王勰為使持節都督南征諸軍事中軍大將軍開府儀同三司【勰音恊使疏吏翻】三月壬午朔崔慧景蕭衍大敗於鄧城【鄧縣漢屬南陽郡宋大明末割襄陽西界為京兆郡鄧縣屬焉其地在隋襄陽郡安養縣界唐貞元中又改安養縣為鄧城縣今鄧城縣在襄陽城北二十里隔漢水按南北對境圖自鄧城南過新河至樊城】時慧景至襄陽五郡已陷没【五郡謂南陽新野南鄉北襄城并西汝南北義陽二郡太守也】慧景與衍及軍主劉山陽傅法憲等帥五千餘人進行鄧城魏數萬騎奄至【帥讀曰率行下孟翻騎奇寄翻】諸軍登城拒守時將士蓐食輕行皆有飢懼之色衍欲出戰慧景曰虜不夜圍人城待日暮自當去既而魏衆轉至慧景於南門拔軍去諸軍不相知相繼皆遁魏兵自北門入劉山陽與部曲數百人斷後死戰【斷丁管翻】且戰且却行慧景過閙溝【據蕭子顯齊書閙溝近沙堨沙堨在宛縣界盖堨水入此溝南流逕鄧城界而入於漢也】軍人相蹈藉【藉慈夜翻】橋皆斷壞魏兵夾路射之【射而亦翻】殺傅法憲士卒赴溝死者相枕【枕之鴆翻】山陽取襖仗填溝乘之得免魏主將大兵追之晡時至沔山陽據城苦戰【沔北有樊城山陽所據盖即此城也】至暮魏兵乃退諸軍恐懼是夕皆下船還襄陽庚寅魏主將十萬衆羽儀華盖以圍樊城曹虎閉門自守魏主臨沔水望襄陽岸乃去如湖陽辛亥如懸瓠魏鎮南將軍王肅攻義陽裴叔業將兵五萬圍渦陽以救義陽【渦陽城在漢沛郡山桑縣東南渦水逕其南時為魏南兖州治所杜佑曰唐為亳州蒙城縣地渦音戈】魏南兖州刺史濟北孟表守渦陽【魏南兖州領下蔡及梁譙沛等郡濟子禮翻】糧盡食草木皮葉叔業積所殺魏人高五丈以示城内【高居傲翻】别遣軍主蕭璝等攻龍亢【龍亢縣漢屬沛郡晉屬譙國後省魏太和十九年置下蔡郡龍亢縣屬焉晉灼曰亢音剛龍亢城南臨渦水璝公回翻】魏廣陵王羽救之叔業引兵擊羽大破之追獲其節魏主使安遠將軍傳永征虜將軍劉藻假輔國將軍高聰救渦陽並受王肅節度叔業進擊大破之聰奔懸瓠永收敗卒徐還叔業再戰凡斬首萬級俘三千餘人獲器械雜畜財物以千萬計魏主命鎖三將詣懸瓠【將即亮翻】劉藻高聰免死徙平州【魏平州治肥如城領遼西北平二郡】傅永奪官爵黜王肅為平南將軍肅表請更遣軍救渦陽魏主報曰觀卿意必以藻等新敗故難於更往朕今少分兵則不足制敵【少詩沼翻】多分兵則禁旅有闕卿審圖之義陽當止則止當下則下若失渦陽卿之過也肅乃解義陽之圍與統軍楊大眼奚康生等步騎十餘萬救渦陽叔業見魏兵盛夜引軍退明日士衆奔潰魏人追之殺傷不可勝數【勝音升下不勝】叔業還保渦口【渦口渦水入淮之口也渦口對淮南岸即齊馬頭郡杜佑曰渦口今臨淮漣水縣非也】 初魏中尉李彪家世孤微【李彪衛國頓丘人家素寒微少孤貧而好學】朝無親援初遊代都以清淵文穆公李冲好士傾心附之【清淵縣漢屬魏郡晉以來屬平陽郡朝直遥翻好呼到翻】冲亦重其材學禮遇甚厚薦於魏主且為之延譽於朝【為於偽翻延譽者為之聲譽使所聞者遠】公私汲引【既公言之於朝而薦之於上又私語同列引而進之引水而上曰汲取此義也】及為中尉彈劾不避貴戚【彈徒丹翻劾戶槩翻又戶得翻下同】魏主賢之以比汲黯彪自以結知人主不復藉冲稍稍疎之唯公坐斂袂而已無復宗敬之意冲浸銜之【復扶又翻坐徂卧翻】及魏主南伐彪與冲及任城王澄共掌留務彪性剛豪意議多所乖異數與冲爭辯形於聲色【數所角翻】自以身為法官他人莫能糾劾事多專恣冲不勝忿乃積其前後過惡禁彪於尚書省上表劾彪違傲高亢【勝音升亢苦浪翻】公行僭逸坐輿禁省【言坐輿而入禁省也漢法不下公門為不敬】私取官材輒駕乘黃【乘黃御馬也乘繩證翻杜佑曰漢有未央廐令魏改為乘黃廐乘黃古之神馬因以為名或亦名飛黃背有角日行萬里淮南子云天下有道飛黃伏皁】無所憚懾臣輒集尚書已下令史已上於尚書都坐【尚書都坐録今僕射尚書园坐處】以彪所犯罪狀告彪訊其虛實彪皆伏罪請以見事免彪所居職付廷尉治罪【見事謂彪見所犯之事也見賢遍翻治直之翻】冲又表稱臣與彪相識以來垂二十載【載子亥翻】見其才優學博議論剛正愚意誠謂拔萃公清之人後稍察其人酷急猶謂益多損少自大駕南行以來彪兼尚書【彪以中尉兼度支尚書】日夕共事始知其專恣無忌尊身忽物聽其言如振古忠恕之賢【振古自古也】校其行寔天下佞暴之賊【行下孟翻】臣與任城卑躬曲己若順弟之奉暴兄其所欲者事雖非理無不屈從依事求實悉有成驗如臣列得實【列謂陳列其事】宜殛彪於北荒以除亂政之姦【詩曰取彼譛人投畀有北毛注云北方寒凉而不毛】所引無證宜投臣於四裔以息青蠅之譛【詩曰營營青蠅止于棘讒人罔極交亂四國】冲手自作表家人不知帝覽表歎悵久之曰不意留臺乃至於此既而曰道固可謂溢矣而僕射亦為滿也【李彪字道固僕射謂冲也】黃門侍郎宋弁素怨冲而與彪同州相善【弁廣平人彪頓丘人二郡皆屬相州】陰左右之【左音佐右音佑】有司處彪大辟【處昌呂翻下久處同辟毗亦翻】帝宥之除名而已【魏孝文於此可謂明矣】冲雅性愠厚及收彪之際親數彪前後過失瞋目大呼投折几案御史皆泥首面縛【數所具翻瞋昌真翻呼火故翻折而設翻中尉得罪而御史皆泥首面縛以謝冲以朝儀言之無是理也魏主所謂僕射亦為滿不亦信哉】冲詈辱肆口遂病荒悸言語錯繆時扼腕大罵稱李彪小人【悸其季翻腕烏貫翻】醫藥皆不能療或以為肝裂【怒氣傷肝怒甚病而醫不能療故以為肝裂】旬餘而卒帝哭之悲不自勝【勝音升】贈司空冲勤敏彊力久處要劇【處昌呂翻】文案盈積終日視事未嘗厭倦職業修舉纔四十而髮白弟兄六人凡四毋少時每多忿競【少詩照翻】及冲貴禄賜皆與共之更成敦睦然多援引族姻私以官爵一家歲禄萬匹有餘時人以此少之【少詩沼翻下同】 魏主以彭城王勰為宗師【魏置宗師見一百二十三卷晉安帝元興三年勰音協】詔使督察宗室有不帥教者以聞【帥讀曰率】 夏四月甲寅改元【改元永泰】 大司馬會稽太守王敬則自以高武舊將心不自安【會工外翻將即亮翻】上雖外禮甚厚而内相疑備【疑備者疑其為變而為之防】數訪問敬則飲食體幹堪宜【堪勝也宜適也問其尚能勝兵及適用與否也數所角翻】聞其衰老且以居内地故得少寛【少詩沼翻】前二歲上遣領軍將軍蕭坦之將齋仗五百人行武進陵【齊自武帝以上諸陵皆在武進行下孟翻】敬則諸子在都憂怖無計【怖普布翻】上知之遣敬則世子仲雄入東安尉之【自建康東入會稽尉與慰同】仲雄善琴上以蔡邕焦尾琴借之【蔡邕在吳吳人有燒桐以爨者邕聞火烈之聲知其良木因請而裁為琴果有美音而其尾猶焦時人因名焦尾琴白虎通曰琴禁也禁止於邪以正人心也廣雅曰琴長三尺六寸六分象三百六十六日五絃象五行桓譚新論五絃第一絃為宫其次商角徵羽文王武王各加一絃以為少宫少商】仲雄於御前鼓琴作懊儂歌【晉志曰懊儂歌者隆安初俗間訛謡之曲歌云春草可攬結女兒可攬擷杜佑曰懊儂歌石崇妾緑珠所作絲布澁難縫一曲而已懊於報翻儂如冬翻仲雄倣其曲而作歌】曰常歎負情儂【儂音農吳語也】郎今果行許又曰君行不淨心那得惡人題【惡烏路翻】上愈猜愧上疾屢危乃以光禄大夫張瓌為平東將軍吳郡太守【瓌古回翻守手又翻】置兵佐以密防敬則中外傳言當有異處分【處昌呂翻分扶問翻】敬則聞之竊曰東今有誰只是欲平我耳東亦何易可平【易以豉翻】吾終不受金甖金甖謂鴆也【賜死者以金甖盛鴆酒故云然】敬則女為徐州行事謝朓妻【朓土了翻】敬則子太子洗馬幼隆遣正員將軍徐岳以情告脁【官至將軍而未有軍號者為正員將軍次為員外將軍洗悉薦翻】為計若同者當往報敬則朓執岳馳啟以聞敬則城局參軍徐庶家在京口其子密以報庶庶以告敬則五官掾王公林【自晉以來諸郡有五官掾】公林敬則族子也常所委信公林勸敬則急送啟賜兒死單舟星夜還都敬則令司馬張思祖草啟既而曰若爾諸郎在都要應有信且忍一夕【言且遲一夜也】其夜呼僚佐文武樗蒱謂衆曰卿諸人欲令我作何計莫敢先荅防閤丁興懷曰官衹應作爾【言應作如此事謂應反也】敬則不應明旦召山陰令王詢臺傳御史鍾離祖願【臺傳御史臺所遣督諸郡錢穀者傳株戀翻】敬則橫刀跂坐【跂坐垂足而坐跟不及地跂去智翻】問詢等發丁可得幾人庫見有幾錢物【見賢遍翻】詢稱縣丁猝不可集祖願稱庫物多未輸入敬則怒將出斬之【將引也】王公林又諫曰凡事皆可悔唯此事不可悔官詎不更思【詎豈也】敬則唾其面曰我作事何關汝小子敬則舉兵反招集配衣【配分給也衣於既翻分給袍甲以衣被之】二三日便前中書令何胤棄官隱居若邪山【若邪山在會稽東南四十里邪讀曰耶】敬則欲刼以為尚書令長史王弄璋等諫曰何令高蹈必不從不從便應殺之舉大事先殺名賢事必不濟敬則乃止胤尚之之孫也【何尚之柄用於宋文武兩朝】 庚午魏發州郡兵二十萬人期八月中旬集懸瓠【復欲南伐也】 魏趙郡靈王幹卒【謚法亂而不損曰靈】 上聞王敬則反收王幼隆及其兄員外郎世雄【此即敬則世子仲雄也仲世二字必有一悮】記室參軍季哲【敬則為大司馬以其子為記室參軍】其弟太子舍人少安等皆殺之【少詩照翻】長子黃門郎元遷將千人在徐州撃魏【長知兩翻將即亮翻】敕徐州刺史徐玄慶殺之前吳郡太守南康侯子恪嶷之子也【豫章王嶷武帝之弟嶷魚力翻】敬則起兵以奉子恪為名子恪亡走未知所在始安王遥光勸上盡誅高武子孫於是悉召諸王侯入宫晉安王寶義江陵公寶覽等處中書省【寶義皇子寶覽姪也處昌呂翻下同】高武諸孫處西省【據蕭子恪傳西省永福省也至唐分三省以門下省為西省中書省為東省】敕人各從左右兩人過此依軍法孩幼者與乳母俱入其夜令太醫煮椒二斛都水辦棺材數十具【前漢都水屬水衡都尉後漢光武省水衡都尉并少府都水屬郡國晉屬大司農蕭子顯志無都水都官尚書有水曹以此考之都水當屬將作大匠然齊大匠卿不常置故都水之官不見於志孩何開翻】須三更當盡殺之【須待也三更丙夜也更工衡翻】子恪徒跣自歸二更達建陽門刺啟【書姓名於奏白曰刺啟奏也既達姓名又啟陳其事】時刻已至而上眠不起中書舍人沈徽孚與上所親左右單景雋共謀少留其事須臾上覺【單上演翻少詩沼翻覺古孝翻寢而寤謂之覺】景雋啟子恪已至上驚問曰未邪未邪景雋具以事對上撫牀曰遥光幾誤人事【單景雋具以子恪所啟之事對上乃謂幾為遥光所誤而濫殺幾居希翻】乃賜王侯供饌【饌雛戀翻又雛晥翻】明日悉遣還第以子恪為太子中庶子寶覽緬之子也【緬上弟也緬彌兖翻】敬則帥實甲萬人過浙江【今之錢塘江也帥讀曰率】張瓌遣兵三千拒敬則於松江【松江在吳郡吳縣南古笠澤也今屬蘇州吳江縣】聞敬則軍鼓聲一時散走瓌棄郡逃民間敬則以舊將舉事百姓擔篙荷鍤隨之者十餘萬衆【將即亮翻擔都甘翻篙古勞翻竹竿也用以撑船荷下可翻鍤楚洽翻鍫也】至晉陵南沙人范脩化殺縣令公上延孫以應之【公上複姓也敬則本晉陵南沙人故范脩化舉縣應之】敬則至武進陵口慟哭而過【蕭氏之先葬武進高帝之殂也從其先兆亦葬武進號泰安陵敬則懷高帝恩故慟哭而過陸游曰自常州西北至呂城過陵口見大石獸偃仆道旁已殘缺盖南朝陵墓齊明帝時王敬則反至陵口慟哭而過是也距丹陽縣三十餘里丹陽古所謂曲阿或曰雲陽宋白曰吳大帝改丹陽為武進縣吳末併入晉陵縣】烏程丘仲孚為曲阿令敬則前鋒奄至仲孚謂吏民曰賊乘勢雖鋭而烏合易離今若收船艦鑿長岡埭【長岡在曲阿縣界今謂之上下夾堽埭即今之上金斗門易以豉翻艦戶黯翻埭徒耐翻】瀉瀆水以阻其路得留數日臺軍必至如此則大事濟矣敬則軍至值瀆涸果頓兵不得進五月詔前軍司馬左興盛後軍將軍崔恭祖輔國將軍劉山陽龍驤將軍馬軍主胡松築壘於曲阿長岡【驤思將翻】右僕射沈文季為持節都督屯湖頭備京口路【湖頭玄武湖頭也其地東接蔣山西巖下西抵玄武湖隄地勢坦平當京口大路】恭祖慧景之族也【前書後軍將軍崔恭祖接魏晉以來官制左右前後將軍是為四軍恭祖位號未能至此齊書王敬則傳作後軍將軍直閤將軍崔恭祖恭祖若為後軍將軍不應下帶直閤將軍此必有誤】敬則急攻興盛山陽二壘臺軍不能敵欲退而圍不開各死戰胡松引騎兵突其後白丁無器仗皆驚散敬則軍大敗索馬再上不能得崔恭祖刺之仆地【索山客翻上時掌翻刺七亦翻】興盛軍客袁文曠斬之【軍客齊書王敬則傳作軍容南史有軍容馬容如桓康為齊高帝軍容蕭摩訶馬容陳智深斬陳叔陵盖皆簡拔魁健有武藝之士使之前驅以壯軍馬之容故以為名】乙酉傳首建康是時上疾已篤敬則倉猝東起朝廷震懼太子寶卷使人上屋望見征虜亭失火【上時掌翻征虜亭在方山南自玄武湖頭大路北出至征虜亭】謂敬則至急裝欲走【急裝謂縛袴也戎裝謂之急裝】敬則聞之喜曰檀公三十六策走為上策計汝父子唯有走耳盖時人譏檀道濟避魏之語也敬則之來聲勢甚盛裁少日而敗【裁少日謂不及二旬也少詩沼翻】臺軍討賊黨晉陵民以附敬則應死者甚衆太守王瞻上言愚民易動不足窮法【窮法謂盡法繩之易以豉翻】上許之所全活以萬數瞻弘之從孫也【王弘之以仕晉宋武帝辟召無所就從才用翻】上賞謝脁之功遷尚書吏部郎【唐六典曰吏部郎職在選舉魏晉用人妙於時選其諸曹郎功高者遷吏部郎歷代品秩皆高於諸曹郎魏晉宋齊吏部郎品第五諸曹郎品第六】朓上表三讓上不許中書疑朓官未及讓國子祭酒沈約曰近世小官不讓遂成恒俗【恒戶登翻】謝吏部今授超階讓别有意【朓自兼殿中郎遷吏部郎故曰超階朓恥以告妻父得官故曰讓别有意】夫讓出人情豈關官之大小邪朓妻常懷刃欲殺朓朓不敢相見 秋七月魏彭城王勰表以一歲國秩職俸親恤裨軍國之用【國秩彭城國秩也職俸勰所居職合受之俸也親恤亦魏朝給勰以恤親者勰音協】魏主詔曰割身存國理為遠矣職俸便停親國聽三分受一【親國謂親恤國秩也】壬午又詔損皇后私府之半六宫嬪御五服男女供恤亦減半【嬪毗賓翻】在軍者三分省一以給軍賞 癸卯以太子中庶子蕭衍為雍州刺史【為後蕭衍以雍州起兵張本雍於用翻】 己酉上殂於正福殿【年四十七】遺詔徐令可重申前命【徐令謂徐孝嗣也孝嗣為尚書令建武四年加開府儀同三司辭不受重直龍翻】沈文季可左僕射江祏可右僕射江祀可侍中劉暄可衛尉軍政可委陳太尉【陳太尉謂顯達】内外衆事無大小委徐孝嗣遥光坦之江袥其大事與沈文季江祀劉暄參懷心膂之任可委劉悛蕭惠休崔慧景【悛七倫翻又丑緣翻】上性猜多慮簡於出入竟不郊天【天子即位當奉珪幣以見上帝於南郊】又深信巫覡【覡刑狄翻】每出先占利害東出云西南出云北初有疾甚袐之聽覽不輟久之敕臺省文簿中求白魚以為藥外始知之【本草曰白魚味甘平無毒主胃氣開胃下食去水氣令人肥健大者六七尺色白頭昂生江湖中按此求文簿中白魚則所謂蠧書魚也本草謂之衣魚亦曰白魚利小便療偏風口喎衍義曰衣魚多在故書中久不動衣帛中或有之身有厚粉手搐之則粉落】太子即位 八月辛亥魏太子自洛陽朝於懸瓠【朝直遥翻下同】 壬子奉朝請鄧學以齊興郡降魏【武帝永明三年置齊興郡屬郢州其地當在西陽弋陽二郡界】 魏主之入寇也遣使高車兵高車憚遠役奉袁紇樹者為主相帥北叛【帥讀曰率】魏主遣征北將軍宇文福討之大敗而還【還從宣翻又如字】福坐黜官更命平北將軍江陽王繼都督北討諸軍事以討之自懷朔以東悉稟節度仍攝鎮平城繼熙之曾孫也【熙道武之子】 八月葬明皇帝於興安陵【陵在曲阿】廟號高宗東昏侯惡靈在太極殿欲速葬【惡烏路翻】徐孝嗣固爭得踰月帝每當哭輒云喉痛太中大夫羊闡入臨【臨力鴆翻哭也】無髮號慟俯仰幘遂脱地帝輟哭大笑謂左右曰秃鶖啼來乎【號戶高翻漢五行志曰鵜鶘或曰禿鶖師古曰鵜鶘一名淘河腹下胡大如數升囊好羣入澤中抒水食魚因名秃鶖亦水鳥也陸佃埤雅曰鶖性貪惡今俗呼秃鶖一名扶老狀如鶴而大長頸赤目其毛辟水毒頭高八尺善與人鬬好㗖蛇鶖音秋】 九月己亥魏主聞高宗殂下詔稱禮不伐喪【春秋左氏傳曰晉士匈侵齊及穀聞喪而還禮也公羊傳曰還者何善辭也何善爾大其不伐喪也】引兵還庚子詔北伐高車 魏主得疾甚篤旬日不見侍臣左右唯彭城王勰等數人而已勰内侍醫藥外總軍國之務遠近肅然人無異議右軍將軍丹陽徐謇善醫【徐謇丹陽人宋明帝之世客青州慕容白曜克東陽謇遂為魏所獲謇九輦翻】時在洛陽急召之既至勰涕泣執手謂曰君能已至尊之疾當獲意外之賞不然有不測之誅非但榮辱乃繫存亡勰又密為壇於汝水之濱依周公故事告天地及顯祖乞以身代魏主【周公金縢之事以周公之至誠行之則可若王莽則偽也】魏主疾有閒【閒如字朱元晦曰閒少差也】丙午發懸瓠舍於汝濱集百官坐徐謇於上席稱揚其功除鴻臚卿封金鄉縣伯賜錢萬緡【臚陵如翻】諸王别餉賚各不減千匹冬十一月辛巳魏主如鄴戊子立妃褚氏為皇后 魏江陽王繼上言高車頑昧避役遁逃若悉追戮恐遂擾亂請遣使鎮别推檢【言六鎮各遣一使令各推檢一鎮使疏吏翻】斬魁首一人自餘加以慰撫若悔悟從役者即令赴軍【令赴南伐之軍也】詔從之於是叛者往往自歸繼先遣人慰諭樹者樹者亡入柔然尋自悔相帥出降【帥讀日率降戶江翻】魏主善之曰江陽可大任也十二月甲寅魏主自鄴班師【北征至鄴而高車已降遂班師】 林邑王諸農入朝海中值風溺死【武帝永明十年范諸農得國朝直遥翻溺奴歷翻】以其子文欵為林邑王<br />
<br />
  資治通鑑卷一百四十一<br />
<br />
<史部,編年類,資治通鑑>  <br>
   </div> 

<script src="/search/ajaxskft.js"> </script>
 <div class="clear"></div>
<br>
<br>
 <!-- a.d-->

 <!--
<div class="info_share">
</div> 
-->
 <!--info_share--></div>   <!-- end info_content-->
  </div> <!-- end l-->

<div class="r">   <!--r-->



<div class="sidebar"  style="margin-bottom:2px;">

 
<div class="sidebar_title">工具类大全</div>
<div class="sidebar_info">
<strong><a href="http://www.guoxuedashi.com/lsditu/" target="_blank">历史地图</a></strong>  
<a href="http://www.880114.com/" target="_blank">英语宝典</a>  
<a href="http://www.guoxuedashi.com/13jing/" target="_blank">十三经检索</a> 
<br><strong><a href="http://www.guoxuedashi.com/gjtsjc/" target="_blank">古今图书集成</a></strong> 
<a href="http://www.guoxuedashi.com/duilian/" target="_blank">对联大全</a> <strong><a href="http://www.guoxuedashi.com/xiangxingzi/" target="_blank">象形文字典</a></strong> 

<br><a href="http://www.guoxuedashi.com/zixing/yanbian/">字形演变</a>  <strong><a href="http://www.guoxuemi.com/hafo/" target="_blank">哈佛燕京中文善本特藏</a></strong>
<br><strong><a href="http://www.guoxuedashi.com/csfz/" target="_blank">丛书&方志检索器</a></strong> <a href="http://www.guoxuedashi.com/yqjyy/" target="_blank">一切经音义</a>  

<br><strong><a href="http://www.guoxuedashi.com/jiapu/" target="_blank">家谱族谱查询</a></strong>  <strong><a href="http://shufa.guoxuedashi.com/sfzitie/" target="_blank">书法字帖欣赏</a></strong> 
<br>

</div>
</div>


<div class="sidebar" style="margin-bottom:0px;">

<font style="font-size:22px;line-height:32px">QQ交流群9:489193090</font>


<div class="sidebar_title">手机APP 扫描或点击</div>
<div class="sidebar_info">
<table>
<tr>
	<td width=160><a href="http://m.guoxuedashi.com/app/" target="_blank"><img src="/img/gxds-sj.png" width="140"  border="0" alt="国学大师手机版"></a></td>
	<td>
<a href="http://www.guoxuedashi.com/download/" target="_blank">app软件下载专区</a><br>
<a href="http://www.guoxuedashi.com/download/gxds.php" target="_blank">《国学大师》下载</a><br>
<a href="http://www.guoxuedashi.com/download/kxzd.php" target="_blank">《汉字宝典》下载</a><br>
<a href="http://www.guoxuedashi.com/download/scqbd.php" target="_blank">《诗词曲宝典》下载</a><br>
<a href="http://www.guoxuedashi.com/SiKuQuanShu/skqs.php" target="_blank">《四库全书》下载</a><br>
</td>
</tr>
</table>

</div>
</div>


<div class="sidebar2">
<center>


</center>
</div>

<div class="sidebar"  style="margin-bottom:2px;">
<div class="sidebar_title">网站使用教程</div>
<div class="sidebar_info">
<a href="http://www.guoxuedashi.com/help/gjsearch.php" target="_blank">如何在国学大师网下载古籍?</a><br>
<a href="http://www.guoxuedashi.com/zidian/bujian/bjjc.php" target="_blank">如何使用部件查字法快速查字?</a><br>
<a href="http://www.guoxuedashi.com/search/sjc.php" target="_blank">如何在指定的书籍中全文检索?</a><br>
<a href="http://www.guoxuedashi.com/search/skjc.php" target="_blank">如何找到一句话在《四库全书》哪一页?</a><br>
</div>
</div>


<div class="sidebar">
<div class="sidebar_title">热门书籍</div>
<div class="sidebar_info">
<a href="/so.php?sokey=%E8%B5%84%E6%B2%BB%E9%80%9A%E9%89%B4&kt=1">资治通鉴</a> <a href="/24shi/"><strong>二十四史</strong></a>&nbsp; <a href="/a2694/">野史</a>&nbsp; <a href="/SiKuQuanShu/"><strong>四库全书</strong></a>&nbsp;<a href="http://www.guoxuedashi.com/SiKuQuanShu/fanti/">繁体</a>
<br><a href="/so.php?sokey=%E7%BA%A2%E6%A5%BC%E6%A2%A6&kt=1">红楼梦</a> <a href="/a/1858x/">三国演义</a> <a href="/a/1038k/">水浒传</a> <a href="/a/1046t/">西游记</a> <a href="/a/1914o/">封神演义</a>
<br>
<a href="http://www.guoxuedashi.com/so.php?sokeygx=%E4%B8%87%E6%9C%89%E6%96%87%E5%BA%93&submit=&kt=1">万有文库</a> <a href="/a/780t/">古文观止</a> <a href="/a/1024l/">文心雕龙</a> <a href="/a/1704n/">全唐诗</a> <a href="/a/1705h/">全宋词</a>
<br><a href="http://www.guoxuedashi.com/so.php?sokeygx=%E7%99%BE%E8%A1%B2%E6%9C%AC%E4%BA%8C%E5%8D%81%E5%9B%9B%E5%8F%B2&submit=&kt=1"><strong>百衲本二十四史</strong></a>  <a href="http://www.guoxuedashi.com/so.php?sokeygx=%E5%8F%A4%E4%BB%8A%E5%9B%BE%E4%B9%A6%E9%9B%86%E6%88%90&submit=&kt=1"><strong>古今图书集成</strong></a>
<br>

<a href="http://www.guoxuedashi.com/so.php?sokeygx=%E4%B8%9B%E4%B9%A6%E9%9B%86%E6%88%90&submit=&kt=1">丛书集成</a> 
<a href="http://www.guoxuedashi.com/so.php?sokeygx=%E5%9B%9B%E9%83%A8%E4%B8%9B%E5%88%8A&submit=&kt=1"><strong>四部丛刊</strong></a>  
<a href="http://www.guoxuedashi.com/so.php?sokeygx=%E8%AF%B4%E6%96%87%E8%A7%A3%E5%AD%97&submit=&kt=1">說文解字</a> <a href="http://www.guoxuedashi.com/so.php?sokeygx=%E5%85%A8%E4%B8%8A%E5%8F%A4&submit=&kt=1">三国六朝文</a>
<br><a href="http://www.guoxuedashi.com/so.php?sokeytm=%E6%97%A5%E6%9C%AC%E5%86%85%E9%98%81%E6%96%87%E5%BA%93&submit=&kt=1"><strong>日本内阁文库</strong></a> <a href="http://www.guoxuedashi.com/so.php?sokeytm=%E5%9B%BD%E5%9B%BE%E6%96%B9%E5%BF%97%E5%90%88%E9%9B%86&ka=100&submit=">国图方志合集</a> <a href="http://www.guoxuedashi.com/so.php?sokeytm=%E5%90%84%E5%9C%B0%E6%96%B9%E5%BF%97&submit=&kt=1"><strong>各地方志</strong></a>

</div>
</div>


<div class="sidebar2">
<center>

</center>
</div>
<div class="sidebar greenbar">
<div class="sidebar_title green">四库全书</div>
<div class="sidebar_info">

《四库全书》是中国古代最大的丛书,编撰于乾隆年间,由纪昀等360多位高官、学者编撰,3800多人抄写,费时十三年编成。丛书分经、史、子、集四部,故名四库。共有3500多种书,7.9万卷,3.6万册,约8亿字,基本上囊括了古代所有图书,故称“全书”。<a href="http://www.guoxuedashi.com/SiKuQuanShu/">详细>>
</a>

</div> 
</div>

</div>  <!--end r-->

</div>
<!-- 内容区END --> 

<!-- 页脚开始 -->
<div class="shh">

</div>

<div class="w1180" style="margin-top:8px;">
<center><script src="http://www.guoxuedashi.com/img/plus.php?id=3"></script></center>
</div>
<div class="w1180 foot">
<a href="/b/thanks.php">特别致谢</a> | <a href="javascript:window.external.AddFavorite(document.location.href,document.title);">收藏本站</a> | <a href="#">欢迎投稿</a> | <a href="http://www.guoxuedashi.com/forum/">意见建议</a> | <a href="http://www.guoxuemi.com/">国学迷</a> | <a href="http://www.shuowen.net/">说文网</a><script language="javascript" type="text/javascript" src="https://js.users.51.la/17753172.js"></script><br />
  Copyright &copy; 国学大师 古典图书集成 All Rights Reserved.<br>
  
  <span style="font-size:14px">免责声明:本站非营利性站点,以方便网友为主,仅供学习研究。<br>内容由热心网友提供和网上收集,不保留版权。若侵犯了您的权益,来信即刪。scp168@qq.com</span>
  <br />
ICP证:<a href="http://www.beian.miit.gov.cn/" target="_blank">鲁ICP备19060063号</a></div>
<!-- 页脚END --> 
<script src="http://www.guoxuedashi.com/img/plus.php?id=22"></script>
<script src="http://www.guoxuedashi.com/img/tongji.js"></script>

</body>
</html>
