資治通鑑卷一百二十九 宋 司馬光 撰

胡三省 音註

宋紀十一|{
	起屠維大淵獻盡閼逢執徐凡六年}


世祖孝武皇帝下

大明三年春正月己巳朔兖州兵與魏皮豹子戰於高平兖州兵不利 己丑以驃騎將軍柳元景為尚書令|{
	驃匹妙翻騎奇寄翻}
右僕射劉遵考為領軍將軍 己酉魏河南公伊馛卒|{
	馛蒲撥翻卒子恤翻}
三月乙卯以揚州六郡為王畿|{
	六郡丹陽淮南宣城吳郡吳興義興也}
更以東揚州為揚州徙治會稽|{
	置東揚州見上卷孝建元年會工外翻}
猶以星變故也|{
	星變見上卷孝建三年}
三月庚寅以義興太守垣閬為兖州刺史閬遵之子也|{
	垣遵即垣苗也武帝西征長安使遵守洛當城城據河濟之會後人謂之垣苗城守式又翻}
夏四月乙巳魏主立其弟子推為京兆王 竟陵王誕知上意忌之亦濳為之備因魏人入寇修城浚隍聚糧治仗|{
	治直之翻}
誕記室參軍江智淵知誕有異志請假先還建康|{
	假古訝翻假休假也}
上以為中書侍郎智淵夷之弟子也|{
	江夷江湛之父夷之弟曰僧安}
少有操行|{
	少詩照翻操七到翻行下孟翻}
沈懷文每稱之曰人所應有盡有人所應無盡無者其唯江智淵乎是時道路皆云誕反會吳郡民劉成上書稱息道龍昔事誕|{
	息子也}
見誕在石頭城修乘輿灋物習唱警蹕|{
	此蓋言誕為揚州刺史時誕時一心奉上必無是事劉成誣告之也乘䋲證翻}
道龍憂懼私與伴侶言之誕殺道龍 |{
	考異曰宋畧作道就今從宋書}
又豫章民陳談之上書稱弟詠之在誕左右見誕書陛下年紀姓諱往巫鄭師鄰家祝詛|{
	祝職叔翻詛莊助翻}
詠之密以啓聞誕誣詠之乘酒罵詈殺之|{
	劉道龍陳談之蓋先皆為誕所殺其父兄希指誣告以報子弟之讐耳詈力智翻}
上乃令有司奏誕罪惡請收付廷尉治罪|{
	冶直之翻}
乙卯詔貶誕爵為侯遣之國詔書未下|{
	下遐稼翻}
先以羽林禁兵配兖州刺史垣閬使以之鎮為名與給事中戴明寶襲誕閬至廣陵誕未悟也明寶夜報誕典籖蔣成使明晨開門為内應成以告府舍人許宗之|{
	宗之竟陵王府舍人也}
宗之入告誕誕驚起呼左右及素所畜養數百人|{
	畜許六翻}
執蔣成勒兵自衛天將曉明寶與閬帥精兵數百人猝至而門不開誕已列兵登陴|{
	帥讀曰率陴頻彌翻}
自在門上斬蔣成赦作徒繫囚|{
	作徒坐徒罪居作者繫囚逮捕在獄者}
開門擊閬殺之 |{
	考異曰宋畧云己亥殺閬按本紀乙卯貶誕爵今從之}
明寶從間道逃還|{
	間古莧翻}
詔内外纂嚴 |{
	考畧曰宋畧乙亥纂嚴按長歷是月戊戌朔無乙亥蓋己亥也}
以始興公沈慶之為車騎大將軍開府儀同三司南兖州刺史將兵討誕|{
	騎奇寄翻將即亮翻下同}
甲子上親總禁兵頓宣武堂司州刺史劉季之誕故將也|{
	誕為會稽季之為參軍及起兵討元凶以季之為將}
素與都督宗慤有隙|{
	宗慤為豫州兼督司州}
聞誕反恐為慤所害委官間道自歸朝廷至盱眙盱眙太守鄭瑗疑季之與誕同謀邀殺之|{
	盱眙音吁怡}
沈慶之至歐陽|{
	水經注曰吳城䢴溝通江淮自永和中江都水斷其水上承歐陽引江入埭六十里至廣陵城余據此地則今之真州閘也}
誕遣慶之宗人沈道愍齎書說慶之餉以玉環刀慶之遣道愍反數以罪惡誕焚郭邑驅居民悉使入城閉門自守|{
	說輸芮翻數所具翻守手又翻}
分遣書檄邀結遠近時山陽内史梁曠家在廣陵誕執其妻子遣使邀曠曠斬使拒之|{
	使疏吏翻}
誕怒滅其家誕奉表投之城外曰陛下信用讒言遂令無名小人來相掩襲不任枉酷|{
	任音壬}
即加誅翦雀鼠偷生仰違詔今親勒部曲鎮扞徐兖先經何福同生皇家|{
	誕於帝同氣也故云然}
今有何愆便成胡越陵鋒蹈戈萬没豈顧|{
	萬没猶言萬死也}
盪定之期冀在旦夕又曰陛下宮帷之醜豈可三緘|{
	家語孔子觀周入后稷之廟有金人三緘其口而銘其背曰占之慎言人也}
上大怒凡誕左右腹心同籍朞親在建康者並誅之|{
	同籍諸同宗屬之籍者朞親謂朞喪之親也}
死者以千數或有家人已死方自城内出奔者慶之至城下誕登樓謂之曰沈公垂白之年|{
	白髮下垂故曰垂白}
何苦來此慶之曰朝廷以君狂愚不足勞少壯故耳|{
	少詩照翻}
上慮誕奔魏使慶之斷其走路|{
	斷音短}
慶之移營白土去城十八里又進軍新亭|{
	此新亭在廣陵城外非建康之新亭也}
豫州刺史宗慤徐州刺史劉道隆並帥衆來會|{
	帥讀曰率}
兖州刺史沈僧明慶之兄子也亦遣兵助慶之先是誕誑其衆云宗慤助我慤至繞城躍馬呼曰我宗慤也|{
	先悉薦翻呼火故翻誑居况翻}
誕見諸軍大集欲棄城北走留中兵參軍申靈賜守廣陵自將步騎數百人|{
	將即亮翻騎奇寄翻}
親信並自隨聲云出戰邪趨海陵道|{
	晉安帝分廣陵立海陵郡今泰州也誕自廣陵北門聲言出戰邪而趨東則海陵之路趨七喻翻}
慶之遣龍驤將軍武念追之|{
	驤思將翻}
誕行十餘里衆皆不欲去互請誕還城誕曰我還易耳卿能為我盡力乎衆皆許諾誕乃復還|{
	易以䜴翻為于偽翻復扶又翻}
築壇㰱血以誓衆|{
	㰱色洽翻}
凡府州文武皆加秩|{
	府司空竟陵王府州南兖州}
以主簿劉琨之為中兵參軍琨之遵考之子也|{
	劉遵考時在朝為尚書右僕射}
辭曰忠孝不得並琨之老父在不敢承命誕囚之十餘日終不受乃殺之右衛將軍垣護之虎賁中郎將殷孝祖等擊魏還至廣陵上並使受慶之節度|{
	賁音奔}
慶之進營逼廣陵城誕餉慶之食提挈者百餘人出自北門慶之不開視悉焚之誕於城上授函表|{
	授南史作投當從之}
請慶之為送慶之曰我受詔討賊不得為汝送表汝必欲歸死朝廷自應開門遣使吾為汝護送|{
	誕之為此以帝猜忍欲以間慶之也慶之峻絶之蓋亦自為謀耳為于偽翻}
東揚州刺史顔竣遭母憂送喪還都上恩待猶厚竣時對親舊有怨言|{
	竣七倫翻}
或語及朝廷得失會王僧逹得罪|{
	僧逹死見上卷上年}
疑竣譖之將死具陳竣前後怨望誹謗之語上乃使御史中丞庾徽之劾奏免竣官|{
	鄭樵曰御史之名周官有之蓋掌贊書而授法令非今任也戰國時秦趙澠池之會各命御史書事又淳于髠謂齊王曰御史在前則皆記事之任至秦漢為糾察之任}
竣愈懼上啓陳謝且請生命上益怒詔答曰卿訕訐怨憤已孤本望乃復過煩思慮懼不自全|{
	訐居謁翻復扶又翻下復沉同}
豈為下事上誠節之至邪及竟陵王誕反上遂誣竣與誕通謀五月收竣付廷尉先折其足|{
	折而設翻}
然後賜死妻予徙交州至宫亭湖|{
	宫亭湖即彭蠡湖在彭澤縣西}
復沉其男口|{
	沉持林翻顔竣失職怨望固為可罪而自尋陽東下之時保護之功不可忘也既殺其身又沉其男口孝武帝亦少恩哉}
六月戊申魏主如隂山 上命沈慶之為三烽於桑里|{
	桑里在廣陵城西南}
若克外城舉一烽克内城舉兩烽擒劉誕舉三烽璽書督趣前後相繼|{
	璽斯氏翻趣讀曰促}
慶之焚其東門塞塹造攻道立行樓土山并諸攻具|{
	塞悉則翻塹七艷翻為樓車推進以攻城故曰行樓}
值久雨不得攻城上使御史中丞庾徽之奏免慶之官詔勿問以激之自四月至於秋七月雨止城猶未拔上怒命太史擇日將自濟江討誕太宰義恭固諫乃止誕初閉城拒使者|{
	使疏吏翻}
記室參軍山隂賀弼固諫誕怒抽刀向之乃止誕遣兵出戰屢敗將佐多踰城出降|{
	將即亮翻降戶江翻下同}
或勸弼宜早出弼曰公舉兵向朝廷此事既不可從荷公厚恩又義無違背|{
	荷下可翻背蒲妹翻}
惟當以死明心耳乃飲藥自殺參軍何康之謀開門納官軍不果斬關出降誕為高樓置康之母於其上暴露之不與食母呼康之數日而死誕以中軍長史濮陽范義為左司馬|{
	濮博木翻}
義母妻子皆在城内或謂義曰事必不振|{
	言誕必䧟敗不能振起也}
子其行乎義曰吾人吏也|{
	人吏言為人之佐吏}
子不可以棄母吏不可以叛君必若何康之而活吾弗為也沈慶之帥衆攻城身先士卒親犯矢石|{
	帥讀曰率先悉薦翻}
乙巳克其外城乘勝而進又克小城誕聞兵入走趨後園|{
	趨七喻翻}
隊主沈胤之等追及之擊傷誕墜水引出斬之誕母妻皆自殺|{
	誕母文帝殷修華妻徐妃}
上聞廣陵平出宣陽門敇左右皆呼萬歲侍中蔡興宗陪輦上顧曰卿何獨不呼興宗正色曰陛下今日正應涕泣行誅豈得皆稱萬歲|{
	謂同氣相殘乃天理人倫之變必若以義滅親應涕泣而行誅也}
上不悦詔貶誕姓留氏廣陵城中士民無大小悉命殺之沈慶之請自五尺以下全之其餘男子皆死女子以為軍賞猶殺三千餘口長水校尉宗越臨决皆先刳腸抉眼|{
	扶於决翻}
或笞面鞭腹苦酒灌創然後斬之|{
	苦酒蓋醯之類也創初良翻}
越對之欣欣若有所得|{
	史言宗越之忍}
上聚其首於石頭南岸為京觀|{
	沈慶之蓋悉獻其首故聚於石頭南岸觀古玩翻}
侍中沈懷文諫不聽|{
	史言當時近侍皆正人但諫不行言不聽耳}
初誕自知將敗使黄門呂曇濟與左右素所信者將世子景粹匿於民間|{
	將也曇徒含翻}
謂曰事若不濟思相全脱如其不免可深埋之|{
	謂深埋景粹之尸也}
各分以金寶齎送既出門並散走唯曇濟不去攜負景粹十餘日捕得斬之臨川内史羊璿坐與誕素善下獄死|{
	璿音旋下遐稼翻}
擢梁曠為後將軍贈劉琨之給事黄門侍郎蔡興宗奉旨慰勞廣陵興宗與范義素善收歛其尸送喪歸豫章|{
	范義蓋寓居豫章也蔡興宗之先亦濟陽人勞力到翻歛力贍翻}
上謂曰卿何敢故觸王憲興宗抗言對曰陛下自殺賊臣自葬故交何不可之有上有慙色|{
	兄弟朋友皆天倫也興宗能不忘故交而帝忍誅屠同氣故慙}
宗越治軍嚴善為營陳每數萬人止頓越自騎馬行前使軍人隨其後馬止營合未嘗參差|{
	治直之翻陳讀曰陣參初今翻差义宜翻又初佳翻參差不齊也}
辛未大赦|{
	廣陵既平故肆赦}
丙子以丹陽尹劉秀之為尚書右僕射 丙戍以南兖州刺史沈慶之為司空刺史如故 八月庚戌魏主如雲中壬戌還平城 九月壬辰築上林苑於玄武湖北|{
	文帝元嘉二十二年築北隄立玄武湖於樂游苑北}
初晉人築南郊壇於已位尚書右丞徐爰以為非禮

詔徙於牛頭山西|{
	牛頭山在今建康府上元縣南四十里兩峯如闕}
直宫城之午位及廢帝即位以舊地為吉復還故處|{
	復扶又翻史終言之}
帝又命尚書左丞荀萬秋造五路依金根車加羽葆蓋|{
	五路之制與金根車不同加羽葆蓋愈非古矣沈約曰秦閱三代之車獨取殷制古曰桑根車秦曰金根車}
四年春正月甲子朔魏大赦改元和平 乙亥上耕籍田大赦 己卯詔祀郊廟初乘玉路 庚寅立皇子子勛為晉安王|{
	勛古勲字}
子房為尋陽王子頊為歷陽王子鸞為襄陽王 魏散騎侍郎馮闡來聘|{
	散悉亶翻騎奇寄翻}
二月魏衛將軍樂安王良討河西叛胡|{
	以下文叛胡詣長安首罪觀之此河西蓋謂自龍門東至華隂河之西岸也}
三月魏人寇北隂平|{
	隂平道漢屬廣漢屬國都尉晉武帝泰始中分立隂平郡宋分立南隂平北隂平二郡五代志普安郡隂平縣宋立北隂平郡宋白曰文州隂平郡戰國時氐羌所據永嘉之後羌虜數叛遂立郡以遏之輿地志云晉永嘉末太守王鍳以郡降李雄晉人於是悉流移於蜀漢其氐羌並屬揚茂搜此郡不復預受正朔故南史諸志悉無所録其晉人流寓於蜀者仍於益州立南北二隂平寓於漢中者亦於梁州立南北二隂平}
朱提太守楊歸子擊破之 |{
	考異曰宋帝紀索虜寇北隂平孔提太守楊歸子擊破之宋畧云索虜寇壯降平朱太守楊歸子擊破之按郡縣名無壯降平及孔提北隂平參酌二書當為朱提 今按魏收地形志武都郡有孔提縣五代志武都建威縣後周併西魏之孔堤郡及縣入焉此時魏人蓋寇北隂平之孔堤為北隂平太守楊歸子所破也當從宋紀朱提郡在南中時屬寜州去隂平甚遠蓋考異誤以宋紀宋畧二書所載合為朱提也當讀作孔提屬上句宋畧所謂壯降平亦北隂平三字之誤朱字於下文無所附着當為衍字}
甲申皇后親桑于西郊皇太后觀禮 夏四月魏太后常氏殂|{
	本保太后尊為太后見一百二十七卷文帝元嘉三十年}
五月癸丑魏葬昭太后於鳴雞山|{
	魏土地記曰下洛城東北三十里冇延河東流北有鳴雞山史記趙襄子殺代王於夏屋其姊為代王夫人襄子迎之至此曰代已亡矣吾將安歸乎遂磨笄於山而自殺代人憐之為立祠焉因名為磨笄山每夜有野雞羣鳴於祠屋上故亦謂之鳴雞山杜佑曰媯州治懷戎縣有鳴雞山本名磨笄山}
丙戌尚書左僕射禇湛之卒 吐谷渾王拾寅兩受宋魏爵命|{
	吐從暾入聲谷音浴}
居止出入擬於王者魏人忿之定陽侯曹安表言拾寅今保白蘭若分軍出其左右必走保南山不過十日人畜乏食可一舉而定六月甲午魏遣征西大將軍陽平王新成等督統萬高平諸軍出南道南郡公中山李惠等督凉州諸軍出北道以擊吐谷渾 魏崔浩之誅也|{
	見一百二十五卷宋文帝元嘉二十七年}
史官遂廢至是復置|{
	復扶又翻}
河西叛胡詣長安首罪|{
	樂安王良兵威臨之故首罪首式又翻}
魏遣使者安慰之 秋七月遣使如魏 甲戌開府儀同三司何尚之卒 壬午魏主如河西 魏軍至西平|{
	西平漢落都之地禿髪所都樂都即落都也唐為鄯州}
吐谷渾王拾寅走保南山九月魏軍濟河追之會疾疫引還獲雜畜三十餘萬|{
	畜許又翻}
庚午魏主還平城 丁亥徙襄陽王子鸞為新安王冬十月庚寅詔沈慶之討緣江蠻前廬陵内史周朗言事切直|{
	見一百二十七卷文帝元嘉三十年}
上衘之使有司奏朗居母喪不如禮傳送寧州|{
	傳知戀翻又直戀翻}
於道殺之朗之行也侍中蔡興宗方在直請與朗别坐白衣領職|{
	蔡興宗立於猜暴之朝葬范義别周朗犯時主之怒而不加刑素行有以孚乎人也}
十一月魏散騎常侍盧度世等來聘|{
	散悉亶翻騎奇寄翻}
是歲上徵青冀二州刺史顔師伯為侍中師伯以諂佞被親任|{
	被皮義翻}
羣臣莫及多納貨賄家累千金上嘗與之樗蒲上擲得雉自謂必勝師伯次擲得盧|{
	樗蒲采名有黑犢有雉有盧得盧者勝}
上失色師伯遽歛子曰幾作盧|{
	子五木也此亦師伯為佞之一端幾居希翻}
是日師伯一輸百萬柔然攻高昌殺沮渠安周滅沮渠氏|{
	文帝元嘉十六年魏克凉州沮}


|{
	渠無諱與弟安周西走保據高昌今為柔然所滅沮子余翻}
以闞伯周為高昌王高昌稱王自此始

五年春正月戊午朔朝賀|{
	朝直遥翻}
雪落太宰義恭衣有六出義恭奏以為瑞上悦義恭以上猜暴懼不自容每卑辭遜色曲意祗奉由是終上之世得免於禍 二月辛卯魏主如中山丙午至鄴遂如信都 三月遣使如魏|{
	使疏吏翻}
魏主發并肆州民五千人治河西獵道|{
	魏道武天賜二年分并州北境為九原鎮太武真君七年置肆州宋白曰十三州志云漢末大亂匈奴侵邊自定襄已西盡雲中鴈門之間遂空曹公集荒郡之戶聚之九原界以立新興郡領五原等縣即唐忻州定襄縣之地後魏書云太平二年置肆州寄理秀容城秀容縣忻州所治即漢末所置九原縣也治直之翻}
辛巳還平城夏四月癸巳更以西陽王子尚為豫章王 庚子詔

經始明堂直作大殿於丙巳之地制如太廟唯十有二間為異 雍州刺史海陵王休茂年十七|{
	雍於用翻}
司馬新野庾深之行府事休茂性急欲自專處决|{
	處昌呂翻}
深之及主帥每禁之|{
	主帥典籖也又齋内亦有主帥謂之齋帥帥所類翻}
常懷忿恨左右張伯超有寵多罪惡主帥屢責之伯超懼說休茂曰|{
	說輸芮翻}
主帥密疏官過失欲以啓聞如此恐無好|{
	疏使去翻記也好如字無好猶今人言無好處言將得罪也}
休茂曰為之奈何伯超曰惟有殺行事及主帥|{
	行事謂庾深之江左率謂長史司馬行府州事者為行事}
舉兵自衛此去都數千里|{
	雍州鎮襄陽去建康水行四千餘里}
縱大事不成不失入虜中為王休茂從之丙午夜休茂與伯超等帥夾轂隊|{
	宋諸王有夾轂隊蓋左右親兵也出則夾車為衛帥讀曰率下同}
殺典籖楊慶於城中出金城殺深之及典籖戴雙徵集兵衆建牙馳檄使佐吏上已為車騎大將軍開府儀同三司加黄鉞|{
	凡府州僚屬皆謂之佐吏上時掌翻騎奇寄翻}
侍讀博士荀詵諫|{
	侍讀博士授諸王經者也詵疎臻翻}
休茂殺之伯超專任軍政生殺在已休茂左右曹萬期挺身斫休茂不克而死休茂出城行營|{
	行下孟翻巡行也}
諮議參軍沈暢之等帥衆閉門拒之休茂馳還不得入義成太守薛繼考為休茂盡力攻城|{
	為于偽翻義成太守治襄陽注詳見前}
克之斬暢之及同謀數十人其日參軍尹玄慶復起兵攻休茂生擒斬之母妻皆自殺|{
	復扶又翻休茂母文帝蔡美人}
同黨伏誅城中擾亂莫相統攝中兵參軍劉恭之秀之之弟也|{
	劉秀之孝建元年不附義宣時為尚書右僕射}
衆共推行府州事繼考以兵脅恭之使作啓事言繼考立義自乘驛還都上以為北中郎諮議參軍賜爵冠軍侯|{
	冠古玩翻}
事尋泄伏誅以玄慶為射聲校尉|{
	校尸教翻}
上自即位以來抑黜諸弟既克廣陵欲更峻其科沈懷文曰漢明不使其子比光武之子前史以為美談|{
	見四十五卷漢明帝永平十五年}
陛下既明管蔡之誅願崇唐衛之寄|{
	周成王既誅管叔囚蔡叔封叔虞於唐封康叔於衛以藩屛周室}
及襄陽平太宰義恭探知上指請裁抑諸王不使任邊州及悉輸器甲禁絶賓客沈懷文固諫以為不可乃止 上畋遊無度嘗出夜還敇開門侍中謝莊居守|{
	守手又翻}
以棨信或虛|{
	棨音啓傳也刻术為合符}
執不奉旨須墨敕乃開|{
	墨敕手敕也}
上後因燕飲從容曰卿欲效郅君章邪|{
	郅惲字君章事見四十三卷漢光武建武十三年從干容翻}
對曰臣聞王者祭祀畋遊出入有節今陛下晨往宵歸|{
	宵夜也}
臣恐不逞之徒妄生矯詐是以伏須神筆乃敢開門耳 魏大旱詔州郡境内神無大小悉灑掃致禱|{
	灑所賣翻掃素報翻又各上聲}
俟豐登各以其秩祭之於是羣祀之廢者皆復其舊|{
	魏罷羣祀見一百二十四卷文帝元嘉二十七年}
秋七月戊寅魏主立其弟小新成為濟陽王|{
	大明元年魏主立其弟新成為陽平王此小新成又陽平王之弟也濟子禮翻}
加征東大將軍鎮平原|{
	平原河津之要時魏未得青齊故於此置鎮}
天賜為汝隂王加征南大將軍鎮虎牢|{
	虎牢宋舊鎮為司州刺史治所魏得之置孫州}
萬壽為樂浪王加征北大將軍鎮和龍|{
	和龍燕舊都魏得之以為鎮後為營州樂浪音洛琅}
洛侯為廣平王 壬午魏主巡山北八月丁丑還平城 戊子立皇子子仁為永嘉王子真為始安王 九月甲寅朔日有食之 沈慶之固讓司空柳元景固讓開府儀同三司詔許之仍命慶之朝會位次司空俸禄依三司|{
	朝直遥翻下同 考異曰宋畧此事在戊戌按長歷是月甲寅朔無戊戌}
元景在從公之上|{
	晉制文官光禄三大夫武官驃騎車騎衛將軍及諸大將軍開府者位從公從從用翻}
慶之目不知書家素富產業累萬金童奴千計再獻錢千萬穀萬斛先有四宅又有園舍在婁湖|{
	按南史齊武帝永明元年望氣者云新林婁湖東府西有王氣正月甲子築青溪舊宫作新婁湖苑以厭之則婁湖當在新林東府間也}
慶之一夕攜子孫及中表親戚徙居婁湖以四宅輸官慶之多蓄妓妾|{
	妓渠綺翻}
優游無事盡意歡娯非朝賀不出門車馬率素從者不過三五人|{
	從才用翻}
遇之者不知其為三公也 甲戌移南豫州治於湖|{
	沈約宋志曰晉江左胡寇彊盛豫部殱覆元帝永昌元年豫州刺史祖約自譙城退屯壽春成帝延和四年僑立豫州庾亮為刺史鎮蕪湖咸康四年毛寶為刺史鎮邾城八年庾懌為刺史又鎮蕪湖穆帝永和元年刺史趙胤鎮牛渚二年刺史謝尚鎮蕪湖四年進屯壽春九年還鎮歷陽十一年進馬頭升平元年刺史謝奕戍譙哀帝隆和元年刺史袁真自譙退守壽春簡文咸安元年刺史桓熙戍歷陽孝武寧康元年刺史桓冲戍姑孰太元十年刺史朱序戍馬頭十二年刺史桓石䖍戍歷陽安帝義熙二年刺史劉豫戍姑孰宋武帝欲開拓河南綏定豫上九年割揚州大江以西大雷以北悉屬豫州豫之基址因此而立十三年刺史劉義慶鎮壽陽永初二年分淮東為南豫州治壽陽淮西為豫州文帝元嘉七年又分五年割揚州之淮南宣城又屬焉徙治姑孰今按自宋元以來分立兩豫豫州治淮西南豫治壽陽孝建之初魯爽以南豫州刺史鎮壽陽居然可知也移南豫州於姑孰寔在大明五年自永初至元嘉七年兩豫必嘗復合而所謂五年割揚州之淮南宣城又屬焉徙治姑孰者蓋指帝之大明五年後人傳寫沈志於文帝元嘉七年又分上下文皆有漏脱而劉義慶鎮壽陽通鑑在義熙十四年罷南豫州入豫州在元嘉二十二年}
丁丑以潯陽王子房為南豫州刺史 閏月戊子皇太子妃何氏卒謚曰獻妃壬寅更以歷陽王子頊為臨海王 |{
	考異曰宋畧作子頊今從宋書}
冬十月甲寅以南徐州刺史劉延孫為尚書左僕射右僕射劉秀之為雍州刺史|{
	雍於用翻}
乙卯以新安王子鸞為南徐州刺史子鸞母殷淑儀寵傾後宫子鸞愛冠諸子|{
	冠古玩翻}
凡為上所眄遇者莫不入子鸞之府|{
	眄彌見翻或作盻}
及為南徐州割吳郡以屬之|{
	吳郡自晉氏渡江以來屬揚州最為近畿大郡}
初巴陵王休若為北徐州刺史以山隂張岱為諮議參軍行府州國事|{
	諸幼王臨州率置行府州事此命岱并巴陵國事行之}
後臨海王子頊為廣州豫章王子尚為揚州晉安王子勛為南兖州|{
	勛古勲翻}
岱歷為三府諮議三王行事與典籖主帥共事|{
	帥所類翻}
事舉而情不相失或謂岱曰主王既幼|{
	江左以來諸王出鎮僚屬呼為主王諸公府僚呼為主公}
執事多門而每能緝和公私|{
	緝一作輯}
云何致此岱曰古人言一心可以事百君我為政端平待物以禮悔吝之事無由而及明闇短長更是才用之多少耳|{
	少詩沼翻}
及子鸞為南徐州復以岱為别駕行事|{
	復扶又翻}
岱永之弟也 魏員外散騎常侍游明根等來聘明根雅之從祖弟也|{
	散悉亶翻騎奇寄翻從才用翻}
魏廣平王洛侯卒 十二月壬申以領軍將軍劉遵考為尚書右僕射 甲戌制民戶歲輸布四匹 是歲詔士族雜婚者皆補將吏|{
	雜婚謂與工商雜戶為婚也將即亮翻}
士族多避役逃亡乃嚴為之制捕得即斬之往往奔竄湖山為盗賊|{
	水則入湖陵則阻山皆依險而為盗賊}
沈懷文諫不聽

六年春正月癸未魏樂浪王萬壽卒|{
	樂浪音洛琅}
辛卯上初祀五帝於明堂大赦 丁未策秀孝於中堂|{
	秀孝秀才孝廉也}
揚州秀才顧灋對策曰源清則流潔神聖則刑全|{
	聖當作王音于况翻刑當作形}
躬化易於上風體訓速於草偃|{
	顧法對策之意欲帝謹厥身於宫帷衽席之間則可以化天下易以豉翻}
上覽之惡其諒也|{
	許慎說文曰諒信也諸儒說經者莫能易此義今此當以諒直為義參考經典則直自是直諒自是諒惡烏路翻}
投策於地 二月乙卯復百官禄|{
	文帝元嘉二十七年以軍興減内外百官俸三分之一繼而國有内難日不暇給今始復百官禄}
三月庚寅立皇子子元為邵陵王 初侍中沈懷文數以直諫忤旨|{
	數所角翻忤五故翻}
懷文素與顔竣周朗善|{
	竣七倫翻}
上謂懷文曰竣若知我殺之亦當不敢如此懷文嘿然侍中王彧言次稱竣朗人才之美懷文與相酬和|{
	彧於六翻和胡卧翻}
顔師伯以白上上益不悦上嘗出射雉|{
	自曹魏以來人主率好自出射雉射而亦翻}
風雨驟至懷文與王彧江智淵約相與諫會召入雉塲懷文曰風雨如此非聖躬所宜冒彧曰懷文所啓宜從智淵未及言上注弩作色曰卿欲效顔竣邪何以恒知人事|{
	恒戶登翻}
又曰顔竣小子恨不先鞭其面每上燕集在坐者皆令沉醉|{
	坐徂卧翻沉持林翻}
嘲謔無度|{
	謔迄却翻}
懷文素不飲酒又不好戲調|{
	好呼到翻調田聊翻}
上謂故欲異已謝莊嘗戒懷文曰卿每與人異亦何可久懷文曰吾少來如此|{
	少詩照翻}
豈可一朝而變非欲異物性所得耳上乃出懷文為晉安王子勛征虜長史|{
	子勛帶號征虜將軍以懷文為長史}
領廣陵太守|{
	子勛鎮南兖州故懷文以長史領廣陵太守}
懷文詣建康朝正|{
	朝正謂赴元正朝會也朝直遥翻}
事畢遣還|{
	還從宣翻又如字下同}
以女病求申期|{
	申重也申期重為之期也}
至是猶未發免官禁錮十年懷文賣宅欲還東|{
	懷文吳興人吳興在建康東}
上聞大怒收付廷尉丁未賜懷文死懷文三子澹淵冲行哭為懷文請命見者傷之|{
	澹徒覽翻為于偽翻}
柳元景欲救懷文言於上曰沈懷文三子塗炭不可見|{
	見視也言其在塗炭之中不堪著眼也}
願陛下速正其罪|{
	言速正其罪者婉而導之謂若正其罪當不至於死也}
上竟殺之 夏四月淑儀殷氏卒 |{
	考異曰殷淑儀南郡王義宣女也義宣敗後帝密取之假姓殷氏左右宣泄者多死或云貴妃是殷琰家人入義宣家義宣敗入宫今從宋書}
追拜貴妃諡曰宣上痛悼不已精神為之罔罔|{
	罔罔失志也若有若無也為于偽翻}
頗廢政事 五月壬寅太宰義恭解領司徒 六月辛酉東昌文穆公劉延孫卒|{
	沈約志廬陵郡有東昌國吳立隋開皇十一年省東昌入泰和縣}
庚午魏主如隂山 魏石樓胡賀畧孫反|{
	石樓胡即吐京胡也吐京有石樓山隋廢吐京郡為石樓縣唐屬隰州}
長安鎮將陸真討平之|{
	將即亮翻}
魏主命真城長蛇鎮|{
	長蛇鎮在南田縣東南有長蛇水唐隴州吳山縣即其地}
氐豪仇傉檀反|{
	傉古沃翻}
真討平之卒城而還|{
	卒子恤翻}
秋七月壬寅魏主如河西 乙未立皇子子雲為晉陵王是日卒諡曰孝 初晉庾冰議使沙門敬王者桓玄復述其議|{
	復扶又翻}
並不果行至是上使有司奏曰儒灋枝?名墨條分|{
	班志儒家助人君順隂陽明教化游文於六經之中留意於仁義之際祖述堯舜憲章文武宗師仲尼以重其言於道最為高法家信賞必罰以輔禮制名家正名名位不同禮亦異數墨家貴儉兼愛上賢右鬼非命尚同}
至於崇親嚴上厥猷靡爽唯浮圖為教反經提傳|{
	釋氏以自西天竺來者為經中國沙門譯而演其義者為傳提拈掇也傳直戀翻}
拘文蔽道在末彌扇夫佛以謙卑自牧忠䖍為道寧有屈膝四輩而簡禮二親|{
	謂拜四輩而不拜父母也釋氏有所謂戒外四聖佛一也菩薩二也圓覺三也聲聞四也亦謂之四輩}
稽顙耆臘而直體萬乘者哉|{
	沙門重戒臘以捨俗為僧之年為始耆老也直體謂不屈身也稽音啓}
臣等參議以為沙門接見比當盡䖍|{
	比毗志翻並也總也}
禮敬之容依其本俗九月戊寅制沙門致敬人主及廢帝即位復舊 乙未以尚書右僕射劉遵考為左僕射丹陽尹王僧朗為右僕射僧朗彧之父也 冬十月壬寅葬宣貴妃於龍山|{
	九域志江寧府有龍山山形似龍江寧府即建康}
鑿岡通道數十里民不堪役死亡甚衆|{
	亡逃亡也}
自江南葬埋之盛未之有也又為之别立廟|{
	古者宗廟之制妾袝於妾祖姑漢氏以來薄太后生文帝鉤弋夫人生昭帝皆就園置寢廟未嘗别立廟也史言帝溺於女寵縱情敗禮為于偽翻}
魏員外散騎常侍游明根等來聘 辛巳加尚書令柳元景司空 壬寅魏主還平城|{
	自河西還也}
南徐州從事史范陽祖冲之|{
	自漢以來諸州皆有從事史假佐}
上言何承天歷疎舛猶多|{
	何承天撰歷見一百二十四卷文帝元嘉二十一年}
更造新歷|{
	更子衡翻}
以為舊灋冬至日有定處未盈百載輒差二度|{
	載子亥翻}
今令冬至日度歲歲微差將來久用無煩屢改又子為辰首位在正北虛為北方列宿之中今歷上元日度發自虛一又日辰之號甲子為先今歷上元歲在甲子又承天灋日月五星各自有元今灋交會遲疾悉以上元歲首為始|{
	所謂今歷今灋皆祖冲之更造者也歷家分上元中元下元甲子各六十年凡一百八十年而下元甲子終又復於上元甲子}
上令善歷者難之不能屈會上晏駕不果施行|{
	難乃旦翻}


七年春正月丁亥以尚書右僕射王僧朗為太常衛將軍顔師伯為尚書僕射上每因宴集使羣臣自相謿訐以為樂|{
	謔人以成其過謂之謿發人之隂私謂之訐訐居謁翻樂音洛}
吏部郎江智淵素恬雅漸不會旨|{
	會合也}
嘗使智淵以王僧朗戲其子彧智淵正色曰恐不宜有此戲上怒曰江僧安癡人癡人自相惜僧安智淵之父也智淵伏席流涕|{
	古人畏聞父母名惟君所無私諱今人雖各有家諱然稠人廣座中往往不敢以為諱吾是以歎隋世以前人士猶為近古也}
由是恩寵大衰|{
	武帝以是怒江智淵何異孫皓之怒韋昭邪}
又議殷貴妃諡曰懷上以為不盡美甚銜之它日與羣臣乘馬至貴妃墓舉鞭指墓前石柱|{
	石柱墓表也}
謂智淵曰此上不容有懷字智淵益懼竟以憂卒 |{
	考異曰宋畧曰帝既以僧安辱智淵自是詆之無度智淵不堪其恥退而自殺今從宋書}
己丑以尚書令柳元景為驃騎大將軍開府儀同三司|{
	驃匹妙翻騎奇寄翻}
二月甲寅上巡南豫南兖二州丁卯校獵於烏江|{
	烏江縣始見於晉書屬淮南郡不記置立宋屬歷陽郡宋白曰晉太康六年於東城界置烏江縣校戶教翻}
壬戌大赦甲子如瓜步山壬申還建康 夏四月甲子詔自非臨軍戰陳並不得專殺其罪應重辟者|{
	陳讀曰陣辟毗亦翻}
皆先上須報|{
	光上其罪狀待報乃行刑此漢法也上時掌翻}
違犯者以殺人論 五月丙子詔曰自今刺史守宰動民興軍皆須手詔施行惟邊隅外警及姦舋内發變起倉猝者不從此例|{
	舋許覲翻}
戊辰以左民尚書蔡興宗|{
	曹魏置左民尚書}
左衛將軍袁粲為吏部尚書粲淑之兄子也|{
	袁淑死於元凶劭之難}
上好狎侮羣臣|{
	好呼報翻}
自太宰義恭以下不免穢辱常呼金紫光禄大夫王玄謨為老傖|{
	江南人呼中州人為傖王玄謨太原人也故呼之為老傖傖助庚翻}
僕射劉秀之為老慳顔師伯為齴|{
	齴魚蹇翻露齒也}
其餘短長肥瘦皆有稱目黄門侍郎宗靈秀體肥拜起不便每至集會多所賜與欲其瞻謝傾踣以為歡笑|{
	踣蒲北翻}
又寵一崑崙奴|{
	崑崙奴者言其狀似崑崙國人也崑崙國在林邑南崙盧昆翻}
令以杖擊羣臣尚書令柳元景以下皆不能免惟憚蔡興宗方嚴不敢侵媟|{
	媟私列翻}
顔師伯謂儀曹郎王耽之曰|{
	曹魏置二十三郎儀曹其一也}
蔡尚書常免昵戲去人實遠|{
	昵尼質翻}
耽之曰蔡豫章昔在相府亦以方嚴不狎武帝宴私之日未嘗相召|{
	蔡豫章興宗父廓也嘗為豫章太守故稱之相府謂武帝相晉時廓為司徒左長史也相息亮翻}
蔡尚書今日可謂能負荷矣|{
	左傳子產曰其父析薪其子不克負荷荷音下可翻又如字}
壬寅魏主如隂山 六月戊辰以秦郡太守劉德願為豫州刺史德願懷慎之子也上既葬殷貴妃數與羣臣至其墓|{
	數所角翻}
謂德願曰卿哭貴妃悲者當厚賞德願應聲慟哭撫膺擗踊涕泗交流|{
	膺胷也擗毘亦翻以手擊胷也詩注曰自月曰涕自鼻曰泗}
上甚悦故用豫州刺史以賞之|{
	用下當有為字}
上又令醫術人羊志哭貴妃志亦嗚咽極悲他日有問志者曰卿那得此副急淚志曰我爾日自哭亡妾耳|{
	史言上淫荒為下所侮弄}
上為人機警勇决學問博洽文章華敏省讀書奏能七行俱下|{
	省悉景翻行戶剛翻一注日間能了七行文義}
又善騎射|{
	騎奇寄翻}
而奢欲無度自晉氏渡江以來宮室草創朝晏所臨東西二堂而已晉孝武末始作清暑殿宋興無所增改上始大修宫室土木被錦繡嬖妾幸臣賞賜傾府藏壞高祖所居隂室|{
	被皮義翻藏徂浪翻壞音怪江左諸帝既崩以其所居殿為隂室藏諸御服}
於其處起玉燭殿與羣臣觀之牀頭有土障壁上挂葛燈籠麻蠅拂|{
	以葛為燈籠以麻為蠅拂}
侍中袁顗因盛稱高祖儉素之德上不答獨曰田舍公得此已為過矣|{
	周公無逸之書曰否則侮厥父母曰昔之人無間知宋孝武是也}
顗淑之兄子也 秋八月乙丑立皇子子孟為淮南王子產為臨賀王 丙寅魏主畋於河西九月辛巳還平城 庚寅以新安王子鸞兼司徒 丙申立皇子子嗣為東平王 冬十月癸亥以東海王禕為司空 己巳上校獵姑孰 魏員外散騎常侍游明根等來聘明根奉使三返上以其長者禮之有加|{
	明根連三年來聘長知兩翻}
十一月癸巳上習水軍於梁山十二月丙午如歷陽甲寅大赦 己未太宰義恭加尚書令 癸亥上還建康

八年春正月丁亥魏主立其弟雲為任城王|{
	任音壬}
戊子以徐州刺史新安王子鸞領司徒夏閏五月壬寅太宰義恭領太尉 上末年尤貪財利刺史二千石罷還必限使獻奉又以蒱戲取之|{
	蒱戲摴蒱之戲也}
要令罄盡乃止終日酣飲少有醒時|{
	少詩沼翻}
常憑儿昏睡或外有奏事即肅然整容無復酒態|{
	復扶又翻}
由是内外畏之莫敢弛惰庚申上殂於玉燭殿|{
	年三十五}
遺詔太宰義恭解尚書令加中書監以驃騎將軍南兖州刺史柳元景領尚書令入居城内|{
	入居臺城之内也建康無外城設六籬門而已百官第宅皆在臺城之外驃匹妙翻騎奇寄翻}
事無巨細悉關二公大事與始興公沈慶之參决若有軍旅悉委慶之尚書中事委僕射顔師伯外監所統委領軍將軍王玄謨|{
	舊制外監不隸領軍宜相統攝者自有别詔文帝元嘉十八年以趙伯符為領軍將軍始統領外監李延壽曰若徵兵動衆大興人役優劇遠近斷於外監之心延壽之言為宋末嬖倖專擅發也}
是日太子即皇帝位|{
	諱子業少字法師孝武帝長子也}
年十六大赦吏部尚書蔡興宗親奉璽綬|{
	璽斯氏翻綬音受}
太子受之傲惰無戚容興宗出告人曰昔魯昭不戚叔孫知其不終|{
	左傳魯襄公薨立昭公叔孫穆子曰是人也居喪而不哀在感而有嘉容是謂不度比葬三易衰衰衽如故衰於是昭公十九年矣猶有童心君子是以知其不終也}
家國之禍其在此乎|{
	為明年帝以狂暴見弑張本}
甲子詔復以太宰義恭録尚書事柳元景如開府儀同三司領丹陽尹解南兖州 六月丁亥魏主如隂山 秋七月己亥以晉安王子勛為江州刺史|{
	為明年子勛起兵張本勛古勲字}
柔然處羅可汗卒子予成立號受羅部真可汗|{
	魏收曰受羅部真魏言惠也可從刋入聲汗音寒}
改元永康部真帥衆侵魏|{
	帥讀曰率}
辛丑魏北鎮遊軍擊破之 壬寅魏主如河西高車五部相聚祭天衆至數萬魏主親往臨視之高車大喜 丙午葬孝武皇帝於景寧陵|{
	景寧陵在丹陽秣陵縣巖山}
廟號世祖 庚戌尊皇太后曰太皇太后皇后曰皇太后 乙卯罷南北二馳道|{
	世祖大明五年立南北二馳道自閶闔門至於朱雀門又自承明門至於玄武湖}
及孝建以來所改制度還依元嘉尚書蔡興宗於都座慨然謂顔師伯曰|{
	此都座謂尚書八座會坐之所猶今之都堂也}
先帝雖非盛德之主要以道始終三年無改古典所貴|{
	論語曰三年無改於父之道可謂孝矣}
今殯宫始撤山陵未遠而凡諸制度興造不論是非一皆刋削雖復禪代亦不至爾|{
	復扶又翻下復非復留同}
天下有識當以此窺人師伯不從太宰義恭素畏戴法興巢尚之等雖受遺輔政而引身避事由是政歸近習法興等專制朝權|{
	朝直遥翻}
威行近遠詔敇皆出其手尚書事無大小咸取决焉義恭與顔師伯但守空名而已|{
	義恭録尚書事師伯為僕射而尚書事决於法興等是守空名也}
蔡興宗自以職管銓衡|{
	興宗為吏部尚書}
每至上朝|{
	上時掌翻朝直遥翻下同}
輒為義恭陳登賢進士之意|{
	為於偽翻}
又箴規得失博論朝政義恭性恇橈|{
	恇怯也橈屈也恇去王翻橈奴教翻}
阿順法興恒慮失旨|{
	恒戶登翻}
聞興宗言輒戰懼無答興宗每奏選事|{
	選須絹翻選事選曹事也}
法興尚之等輒點定回換僅有在者興宗於朝堂謂義恭師伯曰主上諒陰不親萬幾|{
	陰音闇}
而選舉密事多被删改復非公筆|{
	被皮義翻復扶又翻下同}
亦不知是何天子意數與義恭等爭選事往復論執義恭法興皆惡之|{
	數所角翻惡烏路翻下同}
左遷興宗新昌太守|{
	吳孫皓建衡三年分交阯立新興郡晉武帝太康三年更名新昌郡屬交州五代志交州嘉寧縣舊置新昌郡}
既而以其人望復留之建康 丙辰追立何妃曰獻皇后|{
	何妃大明五年薨}
乙丑新安王子鸞解領司徒戴法興等惡王玄謨剛嚴八月丁卯以玄謨為南徐州刺史 王太后疾篤使呼廢帝帝曰病人間多鬼那可往太后怒謂侍者取刀來割我腹那得生如此寧馨兒|{
	寧相傳讀從去聲劉禹錫詩從平聲}
己丑太后殂 九月辛丑魏主還平城|{
	自河西還也}
癸卯以尚書左僕射劉遵考為特進右光禄大夫 乙卯葬文穆皇后于景寧陵|{
	王后從孝武帝謚當作武穆}
冬十二月壬辰以王畿諸郡為揚州|{
	大明三年以丹陽等六郡為王}


|{
	畿}
以揚州為東揚州|{
	以會稽為揚州亦見三年}
癸巳以豫章王子尚為司徒揚州刺史是歲青州移治東陽|{
	青州移治歷城見上卷孝建三年}
宋之境内凡有州二十二郡二百七十四縣千二百九十九戶九十四萬有奇|{
	此大較以沈約宋志為據沈約作志大較以是年為正然是年止二十一州耳沈志所謂二十二州以明帝泰始七年分交廣置越州足之而此時又已省司州盖止二十一州也揚州領丹陽吳興淮南宣城義興五郡東揚州領會稽東陽臨海永嘉新安五郡南徐州領南東海南琅邪晉陵吳南蘭陵南東莞淮陵臨淮南彭城南清河南高平南平昌南濟隂南濮陽南泰山濟陽南魯郡十七郡徐州領彭城沛下邳蘭陵東海東莞東安琅邪陽平濟隂北濟隂十二郡南兖州領廣陵海陵山陽盱眙泰郡南沛鍾離比沛臨江九郡兖州領泰山高平魯東平陽平濟北六郡南豫州領歷陽南譙廬江南汝隂南梁晉熙弋陽安豐南汝南南新蔡東郡南潁南潁川西汝隂南汝陽南陳留南陳左郡邊城左郡光城左郡十九郡豫川領汝南新蔡譙梁陳南頓潁川汝隂汝陽陳留馬頭十一郡江州領尋陽豫章鄱陽臨川廬陵安成南康南新蔡建安晉安十郡青州領齊濟南樂安高密平昌北海東萊太原長廣九郡冀州領廣川平原清河樂陵魏河間頓丘高陽勃海九郡司州領義陽隨陽安陸南汝南四郡荆州領南郡南平宜都巴東汶陽南義陽新興南河東建平長寧武寧十一郡郢州領江夏竟陵隨武陵天門巴陵武昌西陽八郡湘川領長沙衡陽桂陽零陵營陽湘東邵陵始興臨慶始安十郡雍州領襄陽南陽新野順陽京兆始平扶風南上洛河南廣平義成馮翊天水建昌華山北河南弘農十七郡梁州領漢中魏興新興新城上庸晉夀華陽新巴北巴西北隂平南隂平巴渠懷安宋熙白水南上洛北上洛安康南宕渠懷安二十郡秦州領武都略陽安固西京兆南太原南安馮翊隴西始平金城安定天水西扶風北扶風十四郡益州領蜀郡廣漢巴西梓潼巴郡遂寧江陽懷寧寧蜀越嶲汶山南隂平犍為始康晉熙晉原永寧安固南漢中北隂平武都新城南新巴南晉壽宋興南宕渠天水東江陽沈黎二十九郡寧州領建寧晉寧牂柯平蠻夜郎朱提南廣建都西平西河陽東河陽興寧興古梁水十五郡廣州領南海蒼梧晉康新寧永平欝林桂林高凉新會東官義安宋康綏康海昌宋熙寧浦晉興樂昌臨鄣十九郡交州領交阯武平新昌九真九德日南合浦義昌宋平九郡合二百六十八盖以新立百梁蘇永寧安昌富昌南流六郡足為二百七十四其間荒外有郡而無縣有縣而無戶口有尸數而無口數亦不能詳也奇居宜翻}
東方諸郡連歲旱饑|{
	東方諸郡謂三吳及浙江東五郡}
米一升錢數百建康亦至百餘錢餓死什六七

資治通鑑卷一百二十九
