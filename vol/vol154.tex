<!DOCTYPE html PUBLIC "-//W3C//DTD XHTML 1.0 Transitional//EN" "http://www.w3.org/TR/xhtml1/DTD/xhtml1-transitional.dtd">
<html xmlns="http://www.w3.org/1999/xhtml">
<head>
<meta http-equiv="Content-Type" content="text/html; charset=utf-8" />
<meta http-equiv="X-UA-Compatible" content="IE=Edge,chrome=1">
<title>資治通鑒_155-資治通鑑卷一百五十四_155-資治通鑑卷一百五十四</title>
<meta name="Keywords" content="資治通鑒_155-資治通鑑卷一百五十四_155-資治通鑑卷一百五十四">
<meta name="Description" content="資治通鑒_155-資治通鑑卷一百五十四_155-資治通鑑卷一百五十四">
<meta http-equiv="Cache-Control" content="no-transform" />
<meta http-equiv="Cache-Control" content="no-siteapp" />
<link href="/img/style.css" rel="stylesheet" type="text/css" />
<script src="/img/m.js?2020"></script> 
</head>
<body>
 <div class="ClassNavi">
<a  href="/24shi/">二十四史</a> | <a href="/SiKuQuanShu/">四库全书</a> | <a href="http://www.guoxuedashi.com/gjtsjc/"><font  color="#FF0000">古今图书集成</font></a> | <a href="/renwu/">历史人物</a> | <a href="/ShuoWenJieZi/"><font  color="#FF0000">说文解字</a></font> | <a href="/chengyu/">成语词典</a> | <a  target="_blank"  href="http://www.guoxuedashi.com/jgwhj/"><font  color="#FF0000">甲骨文合集</font></a> | <a href="/yzjwjc/"><font  color="#FF0000">殷周金文集成</font></a> | <a href="/xiangxingzi/"><font color="#0000FF">象形字典</font></a> | <a href="/13jing/"><font  color="#FF0000">十三经索引</font></a> | <a href="/zixing/"><font  color="#FF0000">字体转换器</font></a> | <a href="/zidian/xz/"><font color="#0000FF">篆书识别</font></a> | <a href="/jinfanyi/">近义反义词</a> | <a href="/duilian/">对联大全</a> | <a href="/jiapu/"><font  color="#0000FF">家谱族谱查询</font></a> | <a href="http://www.guoxuemi.com/hafo/" target="_blank" ><font color="#FF0000">哈佛古籍</font></a> 
</div>

 <!-- 头部导航开始 -->
<div class="w1180 head clearfix">
  <div class="head_logo l"><a title="国学大师官网" href="http://www.guoxuedashi.com" target="_blank"></a></div>
  <div class="head_sr l">
  <div id="head1">
  
  <a href="http://www.guoxuedashi.com/zidian/bujian/" target="_blank" ><img src="http://www.guoxuedashi.com/img/top1.gif" width="88" height="60" border="0" title="部件查字,支持20万汉字"></a>


<a href="http://www.guoxuedashi.com/help/yingpan.php" target="_blank"><img src="http://www.guoxuedashi.com/img/top230.gif" width="600" height="62" border="0" ></a>


  </div>
  <div id="head3"><a href="javascript:" onClick="javascript:window.external.AddFavorite(window.location.href,document.title);">添加收藏</a>
  <br><a href="/help/setie.php">搜索引擎</a>
  <br><a href="/help/zanzhu.php">赞助本站</a></div>
  <div id="head2">
 <a href="http://www.guoxuemi.com/" target="_blank"><img src="http://www.guoxuedashi.com/img/guoxuemi.gif" width="95" height="62" border="0" style="margin-left:2px;" title="国学迷"></a>
  

  </div>
</div>
  <div class="clear"></div>
  <div class="head_nav">
  <p><a href="/">首页</a> | <a href="/ShuKu/">国学书库</a> | <a href="/guji/">影印古籍</a> | <a href="/shici/">诗词宝典</a> | <a   href="/SiKuQuanShu/gxjx.php">精选</a> <b>|</b> <a href="/zidian/">汉语字典</a> | <a href="/hydcd/">汉语词典</a> | <a href="http://www.guoxuedashi.com/zidian/bujian/"><font  color="#CC0066">部件查字</font></a> | <a href="http://www.sfds.cn/"><font  color="#CC0066">书法大师</font></a> | <a href="/jgwhj/">甲骨文</a> <b>|</b> <a href="/b/4/"><font  color="#CC0066">解密</font></a> | <a href="/renwu/">历史人物</a> | <a href="/diangu/">历史典故</a> | <a href="/xingshi/">姓氏</a> | <a href="/minzu/">民族</a> <b>|</b> <a href="/mz/"><font  color="#CC0066">世界名著</font></a> | <a href="/download/">软件下载</a>
</p>
<p><a href="/b/"><font  color="#CC0066">历史</font></a> | <a href="http://skqs.guoxuedashi.com/" target="_blank">四库全书</a> |  <a href="http://www.guoxuedashi.com/search/" target="_blank"><font  color="#CC0066">全文检索</font></a> | <a href="http://www.guoxuedashi.com/shumu/">古籍书目</a> | <a   href="/24shi/">正史</a> <b>|</b> <a href="/chengyu/">成语词典</a> | <a href="/kangxi/" title="康熙字典">康熙字典</a> | <a href="/ShuoWenJieZi/">说文解字</a> | <a href="/zixing/yanbian/">字形演变</a> | <a href="/yzjwjc/">金 文</a> <b>|</b>  <a href="/shijian/nian-hao/">年号</a> | <a href="/diming/">历史地名</a> | <a href="/shijian/">历史事件</a> | <a href="/guanzhi/">官职</a> | <a href="/lishi/">知识</a> <b>|</b> <a href="/zhongyi/">中医中药</a> | <a href="http://www.guoxuedashi.com/forum/">留言反馈</a>
</p>
  </div>
</div>
<!-- 头部导航END --> 
<!-- 内容区开始 --> 
<div class="w1180 clearfix">
  <div class="info l">
   
<div class="clearfix" style="background:#f5faff;">
<script src='http://www.guoxuedashi.com/img/headersou.js'></script>

</div>
  <div class="info_tree"><a href="http://www.guoxuedashi.com">首页</a> > <a href="/SiKuQuanShu/fanti/">四库全书</a>
 > <h1>资治通鉴</h1> <!--         下载:【右键另存为】即可 --></div>
  <div class="info_content zj clearfix">
  
<div class="info_txt clearfix" id="show">
<center style="font-size:24px;">155-資治通鑑卷一百五十四</center>
    資治通鑑卷一百五十四 宋 司馬光 撰<br />
<br />
  胡三省 音註<br />
<br />
  梁紀十【上章閹茂一年】<br />
<br />
  高祖武皇帝十<br />
<br />
  中大通二年春正月己丑魏益州刺史長孫壽梁州刺史元儁等遣將擊嚴始欣斬之蕭玩等亦敗死【玩援始欣見上卷上年長知兩翻將即亮翻】失亡萬餘人 辛亥魏東徐州城民呂文欣等殺刺史元大賓據城反【魏孝昌昌年置東徐州於下邳】魏遣都官尚書平城樊子鵠討之二月甲寅斬文欣 万俟醜奴侵擾關中【万莫北翻俟渠之翻】魏爾朱榮遣武衛將軍賀拔岳討之岳私謂其兄勝曰醜奴勍敵也【勍其京翻】今攻之不勝固有罪勝之讒嫉將生勝曰然則奈何岳曰願得爾朱氏一人為帥而佐之【帥所類翻】勝為之言於榮【為于偽翻】榮悦以爾朱天光為使持節都督二雍二岐諸軍事驃騎大將軍雍州刺史【後魏雍州治長安北雍州治蕐原縣東雍州治鄭縣岐州治扶風雍縣南岐州治河池故道縣使疏吏翻雍於用翻驃匹妙翻騎奇寄翻】以岳為左大都督又以征西將軍代郡侯莫陳悦為右大都督【侯莫陳其先魏之别部也居庫斛真水世為渠帥遂以為氏其後鎮代郡武川因家焉】竝為天光之副以討之天光初行唯配軍士千人發洛陽以西路次民馬以給之時赤水蜀賊斷路【水經注赤水在鄭縣北即山海經之灌水也北注於渭蜀賊本蜀人之遷關中者乘亂相聚為賊斷丁管翻】詔侍中楊侃先行慰諭并税其馬【蕐隂諸楊仕魏奕世貴顯關西所歸重故使之先行慰諭也】賊持疑不下軍至潼關天光不敢進岳曰蜀賊鼠竊公尚遲疑若遇大敵將何以戰天光曰今日之事一以相委岳遂進擊蜀於渭北破之獲馬二千匹簡其壯健以充軍士又税民馬合萬餘匹以軍士尚少【少詩沒翻】淹留未進榮怒遣騎兵參軍劉貴乘驛至軍中責天光杖之一百以軍士二千人益之三月醜奴自將其衆圍岐州遣其大行臺尉遲菩薩【李延壽曰其先魏之别號尉遲部因以為氏尉音欝菩薄胡翻薩桑葛翻】僕射万俟仵自武功南渡渭攻圍趣柵【仵疑古翻 考異曰北史作万俟行醜今從周書】天光使賀拔岳將千騎救之菩薩等已拔柵而還【還從宣翻又如字】岳故殺掠其吏民以挑之菩薩率步騎二萬至渭北【挑徒了翻帥讀曰率】岳以輕騎數十自渭南與菩薩隔水而語稱揚國威菩薩令省事傳語【省事蓋猶今之通事兩敵相向使之往來通傳言語省悉井翻】岳怒曰我與菩薩語卿何人也射殺之【射而亦翻】明曰復引百餘騎隔水與賊語【程扶又翻】稍引而東至水淺可涉之處岳即馳馬東出賊以為走乃棄步兵輕騎南渡渭追岳岳依横岡設伏兵以待之賊半度岡東岳還兵擊之賊兵敗走【岳既還兵擊賊伏兵又故敗走】岳下令賊下馬者勿殺賊悉投馬俄獲三千人馬亦無遺遂擒菩薩仍度渭北降步卒萬餘竝收其輜重【降戶江翻重直用翻】醜奴聞之棄岐州北走安定【走音奏】置柵於平亭天光方自雍至岐與岳合【平亭在涇州北自雍至岐自雍州至岐州也】夏四月天光至汧渭之間【汧水出汧縣西北而入於渭汧口堅翻】停軍牧馬宣言天時將熱未可行師俟秋凉更圖進止獲醜奴覘者縱遣之【覘醜亷翻又丑艶翻】醜奴信之散衆耕於細川【據令狐德棻後周書百里細川在岐州北又據元豐九域志涇州靈臺縣冇百里鎮蓋即細川之地細川平亭當亦相近】使其太尉侯伏侯元進將兵五千據險立柵【侯伏侯虜三字姓將即亮翻】其餘千人以下為柵者甚衆天光知其勢分晡時密嚴諸軍相繼俱黎明圍元進大柵拔之所得俘囚一皆縱遣諸柵聞之皆降【唐末高仁厚平阡能等亦用此術降戶江翻下同】天光晝夜徑進抵安定城下賊涇州刺史侯幾長貴以城降【侯幾虜複姓魏書官氏志内入諸姓有俟幾氏俟侯字相近】醜奴棄平亭走欲趣高平【九域志鎮戎軍古高平地也趣七喻翻】天光遣賀拔岳輕騎追之丁卯及於平凉賊未成列直閤代郡侯莫陳崇單騎入賊中於馬上生擒醜奴因大呼衆皆披靡【呼火故翻披普彼翻】無敢當者後騎益集賊衆崩潰遂大破之天光進逼高平城中執送蕭寶寅以降【万俟醜奴胡琛之將也普通六年破魏將崔延伯其衆始盛蕭寶寅大通元年叛魏至二年敗奔醜奴及是皆平】壬申以吐谷渾王佛輔為西秦河二州刺史【吐從暾入聲谷音浴】甲戌魏以關中平大赦万俟醜奴蕭寶寅至洛陽置<br />
<br />
  閶闔門外都街之中士女聚觀凡三日丹陽王蕭贊表請寶寅之命【贊以寶寅為叔父故請其命】吏部尚書李神儁黄門侍郎高道穆素與寶寅善欲左右之【左右讀曰佐佑】言於魏主曰寶寅叛逆事在前朝【朝音直遙翻】會應詔王道習自外至【應詔猶漢之待詔也】帝問道習在外所聞對曰惟聞李尚書高黄門與蕭寶寅周欵【周至也密也欵愛也】竝居得言之地必能全之且二人謂寶寅叛逆在前朝寶寅為醜奴太傅豈非陛下時邪賊臣不翦法欲安施帝乃賜寶寅死於駞牛署【邪音耶後魏官有駞牛都尉署者其寺舍也五代志太僕寺之屬有駞牛署掌飼駞騾驢牛有令丞】斬醜奴於都市 六月丁巳帝復以魏汝南王悦為魏王【復扶又翻考異曰梁帝紀中大通元年正月甲子魏汝南王悦求還本國許之二年六月丁巳遣悦還北為魏主按魏】<br />
<br />
  【書悦傳悦未嘗歸魏復入梁今刪去元年事】 戊寅魏詔胡氏親屬受爵於朝者皆黜為民【謂靈后親屬也朝直遙翻】 庚申以魏降將范遵為安北將軍司州牧從魏王悦北還【范遵魏北海王顥之舅蓋與顥同來奔降戶江翻將即亮翻】万俟醜奴既敗自涇豳以西至靈州【後魏滅赫連以赫連果城置薄骨律鎮至孝昌中改鎮為靈州杜佑曰薄骨律鎮今靈武郡富平今迴樂縣唐靈州治迴樂括地志云薄骨律鎮城在河渚之中隨水上下未嘗陷沒故號靈州也】賊黨皆降於魏唯所署行臺万俟道洛帥衆六千逃入山中不降【降戶江翻帥讀曰率下同】時高平大旱爾朱天光以馬乏草退屯城東五十里遣都督長孫邪利帥二百人行原州事以鎮之【魏太延二年置高平鎮正光五年改曰原州治高平城領高平長城二郡】道洛濳與城民通謀掩襲邪利并其所部皆殺之天光帥諸軍赴之道洛出戰而敗帥其衆西入牽屯山【班志开頭山在安定郡涇陽縣西涇水所出師古注曰开音牽此山在今靈州東南俗語訛謂之开屯山杜佑曰牽屯山在今原州高平縣】據險自守爾朱榮以天光失邪利不獲道洛復遣使杖之一百【復扶又翻使疏吏翻】以詔書黜天光為撫軍將軍雍州刺史降爵為侯天光追擊道洛於牽屯道洛敗走入隴【隴隴山也】歸略陽賊帥王慶雲【晉武帝分天水置略陽郡隋廢為隴城縣屬秦州 考異曰魏帝紀作白馬龍涸胡王慶雲今從爾朱天光傳帥音所類翻】道洛驍果絶倫【驍堅堯翻】慶雲得之甚喜謂大事可濟遂稱帝於水洛城【水經注水洛水導源隴山西逕水洛亭西南注略陽川九域志水洛城在德順軍西南一百里范仲淹曰朝那之西秦亭之東有水洛城】置百官以道洛為大將軍秋七月天光帥諸軍入隴至水洛城慶雲道洛出戰天光射道洛中臂【射而亦翻中竹仲翻】失弓還走拔其東城賊併兵趣西城【趣七喻翻】城中無水衆渴乏有降者言慶雲道洛欲突走天光恐失之乃遣人招諭慶雲使早降【降戶江翻】曰若未能自决當聽諸人今夜共議明晨早報慶雲等冀得少緩因待夜突出【少詩沒翻】乃報曰請俟明日天光因使謂曰知須水【須者意所欲也】今相為小退【為於偽翻】任取澗水飲之賊衆悦無復走心天光密使軍士多作木槍各長七尺【此即拒馬槍也杜佑曰拒馬槍以木徑二尺長短隨事十字鑿孔縱横安檢長丈鋭其端以塞要路】昏後繞城布列要路加厚又伏人槍中備其衝突兼令密縛長梯於城北其夜慶雲道洛果馳馬突出遇槍馬各傷倒伏兵起即時擒之軍士緣梯入城餘衆皆出城南遇槍而止窮窘乞降【降戶江翻】丙子天光悉收其仗而阬之死者萬七千人分其家口於是三秦河渭瓜凉鄯州皆降【三秦秦東秦南秦也河州乞伏之地也魏太武真君六年置枹罕鎮後改為河州領金城武始洪和臨洮郡渭州領隴西南安南安陽廣寧郡瓜州即古敦煌之地鄯州禿髪氏之地漢金城西部都尉所統也師古曰瓜州即左傳所云允姓之戎居于瓜州者也其地今猶出大瓜長者狐入瓜中食之首尾不出】天光頓軍略陽詔復天光官爵尋加侍中儀同三司以賀拔岳為涇州刺史侯莫陳悦為渭州刺史秦州城民謀殺刺史駱超南秦州城民謀殺刺史辛顯超顯皆覺之走歸天光天光遣兵討平之步兵校尉宇文泰從賀拔岳入關以功遷征西將軍行原州事時關隴彫弊泰撫以恩信民皆感悦曰早遇宇文使君吾輩豈從亂乎【為宇文泰得賀拔岳之衆以創大業於關西張本】 八月庚戍上餞魏王悦於德陽堂遣兵送至境上 【考異曰悦傳云立為魏主號年更興衍遣其將軍王僧辯送至境上以冀侵逼按僧辯傳未嘗送悦蓋王弁耳】 魏爾朱榮雖居外藩遙制朝政【朝直遙翻】樹置親黨布列魏主左右伺察動静大小必知【伺相吏翻】魏主雖受制於榮然性勤政事朝夕不倦數親覽辭訟理寃獄【數所角翻】榮聞之不悦【史言魏主不能養晦】帝又與吏部尚書李神儁議清治選部【治直之翻選須絹翻下同】榮嘗關補曲陽縣令【據榮傳即上曲陽縣也漢晉屬常山郡後魏屬中山郡關補者先補授而後關吏部五代志趙州鼓城縣舊曰曲陽劉昫曰漢上曲陽縣隋改曰恒陽唐元和十五年復曰曲陽趙州之曲陽下曲陽也】神儁以階懸不奏【言階級相去懸絶其人不應補為縣令】别更擬人榮大怒即遣所補者往奪其任神儁懼而辭位榮使尚書左僕射爾朱世隆攝選榮啓北人為河南諸州帝未之許太宰天穆入見面諭【見賢遍翻】帝猶不許天穆曰天柱既有大功為國宰相若請普代天下官恐陛下亦不得違之如何啓數人為州遽不用也帝正色曰天柱若不為人臣朕亦須代如其猶存臣節無代天下百官之理榮聞之大恚恨曰天子由誰得立今乃不用我語爾朱皇后性妬忌屢致忿恚帝遣爾朱世隆語以大理【恚於避翻語牛倨翻大理謂事理之大致也】后曰天子由我家置立今便如此我父本即自作今亦復决【决判也謂天下事有判决也復扶又翻】世隆曰止自不為【止當作正】若本自為之臣今亦封王矣帝既外逼於榮内逼皇后恒快怏不以萬乘為樂【恒音常怏於兩翻乘繩正翻樂音洛下同】惟幸寇盗未息欲使與榮相持及關隴既定告捷之日乃不甚喜謂尚書令臨淮王彧曰即今天下便是無賊彧見帝色不悦曰臣恐賊平之後方勞聖慮帝畏餘人怪之還以他語亂之曰然撫寧荒餘【荒餘謂兵荒之餘民也】彌成不易【易以䜴翻】榮見四方無事奏稱參軍許周勸臣取九錫臣惡其言已斥遣令去榮時望得殊禮故以意諷朝廷帝實不欲與之因稱歎其忠榮好獵【惡烏路翻好呼報翻】不捨寒暑列圍而進令士卒必齊壹雖遇險阻不得違避一鹿逸出必數人坐死有一卒見虎而走榮謂曰汝畏死邪即斬之自是每獵士卒如登戰塲嘗見虎在窮谷中榮令十餘人空手搏之毋得損傷死者數人卒擒得之【魏道武帝因熊而謝于栗磾爾朱榮反是嗜殺人者烏能定天下邪卒音子恤翻】以此為樂其下甚苦之大宰天穆從容謂榮曰【樂音洛從于容翻】大王勲業已盛四方無事唯宜脩政養民順時蒐狩【禮春蒐夏苗秋獮冬狩杜預曰蒐索擇取不孕者苖為苖除害也獮殺也以殺為名順秋氣也狩圍守也冬物畢成獲則取之無所擇也】何必盛夏驅逐感傷和氣榮攘袂曰靈后女主不能自正推奉天子乃人臣常節葛榮之徒本皆奴才乘時作亂譬如奴走擒獲即已頃來受國大恩未能混壹海内何得遽言勳業如聞朝士猶自寛縱今秋欲與兄戒勒士馬校獵嵩高令貪汙朝貴入圍搏虎【朝直遙翻】仍出魯陽歷三荆悉擁生蠻北填六鎮【杜佑曰北荆州今即伊陽縣東荆州後改曰淮州今淮安郡荆州今南陽郡余按榮言出魯陽則已越伊陽而南矣五代志舂陵郡後魏置南荆州當以此足三荆之數生蠻謂諸蠻戶之未附於魏者六鎮叛亂鎮戶荒殘故欲填之】囘軍之際掃平汾胡【稽胡皆居汾州界謂之汾胡】明年簡練精騎分出江淮蕭衍若降乞萬戶侯【騎奇計翻降戶江翻乞丘計翻與也】如其不降以數千騎徑度縛取然後與兄奉天子巡四方乃可稱勳耳今不頻獵兵士懈怠安可復用也【懈七隘翻復扶又翻】城陽王徽之妃帝之舅女侍中李彧延寔之子帝之姊婿也徽彧欲得權寵惡榮為己害日毁榮於帝勸帝除之帝懲河隂之難【河隂之難事見一百五十二卷惡烏路翻難乃旦翻】恐榮終難保由是密有圖榮之意侍中楊侃尚書右僕射元羅亦預其謀【元羅義之弟也】會榮請入朝欲視皇后㝃乳【㝃與免同又音晩師古曰免乳為產子也乳人喻翻唐韻曰子母相解曰免】徽等勸帝因其入刺殺之【刺七亦翻】唯膠東侯李侃晞濟隂王暉業言榮若來必當有備恐不可圖【濟子禮翻】又欲殺其黨與兵拒之帝疑未定而洛陽人懷憂懼中書侍郎郉子才之徒已避之東出【郉邵字子才避魏主兄彭城王邵諱故以字行本傳云少時有避遂不行名】榮乃遍與朝士書相任去留中書舍人温子昇以書呈帝帝恒望其不來【恒戶登翻】及見書以榮必來色甚不悦子才名邵以字行巒之族弟也【邢巒事魏宣武帝屢經將領有功 考異曰北史邢巒卷首排目云族孫臧劭而卷中乃云巒叔祖祐祐從祖虯虯子臧劭魏書亦云巒從祖祐然則臧劭乃巒族弟非族孫也】時人多以字行者舊史皆因之武衛將軍奚毅建義初往來通命【事見二百五十二卷大通二年】帝每期之甚重然猶以榮所親信不敢與之言情毅曰若必有變臣寧死陛下不能事契胡【爾朱氏契胡種也契欺訖翻】帝曰朕保天柱無異心亦不忘卿忠欵【欵誠也】爾朱世隆疑帝欲為變乃為匿名書自牓其門云天子與楊侃高道穆等為計欲殺天柱取以呈榮榮自恃其彊不以為意手毁其書唾地曰世隆無膽誰敢生心榮妻北鄉長公主亦勸榮不行【榮妻非元氏也以榮功封北鄉長公主 考異作鄉郡長公主注曰北史世隆傳作北鄉郡公主今從魏帝紀按考異作鄉郡長公主是也通鑑作北鄉長公主傳寫之誤耳五代志上黨郡鄉縣石勒置武鄉郡後魏去】<br />
<br />
  【武字為鄉郡證以魏收志無北鄉郡則從鄉郡為是唾吐臥翻長知兩翻】榮不從是月榮將四五千騎發幷州【將即亮翻騎奇計翻】時人皆言榮反又云天子必當圖榮九月榮至洛陽 【考異曰魏帝紀曰辛卯榮天穆自晉陽來朝按北史九月初榮至京十五日天穆至是月甲戌朔辛卯乃十八日非也】帝即欲殺之以太宰天穆在幷州恐為後患故忍未幷召天穆有人告榮云帝欲圖之榮即具奏帝曰外人亦言王欲害我豈可信之於是榮不自疑每入謁帝從人不過數十【從才用翻】又皆挺身不持兵伏【挺徒頂翻直也】帝欲止城陽王徽曰縱不反亦何可耐【耐忍也】况不可保邪先是長星出中台掃大角恒州人高榮祖頗知天文榮問之對曰除舊布新之象也【三台中台上星為諸侯三公大角者天王座也傳曰彗所以除舊布新先悉薦翻恒戶登翻】榮甚悦榮至洛陽行臺郎中李顯和曰天柱至那無九錫安須王自索也【李顯和蓋為幷肆九州行臺郎中時從榮至洛陽索山客翻】亦是天子不見機都督郭羅察曰今年真可作禪文【河隂之難榮已募朝士作禪文故羅察云然郭羅察即郭羅刹】何但九錫參軍禇光曰人言幷州城上有紫氣何慮天柱不應之榮下人皆陵侮帝左右無所忌憚故其事皆上聞【皆上時掌翻】奚毅又見帝求間【求間即請間也】帝即下明光殿與語知其至誠乃召城陽王徽及楊侃李彧告以毅語榮小女適帝兄子陳留王寛榮嘗指之曰我終得此壻力徽以白帝曰榮慮陛下終為己患脱有東宫必貪立孩幼【孩何開翻】若皇后不生太子則立陳留耳帝夢手持刀自割落十指惡之【惡烏路翻】告徽及楊侃徽曰蝮蛇螫手壯士解腕【螫音釋腕烏貫翻】割指亦是其類乃吉祥也戊子天穆至洛陽帝出迎之榮與天穆竝從入西林園宴射榮奏曰近來侍官皆不習武陛下宜將五百騎出獵因省辭訟【將即亮翻騎奇計翻省悉井翻】先是奚毅言榮欲因獵挾天子移都【先悉薦翻】由是帝益疑之辛卯帝召中書舍人溫子昇告以殺榮狀并問以殺董卓事子昇具道本末帝曰王允若即赦涼州人必不應至此【董卓王允事見六十卷漢獻帝初平三年】良久語子昇曰【語牛倨翻】朕之情理卿所具知死猶須為况不必死吾寧為高貴鄉公死不為常道鄉公生【謂曹魏高貴鄉公欲誅司馬昭不克而死常道鄉公禪位于晉而生也】帝謂殺榮天穆即赦其黨皆應不動應詔王道習曰爾朱世隆司馬子如朱元龍特為榮所委任具知天下虛實謂不宜留徽及楊侃皆曰若世隆不全仲遠天光豈有來理【爾朱仲遠時鎮徐州天光時鎮關隴】帝亦以為然徽曰榮腰間常有刀或能狼戾傷人【狼當作很孟子樂歲粒米狼戾猶言狼籍也非此義】臨事願陛下起避之乃伏侃等十餘人於明光殿東其日榮與天穆竝入坐食未訖起出侃等從東階上殿【上時掌翻】見榮天穆已至中庭事不果壬辰帝忌日癸巳榮忌日【親喪之日為忌日禮曰忌日不樂】甲午榮暫入即詣陳留王家飲酒極醉遂言病動頻日不入帝謀頗泄世隆又以告榮且勸其速榮輕帝以為無能為曰何怱怱預帝謀者皆懼帝患之城陽王徽曰以生太子為辭榮必入朝因此斃之帝曰后懷孕始九月可乎【朝直遙翻孕以正翻】徽曰婦人不及期而產者多矣彼必不疑帝從之戊戍帝伏兵於明光殿東序聲言皇子生遣徽馳騎至榮第告之【騎奇計翻】榮方與上黨王天穆慱徽脱榮帽懽舞盤旋【唐李太白詩云脱君帽為君笑脱帽懽舞蓋夷禮也】兼殿内文武傳聲趣之【趣讀曰促】榮遂信之與天穆俱入朝帝聞榮來不覺失色中書舍人溫子昇曰陛下色變帝連索酒飲之【酒能變貌又能張膽故連索飲之索山客翻】帝令子昇作赦文既成執以出遇榮自外入問是何文書子昇顔色不變曰勑榮不取視而入帝在東序下西向坐榮天穆在御榻西北南向坐徽入始一拜榮見光禄少卿魯安典御李侃晞等抽刀從東戶入【漢九卿惟正卿一人魏高祖太和十一年始各置少卿一人典御嘗食典御也少詩詔翻】即起趨御座【趨七喻翻】帝先横刀膝下遂手刃之安等亂砍榮與天穆同時俱死榮子菩提及車騎將軍爾朱陽覩等三十人從榮入宫亦為伏兵所殺【菩薄乎翻】帝得榮手板上有數牒啓皆左右去留人名非其腹心者悉在出限【出不使在帝左右】帝曰豎子若過今日遂不可制於是内外喜譟聲滿洛陽城百僚入賀帝登閶闔門下詔大赦遣武衛將軍奚毅前燕州刺史崔淵將兵鎮北中【燕因肩翻將即亮翻】是夜爾朱世隆奉北鄉長公主帥榮部曲焚西陽門出屯河隂【西陽門即洛陽城西明門】衛將軍賀拔勝與滎黨田怡等聞榮死奔赴榮第時宫殿門猶未加嚴防怡等議即攻門勝止之曰天子既行大事必當有備吾輩衆少何可輕爾【少時沼翻】但得出城更為他計怡乃止及世隆等走勝遂不從【考異曰周書及北史云勝復從世隆至河橋勝以為臣無讎君之義遂勒所部還都莊帝大悦今從魏書】帝甚嘉之朱瑞雖為榮所委而善處朝廷之間【朱瑞本榮之行臺郎中榮定魏主于洛陽以瑞為黄門侍郎兼中書舍人處昌呂翻】帝亦善遇之故瑞從世隆走而中道逃還榮素厚金紫光禄大夫司馬子如榮死子如自宫中突出至滎第棄家隨榮妻子走出城世隆即欲還北子如曰兵不厭詐今天下恟恟唯彊是視【恟許勇翻】當此之際不可以弱示人若亟北走恐變生肘腋【腋音亦】不如分兵守河橋還軍向京師出其不意或可成功假使不得所欲亦足示有餘力使天下畏我之彊不敢叛散世隆從之己亥攻河橋擒奚毅等殺之據北中城魏朝大懼遣前華陽太守段育慰諭之【魏分漢中之沔陽西縣置華陽郡以其地在華山之南也朝直遙翻華戶化翻】世隆斬首以徇魏以雍州刺史爾朱天光為侍中儀同三司【雍於用翻】以司空楊津為都督并肆等九州諸軍事驃騎大將軍并州刺史兼尚書令北道行臺經略河汾【驃匹妙翻騎奇計翻】榮之入洛也以高敖曹自隨禁於駝牛署【爾朱榮誘拘高敖曹事見一百五十二卷大通二年】榮死帝引見勞勉之【見賢遍翻勞力到翻】兄乾自東冀州馳赴洛陽【魏孝昌末葛榮作亂高翼聚衆河濟間魏因置東冀州以翼為刺史蓋因劉宋先置冀州於河濟間而置東冀州以别河北之冀州也翼乾之族也按後以翼為乾之父】帝以乾為河北大使【使疏吏翻】敖曹為直閣將軍使歸招集鄉曲為表裏形援帝親送之於河橋【敖曹兄弟歸鄉里路當東出河橋在洛城北帝不應送之於此河橋二字意必有誤】舉酒指水曰卿兄弟冀部豪傑能令士卒致死京城儻有變可為朕河上一揚塵乾垂涕受詔敖曹援劒起舞誓以必死【為于偽翻下亦為吾為為陛下同援于元翻】冬十月癸卯朔世隆遣爾朱拂律歸【考異曰魏書無拂律歸名伽藍記有之按爾朱度律時在世隆所或者拂律歸即度律也】將胡騎一千皆白服來至郭下索太原王尸【榮迎立敬宗封太原王將即亮翻騎奇計翻索山客翻】帝升大夏門望之【洛陽城北有大夏廣莫二門夏戶雅翻】遣主書牛法尚謂之曰太原王立功不終隂圖舋逆【舋許覲翻】王法無親【言法在必行雖親無赦也】已正刑書罪止榮身餘皆不問卿等若降官爵如故拂律歸曰臣等隨太原王入朝忽致寃酷今不忍空歸願得太原王尸生死無恨因涕泣哀不自勝【勝音升】羣胡皆慟哭聲振城邑帝亦為之愴然遣侍中朱瑞齎鐵劵賜世隆世隆謂瑞曰太原王功格天地赤心奉國長樂不顧信誓枉加屠害【魏敬宗本封長樂王樂音洛】今日兩行鐵字何足可信【行戶剛翻】吾為太原王報讎終無降理瑞還白帝帝即出庫物置城西門外募敢死之士以討世隆一日即得萬人與拂律歸等戰於郭外拂律歸等生長戎旅【長知兩翻】洛陽之人不習戰鬭屢戰不克甲辰以前車騎大將軍李叔仁為大都督帥衆討世隆【帥讀曰率】戊申皇子生大赦以中書令魏蘭根兼尚書左僕射為河北行臺定相殷三州皆稟蘭根節度【相悉亮翻】爾朱氏兵猶在城下帝集朝臣博議皆恇懼不知所出通直散騎常侍李苖奮衣起曰【朝直遙翻恇去王翻散悉亶翻騎奇寄翻下同】今小賊唐突如此朝廷有不測之危正是忠臣烈士效節之日臣雖不武請以一旅之衆為陛下徑斷河橋【杜預曰五百人為一旅斷丁管翻】城陽王徽高道穆皆以為善帝許之乙卯苗募人從馬渚上流乘船夜下去橋數里縱火船焚河橋倏忽而至爾朱氏兵在南岸者望之爭橋北渡俄而橋絶溺死者甚衆苗將百許人泊於小渚以待南援【溺奴狄翻將即亮翻下同】官軍不至爾朱氏就擊之左右皆盡苗赴水死帝傷惜之贈車騎大將軍儀同三司封河陽侯謚曰忠烈世隆亦收兵北遁丙辰詔行臺源子恭將步騎一萬出西道楊昱將募士八千出東道以討之【募士即洛城西門外所募者也】子恭仍鎮太行丹谷築壘以防之【水經注丹水出上黨高都縣故城東北阜下東南流注於丹谷晉書地道記曰縣有太行關丹溪為關之東谷塗自此去不復由關矣行戶剛翻 考異曰伽藍記云源子恭楊寛領步騎三萬鎮河内今從魏書】世隆至建州【慕容永分上黨置建興郡魏真君元年省和平五年復置郡永安中罷郡置建州治高都城領高都長平安平恭寧郡據五代志建州即唐澤州之地】刺史陸希質閉城拒守世隆攻拔之殺城中人無遺類以肆其忿唯希質走免詔以前東荆州刺史元顯恭為晉州刺史【魏孝昌中置唐州建義元年改曰晉州治白馬城領平陽西河南絳北絳永安北五城定陽西平城敷城河西五城冀氏義寧郡】兼尚書左僕射西道行臺 魏東徐州刺史廣牧斛斯椿【廣牧縣漢朔方東部都尉治所也魏省朔方以廣牧縣屬新興郡魏收志屬朔州附化郡考椿傳椿廣牧富昌人則又似廣牧自為一郡也斛斯虜複姓】素依附爾朱榮榮死椿懼聞汝南王悦在境上乃帥部衆棄州歸悦【帥讀曰率下同】悦授椿侍中大將軍司空封靈丘郡公又為大行臺前驅都督 汾州刺史爾朱兆聞榮死自汾州帥騎據晉陽【騎奇計翻】世隆至長子【考異曰魏帝紀云世隆停建興之高都今從世隆傳】兆來會之壬申共推太原太守行并州事長廣王曄即皇帝位【守式又翻】大赦改元建明曄英之弟子也【中山王英著功於太和正始之間】以兆為大將軍進爵為王世隆為尚書令賜爵樂平王加太傅司州牧又以榮從弟度律為太尉賜爵常山王【從才用翻】世隆兄天柱長史彦伯為侍中徐州刺史仲遠為車騎大將軍兼尚書左僕射三徐州大行臺仲遠亦起兵向洛陽【三徐州徐州治彭城北徐州治琅邪永安二年置領東泰山琅邪二郡東徐州治下邳此皆長廣王所除授】爾朱天光之克平凉也宿勤明逹請降【宿勤明逹與万俟醜奴皆胡琛將也降戶江翻】既而復叛北走天光遣賀拔岳討之明逹奔東夏【東夏唐之延州夏戶雅翻】岳聞爾朱榮死不復窮追【復扶又翻】還涇州以待天光天光與侯莫陳悦亦下隴與岳謀引兵向洛魏敬宗使朱瑞慰諭天光天光與岳謀欲令帝外奔而更立宗室【更工衡翻】乃頻啟云臣實無異心唯欲仰奉天顔以申宗門之罪又使其下僚屬啓云天光密有異圖願思勝筭以防之【天光設兩端以疑魏朝】范陽太守盧文偉誘平州刺史侯淵出獵閉門拒之【淵本領平州鎮范陽范陽即涿郡後漢章帝改焉誘音酉】淵屯於郡南為榮舉哀【為于偽翻】勒兵南向進至中山行臺僕射魏蘭根邀擊之為淵所敗【敗補賣翻】敬宗以城陽王徽兼大司馬録尚書事總統内外徽意謂榮既死枝葉自應散落及爾朱世隆等兵四起黨衆日盛徽憂怖不知所出【怖普布翻】性多嫉忌不欲人居己前每獨與帝謀議羣臣有獻策者徽輒勸帝不納且曰小賊何慮不平又靳惜財貨賞賜率皆薄少【靳居焮翻少詩沼翻】或多而中減或與而復追故徒有糜費而恩不感物【史言徽誤魏主復扶又翻】十一月癸酉朔敬宗以車騎將軍鄭先護為大都督與行臺楊昱共討爾朱仲遠乙亥以司徒長孫稚為太尉臨淮王彧為司徒丙子進雍州刺史廣宗公爾朱天光爵為王【自此以上至鄭先護官爵皆敬宗所授】長廣王亦以天光為隴西王爾朱仲遠攻西兗州【魏太和中置西兗州於滑臺孝昌中置西兗州於定陶下云仲遠於賀拔勝戰於滑臺東則是時猶以滑臺為西兖州也】丁丑拔之擒刺史王衍衍肅之兄子也【王肅去齊入魏而貴顯】癸未敬宗以右衛將軍賀拔勝為東征都督壬辰又以鄭先護兼尚書左僕射為行臺與勝共討仲遠戊戍詔罷魏蘭根行臺以定州刺史薛曇尚兼尚書為北道行臺【曇徒舍翻】鄭先護疑賀拔勝置之營外庚子勝與仲遠戰於滑臺東兵敗降於仲遠初爾朱榮嘗從容問左右曰【從于容翻】一日無我誰可主軍皆稱爾朱兆榮曰兆雖勇於戰鬭然所將不過三千騎多則亂矣堪代我者唯賀六渾耳因戒兆曰爾非其匹終當為其穿鼻【譬之以牛牛鼻既穿則為人所制】乃以高歡為晉州刺史及兆引兵向洛遣使召歡歡遣長史孫騰詣兆辭以山蜀未平【蜀人徙汾晉依山而居者謂之山蜀】今方攻討不可委去致有後憂定蜀之日當隔河為犄角之勢【犄居蟻翻】兆不悦曰還白高晉州吾得吉夢夢與吾先人登高丘丘旁之地耕之已熟獨餘馬藺【本草蠡實馬藺子也出冀州圖經曰馬蕳子生河東川谷葉似薤而長厚衍義曰馬藺葉牛馬皆不食為纔出土葉已硬也】先人命吾拔之隨手而盡以此觀之往無不克騰還報歡曰兆狂愚如是而敢為悖逆吾勢不得久事爾朱矣【為歡起兵討爾朱張本悖蒲内翻又蒲沒翻】十二月壬寅朔爾朱兆攻丹谷都督崔伯鳳戰死都督史仵龍開壁請降【仵宜古翻降戶江翻】源子恭退走兆輕兵倍道兼行從河橋西涉渡 【考異曰伽藍記云從雷波涉渡今從魏書兆傳】先是敬宗以大河深廣謂兆未能猝濟【先昔薦翻】是日水不沒馬腹甲辰暴風黄塵漲天兆騎叩宫門宿衛乃覺彎弓欲射【騎奇計翻下同射而亦翻】矢不得發一時散走華山王鷙斤之玄孫也【斤亂代事見一百四卷晉孝武太元元年魏以此始亦以此終天邪人邪】素附爾朱氏帝始聞兆南下欲自帥諸軍討之鷙說帝曰黃河萬仭兆安得渡帝遂自安及兆入宫鷙復約止衛兵不使鬭【帥音率說式芮翻復扶又翻】帝步出雲龍門外遇城陽王徽乘馬走帝屢呼之不顧而去【徽預國大謀敗不即死去將安之】兆騎執帝鎻於永寧寺樓上帝寒甚就兆求頭巾【頭巾所謂袹頭也】不與兆營於尚書省用天子金皷設刻漏於庭撲殺皇子【皇子爾朱后所生也撲弼角翻】汙辱嬪御妃主【汙烏故翻嬪毘賓翻】縱兵大掠殺司空臨淮王彧尚書左僕射范陽王誨青州刺史李延寔等城陽王徽走至山南【山南伊潁南山之南也】扺前洛陽令寇祖仁家 【考異曰魏書作寇彌按寇讃諸孫所字皆連祖字或者名彌字祖仁今從伽藍記】祖仁一門三刺史皆徽所引拔以有舊恩故投之徽齎金百斤馬五十匹祖仁利其財外雖容納而私謂子弟曰如聞爾朱兆購募城陽王得之者封千戶侯今日富貴至矣乃怖徽云官捕將至令其逃於他所【怖普布翻】使人於路邀殺之送首於兆【徽背敬宗而祖仁亦背徽惡殃之報何速哉蒼蒼之不可欺也如此】兆亦不加勳賞兆夢徽謂己曰我有金二百斤馬百匹在祖仁家卿可取之兆既覺【覺古孝翻】意所夢為實即掩捕祖仁徵其金馬祖仁謂人密告望風欵服云實得金百斤馬五十匹【欵誠實也獄囚招承之辭曰欵言得其實也】兆疑其隱匿依夢徵之祖仁家舊有金三十斤馬三十匹盡以輸兆兆猶不信怒執祖仁懸首高樹大石墜足捶之至死【捶止橤翻】爾朱世隆至洛陽【自長子至洛陽也】兆自以為己功責世隆曰叔父在朝日久【世隆榮從弟兆榮從子故呼世隆為叔父靈后臨朝之時世隆已在朝故曰日久朝直遥翻】耳目應廣如何令天柱受禍按劒瞋目聲色甚厲【瞋七人翻】世隆遜辭拜謝然後得已由是深恨之【為爾朱兄弟叔姪互相猜疑以致夷滅張本】爾朱仲遠亦自滑臺至洛戊申魏長廣王大赦爾朱榮之死也敬宗詔河西賊帥紇豆陵步蕃使襲秀容【步蕃居北河之西紇豆陵虜三宇姓魏書官氏志次南諸姓有紇豆陵氏帥音所類翻紇音下沒翻】及兆入洛步蕃南下兵勢甚盛故兆不暇久留亟還晉陽以禦之使爾朱世隆度律彦伯等留鎮洛陽甲寅兆遷敬宗於晉陽兆自於河梁監閱財資【河梁即河橋監工咸翻】高歡聞敬宗向晉陽帥騎東巡欲邀之不及因與兆書為陳禍福不宜害天子受惡名兆怒不納爾朱天光輕騎入洛見世隆等即還雍州【帥讀曰率騎奇計翻為于偽翻雍於用翻】初敬宗恐北軍不利【北軍謂源子恭鎮丹谷之軍也】欲為南走之計託云征蠻以高道穆為南道大行臺未及發而兆入洛道穆託疾去世隆殺之主者請追李苗封贈世隆曰當時衆議更一二日即欲縱兵大掠焚燒郭邑賴苗之故京師獲全天下之善一也不宜復追【復扶又翻】爾朱榮之死也世隆等徵兵於大寧太守代人房謨【魏收志魏孝昌中置泰寧郡屬建州其地當在唐澤州沁水縣界大當作泰】謨不應前後斬其三使【使疏吏翻】遣弟毓詣洛陽及兆得志其黨建州刺史是蘭安定【是蘭姓也安定其名】執謨繫州獄郡中蜀人聞之皆叛【此謂蜀人之居泰寧者亦汾蜀絳蜀之類也】安定給謨弱馬令軍前慰勞【勞力到翻】諸賊見謨莫不遙拜謨先所乘馬安定别給將士【將即亮翻】戰敗蜀人得之謂謨遇害莫不悲泣善養其馬不聽人乘之兒童婦女競投草粟皆言此房公馬也爾朱世隆聞之捨其罪以為其府長史北道大行臺楊津以衆少留鄴召募欲自滏口入并州會爾朱兆入洛津乃散衆輕騎還朝【少詩沼翻滏音釜騎奇計翻朝直遙翻】爾朱世隆與兄弟密謀慮長廣王母衛氏干預朝政伺其出行遣數十騎如刼盜者於京巷殺之【直曰街曲曰巷京巷洛京之曲巷也朝直遙翻伺相吏翻】尋懸牓以千萬錢募賊甲子爾朱兆縊敬宗於晉陽三級佛寺【年二十匹廢帝謚帝曰武懷皇帝及孝武帝立以廟諱改謚曰孝莊皇帝廟號敬宗縊於賜翻又於計翻】并殺陳留王寛是月紇豆陵步蕃大破爾朱兆於秀容南逼晉陽兆懼使人召高歡幷力 【考異曰北齊慕容紹宗傳兆召高祖紹宗諫曰今天下擾攘人懷覬覦正是智士用策之秋高晉州才雄氣猛英略盖世譬如蛟龍安可借以雲雨兆怒曰我與晉州推誠相待何得輒相間阻囚紹宗數日乃釋之北史紹宗語在神武請帥降戶就食山東下按兆始召歡以自救非猜嫌之時今從北史】僚屬皆勸歡勿應召歡曰兆方急保無它慮遂行歡所親賀拔焉過兒請緩行以弊之歡往往逗留辭以河無橋不得渡【此河蓋汾河也】步蕃兵日盛兆屢敗告急於歡歡乃往從之兆時避步蕃南出步蕃至平樂郡【平樂郡據爾朱兆傳當作樂平郡後漢獻帝分太原置樂平郡治沾城唐遼州即其地】歡與兆進兵合擊大破之斬步蕃於石皷山【魏收志秀容郡秀容縣有石皷山】其衆退走兆德歡相與誓為兄弟將數十騎詣歡通夜宴飲初葛榮部衆流入并肆者二十餘萬為契胡凌暴【契胡爾朱之種人也契欺訖翻】皆不聊生大小二十六反誅夷者半猶謀亂不止兆患之問計於歡歡曰六鎮反殘不可盡殺【自破六韓拔陵杜洛周之敗其衆盡歸葛榮皆六鎮人也】宜選王腹心使統之有犯者罪其帥【帥所類翻】則所罪者寡矣兆曰善誰可使者賀拔允時在坐【坐徂臥翻】請使歡領之歡拳敺其口折一齒曰平生天柱時奴輩伏處分如鷹犬【敺烏口翻歡自謂也詭為遜辭使兆不疑已折而設翻處昌呂翻分扶問翻下同】今日天下事取捨在王而阿鞠泥敢僭易妄言請殺之【賀拔允字阿鞠泥易以豉翻】兆以歡為誠遂以其衆委焉歡以兆醉恐醒而悔之遂出宣言受委統州鎮兵【魏改六鎮為州葛榮部衆皆六鎮人故曰州鎮兵】可集汾東受號令乃建牙陽曲川【水經注汾水自汾陽縣南流逕陽曲城西陽曲在秀容之南地形志陽曲縣二漢屬太原郡後魏永安中置永安郡陽曲縣屬焉宋白曰唐忻州秀容定襄二縣皆漢陽曲縣地河千里一曲縣當其陽故曰陽曲後漢末移陽曲縣於并州太原縣界於舊陽曲縣置定襄縣又分置九原縣屬新興郡後魏以九原縣為平冠縣隋為秀容縣】陳部分軍士素惡兆而樂屬歡【分扶問翻惡烏故翻樂音洛】莫不皆至居無何【居無何言在事未多時也無何猶言無幾何時】又使劉貴請兆以幷肆頻歲霜旱降戶掘田鼠而食之【降戶江翻】面無穀色徒汙人境内【幷肆之地兆統内也汙烏故翻】請令就食山東【幷肆冀定瀛相殷以太行常山為限幷肆在山西餘州皆在山東歡欲引衆就食山東正欲遠兆得以從容收衆心因之以起兵也】待溫飽更受處分兆從其議長史慕容紹宗諫曰不可方今四方紛擾人懷異望高公雄才盖世復使握大兵於外【復扶又翻】譬如借蛟龍以雲雨將不可制矣兆曰有香火重誓何慮邪【重誓謂與歡誓為兄弟】紹宗曰親兄弟尚不可信何論香火【時爾朱兆與其羣從已搆嫌隙故紹宗以此言諷之】時兆左右已受歡金因稱紹宗與歡有舊隙兆怒囚紹宗趣歡發【趣讀曰促】歡自晉陽出滏口道逢北鄉長公主自洛陽來有馬三百匹盡奪而易之兆聞之乃釋紹宗而問之紹宗曰此猶是掌握中物也兆乃自追歡至襄垣【襄垣縣漢屬上黨即後魏屬鄉郡至敬宗建義元年分置襄垣郡】會漳水暴漲橋壞【水經漳水自屯留縣東北流逕襄垣縣故城南】歡隔水拜曰所以借公主馬非有它故備山東盜耳王信公主之讒自來賜追【歡稱兆為王因長廣王所封也】今不辭度水而死恐此衆便叛兆自陳無此意因輕馬度水與歡坐幕下授歡刀引頸使歡斫之【古之豪雄推赤心置人腹中者必其威望有以服其心智力足以制其命然後行之以安反側然亦未至如爾朱兆之輕率也】歡大哭曰自天柱之薨賀六渾更何所仰但願大家千萬歲以申力用耳今為旁人所搆間大家何忍復出此言【歡之此言亦謬為恭敬耳歡以主事兆故稱為大家間古莧翻復音扶又翻下同】兆投刀於地復斬白馬與歡為誓因留宿夜飲尉景伏壯士欲執兆【尉紆勿翻】歡齧臂止之曰今殺之其黨必奔歸聚結兵饑馬瘦不可與敵若英雄乘之而起則為害滋甚不如且置之兆雖驍勇凶悍無謀不足圖也【不足圖者謂其易圖也史言舉大事者必審而後驍堅堯翻悍侯旴翻】旦日兆歸營復召歡歡將上馬詣之【上時掌翻】孫騰牽歡衣歡乃止兆隔水肆罵馳還晉陽【當是時爾朱兆已知高歡之不可制而無如之何】兆腹心念賢領降戶家屬别為營歡偽與之善觀其佩刀因取殺之【按通鑑念賢後仕於西魏貴顯此豈别有一念賢邪又按李百藥北齊書歡取賢佩刀以殺其從者從者盡散則謂所殺者賢之從者非殺賢也姓譜有念姓】士衆感悦益願附從齊州城民趙洛周聞爾朱兆入洛逐刺史丹陽王蕭<br />
<br />
  贊以城歸兆贊變形為沙門逃入長白山【五代志齊州章丘縣舊曰高唐有長白山杜佑曰長白山在淄州長山縣】流轉卒於陽平【陽平縣漢屬東郡魏晉以來分屬陽平郡隋唐魏州之莘縣即其地卒子恤翻】梁人或盜其柩以歸上猶以子禮葬於陵次【豫章王綜奔魏改名贊事見一百五十卷普通六年贊不以帝為父而帝猶以贊為子可謂愛其所不當愛矣柩音舊】 魏荆州刺史李琰之韶之族弟也南陽太守趙修延以琰之敬宗外族【敬宗母彭城王勰妃李氏也故云然】誣琰之欲奔梁兵襲州城執琰之自行州事 魏王悦改元更興【更工行翻】聞爾朱兆已入洛自知不及事遂南還斛斯椿復棄悦奔魏【為斛斯椿誅爾朱世隆兄弟搆間高歡以分魏為東西張本復扶又翻】 是歲詔以陳慶之為都督南北司等四州諸軍事南北司二州刺史【梁置南司州於安陸北司州於義陽】慶之引兵圍魏懸瓠破魏潁州刺史婁起等於溱水【魏孝昌四年置潁州於汝隂領汝隂弋陽北陳留潁川財丘梁興西恒農陳南東郡汝南清河南陽新蔡南陳留滎陽北通汝南太原等雙郡東恒農新興等郡水經注溱水出汝南平輿縣浮石嶺東北青衣山東南逕朗陵縣故城西東北逕北宜春縣故城北又東北入於汝溱緇詵翻宋白曰蔡州城南有溱水】又破行臺孫騰等於楚城【梁置西楚州於楚城在汝南郡城陽縣界其地當在唐申州界按孫騰此時猶從高歡在幷冀殷相之間慶之破騰必非此年事史䆒言之耳】罷義陽鎮兵停水陸漕運江湖諸州竝得休息【謂瀕江及洞庭彭蠡間諸州也】開田六十頃二年之後倉廩充實<br />
<br />
  資治通鑑卷一百五十四  <br>
   </div> 

<script src="/search/ajaxskft.js"> </script>
 <div class="clear"></div>
<br>
<br>
 <!-- a.d-->

 <!--
<div class="info_share">
</div> 
-->
 <!--info_share--></div>   <!-- end info_content-->
  </div> <!-- end l-->

<div class="r">   <!--r-->



<div class="sidebar"  style="margin-bottom:2px;">

 
<div class="sidebar_title">工具类大全</div>
<div class="sidebar_info">
<strong><a href="http://www.guoxuedashi.com/lsditu/" target="_blank">历史地图</a></strong>  
<a href="http://www.880114.com/" target="_blank">英语宝典</a>  
<a href="http://www.guoxuedashi.com/13jing/" target="_blank">十三经检索</a> 
<br><strong><a href="http://www.guoxuedashi.com/gjtsjc/" target="_blank">古今图书集成</a></strong> 
<a href="http://www.guoxuedashi.com/duilian/" target="_blank">对联大全</a> <strong><a href="http://www.guoxuedashi.com/xiangxingzi/" target="_blank">象形文字典</a></strong> 

<br><a href="http://www.guoxuedashi.com/zixing/yanbian/">字形演变</a>  <strong><a href="http://www.guoxuemi.com/hafo/" target="_blank">哈佛燕京中文善本特藏</a></strong>
<br><strong><a href="http://www.guoxuedashi.com/csfz/" target="_blank">丛书&方志检索器</a></strong> <a href="http://www.guoxuedashi.com/yqjyy/" target="_blank">一切经音义</a>  

<br><strong><a href="http://www.guoxuedashi.com/jiapu/" target="_blank">家谱族谱查询</a></strong>  <strong><a href="http://shufa.guoxuedashi.com/sfzitie/" target="_blank">书法字帖欣赏</a></strong> 
<br>

</div>
</div>


<div class="sidebar" style="margin-bottom:0px;">

<font style="font-size:22px;line-height:32px">QQ交流群9:489193090</font>


<div class="sidebar_title">手机APP 扫描或点击</div>
<div class="sidebar_info">
<table>
<tr>
	<td width=160><a href="http://m.guoxuedashi.com/app/" target="_blank"><img src="/img/gxds-sj.png" width="140"  border="0" alt="国学大师手机版"></a></td>
	<td>
<a href="http://www.guoxuedashi.com/download/" target="_blank">app软件下载专区</a><br>
<a href="http://www.guoxuedashi.com/download/gxds.php" target="_blank">《国学大师》下载</a><br>
<a href="http://www.guoxuedashi.com/download/kxzd.php" target="_blank">《汉字宝典》下载</a><br>
<a href="http://www.guoxuedashi.com/download/scqbd.php" target="_blank">《诗词曲宝典》下载</a><br>
<a href="http://www.guoxuedashi.com/SiKuQuanShu/skqs.php" target="_blank">《四库全书》下载</a><br>
</td>
</tr>
</table>

</div>
</div>


<div class="sidebar2">
<center>


</center>
</div>

<div class="sidebar"  style="margin-bottom:2px;">
<div class="sidebar_title">网站使用教程</div>
<div class="sidebar_info">
<a href="http://www.guoxuedashi.com/help/gjsearch.php" target="_blank">如何在国学大师网下载古籍?</a><br>
<a href="http://www.guoxuedashi.com/zidian/bujian/bjjc.php" target="_blank">如何使用部件查字法快速查字?</a><br>
<a href="http://www.guoxuedashi.com/search/sjc.php" target="_blank">如何在指定的书籍中全文检索?</a><br>
<a href="http://www.guoxuedashi.com/search/skjc.php" target="_blank">如何找到一句话在《四库全书》哪一页?</a><br>
</div>
</div>


<div class="sidebar">
<div class="sidebar_title">热门书籍</div>
<div class="sidebar_info">
<a href="/so.php?sokey=%E8%B5%84%E6%B2%BB%E9%80%9A%E9%89%B4&kt=1">资治通鉴</a> <a href="/24shi/"><strong>二十四史</strong></a>&nbsp; <a href="/a2694/">野史</a>&nbsp; <a href="/SiKuQuanShu/"><strong>四库全书</strong></a>&nbsp;<a href="http://www.guoxuedashi.com/SiKuQuanShu/fanti/">繁体</a>
<br><a href="/so.php?sokey=%E7%BA%A2%E6%A5%BC%E6%A2%A6&kt=1">红楼梦</a> <a href="/a/1858x/">三国演义</a> <a href="/a/1038k/">水浒传</a> <a href="/a/1046t/">西游记</a> <a href="/a/1914o/">封神演义</a>
<br>
<a href="http://www.guoxuedashi.com/so.php?sokeygx=%E4%B8%87%E6%9C%89%E6%96%87%E5%BA%93&submit=&kt=1">万有文库</a> <a href="/a/780t/">古文观止</a> <a href="/a/1024l/">文心雕龙</a> <a href="/a/1704n/">全唐诗</a> <a href="/a/1705h/">全宋词</a>
<br><a href="http://www.guoxuedashi.com/so.php?sokeygx=%E7%99%BE%E8%A1%B2%E6%9C%AC%E4%BA%8C%E5%8D%81%E5%9B%9B%E5%8F%B2&submit=&kt=1"><strong>百衲本二十四史</strong></a>  <a href="http://www.guoxuedashi.com/so.php?sokeygx=%E5%8F%A4%E4%BB%8A%E5%9B%BE%E4%B9%A6%E9%9B%86%E6%88%90&submit=&kt=1"><strong>古今图书集成</strong></a>
<br>

<a href="http://www.guoxuedashi.com/so.php?sokeygx=%E4%B8%9B%E4%B9%A6%E9%9B%86%E6%88%90&submit=&kt=1">丛书集成</a> 
<a href="http://www.guoxuedashi.com/so.php?sokeygx=%E5%9B%9B%E9%83%A8%E4%B8%9B%E5%88%8A&submit=&kt=1"><strong>四部丛刊</strong></a>  
<a href="http://www.guoxuedashi.com/so.php?sokeygx=%E8%AF%B4%E6%96%87%E8%A7%A3%E5%AD%97&submit=&kt=1">說文解字</a> <a href="http://www.guoxuedashi.com/so.php?sokeygx=%E5%85%A8%E4%B8%8A%E5%8F%A4&submit=&kt=1">三国六朝文</a>
<br><a href="http://www.guoxuedashi.com/so.php?sokeytm=%E6%97%A5%E6%9C%AC%E5%86%85%E9%98%81%E6%96%87%E5%BA%93&submit=&kt=1"><strong>日本内阁文库</strong></a> <a href="http://www.guoxuedashi.com/so.php?sokeytm=%E5%9B%BD%E5%9B%BE%E6%96%B9%E5%BF%97%E5%90%88%E9%9B%86&ka=100&submit=">国图方志合集</a> <a href="http://www.guoxuedashi.com/so.php?sokeytm=%E5%90%84%E5%9C%B0%E6%96%B9%E5%BF%97&submit=&kt=1"><strong>各地方志</strong></a>

</div>
</div>


<div class="sidebar2">
<center>

</center>
</div>
<div class="sidebar greenbar">
<div class="sidebar_title green">四库全书</div>
<div class="sidebar_info">

《四库全书》是中国古代最大的丛书,编撰于乾隆年间,由纪昀等360多位高官、学者编撰,3800多人抄写,费时十三年编成。丛书分经、史、子、集四部,故名四库。共有3500多种书,7.9万卷,3.6万册,约8亿字,基本上囊括了古代所有图书,故称“全书”。<a href="http://www.guoxuedashi.com/SiKuQuanShu/">详细>>
</a>

</div> 
</div>

</div>  <!--end r-->

</div>
<!-- 内容区END --> 

<!-- 页脚开始 -->
<div class="shh">

</div>

<div class="w1180" style="margin-top:8px;">
<center><script src="http://www.guoxuedashi.com/img/plus.php?id=3"></script></center>
</div>
<div class="w1180 foot">
<a href="/b/thanks.php">特别致谢</a> | <a href="javascript:window.external.AddFavorite(document.location.href,document.title);">收藏本站</a> | <a href="#">欢迎投稿</a> | <a href="http://www.guoxuedashi.com/forum/">意见建议</a> | <a href="http://www.guoxuemi.com/">国学迷</a> | <a href="http://www.shuowen.net/">说文网</a><script language="javascript" type="text/javascript" src="https://js.users.51.la/17753172.js"></script><br />
  Copyright &copy; 国学大师 古典图书集成 All Rights Reserved.<br>
  
  <span style="font-size:14px">免责声明:本站非营利性站点,以方便网友为主,仅供学习研究。<br>内容由热心网友提供和网上收集,不保留版权。若侵犯了您的权益,来信即刪。scp168@qq.com</span>
  <br />
ICP证:<a href="http://www.beian.miit.gov.cn/" target="_blank">鲁ICP备19060063号</a></div>
<!-- 页脚END --> 
<script src="http://www.guoxuedashi.com/img/plus.php?id=22"></script>
<script src="http://www.guoxuedashi.com/img/tongji.js"></script>

</body>
</html>
