<!DOCTYPE html PUBLIC "-//W3C//DTD XHTML 1.0 Transitional//EN" "http://www.w3.org/TR/xhtml1/DTD/xhtml1-transitional.dtd">
<html xmlns="http://www.w3.org/1999/xhtml">
<head>
<meta http-equiv="Content-Type" content="text/html; charset=utf-8" />
<meta http-equiv="X-UA-Compatible" content="IE=Edge,chrome=1">
<title>資治通鑒_104-資治通鑑卷一百三_104-資治通鑑卷一百三</title>
<meta name="Keywords" content="資治通鑒_104-資治通鑑卷一百三_104-資治通鑑卷一百三">
<meta name="Description" content="資治通鑒_104-資治通鑑卷一百三_104-資治通鑑卷一百三">
<meta http-equiv="Cache-Control" content="no-transform" />
<meta http-equiv="Cache-Control" content="no-siteapp" />
<link href="/img/style.css" rel="stylesheet" type="text/css" />
<script src="/img/m.js?2020"></script> 
</head>
<body>
 <div class="ClassNavi">
<a  href="/24shi/">二十四史</a> | <a href="/SiKuQuanShu/">四库全书</a> | <a href="http://www.guoxuedashi.com/gjtsjc/"><font  color="#FF0000">古今图书集成</font></a> | <a href="/renwu/">历史人物</a> | <a href="/ShuoWenJieZi/"><font  color="#FF0000">说文解字</a></font> | <a href="/chengyu/">成语词典</a> | <a  target="_blank"  href="http://www.guoxuedashi.com/jgwhj/"><font  color="#FF0000">甲骨文合集</font></a> | <a href="/yzjwjc/"><font  color="#FF0000">殷周金文集成</font></a> | <a href="/xiangxingzi/"><font color="#0000FF">象形字典</font></a> | <a href="/13jing/"><font  color="#FF0000">十三经索引</font></a> | <a href="/zixing/"><font  color="#FF0000">字体转换器</font></a> | <a href="/zidian/xz/"><font color="#0000FF">篆书识别</font></a> | <a href="/jinfanyi/">近义反义词</a> | <a href="/duilian/">对联大全</a> | <a href="/jiapu/"><font  color="#0000FF">家谱族谱查询</font></a> | <a href="http://www.guoxuemi.com/hafo/" target="_blank" ><font color="#FF0000">哈佛古籍</font></a> 
</div>

 <!-- 头部导航开始 -->
<div class="w1180 head clearfix">
  <div class="head_logo l"><a title="国学大师官网" href="http://www.guoxuedashi.com" target="_blank"></a></div>
  <div class="head_sr l">
  <div id="head1">
  
  <a href="http://www.guoxuedashi.com/zidian/bujian/" target="_blank" ><img src="http://www.guoxuedashi.com/img/top1.gif" width="88" height="60" border="0" title="部件查字,支持20万汉字"></a>


<a href="http://www.guoxuedashi.com/help/yingpan.php" target="_blank"><img src="http://www.guoxuedashi.com/img/top230.gif" width="600" height="62" border="0" ></a>


  </div>
  <div id="head3"><a href="javascript:" onClick="javascript:window.external.AddFavorite(window.location.href,document.title);">添加收藏</a>
  <br><a href="/help/setie.php">搜索引擎</a>
  <br><a href="/help/zanzhu.php">赞助本站</a></div>
  <div id="head2">
 <a href="http://www.guoxuemi.com/" target="_blank"><img src="http://www.guoxuedashi.com/img/guoxuemi.gif" width="95" height="62" border="0" style="margin-left:2px;" title="国学迷"></a>
  

  </div>
</div>
  <div class="clear"></div>
  <div class="head_nav">
  <p><a href="/">首页</a> | <a href="/ShuKu/">国学书库</a> | <a href="/guji/">影印古籍</a> | <a href="/shici/">诗词宝典</a> | <a   href="/SiKuQuanShu/gxjx.php">精选</a> <b>|</b> <a href="/zidian/">汉语字典</a> | <a href="/hydcd/">汉语词典</a> | <a href="http://www.guoxuedashi.com/zidian/bujian/"><font  color="#CC0066">部件查字</font></a> | <a href="http://www.sfds.cn/"><font  color="#CC0066">书法大师</font></a> | <a href="/jgwhj/">甲骨文</a> <b>|</b> <a href="/b/4/"><font  color="#CC0066">解密</font></a> | <a href="/renwu/">历史人物</a> | <a href="/diangu/">历史典故</a> | <a href="/xingshi/">姓氏</a> | <a href="/minzu/">民族</a> <b>|</b> <a href="/mz/"><font  color="#CC0066">世界名著</font></a> | <a href="/download/">软件下载</a>
</p>
<p><a href="/b/"><font  color="#CC0066">历史</font></a> | <a href="http://skqs.guoxuedashi.com/" target="_blank">四库全书</a> |  <a href="http://www.guoxuedashi.com/search/" target="_blank"><font  color="#CC0066">全文检索</font></a> | <a href="http://www.guoxuedashi.com/shumu/">古籍书目</a> | <a   href="/24shi/">正史</a> <b>|</b> <a href="/chengyu/">成语词典</a> | <a href="/kangxi/" title="康熙字典">康熙字典</a> | <a href="/ShuoWenJieZi/">说文解字</a> | <a href="/zixing/yanbian/">字形演变</a> | <a href="/yzjwjc/">金 文</a> <b>|</b>  <a href="/shijian/nian-hao/">年号</a> | <a href="/diming/">历史地名</a> | <a href="/shijian/">历史事件</a> | <a href="/guanzhi/">官职</a> | <a href="/lishi/">知识</a> <b>|</b> <a href="/zhongyi/">中医中药</a> | <a href="http://www.guoxuedashi.com/forum/">留言反馈</a>
</p>
  </div>
</div>
<!-- 头部导航END --> 
<!-- 内容区开始 --> 
<div class="w1180 clearfix">
  <div class="info l">
   
<div class="clearfix" style="background:#f5faff;">
<script src='http://www.guoxuedashi.com/img/headersou.js'></script>

</div>
  <div class="info_tree"><a href="http://www.guoxuedashi.com">首页</a> > <a href="/SiKuQuanShu/fanti/">四库全书</a>
 > <h1>资治通鉴</h1> <!--         下载:【右键另存为】即可 --></div>
  <div class="info_content zj clearfix">
  
<div class="info_txt clearfix" id="show">
<center style="font-size:24px;">104-資治通鑑卷一百三</center>
    資治通鑑卷一百三   宋 司馬光撰<br />
<br />
  胡三省音註<br />
<br />
  晉紀二十五【起重光協洽盡旃蒙大淵獻凡五年】<br />
<br />
  太宗簡文皇帝【諱昱字道萬元帝之少子也封琅邪王後徙封會稽王海西即位琅邪絶嗣復徙封琅邪固讓故雖封琅邪而不去會稽之號諡法一德不懈曰簡道德博聞曰文】<br />
<br />
  咸安元年【是年十一月海西廢帝即位始改元咸安通鑑編年因以新元繋之】春正月袁瑾朱輔求救於秦秦王堅以瑾為揚州刺史【瑾渠吝翻】輔為交州刺史遣武衛將軍武都王鑒前將軍張蚝帥步騎二萬救之【蚝七吏翻帥讀曰率騎奇寄翻】大司馬温遣淮南太守桓伊南頓太守桓石䖍等撃鑒蚝於石橋【據桓温傳石橋在肥水北守式又翻】大破之秦兵退屯慎城【慎縣漢屬汝南郡晉分屬汝隂郡唐廬州之慎縣則梁魏之閒南梁郡之慎縣漢九江逡遒縣之地非此慎城】伊宣之子也【桓宣佐祖逖拒祖約守襄陽皆有功】丁亥温拔壽春擒瑾及輔并其宗族送建康斬之 秦王堅徙關東豪傑及雜夷十五萬戶于關中處烏桓于馮翊北地丁零翟斌于新安澠池【為翟斌乘秦亂起兵張本處昌呂翻澠彌兖翻斌音彬】諸因亂流移欲還舊業者悉聽之 二月秦以魏郡太守韋鍾為青州刺史【青州刺史治廣固】中壘將軍梁成為兖州刺史射聲校尉徐成為并州刺史武衛將軍王鑒為豫州刺史左將軍彭越為徐州刺史太尉司馬皇甫覆為荆州刺史【晉志曰秦既滅燕以兖州刺史鎮倉垣并州刺史鎮晉陽豫州刺史鎮洛陽徐州刺史鎮彭城秦初以荆州刺史鎮豐陽後移襄陽余按此時秦未得襄陽蓋仍燕之舊鎮魯陽也】屯騎校尉天水姜宇為涼州刺史扶風内史王統為益州刺史【凉州屬張天錫益州晉土也秦蓋置凉州於天水界置益州於扶風界校戶敎翻】秦州刺史西縣侯雅為使持節都督秦晉凉雍州諸軍事秦州牧【雅苻氏也前此未有晉州凉之張氏分西平界置晉興郡秦蓋於此置晉州也雍於用翻】吏部尚書楊安為使持節都督益梁州諸軍事梁州刺史【堅欲進圖梁益故置梁益二州刺史楊安既克仇池始加督南秦州鎮仇池使疏吏翻】復置雍州治蒲阪【秦省雍州見上卷上年復扶又翻雍於用翻阪音反】以長樂公丕為使持節征東大將軍雍州刺史【樂音洛使疏吏翻】成平老之子統擢之子也【穆帝永和十年王擢降秦】堅以關東初平守令宜得人令王猛以便宜簡召英俊補六州守令授訖言臺除正【奏上秦朝除為正官也嗚呼荀卿子有言兼并易也堅凝之難以苻堅之明王猛之畧簡召六州英俊以補守令然鮮卑乘亂一呼翕然為燕以此知天下之勢但觀人心向背何如耳】 三月壬辰益州刺史建成定公周楚卒【諡法大慮静民曰定】 秦後將軍金城俱難攻蘭陵太守張閔子于桃山【俱姓難名魏收地形志蘭陵昌慮縣有桃山】大司馬温遣兵擊却之 秦西縣侯雅楊安王統徐成及羽林左監朱肜揚武將軍姚萇帥步騎七萬伐仇池公楊纂【肜余中翻萇仲良翻帥讀曰率騎奇寄翻】 代將長孫斤謀弑代王什翼犍世子寔格之傷脇遂執斤殺之【代之先拓拔鄰以次兄為拓拔氏後改為長孫氏將即亮翻犍居言翻】 夏四月戊午大赦 秦兵至鷲峽【鷲峽在仇池北亦謂之塞峽鷲音就】楊纂帥衆五萬拒之梁州刺史弘農楊亮遣督護郭寶卜靖帥千餘騎助纂與秦兵戰于峽中纂兵大敗死者什三四寶等亦沒纂收散兵遁還西縣侯雅進攻仇池楊統帥武都之衆降秦【統與纂爭國見上卷上年降戶江翻下同】纂懼面縛出降雅送纂于長安以統為南秦州刺史【秦置秦州於丄邽仇池在其南故置南秦州】加楊安都督南秦州諸軍事鎮仇池王猛之破張天錫于枹罕也【事見一百一卷海西公太和元年枹音膚】獲其將敦煌陰據及甲士五千人【孰徒門翻】秦王堅既克楊纂遣據帥其甲士還凉州使著作郎梁殊閻負送之【穆帝永和十二年秦遣殊負使凉今復遣之】因命王猛為書諭天錫曰昔貴先公稱藩劉石者惟審于彊弱也【張茂稱藩于劉曜事見九十二卷明帝太寧元年張駿稱藩於石勒事見九十四卷成帝咸和五年】今論凉土之力則損于往時語大秦之德則非二趙之匹而將軍翻然自絶【絶秦見一百一卷太和元年】無乃非宗廟之福也歟以秦之威旁振無外可以囘弱水使東流返江河使西注【禹之治水高高下下因天地之性弱水西流江河東注今言能反之囘之喻秦威力之彊也】關東既平將移兵河右恐非六郡士民所能抗也【凉州六郡以張軌初鎮河西之時統治武威張掖酒泉敦煌西郡西海六郡言之也元康以後張氏所分置其為郡多矣】劉表謂漢南可保【事見漢獻帝紀】將軍謂西河可全吉凶在身元龜不遠宜深筭妙慮自求多福無使六世之業一旦而墜地也【自張軌保據河西至天錫凡九主今言六世者不以耀靈祚玄靚為世數】天錫大懼遣使謝罪稱藩堅拜天錫使持節都督河右諸軍事驃騎大將軍開府儀同三司凉州刺史西平公【使疏吏翻下同驃匹妙翻騎奇寄翻】吐谷渾王辟奚聞楊纂敗五月遣使獻馬千匹金銀五百斤于秦【吐從暾入聲谷音浴】秦以辟奚為安遠將軍漒川侯辟奚葉延之子也【葉延見九十四卷成帝咸和四年漒其良翻】好學仁厚無威斷【好呼到翻斷丁亂翻下無斷同】三弟專恣國人患之長史鍾惡地西漒羌豪也【漒渠良翻羌人據漒川之地分為東西】謂司馬乞宿雲曰三弟縱横勢出王右【横戶孟翻右上也】幾亡國矣【幾居希翻】吾二人位為元輔【長史司馬府之元僚】豈得坐而視之詰朝月望【日行遲一年一周天月行速一月一周天而與日會日月之會謂之合朔自合朔之後月又先日而行至十五日日月相望謂之月望詰去吉翻】文武並會吾將討焉王之左右皆吾羌子轉目一顧立可擒也宿雲請先白王惡地曰王仁而無斷白之必不從萬一事泄吾屬無類矣事已出口何可中變遂於坐收三弟殺之【坐徂卧翻】辟奚驚怖【怖普布翻】自投床下惡地宿雲趨而扶之曰臣昨夢先王敕臣云三弟將為逆不可不討故誅之耳辟奚由是發病恍惚【人無精爽謂之恍惚】命世子視連曰吾禍及同生何以見之于地下國事大小任汝治之吾餘年殘命寄食而已遂以憂卒【卒子恤翻下同】視連立不飲酒遊畋者七年軍國之事委之將佐【將即亮翻】鍾惡地諫以為人主當自娛樂【樂音洛】建威布德視連泣曰孤自先世以來以仁孝忠恕相承先王念友愛之不終悲憤而亡孤雖纂業尸存而已聲色遊娯豈所安也威德之建當付之將來耳【辟奚之死視連之立其事非皆在是年通鑑因辟奚入貢于秦遂連而書之以見辟奚父子天性仁孝不可以夷狄異類視之也】 代世子寔病傷而卒【格長孫斤而被傷也】 秋七月秦王堅如洛陽 代世子寔娶東部大人賀野干之女【據北史賀野干即賀蘭部酋長魏書官氏志北方賀蘭後改為賀氏】有遺腹子甲戌生男代王什翼犍為之赦境内【為于偽翻】名曰涉圭【拓跋珪造魏事始此】大司馬温以梁益多寇周氏世有威名【周訪周撫周楚皆著威名于梁益】八月以寧州刺史周仲孫監益梁二州諸軍事領益州刺史【監工銜翻】仲孫光之子也【周光見九十三卷明帝太寧三年】 秦以光禄勲李儼為河州刺史鎮武始【河西張駿以興晉金城武始南安永晉大夏武城漢中為河州武始郡治狄道亦張駿所置】 王猛以潞川之功【見上卷上年】請以鄧羌為司隸秦王堅下詔曰司隸校尉董牧皇畿吏責甚重非所以優禮名將光武不以吏事處功臣【見四十三卷漢光武建武十三年處昌呂翻】實貴之也羌有亷李之才【亷李謂亷頗李牧】朕方委以征伐之事北平匈奴南蕩揚越羌之任也司隸何足以嬰之其進號鎮軍將軍位特進 九月秦王堅還長安歸安元侯李儼卒于上邽【諡法能思辨衆曰元行義說民曰元晉武受禪當時之臣死多有諡元者固非以行定諡也】堅復以儼子辯為河州刺史【復扶又翻】 冬十月秦王堅如鄴獵于西山旬餘忘返伶人王洛叩馬諫曰【鄭玄曰伶官樂官也伶氏世掌樂官而善焉故後世多號樂官為伶官】陛下羣生所繫今久獵不歸一旦患生不虞奈太后天下何堅為之罷獵還宮【為于偽翻】王猛因進言曰畋獵誠非急務王洛之言不可忘也堅賜洛帛百匹拜官箴左右【左傳昔周辛甲之為太史命百官官箴王闕於虞人之箴曰芒芒禹迹畫為九州經啟九道民有寢廟獸有茂草各有攸處德用不擾在帝夷羿冒于原獸忘其國恤而思其麀牡武不可重用不恢于夏家獸臣司原敢告僕夫虞箴如是以戒獵也堅倣其意拜洛為官箴左右】自是不復獵【復扶又翻】 大司馬温恃其材畧位望陰蓄不臣之志嘗撫枕歎曰男子不能流芳百世亦當遺臭萬年【桓温心迹固不畏人之知之也然而不獲逞者制於命也孰謂天位可以智力奸邪】術士杜炅【炅古迥翻】能知人貴賤温問炅以禄位所至炅曰明公勲格宇宙【據孔安國尚書注格至也】位極人臣温不悦【其志願不止于此故不悦】温欲先立功河朔以收時望還受九錫及枋頭之敗威名頓挫【枋頭之敗事見上卷太和四年】既克壽春謂參軍郗超曰【郗丑之翻】足以雪枋頭之恥乎超曰未也久之超就温宿中夜謂温曰明公都無所慮乎温曰卿欲有言邪超曰明公當天下重任今以六十之年敗于大舉不建不世之勲不足以鎮愜民望【愜苦叶翻】温曰然則奈何超曰明公不為伊霍之舉者無以立大威權鎮壓四海温素有心深以為然遂與之定議【超知温心而迎合之温遂與定議】以帝素謹無過而床第易誣【第側里翻又壯士翻床簀也易以䜴翻】乃言帝早有痿疾【楊正衡曰字林痿痺也人垂翻又於佳翻余謂此蓋言陰痿也】嬖人相龍計好朱靈寶等【相與計皆姓也何承天姓苑相悉良翻范曄後漢書有計子勲】參侍内寢二美人田氏孟氏生三男將建儲立王傾移皇基密播此言于民間時人莫能審其虛實十一月癸卯温自廣陵將還姑孰屯于白石【此白石蓋在牛渚西南桓玄破譙王尚之處非陶侃令庾亮所守白石壘也】丁未詣建康諷褚太后請廢帝立丞相會稽王昱并作令草呈之【先草定太后令而呈之於太后會工外翻】太后方在佛屋燒香【建屋於宫中以奉佛故謂之佛屋】内侍啟云外有急奏太后出倚戶視奏數行【行戶剛翻下數十行同】乃曰我本自疑此至半便止索筆益之曰【索山客翻】未亡人不幸罹此百憂感念存沒心焉如割【杜預曰婦人既寡自稱未亡人】己酉温集百官于朝堂【朝直遥翻下同】廢立既曠代所無莫有識其故典者百官震慄温亦色動不知所為尚書左僕射王彪之知事不可止乃謂温曰公阿衡皇家【伊尹曰阿衡放太甲于桐喻温廢立行伊尹之事也孔安國曰阿倚衡平】當依傍先代【傍蒲浪翻】乃命取漢書霍光傳禮度儀制定於須臾【用霍光廢昌邑王故事傳直戀翻】彪之朝服當階神彩毅然曾無懼容文武儀準莫不取定朝廷以此服之【晉朝以此服王彪之余甚恨彪之得此名于晉朝也彪之父彬不畏死以折王敦此為可服耳】於是宣太后令廢帝為東海王以丞相録尚書事會稽王昱統承皇極【會工外翻】百官入太極前殿温使督護竺瑶散騎侍郎劉亨收帝璽綬【散悉亶翻騎奇寄翻璽斯氏翻綬音受 考異曰帝紀三十國春秋亨皆作享後魏書僭晉傳作亨今從之】帝著白帢單衣【著側略翻帢苦洽翻】步下西堂乘犢車出神虎門【晉制諸公給朝車安車皁輪犢車各一乘東漢都雒陽宫有廣義神虎門賢注曰廣義神虎洛陽宫西門也在金商門外然則神虎門亦建康宫西門乎】羣臣拜辭莫不歔欷【歔音虚欷許既翻又音希】侍御史殿中監將兵百人衛送東海第【殿中監掌監天子服御之事將即亮翻】温帥百官具乘輿法駕迎會稽王于會稽邸【帥讀曰率桑繩證翻會工外翻】王於朝堂變服著平巾幘單衣東向流涕拜受璽綬【平巾幘蓋即平上幘單衣江左諸人所以見尊者之服所謂巾褠也】是日即皇帝位改元【改元咸安】温出次中堂分兵屯衛温有足疾詔乘輿入殿【乘如字】温撰辭欲陳述廢立本意【撰雛免翻預撰辭欲入見而陳之】帝引見【見賢遍翻】便泣下數十行【行戶剛翻】温兢懼竟不能一言而出太宰武陵王晞好習武事【好呼到翻】為温所忌欲廢之以事示王彪之彪之曰武陵親尊【武陵王晞亦元帝子出繼武陵王喆後】未有顯罪不可以猜嫌之間便相廢徙公建立聖明當崇奬王室與伊周同美此大事宜更深詳温曰此已成事卿勿復言【王彪之能全晞于會稽輔政之時而不能全之于會稽纘服之日會稽可以理喻而習武者桓温之所忌也復扶又翻】乙卯温表晞聚納輕剽【剽匹妙翻】息綜矜忍【息子也】袁真叛逆事相連染頃日猜懼將成亂階【温以此誣晞】請免晞官以王歸藩從之并免其世子綜梁王㻱等官【㻱與璡同音津】温使魏郡太守毛安之帥所領宿衛殿中【沈約曰南徐州備有徐兖幽冀青并揚七州郡邑】安之虎生之弟也庚戌尊褚太后曰崇德太后初殷浩卒大司馬温使人齎書弔之浩子㳙不荅【㳙圭淵翻】亦不詣温而與武陵王晞遊廣州刺史庾藴希之弟也素與温有隙温惡殷庾宗彊欲去之【惡烏路翻去羌呂翻】辛亥使其弟祕逼新蔡王晃【晃父邈本汝南王祐之子也嗣新蔡王後】詣西堂叩頭自列【西堂太極殿西堂也自列自陳列其事】稱與晞及子綜著作郎殷㳙太宰長史庾倩掾曹秀舍人劉彊散騎常侍庾柔等謀反帝對之流涕温皆收付廷尉倩柔皆藴之弟也【倩千甸翻掾于絹翻】癸丑温殺東海王三子及其母【即田氏孟氏及所生三男也】甲寅御史中丞譙王恬承温旨請依律誅武陵王晞詔曰悲惋惶怛【惋烏貫翻】非所忍聞況言之哉其更詳議恬承之孫也【譙王氶死於王敦之難承當作氶音注見前】乙卯温重表固請誅晞詞甚酷切帝乃賜温手詔曰若晉祚靈長公便宜奉行前詔如其大運去矣請避賢路温覽之流汗變色乃奏廢晞及其三子家屬皆徙新安郡【吳孫權分丹楊立新都郡武帝太康元年更名新安郡唐為歙州今之徽州】丙辰免新蔡王晃為庶人徙衡陽【吳孫亮分長沙西部都尉置衡陽郡今之衡州】殷㳙庾倩曹秀劉彊庾柔皆族誅庾藴飲酖死藴兄東陽太守友子婦桓豁之女也故温特赦之庾希聞難【難乃旦翻】與弟會稽參軍邈及子攸之逃于海陵陂澤中【海陵縣前漢屬臨淮郡後漢晉屬廣陵郡今泰州即其地】温既誅殷庾威埶翕赫【翕盛也赫炎之極也】侍中謝安見温遥拜温驚曰安石卿何事乃爾安曰未有君拜于前臣揖于後【當是時晉之君臣蓋可知矣春秋之義所謂微而顯者也】戊午大赦增文武位二等己未温如白石上書求歸姑孰庚申詔進温丞相大司馬如故留京師輔政温固辭仍請還鎮辛酉温自白石還姑孰秦王堅聞温廢立謂羣臣曰温前敗灞上【見九十九卷穆帝太和十年】後敗枋頭【見上卷太和四年】不能思愆自貶以謝百姓方更廢君以自說【說載記作悦讀當從悦一曰說讀如字謂自解說也】六十之叟舉動如此將何以自容于四海乎諺曰怒其室而作色於父其桓温之謂矣 秦車騎大將軍王猛以六州任重言于秦王堅請改授親賢及府選便宜輒已停寢【騎奇寄翻堅先是命猛以便宜選賢俊補六州郡縣守令】别乞一州自效堅報曰朕之于卿義則君臣親踰骨肉雖復桓昭之有管樂【齊桓公有管仲燕昭王有樂毅】玄德之有孔明自謂踰之夫人主勞于求才逸于得士【王褒聖主得賢臣頌曰君人者勤於求賢而逸于得人】既以六州相委則朕無東顧之憂非所以為優崇乃朕自求安逸也夫取之不易守之亦難【易以䜴翻】苟任非其人患生慮表【表外也孔頴達曰界外之畔為表】豈獨朕之憂亦卿之責也故虛位台鼎而以分陜為先【陜式冉翻】卿未照朕心殊乖素望新政俟才宜速銓補俟東方化洽當衮衣西歸【周公東征周大夫為作九罭之詩其辭曰九罭之魚鱒魴我覯之子衮衣繡裳又曰是以有衮衣兮無以我公歸兮無使我心悲兮箋云王迎周公當以上公之服】仍遣侍中梁讜詣鄴諭旨猛乃視事如故【史言苻堅王猛君臣相與之至所以猛得展其才讜多朗翻】 十二月大司馬温奏廢放之人屏之以遠【屏必政翻又必郢翻】不可以臨黎元東海王宜依昌邑故事【昌邑事見二十四卷漢昭帝元平元年】築第吳郡太后詔曰使為庶人情有不忍可特封王温又奏可封海西縣侯庚寅封海西縣公 【考異曰海西公紀云咸安二年正月降封今從簡文帝紀】温威振内外帝雖處尊位【處昌呂翻】拱默而已常懼廢黜先是熒惑守太微端門【天文志太微南蕃中二星間曰端門先悉薦翻】踰月而海西廢辛卯熒惑逆行入太微帝甚惡之【惡烏路翻】中書侍郎郗超在直【入直省中也】帝謂超曰命之脩短本所不計故當無復近日事邪【帝之為撫軍也辟超為掾故於今敢以情問之復扶又翻】超曰大司馬臣温方内固社稷外恢經略非常之事臣以百口保之及超請急省其父【晉令急假者五日一急一歲以六十日為限史書所稱取急請急皆謂假也省悉景翻】帝曰致意尊公家國之事遂至於此由吾不能以道匡衛愧歎之深言何能諭因詠庾闡詩云志士痛朝危忠臣哀主辱遂泣下霑襟【此亦清談但情溢于言外耳朝直遥翻下同】帝美風儀善容止留心典籍凝塵滿席湛如也雖神識恬暢然無濟世大略謝安以為惠帝之流但清談差勝耳【清談無益於國事謝安當此之時能立此論可謂拔乎流俗者也】郗超以温故朝中皆畏事之謝安嘗與左衛將軍王坦之共詣超日旰未得前【旴古案翻】坦之欲去安曰獨不能為性命忍須臾邪【史言謝安于風流之中能處事應物又郗超勢燄如此桓温既死之後超得終于牖下蓋以智免也為于偽翻】 秦以河州刺史李辯領興晉太守還鎮枹罕【興晉枹罕河西張氏皆置為郡興晉亦當近枹罕界】徙涼州治金城【自天水徙治金城】張天錫聞秦有兼并之志大懼【以秦徙鎮偪之故懼】立壇于姑臧西刑三牲帥其官屬遥與晉三公盟【帥讀曰率下同】遣從事中郎韓博奉表送盟文并獻書於大司馬温期以明年夏會于上邽【欲使晉起兵攻蜀而出會于上邽也】 是歲秦益州刺史王統攻隴西鮮卑乞伏司繁于度堅山司繁帥騎三萬拒統于苑川統潜襲度堅山【乞伏氏先自漠北南出屯高平川又自高平西南遷麥田山司繁又自麥田遷于度堅山水經注苑川在天水勇士縣界杜佑曰在蘭州五泉縣界以下文乞伏吐雷為勇士護軍觀之則水經注為是】司繁部落五萬餘皆降於統其衆聞妻子已降秦不戰而潰司繁無所歸亦詣統降【降戶江翻】秦王堅以司繁為南單于留之長安以司繁從叔吐雷為勇士護軍【勇士漢縣晉省此因漢縣名而置護軍】撫其部衆【為後乞伏步頹以鮮卑叛秦張本】二年春二月秦以清河房曠為尚書左丞徵曠兄默及清河崔逞燕國韓亂為尚書郎北平陽陟田勰陽瑶為著作佐郎【晉志著作郎一人謂之大著作專掌史任又置佐著作郎八人勰音協】郝略為清河相皆關東士望王猛所薦也瑶騖之子也【陽騖仕燕歷事三朝騖音務】冠軍將軍慕容垂言于秦王堅曰臣叔父評燕之惡來輩也【惡來以多力事紂紂嬖之以亡國惡來輩一作惡來革史記曰惡來善毁讒諸侯以此益疏輩當作革】不宜復汚聖朝【復扶又翻汚烏故翻】願陛下為燕戮之【為于偽翻下為人同】堅乃出評為范陽太守燕之諸王悉補邊郡臣光曰古之人滅人之國而人悦何哉為人除害故也【此惟湯武足以當之下此則漢高帝猶庶幾焉為于偽翻】彼慕容評者蔽君專政忌賢疾功愚闇貪虐以喪其國【喪息浪翻】國亡不死逃遁見禽【事見上卷海西公太和五年】秦王堅不以為誅首又從而寵秩之【秩序也官也寵秩謂寵而序其官使不失次也】是愛一人而不愛一國之人也其失人心多矣是以施恩于人而人莫之<br />
<br />
  恩盡誠于人而人莫之誠卒于功名不遂容身無所【卒子恤翻】由不得其道故也<br />
<br />
  三月戊午遣侍中王坦之徵大司馬温入輔温復辭【復扶又翻】 秦王堅詔關東之民學通一經才成一藝者在所以禮送之在官百石以上學不通一經才不成一藝者罷遣還民【苻堅之政如此而猶不能終況不及苻堅者乎】 夏四月徙海西公于吳縣西柴里敕吳國内史刁彞防衛又遣御史顧允監察之【監工銜翻】彞協之子也【刁協元帝信用之】 六月癸酉秦以王猛為丞相中書監尚書令太子太傅司隸校尉特進常侍持節將軍侯如故【仍帶特進散騎常侍使持節車騎大將軍清河郡侯印綬也】陽平公融為使持節都督六州諸軍事鎮東大將軍冀州牧【代王猛鎮鄴使疏吏翻】 庾希庾邈與故青州刺史武沈之子遵聚衆夜入京口城【沈持林翻】晉陵太守卞踰城奔曲阿【丁含翻沈約曰吳時分無錫以西為毗陵郡治丹徒後復還毗陵東海王越世子名毗東海國故食毗陵永嘉五年改為晉陵大興初郡及丹徒縣悉治京口】希詐稱受海西公密旨誅大司馬温建康震擾内外戒嚴卞發諸縣兵二千人擊希希敗閉城自守温遣東海内史周少孫討之【元帝割吳郡海虞縣之北境為東海郡】秋七月壬辰拔其城擒希邈及其親黨皆斬之【庾亮之後滅矣】壼之子也【卞壼事元明二帝死于蘇峻之難】 甲寅帝不豫急召大司馬温入輔一日一夜發四詔温辭不至初帝為會稽王娶王述從妹為妃生世子道生及弟俞生道生疎躁無行【從才用翻行下孟翻】母子皆以幽廢死餘三子郁朱生天流皆早夭【夭於紹翻】諸姬絶孕將十年王使善相者視之【孕以證翻相息亮翻下同】皆曰非其人又使視諸婢媵【媵以證翻卑女為婢婢女之下者送女從嫁曰媵】有李陵容者在織坊中黑而長宮人謂之崑崙【謂其人如崑崙也崑崙國在南海外崙盧昆翻】相者驚曰此其人也王召之侍寢生子昌明及道子【晉書曰初簡文帝見䜟曰晉祚盡昌明及孝武帝之在孕也李太后夢神人謂之曰汝生男以昌明為字及產東方始明因以為名焉帝後悟乃流涕及孝武帝崩晉自此傾矣】己未立昌明為皇太子生十年矣以道子為琅邪王領會稽國以奉帝母鄭太妃之祀【帝封琅邪王所生母鄭夫人薨固請服重徙封會稽王追號鄭夫人為會稽太妃會工外翻】遺詔大司馬温依周公居攝故事又曰少子可輔者輔之如不可君自取之【用漢昭烈屬諸葛亮之言少詩照翻】侍中王坦之自持詔入于帝前毁之帝曰天下儻來之運卿何所嫌坦之曰天下宣元之天下【宣帝肇基帝業元帝中興故云然】陛下何得專之帝乃使坦之改詔曰家國事一稟大司馬如諸葛武侯王丞相故事【王丞相導也】是日帝崩【年五十三】羣臣疑惑未敢立嗣或曰當須大司馬處分【處昌呂翻分扶問翻】尚書僕射王彪之正色曰天子崩太子代立大司馬何容得異若先面諮必反為所責朝議乃定【朝直遥翻】太子即皇帝位大赦崇德太后令【康獻褚太后既歸政于穆帝居崇德宫及哀帝海西公之世復臨朝稱制海西既廢簡文即位尊后為崇德太后】以帝沖幼加在諒闇【闍音陰】令温依周公居攝故事事已施行王彪之曰此異常大事大司馬必當固讓使萬機停滯稽廢山陵未敢奉令謹具封還事遂不行【此事即封還詔書之始也】温望簡文臨終禪位於己不爾便當居攝既不副所望甚憤怨與弟冲書曰遺詔使吾依武侯王公故事耳温疑王坦之謝安所為心銜之詔謝安徵温入輔温又辭 八月秦丞相猛至長安復加都督中外諸軍事【復扶又翻】猛辭曰元相之重儲傅之尊端右事繁京牧任大總督戎機出納帝命【元相丞相也儲傅太子太傅也端右尚書令也京牧司隸校尉也總督戎機都督中外諸軍事也出納帝命中書監常侍之職也】文武兩寄巨細並關以伊呂蕭鄧之賢尚不能兼【謂伊尹呂望蕭何鄧禹也】況臣猛之無似【無似猶言不肖也】章三四上【上時掌翻】秦王堅不許曰朕方混壹四海非卿無可委者卿之不得辭宰相猶朕不得辭天下也猛為相堅端拱于上百官總已於下軍國内外之事無不由之猛剛明清肅善惡著白放黜尸素【尸素尸位素餐者也】顯拔幽滯勸課農桑練習軍旅官必當才刑必當罪由是國富兵彊戰無不克秦國大治【治直吏翻】堅敕太子宏及長樂公丕等曰汝事王公如事我也【樂音洛】陽平公融在冀州高選綱紀【綱紀謂官屬綱紀衆事者也】以尚書郎房默河間相申紹為治中别駕【姓譜房姓本自丹朱舜封為房邑侯子陵以父封為氏】清河崔宏為州從事管記室融年少為政好新奇貴苛察申紹數規正【少詩照翻好呼到翻數所角翻下同】導以寛和融雖敬之未能盡從後紹出為濟北太守【濟子禮翻】融屢以過失聞數致譴讓乃自恨不用紹言融嘗坐擅起學舍為有司所糾遣主簿李纂詣長安自理纂憂懼道卒【卒子恤翻】融問申紹誰可使者紹曰燕尚書郎高泰清辯有膽智可使也先是丞相猛及融屢辟泰泰不起【先悉薦翻】至是融謂泰曰君子救人之急卿不得復辭【復扶又翻】泰乃從命至長安猛見之笑曰高子伯於今乃來何其遟也【高泰字子伯】泰曰罪人來就刑何問遟速猛曰何謂也泰曰昔魯僖公以泮宫發頌【詩魯頌泮水頌僖公能脩泮宫也】齊宣王以稷下垂聲【史記齊宣王喜文學遊說之士騶衍淳于髠田駢慎到接子環淵之徒七十六人皆賜列第稷下學士且數百千人劉向别録曰齊有稷門城門也談說之士期會於稷下也虞喜曰齊有稷山立館其下以待遊士】今陽平公開建學宫追蹤齊魯未聞明詔褒美乃更煩有司舉劾【劾戶槩翻又戶得翻】明公阿衡聖朝懲勸如此下吏何所逃其罪乎猛曰是吾過也事遂得釋猛因歎曰高子伯豈陽平所宜吏乎言于秦王堅堅召見悦之問以為治之本對曰治本在得人得人在審舉審舉在核真未有官得其人而國家不治者也【治直吏翻】堅曰可謂辭簡而理博矣以為尚書郎泰固請還州【還冀州也】堅許之 九月追尊故會稽王妃王氏曰順皇后【即王述從妹也會工外翻】尊帝母李氏為淑妃 冬十月丁卯葬簡文帝于高平陵 彭城妖人盧悚【晉氏南渡僑置彭城郡於晉陵界妖於驕翻】自稱大道祭酒事之者八百餘家十一月遣弟子許龍如吳晨到海西公門稱太后密詔奉迎興復公初欲從之納保母諫而止龍曰大事垂捷焉用兒女子言乎【焉於䖍翻】公曰我得罪于此幸蒙寛宥豈敢妄動且太后有詔便應官屬來何獨使汝也汝必為亂因叱左右縛之龍懼而走甲午悚帥衆三百人晨攻廣莫門【廣莫門建康城北門也帥讀曰率下同】詐稱海西公還由雲龍門突入殿庭【雲龍門建康宫門也】略取武庫甲仗門下吏士駭愕不知所為【吏士守衛雲龍門者也】游擊將軍毛安之聞難帥衆直入雲龍門【難乃旦翻帥讀曰率】手自奮擊左衛將軍殷康中領軍桓祕入止車門與安之并力討誅之并黨與死者數百人海西公深慮横禍【横戶孟翻】專飲酒恣聲色有子不育時人憐之朝廷知其安于屈辱故不復為虞【虞防也備也復扶又翻】 秦都督北蕃諸軍事鎮北大將軍開府儀同三司朔方桓侯梁平老卒平老在鎮十餘年鮮卑匈奴憚而愛之【平老鎮朔方始一百卷穆帝升平三年】 三吳大旱人多餓死【吳郡吳興義興為三吳注已見前】<br />
<br />
  烈宗孝武皇帝上之上【諱曜字昌明簡文帝第三子諡法五宗安之曰孝克定禍亂曰武】<br />
<br />
  寧康元年春正月己卯朔大赦改元 二月大司馬温來朝【朝直遥翻】辛巳詔吏部尚書謝安侍中王坦之迎于新亭是時都下人情恟恟【恟許勇翻】或云欲誅王謝因移晉室坦之甚懼安神色不變曰晉祚存亡決於此行温既至百官拜于道側温大陳兵衛延見朝士【朝直遥翻】有位望者皆戰慴失色【位列位也中庭左右謂之位孟子曰賢者在位能者在職則有位者公卿大臣也望名望也慴質涉翻】坦之流汗沾衣倒執手版【沈約曰手版則古笏矣尚書令僕射尚書手版頭復有白筆以紫皮裹之名笏】安從容就席【從千容翻】坐定謂温曰安聞諸侯有道守在四隣【左傳楚沈尹戌曰天子守在四夷諸侯守在四隣】明公何須壁後置人邪温笑曰正自不能不爾遂命左右撤之與安笑語移日【史言王坦之雖忠於晉室而識度劣於謝安移日言笑語之久不覺日晷之移】郗超常為温謀主【郗丑之翻】安與坦之見温温使超臥帳中聽其言風動帳開安笑曰郗生可謂入幕之賓矣時天子幼弱外有彊臣安與坦之盡忠輔衛卒安晉室【卒子恤翻】温治盧悚入宮事【治直之翻】收尚書陸始付廷尉免桓祕官連坐者甚衆遷毛安之為左衛將軍桓祕由是怨温三月温有疾停建康十四日甲午還姑孰 夏代王什翼犍使燕鳳入貢于秦【犍居言翻燕於賢翻姓也】 秋七月己亥南郡宣武公桓温薨初温疾篤諷朝廷求九錫屢使人趣之【趣讀曰促】謝安王坦之故緩其事【有心為之謂之故】使袁宏具草宏以示王彪之彪之歎其文辭之美因曰卿固大才安可以此示人【言不當為此文也】謝安見其草輒改之由是歷旬不就宏密謀於彪之彪之曰聞彼病日增亦當不復支久自可更小遲迴【安晉之功人皆歸之謝安王坦之彪之實預有力於其間復扶又翻】温弟江州刺史沖問温以謝安王坦之所任温曰渠等不為汝所處分【吳俗謂他人為渠儂處昌呂翻分扶問翻】其意以為已存彼必不敢立異死則非冲所制若害之無益于沖更失時望故也【觀桓温所以待安坦之者如此二人者豈易及哉】温以世子熙才弱使沖領其衆於是桓祕與熙弟濟謀共殺冲冲密知之不敢入俄頃温薨冲先遣力士拘録熙濟而後臨喪【録收也】祕遂被廢棄熙濟俱徙長沙詔葬温依漢霍光及安平獻王故事冲稱温遺命以少子玄為嗣【為桓玄簒晉張本少詩照翻】時方五歲襲封南郡公庚戌加右將軍荆州刺史桓豁征西將軍督荆揚雍交廣五州諸軍事【揚恐當作梁雍於用翻】桓冲為中軍將軍都督揚豫江三州諸軍事揚豫二州刺史鎮姑孰竟陵太守桓石秀為寧遠將軍江州刺史鎮潯陽【三分温所統以授其弟姪】石秀豁之子也冲既代温居任盡忠王室或勸冲誅除時望專執時權冲不從始温在鎮死罪皆專決不請冲以為生殺之重當歸朝廷凡大辟皆先上【辟毗亦翻上時掌翻】須報然後行之【史言桓冲事晉朝忠順】謝安以天子幼冲新喪元輔【喪息浪翻】欲請崇德太后臨朝王彪之曰前世人主幼在襁褓母子一體故可臨朝【朝直遥翻下同】太后亦不能決事要須顧問大臣今上年出十歲垂及冠婚【冠古玩翻】反令從嫂臨朝【帝元帝之孫於康帝為從弟故太后為從嫂從才用翻】示人主幼弱豈所以光揚聖德乎諸公必欲行此豈僕所制所惜者大體耳安不欲委任桓冲故使太后臨朝已得以專獻替裁決遂不從彪之之言【史言彪之所陳者正義謝安所行者時宜】八月壬子太后復臨朝攝政【復扶又翻】 梁州刺史楊亮遣其子廣襲仇池【簡文帝咸安元年秦取仇池】與秦梁州刺史楊安戰廣兵敗沮水諸戍皆委城奔潰【班志沮水出武都沮縣東狼谷東流合為漢水晉蓋阻沮水列戍以備秦沮千余翻】亮懼退守磬險九月安進攻漢川【漢川即漢中郡之地】 丙申以王彪之為尚書令謝安為僕射領吏部共掌朝政【朝直遥翻】安每歎曰朝廷大事衆所不能決者以諮王公無不立決 以吳國内史刁彞為徐兖二州刺史鎮廣陵 冬秦王堅使益州刺史王統祕書監朱肜帥卒二萬出漢川【肜余冲翻帥讀曰率下同】前禁將軍毛當【秦置左右前後四禁將軍】鷹揚將軍徐成帥卒三萬出劔門入寇梁益梁州刺史楊亮帥巴獠萬餘拒之【蜀先無獠李勢之時始自山出獠盧皓翻】戰于青谷【新唐志洋州真符縣本華陽縣開元十八年析興道置天寶八載開清水谷路興道縣即興埶之地】亮兵敗奔固西城【西城縣漢屬漢中郡魏晉屬魏興郡奔固者奔西城以自固也】肜遂拔漢中徐成攻劔閣克之楊安進攻梓潼梓潼太守周虓固守涪城遣步騎數千送母妻自漢水趣江陵朱肜邀而獲之【此漢水蓋蜀人所謂西漢水也與涪水會至渝州入江順江而下則達江陵然朱肜克漢中因得邀獲虓母妻則又似自漢中之漢水趣江陵但秦兵已至梓潼自涪以北皆為秦有虓母妻安能越劔閣取漢水路而趣江陵乎意謂當以此漢水為西漢水虓虚交翻涪音浮趣七喻翻】虓遂降于安【降戶江翻下同】十一月安克梓潼【梓潼縣漢屬廣漢郡劉蜀分為梓潼郡治涪潼音同】荆州刺史桓豁遣江夏相笁瑤救梁益瑶聞廣漢太守趙長戰死引兵退益州刺史周仲孫勒兵拒朱肜于緜竹聞毛當將至成都仲孫帥騎五千奔于南中秦遂取梁益二州卭筰夜郎皆附于秦【卭渠容翻筰才各翻】秦王堅以楊安為益州牧鎮成都毛當為梁州刺史鎮漢中姚萇為寧州刺史屯墊江【墊音疊】王統為南秦州刺史鎮仇池秦王堅欲以周虓為尚書郎虓曰蒙晉厚恩但老母見獲失節于此母子獲全秦之惠也雖公侯之貴不以為榮況郎官乎遂不仕每見堅或箕踞而坐呼為氐賊【堅本氐也故以氐賊呼之此必虓母死後事】嘗值元會【正月一日為元日是日朝會為元會】儀衛甚盛堅問之曰晉朝元會與此何如虓攘袂厲聲曰犬羊相聚何敢比擬天朝【秦之君臣皆六夷也故詆之為犬羊天朝謂晉也朝直遥翻】秦人以虓不遜屢請殺之堅待之彌厚周仲孫坐失守免官桓冲以冠軍將軍毛虎生為益州刺史領建平太守【冠古玩翻】以虎生子球為梓潼太守虎生與球伐秦至巴西以糧乏退屯巴東 以侍中王坦之為中書令領丹楊尹是歲鮮卑勃寒掠隴右【勃寒亦隴西鮮卑也】秦王堅使乞伏司<br />
<br />
  繁討之勃寒請降遂使司繁鎮勇士川【勇士川即漢天水勇士縣之地】有彗星出于尾箕長十餘丈【彗祥歲翻又旋芮翻又徐醉翻長直亮翻】經<br />
<br />
  太微掃東井自四月始見及秋冬不滅秦太史令張孟言于秦王堅曰尾箕燕分東井秦分【天文志尾九星箕四星燕幽州分東井八星秦雍州分見賢遍翻分扶問翻】今彗起尾箕而掃東井十年之後燕當滅秦二十年之後代當滅燕【按天文志雲中入東井一度定襄入東井八度鴈門入東井十六度代郡入東井二十八度是皆拓跋氏所有之地也所以知代當滅燕者天道好還彗起燕分而掃秦分此燕滅秦之徵秦已滅矣代乘天道好還之運反而滅燕自然之大數也太元十年慕容冲破長安距是歲僅十一年安帝隆安元年拓跋珪克中山距是歲二十三年】慕容暐父子兄弟我之仇敵而布列朝廷貴盛莫二臣竊憂之宜翦其魁桀者以消天變堅不聽陽平公融上疏曰東胡跨據六州【鮮卑東胡之餘種也】南面稱帝陛下勞師累年然後得之【事見上卷海西公太和四年五年】本非慕義而來今陛下親而幸之使其父兄子弟森然滿朝【木多為林森然猶林然也朝直遥翻】執權履職埶傾勲舊臣愚以為狼虎之心終不可養星變如此願少留意堅報曰朕方混六合為一家視夷狄為赤子汝宜息慮勿懷耿介【詩曰憂心耿耿賢曰介介猶耿耿也】夫惟修德可以禳災苟能内求諸已何懼外患乎【史言苻堅養虎自遺患為悔不用融言張本】 二年春正月癸未朔大赦 己酉刁彞卒二月癸丑以王坦之為都督徐兖青三州諸軍事徐兖二州刺史鎮廣陵詔謝安摠中書【王坦之出鎮安兼摠中書】安好聲律朞功之慘不廢絲竹【朞功朞及大功小功之喪也好呼到翻】士大夫效之遂以成俗王坦之屢以書苦諫之曰天下之寶當為天下惜之【為于偽翻言禮法為天下之寶】安不能從 三月秦太尉建寧烈公李威卒夏五月蜀人張育楊光起兵擊秦有衆二萬遣使來<br />
<br />
  請兵【使疏吏翻】秦王堅遣鎮軍將軍鄧羌帥甲士五萬討之【帥讀曰率】益州刺史笁瑤威遠將軍桓石䖍帥衆三萬攻墊江姚萇兵敗退屯五城【晉志廣漢郡有五城縣武帝咸寧四年立唐梓州之玄武縣也華陽國志云漢時立倉於此發五縣人尉部主之後因以為五城縣有五城山】瑶石虔屯巴東張育自號蜀王與巴獠酋帥張重尹萬萬餘人進圍成都【獠魯皓翻酋慈由翻帥所類翻】六月育改元黑龍秋七月張育與張重等爭權舉兵相攻秦楊安鄧羌襲育敗之【敗補邁翻下同】育與楊光退屯緜竹八月鄧羌敗晉兵于涪西九月楊安敗張重尹萬于成都南【敗補邁翻】重死斬首二萬三千級鄧羌撃張育楊光于緜竹皆斬之益州復入于秦【復扶又翻】冬十二月有人入秦明光殿大呼曰甲申乙酉魚羊食人悲哉無復遺【魚羊合成鮮字謂鮮卑也是後慕容起兵攻秦果在甲申乙酉之歲呼火故翻】秦王堅命執之不獲祕書監朱肜祕書侍郎略陽趙整【晉祕書省有丞有郎無侍郎秦以整為祕書郎内侍左右故曰侍郎】固請誅鮮卑堅不聽整宦官也博聞彊記能屬文【屬之欲翻】好直言上書及面諫前後五十餘事【好呼到翻上時掌翻】慕容垂夫人得幸于堅【即段夫人也】堅與之同輦遊于後庭整歌曰不見雀來入鷰室但見浮雲蔽白日堅改容謝之命夫人下輦 是歲代王什翼犍擊劉衛辰南走【衛辰之下更有衛辰字文意乃足為下衛辰求救於秦張本犍居言翻】<br />
<br />
  三年春正月辛亥大赦 夏五月丙午藍田獻侯王坦之卒臨終與謝安桓冲書惟以國家為憂言不及私桓冲以謝安素有重望欲以揚州讓之自求外出桓氏族黨皆以為非計莫不扼腕固諫【腕烏貫翻】郗超亦深止之【揚州統攝京畿權仕要重故皆止冲】冲皆不聽處之澹然【處昌呂翻澹徒覽翻】甲寅詔以冲都督徐豫兖青揚五州諸軍事徐州刺史鎮京口以安領揚州刺史並加侍中 六月秦清河武侯王猛寢疾秦王堅親為之祈南北郊及宗廟社稷【為于偽翻下同】分遣侍臣徧禱河嶽諸神【蓋黄河及華嶽諸神不盡徧四嶽也】猛疾少瘳為之赦殊死以下【身首横分為殊死少詩翻】猛上疏曰不圖陛下以臣之命而虧天地之德開闢以來未之有也臣聞報德莫如盡言謹以垂沒之命竊獻遺欵【欵誠也】伏惟陛下威烈振乎八荒【八方之外為八荒爾雅觚竹北戶西王母日下謂之四荒】聲敎光乎六合【六合天地東西南北】九州百郡十居其七平燕定蜀有如拾芥【師古曰草芥之横在地上者俛而拾之言易而必得也】夫善作者不必善成善始者不必善終【樂毅荅燕惠王書之言】是以古先哲王知功業之不易戰戰兢兢如臨深谷【詩小宛惴惴小心如臨于谷戰戰兢兢如履薄冰易以豉翻】伏惟陛下追蹤前聖天下幸甚堅覽之悲慟秋七月堅親至猛第視疾訪以後事猛曰晉雖僻處江南然正朔相承【王猛事秦亦知正統之在江南徐光之論非矣處昌呂翻】上下安和臣没之後願勿以晉為圖鮮卑西羌我之仇敵終為人患【後卒如猛言】宜漸除之以便社稷言終而卒堅比歛三臨哭【比必寐翻及也歛力贍翻臨如字】謂太子宏曰天不欲使吾平壹六合邪何奪吾景略之速也葬之如漢霍光故事 八月癸巳立皇后王氏大赦后濛之孫也【王濛善清談與劉惔齊名】以后父晉陵太守藴為光禄大夫領五兵尚書【魏始置五兵尚書謂總録中兵外兵别兵都兵騎兵事也】封建昌侯藴固辭不受 九月帝講孝經始覽典籍延儒士謝安薦東莞徐邈補中書舍人【晉初中書置通事舍人各一人掌呈奏案及掌詔命沈約曰晉置中書侍郎又置舍人一人通事一人江左初合舍人通事謂之通事舍人掌呈案奏章後省通事莞音官】每被顧問多所匡益帝或晏集酣樂之後好為手詔詩章以賜侍臣或文詞率爾所言穢雜邈應時收歛還省【省謂中書省被皮義翻樂音洛好呼到翻】刋削皆使可觀經帝重覽【重直龍翻】然後出之時議以此多邈 冬十月癸酉朔日有食之 秦王堅下詔曰新喪賢輔百司或未稱朕心可置聽訟觀于未央南【喪息浪翻稱尺證翻下同觀古玩翻】朕五日一臨以求民隱今天下雖未大定權可偃武修文以稱武侯雅旨其增崇儒敎禁老莊圖䜟之學犯者棄市【王猛諡武侯稱尺證翻䜟楚譛翻】妙簡學生太子及公侯百僚之子皆就學受業中外四禁二衛四軍長上將士皆令受學【秦有中軍外軍將軍前禁後禁左禁右禁將軍是為四禁左衛右衛將軍是為二衛衛軍撫軍鎮軍冠軍將軍是為四軍長上者長上宿衛將士也上時掌翻將即亮翻】二十人給一經生教讀音句後宫置典學以教掖庭選閹人及女隸敏慧者詣博士授經【女隸沒入為官婢者奚官次是也】尚書郎王佩讀䜟堅殺之學䜟者遂絶<br />
<br />
  資治通鑑卷一百三  <br>
   </div> 

<script src="/search/ajaxskft.js"> </script>
 <div class="clear"></div>
<br>
<br>
 <!-- a.d-->

 <!--
<div class="info_share">
</div> 
-->
 <!--info_share--></div>   <!-- end info_content-->
  </div> <!-- end l-->

<div class="r">   <!--r-->



<div class="sidebar"  style="margin-bottom:2px;">

 
<div class="sidebar_title">工具类大全</div>
<div class="sidebar_info">
<strong><a href="http://www.guoxuedashi.com/lsditu/" target="_blank">历史地图</a></strong>  
<a href="http://www.880114.com/" target="_blank">英语宝典</a>  
<a href="http://www.guoxuedashi.com/13jing/" target="_blank">十三经检索</a> 
<br><strong><a href="http://www.guoxuedashi.com/gjtsjc/" target="_blank">古今图书集成</a></strong> 
<a href="http://www.guoxuedashi.com/duilian/" target="_blank">对联大全</a> <strong><a href="http://www.guoxuedashi.com/xiangxingzi/" target="_blank">象形文字典</a></strong> 

<br><a href="http://www.guoxuedashi.com/zixing/yanbian/">字形演变</a>  <strong><a href="http://www.guoxuemi.com/hafo/" target="_blank">哈佛燕京中文善本特藏</a></strong>
<br><strong><a href="http://www.guoxuedashi.com/csfz/" target="_blank">丛书&方志检索器</a></strong> <a href="http://www.guoxuedashi.com/yqjyy/" target="_blank">一切经音义</a>  

<br><strong><a href="http://www.guoxuedashi.com/jiapu/" target="_blank">家谱族谱查询</a></strong>  <strong><a href="http://shufa.guoxuedashi.com/sfzitie/" target="_blank">书法字帖欣赏</a></strong> 
<br>

</div>
</div>


<div class="sidebar" style="margin-bottom:0px;">

<font style="font-size:22px;line-height:32px">QQ交流群9:489193090</font>


<div class="sidebar_title">手机APP 扫描或点击</div>
<div class="sidebar_info">
<table>
<tr>
	<td width=160><a href="http://m.guoxuedashi.com/app/" target="_blank"><img src="/img/gxds-sj.png" width="140"  border="0" alt="国学大师手机版"></a></td>
	<td>
<a href="http://www.guoxuedashi.com/download/" target="_blank">app软件下载专区</a><br>
<a href="http://www.guoxuedashi.com/download/gxds.php" target="_blank">《国学大师》下载</a><br>
<a href="http://www.guoxuedashi.com/download/kxzd.php" target="_blank">《汉字宝典》下载</a><br>
<a href="http://www.guoxuedashi.com/download/scqbd.php" target="_blank">《诗词曲宝典》下载</a><br>
<a href="http://www.guoxuedashi.com/SiKuQuanShu/skqs.php" target="_blank">《四库全书》下载</a><br>
</td>
</tr>
</table>

</div>
</div>


<div class="sidebar2">
<center>


</center>
</div>

<div class="sidebar"  style="margin-bottom:2px;">
<div class="sidebar_title">网站使用教程</div>
<div class="sidebar_info">
<a href="http://www.guoxuedashi.com/help/gjsearch.php" target="_blank">如何在国学大师网下载古籍?</a><br>
<a href="http://www.guoxuedashi.com/zidian/bujian/bjjc.php" target="_blank">如何使用部件查字法快速查字?</a><br>
<a href="http://www.guoxuedashi.com/search/sjc.php" target="_blank">如何在指定的书籍中全文检索?</a><br>
<a href="http://www.guoxuedashi.com/search/skjc.php" target="_blank">如何找到一句话在《四库全书》哪一页?</a><br>
</div>
</div>


<div class="sidebar">
<div class="sidebar_title">热门书籍</div>
<div class="sidebar_info">
<a href="/so.php?sokey=%E8%B5%84%E6%B2%BB%E9%80%9A%E9%89%B4&kt=1">资治通鉴</a> <a href="/24shi/"><strong>二十四史</strong></a>&nbsp; <a href="/a2694/">野史</a>&nbsp; <a href="/SiKuQuanShu/"><strong>四库全书</strong></a>&nbsp;<a href="http://www.guoxuedashi.com/SiKuQuanShu/fanti/">繁体</a>
<br><a href="/so.php?sokey=%E7%BA%A2%E6%A5%BC%E6%A2%A6&kt=1">红楼梦</a> <a href="/a/1858x/">三国演义</a> <a href="/a/1038k/">水浒传</a> <a href="/a/1046t/">西游记</a> <a href="/a/1914o/">封神演义</a>
<br>
<a href="http://www.guoxuedashi.com/so.php?sokeygx=%E4%B8%87%E6%9C%89%E6%96%87%E5%BA%93&submit=&kt=1">万有文库</a> <a href="/a/780t/">古文观止</a> <a href="/a/1024l/">文心雕龙</a> <a href="/a/1704n/">全唐诗</a> <a href="/a/1705h/">全宋词</a>
<br><a href="http://www.guoxuedashi.com/so.php?sokeygx=%E7%99%BE%E8%A1%B2%E6%9C%AC%E4%BA%8C%E5%8D%81%E5%9B%9B%E5%8F%B2&submit=&kt=1"><strong>百衲本二十四史</strong></a>  <a href="http://www.guoxuedashi.com/so.php?sokeygx=%E5%8F%A4%E4%BB%8A%E5%9B%BE%E4%B9%A6%E9%9B%86%E6%88%90&submit=&kt=1"><strong>古今图书集成</strong></a>
<br>

<a href="http://www.guoxuedashi.com/so.php?sokeygx=%E4%B8%9B%E4%B9%A6%E9%9B%86%E6%88%90&submit=&kt=1">丛书集成</a> 
<a href="http://www.guoxuedashi.com/so.php?sokeygx=%E5%9B%9B%E9%83%A8%E4%B8%9B%E5%88%8A&submit=&kt=1"><strong>四部丛刊</strong></a>  
<a href="http://www.guoxuedashi.com/so.php?sokeygx=%E8%AF%B4%E6%96%87%E8%A7%A3%E5%AD%97&submit=&kt=1">說文解字</a> <a href="http://www.guoxuedashi.com/so.php?sokeygx=%E5%85%A8%E4%B8%8A%E5%8F%A4&submit=&kt=1">三国六朝文</a>
<br><a href="http://www.guoxuedashi.com/so.php?sokeytm=%E6%97%A5%E6%9C%AC%E5%86%85%E9%98%81%E6%96%87%E5%BA%93&submit=&kt=1"><strong>日本内阁文库</strong></a> <a href="http://www.guoxuedashi.com/so.php?sokeytm=%E5%9B%BD%E5%9B%BE%E6%96%B9%E5%BF%97%E5%90%88%E9%9B%86&ka=100&submit=">国图方志合集</a> <a href="http://www.guoxuedashi.com/so.php?sokeytm=%E5%90%84%E5%9C%B0%E6%96%B9%E5%BF%97&submit=&kt=1"><strong>各地方志</strong></a>

</div>
</div>


<div class="sidebar2">
<center>

</center>
</div>
<div class="sidebar greenbar">
<div class="sidebar_title green">四库全书</div>
<div class="sidebar_info">

《四库全书》是中国古代最大的丛书,编撰于乾隆年间,由纪昀等360多位高官、学者编撰,3800多人抄写,费时十三年编成。丛书分经、史、子、集四部,故名四库。共有3500多种书,7.9万卷,3.6万册,约8亿字,基本上囊括了古代所有图书,故称“全书”。<a href="http://www.guoxuedashi.com/SiKuQuanShu/">详细>>
</a>

</div> 
</div>

</div>  <!--end r-->

</div>
<!-- 内容区END --> 

<!-- 页脚开始 -->
<div class="shh">

</div>

<div class="w1180" style="margin-top:8px;">
<center><script src="http://www.guoxuedashi.com/img/plus.php?id=3"></script></center>
</div>
<div class="w1180 foot">
<a href="/b/thanks.php">特别致谢</a> | <a href="javascript:window.external.AddFavorite(document.location.href,document.title);">收藏本站</a> | <a href="#">欢迎投稿</a> | <a href="http://www.guoxuedashi.com/forum/">意见建议</a> | <a href="http://www.guoxuemi.com/">国学迷</a> | <a href="http://www.shuowen.net/">说文网</a><script language="javascript" type="text/javascript" src="https://js.users.51.la/17753172.js"></script><br />
  Copyright &copy; 国学大师 古典图书集成 All Rights Reserved.<br>
  
  <span style="font-size:14px">免责声明:本站非营利性站点,以方便网友为主,仅供学习研究。<br>内容由热心网友提供和网上收集,不保留版权。若侵犯了您的权益,来信即刪。scp168@qq.com</span>
  <br />
ICP证:<a href="http://www.beian.miit.gov.cn/" target="_blank">鲁ICP备19060063号</a></div>
<!-- 页脚END --> 
<script src="http://www.guoxuedashi.com/img/plus.php?id=22"></script>
<script src="http://www.guoxuedashi.com/img/tongji.js"></script>

</body>
</html>
