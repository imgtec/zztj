<!DOCTYPE html PUBLIC "-//W3C//DTD XHTML 1.0 Transitional//EN" "http://www.w3.org/TR/xhtml1/DTD/xhtml1-transitional.dtd">
<html xmlns="http://www.w3.org/1999/xhtml">
<head>
<meta http-equiv="Content-Type" content="text/html; charset=utf-8" />
<meta http-equiv="X-UA-Compatible" content="IE=Edge,chrome=1">
<title>資治通鑒_215-資治通鑑卷二百十四_215-資治通鑑卷二百十四</title>
<meta name="Keywords" content="資治通鑒_215-資治通鑑卷二百十四_215-資治通鑑卷二百十四">
<meta name="Description" content="資治通鑒_215-資治通鑑卷二百十四_215-資治通鑑卷二百十四">
<meta http-equiv="Cache-Control" content="no-transform" />
<meta http-equiv="Cache-Control" content="no-siteapp" />
<link href="/img/style.css" rel="stylesheet" type="text/css" />
<script src="/img/m.js?2020"></script> 
</head>
<body>
 <div class="ClassNavi">
<a  href="/24shi/">二十四史</a> | <a href="/SiKuQuanShu/">四库全书</a> | <a href="http://www.guoxuedashi.com/gjtsjc/"><font  color="#FF0000">古今图书集成</font></a> | <a href="/renwu/">历史人物</a> | <a href="/ShuoWenJieZi/"><font  color="#FF0000">说文解字</a></font> | <a href="/chengyu/">成语词典</a> | <a  target="_blank"  href="http://www.guoxuedashi.com/jgwhj/"><font  color="#FF0000">甲骨文合集</font></a> | <a href="/yzjwjc/"><font  color="#FF0000">殷周金文集成</font></a> | <a href="/xiangxingzi/"><font color="#0000FF">象形字典</font></a> | <a href="/13jing/"><font  color="#FF0000">十三经索引</font></a> | <a href="/zixing/"><font  color="#FF0000">字体转换器</font></a> | <a href="/zidian/xz/"><font color="#0000FF">篆书识别</font></a> | <a href="/jinfanyi/">近义反义词</a> | <a href="/duilian/">对联大全</a> | <a href="/jiapu/"><font  color="#0000FF">家谱族谱查询</font></a> | <a href="http://www.guoxuemi.com/hafo/" target="_blank" ><font color="#FF0000">哈佛古籍</font></a> 
</div>

 <!-- 头部导航开始 -->
<div class="w1180 head clearfix">
  <div class="head_logo l"><a title="国学大师官网" href="http://www.guoxuedashi.com" target="_blank"></a></div>
  <div class="head_sr l">
  <div id="head1">
  
  <a href="http://www.guoxuedashi.com/zidian/bujian/" target="_blank" ><img src="http://www.guoxuedashi.com/img/top1.gif" width="88" height="60" border="0" title="部件查字,支持20万汉字"></a>


<a href="http://www.guoxuedashi.com/help/yingpan.php" target="_blank"><img src="http://www.guoxuedashi.com/img/top230.gif" width="600" height="62" border="0" ></a>


  </div>
  <div id="head3"><a href="javascript:" onClick="javascript:window.external.AddFavorite(window.location.href,document.title);">添加收藏</a>
  <br><a href="/help/setie.php">搜索引擎</a>
  <br><a href="/help/zanzhu.php">赞助本站</a></div>
  <div id="head2">
 <a href="http://www.guoxuemi.com/" target="_blank"><img src="http://www.guoxuedashi.com/img/guoxuemi.gif" width="95" height="62" border="0" style="margin-left:2px;" title="国学迷"></a>
  

  </div>
</div>
  <div class="clear"></div>
  <div class="head_nav">
  <p><a href="/">首页</a> | <a href="/ShuKu/">国学书库</a> | <a href="/guji/">影印古籍</a> | <a href="/shici/">诗词宝典</a> | <a   href="/SiKuQuanShu/gxjx.php">精选</a> <b>|</b> <a href="/zidian/">汉语字典</a> | <a href="/hydcd/">汉语词典</a> | <a href="http://www.guoxuedashi.com/zidian/bujian/"><font  color="#CC0066">部件查字</font></a> | <a href="http://www.sfds.cn/"><font  color="#CC0066">书法大师</font></a> | <a href="/jgwhj/">甲骨文</a> <b>|</b> <a href="/b/4/"><font  color="#CC0066">解密</font></a> | <a href="/renwu/">历史人物</a> | <a href="/diangu/">历史典故</a> | <a href="/xingshi/">姓氏</a> | <a href="/minzu/">民族</a> <b>|</b> <a href="/mz/"><font  color="#CC0066">世界名著</font></a> | <a href="/download/">软件下载</a>
</p>
<p><a href="/b/"><font  color="#CC0066">历史</font></a> | <a href="http://skqs.guoxuedashi.com/" target="_blank">四库全书</a> |  <a href="http://www.guoxuedashi.com/search/" target="_blank"><font  color="#CC0066">全文检索</font></a> | <a href="http://www.guoxuedashi.com/shumu/">古籍书目</a> | <a   href="/24shi/">正史</a> <b>|</b> <a href="/chengyu/">成语词典</a> | <a href="/kangxi/" title="康熙字典">康熙字典</a> | <a href="/ShuoWenJieZi/">说文解字</a> | <a href="/zixing/yanbian/">字形演变</a> | <a href="/yzjwjc/">金 文</a> <b>|</b>  <a href="/shijian/nian-hao/">年号</a> | <a href="/diming/">历史地名</a> | <a href="/shijian/">历史事件</a> | <a href="/guanzhi/">官职</a> | <a href="/lishi/">知识</a> <b>|</b> <a href="/zhongyi/">中医中药</a> | <a href="http://www.guoxuedashi.com/forum/">留言反馈</a>
</p>
  </div>
</div>
<!-- 头部导航END --> 
<!-- 内容区开始 --> 
<div class="w1180 clearfix">
  <div class="info l">
   
<div class="clearfix" style="background:#f5faff;">
<script src='http://www.guoxuedashi.com/img/headersou.js'></script>

</div>
  <div class="info_tree"><a href="http://www.guoxuedashi.com">首页</a> > <a href="/SiKuQuanShu/fanti/">四库全书</a>
 > <h1>资治通鉴</h1> <!--         下载:【右键另存为】即可 --></div>
  <div class="info_content zj clearfix">
  
<div class="info_txt clearfix" id="show">
<center style="font-size:24px;">215-資治通鑑卷二百十四</center>
    資治通鑑卷二百十四  宋 司馬光 撰<br />
<br />
  胡三省 音注<br />
<br />
  唐紀三十【起閼逢閹茂盡重光大荒落凡八年】<br />
<br />
  玄宗至道大聖大明孝皇帝中之中<br />
<br />
  開元二十二年春正月己巳上發西京己丑至東都張九齡自韶州入見【見賢遍翻 考異曰唐紀二年六月戊子至東都己丑張九齡至自韶州今從實録】求終喪不許 二月壬寅秦州地連震壞公私屋殆盡吏民壓死者四千餘人命左丞相蕭嵩賑恤【壞音怪賑津忍翻】 方士張果自言有神仙術誑人云堯時為侍中於今數千歲多往來恒山中【恒山時屬定州恒陽縣界誑居况翻恒戶登翻】則天以來屢徵不至相州刺史韋濟薦之上遣中書舍人徐嶠齎璽書迎之【相息亮翻璽斯氏翻】庚寅至東都肩輿入宫恩禮甚厚 張九齡請不禁鑄錢三月庚辰敕百官議之裴耀卿等皆曰一啟此門恐小人棄農逐利而濫惡更甚【自武后以來民間多惡錢官不能禁】祕書監崔沔曰若稅銅折役則官冶可成計估度庸則私鑄無利易而可久簡而難誣【沔彌兖翻折之舌翻易以䜴翻】且夫錢之為物貴以通貨利不在多何待私鑄然後足用也右監門録事參軍劉秩曰【唐十六衛府皆有録事參軍正八品下掌受諸曹及五府之外府事句稽抄目印給紙筆監古衘翻】夫人富則不可以賞勸貧則不可以威禁若許其私鑄貧者必不能為之臣恐貧者益貧而役於富富者益富而逞其欲漢文帝時吳王濞富埓天子鑄錢所致也【事見十四卷漢文帝五年濞匹備翻埒力輟翻】上乃止秩子玄之子也【劉子玄即知幾避帝嫌名以字行】 夏四月壬辰以朔方節度使信安王禕兼關内道采訪處置使增領涇原等十二州【處昌呂翻】 吏部侍郎李林甫柔佞多狡數深結宦官及妃嬪家伺候上動靜無不知之【伺相吏翻】由是每奏對常稱旨上悦之【稱尺證翻】時武惠妃寵幸傾後宫生夀王清諸子莫得為比太子浸疎薄林甫乃因宦官言於惠妃願盡力保護夀王惠妃德之隂為内助由是擢黄門侍郎 【考異曰舊傳云初侍中裴光庭妻武三思女詭譎有材略與林甫私中官高力士本出三思家及光庭卒武氏衘哀祈於力士請林甫代其夫位力士未敢言玄宗使中書令蕭嵩擇相久之以右丞韓休對玄宗然之乃令草詔力士遽漏於武氏乃令林甫白休休既入相甚德林甫與嵩不和乃薦林甫堪為宰相惠妃隂助之因拜黄門侍郎玄宗眷遇益深按光庭妻一寡婦耳豈敢遽引所私代其夫為相韓休正直雖得林甫先報必不至薦之為相今不取】五月戊子以裴耀卿為侍中張九齡為中書令林甫為禮部尚書同中書門下三品【為李林甫得權太子廢張本顔真卿疏曰天寶已後閻官袁思藝日宣詔至中書玄宗動静必告林甫林甫先意奏請玄宗驚喜若神以此權柄恩寵日甚】 上種麥於苑中帥太子以下親往芟之【帥讀曰率芟所衘翻】謂曰此所以薦宗廟故不敢不親且欲使汝曹知稼穡艱難耳又徧以賜侍臣曰比遣人視田中稼多不得實故自種以觀之【種藝之事天有雨暘之不時地有肥磽之不等而人力又有至不至故所收有厚薄之異也若人君不奪農時人得盡其力則地無遺利矣豈必待自種而觀其實哉】 六月壬辰幽州節度使張守珪大破契丹【使疏吏翻下同契欺訖翻又音喫 考異曰實録守珪大破林胡按會要契丹事二十三年守珪大破之盖實録以契丹即戰國時林胡地故云然】遣使獻捷 薛王業疾病上憂之容髪為變【為于偽翻】七月己巳薨贈諡惠宣太子 上以裴耀卿為江淮河南轉運使 【考異曰舊紀云充江淮以南囘造使今從舊食貨志】於河口置輸場八月壬寅於輸場東置河隂倉西置柏崖倉【高宗咸亨二年於洛州河陽縣柏崖置倉開元十年廢今復因舊基置之】三門東置集津倉西置鹽倉鑿漕渠十八里以避三門之險【參考新舊志乃是鑿山開車路十八里非漕渠也】先是舟運江淮之米至東都含嘉倉僦車陸運三百里至陜率兩斛用十錢耀卿令江淮舟運悉輸河隂倉更用河舟運至含嘉倉及太原倉自太原倉入渭輸關中凡三歲運米七百萬斛省僦車錢三十萬緡【按舊制東都含嘉倉積江淮之米載以大輿運而西至于陜三百里率兩斛計庸錢千此耀卿所省之大數也十錢誤當從千錢為是先悉薦翻僦即就翻陜失冉翻更工衡翻 考異曰舊志云四十萬貫今從耀卿傳舊志又云明年耀卿拜侍中蕭炅代焉按耀卿二十一年建此議今年為侍中始置河隂倉後三年方見成効則非作侍中時解此職也】或說耀卿獻所省錢【說式芮翻】耀卿曰此公家贏縮之利耳奈何以之市寵乎悉奏以為市糴錢【糴徙歷翻】 張果固請歸恒山制以為銀青光禄大夫號通玄先生厚賜而遣之後卒好異者奏以為尸解【解佳買翻仙家所謂尸解譬猶蟬蜕蟬飛而蜕在也卒子恤翻好呼到翻】上由是頗信神仙【明皇改集仙為集賢殿是其初心不信神仙也至是則頗信矣又至晩年則深信矣史言正心為難漸入於邪而不自覺】冬十二月戊子朔日有食之 乙巳幽州節度使張<br />
<br />
  守珪斬契丹王屈烈及可突干傳首 【考異曰舊守珪傳屈烈作屈刺契丹傅來年正月傅首今從實録】時可突干連年為邊患趙含章薛楚玉皆不能討守珪到官屢擊破之可突干困迫遣使詐降守珪使管記王悔就撫之悔至其牙帳察契丹上下殊無降意【降戶江翻下同】但稍徙營帳近西北密遣人引突厥謀殺悔以叛悔知之【近其靳翻】牙官李過折 【考異曰舊契丹傅作遇折今從實録及守珪傳】與可突干分典兵馬爭權不叶悔說過折使圖之【說式芮翻】過折夜勒兵斬屈烈及可突干盡誅其黨帥餘衆來降【帥讀曰率】守珪出師紫蒙川【據晉書載記秦漢之間東胡邑于紫蒙之野唐書地理志平州有紫蒙白狼昌黎等戍盖平州之北境契丹之南界也】大閱以鎮撫之梟屈烈可突干首于天津之南【梟堅堯翻】突厥毗伽可汗為其大臣梅録啜所毒【厥九勿翻伽求迦翻可從刋入聲汗音寒啜陟劣翻】未死討誅梅録啜及其族黨既卒子伊然可汗立尋卒弟登利可汗立【卒子恤翻舊史曰登利華言果報也 考異曰舊傳伊然立詔宗正卿李詮弔祭冊立伊然為立碑廟無幾伊然病卒又立其弟為登利可汗按張九齡集敕登利可汗書云今又遣從叔金吾大將軍佺弔祭又云建碑立廟貽範紀功然則告喪時登利已立矣實録銓亦作佺】庚戌來告喪 禁京城匄者置病坊以廩之【時病坊分置於諸寺以悲田養病本於釋教也匄古太翻】二十三年春正月契丹知兵馬中郎李過折來獻捷制以過折為北平王檢校松漠州都督 【考異曰實録云同幽州節度副大使舊傳云授特進檢校松漢州都督按過折雖有功唐未必肯使為幽州節度使今從舊傳】乙亥上耕籍田九推乃止【杜佑曰是年親耕有司進儀注天子三推公卿九推庶人終畝帝欲重耕籍遂進耕五十餘步盡壠乃止推吐雷翻】公卿以下皆終畝赦天下都城酺三日【都城謂東都城酺音蒲】上御五鳳樓酺宴觀者諠隘樂不得奏金吾白梃如雨不能遏【隘烏懈翻梃待鼎翻】上患之高力士奏河南丞嚴安之為理嚴【唐赤縣丞從七品理治也唐諱治改曰理】為人所畏請使止之上從之安之至以手板繞場畫地曰犯此者死于是盡三日人指其畫以相戒無敢犯者時命三百里内刺史縣令各帥所部音樂集于樓下各較勝負【帥讀曰率】懷州刺史以車載樂工數百皆衣文繡服箱之牛皆為虎豹犀象之狀【詩大東曰晥彼牽牛不以服箱注云服牝服也箱大車之箱也疏云兩較之間謂之箱甫田云乃求萬斯箱書傳云長幾充箱是車内容物之處丘氏曰服箱猶言駕車也衣於既翻】魯山令元德秀惟遣樂工數人連袂歌于蒍【魯山古魯縣夏孔甲時豢龍氏劉累所遷之地漢為魯陽縣屬南陽郡後魏置魯陽郡隋復為魯縣屬汝州唐為魯山縣以縣有魯山故名于蒍者德秀所為歌也蒍韋委翻 考異曰明皇雜録作于蒍新傳作于蒍于未詳其義今從雜録】上曰懷州之人其塗炭乎立以刺史為散官【散官無職事散蘇旱翻】德秀性介潔質樸士大夫皆服其高 上美張守珪之功欲以為相張九齡諫曰宰相者代天理物非賞功之官也上曰假以其名而不使任其職可乎對曰不可惟名與器不可以假人君之所司也【左傳記孔子之言】且守珪纔破契丹陛下即以為宰相若盡滅奚厥【奚厥謂奚與突厥厥九勿翻】將以何官賞之上乃止二月守珪詣東都獻捷拜右羽林大將軍兼御史大夫賜二子官賞賚甚厚【賚來代翻】初殿中侍御史楊汪既殺張審素【事見上卷十九年】更名萬頃【更工衡翻】審素二子瑝琇皆幼【瑝戶肓翻又音皇琇音秀】坐流嶺表尋逃歸謀伺便復讐【伺相吏翻】三月丁卯手殺萬頃於都城擊表於斧言父寃狀欲之江外殺與萬頃同謀陷其父者至汜水為有司所得【汜音祀】議者多言二子父死非罪穉年孝烈能復父讐宜加矜宥【穉直利翻】張九齡亦欲活之裴耀卿李林甫以為如此壞國法【壞音怪】上亦以為然謂九齡曰孝子之情義不顧死然殺人而赦之此塗不可啟也乃下敕曰國家設法期於止殺各伸為子之志誰非徇孝之人展轉相讐何有限極咎繇作士【咎與臯同古勞翻繇與陶同餘招翻】法在必行曾參殺人亦不可恕宜付河南府杖殺士民皆憐之為作哀誄牓於衢路【為于偽翻誄魯水翻】市人歛錢葬之於北邙恐萬頃家發之仍為疑冢數處【多作冢以疑之使莫知其所葬之的處】 唐初公主實封止三百戶中宗時太平公主至五千戶率以七丁為限開元以來皇妹止千戶皇女又半之皆以三丁為限駙馬皆除三品員外官而不任以職事公主邑入至少至不能具車服【少詩沼翻】左右或言其太薄上曰百姓租賦非我所有戰士出死力賞不過束帛女子何功而享多戶邪且欲使之知儉嗇耳秋七月咸宜公主將下嫁【咸宜公主下嫁楊洄】始加實封至千戶公主武惠妃之女也于是諸公主皆加至千戶 冬十月戊申突騎施寇北庭及安西撥换城【騎奇寄翻】 閏月壬午朔日有食之 【考異曰舊紀作十一月壬申朔按長歷十一月壬子朔今從實録唐歷】 十二月乙亥冊故蜀州司戶楊玄琰女為夀王妃【為帝納妃於後宮以亂國張本 考異曰實録載冊文云玄璬長女按陳鴻長恨歌傳云詔高力士潜搜外宫得楊玄琰女於夀邸舊貴妃傳云玄琰女早孤養於叔父玄璬又云或奏玄琰女容色冠代宜蒙召見時妃衣道士服號太真新傳云始為夀王妃云云遂召内禁中即為自出妃意者匄籍女官號太真更為夀王娶韋昭訓女而太真得幸舊史蓋諱之耳】玄琰汪之曾孫也【楊汪見一百八十三卷隋煬帝大業十二年】 是歲契丹王過折為其臣湼禮所殺【湼奴結翻 考異曰舊傳過折為可突干餘黨泥裏所殺不云朝廷如何處置泥裏今据張九齡集有上賜契丹都督湼禮敕又有賜張守珪敕云湼禮自擅難以義責而未有名位恐其不安卿可宣示朝旨使知無它也蓋泥裏即湼禮也】并其諸子一子刺乾奔安東得免【刺盧逹翻乾音干開元二年移安東都護府於平州】湼禮上言過折用刑殘虐衆情不安故殺之上赦其罪因以湼禮為松漠都督且賜書責之曰卿之蕃法多無義於君長【長知兩翻】自昔如此朕亦知之然過折是卿之王有惡輒殺之為此王者不亦難乎但恐卿為王後人亦爾常不自保誰願作王亦應防慮後事豈得取快目前突厥尋引兵東侵奚契丹湼禮與奚王李歸國擊破之二十四年春正月庚寅敕天下逃戶聽盡今年内自首有舊產者令還本貫無者别俟進止踰限不首【首式又翻】當命專使搜求散配諸軍【使疏吏翻】 北庭都護蓋嘉運擊突騎施大破之【蓋古盍翻姓也】 二月甲寅宴新除縣令於朝堂上作令長新戒一篇賜天下縣令【朝直遥翻長知兩翻】 庚午更皇子名【更工衡翻 考異曰舊紀唐歷二十三年七月景子皇太子諸王皆改名今從實録】鴻曰瑛潭曰琮浚曰璵洽曰琰㳙曰瑶滉曰琬涺曰琚濰曰璲【璲音遂】澐曰璬【璬公了翻】澤曰璘清曰瑁【瑁音冒】洄曰玢【玢方貧翻】沐曰琦溢曰環沔曰理泚曰玼漼曰珪澄曰珙潓曰瑱【瑱他甸翻】漎曰璿【璿從宣翻】滔曰璥【璥居影翻】 舊制考功員外郎掌試貢舉人有進士李權陵侮員外李昂議者以員外郎位卑不能服衆三月壬辰敕自今委禮部侍郎試貢舉人 張守珪使平盧討擊使左騎衛將軍安禄山討奚契丹叛者【擊使疏吏翻驍堅堯翻下同】禄山恃勇輕進為虜所敗夏四月辛亥守珪奏請斬之禄山臨刑呼曰【敗蒲邁翻呼火故翻】大夫不欲滅奚契丹邪奈何殺禄山守珪亦惜其驍勇乃更執送京師張九齡批曰昔穰苴誅莊賈【史記齊景公使司馬穰苴為將穰苴曰願得君之寵臣以監軍景公使莊賈往賈素驕貴穰苴與之約日中會于軍門夕時乃至穰苴以賈後期斬之以令三軍批匹迷翻判也今人謂之批判】孫武斬宫嬪【孫武以兵法見吴王闔廬吴王曰可以勒兵小試於婦人乎曰可於是出宮中美女百八十人分為二隊以王寵姬二人名為隊長皆令持戟約束既布三令五申於是皷之右婦人大笑孫子曰約束不行申令不熟將之罪也復三令五申而皷之左婦人復大笑孫子斬左右隊長以狥用其次為隊長而復皷婦人左右前後跪起皆中䋲墨規矩於是吴王知孫子能用兵以為將】守珪軍令若行禄山不宜免死上惜其才敕令免官以白衣將領【將即亮翻】九齡固爭曰禄山失律喪師於法不可不誅且臣觀其貌有反相不殺必有後患【喪息浪翻相息亮翻】上曰卿勿以王夷甫識石勒枉害忠良【晉石勒年十四隨邑人行販洛陽倚嘯上東門王衍見而異之顧謂左右曰向者胡雛吾觀其聲視有奇志恐將為天下之患馳遣收之會勒已去】竟赦之 【考異曰玄宗實録四月辛亥張守珪奏禄山統戎失律挫敗軍威請依軍法决斬許之禄山臨刑抗聲言曰兩蕃未和忍殺壮士豈為大夫謀也守珪以禄山嘗捷於擒生聞其言遂捨之以聞肅宗實録云禄山為互市牙郎盗羊事發守珪怒追捕至欲擊殺之禄山大呼曰大夫不欲滅奚契丹兩蕃邪而殺壮士守珪奇其貌壮其言遂釋之姚汝能作禄山事迹其盗羊事與肅宗實録同又云二十一年守珪令禄山奏事中書令張九齡見之謂裴光庭曰亂幽州者此胡也又云二十四年禄山為平盧將討奚契丹失利守珪奏請斬之九齡批曰穰苴出征必誅莊賈孫武行令亦斬宫嬪守珪軍令若行禄山不宜免死玄宗惜其勇鋭但令免官白衣展効九齡奏請誅之玄宗曰卿豈以王夷甫識石勒使臆斷禄山難制邪竟不誅之孫樵作西齋録其序曰張守珪以安禄山叛者何貸刑咈教稔禍階也禄山乃張守珪部將嘗犯令張曲江令守珪斬之不從果使亂天下故書曰張守珪以安禄山叛舊張九齡傳云張守珪以禆將安禄山討奚契丹敗衂執送京師請行朝典九齡奏劾曰穰苴出軍必誅莊賈孫武教戰亦斬宫嬪守珪軍令必行禄山不宜免死上特捨之九齡奏曰禄山狼子野心面有逆相臣請因罪戮之冀絶後患上曰卿勿以王夷甫知石勒故事誤害忠良遂放歸蕃新傳語裴光庭事如事迹執送京師事如舊傳舊禄山傳盗羊事如事迹而無失利請斬事新傳亦然舊傳仍云二十年守珪為幽州節度使禄山盗羊事覺按裴光庭二十一年卒是年冬九齡乃為相云與光庭語誤也孫樵云曲江令守珪斬之尤為失實實録二十一年守珪猶在隴右與吐蕃立分界碑未至幽州舊傳云二十年為節度亦誤也按禄山若始為互市牙郎守珪安能知其終亂天下釋而不殺孫樵豈得以叛罪加之邪若如舊九齡傳守珪執送京師玄宗自赦之則守珪何罪而時人咎之也若謂盗羊喪師兩次當死則禄山豈祇用辭而得免兩死邪若如玄宗實録守珪奏請行法得報聽許感其一言輒舍之則守珪必不敢輕易反覆如此且九齡何從而得見其面而云面有逆相邪若云守珪未嘗奏請行灋則張九齡集有賜守珪敕云禄山等輕我兵威曾不審料致令損失宜其就誅卿既行之軍灋合爾又賜平盧將士敕云安禄山之誅緣輕敵太過勿因此畏懦致失後圖是當時曾許之行誅矣若云守珪自捨之非玄宗意則又賜守珪敕云禄山勇而無謀遂至失利衣甲資盗挫我軍威論其輕敵合加重罪然初聞勇鬭亦有誅殺又寇戎未滅軍令從權故不以一敗棄之將欲收其後效也不行薄責又無所懲宜且停官令白衣將領卿更審量本狀亦任隨事處之今以諸書參考蓋禄山失律守珪奏請行灋故前敕云卿既行之軍灋合爾又云禄山之誅緣輕敵太過似謂守珪已誅之矣既而守珪感其言惜其驍勇欲殺則不忍欲捨則已先奏聞且恐不能厭服將士之心或者報未到故執送京師使上自裁之冀上見其材力而赦之亦猶陳平執樊噲衛青囚蘇建耳上因是欲赦之而九齡執奏云守珪軍令若行禄山不宜免死是并劾守珪不斷於閫外乃更執以諉上之辭也九齡因此見之而云面有逆相上終欲赦之故九齡不得已草敕云卿更審量本狀隨事處之守珪得此敕即捨之以聞如此則與玄宗實録相應而於人情差似相近】安禄山者本營州雜胡初名阿犖山其母巫也【新書曰禄山本姓康其母居突厥中禱于軋犖山虜所謂戰鬬神者而生禄山故以為字從母冒姓安氏阿烏葛翻犖呂角翻】父死母擕之再適突厥安延偃會其部落破散與延偃兄子思順俱逃來故冒姓安氏名禄山又有史窣干者【窣蘇骨翻】與禄山同里閈先後一日生 【考異曰舊傳云思明除日生禄山元日生按禄山事迹天寶十載正月二十日上及貴妃為禄山作生日今不取】及長相親愛皆為互市牙郎以驍勇聞【牙郎駔儈也南北物價定於其口而後相與貿易】張守珪以禄山為捉生將禄山每與數騎出輒擒契丹數十人而返狡猾善揣人情【將即亮翻騎奇寄翻揣初委翻】守珪愛之養以為子窣干嘗負官債亡入奚中為奚遊奕所得欲殺之窣干紿曰我唐之和親使也【紿湯亥翻使疏吏翻】汝殺我禍且及汝國遊奕信之送詣牙帳窣干見奚王長揖不拜奚王雖怒而畏唐不敢殺以客禮館之【館古玩翻】使百人隨窣干入朝窣干謂奚王曰王遣人雖多觀其才皆不足以見天子聞王有良將瑣高者何不使之入朝【瑣高者蓋奚中酋豪之號非人名也前已有李詩瑣高將即亮翻朝直遥翻】奚王即命瑣高與牙下三百人隨窣干入朝窣干將至平盧先使人謂軍使裴休子曰奚使瑣高與精鋭俱來聲云入朝實欲襲軍城宜謹為之備先事圖之休子乃具軍容出迎至館悉阬殺其從兵執瑣高送幽州【使疏吏翻先悉薦翻從才用翻】張守珪以窣干為有功奏為果毅累遷將軍後入奏事上與語悦之賜名思明【安史事始此】 故連州司馬武攸望之子温眘坐交通權貴杖死【帝平韋氏武攸望貶死眘時刃翻】乙丑朔方河東節度使信安王禕貶衢州刺史廣武王承宏貶房州别駕涇州刺史薛自勸貶澧州别駕皆坐與温眘交遊故也承宏守禮之子也【王守禮章懷太子賢之子】辛未蒲州刺史王琚貶通州刺史坐與禕交書也 五月醴泉妖人劉志誠作亂驅掠路人將趣咸陽【妖於喬翻趣七喻翻】村民走告縣官焚橋斷路以拒之【斷音短】其衆遂潰數日悉擒斬之 六月初分月給百官俸錢 初上因籍田赦命有司議增宗廟籩豆之薦及服紀未通者太常卿韋縚奏請宗廟每坐籩豆十二【縚土刀翻坐徂臥翻】兵部侍郎張均職方郎中韋述議曰聖人知孝子之情深而物類之無限故為之節制人之嗜好本無憑凖宴私之饌與時遷移故聖人一切同歸於古屈到嗜芰屈建不以薦以為不以私欲干國之典【國語楚屈到嗜芰有疾召其宗老而屬之曰祭我必以芰及祥宗老將薦芰屈建命去之曰國君有牛享大夫有羊饋士有豚犬之奠庶人有魚炙之薦籩豆脯醢則上下共之不羞珍異不陳庶侈不以其私欲干國之典遂不用芰奇寄翻菱一名芰說文曰楚謂之芰秦謂之薢茩今俗但言蓤芰武陵記四角三角曰芰兩角曰蓤好呼到翻去羌呂翻】今欲取甘旨肥濃皆充祭用苟踰舊制其何限焉書曰黍稷非馨明德惟馨【書成王命君陳之言】若以今之珍饌平生所習求神無方何必泥古則簠簋可去而盤盂盃按當在御矣韶濩可息而箜箏笛當在奏矣【舜樂曰韶湯樂曰濩箜漢武帝使樂人侯調所作或云侯輝所作今按其形似瑟而小七絃用撥彈之如琵琶舊唐書曰箜胡樂也漢靈帝好之體曲而長二十三絃豎抱于懷用兩手齊奏俗名擘箜鳳首箜有項加軫七絃鄭善子作開元中進形如阮咸其下缺小而身大旁有小缺取其身便也一曰箜乃鄭衛之音權輿以其亡國之聲故號空國之侯亦曰坎侯風俗通云漢武帝時丘仲作笛按周禮笙師掌敎箎篴又云起於羌人後漢馬融所賦横笛空洞無底剡其二孔五孔一出其背正似今之尺八李善為之注七孔今一尺四寸此乃今之横笛耳太常鼔吹部中所謂横吹非融所賦者融賦易京君明識音律故本四孔加以一君明所加孔後出是謂商聲五音畢沈約宋書亦云京房備其五音周禮笙師注杜子春云篴乃今時所吹五孔篴以融約所記論之則古篴不應有五孔子春之說亦未為然今三禮圖畫篴亦横設而有五孔不知出何典據篴與笛同簠音甫簋居洧翻】既非正物後嗣何觀夫神以精明臨人者也不求豐大苟失於禮雖多何為豈可廢棄禮經以從流俗且君子愛人以禮不求苟合况在宗廟敢忘舊章太子賓客崔沔議曰祭祀之興肇於太古茹毛飲血則有毛血之薦未有麴糵則有玄酒之奠【司烜氏以鑒取明水於月為玄酒糵魚列翻】施及後王禮物漸備【施弋智翻】然以神道致敬不敢廢也籩豆簠簋樽罍之實皆周人之時饌也其用通於宴饗賓客而周公制禮與毛血玄酒同薦鬼神國家由禮立訓因時制範清廟時饗禮饌畢陳用周制也【如簠簋籩豆鉶羮之類饌雛戀翻又雛晥翻】園陵上食時膳具設遵漢法也【如叔孫通請薦含桃之類上時掌翻】職貢來祭致遠物也有新必薦順時令也苑囿之内躬稼所收蒐狩之時親發所中莫不薦而後食盡誠敬也若此至矣復何加焉【中竹仲翻復扶又翻】但當申敕有司無或簡怠則鮮美肥濃盡在是矣不必加籩豆之數也【自此以上諸人之議皆因舊禮而申之】上固欲量加品味【量音良】縚又奏每室加籩豆各六四時各實以新果珍羞從之縚又奏喪服舅緦麻三月從母外祖父母皆小功五月外祖至尊同於從母之服姨舅一等服則輕重有殊【姨即從母也從才用翻】堂姨舅親即未踈恩絶不相為服舅母來承外族不如同㸑之禮竊以古意猶有所未暢者也請加外祖父母為大功九月姨舅皆小功五月堂舅堂姨舅母並加至袒免【五服止於緦麻此外有袒免之服袒者偏脱衣䄂而露其肩免者以布廣一寸從項申而前交於額上又却向後繞於䯻袒音但免音問】崔沔議曰正家之道不可以貳總一定義理歸本宗是以内有齊斬【齊音咨】外皆緦麻尊名所加不過一等此先王不易之道也願守八年明旨一依古禮【崔沔所謂詔旨見二百十二卷七年曰八年者通帝即位先天之年數之也】以為萬代成法韋述議曰喪服傳曰禽獸知母而不知父野人曰父母何等焉都邑之士則知尊禰矣【傳直戀翻禰奴禮翻】大夫及學士則知尊祖矣聖人究天道而厚於祖禰繫族姓而親其子孫母黨比於本族不可同貫明矣今若外祖與舅加服一等堂舅及姨列於服紀則中外之制相去幾何廢禮徇情所務者末古之制作者知人情之易揺恐失禮之將漸别其同異輕重相懸【易以䜴翻别彼列翻】欲使後來之人永不相雜微旨斯在豈徒然哉苟可加也亦可減也往聖可得而非則禮經可得而隳矣先王之制謂之彛倫【彛常也倫道理次叙】奉以周旋猶恐失墜一紊其叙庸可止乎請依議禮喪服為定【紊音問】禮部員外郎楊仲昌議曰【唐禮部郎掌五禮舉其儀制而辯其名數】鄭文貞公魏徵始加舅服至小功五月雖文貞賢也而周孔聖也以賢改聖後學何從竊恐内外乖序親疎奪倫情之所沿何所不至昔子路有姊之喪而不除孔子曰先王制禮行道之人皆不忍也子路除之【見記檀弓】此則聖人援事抑情之明例也記曰毋輕議禮【禮器之言】明其蟠於天地並彼日月賢者由之安敢損益也敕姨舅既服小功舅母不得全降宜服緦麻堂姨舅宜服袒免均說之子也【說讀曰悦】 秋八月壬子千秋節羣臣皆獻寶鏡張九齡以為以鏡自照見形容以人自照見吉凶乃述前世興廢之源為書五卷謂之千秋金鏡録上之【上時掌翻】上賜書褒美 甲寅突騎施遣其大臣胡禄逹干來請降許之【騎奇寄翻降戶江翻】 御史大夫李適之承乾之孫也以才幹得幸於上數為承乾論辯甲戍追贈承乾恒山愍王【承乾廢見一百九十七卷太宗貞觀十七年數所角翻為于偽翻恒戶登翻】 乙亥汴哀王璥薨冬十月戊申車駕發東都先是敕以來年二月二日行幸西京【先悉薦翻】會宫中有恠明日上召宰相即議西還裴耀卿張九齡曰今農收未畢請俟仲冬李林甫潜知上指二相林甫獨留言於上曰長安洛陽陛下東西宫耳往來行幸何更擇時借使妨於農收但應蠲所過租税而已【蠲圭玄翻】臣請宣示百司即日西行上悦從之過陜州以刺史盧奐有善政題贊於其聽事而去【稱人之美曰費陜失冉翻聽讀曰廳】奐懷慎之子也丁卯至西京 朔方節度使牛仙客前在河西能節用度勤職業倉庫充實器械精利上聞而嘉之欲加尚書張九齡曰不可尚書古之納言唐興以來惟舊相及敭歷中外有德望者乃為之仙客本河湟使典【事見上卷十五年相息亮翻使疏吏翻】今驟居清要恐羞朝廷上曰然則但加實封可乎對曰不可封爵所以勸有功也邊將實倉庫修器械乃常務耳【將即亮翻】不足為功陛下賞其勤賜之金帛可也裂土封之恐非其宜上默然李林甫言於上曰仙客宰相才也何有於尚書九齡書生不逹大體 【考異曰舊林甫傳曰林甫以九齡言告仙客仙客翌日見上泣讓官爵按時不聞仙客在京今從唐歷】上悦明日復以仙客實封為言【復扶又翻】九齡固執如初上怒變色曰事皆由卿邪九齡頓首謝曰陛下不知臣愚使待罪宰相事有未允臣不敢不盡言上曰卿嫌仙客寒微如卿有何閥閲九齡曰臣嶺海孤賤【九齡韶州人】不如仙客生於中華【牛仙客涇州人】然臣出入臺閣典司誥命有年矣【九齡歷司勲員外郎中書舍人】仙客邊隅小吏目不知書若大任之恐不愜衆望【愜苦叶翻】林甫而言曰苟有才識何必辭學天子用人有何不可十一月戊戌賜仙客爵隴西縣公食實封三百戶 初上欲以李林甫為相問於中書令張九齡九齡對曰宰相繫國安危陛下相林甫臣恐異日為廟社之憂上不從時九齡方以文學為上所重林甫雖恨猶曲意事之侍中裴耀卿與九齡善林甫并疾之是時上在位歲久漸肆奢欲怠於政事而九齡遇事無細大皆力爭林甫巧伺上意日思所以中傷之【伺相吏翻中竹仲翻】上之為臨淄王也趙麗妃皇甫德儀劉才人皆有寵【帝置六儀德儀其一也】麗妃生太子瑛德儀生鄂王瑶才人生光王琚及即位幸武惠妃麗妃等愛皆弛惠妃生夀王瑁寵冠諸子太子與瑶琚會於内第【時太子諸王皆居禁中冠古玩翻】各以母失職有怨望語駙馬都尉楊洄尚咸宜公主常伺三子過失以告惠妃【咸宜公主武惠妃之女故楊洄黨於惠妃】惠妃泣訴於上曰太子隂結黨與將害妾母子亦指斥至尊上大怒以語宰相欲皆廢之【語牛倨翻】九齡曰陛下踐阼垂三十年太子諸王不離深宫【離力智翻】日受聖訓天下之人皆慶陛下享國久長子孫蕃昌【蕃音煩】今三子皆已成人不聞大過陛下奈何一旦以無根之語喜怒之際盡廢之乎且太子天下本不可輕揺昔晉獻公聽驪姬之讒殺申生三世大亂【左傳晉獻公殺其世子申生立驪姬之子里克殺之公子夷吾重耳及子圉爭國三世大亂】漢武帝信江充之誣罪戾太子京城流血【事見漢紀】晉惠帝用賈后之讚廢愍懷太子中原塗炭【事見晉紀】隋文帝納獨孤后之言黜太子勇立煬帝遂失天下【事見隋紀】由此觀之不可不慎陛下必欲為此臣不敢奉詔上不悦林甫初無所言退而私謂宦官之貴幸者曰此主上家事何必問外人上猶豫未决惠妃密使官奴牛貴兒謂九齡曰有廢必有興公為之援宰相可長處九齡叱之以其語白上上為之動色【處昌呂翻上為于偽翻】故訖九齡罷相太子得無動林甫日夜短九齡於上上浸疎之【考異曰明皇雜録云林甫請見屢陳仙客實封九齡頗懷誹謗于時方秋上命高力士以白羽扇賜之九齡惶恐作賦以獻新傳亦云然按實録仙客加實封在十月而九齡集白羽扇賦序云開元二十四年夏盛暑奉敕使大將軍高力士賜宰相白羽扇九齡預焉竊有所感立獻賦云云敕報曰朕近賜羽扇聊以滌暑佳彼勁翮方資用利與夫棄捐篋笥義不同也然則上以盛夏遍賜宰臣扇非以秋日獨賜九齡但九齡因此獻賦自寄意耳】林甫引蕭炅為戶部侍郎炅素不學【炅古迴翻】嘗對中書侍郎嚴挺之讀伏臘為伏獵挺之言於九齡曰省中豈容有伏獵侍郎由是出炅為岐州刺史故林甫怨挺之九齡與挺之善欲引以為相嘗謂之曰李尚書方承恩【李林甫時以禮部尚書相】足下宜一造門與之欵䁥挺之素負氣薄林甫為人竟不之詣林甫恨之益深挺之先娶妻出之更嫁蔚州刺史王元琰元琰坐贓罪下三司按鞫挺之為之營解林甫因左右使於禁中白上上謂宰相曰挺之為罪人請屬所由【造七到翻更工衡翻蔚紆勿翻下遐嫁翻為于偽翻屬之欲翻】九齡曰此乃挺之出妻不宜有情上曰雖離乃復有私【復扶又翻下無復同】於是上積前事以耀卿九齡為阿黨壬寅以耀卿為左丞相九齡為右丞相並罷政事以林甫兼中書令仙客為工部尚書同中書門下三品領朔方節度如故 【考異曰唐歷曰宰相遥領節度自仙客始按蕭嵩已遥領河西非始此】嚴挺之貶洺州刺史【舊志洺州京師東北一千五百八十五里】王元琰流嶺南上即位以來所用之相姚崇尚通宋璟尚法張嘉貞尚吏張說尚文李元紘杜暹尚儉韓休張九齡尚直各其所長也九齡既得辠自是朝廷之士皆容身保位無復直言李林甫欲蔽塞人主視聽自專大權明召諸諫官謂曰今明主在上羣臣將順之不暇烏用多言諸君不見立仗馬乎食三品料一鳴輒斥去悔之何及【塞悉則翻去羌呂翻唐舊儀每日尚乘以廐馬八匹分為左右廂立於正殿側宫門外俟仗下即散若大陳設則馬在樂懸之北與大象相次進馬二人戎服執鞭侍立于馬之左隨馬進退】補闕杜璡嘗上書言事【璡資辛翻上時掌翻 考異曰唐歷作杜珽今從新書】明日黜為下邽令【唐制上縣令從六品上補闕從七品上以此言之則非黜也蓋唐人重内官而品之高下不論也况遺補供奉言地居清要乎】自是諫爭路絶矣牛仙客既為林甫所引專給唯諾而已【爭讀曰諍唯于癸翻】然二人皆謹守格式百官遷除各有常度雖奇才異行不免終老常調其以巧謟邪險自進者則超騰不次自有它蹊矣林甫城府深密人莫窺其際好以甘言㗖人而隂中傷之不露辭色凡為上所厚者始則親結之及位埶稍逼輒以計去之【行下孟翻調徒弔翻中竹仲翻去羌呂翻】雖老奸巨猾無能逃於其術者【如韋堅楊慎矜王鉷之類是也】<br />
<br />
  二十五年春正月初置玄學博士【崇玄學習老子莊子文子列子亦曰道舉】每歲依明經舉 二月敕曰進士以聲韵為學多昧古今明經以帖誦為功罕窮旨趣自今明經問大義十條對時務策三首進士試大經十帖 戊辰新羅王興光卒【卒子恤翻】子承慶襲位 乙酉幽州節度使張守珪破契丹於捺禄山【使疏吏翻契欺訖翻又音喫捺奴軋翻】 己亥河西節度使崔希逸襲吐蕃破之於青海西初希逸遣使謂吐蕃乞力徐曰兩國通好今為一家何必更置兵守捉妨人耕牧請皆罷之乞力徐曰常侍忠厚【吐從暾入聲崔希逸蓋帶散騎常侍鎮河西故稱之使疏吏翻好呼到翻捉反角翻】言必不欺然朝廷未必專以邊事相委萬一有姦人交闘其間掩吾不備悔之何及希逸固請乃刑白狗為盟各去守備于是吐蕃畜牧被野【朝直遙翻去羌呂翻被皮義翻畜許救翻】時吐蕃西擊勃律勃律來告急上命吐蕃罷兵吐蕃不奉詔遂破勃律上甚怒會希逸傔人孫誨入奏事【傔苦念翻傔從也唐制凡諸軍鎮大使副使以下皆有傔人别奏以為之使大使傔二十五人别奏十人副使傔二十人别奏八人】自欲求功奏稱吐蕃無備請掩擊必大獲上命内給事趙惠琮與誨偕往審察事宜【唐内侍省有内給事十人從五品下掌承旨勞問分判省事凡元日冬至百官賀皇后則出入宣傳宫人衣服費用則具品秩計其多少春秋宣送中書】惠琮等至則矯詔令希逸襲之希逸不得已發兵自凉州南入吐蕃二千餘里至青海西與吐蕃戰大破之斬首二千餘級乞力徐脱身走惠琮誨皆受厚賞自是吐蕃復絶朝貢【復扶又翻下而復同朝直遥翻】 夏四月辛酉監察御史周子諒彈牛仙客非才引䜟書為證【薛居正五代史曰天后朝有䜟辭云首尾三鱗六十年兩角犢子自狂顛龍蛇相鬬血成川當時好事者解云兩角犢子牛也必有牛姓干唐祚監古衘翻彈徒丹翻】上怒命左右㩧于殿庭【㩧蒲角翻擊也又匹角翻】絶而復蘇仍杖之朝堂流瀼州至藍田而【瀼如羊翻又而章翻藍田縣漢晉屬京兆後魏真君七年併入霸城太和十一年復後周置藍田郡隋廢郡為縣屬京兆府九域志在府東南七十里 考異曰舊紀云子諒以妄陳休咎於朝堂决殺實録此月則曰子諒彈奏仙客非才引妖䜟為證上怒召入禁中責之左右拉者數四氣絶而蘇及仙客傳則云子諒竊言於御史大夫李適之曰牛仙客不才濫登相位大夫國之懿親豈得坐觀其事適之遽奏子諒之言上大怒廷詰子諒子諒辭窮於朝堂决杖配流瀼州行至藍田死舊仙客傳亦然今從此月實録及舊紀柳宗元周君墓碣云有唐貞臣汝南周氏諱某字某又曰在天寶年有以諂諛至相位賢臣放退公為御史抗言以白其事得死于墀下宗元集此碣雖無名字然其事則子諒也云在天寶年誤矣】李林甫言子諒張九齡所薦也甲子貶九齡荆州長史 楊洄又奏太子瑛鄂王瑶光王琚云與太子妃兄駙馬薛鏽潜搆異謀【鏽息六翻又息救翻 考異曰新傳曰二十五年洄復搆瑛瑶琚與妃之兄薛鏽異謀惠妃使人詭召太子二王曰宫中有賊請介以入太子從之妃白帝曰太子二王謀反甲而來帝使中人視之如言遽召宰相林甫議答曰陛下家事非臣所宜豫帝意决乃廢瑛等按瑛等與惠妃相猜忌已久雖承妃言豈肯遽被甲入宫又按廢太子制書云䧟元良於不友誤二子於不義不言被甲入宫也盖洄譛瑛等云欲害夀王瑁耳今從舊傳但云潜搆異謀】上召宰相謀之李林甫對曰此陛下家事非臣等所宜豫上意乃决乙丑使宦者宣制於宫中廢瑛瑶琚為庶人【於宫中廢之用李林甫家事之言也 考異曰獨孤及作裴稹行狀云公為起居郎三庶人以罪廢夀王以母寵子愛議者頗有奪宗之嫌道路憫默朝野疑愳公乃從容請間慷慨獻諫上述新城之敫鑒下陳戾園之元龜謂興亡之由在廢立之地天子感悟改容以謝因詔以給事中授公公曰陛下絶招諫之路為日固久今臣一言而荷殊寵則言者衆矣何以錫之上善其敏而多其讓乃止不拜尋除尚書禮部員外郎按稹光庭之子當是時周子諒杖死張九齡遠貶稹若敢為太子直寃則聲振宇宙豈得湮沒無聞而諸書皆不言此事蓋出於及之虚美耳】流鏽於瀼州瑛瑶琚尋賜死城東驛鏽賜死於藍田瑶琚皆好學有才識死不以罪人皆惜之【好呼到翻】丙寅瑛舅家趙氏妃家薛氏瑶舅家皇甫氏坐流貶者數十人惟瑶妃家韋氏以妃賢得免 五月夷州刺史楊濬坐贓當死上命杖之六十流古州【夷州漢牂柯地其後為徼外隋開置綏陽縣屬明陽郡武德四年置夷州於思州寧夷縣明陽屬焉而綏陽屬義州貞觀元年廢夷州而明陽寧夷屬務州四年復置夷州於黔州都上縣尋又自都上移於綏陽貞觀十二年李弘節開夷獠置古州屬容州都督府】左丞相裴耀卿上疏以為决杖贖死恩則甚優解體受笞事頗為辱止可施之徒隸不當及於士人上從之 癸未敕以方隅底定令中書門下與諸道節度使量軍鎮閑劇利害審計兵防定額於諸色征人及客戶中召募丁壯長充邊軍增給田宅務加優恤 辛丑上命有司選宗子有才者授以臺省及法官京縣官敇曰違道慢常義無私於主法修身効節恩豈薄於它人期於帥先勵我風俗【帥讀曰率】 秋七月己卯大理少卿徐嶠 【考異曰舊紀作徐岵今從刑法志通典】奏今歲天下斷死刑五十八大理獄院由來相傳殺氣太盛鳥雀不栖今有鵲巢其樹于是百官以幾致刑措上表稱賀【斷丁亂翻幾居依翻上時掌翻】上歸功宰輔庚辰賜李林甫爵晉國公牛仙客國公 【考異曰實録七月戊寅有司奏囚減少上歸美宰臣制曰斷獄五十殆至無刑遂封二人又十月丙午上因聽政問京城囚徒有司奏有五十人怡然有喜色下制曰日者叢棘之地烏鵲來巢今結諸刑名纔逾五十其刑部侍郎鄭少微等各賜中上考二者未詳其為一事二事今從舊紀】上命李林甫牛仙客與法官刪修律令格式成九月壬申頒行之 先是西北邊數十州多宿重兵【先悉薦翻】地租營田皆不能贍【贍而艶翻】始用和糴之法有彭果者因牛仙客獻策請行糴法於關中戊子敕以歲稔穀賤傷農命增時價什二三和糴東西畿粟各數百萬斛【東畿都畿也西畿京畿也糴他歷翻】停今年江淮所運租自是關中蓄積羨溢【羨延面翻】車駕不復幸東都矣【復扶又翻】癸巳敕河南北租應輸含嘉太原倉者【據李泌傳太原倉在陜州西】皆留輸本州 太常博士王璵【璵音余 考異曰舊傳不言璵鄉里世系新傳云方慶六世孫又新舊傳皆云抗疏請置春壇因遷太常博士不知其本何官也新表王方慶五世孫璵事肅宗按方慶長安二年卒距此才三十六年不應已有五世六世孫能上疏恐璵偶與之同名實非也今不取】上疏請立青帝壇以迎春從之冬十月辛丑制自今立春親迎春於東郊時上頗好祀神鬼【好呼到翻】故璵專習祠祭之禮以干時上悦之以為侍御史領祠祭使璵祈禱或焚紙錢類巫覡【漢以來喪葬有瘞錢後世俚俗稍以紙寓錢為鬼事使疏吏翻覡刑狄翻】習禮者羞之 壬申上幸驪山温泉乙酉還宫 己丑開府儀同三司廣平文貞公宋璟薨 十二月丙午惠妃武氏薨贈諡貞順皇后 是歲命將作大匠康諐素之東都毁明堂【諐與愆同籕文也新書禮樂志作諐素】諐素上言毁之勞人請去上層【去羌呂翻】卑於舊九十五尺仍舊為乾元殿從之 初令租庸調租資課【調徒弔翻】皆以土物輸京都【西京東都租庸調高祖太宗之法租資課必開元以來之法】<br />
<br />
  二十六年春正月乙亥以牛仙客為侍中 丁丑上迎氣於滻水之東【滻音產】 制邊地長征兵召募向足自今鎮兵勿復遣【復扶又翻】在彼者縱還 今天下州縣里别置學 壬辰以李林甫領隴右節度副大使以鄯州都督杜希望知留後【鄯時戰翻又音善】二月乙卯以牛仙客兼河東節度副大使【牛仙客先已領朔方今兼河東】 己未葬貞順皇后于敬陵【武惠妃諡貞順皇后敬陵在京兆萬年縣東南四十里】 壬戌敇河曲六州胡坐康待賓散隸諸州者聽還故土於鹽夏之間置宥州以處之【徙六胡州見二百十二卷十年今併六胡州之地以為宥州舊志宥州去京師二千一百里夏戶雅翻處昌呂翻】 三月吐蕃寇河西節度使崔希逸擊破之鄯州都督知隴右留後杜希望攻吐蕃新城拔之以其地為威戎軍【鄯州星宿川西北三百五十里有威戎軍 考異曰舊傳作威武軍今從實録】置兵一千戌之 夏五月乙酉李林甫兼河西節度使【已領隴右又兼河西】丙申以崔希逸為河南尹希逸自念失信於吐蕃【以背乞力徐之盟也】内懷愧恨未幾而卒【幾居豈翻】 太子瑛既死李林甫數勸上立夀王瑁上以忠王璵年長且仁孝恭謹又好學意欲立之猶豫歲餘不决自念春秋浸高三子同日誅死繼嗣未定常忽忽不樂寢膳為之減高力士乘間請其故【數所角翻長知兩翻好呼到翻樂音洛為于偽翻間古莧翻】上曰汝我家老奴豈不能揣我意【揣初委翻】力士曰得非以郎君未定邪上曰然對曰大家何必如此虚勞聖心但推長而立【長知兩翻 考異曰統紀叙力士語云但從大枒注謂肅宗也大枒語不可曉今從新傳】誰敢復爭【復扶又翻】上曰汝言是也汝言是也由是遂定六月庚子立璵為太子 辛丑以岐州刺史蕭炅為河西節度使總留後事鄯州都督杜希望為隴右節度使太僕卿王昱為劒南節度使 【考異曰舊傳作王昊今從實録唐歷】分道經略吐蕃仍毁所立赤嶺碑【立碑見上卷二十一年】 突騎施可汗蘇禄素亷儉每攻戰所得輒與諸部分之不留私蓄由是衆樂為用【樂音洛】既尚唐公主又潜通突厥及吐蕃突厥吐蕃各以女妻之蘇禄以三國女為可敦又立數子為葉護用度浸廣由是攻戰所得不復更分晚年病風一手攣縮【妻七細翻復扶又翻下面復同攣閭緣翻】諸部離心酋長莫賀逹干都摩度兩部最彊【酋慈由翻長知兩翻 考異曰會要作莫賀咄逹干今從實録新傳作都摩支今從實録舊傳】其部落又分為黄姓黑姓互相乖阻【突騎施種人自謂娑葛後者為黄姓蘇禄部為黑姓】于是莫賀逹干勒兵夜襲蘇禄殺之都摩度初與莫賀逹干連謀既而復與之異立蘇禄之子骨啜為吐火仙可汗以收其餘衆與莫賀逹干相攻莫賀逹干遣使告磧西節度使蓋嘉運【磧七迹翻】上命嘉運招集突騎施拔汗那以西諸國吐火仙與都摩度據碎葉城黑姓可汗爾微特勒據怛邏斯城【碎葉川長千里西屬怛邏斯城其城初屬石國石常分兵鎮之蓋古盍翻騎奇寄翻可從刋入聲汗音寒怛當割翻邏郎佐翻 考異曰唐歷作怛邏斯今從實録】相與連兵以拒唐 太子將受冊命儀注有中嚴外辦及絳紗袍【唐制皇帝大祀致齋之日晝漏上水一刻侍中版奏請中嚴諸衛入陳于殿庭文武五品已上袴褶陪位諸侍從之官服其器服諸侍臣齋者結佩詣閤奉迎二刻侍中版奏外辦乘輿乃出朝會諸衛立仗百官就列已定侍中亦奏外辦不請中嚴皇帝將出駕發前七刻擊三皷爲一嚴前五刻擊二皷為再嚴侍中版奏請中嚴有司陳鹵簿前二刻擊三皷為三嚴諸衛以次入立于殿庭羣官立朝堂侍中中書令已下奉迎于西階侍中奉寶乘黄令進路於太極殿西階南向于牛將軍執長刀立路前北向黄門侍郎立侍臣之前□者二人既外辦太僕卿攝衣而升正立執轡乘輿出升路太后皇后亦有中嚴外辦皆尚儀版奏皇太子中嚴外辦左庶子版奏皇帝冠通天冠則服絳紗袍冬至受朝賀祭還燕羣臣養老之服也太子冠遠遊冠亦服絳紗袍謁廟還宫元日朔日入朝釋奠之服也】太子嫌與至尊同稱表請易之左丞相裴耀卿奏停中嚴改外辦曰外備改絳紗袍為朱明服秋七月己巳上御宣政殿冊太子 【考異曰元載肅宗實録云二十七年七月壬辰行典禮今從玄宗實録】故事太子乘輅至殿門至是太子不就輅自其宫步入是日赦天下己卯冊忠王妃韋氏為太子妃 杜希望將鄯州之衆奪吐蕃河橋築鹽泉城於河左吐蕃發兵三萬逆戰希望衆少不敵【將即亮翻又音如字鄯音善又時戰翻吐從暾入聲少始紹翻】將卒皆懼左威衛郎將王忠嗣帥所部先犯其陳所向闢易殺數百人虜陳亂【將即亮翻嗣祥吏翻帥讀曰率陳讀曰陣】希望縱兵乘之虜遂大敗置鎮西軍於鹽泉【鎮西軍在河州西百八十里】忠嗣以功遷左金吾將軍八月辛巳勃海王武藝卒子欽茂立 九月丙申朔<br />
<br />
  日有食之 初儀鳳中吐蕃陷安戎城而據之【初劒南度茂州之西築安戎城戌之以迮吐蕃南鄙生羌導吐蕃取之因守之遂并西洱河諸蠻東與松茂嶲接】其地險要唐屢攻之不克劒南節度使王昱築兩城於其側頓軍蒲婆嶺下【新舊作蓬婆嶺其地在雪山外杜甫詩所謂次取蓬婆雪外城是也】運資糧以逼之吐蕃大發兵救安戎城昱衆大敗死者數千人 【考異曰舊傳將士數萬人皆沒于賊今從實録】昱脱身走糧仗軍資皆棄之貶昱栝州刺史再貶高要尉而死 戊午冊南詔蒙歸義為雲南王【水經注雲南郡本雲山縣地也蜀劉氏建興二年置郡自唐戎州開邊縣而南七十里至曲州又二千五百里至雲南城】歸義之先本哀牢夷地【哀牢夷漢明帝之時内附】居姚州之西東南接交趾西北接吐蕃蠻語謂王曰詔先有六詔曰蒙舍曰蒙越曰越析曰浪穹曰様備曰越澹 【考異曰新書六詔曰蒙嶲越析浪穹邆睒施浪蒙舍今從竇滂雲南别録】兵刀相埒莫能相壹【埒力輟翻】歷代因之以分其勢蒙舍最在南故謂之南詔高宗時蒙舍細奴邏初入朝細奴邏生邏盛邏盛生盛邏皮盛邏皮生皮邏閤【朝直遥翻邏郎佐翻 考異曰新傳云蒙氏父子以名相屬細奴邏生邏盛炎邏盛炎生炎閤武后時邏盛炎身入朝妻方娠生盛邏皮喜曰我又有子雖死唐地足矣炎閤立死開元時弟盛邏皮立生皮邏閤授特進臺登郡王炎閤未有子時以閤羅鳳為嗣及生子還其宗而名承閤遂不改按邏盛炎之子盛邏皮豈得云以名相屬既有炎閤豈得云我又有子雖死唐地足矣今從舊南詔傳及楊國忠傳雲南别録又舊南詔傳閤皆作閣今從新傳】皮邏閤浸彊大而五詔微弱會有破渳河蠻之功【渳河即西洱河洱音乃吏翻】乃賂王昱求合六詔為一昱為之奏請【為于偽翻】朝廷許之仍賜名歸義於是以兵威脅服羣蠻不從者滅之遂擊破吐蕃徙居大和城其後卒為邊患【南詔事始此其先烏蠻别種夷語山陂陀為和故謂之大和城卒子恤翻】 冬十月戊寅上幸驪山温泉 壬辰上還宫 是歲於西京東都往來之路作行宫千餘間 分左右羽林置龍武軍以萬騎營隸焉【騎奇寄翻】 潤州刺史齊澣奏自瓜步濟江迂六十里請自京口埭下直濟江穿伊婁河二十五里即逹楊子縣立伊婁埭從之【埭音代按舊書本紀齊澣開伊婁河於揚州南瓜州浦則今之瓜州運河是也但楊子縣今為真州治所安能二十五里即逹楊子縣若自瓜洲逹楊子橋則二十五里而近今之楊子橋或者唐之楊子縣治所橋以此得名也】二十七年春正月壬寅命隴右節度大使榮王琬自至本道廵按處置諸軍【處昌呂翻】選募關内河東壯士三五萬人詣隴右防遏至秋末無寇聽還 羣臣請加尊號曰聖文二月己巳許之因赦天下免百姓今年田租 夏四月癸酉敕諸隂陽術數自非昏喪卜擇皆禁之 己丑以牛仙客為兵部尚書兼侍中李林甫為吏部尚書兼中書令總文武選事【蓋令牛仙客總武選李林甫總文選也選須絹翻】 六月癸酉以御史大夫李適之兼幽州節度使幽州將趙堪白真陁羅矯節度使張守珪之命使平盧軍使烏知義擊叛奚餘黨於横水之北【横水當作潢水新書作湟水舊書張守珪傳作潢水今從之潢水在遼國今臨潢府界志云自營州度松陘嶺北行四百里至潢水使疏吏翻將即亮翻】知義不從白真陁羅矯稱制指以迫之知義不得已出師與虜遇先勝後敗守珪隱其敗狀以克獲聞事頗泄上令内謁者監牛仙童往察之【内謁者監唐正六品下掌内宣傳及諸親命婦朝會所司籍其人數送内侍省】守珪重賂仙童歸罪於白真陁羅逼令自縊死【縊於計翻】仙童有寵於上衆宦官疾之共發其事上怒甲戌命楊思朂杖殺之思勗縳格杖之數百刳取其心割其肉㗖之【㗖徒濫翻又徒覽翻】守珪坐貶括州刺史太子太師蕭嵩嘗賂仙童以城南良田數頃李林甫發之嵩坐貶青州刺史 秋八月乙亥磧西節度使蓋嘉運擒突騎施可汗吐火仙【磧七迹翻蓋古盍翻】嘉運攻碎葉城吐火仙出戰敗走擒之於賀邏嶺分遣疏勒鎮守使夫蒙靈詧與拔汗那王阿悉爛逹干潜引兵突入怛邏斯城擒黑姓可汗爾微遂入曳建城取交河公主【交河公主事始二百一十二卷十一年】悉收散髪之民數萬以與拔汗那王威震西陲 壬午吐蕃寇白草安人等軍【白草軍在蔚茹水之西又鄯州星宿州之西有安人軍蔚茹水在原州蕭關縣此時吐蕃兵不能至疑白草軍當作白水軍】隴右節度使蕭炅擊破之【炅火迴翻】 甲申追諡孔子為文宣王先是祀先聖先師周公南向孔子東向坐制自今孔子南向坐被王者之服釋奠用宫懸【先悉薦翻被皮義翻周禮王宫懸諸侯軒懸卿大夫判懸士特懸注云宫懸四面懸象宫室四面有牆故謂之宫懸軒懸三面其形曲判懸又去其一面特懸又去其一面】追贈弟子皆為公侯伯【顔淵兖公閔子騫費侯冉伯牛鄆侯仲弓薛侯冉有徐侯季路衛侯宰我齊侯子貢黎侯子游吴侯子夏魏侯曾參成伯顓孫師陳伯澹臺減明江伯宓子賤單伯原憲原伯公冶長莒伯南宫适郯伯公晢哀郳伯曾點宿伯顔路杞伯商瞿蒙伯高柴其伯漆雕開滕伯公伯寮任伯司馬牛向伯樊遲樊伯有若卞伯公西赤邵伯巫馬期鄫伯梁鱣梁伯顔柳蕭伯冉孺郜伯曹恤豐伯伯䖍鄒伯公孫龍黄伯冉季產東平伯秦子南少梁伯漆雕歛武城伯顔子驕琅邪伯漆雕徒父須句伯壞駟赤北徵伯商澤睢陽伯石作蜀郈邑伯任不齊任城伯公夏首亢父伯公良孺東牟伯后處營丘伯秦開彭衙伯奚容箴下邳伯公肩定新伯顔襄臨沂伯鄔單銅鞮伯句井彊淇陽伯罕父黑乘丘伯秦商上洛伯申黨召陵伯公祖子之期思伯榮子期雩婁伯縣成渠野伯左人郢臨淄伯燕伋漁陽伯鄭子徒滎陽伯秦非汧陽伯施常乘氏伯顔噲朱虚伯步叔乘停干伯顔之僕東武伯原亢籍萊蕪伯樂欬昌平伯亷潔菖父伯顔何開陽伯叔孫會任丘伯狄黑臨濟伯郝巽平陸伯孔患汶陽伯公西與如重丘伯公西箴祝阿伯】 九月戊午處木昆鼠尼施弓月等諸部先隸突騎施者皆帥衆内附【帥讀曰率】仍請徙居安西管内 太子更名紹【更工衡翻】 冬十月辛巳改修東都明堂【按舊書帝紀即東都乾元殿改修明堂】 丙戌上幸驪山温泉十一月辛丑還宫 甲辰明堂成 劒南節度使張宥文吏不習軍旅悉以軍政委團練副使章仇兼瓊【據舊志上元後置團練使余攷唐制凡有團結兵之地則置團練使此時蜀有黎雅卭翼茂五州鎮防團結兵故有團練副使安史亂後諸州皆置團練使矣】兼瓊入奏事盛言安戎城可取上悦之丁巳以宥為光禄卿十二月以兼瓊為劒南節度使 初睿宗喪既除祫于太廟自是三年一祫五年一禘是歲夏既禘冬又當祫【祫疾夾翻】太常議以為祭數則凟【數所角翻】請停今年祫祭自是通計五年一祫一禘從之【史言如此乃合於五年再殷祭之義】<br />
<br />
  二十八年春正月癸巳上幸驪山温泉庚子還宫 二月荆州長史張九齡卒上雖以九齡忤旨逐之【卒子恤翻忤五故翻】然終愛重其人每宰相薦士輒問曰風度得如九齡不【不讀曰否】 三月丁亥朔日有食之 章仇兼瓊潜與安戎城中吐蕃翟都局及維州别駕董承晏結謀使局開門引内唐兵盡殺吐蕃將卒使監察御史許遠將兵守之【將即亮翻監古莧翻】遠敬宗之曾孫也【永徽顯慶之間許敬宗以姦佞致位公輔安史之亂遠乃能効死節以報國史故著其世以勉為臣者】 甲寅蓋嘉運入獻捷上赦吐火仙罪以為左金吾大將軍嘉運請立阿史那懷道之子昕為十姓可汗從之 【考異曰舊傳云嘉運請立懷道之子昕為可汗以鎮撫之莫賀逹干不肯曰討平蘇禄本是我之元謀若立史昕為主則國家何以酬賞於我乃不立史昕便令莫賀逹干統衆二十七年嘉運詣闕獻俘仍令將吐火仙獻于太廟會要二十九年以解瑟羅之子昕為可汗遣兵送之天寶元年昕至碎葉西南俱南城為莫賀咄逹干所殺三年安西節度使馬靈詧斬之更立其酋長為在地米里骨咄禄毗伽可汗按實録開元二十八年三月甲寅蓋嘉運俘吐火仙來獻四月辛未冊十姓可汗阿史那昕妻李氏為交河公主十二月乙卯突騎施可汗莫賀逹干率其妻子及纛官首領百餘帳内屬初莫賀逹干與烏蘇萬洛扇誘諸蕃叛于我上命蓋嘉運宣恩招諭皆相率而降新傳云逹干不肯立昕即誘部落叛詔嘉運招諭乃率妻子等降遂命統其衆後數年復以昕為可汗遣兵護送昕至俱闌城為莫賀咄所殺莫賀咄自為可汗安西節度使天蒙靈詧誅斬之若如舊傳所言嘉運便以莫賀逹干為可汗統衆則莫賀不應復叛且立可汗當須朝廷冊命嘉運豈得擅立于塞外也若未以為可汗則實録十二月不應謂突騎施可汗莫賀逹干也若如會要所言二十九年始立昕為可汗則實録二十八年四月不應已謂昕為十姓可汗也蓋嘉運既平突騎施即奏立昕為十姓可汗故莫賀逹干不服而叛明皇乃以莫賀逹干為小可汗止統突騎施之衆使嘉運招諭之故來降然昕為十姓可汗兼統諸部故明皇遣兵送之而為莫賀逹干所殺事或然也但實録脱略疑不敢質故略采諸書所見存其梗槩書之】 夏四月辛未以昕妻李氏為交河公主 六月吐蕃圍安戎城 上嘉蓋嘉運之功以為河西隴右節度使使之經略吐蕃嘉運恃恩流連不時發【蓋嘉運恃其邊功以自眤於人主是從流於上也在京師以酒色自娱而不即赴鎮是從流於下也史以流連二字言之旨哉】左丞相裴耀卿上疏【上時掌翻疏所去翻】以為臣近與嘉運同班觀其舉措誠勇烈有餘然言氣矜誇恐難成事昔莫敖狃于蒲騷之役卒喪楚師【左傳楚莫敖屈瑕既敗鄖師於蒲騷復伐羅鬭伯比送之曰莫敖必敗舉高心不固矣遂見楚子楚子入告夫人鄧曼鄧曼曰莫敖狃於蒲騷之役將自用也君若不鎮撫其不設備乎莫敖果不設備及羅羅與盧戎兩軍之大敗之狃女九翻陸德明曰騷音蕭又音縿卒子恤翻】今嘉運有驕敵之色臣竊憂之况防秋非遠未言發日若臨事始去則士卒尚未相識何以制敵且將軍受命鑿凶門而出今乃酣飲朝夕殆非憂國愛人之心若不可改易宜速遣進塗仍乞聖恩嚴加訓勵上乃趣嘉運行【趣讀曰促】已而嘉運竟無功【蓋嘉運小器易盈志氣惰矣安能有功】 秋八月甲戌幽州奏破奚契丹 冬十月甲子上幸驪山温泉辛巳還宫 吐蕃寇安戎城及維州發關中彊騎救之【騎奇寄翻】吐蕃引去更命安戎城曰平戎【更工衡翻】 十一月罷牛仙客朔方河東節度使突騎施莫賀逹干聞阿史那昕為可汗怒曰首誅蘇禄我之謀也今立史昕何以賞我遂帥諸部叛上乃立莫賀逹干為可汗使統突騎施之衆命蓋嘉運招諭之十二月乙卯莫賀逹干降【帥讀曰率降戶江翻】 金城公主薨【金城公主事始二百八卷中宗景龍元年】吐蕃告喪且請和上不許 是歲天下縣千五百七十三戶八百四十一萬二千八百七十一口四千八百一十四萬三千六百九西京東都米斛直錢不滿二百絹匹亦如之海内富安行者雖萬里不持寸兵【以開元之承平而戶口猶不及漢之盛時唐興以來治日少而亂日多也】<br />
<br />
  二十九年春正月癸巳上幸驪山温泉 丁酉制承前諸州饑饉皆待奏報然始開倉賑給【承前猶今言從前也然始猶今言然後也】道路悠遠何救懸絶自今委州縣長官與采訪使量事給訖奏聞【長知兩翻使疏吏翻量音良】 庚子上還宫 上夢玄元皇帝告云吾有像在京城西南百餘里汝遣人求之吾當與汝興慶宫相見【有宋大中祥符之事皆唐明皇教之也】上遣使求得之於盩厔樓觀山間【盩厔縣漢屬扶風後魏併入武功尋復後周為周南郡隋廢郡以盩厔縣為缺州唐屬岐州蘇軾曰樓觀山今為崇聖觀乃尹洙舊宅山脚有授經臺尚在使疏吏翻盩厔音舟窒觀古玩翻下同】夏閏四月迎置興慶宫五月命畫玄元真容分置諸州開元觀 六月吐蕃四十萬衆入寇至安仁軍【安仁軍當作安人軍】渾崕峰騎將臧希液帥衆五千擊破之【騎奇寄翻將即亮翻帥讀曰率 考異曰舊傳作盛希液今從唐歷】 秋七月丙寅突厥遣使來告登利可汗之喪初登利從叔二人分典兵馬號左右殺【從才用翻】登利患兩殺之專與其母謀誘右殺斬之【誘音酉】自將其衆【將即亮翻】左殺判闕特勒勒兵攻登利殺之立毗伽可汗之子為可汗俄為骨咄葉護所殺更立其弟【伽求迦翻咄當沒翻更工衡翻】尋又殺之骨咄葉護自立為可汗【考異曰舊傳云左殺自立為烏蘇米施可汗唐歷新傳皆云判闕特勒子為烏蘇米施可汗天寶初立今從之】上以突厥内亂癸酉命左羽林將軍孫老奴招諭囘紇葛邏禄拔悉密等部落【訖下沒翻邏郎佐翻】 乙亥東都洛水溢溺死者千餘人【溺奴狄翻】 平盧兵馬使安禄山傾巧善事人人多譽之【譽音余】上左右至平盧者禄山皆厚賂之由是上益以為賢御史中丞張利貞為河北采訪使至平盧禄山曲事利貞乃至左右皆有賂利貞入奏盛稱禄山之美八月乙未以禄山為營州都督充平盧軍使【考異曰實録此年八月以幽州節度副大使安禄山為營州刺史充平盧渤海黑水軍使舊紀以幽州節度副使安禄山為營州刺史平盧軍節度副使會要二十八年王斛斯為平盧節度使遂為定額按舊傳禄山自平盧兵馬使為平盧軍使蓋以平盧兵馬使帶幽州節度副使之名耳實録衍大字也天寶元年始以平盧為節度會要誤也】兩蕃渤海黑水四府經略使【唐謂奚契丹為兩蕃】 冬十月丙申上幸驪山温泉 壬寅分北庭安西為二節度十一月庚戌司空邠王守禮薨守禮庸鄙無才識每天將雨及霽守禮必先言之已而皆驗岐薛諸王言於上曰邠兄有術上問其故對曰臣無術則天時以章懷之故幽閉宫中十餘年【守禮幽閉事見二百四卷武后天授元年】歲賜敕杖者數四背瘢甚厚將雨則沈悶【瘢蒲官翻沈持林翻】將霽則輕爽臣以此知之耳因流涕霑襟上亦為之慘然【為于偽翻】 辛酉上還宫 辛未太尉寧王憲薨上哀惋特甚曰天下兄之天下也兄固讓於我【事見二百十卷睿宗景雲元年惋烏貫翻】為唐太伯常名不足以處之【處昌呂翻】乃諡曰讓皇帝其子汝陽王璉【璉資辛翻】上表追述先志謙冲不敢當帝號上不許歛日【歛力贍翻】内出服【天子之服也】以手書致於靈座書稱隆基白又名其基曰惠陵【惠陵在同州奉先縣西北十里】追諡其妃元氏曰恭皇后祔葬焉 十二月乙巳吐蕃屠逹化縣【逹化古澆河之地後周置逹化郡及縣隋廢郡以縣屬廓州縣西百二十里有澆河城】䧟石堡城蓋嘉運不能禦【果如裴耀卿之言】<br />
<br />
  資治通鑑卷二百十四<br />
<br />
<史部,編年類,資治通鑑>  <br>
   </div> 

<script src="/search/ajaxskft.js"> </script>
 <div class="clear"></div>
<br>
<br>
 <!-- a.d-->

 <!--
<div class="info_share">
</div> 
-->
 <!--info_share--></div>   <!-- end info_content-->
  </div> <!-- end l-->

<div class="r">   <!--r-->



<div class="sidebar"  style="margin-bottom:2px;">

 
<div class="sidebar_title">工具类大全</div>
<div class="sidebar_info">
<strong><a href="http://www.guoxuedashi.com/lsditu/" target="_blank">历史地图</a></strong>  
<a href="http://www.880114.com/" target="_blank">英语宝典</a>  
<a href="http://www.guoxuedashi.com/13jing/" target="_blank">十三经检索</a> 
<br><strong><a href="http://www.guoxuedashi.com/gjtsjc/" target="_blank">古今图书集成</a></strong> 
<a href="http://www.guoxuedashi.com/duilian/" target="_blank">对联大全</a> <strong><a href="http://www.guoxuedashi.com/xiangxingzi/" target="_blank">象形文字典</a></strong> 

<br><a href="http://www.guoxuedashi.com/zixing/yanbian/">字形演变</a>  <strong><a href="http://www.guoxuemi.com/hafo/" target="_blank">哈佛燕京中文善本特藏</a></strong>
<br><strong><a href="http://www.guoxuedashi.com/csfz/" target="_blank">丛书&方志检索器</a></strong> <a href="http://www.guoxuedashi.com/yqjyy/" target="_blank">一切经音义</a>  

<br><strong><a href="http://www.guoxuedashi.com/jiapu/" target="_blank">家谱族谱查询</a></strong>  <strong><a href="http://shufa.guoxuedashi.com/sfzitie/" target="_blank">书法字帖欣赏</a></strong> 
<br>

</div>
</div>


<div class="sidebar" style="margin-bottom:0px;">

<font style="font-size:22px;line-height:32px">QQ交流群9:489193090</font>


<div class="sidebar_title">手机APP 扫描或点击</div>
<div class="sidebar_info">
<table>
<tr>
	<td width=160><a href="http://m.guoxuedashi.com/app/" target="_blank"><img src="/img/gxds-sj.png" width="140"  border="0" alt="国学大师手机版"></a></td>
	<td>
<a href="http://www.guoxuedashi.com/download/" target="_blank">app软件下载专区</a><br>
<a href="http://www.guoxuedashi.com/download/gxds.php" target="_blank">《国学大师》下载</a><br>
<a href="http://www.guoxuedashi.com/download/kxzd.php" target="_blank">《汉字宝典》下载</a><br>
<a href="http://www.guoxuedashi.com/download/scqbd.php" target="_blank">《诗词曲宝典》下载</a><br>
<a href="http://www.guoxuedashi.com/SiKuQuanShu/skqs.php" target="_blank">《四库全书》下载</a><br>
</td>
</tr>
</table>

</div>
</div>


<div class="sidebar2">
<center>


</center>
</div>

<div class="sidebar"  style="margin-bottom:2px;">
<div class="sidebar_title">网站使用教程</div>
<div class="sidebar_info">
<a href="http://www.guoxuedashi.com/help/gjsearch.php" target="_blank">如何在国学大师网下载古籍?</a><br>
<a href="http://www.guoxuedashi.com/zidian/bujian/bjjc.php" target="_blank">如何使用部件查字法快速查字?</a><br>
<a href="http://www.guoxuedashi.com/search/sjc.php" target="_blank">如何在指定的书籍中全文检索?</a><br>
<a href="http://www.guoxuedashi.com/search/skjc.php" target="_blank">如何找到一句话在《四库全书》哪一页?</a><br>
</div>
</div>


<div class="sidebar">
<div class="sidebar_title">热门书籍</div>
<div class="sidebar_info">
<a href="/so.php?sokey=%E8%B5%84%E6%B2%BB%E9%80%9A%E9%89%B4&kt=1">资治通鉴</a> <a href="/24shi/"><strong>二十四史</strong></a>&nbsp; <a href="/a2694/">野史</a>&nbsp; <a href="/SiKuQuanShu/"><strong>四库全书</strong></a>&nbsp;<a href="http://www.guoxuedashi.com/SiKuQuanShu/fanti/">繁体</a>
<br><a href="/so.php?sokey=%E7%BA%A2%E6%A5%BC%E6%A2%A6&kt=1">红楼梦</a> <a href="/a/1858x/">三国演义</a> <a href="/a/1038k/">水浒传</a> <a href="/a/1046t/">西游记</a> <a href="/a/1914o/">封神演义</a>
<br>
<a href="http://www.guoxuedashi.com/so.php?sokeygx=%E4%B8%87%E6%9C%89%E6%96%87%E5%BA%93&submit=&kt=1">万有文库</a> <a href="/a/780t/">古文观止</a> <a href="/a/1024l/">文心雕龙</a> <a href="/a/1704n/">全唐诗</a> <a href="/a/1705h/">全宋词</a>
<br><a href="http://www.guoxuedashi.com/so.php?sokeygx=%E7%99%BE%E8%A1%B2%E6%9C%AC%E4%BA%8C%E5%8D%81%E5%9B%9B%E5%8F%B2&submit=&kt=1"><strong>百衲本二十四史</strong></a>  <a href="http://www.guoxuedashi.com/so.php?sokeygx=%E5%8F%A4%E4%BB%8A%E5%9B%BE%E4%B9%A6%E9%9B%86%E6%88%90&submit=&kt=1"><strong>古今图书集成</strong></a>
<br>

<a href="http://www.guoxuedashi.com/so.php?sokeygx=%E4%B8%9B%E4%B9%A6%E9%9B%86%E6%88%90&submit=&kt=1">丛书集成</a> 
<a href="http://www.guoxuedashi.com/so.php?sokeygx=%E5%9B%9B%E9%83%A8%E4%B8%9B%E5%88%8A&submit=&kt=1"><strong>四部丛刊</strong></a>  
<a href="http://www.guoxuedashi.com/so.php?sokeygx=%E8%AF%B4%E6%96%87%E8%A7%A3%E5%AD%97&submit=&kt=1">說文解字</a> <a href="http://www.guoxuedashi.com/so.php?sokeygx=%E5%85%A8%E4%B8%8A%E5%8F%A4&submit=&kt=1">三国六朝文</a>
<br><a href="http://www.guoxuedashi.com/so.php?sokeytm=%E6%97%A5%E6%9C%AC%E5%86%85%E9%98%81%E6%96%87%E5%BA%93&submit=&kt=1"><strong>日本内阁文库</strong></a> <a href="http://www.guoxuedashi.com/so.php?sokeytm=%E5%9B%BD%E5%9B%BE%E6%96%B9%E5%BF%97%E5%90%88%E9%9B%86&ka=100&submit=">国图方志合集</a> <a href="http://www.guoxuedashi.com/so.php?sokeytm=%E5%90%84%E5%9C%B0%E6%96%B9%E5%BF%97&submit=&kt=1"><strong>各地方志</strong></a>

</div>
</div>


<div class="sidebar2">
<center>

</center>
</div>
<div class="sidebar greenbar">
<div class="sidebar_title green">四库全书</div>
<div class="sidebar_info">

《四库全书》是中国古代最大的丛书,编撰于乾隆年间,由纪昀等360多位高官、学者编撰,3800多人抄写,费时十三年编成。丛书分经、史、子、集四部,故名四库。共有3500多种书,7.9万卷,3.6万册,约8亿字,基本上囊括了古代所有图书,故称“全书”。<a href="http://www.guoxuedashi.com/SiKuQuanShu/">详细>>
</a>

</div> 
</div>

</div>  <!--end r-->

</div>
<!-- 内容区END --> 

<!-- 页脚开始 -->
<div class="shh">

</div>

<div class="w1180" style="margin-top:8px;">
<center><script src="http://www.guoxuedashi.com/img/plus.php?id=3"></script></center>
</div>
<div class="w1180 foot">
<a href="/b/thanks.php">特别致谢</a> | <a href="javascript:window.external.AddFavorite(document.location.href,document.title);">收藏本站</a> | <a href="#">欢迎投稿</a> | <a href="http://www.guoxuedashi.com/forum/">意见建议</a> | <a href="http://www.guoxuemi.com/">国学迷</a> | <a href="http://www.shuowen.net/">说文网</a><script language="javascript" type="text/javascript" src="https://js.users.51.la/17753172.js"></script><br />
  Copyright &copy; 国学大师 古典图书集成 All Rights Reserved.<br>
  
  <span style="font-size:14px">免责声明:本站非营利性站点,以方便网友为主,仅供学习研究。<br>内容由热心网友提供和网上收集,不保留版权。若侵犯了您的权益,来信即刪。scp168@qq.com</span>
  <br />
ICP证:<a href="http://www.beian.miit.gov.cn/" target="_blank">鲁ICP备19060063号</a></div>
<!-- 页脚END --> 
<script src="http://www.guoxuedashi.com/img/plus.php?id=22"></script>
<script src="http://www.guoxuedashi.com/img/tongji.js"></script>

</body>
</html>
