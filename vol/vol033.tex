資治通鑑卷三十三   宋 司馬光 撰

胡三省 音註

漢紀二十五|{
	起閼逢攝提格盡旃蒙單閼凡二年}


孝成皇帝下

綏和二年春正月上行幸甘泉郊泰畤 二月 |{
	考異荀紀云赦天下今本紀無之故不取}
壬子丞相方進薨時熒惑守心|{
	心為明堂熒惑守心王者惡之火曰熒惑星熒惑天子理也雖有明天子必視熒惑所在見天文志}
丞相府議曹平陵李尋|{
	議曹職在論議自公府至州郡皆有之}
奏記方進言災變廹切大責日加安得保斥逐之戮|{
	師古曰言其事重不但斥逐而已也}
闔府三百餘人|{
	師古曰三百餘人言丞相之官屬也}
唯君侯擇其中與盡節轉凶方進憂之不知所出會郎賁麗善為星|{
	善為甘石之學也師古曰賁姓也麗名也賁音肥}
言大臣宜當之上乃召見方進還歸未及引决|{
	師古曰引决自裁也還從宣翻}
上遂賜冊責讓以政事不治災害並臻百姓窮困|{
	冊即策書也說文冊符命也諸侯進受於王也象其札一長一短中有二䋲之形程大昌演繁露曰策制長二尺短者半之其次一長一短兩編下陮用篆書此漢策拜丞相之制也至策免則以尺一木兩行而隸書與策拜異矣治直吏翻}
曰欲退君位尚未忍使尚書令賜君上尊酒十石養牛一君審處焉|{
	如淳曰漢儀注有天地大變天下大過皇帝使侍中持節乘四白馬賜上尊酒十斛牛一頭策告殃咎使者去半道丞相即上病使者還未白事尚書以丞相不起聞律稻米一斗得酒一斗為上尊稷米一斗得酒一斗為中尊粟米一斗得酒一斗為下尊師古曰稷即粟也宜為黍米不當言稷且作酒自有澆淳之異為上中下耳處昌呂翻}
方進即日自殺上秘之遣九卿策贈印綬賜乘輿祕器少府供張柱檻皆衣素|{
	乘繩證翻祕器東園祕器也供音居用翻張音竹亮翻師古曰柱屋柱也檻軒前闌版也皆以白素衣之衣音於既翻}
天子親臨弔者數至禮賜異於它相故事|{
	師古曰漢舊儀云丞相有疾皇帝法駕親至問疾從西門入即薨移居第中車駕往弔賜棺槨及歛具贈錢葬地葬日公卿已下會葬數所角翻}


臣光曰晏嬰有言天命不慆不貳其命|{
	晏子對齊侯禳彗之辭也杜預曰慆疑也音他刀翻}
禍福之至安可移乎昔楚昭王宋景公不忍移災於卿佐曰移腹心之疾寘諸股肱何益也|{
	左傳哀六年有雲如衆赤鳥夹日而飛三日楚子使問諸周太史周太史曰其當王身乎若禜之可移於令尹司馬王曰移腹心之疾而寘諸股肱何益遂弗禜史記宋景公時熒惑守心景公憂之司星子韋曰可移於相公曰相吾之股肱曰可移於民公曰君者待民曰可移於歲公曰歲饑民困吾誰為君子韋曰天高聽卑君有仁人之言三熒惑宜有動候之果徙三度}
藉其災可移|{
	藉之為言借也假也設為之言以發所欲言之意}
仁君猶不忍為况不可乎使方進罪不至死而誅之以當大變是誣天也方進有罪當刑隱其罪而厚其葬是誣人也孝成欲誣天人而卒無所益|{
	卒于恤翻}
可謂不知命矣

三月上行幸河東祠后土 丙戌帝崩于未央宫|{
	臣瓚曰帝年二十即位即位二十六年夀四十五師古曰即位明年乃改元耳夀四十六}
帝素彊無疾病|{
	自彊以為無疾病也}
是時楚思王衍梁王立來朝|{
	衍楚孝王囂之子}
明旦當辭去上宿供張白虎殿又欲拜左將軍孔光為丞相已刻侯印書贊|{
	師古曰贊謂廷拜之文贊進也廷進而拜之也書贊者書贊辭於策也}
昏夜平善鄉晨傅絝韈欲起|{
	應劭曰傅著也師古曰鄉讀曰嚮傅讀曰附絝古袴字也韈音武伐翻}
因失衣不能言|{
	攬衣而失手緩縱也}
晝漏上十刻而崩|{
	司漏之度有晝漏夜漏是時三月晝漏五十八刻上者漏箭浮而上也上時掌翻}
民間讙譁咸歸罪趙昭儀|{
	讙許元翻}
皇太后詔大司馬莽雜與御史丞相廷尉治問皇帝起居發病狀趙昭儀自殺

班彪贊曰臣姑充後宫為媫妤|{
	媫妤音接予}
父子昆弟侍帷幄數為臣言|{
	數所角翻為于偽翻}
成帝善修容儀升車正立不内顧不疾言不親指|{
	師古曰不内顧者儼然端嚴不迴眄也不疾言者為輕肆也不親指者為惑下也此三句者本論語鄉黨篇述孔子之事班氏引之今論語云車中不内顧不疾言不親指内顧者說者以為前視不過衡軶旁視不過輢較與此不同輢音於綺翻余謂此亦成帝學論語而有得於修容儀者也夫聖人道德之容積于中而發於外帝則因論語之文而剛制其外而已損者三樂帝何不能服膺斯言乎嗚呼豈唯是哉論語二十篇修身齊家治國平天下盡在是矣}
臨朝淵嘿尊嚴若神可謂穆穆天子之容矣|{
	淵深嘿静也師古曰禮記云天子穆穆諸侯皇皇大夫濟濟士蹌蹌毛晃曰穆穆和敬貌朝直遥翻下同}
博覽古今容受直辭公卿奏議可述遭世承平上下和睦然湛于酒色|{
	師古曰湛讀曰耽孔穎達曰耽者過禮之樂}
趙氏亂内外家擅朝言之可為於邑|{
	師古曰於邑短氣貌讀如本字於又音烏邑又音烏合翻}
建始以來王氏始執國命哀平短祚莽遂簒位蓋其威福所由來者漸矣|{
	言王氏之禍始於成帝}


是日孔光於大行前拜受丞相博山侯印綬|{
	大行前謂大行皇帝柩前韋昭曰大行者不反之辭恩澤侯表博山侯國於南陽順陽}
富平侯張放聞帝崩思慕哭泣而死|{
	放自河東都尉徵為侍中光禄勲丞相翟方進奏免放遣就國}
荀悦論曰放非不愛上忠不存焉故愛而不忠仁之賊也

皇太后詔南北郊長安如故|{
	永始三年復甘泉秦畤雍五畤汾隂后土祠罷長安南北郊}
夏四月丙午太子即皇帝位謁高廟尊皇太后曰太皇太后皇后曰皇太后大赦天下哀帝初立躬行儉約省減諸用政事由已出朝廷翕然望至治焉|{
	治直吏翻}
己卯葬孝成皇帝于延陵|{
	臣瓚曰自崩至葬凡五十四日延陵在扶風去長安}


|{
	六十二里 考異曰成紀三月丙戌帝崩于未央宮四月己卯葬延陵臣瓚曰自崩及葬凡五十四日漢紀乃云三月丙午帝崩四月己卯葬延陵自崩及葬三十四日按是年三月己巳朔無丙午四月己亥朔無己卯若依成紀當云五月己卯葬依荀紀當云閏三月丙午崩二者各冇差舛未知孰是按是年閏七月不當頓差四月今且從成紀之文}
太皇太后令傅太后丁姬十日一至未央宫有詔問丞相大司空定陶共王太后宜當何居|{
	共讀曰恭}
丞相孔光素聞傅太后為人剛暴長於權謀自帝在襁褓而養長教道至於成人|{
	養長知兩翻道讀曰導}
帝之立又有力|{
	事見上卷元延四年}
光心恐傅太后與政事|{
	師古曰與讀曰豫}
不欲與帝旦夕相近|{
	近其靳翻}
即議以為定陶太后宜改築宫大司空何武曰可居北宫上從武言北宫有紫房複道通未央宫|{
	長安記桂宫在未央宫北亦曰北宫余按漢書平帝紀成帝趙皇后退居北宫哀帝傅皇后退居桂宫則北宫桂宫自是兩宫}
傅太后果從複道朝夕至帝所求欲稱尊號貴寵其親屬使上不得由直道行|{
	師古曰不得依正直之道也余謂小宗不得間大宗藩后不得位匹長樂私戚不得妄干恩澤所謂正道也}
高昌侯董宏|{
	宏高昌侯董忠子也功臣表高昌侯國于千乘}
希指上書言秦莊襄王母本夏氏而為華陽夫人所子及即位俱稱太后|{
	事見六卷秦孝文王元年上時掌翻華戶化翻}
宜立定陶共王后為帝太后事下有司|{
	下遐嫁翻}
大司馬王莽左將軍關内侯領尚書事師丹劾奏宏知皇太后至尊之號天下一統而稱引亡秦以為比喻詿誤聖朝|{
	劾戶槩翻詿戶卦翻}
非所宜言大不道上新立謙讓納用莽丹言免宏為庶人傅太后大怒要上欲必稱尊號|{
	要一遥翻}
上乃白太皇太后令下詔尊定陶恭王為恭皇 五月丙戌立皇后傅氏傅太后從弟晏之子也|{
	從才用翻}
詔曰春秋母以子貴|{
	見公羊春秋傳隱元年}
宜尊定陶太后曰恭皇太后丁姬曰恭皇后各置左右詹事食邑如長信宫中宫|{
	應劭曰成帝母王太后居長信宫李奇曰傳姬如長信丁姬如中宫也師古曰中宫皇后之宫}
追尊傅父為崇祖侯丁父為褒德侯封舅丁明為陽安侯舅子滿為平周侯皇后父晏為孔鄉侯|{
	師古曰傅父傅太后之父丁父丁太后之父地理志汝南郡有陽安縣恩澤侯表平周侯食邑於南陽湖陽孔鄉侯食邑於沛郡夏丘}
皇太后弟侍中光禄大夫趙欽為新城侯|{
	地理志河南郡有新城縣}
太皇太后詔大司馬莽就第避帝外家莽上疏乞骸骨帝遣尚書令詔起莽又遣丞相孔光大司空何武左將軍師丹衛尉傅喜白太皇太后曰皇帝聞太后詔甚悲大司馬即不起皇帝即不敢聽政太后乃復令莽視事|{
	太皇太后止稱太后史省文復扶又翻}
成帝之世鄭聲尤甚|{
	周末有鄭衛之樂東門溱洧之詩鄭聲也桑中濮上之音衛聲也皆淫聲也其後凡淫聲通謂之鄭聲孔子曰鄭聲淫是也}
黄門名倡丙彊景武之屬富顯於世|{
	倡音齒良翻}
貴戚至與人主争女樂|{
	盖王氏五侯淳于長之屬也}
帝自為定陶王時疾之又性不好音|{
	好呼到翻}
六月詔曰孔子不云乎放鄭聲鄭聲淫|{
	師古曰論語載孔子之言鄭國有溱洧之水男女亟於其間聚會故俗亂而樂淫}
其罷樂府官|{
	立樂府見十九卷元狩三年}
郊祭樂及古兵法武樂在經非鄭衛之樂者别屬他官|{
	郊祭樂亦武帝置今以給祠南北郊大樂鼓嘉至鼓邯鄲鼓騎吹鼓江南鼓淮南鼓巴俞鼓歌鼓楚嚴鼓梁皇鼔臨淮鼓兹邡鼔朝賀置酒陳殿上應古兵法凡鼓十二人員百二十八人郊祭員十三人諸族樂人兼雲招給祠南北郊用六十七人兼給事雅樂用四人夜誦員五人剛别柎員二人給盛德主調箎員二人聽工以日知律冬夏至一人鍾工磬工簫工員各一人僕射二人主領諸樂人皆不可罷竽工員三人罷一琴工員五人罷三柱工員二人罷一繩弦工員六人罷四鄭四會員六十二人留一人給事雅樂餘罷張瑟員八人留一安世樂鼓沛吹鼓族歌鼓陳吹鼔商樂鼓東海鼓長樂鼓縵樂鼔凡鼔八員百二十八人朝賀置酒陳前殿房中不應經法治竽員五人楚鼔員六人常從倡三十人常從象人四人詔隨常從倡十六人秦倡員三十九人秦倡象人員三人詔隨秦倡一人雅大人員九人朝賀置酒為樂楚四會員十七人巴四會員十二人銚四會員十二人齊四會員十九人蔡謳員三人齊謳員六人竽瑟鐘磬員五人皆鄭聲可罷師學百四十四人其七十二人給太官桐馬酒其七十二人可罷大凡八百二十九人其三百八十八人不可罷可領屬大樂其四百四十一人不應經法或鄭衛之聲皆可罷奏可晋灼曰邡音方師古曰招讀與翹同剛及别柎皆鼔名也柎音膚柱工主筝瑟之柱者弦琴瑟之弦繩言主糾合作之也縵樂雜樂也音漫挏音動李奇曰以馬乳為酒撞挏乃成孟康曰象人若今戲蝦魚師子者也韋昭曰著假面者也}
凡所罷省過半然百姓漸漬日久又不制雅樂有以相變豪富吏民湛沔自若|{
	漸讀曰沾師古曰湛讀曰沈又讀曰耽自若言自如故也}
王莽薦中壘校尉劉歆有材行|{
	行下孟翻}
為侍中稍遷光禄大夫貴幸更名秀|{
	歆改名秀冀以應圖䜟更工衡翻}
上復令秀典領五經卒父前業|{
	秀父向典校書見三十卷河平三年師古曰卒終也復扶又翻卒子恤翻}
秀於是總羣書而奏其七畧有輯略有六藝略有諸子略有詩賦略有兵書略有術數略有方技略|{
	師古曰輯略謂羣書之摠要輯與集同六藝六經也諸子即下九流是也詩賦則自屈原荀卿至揚雄等所作也兵書則權謀技巧形勢隂陽之書也術數則天文歷譜五行蓍龜雜占形法之書也方技則醫經經方房中神仙之書也}
凡書六略三十八種|{
	種章勇翻}
五百九十六家萬三千二百六十九卷其叙諸子分為九流曰儒曰道曰隂陽曰法曰名曰墨曰從横曰雜曰農|{
	從子容翻}
以為九家皆起於王道既微諸侯力政時君世主好惡殊方|{
	好呼到翻惡烏路翻}
是以九家之術蠭出並作|{
	師古曰蠭與鋒同}
各引一端崇其所善以此馳說取合諸侯其言雖殊譬如水火相滅亦相生也|{
	水滅火而生木木復生火}
仁之與義敬之與和相反而皆相成也易曰天下同歸而殊途一致而百慮|{
	師古曰下繫之辭}
今異家者推所長窮知究慮以明其指雖有蔽短合其要歸亦六經之支與流裔|{
	師古曰裔衣末也其於六經如水之下流衣之末裔}
使其人遭明王聖主得其所折中|{
	中竹仲翻}
皆股肱之材已|{
	師古曰已語終辭}
仲尼有言禮失而求諸野|{
	師古曰言都邑失禮則於外野求之亦將有獲}
方今去聖久遠道術缺廢無所更索|{
	師古曰索求也索山客翻}
彼九家者不猶愈於野乎|{
	師古曰愈勝也}
若能修六藝之術而觀此九家之言舍短取長|{
	舍讀曰捨}
則可以通萬方之略矣 河間惠王良能修獻王之行|{
	行下孟翻}
母太后薨服喪如禮詔益封萬戶以為宗室儀表|{
	師古曰儀表者言為禮儀之表率余謂有儀可象謂之儀四外望之以正謂之表}
初董仲舒說武帝|{
	說輸芮翻}
以秦用商鞅之法除井田|{
	事見三卷周顯王十九年}
民得賣買富者田連阡陌貧者亡立錐之地|{
	亡讀與無同}
邑有人君之尊里有公侯之富小民安得不困古井田灋雖難卒行|{
	卒讀曰猝}
宜少近古|{
	少詩沼翻}
限民名田以贍不足|{
	師古曰名田占田也各為限制不使富者過制則可使貧弱之家足也}
塞并兼之路去奴婢除專殺之威|{
	服䖍曰不得專殺奴婢也塞悉則翻去羌呂翻}
薄賦歛省繇役|{
	歛力贍翻繇讀曰徭}
以寛民力然後可善治也|{
	治直吏翻}
及上即位師丹復建言|{
	復扶又翻}
今累世承平豪富吏民訾數鉅萬|{
	訾與貲同}
而貧弱愈困宜略為限天子下其議|{
	下遐稼翻}
丞相光大司空武奏請自諸侯王列侯公主名田各有限關内侯吏民名田皆毋過三十頃奴婢毋過三十人|{
	據哀帝紀有司條奏諸侯王列侯得名田國中列侯在長安及公主得名田縣道關内侯吏民名田皆毋得過三十頃諸侯王奴婢二百人列侯公主百人關内侯吏民三十人與此少異食貨志亦與紀同}
期盡三年犯者沒入官時田宅奴婢賈為減賤|{
	賈讀曰價}
貴戚近習皆不便也|{
	皆不以為便於己也}
詔書且須後|{
	師古曰須待也}
遂寢不行又詔齊三服官諸官織綺繡難成害女紅之物皆止無作輸|{
	齊三服官及諸織官皆無作難成之物以輸送也如淳曰紅亦工也其所作已成未成皆止無復作皆輸所近官府也師古曰如說非也謂未成者不作已成者不輸耳余謂如說固非顔說亦未若余說之為簡易明白也}
除任子令及誹謗詆欺灋|{
	應劭曰任子令者漢儀注吏二千石以上視事滿三年得任同產若子一人為郎不以德選故除之師古曰任保也詆誣也}
掖庭宫人年三十以下出嫁之|{
	重絶人道也}
官奴婢五十以上免為庶人益吏三百石以下俸 上置酒未央宫内者令為傅太后張幄坐於太皇太后坐旁|{
	百官志内者令屬少府以宦者為之掌中布張諸衣物為于偽翻師古曰坐音杜卧翻下同}
大司馬莽按行|{
	行下孟翻}
責内者令曰定陶太后藩妾何以得與至尊並徹去更設坐|{
	去羌呂翻更工衡翻}
傅太后聞之大怒不肯會重怨恚莽|{
	師古曰會謂至酒所也重音直用翻}
莽復乞骸骨|{
	復扶又翻}
秋七月丁卯上賜莽黄金五百斤安車駟馬罷就第 |{
	考異曰公卿表十一月丁卯大司馬莽免庚午師丹為大司馬四月徙又曰十月癸酉丹為大司空又曰太子太傅師丹為左將軍五月遷荀紀七月丁巳大司馬莽免按丹若以十一月為司馬四月徙官不得以十月為司空也七月丁卯朔無丁巳年表月誤荀紀日誤}
公卿大夫多稱之者上乃加恩寵置中黄門為莽家給使|{
	蘇林曰使黄門在其家為使令}
十日一賜餐又下詔益封曲陽侯根安陽侯舜新都侯莽丞相光大司空武邑戶各有差|{
	益封根二千戶舜五百戶舜音子也莽三百五十戶光千戶武更以南陽犨之博望鄉為汜鄉侯國益封千戶}
以莽為特進給事中朝朔望見禮如三公|{
	朝直遥翻}
又還紅陽侯立於京師|{
	立就國見上卷去年}
傅太后從弟右將軍喜好學問有志行|{
	從才用翻好呼到翻行下孟翻下同}
王莽既罷退衆庶歸望於喜初上之官爵外親也|{
	外親外家之親}
喜獨執謙稱疾傅太后始與政事數諫之|{
	與讀曰豫數所角翻}
由是傅太后不欲令喜輔政庚午以左將軍師丹為大司馬封高鄉亭侯|{
	按丹傅及恩澤侯表皆云封高樂侯國於東海}
賜喜黄金百斤上右將軍印綬以光禄大夫養病以光禄勲淮陽彭宣為右將軍大司空何武尚書令唐林皆上書言喜行義修潔忠誠憂國内輔之臣也|{
	言可為内朝輔弼之臣}
今以寢病一旦遣歸衆庶失望皆曰傅氏賢子以議論不合於定陶太后故退百寮莫不為國恨之|{
	為于偽翻}
忠臣社稷之衛魯以季友治亂|{
	師古曰謂季氏亡則魯不昌治直吏翻}
楚以子玉輕重|{
	師古曰謂楚殺子玉而晋侯喜可知}
魏以無忌折衝|{
	事見上卷秦莊襄王三年}
項以范增存亡|{
	事見高帝紀}
百萬之衆不如一賢故秦行千金以間亷頗|{
	事見五卷周赧王五十五年間古莧翻}
漢散萬金以疏亞夫|{
	事見十卷高帝三年疏與疎同}
喜立於朝陛下之光輝傅氏之廢興也|{
	如淳曰傅喜顯則傅氏興其廢亦如之晋灼曰用喜於陛下有光明而傅氏之廢復得興也師古曰如說是余謂晋說亦未可厚非}
上亦自重之故尋復進用焉|{
	明年復進用喜復扶又翻}
建平侯杜業上書詆曲陽侯根高陽侯薛宣安昌侯張禹而薦朱博帝少而聞知王氏驕盛|{
	少詩照翻}
心不能善以初立故且優之後月餘司隸校尉解光|{
	解戶買翻}
奏曲陽侯先帝山陵未成公聘取掖庭女樂五官殷嚴王飛君等置酒歌舞|{
	如淳曰五官官名也外戚傳云五官視三百石}
及根兄子成都侯况亦聘取故掖庭貴人以為妻|{
	况商子也}
皆無人臣禮大不敬不道於是天子曰先帝遇根况父子至厚也今乃背恩忘義|{
	背蒲妹翻}
以根嘗建社稷之策|{
	師古曰謂立哀帝為嗣也事見上卷元延四年}
遣歸國免况為庶人歸故郡|{
	王氏故魏郡元城人}
根及况父商所薦舉為官者皆罷|{
	以具黨也}
九月庚申地震自京師到北邊郡國三十餘處壞城郭|{
	壞音怪}
凡壓殺四百餘人上以災異問待詔李尋 |{
	考異曰尋傳云使侍中衛尉傅喜問尋按公卿表傅喜為衛尉二月遷右將軍十一月罷地震在九月當是時喜已不為衛尉矣}
對曰夫日者衆陽之長|{
	長知兩翻}
人君之表也君不修道則日失其度晻昩亡光|{
	師古曰晻與暗同又音烏感翻}
間者日尤不精光明侵奪失色邪氣珥蜺數作|{
	孟康曰暈適背鐍抱珥虹蜺皆日旁氣也珥形點黑也如淳曰雄為虹雌為蜺凡氣在旁相對為珥在旁如半環向日為抱向外為背有氣刺日為鐍鐍映傷也適者日之將食先有黑之變也蜺讀曰齧珥音仍吏翻數所角翻下同}
小臣不知内事竊以日視陛下志操衰於始初多矣唯陛下執乾剛之德彊志守度|{
	謂守法度也}
毋聽女謁邪臣之態諸保阿乳母甘言卑辭之託斷而勿聽勉彊大義|{
	斷丁管翻彊其兩翻}
絶小不忍良有不得已|{
	良甚也}
可賜以貨財不可私以官位誠皇天之禁也臣聞月者衆隂之長|{
	長知兩翻}
妃后大臣諸侯之象也閒者月數為變此為母后與政亂朝|{
	與讀曰豫朝直遥翻下同}
隂陽俱傷兩不相便外臣不知朝事竊信天文即如此近臣已不足杖矣|{
	師古曰杖謂倚任也}
唯陛下親求賢士無彊所惡|{
	師古曰邪佞之人誠可賤惡勿得寵而異之令其盛彊也惡烏路翻}
以崇社稷尊彊本朝臣聞五行以水為本|{
	五行一曰水水者天一所生}
水為準平王道公正修明則百川理落脉通|{
	師古曰落謂經絡也}
偏黨失綱則湧溢為敗今汝潁漂涌|{
	地理志潁川郡陽城縣陽乾山潁水所出東至沛郡下蔡縣入淮過郡三行千五百里汝水出汝南郡定陵縣高陵山東南至新蔡入淮過郡四行千三百四十里}
與雨水並為民害此詩所謂百川沸騰咎在皇甫卿士之屬|{
	師古曰詩小雅十月之交之詩也皇甫卿士周室女寵之族也}
唯陛下少抑外親大臣臣聞地道柔静隂之常義也聞者關東地數震|{
	數所角翻}
宜務崇陽抑隂以救其咎固志建威|{
	固志以用英俊建威以黜姦邪建立也}
閉絶私路拔進英雋退不任職以彊本朝夫本彊則精神折衝|{
	師古曰言有欲衝突為害者則折挫之}
本弱則招殃致凶為邪謀所陵聞往者淮南王作謀之時其所難者獨有汲黯以為公孫弘等不足言也|{
	事見十九卷武帝元狩元年}
弘漢之名相於今無比而尚見輕何况亡弘之屬乎故曰朝廷亡人則為賊亂所輕|{
	亡讀曰無}
其道自然也 騎都尉平當使領河隄|{
	師古曰為使而領其事使音疏吏翻}
奏九河今皆窴滅|{
	窴與填同}
按經義治水有決河深川|{
	師古曰決分泄也深浚治也治直之翻下同}
而無隄防壅塞之文|{
	塞悉則翻}
河從魏郡以東多溢決水迹難以分明四海之衆不可誣|{
	爾雅九夷八狄七戎六蠻謂之四海孔穎達曰東方曰夷者風俗通云東方人好生萬物觝觸地而出夷者觝也其類有九依東夷傳一曰玄莬二曰樂浪三曰高麗四曰滿飾五曰鳬臾六曰索家七曰東屠八曰倭人九曰天鄙南方曰蠻者風俗通云君臣同川而浴極為簡慢蠻者慢也其類有八李廵注爾雅云一曰天竺二曰咳首三曰僬僥四曰跛踵五曰穿胷六曰儋耳七曰狗軹八曰旁舂西方曰戎者風俗通云斬伐殺生不得其中戎者兇也其類有六李廵注爾雅云一曰僥夷二曰戎央三曰老白四曰耆羌五曰鼻息六曰天剛北方曰狄者風俗通云父子嫂叔同宂無别狄者辟也其行邪辟其類有五李廵注爾雅云一曰月支二曰穢貊三曰匈奴四曰單于五曰白屋諸儒之說畧有異同然平當所謂四海之衆但言四海之内之人耳}
宜博求能浚川疏河者上從之待詔賈讓奏言治河有上中下策古者立國居民疆理土地必遺川澤之分度水埶所不及|{
	師古曰遺留也度計也言川澤水所流聚之處皆留而置之不以為居邑而妄墾殖必計水之所不及然後居而田之也分音扶問翻度音大各翻}
大川無防小水得入陂障卑下以為汙澤|{
	師古曰停水曰汙音一胡翻}
使秋水多得其所休息左右游波寛緩而不廹夫土之有川猶人之有口也治土而防其川猶止兒啼而塞其口豈不遽止然其死可立而待也|{
	塞悉則翻師古曰遽速也音其庶翻}
故曰善為川者決之使道善為民者宣之使言|{
	國語召公諫厲王監謗之辭師古曰道讀曰導導道引也}
盖隄防之作近起戰國雍防百川各以自利|{
	師古曰雍讀曰壅}
齊與趙魏以河為竟|{
	竟讀曰境}
趙魏瀕山|{
	師古曰瀕山猶言以山為邊界也瀕音頻又音賓余謂趙魏之地一邊接山則地勢高非邊界也}
齊地卑下|{
	齊地瀕海故卑下也}
作隄去河二十五里河水東扺齊隄則西泛趙魏趙魏亦為隄去河二十五里雖非其正水尚有所遊盪時至而去則填淤肥美|{
	淤依據翻}
民耕田之或久無害稍築宮宅遂成聚落大水時至漂沒則更起隄防以自救稍去其城郭排水澤而居之湛溺自其宜也|{
	師古曰湛讀曰沈音持林翻}
今隄防陿者去水數百步|{
	陿與狭同}
遠者數里於故大隄之内復有數重|{
	復扶又翻重直龍翻}
民居其間此皆前世所排也河從河内黎陽至魏郡昭陽東西互有石隄激水使還百餘里閒河再西三東廹阨如此不得安息|{
	地理志黎陽縣屬魏郡晉灼曰黎山在其南河水經其東其山上碑云縣取山之名取水之陽以為名按溝洫志具載讓奏曰河從河内北至黎陽為石隄激使扺東郡平剛又為石隄使西北扺黎陽觀下又為石隄使東北扺東郡津北又為石隄使西北扺魏郡昭陽又為石隄激使東北}
今行上策徙冀州之民當水衝者決黎陽遮害亭|{
	遮害亭在淇口東十八里有金隄隄高一丈自淇口東地稍高至遮害亭西五丈水經注曰舊有宿胥口河水於此北入}
放河使北入海河西薄大山東薄金隄埶不能遠泛濫期月自定|{
	薄伯各翻}
難者將曰若如此敗壞城郭田廬冢墓以萬數百姓怨恨|{
	難乃旦翻壞音怪敗補邁翻}
昔大禹治水山陵當路者毁之故鑿龍門闢伊闕析㡳柱破碣石墮斷天地之性|{
	師古曰闢開也析分也墮毁也音火規翻斷丁管翻}
此乃人功所造何足言也|{
	人功所造謂城郭田廬冢墓也}
今瀕河十郡治隄歲費且萬萬|{
	河南河内東郡陳留魏郡平原千乘信都清河渤海凡十郡}
及其大決所殘無數如出數年治河之費以業所徙之民遵古聖之法定山川之位|{
	謂依禹迹也}
使神人各處其所而不相奸|{
	神謂川凟之神人謂居人也處昌呂翻師古曰奸音干}
且大漢方制萬里豈其與水争咫尺之地哉此功一立河定民安千載無患故謂之上策|{
	載子亥翻}
若乃多穿漕渠於冀州地使民得以溉田分殺水怒|{
	殺所介翻减也}
雖非聖人法然亦救敗術也可從淇口以東為石隄|{
	地理志淇水出河内共縣北山東至黎陽入河水經注曰魏晉之枋頭古淇口也共音恭}
多張水門恐議者疑河大川難禁制滎陽漕渠足以卜之|{
	如淳曰今礫谿口是也言作水門流水流不為害也師古曰礫谿谿名即水經所云濟水東過礫谿者}
冀州渠首盡當仰此水門|{
	仰牛向翻}
諸渠皆往往股引取之|{
	如淳曰肢支别也㨿如說股當作肢}
旱則開東方下水門溉冀州水則開西方高門分河流民田適治河隄亦成此誠富國安民興利除害支數百歲故謂之中策若乃繕完故隄增卑倍薄勞費無已數逢其害此最下策也|{
	讓所畫治河三策自漢至今未有能行之者大率古人論事畫為三策者其上策多孟浪駭俗而難行其中策則平實合宜而可用其下策則常人所知也數所角翻}
孔光何武奏迭毁之次當以時定|{
	自元帝時貢禹建毁廟之議韋玄成匡衡皆踵其說以為太祖以下五廟其親廟四親盡而迭毁迄於成帝終莫能定今二府復奏}
請與羣臣雜議於是光禄勲彭宣等五十三人皆以為孝武皇帝雖有功烈親盡宜毁太僕王舜中壘校尉劉歆議曰禮天子七廟七者其正法數可常數者也|{
	禮記曰天子七廟三昭三穆與太祖廟而七}
宗不在此數中宗變也|{
	師古曰言非當數故云變也}
苟有功德則宗之不可預為設數臣愚以為孝武皇帝功烈如彼孝宣皇帝崇立之如此不宜毁|{
	立世宗廟見二十四卷宣帝本始元年}
上覽其議制曰太僕舜中壘校尉歆議可 何武後母在蜀郡|{
	武蜀郡郫縣人}
遣吏歸迎會成帝崩吏恐道路有盗賊後母留止|{
	止不行也}
左右或譏武事親不篤|{
	師古曰左右謂天子側近之臣}
帝亦欲改易大臣冬十月策免武以列侯歸國癸酉以師丹為大司空丹見上多所匡改成帝之政乃上書言古者諒闇不言聽於冢宰|{
	師古曰論語云子張曰書云高宗諒闇三年不言孔子曰何必高宗古之人皆然君薨百官總已以聽于冢宰三年諒信也闇默然也鄭玄曰周之六官皆摠屬於冢宰冢宰於百官無所不主爾雅曰冢大也冢宰太宰也乃上時掌翻}
三年無改於父之道|{
	師古曰論語稱孔子曰父在觀其志父沒觀其行三年無改於父之道可謂孝矣}
前大行屍柩在堂而官爵臣等以及親屬赫然皆貴寵封舅為陽安侯皇后尊號未定豫封父為孔鄉侯出侍中王邑射聲校尉王邯等|{
	王邑王邯太皇太后親屬也邯戶甘翻}
詔書比下變動政事卒暴無漸|{
	師古曰比類也比毗至翻卒讀曰猝下倉卒同}
臣縱不能明陳大義復曾不能牢讓爵位|{
	師古曰牢堅也復扶又翻曾才登翻}
相隨空受封侯增益陛下之過閒者郡國多地動水出流殺人民日月不明五星失行此皆舉錯失中號令不定法度失理隂陽溷濁之應也|{
	錯千故翻師古曰溷音胡頓翻}
臣伏惟人情無子年雖六七十猶博取而廣求|{
	師古曰取讀曰娶}
孝成皇帝深見天命燭知至德|{
	師古曰燭照也至德指謂哀帝}
以壮年克己|{
	已者有我之私克去也}
立陛下為嗣先帝暴棄天下|{
	暴者言無疾而崩}
而陛下繼體四海安寧百姓不懼此先帝聖德當合天人之功也臣聞天威不違顔咫尺|{
	左傳齊桓公對宰孔之言師古曰言常若在前宜自肅懼也}
願陛下深思先帝所以建立陛下之意且克己躬行以觀羣下之從化天下者陛下之家也胏附何患不富貴不宜倉卒若是其不久長矣丹書數十上|{
	上時掌翻}
多切直之言傅太后從弟子遷在左右尤傾邪|{
	從才用翻}
上惡之|{
	惡烏路翻}
免官遣歸故郡|{
	傅氏本河内温人}
傅太后怒上不得已復留遷|{
	復扶又翻下同}
丞相光與大司空丹奏言詔書前後相反天下疑惑無所取信臣請歸遷故郡以銷姦黨卒不得遣|{
	卒子恤翻終也}
復為侍中其逼於傅太后皆此類也|{
	哀帝之時傅氏固為驕横然史家所記如此等語意其出於王氏愛憎之口}
議郎耿育上書寃訟陳湯|{
	成帝永始二年陳湯徙邊寃訟訟其寃也}
曰甘延夀陳湯為聖漢揚鈎深致遠之威|{
	言湯等深入康居遠誅郅支雖其竄伏荒外能揚威而鉤致之也為于偽翻}
雪國家累年之恥討絶域不羈之君|{
	不羈言不可羈屬也}
係萬里難制之虜豈有比哉先帝嘉之仍下明詔宣著其功改年垂歷|{
	師古曰謂改年為竟寧也不以此事蓋當其年上書者附著耳余按元紀詔曰匈奴郅支單于背叛禮義既服其辜呼韓邪單于修朝保塞邊垂長無兵革之事其改元為竟寧則改年亦以此事非附著也}
傳之無窮應是南郡獻白虎|{
	白虎西方之獸主威武故以為湯等之應}
邊垂無警備會先帝寑疾然猶垂意不忘數使尚書責問丞相趣立其功|{
	數所角翻趣使丞相御史立議以序其功也師古曰趣讀曰促}
獨丞相匡衡排而不予|{
	予讀曰與}
封延夀湯數百戶|{
	事見二十九卷元帝竟寧元年}
此功臣戰士所以失望也孝成皇帝承建業之基乘征伐之威兵革不動國家無事而大臣傾邪欲專主威排妒有功|{
	妒與妬同}
使湯塊然被見拘囚|{
	師古曰塊然獨處之意如土塊也塊音口内翻被皮義翻}
不能自明卒以無罪老棄燉煌正當西域通道|{
	通道通行之路也卒子恤翻燉徒門翻}
令威名折衝之臣旋踵及身|{
	謂罪及其身也}
復為郅支遺虜所笑誠可悲也|{
	復扶又翻下同}
至今奉使外蠻者未嘗不陳郅支之誅以揚漢國之盛夫援人之功以懼敵|{
	師古曰援引也音爰}
棄人之身以快讒豈不痛哉且安不忘危盛必慮衰今國家素無文帝累年節儉富饒之畜又無武帝薦延|{
	畜讀與蓄同如淳曰薦延使羣臣薦士而延納之}
梟俊禽敵之臣獨有一陳湯耳|{
	師古曰梟謂斬其首而縣之也俊謂敵之魁率郅支是也春秋左氏傳曰得俊曰克仲馮曰梟俊禽敵之臣宜與薦延通為一句則與上文相配而下言獨有一陳湯耳自不妨梟善鬬故云梟俊猶云梟將也梟堅亮翻}
假使異世不及陛下尚望國家追録其功封表其墓以勸後進也湯幸得身當聖世功曾未久|{
	曾才登翻}
反聽邪臣鞭逐斥遠|{
	遠于願翻}
使亡逃分竄死無處所|{
	師古曰分謂散離也舜典曰分北三苗}
遠覽之士莫不計度以為湯功累世不可及而湯過人情所有|{
	師古曰言湯所犯之罪過人情共有不能免者非特詭異深可誅責也度徒洛翻}
湯尚如此雖復破絶筋骨暴露形骸猶復制於脣舌為嫉妒之臣所係虜耳|{
	言湯功如此之偉猶不免於罪徙繼今者雖復捐身為國終制於吏議䧟於係虜之罪也復扶又翻}
此臣所以為國家尤戚戚也|{
	為于偽翻}
書奏天子還湯卒於長安|{
	卒子恤翻}


孝哀皇帝上|{
	諱欣定陶恭王康之子也成帝立以為嗣荀悦曰諱欣之字曰喜應劭曰諡法恭仁短折曰哀}


建平元年春正月隕石于北地十六 赦天下 司隸校尉解光奏言臣聞許美人及故中宫史曹宫|{
	史女史也中宫皇后宫也趙皇后傳宫屬中宫為學事史通詩授皇后}
皆御幸孝成皇帝產子子隱不見|{
	見賢遍翻}
臣遣吏驗問皆得其狀元延元年宫有身其十月宫乳掖庭牛官令舍|{
	師古曰乳產也音而具翻下皆類此}
中黄門田客|{
	續漢志中黄門比百石掌給事禁中以宦者為之}
持詔記與掖庭獄丞籍武令收置暴室獄|{
	掖庭令屬少府有左右丞暴室丞各一人皆宦者為之暴室丞主中婦人疾病者就此室其皇后貴人有罪亦就此室籍姓晉大夫籍氏之後其先有孫伯黶司晉之典籍以為大政故曰籍氏}
母問兒男女誰兒也宫曰善臧我兒胞|{
	臧古藏字通師古曰胞謂胎之衣也音苞}
丞知是何等兒也|{
	師古曰意言是天子兒耳}
後三日客持詔記與武問兒死未武對未死客曰上與昭儀大怒|{
	趙昭儀也}
奈何不殺武叩頭啼曰不殺兒自知當死殺之亦死|{
	不殺則為違詔命故知當死殺之則後人以害皇子之罪加之故知亦死}
即因客奏封事曰陛下未有繼嗣子無貴賤唯留意奏入客復持詔記取兒付中黄門王舜舜受詔内兒殿中為擇乳母|{
	復扶又翻為于偽翻}
吿善養兒且有賞毋令漏泄舜擇官婢張棄為乳母|{
	官婢蓋以罪沒入掖庭男為官奴女為官婢鄭玄曰古者從坐男女沒入縣官為奴其少才知以為奚今之侍史官婢或謂之奚官女}
後三日客復持詔記并藥以飲宫|{
	師古曰飲音於禁翻}
宫曰果也欲姊弟擅天下我兒男也頟上有壮髪類孝元皇帝|{
	師古曰壮髪當頟前侵下而生今俗呼為圭頭者是也頟鄂格翻}
今兒安在危殺之矣|{
	師古曰危險也猶今人言險不殺耳}
奈何令長信得聞之|{
	師古曰長信謂太后}
遂飲藥死棄所養兒|{
	師古曰棄謂張棄也}
十一月宫長李南以詔書取兒去|{
	晉灼曰漢儀注有女長御比侍中宫長豈此邪余謂宫長者盖老于宫中諸女御因稱之為宫長猶三署諸郎謂久次者為郎署長也前持詔記此以詔書書之與記有以異乎曰有詔記手記也後世謂之手記光武所謂詔書手記不可數得手記出于上手詔書則下為之以璽為信長知兩翻}
不知所置|{
	師古曰終竟不知置何所也}
許美人元延二年懷子十一月乳|{
	乳如注翻㝃乳也}
昭儀謂帝曰常紿我言從中宫來即從中宫來許美人兒何從生中許氏竟當復立邪|{
	晉灼曰昭儀前要帝不得立許美人以為皇后而今有子中許氏竟當復立為皇后邪此前約之言也師古曰此說非也言美人在内中何從得兒而生也故言何從生中次此下乃始言約耳}
懟以手自擣|{
	師古曰懟怨怒也擣築也懟音直類翻}
以頭擊壁戶柱從牀上自投地啼泣不肯食曰今當安置我我欲歸耳帝曰今故告之反怒為|{
	師古曰故以許美人生子告汝何為反怒}
殊不可曉也|{
	殊異甚也}
帝亦不食昭儀曰陛下自知是不食何為陛下嘗自言約不負女|{
	師古曰女讀曰汝}
今美人有子竟負約謂何帝曰約以趙氏故不立許氏使天下無出趙氏上者毋憂也後詔使中黄門靳嚴從許美人取兒去盛以葦箧|{
	靳居翻盛時征翻葦葭類也織以為箧也}
置飾室簾南去|{
	飾室室之以金玉為飾者昭陽舍是也師古曰簾戶簾也音亷}
帝與昭儀坐使御者于客子解箧緘未巳|{
	御者侍者也師古曰緘束箧之繩音古咸翻}
帝使客子及御者皆出自閉戶獨與昭儀在須臾開戶嘑客子|{
	嘑古呼字}
使緘封箧及詔記令中黄門吴恭持以與籍武曰告武箧中有死兒埋屏處|{
	屏處有遮蔽處人所不見者屏必郢翻}
勿令人知武穿獄樓垣下為坎埋其中|{
	獄掖庭獄也}
其它飲藥傷墯者無數事皆在四月丙辰赦令前 |{
	考異曰趙后傳作丙辰按哀帝紀四月丙午即位赦天下盖傳誤也或者即位後十日赦也}
臣謹案永光三年男子忠等發長陵傅夫人冢事更大赦|{
	更工衡翻}
孝元皇帝下詔曰此朕所不當得赦也窮治盡伏辜天下以為當|{
	當丁浪翻}
趙昭儀傾亂聖朝親滅繼嗣家屬當伏天誅而同產親屬皆在尊貴之位廹近帷幄|{
	近其靳翻}
天下寒心請事窮竟|{
	謂窮治其獄而竟其情}
丞相以下議正法|{
	令外朝大議以正其罪}
帝於是免新成侯趙欽欽兄子成陽侯訢皆為庶人|{
	訢臨之子也}
將家屬徙遼西郡議郎耿育上疏言臣聞繼嗣失統廢適立庶|{
	師古曰適讀曰嫡下同}
聖人法禁古今至戒然太伯見歷知適|{
	師古曰歷謂王季即文王之父也知適謂知其當為適嗣也適丁歷翻}
逡循固讓委身吴粤|{
	謂太伯逃之吴粤以避季歷}
權變所設不計常法致位王季以崇聖嗣|{
	聖嗣謂文王}
卒有天下|{
	師古曰卒終也卒子恤翻}
子孫承業七八百載|{
	載子亥翻年也爾雅曰唐虞曰載取物終更始}
功冠三王|{
	冠古玩翻}
道德最備是以尊號追及太王|{
	太王古公亶父也武王克商有天下追王太王王季}
故世必有非常之變然後乃有非常之謀孝成皇帝自知繼嗣不以時立念雖未有皇子萬歲之後末能持國|{
	師古曰末晚暮也萬歲言晏駕也余謂人之生也以死為諱故常人以死後為百年之後天子曰千秋萬歲後}
權柄之重制於女主女主驕盛則耆欲無極|{
	如武帝為鉤弋夫人慮者是也師古曰耆讀曰嗜}
少主幼弱則大臣不使|{
	師古曰不使不可使從命也}
世無周公抱負之輔恐危社稷傾亂天下知陛下有賢聖通明之德仁孝子愛之恩懷獨見之明内斷於身|{
	斷丁亂翻}
故廢後宫就館之漸絶微嗣禍亂之根|{
	師古曰微嗣者謂幼主也}
乃欲致位陛下以安宗廟愚臣既不能深援安危定金櫃之計|{
	師古曰愚臣謂解光等也金匱言長久之法可藏于金匱石室者援音爰}
又不知推演聖德|{
	師古曰演廣也音弋善翻}
述先帝之志乃反覆校省内暴露私燕|{
	師古曰私燕謂成帝閑燕之私也覆音方目翻余謂私燕袵席之私所謂專房燕即此燕也}
誣汙先帝傾惑之過|{
	汙烏故翻}
成結寵妾妬媢之誅|{
	媢莫報翻}
甚失聖賢遠見之明逆負先帝憂國之意夫論大德不拘俗立大功不合衆|{
	用衛鞅語意}
此乃孝成皇帝至思所以萬萬於衆臣陛下聖德盛茂所以符合於皇天也豈當世庸庸斗筲之臣所能及哉|{
	筲竹器也容斗二升音所交翻}
且褒廣將順君父之美匡救銷滅既往之過古今通義也事不當時固争防禍於未然各隨指阿從以求容媚晏駕之後尊號已定萬事已訖乃探追不及之事訐揚幽昩之過此臣所深痛也|{
	耿育之言是也春秋為尊者諱義正如此探吐南翻師古曰訐音居謁翻}
願下有司議|{
	下遐稼翻}
即如臣言宜宣布天下使咸曉知先帝聖意所起不然空使謗議上及山陵下流後世遠聞百蠻|{
	聞音問}
近布海内甚非先帝託後之意也盖孝者善述父之志善成人之事唯陛下省察|{
	省悉井翻}
帝亦以為太子頗得趙太后力|{
	事見上卷成帝元延四年}
遂不竟其事傅太后恩趙太后|{
	師古曰恩謂以厚恩接遇之一曰恩謂銜其立哀帝為嗣之恩也余謂一說是}
趙太后亦歸心故太皇太后及王氏皆怨之|{
	為趙后自殺張本}
丁酉光禄大夫傅喜為大司馬封高武侯|{
	恩澤侯表高武侯國於南陽杜衍縣考異曰公卿表綏和二年十一月庚午師丹為大司馬四月徙建平元年四月丁酉傅喜為大司馬喜傳云明年正月徙師丹為大司空而拜喜為大司馬荀紀亦在正月按長歷此年四月癸亥朔無丁酉今從喜傳漢紀}
秋九月甲辰隕石于虞二|{
	地理志虞縣屬梁國}
郎中令泠襃|{
	師古曰泠音零古者樂工謂之泠人因以為氏周有泠州鳩原父曰按此時無郎中令余謂令字衍}
黄門郎段猶等復奏言定陶共王太后共皇后皆不宜復引定陶藩國之名以冠大號|{
	復扶又翻共讀曰恭冠古玩翻}
車馬衣服宜皆稱皇之意|{
	師古曰皇者至尊之號其服御宜皆副稱之也稱音尺孕翻}
置吏二千石以下各供厥職|{
	師古曰謂詹事太僕少府等衆官也}
又宜為共皇立廟京師|{
	為于偽翻下故為同}
上復下其議|{
	復扶又翻下遐稼翻}
羣下多順指言母以子貴宜立尊號以厚孝道唯丞相光大司馬喜大司空丹以為不可丹曰聖王制禮取法於天地尊卑者所以正天地之位不可亂也|{
	易繫辭曰天尊地卑君臣定矣卑高以陳貴賤位矣又履卦大象曰上天下澤履君子以辯上下定民志履者禮也}
今定陶共皇太后共皇后以定陶共為號者母從子妻從夫之義也|{
	共皇太后之號為母從子共皇后之號為妻從夫}
欲立官置吏車服與太皇太后並非所以明尊無二上之義定陶共皇號諡已前定義不得復改|{
	復扶又翻}
禮父為士子為天子祭以天子其尸服以士服子無爵父之義尊父母也|{
	引禮記喪服小記之言古者祭祀必有尸服以生時之服事亡如事存也鄭玄曰祭以天子養以子道也尸服士服父本無爵不敢以已爵加之嫌于卑之}
為人後者為之子故為所後服斬衰三年|{
	斬衰用麤布其下斬之不緶衰音七雷翻}
而降其父母期明尊本祖而重正統也|{
	本祖所後之祖}
孝成皇帝聖恩深遠故為共王立後|{
	事見上卷成帝綏和元年}
奉承祭祀令共皇長為一國太祖|{
	前稱共王後稱共皇隨其時之所稱而稱之也諸侯之國以始封之君為國太祖}
萬世不毁恩義已備陛下既繼體先帝持重大宗承宗廟天地社稷之祀義不可復奉定陶共皇祭入其廟|{
	復扶又翻下同}
今欲立廟於京師使臣下祭之是無主也又親盡當毁|{
	禮太祖以下親廟四親盡而迭毁匡衡曰孝莫大於嚴父故父之所尊子不敢不承父之所異子不敢同禮公子不得為母信為後則於子祭於孫止李奇曰不得信尊其父也公子去其所而為大宗後尚得私祭其母為孫則止不得祭公子母也明繼祖不復顧其私祖母此皆親盡當毁之義也師古曰信讀曰申}
空去一國太祖不墮之祀而就無主當毁不正之義|{
	共皇立廟于定陶則為一國太祖之廟萬世不毁立廟于京師則其祭莫適為主又親盡當毁而於禮又為不正也墮讀曰隳}
非所以尊厚共皇也丹由是浸不合上意會有上書言古者以龜貝為貨今以錢易之|{
	貝博盖翻海介蟲也居陸名贆在水名蜬古者貨貝而寶龜周有泉至秦廢貝而行錢其後王莽以龜貝為貨盖祖此說也埤雅獸為友貝二為朋貝中肉如科斗而有首尾貝之字從目從八言貝目之所背也鹽鐵論曰教與俗改敝與世易夏后氏以玄貝殷人以紫石孔頴達曰爾雅貝居陸猋在水蜬大者□小者鰿今之細貝亦有紫色者出日南玄貝胎貝黑色者餘蚳黄白文餘泉白黄文白質黄文也詩成貝錦則紫貝也紫貝以紫為質黑為文點也蚆博而頯中廣兩頭鋭蜠大而儉小而惰惰狭而長贆音標蜬音含□音况鰿音積蚳音治蚆音葩頯匡軌翻蜠音囷}
民以故貧宜可改幣上以問丹丹對言可改章下有司議|{
	下遐稼翻下同}
皆以為行錢以來久難卒變易|{
	師古曰卒讀曰猝}
丹老人忘其前語|{
	年老神識衰減則健忘忘音巫放翻}
復從公卿議又丹使吏書奏吏私寫其草丁傅子弟聞之使人上書告丹上封事|{
	上時掌翻}
行道人徧持其書上以問將軍中朝臣|{
	朝直遥翻}
皆對曰忠臣不顯諫大臣奏事不宜漏泄宜下廷尉治|{
	下遐稼翻}
事下廷尉劾丹大不敬|{
	承丁傅風旨也劾戶槩翻}
事未决給事中博士申咸炔欽上書|{
	蘇林曰炔音桂姓也}
言丹經行無比|{
	行下孟翻師古曰比音必寐翻余謂讀如字義自通}
自近世大臣能若丹者少發憤懣奏封事|{
	懣音滿又莫困翻}
不及深思遠慮使主簿書|{
	漢三公府皆有主簿録省衆事簿文籍也以板書之}
漏泄之過不在丹以此貶黜恐不厭衆心|{
	師古曰厭音一贍翻}
上貶咸欽秩各二等|{
	博士秩比六百石貶二等則比四百石}
遂策免丹曰朕惟君位尊任重懷諼迷國|{
	師古曰諼詐也音虚爰翻}
進退違命反覆異言甚為君恥之|{
	為于偽翻下為賢同}
以君嘗托傅位|{
	謂嘗傅上於東宫也}
未忍考于理|{
	理理官也謂廷尉也言未召致廷尉而考問之也}
其上大司空高樂侯印綬罷掃|{
	上時掌翻下同}
尚書令唐林上疏曰竊見免大司空丹策書泰深痛切君子作文為賢者諱|{
	春秋之義為賢者諱}
丹經為世儒宗|{
	言經學為當世儒者所宗也}
德為國黄耉|{
	師古曰黄耉老人之稱也黄謂白髪落盡更生黄者也耉老人面色不浄如垢也}
親傅聖躬位在三公所坐者微海内未見其大過事既以往|{
	丹傳以作已}
免爵太重京師識者咸以為宜復丹爵邑使奉朝請|{
	朝直遥翻師古曰識者謂有識之人也請音材性翻成帝尊禮張禹使奉朝請後遂以為官名沈約曰奉朝會請召而已請讀如字}
唯陛下裁覽衆心有以尉復師傅之臣|{
	尉與慰同安也復報也}
上從林言下詔賜丹爵關侯|{
	自蕭望之以讒間免官賜爵關内侯其後周堪等皆用此比雖曰以恩師傅其實倚閣之使之優閑耳}
上用杜業之言召見朱博起家復為光禄大夫|{
	朱博免官見上卷成帝綏和元年按杜業傳帝初即位業上書言王氏世權日久薛宣張禹惑亂朝廷而薦朱博見賢遍翻}
遷京兆尹冬十月壬午以博為大司空 中山王箕子幼有眚病|{
	箕子中山王興之子孟康曰災眚之眚謂妖病也服䖍曰身盡眚也蘇林曰名為肝厥發時唇口手足十指甲皆眚師古曰下云檮祠解舍孟說是也眚音所領翻字不作眚服說誤矣}
祖母馮太后自養視數禱祠解|{
	數所角翻師古曰解音懈余按韻書解音懈者釋除也禱祠以除災也但顔注上云禱祠解舍則以解為廨舍之廨其說拘矣賈公彦曰求福曰禱禱禮輕得求曰祠祠禮重}
上遣中郎謁者張由將醫治之|{
	續漢志常侍謁者主殿上時節威儀比六百石給事謁者四百石灌謁者郎中比三百石掌賓贊受事及上章報問中郎謁者盖即灌謁者郎中也治直之翻下同}
由素有狂易病|{
	師古曰狂易者狂而變易常性也}
病發怒去西歸長安尚書簿責由擅去狀|{
	師古曰簿責以文簿一一責問也}
由恐因誣言中山太后祝詛上及傅太后|{
	中山太后馮太后也即元帝馮昭儀祝職救翻詛莊助翻}
傅太后與馮太后並事元帝追怨之因是遣御史丁玄案驗數十日無所得更使中謁者令史立治之|{
	師古曰官為中謁者令姓史名立續漢志中官謁者令主報中章宦者為之更工衡翻}
立受傅太后指冀得封侯治馮太后女弟習及弟婦君之|{
	據馮昭儀傳君之寡弟婦也}
死者數十人誣奏云祝詛謀弑上立中山王責問馮太后無服辭立曰熊之上殿何其勇今何怯也|{
	當熊事見二十九卷元帝建昭元年之上時掌翻}
太后還謂左右此乃中語|{
	師古曰中語謂宫中之言語也}
吏何用知之欲䧟我效也|{
	師古曰效徵驗也}
乃飲藥自殺宜鄉侯參君之習及夫子|{
	按馮昭儀傳習夫及子也}
當相坐者或自殺或伏法|{
	伏法謂受刑而死}
凡死者十七人衆莫不憐之司隸孫寶奏請覆治馮氏獄傅太后大怒曰帝置司隸主使察我馮氏反事明白故欲擿抉以揚我惡|{
	師古曰剔抉謂挑發之也擿音他歷翻抉音一决翻挑音他聊翻}
我當坐之上乃順指下寶獄|{
	下遐稼翻}
尚書僕射唐林争之上以林朋黨比周|{
	比毗至翻}
左遷燉煌魚澤障侯|{
	師古曰燉煌效穀縣本魚澤障也}
大司馬傅喜光禄大夫龔勝固争上為言太后出寶復官|{
	為于偽翻}
張由以先告賜爵關内侯史立遷中大僕|{
	張由史立以此受賞豈知乃以此賈禍邪}


資治通鑑卷三十三














































































































































