<!DOCTYPE html PUBLIC "-//W3C//DTD XHTML 1.0 Transitional//EN" "http://www.w3.org/TR/xhtml1/DTD/xhtml1-transitional.dtd">
<html xmlns="http://www.w3.org/1999/xhtml">
<head>
<meta http-equiv="Content-Type" content="text/html; charset=utf-8" />
<meta http-equiv="X-UA-Compatible" content="IE=Edge,chrome=1">
<title>資治通鑒_27-資治通鑑卷二十六_27-資治通鑑卷二十六</title>
<meta name="Keywords" content="資治通鑒_27-資治通鑑卷二十六_27-資治通鑑卷二十六">
<meta name="Description" content="資治通鑒_27-資治通鑑卷二十六_27-資治通鑑卷二十六">
<meta http-equiv="Cache-Control" content="no-transform" />
<meta http-equiv="Cache-Control" content="no-siteapp" />
<link href="/img/style.css" rel="stylesheet" type="text/css" />
<script src="/img/m.js?2020"></script> 
</head>
<body>
 <div class="ClassNavi">
<a  href="/24shi/">二十四史</a> | <a href="/SiKuQuanShu/">四库全书</a> | <a href="http://www.guoxuedashi.com/gjtsjc/"><font  color="#FF0000">古今图书集成</font></a> | <a href="/renwu/">历史人物</a> | <a href="/ShuoWenJieZi/"><font  color="#FF0000">说文解字</a></font> | <a href="/chengyu/">成语词典</a> | <a  target="_blank"  href="http://www.guoxuedashi.com/jgwhj/"><font  color="#FF0000">甲骨文合集</font></a> | <a href="/yzjwjc/"><font  color="#FF0000">殷周金文集成</font></a> | <a href="/xiangxingzi/"><font color="#0000FF">象形字典</font></a> | <a href="/13jing/"><font  color="#FF0000">十三经索引</font></a> | <a href="/zixing/"><font  color="#FF0000">字体转换器</font></a> | <a href="/zidian/xz/"><font color="#0000FF">篆书识别</font></a> | <a href="/jinfanyi/">近义反义词</a> | <a href="/duilian/">对联大全</a> | <a href="/jiapu/"><font  color="#0000FF">家谱族谱查询</font></a> | <a href="http://www.guoxuemi.com/hafo/" target="_blank" ><font color="#FF0000">哈佛古籍</font></a> 
</div>

 <!-- 头部导航开始 -->
<div class="w1180 head clearfix">
  <div class="head_logo l"><a title="国学大师官网" href="http://www.guoxuedashi.com" target="_blank"></a></div>
  <div class="head_sr l">
  <div id="head1">
  
  <a href="http://www.guoxuedashi.com/zidian/bujian/" target="_blank" ><img src="http://www.guoxuedashi.com/img/top1.gif" width="88" height="60" border="0" title="部件查字,支持20万汉字"></a>


<a href="http://www.guoxuedashi.com/help/yingpan.php" target="_blank"><img src="http://www.guoxuedashi.com/img/top230.gif" width="600" height="62" border="0" ></a>


  </div>
  <div id="head3"><a href="javascript:" onClick="javascript:window.external.AddFavorite(window.location.href,document.title);">添加收藏</a>
  <br><a href="/help/setie.php">搜索引擎</a>
  <br><a href="/help/zanzhu.php">赞助本站</a></div>
  <div id="head2">
 <a href="http://www.guoxuemi.com/" target="_blank"><img src="http://www.guoxuedashi.com/img/guoxuemi.gif" width="95" height="62" border="0" style="margin-left:2px;" title="国学迷"></a>
  

  </div>
</div>
  <div class="clear"></div>
  <div class="head_nav">
  <p><a href="/">首页</a> | <a href="/ShuKu/">国学书库</a> | <a href="/guji/">影印古籍</a> | <a href="/shici/">诗词宝典</a> | <a   href="/SiKuQuanShu/gxjx.php">精选</a> <b>|</b> <a href="/zidian/">汉语字典</a> | <a href="/hydcd/">汉语词典</a> | <a href="http://www.guoxuedashi.com/zidian/bujian/"><font  color="#CC0066">部件查字</font></a> | <a href="http://www.sfds.cn/"><font  color="#CC0066">书法大师</font></a> | <a href="/jgwhj/">甲骨文</a> <b>|</b> <a href="/b/4/"><font  color="#CC0066">解密</font></a> | <a href="/renwu/">历史人物</a> | <a href="/diangu/">历史典故</a> | <a href="/xingshi/">姓氏</a> | <a href="/minzu/">民族</a> <b>|</b> <a href="/mz/"><font  color="#CC0066">世界名著</font></a> | <a href="/download/">软件下载</a>
</p>
<p><a href="/b/"><font  color="#CC0066">历史</font></a> | <a href="http://skqs.guoxuedashi.com/" target="_blank">四库全书</a> |  <a href="http://www.guoxuedashi.com/search/" target="_blank"><font  color="#CC0066">全文检索</font></a> | <a href="http://www.guoxuedashi.com/shumu/">古籍书目</a> | <a   href="/24shi/">正史</a> <b>|</b> <a href="/chengyu/">成语词典</a> | <a href="/kangxi/" title="康熙字典">康熙字典</a> | <a href="/ShuoWenJieZi/">说文解字</a> | <a href="/zixing/yanbian/">字形演变</a> | <a href="/yzjwjc/">金 文</a> <b>|</b>  <a href="/shijian/nian-hao/">年号</a> | <a href="/diming/">历史地名</a> | <a href="/shijian/">历史事件</a> | <a href="/guanzhi/">官职</a> | <a href="/lishi/">知识</a> <b>|</b> <a href="/zhongyi/">中医中药</a> | <a href="http://www.guoxuedashi.com/forum/">留言反馈</a>
</p>
  </div>
</div>
<!-- 头部导航END --> 
<!-- 内容区开始 --> 
<div class="w1180 clearfix">
  <div class="info l">
   
<div class="clearfix" style="background:#f5faff;">
<script src='http://www.guoxuedashi.com/img/headersou.js'></script>

</div>
  <div class="info_tree"><a href="http://www.guoxuedashi.com">首页</a> > <a href="/SiKuQuanShu/fanti/">四库全书</a>
 > <h1>资治通鉴</h1> <!--         下载:【右键另存为】即可 --></div>
  <div class="info_content zj clearfix">
  
<div class="info_txt clearfix" id="show">
<center style="font-size:24px;">27-資治通鑑卷二十六</center>
    資治通鑑卷二十六   宋 司馬光 撰<br />
<br />
  胡三省 音註<br />
<br />
  漢紀十八【起上章涒灘盡玄黓閹茂凡三年】<br />
<br />
  中宗孝宣皇帝中<br />
<br />
  神爵元年【以神爵降集紀元】春正月上始行幸甘泉郊泰畤三月行幸河東祠后土上頗修武帝故事謹齋祀之禮以方士言增置神祠【時以方士言為隨侯劍寶玉寶璧周康寶鼎立四祠於未央宫中又祠大室山于即墨三戶山于下密祠天封苑火井于鴻門又立歲星辰星太白熒惑南斗祠于長安城旁又祠參山八神于曲城蓬山石社石鼓于臨朐之罘山於腄成山於不夜莱山于黄成山祠日莱山祠月又祠四時于琅邪蚩尤于夀良京師近縣鄠則有勞谷五牀山日月五帝仙人玉女祠雲陽有徑路神祠又立五龍山仙人祠及黄帝天神帝原水凡四祠於膚施】聞益州有金馬碧雞之神可醮祭而致【後漢志越嶲郡青蛉縣禺同山俗謂有金馬碧雞如淳曰金形似馬碧形似鷄水經注曰禺同山神有金馬碧雞光景儵忽醮即召翻】於是遣諫大夫蜀郡王褒使持節而求之【使疏吏翻】初上聞褒有俊才召見【見賢遍翻】使為聖主得賢臣頌其辭曰夫賢者國家之器用也所任賢則趨舍省而【讀曰趣普博也趨七喻翻舍讀曰捨施式智翻】器用利則用力少而就效衆故工人之用鈍器也勞筋苦骨終日矻矻【應劭曰矻矻勞極貌如淳曰健作貌師古曰如說是也矻口骨翻】及至巧冶鑄干將【干將吳寶劒名闔廬所鑄】使離婁督繩公輸削墨【張晏曰離婁黄帝時明目者也應劭曰公輸魯般性巧者也師古曰督察視也】雖崇臺五層延袤百丈而不溷者工用相得也【師古曰溷亂也音胡頓翻】庸人之御駑馬亦傷吻敝策而不進於行【師古曰吻口角也策所以撃馬】及至駕齧厀驂乘旦【孟康曰良馬低頭口至䣛故曰齧䣛張晏曰駕則旦至故曰乘曰乘食證翻】王良執靶【張晏曰王良郵無恤字伯樂晉灼曰靶音霸謂轡也師古曰參驗左氏傳及國語孟子郵無恤郵良劉無止王良總一人也楚辭云驥躊躇於敝輦遇孫陽而得代王逸云孫陽伯樂姓名也列子云伯樂秦穆公時人考其年代不相當張說云良字伯樂斯失之矣】韓哀附輿【應劭曰世本韓哀作御師古曰宋衷云韓哀韓哀侯也時已有御此復言作者加其精巧也然則善御者耳非始作也】周流八極萬里一息何其遼哉人馬相得也故服絺綌之凉者不苦盛暑之鬱燠襲貂狐之煗者不憂至寒之悽愴【師古曰鬱熱氣也燠温也悽愴寒冷也燠於六翻煗乃短翻】何則有其具者易其備賢人君子亦聖王之所以易海内也【易以豉翻】昔周公躬吐捉之勞故有圉空之隆【師古曰一飯三吐食一沐三捉髪以賓賢士故能成太平之化而刑措不用故囹圄空虚也圉音圄同】齊桓設庭燎之禮故有匡合之功【應劭曰有以九九求見桓公桓公不内其人曰九九小術而君不内之况大于九九者乎于是桓公設庭燎之禮而見之居無幾隰朋自遠而至齊遂以霸師古曰九九計數之書若今筭經也匡謂一匡天下合謂九合諸侯】由此觀之君人者勤於求賢而逸於得人人臣亦然昔賢者之未遭遇也圖事揆策則君不用其謀陳見悃誠【王逸曰悃愊志純一也亦猶實也】則上不然其信進仕不得施効斥逐又非其愆是故伊尹勤於鼎俎太公困於鼔刀【師古曰勒于鼎俎謂負鼎俎以干湯也鼓刀者謂太公屠牛于朝歌也】百里自鬻甯子飯牛【師古曰鬻賣也呂氏春秋曰百里奚之未遇時也虞亡而虜縛鬻以五羊之皮公孫枝得而悅之獻諸穆公應劭曰齊桓公夜出迎客甯戚疾擊其牛角高歌曰南山矸白石爛生不逢堯與舜禪短布單衣適至骭從昏飯牛薄夜半長夜曼曼何時旦桓公乃召與語悅之以為大夫飯扶晚翻】離此患也【師古曰離遭也】及其遇明君遭聖主也運籌合上意諫諍即見聽進退得關其忠任職得行其術剖符錫壤而光祖考故世必有聖知之君【知讀曰智】而後有賢明之臣故虎嘯而風冽【師古曰冽冽風貌也音列】龍興而致雲蟋蟀竢秋唫蜉蝤出以隂【孟康曰蜉蝤渠畧也師古曰蟋蟀今之促織也蜉蝤甲蟲也好叢聚而生也朝生而夕死舍人曰南陽以東曰蜉蝤梁宋之間曰渠畧郭璞曰似蛣蜣身狹而長有角黄黑色聚生糞土中朝生暮死猪好噉之陸機疏云蜉蝣有角大如指長三四寸甲下有翅能飛夏月隂雨時地中出竢即俟字蝤音由】易曰飛龍在天利見大人【師古曰乾卦九五爻辭也言王者居正陽之位賢才見之則利用也】詩曰思皇多士生此王國【師古曰大雅文王之詩也思語辭也皇美也言美哉衆多賢士生此周王之國也】故世平主聖俊艾將自至【師古曰艾讀曰乂】明明在朝穆穆布列聚精會神相得益章【師古曰章明也】雖伯牙操遞鍾【晉灼曰遞音遞送之遞二十四鍾各有節奏擊之不常故曰遞臣瓚曰楚辭云奏伯牙之號鍾號鍾琴名也馬融笛賦曰號鍾高調伯牙以善鼓琴不聞其能擊鍾也師古曰琴名是也字既作遞則與楚辭不同不得即讀為號當依晉音耳】逢門子彎烏號【師古曰逢門善射者即逢蒙也應劭曰楚有柘桑烏栖其上枝下著地不得飛欲墮號呼故曰烏號張揖曰黄帝乘龍上天小臣不得上挽持龍拔墮黄帝弓臣下抱弓而號故名弓烏號師古曰應張二說皆有據逢皮江翻】猶未足以喻其意也故聖主必待賢臣而弘功業俊士亦俟明主以顯其德上下俱欲驩然交欣千載壹合論說無疑翼乎如鴻毛遇順風沛乎如巨魚縱大壑其得意若此則胡禁不止曷令不行【師古曰胡曷皆何也】化溢四表横被無窮【被皮義翻】是以聖主不徧窺望而視已明不殫傾耳而聽已聰【師古曰殫盡也】太平之責塞【師古曰塞滿也塞悉則翻】優游之望得休徵自至夀考無疆何必偃仰屈伸若彭祖呴嘘呼吸如僑松【如淳曰五帝紀彭祖堯舜時人列仙傳彭祖殷大夫也歷夏至商末號年七百師古曰呴嘘者開口出氣也僑王僑松赤松子皆仙人也呴吁于翻嘘音虚】眇然絶俗離世哉【師古曰眇然高遠之意離力智翻】是時上頗好神僊故褒對及之【好呼到翻下同】京兆尹張敞亦上疏諫曰願明主時忘車馬之好斥遠方士之虚語游心帝王之術太平庶幾可興也【遠於願翻幾居希翻】上由是悉罷尚方待詔【此尚方非作器物之尚方尚主也主方藥也司馬相如大人賦詔岐伯使尚方是也】初趙廣漢死後為京兆尹者皆不稱職【稱尺證翻】唯敞能繼其迹其方畧耳目不及廣漢然頗以經術儒雅文之 上頗修飾宫室車服盛于昭帝時外戚許史王氏貴寵諫大夫王吉上疏曰陛下躬聖質總萬方惟思世務將興太平詔書每下民欣然若更生臣伏而思之可謂至恩未可謂本務也【師古曰言天子如此雖於百姓為至恩然未盡政務之本也】欲治之主不世出【師古曰言有時遇之不常值治直吏翻】公卿幸得遭遇其時言聽諫從然未有建萬世之長策舉明主於三代之隆也其務在於期會簿書斷獄聽訟而已【斷丁亂翻】此非太平之基也臣聞民者弱而不可勝愚而不可欺也聖主獨行於深宫得則天下稱誦之失則天下咸言之故宜謹左右審擇所使左右所以正身所使所以宣德此其本也孔子曰安上治民莫善于禮非空言也【師古曰孝經載孔子之言治直之翻】王者未制禮之時引先王禮宜于今者而用之臣願陛下承天心發大業與公卿大臣延及儒生述舊禮明王制一世之民躋之仁夀之域【師古曰以仁撫下則羣生安逸而夀考余謂此以仁夀二字並言仁者不鄙詐夀者不夭折也敺與驅同】則俗何以不若成康夀何以不若高宗【師古曰高宗殷王武丁也享國百年】竊見當世趨務不合於道者謹條奏【師古曰趨讀曰趣趣嚮也】唯陛下財擇焉吉意以為世俗聘妻送女無節則貧人不及故不舉子又漢家列侯尚公主諸侯則國人承翁主【晉灼曰娶天子女則曰尚公主國人娶諸侯女則曰承翁主尚承皆卑下之名也】使男事女夫屈於婦逆隂陽之位故多女亂古者衣服車馬貴賤有章今上下僭奢人人自制【師古曰言無節度】是以貪財誅利不畏死亡【誅責也求也】周之所以能致治刑措而不用者以其禁邪於冥冥絶惡於未萌也【師古曰冥冥言未有端緒也治直吏翻】又言舜湯不用三公九卿之世而舉臯陶伊尹【李奇曰不繼世而爵也言臯陶伊尹非三公九卿之世陶音遥】不仁者遠【師古曰任用賢人放黜讒佞】今使俗吏得任子弟【張晏曰子弟以父兄任為郎】率多驕驁不通古今無益于民宜明選求賢除任子之令外家及故人可厚以財不宜居位去角抵減樂府省尚方明示天下以儉古者工不造琱瑑【師古曰瑑者刻鏤為文瑑音篆】商不通侈靡非工商之獨賢政教使之然也上以其言為迂濶【師古曰迂遠也音于】不甚寵異也吉遂謝病歸 義渠安國至羌中召先零諸豪三十餘人以尤桀黠者皆斬之【師古曰桀堅也言不順從也黠惡也為惡堅也零音憐黠戶入翻】縱兵擊其種人【種章勇翻下同】斬首千餘級於是諸降羌及歸義羌侯楊玉等怨怒無所信郷【師古曰恐中國汎怒不信其心而納嚮之仲馮曰恐怒且恐且怒也羌未有變而漢吏無故誅殺其人故楊玉等謂漢無所信嚮不信漢不嚮漢也作怨怒者通鑑畧改班書之文成一家言降戶江翻】遂刼畧小種背畔犯塞攻城邑殺長吏【背蒲妹翻】安國以騎都尉將騎二千屯備羌至浩亹【浩亹縣屬金城郡有浩亹水出西塞外東至允吾入湟水孟康曰浩亹音合門師古曰浩音誥浩水名也亹者水流峽山岸深若門也詩大雅曰鳬鷖在門亦其義也今俗呼此水為閣門河盖疾言之浩為閣耳杜佑曰浩亹縣即今金城郡廣武縣地又曰廣武縣西南有漢浩亹縣故城】為虜所擊失亡車重兵器甚衆【師古曰重音直用翻】安國引還至令居以聞【令音零】時趙充國年七十餘上老之使丙吉問誰可將者【將即亮翻下同】充國對曰無踰於老臣者矣上遣問焉曰將軍度羌虜何如【師古曰度計也音犬各翻下同】當用幾人充國曰百聞不如一見兵難遥度臣願馳至金城【昭帝元始六年置金城郡唐蘭鄯廓州地】圖上方略【師古曰圖其地形并為攻討方畧俱奏上也上時掌翻下同】羌戎小夷逆天背畔滅亡不久【背蒲妹翻】願陛下以屬老臣【師古曰屬委也屬音之欲翻】勿以為憂上笑曰諾乃大發兵詣金城夏四月遣充國將之以擊西羌【將即亮翻】 六月有星孛于東方【孛蒲内翻】趙充國至金城須兵滿萬騎欲度河恐為虜所遮即夜遣三校衘枚先度【師古曰衘枚者欲其無聲使虜不覺校戶教翻下同】度輒營陳【立營陳則虜不得而犯諸軍可以相繼而度河陳讀曰陣】會明畢遂以次盡度虜數十百騎來出入軍傍充國曰吾士馬新倦不可馳逐此皆驍騎難制又恐其為誘兵也【驍堅堯翻誘音酉】擊虜以殄滅為期小利不足貪令軍勿擊遣騎四望陿中無虜【文穎曰金城有三陿在南六百里師古曰山峭而夾水曰陿四望者陿名也陿音狹】夜引兵上至落都【服䖍曰落都山名也據水經注破羌縣之西有落都城後漢志浩亹縣有雒都谷劉昫曰唐鄯州治故樂都城】召諸校司馬謂曰吾知羌虜不能為兵矣使虜發數千人守杜四望陿中【師古曰杜塞也】兵豈得入哉充國常以遠斥為務行必為戰備止必堅營壁尤能持重愛士卒先計而後戰遂西至西部都尉府【孟康曰在金城】日饗軍士【師古曰饗飲之】士皆欲為用虜數挑戰【數所角翻挑徒了翻】充國堅守捕得生口言羌豪相數責曰語汝無反【數所具翻語牛倨翻】今天子遣趙將軍來年八九十矣善為兵今請欲一鬭而死可得邪【言充國持重不戰羌欲一鬭而死不可得也】初䍐开豪靡當兒使弟雕庫來告都尉曰先零欲反後數日果反雕庫種人頗在先零中都尉即留雕庫為質【金城西部都尉也種章勇翻質音致】充國以為無罪乃遣歸告種豪大兵誅有罪者明白自别毋取并滅【師古曰言勿相和同并取滅亡别彼列翻】天子告諸羌人犯灋者能相捕斬除罪仍以功大小賜錢有差【時募能斬大豪有罪者一人賜錢四十萬中豪十五萬下豪二萬女子及老弱千錢】又以其所捕妻子財物盡與之充國計欲以威信招降䍐开及刼略者解散虜謀徼其疲劇乃擊之【師古曰徼要也音工堯翻】時上已發内郡兵屯邊者合六萬人矣酒泉太守辛武賢【姓譜夏啟封支子於莘莘辛相近遂為辛氏漢初申蒲為趙魏名將及徙家隴西遂為隴西人余按此叙辛武賢之世然既以莘為辛而又以申牽合之以其聲相近也然周自有太史辛甲】奏言郡兵皆屯備南山北邊空虛勢不可久若至秋冬乃進兵此虜在境外之冊今虜朝夕為寇土地寒苦漢馬不耐冬不如以七月上旬齎三十日糧分兵出張掖酒泉合擊䍐开在鮮水上者【劉昫曰漢金城郡之金城縣䍐羌所處也後漢置西海郡晉乞伏乾歸都於此唐為蘭州五泉縣余據漢書羌豪獻鮮水海地于王莽置西海郡即此山海經云北鮮之山鮮水出焉北流注於徐吾非此鮮水也】雖不能盡誅但奪其畜產虜其妻子復引兵還冬復擊之【復扶人翻】大兵仍出虜必震壞【師古曰仍頻也】天子下其書充國【下遐稼翻下同】令議之充國以為一馬自負三十日食為米二斛四斗麥八斛又有衣裝兵器難以追逐虜必商軍進退【師古曰商計度也】稍引去逐水草入山林隨而深入虜即據前險守後阨以絶糧道必有傷危之憂為夷狄笑千載不可復【復報也載子玄翻】而武賢以為可奪其畜產虜其妻子此殆空言非至計也【師古曰殆僅也韻畧云近也】先零首為畔逆它種劫略【師古曰言被刼畧而反畔非其本心】故臣愚冊【冊謀也籌也】欲捐䍐开闇昧之過隐而勿章先行先零之誅以震動之宜悔過反善因赦其罪選擇良吏知其俗者拊循和輯【師古曰拊古撫字輯與集同】此全師保勝安邉之冊天子下其書公卿議者咸以為先零兵盛而負䍐开之助【師古曰負恃也】不先破䍐开則先零未可圖也上乃拜侍中許延壽為彊弩將軍即拜酒泉太守武賢為破羌將軍【師古曰即就也就其郡而拜之】賜璽書嘉納其冊以書敕讓充國曰今轉輸並起百姓煩擾將軍將萬餘之衆不早及秋共水草之利争其畜食【師古曰此畜謂畜產牛羊之屬食謂穀麥之屬也或曰畜食畜之所食即謂草也】欲至冬虜皆當畜食【師古曰此畜讀曰蓄蓄聚積也】多臧匿山中依險阻【臧古藏字】將軍士寒手足皸瘃【師古曰皸坼裂也瘃寒創也皸音軍瘃竹足翻】寧有利哉將軍不念中國之費欲以歲數而勝敵【師古曰久歷年歲乃勝小敵也數音所具翻】將軍誰不樂此者【師古曰言為將軍者皆樂此樂音洛】今詔破羌將軍武賢等將兵以七月擊䍐羌將軍其引兵並進勿復有疑【復扶又翻】充國上書曰陛下前幸賜書欲使人諭䍐以大軍當至漢不誅䍐以解其謀臣故遣开豪雕庫宣天子至德䍐开之屬皆聞知明詔今先零羌楊玉阻石山木便為寇【師古曰謂阻依山之木石以自保固】䍐羌未有所犯乃置先零先擊䍐釋有罪誅無辜【師古曰釋置也放也】起壹難就兩害誠非陛下本計也臣聞兵灋攻不足者守有餘又曰善戰者致人不致於人【師古曰致人者引致而取之致於人為人所引也】今䍐羌欲為燉煌酒泉寇【燉徒門翻】宜飭兵馬練戰士以須其至【師古曰須待也】坐得致敵之術以逸擊勞取勝之道也今恐二郡兵少不足以守而發之行攻釋致虜之術而從為虜所致之道【師古曰釋廢也】臣愚以為不便先零羌虜欲為背畔故與䍐开解仇結約然其私心不能無恐漢兵至而䍐开背之也【背蒲妹翻】臣愚以為其計常欲先赴䍐开之急以堅其約先擊䍐羌先零必助之今虜馬肥糧食方饒擊之恐不能傷害適使先零得施德於䍐羌【師古曰施德自樹恩德也】堅其約合其黨虜交堅黨合精兵二萬餘人廹脅諸小種附著者稍衆【著直畧翻】莫須之屬不輕得離也【服䖍曰莫須小種羌名也】如是虜兵寖多誅之用力數倍臣恐國家憂累【累力瑞翻下累重同】由十年數不二三歲而已於臣之計先誅先零已則䍐开之屬不煩兵而服矣先零已誅而䍐开不服涉正月擊之得計之理又其時也以今進兵誠不見其利戊申充國上奏【上時掌翻】秋七月甲寅璽書報從充國計焉充國乃引兵至先零在所虜久屯聚懈弛【師古曰弛放也】望見大軍棄車重欲度湟水【重直用翻】道阨陿充國徐行驅之或曰逐利行遲【師古曰逐利宜速今行太遅】充國曰此窮寇不可廹也緩之則走不顧急之則還致死【師古曰謂更迴還盡力而死戰】諸校皆曰善虜赴水溺死者數百降及斬首五百餘人【降戶江翻】虜焉牛羊十萬餘頭車四千餘兩【兩音亮】兵至䍐地令軍毋燔聚落芻牧田中【師古曰不得燔燒人居及於田畝之中刈芻放牧也】䍐羌聞之喜曰漢果不擊我矣豪靡忘使人來言願得還復故地【服䖍曰靡忘羌帥名也】充國以聞未報靡忘來自歸充國賜飲食遣還諭種人護軍以下皆争之曰此反虜不可擅遣充國曰諸君但欲便文自營【師古曰苟取文墨之便以自營衛】非為公家忠計也語未卒【為于偽翻卒子恤翻】璽書報令靡忘以贖論後䍐竟不煩兵而下上詔破羌彊弩將軍詣屯所以十二月與充國合擊先零時羌降者萬餘人矣充國度其必壞【度徒洛翻】欲罷騎兵屯田以待其敝作奏未上【上時掌翻】會得進兵璽書充國子中郎將卬懼使客諫充國曰誠令兵出破軍殺將以傾國家【將即亮翻】將軍守之可也即利與病又何足争一旦不合上意遣繡衣來責將軍【師古曰繡衣謂御史】將軍之身不能自保何國家之安充國歎曰是何言之不忠也本用吾言羌虜得至是耶【師古曰言豫防之可無今日之寇也】往者舉可先行羌者【行下孟翻】吾舉辛武賢丞相御史復白遣義渠安國竟沮敗羌【復扶又翻敗補邁翻】金城湟中穀斛八錢吾謂耿中丞【服䖍曰耿夀昌也為司農中丞姓譜耿古國名為晉所滅子孫以為氏謂告語也】糴三百萬斛穀羌人不敢動矣【師古曰言豫儲糧食可以制敵】耿中丞請糴百萬斛乃得四十萬斛耳義渠再使【使疏吏翻】且費其半失此二冊羌人致敢為逆失之毫釐差以千里是既然矣今兵久不决四夷卒有動揺【卒讀曰猝下可卒同又卒死同】相因而起雖有知者不能善其後【知讀曰智】羌獨足憂邪【師古曰言儻如此則所憂不獨在羌】吾固以死守之明主可為忠言遂上屯田奏曰臣所將吏士馬牛食所用糧穀茭槀調度甚廣難久不解【調徒弔翻難乃旦翻】傜役不息恐生他變為明主憂誠非素定廟勝之冊【師古曰廟勝謂謀於廟堂而勝敵也】且羌易以計破難用兵碎也【易以豉翻】故臣愚心以為擊之不便計度臨羌東至浩亹羌虜故田及公田民所未墾可二千頃以上【度徒洛翻】其閒郵亭多壞敗者臣前部士入山伐林木六萬餘枚在水次臣願罷騎兵留步兵萬二百八十一人分屯要害處氷解漕下繕郷亭浚溝渠【師古曰漕下以水運木而下也繕補也浚深治也】治湟陿以西道橋七十所令可至鮮水左右田事出賦人三十畮【師古曰田事出謂至春人出營田也賦謂班與之也畮古畝字】至四月草生發郡騎及屬國胡騎各千就草為田者遊兵以充入金城郡益積畜省大費今大司農所轉穀至者足支萬人一歲食謹上田處及器用簿【上時掌翻】上報曰即如將軍之計虜當何時伏誅兵當何時得决孰計其便復奏【孰與熟同復扶又翻】充國上狀曰臣聞帝王之兵以全取勝是以貴謀而賤戰百戰而百勝非善之善者也故先為不可勝以待敵之可勝【師古曰此兵法之辭言先自完堅令敵不能勝我乃可以勝敵也余據此言本之孫子】蠻夷習俗雖殊於禮義之國然其欲避害就利愛親戚畏死亡一也今虜亡其美地薦草【師古曰薦稠草】愁於寄託遠遯骨肉心離人有畔志而明主班師罷兵【鄧展曰班還也】萬人留田順天時因地利以待可勝之虜雖未即伏辜兵決可朞月而望羌虜瓦解前後降者萬七百餘人及受言去者凡七十輩【如淳曰羌胡言欲降受其言遣去者師古曰如說非也謂羌受充國之言歸相告喻者也羌虜即羌賊耳無預於胡】此坐支解羌虜之具也臣謹條不出兵留田便宜十二事步兵九校【師古曰一部為一校校戶教翻】吏士萬人留屯以為武備因田致穀威德並行一也又因排折羌虜令不得歸肥饒之地貧破其衆以成羌虜相畔之漸二也居民得並田作【師古曰並且讀如本字又音步浪翻仲馮曰並亦俱也】不失農業三也軍馬一月之食度支田士一歲罷騎兵以省大費四也【度徒洛翻】至春省甲士卒循河湟漕穀至臨羌【臨羌縣屬金城郡其西北即塞外】以示羌虜揚威武傳世折衝之具五也以閒暇時下先所伐材繕治郵亭充入金城六也【閒與閑同治直之翻】兵出乘危徼幸【師古曰言不可必勝徼堅堯翻又一遙翻】不出令反畔之虜竄於風寒之地離霜露疾疫瘃墮之患【師古曰墮謂困寒瘃而墮指者】坐得必勝之道七也無經阻遠追死傷之害八也内不損威武之重外不令虜得乘間之勢九也【師古曰間謂軍之間隙者也問古莧翻】又亡驚動河南大开小开【皆羌種在河西之河南亡古無字通】使生它變之憂十也治隍陿中道橋令可至鮮水以制西域伸威千里從枕席上過師十一也【鄭氏曰橋成軍行安易若於枕席上過也】大費既省繇役豫息以戒不虞十二也【繇古傜字通】留屯田得十二便出兵失十二利唯明詔采擇上復賜報曰兵决可期月而望者【復扶又翻下同期讀曰朞】謂今冬邪謂何時也將軍獨不計虜聞兵頗罷且丁壯相聚攻擾田者及道上屯兵復殺略人民將何以止之將軍孰計復奏充國復奏曰臣聞兵以計為本故多筭勝少筭【孫子曰多筭勝少筭不勝】先零羌精兵今餘不過七八千人失地遠客分散飢凍畔還者不絶臣愚以為虜破壞可日月冀遠在來春故曰兵决可期月而望竊見北邉自燉煌至遼東萬一千五百餘里乘塞列地有吏卒數千人虜數以大衆攻之而不能害【燉徒門翻數所角翻】今騎兵雖罷虜見屯田之士精兵萬人從今盡三月虜馬羸瘦【羸倫為翻】必不敢捐其妻子於他種中【種章勇翻】遠涉河山而來為寇亦不敢將其累重還歸故地【師古曰累重謂妻子也累力瑞翻重直用翻】是臣之愚計所以度虜且必瓦解其處【師古曰各于其處自瓦解度徒洛翻】不戰而自破之冊也【冊與策同】至於虜小寇盗時殺人民其原未可卒禁【卒讀曰猝】臣聞戰不必勝不苟接刃攻不必取不苟勞衆誠令兵出雖不能滅先零但能令虜絶不為小寇則出兵也即今同是【師古曰言俱不能止小寇盗】而釋坐勝之道從乘危之勢往終不見利空内自罷敝【罷讀曰疲】貶重以自損【貶重謂貶中國之威重也】非所以示蠻夷也又大兵一出還不可復留【言大兵出塞而還人有歸志不可使復留屯以備羌】湟中亦未可空如是徭役復更發也【復扶又翻下同】臣愚以為不便臣竊自惟念奉詔出塞引軍遠擊窮天子之精兵散車甲於山野雖亡尺寸之功【亡古無字通下同】媮得避嫌之便【師古曰媮苟且也】而亡後咎餘責此人臣不忠之利非明主社稷之福也充國奏每上輒下公卿議臣【上時掌翻下遐嫁翻】初是充國計者什三中什五最後什八有詔詰前言不便者皆頓首服【詰去吉翻】魏相曰臣愚不習兵事利害後將軍數畫軍冊【數所角翻下同】其言常是臣任其計必可用也【師古曰任保也】上於是報充國嘉納之亦以破羌彊弩將軍數言當擊於是兩從其計詔兩將軍與中郎將卬出擊彊弩出降四千餘人破羌斬首二千級中郎將卬斬首降者亦二千餘級而充國所降復得五千餘人詔罷兵獨充國留屯田 大司農朱邑卒上以其循吏閔惜之詔賜其子黄金百斤以奉其祭祀 是歲前將軍龍侯韓增為大司馬車騎將軍【龍頟侯國屬平原郡師古曰今書本雒字或作額而崔浩云有龍頟村作額者非音洛】 丁令比三歲鈔盗匈奴【令音零比毗至翻鈔楚交翻】殺略數千人匈奴遣萬餘騎往擊之無所得【史言匈奴漸衰】<br />
<br />
  二年春正月以鳳皇甘露降集京師赦天下 夏五月趙充國奏言羌本可五萬人軍凡斬首七千六百級降者三萬一千二百人溺河湟餓死者五六千人定計遺脫與煎鞏黄羝俱亡者不過四千人【定計以定數計筭也】羌靡忘等自詭必得【師古曰詭責也自以為憂責言必能得之】請罷屯兵奏可充國振旅而還【書班師振旅孔安國注曰兵入曰振旅振整也杜預曰振整也旅衆也言振衆而還也】所善浩星賜迎說充國曰【鄧展曰浩星姓賜名也孫愐曰漢又有浩星公治穀梁說輸芮翻】衆人皆以破羌彊弩出擊多斬首生降虜以破壞然有識者以為虜埶窮困兵雖不出即自服矣將軍即見【見賢遍翻】宜歸功於二將軍出擊非愚臣所及如此將軍計未失也充國曰吾年老矣爵位已極豈嫌伐一時事以欺明主哉【言一時用兵之事當以實敷奏豈可以自矜伐為嫌】兵埶國之大事當為後灋老臣不以餘命壹為陛下明言兵之利害卒死誰當復言之者卒以其意對【為于偽翻卒子恤翻復扶又翻】上然其計罷遣辛武賢歸酒泉太守官充國復為後將軍秋羌若零離留且種兒庫【師古曰且音子閭翻】共斬先零大豪猶非楊玉首【文穎曰猶非人名也師古曰猶非及楊玉二人也】及諸豪弟澤陽雕良兒靡忘皆帥煎鞏黄羝之屬四千餘人降【帥讀曰率下同 考異曰宣紀五月羌斬猶非楊玉降充國傳五月奏罷屯兵秋羌斬猶非楊玉降今從傳】漢封若零弟澤二人為帥衆王餘皆為侯為君【離留且種二人為侯兒庫為君陽雕為言兵侯良兒為君靡忘為獻牛君】初置金城屬國以處降羌【處昌呂翻】詔舉可護羌校尉者【護羌校尉之官始見於此范曄曰漢武帝時諸羌與匈奴通攻令居安故圍抱罕遣李息徐自為擊定之始置護羌校尉】時充國病四府舉辛武賢小弟湯【四府丞相御史車騎將軍前將軍府也併後將軍府為五府】充國遽起奏湯使酒不可典蠻夷【師古曰使酒因酒而使氣若今言惡酒者使如字】不如湯兄臨衆時湯已拜受節【拜者拜官護羌校尉持節護諸羌】有詔更用臨衆【更改也音工衡翻】後臨衆病免五府復舉湯湯數醉䣱羌人【復扶又翻數所角翻下同師古曰䣱况務翻即酗字也醉怒曰䣱】羌人反畔卒如充國之言【史終言其事卒子恤翻】辛武賢深恨充國【以破羌希賞而格不行也】上書告中郎卭泄省中語【辛武賢在軍中時與卬晏語卬言張安世始不快上上欲誅之卬家將軍以為安世宜全度之由此安世得免武賢恨充國告卬以此罪】下吏自殺【下遐稼翻】司隸校尉魏郡盖寛饒【百官表司隸校尉周官武帝征和四年初置持節從中都官徒千二百人捕巫蠱督大姦猾後罷其兵察三輔三河弘農師古曰以掌徒隸而廵察故云司隸盖音古盍翻齊大夫陳戴食采於蓋其後以為氏至漢初齊有蓋公】剛直公清數干犯上意時上方用刑灋任中書官【武帝游宴後庭用䆠者為中書官宣帝因之遂基恭顯之禍賢曰中書内中之書也】寛饒奏封事曰方今聖道浸微儒術不行以刑餘為周召【師古曰言使奄人當權軸也召讀曰邵】以灋律為詩書【師古曰言以刑法成教化也】又引易傳【傳直戀翻】言五帝官天下三王家天下家以傳子孫官以傳賢聖書奏上以為寛饒怨謗下其書中二千石【下遐稼翻下同】時執金吾議【據公卿表是歲也南陽太守賢為執金吾】以為寛饒旨意欲求禪大逆不道【師古曰言欲使天子傳位于已】諫大夫鄭昌愍傷寛饒忠直憂國以言事不當意而為文吏所詆挫【師古曰詆毁也挫折也】上書訟寛饒曰【訟者訟其寃也】臣聞山有猛獸藜藿為之不采國有忠臣姦邪為之不起【為于偽翻】司隸校尉寛饒居不求安食不求飽【師古曰論語稱孔子曰君子食無求飽居無求安故引之】進有憂國之心退有死節之義上無許史之屬下無金張之託【應劭曰許伯宣帝皇后父史高宣帝外家也金金日磾也張張安世也此四家屬託無不聽師古曰此說非也許氏史氏有外屬之恩金氏張氏自託於近侍也屬讀如本字】職在司察直道而行多仇少與【師古曰仇怨讎也與黨與也】上書陳國事有司劾以大辟【劾戶槩翻辟毗亦翻】臣幸得從大夫之後官以諫為名不敢不言上不聽九月下寛饒吏寛饒引佩刀自剄北闕下【剄古鼎翻】衆莫不憐之匈奴虛閭權渠單于將十餘萬騎旁塞獵【旁步浪翻】欲入<br />
<br />
  邉為寇未至會其民題除渠堂亡降漢言狀漢以為言兵鹿奚鹿盧侯【此侯不見世表盖無食邑猶前羌陽雕侯言兵侯之類也】而遣後將軍趙充國將兵四萬餘騎屯緣邉九郡【文潁曰五原朔方之屬也師古曰九郡者五原朔方雲中代郡鴈門定襄北平上谷漁陽也四萬餘騎分屯之而充國總統領之據充國傳書此事於征羌之前通鑑因匈奴内亂書于此以先事】備虜月餘單于病歐血因不敢入還去即罷兵乃使題王都犂胡次等入漢請和親未報會單于死虛閭權渠單于始立而黜顓渠閼氏【事見二十四卷地節二年閼氏音煙支】顓渠閼氏即與右賢王屠耆堂私通右賢王會龍城而去顓渠閼氏語以單于病甚且勿遠【語牛倨翻】後數日單于死用事貴人郝宿王刑未央使人召諸王未至【師古曰郝音呼各翻】顓渠閼氏與其弟左大將且渠都隆奇謀立右賢王為握衍朐鞮單于【且子余翻朐音劬鞮丁奚翻】握衍朐鞮單于者烏維單于耳孫也【應劭曰耳孫玄孫之子也言去其高曾益遠但耳聞之也李斐曰耳孫曾孫也晉灼曰耳孫玄孫之曾孫也諸侯王表在八世師古曰耳孫諸說不同據平紀及諸侯王表說梁孝王玄孫之子耳孫音仍又匈奴傳說握衍朐鞮單于云烏維單于耳孫以此參之李云曾孫是也然漢書諸處又皆云曾孫非一不應雜兩稱而言據爾雅曾孫之子為玄孫玄孫之子為來孫來孫之子為昆孫昆孫之子為仍孫從己之數是為八葉則與晉說相同仍耳聲相近孟一號也但班氏唯存古名而計其葉數則錯也】握衍朐鞮單于立凶惡殺刑未央等而任用都隆奇又盡免虛閭權渠子弟近親而自以其子弟代之虚閭權渠單于子稽侯㹪既不得立【師古曰㹪音先安翻又音所姦翻杜佑山諫翻】亡歸妻父烏禪幕【師古曰禪音禪】烏禪幕者本康居烏孫間小國數見侵暴【數所角翻】率其衆數千人降匈奴狐鹿姑單于以其弟子日逐王姊妻之使長其衆居右地【師古曰長其衆為之長帥妻七細翻長知兩翻】日逐王先賢撣【鄭氏曰撣音纒束之纒晉灼曰音田師古曰晉音是也】其父左賢王當為單于讓狐鹿姑單于狐鹿姑單于許立之【事見二十二卷武帝太始元年】國人以故頗言日逐王當為單于日逐王素與握衍朐鞮單于有隙即帥其衆欲降漢【帥讀曰率降戶江翻下同】使人至渠犂與騎都尉鄭吉相聞吉發渠犂龜兹諸國五萬人迎日逐王口萬二千人小王將十二人【小王將者以禆小王將兵者也一曰匈奴左右賢王左右谷蠡王左右大將以下凡二十四長為大王將其餘為小王將將即亮翻】隨吉至河曲【黄河千里一曲此當在金城郡界】頗有亡者吉追斬之遂將詣京師【將如字領也挾也】漢封日逐王為歸德侯【功臣表歸德侯食邑於汝南】吉既破車師【事見上卷地節三年】降日逐威震西域遂并護車師以西北道故號都護都護之置自吉始焉【師古曰並護南北二道故謂之都都猶大也總也】上封吉為安遠侯【功臣侯表安遠侯食於汝南之慎縣】邑吉於是中西域而立莫府【師古曰中西域者言最處諸國之中遠近均也中音竹仲翻 考異曰百官表曰西域都護加官地節二年初置盖誤以神爵為地節也西域傳又云神爵三年亦誤】治烏壘城去陽關二千七百餘里【烏壘城與渠犂田官相近陽關在敦煌龍勒縣西宋白曰伊州伊吾郡漢伊吾盧地宣帝時鄭吉為西域都護治烏壘城即此永平末取此地置宜禾都尉】匈奴益弱不敢争西域僮僕都尉由此罷【西域諸國故皆役屬匈奴匈奴西邉日逐王置僮僕都尉使領西域常居焉耆危須尉犂間賦税諸國取富給焉匈奴盖以僮僕視西域也今日逐王既降西域諸國咸服於漢故僮僕都尉罷】都護督察烏孫康居等三十六國動静有變以聞可安輯安輯之不可者誅伐之漢之號令班西域矣【師古曰班布也】握衍朐鞮單于更立其從兄薄胥堂為日逐王【為薄胥堂立為屠耆單于張本從才用翻】烏孫昆彌翁歸靡因長羅侯常惠上書願以漢外孫元貴靡為嗣【元貴靡楚王解憂長男也】得令復尚漢公主結婚重親【復扶又翻下同重直龍翻】畔絶匈奴詔下公卿議【下遐稼翻下同】大鴻臚蕭望之以為烏孫絶域變故難保不可許【臚陵如翻】上美烏孫新立大功【謂本始二年破匈奴也】又重絶故業【師古曰重難也故業謂先與匈奴婚親也】乃以烏孫主解憂弟相夫為公主盛為資送而遣之使常惠送之至燉煌【燉音屯】未出塞聞翁歸靡死烏孫貴人共從本約立岑娶子泥靡為昆彌號狂王【本約見二十四卷本始三年岑娶漢書作岑陬】常惠上書願留少主燉煌【少詩照翻下同燉徒門翻】惠馳至烏孫責讓不立元貴靡為昆彌還迎少主事下公卿望之復以烏孫持兩端難約結【復扶又翻】今少主以元貴靡不立而還信無負于夷狄中國之福也少主不止繇役將興【繇古徭字通】天子從之徵還少主【考異曰烏孫傳請昏在元康二年望之傳云神爵二年按元康二年望之未為鴻臚盖誤以神爵為元康也】<br />
<br />
  三年春三月丙辰高平憲侯魏相薨【恩澤侯表高平侯食邑于淮陽柘縣諡法博聞多能曰憲】夏四月戊辰丙吉為丞相吉上寛大好禮讓【好呼到翻】不親小事時人以為知大體 秋七月甲子大鴻臚蕭望之為御史大夫 八月詔曰吏不亷平則治道衰【治直吏翻下同】今小吏皆勤事而俸禄薄【俸扶用翻】欲無侵漁百姓難矣【如淳曰漁奪也謂奪其利便也晉灼曰許慎云捕魚之字也師古曰漁者若言漁獵也晉說是也】其益吏百石已下俸十五【如淳曰律百石奉月六百韋昭曰若食一石則益五斗 考異曰宣紀云益吏百石以下俸十五韋昭曰若食一石則益五斗荀紀云益吏百石以下俸五十斛盖以十五難曉故改之然詔云以下恐難指五十斛也】 是歲東郡太守韓延夀為左馮翊始延夀為潁川太守潁川承趙廣漢搆會吏民之後【搆會吏民事見二十四卷本始三年師古曰搆結也】俗多怨讐延夀改更教以禮讓【更工衡翻】召故老與議定嫁娶喪祭儀品略依古禮不得過灋百姓遵用其教賣偶車馬下里偽物者棄之市道【張晏曰下里地下蒿里偽物也師古曰偶謂土木為之象真車馬之形也偶對也棄其物於市之道上也】黄霸代延夀居潁川霸因其迹而大治延夀為吏上禮義好古教化【好呼到翻】所至必聘其賢士以禮待用廣謀議納諫争表孝弟有行【争讀曰諍行下孟翻】修治學宫【師古曰學宫謂庠序之舍也治直之翻】春秋郷射陳鍾皷管弦盛升降揖讓【周禮地官郷大夫以郷射之禮五物詢衆庶一曰和二曰容三曰主皮四曰和容五曰興舞】及都試講武設斧鉞旌旗習射御之事【漢諸郡以八月都試講武事也如淳曰太守都尉令長丞尉會都試課殿最也】治城郭收賦租先明布告其日以期會為大事吏民敬畏趨郷之【師古曰趨讀曰趣郷讀曰嚮趣七喻翻】又置正五長【師古曰正若今之郷正里正也伍長同伍之中置一人為長也長知兩翻】相率以孝弟【弟讀曰悌下孝弟同】不得舍姦人【師古曰舍止也】閭里阡陌有非常吏輒聞知姦人不敢入界其始若煩後吏無追捕之苦民無箠楚之憂【師古曰箠杖也楚荆木也即今之荆子也箠止蘂翻】皆便安之接待下吏恩施甚厚而約誓明【施式豉翻】或欺負之者延夀痛自刻責豈其負之何以至此【師古曰言豈我負之其人何以為此事】吏聞者自傷悔其縣尉至自刺死【刺七亦翻】及門下掾自剄人救不殊【掾于絹翻師古曰殊絶也以人救之故身首不相絶也頸古頂翻】延夀涕泣遣吏醫治視【治直之翻】厚復其家【復方目翻】在東郡三歲令行禁止【令之必行禁之必止無違者也】斷獄大減【斷丁亂翻】由是入為馮翊延夀出行縣至高陵【高陵縣屬左馮翊行下孟翻】民有昆弟相與訟田自言延夀大傷之曰幸得備位為郡表率不能宣明教化至令民有骨肉争訟既傷風化重使賢長吏嗇夫三老孝弟受其恥【重直用翻賢長吏謂縣令丞也續漢志縣有嗇夫皆主知民善惡為役先後知民貧富為賦多少平其差品三老掌教化凡有孝子順孫貞女義婦讓財救患及學士為民法式者皆扁表其門以興民行賢曰三老孝弟力田三者皆郷官之名三老高帝置孝弟力田高后置所以勸導郷里助成風化也】咎在馮翊當先退是日移病不聼事因入卧傳舍閉閤思過【傳知戀翻下傳相同】一縣莫知所為令丞嗇夫三老亦皆自繫待罪於是訟者宗族傳相責讓此兩昆弟深自悔皆自髠肉袒謝願以田相移終死不敢復争【師古曰移猶傳也一說兄以讓弟弟又讓兄故云相移復扶又翻】郡中歙然莫不傳相敕厲不敢犯【歙與翕同許及翻】延夀恩信周徧二十四縣【馮翊統高陵櫟陽翟道池陽夏陽衙粟邑谷口蓮勺鄜頻陽臨晉重泉郃陽祋祤武城沈陽□德徵雲陵萬年長陵陽陵雲陽二十四縣】莫敢以辭訟自言者推其至誠吏民不忍欺紿【師古曰紿誑也音蕩亥翻】 匈奴單于又殺先賢撣兩弟烏禪幕請之不聽心恚【師古曰恚恨也音於避翻】其後左奥鞬王死單于自立其小子為奥鞬王留庭【留單于庭也】奥鞬貴人共立故奥鞬王子為王【師古曰與音郁鞬音居言翻】與俱東徙單于遣右丞相將萬騎往擊之失亡數千人不勝<br />
<br />
  資治通鑑卷二十六  <br>
   </div> 

<script src="/search/ajaxskft.js"> </script>
 <div class="clear"></div>
<br>
<br>
 <!-- a.d-->

 <!--
<div class="info_share">
</div> 
-->
 <!--info_share--></div>   <!-- end info_content-->
  </div> <!-- end l-->

<div class="r">   <!--r-->



<div class="sidebar"  style="margin-bottom:2px;">

 
<div class="sidebar_title">工具类大全</div>
<div class="sidebar_info">
<strong><a href="http://www.guoxuedashi.com/lsditu/" target="_blank">历史地图</a></strong>  
<a href="http://www.880114.com/" target="_blank">英语宝典</a>  
<a href="http://www.guoxuedashi.com/13jing/" target="_blank">十三经检索</a> 
<br><strong><a href="http://www.guoxuedashi.com/gjtsjc/" target="_blank">古今图书集成</a></strong> 
<a href="http://www.guoxuedashi.com/duilian/" target="_blank">对联大全</a> <strong><a href="http://www.guoxuedashi.com/xiangxingzi/" target="_blank">象形文字典</a></strong> 

<br><a href="http://www.guoxuedashi.com/zixing/yanbian/">字形演变</a>  <strong><a href="http://www.guoxuemi.com/hafo/" target="_blank">哈佛燕京中文善本特藏</a></strong>
<br><strong><a href="http://www.guoxuedashi.com/csfz/" target="_blank">丛书&方志检索器</a></strong> <a href="http://www.guoxuedashi.com/yqjyy/" target="_blank">一切经音义</a>  

<br><strong><a href="http://www.guoxuedashi.com/jiapu/" target="_blank">家谱族谱查询</a></strong>  <strong><a href="http://shufa.guoxuedashi.com/sfzitie/" target="_blank">书法字帖欣赏</a></strong> 
<br>

</div>
</div>


<div class="sidebar" style="margin-bottom:0px;">

<font style="font-size:22px;line-height:32px">QQ交流群9:489193090</font>


<div class="sidebar_title">手机APP 扫描或点击</div>
<div class="sidebar_info">
<table>
<tr>
	<td width=160><a href="http://m.guoxuedashi.com/app/" target="_blank"><img src="/img/gxds-sj.png" width="140"  border="0" alt="国学大师手机版"></a></td>
	<td>
<a href="http://www.guoxuedashi.com/download/" target="_blank">app软件下载专区</a><br>
<a href="http://www.guoxuedashi.com/download/gxds.php" target="_blank">《国学大师》下载</a><br>
<a href="http://www.guoxuedashi.com/download/kxzd.php" target="_blank">《汉字宝典》下载</a><br>
<a href="http://www.guoxuedashi.com/download/scqbd.php" target="_blank">《诗词曲宝典》下载</a><br>
<a href="http://www.guoxuedashi.com/SiKuQuanShu/skqs.php" target="_blank">《四库全书》下载</a><br>
</td>
</tr>
</table>

</div>
</div>


<div class="sidebar2">
<center>


</center>
</div>

<div class="sidebar"  style="margin-bottom:2px;">
<div class="sidebar_title">网站使用教程</div>
<div class="sidebar_info">
<a href="http://www.guoxuedashi.com/help/gjsearch.php" target="_blank">如何在国学大师网下载古籍?</a><br>
<a href="http://www.guoxuedashi.com/zidian/bujian/bjjc.php" target="_blank">如何使用部件查字法快速查字?</a><br>
<a href="http://www.guoxuedashi.com/search/sjc.php" target="_blank">如何在指定的书籍中全文检索?</a><br>
<a href="http://www.guoxuedashi.com/search/skjc.php" target="_blank">如何找到一句话在《四库全书》哪一页?</a><br>
</div>
</div>


<div class="sidebar">
<div class="sidebar_title">热门书籍</div>
<div class="sidebar_info">
<a href="/so.php?sokey=%E8%B5%84%E6%B2%BB%E9%80%9A%E9%89%B4&kt=1">资治通鉴</a> <a href="/24shi/"><strong>二十四史</strong></a>&nbsp; <a href="/a2694/">野史</a>&nbsp; <a href="/SiKuQuanShu/"><strong>四库全书</strong></a>&nbsp;<a href="http://www.guoxuedashi.com/SiKuQuanShu/fanti/">繁体</a>
<br><a href="/so.php?sokey=%E7%BA%A2%E6%A5%BC%E6%A2%A6&kt=1">红楼梦</a> <a href="/a/1858x/">三国演义</a> <a href="/a/1038k/">水浒传</a> <a href="/a/1046t/">西游记</a> <a href="/a/1914o/">封神演义</a>
<br>
<a href="http://www.guoxuedashi.com/so.php?sokeygx=%E4%B8%87%E6%9C%89%E6%96%87%E5%BA%93&submit=&kt=1">万有文库</a> <a href="/a/780t/">古文观止</a> <a href="/a/1024l/">文心雕龙</a> <a href="/a/1704n/">全唐诗</a> <a href="/a/1705h/">全宋词</a>
<br><a href="http://www.guoxuedashi.com/so.php?sokeygx=%E7%99%BE%E8%A1%B2%E6%9C%AC%E4%BA%8C%E5%8D%81%E5%9B%9B%E5%8F%B2&submit=&kt=1"><strong>百衲本二十四史</strong></a>  <a href="http://www.guoxuedashi.com/so.php?sokeygx=%E5%8F%A4%E4%BB%8A%E5%9B%BE%E4%B9%A6%E9%9B%86%E6%88%90&submit=&kt=1"><strong>古今图书集成</strong></a>
<br>

<a href="http://www.guoxuedashi.com/so.php?sokeygx=%E4%B8%9B%E4%B9%A6%E9%9B%86%E6%88%90&submit=&kt=1">丛书集成</a> 
<a href="http://www.guoxuedashi.com/so.php?sokeygx=%E5%9B%9B%E9%83%A8%E4%B8%9B%E5%88%8A&submit=&kt=1"><strong>四部丛刊</strong></a>  
<a href="http://www.guoxuedashi.com/so.php?sokeygx=%E8%AF%B4%E6%96%87%E8%A7%A3%E5%AD%97&submit=&kt=1">說文解字</a> <a href="http://www.guoxuedashi.com/so.php?sokeygx=%E5%85%A8%E4%B8%8A%E5%8F%A4&submit=&kt=1">三国六朝文</a>
<br><a href="http://www.guoxuedashi.com/so.php?sokeytm=%E6%97%A5%E6%9C%AC%E5%86%85%E9%98%81%E6%96%87%E5%BA%93&submit=&kt=1"><strong>日本内阁文库</strong></a> <a href="http://www.guoxuedashi.com/so.php?sokeytm=%E5%9B%BD%E5%9B%BE%E6%96%B9%E5%BF%97%E5%90%88%E9%9B%86&ka=100&submit=">国图方志合集</a> <a href="http://www.guoxuedashi.com/so.php?sokeytm=%E5%90%84%E5%9C%B0%E6%96%B9%E5%BF%97&submit=&kt=1"><strong>各地方志</strong></a>

</div>
</div>


<div class="sidebar2">
<center>

</center>
</div>
<div class="sidebar greenbar">
<div class="sidebar_title green">四库全书</div>
<div class="sidebar_info">

《四库全书》是中国古代最大的丛书,编撰于乾隆年间,由纪昀等360多位高官、学者编撰,3800多人抄写,费时十三年编成。丛书分经、史、子、集四部,故名四库。共有3500多种书,7.9万卷,3.6万册,约8亿字,基本上囊括了古代所有图书,故称“全书”。<a href="http://www.guoxuedashi.com/SiKuQuanShu/">详细>>
</a>

</div> 
</div>

</div>  <!--end r-->

</div>
<!-- 内容区END --> 

<!-- 页脚开始 -->
<div class="shh">

</div>

<div class="w1180" style="margin-top:8px;">
<center><script src="http://www.guoxuedashi.com/img/plus.php?id=3"></script></center>
</div>
<div class="w1180 foot">
<a href="/b/thanks.php">特别致谢</a> | <a href="javascript:window.external.AddFavorite(document.location.href,document.title);">收藏本站</a> | <a href="#">欢迎投稿</a> | <a href="http://www.guoxuedashi.com/forum/">意见建议</a> | <a href="http://www.guoxuemi.com/">国学迷</a> | <a href="http://www.shuowen.net/">说文网</a><script language="javascript" type="text/javascript" src="https://js.users.51.la/17753172.js"></script><br />
  Copyright &copy; 国学大师 古典图书集成 All Rights Reserved.<br>
  
  <span style="font-size:14px">免责声明:本站非营利性站点,以方便网友为主,仅供学习研究。<br>内容由热心网友提供和网上收集,不保留版权。若侵犯了您的权益,来信即刪。scp168@qq.com</span>
  <br />
ICP证:<a href="http://www.beian.miit.gov.cn/" target="_blank">鲁ICP备19060063号</a></div>
<!-- 页脚END --> 
<script src="http://www.guoxuedashi.com/img/plus.php?id=22"></script>
<script src="http://www.guoxuedashi.com/img/tongji.js"></script>

</body>
</html>
