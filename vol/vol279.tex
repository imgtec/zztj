資治通鑑卷二百七十九 宋 司馬光 撰

胡三省 音註

後唐紀八|{
	起閼逢敦牂二月盡旃蒙恊洽凡一年有奇}


潞王下

清泰元年二月癸酉蜀主以武泰節度使趙季良為司空兼門下侍郎同平章事領節度使如故|{
	趙季良遂為孟蜀佐命元臣}
吳人多不欲遷都者|{
	吳遷都之議始上卷明宗長興四年}
都押牙周宗言於徐知誥曰主上西遷公復須東行|{
	都押牙鎮海寧國兩鎮都押牙也昇州于揚州為西揚州於昇州為東言吳主若西遷金陵徐知誥須東鎮江都也復扶又翻}
不惟勞費甚大且違衆心丙子吳主遣宋齊丘如金陵諭知誥罷遷都先是知誥久有傳禪之志|{
	先悉薦翻下先已同}
以吳主無失德恐衆心不悦欲待嗣君宋齊丘亦以為然一旦知誥臨鏡鑷白髭|{
	鑷尼輒翻髭即移翻在口上曰髭在下曰鬚在頰曰髯}
歎曰國家安而吾老矣奈何周宗知其意請如江都微以傳禪諷吳主且告齊丘齊丘以宗先已心疾之|{
	先悉薦翻}
遣使馳詣金陵手書切諫以為天時人事未可知誥愕然|{
	徐知誥不意宋齊丘立異而忽睹其異議故愕然使疏吏翻}
後數日齊丘至請斬宗以謝吳主乃黜宗為池州副使|{
	池州副使池州團練副使也}
久之節度副使李建勲行軍司馬徐玠等屢陳知誥功業宜早從民望召宗復為都押牙知誥由是疎齊丘|{
	為宋齊丘邀君得禍張本鳴呼為人臣者當易姓之際謹毋以功名自居荀文若以之咀毒而逝劉穆之以之發病而死范雲恐後時不及療疾以求連愈至於促壽而不暇顧若宋齊丘之疾周宗又其輕淺者耳}
朱弘昭馮贇不欲石敬瑭久在太原且欲召孟漢瓊|{
	孟漢瓊權知天雄軍府見上卷上年}
己卯徙成德節度使范延光為天雄節度使代漢瓊徙潞王從珂為河東節度使兼北都留守徙石敬瑭為成德節度使皆不降制書但各遣使臣持宣監送赴鎮|{
	宣樞密院所行文書也是後漢隱帝時郭威以樞密院頭子易置西京留守豈非習於聞見而不以為異邪西班有大使臣小使臣監古銜翻}
吳主詔徐知誥還府舍|{
	徐知誥虛府舍以待吳主見上卷本年}
甲申金陵大火乙酉又火知誥疑有變勒兵自衛|{
	徐知誥之自衛其心猶王建也}
潞王既與朝廷猜阻朝廷又命洋王從璋權知鳳翔從璋性麤率樂禍|{
	樂音洛}
前代安重誨鎮河中手殺之|{
	見二百七十七卷明宗長興二年}
潞王聞其來尤惡之|{
	惡烏路翻}
欲拒命則兵弱糧少|{
	少詩沼翻}
不知所為謀於將佐皆曰主上富於春秋政事出於朱馮大王功名震主離鎮必無全理|{
	離力智翻}
不可受也|{
	言不可受代}
王問觀察判官滴河馬胤孫|{
	隋開皇十六年置滴河縣屬渤海郡唐屬棣州九域志滴河縣在棣州西南八十里注云漢都尉許商鑿此河近海故以商為名後人加水焉}
曰今道過京師當何向為便|{
	發此問以觀衆意}
對曰君命召不俟駕|{
	引論語孔子之言}
臨喪赴鎮又何疑焉諸人凶謀不可從也衆哂之|{
	言當過京師臨大行之喪然後赴太原也馬胤孫之言儒生守經學之言也是時勸潞王拒命者以其言為不達時變故相與哂之哂矢忍翻笑不壞顔為哂}
王乃移檄鄰道言朱弘昭等乘先帝疾亟殺長立少|{
	謂殺從榮而立帝也長知兩翻少詩照翻}
專制朝權别疎骨肉動揺藩垣|{
	謂易置石敬瑭及己也朝直遥翻下同别彼列翻}
懼傾覆社稷今從珂將入朝以清君側之惡而力不能獨辦願乞靈鄰藩以濟之潞王以西都留守王思同當東出之道|{
	自鳳翔趣洛陽道出長安}
尤欲與之相結遣推官郝詡押牙朱廷乂等相繼詣長安說以利害|{
	說式芮翻}
餌以美妓|{
	妓渠綺翻}
不從則令就圖之思同謂將吏曰吾受明宗大恩|{
	王思同自燕降晉梁晉相距思同未嘗有戰功明宗時以久次為節度使故自言受大恩}
今與鳳翔同反借使事成而榮猶為一時之叛臣况事敗而辱流千古之醜跡乎遂執詡等以狀聞時潞王使者多為鄰道所執不則依阿操兩端|{
	不讀曰否操七刀翻}
惟隴州防禦使相里金傾心附之|{
	隴州東至鳳翔一百五十里}
遣判官薛文遇往來計事|{
	薛文遇由此為潞王所信用}
金幷州人也朝廷議討鳳翔康義誠不欲出外恐失軍權|{
	明宗以康義誠為朴忠豈知其陰狡乃爾邪}
請以王思同為統帥|{
	帥所類翻}
以羽林都指揮使侯益為行營馬步軍都虞候|{
	宋白曰長興二年二月敇衛軍神捷神威雄武及魏府廣捷已下指揮改為左右羽林置四十指揮每十指揮立為一軍每一軍置都指揮使一人兼分為左右廂}
益知軍情將變辭不行|{
	侯益曾經鄴都之變故爾}
執政怒之出為商州刺史|{
	洛陽至商州八百八十六里}
辛卯以王思同為西面行營馬步軍都部署|{
	前此用兵置帥率以都招討使命之莊宗時明宗為北面招討使以禦契丹房知温為副都部署當時為都部署者必有其人又孟知祥拒董璋以趙廷隱為行營都部署後遂以為元帥之任宋氏建國之初猶因而用之}
前静難節度使藥彦稠副之|{
	難乃旦翻下同}
前絳州刺史萇從簡為馬步都虞候嚴衛步軍左廂指揮使尹暉|{
	宋白曰應順元年三月改在京羽林左右四十指揮為嚴衛左右軍然此時羽林指揮使楊思權與嚴衛指揮使尹暉並為西征偏禆則似羽林與嚴衛並置}
羽林指揮使楊思權等皆為偏禆暉魏州人也 蜀主以中門使王處回為樞密使丁酉加王思同同平章事知鳳翔行府以護國節度

使安彦威為西面行營都監思同雖有忠義之志而御軍無法潞王老於行陳將士徼幸富貴者心皆向之|{
	行戶剛翻陳讀曰陣徼堅堯翻}
詔遣殿直楚匡祚執亳州團練使李重吉幽於宋州|{
	九域志亳州西北至宋州一百四十五里重直龍翻}
洋王從璋行至關西|{
	函谷關之西也}
聞鳳翔拒命而還|{
	還從宣翻又如字}
三月安彦威與山南西道張䖍釗武定孫漢韶彰義張從賓静難康福等五節度使|{
	梁洋涇邠四帥并安彦威而五難乃旦翻}
奏合兵討鳳翔漢韶李存進之子也|{
	晉王克用義兒百有餘人李存進本姓孫後復本姓}
乙卯諸道兵大集於鳳翔城下攻之克東西關城城中死者甚衆丙辰復進攻城|{
	復扶又翻}
期於必取鳳翔城塹卑淺守備俱乏衆心危急潞王登城泣謂外軍曰吾未冠從先帝百戰出入生死金創滿身|{
	冠古玩翻創初良翻}
以立今日之社稷汝曹從我目睹其事今朝廷信任讒臣猜忌骨肉我何罪而受誅乎因慟哭聞者哀之張䖍釗性急主攻城西南以白刃驅士卒登城士卒怒大詬|{
	褊補典翻詬占侯翻又許侯翻}
反攻之䖍釗躍馬走免楊思權因大呼曰大相公吾主也|{
	楊思權本黨於秦王從榮從榮死思權不自安久矣因乘勢奉潞王王於明宗諸子為長故稱為大相公呼火故翻下同}
遂帥諸軍解甲投兵請降於潞王|{
	帥讀曰率降戶江翻下同}
自西門入以幅紙進潞王曰願王克京城日以臣為節度使勿以為防團|{
	防團謂防禦團練使也}
潞王即書思權可邠寧節度使授之王思同猶未之知趣士卒登城|{
	趣讀曰促}
尹暉大呼曰城西軍已入城受賞矣衆皆弃甲投兵而降其聲震地日中亂兵悉入外軍亦潰思同等六節度使皆遁去|{
	王思同及張䖍釗等五節度為六節度使按孫漢韶時守興元當以藥彦稠足六節度之數}
潞王悉斂城中將吏士民之財以犒軍至於鼎釜皆估直以給之|{
	犒苦到翻估音古}
丁巳王思同藥彦稠等走至長安西京副留守劉遂雍閉門不内乃趣潼關|{
	趣七喻翻}
遂雍鄩之子也|{
	劉鄩梁將也明宗以王淑妃故遂雍皆蒙引拔}
潞王建大將旗鼓整衆而東以孔目官虞城劉延朗為腹心|{
	隋分下邑縣置虞城縣唐屬宋州九域志在州東北五十五里歐史潞王起於鳳翔與共事者五人節度判官韓昭胤掌書記李專美牙將宋審䖍客將房暠孔目官劉延朗及即位審䖍將兵專美與薛文遇主謀議而昭胤暠及延朗掌機密}
潞王始憂王思同等併力據長安拒守至岐山|{
	九域志鳳翔府岐山縣東至長安二百四十三里}
聞劉遂雍不内思同甚喜遣使慰撫之遂雍悉出府庫之財於外軍士前至者即給賞令過比潞王至|{
	比必利翻及也}
前軍賞遍皆不入城庚申潞王至長安遂雍迎謁率民財以充賞|{
	府庫之財僅足以給前軍其隨潞王繼至者率民財以給之}
是日西面步軍都監王景從等自軍前奔還中外大駭帝不知所為謂康義誠等曰先帝弃萬國朕外守藩方|{
	謂鎮天雄也}
當是之時為嗣者在諸公所取耳朕實無心與人爭國既承大業年在幼沖|{
	五代會要明宗崩帝即位年二十}
國事皆委諸公朕於兄弟間不至榛梗|{
	榛梗者隔塞而不通榛側詵翻梗古杏翻}
諸公以社稷大計見告朕何敢違軍興之初皆自夸大以為寇不足平今事至於此何方可以轉禍|{
	言何術可以慱禍為福}
朕欲自迎潞王以大位讓之若不免於臯亦所甘心朱弘昭馮贇大懼不敢對|{
	猜間兄弟以起兵端朱弘昭馮贇為之也事敗而禍集聞帝言乃大懼}
義誠欲悉以宿衛兵迎降為己功乃曰西師驚潰盖主將失策耳|{
	薦王思同者康義誠也咎王思同者亦康義誠也將即亮翻下同}
今侍衛諸軍尚多臣請自往扼其衝要招集離散以圖後効幸陛下勿為過憂帝遣使召石敬瑭欲令將兵拒之義誠固請自行帝乃召將士慰諭空府庫以勞之|{
	勞力到翻}
許以平鳳翔人更賞二百緍府庫不足當以宫中服玩繼之軍士益驕無所畏忌負賜物楊言於路曰至鳳翔更請一分|{
	分扶問翻}
遣楚匡祚殺李重吉於宋州匡祚榜棰重吉責其家財|{
	前已囚重吉於宋州今又使就殺之榜音彭棰止橤翻}
又殺尼惠明|{
	召惠明入禁中見上卷本年}
初馬軍都指揮使朱洪實為秦王從榮所厚及朱弘昭為樞密使洪實以宗兄事之從榮勒兵天津橋洪實首為孟漢瓊繫從榮|{
	事見上卷上年首為于偽翻下為之同}
康義誠由是恨之|{
	康義誠許迎從榮而朱洪實擊之故恨}
辛酉帝親至左藏|{
	藏徂浪翻}
給將士金帛義誠洪實共論用兵利害洪實欲以禁軍固守洛陽曰如此彼亦未敢徑前然後徐圖進取可以萬全義誠怒曰洪實為此言欲反邪洪實曰公自欲反乃謂誰反|{
	康義誠之心事宋洪實知之矣}
其聲漸厲帝聞召而訊之|{
	訊問也}
二人訟於帝前|{
	訟者争辯是非曲直}
帝不能辯其是非遂斬洪實|{
	帝但以階級為曲直而不能察事之是非}
軍士益憤怒|{
	觀上文軍士楊言所云但欲迎降潞王何暇憤朱洪實之枉死盖憤怒者洪實之從兵耳}
壬戌潞王至昭應|{
	宋大中祥符八年改昭應縣為臨潼縣九域志在長安東五十里}
聞前軍獲王思同王曰思同雖失計然盡心所奉亦可嘉也癸亥至靈口|{
	九域志臨潼縣之零口鎮是也}
前軍執思同以至王責讓之對曰思同起行間|{
	行戶剛翻}
先帝擢之位至節將|{
	節將言建節而為大將將即亮翻}
常愧無功以報大恩非不知附大王立得富貴助朝廷自取禍殃但恐死之日無面目見先帝於泉下耳|{
	潞王聞王思同之言豈不内愧乎}
敗而釁鼓固其所也請早就死王為之改容曰公且休矣王欲宥之而楊思權之徒恥見其面|{
	楊思權等背順附逆故恥見思同}
王之過長安|{
	過古禾翻又如字}
尹暉盡取思同家資及妓妾屢言於劉延朗曰若留思同|{
	留者言活之使留於人世妓渠綺翻}
慮失士心屬王醉|{
	屬之欲翻}
不待報擅殺思同及其妻子王醒怒延朗嗟惜者累日 癸亥制以康義誠為鳳翔行營都招討使以王思同副之甲子潞王至華州獲藥彦稠囚之乙丑至閿鄉|{
	九域志華州東至閿鄉九十里自閿鄉東至陜州一百七十里華戶化翻閿武巾翻亦作閺}
朝廷前後所發諸軍遇西軍皆迎降無一人戰者丙寅康義誠引侍衛兵發洛陽詔以侍衛馬軍指揮使安從進為京城巡檢從進已受潞王書潛布腹心矣是日潞王至靈寶|{
	靈寶縣在陜州西四十五里}
護國節度使安彦威匡國節度使安重霸皆降|{
	莊宗同光四年安重霸以秦州降重直龍翻}
惟保義節度使康思立謀固守陜城以俟康義誠|{
	陜失冉翻}
先是捧聖五百騎戍陜西為潞王前鋒至城下呼城上人曰禁軍十萬已奉新帝爾輩數人奚為徒累一城人塗地耳|{
	先悉薦翻累力瑞翻}
於是捧聖卒爭出迎思立不能禁不得已亦出迎丁卯潞王至陜僚佐說王曰今大王將及京畿傳聞乘輿已播遷|{
	說式芮翻乘繩證翻}
大王宜少留於此先移書慰安京城士庶王從之移書諭洛陽文武士庶惟朱弘昭馮贇兩族不赦外自餘勿有憂疑康義誠軍至新安|{
	新安縣西距陜州二百餘里}
所部將士自相結百什為羣弃甲兵争先詣陜降纍纍不絶義誠至乾壕|{
	九域志陜州陜縣有乾壕鎮乾音干}
麾下纔數十人遇潞王候騎十餘人義誠解所佩弓劒為信因候騎請降於潞王戊辰閔帝聞潞王至陝義誠軍潰憂駭不知所為急遣使召朱弘昭謀所向弘昭曰急召我欲罪之也赴井死安從進聞弘昭死殺馮贇於第滅其族 |{
	考異曰張昭閔帝實録帝召弘昭不至俄聞自殺乃令從進殺贇按從進傳贇首於陜則贇死非閔帝之命明矣今不取}
傳弘昭贇首於潞王帝欲奔魏州召孟漢瓊使詣魏州為先置|{
	先置者先路置頓也}
漢瓊不應召單騎奔陜初帝在藩鎮愛信牙將慕容遷及即位以為控鶴指揮使帝將北度河密與之謀使帥部兵守玄武門|{
	玄武門洛陽宫城北門帥讀曰率}
是夕帝以五十騎出玄武門謂遷曰朕且幸魏州徐圖興復汝帥有馬控鶴從我遷曰生死從大家乃陽為團結帝既出即闔門不行|{
	史言自古以來衆叛親離未有甚於此時}
己巳馮道等入朝及端門聞朱馮死帝已北走道及劉昫欲歸|{
	昫香句翻又許羽翻}
李愚曰天子之出吾輩不預謀今太后在宫吾輩當至中書遣小黄門取太后進止然後歸第人臣之義也道曰主上失守社稷人臣惟君是奉無君而入宫城恐非所宜|{
	唐之兩都三省及寺監皆在宫城之内}
潞王已處處張榜不若歸俟教令乃歸至天宮寺安從進遣人語之曰|{
	語牛倨翻}
潞王倍道而來且至矣相公宜帥百官至穀水奉迎|{
	穀水在洛陽城西}
乃止於寺中召百官中書舍人盧導至馮道曰俟舍人久矣所急者勸進文書宜速具草|{
	草者草創其辭}
導曰潞王入朝百官班迎可也設有廢立當俟太后敎令豈可遽議勸進乎道曰事當務實導曰安有天子在外人臣遽以大位勸人者邪若潞王守節北面以大義見責將何辭以對公不如帥百官詣宫門進名問安取太后進止則去就善矣|{
	或問馮道李愚盧導之論其於新舊君之際孰為合於義乎曰皆非也此如羣奴之事主家主死而有二子其一養子也其一親子也養子與親子争家政養子勝而親子不勝一奴曰皆郎君也吾從其勝者而輔之一奴之心本亦附勝者而不敢公言附之也曰吾將决諸主母馮道李愚之謂也或曰盧導之言何如曰盧導之不肯草勸進文書是也若其持論則猶李愚也至於言去就之善若是者得為善乎其言之非殆有甚於李愚矣曰然則為馮道李愚者當何如曰若漢人之論相主在與在主亡與亡可也然亦僅可而已未能盡相道也夫子之言曰危而不持顛而不扶則將焉用彼相矣明乎此則為相者貴於持危扶顛不以但能盡死為貴也}
道未及對從進屢遣人趣之曰潞王至矣太后太妃已遣中使迎勞矣|{
	趣讀曰促勞力到翻}
安得百官無班道等即紛然而去既而潞王未至三相息於上陽門外|{
	三相馮道李愚劉昫也上陽門上陽宫門也上陽宫在洛陽宫城西}
盧導過於前道復召而語之|{
	復扶又翻語牛倨翻}
導對如初李愚曰舍人之言是也吾輩之罪擢髪不足數|{
	用戰國須賈之言擢拔也數所具翻}
康義誠至陜待罪潞王責之曰先帝晏駕立嗣在諸公今上亮陰政事出諸公何為不能終始陷吾弟至此乎義誠大愳叩頭請死王素惡其為人|{
	惡烏路翻}
未欲遽誅且宥之馬步都虞侯萇從簡左龍武統軍王景戡皆為部下所執降於潞王東軍盡降|{
	東軍謂自洛陽來者}
潞王上牋於太后取進止遂自陜而東夏四月庚午朔未明閔帝至衛州東數里遇石敬瑭|{
	石敬瑭自河東來朝至此而遇帝}
帝大喜問以社稷大計敬瑭曰聞康義誠西討何如陛下何為至此帝曰義誠亦叛去矣敬瑭俛首長歎數四|{
	俛音免}
曰衛州刺史王弘䞇宿將習事請與圖之|{
	王弘䞇從敬瑭伐蜀嘗為偏將石敬瑭欲擁帝還衛州以授弘䞇使為之所耳}
乃往見弘䞇問之弘䞇曰前代天子播遷多矣然皆有將相侍衛府庫法物使羣下有所瞻仰今皆無之獨以五十騎自隨雖有忠義之心將若之何敬瑭還見帝於衛州驛|{
	自弘䞇所還見帝}
以弘䞇之言告弓箭庫使沙守榮奔洪進前責敬瑭曰|{
	沙姓古夙沙氏之後史炤曰奔姓也古有賁姓音奔又音肥後遂為奔}
公明宗愛壻|{
	以敬瑭尚明宗女也}
富貴相與共之憂患亦宜相恤今天子播越委計於公冀圖興復乃以此四者為辭|{
	四者謂敬瑭所言無將相侍衛府庫法物從行幸也}
是直欲附賊賣天子耳|{
	直指石敬瑭心術}
守榮抽佩刀欲刺之|{
	刺七亦翻}
敬瑭親將陳暉救之守榮與暉鬭死洪進亦自刎|{
	刎扶粉翻}
敬瑭牙内指揮使劉知遠引兵入盡殺帝左右及從騎獨置帝而去 |{
	考異曰閔帝實録庚午朔四鼓帝至衛州東七八里遇敬瑭竇貞固晉高祖實録始帝欲與少主俱西斷孟津北據壺關南向徵諸侯兵乃啓問康義誠西討作何制置云云蘇逢吉漢高祖實録是夜偵知少帝伏甲欲與從臣謀害晉高祖詐屛人對語方坐庭廡帝密遣御士石敢袖鎚立於後俄頃伏甲者起敢有勇力擁晉祖入一室以巨木塞門敢力當其鋒死之帝解佩刀遇夜晦以在地葦炬未然者奮擊之衆謂短兵也遂散走帝乃匿身長垣下聞帝親將李洪信謂人曰石太尉死矣帝隔垣呼洪信曰太尉無恙乃踰垣出就洪信兵共護晉祖殺建謀者以少主授王弘䞇南唐烈祖實録弘䞇曰今京國阽危百官無主必相率攜神器西向公何不囚少帝西迎潞王此萬全之計敬瑭然其語按為二漢實録者必為二祖飾非今從閔帝實録}
敬瑭遂趣洛陽|{
	趣七喻翻}
是日太后令内諸司至乾壕迎潞王|{
	考異曰廢帝實録三十日太后傳令至并内司迎奉至乾壕帝促令還京按長歷三月辛丑朔四月庚午朔三月無三十日廢帝實録誤也}
王亟遣還洛陽初潞王罷河中歸私第|{
	事見二百七十七卷明宗長興元年}
王淑妃數遣孟漢瓊存撫之|{
	數所角翻}
漢瓊自謂於王有舊恩至澠池西|{
	九域志澠池在洛陽之西一百五十六里澠彌兖翻澠池縣名}
見王大哭欲有所陳王曰諸事不言可知仍自預從臣之列|{
	從才用翻}
王即命斬於路隅 山南西道節度使張䖍釗之討鳳翔也留武定節度使孫漢韶守興元䖍釗既敗奔歸興元與漢韶舉兩鎮之地降于蜀蜀主命奉鑾肅衛馬步都指揮使昭武節度使李肇將兵五千還利州|{
	李肇本鎮昭武蜀主召之入領宿衛今使將兵還鎮以應接梁洋}
右匡聖馬步都指揮使寧江節度使張業將兵一萬屯大漫天以迎之|{
	先是蜀主以兵疲民困不用趙隱取山南之計今乘時而坐得之其庸多矣}
壬申潞王至蔣橋百官班迎於路傳教以未拜梓宫未可相見|{
	傳教謂傳令也王所下令為教}
馮道等皆上牋勸進|{
	終不用盧導之言}
王入謁太后太妃詣西宮伏梓宫慟哭自陳詣闕之由馮道帥百官班見|{
	見賢遍翻}
拜|{
	句絶}
王答拜道等復上牋勸進|{
	復扶又翻}
王立謂道曰予之此行事非獲已俟皇帝歸闕園寢禮終當還守藩服羣公遽言及此甚無謂也癸酉太后下令廢少帝為鄂王 |{
	考異曰閔帝實録云七日廢帝為鄂王今從廢帝實録}
以潞王知軍國事權以書詔印施行|{
	書詔印畫可所用者也閔帝之出奔也蓋以八寶自隨}
百官詣至德宫門待罪|{
	五代會要天成元年中書門下奏請以洛京潛龍舊宅為至德宫蓋明宗舊第也按歐史時潞王入居至德宫}
王命各復其位甲戌太后令潞王宜即皇帝位乙亥即位于柩前帝之發鳳翔也許軍士以入洛人賞錢百緡既至問三司使王玫|{
	玫莫杯翻}
以府庫之實|{
	問其實數}
對有數百萬在既而閱實金帛不過三萬兩匹而賞軍之費計應用五十萬緡帝怒玫請率京城民財以足之數日僅得數萬緡帝謂執政曰軍不可不賞人不可不恤今將柰何執政請據屋為率無問士庶自居及僦者預借五月僦直從之|{
	僦即就翻賃居為僦}
王弘䞇遷閔帝於州廨|{
	廨舌隘翻}
帝遣弘贄之子殿直巒往酖之戊寅巒至衛州謁見|{
	見賢遍翻}
閔帝問來故不對|{
	問巒以所以來之故}
弘䞇數進酒|{
	數所角翻}
閔帝知其有毒不飲巒縊殺之|{
	年二十一}
閔帝性仁厚于兄弟敦睦雖遭秦王忌疾閔帝坦懷待之卒免於患|{
	事見上卷明帝長興三年卒子恤翻}
及嗣位於潞王亦無嫌而朱弘昭孟漢瓊之徒横生猜間|{
	横戶孟翻間古莧翻}
閔帝不能違以致禍敗焉孔妃尚在宫中|{
	妃孔循之女}
潞王使人謂之曰重吉何在|{
	以通鑑書法言之潞王於此當書帝蓋承前史偶失於修改也}
遂殺妃并其四子閔帝之在衛州也惟磁州刺史宋令詢遣使問起居聞其遇害慟哭半日自經死|{
	宋令詢出磁州見上卷上年事閔帝有始終者宋令詢一人而已磁墻之翻}
己卯石敬瑭入朝 庚辰以劉昫判三司 辛巳蜀大赦改元明德 帝之起鳳翔也召興州刺史劉遂清遲疑不至聞帝入洛乃悉集三泉西縣金牛桑林戍兵以歸自散關以南城鎮悉弃之皆為蜀人所有癸未入朝帝欲治罪以其能自歸乃赦之|{
	邊境之臣委弃城鎮乃以其能自歸而不誅安有効死弗去者乎治直之翻}
遂清鄩之姪也 甲申蜀將張業將兵入興元洋州 乙酉改元大赦|{
	改元清泰}
丁亥以宣徽南院使郝瓊權判樞密院前三司使王玫為宣徽北院使鳳翔節度判官韓昭胤為左諫議大夫充端明殿學士 戊子斬河陽節度使判六軍諸衛兼侍中康義誠滅其族|{
	康義誠欲舉宿衛兵迎降以為己功而不免於族滅此傅瑕所以死於鄭厲公之類也}
己丑誅藥彦稠|{
	修河中之怨也}
庚寅釋王景戡萇從簡 有司百方斂民財僅得六萬帝怒下軍巡使獄晝夜督責|{
	凡輸財稽違者則下之軍巡使獄以督責之也下戶嫁翻}
囚繫滿獄至自經赴井而軍士遊市肆皆有驕色市人聚詬之曰汝曹為主力戰立功良苦|{
	詬舌候翻又許候翻為干偽翻下能為同}
反使我輩鞭胷杖背出財為賞汝曹猶揚揚自得獨不愧天地乎是時竭左藏舊物及諸道貢獻乃至太后太妃器服簪珥皆出之|{
	藏徂浪翻珥忍止翻耳當也}
纔及二十萬緡帝患之李專美夜直|{
	李專美本鳳翔掌書記時為樞密直學士}
帝讓之曰卿名有才不能為我謀此留才安所施乎|{
	為于偽翻}
專美謝曰臣駑劣陛下擢任過分|{
	駑音奴分扶問翻}
然軍賞不給非臣之責也竊思自長興之季賞賚亟行卒以是驕|{
	事始見上卷長興四年亟去吏翻卒臧没翻士卒也}
繼以山陵及出師帑藏遂涸|{
	帑它朗翻藏徂浪翻涸戶郭翻以水為諭言枯涸也}
雖有無窮之財終不能滿驕卒之心故陛下拱手於危困之中而得天下|{
	此言在鳳翔時諸軍推戴之事}
夫國之存亡不專繫於厚賞亦在修法度立紀綱陛下苟不改覆車之轍臣恐徒困百姓存亡未可知也|{
	帝起事於鳳翔共事者五人能言及此者獨李專美耳}
今財力盡於此矣宜據所有均給之何必踐初言乎帝以為然壬辰詔禁軍在鳳翔歸命者自楊思權尹暉等各賜二馬一駝錢七十緡下至軍人錢二十緡其在京者各十緡軍士無厭|{
	厭於言翻}
猶怨望為謡言曰除去菩薩扶立生鐵以閔帝仁弱帝嚴有悔心故也|{
	去羌呂翻菩薄乎翻薩桑割翻閔帝小字菩薩}
丙申葬聖德和武欽孝皇帝于徽陵|{
	徽陵在河南府洛陽縣}
廟號明宗帝衰絰護從至陵所宿焉|{
	衰倉回翻從才用翻}
五月丙午以韓昭胤為樞密使以莊宅使劉延朗為樞密副使權知樞密院房暠為宣徽北院使暠長安人也|{
	暠古老翻}
帝與石敬瑭皆以勇力善闘事明宗為左右然心競素不相悦|{
	心競本諸左傳師曠之言競争也}
帝即位敬瑭不得已入朝山陵既畢不敢言歸時敬瑭久病羸瘠|{
	羸倫為翻瘠秦昔翻}
太后及魏國公主屢為之言|{
	魏國公主明宗之女下嫁石敬瑭曹太后所生也歐史公主初號永寧公主是年進封魏國長公主為于偽翻}
而鳳翔將佐多勸帝留之惟韓昭胤李專美以為趙延壽在汴不宜猜忌敬瑭|{
	趙延夀時為宣武帥逼近洛都又其父德鈞在幽州擁彊兵言若猜忌敬瑭趙延壽必懼而生心}
帝亦見其骨立不以為虞乃曰石郎不惟密親兼自少與吾同艱難|{
	少詩照翻}
今我為天子非石郎尚誰託哉乃復以為河東節度使|{
	復扶又翻又如字縱石敬瑭歸鎮乃復疑而徙之此所以速禍也}
戊午以隴州防禦使相里金為保義節度使|{
	賞其先通欵於鳳翔也}
丁未階州刺史趙澄降蜀 戊申以羽林軍使楊思

權為静難節度使|{
	踐鳳翔片紙所書之言也難乃旦翻}
己酉張䖍釗孫漢韶舉族遷于成都 庚戌以司空兼門下侍郎同平章事馮道同平章事充匡國節度使 以天雄節度使兼侍中范延光為樞密使 帝之起鳳翔也悉取天平節度使李從曮家財甲兵以供軍|{
	李從曮家其父茂貞以來再世鎮鳳翔從曮雖移鎮而家財甲兵猶在焉}
將行|{
	謂將東趣洛陽也}
鳳翔之民遮馬請復以從曮鎮鳳翔|{
	復扶又翻}
帝許之至是徙從曮為鳳翔節度使|{
	長興元年從曮自鳳翔入朝徙宣武後徙天平今自天平復還鎮鳳翔}
初明宗為北面招討使|{
	莊宗同光二年始以明宗為北面招討使}
平盧節度使房知温為副都部署帝以别將事之嘗被酒忿争|{
	被皮義翻師古曰被加也被酒者為酒所加}
拔刃相擬及帝舉兵入洛知温密與行軍司馬李沖謀拒之沖請先奉表以觀形埶還言洛中已安定壬戌入廟謝罪帝優禮之知温貢獻甚厚 吳鎮南節度使守中書令東海康王徐知詢卒 蜀人取成州 六月甲戌以皇子左衛上將軍重美為成德節度使同平章兼河南尹判六軍諸衛事 文州都指揮使成延龜舉州附蜀|{
	周文王第五子郕叔武封於郕或言成王封季載於郕其後以國為氏或去邑為成氏}
吳徐知誥將受禪忌昭武節度使兼中書令臨川王

濛|{
	昭武軍利州時屬蜀吳使濛遥領耳}
遣人告濛藏匿亡命擅造兵器丙子降封歷陽公幽于和州命控鶴軍使王宏將兵二百衛之|{
	濛見忌之始見二百七十一卷梁貞明五年}
劉昫與馮道昏姻昫性苛察李愚剛褊道既出鎮|{
	謂出鎮同州也}
二人論議多不合事有應改者愚謂昫曰此賢親家所為更之不亦便乎|{
	傳曰妻父曰昏壻父曰姻凡娶以昏時婦人陰也故謂之昏壻家女之所因故曰姻二父相呼謂之親家更工衡翻下欲更同}
昫恨之由是動成忿争至相詬罵各欲非時求見|{
	見賢遍翻}
事多凝滯帝患之欲更命相問所親信以朝臣閒望宜為相者|{
	聞音問}
皆以尚書左丞姚顗太常卿盧文紀祕書監崔居儉對論其才行互有優劣|{
	行下孟翻}
帝不能决乃寘其名於瑠璃瓶夜焚香祝天且以筯挾之|{
	挾當作梜挾古恊翻記曲禮羮之有菜者用梜注云梜猶箸也今人或謂箸為梜提}
首得文紀次得顗秋七月辛亥以文紀為中書侍郎同平章事居儉蕘之子也|{
	崔蕘見二百五十一卷唐懿宗咸通十年}
帝欲殺楚匡祚|{
	以楚匡祚殺重吉也}
韓昭胤曰陛下為天下父天下之人皆陛下子用法宜存至公匡祚受詔檢校重吉家財不得不爾今族匡祚無益死者恐不厭衆心|{
	厭益涉翻伏也合也}
乙卯長流匡祚於登州丁巳立沛國夫人劉氏為皇后|{
	劉后應州渾元人元一作源}
回鶻入貢者多為河西雜虜所掠詔將軍牛知柔帥禁兵衛送|{
	帥讀曰率}
與邠州兵共討之 吳徐知誥召左僕射兼中書侍郎同平章事宋齊丘還金陵以為諸道都統判官加司空於事皆無所關預|{
	徐知誥疎宋齊丘事始上二月召之還金陵而不使預事者恐其沮止禪代之議故爾}
齊丘屢請退居知誥以南園給之 護國節度使洋王從璋歸德節度使涇王從敏皆罷鎮居洛陽私第帝待之甚薄從敏在宋州預殺重吉帝尤惡之|{
	歸德軍宋州殺重吉於宋見上三月惡烏路翻}
嘗侍晏禁中酒酣顧二王曰爾等皆何物輒據雄藩二王大懼太后叱之曰帝醉矣爾曹速去 蜀置永平軍于雅州以孫漢韶為節度使復以張䖍釗為山南西道節度使同平章事䖍釗固辭不行|{
	孫漢韶張䖍釗同以梁洋降蜀蜀以節鎮授之孫漢韶赴雅州而張䖍釗固辭不赴梁州者無面目以見梁州人士也唐末置永平軍於卭州後徙雅州蓋莊宗滅蜀而廢之今後蜀復置之也}
蜀主得風疾踰年至是增劇甲子立子東川節度使同平章事親衛馬步都指揮使仁贊為太子仍監國|{
	監古銜翻}
召司空同平章事趙季良武信節度使李仁罕保寧節度使趙延隱樞密使王處回捧聖控鶴都指揮使張公鐸奉鑾肅衛指揮副使侯弘實受遺詔輔政是夕殂祕不發喪王處回夜啓義興門告趙季良處回泣不已季良正色曰今彊將握兵專伺時變|{
	伺相吏翻}
宜速立嗣君以絶覬覦|{
	彊將謂李仁罕李肇等覬音冀覦音俞}
豈可但相泣邪處回收淚謝之季良教處回見李仁罕審其詞旨然後告之處回至仁罕第仁罕設備而出遂不以實告|{
	史言李仁罕已遊於趙季良等數内}
丙寅宣遺制命太子仁贊更名昶丁卯即皇帝位|{
	昶蜀主第三子也更工衡翻}
初帝以王玫對左藏見財失實|{
	事見上四月藏徂浪翻見賢遍翻}
故以劉昫代判三司昫命判官高延賞鉤考窮覈皆積年逋欠之數姦吏利其徵責匄取故存之|{
	匄居火翻}
昫具奏其狀且請察其可徵者急督之必無可償者悉蠲之韓昭胤極言其便八月庚午詔長興以前戶部及諸道逋租三百三十八萬虛煩簿籍咸蠲免勿徵貧民大悦而三司吏怨之 辛未以姚顗為中書侍郎同平章事 右龍武統軍索自通以河中之隙|{
	見二百七十七卷明宗長興元年}
心不自安戊子退朝過洛自投于水而卒|{
	洛水貫都城中故自通退朝過之自投于水}
帝聞之大驚贈太尉 丙申以前安國節度使同平章事趙鳳為太子太保 九月癸卯詔鳳翔益兵守東安鎮以備蜀|{
	東安鎮當在鳳翔西界蜀既出關收階成之地故益兵以備之}
蜀衛聖諸軍都指揮使武信節度使李仁罕自恃宿將有功復受顧託|{
	復扶又翻}
求判六軍令進奏吏宋從會以意諭樞密院又至學士院偵草麻|{
	偵丑鄭翻}
蜀主不得已甲寅加仁罕兼中書令判六軍事以左匡聖都指揮使保寧節度使趙延隱兼侍中為之副 己未雲州奏契丹入宼北面招討使石敬瑭奏自將兵屯百井以備契丹辛酉敬瑭奏振武節度使楊檀擊契丹于境上却之蜀奉鑾肅衛都指揮使昭武節度使兼侍中李肇聞蜀

主即位顧望不時入朝至漢州留與親戚燕飲踰旬冬十月庚午始至成都稱足疾扶杖入朝見|{
	見賢遍翻}
見蜀主不拜|{
	李肇之傲幼君亦由武夫倔彊不學無識以自貽禍}
戊寅左僕射門下侍郎同平章事李愚罷守本官吏部尚書兼門下侍郎同平章事判三司劉昫罷為右僕射三司吏聞昫罷相皆相賀無一人從歸第者|{
	以昫奏蠲諸道逋租吏無所緣徵責以漁利也}
蜀捧聖控鶴都指揮使張公鐸與醫官使韓繼勲豐德庫使韓保貞茶酒庫使安思謙等皆事蜀主於藩邸素怨李仁罕共譛之云仁罕有異志蜀主令繼勲等與趙季良趙廷隱謀因仁罕入朝命武士執而殺之|{
	趙廷隱自克東川與李仁罕争功怨隙之深有自來仁罕之求判六軍蜀主命廷隱為之副所以防仁罕仁罕之不覺其冥頑凶悖取死宜矣然趙廷隱終亦不能免近習之讒其得死於牖下者幸也}
癸未下詔暴其罪并其子繼宏及宋從會等數人皆伏誅是日李肇釋杖而拜|{
	李肇事孟知祥於董璋之難陰拱而觀其孰勝董璋既死肇宜不免於死矣孟知祥念其劔州之功不以為罪及事少主釋位入朝倨傲不拜其誰能容之一見李仁罕之誅遽釋杖而拜前倨後恭欲以求免不亦難乎通鑑書之以為武夫恃功驕悖者之戒}
蜀源州都押牙文景琛據城叛|{
	徧考新舊唐志及九域圖志寰宇記皆不載源州建置之由與其地歐史職方考曰州縣凡唐故而廢於五代者若五代所置而見於今者及縣之割隷今因之者皆宜列以備職方之考其餘嘗置而復廢嘗改割而復舊皆不足書則知源州盖蜀所置而尋廢此其所以無傳同光之克蜀也得州六十四見于職方考者五十三州而已如源州等盖皆六十四州之數按薛史後蜀潘仁嗣授武定節度使源壁等州觀察營田處置等使周師攻秦鳳孟貽業駐軍平利為褒源之援則蜀置源州屬武定軍節度}
果州刺史李延厚討平之蜀主左右以李肇倨慢請誅之戊子以肇為太子少

傅致仕徙卭州|{
	卭渠恭翻}
吳主加徐知誥大丞相尚父嗣齊王九錫辭不受 雄武節度使張延朗將兵圍文州|{
	唐末置天雄節度於秦州後唐改為雄武節度}
階州刺史郭知瓊拔尖石寨蜀李延厚將果州兵屯興州遣先登指揮使范延暉將兵救文州延朗解圍而歸興州刺史馮暉自乾渠引戍兵歸鳳翔|{
	時階興二州皆已入于蜀唐蓋使郭知瓊馮暉領二州刺史以進取而不克也薛史曰長興中馮暉為興州刺史以乾渠為治所乾音干}
十一月徐知誥召其子司徒同平章事景通還金陵|{
	自江都還金陵也}
為鎮海寧國節度副大使諸道副都統判中外諸軍事以次子牙内馬步都指揮使海州團練使景遷為左右軍都軍使左僕射參政事留江都輔政 十二月己巳以易州刺史安叔千為振武節度使齊州防禦使尹暉為彰國節度使|{
	安叔千以捍契丹之功尹暉則鳳翔歸命之賞也}
叔千沙陀人也|{
	宋白曰安叔千本貫雲州界戶屬奉誠軍灰泉村}
壬申石敬瑭奏契丹引去罷兵歸|{
	自百井歸晉陽也}
乙亥徵雄武節度使張延朗為中書侍郎同平章事

判三司 辛巳漢皇后馬氏殂|{
	馬氏楚王殷女也}
甲申蜀葬文武聖德英烈明孝皇帝于和陵廟號高祖 乙酉葬鄂王于徽陵城南|{
	唐園陵之制兆域之外繚以垣墻列植栢樹謂之栢城}
封纔數尺觀者悲之 |{
	考異曰閔帝實録及薛史閔帝紀皆云晉高祖即位諡曰閔與秦王及重吉並葬徽陵域中今從廢帝實録}
是歲秋冬旱民多流亡同華蒲絳尤甚|{
	華戶化翻}
漢主命判六軍秦王弘度募宿衛兵千人皆市井無賴子弟弘度昵之|{
	昵尼質翻}
同平章事楊洞潜諫曰秦王國之冢嫡|{
	冢大也}
宜親端士使之治軍已過矣|{
	治直之翻}
况昵羣小乎漢主曰小兒教以戎事過煩公憂終不戒弘度洞潜出見衛士掠商人金帛商人不敢訴歎曰政亂如此安用宰相因謝病歸第久之不召遂卒 二年春正月丙申朔閩大赦改元永和 二月丙寅朔蜀大赦甲戌以樞密使天雄節度使兼侍中范延光為宣武

節度使兼中書令 丁丑夏州節度使李彛超上言疾病|{
	夏戶雅翻上時掌翻疾甚為病}
以兄行軍司馬彛殷權知軍州事彛超尋卒 戊寅蜀主尊母李氏為皇太后太后太原人本莊宗後宫也以賜蜀高祖|{
	孟知祥事莊宗夙蒙親任故以後宫賜之史詳書李氏之所自來以别於福慶長公主}
己丑追尊帝母魯國夫人魏氏曰宣憲皇太后|{
	魏氏本平山王氏婦也少寡與帝皆為明宗所掠}
閩主立淑妃陳氏為皇后初閩主兩娶劉氏皆士族美而無寵陳后本閩太祖侍婢金鳳也陋而淫閩主嬖之|{
	嬖皮藏翻又必計翻}
以其族人守恩匡勝為殿使|{
	殿使閩所置官}
三月辛丑以前宣武節度使兼侍中趙延壽為忠武節度使兼樞密使以李彛殷為定難節度使|{
	李彛殷後避宋朝廟諱改名彛興其子則李繼捧李繼遷也難乃旦翻}
己酉贈吳越王元瓘母陳氏為晉國太夫人元瓘性孝尊禮母黨厚加賜與而未嘗遷官授以重任壬戌以彰聖都指揮使安審琦領順化節度使|{
	五代會要}


|{
	清泰元年六月改捧聖馬軍為彰聖左右軍嚴衛步軍為寧衛左右軍梁嘗改滄州義昌軍為順化軍後唐復唐之舊為横海軍前此吳越錢元珦判明州領順化節度使審琦所領蓋楚州順化軍也}
審琦金全之子也|{
	安金全代北舊將}
太常丞史在德性狂狷上書歷詆内外文武之士|{
	薛史載在德書其畧曰朝廷任事率多濫進稱武士者不閑計策雖被堅執銳戰則弃甲窮則背軍稱文士者鮮有藝能多無士行問策謀則杜口作文字則倩人所謂虛設具員枉費國力逢陛下維新之運是文明革弊之秋臣請應内外所管軍人凡勝衣甲者請宣下本軍大將一一考試武藝短長權謀深淺居下位有將才者便拔為大將居上位無將畧者移之下軍其東班臣僚請内出策題下中書令宰臣面試如下位有大才須拔居大位無大才即移之下僚狷吉掾翻詆下禮翻}
請徧加考試黜陟能否執政及朝士大怒盧文紀及補闕劉濤楊昭儉等皆請加罪帝謂學士馬胤孫曰朕新臨天下宜開言路若朝士以言獲罪誰敢言者卿為朕作詔書宣朕意|{
	馬胤孫時為翰林學士為于偽翻}
乃下詔畧曰㫺魏徵請賞皇甫德參|{
	見一百九十四卷太宗貞觀八年}
今濤等請黜史在德事同言異何其遠哉在德情在傾輸安可責也|{
	傾輸謂傾其胷腹所懷而輸忠於上}
昭儉嗣復之曾孫也|{
	楊嗣復文宗時為相}
吳加徐景遷同平章事知左右軍事徐知誥令尚書郎陳覺輔之 |{
	考異曰江南録時先生權位日隆中外皆知有代謝之勢而以吳主恭謹守道欲待嗣君先主次子景遷吳主之壻也先主鍾愛特甚齊丘使陳覺為景遷教授為之聲價齊丘參决時政多為不法輒歸過於嗣主而盛稱景遷之美幾有奪嫡之計所以然者以吳主少而先主老必不能待他日得國授于景遷易制已為元老威權無上矣此其日夕為謀也先主覺之乃召齊丘如金陵以為已之副遥兼申蔡節度使無所關預從容而已今從十國紀年}
謂覺曰吾少時與宋子嵩論議好相詰難|{
	少詩照翻好呼到翻詰去吉翻難乃旦翻}
或吾捨子嵩還家或子嵩拂衣而起子嵩攜衣笥望秦淮門欲去者數矣吾常戒門者止之|{
	宋齊丘字子嵩秦淮門金陵城門數所角翻}
吾今老矣猶未徧逹時事况景遷年少當國故屈吾子以誨之耳 夏四月庚午蜀以御史中丞龍門毋昭裔為中書侍郎同平章事|{
	龍門縣本漢皮氏縣後魏更名唐屬河中府九域志在府東北九十五里毋姓也毋丘氏望出平昌鉅鹿開元補闕有毋景洛陽人一云吳人毋武夫翻}
癸未加樞密使刑部尚書韓昭胤中書侍郎同平章事辛卯以宣徽南院使劉延皓為刑部尚書充樞密使延皓皇后之弟也癸巳以左領軍衛大將軍劉延朗為本衛上將軍充宣徽北院使兼樞密副使 五月丙申契丹寇新州及振武 庚戌賜振武節度使楊檀名光遠|{
	薛史載中書奏準天成三年正月敕凡廟諱但回避正文其偏旁文字不在減少點畫今鄭州節度使楊檀檀州金壇等名酌情制宜並請改之其表章文案偏旁字缺點畫凡臣僚名涉偏旁亦請改名詔曰偏旁文字音韻懸殊止避正呼不宜全改楊檀宜賜名光遠餘依舊按此以明宗廟諱亶字避偏旁也楊檀時不鎮定州當從通鑑}
六月吳德勝節度使兼中書令柴再用卒先是史官王振嘗詢其戰功|{
	先悉薦翻}
再用曰鷹犬微効皆社稷之靈再用何功之有竟不報|{
	有功而不求聞武人如柴再用者亦可稱也}
契丹寇應州 河東節度使北面總管石敬瑭既還鎮|{
	去年五月帝令石敬瑭還太原}
隂為自全之計帝好咨訪外事|{
	好呼到翻}
常命端明殿學士李專美翰林學士李崧知制誥呂琦薛文遇翰林天文趙延乂等|{
	唐之中世司天臺有天文博士二人正八品下天文觀生九十人天文生五十人皆掌候天文翰林天文居翰林院以候天文者也}
更直於中興殿庭與語或至夜分時敬瑭二子為内使|{
	更工衡翻内使内諸司使按石敬瑭拒命之時其子重殷為右衛上將軍重裔為皇城副使}
曹太后則晉國長公主之母也|{
	敬瑭妻魏國公主是年四月進封晉國長知兩翻}
敬瑭賂太后左右令伺帝之密謀事無巨細皆知之敬瑭多於賓客前自稱羸瘠不堪為帥|{
	羸倫為翻瘠在亦翻帥所類翻}
冀朝廷不之忌時契丹屢寇北邊禁軍多在幽拜敬瑭與趙德鈞求益兵運粮朝夕相繼|{
	敬瑭求兵粮以實幷州趙德鈞求兵粮以實幽州}
甲申詔借河東人有蓄積者菽粟乙酉詔鎮州輸絹五萬匹於總管府糴軍粮|{
	總管府在晉陽石敬瑭時為北面馬步軍都總管故也}
率鎮冀人車千五百乘運粮於代州|{
	九域志鎮州西北至代州六百二十里乘繩證翻}
又詔魏博市糴時水旱民飢敬瑭遣使督趣嚴急|{
	趣讀曰促}
山東之民流散|{
	此謂太行常山之東}
亂始兆矣|{
	史叙致亂之由}
敬瑭將大軍屯忻州朝廷遣使賜軍士夏衣傳詔撫諭軍士呼萬歲者數四|{
	時驕兵習於聞見又欲扶立石敬瑭以希賞}
敬瑭懼幕僚河内段希堯請誅其唱首者敬瑭命都押衙劉知遠斬挾馬都將李暉等三十六人以徇希堯懷州人也帝聞之益疑敬瑭 壬辰詔竊盜不計贓多少并縱火彊盜並行極法 閩福王繼鵬私於宮人李春鷰繼鵬請之於陳后后白閩主而賜之 秋七月以樞密使劉延皓為天雄節度使 乙巳以武寧節度使張敬逹為北面行營副總管將兵屯代州以分石敬瑭之權|{
	為令張敬逹討石敬瑭張本}
帝深以時事為憂嘗從容讓盧文紀等以無所規贊|{
	從才容翻}
丁巳文紀等上言臣等每五日起居與兩班旅見暫獲對揚|{
	見賢遍翻兩班者文武官分為東西兩班書說命說拜稽首曰敢對揚天子之休命注云對答也答受美命而稱揚之後人遂以面對為對揚}
侍衛滿前雖有愚慮不敢敷陳竊見前朝自上元以來置延英殿或宰相欲有奏論天子欲有咨度|{
	上元唐肅宗年號度徒洛翻}
旁無侍衛故人得盡言望復此故事惟聽機要之臣侍側|{
	機要之臣謂樞密}
詔以舊制五日起居百僚俱宰相獨升若常事自可敷奏或事應嚴密不以其日或異日聽于閣門奏牓子當盡屏侍臣|{
	屏必郢翻又卑正翻}
於便殿相待何必襲延英之名也 吳潤州團練使徐知諤狎眤小人|{
	眤尼質翻}
游燕廢務作列肆于牙城西躬自貿易|{
	貿音茂}
徐知誥聞之怒召知諤左右詰責知諤懼或謂知誥曰忠武王最愛知諤|{
	徐温謚忠武王}
而以後事傳于公|{
	徐知誥之得政在於定朱瑾之難若徐温臨没而傳政于知誥非本心也事見二百七十六卷明宗天成二年}
往年知詢失守|{
	謂自昇州召知詢還楊州也}
論議至今未息借使知諤治有能名|{
	治直吏翻}
訓兵養民於公何利知誥感悟待之加厚九月丙申吳大赦改元天祚 己酉以宣徽南院使

房暠為刑部尚書充樞密使|{
	暠古老翻}
宣徽北院使劉延朗為南院使仍兼樞密副使於是延朗及樞密直學士薛文遇等居中用事暠與趙延壽雖為使長|{
	樞密使為樞密院之長長知兩翻}
其聽用之言什不三四暠隨執可否不為事先每幽并遣使入奏樞密諸人環坐議之暠多俛首而寐比覺引頸振衣則使者去矣|{
	俛音免比必利翻覺居效翻}
啓奏除授一歸延朗|{
	為劉延朗受誅於晉房暠獲全張本然二人皆帝之親臣也延朗之好貨非也暠之避事亦非矣}
諸方鎮刺史自外入者必先賂延朗後議貢獻賂厚者先得内地賂薄者晩得邊陲由是諸將帥皆怨憤帝不能察 蜀金州防禦使全師郁寇金州拔水寨|{
	按元和郡縣志漢水去金州城百步故唐置水寨以防蜀兵}
城中兵纔千人都監陳知隱託它事將兵三百沿流遁去防禦使馬全節罄私財以給軍出奇死戰蜀兵乃退戊寅詔斬知隱 初閩主有幸臣曰歸守明出入卧内閩主晩年得風疾陳后與守明及百工院使李可殷私通國人皆惡之莫敢言|{
	惡烏路翻}
可殷嘗譛皇城使李倣於閩主后族陳匡勝無禮於福王繼鵬倣及繼鵬皆恨之閩主疾甚繼鵬有喜色倣以閩主為必不起冬十月己卯使壯士數人持白梃擊李可殷殺之|{
	梃待鼎翻}
中外震驚庚辰閩主疾少間|{
	間如字}
陳后訴之閩主力疾視朝詰可殷死狀倣懼而出俄頃引部兵鼓譟入宫閩主聞變匿於九龍帳下|{
	閩主命錦工作九龍帳國人歌之曰誰謂九龍帳惟貯一歸郎歸郎謂守明也}
亂兵刺之而出|{
	刺七亦翻}
閩主宛轉未絶宫人不忍其苦為絶之|{
	為絶其命也為于偽翻}
倣與繼鵬殺陳后陳守恩陳匡勝歸守明及繼鵬弟繼韜繼韜素與繼鵬相惡故也辛巳繼鵬稱皇太后令監國是日即皇帝位|{
	皇太后璘母黄氏也繼鵬璘之長子}
更名昶|{
	更工衡翻}
謚其父曰齊肅明孝皇帝廟號惠宗既而自稱權知福建節度事遣使奉表於唐大赦境内立李春鷰為賢妃初閩惠宗娶漢主女清遠公主|{
	廣州有清遠縣}
使宦者閩清林延遇置邸於番禺|{
	唐志無閩清縣蓋王氐始分置也九域志閩清縣屬福州在州西北一百五十里宋白曰唐貞元元年割官縣十鄉為梅溪場梁乾化元年改為閩清縣番音潘}
專掌國信漢主賜以大第稟賜甚厚數問以閩事|{
	禀筆錦翻給也數所角翻}
延遇不對謂人曰去閩語閩去越語越處人宫禁可如是乎|{
	處昌呂翻}
漢主聞而賢之以為内常侍使鉤校諸司事延遇聞惠宗遇弑求歸不許素服向其國三日哭|{
	史言林延遇不忘舊君}
荆南節度使高從誨性明逹親禮賢士委任梁震以兄事之震常謂從誨為郎君|{
	門生故吏呼其主之子為郎君梁震事高季興從誨之父也故以郎君呼從誨}
楚王希範好奢靡|{
	好呼到翻下玩好同}
游談者共誇其盛從誨謂僚佐曰如馬王可謂大丈夫矣孫光憲對曰天子諸侯禮有等差彼乳臭子驕侈僭忲|{
	忲它蓋翻奢也}
取快一時不為遠慮危亡無日又足慕乎從誨久而悟曰公言是也它日謂梁震曰吾自念平生奉養固已過矣乃捐去玩好|{
	去羌呂翻好呼到翻}
以經史自娛省刑薄賦境内以安梁震曰先王待我如布衣交以嗣王屬我|{
	先王謂高季興嗣王謂從誨屬之欲翻}
今嗣王能自立不墜其業吾老矣不復事人矣|{
	復扶又翻}
遂固請居從誨不能留乃為之築室於土洲|{
	為于偽翻江陵有九十九洲土洲其一也梁震事高氏始二百六十六卷梁太祖開平二年}
震披鶴氅|{
	氅昌兩翻}
自稱荆臺隱士每詣府跨黄牛至聽事|{
	聽讀曰廳}
從誨時過其家|{
	過音戈}
四時賜與甚厚自是悉以政事屬孫光憲|{
	屬之欲翻}
臣光曰孫光憲見微而能諫高從誨聞善而能徙|{
	高從誨之羨馬希範是侈心之萌芽也而孫光憲力言之以防微高從誨因光憲之言捐玩好而樂經史思所以阜民保境是遷善也}
梁震成功而能退|{
	梁震翼贊高氏父子能保其國是功也}
自古有國家者能如是夫何亡國敗家喪身之有|{
	喪息浪翻}


吳加中書令徐知誥尚父太師大丞相大元帥進封齊王備殊禮以昇潤宣池歙常江饒信海十州為齊國|{
	考徐知誥所封十州自潤循江而上至于江則中斷吳國之腰膂江都之與洪卾脈理不屬矣自常潤波海界淮而有海州則有包舉吳國之勢其規圖自以為得當是時合全吳之人歸心知誥何必如是而後簒也歙書涉翻}
知誥辭尚父丞相殊禮不受 閩皇城使判六軍諸衛李倣專制朝政陰養死士|{
	朝直遥翻}
閩主昶與拱宸指揮使林延皓等圖之延皓等詐親附倣倣待之不疑十一月壬子倣入朝延皓等伏衛士數百於内殿執斬之梟首朝門|{
	梟堅堯翻朝門正朝之門朝直遥翻}
倣部兵千餘持白梃攻應天門不克焚啓聖門奪倣首奔吳越詔暴倣弑君及殺繼韜等罪告諭中外|{
	此閩主之詔也}
以建王繼嚴權判六軍諸衛以六軍判官永泰葉翹為内宣徽使參政事|{
	唐懿宗咸通二年分連江及閩置永泰縣屬福州九域志在州西南三百五十里福州圖經云永泰縣唐永泰二年置以年號為名翹祈消翻}
翹博學質直閩惠宗擢為福王友|{
	閩主昶初封福王}
昶以師傳禮待之多所禆益宫中謂之國翁昶既嗣位驕縱不與翹議國事一旦昶方視事翹衣道士服過庭中趨出|{
	衣於既翻}
昶召還拜之曰軍國事殷久不接對孤之過也翹頓首曰老臣輔導無狀致陛下即位以來無一善可稱願乞骸骨昶曰先帝以孤屬公|{
	屬之欲翻}
政令不善公當極言奈何弃孤去厚賜金帛慰諭令復位昶元妃梁國夫人李氏同平章事敏之女昶嬖李春鷰|{
	昶求春鷰于陳后見上六月嬖卑義翻又博計翻}
待夫人甚薄翹諫曰夫人先帝之甥聘之以禮奈何以新愛而弃之昶不悦由是疎之未幾復上書言事|{
	幾居豈翻復扶又翻}
昶批其紙尾曰一葉隨風落御溝|{
	批匹迷翻筆題之也}
遂放歸永泰|{
	路振九國志葉翹斥歸永春按九域志泉州有永春縣福州有永泰縣未知孰是}
以壽終 帝嘉馬全節之功|{
	却蜀兵全金州之功也}
召詣闕劉延朗求賂全節無以與之延朗欲除全節絳州刺史羣議沸騰帝聞之乙卯以全節為横海留後|{
	帝既聞之而不罪劉延朗善善惡惡郭之所以亡也}
十二月壬申以中書侍郎同平章事充樞密使韓昭胤同平章事充護國節度使 乙酉以前匡國節度使同平章事馮道為司空時久無正拜三公者|{
	喪亂以來以它官兼領及檢校三公者有之無正拜者}
朝議疑其職事盧文紀欲令掌祭祀掃除|{
	隋制三公參議國之大事祭祀則太尉亞獻司徒奉俎司空行掃除盧文紀不深考遂以為司空職掌朝直遥翻}
道聞之曰司空掃除職也吾何憚焉既而文紀自知不可乃止|{
	史言後唐雖自言纂唐舊服而文獻皆不足}
閩主賜洞真先生陳守元號天師信重之乃至更易

將相|{
	更工衡翻}
刑罰選舉皆與之議守元受賂請託言無不從其門如市

資治通鑑卷二百七十九
