資治通鑑卷八十
宋 司馬光 撰

胡三省 音註

晉紀二|{
	起昭陽大荒落盡屠維大淵獻凡七年}


世祖武皇帝上之下

泰始九年春正月辛酉密陵元侯鄭袤卒 |{
	考異曰按本傳袤為司空固辭久之見許以侯就第拜儀同三司而帝紀云司空鄭袤薨誤也}
二月癸巳樂陵武公石苞卒 三月立皇子祗為東海王 吳以陸抗為大司馬荆州牧 夏四月戊辰朔日有食之 初鄧艾之死|{
	事見七十八卷魏元帝咸熙元年}
人皆寃之而朝廷無為之辯者|{
	為于偽翻}
及帝即位議郎敦煌段灼上疏曰|{
	敦徒門翻}
鄧艾心懷至忠而荷反逆之名|{
	荷下可翻}
平定巴蜀而受三族之誅艾性剛急矜功伐善不能協同朋類故莫肯理之臣竊以為艾本屯田掌犢人|{
	鄧艾本義陽棘陽人魏太祖破荆州徙汝南為農民養犢}
寵位已極功名已成七十老公復何所求|{
	復扶又翻}
正以劉禪初降|{
	降戶江翻}
遠郡未附矯令承制權安社稷鍾會有悖逆之心|{
	悖蒲内翻又蒲没翻}
畏艾威名因其疑似構成其事艾被詔書即遣彊兵束身就縛不敢顧望|{
	被皮義翻}
誠知奉見先帝必無當死之理也會受誅之後艾官屬將吏愚戇相聚自共追艾破壞檻車解其囚執|{
	戇直降翻壞音怪}
艾在困地狼狽失據|{
	狼前則跋其胡退則疐其尾狽狼屬也生子或欠一足二足相附而後能行離則顛蹶故猝遽謂之狼狽狽博蓋翻}
未嘗與腹心之人有平素之謀獨受腹背之誅|{
	腹在前背在後謂前後皆不免於誅}
豈不哀哉陛下龍興闡弘大度謂可聽艾歸葬舊墓還其田宅以平蜀之功繼封其後使艾闔棺定謚死無所恨|{
	謚神至翻}
則天下狥名之士思立功之臣必投湯火樂為陛下死矣|{
	樂音洛為于偽翻}
帝善其言而未能從會帝問給事中樊建以諸葛亮之治蜀|{
	樊建故蜀臣治直之翻}
曰吾獨不得如亮者而臣之乎建稽首曰陛下知鄧艾之寃而不能直|{
	稽音啓}
雖得亮得無如馮唐之言乎|{
	言不能用也馮唐事見十四卷漢文帝十四年}
帝笑曰卿言起我意乃以艾孫朗為郎中 吳人多言祥瑞者吳主以問侍中韋昭昭曰此家人筐篋中物耳|{
	言祥瑞而謂之家人筐篋中物者蓋稱引圖緯以言祥瑞之應故謂其書為家人筐篋中物也}
昭領左國史|{
	吳有左右國史皆掌記述}
吳主欲為其父作紀|{
	為于偽翻}
昭曰文皇不登極位當為傳不當為紀|{
	吳主謚其父和曰文皇帝傳直戀翻}
吳主不悦漸見責怒昭憂懼自陳衰老求去侍史二官|{
	侍史侍中及左國史也}
不聽時有疾病醫藥監護持之益恐|{
	監工衘翻}
吳主飲羣臣酒|{
	飲于禁翻}
不問能否率以七升為限至昭獨以茶代之後更見偪強|{
	強其兩翻}
又酒後常使侍臣嘲弄公卿發摘私短以為歡|{
	摘當作擿}
時有愆失輒見收縛至於誅戮昭以為外相毁傷内長尤恨|{
	長丁丈翻令知兩翻}
使羣臣不睦不為佳事故但難問經義而已|{
	難乃旦翻}
吳主以為不奉詔命意不忠盡積前後嫌忿遂收昭付獄昭因獄上辭|{
	辭獄辭也上時掌翻}
獻所著書冀以此求免而吳主怪其書垢故|{
	垢塵也故舊也}
更被詰責|{
	被皮義翻詰去吉翻}
遂誅昭徙其家於零陵 五月以何曾領司徒 六月乙未東海王祗卒 秋七月丁酉朔日有食之 |{
	考異曰宋志無此食今從晉書}
詔選公卿以下女備六宫有蔽匿者以不敬論|{
	以律不敬論罪也}
采擇未畢權禁天下嫁娶帝使楊后擇之后惟取潔白長大而捨其美者帝愛卞氏女欲留之后曰卞氏三世后族|{
	魏武帝卞后諡曰宣后弟秉生蘭及琳蘭孫女為高貴鄉公后琳女又為陳留王后凡三世}
不可屈以卑位帝怒乃自擇之中選者以絳紗繫臂|{
	中竹仲翻}
公卿之女為三夫人|{
	孔頴逹曰夫扶也言扶侍於王也}
九嬪|{
	句斷}
二千石將校女補良人以下|{
	漢制後宫之號十有四等良人視八百石爵比庶長師古曰良善也將即亮翻校戶敎翻}
九月吳主悉封其子弟為十一王王給三千兵大赦|{
	十一王史逸其名}
是歲鄭冲以夀光公罷 吳主愛姬遣人至市奪民物司市中郎將陳聲素有寵於吳主繩之以法姬愬於吳主|{
	愬與訴同}
吳主怒假它事燒鋸斷聲頭投其身於四望之下|{
	據晉書温嶠傳嶠討蘇峻於石頭結壘於四望磯又據内史石頭有四望山蓋山下有磯也斷丁管翻}


十年春正月乙未日有食之 閏月癸酉夀光成公鄭冲卒 丁亥詔曰近世以來多由内寵以登后妃|{
	謂魏三祖立卞郭毛為后}
亂尊卑之序自今不得以妾媵為正嫡|{
	媵以證翻}
分幽州置平州|{
	幽州言北方太隂幽宴也杜佑曰因幽都山為名山海經有幽都山今列北荒統范陽燕北平上谷代遼西漢末公孫度自號平州牧今分昌黎遼東樂浪玄菟帶方五郡置平州}
三月癸亥日有食之 詔又取良家及小將吏女五千人入宫選之母子號哭於宫中聲聞於外|{
	將即亮翻號戶刀翻聞音問}
夏四月己未臨淮康公荀顗卒|{
	諡法温柔好樂曰康顗魚豈翻}
吳

左夫人王氏卒吳主哀念數月不出葬送甚盛時何氏以太后故宗族驕横|{
	横戶孟翻}
吳主舅子何都貌類吳主民間訛言吳主已死立者何都也會稽又訛言章安侯奮當為天子奮母仲姬墓在豫章豫章太守張俊為之掃除|{
	掃糞掃也除芟除荆棘會古外翻為于偽翻}
臨海太守奚熙|{
	吳主休永安三年分會稽東部都尉為臨海郡}
與會稽太守郭誕書|{
	會工外翻}
非議國政誕但白熙書不白妖言|{
	妖言即前訛言妖於驕翻}
吳主怒收誕繫獄誕懼功曹邵疇曰疇在明府何憂遂詣吏自列白|{
	自列猶自陳也}
疇厠身本郡位極朝右|{
	郡功曹位居郡朝之右朝直遥翻}
以噂?之語|{
	噂祖本翻?逹合翻噂?聚語也}
本非事實疾其醜聲不忍聞見欲含垢藏疾|{
	左傳曰川澤納汙山藪藏疾國君含垢}
不彰之翰墨鎮躁歸静使之自息故誕屈其所是默以見從|{
	謂誕從疇之說默而不白妖言也}
此之為愆實由於疇不敢逃死歸罪有司因自殺吳主乃免誕死送付建安作船|{
	宋白曰吳分侯官之地立建安縣又立曲郍都尉主謫徙之人作舟船}
遣其舅三郡督何植收奚熙|{
	江表傳作備海督蓋督臨海建安會稽三郡也}
熙發兵自守其部曲殺熙送首建業又車裂張俊皆夷三族并誅章安侯奮及其五子 |{
	考異曰江表傳曰張布女有寵於皓而死皓厚葬之國人見葬太奢麗皆謂皓已死所葬者是也皓舅子何都顔狀似皓故民間訛言都代立臨海太守奚熙信訛言舉兵欲還秣陵誅都都叔父植時為備海督擊殺熙夷三族訛言乃息又云奮本在章安徙還吳城禁錮使男女不得通婚或年三十四十不得嫁娶奮上表乞自比禽獸使男女自相配偶皓大怒遣察戰齎藥賜奮父子皆飲藥死裴松之按建衡二年至奮之死孫皓即位尚未久若奮未被疑之前男女年二十左右至奮死時不得年三十四十也若先已長大自失時未婚娶不由皓之禁錮矣此雖欲增皓之惡然非實理又吳志孫皓傳鳳凰三年會稽妖言奮為天子遂誅奚熙不言誅奮孫奮傳建衡二年左夫人王氏卒民間訛言遂誅奮及五子三十國晉春秋自皓納張布女至殺奮皆在天冊元年按奮若以建衡二年死不容至鳳凰三年會稽方冇訛言不知奮死果在何年今因奚熙之死終言之}
秋七月丙寅皇后楊氏殂初帝以太子不慧恐不堪為嗣常密以訪后|{
	常當作嘗}
后曰立子以長不以賢|{
	春秋公羊傳之言長知兩翻}
豈可動也鎮軍大將軍胡奮女為貴嬪|{
	晉制貴人夫人貴嬪是為三夫人皆金章紫綬嬪毗賓翻}
有寵於帝后疾篤恐帝立貴嬪為后致太子不安枕帝膝泣曰|{
	枕職任翻}
叔父駿女芷有德色|{
	言有德冇色也}
願陛下以備六宫帝流涕許之 以前太常山濤為吏部尚書濤典選十餘年|{
	帝受禪濤自吏部郎遷尚書居母喪復奪情起典選選息絹翻}
每一官缺輒擇才資可為者啓擬數人|{
	才謂其才足以任資謂其資序當為者}
得詔旨有所向然後顯奏之帝之所用或非舉首衆情不察以濤輕重任意言之於帝帝益親愛之濤甄拔人物各為題目而奏之時稱山公啓事|{
	甄稽延翻明也察也别也}
濤薦嵇紹於帝請以為祕書郎|{
	晉制祕書監屬官有丞有郎}
帝發詔徵之紹以父康得罪|{
	事見七十八卷魏元帝景元三年}
屏居私門欲辭不就|{
	屏必郢翻}
濤謂之曰為君思之久矣天地四時猶有消息况於人乎|{
	為于偽翻下樹為人為同又蠻為同}
紹乃應命帝以為秘書丞初東關之敗|{
	事見七十五卷魏郡陵厲公嘉平四年}
文帝問僚屬曰近日之事誰任其咎|{
	任音壬}
安東司馬王儀修之子|{
	王修見六十四卷漢獻帝建安八年}
對曰責在元帥|{
	文帝時為安東將軍監諸軍}
文帝怒曰司馬欲委罪孤邪引出斬之儀子裒痛父非命隱居敎授三徵七辟皆不就|{
	徵詔召也辟公府及州郡辟也裒萍侯翻}
未嘗西向而坐|{
	裒居城陽晉朝在洛陽故未嘗西向}
廬於墓側旦夕攀柏悲號涕淚着樹|{
	號 刀翻著直略翻}
樹為之枯讀詩至哀哀父母生我劬勞|{
	詩蓼莪之辭}
未嘗不三復流涕門人為之廢蓼莪|{
	以裒悲慘故廢蓼莪之篇不敢講習三息暫翻復扶又翻蓼力竹翻}
家貧計口而田度身而蠶|{
	度徒洛翻}
人或饋之不受助之不聽諸生密為刈麥裒輒棄之遂不仕而終

臣光曰昔舜誅鯀而禹事舜不敢廢至公也嵇康王儀死皆不以其罪二子不仕晉室可也嵇紹苟無蕩陰之忠|{
	蕩陰事見後八十五卷惠帝永興元年余謂蕩陰之難君子以嵇紹為忠於所事可也然未足以塞天性之傷也蕩音湯}
殆不免於君子之譏乎

吴大司馬陸抗疾病上疏曰西陵建平國之蕃表|{
	蕃籬也表外也謂二郡為蕃籬於外也}
既處上流受敵二境|{
	謂二郡之境西距巴夔北接魏興上庸二面皆受敵也處昌呂翻}
若敵汎舟順流星犇電邁非可恃援他部以救倒縣也|{
	縣讀曰懸}
此乃社稷安危之機非徒封疆侵陵小害也臣父遜昔在西埀上言西陵國之西門雖云易守亦復易失|{
	易弋䜴翻}
若有不守非但失一郡荆州非吳有也如其有虞當傾國爭之臣前乞屯精兵三萬而主者循常未肯差赴|{
	主者謂居本兵之職者也差初皆翻}
自步闡以後|{
	步闡反見上卷八年}
益更損耗今臣所統千里外禦彊對|{
	彊對猶言彊敵也}
内懷百蠻而上下見兵財有數萬|{
	見賢遍翻財與纔同}
羸敝日久難以待變|{
	羸倫為翻}
臣愚以為諸王幼冲無用兵馬以防要務|{
	謂十一王各給三千兵也}
又黄門宦官開立占募兵民避役逋逃入占|{
	占章艶翻}
乞特詔簡閲一切料出|{
	料音聊}
以補疆場受敵常處使臣所部足滿八萬省息衆務并力備禦庶幾無虞|{
	幾居希翻}
若其不然深可憂也臣死之後乞以西方為屬|{
	陸抗固知吳之將亡特就職分上言之耳屬之欲翻下屬文同}
及卒吳主使其子晏景玄機雲分將其兵|{
	將即亮翻}
機雲皆善屬文名重於世初周魴之子處膂力絶人不修細行鄉里患之|{
	魴符方翻行下孟翻}
處嘗問父老曰今時和歲豐而人不樂何邪父老歎曰三害不除何樂之有|{
	樂音洛}
處曰何謂也父老曰南山白額虎長橋蛟|{
	南山今湖秀以南諸山也長橋在今常州宜興縣}
并子為三矣|{
	子謂周處}
處曰若所患止此吾能除之乃入山求虎射殺之因投水殺蛟遂從機雲受學篤志讀書砥節礪行比及朞年州府交辟|{
	射而亦翻行下孟翻比必寐翻}
八月戊申葬元皇后于峻陽陵帝及羣臣除喪即吉博士陳逵議以為今時所行漢帝權制太子無有國事自宜終服尚書杜預以為古者天子諸侯三年之喪始同齊斬|{
	謂齊衰斬衰之服其始自天子逹於庶人無以異也齊津夷翻}
既葬除服諒闇以居心喪終制故周公不言高宗服喪三年而云諒闇此服心喪之文也|{
	周公作無逸曰其在高宗作其即位乃或亮陰三年杜預遂引此言以為不服喪之證闇與陰同孔安國曰諒信也隂默也}
叔向不譏景王除喪而譏其宴樂已早明既葬應除而違諒闇之節也|{
	左傳晉荀躒如周葬穆后既葬除喪以文伯宴叔向曰王其不終乎吾聞之所樂必卒焉今王樂憂若卒以憂不可謂終王一歲而有三年之喪二焉於是乎以喪賓宴樂憂甚矣三年之喪雖貴遂服禮也王雖弗遂宴樂以早亦非禮也樂音洛}
子之於禮存諸内而已禮非玉帛之謂|{
	論語孔子曰禮云禮云玉帛云乎哉}
喪豈衰麻之謂乎|{
	衰七囘翻下同}
太子出則撫軍守則監國|{
	左傳晉大夫里克之言監古銜翻}
不為無事宜卒哭除衰麻|{
	卒子恤翻}
而以諒闇終三年帝從之

臣光曰規矩主於方圓然庸工無規矩則方圓不可得而制也衰麻主於哀戚然庸人無衰麻則哀戚不可得而勉也素冠之詩正為是矣|{
	衰倉囘翻詩素冠刺不能三年也為于偽翻}
杜預巧飾經傳以附人情辯則辯矣|{
	傳直戀翻}
臣謂不若陳逵之言質略而敦實也

九月癸亥以大將軍陳騫為太尉 杜預以孟津渡險請建河橋於富平津|{
	水經注孟津又曰富平津杜佑曰富平津在河陽縣南}
議者以為殷周所都歷聖賢而不作者必不可立故也|{
	殷都河内周都洛二代夾河建都不立河橋故以為言}
預固請為之及橋成帝從百寮臨會舉觴屬預曰|{
	屬之欲翻}
非君此橋不立對曰非陛下之明臣亦無所施其巧 是歲邵陵厲公曹芳卒初芳之廢遷金墉也|{
	芳之廢也築宮于河内重門今言遷金墉蓋始廢之時自禁中遷于金墉後乃居于河内也}
太宰中郎陳留范粲素服拜送|{
	晉既受禪避景帝諱採周官名置太宰以代太師魏因漢制上公惟有太傅據粲傳自太宰從事中郎遷太宰中郎時未置太宰宰當作傅}
哀動左右遂稱疾不出陽狂不言|{
	陽發見於外隂蔽伏於中凡人之作事外為是形而内無其實者皆陽為之外若無所營而内濳經畫皆陰為之}
寢所乘車足不履地子孫有婚宦大事輒密諮焉合者則色無變不合則眠寢不安妻子以此知其旨子喬等三人並棄學業絶人事|{
	按晉書喬年二歲祖馨臨終撫其首曰恨不見汝成人因以所用硯與之至五歲祖母以告喬喬便執硯涕泣九歲請學在同輩之中言無媟辭李銓常論揚雄才學優於劉向喬以為向定一代之書正羣籍之篇使雄當之故非所長遂著劉揚優劣論前後辟舉皆不就邑人臘日盜斫其樹人有告者喬陽不聞邑人愧而歸之喬曰卿節日取柴欲與父母相歡娛耳何以愧為嗚呼觀喬之學行如此則棄學業絶人事殆庶幾乎夷齊餓于首陽之下之意}
侍疾家庭足不出邑里及帝即位詔以二千石禄養病加賜帛百匹喬以父疾篤辭不敢受粲不言凡三十六年年八十四終於所寢之車|{
	自卲陵厲公之廢至是方二十一年史因公卒而究言之}
吳比三年大疫|{
	比毗至翻}


咸寧元年春正月戊午朔大赦改元 吳掘地得銀尺上有刻文|{
	吳志曰銀長一尺廣三分刻上有年月字}
吳主大赦改元天冊吳中書令賀邵中風不能言|{
	中竹仲翻}
去職數月吳主疑其詐收付酒藏掠考千數|{
	藏徂浪翻掠音亮}
卒無一言乃燒鋸斷其頭|{
	卒子恤翻斷丅管翻}
徙其家屬於臨海又誅樓玄子孫|{
	殺樓玄見上卷泰始八年}
夏六月鮮卑拓拔力微復遣其子沙漠汗入貢|{
	沙漠汗初入貢見七十八卷魏元帝景元二年汗音寒}
將還幽州刺史衛瓘表請留之又密以金賂其諸部大人離間之|{
	為力微信譖殺沙漠汗張本間古莧翻}
秋七月甲申晦日有食之 冬十二月丁亥追尊宣帝廟曰高祖景帝曰世宗文帝曰太祖 大疫洛陽死者以萬數

二年春令狐豐卒弟宏繼立楊欣討斬之|{
	豐自為敦煌太守見上卷泰始八年}
帝得疾甚劇及愈羣臣上夀詔曰每念疫氣死亡者為之愴然豈以一身之休息忘百姓之艱難邪諸上禮者皆絶之|{
	為于偽翻上時掌翻}
初齊王攸有寵於文帝每見攸輒撫牀呼其小字曰此桃符座也幾為太子者數矣|{
	事見七十八卷魏元帝咸熙元年幾居依翻數所角翻}
臨終為帝叙漢淮南王魏陳思王事而泣|{
	漢文帝誅淮南厲王長魏文帝不能容陳思王植引此二事以戒切帝也}
執攸手以授帝太后臨終亦流涕謂帝曰桃符性急而汝為兄不慈我若不起必恐汝不能相容以是屬汝勿忘我言及帝疾甚朝野皆屬意於攸|{
	屬之欲翻朝直遥翻}
攸妃賈充之長女也|{
	充先娶李氏豐女也生二女長曰荃為齊王攸妃長知兩翻}
河南尹夏侯和謂充曰卿二婿親踈等耳|{
	二婿謂攸及太子也}
立人當立德充不答攸素惡荀勗及左衛將軍馮紞傾陷朂乃使紞說帝曰|{
	惡烏路翻紞都感翻說輸芮翻}
陛下前日疾若不愈齊王為公卿百姓所歸太子雖欲高讓其得免乎宜遣還藩以安社稷帝隂納之乃徙和為光禄勲奪充兵權|{
	充自文帝時領兵}
而位遇無替 吳施但之亂|{
	事見上卷泰始二年}
或譖京下督孫楷於吳主曰楷不時赴討懷兩端吳主數詰讓之徵為宫下鎮驃騎將軍|{
	京下督鎮京口宫下鎮在建業楷孫韶之子數所角翻驃匹妙翻騎奇寄翻}
楷自疑懼夏六月將妻子來犇拜車騎將軍封丹陽侯秋七月吳人或言於吳主曰臨平湖自漢末薉塞|{
	臨平湖今在臨安府仁和縣界有臨平鎮在臨安府城西北四十八里薉荒蕪也音烏廢翻塞悉則翻下同}
長老言此湖塞天下亂此湖開天下平近無故忽更開通此天下當太平青蓋入洛之祥也|{
	青蓋之占見上卷泰始八年}
吳主以問奉禁都尉歷陽陳訓|{
	吳置奉禁都尉蓋以侍奉宮禁為稱}
對曰臣止能望氣不能逹湖之開塞退而告其友曰青蓋入洛者將有銜璧之事非吉祥也或獻小石刻皇帝字云得於湖邉吳主大赦改元天璽|{
	璽斯氏翻}
湘東太守張詠不出筭緡|{
	吳主亮太平二年分長沙東部都尉立湘東郡}
吳主就在所斬之狥首諸郡會稽太守車浚公清有政績|{
	會工外翻車姓出於田千秋車昌遮翻}
值郡旱饑表求振貸吳主以為收私恩遣使梟首|{
	梟堅堯翻}
尚書熊睦微有所諫|{
	黄帝有熊氏姓譜楚鬻熊之後此以名為氏者也}
吳主以刀環撞殺之身無完肌|{
	史詳言吳主之昏虐撞直江翻}
八月己亥以何曾為太傅陳騫為大司馬賈充為太尉齊王攸為司空 吳歷陽山有七穿駢羅穿中黄赤俗謂之石印云石印封發天下當太平歷陽長上言石印發|{
	據吳志鄱陽上言歷陽山石文理成字又江表傳曰歷陽縣有石山臨水高百丈其三十丈所有七穿駢羅今考晉志鄱陽郡無歷陽縣有歷陵縣陽當作陵今饒州圖經亦載鄱陽歷陵縣有石印山長知兩翻}
吳主遣使者以太牢祠之|{
	使疏吏翻}
使者作高梯登其上以朱書石曰楚九州渚吳九州都揚州士作天子四世治太平始還以聞吳主大喜封其山神為王大赦改明年元曰天紀 冬十月以汝陰王駿為征西大將軍羊祜為征南大將軍皆開府辟召儀同三司|{
	此位從公也}
祜上疏請伐吳|{
	陸抗没羊祜始抗疏請伐吳上時掌翻}
曰先帝西平巴蜀|{
	見七十八卷魏元帝景元四年}
南和吳會|{
	見七十八卷魏元帝咸熙元年}
庶幾海内得以休息而吳復背信|{
	事見上卷泰始元年幾居希翻背蒲妹翻}
使邉事更興夫期運雖天所授而功業必因人而成不一大舉掃滅則兵役無時得息也蜀平之時天下皆謂吳當并亡自是以來十有三年矣|{
	景元四年蜀亡至是十三年}
夫謀之雖多決之欲獨凡以險阻得全者謂其埶均力敵耳若輕重不齊強弱異埶雖有險阻不可保也蜀之為國非不險也皆云一夫荷戟千人莫當|{
	荷下可翻}
及進兵之日曾無藩籬之限乘勝席卷徑至成都漢中諸城皆鳥栖而不敢出|{
	謂漢樂諸城也}
非無戰心誠力不足以相抗也及劉禪請降諸營堡索然俱散|{
	索昔各翻}
今江淮之險不如劒閣孫皓之暴過於劉禪吳人之困甚於巴蜀而大晉兵力盛於往時不於此際平壹四海而更阻兵相守使天下困於征戍經歷盛衰不可長久也|{
	謂兵將以盛壯之年出戍經歷營陳至於衰老也}
今若引梁益之兵水陸俱下|{
	王濬唐彬統梁益兵}
荆楚之衆進臨江陵|{
	荆楚祜所統也}
平南豫州直指夏口|{
	胡奮為平南將軍王戎為豫州刺史夏戶雅翻}
徐揚青兖並會秣陵|{
	徐揚王渾所統青兖琅邪王伷所統}
以一隅之吳當天下之衆埶分形散所備皆急巴漢奇兵出其空虚一處傾壞則上下震蕩雖有智者不能為吳謀矣|{
	其後平吳皆如祜所規}
吳緣江為國東西數千里所敵者大無有寧息孫皓恣情任意與下多忌將疑於朝|{
	將即亮翻朝直遥翻}
士困於野無有保世之計一定之心平常之日猶懷去就兵臨之際必有應者終不能齊力致死已可知也其俗急速不能持久弓弩戟楯不如中國唯有水戰是其所便一入其境則長江非復所保還趣城池|{
	趣七喻翻}
去長入短非吾敵也官軍縣進|{
	縣讀曰懸}
人有致死之志吳人内顧各有離散之心如此軍不踰時克可必矣帝深納之而朝議方以秦凉為憂|{
	謂樹機能未平也朝直遥翻}
祜復表曰|{
	復扶又翻}
吳平則胡自定但當速濟大功耳議者多有不同賈充荀朂馮紞尤以伐吳為不可祜歎曰天下不如意事十常居七八天與不取豈非更事者恨於後時哉|{
	言吳可取而不取機會一失經見其事者豈不有後時之恨更工衡翻}
唯度支尚書杜預|{
	魏制度支尚書度徒洛翻}
中書令張華與帝意合贊成其計 丁卯立皇后楊氏大赦后元皇后之從妹也|{
	從才用翻}
美而有婦德帝初聘后后叔父珧|{
	珧余招翻}
上表曰自古一門二后未有能全其宗者乞藏此表於宗廟異日如臣之言得以免禍帝許之|{
	珧雖有此表終不能以免禍}
十二月以后父鎮軍將軍駿為車騎將軍封臨晉侯|{
	國號晉而封后父為臨晉侯不祥之徵也}
尚書禇䂮郭奕皆表駿小器不可任社稷之重|{
	䂮離灼翻任音壬}
帝不從駿驕傲自得胡奮謂駿曰卿恃女更益豪邪歷觀前世與天家婚未有不滅門者但早晩事耳駿曰卿女不在天家乎|{
	天子尊無二上故曰天家言其尊如天也}
奮曰我女與卿女作婢耳何能為損益乎

三年春正月丙子朔日有食之 立皇子裕為始平王庚寅裕卒 三月平虜護軍文鴦督凉秦雍州諸軍討樹機能破之諸胡二十萬口來降|{
	雍於用翻降戶江翻}
夏五月吳將邵顗|{
	顗魚豈翻 考異日武紀作邵凱今從羊祜傳}
夏祥帥衆七千餘人來降|{
	夏戶雅翻帥讀曰率降戶江翻}
秋七月中山王睦坐招誘逋亡貶為丹水縣侯|{
	誘音酉}
有星孛于紫宫|{
	孛蒲内翻}
衛將軍楊珧等建議以為古者封建諸侯所以藩衛王室今諸王公皆在京師非干城之義又異姓諸將居邉宜參以親戚 |{
	考異曰職官志以為珧與荀朂以齊王攸有時望懼太子冇後難故建此議使諸王之國帝初未之察於是下詔議其制按朂傳有異議又時齊王不之國疑此說非實今不取}
帝乃詔諸王各以戶邑多少為三等大國置三軍五千人次國二軍三千人小國一軍一千一百人|{
	時以平原汝南琅邪扶風齊為大國梁趙樂安燕安平義陽為次國餘國為小國}
諸王為都督者各徙其國使相近八月癸亥徙扶風王亮為汝南王出為鎮南大將軍都督豫州諸軍事琅邪王倫為趙王督鄴城守事勃海王輔為太原王監并州諸軍事以東莞王伷在徐州徙封琅邪王|{
	莞音官伷音胄}
汝陰王駿在關中徙封扶風王又徙太原王顒為河間王汝南王柬為南陽王輔孚之子顒孚之孫也|{
	顒魚容翻}
其無官者皆遣就國諸王公戀京師皆涕泣而去又封皇子瑋為始平王允為濮陽王該為新都王遐為清河王其異姓之臣有大功者皆封郡公郡侯封賈充為魯郡公追封王沈為博陵郡公|{
	沈持林翻}
徙封鉅平侯羊祜為南城郡侯|{
	時詔以泰山之南武陽牟南城梁父平陽五縣為南城郡羊祜本泰山南城人也帝制公侯邑萬戶以上為大國五千戶以上為次國不滿五千戶為小國}
祜固辭不受祜每拜官爵常多避讓至心素著故特見申於分列之外|{
	見申謂許之辭爵其志獲申也分列謂分封列爵也}
祜歷事二世|{
	謂事文帝及帝也}
職典樞要凡謀議損益皆焚其草世莫得聞所進逹之人皆不知所由|{
	謂人由祜薦引而進逹不知其所由來也}
常曰拜官公朝謝恩私門吾所不敢也|{
	朝直遥翻}
兖豫徐青荆益梁七州大水冬十二月吳夏口督孫慎入江夏汝南|{
	江夏郡屬荆州汝南郡屬}


|{
	豫州相去甚遠沈約宋志江夏太守治汝南縣本沙羨地晉末汝南郡民流寓夏口因立為汝南則此時江夏郡未有汝南縣也無亦史追書乎夏戶雅翻}
略千餘家而去詔遣侍臣詰羊祜不追討之意|{
	詰去吉翻}
并欲移荆州祜曰江夏去襄陽八百里比知賊問|{
	比必寐翻}
賊已去經日步軍安能追之勞師以免責非臣志也昔魏武帝置都督類皆與州相近|{
	如揚州刺史治夀春都督揚州諸軍事亦治夀春之類近其靳翻}
以兵埶好合惡離故也|{
	好呼到翻惡烏路翻}
疆場之間一彼一此慎守而已|{
	左傳魯桓公曰疆場之間慎守其一而備其不虞}
若輒徙州賊出無常亦未知州之所宜據也 是歲大司馬陳騫自揚州入朝|{
	朝直遙翻}
以高平公罷吳主以會稽張俶多所譖白|{
	會工外翻俶昌六翻}
甚見寵任累

遷司直中郎將封侯其父為山隂縣卒|{
	山隂縣屬會稽郡}
知俶不良上表曰若用俶為司直有罪乞不從坐吳主許之俶表置彈曲二十人專糾司不灋|{
	彈徒干翻}
於是吏民各以愛憎互相告訐獄犴盈溢|{
	訐居謁翻犴音岸犴野犬也野犬所以守故為獄又胡地謂犬為犴}
上下囂然俶大為姧利驕奢暴横|{
	横戶孟翻}
事發父子皆車裂 衛瓘遣拓拔沙漠汗歸國|{
	前年瓘表留沙漠汗讒間既行乃遣歸}
自沙漠汗入質|{
	入質見七十七卷魏元帝景元二年質音致}
力微可汗諸子在側者多有寵及沙漠汗歸諸部大人共譖而殺之既而力微疾篤烏桓主庫賢親近用事受衛瓘賂欲擾動諸部乃礪斧於庭謂諸大人曰可汗恨汝曹讒殺太子|{
	此時鮮卑君長已有可汗之稱可今讀從刋入聲汗音寒}
欲盡收汝曹長子殺之|{
	長知兩翻}
諸大人懼皆散走力微以憂卒時年一百四子悉禄立|{
	悉禄魏收魏書作悉鹿}
其國遂衰初幽并二州皆與鮮卑接東有務桓西有力微多為邉患衛瓘密以計間之|{
	間古莧翻}
務桓降而力微死 |{
	考異曰魏收後魏書鐵弗劉虎匈奴去卑之孫昭成四年死子務桓立按昭成四年晉成帝咸康七年也務桓不應與瓘同時蓋二人皆名務桓耳}
朝廷嘉瓘功封其弟為亭侯

四年春正月庚午朔日有食之 司馬督東平馬隆|{
	晉制二衛前驅由基彊弩為三部司馬各置督沈約曰殿中司馬督晉武帝時殿中宿衛號曰三部司馬與殿中將軍分隸左右二衛}
上言凉州刺史楊欣失羌戎之和必敗|{
	隆言欣必敗猶漢皇甫規之言馬賢蓋懷才欲用故以此自顯耳}
夏六月欣與樹機能之黨若羅拔能等戰于武威敗死 弘訓皇后羊氏殂|{
	景皇后居弘訓宫}
羊祜以病求入朝|{
	朝直遙翻}
既至帝命乘輦入殿不拜而坐祜面陳伐吳之計帝善之以祜病不宜數入|{
	數所角翻}
更遣張華就問籌策祜曰孫皓暴虐已甚於今可不戰而克若皓不幸而没吳人更立令主雖有百萬之衆長江未可窺也將為後患矣華深然之祜曰成吾志者子也帝欲使祜卧護諸將祜曰取吳不必臣行但既平之後當勞聖慮耳功名之際臣不敢居若事了當有所付授願審擇其人也|{
	以東南壤界闊遠當得人以鎮撫之}
秋七月己丑葬景獻皇后于峻平陵|{
	即弘訓后也}
司冀兖豫荆揚州大水|{
	司州即漢司隸校尉所部也漢司隸部察郡縣與州刺史同晉遂定名司州統河南滎陽弘農上洛平陽河東汲郡河内廣平陽平魏郡頓丘冀州者亂則冀安弱則冀強荒則冀豐統趙國鉅鹿安平平原樂陵勃海河間高陽博陵清河中山常山等郡國}
螟傷稼|{
	螟食苖心之蟲}
詔問主者何以佐百姓|{
	主者謂左民及度支二曹也}
度支尚書杜預上疏|{
	度徒洛翻上時掌翻}
以為今者水災東南尤劇宜敕兖豫等諸州留漢氏舊陂繕以蓄水餘皆决瀝令饑者盡得魚菜螺蜯之饒|{
	螺盧戈翻蜯步項翻}
此目下日給之益也水去之後滇淤之田|{
	淤依據翻}
畝收數鍾此又明年之益也典牧種牛有四萬五千餘頭|{
	晉志典牧令屬太僕種章勇翻}
不供耕駕至有老不穿鼻者可分以給民使及春耕種穀登之後責其租税此又數年以後之益也帝從之民賴其利 |{
	考異曰食貨志云咸寧三年杜預傳云四年按五行志三年大水無蟲災四年螟今從預傳}
預在尚書七年|{
	泰始六年預自秦州刺史得罪歸拜度支尚書至是七年矣}
損益庶政不可勝數|{
	勝音升}
時人謂之杜武庫言其無所不有也 九月以何曾為太宰辛巳以侍中尚書令李胤為司徒 吳主忌勝已者侍中中書令張尚紘之孫也|{
	張紘事孫策孫權見漢獻帝紀}
為人辯捷談論每出其表吳主積以致恨後問孤飲酒可以方誰|{
	方比也}
尚曰陛下有百觚之量吳主曰尚知孔丘不王而以孤方之|{
	孔叢子曰趙平原君與孔子高飲強子高酒曰諺云堯飲千鍾孔子百觚子路嗑嗑尚飲十榼古之聖賢無不能飲子何辭焉觚飲器也受二升王于况翻}
因發怒收尚公卿已下百餘人詣宫叩頭請尚罪得减死送建安作船尋就殺之 |{
	考異曰三十國春秋云岑昏等泥頭請代尚死尚得免死徙廣州今從尚傳參取環氏吳紀余觀尚之為人蓋以辯給得親近於孫皓而亦以辯給取怒請其死者必岑昏之徒三十國春秋所書蓋得其實}
冬十月徵征北大將軍衛瓘為尚書令是時朝野咸知太子昏愚不堪為嗣瓘每欲陳啓而未敢發會侍宴陵雲臺|{
	陵雲臺魏文帝所築}
瓘陽醉跪帝牀前曰臣欲有所啓帝曰公所言何邪瓘欲言而止者三因以手撫牀曰此座可惜帝意悟固謬曰公真大醉邪瓘於此不復有言帝悉召東宮官屬為設宴會|{
	復扶又翻為于偽翻下便為同}
而密封尚書疑事令太子决之賈妃大懼倩外人代對|{
	倩七正翻假手於人也}
多引古義給使張泓曰太子不學陛下所知而答詔多引古義必責作草主|{
	言將責問作對草之主名也}
更益譴負不如直以意對妃大喜謂泓曰便為我好答富貴與汝共之|{
	給使給東宮使令張泓蓋庸中之佼佼者後為趙王倫拒齊王冏於陽翟者必是人也}
泓即具草令太子自寫帝省之甚悦|{
	省悉景翻}
先以示瓘瓘大踧踖|{
	踧子六翻踖子昔翻踧踖不自安貌}
衆人乃知瓘嘗有言也賈充密遣人語妃云衛瓘老奴幾破汝家|{
	為賈妃怨衛瓘張本語牛倨翻 考異曰三十國春秋在泰始八年按瓘傳泰始初為青州刺史徙幽州八年不得在京師瓘傳在遷司空後按帝紀太康三年賈充卒十二月瓘為司空故移在入為尚書令下}
吳人大田皖城|{
	佃亭年翻治田也皖戶板翻}
欲謀入宼都督揚州諸軍事王渾遣揚州刺史應綽攻破之斬首五千級焚其積穀百八十餘萬斛踐稻田四千餘頃毁船六百餘艘|{
	艘蘇刀翻}
十一月辛巳太醫司馬程據獻雉頭裘|{
	晉志太醫屬宗正雉頭毛采炫燿集以為裘}
帝焚之於殿前甲申敕内外敢有獻奇技異服者罪之|{
	記王制作淫聲異服奇技奇器以疑衆殺技渠綺翻}
羊祜疾篤舉杜預自代辛卯以預為鎮南大將軍都督荆州諸軍事祜卒|{
	卒子恤翻下同}
帝哭之甚哀是日大寒涕涙霑須鬢皆為氷祜遺令不得以南城侯印入柩|{
	柩音舊}
帝曰祜固讓歷年身没讓存|{
	謂身没而遺令讓侯印也}
今聽復本封以彰高美|{
	祜本封鉅平侯}
南州民聞祜卒為之罷市巷哭聲相接|{
	南州謂荆州也為于偽翻下同}
吳守邉將士亦為之泣祜好遊峴山|{
	好呼到翻峴戶典翻}
襄陽人建碑立廟於其地歲時祭祀望其碑者無不流涕因謂之墮淚碑杜預至鎮簡精鋭襲吳西陵督張政大破之政吳之名將也耻以無備取敗不以實告吳主預欲間之|{
	間古莧翻}
乃表還其所獲吳主果召政還遣武昌監留憲代之|{
	吳之邉鎮有督有監督者督諸軍事之職監者監諸軍事之職}
十二月丁未朗陵公何曾卒曾厚自奉養過於人主司隸校尉東萊劉毅數劾奏曾侈汰無度|{
	數所角翻}
帝以其重臣不問及卒博士新興秦秀議曰|{
	秀新興雲中人朗之子也}
曾驕奢過度名被九域|{
	九域九州之域被皮義翻}
宰相大臣人之表儀若生極其情死又無貶王公貴人復何畏哉謹按諡灋|{
	諡法始於周公以行為諡復扶又翻}
名與實爽曰繆怙亂肆行曰醜宜諡醜繆公帝策諡曰孝|{
	策諡者不用博士議以詔策賜諡}
前司隸校尉傳玄卒 |{
	考異曰玄傳曰五年遷太僕轉司隸景獻皇后崩坐爭位罵尚書免尋卒按景獻后崩在四年玄傳誤也}
玄性峻急每有奏劾或值日暮捧白簡整簪帶|{
	文選任昉彈曹景宗曰謹奉白簡以聞呂向注云簡略狀也晉志曰古者執笏有事則書之故常簪筆今之白筆是其遺意三臺五省二品文官簪之帶革帶也古之鞶帶劾戶槩翻又戶得翻}
竦踊不寐坐而待旦由是貴游震懾|{
	周官師氏凡國之貴游子弟學焉注云貴游子弟王公之子弟游無官司者懾之涉翻}
臺閣生風玄與尚書左丞博陵崔洪善|{
	漢安帝分安平置博陵國}
洪亦清厲骨鯁好面折人過|{
	好呼到翻折之舌翻}
而退無後言人以是重之 鮮卑樹機能久為邉患|{
	泰始六年樹機能為宼至是九年矣}
僕射李憙請發兵討之朝議皆以為出兵重事虜不足憂|{
	朝直遥翻下同}


五年春正月樹機能攻陷凉州|{
	凉州治武威}
帝甚悔之臨朝而歎曰誰能為我討此虜者|{
	為于偽翻}
司馬督馬隆進曰陛下能任臣臣能平之帝曰必能平賊何為不任顧方略何如耳隆曰臣願募勇士三千人無問所從來|{
	應募者或出於農畝或出於營伍或出於逋逃或出於奴隸皆不問其所從來也}
帥之以西虜不足平也|{
	帥讀曰率}
帝許之乙丑以隆為討虜護軍武威太守公卿皆曰見兵已多不宜横設賞募|{
	見賢遍翻横戶孟翻}
隆小將妄言|{
	將即亮翻}
不足信也帝不聽隆募能引弓四鈞挽弩九石者取之|{
	三十斤為鈞四鈞為石石百二十斤}
立標簡試|{
	標表也}
自旦至日中得三千五百人隆曰足矣又請自至武庫選仗武庫令與隆忿爭|{
	晉志武庫令屬衛尉}
御史中丞劾奏隆|{
	自東漢至魏晉以中丞為御史臺主劾戶槩翻又戶得翻}
隆曰臣當畢命戰場武庫令乃給以魏時朽仗非陛下所以使臣之意也帝命惟隆所取仍給三年軍資而遣之 初南單于呼厨泉以兄於扶羅子豹為左賢王及魏武帝分匈奴為五部|{
	五部見上卷泰始六年}
以豹為左部帥|{
	帥所類翻}
豹子淵幼而雋異師事上黨崔游博習經史嘗謂同門生上黨朱紀雁門范隆曰吾常耻隨陸無武絳灌無文隨陸遇高帝而不能建封侯之業絳灌遇文帝而不能興庠序之敎豈不惜哉|{
	隨陸隨何陸賈絳灌絳侯周勃灌將軍嬰}
於是兼學武事及長|{
	長知兩翻}
猨臂善射膂力過人姿貌魁偉為任子在洛陽王渾及子濟皆重之屢薦於帝帝召與語悦之濟曰淵有文武長才陛下任以東南之事吳不足平也孔恂楊珧曰非我族類其心必異|{
	左傳魯季文子曰史佚之志冇之非我族類其心必異珧余招翻}
淵才器誠少比然不可重任也|{
	少詩昭翻}
及凉州覆没帝問將於李憙對曰陛下誠能發匈奴五部之衆假劉淵一將軍之號使將之而西樹機能之首可指日而梟也|{
	使將即亮翻梟堅堯翻}
孔恂曰淵果梟樹機能則凉州之患方更深耳帝乃止東萊王彌家世二千石|{
	世語曰彌魏玄菟太守王頎之孫}
彌有學術勇略善騎射青州人謂之飛豹處士陳留董養見而謂之曰君好亂樂禍若天下有事不作士大夫矣|{
	言將為賊也處昌呂翻好呼到翻樂音洛}
淵與彌友善謂彌曰王李以鄉曲見知|{
	王渾太原人李憙上黨人與澗同州里}
每相稱薦適足為吾患耳因歔欷流涕|{
	歔音虚欷音希又吁既翻}
齊王攸聞之言於帝曰陛下不除劉淵臣恐并州不得久安王渾曰大晉方以信懷殊俗奈何以無形之疑殺人侍子乎何德度之不弘也帝曰渾言是也會豹卒以淵代為左部帥|{
	劉淵事始此史言晉將有亂帥所類翻}
夏四月大赦 除部曲督以下質任|{
	帝受禪之初除部曲將質任今又除部曲督質任質音致}
吳桂林太守修允卒|{
	桂林漢縣屬欝林郡吳主皓鳳凰三年分立桂林郡}
其部曲應分給諸將督將郭馬何典王族等累世舊軍不樂離别會吳主料實廣州戶口|{
	將即亮翻樂音洛料音聊}
馬等因民心不安聚衆攻殺廣州督虞授馬自號都督交廣二州諸軍事使典攻蒼梧族攻始興|{
	吳主皓甘露元年分桂陽南部都尉立始興郡}
秋八月吳以軍師張悌為丞相牛渚都督何植為司徒執金吾滕修為司空未拜更以修為廣州牧帥萬人從東道討郭馬|{
	帥讀曰率}
馬殺南海太守劉略逐廣州刺史徐旗吳主又遣徐陵督陶濬將七千人|{
	徐陵與洞浦對岸吳主權時呂範洞浦之敗魏臧霸度江攻徐陵全琮徐盛擊却之又華覈封徐陵亭侯則徐陵蓋亭名吳以其臨江津置督守之南徐州記曰京口先為徐陵其地蓋丹徒縣之西鄉京口里也}
從西道與交州牧陶璜共擊馬 吳有鬼目菜生工人黄耉家有買菜生工人吳平家|{
	吳志曰鬼日菜依緣棗樹長丈餘莖廣四寸厚三分買菜高四尺厚二分如枇杷形莖廣尺八寸下莖廣五寸兩邉生葉緑色}
東觀案圖書|{
	吳有東觀令觀古玩翻}
名鬼目曰芝草買菜曰平慮草吳主以耉為侍芝郎平為平慮郎皆銀印青綬|{
	以漢制言之銀印青綬中二千石服之}
吳主每宴羣臣咸令沈醉|{
	沈持林翻}
又置黄門郎十人為司過宴罷之後各奏其闕失迕視謬言罔有不舉|{
	沈持林翻迕五故翻逆也}
大者即加刑戮小者記録為罪或剥人面或鑿人眼由是上下離心莫為盡力|{
	為于偽翻}
益州刺史王濬上疏曰孫皓荒淫凶逆宜速征伐若一旦皓死更立賢主則彊敵也|{
	更工衡翻}
臣作船七年|{
	泰始八年濬始作船至是蓋七期年矣}
日有朽敗臣年七十死亡無日三者一乖則難圖也誠願陛下無失事機帝於是决意伐吳會安東將軍王渾表孫皓欲北上|{
	上時掌翻}
邉戍皆戒嚴朝廷乃更議明年出師王濬參軍何攀奉使在洛上疏稱皓必不敢出宜因戒嚴掩取甚易|{
	易以豉翻}
杜預上表曰自閏月以來|{
	是年閏七月}
賊但敕嚴下無兵上|{
	吳自建業宼淮襄皆自下沂江而上上時掌翻}
以理埶推之賊之窮計力不兩完必保夏口以東以延視息|{
	凡人目不能視氣不能息則赫然死人矣}
無緣多兵西上空其國都而陛下過聽便用委棄大計縱敵患生誠可惜也嚮使舉而有敗勿舉可也今事為之制務從完牢若或有成則開太平之基不成不過費損日月之間何惜而不一試之若當須後年|{
	須待也}
天時人事不得如常臣恐其更難也今有萬安之舉無傾敗之慮臣心實了|{
	了决也}
不敢以曖昧之見自取後累|{
	曖時不明也累力瑞翻}
惟陛下察之旬月未報預復上表曰|{
	復扶又翻}
羊祜不先博謀於朝臣|{
	朝直遥翻}
而密與陛下共施此計故益令朝臣多異同之議凡事當以利害相校今此舉之利十有八九而其害一二止於無功耳必使朝臣言破敗之形亦不可得直是計不出已功不在身各耻其前言之失而固守之也|{
	此言指出賈充荀朂馮紞等肺腑}
自頃朝廷事無大小異意鋒起雖人心不同亦由恃恩不慮後患故輕相異同也自秋已來討賊之形頗露今若中止孫皓或怖而生計|{
	怖普布翻}
徙都武昌更完修江南諸城遠其居民城不可攻野無所掠則明年之計或無所及矣帝方與張華圍碁|{
	博物志曰堯造圍碁以敎子丹朱或曰舜以子商均愚故作圍碁以教之其法非智莫能也}
預表適至華推枰歛手曰|{
	推吐雷翻枰音平碁局也}
陛下聖武國富兵彊吳主淫虐誅殺賢能當今討之可不勞而定願勿以為疑帝乃許之以華為度支尚書量計運漕|{
	度徒洛翻量音良}
賈充荀朂馮紞固爭之|{
	紞吐感翻}
帝大怒充免冠謝罪僕射山濤退而告人曰自非聖人外寧必有内憂|{
	左傳晉大夫范文子之言}
今釋吳為外懼豈非筭乎|{
	山濤身為大臣不昌言於朝而退以告人蓋求合於賈充者也}
冬十一月大舉伐吳遣鎮軍將軍琅邪王伷出涂中|{
	伷音胄吳主權作堂邑涂塘即其地蓋從今滁州取真州路涂讀曰滁}
安東將軍王渾出江西|{
	今和州出横江渡路}
建威將軍王戎出武昌平南將軍胡奮出夏口鎮南大將軍杜預出江陵龍驤將軍王濬巴東監軍魯國唐彬下巴蜀|{
	監古銜翻}
東西凡二十餘萬命賈充為使持節假黄鉞大都督|{
	魏文帝以曹真都督中外諸軍事假黄鉞明帝大和四年司馬懿征蜀加號大都督此仍魏制也武王伐紂左杖黄鉞黄钺天子之器非人臣所得專用故曰假使疏吏翻}
以冠軍將軍楊濟副之|{
	冠古玩翻}
充固陳伐吳不利且自言衰老不堪元帥之任|{
	帥讀從所類翻}
詔曰君若不行吾便自出充不得已乃受節钺將中軍南屯襄陽為諸軍節度 馬隆西度温水|{
	武威之東有温圍水}
樹機能等以衆數萬據險拒之隆以山路陿隘乃作扁箱車|{
	陿與狹同車箱扁則可行狹路扁補典翻}
為木屋施於車上|{
	木屋所以蔽風雨捍矢石}
轉戰而前行千餘里殺傷甚衆 |{
	考異曰隆傳曰或夾道累磁石賊被鐵鎧行不得前隆卒悉被犀甲無所留礙賊以為神按此說太誕恐不可信 余謂磁石脅鐵鎧誠有此理}
自隆之西音問斷絶朝廷憂之或謂已没後隆使夜到|{
	使疏吏翻}
帝撫掌歡笑詰朝召羣臣謂曰若從諸卿言無凉州矣|{
	詰去吉翻朝如字}
乃詔假隆節拜宣威將軍|{
	沈約志魏置將軍四十號宣威第二}
隆至武威鮮卑大人猝跋韓且萬能帥萬餘落來降|{
	且子閭翻帥讀曰率降戶江翻}
十二月隆與樹機能大戰斬之凉州遂平 詔問朝臣以政之損益司徒左長史傅咸上書|{
	晉志司徒加置左右長史各一人朝直遥翻}
以為公私不足由設官太多舊都督有四今并監軍乃盈於十|{
	魏初置都督諸軍東南以備吳西以備蜀北以備胡隨其資望輕重而加以征鎮安平之號有四而已其後增置有都督鄴城守諸軍都督秦雍凉諸軍都督梁益諸軍都督荆州諸軍都督揚州諸軍都督徐州諸軍都督淮北諸軍都督豫州諸軍都督幽州諸軍都督并州諸軍凡十其資輕者為監軍}
禹分九州今之刺史幾向一倍|{
	時有司豫徐兖荆揚梁益寧交秦雍凉冀幽并青十八州刺史幾居希翻}
戶口比漢十分之一|{
	漢元始之初民戶千三百二十三萬三千六百一十二口五千九百一十九萬四千九百七十八漢之極盛也桓帝之初戶二千六百七萬九百六口五千六萬六千八百五十六魏既幷蜀景元四年與蜀通計民戶九十四萬三千四百二十三口五百三十七萬二千八百九十一蓋口猶及漢十分之一而戶則未幾及也}
而置郡縣更多虛立軍府動有百數而無益宿衛五等諸侯坐置官屬|{
	軍府謂驃騎車騎衛伏波撫軍都護鎮軍中軍典軍上軍撫國領軍護軍左右衛驍騎游擊左右前後軍及雜號將軍也五等諸侯官屬王置傳友文學郎中令中尉大農左右常侍侍郎典書典祠典衛學官等令典書丞治書中尉司馬世子庶子陵廟牧長謁者中大夫舍人典府公侯以下置官屬隨國小大無定制}
諸所廩給皆出百姓此其所以困乏者也當今之急在於并官息役上下務農而已咸玄之子也時又議省州郡縣半吏以赴農功中書監荀朂以為省吏不如省官省官不如省事省事不如清心昔蕭曹相漢載其清静民以寧一|{
	事見十三卷漢惠帝二年}
所謂清心也抑浮說簡文案略細苛宥小失有好變常以徼利者必行其誅所謂省事也|{
	好呼倒翻徼一遥翻}
以九寺併尚書蘭臺付三府所謂省官也|{
	九寺謂九卿寺也漢初九卿各有所掌東都以後尚書諸曹分掌衆軍九卿殆為具官故欲併之尚書蘭臺御史臺也三府三公府也漢丞相有長史司直御史大夫有中丞侍御史掌察舉非法故朂欲以蘭臺付之三府}
若直作大例凡天下之吏皆減其半恐文武衆官郡國職業劇易不同|{
	易以豉翻}
不可以一槩施之若有曠闕皆須更復或激而滋繁亦不可不重也

資治通鑑卷八十
