<!DOCTYPE html PUBLIC "-//W3C//DTD XHTML 1.0 Transitional//EN" "http://www.w3.org/TR/xhtml1/DTD/xhtml1-transitional.dtd">
<html xmlns="http://www.w3.org/1999/xhtml">
<head>
<meta http-equiv="Content-Type" content="text/html; charset=utf-8" />
<meta http-equiv="X-UA-Compatible" content="IE=Edge,chrome=1">
<title>資治通鑒_111-資治通鑑卷一百一十_111-資治通鑑卷一百一十</title>
<meta name="Keywords" content="資治通鑒_111-資治通鑑卷一百一十_111-資治通鑑卷一百一十">
<meta name="Description" content="資治通鑒_111-資治通鑑卷一百一十_111-資治通鑑卷一百一十">
<meta http-equiv="Cache-Control" content="no-transform" />
<meta http-equiv="Cache-Control" content="no-siteapp" />
<link href="/img/style.css" rel="stylesheet" type="text/css" />
<script src="/img/m.js?2020"></script> 
</head>
<body>
 <div class="ClassNavi">
<a  href="/24shi/">二十四史</a> | <a href="/SiKuQuanShu/">四库全书</a> | <a href="http://www.guoxuedashi.com/gjtsjc/"><font  color="#FF0000">古今图书集成</font></a> | <a href="/renwu/">历史人物</a> | <a href="/ShuoWenJieZi/"><font  color="#FF0000">说文解字</a></font> | <a href="/chengyu/">成语词典</a> | <a  target="_blank"  href="http://www.guoxuedashi.com/jgwhj/"><font  color="#FF0000">甲骨文合集</font></a> | <a href="/yzjwjc/"><font  color="#FF0000">殷周金文集成</font></a> | <a href="/xiangxingzi/"><font color="#0000FF">象形字典</font></a> | <a href="/13jing/"><font  color="#FF0000">十三经索引</font></a> | <a href="/zixing/"><font  color="#FF0000">字体转换器</font></a> | <a href="/zidian/xz/"><font color="#0000FF">篆书识别</font></a> | <a href="/jinfanyi/">近义反义词</a> | <a href="/duilian/">对联大全</a> | <a href="/jiapu/"><font  color="#0000FF">家谱族谱查询</font></a> | <a href="http://www.guoxuemi.com/hafo/" target="_blank" ><font color="#FF0000">哈佛古籍</font></a> 
</div>

 <!-- 头部导航开始 -->
<div class="w1180 head clearfix">
  <div class="head_logo l"><a title="国学大师官网" href="http://www.guoxuedashi.com" target="_blank"></a></div>
  <div class="head_sr l">
  <div id="head1">
  
  <a href="http://www.guoxuedashi.com/zidian/bujian/" target="_blank" ><img src="http://www.guoxuedashi.com/img/top1.gif" width="88" height="60" border="0" title="部件查字,支持20万汉字"></a>


<a href="http://www.guoxuedashi.com/help/yingpan.php" target="_blank"><img src="http://www.guoxuedashi.com/img/top230.gif" width="600" height="62" border="0" ></a>


  </div>
  <div id="head3"><a href="javascript:" onClick="javascript:window.external.AddFavorite(window.location.href,document.title);">添加收藏</a>
  <br><a href="/help/setie.php">搜索引擎</a>
  <br><a href="/help/zanzhu.php">赞助本站</a></div>
  <div id="head2">
 <a href="http://www.guoxuemi.com/" target="_blank"><img src="http://www.guoxuedashi.com/img/guoxuemi.gif" width="95" height="62" border="0" style="margin-left:2px;" title="国学迷"></a>
  

  </div>
</div>
  <div class="clear"></div>
  <div class="head_nav">
  <p><a href="/">首页</a> | <a href="/ShuKu/">国学书库</a> | <a href="/guji/">影印古籍</a> | <a href="/shici/">诗词宝典</a> | <a   href="/SiKuQuanShu/gxjx.php">精选</a> <b>|</b> <a href="/zidian/">汉语字典</a> | <a href="/hydcd/">汉语词典</a> | <a href="http://www.guoxuedashi.com/zidian/bujian/"><font  color="#CC0066">部件查字</font></a> | <a href="http://www.sfds.cn/"><font  color="#CC0066">书法大师</font></a> | <a href="/jgwhj/">甲骨文</a> <b>|</b> <a href="/b/4/"><font  color="#CC0066">解密</font></a> | <a href="/renwu/">历史人物</a> | <a href="/diangu/">历史典故</a> | <a href="/xingshi/">姓氏</a> | <a href="/minzu/">民族</a> <b>|</b> <a href="/mz/"><font  color="#CC0066">世界名著</font></a> | <a href="/download/">软件下载</a>
</p>
<p><a href="/b/"><font  color="#CC0066">历史</font></a> | <a href="http://skqs.guoxuedashi.com/" target="_blank">四库全书</a> |  <a href="http://www.guoxuedashi.com/search/" target="_blank"><font  color="#CC0066">全文检索</font></a> | <a href="http://www.guoxuedashi.com/shumu/">古籍书目</a> | <a   href="/24shi/">正史</a> <b>|</b> <a href="/chengyu/">成语词典</a> | <a href="/kangxi/" title="康熙字典">康熙字典</a> | <a href="/ShuoWenJieZi/">说文解字</a> | <a href="/zixing/yanbian/">字形演变</a> | <a href="/yzjwjc/">金 文</a> <b>|</b>  <a href="/shijian/nian-hao/">年号</a> | <a href="/diming/">历史地名</a> | <a href="/shijian/">历史事件</a> | <a href="/guanzhi/">官职</a> | <a href="/lishi/">知识</a> <b>|</b> <a href="/zhongyi/">中医中药</a> | <a href="http://www.guoxuedashi.com/forum/">留言反馈</a>
</p>
  </div>
</div>
<!-- 头部导航END --> 
<!-- 内容区开始 --> 
<div class="w1180 clearfix">
  <div class="info l">
   
<div class="clearfix" style="background:#f5faff;">
<script src='http://www.guoxuedashi.com/img/headersou.js'></script>

</div>
  <div class="info_tree"><a href="http://www.guoxuedashi.com">首页</a> > <a href="/SiKuQuanShu/fanti/">四库全书</a>
 > <h1>资治通鉴</h1> <!--         下载:【右键另存为】即可 --></div>
  <div class="info_content zj clearfix">
  
<div class="info_txt clearfix" id="show">
<center style="font-size:24px;">111-資治通鑑卷一百一十</center>
    資治通鑑卷一百一十  宋 司馬光 撰<br />
<br />
  胡三省 音註<br />
<br />
  晉紀三十二【著雍閹茂一年】<br />
<br />
  安皇帝乙<br />
<br />
  隆安二年春正月燕范陽王德自鄴帥戶四萬南徙滑臺【帥讀曰率下同】魏衛王儀入鄴收其倉庫追德至河弗及趙王麟上尊號於德【上時掌翻】德用兄垂故事稱燕王【事見一百五卷孝武太元九年】改永康三年為元年以統府行帝制【統府者諸方鎮皆統於燕王府行帝制者稱制以行事】置百官以趙王麟為司空領尚書令慕容法為中軍將軍慕輿拔為尚書左僕射丁通為右僕射麟復謀反德殺之【慕容麟背父叛兄姦詐反覆天下其誰能容之復扶又翻】庚子魏王珪自中山南巡至高邑得王永之子憲<br />
<br />
  喜曰王景略之孫也以為本州中正【王猛青州北海劇縣人太康中分劇屬東筦郡晉東筦屬徐州晉書載記以北海劇縣書之蓋猛自占漢郡縣也然家於魏郡而隱於華隂由是歸秦其子永鎮幽州從苻丕戰死於襄陵故憲流寓高邑今魏以為本州中正則未得青徐蓋使之銓叙東夏人士耳】領選曹事兼掌門下【選曹吏部尚書之職門下侍中常侍給事黄門之職選須絹翻】至鄴置行臺【鄭樵曰行臺自魏晉有之晉文王討諸葛誕散騎常侍裴秀尚書僕射陳泰以行臺從東海王越帥衆屯許昌以行臺自隨後魏謂之尚書大行臺别置官屬】以龍驤將軍日南公和跋為尚書與左丞賈彞帥吏兵五千人鎮鄴【自漢光武委任尚書事歸臺閣謂尚書省曰尚書臺晉惠帝西遷長安置留臺於洛陽主留事於是有留臺之名至拓跋氏置行臺隨其所置掌一道之事魏書官氏志内入諸姓有素和氏後改為和氏驤思將翻】珪自鄴還中山將北歸發卒萬人治直道【治直之翻】自望都鑿恒嶺至代五百餘里【恒嶺恒山之嶺也在上曲陽西北即倒馬關路晉書地道記謂之鴻上關沈括曰北岳恒山今謂之大茂山者是也岳祠舊在山下石晉之後稍遷近裏今其地謂之神棚今祠乃在曲陽祠北有望岳亭新晴氣清則望見大茂飛狐路在大茂之西自銀冶寨北出倒馬關却自石門子令水鋪入缾形梅回兩寨之間至代州然沈括所謂代州乃鴈門也自此亦可至魏之代都但恐非直道耳○水經註祁夷水出平舒縣柬東北流逕蘭亭南又東北逕石門關北舊道出中山故關也魏土地記代城西九里有平舒城此則古代城也恒戶登翻】珪恐已既去山東有變復置行臺於中山【復扶又翻】命衛王儀鎮之以撫軍大將軍略陽公遵為尚書左僕射鎮勃海之合口右將軍尹國督租于冀州聞珪將北還謀襲信都安南將軍長孫嵩執國斬之【長知兩翻】 燕啟倫還至龍城【去年寶遣啟崙南觀形勢倫當作崙音盧昆翻】言中山已䧟燕主寶命罷兵遼西王農言於寶曰今遷都尚新未可南征宜因成師襲庫莫奚取其牛馬以充軍資更審虚實俟明年而議之寶從之己未北行庚申渡澆洛水【澆洛水蓋即饒樂水也賢曰水在今營州北唐太宗時奚内附置饒樂都督府】會南燕王德遣侍郎李延詣寶言涉珪西上【西上謂自中山取恒嶺而西歸雲代也上時掌翻】中國空虚延追寶及之寶大喜即日引還 辛酉魏王珪發中山徙山東六州吏民雜夷十餘萬口以實代【此漢高帝徙關東豪傑以實關中之策也】博陵勃海章武羣盗竝起【漢時章武城屬勃海平舒縣界晉武帝泰始元年置章武國後為郡隋廢屬瀛州入平舒縣】略陽公遵等討平之廣川太守賀賴盧性豪健【廣川縣前漢屬信都國後漢屬清河郡晉屬勃海郡後分為廣川郡守式又翻】恥居冀州刺史王輔之下襲輔殺之驅勒守兵掠陽平頓丘諸郡南渡河奔南燕南燕王德以賴盧為并州刺史封廣甯王 西秦王乾歸遣乞伏益州攻凉支陽鸇武允吾三城克之【支陽允吾皆漢古縣屬金城郡鸇武城當在二縣之間張寔分支陽屬廣武郡允吾蓋仍為金城郡治所劉昫曰唐蘭州廣武縣漢枝陽縣鄯州龍支縣漢允吾縣允吾音鈆牙】虜萬餘人而去 燕主寶還龍城宮詔諸軍就頓【頓者軍行頓舍之地】不聽罷散文武將士皆以家屬隨駕【駕謂車駕猶漢人言乘輿也】遼西王農長樂王盛切諫【樂音洛】以為兵疲力弱魏新得志未可與敵宜且養兵觀釁寶將從之撫軍將軍慕輿騰曰百姓可與樂成難與圖始【用商鞅語意樂音洛】今師衆已集宜獨决聖心乘機進取不宜廣采異同以沮大計【沮在呂翻】寶乃曰吾計决矣敢諫者斬二月乙亥寶出就頓留盛統後事已卯燕軍發龍城慕輿騰為前軍司空農為中軍寶為後軍相去各一頓【觀下文連營百里蓋三十里為一頓】連營百里壬午寶至乙連長上段速骨宋赤眉等因衆心之憚征役遂作亂【凡衛兵皆更番迭上長上者不番代也唐官制懷化執戟長上歸德執戟長上皆武散階九品長上之官尚矣上時掌翻】速骨等皆高陽王隆舊隊共逼隆子高陽王崇為主殺樂浪威王宙中牟熙公段誼及宗室諸王【樂浪音洛琅】河間王熙素與崇善崇擁佑之故獨得免燕主寶將十餘騎奔司空農營農將出迎左右抱其腰止之曰宜小清澄【言衆方亂如水之溷濁宜少俟其定如水之清澄不可輕出也】不可便出農引刀將斫之遂出見寶又馳信追慕輿騰癸未寶農引兵還趣大營【大營謂寶營也】討速骨等農營兵亦厭征役皆棄仗走【以佚道使民雖勞不怨以生道殺民雖死不怨殺者違是鮮有不敗者也】騰營亦潰寶農犇還龍城長樂王盛聞亂引兵出迎寶農僅而得免 會稽王道子忌王殷之逼【會工外翻】以譙王尚之及弟休之有才略引為腹心尚之說道子曰今方鎮彊盛宰相權輕宜密樹腹心於外以自藩衛道子從之以其司馬王愉為江州刺史都督江州及豫州之四郡軍事用為形援日夜與尚之謀議以伺四方之隙【為庾楷說王殷復舉兵張本說輸芮翻伺相吏翻】 魏王珪如繁畤宮【繁畤縣屬鴈門郡魏築宮於此天平初置繁畤郡隋復為縣唐屬代州畤音止】給新徙民田及牛珪畋於白登山【酈道元曰今平城東十七里有臺即白登臺臺南對岡阜即白登山】見熊將數子【師古曰將謂率領也讀如字】謂冠軍將軍于栗磾曰【冠古玩翻磾丁奚翻】卿名勇健能搏此乎對曰獸賤人貴若搏而不勝豈不虚斃一壯士乎乃驅致珪前盡射而獲之【射而亦翻】珪顧謝之秀容川酋長爾朱羽健從珪攻晉陽中山有功拜散騎常侍環其所居割地三百里以封之【此北秀容也為爾朱榮亂魏張本爾朱榮傳云羽健之先世為部落酋帥居爾朱川因氏焉珪初以南秀容川原衍沃欲今居之羽健曰家世奉國給侍左右北秀容既在剗内差近京師豈以沃塉更遷遠地珪許之則北秀容蓋近平城也環音宦酋慈由翻長知兩翻散悉亶翻騎奇寄翻下同】柔然數侵魏邉【數所角翻】尚書中兵郎李先請擊之珪從之大破柔然而還【還從宣翻又如字】 楊軌以其司馬郭緯為西平相帥步騎二萬北赴郭黁秃髪烏孤遣其弟車騎將軍傉檀帥騎一萬助軌【緯于季翻相息亮翻帥讀曰率黁奴昆翻傉奴沃翻】軌至姑臧營于城北 燕尚書頓丘王蘭汗隂與段速骨等通謀引兵營龍城之東城中留守兵至少【汗音寒少詩沼翻】長樂王盛徙内近城之民得丁夫萬餘乘城以禦之速骨等同謀纔百餘人餘皆為所驅脅莫有鬬志三月甲午速骨等將攻城遼西桓烈王農恐不能守且為蘭汗所誘夜潜出赴之冀以自全【農號為有智略乃欲投段速骨以自全不知適以速死殆天奪之鑒也】明旦速骨等攻城城上拒戰甚力速骨之衆死者以百數速骨乃將農循城【將如字引也挾也】農素有忠節威名城中之衆恃以為彊忽見在城下無不驚愕喪氣【喪息浪翻】遂皆逃潰速骨入城縱兵殺掠死者狼籍寶盛與慕輿騰餘崇張真李旱趙恩等輕騎南走速骨幽農於殿内長上阿交羅速骨之謀主也【騎奇寄翻上時掌翻】以高陽王崇幼弱更欲立農崇親信鬷讓出力犍等聞之【鬷祖紅翻春秋左氏傳有鬷蔑晉有鬷戾姓譜鬷姓古鬷夷氏之後犍居言翻】丁酉殺羅及農【使速骨果立農亦必同死於蘭汗之手蓋事勢已去智無所施也】速骨即為之誅讓等【偽于偽翻】農故吏左衛將軍宇文拔亡奔遼西庚子蘭汗襲擊速骨并其黨盡殺之廢崇奉太子策承制大赦遣使迎寶及於薊城【使疏吏翻薊音計】寶欲還長樂王盛等皆曰汗之忠詐未可知今單騎赴之萬一汗有異志悔之無及不如南就范陽王合衆以取冀州若其不捷收南方之衆徐歸龍都亦未晚也寶從之【龍城燕故都故謂之龍都慕容盛智慮逾其父遠矣】 離石胡帥呼延鐵西河胡帥張崇等不樂徙代【帥所類翻樂音洛】聚衆叛魏魏安遠將軍庾岳討平之 魏王珪召衛王儀入輔以略陽公遵代鎮中山夏四月壬戌以征虜將軍穆崇為太尉安南將軍長孫嵩為司徒燕主寶從間道過鄴【間古莧翻】鄴人請留寶不許南至黎陽伏於河西【河水自遮害亭屈而東北流過黎陽縣南河之西岸為黎陽界東岸為滑臺界】遣中黄門令趙思告北地王鍾曰上以二月得丞相表【寶以德為司徒故稱之為丞相】即時南征至乙連會長上作亂失據來此【人主所據者勢也衆叛親離大勢已去失所據矣】王亟白丞相奉迎鍾德之從弟也首勸德稱尊號聞而惡之執思付獄【從才用翻惡烏路翻】以狀白南燕王德德謂羣下曰卿等以社稷大計勸吾攝政吾亦以嗣帝播越【播逋也遷也越遠也走也】民神乏主故權順羣議以繫衆心今天方悔禍嗣帝得還吾將具法駕奉迎謝罪行闕何如【天子行幸所至有行宮宮前闕門謂之行闕】黄門侍郎張華曰今天下大亂非雄才無以寧濟羣生嗣帝闇懦不能紹隆先統陛下若蹈匹夫之節捨天授之業威權一去身首不保况社稷其得血食乎慕輿護曰嗣帝不達時宜委棄國都【寶棄中山見上卷上年】自取敗亡不堪多難【難乃旦翻】亦已明矣昔蒯聵出奔衛輒不納春秋是之【蒯苦怪翻聵五怪翻】以子拒父猶可况以父拒子乎【德於寶為叔父】今趙思之言未明虚實臣請為陛下馳往詗之【為于偽翻詗火迥翻又翾正翻候俟也刺探也】德流涕遣之【流涕遣護將使之殺寶也】護帥壯士數百人【帥讀曰率下同】隨思而北聲言迎衛其實圖之寶既遣思詣鍾於後得樵者言德已稱制懼而北走護至無所見執思以還【還從宣翻又如字】德以思練習典故欲留而用之思曰犬馬猶知戀主思雖刑臣乞還就上【宦者謂之刑臣上謂寶也】德固留之思怒曰周室東遷晉鄭是依【周平王東遷洛邑晉文侯鄭武寔定王室故周桓公曰我周之東遷晉鄭焉依】殿下親則叔父位為上公不能帥先羣后以匡帝室而幸本根之傾為趙王倫之事【事見八十九卷惠帝永寧元年言趙王倫以宗室而簒晉德所為類之倫於惠帝叔祖也德於寶叔父也帥讀曰率】思雖不能如申包胥之存楚【吳破楚入郢申包胥乞師於秦遂破吳師楚昭王復國】猶慕龔君賓不偷生於莽世也【龔勝字君賓事見三十七卷王莽始建國三年】德斬之寶遣扶風忠公慕輿騰與長樂王盛收兵冀州盛以騰素暴横為民所怨乃殺之行至鉅鹿長樂說諸豪傑【横戶孟翻樂音洛說輸芮翻】皆願起兵奉寶寶以蘭汗祀燕宗廟所為似順意欲還龍城不肯留冀州乃北行至建安【建安城在令支之北乙連之南】抵民張曹家曹素武健請為寶合衆【為于偽翻】盛亦勸寶宜且駐留察汗情狀寶乃遣冗從僕射李旱先往見汗【冗而隴翻從才用翻】寶留頓石城【石城縣前漢屬右北平郡後漢晉省縣屬建德郡隋唐併入營州柳城縣界宋白曰石城縣取碣石立如城以名之】會汗遣左將軍蘇超奉迎陳汗忠欵寶以汗燕王垂之舅盛之妃父也謂必無他不待旱返遂行盛流涕固諫寶不聽留盛在後盛與將軍張真下道避匿丁亥寶至索莫汗陘【索昔各翻汗音寒陘音刑】去龍城四十里城中皆喜汗惶怖欲自出請罪【怖普布翻】兄弟共諫止之汗乃遣弟加難帥五百騎出迎又遣兄堤閉門止仗禁人出入城中皆知其將為變而無如之何加難見寶於陘北拜謁已【已者拜謁之禮畢】從寶俱進頴隂烈公餘崇密言於寶曰觀加難形色禍變甚逼宜留三思奈何徑前寶不從行數里加難先執崇崇大呼罵曰汝家幸緣肺附【呼火故翻師古曰肺附謂親戚也舊解云肺附如肺腑之相附著一說肺斫木札也喻其輕薄附著大材也】蒙國寵榮覆宗不足以報今乃敢謀簒逆此天地所不容計旦暮即屠滅但恨我不得手膾汝曹耳【膾細切肉也】加難殺之引寶入龍城外邸弑之【年四十四】汗諡寶曰靈帝殺獻哀太子策及王公卿士百餘人自稱大都督大將軍大單于昌黎王【單音蟬】改元青龍以堤為太尉加難為車騎將軍封河間王熙為遼東公如杞宋故事【周武王封夏之後於杞殷之後於宋】長樂王盛聞之馳欲赴哀張真止之盛曰我今以窮歸汗汗性愚淺必念婚姻不忍殺我旬月之間足以展吾情志遂往見汗汗妻乙氏及盛妃皆泣涕請盛於汗盛妃復頓頭於諸兄弟【復扶又翻】汗惻然哀之乃舍盛於宮中以為侍中左光祿大夫親待如舊堤加難屢請殺盛汗不從堤驕狠荒淫【狠戶墾翻】事汗多無禮盛因而間之【間古莧翻】由是汗兄弟浸相嫌忌【為盛誅汗張本】 凉太原公纂將兵擊楊軌郭黁救之纂敗還 段業使沮渠蒙遜攻西郡【郡在武威西據嶺之要蒙遜得之故晉昌敦煌皆降沮子余翻】執太守呂純以歸純光之弟子也於是晉昌太守王德敦煌太守趙郡孟敏皆以郡降業【敦徒門翻降戶江翻】業封蒙遜為臨池侯以德為酒泉太守敏為沙州刺史 六月丙子魏王珪命羣臣議國號皆曰周秦以前皆自諸侯升為天子因以其國為天下號漢氏以來皆無尺土之資我國家百世相承開基代北遂撫有方夏【據孔安國尚書註方夏謂四方中夏夏戶雅翻】今宜以代為號黄門侍郎崔宏曰昔商人不常厥居故兩稱殷商【契始封於商皇甫謐曰今上洛商是也契孫相土居商丘自契至于成湯八遷湯始居亳從先王居後仲丁遷於囂河亶甲居相祖乙居耿書曰盤庚五遷將治亳殷從先王居謂從帝嚳所居居亳也】代雖舊邦其命維新登國之初已更曰魏【事見一百六卷孝武太元十年更工衡翻】夫魏者大名神州之上國也【左傳卜偃曰魏大名也戰國之時魏為大國中國謂之神州】宜稱魏如故珪從之 楊軌自恃其衆欲與凉王光决戰郭黁每以天道抑止之【言天道未利也郭黁善數故如此黁奴昆翻】凉常山公弘鎮張掖段業使沮渠男成及王德攻之光使太原公纂將兵迎之【將即亮翻】楊軌曰呂弘精兵一萬若與光合則姑臧益彊不可取矣乃與秃髪利鹿孤共邀擊纂纂與戰大破之軌奔王乞基【王乞基田胡也】黁性褊急殘忍不為士民所附【褊補典翻】聞軌敗走降西秦【降戶江翻】西秦王乾歸以為建忠將軍散騎常侍【散悉亶翻騎奇寄翻】弘引兵棄張掖東走段業徙治張掖【治直之翻】將追擊弘沮渠蒙遜諫曰歸師勿遏窮寇勿追【孫子之言】此兵家之戒也業不從大敗而還【還從宣翻】賴蒙遜以免業城西安以其將臧莫孩為太守【業置西安郡於張掖東境孩河開翻】蒙遜曰莫孩勇而無謀知進不知退此乃為之築冢非築城也【為于偽翻冢知隴翻】業不從莫孩尋為呂纂所破 燕太原王奇楷之子蘭汗之外孫也汗亦不殺以為征南將軍得入見長樂王盛盛潛使奇逃出起兵奇起兵於建安衆至數千汗遣蘭堤討之盛謂汗曰善駒小兒未能辦此【善駒奇小字也】豈非有假託其名欲為内應者乎太尉素驕難信不宜委以大衆【汗以堤為太尉故稱之】汗然之罷堤兵【蘇軾有言木必先蠧然後蟲生之人必先疑然後讒入之蘭汗凶逆兄弟自相嫌忌故慕容盛得間之以奮其智報君父之讎】更遣撫軍將軍仇尼慕將兵討奇【更工衡翻】於是龍城自夏不雨至于秋七月汗日詣燕諸廟及寶神座頓首禱請委罪於蘭加難【言弑寶者加難之罪】堤及加難聞之怒且懼誅乙巳相與率所部襲仇尼慕軍敗之【敗補邁翻】汗大懼遣太子穆將兵討之【將即亮翻】穆謂汗曰慕容盛我之仇讎必與奇相表裏此乃腹心之疾不可養也宜先除之汗欲殺盛先引見察之盛妃知之密以告盛盛稱疾不出【蘭妃之為異於雍姬雖曰婦人内夫家而外父母家若蘭妃者處夫妻父母之變得其一而失其一者也】汗亦止不殺李旱衛雙劉忠張豪張真皆盛素所厚也而穆引以為腹心旱雙得出入至盛所潛與盛結謀丁未穆擊堤加難等破之庚戌饗將士汗穆皆醉盛夜如厠因踰垣入于東宮與旱等共殺穆時軍未解嚴皆聚在穆舍聞盛得出呼躍爭先攻汗斬之汗子魯公和陳公揚分屯令支白狼【令音鈴又郎定翻支音祁】盛遣旱真襲誅之堤加難亡匿捕得斬之於是内外帖然士女相慶宇文拔率壯士數百來赴【宇文拔自遼西來也】盛拜拔為大宗正辛亥告于太廟令曰賴五祖之休【五祖謂慕容涉歸廆皝雋垂凡五廟】文武之力宗廟社稷幽而復顯不獨孤以眇眇之身免不同天之責【禮記曰父之讎不與共戴天】凡在臣民皆得明目當世因大赦改元建平盛謙不敢稱尊號以長樂王攝行統制【盛字道運寶之庶長子也樂音洛】諸王皆降稱公以東陽公根為尚書左僕射衛倫陽璆魯恭王滕為尚書【璆渠尤翻】悦真為侍中陽哲為中書監張通為中領軍自餘文武各復舊位改諡寶曰惠閔皇帝廟號烈宗初太原王奇舉兵建安南北之人翕然從之【南人謂自中原來者北人則鮮卑也】蘭汗遣其兄子全討奇奇擊滅之匹馬不返進屯乙連盛既誅汗命奇罷兵奇用丁零嚴生烏桓王龍之謀遂不受命甲寅勒兵三萬餘人進至横溝去龍城十里盛出擊大破之執奇而還斬其黨與百餘人賜奇死桓王之嗣遂絶【慕容恪封太原王諡曰桓楚莊王滅若敖氏而赦箴尹克黄曰子文無後何以勸善以慕容恪之輔成燕業而可使之絶祀乎】羣臣固請上尊號【上時掌翻】盛弗許 魏王珪遷都平城始營宮室建宗廟立社稷宗廟歲五祭用分至及臘【魏都平城置代尹及司州於平城杜佑曰後魏都平城今雲中郡治雲中縣是今馬邑郡北平城即今郡隋為雲内縣恒安鎮此所謂宗廟即代都之東廟也】 桓玄求為廣州會稽王道子忌玄【會工外翻】不欲使居荆州因其所欲以玄為督交廣二州軍事廣州刺史玄受命而不行豫州刺史庾楷以道子割其四郡使王愉督之上疏言江州内地【江州治尋陽在江南故云内地】西府北帶寇戎【晉以京口為北府歷陽為西府豫州治歷陽在江西故云北帶寇戎】不應使愉分督朝廷不許楷怒遣其子鴻說王恭曰尚之兄弟【謂譙王尚之及弟休之也說輸芮翻下同】復秉機權【復扶又翻下同】過於國寶欲假朝威削弱方鎮【朝直遥翻】懲艾前事為禍不測【艾倪祭翻】今及其謀議未成宜早圖之恭以為然以告殷仲堪桓玄仲堪玄許之推恭為盟主刻期同趣京師【趣七喻翻】時内外疑阻津邏嚴急【邏郎佐翻巡也津邏者凡江津之要皆置邏卒】仲堪以斜絹為書内箭簳中【簳古旱翻字林曰箭笴也】合鏑漆之【鏑箭鏃也】因庾楷以送恭恭發書絹文角戾不復能辨仲堪手書【戾曲也乖也斜絹無邊幅經緯不相持故斜角乖曲】疑楷詐為之且謂仲堪去年已違期不赴【事見上卷】今必不動乃先期舉兵【先悉薦翻】司馬劉牢之諫曰將軍國之元舅會稽王天子叔父也會稽王又當國秉政曏為將軍勠其所愛王國寶王緒又送王廞書【事見上卷為于偽翻】其深伏將軍已多矣頃所授任雖未允愜亦非大失割庾楷四郡以配王愉於將軍何損晉陽之甲豈可數興乎【數所角翻】恭不從上表請討王愉司馬尚之兄弟道子使人說楷曰昔我與卿恩如骨肉【楷先黨於王國寶道子亦親之】帳中之飲結帶之言可謂親矣【此必太元二十一年庾楷赴難時事】卿今棄舊交結新援忘王恭疇昔陵侮之恥乎【王恭以元舅之親風神簡貴志氣方嚴視庾楷蔑如也故道子以為陵侮楷】若欲委體而臣之使恭得志必以卿為反覆之人安肯深相親信首身且不可保况富貴乎楷怒曰王恭昔赴山陵相王憂懼無計我知事急尋勒兵而至恭不敢發【事見一百八卷孝武太元二十一年】去年之事我亦俟命而動我事相王無相負者相王不能拒恭反殺國寶及緒自爾以來【自爾以來猶今言自那時以來也又爾言如此也】誰敢復為相王盡力者【復扶又翻為于偽翻】庾楷實不能以百口助人屠滅時楷已應恭檄正徵士馬信返朝廷憂懼内外戒嚴會稽世子元顯言於道子曰前不討王恭故有今日之難今若復從其欲【難乃旦翻復扶又翻】則太宰之禍至矣【道子時為太宰】道子不知所為悉以事委元顯日飲醇酒而已元顯聰警頗涉文義志氣果鋭以安危為已任附會之者謂元顯神武有明帝之風殷仲堪聞恭舉兵自以去歲後期乃勒兵趣發【趣讀曰促】仲堪素不習為將【將即亮翻】悉以軍事委南郡相楊佺期兄弟【相息亮翻】使佺期帥舟師五千為前鋒桓玄次之仲堪帥兵二萬相繼而下【帥讀曰率】佺期自以其先漢太尉震至父亮九世皆以才德著名矜其門地謂江左莫及有以比王珣者佺期猶恚恨而時流以其晩過江婚宦失類【佺期曾祖凖晉太常自震至凖七世有名德祖林少有才望值亂沒胡殳亮少仕偽朝後歸晉比王謝諸家為晚亮及佺期皆以武力為官又與傖荒為婚故云失類時流猶言時輩也恚於避翻】佺期及兄廣弟思平從弟孜敬皆麄獷【從才用翻獷古猛翻獷獷不可附】每排抑之佺期常慷慨切齒欲因事際以逞其志故亦贊成仲堪之謀八月佺期玄奄至湓口【湓口湓浦口也晉人於此築城置戍今其地在江州德化縣西一里湓蒲奔翻】王愉無備惶遽奔臨川【吳孫亮太平二年分豫章東部都尉立臨川郡隋唐為撫州】玄遣偏軍追獲之 燕以河間公熙為侍中車騎大將軍中領軍司隸校尉城陽公元為衛將軍元寶之子也又以劉忠為左將軍張豪為後將軍竝賜姓慕容氏李旱為中常侍輔國將軍衛雙為前將軍張順為鎮西將軍昌黎尹張真為右將軍【燕都龍城以昌黎太守為昌黎尹】皆封公 乙亥燕步兵校尉馬勒等謀反伏誅事連驃騎將軍高陽公崇崇弟東平公澄皆賜死【驃匹妙翻騎奇寄翻】 寧朔將軍鄧啓方南陽太守閭丘羨將兵二萬擊南燕【燕自慕容寶之敗北歸龍城慕容德稱號於滑臺故稱南燕以别之】與南燕中軍將軍法撫軍將軍和戰於管城【魏收志滎陽郡京縣有管城故管叔邑也杜預曰在京縣東北】啓方等兵敗單騎走免 魏王珪命有司正封畿【宋白曰魏道武都平城東至上谷軍都關西至河南至中山隘門塞北至五原地方千里以為甸服】標道里平權衡審度量遣使循行郡國【使疏吏翻行下孟翻】舉奏守宰不法者親考察黜陟之 九月辛卯加會稽王道子黄鉞以世子元顯為征討都督遣衛將軍王珣右將軍謝琰將兵討王恭譙王尚之將兵討庾楷乙未燕以東陽公根為尚書令張通為左僕射衛倫<br />
<br />
  為右僕射慕容豪為幽州刺史鎮肥如 己亥譙王尚之大破庾楷於牛渚楷單騎奔桓玄【為後玄殺楷張本】會稽王道子以尚之為豫州刺史弟恢之為驃騎司馬丹楊尹允之為吳國内史休之為襄城太守【元帝渡江以丹楊春穀縣置襄城郡】各擁兵馬以為已援乙巳桓玄大破官軍於白石【白石在巢縣界水經註柵江水導源巢湖束左會清溪水謂之清溪口柵水又東左會白石山水水發白石山西逕李鵲城南西南注柵水】玄與楊佺期進至横江尚之退走恢之所領水軍皆沒丙午道子屯中堂元顯守石頭己酉王珣守北郊謝琰屯宣陽門以備之【宣陽門建康城南面西頭第一門】王恭素以才地陵物既殺王國寶自謂威無不行仗劉牢之為爪牙而但以部曲將遇之【將即亮翻】牢之負其才深懷恥恨元顯知之遣廬江太守高素說牢之【高素亦北府將故使說之說輸芮翻下可說同】使叛恭許事成即以恭位號授之又以道子書遺牢之為陳禍福【遺于季翻為于偽翻】牢之謂其子敬宣曰王恭昔受先帝大恩今為帝舅不能翼戴王室數舉兵向京師【數所角翻】吾不能審恭之志事捷之日必能為天子相王之下乎【相王謂道子也】吾欲奉國威靈以順討逆何如敬宣曰朝廷雖無成康之美亦無幽厲之惡而恭恃其兵威暴蔑王室【蔑之者視之若無也】大人親非骨肉義非君臣雖共事少時意好不恊【少時言不多時也少詩沼翻好呼到翻】今日討之於情義何有恭參軍何澹之知其謀以告恭【澹徒覽翻】恭以澹之素與牢之有隙不信乃置酒請牢之於衆中拜之為兄精兵堅甲悉以配之【王恭素以部曲將遇牢之及聞何澹之言則拜之為兄此豈能得其死力邪適足以速其背已耳】使帥帳下督顔延為前鋒【帥讀曰率】牢之至竹里斬延以降【降戶江翻】遣敬宣及其壻東莞太守高雅之還襲恭【東莞漢舊縣武帝泰始元年分瑯邪立東莞郡南渡後又置南東莞郡於晉陵界莞音官】恭方出城曜兵敬宣縱騎横擊之【騎奇寄翻下同】恭兵皆潰恭將入城雅之已閉城門恭單騎奔曲阿素不習馬髀中生瘡曲阿人殷確恭故吏也以船載恭將奔桓玄至長塘湖【長塘湖在晉陵延陵縣杜佑曰在潤州金壇縣風土記陽羨縣有洮湖别名長塘湖洮余招翻單鍔曰長塘湖在義興西】為人所告獲之送京師斬於倪塘【倪塘在建康東北方山埭南倪氏築塘因以為名】恭臨刑猶理須神色自若【須與鬚同】謂監刑者曰我闇於信人所以至此【自悔悉以軍事委劉牢之也監工銜翻】原其本心豈不忠於社稷邪但令百世之下知有王恭耳并其子弟黨與皆死以劉牢之為都督兖青冀幽并徐揚州晉陵諸軍事以代恭俄而楊佺期桓玄至石頭殷仲堪至蕪湖元顯自竹里馳還京師遣丹楊尹王愷等發京邑士民數萬人據石頭以拒之佺期玄等上表理王恭求誅劉牢之牢之帥北府之衆馳赴京師軍于新亭【帥讀曰率】佺期玄見之失色回軍蔡洲【蔡洲在今建康府上元縣西二十五里】朝廷未知西軍虛實仲堪等擁衆數萬充斥郊畿内外憂逼【言内憂而外逼也】左衛將軍桓脩冲之子也言於道子曰西軍可說而解也【說輸芮翻】脩知其情矣殷桓之下專恃王恭恭既破滅西軍沮恐【沮恐言氣沮而心恐也沮在呂翻】今若以重利啗玄及佺期【啗徒覽翻餌也】二人必内喜玄能制仲堪佺期可使倒戈取仲堪矣道子納之以玄為江州刺史召郗恢為尚書以佺期代恢為都督梁雍秦三州諸軍事雍州刺史以脩為荆州刺史權領左衛文武之鎮【左衛文武左衛將軍府之僚屬及部曲也之往也郗丑之翻雍於用翻】又令劉牢之以千人送之黜仲堪為廣州刺史遣仲堪叔父太常茂宣詔敇仲堪回軍 張驤子超收合三千餘家據南皮自號烏桓王抄掠諸郡【張驤烏桓種也奉燕見一百五卷孝武帝太元九年歸魏見上卷元年驤思將翻抄楚交翻】魏王珪命庾岳討之 楊軌屯亷川收集夷夏衆至萬餘【夏戶雅翻】王乞基謂軌曰秃髪氏才高而兵盛且乞基之主也不如歸之軌乃遣使降於西平王烏孤【降戶江翻】軌尋為羌酋梁飢所敗【酋慈由翻敗補邁翻】西犇海【闞駰曰金城臨羌縣西有卑和羌海酈道元曰古西零之地也音憐】襲乙弗鮮卑而據其地烏孤謂羣臣曰楊軌王乞基歸誠於我卿等不速救使為羌人所覆孤甚愧之平西將軍渾屯曰【渾古有是姓左傳鄭有渾罕衛有渾良夫吐谷渾氏後改為渾姓渾戶昆翻】梁飢無經遠大略可一戰擒也飢進攻西平西平人田玄明執太守郭倖而代之以拒飢遣子為質於烏孤【質音致】烏孤欲救之羣臣憚飢兵彊多以為疑左司馬趙振曰楊軌新敗呂氏方彊洪池以北未可冀也【洪池嶺名在凉州姑臧之南唐凉州有洪池府】嶺南五郡庶幾可取【嶺南謂洪池嶺南也五郡謂廣武西平樂都澆河湟河也幾居希翻】大王若無開拓之志振不敢言若欲經營四方此機不可失也使羌得西平華夷震動非我之利也烏孤喜曰吾亦欲乘時立功安能坐守窮谷乎【廉川在塞外故謂之窮谷】乃謂羣臣曰梁飢若得西平保據山河不可復制【西平據湟河之要有大小榆谷之饒故云然復扶又翻下同】飢雖驍猛【驍堅堯翻】軍令不整易破也【易以豉翻】遂進擊飢大破之飢退屯龍支堡【唐鄯州有龍支縣劉昫曰龍支漢允吾縣地此時當為西平界】烏孤進攻拔之飢單騎奔澆河【澆河吐谷渾之地呂光開以為郡隋唐之廓州即其地也澆堅堯翻水洄洑曰澆此郡蓋置於洮河洄曲處杜佑曰澆河城在廓州達化縣賀蘭山劉昫曰廓州隋澆河郡治廣威縣即後漢燒當羌之地前凉置湟河郡後魏置石城郡廢帝因縣内化隆谷置化隆縣後周置廓州唐天寶元年改為廣威縣管下有達化縣吐渾澆河城在縣西百二十里杜佑曰澆河城吐谷渾阿豺所築】俘斬數萬以田玄明為西平内史樂都太守田瑶【樂都註已見二十六卷漢宣帝神爵元年五代志西平郡湟水縣後周置樂都郡觀此則呂氏已置郡矣杜佑曰湟水一名樂都水唐鄯州治樂音洛】湟河太守張裯【裯除留翻湟水郡蓋置於此地】澆河太守王稚皆以郡降【降戶江翻】嶺南羌胡數萬落皆附於烏孤 西秦王乾歸遣秦州牧益州武衛將軍慕兀【慕兀晉書載記作慕容兀慕兀蓋亦乞伏氏載記誤也】冠軍將軍翟瑥帥騎二萬伐吐谷渾【冠古玩翻瑥音温】冬十月癸酉燕羣臣復上尊號【復扶又翻上時掌翻】丙子長樂<br />
<br />
  王盛始即皇帝位【樂音洛】大赦尊皇后段氏曰皇太后太妃丁氏曰獻莊皇后初蘭汗之當國也盛從燕主寶出亡蘭妃奉事丁后愈謹及汗誅盛以妃當從坐欲殺之丁后以妃有保全之功固爭之得免然終不為后 大赦 殷仲堪得詔書大怒【得黜廣州之詔書也】趣桓玄楊佺期進軍【趣讀曰促】玄等喜於朝命【朝直遥翻下同】欲受之猶豫未决仲堪聞之遽自蕪湖南歸遣使告諭蔡洲軍士曰汝輩不各自散歸吾至江陵盡誅汝餘口【使疏吏翻餘口謂蔡洲之軍所餘家口留在江陵者】佺期部將劉系帥二千人先歸【帥讀曰率】玄等大懼狼狽西還【還從宣翻又如字】追仲堪至尋陽及之仲堪既失職倚玄等為援玄等亦資仲堪兵雖内相疑阻勢不得不合乃以子弟交質【質音致】壬午盟于尋陽俱不受朝命連名上疏申理王恭求誅劉牢之及譙王尚之并訴仲堪無罪獨被降黜【被皮義翻】朝廷深憚之内外騷然乃復罷桓脩【復扶又翻】以荆州還仲堪優詔慰諭以求和解仲堪等乃受詔御史中丞江績劾奏桓脩專為身計疑誤朝廷【謂分江雍以授桓玄楊佺期自取荆州也劾戶槩翻又戶得翻】詔免脩官初桓玄在荆州所為豪縱仲堪親黨皆勸仲堪殺之仲堪不聽及在尋陽資其聲地【聲謂威聲地謂門地】推玄為盟主玄愈自矜倨楊佺期為人驕悍【悍侯旰翻又下罕翻】玄每以寒士裁之佺期甚恨密說仲堪以玄終為患請於壇所襲之【說輸芮翻】仲堪忌佺期兄弟勇健恐既殺玄不可復制【復扶又翻】苦禁之於是各還所鎮玄亦知佺期之隂謀有取佺期之志【為後玄殺殷楊張本】乃屯於夏口引始安太守濟隂卞範之為長史以為謀主【吳孫皓甘露元年分零陵南部都尉立始安郡屬廣州晉成帝度屬荆州隋唐為桂州之地夏戶雅翻濟子禮翻】是時詔書獨不赦庾楷玄以楷為武昌太守初郗恢為朝廷拒西軍【為于偽翻】玄未得江州欲奪恢雍州【雍於用翻】以恢為廣州恢聞之懼詢於衆衆皆曰楊佺期來者誰不勠力若桓玄來恐難與為敵【桓氏世居西楚故衆畏之】既而聞佺期代已乃與閭丘羨謀阻兵拒之【閭丘羨時為南陽太守雍之部屬也】佺期聞之聲言玄來入沔【沔彌兖翻】以佺期為前驅恢衆信之望風皆潰恢請降【降戶江翻】佺期入府斬閭丘羨放恢還都至楊口殷仲堪隂使人殺之及其四子託言羣蠻所殺 西秦乞伏益州與吐谷渾王視羆戰於度周川【度周川在臨洮塞外龍涸之西吐從暾入聲谷音浴】視羆大敗走保白蘭山遣子宕豈為質於西秦以請和【宕徒浪翻質音致】西秦王乾歸以宗女妻之【妻子細翻】凉建武將軍李鸞以興城降於秃髪烏孤【興城在允吾縣西南】<br />
<br />
  【龍支堡之東】 十一月以琅邪王德文為衛將軍開府儀同三司征虜將軍元顯為中領軍領軍將軍王雅為尚書左僕射 辛亥魏王珪命尚書吏部郎鄧淵立官制恊音律儀曹郎清河董謐制禮儀三公郎王德定律令【吏部儀曹三公郎皆曹魏所置】太史令鼂崇考天象【鼂直遥翻】吏部尚書崔宏總而裁之以為永式淵羌之孫也【鄧羌苻秦之名將】 楊軌王乞基帥戶數千自歸於西平王烏孤【帥讀曰率下同】 十二月己丑魏王珪即皇帝位大赦改元天興命朝野皆束髪加帽【朝直遥翻下同說文曰帽小兒蠻夷蒙頭衣晉書輿服志曰帽猶冠也義取於蒙覆其首其本纚也古者冠無幘冠下有纚以繒為之後世施幘於冠因復裁纚為帽自乘輿宴居下至庶人無爵者皆服之江左時野人已著帽人士亦往往而然但其頂圓耳後乃高其屋云纚所爾翻】追尊遠祖毛以下二十七人皆為皇帝【魏諡毛為成皇帝五世至推寅南遷大澤方千餘里諡宣皇帝七世至鄰始南出居匈奴故地諡獻皇帝獻帝之子曰詰汾諡聖武皇帝】諡六世祖力微曰神元皇帝廟號始祖祖什翼犍曰昭成皇帝廟號高祖父寔曰獻明皇帝【諡力微曰神元皇帝子沙漠汗曰文皇帝沙漠汗之子弗政曰思皇帝弗政卒力微之子祿官立諡曰昭皇帝分國為三部猗㐌猗盧沙漠汗之二子與祿官分統三部猗㐌西略服屬諸國諡曰桓皇帝猗盧自祿官之卒合三部為一又助晉國以益強諡穆皇帝猗盧死祿官之子鬱律繼之諡平文皇帝鬱律弑猗㐌之子賀傉立謚惠皇帝賀傉卒弟紇那立諡煬皇帝翳槐者鬱律之子國人逐紇那而立之諡烈皇帝】魏之舊俗孟夏祀天及東廟【宗廟在東蓋亦左祖之義】季夏帥衆却霜於隂山孟秋祀天於西郊至是始依倣古制定郊廟朝饗禮樂然惟孟夏祀天親行【杜佑曰魏道武天賜二年祀天于西郊為方壇東為二陛土陛無等周垣四門各依方色為名置木主七於壇上牲用白犢黄駒白羊各一祭之日帝御大駕至郊所立青門内近南西面内朝臣皆位於壇北外朝臣及大人方客咸位於青門外后率六宮自黑門入列於青門内近北竝西面廩犧令掌牲陳於壇前女巫執鼓立於壇東西面帝七族子弟七子執酒在巫南西面北上女巫升壇揺鼓帝拜后肅拜内外百官拜祀訖乃殺牲七執酒七人西向以酒洒天神主復拜如此者三禮畢而退自是歲一祭】其餘多有司攝事又用崔宏議自謂黄帝之後【魏收曰魏之先出自黄帝黄帝子曰昌意昌意之子受封北國有大鮮卑山因以為號其後世為君長統幽都之北廣漠之野黄帝以土德王北俗謂土為托謂后為跋故以托跋為氏】以土德王徙六州二十二郡守宰豪傑二千家于代都東至代郡【魏都平城以平城為代都依漢建國之名也漢平城縣本屬鴈門郡而代郡治桑乾後漢徙高柳晉徙平舒魏收地形志之上谷郡晉之代郡也唐為蔚州之地魏之代都唐為雲州雲中縣之地】西及善無南極隂館【善無縣漢屬鴈門郡後漢屬定襄郡元魏天平二年置善無郡班志隂館縣屬鴈門郡本樓煩鄉景帝後三年置隂館縣有累頭山治水所出五代史志代州鴈門縣有纍頭山則漢之隂館縣已併入鴈門縣矣】北盡參合皆為畿内其外四方四維置八部帥以監之【魏書作八部帥八部帥勸課農耕量校收入以為殿最監工銜翻】 己亥燕幽州刺史慕容豪尚書左僕射張通昌黎尹張順坐謀反誅初琅邪人孫泰學妖術於錢唐杜子恭【妖於驕翻】士民多<br />
<br />
  奉之王珣惡之【惡烏路翻】流泰於廣州王雅薦泰於孝武帝云知養性之方召還累官至新安太守泰知晉祚將終因王恭之亂以討恭為名收合兵衆聚貨鉅億【億億為鉅億】三吳之人多從之識者皆憂其為亂以中領軍元顯與之善無敢言者會稽内史謝輶發其謀【輶夷周翻又音酉】己酉會稽王道子使元顯誘而斬之【會工外翻誘音酉】并其六子兄子恩逃入海愚民猶以泰蟬蜕不死【蟬解殻曰蜕神仙家有尸解之說言尸解登仙如蟬之蜕殻也蜕輸芮翻又吐外翻】就海中資給恩恩乃聚合亡命得百餘人以謀復讎【為後孫恩反張本】 西平王禿髪烏孤更稱武威王【更工衡翻】 是歲楊盛遣使附魏魏以盛為仇池王【使疏吏翻】<br />
<br />
  資治通鑑卷一百一十<br />
<br />
<史部,編年類,資治通鑑>  <br>
   </div> 

<script src="/search/ajaxskft.js"> </script>
 <div class="clear"></div>
<br>
<br>
 <!-- a.d-->

 <!--
<div class="info_share">
</div> 
-->
 <!--info_share--></div>   <!-- end info_content-->
  </div> <!-- end l-->

<div class="r">   <!--r-->



<div class="sidebar"  style="margin-bottom:2px;">

 
<div class="sidebar_title">工具类大全</div>
<div class="sidebar_info">
<strong><a href="http://www.guoxuedashi.com/lsditu/" target="_blank">历史地图</a></strong>  
<a href="http://www.880114.com/" target="_blank">英语宝典</a>  
<a href="http://www.guoxuedashi.com/13jing/" target="_blank">十三经检索</a> 
<br><strong><a href="http://www.guoxuedashi.com/gjtsjc/" target="_blank">古今图书集成</a></strong> 
<a href="http://www.guoxuedashi.com/duilian/" target="_blank">对联大全</a> <strong><a href="http://www.guoxuedashi.com/xiangxingzi/" target="_blank">象形文字典</a></strong> 

<br><a href="http://www.guoxuedashi.com/zixing/yanbian/">字形演变</a>  <strong><a href="http://www.guoxuemi.com/hafo/" target="_blank">哈佛燕京中文善本特藏</a></strong>
<br><strong><a href="http://www.guoxuedashi.com/csfz/" target="_blank">丛书&方志检索器</a></strong> <a href="http://www.guoxuedashi.com/yqjyy/" target="_blank">一切经音义</a>  

<br><strong><a href="http://www.guoxuedashi.com/jiapu/" target="_blank">家谱族谱查询</a></strong>  <strong><a href="http://shufa.guoxuedashi.com/sfzitie/" target="_blank">书法字帖欣赏</a></strong> 
<br>

</div>
</div>


<div class="sidebar" style="margin-bottom:0px;">

<font style="font-size:22px;line-height:32px">QQ交流群9:489193090</font>


<div class="sidebar_title">手机APP 扫描或点击</div>
<div class="sidebar_info">
<table>
<tr>
	<td width=160><a href="http://m.guoxuedashi.com/app/" target="_blank"><img src="/img/gxds-sj.png" width="140"  border="0" alt="国学大师手机版"></a></td>
	<td>
<a href="http://www.guoxuedashi.com/download/" target="_blank">app软件下载专区</a><br>
<a href="http://www.guoxuedashi.com/download/gxds.php" target="_blank">《国学大师》下载</a><br>
<a href="http://www.guoxuedashi.com/download/kxzd.php" target="_blank">《汉字宝典》下载</a><br>
<a href="http://www.guoxuedashi.com/download/scqbd.php" target="_blank">《诗词曲宝典》下载</a><br>
<a href="http://www.guoxuedashi.com/SiKuQuanShu/skqs.php" target="_blank">《四库全书》下载</a><br>
</td>
</tr>
</table>

</div>
</div>


<div class="sidebar2">
<center>


</center>
</div>

<div class="sidebar"  style="margin-bottom:2px;">
<div class="sidebar_title">网站使用教程</div>
<div class="sidebar_info">
<a href="http://www.guoxuedashi.com/help/gjsearch.php" target="_blank">如何在国学大师网下载古籍?</a><br>
<a href="http://www.guoxuedashi.com/zidian/bujian/bjjc.php" target="_blank">如何使用部件查字法快速查字?</a><br>
<a href="http://www.guoxuedashi.com/search/sjc.php" target="_blank">如何在指定的书籍中全文检索?</a><br>
<a href="http://www.guoxuedashi.com/search/skjc.php" target="_blank">如何找到一句话在《四库全书》哪一页?</a><br>
</div>
</div>


<div class="sidebar">
<div class="sidebar_title">热门书籍</div>
<div class="sidebar_info">
<a href="/so.php?sokey=%E8%B5%84%E6%B2%BB%E9%80%9A%E9%89%B4&kt=1">资治通鉴</a> <a href="/24shi/"><strong>二十四史</strong></a>&nbsp; <a href="/a2694/">野史</a>&nbsp; <a href="/SiKuQuanShu/"><strong>四库全书</strong></a>&nbsp;<a href="http://www.guoxuedashi.com/SiKuQuanShu/fanti/">繁体</a>
<br><a href="/so.php?sokey=%E7%BA%A2%E6%A5%BC%E6%A2%A6&kt=1">红楼梦</a> <a href="/a/1858x/">三国演义</a> <a href="/a/1038k/">水浒传</a> <a href="/a/1046t/">西游记</a> <a href="/a/1914o/">封神演义</a>
<br>
<a href="http://www.guoxuedashi.com/so.php?sokeygx=%E4%B8%87%E6%9C%89%E6%96%87%E5%BA%93&submit=&kt=1">万有文库</a> <a href="/a/780t/">古文观止</a> <a href="/a/1024l/">文心雕龙</a> <a href="/a/1704n/">全唐诗</a> <a href="/a/1705h/">全宋词</a>
<br><a href="http://www.guoxuedashi.com/so.php?sokeygx=%E7%99%BE%E8%A1%B2%E6%9C%AC%E4%BA%8C%E5%8D%81%E5%9B%9B%E5%8F%B2&submit=&kt=1"><strong>百衲本二十四史</strong></a>  <a href="http://www.guoxuedashi.com/so.php?sokeygx=%E5%8F%A4%E4%BB%8A%E5%9B%BE%E4%B9%A6%E9%9B%86%E6%88%90&submit=&kt=1"><strong>古今图书集成</strong></a>
<br>

<a href="http://www.guoxuedashi.com/so.php?sokeygx=%E4%B8%9B%E4%B9%A6%E9%9B%86%E6%88%90&submit=&kt=1">丛书集成</a> 
<a href="http://www.guoxuedashi.com/so.php?sokeygx=%E5%9B%9B%E9%83%A8%E4%B8%9B%E5%88%8A&submit=&kt=1"><strong>四部丛刊</strong></a>  
<a href="http://www.guoxuedashi.com/so.php?sokeygx=%E8%AF%B4%E6%96%87%E8%A7%A3%E5%AD%97&submit=&kt=1">說文解字</a> <a href="http://www.guoxuedashi.com/so.php?sokeygx=%E5%85%A8%E4%B8%8A%E5%8F%A4&submit=&kt=1">三国六朝文</a>
<br><a href="http://www.guoxuedashi.com/so.php?sokeytm=%E6%97%A5%E6%9C%AC%E5%86%85%E9%98%81%E6%96%87%E5%BA%93&submit=&kt=1"><strong>日本内阁文库</strong></a> <a href="http://www.guoxuedashi.com/so.php?sokeytm=%E5%9B%BD%E5%9B%BE%E6%96%B9%E5%BF%97%E5%90%88%E9%9B%86&ka=100&submit=">国图方志合集</a> <a href="http://www.guoxuedashi.com/so.php?sokeytm=%E5%90%84%E5%9C%B0%E6%96%B9%E5%BF%97&submit=&kt=1"><strong>各地方志</strong></a>

</div>
</div>


<div class="sidebar2">
<center>

</center>
</div>
<div class="sidebar greenbar">
<div class="sidebar_title green">四库全书</div>
<div class="sidebar_info">

《四库全书》是中国古代最大的丛书,编撰于乾隆年间,由纪昀等360多位高官、学者编撰,3800多人抄写,费时十三年编成。丛书分经、史、子、集四部,故名四库。共有3500多种书,7.9万卷,3.6万册,约8亿字,基本上囊括了古代所有图书,故称“全书”。<a href="http://www.guoxuedashi.com/SiKuQuanShu/">详细>>
</a>

</div> 
</div>

</div>  <!--end r-->

</div>
<!-- 内容区END --> 

<!-- 页脚开始 -->
<div class="shh">

</div>

<div class="w1180" style="margin-top:8px;">
<center><script src="http://www.guoxuedashi.com/img/plus.php?id=3"></script></center>
</div>
<div class="w1180 foot">
<a href="/b/thanks.php">特别致谢</a> | <a href="javascript:window.external.AddFavorite(document.location.href,document.title);">收藏本站</a> | <a href="#">欢迎投稿</a> | <a href="http://www.guoxuedashi.com/forum/">意见建议</a> | <a href="http://www.guoxuemi.com/">国学迷</a> | <a href="http://www.shuowen.net/">说文网</a><script language="javascript" type="text/javascript" src="https://js.users.51.la/17753172.js"></script><br />
  Copyright &copy; 国学大师 古典图书集成 All Rights Reserved.<br>
  
  <span style="font-size:14px">免责声明:本站非营利性站点,以方便网友为主,仅供学习研究。<br>内容由热心网友提供和网上收集,不保留版权。若侵犯了您的权益,来信即刪。scp168@qq.com</span>
  <br />
ICP证:<a href="http://www.beian.miit.gov.cn/" target="_blank">鲁ICP备19060063号</a></div>
<!-- 页脚END --> 
<script src="http://www.guoxuedashi.com/img/plus.php?id=22"></script>
<script src="http://www.guoxuedashi.com/img/tongji.js"></script>

</body>
</html>
