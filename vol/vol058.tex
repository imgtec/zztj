\chapter{資治通鑑卷五十八}
宋 司馬光 撰

胡三省 音註

漢紀五十|{
	起重光作噩盡強圉單閼凡七年}


孝靈皇帝中

光和四年春正月初置騄驥廐丞領受郡國調馬|{
	賢曰騄驥善馬也調謂徵發也調徒釣翻下同}
豪右辜榷|{
	前書音義曰辜障也榷專也謂障餘人買賣而自取其利榷古岳翻}
馬一匹至二百萬 夏四月庚子赦天下 交阯烏滸蠻久為亂|{
	烏滸蠻反事始上卷光和元年滸呼古翻}
牧守不能禁交阯人梁龍等復反攻破郡縣|{
	復扶又翻}
詔拜蘭陵令會稽朱儁為交阯刺史|{
	蘭陵縣屬東海郡會古外翻}
擊斬梁龍降者數萬人|{
	降戶江翻}
旬月盡定以功封都亭侯徵為諫議大夫 六月庚辰雨雹如雞子|{
	雨于具翻}
秋九月庚寅朔日有食之 太尉劉寛免衛尉許?為太尉|{
	?於六翻 考異曰袁紀十月許郁坐辟召錯繆免楊賜為太尉今從范書}
閏月辛酉北宫東掖庭永巷署災 司徒楊賜罷 冬十月太常陳耽為司徒 |{
	考異曰袁紀三年閏月楊賜久病罷十月陳耽為司徒蓋誤置閏於去年按長歷此年閏十月以袁紀考之閠九月為是恐長歷差一月今從范書帝紀}
鮮卑寇幽并二州檀石槐死子和連代立和連才力不及父而貪淫後出攻北地北地人射殺之|{
	射而亦翻}
其子騫曼尚幼兄子魁頭立後騫曼長大|{
	長知兩翻}
與魁頭爭國衆遂離散魁頭死弟步度根立 是歲帝作列肆於後宫使諸采女販賣更相盜竊爭鬬|{
	更工衡翻}
帝著商賈服|{
	著陟畧翻下同賈音古}
從之飲宴為樂|{
	樂音洛}
又於西園弄狗著進賢冠帶綬|{
	賢曰三禮圖曰進賢冠文官服之前高七寸後高三寸長八寸續漢志曰靈帝寵用便嬖子弟轉相汲引賣關内侯直五百萬強者貪如豺狼弱者畧不類物眞狗而冠也綬音受}
又駕四驢帝躬自操轡驅馳周旋|{
	續漢志曰驢者乃服重致遠上下山谷野人之所用耳何有帝王君子而驂駕之乎天意若曰國且大亂賢愚倒植凡執政者皆如驢也操千高翻}
京師轉相倣效驢價遂與馬齊帝好為私稸|{
	好呼到翻稸與蓄同}
收天下之珍貨每郡國貢獻先輸中署名為導行費|{
	賢曰中署内署也導引也貢獻外别有所入以為所獻希之導引也}
中常侍呂強上疏諫曰天下之財莫不生之陰陽|{
	賢曰萬物禀陰陽而生}
歸之陛下豈有公私而今中尚方斂諸郡之寶中御府積天下之繒|{
	中尚方中御府皆屬少府天子私藏也繒慈陵翻}
西園引司農之藏中廐聚太僕之馬|{
	中廐即騄驥廐}
而所輸之府輒有導行之財調廣民因費多獻少|{
	調徒弔翻少詩沼翻}
姦吏因其利百姓受其敝又阿媚之臣好獻其私|{
	好呼到翻}
容謟姑息自此而進舊典選舉委任三府尚書受奏御而已|{
	三府選其人而舉之尚書受其奏以進御}
受試任用責以成功功無可察然後付之尚書舉劾請下廷尉覆案虛實行其罪罰|{
	劾戶槩翻又戶得翻下遐稼翻}
於是三公每有所選參議掾屬咨其行狀度其器能|{
	掾俞絹翻行下孟翻度徒洛翻}
然猶有曠職廢官荒穢不治|{
	治直之翻}
今但任尚書或有詔用|{
	詔用者不由三公尚書徑以詔書用之也}
如是三公得免選舉之負尚書亦復不坐責賞無歸豈肯空自勞苦乎書奏不省|{
	復扶又翻省悉井翻}
何皇后性彊忌後宫王美人生皇子協后酖殺美人帝大怒欲廢后諸中官固請得止 大長秋華容侯曹節卒|{
	華容縣屬南郡}
中常侍趙忠代領大長秋

五年春正月辛未赦天下 詔公卿以謡言舉刺史二千石為民蠧害者太尉許?司空張濟承望内官受取貨賂|{
	?許六翻}
其宦者子弟賓客雖貪汙穢濁皆不敢問而虛糾邊遠小郡清脩有惠化者二十六人吏民詣闕陳訴司徒陳耽上言公卿所舉率黨其私所謂放鴟梟而囚鸞鳳 |{
	考異曰劉陶傳光和五年以謠言舉二千石耽與議郎曹操上言按耽已為司徒不應與議郎同上言王沈魏書曰是歲以災異博問得失太祖因此上書切諫不云與耽同上言也今但云陳耽}
帝以讓?濟由是諸坐謠言徵者悉拜議郎 二月大疫三月司徒陳耽免 夏四月旱 以太常袁隗為司徒五月庚申永樂宫署災|{
	樂音洛}
秋七月有星孛于太

微|{
	孛蒲内翻}
板楯蠻寇亂巴郡連年討之不能尅帝欲大發兵以問益州計吏漢中程包對曰板楯七姓|{
	板楯七姓羅朴督鄂度夕龔皆渠帥也楯食尹翻}
自秦世立功復其租賦|{
	復方目翻}
其人勇猛善戰昔永初中羌入漢川郡縣破壞得板楯救之羌死敗殆盡|{
	事見四十九卷安帝元初元年註亦見是年}
羌人號為神兵傳語種輩勿復南行|{
	語牛倨翻種章勇翻復扶又翻下同}
至建和二年羌復大入實賴板楯連摧破之前車騎將軍馮緄南征武陵亦倚板楯以成其功近益州郡亂太守李顒亦以板楯討而平之|{
	緄古本翻又音昆顒魚容翻}
忠功如此本無惡心長吏鄉亭更賦至重|{
	長知兩翻更工衡翻}
僕役箠楚過於奴虜|{
	箠止橤翻}
亦有嫁妻賣子或乃至自剄割雖陳寃州郡而牧守不為通理|{
	為于偽翻}
闕庭悠遠不能自聞含怨呼天無所叩愬故邑落相聚以叛戾非有謀主濳號以圖不軌今但選明能牧守|{
	守式又翻}
自然安集不煩征伐也帝從其言選用太守曹謙宣詔赦之即時皆降|{
	降戶江翻}
八月起四百尺觀於阿亭道|{
	觀古玩翻}
冬十月太尉許?罷以太常楊賜為太尉帝校獵上林苑歷函谷關遂狩于廣成苑十二月還

幸太學 桓典為侍御史宦官畏之典常乘驄馬京師為之語曰行行且止避驄馬御史|{
	驄馬青白雜色}
典焉之孫也|{
	順帝永建初焉為太傳焉榮之孫也}


六年春三月辛未赦天下 夏大旱 爵號皇后母為舞陽君 秋金城河水溢出二十餘里 五原山岸崩|{
	考異曰本紀云大冇年按今夏大旱縱使秋成亦不得為大冇年今不取}
初鉅鹿張角

奉事黄老以妖術教授號太平道呪符水以療病|{
	妖於驕翻呪職救翻}
令病者跪拜過|{
	首式又翻今道家所施符水祖張道陵蓋同此術也}
或時病愈衆共神而信之角分遣弟子周行四方轉相誑誘|{
	誑居况翻誘音酉}
十餘年間徒衆數十萬自青徐幽冀荆揚兖豫八州之人莫不畢應或棄賣財產流移犇赴塡塞道路|{
	塞悉則翻}
未至病死者亦以萬數郡縣不解其意|{
	解戶買翻}
反言角以善道教化為民所歸太尉楊賜時為司徒|{
	賜為司徒熹平五年也}
上書言角誑燿百姓遭赦不悔稍益滋蔓|{
	蔓音萬}
今若下州郡捕討恐更騷擾速成其患宜切敕刺史二千石簡别流民|{
	下遐稼翻别彼列翻}
各護歸本郡以孤弱其黨然後誅其渠帥可不勞而定會賜去位事遂留中|{
	賢曰謂所論事留在禁中未施用之余據賜以熹平六年免帥所類翻}
司徒掾劉陶復上疏申賜前議|{
	掾俞絹翻復扶又翻}
言角等陰謀益甚四方私言云角等竊入京師覘視朝政|{
	覘丑亷翻朝直遥翻}
鳥聲獸心私共鳴呼州郡忌諱不欲聞之但更相告語|{
	更工衡翻}
莫肯公文宜下明詔重募角等賞以國土有敢回避與之同罪帝殊不為意方詔陶次第春秋條例|{
	陶明春秋為之訓詁故詔之次第條例}
角遂置三十六方方猶將軍也大方萬餘人小方六七千 |{
	考異曰袁紀作坊今從范書}
各立渠帥|{
	帥所類翻}
訛言蒼天已死黄天當立歲在甲子天下大吉以白土書京城寺門|{
	寺門在京城諸官寺舍之門}
及州郡官府皆作甲子字大方馬元義等先收荆揚數萬人期會發於鄴元義數往來京師|{
	數所角翻}
以中常侍封諝徐奉等為内應|{
	諝私呂翻}
約以三月五日内外俱起

中平元年|{
	是年十二月改元}
春角弟子濟南唐周上書告之|{
	濟子禮翻 考異曰袁紀云濟陰人唐客今從范書}
於是收馬元義車裂於雒陽|{
	考異曰袁紀曰五月乙卯馬元義等於京都謀反伏誅今從范書}
詔三公司隸案驗宫省直衛及百姓有事角道者誅殺千餘人下冀州逐捕角等|{
	下遐稼翻}
角等知事已露晨夜馳敕諸方一時俱起皆著黄巾以為標幟|{
	著陟畧翻幟尺志翻又音誌}
故時人謂之黄巾賊二月角自稱天公將軍角弟寶稱地公將軍寶弟梁稱人公將軍 |{
	考異曰司馬彪九州春秋云角弟梁梁弟寶袁紀云角弟良寶今從范書}
所在燔燒官府劫略聚邑|{
	聚才喻翻}
州郡失據長吏多逃亡|{
	長知兩翻}
旬月之間天下響應京師震動安平甘陵人各執其王應賊三月戊申以河南尹何進為大將軍封慎侯|{
	慎縣屬汝南郡}
率左右羽林五營營士屯都亭脩理器械以鎮京師置函谷太谷廣成伊闕轘轅旋門孟津小平津八關都尉|{
	函谷關在河南穀城縣賢曰太谷在雒陽東廣成在河南新城縣京相璠曰伊闕在雒陽西南五十里轘轅關在緱氏縣東南水經註曰旋門坂在成臯縣西南十里孟津在河内河陽縣南小平津在河南平縣北賢曰在今鞏縣西北杜佑曰洛州新安縣東北有漢八關城}
帝召羣臣會議北地太守皇甫嵩以為宜解黨禁益出中藏錢西園廐馬以班軍士|{
	中藏府令屬少府宦者為之中藏錢漢所謂禁錢也西園廐馬即騄驥廐馬藏徂浪翻}
嵩規之兄子也上問計於中常侍呂強對曰黨錮久積人情怨憤若不赦宥輕與張角合謀為變滋大悔之無救今請先誅左右貪濁者大赦黨人料簡刺史二千石能否|{
	料音聊量也度也}
則盜無不平矣帝懼而從之壬子赦天下黨人還諸徙者|{
	謂黨人妻子徙邊者也}
唯張角不赦發天下精兵遣北中郎將盧植討張角|{
	漢有三署中郎將五官及左右署又有使匈奴中郎將北中郎將則創置於此時盖以討河北黄巾也}
左中郎將皇甫嵩右中郎將朱儁討潁川黄巾是時中常侍趙忠張讓夏惲郭勝段珪宋典等皆封侯貴寵|{
	夏戶雅翻惲於粉翻}
上常言張常侍是我公趙常侍是我母由是宦官無所憚畏並起第宅擬則宫室上嘗欲登永安候臺|{
	據續漢志永安宫在北宫東北中有候臺洛陽宫殿名曰永安宫周回六百九十八大故基在洛陽故城中}
宦官恐望見其居處乃使中大人尚但諫曰|{
	賢曰尚姓但名姓譜師尚父之後後漢有高士尚子平}
天子不當登高登高則百姓虛散上自是不敢復升臺榭|{
	觀靈帝以尚但之言不敢復升臺榭誠恐百姓虛散也謂無愛民之心可乎使其以信尚但者信諸君子之言則漢之為漢未可知也賢曰春秋濳潭巴曰天子毋高臺榭高臺榭則下叛之盖因此以誑帝也復扶又翻下同}
及封諝徐奉事發上詰責諸常侍曰|{
	詰去吉翻}
汝曹常言黨人欲為不軌皆令禁錮或有伏誅者今黨人更為國用汝曹反與張角通為可斬未皆叩頭曰此王甫侯覽所為也於是諸常侍人人求退各自徵還宗親子弟在州郡者趙忠夏惲等遂共譖呂強云與黨人共議朝廷數讀霍光傳|{
	言其欲謀廢立也數所角翻}
強兄弟所在並皆貪穢帝使中黄門持兵召強強聞帝召怒曰吾死亂起矣丈夫欲盡忠國家豈能對獄吏乎遂自殺忠惲復譖曰強見召未知所問而就外自屛|{
	賢曰自屏謂自殺也屏必郢翻}
有姦明審遂收捕其宗親沒入財產侍中河内向栩上便宜譏刺左右|{
	栩况羽翻上時掌翻下同}
張讓誣栩與張角同心欲為内應收送黄門北寺獄殺之郎中中山張鈞上書曰竊惟張角所以能興兵作亂萬民所以樂附之者|{
	樂音洛}
其源皆由十常侍多放父兄子弟婚親賓客典據州郡辜榷財利|{
	榷古岳翻}
侵掠百姓百姓之寃無所告訴故謀議不軌聚為盗賊宜斬十常侍縣頭南郊以謝百姓|{
	據宦者傳是時張讓趙忠夏惲郭勝孫璋畢嵐栗嵩段珪高望張恭韓悝宋典十二人皆為中常侍言十常侍舉大數也縣讀曰懸考異曰范書宦者傳上列常侍十二人名而下云十常侍未詳}
遣使者布告天下可不須師旅而大寇自消帝以鈞章示諸常侍皆免冠徒跣頓首乞自致雒陽詔獄並出家財以助軍費有詔皆冠履視事如故帝怒鈞曰此眞狂子也十常侍固當有一人善者不|{
	不俯九翻}
御史承旨遂誣奏鈞學黄巾道收掠死獄中|{
	掠音亮}
庚子南陽黄巾張曼成攻殺太守禇貢 帝問太尉楊賜以黄巾事賜所對切直帝不悦夏四月賜坐寇賊免以太僕弘農鄧盛為太尉已而帝閲錄故事得賜與劉陶所上張角奏乃封賜為臨晉侯|{
	臨晉縣屬馮翊賢曰故城在今同州朝邑縣西南上時掌翻}
陶為中陵鄉侯 司空張濟罷以大司農張温為司空 皇甫嵩朱儁合將四萬餘人|{
	將即亮翻}
共討潁川嵩儁各統一軍儁與賊波才戰敗嵩進保長社|{
	長社縣屬潁川郡賢曰今許州縣故城在長葛縣西}
汝南黄巾敗太守趙謙於邵陵|{
	邵陵縣屬汝南郡賢曰故城在今豫州郾陵縣東敗補邁翻}
廣陽黄巾殺幽州刺史郭勲及太守劉衛 波才圍皇甫嵩於長社嵩兵少|{
	少詩沼翻}
軍中皆恐賊依草結營會大風嵩約敕軍士皆束苣乘城|{
	賢曰苣音巨說文云束葦燒之}
使鋭士間出圍外縱火大呼|{
	間古莧翻呼火故翻}
城上舉燎應之嵩從城中鼓譟而出犇擊賊陳|{
	陳讀曰陣}
賊驚亂走會騎都尉沛國曹操將兵適至五月嵩操與朱儁合軍更與賊戰大破之斬首數萬級封嵩都鄉侯操父嵩為中常侍曹騰養子不能審其生出本末或云夏侯氏子也|{
	吳人作曹瞞傳及郭頒世語並云嵩夏侯氏之子夏侯惇之叔父操於惇為從父兄弟}
操少機警有權數而任俠放蕩不治行業|{
	少詩照翻行下孟翻下同}
世人未之奇也唯太尉橋玄及南陽何顒異焉|{
	顒魚容翻}
玄謂操曰天下將亂非命世之才不能濟也能安之者其在君乎顒見操歎曰漢家將亡安天下者必此人也玄謂操曰君未有名可交許子將子將者訓之從子劭也|{
	許劭字子將許訓為公見上卷熹平三年四年從才用翻}
好人倫多所賞識與從兄靖俱有高名好共覈論鄉黨人物每月輒更其品題故汝南俗有月旦評焉|{
	後置州郡中正本於此好呼到翻更工衡翻}
嘗為郡功曹府中聞之莫不改操飾行曹操往造劭而問之|{
	造七到翻}
曰我何如人劭鄙其為人不答操乃劫之劭曰子治世之能臣亂世之姦雄|{
	言其才絶世也天下治則盡其能為世用天下亂則逞其智為時雄}
操大喜而去|{
	曹操事始此}
朱儁之擊黄巾也其護軍司馬北地傳爕上疏曰|{
	護軍司馬官為司馬而使監護一軍}
臣聞天下之禍不由於外皆興於内是故虞舜先除四凶然後用十六相|{
	尚書舜流共工于幽州放驩兜于崇山竄三苖于三危殛鯀于羽山四罪而天下咸服左傳曰高陽氏有才子八人蒼舒隤敳檮戭大臨尨降庭堅仲容叔逹謂之八愷高辛氏有才子八人伯奮仲堪叔獻季仲伯虎仲熊叔豹季貍謂之八元舜臣堯流四凶族舉十六相}
明惡人不去則善人無由進也今張角起於趙魏黄巾亂於六州此皆釁發蕭牆而禍延四海者也臣受戎任奉辭伐罪始到潁川戰無不尅黄巾雖盛不足為廟堂憂也臣之所懼在於治水不自其源末流彌增其廣耳|{
	治直之翻}
陛下仁德寛容多所不忍故閹豎弄權忠臣不進誠使張角梟夷黄巾變服|{
	謂其黨歸順去其黄巾而復服時人之服也梟堅堯翻梟夷謂梟斬而誅夷之}
臣之所憂甫益深耳何者夫邪正之人不宜共國亦猶氷炭不可同器彼知正人之功顯而危亡之兆見皆將巧辭飾說共長虚偽|{
	見賢遍翻長知兩翻}
夫孝子疑於屢至|{
	即曾母投杼事見三卷周赧王七年}
市虎成於三夫|{
	韓子龎共與魏太子質於邯鄲共謂魏王曰今一人言市有虎王信乎王曰否二人言信乎王曰否三人言信乎王曰寡人信矣共曰夫市無虎明矣然三人言成市有虎今邯鄲去魏遠於市謗臣者過三人願王熟察之}
若不詳察眞偽忠臣將復有杜郵之戮矣|{
	白起事見五卷周赧王五十八年復扶又翻郵音尤}
陛下宜思虞舜四罪之舉速行讒佞之誅則善人思進姦凶自息趙忠見其疏而惡之|{
	惡烏路翻}
爕擊黄巾功多當封忠譖訴之帝識爕言|{
	賢曰識記也音志}
得不加罪竟亦不封 張曼成屯宛下百餘日|{
	宛於元翻}
六月南陽太守秦頡擊曼成斬之交阯土多珍貨前後刺史多無清行|{
	行下孟翻}
財計盈給輒求遷代故吏民怨叛執刺史及合浦太守來逹自稱柱天將軍三府選京令東郡賈琮為交阯刺史|{
	京縣屬河南尹琮祖宗翻}
琮到部訊其反狀咸言賦歛過重|{
	歛力贍翻}
百姓莫不空單京師遥遠告寃無所民不聊生故聚為盜賊琮即移書告示各使安其資業招撫荒散蠲復徭役|{
	蠲吉玄翻復音方目翻除也}
誅斬渠帥為大害者|{
	帥所類翻}
簡選良吏試守諸縣歲閒蕩定百姓以安巷路為之歌曰賈父來晚使我先反今見清平吏不敢飯|{
	言吏不敢過民家而飯也飯扶晚翻}
皇甫嵩朱儁乘勝進討汝南陳國黄巾追波才於陽翟擊彭脱於西華|{
	姓譜波姓也其先事王莽為波水將軍子孫以為氏陽翟縣屬潁川郡西華縣屬汝南郡賢曰西華故城在今陳州項城縣西又曰在今溵水縣西北}
並破之餘賊降散|{
	降戶江翻}
三郡悉平嵩乃上言其狀以功歸儁於是進封儁西鄉侯遷鎭賊中郎將|{
	此因欲鎮安黄巾餘賊而置官}
詔嵩討東郡儁討南陽北中郎將盧植連戰破張角斬獲萬餘人角等走保廣宗|{
	廣宗縣屬鉅鹿郡賢曰今貝州宗城縣}
植築圍鑿塹造作雲梯垂當拔之|{
	垂幾也塹七艶翻}
帝遣小黄門左豐視軍或勸植以賂送豐植不肯豐還言於帝曰廣宗賊易破耳|{
	易以䜴翻}
盧中郎固壘息軍以待天誅帝怒檻車徵植減死一等遣東中郎將隴西董卓代之|{
	盧植先為北中郎將卓為東中郎將四中郎將始於此}
巴郡張脩以妖術為人療病|{
	為于偽翻}
其灋畧與張角同令病家出五斗米號五斗米師秋七月脩聚衆反寇郡縣時人謂之米賊 |{
	考異曰范書靈帝紀有此張脩陳壽魏志張魯傳有劉焉司馬張脩劉艾典畧有漢中張脩裴松之以為張脩應是張衡非典畧之失則傳寫之誤案魯傳云祖父陵父衡皆為五斗米道衡死魯復行之劉焉司馬張脩與魯同擊漢中魯襲殺脩非其父也今此据范書}
八月皇甫嵩與黄巾戰於蒼亭|{
	蒼亭在東郡范縣界}
獲其帥卜已|{
	帥所類翻}
董卓攻張角無功扺罪乙巳詔嵩討角 九月安平王續坐不道誅|{
	安帝延光元年改樂成國曰安平以孝王得紹封續得子也}
國除初續為黄巾所虜國人贖之得還朝廷議復其國議郎李爕曰續守藩不稱|{
	稱尺證翻}
損辱聖朝不宜復國朝廷不從爕坐謗毁宗室輸作左校|{
	校戶教翻}
未滿歲王坐誅乃復拜議郎京師為之語曰父不肯立帝|{
	謂李固不肯立質桓二帝也}
子不肯立王冬十月皇甫嵩與張角弟梁戰於廣宗梁衆精勇嵩

不能尅明日乃閉營休士以觀其變知賊意稍懈|{
	懈居隘翻}
乃濳夜勒兵雞鳴馳赴其陳|{
	陳讀曰陣}
戰至晡時大破之斬梁獲首三萬級赴河死者五萬許人角先已病死剖棺戮屍傳首京師十一月嵩復攻角弟寶於下曲陽斬之|{
	下曲陽縣屬鉅鹿郡以常山有上曲陽故此稱下復扶又翻}
斬獲十餘萬人即拜嵩為左車騎將軍領冀州牧封槐里侯嵩能温卹士卒每軍行頓止須營幔脩立然後就舍軍士皆食爾乃嘗飯|{
	爾如此也}
故所嚮有功 北地先零羌及枹罕河關羣盜反|{
	河關枹罕二縣皆屬隴西郡零音憐枹音膚}
共立湟中義從胡北宫伯玉李文侯為將軍|{
	北宫以所居為氏左傳有衛大夫北宫文子孟子有北宫黝從才用翻}
殺護羌校尉泠徵|{
	賢曰泠姓也周有泠州鳩音零}
金城人邉章韓遂素著名西州羣盜誘而劫之使專任軍政|{
	誘音酉任音壬}
殺金城太守陳懿攻燒州郡初武威太守倚恃權貴恣行貪暴|{
	武威太守史失其姓名}
凉州從事武都蘇正和案致其罪刺史梁鵠懼欲殺正和以免其負訪於漢陽長史敦煌蓋勲|{
	續漢志郡太守置丞一人郡當邉戍者丞為長史敦古門翻蓋徒盍翻}
勲素與正和有仇或勸勲因此報之勲曰謀事殺良非忠也乘人之危非仁也乃諫鵠曰夫紲食鷹隼欲其鷙也|{
	賢曰紲繫也廣雅曰鷙執也取其能服執衆鳥隼聳尹翻食讀曰飤}
鷙而亨之|{
	亨讀作烹}
將何用哉鵠乃止正和詣勲求謝勲不見曰吾為梁使君謀不為蘇正和也|{
	為于偽翻}
怨之如初後刺史左昌盜軍穀數萬勲諫之昌怒使勲與從事辛曾孔常别屯阿陽以拒賊|{
	阿陽縣屬漢陽郡}
欲因軍事罪之而勲數有戰功|{
	數所角翻}
及北宫伯玉之攻金城也勲勸昌救之昌不從陳懿既死邉章等進圍昌於冀昌召勲等自救辛曾等疑不肯赴勲怒曰昔莊賈後期穰苴奮劍|{
	齊景公時燕晉侵齊景公以司馬穰苴為將禦之令寵臣莊賈監軍穰苴與期旦日會賈素驕貴夕時乃至穰苴召軍正問曰軍法期而後者云何對曰當斬遂斬賈以狥于三軍}
今之從事豈重於古之監軍乎|{
	監古䘖翻}
曾等懼而從之勲至冀誚讓章等以背叛之罪|{
	誚才笑翻背蒲妹翻}
皆曰左使君若早從君言以兵臨我庶可自改今罪已重不得降也乃解圍去叛羌圍校尉夏育於畜官|{
	前書尹翁歸傳冇論罪輸掌畜官音義曰右扶風畜牧所在有苑師之屬故曰畜官畜音許救翻}
勲與州郡合兵救育至狐槃|{
	晉時秦苻生葬姚弋仲於狐槃載記曰在天水冀縣}
為羌所敗勲餘衆不及百人身被三創|{
	敗補邁翻被皮義翻創初良翻}
堅坐不動指木表曰尸我於此句就種羌滇吾以兵扞衆曰|{
	賢曰句就羌别種句音古侯翻種章勇翻滇音顚}
蓋長史賢人汝曹殺之者為負天勲仰罵曰死反虜汝何知促來殺我衆相視而驚滇吾下馬與勲勲不肯上|{
	上時掌翻}
遂為羌所執羌服其義勇不敢加害送還漢陽後刺史楊雍表勲領漢陽太守 張曼成餘黨更以趙宏為帥衆復盛|{
	帥所類翻下同復扶又翻下同}
至十餘萬據宛城朱儁與荆州刺史徐璆等合兵圍之|{
	宛於元翻璆渠尤翻}
自六月至八月不拔有司奏徵儁司空張温上疏曰昔秦用白起燕任樂毅皆曠年歷載乃能尅敵|{
	史記白起事秦昭王為大良造攻魏破之後五年攻趙拔光狼城後七年攻楚拔鄢鄧五城明年拔郢燒夷陵遂東至竟陵樂毅事燕昭王為上將軍伐齊入臨菑徇齊五歲下七十餘城}
儁討潁川已有功效引師南指方畧已設臨軍易將|{
	將即亮翻}
兵家所忌宜假日月責其成功帝乃止儁擊宏斬之賊帥韓忠復據宛拒儁儁鳴鼓攻其西南賊悉衆赴之儁自將精卒掩其東北乘城而入忠乃退保小城惶懼乞降|{
	將即亮翻降尸江翻並下同}
諸將皆欲聽之儁曰兵固有形同而埶異者昔秦項之際民無定主故賞附以勸來耳今海内一統唯黄巾造逆納降無以勸善討之足以懲惡今若受之更開逆意賊利則進戰鈍則乞降縱敵長寇|{
	長知兩翻}
非良計也因急攻連戰不尅儁登土山望之顧謂司馬張超曰吾知之矣賊今外圍周固内營逼急乞降不受欲出不得所以死戰也萬人一心猶不可當况十萬乎不如徹圍并兵入城忠見圍解埶必自出自出則意散破之道也既而解圍忠果出戰儁因擊大破之斬首萬餘級南陽太守秦頡殺忠餘衆復奉孫夏為帥還屯宛儁急攻之司馬孫堅率衆先登癸巳拔宛城孫夏走儁追至西鄂精山|{
	西鄂縣屬南陽郡賢曰故城在今鄧州向城縣南精山在其南}
復破之斬萬餘級於是黄巾破散其餘州郡所誅一郡數千人 十二月己巳赦天下改元 豫州刺史太原王允破黄巾得張讓賓客書與黄巾交通上之|{
	上時掌翻}
上責怒讓讓叩頭陳謝竟亦不能罪也讓由是以事中允|{
	中竹仲翻中傷也}
遂傳下獄|{
	賢曰傳逮也傳妹戀翻下遐稼翻}
會赦還為刺史旬日間復以他罪被捕|{
	被皮義翻}
楊賜不欲使更楚辱|{
	賢曰更經也楚苦痛更工衡翻}
遣客謝之曰君以張讓之事故一月再徵凶慝難量|{
	量音良}
幸為深計|{
	賢曰深計謂令自死}
諸從事好氣决者|{
	好呼到翻}
共流涕奉藥而進之允厲聲曰吾為人臣獲罪於君當伏大辟以謝天下|{
	辟毗亦翻}
豈有乳藥求死乎|{
	前書王嘉傳何謂咀藥而死乳當作咀}
投杯而起出就檻車既至大將軍進與楊賜袁隗共上疏請之得減死論 |{
	考異曰允傳云大尉袁隗司徒楊賜按隗賜時皆不為此官恐誤也}


二年春正月大疫 二月己酉南宫雲臺災庚戍樂城門災|{
	據續漢志蓋樂成殿門也城當作成五行志作樂城門劉昭曰南宫中門也}
中常侍張讓趙忠說帝歛天下田畮十錢|{
	說輸芮翻歛力贍翻畮古畒字}
以脩宫室鑄銅人樂安太守陸康上疏諫曰昔魯宣税畮而蝝災自生|{
	公羊傳曰初税畒者何履畝而税也何休註云宣公無恩信於人人不肯盡力於公田起履踐案行其畒穀好者税取之蝝螽子也傳曰冬蝝生此其言蝝生何上變古易常也註云上公也謂宣公變易公田舊制而税畝也蝝余專翻}
哀公增賦而孔子非之|{
	左傳季孫欲以田賦使冉有訪諸仲尼仲尼私非冉有曰子季孫若欲行而灋則周公之典在若欲苟而行又何訪焉}
豈有聚奪民物以營無用之銅人捐捨聖戒自蹈亡王之灋哉内倖譖康援引亡國以譬聖明|{
	援于元翻}
大不敬檻車徵諸廷尉侍御史劉岱表陳解釋得免歸田里康續之孫也|{
	陸續事見四十五卷明帝永平十四年}
又詔發州郡材木文石部送京師黄門常侍輒令譴呵不中者因強折賤買僅得本賈十分之一|{
	中竹仲翻賈讀曰價}
因復貨之宦官復不為即受材木遂至腐積宫室連年不成刺史太守復增私調|{
	復扶又翻調徒弔翻}
百姓呼嗟又令西園騶分道督趣|{
	騶側尤翻趣讀曰促}
恐動州郡多受賕賂刺史二千石及茂才孝亷遷除皆責助軍脩宫錢大郡至二三千萬餘各有差當之官者皆先至西園諧價然後得去|{
	賢曰諧謂平定其價也}
其守清者乞不之官皆廹遣之時鉅鹿太守河内司馬直新除以有清名減責三百萬直被詔悵然曰為民父母而反割剝百姓以稱時求|{
	被皮義翻稱尺證翻}
吾不忍也辭疾不聽行至孟津上書極陳當世之失即吞藥自殺書奏帝為暫絶脩宫錢|{
	為于偽翻}
以朱儁為右車騎將軍 自張角之亂所在盜賊並起博陵張牛角常山禇飛燕及黄龍左校于氐根張白騎劉石左髭文八平漢大計司隸綠城雷公浮雲白雀楊鳳于毒五鹿李大目白繞眭固苦蝤之徒不可勝數|{
	朱儁傳曰輕便者言飛燕于氐根賢註曰左傳曰于思于思杜預云于思多須之貌騎白馬者為張白騎大聲者稱雷公大眼者為大目左髭文八作左髭大八校戶教翻騎奇寄翻眭息隨翻蝤才由翻勝音升}
大者二三萬小者六七千人張牛角褚飛燕合軍攻癭陶|{
	癭於郢翻}
牛角中流矢|{
	中竹仲翻}
且死令其衆奉飛燕為帥|{
	帥所類翻}
改姓張飛燕名燕輕勇趫捷故軍中號曰飛燕|{
	趫邱妖翻}
山谷寇賊多附之部衆寖廣殆至百萬號黑山賊|{
	杜佑曰衛州衛縣漢朝歌縣也紂都朝歌在今縣西縣西北有黑山}
河北諸郡縣並被其害|{
	被皮義翻}
朝廷不能討燕乃遣使至京師奏書乞降|{
	降戶江翻}
遂拜燕平難中郎將|{
	難乃旦翻}
使領河北諸山谷事歲得孝亷計吏 司徒袁隗免|{
	隗五罪翻}
二月以廷尉崔烈為司徒烈寔之從兄也|{
	崔寔作政論從才用翻}
是時三公往往因常侍阿保入錢西園而得之|{
	賢曰阿保謂傅母也余謂阿母保母也}
段熲張温等雖有功勤名譽|{
	熲古迥翻}
然皆先輸貨財乃登公位烈因傳母入錢五百萬故得為司徒及拜日天子臨軒百僚畢會帝顧謂親幸者曰悔不少靳可至千萬|{
	賢曰靳固之也居焮翻}
程夫人於傍應曰崔公冀州名士豈肯買官賴我得是反不知姝邪|{
	賢曰姝美也言反不知斯事之美也姝春朱翻}
烈由是聲譽頓衰 北宫伯玉等寇三輔詔左車騎將軍皇甫嵩鎮長安以討之時凉州兵亂不止徵發天下役賦無已崔烈以為宜棄凉州詔會公卿百官議之議郎傳爕厲言曰斬司徒天下乃安尚書奏爕廷辱大臣帝以問爕對曰樊噲以冒頓悖逆憤激思奮未失人臣之節季布猶曰噲可斬也|{
	事見十二卷惠帝三年}
今凉州天下要衝國家藩衛高祖初興使酈商别定隴右|{
	高祖以將軍酈商為隴西都尉别定北地郡}
世宗拓境列置四郡|{
	武帝元狩二年匈奴渾邪王降太初元年置酒泉張掖郡四年以休屠王地為武威郡後元年分酒泉郡置敦煌郡}
議者以為斷匈奴右臂|{
	斷丁管翻}
今牧御失和使一州叛逆烈為宰相不念為國思所以弭之之策|{
	為于偽翻}
乃欲割棄一方萬里之土臣竊惑之若使左袵之虜得居此地|{
	說文曰袵衣衿夷狄之人左袵}
士勁甲堅因以為亂此天下之至慮社稷之深憂也若烈不知是極蔽也知而故言是不忠也帝善而從之 夏四月庚戌大雨雹|{
	雨于具翻}
五月太尉鄧盛罷以太僕河南張延為太尉 六月以討張角功封中常侍張讓等十二人為列侯 秋七月三輔螟|{
	說文曰螟蟲食穀葉者}
皇甫嵩之討張角也過鄴見中常侍趙忠舍宅踰制奏沒入之又中常侍張讓私求錢五千萬嵩不與二人由是奏嵩連戰無功所費者多徵嵩還收左車騎將軍印綬削戶六千|{
	綬音受}
八月以司空張温為車騎將軍執金吾袁滂為副以討北宫伯玉拜中郎將董卓為破虜將軍與盪寇將軍周愼並統於温 九月以特進楊賜為司空冬十月庚寅臨晉文烈侯楊賜薨以光禄大夫許相為司空相訓之子也|{
	建寧二年許訓為司徒}
諫議大夫劉陶上言天下前遇張角之亂後遭邉章之寇今西羌逆類已攻河東恐遂轉盛豕突上京|{
	河東東南至雒陽五百里耳}
民有百走退死之心而無一前鬭生之計西寇浸前車騎孤危|{
	車騎謂張温也}
假令失利其敗不救臣自知言數見厭|{
	數所角翻}
而言不自裁者以為國安則臣蒙其慶國危則臣亦先亡也謹復陳當今要急八事|{
	復扶又翻}
大較言天下大亂皆由宦官宦官共讒陶曰前張角事發詔書示以威恩自此以來各各改悔今者四方安静而陶疾害聖政專言妖孽|{
	妖於驕翻孽魚列翻}
州郡不上|{
	上時掌翻}
陶何緣知疑陶與賊通情於是收陶下黄門北寺獄掠按日急|{
	下遐稼翻掠音亮}
陶謂使者曰臣恨不與伊呂同疇而以三仁為輩|{
	孔子曰殷有三仁焉微子去之箕子為之奴比干諫而死}
今上殺忠謇之臣下有憔悴之民|{
	悴泰醉翻}
亦在不久後悔何及遂閉氣而死前司徒陳耽為人忠正宦官怨之亦誣䧟死獄中 張温將諸郡兵步騎十餘萬屯美陽|{
	美陽縣屬扶風賢曰在今雍州武功縣北杜佑曰美陽本前漢頻陽縣}
邉章韓遂亦進兵美陽温與戰輒不利十一月董卓與右扶風鮑鴻等并兵攻章遂大破之章遂走榆中|{
	榆中縣屬金城郡賢曰故城在今蘭州金城縣東杜佑曰蘭州治五泉縣漢榆中故城在今縣東}
温遣周慎將三萬人追之參軍事孫堅說愼曰賊城中無穀當外轉糧食堅願得萬人斷其運道|{
	參軍事之官始見於此杜佑曰漢靈帝時陶謙幽州刺史參司空車騎將軍張温軍事時孫堅亦為參軍晉時軍府乃置為官員說輸芮翻斷丁管翻下同}
將軍以大兵繼後賊必困乏而不敢戰走入羌中并力討之則凉州可定也愼不從引軍圍榆中城而章遂分屯葵園峽反斷愼運道愼懼棄車重而退|{
	重直用翻}
温又使董卓將兵三萬討先零羌|{
	零音隣}
羌胡圍卓於望垣北|{
	望垣縣屬漢陽郡陳壽三國志曰望垣峽名}
糧食乏絶乃於所度水中立以捕魚而濳從下過軍|{
	賢曰續漢書字作堰其字義則同但異體耳}
比賊追之|{
	比必寐翻}
决水已深不得度遂還屯扶風張温以詔書召卓卓良久乃詣温温責讓卓應對不順孫堅前耳語謂温曰|{
	耳語附耳而語也}
卓不怖罪|{
	怖普布翻}
而䲭張大語宜以召不時至陳軍灋斬之温曰卓素著威名於河隴之間今日殺之西行無依堅曰明公親率王師威震天下何賴於卓觀卓所言不假明公輕上無禮一罪也章遂跋扈經年當以時進討而卓云未可沮軍疑衆二罪也|{
	沮在呂翻}
卓受任無功應召稽留而軒昂自高三罪也古之名將仗钺臨衆未有不斷斬以成功者也今明公垂意於卓|{
	垂意猶言降意也斷丁亂翻}
不即加誅虧損威刑於是在矣温不忍發乃曰君且還卓將疑人堅遂出 是歲帝造萬金堂於西園引司農金錢繒帛牣積堂中|{
	賢曰牣滿也}
復藏寄小黄門常侍家錢各數千萬|{
	復扶又翻}
又於河間買田宅起第觀|{
	帝故封河間解凟亭侯觀古玩翻}


三年春二月江夏兵趙慈反|{
	夏戶雅翻}
殺南陽太守秦頡庚戌赦天下 太尉張延罷遣使者持節就長安拜張温為太尉三公在外始於温 以中常侍趙忠為車騎將軍帝使忠論討黄巾之功執金吾甄舉謂忠曰|{
	甄之人翻}
傅南容前在東軍有功不侯|{
	傳爕字南容不侯事見上年}
天下失望今將軍親當重任宜進賢理屈以副衆心忠納其言遣弟城門校尉延致殷勤於傳爕延謂爕曰南容少答我常侍|{
	少詩沼翻}
萬戶侯不足得也爕正色拒之曰有功不論命也傅爕豈求私賞哉忠愈懷恨然憚其名不敢害出為漢陽太守 |{
	考異曰袁紀在明年九月今從范書}
帝使鉤盾令宋典脩南宫玉堂|{
	南宫有玉堂殿}
又使掖庭令畢嵐鑄四銅人又鑄四鐘皆受二千斛|{
	賢曰銅人列於蒼龍玄武闕外鐘懸於雲臺及玉堂殿前}
又鑄天祿蝦蟇吐水於平門外橋東轉水入宫|{
	賢曰天祿獸也案今鄧州南陽縣北有宗資碑旁有兩石獸鐫其膞一曰天禄一曰辟邪此即天禄辟邪並獸名也漢有天禄閣亦因獸以立名}
又作翻車渇烏施於橋西用灑南北郊路|{
	賢曰翻車設機車以引水渇烏為曲桶以氣引水上也車尺遮翻}
以為可省百姓灑道之費 五月壬辰晦日有食之 六月荆州刺史王敏討趙慈斬之 車騎將軍趙忠罷 冬十月武陵蠻反郡兵討破之 前太尉張延為宦官所譖下獄死 十二月鮮卑寇幽并二州 徵張温還京師

四年春正月己卯赦天下 二月滎陽賊殺中牟令|{
	中牟縣屬河南尹賢曰今鄭州縣}
三月河南尹何苖討滎陽賊破之拜苖為車騎將軍 韓遂殺邉章及北宫伯玉李文侯擁兵十餘萬進圍隴西太守李相如叛與遂連和凉州刺史耿鄙率六郡兵討遂鄙任治中程球|{
	百官志州刺史置從事史員職畧與司隸同無都官從事其功曹從事為治中從事主州選署及衆事}
球通姦利士民怨之漢陽太守傅爕謂鄙曰使君統政日淺民未知教賊聞大軍將至必萬人一心邉兵多勇其鋒難當而新合之衆上下未和萬一内變雖悔無及不若息軍養德明賞必罰賊得寛挺|{
	賢曰挺解也又緩也}
必謂我怯羣惡爭勢其離可必然後率已教之民討成離之賊其功可坐而待也鄙不從夏四月鄙行至狄道州别駕反應賊|{
	别駕從事刺史行部則奉引録衆事}
先殺程球次害鄙賊遂進圍漢陽城中兵少糧盡爕猶固守時北地胡騎數千隨賊攻郡皆夙懷爕思共於城外叩頭求送爕歸鄉里|{
	傳爕北地靈州人}
爕子幹年十三言於爕曰國家昏亂遂令大人不容於朝|{
	朝直遥翻}
今兵不足以自守宜聽羌胡之請還鄉里徐俟有道而輔之言未終爕慨然歎曰汝知吾必死邪聖逹節次守節|{
	左傳曹公子臧曰聖逹節次守節下失節}
殷紂暴虐伯夷不食周粟而死吾遭世亂不能養浩然之志食禄又欲避其難乎|{
	難乃旦翻}
吾行何之必死於此汝有才智勉之勉之主簿楊會吾之程嬰也|{
	史記趙朔娶晉成公姊為夫人晉景公三年屠岸賈殺趙朔滅其族朔妻冇遺腹走公宫朔客公孫杵臼謂客程嬰曰胡不死嬰曰朔之婦冇遺腹即幸而生男吾奉之即女也吾徐死耳居無何朔妻生男屠岸賈聞之乃索於公宫朔妻置兒於絝中祝曰趙宗滅乎若啼即不滅若無聲及索兒竟無聲程嬰曰今一索不得後必復索之杵臼乃取他嬰兒負之匿山中諸將攻殺杵臼并兒然趙孤兒乃在程嬰所即趙武也居十五年景公乃立趙武為卿而復其田邑}
狄道人王國使故酒泉太守黄衍說爕曰天下已非復漢有府君寧有意為吾屬帥乎|{
	帥所類翻}
爕按劍叱衍曰若剖符之臣反為賊說邪遂麾左右進兵臨陳戰殁|{
	說輪芮翻為于偽翻陳讀曰陣 考異曰袁紀在明年五月今從范書}
耿鄙司馬扶風馬騰亦擁兵反與韓遂合共推王國為主寇掠三輔 太尉張温以寇賊未平免以司徒崔烈為太尉五月以司空許相為司徒光禄勲沛國丁宫為司空初張温發幽州烏桓突騎三千以討凉州故中山相漁

陽張純請將之温不聽而使涿令遼西公孫瓚將之|{
	涿郡治涿縣瓚藏旱翻}
軍到薊中烏桓以牢禀逋縣|{
	縣讀曰懸牢價直也稟給也賢曰前書音義牢廪食也古者名廪為牢}
多叛還本國張純忿不得將|{
	將即亮翻}
乃與同郡故泰山太守張舉及烏桓大人邱力居等連盟劫略薊中|{
	薊音計}
殺護烏桓校尉公綦稠|{
	公綦復姓}
右北平太守劉政遼東太守陽終等衆至十餘萬屯肥如|{
	肥如縣屬遼西郡應劭曰肥子奔燕燕封於此賢曰故城今平州}
舉稱天子純稱彌天將軍安定王移書州郡云舉當代漢告天子避位敕公卿奉迎 冬十月長沙賊區星自稱將軍|{
	區烏侯翻姓也又如字 考異曰范書作觀鵠今從陳壽吳志}
衆萬餘人詔以議郎孫堅為長沙太守討擊平之封堅烏程侯|{
	烏程縣屬吳郡為堅以長沙兵討董卓張本}
十一月太尉崔烈罷以大司農曹嵩為太尉 十二月屠各胡反|{
	屠各胡即匈奴也屠直於翻}
是歲賣關内侯直五百萬錢 前太丘長陳寔卒|{
	長知兩翻}
海内赴弔者三萬餘人寔在鄉閭平心率物其有爭訟輒求判正|{
	判分也剖也剖析而見正理也}
曉譬曲直退無怨者至乃歎曰寧為刑罰所加不為陳君所短楊賜陳耽每拜公卿羣僚畢賀輒歎寔大位未登愧於先之|{
	先悉薦翻}


資治通鑑卷五十八
