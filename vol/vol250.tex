\section{資治通鑑卷二百五十}
宋 司馬光 撰

胡三省 音註

唐紀六十六|{
	起上章執徐盡彊圉大淵獻凡八年}


懿宗昭聖恭惠孝皇帝上|{
	諱漼宣宗長子也初諱溫嗣位更名}


咸通元年|{
	是年十一月始改元咸通}
春正月乙卯浙東軍與裘甫戰於桐栢觀前|{
	桐栢觀在台州唐興縣天台山宋改唐興縣為天台縣桐栢觀賜額崇道觀觀古玩翻}
范居植死劉勍僅以身免乙丑甫帥其徒千餘人陷剡縣|{
	帥讀曰率}
開府庫募壯士衆至數千人越州大恐時二浙久安人不習戰甲兵朽鈍見卒不滿三百|{
	見賢遍翻}
鄭祗德更募新卒以益之軍吏受賂率皆得孱弱者|{
	孱鉏山翻}
祗德遣子將沈君縱副將張公署望海鎭將李珪|{
	子將小將也望海鎮在明州界今定海縣即其地元和十四年浙東觀察使薛戎奏望海鎮去明州七十餘里俯臨大海與新羅日本諸蕃接界將即亮翻下同}
將新卒五百擊裘甫二月辛卯與甫戰於剡西賊設伏於三溪之南而陳於三溪之北|{
	三溪在今縣西南一溪自新昌縣東來一溪自磕下山南來與新昌溪會於湖塍屈而西北流溪流若三派然故謂之三溪}
壅溪上流使可涉既戰陽敗走官軍追之半涉决壅水大至官軍大敗三將皆死官軍幾盡|{
	幾居依翻}
於是山海諸盜及它道無賴亡命之徒四面雲集衆至三萬分為三十二隊其小帥有謀略者推劉暀|{
	帥所類翻暀于放翻又乎曠翻}
勇力推劉慶劉從簡羣盜皆遙通書幣求屬麾下甫自稱天下都知兵馬使改元曰羅平鑄印曰天平大聚資糧購良工治器械聲震中原|{
	治直之翻}
丙申葬聖武獻文孝皇帝于貞陵|{
	此諡正葬貞陵陵中册諡也貞陵在京兆雲陽縣西北四十里}
廟號宣宗丙午白敏中入朝墜陛傷腰肩輿以歸 鄭祗德累

表告急且求救於鄰道浙西遣牙將凌茂貞將四百人宣歙遣牙將白琮將三百人赴之|{
	歙書涉翻}
祗德始令屯郭門及東小江|{
	越州有東小江西小江東小江出剡溪至曹娥百官渡而東入海西小江出諸暨至錢清渡而東入于海皆曰小江者以浙江為大江也}
尋復召還府中以自衛|{
	復扶又翻}
祗德饋之比度支常饋多十三倍而宣潤將士猶以為不足|{
	史言元帥威令不振則惠䙝而將士不以為德度徒洛翻}
宣潤將士請土軍為導以與賊戰諸將或稱病或陽墜馬其肯行者必先邀職級|{
	職者軍職級者勲級}
竟不果遣賊遊騎至平水東小江|{
	越州會稽縣東南有平水鎮又東踰山即小江也此又一小江源出大木山南流合于剡江故係平水東以别東小江}
城中士民儲舟裹糧夜坐待旦各謀逃潰朝廷知祗德懦怯議選武將代之夏侯孜曰浙東山海幽阻可以計取難以力攻西班中無可語者|{
	唐凡朝會文官班於東武官班於西故謂武官為西班}
前安南都護王式雖儒家子|{
	王式王播弟起之子也舊史以為播子}
在安南威服華夷名聞遠近|{
	聞音問}
可任也諸相皆以為然|{
	相息亮翻}
遂以式為觀察使徵祗德為賓客|{
	太子賓客間慢局員也}
三月辛亥朔式入對上問以討賊方略對曰但得兵賊必可破有宦官侍側曰發兵所費甚大式曰臣為國家惜費則不然|{
	為于偽翻}
兵多賊速破其費省矣若兵少不能勝賊延引歲月賊勢益張|{
	張知亮翻}
則江淮羣盜將蜂起應之國家用度盡仰江淮|{
	仰牛向翻}
若阻絶不通則上自九廟下及十軍|{
	肅宗以後羽林龍武神武神威神策皆分左右號北門十軍元和二年省神武軍明年又省神威軍以其兵騎分隸左右神策而猶存十軍之名}
皆無以供給其費豈可勝計哉|{
	勝音升}
上顧宦官曰當與之兵乃詔發忠武義成淮南等諸道兵授之裘甫分兵掠衢婺州婺州押牙房郅散將樓曾|{
	散將者牙將之散員也散悉但翻將即亮翻下同}
衢州十將方景深將兵拒險賊不得入又分兵掠明州明州之民相與謀曰賊若入城妻子皆為葅醢况貨財能保之乎乃自相帥出財募勇士|{
	帥讀曰率}
治器械樹栅浚溝斷橋為固守之備|{
	治直之翻斷丁管翻}
賊又遣兵掠台州破唐興|{
	吳分章安之西界置始平縣晉改為始豐縣宋廢唐武德初分臨海置唐興縣宋改曰天台九域志在台州西一百一十里}
己巳甫自將萬餘人掠上虞焚之|{
	上虞漢古縣唐屬越州北域志在州東一百一十里}
癸酉入餘姚殺丞尉|{
	餘姚漢古縣唐屬越州九域志在州東一百四十七里宋白曰餘姚舊縣在餘姚山西風土記云舜支庶所封舜姓姚故曰餘姚}
東破慈溪入奉化抵寧海殺其令而據之|{
	開元二十六年分明州之鄮縣置慈溪縣在州西三十七里又分鄮縣置奉化縣在州南八十里武德四年分臨海縣置寧海縣屬台州九域志在州東北一百七十里}
分兵圍象山所過俘其少壯|{
	少詩照翻}
餘老弱者蹂踐殺之|{
	蹂忍久翻踐慈演翻}
及王式除書下浙東人心稍安|{
	下遐稼翻}
裘甫方與其徒飲酒聞之不樂|{
	聞王式來心有所憚樂音洛}
劉暀歎曰有如此之衆而策畫未定良可惜也今朝廷遣王中丞將兵來|{
	王式蓋檢校御史中丞}
聞其人智勇無敵不四十日必至兵馬使宜急引兵取越州憑城郭據府庫遣兵五千守西陵循浙江築壘以拒之|{
	西陵渡在越州西一百二十二里今西興渡是也吳越王錢鏐惡西陵之名改曰西興}
大集舟艦得間則長驅進取浙西|{
	間古莧翻}
過大江掠揚州貨財以自實|{
	揚州江淮之都會也轉運鹽鐵使及度支之貨財聚焉故劉暀朶頤}
還修石頭城而守之宣歙江西必有響應者遣劉從簡以萬人循海而南襲取福建如此則國家貢賦之地盡入於我矣|{
	唐自中世以後貢賦皆仰東南故云然}
但恐子孫不能守耳終吾身保無憂也|{
	觀劉暀策畫豈可以小盜待之乎}
甫曰醉矣明日議之暀以甫不用其言怒陽醉而出有進士王輅在賊中賊客之輅說甫曰如劉副使之謀乃孫權所為也彼乘天下大亂故能據有江東今中國無事此功未易成也|{
	說式芮翻易以豉翻}
不如擁衆據險自守陸耕海漁急則逃入海島此萬全策也甫畏式猶豫未决夏四月式行至柿口義成軍不整式欲斬其將久乃釋之|{
	將即亮翻}
自是軍所過若無人至西陵裘甫遣使請降式曰是必無降心直欲窺吾所為且欲使吾驕怠耳乃謂使者曰甫面縛以來當免而死|{
	而汝也}
乙未式入越州既交政為鄭祗德置酒|{
	為于偽翻}
曰式主軍政不可以飲監軍但與衆賓盡醉迨夜繼以燭曰式在此賊安能妨人樂飲|{
	樂音洛下同}
丙申餞祗德于遠郊復樂飲而歸|{
	杜子春周禮注曰五十里為近郊百里為遠郊以今地里考之越州百里至蕭山縣王式豈能送鄭祗德至此邪記事者華言耳復扶又翻}
於是始修軍令告饋餉不足者息矣稱疾臥家者起矣先求遷職者默矣賊别帥洪師簡許會能帥所部降|{
	别帥所類翻能帥讀曰率降戶江翻}
式曰汝降是也當立効以自異|{
	立効謂立功也}
使帥其徒為前鋒|{
	帥讀曰率}
與賊戰有功乃奏以官先是賊諜入越州軍吏匿而飲食之|{
	先悉薦翻諜徒協翻飲於禁翻食祥吏翻}
文武將吏往往潛與賊通求城破之日免死及全妻子或詐引賊將來降實窺虚實城中密謀屏語|{
	屏必郢翻}
賊皆知之式隂察知悉捕索斬之刑將吏尤横猾者|{
	索山客翻横戶孟翻}
嚴門禁無驗者不得出入警夜周密賊始不知我所為矣式命諸縣開倉廩以賑貧乏或曰賊未滅軍食方急不可散也式曰非汝所知官軍少騎卒|{
	少詩沼翻}
式曰吐蕃囘鶻比配江淮者|{
	比毗至翻}
其人習險阻便鞍馬可用也舉籍府中得驍健者百餘人|{
	凡吐蕃囘鶻之配隸浙東觀察府者舉其籍而取之}
虜久覊旅所部遇之無狀|{
	無善狀也}
困餧甚|{
	餧與餒同}
式既犒飲又賙其父母妻子皆泣拜讙呼|{
	讙與喧同}
願效死悉以為騎卒使騎將石宗本將之凡在管内者皆視此籍之又奏得龍陂監馬二百匹|{
	龍陂漢穎川郟縣之摩陂也唐在汝州界置馬監宋白曰元和十三年十一月賜蔡州羣牧號龍陂牧}
於是騎兵足矣或請為烽燧以詗賊遠近衆寡|{
	詗翾正翻又火迥翻}
式笑而不應選懦卒使乘健馬少與之兵以為候騎|{
	少詩沼翻}
衆怪之不敢問於是閲諸營見卒|{
	見賢遍翻}
及土團子弟得四千人使導軍分路討賊府下無守兵更籍土團千人以補之乃命宣歙將白琮浙西將凌茂貞帥本軍北來將韓宗政等帥土團合千人石宋本帥騎兵為前鋒自上虞趨奉化解象山之圍號東路軍|{
	將即亮翻帥讀曰率趨七喻翻}
又以義成將白宗建忠將游君楚|{
	唐無建忠軍按此時發忠武軍從王式史逸武字也白宗建人姓名}
淮南將萬璘帥本軍與台州唐興軍合號南路軍令之曰毋爭險易|{
	易以豉翻}
毋焚廬舍毋殺平民以增首級平民脅從者募降之|{
	降戶江翻}
得賊金帛官無所問俘獲者皆越人也釋之癸卯南路軍拔賊沃洲寨|{
	沃洲在今越州新昌縣東南}
甲辰拔新昌寨|{
	新昌時屬剡縣界今置新昌縣在越州東南二百二十里}
破賊將毛應天進拔唐興 白敏中三表辭位上不許右補闕王譜上疏以為陛下致理之初乃宰相盡心之日不可暫闕敏中自正月臥疾今四月矣陛下雖與他相坐語未嘗三刻天下之事陛下嘗暇與之講論乎|{
	相息亮翻}
願聼敏中罷去延訪碩德以資聰明己酉貶譜為陽翟令譜珪之六世孫也|{
	王珪事太宗以直聞譜博古翻}
五月庚戌朔給事中鄭公輿封還貶譜敕書上令宰相議之宰相以為譜侵敏中竟貶之 辛亥浙東東路軍破賊將孫馬騎於寧海戊午南路軍大破賊將劉暀毛應天於唐興南谷斬應天先是王式以兵少奏更發忠武義成軍及請昭義軍詔從之|{
	先悉薦翻}
三道兵至越州式命忠武將張茵將三百人屯唐興斷賊南出之道|{
	斷音短下同}
義成將高羅鋭將三百人益以台州土軍徑趨寧海|{
	趨七喻翻}
攻賊巢穴昭義將跌戣將四百人|{
	奚結翻跌徒結翻戣渠龜翻}
益東路軍斷賊入明州之道庚申南路軍大破賊於海遊鎭|{
	海遊鎮在寧海南九十里}
賊入甬溪洞|{
	甬溪洞在寧海西南百餘里屬唐興縣界又西則楢溪產鐵}
戊辰官軍屯於洞口賊出洞戰又破之己巳高羅鋭襲賊别帥劉平天寨破之|{
	帥所類翻}
自是諸軍與賊十九戰賊連敗劉暀謂裘甫曰曏從吾謀入越州寧有此困邪王輅等進士數人在賊中皆衣緑|{
	衣於既翻}
暀悉斬之曰亂我謀者此青蟲也高羅鋭克寧海收其逃散之民得七千餘人王式曰賊窘且飢必逃入海入海則歲月間未可擒也命羅鋭軍海口以拒之|{
	海口在寧海東北四十餘里}
又命望海鎭將雲思益浙西將王克容將水軍巡海澨|{
	澨市制翻水際曰澨}
思益等遇賊將劉簡於寧海東賊不虞水軍遽至|{
	虞度也}
皆棄船走山谷|{
	走音奏}
得其船十七盡焚之式曰賊無所逃矣惟黃罕嶺可入剡|{
	黄罕嶺在奉化縣西北剡縣之東其路深險度黃罕嶺則平川四十里至剡}
恨無兵以守之雖然亦成擒矣裘甫既失寧海乃帥其徒屯南陳館下|{
	南陳館在寧海西南六十餘里帥讀曰率}
衆尚萬餘人辛未東路軍破賊將孫馬騎於上疁村|{
	上疁村在寧海西北四十餘里今謂之上寮山疁力留翻}
賊將王臯懼請降 壬申右拾遺内供奉薛調上言以為兵興以來賦歛無度|{
	上時掌翻歛力贍翻}
所在羣盜半是逃戶固須翦滅亦可閔傷望敕州縣税外毋得科率仍敕長吏嚴加糾察從之 袁王紳薨|{
	紳順宗子}
戊寅浙東東路軍大破裘甫於南陳館斬首數千級賊委棄繒帛盈路|{
	繒慈陵翻}
以緩追者跌戣令士卒敢顧者斬毋敢犯者賊果自黄罕嶺遁去六月甲申復入剡|{
	復扶又翻下同}
諸軍失甫不知所在義成將張茵在唐興獲俘將苦之俘曰賊入剡矣苟捨我我請為軍導從之茵後甫一日至剡壁其東南府中聞甫入剡復大恐王式曰賊來就擒耳命趣東南兩路軍會於剡|{
	趣讀曰促}
辛卯圍之賊城守甚堅攻之不能拔諸將議絶溪水以渴之|{
	剡城東南臨溪西北負山城中多鑿井以引山泉非絶溪水所能渇作史者乃北人臆說耳今浙東諸縣皆無城獨剡縣有城猶為完壯}
賊知之乃出戰三日凡八十三戰賊雖敗官軍亦疲賊請降諸將以白式式曰賊欲少休耳|{
	少詩沼翻}
益謹備之功垂成矣賊果復出又三戰庚子夜裘甫劉暀劉慶從百餘人出降遙與諸將語離城數十步官軍疾趨斷其後|{
	離力智翻斷音短}
遂擒之壬寅甫等至越州式腰斬暀慶等二十餘人械甫送京師 |{
	考異曰平剡録曰諸軍圍賊於剡賊悍甚其所謂女軍者亦乘城摘礫以中人三日凡八十三戰賊雖衂官軍亦疲裘甫佯言乞降諸將使騎來白公曰賊憊蹔休耳謹備之仍遣押牙薛敬義謂諸將曰功成矣勉之勿怠也果復三戰二十一日夜甫與劉暀劉慶十餘輩又從百餘人出遙與諸將語伺我軍之懈將使勇者潰圍焉諸將得公誡夜皆設伏於營前甫輩離城數十步伏兵疾走以間之鋭師數百復繼之城中賊不出甫遽甚不知所為遂成擒焉至是用兵六十六日矣二十三日縛致府城公於衙門陳兵以見執其徒劉暀劉慶二十餘輩三斬之械裘甫獻闕下玉泉子見聞録曰王式討裘甫甫始起於剡既為官軍所敗復入於剡城堅卒鋭不可遽拔式乃約降許奏以金吾將軍甫許焉其將劉暀獨以為不可比及越城左右則械手以木曳頸以組甫曰吾既已降何用是為左右曰法也到越則釋去公且行有命矣既至式登南樓俟之曰裘甫何罪罪皆劉暀輩命三斬之暀顧謂甫曰君竟拜金吾乎斬甫于長安東市初甫之入剡也雖已屢敗向使城守朞歲未可平也玉泉子曰古人有言殺降不祥李廣所以不侯良有以也王公亦不聞大貴鄭公述平剡録一何曲筆哉雖驟歷清顯而卒以喪明不復起可不愼哉按二書所言莫知孰是然裘甫在剡城窮困已極勢不能久式不必更以詐誘之或者諸將為之不可知也甫之出降也或欲突走或被誘而來皆不可知要之為出城乞降官軍因邀斷其後擒之耳}
剡城猶未下諸將已擒甫不復設備劉從簡帥壯士五百突圍走諸將追至大蘭山|{
	今明州奉化縣西北有大蘭山山在越州分界復扶又翻帥讀曰率}
從簡據險自守秋七月丁巳諸將共攻克之台州刺史李師望募賊相捕斬之以自贖所降數百人得從簡首獻之|{
	大蘭既破劉從簡走入台州界方為其黨所殺}
諸將還越式大置酒諸將乃請曰某等生長軍中久更行陳|{
	長知兩翻更工衡翻行戶剛翻}
今年得從公破賊然私有所不諭者敢問公之始至軍食方急而遽散以賑貧乏何也式曰此易知耳|{
	易以䜴翻}
賊聚穀以誘飢人吾給之食則彼不為盜矣且諸縣無守兵賊至則倉穀適足資之耳又問不置烽燧何也式曰烽燧所以趣救兵也|{
	趣讀曰促}
兵盡行城中無兵以繼之徒驚士民使自潰亂耳又問使懦卒為候騎而少給兵何也式曰彼勇卒操利兵|{
	操七高翻}
遇敵且不量力而鬭鬬死則賊至不知矣皆曰非所及也|{
	自至德以來浙東盜起者再袁晁裘甫是也裘甫之禍不烈於袁晁袁晁之難張伯儀平之通鑑所書數語而已今王式之平裘甫通鑑書之視張伯儀平袁晁事為詳蓋唐中世之後家有私史王式儒家子也成功之後紀事者不無張大通鑑因其文而序之弗覺其煩耳容齋隨筆曰通鑑書討裘甫事用平剡録蓋亦有見於此考異二十卷辯訂唐事者居大半焉亦以唐私史之多也}
封憲宗子為信王|{
	彌遣翻}
八月裘甫至京師斬于東市加王式檢校右散騎常侍諸將官賞各有差先是上每以越盜為憂|{
	先悉薦翻}
夏侯孜曰王式才有餘不日告捷矣孜與式書曰公專以執裘甫為事軍須細大此期悉力|{
	軍須謂行軍所須糧仗衣物悉力謂盡力應辦也}
故式所奏求無不從由是能成其功 衛王灌薨|{
	灌上弟也}
九月白敏中五上表辭位辛亥以敏中為司徒中書令 右拾遺句容劉鄴上言李德裕父子為相有聲迹功効|{
	李德裕父吉甫相憲宗德裕相武宗皆有勲勞在於王室備著前紀}
竄逐以來血屬將盡生涯已空宜賜哀閔贈以一官冬十月丁亥敕復李德裕太子少保衛國公贈左僕射|{
	德裕貶見二百四十八卷宣宗大中元年 考異曰裴旦李太尉南行録載咸通二年九月二十六日右拾遺内供奉劉鄴表略云子曄貶立山尉去年獲遇陛下惟新之命覃作解之恩移授郴縣尉今已沒於貶所又曰血屬已盡生涯悉空又曰枯骨未歸於塋域一男又殞於江湘又曰其李德裕請特賜贈官敕依奏實録注引東觀奏記云令狐相綯夢德裕曰某已謝明時幸相公哀之許歸葬故里綯具為其子滈言之滈曰李衛公犯衆怒又崔相鉉魏相謩皆敵人也見持政必將上前異同未可言也後數日上將坐延英綯又夢德裕曰某委骨海上思還故里與相公有舊幸憫而許之既寤復謂滈曰向見衛公精爽尚可畏吾不言必掇禍明日入中書且為同列言之既而於帝前論奏許其子蒙州立山尉曄護喪歸葬又是時柳仲郢鎮東蜀設奠於荆南命從事李商隱為文曰躬承新渥言還舊止又云身留蜀郡路隔伊川鄴奏乃云孤骨未歸塋域曄懿宗初纔徙郴縣尉未詳或者後人偽作之非鄴本奏也實録注又云白敏中為中書令時與右庶子段全緯書云故衛公太尉災興鴿鳥怨結江魚親交雨散於西園子弟蓬飄於南土嘗蒙一顧繼履三台保持獲盡於天年論請爰加於寵贈全緯嘗為德裕西川從事故敏中語及云按此似繇敏中開發而數本追復贈官多連鄴奏德裕素有恩於敏中敏中前作相既遠貶之至此又掠其美鄙哉按劉鄴表云去年獲遇陛下惟新之命覃作解之恩則上此表在咸通元年非二年也舊傳鄴為翰林學士承旨以李德裕貶死朱崖大中朝令狐綯當權累有赦宥不蒙恩列懿宗即位綯在方鎮屬郊天大赦鄴奏論之李太尉南行録鄴此時未為翰林學士因上此表敕批便令内養宣喚入翰林充學士餘依奏金華子雜編曰宣宗嘗私行經延資庫見廣厦連綿錢帛山積問左右曰誰為此庫侍臣對曰宰相李德裕執政日以天下每歲備用之餘自是已來邊庭有急支備無乏者兹實有賴上曰今何在曰頃以坐吳湘獄貶于崖州上曰如有此功於國微罪豈合深譴由是劉公鄴得以進表乞追雪之上一覽表遂許其加贈歸葬焉按宣宗素惡德裕故始即位即逐之豈有不知其在崖州而云豈合深譴又劉鄴追雪在懿宗時此說殊為淺陋今不取}
己亥以門下侍郎同平章事夏侯孜同平章事充西川節度使以戶部尚書判度支畢諴為禮部尚書同平章事 安南都護李鄠復取播州|{
	播州屬黔中道大中十三年為雲南所陷此非安南巡屬也李鄠越境收復欲以為功而不知蠻兵乘虚已陷安南也鄠音戶復扶又翻}
十一月丁丑上祀圓丘赦改元 十二月戊申安南土蠻引南詔兵合三萬餘人乘虚攻交趾陷之|{
	考異曰新南詔傳大中時李琢為安南經略使苛暴自私以斗鹽易一牛夷人不堪結南詔將段酋遷陷安}


|{
	南都護府號白衣沒命軍懿宗絶其朝貢乃陷播州安南都護李鄠屯武州咸通元年為蠻所攻棄州走天子斥鄠以王寛代之按宣宗時南詔未嘗陷安南据新傳則似大中時已陷安南咸通元年又陷武州也且李鄠安南失守然後奔武州非在武州而棄之新傳誤也今從實録}
都護李鄠與監軍奔武州|{
	新志邕管所領又有顯州武州沈州後皆廢省據此則武州當在宜州界}


二年春正月詔發邕管及鄰道兵救安南擊南蠻 二月以中書令白敏中兼中書令充鳳翔節度使以左僕射判度支杜悰兼門下侍郎同平章事一日兩樞密使詣中書宣徽使楊公慶繼至獨揖悰受宣|{
	受宣受宣命也}
三相起避之西軒|{
	三相畢諴杜審權蔣伸也}
公慶出斜封文書以授悰發之乃宣宗大漸時請鄆王監國奏也且曰當時宰相無名者當以反法處之悰反復讀|{
	處昌呂翻復音覆又如字}
良久曰聖主登極萬方欣戴今日此文書非臣下所宜窺復封以授公慶曰主上欲罪宰相當於延英面示聖旨明行誅譴公慶去悰復與兩樞密坐謂曰内外之臣事猶一體宰相樞密共參國政今主上新踐阼未熟萬機資内外裨補固當以仁愛為先刑殺為後豈得遽贊成殺宰相事若主上習以性成則中尉樞密權重禁闈|{
	時以兩中尉兩樞密為四貴}
豈得不自憂乎|{
	言殺宰相則上手滑矣中尉樞密亦將及禍豈得不自以為憂}
悰受恩六朝|{
	六朝謂憲穆敬文武宣}
所望致君堯舜不欲朝廷以愛憎行法兩樞密相顧默然徐曰當具以公言白至尊非公重德無人及此慙悚而退三相復來見悰微請宣意悰無言三相惶怖乞存家族悰曰勿為它慮既而寂然無復宣命及延英開上色甚悦|{
	意此亦是據杜悰家傳書之其詞旨抑揚容有過其實者洪邁隨筆曰按懿宗即位之日宰相四人曰令狐綯曰蕭鄴曰夏侯孜曰蔣伸至是惟有伸在三人者罷去矣諴及審權乃懿宗自用者無有斯事蓋野史之妄溫公以唐事屬之范祖禹其審取可謂詳盡尚如此信乎修史之難哉 考異曰新傳云宣宗大漸樞密使王歸長等矯詔迎鄆王立之懿宗即位欲罪大臣悰解之按立鄆王者王宗實新傳云歸長誤也今從補國史}
是時士大夫深疾宦官事有小相涉則衆共棄之建州進士葉京嘗預宣武軍宴識監軍之面既而及第在長安與同年出遊遇之於塗馬上相揖因之謗議諠然遂沈廢終身其不相悦如此|{
	東漢黨錮之禍蓋亦如此但李杜諸公風節凜凜千載之下讀其事者猶使人心神肅然晩唐諸人不能企其萬一也而亦以胎清流之禍哀哉沈持林翻}
福王綰薨|{
	綰順宗子}
夏六月癸丑以鹽州防禦使王寛為安南經略使時李鄠自武州牧集土軍攻羣蠻復取安南朝廷責其失守貶儋州司戶鄠初至安南殺蠻酋杜守澄其宗黨遂誘道羣蠻陷交趾|{
	道讀曰導}
朝廷以杜氏彊盛務在姑息冀收其力用乃贈守澄父存誠金吾將軍再舉鄠殺守澄之罪長流崖州|{
	劉昫曰唐武德四年以隋朱崖郡為崖州自雷州徐聞縣南舟行四百三十里度大海達崖州宋白曰宋開寶六年割舊崖州之地屬瓊州却改振州為崖州 考異曰實録又賜寛手詔云云如聞李琢在安南日殺害杜存誠李鄠又處置其子守澄使誘導羣蠻陷沒城邑卿到鎮日於李鄠處索取前後敕詔一一參詳初李琢在鎮蠻首領愛州刺史兼土軍兵馬使杜存誠密誘溪洞夷獠為之鄉導琢察其不忠戮死焉及李鄠至鎮蠻陷安南鄠走武州召土軍收復城邑而存誠家兵甚衆朝廷務姑息乃贈存誠金吾將軍鄠以失備貶儋州補國史蠻陷安南李鄠投武州召土軍收復頗有功績殺首領杜存誠以捍禦盤桓不戮力盡敵兼洞夷獠為鄉導之罪也鄠貶儋州後以存誠谿洞強獷家兵數多子弟繼總軍旅皆輸忠勇軍府倚賴方甚朝廷亦加姑息乃再舉憲章長流鄠崖州贈存誠金吾將軍以誘其竭力命前鹽州刺史王宙為都護按鄠所殺存誠之子守澄已為王式所逐鄠至旬日殺之非因扞禦不戮力也代鄠者乃王寛非王宙補國史誤也今獨取鄠克復安南一事餘皆從平剡録實録 按唐朝若以杜守澄之戮為李鄠罪則當贈守澄官不當贈其父官此余所以致疑於前也}
秋七月南詔攻邕州陷之先是廣桂容三道共發兵三千人戍邕州三年一代|{
	先悉薦翻}
經略使段文楚請以三道衣糧自募土軍以代之朝廷許之所募纔得五百許人文楚入為金吾將軍經略使李蒙利其闕額衣糧以自入悉罷遣三道戍卒止以所募兵守左右江比舊什減七八故蠻人乘虚入寇時蒙已卒經略使李弘源至鎮纔十日無兵以禦之城陷弘源與監軍脱身奔巒州|{
	宋白曰邕州古南越城晉置晉興郡隋廢郡為宣化縣唐武德四年於此置南晉州貞觀六年改邕州至長安五千六百里巒州秦桂林郡地唐置淳州後改巒州至京師五千三百里西至邕州三百里}
二十餘日蠻去乃還|{
	還從宣翻又如字}
弘源坐貶建州司戶文楚時為殿中監復以為邕管經略使至鎭城邑居人什不存一文楚秀實之孫也|{
	段秀實死於朱泚之難}
杜悰上言南詔向化七十年|{
	貞元間南詔復向化}
蜀中寢兵無事羣蠻率服|{
	率服謂相率而服從也}
今西川兵食單寡未可輕與之絶且應遣使弔祭曉諭清平官等以新王名犯廟諱故未行册命|{
	事始見上卷大中十三年}
待其更名謝恩|{
	更工衡翻}
然後遣使册命庶全大體上從之命左司郎中孟穆為弔祭使未發會南詔寇嶲州攻卭崍關穆遂不行 |{
	考異曰實録在此年十二月按補國史杜邠公再入輔建議遣使弔祭令其改名纔命使臣已破越嶲城池攻卭崍關鎮使臣逗留數月不發然則命穆充使當在寇嶲州前實録書於十二月誤也按南詔已稱帝陷安南豈可彌悰但欲姑息故陽不知其僭號及以陷安南者為土蠻耳}
冬十月以御史大夫鄭涯為山南東道節度使十一月加同平章事

三年春正月庚寅朔羣臣上尊號曰睿文明聖孝德皇帝赦天下 以中書侍郎同平章事蔣伸同平章事充河中節度使 二月棣王惴薨 南詔復寇安南經略使王寛數來告急|{
	復扶又翻數所角翻}
朝廷以前湖南觀察使蔡襲代之 |{
	考異曰補國史王宙有緝理撫衆才遠人懷惠纔未周歲南蠻復侵封部請兵設備累以危急上聞乃命桂管都防禦使蔡襲代之實録以前湖南觀察使為安南經略等使王寛亦制置失宜諸部蠻相帥内寇故命襲往代焉今從之}
仍發許滑徐汴荆襄潭鄂等道兵各三萬人|{
	各三萬人則八道之兵為二十四萬不既多乎疑各字誤否則萬字誤蜀本作合三萬人良是}
授襲以禦之兵勢既盛蠻遂引去 |{
	考異曰實録咸通三年二月以蔡襲為安南經略招討處置等使三月以蔡京充荆襄以南宣慰安撫使五月以京為嶺南西道節度使舊紀三年十一月遣蔡襲帥禁軍三千赴援安南按補國史云咸通三年使左庶子蔡京制置嶺南事又云命桂管都防禦使蔡襲代王宙然則襲除安南似在咸通二年也又按樊綽蠻書云臣咸通三年三月四日奉本使尚書蔡襲手示密委臣深入賊帥朱道古營寨三月八日入賊重圍之中臣却囘一一白於都護王寛領得臣書牒全無指揮擅放軍囘苟求朝奬致襲枉傷矢石陷失城池徵之其由莫非蔡京王寛之過綽既謂襲為本使為之入蠻則是襲已到官又云囘白都護王寛則是寛猶未去任也不知綽不白襲而白寛何故也又襲將兵代寛寛為已替之人安能擅放軍囘令襲陷沒疑蠻書擅放軍囘字上少蔡京二字襲除安南不知的在何年月今從實録}
邕管經略使段文楚坐變更舊制|{
	謂募土軍以代廣桂容戍軍更工衡翻}
左遷威衛將軍分司 |{
	考異曰補國史文楚到後城邑牢落人戶彫殘纔得數月朝廷責其更改舊制降授威衛分司蓋文楚既之官而朝議責邕州陷沒由文楚請罷三道戍兵自募土軍故云更改舊制而實録云及文楚再至城池圯廢人戶殘耗由是頗更舊制未數月朝廷慮致煩擾復改命懷玉焉新傳文楚數更條約衆不悦以胡懷玉代之蓋因補國史改更舊制之語相承致誤也}
左庶子蔡京性貪虐多詐時相以為有吏才|{
	相息亮翻}
奏遣制置嶺南事三月京還奏事稱旨|{
	稱尺證翻}
復以京權知太僕卿充荆襄以南宣慰安撫使|{
	為蔡京奔敗張本復扶又翻}
夏四月己亥朔敕於兩街四寺各置戒壇度人三七日|{
	兩街四寺謂慈恩薦福西明莊嚴也三七二十一日}
上奉佛太過怠於政事嘗於咸泰殿築壇為内寺尼受戒|{
	内寺尼蓋宮人捨俗者就禁中為寺以處之非敎也}
兩街僧尼皆入預又於禁中設講席自唱經手録梵夾|{
	梵夾者貝葉經也以板夾之謂之梵夾段成式曰貝多葉出摩伽佗西國土用以寫經其樹長六七丈經冬不凋}
又數幸諸寺施與無度|{
	數所角翻施式豉翻}
吏部侍郎蕭倣上疏以為玄祖之道慈儉為先素王之風仁義為首|{
	玄祖謂唐祖老子尊為玄元皇帝也素王謂孔子也}
垂範百代必不可加佛者棄位出家割愛中之至難取滅後之殊勝|{
	人情莫不愛其親莫不愛富貴佛者棄父母之親捨王子之貴而出家是割愛中之至難又釋氏為宏闊勝大之言以為佛滅度後諸天神王供養莊嚴皆人世所希有後人又奉其法而尊事之是取滅後之殊勝也}
非帝王所宜慕也願陛下時開延英接對四輔力求人瘼|{
	瘼音莫病也}
䖍奉宗祧思繆賞與濫刑其殃必至知勝殘而去殺得福甚多罷去講筵|{
	繆靡幼翻勝音升去羌呂翻講筵與僧尼講經之筵}
躬勤政事上雖嘉奬竟不能從 嶺南舊分五管廣桂邕容安南皆隸嶺南節度使蔡京奏請分嶺南為兩道節度從之五月敕以廣州為東道邕州為西道又割桂管龔象二州容管藤巖二州隸邕管尋以嶺南節度使韋宙為東道節度使以蔡京為西道節度使蔡襲將諸道兵在安南蔡京忌之恐其立功奏稱南蠻遠遁邊徼無虞|{
	徼吉弔翻}
武夫邀功妄占戍兵|{
	占之贍翻}
虚費餽運蓋以荒陬路遠|{
	陬將侯翻}
難於覆驗故得肆其姦詐請罷戍兵各還本道朝廷從之襲累奏羣蠻伺隙日久不可無備|{
	伺相吏翻}
乞留戍兵五千人不聽襲以蠻寇必至交趾兵食皆闕謀力兩窮作十必死狀申中書時相信京之言終不之省|{
	時相苟求省餽運之費故京之言易入襲之請不行省悉景翻}
秋七月徐州軍亂逐節度使溫璋 |{
	考異曰舊傳曰璋咸通末為徐泗節度使徐州牙卒曰銀刀軍頗驕横璋至誅其凶惡者五百人自是軍中畏法按誅銀刀軍者王式也舊傳誤}
初王智興既得徐州募勇悍之士二千人號銀刀彫旗門槍挾馬等七軍常以三百餘人自衛露刃坐於兩廡夾幕之下每月一更|{
	更工衡翻}
其後節度使多儒臣其兵浸驕小不如意一夫大呼|{
	呼火故翻}
其衆皆和之|{
	和戶臥翻}
節度使輒自後門逃去前節度使田牟至與之雜坐飲酒把臂拊背或為之執板唱歌|{
	為于偽翻}
犒賜之費日以萬計風雨寒暑復加勞來|{
	復扶又翻勞力到翻來力代翻}
猶時喧譁邀求不已牟薨璋代之驕兵素聞璋性嚴憚之璋開懷慰撫而驕兵終懷猜忌賜酒食皆不歷口一旦竟聚譟而逐之朝廷知璋無辜乙亥以璋為邠寧節度使以浙東觀察使王式為武寧節度使 以前西川節度使同平章事夏侯孜為左僕射同平章事 忠武義成兩軍從王式討裘甫者猶在浙東詔式帥以赴徐州|{
	帥讀曰率}
驕兵聞之甚懼八月式至大彭館|{
	大彭館在徐州城外大彭即彭祖所謂商有大彭霸諸侯者也一曰彭祖姓籛名鏗事帝堯歷虞夏至商年八百歲封於彭城故彭城人以名館}
始出迎謁式視事三日饗兩鎭將士遣還鎭擐甲執兵命圍驕兵盡殺之銀刀都將邵澤等數千人皆死 |{
	考異曰舊傳曰式至鎮盡誅銀刀等七軍徐方平定金華子雜編曰温璋失律於徐州自河陽移式往鎮之式領河陽全軍赴任徐州將士聞式到近境遣衙隊三百人遠接式衩衣坐胡床受參既畢乃問其逐帥之罪命皆斬於帳前不留一人既而相次繼來莫知前死者音耗至則又斬之亦無脱者如是數日銀刀都數千人垂盡虎狼之衆居常咸謂能吞噬於人及于斯際式衣襖子半臂曳屐危坐逐人皆拱手就戮無一敢旅拒者其後親戚相訝不能自會焉按若頓殺數千人豈有人不知者又式自浙東除武寧非河陽也今從實録}
甲子敕以徐州先隸湽青道李洧自歸始置徐海使額|{
	見二百二十七卷德宗建中三年}
及張建封以威名寵任特帖濠泗二州|{
	見二百三十三卷貞元四年}
當時本以控扼湽青光蔡自寇孽消弭而武寧一道職為亂階今改為徐州團練使隸兖海節度復以濠州歸淮南道更於宿州置宿泗都團練觀察使|{
	憲宗元和四年析徐州之符離蘄泗州之虹置宿州治埇橋在徐州南界汴水上當舟車之會宋白曰宿州取古宿國為名}
留將士三千人守徐州餘皆分隸兖宿且以王式為武寧節度使兼徐泗濠宿制置使委式與監軍楊玄質分配將士赴諸道訖然後將忠武義成兩道兵至汴滑各遣歸本道身詣京師其銀刀等軍逃匿將士聽一月内自首|{
	首手又翻}
一切勿問 嶺南西道節度使蔡京為政苛慘設炮烙之刑闔境怨之遂為邕州軍士所逐|{
	嶺南分二節鎮西道治邕州}
奔藤州|{
	藤州漢猛陵縣地唐置藤州至京師五千六百里}
詐為敕書及攻討使印募鄉丁及旁側土軍以攻邕州衆既烏合動輒潰敗往依桂州桂州人怨其分裂不納|{
	以其割桂管巡屬隸西道節度也}
京無所自容敕貶崖州司戶不肯之官還至零陵敕賜自盡以桂管觀察使鄭愚為嶺南西道節度使 冬十月丙申朔立皇子佾為魏王侹為凉王佶為蜀王|{
	侹他鼎翻佶其吉翻}
十一月立順宗子緝為蘄王憲宗子憤為榮王 南詔帥羣蠻五萬寇安南|{
	帥讀曰率 考異曰補國史云四年春南蠻帥衆五萬攻安南按蠻書咸通三年十二月二十一日桃花人安南城西南角下營茫蠻於蘇歷江岸屯聚裸形蠻亦當陳面二十七日蠻賊逼交州城則是今年冬末蠻已圍交州也今從實録}
都護蔡襲告急敕發荆南湖南兩道兵二千桂管義征子弟三千詣邕州|{
	義征子弟因其應募從軍名之}
受鄭愚節度 嶺南東道節度使韋宙奏蠻寇必向邕州若不先保護遽欲遠征恐蠻於後乘虚扼絶餉道乃敕蔡襲屯海門 |{
	考異曰實録詔襲且住海門是令棄交趾退屯海門也按襲死時猶在交趾蓋詔書到時襲已被圍不得通也}
鄭愚分兵備禦十二月襲又求益兵敕山南東道發弩手千人赴之時南詔已圍交趾襲嬰城固守救兵不得至 翼王繟薨|{
	繟順宗子音齒善翻}
是歲嗢末始入貢嗢末者吐蕃之奴號也|{
	嗢烏沒翻}
吐蕃每發兵其富室多以奴從|{
	從才用翻}
往往一家至十數人由是吐蕃之衆多及論恐熱作亂奴多無主遂相糾合為部落散在甘肅瓜沙河渭岷廓疊宕之間|{
	宕徒浪翻}
吐蕃微弱者反依附之

四年春正月庚午上祀圓丘赦天下 是日南詔陷交趾蔡襲左右皆盡徒步力戰身集十矢欲趣監軍船|{
	趣七喻翻}
船已離岸遂溺海死|{
	離力智翻蔡襲死矣而十必死之狀曾無朝臣一人為之申理自是之後唐之紀綱大壞凡藩鎮有片言隻字則朝廷聳動惟恐拂其意朝臣反與之關通依以為外主矣}
幕僚樊綽携其印浮度江|{
	自白州博白縣西南百里下北戍灘出馬門江度海抵安南界樊綽携印度處即此江}
荆南江西鄂岳襄州將士四百餘人走至城東水際荆南虞候元惟德等謂衆曰吾輩無船入水則死不若還向城與蠻鬭人以一身易二蠻亦為有利遂還向城入東羅門|{
	東羅門安南羅城東門也}
蠻不為備惟德等縱兵殺蠻二千餘人 |{
	考異曰實録二月安南經略使蔡襲奏蠻賊楊思僭羅伏州扶耶縣令麻光高部領其衆於城西角下營嶺南東道節度使韋宙奏蠻賊去十二月二十七日逼安南城池經略使檢校工部尚書蔡襲出兵格鬬殺傷相當正月三日賊衆圍城進攻甚急襲城上以車弩射之至七日城陷襲左膊中弩箭死家口并元從七千餘人悉隕於賊從事樊綽携印度江其荆南江西鄂岳襄州兵突到城東水際無船却囘相率入東羅門殺蠻僅一二千人至夜賊救兵至遂屠其城按此二奏似後人采集蠻書為之其中又多差舛如楊思僭蠻書中兩處有之皆作楊思縉蓋草書誤為僭耳彼雖蠻夷豈肯名思僭也張錦里耆舊傳載高駢與雲南牒亦云楊思縉善蘭節度新書亦承此誤為僭又蠻書所云思縉光高部領者桃花蠻五六千人耳非謂盡將羣蠻也補國史云蠻衆五萬攻安南非止五六千人也又十二月二十一日裸形蠻茫蠻桃花人已在城下豈至二十七日始逼安南也蠻書言二十七日逼城者但記見河蠻尋傳蠻之日耳又言正月二日三日者但記以車弩射得苴子之日耳非其日始圍城也且城陷奔迸之際非樊綽身在其間豈知其詳然四道兵所殺人數猶因僧無碍說始知之韋宙身在廣州何得所奏一如樊綽之書其偽明矣新傳曰是夜蠻遂屠城亦因實録而誤}
逮夜蠻將楊思縉始自子城出救之|{
	子城城内小城也}
惟德等皆死南詔兩陷交趾所殺虜且十五萬人留兵二萬使思縉據交趾城谿洞夷獠無遠近皆降之|{
	獠魯皓翻降戶江翻}
詔諸道兵赴安南者悉召還分保嶺南西道 上遊宴無節左拾遺劉蜕上疏曰|{
	蜕輸芮翻}
今西凉築城應接未决於與奪|{
	西凉即凉州蓋此時謀進築也}
南蠻侵軼|{
	軼徒結翻突也}
干戈悉在於道塗旬月以來不為無事陛下不形憂閔以示遠近則何以責其死力望節娛遊以待遠人乂安未晩|{
	言待遠人乂安之後然後娛遊尚未為晩}
弗聽 二月甲午朔上歷拜十六陵|{
	十六陵謂獻陵昭陵乾陵定陵橋陵泰陵建陵元陵崇陵豐陵景陵光陵莊陵章陵端陵貞陵考異曰拜十六陵非一日可了而舊史無還宮之日唐年補録云二月庚子一日拜十六陵尤難信也}


置天雄軍於秦州|{
	代宗姑息田承嗣以天雄軍號寵魏博尋以其悖傲削之今復於秦州置天雄軍至於唐末魏博復天雄軍號秦州不復號天雄矣}
以成河渭三州隸焉以前左金吾將軍王晏實為天雄觀察使|{
	晏實宰之子宰父智興子之見二百四十七卷會昌四年}
三月歸義節度使張義潮奏自將蕃漢兵七千克復凉州|{
	將即亮翻下同}
南蠻寇左右江浸逼邕州鄭愚懼自言儒臣無將略請任武臣朝廷召義武節度使康承訓詣闕欲使之代愚仍詔選軍校數人士卒數百人自隨|{
	就義武軍中選之也校戶敎翻}
中書侍郎同平章事畢諴以同列多狥私不法稱疾辭位夏四月罷為兵部尚書庚戌羣盜入徐州殺官吏刺史曹慶討平之 康承訓至京師以為嶺南西道節度使發荆襄洪鄂四道兵萬人與之俱 五月戊辰以翰林學士承旨兵部侍郎楊收同平章事收發之弟也|{
	宣宗以開河湟追加順憲二宗尊號有司議改造廟主署新諡發以為作主求古無其文執不可知禮者韙之由是知名}
與左軍中尉楊玄价敘同宗相結故得為相|{
	价音介為楊收與玄价交惡張本}
乙亥廢容管隸嶺南西道復以龔象二州隸桂管|{
	去年以龔象隸嶺南西道}
戊子以門下侍郎同平章事杜審權同平章事充鎮海節度使 六月廢安南都護府置行交州於海門鎮以右監門將軍宋戎為行交州刺史以康承訓兼領安南及諸軍行營 閏月以門下侍郎同平章事杜悰同平章事充鳳翔節度使以兵部侍郎判度支河南曹確同平章事 秋七月辛卯朔日有食之 復置安南都護府於行交州 |{
	考異曰實録以郡州為交州補國史亦同又云夏侯貞孝公請用高駢為郡州進討使按地理志無郡州補國史又云海門今晏州地理志晏州乃屬瀘州都督府嶺南亦無之}
以宋戎為經略使發山東兵萬人鎭之時諸道兵援安南者屯聚嶺南|{
	句斷}
江西湖南|{
	此四字衍}
江西湖南餽運者皆泝湘江入澪渠灕水|{
	酈道元曰湘灕同源分為二水南則灕水北則湘川湘灕之間陸地廣百餘步謂之始安嶠漢伐南越出零陵下灕水即此路也湘水出零陵始安縣陽朔山自零陵西南謂之澪渠新書曰桂州有灕水出海陽山世言秦命史禄伐越鑿為漕馬援討徵側復治以通餽後為江水潰毁渠遂廞淺唐李渤復浚之范成大桂海虞衡志曰湘灕二水皆出靈川之海陽行百里分南北下北下曰湘稠灘急瀧又二千里至長沙水始緩南下曰灕名灘三百六十又千二百里至番禺以入海又曰靈渠在桂之興安縣秦始皇戍嶺時史禄鑿此以運之遺迹湘水源於雲泉之陽海山在此下瀜江牂柯下流本南下廣西興安水行其間地勢最高二水遠不相謀禄始作此渠派湘之流而注之瀜使北水南合北舟踰嶺其作渠之法於湘流沙磕中壘石作鏵觜鋭其前逆分湘流為兩激之六十里行渠中以入瀜江與俱南渠繞興安界深不數尺廣丈餘六十里間置斗門三十六土人但謂之斗舟入一斗則復閘斗伺水積漸進故能循崖而上建瓴而下千斛之舟亦可往來治水巧妙無如靈渠者澪音零灕音离}
勞費艱澁諸軍乏食潤州人陳磻石上言|{
	磻薄官翻}
請造千斛大舟自福建運米泛海不一月至廣州從之軍食以足然有司以和雇為名奪商人舟委其貨於岸側舟入海或遇風濤沒溺有司囚繫綱吏舟人使償其米人頗苦之 八月嶺南東道節度使韋宙奏蠻必向邕州請分兵屯容藤州|{
	容藤二州相去二百七十里}
夔王滋薨|{
	滋上弟也}
敕以閤門使吳德應等為館驛使臺諫上言故事御史巡驛|{
	唐中世置閤門使以宦者為之掌供奉朝會贊引親王宰相百官蕃客朝見辭唐初中書通事舍人之職也玄宗開元中以監察御史兼廵傳驛至二十五年以監察御史檢校兩京館驛大歷十四年兩京以御史一人知館驛號館驛使宋白曰元和初征劉闘郵傳多事憲宗命中人為館驛使監察御史薛存誠及諫官相繼論奏罷之}
不應忽以内人代之上諭以敕命已行不可復改左拾遺劉蜕上言昔楚子縣陳得申叔一言而復封之|{
	左傳楚子為陳夏氏亂故伐陳遂入陳殺夏徵舒因縣陳申叔時不賀楚子問其故對曰夏徵舒弑其君其罪大矣討而戮之君之義也今縣陳貪其富也無乃不可乎王曰善哉乃復封陳}
太宗發卒修乾元殿聞張玄素諫即日罷之|{
	見一百九十三卷貞觀四年}
自古明君所尚者從諫如流豈有已行而不改且敕自陛下出之自陛下改之何為不可弗聽 戞斯遣其臣合伊難支表求經籍及每年遣使走馬請歷又欲討囘鶻使安西以來悉歸唐不許 冬十月甲戌以長安尉集賢校理令狐滈為左拾遺乙亥左拾遺劉蜕上言滈專家無子弟之法布衣行公相之權|{
	相息亮翻}
起居郎張雲言滈父綯用李涿為安南|{
	見上卷宣宗大中十三年}
致南蠻至今為梗由滈納賄陷父於惡十一月丁酉雲復上言滈父綯執政之時|{
	復扶又翻}
人號白衣宰相滈亦上表引避乃改詹事府司直|{
	唐太子詹事府有司直二人正七品上掌糾劾宮寮及率府之兵}
辛巳廢宿泗觀察使復以徐州為觀察府以濠泗隸焉|{
	去年八月廢徐州軍額}
十二月南詔寇西川 昭義節度使沈詢奴歸秦與詢侍婢通詢欲殺之未果乙酉歸秦結牙將作亂攻府第殺詢

五年春正月以京兆尹李蠙為昭義節度使|{
	蠙部田翻}
取歸秦心肝以祭沈詢 淮南節度使令狐綯為其子滈訟寃|{
	為于偽翻}
貶張雲興元少尹劉蜕華隂令|{
	華戶化翻}
敕曰雖嘉蹇諤之忠難逃疎易之責|{
	易以䜴翻}
丙午西川奏南詔寇嶲州刺史喻士珍破之獲千餘人|{
	觀明年喻士珍以貪獪而失守則此捷虚張功狀也}
詔發右神策兵五千及諸道兵戍之忠武大將顔慶復請築新安遏戎二城從之|{
	二城蓋築於嶲州界}
以容管經略使張茵兼句當交州事|{
	句古侯翻當丁浪翻時交州寄治海門欲使張茵進取}
益海門鎮兵滿二萬五千人令茵進取安南 二月己巳以刑部尚書鹽鐵轉運使李福同平章事充西川節度使 甲申前西川節度使蕭鄴左遷山南西道觀察使 三月丁酉彗星出於婁長三尺|{
	彗祥歲翻又徐醉翻又音歲長直亮翻}
己亥司天監奏按星經是名舍譽瑞星也上大喜|{
	唐司天監正三品掌察天文稽歷數史言唐末司天官昏迷天象以妖為祥}
請宣示中外編諸史策從之 康承訓至邕州蠻寇益熾詔發許滑青汴兖鄆宣潤八道兵以授之承訓不設斥候南詔帥羣蠻近六萬寇邕州|{
	帥讀曰率近其靳翻}
將入境承訓乃遣六道兵凡萬人拒之以獠為導紿之|{
	熾昌志翻獠音老紿徒亥翻}
敵至不設備五道兵八千人皆沒惟天平軍後一日至得免|{
	天平軍鄆兵也}
承訓聞之惶怖不知所為|{
	怖普布翻}
節度副使李行素帥衆治壕栅甫畢蠻軍已合圍留四日治攻具將就諸將請夜分道斫蠻營承訓不許有天平小校再三力爭乃許之小校將勇士三百夜縋而出|{
	將即亮翻校戶敎翻縋馳偽翻}
散燒蠻營斬首五百餘級蠻大驚間一日解圍去承訓乃遣諸軍數千追之所殺虜不滿三百級皆溪獠脅從者承訓騰奏告捷云大破蠻賊中外皆賀 夏四月以兵部侍郎判戶部蕭寘同平章事寘復之孫也|{
	蕭復相德宗}
加康承訓檢校右僕射賞破蠻之功也自餘奏功受賞者皆承訓子弟親昵|{
	昵尼質翻}
燒營將校不遷一級由是軍中怨怒聲流道路 五月敕徐州土風雄勁甲士精彊比因罷節|{
	比毗至翻四年罷徐州武寧節度}
頗多逃匿宜令徐泗團練使選募軍士三千人赴邕州防戍待嶺外事寧即與代歸秋七月西川奏兩林鬼主邀南詔蠻敗之|{
	史炤曰兩林部落東蠻國也去勿鄧國七十里地雖狹而諸部推為長號大鬼主敗補邁翻}
殺獲甚衆保塞城使杜守連不從南詔率衆詣黎州降|{
	帥讀曰率降戶江翻}
嶺南東道節度使韋宙具知康承訓所為以書白宰相承訓亦自疑懼累表辭疾乃以承訓為右武衛大將軍分司|{
	考異曰補國史嶺南東道節度使韋宙兼領供軍使將吏在邕州者潜令申報事無巨細莫不知之復究尋克捷事多虛妄具所聞啓於丞相承訓已自懷疑懼辭疾免責授右武衛大將軍分司東都僖宗實録承訓傳曰南蠻陷交趾以承訓為嶺南西道節度使踰歲討平之加檢校右僕射與鄰帥不叶以右武衛大將軍罷歸蓋其家行狀云爾今從補國史懿宗實録新傳}
以容管經略使張茵為嶺南西道節度使復以容管四州别為經略使|{
	新書方鎮表咸通元年罷容管以所管州隸邕管}
時南詔知邕州空竭不復入寇茵久之不敢進軍取安南夏侯孜薦驍衛將軍高駢代之 |{
	考異曰補國史茵驍將無遠略經年不敢進軍丞相夏侯貞孝公獨獻密疏請用驍衛將軍高駢有制以本官充郡州進討使旋拜安南節度使其茵所領兵並付高公指揮按今年正月詔茵進軍收復安南若經年則孜已罷相今從實録附於此實録駢官為右領軍上將軍太高今從補國史舊紀五年四月南蠻寇邕管以秦州經略使高駢率禁軍五千會諸侯之師禦之今不取}
乃以駢為安南都護本管經略招討使茵所將兵悉以授之駢崇文之孫也|{
	憲宗朝高崇文有定蜀之功}
世在禁軍駢頗讀書好談今古|{
	好呼到翻}
兩軍宦官多譽之|{
	兩軍謂左右神策兩軍也譽音余}
累遷右神策都虞候党項叛將禁兵萬人戍長武屢有功遷秦州防禦使復有功故委以安南|{
	復扶又翻}
冬十一月以門下侍郎同平章事夏侯孜同平章事充河東節度使 壬寅以翰林學士承旨兵部侍郎路巖同平章事時年三十六|{
	為路巖以高位疾僨張本}


六年春正月丁巳始以懿安皇后配饗憲宗廟時王皥復為禮院檢討官更申前議朝廷竟從之|{
	王皥議見二百四十八卷宣宗大中二年}
諸道進私白者|{
	唐時諸道歲進閩兒號曰私白}
閩中為多故宦官多閩人福建觀察使杜宣猷每寒食遣吏分祭其先壟宦官德之庚申以宣猷為宣歙觀察使時人謂之敕使墓戶 三月中書侍郎同平章事蕭寘薨 夏四月以前東川節度使高璩為兵部侍郎同平章事璩元裕之子也|{
	璩其於翻元裕見二百四十五卷文宗太和八年}
楊收建議以蠻寇積年未平兩河兵戍嶺南冒瘴霧物故者什六七請於江西積粟募彊弩三萬人以應接嶺南道近便仍建節以重其權從之五月辛丑置鎮南軍於洪州 嶲州刺史喻士珍貪獪|{
	獪古外翻}
掠兩林蠻以易金南詔復寇嶲州|{
	復扶又翻}
兩林蠻開門納之南詔盡殺戍卒士珍降之|{
	降戶江翻}
壬寅以桂管觀察使嚴譔為鎮南節度使譔震之從孫也|{
	譔雛免翻嚴震鎮興元德宗播遷震有迎奉之功從才用翻}
六月高璩薨以御史大夫徐商為兵部侍郎同平章事 秋七月

立皇子侃為郢王儼為普王 高駢治兵於海門未進監軍李維周惡駢欲去之屢趣駢使進軍|{
	治直之翻惡烏路翻去羌呂翻趣讀曰促}
駢以五千人先濟約維周發兵應援駢既行維周擁餘衆不發一卒以繼之九月駢至南定|{
	高祖武德四年分交趾所管宋平縣置南定縣時屬安南府安南府即交趾宋白曰南定縣漢日南郡西捲縣地}
峯州蠻衆近五萬方穫田|{
	近其靳翻劉昫曰峯州隋交趾郡之嘉寧縣唐武德四年置峯州嘉寧漢麊令縣地}
駢掩擊大破之 |{
	考異曰舊紀實録皆云五月駢奏於邕管大敗林邑蠻按林邑在海南自至德後號環王與中國久絶劉昫但見南蠻則謂之林邑誤也新南詔傳亦云駢以選士五千度江敗林邑兵於邕州亦承此而誤也舊紀又云是歲秋高駢自海門進軍破蠻軍收復安南府蓋因駢今秋發海門遂云復安南耳復安南實在明年也補國史云五年九月高公力戰破峯州蠻於南定縣按張茵以五年正月句當交州受詔收復安南補國史云經年不進軍乃以駢代之則駢豈得以其年九月已破峯州蠻乎補國史又云駢破峯州蠻後近四月餘日表報不至朝廷以王晏權代之六月高公進軍收復安南亦不云幾年六月蓋駢以六年六月破峯州蠻七年六月破安南耳實録又云九月駢奏破蠻龍州營寨并燒食糧等事詔駢令於當界守備緣近有赦文已許恩宥伺其悛改亦未要更深加討逐按赦在明年十一月此詔必在駢已平安南後實録誤也新傳又云駢擊南詔龍州屯蠻酋燒貲畜走龍州即安南所管龍編縣也}
收其所穫以食軍|{
	獲戶郭翻刈稻也食祥吏翻}
冬十二月壬子太皇太后鄭氏崩 |{
	考異曰舊傳大中末崩誤也今從實録}


七年春二月歸義節度使張義潮奏北庭囘鶻固俊克西州北庭輪臺清鎭等城|{
	北庭本貞觀所置之庭州長安二年置北庭都護府西七百里有清海鎮又西延城西行三百二十里至輪臺縣囘鶻固俊新書及考異正文皆作僕固俊 考異曰實録義潮奏俊收西河及部落胡漢皆歸伏并表賀收西州等城事新吐蕃傳曰七年俊擊取西州收諸部按大中五年義潮以十一州圖籍來上西州已在其中今始云收西州者蓋當時雖得其圖籍其地猶為吐蕃所據耳}
論恐熱寓居廓州糾合旁側諸部欲為邊患皆不從所向盡為仇敵無所容仇人以告拓拔懷光於鄯州懷光引兵擊破之|{
	宋白曰鄯州南至廓州一百八十里考異曰實録義潮又奏鄯州城使張季顒押領拓拔懷光下使到尚恐熱將并隨身器甲等並以進奉新吐蕃傳曰鄯州城使張季顒與尚恐熱戰破之收器鎧以獻今從補國史實録}
三月戊寅以河東節度使劉潼為西川節度使初南詔圍嶲州東蠻浪稽部竭力助之遂屠其城|{
	謂去年陷嶲州也}
卑籠部怨南詔殺其父兄導忠武戍兵襲浪稽滅之南詔由是怨唐南詔遣清平官董成等詣成都節度使李福盛儀衛以見之故事南詔使見節度使拜伏於庭成等曰驃信已應天順人|{
	南詔自尋夢湊以來自稱驃信夷語君也因僭號自謂應天順人}
我見節度使當抗禮傳言往返自旦至日中不决將士皆憤怒福乃命捽而毆之|{
	捽昨沒翻毆烏口翻}
因械繫於獄劉潼至鎮釋之奏遣還國詔召成等至京師見於别殿厚賜勞而遣之|{
	見賢遍翻勞力到翻}
成德節度使王紹懿在鎭十年|{
	大中十一年紹懿襲鎮}
為政寛簡軍民便之疾病召兄紹鼎之子都知兵馬使景崇而告之曰吾兄以汝之幼以軍政授我汝今長矣|{
	長知兩翻}
我復以軍政歸汝努力為之上忠朝廷下和鄰藩勿墜吾兄之業汝之功也言竟而薨|{
	史言王紹懿垂沒精神不亂}
閏月吐蕃寇邠寧節度使薛弘宗拒却之 夏四月貶前西川節度使李福為蘄王傅|{
	以毆繫南詔使者也蘄王緝順宗子}
五月葬孝明皇后於景陵之側主祔别廟|{
	孝明皇后宣宗母鄭太后也懿安郭后憲宗之元妃也配食于太廟鄭后側室也祔别廟禮也}
六月魏博節度使何弘敬薨軍中立其子左司馬全皥為留後 以王景崇為成德留後 南詔酋龍遣善闡節度使楊緝助安南節度使段酋遷守交趾|{
	善闡府南詔别都也在交趾西北}
以范昵些為安南都統|{
	昵尼質翻些蘇个翻又音細}
趙諾眉為扶邪都統|{
	按實録扶邪縣屬羅伏州蓋南詔所置也}
監陳敕使韋仲宰將七千人至峯州|{
	監古銜翻陳讀曰陣}
高駢得以益其軍進擊南詔屢破之捷奏至海門李維周皆匿之數月無聲問上怪之以問維周維周奏駢駐軍峯州玩寇不進上怒以右武衛將軍王晏權代駢鎮安南 |{
	考異曰補國史謂駢及晏權皆云安南節度使按時安南止有都護經略招討使耳無節度使也舊王智興傳九子無晏權名實録亦云命晏權代駢為節度而無月日蓋闕漏也}
召駢詣闕欲重貶之晏權智興之從子也|{
	王智興歷德順憲穆四朝後為武寧帥尤貪横}
是月駢大破南詔蠻於交趾殺獲甚衆遂圍交趾城 秋七月以何全皥為魏博留後 冬十月甲申以門下侍郎同平章事楊收為宣歙觀察使收性侈靡門吏僮奴多倚為姦利楊玄价兄弟受方鎮之賂屢有請託收不能盡從玄价怒以為叛已故出之 拓跋懷光以五百騎入廓州生擒論恐熱先刖其足數而斬之|{
	數所具翻}
傳首京師其部衆東奔秦州尚延心邀擊破之悉奏遷於嶺南吐蕃自是衰絶乞離胡君臣不知所終|{
	乞離胡事始見二百四十六卷武宗會昌二年}
高駢圍交趾十餘日蠻困蹙甚城且下會得王晏權牒已與李維周將大軍發海門駢即以軍事授韋仲宰與麾下百餘人北歸先是仲宰遣小使王惠贊駢遣小校曾衮入告交趾之捷|{
	先悉薦翻}
至海中望見旌旗東來問遊船|{
	遊船遊奕之船}
云新經略使與監軍也二人謀曰維周必奪表留我乃匿於島間維周過即馳詣京師上得奏大喜即加駢檢校工部尚書復鎮安南駢至海門而還王晏權闇懦動稟李維周之命維周凶貪諸將不為之用遂解重圍|{
	重直龍翻}
蠻遁去者大半駢至復督勵將士攻城|{
	復扶又翻}
遂克之殺段酋遷及上蠻為南詔鄉導者朱道古|{
	鄉讀曰嚮蠻居安南界内者為上蠻}
斬首三萬餘級 |{
	考異曰舊紀十月蠻寇悉平實録九月駢奏殺戮都蠻統皈首遷朱道古及斬首三千餘級十月丙申日下又云駢奏收復安南蠻寇遁散又云敗楊緝思段酋遷朱道古殺戮三萬餘級新紀十月高駢克安南按皈首遷即段酋遷字之誤也補國史收城與敗緝思等共是一事實録分在兩月不知其何所據也新南詔傳曰七年六月駢次交州戰數勝士酣鬭斬其將張詮李溠龍舉衆萬人降抜波風三壁緝思出戰敗還走城士乘之超堞入斬酋遷昵些諾眉上首三萬級安南平蓋因駢以六月至安南終言之耳安南實不以六月平也今從新舊紀}
南詔遁去駢又破土蠻附南詔者二洞誅其酋長土蠻帥衆歸附者萬七千人|{
	酋慈由翻長知兩翻帥讀曰率}
十一月壬子赦天下詔安南邕州西川諸軍各保疆域勿復進攻南詔委劉潼曉諭如能更修舊好|{
	復扶又翻好呼到翻下同}
一切不問 置静海軍於安南以高駢為節度使|{
	自此迄宋朝安南遂為靜海軍節鎮}
自李涿侵擾羣蠻|{
	事見上卷宣宗大中十二年}
為安南患殆將十年至是始平駢築安南城周三千步造屋四十餘萬間 十二月戞斯遣將軍乙支連幾入貢奏遣鞍馬迎册立使及請亥年歷日|{
	是年丙戌亥明年也}
以成德留後王景崇為節度使 上好音樂宴遊殿前供奉樂工常近五百人|{
	近其靳翻}
每月宴設不減十餘|{
	宴設謂宮中置宴也宋朝内臣謂之排當}
水陸皆備|{
	言殽膳備水陸之品}
聼樂觀優不知厭倦賜與動及千緡曲江昆明灞滻南宮北苑|{
	南宮即興慶宮禁苑在皇城之北}
昭應咸陽|{
	昭應有華清宫咸陽有望賢樓}
所欲遊幸即行不待供置有司常具音樂飲食幄帟|{
	帟羊益翻小幕曰帟}
諸王立馬以備陪從|{
	從才用翻下同}
每行幸内外諸司扈從者十餘萬人所費不可勝紀|{
	勝音升}


八年春正月以魏博留後何全皥為節度使 二月歸義節度使張義潮入朝|{
	宣宗大中五年張義潮以沙州降尋授以歸義節至是入朝}
以為右神武統軍命其族子惟深守歸義 自安南至邕廣海路多潜石覆舟靜海節度使高駢募工鑿之漕運無滯 西川近邊六姓蠻|{
	六姓蠻一曰蒙蠻二曰夷蠻三曰訛蠻四曰狼蠻五曰勿鄧蠻六曰白蠻近其靳翻}
常持兩端無寇則稱效順有寇必為前鋒卑籠部獨盡心於唐與羣蠻為讎朝廷賜姓李除為刺史節度使劉潼遣將將兵助之|{
	將即亮翻}
討六姓蠻焚其部落斬首五千餘級 樂工李可及善為新聲三月上以可及為左威衛將軍曹確諫曰太宗定文武官六百餘員謂房玄齡曰朕以待天下賢士工商雜流不可處也|{
	處昌呂翻}
太和中文宗欲以樂工尉遲璋為王府率|{
	尉紆勿翻東宮有十率諸王有府率}
拾遺竇洵直諫即改光州長史乞以兩朝故事别除可及官不從 夏四月上不豫羣臣希進見|{
	見賢遍翻}
五月丙辰疎理天下繫囚非巨蠧不可赦者皆逓降一等 秋七月壬寅蘄王緝薨|{
	緝順宗子}
懷州民訴旱刺史劉仁規揭牓禁之|{
	揭其列翻}
民怒相與作亂逐仁規仁規逃匿村舍民入州宅掠其家貲登樓擊鼓久之乃定甲子以兵部侍郎充諸道鹽鐵轉運等使駙馬都尉

于琮同平章事 宣歙觀察使楊收過華嶽廟|{
	華嶽廟在華州華隂縣華戶化翻}
施衣物|{
	施式豉翻}
使巫祈禱縣令誣以為收罪右拾遺韋保衡復言收前為相除嚴譔江西節度使受錢百萬又置造船務人訟其侵隱|{
	復扶又翻}
八月庚寅貶收端州司馬 |{
	考異曰舊傳曰韋保衡作相又發收隂事言前用嚴譔為江西節度納賂百萬明年貶為端州司馬按是時保衡未作相舊傳誤今從實録}
九月上疾瘳 冬十二月信王薨|{
	憲宗子音彌兖翻}
加嶺南東道節度使韋宙同平章事

資治通鑑卷二百五十
