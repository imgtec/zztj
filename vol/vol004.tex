<!DOCTYPE html PUBLIC "-//W3C//DTD XHTML 1.0 Transitional//EN" "http://www.w3.org/TR/xhtml1/DTD/xhtml1-transitional.dtd">
<html xmlns="http://www.w3.org/1999/xhtml">
<head>
<meta http-equiv="Content-Type" content="text/html; charset=utf-8" />
<meta http-equiv="X-UA-Compatible" content="IE=Edge,chrome=1">
<title>資治通鑒_5-資治通鑑卷四_5-資治通鑑卷四</title>
<meta name="Keywords" content="資治通鑒_5-資治通鑑卷四_5-資治通鑑卷四">
<meta name="Description" content="資治通鑒_5-資治通鑑卷四_5-資治通鑑卷四">
<meta http-equiv="Cache-Control" content="no-transform" />
<meta http-equiv="Cache-Control" content="no-siteapp" />
<link href="/img/style.css" rel="stylesheet" type="text/css" />
<script src="/img/m.js?2020"></script> 
</head>
<body>
 <div class="ClassNavi">
<a  href="/24shi/">二十四史</a> | <a href="/SiKuQuanShu/">四库全书</a> | <a href="http://www.guoxuedashi.com/gjtsjc/"><font  color="#FF0000">古今图书集成</font></a> | <a href="/renwu/">历史人物</a> | <a href="/ShuoWenJieZi/"><font  color="#FF0000">说文解字</a></font> | <a href="/chengyu/">成语词典</a> | <a  target="_blank"  href="http://www.guoxuedashi.com/jgwhj/"><font  color="#FF0000">甲骨文合集</font></a> | <a href="/yzjwjc/"><font  color="#FF0000">殷周金文集成</font></a> | <a href="/xiangxingzi/"><font color="#0000FF">象形字典</font></a> | <a href="/13jing/"><font  color="#FF0000">十三经索引</font></a> | <a href="/zixing/"><font  color="#FF0000">字体转换器</font></a> | <a href="/zidian/xz/"><font color="#0000FF">篆书识别</font></a> | <a href="/jinfanyi/">近义反义词</a> | <a href="/duilian/">对联大全</a> | <a href="/jiapu/"><font  color="#0000FF">家谱族谱查询</font></a> | <a href="http://www.guoxuemi.com/hafo/" target="_blank" ><font color="#FF0000">哈佛古籍</font></a> 
</div>

 <!-- 头部导航开始 -->
<div class="w1180 head clearfix">
  <div class="head_logo l"><a title="国学大师官网" href="http://www.guoxuedashi.com" target="_blank"></a></div>
  <div class="head_sr l">
  <div id="head1">
  
  <a href="http://www.guoxuedashi.com/zidian/bujian/" target="_blank" ><img src="http://www.guoxuedashi.com/img/top1.gif" width="88" height="60" border="0" title="部件查字,支持20万汉字"></a>


<a href="http://www.guoxuedashi.com/help/yingpan.php" target="_blank"><img src="http://www.guoxuedashi.com/img/top230.gif" width="600" height="62" border="0" ></a>


  </div>
  <div id="head3"><a href="javascript:" onClick="javascript:window.external.AddFavorite(window.location.href,document.title);">添加收藏</a>
  <br><a href="/help/setie.php">搜索引擎</a>
  <br><a href="/help/zanzhu.php">赞助本站</a></div>
  <div id="head2">
 <a href="http://www.guoxuemi.com/" target="_blank"><img src="http://www.guoxuedashi.com/img/guoxuemi.gif" width="95" height="62" border="0" style="margin-left:2px;" title="国学迷"></a>
  

  </div>
</div>
  <div class="clear"></div>
  <div class="head_nav">
  <p><a href="/">首页</a> | <a href="/ShuKu/">国学书库</a> | <a href="/guji/">影印古籍</a> | <a href="/shici/">诗词宝典</a> | <a   href="/SiKuQuanShu/gxjx.php">精选</a> <b>|</b> <a href="/zidian/">汉语字典</a> | <a href="/hydcd/">汉语词典</a> | <a href="http://www.guoxuedashi.com/zidian/bujian/"><font  color="#CC0066">部件查字</font></a> | <a href="http://www.sfds.cn/"><font  color="#CC0066">书法大师</font></a> | <a href="/jgwhj/">甲骨文</a> <b>|</b> <a href="/b/4/"><font  color="#CC0066">解密</font></a> | <a href="/renwu/">历史人物</a> | <a href="/diangu/">历史典故</a> | <a href="/xingshi/">姓氏</a> | <a href="/minzu/">民族</a> <b>|</b> <a href="/mz/"><font  color="#CC0066">世界名著</font></a> | <a href="/download/">软件下载</a>
</p>
<p><a href="/b/"><font  color="#CC0066">历史</font></a> | <a href="http://skqs.guoxuedashi.com/" target="_blank">四库全书</a> |  <a href="http://www.guoxuedashi.com/search/" target="_blank"><font  color="#CC0066">全文检索</font></a> | <a href="http://www.guoxuedashi.com/shumu/">古籍书目</a> | <a   href="/24shi/">正史</a> <b>|</b> <a href="/chengyu/">成语词典</a> | <a href="/kangxi/" title="康熙字典">康熙字典</a> | <a href="/ShuoWenJieZi/">说文解字</a> | <a href="/zixing/yanbian/">字形演变</a> | <a href="/yzjwjc/">金 文</a> <b>|</b>  <a href="/shijian/nian-hao/">年号</a> | <a href="/diming/">历史地名</a> | <a href="/shijian/">历史事件</a> | <a href="/guanzhi/">官职</a> | <a href="/lishi/">知识</a> <b>|</b> <a href="/zhongyi/">中医中药</a> | <a href="http://www.guoxuedashi.com/forum/">留言反馈</a>
</p>
  </div>
</div>
<!-- 头部导航END --> 
<!-- 内容区开始 --> 
<div class="w1180 clearfix">
  <div class="info l">
   
<div class="clearfix" style="background:#f5faff;">
<script src='http://www.guoxuedashi.com/img/headersou.js'></script>

</div>
  <div class="info_tree"><a href="http://www.guoxuedashi.com">首页</a> > <a href="/SiKuQuanShu/fanti/">四库全书</a>
 > <h1>资治通鉴</h1> <!--         下载:【右键另存为】即可 --></div>
  <div class="info_content zj clearfix">
  
<div class="info_txt clearfix" id="show">
<center style="font-size:24px;">5-資治通鑑卷四</center>
    資治通鑑卷四     宋 司馬光 撰<br />
<br />
  胡三省 音註<br />
<br />
  周紀四【起閼逢困敦盡著雍困敦凡二十五年始甲子終戊子】<br />
<br />
  赧王中<br />
<br />
  十八年楚懷王亡歸秦人覺之遮楚道【遮其歸楚之路也】懷王從間道走趙【間隙也從空隙之路而行也間古莧翻走音奏又音如字】趙主父在代趙人不敢受懷王將走魏秦人追及之以歸<br />
<br />
  魯平公薨子緡公賈立【世本緡作湣】<br />
<br />
  十九年楚懷王發病薨於秦秦人歸其喪【喪息郎翻】楚人皆憐之如悲親戚諸侯由是不直秦 齊韓魏趙宋同擊秦至鹽氏而還【括地志鹽氏故城一名司鹽城在蒲州安邑縣掌鹽池之官因稱鹽氏徐廣曰鹽一作監】秦與韓武遂與魏封陵以和【十二年秦取魏封陵又取韓武遂今皆歸之以和】 趙主父行新地【趙新取中山之地也行下孟翻】遂出代西遇樓煩王於西河而致其兵【趙北有林胡樓煩之戌漢鴈門郡樓煩縣樓煩胡所居地西河即漢西河郡之地】 魏襄王薨子昭王立【世本曰昭王名遫】 韓襄王薨子釐王咎立【釐讀曰僖】<br />
<br />
  二十年秦尉錯伐魏襄城【尉蓋國尉也班志襄城縣屬潁川郡以分地考之潁川屬韓境蓋魏與韓分有潁川之地用兵争彊疆場之間朝韓暮魏則此時襄城或為魏土容亦有之場音亦】趙主父與齊燕共滅中山遷其王於膚施【燕因肩翻班志膚施縣屬上郡唐屬延州為州治所】歸行賞大赦置酒酺五日【酺音蒲說文曰王德布大飲酒也師古曰酺之為言布也王德布於天下而合聚飲食為酺師古所註漢法也此言趙國内酺耳赦者宥有罪也】 趙主父封其長子章於代號曰安陽君【長知丈翻班志代郡有東安陽縣括地志東安陽故城在朔州定襄縣界】安陽君素侈心不服其弟【不服其弟為王也】主父使田不禮相之【相息亮翻】李兑謂肥義曰公子章彊壯而志驕黨衆而欲大田不禮忍殺而驕二人相得必有陰謀夫小人有欲輕慮淺謀徒見其利不顧其害難必不久矣【夫音扶難乃旦翻下同】子任重而勢大亂之所始而禍之所集也子何不稱疾毋出而傳政於公子成毋為禍梯【梯猶階也以木為之以升高者也禍梯猶言禍階也梯天黎翻】不亦可乎肥義曰昔者主父以王屬義也【屬之欲翻】曰毋變而度毋易而慮【而猶汝也】堅守一心以殁而世義再拜受命而籍之【記王命於籍也】今畏不禮之難而忘吾籍變孰大焉諺曰死者復生生者不愧吾欲全吾言安得全吾身乎子則有賜而忠我矣雖然吾言已在前矣終不敢失李兑曰諾子勉之矣吾見子已今年耳【已止也言肥義命止於今年也】涕泣而出李兑數見公子成以備田不禮【數見者相與謀為之備也數所角翻】肥義謂信期曰【索隱曰即下文高信也史記正義曰信音申康曰如字】公子章與田不禮聲善而實惡内得主而外為暴【得主謂章為主父所憐也】矯令以擅一旦之命不難為也【矯令矯主父之令也令力正翻擅時戰翻】今吾憂之夜而忘寐飢而忘食盜出入不可以不備【言盜在主父左右出入不可不備也】自今以來有召王者必見吾面我將以身先之無故而後王可入也【先悉薦翻】信期曰善主父使惠文王朝羣臣而自從旁窺之見其長子傫然也【朝直遥翻長知丈翻傫倫追翻懶懈貌少子臨朝而長子朝之故其貌如此】反北面為臣詘於其弟【詘與屈同】心憐之於是乃欲分趙而王公子章於代【王于況翻又音如字】計未决而輟主父及王游沙丘【史記正義曰沙丘在邢州平鄉縣東北二十里余按沙丘臺紂所作也班志云沙丘在鉅鹿郡鉅鹿縣東北七十里】異宫【異宫而處也】公子章田不禮以其徒作亂詐以主父令召王肥義先入殺之高信即與王戰【高信以王與公子章之徒戰也】公子成與李兌自國至【趙都邯鄲自邯鄲至也邯鄲音寒丹也】乃起四邑之兵入距難【距猶拒也難乃旦翻】殺公子章及田不禮滅其黨公子成為相號安平君【相息亮翻班志涿郡有安平縣非趙地也以公子成能平難而安國故以為號】李兑為司寇【司寇周六卿之一也掌刑】是時惠文王少【少詩照翻】成兑專政公子章之敗也往走主父主父開之【謂開宫門内之也走音奏】成兑因圍主父公子章死成兑謀曰以章故圍主父即解兵吾屬夷矣【夷誅也滅也】乃遂圍之令宫中人後出者夷【令力正翻】宫中人悉出主父欲出不得又不得食探雀鷇而食之【爾雅曰生哺鷇生噣雛釋云辯鳥子之異名也鳥子生而須母哺食者為鷇謂燕雀之屬也生而能自啄食者為雛謂雞雉之屬也探吐南翻鷇居候翻】三月餘餓死沙丘宫主父定死乃發喪赴諸侯主父初以長子章為太子後得吳娃愛之【長知丈翻吳娃謂吳廣之女孟姚也見上卷五年吳楚之間謂美女曰娃娃音於佳翻】為不出者數歲【為于偽翻】生子何乃廢太子章而立之【何即惠文王也】吳娃死愛弛憐故太子欲兩王之猶豫未决故亂起 秦樓緩免相魏冉代之【相息亮翻】<br />
<br />
  二十一年秦敗魏師于解【敗補邁翻班志解縣屬河東郡宋白曰解縣地即夏鳴條之野有鹽池之利後漢乾祐元年騎帥李茂貞奏置解州師古曰解蒲買翻】<br />
<br />
  二十二年韓公孫喜魏人伐秦【魏書人其將微也將即亮翻】穰侯薦左更白起於秦王以代向壽將兵【白姓也春秋之時秦有白乙丙穰人羊翻更工衡翻向息亮翻】敗魏師韓師于伊闕斬首二十四萬級虜公孫喜拔五城秦王以白起為國尉【戰國之時有國尉有郡尉應劭曰自上安下曰尉武官悉以為稱敗補邁翻】 秦王遺楚王書曰楚倍秦秦且率諸侯伐楚願王之飭士卒【飭治也整也遺于季翻倍蒲妹翻】得一樂戰【樂音洛快意也言一戰以快其意】楚王患之乃復與秦和親【和親者結和以相親也復扶又翻又音如字】<br />
<br />
  二十三年楚襄王迎婦於秦<br />
<br />
  臣光曰甚哉秦之無道也殺其父而劫其子【謂楚懷王留於秦而以困死秦王復遺襄王書以兵威劫之】楚之不競也【杜預曰競強也或曰競争也言不能與秦争也】忍其父而婚其讐【謂楚襄王父死于秦是仇讐之國也忍恥而與之婚】烏呼楚之君誠得其道臣誠得其人秦雖彊烏得陵之哉善乎荀卿論之曰夫道善用之則百里之地可以獨立不善用之則楚六千里而為讐人役【夫音扶】故人主不務得道而廣有其勢是其所以危也<br />
<br />
  秦魏冉謝病免以客卿燭壽為丞相【燭姓也左傳鄭有大夫燭之武】二十四年秦伐韓拔宛【宛故申伯國班志宛縣屬南陽郡唐為鄧州南陽縣宛於元翻】秦燭壽免魏冉復為丞相【相息亮翻】封於穰與陶謂之穰<br />
<br />
  侯又封公子市于宛公子悝于鄧【悝苦回翻】<br />
<br />
  二十五年魏入河東地四百里【河東地蓋安邑大陽蒲阪解縣瀕河之地阪音反解下買翻】韓入武遂地二百里于秦【武遂地十八年秦以予韓予讀曰與】魏芒卯始以詐見重【芒謨郎翻姓也卯其名】<br />
<br />
  二十六年秦大良造白起客卿錯伐魏至軹取城大小六十一【大良造即大上造之良者大上造秦十六爵軹音只軹縣班志屬河内郡唐為孟州濟源縣】二十七年冬十月秦王稱西帝遣使立齊王為東帝欲約與共伐趙蘇代自燕來【使疏吏翻燕因肩翻】齊王曰秦使魏冉致帝子以為何如對曰願王受之而勿稱也秦稱之天下安之王乃稱之無後也【無後猶言未晩】秦稱之天下惡之【惡烏路翻】王因勿稱以收天下此大資也且伐趙孰與伐桀宋利【桀宋見下二十九年】今王不如釋帝以收天下之望發兵以伐桀宋宋舉則楚趙梁衛皆懼矣是我以名尊秦而令天下憎之【令盧經翻】所謂以卑為尊也【古人有言曰自卑者人尊之】齊王從之稱帝二日而復歸之【歸帝號而不稱也復扶又翻又音如字】十二月呂禮自齊入秦【姓譜曰太嶽為禹心呂之臣故封呂侯後因以為氏古字脊骨之膂本作呂】秦王亦去帝復稱王【去羌呂翻除也後以義推】 秦攻趙拔杜陽【徐廣曰杜一作梗按班志杜陽在太原郡榆次縣界杜陽縣屬扶風注云杜水南入渭詩曰自土師古注云緜詩自土沮漆齊詩作自杜言公劉自狄而來居杜與沮漆之地又云扶風栒邑有豳鄉公劉所都審爾則杜陽近栒邑接上郡北地之境趙地西至上郡膚施或者其時併有杜陽歟沮七余翻枸須倫翻豳彼貧翻】<br />
<br />
  二十八年秦攻趙拔新垣曲陽【史記正義曰年表及括地志曲陽故城在懷州濟源縣西四十里新垣近曲陽未詳端的之處余按班志王屋山在河東垣縣沇水所出東流為濟水經濟水出河東垣縣王屋山為沇水注云濟水重源出温西北平地水有二源東源出原城東北晉文公以信降原即此城也俗以濟水重源所發因復謂之濟源城如此則濟源去垣不遠矣蓋新垣即河東之垣縣也以縣有遷徙謂其新邑為新垣也垣干元翻濟子禮翻沇以轉翻降戶江翻重直龍翻復扶又翻】<br />
<br />
  二十九年秦司馬錯擊魏河内【漢河内郡即魏河内之地秦并屬河東郡孟子記梁惠王曰河内凶則移其民於河東移其粟於河内蓋魏之有國河東河内自為二郡也錯七各翻又千故翻】魏獻安邑以和秦出其人歸之魏 秦敗韓師於夏山【敗補邁翻夏戶雅翻】 宋有雀生於城之陬【劉向說苑作鸇字林曰鷂屬陸璣曰鸇似鷂青黄色燕頷句啄向風揺翅乃因風飛急疾擊鳩鴿燕雀食之陬子侯翻隅也句古侯翻】史占之【史太史之屬掌卜筮者】曰吉【凶人吉其凶】小而生巨必覇天下宋康王喜起兵滅滕伐薛【班志沛郡公丘縣古滕國水經注滕城在蕃縣西唐志滕縣屬徐州薛即孟嘗君所封地蕃音反又音如字】東敗齊取五城南敗楚取地三百里西敗魏軍與齊魏為敵國乃愈自信其覇欲覇之亟成故射天笞地【敗補邁翻亟己力翻射而亦翻後以義推笞擊也音丑之翻】斬社稷而焚滅之【記曰共工氏有子曰句龍能平水土故祀以為社烈山氏之子曰柱為稷自夏以上祀之周棄亦為稷自商以來祀之自漢以下夏禹配食官社后稷配食官稷周禮注社稷土穀之神共讀曰恭夏戶雅翻】以示威服鬼神為長夜之飲於室中室中人呼萬歲則堂上之人應之堂下之人又應之門外之人又應之以至於國中無敢不呼萬歲者天下之人謂之桀宋【言其昏暴如桀也】齊湣王起兵伐之民散城不守宋王奔魏死於温【温周司寇蘇忿生之邑班志温縣屬河内郡宋至此而滅湣讀曰閔】<br />
<br />
  三十年秦王會楚王於宛【宛於元翻】會趙王於中陽【班志中陽縣屬西河郡水經註文水逕太原兹氏縣故城之東瀦為文湖文湖水又東逕中陽縣故城東】秦蒙武擊齊拔九城【風俗通東蒙主以蒙山為氏】 齊湣王既滅宋而驕乃南侵楚西侵三晉欲并二周為天子狐喧正議斮之檀衢【狐姓也春秋之時晉有狐突狐毛狐偃父子左傳齊簡公與婦人飲酒于檀臺檀衢意其地為通檀臺之衢路也爾雅四達謂之衢咺况晚翻又况遠翻斮側略翻斬也】陳舉直言殺之東閭【左傳晉圍齊州綽門于東閭杜預注曰齊東門】燕昭王日夜撫循其人益以富實【燕因肩翻】乃與樂毅謀伐齊樂毅曰齊霸國之餘業也【毅魚器翻自齊桓公霸天下國以彊大田氏藉其餘業】地大人衆未易獨攻也【易弋䜴翻】王必欲伐之莫如約趙及楚魏於是使樂毅約趙别使使者連楚魏且令趙嚪秦以伐齊之利【使者上疏吏翻令力丁翻以利誘之曰嚪嚪音田濫翻誘羊久翻】諸侯害齊王之驕暴皆爭合謀與燕伐齊【燕因肩翻】<br />
<br />
  三十一年燕王悉起兵以樂毅為上將軍【上將軍猶春秋之元帥帥所類翻】秦尉斯離帥師與三晉之師會之【尉秦官也斯離其名或曰斯姓也離名也斯蜀之西南夷種遂以為姓帥讀曰率】趙王以相國印授樂毅樂毅并將秦魏韓趙之兵以伐齊齊湣王悉國中之衆以拒之戰於濟西【相息亮翻將即亮翻湣讀曰閔水經濟水東北過壽張縣西界北逕須昌穀城臨邑縣西又北逕北平隂城西又東北過盧縣北皆齊地也濟西地在濟水之西濟子禮翻下同】齊師大敗樂毅還秦韓之師分魏師以略宋地部趙師以收河間【秦韓與齊隔遠故先還其師宋地近於魏故使略之河間近於趙故以方略部趙取之此其部分非人所能及也宋地齊滅宋所取之地】身率燕師長驅逐北劇辛曰齊大而燕小【燕因肩翻劇竭戟翻】賴諸侯之助以破其軍宜及時攻取其邊城以自益此長久之利也【劇竭戟翻姓也】今過而不攻以深入為名無損於齊無益於燕而結深怨後必悔之樂毅曰齊王伐功矜能謀不逮下廢黜賢良信任謟諛政令戾虐百姓怨懟【懟直類翻】今軍皆破亡若因而乘之其民必叛禍亂内作則齊可圖也若不遂乘之待彼悔前之非改過恤下而撫其民則難慮也遂進軍深入齊人果大亂失度【難慮謂難為計慮也失度失其常度也】湣王出走樂毅入臨淄取寶物祭器輸之於燕燕王親至濟上勞軍行賞饗士【燕因肩翻濟子禮翻勞力到翻】封樂毅為昌國君【班志昌國縣屬齊郡封毅為昌國君以其能昌大燕國也】遂使留徇齊城之未下者齊王出亡之衛衛君辟宮舍之稱臣而共具【辟讀曰避共音供又居用翻】齊王不遜衛人侵之齊王去奔鄒魯有驕色鄒魯弗内遂走莒【莒春秋莒子之國齊滅之班志莒縣屬城陽國國都也宋白曰周武王封少昊之後嬴姓兹輿於莒始都計斤城在今高密縣東南四十里春秋時徙於莒隱公二年莒人入向註云今城陽莒縣莒自初封二十三君為楚簡王所滅漢為莒縣城陽王所都莒音居禦翻】楚使淖齒將兵救齊因為齊相淖齒欲與燕分齊地【索隱曰淖女教翻康曰竹角切姓也相息亮翻】乃執湣王而數之【數其罪也師古曰數所具翻宋祁曰所主翻】曰千乘博昌之間方數百里雨血沾衣【漢置千乘郡博昌縣屬焉後漢更千乘郡為樂安國十三州志曰昌水其勢平故曰博昌唐志千乘博昌二縣皆屬青州乘繩證翻雨主遇翻自上而下曰雨】王知之乎曰知之嬴博之間地坼及泉【班志嬴博二縣屬泰山郡】王知之乎曰知之有人當闕而哭者求之不得去則聞其聲王知之乎曰知之淖齒曰天雨血沾衣者天以告也地坼及泉者地以告也有人當闕而哭者人以告也天地人皆告矣而王不知誡焉【誡與戒同戒警敕也毛晃曰警敇之辭曰誡此言天地人皆以相警敕也】何得無誅遂弑王於鼓里【鼔里莒中地名近齊廟】<br />
<br />
  荀子論之曰國者天下之利勢也得道以持之則大安也大榮也積美之源也不得道以持之則大危也大累也【累力偽翻事相緣及也】有之不如無之及其綦也【齊人謂極為綦音其下綦之同】索為匹夫不可得也【索山客翻求也】齊湣宋獻是也【湣讀曰閔宋獻意即指宋康王】故用國者義立而王信立而霸權謀立而亡挈國以呼禮義而無以害之【挈即提挈之挈音詰結翻】行一不義殺一無罪而得天下仁者不為也擽然扶持心國且若是其固也【擽然落石貌言其持心持國擽然如石之固擽歷各翻】之所與為之者之人則舉義士也之所以為布陳於國家刑法者則舉義法也主之所極然【毛晃曰然如也是也】帥羣臣而首嚮之者則舉義志也【帥讀曰率首所救翻志者心之所主也】如是則下仰上以義矣【仰魚亮翻凡仰給仰成之仰皆同音】是基定也基定而國定國定而天下定故曰以國濟義一日而白湯武是也【基址也本也為土立址曰基為木立根本亦曰基白明也】是所謂義立而王也德雖未至也義雖未濟也然而天下之理略奏矣【楊倞曰略有節奏也】刑賞已諾信於天下矣【諾人應聲也信人不疑而心孚也】臣下曉然皆知其可要也【要一遥翻約也勤也求也】政令已陳雖覩利敗不欺其民【令力正翻】約結已定雖覩利敗不欺其與【與黨與也即下文所謂與國也】如是則兵勁城固敵國畏之國一綦明【楊倞曰此綦當作基今謂此綦字從上註所謂齊人之言其義亦通明顯也】與國信之雖在僻陋之國威動天下五伯是也是所謂信立而霸也【伯讀曰霸五霸夏昆吾商大彭豕韋周齊桓晉文或曰齊桓晉文宋襄秦穆楚莊為五霸】挈國以呼功利不務張其義齊其信唯利之求内則不憚詐其民而求小利焉外則不憚詐其與而求大利焉内不修正其所以有然常欲人之有如是則臣下百姓莫不以詐心待其上矣上詐其下下詐其上則是上下析也【析先的翻分也離也】如是則敵國輕之與國疑之權謀日行而國不免危削綦之而亡齊湣薛公是也【湣讀曰閔薛公謂孟嘗君孟嘗君卒齊與諸侯共滅薛卒子恤翻】故用彊齊非以修禮義也非以本政教也非以一天下也綿綿常以結引馳外為務【引讀曰靷音羊晉翻丁度曰靷駕牛具在胷曰靷蓋駕馬亦用靷也】故彊南足以破楚西足以詘秦北足以敗燕中足以舉宋【史記齊閔王十年伐燕取之二十三年與秦敗楚于重丘南割楚之淮北三十六年與韓魏攻秦至函谷三十八年伐宋滅之通鑑據孟子以取燕事屬之齊宣王敗補邁翻詘與屈同音渠勿翻燕因肩翻】及以燕趙起而攻之若振槁然【燕因肩翻槁枯木也振揺也振已枯之木則其葉摧落而木根撥矣槁苦皓翻又音古老翻】而身死國亡為天下大戮後世言惡則必稽焉【稽考也又計校也】是無他故焉唯其不由禮義而由權謀也三者明主之所謹擇也仁人之所務白也【白明白也】善擇者制人不善擇者人制之<br />
<br />
  樂毅聞晝邑人王蠋賢【劉熙曰畫齊西南近邑音獲索隱曰音胡卦翻括地志戟里城在臨淄城西北三十里春秋時棘邑又云澅邑蠋所居即此邑因澅水為名也京相璠曰今臨淄有澅水西北入沛即班志所謂如水如時聲相似然則澅水即時水也余按後漢耿弇攻張步進軍畫中在臨淄西安二邑之間蠋班固古今人表作歜音觸據蠋字則當音蜀或音之欲翻康珠玉切通鑑以畫邑為晝邑以孟子去齊宿于晝為據也若以孟子為據則晝讀如字】令軍中環晝邑三十里無入【環據漢書音義音宦】使人請蠋蠋謝不往燕人曰不來吾且屠晝邑蠋曰忠臣不事二君烈女不更二夫【燕因肩翻蠋音蜀更工衡翻】國破君亡吾不能存而又欲劫之以兵吾與其不義而生不若死遂經其頸於樹枝自奮絶脰而死【經絞也縊也頸居郢翻頭莖也自奮自奮起而還擲也脰大透翻頸也】燕師乘勝長驅齊城皆望風奔潰樂毅修整燕軍禁止侵掠求齊之逸民顯而禮之寛其賦歛【歛力驗翻後以義推】除其暴令修其舊政齊民喜悦乃遣左軍渡膠東東萊【膠東漢為王國水經膠水出琅邪黔陬縣膠山北過膠東下密又北過東萊當利縣入海膠水之東為膠東國膠水之西為膠西國東萊春秋萊子之國漢置東萊郡邪音耶】前軍循泰山以東至海略琅邪【琅邪秦置為郡其地東至海南距淮】右軍循河濟屯阿鄄以連魏師【河濟注己見一卷安王十五年然僅及濟水入河而溢為滎一節今據水經濟水自滎澤東流至濟陰乘氏縣西分為二瀆其南瀆為荷水東南流至山陽湖陸縣與泗水合其北凟東北流入于鉅野澤東北過東郡壽良縣西界北逕須昌穀城至臨邑縣四瀆津口與河水合此蓋言齊地在河濟之間者也參考上濟西注可見濟子禮翻阿東阿鄄鄄城鄄音絹乘繩證翻荷音柯】後軍旁北海以撫千乘【旁步浪翻自臨淄東北至海北海地也漢置郡乘繩證翻】中軍據臨淄而鎭齊都祀桓公管仲於郊表賢者之閭封王蠋之墓【王蠋墓蓋在晝】齊人食邑於燕者二十餘君【封為君也】有爵位於薊者百有餘人【薊燕都也班志薊縣屬廣陽國唐為幽州治所今為燕京水經注薊城西北隅有薊丘故名燕因肩翻薊音計】六月之間下齊七十餘城皆為郡縣秦王魏王韓王會于京師<br />
<br />
  三十二年秦趙會于穰秦拔魏安城【班志安城縣屬汝南郡司馬彪志作安城時魏地南至汝南秦自武關出兵攻拔之括地志安城在豫州汝陽縣東南十七里一曰在豫州吳房縣東南穰人羊翻】兵至大梁而還【還音旋】 齊淖齒之亂湣王子法章變姓名為莒太史敫家傭【淖女教翻湣讀曰閔徐廣曰敫音躍一音皎康吉了切余按班書王子侯表有敫字師古曰古穆字今從之傭顧身為人力作為于偽翻】太史敫女奇法章狀貌以為非常人憐而常竊衣食之【竊私也私為之而不使人知衣於既翻食祥吏翻】因與私通王孫賈從湣王失王之處其母曰汝朝出而晩來則吾倚門而望汝暮出而不還則吾倚閭而望【閭里門也周禮二十五家為閭】汝今事王王走汝不知其處汝尚何歸焉王孫賈乃入市中呼曰【呼火故翻叫號也號戶高翻】淖齒亂齊國殺湣王欲與我誅之者袒右【袒右肩也】市人從者四百人與攻淖齒殺之於是齊亡臣相與求湣王子欲立之法章懼其誅己久之乃敢自言遂立以為齊王保莒城以拒燕【燕因肩翻】布告國中曰王已立在莒矣【其時樂毅以燕中軍鎭臨淄法章已立而保莒田單自安平保即墨奔敗之餘猶可置之不問法章布告國中自言已立在莒可安坐而不問乎後人論樂毅以為善藏其用吾未敢以為然也】 趙王得楚和氏璧【楚人卞和得玉璞獻之楚厲王王使玉人視之曰石也王以和為詐刖其左足及武王立和又獻之玉人又曰石也王又以為詐而刖其右足及文王立和乃抱璞而泣於荆山之下王聞之使玉人理其璞而得寶因命曰和氏之璧爾雅肉倍好謂之璧外圓象天内方象地】秦昭王欲之請易以十五城趙王欲勿與畏秦彊欲與之恐見欺以問藺相如【姓譜曰韓獻子玄孫曰康食采於藺因氏焉藺力刃翻】對曰秦以城求璧而王不許曲在我矣我與之璧而秦不與我城則曲在秦均之二策寧許以負秦【使秦負曲也】臣願奉璧而往使秦城不入臣請完璧而歸之趙王遣之相如至秦秦王無意償趙城相如乃以詐紿秦王復取璧【償辰羊翻紿蕩亥翻欺也誑也】遣從者懷之間行歸趙【從才用翻間古莧翻】而以身待命於秦秦王以為賢而弗誅禮而歸之趙王以相如為上大夫 衛嗣君薨子懷君立嗣君好察微隱縣令有發褥而席弊者【嗣祥吏翻好呼到翻令力正翻古者縣大夫至春秋時有邑大夫縣令起於戰國之時秦漢因之】嗣君聞之乃賜之席令大驚以君為神又使人過關市賂之以金【此蓋賂掌關市之官周禮司關掌國貨之節以聨門市司貨賄之出入者掌其治禁與其征廛司市掌市之治教政刑量度禁令戰國之時合為一官量音亮】既而召關市問有客過與汝金汝回遣之【回遣謂還其金也】關市大恐又愛泄姬重如耳而恐其因愛重以壅已也【泄姓也與洩同春秋時鄭有大夫洩駕陳有大夫洩冶如亦姓也張守節以如耳為魏大夫姓名非也蓋衛大夫是時魏王有如姬重音輕重之重】乃貴薄疑以敵如耳【薄姓也風俗通衛賢人薄疑敵當也】尊魏妃以偶泄姬【偶匹也對也】曰以是相參也【參三也相參列也間厠也】<br />
<br />
  荀子論之曰成侯嗣君聚歛計數之君也未及取民也子產取民者也未及為政也管仲為政者也未及修禮也故修禮者王為政者彊取民者安聚歛者亡【斂力艷翻】<br />
<br />
  三十三年秦伐趙拔兩城<br />
<br />
  三十四年秦伐趙拔石城【史記正義曰地理志右北平有石城縣括地志石城在相州林慮縣西南九十里疑相州石城是余謂北平之石城燕境也相州之石城魏境也皆非趙地此石城即漢西河之離石縣城拓拔魏分西河置五城郡又置石城縣蓋此地是也相息亮翻慮音廬燕因肩翻】 秦穰侯復為丞相【穰人羊翻復扶又翻】 楚欲與齊韓共伐秦因欲圖周王使東周武公謂楚令尹昭子曰周不可圖也【令尹楚上卿執其國之政猶秦之丞相也令力正翻】昭子曰乃圖周則無之雖然何不可圖武公曰西周之地絶長補短不過百里名為天下共主【言天下共宗周以為諸侯主杜佑曰洛陽古成周之地今洛陽城東三十餘里故城是周之下都也晉帥諸侯城之以居敬王至孝王封其弟桓公于河南以續周公之官職至孫惠公乃封少子於鞏號曰東周王赧立東西周分理又徙都西周則王城也帥讀曰率少始照翻鞏居勇翻赧奴版翻】裂其地不足以肥國得其衆不足以勁兵雖然攻之者名為弑君然而猶有欲攻之者見祭器在焉故也【謂三代所傳之祭器如九鼎之類是也】夫虎肉臊而兵利身人猶攻之若使澤中之麋蒙虎之皮人之攻之也必萬倍矣【夫音扶劉伯莊曰虎之爪牙如兵之利刃在身其肉雖臊而人猶攻之者以其皮之所在也鹿之大者曰麋麋無爪牙之利而肉可食若更蒙虎之皮人之攻之必萬倍於虎矣臊蘇遭翻魚腥肉臊麋武悲翻更古孟翻】裂楚之地足以肥國詘楚之名足以尊王【詘讀曰黜言黜其僭王之名也】今子欲誅殘天下之共主居三代之傳器器南則兵至矣於是楚計輟不行【共音如字傳直專翻言三代之器傳於周周亡則所傳之器將南歸於楚天下將合兵至楚而共討其罪也】<br />
<br />
  三十五年秦白起敗趙軍斬首二萬取代光狼城【索隱曰地志不載光狼城蓋屬趙國史記正義曰光狼故城在澤州高平縣西二十里康曰本中山地趙武靈王取之其地在代余考史以代光狼城聨而書之康以為其地在代何也又云本中山地中山與代舊為兩國代在山之陰中山在山之陽既云在代不當又云本中山地如康意抑以為光狼本代地趙襄子滅代而中山侵有光狼地武靈王既滅中山始有光狼之地白起自上郡九原雲中下兵始能敗趙軍取光狼史既不先序其兵行之路後又無考光狼城之所闕疑可也】又使司馬錯發隴西兵【扶風汧縣之西有大隴山名隴坻上者七日方越自隴以西本冀戎䝠戎氐羌之地秦累世攘拓以其地置隴西郡錯七各翻又倉故翻汧若堅翻䝠戶官翻】因蜀攻楚黔中拔之【按秦兵時因蜀出巴郡和縣路以攻拔楚之黔中黔音琴】楚獻漢北及上庸地【漢北謂漢水以北宛葉樊鄧隋唐之地上庸曹魏新城唐房陵郡之地】<br />
<br />
  三十六年秦白起伐楚取鄢鄧西陵【史記正義曰鄢鄧二城並在襄州括地志故鄢城在襄州安養北三里古鄢子之國又按水經注鄢城當在宜城南有鄢水左傳楚屈瑕伐羅及鄢鄢亂次而濟即其地徐廣曰西陵屬江夏余謂西陵即夷陵班志夷陵縣屬南郡水經江水東逕夷陵縣又東逕西陵峽蓋縣城去峽不遠夏戶雅翻】 秦王使使者告趙王願為好會於河外澠池【使使之使疏吏翻好呼到翻凡和好之好皆同音漢志澠池縣屬弘農郡杜佑曰澠池有東西俱利二城即秦趙會處宋白曰在今縣西十三里澠莫踐翻又莫忍翻】趙王欲毋行廉頗藺相如計曰【姓譜廉姓顓帝曾孫大廉之後頗普河翻】王不行示趙弱且怯也趙王遂行相如從【從才用翻】廉頗送至境與王訣【訣音决别也】曰王行度道里會遇之禮畢【度徒洛翻】還不過三十日三十日不還【還從宣翻又音如字】則請立太子以絶秦望王許之會于澠池王與趙王飲【此句作秦王與趙王飲文意乃明】酒酣【酣戶甘翻樂也洽也】秦王請趙王鼓瑟【瑟二十五絃伏羲所作史記曰泰帝使素女鼔五十絃瑟帝悲不止故破其瑟為二十五絃趙人善瑟故秦請鼓之瑟色櫛翻】趙王鼓之藺相如復請秦王擊缶【缶瓦器爾雅曰盎謂之缶注云盆也楊惲曰仰天拊缶而歌嗚嗚秦聲也說文曰缶所以盛酒秦人鼓之以節樂劉昫曰缶如足盆古西戎之樂秦俗因而用之其形如覆盆以四杖擊之復扶又翻又音如字缶方九翻惲於粉翻盛時征翻昫旴句翻】秦王不肯相如曰五步之内臣請得以頸血濺大王矣【言將殺秦王也頸居郢翻濺音箭康音贊汙灑也汙烏故翻】左右欲刃相如相如張目叱之左右皆靡【靡委靡不振之貌】王不懌【不悦也】為一擊缶【為于偽翻】罷酒秦終不能有加於趙趙人亦盛為之備秦不敢動趙王歸國以藺相如為上卿位在廉頗之右【藺力刃翻頗普何翻毛晃曰人道尚右故左右手之右以右為尊】廉頗曰我為趙將有攻城野戰之功【將即亮翻】藺相如素賤人徒以口舌而位居我上吾羞不忍為之下宣言曰我見相如必辱之【宣言者宣布其言於外也】相如聞之不肯與會每朝常稱病不欲争列【朝直遥翻毛晃曰列行次也位序也】出而望見輒引車避匿【匿藏也隱也】其舍人皆以為恥相如曰子視廉將軍孰與秦王曰不若【若猶如也】相如曰夫以秦王之威而相如廷叱之辱其羣臣【此謂請秦王擊缶時也】相如雖駑【夫音扶駑音奴字林曰駘也駘堂來翻】獨畏廉將軍哉顧吾念之彊秦所以不敢加兵於趙者徒以吾兩人在也今兩虎共鬭其勢不俱生吾所以為此者先國家之急而後私讐也【先悉薦翻後戶遘翻】廉頗聞之肉袒負荆至門謝罪【荆所以笞故負之以請罪肉袒者袒而露其肉笞丑之翻】遂為刎頸之交【刎式粉翻頸居郢翻言襟相契雖刎斷其首無所顧也崔顥曰言要齊生死斷首無悔】初燕人攻安平【燕因肩翻班志東安平縣屬淄川司馬彪志屬北海郡括地志安平城在青州臨淄縣東十九里古紀國之酅邑唐志青州有安平縣後省入博昌縣按三十一年樂毅入臨淄以中軍據之燕人攻安平當在三十二年三十三年之間故通鑑於是年以初字發之酅戶圭翻】臨淄市掾田單在安平使其宗人皆以鐵籠傅車轊【掾以絹翻掌市官屬也卷鐵以傅車轊故謂之鐵籠籠盧東翻傅音附轊音衛車軸頭謂之轊】及城潰人争門而出皆以軸折車敗為燕所擒【潰戶對翻潰散也折食列翻燕因肩翻】獨田單宗人以鐵籠得免遂犇即墨是時齊地皆屬燕獨莒即墨未下樂毅乃并右軍前軍以圍莒左軍後軍圍即墨即墨大夫出戰而死即墨人曰安平之戰田單宗人以鐵籠得全是多智習兵因共立以為將以拒燕【燕因肩翻將即亮翻】樂毅圍二邑朞年不剋乃令解圍各去城九里而為壘令曰城中民出者勿獲困者賑之【令力正翻勿獲勿禽之以為俘獲賑即忍翻救也恤也毅欲懷柔二邑使之自服不及計其死守也】使即舊業【即就也】以鎭新民【恐新民思為齊而反則以此鎭之】三年而猶未下或讒之於燕昭王曰樂毅智謀過人伐齊呼吸之間剋七十餘城今不下者兩城耳非其力不能拔所以三年不攻者欲久仗兵威以服齊人南面而王耳今齊人已服所以未發者以其妻子在燕故也且齊多美女又將忘其妻子願王圖之昭王於是置酒大會引言者而讓之曰【讓責也】先王舉國以禮賢者非貪土地以遺子孫也【遺于季翻】遭所傳德薄不能堪命國人不順齊為無道乘孤國之亂以害先王【謂王噲讓國於子之以至亡國殺身也事見上卷愼靚王五年今王元年堪勝也任也不能堪命者言王噲命子之子之不能勝王噲所命而任燕國之事也噲苦怪翻靚疾正翻勝音升任音壬】寡人統位【統他綜翻丁度曰統攝理也】痛之入骨故廣延羣臣外招賓客以求報讐其有成功者尚欲與之同共燕國今樂君親為寡人破齊夷其宗廟【夷平也為于偽翻】報塞先仇【塞悉則翻】齊國固樂君所有非燕之所得也樂君若能有齊與燕並為列國結歡同好以抗諸侯之難燕國之福寡人之願也【塞悉則翻燕因肩翻好呼到翻難乃旦翻】汝何敢言若此乃斬之賜樂毅妻以后服賜其子以公子之服輅車乘馬後屬百兩【夏奚仲作車至周而備其制輿方象地蓋圓象天三十輻以象日月蓋弓二十八以象列星龍旂九斿七仭齊軫以象大火鳥旟七斿五仞齊較以象鶉火熊旂六斿五仞齊肩以象參伐龜旐四斿四仞齊首以象營室弧旌枉矢以象弧此諸侯以下所建者也輅車之後又有屬車百兩亦當時諸國之儀乘馬四馬也孔頴達曰書序云武王戎車三百兩皆以一乘為一兩謂之兩者風俗通以車有兩輪故車稱兩乘繩正翻好呼到翻難乃旦翻屬音蜀兩音亮旂渠希翻斿夷周翻旒也參所今翻列宿星名也旐音兆】遣國相奉而致之樂毅立樂毅為齊王【相息亮翻】樂毅惶恐不受拜書以死自誓由是齊人服其義諸侯畏其信莫敢復有謀者頃之昭王薨惠王立【頃之言無幾何時毅魚器翻復扶又翻】惠王自為太子時嘗不快於樂毅田單聞之乃縱反間於燕【孫子五間有反間因其敵間而用之又曰敵間之間我者因而利之導而舍之故反間可得而用也間古莧翻】宣言曰齊王已死城之不拔者二耳樂毅與燕新王有隙畏誅而不敢歸以伐齊為名實欲連兵南面王齊【王于况翻又音如字】齊人未附故且緩攻即墨以待其事齊人所懼唯恐他將之來即墨殘矣燕王固已疑樂毅得齊反間乃使騎劫代將而召樂毅【將即亮翻燕因肩翻下同騎奇寄翻康曰姓也余謂騎劫時以能而將騎以官稱非姓也毅魚器翻】樂毅知王不善代之【知王遣代其意不善將誅之也】遂犇趙燕將士由是憤惋不和【燕因肩翻將即亮翻惋烏貫翻】田單令城中人食必祭其先祖於庭飛鳥皆翔舞而下城中燕人怪之田單因宣言曰當有神師下教俄有一卒曰臣可以為師乎因反走田單起引還坐東鄉師事之【鄉讀曰嚮】卒曰臣欺君田單曰子勿言也因師之每出約束必稱神師【田單恐衆心未一故假神以令其衆】乃宣言曰吾唯懼燕軍之劓所得齊卒【劓魚器翻割鼻也】置之前行【行戶剛翻】即墨敗矣燕人聞之如其言城中見降者盡劓皆怒堅守唯恐見得【降戶江翻】單又縱反間言吾懼燕人掘吾城外冢墓可為寒心【掘其月翻冢而隴翻】燕軍盡掘冢墓燒死人齊人從城上望見皆涕泣其欲出戰怒自十倍田單知士卒之可用乃身操版鍤與士卒分功【操七刀翻鍤則洽翻鍫也】妻妾編於行伍之間盡散飲食饗士令甲卒皆伏使老弱女子乘城【乘登也登城而守也】遣使約降於燕【使疏吏翻降戶江翻燕因肩翻】燕軍皆呼萬歲田單又收民金得千鎰令即墨富豪遺燕將曰即降願無虜掠吾族家燕將大喜許之燕軍益懈田單乃收城中得牛千餘為絳繒衣【鎰弋質翻遺于季翻令盧經翻懈古隘翻繒慈陵翻絹也】畫以五采龍文【畫古畫字通】束兵刃於其角而灌脂束葦於其尾【葦于鬼翻葭也】燒其端鑿城數十穴夜縱牛壯士五千隨其後【史記田單傳壯士五千下有人字】牛尾熱怒而犇燕軍燕軍大驚視牛皆龍文所觸盡死傷而城中鼓譟從之【譟先到翻羣呼也】老弱皆擊銅器為聲聲動天地燕軍大駭敗走齊人殺騎劫追亡逐北【亡逃亡也北奔北也逃亡者追之奔北者逐之楊倞曰北者乖背之名故以敗走為北毛晃曰人道面南偝北北者偝也故古以堂北為背背亦偝也以敗走為北者取偝之而走耳】所過城邑皆叛燕復為齊田單兵日益多乘勝燕日敗亡走至河上而齊七十餘城皆復焉【燕因肩翻為于偽翻又音如字】乃迎襄王於莒入臨淄封田單為安平君【齊以田單安國平難又嘗保安平故因以安平封之】齊王以太史敫之女為后生太子建太史敫曰女不取媒因自嫁非吾種也汙吾世【徐廣曰敫音躍一音皎師古曰敫古穆字種之隴翻汙烏故翻凡染汙之汙皆同音】終身不見君王后君王后亦不以不見故失人子之禮趙王封樂毅於觀津【班志觀津縣屬信都國觀工喚翻】尊寵之以警動於燕齊燕惠王乃使人讓樂毅且謝之曰將軍過聼以與寡人有隙遂捐燕歸趙【捐余專翻弃也】將軍自為計則可矣而亦何以報先王之所以遇將軍之意乎樂毅報書曰昔伍子胥說聼於闔閭而吳遠迹至郢夫差弗是也賜之鴟夷而浮之江吳王不寤先論之可以立功故沈子胥而不悔子胥不蚤見主之不同量是以至於入江而不化【遠迹言自吳至郢其道里甚遠而行迹得至也弗是謂夫差弗以子胥之言為是也伍子胥楚人也楚平王信讒殺其父兄子胥奔吳吳王闔閭信而用之伐楚入郢闔閭卒夫差立子胥屢諫不聼賜之屬鏤以死子胥既死夫差取其尸盛之鴟夷浮之江中應劭曰鴟夷搕形也以馬革為之韋昭曰革囊也或曰生牛皮也索隱曰言子胥怨恨故雖投江而神不化猶為波濤之神也郢以井翻說式芮翻夫音扶差初加翻鴟丑之翻沈持林翻量力讓翻闔戶臘翻卒子恤翻鏤力俱翻又力侯翻盛時征翻榼戶盍翻】夫免身立功以明先王之迹臣之上計也【夫音扶】離毀辱之誹謗墮先王之名臣之所大恐也【離與罹同墮與隳同音火規翻後凡墮毀之墮皆同音】臨不測之罪以幸為利義之所不敢出也【謂不敢與趙謀燕】臣聞古之君子交絶不出惡聲忠臣去國不潔其名臣雖不佞【不佞猶言不才也】數奉教於君子矣【數所角翻】唯君王之留意焉於是燕王復以樂毅子閒為昌國君【燕因肩翻索隱曰閒音紀閑翻】而樂毅往來復通燕卒於趙號曰望諸君【望諸澤名本齊地毅自齊奔趙趙人以此號之本其所從來也卒子恤翻】田單相齊【相息亮翻】過淄水【水經淄水出泰山萊蕪縣原山東北過臨淄縣又東過利縣東東北入于海】有老人涉淄而寒出水不能行田單解其裘而衣之襄王惡之曰田單之施於人【衣于既翻惡烏路翻施式䜴翻後凡布施之施皆同音】將以取我國乎不早圖恐後之變也左右顧無人巖下有貫珠者【巖下殿巖之下也昔舜遊巖廊】襄王呼而問之曰汝聞吾言乎對曰聞之王曰汝以為何如對曰王不如因以為己善王嘉單之善下令曰寡人憂民之饑也單收而食之寡人憂民之寒也單解裘而衣之寡人憂勞百姓而單亦憂稱寡人之意【食祥吏翻稱昌孕翻愜也後凡稱愜之稱皆同音】單有是善而王嘉之單之善亦王之善也王曰善乃賜單牛酒後數日貫珠者復見王曰王朝日宜召田單而揖之於庭口勞之【復扶又翻朝陟遥翻旦日又直遥翻朝羣臣之日也勞力到翻几撫勞之勞皆同音】乃布令求百姓之饑寒者收穀之【穀如字養也收穀收而養乏也】乃使人聼於閭里聞大夫之相與語者曰田單之愛人嗟乃王之教也田單任貂勃於王【任汝鴆翻保也今之任子義亦如此貂丁聊翻康曰姓也】王有所幸臣九人欲傷安平君【傷譛毀也害也損也】相與語於王曰燕之伐齊之時楚王使將軍將萬人而佐齊【軍將即亮翻又音如字領也】今國已定而社稷已安矣何不使使者謝於楚王王【曰曰】左右孰可九人之屬曰貂勃可貂勃使楚楚王受而觴之數月不反【使疏吏翻觴之者舉觴以禮之也】九人之屬相與語曰夫一人之身而牽留萬乘者豈不以據勢也哉【謂貂勃以安平君之重楚王留而禮遇之也夫音扶乘繩證翻】且安平君之與王也君臣無異而上下無别【别彼列翻】且其志欲為不善内撫百姓外懷戎翟【翟與狄同】禮天下之賢士其志欲有為願王察之異日王曰召相單而來【異日猶言他日也相息亮翻】田單免冠徒跣肉袒而進【徒跣徒行而跣足也跣先典翻不屨而以足親地也李巡曰䄠裼脱衣袒肩見體曰肉袒䄠與袒同裼先的翻】退而請死罪五日而王曰子無罪於寡人子為子之臣禮吾為吾之王禮而已矣貂勃從楚來王賜之酒酒酣王曰召相單而來貂勃避席稽首【酣戶甘翻酒樂也應劭曰洽也稽音啟下首拜也】曰王上者孰與周文王王曰吾不若也貂勃曰然臣固知王不若也下者孰與齊桓公王曰吾不若也貂勃曰然臣固知王不若也然則周文王得呂尚以為太公【呂尚釣于渭濱周文王出獵載與俱歸曰吾太公望子久矣因號曰太公望】齊桓公得管夷吾以為仲父【齊公子無知之亂管夷吾奉公子糾與桓公争國子糾死管仲囚桓公釋其罪任之以政號曰仲父姓譜管姓周文王子管叔之後】今王得安平君而獨曰單安得此亡國之言乎且自天地之闢民人之始為人臣之功者誰有厚於安平君者哉王不能守王之社稷燕人興師而襲齊王走而之城陽之山中【襄王從湣王走莒班志莒縣屬城陽國故云城陽之山中湣讀曰閔莒許與翻】安平君以惴惴即墨三里之城五里之郭敝卒七千人禽其司馬而反千里之齊安平君之功也【惴之睡翻危恐之貌司馬蓋指騎劫】當是之時舍城陽而自王【舍讀曰捨王于况翻】天下莫之能止然而計之於道歸之於義以為不可故棧道木閣而迎王與后於城陽山中【架木通路曰棧道棧士限翻康士諫切非】王乃得反子臨百姓今國已定民已安矣王乃曰單嬰兒之計不為此也王亟殺此九子者以謝安平君不然國其危矣乃殺九子而逐其家益封安平君以夜邑萬戶【夜邑戰國策作掖邑班志掖縣屬東萊郡掖羊益翻】田單將攻狄【班志狄縣屬千乘郡後漢安帝改曰臨濟徐廣曰狄今樂安臨濟縣也史記正義曰故狄城在淄州高苑縣西北】往見魯仲連【姓譜魯以國為姓】魯仲連曰將軍攻狄不能下也田單曰臣以即墨破亡餘卒破萬乘之燕復齊之墟今攻狄而不下何也上車弗謝而去【乘繩證翻燕因肩翻上時掌翻】遂攻狄三月不克齊小兒謡曰大冠若箕【謡余韶翻徒歌曰謡大冠武冠也】脩劒拄頤【脩長也拄冢庾翻】攻狄不能下壘枯骨成丘田單乃懼問魯仲連曰先生謂單不能下狄請聞其說魯仲連曰將軍之在即墨坐則織蕢【蕢其位翻草器也】立則仗鍤【顔師古曰仗直亮翻憑荷也鍤則洽翻】為士卒倡曰無可往矣宗廟亡矣今日尚矣歸於何黨矣【毛晃曰尚庶幾也言單于其時蓋言曰今日之事尚庶幾焉黨類也言戰有勝負不死則降將歸于何類也】當此之時將軍有死之心士卒無生之氣聞君言莫不揮泣奮臂而欲戰此所以破燕也當今將軍東有夜邑之奉西有淄上之娛【此蓋言安平封邑益之以夜邑夜邑在安平東淄水在安平西夜邑有租賦之奉淄上有遊觀之樂故魯仲連云然燕因肩翻夜讀曰掖音羊益翻】黄金横帶而騁乎淄澠之間【水經註淄水自利縣東北流逕安平城北又東逕廣饒縣與濁水會濁水出廣饒縣冶嶺山亦謂之繩水又北與時澠之水會時水出齊城西北北會繩水繩水出管城東世謂漢溱水西逕樂安博縣與時水合孔子謂淄澠之合易身嘗而知之即斯水也騁丑郢翻馳騖也澠時陵翻溱仄詵翻又音秦】有生之樂無死之心所以不勝也【樂音洛】田單曰單之有心先生志之矣【志者心之所主也】明日乃厲氣循城【厲嚴厲也勉厲也奮厲也振厲也是三者有修飭振起之意】立於矢石之所援枹鼔之狄人乃下【援于元翻引也後以義推枹芳無翻擊鼔杖】初齊湣王既滅宋欲去孟嘗君【二十九年書齊滅宋先書宋滅薛時孟嘗君已封於薛宋所滅者何薛邪去羌呂翻】孟嘗君奔魏魏昭王以為相與諸侯共伐破齊湣王死襄王復國而孟嘗君中立為諸侯無所屬襄王新立畏孟嘗君與之連和孟嘗君卒諸子争立而齊魏共滅薛孟嘗君絶嗣【嗣祥吏翻】<br />
<br />
  三十七年秦大良造白起伐楚拔郢【括地志郢城在江陵縣東北六里楚平王築都之地】燒夷陵楚襄王兵散遂不復戰東北徙都於陳【陳即古陳國班志陳縣屬淮陽國註云楚頃襄王自郢徙北復扶又翻頃窺營翻】秦以郢為南郡封白起為武安君【班志武安縣屬魏郡戰國之君分封其臣如平原武安之類非眞食其縣之入也張守節曰言能撫養軍士戰必克得百姓安集故曰武安】<br />
<br />
  三十八年秦武安君定巫黔中初置黔中郡【括地志黔中故城在辰州沅陵縣西二十二里江南今黔府亦其地按秦黔中郡地非唐黔州地也宋白曰秦黔中郡所理在今辰州西二十里黔中故郡城是漢改黔中為武陵郡移理義陵即今辰州溆浦縣是後漢移理臨沅即今朗州所理是今辰州叙奬溪澧朗施八州是秦漢黔中郡之地自永嘉以後没于夷䝤元魏之後圖記不傳至後周保定四方涪陵首領田思鶴歸化初于其地立奉州續改為黔州大業中又改為黔安郡因周隋州郡之名遂為秦漢黔中郡交㸦難辯今黔州及夷費思播與秦黔中郡隔越峻嶺以山川言之炳然自分黔其今翻又其炎翻沅音元溆音叙澧里弟翻䝤魯皓翻涪音浮㸦與互同費兵媚翻以水名】 魏昭王薨子安釐王立【世本曰安釐王名圉釐讀曰僖】<br />
<br />
  三十九年秦武安君伐魏拔兩城 楚王收東地兵【東地蓋楚之東境淮汝之地也】得十餘萬復西取江南十五邑【復扶又翻又音如字】魏安釐王封其弟無忌為信陵君【宋白曰信陵君邑于甯今宋州寜陵】<br />
<br />
  【縣古甯城也】<br />
<br />
  四十年秦相國穰侯伐魏韓暴鳶救魏【暴曰報翻姓也周有卿士暴公其後遂以為氏鳶以專翻名也】穰侯大破之【穰人羊翻】斬首四萬暴鳶走開封【班志開封縣屬河南郡賢曰開封故城在今汴城南宋白曰今汴州開封縣南五十里開封故城是漢理所汴皮變翻】魏納八城以和穰侯復伐魏走芒卯入北宅【北宅即宅陽復扶又翻芒莫郎翻】魏人割温以和【温縣即春秋温邑屬晉唐屬孟州】四十一年魏復與齊合從秦穰侯伐魏拔四城斬首四萬【從子容翻穰人羊翻】 魯湣公薨子頃公讐立【湣讀曰閔頃音傾謚法甄心動懼曰頃敏以敬慎曰頃】<br />
<br />
  四十二年趙人魏人伐韓華陽【司馬彪曰華陽山名在河南密縣括地志在鄭州管城縣南四十里水經註黄水出新鄭縣太山黄泉東南流逕華城西史伯謂鄭桓公曰華君之土也韋昭曰華國名也華戶化翻】韓人告急於秦秦王弗救韓相國謂陳筮曰事急矣【相息亮翻】願公雖病為一宿之行陳筮如秦【如往也】見穰侯穰侯曰事急乎故使公來陳筮曰未急也穰侯怒曰何也陳筮曰彼韓急則將變而他從【謂從趙魏也】以未急故復來耳【復扶又翻】穰侯曰請發兵矣乃與武安君及客卿胡陽救韓【姓譜舜後胡公滿封於陳子孫以為氏又陸終氏六子長曰昆吾次曰參胡董姓封於韓墟周為胡國楚滅之】八日而至敗魏軍於華陽之下【蓋華陽城下也】走芒卯虜三將斬首十三萬武安君又與趙將賈偃戰沈其卒二萬人于河【將即亮翻沈持林翻人皆貪生而畏死二萬人與戰烏得盡沈諸河以計沈之也】魏段干子請割南陽予秦以和【古予與字通下書南陽寔脩武班志脩武縣屬河内郡應劭曰晉始啟南陽今南陽城是也其地在晉山南河北故曰南陽劉原父曰脩武即晉之甯邑武王伐紂名之韓詩外傳武王伐紂勒兵於甯故曰脩武有古南陽城父音甫傳直戀翻】蘇代謂魏王曰欲璽者段干子也欲地者秦也今王使欲地者制璽欲璽者制地魏地盡矣【璽印也言段干子欲得秦相印故請魏割地璽斯氏翻】夫以地事秦猶抱薪救火【夫音扶】薪不盡火不滅王曰是則然也雖然事始已行不可更矣【更工衡翻】對曰夫博之所以貴梟者便則食不便則止今何王之用智不如用梟也【夫音扶鄭司農註考工記曰博立梟棊宋玉楚辭曰箟蔽象棊有六博成梟而牟呼五白謝艾曰六博得梟者勝史記正義曰博頭有刻為梟鳥形者擲得梟者合食其子不便則為餘行也梁湘東王繹博食子未下以其有便不便也梟堅堯翻】魏王不聼卒以南陽為和【卒子恤翻】寔脩武 韓釐王薨子桓惠王立 韓魏既服於秦秦王將使武安君與韓魏伐楚未行而楚使者黄歇至【姓譜陸終之後受封於黄為楚所滅其後以國為氏使疏吏翻歇許竭翻】聞之畏秦乘勝一舉而滅楚也乃上書曰臣聞物至則反冬夏是也【上時掌翻至極也物極則反也冬至陰之極而陽生焉夏至陽之極而陰生焉】致至則危累棊是也【致亦極也極其至則危也累棊至於極高則必危矣楚司馬子期累十二博棊不墜王曰危哉】今大國之地徧天下有其二垂【史記正義曰極東極西也余謂秦國之地有天下西北之二垂也】此從生民以來萬乘之地未嘗有也先王三世不忘接地於齊以絶從親之要【乘繩證翻從子容翻索隱曰要讀曰腰以言山東合從韓魏是其腰康曰於笑切約也余謂索隱說是】今王使盛橋守事於韓盛橋以其地入秦【索隱曰秦使盛橋守事於韓亦猶楚使召滑相趙然也盛姓也相息亮翻】是王不用甲不信威而得百里之地王可謂能矣【信讀曰申後屈信之信皆同音】王又舉甲而攻魏杜大梁之門舉河内拔燕酸棗虚桃入邢魏之兵雲翔而不敢捄【徐廣曰始皇五年取酸棗燕虚蘇代曰决宿胥之口魏無虛頓丘又曰燕縣有桃城班志東郡有燕縣陳留郡有酸棗縣水經註濮渠東北逕燕城内為陽清湖又逕桃城南即戰國策所謂燕酸棗虛桃者史記正義曰故桃城在滑州昨城縣東三十里燕於賢翻虛如字徐廣曰平臯有邢丘劉昭曰邢丘故邢國史記正義曰邢丘在懷州武德縣東南二十里捄與救同】王之功亦多矣王休甲息衆三年而後復之又并蒲衍首垣以臨仁平丘黄濟陽嬰城而魏氏服【徐廣曰皆屬陳留索隱曰仁及平丘二縣名班志平丘外黄濟陽三縣屬陳留仁地闕張晏曰魏郡有内黄故加外臣瓚曰縣有黄溝師古曰左傳魯惠公敗宋師於黄杜預以為外黄縣有黄城即此地也索隱又曰謂秦以兵臨仁平丘二縣則黄濟陽嬰城而自守也平丘句斷史記正義曰故黄城在曹州考城縣東按水經註黄溝名也河水舊於白馬南泆通濮濟黄溝濟陽故城在曹州寃句縣西南康曰蒲在長垣之蒲鄉衍在河南與卷近首蓋牛首垣即長垣濟子禮翻垣于元翻瓚藏旱翻索山客翻傳直戀翻】王又割濮磨之北注齊秦之要絶楚趙之脊天下五合六聚而不敢捄王之威亦單矣【濮博木翻要讀曰腰脊資昔翻徐廣曰濮水北於鉅野入濟索隱曰濮磨地名近濮水水經濮水上承濟水於封丘縣班志所謂濮水首濟者也東北流左會别濮水水受河於酸棗縣杜預所謂濮水出酸棗縣首受河者也東至乘氏縣與濟同入鉅野澤磨康莫賀切言秦既服魏又割濮磨之北則地連於齊是注齊之要也魏地既入於秦則楚趙之聲勢不接是絶楚趙之脊也單與殫同索隱曰單盡也言秦王之威盡行也濟子禮翻乘繩證翻索山客翻】王若能保功守威絀攻取之心【絀敕律翻黜也】而肥仁義之地使無後患三王不足四五伯不足六也【伯讀曰覇】王若負人徒之衆仗兵革之彊乘毀魏之威而欲以力臣天下之主臣恐其有後患也詩曰靡不有初鮮克有終【詩變大雅蕩之辭鮮息善翻少也後以義推】易曰狐涉水濡其尾【易未濟小狐汔濟濡其尾彖曰小狐汔濟未出中也濡其尾無攸利不續終也濡汝朱翻】此言始之易終之難也【易弋豉翻】昔吳之信越也從而伐齊既勝齊人於艾陵還為越王禽於三江之浦【事見左傳史記正義曰艾山在兗州博縣南六十里三江即禹貢所謂三江既入震澤底定者也吳地記松江東北行七十里得三江口東北入海為婁江東南入海為東江併松江為三江水瀕曰浦】智氏之信韓魏也從而伐趙攻晉陽城勝有日矣韓魏叛之殺智伯瑶於鑿臺之下【事見一卷威烈王二十三年水經註太原榆次縣同過水側有鑿臺】今王妬楚之不毀【孔頴達曰本以色曰妬以行曰忌但後之作者妬亦兼行】而忘毀楚之彊韓魏也臣為王慮而不取也【為于偽翻】夫楚國援也鄰國敵也【夫音扶】今王信韓魏之善王此正吳之信越也臣恐韓魏卑辭除患而實欲欺大國也何則王無重世之德於韓魏而有累世之怨焉【索隱曰重世猶再世也重直龍翻累魯水翻】夫韓魏父子兄弟接踵而死於秦將十世矣故韓魏之不亡秦社稷之憂也今王資之與攻楚不亦過乎【資之謂資以兵也】且攻楚將惡出兵【惡音烏】王將借路於仇讐之韓魏乎兵出之日而王憂其不反也王若不借路於仇讐之韓魏必攻隨水右壤【索隱曰楚都陳隨水右壤蓋在隨水之西今鄧州之西其地多山林者是余謂右壤蓋其地在楚都之右】此皆廣川大水山林谿谷不食之地【記擅弓成子高曰死則擇不食之地而葬註云不食謂不墾耕】是王有毀楚之名而無得地之實也且王攻楚之日四國必悉起兵而應王秦楚之兵構而不離魏氏將出而攻留方與銍湖陵碭蕭相故宋必盡【班志留縣屬楚國方與湖陵縣屬山陽郡銍蕭相三縣屬沛郡碭縣屬梁國銍竹乙翻方音房與音預碭音唐又徒浪翻相息亮翻史記正義曰徐州西宋州東兗州南並故宋地】齊人南面攻楚泗上必舉【時楚蠶食魯國有泗上之地】此皆平原四達膏腴之地如此則天下之國莫彊於齊魏矣臣為王慮莫若善楚秦楚合而為一以臨韓韓必斂手而朝王施以東山之險帶以曲河之利【東山謂華山以至崤塞諸山皆在咸陽之東曲河謂河千里一曲按水經河水自雲中沙南縣屈而南流至華陰潼關曲而東流所謂曲河也春秋說題辭曰河之為言荷也荷精分布懷陰引度也釋名曰河下也隨地下處而通流也】韓必為關内之侯若是而王以十萬戍鄭【鄭韓之國都也】梁氏寒心許鄢陵嬰城而上蔡召陵不往來也【許春秋許國班志許鄢陵二縣皆屬潁川郡上蔡故蔡國蔡仲所封後徙新蔡故此為上蔡召陵即齊桓公伐楚所次之地二縣班志皆屬汝南郡魏都大梁其境南至汝南許鄢陵居其間二邑皆脅於秦兵嬰城自守則楚之上蔡召陵不能與大梁往來矣嬰繞也嬰城者謂以兵繞城而守也郾漢書音義音甚多丁度毛晃音從於建翻召讀曰邵】如此魏亦關内侯矣大王壹善楚而關内兩萬乘之主【乘繩證翻】注地於齊齊右壤可拱手而取也【齊右壤謂濟西之地也】王之地一經兩海要約天下【東西為經兩海東海西海也謂自西海至東海其地一為秦所有也要約猶約束也要於遥翻】是燕趙無齊楚齊楚無燕趙也【此極言山東諸國連從之為秦害也燕因肩翻】然後危動燕趙直揺齊楚此四國者不待痛而服矣【燕因肩翻】王從之止武安君而謝韓魏使黄歇歸約親於楚<br />
<br />
  資治通鑑卷四<br />
<br />
<史部,編年類,資治通鑑>  <br>
   </div> 

<script src="/search/ajaxskft.js"> </script>
 <div class="clear"></div>
<br>
<br>
 <!-- a.d-->

 <!--
<div class="info_share">
</div> 
-->
 <!--info_share--></div>   <!-- end info_content-->
  </div> <!-- end l-->

<div class="r">   <!--r-->



<div class="sidebar"  style="margin-bottom:2px;">

 
<div class="sidebar_title">工具类大全</div>
<div class="sidebar_info">
<strong><a href="http://www.guoxuedashi.com/lsditu/" target="_blank">历史地图</a></strong>  
<a href="http://www.880114.com/" target="_blank">英语宝典</a>  
<a href="http://www.guoxuedashi.com/13jing/" target="_blank">十三经检索</a> 
<br><strong><a href="http://www.guoxuedashi.com/gjtsjc/" target="_blank">古今图书集成</a></strong> 
<a href="http://www.guoxuedashi.com/duilian/" target="_blank">对联大全</a> <strong><a href="http://www.guoxuedashi.com/xiangxingzi/" target="_blank">象形文字典</a></strong> 

<br><a href="http://www.guoxuedashi.com/zixing/yanbian/">字形演变</a>  <strong><a href="http://www.guoxuemi.com/hafo/" target="_blank">哈佛燕京中文善本特藏</a></strong>
<br><strong><a href="http://www.guoxuedashi.com/csfz/" target="_blank">丛书&方志检索器</a></strong> <a href="http://www.guoxuedashi.com/yqjyy/" target="_blank">一切经音义</a>  

<br><strong><a href="http://www.guoxuedashi.com/jiapu/" target="_blank">家谱族谱查询</a></strong>  <strong><a href="http://shufa.guoxuedashi.com/sfzitie/" target="_blank">书法字帖欣赏</a></strong> 
<br>

</div>
</div>


<div class="sidebar" style="margin-bottom:0px;">

<font style="font-size:22px;line-height:32px">QQ交流群9:489193090</font>


<div class="sidebar_title">手机APP 扫描或点击</div>
<div class="sidebar_info">
<table>
<tr>
	<td width=160><a href="http://m.guoxuedashi.com/app/" target="_blank"><img src="/img/gxds-sj.png" width="140"  border="0" alt="国学大师手机版"></a></td>
	<td>
<a href="http://www.guoxuedashi.com/download/" target="_blank">app软件下载专区</a><br>
<a href="http://www.guoxuedashi.com/download/gxds.php" target="_blank">《国学大师》下载</a><br>
<a href="http://www.guoxuedashi.com/download/kxzd.php" target="_blank">《汉字宝典》下载</a><br>
<a href="http://www.guoxuedashi.com/download/scqbd.php" target="_blank">《诗词曲宝典》下载</a><br>
<a href="http://www.guoxuedashi.com/SiKuQuanShu/skqs.php" target="_blank">《四库全书》下载</a><br>
</td>
</tr>
</table>

</div>
</div>


<div class="sidebar2">
<center>


</center>
</div>

<div class="sidebar"  style="margin-bottom:2px;">
<div class="sidebar_title">网站使用教程</div>
<div class="sidebar_info">
<a href="http://www.guoxuedashi.com/help/gjsearch.php" target="_blank">如何在国学大师网下载古籍?</a><br>
<a href="http://www.guoxuedashi.com/zidian/bujian/bjjc.php" target="_blank">如何使用部件查字法快速查字?</a><br>
<a href="http://www.guoxuedashi.com/search/sjc.php" target="_blank">如何在指定的书籍中全文检索?</a><br>
<a href="http://www.guoxuedashi.com/search/skjc.php" target="_blank">如何找到一句话在《四库全书》哪一页?</a><br>
</div>
</div>


<div class="sidebar">
<div class="sidebar_title">热门书籍</div>
<div class="sidebar_info">
<a href="/so.php?sokey=%E8%B5%84%E6%B2%BB%E9%80%9A%E9%89%B4&kt=1">资治通鉴</a> <a href="/24shi/"><strong>二十四史</strong></a>&nbsp; <a href="/a2694/">野史</a>&nbsp; <a href="/SiKuQuanShu/"><strong>四库全书</strong></a>&nbsp;<a href="http://www.guoxuedashi.com/SiKuQuanShu/fanti/">繁体</a>
<br><a href="/so.php?sokey=%E7%BA%A2%E6%A5%BC%E6%A2%A6&kt=1">红楼梦</a> <a href="/a/1858x/">三国演义</a> <a href="/a/1038k/">水浒传</a> <a href="/a/1046t/">西游记</a> <a href="/a/1914o/">封神演义</a>
<br>
<a href="http://www.guoxuedashi.com/so.php?sokeygx=%E4%B8%87%E6%9C%89%E6%96%87%E5%BA%93&submit=&kt=1">万有文库</a> <a href="/a/780t/">古文观止</a> <a href="/a/1024l/">文心雕龙</a> <a href="/a/1704n/">全唐诗</a> <a href="/a/1705h/">全宋词</a>
<br><a href="http://www.guoxuedashi.com/so.php?sokeygx=%E7%99%BE%E8%A1%B2%E6%9C%AC%E4%BA%8C%E5%8D%81%E5%9B%9B%E5%8F%B2&submit=&kt=1"><strong>百衲本二十四史</strong></a>  <a href="http://www.guoxuedashi.com/so.php?sokeygx=%E5%8F%A4%E4%BB%8A%E5%9B%BE%E4%B9%A6%E9%9B%86%E6%88%90&submit=&kt=1"><strong>古今图书集成</strong></a>
<br>

<a href="http://www.guoxuedashi.com/so.php?sokeygx=%E4%B8%9B%E4%B9%A6%E9%9B%86%E6%88%90&submit=&kt=1">丛书集成</a> 
<a href="http://www.guoxuedashi.com/so.php?sokeygx=%E5%9B%9B%E9%83%A8%E4%B8%9B%E5%88%8A&submit=&kt=1"><strong>四部丛刊</strong></a>  
<a href="http://www.guoxuedashi.com/so.php?sokeygx=%E8%AF%B4%E6%96%87%E8%A7%A3%E5%AD%97&submit=&kt=1">說文解字</a> <a href="http://www.guoxuedashi.com/so.php?sokeygx=%E5%85%A8%E4%B8%8A%E5%8F%A4&submit=&kt=1">三国六朝文</a>
<br><a href="http://www.guoxuedashi.com/so.php?sokeytm=%E6%97%A5%E6%9C%AC%E5%86%85%E9%98%81%E6%96%87%E5%BA%93&submit=&kt=1"><strong>日本内阁文库</strong></a> <a href="http://www.guoxuedashi.com/so.php?sokeytm=%E5%9B%BD%E5%9B%BE%E6%96%B9%E5%BF%97%E5%90%88%E9%9B%86&ka=100&submit=">国图方志合集</a> <a href="http://www.guoxuedashi.com/so.php?sokeytm=%E5%90%84%E5%9C%B0%E6%96%B9%E5%BF%97&submit=&kt=1"><strong>各地方志</strong></a>

</div>
</div>


<div class="sidebar2">
<center>

</center>
</div>
<div class="sidebar greenbar">
<div class="sidebar_title green">四库全书</div>
<div class="sidebar_info">

《四库全书》是中国古代最大的丛书,编撰于乾隆年间,由纪昀等360多位高官、学者编撰,3800多人抄写,费时十三年编成。丛书分经、史、子、集四部,故名四库。共有3500多种书,7.9万卷,3.6万册,约8亿字,基本上囊括了古代所有图书,故称“全书”。<a href="http://www.guoxuedashi.com/SiKuQuanShu/">详细>>
</a>

</div> 
</div>

</div>  <!--end r-->

</div>
<!-- 内容区END --> 

<!-- 页脚开始 -->
<div class="shh">

</div>

<div class="w1180" style="margin-top:8px;">
<center><script src="http://www.guoxuedashi.com/img/plus.php?id=3"></script></center>
</div>
<div class="w1180 foot">
<a href="/b/thanks.php">特别致谢</a> | <a href="javascript:window.external.AddFavorite(document.location.href,document.title);">收藏本站</a> | <a href="#">欢迎投稿</a> | <a href="http://www.guoxuedashi.com/forum/">意见建议</a> | <a href="http://www.guoxuemi.com/">国学迷</a> | <a href="http://www.shuowen.net/">说文网</a><script language="javascript" type="text/javascript" src="https://js.users.51.la/17753172.js"></script><br />
  Copyright &copy; 国学大师 古典图书集成 All Rights Reserved.<br>
  
  <span style="font-size:14px">免责声明:本站非营利性站点,以方便网友为主,仅供学习研究。<br>内容由热心网友提供和网上收集,不保留版权。若侵犯了您的权益,来信即刪。scp168@qq.com</span>
  <br />
ICP证:<a href="http://www.beian.miit.gov.cn/" target="_blank">鲁ICP备19060063号</a></div>
<!-- 页脚END --> 
<script src="http://www.guoxuedashi.com/img/plus.php?id=22"></script>
<script src="http://www.guoxuedashi.com/img/tongji.js"></script>

</body>
</html>
