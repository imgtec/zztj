<!DOCTYPE html PUBLIC "-//W3C//DTD XHTML 1.0 Transitional//EN" "http://www.w3.org/TR/xhtml1/DTD/xhtml1-transitional.dtd">
<html xmlns="http://www.w3.org/1999/xhtml">
<head>
<meta http-equiv="Content-Type" content="text/html; charset=utf-8" />
<meta http-equiv="X-UA-Compatible" content="IE=Edge,chrome=1">
<title>資治通鑒_169-資治通鑑卷一百六十八_169-資治通鑑卷一百六十八</title>
<meta name="Keywords" content="資治通鑒_169-資治通鑑卷一百六十八_169-資治通鑑卷一百六十八">
<meta name="Description" content="資治通鑒_169-資治通鑑卷一百六十八_169-資治通鑑卷一百六十八">
<meta http-equiv="Cache-Control" content="no-transform" />
<meta http-equiv="Cache-Control" content="no-siteapp" />
<link href="/img/style.css" rel="stylesheet" type="text/css" />
<script src="/img/m.js?2020"></script> 
</head>
<body>
 <div class="ClassNavi">
<a  href="/24shi/">二十四史</a> | <a href="/SiKuQuanShu/">四库全书</a> | <a href="http://www.guoxuedashi.com/gjtsjc/"><font  color="#FF0000">古今图书集成</font></a> | <a href="/renwu/">历史人物</a> | <a href="/ShuoWenJieZi/"><font  color="#FF0000">说文解字</a></font> | <a href="/chengyu/">成语词典</a> | <a  target="_blank"  href="http://www.guoxuedashi.com/jgwhj/"><font  color="#FF0000">甲骨文合集</font></a> | <a href="/yzjwjc/"><font  color="#FF0000">殷周金文集成</font></a> | <a href="/xiangxingzi/"><font color="#0000FF">象形字典</font></a> | <a href="/13jing/"><font  color="#FF0000">十三经索引</font></a> | <a href="/zixing/"><font  color="#FF0000">字体转换器</font></a> | <a href="/zidian/xz/"><font color="#0000FF">篆书识别</font></a> | <a href="/jinfanyi/">近义反义词</a> | <a href="/duilian/">对联大全</a> | <a href="/jiapu/"><font  color="#0000FF">家谱族谱查询</font></a> | <a href="http://www.guoxuemi.com/hafo/" target="_blank" ><font color="#FF0000">哈佛古籍</font></a> 
</div>

 <!-- 头部导航开始 -->
<div class="w1180 head clearfix">
  <div class="head_logo l"><a title="国学大师官网" href="http://www.guoxuedashi.com" target="_blank"></a></div>
  <div class="head_sr l">
  <div id="head1">
  
  <a href="http://www.guoxuedashi.com/zidian/bujian/" target="_blank" ><img src="http://www.guoxuedashi.com/img/top1.gif" width="88" height="60" border="0" title="部件查字,支持20万汉字"></a>


<a href="http://www.guoxuedashi.com/help/yingpan.php" target="_blank"><img src="http://www.guoxuedashi.com/img/top230.gif" width="600" height="62" border="0" ></a>


  </div>
  <div id="head3"><a href="javascript:" onClick="javascript:window.external.AddFavorite(window.location.href,document.title);">添加收藏</a>
  <br><a href="/help/setie.php">搜索引擎</a>
  <br><a href="/help/zanzhu.php">赞助本站</a></div>
  <div id="head2">
 <a href="http://www.guoxuemi.com/" target="_blank"><img src="http://www.guoxuedashi.com/img/guoxuemi.gif" width="95" height="62" border="0" style="margin-left:2px;" title="国学迷"></a>
  

  </div>
</div>
  <div class="clear"></div>
  <div class="head_nav">
  <p><a href="/">首页</a> | <a href="/ShuKu/">国学书库</a> | <a href="/guji/">影印古籍</a> | <a href="/shici/">诗词宝典</a> | <a   href="/SiKuQuanShu/gxjx.php">精选</a> <b>|</b> <a href="/zidian/">汉语字典</a> | <a href="/hydcd/">汉语词典</a> | <a href="http://www.guoxuedashi.com/zidian/bujian/"><font  color="#CC0066">部件查字</font></a> | <a href="http://www.sfds.cn/"><font  color="#CC0066">书法大师</font></a> | <a href="/jgwhj/">甲骨文</a> <b>|</b> <a href="/b/4/"><font  color="#CC0066">解密</font></a> | <a href="/renwu/">历史人物</a> | <a href="/diangu/">历史典故</a> | <a href="/xingshi/">姓氏</a> | <a href="/minzu/">民族</a> <b>|</b> <a href="/mz/"><font  color="#CC0066">世界名著</font></a> | <a href="/download/">软件下载</a>
</p>
<p><a href="/b/"><font  color="#CC0066">历史</font></a> | <a href="http://skqs.guoxuedashi.com/" target="_blank">四库全书</a> |  <a href="http://www.guoxuedashi.com/search/" target="_blank"><font  color="#CC0066">全文检索</font></a> | <a href="http://www.guoxuedashi.com/shumu/">古籍书目</a> | <a   href="/24shi/">正史</a> <b>|</b> <a href="/chengyu/">成语词典</a> | <a href="/kangxi/" title="康熙字典">康熙字典</a> | <a href="/ShuoWenJieZi/">说文解字</a> | <a href="/zixing/yanbian/">字形演变</a> | <a href="/yzjwjc/">金 文</a> <b>|</b>  <a href="/shijian/nian-hao/">年号</a> | <a href="/diming/">历史地名</a> | <a href="/shijian/">历史事件</a> | <a href="/guanzhi/">官职</a> | <a href="/lishi/">知识</a> <b>|</b> <a href="/zhongyi/">中医中药</a> | <a href="http://www.guoxuedashi.com/forum/">留言反馈</a>
</p>
  </div>
</div>
<!-- 头部导航END --> 
<!-- 内容区开始 --> 
<div class="w1180 clearfix">
  <div class="info l">
   
<div class="clearfix" style="background:#f5faff;">
<script src='http://www.guoxuedashi.com/img/headersou.js'></script>

</div>
  <div class="info_tree"><a href="http://www.guoxuedashi.com">首页</a> > <a href="/SiKuQuanShu/fanti/">四库全书</a>
 > <h1>资治通鉴</h1> <!--         下载:【右键另存为】即可 --></div>
  <div class="info_content zj clearfix">
  
<div class="info_txt clearfix" id="show">
<center style="font-size:24px;">169-資治通鑑卷一百六十八</center>
    資治通鑑卷一百六十八 宋 司馬光 撰<br />
<br />
  胡三省 音註<br />
<br />
  陳紀二【起上章執徐盡玄黓敦牂凡三年】<br />
<br />
  世祖文皇帝上【諱蒨字子華高祖兄始興王道譚之長子】<br />
<br />
  天嘉元年春正月癸丑朔大赦改元 齊大赦改元乾明 辛酉上祀南郊 齊高陽王湜以滑稽便辟有寵於顯祖【湜常職翻史記索隱曰滑謂亂也稽同也以言辯捷之人言非若是言是若非能亂同異也楚辭將突梯滑稽如脂如韋崔浩云滑音骨稽流酒器也轉注吐酒終日不已言出口成章詞不空竭若滑稽之吐酒故楊雄酒賦云鴟夷滑稽腹如大壺盡日盛酒人復藉沽是也又姚察云滑稽猶俳諧也滑讀如字稽音計以其言語滑利智計捷出故云滑稽也尹焞曰便辟足恭也朱元晦曰便辟謂習於威儀而不直便毘連翻辟匹亦翻】常在左右執杖以撻諸王太皇太后深銜之及顯祖殂湜有罪太皇太后杖之百餘癸亥卒 辛未上祀北郊 齊主自晉陽還至鄴 二月乙未高州刺史紀機自軍所逃還宣城【軍所侯瑱軍前也】據郡應王琳涇令賀當遷討平之王琳至柵口【柵口在濡須口之東水導巢湖今謂之柵江口宋白曰廬州東南至柵口三百九十里今謂之新婦口】侯瑱督諸軍出屯蕪湖【瑱他甸翻又音鎮】相持百餘日東關春水稍長【長知两翻】舟艦得通琳引合肥漅湖之衆舳艫相次而下【艦戶黯翻漅音巢又子小翻舳艫音逐盧】軍勢甚盛瑱進軍虎檻洲琳亦出船列於江西隔洲而泊明日合戰琳軍少却退保西㟁【少詩沼翻】及夕東北風大起吹其舟艦並壞沒於沙中浪大不得還浦及旦風静琳入浦治船【治直之翻】瑱等亦引軍退入蕪湖周人聞琳東下遣都督荆襄等五十二州諸軍事荆州刺史史寜將兵數萬乘虛襲郢州孫瑒嬰城自守【去年王琳東下留孫瑒守郢州將即亮翻瑒雉杏翻又音暢】琳聞之恐其衆潰乃帥舟師東下去蕪湖十里而泊擊柝聞於陳軍【帥讀曰率柝他各翻聞音問】齊儀同三司劉伯球將兵萬餘人助琳水戰行臺慕容恃德之子子會將鐵騎二千屯蕪湖西㟁為之聲勢【騎奇寄翻】丙申瑱令軍中晨炊蓐食以待之時西南風急琳自謂得天助引兵直趣建康【趣七喻翻】瑱等徐出蕪湖躡其後西南風翻為瑱用琳擲火炬以燒陳船皆反燒其船【逆風而用火攻此王琳所以敗也】瑱發拍以擊琳艦【戰船前後置拍竿以拍敵船】又以牛皮冒蒙衝小船以觸其艦并鎔鐵灑之琳軍大敗軍士溺死者什二三【溺奴狄翻】餘皆棄船登㟁為陳軍所殺殆盡齊步軍在西㟁者自相蹂踐並陷於蘆荻泥淖中【蹂人九翻踐慈演翻淖奴敎翻濘泥也】騎皆棄馬脫走得免者什二三擒劉伯球慕容子會斬獲萬計盡收梁齊軍資器械琳乘舴艋冒陳走【舴陟格翻艋音猛陳讀曰陣】至湓城欲收合離散衆無附者乃與妻妾左右十餘人奔齊先是琳使侍中袁泌【先悉薦翻泌毘必翻又兵媚翻 考異曰北齊書作長史袁泌今從陳書】御史中丞劉仲威侍衛永嘉王莊及敗左右皆散泌以輕舟送莊達於齊境拜辭而還遂來降【還從宣翻又如字降戶江翻下同】仲威奉莊奔齊泌昂之子也【袁昂著名節於齊梁之間】樊猛及其兄毅帥部曲來降【自此江南皆為陳有矣帥讀曰率】 齊葬文宣皇帝於武寜陵廟號高祖後改曰顯祖 戊戌詔衣冠士族將帥戰兵陷在王琳黨中者皆赦之隨材銓叙【將即亮翻帥所類翻】 己亥齊以常山王演為太師錄尚書事【此鄴省尚書也】以長廣王湛為大司馬并省錄尚書事【晉陽并州故曰并省并必經翻】以尚書左僕射平秦王歸彥為司空趙郡王叡為尚書左僕射詔諸元良口配没入官及賜人者並縱遣【去年齊顯祖夷諸元没其家口今縱遣良口奴婢仍不縱也】 乙巳以太尉侯瑱都督湘巴等五州諸軍事鎮湓城 齊顯祖之喪常山王演居禁中護喪事婁太后欲立之而不果太子即位乃就朝列【朝直遙翻】以天子諒隂詔演居東館【東館盖在鄴宮昭陽殿東】欲奏之事皆先咨决楊愔等以演與長廣王湛位地親逼恐不利於嗣主心忌之居頃之演出歸第 【考異曰北齊書孝昭紀云除太傅錄尚書朝政皆决於帝月餘乃居藩邸今從楊愔傳 藩邸常山第也】自是詔敕多不關預或謂演曰鷙鳥離巢必有探卵之患【離力智翻探吐南翻】今日王何宜屢出中山太守陽休之詣演演不見休之謂王友王晞曰昔周公朝讀百篇書夕見七十士猶恐不足錄王何所嫌疑乃爾拒絶賓客【演以常山王錄尚書事故稱為錄王】先是顯祖之世羣臣人不自保【先悉薦翻】及濟南王立【濟子禮翻】演謂王晞曰一人垂拱吾曹亦保優閒因言朝廷寛仁真守文良主晞曰先帝時東宫委一胡人傅之今春秋尚富驟覽萬機殿下宜朝夕先後【先悉薦翻後戶遘翻】親承音旨而使他姓出納詔命大權必有所歸殿下雖欲守藩其可得耶借令得遂冲退自審家祚得保靈長乎【家祚猶云國祚也演以叔父之親與國同休等戚故言家祚王晞勸常山傾大宗其來久矣】演默然久之曰何以處我【處昌呂翻】晞曰周公抱成王攝政七年然後復子明辟惟殿下慮之演曰我何敢自比周公晞曰殿下今日地望欲不為周公得邪演不應顯祖常遣胡人康虎兒保護太子故晞言及之齊主將發晉陽【發晉陽者嗣位而詣鄴】時議謂常山王必當留守根本之地【高歡建大丞相府於晉陽文宣席之以移魏鼎宿將勁兵咸在焉故以為根本之地】執政欲使常山王從帝之鄴留長廣王鎮晉陽既而又疑之乃敕二王俱從至鄴外朝聞之莫不駭愕【朝直遙翻】又敕以王晞為并州長史演既行晞出郊送之演恐有覘察【覘丑亷翻又丑艶翻】命晞還城執晞手曰努力自慎躍馬而去平秦王歸彥總知禁衛楊愔宣敕留從駕五千兵於西中【晉陽在鄴西故謂之西中從才用翻】隂備非常至鄴數日歸彥乃知之由是怨愔領軍大將軍可朱渾天和道元之子也【可朱渾道元自隴右歸高歡考異曰典畧云道元弟今從北齊書】尚帝姑東平公主【齊主之姑則高歡之女也】每曰若不誅二王少主無自安之理【少詩照翻】燕子獻謀處太皇太后於北宫【燕因肩翻鄴城有北宫處昌呂翻】使歸政皇太后又自天保八年已來爵賞多濫楊愔欲加澄汰【以水為諭也澄者去泥滓汰者去砂石】乃先自表解開府及開封王諸叨竊恩榮者皆從黜免由是嬖寵失職之徒盡歸心二叔【演湛皆齊主之叔】平秦王歸彥初與楊燕同心既而中變盡以疎忌之迹告二王侍中宋欽道弁之孫也【宋弁見任於魏孝文帝】顯祖使在東宫敎太子以吏事欽道面奏帝稱二叔威權既重宜速去之【去羌呂翻】帝不許曰可與令公共詳其事【楊愔時為尚書令故稱之為令公】愔等議出二王為刺史以帝慈仁恐不可所奏乃通啓皇太后具述安危宫人李昌儀【昌儀意亦内職而北史后妃傳無之盖太后女官之名】高仲密之妻也【高仲密因妻而外叛事見一百五十八卷梁武帝大同九年】李太后以其同姓甚相昵愛【昵尼質翻】以啓示之昌儀密啓太皇太后【史言謀及婦人之禍】愔等又議不可令二王俱出乃奏以長廣王湛鎮晉陽以常山王演錄尚書事二王既拜職乙巳於尚書省大會百僚愔等將赴之散騎常侍兼中書侍郎鄭頤止之【散悉亶翻騎奇寄翻】曰事未可量不宜輕脫【量音良】愔曰吾等至誠體國豈常山拜職有不赴之理長廣王湛旦伏家僮數十人於錄尚書後室【錄尚書後室錄尚書者宴息之所】仍與席上勲貴賀拔仁斛律金等數人相知約曰行酒至愔等我各勸雙盃彼必致辭我一曰執酒二曰執酒三曰何不執爾輩即執之及宴如之愔大言曰諸王反逆欲殺忠良邪尊天子削諸侯赤心奉國何罪之有常山王演欲緩之湛曰不可於是拳杖亂毆【毆烏口翻】愔及天和欽道皆頭面血流各十人持之燕子獻多力頭又少髪【少詩沼翻】狼狽排衆走出門斛律光逐而擒之子獻歎曰丈夫為計遲遂至於此使太子太保薛孤延等執頤於尚藥局【五代志尚藥局總知御藥事屬門下省後齊制也】頤曰不用智者言至此豈非命也二王與平秦王歸彥賀拔仁斛律金擁愔等唐突入雲龍門【唐突廣韻作傏不遜也今時謂千乘輿者為唐突】見都督叱利騷招之不進使騎殺之【叱利虜複姓】開府儀同三司成休寜抽刃呵演演使歸彥諭之休寜厲聲不從歸彥久為領軍素為軍士所服皆弛仗休寜方歎息而罷演入至昭陽殿湛及歸彥在朱華門外【後齊禁中有朱華閤門下省領左右局領左右二人掌知朱華閤内諸事宣傳已下白衣齋子已上皆主之又有左右直長四人朱華閤以外左右衛將軍各一人主之各武衛將軍二人貳之御仗屬官直盪屬官直衛屬官直突屬官直閤屬官皆屬焉分為左右廂】帝與太皇太后並出【太皇太后之下疑當更有皇太后三字】太皇太后坐殿上皇太后及帝側立演以塼叩頭【塼朱緣翻範土為塼陶而成之】進言曰臣與陛下骨肉至親楊遵彥等欲獨擅朝權【楊愔字遵彥朝直遙翻】威福自已自王公已下皆重足屏氣【重直龍翻屛必郢翻】共相唇齒以成亂階若不早圖必為宗社之害臣與湛為國事重【為於偽翻】賀拔仁斛律金惜獻武皇帝之業共執遵彥等入宫未敢刑戮專輒之罪誠當萬死時庭中及兩廡衛士二千餘人皆被甲待詔【被皮義翻】武衛娥永樂武力絶倫素為顯祖所厚叩刀仰視【娥姓也後魏有娥清樂音洛叩刀者拔刀離鞘纔寸許 考異曰北齊書作領軍劉桃枝今從北史】帝不睨之【睨研計翻邪視也】帝素吃訥【吃音訖吃訥之由見上卷武帝永定三年】倉猝不知所言太皇太后令却仗不退又厲聲曰奴輩即今頭落乃退【衛士被甲者皆退當未退之時使宇文覺處之常山長廣身首分矣】永樂内刀而泣太皇太后因問楊郎何在賀拔仁曰一眼已出太皇太后愴然曰楊郎何所能為留使豈不佳邪【楊愔主壻故謂之楊郎留使留之以任使愴初亮翻】乃讓帝曰此等懷逆欲殺我二子【演湛皆太皇太后之子】次將及我爾何為縱之帝猶不能言太皇太后怒且悲曰豈可使我母子受漢老嫗斟酌【婁太后鮮卑也李太后華族也故云然嫗威遇翻】太后拜謝太皇太后又為太后誓言演無異志但欲去逼而已【逼謂楊愔等以疏踰戚故欲去之為於偽翻下敢為同去羌呂翻】演叩頭不止太后謂帝何不安慰爾叔帝乃曰天子亦不敢為叔惜况此漢輩但匄兒命兒自下殿去此屬任叔父處分【處昌呂翻分扶問翻】遂皆斬之【楊愔受託孤之寄不能尊主庇身者鮮卑之勢素盛華人不足以制之也】長廣王湛以鄭頤昔嘗讒已【楊愔傳云湛以頤昔讒已作詔書】先拔其舌截其手而殺之演令平秦王歸彥引侍衛之士向華林園【鄴都華林園魏武之舊也】以京畿軍士入守門閤【高歡遷魏主於鄴而身居晉陽以其子為京畿大都督防遏内外故有京畿軍士】斬娥永樂於園太皇太后臨愔喪哭曰楊郎忠而獲罪【婁后此言出於人心是非之真也】以御金為之一眼【御金御府之金也】親内之曰以表我意演亦悔殺之於是下詔罪狀愔等且曰罪止一身家屬不問頃之復簿錄五家【楊愔可朱渾天和燕子獻宋欽道鄭頤凡五家復扶又翻】王晞固諫乃各没一房孩幼盡死兄弟皆除名以中書令趙彥深代楊愔總機務鴻臚少卿陽休之私謂人曰將涉千里殺騏驎而策蹇驢可悲之甚也【馬黑脊曰騏驎毛晃曰馬青驪色曰騏驎東方朔傳騏驎騄駬蜚鳴驊騮天下良馬也驎力珍翻】戊申演為大丞相都督中外諸軍錄尚書事【自後魏敬宗以爾朱榮為大丞相後高歡復為之位絶羣后威權震主】湛為太傅京畿大都督段韶為大將軍平陽王淹為太尉平秦王歸彥為司徒彭城王浟為尚書令【浟夷周翻】 江陵之陷也【見一百六十五卷梁元帝承聖三年】長城世子昌【武帝封長城公昌為世子】及中書侍郎頊皆没於長安高祖即位屢請之於周周人許而不遣高祖殂周人乃遣昌還【高祖存而不遣高祖殂而遣還欲以間陳使兄弟爭國也】以王琳之難居於安陸【王琳據中流昌還建康路梗故居安陸難乃旦翻】琳敗昌發安陸將濟江致書於上辭甚不遜上不懌召侯安都從容謂曰【從千容翻】太子將至須别求一藩為歸老之地安都曰自古豈有被代天子【被皮義翻】臣愚不敢奉詔因請自迎昌於是羣臣上表請加昌爵命庚戌以昌為驃騎將軍湘州牧【驃匹妙翻騎奇寄翻】封衡陽王 齊大丞相演如晉陽既至謂王晞曰【演從少帝還鄴晞為并州長史留晉陽】不用卿言幾至傾覆今君側雖清終當何以處我【幾居依翻處昌呂翻下出處同】晞曰殿下往時位地猶可以名教出處今日事勢遂關天時非復人理所及【晞勸演簒史䆒言之復扶又翻】演奏趙郡王叡為長史王晞為司馬三月甲寅詔軍國之政皆申晉陽稟大丞相䂓算【詔演志也】周軍初至郢州助防張世貴舉外城以應之所失軍民三千餘口周人起土山長梯晝夜攻之因風縱火燒其内城南面五十餘樓孫瑒兵不滿千人身自撫循行酒賦食士卒皆為之死戰周人不能克【史言千人一心雖大敵不能克郢人之死戰不下者畏江陵之俘戮也為於偽翻】乃授瑒柱國郢州刺史封萬戶郡公瑒偽許以緩之而潛脩戰守之備一朝而具乃復拒守【復扶又翻】既而周人聞王琳敗陳兵將至乃解圍去瑒集將佐謂之曰吾與王公同奬梁室勤亦至矣今時事如此豈非天乎遂遣使奉表舉中流之地來降【將即亮翻使疏吏翻降戶江翻】王琳之東下也帝徵南川兵江州刺史周廸高州刺史黄法帥舟師將赴之熊曇朗據城列艦塞其中路【熊曇朗時據豫章巨俱翻帥讀曰率曇徒含翻艦戶黯翻塞悉則翻】廸等與周敷共圍之琳敗曇朗部衆離心廸攻拔其城虜男女萬餘口曇朗走入村中村民斬之丁巳傳首建康盡滅其族齊軍先守魯山戊午棄城走詔南豫州刺史程靈洗守之 甲子置沅州武州【梁置武州於武陵帝分荆州之義陽天門郡郢州之武陵郡置武州督沅州領武陵太守治武陵郡其都尉所部六縣為沅州别置通寜郡以刺史領太守治都尉城省舊都尉沅音元】以右衛將軍吳明徹為武州刺史以孫瑒為湘州刺史瑒懷不自安固請入朝【史言孫瑒能自全朝直遙翻】徵為中領軍未拜除吳郡太守 壬申齊封世宗之子孝珩為廣寜王【珩音行】長恭為蘭陵王 甲戌衡陽獻王昌入境詔主書舍人緣道迎候【主書及中書舍人皆當時要近之職也】丙子濟江中流殞之使以溺告【溺奴狄翻】侯安都以功進爵清遠公【以殺昌之功也五代志南海郡翁源縣陳置清遠郡】初高祖遣滎陽毛喜從安成王頊詣江陵梁世祖以喜為侍郎沒於長安與昌俱還因進和親之策上乃使侍中周弘正通好於周【好呼到翻】 夏四月丁亥立皇子伯信為衡陽王奉獻王祀【昌諡曰獻】 周世宗明敏有識量晉公護憚之使膳部中大夫李安寘毒於糖䭔而進之【周禮有膳夫唐六典紀前世官制沿革以後周之典庖中士為唐太官署令之職肴藏中士為珍羞署令之職掌醢中士為掌醢署令之職獨不言膳部中大夫以類推之則後周之膳部中大夫唐光祿卿之職也杜佑通典後周膳部中大夫屬冢宰六命又有膳部下大夫五命䭔都囘翻丸餅也江陵未敗時梁將陸法和有道術先具大䭔薄餅及江陵陷梁人入魏果見䭔餅盖北食也今城市間元宵所賣焦䭔即其物但較小耳糖出南方煎蔗為之絶甘】帝頗覺之庚子大漸口授遺詔五百餘言且曰朕子年幼未堪當國魯公朕之介弟【杜預曰介大也】寛仁大度海内共聞能弘我周家必此子也【弘大也世宗之知武帝史所謂明敏有識孰大於此】辛丑殂【年二十七】魯公幼有器質特為世宗所親愛朝廷大事多與之參議性深沈有遠識【沈持林翻】非因顧問終不輒言世宗每歎曰夫人不言言必有中【引論語孔子之言夫音扶中竹仲翻】壬寅魯公即皇帝位【諱邕字彌羅突安定公泰之第四子也】大赦 五月壬子齊以開府儀同三司劉洪徽為尚書右僕射 侯安都父文捍為始興内史卒官【卒官卒於官也卒子恤翻】上迎其母還建康母固求停鄉里乙卯為置東衡州【梁先已置東衡州於始興盖中廢而今復置也為於偽翻】以安都從弟曉為刺史【從才用翻】安都子袐纔九歲上以為始興内史並令在鄉侍養【以安都能定策以安國家故寵之養余亮翻】 六月壬辰詔葬梁元帝於江寜【梁敬帝太平二年周人歸元帝之柩於王琳琳敗陳人乃得而葬之】車旗禮章悉用梁典 齊人收永安上黨二王遺骨葬之【齊二王死見上卷武帝永定二年】敕上黨王妃李氏還第馮文洛尚以故意脩飾詣之妃盛列左右立文洛於階下數之曰遭難流離【數所具翻難乃旦翻】以至大辱志操寡薄不能自盡【言不能自殺也】幸蒙恩詔得反藩闈【藩闈言藩王之閨闈也】汝何物奴猶欲見侮杖之一百血流灑地 秋七月丙辰封皇子伯山為鄱陽王 齊丞相演以王晞儒緩恐不允武將之意【將即亮翻】每夜載入晝則不與語嘗進晞密室謂曰比王侯諸貴每見敦廹【比毘至翻】言我違天不祥恐當或有變起吾欲以法繩之何如晞曰朝廷比者疎遠親戚殿下倉猝所行非復人臣之事芒刺在背【用漢霍光事遠於願翻復扶又翻】上下相疑何由可久殿下謙退粃糠神器實恐違上元之意【上元天也】墜先帝之基【先帝謂高歡】演曰卿何敢發此言須致卿於法晞曰天時人事皆無異謀是以敢冒犯斧钺抑亦神明所贊耳演曰拯難匡時【難乃旦翻】方俟聖哲吾何敢私議幸勿多言丞相從事中郎陸杳將出使【使疏吏翻】握晞手使之勸進晞以杳言告演演曰若内外咸有此意趙彥深朝夕左右何故初無一言【史言演非不欲簒特覘衆心】晞乃以事隙密問彥深【事隙公事之隙少暇之時也】彥深曰我比亦驚此聲論【聲論謂輿論皆歸演聲滿朝野也比毘至翻】每欲陳聞則口噤心悸【噤其禁翻悸其季翻】弟既發端吾亦當昧死一披肝膽因共勸演演遂言於太皇太后趙道德曰相王不效周公輔成王【演為丞相故呼之為相王演於齊主居親親之地猶周公之於成王而不能以周公自任故趙道德責之】而欲骨肉相奪不畏後世謂之簒邪太皇太后曰道德之言是也未幾【幾居豈翻】演又啓云天下人心未定恐奄忽變生須蚤定名位太皇太后乃從之八月壬午太皇太后下令廢齊主為濟南王出居别宫【濟子禮翻】以常山王演入纂大統且戒之曰勿令濟南有他也【為演殺濟南王太后怒張本】肅宗即皇帝位於晉陽【諱演字延安勃海王歡第六子文宣帝之母弟也】大赦改元皇建太皇太后還稱皇太后皇太后稱文宣皇后宫曰昭信乙酉詔紹封功臣禮賜耆老延訪直言褒賞死事追贈名德帝謂王晞曰卿何為自同外客畧不可見自今假非局司但有所懷隨宜作一牒【毛晃曰牒書板小簡也】候少隙即徑進也【少隙言少有間隙也少詩沼翻】因敕與尚書陽休之鴻臚卿崔劼等三人【臚陵如翻劼邱八翻】每日職務罷並入東廊共舉録歷代禮樂職官及田市徵稅或不便於時而相承施用或自古為利而於今廢墜或道德高儁久在沈淪【沈持林翻下沈敏同】或巧言眩俗妖邪害政者悉令詳思以漸條奏朝晡給御食畢景聽還【景日景日入而後聽還私舍故云畢景聽還妖於驕翻晡奔謨翻】帝識度沈敏少居臺閣明習吏事即位尤自勤勵大革顯祖之弊時人服其明而譏其細【人君而親小事為細所謂元首叢脞也少詩照翻】嘗問舍人裴澤在外議論得失澤率爾對曰陛下聰明至公自可遠侔古昔而有識之士咸言傷細帝王之度頗為未宏帝笑曰誠如卿言朕初臨萬機慮不周悉故致爾耳【顔之推曰如是為爾而已為耳】此事安可久行恐後又嫌疎漏澤由是被寵遇庫狄顯安侍坐【被皮義翻坐徂卧翻】帝曰顯安我姑之子【庫狄顯安父干娶勃海王歡之妹樂陵長公主】今序家人之禮除君臣之敬可言我之不逮顯安曰陛下多妄言帝曰何故對曰陛下昔見文宣以馬鞭撻人常以為非今自行之非妄言邪帝握其手謝之又使直言對曰陛下太細天子乃更似吏帝曰朕甚知之然無法日久將整之以至無為耳又問王晞晞曰顯安言是也顯安干之子也羣臣進言帝皆從容受納【從千容翻】性至孝太后不豫帝行不能正履容色貶悴【悴秦醉翻】衣不解帶殆將四旬太后疾小增即寢伏閤外食飲藥物皆手親之太后嘗心痛不自堪帝立侍帷前以爪搯掌代痛【搯苦洽翻】血流出袖友愛諸弟無君臣之隔戊子以長廣王湛為右丞相平陽王淹為太傅彭城王浟為大司馬【浟夷周翻】 周軍司馬賀若敦【唐六典曰周官大司馬屬官有軍司馬下大夫盖兵部郎中之任也後周依周官其爵列中大夫也六命若人者翻】帥衆一萬奄至武陵【帥讀曰率】武州刺史吳明徹不能拒引軍還巴陵 江陵之陷也巴湘之地皆入於周周使梁人守之太尉侯瑱等將兵逼湘州賀若敦將步騎救之乘勝深入【按賀若敦傳屢戰破瑱乘勝深入】軍於湘川九月乙卯周將獨孤盛將水軍與敦俱進辛酉遣儀同三司徐度將兵會侯瑱於巴邱【將即亮翻】會秋水汎溢盛敦糧援斷絶分軍抄掠以供資費【抄楚交翻】敦恐瑱知其糧少乃於營内多為土聚覆之以米【少詩沼翻覆敷又翻此檀道濟量沙之故智也】召旁村人【營旁之村人也】陽有訪問隨即遣之瑱聞之良以為實敦又增脩營壘造廬舍為久留之計湘羅之間遂廢農業【梁置湘州於長沙置羅州於湘隂縣】瑱等無如之何先是土人亟乘輕船【先悉薦翻亟去吏翻數也】載米粟雞鴨以餉瑱軍敦患之乃偽為土人裝船伏甲士於中瑱軍人望見謂餉船之至逆來争取敦甲士出而擒之【唐裴行儉詐為糧車以破突厥亦用此策】又敦軍數有叛人乘馬投瑱者敦乃别取一馬牽以趣船令船中逆以鞭鞭之如是者再三馬畏船不上【數所角翻趣七喻翻上時掌翻】然後伏兵於江㟁使人乘畏船馬以招瑱軍詐云投附瑱遣兵迎接競來牽馬馬既畏船不上伏兵發盡殺之此後實有饋餉及亡降者瑱猶謂之詐並拒擊之冬十月癸巳瑱襲破獨孤盛於楊葉洲【據姚思亷陳書楊葉洲在西江口西江謂湘江也】盛收兵登㟁築城自保丁酉詔司空侯安都帥衆會瑱南討【帥讀曰率】十一月辛亥齊主立妃元氏為皇后世子百年為太子百年時纔五歲齊主徵前開府長史盧叔虎為中庶子【太子中庶子職如侍中後齊門下坊之長也】叔虎柔之從叔也【從才用翻】帝問時務於叔虎叔虎請伐周曰我彊彼弱我富彼貧其勢相懸然干戈不息未能并吞者此失於不用彊富也【以當時東西二國觀之齊若富彊而其根本實撥周若貧弱而其根本實牢若齊孝昭欲用其富彊周固有以待之】輕兵野戰勝負難必是胡騎之法非萬全之術也宜立重鎮於平陽與彼蒲州相對【魏神䴥元年置雍州於河東延和元年改曰秦州太和中罷魏既分為東西東魏天平初復置秦州於河東沙苑敗後河東之地入於西魏後周因蒲坂舊名而置蒲州】深溝高壘運糧積甲彼閉關不出則稍蠶食其河東之地日使窮蹙若彼出兵非十萬以上不足為我敵所損糧食【損當作資】咸出關中我軍士年别一代【一年一更戍也】穀食豐饒彼來求戰我則不應彼若退去我乘其弊自長安以西民疏城遠敵兵來往實自艱難與我相持農業且廢不過三年彼自破矣帝深善之 齊主自將擊庫莫奚【將即亮翻下同】至天池庫莫奚出長城北遁【此文宣帝所築長城也】齊主分兵追擊獲牛羊七萬而還【還從宣翻又如字】 十二月乙未詔自今孟春訖於夏首大辟事已款者【已款謂囚已款服也今人謂獄辭為獄款辟毘至翻】宜且申停【及秋冬乃行刑也】 己亥周巴陵城主尉遲憲降【尉紆勿翻降戶江翻】遣巴州刺史侯安鼎守之庚子獨孤盛將餘衆自楊葉洲潛遁【賀若敦之勢愈孤矣】 丙午齊主還晉陽齊主斬人於前問王晞曰是人應死不【不讀曰否齊主以文宣殺人多非其罪自謂誅當其罪故以問晞】晞曰應死但恨死不得其地耳臣聞刑人於市與衆棄之【記王制之言】殿庭非行戮之所帝改容謝曰自今當為王公改之【為於偽翻】帝欲以晞為侍郎【按北史王晞傳侍郎當作侍中】苦辭不受或勸晞勿自踈晞曰我少年以來閱要人多矣【少詩照翻要人謂位居勢要者】得志少時鮮不顛覆【少時言不多時也少始紹翻鮮息翦翻】且吾性實疎緩不堪時務人主恩私何由可保萬一披猖求退無地非不好作要官但思之爛熟耳【好呼到翻】 初齊顯祖之末穀糴踊貴濟南王即位【濟子禮翻】尚書左丞蘇珍芝建議脩石鼈等屯自是淮南軍防足食【杜佑曰石鼈在楚州安宜縣西八十里鄧艾築城於此作白水塘北接連洪澤屯田一萬三千頃安宜唐寶應元年改為寶應縣】肅宗即位平州刺史嵇曄建議開督亢陂置屯田歲收稻粟數十萬石北境周贍【督亢陂在唐涿州新城縣界燕荆軻獻圖於秦即此地亢音剛】又於河内置懷義等屯以給河南之費【齊分河内汲郡為義州置懷義等屯】由是稍止轉輸之勞【此是五代志序齊濟南王至孝昭時軍餉鑑取之附見於此】<br />
<br />
  二年春正月戊申周改元保定以大冢宰護為都督中外諸軍事令五府總於天官事無巨細皆先斷後聞【五府地官春官夏官秋官冬官府也史言宇文護之權愈重斷丁亂翻】 庚戌大赦 周主祀圜丘 辛亥齊主祀圜丘壬子禘於太廟 周主祀方丘甲寅祀感生帝於南郊【用鄭玄之說祀感生帝靈威仰於南郊以祈穀】乙卯祭太社 齊主使王琳出合肥召募傖楚更圖進取【傖助庚翻】合州刺史裴景徽 【考異曰北齊書作景暉今從陳書】琳兄珉之壻也請以私屬為鄉導【鄉讀曰嚮】齊主使琳與行臺左丞盧潛將兵赴之琳沈吟不决景徽恐事泄挺身奔齊【按梁置合州於合肥侯景之亂已入於齊齊之境土南盡歷陽陳盖僑置合州於江濱以景徽為刺史沈持林翻】齊主以琳為驃騎大將軍開府儀同三司揚州刺史鎮夀陽己巳周主享太廟班太祖所述六官之法【宇文泰廟號太祖泰】<br />
<br />
  【之相魏也建六官述周禮六典以為六官之法】 辛未周湘州城主殷亮降【降戶江翻下同】湘州平侯瑱與賀若敦相持日久瑱不能制乃借船送敦等渡江【按賀若敦傳借船之上有求字】敦慮其詐不許報云湘州我地為爾侵逼必須我歸可去我百里之外瑱留船江㟁引兵去之敦乃自拔北歸軍士病死者什五六武陵天門南平義陽河東宜都郡悉平【五代志澧陽郡孱陵縣舊置南平郡安鄉縣舊置義陽郡南郡松滋縣舊置河東郡宋白曰澧陽郡安鄉縣本漢孱陵縣地後漢為漢夀縣地晉曾立義陽郡】晉公護以敦失地無功除名為民 二月甲午周主朝日於東郊【三代之禮春朝朝日秋暮夕月周人慕古舉行其禮朝直遙翻】 周人以小司徒韋孝寛嘗立勲於玉壁【事見一百五十九卷梁武帝中大同元年後周之制小司徒六命上大夫也】乃置勲州於玉壁以孝寛為刺史孝寛有恩信善用間諜【間古莧翻】或齊人受孝寛金貨遙通書疏故齊之動静周人皆先知之有主帥許盆以所戌城降齊孝寛遣諜取之俄斬首而還【帥所類翻諜徒協翻】離石以南生胡數為抄掠【五代志離石郡後齊置西汾州生胡即稽胡之不附屬周者數所角翻抄楚交翻】而居於齊境不可誅討孝寛欲築城於險要以制之乃發河西役徒十萬甲士百人【河西龍門河之西也】遣開府儀同三司姚岳監築之岳以兵少懼不敢前【監工銜翻少詩沼翻】孝寛曰計此城十日可畢城距晉州四百餘里吾一日創手二日敵境始知設使晉州徵兵三日方集謀議之間自稽二日計其軍行二日不到我之城隍足得辦矣乃令築之齊人果至境上疑有大軍停留不進其夜孝寛使汾水以南傍介山稷山諸村縱火【傍蒲浪翻唐志蒲州萬泉縣有介山介子推隱處稷山縣有稷山】齊人以為軍營收兵自固岳卒城而還【卒子恤翻】 三月乙卯太尉零陵壯肅公侯瑱卒 丙寅周改八丁兵為十二丁兵率歲一月而役【八丁兵者凡境内民丁分為八番遞上就役十二丁兵者分為十二番月上就役周而復始】 夏四月丙子朔日有食之 周以少傅尉遲綱為大司空【尉紆勿翻】 丙午周封愍帝子康為紀國公皇子贇為魯公贇李后之子也【贇於倫翻】六月乙酉周使御正殷不害來聘【周書申徽傳曰御正任專絲綸盖中】<br />
<br />
  【書舍人之職也北史盧辯傳武成元年增置御正四人位上大夫考之唐六典則曰後周依周官春官府置内史中大夫掌王言盖比中書監令之任後又增為上大夫小史下大夫比中書侍郎之任小史上士比中書舍人之任然則為御正者亦代言之職在帝左右又親密於中書杜佑通典御正屬天官府】 秋七月周更鑄錢【更工衡翻】文曰布泉一當五與五銖並行 己酉周追封皇伯父顥為邵國公以晉公護之子會為嗣顥弟連為杞國公以章武公導之子亮為嗣連弟洛生為莒國公以護之子至為嗣追封太祖之子武邑公震為宋公以世宗之子實為嗣【顥與衛可孤戰殁有子什肥導護什肥與其叔連皆為高歡所殺無後故以會亮嗣之洛生為爾朱榮所殺震早卒皆無後故亦立嗣】 齊主之誅楊燕也【燕因肩翻】許以長廣王湛為太弟既而立太子百年湛心不平帝在晉陽湛居守於鄴【楊燕謂楊愔燕子獻守手又翻】散騎常侍高元海高祖之從孫也【高歡廟號高祖元海父思宗歡之從子散悉亶翻騎奇寄翻從用翻】才留典機密帝以領軍代人庫狄伏連為幽州刺史斛律光之弟羨為領軍以分湛權湛留伏連不聽羨視事【齊主以伏連代羨為幽州以羨代伏連為領軍以分鄴下之權湛知其故乃留伏連不使之幽州而羨至又不聽其視領軍府事】先是濟南閔悼王常在鄴【濟南王殷諡閔悼先悉薦翻濟子禮翻】望氣者言鄴中有天子氣平秦王歸彥恐濟南復立為己不利【齊主藉歸彥握兵以殺楊燕楊燕死而濟南廢矣故恐其復立為己不利復扶又翻】勸帝除之帝乃使歸彥至鄴徵濟南王如晉陽湛内不自安問計於高元海元海曰皇太后萬福至尊孝友異常殿下不須異慮湛曰此豈我推誠之意邪元海乞還省一夜思之湛即留元海於後堂元海達旦不眠唯遶牀徐步夜漏未盡湛遽出曰神算如何元海曰有三策恐不堪用耳請殿下如梁孝王故事從數騎入晉陽先見太后求哀【梁孝王事見十六卷漢景帝中二年】後見主上請去兵權【見賢遍翻去羌呂翻】以死為限不干朝政【朝直遙翻】必保太山之安此上策也不然當具表云威權太盛恐取謗衆口請青齊二州刺史沈靖自居【沈持林翻】必不招物議此中策也更問下策曰發言即恐族誅固逼之元海曰濟南世嫡主上假太后令而奪之今集文武示以徵濟南之敕執斛律豐樂【斛律羨字豐樂樂音洛】斬高歸彥尊立濟南號令天下以順討逆此萬世一時也湛大悅然性怯狐疑未能用使術士鄭道謙等卜之皆曰不利舉事静則吉有林慮令潘子密曉占候【林慮縣漢屬河内郡晉屬汲郡魏敬宗永安元年置林慮郡帶林慮縣慮讀如閭】潛謂湛曰宫車當晏駕殿下為天下主湛拘之於内以候之又令巫覡卜之【覡刑狄翻】多云不須舉兵自有大慶湛乃奉詔令數百騎送濟南王至晉陽九月帝使人酖之濟南王不從乃扼殺之帝尋亦悔之 冬十月甲戌朔日有食之 丙子齊以彭城王浟為太保長樂王尉粲為太尉【樂音洛下同】 齊肅宗出畋有兎驚馬墜地絶肋婁太后視疾問濟南所在者三齊主不對太后怒曰殺之邪不用吾言死其宜矣遂去不顧十一月甲辰詔以嗣子冲眇可遣尚書右僕射趙郡王叡諭旨徵長廣王湛統兹大寶又與湛書曰百年無罪汝可以樂處置之勿效前人也【樂音洛楚靈王乾谿之役楚人殺其諸子王聞之自投於車下曰余殺人子多矣能無及此乎齊肅宗殺其兄之子臨終乃戒其弟勿殺已之子良可憫笑】是日殂於晉陽宫【年二十七】臨終言恨不見太后山陵<br />
<br />
  顔之推論曰孝昭天性至孝而不知忌諱乃至於此良由不學之所為也<br />
<br />
  趙郡王叡先使黄門侍郎王松年馳至鄴宣肅宗遺命湛猶疑其詐使所親先詣殯所發而視之使者復命【使疏吏翻下同】湛喜馳赴晉陽使河南王孝瑜先入宫改易禁衛癸丑世祖即皇帝位於南宫【諱湛勃海王歡第九子孝昭帝之母弟南宫晉陽南宫也】大赦改元太寜 周人許歸安成王頊【頊吁玉翻】使司會上士杜杲來聘【周禮天官之屬有司會凡邦國都鄙官府之治及其財用在書契版圖者皆聽其會計以歲月日考其成鄭元曰會大計也司會主天下之大計計官之長若今尚書余按後周地官即唐戶部尚書之任司會當如唐之度支郎中而六典不言所以杜佑通典後周司會屬天官府有中大夫上士中士】上悅即遣使報之并賂以黔中地及魯山郡【周得黔中則全有巴蜀得魯山則全有漢沔故因其所欲而餌之】 齊以彭城王浟為太師錄尚書事平秦王歸彥為太傅尉粲為太保平陽王淹為太宰博陵王濟為太尉段韶為大司馬豐州刺史婁叡為司空【五代志上黨郡鄉縣後魏置南垣州尋改曰豐州】趙郡王叡為尚書令任城王湝為尚書左僕射【任音壬湝居諧翻】并州刺史斛律光為右僕射婁叡昭之兄子也【婁昭婁太后之弟叡昭兄拔之子】立太子百年為樂陵王 丁巳周主畋於岐陽十二月壬午還長安 太子中庶子餘姚虞荔御史中丞孔奐以國用不足奏立煮海鹽賦及榷酤之科【吳王濞煮海為鹽今淮鹽也至此則東南瀕海煮鹽之地皆歸於筦榷矣酤音固荔力計翻榷古岳翻】詔從之 初高祖以帝女豐安公主妻留異之子貞臣徵異為南徐州刺史異遷延不就帝即位復以異為縉州刺史領東陽太守【自侯景之亂梁南郡王大連之敗留異跨據東陽陳興以為縉州刺史因縉雲山以名州妻七細翻復扶又翻下異復同】異屢遣其長史王澌入朝澌每言朝廷虛弱【澌斯義翻】異信之雖外示臣節恒懷兩端【恒戶登翻】與王琳自鄱陽信安嶺潛通使往來【今有嶺路自衢州經信州達於鄱陽使疏吏翻】琳敗上遣左衛將軍沈恪代異實以兵襲之異出軍下淮以拒恪恪與戰而敗退還錢塘異復上表遜謝時衆軍方事湘郢乃降詔書慰諭且羈縻之異知朝廷終將討已乃以兵戍下淮及建德以備江路【劉昫曰建德縣漢會稽富春縣地吳分置建德縣隋廢唐復置建德縣為睦州治所】丙午詔司空南徐州刺史侯安都討之<br />
<br />
  三年春正月乙亥齊主至鄴【自晉陽宫至鄴】辛巳祀南郊壬午享太廟丙戌立妃胡氏為皇后子緯為皇太子【緯于貴翻】后魏兖州刺史安定胡延之之女也戊子齊大赦己亥以馮翊王潤為尚書左僕射 周凉景公賀蘭祥卒【凉國公景諡也】 壬寅周人鑿河渠於蒲州龍首渠於同州【二渠皆以灌溉】丁未周以安成王頊為柱國大將軍遣杜杲送之南<br />
<br />
  歸 【考異曰典畧作杜杲今從周書】 辛亥上祀南郊以胡公配天【胡公始封於陳故郊祀之以配天】二月辛酉祀北郊 閏月丁未齊以太宰平陽王淹為青州刺史太傅平秦王歸彥為太宰冀州刺史歸彥為肅宗所厚【歸彥以殺楊燕之功為肅宗所厚】恃勢驕盈陵侮貴戚世祖即位侍中開府儀同三司高元海御史中丞畢義雲黄門郎高乾和數言其短【數所角翻】且云歸彥威權震主必為禍亂帝亦㝷其反覆之跡漸忌之【歸彥始為文宣所親任其後背楊愔附孝昭以成濟南之禍又為孝昭所委信孝昭既殂又迎武成以貪天之功故武成跡其反覆而忌之】伺歸彥還家召魏收於帝前作詔草除歸彥冀州【伺相吏翻】使乾和繕寫晝日仍敕門司不聽歸彥輒入宫時歸彥縱酒為樂經宿不知至明欲參【參朝參也毛晃曰參造也趨承也樂音洛】至門知之大驚而退及通名謝敕令早發别賜錢帛等物甚厚又敕督將悉送至清陽宫【齊有别宫在清淇之陽因以為名五代志清河郡清陽縣舊曰清河縣後齊省貝邱入焉改為貝邱隋開皇六年改為清陽將即亮翻】拜辭而退莫敢與語唯趙郡王叡與之久語時無聞者帝之為長廣王也清都和士開以善握槊彈琵琶有寵辟為開府行參軍及即位累遷給事黄門侍郎高元海畢義雲高乾和皆疾之將言其事士開乃奏元海等交結朋黨欲擅威福乾和由是被疎【被皮義翻】義雲納賂於士開得為兗州刺史【為和士開怙寵亂齊張本】 帝徵江州刺史周廸出鎮湓城【周廸領江州刺史而屯據臨川徵之鎮湓城若以江州授之者】又徵其子入朝【朝直遙翻下同】廸趑且顧望並不至【趑子移翻且七余翻趑且不進之貌】其餘南江酋帥私署令長多不受召【酋慈秋翻帥所類翻長知兩翻】朝廷未暇致討但羈縻之豫章太守周敷獨先入朝進號安西將軍給鼓吹一部賜以女妓金帛令還豫章【周敷先與周廸分據臨川既破熊曇朗敷移據豫章吹尺瑞翻妓渠綺翻】廸以敷素出已下深不平之乃隂與留異相結遣其弟方興襲敷敷與戰破之又遣其兄子伏甲船中詐為賈人欲襲湓城【賈音古】未發事覺尋陽太守監江州事晉陵華皎遣兵逆擊之盡獲其船仗【監工銜翻】上以閩州刺史陳寶應之父為光禄大夫【五代志建安郡陳置閩州陳寶應父羽】子女皆受封爵命宗正編入屬籍而寶應以留異女為妻隂與異合虞荔弟寄流寓閩中荔思之成疾上為荔徵之寶應留不遣寄嘗從容諷以逆順【為於偽翻從千容翻】寶應輒引它語以亂之寶應嘗使人讀漢書臥而聽之至蒯通說韓信曰相君之背貴不可言蹶然起坐曰可謂智士寄曰通一說殺三士何足稱智【班固曰蒯通一說而喪三儁應劭注云謂烹酈生敗田横驕韓信也說式芮翻相息亮翻】豈若班彪王命識所歸乎【王命論見四十一卷漢光武建武五年】寄知寶應不可諫恐禍及已乃著居士服【著陟畧翻】居東山寺陽稱足疾寶應使人燒其屋寄安臥不動親近將扶之出寄曰吾命有所懸避將安往【言託跡閩中生死之命懸於人手無所避之也】縱火者自救之 乙卯齊以任城王湝為司徒【任音壬湝戶皆翻】 齊揚州刺史行臺王琳數欲南侵尚書盧潛以為時事未可上遣移書夀陽欲與齊和親潛以其書奏齊朝仍上啟請且息兵【數所角翻上時掌翻朝直遙翻】齊主許之遣散騎常侍崔瞻來聘且歸南康愍王曇朗之喪【曇朗為齊所殺見一百六十六卷梁敬帝太平元年】琳於是與潛有隙更相表列【更工衡翻】齊主徵琳赴鄴以潛為揚州刺史領行臺尚書瞻㥄之子也【高歡起兵於信都崔㥄為參佐㥄力膺翻】梁末喪亂鐵錢不行【梁普通中鑄鐵錢喪息浪翻】民間私用鵝眼錢甲子改鑄五銖錢一當鵝眼之十 【考異曰隋志在天嘉五年今從陳帝紀】後梁主安於儉素不好酒色【好呼到翻】雖多猜忌而撫將<br />
<br />
  士有恩以封疆褊隘邑居殘毁干戈日用鬱鬱不得志疽發背而殂【年四十四】葬平陵諡曰宣皇帝廟號中宗太子巋即皇帝位【巋字仁遠宣帝之第三子也巋音歸又區胃翻】改元天保尊龔太后為太皇太后王后曰皇太后母曹貴嬪為皇太妃【嬪昆賓翻】 二月丙子安成王頊至建康詔以為中書監中衛將軍上謂杜杲曰家弟今蒙禮遣實周朝之惠然魯山不返亦恐未能及此【言若不賂以魯山亦恐未及遣安成王還也朝直遙翻下同】杲對曰安成長安一布衣耳而陳之介弟也【介大也】其價豈止一城而已哉本朝敦睦九族恕已及物上遵太祖遺旨下思繼好之義【好呼到翻】是以遣之南歸今乃云以尋常之土易骨肉之親非使臣之所敢聞也【使疏吏翻】上甚慚曰前言戲之耳待杲之禮有加焉頊妃柳氏及子叔寶猶在穰城上復遣毛喜如周請之【頊吁玉翻復扶又翻】周人皆歸之丁丑以安右將軍吳明徹為江州刺史督高州刺史<br />
<br />
  黄法【巨俱翻】豫章太守【守式又翻】周敷共討周廸 甲申大赦 留異始謂臺軍必自錢塘上既而侯安都步由諸暨出永康【上時掌翻諸暨縣自漢以來屬會稽郡永康縣吳赤烏八年分上虞烏傷立屬東陽郡自永安至東陽一百九里】異大驚奔桃枝嶺於巖口竪柵以拒之【堅而主翻】安都為流矢所中血流至踝【中竹仲翻踝胡瓦翻足跟也】乘轝指麾容止不變因其山勢迮而為堰【迮側百翻廹也堰於建翻】會潦水漲滿安都引船入堰起樓艦與異城等發拍碎其樓堞【潦盧皓翻艦戶黯翻堞達協翻】異與其子忠臣脫身奔晉安依陳寶應安都虜其妻及餘子盡收鎧仗而還【鎧可亥翻還從宣翻又音如字】異黨向文政據新安上以貞毅將軍程文季為新安太守【梁置貞毅將軍班第二十二在五德將軍之下陳制擬官品第五】帥精甲三百輕往攻之文政戰敗遂降文季靈洗之子也【陳氏建國程靈洗蕭摩訶等俱為健將帥讀曰率降戶江翻】 夏四月辛丑齊武明婁太后殂齊主不改服緋袍如故未幾【未幾言未幾時也幾居豈翻】登三臺置酒作樂宫女進白袍帝投諸臺下散騎常侍和士開請止樂帝怒撾之【和士開長君之惡者也自此益無所忌憚矣】 乙巳齊遣使來聘【使疏吏翻下同】 齊青州上言河水清齊主遣使祭之改元河清先是周之君臣受封爵者皆未給租賦【先悉薦翻】癸亥始<br />
<br />
  詔柱國等貴臣邑戶聽寄食佗縣 五月庚午周大赦己丑齊以右僕射斛律光為尚書令 壬辰周以柱<br />
<br />
  國楊忠為大司空六月己亥以柱國蜀國公尉遲迥為大司馬【尉紆勿翻】 秋七月己丑納太子妃王氏金紫光禄大夫周之女也【姚思亷陳書周作固】 齊平秦王歸彦至冀州内不自安欲待齊主如晉陽乘虛入鄴其郎中令呂思禮告之【此王國郎中令也】詔大司馬段韶司空婁叡討之歸彥於南境置私驛聞大軍將至即閉城拒守長史宇文仲鸞等不從皆殺之歸彥自稱大丞相有衆四萬齊主以都官尚書封子繪冀州人祖父世為本州刺史得人心【子繪父隆之祖囘皆為冀州刺史】使乘傳至信都【傳張戀翻】巡城諭以禍福吏民降者相繼城中動静大小皆知之歸彥登城大呼云【降戶江翻呼火故翻】孝昭皇帝初崩六軍百萬悉在臣手投身向鄴奉迎陛下當時不反今日豈反邪正恨高元海畢義雲高乾和誑惑聖上疾忌忠良但為殺此三人【誑居况翻為於偽翻】即臨城自刎【刎扶粉翻】既而城破單騎北走至交津獲之【水經注衡漳水逕武邑郡南又東逕武強縣北又東北逕武隧縣故城南白馬河注之水上承呼沱東逕樂鄉縣北饒陽縣南又東南逕武邑郡北而東入衡漳水謂之交津口】鎻送鄴乙未載以露車銜木面縛【李百藥北齊書銜木作銜枚】劉桃枝臨之以刃擊鼓隨之并其子孫十五人皆棄市命封子繪行冀州事齊主知歸彥前譖清河王岳【事見一百六十六卷梁敬帝紹泰元年】以歸彥家良賤百口賜岳家贈岳太師丁酉以段韶為太傅婁叡為司徒平陽王淹為太宰斛律光為司空趙郡王叡為尚書令河間王孝琬為左僕射 癸亥齊主如晉陽上遣使聘齊 九月戊辰朔日有食之 以侍中都官尚書到仲舉為尚書右僕射丹陽尹仲舉溉之弟子也【到溉彥之之曾孫梁初以文學顯以亷白稱】 吳明徹至臨川攻周廸不能克丁亥詔安成王頊代之 【考異曰陳書帝紀云丁亥廸請降詔安成王諱督衆軍以招納之今從南史廸傳】 冬十月戊戌詔以軍旅費廣百姓空虛凡供乘輿飲食衣服及宫中調度悉從減削【乘繩證翻調徒釣翻】至於百司宜亦思省約 十一月丁卯周以趙國公招為益州總管 丁丑齊遣兼散騎常侍封孝琰來聘十二月丙辰齊主還鄴【自晉陽還鄴】齊主逼通昭信李后【文宣李后宫曰昭信】曰若不從我我殺爾兒后懼從之既而有娠太原王紹德至閤不得見【見賢遍翻】愠曰兒豈不知邪姊腹大故不見兒后大慚由是生女不舉帝横刀詬曰【詬時候翻】殺我女我何得不殺爾兒對后以刀環築殺紹德后大哭帝愈怒裸后亂撾之后號天不已帝命盛以絹囊【號戶刀翻盛時征翻】流血淋漉投諸渠水良久乃蘇犢車載送妙勝寺為尼【武成之淫虐文宣教之也是以詩貴正始】<br />
<br />
  資治通鑑卷一百六十八<br />
<br />
<史部,編年類,資治通鑑>  <br>
   </div> 

<script src="/search/ajaxskft.js"> </script>
 <div class="clear"></div>
<br>
<br>
 <!-- a.d-->

 <!--
<div class="info_share">
</div> 
-->
 <!--info_share--></div>   <!-- end info_content-->
  </div> <!-- end l-->

<div class="r">   <!--r-->



<div class="sidebar"  style="margin-bottom:2px;">

 
<div class="sidebar_title">工具类大全</div>
<div class="sidebar_info">
<strong><a href="http://www.guoxuedashi.com/lsditu/" target="_blank">历史地图</a></strong>  
<a href="http://www.880114.com/" target="_blank">英语宝典</a>  
<a href="http://www.guoxuedashi.com/13jing/" target="_blank">十三经检索</a> 
<br><strong><a href="http://www.guoxuedashi.com/gjtsjc/" target="_blank">古今图书集成</a></strong> 
<a href="http://www.guoxuedashi.com/duilian/" target="_blank">对联大全</a> <strong><a href="http://www.guoxuedashi.com/xiangxingzi/" target="_blank">象形文字典</a></strong> 

<br><a href="http://www.guoxuedashi.com/zixing/yanbian/">字形演变</a>  <strong><a href="http://www.guoxuemi.com/hafo/" target="_blank">哈佛燕京中文善本特藏</a></strong>
<br><strong><a href="http://www.guoxuedashi.com/csfz/" target="_blank">丛书&方志检索器</a></strong> <a href="http://www.guoxuedashi.com/yqjyy/" target="_blank">一切经音义</a>  

<br><strong><a href="http://www.guoxuedashi.com/jiapu/" target="_blank">家谱族谱查询</a></strong>  <strong><a href="http://shufa.guoxuedashi.com/sfzitie/" target="_blank">书法字帖欣赏</a></strong> 
<br>

</div>
</div>


<div class="sidebar" style="margin-bottom:0px;">

<font style="font-size:22px;line-height:32px">QQ交流群9:489193090</font>


<div class="sidebar_title">手机APP 扫描或点击</div>
<div class="sidebar_info">
<table>
<tr>
	<td width=160><a href="http://m.guoxuedashi.com/app/" target="_blank"><img src="/img/gxds-sj.png" width="140"  border="0" alt="国学大师手机版"></a></td>
	<td>
<a href="http://www.guoxuedashi.com/download/" target="_blank">app软件下载专区</a><br>
<a href="http://www.guoxuedashi.com/download/gxds.php" target="_blank">《国学大师》下载</a><br>
<a href="http://www.guoxuedashi.com/download/kxzd.php" target="_blank">《汉字宝典》下载</a><br>
<a href="http://www.guoxuedashi.com/download/scqbd.php" target="_blank">《诗词曲宝典》下载</a><br>
<a href="http://www.guoxuedashi.com/SiKuQuanShu/skqs.php" target="_blank">《四库全书》下载</a><br>
</td>
</tr>
</table>

</div>
</div>


<div class="sidebar2">
<center>


</center>
</div>

<div class="sidebar"  style="margin-bottom:2px;">
<div class="sidebar_title">网站使用教程</div>
<div class="sidebar_info">
<a href="http://www.guoxuedashi.com/help/gjsearch.php" target="_blank">如何在国学大师网下载古籍?</a><br>
<a href="http://www.guoxuedashi.com/zidian/bujian/bjjc.php" target="_blank">如何使用部件查字法快速查字?</a><br>
<a href="http://www.guoxuedashi.com/search/sjc.php" target="_blank">如何在指定的书籍中全文检索?</a><br>
<a href="http://www.guoxuedashi.com/search/skjc.php" target="_blank">如何找到一句话在《四库全书》哪一页?</a><br>
</div>
</div>


<div class="sidebar">
<div class="sidebar_title">热门书籍</div>
<div class="sidebar_info">
<a href="/so.php?sokey=%E8%B5%84%E6%B2%BB%E9%80%9A%E9%89%B4&kt=1">资治通鉴</a> <a href="/24shi/"><strong>二十四史</strong></a>&nbsp; <a href="/a2694/">野史</a>&nbsp; <a href="/SiKuQuanShu/"><strong>四库全书</strong></a>&nbsp;<a href="http://www.guoxuedashi.com/SiKuQuanShu/fanti/">繁体</a>
<br><a href="/so.php?sokey=%E7%BA%A2%E6%A5%BC%E6%A2%A6&kt=1">红楼梦</a> <a href="/a/1858x/">三国演义</a> <a href="/a/1038k/">水浒传</a> <a href="/a/1046t/">西游记</a> <a href="/a/1914o/">封神演义</a>
<br>
<a href="http://www.guoxuedashi.com/so.php?sokeygx=%E4%B8%87%E6%9C%89%E6%96%87%E5%BA%93&submit=&kt=1">万有文库</a> <a href="/a/780t/">古文观止</a> <a href="/a/1024l/">文心雕龙</a> <a href="/a/1704n/">全唐诗</a> <a href="/a/1705h/">全宋词</a>
<br><a href="http://www.guoxuedashi.com/so.php?sokeygx=%E7%99%BE%E8%A1%B2%E6%9C%AC%E4%BA%8C%E5%8D%81%E5%9B%9B%E5%8F%B2&submit=&kt=1"><strong>百衲本二十四史</strong></a>  <a href="http://www.guoxuedashi.com/so.php?sokeygx=%E5%8F%A4%E4%BB%8A%E5%9B%BE%E4%B9%A6%E9%9B%86%E6%88%90&submit=&kt=1"><strong>古今图书集成</strong></a>
<br>

<a href="http://www.guoxuedashi.com/so.php?sokeygx=%E4%B8%9B%E4%B9%A6%E9%9B%86%E6%88%90&submit=&kt=1">丛书集成</a> 
<a href="http://www.guoxuedashi.com/so.php?sokeygx=%E5%9B%9B%E9%83%A8%E4%B8%9B%E5%88%8A&submit=&kt=1"><strong>四部丛刊</strong></a>  
<a href="http://www.guoxuedashi.com/so.php?sokeygx=%E8%AF%B4%E6%96%87%E8%A7%A3%E5%AD%97&submit=&kt=1">說文解字</a> <a href="http://www.guoxuedashi.com/so.php?sokeygx=%E5%85%A8%E4%B8%8A%E5%8F%A4&submit=&kt=1">三国六朝文</a>
<br><a href="http://www.guoxuedashi.com/so.php?sokeytm=%E6%97%A5%E6%9C%AC%E5%86%85%E9%98%81%E6%96%87%E5%BA%93&submit=&kt=1"><strong>日本内阁文库</strong></a> <a href="http://www.guoxuedashi.com/so.php?sokeytm=%E5%9B%BD%E5%9B%BE%E6%96%B9%E5%BF%97%E5%90%88%E9%9B%86&ka=100&submit=">国图方志合集</a> <a href="http://www.guoxuedashi.com/so.php?sokeytm=%E5%90%84%E5%9C%B0%E6%96%B9%E5%BF%97&submit=&kt=1"><strong>各地方志</strong></a>

</div>
</div>


<div class="sidebar2">
<center>

</center>
</div>
<div class="sidebar greenbar">
<div class="sidebar_title green">四库全书</div>
<div class="sidebar_info">

《四库全书》是中国古代最大的丛书,编撰于乾隆年间,由纪昀等360多位高官、学者编撰,3800多人抄写,费时十三年编成。丛书分经、史、子、集四部,故名四库。共有3500多种书,7.9万卷,3.6万册,约8亿字,基本上囊括了古代所有图书,故称“全书”。<a href="http://www.guoxuedashi.com/SiKuQuanShu/">详细>>
</a>

</div> 
</div>

</div>  <!--end r-->

</div>
<!-- 内容区END --> 

<!-- 页脚开始 -->
<div class="shh">

</div>

<div class="w1180" style="margin-top:8px;">
<center><script src="http://www.guoxuedashi.com/img/plus.php?id=3"></script></center>
</div>
<div class="w1180 foot">
<a href="/b/thanks.php">特别致谢</a> | <a href="javascript:window.external.AddFavorite(document.location.href,document.title);">收藏本站</a> | <a href="#">欢迎投稿</a> | <a href="http://www.guoxuedashi.com/forum/">意见建议</a> | <a href="http://www.guoxuemi.com/">国学迷</a> | <a href="http://www.shuowen.net/">说文网</a><script language="javascript" type="text/javascript" src="https://js.users.51.la/17753172.js"></script><br />
  Copyright &copy; 国学大师 古典图书集成 All Rights Reserved.<br>
  
  <span style="font-size:14px">免责声明:本站非营利性站点,以方便网友为主,仅供学习研究。<br>内容由热心网友提供和网上收集,不保留版权。若侵犯了您的权益,来信即刪。scp168@qq.com</span>
  <br />
ICP证:<a href="http://www.beian.miit.gov.cn/" target="_blank">鲁ICP备19060063号</a></div>
<!-- 页脚END --> 
<script src="http://www.guoxuedashi.com/img/plus.php?id=22"></script>
<script src="http://www.guoxuedashi.com/img/tongji.js"></script>

</body>
</html>
