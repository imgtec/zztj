資治通鑑卷二百六十
宋 司馬光 撰

胡三省 音註

唐紀七十六|{
	起旃蒙單閼書柔兆執徐凡二年}


昭宗聖穆景文孝皇帝上之下

乾寧二年春正月辛酉幽州軍民數萬以麾蓋歌鼓迎李克用入府舍克用命李存審劉仁恭將兵略定巡屬|{
	幽涿瀛莫媯檀薊順營平新武等州皆盧龍巡屬也}
癸未朱全忠遣其將朱友恭圍兖州|{
	朱瑾據兖州屢為汴人所敗兵力俱困至是受圍}
朱瑄自鄆以兵糧救之友恭設伏敗之於高梧|{
	敗補邁翻高梧即春秋魯國之高魚杜預注曰高魚在東郡廩丘縣東南續漢志廩丘有鄆城高魚城}
盡奪其餉擒河東將安福順安福慶|{
	去年河東遣安福順等救兖鄆事見上卷}
己巳以給事中陸希聲為戶部侍郎同平章事希聲元方五世孫也|{
	陸元方見二百五卷武后長夀二年}
壬申護國節度使王重盈薨軍中請以重榮子行軍司馬珂知留後事珂重盈兄重簡之子也重榮養以為子|{
	為王珙王珂争河中張本重直龍翻珂丘何翻}
楊行密表朱全忠罪惡請會易定兖鄆河東兵討之 董昌將稱帝集將佐議之節度副使黄碣曰|{
	碣其謁翻}
今唐室雖微天人未厭齊桓晉文皆翼戴周室以成霸業大王興於畎畝|{
	昌爵隴西郡王故稱之畎古泫翻}
受朝廷厚恩位至將相富貴極矣奈何一旦忽為族滅之計乎碣寧死為忠臣不生為叛逆昌怒以為惑衆斬之投其首於厠中罵之曰奴賊負我好聖明時三公不能待而先求死也并殺其家八十口同坎瘞之|{
	瘞於計翻}
又問會稽令吳鐐|{
	會古外翻鐐力彫翻又力弔翻}
對曰大王不為真諸侯以傳子孫乃欲假天子以取滅亡邪|{
	乃欲之下有為字文意方足}
昌亦族誅之又謂山隂令張遜曰汝有能政吾深知之俟吾為帝命汝知御史臺遜曰大王起石鏡鎮|{
	見二百五十三卷僖宗乾符五年}
建節浙東榮貴近二十年|{
	近其靳翻}
何苦效李錡劉闢之所為乎|{
	李錡劉闢以反誅事皆見憲宗紀}
浙東僻處海隅|{
	處昌呂翻}
巡屬雖有六州大王若稱帝彼必不從|{
	台明温處婺衢浙東巡屬也時豪傑並起各自為刺史昌羈縻而已}
徒守空城為天下笑耳昌又殺之謂人曰無此三人者則人莫我違矣二月辛卯昌被衮冕登子城門樓即皇帝位|{
	被皮義翻 考異曰吳越備史云癸卯昌僭號按會稽錄昌自云應兔子之䜟欲以二月二日僭號取卯月卯日也而實錄長歷皆云二月己丑朔非當時歷誤即今日歷誤要之昌必以二月辛卯日僭號}
悉陳瑞物於庭以示衆先是咸通末|{
	先悉薦翻}
吳越間訛言山中有大鳥四目三足聲云羅平天冊見者有殃民間多畫像以祀之及昌僭號曰此吾鸑鷟也|{
	鸑五角翻鷟士角翻鸑鷟鳳屬}
乃自稱大越羅平國改元順天 |{
	考異曰吳越備史曰癸卯昌僭稱皇帝建元順天國號羅平年號或云天冊或云大聖皆非也羅隱撰吳越行營露布曰羅平者啓國之名順天者建元之始又曰將軍門稱天冊之樓以會府為宣室之地明告我其所稱曰權即羅平國位昌狀印文曰順天治國之印十國紀年亦云年號頤天會稽錄云天冊蓋誤今從備史}
署城樓曰天冊之樓令羣下謂已曰聖人以前杭州刺史李邈前婺州刺史蔣瓌兩浙鹽鐵副使杜郢前屯田郎中李瑜為相又以吳瑤等皆為翰林學士李暢之等皆為大將軍昌移書錢鏐告以權即羅平國位以鏐為兩浙都指揮使鏐遺昌書曰|{
	遺唯季翻}
與其閉門作天子與九族百姓俱陷塗炭豈若開門作節度使終身富貴邪及今悛悔|{
	悛且緣翻改也}
尚可及也昌不聽鏐乃將兵三萬詣越州城下至迎恩門|{
	迎恩門越州城西門}
見昌再拜言曰大王位兼將相奈何捨安就危鏐將兵此來以俟大王改過耳縱大王不自惜鄉里士民何罪隨大王族滅乎昌懼致犒軍錢二百萬執首謀者吳瑤及巫覡數人送於鏐|{
	犒苦到翻覡刑狄翻}
且請待罪天子鏐引兵還以狀聞|{
	聞於朝也}
王重盈之子保義節度使珙|{
	王重盈先鎮陜虢王重榮為其下所殺重盈代鎮河中以其子珙繼鎮陜虢陜虢號保義軍珙居勇翻}
晉州刺史瑤舉兵擊王珂表言珂非王氏子與朱全忠書言珂本吾家蒼頭不應為嗣珂上表自陳|{
	珂丘何翻上時掌翻}
且求援於李克用上遣中使諭解之 上重李谿文學乙未復以谿為戶部侍郎同平章事|{
	去年命李谿為相劉崇魯沮之而止事見上卷}
朱全忠軍于單父|{
	單父縣時帶單州單音善父音甫}
為朱友恭聲援|{
	朱友恭時圍朱瑾於兖州}
李克用表劉仁恭為盧龍留後留兵戌之壬子還晉陽媯州人高思繼兄弟有武幹為燕人所服克用皆以為都將分掌幽州兵部下士卒皆山北之豪也|{
	媯檀諸州皆在幽州山北亦謂之山後}
仁恭憚之久之河東兵戍幽州者暴横|{
	横戶孟翻}
思繼兄弟以法裁之所誅殺甚多克用怒以讓仁恭仁恭訴稱高氏兄弟所為克用俱殺之仁恭欲收燕人心復引其諸子置帳下厚撫之|{
	為仁恭叛克用張本復扶又翻下同}
崔昭緯與李茂貞王行瑜深相結得天子過失朝廷機事悉以告之邠寧節度副使崔鋋昭緯之族也|{
	鋋音蟬}
李谿再入相昭緯使鋋告行瑜曰曏者尚書令之命巳行矣而韋昭度沮之|{
	事見上卷景福二年}
今又引李谿為同列相與熒惑聖聽恐復有杜太尉之事|{
	杜讓能事亦見上卷景福二年}
行瑜乃與茂貞表稱谿姦邪昭度無相業宜罷居散秩|{
	散悉亶翻}
上報曰軍旅之事朕則與藩鎮圖之至於命相當出朕懷行瑜等論列不已三月谿復罷為太子少師|{
	復扶又翻}
王珙王瑤請朝廷命河中帥|{
	帥所類翻下同}
詔以中書侍郎同平章事崔胤同平章事充護國節度使以戶部侍郎判戶部王摶為中書侍郎同平章事 王珂李克用之壻也克用表重榮有功於國|{
	言破黄巢黜襄王王重榮皆有功也}
請賜其子珂節钺王珙厚結王行瑜李茂貞韓建三帥更上表稱珂非王氏子|{
	更工衡翻迭也}
請以珂為陜州珙為河中上諭以先已允克用之奏不許|{
	允從也為三帥稱兵入京城克用誅王瑤討三帥張本陜失冉翻}
加王鎔兼侍中 楊行密浮淮至泗州防禦使臺濛盛飾供帳|{
	姓苑臺姓臺駘之後後漢有高士臺佟晉有術士臺彦前趙有特進臺彦皋供居用翻}
行密不悦既行濛於卧内得補綻衣馳使歸之|{
	綻當作䘺文莧翻䘺亦補也使疏吏翻}
行密笑曰吾少貧賤不敢忘本|{
	少詩照翻}
濛甚慙行密攻濠州拔之執刺史張璲|{
	璲附朱全忠見上卷景福元年}
行密軍士掠得徐州人李氏之子生八年矣行密養以為子|{
	南唐世家曰李昇徐州人李榮之子榮遇亂不知所終昇少孤流寓濠泗間楊行密攻濠州得之養為子}
行密長子渥憎之行密謂其將徐温曰此兒質狀性識頗異於人吾度渥必不能容|{
	度徒洛翻}
今賜汝為子温名之曰知誥知誥事温勤孝過於諸子嘗得罪於温温笞而逐之及歸知誥迎拜於門温問何故猶在此知誥泣對曰人子捨父母將何之父怒而歸母人情之常也温以是益愛之使掌家事家人無違言及長喜書善射|{
	長知兩翻喜許既翻}
識度英偉行密常謂温日知誥俊傑諸將子皆不及也|{
	徐知誥事始此後復姓李名昇}
丁亥行密圍夀州 上以郊畿多盜至有踰垣入宫或侵犯陵寢者欲令宗室諸王將兵巡警又欲使之四方撫慰藩鎮南北司用事之臣恐其不利於已交章論諫上不得已夏四月下詔悉罷之 朝廷以董昌有貢輸之勤|{
	輸春遇翻}
今日所為類得心疾詔釋其罪縱歸田里戶部侍郎同平章事陸希聲罷為太子少師 楊行

密圍夀州不克將還庚寅其將朱延夀請試往更攻一鼓拔之|{
	以行密將還而懈於守溝故一鼓而拔}
執刺史江從朂|{
	高彦温舉夀州附朱全忠全忠以江從朂為刺史楊行密執之遂有濠夀二州}
行密以延夀權知夀州圑練使未幾|{
	幾居豈翻}
汴兵數萬攻夀州州中兵少吏民忷懼|{
	忷許勇翻}
延夀制軍中每旗二十五騎命黑雲隊長李厚將十旗擊汴兵不勝延夀將斬之|{
	長知兩翻}
厚稱衆寡不敵願益兵更往不勝則死都押牙汝陽柴再用亦為之請|{
	路振九國志柴再用始名存事孫儒與一小校結死友有告小校反儒斬之執存至詰何故反不對又問對曰與彼結死友彼反則某反公誅之復何問焉儒奇之曰汝果不反吾再用汝因改名為于偽翻}
乃益以五旗厚殊死戰再用助之延夀悉衆乘之汴兵敗走厚蔡州人也|{
	李厚者孫儒之遺兵}
行密又遣兵襲漣水拔之|{
	史言楊行密壤地浸廣泗州漣水縣杜佑曰漢仇猶縣宋白曰按厹猶城今宿預縣也魏曰海安縣晉為宿預之境宋置東海郡後魏改海安郡隋廢郡置漣水縣}
錢鏐表董昌僭逆不可赦請以本道兵討之|{
	錢鏐本有并董昌之心因其僭號仗大順而請討之}
太傅門下侍郎同平章事韋昭度以太保致仕 戊戌以劉建鋒為武安節度使建鋒以馬殷為内外馬步軍都指揮使|{
	為馬殷代建鋒張本}
楊行密遣使詣錢鏐言董昌已改過宜釋之|{
	楊行密欲存董昌以制錢鏐之後使不得與己争衡耳}
亦遣詣昌使趣朝貢|{
	趣讀曰促朝直遙翻下同}
河東遣其將史儼李承嗣以萬騎馳入于鄆|{
	李克用遣史儼等再往救兖鄆則不得還矣勞師遠圖自古忌之}
朱友恭退歸于汴 五月詔削董昌官爵委錢鏐討之 初王行瑜求尚書令不獲|{
	見上卷景福二年}
由是怨朝廷畿内有八鎮兵隸左右軍|{
	左右神策軍也}
郃陽鎮近華州韓建求之|{
	郃陽漢縣唐屬同州九域志縣在州東一百二十里郃音合近其靳翻下同}
良原鎮近邠州王行瑜求之|{
	良原縣屬涇州}
宦官曰此天子禁軍何可得也王珂王珙争河中行瑜建及李茂貞皆為珙請不能得恥之珙使人語三帥曰|{
	為于偽翻語于倨翻帥所類翻}
珂不受代而與河東婚姻必為諸公不利請討之行瑜使其弟匡國節度使行約攻河中|{
	時以同州為匡國軍九域志同州東至河中七十五里}
珂求救於李克用行瑜乃與茂貞建各將精兵數千入朝甲子至京師坊市民皆竄匿上御安福門以待之三帥盛陳甲兵拜伏舞蹈於門下上臨軒親詰之曰|{
	宇末曰軒詰去吉翻}
卿等不奏請俟報輒稱兵入京城其志欲何為乎若不能事朕今日請避賢路行瑜茂貞流汗不能言獨韓建粗述入朝之由|{
	粗音坐五翻}
上與三帥宴三帥奏稱南北司互有朋黨墮紊朝政|{
	墮讀曰隳紊音問}
韋昭度討西川失策|{
	討西川事見二百五十七卷二百五十八卷}
李谿作相不合衆心請誅之上未之許是日行瑜等殺昭度谿於都亭驛|{
	都亭驛在朱雀門外西街含光門北來第二坊}
又殺樞密使康尚弼及宦官數人又言王珂王珙嫡庶不分請除王珙河中徙王行約於陜王珂於同州上皆許之始三帥謀廢上立吉王保至是聞李克用已起兵於河東行瑜茂貞各留兵二千人宿衛京師與建皆辭還鎮貶戶部尚書楊堪為雅州刺史堪虞卿之子|{
	楊虞卿見文宗紀}
昭度之舅也初崔胤除河中節度使河東進奏官薛志勤揚言曰崔公雖重德以之代王珂不若光德劉公於我公厚也光德劉公者太常卿劉崇望也|{
	光德里名在長安城中唐末大臣有時望者時人率以其所居里稱之光德坊朱雀街西第三街北來第六坊京兆府在焉}
及三帥入朝聞志勤之言貶崇望昭州司馬李克用聞三鎮兵犯闕即日遣使十三輩發北部兵|{
	北部兵代北諸蕃落兵也}
期以來月度河入關 六月庚寅以錢鏐為浙東招討使鏐復發兵撃董昌|{
	復扶又翻}
辛卯以前均州刺史孔緯繡州司戶張濬並為太子賓客壬辰以緯為吏部尚書復其階爵癸巳拜司空兼門下侍郎同平章事以張濬為兵部尚書諸道租庸使|{
	孔緯張濬貶見二百五十八卷大順元年今欲復用之}
時緯居華州濬居長水上以崔昭緯等外交藩鎮朋黨相傾思得骨鯁之士故驟用緯濬緯以有疾扶輿至京師見上涕泣固辭上不許 李克用大舉蕃漢兵南下上表稱王行瑜李茂貞韓建稱兵犯闕賊害大臣請討之|{
	李克用實黨王珂聲三帥之罪而表請致討}
又移檄三鎮行瑜等大懼克用軍至絳州刺史王瑤閉城拒之克用進攻旬日拔之斬瑤於軍門殺城中違拒者千餘人秋七月丙辰朔克用至河中王珂迎謁於路匡國節度使王行約敗於朝邑|{
	朝直遙翻}
戊午行約棄同州走己未至京師行約弟行實時為左軍指揮使|{
	神策左軍非此}
帥衆與行約大掠西市|{
	朱雀街西謂之西市}
行實奏稱同華已沒沙陀將至請車駕幸邠州庚申樞密使駱全瓘奏請車駕幸鳳翔上曰朕得克用表尚駐軍河中就使沙陀至此朕自有以枝梧卿等但各撫本軍勿令搖動右軍指揮使李繼鵬茂貞假子也|{
	程大昌雍錄曰北軍左右兩軍皆在苑内左三軍在内東苑之東大明宫苑東也右三軍在九仙門之西九仙在内東苑之西北角左三軍左神策左龍武左羽林軍也右三軍右神策右龍武右羽林軍也余按雍錄所云左右六軍代德以後宿衛者也僖宗廣明幸蜀此六軍潰散田令孜於成都募新軍五十二都分屬左右神策軍自時厥後凡所謂左右軍者皆此軍也分營於京城内外又不專在苑中若此時王行實李繼鵬為左右軍指揮使疑是邠岐二帥所留兵以宿衛者自分為左右也}
本姓名閻珪與駱全瓘謀刼上幸鳳翔中尉劉景宣與王行實知之欲刼上幸邠州孔緯面折景宣以為不可輕離宫闕|{
	折之舌翻離力智翻}
向晚繼鵬連奏請車駕出幸於是王行約引左軍攻右軍鼓譟震地上聞亂登承天樓欲諭止之捧日都頭李筠將本軍於樓前侍衛李繼鵬以鳳翔兵攻筠|{
	王行約以李繼鵬欲先刼車駕幸岐故攻右軍李繼鵬當與行約戰而乃攻李筠者以筠衛上不得而刼幸也}
矢拂御衣著于樓桷|{
	著直略翻桷櫰也椽方曰桷}
左右扶上下樓繼鵬復縱火焚宫門煙炎蔽天時有鹽州六都兵屯京師|{
	炎讀與燄同鹽州六都兵孫德昭等所領兵也}
素為兩軍所憚上急召令入衛既至兩軍退走各歸邠州及鳳翔城中大亂互相剽掠|{
	剽匹妙翻}
上與諸王及親近幸李筠營護蹕都頭李居實帥衆繼至|{
	護蹕都亦神策五十四都之一或曰即扈蹕都帥讀曰率}
或傳王行瑜李茂貞欲自來迎車駕上懼為所迫辛酉以筠居實兩都兵自衛出啓夏門|{
	啓夏門長安城南面東來第一門}
趣南山宿莎城鎮|{
	莎城鎮在長安城南近郊之地也趣七喻翻}
士民追從車駕者數十萬人比至谷口暍死者三之一|{
	谷口南山谷口也暍於歇翻暍死者中熱而死比必寐翻}
夜復為盜所掠哭聲震山谷時百官多扈從不及|{
	從才用翻}
戶部尚書判度支及鹽鐵轉運使薛王知柔獨先至|{
	知柔薛王業之曾孫}
上命權知中書事及置頓使壬戌李克用入同州崔昭緯徐彦若王摶至莎城甲子上徙幸石門鎮|{
	路振九國志昭宗出啓夏門駐華嚴寺晡晚出幸南山之莎城駐于石門山之佛寺與此稍異}
命薛王知柔與知樞密院劉光裕還京城制置守衛宫禁丙寅李克用遣節度判官王瓌奉表問起居丁卯上遣内侍郗廷昱|{
	新書百官志内侍在内侍監之下内常侍之上員四人從四品上郗丑之翻}
齎詔詣李克用軍令與王珂各發萬騎同赴新平|{
	赴新平以討王行瑜邠州新平郡}
又詔彰義節度使張鐇以涇原兵控扼鳳翔李克用遣兵攻華州韓建登城呼曰|{
	呼火故翻}
僕於李公未嘗失禮何為見攻克用使謂之曰公為人臣逼逐天子公為有禮孰為無禮者乎會郗廷昱至言李茂貞將兵三萬至盩厔王行瑜將兵至興平皆欲迎車駕克用乃釋華州之圍移兵營渭橋 |{
	考異曰唐太祖紀年錄王師攻華州俄而郗廷昱至且言茂貞領兵三萬至盩厔行瑜領軍至興平欲往石門迎駕乃解華圍進營渭橋按實錄八月延王戒丕至河中克用已發前鋒至渭北己丑克用進營渭橋又紀年錄載詔曰省表已部領大軍前月二十七日離河中蓋克用不親圍華州但遣别將將兵往及聞邠岐謀迎駕乃遣華兵詣渭橋即所謂前鋒者也克用既以七月二十七日離河中則戒丕至彼必在其前實錄云八月至河中誤也今從紀年錄}
以薛王知柔為清海節度使|{
	是年賜嶺南節度使軍額曰清海}
同平章事仍權知京兆尹判度支充鹽鐵轉運使俟反正日赴鎮上在南山旬餘士民從車駕避亂者日相驚曰邠岐兵至矣上遣延王戒丕詣河中趣李克用令進兵|{
	延王邠玄宗子戒丕其後也趣讀曰促}
壬午克用發河中上遣供奉官張承業詣克用軍|{
	張承業内供奉官也}
承業同州人屢奉使於克用因留監其軍|{
	為張承業盡心於李克用父子張本}
己丑克用進軍渭橋遣其將李存貞為前鋒辛卯拔永夀又遣史儼將三千騎詣石門侍衛癸巳遣李存信李存審會保大節度使李思孝攻王行瑜棃園寨|{
	棃園寨在京兆雲陽縣九域志雲陽在華州西北九十里 考異曰莊宗列傳曰三鎮亂長安李存信從太祖入關以前軍先自夏陽度河攻同華屬邑下之時太祖在渭北伶官羣小或勸太祖入朝自握兵柄太祖亦以全忠圖已朝廷不能斷心微有望月餘不進軍存信與蓋寓乘間密啓曰大王家世効忠此行討逆上為邠鳳不臣但令臣節為天下所知即三賊不足平也而悠悠之徒不達大體或以弗詢之畫苟合台情雖俳優之言不宜縱其如此京師咫尺天聽非遙實無益於英德也今三凶正蹙須速圖之事留變生無宜猶豫太祖曰公言是也即日出師下棃園砦按克用謀大事固非伶官所豫又實錄己丑克用進營渭橋癸已克棃園中間四日耳無月餘不進事且既云羣小觀入朝即當詣行在不當留渭北此特李存信之人欲歸功於存信耳今不取}
擒其將王令陶等獻於行在思孝本姓拓跋思恭之弟也李茂貞懼斬李繼鵬傳首行在|{
	李茂貞委刼乘輿之罪於繼鵬}
上表請罪且遣使求和於克用上復遣延王戒丕丹王允諭克用|{
	丹王逾代宗子允其後也復扶又翻}
令且赦茂貞併力討行瑜俟其殄平當更與卿議之且命二王拜克用為兄 以前河中節度使崔胤為中書侍郎同平章事戊戌削奪王行瑜官爵癸卯以李克用為邠寧四面行營都招討使保大節度使李思孝為北面招討使定難節度使李思諫為東面招討使|{
	難乃旦翻}
彰義節度使張鐇為西面招討使|{
	命李克用自南臨討之}
克用遣其子存朂詣行在|{
	李存朂始此 考異曰實錄作存貞据後唐實錄薛居正五代史莊宗未嘗名存貞實錄蓋誤}
年十一上奇其狀貌撫之曰兒方為國之棟梁他日宜盡忠於吾家克用表請上還京上許之令克用遣騎三千駐三橋為備禦辛亥車駕還京師壬子司空兼門下侍郎同平章事崔昭緯罷為右僕射 以護國留後王珂盧龍留後劉仁恭各為本鎮節度使|{
	李克用之志也}
時宫室焚毁未暇完葺上寓居尚書省|{
	程大昌曰尚書省在朱雀門正街之東自占一坊六部附麗其旁}
百官往往無袍笏僕馬 以李克用為行營都統九月癸亥司空兼門下侍郎同平章事孔緯薨 辛

未朱全忠自將擊朱瑄戰於梁山|{
	新志鄆州夀張縣有刀梁山水經注梁山在夀張縣濟水逕其東}
瑄敗走還鄆 李克用急攻棃園王行瑜求救於李茂貞茂貞遣兵萬人屯龍泉鎮|{
	九域志邠州三水縣有龍泉鎮在州東北}
自將兵三萬屯咸陽之旁克用請詔茂貞歸鎮仍削奪其官爵欲分兵討之上以茂貞自誅繼鵬前已赦宥不可復削奪誅討|{
	復扶又翻}
但詔歸鎮仍令克用與之和解以昭義節度使李罕之檢校侍中充邠寧四面行營副都統史儼敗邠寧兵於雲陽|{
	敗補邁翻下同}
擒雲陽鎮使王令誨等獻之 王建遣簡州刺史王宗瑤等將兵赴難|{
	難乃旦翻}
甲戍軍于綿州|{
	春秋之法書救而書次者以次為貶貶者以其頓兵觀望不進無救難解急之意也王建遣兵赴難而軍于綿州何日至長安邪}
董昌求救於楊行密行密遣泗州防禦使臺濛攻蘇州以救之|{
	蘇州時屬錢鏐攻之所以牽制鏐兵不得專攻董昌}
且表昌引咎願修職貢請復官爵又遺錢鏐書稱昌狂疾自立已畏兵諫|{
	遺唯季翻春秋左氏傳鬻拳彊諫楚子不從臨之以兵}
執送同惡|{
	謂董昌執首謀僭吳瑤及巫覡數人送於鏐也}
不當復伐之|{
	復扶又翻}
冬十月丙戌河東將李存貞敗邠寧軍於棃園北殺

千餘人|{
	敗補邁翻}
自是棃園閉壁不敢出 貶右僕射崔昭緯為梧州司馬|{
	以黨附邠岐也}
魏國夫人陳氏才色冠後宫|{
	冠古玩翻}
戊子上以賜李克用|{
	薛史曰後克用薨陳氏為尼至晉天福中乃卒}
克用令李罕之李存信等急攻棃園城中食盡棄城走罕之等邀擊之所殺萬餘人克棃園等三寨獲王行瑜子知進及大將李元福等克用進屯棃園庚寅王行約王行實燒寧州遁去|{
	九域志寧州南至邠州一百二十五里}
克用奏請以匡國節度使蘇文建為静難節度使趣令赴鎮且理寧州招撫降人|{
	以蘇文建代王行瑜也時邠州未下故令且治寧州趣讀曰促降戶江翻}
上遷居大内|{
	葺理稍完自尚書省還居大内}
朱全忠遣都將葛從周擊兖州自以大軍繼之癸卯圍兖州|{
	是年春汴兵圍兖州以河東救至而退今復圍之}
楊行密遣寧國節度使田頵|{
	景福元年升宣歙團練使為寧國節度使}
潤

州團練使安仁義攻杭州鎮戍以救董昌昌使湖州將徐淑會淮南將魏約共圍嘉興錢鏐遣武勇都指揮使顧全武救嘉興破烏墩光福二寨|{
	九域志湖州烏程縣有烏墩鎮墩都昆翻}
淮南將柯厚破蘇州水柵全武餘姚人也義陽節度使王處存薨軍中推其子節度副使郜為留後|{
	郜古到翻}
以京兆尹武邑孫偓為兵部侍郎同平章事 王行瑜以精甲五千守龍泉寨李克用攻之李茂貞以兵五千救之營於鎮西|{
	鎮西龍泉鎮之西也}
李罕之擊鳳翔兵走之十一月丁巳拔龍泉寨行瑜走入邠州遣使請降於李克用齊州刺史朱瓊舉州降於朱全忠|{
	為朱瑾誘斬瓊張本考異曰薛居正五代史梁紀瓊降及死皆在十月按編遺錄十一月丁巳瓊遺軍將王自新奉檄歸義壬申瓊自來辛巳死今從之}
瓊瑾之從父兄也|{
	從才用翻}
衢州刺史陳儒卒弟岌代之 李克用引兵逼邠州王行瑜登城號哭|{
	號戶刀翻}
謂克用曰行瑜無罪迫脅乘輿皆李茂貞及李繼鵬所為請移兵問鳳翔行瑜願束身歸朝|{
	乘繩證翻朝直遙翻}
克用曰王尚父何恭之甚|{
	王行瑜賜號尚父時已削奪克用稱之以戲之}
僕受詔討三賊臣|{
	謂王行瑜李茂貞韓建也}
公預其一束身歸朝非僕所得專也丁卯行瑜挈族棄城走克用入邠州封府庫撫居人命指揮使高爽權巡撫軍城奏趣蘇文建赴鎮|{
	趣讀曰促}
行瑜走至慶州境部下斬行瑜傳首|{
	光啓三年王行瑜得静難節至是而誅}
朱瑄遣其將賀瓌柳存及河東將薛懷寶將兵萬餘人襲曹州|{
	曹州降汴見二百五十八卷大順二年}
以解兖州之圍瓌濮陽人也|{
	濮博木翻}
丁卯全忠自中都引兵夜追之比明至鉅野南及之|{
	比必利翻中都漢平陸縣天寶元年改曰中都鉅野漢古縣唐並屬鄆州九域志中都縣在州東南六十里鉅野縣在州南百八十里}
屠殺殆盡生擒瓌存懷寶俘士卒三千餘人是日晡後大風沙塵晦冥全忠曰此殺人未足耳下令所得之俘盡殺之庚午縳瓌等徇於兖州城下謂朱瑾曰卿兄已敗何不早降 丁丑雅州刺史王宗侃攻拔利州執刺史李繼顒斬之|{
	王宗侃西川將李繼顒鳳翔將}
朱瑾偽遣使請降於朱全忠|{
	因其誘降而行詐}
全忠自就延夀門下與瑾語|{
	延夀門蓋兖州城門也}
瑾曰欲送符印願使兄瓊來領之辛巳全忠使瓊往瑾立馬橋上伏驍果董懷進於橋下瓊至懷進突出擒之以入須臾擲首城外全忠乃引兵還|{
	全忠知瑾無降心攻之未易猝下故還}
以瓊弟玭為齊州防禦使|{
	玭蒲田翻}
殺柳存懷寶聞賀瓌名釋而用之|{
	賀瓖自此遂為朱氏用}
李克用旋軍渭北|{
	自邠寧回屯渭北}
加静難節度使蘇文建同平章事 蔣勛求為邵州刺史劉建鋒不許|{
	乾寧二年蔣勛棄回龍關以開劉建鋒之取長沙故邀之以求邵州}
勛乃與鄧繼崇起兵連飛山梅山蠻寇湘潭|{
	飛山蠻在邵州西北界今其山在靖州北十五里比諸山為最高峻四面絶壁千仭梅山蠻在潭州界宋朝開為安化縣在州西三百二十里湘潭後漢湘南縣地吳分湘南置衡陽縣天寶八年移治於洛口因改名湘潭縣屬潭州九域志在州南一百六十里}
據邵州使其將申德昌屯定勝鎮|{
	定勝鎮在邵州東北界}
以扼潭人 十二月甲申閬州防禦使李繼雍蓬州刺史費存|{
	費父沸翻}
渠州刺史陳璠各帥所部兵奔王建|{
	三人皆鳳翔將帥讀曰率}
乙酉李克用軍于雲陽王建奏東川節度使顧彦暉不發兵赴難而掠奪輜

重遣瀘州刺史馬敬儒斷峽路請興兵討之|{
	難乃旦翻重直用翻觀此則去年王宗瑤赴難之軍非真有勤王之心特借此以開東川兵端耳斷音短}
戊子華洪大破東川兵於楸林俘斬數萬拔楸林寨|{
	楸七由翻}
乙未進李克用爵晉王|{
	自隴西郡王進爵晉王}
加李罕之兼侍中以河東大將盖寓領容管觀察使|{
	蓋古盍翻領遙領也}
自餘克用將佐子孫並進官爵克用性嚴急左右小有過輒死無敢違忤惟蓋寓敏慧能揣其意|{
	忤五故翻揣初委翻}
婉辭禆益無不從者克用或以非罪怒將吏寓必陽助之怒克用常釋之有所諫諍必徵近事為喻由是克用愛信之境内無不依附權與克用侔朝廷及鄰道遣使至河東其賞賜賂遺先入克用次及寓家朱全忠數遣人間之|{
	遺于季翻數所角翻間古莧翻}
及揚言云蓋寓已代克用而克用待之益厚|{
	自古英雄之争天下必倚勇智之士以為用而出入左右伺候顔色者亦有敏慧軟媚之人若蓋寓之於李克用是也}
丙申王建攻東川别將王宗弼為東川兵所擒|{
	路振九國志曰王宗弼掠地飛烏為顧彦暉所獲}
顧彦暉畜以為子|{
	畜吁玉翻}
戊戌通州刺史李彦昭將所部兵二千降於建|{
	通州今之達州李彦昭亦鳳翔將}
李克用遣掌書記李襲吉入謝恩|{
	景鳳元年行軍府置掌書記開元以後諸節鎮皆置之掌朝覲聘慰薦祭祀祈祝之文與號令升絀之事}
密言於上曰比年以來|{
	比毗至翻}
關輔不寧|{
	關謂蒲潼隴蜀藍田諸關輔謂三輔關内即漢三輔之地}
乘此勝勢遂取鳳翔一勞永逸時不可失臣屯軍渭北專俟進止上謀於貴近或曰茂貞復滅|{
	復扶又翻下同}
則沙陁太盛朝廷危矣上乃賜克用詔褒其忠款|{
	款誠也}
而言不臣之狀行瑜為甚自朕出幸以來茂貞韓建自知其罪不忘國恩職貢相繼且當休兵息民克用奉詔而止既而私於詔使曰觀朝廷之意似疑克用有異心也然不去茂貞|{
	去羌呂翻}
關中無安寧之日|{
	其浚李茂貞再犯京師克用亦不能救矣}
又詔免克用入朝將佐或言今密邇闕庭豈可不入見天子|{
	見賢遍翻}
克用猶豫未决蓋寓言於克用曰曏者王行瑜輩縱兵狂悖|{
	悖蒲妹翻又蒲没翻}
致鑾輿播越百姓奔散今天子還未安席人心尚危大王若引兵度渭竊恐復驚駭都邑|{
	蓋寓言李克用既不可釋兵入朝若以衆入是復邠岐華三帥之事耳}
人臣盡忠在於勤王不在入覲願熟圖之克用笑曰蓋寓尚不欲吾入朝况天下之人乎乃表稱臣總帥大軍|{
	帥讀曰率}
不敢徑入朝覲且懼部落士卒侵擾渭北居人辛亥引兵東歸表至京師上下始安詔賜河東士卒錢三十萬緡克用既去李茂貞驕横如故|{
	横下孟翻}
河西州縣多為茂貞所據|{
	河西謂涼瓜沙肅諸州}
以其將胡敬璋為河西節度使 朱全忠之去兖州也|{
	朱瓊死而全忠還}
留葛從周將兵守之朱瑾閉城不復出從周將還乃揚言天平河東救兵至引兵西北邀之夜半潛歸故寨瑾以從周精兵悉出果出兵攻寨從周突出奮擊殺千餘人擒其都將孫漢筠而還 加鎮海節度使錢鏐兼侍中 彰義節度使張鐇薨以其子璉權知留後|{
	璉力展翻}
朱瑄朱瑾屢為朱全忠所攻民失耕稼財力俱弊告急於河東李克用遣大將史儼李承嗣將數千騎假道於魏以救之|{
	史儼李承嗣自此遂與朱瑾入淮南矣}
安州防禦使家晟|{
	姓苑家姓周大夫家父之後又魯公族有子家氏}
與朱全忠親吏蔣玄暉有隙恐及禍與指揮使劉士政兵馬監押陳可璠將兵三千襲桂州殺經略使周元静而代之|{
	自安州遠襲桂州而克之者江湘城邑荒殘守兵單弱道無邀截之患桂人不意其至遂殺其帥而代之璠孚袁翻}
晟醉侮可璠可璠手刃之推士政知軍府事可璠自為副使詔即以士政為經略使玄暉吳人也|{
	為劉陳又為馬殷所併張本}


三年春正月西川將王宗夔攻拔龍州殺刺史田昉|{
	此時龍州當屬李茂貞眆方往翻}
丁巳劉建鋒遣都指揮使馬殷將兵討蔣勛攻定勝寨破之|{
	去年蔣勛遣兵守定勝寨}
辛未安仁義以舟師至湖州欲度江應董昌|{
	安仁義自潤州以舟師至湖州何從而度江哉蓋欲自湖州舟行入柳浦而度西陵耳然錢鏐在杭未容得至西陵}
錢鏐遣武勇都指揮使顧全武都知兵馬使許再思守西陵仁義不能度昌遣其將湯臼守石城|{
	會稽志石城山在山隂縣東北三十}
袁邠守餘姚閏月克用遣蕃漢都指揮使李存信|{
	此又是一段起事克用之上當有李字}
將萬騎假道于魏以救兖鄆軍于莘縣朱全忠使人謂羅弘信曰克用志吞河朔師還之日貴道可憂存信戢衆不嚴|{
	戢則立翻}
侵暴魏人弘信怒發兵三萬夜襲之存信軍潰退保洺州喪士卒什二三|{
	按九域志莘縣西距魏州九十里羅弘信欲襲李存信亦必朝出軍而後能乘夜而至李存信之敗斥候不明故也喪息浪翻}
委棄資糧兵械萬數史儼李承嗣之軍隔絶不得還弘信自是與河東絶專志於汴全忠方圖兖鄆畏弘信議其後弘信每有贈遺|{
	遺于季翻}
全忠必對使者北向拜授之曰六兄於予倍年以長固非諸鄰之比|{
	授當作受羅弘信第六記曲禮年長以倍則父事之朱全忠豈知禮者繆為恭敬以離并魏之交耳長知兩翻諸鄰謂與宣武鄰道諸帥也}
弘信信之全忠以是得專意東方|{
	謂專意攻兖鄆也}
丁亥果州刺史張雄降于王建|{
	宋白曰果州南充郡劉璋初分墊江已上置巴郡理此建安六年璋改郡為巴西徙理閬中今郡在嘉陵江之西魏平蜀於今州北三十七里石苟埧置南宕渠郡其縣亦移就郡理隋廢郡併入閬中復為巴西縣地仍移巴西縣理安漢城開皇十八年改為南充縣唐武德四年分置果州以郡南八里有果山為名}
二月戊辰顧全武許再思敗湯臼於石城|{
	敗補邁翻下同}
上用楊行密之請赦董昌復其官爵錢鏐不從 以通王滋判侍衛諸將事|{
	通王滋宣宗子}
朱全忠薦兵部尚書張濬上欲復相之李克用表請發兵擊全忠且言濬朝為相臣則夕至闕庭|{
	觀李克用此表謂非脅君吾不信也}
京師震懼上下詔和解之 三月以天雄留後李繼徽為節度使 保大節度使李思孝表請致仕薦弟思敬自代詔以思孝為太師致仕思敬為保大留後 朱全忠遣龎師古將兵伐鄆州敗鄆兵於馬頰|{
	馬頰禹疏九河之一也水經注濟水自須昌縣北逕魚山東左合馬頰水水首受濟西北流歷安民山北又逕桃城東又東北逕魚山南又東注于濟曰馬頰口敗補邁翻}
遂抵其城下 己酉顧全武等攻餘姚明州刺史黄晟遣兵助之董昌遣其將徐章救餘姚全武擊擒之 夏四月辛酉河漲將毁滑州城朱全忠命决為二河夾滑城而東為害滋甚 李克用擊羅弘信|{
	報李存信之敗也}
攻洹水|{
	洹于元翻}
殺魏兵萬餘人進攻魏州 武安節度使劉建鋒既得志|{
	小人之器易盈劉建鋒甫得長沙已得志矣}
嗜酒不親政事長直兵陳贍妻美建鋒私之贍袖鐵撾擊殺建鋒諸將殺贍迎行軍司馬張佶為留後佶將入府馬忽踶齧傷左髀|{
	佶巨乙翻踶大計翻齧五結翻}
時馬殷攻邵州未下|{
	是年正月劉建鋒遣馬殷攻邵州}
佶謝諸將曰馬公勇而有謀寛厚樂善吾所不及真乃主也|{
	乃汝也樂音洛}
乃以牒召之殷猶豫未行聽直軍將姚彦章說殷曰公與劉龍驤張司馬一體之人也|{
	聽讀曰廳直廳事之軍將也劉建鋒張佶馬殷同在孫儒軍中儒敗三人者叶力成軍以取湖南故彦章云然路振曰乾符中黄巢亂詔遣忠武决勝指揮使孫儒龍驤指揮使劉建鋒戍淮西隷秦宗權宗權為□所敗遂降之儒等皆為所脅制}
今龍驤遇禍司馬傷髀天命人望捨公尚誰屬哉|{
	屬之欲翻}
殷乃使親從都副指揮使李瓊留攻邵州|{
	從才用翻}
徑詣長沙 淮南兵與鎮海兵戰于皇天蕩|{
	大江過昇州界浸以深廣自老鸛觜渡白沙横關三十餘里俗呼為皇天蕩是時淮南兵既敗浙兵於皇天蕩遂圍蘇州則非前所言皇天蕩矣宋熙寧二年崑山人郟亶上疏言水利謂長洲縣界有長蕩黄天蕩其水上承湖下通海正淮浙兵戰處也}
鎮海兵不利楊行密遂圍蘇州 錢鏐鍾傳杜洪畏楊行密之彊皆求援於朱全忠|{
	其後鍾杜皆不能保其土而錢氏獨傳子及孫以此知有國有家者久近存乎其人}
全忠遣許州刺史朱友恭將兵萬人度淮聽以便宜從事 董昌使人覘錢鏐兵|{
	覘丑亷翻}
有言其彊盛者輒怒斬之言兵疲食盡則賞之戊寅袁邠以餘姚降於鏐顧全武許再思進兵至越州城下五月昌出戰而敗嬰城自守全武等圍之昌始懼去帝號|{
	去羌呂翻}
復稱節度使 馬殷至長沙張佶肩輿入府坐受殷拜謁已乃命殷升聽事以留後讓之即趨下帥將吏拜賀|{
	坐受拜謁留後受將校牙參之禮帥將吏拜賀行軍司馬賀新留後之禮帥讀曰率}
復為行軍司馬代殷將兵攻邵州 癸未蘇州常熟鎮使陸郢以州城應楊行密虜刺史成及行密閲及家所蓄惟圖書藥物賢之歸署行軍司馬及拜且泣曰及百口在錢公所失蘇州不能死敢求富貴願以一身易百口之死引佩刀欲自刺|{
	刺七亦翻}
行密遽執其手止之館於府舍|{
	館古玩翻}
其室中亦有兵仗行密每單衣詣之與之共飲膳無所疑|{
	使楊行密待俘虜皆如成及不亦汎乎是必所有見也}
錢鏐聞蘇州陷急召顧全武使趨西陵備行密|{
	趨七喻翻既恐其得蘇而乘勝攻杭又恐其自海道趨西陵也}
全武曰越州賊之根本奈何垂克棄之請先取越州後復蘇州鏐從之|{
	史言顧全武頗䜟用兵先後}
淮南將朱延夀奄至蘄州圍其城大將賈公鐸方獵不得還伏兵林中命勇士二人衣羊皮夜入延夀所掠羊羣潛入城約夜半開門舉火為應復衣皮反命|{
	衣於既翻}
公鐸如期引兵至城南門中火舉力戰突圍而入延夀驚曰吾常恐其潰圍而出反潰圍而入如此城安可猝拔乃白行密求軍中與公鐸有舊者持誓書金帛往說之許以昏|{
	說式芮翻}
夀州團練副使柴再用請行臨城與語為陳利害|{
	為于偽翻}
數日公鐸及刺史馮敬章請降以敬章為左都押牙|{
	淮南左都押牙也}
公鐸為右監門衛將軍|{
	此是領環衛官僖宗光啓三年馮敬章陷蘄州至是降路振九國志曰賈鐸生於上蔡叛秦宗權度淮遇故人馮敬章導之襲破蘄春推敬章為帥鐸為牙將塹城礪兵以自固}
延夀進拔光州殺刺史劉存|{
	楊行密自此全有淮南之地}
丙戌上遣中使詣梓州和解兩川王建雖奉詔還成都然猶連兵未解 崔昭緯復求救於朱全忠戊子遣中使賜昭緯死行至荆南追及斬之中外咸以為快|{
	崔昭緯結邠岐以殺杜讓能韋昭度李谿卒亦以殺其身朋比為姦果何益哉}
荆南節度使成汭與其將許存泝江略地盡取濱江州縣武泰節度使王建肇棄黔州收餘衆保豐都|{
	豐都漢巴郡枳縣地後漢置平都縣因山以名縣也梁置臨江郡隋廢郡為縣義寧二年分臨江置豐都縣唐屬忠州九城志在州西九十里}
存又引兵西取渝涪二州汭以其將趙武為黔中留後存為萬州刺史汭知存不得志使人詗之曰存不治州事日出蹴鞠汭曰存將逃走先匀足力也|{
	詗古迥翻又朝正翻冶直之翻蹴子六翻鞫居六翻蹴蹋也鞠毬也顧師古曰鞠以皮為之實以毛崔豹曰蹴鞠起黄帝習用兵之勢匀于倫翻}
遣兵襲之存棄城走|{
	成汭不見容於張瓌而已又不能容許存忌賢疾能常人之情也}
其衆稍稍歸之屯于茅埧|{
	埧必駕翻蜀人謂平川為埧宋白曰渝州江津縣有茅㙋驛}
趙武數攻豐都王建肇不能守|{
	文德元年王建肇得黔中節今敗走數所角翻}
與存皆降于王建建忌存勇略欲殺之掌書記高燭曰公方總攬英雄以圖霸業彼窮來歸我奈何殺之建使戍蜀州隂使知蜀州王宗綰察之宗綰密言存忠勇謙謹有良將才建乃捨之更其姓名曰王宗播|{
	更工衡翻}
而宗綰竟不使宗播知其免已也宗播元從孔目官柳修業每勸宗播慎静以免禍|{
	從才用翻}
其後宗播為建將遇彊敵諸將所憚者以身先之|{
	先悉薦翻}
及有功輒稱病不自伐由是得以功名終甲午夜顧全武急攻越州乙未旦克其外郭董昌猶

據牙城拒之戊戌鏐遣昌故將駱團紿昌云奉詔令大王致仕歸臨安昌乃送牌印出居清道坊|{
	今越州牙城外東街猶有橋曰清道橋}
己亥全武遣武勇都監使吳璋以舟載昌如杭州至小江南斬之|{
	據新書董昌傳小江西江也蓋錢清江也源出諸暨縣界東流過錢清鎮又東入于海去越州四十五里又西至杭州八十里光啓三年董昌得越州至是而亡監古銜翻}
并其家二百餘人宰相李邈蔣瓌以下百餘人昌在圍城中貪吝日甚口率民間錢帛|{
	計口而率之}
減戰士糧及城破庫有雜貨五百閒倉有糧三百萬斛錢鏐傳昌首於京師散金帛以賞將士開倉以賑貧乏 李克用攻魏博侵掠徧六州|{
	魏博貝衛澶相六州}
朱全忠召葛從周於鄆州使將兵營洹水以救魏博|{
	葛從周汴之騎將也沙陀便於鞍馬故召使敵之}
留龎師古攻鄆州六月克用引兵擊從周汴人多鑿坎於陳前|{
	陳讀曰陣}
戰方酣克用之子鐵林指揮使落落馬遇坎而躓|{
	躓陟利翻}
汴人生擒之 |{
	考異曰唐太祖紀年錄薛居正五代史武皇紀實錄禽落落皆在七月葛從周李存信傳在五月今從梁太祖紀}
克用自往救之馬亦躓幾為汴人所獲克用顧射汴將一人斃之|{
	幾居依翻射而亦翻}
乃得免克用請修好以贖落落全忠不許以與羅弘信使殺之|{
	羅弘信既殺李克用之子則與克用為深仇而汴魏之交益固矣此全忠之術也好呼到翻}
克用引軍還葛從周自洹水引兵濟河屯于楊劉復擊鄆|{
	復扶又翻}
及兖鄆河東之兵戰于故樂亭破之兖鄆屬城皆為汴人所據屢求救於李克用克用發兵赴之為羅弘信所拒不得前兖鄆由是不振 初李克用屯渭北|{
	謂自邠寧還屯渭北時也}
李茂貞韓建憚之事朝廷禮甚恭克用去|{
	謂歸河東也}
二鎮貢獻漸疎表章驕慢上自石門還於神策兩軍之外更置安聖捧宸保寧宣化等軍選補數萬人使諸王將之嗣延王戒丕嗣覃王嗣周又自募麾下數千人茂貞以為欲討己語多怨望嫌隙日構茂貞亦勒兵揚言欲詣闕訟寃京師士民争亡匿山谷上命通王滋及嗣周戒丕分將諸軍以衛近畿戒丕屯三橋茂貞遂表言延王無故稱兵討臣臣今勒兵入朝請罪 |{
	考異曰薛居正五代史五月制授茂貞東川節度使仍命通王覃王治禁軍於闕下如茂貞違詔即討之茂貞懼將赴鎮王師至興平夜自驚潰茂貞因出乘之官軍大敗唐補紀曰五月朝廷除覃王為鳳翔節度使除茂貞為興元節度使茂貞拒命不發亦無向闕之心自是京國人心驚憂出投郊坰京城為之一空上潛謀行幸按實錄新舊紀諸書茂貞未嘗除東川薛史誤移鎮興元乃景福二年事唐補紀誤今從實錄}
上遽遣使告急於河東丙寅茂貞引兵逼京畿覃王與戰於婁館官軍敗績|{
	婁館蓋在京兆興平縣西 考異曰舊紀茂貞請入覲上令通王覃王延王分統四軍以衛近畿丙寅鳳翔軍犯京畿覃王拒之於婁館接戰不利實錄命延王部神策諸軍於三橋防遏茂貞上言延王稱兵討臣臣有何罪言將朝覲丙寅李茂貞大軍犯京師覃王拒之於婁館王師戰不利新紀六月庚戌李茂貞犯京師嗣延王戒丕禦之丙寅及茂貞戰于婁館敗績今從舊紀}
秋七月茂貞進逼京師|{
	果如李克用之言}
延王戒丕曰今關中藩鎮無可依者不若自鄜州濟河幸太原|{
	自鄜州濟河道汾隰至太原路甚回遠以韓建在華州李茂貞養子繼塘在同州不敢由同州出河中也鄜音夫}
臣請先往告之辛卯詔幸鄜州壬辰上出至渭北韓建遣其子從允奉表請幸華州上不許|{
	華戶化翻}
以建為京畿都指揮安撫制置及開通四面道路催促諸道綱運等使而建奉表相繼上及從官亦憚遠去|{
	從才用翻}
癸巳至富平遣宣徽使元公訊召建面議去留甲午建詣富平見上頓首涕泣言方今藩臣跋扈者非止茂貞陛下若去宗廟園陵遠巡邊鄙臣恐車駕濟河無復還期今華州兵力雖微控帶關輔亦足自固臣積聚訓厲十五年矣|{
	按韓建從鹿晏弘至興元之時僖宗在蜀遂奔行在中和四年也僖宗還長安光啓元年也建刺華州當在此時至是纔十二年耳}
西距長安不遠|{
	九域志華州西至長安一百五十里}
願陛下臨之以圖興復上乃從之乙未宿下邽丙申至華州|{
	九域志自富平至下邽三十五里自下邽至華州六十五里}
以府署為行宫建視事於龍興寺茂貞遂入長安自中和以來所葺宫室市肆燔燒俱盡|{
	黄巢之亂宫室燔毁中和以來留守王徽補葺粗完襄王之亂又為亂兵所焚及僖宗還京復加完葺上出石門重罹燒爇還又葺之至是為茂貞所燔}
乙巳以中書侍郎同平章事崔胤同平章事充武安節度使上以胤崔昭緯之黨也故出之 丙午以翰林學士承旨尚書左丞陸扆為戶部侍郎同平章事扆陜人也|{
	扆隱豈翻陜失冉翻}
水部郎中何迎|{
	新書百官志水部郎中屬工部尚書掌津濟船艫渠梁堤堰溝洫漁捕運漕碾磑之事此時惟具官不復能舉其職矣}
表薦國子毛詩博士襄陽朱朴才如謝安|{
	唐制國子監置五經博士各二人掌以其經之學教國子周易尚書毛詩左氏春秋禮記為五經}
道士許巖士亦薦朴有經濟才上連日召對朴有口辯上悦之曰朕雖非太宗得卿如魏徵矣賜以金帛并賜何迎 以徐彦若為大明宫留守兼京畿安撫制置等使楊行密表請上遷都江淮王建請上幸成都|{
	皆欲迎天子挾}


|{
	之以令諸侯}
宰相畏韓建不敢專决政事八月丙辰詔建關議朝政建上表固辭乃止|{
	韓建非避權勢者目不知書故辭耳朝直遥翻上時掌翻}
韓建移檄諸道令共輸資糧詣行在李克用聞之歎曰去歲從余言豈有今日之患|{
	謂欲討李茂貞上不許也}
又曰韓建天下癡物為賊臣弱帝室|{
	為于偽翻}
是不為李茂貞所擒則為朱全忠所虜耳因奏將與鄰道發兵入援|{
	曰將入援亦虛言耳}
加錢鏐兼中書令 癸丑以王建為鳳翔西面行營招討使|{
	欲使王建討李茂貞也}
甲寅以門下侍郎同平章事王摶同平章事充威勝節度使|{
	先是已升浙東觀察使為威勝節度使方鎮表乾寧元年以乾州置威勝軍節度參考下文則朝議以董昌已誅欲以王摶代鎮浙東然則此時藩鎮有兩威勝軍邪}
上憤天下之亂思得奇傑之士不次用之國子博士朱朴自言得為宰相月餘可致太平上以為然乙丑以朴為左諫議大夫同平章事 |{
	考異曰舊傳曰朴腐儒木強無他才伎道士許巖士出入禁中常依朴為姦利從容上前薦朴有經濟才昭宗召見對以經義甚悦即日拜平章事在中書與名公齒筆札議論動為笑端唐補紀曰朴亦有文詞託識諸王下吏人以通意旨言方今宰相皆非時才致今宗社不安頻有傾動若使朴在相位月餘能致太平諸王以為然乃奉天聽翌日宣喚顧問機宜便入中書令參知政事諸相座愕然莫測聽其籌謨經四五月並無所聞遂貶出嶺外按朴雖庸鄙恐不至如舊傳所云唐補史亦恐得之傳聞非詳實今從新傳}
朴為人庸鄙迂僻無他長制出中外大驚 丙寅加韓建兼中書令 九月庚辰升福建為威武軍以觀察使王潮為節度使 以湖南留後馬殷判湖南軍府事殷以高郁為謀主郁揚州人也殷畏楊行密成汭之彊議以金帛結之高郁曰成汭不足畏也行密公之讎|{
	言馬殷從孫儒攻楊行密積年交戰已為仇讎}
雖以萬金賂之安肯為吾援乎不若上奉天子下奉士民訓卒厲兵以修霸業則誰與為敵矣殷從之|{
	史言馬殷能用高郁以保據湖南}
崔胤出鎮湖南|{
	出崔胤為武安節度見上}
韓建之志也胤密求援於朱全忠且教之營東都宫闕表迎車駕全忠與河南尹張全義表請上遷都洛陽全忠仍請以兵二萬迎車駕且言崔胤忠臣不宜出外韓建懼復奏召胤為相遣使諭全忠以且宜安静全忠乃止乙未復以胤為中書侍郎同平章事|{
	崔胤自此與朱全忠相為表裏 考異曰舊傳胤檢校兵部尚書嶺南東道節度使胤密致書全忠求援全忠上書理之胤已至湖南復召拜平章事新傳昭緯以罪誅罷為武安節度使陸扆當國時南北司各樹黨結藩鎮胤素厚朱全忠委心結之全忠為言胤有功不宜處外故還相而逐扆按胤出為清海節度使在後非此年舊傳誤今從實錄}
以翰林學士承旨兵部侍郎崔遠同平章事遠珙弟璵之孫也|{
	崔珙見二百四十六卷開成五年珙居勇翻璵音余}
丁酉貶中書侍郎同平章事陸扆為硤州刺史|{
	考異曰舊傳曰九月覃王率師送徐彦若赴鳳翔師之起也扆堅請曰播越之後國步初集不宜與近輔交惡必為他盜所窺加以親王統兵物議騰口無益於事祇貽後患昭宗已發兵怒扆沮議是月十九日責授硤州刺史師出果敗車駕出幸按此乃景福二年杜讓能討鳳翔事時扆未為相舊傳誤新傳亦同今從實錄}
崔胤恨扆代已|{
	事見上}
誣扆云黨於李茂貞而貶之己亥以朱朴兼判戶部凡軍旅財賦之事上一以委之以孫偓為鳳翔四面行營都統又以前定難節度使李思諫為静難節度使兼副都統|{
	皆欲使之討李茂貞難乃旦翻}
以保大留後李思敬為節度使 河東將李存信攻臨清敗汴將葛從周於宗城北|{
	敗補邁翻下同}
乘勝至魏州北門|{
	九域志臨清縣在魏州北一百五十里宗城縣在魏州西北一百七十里}
冬十月壬子加孫偓行營節度招討處置等使|{
	處昌呂翻}
丁巳以韓建權知京兆尹兼把截使 |{
	考異曰李巨川許國公勤王錄十月十日敕命公權知京兆尹并充把截使實錄作癸丑是月戊申朔今從勤王錄}
戊午李茂貞上表請罪願得自新仍獻助修宫室錢|{
	考異曰舊紀實錄皆云茂貞進錢十五萬助修宫闕按十五萬乃百五十貫太少蓋脫貫字耳}
韓建復佐佑之|{
	復扶又翻下同}
竟不出師 錢鏐令兩浙吏民上表請以鏐兼領浙東朝廷不得已復以王摶為吏部尚書同平章事以鏐為鎮海威勝兩軍節度使|{
	錢鏐自此遂跨有浙東西}
丙子更名威勝曰鎮東軍|{
	更工衡翻}
李克用自將攻魏州敗魏兵於白龍潭|{
	按薛史梁太祖紀乾化元年九月丙辰幸魏縣戊辰幸邑西白龍潭則其地在魏縣西也}
追至觀音門|{
	薛史魏州羅城西門曰觀音門晉天福五年閏三月改日金明門}
朱全忠復遣葛從周救之屯于洹水全忠以大軍繼之克用乃還 加河中節度使王珂同平章事 十一月朱全忠還大梁復遣葛從周東會龎師古攻鄆州 湖州刺史李師悦求旌節詔置忠國軍於湖州以師悦為節度使賜告身旌節者未入境戊子師悦卒楊行密表師悦子前綿州刺史彦徽知州事 |{
	考異曰實錄乾寧二年四月忠國節度使李師悦卒以其孫彦徽知留後今從新紀十國紀年}
淮南將安仁義攻婺州十二月東川兵焚掠漢眉資簡之境|{
	漢眉資簡四州皆西川巡屬}
清海節度使薛王知柔行至湖南廣州牙將盧琚譚弘玘據境拒之使弘玘守端州弘玘結封州刺史劉隱許妻以女隱偽許之託言親迎|{
	玘起里翻妻七細翻迎魚敬翻}
伏甲舟中夜入端州斬弘玘遂襲廣州斬琚|{
	按九域志自封州東南歷康州界而後至端州自端州東至廣州二百四十里}
具軍容迎知柔入視事|{
	具軍容以迎新帥如承平儀注}
知柔表隱為行軍司馬

資治通鑑卷二百六十
