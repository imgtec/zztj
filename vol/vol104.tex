










 


 
 


 

  
  
  
  
  





  
  
  
  
  
 
  

  

  
  
  



  

 
 

  
   




  

  
  


    資治通鑑卷一百四   宋 司馬光 撰

  胡三省 音註

  晉紀二十六【起柔兆困敦盡玄黓敦牂凡七年}


  烈宗孝武皇帝上之中

  太元元年春正月壬寅朔帝加元服皇太后下詔歸政【太后攝政見上卷上年}
復稱崇德太后甲辰大赦改元丙午帝始臨朝【朝直遥翻}
以會稽内史郗愔為鎮軍大將軍都督浙江東五郡諸軍事【浙江東五郡會稽東陽臨海永嘉新安也會工外翻郗丑之翻愔挹淫翻}
徐州刺史桓冲為車騎將軍都督豫江二州之六郡諸軍事【豫州之歷陽淮南廬江安豐襄城及江州之尋陽共六郡騎奇寄翻}
自京口徙鎮姑孰謝安欲以王藴為方伯故先解冲徐州乙卯加謝安中書監録尚書事 二月辛卯秦王堅下詔曰朕聞王者勞于求賢逸於得士【齊桓公用管仲之言}
斯言何其驗也往得丞相常謂帝王易為【易以䜴翻}
自丞相違世鬚髮中白【丞相謂王猛中半也中丁仲翻}
每一念之不覺酸慟今天下既無丞相或政教淪替【替廢也}
可分遣侍臣周巡郡縣問民疾苦 三月秦兵寇南鄉拔之山蠻三萬戶降秦【自春秋之時伊洛以南巴巫漢沔以北大山長谷皆蠻居之文公十六年庸人率羣蠻以叛楚庸則漢之上庸縣也哀公四年楚人襲梁及霍以圍蠻氏執蠻子赤梁則漢河南之梁縣霍則梁縣南之霍陽山也漢高帝用巴渝蠻以定三秦則板楯蠻也後漢祭遵攻新城蠻柏華蠻破霍陽聚則春秋蠻氏之聚落也其後又有巫蠻南郡蠻江夏蠻襄陽以西中廬宜城之西山皆蠻居之所謂山蠻也宋齊以後謂之雍州蠻降戶江翻}
 夏五月甲寅大赦初張天錫之殺張邕也劉肅及安定梁景皆有功【事見一百一卷穆帝升平五年}
二人由是有寵賜姓張氏以為己子使預政事天錫荒于酒色不親庶務黜世子大懷而立嬖妾之子大豫【嬖卑義翻又博計翻}
以焦氏為左夫人人情憤怨從弟從事中郎憲輿櫬切諫不聽【從才用翻櫬初覲翻}
秦王堅下詔曰張天錫雖稱藩受位然臣道未純可遣使持節武衛將軍苟萇左將軍毛盛中書今梁熙步兵校尉姚萇等將兵臨西河【河水過敦煌酒泉張掖郡南武威郡東北為西河使疏吏翻萇仲良翻將即亮翻}
尚書郎閻負梁殊奉詔徵天錫入朝【朝直遥翻}
若有違王命即進師撲討【撲普卜翻}
是時秦步騎十三萬軍司段鏗謂周虓曰以此衆戰誰能敵之【用左傳齊桓公之言鏗丘耕翻虓虚交翻}
虓曰戎狄以來未之有也【周虓拘執於秦其尊本朝之心雖造次不忘也 考異曰虓傳曰呂光征西域堅出餞之戎士二十萬旌旗數百里問虓曰朕衆力何如虓曰戎夷以來未之有也按建元十八年二月虓謀反徙朔方十九年正月呂光發長安故知在伐涼州時今從十六國春秋}
堅又命秦州刺史苟池河州刺史李辯涼州刺史王統帥三州之衆為苟萇後繼【帥讀曰率}
秋七月閻負梁殊至姑臧張天錫會官屬謀之曰今入朝必不返如其不從秦兵必至將若之何禁中録事席仂曰【禁中録事張氏所置使摠録禁中事也仂與力同又音勒}
以愛子為質【質音致}
賂以重寶以退其師然後徐為之計此屈伸之術也衆皆怒曰吾世事晉朝【朝直遥翻}
忠節著于海内今一旦委身賊庭辱及祖宗醜莫大焉且河西天險百年無虞若悉境内精兵右招西域北引匈奴以拒之何遽知其不捷也天錫攘袂大言曰孤計決矣言降者斬【降戶江翻下同}
使謂閻負梁殊曰君欲生歸乎死歸乎殊等辭氣不屈天錫怒縳之軍門命軍士交射之曰射而不中【射而亦翻中竹仲翻}
不與我同心者也其母嚴氏泣曰秦主以一州之地横制天下東平鮮卑南取巴蜀兵不留行汝若降之猶可延數年之命今以蕞爾一隅抗衡大國【蕞徂外翻}
又殺其使者亡無日矣天錫使龍驤將軍馬建帥衆二萬拒秦【驤思將翻帥讀曰率}
秦人聞天錫殺閻負梁殊八月梁熙姚萇王統李辯濟自清石津攻涼驍烈將軍梁濟于河會城降之【驍烈將軍蓋張氏置五代志允吾縣有青巖山水經注湟河至允吾與大河會意者清石津在青巖山之下河會城在二河之會歟驍堅堯翻}
甲申苟萇濟自石城津【闞駰曰石城津在金城西北}
與梁熙會攻纒縮城拔之馬建懼自楊非退屯清塞【水經注逆水出允吾縣之參街谷東南流逕街亭城南又東南逕楊非亭北又東南逕廣武城西據載記楊非在支陽東北三百餘里}
天錫又遣征東將軍掌據帥衆三萬軍于洪池【洪池嶺名在姑臧南掌據晉書作常據當從之}
天錫自將餘衆五萬軍于金昌城【金昌城在赤岸西北}
安西將軍敦煌宋皓言於天錫曰【敦徒門翻}
臣晝察人事夜觀天文秦兵不可敵也不如降之天錫怒貶皓為宣威護軍廣武太守辛章曰【張寔分金城之令居枝陽置廣武郡宋白曰蘭州廣武縣本漢枝陽縣地張駿分晉興置廣武郡}
馬建出于行陳【行戶剛翻陳讀曰陣}
必不為國家用苟萇使姚萇帥甲士三千為前驅庚寅馬建帥萬人迎降餘兵皆散走辛卯苟萇及常據戰于洪池據兵敗馬為亂兵所殺其屬董儒授之以馬據曰吾三督諸軍再秉節鉞八將禁旅十摠禁兵寵任極矣【天錫之攻李儼也常據首破其兵蓋河西推為良將故其言如此}
今卒困於此【卒子恤翻}
此吾之死地也尚安之乎乃就帳免胄西向稽首伏劔而死【稽音啟}
秦兵殺軍司席仂癸巳秦兵入清塞天錫遣司兵趙充哲帥衆拒之【河西張氏置官僚擬于王者而微異其名司兵蓋晉五兵尚書之職也}
秦兵與充哲戰于赤岸大破之【水經注河水自左南而東逕赤岸北亦謂之河夾岸秦州記曰枹罕有河夾岸}
俘斬三萬八千級充哲死天錫出城自戰城内又叛天錫與數千騎奔還姑臧甲午秦兵至姑臧天錫素車白馬面縛輿櫬降于軍門苟萇釋縛焚櫬送于長安【惠帝永寜元年張軌為涼州刺史遂有涼土共九主七十五年而亡櫬初覲翻}
涼州郡縣悉降于秦九月秦王堅以梁熙為涼州刺史鎮姑臧徙豪右七千餘戶於關中餘皆按堵如故封天錫為歸義侯拜北部尚書【秦置北部尚書以掌北蕃}
初秦兵之出也先為天錫築第於長安【為于偽翻}
至則居之以天錫晉興太守隴西彭和正為黄門侍郎【張軌分西平界置晉興郡}
治中從事武興蘇膺【張軌以秦雍移人于姑臧西北置武興郡}
敦煌太守張烈為尚書郎【敦徒門翻}
西平太守金城趙凝為金城太守高昌楊幹為高昌太守【高昌漢車師之高昌壁也張氏始置郡後為高昌國唐以其地置西州}
餘皆隨才擢叙梁熙清儉愛民河右安之【為梁熙為呂光所殺張本}
以天錫武威太守敦煌索泮為别駕【索昔各翻}
宋皓為主簿西平郭護起兵攻秦熙以皓為折衝將軍討平之桓冲聞秦攻涼州遣兖州刺史朱序江州刺史桓石秀與荆州督護桓羆遊軍沔漢為涼州聲援【沔彌兖翻}
又遣豫州刺史桓伊帥衆向壽陽【帥讀曰率下同}
淮南太守劉波汎舟淮泗欲橈秦以救涼【橈奴敎翻}
聞涼州敗沒皆罷兵 初哀帝減田租畝收二升【見一百一卷隆和元年}
乙巳除度田收租之制【度徒洛翻}
王公以下口税米三斛蠲在役之身 冬十月移淮北民於淮南【畏秦也}
 劉衛辰為代所逼求救於秦秦王堅以幽州刺史行唐公洛為北討大都督帥幽冀兵十萬擊代【帥讀曰率}
使并州刺史俱難鎮軍將軍鄧羌尚書趙遷李柔前將軍朱肜前禁將軍張蚝【蚝七吏翻}
右禁將軍郭慶帥步騎二十萬東出和龍西出上郡皆與洛會以衛辰為鄉導洛菁之弟也【秦主健之入關菁有功焉健之垂没也菁以逆誅鄉讀曰嚮}
苟萇之伐涼州也遣揚武將軍馬暉建武將軍杜周帥八千騎西出恩宿邀張天錫走路期會姑臧暉等行澤中值水失期於法應斬有司奏徵下獄【下遐稼翻}
秦王堅曰水春冬耗竭秋夏盛漲此乃苟萇量事失宜【量音良}
非暉等罪今天下方有事宜宥過責功命暉等回赴北軍擊索虜以自贖【代本鮮卑索頭種故謂之索虜索昔各翻}
衆咸以為萬里召將非所以應速【將即亮翻下同}
堅曰暉等喜于免死不可以常事疑也暉等果倍道疾驅遂及東軍【暉等自西方囘故謂伐代之軍為東軍}
十一月己巳朔日有食之 代王什翼犍使白部獨孤部南禦秦兵皆不勝【鮮卑有白部後漢時鮮卑居白山者最為強盛後因曰白部令狐德棻曰魏氏之初三十六部其先伏留屯者與魏俱起為部落大人遂為獨孤部犍居言翻}
又使南部大人劉庫仁將十萬騎禦之庫仁者衛辰之族什翼犍之甥也與秦兵戰於石子嶺【石子嶺當雲中盛樂西南新唐書曰自夏州北渡烏水一百二十里至可朱渾水源又百餘里至石子嶺}
庫仁大敗什翼犍病不能自將乃帥諸部奔陰山之北高車雜種盡叛【李延夀曰高車蓋赤狄之餘種也北方以為高車丁零或云其先匈奴甥也其遷徙隨水草衣皮食肉牛羊畜產並與柔然同唯車輪高大輻數至多因以為號種章勇翻}
四面寇鈔【鈔楚交翻}
不得芻牧什翼犍復度漠南【復扶又翻}
聞秦兵稍退十二月什翼犍還雲中初什翼犍分國之半以授弟孤【事見九十六卷成帝咸康四年}
孤卒子斤失職怨望【不復得國之半故自以為失職而怨卒子恤翻}
世子寔及弟翰早卒【寔卒見上卷簡文帝咸安元年}
寔子珪尚幼慕容妃之子閼婆壽鳩紇根地干力真窟咄皆長【閼於葛翻紇下没翻窟苦骨翻咄當没翻長知兩翻下同慕容妃燕女也什翼犍娶燕女為妃見九十七卷康帝建元二年}
繼嗣未定時秦兵尚在君子津【水經河水南入雲中楨陵縣西北又南過赤城東又南過定襄桐過縣西河水於二縣之間濟有君子之名酈道元注曰昔漢桓帝西幸榆中東行代地洛陽大賈賫金貨隨帝後行夜迷失道往投津長曰子封送之度河賈人卒死津長埋之其子尋求父喪發冢舉尸資貨一無所損其子悉以金與之津長不受事聞于帝曰君子也即名其津為君子濟在雲中城西南二百餘里}
諸子每夜執兵警衛斤因說什翼犍之庶長子寔君曰【說輸芮翻}
王將立慕容妃之子欲先殺汝故頃來諸子每夜戎服以兵遶廬帳【北狄之長居大氈帳環設兵衛氈帳漢人謂之穹廬因曰廬帳}
伺便將發耳【伺相吏翻}
寔君信之遂殺諸弟并弑什翼犍是夜諸子婦及部人奔告秦軍秦李柔張蚝勒兵趨雲中【趨七喻翻}
部衆逃潰國中大亂珪母賀氏以珪走依賀訥訥野干之子也【賀野干見上卷簡文帝咸安元年}
秦王堅召代長史燕鳳問其所以亂故鳳具以狀對堅曰天下之惡一也【左傳載石祁子之言}
乃執寔君及斤至長安車裂之堅欲遷珪於長安鳳固請曰代王初亡羣下叛散遺孫冲幼莫相統攝其别部大人劉庫仁勇而有智鐵弗衛辰狡猾多變【劉衛辰本匈奴鐵弗種李延壽曰鐵弗南單于苗裔衛辰者左賢王去卑之玄孫北人謂父為鮮卑母為鐵弗因以為姓}
皆不可獨任宜分諸部為二令此兩人統之兩人素有深讐其勢莫敢先發俟其孫稍長引而立之是陛下有存亡繼絶之德於代使其子子孫孫永為不侵不叛之臣【用左傳戎子駒支之言}
此安邊之良策也堅從之分代民為二部自河以東屬庫仁自河以西屬衛辰各拜官爵使統其衆賀氏以珪歸獨孤部與南部大人長孫嵩【拓跋鬱律生二子長曰沙莫雄次曰什翼犍沙莫雄為南部大人後改名仁號為拔拔氏生嵩道武以嵩宗室之長改為長孫氏此言長孫所出與前注略不同}
元佗等皆依庫仁行唐公洛以什翼犍子窟咄年長【長知兩翻}
遷之長安堅使窟咄入太學讀書下詔曰張天錫承祖父之資藉百年之業擅命河右叛換偏隅【鄭康成曰叛換猶跋扈也韓詩曰叛換武強也}
索頭世跨朔北中分區域東賓穢貊【穢當作濊}
西引烏孫控弦百萬虎視雲中爰命兩師【兩師謂苟萇伐河西之師行唐公洛伐代之師也}
分討黠虜【黠下八翻}
役不淹歲窮殄二兇俘降百萬【降戶江翻}
闢土九千五帝之所未賓周漢之所未至莫不重譯來王【重直龍翻}
懷風率職有司可速班功受爵【杜預曰班次也受當作授}
戎士悉復之五歲【復方目翻}
賜爵三級於是加行唐公洛征西將軍以鄧羌為并州刺史陽平國常侍慕容紹私謂其兄楷曰秦恃其彊大務勝不休北戍雲中南守蜀漢轉運萬里道殣相望【左傳之言詩云行有死人尚或殣之毛氏曰殣路冢也殣音覲說文曰道中死人人所覆也又餓殍為殣}
兵疲於外民困於内危亡近矣冠軍叔仁智度英拔必能恢復燕祚【秦以慕容垂為冠軍將軍楷紹之叔父也叔仁當作叔父冠古玩翻}
吾屬但當愛身以待時耳【史言鮮卑窺秦有乘釁報復之志}
初秦人既克涼州議討西障氐羌【西障西邊也}
秦王堅曰彼種落雜居【種章勇翻}
不相統壹不能為中國大患宜先撫諭徵其租税若不從命然後討之乃使殿中將軍張旬前行宣慰庭中將軍魏曷飛帥騎二萬七千隨之【庭中將軍秦所置蓋立仗殿庭中者也帥讀曰率騎奇寄翻}
曷飛忿其恃險不服縱兵擊之大掠而歸堅怒其違命鞭之二百斬前鋒督護儲安以謝氐羌氐羌大悦降附貢獻者八萬三千餘落【降戶江翻}
雍州士族先因亂流寓河西者皆聽還本【雍於用翻}
劉庫仁招撫離散恩信甚著奉事拓拔珪恩勤周備不以廢興易意常謂諸子曰此兒有高天下之志必能恢隆祖業汝曹當謹遇之【天下之英雄雖在童穉中固不與羣兒同也}
秦王堅賞其功加廣武將軍給幢麾鼔蓋【幢直江翻}
劉衛辰恥在庫仁之下怒殺秦五原太守而叛【五原漢郡也魏晉省弃其地於荒外秦復置郡隋唐為豐鹽二州}
庫仁擊衛辰破之追至陰山西北千餘里獲其妻子又西擊庫狄部徙其部落置之桑乾川【桑乾縣漢屬代郡晉省孟康曰乾音干拓跋魏後置桑乾郡唐屬朔州善陽縣界魏收志拓跋力微時次南諸部有庫狄部後改為狄氏}
久之堅以衛辰為西單于督攝河西雜類屯代來城【代來城在北河西蓋秦築以居衛辰言自代來者居此城也單音蟬}
 是歲乞伏司繁卒子國仁立【為乞伏國仁乘秦亂據隴西張本}


  二年春高句麗新羅西南夷皆遣使入貢于秦【新羅弁韓苗裔也居漢樂浪地杜佑曰新羅本辰韓種魏時為斬盧國晉宋曰新羅其國在百濟東南五百餘里兼有沃沮不耐韓濊地句如字又音駒麗力知翻使疏吏翻}
趙故將作功曹熊邈屢為秦王堅言石氏宮室器玩之盛堅以邈為將作長史領將作

  丞【晉將作大匠有丞闕    無長史長史蓋秦所置屢為于偽翻}
大脩舟艦兵器飾以金銀頗極精巧【艦戶黯翻}
慕容農私言于慕容垂曰自王猛之死秦之法制日以頹靡今又重之以奢侈【重直用翻}
殃將至矣圖䜟之言行當有驗大王宜結納英傑以承天意時不可失垂笑曰天下事非爾所及【慕容農所見猶紹楷也}
桓豁表兖州刺史朱序為梁州刺史鎮襄陽 秋七月丁未以尚書僕射謝安為司徒安讓不拜復加侍中都督揚豫徐兖青五州諸軍事【復扶又翻}
丙辰征西大將軍荆州刺史桓豁卒冬十月辛丑以桓冲都督江荆梁益寧交廣七州諸軍事領荆州刺史以冲子嗣為江州刺史又以五兵尚書王藴都督江南諸軍事領徐州刺史【江南諸軍謂晉陵諸軍也}
征西司馬領南郡相謝玄為兖州刺史領廣陵相監江北諸軍事【桓豁為征西將軍以玄為司馬監工銜翻}
桓冲以秦人彊盛欲移阻江南【此江南即上明也}
奏自江陵徙鎮上明【晉志上明在漢武陵郡孱陵縣界水經注上明城在枝江縣其地夷敞北據大江江汜枝分東入大江縣治洲上故以枝江為稱杜佑曰上明即今江陵松滋縣西廢大明城桓冲所築也冲疏曰南平孱陵縣界地名上明田土膏良可以資業軍人在吳時樂鄉城以上四十餘里北枕大江西接三峽宋白曰上明城桓冲所築在今松滋縣西}
使冠軍將軍劉波守江陵【冠古玩翻}
諮議參軍楊亮守江夏王藴固讓徐州謝安曰卿居后父之重不應妄自菲薄以虧時遇【時遇謂一時之恩遇也}
藴乃受命初中書郎郗超自以其父愔位遇應在謝安之右而安入掌機權愔優遊散地【郗愔自徐兖二州刺史移鎮會稽郗丑之翻愔挹淫翻散悉亶翻}
常憤邑形于辭色由是與謝氏有隙是時朝廷方以秦寇為憂詔求文武良將可以鎮禦北方者【將即亮翻}
謝安以兄子玄應詔超聞之歎曰安之明乃能違衆舉親玄之才足以不負所舉衆咸以為不然超曰吾嘗與玄共在桓公府【桓公謂桓温超玄同府事見一百一卷哀帝興寧元年}
見其使才雖履展間未嘗不得其任是以知之【履以皮為之屐以木為之屐竭戟翻}
玄募驍勇之士【驍堅堯翻}
得彭城劉牢之等數人以牢之為參軍常領精鋭為前鋒戰無不捷時號北府兵【晉人謂京口為北府謝安破俱難等始兼領徐州號北府兵者史終言之}
敵人畏之 壬寅護軍將軍散騎常侍王彪之卒【散悉亶翻騎奇寄翻}
初謝安欲增修宮室彪之曰中興之初即東府為宮【東府在建康臺城之東}
殊為儉陋蘇峻之亂成帝止蘭臺都坐【蘭臺御史臺也都坐御史臺官會坐之地坐徂卧翻}
殆不蔽寒暑是以更營新宮【見九十四卷成帝咸和五年}
比之漢魏則為儉比之初過江則為侈矣今寇敵方彊豈可大興功役勞擾百姓邪安曰宮室弊陋後人謂人無能彪之曰凡任天下之重者當保國寧家緝熙政事【緝續也熙廣也鄭玄曰緝熙光明也}
乃以脩室屋為能邪安不能奪其議故終彪之之世無所營造 十二月臨海太守郗超卒【臨海本會稽東部都尉治沈約曰前漢都尉治鄞後漢分會稽為吳郡疑是都尉徙治章安孫亮太平二年立臨海郡}
初超黨于桓氏以父愔忠于王室不令知之及病甚出一箱書授門生曰公年尊我死之後若以哀惋害寢食者可呈此箱不爾即焚之既而愔果哀惋成疾【惋烏貫翻}
門生呈箱皆與桓温往反密計愔大怒曰小子死已晩矣遂不復哭【復扶又翻}


  三年春二月乙巳作新宮帝移居會稽王邸【會工外翻}
秦王堅遣征南大將軍都督征討諸軍事守尚書令長樂公丕武衛將軍苟萇尚書慕容暐帥步騎七萬寇襄陽以荆州刺史楊安帥樊鄧之衆為前鋒征虜將軍始平石越帥精騎一萬出魯陽關【南陽郡魯陽縣有魯陽關樂音洛萇仲良翻帥讀曰率下同騎奇寄翻下同}
京兆尹慕容垂揚武將軍姚萇帥衆五萬出南鄉領軍將軍苟池右將軍毛當強弩將軍王顯帥衆四萬出武當會攻襄陽夏四月秦兵至沔北【沔彌兖翻}
梁州刺史朱序以秦無舟檝不以為虞【虞防也備也}
既而石越帥騎五千浮渡漢水序惶駭固守中城越克其外郭獲船百餘艘以濟餘軍【艘蘇遭翻}
長樂公丕督諸將攻中城序母韓氏聞秦兵將至自登城履行【行下孟翻}
至西北隅以為不固帥百餘婢及城中女丁築邪城於其内【邪即斜翻}
及秦兵至西北隅果潰衆移守新城襄陽人謂之夫人城桓冲在上明擁衆七萬憚秦兵之彊不敢進丕欲急攻襄陽苟萇曰吾衆十倍于敵糗糧山積【糗去九翻}
但稍遷漢沔之民于許洛塞其運道【塞悉則翻}
絶其援兵譬如網中之禽何患不獲而多殺將士急求成功哉丕從之慕容垂拔南陽執太守鄭裔與丕會襄陽 秋七月新宮成辛巳帝入居之 秦兖州刺史彭超請攻沛郡太守戴?於彭城【?領沛郡太守戊彭城楊正衛曰?古遁字}
且曰願更遣重將攻淮南諸城為征南棊刼之勢【征南謂苻丕也時督諸軍攻襄陽棊刼者以棊勢喻兵勢也圍棊者攻其右而敵手應之則擊其左取之謂之刼}
東西並進丹陽不足平也【晉都建康漢丹陽秣陵縣地}
秦王堅從之使都督東討諸軍事後將軍俱難右禁將軍毛盛洛州刺史邵保帥步騎七萬寇淮陽盱眙【俱姓也秦初以洛州刺史治陜城晉志曰滅燕之後移洛州治豐陽參考前鄧羌以洛州刺史鎮洛陽則是時洛州刺史猶治洛陽是後北海公重以豫州刺史及平原公暉以豫州牧鎮洛陽洛州刺史始移治豐陽淮陽晉書載記作淮陰當從之淮陰盱眙前漢並屬臨淮郡後漢晉以淮陰屬廣陵}
超越之弟保羌之從弟也【邵羌見一百一卷海西公太和二年從才用翻}
八月彭超攻彭城詔右將軍毛虎生帥衆五萬鎮姑孰以禦秦兵秦梁州刺史韋鍾圍魏興太守吉挹于西城【杜佑曰金州西城縣南九里吉挹於峻山築壘今其山曰魏山}
 九月秦王堅與羣臣飲酒以祕書監朱肜為正【正酒正也肜余中翻}
人以極醉為限祕書侍郎趙整作酒德之歌曰地列酒泉天垂酒池【九州春秋曰曹公禁酒孔融以書嘲之曰天有酒旗之星地列酒泉之郡天文志曰軒轅右角南二星曰酒旗酒官之旗也此曰天垂酒池既曰垂矣池當作旗}
杜康妙識儀狄先知【魏武樂府短歌行云何以解憂唯有杜康注云杜康占之造酒者戰國策曰昔帝女儀狄作酒以進于禹禹飲而甘之遂疏儀狄曰後世必有以酒亡國者}
紂喪殷邦桀傾夏國由此言之前危後則【紂為酒池肉林長夜之飲以亡殷史曰夏桀淫驕乃放鳴條蓋亦以酒也前危後則謂前人之危後人之法則也喪息浪翻夏戶雅翻}
堅大悦命整書之以為酒戒自是宴羣臣禮飲而已【禮臣侍君宴不過三爵}
 秦涼州刺史梁熙遣使入西域揚秦威德冬十月大宛獻汗血馬【使疏吏翻宛於元翻}
秦王堅曰吾嘗慕漢文帝之為人用千里馬何為【文帝却千里馬見十三卷元年}
命羣臣作止馬之詩而反之【反則反之何以作詩為哉此亦好名之過也}
 巴西人趙寶起兵涼州自稱晉西蠻校尉巴郡太守【史言蜀人思晉}
 秦豫州刺史北海公重鎮洛陽謀反秦王堅曰長史呂光忠正必不與之同即命光收重檻車送長安赦之以公就第重洛之兄也 十二月秦御史中丞李柔劾奏長樂公丕等擁衆十萬攻圍小城日費萬金久而無効請徵下廷尉【劾戶槩翻又戶得翻下遐稼翻}
秦王堅曰丕等廣費無成實宜貶戮但師已淹時【淹滯也久留也}
不可虛返其特原之令以成功贖罪使黄門侍郎韋華持節切讓丕等賜丕劔曰來春不捷汝可自裁勿復持面見吾也 周虓在秦密與桓冲書言秦陰計又逃奔漢中秦人獲而赦之【虓虛交翻}
四年春正月辛酉大赦 秦長樂公丕等得詔惶恐乃命諸軍并力攻襄陽秦王堅欲自將攻襄陽【將即亮翻}
詔陽平公融以關東六州之兵會壽春梁熙以河西之兵為後繼陽平公融諫曰陛下欲取江南固當博謀熟慮不可倉猝若止取襄陽又豈足親勞大駕乎未有動天下之衆而為一城者【為于偽翻}
所謂以隨侯之珠彈千仭之雀也【呂氏春秋曰以隨侯之珠彈千仭之雀世必笑之所用重所要輕也搜神記曰隨侯行見大蛇傷救而治之其後蛇含珠以報之徑盈寸純白而夜光可燭堂故歷世稱隨珠焉}
梁熙諫曰晉主之暴未如孫皓江山險固易守難攻【易以䜴翻}
陛下必欲廓清江表亦不過分命將帥【將即亮翻帥所類翻}
引關東之兵南臨淮泗下梁益之卒東出巴峽又何必親屈鸞輅遠幸沮澤乎【沮將豫翻下濕之地曰沮}
昔漢光武誅公孫述晉武帝擒孫皓未聞二帝自統六師親執枹鼓蒙矢石也【光武用岑彭吳漢以滅公孫述晉武帝用王濬王渾以平孫皓苻融梁熙未嘗離所鎮皆上疏以諫枹音膚}
堅乃止詔冠軍將軍南郡相劉波帥衆八千救襄陽波畏秦不敢進朱序屢出戰破秦兵引退稍遠序不設備二月襄陽督護李伯護密遣其子送欵于秦請為内應長樂公丕命諸軍進攻之戊午克襄陽執朱序送長安秦王堅以序能守節拜度支尚書【曹魏置度支尚書度徙洛翻}
以李伯護為不忠斬之秦將軍慕容越拔順陽【晉志曰太康中置順陽郡唐鄧州臨湍菊潭二縣古順陽地}
執太守譙國丁穆堅欲官之穆固辭不受堅以中壘將軍梁成為荆州刺史配兵一萬鎮襄陽選其才望禮而臣之桓冲以襄陽䧟沒上疏送章節【章印也上時掌翻}
請解職不許詔免劉波官俄復以為冠軍將軍 秦以前將軍張蚝為并州刺史【蚝七吏翻}
兖州刺史謝玄帥衆萬餘救彭城【帥讀曰率}
軍于泗口欲遣間使報戴?而不可得【間古莧翻}
部曲將田泓請沒水潜行趣彭城【趣七喻翻}
玄遣之泓為秦人所獲厚賂之使云南軍已敗泓偽許之既而告城中曰南軍垂至我單行來報為賊所得勉之秦人殺之彭超置輜重于留城【留縣城也自漢以來屬彭城郡重直用翻下同}
謝玄揚聲遣後軍將軍何謙向留城超聞之釋彭城圍引兵還保輜重戴?帥彭城之衆隨謙奔玄超遂據彭城 【考異曰謝玄傳云何謙進解彭城圍又云於是罷彭城下邳二戍帝紀及諸傳皆不言此年彭城䧟沒而十六國秦春秋云超據彭城又云超分兵下邳留徐褒守彭城至七月以毛當為徐州刺史鎮彭城王顯為揚州戍下邳是二城俱䧟也}
留兖州治中徐褒守之南攻盱眙【盱眙音吁怡}
俱難克淮陰【南北對境圖曰淮陰縣距淮五十步北對清河口十里進可以窺山東内則蔽沿江晉宋以為重鎮}
留邵保戍之 三月壬戌詔以疆場多虞【場音亦}
年穀不登其供御所須事從儉約九親供給【九親即九族}
衆官廩俸權可減半凡諸役費自非軍國事要皆宜停省 癸未使右將軍毛虎生帥衆三萬擊巴中以救魏興【巴中即巴郡}
前鋒督護趙福等至巴西為秦將張紹等所敗【敗補邁翻}
亡七千餘人虎生退屯巴東蜀人李烏聚衆二萬圍成都以應虎生秦王堅使破虜將軍呂光擊滅之【破虜將軍蓋苻秦所置}
夏四月戊申韋鍾拔魏興吉挹引刀欲自殺左右奪其刀會秦人至執之挹不言不食而死秦王堅歎曰周孟威不屈于前丁彦遠潔已于後吉祖冲閉口而死何晉氏之多忠臣也【周虓字孟威丁穆字彦遠吉挹字祖冲}
挹參軍史頴得歸得挹臨終手疏詔贈益州刺史 秦毛當王顯帥衆二萬自襄陽東會俱難彭超攻淮南五月乙丑難超拔盱眙執高密内史毛璪之【高密僑國也璪之領内史戍盱眙璪子皓翻}
秦兵六萬圍幽州刺史田洛于三阿【晉僑置幽冀青并四州於江北三阿今寶應軍即其地}
去廣陵百里朝廷大震臨江列戍遣征虜將軍謝石帥舟師屯涂中【涂讀曰除}
石安之弟也右衛將軍毛安之等帥衆四萬屯堂邑秦毛當毛盛帥騎二萬襲堂邑安之等驚潰兖州刺史謝玄自廣陵救三阿丙子難超戰敗退保盱眙六月戊子玄與田洛帥衆五萬進攻盱眙難超又敗退屯淮陰玄遣何謙等帥舟師乘潮而上夜焚淮橋【秦作橋于淮水以渡兵上時掌翻}
邵保戰死難超退屯淮北玄與何謙戴?田洛共追之戰于君川【今盱眙縣北六里有君山此蓋君山之川也}
復大破之【復扶又翻}
難超北走僅以身免謝玄還廣陵詔進號冠軍將軍加領徐州刺史【冠古玩翻}
秦王堅聞之大怒秋七月檻車徵超下廷尉【下遐稼翻}
超自殺難削爵為民以毛當為徐州刺史鎮彭城毛盛為兖州刺史鎮湖陸【續漢志湖陸故湖陵章帝更名前漢志曰王莽改曰湖陸今按湖陸縣漢屬山陽郡晉分屬高平國魏收地形志高平縣有湖陵城當在唐兖州任城縣界}
王顯為揚州刺史戍下邳謝安為宰相秦人屢入寇邊兵失利安每鎮之以和静其為政務舉大綱不為小察時人比安于王導而謂其文雅過之 八月丁亥以左將軍王藴為尚書僕射頃之遷丹陽尹藴自以國姻【藴后父也}
不欲在内苦求外出復以為都督浙江東五郡諸軍事會稽内史【藴先督徐州今復督浙東復扶又翻會工外翻}
 是歲秦大饑

  五年春正月秦王堅復以北海公重為鎮北大將軍鎮薊【重謀反而不誅復任之以方面宜其與弟洛反也復扶又翻}
二月作敎武堂于渭城【漢高帝元年改咸陽曰新城武帝元鼎三年更名渭城後漢晉省石勒置石安縣苻秦復曰渭城}
命太學生明陰陽兵法者敎授諸將【將即亮翻}
祕書監朱肜諫曰陛下東征西伐所向無敵四海之地什得其八雖江南未服蓋不足言是宜稍偃武事增修文德乃更始立學舍教人戰鬬之術殆非所以馴致升平也【馴從也言從此而致升平也}
且諸將皆百戰之餘何患不習於兵而更使受敎于書生非所以彊其志氣也此無益於實而有損于名惟陛下圖之堅乃止 秦征北將軍幽州刺史行唐公洛【洛以幽州刺史鎮和龍行唐戰國時趙邑秦以為縣魏晉因之}
勇而多力能坐制奔牛射洞犂耳【犂耳之鐵厚而堅}
自以有滅代之功【滅代見上元年}
求開府儀同三司不得由是怨憤三月秦王堅以洛為使持節都督益寧西南夷諸軍事征南大將軍益州牧【使疏吏翻}
使自伊闕趨襄陽泝漢而上【趨七喻翻上時掌翻}
洛謂官屬曰孤帝室至親【洛苻健兄子也}
不得入為將相而常擯棄邊鄙今又投之西裔復不聽過京師【復扶又翻過古禾翻}
此必有陰計欲使梁成沈孤於漢水耳【梁成時鎮襄陽沈持林翻}
幽州治中平規曰逆取順守湯武是也【漢陸賈曰湯武逆取而順守之}
因禍為福桓文是也【齊桓晉文皆因兄弟爭國得國而霸}
主上雖不為昏暴然窮兵黷武民思有所息肩者十室而九若明公神旗一建必率土雲從今跨據全燕地盡東海北摠烏桓鮮卑東引句麗百濟【燕於賢翻句音駒麗力知翻}
控弦之士不減五十餘萬奈何束手就徵蹈不測之禍乎洛攘袂大言曰孤計决矣沮謀者斬【沮在呂翻}
於是自稱大將軍大都督秦王以平規為幽州刺史玄莬太守吉貞為左長史【莬同都翻}
遼東太守趙讃為左司馬昌黎太守王緼為右司馬遼西太守王琳北平太守皇甫傑牧官都尉魏敷等為從事中郎【漢邊郡有牧官秦置牧官都尉}
分遣使者徵兵于鮮卑烏桓高句麗百濟新羅休忍諸國遣兵三萬助北海公重戍薊諸國皆曰吾為天子守藩【為于偽翻}
不能從行唐公為逆洛懼欲止猶豫未決王緼王琳皇甫傑魏敷知其無成欲告之洛皆殺之吉貞趙讃曰今諸國不從事乖本圖明公若憚益州之行者當遣使奉表乞留【使疏吏翻下同}
主上亦不慮不從平規曰今事形已露何可中止宜聲言受詔盡幽州之兵南出常山陽平公必郊迎因而執之進據冀州【陽平公融以冀州牧鎮鄴平規使洛出中山以臨鄴}
摠關東之衆以圖西土天下可指麾而定也洛從之夏四月洛帥衆七萬發和龍【帥讀曰率下同}
秦王堅召羣臣謀之步兵校尉呂光曰行唐公以至親為逆此天下所共疾願假臣步騎五萬取之如拾遺耳堅曰重洛兄弟據東北一隅兵賦全資未可輕也光曰彼衆迫於凶威一時蟻聚耳若以大軍臨之勢必瓦解不足憂也堅乃遣使讓洛使還和龍當以幽州永為世封洛謂使者曰汝還白東海王【堅本封東海王}
幽州褊狹不足以容萬乘須王秦中以承高祖之業【苻健廟號高祖乘紳證翻王于況翻}
若能迎駕潼關者當位為上公爵歸本國堅怒遣左將軍武都竇衝及呂光帥步騎四萬討之右將軍都貴馳傳詣鄴【都姓貴名鄭公孫閼字子都子孫以為氏傳株戀翻}
將冀州兵三萬為前鋒【將即亮翻}
以陽平公融為征討大都督北海公重悉薊城之衆與洛會屯中山有衆十萬【薊音計}
五月竇衝等與洛戰于中山洛兵大敗生擒洛送長安北海公重走還薊呂光追斬之屯騎校尉石越自東萊帥騎一萬浮海襲和龍斬平規幽州悉平堅赦洛不誅徙涼州之西海郡【漢獻帝興平二年武威太守張雅請置西海郡於居延}


  臣光曰夫有功不賞有罪不誅雖堯舜不能為治【用漢宣帝詔而畧變其文治直吏翻}
況他人乎秦王堅每得反者輒宥之使其臣狃于為逆【狃狎也}
行險徼幸【徼堅堯翻}
雖力屈被擒猶不憂死亂何自而息哉書曰威克厥愛允濟愛克厥威允罔功【書征之辭}
詩云毋縱詭隨以謹罔極式遏寇虐無俾作慝【詩民勞第三章之辭}
今堅違之能無亡乎

  朝廷以秦兵之退為謝安桓冲之功拜安衛將軍與冲皆開府儀同三司 六月甲子大赦 丁卯以會稽王道子為司徒【會工外翻}
固讓不拜 秦王堅召陽平公融為侍中中書監都督中外諸軍事車騎大將軍司隸校尉録尚書事以征南大將軍守尚書令長樂公丕為都督關東諸軍事征東大將軍冀州牧【樂音洛}
堅以諸氐種類繁滋【種章勇翻}
秋七月分三原九嵕武都汧雍氐十五萬戶【九嵕山在漢馮翊雲陽縣界唐在醴泉縣嵕祖紅翻汧苦堅翻雍於用翻}
使諸宗親各領之散居方鎮如古諸侯長樂公丕領氐三千戶以仇池氐酋射聲校尉楊膺為征東左司馬九嵕氐酋長水校尉齊午為右司馬各領一千五百戶為長樂世卿【古者封建諸侯國命卿皆世其官堅分諸宗親散居方鎮各以種類為世卿樂音洛酋慈由翻}
長樂郎中令略陽垣敞為録事參軍【垣氐姓也後隨宋武南歸遂為累世將家}
侍講扶風韋幹為參軍事申紹為别駕膺丕之妻兄也午膺之妻父也八月分幽州置平州【晉書曰按平州禹貢冀州之域于周為幽州界漢屬北平郡後漢末公孫度自號平州牧至孫文懿為魏所㓕因置平州統遼東昌黎玄菟帶方五郡後還合於幽州苻秦滅燕復分幽州置平州公孫淵字文懿唐避高祖諱稱其字}
以石越為平州刺史鎮龍城中書令梁讜為幽州刺史鎮薊城撫軍將軍毛興為都督河秦二州諸軍事河州刺史鎮枹罕【枹音膚}
長水校尉王騰為并州刺史鎮晉陽河并二州各配氐戶三千興騰並苻氏婚姻氐之崇望也平原公暉為都督豫洛荆南兖東豫陽六州諸軍事鎮東大將軍豫州牧鎮洛陽【秦兖州刺史鎮倉垣南兖州鎮湖陸又秦初以豫州刺史鎮許昌㓕燕之後以豫州刺史鎮洛陽於許昌置東豫州陽當作揚按後魏書地形志天平初始置揚州于宜陽苻堅以王顯為揚州刺史戍下邳正屬暉所統}
移洛州刺史治豐陽【苻秦初以洛州刺史鎮陜城荆州刺史鎮豐陽既得襄陽以為荆州徙洛州于豐陽豐陽漢上洛縣地也宋白曰豐陽漢商縣地晉泰始三年分置豐陽縣在豐陽川}
鉅鹿公叡為雍州刺史【雍於用翻}
各配氐戶三千二百堅送丕至灞上諸氐别其父兄皆慟哭哀感路人趙整因侍宴援琴而歌曰【項安世家說伏羲作琴長三尺六寸六分象三百六十六日也廣六寸象六合也文上曰池下曰宕池水平也前廣後狹象尊卑也上圜下方法天地也五弦官也大弦君也寛和而温小弦臣也清亷而不亂文王加三弦合君臣恩也援干元翻杜佑曰世本云琴神農所造琴操云伏羲作琴所以修身理性反其天真白虎通曰琴禁也禁止于邪以正人心也廣雅曰文王武王加二絃以合君臣之恩也揚雄琴清英曰舜彈五絃而天下化堯加二絃以合君臣之恩}
阿得脂阿得脂博勞舅父是仇綏【爾雅鵙伯勞郭璞曰伯勞似鶷鶡而大飛不能翺翔竦翅上下而已廣雅曰伯勞一曰傳勞一名伯趙仇綏不知為何物}
尾長翼短不能飛遠徙種人留鮮卑【謂徙諸氐而留慕容也種章勇翻}
一旦緩急當語誰堅笑而不納【語牛倨翻}
 九月癸未皇后王氏崩 冬十月九真太守李遜據交州反秦王堅以左禁將軍揚壁為秦州刺史尚書趙遷為

  洛州刺史南巴校尉姜宇為寧州刺史【苻秦于南中置南巴校尉}
十一月乙酉葬定皇后于隆平陵 十二月秦以左將軍都貴為荆州刺史鎮彭城【都貴鎮襄陽彭城誤也}
置東豫州以毛當為刺史鎮許昌 是歲秦王堅遣高密太守毛璪之等二百餘人來歸【毛璪之被禽見上四年}


  六年春正月帝初奉佛法立精舍于殿内【後漢書姜肱傳曰就精廬求見徵君賢曰精廬即精舍也蓋以專精講習所業為義今儒釋肄業之地通曰精舍}
引諸沙門居之尚書左丞王雅表諫不從雅肅之曾孫也【王肅仕曹魏以經學著名武帝肅外孫也}
 丁酉以尚書謝石為僕射 二月東夷西域六十二國入貢于秦 夏六月庚子朔日有食之秋七月甲午交阯太守杜瑗斬李遜交州平 冬十

  月故武陵王晞卒于新安【晞徙新安見上卷簡文帝咸安元年}
追封新寧郡王命其子遵為嗣 十一月己亥以前會稽内史郗愔為司空愔固辭不起 秦荆州刺史都貴遣其司馬閻振中兵參軍吳仲帥衆二萬寇竟陵【竟陵侯國前漢屬江夏郡惠帝分立竟陵郡}
桓冲遣南平太守桓石䖍衛軍參軍桓石民等帥水陸二萬拒之【帥讀曰率}
石民石䖍之弟也十二月甲辰石䖍襲擊振仲大破之振仲退保管城石䖍進攻之癸亥拔管城【據載記石䖍襲破振仲于滶水振仲退保管城又據水經沔水逕郡縣故城南又東滶水注之滶水西南注于沔寔曰滶口沔水又南逕石城西城因山為固晉竟陵郡所治也以此攷之管城當在滶水北}
獲振仲斬首七千級俘虜萬人詔封桓冲子謙為宜陽侯以桓石䖍領河東太守【沈約曰成帝咸康三年征西將軍庾亮以司州僑戶立南河東郡屬荆州五代志南郡松滋縣江左置河東郡}
 是歲江東大饑

  七年秦大司農東海公陽員外散騎侍郎王皮【晉職官志散騎侍郎四人魏初與散騎常侍同置員外散騎侍郎晉武帝置散悉亶翻騎奇寄翻}
尚書郎周虓謀反【虓虛交翻}
事覺收下廷尉【下遐稼翻}
陽法之子皮猛之子也秦王堅問其反狀陽曰臣父哀公死不以罪【法死見一百卷穆帝升平元年}
臣為父復讐耳【為于偽翻下嘗為同}
堅泣曰哀公之死事不在朕卿豈不知之王皮曰臣父丞相有佐命之勲而臣不免貧賤故欲圖富貴耳堅曰丞相臨終託卿以十具牛為治田之資未嘗為卿求官【治直之翻為于偽翻}
知子莫若父何其明也周虓曰虓世荷晉恩生為晉臣死為晉鬼復何問乎【荷下可翻復扶又翻下同}
先是虓屢謀反叛【先悉薦翻}
左右皆請殺之堅曰孟威烈士秉志如此豈憚死乎殺之適足成其名耳皆赦不誅徙陽于涼州之高昌郡【徵諸晉志河西張氏未嘗置高昌郡苻堅之平河西也以高昌楊幹為高昌太守疑張氏置是郡苻氏因之高昌即漢車師後部高昌壁之地注又見後}
皮虓于朔方之北虓卒于朔方【卒子恤翻}
陽勇力兼人尋復徙鄯善及建元之末秦國大亂【建元十九年堅伐晉而敗秦遂以亂二十年堅死是建元十八年也復扶又翻鄯上扇翻}
陽刼鄯善之相欲求東歸鄯善王殺之【史終言之}
 秦王堅徙鄴銅駞銅馬飛亷翁仲于長安【石虎所置于鄴者}
 夏四月堅扶風太守王永為幽州刺史【堅下當有以字}
永皮之兄也皮凶險無行而永清修好學【行下孟翻好呼到翻}
故堅用之以陽平公融為司徒融固辭不受堅方謀伐晉乃以融為征南大將軍開府儀同三司五月幽州蝗生廣袤千里【廣古曠翻}
秦王堅使散騎常侍彭城劉蘭發幽冀青并民撲除之【撲普卜翻}
 秋八月癸卯大赦 秦王堅以諫議大夫裴元略為巴西梓潼二郡太守使密具舟師【欲祖王濬之故智順流東下而伐晉也}
 九月車師前部王彌窴鄯善王休密馱【窴堂見翻馱堂何翻}
入朝于秦【朝直遥翻}
請為鄉導以伐西域之不服者【鄉讀曰嚮}
因如漢法置都護以統理之秦王堅以驍騎將軍呂光為使持節都督西域征討諸軍事【驍堅堯翻騎奇寄翻下同使疏吏翻}
與凌江將軍姜飛【凌江將軍晉文王所置以授羅憲}
輕車將軍彭晃將軍杜進康盛等【杜進康盛位至將軍未有將軍號}
摠兵十萬鐵騎五千以伐西域陽平公融諫曰西域荒遠得其氏不可使得其地不可食漢武征之得不補失【謂漢武伐大宛破樓蘭姑師田車師也}
今勞師萬里之外以踵漢氏之過舉臣竊惜之不聽 桓冲使揚威將軍朱綽擊秦荆州刺史都貴于襄陽焚踐沔北屯田掠六百餘戶而還【沔彌兖翻}
冬十月秦王堅會羣臣于太極殿議曰自吾承業垂三十載【堅以升平元年自立至是凡二十六年惟年之久長懼于不終尚庶幾焉乃欲疲民以逞宜其亡也}
四方略定唯東南一隅未霑王化今略計吾士卒可得九十七萬吾欲自將以討之【將即亮翻}
何如祕書監朱肜曰陛下恭行天罰必有征無戰晉主不銜璧軍門則走死江海陛下返中國士民使復其桑梓【謂永嘉之末避亂南渡之子孫也}
然後囘輿東巡告成岱宗【杜佑曰岱宗東岳也特謂太山為岱宗者以其處東北居寅丑之間萬物終始之地陰陽交代之所為衆山之宗故曰岱宗}
此千載一時也【載子亥翻}
堅喜曰是吾志也尚書左僕射權翼曰昔紂為無道三仁在朝武王猶為之旋師【論語微子去之箕子為之奴比干諫而死孔子曰殷有三仁焉史記武王即位九年東觀兵至于盟津諸侯不期而會者八百皆曰紂可伐矣武王曰未可也乃還師居二年紂暴虐滋甚殺王子比干囚箕子微子奔周武王告諸侯曰殷有重罪不可不伐遂滅之朝直遥翻猶為干偽翻}
今晉雖微弱未有大惡謝安桓冲皆江表偉人君臣輯睦内外同心以臣觀之未可圖也堅嘿然良久曰諸君各言其志太子左衛率石越曰今歲鎮守斗福德在吳【歲木星鎮土星斗牛女吳越揚州分}
伐之必有夭殃且彼據長江之險民為之用殆未可伐也堅曰昔武王伐紂逆歲違卜【荀子曰武王之誅紂也東面而迎太歲揚倞注曰迎謂逆太歲也尸子曰武王伐紂魚辛諫曰歲在北方不可北征武王不從史記齊世家武王將伐紂卜龜兆不吉風雨暴至羣公盡懼唯太公彊之勸武王武王遂行}
天道幽遠未易可知【易以䜴翻}
夫差孫皓皆保據江湖不免于亡今以吾之衆投鞭于江足斷其流【斷丁管翻}
又何險之足恃乎對曰三國之君皆淫虐無道【三國之君謂紂夫差孫皓}
故敵國取之易于拾遺【易以豉翻}
今晉雖無德未有大罪願陛下且案兵積穀以待其釁于是羣臣各言利害久之不決堅曰此所謂築舍道傍無時可成【詩曰如彼築室于道謀是用不潰于成}
吾當内斷於心耳【斷丁亂翻}
羣臣皆出獨留陽平公融謂之曰自古定大事者不過一二臣而已今衆言紛紛徒亂人意吾當與汝決之對曰今伐晉有三難天道不順一也晉國無釁二也我數戰兵疲【數所角翻}
民有畏敵之心三也羣臣言晉不可伐者皆忠臣也願陛下聽之堅作色曰汝亦如此吾復何望【復扶又翻}
吾彊兵百萬資仗如山吾雖未為令主亦非闇劣【劣弱也}
乘累捷之勢擊垂亡之國何患不克豈可復留此殘寇使長為國家之憂哉【漢魏相有言恃國家之大矜人民之衆欲見威于敵者謂之驕兵兵驕者滅其苻堅之謂歟復扶又翻下復留同}
融泣曰晉未可滅昭然甚明今勞師大舉恐無萬全之功且臣之所憂不止于此陛下寵育鮮卑羌羯布滿畿甸此屬皆我之深仇太子獨與弱卒數萬留守京師臣懼有不虞之變生于腹心肘腋不可悔也臣之頑愚誠不足采王景畧一時英傑陛下常比之諸葛武侯【諸葛亮諡武侯}
獨不記其臨沒之言乎【見上卷寧康三年}
堅不聽于是朝臣進諫者衆堅曰以吾擊晉校其彊弱之埶猶疾風之掃秋葉而朝廷内外皆言不可誠吾所不解也【朝直遥翻下同解戶買翻曉也}
太子宏曰今歲在吳分【分扶問翻}
又晉君無罪若大舉不捷恐威名外挫財力内竭此羣下所以疑也堅曰昔吾滅燕亦犯歲而捷天道固難知也秦滅六國六國之君豈皆暴虐乎冠軍京兆尹慕容垂【冠軍即冠軍將軍也晉書載記所書率書將軍號而不繫將軍通鑑因之冠古玩翻}
言于堅曰弱併于彊小併於大此理勢自然非難知也以陛下神武應期威加海外虎旅百萬韓白滿朝【韓白謂韓信白起言秦多良將也}
而蕞爾江南【蕞徂外翻小也}
獨違王命豈可復留之以遺子孫哉【復扶又翻遺于季翻}
詩云謀夫孔多是用不集【詩小旻之辭}
陛下斷自聖心足矣【斷丁亂翻}
何必廣詢朝衆晉武平吳所仗者張杜二三臣而已若從朝衆之言豈有混壹之功【謂張華杜預也事見八十卷武帝咸寧五年朝直遥翻}
堅大悦曰與吾共定天下者獨卿而已賜帛五百匹堅鋭意欲取江東寢不能旦陽平公融諫曰知足不辱知止不殆【老子道德經立戒篇之辭}
自古窮兵極武未有不亡者且國家本戎狄也正朔會不歸人【會要也言大要中國正朔相傳不歸夷狄也}
江東雖微弱僅存然中華正統天意必不絶之堅曰帝王歷數豈有常邪惟德之所在耳劉禪豈非漢之苗裔邪終為魏所滅汝所以不如吾者正病此不達變通耳堅素信重沙門道安【道安在襄陽堅破襄陽輿而致之}
羣臣使道安乘間進言【間古莧翻}
十一月堅與道安同輦遊于東苑堅曰朕將與公南遊吳越泛長江臨滄海不亦樂乎【樂音洛}
安曰陛下應天御世居中土而制四維自足比隆堯舜何必櫛風沐雨經畧遐方乎且東南卑濕沴氣易構【沴音戾五行之氣相克勝則為沴氣}
虞舜遊而不歸大禹往而不復【虞舜南巡狩崩于蒼梧之野禹東巡狩至于會稽而崩}
何足以上勞大駕也堅曰天生烝民而樹之君使司牧之朕豈敢憚勞使彼一方獨不被澤乎【被皮義翻}
必如公言是古之帝王皆無征伐也道安曰必不得已陛下宜駐蹕洛陽遣使者奉尺書于前諸將摠六師于後彼必稽首入臣不必親涉江淮也【稽音啟}
堅不聽堅所幸張夫人諫曰妾聞天地之生萬物聖王之治天下【治直之翻}
皆因其自然而順之故功無不成是以黄帝服牛乘馬因其性也【言因牛馬之性故可引重而致遠}
禹濬九川障九澤因其勢也【言因高下之勢故可滌源而陂澤}
后稷播殖百穀因其時也【因天時而播殖則百穀成}
湯武帥天下而攻桀紂因其心也【因人心而用兵則天下服帥讀曰率}
皆有因則成無因則敗今朝野之人皆言晉不可伐陛下獨決意行之妾不知陛下何所因也書曰天聰明自我民聰明【書臯陶謨之辭}
天猶因民而况人乎妾又聞王者出師必上觀天道下順人心今人心既不然矣請驗之天道諺云雞夜鳴者不利行師犬羣嘷者宮室將空【嘷戶刀翻}
兵動馬驚軍敗不歸自秋冬以來衆雞夜鳴羣犬哀嘷廄馬多驚武庫兵器自動有聲此皆非出師之祥也堅曰軍旅之事非婦人所當預也堅幼子中山公詵最有寵亦諫曰臣聞國之興亡繫賢人之用捨今陽平公國之謀主而陛下違之晉有謝安桓冲而陛下伐之臣竊惑之堅曰天下大事孺子安知 秦劉蘭討蝗經秋冬不能滅十二月有司奏徵蘭下廷尉【下遐稼翻}
秦王堅曰災降自天非人力所能除此由朕之失政蘭何罪乎是歲秦大熟上田畝收七十石下者三十石蝗不出幽州之境不食麻豆上田畝收百石下者五十石【物反常為妖蝗之為災尚矣蝗生而不食五穀妖之大者也農人服田力穡至于有秋自古以來未有畝收百石七十石之理而畝收五十石三十石亦未之聞也使其誠有之又豈非反常之大者乎使其無之則州縣相與誣飾以罔上亦不祥之大者也秦亡宜矣}


  資治通鑑卷一百四


    


 


 



 

 
  







 


  
  
 
 
 


  

 















	
	









































 
  



















 





 












  
  
  

 





