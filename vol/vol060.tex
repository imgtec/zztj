資治通鑑卷六十    宋 司馬光 撰

胡三省 音註

漢紀五十二|{
	起重光協洽盡昭陽作噩凡三年}


孝獻皇帝乙

初平二年春正月辛丑赦天下 關東諸將議以朝廷幼冲廹於董卓遠隔關塞|{
	關塞謂函谷關桃林塞也}
不知存否幽州牧劉虞宗室賢儁欲共立為主曹操曰吾等所以舉兵而遠近莫不響應者以義動故也今幼主微弱制於姦臣非有昌邑亡國之釁|{
	昌邑謂昌邑王賀也}
而一旦改易天下其孰安之諸君北面我自西向|{
	幽州在北長安在西故操云然}
韓馥袁紹以書與袁術曰帝非孝靈子欲依絳灌誅廢少主迎立代王故事|{
	少詩照翻}
奉大司馬虞為帝術陰有不臣之心不利國家有長君|{
	長知兩翻}
乃外託公義以拒之紹復與術書曰今西名有幼君無血脉之屬|{
	謂帝非靈帝子也復扶又翻下同}
公卿以下皆媚事卓安可復信|{
	復扶又翻下同}
但當使兵往屯關要皆自蹙死東立聖君太平可冀如何有疑又室家見戮不念子胥|{
	謂子胥能報父兄之讎也}
可復北面乎|{
	以殺袁隗等為出於帝}
術答曰聖主聰叡有周成之質賊卓因危亂之際威服百寮此乃漢家小厄之會乃云今上無血脉之屬豈不誣乎又曰室家見戮可復北面此卓所為豈國家哉慺慺赤心|{
	慺力侯翻}
志在滅卓不識其他馥紹竟遣故樂浪太守張岐等齎議上虞尊號|{
	樂浪音洛琅上時掌翻}
虞見岐等厲色叱之曰今天下崩亂主上蒙塵吾被重恩|{
	被皮義翻}
未能清雪國恥諸君各據州郡宜共戮力盡心王室而反造逆謀以相垢汙邪|{
	汙烏故翻}
固拒之馥等又請虞領尚書事承制封拜復不聽欲犇匈奴以自絶紹等乃止 二月丁丑以董卓為太師位在諸侯王上 孫堅移屯梁東為卓將徐榮所敗|{
	敗補邁翻}
復收散卒進屯陽人|{
	賢曰梁縣屬河南郡今汝州縣陽人聚故城在梁縣西}
卓遣東郡太守胡軫督步騎五千擊之以呂布為騎督軫與布不相得堅出擊大破之梟其都督華雄|{
	梟古堯翻華戶化翻}
或謂袁術曰堅若得雒不可復制此為除狼而得虎也術疑之不運軍糧堅夜馳見術|{
	陽人去魯陽百餘里}
畫地計校曰所以出身不顧者上為國家討賊|{
	為于偽翻}
下慰將軍家門之私讎堅與卓非有骨肉之怨也而將軍受浸潤之言|{
	浸潤之譖出論語}
還相嫌疑何也術踧踖|{
	踧子六翻踖資昔翻踧踖不自安貌}
即調發軍糧|{
	調徒釣翻}
堅還屯卓遣將軍李傕說堅|{
	傕克角翻說輸芮翻後同}
欲與和親令堅疏子弟任刺史郡守者許表用之堅曰卓逆天無道今不夷汝三族縣示四海|{
	縣讀曰懸}
則吾死不瞑目|{
	瞑莫定翻}
豈將與乃和親邪|{
	乃汝也}
復進軍大谷距雒九十里|{
	賢曰大谷口在故嵩陽西北八十五里北出對雒陽故城張衡東京賦云盟津逹其後大谷通其前是也距至也}
卓自出與堅戰於諸陵間卓敗走卻屯澠池聚兵於陜|{
	澠彌兖翻陜式冉翻}
堅進至雒陽擊呂布復破走堅乃掃除宗廟祠以太牢得傳國璽於城南甄官井中|{
	甄官署之井中也晉職官志少府之屬冇甄官令而續漢志無之蓋屬於他署未置專官也甄官掌琢石陶土之事為後建安元年袁術奪璽張本璽斯氏翻}
分兵出新安澠池間以要卓卓謂長史劉艾曰關東軍敗數矣|{
	要一遥翻數所角翻}
皆畏孤無能為也惟孫堅小戇|{
	說文曰戇愚也音都降翻}
頗能用人當語諸將使知忌之|{
	語牛倨翻下同}
孤昔與周愼西征邊韓於金城孤語張温求引所將兵為慎作後駐温不聽温又使孤討先零叛羌|{
	零音隣}
孤知其不克而不得止遂行留别部司馬劉靖將步騎四千屯安定以為聲勢叛羌欲歸道|{
	即截字}
孤小擊輒開畏安定有兵故也虜謂安定當數萬人不知但靖也而孫堅隨周愼行謂愼求先將萬兵造金城|{
	將即亮翻造七到翻}
使愼以二萬作後駐邊韓畏愼大兵不敢輕與堅戰而堅兵足以斷其運道|{
	斷丁管翻}
兒曹用其言凉州或能定也温既不能用孤愼又不能用堅卒用敗走|{
	事見五十八卷靈帝中平二年卒子恤翻}
堅以佐軍司馬所見畧與人同固自為可|{
	言其才可用也}
但無故從諸袁兒終亦死耳乃使東中郎將董越屯澠池中郎將段煨屯華陰|{
	煨烏回翻華戶化翻}
中郎將牛輔屯安邑|{
	姓譜牛木自殷周封微子於宋其裔司寇牛父敗狄於長丘死之其子孫以王父字為氏}
其餘諸將布在諸縣以禦山東輔卓之壻也卓引還長安孫堅修塞諸陵|{
	塞悉則翻}
引軍還魯陽 夏四月董卓至長安公卿皆迎拜車下卓抵手謂御史中丞皇甫嵩曰義眞怖未乎|{
	皇甫嵩字義真怖普布翻}
嵩曰明公以德輔朝廷大慶方至何怖之有若淫刑以逞將天下皆懼豈獨嵩乎卓黨欲尊卓比太公稱尚父卓以問蔡邕邕曰明公威德誠為巍巍然比之太公愚意以為未可宜須關東平定車駕還反舊京然後議之卓乃止卓使司隸校尉劉囂籍吏民有為子不孝為臣不忠為吏不清為弟不順者皆身誅財物没官於是更相誣引|{
	更工衡翻}
寃死者以千數百姓囂囂道路以目|{
	囂五羔翻韋昭曰不敢發言以目相眄而已}
六月丙戌地震 秋七月司空种拂免以光禄大夫

濟南淳于嘉為司空|{
	濟子禮翻}
太尉趙謙罷以太常馬日磾為太尉|{
	磾丁奚翻}
初何進遣雲中張楊還并州募兵會進敗楊留上黨有衆數千人袁紹在河内楊往歸之與南單于於扶羅屯漳水|{
	濁漳水出上黨長子而東過鄴鄴則韓馥所居也}
韓馥以豪傑多歸心袁紹忌之陰貶節其軍糧欲使其衆離散會馥將麴義叛|{
	姓譜漢有平原鞠譚其子閟避難改曰麴氏後遂為西平著姓}
馥與戰而敗紹因與義相結紹客逢紀謂紹曰|{
	逢蒲江翻}
將軍舉大事而仰人資給|{
	仰牛向翻}
不據一州無以自全紹曰冀州兵強吾士饑乏設不能辦無所容立紀曰韓馥庸才可密要公孫瓚|{
	要讀曰邀}
使取冀州馥必駭懼因遣辯士為陳禍福|{
	為于偽翻}
馥迫於倉卒|{
	卒讀曰猝}
必肯遜讓紹然之即以書與瓚瓚遂引兵而至外託討董卓而陰謀襲馥馥與戰不利會董卓入關紹還軍延津|{
	續漢志酸棗縣北有延津}
使外甥陳留高幹及馥所親潁川辛評荀諶郭圖等說馥曰|{
	諶時壬翻說輸芮翻}
公孫瓚將燕代之卒乘勝來南而諸郡應之其鋒不可當袁車騎引軍東向|{
	自河内至延津為東向}
其意未可量也|{
	量音良}
竊為將軍危之|{
	為于偽翻}
馥懼曰然則為之奈何諶曰君自料寛仁容衆為天下所附孰與袁氏馥曰不如也臨危吐决|{
	吐决謂吐奇决策也}
智勇過人又孰與袁氏馥曰不如也世布恩德天下家受其惠又孰與袁氏馥曰不如也諶曰袁氏一時之傑將軍資三不如之勢久處其上|{
	處昌呂翻}
彼必不為將軍下也夫冀州天下之重資也彼若與公孫瓚并力取之危亡可立而待也夫袁氏將軍之舊且為同盟|{
	謂同盟討董卓}
當今之計若舉冀州以讓袁氏彼必厚德將軍瓚亦不能與之争矣是將軍有讓賢之名而身安於泰山也馥性恇怯因然其計|{
	恇去王翻}
馥長史耿武别駕閔純治中李歷聞而諫曰 |{
	考異曰九州春秋作耿彧今從范書魏志袁紀又范書騎都尉沮授諫無李歷今從魏志袁紀}
冀州帶甲百萬穀支十年袁紹孤客窮軍仰我鼻息|{
	鼻息氣一出入之頃也鼻氣嘘之則温吸之則寒故云然醫書云血為脉氣為息脉息之名自是而分呼吸者氣之槖籥動應者血之波瀾其經以身寸度之計十六丈二尺一呼脉再動一吸脉再動呼吸定息脉五動閏以大息則六動一動一寸故一息脉行六寸十息六尺百息六大二百息十二丈七十息四丈二尺計二百七十息漏水下二刻盡十六丈二尺營周一身百刻之中得五十營故曰脉行陽二十五度行陰二十五度也息者以呼吸定之一日計一萬三千五百息呼吸進退既遲於脉故一日一夜方行盡十六丈二尺經絡而氣周於一身大會於風府脉屬陰陰行速猶太陰一月一周天息屬陽陽行遲猶太陽一歲一周天如是則應天常度閏當作間}
譬如嬰兒在股掌之上絶其哺乳立可餓殺奈何欲以州與之馥曰吾袁氏故吏且才不如本初度德而讓|{
	度徒洛翻}
古人所貴諸君獨何病焉先是馥從事趙浮程渙將彊弩萬張屯孟津|{
	先悉薦翻將即亮翻}
聞之率兵馳還時紹在朝歌清水|{
	據水經清水出河内脩武縣逕獲嘉汲縣而入于河不至朝歌惟淇水則逕朝歌耳蓋俗亦呼淇水為清水據九州春秋紹時在朝歌清水口浮等自孟津東下則兩軍皆舟行大河而向鄴也清水口即淇口南㟁即延津}
浮等從後來船數百艘衆萬餘人整兵鼓夜過紹營紹甚惡之|{
	惡烏路翻}
浮等到謂馥曰袁本初軍無斗糧各已離散雖有張楊於扶羅新附未肯為用不足敵也小從事等請以見兵拒之|{
	見賢遍翻}
旬日之間必土崩瓦解明將軍但當開閣高枕|{
	枕職任翻}
何憂何懼馥又不聽乃避位出居中常侍趙忠故舍遣子送印綬以讓紹紹將至從事十人爭棄馥去獨耿武閔純杖刀拒之不能禁乃止紹皆殺之紹遂領冀州牧承制以馥為奮威將軍而無所將御|{
	將即亮翻將御猶言統御也}
亦無官屬紹以廣平沮授為奮武將軍|{
	廣平縣屬鉅鹿郡沮千余翻又音諸姓也黄帝史官沮誦之後}
使監護諸將寵遇甚厚|{
	監古銜翻}
魏郡審配鉅鹿田豐竝以正直不得志於韓馥紹以豐為别駕配為治中及南陽許攸逢紀潁川荀諶皆為謀主紹以河内朱漢為都官從事|{
	紹置都官從事則猶領司隸校尉也}
漢先為韓馥所不禮且欲徼迎紹意|{
	徼一遥翻}
擅發兵圍守馥第拔刃登屋馥走上樓|{
	上時掌翻}
收得馥大兒槌折兩腳|{
	折而設翻}
紹立收漢殺之馥猶憂怖從紹索去|{
	怖普布翻索山客翻}
往依張邈後紹遣使詣邈有所計議與邈耳語|{
	耳語附耳而語也}
馥在坐上|{
	坐徂卧翻}
謂為見圖無何起至溷以書刀自殺|{
	溷戶困翻圊也厠也時雖已有紙猶多用刀筆書故有書刀}
鮑信謂曹操曰袁紹為盟主因權專利將自生亂是復有一卓也|{
	復扶又翻}
若抑之則力不能制祇以遘難|{
	遘與構同難乃旦翻}
且可規大河之南以待其變操善之會黑山于毒白繞眭固等十餘萬衆略東郡王肱不能禦曹操引兵入東郡擊白繞於濮陽破之|{
	眭息為翻濮博木翻}
袁紹因表操為東郡太守治東武陽|{
	東武陽縣屬東郡應劭曰縣在武水之陽水經註曰武水即漯水賢曰故城在今魏州莘縣南守式又翻}
南單于劫張楊以叛袁紹屯於黎陽董卓以楊為建義將軍河内太守 太史望氣言當有大臣戮死者董卓使人誣衛尉張温與袁術交通冬十月壬戌笞殺温於市以應之|{
	張温不能斬卓於西征之時死於卓手可哀也已}
青州黄巾寇勃海衆三十萬欲與黑山合公孫瓚率步騎二萬人逆擊於東光南大破之|{
	東光縣屬勃海賢曰今滄州縣}
斬首三萬餘級賊棄其輜重|{
	重直用翻}
犇走度河瓚因其半濟薄之賊復大破|{
	復扶又翻下同}
死者數萬流血丹水|{
	言水為之丹也}
收得生口七萬餘人車甲財物不可勝筭|{
	勝音升}
威名大震 劉虞子和為侍中帝思東歸使和偽逃董卓濳出武關詣虞令將兵來迎 |{
	考異曰范書劉虞傳虞使田疇使長安時和為侍中因遣從武關出按魏志公孫瓚傳但云天子思歸不云因疇至也若爾當令和與疇俱還不應出武關又疇未還劉虞已死虞死在初平四年冬界橋戰在三年春范書誤也}
和至南陽袁術利虞為援留和不遣許兵至俱西令和為書與虞虞得書遣數千騎詣和公孫瓚知術有異志止之虞不聽瓚恐術聞而怨之亦遣其從弟越將千騎詣術|{
	從才用翻下同}
而陰教術執和奪其兵由是虞瓚有隙|{
	虞先與瓚有隙至是而隙愈深}
和逃術來北復為袁紹所留是時關東州郡務相兼并以自彊大袁紹袁術亦自離貳術遣孫堅擊董卓未返紹以會稽周昂為豫州刺史襲奪堅陽城|{
	陽城縣屬潁川郡堅領豫州刺史屯陽城}
堅歎曰同舉義兵將救社稷逆賊垂破而各若此吾當誰與戮力乎引兵擊昂走之袁術遣公孫越助堅攻昂越為流矢所中死|{
	中竹仲翻}
公孫瓚怒曰余弟死禍起於紹遂出軍屯磐河|{
	水經大河故瀆東北逕西平昌縣故城北分?東入般縣為般河余據賢註般音卜滿翻此作磐讀當如字賢又曰般即爾雅九河鉤磐河也其枯河在今滄州樂陵縣東南魏儀地形志安德郡般縣有故般河}
上書數紹罪惡|{
	數所具翻}
進兵攻紹冀州諸城多叛紹從瓚紹懼以所佩勃海太守印綬授瓚從弟範遣之郡而範遂背紹領勃海兵以助瓚|{
	背蒲妺翻}
瓚乃自署其將帥嚴綱為冀州刺史田楷為青州刺史單經為兖州刺史|{
	單音善姓也姓譜周卿士單襄公之後}
又悉改置郡縣守令初涿郡劉備中山靖王之後也|{
	蜀書云備中山靖王勝子陸城亭侯貞之後然自祖父以上世系不可攷}
少孤貧與母以販履為業|{
	少詩照翻}
長七尺五寸垂手下䣛顧自見其耳|{
	長直亮翻䣛與膝同言其有異相也}
有大志少語言喜怒不形於色|{
	少詩照翻}
嘗與公孫瓚同師事盧植由是往依瓚瓚使備與田楷徇青州有功因以為平原相備少與河東關羽涿郡張飛相友善|{
	少詩照翻}
以羽飛為别部司馬分統部曲備與二人寢則同牀恩若兄弟而稠人廣坐|{
	坐徂卧翻}
侍立終日隨備周旋不避艱險常山趙雲為本郡將吏兵詣公孫瓚|{
	為于偽翻將即亮翻}
瓚曰聞貴州人皆願袁氏|{
	願下當有從字}
尹何獨迷而能反乎雲曰天下訩訩|{
	訩許容翻衆語喧嘵之貌}
未知孰是民有倒縣之厄|{
	縣讀曰懸}
鄙州論議從仁政所在不為忽袁公私明將軍也|{
	為于偽翻下同}
劉備見而奇之深加接納雲遂從備至平原為備主騎兵|{
	劉備事始此}
初袁術之得南陽戶口數百萬而術奢淫肆欲徵斂無度|{
	歛力贍翻}
百姓苦之稍稍離散既與袁紹有隙各立黨援以相圖謀術結公孫瓚而紹連劉表豪傑多附於紹術怒曰羣豎不吾從而從吾家奴乎|{
	據袁山松書紹司空逢之孽子出後伯父成故術云然}
又與公孫瓚書曰紹非袁氏子紹聞大怒術使孫堅擊劉表表遣其將黄祖逆戰於樊鄧之間|{
	鄧縣屬南陽郡樊城周仲山甫之邑在漢水北杜佑曰樊城今襄州安養縣劉昫曰鄧城縣漢之鄧縣古樊城也宋改安養縣天寶元年改為臨漢縣貞元二十一年移縣古鄧城乃改為鄧城縣}
堅擊破之遂圍襄陽表夜遣黄祖濳出發兵祖將兵欲還堅逆與戰祖敗走竄峴山中|{
	峴山去襄陽十里峴戶典翻}
堅乘勝夜追祖祖部曲兵從竹木間暗射堅殺之|{
	射而亦翻 考異曰范書初平三年春堅死吳志孫堅傳亦云初平三年英雄記曰初平四年正月七日死袁紀初平三年五月山陽公載記載策表曰臣年十七喪失所怙裴松之按策以建安五年卒時年二十六計堅之亡策應十八而此表云十七則為不符張璠漢紀及胡冲吳歷並以堅初平二年死此為是而本傳誤也今從之}
堅所舉孝亷長沙桓階詣表請堅喪表義而許之堅兄子賁率其士衆就袁術術復表賁為豫州刺史|{
	復扶又翻下同}
術由是不能勝表 初董卓入關留朱儁守雒陽而儁濳與山東諸將通謀懼為卓所襲出犇荆州卓以弘農楊懿為河南尹儁復引兵還雒擊懿走之儁以河南殘破無所資乃東屯中牟移書州郡請師討卓徐州刺史陶謙上儁行車騎將軍|{
	上時掌翻}
遣精兵三千助之餘州郡亦有所給謙丹陽人|{
	丹陽縣屬丹陽郡今潤州縣}
朝廷以黄巾寇亂徐州用謙為刺史謙至擊黄巾大破走之州境晏然 劉焉在益州陰圖異計沛人張魯自祖父陵以來世為五斗米道|{
	陵即今所謂天師者也後魏寇謙之祖其道}
客居于蜀魯母以鬼道常往來焉家焉乃以魯為督義司馬|{
	洪氏隸釋曰劉焉在蜀創置督義司馬助義褒義校尉劉表在荆州亦置綏民校尉漢衰諸侯擅命率意各置官屬}
以張脩為别部司馬合兵掩殺漢中太守蘇固斷絶斜谷閣|{
	斜谷在漢中西北今興元府西北入斜谷路至鳳州界百五十里有棧閣二千九百八十九間板閣二千八百九十二間郡國志曰褒城縣北有褒谷北口曰斜南口曰褒長四百七十里同為一谷兩山高峻中間谷道褒水所流曹操謂斜谷道為五百里石六者此也余據班志斜水出衙嶺山北至郿入渭褒水亦出衙嶺南至南鄭入沔則褒斜雖同為一谷而衙嶺乃其分水處也斷丁管翻斜音余奢翻谷音穀又音浴}
殺害漢使焉上書言米賊斷道不得復通|{
	斷丁管翻}
又託他事殺州豪彊王咸李權等十餘人以立威刑犍為太守任岐及校尉賈龍由此起兵攻焉焉擊殺岐龍焉意漸盛作乘輿車具千餘乘|{
	乘繩證翻}
劉表上焉有似子夏在西河疑聖人之論|{
	禮記檀弓曾子責子夏曰吾與子事夫子於洙泗之間退歸老於西河之上使西河之人疑汝於夫子而罪一也表蓋言焉在蜀僭擬使蜀人疑為天子也上時掌翻}
時焉子範為左中郎將誕為治書御史|{
	續漢志曰治書侍御史二人秩六百石掌選明法律者為之凡天下諸讞疑事掌以法律當其是非蔡質曰選御史高第補之胡廣曰宣帝幸宣室齋居而决事 令御史二人治書治書御史起此治直之翻}
璋為奉車都尉皆從帝在長安惟小子别部司馬瑁素隨焉帝使璋曉喻焉焉留璋不遣 公孫度威行海外中國人士避亂者多歸之北海管寧邴原王烈皆往依焉寧少時與華歆為友|{
	少詩照翻華戶化翻}
嘗與歆共鋤菜見地有金寧揮鋤不顧與瓦石無異歆捉而擲之人以是知其優劣邴原遠行遊學八九年而歸師友以原不飲酒會米肉送之原曰本能飲酒但以荒思廢業故斷之耳|{
	思相吏翻斷音短}
今當遠别可一飲燕於是共坐飲酒終日不醉寧原俱以操尚稱度虛館以候之|{
	操七到翻候者伺其至也}
寧既見度乃廬於山谷時避難者多居郡南|{
	難乃旦翻}
而寧獨居北示無還志後漸來從之旬月而成邑寧每見度語唯經典不及世事還山專講詩書習俎豆非學者無見也由是度安其賢民化其德邴原性剛直清議以格物|{
	格正也}
度以下心不安之寧謂原曰濳龍以不見成德|{
	乾初九濳龍勿用孔子曰君子以成德為行濳之為言也隱而未見行而未成是以君子弗用見賢遍翻}
言非其時皆招禍之道也密遣原逃歸度聞之亦不復追也|{
	復扶又翻}
王烈器業過人少時名聞在原寧之右|{
	少詩照翻聞文運翻名聲所至曰聞}
善於教誘鄉里有盜牛者主得之盜請罪曰刑戮是甘乞不使王彦方知也|{
	王烈字彦方}
烈聞而使人謝之遺布一端|{
	布帛六丈曰端一曰八丈曰端按古以二丈為端遺于季翻}
或問其故烈曰盜懼吾聞其過是有耻惡之心既知恥惡則善心將生故與布以勸為善也後有老父遺劍於路行道一人見而守之至暮老父還尋得劍怪之以事告烈烈使推求|{
	推尋也}
乃先盜牛者也諸有爭訟曲直往質之於烈|{
	質正也}
或至塗而反或望廬而還|{
	還旬緣翻}
皆相推以直|{
	推移也前書韓延壽傳以田相移即此義也}
不敢使烈聞之度欲以為長史烈辭之為商賈以自穢乃免|{
	賈音古}


三年春正月丁丑赦天下 董卓遣牛輔將兵屯陜輔分遣校尉北地李傕張掖郭汜武威張濟將步騎數萬擊破朱儁於中牟|{
	傕古岳翻汜音祀又孚梵翻}
因掠陳留潁川諸縣所過殺虜無遺初荀淑有孫曰彧少有才名|{
	少詩照翻}
何顒見而異之曰王佐才也及天下亂彧謂父老曰潁川四戰之地|{
	言其地平四面受敵}
宜亟避之鄉人多懷土不能去彧獨率宗族去依韓馥會袁紹已奪馥位待彧以上賓之禮彧度紹終不能定大業|{
	度徒洛翻}
聞曹操有雄畧乃去紹從操操與語大悦曰吾子房也|{
	比之張良}
以為奮武司馬|{
	操初起兵為奮武將軍故以彧為奮武司馬}
其鄉人留者多為傕汜等所殺 袁紹自出拒公孫瓚與瓚戰於界橋南二十里|{
	水經大河右瀆東北逕鉅鹿郡廣宗縣故城南又東北逕界城亭北又東北逕信都郡武彊縣故城東此蓋於河瀆上作橋註又云清河東北逕界城亭東水上有大梁謂之界城橋賢曰今貝州宗城縣側有古畧城此城近枯漳水界橋當在此水上杜佑曰界橋在貝州宗城縣東}
瓚兵三萬其鋒甚鋭紹令麴義領精兵八百先登彊弩千張夾承之瓚輕其兵少縱騎騰之義兵伏楯下不動未至十數步一時同發讙呼動地|{
	讙許元翻}
瓚軍大敗斬其所置冀州刺史嚴綱 |{
	考異曰九州春秋作劉綱今從范書魏志}
獲甲首千餘級追至界橋瓚歛兵還戰義復破之|{
	復扶又翻}
遂到瓚營抜其牙門|{
	賢曰眞人水鏡經曰凡軍始出必令完堅若有折將軍不利牙門旗竿軍之精也即周禮司職云軍旅會同置旌門是也}
餘衆皆走初兖州刺史劉岱與紹瓚連和紹令妻子居岱所瓚亦遣從事范方將騎助岱及瓚擊破紹軍語岱令遣紹妻子|{
	語牛倨翻}
别敕范方若岱不遣紹家將騎還吾定紹將加兵於岱岱與官屬議連日不决聞東郡程昱有智謀召而問之昱曰若棄紹近援而求瓚遠助此假人於越以救溺子之說也|{
	言勢不能相及也越人習水故以為能救溺溺奴歷翻}
夫公孫瓚非袁紹之敵也今雖壞紹軍|{
	壞音怪}
然終為紹所禽岱從之范方將其騎歸未至而瓚敗 曹操軍頓丘|{
	頓丘縣屬東郡師古曰以丘名縣也丘一成為頓丘謂一頓而成也或曰成重也一重之丘也}
于毒等攻東武陽操引兵西入山攻毒等本屯|{
	毒等時掠魏郡屯于西山}
諸將皆請救武陽操曰使賊聞我西而還武陽自解也不還我能敗其本屯虜不能拔武陽必矣|{
	敗補邁翻}
遂行毒聞之棄武陽還操遂擊眭固及匈奴於扶羅於内黄|{
	内黄縣屬魏郡陳留有外黄故加内眭息隨翻}
皆大破之 董卓以其弟旻為左將軍兄子璜為中軍校尉皆典兵事宗族内外並列朝廷卓侍妾懷抱中子皆封侯弄以金紫卓車服僭擬天子召呼三臺|{
	三臺尚書臺御史臺符節臺也晉書曰漢官尚書為中臺御史為憲臺謁者為外臺是謂三臺}
尚書以下皆自詣卓府啟事又築塢於郿|{
	英雄記曰郿去長安二百六十里漢書郿音媚地名}
高厚皆七丈積穀為三十年儲自云事成雄據天下不成守此足以畢老卓忍於誅殺諸將言語有蹉跌者|{
	蹉倉何翻跌徒結翻}
便戮於前人不聊生司徒王允與司隸校尉黄琬僕射士孫瑞尚書楊瓚密謀誅卓中郎將呂布便弓馬膂力過人|{
	膂脊骨也}
卓自以遇人無禮行止常以布自衛甚愛信之誓為父子然卓性剛嘗小失卓意卓拔手戟擲布|{
	手戟小戟便于擊刺者}
布拳捷|{
	勇力為拳迅疾為捷}
避之而改容顧謝卓意亦解布由是陰怨于卓卓又使布守中閣而私於傅婢益不自安王允素善待布布見允自陳卓幾見殺之狀|{
	幾居希翻}
允因以誅卓之謀告布使為内應布曰如父子何曰君自姓呂本非骨肉今憂死不暇何謂父子擲戟之時豈有父子情邪布遂許之夏四月丁巳帝有疾新愈大會未央殿卓朝服乘車而入|{
	魏祕書監秦静曰漢氏承秦改六冕之制朝服俱玄冠絳衣而已晉名曰五時朝服有四時朝服又有朝服}
陳兵夾道自營至宫左步右騎屯衛周帀|{
	帀作答翻}
令呂布等扞衛前後王允使士孫瑞自書詔以授布|{
	使尚書僕射自書詔者懼其泄也}
布令同郡騎都尉李肅 |{
	考異曰袁紀作李順今從范書魏志}
與勇士秦誼陳衛等十餘人偽著衛士服|{
	著陟畧翻}
守北掖門内以待卓卓入門肅以戟刺之|{
	刺七亦翻下同}
卓衷甲不入|{
	衷甲者被甲於内而加衣甲上}
傷臂墮車顧大呼曰|{
	呼火故翻}
呂布何在布曰有詔討賊臣卓大罵曰庸狗敢如是邪布應聲持矛刺卓趣兵斬之|{
	趣讀曰促}
主簿田儀及卓蒼頭前赴其尸布又殺之凡所殺三人布即出懷中詔版以令吏士曰詔討卓耳餘皆不問吏士皆正立不動大稱萬歲百姓歌舞於道長安中士女賣其珠玉衣裝市酒肉相慶者塡滿街肆弟旻璜等及宗族老弱在郿皆為其羣下斫射死|{
	射而亦翻}
暴卓尸於市|{
	暴薄木翻又薄報翻}
天時始熱卓素充肥脂流於地守尸吏為大炷|{
	炷燈也燼所著者}
置卓臍中然之光明逹曙如是積日諸袁門生聚董氏之尸焚灰揚之於路塢中有金二三萬斤銀八九萬斤錦綺奇玩積如丘山以王允録尚書事呂布為奮威將軍假節儀比三司|{
	奮武將軍始於漢元帝用任千秋為之沈約曰呂布為奮武將軍儀比三司猶儀同三司也}
封温侯|{
	温縣屬河内郡周大夫蘇忿生之邑}
共秉朝政|{
	朝直遥翻}
卓之死也左中郎將高陽侯蔡邕在王允坐|{
	高陽縣屬涿郡又陳留圉縣有高陽亭坐徂卧翻}
聞之驚歎允勃然叱之曰董卓國之大賊幾亡漢室|{
	幾居希翻}
君為王臣所宜同疾而懷其私遇反相傷痛豈不共為逆哉即收付廷尉邕謝曰身雖不忠古今大義耳所厭聞口所常玩豈當背國而嚮卓也|{
	背蒲妹翻}
願黥首刖足繼成漢史|{
	初邕徙朔方自徙中上書乞續漢書諸志蓋其所學所志者在此}
士大夫多矜救之不能得太尉馬日磾謂允曰伯喈曠世逸才|{
	蔡邕字伯喈}
多識漢事當續成後史為一代大典而所坐至微誅之無乃失人望乎允曰昔武帝不殺司馬遷使作謗書流於後世|{
	賢曰凡史官記事善惡必書謂遷所記但是漢家不善之事皆為謗也非獨指武帝之身即高祖善家令之言武帝筭緡榷酤之類是也班固集云史遷著書成一家之言至以身陷刑故微文譏刺貶損當世非義士也}
方今國祚中衰|{
	中竹仲翻}
戎馬在郊不可令佞臣執筆在幼主左右既無益聖德復使吾黨蒙其訕議|{
	復扶又翻}
日磾退而告人曰王公其無後乎善人國之紀也制作國之典也滅紀廢典其能久乎邕遂死獄中初黄門侍郎荀攸與尚書鄭泰侍中种輯等謀曰董卓驕忍無親雖資彊兵實一匹夫耳可直刺殺也|{
	刺七亦翻 考異曰魏志云攸與何顒伍瓊同謀按顒瓊死已久恐誤}
事垂就而覺收攸繫獄泰逃犇袁術攸言語飲食自若會卓死得免 青州黄巾寇兖州劉岱欲擊之濟北相鮑信諫曰今賊衆百萬百姓皆震恐士卒無鬭志不可敵也然賊軍無輜重唯以鈔畧為資|{
	重直用翻}
今不若畜士衆之力先為固守彼欲戰不得攻又不能其勢必離散然後選精鋭據要害擊之可破也岱不從遂與戰果為所殺曹操部將東郡陳宫謂操曰州今無主而王命斷絶宫請說州中綱紀|{
	綱紀即謂州别駕及治中諸從事也說輸芮翻下同}
明府尋往牧之|{
	牧之謂為州牧}
資之以收天下此霸王之業也宫因往說别駕治中曰今天下分裂而州無主曹東郡命世之才也若迎以牧州必寧生民鮑信等亦以為然乃與州吏萬濳等至東郡迎操領兖州刺史操遂進兵擊黄巾於壽張東不利賊衆精悍|{
	悍下罕翻又侯旰翻}
操兵寡弱操撫循激勵明設賞罰承間設奇|{
	間古莧翻}
晝夜會戰戰輒禽獲賊遂退走鮑信戰死操購求其喪不得乃刻木如信狀祭而哭焉詔以京兆金尚為兖州刺史將之部操逆擊之尚犇袁術|{
	為後建安二年尚不屈於術張本}
五月 |{
	考異曰范書丁酉大赦袁紀丁未大赦按是年正月丁丑大赦及李傕求赦王允曰一歲不再赦然則五月必無赦也}
以征西將軍皇甫嵩為車騎將軍 初呂布勸王允盡殺董卓部曲允曰此輩無罪不可布欲以卓財物班賜公卿將校允又不從允素以劍客遇布布負其功勞多自誇伐既失意望漸不相平允性剛稜疾惡|{
	賢曰稜威稜也音力登翻余謂稜方稜也剛稜猶言剛方}
初懼董卓故折節下之|{
	下遐稼翻}
卓既殱滅自謂無復患難|{
	殱息廉翻復扶又翻難乃旦翻}
頗自驕傲以是羣下不甚附之允始與士孫瑞議特下詔赦卓部曲既而疑曰部曲從其主耳今若名之惡逆而赦之恐適使深自疑非所以安之也乃止又議悉罷其軍或說允曰凉州人素憚袁氏而畏關東今若一旦解兵開關必人人自危可以皇甫義眞為將軍就領其衆因使留陜以安撫之允曰不然關東舉義兵者皆吾徒也今若距險屯陜雖安凉州而疑關東之心不可也|{
	陜失冉翻}
時百姓訛言當悉誅凉州人卓故將校遂轉相恐動皆擁兵自守|{
	將即亮翻校戶教翻}
更相謂曰蔡伯喈但以董公親厚尚從坐今既不赦我曹而欲使解兵今日解兵明日當復為魚肉矣|{
	更工衡翻復扶又翻}
呂布使李肅至陜以詔命誅牛輔輔等逆與肅戰肅敗走弘農布誅殺之輔恇怯失守|{
	恇去王翻}
會營中無故自驚輔欲走為左右所殺李傕等還|{
	傕等自陳留潁川還也}
輔已死傕等無所依遣使詣長安求赦王允曰一歲不可再赦不許傕等益懼不知所為欲各解散間行歸鄉里|{
	間古莧翻}
討虜校尉武威賈詡曰諸君若棄軍單行則一亭長能束君矣|{
	長知兩翻}
不如相率而西以攻長安為董公報仇|{
	為于偽翻}
事濟奉國家以正天下若其不合|{
	不合謂事不濟不與本計合也}
走未晚也傕等然之乃相與結盟率軍數千晨夜西行王允以胡文才楊整脩皆凉州大人|{
	賢曰大人謂大家豪右又曰大人長老之稱言尊事之也}
召使東解釋之不假借以温顔謂曰關東鼠子欲何為邪卿往呼之於是二人往實召兵而還傕隨道收兵比至長安|{
	比必寐翻及也}
已十餘萬與卓故部曲樊稠李蒙等合圍長安城城峻不可攻守之八日 |{
	考異曰魏志云十日今從范書}
呂布軍有叟兵内反|{
	賢曰叟兵即蜀兵也漢代謂蜀為叟}
六月戊午引傕衆入城放兵虜掠布與戰城中不勝將數百騎以卓頭繫馬鞍出走駐馬青瑣門外|{
	衛瓘曰青瑣戶邊青鏤也一曰天子門内有眉格再重裏青畫曰瑣}
招王允同去允曰若蒙社稷之靈上安國家吾之願也如其不獲則奉身以死之朝廷幼少|{
	朝廷謂天子也}
恃我而已臨難苟免吾不忍也|{
	難乃旦翻}
努力謝關東諸公勤以國家為念太常种拂曰為國大臣不能禁暴禦侮使白刃向宫去將安之遂戰而死傕汜屯南宫掖門殺太僕魯馗|{
	馗音逵}
大鴻臚周奐|{
	臚陵如翻}
城門校尉崔烈越騎校尉王頎|{
	頎音祈}
吏民死者萬餘人狼籍滿道王允扶帝上宣平門避兵|{
	三輔黄圖曰長安城東向北頭門號宣平門上時掌翻}
傕等於城門下伏地叩頭帝謂傕等曰卿等放兵縱横欲何為乎|{
	横戶孟翻}
傕等曰董卓忠於陛下而無故為呂布所殺臣等為卓報讎|{
	為于偽翻}
非敢為逆也請事畢詣廷尉受罪傕等圍門樓共表請司徒王允出問太師何罪允窮蹙乃下見之己未赦天下以李傕為揚武將軍郭汜為揚烈將軍|{
	揚武將軍始於建武之初馬成為之揚烈將軍蓋始於是時}
樊稠等皆為中郎將傕等收司隸校尉黄琬殺之初王允以同郡宋翼為左馮翊王宏為右扶風|{
	允太原人}
傕等欲殺允恐二郡為患乃先徵翼宏宏遣使謂翼曰郭汜李傕以我二人在外故未危王公|{
	危謂殺也}
今日就徵明日俱族計將安出翼曰雖禍福難量|{
	量音良}
然王命所不得避也宏曰關東義兵鼎沸欲誅董卓今卓已死其黨與易制耳|{
	易以豉翻}
若舉兵共討傕等與山東相應此轉禍為福之計也翼不從宏不能獨立遂俱就徵甲子傕收允及翼宏并殺之允妻子皆死宏臨命詬曰|{
	詬許候翻又古候翻怒罵也}
宋翼豎儒不足議大計|{
	賢曰豎者言賤劣如僮豎}
傕尸王允於市莫敢收者故吏平陵令京兆趙戩棄官收而葬之|{
	戩子踐翻}
始允自專討卓之勞士孫瑞歸功不侯故得免於難臣光曰易稱勞謙君子有終吉|{
	易繫辭曰勞而不伐冇功而不德厚之至也語以其功下人者也德言盛禮言恭謙也者致恭以存其位者也程頤註曰有勞而能謙又須君子行之有終則吉夫樂高喜勝人之常情平時能謙固已鮮矣况有功勞可尊乎雖使知謙之善勉而為之若矜負之心不忘則不能常久欲其有終不可得也惟君 子安履謙順故久而不變乃所謂有終則吉也}
士孫瑞有功不伐以保其身可不謂之智乎

傕等以賈詡為左馮翊欲侯之詡曰此救命之計何功之有固辭不受又以為尚書僕射詡曰尚書僕射官之師長|{
	長知兩翻}
天下所望詡名不素重非所以服人也乃以為尚書 呂布自武關犇南陽袁術待之甚厚布自恃有功於袁氏|{
	謂殺董卓為袁氏報仇也}
恣兵鈔掠|{
	鈔楚交翻}
術患之布不自安去從張楊於河内李傕等購求布急布又逃歸袁紹 丙子以前將軍趙謙為司徒 秋七月庚子以太尉馬日磾為太傳録尚書事|{
	磾丁奚翻}
八月以車騎將軍皇甫嵩為太尉 詔太傅馬日磾太僕趙岐杖節鎮撫關東 九月以李傕為車騎將軍領司隸校尉假節郭汜為後將軍樊稠為右將軍張濟為驃騎將軍皆封侯|{
	驃匹玅翻}
傕汜稠筦朝政|{
	筦與管同}
濟出屯弘農 司徒趙謙罷甲申以司空淳于嘉為司徒光禄大夫楊彪為司空録

尚書事 初董卓入關說韓遂馬騰與共圖山東|{
	說輸芮翻}
遂騰率衆詣長安會卓死李傕等以遂為鎮西將軍遣還金城騰為征西將軍遣屯郿|{
	晉書職官志曰四征起於漢代四鎮通於柔遠}
冬十月荆州刺史劉表遣使貢獻以表為鎮南將軍荆州牧封成武侯|{
	成武縣前漢屬山陽郡後漢屬濟陰郡}
十二月太尉皇甫嵩免以光禄大夫周忠為太尉參録尚書事曹操追黄巾至濟北悉降之|{
	濟子禮翻降戶江翻}
得戎卒三十餘萬男女百餘萬口收其精鋭者號青州兵|{
	所降者青州黄巾也故號青州兵}
操辟陳留毛玠為治中從事玠言於操曰今天下分崩乘輿播蕩生民廢業饑饉流亡公家無經歲之儲百姓無安固之志難以持久夫兵義者勝|{
	魏相嘗引是言}
守值以財|{
	易大傳曰何以聚人曰財何以守位曰仁}
宜奉天子以令不臣脩耕植以畜軍資|{
	操之所以芟羣雄者在迎天子都許屯田積穀而已二事乃玠發其謀也}
如此則覇王之業可成也操納其言遣使詣河内太守張楊欲假塗西至長安楊不聽定陶董昭說楊曰|{
	說輪内翻}
袁曹雖為一家勢不久羣曹今雖弱然實天下之英雄也當故結之|{
	故者結交之因也謂因事而結之}
况今有緣宜通其上事|{
	上時掌翻下同}
并表薦之若事冇成永為深分|{
	分扶問翻契分也}
楊於是通操上事仍表薦操昭為操作書與李傕郭汜等|{
	為于偽翻}
各隨輕重致殷勤傕汜見操使以為關東欲自立天子今曹操雖有使命非其誠實|{
	使疏吏翻}
議留操使黄門侍郎鍾繇說傕汜曰方今英雄並起各矯命專制唯曹兖州乃心王室而逆其忠欵非所以副將來之望也傕汜乃厚加報答|{
	當是時董昭在河内鍾繇在長安操不能使也而各為操道地蓋聞其雄畧先為效用以自結也}
繇皓之曾孫也|{
	鍾皓見五十三卷桓帝建和三年}
徐州刺史陶謙與諸守相共奏記推朱儁為太師|{
	守式又翻相悉亮翻}
因移檄牧伯欲以同討李傕等奉迎天子會李傕用太尉周忠尚書賈詡策徵儁入朝儁乃辭謙議而就徵復為太僕 公孫瓚復遣兵擊袁紹至龍湊|{
	龍湊地名在平原界漢晉春秋載紹與瓚書曰龍河之師羸兵前誘大兵未濟而足下膽破衆散不鼓而敗則龍湊蓋河津也詳味紹書龍湊宜在勃海界又袁譚軍龍湊曹操攻之拔平原走保南皮蓋在平原界也復扶又翻下同}
紹擊破之瓚遂還幽州不敢復出 揚州刺史汝南陳温卒袁紹使袁遺領揚州袁術擊破之遺走至沛為兵所殺術以下邳陳瑀為揚州刺史 |{
	考異曰獻帝紀四年三月袁術殺陳温據淮南魏志術傳云術殺温領其州裴松之按英雄記温自病死不為術所殺九州春秋曰初平三年揚州刺史陳禕死術以瑀領揚州蓋陳禕當為陳温實以三年卒今從之}
四年春正月甲寅朔日有食之 丁卯赦天下 |{
	考異曰袁紀五月丁卯赦今從范書}
曹操軍甄城|{
	甄城縣屬濟陰郡賢曰今濮州縣也甄音絹蜀本作鄄為是}
袁術為劉表所逼引兵屯封丘黑山别部及匈奴於扶羅皆附之曹操擊破術軍遂圍封丘術走襄邑又走寧陵|{
	封丘襄邑二縣屬陳留郡寧陵縣屬梁國宋白曰封丘古封國之地左傳所謂封父之䌓弱是也漢爲封丘縣寧陵縣古甯城漢高祖改為寧陵縣}
操追擊連破之術走九江揚州刺史陳瑀拒術不納術退保陰陵集兵於淮北復進向壽春|{
	陰陵壽春二縣皆屬九江郡壽春揚州刺史治所復扶又翻}
瑀懼走歸下邳術遂領其州兼稱徐州伯李傕欲結術為援以術為左將軍封陽翟侯|{
	陽翟縣屬潁川郡}
假節 袁紹與公孫瓚所置青州刺史田楷連戰二年士卒疲困糧食並盡互掠百姓野無青草紹以其子譚為青州刺史楷與戰不勝會趙岐來和解關東瓚乃與紹和親各引兵去 三月袁紹在薄落津|{
	續漢志安平國經縣西有漳水津名薄落津鉅鹿郡癭陶縣有薄落亭水經註漳水逕鉅鹿縣故城西水冇故津謂之薄落津}
魏郡兵反與黑山賊于毒等數萬人共覆鄴城殺其太守紹還屯斥丘|{
	斥丘縣屬鉅鹿郡賢曰故城在今相州成安縣東南十三州志云土地斥鹵故云斥丘}
夏曹操還軍定陶 徐州治中東海王朗及别駕琅邪趙昱說刺史陶謙曰求諸侯莫如勤王|{
	左傳晉大夫狐偃之言說輸芮翻}
今天子越在西京宜遣使奉貢謙乃遣昱奉章至長安詔拜謙徐州牧加安東將軍封溧陽侯|{
	溧陽縣屬丹陽郡}
以昱為廣陵太守朗為會稽太守是時徐方百姓殷盛|{
	古語多謂州為方故八州八伯謂之方伯書曰惟此陶唐有此冀方詩曰徐方不庭是也}
穀實差豐流民多歸之而謙信用讒邪疎遠忠直|{
	遠于願翻}
刑政不治由是徐州漸亂許劭避地廣陵謙禮之甚厚劭告其徒曰陶恭祖外慕聲名内非真正|{
	陶謙字恭祖}
待吾雖厚其勢必薄遂去之後謙果捕諸寓士人乃服其先識 六月扶風大雨雹|{
	雨于具翻}
華山崩裂|{
	華戶化翻}
太尉周忠免以太僕朱儁為太尉錄尚書事下邳闕宣聚衆數千人|{
	賢曰風俗通曰闕姓也承闕黨童子之後縱横家有闕子}


|{
	著書}
自稱天子陶謙擊殺之 |{
	考異曰范書謙傳作闕宣今從魏志武紀及謙傳魏武紀又曰謙與宣共舉兵取泰山華費掠任城謙傳亦云謙始與合從後遂殺之并其衆按謙據有徐州託義勤王何藉宣數千之衆而與之合從盖謙别將與宣共襲曹嵩故曹操以此為謙罪而伐之耳}
大雨晝夜二十餘日漂沒民居 袁紹出軍入朝哥鹿膓山|{
	朝哥縣屬河内郡賢曰朝哥故城在今衛縣西續漢志曰朝哥有鹿膓山}
討于毒圍攻五日破之斬毒及其衆萬餘級紹遂尋山北行進擊諸賊左髭丈八等皆斬之又擊劉石青牛角黄龍左校郭大賢李大目于氐根等復斬數萬級皆屠其屯壁遂與黑山賊張燕及四營屠各鴈門烏桓戰於常山|{
	復扶又翻屠直於翻屠各匈奴種}
燕精兵數萬騎數千匹紹與呂布共擊燕連戰十餘日燕兵死傷雖多紹軍亦疲遂俱退呂布將士多暴横|{
	横戶孟翻}
紹患之布因求還雒陽紹承制以布領司隸校尉遣壯士送布而陰圖之布使人鼓筝於帳中|{
	說文筝樂也鼔弦竹身十三弦蒙恬所造一說秦人薄義父子爭瑟而分之因以為名案筝制與瑟同瑟二十五弦而筝十三弦故云然風俗通筝秦聲五弦筑身筝者上圓象天下平象地中空凖六合弦柱十二擬十二月乃仁智之器也今并凉二州筝形如瑟不知誰改也釋名筝施絃高筝筝然音争}
密亡去送者夜起斫帳被皆壞明旦紹聞布尚在懼閉城自守布引軍復歸張楊 前太尉曹嵩避難在琅邪|{
	難乃旦翻}
其子操令泰山太守應劭迎之嵩輜重百餘兩|{
	重直用翻兩音亮}
陶謙别將守陰平|{
	陰平縣屬東海郡賢曰故城在今沂州承縣西南}
士卒利嵩財寶掩襲嵩於華費間殺之|{
	前漢志華費二縣皆屬泰山郡續漢志泰山有費縣無華縣蓋并省也水經時水南過華縣東又南過費縣東入沂賢曰費縣故城在費縣東北費音秘}
并少子德秋操引兵擊謙攻拔十餘城至彭城大戰謙兵敗走保郯|{
	郯縣屬東海郡徐州刺史治所}
初京雒遭董卓之亂民流移東出多依徐土遇操至坑殺男女數十萬口於泗水水為不流|{
	為于偽翻}
操攻郯不能克乃去攻取慮睢陵夏丘|{
	三縣皆屬下邳國杜佑曰泗川下邳縣北冇漢武原故城又北有郯縣故城睢陵故城在下邳東南夏丘堯封禹為夏伯邑於此漢為夏丘縣師古曰取音趨又音秋慮音廬睢音雖夏戶雅翻}
皆屠之雞犬亦盡墟邑無復行人冬十月辛丑京師地震 有星孛于天市|{
	孛蒲内翻}
司

空楊彪免丙午以太常趙温為司空録尚書事 劉虞與公孫瓚積不相能瓚數與袁紹相攻|{
	數所角翻下同}
虞禁之不可而稍節其禀假瓚怒屢違節度又復侵犯百姓虞不能制乃遣驛使奉章陳其暴掠之罪瓚亦上虞稟糧不周|{
	上時掌翻}
二奏交馳互相非毁朝廷依違而已|{
	依違言甲奏上則依甲而違乙乙奏上則依乙而違甲無决然之是非也}
瓚乃築小城於薊城東南以居之|{
	薊縣屬廣陽國幽州牧所治也薊音計}
虞數請會瓚輒稱病不應虞恐其終為亂乃率所部兵合十萬人以討之時瓚部曲放散在外倉卒掘東城欲走|{
	卒與猝同}
虞兵無部伍不習戰又愛民廬舍敕不聽焚燒戒軍士曰無傷餘人殺一伯珪而已|{
	公孫瓚字伯珪}
攻圍不下瓚乃簡募鋭士數百人因風縱火直衝突之虞衆大潰虞與官屬北犇居庸|{
	居庸縣屬上谷郡胡嶠曰自幽州西北入居庸關宋祁曰唐媯州懷戎縣東南五十里有居庸塞東連盧龍碣石西屬太行常山實天下之險}
瓚追攻之三日城陷執虞并妻子還薊猶使領州文書會詔遣使者段訓增虞封邑督六州事拜瓚前將軍封易侯|{
	易縣前漢屬涿郡後漢省}
瓚乃誣虞前與袁紹等謀稱尊號脅訓斬虞及妻子於薊市故常山相孫瑾掾張逸張瓚等相與就虞罵瓚極口然後同死|{
	掾俞絹翻}
瓚傳虞首於京師故吏尾敦於路劫虞首歸葬之|{
	賢曰尾姓敦名余按古有尾生}
虞以恩厚得衆心北州百姓流舊莫不痛惜|{
	流者他州人流入幽州者也舊者舊著籍幽州者也}
初虞欲遣使奉章詣長安而難其人衆咸曰右北平田疇年二十二年雖少|{
	少詩照翻}
然有奇材虞乃備禮請以為掾具車騎將行疇曰今道路阻絶寇虜縱横稱官奉使為衆所指願以私行期於得逹而已虞從之疇乃自選家客二十騎俱上西關出塞傍北山|{
	傍步浪翻西關即居庸關北山即陰山}
直趣朔方循間道至長安致命|{
	趣七喻翻間古莧翻}
詔拜疇為騎都尉疇以天子方蒙塵未安不可以荷佩榮寵|{
	荷下可翻}
固辭不受得報馳還比至虞已死|{
	比必寐翻}
疇謁祭虞墓陳發章表|{
	章表當依下文作章報}
哭泣而去公孫瓚怒購求獲疇謂曰汝不送章報我何也疇曰漢室衰穨人懷異心唯劉公不失忠節章報所言於將軍未美恐非所樂聞|{
	樂音洛}
故不進也且將軍既滅無罪之君又讎守義之臣疇恐燕趙之士皆將蹈東海而死莫有從將軍者也瓚乃釋之疇北歸無終|{
	無終縣屬右北平郡春秋無終子之國疇盖其縣人宋白曰無終唐為薊州玉田縣}
率宗族及他附從者數百人埽地而盟曰君仇不報吾不可以立於世遂入徐無山中|{
	徐無縣屬右北平郡冇徐無山}
營深險平敞地而居躬耕以養父母|{
	養羊亮翻}
百姓歸之數年間至五千餘家疇謂其父老曰今衆成都邑而莫相統一又無法制以治之|{
	治直之翻}
恐非久安之道疇有愚計願與諸君共施之可乎皆曰可疇乃為約束相殺傷犯盗諍訟者|{
	諍讀曰争晉王沈釋時論闟茸勇敢於饕諍叶韻平聲}
隨輕重抵罪重者至死凡一十餘條又制為婚姻嫁娶之禮與學校講授之業|{
	校戶教翻}
班行於衆衆皆便之至道不拾遺北邊翕然服其威信烏桓鮮卑各遣使致饋疇悉撫納令不為寇 十二月辛丑地震 司空趙温免乙巳以衛尉張喜為司空

資治通鑑卷六十
