<!DOCTYPE html PUBLIC "-//W3C//DTD XHTML 1.0 Transitional//EN" "http://www.w3.org/TR/xhtml1/DTD/xhtml1-transitional.dtd">
<html xmlns="http://www.w3.org/1999/xhtml">
<head>
<meta http-equiv="Content-Type" content="text/html; charset=utf-8" />
<meta http-equiv="X-UA-Compatible" content="IE=Edge,chrome=1">
<title>資治通鑒_166-資治通鑑卷一百六十五_166-資治通鑑卷一百六十五</title>
<meta name="Keywords" content="資治通鑒_166-資治通鑑卷一百六十五_166-資治通鑑卷一百六十五">
<meta name="Description" content="資治通鑒_166-資治通鑑卷一百六十五_166-資治通鑑卷一百六十五">
<meta http-equiv="Cache-Control" content="no-transform" />
<meta http-equiv="Cache-Control" content="no-siteapp" />
<link href="/img/style.css" rel="stylesheet" type="text/css" />
<script src="/img/m.js?2020"></script> 
</head>
<body>
 <div class="ClassNavi">
<a  href="/24shi/">二十四史</a> | <a href="/SiKuQuanShu/">四库全书</a> | <a href="http://www.guoxuedashi.com/gjtsjc/"><font  color="#FF0000">古今图书集成</font></a> | <a href="/renwu/">历史人物</a> | <a href="/ShuoWenJieZi/"><font  color="#FF0000">说文解字</a></font> | <a href="/chengyu/">成语词典</a> | <a  target="_blank"  href="http://www.guoxuedashi.com/jgwhj/"><font  color="#FF0000">甲骨文合集</font></a> | <a href="/yzjwjc/"><font  color="#FF0000">殷周金文集成</font></a> | <a href="/xiangxingzi/"><font color="#0000FF">象形字典</font></a> | <a href="/13jing/"><font  color="#FF0000">十三经索引</font></a> | <a href="/zixing/"><font  color="#FF0000">字体转换器</font></a> | <a href="/zidian/xz/"><font color="#0000FF">篆书识别</font></a> | <a href="/jinfanyi/">近义反义词</a> | <a href="/duilian/">对联大全</a> | <a href="/jiapu/"><font  color="#0000FF">家谱族谱查询</font></a> | <a href="http://www.guoxuemi.com/hafo/" target="_blank" ><font color="#FF0000">哈佛古籍</font></a> 
</div>

 <!-- 头部导航开始 -->
<div class="w1180 head clearfix">
  <div class="head_logo l"><a title="国学大师官网" href="http://www.guoxuedashi.com" target="_blank"></a></div>
  <div class="head_sr l">
  <div id="head1">
  
  <a href="http://www.guoxuedashi.com/zidian/bujian/" target="_blank" ><img src="http://www.guoxuedashi.com/img/top1.gif" width="88" height="60" border="0" title="部件查字,支持20万汉字"></a>


<a href="http://www.guoxuedashi.com/help/yingpan.php" target="_blank"><img src="http://www.guoxuedashi.com/img/top230.gif" width="600" height="62" border="0" ></a>


  </div>
  <div id="head3"><a href="javascript:" onClick="javascript:window.external.AddFavorite(window.location.href,document.title);">添加收藏</a>
  <br><a href="/help/setie.php">搜索引擎</a>
  <br><a href="/help/zanzhu.php">赞助本站</a></div>
  <div id="head2">
 <a href="http://www.guoxuemi.com/" target="_blank"><img src="http://www.guoxuedashi.com/img/guoxuemi.gif" width="95" height="62" border="0" style="margin-left:2px;" title="国学迷"></a>
  

  </div>
</div>
  <div class="clear"></div>
  <div class="head_nav">
  <p><a href="/">首页</a> | <a href="/ShuKu/">国学书库</a> | <a href="/guji/">影印古籍</a> | <a href="/shici/">诗词宝典</a> | <a   href="/SiKuQuanShu/gxjx.php">精选</a> <b>|</b> <a href="/zidian/">汉语字典</a> | <a href="/hydcd/">汉语词典</a> | <a href="http://www.guoxuedashi.com/zidian/bujian/"><font  color="#CC0066">部件查字</font></a> | <a href="http://www.sfds.cn/"><font  color="#CC0066">书法大师</font></a> | <a href="/jgwhj/">甲骨文</a> <b>|</b> <a href="/b/4/"><font  color="#CC0066">解密</font></a> | <a href="/renwu/">历史人物</a> | <a href="/diangu/">历史典故</a> | <a href="/xingshi/">姓氏</a> | <a href="/minzu/">民族</a> <b>|</b> <a href="/mz/"><font  color="#CC0066">世界名著</font></a> | <a href="/download/">软件下载</a>
</p>
<p><a href="/b/"><font  color="#CC0066">历史</font></a> | <a href="http://skqs.guoxuedashi.com/" target="_blank">四库全书</a> |  <a href="http://www.guoxuedashi.com/search/" target="_blank"><font  color="#CC0066">全文检索</font></a> | <a href="http://www.guoxuedashi.com/shumu/">古籍书目</a> | <a   href="/24shi/">正史</a> <b>|</b> <a href="/chengyu/">成语词典</a> | <a href="/kangxi/" title="康熙字典">康熙字典</a> | <a href="/ShuoWenJieZi/">说文解字</a> | <a href="/zixing/yanbian/">字形演变</a> | <a href="/yzjwjc/">金 文</a> <b>|</b>  <a href="/shijian/nian-hao/">年号</a> | <a href="/diming/">历史地名</a> | <a href="/shijian/">历史事件</a> | <a href="/guanzhi/">官职</a> | <a href="/lishi/">知识</a> <b>|</b> <a href="/zhongyi/">中医中药</a> | <a href="http://www.guoxuedashi.com/forum/">留言反馈</a>
</p>
  </div>
</div>
<!-- 头部导航END --> 
<!-- 内容区开始 --> 
<div class="w1180 clearfix">
  <div class="info l">
   
<div class="clearfix" style="background:#f5faff;">
<script src='http://www.guoxuedashi.com/img/headersou.js'></script>

</div>
  <div class="info_tree"><a href="http://www.guoxuedashi.com">首页</a> > <a href="/SiKuQuanShu/fanti/">四库全书</a>
 > <h1>资治通鉴</h1> <!--         下载:【右键另存为】即可 --></div>
  <div class="info_content zj clearfix">
  
<div class="info_txt clearfix" id="show">
<center style="font-size:24px;">166-資治通鑑卷一百六十五</center>
    資治通鑑卷一百六十五 宋 司馬光 撰<br />
<br />
  胡三省 音註<br />
<br />
  梁紀二十一【起昭陽作噩盡閼逢閹茂凡二年】<br />
<br />
  世祖孝元皇帝下<br />
<br />
  承聖二年春正月王僧辯發建康承制使陳霸先代鎮揚州【使陳霸先自京口代鎮揚州】 丙子山胡圍齊離石戊寅齊主討之未至胡已走因廵三堆【魏收地形志永安郡平寇縣魏真君七年併三堆屬焉隋鴈門郡崞縣有平寇縣】大獵而歸 以吏部尚書王褒為左僕射己丑齊改鑄錢文曰常平五銖【五代志齊文宣除魏永安五銖改鑄常平】<br />
<br />
  【五銖重如其文其錢甚貴且制造甚精】 二月庚子李洪雅力屈以空雲城降陸納【去年吳藏攻李洪雅降戶江翻下同】納囚洪雅殺丁道貴納以沙門寶誌詩䜟有十八子以為李氏當王【天監中寶誌為䜟云太歲龍將無理蕭經霜草應死餘人散十八子時言蕭氏當滅李氏代興䜟楚譛翻】甲辰推洪雅為主號大將軍使乘平肩輿列鼓吹納帥衆數千左右翼從【吹尺瑞翻帥讀曰率下同從才用翻】 魏太師泰去丞相大行臺【去羌呂翻】為都督中外諸軍事 王雄至東梁州黄衆寶帥衆降【黄衆寶反見上卷上年】太師泰赦之遷其豪帥於雍州【帥所類翻雍於用翻】 齊主送柔然可汗鐵伐之父登注及兄庫提還其國【登注等奔齊見上卷上年可從刋入聲汗音寒】鐵伐尋為契丹所殺【契欺訖翻又音喫】國人立登注為可汗登注復為其大人阿富提所殺【復扶又翻】國人立庫提 突厥伊利可汗卒子科羅立號乙息記可汗【厥君勿翻 考異曰顔師古隋書突厥傳云弟逸可汗立今從周書及北史】三月遣使獻馬五萬于魏【使疏吏翻】柔然别部又立阿那瓌叔父鄧叔子為可汗 【考異曰魏書北史蠕蠕傳皆云立鐵伐為可汗突厥傳皆云立鄧叔子為可汗蓋諸部分散各有所立也】乙息記撃破鄧叔子於沃野北木賴山乙息記卒捨其子攝圖而立其弟俟斤號木杆可汗【俟渠之翻杆公旦翻為後佗鉢卒攝圖爭國張本 考異曰周書作术汗隋書作俟斗木杆今從北史】木杆狀貌奇異性剛勇多智畧善用兵鄰國畏之上聞武陵王紀東下使方士畫版為紀像【畫與畫同】親釘<br />
<br />
  支體以厭之【釘丁定翻厭於叶翻】又執侯景之俘以報紀初紀之舉兵皆太子圓照之謀也圓照時鎮巴東【巴東信州】執留使者啓紀云侯景未平宜急進討已聞荆鎮為景所破紀從之趣兵東下【使疏吏翻趣讀曰促】上甚懼與魏書曰子糾親也請君討之【左傳齊無知弑其君雍廩殺無知公子小白自莒入於齊魯莊公伐齊納子糾魯師敗績齊鮑叔帥師來言曰子糾親也請君討之乃殺子糾】太師泰曰取蜀制梁在兹一舉諸將咸難之【咸以為難也】大將軍代人尉遲迥泰之甥也【迥傳其先魏之别種號尉遟部因氏焉尉紆勿翻】獨以為可克泰問以方略迥曰蜀與中國隔絶百有餘年恃其險不虞我至若以鐵騎兼行襲之無不克矣【騎奇寄翻下同】泰乃遣迥督開府儀同三司原珍等六軍甲士萬二千騎萬匹自散關伐蜀【考異曰典略在正月戊辰今從周紀】 陸納遣其將吳藏潘烏黑李賢明等下據車輪【將即亮翻下同按下文陸納夾岸為城甲子王僧辯攻拔之乙丑進圍長沙則車輪之地蓋據湘江之要去長沙不遠也 考異曰梁紀云二月丙子按長歷二月無丙子梁紀誤】王僧辯至巴陵 【考異曰典略云三月辛酉按長歷是月癸亥朔無辛酉典略誤】宜豐侯循讓都督於僧辯 【考異曰僧辯傳云與陳霸先讓都督今從典略】僧辯弗受上乃以僧辯循為東【西】都督夏四月丙申僧辯軍于車輪 【考異曰典略作甲子非也今從梁紀】 吐谷渾可汗夸呂雖通使于魏而寇抄不息【吐從暾入聲谷音浴使疏吏翻下同抄楚交翻】宇文泰將騎三萬踰隴至姑臧討之夸呂懼請服既而復通使於齊凉州刺史史寧覘知其還襲之於赤泉【唐志凉州姑臧縣有赤水軍本赤烏鎮有赤烏泉因名幅員五千一百八十里軍之最大者也復扶又翻覘丑鹽翻又丑艶翻】獲其僕射乞伏觸狀 陸納夾岸為城以拒王僧辯納士卒皆百戰之餘僧辯憚之不敢輕進稍作連城以逼之納以僧辯為怯不設備五月甲子僧辯命諸軍水陸齊進急攻之僧辯親執旗鼓宜豐侯循親受矢石拔其二城納衆大敗步走保長沙乙丑僧辯進圍之僧辯坐壟上視築圍壘吳藏李賢明帥銳卒千人開門突出蒙楯直進趨僧辯【帥讀曰率楯食尹翻趨七喻翻】時杜崱杜龕並侍左右甲士衛者止百餘人力戰拒之【崱士力翻龕苦含翻】僧辯據胡牀不動裴之横從旁擊藏等藏等敗退賢明死【李賢明本侯景將也景敗歸王琳】藏脫走入城 武陵王紀至巴郡聞有魏兵遣前梁州刺史巴西譙淹還軍救蜀初楊乾運求為梁州刺史紀以為潼州刺史【五代志金山郡西魏置潼州蓋梁已置此州也治涪城】楊灋琛求為黎州刺史以為沙州【蓋即以平興為沙州也潼音同琛丑林翻】二人皆不悅乾運兄子畧說乾運曰今侯景初平宜同心戮力【說式芮翻戮音留又音六并力也】保國寧民而兄弟尋戈【左傳子產曰昔高辛氏有二子伯曰閼伯季曰實沈不相能也日尋干戈以相征討】此自亡之道也夫木朽不雕【論語孔子曰朽木不可雕也】世衰難佐不如送欵關中可以功名兩全乾運然之令略將二千人鎮劍閣又遣其壻樂廣鎮安州【五代志普安郡梁置南安州後改為安州普安舊曰南安西魏改普安將即亮翻】與灋琛皆濳通於魏魏太師泰密賜乾運鐵劵授驃騎大將軍開府儀同三司梁州刺史【西魏之官驃騎大將軍開府儀同三司位次柱國大將軍驃匹妙翻】尉遟迥以開府儀同三司侯呂陵始為前軍【侯呂陵虜三字姓】至劔閣畧退就樂廣翻城應始始入據安州甲戌迥至涪水【涪水自龍州入潼州界潼州治涪其城西臨涪水涪音浮】乾運以州降【降戶江翻】迥分軍守之進襲成都時成都見兵不滿萬人【見賢遍翻】倉庫空竭永豐侯撝嬰城自守迥圍之譙淹遣江州刺史景欣幽州刺史趙抜扈援成都【五代志隆山郡隆山縣舊曰犍為縣置江州】迥使原珍等擊走之武陵王紀至巴東聞侯景已平乃自悔召太子圓照責之對曰侯景雖平江陵未服紀亦以既稱尊號【紀稱尊號見上卷上年】不可復為人下【復扶又翻下上復復送乃復無復同】欲遂東進將卒日夜思歸其江州刺史王開業以為宜還救根本更思後圖諸將皆以為然【將即亮翻】圓照及劉孝勝固言不可紀從之宣言於衆曰敢諫者死己丑紀至西陵軍勢甚盛舳艫翳川【舳音逐艫音盧翳蔽也】護軍陸灋和築二城於峽口兩岸運石塡江鐵鎻斷之【斷音短】帝抜任約於獄以為晉安王司馬【帝封子方智為晉安王任音壬】使助灋和拒紀【赤亭之戰法和活約是有舊恩故使助之】謂之曰汝罪不容誅我不殺本為今日因撤禁兵以配之仍許妻以廬陵王續之女使宣猛將軍劉棻與之俱【為于偽翻妻七細翻棻符分翻】 庚辰巴州刺史余孝頃將兵萬人會王僧辯於長沙【將即亮翻下同】 豫章太守觀寧侯永昏而少斷【少詩沼翻斷丁亂翻】左右武蠻奴用事軍主文重疾之永將兵討陸納至宫亭湖重殺蠻奴永軍潰奔江陵重將其衆奔開建侯蕃蕃殺之而有其衆【開建侯蕃時鎮鄱陽沈約曰宋文帝分臨賀郡之封陽縣立開建縣】 六月壬辰武陵王紀築連城攻絶鐵鎻陸法和告急相繼上復抜謝答仁於獄【去年侯景敗得謝答仁不殺而囚之】以為步兵校尉【梁東宫有步兵等三校尉校戶敎翻】配兵使助法和又遣使送王琳令說輸陸納【使疏吏翻說式芮翻】乙未琳至長沙僧辯使送示之納衆悉拜且泣使謂僧辯曰朝廷若赦王郎乞聽入城僧辯不許復送江陵陸法和求救不已上欲召長沙兵恐失陸納乃復遣琳許其入城琳既入納遂降湘州平【降戶江翻 考異曰梁紀乙酉湘州平按長歷是月無乙酉梁紀誤】上復琳官爵使將兵西援峽口【將即亮翻】 甲辰齊章武景王庫狄干卒 武陵王紀遣將軍侯叡將衆七千築壘與陸法和相拒上遣使與紀書許其還蜀專制一方紀不從報書如家人禮【不肯定君臣之分而用兄弟之禮使疏吏翻下同】陸納既平湘州諸軍相繼西上【上時掌翻】上復與紀書曰吾年為一日之長屬有平亂之功膺此樂推事歸當璧【長陟丈翻屬之欲翻樂音洛左傳楚共王無冢適有寵子五人無適立焉乃大有事於羣望而祈曰請神擇於五人者使主社稷乃徧以璧見於羣望曰當璧而拜者神所立也既乃密埋璧於太室之庭使五人者齊而長入拜康王跨之靈王肘加焉子干子晳皆遠之平王弱抱以入再拜皆壓紐後平王卒有楚國】儻遣使乎良所遟也【遟直利翻待也】如曰不然於此投筆友于兄弟分形共氣兄肥弟瘦無復相見之期讓棗推梨永罷懽愉之日【漢孔融兄弟七人融第六四歲時與諸兄共食梨棗輒引小者人問其故答曰我小兒法當取小者人皆異之推吐雷翻】心乎愛矣書不盡言【言兄弟之愛存之於心非書翰之間所能盡言也】紀頓兵日久頻戰不利又聞魏寇深入成都孤危憂懣不知所為【懣音悶又音滿】乃遣其度支尚書樂奉業詣江陵求和請依前旨還蜀【度徒洛翻】奉業知紀必敗啓上曰蜀軍乏糧士卒多死危亡可待上遂不許其和【史言上兄弟皆阻兵而安忍】紀以黄金一斤為餅餅百為篋至有百篋銀五倍於金錦罽繒綵稱是每戰懸示將士不以為賞【罽音計繒慈陵翻稱尺證翻將即亮翻下同】寧州刺史陳智祖請散之以募勇士弗聽智祖哭而死有請事者紀稱疾不見由是將卒解體秋七月辛未巴東民符昇等斬峽口城主公孫晃降於王琳【將即亮翻降戶江翻 考異曰典畧作丙戌今從梁書】謝荅仁任約進攻侯叡破之抜其三壘於是兩岸十四城俱降紀不獲退【諸城已降江陵兵斷道故不獲退】順流東下遊擊將軍樊猛追擊之紀衆大潰赴水死者八千餘人猛圍而守之上密敕猛曰生還不成功也猛引兵至紀所紀在舟中繞牀而走以金囊擲猛曰以此雇卿送我一見七官猛曰天子何由可見殺足下金將安之【之往也】遂斬紀及其幼子圓滿陸法和收太子圓照兄弟三人送江陵上絶紀屬籍賜姓饕餮氏【饕他刀翻餮他結翻貪財為饕貪食為餮以帝鴻氏不才子比紀也】下劉孝勝獄已而釋之【紀之稱帝舉兵劉孝勝實鼓成之此而不誅亦失刑也下遐稼翻】上使謂江安侯圓正曰西軍已敗汝父不知存亡意欲使其自裁【圓正見囚見上卷簡文帝大寶二年】圓正聞之號哭稱世子不絶聲【咎圓照之誤紀也號戶刀翻】上頻使覘之知不能死移送廷尉獄見圓照曰兄何乃亂人骨肉使痛酷如此圓照唯云計誤上並命絶食於獄至齧臂啖之十三日而死遠近聞而悲之【覘丑亷翻又丑艶翻齧魚結翻啖徒敢翻又徒濫翻】乙未王僧辯還江陵詔諸軍各還所鎭 魏尉遟迥圍成都五旬永豐侯撝屢出戰皆敗乃請降諸將欲不許迥曰降之則將士全遠人悦攻之則將士傷遠人懼遂受之八月戊戌撝與宜都王圓肅帥文武詣軍門降【撝許韋翻降戶江翻將即亮翻帥讀曰率】迥以禮接之與盟於益州城北吏民皆復其業唯收奴婢及儲積以賞將士軍無私焉【史言尉遟迥能凝蜀人之心】魏以撝及圓肅並為開府儀同三司以迥為大都督益潼等十二州諸軍事益州刺史 庚子下詔將還建康領軍將軍胡僧祐太府卿黄羅漢吏部尚書宗懔【懔力荏翻又巨禁翻】御史中丞劉㲄【㲄音訖岳翻】諫曰建業王氣已盡與虜止隔一江若有不虞悔無及也【建業與齊止隔一江固也獨不思江陵介在江北逼近襄陽岳陽有復讐之志宇文有啓疆之思乎】且古老相承云荆州洲數滿百當出天子今枝江生洲百數已滿【盛弘之荆州記曰自枝江縣西至上明東及江津其中有九十九洲楚諺云洲不百故不出王者】陛下龍飛是其應也上令朝臣議之【朝直遥翻】黄門侍郎周弘正尚書右僕射王褒曰今百姓未見輿駕入建康謂是列國諸王願陛下從四海之望時羣臣多荆州人皆曰弘正等東人也【周覬王導自南渡以來世居建康故謂為東人】志願東下恐非良計弘正面折之曰【折之舌翻】東人勸東謂非良計西人欲西豈成長策上笑又議於後堂會者五百人上問之曰吾欲還建康諸卿以為如何衆莫敢先對上曰勸吾去者左袒左袒者過半武昌太守朱買臣言於上曰建康舊都山陵所在【梁氏自簡文以上葬建康武帝以上葬晉陵守式又翻】荆鎮邊疆非王者之宅【荆州被邊自晉以來為重鎮】願陛下勿疑以致後悔臣家在荆州豈不願陛下居此但恐是臣富貴非陛下富貴耳上使術士杜景豪卜之不吉對上曰未去退而言曰此兆為鬼賊所留也上以建康彫殘江陵全盛意亦安之卒從僧祐等議【史言上懷居違卜以成亡國之禍卒子恤翻】以湘州刺史王琳為衡州刺史 九月庚午詔王僧辯還鎮建康陳霸先復還京口【復扶又翻】丙子以護軍將軍陸法和為郢州刺史法和為政不用刑獄專以沙門法及西域幻術【幻胡辦翻】敎化部曲數千人通謂之弟子 契丹寇齊邊【契欺訖翻】壬午齊主北廵冀定幽安【冀定幽安四州名】遂伐契丹 齊主使郭元建治水軍二萬餘人於合肥【治直之翻】將襲建康納湘潭侯退【退北奔見一百六十二卷武帝太清二年】又遣將軍邢景遠步大汗薩帥衆繼之【步大汗虜三字姓汗音寒薩桑葛翻 考異曰梁書作邢杲遠步六汗薩今從北齊書北史】陳霸先在建康聞之白上上詔王僧辯鎮姑孰以禦之 冬十月丁酉齊主至平州從西道趣長塹【曹操征烏桓出盧龍塞塹山堙谷五百餘里後人因謂之長塹趣七喻翻塹七艶翻】使司徒潘相樂帥精騎五千自東道趣青山辛丑至白狼城壬寅至昌黎城使安德王韓軌帥精騎四千東斷契丹走路【魏收地志營州統内建德郡治白狼城中興初分樂陵置安德郡治般縣帥讀曰率騎奇寄翻】癸卯至陽師水【唐志貞觀三年以契丹室韋部落置師州及陽師縣於營州之廢陽師鎮即此】倍道兼行掩襲契丹齊主露髻肉袒晝夜不息行千餘里踰越山嶺為士卒先唯食肉飲水壯氣彌厲甲辰與契丹遇奮擊大破之虜獲十餘萬口雜畜數百萬頭【畜許救翻】潘相樂又於青山破契丹别部丁未齊主還至營州己酉王僧辯至姑孰遣婺州刺史侯瑱【東陽郡梁置婺州瑱他甸】<br />
<br />
  【翻又音鎮】吳郡太守張彪吳興太守裴之横築壘東關以待齊師 丁巳齊主登碣石山臨滄海遂如晉陽以肆州刺史斛律金為太師召還晉陽拜其子豐樂為武衛大將軍【樂音洛】命其孫武都尚義寧公主寵待之厚羣臣莫及【金子羨字豐樂武都光之子也樂音洛】 閏月丁丑南豫州刺史侯瑱【南豫州時治姑孰瑱他甸翻又音鎮】與郭元建戰於東關齊師大敗溺死者萬計【溺奴狄翻】湘潭侯退復歸於鄴【復扶又翻】王僧辯還建康吳州刺史開建侯蕃恃其兵彊貢獻不入【五代志鄱陽郡梁置】<br />
<br />
  【吳州】上密令其將徐佛受圖之【將即亮翻】佛受使其徒詐為訟者詣蕃遂執之上以佛受為建安太守以侍中王質為吳州刺史質至鄱陽佛受置之金城自據羅城掌門管【左傳秦子曰鄭人使我掌北門之管杜預注曰管籥也】繕治舟艦甲兵【治直之翻艦戶黯翻】質不敢與爭故開建侯部曲數千人攻佛受佛受奔南豫州侯瑱殺之質始得行州事 十一月戊戌以尚書右僕射王褒為左僕射湘東太守張綰為右僕射 己未突厥復攻柔然柔然舉國奔齊【復扶又翻】 癸亥齊主自晉陽北擊突厥迎納柔然廢其可汗庫提立阿那瓌子菴羅辰為可汗置之馬邑川給其廩餼繒帛【餼許氣翻繒慈陵翻】親追突厥於朔州突厥請降【降戶江翻下同】許之而還【還從宣翻又如字】自是貢獻相繼 魏尚書元烈謀殺宇文泰事泄泰殺之【宇文泰於此獨不書其官因舊史成文也】 丙寅上使侍中王琛使於魏【琛丑林翻使疏吏翻】太師泰隂有圖江陵之志梁王詧聞之益重其貢獻【梁王詧欲倚魏以報河東王譽之仇通鑑至此復書梁王詧】 十二月齊宿預民東方白額以城降江西州郡皆起兵應之【江淮之民苦於齊之虐政欲相率而歸江南】<br />
<br />
  三年春正月癸巳齊主自離石討山胡遣斛律金從顯州道【魏收地形志永安中置顯州治汾州六壁城蓋在隋西河郡界】常山王演從晉州道夹攻大破之男子十三以上皆斬 【考異曰北史作十二以上今從典畧】女子及幼弱以賞軍遂平石樓石樓絶險自魏世所不能至【水經河水東逕蒲川石樓山南又南逕蒲城東蓋其地在蒲子縣西五代志汾州樓山縣有北石樓山又有石樓縣舊置吐京郡宋白曰石樓縣本漢土軍縣後魏置吐京郡蓋胡俗譯言音訛變故曰吐京也隋改縣曰石樓】於是遠近山胡莫不懾服【懾之涉翻】有都督戰傷其什長路暉禮不能救帝命刳其五藏【什長十人之長也五藏心肺肝膽腎長陟丈翻藏徂浪翻】令九人食之肉及穢惡皆盡自是始為威虐 陳霸先自丹徒濟江圍齊廣陵秦州刺史嚴超達自秦郡進圍涇州【五代志江都郡永福縣舊曰沛梁置涇州領涇城東陽二郡陳廢涇州併二郡為沛郡後周改沛郡為石梁縣唐併石梁縣入六合北史梁涇州在石梁杜佑曰楊州天長縣梁於石梁置涇州】南豫州刺史侯瑱吳郡太守張彪皆出石梁為之聲援辛丑使晉陵太守杜僧明帥三千人助東方白額【帥讀曰率】 魏太師泰始作九命之典以叙内外官爵改流外品為九秩【五代志曰泰命尚書盧辯遠師周之建職置三公三孤以為論道之官次置六卿以分司庶務其内命謂王朝之臣三公九命三孤八命六卿七命上大夫六命中大夫五命下大夫四命上士三命中士再命下士一命外命謂諸侯及其臣諸公九命諸侯八命諸伯七命諸子六命諸男五命諸公之孤卿四命侯之孤卿公之大夫三命子男之孤卿侯伯之大夫公之上士再命公之中士侯伯之上士一命公之下士侯伯之中士下士子男之士不命其制禄秩下士一百二十五石中士以上至於上大夫各倍之上大夫是為四千石卿二分孤三分公四分各益其一公因盈數為一萬石其九秩一百二十石八秩至於七秩每二秩六分而下各去其一二秩俱為四十石凡頒禄視其年之上下畝至四釡為上年上年頒其正三釡為中年中年頒其半二釡為下年下年頒其一無年為凶荒不頒禄盧辯傳曰柱國大將軍建德四年增置上柱國上將軍也正九命驃騎大將軍開府儀同三司建德四年改為開府儀同大將軍仍增上將軍開府儀同大將軍雍州牧九命驃騎大將軍右光禄大夫車騎將軍左光禄大夫戶三萬以上州刺史正八命征東征南征西征北等將軍右金紫光禄大夫中軍鎮軍撫軍等將軍左金紫光禄大夫都督二萬戶以上州刺史京兆尹八命平東平西平南平北等將軍右銀青光禄大夫前右左後等將軍左銀青光禄大夫帥都督柱國大將軍府長史司馬司錄戶一萬以上州刺史正七命冠軍將軍太中大夫輔國將軍中散大夫都督五千戶以上州刺史戶一萬五千戶以上郡守七命鎮遠將軍諫議大夫建忠將軍朝散大夫州將開府長史司馬司錄戶不滿五千以下州刺史戶一萬以上郡守正六命中堅將軍右中郎將寧朔將軍左中郎將儀同府正八命州長史司馬司錄戶五千以上郡守大呼藥六命寧遠將軍右員外常侍揚烈將軍左員外常侍統軍驃騎車騎將軍府八命州長史司馬司錄柱國大將軍府中郎掾屬戶一千以上郡守長安萬年縣令正五命伏波將軍奉車都尉輕車將軍奉騎都尉四征中鎮撫將軍府正七命州長史司馬司錄開府正中郎掾屬戶不滿一千以下郡守戶七千以上縣令正八命州呼藥五命宣威將軍虎賁給事明威將軍冗從給事儀同府中郎掾屬柱國大將軍府列曹參軍冠軍輔國將軍府正六命州長史司馬司錄正七命州中從事七命郡丞戶四千以上縣令八命州呼藥正四命給事中襄威厲威將軍奉朝請軍主開府列曹參軍冠軍輔國將軍府正六命州長史司馬司錄正七命州别駕正八命州從事七命郡丞戶二千以上縣令正七命州呼藥四命威烈將軍右員外侍郎討寇將軍左員外侍郎幢主儀同府正八命州列曹參軍柱國大將軍府參軍鎮遠建忠中堅寧朔將軍府長史司馬正六命州别駕正七命州從事正六命郡丞五百戶以上縣令七命州呼藥正三命蕩寇將軍武騎常侍蕩難將軍武騎侍郎開府參軍驃騎車騎將軍府正八命州列曹參軍寧遠揚烈伏波輕車將軍府長史正六命州中從事六命郡丞戶不滿五百以下縣令戍主正六命州呼藥三命殄寇將軍彊弩司馬殄難將軍積弩司馬四征中鎮撫將軍府正七命州列曹參軍正五命郡丞正二命掃寇將軍武騎司馬掃難將軍武威司馬四平前右左後將軍府七命州列曹參軍五命郡丞戍副二命曠野將軍殿中司馬横野將軍員外司馬冠軍輔國將軍府正六命州列曹參軍正一命武威將軍淮海都尉虎牙將軍山林都尉鎮遠建中中堅寧朔寧遠揚烈伏波輕車將軍府列曹參軍一命】魏主自元烈之死有怨言密謀誅太師泰臨淮王育廣平王贊垂涕切諫不聽泰諸子皆幼兄子章武公導中山公護皆出鎮【導護皆泰兄顥之子也導鎮上邽】唯以諸婿為心膂大都督清河公李基義城李暉常山公于翼俱為武衛將軍【魏武為丞相有武衛營元魏之制迄于高齊左右衛將軍各一人掌左右廂所主朱華閤以外各武衛將軍二人貳之宇文相魏亦置武衛將軍以掌宿衛而盧辯所定九命無其官此蓋猶在盧辯定官之前以武衛授諸婿然宇文所置如大都督八命帥師都督正七命抑李基等皆以大都督叙官邪至隋始置左右武衛府列于十二衛】分掌禁兵基遠之子暉弼之子翼謹之子也由是魏主謀泄【禁兵既泰諸壻所掌魏主誰與謀哉由是事泄】泰廢魏主置之雍州【置之雍州廨舍雍於用翻】立其弟齊王廓【廓文帝之第四子 考異曰國典云三月廢帝四月立恭帝北史皆在正月今從之】去年號稱元年【去羌呂翻】復姓拓跋氏九十九姓改為單者皆復其舊【單音丹魏改姓見一百四十卷齊明帝建武三年】魏初統國三十六大姓九十九【魏始祖成帝毛統國三十六大姓九十九蓋後漢時匈奴既衰鮮卑始盛之際也】後多滅絶泰乃以諸將功高者為三十六姓次者為九十九姓所將士卒亦改從其姓【洪邁曰西魏以中原故家易賜蕃姓如李弼為徒河氏趙肅趙貴為乙弗氏劉亮為侯莫陳氏楊忠為普六茹氏王雄為可頻氏李虎閻慶為大野氏辛威為普毛氏田宏為紇干氏耿豪為和稽氏王勇為庫汗氏楊紹為叱利氏侯植為侯伏侯氏竇熾為紇豆陵氏李穆為㩉拔氏陸通為步六孤氏楊纂為莫胡盧氏寇售為若口引氏段永為爾綿氏韓褒為侯呂陵氏裴文舉為賀蘭氏陳析為尉遲氏樊深為萬紐于氏將即亮翻】 三月丁亥長沙王韶取巴郡【魏得成都未暇東畧故韶得乘而取之取言易也】 甲辰以王僧辯為太尉車騎大將軍【考異曰典畧作二月甲子今從梁紀】 丁未齊將王球攻宿預杜僧明<br />
<br />
  出擊大破之球歸彭城【將即亮翻下同】 郢州刺史陸法和上啓自稱司徒上怪之王褒曰法和既有道術容或先知戊申上就拜法和為司徒 己酉魏侍中宇文仁恕來聘會齊使者亦至江陵帝接仁恕不及齊使【使疏吏翻】仁恕歸以告太師泰帝又請據舊圖定疆境辭頗不遜泰曰古人有言天之所弃誰能興之【左傳晉胥午之言】其蕭繹之謂乎荆州刺史長孫儉屢陳攻取之策泰徵儉入朝問以經畧復命還鎮密為之備馬伯符密使告帝【武帝太清三年楊忠入寇伯符以下溠城降之因留于魏復扶又翻朝直遙翻】帝弗之信 柔然可汗菴羅辰叛齊齊主自將出擊大破之菴羅辰父子北走太保安定王賀抜仁獻馬不甚駿齊主抜其髪免為庶人輸晉陽負炭 齊中書令魏收撰魏書頗用愛憎為褒貶每謂人曰何物小子敢與魏收作色舉之則使升天按之則使入地既成【東魏孝静天保二年詔魏收撰魏史至是而成】中書舍人盧潜奏收誣罔一代罪當誅尚書左丞盧斐頓丘李庶皆言魏史不直收啓齊主云臣既結怨彊宗【盧李山東望族故以為彊宗】將為刺客所殺帝怒於是斐庶及尚書郎中王松年皆坐謗史鞭二百配甲坊【甲坊造甲之所】斐庶死於獄中濳亦坐繫獄然時人終不服謂之穢史濳度世之曾孫斐同之子松年遵業之子也【盧度世見一百三十二卷宋明帝泰始三年盧同見一百四十八卷梁武帝天監八年王遵業見一百五十二卷大通二年】 夏四月柔然寇齊肆州齊主自晉陽討之至恒州【恒戶登翻】柔然散走帝以二千餘騎為殿【殿丁練翻】宿黄瓜堆柔然别部數萬騎奄至帝安卧平明乃起神色自若指畫形勢縱兵奮擊柔然披靡【披普彼翻】因潰圍而出柔然走追擊之伏尸二十餘里獲菴羅辰妻子虜三萬餘口令都督善無高阿那肱帥騎數千塞其走路時柔然軍猶盛阿那肱以兵少請益【帥讀曰率塞悉則翻少詩詔翻】帝更減其半阿那肱奮擊大破之菴羅辰超越巖谷僅以身免【同一高阿那肱也齊文宣用之則致死以破敵後主用之則賣主以求生盖厲威猶可使之知懼濫恩不足以得其死力也塞悉則反】 丙寅上使散騎常侍庾信等聘於魏【散悉亶翻騎奇寄翻】 癸酉以陳霸先為司空丁未齊主復自擊柔然大破之【復扶又翻】 庚戌魏太師泰酖殺廢帝 五月魏直州人樂熾洋州人黄國等作亂【五代志西城郡安康縣齊置安康郡魏置東梁州西魏改曰直州漢川郡西鄉縣舊曰豐寧置洋州及洋川郡考漢川志蜀分漢成固縣立南鄉縣晉改為西郷縣魏廢縣仍於豐寧戍置豐寧縣】開府儀同三司高平田弘河南賀若敦討之不克【若人者翻】太師泰命車騎大將軍李遷哲與敦共討熾等平之仍與敦南出狥地至巴州【後漢分宕渠北界置漢昌縣蜀先主置巴西郡宋武帝置歸化郡魏於漢昌縣治置大谷郡又於郡北置巴州五代志清化郡化成縣梁置歸化郡及巴州】巴州刺史牟安民降之 【考異曰典畧云斬梁巴州刺史牟安平今從周書北史】巴濮之民皆附於魏【春秋巴子之國三巴郡地是也春秋百濮之地在西城上庸之間濮博木翻】蠻酋向五子王䧟白帝【酋慈秋翻】遷哲擊之五子王遁去遷哲追擊破之泰以遷哲為信州刺史鎮白帝信州先無儲蓄遷哲與軍士共采葛根為糧時有異味輒分嘗之軍士感悦屢擊叛蠻破之羣蠻懾服皆送糧餼遣子弟入質【懾之涉翻愾許氣翻質音致】由是州境安息軍儲亦贍 柔然乙旃達官寇魏廣武【魏收志東夏州偏城郡帶廣武縣五代志延安郡豐林縣後魏置廣武縣及偏城郡宋熙寧九年省豐林為鎮併屬膚施縣】柱國李弼遣擊破之【遣擊恐當作追擊】 廣州刺史曲江侯勃自以非上所授【陳霸先推蕭勃為廣州刺史見一百六十二卷武帝太清三年】内不自安上亦疑之勃唘求入朝【朝直遥翻】五月乙巳上以王琳為廣州刺史勃為晉州刺史【五代志同安郡梁置豫州後改曰晉州】上以琳部衆彊盛又得衆心故欲遠之【遠于願翻】琳與主書廣漢李膺厚善私謂膺曰琳小人也蒙官拔擢至此今天下未定遷琳嶺南如有不虞安得琳力竊揆官意不過疑琳琳分望有限【言自揆分不敢懷非望也分扶問翻】豈與官爭為帝乎何不以琳為雍州刺史鎮武寧【雍於用翻】琳自放兵作田為國禦捍膺然其言而弗敢啓【史言王琳忠於所事而帝不能用為國于偽翻】 散騎郎新野庾季才言於上曰去年八月丙申月犯心中星今月丙戌赤氣干北斗心為天王丙主楚分【分扶問翻】臣恐建子之月有大兵入江陵陛下宜留重臣鎮江陵整斾還都以避其患假令魏虜侵蹙止失荆湘在於社稷猶得無慮上亦曉天文知楚有災歎曰禍福在天避之何益【天之警帝未棄帝也帝不知避是自棄也】 六月壬午齊步大汗薩將兵四萬趣涇州王僧辯使侯瑱張彪自石梁引兵助嚴超達拒之瑱彪遲留不進將軍尹令思將萬餘人謀襲盱眙【盱眙音怡】齊冀州刺史段韶將兵討東方白額於宿預廣陵涇州皆來告急諸將患之韶曰梁氏喪亂【喪息浪翻】國無定主人懷去就彊者從之霸先等外託同德内有離心諸君不足憂吾揣之熟矣【揣初委翻】乃留儀同三司敬顯携等圍宿預【敬顯携當作敬顯儁】自引兵倍道趣涇州塗出盱眙令思不意齊師猝至望風退走韶進擊超達破之囘趣廣陵陳霸先解圍走【趣七喻翻】杜僧明還丹徒侯瑱張彪還秦郡吳明徹圍海西【海西縣前漢屬東海郡後漢屬廣陵郡齊明帝置東海郡東魏武定七年改海西郡今西海州即其地】鎮將中山郎基固守削木為箭翦紙為羽圍之十旬卒不能克而還【將即亮翻下同卒子恤翻】 柔然帥餘衆東徙且欲南寇齊主帥輕騎邀之於金川【唐志單于府帶金河縣其即金川歟帥讀曰率】柔然聞之遠遁營州刺史靈丘王峻設伏擊之獲其名王數十人鄧至羌檐□失國【檐余亷翻□戶庚翻】奔魏太師泰使秦州刺史宇文導將兵納之 齊段韶還至宿預使辯士說東方白額【說式芮翻】白額開門請盟因執而斬之 秋七月庚戌齊主還鄴 魏太師泰西廵至原州 八月壬辰齊以司州牧清河王岳為太保司空尉粲為司徒太子太師侯莫陳相為司空尚書令平陽王淹錄尚書事常山王演為尚書令中書令上黨王渙為左僕射 乙亥齊儀同三司元旭坐事賜死丁丑齊主如晉陽齊主之未為魏相也【相息亮翻】太保錄尚書事平原王高隆之常侮之及將受禪隆之復以為不可【事見一百六十三卷簡文帝大寶元年復扶又翻】齊主由是銜之崔季舒譖隆之每見訴訟者輒加哀矜之意以示非已能裁帝禁之尚書省【崔季舒報徙邊之怨也】隆之嘗與元旭飲謂旭曰與王交當生死不相負人有密言之者帝由是怒令壯士築百餘拳而捨之辛巳卒於路【卒子恤翻】久之帝追忿隆之執其子慧登等二十人於前帝以鞭叩鞍一時頭絶並投尸漳水又發隆之冢出其尸斬截骸骨焚之棄於漳水 齊主使常山王演上黨王渙清河王岳平原王段韶帥衆於洛陽西南築伐惡城新城嚴城河南城【帥讀曰率】九月齊主廵四城欲以致魏師魏師不出【史言齊強宇文泰畏之】乃如晉陽 魏宇文泰命侍中崔猷開囘車路以通漢中【按北史崔猷傳泰欲開梁漢舊路乃命猷開通車路鑿山堙谷五百餘里至於梁州此特因舊路開而廣之以通車耳前史蓋誤以通字為迴傳寫者又去其傍為囘也泰不書官而書姓亦無義例之可言也】 帝好玄談【好呼到翻】辛卯於龍光殿講老子 曲江侯勃遷居始興王琳使副將孫瑒先行據番禺【將即亮翻番禺音潘愚】 乙巳魏遣柱國常山公于謹中山公宇文護大將軍楊忠將兵五萬入寇冬十月壬戌長安長孫儉問謹曰為蕭繹之計將如之何謹曰耀兵漢沔席捲度江直據丹楊上策也【謂東遷建康也卷讀曰捲】移郭内居民退保子城峻其陴堞以待援軍中策也【陴頻彌翻堞達協翻】若難於移動據守羅郭下策也儉曰揣繹定出何策【揣初委翻下同】謹曰下策儉曰何故謹曰蕭氏保據江南綿歷數紀【十二年為一紀】屬中原多故【屬之欲翻】未遑外畧又以我有齊氏之患必謂力不能分且繹懦而無謀多疑少斷【少詩詔翻斷丁亂翻】愚民難與慮始皆戀邑居所以知其用下策也癸亥武寧太守宗均告魏兵且至帝召公卿議之領軍胡僧祐太府卿黄羅漢曰二國通好未有嫌隙必應不爾【江陵諸將胡僧祐其巨擘也識見如此烏能敵于謹哉好呼到翻】侍中王琛曰臣揣宇文容色必無此理【去年王琛使魏故自謂揣其容色必無此事可謂不善於覘國者矣】乃復使琛使魏【復扶又翻下復講同】丙寅于謹至樊鄧梁王詧帥衆會之辛卯帝停講【停講老子也帥讀曰率下同】内外戒嚴王琛至石梵【杜佑曰石梵在沔州沔口上又據梁書安成王秀傳石梵時屬竟陵界】未見魏軍馳書報黄羅漢曰吾至石梵境上帖然前言皆兒戲耳帝聞而疑之庚午復講百官戎服以聽辛未帝使主書李膺至建康徵王僧辯為大都督荆州刺史命陳霸先徙鎭揚州僧辯遣豫州刺史侯瑱帥新安程靈洗等為前軍兖州刺史杜僧明帥吳明徹等為後軍【王僧辯一聞徵命當投袂勤王可也外言部分諸軍不聞星馳電赴江陵覆没僧辯之罪也帥讀曰率】甲戌帝夜登鳳凰閣徙倚歎息曰客星入翼軫【倚欄而又徙處為徙倚翼軫楚荆州分】今必敗矣嬪御皆泣【嬪毘賓翻】陸法和聞魏師至自郢州入漢口將赴江陵帝使逆之曰此自能破賊但鎭郢州不須動也法和還州堊其城門【堊烏各翻以白土塗城門示有喪也】著衰絰坐葦席終日乃脫之【著陟畧翻喪倉回翻此法和預為喪君之服設使法和果至江陵亦不能制魏兵之攻圍此其徒欲神法和之術託為之言以為能知來耳】十一月帝大閱於津陽門外【江左都建康外城十二門門名皆用洛城門名帝都江陵外城門亦依建康城門名之津陽門城南面東來第二門】遇北風暴雨輕輦還宫癸未魏軍濟漢于謹令宇文護楊忠帥精騎先據江津斷東路【斷音短】甲申護克武寧執宗均是日帝乘馬出城行栅【行下孟翻下廵行同】挿木為之周圍六十餘里以領軍將軍胡僧祐都督城東諸軍事尚書右僕射張綰為之副左僕射王褒都督城西諸軍事四廂領直元景亮為之副王公已下各有所守丙戌命太子廵行城樓令居人助運木石夜魏軍至黄華去江陵四十里丁亥至柵下戊子嶲州刺史裴畿【越嶲郡梁置嶲州嶲音髓】畿弟新興太守機武昌太守朱買臣衡陽太守謝答仁開枇杷門出戰裴機殺魏儀同三司胡文伐畿之高之子也【臺城既没裴之高赴江陵】帝徵廣州刺史王琳為湘東刺史使引兵入援丁酉栅内火焚數千家及城樓二十五帝臨所焚樓望魏軍濟江四顧歎息是夜遂止宫外宿民家己亥移居祇洹寺【祇巨支翻洹胡官翻】于謹令築長圍中外信命始絶庚子信州刺史徐世譜晉安王司馬任約等築壘於馬頭【江陵南岸謂馬頭岸】遥為聲援是夜帝廵城猶口占為詩羣臣亦有和者【和戶卧翻】帝裂帛為書趣王僧辯曰【趣讀曰促】吾忍死待公可以至矣壬寅還宫癸卯出長沙寺戊申王褒胡僧祐朱買臣謝答仁等開門出戰皆敗還己酉帝移居天居寺癸丑移居長沙寺朱買臣按劒進曰唯斬宗懍黄羅漢可以謝天下【買臣罪其諫還建康也懍力任翻又力禁翻】帝曰曩實吾意宗黄何罪二人退入衆中王琳軍至長沙鎮南府長史裴政請間道先報江陵【王琳為鎮南將軍以裴政為府長史間古莧翻下間使同】至百里洲為魏人所獲梁王詧謂政曰我武皇帝之孫也不可為爾君乎若從我計貴及子孫如或不然腰領分矣政詭對曰唯命詧鎻之至城下使言曰王僧辯聞臺城被圍【時都江陵上臺所在故亦謂之臺城】已自為帝王琳孤弱不能復來【復扶又翻】政告城中曰援兵大至各思自勉吾以間使被擒當碎身報國監者擊其口【使疏吏翻監工銜翻】詧怒使速殺之西中郎參軍蔡大業【梁置西中郎將於襄陽以蔡大業為參軍】諫曰此民望也殺之則荆州不可下矣乃釋之政之禮之子【裴之禮邃之子也】大業大寶之弟也時徵兵四方皆未至甲寅魏人百道攻城 【考異曰梁紀作辛卯誤也今從典畧】城中負戶蒙楯【楯食尹翻】胡僧祐親當矢石晝夜督戰奬勵將士明行賞罰衆咸致死所向摧殄魏不得前俄而僧祐中流矢死【中竹仲翻】内外大駭魏悉衆攻栅反者開西門納魏師帝與太子王褒謝答仁朱買臣退保金城令汝南王大封晉熙王大圓質於于謹以請和【大封大圓皆簡文帝之子質音致下同】魏軍之初至也衆以王僧辯子侍中頠可為都督【頠魚豈翻】帝不用更奪其兵使與左右十人入守殿中及胡僧祐死乃用為都督城中諸軍事裴畿裴機歷陽侯峻皆出降【降戶江翻下同】于謹以機手殺胡文伐并畿殺之峻淵猷之子也【淵猷長沙王懿之子】時城南雖破而城北諸將猶苦戰日暝聞城䧟乃散【暝莫定翻】帝入東閣竹殿命舍人高善寶焚古今圖書十四萬卷 【考異曰隋經籍志云焚七萬卷南史云十餘萬卷按周僧辯所送建康書已八萬卷并江陵舊書豈止七萬卷乎今從典略周當作王】將自赴火宫人左右共止之又以寶劒斫柱令折歎曰文武之道今夜盡矣【焚書折劔以為文武道盡折而設翻】乃使御史中丞王孝祀作降文謝荅仁朱買臣諫曰城中兵衆猶彊乘闇突圍而出賊必驚因而薄之可度江就任約【任約築壘馬頭岸與江陵僅隔一江耳】帝素不便走馬曰事必無成祗增辱耳荅仁求自扶帝以問王褒褒曰荅仁侯景之黨豈足可信成彼之勲不如降也荅仁又請守子城收兵可得五千人帝然之即授城中大都督配以公主既而召王褒謀之以為不可荅仁請入不得歐血而去【歐烏口翻】于謹徵太子為質帝使王褒送之謹子以褒善書給之紙筆乃書曰柱國常山公家奴王褒【謹為柱國大將軍封常山公褒以此自處安能為帝謀乎】有頃黄門郎裴政犯門而出帝遂去羽儀文物【去羌呂翻】白馬素衣出東門抽劍擊闔曰蕭世誠一至此乎【左傳晉州綽攻齊東門以枚數闔杜預註曰闔門扇也世誠帝字也】魏軍士度塹牽其轡【塹七艶翻】至白馬寺北奪其所乘駿馬以駑馬代之遣長壯胡人手扼其背以行逢于謹胡人牽帝使拜 【考異曰典畧云謹撝梁主令西至龍泉廟出武陵河東二王子孫於獄列於沙州鎻械嚴酷瘡痍腐爛引梁主使視之謂曰此皆骨肉忍虐如此何以為君上無以應按武陵諸子先已餓死河東子孫亦應不存今不取】梁王詧使鐵騎擁帝入營囚于烏幔之下【幔莫半翻】甚為詧所詰辱【詰去吉翻】乙卯于謹令開府儀同三司長孫儉入據金城帝紿儉云城中埋金千斤欲以相贈儉乃將帝入城帝因述詧見辱之狀謂儉曰向聊相紿欲言此耳【紿蕩亥翻】豈有天子自埋金乎儉乃留帝於主衣庫【此主衣庫在江陵金城中之禁中】帝性殘忍且懲高祖寛縱之弊故為政尚嚴及魏師圍城獄中死囚且數千人有司請釋之以充戰士帝不許悉令棓殺之【棓蒲頃翻】事未成而城䧟中書郎殷不害先於别所督戰城䧟失其母時氷雪交積凍死者填滿溝塹不害行哭於道求其母尸無所不至見溝中死人輒投下捧視舉體凍濕水漿不入口號哭不輟聲【號戶刀翻】如是七日乃得之十二月丙辰徐世譜任約退戍巴陵于謹逼帝使為書召王僧辯帝不可使者曰王今豈得自由帝曰我既不自由僧辯亦不由我又從長孫儉求宫人王氏苟氏及幼子犀首儉並還之或問何意焚書帝曰讀書萬卷猶有今日故焚之【帝之亡國固不由讀書也】 庚申齊主北廵至達速嶺行視山川險要將起長城 辛未帝為魏人所殺【年四十七】梁王詧遣尚書傅凖監刑【監工銜翻】以土囊隕之詧使以布帊纒尸【帊普駕翻通俗文曰三幅為帊】歛以蒲席束以白茅【歛力贍翻】葬於津陽門外并殺愍懷太子元良始安王方略桂陽王大成等世祖性好書【梁王方智承制諡帝曰元廟號世祖好呼到翻】常令左右讀書晝夜不絶雖熟睡卷猶不釋或差誤及欺之帝輒驚窹作文章援筆立就常言我韜於文士【今人謂器幣有餘用者為寛韜與此韜同義】愧於武夫論者以為得言【得言謂其自言者此為得之】魏立梁王詧為梁主資以荆州之地延袤三百里【資以江陵緣江之地延袤三百里廣不及三百里也袤音茂】仍取其雍州之地【雍於用翻】詧居江陵東城魏置防主將兵居西城名曰助防外示助詧備禦内實防之【魏克江陵因取襄樊之地此正滅虢取虞之計詧雖悔之何及矣】以前儀同三司王悦留鎮江陵于謹收府庫珍寶及宋渾天儀梁銅晷表【宋渾天儀元嘉十三年錢樂之所鑄也梁銅晷表武帝所造】大玉徑四尺及諸法物盡俘王公以下及選百姓男女數萬口為奴婢 【考異曰典畧作五十萬今從梁紀南史】分賞三軍驅歸長安小弱者皆殺之得免者三百餘家而人馬所踐及凍死者十二三【踐慈演翻下同】魏師之在江陵也梁王詧將尹德毅說詧曰魏虜貪惏【說式芮翻惏與婪同盧含翻】肆其殘忍殺掠士民不可勝紀【勝音升】江東之人塗炭至此咸謂殿下為之殿下既殺人父兄孤人子弟人盡讐也誰與為國今魏之精銳盡萃於此若殿下為設享會【下為于偽翻】請于謹等為歡預伏武士因而斃之分命諸將掩其營壘大殱羣醜俾無遺類【殱息亷翻】收江陵百姓撫而安之文武羣僚隨材銓授魏人懾息未敢送死【懾之涉翻】王僧辯之徒折簡可致然後朝服濟江入踐皇極【謂還建康即位也朝直遥翻】晷刻之間大功可立古人云天與不取反受其咎【漢蒯通之言】願殿下恢弘遠畧勿懷匹夫之行【匹夫之行小亷小謹以自託於郷黨行下孟翻】詧曰卿此策非不善也然魏人待我厚未可背德【背蒲妹翻】若遽為卿計人將不食吾餘【左傳鄧祁侯之言】既而闔城老幼被虜又失襄陽【被皮義翻】詧乃歎曰恨不用尹德毅之言王僧辯陳霸先等共奉江州刺史晉安王方智為太宰承制王褒王克劉㲄宗懔殷不害及尚書右丞吳興沈烱至長安【㲄古岳翻烱古㢠翻】太師泰皆厚禮之泰親至于謹第宴勞極歡【勞力到翻】賞謹奴婢千口及梁之寶物并雅樂一部别封新野公【既封常山又封新野故曰别封】謹固辭不許謹自以久居重任功名既立欲保優閑乃上先所乘駿馬及所著鎧甲等【上時掌翻著陟畧翻鎧可亥翻】泰識其意曰今巨猾未平公豈得遽爾獨善【巨猾謂齊孟子曰達則兼善天下窮則獨善其身】遂不受 是歲魏秦州刺史章武孝公宇文導卒 魏加益州刺史尉遟㢠督六州通前十八州自劍閣以南得承制封拜及黜陟㢠明賞罰布威恩綏輯新民經畧未附華夷懷之<br />
<br />
  資治通鑑卷一百六十五  <br>
   </div> 

<script src="/search/ajaxskft.js"> </script>
 <div class="clear"></div>
<br>
<br>
 <!-- a.d-->

 <!--
<div class="info_share">
</div> 
-->
 <!--info_share--></div>   <!-- end info_content-->
  </div> <!-- end l-->

<div class="r">   <!--r-->



<div class="sidebar"  style="margin-bottom:2px;">

 
<div class="sidebar_title">工具类大全</div>
<div class="sidebar_info">
<strong><a href="http://www.guoxuedashi.com/lsditu/" target="_blank">历史地图</a></strong>  
<a href="http://www.880114.com/" target="_blank">英语宝典</a>  
<a href="http://www.guoxuedashi.com/13jing/" target="_blank">十三经检索</a> 
<br><strong><a href="http://www.guoxuedashi.com/gjtsjc/" target="_blank">古今图书集成</a></strong> 
<a href="http://www.guoxuedashi.com/duilian/" target="_blank">对联大全</a> <strong><a href="http://www.guoxuedashi.com/xiangxingzi/" target="_blank">象形文字典</a></strong> 

<br><a href="http://www.guoxuedashi.com/zixing/yanbian/">字形演变</a>  <strong><a href="http://www.guoxuemi.com/hafo/" target="_blank">哈佛燕京中文善本特藏</a></strong>
<br><strong><a href="http://www.guoxuedashi.com/csfz/" target="_blank">丛书&方志检索器</a></strong> <a href="http://www.guoxuedashi.com/yqjyy/" target="_blank">一切经音义</a>  

<br><strong><a href="http://www.guoxuedashi.com/jiapu/" target="_blank">家谱族谱查询</a></strong>  <strong><a href="http://shufa.guoxuedashi.com/sfzitie/" target="_blank">书法字帖欣赏</a></strong> 
<br>

</div>
</div>


<div class="sidebar" style="margin-bottom:0px;">

<font style="font-size:22px;line-height:32px">QQ交流群9:489193090</font>


<div class="sidebar_title">手机APP 扫描或点击</div>
<div class="sidebar_info">
<table>
<tr>
	<td width=160><a href="http://m.guoxuedashi.com/app/" target="_blank"><img src="/img/gxds-sj.png" width="140"  border="0" alt="国学大师手机版"></a></td>
	<td>
<a href="http://www.guoxuedashi.com/download/" target="_blank">app软件下载专区</a><br>
<a href="http://www.guoxuedashi.com/download/gxds.php" target="_blank">《国学大师》下载</a><br>
<a href="http://www.guoxuedashi.com/download/kxzd.php" target="_blank">《汉字宝典》下载</a><br>
<a href="http://www.guoxuedashi.com/download/scqbd.php" target="_blank">《诗词曲宝典》下载</a><br>
<a href="http://www.guoxuedashi.com/SiKuQuanShu/skqs.php" target="_blank">《四库全书》下载</a><br>
</td>
</tr>
</table>

</div>
</div>


<div class="sidebar2">
<center>


</center>
</div>

<div class="sidebar"  style="margin-bottom:2px;">
<div class="sidebar_title">网站使用教程</div>
<div class="sidebar_info">
<a href="http://www.guoxuedashi.com/help/gjsearch.php" target="_blank">如何在国学大师网下载古籍?</a><br>
<a href="http://www.guoxuedashi.com/zidian/bujian/bjjc.php" target="_blank">如何使用部件查字法快速查字?</a><br>
<a href="http://www.guoxuedashi.com/search/sjc.php" target="_blank">如何在指定的书籍中全文检索?</a><br>
<a href="http://www.guoxuedashi.com/search/skjc.php" target="_blank">如何找到一句话在《四库全书》哪一页?</a><br>
</div>
</div>


<div class="sidebar">
<div class="sidebar_title">热门书籍</div>
<div class="sidebar_info">
<a href="/so.php?sokey=%E8%B5%84%E6%B2%BB%E9%80%9A%E9%89%B4&kt=1">资治通鉴</a> <a href="/24shi/"><strong>二十四史</strong></a>&nbsp; <a href="/a2694/">野史</a>&nbsp; <a href="/SiKuQuanShu/"><strong>四库全书</strong></a>&nbsp;<a href="http://www.guoxuedashi.com/SiKuQuanShu/fanti/">繁体</a>
<br><a href="/so.php?sokey=%E7%BA%A2%E6%A5%BC%E6%A2%A6&kt=1">红楼梦</a> <a href="/a/1858x/">三国演义</a> <a href="/a/1038k/">水浒传</a> <a href="/a/1046t/">西游记</a> <a href="/a/1914o/">封神演义</a>
<br>
<a href="http://www.guoxuedashi.com/so.php?sokeygx=%E4%B8%87%E6%9C%89%E6%96%87%E5%BA%93&submit=&kt=1">万有文库</a> <a href="/a/780t/">古文观止</a> <a href="/a/1024l/">文心雕龙</a> <a href="/a/1704n/">全唐诗</a> <a href="/a/1705h/">全宋词</a>
<br><a href="http://www.guoxuedashi.com/so.php?sokeygx=%E7%99%BE%E8%A1%B2%E6%9C%AC%E4%BA%8C%E5%8D%81%E5%9B%9B%E5%8F%B2&submit=&kt=1"><strong>百衲本二十四史</strong></a>  <a href="http://www.guoxuedashi.com/so.php?sokeygx=%E5%8F%A4%E4%BB%8A%E5%9B%BE%E4%B9%A6%E9%9B%86%E6%88%90&submit=&kt=1"><strong>古今图书集成</strong></a>
<br>

<a href="http://www.guoxuedashi.com/so.php?sokeygx=%E4%B8%9B%E4%B9%A6%E9%9B%86%E6%88%90&submit=&kt=1">丛书集成</a> 
<a href="http://www.guoxuedashi.com/so.php?sokeygx=%E5%9B%9B%E9%83%A8%E4%B8%9B%E5%88%8A&submit=&kt=1"><strong>四部丛刊</strong></a>  
<a href="http://www.guoxuedashi.com/so.php?sokeygx=%E8%AF%B4%E6%96%87%E8%A7%A3%E5%AD%97&submit=&kt=1">說文解字</a> <a href="http://www.guoxuedashi.com/so.php?sokeygx=%E5%85%A8%E4%B8%8A%E5%8F%A4&submit=&kt=1">三国六朝文</a>
<br><a href="http://www.guoxuedashi.com/so.php?sokeytm=%E6%97%A5%E6%9C%AC%E5%86%85%E9%98%81%E6%96%87%E5%BA%93&submit=&kt=1"><strong>日本内阁文库</strong></a> <a href="http://www.guoxuedashi.com/so.php?sokeytm=%E5%9B%BD%E5%9B%BE%E6%96%B9%E5%BF%97%E5%90%88%E9%9B%86&ka=100&submit=">国图方志合集</a> <a href="http://www.guoxuedashi.com/so.php?sokeytm=%E5%90%84%E5%9C%B0%E6%96%B9%E5%BF%97&submit=&kt=1"><strong>各地方志</strong></a>

</div>
</div>


<div class="sidebar2">
<center>

</center>
</div>
<div class="sidebar greenbar">
<div class="sidebar_title green">四库全书</div>
<div class="sidebar_info">

《四库全书》是中国古代最大的丛书,编撰于乾隆年间,由纪昀等360多位高官、学者编撰,3800多人抄写,费时十三年编成。丛书分经、史、子、集四部,故名四库。共有3500多种书,7.9万卷,3.6万册,约8亿字,基本上囊括了古代所有图书,故称“全书”。<a href="http://www.guoxuedashi.com/SiKuQuanShu/">详细>>
</a>

</div> 
</div>

</div>  <!--end r-->

</div>
<!-- 内容区END --> 

<!-- 页脚开始 -->
<div class="shh">

</div>

<div class="w1180" style="margin-top:8px;">
<center><script src="http://www.guoxuedashi.com/img/plus.php?id=3"></script></center>
</div>
<div class="w1180 foot">
<a href="/b/thanks.php">特别致谢</a> | <a href="javascript:window.external.AddFavorite(document.location.href,document.title);">收藏本站</a> | <a href="#">欢迎投稿</a> | <a href="http://www.guoxuedashi.com/forum/">意见建议</a> | <a href="http://www.guoxuemi.com/">国学迷</a> | <a href="http://www.shuowen.net/">说文网</a><script language="javascript" type="text/javascript" src="https://js.users.51.la/17753172.js"></script><br />
  Copyright &copy; 国学大师 古典图书集成 All Rights Reserved.<br>
  
  <span style="font-size:14px">免责声明:本站非营利性站点,以方便网友为主,仅供学习研究。<br>内容由热心网友提供和网上收集,不保留版权。若侵犯了您的权益,来信即刪。scp168@qq.com</span>
  <br />
ICP证:<a href="http://www.beian.miit.gov.cn/" target="_blank">鲁ICP备19060063号</a></div>
<!-- 页脚END --> 
<script src="http://www.guoxuedashi.com/img/plus.php?id=22"></script>
<script src="http://www.guoxuedashi.com/img/tongji.js"></script>

</body>
</html>
