<!DOCTYPE html PUBLIC "-//W3C//DTD XHTML 1.0 Transitional//EN" "http://www.w3.org/TR/xhtml1/DTD/xhtml1-transitional.dtd">
<html xmlns="http://www.w3.org/1999/xhtml">
<head>
<meta http-equiv="Content-Type" content="text/html; charset=utf-8" />
<meta http-equiv="X-UA-Compatible" content="IE=Edge,chrome=1">
<title>資治通鑒_141-資治通鑑卷一百四十_141-資治通鑑卷一百四十</title>
<meta name="Keywords" content="資治通鑒_141-資治通鑑卷一百四十_141-資治通鑑卷一百四十">
<meta name="Description" content="資治通鑒_141-資治通鑑卷一百四十_141-資治通鑑卷一百四十">
<meta http-equiv="Cache-Control" content="no-transform" />
<meta http-equiv="Cache-Control" content="no-siteapp" />
<link href="/img/style.css" rel="stylesheet" type="text/css" />
<script src="/img/m.js?2020"></script> 
</head>
<body>
 <div class="ClassNavi">
<a  href="/24shi/">二十四史</a> | <a href="/SiKuQuanShu/">四库全书</a> | <a href="http://www.guoxuedashi.com/gjtsjc/"><font  color="#FF0000">古今图书集成</font></a> | <a href="/renwu/">历史人物</a> | <a href="/ShuoWenJieZi/"><font  color="#FF0000">说文解字</a></font> | <a href="/chengyu/">成语词典</a> | <a  target="_blank"  href="http://www.guoxuedashi.com/jgwhj/"><font  color="#FF0000">甲骨文合集</font></a> | <a href="/yzjwjc/"><font  color="#FF0000">殷周金文集成</font></a> | <a href="/xiangxingzi/"><font color="#0000FF">象形字典</font></a> | <a href="/13jing/"><font  color="#FF0000">十三经索引</font></a> | <a href="/zixing/"><font  color="#FF0000">字体转换器</font></a> | <a href="/zidian/xz/"><font color="#0000FF">篆书识别</font></a> | <a href="/jinfanyi/">近义反义词</a> | <a href="/duilian/">对联大全</a> | <a href="/jiapu/"><font  color="#0000FF">家谱族谱查询</font></a> | <a href="http://www.guoxuemi.com/hafo/" target="_blank" ><font color="#FF0000">哈佛古籍</font></a> 
</div>

 <!-- 头部导航开始 -->
<div class="w1180 head clearfix">
  <div class="head_logo l"><a title="国学大师官网" href="http://www.guoxuedashi.com" target="_blank"></a></div>
  <div class="head_sr l">
  <div id="head1">
  
  <a href="http://www.guoxuedashi.com/zidian/bujian/" target="_blank" ><img src="http://www.guoxuedashi.com/img/top1.gif" width="88" height="60" border="0" title="部件查字,支持20万汉字"></a>


<a href="http://www.guoxuedashi.com/help/yingpan.php" target="_blank"><img src="http://www.guoxuedashi.com/img/top230.gif" width="600" height="62" border="0" ></a>


  </div>
  <div id="head3"><a href="javascript:" onClick="javascript:window.external.AddFavorite(window.location.href,document.title);">添加收藏</a>
  <br><a href="/help/setie.php">搜索引擎</a>
  <br><a href="/help/zanzhu.php">赞助本站</a></div>
  <div id="head2">
 <a href="http://www.guoxuemi.com/" target="_blank"><img src="http://www.guoxuedashi.com/img/guoxuemi.gif" width="95" height="62" border="0" style="margin-left:2px;" title="国学迷"></a>
  

  </div>
</div>
  <div class="clear"></div>
  <div class="head_nav">
  <p><a href="/">首页</a> | <a href="/ShuKu/">国学书库</a> | <a href="/guji/">影印古籍</a> | <a href="/shici/">诗词宝典</a> | <a   href="/SiKuQuanShu/gxjx.php">精选</a> <b>|</b> <a href="/zidian/">汉语字典</a> | <a href="/hydcd/">汉语词典</a> | <a href="http://www.guoxuedashi.com/zidian/bujian/"><font  color="#CC0066">部件查字</font></a> | <a href="http://www.sfds.cn/"><font  color="#CC0066">书法大师</font></a> | <a href="/jgwhj/">甲骨文</a> <b>|</b> <a href="/b/4/"><font  color="#CC0066">解密</font></a> | <a href="/renwu/">历史人物</a> | <a href="/diangu/">历史典故</a> | <a href="/xingshi/">姓氏</a> | <a href="/minzu/">民族</a> <b>|</b> <a href="/mz/"><font  color="#CC0066">世界名著</font></a> | <a href="/download/">软件下载</a>
</p>
<p><a href="/b/"><font  color="#CC0066">历史</font></a> | <a href="http://skqs.guoxuedashi.com/" target="_blank">四库全书</a> |  <a href="http://www.guoxuedashi.com/search/" target="_blank"><font  color="#CC0066">全文检索</font></a> | <a href="http://www.guoxuedashi.com/shumu/">古籍书目</a> | <a   href="/24shi/">正史</a> <b>|</b> <a href="/chengyu/">成语词典</a> | <a href="/kangxi/" title="康熙字典">康熙字典</a> | <a href="/ShuoWenJieZi/">说文解字</a> | <a href="/zixing/yanbian/">字形演变</a> | <a href="/yzjwjc/">金 文</a> <b>|</b>  <a href="/shijian/nian-hao/">年号</a> | <a href="/diming/">历史地名</a> | <a href="/shijian/">历史事件</a> | <a href="/guanzhi/">官职</a> | <a href="/lishi/">知识</a> <b>|</b> <a href="/zhongyi/">中医中药</a> | <a href="http://www.guoxuedashi.com/forum/">留言反馈</a>
</p>
  </div>
</div>
<!-- 头部导航END --> 
<!-- 内容区开始 --> 
<div class="w1180 clearfix">
  <div class="info l">
   
<div class="clearfix" style="background:#f5faff;">
<script src='http://www.guoxuedashi.com/img/headersou.js'></script>

</div>
  <div class="info_tree"><a href="http://www.guoxuedashi.com">首页</a> > <a href="/SiKuQuanShu/fanti/">四库全书</a>
 > <h1>资治通鉴</h1> <!--         下载:【右键另存为】即可 --></div>
  <div class="info_content zj clearfix">
  
<div class="info_txt clearfix" id="show">
<center style="font-size:24px;">141-資治通鑑卷一百四十</center>
    資治通鑑卷一百四十  宋 司馬光 撰<br />
<br />
  胡三省 音註<br />
<br />
  齊紀六【起旃蒙大淵獻盡柔兆困敦凡二年】<br />
<br />
  高宗明皇帝中<br />
<br />
  建武二年春正月壬申遣鎮南將軍王廣之督司州右衛將軍蕭坦之督徐州尚書右僕射沈文季督豫州諸軍以拒魏癸酉魏詔淮北之人不得侵掠犯者以大辟論【淮北時已屬魏故詔不得侵掠其人辟毗亦翻】乙未拓拔衍攻鍾離徐州刺史蕭惠休乘城拒守間出襲擊魏兵破之惠休惠明之弟也【間古莧翻蕭惠明見一百三十三卷宋蒼梧王元徽二年】劉昶王肅攻義陽【昶知兩翻】司州刺史蕭誕拒之肅屢破誕兵招降萬餘人【降戶江翻】魏以肅為豫州刺史劉昶性褊躁御軍嚴暴【褊補典翻躁則到翻】人莫敢言法曹行參軍北平陽固苦諫昶怒欲斬之使當攻道【攻道攻城之道矢石之所集也】固志意閒雅臨敵勇决昶始奇之丁酉中外纂嚴以太尉陳顯達為使持節都督西北討諸軍事往來新亭白下以張聲埶【使疏吏翻下同】己亥魏主濟淮二月至夀陽衆號三十萬鐵騎彌望【彌望猶言極望也孔穎達曰人目所望三十里而天地合於三十里外不復見之是為極望騎奇寄翻】甲辰魏主登八公山賦詩道遇甚雨命去蓋【去羌呂翻】見軍士病者親撫慰之魏主遣使呼城中人豐城公遥昌使崔慶遠出應之慶遠問師故【左傳齊桓公以諸侯之師伐楚楚子使與師言曰不虞君之涉吾地也何故】魏主曰固當有故卿欲我斥言之乎【斥指也直言以指人之罪過無所回避謂之斥】欲我含垢依違乎慶遠曰未承來命無所含垢【左傳曰國君含垢杜預注曰含垢忍垢恥】魏主曰齊主何故廢立慶遠曰廢昏立明古今非一未審何疑魏主曰武帝子孫今皆安在慶遠曰七王同惡已伏管蔡之誅【子隆子懋子敬子真子倫并鬱林海陵為七王】其餘二十餘王或内列清要或外典方牧魏主曰卿主若不忘忠義何以不立近親如周公之輔成王而自取之乎慶遠曰成王有亞聖之德故周公得而相之【相息亮翻】今近親皆非成王之比故不可立且霍光亦捨武帝近親而立宣帝唯其賢也魏主曰霍光何以不自立慶遠曰非其類也主上正可比宣帝安得比霍光若爾武王伐紂不立微子而輔之亦為苟貪天下乎【史言崔慶遠之機辨】魏主大笑曰朕來問罪如卿之言便可釋然慶遠曰見可而進知難而退【左傳載晉大夫隨武子之言】聖人之師也魏主曰卿欲吾和親為不欲乎慶遠曰和親則二國交歡生民蒙福否則二國交惡生民塗炭和親與否裁自聖衷魏主賜慶遠酒殽衣服而遣之戊申魏主循淮而東【過夀陽不攻引兵東下】民皆安堵租運屬路【屬之欲翻此謂淮北之民耳】丙辰至鍾離【自夀陽至鍾離三百三十餘里】上遣左衛將軍崔慧景寧朔將軍裴叔業救鍾離劉昶王肅衆號二十萬塹柵三重【塹七艷翻重直龍翻】并力攻義陽城中負楯而立【攻城甚急矢石交至故負楯而立以自蔽楯食尹翻】王廣之引兵救義陽去城百餘里畏魏彊不敢進城中益急黃門侍郎蕭衍請先進廣之分麾下精兵配之衍間道夜發【間古莧翻】與太子右率蕭誄等【率所律翻右率太子右衛率也誄魯水翻】徑上賢首山【水經注溮水南出大潰山北逕賢首山西又北出東南屈逕義陽縣城南上時掌翻】去魏軍數里魏人出不意未測多少不敢逼【少詩沼翻】黎明城中望見援軍至蕭誕遣長史王伯瑜出攻魏柵因風縱火衍等衆軍自外擊之魏不能支解圍去己未誕等追擊破之誄諶之弟也先是上以義陽危急【先悉薦翻】詔都督青冀二州諸軍事張冲出軍攻魏以分其兵埶冲遣軍主桑係祖攻魏建陵驛馬厚丘三城又遣軍主杜僧護攻魏虎阬馮時即丘三城皆拔之青冀二州刺史王洪範遣軍主崔延襲魏紀城據之【宋泰始初青冀二州入於魏乃置青冀二州刺史治朐山杜佑曰宋明帝立青冀二州寄治贑榆齊青州治朐山冀州理漣口今臨淮郡漣水縣魏收志郯郡有建陵縣漢古縣也宋白曰厚丘故城在海州沭陽縣北四十五里又東彭城郡龍沮縣有即丘城即丘亦漢縣本屬琅邪郡賢曰即丘即左傳之祝丘故城在今沂州臨沂縣東南紀城春秋紀鄣之故城也杜預曰東海贑榆縣東北有故紀城】魏主欲南臨江水辛酉鍾離司徒長樂元懿公馮誕病不能從【樂音洛從才用翻】魏主與之泣訣行五十里聞誕卒時崔慧景等軍去魏主營不過百里魏主輕將數千人夜還鍾離【將即亮翻】拊尸而哭達旦聲淚不絶壬戌敕諸軍罷臨江之行葬誕依晉齊獻王故事【齊獻王攸葬事見八十一卷晉武帝太康四年】誕與帝同年幼同硯席尚帝妹樂安長公主【長知兩翻】雖無學術而質性淳篤故特有寵丁卯魏主遣使臨江數上罪惡【使疏吏翻數所具翻】魏久攻鍾離不克士卒多死三月戊寅魏主如邵陽築城於洲上【邵陽洲在鍾離城北淮水中】柵斷水路夾築二城【既築城於洲上又於淮水南北兩岸夾築二城樹柵水中以斷援兵之路斷丁管翻下先斷邀斷欲斷同】蕭坦之遣軍主裴叔業攻二城拔之魏主欲築城置戍於淮南以撫新附之民賜相州刺史高閭璽書具論其狀【相息亮翻璽斯氏翻】閭上表以為兵法十則圍之五則攻之【孫子兵法有是言】曏者國家止為受降之計【謂欲受曹虎降也降戶江翻下同】發兵不多東西遼闊難以成功今又欲置戍淮南招撫新附昔世祖以回山倒海之威步騎數十萬南臨瓜步諸郡盡降而盱眙小城攻之不克【事見一百二十五卷宋文帝元嘉二十七年騎奇寄翻盱眙音吁怡】班師之日兵不戍一城土不闢一㕓【說文曰㕓一畝半一家之居地】夫豈無人以為大鎮未平【宋時淮上以夀陽廣陵為大鎮】不可守小故也夫壅水者先塞其原【塞悉則翻】伐木者先斷其本本原尚在而攻其末流終無益也夀陽盱眙淮陰淮南之本原也【夀陽盱眙淮陰皆淮津之要地即皆以重兵守之故云本原】三鎮不克其一而留守孤城其不能自全明矣敵之大鎮逼其外長淮隔其内少置兵則不足以自固【少詩沼翻】多置兵則糧運難通大軍既還士心孤怯夏水盛漲救援甚難以新擊舊以勞禦逸【久於屯戍魏師已老齊以生兵攻之是之謂以新擊舊魏以孤軍守孤城勞於備禦齊師迭出而攻之士有餘力是之謂以勞禦逸】若果如此必為敵擒雖忠勇奮發終何益哉【言將士效死弗去而城破身没雖忠勇奮而無益於國事】且安土戀本人之常情昔彭城之没既克大鎮城戍已定而不服思叛者猶踰數萬【宋明帝泰始二年魏得彭城至高帝建元之初淮北之民猶不樂屬魏思歸江南遂有五固之役】角城蕞爾處在淮北【蕞徂外翻小貌處昌呂翻】去淮陽十八里五固之役攻圍歷時卒不能克【事見一百二十五卷高帝建元三年卒子恤翻】以今凖昔事兼數倍天時尚熱【尚當作向】雨水方降願陛下踵世祖之成規旋轅反斾經營洛邑蓄力觀釁【釁許覲翻】布德行化中國既和遠人自服矣尚書令陸叡上表以為長江浩蕩彼之巨防又南土昏霧暑氣鬱蒸師人經夏必多疾病而遷鼎草創【武王遷九鼎於洛邑故引以為言】庶事甫爾臺省無論政之館府寺靡聽治之所【治直吏翻】百僚居止事等行路沈雨炎陽自成癘疫【沈與霃同持林翻說文久陰曰霃炎陽炎日也】且兵徭並舉聖王所難今介胄之士外攻寇讐羸弱之夫内勤土木運給之費日損千金驅罷弊之兵討堅城之虜將何以取勝乎【羸倫為翻罷讀與疲同】陛下去冬之舉正欲曜武江漢耳今自春幾夏【幾居希翻近也】理宜釋甲願早還洛邑使根本深固聖懷無内顧之憂兆民休斤板之役【斤謂斧斤之役板謂板築之役】然後命將出師【將即亮翻下同】何憂不服魏主納其言崔慧景以魏人城邵陽患之張欣泰曰彼有去志所以築城者外自誇大懼我躡其後耳今若說之以兩願罷兵【說輸芮翻】彼無不聽矣慧景從之使欣泰詣城下語魏人【語牛倨翻】魏主乃還濟淮餘五將未濟齊兵據渚邀斷津路【斷下管翻下同】魏主募能破中渚兵者以為直閤將軍軍主代人奚康生應募【據北史康生本姓達奚魏孝文改複姓於是姓奚】縳筏積柴因風縱火燒齊船艦【艦戶黯翻】依煙焰進飛刀亂斫中渚兵遂潰魏主假康生直閤將軍魏主使前將軍楊播將步卒三千騎五百為殿【將步即亮翻騎奇寄翻下同殿都輦翻斷後曰殿】時春水方長齊兵大至戰艦塞川播結陳於南岸以禦之【長知兩翻艦戶黯翻塞悉則翻陳讀曰陣下為陳同】諸軍盡濟齊兵四集圍播播為圓陳以禦之身自戰所殺甚衆相拒再宿軍中食盡圍兵愈急魏主在北岸望之以水盛不能救既而水稍減播引精兵三百歷齊艦大呼曰【呼火故翻】我今欲渡能戰者來遂擁衆而濟播椿之兄也【楊椿見一百三十七卷武帝永明八年】魏軍既退邵陽洲上餘兵萬人求輸馬五百匹假道以歸崔慧景欲斷路攻之張欣泰曰歸師勿遏古人畏之【兵法歸師勿遏窮寇勿追】兵在死地不可輕也今勝之不足為武不勝徒喪前功【喪息浪翻】不如許之慧景從之蕭坦之還言於上曰邵陽洲有死賊萬人慧景欣泰縱而不取由是皆不加賞甲申解嚴【魏師已退故解嚴】初上聞魏主欲飲馬於江懼敕廣陵太守行南兖州事蕭穎胄移居民入城民驚恐欲席卷南渡【卷讀曰捲】頴胄以魏寇尚遠不即施行魏兵竟不至頴胄太祖之從子也【蕭穎胄太祖從弟赤斧之子從才用翻】上遣尚書左僕射沈文季助豐城公遥昌守夀陽【是年春正月遣沈文季督豫州諸軍豫州治夀陽】文季入城止游兵不聽出洞開城門嚴加守備魏兵尋退魏之入寇也盧昶等猶在建康【海陵王即位魏遣昶來聘昶至建康而帝已立】齊人恨之飼以蒸豆【飼祥吏翻馬牛待之】昶怖懼食之【怖普布翻】淚汗交横謁者張思寧辭氣不屈死於館下及還魏主讓昶曰人誰不死何至自同牛馬屈身辱國縱不遠慙蘇武【蘇武使匈奴十九年不屈節】獨不近愧思寧乎乃黜為民 戊子魏太師京兆武公馮熙卒於平城 乙未魏主如下邳夏四月庚子如彭城辛丑為馮熙舉哀【為於偽翻】太傅録尚書事平陽公丕不樂南遷【樂音洛】與陸叡表請魏主還臨熙葬【丕叡時留守平城】帝曰開闢以來安有天子遠奔舅喪者乎今經始洛邑【經度之也始初也詩云經始靈臺】豈宜妄相誘引陷君不義令僕以下可付法官貶之【此平城留臺令僕也法官謂御史誘音酉】仍詔迎熙及博陵長公主之柩【長知兩翻柩巨救翻】南葬洛陽禮如晉安平獻王故事【晉安平王孚葬見七十九卷武帝泰始八年魏之葬熙其禮又加於誕】 魏主之在鍾離仇池鎮都大將梁州刺史拓跋英請以州兵會劉藻擊漢中【去年十一月魏遣劉藻向南鄭魏梁州刺史治仇池齊梁州刺史治南鄭將即亮翻下同】魏主許之梁州刺史蕭懿遣部將尹紹祖梁季羣等將兵二萬據險立五柵以拒之【據蕭子顯齊書時據角弩谷白馬沮水立五柵】英曰彼帥賤莫相統壹【帥所類翻】我選精卒并攻一營彼必不相救若克一營四營皆走矣乃引兵急攻一營拔之四營俱潰生擒梁季羣斬三千餘級俘七百餘人乘勝長驅進逼南鄭懿又遣其將姜修擊英英掩擊盡獲之將還懿别軍繼至將士皆已疲不意其至大懼欲走英故緩轡徐行神色自若登高望敵東西指麾狀若處分【處昌呂翻】然後整列而前懿軍疑有伏兵遷延引退英追擊破之遂圍南鄭禁將士毋得侵暴遠近悦附爭供租運懿嬰城自守軍主范絜先將三千餘人在外還救南鄭英掩擊盡獲之圍城數十日城中恟懼【將即亮翻恟許記翻】録事參軍新野庾域封題空倉數十指示將士曰此中粟皆滿足支二年但努力固守衆心乃安會魏主召英還使老弱先行自將精兵為後拒【殿軍後以拒追兵曰後拒】遣使與懿告别【使疏吏翻】懿以為詐英去一日猶不開門二日乃遣將追之英與士卒下馬交戰懿兵不敢逼行四日四夜懿兵乃返英入斜谷會天大雨士卒截竹貯米執炬火於馬上炊之【貯丁呂翻】先是懿遣人誘說仇池諸氐使起兵斷英運道及歸路英勒兵奮擊且戰且前矢中英頰卒全軍還仇池【英乘勝深入後無繼援雖僅獲全軍而返亦已危矣先悉薦翻說輸芮翻斷丁管翻中竹仲翻卒子恤翻】討叛氐平之英楨之子【南安王楨見一百三十八卷武帝永明十一年】懿衍之兄也英之攻南鄭也魏主詔雍涇岐三州發兵六千人戍南鄭【魏雍州治長安領京兆馮翊扶風咸陽北地等郡太和中置涇州治臨涇城領安定隴東新平平凉平原等郡十一年置岐州治雍城鎮領平秦武功武都郡雍於用翻】俟克城則遣之侍中兼左僕射李冲表諫曰秦州險阨地接羌夷自西師出後餉援連續加氐胡叛逆所在奔命運糧擐甲迄兹未已今復豫差戍卒【復扶又翻差初皆翻下差遣同】懸擬山外【漢中之地在關中南山之南故曰山外】雖加優復【復方目翻】恐猶驚駭脱終攻不克徒動民情連胡結夷事或難測輒依旨密下刺史待軍克鄭城【下戶嫁翻鄭城謂南鄭城】然後差遣如臣愚見猶謂未足何者西道險阨單徑千里【謂褒斜之道也】今欲深戍絶界之外孤據羣賊之中敵攻不可猝援食盡不可運糧古人有言雖鞭之長不及馬腹【左傳晉伯宗之言】南鄭於國實為馬腹也且魏境所掩九州過八【此指禹貢九州為言】民人所臣十分而九所未民者唯漠北之與江外耳【漠北謂柔然江外謂齊言唯此二國未為魏民】羈之在近【謂以繩羈係其君而致之在近言不遠也】豈汲汲於今日也宜待疆宇既廣糧食既足然後置邦樹將【樹立也將帥也將即亮翻】為吞併之舉今夀陽鍾離密邇未拔赭城新野跬步弗降【赭城即赭陽城也降戶江翻】東道既未可以近力守西藩寧可以遠兵固【李冲盖謂淮漢之地為東道謂南鄭為西藩】若果欲置者臣恐終以資敵也又建都土中【洛陽為土中】地接寇壤方須大收死士平蕩江會【建康為江南都會之地故曰江會】若輕遣單寡弃令陷没恐後舉之日衆以留守致懼求其死効未易可獲【易以豉翻】推此而論不戍為上魏主從之 癸丑魏主如小沛己未如瑕丘庚申如魯城【魏收地形志魯郡魯縣之魯城】親祠孔子辛酉拜孔氏四人顔氏二人官仍選諸孔宗子一人封崇聖侯奉孔子祀命兖州修孔子墓【大宗之子為宗子孔子墓亦在魯縣】更建碑銘戊辰魏主如碻磝命謁者僕射成淹具舟楫欲自泗入河泝流還洛淹諫以為河流悍猛非萬乘所宜乘【泝蘇故翻悍侯盱翻又下罕翻萬乘繩證翻】帝曰我以平城無漕運之路故京邑民貧今遷都洛陽欲通四方之運而民猶憚河流之險故朕有此行所以開百姓之心也 魏城陽王鸞等攻赭陽諸將不相統壹圍守百餘日諸將欲案甲不戰以疲之李佐獨晝夜攻擊士卒死者甚衆帝遣太子右衛率垣歷生救之諸將以衆寡不敵欲退佐獨帥騎二千逆戰而敗【將即亮翻帥所律翻帥讀曰率騎奇寄翻】盧淵等引去歷生追擊大破之歷生榮祖之從弟也【垣榮祖著名於宋泰始之間從才用翻下同】南陽太守房伯玉等又敗薛真度於沙堨【敗蒲邁翻 考異曰齊書魏虜傳真度敗在建武元年下魏帝紀城陽王鸞以敗軍獲罪在太和十九年五月今從之】鸞等見魏主於瑕丘【見賢遍翻】魏主責之曰卿等沮辱威靈罪當大辟【沮在呂翻辟毗亦翻】朕以新遷洛邑特從寛典五月己巳降封鸞為定襄縣王削戶五百盧淵李佐韋珍皆削官爵為民佐仍徙瀛州【太和十一年分定州河間高陽冀州章武浮陽置瀛州治趙都軍城】以薛真度與其從兄安都有開徐方之功【謂以彭城降魏也從才用翻】聽存其爵及荆州刺史餘皆削奪曰進足明功退足彰罪矣魏廣川剛王諧卒諧畧之子也【魏廣川王畧見一百三十五卷高帝之建】<br />
<br />
  【元二年謚法追補前過曰剛】魏主曰古者大臣之喪有三臨之禮【賈山曰古者賢君之於臣也死則往弔哭之臨其小歛大歛已棺除而為之服錫衰麻絰而三臨其喪】魏晉以來王公之喪哭於東堂自今諸王之喪期親三臨【期親期喪之親期讀曰朞】大功再臨小功緦麻一臨【大功九月服小功五月服緦麻三月服】罷東堂之哭廣川王於朕大功也【廣川王畧顯祖之弟諧於魏主從兄弟也其服大功】將大歛【歛力贍翻】素服深衣往哭之 甲戌魏主如滑臺丙子舍于石濟庚申太子出迎於平桃城【魏收志濟陰郡離狐縣有桃城水經注曰榮陽縣有䝞亭俗謂之平城】趙郡王幹在洛陽貪淫不法御史中尉李彪私戒之【魏置御史中尉以糾察百官猶御史中丞也】且曰殿下不悛不敢不以聞幹悠然不以為意【悠遠也悠然夷曠自得之意悛七緣翻】彪表彈之【彈徒丹翻】魏主詔幹與北海王祥俱從太子詣行在既至見祥而不見幹陰使左右察其意色知無憂悔【言既無憂色又無悔過之意】乃親數其罪杖之一百免官還第【數所具翻】癸未魏主還洛陽告于太廟甲申減冗官之禄以助軍國之用乙酉行飲至之禮【左傳臧僖伯曰三年而治兵入而振旅歸而飲至以數軍實又曰反行飲至舍爵策勲焉飲至者告至於廟而飲酒也】班賞有差【班南伐之賞也】 甲午魏太子冠於廟【記冠義曰古者重冠冠故行之於廟行之於廟者所以自卑而尊先祖也鄭樵曰曹魏冠太子再加宋一加余謂魏孝文好古其必用三加之禮冠於廟禮也曹魏以來不復在廟冠古玩翻】 魏主欲變北俗引見羣臣【見賢遍翻】謂曰卿等欲朕遠追商周為欲不及漢晉邪咸陽王禧對曰羣臣願陛下度越前王耳帝曰然則當變風易俗當因循守故邪對曰願聖政日新帝曰為止於一身為欲傳之子孫邪對曰願傳之百世帝曰然則必當改作卿等不得違也對曰上令下從其誰敢違帝曰夫名不正言不順則禮樂不可興【用論語孔子之言】今欲斷諸北語一從正音【斷音短正音華言也】其年三十已上習性已久容不可猝革三十已下見在朝廷之人【見賢遍翻】語音不聽仍舊若有故為【謂故意為北語不肯從華言者】當加降黜各宜深戒王公卿士以為然不【不讀曰否】對曰實如聖旨帝曰朕嘗與李冲論此冲曰四方之語竟知誰是【謂四方之人言語不同不知當以誰為是】帝者言之即為正矣冲之此言其罪當死因顧冲曰卿負社稷當令御史牽下冲免冠頓首謝又責留守之官曰【守手又翻】昨望見婦女猶服夾領小袖卿等何為不遵前詔皆謝罪帝曰朕言非是卿等當廷爭【爭讀曰諍】如何入則順旨退則不從乎六月己亥下詔不得為北俗之語於朝廷違者免所居官癸卯魏主使太子如平城赴太師熙之喪 癸丑魏<br />
<br />
  詔求遺書袐閤所無【漢時書府在外則有太常太史博士掌之内則有延閣廣内石渠之藏後漢則藏之東觀晉有中外三閣經書陸機謝表云身登三閣謂為祕書郎掌中外三閣祕書也此祕閣之名所由始】有益時用者加以優賞 魏有司奏廣川王妃葬於代都未審以新尊從舊卑以舊卑就新尊【夫尊婦卑廣川王諧新卒故曰新尊其妃先卒故曰舊卑】魏主曰代人遷洛者宜悉葬邙山【邙山在洛城北邙謨郎翻】其先有夫死於代者聽妻還葬夫死於洛者不得還代就妻其餘州之人自聽從便丙辰詔遷洛之民死葬河南不得還北於是代人遷洛者悉為河南洛陽人 戊午魏改用長尺大斗其法依漢志為之【漢律歷志以子穀秬黍中者一黍之廣度之九十分黄鍾之長一為一分十分為寸十寸為尺又以子穀秬黍中者千有二百實其龠十龠為合十合為升十升為斗】 上之廢鬱林王也【見上卷上年】許蕭諶以揚州既而除領軍將軍南徐州刺史諶恚曰見炊飯推以與人【諶氏壬翻恚於避翻見賢遍翻推吐雷翻】諶恃功頗干預朝政【朝直遥翻】所欲選用輒命尚書使為申論【為于偽翻】上聞而忌之以蕭誕蕭誄方將兵拒魏【誄魯水翻將即亮翻】隱忍不壬戍上遊華林園與諶及尚書令王晏等數人宴盡歡坐罷留諶晩出至華林園仗身執還省【仗身執仗之衛士天子禁衛有齋内仗身見齊書蕭諶傳又按杜佑通典曰唐制鎮戍之官給仗身其人數眡鎮戍之上中下為差京官五品已上亦有仗身職員】上遣左右莫智明數諶曰隆昌之際非卿無有今日今一門二州兄弟三封【諶為南徐州誕為司州所謂二州也諶封衡陽郡公誄封西昌侯誕封安復侯所謂三封也數所具翻】朝廷相報止可極此卿恒懷怨望【恒戶登翻】乃云炊飯已熟合甑與人邪今賜卿死遂殺之并其弟誄以黃門郎蕭衍為司州别駕往執誕殺之諶好術數吳興沈文猷常語之曰君相不減高帝【好呼到翻語牛倨翻相息亮翻相貌也】諶死文猷亦伏誅諶死之日上又殺西陽王子明南海王子罕邵陵王子貞【三王皆武帝子也】 乙丑以右衛將軍蕭坦之為領軍將軍 魏高閭上言鄴城密皇后廟頹圯請更葺治若謂已配饗太廟即宜罷毁詔罷之【密皇后世祖母杜皇后也后鄴人神䴥三年立廟於鄴高閭為相州刺史相州治鄴故上言之圯都鄙翻毁也治直之翻】 魏拓跋英之寇漢中也沮水氐楊馥之為齊擊武興氐楊集始破之【按漢志武都郡沮縣有東狼谷沮水所出也水在廣業郡界唐鳳州同谷縣魏之廣業郡地也氐居沮水上因以為種落之名沮千余翻為于偽翻】秋七月辛卯以馥之為北秦州刺史【蕭子顯曰永明郡國志秦州寄治漢中南鄭不曰南北元嘉計偕亦曰秦州而荆州刺史嘗督二秦梁是則志所載秦州為南秦氐為北秦然是時秦州所領諸郡皆僑郡與荒郡也】仇池公 八月乙巳魏選武勇之士十五萬人為羽林虎賁以充宿衛【為後虎賁羽林作亂殺張彛父子張本賁音奔】 魏金墉宫成立國子太學四門小學於洛陽【四門學始此】 魏高祖遊華林園觀故景陽山【華林園及景陽山皆魏明帝所築】黃門侍郎郭祚曰山水者仁智之所樂【論語孔子曰仁者樂山智者樂水故郭祚引以為言樂魚教翻】宜復修之【復扶又翻】帝曰魏明帝以奢失之於前朕豈可襲之於後乎帝好讀書手不釋卷在輿據鞍不忘講道善屬文【好呼到翻下好賢同屬之欲翻】多於馬上口占既成不更一字【更工衡翻】自太和十年以後詔策皆自為之好賢樂善情如饑渴所與遊接常寄以布素之意【樂音洛言寄以布衣雅素相與之意】如李冲李彪高閭王肅郭祚宋弁劉芳崔光邢巒之徒皆以文雅見親貴顯用事制禮作樂鬱然可觀有太平之風焉【史言魏高祖能以文治】治書侍御史薛聰辯之曾孫也【薛辯見一百二十四卷宋文帝元嘉二十一年治直之翻】彈劾不避彊禦【彈徒丹翻劾戶槩翻又戶得翻】帝或欲寛貸者聰輒爭之帝每曰朕見薛聰不能不憚何况諸人也自是貴戚歛手累遷直閤將軍兼給事黃門侍郎散騎常侍【散悉亶翻騎奇寄翻】帝外以德器遇之内以心膂為寄親衛禁兵悉聰管領故終太和之世恒帶直閤將軍羣臣罷朝之後聰恒陪侍帷幄言兼晝夜時政得失動輒匡諫事多聽允而重厚沈密【恒戶登翻朝直遥翻沈持林翻】外莫窺其際帝欲進以名位輒苦讓不受帝亦雅相體悉謂之曰卿天爵自高固非人爵所能榮也【孟子曰公卿大夫此人爵也仁義忠信此天爵也】 九月庚午魏六宫文武悉遷於洛陽【六宫后妃夫人嬪御也文武内外文武百官也】 丙戌魏主如鄴屢至相州刺史高閭之館【館謂刺史官舍相息亮翻】美其治效【治直吏翻】賞賜甚厚閭數請本州【數所角翻】詔曰閭以懸車之年方求衣錦【漢薛廣漢致仕懸其安車以示子孫古人有言富貴不歸故鄉如衣錦夜行衣於既翻】知進忘退有塵謙德可降號平北將軍朝之老成宜遂情願徙授幽州刺史【高閭漁陽雍奴人幽州統内也朝直遥翻】令存勸兩修恩法並舉【從所請以勸善示恩降號以存法】以高陽王雍為相州刺史戒之曰作牧亦易亦難其身正不令而行所以易其身不正雖令不從所以難【用孔子之言而難易之論易以豉翻】 己丑徙南平王寶攸為邵陵王蜀郡王子文為西陽王廣漢王子峻為衡陽王臨海王昭秀為巴郡王永嘉王昭粲為桂陽王【寶攸皇子餘皆高武子孫】 乙未魏主自鄴還【還洛陽】冬十月丙辰至洛陽 壬戌魏詔諸州精品屬官考其得失為三等以聞又詔徐兖光南青荆洛六州嚴纂戎備應須赴集【魏徐州領彭城南陽平沛蘭陵北濟陰等郡兖州領泰山魯高平任城東平東陽平等郡光州治掖城皇興四年分青州置領東萊長廣東牟等郡南青州即東徐州魏主更名領東安東莞郡魏先置荆州於上洛領上洛上庸魏興等郡太和十一年改為洛州置荆州於穰城領南陽順陽新野襄城等郡詔纂戎備將復南伐也】 十一月丁卯詔罷世宗東田毁興光樓【東田見武帝紀興光樓盖亦文惠太子所建】己卯納太子妃禇氏大赦妃澄之女也【禇澄見一百三十三卷宋蒼梧王元徽二年】 庚午魏主如委粟山定圜丘己卯帝引諸儒議圜丘禮祕書令李彪建言魯人將有事於上帝必先有事於泮宫【記禮器之言鄭玄注曰泮宫郊學也】請前一日告廟從之甲申魏主祀圜丘大赦 十二月乙未朔魏主見羣臣於光極堂宣下品令為大選之始【下遐嫁翻品令九品之令也大選者謂將大選羣臣也】光禄勲于烈子登引例求遷官烈上表曰方今聖明之朝理應廉讓而臣子登引人求進【引人謂引他人之例也朝直遥翻】是臣素無教訓乞行黜落【黜落謂黜官落職也】魏主曰此乃有識之言不謂烈能辦此乃引見登謂曰朕將流化天下以卿父有謙遜之美直士之風故進卿為太子翊軍校尉又加烈散騎常侍封聊城縣子【校戶教翻散悉亶翻騎奇寄翻】魏主謂羣臣曰國家從來有一事可歎臣下莫肯公言得失是也夫人君患不能納諫人臣患不能盡忠自今朕舉一人如有不可卿等直言其失若有才能而朕所不識卿等亦當舉之如是得人者有賞不言者有罪卿等當知之【以魏孝文之求諫求才如此而一時之臣猶未能稱上意豈非朝廷之議帝務騁辭氣以加之故有有懷而不敢盡者】 丁酉詔修晉帝諸陵增置守衛【此晉帝諸陵謂在江南者】 甲子魏主引見羣臣於光極堂頒賜冠服【賜冠服以易胡服】 先是魏人未嘗用錢【先悉薦翻】魏主始命鑄太和五銖是歲鼔鑄粗備【粗坐五翻】詔公私用之 魏以光城蠻帥田益光為南司州刺史【帥所類翻】所統守宰聽其銓置後更於新蔡立東豫州以益光為刺史【據北史益光當作益宗魏以益宗既渡淮北不可仍為司州乃於新蔡立東豫州又按五代志及水經注新蔡當作新息】 氐王楊炅卒【炅古迥翻又古惠翻】<br />
<br />
  三年春正月丁卯以楊炅子崇祖為沙州刺史封陰平王 【考異曰齊本紀作丁酉按長歷是月乙丑朔無丁酉下有己巳當作丁卯】 魏主下詔以為北人謂上為拓后為跋魏之先出於黃帝以土德王【王于况翻】故為拓跋氏夫土者黃中之色萬物之元也宜改姓元氏諸功臣舊族自代來者姓或重複皆改之【重直龍翻】於是始改拔拔氏為長孫氏達奚氏為奚氏乙旃氏為叔孫氏丘穆陵氏為穆氏步六孤氏為陸氏賀賴氏為賀氏獨孤氏為劉氏賀樓氏為樓氏勿忸于氏為于氏尉遲氏為尉氏其餘所改不可勝紀【如長孫嵩奚斤叔孫建穆崇于栗磾之類史皆因其後改姓從簡便而書之非其舊也其餘北人諸姓改從後姓注已畧見於前盖其所改後姓有與華人舊姓相犯者也忸女九翻又女六翻 考異曰魏初功臣姓皆複重奇僻孝文太和中變胡俗始改之魏收作魏書已盡用新姓不用舊姓宋書索虜傳南齊書魏虜傳所稱者盖其舊姓名耳今並從魏書以就簡易】魏主雅重門族以范陽盧敏清河崔宗伯滎陽鄭羲太原王瓊四姓衣冠所推咸納其女以充後宫隴西李冲以才識見任當朝貴重所結姻㜕莫非清望【朝直遥翻㜕音連史記南越傳呂嘉宗室兄弟及蒼梧秦王有連漢書音義曰連親婚也史記索隱曰有連者皆親姻也後人因以姻連之連其旁加女遂為㜕字】帝亦以其女為夫人詔黃門郎司徒左長史宋弁定諸州士族多所升降又詔以代人先無姓族雖功賢之胤無異寒賤故宦達者位極公卿其功衰之親仍居猥任【功衰自小功大功以上至齊衰也猥卑下也衰倉囘翻猥烏賄翻鄙也】其穆陸賀劉樓于稽尉八姓【嵇恐當作奚今按魏書官氏志自有嵇姓嵇敬之嵇是也尉紆勿翻】自太祖已降勳著當世位盡王公灼然可知者且下司州吏部勿充猥官一同四姓【四姓盧崔鄭王也下戶嫁翻】自此以外應班士流者尋續别敕其舊為部落大人而皇始以來三世官在給事已上及品登王公者為姓若本非大人而皇始以來三世官在尚書已上及品登王公者亦為姓其大人之後而官不顯者為族若本非大人而官顯者亦為族凡此姓族皆應審覈勿容偽冒【覈戶革翻】令司空穆亮尚書陸琇等詳定務令平允琇馥之子也【魏孝文受内禪陸馛傳之故其子皆通顯琇音秀馛蒲撥翻】魏舊制王國舍人皆應娶八族及清修之門【王國舍人舍謂諸王妃嬪之舍其人即妃嬪也八族即前自代來八姓】咸陽王禧娶隸戶為室【隸戶謂没入為奴隸之戶】帝深責之因下詔為六弟聘室【為于偽翻】前者所納可為妾媵【媵以證翻】咸陽王禧可聘故潁川太守隴西李輔女河南王幹可聘故中散大夫代郡穆明樂女【太和十八年河南王幹已徙封趙郡王史盖以舊封書之散悉亶翻】廣陵王羽可聘驃騎諮議參軍滎陽鄭平城女【驃匹妙翻騎奇寄翻】潁川王雍可聘故中書博士范陽盧神寶女【潁川王雍亦以太和十八年徙封高陽史以舊封書之】始平王勰可聘廷尉卿隴西李冲女【勰音協】北海王祥可聘吏部郎中滎陽鄭懿女【魏定氏族固亦未能盡允清議至令詔諸王改納室則大悖於人倫夫妻者齊也一與之齊終身不改富而易妻人士猶或羞之况天子之弟乎此詔一出天下何觀】懿羲之子也【宋泰始之初鄭羲從拓拔石平汝潁】時趙郡諸李人物尤多各盛家風故世之言高華者以五姓為首【盧崔鄭王并李為五姓趙郡諸李北人謂之趙李李靈李順李孝伯羣從子姪皆趙李也】衆議以薛氏為河東茂族帝曰薛氏蜀也豈可入郡姓直閤薛宗起執戟在殿下出次對曰臣之先人漢末仕蜀二世復歸河東今六世相襲非蜀人也伏以陛下黃帝之胤受封北土豈可亦謂之胡邪今不預郡姓何以生為乃碎戟於地帝徐曰然則朕甲卿乙乎乃入郡姓仍曰卿非宗起乃起宗也【郡姓者郡之大姓著姓也今百氏郡望盖始於此考異曰北史薛聰傳為羽林監帝曾與朝臣論海内姓地人物戲謂聰曰人謂卿諸薛是蜀人定是蜀人不聰對曰臣遠祖廣德世事漢朝時人呼為漢臣九世祖永隨劉備入蜀時人呼為蜀臣今事陛下是虜非蜀也帝撫掌笑曰卿可自明非蜀何乃遂復苦朕聰因投戟而出帝曰薛監醉耳其見知如此今從元行冲後魏國典】帝與羣臣論選調【選須絹翻調徒弔翻】曰近世高卑出身各有常分【分扶問翻】此果如何李冲對曰未審上古以來張官列位為膏粱子弟乎為致治乎【為于偽翻治直吏翻】帝曰欲為治耳冲曰然則陛下何為專取門品不拔才能乎帝曰苟有過人之才不患不知然君子之門借使無當世之用要自德行純篤朕故用之冲曰傳說呂望豈可以門地得之【謂傳說起於版築呂望起於屠釣也行下孟翻說讀曰悦】帝曰非常之人曠世乃有一二耳祕書令李彪曰陛下若專取門地不審魯之三卿孰若四科【魯三卿季孫孟孫叔孫氏也孔門四科德行言語政事文學也】著作佐郎韓顯宗曰陛下豈可以貴襲貴以賤襲賤帝曰必有高明卓然出類拔萃者朕亦不拘此制頃之劉昶入朝【劉昶自彭城入朝朝直遥翻】帝謂昶曰或言唯能是寄不必拘門朕以為不爾何者清濁同流混齊一等君子小人名器無别【别彼列翻】此殊為不可我今八族以上士人品第有九九品之外小人之官復有七等【後之流内銓流外銓盖分於此復扶又翻】若有其人可起家為三公正恐賢才難得不可止為一人渾我典制也【為于偽翻渾翻本翻】<br />
<br />
  臣光曰選舉之法先門地而後賢才【先後皆去聲】此魏晉之深弊而歷代相因莫之能改也夫君子小人不在於世禄與側微【書序虞舜側微孔頴達疏曰不在朝廷謂之側其人貧賤謂之微】以今日視之愚智所同知也當是之時雖魏孝文之賢猶不免斯弊故夫明辯是非而不惑於世俗者誠鮮矣【鮮息淺翻】<br />
<br />
  壬辰魏徙始平王勰為彭城王復定襄縣王鸞為城陽王【鸞以赭陽之敗降封今復之勰音協】 二月壬寅魏詔羣臣自非金革聽終三年喪 丙午魏詔畿内七十已上暮春赴京師行養老之禮三月丙寅宴羣臣及國老庶老於華林園詔國老黃耇已上假中散大夫郡守耆年已上假給事中縣令庶老直假郡縣各賜鳩杖衣裳【熊氏曰國老謂卿大夫致仕者庶老謂士也皇氏曰庶老兼庶人在官者毛萇曰黃黃髪也耇老艾也陸德明曰耆至也言至老境也漢儀仲秋之月縣道皆案戶比民年始七十者授以玉杖餔之糜粥八十者禮有加賜玉杖長九尺端以鳩鳥為飾鳩者不噎之鳥也欲老人不噎耳耇音苟郡縣之下當有逸字】 丁丑魏詔諸州中正各舉其鄉之民望年五十以上守素衡門者授以令長【毛萇曰衡門横木為門言淺陋也長知兩翻】 壬午詔乘輿有金銀飾校者皆剔除之【乘繩證翻校戶教翻校欄格也飾其校飾其欄格也又居效翻義與鉸同以金飾器謂之鉸】上志慕節儉太官嘗進裹蒸上曰我食此不盡可四破之餘充晩食【今之裏蒸以糖和糯米入香藥松子胡桃仁等物以竹籜裹而蒸之大纔二指許不勞四破也者】又嘗用皁莢以餘濼授左右曰此可更用【皁莢不極高大莢形如豬牙去垢膩洗沐多用之濼音郎狄翻更居孟翻再也】太官元日上夀有銀酒鎗上欲壞之【太平御覽云鎗即當字壞音怪下同】王晏等咸稱盛德衛尉蕭頴胄曰朝廷盛禮莫若三元【玉燭寶典曰正月為端月其一日為上日亦云三元謂歲之元月之元時之元也】此一器既是舊物不足為侈上不悦後預曲宴銀器滿席【内宴於宫中謂之曲宴】頴胄曰陛下前欲壞酒鎗恐宜移在此器上甚慙上躬親細務綱目亦密於是郡縣及六署九府常行職事莫不啟聞取决詔敇【按蕭子顯齊志六署者尚書左右僕射左右丞所通署除署功論封爵貶黜八議疑讞六案也九府者太常光禄勲衛尉廷尉大司農少府將作大匠太僕大鴻臚九卿府也】文武勲舊皆不歸選部【選須絹翻】親戚憑藉互相通進人君之務過繁密南康王侍郎潁川鍾嶸上書言古者明君揆才頒政量能授職三公坐而論道九卿作而成務【嶸乎萌翻古者三公論道六卿分職周官考工記坐而論道謂之三公作而行之謂之士大夫注云親受其職居其官也】天子唯恭已南面而已書奏上不懌謂大中大夫顧暠曰鍾嶸何人欲斷朕機務卿識之不【暠古老翻斷音短不讀曰否】對曰嶸雖位末名卑而所言或有可采且繁碎軄事各有司存今人主總而親之是人主愈勞而人臣愈逸所謂代庖人宰而為大匠斵也上不顧而言他【齊明帝以吏事權詐得國猜防羣下故親攬機務王莽之親御燈火其計慮亦如此耳為于偽翻】 夏四月甲辰魏廣州刺史薛法護來降【以蕭子顯齊書考之廣州不在太和十年分置三十八州之數魏牧地形志永安中置廣州治魯陽意此時廣州亦當置於魯陽也降戶江翻】魏寇司州櫟城戍主魏僧珉拒破之【櫟城即左傳吳伐楚入棘櫟麻之櫟杜預注曰汝陰新蔡縣東北有櫟亭】 五月丙戌魏營方澤於河陰又詔漢魏晉諸帝陵百步内禁樵蘇【此諸陵皆謂在河南者】丁亥魏主有事於方澤 秋七月魏廢皇后馮氏初文明太后欲其家貴重簡馮熙二女入掖庭其一早卒其一得幸於魏主未幾有疾還家為尼及太后殂帝立熙少女為皇后【幾居豈翻少詩沼翻】既而其姊疾愈帝思之復迎入宫拜左昭儀后寵浸衰昭儀自以年長且先入宫不率妾禮【復扶又翻長丁丈翻今知兩翻率循也】后頗愧恨昭儀因譛而廢之【為後昭儀為后及不終張本】后素有德操遂居瑤光寺為練行尼【練行謂修練戒行也瑶光寺在洛陽宫側行下孟翻】 魏主以久旱自癸未不食至於乙酉羣臣皆詣中書省請見帝在崇虚樓【武帝永明九年魏移道壇於桑乾之陰改曰崇虚寺此盖遷洛後建崇虛樓於禁中齊戒則居之見賢遍翻】遣舍人辭焉且問來故【舍人即中書舍人問其所以來請見之故】豫州刺史王肅對曰今四郊雨已霑洽獨京城微少細民未乏一餐而陛下輟膳三日臣下惶惶無復情地【少詩沼翻復扶又翻】帝使舍人應之曰朕不食數日猶無所感比來中外貴賤皆言四郊有雨【比毗至翻】朕疑其欲相寛免未必有實方將遣使視之【使疏吏翻】果如所言即當進膳如其不然朕何以生為當以身為萬民塞咎耳【塞悉則翻】是夕大雨 魏太子恂不好學體素肥大苦河南地熱常思北歸魏主賜之衣冠恂常私著胡服【好呼到翻著陟略翻】中庶子遼東高道悦數切諫恂惡之【數所角翻惡烏路翻】八月戊戌帝如嵩高恂與左右密謀召牧馬輕騎奔平城手刃道悦於禁中中領軍元儼勒門防遏【嚴勒門衛以防遏其變騎奇寄翻】入夜乃定詰旦尚書陸琇馳以啟帝【詰去吉翻琇音秀】帝大駭祕其事仍至汴口而還【汴口汴水與河通之口至此而後還以安人心還從宣翻】甲寅入宫引見恂數其罪親與咸陽王禧更代杖之百餘下【見賢遍翻數所具翻更工衡翻】扶曳出外囚於城西月餘乃能起 丁巳魏相州刺史南安惠王禎卒【相息亮翻謚法柔質愛民曰惠愛民好與曰惠】 九月戊辰魏主講武於小平津癸酉還宫 冬十月戊戌魏詔軍士自代來者皆以為羽林虎賁【賁音奔】司州民十二夫調一吏以供公私力役【此時魏以洛為司州調徒弔翻】 魏吐京胡反【魏世祖太平真君九年置吐京郡水經注曰吐京即漢西河郡吐軍縣夷夏俗音訛也】詔朔州刺史元彬行汾州事帥幷肆之衆以討之【大和十二年置汾州治蒲子縣西河吐京定陽北鄉正平五城中陽絳郡皆屬焉并州領太原上黨樂平鄉郡太平真君七年置肆州領新興秀容鴈門郡帥讀曰率下同】彬禎之子也彬遣統軍奚康生擊叛胡破之追至車突谷又破之【五代史志離石郡太和縣後周置烏突郡烏突縣盖因車突谷而名之也】俘雜畜以萬數【畜許救翻】詔以彬為汾州刺史胡去居等六百餘人保險不服彬請兵二萬以討之有司奏許之魏主大怒曰小寇何有發兵之理可隨宜討治【治直之翻】若不能克必須大兵者則先斬刺史然後發兵彬大懼督帥州兵身先將士【身先悉薦翻】討去居平之 魏主引見羣臣於清徽堂【見賢遍翻】議廢太子恂太子太傅穆亮少保李冲免冠頓首謝帝曰卿所謝者私也我所議者國也大義㓕親古人所貴【左傳以是語美石碏】今恂欲違父逃叛跨據恒朔【魏太祖天興中置司州治代都平城太和都洛改為恒州杜佑曰魏恒州在唐代郡安邊馬邑縣界朔朔州也宋白曰後魏都平城置司州及代尹及遷洛陽置司州於洛以平城為恒州隋雲中郡恒安鎮即其地後魏懷朔鎮孝文遷洛於定襄故城置朔州在唐朔州北三百八十里恒戶登翻下同】天下之惡孰大焉若不去之【去羌呂翻】乃社稷之憂也閏月丙寅廢恂為庶人考【異曰齊書魏虜傳云大馮有寵日夜讒恂魏書無之又魏帝紀在十二月丙寅按長歷魏閏十一月齊閏十二月今從齊歷】置於河陽無鼻城【水經溴水出河内軹縣原山南流注於河水東有無辟邑謂之無鼻城蕭子顯曰在河橋北二里】以兵守之服食所供粗免饑寒而已【粗坐五翻】 戊辰魏置常平倉 戊寅太子寶卷冠【卷讀曰捲冠古玩翻】 初魏文明太后欲廢魏主穆泰切諫而止【見一百三十七卷世祖永明八年】由是有寵及帝南遷洛陽所親任者多中州儒士宗室及代人往往不樂【樂音洛】泰自尚書右僕射出為定州刺史自陳久病土温則甚乞為恒州帝為之徙恒州刺史陸叡為定州以泰代之【為于偽翻下強為同】泰至叡未遂相與謀作亂陰結鎮北大將軍樂陵王思譽安樂侯隆撫冥鎮將魯郡侯業驍騎將軍超等共推朔州刺史陽平王頤為主思譽天賜之子【汝陰王天賜景穆太子之子於魏主為叔祖樂音洛將即亮翻驍堅堯翻騎奇寄翻】業丕之弟隆超皆丕之子也叡以為洛陽休明【左傳楚子伐陸渾之戎遂至於洛觀兵於周疆定王使王孫滿勞楚子楚子問鼎之大小輕重焉王孫滿曰德之休明雖小重也其姧回昏亂雖大輕也天祚明德有所底止周德雖衰天命未改鼎之輕重未可問也】勸泰緩之泰由是未頤偽許泰等以安其意而密以狀聞行吏部尚書任城王澄有疾【行吏部尚書者行吏部尚書事未為真也任音壬】帝召見於凝閒堂【見賢遍翻】謂之曰穆泰謀為不軌扇誘宗室【誘音酉】脫或必然今遷都甫爾北人戀舊南北紛擾朕洛陽不立也此國家大事非卿不能辦卿雖疾強為我北行【強其兩翻為于偽翻】審觀其埶儻其微弱直往擒之若已彊盛可承制發并肆兵擊之對曰泰等愚惑正由戀舊為此計耳非有深謀遠慮臣雖駑怯足以制之【駑音奴】願陛下勿憂雖有犬馬之疾何敢辭也帝笑曰任城肯行朕復何憂【復扶又翻下正復同】遂授澄節銅虎竹使符御杖左右【漢文帝二年初與郡守為銅虎符竹使符應劭曰銅虎符第一至第五國家當兵遣使者至郡合符符合乃聽受之竹使符皆以竹箭五故長五寸鐫刻篆書第一至第五魏晉以下竹使符第一至第十御杖左右帶御杖在天子左右者授澄以為衛使疏吏翻】仍行恒州事行至鴈門鴈門太守夜告云泰已引兵西就陽平【陽平王頤刺朔州在平城西宋白曰朔州東北至平城二百六十里】澄遽令進發右丞孟斌曰事未可量宜依敇召幷肆兵然後徐進澄曰泰既謀亂應據堅城而更迎陽平度其所為當似埶弱【斌音彬量音良度徒洛翻】泰既不相拒無故發兵非宜也但速往鎮之民心自定遂倍道兼行先遣治書侍御史李煥單騎入代【漢宣帝幸宣室齋居而决事令侍御史二人治書侍側後因别置謂之治書侍御史魏謂平城為代郡治直之翻騎奇寄翻】出其不意曉諭泰黨示以禍福皆莫為之用泰計無所出帥麾下數百人攻煥不克【帥讀曰率】走出城西追擒之澄亦尋至【尋繼也】窮治黨與收陸叡等百餘人皆繫獄民間帖然澄具狀表聞帝喜召公卿以表示之曰任城可謂社稷臣也觀其獄辭正復皋陶何以過之【陶餘招翻】顧謂咸陽王禧等曰汝曹當此不能辦也 魏主謀入寇引見公卿於清徽堂曰朕卜宅土中綱條粗舉【書說命曰若網在綱有條而不紊見賢遍翻粗坐五翻】唯南寇未平安能效近世天子下帷於深宮之中乎朕今南征决矣但未知早晩之期比來術者皆云今往必克【比毗至翻】此國之大事宜君臣各盡所見勿以朕先言而依違於前同異於後也李冲對曰凡用兵之法須先論人事後察天道今卜筮雖吉而人事未備遷都尚新秋穀不稔未可以興師旅如臣所見宜俟來秋帝曰去十七年朕擁兵二十萬【齊世祖永明十一年魏高祖之太和十七年也魏定遷洛之議而止南伐之師至去年方入寇盖十九年也二十萬亦當作三十萬事並見上年去猶昨也又按當時衆號三十萬實則二十萬耳】此人事之盛也而天時不利今天時既從復云人事未備【復扶又翻】如僕射之言是終無征伐之期也寇戎咫尺異日將為社稷之憂朕何敢自安若秋行不捷諸君當盡付司寇不可不盡懷也【魏既都洛逼近淮漢故急於南伐以攘斥境土】 魏主以有罪徙邊者多逋亡乃制一人逋亡闔門充役光州刺史博陵崔挺上書諫曰天下善人少惡人多【少詩沼翻】若一人有罪延及闔門則司馬牛受桓魋之罰柳下惠嬰盗跖之誅【司馬牛之於桓魋柳下惠之於盗跖皆兄弟賢不肖既相遠而兄弟罪不相及古法也魋徒回翻跖之石翻】豈不哀哉帝善之遂除其制<br />
<br />
  資治通鑑卷一百四十  <br>
   </div> 

<script src="/search/ajaxskft.js"> </script>
 <div class="clear"></div>
<br>
<br>
 <!-- a.d-->

 <!--
<div class="info_share">
</div> 
-->
 <!--info_share--></div>   <!-- end info_content-->
  </div> <!-- end l-->

<div class="r">   <!--r-->



<div class="sidebar"  style="margin-bottom:2px;">

 
<div class="sidebar_title">工具类大全</div>
<div class="sidebar_info">
<strong><a href="http://www.guoxuedashi.com/lsditu/" target="_blank">历史地图</a></strong>  
<a href="http://www.880114.com/" target="_blank">英语宝典</a>  
<a href="http://www.guoxuedashi.com/13jing/" target="_blank">十三经检索</a> 
<br><strong><a href="http://www.guoxuedashi.com/gjtsjc/" target="_blank">古今图书集成</a></strong> 
<a href="http://www.guoxuedashi.com/duilian/" target="_blank">对联大全</a> <strong><a href="http://www.guoxuedashi.com/xiangxingzi/" target="_blank">象形文字典</a></strong> 

<br><a href="http://www.guoxuedashi.com/zixing/yanbian/">字形演变</a>  <strong><a href="http://www.guoxuemi.com/hafo/" target="_blank">哈佛燕京中文善本特藏</a></strong>
<br><strong><a href="http://www.guoxuedashi.com/csfz/" target="_blank">丛书&方志检索器</a></strong> <a href="http://www.guoxuedashi.com/yqjyy/" target="_blank">一切经音义</a>  

<br><strong><a href="http://www.guoxuedashi.com/jiapu/" target="_blank">家谱族谱查询</a></strong>  <strong><a href="http://shufa.guoxuedashi.com/sfzitie/" target="_blank">书法字帖欣赏</a></strong> 
<br>

</div>
</div>


<div class="sidebar" style="margin-bottom:0px;">

<font style="font-size:22px;line-height:32px">QQ交流群9:489193090</font>


<div class="sidebar_title">手机APP 扫描或点击</div>
<div class="sidebar_info">
<table>
<tr>
	<td width=160><a href="http://m.guoxuedashi.com/app/" target="_blank"><img src="/img/gxds-sj.png" width="140"  border="0" alt="国学大师手机版"></a></td>
	<td>
<a href="http://www.guoxuedashi.com/download/" target="_blank">app软件下载专区</a><br>
<a href="http://www.guoxuedashi.com/download/gxds.php" target="_blank">《国学大师》下载</a><br>
<a href="http://www.guoxuedashi.com/download/kxzd.php" target="_blank">《汉字宝典》下载</a><br>
<a href="http://www.guoxuedashi.com/download/scqbd.php" target="_blank">《诗词曲宝典》下载</a><br>
<a href="http://www.guoxuedashi.com/SiKuQuanShu/skqs.php" target="_blank">《四库全书》下载</a><br>
</td>
</tr>
</table>

</div>
</div>


<div class="sidebar2">
<center>


</center>
</div>

<div class="sidebar"  style="margin-bottom:2px;">
<div class="sidebar_title">网站使用教程</div>
<div class="sidebar_info">
<a href="http://www.guoxuedashi.com/help/gjsearch.php" target="_blank">如何在国学大师网下载古籍?</a><br>
<a href="http://www.guoxuedashi.com/zidian/bujian/bjjc.php" target="_blank">如何使用部件查字法快速查字?</a><br>
<a href="http://www.guoxuedashi.com/search/sjc.php" target="_blank">如何在指定的书籍中全文检索?</a><br>
<a href="http://www.guoxuedashi.com/search/skjc.php" target="_blank">如何找到一句话在《四库全书》哪一页?</a><br>
</div>
</div>


<div class="sidebar">
<div class="sidebar_title">热门书籍</div>
<div class="sidebar_info">
<a href="/so.php?sokey=%E8%B5%84%E6%B2%BB%E9%80%9A%E9%89%B4&kt=1">资治通鉴</a> <a href="/24shi/"><strong>二十四史</strong></a>&nbsp; <a href="/a2694/">野史</a>&nbsp; <a href="/SiKuQuanShu/"><strong>四库全书</strong></a>&nbsp;<a href="http://www.guoxuedashi.com/SiKuQuanShu/fanti/">繁体</a>
<br><a href="/so.php?sokey=%E7%BA%A2%E6%A5%BC%E6%A2%A6&kt=1">红楼梦</a> <a href="/a/1858x/">三国演义</a> <a href="/a/1038k/">水浒传</a> <a href="/a/1046t/">西游记</a> <a href="/a/1914o/">封神演义</a>
<br>
<a href="http://www.guoxuedashi.com/so.php?sokeygx=%E4%B8%87%E6%9C%89%E6%96%87%E5%BA%93&submit=&kt=1">万有文库</a> <a href="/a/780t/">古文观止</a> <a href="/a/1024l/">文心雕龙</a> <a href="/a/1704n/">全唐诗</a> <a href="/a/1705h/">全宋词</a>
<br><a href="http://www.guoxuedashi.com/so.php?sokeygx=%E7%99%BE%E8%A1%B2%E6%9C%AC%E4%BA%8C%E5%8D%81%E5%9B%9B%E5%8F%B2&submit=&kt=1"><strong>百衲本二十四史</strong></a>  <a href="http://www.guoxuedashi.com/so.php?sokeygx=%E5%8F%A4%E4%BB%8A%E5%9B%BE%E4%B9%A6%E9%9B%86%E6%88%90&submit=&kt=1"><strong>古今图书集成</strong></a>
<br>

<a href="http://www.guoxuedashi.com/so.php?sokeygx=%E4%B8%9B%E4%B9%A6%E9%9B%86%E6%88%90&submit=&kt=1">丛书集成</a> 
<a href="http://www.guoxuedashi.com/so.php?sokeygx=%E5%9B%9B%E9%83%A8%E4%B8%9B%E5%88%8A&submit=&kt=1"><strong>四部丛刊</strong></a>  
<a href="http://www.guoxuedashi.com/so.php?sokeygx=%E8%AF%B4%E6%96%87%E8%A7%A3%E5%AD%97&submit=&kt=1">說文解字</a> <a href="http://www.guoxuedashi.com/so.php?sokeygx=%E5%85%A8%E4%B8%8A%E5%8F%A4&submit=&kt=1">三国六朝文</a>
<br><a href="http://www.guoxuedashi.com/so.php?sokeytm=%E6%97%A5%E6%9C%AC%E5%86%85%E9%98%81%E6%96%87%E5%BA%93&submit=&kt=1"><strong>日本内阁文库</strong></a> <a href="http://www.guoxuedashi.com/so.php?sokeytm=%E5%9B%BD%E5%9B%BE%E6%96%B9%E5%BF%97%E5%90%88%E9%9B%86&ka=100&submit=">国图方志合集</a> <a href="http://www.guoxuedashi.com/so.php?sokeytm=%E5%90%84%E5%9C%B0%E6%96%B9%E5%BF%97&submit=&kt=1"><strong>各地方志</strong></a>

</div>
</div>


<div class="sidebar2">
<center>

</center>
</div>
<div class="sidebar greenbar">
<div class="sidebar_title green">四库全书</div>
<div class="sidebar_info">

《四库全书》是中国古代最大的丛书,编撰于乾隆年间,由纪昀等360多位高官、学者编撰,3800多人抄写,费时十三年编成。丛书分经、史、子、集四部,故名四库。共有3500多种书,7.9万卷,3.6万册,约8亿字,基本上囊括了古代所有图书,故称“全书”。<a href="http://www.guoxuedashi.com/SiKuQuanShu/">详细>>
</a>

</div> 
</div>

</div>  <!--end r-->

</div>
<!-- 内容区END --> 

<!-- 页脚开始 -->
<div class="shh">

</div>

<div class="w1180" style="margin-top:8px;">
<center><script src="http://www.guoxuedashi.com/img/plus.php?id=3"></script></center>
</div>
<div class="w1180 foot">
<a href="/b/thanks.php">特别致谢</a> | <a href="javascript:window.external.AddFavorite(document.location.href,document.title);">收藏本站</a> | <a href="#">欢迎投稿</a> | <a href="http://www.guoxuedashi.com/forum/">意见建议</a> | <a href="http://www.guoxuemi.com/">国学迷</a> | <a href="http://www.shuowen.net/">说文网</a><script language="javascript" type="text/javascript" src="https://js.users.51.la/17753172.js"></script><br />
  Copyright &copy; 国学大师 古典图书集成 All Rights Reserved.<br>
  
  <span style="font-size:14px">免责声明:本站非营利性站点,以方便网友为主,仅供学习研究。<br>内容由热心网友提供和网上收集,不保留版权。若侵犯了您的权益,来信即刪。scp168@qq.com</span>
  <br />
ICP证:<a href="http://www.beian.miit.gov.cn/" target="_blank">鲁ICP备19060063号</a></div>
<!-- 页脚END --> 
<script src="http://www.guoxuedashi.com/img/plus.php?id=22"></script>
<script src="http://www.guoxuedashi.com/img/tongji.js"></script>

</body>
</html>
