<!DOCTYPE html PUBLIC "-//W3C//DTD XHTML 1.0 Transitional//EN" "http://www.w3.org/TR/xhtml1/DTD/xhtml1-transitional.dtd">
<html xmlns="http://www.w3.org/1999/xhtml">
<head>
<meta http-equiv="Content-Type" content="text/html; charset=utf-8" />
<meta http-equiv="X-UA-Compatible" content="IE=Edge,chrome=1">
<title>資治通鑒_185-資治通鑑卷一百八十四_185-資治通鑑卷一百八十四</title>
<meta name="Keywords" content="資治通鑒_185-資治通鑑卷一百八十四_185-資治通鑑卷一百八十四">
<meta name="Description" content="資治通鑒_185-資治通鑑卷一百八十四_185-資治通鑑卷一百八十四">
<meta http-equiv="Cache-Control" content="no-transform" />
<meta http-equiv="Cache-Control" content="no-siteapp" />
<link href="/img/style.css" rel="stylesheet" type="text/css" />
<script src="/img/m.js?2020"></script> 
</head>
<body>
 <div class="ClassNavi">
<a  href="/24shi/">二十四史</a> | <a href="/SiKuQuanShu/">四库全书</a> | <a href="http://www.guoxuedashi.com/gjtsjc/"><font  color="#FF0000">古今图书集成</font></a> | <a href="/renwu/">历史人物</a> | <a href="/ShuoWenJieZi/"><font  color="#FF0000">说文解字</a></font> | <a href="/chengyu/">成语词典</a> | <a  target="_blank"  href="http://www.guoxuedashi.com/jgwhj/"><font  color="#FF0000">甲骨文合集</font></a> | <a href="/yzjwjc/"><font  color="#FF0000">殷周金文集成</font></a> | <a href="/xiangxingzi/"><font color="#0000FF">象形字典</font></a> | <a href="/13jing/"><font  color="#FF0000">十三经索引</font></a> | <a href="/zixing/"><font  color="#FF0000">字体转换器</font></a> | <a href="/zidian/xz/"><font color="#0000FF">篆书识别</font></a> | <a href="/jinfanyi/">近义反义词</a> | <a href="/duilian/">对联大全</a> | <a href="/jiapu/"><font  color="#0000FF">家谱族谱查询</font></a> | <a href="http://www.guoxuemi.com/hafo/" target="_blank" ><font color="#FF0000">哈佛古籍</font></a> 
</div>

 <!-- 头部导航开始 -->
<div class="w1180 head clearfix">
  <div class="head_logo l"><a title="国学大师官网" href="http://www.guoxuedashi.com" target="_blank"></a></div>
  <div class="head_sr l">
  <div id="head1">
  
  <a href="http://www.guoxuedashi.com/zidian/bujian/" target="_blank" ><img src="http://www.guoxuedashi.com/img/top1.gif" width="88" height="60" border="0" title="部件查字,支持20万汉字"></a>


<a href="http://www.guoxuedashi.com/help/yingpan.php" target="_blank"><img src="http://www.guoxuedashi.com/img/top230.gif" width="600" height="62" border="0" ></a>


  </div>
  <div id="head3"><a href="javascript:" onClick="javascript:window.external.AddFavorite(window.location.href,document.title);">添加收藏</a>
  <br><a href="/help/setie.php">搜索引擎</a>
  <br><a href="/help/zanzhu.php">赞助本站</a></div>
  <div id="head2">
 <a href="http://www.guoxuemi.com/" target="_blank"><img src="http://www.guoxuedashi.com/img/guoxuemi.gif" width="95" height="62" border="0" style="margin-left:2px;" title="国学迷"></a>
  

  </div>
</div>
  <div class="clear"></div>
  <div class="head_nav">
  <p><a href="/">首页</a> | <a href="/ShuKu/">国学书库</a> | <a href="/guji/">影印古籍</a> | <a href="/shici/">诗词宝典</a> | <a   href="/SiKuQuanShu/gxjx.php">精选</a> <b>|</b> <a href="/zidian/">汉语字典</a> | <a href="/hydcd/">汉语词典</a> | <a href="http://www.guoxuedashi.com/zidian/bujian/"><font  color="#CC0066">部件查字</font></a> | <a href="http://www.sfds.cn/"><font  color="#CC0066">书法大师</font></a> | <a href="/jgwhj/">甲骨文</a> <b>|</b> <a href="/b/4/"><font  color="#CC0066">解密</font></a> | <a href="/renwu/">历史人物</a> | <a href="/diangu/">历史典故</a> | <a href="/xingshi/">姓氏</a> | <a href="/minzu/">民族</a> <b>|</b> <a href="/mz/"><font  color="#CC0066">世界名著</font></a> | <a href="/download/">软件下载</a>
</p>
<p><a href="/b/"><font  color="#CC0066">历史</font></a> | <a href="http://skqs.guoxuedashi.com/" target="_blank">四库全书</a> |  <a href="http://www.guoxuedashi.com/search/" target="_blank"><font  color="#CC0066">全文检索</font></a> | <a href="http://www.guoxuedashi.com/shumu/">古籍书目</a> | <a   href="/24shi/">正史</a> <b>|</b> <a href="/chengyu/">成语词典</a> | <a href="/kangxi/" title="康熙字典">康熙字典</a> | <a href="/ShuoWenJieZi/">说文解字</a> | <a href="/zixing/yanbian/">字形演变</a> | <a href="/yzjwjc/">金 文</a> <b>|</b>  <a href="/shijian/nian-hao/">年号</a> | <a href="/diming/">历史地名</a> | <a href="/shijian/">历史事件</a> | <a href="/guanzhi/">官职</a> | <a href="/lishi/">知识</a> <b>|</b> <a href="/zhongyi/">中医中药</a> | <a href="http://www.guoxuedashi.com/forum/">留言反馈</a>
</p>
  </div>
</div>
<!-- 头部导航END --> 
<!-- 内容区开始 --> 
<div class="w1180 clearfix">
  <div class="info l">
   
<div class="clearfix" style="background:#f5faff;">
<script src='http://www.guoxuedashi.com/img/headersou.js'></script>

</div>
  <div class="info_tree"><a href="http://www.guoxuedashi.com">首页</a> > <a href="/SiKuQuanShu/fanti/">四库全书</a>
 > <h1>资治通鉴</h1> <!--         下载:【右键另存为】即可 --></div>
  <div class="info_content zj clearfix">
  
<div class="info_txt clearfix" id="show">
<center style="font-size:24px;">185-資治通鑑卷一百八十四</center>
    資治通鑑卷一百八十四 宋 司馬光 撰<br />
<br />
  胡三省 音註<br />
<br />
  隋紀八【起彊圉赤奮若六月不滿一年】<br />
<br />
  恭皇帝下<br />
<br />
  義寧元年六月己卯李建成等至晉陽 劉文静勸李淵與突厥相結【厥九勿翻 考異曰創業注突厥去覘人來報文武入賀帝曰且勿相賀當為諸君召而使之即自手與突厥書蓋温大雅欲歸功高祖耳今從唐書劉文静傳】資其士馬以益兵勢淵從之自為手啓卑辭厚禮遺始畢可汗【遺于季翻可從刋入聲汗音寒 考異曰創業注云仍命封題署云名啓所司請改啓為書帝不許按太宗云太上皇稱臣於突厥蓋謂此時但温大雅諱之耳】云欲大舉義兵遠迎主上復與突厥和親如開皇之時若能與我俱南願勿侵暴百姓若但和親坐受寶貨亦唯可汗所擇始畢得啓謂其大臣曰隋主為人我所知也若迎以來必害唐公而擊我無疑矣苟唐公自為天子我當不避盛暑以兵馬助之即命以此意為復書使者七日而返將佐皆喜請從突厥之言【使疏吏翻將即亮翻下同】淵不可裴寂劉文静皆曰今義兵雖集而戎馬殊乏胡兵非所須【須者意所欲也】而馬不可失若復稽回【復扶又翻】恐其有悔淵曰諸君宜更思其次寂等乃請尊天子為太上皇立代王為帝以安隋室移檄郡縣改易旗幟雜用絳白以示突厥【隋色尚赤今用絳而雜之以白示若不純於隋幟昌志翻厥九勿翻】淵曰此可謂掩耳盗鍾【此鄙語也言盜鍾者惡鍾聲之聞而掩耳盜之此可以自欺而不可以欺人也】然逼於時事不得不爾乃許之遣使以此議告突厥【使疏吏翻厥九勿翻】西河郡不從淵命甲申淵使建成世民將兵擊西河【將即亮翻又音如字領也 考異曰創業注云命大郎二郎率衆討西河高祖太宗實錄但云命太宗徇西河蓋史官没建成之名耳唐殷嶠傳從隱太子攻西河今從創業注】命太原令太原温大有與之偕行【隋志太原縣舊曰晉陽開皇十年分置太原縣而改後齊所置龍山縣為晉陽縣二縣並帶太原郡令力正翻】曰吾兒年少【少詩照翻】以卿參謀軍事事之成敗當以此行卜之時軍士新集咸未閲習建成世民與之同甘苦遇敵則以身先之【先悉薦翻】近道菜果非買不食軍士有竊之者輒求其主償之亦不詰竊者【詰去吉翻】軍士及民皆感悦至西河城下民有欲入城者皆聽其入郡丞高德儒閉城拒守己丑攻拔之執德儒至軍門世民數之曰汝指野鳥為鸞以欺人主取高官【事見一百八十二卷大業十一年數所具翻又所主翻】吾興義兵正為誅佞人耳【為于偽翻】遂斬之自餘不戮一人秋毫無犯各尉撫使復業【尉與慰同】遠近聞之大悦【義師初起而人心如此固可以取天下矣】建成等引兵還晉陽往返凡九日【還從宣翻又音如字】淵喜曰以此行兵雖橫行天下可也【言世民行兵有紀律也】遂定入關之計淵開倉以賑貧民【賑津忍翻】應募者日益多淵命為三軍分左右通謂之義士裴寂等上淵號為大將軍【上時掌翻】癸巳建大將軍府以寂為長史【長知兩翻】劉文靜為司馬唐儉及前長安尉温大雅為記室大雅仍與弟大有共掌機密武士彠為鎧曹劉政會及武城崔善為太原張道源為戶曹晉陽長上邽姜謩為司功參軍太谷長殷開山為府掾【此唐公開大將軍府署置官属參用隋親王府大將軍府州郡官屬之制也隋制唯親王有掾有属有記室大將軍府有鎧曹州郡有戶曹皆行參軍也煬帝改州為郡郡置諸司書佐而書佐即參軍之職行書佐即行參軍之職也隋志武城縣屬清河郡上邽縣帶天水郡太谷縣屬太原郡舊曰陽邑開皇十八年改名彠一虢翻鎧可亥翻謩與謨同長知兩翻掾以絹翻】長孫順德劉弘基竇琮及鷹揚郎將高平王長諧天水姜寶誼陽屯為左右統軍【高平縣後魏置高平郡隋已改為平高縣煬帝改秦州為天水郡因古郡名也統軍後魏所置將即亮翻統他綜翻】自餘文武隨才授任又以世子建成為隴西公左領軍大都督左三統軍隸焉世民為敦煌公【敦大門翻】右領軍大都督右三統軍隸焉各置官屬以柴紹為右領軍府長史【此左右領軍以總領左右軍而名非取隋十二衛左右領軍之職而名也】諮議譙人劉贍領西河通守【此大將軍府諮議參軍也譙縣屬譙郡贍而艶翻守式又翻】道源名河開山名嶠皆以字行開山不害之孫也【殷不害以孝行聞於陳隋之間】 李密復帥衆向東都【復扶又翻下同帥讀曰率】丙申大戰于平樂園【此蓋即漢魏平樂觀之地為園也然漢魏平樂觀在洛城西隋既遷營新都則平樂圉當在都城東樂音洛】密左騎右步【騎奇寄翻】中列彊弩鳴千鼓以衝之東都兵大敗密復取回洛倉 突厥遣其柱國康鞘利等【厥九勿翻鞘所交翻】送馬千匹詣李淵為互市許發兵送淵入關多少隨所欲【少詩沼翻下同】丁酉淵引見康鞘利等受可汗書禮容盡恭【可從刋入聲汗音寒】贈遺康鞘利等甚厚擇其馬之善者止市其半義士請以私錢市其餘淵曰虜饒馬而貪利其來將不已恐汝不能市也吾所以少取者示貧且不以為急故也當為汝貰之【為于偽翻貰時制翻賒也】不足為汝費乙巳靈壽賊帥郗士陵【隋志靈壽縣屬恒山郡帥所類翻郗丑之翻】帥衆數千降於淵淵以為鎮東將軍燕郡公【帥讀曰率降戶江翻燕因肩翻】仍置鎮東府補僚屬以招撫山東郡縣己巳康鞘利北還淵命劉文靜使於突厥以請兵【使疏吏翻下同】私謂文靜曰胡騎入中國生民之大蠧也【騎奇寄翻】吾所以欲得之者恐劉武周引之共為邊患又胡馬行牧不費芻粟聊欲藉之以為聲勢耳數百人之外無所用之【觀唐公之言豈若肅代及石晉之君所為哉】 秋七月煬帝遣江都通守王世充將江淮勁卒將軍王隆帥卭黃蠻【按唐書卭部有烏蠻白蠻又謂羣蠻種類多不可記意必有黃蠻也守式又翻下同充將即亮翻又音如字領也帥讀曰率下同卭渠容翻】河北大使太常少卿韋霽河南大使虎牙郎將王辯等【二人蓋皆討捕大使也使疏吏翻少始照翻將即亮翻下同】各帥所領同赴東都相知討李密【帥讀曰率 考異曰雜記四月世充帥淮南兵萬人援東都世充行至彭城懼密衆之盛自以兵少不敵乃間行自黎陽濟河而至七月世充帥留守兵二萬擊密無功今從畧記蒲山公傳】霽世康之子也【韋世康開皇四大總管之一】壬子李淵以子元吉為太原太守留守晉陽宫後事悉以委之【守式又翻】癸丑淵帥甲士三萬發晉陽立軍門誓衆并移檄郡縣諭以尊立代王之意西突厥阿史那大奈亦帥其衆以從【大業八年分大奈之衆居樓煩故今亦從淵帥讀曰率厥九勿翻從才用翻】甲寅遣通議大夫張綸將兵徇稽胡【稽胡部落居邠石間】丙辰淵至西河慰勞吏民【勞力到翻】賑贍窮乏民年七十以上皆除散官【朝議等八郎武騎等八尉皆散官也賑津忍翻贍而艶翻散悉亶翻】其餘豪俊隨才授任口詢功能手注官秩一日除千餘人受官皆不取告身【唐志補官者皆給以符謂之告身猶今言付身也】各分淵所書官名而去淵入雀鼠谷壬戍軍賈胡堡【賈胡堡在霍邑西北括地志汾州靈石縣有賈胡堡賈咅古】去霍邑五十餘里代王侑遣虎牙郎將宋老生帥精兵二萬屯霍邑【將即亮翻】左武侯大將軍屈突通屯河東以拒淵【屈區勿翻】會積雨淵不得進遣府佐沈叔安等將羸兵還太原更運一月糧【將如字即良翻羸倫為翻還從宣翻又音如字】乙丑張綸克離石殺太守楊子崇【煬帝改石州為離石郡】劉文靜至突厥見始畢可汗請兵且與之約【厥九勿翻可從刋入聲汗音寒 考異曰唐劉文靜傳曰始畢曰唐公起事今欲何為文靜曰皇帝廢冢嫡傳位後主致斯禍亂唐公國之懿戚不忍坐觀成敗故起義軍欲黜不當立者創業起居注先已再遣使至突厥不容始畢方有此問今不取】曰若入長安民衆土地入唐公金玉繒帛歸突厥【繒慈陵翻厥九勿翻】始畢大喜丙寅遣其大臣級失特勒先至淵軍告以兵已上道【上時掌翻】淵以書招李密 【考異曰壺關錄云高祖屯壽陽遣右衛將軍張仁則齎書招李密蒲山公傳密答書曰使至辱今月十九日書按長歷是月己酉朔十九日丁卯不應己巳還至霍邑又發書日不應猶在壽陽今皆不取】密自恃兵彊欲為盟主使祖君彦復書曰與兄派流雖異根系本同【唐公出於李虎密出於李弼是異派也然李弼之先本遼東襄平人李虎祖西凉木隴西成紀人所謂根系但同姓耳】自唯虚薄為四海英雄共推盟主【唯當作惟惟思也詳見審是】所望左提右挈戮力同心執子嬰於咸陽【以謂代王】殪商辛於牧野【以謂煬帝殪於計翻】豈不盛哉且欲使淵以步騎數千自至河内【煬帝改懷州為河内郡騎奇寄翻】面結盟約淵得書笑曰密妄自矜大非折簡可致吾方有事關中若遽絶之乃是更生一敵不如卑辭推奬以驕其志使為我塞成臯之道綴東都之兵【塞成臯之道則江都信使不通綴東都之兵則不得西應長安折之舌翻為于偽翻塞悉則翻】我得專意西征俟關中平定據險養威徐觀鷸蚌之勢以收漁人之功未為晚也【戰國策趙且伐燕蘇代為燕說趙惠王曰今者臣來過易水蚌方出曝鷸啄其肉蚌合而拑其喙鷸曰今日不雨明日不雨即有蚌脯蚌亦謂鷸曰今日不出明日不出必見死鷸漁父見而并獲之今燕趙相持臣恐強秦之為漁父也唐公欲使李密與東都相持而已收漁人之利鷸餘律翻】乃使温大雅復書曰吾雖庸劣幸承餘緒出為八使【漢順帝遣八使唐公使山西河東故云然使疏吏翻下同】入典六屯【隋制六軍十二衛唐公嘗為將軍故云】顛而不扶通賢所責所以大會義兵和親北狄共匡天下志在尊隋天生烝民必有司牧當今為牧非子而誰老夫年逾知命【孔子曰五十而知天命】願不及此欣戴大弟攀鱗附翼唯弟早膺圖籙以寧兆民宗盟之長屬籍見容【屬籍宗屬之籍長知兩翻】復封於唐斯榮足矣殪商辛於牧野所不忍言執子嬰於咸陽未敢聞命汾晉左右尚須安輯盟津之會未暇卜期【盟讀曰孟】密得書甚喜以示將佐曰唐公見推天下不足定矣【將即亮翻】自是信使往來不絶雨久不止淵軍中糧乏劉文靜未返或傳突厥與劉武周乘虚襲晉陽淵召將佐謀北還【厥九勿翻還從宣翻又音如字】裴寂等皆曰宋老生屈突通連兵據險未易猝下【屈居勿翻易以豉翻】李密雖云連和姦謀難測突厥貪而無信唯利是視武周事胡者也太原一方都會且義兵家屬在焉不如還救根本更圖後舉李世民曰今禾菽被野何憂乏糧老生輕躁一戰可擒【被皮義翻躁則到翻】李密顧戀倉粟未遑遠略武周與突厥外雖相附内實相猜武周雖遠利太原豈可近忘馬邑本興大義奮不顧身以救蒼生當先入咸陽號令天下今遇小敵遽已班師恐從義之徒一朝解體還守太原一城之地為賊耳【還從宣翻又音如字】何以自全李建成亦以為然淵不聽促令引發世民將復入諫【令力丁翻復扶又翻】會日暮淵已寢世民不得入號哭於外聲聞帳中【號戶刀翻聞音問】淵召問之世民曰今兵以義動進戰則克退還則散衆散於前敵乘於後死亡無日何得不悲淵乃悟曰軍已發奈何世民曰右軍嚴而未發【嚴裝也】左軍雖去計亦未遠請自追之淵笑曰吾之成敗皆在爾知復何言【復扶又翻】唯爾所為世民乃與建成夜追左軍復還【復音如字 考異曰創業注帝集文武官人及大郎二郎等而謂之曰以天贊我應無此勢以人事見機而發無冇不為借遣吾當突厥武周之地何有不來之理諸公謂云何議者以老生屈突通相去不遠李密譎誑奸謀難測突厥見利而行武周事胡者也太原一都之會義兵家屬在焉愚夫所慮伏聽教旨唐公顧謂大郎二郎曰爾輩何如對曰武周位極而志滿突厥少信而貪利外雖相附内實相猜突厥必欲來利太原寧肯近忘馬邑武周悉其此勢未必同謀同志老生突厥奔競來拒進闕圖南退窮自北還無所入往無所之畏溺先沈近於斯矣今禾菽被野人馬無憂坐即有糧行即得衆李密戀於倉粟未遑遠畧老生輕躁破之不疑定業取威在兹一决諸人保家愛命言不可聽雨罷進軍若不殺老生兒等敢以死謝唐公喜曰爾謀得之吾其决矣三占從二何籍與言懦夫之徒幾敗乃公事耳太宗實錄盡以為太宗之策無建成名蓋没之耳據建成同追左軍則建成意亦不欲還也今從創業注】丙子太原運糧亦至 武威鷹揚府司馬李軌【煬帝改凉州為武威郡各郡置鷹揚府有郎將副郎將長史司馬】家富好任俠【好呼到翻下同】薛舉作亂於金城【是年夏四月薛舉起】軌與同郡曹珍關謹梁碩李贇安修仁等謀曰【贇於倫翻】薛舉必來侵暴郡官庸怯勢不能禦吾輩豈可束手并妻孥為人所虜邪【孥音奴邪音耶】不若相與并力拒之保據河右以待天下之變衆皆以為然欲推一人為主各相讓莫肯當曹珍曰久聞圖䜟李氏當王【䜟楚譛翻】今軌在謀中乃天命也遂相與拜軌奉以為主丙辰軌令修仁集諸胡【令力丁翻安氏凉州豪望世為民夷所附故使之集諸胡】軌結民間豪傑共起兵執虎賁郎將謝統師郡丞韋士政【賁音奔將即亮翻統他綜翻】軌自稱河西大涼王置官屬並擬開皇故事關謹等欲盡殺隋官分其家貲軌曰諸人既逼以為主當禀其號令今興義兵以救生民乃殺人取貨此羣盜耳將何以濟於是以統師為太僕卿士政為太府卿西突厥闕度設據會寧川【大業八年分闕度設居會寧厥九勿翻】自稱闕可汗請降於軌【可從刋入聲汗咅寒降戶江翻】 薛舉自稱秦帝【考異曰唐高祖實錄武德元年四月辛卯舉稱尊號按今冬舉敗問禇亮曰天子有降事否是則已稱尊號也今從唐書舉傳】立其妻鞠氏為皇后子仁果為皇太子遣仁果將兵圍天水克之舉自金城徙都之仁果多力善騎射【將即亮翻騎奇寄翻】軍中號萬人敵然性貪而好殺嘗獲庾信子立怒其不降磔於火上稍割以噉軍士【庾信自梁入關有文名史言薛仁果在兵間不能收禮文藝名義之士卒以敗亡好呼到翻降戶江翻磔陟格翻噉徙濫翻】及克天水悉召富人倒懸之以醋灌鼻責其金寶舉每戒之曰汝之才略足以辦事然苛虐無恩終當覆我國家舉遣晉王仁越將兵趨劍口至河池郡太守蕭瑀拒却之【劒口劒門關口舉指授仁越使之趨劍口未至而蕭瑀以河池拒之遂退却將即亮翻下同趨七喻翻又逡須翻瑀咅禹守式又翻】又遣其將常仲興濟河擊李軌與軌將李贇戰於昌松【隋志昌松縣屬武威郡】仲興舉軍敗没軌欲縱遣之贇曰力戰獲俘復縱以資敵將焉用之不如盡阬之【復扶又翻下同焉於乾翻】軌曰天若祚我當擒其主此屬終為我有若其無成留之何益乃縱之【李軌不殺隋官縱薛舉兵皆有人君之言其才略不足以濟則徒言無益也】未幾攻張掖敦煌西平枹罕皆克之【幾居豈翻敦徒門翻枹咅膚】盡有河西五郡之地 煬帝詔左禦衛大將軍涿郡留守薛世雄將燕地精兵三萬討李密【燕因肩翻】命王世充等諸將皆受世雄節度所過盜賊隨便誅翦世雄行至河間軍於七里井【七里井蓋其地去河間七里故名】竇建德士衆惶懼悉拔諸城南遁聲言還入豆子䴚【䴚各朗翻】世雄以為畏已不復設備建德謀還襲之其處去世雄營百四十里建德帥敢死士二百八十人先行【帥讀曰率】令餘衆續發建德與其士衆約曰夜至則擊其營已明則降之【降戶江翻下同】未至一里所天欲明建德惶惑議降會天大霧人咫尺不相辨建德喜曰天贊我也【贊助也】遂突入其營擊之世雄士卒大亂皆騰栅走世雄不能禁與左右數十騎遁歸涿郡 【考異曰革命記帝以李密在洛口征遼回日令右翊衛將軍薛世雄於留鎮兵内簡練精鋭及幽易驍勇討密經過之處若有草竊隨便誅翦仍令王世充等諸軍並取世雄處分世雄乃自領精兵六萬四月末至河間郡城下作營州縣皆備牛酒軍糧以待薛將軍時建德以無糧食兵士先皆分散餘軍不滿千人在武強縣境收麥充食聞世雄兵至河間惶懼無計問一女巫欲走避之如何巫云不免問欲首如何巫云亦不吉問欲掩其不備擊之如何巫云今夜天未明到大吉卜時日己午卜處去河間一百四十里建德簡精兵二百八十人先行餘勒續發建德與衆决云夜到即打明即降之吉凶之事在此舉耳遂行去世雄營二里天已屬明又聞吹角聲擬發建德惶惑欲降須臾大霧忽起建德曰此天助我也遂引兵入營攻之兵遂大亂世雄左右先已裝束擬發世雄遂得上馬奔走仍中數槍傼而獲免幽易之士並不欲作留鎮兵先無闘意既不知賊多少悉弃甲奔亡遂使山東賊勢轉盛李密先招慰河北州縣多悉從之世雄慙憤而卒唐竇建德傳云七月世雄討之建德帥敢死士千人襲之世雄以數百騎遁去今從隋薛世雄傳以建德傳革命記參之】慙恚發病卒【恚於避翻卒子恤翻】建德遂圍河間 八月己卯雨霽庚辰李淵命軍中曝鎧仗行裝【鎧可亥翻】辛巳旦東南由山足細道趣霍邑【趣七喻翻又逡須翻】淵恐宋老生不出李建成李世民曰老生勇而無謀以輕騎挑之【挑徒了翻】理無不出脱其固守則誣以貳於我彼恐為左右所奏安敢不出淵曰汝測之善老生不能逆戰賈胡【謂淵屯賈胡堡時老生不能逆戰賈音古】吾知其無能為也淵與數百騎先至霍邑城東數里以待步兵使建成世民將數十騎至城下舉鞭指麾若將圍城之狀且詬之【騎奇寄翻詬苦候翻】老生怒引兵三萬自東門南門分道而出淵使殷開山趣召後軍【趣讀曰促】後軍至淵欲使軍士先食而戰世民曰時不可失淵乃與建成陳於城東世民陳於城南【陳讀曰陣下同】淵建成戰小却世民與軍頭臨淄段志玄自南原引兵馳下【新唐志曰武德元年改鷹揚郎將曰軍頭蓋起兵之初已置軍頭也後又改軍頭為驃騎將軍隋志臨淄縣屬北海郡】衝老生陳出其背世民手殺數十人兩刀皆缺流血滿袖灑之復戰淵兵復振【復扶又翻】因傳呼曰已獲老生矣老生兵大敗淵兵先趣其門【趣七喻翻】門閉老生下馬投塹劉弘基就斬之僵尸數里【塹七艶翻僵居良翻】日已暮淵即命登城時無攻具將士肉薄而登遂克之淵賞霍邑之功軍吏疑奴應募者不得與良人同淵曰矢石之間不辨貴賤論勲之際何有等差宜並從本勲授壬午淵引見霍邑吏民勞賞如西河【勞力到翻】選其丁壯使從軍關中軍士欲歸者並授五品散官【煬帝置散職九大夫朝請大夫正五品朝散大夫從五品散悉但翻】遣歸【既順其歸志又以動關中士民之心】或諫以官太濫淵曰隋氏吝惜勲賞此所以失人心也奈何效之且收衆以官不勝於用兵乎丙戌淵入臨汾郡【平陽古郡名後改置唐州後改為晉州開皇初改郡曰平河平陽縣改曰臨汾縣惡平陽之名也大業初改曰臨汾郡】慰撫如霍邑庚寅宿鼓山【鼔山在絳郡北】絳郡通守陳叔達拒守【通守式又翻煬帝改絳州為絳郡】辛卯進攻克之叔達陳高宗之子有才學淵禮而用之癸巳淵至龍門【龍門縣屬河東郡在郡東北】劉文靜康鞘利以突厥兵五百人馬二千匹來至淵喜其來緩謂文靜曰吾西行及河突厥始至兵少馬多皆君將命之功也【厥九勿翻少詩沼翻】汾陽薛大鼎說淵【按新唐書薛大鼎蒲州汾隂人隋唐志亦皆無汾陽縣陽當作隂說式芮翻】請勿攻河東自龍門直濟河據永豐倉傳檄遠近關中可坐取也淵將從之諸將請先攻河東乃以大鼎為大將軍府察非掾【察非掾言使之察姦非若漢刺姦掾也煬帝時左右候衛府增置察非掾諸將即亮翻掾俞絹翻】河東縣戶曹任瓌【河東縣帶河東郡舊曰蒲坂開皇十六年改名隋制縣置金戶兵法士等曹佐任音壬瓌古回翻】說淵曰關中豪傑皆企踵以待義兵瓌在馮翊積年【瓌仁壽中為馮翊韓城尉說式芮翻企去智翻】知其豪傑請往諭之必從風而靡義師自梁山濟河指韓城逼郃陽【梁山在韓城縣界臨河即左傳所謂梁山崩者也韓城郃陽二縣皆屬馮翊郡隋所置也杜佑曰同州韓城縣漢為夏陽縣有梁山龍門山宋白曰今韓城縣西南三里有夏陽故城乃韓國故城今縣理南二十五里有少梁故城隋文帝分郃陽故城於此置韓城縣以古韓城為名郃古沓翻】蕭造文吏必當望塵請服孫華之徒皆當遠迎然後鼔行而進直據永豐雖未得長安關中固已定矣淵悦以瓌為銀青光祿大夫【隋制銀青光祿散職從三品】時關中羣盜孫華最彊丙申淵至汾隂以書招之【汾隂縣屬河東郡】己亥淵進軍壺口【隋志文城郡昌寧縣冇壺口山】河濱之民獻舟者日以百數仍置水軍壬寅孫華自郃陽輕騎渡河見淵【騎奇寄翻下同】淵握手與坐慰奬之以華為左光祿大夫武鄉縣公領馮翊太守【隋制散職左光祿正二品馮翊縣後魏曰華隂西魏改曰武鄉大業初改曰馮翊今以開皇舊縣名封華守式又翻】其徒有功者委華以次授官賞賜甚厚使之先濟繼遣左右統軍王長諧劉弘基及左領軍長史陳演壽【陳演壽建成府元僚長知兩翻】金紫光祿大夫史大奈【金紫光祿散職正三品】將步騎六千自梁山濟營於河西以待大軍以任瓌為招慰大使瓌說韓城下之淵謂長諧曰屈突通精兵不少【任咅壬瓌古回翻使疏吏翻說式芮翻少詩沼翻】相去五十餘里不敢來戰足明其衆不為之用然通畏罪不敢不出若自濟河擊卿等則我進攻河東必不能守若全軍守城則卿等絶其河梁【河梁謂蒲津橋】前扼其喉後拊其背彼不走必為擒矣 驍果從煬帝在江都者多逃去【驍堅堯翻】帝患之以問裴矩對曰人情非有匹偶難以久處【處昌呂翻】請聽軍士於此納室帝從之九月悉召江都境内寡婦處女集宫下恣將士所取或先與姦者聽自首【處昌呂翻將即亮翻首羊又翻】即以配之 武陽郡丞元寶藏以郡降李密【煬帝改魏州為武陽郡降戶江翻下同】甲寅密以寶藏為上柱國武陽公寶藏使其客鉅鹿魏徵為啓謝密【隋志鉅鹿縣屬襄國郡】且請改武陽為魏州又請帥所部西取魏郡【煬帝改相州為魏郡帥讀曰率】南會諸將取黎陽倉【汲郡黎陽縣有黎陽倉將即亮翻】密喜即以寶藏為魏州總管召魏徵為元帥府文學參軍掌記室【帥所類翻】徵少孤貧好讀書有大志【少詩照翻好呼到翻】落拓不事生業始為道士寶藏召典書記密愛其文辭故召之初貴鄉長弘農魏德深【隋志貴鄉縣帶武陽郡劉昫曰魏州漢魏郡元城縣之地後魏天平二年分舘陶西界於今州西北三十里古趙城置貴鄉縣後魏建德七年以趙城卑濕西南移三十里就孔思集寺為貴鄉縣大象二年於縣置魏州隋改名武陽郡隋志魏德深本鉅鹿人家弘農隋河南郡陜縣後魏之弘農郡也弘農郡之弘農縣後魏之西弘農郡也魏避諱弘作恒長知兩翻】為政清靜不嚴而治【治直吏翻】遼東之役徵税百端使者旁午責成郡縣民不堪命唯貴鄉閭里不擾有無相通不竭其力所求皆給元寶藏受詔捕賊數調器械【數所角翻調徒釣翻】動以軍法從事其鄰城營造皆聚於聽事【聽讀曰廳】官吏逓相督責晝夜喧囂猶不能濟德深聽隨便修營官府寂然恒若無事【恒戶登翻】唯戒吏以不須過勝餘縣使百姓勞苦然民各自竭心常為諸縣之最民愛之如父母寶藏深害其能遣將千兵赴東都【將即亮翻】所領兵聞寶藏降密思其親戚輒出都門東向慟哭而返或勸之降密皆泣曰我與魏明府同來何忍弃去河南山東大水餓殍滿野【殍平表翻】煬帝詔開黎陽倉賑之吏不時給死者日數萬人徐世勣言於李密曰天下大亂本為饑饉【為于偽翻】今更得黎陽倉大事濟矣密遣世勣帥麾下五千人自原武濟河【隋志原武縣屬滎陽郡開皇十六年置帥讀曰率】會元寶藏郝孝德李文相及洹水賊帥張升清河賊帥趙君德兵襲破黎陽倉據之【隋志洹水縣屬魏郡後周置洹于元翻又音桓帥所類翻 考異曰河洛記今年四月祖君彦檄云又得回洛復取黎陽天下之倉盡非隋有而九月魏徵啓方勸取黎陽蓋君彦為檄欲虚張聲勢非事實也】開倉恣民就食浹旬間得勝兵二十餘萬【浹子恊翻勝音升 考異曰唐李勣傳勣初得黎陽倉就食者數十萬人魏徵高季輔杜正倫郭孝恪皆客游其所一見於衆人中即加禮敬引之卧内談謔忘倦按徵為元寶藏作啓方謀取黎陽倉高季輔已為汲令杜正倫為羽騎都尉郭孝恪先在密所足知此事為虚今不取 余按隋置羽騎尉都字衍】武安永安義陽弋陽齊郡相繼降密【煬帝改洛州為武安郡黃州為永安郡義陽郡齊梁曰司州後魏曰郢州後周改申州大業二年改義州尋改為郡改光州為弋陽郡改齊州為齊郡】竇建德朱粲之徒亦遣使附密【使疏吏翻】密以粲為揚州總管鄧公【以粲總管揚州而爵為鄧公也】泰山道士徐洪客獻書於密以為大衆久聚恐米盡人散師老厭戰難可成功勸密乘進取之機因士馬之鋭沿流東指直向江都執取獨夫號令天下密壯其言以書招之洪客竟不出莫知所之 乙卯張綸徇龍泉文成等郡【煬帝改隰州為龍泉郡治隰川縣漢之蒲子縣也改汾州為文城郡治吉昌縣後魏定陽縣也】皆下之獲文成太守鄭元璹元璹譯之子也【成隋文帝業者鄭譯也守式又翻璹殊玉翻】 屈突通遣虎牙郎將桑顯和將驍果數千人夜襲王長諧等營長諧等戰不利【將即亮翻 考異曰創業注云桑顯和帥驍果精兵數千夜馳掩襲長諧等軍營諧及孫華等奉教備豫故並覺之伺和赴營設伏分擊應時摧散唐高祖本紀云義師不利太宗以遊騎數百掩其後顯和潰散按太宗時未過河西今從高祖實錄及唐史大奈傳】孫華史大奈以遊騎自後擊顯和大破之【騎奇寄翻】顯和脱走入城仍自絶河梁丙辰馮翊太守蕭造降於李淵造脩之子也【梁宜豐侯循一作脩】戊午淵帥諸軍圍河東【降戶江翻帥讀曰率 考異曰創業注戊午唐公親率諸軍圍河東郡屈突通不敢出閉門自守城甚高峻不易可攻唐公觀義士等志試遣登之南面千餘人應時而上時值雨甚公命旋師軍人時速上城不時速下公曰屈突宿衛舊人解安陣隊野戰非其所長嬰城善為捍禦我師常勝入必輕之驍鋭先登恐無還路今且示威而已未是攻城之時殺人得城知何所用乃命還唐高祖實錄云驍勇千餘人已登其南城高祖在東原會暴雨高祖鳴角收衆由是不克温大雅因為虚美耳今不取】屈突通嬰城自守將佐復推淵領太尉【復扶又翻】增置官屬淵從之時河東未下三輔豪傑至者日以千數淵欲引兵西趣長安【趣七喻翻又逡須翻】猶豫未决裴寂曰屈突通擁大衆憑堅城吾捨之而去若進攻長安不克退為河東所踵腹背受敵此危道也不若先克河東然後西上【上時掌翻】長安恃通為援通敗長安必破矣李世民曰不然兵貴神速吾席累勝之威撫歸順之衆鼓行而西長安之人望風震駭智不及謀勇不及斷【斷丁亂翻】取之若振槁葉耳若淹留自弊於堅城之下彼得成謀脩備以待我坐費日月衆心離沮【沮在呂翻】則大事去矣且關中蜂起之將未有所屬不可不早招懷也【將即亮翻下同】屈突通自守虜耳不足為慮淵兩從之留諸將圍河東自引軍而西朝邑法曹武功靳孝謨以蒲津中潬二城降【隋志朝邑縣屬馮翊郡後魏曰南五泉西魏改焉其地當蒲津橋西唐改為河西縣梁大河為橋故有中潬朝直遥翻靳居焮翻潬徒旱翻降戶江翻下同】華隂令李孝常以永豐倉降【隋志華隂縣屬京兆郡華戶化翻】仍應接河西諸軍孝常圓通之子也【李圓通寵任於開皇之初】京兆諸縣亦多遣使請降【使疏吏翻】 王世充韋霽王辯及河内通守孟善誼河陽郡尉獨孤武都【河陽非郡也隋制舊有兵處州刺史帶諸軍事以統之煬帝罷州置郡别置都尉領兵與郡不相知郡尉當作都尉】各帥所領會東都【帥讀曰率下同】唯王隆後期不至【王隆帥卭黄蠻者也】己未越王侗使虎賁郎將劉長恭等帥留守兵龎玉等帥偃師兵與世充等合十餘萬衆擊李密於洛口【賁音奔守式又翻帥讀曰率 考異曰畧記作乙丑河洛記作十二日蒲山公傳九月十一日師出東都按長歷是月己酉朔乙丑十七日也今從蒲山公傳】與密夾洛水相守煬帝詔諸軍皆受世充節度 【考異曰畧記云世充擊密罔不摧破露布相續而來百姓忻忻歡詠於道蒲山公傳云自秋徂冬凡經三十餘戰世充多敗績河洛記云四十餘戰世充無功三書相違莫知孰是今皆不取唯勝負有顯狀者存之】帝遣攝江都郡丞馮慈明向東都為密所獲密素聞其名【慈明事煬帝於并省歷位于朝其名夙著】延坐勞問【勞力到翻】禮意甚厚因謂曰隋祚已盡公能與孤立大功乎慈明曰公家歷事先朝【朝直遥翻】榮祿兼備不能善守門閥乃與玄感舉兵偶脱罔羅得有今日唯圖反噬未諭高旨莽卓敦玄【王莽董卓王敦桓玄】非不彊盛一朝夷滅罪及祖宗僕死而後已不敢聞命密怒囚之慈明說防人席務本【說輪芮翻】使亡走奉表江都及致書東都論賊形勢至雍丘為密將李公逸所獲【將即亮翻下同】密又義而釋之出至營門翟讓殺之慈明子琮之子也【馮子琮事高齊死於琅邪王儼之難】密之克洛口也【是年二月密克洛口】箕山府郎將張季珣固守不下【大業十二年移箕山公路二府守洛口倉】密以其寡弱遣人呼之季珣罵密極口密怒遣兵攻之不能克時密衆數十萬在其城下季珣四面阻絶所領不過數百人而執志彌固誓以必死久之糧盡水竭【城在原上汲道不通故水竭】士卒羸病【羸倫為翻】季珣撫循之一無離叛自三月至于是月城遂陷季珣見密不肯拜曰天子爪牙何容拜賊密猶欲降之誘諭終不屈乃殺之【降戶江翻 考異曰隋書季珣傳云密攻之經三年遂為所陷又云密壯而釋之翟讓從求金不得遂殺之河洛記曰自三月至九月不下後為糧盡水竭乃被摧陷生獲珣於牙門遣人宣之以降為度珣更張目極罵不肯低屈遂殺之按密明年已降唐安得三年攻守箕山之事今參取二書去其牴牾者而已】季珣祥之子也【漢王諒舉兵張祥守井陘不下】庚申李淵帥諸軍濟河【帥讀曰率下同】甲子至朝邑舍於長春宫【隋志朝邑縣有長春宫】關中士民歸之者如市丙寅淵遣世子建成司馬劉文靜帥王長諧等諸軍數萬人屯永豐倉守潼關以備東方兵慰撫使竇軌等受其節度敦煌公世民帥劉弘基等諸軍數萬人徇渭北慰撫使殷開山等受其節度軌琮之兄也冠氏長于志寧安養尉顔師古【冠氏春秋邑名隋分館陶東界置冠氏縣屬武陽郡安養縣屬襄陽郡劉昫曰漢鄧城古樊城也宋改安養縣後周廢山都樊城二縣入焉使疏吏翻敦徒門翻】及世民婦兄長孫無忌謁見淵於長春宫【兄長知兩翻見賢遍翻】師古名籀以字行【籀直又翻】志寧宣敏之兄子【于宣敏見一百七十五卷陳宣帝太建十三年】師古之推之孫也【顔之推見一百七十三卷陳宣帝大建九年】皆以文學知名無忌仍有才畧淵皆禮而用之以志寧為記室師古為朝散大夫無忌為渭北行軍典籖【朝散大夫隋散職從五品自親王府至州郡皆有典籖朝直遥翻】屈突通聞淵西入署鷹揚郎將湯隂堯君素領河東通守【隋志湯隂縣屬汲郡】使守蒲坂【隋河東郡治河東縣古蒲坂也坂音反】自引兵數萬趣長安【趣七喻翻 考異曰唐書通傳云將自武關趨藍田赴長安疑其太迂今但云趨長安】為劉文靜所遏將軍劉綱戍潼關屯都尉南城【隋潼關有守兵故置都尉】通欲往依之王長諧先引兵襲斬綱據城以拒通通退保北城淵遣其將呂紹宗等攻河東不能克【將即亮翻】柴紹之自長安赴太原也【是年五月紹赴太原】謂其妻李氏曰尊公舉兵今偕行則不可留此則及禍奈何李氏曰君弟速行【弟與第同】我一婦人易以濳匿【易以豉翻】當自為計紹遂行李氏歸鄠縣别墅【隋志鄠縣屬京兆郡鄠音戶墅承與翻】散家貲聚徒衆淵從弟神通在長安【從才用翻下之從同】亡入鄠縣山中與長安大俠史萬寶等起兵以應淵西域商胡何潘仁入司竹園為盜【隋志京兆府盩厔縣有司竹園】有衆數萬劫前尚書右丞李綱為長史【長知兩翻】李氏使其奴馬三寶說潘仁【說輸芮翻】與之就神通合勢攻鄠縣下之神通衆逾一萬自稱關中道行軍總管以前樂城長令狐德棻為記室【考之隋志云信安郡有樂城縣又河南郡樂壽縣舊曰樂城長知兩翻令力定翻棻扶分翻下同】德棻熙之子也【今狐熙事宇文氏著勞績於河西】李氏又使馬三寶說羣盜李仲文向善志丘師利等皆帥衆從之仲文密之從父師利和之子也【丘和以饋食為煬帝所寵用說輸芮翻帥讀曰率】西京留守屢遣將討潘仁等皆為所敗【敗補邁翻】李氏徇盩厔武功始平皆下之【盩厔音舟窒隋志始平縣屬京兆郡唐改曰興平】衆至七萬左親衛段綸文振之子也娶淵女【段文振見一百八十一卷大業八年 考異曰唐太宗實錄云隱太子以琅邪長公主妻之劉子玄唐高祖實錄及新唐書皆云高密大長公主適段綸蓋改封】亦聚徒於藍田【隋志藍田縣屬京兆郡】得萬餘人及淵濟河神通李氏綸各遣使迎淵【使疏吏翻下同】淵以神通為光祿大夫子道彦為朝請大夫綸為金紫光祿大夫【隋散職光祿從一品金紫正三品朝請正五品朝直遥翻】使柴紹將數百騎並南山迎李氏【將即亮翻騎奇計翻並步浪翻自華山而南接盩厔鄠杜諸山皆長安南山也】何潘仁李仲文向善志及關中羣盜皆請降於淵淵一一以書慰勞授官【降戶江翻勞力到翻下同】使各居其所受敦煌公【敦徒門翻】世民節度刑部尚書領京兆内史衛文昇年老【煬帝改京兆河南尹為内史】聞淵兵向長安憂懼成疾不復預事【復扶又翻】獨左翊衛將軍隂世師京兆郡丞骨儀奉代王侑乘城拒守己巳淵如蒲津庚午自臨晉濟渭【朝邑古臨晉地】至永豐勞軍開倉賑飢民【賑津忍翻】辛未還長春宫壬申進屯馮翊【隋志馮翊縣帶郡】世民所至吏民及羣盜歸之如流世民收其豪俊以備僚屬營于涇陽【涇陽縣屬京兆郡】勝兵九萬【勝音升下同】李氏將精兵萬餘會世民於渭北與柴紹各置幕府號娘子軍先是平凉奴賊數萬圍扶風太守竇璡【帝改原州為平凉郡岐州為扶風郡先悉薦翻守式又翻璡將鄰翻】數月不下賊中食盡丘師利遣其弟行恭帥五百人負米麥持牛酒詣奴賊營【帥讀曰率】奴帥長揖行恭手斬之【帥所類翻】謂其衆曰汝輩皆良人何故事奴為主使天下謂之奴賊衆皆俯伏曰願改事公行恭即帥其衆與師利共謁世民於渭北【帥讀曰率下同】世民以為光祿大夫璡琮之從子也【從才用翻】隰城尉房玄齡謁世民於軍門 【考異曰舊唐書玄齡傳云温彦博又薦焉按彦博時在羅藝所今不取】世民一見如舊識署記室參軍引為謀主玄齡亦自以為遇知己罄竭心力知無不為淵命劉弘基殷開山分兵西畧扶風有衆六萬南度渭水屯長安故城【考異曰創業注云敦煌公自涇陽趨司竹留弘基開山屯長安故城今從唐書弘基傳】城中出戰弘基逆擊破之世民引兵趣司竹李仲文何潘仁向善志皆帥衆從之頓于阿城【趣七喻翻帥讀曰率阿城即秦阿房宫城】勝兵十三萬軍令嚴整秋毫不犯乙亥世民自盩厔遣使白淵請期日赴長安淵曰屈突東行不能復西不足虞矣【屈居勿翻復扶又翻】乃命建成選倉上精兵自新豐趣長樂宫【新豐縣屬京兆郡長樂宫故漢宫也樂音洛】世民帥新附諸軍北屯長安故城【自盩厔趣長安故謂之北】至並聽教【並至所期之地聽教令】延安上郡雕隂皆請降於淵丙子淵引軍西行【自馮翊西行降戶江翻】所過離宫園苑皆罷之出宫女還其親屬冬十月辛巳淵至長安營於春明門之西北【春明門長安城東面三門之中門也】諸軍皆集合二十餘萬淵命各依壁壘毋得入村落侵暴屢遣使至城下諭衛文昇等以欲尊隋之意不報辛卯命諸軍進圍城甲午淵遷館於安興坊【安興坊蓋在安興門外雍錄長安城東面三門通化春明安興帥讀曰率】巴陵校尉鄱陽董景珍雷世猛旅帥鄭文秀許玄徹<br />
<br />
  萬瓚徐德基郭華沔陽張繡等謀據郡叛隋【隋志巴陵郡梁置巴州平陳改曰岳州大業初改曰羅州尋改為郡煬帝改大都督為校尉帥都督為旅帥沔陽郡後置復州大業初改曰沔州尋改為郡校戶教翻帥所類翻下同瓚藏旱翻沔彌兖翻】推景珍為主景珍曰吾素寒賤不為衆所服羅川令蕭銑梁室之後【按隋書帝紀羅川縣即巴陵郡之羅縣銑梁宣帝曾孫巖之孫】寛仁大度請奉之以從衆望乃遣使報銑【使疏吏翻】銑喜從之聲言討賊召募得數千人銑巖之孫也【蕭巖奔陳見開皇八年見殺見九年】會潁川賊帥沈柳生寇羅川【煬帝改許州為潁川郡】銑與戰不利因謂其衆曰今天下皆叛隋政不行巴陵豪傑起兵欲奉吾為主若從其請以號令江南可以中興梁祚以此召柳生亦當從吾矣衆皆悦聽命乃自稱梁公改隋服色旗幟皆如梁舊柳生即帥衆歸之以柳生為車騎大將軍起兵五日遠近歸附者至數萬人遂帥衆向巴陵景珍遣徐德基帥郡中豪傑數百人出迎【幟昌志翻帥讀曰率騎奇寄翻】未及見銑柳生與其黨謀曰我先奉梁公勲居第一今巴陵諸將皆位高兵多我若入城返出其下不如殺德基質其首領【將即亮翻質音致】獨挾梁公進取郡城則無出我右者矣遂殺德基入白銑銑大驚曰今欲撥亂反正忽自相殺吾不能為若主矣因步出軍門柳生大懼伏地請罪銑責而赦之陳兵入城景珍言於銑曰徐德基建義功臣而柳生無故擅殺之此而不誅何以為政且柳生為盜日久今雖從義凶悖不移【悖蒲妹翻又蒲没翻】共處一城【處昌呂翻】勢必為變失今不取後悔無及銑又從之景珍收柳生斬之其徒皆潰去丙申銑築壇燔燎自稱梁王改元鳴鳳 壬寅王世充夜度洛水營於黑石明日分兵守營自將精兵陳於洛北李密聞之引兵度洛逆戰密兵大敗柴孝和溺死密帥麾下精騎度洛南【將即亮翻陳讀曰陣帥讀曰率】餘衆東走月城【月城蓋臨洛水築偃月城與倉城相應】世充追圍之密自洛南策馬直趣黑石【趣七喻翻】營中懼連舉六烽世充釋月城之圍狼狽自救密還與戰大破之斬首三千餘級 甲辰李淵命諸軍攻城約毋得犯七廟及代王宗室違者夷三族孫華中流矢卒【中竹仲翻】十一月丙辰軍頭雷永吉先登 【考異曰唐高祖實錄作雷紹今從創業注】遂克長安代王在東宫左右奔散唯侍讀姚思廉侍側軍士將登殿思廉厲聲訶之曰唐公舉義兵匡帝室卿等毋得無禮衆皆愕然布立庭下【訶虎何翻愕五各翻】淵迎王於東宫遷居大興殿後【大興殿隋宫正殿也未即尊位故居殿後】聽思廉扶王至順陽閤下泣拜而去思廉察之子也【姚察事陳以文義稱】淵還舍於長樂宫【樂音洛】與民約法十二條悉除隋苛禁淵之起兵也留守官發其墳墓毁其五廟【隋制諸公立五廟】至是衛文昇已卒戊午執隂世師骨儀等數以貪婪苛酷且拒義師俱斬之【卒子恤翻數所具翻又所主翻按隋書稱隂世師少有節槩性忠厚多武藝骨儀性剛鯁有不可奪之志于時朝政浸亂濁貨公行天下士大夫莫不變節儀獨厲志守常介然獨立如此則皆隋之良也唐公特以其發墳墓毁家廟拒守不下而誅之數以貪婪苛酷非其罪也觀通鑑所書可謂微而顯矣婪盧含翻 考異曰隋書北史衛玄傳皆曰城陷歸于家義寧中卒按文昇與二人俱為留守官不容獨免今從唐本紀】死者十餘人餘無所問馬邑郡丞三原李靖素與淵有隙【隋志三原縣屬京兆郡煬帝改朔州為馬邑郡 考異曰柳芳唐歷及唐書靖傳云高祖擊突厥於塞外靖察高祖知有四方之志因自鎻上變將詣江都至長安道塞不通而止按太宗謀起兵高祖尚未知知之猶不從當擊突厥之時未有異志靖何從察知之又上變當乘驛取疾何為自鎻也今依靖行狀云昔在隋朝曾經忤旨及兹城䧟高祖追責舊言公忼慨直論特蒙宥釋但行狀題云魏徵撰非也按徵以貞觀十七年卒靖三十三年乃卒蓋後人為之託徵名又叙靖事極怪誕無取唯此可為據耳】淵入城將斬之靖大呼曰【呼火故翻】公興義兵欲平暴亂乃以私怨殺壯士乎世民為之固請【為于偽翻】乃捨之世民因召置幕府靖少負志氣有文武才畧其舅韓擒虎每撫之曰可與言將帥之畧者獨此子耳【少詩照翻將即亮翻帥所類翻】 王世充自洛北之敗堅壁不出越王侗遣使勞之【侗他紅翻又音同使疏吏翻勞力到翻】世充慙懼請戰於密丙辰世充與密夾石子河而陳密布陳南北十餘里【陳讀曰陣】翟讓先與世充戰不利而退世充逐之王伯當裴仁基從旁横斷其後密勒中軍擊之世充大敗西走 【考異曰前已有丙辰戊午欲各叙西京東都事使不相亂故重出按通鑑下文書戊午殺翟讓考異於此兼言之】翟讓司馬王儒信勸讓自為大冢宰總領衆務以奪密權讓不從讓兄柱國榮陽公弘【考異曰河洛記作洪今從蒲山公傳】粗愚人也謂讓曰天子汝當自為奈何與人汝不為者我當為之讓但大笑不以為意密聞而惡之【惡乃路翻】總管崔世樞自鄢陵初附於密【鄢陵縣隋屬潁川郡鄢謁晚翻又於建翻又音偃】讓囚之私府責其貨世樞營求未辦遽欲加刑讓召元帥府記室邢義期博逡巡未就杖之八十【帥所類翻逡七旬翻】讓謂左長史房彦藻曰君前破汝南【長知兩翻煬帝改蔡州為汝南郡】大得寶貨獨與魏公全不與我魏公我之所立事未可知彦藻懼以狀告密因與左司馬鄭頲共說密曰讓貪愎不仁【頲他鼎翻說式芮翻愎符逼翻】有無君之心宜早圖之密曰今安危未定遽相誅殺何以示遠頲曰毒虵螫手壯士解腕【螫音釋腕烏貫翻】所全者大故也彼先得志悔無所及密乃從之置酒召讓戊午讓與兄弘及兄子司徒府長史摩侯同詣密密與讓弘裴仁基郝孝德共坐單雄信等皆立侍【長知兩翻郝呼各翻單慈淺翻 考異曰河洛記云密讓讓兄子摩侯王儒信同榻而坐今從蒲山公傳】房彦藻鄭頲往來檢校密曰今日與達官飲不須多人【達官猶言顯官也】左右止留給使而已密左右皆引去讓左右猶在彦藻白密曰今方為樂【樂音洛】天時甚寒司徒左右請給酒食密曰聽司徒進止讓曰甚佳乃引讓左右盡出獨密下壯士蔡建德持刀立侍食未進密出良弓與讓習射讓方引滿建德自後斫之踣於牀前【踣蒲北翻】聲若牛吼并弘摩侯儒信皆殺之徐世勣走出門者斫之傷頸王伯當遥訶止之單雄信叩頭請命密釋之左右驚擾莫知所為密大言曰與君等同起義兵本除暴亂司徒專行暴虐陵辱羣僚無復上下今所誅止其一家諸君無預也命扶徐世勣置幕下親為傅瘡【為于偽翻】讓麾下欲散密使單雄信前往宣慰密尋獨騎入其營【獨騎猶言單騎也騎奇寄翻】歷加撫諭令世勣雄信伯當分領其衆中外遂定讓殘忍摩侯猜忌儒信貪縱故死之日所部無哀之者然密之將佐始有自疑之心矣始王世充知讓與密必不久睦冀其相圖得從而乘之及聞讓死大失望歎曰李密天資明决為龍為虵固不可測也 壬戌李淵備灋駕迎代王即皇帝位於天興殿【天興殿當作大興殿】時年十三大赦改元【改元義寧】遥尊煬帝為太上皇甲子淵自長樂宫入長安以淵為假黃鉞使持節大都督内外諸軍事尚書令大丞相進封唐王【樂音洛使疏吏翻令力定翻相息亮翻】以武德殿為丞相府改教稱令日於䖍化門視事【䖍化門在大興殿前東偏】乙丑榆林靈武平凉安定諸郡皆遣使請命【使疏吏翻】丙寅詔軍國機務事無大小文武設官位無貴賤憲章賞罰咸歸相府唯郊祀天地四時禘祫奏聞【祫戶夾翻】置丞相府官屬 【考異曰唐帝紀在十二月癸未今從創業注】以裴寂為長史劉文靜為司馬何潘仁使李綱入見【長知兩翻見賢遍翻】淵留之以為丞相府司錄【錄者總錄一府之事隋自文帝受禪後不復有丞相府亦無官屬唐公輔政位絶羣后凡官屬皆復特置之】專掌選事【選宣戀翻】又以前考功郎中竇威為司錄參軍使定禮儀威熾之子也【竇熾隋初三公】淵傾府庫以賜勲人國用不足右光祿大夫劉世龍獻策【隋散職左右光祿從二品】以為今義師數萬並在京師樵蘇貴而布帛賤請伐六街及苑中樹為樵【長安城中六街苑城包漢故都抵渭水】以易布帛可得數十萬匹淵從之己巳以李建成為唐世子李世民為京兆尹秦公李元吉為齊公 河南諸郡盡附李密唯滎陽太守楊汪尚為隋守【慶河間王弘之子弘高祖從祖弟也煬帝改宋州為梁郡郇音荀為于偽翻下同】密以書招慶為陳利害且曰王之家世本住山東本姓郭氏乃非楊族芝焚蕙歎事不同此初慶祖父元孫早孤隨母郭氏養於舅族及武元帝從周文起兵關中【楊忠謚武元皇帝】元孫在鄴恐為高氏所誅【北齊高氏】冒姓郭氏故密云然慶得書惶恐即以郡降密【降戶江翻下同】復姓郭氏 十二月癸未追謚唐王淵大父襄公為景王考仁公為元王夫人竇氏為穆妃【襄公虎仁公昞竇氏毅之女是為太穆皇后謚神至翻】 薛舉遣其子仁果寇扶風唐弼據汧源拒之【汧源縣隋屬扶風郡汧苦堅翻】舉遣使招弼弼乃殺李弘芝請降於舉【唐弼立李弘芝見一百八十二卷大業十年使疏吏翻降戶江翻】仁果乘其無備襲破之悉并其衆弼以數百騎走詣扶風請降扶風太守竇璡殺之舉勢益張【守式又翻張知亮翻】衆號三十萬謀取長安聞丞相淵已定長安遂圍扶風淵使李世民將兵擊之【將即亮翻】又使姜謩竇軌俱出散關【大散關在扶風郡陳倉縣西南散悉亶翻】安撫隴右左光祿大夫李孝恭招慰山南府戶曹張道源招慰山東【道源丞相府戶曹也】孝恭淵之從父兄子也【從才用翻】癸巳世民擊薛仁果於扶風大破之追奔至壠坻而還【坻丁禮翻又丁計翻還從宣翻又如字】薛舉大懼問其羣臣曰自古天子有降事乎黃門侍郎錢唐禇亮曰【隋志錢唐縣屬餘杭郡降戶江翻下同】趙佗歸漢【事見漢高祖文帝紀佗徒何翻】劉禪仕晉【事見魏紀晉紀】近世蕭琮至今猶貴【謂蕭氏子弟也】轉禍為福自古有之衛尉卿郝瑗趨進曰陛下失問禇亮之言又何悖也【郝呼各翻瑗于眷翻悖蒲妹翻】昔漢高祖屢經奔敗【見本紀】蜀先主亟亡妻子【見漢獻帝紀亟去吏翻】卒成大業【卒子恤翻】陛下奈何以一戰不利遽為亡國之計乎舉亦悔之曰聊以此試君等耳乃厚賞瑗引為謀主 乙未平凉留守張隆丁酉河池太守蕭瑀及扶風漢陽郡相繼來降【煬帝改成州為漢陽郡武都仇池之地也守式又翻降戶江翻】以竇璡為工部尚書燕國公【璡則鄰翻燕因肩翻】蕭瑀為禮部尚書宋國公【瑀音禹】 姜謩竇軌進至長道【元魏分上祿置長道縣隋屬漢陽郡】為薛舉所敗引還【敗補邁翻還從宣翻】淵使通議大夫醴泉劉世讓安集唐弼餘黨【通議大夫隋散職從四品隋志醴泉縣屬京兆郡後魏之寧夷縣開皇十八年改名】與舉相遇戰敗為舉所虜 李孝恭擊破朱粲諸將請盡殺其俘【將即亮翻】孝恭曰不可自是以往誰復肯降矣【復扶又翻】於是自金川出巴蜀檄書所至降附者三十餘州【隋志金川縣帶西城郡漢西城縣地梁初曰上廉後曰吉陽西魏改曰吉安後周以西城入焉大業三年改曰金川以其地產金也自金川出巴中自巴中則至蜀矣】 屈突通與劉文靜相持月餘通復使桑顯和夜襲其營【屈居勿翻復扶又翻】文靜與左光祿大夫段志玄悉力苦戰顯和敗走盡俘其衆通勢益蹙或說通降通泣曰吾歷事兩主【兩主謂文帝煬帝說式芮翻降戶江翻下同】恩顧甚厚食人之祿而違其難【難乃旦翻】吾不為也每自摩其頸曰要當為國家受一刀勞勉將士未嘗不流涕【為于偽翻勞力到翻將即亮翻】人亦以此懷之丞相淵遣其家僮召之通立斬之及聞長安不守家屬悉為淵所虜乃留顯和鎮潼關引兵東出將趣洛陽【趣七喻翻又逡須翻】通適去顯和即以城降文靜【降戶江翻】文靜遣竇琮等將輕騎與顯和追之及於稠桑【虢州湖城縣有稠桑驛琮徂宗翻將即亮翻又音如字領也騎奇寄翻】通結陳自固【陳讀曰陣】竇琮遣通子壽往諭之通罵曰此賊何來昔與汝為父子今與汝為仇讎命左右射之【射而亦翻】顯和謂其衆曰今京城巳陷汝輩皆關中人去欲何之衆皆釋仗而降通知不免下馬東南向再拜號哭曰【號戶刀翻】臣力屈至此非敢負國天地神祇實知之【祇其支翻】軍人執通送長安 【考異曰革命記高祖令諸將擊通通走出潼關仍令通子壽隨軍喚父至稠桑追及之夀告通云天下今既喪亡相王舉義兵平定禍亂大人須轉禍為福以自保全單馬輕身將欲何往通叱夀云此賊何由可耐引弓射之夀招喚通兵士並悉放仗來降夀乃馳走抱通請大人屈節歸義通遂回首東南向泣號哭口稱至尊臣力屈以至於此非臣敢虧名節違背國恩然始收涙赴軍以見唐王今從唐書唐裴矩傳屈突通敗問至江都煬帝問矩方畧矩曰太原有變京畿不靜遥為處分恐失事機唯鑾輿早還方可平定按隋失天下皆因矩諂諛所致豈敢輒勸帝西還蓋矩經事唐朝其子孫及史官附益此語欲蓋其惡耳今所不取】淵以為兵部尚書賜爵蔣公【蔣古國名】兼秦公元帥府長史【長知兩翻】淵遣通至河東城下招諭堯君素君素見通歔欷不自勝【歔音虚欷音希又許既翻勝音升】通亦泣下霑衿因謂君素曰吾軍已敗義旗所指莫不響應事勢如此卿宜早降君素曰公為國大臣主上委公以關中代王付公以社稷奈何負國生降乃更為人作說客邪【降戶江翻為于偽翻說輸芮翻邪音耶】公所乘馬即代王所賜也公何面目乘之哉通曰吁君素我力屈而來君素曰方今力猶未屈何用多言通慙而退 東都米斗三錢人餓死者什二三庚子王世充軍士有亡降李密者密問世充軍中何<br />
<br />
  所為軍士曰比見益募兵再饗將士不知其故【比毗至翻】密謂裴仁基曰吾幾落奴度中【幾居依翻】光祿知之乎吾久不出兵世充芻糧將竭求戰不得故募兵饗士欲乘月晦以襲倉城耳宜速備之乃命平原公郝孝德琅邪公王伯當齊郡公孟讓勒兵分屯倉城之側以待之【郝呼各翻邪音耶】其夕三鼔世充兵果至伯當先遇之與戰不利世充兵即陵城總管魯儒拒却之伯當更收兵擊之世充大敗斬其驍將費青奴【驍堅堯翻將即亮翻費扶沸翻】士卒戰溺死者千餘人世充屢與密戰不勝 【考異曰蒲山公傳云自洛北敗至此七十餘戰河洛記云四十餘戰再三失利今但云屢與密戰】越王侗遣使勞之【侗他紅翻使疏吏翻勞力到翻】世充訴以兵少數戰疲弊【少詩沼翻數所角翻】侗以兵七萬益之劉文靜等引兵東略地取弘農郡遂定新安以西【隋志】<br />
<br />
  【河南郡陜縣舊置弘農郡大業初置弘農宫别自有弘農郡領弘農盧氏長泉朱陽等縣新安縣亦屬河南郡其地在陜東則取弘農郡併弘農宫取之矣】 甲辰李淵遣雲陽令詹俊武功縣正李仲衮徇巴蜀下之【隋志雲陽武功二縣皆屬京兆郡煬帝改縣尉為縣正詹姓也周有詹父楚有詹尹 考異曰創業注十一月甲子遣使慰諭巴蜀實錄在十二月甲辰唐歷在十二月丙午未知創業注所云者即俊等邪為别使也今從實錄】 乙巳方與賊帥張善安襲陷廬江郡【隋志方與縣屬彭城郡煬帝改廬州為廬江郡方與音房豫帥所類翻】因度江歸林士弘於豫章士弘疑之營於南塘上【煬帝改洪州為豫章郡水經注南昌縣南塘本通大江漢永元中太守張躬築塘以通南路大江南江也】善安恨之襲破士弘焚其郛郭而去士弘徙居南康蕭銑遣其將蘇胡兒襲豫章克之【將即亮翻】士弘退保餘干【煬帝改䖍州為南康郡餘干縣屬鄱陽郡】<br />
<br />
  資治通鑑卷一百八十四  <br>
   </div> 

<script src="/search/ajaxskft.js"> </script>
 <div class="clear"></div>
<br>
<br>
 <!-- a.d-->

 <!--
<div class="info_share">
</div> 
-->
 <!--info_share--></div>   <!-- end info_content-->
  </div> <!-- end l-->

<div class="r">   <!--r-->



<div class="sidebar"  style="margin-bottom:2px;">

 
<div class="sidebar_title">工具类大全</div>
<div class="sidebar_info">
<strong><a href="http://www.guoxuedashi.com/lsditu/" target="_blank">历史地图</a></strong>  
<a href="http://www.880114.com/" target="_blank">英语宝典</a>  
<a href="http://www.guoxuedashi.com/13jing/" target="_blank">十三经检索</a> 
<br><strong><a href="http://www.guoxuedashi.com/gjtsjc/" target="_blank">古今图书集成</a></strong> 
<a href="http://www.guoxuedashi.com/duilian/" target="_blank">对联大全</a> <strong><a href="http://www.guoxuedashi.com/xiangxingzi/" target="_blank">象形文字典</a></strong> 

<br><a href="http://www.guoxuedashi.com/zixing/yanbian/">字形演变</a>  <strong><a href="http://www.guoxuemi.com/hafo/" target="_blank">哈佛燕京中文善本特藏</a></strong>
<br><strong><a href="http://www.guoxuedashi.com/csfz/" target="_blank">丛书&方志检索器</a></strong> <a href="http://www.guoxuedashi.com/yqjyy/" target="_blank">一切经音义</a>  

<br><strong><a href="http://www.guoxuedashi.com/jiapu/" target="_blank">家谱族谱查询</a></strong>  <strong><a href="http://shufa.guoxuedashi.com/sfzitie/" target="_blank">书法字帖欣赏</a></strong> 
<br>

</div>
</div>


<div class="sidebar" style="margin-bottom:0px;">

<font style="font-size:22px;line-height:32px">QQ交流群9:489193090</font>


<div class="sidebar_title">手机APP 扫描或点击</div>
<div class="sidebar_info">
<table>
<tr>
	<td width=160><a href="http://m.guoxuedashi.com/app/" target="_blank"><img src="/img/gxds-sj.png" width="140"  border="0" alt="国学大师手机版"></a></td>
	<td>
<a href="http://www.guoxuedashi.com/download/" target="_blank">app软件下载专区</a><br>
<a href="http://www.guoxuedashi.com/download/gxds.php" target="_blank">《国学大师》下载</a><br>
<a href="http://www.guoxuedashi.com/download/kxzd.php" target="_blank">《汉字宝典》下载</a><br>
<a href="http://www.guoxuedashi.com/download/scqbd.php" target="_blank">《诗词曲宝典》下载</a><br>
<a href="http://www.guoxuedashi.com/SiKuQuanShu/skqs.php" target="_blank">《四库全书》下载</a><br>
</td>
</tr>
</table>

</div>
</div>


<div class="sidebar2">
<center>


</center>
</div>

<div class="sidebar"  style="margin-bottom:2px;">
<div class="sidebar_title">网站使用教程</div>
<div class="sidebar_info">
<a href="http://www.guoxuedashi.com/help/gjsearch.php" target="_blank">如何在国学大师网下载古籍?</a><br>
<a href="http://www.guoxuedashi.com/zidian/bujian/bjjc.php" target="_blank">如何使用部件查字法快速查字?</a><br>
<a href="http://www.guoxuedashi.com/search/sjc.php" target="_blank">如何在指定的书籍中全文检索?</a><br>
<a href="http://www.guoxuedashi.com/search/skjc.php" target="_blank">如何找到一句话在《四库全书》哪一页?</a><br>
</div>
</div>


<div class="sidebar">
<div class="sidebar_title">热门书籍</div>
<div class="sidebar_info">
<a href="/so.php?sokey=%E8%B5%84%E6%B2%BB%E9%80%9A%E9%89%B4&kt=1">资治通鉴</a> <a href="/24shi/"><strong>二十四史</strong></a>&nbsp; <a href="/a2694/">野史</a>&nbsp; <a href="/SiKuQuanShu/"><strong>四库全书</strong></a>&nbsp;<a href="http://www.guoxuedashi.com/SiKuQuanShu/fanti/">繁体</a>
<br><a href="/so.php?sokey=%E7%BA%A2%E6%A5%BC%E6%A2%A6&kt=1">红楼梦</a> <a href="/a/1858x/">三国演义</a> <a href="/a/1038k/">水浒传</a> <a href="/a/1046t/">西游记</a> <a href="/a/1914o/">封神演义</a>
<br>
<a href="http://www.guoxuedashi.com/so.php?sokeygx=%E4%B8%87%E6%9C%89%E6%96%87%E5%BA%93&submit=&kt=1">万有文库</a> <a href="/a/780t/">古文观止</a> <a href="/a/1024l/">文心雕龙</a> <a href="/a/1704n/">全唐诗</a> <a href="/a/1705h/">全宋词</a>
<br><a href="http://www.guoxuedashi.com/so.php?sokeygx=%E7%99%BE%E8%A1%B2%E6%9C%AC%E4%BA%8C%E5%8D%81%E5%9B%9B%E5%8F%B2&submit=&kt=1"><strong>百衲本二十四史</strong></a>  <a href="http://www.guoxuedashi.com/so.php?sokeygx=%E5%8F%A4%E4%BB%8A%E5%9B%BE%E4%B9%A6%E9%9B%86%E6%88%90&submit=&kt=1"><strong>古今图书集成</strong></a>
<br>

<a href="http://www.guoxuedashi.com/so.php?sokeygx=%E4%B8%9B%E4%B9%A6%E9%9B%86%E6%88%90&submit=&kt=1">丛书集成</a> 
<a href="http://www.guoxuedashi.com/so.php?sokeygx=%E5%9B%9B%E9%83%A8%E4%B8%9B%E5%88%8A&submit=&kt=1"><strong>四部丛刊</strong></a>  
<a href="http://www.guoxuedashi.com/so.php?sokeygx=%E8%AF%B4%E6%96%87%E8%A7%A3%E5%AD%97&submit=&kt=1">說文解字</a> <a href="http://www.guoxuedashi.com/so.php?sokeygx=%E5%85%A8%E4%B8%8A%E5%8F%A4&submit=&kt=1">三国六朝文</a>
<br><a href="http://www.guoxuedashi.com/so.php?sokeytm=%E6%97%A5%E6%9C%AC%E5%86%85%E9%98%81%E6%96%87%E5%BA%93&submit=&kt=1"><strong>日本内阁文库</strong></a> <a href="http://www.guoxuedashi.com/so.php?sokeytm=%E5%9B%BD%E5%9B%BE%E6%96%B9%E5%BF%97%E5%90%88%E9%9B%86&ka=100&submit=">国图方志合集</a> <a href="http://www.guoxuedashi.com/so.php?sokeytm=%E5%90%84%E5%9C%B0%E6%96%B9%E5%BF%97&submit=&kt=1"><strong>各地方志</strong></a>

</div>
</div>


<div class="sidebar2">
<center>

</center>
</div>
<div class="sidebar greenbar">
<div class="sidebar_title green">四库全书</div>
<div class="sidebar_info">

《四库全书》是中国古代最大的丛书,编撰于乾隆年间,由纪昀等360多位高官、学者编撰,3800多人抄写,费时十三年编成。丛书分经、史、子、集四部,故名四库。共有3500多种书,7.9万卷,3.6万册,约8亿字,基本上囊括了古代所有图书,故称“全书”。<a href="http://www.guoxuedashi.com/SiKuQuanShu/">详细>>
</a>

</div> 
</div>

</div>  <!--end r-->

</div>
<!-- 内容区END --> 

<!-- 页脚开始 -->
<div class="shh">

</div>

<div class="w1180" style="margin-top:8px;">
<center><script src="http://www.guoxuedashi.com/img/plus.php?id=3"></script></center>
</div>
<div class="w1180 foot">
<a href="/b/thanks.php">特别致谢</a> | <a href="javascript:window.external.AddFavorite(document.location.href,document.title);">收藏本站</a> | <a href="#">欢迎投稿</a> | <a href="http://www.guoxuedashi.com/forum/">意见建议</a> | <a href="http://www.guoxuemi.com/">国学迷</a> | <a href="http://www.shuowen.net/">说文网</a><script language="javascript" type="text/javascript" src="https://js.users.51.la/17753172.js"></script><br />
  Copyright &copy; 国学大师 古典图书集成 All Rights Reserved.<br>
  
  <span style="font-size:14px">免责声明:本站非营利性站点,以方便网友为主,仅供学习研究。<br>内容由热心网友提供和网上收集,不保留版权。若侵犯了您的权益,来信即刪。scp168@qq.com</span>
  <br />
ICP证:<a href="http://www.beian.miit.gov.cn/" target="_blank">鲁ICP备19060063号</a></div>
<!-- 页脚END --> 
<script src="http://www.guoxuedashi.com/img/plus.php?id=22"></script>
<script src="http://www.guoxuedashi.com/img/tongji.js"></script>

</body>
</html>
