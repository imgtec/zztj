<!DOCTYPE html PUBLIC "-//W3C//DTD XHTML 1.0 Transitional//EN" "http://www.w3.org/TR/xhtml1/DTD/xhtml1-transitional.dtd">
<html xmlns="http://www.w3.org/1999/xhtml">
<head>
<meta http-equiv="Content-Type" content="text/html; charset=utf-8" />
<meta http-equiv="X-UA-Compatible" content="IE=Edge,chrome=1">
<title>資治通鑒_57-資治通鑑卷五十六_57-資治通鑑卷五十六</title>
<meta name="Keywords" content="資治通鑒_57-資治通鑑卷五十六_57-資治通鑑卷五十六">
<meta name="Description" content="資治通鑒_57-資治通鑑卷五十六_57-資治通鑑卷五十六">
<meta http-equiv="Cache-Control" content="no-transform" />
<meta http-equiv="Cache-Control" content="no-siteapp" />
<link href="/img/style.css" rel="stylesheet" type="text/css" />
<script src="/img/m.js?2020"></script> 
</head>
<body>
 <div class="ClassNavi">
<a  href="/24shi/">二十四史</a> | <a href="/SiKuQuanShu/">四库全书</a> | <a href="http://www.guoxuedashi.com/gjtsjc/"><font  color="#FF0000">古今图书集成</font></a> | <a href="/renwu/">历史人物</a> | <a href="/ShuoWenJieZi/"><font  color="#FF0000">说文解字</a></font> | <a href="/chengyu/">成语词典</a> | <a  target="_blank"  href="http://www.guoxuedashi.com/jgwhj/"><font  color="#FF0000">甲骨文合集</font></a> | <a href="/yzjwjc/"><font  color="#FF0000">殷周金文集成</font></a> | <a href="/xiangxingzi/"><font color="#0000FF">象形字典</font></a> | <a href="/13jing/"><font  color="#FF0000">十三经索引</font></a> | <a href="/zixing/"><font  color="#FF0000">字体转换器</font></a> | <a href="/zidian/xz/"><font color="#0000FF">篆书识别</font></a> | <a href="/jinfanyi/">近义反义词</a> | <a href="/duilian/">对联大全</a> | <a href="/jiapu/"><font  color="#0000FF">家谱族谱查询</font></a> | <a href="http://www.guoxuemi.com/hafo/" target="_blank" ><font color="#FF0000">哈佛古籍</font></a> 
</div>

 <!-- 头部导航开始 -->
<div class="w1180 head clearfix">
  <div class="head_logo l"><a title="国学大师官网" href="http://www.guoxuedashi.com" target="_blank"></a></div>
  <div class="head_sr l">
  <div id="head1">
  
  <a href="http://www.guoxuedashi.com/zidian/bujian/" target="_blank" ><img src="http://www.guoxuedashi.com/img/top1.gif" width="88" height="60" border="0" title="部件查字,支持20万汉字"></a>


<a href="http://www.guoxuedashi.com/help/yingpan.php" target="_blank"><img src="http://www.guoxuedashi.com/img/top230.gif" width="600" height="62" border="0" ></a>


  </div>
  <div id="head3"><a href="javascript:" onClick="javascript:window.external.AddFavorite(window.location.href,document.title);">添加收藏</a>
  <br><a href="/help/setie.php">搜索引擎</a>
  <br><a href="/help/zanzhu.php">赞助本站</a></div>
  <div id="head2">
 <a href="http://www.guoxuemi.com/" target="_blank"><img src="http://www.guoxuedashi.com/img/guoxuemi.gif" width="95" height="62" border="0" style="margin-left:2px;" title="国学迷"></a>
  

  </div>
</div>
  <div class="clear"></div>
  <div class="head_nav">
  <p><a href="/">首页</a> | <a href="/ShuKu/">国学书库</a> | <a href="/guji/">影印古籍</a> | <a href="/shici/">诗词宝典</a> | <a   href="/SiKuQuanShu/gxjx.php">精选</a> <b>|</b> <a href="/zidian/">汉语字典</a> | <a href="/hydcd/">汉语词典</a> | <a href="http://www.guoxuedashi.com/zidian/bujian/"><font  color="#CC0066">部件查字</font></a> | <a href="http://www.sfds.cn/"><font  color="#CC0066">书法大师</font></a> | <a href="/jgwhj/">甲骨文</a> <b>|</b> <a href="/b/4/"><font  color="#CC0066">解密</font></a> | <a href="/renwu/">历史人物</a> | <a href="/diangu/">历史典故</a> | <a href="/xingshi/">姓氏</a> | <a href="/minzu/">民族</a> <b>|</b> <a href="/mz/"><font  color="#CC0066">世界名著</font></a> | <a href="/download/">软件下载</a>
</p>
<p><a href="/b/"><font  color="#CC0066">历史</font></a> | <a href="http://skqs.guoxuedashi.com/" target="_blank">四库全书</a> |  <a href="http://www.guoxuedashi.com/search/" target="_blank"><font  color="#CC0066">全文检索</font></a> | <a href="http://www.guoxuedashi.com/shumu/">古籍书目</a> | <a   href="/24shi/">正史</a> <b>|</b> <a href="/chengyu/">成语词典</a> | <a href="/kangxi/" title="康熙字典">康熙字典</a> | <a href="/ShuoWenJieZi/">说文解字</a> | <a href="/zixing/yanbian/">字形演变</a> | <a href="/yzjwjc/">金 文</a> <b>|</b>  <a href="/shijian/nian-hao/">年号</a> | <a href="/diming/">历史地名</a> | <a href="/shijian/">历史事件</a> | <a href="/guanzhi/">官职</a> | <a href="/lishi/">知识</a> <b>|</b> <a href="/zhongyi/">中医中药</a> | <a href="http://www.guoxuedashi.com/forum/">留言反馈</a>
</p>
  </div>
</div>
<!-- 头部导航END --> 
<!-- 内容区开始 --> 
<div class="w1180 clearfix">
  <div class="info l">
   
<div class="clearfix" style="background:#f5faff;">
<script src='http://www.guoxuedashi.com/img/headersou.js'></script>

</div>
  <div class="info_tree"><a href="http://www.guoxuedashi.com">首页</a> > <a href="/SiKuQuanShu/fanti/">四库全书</a>
 > <h1>资治通鉴</h1> <!--         下载:【右键另存为】即可 --></div>
  <div class="info_content zj clearfix">
  
<div class="info_txt clearfix" id="show">
<center style="font-size:24px;">57-資治通鑑卷五十六</center>
    資治通鑑卷五十六  宋 司馬光 撰<br />
<br />
  胡三省 音註<br />
<br />
  漢紀四十八【起彊圉協洽盡重光大淵獻凡五年】<br />
<br />
  孝桓皇帝下<br />
<br />
  永康元年【是年六月始改元】春正月東羌先零圍祋祤掠雲陽【二縣皆屬左馮翊宋白曰耀州華原同官縣本漢祋祤縣地雲陽故城在今縣西北六十里零音憐祋音丁活翻又音丁外翻祤音詡】煎當諸種復反【種章勇翻復扶又翻下同】段熲擊之於鸞鳥【熲高迥翻鸞音雚鳥讀曰雀】大破之西羌遂定 夫餘王夫台寇玄菟【夫音扶莬同都翻】玄菟太守公孫域擊破之【守式又翻】 夏四月先零羌寇三輔攻没兩營【兩營京兆虎牙營扶風雍營零音憐】殺千餘人 五月壬子晦日有食之 陳蕃既免朝臣震栗莫敢復為黨人言者【復扶又翻朝直遥翻下同為于偽翻】賈彪曰吾不西行大禍不解【賈彪潁川定陵人自潁川至洛陽為西行】乃入雒陽說城門校尉竇武尚書魏郡霍諝等【說輸芮翻諝私呂翻】使訟之武上疏曰陛下即位以來未聞善政常侍黄門競作譎詐妄爵非人伏尋西京佞臣執政終喪天下【譎古穴翻喪息浪翻】今不慮前事之失復循覆車之軌臣恐二世之難【難乃旦翻】必將復及趙高之變不朝則夕【謂望夷宫之事也】近者姦臣牢脩造設黨議遂收前司隷校尉李膺等逮考連及數百人曠年拘録事無效驗【謂自去年興獄至今年事終無其實也校戶教翻】臣惟膺等忠心抗節志經王室此誠陛下稷卨伊呂之佐【卨古契字音息列翻】而虚為姦臣賊子之所誣枉天下寒心四海失望惟陛下留神澄省【澄清也省察也省悉井翻】時見理出【賢曰時謂即時也】以厭神鬼喁喁之心【喁魚恭翻】今臺閣近臣尚書朱㝢荀鯤劉祐魏朗劉矩尹勲等皆國之貞士朝之良佐【緄古本翻 考異曰武傳武上疏曰令臺閻近臣尚書令陳蕃僕射胡廣尚書朱㝢等按蕃廣時不為令僕故去之】尚書郎張陵媯皓【媯俱為翻姓譜媯帝舜之後】苑康【姓譜苑姓商武丁之子受封于苑因以為氏左傳齊有大夫苑何忌】楊喬邊韶【陳留風俗傳邊祖于宋平公子戍字子邊又左傳周有大夫邊伯】戴恢等文質彬彬明逹國典内外之職羣才並列而陛下委任近習專樹饕餮【饕吐刀翻餮他結翻】外典州郡内幹心膂宜以次貶黜案罪糾罰信任忠良平决臧否使邪正毁譽各得其所【否音鄙譽音余】寶愛天官唯善是授【天官言天命有德人君不可以私授】如此咎徵可消天應可待間者有嘉禾芝草黄龍之見【是年魏郡言嘉禾生巴郡言黄龍見見賢遍翻】夫瑞生必於嘉士福至實由善人在德為瑞無德為災陛下所行不合天意不宜稱慶書奏因以病上還城門校尉槐里侯印綬霍諝亦為表請【上時掌翻為于偽翻下同】帝意稍解使中常侍王甫就獄訊黨人范滂等皆三木囊頭暴於階下【賢曰三木頭及手足皆有械更以物蒙覆其頭也】甫以次辯詰曰卿等更相拔舉【更工衡翻】迭為唇齒其意如何滂曰仲尼之言見善如不及見惡如探湯【賢曰探湯喻去之疾也見論語探吐南翻】滂欲使善善同其清惡惡同其汙謂王政之所願聞不悟更以為黨古之修善自求多福今之修善身陷大戮身死之日願埋滂於首陽山側上不負皇天下不愧夷齊【賢曰伯夷叔齊餓死首陽山事見史記首陽山在雒陽東北杜佑曰偃師縣有首陽山】甫愍然為之改容乃得並解桎梏【鄭玄註周禮曰木在手曰桎在足曰梏桎之日翻梏工沃翻】李膺等又多引宦官子弟宦官懼請帝以天時宜赦六月庚申赦天下改元黨人二百餘人皆歸田里書名三府禁錮終身 【考異曰帝紀於去年冬書李膺等二百餘人受誣為黨人並坐下獄書名三府案陳蕃以訟李膺免即膺等下獄已在前後遇赦方得書名三府則帝紀所紀為兩無所用故去之又故書三府為王府劉攽曰當為三府】范滂往候霍諝而不謝或讓之滂曰昔叔向不見祁奚【晉范宣子囚叔向祁奚請而免之不見叔向而歸叔向亦不告免焉而朝】吾何謝焉滂南歸汝南南陽士大夫迎之者車數千兩【兩音亮】鄉人殷陶黄穆侍衛於旁應對賓客滂謂陶等曰今子相隨是重吾禍也遂遁還鄉里初詔書下舉鉤黨【賢曰鉤謂相連也下遐稼翻】郡國所奏相連及者多至百數唯平原相史弼獨無所上【上時掌翻】詔書前後迫切州郡髠笞掾史從事坐傳舍責曰【掾俞絹翻賢曰續漢志每州有從事史及諸曹掾史傳客舍也音知戀翻坐傳舍召弼而責余謂髠笞掾史句絶言詔書督迫州郡至於髠笞掾史青州從事則坐平原傳舍而責史弼也】詔書疾惡黨人【惡烏路翻】旨意懇惻青州六郡其五有黨平原何治而得獨無弼曰先王疆理天下【賢曰疆界也理正也】畫界分境水土異齊風俗不同【記王制曰凡居民財必因天地寒暖燥濕廣谷大川異制民生其間者異俗剛柔輕重遲速異齊齊才細翻前書曰凡民函五常之性而其剛柔緩急音聲不同繫水土之風氣故謂之風好惡取舍動靜無常隨君上之情欲故謂之俗】它郡自有平原自無胡可相比若承望上司誣陷良善淫刑濫罰以逞非理則平原之人戶可為黨相有死而已【相息亮翻】所不能也從事大怒即收郡僚職送獄【郡僚職謂郡諸曹掾史也】遂舉奏弼會黨禁中解弼以俸贖罪所脱者甚衆竇武所薦朱㝢沛人苑康勃海人楊喬會稽人【會工外翻】邊韶陳留人喬容儀偉麗數上言政事【數所角翻】帝愛其才貌欲妻以公主【妻七細翻】喬固辭不聽遂閉口不食七日而死 秋八月巴郡言黄龍見【見賢遍翻】初郡人欲就池浴見池水濁因戲相恐此中有黄龍語遂行民間太守欲以為美故上之【上時掌翻】郡吏傅堅諫曰此走卒戲言耳太守不聽 六月大水勃海溢 冬十月先零羌寇三輔【零音憐】張奐遣司馬尹端董卓拒擊大破之斬其酋豪首虜萬餘人【酋慈由翻】三州清定【時奐督幽并涼三州】奐論功當封以不事宦官故不果封唯賜錢二十萬除家一人為郎奐辭不受請徙屬弘農舊制邊人不得内徙詔以奐有功特許之【奐燉煌淵泉人】拜董卓為郎中卓隴西人性粗猛有謀羌胡畏之【董卓事始此】十二月壬申復癭陶王悝為勃海王【悝貶事見上卷延熹八年癭於】<br />
<br />
  【郢翻悝苦回翻】 丁丑帝崩于德陽前殿【年三十六】戊寅尊皇后曰皇太后太后臨朝初竇后既立御見甚稀【見賢遍翻】唯采女田聖等有寵后素忌忍帝梓宫尚在前殿遂殺田聖城門校尉竇武議立嗣召侍御史河間劉鯈【鯈式竹翻】問以國中宗室之賢者鯈稱解瀆亭侯宏【賢曰解瀆亭在今定州義豐縣東北杜佑曰義豐漢之安國縣也】宏者河間孝王之曾孫也祖淑父萇世封解瀆亭侯武乃入白太后定策禁中以鯈守光祿大夫與中常侍曹節並持節將中黄門虎賁羽林千人【將即亮翻】奉迎宏時年十二 【考異曰范書云即帝位年十三袁紀初立為嗣詔書云年十有二建寧二年誅黨人時云年十四袁紀是也】<br />
<br />
  孝靈皇帝上之上【諱宏諡法亂而不損曰靈伏侯古今注宏之字曰大】<br />
<br />
  建寧元年春正月壬午以城門校尉竇武為大將軍【考異曰袁紀延熹九年四月戊寅特進竇武為大將軍武移病固讓至于數十不許范書在今年正月壬午武傳為大將軍亦在迎立靈帝後今從之】前太尉陳蕃為太傅 【考異曰帝紀拜蕃太傅在即位後傳在前緣有蕃責尚書等語故知從傳是也】與武及司徒胡廣參録尚書事【三人謂之參】時新遭大喪國嗣未立諸尚書畏懼多託病不朝陳蕃移書責之曰古人立節事亡如存【賢曰言人主雖亡法度尚在當行之與不亡時同故曰如存余謂事死如事生事亡如事存中庸之文言人主雖死亡事之如生存也】今帝祚未立政事日蹙諸君奈何委荼蓼之苦息偃在牀【詩國風曰誰謂荼苦其甘如薺周頌曰未堪家多難予又集于蓼小雅曰或息偃在牀】於義安乎諸尚書惶怖【怖普布翻】皆起視事 己亥解瀆亭侯至夏門亭使竇武持節以王青蓋車迎入殿中庚子即皇帝位改元 二月辛酉葬孝桓皇帝于宣陵【賢曰宣陵在雒陽東南三卜里】廟曰威宗 辛未赦天下 初護羌校尉段熲既定西羌【去年熲定西羌】而東羌先零等種猶未服度遼將軍皇甫規中郎將張奐招之連年既降又叛【桓帝延熹四年皇甫規招降東羌六年規薦張奐至永康元年七年之間羌之叛服無常降戶江翻】桓帝詔問熲曰先零東羌造惡反逆而皇甫規張奐各擁強衆不時輯定欲令熲移兵東討未識其宜可參思術畧熲上言曰臣伏見先零東羌雖數叛逆【數所角翻】而降於皇甫規者已二萬許落善惡既分餘寇無幾今張奐躊躇久不進者【躊躇猶豫也又住足也】當慮外離内合兵往必驚且自冬踐春屯結不散人畜疲羸有自亡之埶欲更招降坐制強敵耳臣以為狼子野心難以恩納埶窮雖服兵去復動【復扶又翻】唯當長矛挾脅白刃加頸耳計東種所餘三萬餘落【種章勇翻下同】近居塞内路無險折非有燕齊秦趙從横之埶【從子容翻】而久亂并凉累侵三輔西河上郡已各内徙【事見五十二卷順帝永和五年】安定北地復至單危【復扶又翻下同】自雲中五原西至漢陽二千餘里匈奴諸羌並擅其地是為癰疽伏疾留滯脇下如不加誅轉就滋大若以騎五千步萬人車三千兩【兩音亮】三冬二夏足以破定無慮用費為錢五十四億【賢曰無慮都凡也毛晃曰總計曰無慮猶言多少如是無疑也】如此則可令羣羌破盡匈奴長服内徙郡縣得反本土伏計永初中諸羌反叛十有四年用二百四十億【事見五十卷安帝元初五年】永和之末復經七年用八十餘億【事見五十三卷冲帝永嘉元年】費耗若此猶不誅盡餘孽復起于兹作害今不暫疲民則永寧無期臣庶竭駑劣伏待節度【駑音奴】帝許之悉聽如所上【上時掌翻】熲於是將兵萬餘人齎十五日糧從彭陽直指高平【賢曰彭陽高平並縣名屬安定郡彭陽縣即今原州彭原縣也高平縣今原州也】與先零諸種戰於逢義山【賢曰山在今原州平高縣杜佑曰平高縣即漢之高平也】虜兵盛熲衆皆恐熲乃令軍中長鏃利刃【范書段熲傳作張鏃利刃】長矛三重【重直龍翻】挾以強弩列輕騎為左右翼謂將士曰今去家數千里進則事成走必盡死努力共功名因大呼【呼火故翻】衆皆應聲騰赴馳騎於傍突而擊之虜衆大潰斬首八千餘級太后賜詔書褒美曰須東羌盡定當并録功勤今且賜熲錢二十萬以家一人為郎中敇中藏府調金錢綵物增助軍費【百官志中藏府今屬少府掌中幣帛金銀諸貨物調徒弔翻藏沮浪翻】拜熲破羌將軍閏月甲午追尊皇祖為孝元皇【沈約曰孝元皇謚法所不載今按周公謚】<br />
<br />
  【法能思辨衆曰元行義說民曰元主義行德曰元靖民則法曰元】夫人夏氏為孝元后【夏戶雅翻】考為孝仁皇【謚法貴賢親親曰仁】尊帝母董氏為慎園貴人【皇祖解瀆亭侯淑也皇考侯萇也賢曰慎園在今瀛州樂夀縣東南俗呼為二皇陵】 夏四月戊辰太尉周景薨司空宣酆免以長樂衛尉王暢為司空【樂音洛】 五月丁未朔日有食之 以太中大夫劉矩為太尉 六月京師大水 癸巳録定策功封竇武為聞喜侯武子機為渭陽侯【考兩漢志無渭陽縣蓋因舅氏之親而為封國之名】兄子紹為鄠侯【鄠音戶】靖為西鄉侯中常侍曹節為長安鄉侯侯者凡十一人涿郡盧植上書說武曰【說輸芮翻】足下之於漢朝猶旦奭之在周室建立聖主四海有繫論者以為吾子之功於斯為重今同宗相後披圖案牒以次建之何勲之有【自和帝無嗣安帝以肅宗之孫入立冲質短祚桓帝以肅宗曾孫入立桓帝無嗣又以肅宗玄孫入立是同宗相後以次建之也圖以族屬之遠近寫為圖也牒譜第之也】豈可横叨天功以為已力乎【横戶孟翻】宜辭大賞以全身名武不能用植身長八尺二寸【長直亮翻】音聲如鍾性剛毅有大節少事馬融【少詩照翻】融性豪侈多列女倡歌舞於前【倡音昌】植侍講積年未嘗轉盻融以是敬之太后以陳蕃舊德特封高陽鄉侯蕃上疏讓曰臣聞割地之封功德是為【為于偽翻】臣雖無素潔之行【行下孟翻】竊慕君子不以其道得之不居也【孔子曰富與貴是人之所欲也不以其道得之不處也】若受爵不讓掩面就之【詩云受爵不讓至于已斯亡】使皇天震怒災流下民於臣之身亦何所寄太后不許蕃固讓章前後十上【上時掌翻】竟不受封 段熲將輕兵追羌出橋門【據東觀記橋門谷名水經注云橋門即橋山之長城門也】晨夜兼行與戰於奢延澤落川令鮮水上【賢曰即上郡奢延縣界也水經註奢延水出奢延縣西南赤沙阜東流入于河洛川在奢延水南賢曰令鮮水名在今甘州張掖縣界一名合黎水一名羌谷水余攷鮮水既捷乃追戰于靈武谷此鮮水非甘州之鮮水明矣當在上郡北地界】連破之又戰於靈武谷【賢曰靈武縣名有谷在今靈州懷遠縣西北余據前書地理志北地郡有靈武縣靈武谷當在此縣界非唐靈州之靈武縣也】羌遂大敗秋七月熲至涇陽【涇陽縣屬安定郡賢曰故城在今原州平凉縣南】餘寇四千落悉散入漢陽山谷間護匈奴中郎將張奐上言東羌雖破餘種難盡【種章勇翻】段熲性輕果慮負敗難常宜且以恩降可無後悔詔書下熲【下遐稼翻】熲復上言臣本知東羌雖衆而輭弱易制【復扶又翻下同輭乳兖翻柔也】所以比陳愚慮【比毗至翻】思為永寧之算而中郎將張奐說虜強難破宜用招降【降戶江翻】聖朝明鑒信納瞽言故臣謀得行奐計不用事勢相反遂懷猜恨信叛羌之訴飾潤辭意云臣兵累見折衂【賢曰傷敗曰衂音女六翻】又言羌一氣所生不可誅盡【賢曰言羌亦禀天之一氣所生誅之不可盡也】山谷廣大不可空靜血流汚野傷和致災【汚烏故翻】臣伏念周秦之際戎狄為害中興以來羌寇最盛誅之不盡雖降復叛今先零雜種累以反覆攻没縣邑剽略人物【剽匹妙翻】發冢露尸禍及生死上天震怒假手行誅昔邢為無道衛國伐之師興而雨【左傳曰衛大旱卜有事于山川不吉甯莊子曰昔周飢克殷而年豐今邢方無道天其欲衛討邢乎從之師興而雨】臣動兵涉夏連獲甘澍【澍音樹又音注時雨也】歲時豐稔人無疵疫上占天心不為災傷下察人事衆和師克自橋門以西落川以東故宫縣邑更相通屬【杜佑曰橋門以西落川以東今金城會寧平凉郡地屬之欲翻】非為深險絕域之地車騎安行無應折衂案奐為漢吏身當武職駐軍二年不能平寇虛【桓帝延熹九年奐督三州二營】欲修文戢戈招降獷敵【賢曰獷惡貌也音各猛翻】誕辭空說僭而無徵【左傳臧會卜為信與僭杜預註曰僭不信也】何以言之昔先零作寇趙充國徙令居内【宣帝時趙充國擊西羌降者三萬餘人徙之金城置金城屬國以處之令使也音零】煎當亂邊馬援遷之三輔始服終叛至今為鯁【賢曰鯁與梗同梗病也大雅云至今為梗】故遠識之士以為深憂今傍郡戶口單少數為羌所創毒【數所角翻下同創初良翻傷也】而欲令降徒與之雜居【降戶江翻】是猶種枳棘於良田養蛇虺於室内也故臣奉大漢之威建長久之策欲絶其本根不使能殖【賢曰殖生也左傳曰見惡如農夫之務去草焉絶其本根勿使能殖】本規三歲之費用五十四億今適朞年所耗未半而餘寇殘燼【杜預曰燼火餘木也】將向殄滅臣每奉詔書軍不内御【賢曰御制御也淮南子曰國不可從外理軍不可從中御】願卒斯言【卒子恤翻終也】一以任臣臨時量宜【量音亮】不失權便 八月司空王暢免宗正劉寵為司空 初竇太后之立也陳蕃有力焉【事見五十五卷桓帝延熹八年】及臨朝政無大小皆委於蕃蕃與竇武同心戮力以奬王室徵天下名賢李膺杜密尹勲劉瑜等皆列於朝廷與共參政事於是天下之士莫不延頸想望太平而帝乳母趙嬈及諸女尚書【賢曰女尚書内官也嬈音乃了翻】旦夕在太后側中常侍曹節王甫等共相朋結諂事太后太后信之數出詔命有所封拜蕃武疾之嘗共會朝堂【數所角翻朝直遥翻】蕃私謂武曰曹節王甫等自先帝時操弄國權【操千高翻】濁亂海内今不誅之後必難圖武深然之蕃大喜以手推席而起武於是引同志尚書令尹勲等共定計策會有日食之變蕃謂武曰昔蕭望之困一石顯【事見二十八卷元帝初元二年】况今石顯數十輩乎蕃以八十之年欲為將軍除害今可因日食斥罷宦官以塞天變【為于偽翻塞悉則翻】武乃白太后曰故事黄門常侍但當給事省内門戶主近署財物耳【省内謂禁中也近署財物謂少府所掌中藏府尚方内省諸署也】今乃使與政事任重權【與讀曰預】子弟布列專為貪暴天下匈匈正以此故宜悉誅廢以清朝廷太后曰漢元以來故事【漢元漢初也】世有宦官但當誅其有罪者豈可盡廢邪時中常侍管霸頗有才畧專制省内武先白收霸及中常侍蘇康等皆坐死武復數白誅曹節等【復扶又翻數所角翻】太后冘豫未忍【冘音淫】故事久不發蕃上疏曰今京師嚻嚻道路諠譁言侯覽曹節公乘昕王甫鄭颯等與趙夫人諸尚書並亂天下【公乘秦爵也此以爵為氏乘繩證翻昕許斤翻趙夫人即趙嬈颯音立】附從者升進忤逆者中傷【忤五故翻中竹仲翻】一朝羣臣如河中木耳【朝直遥翻謂舉朝之臣也】汎汎東西耽禄畏害陛下今不急誅此曹必生變亂傾危社稷其禍難量【量音良】願出臣章宣示左右并令天下諸姦知臣疾之太后不納是月太白犯房之上將入太微【晉書天文志房四星為明堂天子布政之宫也亦四輔也第一星上將也次次將也次次相也上星上相也大微天子庭也】侍中劉瑜素善天官【天官即天文也史記天官書猶後之天文志】惡之【惡烏路翻】上書皇太后曰案占書宫門當閉將相不利姦人在主傍願急防之又與武蕃書以星辰錯繆不利大臣宜速斷大計【斷丁亂翻】於是武蕃以朱㝢為司隸校尉劉祐為河南尹虞祁為雒陽令武奏免黄門令魏彪以所親小黄門山氷代之【姓譜周有山師之官子孫以為氏或云烈山氏之後】使氷奏收長樂尚書鄭颯【長樂尚書蓋以太后臨朝置之以掌奏下外朝文書衆事也樂音洛下同】送北寺獄蕃謂武曰此曹子便當收殺何復考為【復扶又翻下同】武不從令氷與尹勲侍御史祝瑨雜考颯辭連及曹節王甫勲氷即奏收節等使劉瑜内奏九月辛亥武出宿歸府典中書者先以告長樂五官史朱瑀瑀盜發武奏【長樂太后宫也太后宫有女尚書五人五官史主之考異曰范書帝紀作丁亥袁紀作辛亥按長歷是年九月乙巳朔無丁亥今從袁紀】罵曰中官放縱者自可誅耳我曹何罪而當盡見族滅因大呼曰【呼火故翻下同】陳蕃竇武奏白太后廢帝為大逆乃夜召素所親壯健者長樂從官史共普張亮等十七人【長樂從官史掌太后宫從官從才用翻共音龔】喢血共盟【喢色洽翻】謀誅武等曹節白帝曰外間切切【切切猶言迫急也】請出御德陽前殿令帝拔劔踴躍使乳母趙嬈等擁衛左右取棨信閉諸禁門【賢曰棨有衣戟也漢官儀曰凡居宮中皆施籍於掖門案姓名省入者本官為封棨傳審印信然後受也】召尚書官屬脅以白刃使作詔板拜王甫為黄門令【詔板所謂尺一也】持節至北寺獄收尹勲山氷氷疑不受詔甫格殺之并殺勲出鄭颯還兵刼太后奪璽綬【璽斯氏翻綬音受】令中謁者守南宫閉門絶複道【謁者掌守門戶文帝自代邸人立有謁者十人持戟衛端門是也雒陽南北宫有複道相通】使鄭颯等持節及侍御史謁者捕收武等武不受詔馳入步兵營與其兄子步兵校尉紹共射殺使者【射而亦翻】召會北軍五校士數千人屯都亭【雒陽都亭也校戶教翻】下令軍士曰黄門常侍反盡力者封侯重賞陳蕃聞難將官屬諸生八十餘人並拔刃突入承明門【難乃旦翻 考異曰袁紀蕃到承明門使者不内曰公未被詔召何得勒兵入宮蕃曰趙鞅專兵向宫以逐君側之惡春秋義之有使者出開門蕃到尚書門正色云云今從范書】到尚書門攘臂呼曰【呼火故翻】大將軍忠以衛國黄門反逆何云竇氏不道邪王甫時出與蕃相遇適聞其言而讓蕃曰先帝新棄天下山陵未成武有何功兄弟父子並封三侯【謂武子機封渭陽侯兄子紹封鄠侯紹弟靖封西鄉侯】又設樂飲讌多取掖庭宫人旬日之間貲財巨萬大臣若此為是道邪【謂此非不道而何】公為宰輔苟相阿黨復何求賊【復扶又翻】使劔士收蕃蕃拔劔叱甫辭色逾厲遂執蕃送北寺獄 【考異曰范書蕃傳曰蕃拔劔叱甫甫兵不敢近乃益人圍之數十重遂執蕃送獄今從袁紀】黄門從官騶蹋踧蕃曰【從才用翻騶側尤翻賢曰騶騎士也踧子六翻】死老魅【魅明祕翻物老而能為精怪曰魅】復能損我曹員數奪我曹稟假不【稟給也假借也不俯九翻】即日殺之時護匈奴中郎將張奐徵還京師曹節等以奐新至不知本謀矯制以少府周靖行車騎將軍加節與奐率五營士討武夜漏盡【天且明也】王甫將虎賁羽林等合千餘人出屯朱雀掖門【北宫南掖門曰朱雀門將即亮翻】與奐等合已而悉軍闕下與武對陳【陳讀曰陣】甫兵漸盛使其士大呼武軍曰竇武反汝皆禁兵當宿衛宫省何故隨反者乎先降有賞營府兵素畏服中官【營府謂五營校尉府也】於是武軍稍稍歸甫自旦至食時兵降畧盡【降戶江翻】武紹走諸軍追圍之皆自殺梟首雒陽都亭【梟工堯翻】收捕宗親賓客姻屬悉誅之及侍中劉瑜屯騎校尉馮述皆夷其族宦官又譖虎賁中郎將河間劉淑故尚書會稽魏朗云與武等通謀皆自殺【會工外翻】遷皇太后於南宫徙武家屬於日南自公卿以下嘗為蕃武所舉者及門生故吏皆免官禁錮議郎勃海巴肅【姓譜巴巴國之後後漢又有揚州刺史巴祗】始與武等同謀曹節等不知但坐禁錮後乃知而收之肅自載詣縣【肅勃海高城縣人】縣令見肅入閤解印綬欲與俱去肅曰為人臣者有謀不敢隱有罪不逃刑既不隱其謀矣又敢逃其刑乎遂被誅【被皮義翻】曹節遷長樂衛尉封育陽侯【育陽縣屬南陽郡】王甫遷中常侍黄門令如故朱瑀共普張亮等六人皆為列侯【共音龔姓譜共商諸侯之國晉有左行共華乂云鄭共叔段之後】十一人為關内侯於是羣小得志士大夫皆喪氣【喪息浪翻】蕃友人陳留朱震收葬蕃尸匿其子逸事覺繫獄合門桎梏震受考掠【桎之日翻梏工沃翻掠音亮】誓死不言逸由是得免武府掾桂陽胡騰殯歛武尸行喪【掾俞絹翻歛力贍翻】坐以禁錮武孫輔年二歲騰詐以為已子與令史南陽張敞【百官志大將軍府令史及御屬三十一人】共匿之於零陵界中亦得免張奐遷大司農以功封侯奐深病為曹節等所賣固辭不受 以司徒胡廣為太傅録尚書事司空劉寵為司徒大鴻臚許栩為司空【臚陵如翻栩况羽翻】冬十月甲辰晦日有食之 十一月太尉劉矩免以<br />
<br />
  太僕沛國聞人襲為太尉【聞人姓也風俗通曰少正卯魯之聞人其後氏焉】十二月鮮卑及濊貊寇幽并二州【濊音穢貊莫百翻】 是歲疏勒王季父和得殺其王自立 烏桓大人上谷難樓有衆九千餘落遼西丘力居有衆五千餘落自稱王遼東蘇僕延有衆千餘落自稱峭王【峭音七笑翻】右北平烏延有衆八百餘落自稱汗魯王【史言烏桓強盛】<br />
<br />
  二年春正月丁丑赦天下 帝迎董貴人於河間三月乙巳尊為孝仁皇后居永樂宫【樂音洛】拜其兄寵為執金吾兄子重為五官中郎將 夏四月壬辰有青蛇見於御坐上【見賢遍翻坐徂卧翻】癸巳大風雨雹霹靂【霹靂震霆也考異曰帝紀建寧二年四月癸巳大風雨雹楊賜傳熹平元年青蛇見御坐續漢志熹平元年四月甲午青蛇見御坐袁紀建寧二年四月壬辰青蛇見癸巳大風按張奐傳論陳竇薦王李與袁紀相應今從之】拔大木百餘詔公卿以下各上封事大司農張奐上疏曰昔周公葬不如禮天乃動威【尚書大傳曰周公薨成王欲葬之成周天乃雷電以風禾則盡偃大木斯拔邦人大恐王葬周公于畢示不敢臣也】今竇武陳蕃忠貞未被明宥【被皮義翻】妖眚之來皆為此也【為于偽翻】宜急為收葬【為于偽翻】徙還家屬其從坐禁錮一切蠲除【蠲吉玄翻】又皇太后雖居南宫而恩禮不接朝臣莫言遠近失望宜思大義顧復之報【賢曰顧旋視也復反復也小雅曰父兮生我母兮鞠我顧我復我出入腹我】上深嘉奐言以問諸常侍左右皆惡之【惡烏路翻】帝不得自從奐又與尚書劉猛等共薦王暢李膺可參三公之選曹節等彌疾其言遂下詔切責之奐等皆自囚廷尉數日乃得出並以三月俸贖罪【俸扶用翻】郎中東郡謝弼上封事曰臣聞惟虺惟蛇女子之祥【詩小雅無羊之辭鄭玄註云虺蛇穴處隂之祥也】伏惟皇太后定策宫闥援立聖明書曰父子兄弟罪不相及【左傳胥臣曰康誥曰父不慈子不祇兄不友弟不恭不相及也今尚書康誥無此語】竇氏之誅豈宜咎延太后幽隔空宫愁感天心如有霧露之疾陛下當何面目以見天下孝和皇帝不絶竇氏之恩【事見四十七卷永元九年】前世以為美談禮為人後者為之子今以桓帝為父豈得不以太后為母哉願陛下仰慕有虞烝烝之化凱風慰母之念【書堯典曰烝烝乂不格姦孔安國註云烝烝猶進進也言舜進于善道詩凱風曰有子七人莫慰母心】臣又聞開國承家小人勿用【易師卦上六爻辭】今功臣久外未蒙爵秩阿母寵私乃享大封大風雨雹亦由於兹【雨于具翻】又故太傅陳蕃勤身王室而見陷羣邪一旦誅滅其為酷濫駭動天下而門生故吏並離徙錮【離遭也】蕃身已往人百何贖【詩國風黄鳥曰如可贖兮人百其身】宜還其家屬解除禁網夫台宰重器國命所繫今之四公唯司空劉寵斷斷守善餘皆素餐致寇之人必有折足覆餗之凶【賢曰四公謂劉矩為太尉許訓為司徒胡廣為太傅及寵也書曰如有一个臣斷斷猗無他技孔安國註云斷斷猗然專一之臣也素空乜無德而食其禄曰素餐易曰負且乘致寇至又曰鼎折足覆公餗鼎以喻三公餗鼎實也折足覆餗言不勝具任據是年閒人襲已代劉矩為太尉餘三公亦不與賢註合斷丁亂翻折而設翻餗音速】可因災異並加罷黜徵故司空王暢長樂少府李膺並居政事庶災變可消國祚惟永左右惡其言【惡烏路翻】出為廣陵府丞【府丞即郡丞也】去官歸家曹節從子紹為東郡太守以它罪收弼掠死於獄【掠音亮】帝以蛇妖問光禄勲楊賜賜上封事曰夫善不妄來災不空發王者心有所想雖未形顔色而五星以之推移隂陽為其變度夫皇極不建則有龍蛇之孽【賢曰洪範五行傳曰皇大也極中也建立也孽災也君不合大中是謂不立蛇龍隂類也】詩云惟虺惟蛇女子之祥惟陛下思乾剛之道别内外之宜抑皇甫之權割艶妻之愛【賢曰艶妻周幽王后褒姒也皇甫卿士皆后之黨用后嬖寵而居位也詩云皇甫卿士艶妻煽方處别彼列翻】則蛇變可消禎祥立應賜秉之子也 五月太尉聞人襲司空許栩免六月以司徒劉寵為太尉太常汝南許訓為司徒太僕長沙劉囂為司空囂素附諸常侍故致位公輔 詔遣謁者馮禪說降漢陽散羌【說輸芮翻降戶江翻】段熲以春農百姓布野羌雖暫降而縣官無廩必當復為盜賊【復扶又翻】不如乘虛放兵【放兵謂縱兵擊羌也】埶必殄滅熲於是自進營去羌所屯凡亭山四五十里【魏收地形志安定鶉隂縣有凡亭杜佑作瓦亭山注云瓦亭山在今平凉郡蕭關縣】遣騎司馬田晏假司馬夏育將五千人先進擊破之【夏戶雅翻】羌衆潰東犇復聚射虎谷【復扶又翻下同】分兵守谷上下門熲規一舉滅之不欲復令散走秋七月熲遣千人於西縣結木為柵【西縣前漢屬隴西郡後漢屬漢陽郡參據二志皆云縣有嶓冢山西漢水所出是則禹貢所謂嶓冢導漾東流為漢其發源之地也段熲討羌起於安定高平羌敗則追至上郡奢延及大敗于靈武谷乃追至安定涇陽諸羌散入漢陽山谷間聚屯凡亭山凡亭既破復聚射虎谷熲乃於西縣結柵以遮之以羌奔潰所趨考之射虎谷在西縣東北凡亭山當在射虎谷東北蓋東羌為熲兵所迫復欲西犇出塞歸其舊來巢穴而殱於是谷也賢曰西縣故城在今秦州上邽縣西南】廣二十步長四十里遮之【廣古曠翻長直亮翻】分遣晏育等將七千人衘枚夜上西山結營穿塹去虜一里許又遣司馬張愷等將三千人上東山【上時掌翻】虜乃覺之熲因與愷等挾東西山縱兵奮擊破之追至谷上下門窮山深谷之中處處破之斬其渠帥以下萬九千級【帥所類翻】馮禪等所招降四千人分置安定漢陽隴西三郡於是東羌悉平熲凡百八十戰斬三萬八千餘級獲雜畜四十二萬七千餘頭【畜許又翻】費用四十四億軍士死者四百餘人更封新豐縣侯邑萬戶臣光曰書稱天地萬物父母惟人萬物之靈亶聰明作元后元后作民父母【周書泰誓之辭亶誠也】夫蠻夷戎狄氣類雖殊其就利避害樂生惡死【樂音洛惡烏路翻】亦與人同耳御之得其道則附順服從失其道則離叛侵擾固其宜也是以先王之政叛則討之服則懷之處之四裔【裔邊也處昌呂翻】不使亂禮義之邦而已若乃視之如草木禽獸不分臧否不辨去來悉艾殺之【否音鄙艾讀曰刈】豈作民父母之意哉且夫羌之所以叛者為郡縣所侵寃故也【侵寃者為所侵刻而衘寃】叛而不即誅者將帥非其人故也苟使良將驅而出之塞外擇良吏而牧之則疆場之臣也豈得專以多殺為快邪夫御之不得其道雖華夏之民亦將蠭起而為寇又可盡誅邪然則段紀明之為將【段熲字紀明犯太宗嫌諱故稱字】雖克捷有功君子所不與也<br />
<br />
  九月江夏蠻反【夏戶雅翻】州郡討平之 丹楊山越圍太守陳夤夤擊破之【山越本亦越人依阻山險不納王租故曰山越寇擾郡縣蓋自此始其後孫吳悉取其地以民為兵遂為王土】 初李膺等雖廢錮【事見上卷桓帝延熹九年】天下士大夫皆高尚其道而汙穢朝廷希之者唯恐不及更共相標榜為之稱號【賢曰標榜猶相稱揚也余謂立表以示人曰標揭書以示人曰榜標榜猶言表揭也更工衡翻】以竇武陳蕃劉淑為三君君者言一世之所宗也李膺荀翌【翌范書作昱】杜密王暢劉祐魏朗趙典朱㝢為八俊俊者言人之英也郭泰范滂尹勲巴肅及南陽宗慈陳留夏馥汝南蔡衍泰山羊陟為八顧顧者言能以德行引人者也【行下孟翻】張儉翟超岑晊苑康【翟萇伯翻晊之日翻】及山陽劉表汝南陳翔魯國孔昱山陽檀敷為八及及者言其能導人追宗者也【賢曰導引也言謂所宗仰者】度尚及東平張邈王孝東郡劉儒泰山胡母班【風俗通曰胡母姓本陳胡公之後也公子完奔齊遂有齊國齊宣王母弟别封母鄉遠取胡公近取母邑故曰胡母氏】陳留秦周魯國蕃嚮【賢曰蕃姓也音皮】東萊王章為八厨厨者言能以財救人者也及陳竇用事復舉拔膺等陳竇誅膺等復廢宦官疾惡膺等每下詔書輒申黨人之禁【復扶又翻惡烏路翻下遐稼翻】侯覽怨張險尤甚【以破其冢宅也事見上卷桓帝延熹九年】覽鄉人朱並素佞邪為儉所棄承覽意指上書告儉與同鄉二十四人别相署號共為部黨圖危社稷而儉為之魁詔刋章捕儉等【刋章者刋去並姓名而下其章也】冬十月大長秋曹節因此諷有司奏諸鉤黨者故司空虞放及李膺杜密朱㝢荀翌翟超劉儒范滂等【賢曰鉤謂相牽引也】請下州郡考治【下遐稼翻治直之翻】是時上年十四問節等曰何以為鉤黨對曰鉤黨者即黨人也上曰黨人何用為惡而欲誅之邪對曰皆相舉羣輩欲為不軌上曰不軌欲如何對曰欲圖社稷上乃可其奏【軌法度也君君臣臣所謂法也為人臣而欲圖危社稷謂之不法誠是也而諸閹以此罪加之君子帝不之悟眂元帝之不省召致廷尉為下獄者闇乂甚焉悲夫】或謂李膺曰可去矣對曰事不辭難罪不逃刑臣之節也【左傳羊舌赤之言曰事君不避難有罪不逃刑】吾年已六十死生有命去將安之乃詣詔獄考死門生故吏並被禁錮【被皮義翻】侍御史蜀郡景毅子顧為膺門徒未有録牒不及於譴【録記也牒籍也時聚徒教授多者以千計各錄記其姓名于譜牒】毅慨然曰本謂膺賢遣子師之豈可以漏脱名籍苟安而已遂自表免歸汝南督郵吳導受詔捕范滂至征羌抱詔書閉傳舍【征羌縣屬汝南郡木當鄉縣光武以來歙有平羌之功改為征羌侯國以封之因名焉滂縣人也賢曰傳驛舍也音知戀翻征羌故城在今豫州郾陵縣東南】伏牀而泣一縣不知所為滂聞之曰必為我也【為于偽翻】即自詣獄縣令郭揖大驚出解印綬引與俱亡曰天下大矣子何為在此滂曰滂死則禍塞何敢以罪累君【塞悉則翻累力瑞翻】又令老母流離乎其母就與之訣滂白母曰仲博孝敬足以供養【仲博滂弟字也供俱用翻養羊尚翻】滂從龍舒君歸黄泉存亡各得其所惟大人割不可忍之恩勿增感戚仲博者滂弟也龍舒君者滂父龍舒侯相顯也【相息亮翻】母曰汝今得與李杜齊名死亦何恨【李杜謂李膺杜密】既有令名復求夀考可兼得乎【復扶又翻】滂跪受教再拜而辭顧其子曰吾欲使汝為惡惡不可為使汝為善則我不為惡行路聞之莫不流涕凡黨人死者百餘人妻子皆徙邊天下豪桀及儒學有行義者【行下孟翻】宦官一切指為黨人有怨隙者因相陷害睚眦之忿濫入黨中【睚牛懈翻眦士解翻】州郡承旨或有未嘗交關亦離禍毒【離與罹同遭也】其死徙廢禁者又六七百人【廢禁謂廢棄而禁錮】郭泰聞黨人之死私為之慟曰詩云人之云亡邦國殄瘁【為于偽翻詩大雅瞻卬之辭毛氏曰殄盡也瘁病也瘁似醉翻】漢室滅矣但未知瞻烏爰止于誰之屋耳【詩小雅正月之辭毛氏註曰富人之屋烏所集也鄭氏曰視烏集於富人之屋以言今民亦當求明君而歸之考異曰范書以泰此語為哭陳竇袁紀以為哭三君八俊今從之】泰雖好臧否人倫【好呼到翻否音鄙】而不為危言覈論【覈謂深探其實也刻覈也】故能處濁世而怨禍不及焉【處昌呂翻】張儉亡命困迫望門投止【望門而投之以求止舍困急之甚也】莫不重其名行【行下孟翻】破家相容後流轉東萊止李篤家外黄令毛欽操兵到門【考兩漢志外黄縣屬陳留郡黄縣屬東萊郡毛欽蓋為黄縣令外字衍操千高翻】篤引欽就席曰張儉負罪亡命篤豈得藏之若審在此此人名士明廷寧宜執之乎欽因起撫篤曰蘧伯玉恥獨為君子足下如何專取仁義篤曰今欲分之明廷載半去矣【賢曰明廷猶言明府言不執儉得義之半也】欽歎息而去篤導儉經北海戲子然家【戲許宜翻姓譜伏戲氏之後】遂入漁陽出塞其所經歷伏重誅者以十數連引收考者布徧天下宗親並皆殄滅郡縣為之殘破【為于偽翻】儉與魯國孔襃有舊亡抵襃不遇【賢曰抵歸也】襃弟融年十六匿之後事泄儉得亡走國相收襃融送獄【相息亮翻】未知所坐融曰保納舍藏者融也當坐【謂自保無它而納儉因舍止而藏匿之】襃曰彼來求我非弟之過吏問其母母曰家事任長【任音壬長知兩翻】妾當其辜一門爭死郡縣疑不能决乃上讞之【賢曰前書音義曰讞請也上時掌翻讞音宜桀翻】詔書竟坐襃及黨禁解儉乃還鄉里後為衛尉卒年八十四【儉傳云建安初徵為衛尉不得已而起儉見曹氏世德已萌乃闔門縣車不豫政事歲餘卒於許下】夏馥聞張儉亡命歎曰孽自已作空汙良善【汙烏路翻】一人逃死禍及萬家何以生為乃自翦須變形【須與鬚同】入林慮山中【慮音廬】隱姓名為冶家傭親突煙炭形貌毁瘁【瘁似醉翻】積二三年人無知者馥弟靜載縑帛追求餉之馥不受曰弟奈何載禍相餉乎黨禁未解而卒初中常侍張讓父死歸葬潁川雖一郡畢至而名士無往者讓甚恥之陳寔獨弔焉及誅黨人讓以寔故多所全宥南陽何顒素與陳蕃李膺善亦被收捕【顒魚容翻被皮義翻】乃變名姓匿汝南間與袁紹為奔走之交常私入雒陽從紹計議為諸名士罹黨事者求救援設權計使得逃隱所全免甚衆初太尉袁湯三子成逢隗成生紹逢生術【據術字公路當讀如月令審端徑術之術音遂又據說文術邑中道讀從入聲則二音皆通隗五罪翻】逢隗皆有名稱少歷顯官【稱尺證翻少詩照翻】時中常侍袁赦 【考異曰袁紀作袁朗今從范書袁隗傳】以逢隗宰相家與之同姓推崇以為外援故袁氏貴寵於世富奢甚不與它公族同紹壯健有威容愛士養名賓客輻湊歸之輜軿柴轂填接街陌【賢曰說文曰軿車衣車也鄭玄注周禮曰軿猶屏也取其自蔽隱柴轂賤者之車袁紹事始此黨錮既死而誅宦官者二袁也人不為善而欲去害己者天其許之乎】術亦以俠氣聞逢從兄子閎少有操行【俠戶頰翻從才用翻少詩照翻行下孟翻】以耕學為業逢隗數餽之無所受【數所角翻】閎見時方險亂而家門富盛常對兄弟歎曰吾先公福祚後世不能以德守之而競為驕奢與亂世爭權此即晉之三郤矣【先公謂袁安也三郤謂晉大夫郤錡郤犫郤至也郤氏世為晉卿三子者憑藉世資驕奢侵權為厲公所殺】及黨事起閎欲投迹深林以母老不宜遠遁乃築土室四周於庭不為戶自牖納飲食母思閎時往就視母去便自掩閉兄弟妻子莫得見也潛身十八年卒於土室初范滂等非訐朝政【賢曰訐謂横議是非也訐居謁翻朝直遥翻】自公卿以下皆折節下之【折而設翻下遐稼翻】太學生爭慕其風以為文學將興處士復用申屠蟠獨歎曰昔戰國之世處士横議列國之王至為擁篲先驅【史記鄒衍如燕昭王擁篲先驅請列弟子之座而受業築碣石宫身親往師之處昌呂翻復扶又翻横戶孟翻為于偽翻篲祥歲翻】卒有坑儒燒書之禍【事見七卷秦始皇三十四年三十五年卒子恤翻】今之謂矣乃絶迹於梁碭之間【碭音唐】因樹為屋自同傭人居二年滂等果罹黨錮之禍唯蟠超然免於評論<br />
<br />
  臣光曰天下有道君子揚于王庭以正小人之罪而莫敢不服天下無道君子囊括不言以避小人之禍而猶或不免【坤之六四居近五之位而無相得之義乃上下閉隔之時羣隂既盛故當括囊以避禍夬以五陽決一隂小人衰微君子道盛故可揚于王庭以聲小人之罪】黨人生昬亂之世不在其位四海横流而欲以口舌救之臧否人物【横戶孟翻否音鄙】激濁揚清撩虺蛇之頭【撩連條翻】踐虎狼之尾以至身被淫刑【被皮義翻】禍及朋友士類殱滅而國隨以亡不亦悲乎【殱息亷翻】夫唯郭泰既明且哲以保其身【以尹吉甫美仲山甫者美郭泰】申屠蟠見幾而作不俟終日【謂申屠蟠得豫之六二幾居希翻】卓乎其不可及已<br />
<br />
  庚子晦日有食之 十一月太尉劉寵免太僕扶溝郭禧為太尉 鮮卑寇并州 長樂太僕曹節病困詔拜車騎將軍有頃疾瘳上印綬【上時掌翻】復為中常侍位特進秩中二千石 高句驪王伯固寇遼東玄菟太守耿臨討降之【句如字又音駒驪力知翻】<br />
<br />
  三年春三月丙寅晦日有食之 徵段熲還京師拜侍中熲在邊十餘年未嘗一日蓐寢【熲古迥翻郭璞曰蓐席也】與將士同甘苦故皆樂為死戰【樂音洛】所嚮有功 夏四月太尉郭禧罷以太中大夫聞人襲為太尉 秋七月司空劉囂罷八月以大鴻臚梁國橋玄為司空【姓譜黄帝葬橋山子孫守冢因氏焉】 九月執金吾董寵坐矯永樂太后屬請下獄死【屬之欲翻下遐稼翻】 冬鬱林太守谷永以恩信招降烏滸人十餘萬【萬震曰烏滸之地在廣州之南交州之北賢曰烏滸南方夷號也廣州記曰其俗食人以鼻飲酒口中進噉如故劉昫曰貴州鬱平縣漢鬱林廣鬱縣地古西甌駱越所居谷永招降烏滸開置七縣即此也杜佑曰烏滸地在今南海郡之西南安南府北朔寧郡管滸呼古翻】皆内屬受冠帶開置七縣凉州刺史扶風孟佗【賢曰佗音駝】遣從事任涉將敦煌兵<br />
<br />
  五百人與戊巳校尉曹寛西域長史張宴將焉耆龜兹車師前後部合三萬餘人討疏勒【以元年疏勒弑其王也任音壬敦徒門翻校戶孝翻龜兹音丘慈】攻楨中城四十餘日不能下引去其後疏勒王連相殺害朝廷亦不能復治【復扶又翻治直之翻】初中常侍張讓有監奴典任家事威形諠赫【諠况遠翻】孟佗資產饒贍【贍而艶翻】與奴朋結傾竭饋問無所遺愛【言其汎愛無有遺者】奴咸德之問其所欲佗曰吾望汝曹為我一拜耳【為于偽翻】時賓客求謁讓者車常數百千兩【兩音亮】佗詣讓後至不得進監奴乃率諸倉頭迎拜於路遂共轝車入門【轝羊茹翻】賓客咸驚謂佗善於讓皆爭以珍玩賂之佗分以遺讓【遺于季翻】讓大喜由是以佗為凉州刺史<br />
<br />
  四年春正月甲子帝加元服赦天下唯黨人不赦 二月癸卯地震 三月辛酉朔日有食之 太尉聞人襲免以太僕汝南李咸為太尉 大疫 司徒許訓免以司空橋玄為司徒 夏四月以太常南陽來艶為司空秋七月司空來艶免 癸丑立貴人宋氏為皇后后執金吾酆之女也 司徒橋玄免以太常南陽宗俱為司空前司空許栩為司徒 帝以竇太后有援立之功冬十月戊子朔率羣臣朝太后於南宫親饋上夀【朝直遥翻饋進食也】黄門令董萌因此數為太后訴寃【數所角翻為于偽翻】帝深納之供養資奉有加於前【供居用翻養羊尚翻】曹節王甫疾之誣萌以謗訕永樂宫【帝母孝仁董太后所居也樂音洛】下獄死【下遐稼翻】 鮮卑寇并州<br />
<br />
  資治通鑑卷五十六<br />
<br />
<史部,編年類,資治通鑑>  <br>
   </div> 

<script src="/search/ajaxskft.js"> </script>
 <div class="clear"></div>
<br>
<br>
 <!-- a.d-->

 <!--
<div class="info_share">
</div> 
-->
 <!--info_share--></div>   <!-- end info_content-->
  </div> <!-- end l-->

<div class="r">   <!--r-->



<div class="sidebar"  style="margin-bottom:2px;">

 
<div class="sidebar_title">工具类大全</div>
<div class="sidebar_info">
<strong><a href="http://www.guoxuedashi.com/lsditu/" target="_blank">历史地图</a></strong>  
<a href="http://www.880114.com/" target="_blank">英语宝典</a>  
<a href="http://www.guoxuedashi.com/13jing/" target="_blank">十三经检索</a> 
<br><strong><a href="http://www.guoxuedashi.com/gjtsjc/" target="_blank">古今图书集成</a></strong> 
<a href="http://www.guoxuedashi.com/duilian/" target="_blank">对联大全</a> <strong><a href="http://www.guoxuedashi.com/xiangxingzi/" target="_blank">象形文字典</a></strong> 

<br><a href="http://www.guoxuedashi.com/zixing/yanbian/">字形演变</a>  <strong><a href="http://www.guoxuemi.com/hafo/" target="_blank">哈佛燕京中文善本特藏</a></strong>
<br><strong><a href="http://www.guoxuedashi.com/csfz/" target="_blank">丛书&方志检索器</a></strong> <a href="http://www.guoxuedashi.com/yqjyy/" target="_blank">一切经音义</a>  

<br><strong><a href="http://www.guoxuedashi.com/jiapu/" target="_blank">家谱族谱查询</a></strong>  <strong><a href="http://shufa.guoxuedashi.com/sfzitie/" target="_blank">书法字帖欣赏</a></strong> 
<br>

</div>
</div>


<div class="sidebar" style="margin-bottom:0px;">

<font style="font-size:22px;line-height:32px">QQ交流群9:489193090</font>


<div class="sidebar_title">手机APP 扫描或点击</div>
<div class="sidebar_info">
<table>
<tr>
	<td width=160><a href="http://m.guoxuedashi.com/app/" target="_blank"><img src="/img/gxds-sj.png" width="140"  border="0" alt="国学大师手机版"></a></td>
	<td>
<a href="http://www.guoxuedashi.com/download/" target="_blank">app软件下载专区</a><br>
<a href="http://www.guoxuedashi.com/download/gxds.php" target="_blank">《国学大师》下载</a><br>
<a href="http://www.guoxuedashi.com/download/kxzd.php" target="_blank">《汉字宝典》下载</a><br>
<a href="http://www.guoxuedashi.com/download/scqbd.php" target="_blank">《诗词曲宝典》下载</a><br>
<a href="http://www.guoxuedashi.com/SiKuQuanShu/skqs.php" target="_blank">《四库全书》下载</a><br>
</td>
</tr>
</table>

</div>
</div>


<div class="sidebar2">
<center>


</center>
</div>

<div class="sidebar"  style="margin-bottom:2px;">
<div class="sidebar_title">网站使用教程</div>
<div class="sidebar_info">
<a href="http://www.guoxuedashi.com/help/gjsearch.php" target="_blank">如何在国学大师网下载古籍?</a><br>
<a href="http://www.guoxuedashi.com/zidian/bujian/bjjc.php" target="_blank">如何使用部件查字法快速查字?</a><br>
<a href="http://www.guoxuedashi.com/search/sjc.php" target="_blank">如何在指定的书籍中全文检索?</a><br>
<a href="http://www.guoxuedashi.com/search/skjc.php" target="_blank">如何找到一句话在《四库全书》哪一页?</a><br>
</div>
</div>


<div class="sidebar">
<div class="sidebar_title">热门书籍</div>
<div class="sidebar_info">
<a href="/so.php?sokey=%E8%B5%84%E6%B2%BB%E9%80%9A%E9%89%B4&kt=1">资治通鉴</a> <a href="/24shi/"><strong>二十四史</strong></a>&nbsp; <a href="/a2694/">野史</a>&nbsp; <a href="/SiKuQuanShu/"><strong>四库全书</strong></a>&nbsp;<a href="http://www.guoxuedashi.com/SiKuQuanShu/fanti/">繁体</a>
<br><a href="/so.php?sokey=%E7%BA%A2%E6%A5%BC%E6%A2%A6&kt=1">红楼梦</a> <a href="/a/1858x/">三国演义</a> <a href="/a/1038k/">水浒传</a> <a href="/a/1046t/">西游记</a> <a href="/a/1914o/">封神演义</a>
<br>
<a href="http://www.guoxuedashi.com/so.php?sokeygx=%E4%B8%87%E6%9C%89%E6%96%87%E5%BA%93&submit=&kt=1">万有文库</a> <a href="/a/780t/">古文观止</a> <a href="/a/1024l/">文心雕龙</a> <a href="/a/1704n/">全唐诗</a> <a href="/a/1705h/">全宋词</a>
<br><a href="http://www.guoxuedashi.com/so.php?sokeygx=%E7%99%BE%E8%A1%B2%E6%9C%AC%E4%BA%8C%E5%8D%81%E5%9B%9B%E5%8F%B2&submit=&kt=1"><strong>百衲本二十四史</strong></a>  <a href="http://www.guoxuedashi.com/so.php?sokeygx=%E5%8F%A4%E4%BB%8A%E5%9B%BE%E4%B9%A6%E9%9B%86%E6%88%90&submit=&kt=1"><strong>古今图书集成</strong></a>
<br>

<a href="http://www.guoxuedashi.com/so.php?sokeygx=%E4%B8%9B%E4%B9%A6%E9%9B%86%E6%88%90&submit=&kt=1">丛书集成</a> 
<a href="http://www.guoxuedashi.com/so.php?sokeygx=%E5%9B%9B%E9%83%A8%E4%B8%9B%E5%88%8A&submit=&kt=1"><strong>四部丛刊</strong></a>  
<a href="http://www.guoxuedashi.com/so.php?sokeygx=%E8%AF%B4%E6%96%87%E8%A7%A3%E5%AD%97&submit=&kt=1">說文解字</a> <a href="http://www.guoxuedashi.com/so.php?sokeygx=%E5%85%A8%E4%B8%8A%E5%8F%A4&submit=&kt=1">三国六朝文</a>
<br><a href="http://www.guoxuedashi.com/so.php?sokeytm=%E6%97%A5%E6%9C%AC%E5%86%85%E9%98%81%E6%96%87%E5%BA%93&submit=&kt=1"><strong>日本内阁文库</strong></a> <a href="http://www.guoxuedashi.com/so.php?sokeytm=%E5%9B%BD%E5%9B%BE%E6%96%B9%E5%BF%97%E5%90%88%E9%9B%86&ka=100&submit=">国图方志合集</a> <a href="http://www.guoxuedashi.com/so.php?sokeytm=%E5%90%84%E5%9C%B0%E6%96%B9%E5%BF%97&submit=&kt=1"><strong>各地方志</strong></a>

</div>
</div>


<div class="sidebar2">
<center>

</center>
</div>
<div class="sidebar greenbar">
<div class="sidebar_title green">四库全书</div>
<div class="sidebar_info">

《四库全书》是中国古代最大的丛书,编撰于乾隆年间,由纪昀等360多位高官、学者编撰,3800多人抄写,费时十三年编成。丛书分经、史、子、集四部,故名四库。共有3500多种书,7.9万卷,3.6万册,约8亿字,基本上囊括了古代所有图书,故称“全书”。<a href="http://www.guoxuedashi.com/SiKuQuanShu/">详细>>
</a>

</div> 
</div>

</div>  <!--end r-->

</div>
<!-- 内容区END --> 

<!-- 页脚开始 -->
<div class="shh">

</div>

<div class="w1180" style="margin-top:8px;">
<center><script src="http://www.guoxuedashi.com/img/plus.php?id=3"></script></center>
</div>
<div class="w1180 foot">
<a href="/b/thanks.php">特别致谢</a> | <a href="javascript:window.external.AddFavorite(document.location.href,document.title);">收藏本站</a> | <a href="#">欢迎投稿</a> | <a href="http://www.guoxuedashi.com/forum/">意见建议</a> | <a href="http://www.guoxuemi.com/">国学迷</a> | <a href="http://www.shuowen.net/">说文网</a><script language="javascript" type="text/javascript" src="https://js.users.51.la/17753172.js"></script><br />
  Copyright &copy; 国学大师 古典图书集成 All Rights Reserved.<br>
  
  <span style="font-size:14px">免责声明:本站非营利性站点,以方便网友为主,仅供学习研究。<br>内容由热心网友提供和网上收集,不保留版权。若侵犯了您的权益,来信即刪。scp168@qq.com</span>
  <br />
ICP证:<a href="http://www.beian.miit.gov.cn/" target="_blank">鲁ICP备19060063号</a></div>
<!-- 页脚END --> 
<script src="http://www.guoxuedashi.com/img/plus.php?id=22"></script>
<script src="http://www.guoxuedashi.com/img/tongji.js"></script>

</body>
</html>
