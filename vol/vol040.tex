<!DOCTYPE html PUBLIC "-//W3C//DTD XHTML 1.0 Transitional//EN" "http://www.w3.org/TR/xhtml1/DTD/xhtml1-transitional.dtd">
<html xmlns="http://www.w3.org/1999/xhtml">
<head>
<meta http-equiv="Content-Type" content="text/html; charset=utf-8" />
<meta http-equiv="X-UA-Compatible" content="IE=Edge,chrome=1">
<title>資治通鑒_41-資治通鑑卷四十_41-資治通鑑卷四十</title>
<meta name="Keywords" content="資治通鑒_41-資治通鑑卷四十_41-資治通鑑卷四十">
<meta name="Description" content="資治通鑒_41-資治通鑑卷四十_41-資治通鑑卷四十">
<meta http-equiv="Cache-Control" content="no-transform" />
<meta http-equiv="Cache-Control" content="no-siteapp" />
<link href="/img/style.css" rel="stylesheet" type="text/css" />
<script src="/img/m.js?2020"></script> 
</head>
<body>
 <div class="ClassNavi">
<a  href="/24shi/">二十四史</a> | <a href="/SiKuQuanShu/">四库全书</a> | <a href="http://www.guoxuedashi.com/gjtsjc/"><font  color="#FF0000">古今图书集成</font></a> | <a href="/renwu/">历史人物</a> | <a href="/ShuoWenJieZi/"><font  color="#FF0000">说文解字</a></font> | <a href="/chengyu/">成语词典</a> | <a  target="_blank"  href="http://www.guoxuedashi.com/jgwhj/"><font  color="#FF0000">甲骨文合集</font></a> | <a href="/yzjwjc/"><font  color="#FF0000">殷周金文集成</font></a> | <a href="/xiangxingzi/"><font color="#0000FF">象形字典</font></a> | <a href="/13jing/"><font  color="#FF0000">十三经索引</font></a> | <a href="/zixing/"><font  color="#FF0000">字体转换器</font></a> | <a href="/zidian/xz/"><font color="#0000FF">篆书识别</font></a> | <a href="/jinfanyi/">近义反义词</a> | <a href="/duilian/">对联大全</a> | <a href="/jiapu/"><font  color="#0000FF">家谱族谱查询</font></a> | <a href="http://www.guoxuemi.com/hafo/" target="_blank" ><font color="#FF0000">哈佛古籍</font></a> 
</div>

 <!-- 头部导航开始 -->
<div class="w1180 head clearfix">
  <div class="head_logo l"><a title="国学大师官网" href="http://www.guoxuedashi.com" target="_blank"></a></div>
  <div class="head_sr l">
  <div id="head1">
  
  <a href="http://www.guoxuedashi.com/zidian/bujian/" target="_blank" ><img src="http://www.guoxuedashi.com/img/top1.gif" width="88" height="60" border="0" title="部件查字,支持20万汉字"></a>


<a href="http://www.guoxuedashi.com/help/yingpan.php" target="_blank"><img src="http://www.guoxuedashi.com/img/top230.gif" width="600" height="62" border="0" ></a>


  </div>
  <div id="head3"><a href="javascript:" onClick="javascript:window.external.AddFavorite(window.location.href,document.title);">添加收藏</a>
  <br><a href="/help/setie.php">搜索引擎</a>
  <br><a href="/help/zanzhu.php">赞助本站</a></div>
  <div id="head2">
 <a href="http://www.guoxuemi.com/" target="_blank"><img src="http://www.guoxuedashi.com/img/guoxuemi.gif" width="95" height="62" border="0" style="margin-left:2px;" title="国学迷"></a>
  

  </div>
</div>
  <div class="clear"></div>
  <div class="head_nav">
  <p><a href="/">首页</a> | <a href="/ShuKu/">国学书库</a> | <a href="/guji/">影印古籍</a> | <a href="/shici/">诗词宝典</a> | <a   href="/SiKuQuanShu/gxjx.php">精选</a> <b>|</b> <a href="/zidian/">汉语字典</a> | <a href="/hydcd/">汉语词典</a> | <a href="http://www.guoxuedashi.com/zidian/bujian/"><font  color="#CC0066">部件查字</font></a> | <a href="http://www.sfds.cn/"><font  color="#CC0066">书法大师</font></a> | <a href="/jgwhj/">甲骨文</a> <b>|</b> <a href="/b/4/"><font  color="#CC0066">解密</font></a> | <a href="/renwu/">历史人物</a> | <a href="/diangu/">历史典故</a> | <a href="/xingshi/">姓氏</a> | <a href="/minzu/">民族</a> <b>|</b> <a href="/mz/"><font  color="#CC0066">世界名著</font></a> | <a href="/download/">软件下载</a>
</p>
<p><a href="/b/"><font  color="#CC0066">历史</font></a> | <a href="http://skqs.guoxuedashi.com/" target="_blank">四库全书</a> |  <a href="http://www.guoxuedashi.com/search/" target="_blank"><font  color="#CC0066">全文检索</font></a> | <a href="http://www.guoxuedashi.com/shumu/">古籍书目</a> | <a   href="/24shi/">正史</a> <b>|</b> <a href="/chengyu/">成语词典</a> | <a href="/kangxi/" title="康熙字典">康熙字典</a> | <a href="/ShuoWenJieZi/">说文解字</a> | <a href="/zixing/yanbian/">字形演变</a> | <a href="/yzjwjc/">金 文</a> <b>|</b>  <a href="/shijian/nian-hao/">年号</a> | <a href="/diming/">历史地名</a> | <a href="/shijian/">历史事件</a> | <a href="/guanzhi/">官职</a> | <a href="/lishi/">知识</a> <b>|</b> <a href="/zhongyi/">中医中药</a> | <a href="http://www.guoxuedashi.com/forum/">留言反馈</a>
</p>
  </div>
</div>
<!-- 头部导航END --> 
<!-- 内容区开始 --> 
<div class="w1180 clearfix">
  <div class="info l">
   
<div class="clearfix" style="background:#f5faff;">
<script src='http://www.guoxuedashi.com/img/headersou.js'></script>

</div>
  <div class="info_tree"><a href="http://www.guoxuedashi.com">首页</a> > <a href="/SiKuQuanShu/fanti/">四库全书</a>
 > <h1>资治通鉴</h1> <!--         下载:【右键另存为】即可 --></div>
  <div class="info_content zj clearfix">
  
<div class="info_txt clearfix" id="show">
<center style="font-size:24px;">41-資治通鑑卷四十</center>
    資治通鑑卷四十    宋 司馬光 撰<br />
<br />
  胡三省 音註<br />
<br />
  漢紀三十二【起旃蒙作噩盡柔兆閹茂凡二年】<br />
<br />
  世祖光武皇帝上之上【諱秀字文叔賢曰禮祖有功而宗有德光武中葉興故廟稱世祖謚法能紹前業曰光克定禍亂曰武伏侯古今註曰秀之字曰茂伯仲叔季兄弟之次長兄伯升次仲故字文叔焉】<br />
<br />
  建武元年【是年六月即位改元】春正月方望與安陵人弓林【姓譜弓魯大夫叔弓之後又孔子弟子有仲弓又有馯臂子弓】共立前定安公嬰為天子聚黨數千人居臨涇【臨涇縣屬安定郡賢曰今涇州縣】更始遣丞相松等擊破皆斬之 鄧禹至箕關【賢曰箕關在今王屋縣東余據唐王屋縣屬懷州水經註箕關故城在垣縣】擊破河東都尉進圍安邑【縣名屬河東郡】 赤眉二部俱會弘農更始遣討難將軍蘇茂拒之【難乃旦翻】茂軍大敗赤眉衆遂大集乃分萬人為一營凡三十營三月更始遣丞相松與赤眉戰於蓩鄉【賢曰蓩音莫老翻字林曰毒草也因以為地名續漢志弘農有蓩鄉東觀記曰崇等入至弘農枯樅山下與茂戰崇比至蓩鄉轉至湖湖即湖城縣也以此而言其地蓋在今虢州湖城縣之間】松等大敗死者三萬餘人赤眉遂轉北至湖 蜀郡功曹李熊說公孫述宜稱天子【說輸芮翻】夏四月述即帝位號成家【賢曰以起成都故號成家】改元龍興【時有龍出其府因以紀元】李熊為大司徒述弟光為大司馬恢為大司空越嶲任貴據郡降述【王莽天鳳三年任貴據越嶲嶲音髓任音壬】 蕭王北擊尤來大槍五幡於元氏【地理志元氏縣屬常山郡闞駰曰趙公子元之封邑故曰元氏】追至北平連破之【賢曰北平縣屬中山國今易州永樂縣也】又戰於順水北【賢曰水經註云徐水經北平縣故城北光武追銅馬五幡破之於順水即徐水之别名也今在易州括地志徐水過北平縣界而東流又東逕清苑城】乘勝輕進反為所敗【敗補邁翻】王自投高岸突騎王豐下馬授王【騎奇寄翻下同】王僅而得免散兵歸保范陽【賢曰縣名在范水之陽屬涿郡范陽故城在今易州易縣東南】軍中不見王或云已歿諸將不知所為吳漢曰卿曹努力王兄子在南陽何憂無主【兄子謂伯升子章及興也】衆恐懼數日乃定賊雖戰勝而憚王威名夜遂引去大軍復進至安次連戰破之【復扶又翻賢曰安次縣名屬渤海郡今幽州縣也故城在縣東按我宋朝霸州文安縣本漢安次縣地】賊退入漁陽所過虜掠彊弩將軍陳俊言於王曰賊無輜重【重直用翻】宜令輕騎出賊前使百姓各自堅壁以絶其食可不戰而殄也王然之遣俊將輕騎馳出賊前視人保壁堅完者敕令固守放散在野者因掠取之賊至無所得遂散敗王謂俊曰困此虜者將軍策也馮異遺李軼書為陳禍福【遺于季翻為于偽翻】勸令歸附蕭王軼知長安已危而以伯升之死心不自安【事見上卷更始元年】乃報書曰軼本與蕭王首謀造漢【事見三十八卷王莽地皇三年】今軼守洛陽將軍鎮孟津俱據機軸【賢曰機弩牙也軸車軸也皆在物之要故取喻焉】千載一會思成斷金【易曰二人同心其利斷金陸德明曰斷丁亂翻王肅丁管翻】唯深達蕭王願進愚策以佐國安民軼自通書之後不復與異爭鋒【復扶又翻】故異得北攻天井關【劉昭志曰上黨高都縣有天井關賢曰在今澤州晉城縣南今太行山上關南有天井泉三所】拔上黨兩城又南下河南成臯以東十三縣降者十餘萬武勃將萬餘人攻諸畔者異與戰於士鄉下【劉昭志河南雒陽縣有士鄉聚續漢志曰士鄉亭名屬河南郡】大破斬勃軼閉門不救異見其信效具以白王王報異曰季文多詐【李軼字季文】人不能得其要領【要一遥翻】令移其書告守尉當警備者衆皆怪王宣露軼書朱鮪聞之【鮪于軌翻】使人刺殺軼【刺七亦翻】由是城中乖離多有降者【降戶江翻】 朱鮪聞王北征而河内孤乃遣其將蘇茂賈彊將兵三萬餘人度鞏河攻溫【鞏縣屬河南郡周鞏伯之國也河水過鞏縣北謂之鞏河即五社津也温縣屬河内郡周大夫蘇子邑賢曰鞏温並今洛州縣也】鮪自將數萬人攻平隂以綴異【賢曰平隂縣名屬河南郡杜佑曰漢平隂縣城在今洛陽縣北五十里水經註平隂即晉之隂地故隂戎所居魏文帝改曰河隂綴謂連綴也將即亮翻】檄書至河内宼恂即勒軍馳出並移告屬縣發兵會溫下軍吏皆諫曰今洛陽兵度河前後不絶宜待衆軍畢集乃可出也恂曰溫郡之藩蔽失溫則郡不可守遂馳赴之且日合戰而馮異遣救及諸縣兵適至恂令士卒乘城鼓譟大呼言曰劉公兵到蘇茂軍聞之陳動【呼火故翻陳讀曰陣】恂因奔擊大破之馮異亦渡河擊朱鮪鮪走異與恂追至洛陽環城一帀而歸【環音宦帀作荅翻周回也】自是洛陽震恐城門晝閉異恂移檄上狀諸將入賀因上尊號【上時掌翻下同】將軍南陽馬武先進曰大王雖執謙退奈宗廟社稷何宜先即尊位乃議征伐今此誰賊而馳騖擊之乎【賢曰誰謂未有主也前書音義曰直馳曰馳亂馳曰騖余謂誰賊者蓋謂位號未正指誰為賊也】王驚曰何將軍出此言可斬也乃引軍還薊復遣吳漢率耿弇景丹等十三將軍追尤來等【復扶又翻下除賈復外皆同】斬首萬三千餘級遂窮追至浚靡而還【賢曰浚靡縣名屬右北平郡故城在今漁陽縣北靡音麻】賊散入遼西遼東為烏桓貊人所鈔擊畧盡【貊莫白翻鈔楚交翻】都護將軍賈復【漢宣帝置西域都護盡護南北道諸國甘延壽之擊郅支也自謂為都護將軍漢朝未以為將軍號也至光武乃以命賈復】與五校戰於眞定復傷創甚【校戶教翻創初良翻】王大驚曰我所以不令賈復别將者【將即亮翻】為其輕敵也【為于偽翻】果然失吾名將聞其婦有孕生女邪我子娶之生男邪我女嫁之不令其憂妻子也復病尋愈追及王於薊相見甚驩【薊音計】還至中山諸將復上尊號王又不聽行到南平棘【賢曰縣名屬常山郡今趙州縣故城在縣南】諸將復固請之王不許諸將且出耿純進曰天下士大夫捐親戚棄土壤從大王於矢石之間者其計固望攀龍鱗附鳳翼以成其所志耳今大王留時逆衆不正號位純恐士大夫望絶計窮則有去歸之思無為久自苦也大衆一散難可復合純言甚誠切王深感曰吾將思之行至鄗【續漢志鄗縣屬常山國帝於此即位改曰高邑鄗呼各翻】召馮異問四方動静異曰更始必敗宗廟之憂在於大王宜從衆議會儒生彊華自關中奉赤伏符來詣王曰劉秀發兵捕不道四夷雲集龍鬭野四七之際火為主【賢曰彊音其兩翻姓譜其良翻風俗通作疆華系之曰晉有大夫疆劒四七二十八也自高祖至光武初起合二百二十八年即四七之際也漢火德故火為主也】羣臣因復奏請六月己未王即皇帝位于鄗南【時設壇於鄗南千秋亭五城陌賢曰其地在今趙州栢鄉縣 考異曰光武本紀馮異破蘇茂諸將上尊號光武還至薊皆在四月前而馮異傳異與李軼書云長安壞亂赤眉臨郊王侯構難大臣乖離綱紀已絶又勸光武稱尊號亦曰三王反叛更始敗亡按是年六月己未光武即位是月甲子鄧禹破王匡等於安邑王匡張卭等還奔長安乃謀以立秋貙膢時共刼更始然則三王反叛應在光武即位之後夏秋之交馮異安得於四月之前已言之也或者史家潤色其言致此差失耳】改元大赦 鄧禹圍安邑數月未下更始大將軍樊參將數萬人度大陽【賢曰大陽縣屬河東郡前書音義曰大河之陽春秋秦伯伐晉自茅津濟杜預曰河東大陽縣也】欲攻禹禹逆擊於解南斬之【賢曰解縣屬河東郡故城在今蒲州桑泉縣東南也師古曰解音蟹】王匡成丹劉均合軍十餘萬復共擊禹禹軍不利明日癸亥匡等以六甲窮日不出禹因得更治兵【治直之翻】甲子匡悉軍出攻禹禹令軍中毋得妄動既至營下因傳發諸將【孟康曰傳令軍中使發也】鼓而並進大破之匡等皆走禹追斬均及河東太守楊寶遂定河東匡等犇還長安 【考異曰劉玄傳王匡張卭守河東為鄧禹所破奔還長安鄧禹傳無張卭名今從之】張卬與諸將議曰赤眉旦暮且至見滅不久不如掠長安東歸南陽事若不集復入湖池中為盜耳乃共入說更始【說輸芮翻】更始怒不應莫敢復言【復扶又翻】更始使王匡陳牧成丹趙萌屯新豐李松軍掫以拒赤眉【賢曰掫音子侯翻及續漢志新豐有鴻門亭掫城即此也】張卬廖湛胡殷申屠建與隗囂合謀欲以立秋日貙膢時【賢曰前書音義曰貙獸以立秋日祭獸王者亦此日出獵用祭宗廟冀州北郡以八月朝作飲食為膢其俗語曰膢臘社伏風俗通嘗新始殺食曰貙膢漢儀立秋日郊禮畢始揚威武乃祀先虞告以烹鮮天子御戎輅白馬朱鬛躬執弩射牲牲以鹿麛斬牲於郊東門載獲車馳駟以薦陵廟名貙劉劉殺也貙於時殺物故以應之又謂之貙膢廖力弔翻貙去于翻膢音婁】共刼更始俱成前計 【考異曰袁紀云申屠建等勸更始讓帝位更始不應建等謀劫之今從范書】更始知之託病不出召張卭等入將悉誅之唯隗囂稱疾不入會客王遵周宗等勒兵自守更始狐疑不決卬湛殷疑有變遂突出獨申屠建在更始斬建使執金吾鄧曅將兵圍隗囂第卬湛殷勒兵燒門入戰宫中更始大敗囂亦潰圍走歸天水明旦更始東犇趙萌於新豐更始復疑王匡陳牧成丹與張卬等同謀乃並召入牧丹先至即斬之王匡懼將兵入長安與張卬等合 赤眉進至華隂【華戶化翻】軍中有齊巫【齊巫齊國之巫】常鼔舞祠城陽景王【城陽景王章有誅諸呂之功故齊人祠之以求福助】巫狂言景王大怒曰當為縣官何故為賊【賢曰縣官謂天子也】有笑巫者輒病軍中驚動方望弟陽說樊崇等曰今將軍擁百萬之衆西向帝城而無稱號【說輸芮翻稱尺證翻】名為羣賊不可以久不如立宗室挾義誅伐以此號令誰敢不從崇等以為然而巫言益甚前至鄭【鄭縣屬京兆賢曰今華州縣】乃相與議曰今迫近長安而鬼神若此當求劉氏共尊立之先是赤眉過式【地理志式縣屬泰山郡近其靳翻先悉薦翻】掠故式侯萌之子恭茂盆子三人自隨【萌之父曰憲城陽景王五世孫荒王順之子元帝時封式侯】恭少習尚書【少詩照翻】隨樊崇等降更始於洛陽【樊崇等降見上卷更始元年降戶江翻】復封式侯為侍中在長安茂與盆子留軍中屬右校卒史劉俠卿主牧牛【漢注卒史秩百石九卿寺及諸郡及軍行部校皆有之校戶教翻俠戶頰翻】及崇等欲立帝求軍中景王後得七十餘人唯茂盆子及前西安侯孝最為近屬崇等曰聞古者天子將兵稱上將軍乃書札為符曰上將軍又以兩空札置笥中【賢曰札簡也笥箧也】於鄭北設壇場祠城陽景王諸三老從事皆大會【赤眉諸帥最尊者號三老次從事】列盆子等三人居中立以年次探札盆子最幼後探得符【探吐南翻】諸將皆稱臣拜盆子時年十五被髪徒跣敝衣赭汗見衆拜恐畏欲啼【被皮義翻】茂謂曰善臧符【臧讀曰藏】盆子即齧折棄之【折而設翻】以徐宣為丞相樊崇為御史大夫逢安為左大司馬【逢皮江翻】謝禄為右大司馬其餘皆列卿將軍盆子雖立猶朝夕拜劉俠卿時欲出從牧兒戲俠卿怒止之崇等亦不復視也【復扶又翻】 秋七月辛未帝使使持節拜鄧禹為大司徒封鄼侯食邑萬戶【賢曰鄼縣屬南陽郡故城在今襄州穀城縣東北余謂蓋以禹功比蕭何故封之鄼鄼音贊】禹時年二十四又議選大司空帝以赤伏符曰王梁主衛作玄武丁丑以野王令王梁為大司空【帝以野王衛之所徙玄武水神之名司空水土之官也於是用梁賢曰玄武北方之神龜蛇合體野王縣屬河内郡宋白曰懷州河内縣古野王也】又欲以䜟文用平狄將軍孫咸行大司馬衆咸不悦【䜟楚譛翻】壬午以吳漢為大司馬初更始以琅邪伏湛為平原太守【姓譜伏本自伏羲之後漢初有濟南伏生守式又翻】時天下兵起湛獨晏然撫循百姓門下督謀為湛起兵湛收斬之【諸郡各有門下督主兵衛為于偽翻】於是吏民信向平原一境賴湛以全帝徵湛為尚書使典定舊制又以鄧禹西征拜湛為司直行大司徒事【東都之司徒西都之丞相也司直即丞相司直】車駕每出征伐常留鎮守 鄧禹自汾隂渡河入夏陽【汾隂縣屬河東夏陽縣屬馮翊】更始左輔都尉公乘歙引其衆十萬與左馮翊兵共拒禹於衙【地理志左輔都尉治高陵賢曰左輔即左馮翊也三輔皆有都尉衙縣名屬左馮翊故城在今同州白水縣東北左傳秦晉戰于彭衙即此地公乘姓也以秦爵為氏乘繩證翻歙許及翻】禹復破走之【復扶又翻】宗室劉茂聚衆京密間【茂元氏王歙從父弟也賢曰京縣屬河南郡鄭之京邑故城在今鄭州滎陽縣東南密縣屬河南郡故城在今密縣東南】自稱厭新將軍【厭一葉翻厭伏也新謂新室也】攻下潁川汝南衆十餘萬人帝使驃騎大將軍景丹建威大將軍耿弇彊弩將軍陳俊攻之茂來降【降戶江翻】封為中山王 己亥帝幸懷【懷故城在武陟縣西南十餘里賢曰縣名屬河内郡故城在懷州武陟縣西余據河内郡治懷在雒陽北百四十里】遣耿弇陳俊軍五社津【即鞏河也水經注河水東過鞏縣北於此有五社渡為五社津杜佑曰一名五渡津】備滎陽以東使吳漢率建議大將軍朱祜等十一將軍圍朱鮪於洛陽八月進幸河陽【地理志河陽縣屬河内郡】 李松自掫引兵還從更始與趙萌共攻王匡張卬於長安連戰月餘匡等敗走更始徙居長信宫【三輔黄圖曰從洛門至周廟門有長信宫在其中】赤眉至高陵【地理志高陵縣屬馮翊】王匡張卬等迎降之遂共連兵進攻東都門李松出戰赤眉生得松松弟況為城門校尉開門納之九月赤眉入長安更始單騎走從厨城門出【三輔黄圖曰洛城門王莽改曰建子門其内有長安厨官俗名之為厨城門今長安故城北面之中門是也騎奇寄翻】式侯恭以赤眉立其弟自繫詔獄聞更始敗走乃出見定陶王祉祉為之除械【為于偽翻】相與從更始於渭濱右輔都尉嚴本恐失更始為赤眉所誅即將更始至高陵【將如字領也携也挾也】本將兵宿衛其實圍之【右輔都尉治郿高陵左輔都尉治所也右恐當作左】更始將相皆降赤眉獨丞相曹竟不降手劒格死【手守又翻】 辛未詔封更始為淮陽王吏民敢有賊害者罪同大逆其送詣吏者封列侯 初宛人卓茂【卓姓也史記貨殖傳有蜀卓氏宛於元翻】寛仁恭愛恬蕩樂道【恬安恬蕩坦蕩蕩也樂音洛】雅實不為華貌行已在於清濁之間自束髮至白首未嘗與人有爭競鄉黨故舊雖行能與茂不同【行下孟翻】而皆愛慕欣欣焉哀平間為密令【宋白續通典曰密縣古鄶國密國之地左傳諸侯伐鄭圍新密漢為縣屬河南郡今縣東南三十里有故密城即漢理所】視民如子舉善而教口無惡言吏民親愛不忍欺之民嘗有言部亭長受其米肉遺者【賢曰部謂所部也遺于季翻下同】茂曰亭長為從汝求乎為汝有事囑之而受乎【囑之欲翻託也私請也】將平居自以恩意遺之乎民曰往遺之耳茂曰遺之而受何故言邪民曰竊聞賢明之君使民不畏吏吏不取民今我畏吏是以遺之吏既卒受【卒終也音子恤翻】故來言耳茂曰汝為敝民矣凡人所以羣居不亂異於禽獸者以有仁愛禮義知相敬事也汝獨不欲修之寧能高飛遠走不在人間邪吏顧不當乘威力彊請求耳亭長素善吏歲時遺之禮也民曰苟如此律何故禁之茂笑曰律設大法禮順人情今我以禮教汝汝必無怨惡以律治汝汝何所措其手足乎【治直之翻下同】一門之内小者可論大者可殺也且歸念之初茂到縣有所廢置吏民笑之鄰城聞者皆蚩其不能【蚩笑也】河南郡為置守令茂不為嫌治事自若【茂正為令郡復置守令使與茂並居郡為于偽翻】數年教化大行道不拾遺遷京部丞【王莽秉政置大司農部丞十三人勸課農桑京部丞主司隸所部】密人老少皆涕泣隨送及王莽居攝以病免歸上即位先訪求茂茂時年七十餘甲申詔曰夫名冠天下【冠古玩翻】當受天下重賞今以茂為太傅【東都之制太傳位上公絶席在三公之右】封褒德侯<br />
<br />
  臣光曰孔子稱舉善而教不能則勸【論語孔子荅季康子之言】是以堯舉臯陶湯舉伊尹而不仁者遠【論語子夏荅樊遟之言陶音遙】有德故也光武即位之初羣雄競逐四海鼎沸彼摧堅陷敵之人權略詭辯之士方見重於世而獨能取忠厚之臣旌循良之吏抜於草萊之中寘諸羣公之首宜其光復舊物享祚久長蓋由知所先務而得其本原故也<br />
<br />
  諸將圍洛陽數月朱鮪堅守不下帝以廷尉岑彭嘗為鮪校尉【朱鮪為大司馬以彭為校尉後從邑人韓歆于河内遂歸光武校戶教翻】令往說之【說輸芮翻】鮪在城上彭在城下為陳成敗【為于偽翻】鮪曰大司徒被害時鮪與其謀又諫更始無遣蕭王北伐【事並見上卷更始元年與讀曰預】誠自知罪深不敢降彭還具言於帝【還從宣翻又如字】帝曰舉大事者不忌小怨鮪今若降官爵可保况誅罰乎河水在此吾不食言【賢曰指河以為信言其明白也索隱曰左傳日食言多矣能無肥乎是謂食言為妄言】彭復往告鮪鮪從城上下索【復扶又翻下遐稼翻索昔各翻】曰必信可乘此上【上時掌翻下同】彭趣索欲上【賢曰趣向也春遇翻】鮪見其誠即許降辛卯朱鮪面縛與岑彭俱詣河陽帝解其縛召見之復令彭夜送鮪歸城明旦與蘇茂等悉其衆出降拜鮪為平狄將軍封扶溝侯【地理志扶溝縣屬淮陽郡陳留風俗傳小扶亭有洧水之溝因以名縣】後為少府傳封累世帝使侍御史河内杜詩安集洛陽將軍蕭廣縱兵士暴横【横戶孟翻】詩敕曉不改【敕戒也曉開諭也】遂格殺廣還以狀聞上召見賜以棨戟【漢雜事曰漢制假棨戟以代斧鉞崔豹古今注曰棨戟前驅之器也以木為之後代刻偽無復典刑以赤油韜之亦謂之油戟亦曰棨戟王公以下通用之以前驅】遂擢任之 冬十月癸丑車駕入洛陽幸南宫遂定都焉【蔡質漢儀曰南宫至北宫中央作大屋複道三道行天子從中道從官夾左右十步一衛兩宫相去七里】 赤眉下書曰聖公降者封為長沙王過二十日勿受更始遣劉恭請降赤眉使其將謝禄往受之更始隨祿肉袒上璽綬於盆子【璽斯氏翻綬音受】赤眉坐更始置庭中將殺之劉恭謝禄為請不能得【為于偽翻下同】遂引更始出劉恭追呼曰【呼火故翻】臣誠力極請得先死拔劒欲自刎【刎武粉翻】樊崇等遽共救止之乃赦更始封為畏威侯劉恭復為固請【復扶又翻】竟得封長沙王更始常依謝禄居劉恭亦擁護之 劉盆子居長樂宫【樂音洛】三輔郡縣營長遣使貢獻【時三輔豪傑處處屯聚各有營長長知兩翻】兵士輒剽奪之又數暴掠吏民由是皆復固守【剽匹妙翻數所角翻復扶又翻】百姓不知所歸聞鄧禹乘勝獨克而師行有紀【賢曰紀綱紀也言有條貫而不殘暴】皆望風相攜負以迎軍降者日以千數衆號百萬禹所止輒停車拄節以勞來之【勞力到翻來力代翻】父老童稺垂髪戴白滿其車下莫不感悦【賢曰垂髪童幼也戴白父老也稺直利翻】於是名震關西諸將豪桀皆勸禹徑攻長安禹曰不然今吾衆雖多能戰者少前無可仰之積【賢曰仰魚向翻】後無轉饋之資赤眉新抜長安財穀充實鋒鋭未可當也夫盜賊羣居無終日之計財穀雖多變故萬端寧能堅守者也上郡北地安定三郡土廣人稀饒穀多畜【畜許救翻謂六畜也】吾且休兵北道就糧養士以觀其敝乃可圖也於是引軍北至栒邑【賢曰栒邑縣屬右扶風故城在今豳州三水縣東北宋白曰三水縣東北二十五里邠邑原上有栒邑故城栒音荀 考異曰袁紀禹曰璽書每至輒曰無與窮赤眉争鋒按世祖賜禹書責其不攻長安不容有此語二年十一月詔徵禹還乃曰毋與窮宼争鋒袁紀誤也】所到諸營保郡邑皆開門歸附 上遣岑彭擊荆州羣賊下犨葉等十餘城【地理志犨葉二縣皆屬南陽郡賢曰犨故城在今汝州魯山縣東南葉今許州葉縣也師古曰犨音昌牛翻葉音式涉翻】 十一月甲午上幸懷 梁王永稱帝於睢陽【睢音雖】 十二月丙戌上還洛陽 三輔苦赤眉暴虐皆憐更始欲盜出之張卬等深以為慮【卬等攻更始恐其得位而禍及已故深以為慮】使謝禄縊殺之劉恭夜往收藏其尸帝詔鄧禹葬之於霸陵中郎將宛人趙熹將出武關【宛於元翻熹許計翻又許里翻】道遇更始親屬皆裸跣飢困【裸郎果翻】熹竭其資糧以與之將護而前【將送也】宛王賜聞之迎還鄉里【還從宣翻又如字】隗囂歸天水復招聚其衆【復扶又翻】興修故業自稱西州<br />
<br />
  上將軍三輔士大夫避亂者多歸囂囂傾身引接為布衣交以平陵范逡為師友【逡七倫翻】前涼州刺史河内鄭興為祭酒【前書音義曰禮飲酒必祭示有先也故稱祭酒祭祀時唯長者以酒沃酹】茂陵申屠剛杜林為治書【賢曰治書即治書侍御史治直之翻】馬援為綏德將軍楊廣王遵周宗及平襄行廵阿陽王捷【地理志平襄縣阿陽縣屬天水郡行姓廵名姓譜周有大行人之官其後氏焉】長陵王元為大將軍安陵班彪之屬為賓客由此名震西州聞於山東【聞音問】馬援少時以家用不足辭其兄況欲就邊郡田牧況曰汝大才當晚成良工不示人以朴且從所好【賢曰從其所請也少詩照翻好呼到翻下同】遂之北地田牧常謂賓客曰丈夫為志窮當益堅老當益壯後有畜數千頭穀數萬斛【畜許救翻】旣而歎曰凡殖財產貴其能賑施也【施式䜴翻】否則守錢虜耳乃盡散於親舊聞隗囂好士往從之囂甚敬重與決籌策班彪稺之子也【班稺見三十六卷平帝元始元年】 初平陵竇融累世仕宦河西知其土俗與更始右大司馬趙萌善私謂兄弟曰天下安危未可知河西殷富帶河為固張掖屬國精兵萬騎【漢邊郡皆置屬國有都尉以領之】一旦緩急杜絶河津足以自守此遺種處也【賢曰遺留也可以保全不畏絶滅種章勇翻】乃因萌求往河西萌薦融於更始以為張掖屬國都尉融旣到撫結雄桀懷輯羌虜【輯和也】甚得其歡心是時酒泉太守安定梁統【姓譜梁姓本自秦仲周平王封其少子康於夏陽梁山是為梁伯後為秦所併子孫以國為氏】金城太守庫鈞【賢曰前書音義曰庫姓即倉庫吏後也今羌中有姓庫者音舍承鈞之後也】張掖都尉茂陵史苞酒泉都尉竺曾【姓譜竺孤竹君之後本姓竺後漢擬陽侯竺晏報怨有仇以胄始名賢不改其族乃加二字以存夷齊一曰天竺國之後】敦煌都尉辛肜【敦徒門翻肜余中翻】並州郡英俊融皆與厚善及更始敗融與梁統等計議曰今天下擾亂未知所歸河西斗絶在羌胡中【賢曰斗峻絶也余謂斗僻絶也】不同心戮力則不能自守權鈞力齊復無以相率【復扶又翻】當推一人為大將軍共全五郡觀時變動議旣定而各謙讓以位次咸共推梁統統固辭乃推融行河西五郡大將軍事武威太守馬期張掖太守任仲【任音壬】並孤立無黨乃共移書告示之二人即解印綬去於是以梁統為武威太守史苞為張掖太守竺曾為酒泉太守辛肜為敦煌太守融居屬國領都尉職如故置從事監察五郡【監古衘翻】河西民俗質樸而融等政亦寛和上下相親晏然富殖脩兵馬習戰射明烽燧羌胡犯塞融輒自將與諸郡相救皆如符要【賢曰赴敵不失期契也將即亮翻】每輒破之其後羌胡皆震服親附内郡流民避凶飢者歸之不絶王莽之世天下咸思漢德安定三水盧芳居左谷中<br />
<br />
  【續漢志曰三水縣有左右谷故城在今涇州安定縣南水經註肥水出高平西北牽條山東北出峽注于高平川水東有山山東有三水縣故城本屬國都尉治】詐稱武帝曾孫劉文伯云曾祖母匈奴渾邪王之姊也常以是言誑惑安定間王莽末乃與三水屬國羌胡起兵更始至長安徵芳為騎都尉使鎮撫安定以西更始敗三水豪桀共立芳為上將軍西平王【賢曰欲平定西方故以為號】使使與西羌匈奴結和親單于以為漢氏中絶劉氏來歸我亦當如呼韓邪立之令尊事我乃使句林王將數千騎迎芳兄弟入匈奴【賢曰句古侯翻】立芳為漢帝以芳弟程為中郎將將胡騎還入安定帝以關中未定而鄧禹久不進兵賜書責之曰司徒<br />
<br />
  堯也亡賊桀也長安吏民遑遑無所依歸宜以時進討鎮慰西京繫百姓之心禹猶執前意别攻上郡諸縣更徵兵引穀歸至大要【賢曰大要縣屬北地郡】積弩將軍馮愔車騎將軍宗歆守栒邑二人爭權相攻愔遂殺歆【愔於今翻】因反擊禹禹遣使以聞帝問使人【使疏吏翻】愔所親愛為誰對曰護軍黄防帝度愔防不能久和埶必相忤【度徒洛翻忤五故翻】因報禹曰縛馮愔者必黄防也乃遣尚書宗廣持節往降之【降戶江翻】後月餘防果執愔將其衆歸罪更始諸將王匡胡殷成丹等皆詣廣降廣與東歸至安邑道欲亡廣悉斬之愔之叛也引兵西向天水隗囂逆擊破之於高平【地理志高平縣屬安定郡賢曰今原州高平縣杜佑曰原州他樓縣漢高平縣地又曰原州平高縣即漢高平縣 考異曰鄧禹傳愔叛在建武元年隗囂傳在二年蓋愔以元年冬末叛延及二年囂拜官在二年也】盡獲其輜重【重直用翻】於是禹承制遣使持節命囂為西州大將軍得專制涼州朔方事【鄧禹西征任專方面權宜命囂故曰承制言承制詔而命之也後之承制始此】 臘日赤眉設樂大會酒未行羣臣更相辯鬬【更工衡翻】而兵衆遂各踰宫斬關入掠酒肉互相殺傷衛尉諸葛稺聞之【稺直利翻】勒兵入格殺百餘人乃定劉盆子惶恐日夜啼泣從官皆憐之【從才用翻】 帝遣宗正劉延攻天井關與田邑連戰十餘合延不得進及更始敗邑遣使請降即拜為上黨太守帝又遣諫議大夫儲大伯持節徵鮑永永未知更始存亡疑不肯從收繫大伯遣使馳至長安詗問虛實【詗翾正翻伺也又古迥翻】 初帝從更始在宛【宛於元翻】納新野隂氏之女麗華【風俗通管脩自齊適楚為隂大夫其後氏焉】是歲遣使迎麗華與帝姊湖陽公主妹寧平公主俱到洛陽【賢曰寧平縣屬淮陽故城在今亳州谷陽縣西南】以麗華為貴人更始西平王李通先娶寧平公主上徵通為衛尉 初更始以王閎為琅邪太守張步據郡拒之閎諭降得贛榆等六縣【地理志贛榆縣屬琅邪郡師古曰贛音紺榆音踰賢曰贛音貢今海州東海縣也余據今人皆從顔音】收兵與步戰不勝步旣受劉永官號治兵於劇【地理志劇縣屬北海郡賢曰故城在今青州夀光縣南故紀國城也治直之翻】遣將徇泰山東萊城陽膠東北海濟南齊郡皆下之閎力不敵乃詣步相見步大陳兵而見之怒曰步有何罪君前見攻之甚閎按劒曰太守奉朝命【朝直遥翻】而文公擁兵相拒【張步字文公】閎攻賊耳何謂甚邪步起跪謝與之宴飲待為上賓令閎關掌郡事【賢曰關通也】<br />
<br />
  二年春正月甲子朔日有食之 劉恭知赤眉必敗密教弟盆子歸璽綬習為辭讓之言及正旦大會恭先曰諸君共立恭弟為帝德誠深厚立且一年殽亂日甚誠不足以相成恐死而無益願得退為庶人更求賢知唯諸君省察【知讀曰智省悉景翻】樊崇等謝曰此皆崇等罪也恭復固請【復扶又翻下同】或曰此寧式侯事邪【賢曰劉恭為式侯言衆立天子非恭所預】恭惶恐起去盆子乃下牀解璽綬叩頭曰今設置縣官而為賊如故四方怨恨不復信向此皆立非其人所致願乞骸骨避賢聖路必欲殺盆子以塞責者無所離死【賢曰離避也塞悉則翻】因涕泣嘘唏【賢曰唏與欷同】崇等及會者數百人莫不哀憐之乃皆避席頓首曰臣無狀負陛下【無狀無善狀也】請自今已後不敢復放縱因共抱持盆子帶以璽綬盆子號呼不得已【號戶刀翻】旣罷出各閉營自守三輔翕然稱天子聰明百姓爭還長安市里且滿後二十餘日復出大掠如故 力子都為其部曲所殺餘黨與諸賊會檀鄉號檀鄉賊【力依考異當作刁賢曰今兖州瑕丘縣東北有檀鄉】宼魏郡清河魏郡大吏李熊弟陸謀反城迎檀鄉【反音翻】或以告魏郡太守潁川銚期【賢曰鈍音姚姓也魏郡秦置故城在今相州安陽縣東北】期召問熊熊叩頭首服【首式救翻】願與老母俱就死期曰為吏儻不若為賊樂者可歸與老母往就陸也【賢曰必以在城中為吏不如為賊之樂即任將母往就弟樂音洛】使吏送出城熊行求得陸將詣鄴城西門【魏城治鄴城將如字】陸不勝愧感【勝音升】自殺以謝期期嗟歎以禮葬之而還熊故職於是郡中服其威信帝遣吳漢率王梁等九將軍擊檀鄉於鄴東漳水上【水經漳水源出上黨長子縣西發鳩山東過壺關屯留潞武安等縣又東出山過鄴縣】大破之十餘萬衆皆降又使梁與大將軍杜茂將兵安輯魏郡清河東郡悉平諸營保【保與堡同】三郡清静邊路流通【自雒陽至漁陽上谷路出三郡三郡旣平則邊路流通矣范史杜茂傳邊作道】 庚辰悉封諸功臣為列侯【盤州洪氏曰西京列侯其傳國皆有世次東都枝葉不蕃而史筆又簡畧】梁侯鄧禹【禹始封鄼是年改封梁侯地理志梁縣屬河南郡唐汝州治梁縣宋白曰漢梁縣故城在汝水之南】廣平侯吳漢【賢曰廣平縣屬廣平郡故城在今洺州永年縣西北】皆食四縣博士丁恭議曰古者封諸侯不過百里強幹弱枝所以為治也【治直吏翻】今封四縣不合法制帝曰古之亡國皆以無道未嘗聞功臣地多而滅亡者也隂鄉侯隂識貴人之兄也以軍功當增封識叩頭讓曰天下初定將帥有功者衆臣託屬掖廷仍加爵邑不可以示天下此為親戚受賞國人計功也【戰國公孫龍告平原君之言】帝從之帝令諸將各言所樂【樂音洛】皆占美縣【占之瞻翻】河南太守潁川丁綝獨求封本鄉或問其故綝曰綝能薄功微得鄉亭厚矣帝從其志封新安鄉侯【綝丑林翻漢法大縣侯位視三公小縣侯位視上卿鄉亭侯位視中二千石綝潁川定陵人新安鄉蓋在定陵】帝使郎中魏郡馮勤典諸侯封事勤差量功次輕重國土遠近地埶豐薄不相踰越莫不厭服焉【量音良厭於艷翻】帝以為能尚書衆事皆令總録之故事尚書郎以令史久次補之帝始用孝廉為尚書郎【百官志尚書令史十八人秩二百石侍郎三十六人秩四百石主作文書起草蔡質漢儀曰尚書郎初從三署詣臺試初上臺稱守尚書郎中歲滿稱尚書郎三年稱侍郎】 起高廟於洛陽 【考異曰帝紀正月壬子按正月甲子朔不應有壬子誤】四時合祀高祖太宗世宗建社稷于宗廟之右立郊兆于城南【續漢志曰立社稷於雒陽在宗廟之右皆方壇四面及中各依方色無屋有牆門而已白虎通曰天子之壇方五丈諸侯之壇半天子之壇社者土也人非土不立非穀不食故封土立社示有上也稷者五穀之長得隂陽中和之氣故祭之也沈約曰禮云共工氏之霸九州其子句龍曰后土能平九土故祀以為社烈山氏之有天下其子曰神農能殖百穀其裔曰柱佐顓頊為稷官主農事周弃繼之法施於人故祀以為稷禮王為羣姓立社曰太社王自為立社曰王社故國有二社而稷亦有二也漢魏則有官社無稷故常二社一稷也傅咸曰天子親耕以供粢盛親耕自報故自為立社為籍而報也國以人為本人以國為命故又為百姓立社而祈報也此社之所以有二也王肅論王社謂春祈籍田秋而報之也其論太社則曰王者布下圻内為百姓立之謂之太社不自立之京師也杜佑曰社者五土之神五土者山林川澤丘陵墳衍原隰等各有所育羣生賴之故特於吐生物處别立其名為社稷者於五土之中特指原隰之祗以五土雖各有所生而山林川澤丘陵墳衍此四者雜出材用等物於五穀之功則少且生人所急者食故於五土之中别旌異原隰之祗以報之以其能生五穀名其神但五穀不可遍言以稷為五穀之長春生秋成之主稷者原隰之中能生五穀之祗是也續漢書曰制郊兆於雒陽城南七里為壇八階中又為重壇天地位皆在壇上其外壇上為五帝位青帝位在甲寅赤帝位在丙巳黄帝位在丁未白帝位在庚申黑帝位在壬亥其外為壝重營皆紫以象紫宫營有通道以為門日月在營内南道日在東月在西北斗在北道之西外營中營凡千五百一十四神高皇帝配食焉】 長安城中糧盡赤眉收載珍寶大縱火燒宫室市里恣行殺掠長安城中無復人行【復扶又翻】乃引兵而西衆號百萬自南山轉掠城邑遂入安定北地鄧禹引兵南至長安軍昆明池謁祠高廟收十一帝神主送詣洛陽【高惠文景武昭宣元成哀平十一帝賢曰神主以木為之方尺二寸穿中央達四方諸侯王長一尺虞主用桑練主用栗衛宏漢舊儀曰已葬收主為木函藏廟太室中西壁坎中去地六尺一寸祭則立主於坎下】因廵行園陵為置吏士奉守焉【行下孟翻為于偽翻】 真定王楊造䜟記曰赤九之後癭楊為主【賢曰漢以火德故云赤也光武於高祖九代孫故云九癭於郢翻癭生於頸而附於咽】楊病癭欲以惑衆與綿曼賊交通【賢曰綿曼縣名屬真定國故城在今恒州石邑縣西北俗音訛謂之人文故城也】帝遣騎都尉陳副游擊將軍鄧隆徵之楊閉城門不内帝復遣前將軍耿純持節行幽冀所過勞慰王侯【復扶又翻行下孟翻勞力到翻】密敕收楊純至真定止傳舍【傳株戀翻】邀楊相見純眞定宗室之出也故楊不以為疑且自恃衆強而純意安静即從官屬詣之【賢曰男子謂姊妹之子為出純母蓋真定宗室之女故楊不疑而來見純】楊兄弟並將輕兵在門外【將即亮翻】楊入見純純接以禮敬因延請其兄弟皆入乃閉閤悉誅之因勒兵而出眞定震怖無敢動者【怖普布翻】帝憐楊謀未發而誅復封其子為眞定王【楊子德】 二月己酉車駕幸脩武【賢曰縣名屬河内郡本殷之甯邑韓詩外傳曰武王伐紂勒兵于甯故曰修武今懷州縣也】 鮑永馮衍審知更始已亡乃發喪出儲大伯等封上印綬悉罷兵幅巾詣河内【杜佑曰按巾六國時趙魏之間通謂之承露庶人及軍旅皆服之賢曰幅巾謂不著冠但幅巾束首也傅玄子曰漢末王公卿士多委王服以幅巾為雅是以袁紹崔鈞之徒雖為將帥皆著㡘巾上時掌翻考異曰鮑永傳稱永降於河内時攻懷未拔帝謂永曰我攻懷三日而城不下關東畏服卿可且將故人自往城下譬之即拜永諫議大夫至懷乃說更始河内太守於是開城而降按光武未都洛陽以前屢幸懷又祠高祖於懷宫並無更始河内太守據懷事本紀亦無攻懷一節按田邑書稱主亡一歲莫知定所則永衍之降必在此年而帝紀光武此年不曾幸河内但有幸脩武事然則永衍實降于脩武脩武亦河内縣也其稱降懷等事當是史誤故皆畧之】帝見永問曰卿衆安在永離席叩頭曰【離力智翻】臣事更始不能令全誠慙以其衆幸富貴故悉罷之帝曰卿言大而意不悦【帝雖謂永言大而以其降晚意懷不悦也】旣而永以立功見用【賢曰謂說下懷余按考異不取下懷事當以永討平魯郡為功也按永傳時董憲裨將討魯侵害百姓乃拜永為魯郡太守永到討擊大破之唯别帥彭豐虡休皮常等各千餘人稱將軍不肯下永以計誘手格殺豐等禽破黨與以功封關内侯遷揚州牧】衍遂廢棄永謂衍曰昔高祖賞季布之罪誅丁固之功【事見十一卷高帝五年丁固即丁公】今遭明主亦何憂哉衍曰人有挑其鄰人之妻者其長者罵而少者報之【挑徒了翻長知兩翻少詩沼翻下同】後其夫死取其長者或謂之曰夫非罵爾者邪【夫非之夫音扶】曰在人欲其報我在我欲其罵人也【賢曰此並陳軫對秦王之辭見戰國策引之者言已為故主守節亦冀新帝重之也為于偽翻】夫天命難知人道易守【易以䜴翻】守道之臣何患死亡 大司空王梁屢違詔命【梁與吳漢俱擊檀鄉詔軍事一屬漢而梁輒發野王兵帝以其不奉詔勅令止在所縣而梁復以便宜進軍是屢違詔命也】帝怒遣尚書宗廣持節即軍中斬梁廣檻車送京師旣至赦之以為中郎將北守箕關【水經註濝水出河東垣縣王屋西山濝溪夾山東南流逕故城東即濝關也光武遣王梁守之】 壬子以太中大夫京兆宋弘為大司空弘薦沛國桓譚為議郎給事中【帝以沛郡為沛國】帝令譚鼔琴愛其繁聲弘聞之不悦伺譚内出【内出從禁中出也伺相吏翻】正朝服坐府上【朝直遥翻】遣吏召之譚至不與席而讓之且曰能自改邪將令相舉以法乎譚頓首辭謝良久乃遣之後大會羣臣帝使譚鼓琴譚見弘失其常度帝怪而問之弘乃離席免冠謝曰【離力智翻】臣所以薦桓譚者望能以忠正導主而令朝廷耽悦鄭聲臣之罪也帝改容謝之湖陽公主新寡帝與共論朝臣微觀其意主曰宋公威容德器羣臣莫及帝曰方且圖之後弘被引見【被皮義翻見賢遍翻】帝令主坐屏風後【釋名屏風障風也】因謂弘曰諺言貴易交富易妻人情乎弘曰臣聞貧賤之知不可忘糟糠之妻不下堂帝顧謂主曰事不諧矣 帝之討王郎也彭寵發突騎以助軍【事見上卷二年】轉糧食前後不絶及帝追銅馬至薊寵自負其功意望甚高帝接之不能滿以此懷不平【賢曰負恃也不能滿其望故心不平也按寵傳先是吳漢北發兵帝遺寵以所服劒倚為北道主人及追銅馬北至薊寵來謁謂當迎閤握手交歡並坐帝接之不能滿其意所以失望】及即位吳漢王梁寵之所遣【事見上卷更始二年】並為三公而寵獨無所加愈怏怏不得志【怏於兩翻】歎曰如此我當為王但爾者陛下忘我邪【爾猶言如此也】是時北州破散而漁陽差完有舊鐵官【地理志漁陽郡漁陽有鐵官】寵轉以貿穀積珍寶益富彊【貿音茂】幽州牧朱浮年少有俊才欲厲風迹【賢曰風化之迹也少詩照翻】收士心辟召州中名宿【有名耆宿之士】及王莽時故吏二千石皆引置幕府多發諸郡倉穀稟贍其妻子寵以為天下未定師旅方起不宜多置官屬以損軍實【賢曰謂甲兵糧儲也左傳曰隳軍實也】不從其令浮性矜急自多【賢曰矜誇多自取也】寵亦狠彊嫌怨轉積浮數譛構之密奏寵多聚兵穀意計難量【狠戶墾翻數所角翻量音良】上輒漏泄令寵聞以脅恐之【恐欺用翻又如字】至是有詔徵寵寵上疏願與浮俱徵帝不許寵益以自疑其妻素剛不堪抑屈固勸無受徵曰天下未定四方各自為雄漁陽大郡兵馬㝡精何故為人所奏而棄此去乎寵又與所親信吏計議皆懷怨於浮莫有勸行者帝遣寵從弟子后蘭卿喻之寵因留子后蘭卿遂發兵反拜署將帥自將二萬餘人攻朱浮於薊又以與耿況俱有重功而恩賞並薄數遣使邀誘況【要一遥翻】況不受斬其使延岑復反圍南鄭【岑降嘉見上卷更始二年復扶又翻下同】漢中王嘉兵<br />
<br />
  敗走岑遂據漢中進兵武都【地理志武都縣屬武都郡】為更始柱功侯李寶所破岑走天水【走音奏】公孫述遣將侯丹取南鄭嘉收散卒得數萬人以李寶為相從武都南擊侯丹不利還軍河池下辨【賢曰河池縣屬武都郡一名仇池今鳳州縣也下辨道亦屬武都郡今成州同谷縣師古曰辨音皮莧翻】復與延岑連戰岑引北入散關至陳倉【賢曰散關故城在今陳倉縣南十里有散谷水因取名焉地理志陳倉縣屬右抉風唐為寶雞縣屬岐州】嘉追擊破之公孫述又遣將軍任滿從閬中下江州東據扞關【賢曰閬中江州皆縣名並屬巴郡閬中今隆州縣也江州故城在渝州巴縣西宋白曰今渝州江津縣本漢江州縣史記曰楚肅王為扞關以拒蜀故基在今峽州巴山縣】於是盡有益州之地【漢益州部漢中巴郡廣漢蜀郡犍為牂柯越嶲益州等郡】 辛卯上還洛陽 三月乙未大赦 更始諸大將在南方未降者尚多帝召諸將議兵事以檄叩地曰郾最彊宛為次誰當擊之【叩去后翻又丘候翻師古曰郾一戰翻宛於元翻】賈復率然對曰【率然輕遽之貌】臣請擊郾帝笑曰執金吾擊郾吾復何憂【復扶又翻】大司馬當擊宛遂遣復擊郾破之尹尊降又東擊更始淮陽太守暴汜汜降【賢曰淮陽故城在今陳州宛邱縣東南汜音汎又音凡降戶江翻】 夏四月虎牙大將軍蓋延【蓋古盍翻】督駙馬都尉馬武等四將軍擊劉永破之遂圍永於睢陽故更始將蘇茂反【茂隨朱鮪降今復反】殺淮陽太守潘蹇【姓譜周文王之子季孫食采於潘因氏焉晉有潘父楚有潘崇】據廣樂而臣於永【賢曰廣樂地闕今宋州虞城縣有長樂故城蓋避隋煬帝諱改】永以茂為大司馬淮陽王 吳漢擊宛宛王賜奉更始妻子詣洛陽降帝封賜為慎侯【賢曰慎縣屬汝南郡故城在今潁州潁上縣西北】叔父良族父歙族兄祉皆自長安來甲午封良為廣陽王祉為城陽王【宛王賜於光武為族兄更始近屬也歙許及翻亦更始近屬更始封為元氏王祉族兄舂陵康侯敞之子光武之族姪而舂陵節侯買之嫡曾孫也更始封為定陶王】又封兄縯子章為太原王興為魯王更始三子求歆鯉皆為列侯【求為襄邑侯歆為穀熟侯鯉為夀光侯】 鄧王王常降帝見之甚歡曰吾見王廷尉不憂南方矣【更始以王常為廷尉故帝稱之常降則得南陽一郡故云不憂南方】拜為左曹【前書百官表左右曹加官受尚書事此時蓋為專官也】封山桑侯【賢曰山桑縣屬沛郡今亳州縣】五月庚辰封族父歙為泗水王 帝以隂貴人雅性<br />
<br />
  寛仁欲立以為后貴人以郭貴人有子【西都後宫之號十四等未有貴人光武中興斵琱為樸六宫稱號唯有皇后貴人貴人金印紫綬奉不過數十斛】終不肯當六月戊戌立貴人郭氏為皇后立其子彊為皇太子大赦丙午封泗水王子終為淄川王【終歙子也與帝少相親愛故封為王】<br />
<br />
  秋賈復南擊召陵新息平之【召陵新息二縣並屬汝南郡賢曰新息故城在今豫州新息縣西南】復部將殺人於潁川潁川太守宼恂捕得繫獄時尚草創軍營犯法率多相容恂戮之於市復以為恥還過潁川謂左右曰吾與寇恂並列將帥而為其所䧟今見恂必手劒之恂知其謀不欲與相見姊子谷崇曰崇將也得帶劒侍側卒有變足以相當【卒讀曰猝】恂曰不然昔藺相如不畏秦王而屈於亷頗者為國也【事見四卷周赧王三十六年為于偽翻】乃敕屬縣盛供具儲酒醪【說文曰醪兼汁滓酒】執金吾軍入界一人皆兼二人之饌【賢曰饌具也雛晥翻又音雛戀翻】恂出迎於道稱疾而還復勒兵欲追之而吏士皆醉遂過去恂遣谷崇以狀聞帝乃徵恂恂至引見時賈復先在坐【坐徂卧翻】欲起相避帝曰天下未定兩虎安得私鬬今日朕分之【分猶解也】於是並坐極歡遂共車同出結友而去 八月帝自率諸將征五校【校戶敎翻】丙辰幸内黄【賢曰内黄縣屬魏郡今相州縣】大破五校於羛陽降其衆五萬人【賢曰羛陽縣名屬魏郡故城在今相州堯城縣東余據左傳晉荀盈如齊逆女還卒於戲陽杜預註内黄縣北有戲陽城堯城縣本漢内黄縣隋開皇十八年更名唐未改宋定縣戲與羛同許宜翻降戶江翻下同】 帝遣游擊將軍鄧隆助朱浮討彭寵隆軍潞南浮軍雍奴遣吏奏狀【潞雍奴二縣皆屬漁陽郡水經曰鮑邱水過潞縣南曰潞河鄧隆軍於是水之南為彭寵所破宋白曰幽州武清縣本漢雍奴縣水經注云雍奴藪澤之名四面有水曰雍水不流曰奴】帝讀檄怒謂使吏曰【遣吏來使故曰使吏使疏吏翻】營相去百里其埶豈可得相及比若還【若汝也比必寐翻及也】北軍必敗矣彭寵果遣輕兵擊隆軍大破之浮遠遂不能救 蓋延圍睢陽數月克之【蓋古盍翻】劉永走至虞【賢曰虞縣屬梁國故城在今宋州虞城縣】虞人反殺其母妻永與麾下數十人奔譙【地理志譙縣屬沛郡賢曰今亳州縣】蘇茂佼彊周建合軍三萬餘人救永延與戰於沛西【地理志沛縣屬沛郡賢曰今徐州縣佼音絞又音効】大破之永彊建走保湖陵【地理志湖陵縣屬山陽郡】茂犇還廣樂延遂定沛楚臨淮【三郡也】帝使太中大夫伏隆持節使青徐二州招降郡國青徐羣盜聞劉永破敗皆惶怖請降【怖普布翻】張步遣其掾孫昱隨隆詣闕上書獻鰒魚【鰒步各翻】隆湛之子也 堵鄉人董訢反宛城【水經註曰堵水南經小堵鄉賢曰在今唐州方城縣堵音者宛於元翻】執南陽太守劉驎揚化將軍堅鐔攻宛拔之【驎離珍翻鐔徒含翻堅姓鐔名】訢走還堵鄉 吳漢徇南陽諸縣所過多侵暴破虜將軍鄧奉謁歸新野【謁歸謁告而歸也】怒漢掠其鄉里遂反擊破漢軍屯據淯陽與諸賊合從【從子容翻】 九月壬戌帝自内黄還 陜賊蘇况攻破弘農【陜失冉翻】帝使景丹討之會丹薨征虜將軍祭遵擊弘農栢華蠻中賊皆平之【東觀記曰栢華聚也酈道元曰河南郡新城縣故蠻子國也縣有鄤聚今名蠻中括地志故麻城謂之蠻中在汝州梁縣界祭則介翻】 赤眉引兵欲西上隴【隴縣屬天水郡有大坂名隴坻三秦記曰其坂九回不知高幾許欲上者七日乃越高處可容百餘家清水四注下郭仲產秦州記曰隴山東西百八十里登山嶺東望秦川四五百里極目泯然山東人行役升此而顧瞻者莫不悲思故歌曰隴頭流水分離四下念我行役飄然曠野登高遠望涕零雙墮度汧隴無蠶桑八月乃麥五月乃凍解】隗囂遣將軍楊廣迎擊破之又追敗之於烏氏涇陽間【烏氏涇陽二縣屬安定郡賢曰烏氏故城在今涇州安定縣東四十里涇陽故城在今原州平高縣之南敗補邁翻氏音支】赤眉至陽城番須中【酈道元曰陽城在安民縣成帝永始二年罷安定呼他苑以為安民縣賢曰番須口與回中相近並在汧番音盤】逢大雪坑谷皆滿士多凍死乃復還發掘諸陵【復扶又翻下同】取其寶貨凡有玉匣殮者率皆如生【殮力贍翻】賊遂汙辱呂后尸【汙烏故翻關中記呂后合葬長陵高祖陵在西呂后陵在東】鄧禹遣兵擊之於郁夷【地理志郁夷縣屬右扶風】反為所敗【敗補邁翻】禹乃出之雲陽【地理志雲陽縣屬左馮翊】赤眉復入長安延岑屯杜陵【賢曰縣名屬京兆周之杜伯國在今萬年縣東南】赤眉將逢安擊之鄧禹以安精兵在外引兵襲長安會謝禄救至禹兵敗走延岑擊逢安大破之【逢音龎】死者十餘萬人廖湛將赤眉十八萬攻漢中王嘉嘉與戰於谷口【地理志谷口縣屬馮翊賢曰故城在今醴泉縣東北四十里水經註曰涇水東經九嵕山東中山西謂之谷口杜佑曰谷口今雲陽縣治谷是宋白曰當涇水所出之處故謂之谷口廖力弔翻又力救翻】大破之嘉手殺湛遂到雲陽就穀嘉妻兄新野來歙帝之姑子也帝令鄧禹招嘉嘉因歙詣禹降【歙許及翻】李寶倨慢禹斬之 【考異曰更始柱功侯李寶時為劉嘉相此蓋别一人同姓名余參考范書究其本末漢中王嘉即以更始柱功侯李寶為相禹誅之非别一人也】 冬十一月以廷尉岑彭為征南大將軍帝於大會中指王常謂羣臣曰此家率下江諸將輔翼漢室心如金石眞忠臣也【此家猶言此人也】即日拜常為漢忠將軍使與岑彭率建義大將軍朱祜等七將軍討鄧奉董訢彭等先擊堵鄉【堵音者】鄧奉救之朱祜軍敗為奉所獲 銅馬青犢尤來餘賊共立孫登為天子登將樂玄殺登以其衆五萬餘人降 鄧禹自馮愔叛後威名稍損又乏糧食戰數不利【數所角翻】歸附者日益離散赤眉延岑暴亂三輔郡縣大姓各擁兵衆禹不能定帝乃遣偏將軍馮異代禹討之車駕送至河南【地理志河南縣屬河南郡故郟鄏地周武王遷九鼎周公營以為都是為王城雒陽周公遷殷民是為成周晉地道記河南城去雒城四十里宋白曰河南縣周平王徙居于此至敬王乃徙居成周漢為河南縣歷魏晉及後魏皆理於唐苑城東北隅】敕異曰三輔遭王莽更始之亂重以赤眉延岑之醜【重直用翻】元元塗炭【賢曰塗炭者若䧟泥墜火喻窮困之極也】無所依訴將軍今奉辭討諸不軌營保降者遣其渠帥詣京師【帥所類翻】散其小民令就農桑壞其營壁無使復聚【壞音怪復扶又翻】征伐非必畧地屠城要在平定安集之耳諸將非不健鬬然好虜掠【好呼到翻】卿本能御吏士念自修敕無為郡縣所苦異頓首受命引而西所至布威信羣盜多降<br />
<br />
  臣光曰昔周人頌武王之德曰敷時繹思我徂惟求定【周頌賚之詩也敷布也繹陳也徂往也求定謂安天下也】言王者之兵志在布陳威德安民而已觀光武之所以取關中用是道也豈不美哉<br />
<br />
  又詔徵鄧禹還曰慎毋與窮宼爭鋒【窮宼者言其勢已窮埶必致死也兵法曰窮宼勿追】赤眉無穀自當來東吾以飽待飢以逸待勞【孫武子之言也】折箠笞之【箠杖也折杖笞之言易也】非諸將憂也無得復妄進兵【復扶又翻下同】帝以伏隆為光禄大夫復使於張步【使疏吏翻】拜步東萊太守并與新除青州牧守都尉俱東詔隆輒拜令長以下【令力政翻長知兩翻】 十二月戊午詔宗室列侯為王莽所絶者皆復故國【王莽始建國二年免漢宗室列侯為民事見三十七卷復如字】三輔大饑人相食城郭皆空白骨蔽野遺民往往聚為營保各堅壁清野赤眉虜掠無所得乃引而東歸衆尚二十餘萬隨道復散【復扶又翻】帝遣破姦將軍侯進等屯新安建威大將軍耿弇等屯宜陽以要其還路【地理志新安宜陽二縣皆屬弘農郡要與邀同】敕諸將曰賊若東走可引宜陽兵會新安賊若南走可引新安兵會宜陽馮異與赤眉遇於華隂【華戶化翻】相距六十餘日戰數十合降其將卒五千餘人<br />
<br />
  資治通鑑卷四十<br />
<br />
<史部,編年類,資治通鑑>  <br>
   </div> 

<script src="/search/ajaxskft.js"> </script>
 <div class="clear"></div>
<br>
<br>
 <!-- a.d-->

 <!--
<div class="info_share">
</div> 
-->
 <!--info_share--></div>   <!-- end info_content-->
  </div> <!-- end l-->

<div class="r">   <!--r-->



<div class="sidebar"  style="margin-bottom:2px;">

 
<div class="sidebar_title">工具类大全</div>
<div class="sidebar_info">
<strong><a href="http://www.guoxuedashi.com/lsditu/" target="_blank">历史地图</a></strong>  
<a href="http://www.880114.com/" target="_blank">英语宝典</a>  
<a href="http://www.guoxuedashi.com/13jing/" target="_blank">十三经检索</a> 
<br><strong><a href="http://www.guoxuedashi.com/gjtsjc/" target="_blank">古今图书集成</a></strong> 
<a href="http://www.guoxuedashi.com/duilian/" target="_blank">对联大全</a> <strong><a href="http://www.guoxuedashi.com/xiangxingzi/" target="_blank">象形文字典</a></strong> 

<br><a href="http://www.guoxuedashi.com/zixing/yanbian/">字形演变</a>  <strong><a href="http://www.guoxuemi.com/hafo/" target="_blank">哈佛燕京中文善本特藏</a></strong>
<br><strong><a href="http://www.guoxuedashi.com/csfz/" target="_blank">丛书&方志检索器</a></strong> <a href="http://www.guoxuedashi.com/yqjyy/" target="_blank">一切经音义</a>  

<br><strong><a href="http://www.guoxuedashi.com/jiapu/" target="_blank">家谱族谱查询</a></strong>  <strong><a href="http://shufa.guoxuedashi.com/sfzitie/" target="_blank">书法字帖欣赏</a></strong> 
<br>

</div>
</div>


<div class="sidebar" style="margin-bottom:0px;">

<font style="font-size:22px;line-height:32px">QQ交流群9:489193090</font>


<div class="sidebar_title">手机APP 扫描或点击</div>
<div class="sidebar_info">
<table>
<tr>
	<td width=160><a href="http://m.guoxuedashi.com/app/" target="_blank"><img src="/img/gxds-sj.png" width="140"  border="0" alt="国学大师手机版"></a></td>
	<td>
<a href="http://www.guoxuedashi.com/download/" target="_blank">app软件下载专区</a><br>
<a href="http://www.guoxuedashi.com/download/gxds.php" target="_blank">《国学大师》下载</a><br>
<a href="http://www.guoxuedashi.com/download/kxzd.php" target="_blank">《汉字宝典》下载</a><br>
<a href="http://www.guoxuedashi.com/download/scqbd.php" target="_blank">《诗词曲宝典》下载</a><br>
<a href="http://www.guoxuedashi.com/SiKuQuanShu/skqs.php" target="_blank">《四库全书》下载</a><br>
</td>
</tr>
</table>

</div>
</div>


<div class="sidebar2">
<center>


</center>
</div>

<div class="sidebar"  style="margin-bottom:2px;">
<div class="sidebar_title">网站使用教程</div>
<div class="sidebar_info">
<a href="http://www.guoxuedashi.com/help/gjsearch.php" target="_blank">如何在国学大师网下载古籍?</a><br>
<a href="http://www.guoxuedashi.com/zidian/bujian/bjjc.php" target="_blank">如何使用部件查字法快速查字?</a><br>
<a href="http://www.guoxuedashi.com/search/sjc.php" target="_blank">如何在指定的书籍中全文检索?</a><br>
<a href="http://www.guoxuedashi.com/search/skjc.php" target="_blank">如何找到一句话在《四库全书》哪一页?</a><br>
</div>
</div>


<div class="sidebar">
<div class="sidebar_title">热门书籍</div>
<div class="sidebar_info">
<a href="/so.php?sokey=%E8%B5%84%E6%B2%BB%E9%80%9A%E9%89%B4&kt=1">资治通鉴</a> <a href="/24shi/"><strong>二十四史</strong></a>&nbsp; <a href="/a2694/">野史</a>&nbsp; <a href="/SiKuQuanShu/"><strong>四库全书</strong></a>&nbsp;<a href="http://www.guoxuedashi.com/SiKuQuanShu/fanti/">繁体</a>
<br><a href="/so.php?sokey=%E7%BA%A2%E6%A5%BC%E6%A2%A6&kt=1">红楼梦</a> <a href="/a/1858x/">三国演义</a> <a href="/a/1038k/">水浒传</a> <a href="/a/1046t/">西游记</a> <a href="/a/1914o/">封神演义</a>
<br>
<a href="http://www.guoxuedashi.com/so.php?sokeygx=%E4%B8%87%E6%9C%89%E6%96%87%E5%BA%93&submit=&kt=1">万有文库</a> <a href="/a/780t/">古文观止</a> <a href="/a/1024l/">文心雕龙</a> <a href="/a/1704n/">全唐诗</a> <a href="/a/1705h/">全宋词</a>
<br><a href="http://www.guoxuedashi.com/so.php?sokeygx=%E7%99%BE%E8%A1%B2%E6%9C%AC%E4%BA%8C%E5%8D%81%E5%9B%9B%E5%8F%B2&submit=&kt=1"><strong>百衲本二十四史</strong></a>  <a href="http://www.guoxuedashi.com/so.php?sokeygx=%E5%8F%A4%E4%BB%8A%E5%9B%BE%E4%B9%A6%E9%9B%86%E6%88%90&submit=&kt=1"><strong>古今图书集成</strong></a>
<br>

<a href="http://www.guoxuedashi.com/so.php?sokeygx=%E4%B8%9B%E4%B9%A6%E9%9B%86%E6%88%90&submit=&kt=1">丛书集成</a> 
<a href="http://www.guoxuedashi.com/so.php?sokeygx=%E5%9B%9B%E9%83%A8%E4%B8%9B%E5%88%8A&submit=&kt=1"><strong>四部丛刊</strong></a>  
<a href="http://www.guoxuedashi.com/so.php?sokeygx=%E8%AF%B4%E6%96%87%E8%A7%A3%E5%AD%97&submit=&kt=1">說文解字</a> <a href="http://www.guoxuedashi.com/so.php?sokeygx=%E5%85%A8%E4%B8%8A%E5%8F%A4&submit=&kt=1">三国六朝文</a>
<br><a href="http://www.guoxuedashi.com/so.php?sokeytm=%E6%97%A5%E6%9C%AC%E5%86%85%E9%98%81%E6%96%87%E5%BA%93&submit=&kt=1"><strong>日本内阁文库</strong></a> <a href="http://www.guoxuedashi.com/so.php?sokeytm=%E5%9B%BD%E5%9B%BE%E6%96%B9%E5%BF%97%E5%90%88%E9%9B%86&ka=100&submit=">国图方志合集</a> <a href="http://www.guoxuedashi.com/so.php?sokeytm=%E5%90%84%E5%9C%B0%E6%96%B9%E5%BF%97&submit=&kt=1"><strong>各地方志</strong></a>

</div>
</div>


<div class="sidebar2">
<center>

</center>
</div>
<div class="sidebar greenbar">
<div class="sidebar_title green">四库全书</div>
<div class="sidebar_info">

《四库全书》是中国古代最大的丛书,编撰于乾隆年间,由纪昀等360多位高官、学者编撰,3800多人抄写,费时十三年编成。丛书分经、史、子、集四部,故名四库。共有3500多种书,7.9万卷,3.6万册,约8亿字,基本上囊括了古代所有图书,故称“全书”。<a href="http://www.guoxuedashi.com/SiKuQuanShu/">详细>>
</a>

</div> 
</div>

</div>  <!--end r-->

</div>
<!-- 内容区END --> 

<!-- 页脚开始 -->
<div class="shh">

</div>

<div class="w1180" style="margin-top:8px;">
<center><script src="http://www.guoxuedashi.com/img/plus.php?id=3"></script></center>
</div>
<div class="w1180 foot">
<a href="/b/thanks.php">特别致谢</a> | <a href="javascript:window.external.AddFavorite(document.location.href,document.title);">收藏本站</a> | <a href="#">欢迎投稿</a> | <a href="http://www.guoxuedashi.com/forum/">意见建议</a> | <a href="http://www.guoxuemi.com/">国学迷</a> | <a href="http://www.shuowen.net/">说文网</a><script language="javascript" type="text/javascript" src="https://js.users.51.la/17753172.js"></script><br />
  Copyright &copy; 国学大师 古典图书集成 All Rights Reserved.<br>
  
  <span style="font-size:14px">免责声明:本站非营利性站点,以方便网友为主,仅供学习研究。<br>内容由热心网友提供和网上收集,不保留版权。若侵犯了您的权益,来信即刪。scp168@qq.com</span>
  <br />
ICP证:<a href="http://www.beian.miit.gov.cn/" target="_blank">鲁ICP备19060063号</a></div>
<!-- 页脚END --> 
<script src="http://www.guoxuedashi.com/img/plus.php?id=22"></script>
<script src="http://www.guoxuedashi.com/img/tongji.js"></script>

</body>
</html>
