\chapter{資治通鑑卷二百三十八}
宋 司馬光 撰

胡三省 音註

唐紀五十四|{
	起屠維赤奮若七月盡玄黓執徐九月凡三年有奇}


憲宗昭文章武大聖至神孝皇帝上之下

元和四年秋七月壬戌御史中丞李夷簡彈京兆尹楊憑前為江西觀察使貪汚僭侈丁卯貶憑臨賀尉|{
	臨賀漢縣屬蒼梧郡以臨賀水故名唐帶賀州}
夷簡元懿之玄孫也|{
	鄭王元懿高祖之子}
上命盡籍憑資產|{
	財物田園人資以生謂之資產}
李絳諫曰舊制非反逆不籍其家上乃止憑之親友無敢送者櫟陽尉徐晦獨至藍田與别|{
	櫟音藥}
太常卿權德輿素與晦善謂之曰君送楊臨賀誠為厚矣無乃為累乎|{
	累良瑞翻}
對曰晦自布衣蒙楊公知奬今日遠謫豈得不與之别借如明公它日為讒人所逐晦敢自同路人乎德輿嗟嘆稱之於朝|{
	朝直遥翻}
後數日李夷簡奏為監察御史晦謝曰晦平生未嘗得望公顔色公何從而取之夷簡曰君不負楊臨賀肯負國乎 上密問諸學士曰今欲用王承宗為成德留後割德棣二州更為一鎮以離其勢并使承宗輸二税請官吏一如師道何如|{
	李師道事見上卷元年}
李絳等對曰德棣之隸成德為日已久|{
	貞元初王武俊破朱滔取德棣}
今一且割之恐承宗及其將士憂疑怨望得以為辭况其鄰道情狀一同各慮他日分割或潛相構扇萬一旅拒倍難處置願更三思|{
	旅衆也旅拒者挾衆而拒上命也處昌呂翻三息暫翻又如字}
所是二税官吏願因弔祭使至彼自以其意諭承宗令上表陳乞如師道例勿令知出陛下意如此則幸而聽命於理固順若其不聽體亦無損上又問今劉濟田季安皆有疾若其物故|{
	物故注已見漢紀史炤曰顔師古曰物故死也言其同於鬼物而故也一曰不欲斥言 云其所服用之物皆已故也}
豈可盡如成德付授其子天下何時當平議者皆言宜乘此際代之不受則發兵討之時不可失如何對曰羣臣見陛下西取蜀東取吳|{
	蜀謂劉闢吳謂李錡}
易於反掌|{
	易以豉翻下同}
故謟諛躁競之人|{
	躁輕也競爭也}
爭獻策畫勸開河北不為國家深謀遠慮|{
	為于偽翻}
陛下亦以前日成功之易而信其言臣等夙夜思之河北之勢與二方異何則西川浙西皆非反側之地其四鄰皆國家臂指之臣|{
	臂指用賈誼語意言其順使也}
劉闢李錡獨生狂謀其下皆莫之與闢錡徒以貨財㗖之大軍一臨則渙然離耳故臣等當時亦勸陛下誅之以其萬全故也成德則不然内則膠固歲深外則蔓連勢廣|{
	膠固如膠之附著堅固也蔓連如蔓草之曼衍連屬也}
其將士百姓懷其累代喣嫗之恩|{
	喣吁句翻嫗衣遇翻鄭玄曰氣曰喣體曰嫗}
不知君臣逆順之理諭之不從威之不服將為朝廷羞又鄰道平居或相猜恨及聞代易必合為一心蓋各為子孫之謀亦慮他日及此故也萬一餘道或相表裏兵連禍結財盡力竭西戎北狄乘間窺窬|{
	西戎謂吐蕃北狄謂回鶻間古莧翻下同}
其為憂患可勝道哉|{
	勝音升}
濟季安與承宗事體不殊若物故之際有間可乘當臨事圖之於今用兵則恐未可太平之業非朝夕可致願陛下審處之|{
	處昌呂翻}
時吳少誠病甚絳等復上言少誠病必不起|{
	復扶又翻上時掌翻}
淮西事體與河北不同四旁皆國家州縣不與賊鄰無黨援相助朝廷命帥|{
	帥所類翻}
今正其時萬一不從可議征討臣願捨恒冀難致之策就申蔡易成之謀脱或恒冀連兵事未如意蔡州有舋勢可興師南北之役俱興財力之用不足儻事不得已須赦承宗|{
	絳等之言後無不驗}
則恩德虚施威令頓廢不如早賜處分|{
	處昌呂翻分扶問翻}
以收鎮冀之心|{
	此時未改恒州為鎮州史以後來所改州名書之耳}
坐待機宜必獲申蔡之利既而承宗久未得朝命頗懼累表自訴八月壬午上乃遣京兆少尹裴武詣真定宣慰|{
	恒州古真定}
承宗受詔甚㳟曰三軍見廹不暇俟朝旨議獻德棣二州以明懇欵 丙申安南都護張舟奏破環王三萬衆|{
	林邑國至德後改號環王}
九月甲辰朔裴武復命庚戍以承宗為成德節度使恒冀深趙州觀察使德州刺史薛昌朝為保信軍節度德棣二州觀察使|{
	考異曰李司空論事初武衘命使鎮州令諭王承宗割德棣兩州歸朝廷武飛表上言一如朝廷意旨遂除昌朝德棣節度及旌節至德州而昌朝已追到鎮州朝命遂不行比及武還事宜與先上表參差按實録甲辰武至自鎮州庚戌除昌朝非武未還據所上表除之也論事集誤今從實録}
昌朝嵩之子|{
	薛嵩亦安史舊將代宗初來降}
王氏之壻也故就用之田季安得飛報先知之使謂承宗曰昌朝隂與朝廷通故受節鉞承宗遽遣數百騎馳入德州執昌朝至真定囚之中使送昌朝節過魏州季安陽為宴勞留使者累日比至德州已不及矣|{
	勞力到翻比必利翻及也}
上以裴武為欺罔又有譖之者曰武使還|{
	使疏吏翻還音旋}
先宿裴垍家明旦乃入見上怒甚以語李絳|{
	見賢遍翻語牛倨翻}
欲貶武於嶺南絳曰武昔陷李懷光軍中守節不屈|{
	蓋貞元初李懷光據河中時也}
豈容今日遽為姦回蓋賊多變詐人未易盡其情承宗始懼朝廷誅討故請獻二州既蒙恩貸而鄰道皆不欲成德開分割之端計必有間說誘而脅之使不得守其初心者|{
	李絳可謂洞見田季安王承宗之情間古莧翻說式芮翻誘音酉}
非武之罪也今陛下選武使入逆亂之地使還一語不相應遽竄之遐荒臣恐自今奉使賊庭者以武為戒苟求便身率為依阿兩可之言|{
	史炤曰依阿謂不特立其說常附順人言兩可謂無所可否}
莫肯盡誠具陳利害如此非國家之利也且垍武久處朝廷|{
	處昌呂翻}
諳練事體|{
	諳烏含翻}
豈有使還未見天子而先宿宰相家乎臣敢為陛下必保其不然|{
	為于偽翻}
此殆有讒人欲傷武及垍者願陛下察之上良久曰理或有此遂不問 丙辰振武奏吐蕃五萬餘騎至拂梯泉|{
	史炤曰拂薄勿切梯天黎切本又作鸊鵜泉在豐州西受降城北三百里}
辛未豐州奏吐蕃萬餘騎至大石谷掠回鶻入貢還國者 左神策軍吏李昱貣長安富人錢八千緡滿三歲不償|{
	貣吐得翻假貣也}
京兆尹許孟容收捕械繫立期使償曰期滿不足當死一軍大驚中尉訴於上上遣中使宣旨付本軍孟容不之遣中使再至孟容曰臣不奉詔當死然臣為陛下尹京畿|{
	京兆以長安萬年為京縣餘屬縣為畿縣}
非抑制豪彊何以肅清輦下錢未畢償昱不可得上嘉其剛直而許之京城震栗 上遣中使諭王承宗使遣薛昌朝還鎮|{
	使之遣還德州}
承宗不奉詔冬十月癸未制削奪承宗官爵以左神策中尉吐突承璀為左右神策河中河陽浙西宣歙等道行營兵馬使招討處置等使|{
	唐實録云開元二十年置諸道採訪處置使專以觀省風俗黜陟官吏其後伐叛討有罪則置招討處置使處昌呂翻}
翰林學士白居易上奏以為國家征伐當責成將帥近歲始以中使為監軍自古及今未有徵天下之兵專令中使統領者也今神策軍既不置行營節度使則承璀乃制將也|{
	制將言諸軍進退皆受制於承璀將即亮翻}
又充諸軍招討處置使則承璀乃都統也|{
	都統謂都統諸軍唐中世以後專征之任}
臣恐四方聞之必窺朝廷四夷聞之必笑中國|{
	白居易之言自春秋書多魚漏師左傳夙沙衛殿齊師來况吐突承璀以寺人專征乎崇觀間金人有所侮而動正如此}
陛下忍令後代相傳云以中官為制將都統自陛下始乎臣又恐劉濟茂昭及希朝從史乃至諸道將校皆恥受承璀指麾心既不齊功何由立此是資承宗之計而挫諸將之勢也陛下念承璀勤勞貴之可也憐其忠赤富之可也至於軍國權柄動關理亂朝廷制度出自祖宗陛下寧忍徇下之情而自隳法制從人之欲而自損聖明何不思於一時之間而取笑於萬代之後乎時諫官御史論承璀職名太重者相屬|{
	屬之欲翻}
上皆不聽戊子上御延英殿度支使李元素鹽鐵使李鄘京兆尹許孟容御史中丞李夷簡給事中呂元膺穆質右補闕獨孤郁等極言其不可 |{
	考異曰舊承璀傳曰諫官御史上疏相相屬言自古無中貴人為兵馬統帥者補闕獨孤郁段平仲尤激切呂元膺傳元膺與給事中穆質孟簡兵部侍郎許孟容等八人抗論不可若据承璀傳則是九人又平仲時為諫議大夫非補闕恐誤今從實録}
上不得已明日削承璀四道兵馬使改處置為宣慰而已李絳嘗極言宦官驕横|{
	横戶孟翻}
侵害政事讒毁忠貞上曰此屬安敢為讒就使為之朕亦不聽絳曰此屬大抵不知仁義不分枉直惟利是嗜得賂則譽跖蹻為亷良怫意則毁龔黄為貪暴|{
	李奇曰跖秦大盗也楚之大盗為莊蹻師古曰莊周云跖柳下惠之弟蓋寓言也龔黄龔遂黄霸也譽音余蹻居畧翻怫符弗翻}
能用傾巧之智構成疑似之端朝夕左右浸潤以入之陛下必有時而信之矣自古宦官敗國者|{
	敗蒲邁翻}
備載方冊陛下豈得不防其漸乎己亥吐突承璀將神策兵發長安命恒州四面藩鎮各進兵招討 初吳少誠寵其大將吳少陽名以從弟|{
	從才用翻}
署為軍職出入少誠家如至親累遷申州刺史少誠病不知人家僮鮮于熊兒詐以少誠命召少陽攝副使知軍州事少誠有子元慶少陽殺之十一月己巳少誠薨少陽自為留後 是歲雲南王尋閤勸卒子勸龍晟立 田季安聞吐突承璀將兵討王承宗聚其徒曰師不跨河二十五年矣|{
	自德宗討田悦不克王師不復跨河}
今一旦越魏伐趙趙虜魏亦虜矣計為之奈何其將有超伍而言者|{
	超伍出位而言也蓋超出儔伍之中而言}
曰願借騎五千以除君憂季安大呼曰壯哉兵决出格沮者斬|{
	呼火故翻格音閣}
幽州牙將絳人譚忠為劉濟使魏|{
	為于偽翻使疏吏翻}
知其謀入謂季安曰如某之謀是引天下之兵也何者今王師越魏伐趙不使耆臣宿將而專付中臣|{
	耆老也宿舊也}
不輸天下之甲而多出秦甲|{
	關中之地古秦地也故謂關中之兵為秦甲}
君知誰為之謀此乃天子自為之謀欲將夸服於臣下也|{
	夸服謂欲自衒於筭畧以服臣下之心}
若師未叩趙而先碎於魏是上之謀反不如下且能不恥於天下乎既恥且怒必任智士畫長策仗猛將練精兵畢力再舉涉河鑑前之敗必不越魏而伐趙校罪輕重必不先趙而後魏|{
	先悉薦翻後戶遘翻}
是上不上下不下當魏而來也季安曰然則若之何忠曰王師入魏君厚犒之於是悉甲壓境號曰伐趙而可隂遺趙人書曰|{
	犒苦到翻遺唯季翻下遺魏同}
魏若伐趙則河北義士謂魏賣友魏若與趙則河南忠臣謂魏反君賣友反君之名魏不忍受執事若能隂解陴障遺魏一城魏得持之奏捷天子以為符信此乃使魏北得以奉趙西得以為臣|{
	長安在魏西為臣言能承上命不悖臣道}
於趙有角尖之耗|{
	角尖言所耗者小}
於魏獲不世之利執事豈能無意於魏乎趙人脱不拒君是魏霸基安矣季安曰善先生之來是天眷魏也遂用忠之謀與趙隂計得其堂陽|{
	堂陽漢縣屬鉅鹿郡唐屬冀州在州西南}
忠歸幽州謀欲激劉濟討王承宗會濟合諸將言曰天子知我怨趙今命我伐之|{
	今當作必}
趙亦必大備我伐與不伐孰利忠疾對曰天子終不使我伐趙趙亦不備燕濟怒曰爾何不直言濟與承宗反乎命繫忠獄使人視成德之境果不為備後一日詔果來令濟專護北疆勿使朕復掛胡憂而得專心於承宗|{
	復扶又翻}
濟乃解獄召忠曰信如子斷矣|{
	解獄謂釋其囚也斷丁亂翻}
何以知之忠曰盧從史外親燕内實忌之外絶趙内實與之此為趙畫曰|{
	為于偽翻}
燕以趙為障雖怨趙必不殘趙不必為備一且示趙不敢抗燕二且使燕獲疑天子趙人既不備燕潞人則走告于天子曰|{
	盧從史鎮潞州故謂之潞人}
燕厚怨趙趙見伐而不備燕是燕反與趙也此所以知天子終不使君伐趙趙亦不備燕也濟曰今則奈何忠曰燕趙為怨天下無不知|{
	自朱滔以來燕趙交惡}
今天子伐趙君坐全燕之甲一人未濟易水此正使潞人以燕賣恩於趙敗忠於上|{
	言燕本忠於上而盧從史以計敗之敗補邁翻}
兩皆售也|{
	賣物去手曰售}
是燕貯忠義之心卒染私趙之口不見德於趙人惡聲徒嘈嘈於天下耳|{
	貯丁呂翻卒子恤翻嘈昨勞翻}
惟君熟思之濟曰吾知之矣乃下令軍中曰五日畢出後者醢以徇|{
	譚忠頗有戰國說士之風而心為唐}


五年春正月劉濟自將兵七萬人擊王承宗時諸軍皆未進濟獨前奮擊拔饒陽束鹿河東河中振武義武四軍為恒州北面招討會于定州會望夜軍吏以有外軍請罷張燈張茂昭曰三鎮官軍也|{
	三鎮謂河中河東振武}
何謂外軍命張燈不禁行人不閉里門三夜如平日亦無敢喧嘩者|{
	唐制兩京及諸州縣街巷率置邏卒曉暝傳呼以禁夜行惟元夕張燈弛禁前後各一日}
丁卯河東將王榮拔王承宗洄湟鎮吐突承璀至行營威令不振與承宗戰屢敗左神策大將軍酈定進戰死定進驍將也|{
	酈定進擒劉闢有驍名}
軍中奪氣 河南尹房式有不法事東臺監察御史元稹奏攝之|{
	唐制御史分司東都謂之東臺攝收也}
擅令停務朝廷以為不可罸一季俸召還西京至敷水驛|{
	華州華隂縣西二十四里有敷水渠九域志華隂縣有敷水鎮}
有内侍後至破驛門呼罵而入以馬鞭擊稹傷面 |{
	考異曰實録云中使仇士良與元稹爭廳按稹及白居易傳皆云劉士元而實録云仇士良恐誤今止云内侍}
上復引稹前過貶江陵士曹|{
	復扶又翻前過謂擅令河南尹停務上知曲在中官故引前過以貶稹}
翰林學士李絳崔羣言稹無罪白居易上言中使陵辱朝士中使不問而稹先貶恐自今中使出外益暴横|{
	横戶孟翻}
人無敢言者又稹為御史多所舉奏不避權勢切齒者衆恐自今無人肯為陛下當官執法疾惡繩愆|{
	為于偽翻}
有大姦猾陛下無從得知上不聽 上以河朔方用兵不能討吳少陽三月己未以少陽為淮西留後|{
	果如李絳之言}
諸軍討王承宗者久無功白居易上言以為河北本不當用兵今既出師承璀未嘗苦戰已失大將|{
	謂酈定進戰死也}
與從史兩軍入賊境遷延進退不惟意在逗留亦是力難支敵希朝茂昭至新市鎮竟不能過|{
	新市漢縣名屬中山郡唐初新市縣屬觀州武德五年廢州并廢新市為鎮屬九門縣}
劉濟引全軍攻圍樂壽久不能下|{
	按劉濟時軍瀛洲而攻樂壽樂壽時屬深州在瀛洲南六十里}
師道季安元不可保察其情狀似相計會各收一縣遂不進軍|{
	譚忠之為田季安計者白居易已窺見之矣}
陛下觀此事勢成功有何所望以臣愚見須速罷兵若又遲疑其害有四可為痛惜者二可為深憂者二何則若保有成即不論用度多少既的知不可即不合虛費貲糧|{
	貲財也或曰當作資}
悟而後行事亦非晩今遲校一日則有一日之費更延旬月所費滋多終須罷兵何如早罷以府庫錢帛百姓脂膏資助河北諸侯轉令彊大此臣為陛下痛惜者一也|{
	為于偽翻下同}
臣又恐河北諸將見吳少陽已受制命|{
	言制以吳少陽為淮西留後}
必引事例輕重同詞請雪承宗若章表繼來即義無不許請而後捨體勢可知轉令承宗膠固同類如此則與奪皆由鄰道恩信不出朝廷實恐威權盡歸河北此為陛下痛惜者二也今天時已熱兵氣相蒸至於饑渇疲勞疾疫暴露驅以就戰人何以堪縱不惜身亦難忍苦况神策烏雜城市之人例皆不慣如此忽思生路|{
	連兵不解不死於戰亦死於久屯必思逃奔潰散為求生之路}
一人若逃百人相扇一軍若散諸軍必揺事忽至此悔將何及此為陛下深憂者一也臣聞回鶻吐蕃皆有細作|{
	細作古之諜者}
中國之事小大盡知今聚天下之兵唯討承宗一賊自冬及夏都未立功則兵力之彊弱資費之多少豈宜使西戎北虜一一知之忽見利生心乘虛入寇以今日之勢力可能救其首尾哉兵連禍生何事不有萬一及此實關安危此其為陛下深憂者二也|{
	其字衍考異曰白氏集云五月十日進据此疏云從史雖經接戰與賊勝負畧均則是未就縛也此月戊戌從史已流驩州疑五月當為四月故移於此}
盧從史首建伐王承宗之謀|{
	事見上卷上年五月}
及朝廷興師從史逗留不進隂與承宗通謀令軍士潛懷承宗號|{
	凡行軍各有號以相識别}
又高芻粟之價以販度支|{
	時吐突承璀總行營兵屯邢趙界邢州昭義廵屬也度夬芻粟不能遠致以給行營就昭義市糴故盧從史得高其價以牟利度徒洛翻}
諷朝廷求平章事誣奏諸道與賊通不可進兵上甚患之會從史遣牙將王翊元入奏事裴垍引與語為言為臣之義|{
	為言于偽翻}
微動其心翊元遂輸誠言從史隂謀及可取之狀垍令翊元還本軍經營復來京師|{
	復扶又翻}
遂得其都知兵馬使烏重胤等款要|{
	款誠也}
垍言於上曰從史狡猾驕狠必將為亂今聞其與承璀對營視承璀如嬰兒往來都不設備失今不取後雖興大兵未可以歲月平也上初愕然熟思良久乃許之從史性貪承璀盛陳奇玩視其所欲稍以遺之|{
	遺唯季翻}
從史喜益相昵狎|{
	昵尼質翻}
甲申承璀與行營兵馬使李聽謀召從史入營慱伏壯士於幕下突出擒詣帳後縳之内車中馳詣京師|{
	考異曰承璀傳曰承璀出師經年無功乃遣密人告王承宗令上疏待罪許以罷兵為解仍奏昭義節度使}


|{
	盧從史素與賊通許為承宗求節鉞乃誘潞洲牙將烏重胤謀執從史送京師今從裴垍等傳}
左右驚亂|{
	從史之左右也}
承璀斬十餘人諭以詔旨從史營中士聞之皆甲以出操兵趨譁|{
	操七刀翻趨譁言趨走而喧譁也}
烏重胤當軍門叱之曰天子有詔從者賞敢違者斬士卒皆歛兵還部伍會夜車疾驅未明已出境重胤承洽之子|{
	新書作承玼之子韓愈烏氏先廟碑亦作承玼一本云玼或作洽}
聽晟之子也 丁亥范希朝張茂昭大破承宗之衆於木刀溝|{
	新唐書地理志定州新樂縣東南二十里有木刀溝有民木刀居溝旁因名之}
上加烏重胤之功欲即授以昭義節度使李絳以為不可請授重胤河陽以河陽節度使孟元陽鎮昭義會吐突承璀奏已牒重胤句當昭義留後|{
	句舌候翻當丁浪翻}
絳上言昭義五州據山東要害|{
	五州澤潞邢洺磁要害者於我為要於敵為害}
魏博恒幽諸鎮蟠結|{
	魏博一鎮恒一鎮幽一鎮謂之河朔三鎮}
朝廷惟恃此以制之邢磁洺入其腹内|{
	邢州臨趙境磁洺臨魏境其界犬牙相入}
誠國之寶地安危所繫也曏為從史所據使朝廷旰食今幸而得之承璀復以與重胤|{
	復扶又翻}
臣聞之驚歎實所痛心昨國家誘執從史雖為長策已失大體|{
	不能明斥從史之罪而行天討乃誘執之是為失體}
今承璀又以文牒差人為重鎮留後為之求旌節|{
	為于偽翻}
無君之心孰甚於此陛下昨日得昭義人神同慶威令再立今日忽以授本軍牙將物情頓沮紀綱大紊校計利害|{
	校數也考也計算也度也}
更不若從史為之何則從史雖蓄姦謀已是朝廷牧伯重胤出於列校|{
	校戶教翻}
以承璀一牒代之竊恐河南北諸侯聞之無不憤怒恥與為伍且謂承璀誘重胤逐從史而代其位彼人人麾下各有將校能無自危乎儻劉濟茂昭季安執㳟韓弘師道繼有章表陳其情狀|{
	張茂昭田季安程執㳟李師道}
并指承璀專命之罪不知陛下何以處之|{
	處昌呂翻}
若皆不報則衆怒益甚若為之改除|{
	為于偽翻}
則朝廷之威重去矣上復使樞密使梁守謙密謀於絳曰|{
	復扶又翻}
今重胤已總軍務事不得已須應與節對曰從史為帥不由朝廷|{
	事見二百三十六卷德宗貞元二十年帥所類翻下同}
故啓其邪心終成逆節今以重胤典兵即授之節威福之柄不在朝廷可以異於從史乎重胤之得河陽已為望外之福豈敢更為旅拒况重胤所以能執從史本以杖順成功一旦自逆詔命安知同列不襲其跡而動乎重胤軍中等夷甚多必不願重胤獨為主帥移之它鎮乃惬衆心|{
	惬苦叶翻}
何憂其致亂乎上悦皆如其請壬辰以重胤為河陽節度使元陽為昭義節度使戊戌貶盧從史驩州司馬 五月乙巳昭義軍三千餘人夜潰奔魏州|{
	潰奔者盧從史之黨也}
劉濟奏拔安平 庚申吐蕃遣其臣論思邪熱入見|{
	見賢遍翻}
且歸路泌鄭叔矩之柩|{
	平凉刼盟泌叔矩沒于吐蕃柩巨救翻鄭注曰在床曰尸在棺曰柩}
甲子奚寇靈州六月甲申白居易復上奏以為臣比請罷兵|{
	易以豉翻下同}


|{
	復扶又翻下同上時掌翻下同上言比毗至翻}
今之事勢又不如前不知陛下復何所待是時上每有軍國大事必與諸學士謀之嘗踰月不見學士李絳等上言臣等飽食不言其自為計則得矣如陛下何陛下詢訪理道|{
	理道治道也}
開納直言實天下之幸豈臣等之幸上遽令明日三殿對來|{
	三殿麟德殿也殿有三面故曰三殿三殿之西即翰林學士院對來者言明日當召對可前來也時召對廷臣詔旨率有對來之語}
白居易嘗因論事言陛下錯上色莊而罷密召承旨李絳|{
	唐置翰林學士之始無承旨永貞元年上始命鄭絪為承旨大誥令大廢置丞相之密畫内外之密奏上之所甚注意者莫不專受專對翰林學士凡十廳南廳五間北廳五間中隔花甎道承旨居北廳東第一間}
謂白居易小臣不遜|{
	白當作曰}
須令出院|{
	欲出居易不令復入翰林}
絳曰陛下容納直言故羣臣敢竭誠無隱居易言雖少思|{
	少思猶今人言欠入思慮也少詩紹翻}
志在納忠陛下今日罪之臣恐天下各思箝口|{
	箝其亷翻}
非所以廣聰明昭聖德也上悦待居易如初 |{
	考異曰舊居易傳曰吐突承璀為招討使諫官上章者十七八居易面論辭情切至既而又請罷河北用兵凡數千百言皆人之所難言者上多聽納唯諫承璀事切上頗不悦謂李絳曰白居易小子是朕拔擢而無禮於朕朕實難耐絳對曰居易所以不避死亡之誅事無巨細必言者蓋欲酬陛下特力拔擢耳陛下欲開諫諍之路不宜阻居易言上曰卿言是也由是多見聽納今從李司空論事}
上嘗欲近獵苑中至蓬萊池西|{
	蓬萊池在蓬萊殿之北一曰太液池池中有蓬萊山自蓬萊池西出玄武門入重元門即苑中重元門苑之南門南對宫城玄武門}
謂左右曰李絳必諫不如且止 秋七月庚子王承宗遣使自陳為盧從史所離間|{
	間古莧翻}
乞輸貢賦請官吏許其自新李師道等數上表請雪承宗|{
	數所角翻 考異曰實録淄青幽州累有章表請赦承宗按劉濟素與成德有怨攻之最力白居易請罷兵狀云劉濟近日情似近忠今忽罷兵慮傷其意又豈緣劉濟一人惆悵而不顧天下遠圖然則濟豈肯請赦承宗今不取}
朝廷亦以師久無功丁未制洗雪承宗以為成德軍節度使復以德棣二州與之|{
	復扶又翻}
悉罷諸道行營將士共賜布帛二十八萬端匹|{
	唐制布帛六丈為端四丈為匹}
加劉濟中書令 劉濟之討王承宗也以長子緄為副大使|{
	長知兩翻緄古本翻}
掌幽州留務濟軍瀛州次子緫為瀛州刺史濟署行營都知兵馬使使屯饒陽濟有疾緫與判官張玘|{
	玘瓐里翻}
孔目官成國寶謀詐使人從長安來曰朝廷以相公逗留無功已除副大使為節度使矣明日又使人來告曰副大使旌節已至太原又使人走而呼曰|{
	呼火故翻}
旌節已過代州舉軍驚駭濟憤怒不知所為殺大將素與緄厚者數十人追緄詣行營以張玘兄臯代知留務濟自朝至日昃不食渇索飲|{
	索山客翻}
緫因寘毒而進之乙卯濟薨緄行至涿州|{
	涿州南至莫州百六十里莫州南至瀛州八十八里}
緫矯以父命杖殺之遂領軍務 嶺南監軍許遂振以飛語毁節度使楊於陵於上上命召於陵還除宂官|{
	楊於音烏召於同宂官散官也宂而隴翻}
裴垍曰於陵性亷直陛下以遂振故黜藩臣不可丁巳以於陵為吏部侍郎遂振尋自抵罪八月乙亥上與宰相語及神仙問果有之乎|{
	憲宗信方士之}


|{
	心已露於此}
李藩對曰秦始皇漢武帝學仙之效具載前史|{
	事各見本紀}
太宗服天竺僧長年藥致疾|{
	事見二百一卷高宗緫章二年}
此古今之明戒也陛下春秋鼎盛方勵志太平宜拒絶方士之說苟道盛德充人安國理何憂無堯舜之壽乎九月己亥吐突承璀自行營還|{
	自討王承宗還也還從宣翻又如字}
辛亥復為左衛上將軍充左軍中尉裴垍曰承璀首唱用兵|{
	事見上卷上年四月}
疲弊天下卒無成功|{
	卒子恤翻}
陛下縱以舊恩不加顯戮|{
	吐突承璀事帝於東宫故言舊恩}
豈得全不貶黜以謝天下乎給事中段平仲呂元膺言承璀可斬李絳奏稱陛下不責承璀它日復有敗軍之將何以處之|{
	復扶又翻處昌呂翻}
若或誅之則同罪異罰彼必不服若或釋之則誰不保身而玩寇乎願陛下割不忍之恩行不易之典|{
	有功必賞敗軍必誅此古今不易之典}
使將帥有所懲勸間二日|{
	間如字}
上罷承璀中尉降為軍器使|{
	唐中世以後置内諸司使以宦官為之軍器庫使其一也宋白曰軍器本屬軍器監中世置軍器使貞元四年廢武庫其器械隸於軍器使}
中外相賀 裴垍得風疾上甚惜之中使候問旁午於道|{
	一縱一横為旁午}
丙寅以太常卿權德輿為禮部尚書同平章事 義武節度使張茂昭請除代人欲舉族入朝河北諸鎮互遣人說止之|{
	說輸芮翻}
茂昭不從凡四上表上乃許之以左庶子任廸簡為義武行軍司馬茂昭悉以易定二州簿書管鑰授廸簡遣其妻子先行曰吾不欲子孫染於汚俗茂昭既去冬十月戊寅虞候楊伯玉作亂囚廸簡辛巳義武將士共殺伯玉兵馬使張佐元又作亂囚廸簡廸簡乞歸朝既而將士復殺佐元奉廸簡主軍務|{
	復扶又翻}
時易定府庫罄竭閭閻亦空|{
	周禮五家為比五比為閭閻里中門也}
廸簡無以犒士乃設糲飯與士卒共食之|{
	糲盧逹翻脱粟飯也}
身居戟門下經月|{
	藩鎮府門列戟因謂之戟門}
將士感之共請廸簡還寢然後得安其位上命以綾絹十萬匹賜易定將士壬辰以廸簡為義武節度使|{
	憲宗用任廸簡而得易定穆宗用張弘靖而失幽燕節鎮命代可不謹哉}
甲午以張茂昭為河中慈隰晉絳節度使從行將校皆拜官 右金吾大將軍伊慎以錢三萬緡賂右軍中尉第五從直求河中節度使從直恐事泄奏之十一月庚子貶慎為右衛將軍坐死者三人初慎自安州入朝|{
	入朝見上卷元和元年}
留其子宥主留事朝廷因以為安州刺史未能去也|{
	去羌呂翻}
會宥母卒於長安宥利於兵權不時發喪鄂岳觀察使郗士美遣僚屬以事過其境宥出迎因告以凶問|{
	凶問母卒之問也}
先備籃輿即日遣之|{
	籃輿即今之轎也}
甲辰會王纁薨|{
	纁上弟也薨呼肱翻}
庚戍以前河中節度使王鍔為河東節度使上左右受鍔厚賂多稱譽之|{
	譽音余}
上命鍔兼平章事李藩固執以為不可權德輿曰宰相非序進之官唐興以來方鎮非大忠大勲則跋扈者朝廷或不得已而加之今鍔既無忠勲朝廷又非不得已何為遽以此名假之上乃止 |{
	考異曰舊李藩傳曰鍔以錢數千萬賂權侍求兼宰相藩與權德輿在中書有密旨曰王鍔可兼宰相宜即擬來藩遂以筆塗兼宰相字却奏上云不可德輿失色曰縱不可宜别作奏豈可以筆塗詔邪曰勢廹矣出今日便不可止日又暮何暇别作奏事果寢會要崔鉉曰此乃不諳故事者之妄傳史官之謬記耳既稱奉密旨宜擬狀中陳論固不假以筆塗詔矣凡欲降白麻若啇量於中書門下皆前一日進文書然後付翰林草麻又稱藩曰勢廹矣出今日便不可止尤為疎闊蓋由史氏以藩有直亮之名欲委曲成其美豈所謂直筆哉舊德輿傳曰初鍔來朝貴倖多譽鍔者上將加平章事李藩堅執以為不可德輿繼奏云云乃止今從之}
鍔有吏才工於完聚范希朝以河東全軍出屯河北|{
	謂討王承宗也}
耗散甚衆鍔到鎮之初兵不滿三萬人馬不過六百匹歲餘兵至五萬人馬有五千匹器械精利倉庫充實又進家財三十萬緡上復欲加鍔平章事李絳諫曰鍔在太原雖頗著績効今因獻家財而命之若後世何上乃止|{
	復扶又翻}
中書侍郎裴垍數以疾辭位|{
	數所角翻}
庚申罷為兵部尚書 十二月戊寅張茂昭入朝請遷祖考之骨于京兆|{
	張茂昭祖謐父孝忠皆葬河北}
壬午以御史中丞呂元膺為鄂岳觀察使元膺嘗欲夜登城門已鏁守者不為開|{
	鏁蘇果翻不為于偽翻}
左右曰中丞也對曰夜中難辯真偽雖中丞亦不可元膺乃還|{
	還音旋又如字}
明日擢為重職翰林學士司勲郎中李絳面陳吐突承璀專横語極

懇切|{
	横戶孟翻懇誠至也}
上作色曰卿言太過絳泣曰陛下置臣於腹心耳目之地若臣畏避左右愛身不言是臣負陛下言之而陛下惡聞|{
	惡烏路翻}
乃陛下負臣也上怒解曰卿所言皆人所不能言使朕聞所不聞真忠臣也它日盡言皆應如是己丑以絳為中書舍人學士如故絳嘗從容諫上聚財|{
	從千容翻}
上曰今兩河數十州皆國家政令所不及河湟數千里淪於左祍朕日夜思雪祖宗之恥而財力不贍故不得不蓄聚耳不然朕宫中用度極儉薄多藏何用邪|{
	淮西既平帝之所聚適為驕侈之資耳}


六年春正月甲辰以彰義留後吳少陽為節度使 庚申以前淮南節度使李吉甫為中書侍郎同平章事二月壬申李藩罷為太子詹事 己丑忻王造薨|{
	造代宗之子皇叔祖也}
宦官惡李絳在翰林|{
	惡烏路翻}
以為戶部侍郎判本司|{
	判本司者判戶部職事唐自中世以後戶部侍郎或判度支故以判戶部為判本司此二十四司之司也}
上問故事戶部侍郎皆進羨餘|{
	羨弋線翻}
卿獨無進何也對曰守土之官厚歛於人以市私恩天下猶共非之况戶部所掌皆陛下府庫之物給納有籍安得羨餘若自左藏輸之内藏|{
	歛力贍翻藏徂浪翻}
以為進奉是猶東庫移之西庫臣不敢踵此弊也|{
	自玄宗時王鉷歲進錢以供天子燕私至裴延齡而其弊極矣}
上嘉其直益重之 乙巳上問宰相為政寛猛何先權德輿對曰秦以慘刻而亡漢以寛大而興太宗觀明堂圖禁抶人背|{
	事見一百九十三卷貞觀四年抶丑栗翻}
是故安史以來屢有悖逆之臣皆旋踵自亡|{
	悖蒲内翻又蒲沒翻}
由祖宗仁政結於人心人不能忘故也然則寛猛之先後可見矣上善其言 夏四月戊辰以兵部尚書裴垍為太子賓客李吉甫惡之也|{
	惡烏路翻}
庚午以刑部侍郎鹽鐵轉運使盧坦為戶部侍郎判度支或告泗州刺史薛謇為代北水運使有異馬不以獻事下度支|{
	謇知輦翻下戶嫁翻}
使廵官往驗未返上遲之使品官劉泰昕按其事|{
	唐内待省有品官白身二千九百三十二人昕許斤翻}
盧坦曰陛下既使有司驗之又使品官繼往豈大臣不足信於品官乎臣請先就黜免上召泰昕還|{
	還音旋又如字}
五月前行營糧料使于臯謩董溪|{
	行營謂前討恒州行營}
坐贓數千緡敕貸其死臯謨流春州溪流封州行至潭州並追遣中使賜死|{
	春州漢合浦郡高凉縣地隋為高凉郡之陽春縣唐置春州京師東南六千四百四十八里封州至京師水陸四千五百一十里潭州古長沙郡晉置湘州隋改潭州京師南二千四百四十五里}
權德輿上言以為臯謨等罪當死陛下肆諸市朝|{
	何晏曰已刑而陳其尸曰肆朝直遥翻}
誰不懼法不當已赦而殺之溪晉之子也|{
	董晉相德宗後鎮宣武薨于鎮}
庚子以金吾大將軍李惟簡為鳳翔節度使|{
	李惟簡惟岳之弟也}
隴州地與吐蕃接舊常朝夕相伺|{
	伺相吏翻}
更入攻抄|{
	更工衡翻抄楚交翻}
人不得息惟簡以為邊將當謹守備蓄財穀以待寇不當覩小利起事盗恩|{
	生事邀功竊取官賞是為盗恩}
禁不得妄入其地|{
	禁妄入吐蕃界}
益市耕牛鑄農器以給農之不能自具者增懇田數十萬畝屬歲屢稔|{
	屬之欲翻屢良遇翻又如字}
公私有餘販者流及他方 賜振武節度使阿跌光進姓李氏 六月丁卯李吉甫奏自秦至隋十有三代|{
	吉甫所謂十三代以秦漢魏晉宋齊梁陳北魏北齊周隋為數也}
設官之多無如國家者天寶以後中原宿兵見在可計者八十餘萬|{
	見賢遍翻}
其餘為啇賈僧道不服田畝者什有五六|{
	賈音古}
是常以三分勞筋苦骨之人奉七分待衣坐食之輩也今内外官以税錢給俸者不下萬員天下三百餘縣或以一縣之地而為州一鄉之民而為縣者甚衆請敕有司詳定廢置吏員可省者省之州縣可併者併之入仕之塗可減者減之又國家舊章依品制俸官一品月俸錢三十緡|{
	永徽之制一品月俸八千開元二十四年令百官防閤庶僕俸食雜用以月給之揔稱月俸一品為錢三萬一千}
職田禄米不過千斛|{
	唐初給一品職田六十頃禄七百石}
艱難以來增置使額厚給俸錢|{
	自兵興後權臣增領諸使月給厚俸比開元制禄數倍}
大歷中權臣月俸至九千緡州無大小刺史皆千緡|{
	新志云權臣月俸有至九十萬者刺史亦至十萬即此數也}
常衮為相始立限約|{
	事見二百二十五卷代宗大歷十二年}
李泌又量其閒劇隨事增加|{
	事見一百三十三卷德宗貞元四年量音良下同}
時謂通濟理難減削然猶有名存職廢或額去俸存閒劇之間厚薄頓異請敕有司詳考俸料雜給量定以聞|{
	按常衮為相增京官正員及諸道觀察使都團練使副使以下料錢李泌為相又增百官及畿内官月俸復置手力資課歲給錢左右衛上將軍以下又有六雜給一曰糧米二曰鹽三曰私馬四曰手力五曰隨身六曰春冬服私馬則有芻豆手力則有資錢隨身則有糧米鹽春冬服則有布絹絁紬綿射生神策大將軍增以鞋州縣官有手力雜給錢李吉甫請就加詳校而量定之也}
於是命給事中段平仲中書舍人韋貫之兵部侍郎許孟容戶部侍郎李絳同詳定 秋九月富平人梁悦報父仇殺秦杲自詣縣請罪敕復讐據禮經則義不同天|{
	禮記曰父之讐不與共戴天}
徵法令則殺人者死禮法二事皆王教之大端有此異同固資論辯宜令都省集議聞奏|{
	都省尚書都省}
職方員外郎韓愈議以為律無其條非闕文也蓋以不許復讐則傷孝子之心而乖先王之訓許復讐則人將倚法專殺無以禁止其端矣故聖人丁寧其義於經而深沒其文於律其意將使法吏一斷於法|{
	斷丁亂翻}
而經術之士得引經而議也宜定其制曰凡復父讐者事發具申尚書省集議奏聞酌其宜而處之|{
	處昌呂翻}
則經律無失其指矣敕梁悦杖一百流循州|{
	循州古龍川縣地舊志至東都四千八百里加東都至京師道里從可知也}
甲寅吏部奏凖敕併省内外官計八百八員諸司流外一千七百六十九人 黔州大水壞城郭|{
	黔音禽又其亷翻壞音怪}
觀察使竇羣發溪洞蠻以治之|{
	黔中觀察使領辰錦施叙奬夷播思費南溪溱等州又有羈縻州五十大率皆溪洞蠻也治直之翻}
督役太急於是辰溆二州蠻反|{
	溆州本巫州天授二年改沅州開元十三年以沅原聲相近復為巫州大歷五年更名溆州 考異曰舊傳作辰錦二州今從實錄}
羣討之不能定戊午貶羣開州刺史|{
	開州治開江縣因縣名州京師南一千四百六十里}
冬十一月弓箭庫使劉希光|{
	唐内諸司使弓箭庫使在軍器庫使之下}
受羽林大將軍孫璹錢二萬緡為求方鎮|{
	璹神六翻為于偽翻}
事覺賜死事連左衛上將軍知内侍省事吐突承璀丙申以承璀為淮南監軍上問李絳朕出承璀何如對曰外人不意陛下遽能如是上曰此家奴耳曏以其驅使之久|{
	承璀事帝於東宫}
故假以恩私若有違犯朕去之輕如一毛耳|{
	去羌呂翻}
十六宅諸王既不出閤 |{
	考異曰新李吉甫傳作十宅按舊紀自此至唐末皆云十六宅新傳誤也 余按開元以來皇子多居禁中詔附苑城為大宫分院而處號十王宅中人押之就夾城參天子起居其後增為十六宅舊史曰開元於安國寺東附苑城為大宅分院而居號十王宅十王謂慶忠棣鄂儀潁永榮延濟其後盛儀壽豐恒梁六王又就封入内宅此十六宅得名之始也}
其女嫁不以時選尚者皆由宦官率以厚賂自逹李吉甫上言自古尚主必擇其人獨近世不然十二月壬申詔封恩王等六女為縣主委中書門下宗正吏部選門地人才稱可者嫁之|{
	稱尺證翻}
己丑以戶部侍郎李絳為中書侍郎同平章事 |{
	考異曰舊傳曰吐突承璀恩寵莫二是歲將用絳為宰相前一日出璀為淮南監軍翌日降制以絳同平章事新傳曰絳所言無不聽帝欲遂以為相而承璀寵方盛忌其進隂有毀短帝乃出璀淮南監軍翌日拜絳同平章事今據實録出承璀至絳入相五十四日舊傳云翌日誤也}
李吉甫為相多修舊怨上頗知之故擢絳為相吉甫善逢迎上意而絳鯁直數爭論於上前|{
	數所角翻}
上多直絳而從其言由是二人有隙 閏月辛卯朔黔州奏辰溆賊帥張伯靖寇播州費州|{
	溆音叙}
試太子通事舍人李涉|{
	唐太子通事舍人屬右春坊員八人正七品下掌導宫臣辭見承令勞問此職事官也若李涉則試官}
知上於吐突承璀恩顧未衰乃投匭上疏稱承璀有功希光無罪承璀久委心腹不宜遽弃知匭使諫議大夫孔戣見其副章詰責不受涉乃行賂詣光順門通之|{
	戣渠龜翻武后垂拱四年置匭四枚共為一室列於朝堂東方木位主春色青配仁仁者以亭育為本以青匭置於東有能告養人及勸農之事者投之銘曰延恩匭南方火位主夏色赤配信信者風化之本以丹匭置於南有能正諫論時政得失者投之銘曰招諫匭西方金位主秋色白配義義者以斷决為本以素匭置於西有欲自陳抑屈者投之銘曰申寃匭北方水位主冬色玄配智智者謀慮之本以玄匭置於北能告以謀智者投之銘曰通玄匭以諫議補拾充使於朝堂知匭事每日所有投書至暮並即進入其詣光順門進狀者閤門使收而進之宋朝改知匭使為理檢使宋白曰光順門外即昭慶門匭居洧翻}
戣聞之上疏極言涉姦險欺天請加顯戮戊申貶涉峽州司倉|{
	峽州古夷陵地蜀置宜都郡梁置宜州後魏改拓州取開拓之義周武帝以州扼三峽之口改曰峽州舊志峽州京師東南一千八百八十八里}
涉渤之兄|{
	李渤時隱於少室山}
戣巢父之子也|{
	孔巢父死於李懷光之難}
辛亥惠昭太子寧薨|{
	寧立為太子見上卷四年三月}
是歲天下大稔米斗有直二錢者

七年春正月辛未以京兆尹元義方為鄜坊觀察使初義方媚事吐突承璀李吉甫欲自託於承璀擢義方為京兆尹李絳惡義方為人故出之|{
	惡烏路翻}
義方入謝因言李絳私其同年許季同除京兆少尹出臣鄜坊專作威福欺罔聰明上曰朕諳李絳不如是|{
	諳烏舍翻}
明日將問之義方惶愧而出明日上以詰絳曰人於同年固有情乎對曰同年乃九州四海之人偶同科第或登科然後相識情於何有|{
	唐人謂同榜進士為同年至今猶然}
且陛下不以臣愚備位宰相宰相職在量才授任若其人果才雖在兄弟子姪之中猶將用之况同年乎避嫌而弃才是乃便身非徇公也上曰善朕知卿必不爾遂趣義方之官|{
	趣讀曰促}
振武河溢毁東受降城|{
	東受降城瀕河河溢故毀城}
三月丙戌上御延英殿李吉甫言天下已太平陛下宜為樂|{
	樂音洛下同}
李絳曰漢文帝時兵木無刃家給人足賈誼猶以為厝火積薪之下不可謂安|{
	見十四卷漢文帝六年}
今法令所不能制者河南北五十餘州犬戎腥羶近接涇隴烽火屢驚|{
	唐六典烽候所置大率三十里若有山岡隔絶須逐便安置得相望見不必要限三十里其逼邊境者築城而置之每烽置帥副各一人其放烽有一炬兩炬三炬四炬隨賊多少為差}
加之水旱時作倉廩空虛此正陛下宵衣旰食之時豈得謂之太平遽為樂哉|{
	旰古按翻}
上欣然曰卿言正合朕意退謂左右曰吉甫專為悦媚如李絳真宰相也上嘗問宰相貞元中政事不理何乃至此李吉甫對曰德宗自任聖智不信宰相而信他人是使姦臣得乘間弄威福|{
	間古莧翻}
政事不理職此故也上曰然此亦未必皆德宗之過朕幼在德宗左右見事有得失當時宰相亦未有再三執奏者皆懷禄偷安今日豈得專歸咎於德宗邪卿輩宜用此為戒事有非是當力陳不已勿畏朕譴怒而遽止也李吉甫常言人臣不當強諫|{
	左傳宫之奇之為人也懦而不能強諫陸德明音義曰彊其良翻又其兩翻}
使君悦臣安不亦美乎李絳曰人臣當犯顔苦口指陳得失若陷君於惡豈得為忠上曰絳言是也吉甫至中書卧不視事長吁而已李絳或久不諫上輒詰之曰豈朕不能容受邪將無事可諫也李吉甫又嘗言於上曰賞罰人主之二柄不可偏廢陛下踐阼以來惠澤深矣而威刑未振中外懈惰|{
	懈古隘翻怠也}
願加嚴以振之上顧李絳曰何如對曰王者之政尚德不尚刑豈可捨成康文景而效秦始皇父子乎上曰然後旬餘于頔入對亦勸上峻刑又數日上謂宰相曰于頔大是姦臣勸朕峻刑卿知其意乎皆對曰不知也上曰此欲使朕失人心耳吉甫失色退而抑首不言笑竟日|{
	上以于頔峻刑之言為姦故吉甫愧前之失言}
夏四月丙辰以庫部郎中翰林學士崔羣為中書舍

人學士如故|{
	庫部郎掌戎器鹵簿儀仗屬兵部}
上嘉羣讜直|{
	讜音黨}
命學士自今奏事必取崔羣連署然後進之羣曰翰林舉動皆為故事必如是後來萬一有阿媚之人為之長|{
	長知兩翻}
則下位直言無從而進矣固不奉詔章三上|{
	上時掌翻}
上乃從之 五月庚申上謂宰相曰卿輩屢言淮浙去歲水旱近有御史自彼還言不至為災事竟如何李絳對曰臣按淮南浙西浙東奏狀皆云水旱人多流亡求設法招撫|{
	設為法制以招撫流亡之民}
其意似恐朝廷罪之者豈肯無災而妄言有災邪此蓋御史欲為姦謏以悦上意耳願得其主名按致其法上曰卿言是也國以人為本聞有災當亟救之豈可尚復疑之邪|{
	復扶又翻}
朕適者不思失言耳命速蠲其租賦上嘗與宰相論治道於延英殿|{
	治直吏翻}
日旰暑甚汗透御服宰相恐上體倦求退上留之曰朕入禁中所與處者獨宫人宦官耳故樂與卿等且共談為理之要殊不知倦也|{
	為理猶言為治唐避高宗諱改治為理處昌呂翻樂音洛}
六月癸巳司徒同平章事杜佑以太保致仕 秋七月乙亥立遂王宥為太子更名恒|{
	更工衡翻恒戶登翻 考異曰舊澧王惲傳曰時吐突承璀恩寵特異惠昭太子薨議立儲副承璀獨排羣議屬澧王欲以威權自樹賴上明斷不惑承璀傳曰八年欲召承璀還乃罷絳相位承璀還復為神策中尉惠昭太子薨承璀建議請立澧王寛為太子憲宗不納立遂王宥崔羣傳曰憲宗以澧王居長又多内助新傳亦曰惠昭太子薨承璀請立澧王不從据實録六年十一月承璀監淮南軍閏十二月惠昭太子薨明年承璀乃召還而新舊傳皆如此穆宗卒以此殺承璀蓋憲宗末年承璀欲廢太子立澧王耳非惠昭初薨時也}
恒郭貴妃之子也諸姫子澧王寛長於恒|{
	長知兩翻}
上將立恒命崔羣為寛草讓表|{
	為于偽翻}
羣曰凡推己之有以與人謂之讓|{
	推吐雷翻}
遂王嫡子也寛何讓焉|{
	史言崔羣力為憲宗言立子以嫡不以長之義}
上乃止 八月戊戌魏博節度使田季安薨初季安娶洺州刺史元誼女|{
	元誼奔魏見二百三十五卷德宗貞元十二年}
生子懷諫為節度副使|{
	新志節度副使在行軍司馬之下節度副大使則在行軍司馬之上河北三鎮以為儲帥}
牙内兵馬使田興庭玠之子也|{
	田庭玠見二百二十六卷德宗建中二年}
有勇力頗讀書性㳟遜季安淫虐興數規諫|{
	數所角翻}
軍中賴之季安以為收衆心出為臨清鎮將將欲殺之|{
	將欲如字}
興陽為風痺|{
	痺必至翻冷濕病也}
炙灼滿身|{
	炙居又翻灼艾也}
乃得免季安病風殺戮無度軍政廢亂夫人元氏召諸將立懷諫為副大使知軍務時年十一|{
	考異曰論事集作十二今從實録及舊傳}
遷季安於别寢月餘而薨召田興為步射都知兵馬使辛亥以左龍武大將軍薛平為鄭滑節度使欲以控制魏博上與宰相議魏博事李吉甫請興兵討之李絳以為魏博不必用兵當自歸朝廷吉甫盛陳不可不用兵之狀上曰朕意亦以為然絳曰臣竊觀兩河藩鎮之跋扈者皆分兵以隸諸將不使專在一人恐其權任太重乘間而謀己故也|{
	間古莧翻}
諸將勢均力敵莫能相制欲廣相連結則衆心不同其謀必泄欲獨起為變則兵少力微勢必不成加以購賞既重刑誅又峻是以諸將互相顧忌莫敢先發跋扈者恃此以為長策然臣竊思之若常得嚴明主帥能制諸將之死命者以臨之則粗能自固矣|{
	帥所類翻粗坐五翻讀從去聲}
今懷諫乳臭子不能自聽斷|{
	斷丁亂翻}
軍府大權必有所歸諸將厚薄不均怨怒必起不相服從則曏日分兵之策適足為今日禍亂之階也田氏不為屠肆|{
	謂舉家見屠骨肉分裂若屠家之屠羊豕然掛肉於枅以為列肆}
則悉為俘囚矣何煩天兵哉|{
	天子之兵謂之天兵}
彼自列將起代主帥鄰道所惡莫甚於此|{
	惡烏路翻}
彼不倚朝廷之援以自存則立為鄰道所□粉矣|{
	□與韲同牋西翻碎切薑蒜為之}
故臣以為不必用兵可坐待魏博之自歸也但願陛下按兵養威嚴敕諸道選練士馬以須後敕|{
	須待也}
使賊中知之不過數月必有自效於軍中者矣至時惟在朝廷應之敏速中其機會|{
	中竹仲翻}
不愛爵禄以賞其人使兩河藩鎮聞之恐其麾下效之以取朝廷之賞必皆恐懼爭為㳟順矣此所謂不戰而屈人兵者也上曰善它日吉甫復於延英盛陳用兵之利|{
	復扶又翻下同}
且言芻糧金帛皆已有備上顧問絳|{
	顧迴視也}
絳對曰兵不可輕動前年討恒州|{
	恒戶登翻}
四面發兵二十萬又發兩神策兵自京師赴之天下騷動所費七百餘萬緡訖無成功為天下笑|{
	謂吐突承璀討王承宗也}
今瘡痍未復人皆憚戰若又以敕命驅之臣恐非直無功或生他變况魏博不必用兵事勢明白願陛下勿疑上奮身撫案曰朕不用兵决矣|{
	撫拍也 考異曰新吉甫傳魏博節度使田季安疾甚吉甫請任薛平為義成節度使以重兵控邢洺因圖上河北險要所在帝張於浴堂門壁每議河北事必指吉甫曰朕目按圖信如卿料矣按憲宗竟用李絳之策不用兵而魏博平不如新傳所言今不取}
絳曰陛下雖有是言恐退朝之後復有熒惑聖聽者上正色厲聲曰朕志已决誰能惑之絳乃拜賀曰此社稷之福也既而田懷諫幼弱軍政皆决於家僮蔣士則數以愛憎移易諸將|{
	數所角翻}
衆皆憤怒朝命久不至|{
	朝直遥翻}
軍中不安田興晨入府士卒數千人大譟環興而拜|{
	環音宦}
請為留後興驚仆於地衆不散久之興度不免|{
	度徒洛翻}
乃謂衆曰汝肯聽吾言乎皆曰惟命興曰勿犯副大使守朝廷法令申版籍請官吏然後可皆曰諾興乃殺蔣士則等十餘人遷懷諫於外|{
	代宗廣德元年田承嗣帥魏博四世四十九年而滅}


資治通鑑卷二百三十八
