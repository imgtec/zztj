<!DOCTYPE html PUBLIC "-//W3C//DTD XHTML 1.0 Transitional//EN" "http://www.w3.org/TR/xhtml1/DTD/xhtml1-transitional.dtd">
<html xmlns="http://www.w3.org/1999/xhtml">
<head>
<meta http-equiv="Content-Type" content="text/html; charset=utf-8" />
<meta http-equiv="X-UA-Compatible" content="IE=Edge,chrome=1">
<title>資治通鑒_106-資治通鑑卷一百五_106-資治通鑑卷一百五</title>
<meta name="Keywords" content="資治通鑒_106-資治通鑑卷一百五_106-資治通鑑卷一百五">
<meta name="Description" content="資治通鑒_106-資治通鑑卷一百五_106-資治通鑑卷一百五">
<meta http-equiv="Cache-Control" content="no-transform" />
<meta http-equiv="Cache-Control" content="no-siteapp" />
<link href="/img/style.css" rel="stylesheet" type="text/css" />
<script src="/img/m.js?2020"></script> 
</head>
<body>
 <div class="ClassNavi">
<a  href="/24shi/">二十四史</a> | <a href="/SiKuQuanShu/">四库全书</a> | <a href="http://www.guoxuedashi.com/gjtsjc/"><font  color="#FF0000">古今图书集成</font></a> | <a href="/renwu/">历史人物</a> | <a href="/ShuoWenJieZi/"><font  color="#FF0000">说文解字</a></font> | <a href="/chengyu/">成语词典</a> | <a  target="_blank"  href="http://www.guoxuedashi.com/jgwhj/"><font  color="#FF0000">甲骨文合集</font></a> | <a href="/yzjwjc/"><font  color="#FF0000">殷周金文集成</font></a> | <a href="/xiangxingzi/"><font color="#0000FF">象形字典</font></a> | <a href="/13jing/"><font  color="#FF0000">十三经索引</font></a> | <a href="/zixing/"><font  color="#FF0000">字体转换器</font></a> | <a href="/zidian/xz/"><font color="#0000FF">篆书识别</font></a> | <a href="/jinfanyi/">近义反义词</a> | <a href="/duilian/">对联大全</a> | <a href="/jiapu/"><font  color="#0000FF">家谱族谱查询</font></a> | <a href="http://www.guoxuemi.com/hafo/" target="_blank" ><font color="#FF0000">哈佛古籍</font></a> 
</div>

 <!-- 头部导航开始 -->
<div class="w1180 head clearfix">
  <div class="head_logo l"><a title="国学大师官网" href="http://www.guoxuedashi.com" target="_blank"></a></div>
  <div class="head_sr l">
  <div id="head1">
  
  <a href="http://www.guoxuedashi.com/zidian/bujian/" target="_blank" ><img src="http://www.guoxuedashi.com/img/top1.gif" width="88" height="60" border="0" title="部件查字,支持20万汉字"></a>


<a href="http://www.guoxuedashi.com/help/yingpan.php" target="_blank"><img src="http://www.guoxuedashi.com/img/top230.gif" width="600" height="62" border="0" ></a>


  </div>
  <div id="head3"><a href="javascript:" onClick="javascript:window.external.AddFavorite(window.location.href,document.title);">添加收藏</a>
  <br><a href="/help/setie.php">搜索引擎</a>
  <br><a href="/help/zanzhu.php">赞助本站</a></div>
  <div id="head2">
 <a href="http://www.guoxuemi.com/" target="_blank"><img src="http://www.guoxuedashi.com/img/guoxuemi.gif" width="95" height="62" border="0" style="margin-left:2px;" title="国学迷"></a>
  

  </div>
</div>
  <div class="clear"></div>
  <div class="head_nav">
  <p><a href="/">首页</a> | <a href="/ShuKu/">国学书库</a> | <a href="/guji/">影印古籍</a> | <a href="/shici/">诗词宝典</a> | <a   href="/SiKuQuanShu/gxjx.php">精选</a> <b>|</b> <a href="/zidian/">汉语字典</a> | <a href="/hydcd/">汉语词典</a> | <a href="http://www.guoxuedashi.com/zidian/bujian/"><font  color="#CC0066">部件查字</font></a> | <a href="http://www.sfds.cn/"><font  color="#CC0066">书法大师</font></a> | <a href="/jgwhj/">甲骨文</a> <b>|</b> <a href="/b/4/"><font  color="#CC0066">解密</font></a> | <a href="/renwu/">历史人物</a> | <a href="/diangu/">历史典故</a> | <a href="/xingshi/">姓氏</a> | <a href="/minzu/">民族</a> <b>|</b> <a href="/mz/"><font  color="#CC0066">世界名著</font></a> | <a href="/download/">软件下载</a>
</p>
<p><a href="/b/"><font  color="#CC0066">历史</font></a> | <a href="http://skqs.guoxuedashi.com/" target="_blank">四库全书</a> |  <a href="http://www.guoxuedashi.com/search/" target="_blank"><font  color="#CC0066">全文检索</font></a> | <a href="http://www.guoxuedashi.com/shumu/">古籍书目</a> | <a   href="/24shi/">正史</a> <b>|</b> <a href="/chengyu/">成语词典</a> | <a href="/kangxi/" title="康熙字典">康熙字典</a> | <a href="/ShuoWenJieZi/">说文解字</a> | <a href="/zixing/yanbian/">字形演变</a> | <a href="/yzjwjc/">金 文</a> <b>|</b>  <a href="/shijian/nian-hao/">年号</a> | <a href="/diming/">历史地名</a> | <a href="/shijian/">历史事件</a> | <a href="/guanzhi/">官职</a> | <a href="/lishi/">知识</a> <b>|</b> <a href="/zhongyi/">中医中药</a> | <a href="http://www.guoxuedashi.com/forum/">留言反馈</a>
</p>
  </div>
</div>
<!-- 头部导航END --> 
<!-- 内容区开始 --> 
<div class="w1180 clearfix">
  <div class="info l">
   
<div class="clearfix" style="background:#f5faff;">
<script src='http://www.guoxuedashi.com/img/headersou.js'></script>

</div>
  <div class="info_tree"><a href="http://www.guoxuedashi.com">首页</a> > <a href="/SiKuQuanShu/fanti/">四库全书</a>
 > <h1>资治通鉴</h1> <!--         下载:【右键另存为】即可 --></div>
  <div class="info_content zj clearfix">
  
<div class="info_txt clearfix" id="show">
<center style="font-size:24px;">106-資治通鑑卷一百五</center>
    資治通鑑卷一百五   宋 司馬光 撰<br />
<br />
  胡三省 音註<br />
<br />
  晉紀二十七【起昭陽恊洽盡閼逢涒灘凡二年】<br />
<br />
  烈宗孝武皇帝上之下<br />
<br />
  太元八年春正月秦呂光發長安以鄯善王休密馱車師前部王彌窴為鄉導【鄯上扇翻馱唐何翻窴徒賢翻又唐見翻鄉讀曰嚮】 三月丁巳大赦 夏五月桓冲帥衆十萬伐秦攻襄陽【帥讀曰率】遣前將軍劉波等攻沔北諸城【沔彌兖翻】輔國將軍楊亮攻蜀拔五城進攻涪城【涪音浮】鷹揚將軍郭銓攻武當六月冲别將攻萬歲筑陽拔之【萬歲城名盖近筑陽筑陽縣漢屬南陽郡晉屬順陽郡春秋穀伯之國唐為襄州穀城縣師古曰筑音逐】秦王堅遣征南將軍鉅鹿公叡冠軍將軍慕容垂等帥步騎五萬救襄陽【冠古玩翻騎奇寄翻】兖州刺史張崇救武當後將軍張蚝步兵校尉姚萇救涪城【蚝七吏翻萇仲良翻】叡軍於新野垂軍於鄧城【鄧城縣屬襄陽郡蓋晉置也】桓冲退屯沔南秋七月郭銓及冠軍將軍桓石䖍敗張崇於武當【敗補邁翻】掠二千戶以歸鉅鹿公叡遣慕容垂為前鋒進臨沔水垂夜命軍士人持十炬繫於樹枝光照數十里冲懼退還上明【冲鎮上明見上卷二年】張蚝出斜谷【斜余遮翻谷音浴又古祿翻】楊亮引兵還冲表其兄子石民領襄城太守戍夏口冲自求領江州刺史詔許之 秦王堅下詔大舉入寇民每十丁遣一兵其良家子年二十已下有材勇者皆拜羽林郎又曰其以司馬昌明為尚書左僕射謝安為吏部尚書桓冲為侍中勢還不遠【謂以勢言之克晉之期近在旦夕還師不遠也還音旋又如字】可先為起第【為于偽翻】良家子至者三萬餘騎【騎奇寄翻下同】拜秦州主簿趙盛之為少年都統【都統官名起于此少詩照翻下同】是時朝臣皆不欲堅行【朝直遥翻】獨慕容垂姚萇及良家子勸之陽平公融言於堅曰鮮卑羌虜我之仇讐【慕容垂鮮卑也姚萇羌也其國皆為秦所滅雖曰臣服其實仇讐】常思風塵之變以逞其志所陳策畫何可從也良家少年皆富饒子弟不閑軍旅苟為謟諛之言以會陛下之意【會會合也】今陛下信而用之輕舉大事臣恐功既不成仍有後患悔無及也堅不聽八月戊午堅遣陽平公融督張蚝慕容垂等步騎二十五萬為前鋒以兖州刺史姚萇為龍驤將軍督益梁州諸軍事堅謂萇曰昔朕以龍驤建業【堅以龍驤將軍殺苻生得秦國驤思將翻】未嘗輕以授人卿其勉之左將軍竇衝曰王者無戲言此不祥之徵也堅默然慕容楷慕容紹言於慕容垂曰主上驕矜已甚叔父建中興之業在此行也垂曰然非汝誰與成之【至此垂知堅必敗方與兄子明言之】甲子堅發長安戎卒六十餘萬騎二十七萬旗鼓相望前後千里九月堅至項城涼州之兵始達咸陽蜀漢之兵方順流而下幽冀之兵至於彭城東西萬里水陸齊進運漕萬艘【艘蘇遭翻】陽平公融等兵三十萬先至潁口【潁水入淮之口也地理志潁水出陽城縣陽乾山東下至蔡入淮】詔以尚書僕射謝石為征虜將軍征討大都督以徐兖二州刺史謝玄為前鋒都督與輔國將軍謝琰西中郎將桓伊等衆共八萬拒之使龍驤將軍胡彬以水軍五千援夀陽琰安之子也是時秦兵既盛都下震恐謝玄入問計於謝安安夷然【夷坦也平也言坦然無異平日也】答曰已别有旨既而寂然玄不敢復言乃令張玄重請【復扶又翻重直用翻】安遂命駕出遊山墅【墅承與翻園廬也】親朋畢集與玄圍棊賭墅安棊常劣於玄是日玄懼便為敵手而又不勝【敵手謂下子爭行刼智筭相敵也玄意不在棊故不能勝安】安遂遊陟至夜乃還【還從宣翻又如字】桓沖深以根本為憂遣精鋭三千入衛京師謝安固却之曰朝廷處分已定【處昌呂翻分扶問翻】兵甲無闕西藩宜留以為防沖對佐吏歎曰【諸藩府參佐為佐吏】謝安石有廟堂之量不閑將略【將即亮翻】今大敵垂至方遊談不暇遣諸不經事少年拒之衆又寡弱天下事已可知吾其左衽矣 以琅邪王道子錄尚書六條事【錄尚書六條事始於劉聰】 冬十月秦陽平公融等攻夀陽癸酉克之執平虜將軍徐元喜等融以其參軍河南郭褒為淮南太守【淮南郡本治夀陽秦既得之以郭褒為太守】慕容垂拔鄖城【杜預曰江夏雲杜縣東南有鄖城鄖于分翻】胡彬聞夀陽陷退保硖石【水經注淮水東過夀春縣北右合肥水又北逕山峽中謂之峽石對岸山上結二城以防津要杜佑曰硖石今汝隂郡下蔡縣】融進攻之秦衛將軍梁成等帥衆五萬屯于洛澗柵淮以遏東兵【水經注洛澗上承苑馬塘水北歷秦墟下注淮謂之洛口帥讀曰率下同】謝石謝玄等去洛澗二十五里而軍憚成不敢進胡彬糧盡潜遣使告石等曰今賊盛糧盡恐不復見大軍【復扶又翻】秦人獲之送於陽平公融融馳使白秦王堅曰賊少易擒但恐逃去宜速赴之【使疏吏翻下同融持議以為晉不可伐今臨敵乃輕脱如此亦天奪其鍳也少詩沼翻易以䜴翻】堅乃留大軍於項城引輕騎八千兼道就融於夀陽遣尚書朱序來說謝石等以為彊弱異勢不如速降【三年堅執朱序于襄陽拜為度支尚書說輸芮翻降戶江翻】序私謂石等曰若秦百萬之衆盡至誠難與為敵今乘諸軍未集宜速擊之若敗其前鋒【敗捕邁翻】則彼已奪氣可遂破也石聞堅在夀陽甚懼欲不戰以老秦師謝琰勸石從序言十一月謝玄遣廣陵相劉牢之帥精兵五千趣洛澗【趣七喻翻】未至十里梁成阻澗為陳以待之【陳讀曰陣下同】牢之直前渡水撃成大破之斬成及弋陽太守王詠【曹魏分西陽蘄春置弋陽郡秦未能有其地也王詠領太守耳弋陽唐為光蘄黄三州之地】又分兵斷其歸津【斷丁管翻】秦步騎崩潰爭赴淮水士卒死者萬五千人執秦揚州刺史王顯等盡收其器械軍實于是謝石等諸軍水陸繼進秦王堅與陽平公融登壽陽城望之見晉兵部陣嚴整又望八公山上草木皆以為晉兵【八公山在今夀春縣北四里世傳漢淮南王安好神仙忽有八公皆鬚眉皓素詣門求見門者曰吾王好長生今先生無駐衰之術未敢以聞八公皆變成童遂立廟于山上或言今廟食于此山者乃右吳朱驕任被雲被雪八人皆淮南王之客世以八公為仙誕也】顧謂融曰此亦勍敵何謂弱也【勍渠京翻強也】憮然始有懼色【憮罔甫翻悵然失意貌】秦兵逼肥水而陳晋兵不得渡謝玄遣使謂陽平公融曰君懸軍深入而置陳逼水此乃持久之計非欲速戰者也若移陳少却【少詩沼翻下同】使晉兵得渡以决勝負不亦善乎秦諸將皆曰我衆彼寡不如遏之使不得上【上時掌翻】可以萬全堅曰但引兵少却使之半渡我以鐵騎蹙而殺之蔑不勝矣融亦以為然遂麾兵使却秦兵遂退不可復止【兩陳相向退者先敗此用兵之常勢也復扶又翻】謝玄謝琰桓伊等引兵渡水撃之融馳騎略陳欲以帥退者【帥讀曰率】馬倒為晉兵所殺秦兵遂潰玄等乘勝追擊至于青岡【青岡去今夀春縣三十里】秦兵大敗自相蹈藉而死者蔽野塞川【言敗兵自相蹈踐枕藉而死也藉慈夜翻塞悉則翻】其走者聞風聲鶴唳皆以為晉兵且至晝夜不敢息草行露宿【草行者涉草而行不敢由路露宿者宿於野次不敢入人家皆懼追兵也】重以饑凍【重直用翻】死者什七八初秦兵少却朱序在陳後呼曰【呼火故翻】秦兵敗矣衆遂大奔序因與張天錫徐元喜皆來奔獲秦王堅所乘雲母車【晉制雲母車以雲母飾犢車臣下不得乘以賜王公耳趙彦絟續古今注石虎皇后乘輦以純雲母代紗四望皆通徹】復取夀陽執其淮南太守郭褒【晉復取夀陽故秦所置太守見執】堅中流矢【中竹仲翻】單騎走至淮北饑甚民有進壺飱豚髀者【飱蘇昆翻熟食曰飱字林曰水澆飯也】堅食之賜帛十匹綿十斤辭曰陛下厭苦安樂【樂音洛】自取危困臣為陛下子陛下為臣父安有子飼其父而求報乎弗顧而去【飼祥吏翻】堅謂張夫人曰吾今復何面目治天下乎【復扶又翻治直之翻】澘然流涕【澘所姦翻流涕貌又所版翻所晏翻】是時諸軍皆潰惟慕容垂所將三萬人獨全【垂别撃鄖城不與肥水之戰且持軍嚴整故諸軍皆潰而垂軍獨全將即亮翻】堅以千餘騎赴之世子寶言於垂曰家國傾覆天命人心皆歸至尊但時運未至故晦迹自藏耳今秦主兵敗委身于我是天借之便以復燕祚此時不可失也願不以意氣微恩忘社稷之重【意氣微恩謂堅厚禮垂父子也】垂曰汝言是也然彼以赤心投命於我若之何害之天苟棄之不患不亡不若保護其危以報德徐俟其釁而圖之既不負宿心且可以義取天下【慕容垂此言猶有君人之度】奮威將軍慕容德曰秦彊而并燕秦弱而圖之此為報仇雪恥非負宿心也兄奈何得而不取釋數萬之衆以授人乎垂曰我昔為太傅所不容置身無所逃死於秦【見一百二卷海西公太和四年】秦主以國士遇我恩禮備至後復為王猛所賣【復扶又翻下尚復德復同】無以自明秦主獨能明之【見太和五年】此恩何可忘也若氐運必窮吾當懷集關東以復先業耳關西會非吾有也冠軍行參軍趙秋曰明公當紹復燕祚著於圖䜟【冠古玩翻䜟楚譛翻】今天時已至尚復何待若殺秦主據鄴都鼔行而西三秦亦非苻氏之有也垂親黨多勸垂殺堅垂皆不從悉以兵授堅平南將軍慕容暐屯鄖城聞堅敗棄其衆遁去至滎陽慕容德復說暐起兵以復燕祚【尚復德復扶又翻說輸芮翻】暐不從謝安得驛書知秦兵已敗時方與客圍棊攝書置牀上了無喜色【攝收也】圍棊如故客問之徐答曰小兒輩遂已破賊既罷還内過戶限不覺屐齒之折【言其喜甚也史言安矯情鎮物人臣以安社稷為悦者也大敵壓境一戰而破之安得不喜乎屐齒之折亦非安之訾也】丁亥謝石等歸建康得秦樂工能習舊聲于是宗廟始備金石之樂【永嘉之亂伶官樂器皆沒於劉石江左初立宗廟以無雅樂及伶人省太樂并鼓吹令是後頗得登歌食舉之樂猶有未備太寧末明帝又訪阮孚等增益之咸和中成帝乃復置太樂官鳩集遺工而尚未有金石也及慕容雋平冉閔兵戈之際鄴下樂人頗亦有來者謝尚鎮夀陽採拾樂人以備太樂并制石磬雅樂始頗具而王猛平鄴慕容氏所得樂聲又入關右今破苻堅獲其樂工楊蜀等閑習舊樂于是金石始備焉】乙未以張天錫為散騎常侍【散悉亶翻騎奇寄翻】朱序為琅邪内史 秦王堅收集離散比至洛陽【比必寐翻】衆十餘萬百官儀物軍容粗備【粗坐五翻】慕容農謂慕容垂曰尊不迫人於險【尊謂其父垂也慕容令亦呼垂為尊蓋其父子間常稱也】其義聲足以感動天地農聞祕記曰燕復興當在河陽【燕于賢翻】夫取果於未熟與自落不過晚旬日之間然其難易美惡相去遠矣【易以豉翻】垂心善其言行至澠池【澠彌兖翻】言于堅曰北鄙之民聞王師不利輕相扇動臣請奉詔書以鎮慰安集之因過謁陵廟【垂欲因行自謁其祖父陵廟也】堅許之權翼諫曰國兵新破四方皆有離心宜徵集名將置之京師以固根本鎮枝葉【將即亮翻】垂勇略過人世豪東夏頃以避禍而來其心豈止欲作冠軍而已哉【夏戶雅翻冠古玩翻】譬如養鷹飢則附人每聞風飊之起常有陵霄之志正宜謹其絛籠【飊扶搖風也釋曰疾風自下而上曰飊音卑遥翻絛他刀翻絲䋲也所以紲鷹】豈可解縱任其所欲哉堅曰卿言是也然朕已許之匹夫猶不食言【孔安國曰食言者食盡其言偽不實】况萬乘乎【乘䋲證翻】若天命有廢興固非智力所能移也翼曰陛下重小信而輕社稷臣見其往而不返關東之亂自此始矣堅不聽遣將軍李蠻閔亮尹固帥衆三千送垂又遣驍騎將軍石越率精卒三千戍鄴驃騎將軍張蚝帥羽林五千戍并州鎮軍將軍毛當帥衆四千戍洛陽【驍堅堯翻騎奇寄翻下同帥讀曰率下同驃匹妙翻蚝七吏翻】權翼密遣壯士邀垂於河橋南空倉中垂疑之自涼馬臺結草筏以渡【水經注東郡白馬縣有涼城河水逕其北有神馬亭西去白馬津可二十許里實中層峙南北二百步東西五十許步今按神馬亭既在東郡白馬正對黎陽岸垂安得越滎洛而至此渡河乎此涼馬臺蓋在富平津橋之西也涼馬臺由昔人于河渚浴馬浴竟驅馬就高納涼因名】使典軍程同衣已衣乘已馬與僮僕趣河橋【典軍蓋王國官垂在燕為吳王時所置也同衣于既翻趣七喻翻】伏兵發同馳馬獲免十二月秦王堅至長安哭陽平公而後入諡曰哀公大赦復死事者家【復方目翻復其家之賦役也】 庚午大赦以謝石為尚書令進謝玄號前將軍固讓不受 謝安壻王國寶坦之之子也安惡其為人【惡烏路翻】每抑而不用以為尚書郎國寶自以望族故事唯作吏部不為餘曹【尚書郎晉制三十五曹置郎二十三人更相統攝及江左無直事右民屯田車部别兵都兵騎兵左右士運曹十曹郎康穆以後又無虞曹二千石二郎但有殿中祠部吏部儀曹三公比部金部倉部度支都官起部水部主客駕部庫部中兵外兵十八曹後又省主客起部水部餘十五曹而吏部最為清選】固辭不拜由是怨安國寶從妹為會稽王道子妃帝與道子皆嗜酒狎昵邪諂【從才用翻昵尼質翻】國寶乃譛安于道子使離間之於帝【間古莧翻】安功名既盛而險詖求進之徒多毀短安帝由是稍疎忌之【詖彼義翻】 初開酒禁【漢建安中曹公嚴酒禁】增民税米口五石【咸和五年成帝始度百姓田取十分之一率畝税米三升哀帝即位減田租畝收二升太元二年帝除度田收租之制王公以下口租三斛唯蠲在役之身至是年又增税米口五石】 秦呂光行越流沙三百餘里【自玉門出渡流沙西行至鄯善北行至車師又且未國在鄯善西其國之西北有流沙數百里夏日有熱風為行旅之患風之所至唯老駞預知之即嗔而聚立埋其口鼻於沙中人每以為候亦即將氈擁蔽鼻口其風迅駛斯須過盡若不防者必致危斃桑欽曰流沙地在張掖居延縣西北杜佑曰流沙在沙州敦煌郡西八十里酈道元曰弱水入流沙流沙與水流行也亦言出鍾山西行極崦嵫之山在西海郡北流沙又逕浮渚歷壑市之國又逕于鳥山之東朝雲國西歷崐山西南出于過瀛之山大荒山經曰西南海之外流沙出焉】焉耆等諸國等皆降【降戶江翻】惟龜兹王帛純拒之【龜兹音邱慈】嬰城固守光進軍攻之 秦王堅之入寇也以乞伏國仁為前將軍領先鋒騎【騎奇寄翻】會國仁叔父步頹反於隴西堅遣國仁還討之【國仁代司繁鎮勇士見上卷元年】步頹聞之大喜迎國仁於路國仁置酒大言曰苻氏疲民逞兵殆將亡矣吾當與諸君共建一方之業及堅敗國仁遂迫脇諸部有不從者撃而併之衆至十餘萬 慕容垂至安陽【安陽在鄴城西南】遣參軍田山修牋於長樂公丕【樂音洛】丕聞垂北來疑其欲為亂然猶身自迎之趙秋勸垂於座取丕因據鄴起兵垂不從丕謀襲撃垂侍郎天水姜讓諫曰【晉制王國置侍郎二人】垂反形未著而明公擅殺之非臣子之義不如待以上賓之禮嚴兵衛之密表情狀聽敕而後圖之丕從之館垂於鄴西【館音貫】垂潜與燕之故臣謀復燕祚會丁零翟斌起兵叛秦【丁零種落本居中山苻堅之滅燕也徙于新安斌仕秦為衛軍從事中郎斌音彬】謀攻豫州牧平原公暉於洛陽秦王堅驛書使垂將兵討之【將即亮翻】石越言於丕曰王師新敗民心未安負罪亡匿之徒思亂者衆故丁零一唱旬日之中衆已數千此其驗也慕容垂燕之宿望有興復舊業之心今復資之以兵此為虎傅翼也【今復扶又翻為于偽翻傅讀曰附】丕曰垂在鄴如藉虎寢蛟【藉慈夜翻】常恐為肘腋之變【腋音亦】今遠之於外不猶愈乎【遠于願翻】且翟斌凶悖【悖蒲内翻又蒲没翻】必不肯為垂下使兩虎相斃吾從而制之此卞莊子之術也乃以羸兵二千及鎧仗之獘者給垂【羸倫為翻鎧可亥翻】又遣廣武將軍苻飛龍帥氐騎一千為垂之副【帥讀曰率騎奇寄翻】密戒飛龍曰垂為三軍之帥卿為謀垂之將行矣勉之【成都王潁使和演圖王浚殷浩使魏憬圖姚襄苻丕使苻飛龍圖慕容垂智略不足以濟其敗同一轍也帥所類翻將即亮翻】垂請入鄴城拜廟【燕都鄴故廟在鄴城】丕弗許乃潜服而入亭吏禁之垂怒斬吏燒亭而去石越言於丕曰垂敢輕侮方鎮殺吏燒亭反形已露可因此除之丕曰淮南之敗垂侍衛乘輿【乘䋲證翻】此功不可忘也越曰垂尚不忠於燕安能盡忠於我失今不取必為後患丕不從越退告人曰公父子好為小仁不顧大計終當為人禽耳【丕父子後卒如越之言好呼到翻】垂留慕容農慕容楷慕容紹於鄴行至安陽之湯池閔亮李毗自鄴來以丕與苻飛龍所謀告垂【幾事不密則害成苻飛龍固不足以辦垂况其謀已泄邪】垂因激怒其衆曰吾盡忠於苻氏而彼專欲圖我父子吾雖欲已得乎【已止也】乃託言兵少【少詩沼翻】停河内募兵旬日間有衆八千平原公暉遣使讓垂趣使進兵【遣使疏吏翻趣讀曰促】垂謂飛龍曰今宼賊不遠【河内距新安洛陽止隔大河耳故云然】當晝止夜行襲其不意飛龍以為然壬午夜垂遣世子寶將兵居前少子隆勒兵從已【將即亮翻少詩照翻】令氐兵五人為伍隂與寶約聞鼓聲前後合撃氐兵及飛龍盡殺之參佐家在西者皆遣還并以書遺秦王堅言所以殺飛龍之故【蓋言丕使飛龍圖已故殺之也遺于季翻】初垂從堅入鄴【見一百二卷海西公太和五年】以其子麟屢嘗告變於燕【事見太和四年】立殺其母然猶不忍殺麟置之外舍希得侍見【見賢遍翻】及殺苻飛龍麟屢進策畫啟發垂意【垂意所欲為而思慮偶有所未及麟能迎其機言之故謂之啟發】垂更奇之寵待與諸子均矣【為後麟亂燕張本】慕容鳳及燕故臣之子燕郡王騰遼西段延等【段延蓋段國之種燕于賢翻】聞翟斌起兵各帥部曲歸之【帥讀曰率】平原公暉使武平武侯毛當討斌慕容鳳曰鳳今將雪先王之恥【燕之亡也鳳父桓死難事見一百二卷海西公太和五年】請為將軍斬此氐奴【為于偽翻】乃擐甲直進【擐音宦】丁零之衆随之大敗秦兵【敗補邁翻】斬毛當遂進攻陵雲臺戍克之收萬餘人甲仗【陵雲臺魏文帝所築在洛城西秦置戍焉】癸未慕容垂濟河焚橋有衆三萬留遼東鮮卑可足渾譚集兵於河内之沙城【言河内以别魏郡之沙城燕主皝后可足渾氏譚蓋亦燕之戚黨也】垂遣田山如鄴密告慕容農等使起兵相應時日已暮農與慕容楷留宿鄴中慕容紹先出至蒲池【蒲池在鄴城外慕容雋與羣臣宴處】盜丕駿馬數百疋以待農楷甲申晦農楷將數十騎微服出鄴遂同奔列人【列人縣漢屬鉅鹿郡魏晉屬廣平郡其地在鄴城東北魏收地形志魏郡臨漳縣有列人城又别有列人縣亦屬魏郡】 九年春正月乙酉朔秦長樂公丕大會賓客【樂音洛】請慕容農不得始覺有變遣人四出求之三日乃知其在列人已起兵矣慕容鳳王騰段延皆勸翟斌奉慕容垂為盟主斌從之垂欲襲洛陽且未知斌之誠偽乃拒之曰吾來救豫州【秦平原公暉以豫州牧鎮洛陽】不來赴君君既建大事成享其福敗受其禍吾無預焉丙戌垂至洛陽平原公暉聞其殺苻飛龍閉門拒之翟斌復遣長史郭通往說垂【復扶又翻說輸芮翻下同】垂猶未許通曰將軍所以拒通者豈非以翟斌兄弟山野異類無奇才遠略必無所成故邪獨不念將軍今日憑之可以濟大業乎【謂憑其衆可以成功也】垂乃許之於是斌帥其衆來與垂會【帥讀曰率下同】勸垂稱尊號垂曰新興侯吾主也【秦獲慕容暐封為新興侯】當迎歸返正耳垂以洛陽四面受敵欲取鄴而據之乃引兵而東故扶餘王餘蔚為滎陽太守【餘蔚即太和五年開鄴北門納秦兵者蔚音紆勿翻】及昌黎鮮卑衛駒各率其衆降垂【降戶江翻】垂至滎陽羣下固請上尊號垂乃依晉中宗故事【晉元帝廟號中宗上時掌翻】稱大將軍大都督燕王承制行事【晉元帝稱王承制見九十卷建武元年】謂之統府羣下稱臣文表奏疏封拜官爵皆如王者以弟德為車騎大將軍封范陽王【騎奇寄翻】兄子楷為征西大將軍封太原王【燕本封德為范陽王今復其故楷恪子也恪封太原王今令楷襲父爵】翟斌為建義大將軍封河南王餘蔚為征東將軍統府左司馬封扶餘王衛駒為鷹揚將軍慕容鳳為建策將軍【建策將軍亦慕容垂一時所署置也】帥衆二十餘萬【帥讀曰率】自石門濟河長驅向鄴慕容農之奔列人也止於烏桓魯利家【魯利本烏桓種而家於列人】利為之置饌【為于偽翻饌雛皖翻又雛戀翻饔也】農笑而不食利謂其妻曰惡奴【句絶惡奴蓋詈其妻之語】郎貴人家貧無以饌之奈何妻曰郎有雄才大志今無故而至必將有異非為飲食來也【為于偽翻】君亟出遠望以備非常利從之農謂利曰吾欲集兵列人以圖興復卿能從我乎利曰死生唯郎是從【今世俗多呼其主為郎主又呼其主之子為郎君】農乃詣烏桓張驤說之曰【驤思將翻說輸芮翻下同】家王已舉大事翟斌等咸相推奉遠近響應故來相告耳驤再拜曰得舊主而奉之敢不盡死於是農驅列人居民為士卒斬桑榆為兵裂襜裳為旗【襜昌占翻爾雅曰衣蔽前也郭璞曰衣蔽膝也】使趙秋說屠各畢聰聰與屠各卜勝張延李白郭超及東夷餘和敕勃【屠直於翻餘和敕勃蓋二人】易陽烏桓劉大【易陽縣漢屬趙國魏晉屬陽平郡劉昫曰唐洛州臨洺縣古易陽縣也隋開皇六年更名】各帥部衆數千赴之【帥讀曰率】農假張驤輔國將軍劉大安遠將軍魯利建威將軍農自將攻破館陶【館陶縣漢屬魏郡魏晉屬陽平郡將即亮翻】收其軍資器械遣蘭汗段讃趙秋慕輿悕略取康臺牧馬數千匹【悕香衣翻魏收地形志廣平郡平恩縣有康臺澤杜預曰不以道取曰略】汗燕王垂之從舅讃聰之子也【從才用翻】於是步騎雲集衆至數萬驤等共推農為使持節都督河北諸軍事驃騎大將軍監統諸將【使疏吏翻驃匹妙翻騎奇寄翻監工衘翻】隨才部署上下肅然農以燕王垂未至不敢封賞將士趙秋曰軍無賞士不往【言無賞以奨激之則士不往赴戰也】今之來者皆欲建一時之功規萬世之利【規圖也】宜承制封拜以廣中興之基農從之於是赴者相繼垂聞而善之農間招庫傉官偉於上黨東引乞特歸於東阿北召光烈將軍平叡及叡兄汝陽太守幼於燕國偉等皆應之【間古莧翻遣間使招之也傉奴沃翻東阿縣漢屬東郡晉屬濟北郡唐屬濟州汝陽縣漢晉屬汝南郡後分為汝陽郡平幼蓋先嘗為汝陽太守時居燕國也偉等皆燕之舊臣故招之而應光烈將軍蓋亦前燕以授平叡】又遣蘭汗攻頓丘克之【頓丘縣漢屬東郡武帝泰始二年分置頓丘郡】農號令整肅軍無私掠【言其軍不敢掠居民而私其物】士女喜悦長樂公丕使石越將步騎萬餘討之農曰越有智勇之名今不南拒大軍而來此是畏王而陵我也【王謂慕容垂】必不設備可以計取之衆請治列人城【治直之翻下同】農曰善用兵者結士以心不以異物【異物猶言别物也】今起義兵唯敵是求當以山河為城池何列人之足治也辛卯越至列人西農使趙秋及參軍綦母滕撃越前鋒破之【越之氣已挫矣】參軍太原趙謙言於農曰越甲仗雖精人心危駭易破也【易以豉翻】宜急撃之農曰彼甲在人我甲在心【士心欲鬬則雖無甲胄而勇于赴戰故曰甲在心】晝戰則士卒見其外貌而憚之不如待暮撃之可以必克令軍士嚴備以待毋得妄動越立栅自固農笑謂諸將曰越兵精士衆不乘初至之鋭以撃我方更立柵吾知其無能為也向暮農鼔譟出陳於城西【陳讀曰陣】牙門劉木請先攻越柵農笑曰凡人見美食誰不欲之何得獨請然汝猛鋭可嘉當以先鋒惠汝木乃帥壯士四百騰栅而入秦兵披靡【帥讀曰率下同披普彼翻】農督大衆随之大敗秦兵斬越送首於垂【敗補邁翻】越與毛當皆秦之驍將也【驍堅堯翻將即亮翻】故秦王堅使助二子鎮守既而相繼敗沒人情騷動【騷愁也擾也】所在盜賊羣起庚戌燕王垂至鄴改秦建元二十年為燕元年服色朝儀皆如舊章【朝直遥翻】以前岷山公庫傉官偉為左長史前尚書段崇為右長史滎陽鄭豁等為從事中郎【凡帶前字者皆前燕所授官也】慕容農引兵會垂於鄴垂因其所稱之官而授之【即張驤等所推之官也】立世子寶為太子封從弟拔等十七人及甥宇文輸舅子蘭審皆為王【從才用翻】其餘宗族及功臣封公者三十七人侯伯子男者八十九人可足渾譚集兵得二萬餘人攻野王拔之【垂使譚集兵於河内之沙城遂因而攻拔野王】引兵會攻鄴平幼及其弟叡規亦帥衆數萬會垂於鄴長樂公丕使姜讓誚讓燕王垂且說之曰過而能改今猶未晚也【誚才笑翻說輸芮翻】垂曰孤受主上不世之恩故欲安全長樂公使盡衆赴京師【京師謂長安也】然後修復國家之業與秦永為鄰好【好呼到翻】何故闇於機運不以鄴城見歸若迷而不復當窮極兵勢恐單馬求生亦不可得也讓厲色責之曰將軍不容於家國投命聖朝【朝直遥翻】燕之尺土將軍豈有分乎【分扶問翻】主上與將軍風殊類别【言氐處關西鮮卑在東北既不同風族類又别也】一見傾心親如宗戚寵踰勲舊自古君臣際遇有如是之厚者乎一旦因王師小敗遽有異圖長樂公主上元子受分陕之任【陕失冉翻】寧可束手輸將軍以百城之地乎將軍欲裂冠毀冕【左傳晉率隂戎伐潁景王使詹桓伯辭於晉曰我在伯父猶衣服之有冠冕木水之有本原民人之有謀主也伯父若裂冠毀冕拔本塞原專棄謀主雖戎狄其何有余一人】自可極其兵勢奚更云云但惜將軍以七十之年懸首白旗【武王斬紂首懸於太白之旗】高世之忠更為逆鬼耳垂默然【姜讓之辭直垂心内愧故默然無以答】左右請殺之垂曰彼各為其主耳何罪禮而歸之遺丕書及上秦王堅表陳述利害請送丕歸長安堅及丕怒復書切責之【為于偽翻遺于季翻上時掌翻】 鷹揚將軍劉牢之攻秦譙城拔之桓冲遣上庸太守郭寶攻秦魏興上庸新城三郡拔之將軍楊佺期進據成固撃秦梁州刺史潘猛走之佺期亮之子也【楊亮見上卷太元二年】壬子燕王垂攻鄴拔其外郭長樂公丕退守中城關<br />
<br />
  東六州郡縣多送任請降於燕【降戶江翻】癸丑垂以陳留王紹行冀州刺史屯廣阿【廣阿縣前漢屬鉅鹿郡後漢併入鉅鹿縣有廣阿澤在鉅鹿縣界即大陸澤也】 豐城宣穆公桓沖聞謝玄等有功自以失言【謂去年吾其左衽之言也】慙恨成疾二月辛巳卒朝議欲以謝玄為荆江二州刺史【朝直遥翻】謝安自以父子名位太盛又懼桓氏失職怨望乃以梁郡太守桓石民為荆州刺史河東太守桓石䖍為豫州刺史豫州刺史桓伊為江州刺史燕王垂引丁零烏桓之衆二十餘萬為飛梯地道以<br />
<br />
  攻鄴不拔乃築長圍守之分處老弱於肥鄉【晉志肥郷縣屬廣平郡魏收曰天平初併入魏郡臨漳縣隋復分置肥鄉縣唐屬洺州處昌呂翻】築新興城以置輜重【重直用翻】 秦征東府官屬疑參軍高泰燕之舊臣有貳心【苻丕為征東大將軍高泰先仕燕慕容垂以為從事中郎】泰懼與同郡虞曹從事吳韶逃歸勃海【秦征東府置虞曹從事掌所部山澤泰與韶皆勃海人也】韶曰燕軍近在肥鄉宜從之泰曰吾以避禍耳去一君事一君吾所不為也申紹見而歎曰去就以道可謂君子矣燕范陽王德撃秦枋頭取之置戍而還【還音旋又讀如字】東胡王晏據館陶為鄴中聲援鮮卑烏桓及郡縣民據塢壁不從燕者尚衆燕王垂遣太原王楷與鎮南將軍陳留王紹討之楷謂紹曰鮮卑烏桓及冀州之民本皆燕臣今大業始爾人心未洽所以小異惟宜綏之以德不可震之以威吾當止一處為軍聲之本汝巡撫民夷示以大義彼必當聽從楷乃屯於辟陽【地理風俗記曰廣川西南六十里有辟陽亭故縣也漢高帝封審食其為侯國魏收地形志長樂郡信都縣有辟陽城】紹帥騎數百往說王晏為陳禍福【帥讀曰率說輸芮翻為于偽翻】晏隨紹詣楷降于是鮮卑烏桓及塢民降者數十萬口【降戶江翻】楷留其老弱置守宰以撫之發其丁壯十餘萬與王晏詣鄴垂大悦曰汝兄弟才兼文武足以繼先王矣【言足以繼慕容恪也】 三月以衛將軍謝安為太保 秦北地長史慕容泓聞燕王垂攻鄴亡奔關東收集鮮卑衆至數千還屯華隂敗秦將軍強永【華戶化翻敗補邁翻強其兩翻】其衆遂盛自稱都督陕西諸軍事大將軍雍州牧濟北王【陕式冉翻雍于用翻濟子禮翻】推垂為丞相都督陕東諸軍事領大司馬冀州牧吳王秦王堅謂權翼曰不用卿言【謂不用翼之言而遣慕容垂也】使鮮卑至此關東之地吾不復與之爭【復扶又翻】將若泓何乃以廣平公熙為雍州刺史鎮蒲阪【阪音反】徵雍州牧鉅鹿公叡為都督中外諸軍事衛大將軍錄尚書事配兵五萬以左將軍竇衝為長史龍驤將軍姚萇為司馬以討泓【驤思將翻】平陽太守慕容沖亦起兵于平陽有衆二萬進攻蒲坂堅使竇衝討之 庫傉官偉帥營部數萬至鄴燕王垂封偉為安定王 秦冀州刺史阜城侯定守信都高城男紹在其國【高城縣屬勃海郡唐為滄州鹽山縣】高邑侯亮重合侯謨守常山【重直龍翻】固安侯鑒守中山燕王垂遣前將軍樂浪王温督諸軍攻信都不克【樂浪音洛琅】夏四月丙辰遣撫軍大將軍麟益兵助之定鑒秦王堅之從叔紹謨從弟亮從子也【從才用翻】温燕王垂之弟子也 慕容泓聞秦兵且至懼帥衆將奔關東【帥讀曰率下同】秦鉅鹿愍公叡麄猛輕敵欲馳兵邀之姚萇諫曰鮮卑皆有思歸之志故起而為亂宜驅令出關不可遏也夫執鼷鼠之尾猶能反噬於人【鼷鼠一名甘口鼠食物不痛爾雅曰有螫毒者鼷音奚】彼自知困窮致死於我萬一失利悔將何及但可鳴鼓随之彼將奔敗不暇矣【使苻叡能用姚萇之言慕容泓必東奔慕容沖敗而無所歸必亦就禽矣】叡弗從戰于華澤【華澤即華隂之澤也華戶化翻】叡兵敗為泓所殺萇遣龍驤長史趙都參軍姜協詣秦王堅謝罪【驤思將翻】堅怒殺之萇懼奔渭北馬牧【馬牧牧馬之地猶漢之牧苑也】於是天水尹緯尹詳南安龎演等【龎皮江翻】糾扇羌豪帥其戶口歸萇者五萬餘家【帥讀曰率】推萇為盟主萇自稱大將軍大單于萬年秦王大赦改元白雀【通鑑目錄年經國緯自此以後姚萇繋之後秦單音蟬】以尹詳龎演為左右長史南安姚晃及尹緯為左右司馬天水狄伯支等為從事中郎羌訓等為掾屬王據等為參軍王欽盧姚方成等為將帥【掾以絹翻將即亮翻帥所類翻】 秦竇衝撃慕容沖於河東大破之沖帥鮮卑騎八千奔慕容泓泓衆至十餘萬遣使謂秦王堅曰吳王已定關東可速資備大駕奉送家兄皇帝【暐泓之兄也】泓當帥關中燕人翼衛乘輿還返鄴都【乘䋲證翻還音旋又如字】與秦以虎牢為界永為鄰好【好呼到翻】堅大怒召慕容暐責之曰今泓書如此卿欲去者朕當相資卿之宗族可謂人面獸心不可以國士期也暐叩頭流血涕泣陳謝堅久之曰此自三豎所謂非卿之過【三豎謂垂泓沖】復其位待之如初命暐以書招諭泓沖及垂暐密遣使謂泓曰吾籠中之人必無還理且燕室之罪人也【暐不能保燕之社稷故自謂為罪人】不足復顧【復扶又翻】汝勉建大業以吳王為相國中山王為太宰領大司馬汝可為大將軍領司徒承制封拜聽吾死問汝便即尊位泓于是進向長安改元燕興 燕王垂以鄴城猶固會僚佐議之右司馬封衡請引漳水灌之【用曹操攻鄴故智也】從之垂行圍【行下孟翻】因飲於華林園【洛都鄴都皆有華林園鄴之華林則魏武所築也】秦人密出兵掩之矢下如雨垂幾不得出【幾居依翻】冠軍大將軍隆將騎衝之垂僅而得免【將即亮翻騎奇寄翻】 竟陵太守趙統攻襄陽秦荆州刺史都貴奔魯陽五月秦洛州刺史張五虎據豐陽來降【降戶江翻】 梁州<br />
<br />
  刺史楊亮帥衆五萬伐蜀【帥讀曰率】遣巴西太守費統將水陸兵三萬為前鋒【費扶瑞翻將即亮翻】亮屯巴郡秦益州刺史王廣遣巴西太守康回等拒之【姓譜西胡自有康姓】 秦苻定苻紹皆降於燕【定以信都降紹以高城降降戶江翻下同】燕慕容麟引兵西攻常山【苻謨守常山】 後秦王萇進屯北地【姚萇書後秦以别於苻秦也】秦華隂北地新平安定羌胡降之者十餘萬 六月癸丑朔崇德太后禇氏崩 秦王堅自帥步騎二萬以撃後秦軍於趙氏塢【據晉書載記趙氏塢在北地帥讀曰率騎奇寄翻】使護軍將軍楊璧等分道攻之後秦兵屢敗斬後秦王萇之弟鎮軍將軍尹買後秦軍中無井秦人塞安公谷堰同官水以困之【安公谷同官水皆在今耀州界魏收地形志北地郡有銅官縣真君七年置杜佑曰銅官本漢祋祤縣地晉為頻陽苻秦於祋祤北銅官川置銅官護軍後魏太武罷護軍置銅官縣後周武帝移于今所隋以後唯作同官塞悉則翻】後秦人恟懼有渴死者【恟許拱翻】會天大雨後秦營中水三尺繞營百步之外寸餘而已後秦軍復振【復扶又翻】秦王堅歎曰天亦佑賊乎【天不助秦不可復支矣】 慕容泓謀臣高蓋等以泓德望不如慕容沖且持法苛峻乃殺泓立沖為皇太弟承制行事置百官以蓋為尚書令後秦王萇遣子嵩為質於沖以請和【欲連兵以斃秦且畏沖兵之彊也質音致】 將軍劉春攻魯陽都貴奔還長安 後秦王萇帥衆七萬撃秦秦王堅遣楊璧等拒之為萇所敗【敗補邁翻】獲楊璧及右將軍徐成鎮軍將軍毛盛等將吏數十人萇皆禮而遣之 燕慕容麟拔常山秦苻亮苻謨皆降【降戶江翻】麟進圍中山秋七月克之執苻鑒【冀州皆為燕有惟苻丕守鄴而已】麟威聲大振留屯中山 秦幽州刺史王永平州刺史苻沖帥二州之衆以撃燕【帥讀曰率】燕王垂遣平朔將軍平規撃永【此平規别是一平規平幼之弟非與苻洛同反之平規也】永遣昌黎太守宋敞逆戰於范陽敞兵敗規進據薊南【薊音計】 秦平原公暉帥洛陽陕城之衆七萬歸於長安【陕失冉翻】 秦王堅聞慕容沖去長安浸近乃引兵歸【歸自北地趙氏塢使沖不逼長安堅尚與萇相持勝負之勢未有所定也沖兵既逼堅不容不還長安萇得收嶺北以為資堅沖血戰而萇伺其敝堅死而鮮卑東出萇坐而取關中真所謂鷸蚌相持漁人之利也】遣撫軍大將軍方戍驪山【驪力知翻】拜平原公暉為都督中外諸軍事車騎大將軍錄尚書事配兵五萬以拒沖沖與暉戰于鄭西大破之堅又遣前將軍姜宇與少子河間公琳帥衆三萬拒沖於灞上【少詩照翻】琳宇皆敗死沖遂據阿房城【即秦之阿房宮城】 秦康回兵數敗【數所角翻】退還成都梓潼太守壘襲以涪城來降【此晉西師之捷姓譜曰後趙錄有壘澄本姓裴氏】荆州刺史桓石民據魯陽遣河南太守高茂北戍洛陽【此晉自襄沔北向之師也】 己酉葬康獻皇后於崇平陵 燕翟斌恃功驕縱邀求無厭【斌音彬厭於鹽翻】又以鄴城久不下潜有貳心太子寶請除之燕王垂曰河南之盟不可負也【斌引兵會垂於洛陽垂與之盟盖在河南縣】若其為難【難乃旦翻】罪由於斌今事未有形而殺之人必謂我忌憚其功能吾方收攬豪傑以隆大業不可示人以狹失天下之望也藉彼有謀吾以智防之無能為也范陽王德陳留王紹驃騎大將軍農皆曰翟斌兄弟恃功而驕必為國患垂曰驕則速敗焉能為患【焉于乾翻何也】彼有大功當聽其自斃耳禮遇彌重斌諷丁零及其黨請斌為尚書令垂曰翟王之功宜居上輔但臺既未建此官不可遽置耳斌怒密與前秦長樂公丕通謀【通鑑凡苻秦事書曰秦此前字衍】使丁零決隄潰水【燕引漳水以灌鄴故斌欲決隄以潰之】事覺垂殺斌及其弟檀敏餘皆赦之斌兄子真夜將營衆北奔邯鄲【將即亮翻邯鄲音寒丹】引兵還向鄴圍欲與丕内外相應太子寶與冠軍大將軍隆撃破之【冠古玩翻】真還走邯鄲【走音奏】太原王楷陳留王紹言於垂曰丁零非有大志但寵過為亂耳今急之則屯聚為寇緩之則自散散而撃之無不克矣垂從之 龜兹王帛純窘急【呂光自去年進軍攻龜兹龜兹音丘慈窘渠隕翻】重賂獪胡以求救【獪胡盖又在龜兹之西楊正衡曰獪古邁字】獪胡王遣其弟呐龍侯將馗帥騎二十餘萬【呐龍一人馗又一人侯將官稱也漢時西域諸國各有輔國侯安國侯左右將其後盖併侯將為一官呐女劣翻又女鬱翻將即亮翻馗渠追翻】并引温宿尉頭等諸國兵合七十餘萬以救龜兹秦呂光與戰於城西大破之帛純出走王侯降者三十餘國【降戶江翻】光入其城城如長安市邑宮室甚盛光撫寧西域威恩甚著遠方諸國前世所不能服者皆來歸附上漢所賜節傳【上時掌翻傳張戀翻】光皆表而易之立帛純弟震為龜兹王 八月翟真自邯鄲北走燕王垂遣太原王楷驃騎大將軍農帥騎追之及於下邑楷欲戰農曰士卒饑倦且視賊營不見丁壯殆有他伏楷不從進戰燕兵大敗真北趨中山屯於承營【羸師示弱者必有伏兵衆所通知也然而往往堕其中而不自覺以致覆軍者多矣趨七諭翻下同】鄴中芻糧俱盡削松木以飼馬【飼祥史翻】燕王垂謂諸將曰苻丕窮寇必無降理【降戶江翻】不如退屯新城【即肥鄉之新興城也】開丕西歸之路以謝秦王疇昔之恩且為討翟真之計丙寅夜垂解圍趨新城遣慕容農狗清河平原徵督租賦農明立約束均適有無軍令嚴整無所侵暴由是穀帛屬路軍資豐給【屬之欲翻】 戊寅南昌文穆公郗愔薨【郗丑之翻愔挹淫翻】 太保安奏請乘苻氏傾敗開拓中原以徐兗二州刺史謝玄為前鋒都督帥豫州刺史桓石䖍伐秦【帥讀曰率】玄至下邳秦徐州刺史趙遷棄彭城走玄進據彭城【此晉自淮泗北向之師也】 秦王堅聞呂光平西域以光為都督玉門以西諸軍事西域校尉道絶不通 秦幽州刺史王永求救於振威將軍劉庫仁【先是秦盖授劉庫仁振武將軍】庫仁遣其妻兄公孫希帥騎三千救之大破平規於薊南乘勝長驅進據唐城【中山郡之唐縣城也】 九月謝玄使彭城内史劉牢之攻秦兖州刺史張崇辛卯崇棄鄄城奔燕牢之據鄄城【鄄古掾翻】河南城堡皆來歸附 太保安上疏自求北征加安都督揚江等十五州諸軍事加黄鉞【十五州蓋揚徐南徐兖南兖豫南豫江青冀幽并司荆雍也】 慕容沖進逼長安秦王堅登城觀之歎曰此虜何從出哉大呼責沖曰【呼火故翻】奴何苦來送死沖曰奴厭奴苦欲取汝為代耳沖少有寵於堅【沖少有龍陽色得幸於堅少詩照翻】堅遣使以錦袍稱詔遺之【遺於季翻】沖遣詹事稱皇太弟令答之曰孤今心在天下豈顧一袍小惠苟能知命君臣束手早送皇帝【皇帝謂慕容暐】自當寛貸苻氏以酬曩好【好呼到翻】堅大怒曰吾不用王㬌略陽平公之言【事見一百三卷寧康三年及上卷大元七年】使白虜敢至於此【載記曰秦人率謂鮮卑為白虜】 冬十月辛亥朔日有食之 乙丑大赦謝玄遣隂陵太守高素攻秦青州刺史苻朗【隂陵縣漢屬九江郡晉書謝玄傳作淮陵淮陵縣前漢屬臨淮郡後漢屬下邳郡晉復屬臨淮郡惠帝元康七年分置淮陵郡隂當作淮】軍至琅邪朗來降朗堅之從子也【降戶江翻從才用翻】 翟真在承營與公孫希宋敞遥相首尾【公孫希劉庫仁所遣宋敞王永所遣】長樂公丕遣宦者冗從僕射清河光祚【姓譜光姓燕人田光之後秦末子孫避地以光為氏冗而隴翻從才用翻】將兵數百赴中山與真相結又遣陽平太守邵興將數千騎招集冀州故郡縣與祚期會襄國是時燕軍疲弊秦勢復振【復扶又翻】冀州郡縣皆觀望成敗趙郡人趙粟等起兵柏鄉以應興【魏收地形志南趙郡柏人縣有柏鄉城九域志曰柏鄉故城春秋時晉鄗邑五代志隋文帝開皇十六年置柏鄉縣屬趙郡】燕王垂遣冠軍大將軍隆龍驤將軍張崇將兵邀擊興命驃騎大將軍農自清河引兵會之【冠古玩翻驤思將翻驃匹妙翻騎奇寄翻】隆與興戰于襄國大破之興走至廣阿遇慕容農執之光祚聞之循西山走歸鄴隆遂撃趙粟等皆破之冀州郡縣復從燕【復扶又翻】 劉庫仁聞公孫希已破平規欲大舉兵以救長樂公丕發鴈門上谷代郡兵屯繁畤【畤音止】燕太子太保慕輿句之子文零陵公慕輿䖍之子常【慕輿句見九十八卷穆帝永和六年慕輿䖍見一百一卷哀帝興寧三年句音鉤】時在庫仁所知三郡兵不樂遠征【樂音洛】因作亂夜攻庫仁殺之竊其駿馬奔燕公孫希之衆聞亂自潰希奔翟真庫仁弟頭眷代領庫仁部衆 秦長樂公丕遣光祚及參軍封孚召驃騎將軍張蚝并州刺史王騰於晉陽以自救蚝騰以衆少不能赴【秦以鄧羌張蚝為萬人敵是時鄧羌死矣張蚝卒不能救秦之亡是知徒勇而無謀者無益于成敗之數也蚝七吏翻】丕進退路窮謀於僚佐司馬楊膺請自歸於晉丕未許會謝玄遣龍驤將軍劉牢之等據碻磝【碻磝城濟北郡治所沿河要地也碻丘交翻磝牛交翻楊正衡曰碻口勞翻杜佑曰碻口交翻磝音敖】濟陽太守郭滿據滑臺【沈約曰晉惠帝分陳留為濟陽國滑臺漢之白馬唐之滑州也宋南渡後遣范成大北使時河已南徙滑州及白馬縣皆在河北古滑州已淪于河中矣剩水在濬州西南積水若河對濬州城即黎陽山濟子禮翻】將軍顔肱劉襲軍於河北【河北滑臺之北岸也】丕遣將軍桑據屯黎陽以拒之劉襲夜襲據走之遂克黎陽丕懼乃遣從弟就與參軍焦逵請救於玄【從才用翻】致書稱欲假塗求糧西赴國難【難乃旦翻】須援軍既接以鄴與之若西路不通長安陷沒請帥所領保守鄴城【帥讀曰率】逵與參軍姜讓密謂膺曰今喪敗如此【喪息浪翻】長安阻絶存亡不可知屈節竭誠以求糧援猶懼不獲而公豪氣不除方設兩端事必無成宜正書為表許以王師之至當致身南歸如其不從可逼縛與之膺自以力能制丕【楊膺丕之妃兄故自以為力能制丕】乃改書而遣之 謝玄遣晉陵太守滕恬之渡河守黎陽恬之修之曾孫也【滕修為吳將孫皓之亡修歸晉】朝廷以兖青司豫既平加玄都督徐兖青司冀幽并七州諸軍事 後秦王萇聞慕容沖攻長安會羣僚議進止皆曰大王宜先取長安建立根本然後經營四方萇曰不然燕人因其衆有思歸之心以起兵若得其志必不久留關中吾當移屯嶺北【嶺北謂九嵕之北凡新平北地安定之地皆是也】廣收資實以待秦亡燕去然後拱手取之耳乃留其長子興守北地【長知兩翻】使寧北將軍姚穆守同官川自將其衆攻新平【將即亮翻】初新平人殺其郡將秦王堅缺其城角以恥之【石虎之末清河崔悦為新平相為郡人所殺悦子液仕堅為尚書郎自表父仇不同天地請還冀州堅愍之禁錮新平人缺其城角以恥之將即亮翻】新平民望深以為病【民望郡之賢豪為一郡所宗嚮者】欲立忠義以雪之及後秦王萇至新平新平太守南安苟輔欲降之【苟輔氐也秦之外威降戶江翻下同】郡人遼西太守馮傑蓮勺令馮羽尚書郎趙義汶山太守馮苗諫曰昔田單以一城存齊【傑等皆新平人太康地志曰汶山郡漢武帝立孝宣地節三年合蜀郡劉蜀又立郡蓮音輦汶讀曰岷田單事見四卷周赧王三十六年】今秦之州鎮猶連城過百奈何遽為叛臣乎輔喜曰此吾志也但恐久而無救郡人横被無辜【横戶孟翻】諸君能爾吾豈顧生哉於是憑城固守後秦為土山地道輔亦於内為之或戰地下或戰山上後秦之衆死者萬餘人輔詐降以誘萇萇將入城覺之而返輔伏兵邀撃幾獲之【幾巨依翻】又殺萬餘人 隴西處士王嘉隱居倒虎山【水經注倒虎山在新豐縣南處昌呂翻】有異術能知未然秦人神之秦王堅後秦王萇及慕容沖皆遣使迎之十一月嘉入長安衆聞之以為堅有福故聖人助之三輔堡壁及四山氐羌歸堅者四萬餘人堅置嘉及沙門道安於外殿動静咨之 燕慕容農自信都西撃丁零翟遼于魯口破之遼退屯無極農屯藁城以逼之【無極縣漢屬中山晉省後魏復置無極縣唐末為祁州治所藁城縣前漢屬真定後漢屬鉅鹿晉省今所屯蓋故縣城也唐復置藁城縣屬恒州】遼真之從兄也【從才用翻】 鮮卑在長安城中者猶千餘人慕容紹之兄肅與慕容暐隂謀結鮮卑為亂十二月暐白堅以其子新昏請堅幸其家置酒欲伏兵殺之堅許之會天大雨不果往事覺堅召暐及肅肅曰事必洩矣入則俱死今城内已嚴【已嚴者謂鮮卑之衆也】不如殺使者馳出既得出門大衆便集暐不從遂俱入堅曰吾相待何如而起此意暐飾辭以對肅曰家國事重何論意氣【意氣謂堅相待之厚】堅先殺肅乃殺暐及其宗族城内鮮卑無少長男女皆殺之【少詩照翻長知兩翻】燕王垂幼子柔養於宦者宋牙家為牙子故得不坐與太子寶之子盛乘間得出奔慕容沖【為後慕容盛等自長子歸燕張本間古莧翻】 燕慕容麟慕容農合兵襲翟遼大破之遼單騎奔翟真 燕王垂以秦長樂公丕猶據鄴不去乃更引兵圍鄴開其西走之路【垂志在得鄴故開其走路所謂圍城為之缺也】焦逵見謝玄玄欲徵丕任子然後出兵逵固陳丕欵誠并述楊膺之意玄乃遣劉牢之滕恬之等帥衆二萬救鄴【帥讀曰率】丕告饑玄水陸運米二千斛以饋之 秦梁州刺史潘猛棄漢中奔長安【梁州之地自此復歸于晉】<br />
<br />
  資治通鑑卷一百五  <br>
   </div> 

<script src="/search/ajaxskft.js"> </script>
 <div class="clear"></div>
<br>
<br>
 <!-- a.d-->

 <!--
<div class="info_share">
</div> 
-->
 <!--info_share--></div>   <!-- end info_content-->
  </div> <!-- end l-->

<div class="r">   <!--r-->



<div class="sidebar"  style="margin-bottom:2px;">

 
<div class="sidebar_title">工具类大全</div>
<div class="sidebar_info">
<strong><a href="http://www.guoxuedashi.com/lsditu/" target="_blank">历史地图</a></strong>  
<a href="http://www.880114.com/" target="_blank">英语宝典</a>  
<a href="http://www.guoxuedashi.com/13jing/" target="_blank">十三经检索</a> 
<br><strong><a href="http://www.guoxuedashi.com/gjtsjc/" target="_blank">古今图书集成</a></strong> 
<a href="http://www.guoxuedashi.com/duilian/" target="_blank">对联大全</a> <strong><a href="http://www.guoxuedashi.com/xiangxingzi/" target="_blank">象形文字典</a></strong> 

<br><a href="http://www.guoxuedashi.com/zixing/yanbian/">字形演变</a>  <strong><a href="http://www.guoxuemi.com/hafo/" target="_blank">哈佛燕京中文善本特藏</a></strong>
<br><strong><a href="http://www.guoxuedashi.com/csfz/" target="_blank">丛书&方志检索器</a></strong> <a href="http://www.guoxuedashi.com/yqjyy/" target="_blank">一切经音义</a>  

<br><strong><a href="http://www.guoxuedashi.com/jiapu/" target="_blank">家谱族谱查询</a></strong>  <strong><a href="http://shufa.guoxuedashi.com/sfzitie/" target="_blank">书法字帖欣赏</a></strong> 
<br>

</div>
</div>


<div class="sidebar" style="margin-bottom:0px;">

<font style="font-size:22px;line-height:32px">QQ交流群9:489193090</font>


<div class="sidebar_title">手机APP 扫描或点击</div>
<div class="sidebar_info">
<table>
<tr>
	<td width=160><a href="http://m.guoxuedashi.com/app/" target="_blank"><img src="/img/gxds-sj.png" width="140"  border="0" alt="国学大师手机版"></a></td>
	<td>
<a href="http://www.guoxuedashi.com/download/" target="_blank">app软件下载专区</a><br>
<a href="http://www.guoxuedashi.com/download/gxds.php" target="_blank">《国学大师》下载</a><br>
<a href="http://www.guoxuedashi.com/download/kxzd.php" target="_blank">《汉字宝典》下载</a><br>
<a href="http://www.guoxuedashi.com/download/scqbd.php" target="_blank">《诗词曲宝典》下载</a><br>
<a href="http://www.guoxuedashi.com/SiKuQuanShu/skqs.php" target="_blank">《四库全书》下载</a><br>
</td>
</tr>
</table>

</div>
</div>


<div class="sidebar2">
<center>


</center>
</div>

<div class="sidebar"  style="margin-bottom:2px;">
<div class="sidebar_title">网站使用教程</div>
<div class="sidebar_info">
<a href="http://www.guoxuedashi.com/help/gjsearch.php" target="_blank">如何在国学大师网下载古籍?</a><br>
<a href="http://www.guoxuedashi.com/zidian/bujian/bjjc.php" target="_blank">如何使用部件查字法快速查字?</a><br>
<a href="http://www.guoxuedashi.com/search/sjc.php" target="_blank">如何在指定的书籍中全文检索?</a><br>
<a href="http://www.guoxuedashi.com/search/skjc.php" target="_blank">如何找到一句话在《四库全书》哪一页?</a><br>
</div>
</div>


<div class="sidebar">
<div class="sidebar_title">热门书籍</div>
<div class="sidebar_info">
<a href="/so.php?sokey=%E8%B5%84%E6%B2%BB%E9%80%9A%E9%89%B4&kt=1">资治通鉴</a> <a href="/24shi/"><strong>二十四史</strong></a>&nbsp; <a href="/a2694/">野史</a>&nbsp; <a href="/SiKuQuanShu/"><strong>四库全书</strong></a>&nbsp;<a href="http://www.guoxuedashi.com/SiKuQuanShu/fanti/">繁体</a>
<br><a href="/so.php?sokey=%E7%BA%A2%E6%A5%BC%E6%A2%A6&kt=1">红楼梦</a> <a href="/a/1858x/">三国演义</a> <a href="/a/1038k/">水浒传</a> <a href="/a/1046t/">西游记</a> <a href="/a/1914o/">封神演义</a>
<br>
<a href="http://www.guoxuedashi.com/so.php?sokeygx=%E4%B8%87%E6%9C%89%E6%96%87%E5%BA%93&submit=&kt=1">万有文库</a> <a href="/a/780t/">古文观止</a> <a href="/a/1024l/">文心雕龙</a> <a href="/a/1704n/">全唐诗</a> <a href="/a/1705h/">全宋词</a>
<br><a href="http://www.guoxuedashi.com/so.php?sokeygx=%E7%99%BE%E8%A1%B2%E6%9C%AC%E4%BA%8C%E5%8D%81%E5%9B%9B%E5%8F%B2&submit=&kt=1"><strong>百衲本二十四史</strong></a>  <a href="http://www.guoxuedashi.com/so.php?sokeygx=%E5%8F%A4%E4%BB%8A%E5%9B%BE%E4%B9%A6%E9%9B%86%E6%88%90&submit=&kt=1"><strong>古今图书集成</strong></a>
<br>

<a href="http://www.guoxuedashi.com/so.php?sokeygx=%E4%B8%9B%E4%B9%A6%E9%9B%86%E6%88%90&submit=&kt=1">丛书集成</a> 
<a href="http://www.guoxuedashi.com/so.php?sokeygx=%E5%9B%9B%E9%83%A8%E4%B8%9B%E5%88%8A&submit=&kt=1"><strong>四部丛刊</strong></a>  
<a href="http://www.guoxuedashi.com/so.php?sokeygx=%E8%AF%B4%E6%96%87%E8%A7%A3%E5%AD%97&submit=&kt=1">說文解字</a> <a href="http://www.guoxuedashi.com/so.php?sokeygx=%E5%85%A8%E4%B8%8A%E5%8F%A4&submit=&kt=1">三国六朝文</a>
<br><a href="http://www.guoxuedashi.com/so.php?sokeytm=%E6%97%A5%E6%9C%AC%E5%86%85%E9%98%81%E6%96%87%E5%BA%93&submit=&kt=1"><strong>日本内阁文库</strong></a> <a href="http://www.guoxuedashi.com/so.php?sokeytm=%E5%9B%BD%E5%9B%BE%E6%96%B9%E5%BF%97%E5%90%88%E9%9B%86&ka=100&submit=">国图方志合集</a> <a href="http://www.guoxuedashi.com/so.php?sokeytm=%E5%90%84%E5%9C%B0%E6%96%B9%E5%BF%97&submit=&kt=1"><strong>各地方志</strong></a>

</div>
</div>


<div class="sidebar2">
<center>

</center>
</div>
<div class="sidebar greenbar">
<div class="sidebar_title green">四库全书</div>
<div class="sidebar_info">

《四库全书》是中国古代最大的丛书,编撰于乾隆年间,由纪昀等360多位高官、学者编撰,3800多人抄写,费时十三年编成。丛书分经、史、子、集四部,故名四库。共有3500多种书,7.9万卷,3.6万册,约8亿字,基本上囊括了古代所有图书,故称“全书”。<a href="http://www.guoxuedashi.com/SiKuQuanShu/">详细>>
</a>

</div> 
</div>

</div>  <!--end r-->

</div>
<!-- 内容区END --> 

<!-- 页脚开始 -->
<div class="shh">

</div>

<div class="w1180" style="margin-top:8px;">
<center><script src="http://www.guoxuedashi.com/img/plus.php?id=3"></script></center>
</div>
<div class="w1180 foot">
<a href="/b/thanks.php">特别致谢</a> | <a href="javascript:window.external.AddFavorite(document.location.href,document.title);">收藏本站</a> | <a href="#">欢迎投稿</a> | <a href="http://www.guoxuedashi.com/forum/">意见建议</a> | <a href="http://www.guoxuemi.com/">国学迷</a> | <a href="http://www.shuowen.net/">说文网</a><script language="javascript" type="text/javascript" src="https://js.users.51.la/17753172.js"></script><br />
  Copyright &copy; 国学大师 古典图书集成 All Rights Reserved.<br>
  
  <span style="font-size:14px">免责声明:本站非营利性站点,以方便网友为主,仅供学习研究。<br>内容由热心网友提供和网上收集,不保留版权。若侵犯了您的权益,来信即刪。scp168@qq.com</span>
  <br />
ICP证:<a href="http://www.beian.miit.gov.cn/" target="_blank">鲁ICP备19060063号</a></div>
<!-- 页脚END --> 
<script src="http://www.guoxuedashi.com/img/plus.php?id=22"></script>
<script src="http://www.guoxuedashi.com/img/tongji.js"></script>

</body>
</html>
