<!DOCTYPE html PUBLIC "-//W3C//DTD XHTML 1.0 Transitional//EN" "http://www.w3.org/TR/xhtml1/DTD/xhtml1-transitional.dtd">
<html xmlns="http://www.w3.org/1999/xhtml">
<head>
<meta http-equiv="Content-Type" content="text/html; charset=utf-8" />
<meta http-equiv="X-UA-Compatible" content="IE=Edge,chrome=1">
<title>資治通鑒_115-資治通鑑卷一百十四_115-資治通鑑卷一百十四</title>
<meta name="Keywords" content="資治通鑒_115-資治通鑑卷一百十四_115-資治通鑑卷一百十四">
<meta name="Description" content="資治通鑒_115-資治通鑑卷一百十四_115-資治通鑑卷一百十四">
<meta http-equiv="Cache-Control" content="no-transform" />
<meta http-equiv="Cache-Control" content="no-siteapp" />
<link href="/img/style.css" rel="stylesheet" type="text/css" />
<script src="/img/m.js?2020"></script> 
</head>
<body>
 <div class="ClassNavi">
<a  href="/24shi/">二十四史</a> | <a href="/SiKuQuanShu/">四库全书</a> | <a href="http://www.guoxuedashi.com/gjtsjc/"><font  color="#FF0000">古今图书集成</font></a> | <a href="/renwu/">历史人物</a> | <a href="/ShuoWenJieZi/"><font  color="#FF0000">说文解字</a></font> | <a href="/chengyu/">成语词典</a> | <a  target="_blank"  href="http://www.guoxuedashi.com/jgwhj/"><font  color="#FF0000">甲骨文合集</font></a> | <a href="/yzjwjc/"><font  color="#FF0000">殷周金文集成</font></a> | <a href="/xiangxingzi/"><font color="#0000FF">象形字典</font></a> | <a href="/13jing/"><font  color="#FF0000">十三经索引</font></a> | <a href="/zixing/"><font  color="#FF0000">字体转换器</font></a> | <a href="/zidian/xz/"><font color="#0000FF">篆书识别</font></a> | <a href="/jinfanyi/">近义反义词</a> | <a href="/duilian/">对联大全</a> | <a href="/jiapu/"><font  color="#0000FF">家谱族谱查询</font></a> | <a href="http://www.guoxuemi.com/hafo/" target="_blank" ><font color="#FF0000">哈佛古籍</font></a> 
</div>

 <!-- 头部导航开始 -->
<div class="w1180 head clearfix">
  <div class="head_logo l"><a title="国学大师官网" href="http://www.guoxuedashi.com" target="_blank"></a></div>
  <div class="head_sr l">
  <div id="head1">
  
  <a href="http://www.guoxuedashi.com/zidian/bujian/" target="_blank" ><img src="http://www.guoxuedashi.com/img/top1.gif" width="88" height="60" border="0" title="部件查字,支持20万汉字"></a>


<a href="http://www.guoxuedashi.com/help/yingpan.php" target="_blank"><img src="http://www.guoxuedashi.com/img/top230.gif" width="600" height="62" border="0" ></a>


  </div>
  <div id="head3"><a href="javascript:" onClick="javascript:window.external.AddFavorite(window.location.href,document.title);">添加收藏</a>
  <br><a href="/help/setie.php">搜索引擎</a>
  <br><a href="/help/zanzhu.php">赞助本站</a></div>
  <div id="head2">
 <a href="http://www.guoxuemi.com/" target="_blank"><img src="http://www.guoxuedashi.com/img/guoxuemi.gif" width="95" height="62" border="0" style="margin-left:2px;" title="国学迷"></a>
  

  </div>
</div>
  <div class="clear"></div>
  <div class="head_nav">
  <p><a href="/">首页</a> | <a href="/ShuKu/">国学书库</a> | <a href="/guji/">影印古籍</a> | <a href="/shici/">诗词宝典</a> | <a   href="/SiKuQuanShu/gxjx.php">精选</a> <b>|</b> <a href="/zidian/">汉语字典</a> | <a href="/hydcd/">汉语词典</a> | <a href="http://www.guoxuedashi.com/zidian/bujian/"><font  color="#CC0066">部件查字</font></a> | <a href="http://www.sfds.cn/"><font  color="#CC0066">书法大师</font></a> | <a href="/jgwhj/">甲骨文</a> <b>|</b> <a href="/b/4/"><font  color="#CC0066">解密</font></a> | <a href="/renwu/">历史人物</a> | <a href="/diangu/">历史典故</a> | <a href="/xingshi/">姓氏</a> | <a href="/minzu/">民族</a> <b>|</b> <a href="/mz/"><font  color="#CC0066">世界名著</font></a> | <a href="/download/">软件下载</a>
</p>
<p><a href="/b/"><font  color="#CC0066">历史</font></a> | <a href="http://skqs.guoxuedashi.com/" target="_blank">四库全书</a> |  <a href="http://www.guoxuedashi.com/search/" target="_blank"><font  color="#CC0066">全文检索</font></a> | <a href="http://www.guoxuedashi.com/shumu/">古籍书目</a> | <a   href="/24shi/">正史</a> <b>|</b> <a href="/chengyu/">成语词典</a> | <a href="/kangxi/" title="康熙字典">康熙字典</a> | <a href="/ShuoWenJieZi/">说文解字</a> | <a href="/zixing/yanbian/">字形演变</a> | <a href="/yzjwjc/">金 文</a> <b>|</b>  <a href="/shijian/nian-hao/">年号</a> | <a href="/diming/">历史地名</a> | <a href="/shijian/">历史事件</a> | <a href="/guanzhi/">官职</a> | <a href="/lishi/">知识</a> <b>|</b> <a href="/zhongyi/">中医中药</a> | <a href="http://www.guoxuedashi.com/forum/">留言反馈</a>
</p>
  </div>
</div>
<!-- 头部导航END --> 
<!-- 内容区开始 --> 
<div class="w1180 clearfix">
  <div class="info l">
   
<div class="clearfix" style="background:#f5faff;">
<script src='http://www.guoxuedashi.com/img/headersou.js'></script>

</div>
  <div class="info_tree"><a href="http://www.guoxuedashi.com">首页</a> > <a href="/SiKuQuanShu/fanti/">四库全书</a>
 > <h1>资治通鉴</h1> <!--         下载:【右键另存为】即可 --></div>
  <div class="info_content zj clearfix">
  
<div class="info_txt clearfix" id="show">
<center style="font-size:24px;">115-資治通鑑卷一百十四</center>
    資治通鑑卷一百十四<br />
<br />
  宋 司馬光 撰<br />
<br />
  胡三省 音注<br />
<br />
  晉紀三十六【起旃蒙大荒落盡著雍涒灘凡四年】<br />
<br />
  安皇帝已<br />
<br />
  義熙元年春正月南陽太守扶風魯宗之起兵襲襄陽桓蔚走江陵【蔚紆勿翻】己丑劉毅等諸軍至馬頭桓振挾帝出屯江津【江津戍在江陵南臨江滸荆州記曰江陵縣東三里有津鄉水經注江陵城南有馬牧城此洲始自枚回下迄于此長七十餘里洲上有奉城江津長所治】遣使求割江荆二州奉送天子毅等不許辛卯宗之擊破振將温楷于柞溪【水經注柞溪水出江陵縣北蓋諸池散流咸所會合積以成川東流逕魯宗之壘南又東注船官湖將即亮翻柞才各翻又音作】進屯紀南【郡國志江陵縣北十餘里有紀南城】振留桓謙馮該守江陵引兵與宗之戰大破之劉毅等擊破馮該於豫章口【水經注江水過江陵而東得豫章口夏水所通也西北有豫章岡蓋因岡而得名其地去江陵城二十里】桓謙棄城走毅等入江陵執卞範之等斬之桓振還望見火起知城已陷其衆皆潰振逃于溳川【水經注溳水出漢南陽郡蔡陽縣東南大洪山東南流過隨縣西又南過江夏安陸縣西又東南入于夏溳音云】乙未詔大處分悉委冠軍將軍劉毅【處昌呂翻分扶問翻冠古玩翻】戊戌大赦改元惟桓氏不原以桓冲忠于王室特宥其孫胤以魯宗之為雍州刺史毛璩為征西將軍都督益梁秦凉寧五州諸軍事璩弟瑾為梁秦二州刺史瑗為寧州刺史【雍於用翻璩求於翻瑾渠吝翻瑗于眷翻】劉懷肅追斬馮該於石城桓謙桓怡桓蔚桓謐何澹之温楷皆奔秦【蔚紆勿翻澹徒覽翻】怡弘之弟也【桓弘死見上卷上年】燕王熙伐高句麗【句如字又音駒麗力知翻】戊申攻遼東城且陷熙命將士毋得先登俟剗平其城朕與皇后乘輦而入【剗楚限翻】由是城中得嚴備不克而還【後齊高緯之攻晉州亦若是矣還從宣翻又如字】秦王興以鳩摩羅什為國師奉之如神親帥羣臣及沙門聽羅什講佛經【帥讀曰率下同】又命羅什翻譯西域經論三百餘卷【古之譯者傳四夷之言今羅什翻夷言為華言故曰譯】大營塔寺沙門坐禪者常以千數【禪静也寂也傳燈録曰禪有五有凡夫禪有外道禪有小乘禪有大乘禪有最上乘禪禪時連翻】公卿以下皆奉佛由是州郡化之事佛者十室而九 乞伏乾歸擊吐谷渾大孩大破之俘萬餘口而還大孩走死胡園【晉書吐谷渾傳吐谷渾王烏紇堤一名大孩胡園作胡國孩何開翻】視羆世子樹洛干帥其餘衆數千家奔莫何川【莫何山在西傾山東北西傾亦名嵹臺山帥讀曰率】自稱車騎大將軍大單于吐谷渾王【騎奇寄翻單音蟬】樹洛干輕徭薄賦信賞必罰吐谷渾復興【復扶又翻】沙漒諸戎皆附之【段國曰澆河郡西南一百七十里有黄沙南北一百二十里東西七十里西極大揚川望之若人委糒糠於地不生草木蕩然黄沙周迴數百里洮水出嵹臺山東北逕吐谷渾中自洮嵹南北三百里中地草皆是龍鬚而無樵柴謂之嵹川漒渠良翻】 西涼公暠自稱大將軍大都督領秦凉二州牧【暠古老翻】大赦改元建初遣舍人黄始梁興間行奉表詣建康【間古莧翻】 二月丁巳留臺備法駕迎帝於江陵劉毅劉道規留屯夏口【夏戶雅翻】何無忌奉帝東還 初毛璩聞桓振陷江陵帥衆三萬順流東下將討之使其弟西夷校尉瑾蜀郡太守瑗出外水【璩求於翻瑾渠吝翻瑗于眷翻蜀有内水外水内水涪水也外水即蜀江源於岷山者】參軍巴西譙縱侯暉出涪水蜀人不樂遠征暉至五城水口【涪音浮樂音洛水經注涪水自南安郡南流其枝流西逕廣漢五城縣為五城水又西至成都入于江又曰江水東絶緜洛逕五城界至廣都北岸南入于江謂之五城水口斯為北江沈約宋志五城縣屬廣漢郡晉武帝咸寜四年立華陽國志云漢時立倉五縣人尉部主之晉因立五城縣在五城山按五代志蜀郡玄武縣舊曰伍城玄武縣唐屬梓州】與巴西陽昧謀作亂【昩莫葛翻】縱為人和謹蜀人愛之暉昩共逼縱為主縱不可走投于水引出以兵逼縱登輿縱又投地叩頭固辭暉縛縱於輿還襲毛瑾於涪城殺之【涪音浮】推縱為梁秦二州刺史璩至略城【據晉書毛璩傳略城去成都四百里】聞變奔還成都遣參軍王瓊將兵討之【將即亮翻】為縱弟明子所敗【敗補邁翻】死者什八九益州營戶李騰開城納縱兵【民有流離逃叛分配軍營者為營戶】殺璩及弟瑗滅其家縱稱成都王以從弟洪為益州刺史以明子為巴州刺史屯白帝於是蜀大亂漢中空虚氐王楊盛遣其兄子平南將軍撫據之 癸亥魏主珪還自豺山罷尚書三十六曹【魏三十六曹始見於一百九卷隆安元年】 三月桓振自鄖城襲江陵【杜預曰江夏雲杜縣東南有鄖城古鄖子之國鄖音云振先逃于鄖川鄖城蓋在鄖川也鄖音云】荆州刺史司馬休之戰敗奔襄陽振自稱荆州刺史建威將軍劉懷肅自雲杜引兵馳赴與振戰於沙橋【沙橋在江陵城北】劉毅遣廣武將軍唐興助之臨陳斬振【陳讀曰陣】復取江陵【復扶又翻下復詣尋復復說同】甲午帝至建康乙未百官詣闕請罪詔令復職尚書殷仲文以朝廷音樂未備言於劉裕請治之【治直之翻】裕曰今日不暇給且性所不解【解戶買翻曉也下同】仲文曰好之自解裕曰正以解則好之故不習耳【英雄之言政自度越常流世之嗜音者可以自省矣好呼到翻】 庚子以琅邪王德文為大司馬武陵王遵為太保劉裕為侍中車騎將軍都督中外諸軍事徐青二州刺史如故劉毅為左將軍何無忌為右將軍督豫州揚州五郡軍事豫州刺史劉道規為輔國將軍督淮北諸軍事并州刺史魏詠之為征虜將軍吳國内史裕固讓不受加録尚書事又不受屢請歸藩【歸藩歸京口也】詔百官敦勸帝親幸其第裕惶懼復詣闕陳請乃聽歸藩以魏詠之為荆州刺史代司馬休之初劉毅嘗為劉敬宣寧朔參軍【劉敬宣為寜朔將軍毅為參軍】時人或以雄傑許之敬宣曰夫非常之才自有調度【調徒弔翻】豈得便謂此君為人豪邪此君之性外寛而内忌自伐而尚人若一旦遭遇亦當以陵上取禍耳【敬宣之論毅其知之固審矣然幾以此掇禍聖人包周身之防正為是耳】毅聞而恨之及敬宣為江州辭以無功不宜授任先於毅等【先悉薦翻】裕不許毅使人言於裕曰劉敬宣不豫建義猛將勞臣方須叙報【將即亮翻】如敬宣之比宜令在後若使君不忘平生【裕參劉牢之軍事牢之父子雅敬待之故云然】止可為員外常侍耳【員外散騎常侍魏末置】聞已授郡實為過優【敬宣自北來歸裕以為晉陵太守】尋復為江州尤用駭惋【惋烏貫翻】敬宣愈不自安自表解職乃召還為宣城内史 夏四月劉裕旋鎮京口改授都督荆司等十六州諸軍事加領兖州刺史 盧循遣使貢獻【使疏吏翻】時朝廷新定未暇征討壬申以循為廣州刺史徐道覆為始興相循遺劉裕益智粽【遺于季翻本草曰益智子生崑崙國今嶺南州郡往往有之顧微交州記曰益智葉如蘘荷莖如竹箭子從心出一枝有十子子肉白滑四破去之蜜煮為粽味辛粽作弄翻角黍也】裕報以續命湯【循以益智調裕裕以續命報之此雖淺陋亦兵機也】循以前琅邪内史王誕為平南長史誕說循曰誕本非戎旅在此無用【說輸芮翻王氏江南衣冠稱首故云本非戎旅】素為劉鎮軍所厚若得北歸必蒙寄任公私際會仰答厚恩循甚然之劉裕與循書令遣吳隱之還循不從誕復說循曰【復扶又翻】將軍今留吳公公私非計孫伯符豈不欲留華子魚邪但以一境不容二君耳於是循遣隱之與誕俱還【元興元年桓玄流王誕於嶺南二年盧循破廣州虜吳隱之誕并沒於循所漢獻帝建安四年華歆以豫章歸孫策策死曹操表召歆孫權遣還許華戶化翻】 初南燕主備德仕秦為張掖太守【事見一百二卷海西公太和五年】其兄納與母公孫氏居于張掖備德之從秦王堅寇淮南也【寇淮南見一百五卷孝武帝太元八年】留金刀與其母别備德與燕王垂舉兵於山東張掖太守苻昌收納及備德諸子皆誅之公孫氏以老獲免納妻段氏方娠未決獄掾呼延平備德之故吏也【掾于絹翻】竊以公孫氏及段氏逃于羌中段氏生子超十歲而公孫氏病臨卒以金刀授超曰汝得東歸當以此刀還汝叔也呼延平又以超母子奔涼及呂隆降秦超隨涼州民徙長安【秦徙涼州民事見上卷元興二年】平卒段氏為超娶其女為婦超恐為秦人所録【為于偽翻録采也收也為所收采則不得歸南燕矣】乃陽狂行乞秦人賤之惟東平公紹見而異之言於秦王興曰慕容超姿幹瓌偉【瓌公回翻】殆非真狂願微加官爵以縻之興召見與語超故為謬對或問而不答興謂紹曰諺云妍皮不裹癡骨徒妄語耳乃罷遣之備德聞納有遺腹子在秦遣濟隂人吳辯往視之【濟子禮翻】辯因鄉人宗正謙賣卜在長安以告超【宗正以官為氏】超不敢告其母妻潛與謙變姓名逃歸南燕行至梁父【父音甫】鎮南長史悦夀以告兖州刺史慕容法【南燕以法為兖州刺史鎮梁父】法曰昔漢有卜者詐稱衛太子【見二十三卷漢昭帝始元五年】今安知非此類也不禮之超由是與法有隙【為超立法謀反張本】備德聞超至大喜遣騎三百迎之【騎奇寄翻下同】超至廣固以金刀獻於備德備德慟哭悲不自勝【勝音升】封超為北海王拜侍中驃騎大將軍司隸校尉【驃匹妙翻】開府妙選時賢為之僚佐備德無子欲以超為嗣超入則侍奉盡歡出則傾身下士【下戶嫁翻】由是内外譽望翕然歸之 五月桂陽太守章武王秀【義陽王望子河間王洪生子威徙封章武傳至孫無嗣河間王欽以子範之繼之秀範之子也】及益州刺史司馬軌之謀反伏誅秀妻桓振之妹也故自疑而反桓玄餘黨桓亮苻宏等擁衆寇亂郡縣者以十數劉<br />
<br />
  毅劉道規檀祗等分兵討滅之荆湘江豫皆平詔以毅為都督淮南等五郡軍事豫州刺史【淮南廬江歷陽晉熙安豐凡五郡】何無忌為都督江東五郡軍事會稽内史【會工外翻】 北青州刺史劉該反引魏為援【隆安五年劉該固嘗降魏矣沈約曰江左青州治廣陵】清河陽平二郡太守孫全聚衆應之六月魏豫州刺史索度真大將斛斯蘭【斛斯亦虜姓也】寇徐州圍彭城劉裕遣其弟南彭城内史道憐東海太守孟龍符將兵救之【將即亮翻】斬該及全魏兵敗走龍符懷玉之弟也 秦隴西公碩德伐仇池屢破楊盛兵將軍斂俱攻漢中【歛羌之種姓俱其名】拔成固徙流民三千餘家於關中秋七月楊盛請降於秦【降戶江翻】秦以盛為都督益寧二州諸軍事征南大將軍益州牧 劉裕遣使求和於秦【使疏吏翻】且求南鄉等諸郡秦王興許之羣臣咸以為不可興曰天下之善一也劉裕拔起細微能誅討桓玄興復晉室内釐庶政外修封疆吾何惜數郡不以成其美乎遂割南鄉順陽新野舞隂等十二郡歸于晉【隆安二年淮漢以北多降于秦此十二郡蓋皆在漢北漢建安中割南陽右壤為南鄉郡晉立順陽郡以南鄉為縣盖其後復分立郡也案晉南鄉郡秦漢隂縣及鄼縣之地今為光化軍舞隂縣屬南陽郡未知立郡之始】八月燕遼西太守邵顔有罪亡命為盜九月中常侍郭仲討斬之 汝水竭【汝當作女郭緣生述征記齊桓公冢在齊城南二十里冢東有女水或曰齊桓公女冢在其上故以名水女水導川東北流甚有神焉化隆則水生政薄則津竭地理志葂頭山女水所出東北至臨菑入鉅淀鉅淀即漢鉅定地晉書地理志女水出齊國東安平縣東北】南燕主備德惡之【惡烏路翻】俄而寢疾北海王超請禱之備德曰人主之命短長在天非汝水所能制也固請不許戊午備德引見羣臣于東陽殿【見賢遍翻】議立超為太子俄而地震百官驚恐備德亦不自安還宫是夜疾篤瞑不能言段后大呼今召中書作詔立超可乎【呼火故翻】備德開目頷之乃立超為皇太子大赦備德尋卒【年七十】為十餘棺夜分出四門潛瘞山谷【瘞一計翻】己未超即皇帝位【超字祖明德兄北海王納之子】大赦改元太上尊段后為皇太后以北地王鍾都督中外諸軍録尚書事慕容法為征南大將軍都督徐兖揚南兖四州諸軍事加慕容鎮開府儀同三司以尚書令封孚為大尉麴【麴當作鞠】仲為司空封嵩為尚書左僕射癸亥虚葬備德於東陽陵諡曰獻武皇帝廟號世宗超引所親公孫五樓為腹心備德故大臣北地王鍾段宏等皆不自安求補外職超以鍾為青州牧宏為徐州刺史公孫五樓為武衛將軍領屯騎校尉内參政事封孚諫曰臣聞親不處外羇不處内【左傳申無宇諫楚靈王曰親不在外羇不在内處昌呂翻】鍾國之宗臣社稷所賴宏外戚懿望百姓具瞻正應參翼百揆不宜遠鎮外方今鍾等出藩五樓内輔臣竊未安超不從鍾宏心皆不平相謂曰黄犬之皮恐終補狐裘也【史記騶忌相齊淳于髠謂之曰狐裘雖弊不可補以黄狗之皮騶忌曰謹受令請謹擇君子母雜小人其間】五樓聞而恨之 魏詠之卒江陵令羅脩謀舉兵襲江陵奉王慧龍為主劉裕以并州刺史劉道規為都督荆寧等六州諸軍事荆州刺史脩不果發奉慧龍奔秦【慧龍得免見上卷元興三年】 乞伏乾歸伐仇池為楊盛所敗【敗補邁翻】西涼公暠與長史張邈謀徙都酒泉以逼沮渠蒙遜【沮子余翻】以張體順為建康太守鎮樂涫【漢志樂涫縣屬酒泉郡張氏分為建康郡涫音官】以宋繇為敦煌護軍與其子敦煌太守讓鎮敦煌遂遷于酒泉【李暠遷酒泉欲以逼沮渠蒙遜安知反為蒙遜所逼邪敦徒門翻】暠手令戒諸子以為從政者當審慎賞罰勿任愛憎近忠正遠佞諛【近其靳翻遠于願翻】勿使左右竊弄威福毁譽之來當研覈真偽聽訟折獄必和顔任理慎勿逆詐億必【朱子曰逆未至而迎之也詐謂人欺已也億未見而意之也必期必也譽音余】輕加聲色務廣咨詢勿自專用吾莅事五年雖未能息民然含垢匿瑕朝為寇讐夕委心膂粗無負於新舊【粗坐五翻】事任公平坦然無纇【纇盧對翻絲節也庇也】初不容懷有所損益計近則如不足經遠乃為有餘庶亦無愧前人也 十二月燕王熙襲契丹【契丹本東胡種其先為匈奴所破保鮮卑山魏青龍中部酋軻比能桀驁為幽州刺史王雄所殺部衆遂微逃潢水之南黄龍之北後自號曰契丹種類繁盛契欺吉翻程大昌曰契丹之契讀如喫】<br />
<br />
  二年春正月甲申魏主珪如豺山宫諸州置三刺史郡置三太守縣置三令長刺史令長各之州縣【長知兩翻之往也】太守雖置而未臨民功臣為州者皆徵還京師以爵歸第 益州刺史司馬榮期擊譙明子于白帝破之 燕王熙至陘北【陘北冷陘山之北也陘音刑】畏契丹之衆欲還苻后不聽戊申遂棄輜重【重直用翻】輕兵襲高句麗 南燕主超猜虐日甚政出權倖盤于遊畋【盤樂也言樂于田獵遊逸】封孚韓屢諫不聽【丁度曰竹角翻】超嘗臨軒問孚曰朕可方前世何主對曰桀紂超慙怒孚徐步而出不為改容【為于偽翻】鞠仲謂孚曰與天子言何得如是宜還謝孚曰行年七十惟求死所耳竟不謝超以其時望優容之 桓玄之亂河間王曇之子國璠叔璠奔南燕【河間王顒死無後元帝以彭城王植子融為顒嗣薨又無子帝復尋以彭城王釋子欽為融嗣欽薨曇之嗣曇之薨國鎮嗣國璠盖國鎮兄弟劉裕興復簒意未彰國璠宜如劉敬宣輩南歸可也乃攻擾晉邊者欽孫秀嗣封章武國璠從兄弟也秀以桓振妹婿謀反誅故國璠兄弟不敢南歸耳豈知裕之必簒哉曇徒含翻璠孚袁翻】二月甲戌國璠等攻陷弋陽 燕軍行三千餘里士馬疲凍死者屬路【屬之欲翻】攻高句麗木底城不克而還【木底城在南蘇之東唐置木底州句如字又音駒麗力知翻】夕陽公雲傷於矢且畏燕王熙之虐遂以疾去官【為後燕人弑熙立雲張本】 三月庚子魏主珪還平城夏四月庚申復如豺山宫【復扶又翻】甲午還平城 柔然社崘侵魏邊【崘盧昆翻】五月燕主寶之子博陵公䖍上黨公昭皆以嫌疑賜死六月秦隴西公碩德自上邽入朝秦王興為之大赦<br />
<br />
  【為于偽翻】及歸送之至雍乃還【雍於用翻】興事晉公緒及碩德皆如家人禮車馬服玩先奉二叔而自服其次國家大政皆咨而後行 秃髪傉檀伐沮渠蒙遜【傉奴沃翻沮子余翻】蒙遜嬰城固守傉檀至赤泉而還【赤泉在張掖氐池縣北】獻馬三千匹羊三萬口于秦秦王興以為忠以傉檀為都督河右諸軍事車騎大將軍凉州刺史鎮姑臧徵王尚還長安凉州人申屠英等遣主簿胡威詣長安請留尚興弗許威見興流涕言曰臣州奉戴王化於兹五年【隆安五年九月呂隆降秦至是猶未五朞】土宇僻遠威靈不接士民嘗膽抆血共守孤城【抆武粉翻又文運翻拭也】仰恃陛下聖德俯杖良牧仁政克自保全以至今日陛下奈何乃以臣等貿馬三千匹羊三萬口【貿音茂易也市賣也】賤人貴畜【畜許又翻】毋乃不可若軍國須馬直煩尚書一符臣州三千餘戶各輸一馬朝下夕辦何難之有昔漢武傾天下之資力開拓河西以斷匈奴右臂今陛下無故棄五郡之地忠良華族以資暴虜【斷丁管翻此五郡謂漢所開武威張掖敦煌酒泉金城】豈惟臣州士民墜于塗炭恐方為聖朝旰食之憂【旰古案翻】興悔之使西平人車普馳止王尚又遣使諭傉檀會傉檀已帥步騎三萬軍于五澗【五澗在姑臧南車尺遮翻使疏吏翻帥讀曰率】普先以狀告之傉檀遽逼遣王尚尚出自清陽門傉檀入自凉風門【清陽當作青陽凉風門姑臧城南門】别駕宗敞送尚還長安【宗姓也漢有南陽宗資】傉檀謂敞曰吾得凉州三千餘家情之所寄惟卿一人奈何捨我去乎敞曰今送舊君所以忠於殿下也傉檀曰吾新牧貴州懷遠安邇之略如何敞曰凉土雖弊形勝之地殿下惠撫其民收其賢俊以建功名其何求不獲因薦本州文武名士十餘人傉檀嘉納之王尚至長安興以為尚書傉檀燕羣臣於宣德堂仰視嘆曰古人有言作者不居居者不作信矣武威孟禕曰昔張文王始為此堂於今百年十有二主矣【張駿卒私諡曰文王張氏自駿至重華耀靈祚玄靚天錫凡六主梁熙呂光呂紹呂纂呂隆王尚又六主通十二主禕許韋翻】惟履信思順者可以久處【處昌呂翻】傉檀善之 魏主珪規度平城【度徒洛翻】欲擬鄴洛長安修廣宫室以濟陽太守莫題有巧思【惠帝分陳留為濟陽國後因以為郡濟子禮翻思相吏翻】召見與之商功題久侍稍怠珪怒賜死題含之孫也【此莫題非高邑公莫題莫含見八十九卷愍帝建興三年】於是發八部五百里内男丁築灅南宫【魏先有八部大人既得中原建平城為代都分布八部於畿内灅力水翻】闕門高十餘丈【高居傲翻】穿溝池廣苑囿規立外城方二十里分置市里三十日罷 秋七月魏太尉宜都丁公穆崇薨【周公諡法述義不克曰丁】 八月秃髮傉檀以興城侯文支鎮姑臧自還樂都【樂音洛】雖受秦爵命然其車服禮儀皆如王者甲辰魏主珪如豺山宫遂之石漠【自隂山以北皆大漠有白漠黑漠石漠白黑二漠以其色為名石漠蓋其地皆石據北史石漠在漢定襄郡武要縣西北塞外】九月度漠北癸巳南還長川【水經注長川城在柔玄鎮西】劉裕聞譙縱反遣龍驤將軍毛修之將兵與司馬榮期文處茂時延祖共討之【處昌呂翻】修之至宕渠【宕渠縣漢屬巴郡劉蜀分屬巴西郡惠帝復分巴西置宕渠郡按五代志果州南充縣舊置宕渠郡合州石鏡縣亦置宕渠郡皆當自白帝上成都之路宕徒浪翻】榮期為其參軍楊承祖所殺承祖自稱巴州刺史修之退還白帝秃髮傉檀求好於西凉【好呼到翻】西凉公暠許之沮渠蒙<br />
<br />
  遜襲酒泉至安珍【安珍即漢酒泉郡安彌縣也後人從省書之以彌為弥傳寫之訛又以弥為珍】暠戰敗城守蒙遜引還 南燕公孫五樓欲擅朝權【朝直遙翻】譛北地王鍾於南燕主超請誅之南燕主備德之卒也【卒子恤翻】慕容法不奔喪超遣使讓之【使疏吏翻】法懼遂與鍾及段宏謀反超聞之徵鍾鍾稱疾不至超收其黨侍中慕容統等殺之征南司馬卜珍告左僕射封嵩數與法往來疑有姧超收嵩下廷尉太后懼泣告超曰嵩數遣黄門令牟常說吾云帝非太后所生恐依永康故事【數所角翻說輸芮翻下遐稼翻燕主寶永康元年逼殺其母段氏事見一百八卷孝武帝太元二十一年】我婦人識淺恐帝見殺即以語法法為謀見誤知復何言【語牛倨翻復扶又翻】超乃車裂嵩西中郎將封融犇魏超遣慕容鎮攻青州慕容昱攻徐州右僕射濟陽王凝及韓範攻兖州【南燕青州刺史鎮東萊徐州刺史鎮莒城兖州刺史鎮梁父濟子禮翻】昱拔莒城段宏犇魏封融與羣盜襲石塞城殺鎮西大將軍餘鬱國中振恐濟陽王凝謀殺韓範襲廣固範知之勒兵攻凝凝犇梁父範并將其衆攻梁父克之【父音甫】法出犇魏凝出犇秦慕容鎮克青州鍾殺其妻子為地道以出與高都公始皆奔秦秦以鍾為始平太守凝為侍中南燕主超好變更舊制朝野多不悦又欲復肉刑增置烹轘之法【好呼到翻更工衡翻轘戶慣翻車裂也】衆議不合而止冬十月封孚卒【慕容超之立雖非少主乃國疑而大臣未附之時超不能推心和輯使之阻兵以至于奔亡超誰與立哉雖劉裕之兵未至固知其必滅矣】 尚書論建義功奏封劉裕豫章郡公劉毅南平郡公何無忌安成郡公自餘封賞有差 梁州刺史劉稚反劉毅遣將討禽之【將即亮翻】 庚申魏主珪還平城 乙亥以左將軍孔安國為尚書左僕射 十一月秃髮傉檀遷于姑臧 乞伏乾歸入朝于秦【朝直遙翻】十二月以何無忌為都督荆江豫三州八郡軍事江<br />
<br />
  州刺史【八郡蓋荆州之武昌江州之尋陽豫章廬陵臨川鄱陽南康豫州之晉熙】 是歲桓石綏與司馬國璠陳襲聚衆胡桃山為寇【胡桃山當在歷陽郡界璠孚袁翻】劉毅遣司馬劉懷肅討破之石綏石生之弟也三年春正月辛丑朔燕大赦改元建始 秦王興以乞伏乾歸寖彊難制留為主客尚書【漢成帝置四曹尚書其四曰主客主外國夷狄事】以其世子熾磐行西夷校尉監其部衆【是後秦政漸衰熾磐日以盛而乾歸亦不可得而留矣熾昌志翻校戶教翻監工衡翻】 二月己酉劉裕詣建康固辭新所除官欲詣廷尉詔從其所守裕乃還丹徒 魏主珪立其子修為河間王處文為長樂王【處昌呂翻樂音洛】連為廣平王黎為京兆王殷仲文素有才望自謂宜當朝政悒悒不得志【悒悒憂悒不自安之意仲文黨於桓玄以才望希進而不得進故不自安也朝直遙翻】出為東陽太守尤不樂【樂音洛】何無忌素慕其名東陽無忌所統【無忌都督浙江東五郡東陽其一也】仲文許便道修謁無忌喜欽遲之【遲直吏翻待也後企遲同】而仲文失志恍惚遂不過府【府謂督府何無忌治所也恍呼廣翻惚音忽】無忌以為薄已大怒會南燕入寇無忌言於劉裕曰桓胤殷仲文乃腹心之疾北虜不足憂也閏月劉裕府將駱氷謀作亂【將即亮翻】事覺裕斬之因言氷與仲文桓石松曹靖之卞承之劉延祖潛相連結謀立桓胤為主皆族誅之 燕王熙為其后苻氏起承華殿【為于偽翻】負土於北門土與穀同價宿軍典軍杜静載棺詣闕極諫熙斬之【北燕營州刺史鎮宿軍】苻氏嘗季夏思凍魚【煎魚為凍今人多能之季夏六月暑盛則不能凍】仲冬須生地黄【本草曰地黄葉如甘露子花如脂麻花但有細斑點北人謂之牛妳子二月八月採根隂乾解諸熱破血通利月水】熙下有司切責不得而斬之【下遐嫁翻】夏四月癸丑苻氏卒熙哭之懣絶久而復蘇【懣音悶】喪之如父母服斬衰食粥【喪息郎翻衰倉回翻】命百官於宫内設位而哭使人案檢哭者無淚則罪之羣臣皆含辛以為淚高陽王妃張氏熙之嫂也【高陽王隆之妃】美而有巧思【思相吏翻】熙欲以為殉乃毁其禭鞾中得弊氈【禭音遂送終曰禭鞾許加翻】遂賜死右僕射韋璆等皆恐為殉沐浴俟命公卿以下至兵民戶率營陵費殫府藏【璆渠尤翻藏徂浪翻】陵周圍數里熙謂監作者曰善為之朕將繼往【觀熙此言死期將至監工銜翻】丁酉燕太后段氏去尊號出居外宫【去羌呂翻段氏熙之慈母也見上卷元興二年】 氐王楊盛以平北將軍符宣為梁州督護將兵入漢中【將即亮翻】秦梁州别駕呂瑩等起兵應之【呂營盖為秦之梁州别駕以後面秦王興遣南梁州刺史王敏敕譙縱觀之可見】刺史王敏攻之瑩等求援於盛盛遣軍臨濜口敏退屯武興【水經沔水東逕白馬戍南濜水入焉注云濜水北武都氐中南逕張魯城東又南過陽平關西而南入于沔謂之濜口有濜口城郡國縣道記梁州西縣本名白馬城又曰濜口城劉蜀置武興督於漢中沔陽縣隋唐為興州今沔州城古武興城也濜音徐刃翻】盛復通於晉【孝武太元十九年楊盛來稱藩其所以不與晉通者以桓玄之亂也復音扶又翻】晉以盛為都督隴右諸軍事征西大將軍開府儀同三司盛因以宣行梁州刺史【通鑑以晉紀年則以盛為都督之上不必書晉晉字當作詔字】 五月壬戌燕尚書郎苻進謀反誅進定之子也【孝武太元十一年苻定降燕見一百六卷】 魏主珪北巡至濡源【濡音乃官翻】 魏常山王遵以罪賜死 初魏主珪滅劉衛辰其子勃勃奔秦【見一百卷孝武太元十六年】秦高平公沒奕干以女妻之【妻音七細翻】勃勃魁岸美容儀性辯慧秦王興見而奇之與論軍國大事寵遇踰于勲舊興弟邕諫曰勃勃不可近也【近音其靳翻】興曰勃勃有濟世之才吾方與之平天下奈何逆忌之乃以為安遠將軍使助沒奕干鎮高平以三城朔方雜夷【魏收地形志偏城郡廣武縣有三城唐延州豐林縣古廣武縣地】及衛辰部衆三萬配之使伺魏間隙【間音古莧翻】邕固争以為不可興曰卿何以知其為人邕曰勃勃奉上慢御衆殘貪猾不仁輕為去就寵之踰分【分音扶問翻】恐終為邊患興乃止久之竟以勃勃為安北將軍五原公配以三交五部鮮卑及雜虜二萬餘落鎮朔方【為勃勃病秦興悔不用邕言張本】魏主珪歸所虜秦將唐小方于秦【將音即亮翻】秦王興請歸賀狄干仍送良馬千匹以贖狄伯支珪許之【秦留賀狄干見一百十二卷元興元年狄伯支唐小方被禽亦見是年】勃勃聞秦復與魏通而怒【復音扶又翻】乃謀叛秦【楚白公勝所以作亂於楚者其事正如此】柔然可汗社崘獻馬八千匹于秦至大城【大城縣前漢屬西河郡後漢屬朔方郡魏晉省崘盧昆翻】勃勃掠取之悉集其衆三萬餘人偽畋於高平川因襲殺沒奕干而并其衆勃勃自謂夏后氏之苗裔【史記及漢書皆云匈奴夏后氏苗裔淳維之後勃勃匈奴餘種故云然夏戶雅翻下同】六月自稱大夏天王大單于【晉書曰勃勃字屈孑匈奴右賢王去卑之後劉元海之族劉武之曾孫劉衛辰之子劉武即劉虎晉書避唐國諱改虎為武單音蟬】大赦改元龍升置百官以其兄右地代為丞相封代公力俟提為大將軍封魏公叱干阿利為御史大夫封梁公弟阿利羅引為司隸校尉若門為尚書令叱以鞬為左僕射【鞬居言翻】乙斗為右僕射賀狄干久在長安常幽閉因習讀經史舉止如儒者及還魏主珪見其言語衣服皆類秦人以為慕而效之怒并其弟歸殺之 秦王興以太子泓録尚書事 秋七月戊戌朔日有食之 汝南王遵之坐事死遵之亮之五世孫也【汝南王亮死於楚王瑋之難】 癸亥燕王熙葬其后苻氏于徽平陵喪車高大毁北門而出熙被髮徒跣步從二十餘里【被皮義翻從才用翻】甲子大赦初中衛將軍馮跋及弟侍御郎素弗皆得罪於熙熙欲殺之跋亡命山澤熙賦役繁數民不堪命【數所角翻】跋素弗與其從弟萬泥謀曰吾輩還首無路【還首自歸請罪也首式救翻】不若因民之怨共舉大事可以建公侯之業事之不捷死未晚也遂相與乘車使婦人御潛入龍城匿於北部司馬孫護之家及熙出送葬跋等與左衛將軍張興及苻進餘黨作亂跋素與慕容雲善乃推雲為主雲以疾辭【雲稱疾見上年】跋曰河間淫虐人神共怒此天亡之時也公高氏名家何能為人養子【為養子事見百九卷隆安元年】而棄難得之運乎扶之而出跋弟乳陳等帥衆攻弘光門【帥讀曰率】鼓噪而進禁衛皆散走遂入宫授甲閉門拒守中黄門趙洛生走告于熙熙曰鼠盜何能為朕當還誅之乃置后柩於南苑【柩巨救翻】收髮貫甲馳還赴難【難乃旦翻】夜至龍城攻北門不克宿於門外乙丑雲即天王位【雲字子雨祖父高和句麗之支庶慕容寶養以為子】大赦改元正始熙退入龍騰苑尚方兵褚頭踰城從熙稱營兵同心效順唯俟軍至熙聞之驚走而出左右莫敢迫熙從溝下潛遁良久左右怪其不還相與尋之唯得衣冠不知所適中領軍慕容拔謂中常侍郭仲曰大事垂捷而帝無故自驚深可怪也然城内企遲【遲直利翻待也】至必成功不可稽留吾當先往趣城【趣七喻翻】卿留待帝得帝速來若帝未還吾得如意安撫城中徐迎未晚乃分將壯士二千餘人登北城將士謂熙至皆投仗請降【將即亮翻降戶江翻】既而熙久不至拔兵無後繼衆心疑懼復下城赴苑【復扶又翻下復貳可復同】遂皆潰去拔為城中人所殺丙寅熙微服匿於林中為人所執送於雲雲數而殺之【年二十三史言慕容熙淫虐天奪其魄身死國滅載記曰自垂至熙四世凡二十四年而滅數所具翻】并其諸子雲復姓高氏幽州刺史上庸公懿以令支降魏【令音鈴又郎定翻支音祁】魏以懿為平州牧昌黎王懿評之孫也【前燕之亡慕容評之罪也】 魏主珪自濡源西如參合陂【濡乃官翻】乃還平城 秃髪傉檀復貳於秦【傉奴沃翻復扶又翻下同】遣使邀乞伏熾磐熾磐斬其使送長安【為秦襲傉檀張本使疏吏翻下同熾昌志翻】 南燕主超母妻猶在秦超遣御史中丞封愷使於秦以請之秦王興曰昔苻氏之敗太樂諸伎悉入于燕【長安之陷太樂諸伎入于西燕西燕之亡慕容垂收以歸于中山中山之陷相率奔鄴由是南燕得之伎渠綺翻】燕今稱藩送伎或送吳口千人所請乃可得也超與羣臣議之左僕射段暉曰陛下嗣守社稷不宜以私親之故遂降尊號且太樂先代遺音不可與也不如掠吳口與之尚書張華曰侵掠鄰國兵連禍結此既能往彼亦能來非國家之福也陛下慈親在人掌握豈可靳惜虚名不為之降屈乎【靳居焮翻為于偽翻】中書令韓範嘗與秦王俱為苻氏太子舍人若使之往必得如志超從之乃使韓範聘于秦稱藩奉表慕容凝言於興曰燕王得其母妻不可復臣宜先使送伎興乃謂範曰朕歸燕王家屬必矣然今天時尚熱當俟秋凉八月秦使員外散騎常侍韋宗聘於燕超與羣臣議見宗之禮張華曰陛下前既奉表今宜北面受詔封逞曰大燕七聖重光【自廆皝雋暐至垂德超凡七主重直龍翻】奈何一旦為豎子屈節超曰吾為太后屈願諸君勿復言【為于偽翻復扶又翻】遂北面受詔 毛修之與漢嘉太守馮遷合兵擊楊承祖斬之修之欲進討譙縱益州刺史鮑陋不可修之上表言人之所以重生實有生理可保臣之情地生塗已竭【謂其父瑾伯璩舉家為蜀人所滅修之欲致死復讐不復求生路也】所以借命朝露者【朝露易晞言不久生也】庶憑天威誅夷讐逆今屢有可乘之機而陋每違期不赴臣雖効死寇庭而救援理絶將何以濟劉裕乃表襄城太守劉敬宣帥衆五千伐蜀【晉氏南渡置襄城郡於江南仍領繁昌等縣孝武罷襄城郡為繁昌縣屬淮南僑郡今太平州繁昌縣即其地繁昌本漢潁川郡屬縣因僑立而是縣之名遂移于江南此襄城蓋敬宣以舊郡僑領太守也帥讀曰率】以劉道規為征蜀都督 魏主珪如豺山宫候官告司空庾岳服飾鮮麗行止風采擬則人君【候官見上卷元興三年】珪收岳殺之 北燕王雲以馮跋為都督中外諸軍事開府儀同三司録尚書事馮萬泥為尚書令馮素弗為昌黎尹馮弘為征東大將軍孫護為尚書左僕射張興為輔國大將軍弘跋之弟也 九月譙縱稱藩於秦秃髪傉檀將五萬餘人伐沮渠蒙遜【傉奴沃翻將即亮翻沮子余翻】蒙遜與戰於均石大破之【均石在張掖之東西陜之西蓋西郡界】蒙遜進攻西郡太守楊統於日勒降之【日勒縣漢屬張掖郡後分置西郡治日勒賢曰日勒故城在今甘州剛丹縣東南】 冬十月秦河州刺史彭奚念叛降於秃髮傉檀秦以乞伏熾磐行河州刺史【熾昌志翻】 南燕主超使左僕射張華給事中宗正元獻太樂伎一百二十人於秦秦王興乃還超母妻厚其資禮而遣之超親帥六宫迎於馬耳關【伎渠綺翻帥讀曰率據水經濟南臺縣有馬耳山關盧水出焉魏收地形志泰山郡臺縣有馬耳山】 夏王勃勃破鮮卑薛千等三部【薛千晉書載記作薛干蜀本作薛干】降其衆以萬數【降戶江翻】進攻秦三城已北諸戍斬秦將楊丕姚石生等諸將皆曰陛下欲經營關中宜先固根本使人心有所憑係高平山川險固土田饒沃可以定都勃勃曰卿知其一未知其二吾大業草創士衆未多姚興亦一時之雄諸將用命關中未可圖也我今專固一城彼必并力於我衆非其敵亡可立待不如以驍騎風馳【將即亮翻驍堅堯翻騎奇寄翻】出其不意救前則擊後救後則擊前使彼疲於犇命我則游食自若不及十年嶺北河東盡為我有待興既死嗣子闇弱徐取長安在吾計中矣於是侵掠嶺北嶺北諸城門不晝啟興乃歎曰吾不用黄兒之言以至于此【黄兒興弟邕小字也】勃勃求婚於秃髮傉檀傉檀不許十一月勃勃帥騎二萬擊傉檀至于支陽【枝陽縣漢屬金城郡晉張寔分屬廣武郡劉昫曰唐蘭州廣武縣漢枝陽縣杜佑曰唐會州會寧縣漢枝陽縣】殺傷萬餘人驅掠二萬七千餘口牛馬羊數十萬而還【還從宣翻又如字】傉檀帥衆追之【帥讀曰率】焦朗曰勃勃天姿雄健御軍嚴整未可輕也不如從温圍北渡趣萬斛堆【温圍水名水經河水北過武威温圍縣東北温圍其即漢之温圍縣歟趣七喻翻】阻水結營扼其咽喉【咽音烟】百戰百勝之術也傉檀將賀連怒曰勃勃敗亡之餘烏合之衆奈何避之示之以弱宜急追之傉檀從之勃勃於陽武下峽鑿凌埋車以塞路【凌力證翻氷也又閭承翻鑿氷塞路置兵死地使人自為戰塞悉則翻】勒兵逆擊傉檀大破之追奔八十餘里殺傷萬計名臣勇將死者什六七【將即亮翻】傉檀與數騎犇南山【漢書地理志武威郡蒼松縣有南山松陜予謂此南山自羌中連延西平金城之界東出秦雍至于終南皆此山也傉檀所奔枝陽之南山也】幾為追騎所得【幾居希翻騎奇寄翻】勃勃積尸而封之號曰髑髏臺【髑徒谷翻髏音婁】勃勃又敗秦將張佛生于青石原【敗補邁翻後漢書西羌傳安定有青石岸安定唐之涇州涇州有青石嶺】俘斬五千餘人傉檀懼外寇之逼徙三百里内民皆入姑臧國人駭怨屠各成七兒因之作亂【屠直於翻】一夕聚衆至數千人殿中都尉張猛大言於衆曰主上陽武之敗蓋恃衆故也責躬悔過何損於明而諸君遽從此小人為不義之事殿中兵今至禍在目前矣衆聞之皆散七兒犇晏然追斬之軍諮祭酒梁裒輔國司馬邊憲等謀反傉檀皆殺之【自是之後禿髪氏之勢日以衰矣】魏主珪還平城 十二月戊子武岡文恭侯王謐薨【諡法既過能改曰恭】 是歲西凉公暠以前表未報【前奉表見上元年暠古老翻】復遣沙門法泉間行奉表詣建康【復扶又翻間古莧翻】<br />
<br />
  四年春正月甲辰以琅邪王德文領司徒劉毅等不欲劉裕入輔政議以中領軍謝混為揚州刺史【王謐薨揚州刺史缺官故議用其人】或欲令裕於丹徒領揚州以内事付孟昶【昶丑兩翻】遣尚書右丞皮沈以二議諮裕【皮姓也沈持林翻】沈先見裕記室録事參軍劉穆之具道朝議【朝直遙翻】穆之偽起如厠密疏白裕曰皮沈之言不可從裕既見沈且令出外呼穆之問之穆之曰晉朝失政日久【朝直遙翻下同】天命已移公興復皇祚勲高位重今日形勢豈得居謙遂為守藩之將耶【將即亮翻】劉孟諸公與公俱起布衣共立大義以取富貴事有先後故一時相推非為委體心服宿定臣主之分也【分扶問翻】力敵勢均終相吞噬【後果如穆之之言】揚州根本所係不可假人前者以授王謐事出權道【見上卷元興三年】今若復以他授【復扶又翻下而復同】便應受制於人一失權柄無由可得將來之危難可熟念今朝議如此宜相酧答必云在我措辭又難唯應云神州治本宰輔崇要【治直吏翻】此事既大非可懸論便蹔入朝共盡同異公至京邑彼必不敢越公更授餘人明矣裕從之朝廷乃徵裕為侍中車騎將軍開府儀同三司揚州刺史録尚書事徐兖二州刺史如故【騎奇寄翻】裕表解兖州以諸葛長民為青州刺史鎮丹徒劉道憐為并州刺史戍石頭 庚申武陵忠敬王遵薨 魏主珪如豺山宫遂至甯川【甯川即後漢上谷郡之甯縣也前漢曰寧縣地理志曰于延水出代郡且如縣塞外東至寧入沽水經注曰于延水逕罡城南又東左與寧川水合水出小寧縣西北東南流注于延水又東逕小甯縣故城南地理志寧縣也師古曰且子如翻】 南燕主超尊其母段氏為皇太后妻呼延氏為皇后超祀南郊有獸如鼠而赤大如馬來至壇側須臾大風晝晦羽儀帷幄皆毁裂【賈公彦曰在旁曰帷四合象宫室曰幄】超懼以問太史令成公綏對曰陛下信用姧佞誅戮賢良賦斂繁多事役殷重之所致也【斂力贍翻】超乃大赦黜公孫五樓等俄而復用之【復扶又翻】 北燕主雲立妻李氏為皇后子彭城為太子三月庚申葬燕王熙及苻后于徽平陵諡熙曰昭文<br />
<br />
  皇帝 高句麗遣使聘北燕且叙宗族【雲本高句麗支屬詳見一百九卷隆安元年使疏吏翻】北燕王雲遣侍御史李拔報之 夏四月尚書左僕射孔安國卒甲午以吏部尚書孟昶代之【昶丑兩翻】 北燕大赦 五月北燕以尚書令馮萬泥為幽冀二州牧鎮肥如中軍將軍馮乳陳為并州牧鎮白狼【前漢右北平郡有白狼縣師古曰有白狼山故以名縣後漢晉省縣魏收地形志曰世祖太平真君八年置建德郡治白狼城其地屬唐營州柳城縣界】撫軍大將軍馮素弗為司隸校尉司隸校尉務銀提為尚書令 譙縱遣使稱藩於秦【使疏吏翻】又與盧循潛通縱上表請桓謙於秦欲與之共擊劉裕秦王興以問謙謙曰臣之累世著恩荆楚若得因巴蜀之資順流東下士民必翕然響應興曰小水不容巨魚若縱之才力自足辦事亦不假君以為鱗翼宜自求多福遂遣之謙至成都虚懷引士縱疑之置於龍格使人守之【龍格蓋即今成都府廣都縣龍爪灘之地】謙泣謂諸弟曰姚主之言神矣 秦王興以秃髮傉檀外内多難【時傉檀軍諮祭酒梁裒輔國司馬邊憲等謀反傉檀悉誅之晉書載記曰傉檀外有陽武之敗内有邊梁之亂難乃旦翻】欲因而取之使尚書郎韋宗往覘之【覘丑簾翻又丑艷翻】傉檀與宗論當世大略縱横無窮【縱子容翻】宗退歎曰奇才英器不必華夏【夏戶雅翻】明智敏識不必讀書吾乃今知九州之外五經之表復自有人也【傉檀之才辯内足以欺其父兄外足以欺敵人之覘國者而卒以敗亡者輕用兵也揆之於古蓋智伯瑶之流而才識又不及焉復扶又翻】歸言於興曰涼州雖弊傉檀權譎過人未可圖也【譎古宂翻】興曰劉勃勃以烏合之衆猶能破之况我舉天下之兵以加之乎宗曰不然形移勢變返覆萬端【蜀本作返復當從之】陵人者易敗【易以豉翻】戒懼者難攻傉檀之所以敗於勃勃者輕之也今我以大軍臨之彼必懼而求全臣竊觀羣臣才略無傉檀之比者雖以天威臨之亦未敢保其必勝也興不聽使其子中軍將軍廣平公弼後軍將軍歛成鎮遠將軍乞伏乾歸帥步騎三萬襲傉檀左僕射齊難帥騎二萬討勃勃【帥讀曰率下同】吏部尚書尹昭諫曰傉檀恃其險遠故敢違慢不若詔沮渠蒙遜及李暠討之使自相困斃不必煩中國之兵也亦不聽興遺傉檀書曰【遺于季翻】今遣齊難討勃勃恐其西逸故令弼等於河西邀之傉檀以為然遂不設備弼濟自金城【自金城濟河也】姜紀言於弼曰今王師聲言討勃勃傉檀猶豫守備未嚴願給輕騎五千【騎奇寄翻】掩其城門則山澤之民皆為吾有孤城無援可坐克也弼不從進至漠口【漠口在昌松郡界謂之昌松漠口魏收地形志昌松郡有漠口縣】昌松太守蘇霸閉城拒之弼遣人諭之使降霸曰汝棄信誓而伐與國吾有死而已何降之有【降戶江翻】弼進攻斬之長驅至姑臧傉檀嬰城固守出奇兵擊弼破之弼退據西苑城中人王鍾等謀為内應事洩傉檀欲誅首謀者而赦其餘前軍將軍伊力延侯曰今彊寇在外而姧人竊發於内危孰甚焉不悉阬之何以懲後傉檀從之殺五千餘人命郡縣悉散牛羊于野斂成縱兵鈔掠【鈔楚交翻】傉檀遣鎮北大將軍俱延鎮軍將軍敬歸等擊之秦兵大敗斬首七千餘級【傉檀散牛羊以餌敵而斂成掠之宜其敗也】姚弼固壘不出傉檀攻之未克秋七月興遣衛大將軍常山公顯帥騎二萬為諸軍後繼至高平聞弼敗倍道赴之顯遣善射者孟欽等五人挑戰於凉風門【挑徒了翻】弦未及發傉檀材官將軍宋益等迎擊斬之顯乃委罪歛成遣使謝傉檀慰撫河外引兵還傉檀遣使者徐宿詣秦謝罪【使疏吏翻】夏王勃勃聞秦兵且至退保河曲【河曲在朔方東北黄河千里一曲】齊難以勃勃既遠縱兵野掠勃勃潛師襲之俘斬七千餘人難引兵退走勃勃追至木城禽之虜其將士萬三千人於是嶺北夷夏附於勃勃者以萬數【夏戶雅翻】勃勃皆置守宰以撫之【姚弼之敗秃髮未能為秦患也齊難之敗則赫連之患熾矣】 司馬叔璠自蕃城寇鄒山【璠孚袁翻蕃音皮又音反讀曰翻】魯郡太守徐邕棄城走車騎長史劉鍾擊却之 北燕王雲封慕容歸為遼東公使主燕祀 劉敬宣既入峽【所謂三峽也】遣巴東太守温祚以二千人出外水自帥益州刺史鮑陋輔國將軍文處茂龍驤將軍時延祖由墊江轉戰而前【帥讀曰率處昌呂翻驤思將翻此由内水而進也墊音疊】譙縱求救于秦秦王興遣平西將軍姚賞南梁州刺史王敏將兵二萬赴之敬宣軍至黄虎去成都五百里【黄虎近涪城】縱輔國將軍譙道福悉衆拒嶮相持六十餘日敬宣不得進食盡軍中疾疫死者大半乃引軍還【還從宣翻又如字】敬宣坐免官削封三分之一荆州刺史劉道規以督統降號建威將軍【春秋責帥之義也道規時為征蜀都督】九月劉裕以敬宣失利請遜位詔降為中軍將軍開府如故劉毅欲以重法繩敬宣裕保護之何無忌謂毅曰奈何以私憾傷至公【私憾見上元年】毅乃止 乞伏熾磐以秦政浸衰且畏秦之攻襲【熾昌志翻】冬十月招結諸部二萬餘人築城于嵻㟍山而據之【下度曰嵻㟍山在西羌予據乞伏氏據苑川其地西至枹罕東極隴坻北限赫連南界吐谷渾嵻㟍山當在苑川西南宋朝西境盡秦渭嵻㟍山始在西羌中嵻丘岡翻㟍盧當翻】 十一月秃髪傉檀復稱涼王大赦改元嘉平置百官立夫人折掘氏為王后世子武臺為太子録尚書事【武臺本名虎臺唐人作晉書避唐祖諱改虎為武通鑑因之】左長史趙鼂右長史郭倖為尚書左右僕射【鼂古朝字音直遙翻】昌松侯俱延為太尉 南燕汝水竭【汝當作女】河凍皆合而澠水不氷【水經注澠水出營城東西北流入時水營城即臨淄城時水通有澠水之名亦謂之時澠水時水東北入淄水淄水又東北合濁水濁水東北流逕廣固城西濁水亦或通名之為澠水昔趙攻廣固望氣者以為澠水帶城非可攻拔若塞五龍口城必當陷指是水也澠神陵翻】南燕主超惡之問於李宣對曰澠水無氷良由逼帶京城近日月也【惡烏路翻近其靳翻】超大悦賜朝服一具【朝直遙翻】 十二月乞伏熾磐攻彭奚念於枹罕為奚念所敗而還【枹音膚敗補邁翻還從宣翻又如字】 是歲魏主珪殺高邑公莫題初拓跋窟咄之伐珪也【見一百六卷孝武大元十一年】題以珪年少【少詩沼翻】潛以箭遺窟咄曰三歲犢豈能勝重載邪【遺干季翻勝音升載才再翻】珪心銜之至是或告題居處倨傲擬則人主者珪使人以箭示題而謂之曰三歲犢果如何題父子對泣詰朝收斬之【處昌呂翻詰去吉翻】<br />
<br />
  資治通鑑卷一百十四<br />
<br />
<史部,編年類,資治通鑑>  <br>
   </div> 

<script src="/search/ajaxskft.js"> </script>
 <div class="clear"></div>
<br>
<br>
 <!-- a.d-->

 <!--
<div class="info_share">
</div> 
-->
 <!--info_share--></div>   <!-- end info_content-->
  </div> <!-- end l-->

<div class="r">   <!--r-->



<div class="sidebar"  style="margin-bottom:2px;">

 
<div class="sidebar_title">工具类大全</div>
<div class="sidebar_info">
<strong><a href="http://www.guoxuedashi.com/lsditu/" target="_blank">历史地图</a></strong>  
<a href="http://www.880114.com/" target="_blank">英语宝典</a>  
<a href="http://www.guoxuedashi.com/13jing/" target="_blank">十三经检索</a> 
<br><strong><a href="http://www.guoxuedashi.com/gjtsjc/" target="_blank">古今图书集成</a></strong> 
<a href="http://www.guoxuedashi.com/duilian/" target="_blank">对联大全</a> <strong><a href="http://www.guoxuedashi.com/xiangxingzi/" target="_blank">象形文字典</a></strong> 

<br><a href="http://www.guoxuedashi.com/zixing/yanbian/">字形演变</a>  <strong><a href="http://www.guoxuemi.com/hafo/" target="_blank">哈佛燕京中文善本特藏</a></strong>
<br><strong><a href="http://www.guoxuedashi.com/csfz/" target="_blank">丛书&方志检索器</a></strong> <a href="http://www.guoxuedashi.com/yqjyy/" target="_blank">一切经音义</a>  

<br><strong><a href="http://www.guoxuedashi.com/jiapu/" target="_blank">家谱族谱查询</a></strong>  <strong><a href="http://shufa.guoxuedashi.com/sfzitie/" target="_blank">书法字帖欣赏</a></strong> 
<br>

</div>
</div>


<div class="sidebar" style="margin-bottom:0px;">

<font style="font-size:22px;line-height:32px">QQ交流群9:489193090</font>


<div class="sidebar_title">手机APP 扫描或点击</div>
<div class="sidebar_info">
<table>
<tr>
	<td width=160><a href="http://m.guoxuedashi.com/app/" target="_blank"><img src="/img/gxds-sj.png" width="140"  border="0" alt="国学大师手机版"></a></td>
	<td>
<a href="http://www.guoxuedashi.com/download/" target="_blank">app软件下载专区</a><br>
<a href="http://www.guoxuedashi.com/download/gxds.php" target="_blank">《国学大师》下载</a><br>
<a href="http://www.guoxuedashi.com/download/kxzd.php" target="_blank">《汉字宝典》下载</a><br>
<a href="http://www.guoxuedashi.com/download/scqbd.php" target="_blank">《诗词曲宝典》下载</a><br>
<a href="http://www.guoxuedashi.com/SiKuQuanShu/skqs.php" target="_blank">《四库全书》下载</a><br>
</td>
</tr>
</table>

</div>
</div>


<div class="sidebar2">
<center>


</center>
</div>

<div class="sidebar"  style="margin-bottom:2px;">
<div class="sidebar_title">网站使用教程</div>
<div class="sidebar_info">
<a href="http://www.guoxuedashi.com/help/gjsearch.php" target="_blank">如何在国学大师网下载古籍?</a><br>
<a href="http://www.guoxuedashi.com/zidian/bujian/bjjc.php" target="_blank">如何使用部件查字法快速查字?</a><br>
<a href="http://www.guoxuedashi.com/search/sjc.php" target="_blank">如何在指定的书籍中全文检索?</a><br>
<a href="http://www.guoxuedashi.com/search/skjc.php" target="_blank">如何找到一句话在《四库全书》哪一页?</a><br>
</div>
</div>


<div class="sidebar">
<div class="sidebar_title">热门书籍</div>
<div class="sidebar_info">
<a href="/so.php?sokey=%E8%B5%84%E6%B2%BB%E9%80%9A%E9%89%B4&kt=1">资治通鉴</a> <a href="/24shi/"><strong>二十四史</strong></a>&nbsp; <a href="/a2694/">野史</a>&nbsp; <a href="/SiKuQuanShu/"><strong>四库全书</strong></a>&nbsp;<a href="http://www.guoxuedashi.com/SiKuQuanShu/fanti/">繁体</a>
<br><a href="/so.php?sokey=%E7%BA%A2%E6%A5%BC%E6%A2%A6&kt=1">红楼梦</a> <a href="/a/1858x/">三国演义</a> <a href="/a/1038k/">水浒传</a> <a href="/a/1046t/">西游记</a> <a href="/a/1914o/">封神演义</a>
<br>
<a href="http://www.guoxuedashi.com/so.php?sokeygx=%E4%B8%87%E6%9C%89%E6%96%87%E5%BA%93&submit=&kt=1">万有文库</a> <a href="/a/780t/">古文观止</a> <a href="/a/1024l/">文心雕龙</a> <a href="/a/1704n/">全唐诗</a> <a href="/a/1705h/">全宋词</a>
<br><a href="http://www.guoxuedashi.com/so.php?sokeygx=%E7%99%BE%E8%A1%B2%E6%9C%AC%E4%BA%8C%E5%8D%81%E5%9B%9B%E5%8F%B2&submit=&kt=1"><strong>百衲本二十四史</strong></a>  <a href="http://www.guoxuedashi.com/so.php?sokeygx=%E5%8F%A4%E4%BB%8A%E5%9B%BE%E4%B9%A6%E9%9B%86%E6%88%90&submit=&kt=1"><strong>古今图书集成</strong></a>
<br>

<a href="http://www.guoxuedashi.com/so.php?sokeygx=%E4%B8%9B%E4%B9%A6%E9%9B%86%E6%88%90&submit=&kt=1">丛书集成</a> 
<a href="http://www.guoxuedashi.com/so.php?sokeygx=%E5%9B%9B%E9%83%A8%E4%B8%9B%E5%88%8A&submit=&kt=1"><strong>四部丛刊</strong></a>  
<a href="http://www.guoxuedashi.com/so.php?sokeygx=%E8%AF%B4%E6%96%87%E8%A7%A3%E5%AD%97&submit=&kt=1">說文解字</a> <a href="http://www.guoxuedashi.com/so.php?sokeygx=%E5%85%A8%E4%B8%8A%E5%8F%A4&submit=&kt=1">三国六朝文</a>
<br><a href="http://www.guoxuedashi.com/so.php?sokeytm=%E6%97%A5%E6%9C%AC%E5%86%85%E9%98%81%E6%96%87%E5%BA%93&submit=&kt=1"><strong>日本内阁文库</strong></a> <a href="http://www.guoxuedashi.com/so.php?sokeytm=%E5%9B%BD%E5%9B%BE%E6%96%B9%E5%BF%97%E5%90%88%E9%9B%86&ka=100&submit=">国图方志合集</a> <a href="http://www.guoxuedashi.com/so.php?sokeytm=%E5%90%84%E5%9C%B0%E6%96%B9%E5%BF%97&submit=&kt=1"><strong>各地方志</strong></a>

</div>
</div>


<div class="sidebar2">
<center>

</center>
</div>
<div class="sidebar greenbar">
<div class="sidebar_title green">四库全书</div>
<div class="sidebar_info">

《四库全书》是中国古代最大的丛书,编撰于乾隆年间,由纪昀等360多位高官、学者编撰,3800多人抄写,费时十三年编成。丛书分经、史、子、集四部,故名四库。共有3500多种书,7.9万卷,3.6万册,约8亿字,基本上囊括了古代所有图书,故称“全书”。<a href="http://www.guoxuedashi.com/SiKuQuanShu/">详细>>
</a>

</div> 
</div>

</div>  <!--end r-->

</div>
<!-- 内容区END --> 

<!-- 页脚开始 -->
<div class="shh">

</div>

<div class="w1180" style="margin-top:8px;">
<center><script src="http://www.guoxuedashi.com/img/plus.php?id=3"></script></center>
</div>
<div class="w1180 foot">
<a href="/b/thanks.php">特别致谢</a> | <a href="javascript:window.external.AddFavorite(document.location.href,document.title);">收藏本站</a> | <a href="#">欢迎投稿</a> | <a href="http://www.guoxuedashi.com/forum/">意见建议</a> | <a href="http://www.guoxuemi.com/">国学迷</a> | <a href="http://www.shuowen.net/">说文网</a><script language="javascript" type="text/javascript" src="https://js.users.51.la/17753172.js"></script><br />
  Copyright &copy; 国学大师 古典图书集成 All Rights Reserved.<br>
  
  <span style="font-size:14px">免责声明:本站非营利性站点,以方便网友为主,仅供学习研究。<br>内容由热心网友提供和网上收集,不保留版权。若侵犯了您的权益,来信即刪。scp168@qq.com</span>
  <br />
ICP证:<a href="http://www.beian.miit.gov.cn/" target="_blank">鲁ICP备19060063号</a></div>
<!-- 页脚END --> 
<script src="http://www.guoxuedashi.com/img/plus.php?id=22"></script>
<script src="http://www.guoxuedashi.com/img/tongji.js"></script>

</body>
</html>
