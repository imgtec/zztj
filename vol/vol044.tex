資治通鑑卷四十四   宋 司馬光 撰

胡三省 音註

漢紀三十六|{
	起強圉協洽盡上章涒灘凡十四年}


世祖光武皇帝下

建武二十三年春正月南郡蠻叛|{
	郡國志南郡在雒陽南一千五百里蠻即緣沔諸山蠻也杜佑曰時南郡潳山蠻反劉尚討破之徙其種人七千餘口置江夏界中其後沔中蠻是也}
遣武威將軍劉尚討破之 夏五月丁卯大司徒蔡茂薨 秋八月丙戍大司空杜林薨 九月辛未以陳留玉況為大司徒|{
	賢曰玉音肅姓也}
冬十月丙申以太僕張純為大司空 武陵蠻精夫相單程等反|{
	秦昭王使白起伐楚畧取蠻夷始置黔中郡漢興改為武陵范書曰長沙武陵蠻名渠帥曰精夫槃瓠之後也}
遣劉尚發兵萬餘人泝沅水入武谿擊之|{
	賢曰沅水出牂柯故且蘭東北經辰州潭州岳州經洞庭湖入江武谿在今辰州盧谿縣西百八十里即五谿之一也沅音元}
尚輕敵深入蠻乘險邀之尚一軍悉沒 初匈奴單于輿弟右谷蠡王知牙師以次當為左賢王|{
	谷蠡音鹿黎}
左賢王次即當為單于單于欲傳其子遂殺知牙師烏珠留單于有子曰比為右薁鞬日逐王|{
	薁音郁鞬居言翻}
領南邊八部比見知牙師死出怨言曰以兄弟言之右谷蠡王次當立以子言之我前單于長子我當立|{
	呼韓邪單于約其諸子以兄弟次相傳單于輿殺其弟知牙師而立其子亂呼韓邪之約而比則烏珠留之長子也比自謂若父子相傳則烏珠留死比當立為單于何待至輿而始傳其子也師古曰谷音鹿蠡盧奚翻}
遂内懷猜懼庭會稀闊|{
	匈奴諸王歲正月會單于庭}
單于疑之乃遣兩骨都侯監領比所部兵|{
	監古銜翻}
及單于蒲奴立比益恨望密遣漢人郭衡奉匈奴地圖詣西河太守求内附|{
	郡國志西河郡在雒陽北千二百里守式又翻}
兩骨都侯頗覺其意會五月龍祠|{
	匈奴諸王每歲五月會龍城祠南匈奴傳曰匈奴俗歲有三龍祠常以正月五月九月戊日}
勸單于誅比比弟漸將王在單于帳下|{
	南匈奴傳大臣貴者左賢王次左谷蠡王次右賢王次右谷蠡王謂之四角次左右日逐王次左右温禺鞮王次左右斬將王是為六角漸當作斬傳寫誤加水旁耳}
聞之馳以報比比遂聚八部兵四五萬人待兩骨都侯還欲殺之骨都侯且到知其謀亡去單于遣萬騎擊之見比衆盛不敢進而還 是歲鬲侯朱祐卒|{
	范書朱祐傳二十四年卒}
祐為人質直尚儒學為將多受降|{
	將即亮翻降戶江翻}
以克定城邑為本不存首級之功又禁制士卒不得虜掠百姓軍人樂放縱|{
	樂音洛}
多以此怨之二十四年春正月乙亥赦天下 匈奴八部大人共議立日逐王比為呼韓邪單于欵五原塞願永為藩蔽扞禦北虜事下公卿|{
	下遐稼翻}
議者皆以為天下初定中國空虚夷狄情偽難知不可許五官中郎將耿國|{
	五官中郎將掌五官郎杜佑曰漢制三署郎年五十以上屬五官其次分屬左右署}
獨以為宜如孝宣故事受之|{
	事見二十七卷宣帝甘露黄龍間}
令東扞鮮卑北拒匈奴率厲四夷完復邊郡|{
	時邊郡皆創殘有南匈奴為扞蔽則可以完復矣}
帝從之 秋七月武陵蠻寇臨沅|{
	賢曰臨沅縣名屬武陵郡故城在今朗州武陵縣}
遣謁者李嵩中山太守馬成討之不克馬援請行帝愍其老未許援曰臣尚能被甲上馬|{
	被皮義翻}
帝令試之援據鞍顧眄以示可用帝笑曰矍鑠哉是翁|{
	賢曰矍鑠勇貌也}
遂遣援率中郎將馬武耿舒等將四萬餘人征五溪|{
	酈道元註水經云武陵有五溪謂雄溪樠溪酉溪潕溪辰溪悉是蠻夷所居故謂五溪皆槃瓠之子孫也土俗雄作熊樠作朗潕作武賢曰五溪在今辰州界}
援謂友人杜愔曰吾受厚恩年迫日索|{
	索盡也愔於今翻索昔各翻}
常恐不得死國事今獲所願甘心瞑目但畏長者家兒或在左右或與從事殊難得調介介獨惡是耳|{
	賢曰長者家兒謂權要子弟等介介猶耿耿也余謂調和也援固已慮耿舒之難與共事梁松竇固之邇言矣惡烏路翻}
冬十月匈奴日逐王比自立為南單于遣使詣闕奉藩稱臣上以問朗陵侯臧宫|{
	賢曰朗陵縣名屬汝南郡故城在今豫州朗山縣西南}
宫曰匈奴飢疫分爭臣願得五千騎以立功帝笑曰常勝之家難與慮敵吾方自思之

二十五年春正月遼東徼外貊人寇邊|{
	徼古弔翻貊莫百翻}
太守祭肜招降之|{
	降戶江翻}
肜又以財利撫約鮮卑大都護偏何使招致異種駱驛欵塞|{
	種章勇翻駱驛相繼也欵叩也至也}
肜曰審欲立功當歸撃匈奴斬送頭首乃信耳偏何等即撃匈奴斬首二千餘級持頭詣郡其後歲歲相攻輒送首級受賞賜自是匈奴衰弱邊無寇警鮮卑烏桓並入朝貢|{
	朝直遥翻}
肜為人質厚重毅撫夷狄以恩信故皆畏而愛之得其死力 南單于遣其弟左賢王莫|{
	莫者左賢王之名}
將兵萬餘人擊北單于弟薁鞬左賢王生獲之北單于震怖|{
	怖普布翻}
却地千餘里北部薁鞬骨都侯與右骨都侯率衆三萬餘人歸南單于三月南單于復遣使詣闕貢獻求使者監護|{
	復扶又翻監古衘翻}
遣侍子修舊約|{
	舊約宣帝舊約}
戊申晦日有食之 馬援軍至臨鄉|{
	水經註武陵郡沅南縣建武中所置縣在沅水之隂因以沅南為名縣治故城昔馬援討臨鄉所築也}
擊破蠻兵斬獲二千餘人初援嘗有疾虎賁中郎將梁松來候之|{
	虎賁中郎將掌虎賁郎賁音奔}
獨拜牀下援不答松去後諸子問曰梁伯孫帝婿|{
	梁松字伯孫尚帝女舞隂公主爾雅曰女子之夫為婿}
貴重朝廷公卿已下莫不憚之大人柰何獨不為禮援曰我乃松父友也雖貴何得失其序乎援兄子嚴敦並喜譏議|{
	賢曰喜許吏翻}
通輕俠援前在交阯還書誡之曰吾欲汝曹聞人過失如聞父母之名耳可得聞口不可得言也好論議人長短|{
	好呼到翻下同}
妄是非政灋|{
	賢曰謂譏刺時政也}
此吾所大惡也寧死不願聞子孫有此行也|{
	惡烏路翻行下孟翻下同}
龍伯高敦厚周慎口無擇言謙約節儉亷公有威吾愛之重之願汝曹效之杜季良豪俠好義憂人之憂樂人之樂|{
	樂音洛}
父喪致客數郡畢至吾愛之重之不願汝曹效也效伯高不得猶為謹敕之士所謂刻鵠不成尚類鶩者也|{
	賢曰鶩鴨也鶩莫卜翻毛晃曰舒鳬俗謂之鴨可畜而不能高飛者曰鴨野生而高飛者曰鶩}
效季良不得䧟為天下輕薄子所謂畫虎不成反類狗者也伯高者山都長龍述也|{
	龍姓述名賢曰山都縣名屬南陽郡舊南陽之赤鄉秦以為縣故城在今襄州義清縣東北長知兩翻}
季良者越騎司馬杜保也|{
	百官志越騎校尉其屬有司馬秩千石}
皆京兆人會保仇人上書訟保為行浮薄亂羣惑衆伏波將軍萬里還書以誡兄子而梁松竇固與之交結將扇其輕偽敗亂諸夏|{
	敗補邁翻}
書奏帝召責松固以訟書及援誡書示之松固叩頭流血而得不罪詔免保官擢拜龍述為零陵太守|{
	賢曰零陵今永州守式又翻}
松由是恨援及援討武陵蠻軍次下雋|{
	賢曰下雋縣名屬長沙國故城在今辰州沅陵縣宋白曰岳州巴陵縣漢地理志下雋縣属長沙郡在今卾州蒲沂縣界即此地按水經江水東至長沙下雋縣北澧水資水沅水合東流注之則宋說為是賢說非雋字兖翻}
有兩道可入從壺頭則路近而水嶮|{
	水經註夷水南出夷山北流注沅夷山東接壺頭山山下水際有馬援停軍處賢曰壺頭山在今辰州沅陵東}
從充則塗夷而運遠|{
	賢曰充縣名屬武陵郡充昌容翻}
耿舒欲從充道援以為棄日費糧不如進壼頭搤其喉咽|{
	搤持也咽音烟喉嚨也}
充賊自破以事上之|{
	上時掌翻下同}
帝從援策進營壼頭賊乘高守隘水疾船不得上會暑甚士卒多疫死援亦中病乃穿岸為室以避炎氣|{
	武陵記曰壼頭山邊有石窟即援所穿室也中竹仲翻}
賊每升險鼔譟援輒曳足以觀之左右哀其壯意莫不為之流涕|{
	為于偽翻}
耿舒與兄好畤侯弇書曰|{
	好畤縣屬扶風畤音止}
前舒上書當先擊充糧雖難運而兵馬得用軍人數萬爭欲先奮今壼頭竟不得進大衆怫鬱行死|{
	師古曰怫欝憂不樂也怫符弗翻怫欝氣藴積而不得舒也行死謂行將疫死也}
誠可痛惜前到臨鄉賊無故自致若夜擊之即可殄滅伏波類西域賈胡到一處輒止|{
	賢曰言似商胡所至之處輒停留也賈音古}
以是失利今果疾疫皆如舒言弇得書奏之帝乃使梁松乘驛責問援因代監軍|{
	監古銜翻}
會援卒松因是構䧟援帝大怒追收援新息侯印綬|{
	郡國志新息侯國屬汝南郡應劭曰古息國其後東徙加新字}
初援在交阯常餌薏苡|{
	神農本草經曰薏苡味甘微寒主風濕痺下氣除筋骨邪氣久服輕身益氣}
實能輕身勝障氣|{
	障與瘴同}
軍還載之一車及卒後有上書譖之者以為前所載還皆明珠文犀|{
	文犀犀之有文彩者}
帝益怒援妻孥惶懼|{
	孥音奴子也}
不敢以喪還舊塋槀葬域西|{
	賢曰槀草也以不歸舊塋時權葬故稱槀馬援傳作城西說文曰塋墓地廣雅曰塋域葬地也}
賓客故人莫敢弔會|{
	不敢弔及會葬}
嚴與援妻子草索相連詣闕請罪|{
	索昔各翻}
帝乃出松書以示之方知所坐上書訴寃前後六上辭甚哀切|{
	上時掌翻下同}
前雲陽令扶風朱勃|{
	雲陽縣屬左馮翊有秦雲陽宫鈎弋夫人葬雲陽昭帝為起雲陵邑後為縣}
詣闕上書曰竊見故伏波將軍馬援拔自西州欽慕聖義間關險難|{
	難乃旦翻}
觸冒萬死經營隴冀|{
	謂征隗囂時也}
謀如涌泉埶如轉規|{
	規圓也}
兵動有功師進輒克誅鋤先零飛矢貫脛|{
	零音憐建武十一年援擊破先零飛矢貫脛脛形定翻}
出征交阯與妻子生訣|{
	征交阯事見上卷十七年十八年十九年}
間復南討|{
	復扶又翻}
立䧟臨鄉師已有業|{
	業緒也}
未竟而死吏士雖疫援不獨存夫戰或以久而立功或以速而致敗深入未必為得不進未必為非人情豈樂久屯絶地不生歸哉|{
	樂音洛}
惟援得事朝廷二十二年北出塞漠|{
	謂討烏桓}
南度江海觸冒害氣僵死軍事名滅爵絶國土不傳海内不知其過衆庶未聞其毁家屬杜門葬不歸墓怨隙並興宗親怖慄|{
	怖普布翻}
死者不能自列生者莫為之訟|{
	為于偽翻}
臣竊傷之夫明主醲於用賞約於用刑高祖嘗與陳平金四萬斤以間楚軍不問出入所為|{
	事見十卷高帝三年間古莧翻}
豈復疑以錢穀間哉|{
	復扶又翻}
願下公卿平援功罪宜絶宜續以厭海内之望|{
	下遐稼翻厭一葉翻}
帝意稍解初勃年十二能誦詩書常候援兄況辭言雅|{
	賢曰音閑雅猶言沈静也余謂習也屈原傳於辭令}
援裁知書見之自失況知其意乃自酌酒慰援曰朱勃小器速成智盡此耳卒當從汝禀學|{
	卒子恤翻終也賢曰禀受也}
勿畏也勃未二十右扶風請試守渭城宰|{
	前書音義曰試守者試守一歲乃為真食其全俸賢曰渭城縣名故城在今咸陽縣東北}
及援為將軍封侯而勃位不過縣令援後雖貴常待以舊恩而卑侮之勃愈身自親及援遇讒唯勃能終焉謁者南陽宗均監援軍|{
	宗均列傳作宋均趙明誠金石錄有漢司空宗俱碑按後漢宋均傳均族子意意孫俱靈帝時為司空余嘗得宗資墓前碑龜膊上刻字因以後漢帝紀及姓苑姓纂諸書參考以謂自均以下其姓皆作宗而列傳轉寫為宋誤也後得此碑益知前言之不繆}
援旣卒軍士疫死者太半蠻亦飢困均乃與諸將議曰今道遠士病不可以戰欲權承制降之何如諸將皆伏地莫敢應|{
	降戶江翻}
均曰夫忠臣出竟有可以安國家專之可也|{
	公羊傳曰聘禮大夫受命不受辭出境有可以安社稷全國家者則專之可也竟讀曰境}
乃矯制調伏波司馬呂种守沅陵長|{
	調徒弔翻}
命种奉詔書入虜營告以恩信因勒兵隨其後蠻夷震怖冬十月共斬其大帥而降|{
	帥所類翻}
於是均入賊營散其衆遣歸本郡為置長吏而還|{
	為于偽翻還從宣翻又如字}
羣蠻遂平均未至先自劾矯制之罪|{
	劾戶槩翻又戶得翻}
上嘉其功迎賜以金帛令過家上冢|{
	受命而出未復命則不當先過家今使過家上冢所以示寵榮也上時掌翻}
是歲遼西烏桓大人郝旦等率衆内屬|{
	考異曰帝紀今春旣著烏桓來朝歲末又紀是歲烏桓朝貢内屬盖始獨大人來朝後乃率種族内屬耳}
詔封烏桓渠帥為侯王君長者八十一人|{
	帥所類翻長知兩翻}
使居塞内布於緣邊諸郡令招來種人|{
	種章勇翻}
給其衣食遂為漢偵候|{
	偵丑鄭翻}
助撃匈奴鮮卑時司徒掾班彪上言烏桓天性輕黠好為寇賊若久放縱而無總領者必復掠居人|{
	掾俞絹翻黠下八翻好呼到翻復扶又翻}
但委主降掾吏|{
	賢曰盖當時權置也降戶江翻}
恐非所能制臣愚以為宜復置烏桓校尉|{
	西都置護烏桓校尉至王莽時烏桓叛校尉由是罷闞駰十三州志曰護烏桓擁節秩比二千石武帝置以護内附烏桓旣而并於匈奴中郎將余據匈奴中郎將亦此時方置未知并於匈奴中郎將果何時也校戶教翻}
誠有益於附集省國家之邊慮帝從之於是始復置校尉於上谷甯城|{
	賢曰甯城縣名前書甯作寧寧甯兩字通也杜佑曰甯城在媯川郡懷戎縣西北俗名西吐㪍城}
開營府并領鮮卑賞賜質子歲時互市焉|{
	質音致}


二十六年正月詔增百官奉|{
	百官志大將軍三公奉月三百五十斛秩中二千石俸月百八十斛二千石月百二十斛比二千石月百斛千石月九十斛比千石月八十斛六百石月七十斛比六百石月五十五斛四百石月五十斛比四百石月四十五斛三百石月四十斛比三百石月三十七斛二百石月三十斛比二百石月二十七斛百石月十六斛斗食月十一斛佐史月八斛凡諸受俸錢穀各半奉音扶用翻}
其千石已上減於西京舊制六百石已下增於舊秩初作夀陵|{
	賢曰初作陵未有名故號夀陵盖取久長之義也}
帝曰古者帝王

之葬皆陶人瓦器木車茅馬使後世之人不知其處太宗識終始之義景帝能述遵孝道遭天下反覆而霸陵獨完受其福豈不美哉|{
	謂赤眉入長安惟霸陵不掘}
今所制地不過二三頃無山陵陂池裁令流水而已|{
	賢曰言不起山陵裁令封土陂池不停水而已陂音普何翻池音徒河翻}
使迭興之後與丘隴同體|{
	迭興謂易姓而王者}
詔遣中郎將段彬|{
	彬丑林翻}
副校尉王郁使南匈奴立其

庭去五原西部塞八十里|{
	地理志五原西部都尉治田辟師古曰辟讀曰壁}
使者令單于伏拜受詔單于顧望有頃乃伏稱臣拜訖令譯曉使者曰單于新立誠慙於左右願使者衆中無相屈折也詔聽南單于入居雲中|{
	賢曰雲中郡名在今勝州北宋白曰漢雲中故城在勝州東北四十里榆林縣界趙武侯所築}
始置使匈奴中郎將將兵衛護之 夏南單于所獲北虜薁鞬左賢王將其衆及南部五骨都侯|{
	韓氏骨都侯當干骨都侯呼衍骨都侯郎氏骨都侯粟藉骨都侯凡五薁音郁鞬居言翻}
合三萬餘人畔歸去北庭三百餘里自立為單于月餘日更相攻撃|{
	更工衡翻}
五骨都侯皆死左賢王自殺諸骨都侯子各擁兵自守 秋南單于遣子入侍詔賜單于冠帶璽綬|{
	南匈奴傳黄金璽盭緺綬賢曰盭音戾草名以戾草染綬因以為名别漢諸侯王制戾綠色緺紫青色音瓜璽斯氏翻綬音受}
車馬金帛甲兵什器|{
	賢曰古之師行二五為什食器之類必供之故曰什物食具今人通謂生生之具為什物}
又轉河東米糒二萬五千斛牛羊三萬六千頭以贍給之|{
	糒音備糗也}
令中郎將將弛刑五十人隨單于所處參辭訟察動静|{
	弛刑者弛刑徒也說文弓解曰弛此謂解其罪而輸作者處昌呂翻 考異曰帝紀今年春使段彬賜璽綬置使匈奴中郎將據匈奴傳賜璽綬在秋其置中郎將亦未知決在何時或者今春置之至是更為之約束制度耳}
單于歲盡輒遣奉奏送侍子入朝漢遣謁者送前侍子還單于庭賜單于及閼氏左右賢王以下繒綵合萬匹歲以為常|{
	閼音煙氏音支}
於是雲中五原朔方北地定襄鴈門上谷代八郡民歸於本土|{
	前此避匈奴内徙者令皆歸復本土}
遣謁者分將弛刑補治城郭|{
	將即亮翻下同治直之翻}
發遣邊民在中國者布還諸縣皆賜以裝錢轉給糧食時城郭丘墟掃地更為上乃悔前徙之|{
	徙民見上卷十五年}
冬南匈奴五骨都侯子復將其衆三千人歸南部北單于使騎追撃悉獲其衆南單于遣兵拒之逆戰不利於是復詔單于徙居西河美稷|{
	復扶又翻}
因使段彬王郁留西河擁護之|{
	使匈奴中郎將自是亦屯西河美稷杜佑曰汾州隰城縣有美稷郷即漢美稷縣也隰城漢之兹氏縣也}
令西河長史歲將騎二千弛刑五百人助中郎將衛護單于冬屯夏罷自後以為常南單于旣居西河亦列置諸部王助漢扞戍北地朔方五原雲中定襄鴈門代郡皆領部衆為郡縣偵邏耳目|{
	偵丑鄭翻賢曰邏音力賀翻}
北單于惶恐頗還所掠漢民以示善意鈔兵每到南部下|{
	鈔楚交翻}
還過亭候輒謝曰自擊亡虜薁鞬日逐耳|{
	薁於六翻鞬居言翻}
非敢犯漢民也

二十七年夏四月戊午大司徒王況薨 五月丁丑詔司徒司空並去大名|{
	去羌呂翻}
改大司馬為太尉驃騎大將軍行大司馬劉隆即日罷以太僕趙熹為太尉大司農馮勤為司徒 北匈奴遣使詣武威求和親|{
	自北地以東南部分居塞内北使不敢至塞下故詣武威求和賢曰武威郡故城在今涼州姑臧縣西北故涼城是也}
帝召公卿廷議不決皇太子言曰南單于新附北虜懼於見伐故傾耳而聽爭欲歸義耳今未能出兵而反交通北虜臣恐南單于將有二心北虜降者且不復來矣|{
	復扶又翻下同}
帝然之告武威太守勿受其使 朗陵侯臧宫揚虚侯馬武上書曰|{
	朗陵侯國屬汝南郡水經註揚虚縣屬平原漯水逕其東商河發源於此}
匈奴貪利無有禮信窮則稽首安則侵盜|{
	稽音啓}
虜今人畜疫死旱蝗赤地疲困乏力不當中國一郡萬里死命縣在陛下|{
	縣讀曰懸下同}
福不再來時或易失豈宜固守文德而墮武事乎|{
	左傳曰大福不再蒯通曰時難得而易失易以豉翻墮讀曰隳}
今命將臨塞厚縣購賞|{
	將即亮翻縣讀曰懸}
喻告高句驪烏桓鮮卑攻其左|{
	句如字又音駒驪力知翻}
發河西四郡天水隴西羌胡擊其右如此北虜之滅不過數年臣恐陛下仁恩不忍謀臣狐疑令萬世刻石之功不立於聖世詔報曰黄石公記曰柔能制剛弱能制彊|{
	賢曰黄石公即張良於下邳圯上所見老父出一編書者}
舍近謀遠者勞而無功舍遠謀近者逸而有終|{
	舍讀曰捨}
故曰務廣地者荒務廣德者彊有其有者安貪人有者殘殘滅之政雖成必敗今國無善政災變不息百姓驚惶人不自保而復欲遠事邊外乎孔子曰吾恐季孫之憂不在顓臾|{
	見論語}
且北狄尚彊而屯田警備傳聞之事恒多失實|{
	恒戶登翻}
誠能舉天下之半以滅大寇豈非至願苟非其時不如息民自是諸將莫敢復言兵事者 上問趙憙以久長之計憙請遣諸王就國冬上始遣魯王興齊王石就國|{
	興縯之次子石章之子縯之嫡孫也}
是歲帝舅夀張恭侯樊宏薨|{
	夀張縣屬東平國春秋曰良漢曰夀良帝避叔父趙王良諱改曰夀張宏帝舅也謚敬侯曰恭侯温公避國諱也 考異曰袁紀宏皆作密今從范書}
宏為人謙柔畏慎每當朝會輒迎期先到俯伏待事所上便宜|{
	朝直遥翻下同上時掌翻}
手自書寫毁削草本公朝訪逮|{
	逮及也}
不敢衆對宗族染其化未嘗犯灋帝甚重之及病困遺令薄葬一無所用以為棺柩一藏不宜復見|{
	復扶又翻}
如有腐敗傷孝子之心使與夫人同墳異藏|{
	古夫婦合葬詩曰穀則異室死則同穴是也同墳異藏自宏始}
帝善其令以書示百官因曰今不順夀張侯意無以彰其德且吾萬歲之後欲以為式二十八年春正月己巳徙魯王興為北海王以魯益東海帝以東海王彊去就有禮|{
	謂以天下讓}
故優以大封食二十九縣賜虎賁旄頭設鍾虡之樂|{
	漢官儀曰虎賁千五百人戴鶡尾屬虎賁中郎將旄頭注見前爾雅木謂之虡所以懸鐘磬也說文曰虡飾為猛獸虡音巨}
擬於乘輿|{
	東䋲證翻}
夏六月丁卯沛太后郭氏薨 初馬援兄子壻王磐平阿侯仁之子也王莽敗磐擁富貲為游俠|{
	俠戶頰翻}
有名江淮間後游京師與諸貴戚友善援謂姊子曹訓曰王氏廢姓也子石當屏居自守|{
	磐字子石屏必郢翻}
而反游京師長者|{
	賢曰長者謂豪俠者也余謂長者正指諸貴戚耳前所謂長者家兒可以槩推}
用氣自行多所陵折其敗必也後歲餘磐坐事死磐子肅復出入王侯邸第|{
	復扶又翻}
時禁罔尚疏諸王皆在京師競脩名譽招游士馬援謂司馬呂种曰建武之元名為天下重開|{
	种持中翻重直龍翻}
自今以往海内日當安耳但憂國家諸子並壯而舊防未立若多通賓客則大獄起矣|{
	賢曰舊防諸侯王子不許交通賓客}
卿曹戒慎之至是有上書告肅等受誅之家為諸王賓客慮因事生亂會更始之子夀光侯鯉得幸於沛王|{
	賢曰夀光縣屬北海郡今青州縣}
怨劉盆子結客殺故式侯恭帝怒沛王坐繋詔獄三日乃得出因詔郡縣收捕諸王賓客更相牽引|{
	更工衡翻}
死者以千數呂种亦與其禍|{
	與讀曰豫}
臨命嘆曰馬將軍誠神人也 秋八月戊寅東海王彊沛王輔楚王英濟南王康淮陽王延始就國|{
	濟子禮翻}
上大會羣臣問誰可傅太子者羣臣承望上意皆言

太子舅執金吾原鹿侯隂識可|{
	原鹿縣屬汝南郡春秋之鹿上也可言可任也}
博士張佚正色曰今陛下立太子為隂氏乎為天下乎即為隂氏則隂侯可為天下則固宜用天下之賢才|{
	為于偽翻}
帝稱善曰欲置傅者以輔太子也今博士不難正朕況太子乎即拜佚為太子太傅以博士桓榮為少傅賜以輜車乘馬|{
	乘䋲證翻}
榮大會諸生陳其車馬印綬曰今日所蒙稽古之力也可不勉哉 北匈奴遣使貢馬及裘更乞和親并請音樂又求率西域諸國胡客俱獻見帝下三府議酬荅之宜|{
	三府太尉司徒司空府也見賢遍翻下遐稼翻}
司徒掾班彪曰臣聞孝宣皇帝敕邊守尉曰匈奴大國多變詐交接得其情則却敵折衝應對入其數則反為輕欺|{
	數術數也言入其術中也}
今北單于見南單于來附懼謀其國故數乞和親|{
	數所角翻下同}
又遠驅牛馬與漢合市|{
	合市與漢和合為市也}
重遣名王多所貢獻斯皆外示富彊以相欺誕也臣見其獻益重知其國益虚歸親愈數為懼愈多然今旣未獲助南則亦不宜絶北覊縻之義禮無不答謂可頗加賞賜略與所獻相當報答之辭令必有適|{
	賢曰適猶所也言報答之辭必令得所也余謂適當也言報答之辭必有當乎事情也}
今立稾草并上曰單于不忘漢恩追念先祖舊約|{
	謂呼韓邪舊約也上時掌翻}
欲修和親以輔身安國計議甚高為單于嘉之|{
	為于偽翻}
往者匈奴數有乖亂呼韓邪郅支自相讎隙並蒙孝宣帝垂恩救護故各遣侍子稱藩保塞其後郅支忿戾自絶皇澤而呼韓附親忠孝彌著及漢滅郅支遂保國傳嗣子孫相繼|{
	事並見前紀}
今南單于攜衆向南欵塞歸命自以呼韓嫡長次第當立而侵奪失職猜疑相背數請兵將歸埽北庭|{
	長知兩翻背蒲妹翻將即亮翻}
策謀紛紜無所不至惟念斯言不可獨聽|{
	惟思也}
又以北單于比年貢獻|{
	比毗至翻}
欲修和親故拒而未許將以成單于忠孝之義漢秉威信總率萬國日月所照皆為臣妾殊俗百蠻義無親踈服順者褒賞畔逆者誅罰善惡之效呼韓郅支是也今單于欲修和親欵誠已達何嫌而欲率西域諸國俱來獻見西域國屬匈奴與屬漢何異單于數連兵亂國内虚耗貢物裁以通禮何必獻馬裘今齎雜繒五百匹弓鞬韥丸一|{
	賢曰鞬音居言翻方言曰藏弓為鞬藏箭為韥丸即箭箙也韥與韣同徒谷翻}
矢四發遺單于|{
	遺于季翻}
又賜獻馬左骨都侯右谷蠡王|{
	谷音鹿蠡音黎}
雜繒各四百匹斬馬劍各一單于前言先帝時所賜呼韓邪竽瑟空侯皆敗|{
	竽管三十六簧劉昫曰女媧氏造匏列管於匏上内簧其中爾雅謂之巢大者曰竽小者曰和竽煦也立春之氣煦生萬物也竽管三十六宫管在左和管十三宫管居中今之竽笙並以木代匏而漆之無復八音矣瑟註見前空侯世本云空國侯所造劉昫曰漢武帝使樂人侯調所作以祠太廟或曰侯暉所作其聲坎坎應節謂之坎侯聲訛為箜或謂師賢靡靡樂非也舊說一依琴制今案其形似瑟而小七絃用撥彈之如琵琶}
願復裁賜|{
	賢曰言更請裁賜也余謂裁量也量多少以賜也復扶又翻}
念單于國尚未安方厲武節以戰攻為務竽瑟之用不如良弓利劍故未以齎朕不愛小物於單于便宜所欲遣驛以聞帝悉納從之二十九年春二月丁巳朔日有食之

三十年春二月車駕東廵羣臣上言即位三十年宜封禪泰山詔曰即位三十年百姓怨氣滿腹吾誰欺欺天乎曾謂泰山不如林放乎|{
	論語記孔子之言}
何事汚七十二代之編錄|{
	賢曰莊子曰易姓而王封於泰山禪於梁父者七十有二代其有形兆垠堮勒石凡千八百餘處許慎說文序曰蒼頡之初作書盖依類象形故謂之文其有形聲相益即謂之字字者言孶乳而滋多也著於竹帛謂之書書者如也以迄五帝三王之世改易殊體封於泰山者七十有二代靡有同焉汙烏故翻}
若郡縣遠遣吏上夀盛稱虛美必髠令屯田於是羣臣不敢復言|{
	復扶又翻}
甲子上幸魯濟南|{
	濟子禮翻}
閏月癸丑還宫 有星孛于紫宫|{
	孛蒲内翻}
夏四月戊子徙左翊王焉為中山王 五月大水 秋七月丁酉上行幸魯冬十一月丁酉還宫 膠東剛侯賈復薨|{
	謚法能補前過曰剛此直以賈復剛毅而謚之耳考異曰本傳在三十一年今從袁紀}
復從征伐未嘗喪敗數與諸將潰圍解急身被十二創|{
	喪息浪翻數所角翻被皮義翻創初良翻}
帝以復敢深入希令遠征而壯其勇節常自從之|{
	常以復自從也}
故復少方面之勲|{
	少詩沼翻}
諸將每論功伐復未嘗有言帝輒曰賈君之功我自知之

三十一年夏五月大水 癸酉晦日有食之 蝗 京兆掾第五倫|{
	倫之先齊諸田徙長陵諸田徙園陵者多故以次第為氏掾俞絹翻}
領長安市公平亷介市無姦枉每讀詔書常嘆息曰此聖主也一見決矣等輩笑之曰爾說將尚不能下|{
	賢曰將謂州將說輸芮翻將即亮翻}
安能動萬乘乎|{
	乘䋲證翻}
倫曰未遇知已道不同故耳後舉孝亷補淮陽王醫工長|{
	百官志王國官有禮樂長主樂人衛士長主衛士醫工長主醫藥永巷長主宫中婢使祠祀長主祠祀皆比四百石長知兩翻}


中元元年|{
	洪氏隸釋曰城都有漢蜀郡太守何君造尊楗閣碑其末云建武中元二年六月按范史本紀建武止三十一年次年改為中元直書為中元元年觀此所刻乃是雖别為中元猶冠以建武如文景中元後元之類也又祭祀志載封禪後赦天下詔明言云改建武三十二年為建武中元元年東夷倭國傳建武中元二年來奉貢證据甚明宋莒公紀元通譜云紀志俱出范史必傳寫脱誤學者失於精審以意剛去梁武帝大同大通俱有中字是亦憲章於此司馬公作通鑑不取其說余按考異温公非不取宋說也從袁范書中元者從簡易耳}
春正月淮陽王入朝第五倫隨官屬得會見|{
	見賢遍翻}
帝問以政事倫因此酬對帝大悦明日復特召入與語至夕|{
	復扶又翻}
帝謂倫曰聞卿為吏篣婦公|{
	篣音彭}
不過從兄飯寧有之邪|{
	過工禾翻從才用翻飯扶晚翻}
對曰臣三娶妻皆無父少遭飢亂|{
	少詩照翻}
實不敢妄過人食衆人以臣愚蔽故生是語耳帝大笑以倫為扶夷長|{
	賢曰扶夷縣屬零陵郡故城在今邵州武岡縣東北水經志夫夷縣在卲陵西}
未到官追拜會稽太守|{
	會古外翻守式又翻}
為政清而有惠百姓愛之 上讀河圖會昌符曰赤劉之九會命岱宗|{
	風俗通曰岱始也泰山山之尊者一曰岱宗岱始也宗長也萬物之始隂陽交代故為五岳之長}
上感此文乃詔虎賁中郎將梁松等按察河雒讖文言九世當封禪者凡三十六事|{
	讖楚譛翻}
於是張純等復奏請封禪|{
	復扶又翻史記集註曰泰山上築土為壇以祭天報天之功故曰封泰山下小山上除地報地之功故曰禪}
上乃許焉詔有司求元封故事當用方石再累玉檢金泥|{
	元封故事武帝封禪故事也用方石再累置壇中皆方五尺厚一尺用玉牒書藏方石牒厚五寸長尺三寸廣五寸有玉檢又有石檢十枚列於石旁東西各三南北各二皆長五尺廣三尺厚七寸檢中刻三處深四寸方五寸有盖檢用金縷五周以水銀和金以為泥}
上以石功難就欲因孝武故封石置玉牒其中梁松等爭以為不可乃命石工取完青石無必五色|{
	舊制用石盖各依方色也}
丁卯車駕東廵二月己卯幸魯進幸泰山辛卯晨燎祭天於泰山下南方羣神皆從|{
	從從祀也從才用翻}
用樂如南郊事畢至食時天子御輦登山日中後到山上|{
	郭璞註山海經曰泰山從山下至頭四十八里二百步}
更衣|{
	易服乃即事也更工衡翻}
晡時升壇北面尚書令奉玉牒檢天子以寸二分璽親封之|{
	璽斯氏翻}
訖太常命騶騎二千餘人|{
	騶側尤翻}
發壇上方石尚書令藏玉牒已復石覆訖|{
	覆敷救翻}
尚書令以五寸印封石檢事畢天子再拜羣臣稱萬歲乃復道下|{
	謂復故道而下山也}
夜半後上乃到山下百官明旦乃訖甲午禪祭地於梁隂|{
	梁父之隂也禪時戰翻}
以高后配山川羣神從|{
	從從祀也從才用翻}
如元始中北郊故事 三月戊辰司空張純薨 夏四月癸酉車駕還宫己卯赦天下改元 |{
	考異曰續漢志云以建武三十二年為建武中元元年紀年通譜云據紀志俱出范氏而所載不同此必傳寫脱誤今官書累經校定學者失於精審但見紀元復有建武二字輒以意刪去斯為繆矣梁武帝大同大通之號俱有中字是亦憲章於此今從袁紀范書}
上行幸長安五月乙丑還宫 六月辛卯以太僕馮

魴為司空|{
	魴符方翻}
乙未司徒馮勤薨 京師醴泉湧出|{
	爾雅甘雨時降萬物以嘉謂之醴泉}
又有赤草生於水崖|{
	䝨曰曰赤草朱草也也大戴禮曰朱草日生一葉至十五日以後日落一葉週而復始}
郡國頻上甘露|{
	上時掌翻下同}
羣臣奏言靈物仍降宜令太史撰集以傳來世|{
	賢曰太史史官之長也撰雛免翻}
帝不納帝自謙無德每郡國所上輒抑而不當故史官罕得記焉 秋郡國三蝗 冬十月辛未以司隸校尉東萊李訢為司徒|{
	郡國志東萊郡在雒陽東三千一百二十八里訢許斤翻}
甲申使司空告祠高廟上薄太后尊號曰高皇后配食地祇|{
	上時掌翻}
遷呂太后廟主于園|{
	以呂太后幾危劉氏也賢曰園謂塋域也於中置寢}
四時上祭|{
	上時掌翻}
十一月甲子晦日有食之 是歲起明堂靈臺辟雍|{
	賢曰漢官儀明堂去平城門二里所天子出從平城門先歷明堂乃至郊祀又曰辟雍去明堂三百步車駕臨辟雍從北門入三月九月皆於中行鄉射禮辟雍以水周其外以節觀者漢宫闕疏曰靈臺高三丈十二門楊衒之雒陽記曰平昌門直南大道東是明堂大道西是靈臺}
宣布圖䜟於天下初上以赤伏符即帝位|{
	見四十卷建武元年}
由是信用䜟文多以決定嫌疑給事中桓譚上疏諫曰凡人情忽於見事而貴於異聞|{
	見賢遍翻}
觀先王之所記述咸以仁義正道為本非有奇怪虛誕之事盖天道性命聖人所難言也自子貢以下不得而聞|{
	論語子貢曰夫子之言性與天道不可得而聞也}
況後世淺儒能通之乎今諸巧慧小才伎數之人增益圖書矯稱䜟記|{
	伎謂方伎醫方之家也數謂數術明堂羲和史卜之官也圖書即䜟緯符命之類是也伎渠綺翻}
以欺惑貪邪詿誤人主焉可不抑遠之哉|{
	詿古賣翻又戶卦翻焉於䖍翻遠于願翻}
臣譚伏聞陛下窮折方士黄白之術甚為明矣|{
	黄白謂以藥化成金銀也方士有方術之士也}
而乃欲聽納䜟記又何誤也其事雖有時合譬猶卜數隻偶之類|{
	賢曰言偶中也}
陛下宜垂明聽發聖意屛羣小之曲說|{
	屏必郢翻又卑正翻}
述五經之正義疏奏帝不悦會議靈臺所處|{
	處昌呂翻}
帝謂譚曰吾欲以䜟決之何如譚默然良久曰臣不讀䜟帝問其故譚復極言䜟之非經|{
	復扶又翻}
帝大怒曰桓譚非聖無法將下斬之|{
	將資良翻持也領也}
譚叩頭流血良久乃得解出為六安郡丞|{
	賢曰六安郡故城在今夀州安豐縣南余據郡國志建武十六年省六安國以其縣屬廬江郡譚出為郡丞不必在是年通鑑因靈臺事併書於此}
道病卒

范曄論曰桓譚以不善䜟流亡鄭興以遜辭僅免賈逵能傅會文致最差貴顯|{
	鄭興事見四十二卷七年明帝永平中賈逵上言左氏與圖䜟合明劉氏為堯後帝嘉之歷遷侍中領騎都尉甚見信用傅讀曰附}
世主以此論學悲哉逵扶風人也

南單于比死弟左賢王莫立為丘浮尤鞮單于|{
	鞮丁奚翻}
帝遣使齎璽書拜授璽綬賜以衣冠及繒綵|{
	繒慈陵翻}
是後遂以為常

二年春正月辛未初立北郊祀后土 二月戊戍帝崩於南宮前殿年六十二帝每旦視朝日昃乃罷|{
	日過中則昃朝直遥翻}
數引公卿郎將|{
	數所角翻}
講論經理夜分乃寐|{
	賢曰分猶半也}
皇太子見帝勤勞不怠承間諫曰|{
	間古莧翻}
陛下有禹湯之明而失黄老養性之福願頤愛精神優游自寧帝曰我自樂此不為疲也|{
	樂音洛}
雖以征伐濟大業及天下旣定乃退功臣而進文吏明慎政體總攬權綱量時度力|{
	量音良度徒洛翻}
舉無過事故能恢復前烈身致太平太尉趙憙典喪事時經王莽之亂舊典不存皇太子與諸王雜止同席藩國官屬出入宫省|{
	宫省即宫禁也}
與百僚無别|{
	别彼列翻}
憙正色横劒殿階扶下諸王以明尊卑奏遣謁者將護官屬分止他縣諸王並令就邸|{
	諸王國各置邸洛陽}
唯得朝晡入臨|{
	臨臨哭也力鴆翻下同}
整禮儀嚴門衛|{
	賈公彦曰漢宫殿門每門皆使司馬二人守門比千石皆號司馬殿門}
内外肅然 太子即皇帝位尊皇后曰皇太后 山陽王荆哭臨不哀而作飛書令蒼頭詐稱大鴻臚郭況書與東海王彊言其無罪被廢|{
	被皮義翻}
及郭后黜辱勸令東歸舉兵以取天下且曰高祖起亭長陛下興白水|{
	謂光武起於南陽舂陵之白水卿也長知兩翻}
何況於王陛下長子故副主哉|{
	故副主謂舊為太子也長知兩翻}
當為秋霜毋為檻羊|{
	賢曰秋霜肅殺於物檻羊受制於人}
人主崩亡閭閻之伍尚為盜賊欲有所望何況王邪彊得書惶怖|{
	怖普故翻}
即執其使|{
	使疏吏翻}
封書上之|{
	上時掌翻}
明帝以荆母弟|{
	帝及荆皆隂后所生}
祕其事遣荆出止河南宮|{
	宫在河南縣}
三月丁卯葬光武皇帝於原陵|{
	帝王紀曰原陵在臨平亭東南去雒陽十五里水經註光武葬臨平亭南西望平隂大河逕其北}
夏四月丙辰詔曰方今上無天子下無方伯若涉淵水而無舟楫夫萬乘至重而壯者慮輕實賴有德左右小子|{
	帝謙言年尚少壯思慮輕淺故須賢人輔弼賴恃也左右助也左右音佐佑}
高密侯禹元功之首東平王蒼寛博有謀其以禹為太傅蒼為驃騎將軍蒼懇辭帝不許又詔驃騎將軍置長史掾史員四十人位在三公上|{
	賢曰四府掾史皆無四十人今特置以優之也驃匹妙翻掾俞絹翻}
蒼嘗薦西曹掾齊國吳良|{
	百官志西曹主府史署用掾秩比四百石}
帝曰薦賢助國宰相之職也蕭何舉韓信設壇而拜不復考試|{
	復扶又翻下同}
今以良為議郎 初燒當羌豪滇良擊破先零奪居其地|{
	羌無弋爰劒玄孫研居湟中至豪健羌中號其種為研種至研十三世孫燒當復豪健其子孫更以燒當為種號滇良者燒當之玄孫也自燒當至滇良世居河北大允谷而先零卑湳並皆強富滇良集諸雜種掩擊先零卑湳大破之奪居大榆中地繇是始強滇音顛零音憐}
滇良卒子滇吾立附落轉盛秋滇吾與弟滇岸率衆寇隴西敗太守劉盱於允街|{
	敗補邁翻賢曰允音鈆街音皆屬金城郡故城在今涼州昌松縣東南城臨麗水一名麗水城}
於是守塞諸羌皆叛詔謁者張鴻領諸郡兵撃之戰於允吾|{
	賢曰允吾縣名屬金城郡故城在今蘭州廣武縣西南允音鈆吾音牙杜佑曰西平郡龍支縣漢允吾縣地後漢為龍耆縣}
鴻軍敗沒冬十一月復遣中郎將竇固監捕虜將軍馬武等二將軍四萬人討之|{
	監古銜翻}
是歲南單于莫死弟汗立為伊伐於慮鞮單于|{
	鞮丁奚翻}


顯宗孝明皇帝上|{
	幼名陽後改名莊伏侯古今註曰莊之字曰嚴諡法照臨四方曰明光武第四子也}


永平元年春正月帝率公卿已下|{
	已下即以下孔頴達曰已與以字本同}
朝于原陵如元會儀|{
	朝陵如元會儀事死如事生也朝直遥翻}
乘輿拜神坐|{
	乘䋲證翻坐徂卧翻}
退坐東廂侍衛官皆在神坐後太官上食|{
	上時掌翻下同}
太常奏樂郡國上計吏以次前當神軒占其郡穀價及民所疾苦是後遂以為常 夏五月高密元侯鄧禹薨|{
	諡法行義說民曰元主義行德曰元此特以鄧禹中興元功而諡之耳後世謚法始有茂德丕績曰元}
東海恭王彊病上遣使者太醫乘驛視疾駱驛不絶|{
	驛傳逓馬也左傳謂之乘驛者乘驛馬也西漢謂之置傳馳傳駱驛往來不絶也}
詔沛王輔濟南王康淮陽王延詣魯省疾|{
	省悉景翻}
戊寅彊薨臨終上書謝恩言身旣夭命孤弱復為皇太后陛下憂慮|{
	言身既夭死而子孫又貽上之人憂慮也夭於紹翻復扶又翻下同}
誠悲誠慙息政小人也|{
	息子也政其名}
猥當襲臣後必非所以全利之也願還東海郡今天下新罹大憂|{
	謂光武崩也}
惟陛下加供養皇太后數進御餐|{
	供居用翻養羊亮翻數所角翻}
臣彊困劣言不能盡意願並謝諸王不意永不復相見也帝覽書悲慟從太后出幸津門亭發哀|{
	賢曰津門雒陽城南面西頭門也一名津陽門每門皆有亭李尤銘津門位未}
使大司空持節護喪事|{
	百官志司空掌水土事大喪掌將校復土今使護藩主喪殊禮也}
贈送以殊禮詔楚王英趙王栩北海王興及京師親戚皆會葬|{
	栩況羽翻}
帝追惟彊深執謙儉|{
	惟思也}
不欲厚葬以違其意於是特詔遣送之物務從約省衣足歛形|{
	歛力贍翻}
茅車瓦器物減於制以彰王卓爾獨行之志將作大匠留起陵廟|{
	秦曰將作少府景帝改為將作大匠掌修作宗廟路寢宮室陵園土木之工并樹桐梓之類列於道側}
秋七月馬武等撃燒當羌大破之餘皆降散|{
	降戶江翻}
山陽王荆私迎能為星者與謀議冀天下有變帝聞之徙封荆廣陵王遣之國|{
	郡國志廣陵在雒陽東一千六百四十里}
遼東太守祭肜使偏何討赤山|{
	偏氏高辛後急就章有偏呂何}
烏桓|{
	烏桓傳赤山在遼東西北數千里鮮卑傳云偏何撃漁陽赤山烏桓欽志賁盖歆志賁本赤山種而居漁陽塞外也}
大破之斬其魁帥|{
	帥所類翻}
塞外震讋|{
	讋之涉翻}
西自武威東盡玄菟|{
	郡國志武威郡在雒陽西三千五百里玄菟郡在雒陽東北四千里菟同都翻}
皆來内附野無風塵乃悉罷緣邊屯兵 東平王蒼以為中興三十餘年四方無虞宜修禮樂乃與公卿共議定南北郊冠冕車服制度|{
	光武建武二年立南郊中元元年立北郊於雒陽城北四里今定其冠冕車服制度漢官儀曰北郊壇在城西北角去城一里所}
及光武廟登歌八佾舞數上之|{
	記曰歌者在上貴人聲也天子樂舞八佾六十四人八八六十四人也佾音逸舞行列也上時掌翻}
好畤愍侯耿弇薨|{
	畤音止諡法在國遭憂曰愍時國有大喪故以諡弇言與國同戚也弇古含翻}


二年春正月辛未宗祀光武皇帝於明堂|{
	宗尊也尊而祀之以配上帝}
帝及公卿列侯始服冠冕玉佩以行事|{
	漢官儀曰天子冠通天諸侯王冠遠遊三公諸侯冠進賢三梁卿大夫尚書二千石博士冠兩梁千石以下至小吏冠一梁天子公卿特進諸侯祀天地明堂皆冠平冕天子十二旒三公九卿諸侯七其纓各如其綬色玄衣纁裳周禮曰王祀昊天上帝則服大裘而冕祀五帝亦如之三禮圖曰冕以三十升布漆而為之廣八寸長尺六寸前圜後方前下後高有俛伏之形故謂之冕欲人之位彌高而志彌下故以名焉董巴輿服志曰顯宗初服冕衣裳以祀天地衣裳以玄上纁下乘輿備文日月星辰十二章三公諸侯用山龍九章卿已下用華蟲七章皆五色采乘輿刺繡公卿已下皆織成陳留襄邑獻之徐廣車服注曰漢明帝按古禮備服章天子郊廟衣皁上絳下前三幅後四幅衣畫而裳繡禮記古之君子必佩玉君子於玉比德焉天子佩白玉公侯佩山玄玉大夫佩水蒼玉世子佩瑜玉晉志曰周禮弁師掌六冕司服掌六服自后王至庶人各有等差秦變古制郊祭之服皆以袀玄舊法掃地盡矣漢承秦故二百餘年未能有所制立及中興後明帝乃始采周官禮記尚書及諸儒記說備衮冕之服天子車乘冠服從歐陽氏說公卿已下從大小夏侯氏說}
禮畢登靈臺望雲物|{
	春秋左氏傳曰分至啓閉必書雲物杜預註曰雲物氣色灾變也素察妖祥逆為之備前書天文志曰歲正月旦旦至食為麥食至日昳為稷昳至晡為黍晡至下晡為菽下晡至日入為麻各以其時用雲色占種所宜}
赦天下 三月臨辟雍初行大射禮|{
	儀禮曰大射之禮王將祭射宫擇士以助祭也張虎侯熊侯豹侯其制若今之射的也}
冬十月壬子上幸辟雍初行養老禮以李躬為三老桓榮為五更|{
	更工衡翻}
三老服都紵大袍冠進賢扶玉杖|{
	紵直呂翻說文曰紵檾屬績紵以為美布故曰都紵續漢志進賢冠古緇布冠也文儒者之服也前高七寸後高三寸長八寸公侯三梁中二下石至博士兩梁自博士以下至小吏私學弟子皆一梁又仲春之月縣道皆案戶比民民年始七十者授之以玉杖玉杖長九尺端以鳩鳥為飾鳩者不噎之鳥也欲老人不噎爾雅翼曰刻玉為鳩置之杖端謂之鳩杖亦曰玉杖}
五更亦如之不杖乘輿到辟雍禮殿|{
	乘䋲證翻}
御坐東廂遣使者安車迎三老五更於太學講堂天子迎于門屏交禮道自阼階|{
	道讀曰導}
三老升自賓階至階天子揖如禮三老升東面三公設几九卿正履天子親袒割牲執醬而饋|{
	饋進食也醬食味之主故執之而饋}
執爵而酳|{
	酳音胤又土覲翻}
祝鯁在前祝饐在後|{
	饐一結翻食窒氣不通}
五更南面三公進供禮亦如之|{
	賢曰宋均曰三老老人知天地人之事者五更老人知五行更代事者鄭康成曰三老五更皆年老更事致仕者也天子以父兄養之示天下之孝弟也名以三五者取象三辰五星天所以照明天下者都布布名進賢冠古緇布冠也文儒者之服前高七寸後高三寸長八寸禮殿先聖先師也阼階東階主階也賓階西階也賢曰醬醢也酳嗽也所以潔口也陸德明曰以酒曰酳以水曰漱音義隱云飯畢盪口也音胤老人食多鯁饐故置人於前後祝之令其不鯁饐也都布之美者也進賢冠古緇布冠也玉杖長七尺端以鳩鳥為飾鳩者不噎之鳥也欲老人不噎更工衡翻}
禮畢引桓榮及弟子升堂上自為下說|{
	賢曰下說謂下語而講說也}
諸儒執經問難於前|{
	難乃旦翻}
冠帶縉紳之人圜橋門而觀聽者蓋億萬計|{
	漢官儀曰辟雍四門外有水以節觀者門外皆有橋觀者在水外故云圜橋門也圜繞也}
於是下詔賜榮爵關内侯 |{
	考異曰帝紀載詔文上言李躬而下獨封榮似脱躬字榮傳袁紀詔獨言桓榮不及李躬今闕疑}
三老五更皆以二千石祿養終厥身賜天下三老酒人一石肉四十斤上自為太子受尚書於桓榮及即帝位猶尊榮以師禮嘗幸太常府令榮坐東面設几杖會百官及榮門生數百人|{
	門生受業於門者也}
上親自執業|{
	執業猶執經也}
諸生或避位發難|{
	發難發疑難也難乃旦翻}
上謙曰大師在是旣罷悉以太官供具賜太常家榮每疾病帝輒遣使者存問太官太醫相望於道及篤上疏謝恩讓還爵土帝幸其家問起居入街下車擁經而前撫榮垂涕賜以牀茵帷帳刀劒衣被良久乃去自是諸侯將軍大夫問疾者不敢復乘車到門|{
	復扶又翻}
皆拜牀下榮卒帝親自變服臨喪送葬賜冢塋于首山之陽|{
	賢曰首陽山在今偃師縣西北}
子郁當嗣讓其兄子汎帝不許郁乃受封而悉以租入與之帝以郁為侍中 上以中山王焉郭太后少子太后尤愛之故獨留京師至是始與諸王俱就國賜以虎賁官騎|{
	賢曰漢官儀騶騎王家名官騎余據焉傳時賜以北軍胡騎百人便兵善射騶側尤翻}
恩寵尤厚獨得往來京師帝禮待隂郭每事必均數受賞賜|{
	數所角翻下同}
恩寵俱渥 甲子上行幸長安十一月甲申遣使者以中牢祠蕭何霍光帝過式其墓進幸河東癸卯還宮 十二月護羌校尉竇林坐欺罔及臧罪下獄死|{
	時羌滇吾叛滇岸來降林奏以滇岸為大豪後滇吾復降林又奏其為第一豪帝怪其一種兩豪以詰林窮驗知之怒而免林官涼州刺史又奏林臧罪遂下獄死下遐稼翻}
林者融之從兄子也|{
	從才用翻}
於是竇氏一公兩侯三公主四二千石相與並時|{
	賢曰一公大司空也兩侯安豐顯親也四二千石衛尉城門校尉護羌校尉中郎將也余據融傳融子穆尚内黄公主穆子勲尚東海王彊女沘陽公主友子固尚光武女温陽公主}
自祖及孫官府邸第相望京邑於親戚功臣中莫與為比及林誅帝數下詔切責融融惶恐乞骸骨詔令歸第養病 是歲初迎氣於五郊|{
	續漢書曰迎氣五郊之兆四方之兆各依其位中央之兆在未壇皆三尺立春之日迎春於東郊祭青帝句芒車服皆青歌青陽八佾舞雲翹之舞立夏之日迎夏於南郊祭赤帝祝融車服皆赤歌朱明舞如迎春先立秋十八日迎黄靈於中兆祭黄帝后土車服皆黄歌朱明八佾舞雲翹育命之舞立秋之日迎秋於西郊祭白帝蓐收車服皆白歌白藏八佾舞育命之舞立冬之日迎冬於北郊祭黑帝玄冥車服皆黑歌玄冥舞如迎秋}
新陽侯隂就子豐尚酈邑公主公主驕妬豐殺之被誅父母皆自殺|{
	公主光武女賢曰酈縣屬南陽郡酈音櫟}
南單于汗死單于比之子適立為䤈僮尸逐侯鞮單于|{
	賢曰䤈火奚翻}


三年春二月甲寅太尉趙憙司徒李訢免 丙辰以左馮翊郭丹為司徒 己未以南陽太守虞延為太尉甲子立貴人馬氏為皇后皇子炟為太子|{
	賢曰炟音丁達翻}
后援之女也光武時以選入太子宫能奉承隂后傍接同列禮則脩備上下安之遂見寵異及帝即位為貴人時后前母姊女賈氏亦以選入生皇子炟帝以后無子命養之謂曰人未必當自生子但患愛養不至耳后於是盡心撫育勞悴過於所生|{
	悴秦醉翻}
太子亦孝性淳篤母子慈愛始終無纎介之間|{
	賢曰纎介猶細微也間隙也間古莧翻}
后常以皇嗣未廣薦達左右若恐不及後宫有進見者每加慰納若數所寵引|{
	見賢遍翻數所角翻下同}
輒加隆遇及有司奏立長秋宫|{
	皇后宫謂之長秋宫}
帝未有所言皇太后曰馬貴人德冠後宫|{
	冠古玩翻}
即其人也后旣正位宫闈愈自謙肅好讀書常衣大練|{
	賢曰大練大帛也杜預註左傳曰大帛厚繒也好呼到翻衣於旣翻}
裙不加緣|{
	緣俞絹翻}
朔望諸姬主朝請|{
	朝直遥翻}
望見后袍衣踈麄以為綺縠就視乃笑后曰此繒特宜染色故用之耳羣臣奏事有難平者|{
	平決也難平難決者也}
帝數以試后后輒分解趣理各得其情然未嘗以家私干政事帝由是寵敬始終無衰焉帝思中興功臣乃圖畫二十八將於南宫雲臺以鄧禹為首次馬成吳漢王梁賈復陳俊耿弇杜茂寇恂傅俊岑彭堅鐔馮異王霸朱祐任光祭遵李忠景丹萬修蓋延邳肜銚期劉植耿純臧宫馬武劉隆又益以王常李通竇融卓茂合三十二人|{
	雲臺功臣之次以鄧禹吳漢賈復耿弇寇恂岑彭馮異朱祐祭遵景丹盖延銚期耿純臧宫馬武劉隆為一列馬成王梁陳俊杜茂傅俊堅鐔王霸任光李忠萬修邳肜劉植王常李通竇融卓茂為一列此序其次不與前史合鐔音覃又音尋祭則介翻盖古盍翻銚音姚}
馬援以椒房之親獨不與焉|{
	與讀曰預}
夏四月辛酉封皇子建為千乘王羨為廣平王|{
	郡國志高帝以西平昌置千乘郡在雒陽東千五百三十里地理志武帝征和元年置為平于國宣帝五鳳二年復為廣平國郡國志光武建武十三年省廣平國以其縣屬鉅鹿郡賢曰廣平縣故城在今洛州永年縣北千乘今青州縣故城在今淄州高苑縣北乘䋲證翻}
六月丁卯有星孛於天船北|{
	晉天文志大陵八星在胃北又北九星曰天船一曰舟星所以濟不通也天漢西南行絡大陵天船卷舌而南行孛蒲内翻}
帝大起北宫時天旱尚書僕射會稽鍾離意|{
	會古外翻}
詣闕免冠上疏曰昔成湯遭旱以六事自責曰政不節邪使民疾邪宫室營邪|{
	營范書作榮}
女謁盛邪苞苴行邪讒夫昌邪|{
	帝王記曰成湯大旱七年齋戒剪髪斷爪以己為犧牲禱於桑林之社以六事自責}
竊見北宫大作民失農時自古非苦宫室小狹但患民不安寧宜且罷止以應天心帝策詔報曰湯引六事咎在一人其冠履勿謝|{
	策詔者書詔於策也}
又勑大匠止作諸宫減省不急詔因謝公卿百僚遂應時澍雨|{
	說文曰雨所以澍注萬物故曰澍音注}
意薦全椒長劉平|{
	全椒縣屬九江郡賢曰今滁州縣}
詔徵拜議郎平在全椒政有恩惠民或增貲就賦或減年從役刺史太守行部|{
	行戶孟翻}
獄無繋囚人自以得所不知所問唯班詔書而去帝性褊察好以耳目隱發為明|{
	賢曰隱猶私也余謂隱者人耳目之所不及帝好以耳目窺其隱而發之好呼到翻}
公卿大臣數被詆毁|{
	數所角翻下同}
近臣尚書以下至見提曳|{
	提讀如冒絮提文帝之提音大計翻擲物以撃之也曳讀曰拽音奚結翻拖也引也一說提曳讀皆如字}
常以事怒郎藥崧|{
	藥姓崧名}
以杖撞之|{
	撞直江翻}
崧走入牀下帝怒甚疾言曰郎出崧乃曰天子穆穆諸侯皇皇|{
	記曲禮之文鄭曰皆行容止之貌也賢曰穆穆美也皇皇盛也}
未聞人君自起撞郎帝乃赦之是時朝廷莫不悚慄爭為嚴切以避誅責唯鍾離意獨敢諫爭|{
	爭讀曰諍}
數封還詔書臣下過失輒救解之會連有變異上疏曰陛下敬畏鬼神憂恤黎元而天氣未和寒暑違節者咎在羣臣不能宣化治職|{
	治直之翻}
而以苛刻為俗百官無相親之心吏民無雍雍之志|{
	爾雅曰雍雍和也}
至於感逆和氣以致天災百姓可以德勝難以力服鹿鳴之詩必言宴樂者|{
	鹿鳴詩小雅宴羣臣也其詩曰呦呦鹿鳴食野之草我有嘉賓鼔瑟吹笙又曰我有旨酒以宴樂嘉賓之心樂音洛}
以人神之心洽然後天氣和也願陛下垂聖德緩刑罰順時氣以調隂陽帝雖不能時用然知其至誠終愛厚之 秋八月戊辰詔改太樂官曰太予用讖文也|{
	賢曰尚書璇璣鈴曰有帝漢出德洽作樂名予故據璇璣鈴改之漢官儀曰太予樂令一人秩六百石蔡邕禮樂志曰漢樂四品一曰太予樂典郊廟上陵殿諸食舉之樂一曰周頌雅樂典辟雍饗射六宗社稷之樂三曰黄門鼔吹天子所以宴樂羣臣四曰短簫鐃歌軍樂也}
壬申晦日有食之詔曰昔楚莊無灾以致戒懼|{
	說苑曰楚莊王見天不見妖而地不出孽則曰天其忘予歟此能求過於天必不逆諫矣}
魯哀禍大天不降譴|{
	春秋感精符曰魯哀公時政彌亂絶不日食政亂之類當致日食之變而不應者譴之何益告之不悟故哀公之篇絶無日食之異}
今之動變儻尚可救有司勉思厥職以匡無德 冬十月甲子車駕從皇太后幸章陵|{
	光武建武六年改舂陵鄉為章陵縣}
荆州刺史郭賀官有殊政|{
	荆州統南陽南郡江夏零陵桂陽武陵長沙等郡}
上賜以三公之服黼黻冕旒|{
	東漢之制冕冠垂旒前後邃延三公諸侯之旒青玉為珠}
敕行部去襜帷|{
	爾雅曰襜帷蔽前襜帷者車之前帷也孔穎達曰幨帷山東謂之裳或曰潼容泯之詩曰淇水湯湯漸車帷裳注帷裳潼容也其上有盖 四方旁垂而下謂之襜行下孟翻去羌呂翻襜蚩占翻}
使百姓見其容服以章有德戊辰還自章陵 是歲京師及郡國七大水 莎車王賢以兵威逼奪于寘大宛媯塞王國|{
	媯塞國塞種臨媯水而居者因以為國名莎素禾翻寘徒賢翻宛於元翻媯居為翻塞悉則翻}
使其將守之于寘人殺其將君德立大人休莫霸為王賢率諸國兵數萬擊之大為休莫霸所敗脱身走還休莫霸進圍莎車中流矢死|{
	敗蒲邁翻中竹仲翻}
于寘人復立其兄子廣德為王廣德使其弟仁攻賢廣德父先拘在莎車賢乃歸其父以女妻之|{
	復扶又翻妻七細翻}
與之和親|{
	為廣德殺賢張本}
資治通鑑卷四十四














































































































































