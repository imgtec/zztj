<!DOCTYPE html PUBLIC "-//W3C//DTD XHTML 1.0 Transitional//EN" "http://www.w3.org/TR/xhtml1/DTD/xhtml1-transitional.dtd">
<html xmlns="http://www.w3.org/1999/xhtml">
<head>
<meta http-equiv="Content-Type" content="text/html; charset=utf-8" />
<meta http-equiv="X-UA-Compatible" content="IE=Edge,chrome=1">
<title>資治通鑒_231-資治通鑑卷二百三十_231-資治通鑑卷二百三十</title>
<meta name="Keywords" content="資治通鑒_231-資治通鑑卷二百三十_231-資治通鑑卷二百三十">
<meta name="Description" content="資治通鑒_231-資治通鑑卷二百三十_231-資治通鑑卷二百三十">
<meta http-equiv="Cache-Control" content="no-transform" />
<meta http-equiv="Cache-Control" content="no-siteapp" />
<link href="/img/style.css" rel="stylesheet" type="text/css" />
<script src="/img/m.js?2020"></script> 
</head>
<body>
 <div class="ClassNavi">
<a  href="/24shi/">二十四史</a> | <a href="/SiKuQuanShu/">四库全书</a> | <a href="http://www.guoxuedashi.com/gjtsjc/"><font  color="#FF0000">古今图书集成</font></a> | <a href="/renwu/">历史人物</a> | <a href="/ShuoWenJieZi/"><font  color="#FF0000">说文解字</a></font> | <a href="/chengyu/">成语词典</a> | <a  target="_blank"  href="http://www.guoxuedashi.com/jgwhj/"><font  color="#FF0000">甲骨文合集</font></a> | <a href="/yzjwjc/"><font  color="#FF0000">殷周金文集成</font></a> | <a href="/xiangxingzi/"><font color="#0000FF">象形字典</font></a> | <a href="/13jing/"><font  color="#FF0000">十三经索引</font></a> | <a href="/zixing/"><font  color="#FF0000">字体转换器</font></a> | <a href="/zidian/xz/"><font color="#0000FF">篆书识别</font></a> | <a href="/jinfanyi/">近义反义词</a> | <a href="/duilian/">对联大全</a> | <a href="/jiapu/"><font  color="#0000FF">家谱族谱查询</font></a> | <a href="http://www.guoxuemi.com/hafo/" target="_blank" ><font color="#FF0000">哈佛古籍</font></a> 
</div>

 <!-- 头部导航开始 -->
<div class="w1180 head clearfix">
  <div class="head_logo l"><a title="国学大师官网" href="http://www.guoxuedashi.com" target="_blank"></a></div>
  <div class="head_sr l">
  <div id="head1">
  
  <a href="http://www.guoxuedashi.com/zidian/bujian/" target="_blank" ><img src="http://www.guoxuedashi.com/img/top1.gif" width="88" height="60" border="0" title="部件查字,支持20万汉字"></a>


<a href="http://www.guoxuedashi.com/help/yingpan.php" target="_blank"><img src="http://www.guoxuedashi.com/img/top230.gif" width="600" height="62" border="0" ></a>


  </div>
  <div id="head3"><a href="javascript:" onClick="javascript:window.external.AddFavorite(window.location.href,document.title);">添加收藏</a>
  <br><a href="/help/setie.php">搜索引擎</a>
  <br><a href="/help/zanzhu.php">赞助本站</a></div>
  <div id="head2">
 <a href="http://www.guoxuemi.com/" target="_blank"><img src="http://www.guoxuedashi.com/img/guoxuemi.gif" width="95" height="62" border="0" style="margin-left:2px;" title="国学迷"></a>
  

  </div>
</div>
  <div class="clear"></div>
  <div class="head_nav">
  <p><a href="/">首页</a> | <a href="/ShuKu/">国学书库</a> | <a href="/guji/">影印古籍</a> | <a href="/shici/">诗词宝典</a> | <a   href="/SiKuQuanShu/gxjx.php">精选</a> <b>|</b> <a href="/zidian/">汉语字典</a> | <a href="/hydcd/">汉语词典</a> | <a href="http://www.guoxuedashi.com/zidian/bujian/"><font  color="#CC0066">部件查字</font></a> | <a href="http://www.sfds.cn/"><font  color="#CC0066">书法大师</font></a> | <a href="/jgwhj/">甲骨文</a> <b>|</b> <a href="/b/4/"><font  color="#CC0066">解密</font></a> | <a href="/renwu/">历史人物</a> | <a href="/diangu/">历史典故</a> | <a href="/xingshi/">姓氏</a> | <a href="/minzu/">民族</a> <b>|</b> <a href="/mz/"><font  color="#CC0066">世界名著</font></a> | <a href="/download/">软件下载</a>
</p>
<p><a href="/b/"><font  color="#CC0066">历史</font></a> | <a href="http://skqs.guoxuedashi.com/" target="_blank">四库全书</a> |  <a href="http://www.guoxuedashi.com/search/" target="_blank"><font  color="#CC0066">全文检索</font></a> | <a href="http://www.guoxuedashi.com/shumu/">古籍书目</a> | <a   href="/24shi/">正史</a> <b>|</b> <a href="/chengyu/">成语词典</a> | <a href="/kangxi/" title="康熙字典">康熙字典</a> | <a href="/ShuoWenJieZi/">说文解字</a> | <a href="/zixing/yanbian/">字形演变</a> | <a href="/yzjwjc/">金 文</a> <b>|</b>  <a href="/shijian/nian-hao/">年号</a> | <a href="/diming/">历史地名</a> | <a href="/shijian/">历史事件</a> | <a href="/guanzhi/">官职</a> | <a href="/lishi/">知识</a> <b>|</b> <a href="/zhongyi/">中医中药</a> | <a href="http://www.guoxuedashi.com/forum/">留言反馈</a>
</p>
  </div>
</div>
<!-- 头部导航END --> 
<!-- 内容区开始 --> 
<div class="w1180 clearfix">
  <div class="info l">
   
<div class="clearfix" style="background:#f5faff;">
<script src='http://www.guoxuedashi.com/img/headersou.js'></script>

</div>
  <div class="info_tree"><a href="http://www.guoxuedashi.com">首页</a> > <a href="/SiKuQuanShu/fanti/">四库全书</a>
 > <h1>资治通鉴</h1> <!--         下载:【右键另存为】即可 --></div>
  <div class="info_content zj clearfix">
  
<div class="info_txt clearfix" id="show">
<center style="font-size:24px;">231-資治通鑑卷二百三十</center>
    資治通鑑卷二百三十  宋 司馬光 撰<br />
<br />
  胡三省 音註<br />
<br />
  唐紀四十六【起閼逢困敦二月盡四月不滿一年】<br />
<br />
  德宗神武聖文皇帝五<br />
<br />
  興元元年二月戊申詔贈段秀實太尉諡曰忠烈厚恤其家【段秀實死節事見二百二十八卷建中四年諡神至翻】時賈隱林已卒贈左僕射賞其能直言也【卒子恤翻射寅謝翻直言事見上卷上年】 李希烈將兵五萬圍寧陵引水灌之濮州刺史劉昌以三千人守之【將即亮翻又音如字濮博木翻李希烈自建中四年改寧陵】滑州刺史李澄密遣使請降【李澄降賊見上卷上年使疏吏翻降戶江翻】上許以澄為汴滑節度使澄猶外事希烈希烈疑之遣養子六百人戍白馬【汴皮變翻白馬滑州治所】召澄共攻寧陵澄至石柱使其衆陽驚燒營而遁又諷養子令剽掠【令力丁翻剽匹妙翻下同】澄悉收斬之以白希烈希烈無以罪也劉昌守寧陵凡四十五日不釋甲韓滉遣其將王栖曜將兵助劉洽拒希烈栖曜以彊弩數千游汴水夜入寧陵城【滉呼廣翻將即亮翻將兵之將音同上 考異曰新書柏良器傳曰良器為武衛中郎將以兵隸浙西希烈圍寧陵遏水灌之親令軍中明日拔城良器以救兵至擇弩手善游者沿河渠夜入及旦伏弩乘城者皆死疑韓滉遣栖曜及良器同救寧陵舊栖曜傳曰將強弩數千夜入寧陵與此共是一事今參取之】明日從城上射希烈【射而亦翻】及其坐幄【坐才卧翻】希烈驚曰宣潤弩手至矣遂解圍去 朱泚自奉天敗歸【事見上卷建中四年泚且禮翻又音此】李晟謀取長安劉德信與晟俱屯東渭橋【劉德信屯東渭橋事始見二百二十八卷建中四年晟成正翻】不受晟節制晟因德信至營中數以滬澗之敗及所過剽掠之罪斬之【數所具翻又所主翻滬俟古翻剽匹妙翻滬澗之敗見二百二十八卷建中四年是年十一月既加李晟神策行營節度劉德信可得而不受節制乎况又有敗軍及剽掠之罪斬之宜矣】因以數騎馳入德信軍勞其衆【勞力到翻騎奇寄翻】無敢動者遂并將之軍勢益振【將即亮翻又音如字】李懷光既脅朝廷逐盧等【事見上卷上年朝直遥翻】内不自安遂有異志又惡李晟獨當一面【惡烏路翻下同】恐其成功奏請與晟合軍詔許之晟與懷光會于咸陽西陳濤斜築壘未畢【壘魯水翻】泚衆大至晟謂懷光曰賊若固守宫苑【宫苑謂宫城及苑城也】或曠日持久未易攻取【易以豉翻】今去其巢穴敢出求戰此天以賊賜明公不可失也懷光曰軍適至馬未秣士未飯【飯扶晩翻】豈可遽戰邪【邪音耶】晟不得已乃就壁晟每與懷光同出軍懷光軍士多掠人牛馬晟軍秋豪不犯懷光軍士惡其異已分所獲與之晟軍終不敢受懷光屯咸陽累月逗留不進【逗音豆 考異曰實録云懷光堅壁自守凡八十餘日按懷光以十一月癸巳解奉天圍李晟以二月戊申徙東渭橋其間纔七十六日實録所言謂懷光奔河中以前耳今但云累月】上屢遣中使趣之【使疏吏翻趣讀曰促】辭以士卒疲弊且當休息觀釁諸將數勸之攻長安【將即亮翻數所角翻下同】懷光不從密與朱泚通謀李晟屢奏恐其有變為所併請移軍東渭橋【泚且禮翻又音此晟成正翻李懷光既有異謀李晟與之連營於咸陽有不能一息安者其奏請移軍當也然必歸東渭橋者晟之本規也盖朱泚擁涇卒而據長安其敗也必當西奔晟以師自東逼之所以開其走路耳兵法圍城為之闕此其近之】上猶冀懷光革心收其力用寢晟奏不下【下戶嫁翻】懷光欲緩戰期且激怒諸軍奏言諸軍糧賜薄神策獨厚厚薄不均難以進戰上以財用方窘【窘巨隕翻】若糧賜皆比神策則無以給之不然又逆懷光意恐諸軍觖望【觖古穴翻怨望也】乃遣陸䞇詣懷光營宣慰因召李晟參議其事懷光意欲晟自乞减損使失士心沮敗其功【沮在呂翻敗補邁翻】乃曰將士戰鬭同而糧賜異何以使之恊力【將即亮翻】贄未有言數顧晟晟曰公為元帥得專號令晟將一軍受指蹤而已【數所角翻帥所類翻將即亮翻又音如字】至於增减衣食公當裁之懷光默然又不欲自减之遂止【李晟之答懷光氣和而辭正故能伐其謀】時上遣崔漢衡詣吐蕃兵【見上卷本年正月吐從暾入聲】吐蕃相尚結贊【相息亮翻】言蕃法兵以主兵大臣為信今制書無懷光署名故不敢進上命陸贄諭懷光懷光固執以為不可曰若克京城吐蕃必縱兵焚掠誰能遏之此一害也前有勑旨募士卒克城者人賞百緡彼兵五萬若援敕求賞五百萬緡何從可得此二害也虜騎雖來必不先進勒兵自固觀我兵勢勝則從而分功敗則從而圖變譎詐多端不可親信此三害也【李懷光雖欲養寇以自資然其陳用吐蕃三害其言亦各有理緡眉巾翻騎奇寄翻譎古穴翻】竟不肯署敕尚結贊亦不進軍陸贄自咸陽還上言賊泚稽誅保聚宫苑【上時掌翻還從宣翻又音如字泚且禮翻又音此朱泚自據長安居白華殿重兵多在苑中故言保聚宫苑】勢窮援絶引日偷生懷光總仗順之師乘制勝之氣【謂醴泉之勝也】鼓行芟翦易若摧枯【芟所銜翻易以豉翻】而乃寇奔不追師老不用諸帥每欲進取懷光輒沮其謀【諸帥謂李晟楊惠元等帥所類翻沮在呂翻】據兹事情殊不可解【解戶買翻曉也】陛下意在全護委曲聽從觀其所為亦未知感若不别務規略漸思制持惟以姑息求安終恐變故難測此誠事機危迫之秋也固不可以尋常容易處之【易弋䜴翻處昌呂翻】今李晟奏請移軍適遇臣銜命宣慰【晟成正翻銜戶緘翻】懷光偶論此事臣遂汎問所宜懷光乃云李晟既欲别行某亦都不要藉【要者須其用藉者借其力當時諸鎮有要藉官所以名官之意亦如此】臣猶慮有飜覆因美其軍盛彊懷光大自矜誇轉有輕晟之意臣又從容問云【從千容翻】囘日或聖旨顧問事之可否决定何如懷光已肆輕言不可中變遂云恩命許去事亦無妨【言上已許李晟去咸陽則其移軍於事體無妨也】要約再三【要一遥翻】非不詳審雖欲追悔固難為辭伏望即以李晟表出付中書敕下依奏【敕下李晟依其所奏也下戶嫁翻】别賜懷光手詔示以移軍事由【事由猶言事因也】其手詔大意云昨得李晟奏請移軍城東以分賊勢【東渭橋在京城東故云然晟成正翻】朕本欲委卿商量適會陸䞇囘奏云見卿語及於此仍言許去事亦無妨遂勑本軍允其所請如此則詞婉而直理順而明雖蓄異端何由起怨上從之晟自咸陽結陳而行【結陳而行以防李懷光追掩陳讀曰陣】歸東渭橋時鄜坊節度使李建徽神策行營節度使楊惠元猶與懷光聨營陸䞇復上奏曰懷光當管師徒【鄜音虜使疏吏翻復扶又翻上時掌翻當管猶言見管也】足以獨制兇寇逗留未進抑有它由所患太彊不資傍助比者又遣李晟李建徽楊惠元三節度之衆附麗其營【比毘至翻近也】無益成功祗足生事何則四軍接壘羣帥異心【李晟李建徽楊惠元之軍及李懷光之軍為四軍帥所類翻】論勢力則懸絶高卑【言懷光之軍最彊懷光之官最高相去懸絶】據職名則不相統屬【言懷光晟建徽惠元四人並為節度使各總一軍不相統屬】懷光輕晟等兵微位下而忿其制不從心晟等疑懷光養寇蓄姦而怨其事多陵已端居則互防飛謗欲戰則遞恐分功齟齬不和【齟壯所翻齬偶許翻】嫌釁遂構俾之同處必不兩全【處昌呂翻】彊者惡積而後亡弱者勢危而先覆【陸䞇言李懷光李建徽楊惠元之禍敗如燭照龜卜】覆亡之禍翹足可期【人立而翹一足則不能久翹足可期者言禍來之速也】舊寇未平新患方起憂歎所切實堪疚心【疚病也】太上消慝於未萌【太上猶言極上也慝惡也】其次救失於始兆况乎事情已露禍難垂成【難乃旦翻】委而不謀何以寧亂李晟見機慮變先請移軍建徽惠元勢轉孤弱為其吞噬理在必然【晟成正翻噬時制翻㗖也】它日雖有良圖亦恐不能自拔拯其危急唯在此時【拯救也】今因李晟願行便遣合軍同往託言晟兵素少【少詩沼翻】慮為賊泚所邀藉此兩軍迭為犄角【泚且禮翻犄居蟻翻】仍先諭旨密使促裝詔書至營即日進路懷光意雖不欲然亦計無所施是謂先人有奪人之心【左傳趙宣子之言先悉薦翻】疾雷不及掩耳者也【淮南子之言】解鬭不可以不離救焚不可以不疾理盡於此惟陛下圖之上曰卿所料極善然李晟移軍懷光不免悵望【悵丑亮翻怨也】若更遣建徽惠元就東【謂自咸陽東就李晟也】恐因此生辭【生辭猶今人言生言語也】轉難調息【調息猶今人言調停也】且更俟旬時【旬時猶言旬日也】 辛酉加王武俊同平章事兼幽州盧龍節度使【欲使之討朱滔也使疏吏翻】 李晟以為懷光反狀已明緩急宜有備蜀漢之路不可壅【此指漢蜀郡漢中郡二郡大界而言】請以禆將趙光銑等為洋利劍三州刺史【三州皆當入蜀之道之要禆賓彌翻將即亮翻洋音】祥各將兵五百以防未然【將音同上又音如字】上疑未决欲親總禁兵幸咸陽以慰撫為名趣諸將進討【趣讀曰促】或謂懷光曰此漢祖遊雲夢之策也【遊雲夢事見十一卷漢高祖六年】懷光大懼反謀益甚上垂欲行懷光辭益不遜上猶疑讒人間之【間古莧翻】甲子加懷光太尉增實食賜鐵劵【實食食實封也】遣神策右兵馬使李卞等往諭旨【使疏吏翻下同 考異曰邠志曰十六日詔加懷光太尉按實録甲子二十三日邠志誤幸奉天録舊傳李卞作李昇今從奉天録】懷光對使者投鐵劵於地曰聖人疑懷光邪【唐之臣子率稱君父為聖人邪音耶】人臣反賜鐵劵懷光不反今賜鐵劵是使之反也辭氣甚悖【悖蒲妺翻又蒲没翻】朔方左兵馬使張名振當軍門大呼曰【呼火故翻】太尉視賊不許擊待天使不敬【使疏吏翻朝廷所遣謂之天使盖謂君天也君之所遣猶天之所遣也】果欲反邪功高太山一旦弃之自取族滅富貴它人何益哉【言懷光反是自取族滅它人平其亂以為功而得富貴是富貴它人也】我今日必以死爭之懷光聞之謂曰我不反以賊方彊故須蓄鋭俟時耳懷光又言天子所居必有城隍【有水曰池無水曰隍】乃卒城咸陽未幾移軍據之【幾居豈翻】張名振曰乃者言不反【乃者猶言昨者也】今日拔軍此來何也何不攻長安殺朱泚取富貴引軍還邠邪【泚且禮翻又音此還從宣翻又音如字邠卑旻翻懷光所統朔方軍本屯邠州】懷光曰名振病心矣命左右引去拉殺之【拉落合翻】右武鋒兵馬使石演芬本西域胡人懷光養以為子懷光潛與朱泚通謀演芬遣其客郜成義詣行在告之【泚且禮翻又音此演以淺翻郜古到翻史炤曰郜姓也出自周文王子封郜國國在濟隂晉有尚書高昌郜久】請罷其都統之權成義至奉天告懷光子璀【統他綜翻俗音從上聲璀七罪翻】璀密白其父懷光召演芬責之曰我以爾為子柰何欲破我家今日負我死甘心乎演芬曰天子以太尉為股肱太尉以演芬為心腹太尉既負天子演芬安得不負太尉乎演芬胡人不能異心惟知事一人【一人謂天子也】苟免賊名而死死甘心矣懷光使左右臠食之皆曰義士也可令快死以刀斷其喉而去【臠力兖翻令力丁翻斷音短 考異曰邠志曰懷光投鐵劵于地使者懼焉名振呼於軍門又曰二月二十一日懷光拔其軍居咸陽又曰三月三日懷光廵咸陽城名振曰昨日言不反今悉軍此來何也又曰懷光既殺名振召演芬責之按名振云昨日言不反今何此來則是呼軍門之明日懷光即移軍咸陽若至咸陽已十三日因廵城而名振言之何得云昨日又何得云悉軍此來又名振與演芬同日死按舊傳云郜成義至奉天乃反其言告懷光子璀璀密告其父懷光若三月三日則車駕已幸梁洋不在奉天且是時反狀已彰灼如此豈能尚欺人云不反邪今從幸奉天録悉因投鐵劵言之】李卞等還言懷光驕慢之狀【還從宣翻又音如字】於是行在始嚴門禁【嚴門關出入之禁以防不虞】從臣皆密裝以待【史炤曰密具裝束所以備行從才用翻】乙丑加李晟河中同絳節度使上猶以為薄【德宗當患難之時進人若將加諸當事定之後退人若將墜諸淵晟成正翻使疏吏翻】丙寅又加同平章事上將幸梁州【梁州古漢中】山南節度使鹽亭嚴震聞之【鹽亭漢廣漢縣地梁置鹽亭縣唐屬梓州以產鹽名縣】遣使詣奉天奉迎又遣大將張用誠將兵五千至盩厔以來迎衛【至盩厔以來者言若迎衛之兵至盩厔而乘輿未至則當沿道漸進來前以迎乘輿不指定一處也盩厔音舟窒將即亮翻誠將音同上又音如字】用誠為懷光所誘隂與之通謀【誘音酉】上聞而患之會震繼遣牙將馬勛奉表上語之故【勛許云翻語牛倨翻】勛請亟詣梁州取嚴震符召用誠還府若不受召臣請殺之上喜曰卿何時復至此【還從宣翻又音如字復扶又翻又音如字】勛刻日時而去既得震符請壯士五人與之俱出駱谷用誠不知事泄以數百騎迎之【漢中取鳳翔之路南谷曰褒北谷曰駱騎奇寄翻】勛與之俱入驛時天寒勛多然藁火於驛外【然與燃同藁禾稈也】軍士皆往附火勛乃從容出懷中符以示用誠曰大夫召君用誠錯愕起走【從于容翻錯愕猝然驚也】壯士自後執其手擒之用誠子在勛後斫傷勛首壯士格殺其子仆用誠於地跨其腹以刀擬其喉曰出聲則死勛入其營士卒已擐甲執兵矣【仆方遇翻頓也擐戶慣翻】勛大言曰汝曹父母妻子皆在漢中一朝弃之與張用誠同反於汝曹何利乎大夫令我取用誠不問汝曹無自取族滅衆皆讋服【令力丁翻讋之涉翻失氣也】勛送用誠詣梁州震杖殺之命副將領其衆【將即亮翻】勛裹其首復命於行在愆期半日【愆期過期也】李懷光夜遣人襲奪李建徽楊惠元軍建徽走免惠元將奔奉天懷光遣兵追殺之懷光又宣言曰吾今與朱泚連和車駕且當遠避懷光以韓遊瓌朔方將也【泚且禮翻又音此韓遊瓌初事郭子儀李懷光東征遊瓌為邠寧留後瓌古囘翻將即亮翻】掌兵在奉天與遊瓌書約使為變遊瓌密奏之明日又以書趣之【懷光又以書趣遊瓌遊瓌盖又奏之也若據考異則後書為渾瑊所獲通鑑疑而不取趣讀曰促】上稱其忠義因問策安出對曰懷光總諸道兵故敢恃衆為亂今邠寧有張昕靈武有甯景璿【邠卑旻翻昕許斤翻璿似宣翻】河中有呂鳴岳振武有杜從政潼關有唐朝臣渭北有竇覦【潼音同朝直遥翻覦音俞】皆守將也【言此諸將各守其地也】陛下各以其地及其衆授之尊懷光之官罷其權則行營諸將各受本府指麾矣【罷懷光兵權則諸路兵雖在行營將不肯稟命於懷光而各稟本府之命】懷光獨立安能為亂上曰罷懷光兵權若朱泚何【言罷懷光恐無以制朱泚】對曰陛下既許將士以克城殊賞將士奉天子之命以討賊取富貴誰不願之邠府兵以萬數借使臣得而將之【將即亮翻又音如字】足以誅泚况諸道必有杖義之臣泚不足憂也上然之丁卯懷光遣其將趙昇鸞入奉天約其夕使别將逹奚小俊燒乾陵 【考異曰邠志作逹奚小進今從實録】令昇鸞為内應以驚脅乘輿【令力丁翻乘繩證翻】昇鸞詣渾瑊自言瑊遽以聞且請决幸梁州【軍戶昆翻又戶本翻瑊古銜翻 考異曰邠志二十六日懷光又使持書趣遊瓌渾公獲而奏之且使其卒物色我軍遊瓌不知不得以聞又怒瑊之虞已也慢罵于途上疑其變即日幸梁州今從實録奉天記曰上初拔奉天而車駕至宜夀縣渭水之陽謂侍臣曰朕之此行莫同永嘉之勢因然流涕渾瑊對曰臨大難無憂懼者聖人之勇也言訖濟河按新傳李惟簡追及上於盩厔西然後渾瑊繼至則上至渭陽時瑊猶未來今不取】上命瑊戒嚴瑊出部勒未畢上已出城西命戴休顔守奉天朝臣將士狼狽扈從【朝直遥翻將即亮翻從才用翻】戴休顔狥於軍中曰懷光已反遂乘城拒守朱泚之稱帝也【朱泚稱帝見二百二十八卷建中四年泚且禮翻又音此】兵部侍郎劉廼卧病在家泚召之不起使蔣鎮自往說之【說式芮翻】凡再往知不可誘脅【誘音酉】乃歎曰鎮亦忝列曹不能捨生以至於此【蔣鎮仕唐為工部侍郎故云亦忝列曹為泚所得不能死而受泚官自愧不能捨生取義】豈可復以己之腥臊汚漫賢者乎【復扶又翻臊蘇遭翻汚烏故翻漫謨官翻塗也】歔欷而返【歔音虛欷音希又許既翻】廼聞帝幸山南膺大呼【呼火故翻】自投於牀不食數日而卒【梁州在長安南山之南劉廼以乘輿播遷浸以益遠故自絶於衾衽之間】太子少師喬琳從上至盩厔稱老疾不堪山險削髪為僧匿於仙遊寺泚聞之召至長安以為吏部尚書於是朝士之竄匿者多出仕泚矣【劉廼以乘輿不能復還而自絶義不臣賊也喬琳等以乘輿不能復還出仕於泚苟性命而貪禄利也唐於此時亦云殆矣少始照翻盩厔音舟窒尚辰羊翻】懷光遣其將孟保【考異曰邠志作孟廷寶今從實録】惠靜壽孫福逹將精騎趣南山邀車<br />
<br />
  駕【逹將即亮翻又如字騎奇寄翻趣逡喻翻】遇諸軍糧料使張增於盩厔【使疏吏翻】三將曰彼使我為不臣我以追不及報之不過不使我將耳【過古禾翻又古卧翻將即亮翻言不過不使之為將也】因目增曰【目增示之以意欲因其言以紿衆】軍士未朝食如何增紿其衆曰此東數里有佛祠吾貯糧焉三將帥衆而東縱之剽掠【紿蕩亥翻貯丁呂翻帥讀曰率剽匹妙翻】由是百官從行者皆得入駱谷以追不及還報【還從宣翻又音如字 考異曰實録曰纔入駱谷懷光遣其將孟保等以數百騎來襲為後軍將侯仲莊所拒而退遂焚店驛而去舊嚴震傳曰賴山南兵擊之而退輿駕無警急之患今從邠志】懷光皆黜之河東將王權馬彚引兵歸太原【將即亮翻彚于貴翻權彚入援見上卷上】<br />
<br />
  【年以上幸山南聲聞不接故引兵歸史言馬燧怠於勤王】李晟得除官制拜哭受命【謂河中同絳及加同平章事之命晟成正翻】謂將佐曰長安宗廟所在天下根本若諸將皆從行誰當滅賊者乃治城隍繕甲兵為復京城之計【城隍即為東渭橋營塹治直之翻】先是東渭橋有積粟十餘萬斛度支給李懷光軍幾盡【先悉薦翻度徒洛翻幾居希翻】是時懷光朱泚連兵聲勢甚盛車駕南幸人情擾擾晟以孤軍處二彊寇之間【泚且禮翻又音此處昌呂翻】内無資糧外無救援徒以忠義感激將士故其衆雖單弱而鋭氣不衰又以書遺懷光辭禮卑遜【遺唯季翻】雖示尊崇而諭以禍福勸之立功補過故懷光慙恧未忍擊之【恧女六翻】晟曰畿内雖兵荒之餘猶可賦斂【斂力瞻翻】宿兵養寇患莫大焉乃以判官張彧假京兆尹擇四十餘人假官以督渭北芻粟不旬日皆充羨【羨弋線翻】乃流涕誓衆决志平賊【李懷光自河北千里赴難不可謂不勇於勤王以其兵力固可以指期收復君臣猜嫌反忠為逆張名振所謂自取族滅富貴它人有味乎其言也後之觀史者觀懷光之勤王始末與張名振所以諫懷光之言與夫史家歸功李晟之言則凡居功名之際者可不戒哉】 田悦用兵數敗【事並見前數所角翻】士卒死者什六七其下皆厭苦之上以給事中孔巢父為魏博宣慰使巢父性辯博至魏州對其衆為陳逆順禍福【為于偽翻】悦及將士皆喜兵馬使田緒承嗣之子也凶險多過失悦不忍殺杖而拘之悦既歸國内外撤警備三月壬申朔悦與巢父宴飲緒對弟姪有怨言其姪止之緒怒殺姪既而悔之曰僕射必殺我【僕射謂田悦也】既夕悦醉歸寢緒與左右密穿後垣入殺悦及其母妻等十餘人即帥左右執刀立於中門之内夾道【帥讀曰率】將旦以悦命召行軍司馬扈㠋判官許士則都虞候蔣濟議事府署深邃外不知有變士則濟先至召入亂斫殺之緒恐旣明事泄乃出門【出中門也】遇悦親將劉忠信方排牙【排牙者牙前將士各執其物以立於庭下俟節度使升聽事以次參謁也】緒疾呼謂衆曰劉忠信與扈㠋謀反昨夜刺殺僕射【呼火故翻下同刺七亦翻】衆大驚諠譁忠信未及自辯衆分裂殺之扈㠋來及戟門遇亂【節鎭外門列戟故謂之戟門】招諭將士將士從之者三分之一緒懼登城而立【田緒所登者魏州牙城也】大呼謂衆曰緒先相公之子諸君受先相公恩【先相公謂田承嗣也】若能立緒兵馬使賞緡錢二千大將半之下至士卒人賞百緡竭公私之貨五日取辦於是將士囘首殺扈㠋皆歸緒軍府乃安因請命於孔巢父巢父命緒權知軍府後數日衆乃知緒殺其兄【田悦者緒之從兄】雖悔怒【怒其殺兄而悔立之】而緒已立無如之何緒又殺悦親將薛有倫等二十餘人李抱眞王武俊引兵將救貝州聞亂不敢進朱滔聞悦死喜曰悦負恩天假手於緒也即遣其執憲大夫鄭景濟等【執憲大夫猶天朝御史大夫】將步騎五千助馬寔合兵萬二千人攻魏州寔軍王莽河縱騎兵及囘紇四出剽掠滔别遣人說緒許以本道節度使緒方危急遣隨軍侯臧詣貝州送欵於滔滔喜遣臧還報使亟定盟約時緒部署城内已定【謂魏州城内也】李抱眞王武俊又遣使詣緒許以赴援如悦存日之約緒召將佐議之幕僚曾穆盧南史曰用兵雖尚威武亦本仁義然後有功今幽陵之兵恣行殺掠白骨蔽野雖先僕射背德【背蒲妹翻】其民何罪今雖盛彊其亡可跂立而待也【跂去智翻舉踵而立也】况昭義恒冀方相與攻之【昭義李抱眞恒冀王武俊】柰何以目前之急欲從人為反逆乎不若歸命朝廷天子方蒙塵於外聞魏博使至必喜官爵旋踵而至矣【旋踵轉足也】緒從之遣使奉表詣行在城守以俟命 上之奉天也【謂自奉天幸山南】韓遊瓌帥其麾下八百餘人還邠州【帥讀曰率下同 考異曰邠志曰韓遊瓌使其子欽緒扈從懷光知之以戴休顔代領其職仍假遊瓌邠州刺史將使其黨張昕害之遊瓌既失兵柄未知所從說客劉南金曰竊觀人心莫不戀主邠有留甲可以圖變公得之邠殆天假也乃使麾下將范希朝趙懷誘其軍歸邠士皆從之休顔率麾下卒據城門士不得盡出其從遊瓌至邠者八百餘人按舊遊瓌傳無受懷光邠州刺史事休顔傳云及李懷光叛據咸陽使誘休顔休顔集三軍斬其使嬰城自守懷光大駭遂自涇陽夜遁其月拜檢校工部尚書奉天行營節度使且上幸山南命休顔留守奉天遊瓌先懷光隂謀二人豈更受懷光節度盖當時出幸倉卒遊瓌扈從不及或以與渾瑊有隙不敢南行故率麾下歸邠州耳】李懷光以李晟軍浸盛惡之【惡烏路翻】欲引軍自咸陽襲東渭橋三令其衆衆不應竊相謂曰若與我曹擊朱泚惟力是視若欲反我曹有死不能從也懷光知衆不可強【強其兩翻】問計於賓佐節度廵官良鄉李景略曰【良鄉漢縣屬涿郡唐屬幽州】取長安殺朱泚散軍還諸道單騎詣行在如此臣節亦未虧功名猶可保也頓首懇請至於流涕懷光許之都虞候閻晏等勸懷光東保河中徐圖去就懷光乃說其衆曰【說式芮翻】今且屯涇陽召妻孥於邠俟至與之俱往河中春裝既辦還攻長安未晩也東方諸縣皆富實軍之日聽爾俘掠衆許之【東方諸縣謂涇陽以東諸縣也 考異曰幸奉天録李晟至東渭橋旬日之後軍用整備懷光患之稍移軍涇陽與朱泚約同㓕晟軍舊懷光傳曰懷光劫李建徽等軍移於好畤又曰居二旬乃驅兵掠涇陽富平自同州往河中朱泚傳曰懷光為泚所賣慙怒憤耻移於好畤按實録三月甲申懷光自咸陽燒營走歸河中幸奉天録曰三月懷光拔咸陽掠三原等十二縣雞犬無遺老小步騎百餘萬皆不云移軍好畤及涇陽今從邠志及幸奉天録】懷光乃謂景略曰曏者之議軍衆不從子宜速去不且見害遣數騎送之景略出軍門慟哭曰不意此軍䧟於不義【朔方軍平安史拒囘紇吐蕃功高天下備盡忠力一旦從懷光反是䧟於不義】懷光遣使詣邠州令留後張昕悉所留兵萬餘人及行營將士家屬會涇陽仍遣其將劉禮等將三千餘騎脅遷之韓遊瓌說昕曰【說式芮翻】李太尉功高自蹈禍機中丞今日可以自求富貴遊瓌請帥麾下以從【從才用翻】昕曰昕微賤賴李太尉得至此不忍負也遊瓌乃謝病不出隂與諸將高固楊懷賓等相結時崔漢衡以吐蕃兵營于邠南高固曰昕以衆去則邠城空矣乃詐為渾瑊書召吐蕃使稍逼邠城昕等懼竟不敢出昕等謀殺諸將之不從者遊瓌知之先與高固等舉兵殺昕【昕許斤翻將即亮翻瓌古囘翻 考異曰邠志曰三月二十三日張昕戒劉禮等衷甲而入昕小史李岌密報遊瓖遊瓖伏甲先起高固等帥衆應之遂斬昕于府中遊瓖既據邠府遣李旻懷光乃走蒲州按實録甲申懷光自咸陽燒營走歸河中然則遊瓌殺昕必在其前今因懷光走見之】遣楊懷賓奉表以聞且遣人告崔漢衡漢衡矯詔以遊瓌知軍府事軍中大喜懷光子旻在邠【邠卑旻翻】遊瓌遣之或曰不殺旻何以自明【言遣旻則上疑遊瓌與懷光通將無以自明也】遊瓌曰殺旻則懷光怒其衆必至不如釋旻以走之時楊懷賓子朝晟在懷光軍中為右廂兵馬使【朝直遥翻晟成正翻使疏吏翻】聞之泣白懷光曰父立功於國【言其父殺張昕以邠城返正也】子當誅夷不可典兵懷光囚之【為後赦朝晟張本】於是遊瓌屯邠寧戴休顔屯奉天駱元光屯昭應尚可孤屯藍田皆受李晟節度晟軍聲大振始懷光方彊朱泚畏之與懷光書以兄事之約分帝關中永為鄰國及懷光决反逼乘輿南幸【泚且禮翻又音此乘繩證翻】其下多叛之勢益弱泚乃賜懷光詔書以臣禮待之且徵其兵懷光慙怒内憂麾下為變外恐李晟襲之遂燒營東走掠涇陽等十二縣雞犬無遺 【考異曰舊高郢傳曰懷光將歸河中郢言西迎大駕豈非忠乎懷光不聽按德宗因懷光迫逐遂幸梁州借使懷光欲迎駕德宗豈肯來乎今不取】及富平【懷光行及富平也】大將孟涉段威勇將數千人奔于李晟將士在道散亡相繼至河中或勸河中守將呂鳴岳焚橋拒之鳴岳以兵少恐不能支遂納之【若呂鳴岳焚蒲津橋懷光將士之心已離必潰散於河西不得至河中矣將即亮翻少詩紹翻】河中尹李齊運弃城走懷光遣其將趙貴先築壘於同州【備唐兵討之也】刺史李紓懼奔行在幕僚裴向攝州事詣貴先責以逆順之理貴先感寤遂請降同州由是獲全向遵慶之子也【裴遵慶肅宗朝為相】懷光使其將符嶠襲坊州據之渭北守將竇覦帥獵團七百圍之【團結獵戶為兵謂之獵團帥讀曰率】嶠請降詔以覦為渭北行軍司馬 丁亥以李晟兼京畿渭北鄜坊丹延節度使【鄜音夫】 庚寅車駕至城固唐安公主薨【蜀州唐安郡】上長女也 上在道民有獻瓜果者上欲以散試官授之【散官即文散階武散階也試官事始見二百五卷武后長夀元年】訪於陸贄贄上奏以為爵位恒宜慎惜不可輕用【恒戶登翻】起端雖微流弊必大獻瓜果者止可賜以錢帛不當酬以官上曰試官虛名無損於事贄又上奏其畧曰自兵興以來財賦不足以供賜而職官之賞興焉青朱雜㳫於胥徒【周禮六官之屬大夫士之下有府史胥徒鄭氏注曰胥徒民之給徭役者若今衛士矣胥讀如諝謂其有才智為什長胥私呂翻又思餘翻】金紫普施於輿皁【左傳芋無宇曰人有十等王臣公公臣大夫大夫臣士士臣皁皁臣輿輿臣隸隸臣僚僚臣僕僕臣臺】當今所病方在爵輕設法貴之猶恐不重若又自弃將何勸人夫誘人之方惟名與利名近虛而於教為重利近實而於德為輕【近其靳翻】專實利而不濟之以虛則耗匱而物力不給專虛名而不副之以實則誕謾而人情不趨【誕謾虛言也趨七喻翻又音如字】故國家命秩之制有職事官有散官有勲官有爵號然掌務而授俸者唯繫職事之一官也此所謂施實利而寓虛名者也其勲散爵號三者所繫大抵止於服色資䕃而已【服色謂紫緋淺緋深緑淺緑深青淺青及黄其色各以品為差資廕謂隨資品得廕其子若孫及曾孫也】此所謂假虛名而佐實利者也今之員外試官頗同勲散爵號雖則授無費禄受不占員【占音之贍翻】然而突銛鋒排患難者則以是賞之【銛息亷翻利也難乃旦翻】竭筋力展勞效者又以是酬之若獻瓜果者亦授試官則彼必相謂曰吾以忘軀命而獲官此以進瓜果而獲官是乃國家以吾之軀命同於瓜果矣視人如草木誰復為用哉【復扶又翻】今陛下既未有實利以敦勸又不重虛名而濫施人無藉焉則後之立功者將曷用為賞哉䞇在翰林為上所親信居艱難中雖有宰相大小之事上必與贄謀之故當時謂之内相【相息亮翻下同】上行止必與之俱梁洋道險嘗與䞇相失經夕不至上驚憂涕泣募得䞇者賞千金久之乃至上喜甚太子以下皆賀然贄數直諫迕上意【數所角翻迕五故翻】盧雖貶官【杞貶官見上卷上年】上心庇之䞇極言姦邪致亂上雖貌從心頗不悦故劉從一姜公輔皆自下陳登用【二人為相見上卷上年劉從一自吏部郎中姜公輔自翰林學士下陳猶下列也】贄恩遇雖隆未得為相【為上追仇陸贄盡言而貶贄張本】壬辰車駕至梁州山南地薄民貧自安史以來盗賊攻剽【剽匹妙翻】戶口減耗大半雖節制十五州【十五州梁洋興鳳開通渠集蓬利壁巴閬果金也】租賦不及中原數縣及大駕駐蹕糧用頗窘上欲西幸成都嚴震言於上曰山南地接京畿李晟方圖收復籍六軍以為聲援若幸西川則晟未有收復之期也衆議未决會李晟表至言陛下駐蹕漢中所以繫億兆之心成滅賊之勢若規小捨大【規小謂欲幸成都以便資用捨大謂捨興復之功而苟安於一隅】遷都岷峨則士庶失望雖有猛將謀臣無所施矣上乃止嚴震百方以聚財賦民不至困窮而供億無乏牙將嚴礪震之從祖弟也震使掌轉餉事甚修辨【史言嚴震供奉車駕無闕之功辨讀曰辦】 初奉天圍既解李楚琳遣使入貢上不得已除鳳翔節度使而心惡之【惡其殺張鎰而附朱泚且在肘腋之下也惡烏路翻】議者言楚琳凶逆反覆若不隄防恐生窺伺【伺相吏翻】由是楚琳使者數輩至上皆不引見【見賢遍翻】留之不遣甫至漢中欲以渾瑊代楚琳鎮鳳翔陸䞇上奏以為楚琳殺帥助賊【事見二百二十八卷建中四年帥所類翻】其罪固大但以乘輿未復大憝猶存【書云元惡大憝憝亦惡也音徒對翻】勤王之師悉在畿内急宣速告晷刻是爭【言較晷刻而爭遲速也】商嶺則道迃且遥駱谷復為盗所扼【復扶又翻】僅通王命唯在褒斜【據九域志商州之路逹金洋皆數百里而洋又遠於金自商州西至長安復二百餘里則其路迂遥至長安盖一千一百餘里自駱谷關至洋州亦五百餘里惟寶雞南入大散關至梁州五百里而近宋白曰興元府東北至長安取駱谷路六百五十二里取斜谷路九百二十三里驛路一千二百二十三里】此路若又阻艱南北遂將夐絶【夐休正翻】以諸鎮危疑之勢居二逆誘脅之中【二逆謂朱泚李懷光也誘音酉】洶洶羣情各懷向背【背蒲妹翻】儻或楚琳憾公肆猖狂南塞要衝【塞悉則翻】東延巨猾則我咽喉梗而心膂分矣【咽因肩翻】今楚琳能兩端顧望乃是天誘其衷【兩端顧望謂李楚琳外奉朝廷而隂事朱泚杜預曰衷中也陸德明曰衷音中或丁仲翻】故通歸塗將濟大業陛下誠宜深以為念厚加撫循得其遲疑便足集事必欲精求素行追抉宿疵【行下孟翻抉於决翻】則是改過不足以補愆自新不足以贖罪凡今將吏豈得盡無疵瑕人皆省思【省悉景翻】孰免疑畏又况阻命之輩脅從之流自知負恩安敢歸化斯釁非小所宜速圖伏願陛下思英主大略勿以小不忍虧撓興復之業也【撓奴教翻】上釋然開悟善待楚琳使者優詔存慰之丁酉加宣武節度使劉洽同平章事 己亥以行在<br />
<br />
  都知兵馬使渾瑊同平章事兼朔方節度使朔方邠寧振武永平奉天行營兵馬副元帥【將罷李懷光兵權故先用渾瑊】 庚子詔數李懷光罪惡【數所具翻又所主翻】叙朔方將士忠順功名猶以懷光舊勲曲加容貸其副元帥太尉中書令河中尹并朔方諸道節度觀察等使宜並罷免【將即亮翻貸來戴翻 考異曰舊高郢傳曰懷光歸河中又欲悉衆而西時渾瑊軍孤羣帥未集郢與李鄘誓死駐之屬懷光長子璀候郢郢乃諭以逆順曰人臣所宜效順且自天寶以來阻兵者今復誰在况國家自有天命非獨人力今若恃衆西向自絶于天安知三軍不有奔潰者乎李璀震懼流淚氣索明年郢與都知兵馬使呂鳴岳都虞候張延英同謀間道上表及受密詔事泄二將立死懷光乃大集將卒白刃盈庭引郢詰之郢挺然抗詞無所慙隱憤氣感觀者淚下懷光慙沮而止按實録懷光以興元元年正月甲申走歸河中己亥以渾瑊為副元帥四月辛丑朔始臨軒授瑊節鉞與郢傳年月全不相應今不取】授太子太保其所管兵馬委本軍自舉一人功高望重者便宜統領速具奏聞當授旌旄以從人欲【旌旄猶言節旄也】 夏四月壬寅以邠寧兵馬使韓遊瓌為邠寧節度使癸卯以奉天行營兵馬使戴休顔為奉天行營節度使 靈武守將甯景璿為李懷光治第别將李如暹曰李太尉逐天子而景璿為之治第【治直之翻為于偽翻】是亦反也攻而殺之 甲辰加李晟鄜坊京畿渭北商華副元帥【分李懷光兵柄以授李晟渾瑊鄜音夫華戶化翻】晟家百口及神策軍士家屬皆在長安朱泚善遇之軍中有言及家者晟泣曰天子何在敢言家乎泚使晟親近以家書遺晟【遺唯季翻】曰公家無恙晟怒曰爾敢為賊為間【為賊于偽翻間古莧翻】立斬之軍士未授春衣盛夏猶衣裘褐【猶衣於既翻】終無叛志【史言李晟以忠義感激士心】乙巳以陝虢防遏使唐朝臣為河中同絳節度使【陜失冉翻】前河中尹李齊運為京兆尹供晟軍糧役【役者輓輸浚築之事】 庚戌以魏博兵馬使田緒為魏博節度使 渾瑊帥諸軍出斜谷【帥讀曰率】崔漢衡勸吐蕃出兵助之尚結贊曰邠軍不出將襲我後韓遊瓌聞之遣其將曹子逹將兵三千往會瑊軍吐蕃遣其將論莽羅依將兵二萬從之李楚琳遣其將石鍠將卒七百從瑊拔武功【鍠戶盲翻】庚戌朱泚遣其將韓旻攻武功鍠以其衆迎降瑊戰不利收兵登西原【其地高平在武功縣西故曰西原】會曹子逹以吐蕃至擊旻大破之於武亭川 【考異曰邠志十日日破旻等而實録云乙丑盖奏到之日也今從邠志】斬首萬餘級旻僅以身免瑊遂引兵屯奉天與李晟東西相應以逼長安 上欲為唐安公主造塔厚葬之【時唐安公主薨於城固塔浮圖也為于偽翻】諫議大夫同平章事姜公輔表諫以為山南非久安之地公主之葬會歸上都【會合也要也上都謂長安】此宜儉薄以副軍須之急【凡行軍資糧器械所須者皆謂之軍須】上使謂陸贄曰唐安造塔其費甚微非宰相所宜論公輔正欲指朕過失自求名耳相負如此當如何處之【相息亮翻處昌呂翻】贄上奏以為公輔任居宰相遇事論諫不當罪之其略曰公輔頃與臣同在翰林臣今據理辯直則涉於私黨之嫌希旨順成則違於匡輔之義涉嫌上貽於身患違義實玷於君恩徇身忘君臣之恥也【玷都念翻玉病也】又曰唯闇惑之主則怨讟溢於下國而耳不欲聞腥德逹於上天而心不求寤【讟徒牧翻謗也書呂刑曰德刑聞惟腥】迨乎顛覆猶未知非又曰當問理之是非豈論事之大小虞書曰兢兢業業一日二日萬幾【幾居希翻見臯陶謨】唐虞之際主聖臣賢慮事之微日至萬數然則微之不可不重也如此陛下又安可忽而不念乎又曰若以諫爭為指過【爭讀曰諍】則剖心之主不宜見罪於哲王【武王數紂之罪曰斮朝涉之脛剖賢人之心】以諫爭為取名則匪躬之臣不應垂訓於聖典【易曰王臣蹇蹇匪躬之故應一凌翻】又曰假有意將指過諫以取名但能聞善而遷見諫不逆則所指者適足以彰陛下莫大之善所取者適足以資陛下無疆之休因而利焉所獲多矣儻或怒其指過而不改則陛下招惡直之譏【惡烏路翻又如字】黜其取名而不容則陛下被違諫之謗【被皮義翻】是乃掩己過而過彌著損彼名而名益彰果而行之所失大矣上意猶怒甲寅罷公輔爲左庶子加西川節度使張延賞同平章事賞其供億無乏故也【上在漢中藉西川供億為張延賞入相張本上時掌翻使疏吏翻】 朱泚姚令言數遣人誘涇原節度使馮河清【泚且禮翻又音此數所角翻誘音酉使疏吏翻】河清皆斬其使者大將田希鑒密與朱泚通殺河清以軍府附於泚泚以希鑒為涇原節度使【將即亮翻 考異曰邠志曰興元元年四月渾公受鉞專征出斜谷崔公勸吐蕃分軍應援尚結贊曰邠軍不出乘我也韓公使曹子逹帥甲三千赴于渾公吐蕃乃以三萬餘從之李楚琳使石鍠以卒七百人從渾瑊進收武功遂居之十日朱泚使韓旻田旻以卒三千寇武功渾公禦之陳于東郊石鍠以其卒降旻於陳渾公軍敗乃馳登西原建旗收卒會邠師以吐蕃至賊不知乃悉衆追渾公遂為吐蕃所覆皆死焉田旻以馬逸獲免吐蕃既勝泚軍乃大掠而去涇人相傳言吐蕃助國有功將以叛卒之孥賞而歸之涇人曰不殺馮公雖吾親族亦將不免矣十四日涇卒殺河清以田希鑒請命於泚泚授希鑒涇原節度大使賜金帛使和西戎西戎皆受賂焉希鑒疏涇將之不與己者以告朱泚請殺之泚曰我曲彼直不許按希鑒殺河清必有宿謀或為此訛言以揺衆耳今從實録河清死在三月今從邠志】上問陸䞇近有卑官自山北來者【梁州在山南岐雍在山北】率非良士有邢建者論說賊勢語最張皇【皇大也】察其事情頗似窺覘【覘丑亷翻又丑艶翻】今已於一所安置如此之類更有數人若不追尋恐成姦計卿試思之如何為便䞇上奏以為今盗據宫闕有涉險遠來赴行在者當量加恩賞豈得復猜慮拘囚【量音良復扶又翻】其畧曰以一人之聽覽而欲窮宇宙之變態以一人之防慮而求勝億兆之姦欺役智彌精失道彌遠項籍納秦降卒二十萬慮其懷詐復叛一舉而盡阬之其於防虞亦已甚矣【阬降卒事見九卷漢高祖元年】漢高豁逹大度天下之士至者納用不疑其於備慮可謂疏矣【疏與踈同】然而項氏以滅劉氏以昌蓄疑之與推誠其效固不同也秦皇嚴肅雄猜而荆軻奮其隂計【事見七卷秦始皇二十年】光武寛容博厚而馬援輸其欵誠【事見四十一卷漢世祖建武四年】豈不以虚懷待人人亦思附任數御物物終不親情思附則感而悦之雖寇讐化為心膂矣意不親則懼而阻之雖骨肉結為仇慝矣又曰陛下智出庶物有輕待人臣之心思用萬機有獨馭區㝢之意謀吞衆畧有過慎之防明照羣情有先事之察【先悉薦翻】嚴束百辟有任刑致理之規威制四方有以力勝殘之志【此數語曲盡德宗心事異日安免追仇乎】由是才能者怨於不任忠藎者憂於見疑【藎徐刃翻詩王之藎臣毛氏傳曰藎進也】著勲業者懼於不容懷反側者迫於及討馭致離叛搆成禍災天子所作天下式瞻小猶慎之矧又非小願陛下以覆車之轍為戒實宗社無疆之休丁巳以前山南東道節度使南皮賈耽為工部尚書先是耽使行軍司馬樊澤奏事行在【先悉薦翻】澤既復命方大宴有急牒至以澤代耽為節度使【事見上卷興元元年】耽内牒懷中宴飲如故顔色不改宴罷召澤告之且命將吏謁澤牙將張獻甫怒曰行軍為尚書問天子起居【為于偽翻】乃敢自圖節鉞奪尚書土地事人不忠請殺之耽曰是何言也天子所命即為節度使矣即日離鎮以獻甫自隨軍府遂安【即日離鎮既得君命召不俟駕之義亦所以遏亂原以張獻甫自隨則樊澤無所猜嫌亦所以全獻甫也離力智翻】 左僕射李揆自吐蕃還甲子薨於鳳州【李揆入吐蕃見二百二十八卷建中四年盖自吐蕃還赴興元至鳳州而薨】 韓遊瓌引兵會渾瑊於奉天 丙寅加平盧節度使李納同平章事丁卯義王玼薨【玼玄宗子玼音此又且禮翻】 朱滔攻貝州百餘日馬寔攻魏州亦踰四旬皆不能下賈林復為李抱真說王武俊【復扶又翻為于偽翻說輸芮翻】曰朱滔志呑貝魏復值田悦被害【復抉又翻】儻旬日不救則魏博皆為滔有矣魏博既下則張孝忠必為之臣【張孝忠時鎮易定】滔連三道之兵【三道為幽州易定魏博】益以囘紇【時囘紇遣兵助滔】進臨常山【恒州常山郡王武俊居之】明公欲保其宗族得乎常山不守則昭義退保西山【自常山南至趙州皆恒冀廵屬又西南抵邢州界即昭義廵屬阻山以為固】河朔盡入於滔矣不若乘貝魏未下與昭義合兵救之滔既破亡則關中喪氣朱泚不日梟夷【朱泚竊據關中滔破則泚喪氣矣喪息浪翻】鑾輿反正諸將之功孰有居明公之右者哉武俊悦從之戊辰武俊軍于南宫東南抱真自臨洺引兵會之與武俊營相距十里兩軍尚相疑明日抱真以數騎詣武俊營賓客共諫止之抱真命行軍司馬盧玄卿勒兵以俟曰吾之此舉繫天下安危若其不還領軍事以聽朝命亦惟子勵將士以雪讎耻亦惟子言終遂行武俊嚴備以待之抱真見武俊叙國家禍難【難乃旦翻】天子播遷持武俊哭流涕縱横【縱子容翻】武俊亦悲不自勝【勝音升】左右莫能仰視遂與武俊約為兄弟誓同滅賊武俊曰相公十兄名高四海【李抱真第十故呼為十兄】曏蒙開諭得棄逆從順免葅醢之罪享王公之榮今又不間胡虜【間古莧翻王武俊本出於夷落】辱為兄弟武俊當何以為報乎滔所恃者囘紇耳不足畏也戰日願十兄按轡臨視武俊决為十兄破之【决為于偽翻】抱真退入武俊帳中酣寢久之武俊感激待之益恭指心仰天曰此身已許十兄死矣【史言抱真推心待武俊以成大功】遂連營而進 山南地熱上以軍士未有春服亦自御裌衣【裌音夾】<br />
<br />
  資治通鑑卷二百三十<br />
<br />
<史部,編年類,資治通鑑>  <br>
   </div> 

<script src="/search/ajaxskft.js"> </script>
 <div class="clear"></div>
<br>
<br>
 <!-- a.d-->

 <!--
<div class="info_share">
</div> 
-->
 <!--info_share--></div>   <!-- end info_content-->
  </div> <!-- end l-->

<div class="r">   <!--r-->



<div class="sidebar"  style="margin-bottom:2px;">

 
<div class="sidebar_title">工具类大全</div>
<div class="sidebar_info">
<strong><a href="http://www.guoxuedashi.com/lsditu/" target="_blank">历史地图</a></strong>  
<a href="http://www.880114.com/" target="_blank">英语宝典</a>  
<a href="http://www.guoxuedashi.com/13jing/" target="_blank">十三经检索</a> 
<br><strong><a href="http://www.guoxuedashi.com/gjtsjc/" target="_blank">古今图书集成</a></strong> 
<a href="http://www.guoxuedashi.com/duilian/" target="_blank">对联大全</a> <strong><a href="http://www.guoxuedashi.com/xiangxingzi/" target="_blank">象形文字典</a></strong> 

<br><a href="http://www.guoxuedashi.com/zixing/yanbian/">字形演变</a>  <strong><a href="http://www.guoxuemi.com/hafo/" target="_blank">哈佛燕京中文善本特藏</a></strong>
<br><strong><a href="http://www.guoxuedashi.com/csfz/" target="_blank">丛书&方志检索器</a></strong> <a href="http://www.guoxuedashi.com/yqjyy/" target="_blank">一切经音义</a>  

<br><strong><a href="http://www.guoxuedashi.com/jiapu/" target="_blank">家谱族谱查询</a></strong>  <strong><a href="http://shufa.guoxuedashi.com/sfzitie/" target="_blank">书法字帖欣赏</a></strong> 
<br>

</div>
</div>


<div class="sidebar" style="margin-bottom:0px;">

<font style="font-size:22px;line-height:32px">QQ交流群9:489193090</font>


<div class="sidebar_title">手机APP 扫描或点击</div>
<div class="sidebar_info">
<table>
<tr>
	<td width=160><a href="http://m.guoxuedashi.com/app/" target="_blank"><img src="/img/gxds-sj.png" width="140"  border="0" alt="国学大师手机版"></a></td>
	<td>
<a href="http://www.guoxuedashi.com/download/" target="_blank">app软件下载专区</a><br>
<a href="http://www.guoxuedashi.com/download/gxds.php" target="_blank">《国学大师》下载</a><br>
<a href="http://www.guoxuedashi.com/download/kxzd.php" target="_blank">《汉字宝典》下载</a><br>
<a href="http://www.guoxuedashi.com/download/scqbd.php" target="_blank">《诗词曲宝典》下载</a><br>
<a href="http://www.guoxuedashi.com/SiKuQuanShu/skqs.php" target="_blank">《四库全书》下载</a><br>
</td>
</tr>
</table>

</div>
</div>


<div class="sidebar2">
<center>


</center>
</div>

<div class="sidebar"  style="margin-bottom:2px;">
<div class="sidebar_title">网站使用教程</div>
<div class="sidebar_info">
<a href="http://www.guoxuedashi.com/help/gjsearch.php" target="_blank">如何在国学大师网下载古籍?</a><br>
<a href="http://www.guoxuedashi.com/zidian/bujian/bjjc.php" target="_blank">如何使用部件查字法快速查字?</a><br>
<a href="http://www.guoxuedashi.com/search/sjc.php" target="_blank">如何在指定的书籍中全文检索?</a><br>
<a href="http://www.guoxuedashi.com/search/skjc.php" target="_blank">如何找到一句话在《四库全书》哪一页?</a><br>
</div>
</div>


<div class="sidebar">
<div class="sidebar_title">热门书籍</div>
<div class="sidebar_info">
<a href="/so.php?sokey=%E8%B5%84%E6%B2%BB%E9%80%9A%E9%89%B4&kt=1">资治通鉴</a> <a href="/24shi/"><strong>二十四史</strong></a>&nbsp; <a href="/a2694/">野史</a>&nbsp; <a href="/SiKuQuanShu/"><strong>四库全书</strong></a>&nbsp;<a href="http://www.guoxuedashi.com/SiKuQuanShu/fanti/">繁体</a>
<br><a href="/so.php?sokey=%E7%BA%A2%E6%A5%BC%E6%A2%A6&kt=1">红楼梦</a> <a href="/a/1858x/">三国演义</a> <a href="/a/1038k/">水浒传</a> <a href="/a/1046t/">西游记</a> <a href="/a/1914o/">封神演义</a>
<br>
<a href="http://www.guoxuedashi.com/so.php?sokeygx=%E4%B8%87%E6%9C%89%E6%96%87%E5%BA%93&submit=&kt=1">万有文库</a> <a href="/a/780t/">古文观止</a> <a href="/a/1024l/">文心雕龙</a> <a href="/a/1704n/">全唐诗</a> <a href="/a/1705h/">全宋词</a>
<br><a href="http://www.guoxuedashi.com/so.php?sokeygx=%E7%99%BE%E8%A1%B2%E6%9C%AC%E4%BA%8C%E5%8D%81%E5%9B%9B%E5%8F%B2&submit=&kt=1"><strong>百衲本二十四史</strong></a>  <a href="http://www.guoxuedashi.com/so.php?sokeygx=%E5%8F%A4%E4%BB%8A%E5%9B%BE%E4%B9%A6%E9%9B%86%E6%88%90&submit=&kt=1"><strong>古今图书集成</strong></a>
<br>

<a href="http://www.guoxuedashi.com/so.php?sokeygx=%E4%B8%9B%E4%B9%A6%E9%9B%86%E6%88%90&submit=&kt=1">丛书集成</a> 
<a href="http://www.guoxuedashi.com/so.php?sokeygx=%E5%9B%9B%E9%83%A8%E4%B8%9B%E5%88%8A&submit=&kt=1"><strong>四部丛刊</strong></a>  
<a href="http://www.guoxuedashi.com/so.php?sokeygx=%E8%AF%B4%E6%96%87%E8%A7%A3%E5%AD%97&submit=&kt=1">說文解字</a> <a href="http://www.guoxuedashi.com/so.php?sokeygx=%E5%85%A8%E4%B8%8A%E5%8F%A4&submit=&kt=1">三国六朝文</a>
<br><a href="http://www.guoxuedashi.com/so.php?sokeytm=%E6%97%A5%E6%9C%AC%E5%86%85%E9%98%81%E6%96%87%E5%BA%93&submit=&kt=1"><strong>日本内阁文库</strong></a> <a href="http://www.guoxuedashi.com/so.php?sokeytm=%E5%9B%BD%E5%9B%BE%E6%96%B9%E5%BF%97%E5%90%88%E9%9B%86&ka=100&submit=">国图方志合集</a> <a href="http://www.guoxuedashi.com/so.php?sokeytm=%E5%90%84%E5%9C%B0%E6%96%B9%E5%BF%97&submit=&kt=1"><strong>各地方志</strong></a>

</div>
</div>


<div class="sidebar2">
<center>

</center>
</div>
<div class="sidebar greenbar">
<div class="sidebar_title green">四库全书</div>
<div class="sidebar_info">

《四库全书》是中国古代最大的丛书,编撰于乾隆年间,由纪昀等360多位高官、学者编撰,3800多人抄写,费时十三年编成。丛书分经、史、子、集四部,故名四库。共有3500多种书,7.9万卷,3.6万册,约8亿字,基本上囊括了古代所有图书,故称“全书”。<a href="http://www.guoxuedashi.com/SiKuQuanShu/">详细>>
</a>

</div> 
</div>

</div>  <!--end r-->

</div>
<!-- 内容区END --> 

<!-- 页脚开始 -->
<div class="shh">

</div>

<div class="w1180" style="margin-top:8px;">
<center><script src="http://www.guoxuedashi.com/img/plus.php?id=3"></script></center>
</div>
<div class="w1180 foot">
<a href="/b/thanks.php">特别致谢</a> | <a href="javascript:window.external.AddFavorite(document.location.href,document.title);">收藏本站</a> | <a href="#">欢迎投稿</a> | <a href="http://www.guoxuedashi.com/forum/">意见建议</a> | <a href="http://www.guoxuemi.com/">国学迷</a> | <a href="http://www.shuowen.net/">说文网</a><script language="javascript" type="text/javascript" src="https://js.users.51.la/17753172.js"></script><br />
  Copyright &copy; 国学大师 古典图书集成 All Rights Reserved.<br>
  
  <span style="font-size:14px">免责声明:本站非营利性站点,以方便网友为主,仅供学习研究。<br>内容由热心网友提供和网上收集,不保留版权。若侵犯了您的权益,来信即刪。scp168@qq.com</span>
  <br />
ICP证:<a href="http://www.beian.miit.gov.cn/" target="_blank">鲁ICP备19060063号</a></div>
<!-- 页脚END --> 
<script src="http://www.guoxuedashi.com/img/plus.php?id=22"></script>
<script src="http://www.guoxuedashi.com/img/tongji.js"></script>

</body>
</html>
