資治通鑑卷一百二十
宋 司馬光 撰

胡三省 音註

宋紀二|{
	起閼逢困敦盡彊圉單閼凡四年}


太祖文皇帝上之上|{
	諱義隆小字車兒武帝第三子也}


元嘉元年|{
	是年八月始改元}
春正月|{
	考異曰宋本紀正月癸巳朔日有食之宋紀二月己巳宋畧二月癸巳李延夀南史二月己卯朔皆悞也按長歷是年正月丁巳二月丁亥朔後魏書紀紀志是年無日食今從之}
魏改元始光 丙寅魏安定殤王彌卒 營陽王居喪無禮好與左右狎暱|{
	好呼到翻下情好同}
遊戲無度特進致仕范泰上封事曰伏聞陛下時在後園頗習武備鼔鞞在宫聲聞于外|{
	鞞與鼙同音駢迷翻聞音問}
黷武掖庭之内諠譁省闥之間非徒不足以威四夷秖生遠近之怪陛下踐祚委政宰臣實同高宗諒闇之美|{
	闇讀如隂}
而更親狎小人懼非社稷至計經世之道也不聽泰甯之子也|{
	范甯汪之子以儒學為孝武帝所親}
南豫州刺史廬陵王義真警悟愛文義而性輕易|{
	易以豉翻}
與太子左衛率謝靈運員外常侍顔延之|{
	曹魏之末置員外散騎常侍率所律翻}
慧琳道人情好欵密嘗云得志之日以靈運延之為宰相慧琳為西豫州都督|{
	西豫州即豫州也宋南豫州治歷陽豫州治夀陽夀陽在歷陽西故亦謂豫州為西豫州}
靈運玄之孫也|{
	靈運玄子瑍之子也}
性褊傲不遵法度|{
	褊方緬翻}
朝廷但以文義處之|{
	處昌呂翻}
不以為有實用靈運自謂才能宜參權要常懷憤邑延之含之曾孫也|{
	顔含見九十六卷晉成帝咸康四年}
嗜酒放縱徐羨之等惡義真與靈運等遊義真故吏范晏從容戒之義真曰靈運空疎延之隘薄魏文帝所謂古今文人類不護細行者也|{
	惡烏路翻從千容翻行下孟翻}
但性情所得未能忘言於悟賞耳|{
	悟開覺也賞褒嘉也}
於是羨之等以為靈運延之構扇異同非毁執政出靈運為永嘉太守延之為始平太守|{
	守式又翻}
義真在歷陽多所求索執政每裁量不盡與|{
	裁剸節也量槩度也索山谷翻量音良}
義真深怨之數有不平之言|{
	數所角翻}
又表求還都諮議參軍廬江何尚之屢諫不聽時羨之等已密謀廢帝而次立者應在義真乃因義真與帝有隙先奏列其罪惡廢為庶人徙新安郡前吉陽令堂邑張約之上疏曰|{
	吉陽縣屬廬陵郡今吉州有吉水縣蓋吳立縣於吉水之陽因以為名也}
廬陵王少蒙先皇優慈之遇長受陛下睦愛之恩故在心必言所懷必亮|{
	亮信也明也導也言義真凡有所懷自信以為是必明而導之無所囬避也少詩照翻長知兩翻}
容犯臣子之道致招驕恣之愆|{
	言容有犯臣道之事以致招驕恣之罪}
至於天姿夙成實有卓然之美宜在容養錄善掩瑕訓盡義方進退以漸今猥加剝辱幽徙遠郡|{
	剝辱謂褫爵為庶人}
上傷陛下常棣之篤下令遠近恇然失圖|{
	恇去王翻}
臣伏思大宋開基造次|{
	造七到翻}
根條未繁宜廣樹藩戚敦睦以道人誰無過貴能自新以武皇之愛子陛下之懿弟豈可以其一眚長致淪棄哉書奏以約之為梁州府參軍尋殺之 夏四月甲辰魏主東巡大甯 秦王熾磐遣鎮南將軍吉毗等帥步騎一萬南伐白苟車孚崔提旁為四國皆降之|{
	白狗國至唐猶存蓋生羌也其地與東會州接車孚崔提旁為無所考帥讀曰率騎奇寄翻降戶江翻}
徐羨之等以南兖州刺史檀道濟|{
	沈約曰中原亂北州流民多南渡晉}


|{
	成帝立南兖州寄治京口文帝始割江淮間為境治廣陵}
先朝舊將威服殿省且有兵衆乃召道濟及江州刺史王弘入朝|{
	將即亮翻朝直遥翻}
五月皆至建康以廢立之謀告之甲申謝晦以領軍府屋敗悉令家人出外聚將士於府内又使中書舍人邢安泰潘盛為内應夜邀檀道濟同宿晦悚動不得眠道濟就寢便熟晦以此服之|{
	服其處六事而不變其常度也}
時帝於華林園為列肆親自沽賣又與左右引船為樂夕遊天淵池即龍舟而寢|{
	樂音洛魏氏作華林園天淵池於洛中晉氏南渡放其制作之於建康華林園在宫城北隅}
乙酉詰旦|{
	詰去吉翻}
道濟引兵居前羨之等繼其後入自雲龍門安泰等先誡宿衛莫有禦者帝未興軍士進殺二侍者傷帝指扶出東閣收璽綬|{
	璽斯氏翻綬音受}
羣臣拜辭衛送故太子宫侍中程道惠勸羨之等立皇弟南豫州刺史義恭羨之等以宜都王義隆素有令望又多符瑞|{
	景平初有黑龍見西方五色雲隨之二年江陵城上有紫雲望氣者皆以為帝王之符當在西方又江陵西至上明及江津其間有九十九洲楚諺云洲滿百當出王者時忽有一洲自生汙流逥薄而成皆為上龍飛之應}
乃稱皇太后令數帝過惡|{
	數所具翻}
廢為營陽王以宜都王纂承大統赦死罪以下又稱皇太后令奉還璽紱|{
	紱音弗}
并廢皇后為營陽王妃遷營陽王於吳使檀道濟入守朝堂|{
	朝直遥翻}
王至吳止金昌亭六月癸丑羨之等使邢安泰就弑之王多力突走出昌門|{
	金昌亭在昌門内孫權記注云閶門吳西郭門夫差作以天門通閶闔故名之後春申君改為昌門金昌亭以其在西門内故曰金昌}
追者以門關踣而弑之|{
	踣蒲北翻}


裴子野論曰古者人君養子能言而師授之辭能行而傅相之禮|{
	相息亮翻}
宋之教誨雅異於斯居中則任僕妾處外則近趨走|{
	處昌呂翻近其靳翻趨走執役者也}
太子皇子有帥有侍|{
	帥所類翻}
是二職者皆臺皁也|{
	左傳申無宇曰士臣皁僕臣臺}
制其行止授其法則導達臧否|{
	否音鄙}
罔弗由之言不及於禮義識不達於今古謹敇者能勸之以吝嗇狂愚者或誘之以凶慝|{
	誘音酉慝吐得翻}
雖有師傅多以耆艾大夫為之雖有友及文學多以膏粱年少為之具位而已亦弗與遊|{
	王置文學師友各一人晉制也禮五十曰艾服官政六十曰耆指使孔穎達曰艾者年至五十氣力已衰髮蒼白如艾也賀場曰耆至也至老之境也少詩照翻}
幼王臨州長史行事宣傳敎命|{
	行事行府州事也}
又有典籖往往專恣竊弄威權|{
	南史曰故事府州部内論事皆籖前直敍所論之事後云謹籖日月下又云某官某籖故府州置典籖以領之本五品吏宋初改為士職宋末多以幼少皇子為藩鎮時主以左右親近領典籖其權任遂重}
是以本根雖茂而端良甚寡嗣君冲幼世繼姦囬雖惡物醜類天然自出然習則生常其流遠矣降及太宗舉天下而棄之亦昵比之為也|{
	昵尼質翻比毗至翻}
嗚呼有國有家其鑑之矣|{
	裴子野究言宋氏亡國之禍通鑑載之於此欲使有國有家謹於其初也}


傅亮帥行臺百官奉法駕迎宜都王于江陵|{
	帥讀曰}
祠部尚書蔡廓|{
	晉氏渡江始有祠部尚書常與右僕射通職不常置以右僕射攝之若右僕射闕則祠部尚書攝知右事}
至尋陽遇疾不堪前亮與之别廓曰營陽在吳宜厚加供奉一旦不幸卿諸人有弑主之名欲立於世將可得邪時亮已與羨之議害營陽王乃馳信止之不及羨之大怒曰與人共計議如何旋背即賣惡於人邪|{
	旋背猶今人言轉背也背如字}
羨之等又遣使者殺前廬陵王義真於新安 |{
	考異曰宋南史本紀二月廢義真徙新安之下即云執政使使者誅義真于新安宋義真傳六月癸未羨之等遣使殺義真於徙所羨之傳亦云廢帝後殺義真於新安殺帝於吳縣按長歷六月庚寅朔無癸未蓋癸丑也}
羨之以荆州地重恐宜都王至或用别人乃亟以錄命除領軍將軍謝晦行都督荆湘等七州諸軍事荆州刺史|{
	錄命錄尚書自出命也}
欲令居外為援精兵舊將悉以配之|{
	羨之亮晦所以為身謀者如此而亦無救於廢弑之誅伊霍以至公血誠處之而師春所紀有異於書蓋不羨于伊尹霍光僅能保其身而不能保其族此天地之大變固人臣之所難居也將即亮翻}
秋七月行臺至江陵立行門於城南題曰大司馬門傅亮帥百僚詣門上表進璽紱儀物甚盛|{
	帥讀曰率上時掌翻璽斯氏翻紱音弗}
宜都王時年十八下教曰猥以不德謬降大命顧已兢悸何以克堪|{
	悸其季翻}
輒當暫歸朝廷展哀陵寢并與賢彦申寫所懷望體其心勿為辭費府州佐史並稱臣請題牓諸門一依宫省王皆不許教州府國綱紀宥所統内見刑原逋責|{
	州荆州府都督府國宜都國綱紀上佐掾屬也見賢遍翻逋欠也負也青如字又仄懈翻漢書高紀兩家折券棄責無音淮陽王傳張博負責仄懈翻}
諸將佐聞營陽廬陵王死皆以為疑勸王不可東下司馬王華曰先帝有大功於天下四海所服雖嗣主不綱人望未改徐羨之中才寒士傳亮布衣諸生非有晉宣帝王大將軍之心明矣|{
	王大將軍敦也}
受寄崇重未容遽敢背德畏廬陵嚴斷|{
	背蒲妹翻斷丁亂翻}
將來必不自容以殿下寛叡慈仁遠近所知且越次奉迎冀以見德|{
	冀以定策為德也}
悠悠之論殆必不然又羨之等五人同功竝位孰肯相讓|{
	五人謂徐羨之傅亮謝晦檀道濟王弘也}
就懷不軌勢必不行廢主若存慮其將來受禍致此殺害蓋由貪生過深寧敢一朝頓懷逆志不過欲握權自固以少主仰待耳殿下但當長驅六轡以副天人之心王曰卿復欲為宋昌邪|{
	少詩照翻復扶又翻宋昌事見十三卷漢高后八年}
長史王曇首南蠻校尉到彦之皆勸王行|{
	姓譜到本自高陽氏楚令尹屈到之後後漢有東平太守到質曇徒含翻}
曇首仍陳天人符應王乃曰諸公受遺不容背義|{
	背滿妹翻}
且勞臣舊將内外充滿今兵力又足以制物夫何所疑乃命王華摠後任留鎮荆州王欲使到彦之將兵前驅彦之曰了彼不反|{
	了決知也將即亮翻}
便應朝服順流|{
	朝直遥翻}
若使有虞此師既不足恃更開嫌隙之端非所以副遠邇之望也|{
	彦之此言誠合天理而亦自知其才不足以制檀道濟也}
會雍州刺史褚叔度卒|{
	雍於用翻}
乃遣彦之權鎮襄陽甲戌王江陵引見傅亮號泣哀動左右|{
	見賢遍翻號戶高翻}
既而問義真及少帝薨廢本末|{
	少詩照翻}
悲哭嗚咽侍側者莫能仰視亮流汗沾背不能對乃布腹心於到彦之王華等深自結納王以府州文武嚴兵自衛臺所遣百官衆力不得近部伍|{
	近其靳翻}
中兵參軍朱容子抱刀處王所乘舟戶外不解帶者累旬|{
	防非常也處昌呂翻}
魏主還宫 秦王熾磐遣太子暮末帥征北將軍木弈干等步騎三萬出貂渠谷攻河西白草嶺臨松郡皆破之|{
	水經註西平鮮谷塞東南有白草嶺帥讀曰率騎奇寄翻}
徙民二萬餘口而還|{
	還從宣翻}
八月丙申宜都王至建康羣臣迎拜於新亭徐羨之問傅亮曰王可方誰亮曰晉文景以上人羨之曰必能明我赤心亮曰不然|{
	亮固知其不得免矣}
丁酉王謁初寧陵還止中堂|{
	晉孝武以太學在秦淮南去臺城懸遠權以中堂為太學親釋奠於先聖則中堂亦在秦淮北但在臺城之外耳}
百官奉璽綬|{
	綬音受}
王辭讓數四乃受之即皇帝位于中堂備法駕入宫御太極前殿大赦改元文武賜位二等戊戌謁太廟詔復廬陵王先封迎其柩及孫脩華謝妃還建康|{
	孫脩華義真之母謝義真之妃也南史云晉武帝采漢魏之制以淑妃淑媛淑儀脩華脩容脩儀婕妤容華充華是為九嬪位視九卿李延夀曰貴嬪魏文帝所制夫人魏武初建魏國所制貴人漢光武所制晉為三夫人位視三公淑妃魏明帝所制淑媛魏文帝所制淑儀脩華晉武帝所制脩容魏文帝所制脩儀魏明帝所制婕妤容華前漢舊號充華晉武帝所制美人漢光武所制}
庚子以行荆州刺史謝晦為真晦將行與蔡廓别屛人問曰|{
	屏必郢翻}
吾其免乎廓曰卿受先帝顧命任以社稷廢昏立明義無不可但殺人二兄而以之北面挾震主之威據上流之重以古推今自免為難|{
	蔡廓父子以亮直名于宋朝觀其抗言無所避就若不足以保身而卒能以身名終何也蓋其素行已孚乎人而言事無所依違又所以遂其直彼其問者方怵于利害就以求决則聽之也固合于心而焉敢以為諱乎}
晦始懼不得去既顧望石頭城喜曰今得脱矣癸卯徐羨之進位司徒王弘進位司空傅亮加開府儀同三司謝晦進號衛將軍檀道濟進號征北將軍|{
	謝晦自領軍將軍進號檀道濟自鎮北將軍進號}
有司奏車駕依故事臨華林園聽訟詔曰政刑多所未悉可如先者二公推訊|{
	二公謂徐羨之王弘}
帝以王曇首王華為侍中曇首領右衛將軍華領驍騎將軍朱容子為右衛將軍|{
	魏明帝有左軍將軍晉武帝置前軍右軍又置後軍是為四軍驍騎將軍游撃將軍並漢雜號將軍也魏置為中軍及晉以領護左右衛驍騎游撃為六軍驍堅堯翻騎奇寄翻}
甲辰追尊帝母胡婕妤曰章皇后|{
	婕妤音接予諡法敬慎高明曰章}


封皇弟義恭為江夏王|{
	夏戶雅翻}
義宣為竟陵王義季為衡陽王仍以義宣為左將軍鎮石頭徐羨之等欲即以到彦之為雍州|{
	雍於用翻}
帝不許徵彦之為中領軍委以戎政彦之自襄陽南下謝晦已至鎮慮彦之不過已|{
	過古禾翻}
彦之至楊口步往江陵深布誠欵晦亦厚自結納彦之留馬及利劍名刀以與晦晦由此大安 柔然紇升蓋可汗聞魏太宗殂將六萬騎入雲中殺掠吏民攻拔盛樂宫|{
	魏之先什翼犍始居雲中之盛樂宫築盛樂城於故城南八里紇戶骨翻可從刋入聲汗音寒將即亮翻騎奇寄翻樂音洛}
魏世祖自將輕騎討之三日二夜至雲中紇升蓋引騎圍魏主五十餘重騎逼馬首相次如堵將士大懼魏主顔色自若衆情乃安紇升蓋以弟子於陟斤為大將魏人射殺之紇升蓋懼遁去|{
	重直龍翻射而亦翻 考異曰後魏本紀云赭楊子尉普丈率輕騎討之虜乃退走李延夀北史紀云帝帥輕騎討之虜乃退走今據蠕蠕傳從北史}
尚書令劉絜言於魏主曰大檀自恃其衆必將復來|{
	復扶又翻}
請俟收田畢大兵為二道東西並進以討之魏主然之 九月丙子立妃袁氏為皇后耽之曾孫也|{
	袁耽見九十五卷成帝咸康元年}
冬十月吐谷渾威王阿柴卒阿柴有子二十人疾病召諸子弟謂之曰先公車騎以大業之故捨其子拾䖍而授孤孤敢私於緯代而忘先君之志乎|{
	先公謂樹洛干也樹洛干自號車騎將軍授阿柴國見一百一十八卷晉安帝義熙十三年緯于貴翻}
我死汝曹當奉慕璝為主緯代者阿柴之長子慕璝者阿柴之母弟叔父烏紇提之子也|{
	烏紇提之立也妻樹洛干母生二子慕璝慕利延緯於貴翻璝姑回翻提當作堤}
阿柴又命諸子各獻一箭取一箭授其弟慕利延使折之慕利延折之又取十九箭使折之慕利延不能折|{
	折而設翻}
阿柴乃諭之曰汝曹知之乎孤則易折|{
	易以豉翻}
衆則難摧汝曹當勠力一心然後可以保國寧家言終而卒慕璝亦有才畧撫秦涼失業之民及氐羌雜種至五六百落部衆轉盛|{
	種章勇翻}
十二月魏主命安集將軍長孫翰安北將軍尉眷北撃柔然魏主自將屯柞山|{
	柞山在平城之西大河之東柞則洛翻}
柔然北遁諸軍追之大獲而還翰肥之子也|{
	長孫肥事魏主珪為將數有功}
詔拜營陽王母張氏為營陽太妃 林邑王范陽邁寇日南九德諸郡|{
	沈約曰九德郡故屬九真孫吳分立九德郡隋唐為驩州}
宕昌王梁彌怱遣子彌黄入見于魏|{
	宕徒浪翻見賢遍翻}
宕昌羌之别種也羌地東接中國西通西域長數千里|{
	長直亮翻}
各有酋帥部落分地不相統攝而宕昌最疆有民二萬餘落諸種畏之|{
	北史曰宕昌蓋三苖之胤杜佑曰其界自仇池以西東西千里席水以南南北八百里地多山阜宕徒浪翻酋慈由翻帥所類翻種章勇翻}
夏主將廢太子璝而立少子酒泉公倫璝聞之將兵七萬北伐倫|{
	璝錄南臺自長安北伐倫璝姑囬翻少詩照翻將即亮翻下同}
倫將騎三萬拒之戰于高平倫敗死倫兄太原公昌將騎一萬襲璝殺之并其衆八萬五千歸于統萬夏主大悦立昌為太子夏主好自矜大|{
	好呼到翻}
名其四門東曰招魏南曰朝宋西曰服涼北曰平朔|{
	朝直遥翻}


二年春正月徐羨之傅亮上表歸政表三上|{
	上時掌翻}
帝乃許之丙寅始親萬機羨之仍遜位還第徐珮之程道惠及吳興太守王韶之等並謂非宜|{
	守式又翻}
敦勸甚苦乃復奉詔視事|{
	速徐傅之死者珮之諸公也復扶又翻}
辛未帝祀南郊大赦己卯魏主還平城 二月燕有女子化為男燕王以

問羣臣尚書左丞傅權對曰西漢之末雌雞化為雄猶有王莽之禍|{
	漢書五行志宣帝黄龍元年未央殿輅軨中雌雞化為雄毛衣變化而不鳴不將無距元帝初元丞相府史家雌雞伏子漸化為雄冠距鳴將其後王后羣弟世權以至於莽遂篡天下}
况今女化為男臣將為君之兆也 三月丙寅魏主尊保母竇氏為保太后密后之殂也|{
	密后即魏主之母杜貴嬪}
世祖尚幼太宗以竇氏慈良有操行|{
	行下孟翻}
使保養之竇氏撫視有恩訓導有禮世祖德之故加以尊號奉養不異所生|{
	養羊亮翻}
丁巳魏以長孫嵩為太尉長孫翰為司徒奚斤為司

空 夏四月秦王熾磐遣平遠將軍叱盧犍等襲河西鎮南將軍沮渠白蹄於臨松擒之徙其民五千餘戶於枹罕|{
	犍居言翻沮子余翻枹音膚}
魏主遣龍驤將軍步堆等來聘始復通好|{
	魏書官氏志西方步鹿孤氏改為步氏驤思將翻復扶又翻好呼到翻}
六月武都惠文王楊盛卒初盛聞晉亡不改義熙年號謂世子玄曰吾老矣當終為晉臣汝善事宋帝及盛卒|{
	辛子恤翻}
玄自稱都督隴右諸軍事征西大將軍開府儀同三司秦州刺史武都王遣使來告喪始用元嘉年號|{
	使疏吏翻}
秋七月秦王熾磐遣鎮南將軍吉毗等南擊黑水羌酋丘擔大破之|{
	黑水羌在鄧至西北水經註曰白水出臨洮縣西南西傾山東南流與黑水合黑水出羌中西南逕黑水城西又西南入于白水擔都甘翻}
八月夏武烈帝殂葬嘉平陵廟號世祖太子昌即皇帝位|{
	昌字還國勃勃第二子也}
大赦改元承光 王弘自以始不預定策不受司空表讓彌年乃許之|{
	弘以此得免徐傳之禍}
乙酉以弘為車騎大將軍開府儀同三司|{
	騎奇寄翻}
冬十月丘擔以其衆降秦|{
	降戶江翻}
秦以擔為歸善將軍拜折衝將軍乞伏信帝為平羌校尉以鎮之 癸卯魏主大伐柔然五道並進長孫翰等從東道出黑漠 |{
	考異曰翰傳云與娥清出長川今從蠕蠕傳}
廷尉卿長孫道生等出白黑二漠之間|{
	長川牛川同是大漠之地拓拔分其地名耳長川有白黑二漠黑在東白在西}
魏主從中道東平公娥清出栗園|{
	栗園在中道之西西道之東 考異曰清傳云與長孫翰出長川今從蠕蠕傳}
奚斤等從西道出爾寒山諸軍至漠南舍輜重|{
	舍讀曰捨重直用翻}
輕騎齎十五日糧度漠擊之|{
	騎奇寄翻}
柔然部落大驚絶迹北走 十一月以武都世子玄為北秦州刺史武都王|{
	時南秦州治漢中故以武都為北秦州 考異曰宋本紀癸酉南史庚午按十一月壬午朔無癸酉及庚午今不書日}
初會稽孔甯子為帝鎮西諮議參軍|{
	會工外翻}
及即位以甯子為步兵校尉與侍中王華並有富貴之願疾徐羨之傅亮專權日夜搆之於帝|{
	史言徐傅偪上固當誅而王華等之構間亦非也}
會謝晦二女當適彭城王義康新野侯義賓遣其妻曹氏及長子世休送女至建康帝欲誅羨之亮并發兵討晦聲言當伐魏又言拜京陵|{
	京陵興寧陵也}
治行裝艦|{
	治直之翻艦戶黯翻}
亮與晦書曰薄伐河朔事猶未已朝野之慮憂懼者多又言朝士多諫北征|{
	朝直遥翻}
上當遣外監萬幼宗往相諮訪|{
	南史曰制局監外監領器仗兵役多以嬖倖為之}
時朝廷處分異常其謀頗泄|{
	處昌呂翻分扶問翻}


三年春正月謝晦弟黄門侍郎㬭馳使告晦|{
	㬭子肖翻}
晦猶謂不然以傅亮書示諮議參軍何承天曰計幼宗一二日必至傳公慮我好事故先遣此書承天曰外間所聞咸謂西討已定幼宗豈有上理|{
	好事猶言好生事微省其辭苦隱語然好呼到翻上時掌翻}
晦尚謂虚妄使承天豫立答詔啟草言伐虜宜須明年江夏内史程道惠得尋陽人書|{
	夏戶雅翻下同}
言朝廷將有大處分其事已審使其輔國府中兵參軍樂冏封以示晦|{
	道惠蓋帶輔國將軍也}
晦問承天曰若果爾卿令我云何對曰蒙將軍殊顧常思報德事變至矣何敢隱情然明日戒嚴動用軍法區區所懷懼不得盡晦懼曰卿豈欲我自裁邪承天曰尚未至此以王者之衆舉天下以攻一州大小既殊逆順又異境外求全上計也其次以腹心將兵屯義陽將軍自帥大衆戰於夏口若敗即趨義陽以出北境其次也|{
	承天二策皆勸晦奔魏以求全將即亮翻帥讀曰率趨七喻翻}
晦良久曰荆州用武之地兵糧易給聊且決戰走復何晩|{
	易以豉翻復扶又翻}
乃使承天造立表檄又與衛軍諮議參軍琅邪顔邵謀舉兵|{
	晦帶衛將軍}
邵飲藥而死晦立幡戒嚴謂司馬庾登之曰今當自下欲屈卿以三千人守城備禦劉粹登之曰下官親老在都又素無部衆情計二三不敢受此旨晦仍問諸將佐戰士三千足守城否南蠻司馬周超對曰非徒守城而已若有外寇可以立功|{
	超盖為南蠻校尉府司馬}
登之因曰超必能辦下官請解司馬南郡以授之|{
	登之晦府司馬領南郡太守乞解以授超}
晦即於坐命超為司馬領南義陽太守|{
	沈約曰晉末以義陽流民僑立南義陽郡屬荆州領厥西平氏二縣坐徂卧翻}
轉登之為長史南郡如故登之蘊之孫也|{
	庾藴死于海西之廢}
帝以王弘檀道濟始不預廢弑之謀弘第曇首又為帝所親委事將發密使報弘且召道濟欲使討晦王華等皆以為不可帝曰道濟止於脅從本非創謀殺害之事又所不關吾撫而使之必將無慮乙丑道濟至建康丙寅下詔暴羨之亮晦殺營陽廬陵王之罪命有司誅之且曰晦據有上流或不即罪|{
	即就也}
朕當親帥六師為其過防|{
	帥讀曰率}
可遣中領軍到彦之即日電發征北將軍檀道濟駱驛繼路符衛軍府州以時收翦|{
	符衛軍府及荆州官屬使收誅晦也}
已命雍州刺史劉粹等|{
	雍於用翻}
斷其走伏|{
	走逃也代匿也斷其逃匿之路也斷丁管翻}
罪止元凶餘無所問是日詔召羨之亮羨之行至西明門外|{
	洛城西面有廣陽西明閶闔三門建康倣之}
謝㬭正直|{
	㬭為黄門侍郎正入直省内也}
遣報亮云殿内有異處分亮辭以嫂病暫還遣使報羨之羨之還西州|{
	揚州刺史治臺城西故曰西州}
乘内人問訊車出郭步走至新林|{
	新林浦去建康城二十里}
入陶竈中自經死亮乘車出郭門乘馬奔兄迪墓屯騎校尉郭泓收之至廣莫門|{
	廣莫門建康城北門亦放洛城之制騎奇寄翻校戶教翻}
上遣中書舍人以詔書示亮幷謂曰以公江陵之誠|{
	謂亮迎帝於江陵也}
當使諸子無恙亮讀詔書訖曰亮受先帝布衣之眷遂蒙顧託黜昏立明社稷之計也欲加之罪其無辭乎|{
	晉大夫里克之言}
於是誅亮而徙其妻子於建安誅羨之二子而宥其兄子珮之又誅晦子世休收繋謝㬭帝將討謝晦問策於檀道濟對曰臣昔與晦同從北征|{
	事見一百一十八卷晉安帝義熙十三年也}
入關十策晦有其九才畧明練殆為少敵然未嘗孤軍決勝戎事恐非其長|{
	其下當有所字}
臣悉晦智晦悉臣勇今奉王命以討之可未陳而擒也|{
	陳與陣同}
丁卯徵王弘為侍中司徒錄尚書事揚州刺史以彭城王義康為都督荆湘等八州諸軍事荆州刺史樂冏復遣使告謝晦以徐傅及㬭等已誅|{
	復扶又翻下磐復同使疏吏翻}
晦先舉羨之亮哀次發子弟凶問既而自出射堂勒兵晦從高祖征討指麾處分莫不曲盡其宜數日間四遠投集得精兵三萬人乃奉表稱羨之亮等忠貞横被寃酷|{
	横戶孟翻}
且言臣等若志欲執權不專為國|{
	為于偽翻}
初廢營陽陛下在遠武皇之子尚有童幼擁以號令誰敢非之豈得泝流三千里|{
	自建康至江陵泝流而上凡三千里}
虚館七旬仰望鸞旗者哉|{
	景平二年五月乙酉廢少帝八月丙申帝至建康凡七旬}
故廬陵王於營陽之世積怨犯上自貽非命不有所廢將何以興|{
	亦晉里克之言}
耿弇不以賊遺君父|{
	此耿弇討張步之言晦引以為言自謂殺廬陵所以除偪不以累帝也遺于季翻}
臣亦何負於宋室邪此皆王弘王曇首王華險躁猜忌讒搆成禍|{
	躁則到翻}
今當舉兵以除君側之惡 秦王熾磐復遣使如魏請用師于夏|{
	秦入貢于魏以請伐夏事始上卷營陽王景平元年復扶又翻使疏吏翻}
初袁皇后生皇子劭后自詳視使馳白帝曰此兒形貌異常必破國亡家不可舉即欲殺之帝狼狽至后殿戶外手撥幔禁之乃止|{
	為劭弑逆張本幔莫半翻}
以尚在諒闇|{
	闇音隂}
故祕之閏月丙戌始言劭生 帝下詔戒嚴大赦諸軍相次進路以討謝晦晦以弟遯為竟陵内史將萬人摠留任帥衆二萬發江陵列舟艦自江津至于破冢|{
	將即亮翻帥讀曰率艦戶黯翻冢知隴翻}
旌旗蔽日歎曰恨不得以此為勤王之師晦欲遣兵襲湘州刺史張邵何承天以邵兄益州刺史茂度與晦善曰邵意趣未可知不宜遽擊之晦以書招邵邵不從 二月戊午以金紫光祿大夫王敬弘為尚書左僕射|{
	晉制左右光祿大夫金章紫綬後遂為金紫光禄大夫}
建安太守鄭鮮之為右僕射敬弘廙之曾孫也|{
	王廙見八十九卷晉愍帝建興三年廙羊至翻又逸職翻}
庚申上建康命王弘與彭城王義康居守入居中書下省|{
	中書有上省下省守手又翻}
侍中殷景仁參掌留任帝姊會稽長公主留止臺内摠攝六宫|{
	臺内郎禁中會工外翻長知兩翻}
謝晦自江陵東下何承天留府不從晦至江口|{
	江口即西江口從才用翻}
到彦之已至彭城洲庾登之據巴陵畏懦不敢進會霖雨連日參軍劉和之曰彼此共有雨耳檀征北尋至東軍方彊唯宜速戰登之恇怯使小將陳祐作大囊貯茅懸於帆檣云可以焚艦|{
	恇去王翻將即亮翻貯丁呂翻艦戶黯翻}
用火宜須晴以緩戰期晦然之停軍十五日乃使中兵參軍孔延秀攻將軍蕭欣於彭城洲破之|{
	水經注江水過長沙下雋縣北又東逕彭城口水東冇彭城磯}
又攻洲口栅陷之諸將咸欲退還夏口到彦之不可乃保隱圻|{
	水經註江水自彭城磯東逕如山北山北對隱磯}
晦又上表自訟且自矜其捷曰陛下若梟四凶於廟庭懸三監於絳闕|{
	以王弘王曇首王華比虞之共工驩兜苗鯀周之管叔蔡叔霍叔也梟堅堯翻}
臣便勒衆旋旗還保所任初晦與徐羨之傅亮為自全之計以為晦據上流而檀道濟鎮廣陵各有彊兵足以制朝廷羨之亮居中秉權可得持久及聞道濟帥衆來上惶懼無計道濟既至與到彦之軍合牽艦緣岸晦始見艦數不多輕之不即出戰至晩因風帆上前後連咽|{
	上時掌翻連謂沿江戰艦連接不斷咽謂戰艦塞江前後填咽}
西人離沮無復鬬心戊辰臺軍至忌置洲尾|{
	沮在呂翻水經註江水東過長沙下雋縣北湘水自南注之又東左得二夏浦俗謂之西江口又東逕忌置山南又東過彭城口復扶又翻}
列艦過江晦軍一時皆潰晦夜出投巴陵得小船還江陵先是帝遣雍州刺史劉粹自陸道帥步騎襲江陵至沙橋|{
	先悉薦翻雍於用翻帥讀曰率下同騎奇寄翻沙橋在江陵北}
周超帥萬餘人逆戰大破之士卒傷死者過半俄而晦敗問至初晦與粹善以粹子曠之為參軍帝疑之王弘曰粹無私必無憂也及受命南討一無所顧帝以此嘉之晦亦不殺曠之遣還粹所丙子帝自蕪湖東還晦至江陵無他處分唯愧謝周超而已其夜超捨軍單舸詣到彦之降|{
	舸古我翻}
晦衆散畧盡乃攜其弟遯等七騎北走遯肥壯不能乘馬晦每待之行不得速己卯至安陸延頭|{
	水經註武口水上通安陸之延頭九域志武湖在黄洲界蓋此湖上接延頭也杜佑曰武湖在黄洲黄陂縣東黄祖習戰閲武之所}
為戍主光順之所執|{
	戍主戍副宋齊以下至隋咸有其官姓譜光姓也晉書有光逸}
檻送建康到彦之至馬頭何承天自歸彦之因監荆州府事|{
	監工銜翻}
以周超為參軍劉粹以沙橋之敗告乃執之於是誅晦㬭遯及其兄弟之子并同黨孔延秀周超等晦女彭城王妃被髪徒跣與晦訣曰大丈夫當横尸戰場奈何狼籍都市|{
	被皮義翻籍而亦翻}
庾登之以無任免官禁錮何承天及南蠻行參軍新興王玄謨等皆見原|{
	據南史王玄謨太原祁縣人漢建安二十年集塞下荒地置新興郡魏黄初初遷于陘嶺之南玄謨蓋本新興人而居太原之祁縣界也}
晦之走也左右皆棄之唯延陵蓋追隨不捨帝以蓋為鎮軍功曹督護|{
	延陵複姓蓋吳延陵季子之後蓋其名也為鎮軍府功曹又兼督護之官也晉氏渡江有參軍督護功曹參軍兼督護即參軍督護之任也洪适曰參軍督護江左置皆有部曲宋則無矣}
晦之起兵引魏南蠻校尉王慧龍為援|{
	魏以王慧龍為南蠻校尉以擾汝潁之間}
慧龍帥衆一萬拔思陵戍|{
	思陵戍在陳郡西北}
進圍項城聞晦敗乃退益州刺史張茂度受詔襲江陵晦敗茂度軍始至白帝議者疑茂度有二心帝以茂度弟邵有誠節赦不問代還三月辛巳帝還建康徵謝靈運為祕書監顔延之為中書侍郎賞遇甚厚帝以惠琳道人善談論因與議朝廷大事遂參權要賓客輻湊門車常有數十兩|{
	門車謂門前候見之車兩音亮}
四方贈賂相係方筵七八座上恒滿琳著高屐|{
	著陟畧翻}
披貂裘置通呈書佐|{
	通呈典謁之職書佐掌書翰}
會稽孔覬嘗詣之遇賓客填咽暄涼而已|{
	言但叙寒温不及餘語會工外翻覬音冀}
覬慨然曰遂有黑衣宰相可謂冠屨失所矣|{
	廬陵廢而三人斥徐傅誅而三人進可謂矯枉過正矣}
夏五月乙未以檀道濟為征南大將軍開府儀同三司江州刺史到彦之為南豫州刺史遣散騎常侍袁渝等十六人分行諸州郡縣觀察吏政訪求民隱|{
	散悉亶翻騎奇寄翻行下孟翻}
又使郡縣各言損益丙午上臨延賢堂聽訟|{
	延賢堂在建康華林園}
自是每歲三訊|{
	周禮秋官以三刺斷庶民獄訟之中一曰訊羣臣二曰訊羣吏三曰訊萬民註云刺殺也三訊罪定則殺之訊言也}
左僕射王敬弘性恬淡有重名關署文案初不省讀|{
	省悉景翻}
嘗預聽訟上問以疑獄敬弘不對上變色問左右何故不以訊牒副僕射|{
	謂不以訊牒副本納呈敬弘也}
敬弘曰臣乃得訊牒讀之正自不解|{
	解戶買翻曉也}
上甚不悦雖加禮敬不復以時務及之|{
	復扶又翻禮敬不替而不以時務及之此法正勸蜀先主以加禮許靖之智也}
六月以右衛將軍王華為中護軍侍中如故華以王弘輔政王曇首為上所親任與已相埓|{
	埒龍輟翻}
自謂力用不盡每歎息曰宰相頓有數人天下何由得治|{
	治直吏翻}
是時宰相無常官唯人主所與議論政事委以機密者皆宰相也故華有是言亦有任侍中而不為宰相者然尚書令僕中書監令侍中侍郎給事中皆當時要官也華與劉湛王曇首殷景仁俱為侍中風力局幹冠冕一時上嘗與四人於合殿宴飲甚悦|{
	合殿在齋閣之後李延夀曰晉世諸帝多處内房朝宴所臨東西二堂而已孝武末年清暑方構永初受命無所改作所居惟稱西殿不製嘉名文帝因之亦有合殿之稱}
既罷出上目送良久歎曰此四賢一時之秀同管喉脣恐後世難繼也|{
	喉脣言出納王命也}
黄門侍郎謝弘微與華等皆上所重當時號曰五臣弘微琰之從孫也|{
	琰安之子也從才用翻下同}
精神端審時然後言婢僕之前不妄語笑由是尊卑大小敬之若神從叔混特重之常曰微子異不傷物同不害正吾無間然|{
	呂大臨曰無間隙可言其失謝顯道曰猶言我無得而議之也嗚呼此江左所謂清談也問古莧翻}
上欲封王曇首王華等拊御牀曰此坐非卿兄弟無復今日因出封詔以示之|{
	以誅徐傅等為曇首華之功坐徂卧翻}
曇首固辭曰近日之事賴陛下英明罪人斯得臣等豈可因國之災以為身幸上乃止 魏主詔問公卿今當用兵赫連蠕蠕二國何先|{
	杜佑曰柔然後魏大武以其無知狀類於蟲故改其號曰蠕蠕宋齊謂之芮芮蠕人兖翻}
長孫嵩長孫翰奚斤皆曰赫連土著|{
	著直畧翻}
未能為患不如先伐蠕蠕若追而及之可以大獲不及則獵於隂山取其禽獸皮角以充軍實太常崔浩曰蠕蠕鳥集獸逃|{
	言其來則如鳥之集走則如獸之逃也}
舉大衆追之則不能及輕兵追之又不足以制敵赫連氏土地不過千里政刑殘虐人神所棄宜先伐之尚書劉絜武京侯安原請先伐燕於是魏主自雲中西巡至五原因畋於隂山東至和兜山|{
	和兜山蓋在隂山之東長川之南}
秋八月還平城 詔殿中將軍吉恒聘于魏|{
	恒戶登翻}
燕太子永卒立次子翼為太子 秦王熾磐伐河西至廉川遣太子暮末等步騎三萬攻西安不克又攻番禾河西王蒙遜兵禦之且遣使說夏主|{
	騎奇寄翻番音磐使疏吏翻說輸芮翻}
使乘虛襲枹罕|{
	枹音膚}
夏主遣征南大將軍呼盧古將騎二萬攻苑川車騎大將軍韋伐將騎三萬攻南安熾磐聞之引歸|{
	蒙遜借助於夏以退秦師秦既敝於夏夏亦僨於魏而涼亦不能以自立是以親仁善鄰國之寶也}
九月徙其境内老弱畜產於澆河|{
	杜佑曰澆河城在廓州達化縣西一百二十里}
及莫河仍寒川留左丞相曇達守枹罕韋伐攻抜南安獲秦秦州刺史翟爽南安太守李亮 吐谷渾握逵等帥部衆二萬落叛秦奔昂川附於吐谷渾王慕璝|{
	史言乞伏兵勢漸衰帥讀曰率璝姑回翻}
大旱蝗左光祿大夫范泰上表曰婦人有三從之義

|{
	婦人在室從父母既嫁從夫夫死從子}
無自專之道謝晦婦女猶在尚方唯陛下留意有詔原之 魏主聞夏世祖殂諸子相圖|{
	謂倫璝昌相殺也}
國人不安欲伐之長孫嵩等皆曰彼若城守以逸待勞大檀聞之乘虚入寇此危道也崔浩曰往年以來熒惑再守羽林鉤已而行其占秦亡|{
	事見一百十七卷晉安帝義熙十一年}
今年五星并出東方利以西伐天人相應不可失也嵩固争之帝大怒責嵩在官貪汚命武士頓辱之|{
	嵩歷事四朝魏之元臣也頓辱捽其首使頓地以辱之}
於是遣司空奚斤帥四萬五千人襲蒲阪|{
	阪音反}
宋兵將軍周幾帥萬人襲陜城|{
	陜失冉翻}
以河東太守薛謹為鄉導謹辯之子也|{
	薛辯見一百十八卷晉安帝義熙十三年鄉讀曰嚮}
魏主欲以中書博士平棘李順摠前驅之兵|{
	平棘縣二漢屬常山晉魏屬趙郡}
訪於崔浩浩曰順誠有籌畧然臣與之婚姻深知其為人果於去就不可專委帝乃止浩與順由是有隙|{
	為後魏主以浩言誅順張本}
冬十月丁巳魏主平城 秦左丞相曇達與夏呼盧古戰於嵻㟍山|{
	曇徒含翻嵻丘岡翻㟍盧當翻}
曇達兵敗十一月呼盧古韋伐進攻枹罕|{
	枹音膚}
秦王熾磐遷保定連呼盧古入南城|{
	南城枹罕南城}
鎮京將軍趙夀生率死士三百人力戰却之呼盧古韋伐又攻沙州刺史出連䖍于湟河䖍遣後將軍乞伏萬年擊敗之|{
	敗補邁翻}
又攻西平執安西將軍庫洛干阬戰士五千餘人掠民二萬餘戶而去 仇池氐楊興平求内附梁南秦二州刺史吉翰|{
	晉泰始之初立梁州於漢中至安帝之世秦州又治漢中自是鎮漢中者帶梁南秦二州刺史}
遣始平太守龎諮據武興|{
	武興漢武都郡之沮縣也蜀以其地當衝要置武興督以守之宋立東益州梁立武興蕃王國西魏改東益為興州因武興郡為名至我本朝以吳曦之變改為沔州}
氐王楊玄遣其弟難當將兵拒諮|{
	將即亮翻}
諮擊走之 魏主行至君子津會天暴寒冰合戊寅帥輕騎二萬濟河襲統萬|{
	帥讀曰率騎奇寄翻}
壬午冬至夏主方燕羣臣魏師奄至上下驚擾魏主軍於黑水去城三十餘里夏主出戰而敗退走入城門未及閉内三郎豆代田帥衆乘勝入西宫|{
	内三郎魏宿衛之官也三郎將領之又按魏道武帝天興初置幢將幢將主内三郎内三郎衛士也}
焚其西門宫門閉代田踰宫垣而出魏主拜代田勇武將軍|{
	勇武將軍之號魏始置}
魏軍夜宿城北癸未分兵四掠殺獲數萬得牛馬十餘萬魏主謂諸將曰|{
	將即亮翻下守將同}
統萬未可得也它年當與卿等取之乃徙其民萬餘家而還|{
	還從宣翻又如字}
夏弘農太守曹達聞周幾將至不戰而走魏師乘勝長驅遂入三輔會幾卒于軍中蒲阪守將東平公乙斗聞奚斤將至遣使詣統萬告急使者至統萬|{
	阪音反使疏吏翻下同}
魏軍已圍其城還告乙斗曰統萬已敗矣乙斗懼棄城西奔長安斤遂克蒲阪夏主之弟助興先守長安乙斗至與助興棄長安西犇安定 |{
	考異曰奚斤傳作乙升今從帝紀}
十二月斤入長安秦雍氐羌皆詣斤降|{
	雍於用翻降戶江翻}
河西王蒙遜及氐王楊玄聞之皆遣使附魏|{
	兵以氣勢為用者也統萬圍而諸鎮失守氣勢然也}
前吳郡太守徐珮之聚黨百餘人謀以明年正會於殿中作亂事覺收斬之|{
	正會明年正月朔旦朝會也亦曰元會}
營陽太妃張氏卒 秦征南將軍吉毗鎮南漒|{
	乞伏國仁置十二郡漒川其一也南漒當又在漒川之南漒其良翻}
隴西人辛澹帥戶三千據城逐毗毗走還枹罕澹南奔仇池|{
	澹徒覽翻帥讀曰率}
魏初得中原民多逃隱天興中詔采諸漏戶令輸繒帛|{
	魏皇始二年克中山始得中原晉安帝之隆安元年也明年改元天興繒慈陵翻}
於是自占為紬繭羅縠戶者甚衆|{
	占之贍翻紬除留翻穀戶谷翻}
不隸郡縣賦役不均是歲始詔一切罷之以屬郡縣

四年春正月辛巳帝祀南郊 乙酉魏主還平城統萬徙民在道多死能至平城者什纔六七己亥魏主如幽州夏主遣平原公定帥衆二萬向長安魏主聞之伐木隂山大造攻具再謀伐夏 山羌叛秦|{
	羌分居武始洮陽南山者曰山羌}
二月秦王熾磐遣左丞相曇達招慰武始諸羌征南將軍吉毗招慰洮陽諸羌|{
	晉惠帝置洮陽縣屬狄道郡曇徒含翻}
羌人執曇達送夏吉毗為羌所撃犇還士馬死傷者什八九魏主還平城 乙卯帝如丹徒己巳謁京陵初高祖既貴命藏微時耕具以示子孫帝至故宫|{
	晉之東遷也劉氏自彭城移居晉陵丹徒之京口里陵墓及故宮在焉}
見之有慙色近侍或進曰大舜躬耕歷山伯禹親事水土|{
	舜耕于歷山歷山之人皆讓畔伯禹親事水土手足胼胝}
陛下不覩遺物安知先帝之至德稼穡之艱難乎 二月丙子魏主遣高涼王禮鎮長安禮斤之孫也|{
	拓跋斤見一百四卷晉孝武太元元年}
又詔執金吾桓貸造橋於君子津 丁丑魏廣平王連卒 丁亥帝還建康 戊子尚書右僕射鄭鮮之卒 秦王熾磐以輔國將軍段暉為涼州刺史鎮樂都|{
	樂音洛}
平西將軍麴景為沙州刺史鎮西平寧朔將軍出連輔政為梁州刺史鎮赤水 夏四月丁未魏員外散騎常侍步堆等來聘|{
	曹魏之末置員外散騎常侍散悉亶翻騎奇寄翻}
庚戌以廷尉王徽之為交州刺史徵前刺史杜弘文弘文有疾自輿就路或勸之待病愈弘文曰吾杖節三世|{
	弘文父慧度祖瑗三世鎮交州}
常欲投軀帝庭况被徵乎遂行卒於廣州|{
	被皮義翻卒子恤翻}
弘文慧度之子也 魏奚斤與夏平原公定相持於長安魏主欲乘虚伐統萬簡兵練士部分諸將|{
	分扶問翻}
命司徒長孫翰等將三萬騎為前驅常山王素等將步兵三萬為後繼南陽王伏真等將步兵三萬部送攻具將軍賀多羅將精騎三千為前候|{
	前候者居前為候騎將即亮翻騎奇寄翻}
素遵之子也|{
	拓跋遵見一百八卷晉孝武太元二十年}
五月魏主平城命龍驤將軍代人陸俟督諸軍鎮大磧以備柔然|{
	魏書官氏志内入諸姓步六孤氏後改為陸氏驤用將翻磧七迹翻}
辛巳濟君子津 壬午中護軍王華卒 魏主至拔鄰山|{
	拔鄰山在黑水東北}
築城捨輜重|{
	北史捨作舍當從之讀如字重直用翻}
以輕騎三萬倍道先行羣臣咸諫曰統萬城堅非朝夕可拔今輕軍討之進不可克退無所資不若與步兵攻具一時俱往帝曰用兵之術攻城最下必不得已然後用之今以步兵攻具皆進彼必懼而堅守若攻不時拔食盡兵疲外無所掠進退無地不如以輕騎直扺其城彼見步兵未至意必寛弛吾羸形以誘之|{
	羸倫為翻誘音酉}
彼或出戰則成擒矣所以然者吾之軍士去家二千餘里又隔大河所謂置之死地而後生者也|{
	去國遠鬬人皆致死故其鋒不可當}
故以之攻城則不足決戰則有餘矣遂行 六月癸卯朔日有食之 魏主至統萬分軍伏於深谷以少衆至城下|{
	藏匿其衆以少衆至城下誘其出戰少詩沼翻}
夏將狄子玉降魏|{
	將即亮翻降戶江翻}
言夏主聞有魏師遣使召平原公定定曰統萬堅峻未易攻拔|{
	易以䜴翻}
待我擒奚斤然後徐往内外擊之蔑不濟矣故夏王堅守以待之魏主患之|{
	患其不出戰也}
乃退軍以示弱遣娥清及永昌王健帥騎五千西掠居民|{
	帥讀曰率騎奇寄翻下同}
魏軍士有得罪亡犇夏者言魏軍糧盡士卒食菜輜重在後步兵未至宜急擊之夏主從之|{
	使魏主用間亦不如是之巧殆天啟之也重直用翻}
甲辰將步騎三萬出城長孫翰等皆言夏兵步陳難陷|{
	陳讀曰陣}
宜避其鋒魏主曰吾遠來求賊惟恐不出今既出矣乃避而不撃彼奮我弱非計也遂收衆偽遁引而疲之夏兵為兩翼鼓譟追之行五六里會有風雨從東南來揚沙晦冥宦者趙倪頗曉方術言於魏主曰今風雨從賊上來我向之彼背之|{
	背蒲妹翻}
天不助人且將士饑渇願陛下攝騎避之|{
	攝收也}
更待後日崔浩叱之曰是何言也吾千里制勝一日之中豈得變易|{
	言先定必勝之計故千里行師不可以風雨之故變易成算於一日之間}
賊貪進不止後軍已絶宜隱軍分出掩撃不意風道在人豈有常也|{
	言風在人用之分兵出其後順風擊之則風為我用豈有常勢哉}
魏主曰善乃分騎為左右隊以掎之|{
	掎居蟻翻}
魏主馬蹶而墜幾為夏兵所獲拓跋齊以身捍蔽決死力戰夏兵乃退魏主騰馬得上刺夏尚書斛黎文殺之又殺騎兵十餘人身中流矢|{
	幾巨依翻上時掌翻刺七亦翻中竹仲翻斛黎虜複姓}
奮擊不輟夏衆大潰齊翳槐之玄孫也|{
	翳槐什翼犍之兄晉成帝咸和四年立}
魏人乘勝逐夏主至城北殺夏主之弟河南公滿及兄子蒙遜死者萬餘人夏主不及入城遂犇上邽魏主微服逐犇者入其城拓跋齊固諫不聽夏人覺之諸門悉閉魏主因與齊等入其宫中得婦人裙繋之槊上魏主乘之而上僅乃得免會日暮夏尚書僕射問至奉夏主之母出走|{
	問姓至名孫愐曰襄州有問姓}
長孫翰將八千騎追夏主至高平不及而還乙巳魏主入城獲夏王公卿將校及諸母后妃姊妹宫人以萬數馬三十餘萬匹牛羊數千萬頭府庫珍寶車旗器物不可勝計|{
	將即亮翻校戶教翻勝音升}
頒賜將士有差初夏世祖性豪侈|{
	夏主勃勃廟號世祖}
築統萬城|{
	事見一百十六卷晉安帝義熙九年}
高十仞基厚三十步上廣十步官牆高五仞|{
	高古號翻厚戶遘翻廣古曠翻}
其堅可以厲刀斧臺榭壯大皆雕鏤圖畫被以綺繡窮極文采魏主顧謂左右曰蕞爾國而用民如此|{
	被皮義翻蕞徂内翻}
欲不亡得乎得夏太史令張淵徐辯復以為太史令得故晉將毛脩之秦將軍庫洛干歸庫洛干於秦以毛脩之善烹調用為太官令|{
	毛脩之為夏所禽見一百十八卷晉安帝義熙十四年庫洛千被禽見上年}
魏主見夏著作郎天水趙逸所為文譽夏主太過|{
	譽音余}
怒曰此竪無道何敢如是誰所為邪當速推之|{
	欲案其罪也}
崔浩曰文士褒貶多過其實蓋非得已不足罪也乃止魏主納夏世祖三女為貴人奚斤與夏平原公定猶相拒於長安魏主命宗正娥清太僕丘堆帥騎五千畧地關右定聞統萬已破遂奔上邽斤追至雍不及而還|{
	雍於用翻}
清堆攻夏貳城拔之魏主詔斤等班師斤上言赫連昌亡保上邽鳩合餘燼未有蟠據之資今因其危滅之為易|{
	易以豉翻}
請益鎧馬平昌而還|{
	鎧可亥翻}
魏主不許斤固請乃許之給斤兵萬人遣將軍劉拔送馬三千匹并留娥清丘堆使共擊夏辛酉魏主自統萬東還以常山王素為征南大將軍假節與執金吾桓貸莫雲留鎮統萬|{
	魏書官氏志内入諸姓烏丸氏為恒氏}
雲題之弟也|{
	莫題見一百十四卷晉安帝義熙四年}
秦王熾磐還枹罕|{
	夏既破故熾磐還枹音膚}
秋七月己卯魏主至柞嶺|{
	柞嶺即柞山之嶺柞則洛翻}
柔然寇雲中聞魏已克統萬乃遁去 秦王熾磐謂羣臣曰孤知赫連氏必無成冒險歸魏|{
	事見上卷營陽王景平元年}
今果如孤言八月遣其叔父平遠將軍渥頭等入貢于魏 壬子魏主還至平城以所獲頒賜留臺百官有差魏主為人壯健鷙勇臨城對陳|{
	陳讀曰陣}
親犯矢石左右死傷相繼神色自若由是將士畏服咸盡死力|{
	將即亮翻下同}
性儉率服御飲膳取給而已羣臣請增峻京城及修宫室曰易云王公設險以守其國|{
	易坎卦彖辭}
又蕭何云天子以四海為家不壯不麗無以重威|{
	事見十一卷漢高帝七年}
帝曰古人有言在德不在險|{
	吳起之言}
屈丐蒸土築城而朕滅之豈在城也今天下未平方須民力土功之事朕所未為蕭何之對非雅言也|{
	雅正也}
每以為財者軍國之本不可輕費至於賞賜皆死事勲績之家親戚貴寵未嘗横有所及|{
	横戶孟翻}
命將出師指授節度違之者多致負敗明於知人或拔士於卒伍之中唯其才用所長不論本末聽察精敏下無遁情賞不違賤罰不避貴雖所甚愛之人終無寛假常曰法者朕與天下共之何敢輕也然性殘忍果於殺戮往往已殺而復悔之|{
	如崔浩之類是也復扶又翻}
九月丁酉安定民舉城降魏|{
	夏都既破安定亦降降戶江翻下同}
氐主楊玄遣將軍苻白作圍秦梁州刺史出連輔政于赤水城中糧盡民執輔政以降輔政至駱谷逃還冬十月秦以驍騎將軍吳漢為平南將軍梁州刺史鎮南漒|{
	驍堅堯翻騎奇寄翻漒其良翻}
十一月魏主遣軍司馬公孫軌兼大鴻臚持節策拜楊玄為都督荆梁等四州諸軍事梁州刺史南秦王|{
	封玄為南秦王以别乞伏熾磐臚陵如翻}
及境玄不出迎軌責讓之欲奉策以還玄懼而郊迎魏主善之以軌為尚書軌表之子也|{
	表死於營陽王景平元年}
十二月秦梁州刺史吳漢為羣羌所攻帥戶二千還于枹罕|{
	帥讀曰率}
魏主行如中山癸卯還平城

資治通鑑卷一百二十
