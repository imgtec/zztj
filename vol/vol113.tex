\chapter{資治通鑑卷一百十三}


宋 司馬光 撰

胡三省 音註

晉紀三十五|{
	起昭陽單閼盡閼逢執徐凡二年}


安皇帝戊

元興二年春正月盧循使司馬徐道覆寇東陽二月辛丑建武將軍劉裕擊破之道覆循之姊夫也 乙卯以太尉玄為大將軍|{
	大將軍自漢以來職名崇重居其位者皆擅朝權晉初以司馬孚為太尉奏以大將軍位太尉下後復舊在三司上}
丁巳玄殺冀州刺史孫無終|{
	孫無終亦北府舊將也}
玄上表請帥諸軍掃平關洛既而諷朝廷下詔不許|{
	上時掌翻帥讀曰率朝直遙翻}
乃云奉詔故止玄初欲飭裝先命作輕舸載服玩書畫|{
	舸加我翻大舡也方言南楚江湖謂之舸畫與畫同}
或問其故玄曰兵凶戰危脱有意外當使輕而易運衆皆笑之|{
	桓玄意態終始如此耳時人誤以為雄豪而憚之故每遇輒敗峥嶸洲之戰劉道規等知其為人而徑突之一敗而不能復振矣易以豉翻}
夏四月癸巳朔日有食之 南燕主備德故吏趙融自長安來始得母兄凶問備德號慟吐血|{
	號戶刀翻吐土故翻}
因而寢疾司隸校尉慕容達謀反遣牙門皇璆攻端門|{
	璆渠尤翻}
殿中帥侯赤眉開門應之|{
	殿中帥猶晉之殿中三部督也帥所類翻}
中黄門孫進扶備德踰城匿於進舍段宏等聞宫中有變勒兵屯四門|{
	廣固城四門也}
備德入宫誅赤眉等達出奔魏備德優遷徙之民使之長復不役|{
	復方目翻復除也}
民緣此迭相䕃冒或百室合戶或千丁共籍以避課役尚書韓請加隱覈|{
	合音閤竹角翻隱度也覈實也隱覈度其實也}
備德從之使巡行郡縣|{
	行下孟翻}
得䕃戶五萬八千 泰山賊正始聚衆數萬自稱太平皇帝署置公卿南燕桂林王鎭討禽之臨刑或問其父及兄弟安在始曰太上皇蒙塵于外征東征西為亂兵所害其妻怒之曰君正坐此口奈何尚爾始曰皇后不知自古豈有不亡之國朕則崩矣終不改號|{
	史言王始僭舉大號至敗亡而不悔}
五月燕王熙作龍騰苑方十餘里役徒二萬人築景雲山於苑内基廣五百步峯高十七丈|{
	廣古曠翻高古號翻}
秋七月戊子魏王珪北巡作離宫於豺山平原太守和跋奢豪喜名|{
	喜許記翻}
珪惡而殺之|{
	惡烏路翻}
使其弟毗等就與訣跋曰灅北土瘠可遷水南勉為生計|{
	灅力水翻}
且使之背已|{
	背蒲妹翻}
曰汝何忍視吾之死也毗等諭其意詐稱使者逃入秦珪怒滅其家中壘將軍鄧淵從弟尚書暉與跋善|{
	徙才用翻}
或譖諸珪曰毗之出亡暉實送之珪疑淵知其謀賜淵死 南凉王傉檀及沮渠蒙遜互出兵攻呂隆|{
	傉奴沃翻沮子余翻}
隆患之秦之謀臣言於秦王興曰隆藉先世之資專制河外今雖飢窘尚能自支|{
	窘渠隕翻}
若將來豐贍終不為吾有凉州險絶土田饒沃不如因其危而取之興乃遣使徵呂超入侍|{
	使疏吏翻}
隆念姑臧終無以自存乃因超謀迎于秦興遣尚書左僕射齊難鎮西將軍姚誥左賢王乞伏乾歸鎮遠將軍趙曜帥步騎四萬迎隆于河西|{
	詰去吉翻帥讀曰率騎奇寄翻}
南凉王傉檀攝昌松魏安二戍以避之|{
	攝收也傉奴沃翻}
八月齊難等至姑臧隆素車白馬迎于道旁隆勸難擊沮渠蒙遜|{
	沮子余翻}
蒙遜使臧莫孩拒之敗其前軍|{
	孩何聞翻敗補邁翻}
難乃與蒙遜結盟蒙遜遣弟拏入貢于秦|{
	挐女居翻}
難以司馬王尚行凉州刺史配兵三千鎮姑臧以將軍閻松為倉松太守|{
	倉松即漢昌松縣}
郭將為番禾太守|{
	番音盤}
分戍二城徙隆宗族僚屬及民萬戶于長安|{
	載記曰自光至隆十三載而滅}
興以隆為散騎常侍|{
	散悉亶翻騎奇寄翻}
超為安定太守自餘文武隨才擢叙初郭黁常言代呂者王故其起兵先推王詳後推王乞基|{
	事見一百九卷元年黁奴昆翻}
及隆東遷王尚卒代之黁從乞伏乾歸降秦|{
	卒子恤翻降戶江翻}
以為滅秦者晉也遂來奔秦人追得殺之|{
	郭黁自信其術幸亂以徼福而卒以殺身足以明天道之難知矣}
沮渠蒙遜伯父中田護軍親信臨松太守孔篤皆驕恣為民患|{
	據晉書蒙遜載記中田護軍蓋呂光所置鎮臨松}
蒙遜曰亂吾法者二伯父也皆逼之使自殺秦遣使者梁構至張掖蒙遜問曰秃髪傉檀為公而身為侯何也|{
	秦封傉檀為廣武公封蒙遜為西海侯事見上卷上年}
構曰傉檀凶狡欵誠未著故朝廷以重爵虚名覊縻之將軍忠貫白日當入贊帝室豈可以不信相待也聖朝爵必稱功|{
	朝直遙翻稱尺證翻}
如尹緯姚晃佐命之臣齊難徐洛一時猛將爵皆不過侯伯|{
	緯于貴翻將即亮翻}
將軍何以先之乎|{
	先悉薦翻}
昔竇融殷勤固讓不欲居舊臣之右|{
	事見四十三卷漢光武建武十三年}
不意將軍忽有此問蒙遜曰朝廷何不即封張掖而更遠封西海邪構曰張掖將軍已自有之所以遠授西海者欲廣大將軍之國耳蒙遜悦乃受命 荆州刺史桓偉卒大將軍玄以桓修代之從事中郎曹靖之說玄曰|{
	說輸芮翻}
謙修兄弟專據内外權勢太重玄乃以南郡相桓石康為荆州刺史石康豁之子也|{
	桓豁温之次弟}
劉裕破盧循於永嘉追至晉安|{
	武帝太康三年分建安立晉安郡今泉州南安縣即其地宋白曰東晉南渡衣冠士族多萃此地以求安堵因立晉安郡隋為泉州}
屢破之循浮海南走何無忌潛詣裕勸裕於山隂起兵討桓玄裕謀於土豪孔靖靖曰山隂去都道遠舉事難成且玄未簒位不如待其已簒於京口圖之裕從之靖愉之孫也|{
	孔愉歷事元明成三帝}
九月魏主珪如南平城|{
	愍帝建興元年代公猗盧城盛樂以為北都修故平城以為南都更南百里於灅水之陽黄瓜堆築新平城所為南平城也唐朔州西南有新城即其地}
規度灅南|{
	自灅水南抵夏屋山皆灅南地也度徒洛翻灅力水翻}
將建新都 侍中殷仲文散騎常侍卞範之勸大將軍玄早受禪隂撰九錫文及冊命|{
	散悉亶翻騎奇寄翻禪時戰翻撰士免翻}
以桓謙為侍中開府録尚書事王謐為中書監領司徒桓胤為中書令加桓修撫軍大將軍胤冲之孫也丙子冊命玄為相國總百揆封十郡為楚王加九錫楚國置丞相以下官桓謙私問彭城内史劉裕曰楚王勲德隆重朝廷之情咸謂宜有揖讓卿以為何如裕曰楚王宣武之子|{
	桓温諡曰宣武}
勲德蓋世晉室微弱民望久移乘運襌代有何不可謙喜曰卿謂之可即可耳|{
	劉裕一世之雄桓謙問之以決可否裕詭辭以順其意故喜}
新野人庾仄殷仲堪之黨也|{
	新野縣漢屬南陽郡晉武帝太康中分屬義陽郡惠帝又分立新野郡仄阻力翻從厂不從□}
聞桓偉死石康未至乃起兵襲雍州刺史馮該於襄陽走之|{
	雍於用翻}
仄有衆七千設壇祭七廟云欲討桓玄江陵震動石康至州發兵攻襄陽仄敗奔秦 高雅之表南燕主備德請伐桓玄曰縱未能廓清吳會亦可收江北之地中書侍郎韓範亦上疏曰今晉室衰亂江淮南北戶口無幾戎馬單弱重以桓玄悖逆|{
	重直用翻悖蒲内翻又蒲沒翻}
上下離心以陛下神武發步騎一萬臨之彼必土崩瓦解兵不留行矣得而有之秦魏不足敵也拓地定功正在今日失時不取彼之豪傑誅滅桓玄更脩德政豈惟建康不可得江北亦無望矣備德曰朕以舊邦覆沒欲先定中原乃平蕩荆揚故未南征耳其令公卿議之因講武城西步卒二十七萬人騎五萬三千匹車萬七千乘|{
	乘繩證翻}
公卿皆以為玄新得志未可圖乃止|{
	慕容德取青州至是纔五年耳有衆如此不能乘時而用之自審其才不足以辨桓玄也}
冬十月楚王玄上表請歸藩使帝作手詔固留之又詐言錢塘臨平湖開|{
	臨平湖草常蓁塞開則天下太平}
江州甘露降使百寮集賀用為已受命之符又以前世皆有隱士恥於已時獨無求得西朝隱士安定皇甫謐六世孫希之|{
	晉氏東遷以洛陽為西朝皇甫謐在魏晉之間徵辟不行自號玄晏先生朝直遙翻}
給其資用使隱居山林徵為著作郎使希之固辭不就然後下詔旌禮號曰高士時人謂之充隱|{
	實非隱者而以之備數故謂之充隱}
又欲廢錢用穀帛及復肉刑制作紛紜志無一定變更回復|{
	更工衡翻}
卒無所施行性復貪鄙人士有法書好畫|{
	卒子恤翻性復扶又翻法書謂如史籕程邈李斯張芝師宜梁鵠衛瓘索靖鍾繇諸人真蹟各有家法者畫與畫同}
及佳園宅必假蒲博而取之尤愛珠玉未嘗離手|{
	離力智翻史言桓玄志度凡近}
乙卯魏主珪立其子嗣為齊王加位相國紹為清河王加征南大將軍熙為陽平王曜為河南王 丁巳魏將軍伊謂帥騎二萬襲高車餘種袁紇烏頻十一月庚午大破之|{
	帥讀曰率騎奇寄翻種章勇翻紇戶骨翻}
詔楚王玄行天子禮樂妃為王后世子為太子丁丑卞範之為禪詔|{
	禪時戰翻下同}
使臨川王寶逼帝書之寶晞之曾孫也|{
	武陵王晞死於桓温廢立之際}
庚辰帝臨軒遣兼太保領司徒王謐奉璽綬禪位于楚|{
	璽斯氏翻綬音受}
壬午帝出居永安宫癸未遷太廟神主于琅邪國|{
	永嘉之亂琅邪國人隨元帝過江者千餘戶太興二年立懷德縣丹陽雖有琅邪相而無其地成帝咸康元年桓温領琅邪太守鎮江乘之蒲洲金城求割江乘縣境立郡始有實土}
穆章何皇后及琅邪王德文皆徙居司徒府百官詣姑孰勸進十二月庚寅朔玄築壇於九井山北|{
	九域志太平州有九井山今太平州古姑孰之地也蕪湖縣南有溪猶曰姑孰溪北征記云九井山在丹陽南}
壬辰即皇帝位冊文多非薄晉室或諫之玄曰揖讓之文止可陳之於下民耳豈可欺上帝乎大赦改元永始以南康之平固縣|{
	武帝太康三年以盧陵南部都尉立南康郡平固吳所置平陽縣也太康元年更名平固九域志䖍州贑縣有平固鎮}
封帝為平固王降何后為零陵縣君琅邪王德文為右陽縣公武陵王遵為彭澤縣侯追尊父温為宣武皇帝廟號太祖南康公主為宣皇后封子昇為豫章王以會稽内史王愉為尚書僕射愉子相國左長史綏為中書令綏桓氏之甥也戊戌玄入建康宫登御坐而床忽陷|{
	坐讀曰座}
羣下失色殷仲文曰將由聖德深厚地不能載玄大悦梁王珍之國臣孔樸奉珍之犇夀陽珍之晞之曾孫也|{
	晞子㻱出繼梁國珍之之祖也}
戊申燕王熙尊燕主垂之貴嬪段氏為皇太后段氏熙之慈母也己酉立苻貴嬪為皇后|{
	嬪毗賓翻}
大赦 辛亥桓玄遷帝於尋陽|{
	尋陽郡時治柴桑}
燕以衛尉悦真為青州刺史鎮新城光禄大夫衛駒為并州刺史鎮凡城癸丑納桓温神主于太廟桓玄臨聽訟觀閲囚徒|{
	洛都華林園北有聽訟觀本平望觀也魏明帝以刑獄天下大命也每斷大獄常幸觀聽之太和三年更名聽訟觀建康倣洛都之制亦置之觀古玩翻}
罪無輕重多得原放有干輿乞者時或卹之其好行小惠如此|{
	干犯也干輿行犯乘輿也乞者丐衣食之物好呼到翻}
是歲魏主珪始命有司制冠服以品秩為差然法度草創多不稽古

三年春正月桓玄立其妻劉氏為皇后劉氏喬之曾孫也|{
	劉喬見八十六卷惠帝永興三年}
玄以其祖彛以上名位不顯不復追尊立廟|{
	復扶又翻}
散騎常侍徐廣曰敬其父則子悦|{
	孝經載孔子之言}
請依故事立七廟玄曰禮太祖東向左昭右穆|{
	禮天子七廟太祖正東向之位左三昭右三穆决疑要録曰父南面故曰昭昭明也子北面故曰穆穆順也昭本如字為漢諱昭改音韶或云晉文帝名昭改音韶}
晉立七廟宣帝不得正東向之位何足法也秘書監卞承之謂廣曰若宗廟之祭果不及祖有以知楚德之不長矣廣邈之弟也|{
	徐邈以文學為孝武所親信}
玄自即位心常不自安二月己丑朔夜濤水入石頭流殺人甚多讙譁震天玄聞之懼曰奴輩作矣玄性苛細好自矜伐|{
	讙許元翻好呼到翻下性好同}
主者奏事或一字不體|{
	謂字之上下偏傍不合體也}
或片辭之謬必加糾擿以示聰明|{
	擿他狄翻}
尚書答詔誤書春蒐為春菟|{
	菟同都翻}
自左丞王納之以下凡所關署|{
	關通也}
皆被降黜|{
	被皮義翻}
或手注直官|{
	直官入直官也}
或自用令史|{
	令史尚書令僕所署用}
詔令紛紜有司奉答不暇而紀綱不治奏案停積不能知也又性好遊畋或一日數出|{
	數所角翻}
遷居東宫更繕宫室|{
	更工衡翻}
土木並興督迫嚴促朝野騷然思亂者衆玄遣使加益州刺史毛璩散騎常侍左將軍璩執留玄使不受其命璩寶之孫也|{
	使疏吏翻散悉亶翻騎奇寄翻璩彊魚翻}
玄以桓希為梁州刺史分命諸將戍三巴以備之|{
	三巴巴郡巴東巴西也杜佑曰渝州古巴國謂之三巴以閬白二水東南流曲折三迴如巴字也}
璩傳檄遠近列玄罪狀遣巴東太守柳約之建平太守羅述征虜司馬甄季之擊破希等|{
	甄之人翻}
仍帥衆進屯白帝|{
	史言劉裕未起毛璩已仗義舉兵討玄帥讀曰率}
劉裕從徐兖二州刺史安成王桓脩入朝|{
	朝直遙翻}
玄謂王謐曰裕風骨不常蓋人傑也每遊集必引接殷勤贈賜甚厚玄后劉氏有智鑒謂玄曰劉裕龍行虎步視瞻不凡恐終不為人下不如早除之玄曰我方平蕩中原非裕莫可用者俟關河平定然後别議之耳玄以桓弘為青州刺史鎮廣陵刁逵為豫州刺史鎮歷陽弘脩之弟逵彛之子也|{
	刁彛見一百三卷簡文帝咸安二年}
劉裕與何無忌同舟還京口密謀興復晉室劉邁弟毅家於京口亦與無忌謀討玄無忌曰桓氏彊盛其可圖乎毅曰天下自有彊弱苟為失道雖彊易弱|{
	易以豉翻}
正患事主難得耳|{
	謂舉大事難得一人為主}
無忌曰天下草澤之中非無英雄也毅曰所見惟有劉下邳|{
	裕先領下邳太守故稱之}
無忌笑而不答還以告裕遂與毅定謀初太原王元德及弟仲德為苻氏起兵攻燕主垂不克來犇|{
	王叡字元德王懿字仲德名犯宣元二帝諱故以字行仲德為燕所敗渡河依段遼自遼所來奔為于偽翻}
朝廷以元德為弘農太守仲德見桓玄稱帝謂人曰自古革命誠非一族然今之起者恐不足以成大事平昌孟昶為青州主簿|{
	平昌縣漢屬城陽國魏文帝分城陽立平昌郡後省晉惠帝又立平昌郡其地今屬密州安丘縣界昶丑兩翻}
桓弘使昶至建康玄見而悦之謂劉邁曰素士中得一尚書郎|{
	起於白厔者謂之素士}
卿與其州里寧相識否|{
	孟昶平昌人平昌郡屬青州劉邁彭城沛人彭城屬徐州蓋二人並僑居京口故謂之同州里}
邁素與昶不善對曰臣在京口不聞昶有異能唯聞父子紛紛更相贈詩耳|{
	更工衡翻}
玄笑而止昶聞而恨之既還京口裕謂昶曰草間當有英雄起卿頗聞乎昶曰今日英雄有誰正當是卿耳於是裕毅無忌元德仲德昶及裕弟道規任城魏詠之|{
	任城縣前漢屬東平國後漢章帝元和元年分東平為任城國而任城縣屬焉晉氏南渡省任城郡為任城縣屬高平郡任音壬}
高平檀憑之琅邪諸葛長民河内太守隴西辛扈興振威將軍東莞童厚之|{
	莞音官}
相與合謀起兵道規為桓弘中兵參軍裕使毅就道規及昶於江北共殺弘據廣陵長民為刁逵參軍使長民殺逵據歷陽元德扈興厚之在建康使之聚衆攻玄為内應刻期齊孟昶妻周氏富於財昶謂之曰劉邁毁我於桓公使我一生淪陷我決當作賊卿幸早離絶脱得富貴相迎不晚也周氏曰君父母在堂欲建非常之謀豈婦人所能諫事之不成當於奚官中奉養大家|{
	奚如字又胡禮翻周禮注曰古者從坐男女没入縣官為奴其少才智以為奚今之侍史官婢或曰奚官女此言事君敗没為官婢當於奚官中養姑晉志奚官令屬少府晉宋間子婦稱其姑曰大家攷南史孝義孫棘傳可見}
義無歸志也昶愴然久之而起|{
	愴丑亮翻悵然失志貌}
周氏追昶坐曰觀君舉措非謀及婦人者不過欲得財物耳因指懷中兒示之曰此而可賣亦當不惜遂傾貲以給之昶弟顗妻周氏之從妹也|{
	顗魚豈翻從才用翻}
周氏紿之曰昨夜夢殊不祥|{
	紿待亥翻}
門内絳色物宜悉取以為厭勝|{
	厭於涉翻又於檢翻}
妹信而與之遂盡縫以為軍士袍何無忌夜於屏風裏草檄文其母劉牢之姊也登橙密窺之|{
	橙都鄧翻床屬}
泣曰吾不及東海呂母明矣|{
	呂母事見三十八卷王莽天鳳四年}
汝能如此吾復何恨問所與同謀者曰劉裕母尤喜因為言玄必敗舉事必成之理以勸之|{
	復扶又翻為于偽翻}
乙卯裕託以遊獵與無忌收合徒衆得百餘人丙辰詰旦京口城開無忌著傳詔服|{
	詰去吉翻著則略翻著傳詔之服因自稱敕使}
稱敕使居前|{
	使疏吏翻}
徒衆隨之齊入即斬桓脩以徇脩司馬刁弘帥文武佐吏來赴|{
	帥讀曰率}
裕登城謂之曰|{
	城謂京口之金城}
郭江州已奉乘輿返正於尋陽|{
	郭江州謂郭昶之也時帝在尋陽裕詭言以誑弘等乘䋲證翻}
我等並被密詔誅除逆黨|{
	被皮義翻}
今日賊玄之首已當梟於大航矣|{
	梟堅堯翻說文曰日至捕梟磔之以頭掛木上今謂掛首為梟}
諸君非大晉之臣乎今來欲何為弘等信之收衆而退裕問無忌曰今急須一府主簿何由得之無忌曰無過劉道民道民者東莞劉穆之也|{
	晉陵有南東莞郡故穆之居京口莞音官}
裕曰吾亦識之即馳信召焉時穆之聞京口讙噪聲|{
	讙許云翻噪當作譟}
晨起出陌頭屬與信會|{
	信使也屬之欲翻使疏吏翻}
穆之直視不言者久之|{
	直視注目直視不他屬}
既而返室壞布裳為袴|{
	袴脛衣也晉志曰袴褶之制未詳所起近世以為戎服壞音怪}
往見裕裕曰始舉大義方造艱難|{
	言造事之初事事艱難也}
須一軍吏甚急卿謂誰堪其選穆之曰貴府始建軍吏實須其才倉猝之際略當無見踰者裕笑曰卿能自屈吾事濟矣即於坐署主簿|{
	坐讀曰座}
孟昶勸桓弘其日出獵天未明開門出獵人昶與劉毅劉道規帥壯士數十人直入|{
	帥讀曰率下同}
弘方噉粥即斬之因收衆濟江|{
	噉徒濫翻}
裕使毅誅刁弘先是裕遣同謀周安穆入建康報劉邁|{
	先悉薦翻}
邁雖酬許意甚惶懼安穆慮事泄乃馳歸玄以邁為竟陵太守邁欲亟之郡是夜玄與邁書曰北府人情云何卿近見劉裕何所道邁謂玄已知其謀晨起白之玄大驚封邁為重安侯|{
	重直龍翻}
既而嫌邁不執安穆使得逃去乃殺之悉誅元德扈興厚之等衆推劉裕為盟主總督徐州事以孟昶為長史守京口檀憑之為司馬彭城人應募者裕悉使郡主簿劉鍾統之丁巳裕帥二州之衆千七百人|{
	二州兖徐也}
軍于竹里移檄遠近聲言益州刺史毛璩已定荆楚江州刺史郭昶之奉迎主上返正於尋陽鎮北參軍王元德等並帥部曲保據石頭揚武將軍諸葛長民已據歷陽玄移還上宫|{
	玄始遷東宫今以裕起移還上宫}
召侍官皆入止省中|{
	侍官自侍中下至黄散之屬}
加揚州刺史新野王桓謙征討都督以殷仲文代桓脩為徐兖二州刺史謙等請亟遣兵擊裕玄曰彼兵鋭甚計出萬死若有蹉跌|{
	蹉七何翻跌徒結翻}
則彼氣成而吾事去矣不如屯大衆於覆舟山以待之|{
	成帝咸康八年於覆舟山南立北郊山蓋在建康城北也形如覆舟故名}
彼空行二百里無所得鋭氣已挫忽見大軍必驚愕我案兵堅陣勿與交鋒彼求戰不得自然散走此策之上也謙等固請擊之乃遣頓丘太守吳甫之右衛將軍皇甫敷相繼北上|{
	自建康趣京口為北上上時掌翻}
玄憂懼特甚或曰裕等烏合微弱勢必無成陛下何慮之深玄曰劉裕足為一世之雄劉毅家無擔石之儲|{
	擔與儋同丁濫翻言一儋一石也儲無儋石家貧之至也揚雄家無儋石之儲應劭注曰齊人名小罌為儋石受二斛晉灼曰石斗石也前書音義曰儋言一斗之儲方言曰儋罌也齊東北海岱之間謂之儋余據今江淮人謂一石為一擔}
摴蒲一擲百萬何無忌酷似其舅共舉大事何謂無成 南涼王傉檀畏秦之彊乃去年號|{
	傉奴沃翻元興元年傉檀改元弘昌去羌呂翻}
罷尚書丞郎官遣參軍關尚使于秦|{
	使疏吏翻}
秦王興曰車騎獻欵稱藩而擅興兵造大城豈為臣之道乎|{
	興拜傉檀為車騎將軍故稱之}
尚曰王公設險以守其國|{
	易坎卦彖辭}
先王之制也車騎僻在遐藩密邇勍寇|{
	勍渠京翻}
蓋為國家重門之防|{
	重直龍翻}
不圖陛下忽以為嫌興善之傉檀求領涼州興不許 初袁真殺朱憲|{
	見一百二卷海西公太和五年}
憲弟綽逃奔桓温温克夀陽綽輒發真棺戮其尸温怒將殺之桓沖請而免之綽事沖如父沖薨綽嘔血而卒|{
	卒子恤翻}
劉裕克京口以綽子齡石為建武參軍|{
	裕本為建武將軍以齡石參軍事}
三月戊午朔裕軍與吳甫之遇於江乘|{
	江乘漢舊縣屬丹楊郡成帝咸康元年桓温領琅邪太守鎮江乘之蒲洲奏割丹楊之江乘立南琅邪郡江乘縣屬焉}
將戰齡石言於裕曰齡石世受桓氏厚恩不欲以兵刃相向乞在軍後裕義而許之甫之玄驍將也|{
	驍堅堯翻將即亮翻}
其兵甚鋭裕手執長刀大呼以衝之衆皆披靡|{
	呼火故翻披普彼翻}
即斬甫之進至羅落橋|{
	羅落橋在江乘縣南蓋緣水設羅落因以為名}
皇甫敷帥數千人逆戰|{
	帥讀曰率}
寧遠將軍檀憑之敗死裕進戰彌厲敷圍之數重裕倚大樹挺戰|{
	重直龍翻挺戰挺身獨戰也挺他鼎翻直也}
敷曰汝欲作何死拔戟將刺之裕瞋目叱之敷辟易|{
	刺七亦翻瞋七人翻辟讀曰闢易如字}
裕黨俄至射敷中額而踣|{
	射而亦翻中竹仲翻踣蒲北翻}
裕援刀直進|{
	援于元翻}
敷曰君有天命以子孫為託裕斬之厚撫其孤裕以檀憑之所領兵配參軍檀祗祗憑之之從子也|{
	從才用翻}
玄聞二將死大懼召諸道術人推算及為厭勝|{
	將即亮翻厭於叶翻又一琰翻}
問羣臣曰朕其敗乎吏部郎曹靖之對曰民怨神怒臣實懼焉玄曰民或可怨神何為怒對曰晉氏宗廟飄泊江濱|{
	謂遷晉宗廟主於琅邪國尋又隨帝上尋陽也}
大楚之祭上不及祖|{
	謂止祭桓温於太廟}
此其所以怒也玄曰卿何不諫對曰輦上君子皆以為堯舜之世臣何敢言玄默然使桓謙及游擊將軍何澹之屯東陵|{
	游擊將軍漢雜號將軍也魏置為中軍及晉以領護左右衛驍騎游擊為六軍建康之西有西陵其東有東陵東陵在覆舟山東北澹徒覽翻}
侍中後將軍卞範之屯覆舟山西衆合二萬己未裕軍食畢悉棄其餘糧進至覆舟山東使羸弱登山張旗幟為疑兵|{
	羸倫為翻幟昌志翻}
數道並前布滿山谷玄偵候者還云裕軍四塞|{
	偵丑鄭翻塞悉則翻}
不知多少玄益憂恐遣武衛將軍庾賾之帥精卒副援諸軍|{
	魏文帝踐阼置領軍將軍主五校中壘武衛等三營後遂各置將軍賾士革翻}
謙等士卒多北府人素畏伏裕莫有鬭志裕與劉毅等分為數隊進突謙陳裕以身先之|{
	陳讀曰陣先悉薦翻}
將士皆殊死戰無不一當百呼聲動天地時東北風急因縱火焚之煙炎熛天|{
	呼火故翻炎讀曰燄熛必遙翻}
鼓噪之聲震動京邑謙等諸軍大潰玄時雖遣軍拒裕而走意已決潛使領軍將軍殷仲文具舟於石頭聞謙等敗帥親信數千人聲言赴戰|{
	帥讀曰率下同}
遂將其子昇兄子濬出南掖門遇前相國參軍胡藩執馬鞚諫曰今羽林射手猶有八百皆是義故西人受累世之恩|{
	鞚音控馬勒也桓氏世居荆楚西人皆其義舊此蓋從玄東下玄既簒因以為羽林}
不驅令一戰一旦舍此欲安之乎玄不對但舉策指天|{
	玄舉策指天亦項羽所謂天之亡我之意}
因鞭馬而走西趨石頭與仲文等浮江南走經日不食左右進粗飯|{
	趨七喻翻粗與麄同}
玄咽不能下昇抱其胷而撫之玄悲不自勝|{
	咽於甸翻勝音升}
裕入建康王仲德抱元德子方回出候裕裕於馬上抱方回與仲德對哭追贈元德給事中以仲德為中兵參軍裕止桓謙故營遣劉鍾據東府庚申裕屯石頭城立留臺百官焚桓温神主於宣陽門外造晉新主納于太廟|{
	桓玄初簒遷七廟神主于琅邪國既而遷帝於尋陽宗廟主祏皆隨帝西上故權造新主}
遣諸將追玄尚書王嘏帥百官奉迎乘輿|{
	乘繩證翻}
誅玄宗族在建康者裕使臧熹入宫收圖書器物封閉府庫有金飾樂器裕問熹卿得無欲此乎熹正色曰皇上幽逼播越非所將軍首建大義劬勞王家雖復不肖實無情於樂|{
	復扶又翻}
裕笑曰聊以戲卿耳熹燾之弟也|{
	劉裕娶于臧氏}
壬戌玄司徒王謐與衆議推裕領揚州裕固辭乃以謐為侍中領司徒揚州刺史録尚書事謐推裕為使持節都督楊徐兖豫青冀幽并八州諸軍事徐州刺史|{
	使疏吏翻}
劉毅為青州刺史何無忌為琅邪内史孟昶為丹陽尹劉道規為義昌太守|{
	宋永初郡國志安豐有義昌縣蓋晉末嘗立郡宋初廢為縣也裕取義昌美名使道規領太守}
裕始至建康諸大處分皆委於劉穆之|{
	處昌呂翻分扶問翻}
倉猝立定無不允愜|{
	愜苦叶翻}
裕遂託以腹心動止諮焉穆之亦竭節盡誠無所遺隱時晉政寛弛綱紀不立豪族陵縱小民窮蹙重以司馬元顯政令違舛|{
	重直用翻}
桓玄雖欲釐整而科條繁密衆莫之從穆之斟酌時宜隨方矯正|{
	揉曲為矯言隨事矯揉使歸於正}
裕以身範物先以威禁内外百官皆肅然奉職不盈旬日風俗頓改|{
	史言劉裕有撥亂反正之才}
初諸葛長民至豫州失期不得發刁逵執長民檻車送桓玄至當利而玄敗|{
	當利浦名}
送人共破檻出長民還趣歷陽|{
	趣七喻翻}
逵棄城走為其下所執斬於石頭子姪無少長皆死|{
	少詩照翻長知兩翻}
唯赦其季弟給事中騁逵故吏匿其弟子雍送洛陽|{
	刁雍後自秦入魏雍於容翻}
秦王興以為太子中庶子裕以魏詠之為豫州刺史鎮歷陽諸葛長民為宣城内史初裕名微位薄輕狡無行盛流皆不與相知|{
	行下孟翻盛流謂當時貴盛之流}
惟王謐獨奇貴之謂裕曰卿當為一代英雄裕嘗與刁逵摴蒲不時輸直|{
	摴蒲不勝而不即納其所負之直此亦博徒輕狡之常態}
逵縛之馬柳|{
	柳魚浪翻繫馬柱也又五剛翻}
謐見之責逵而釋之代之還直由是裕深憾逵而德謐 蕭方等曰|{
	蕭方等梁元帝之嫡長子撰三十國春秋}
夫蛟龍潛伏魚鰕䙝之是以漢高赦雍齒魏武免梁鵠|{
	雍齒事見十一卷漢高帝十一年漢靈帝時梁鵠為選部尚書魏武欲為洛陽令鵠以為北部尉董卓之亂鵠奔劉表魏武破荆州鵠懼而自縛詣門使在祕書以勤書自効雍於用翻}
安可以布衣之嫌而成萬乘之隙也|{
	乘繩證翻}
今王謐為公刁逵亡族醻恩報怨何其狹哉 尚書左僕射王愉及子荆州刺史綏謀襲裕事泄族誅綏弟子慧龍為僧彬所匿得免|{
	慧龍後遂逃奔秦又自秦奔魏}
魏以中土蕭條詔縣戶不滿百者罷之 丁卯劉裕還鎮東府 桓玄至尋陽郭昶之給其器用兵力辛未玄逼帝西上|{
	上時掌翻}
劉毅帥何無忌劉道規等諸軍追之|{
	帥讀曰率}
玄留龍驤將軍何澹之前將軍郭銓與郭昶之守湓口|{
	驤思將翻澹徒覽翻昶丑兩翻湓蒲奔翻}
玄於道自作起居注|{
	杜佑通典曰周官有左右史蓋今起居注之本動則左史書之言則右史書之左史記言右史記事漢武帝有禁中起居注後漢馬皇后撰明帝起居注則漢起居注似在宫中為女史之任其後起居皆近侍之臣録記也歷代有其職而無其官後魏始置起居令史每行幸宴會則在御左右記録帝言後又别置修起居注}
叙討劉裕事自謂經略舉無遺策諸軍違節度以致犇敗專覃思著述|{
	覃深也廣也思相吏翻}
不暇與羣下議時事起居注既成宣示遠近 丙戌劉裕稱受帝密詔以武陵王遵承制總百官行事加侍中大將軍因大赦惟桓玄一族不宥 劉敬宣高雅之結青州大姓及鮮卑豪帥謀殺南燕主備德|{
	帥所類翻}
推司馬休之為主備德以劉軌為司空甚寵信之雅之欲邀軌同謀敬宣曰劉公衰老有安齊之志不可告也雅之卒告之|{
	卒子恤翻}
軌不從謀頗泄敬宣等南走南燕人收軌殺之追及雅之又殺之敬宣休之至淮泗間聞桓玄敗遂來歸|{
	敬宣等奔南燕事見上卷元興元年}
劉裕以敬宣為晉陵太守 南燕主備德聞桓玄敗命北地王鍾等將兵欲取江南會備德有疾而止|{
	昔魯莊公伐齊納子糾小白自莒先入所以有乾時之敗當此之時建康已定使慕容鍾等之師果進劉裕固有以待之矣將即亮翻}
夏四月己丑武陵王遵入居東宫内外畢敬遷除百官稱制書教稱令書以司馬休之監荆益梁寧秦雍六州諸軍事領荆州刺史|{
	劉毅等之兵既進故預以休之鎮南蕃監工銜翻雍於用翻}
庚寅桓玄挾帝至江陵桓石康納之玄更署置百官以卞範之為尚書僕射自以犇敗之後恐威令不行乃更增峻刑罰衆益離怨殷仲文諫玄怒曰今以諸將失律天文不利故還都舊楚而羣小紛紛妄興異議方當糾之以猛未可施之以寛也荆江諸郡聞玄播越有上表犇問起居者玄皆不受更令所在賀遷新都|{
	唐人所謂難將一人手掩盡天下目桓玄是也}
初王謐為玄佐命元臣玄之受禪|{
	禪時戰翻}
謐手解帝璽綬|{
	璽斯氏翻綬音受}
及玄敗衆謂謐宜誅劉裕特保全之劉毅嘗因朝會問謐璽綬所在|{
	璽斯氏翻綬音受朝直遙翻}
謐内不自安逃犇曲阿|{
	劉昫曰唐潤州丹楊縣古曲阿縣也}
裕牋白武陵王迎還復位 桓玄兄子歆引氐帥楊秋寇歷陽|{
	帥所類翻}
魏詠之帥諸葛長民劉敬宣劉鍾共擊破之|{
	帥讀曰率下同}
斬楊秋於練固|{
	練固在歷陽西北}
玄使武衛將軍庾稚祖江夏太守桓道恭帥數千人就何澹之等共守湓口何無忌劉道規至桑落洲|{
	桑落洲在湓城東北大江中杜佑曰桑落洲在江洲都昌縣漢之彭澤縣也}
庚戌澹之等引舟師逆戰澹之常所乘舫|{
	舫甫妄翻方舟也}
羽儀旗幟甚盛|{
	幟昌志翻}
無忌曰賊帥必不居此|{
	帥所類翻}
欲詐我耳宜亟攻之衆曰澹之不在其中得之無益無忌曰今衆寡不敵戰無全勝澹之既不居此舫戰士必弱我以勁兵攻之必得之得之則彼勢沮而我氣倍因而薄之破賊必矣|{
	沮在呂翻}
道規曰善遂往攻而得之因傳呼曰已得何澹之矣澹之軍中驚擾無忌之衆亦以為然乘勝進攻澹之等大破之無忌等克湓口進據尋陽遣使奉送宗廟主祏還京師|{
	祏音石廟中藏木主石室也既克尋陽宗廟主祏乃得還}
加劉裕都督江州諸軍事桑落之戰胡藩所乘艦為官軍所燒藩全鎧入水潛行三十許步乃得登岸|{
	艦戶黯翻鎧苦亥翻}
時江陵路已絶|{
	官軍既克尋陽故江陵之路絶}
乃還豫章劉裕素聞藩為人忠直引參領軍軍事 桓玄收集荆州兵曾未三旬有衆二萬樓船器械甚盛甲寅玄復帥諸軍挾帝東下以苻宏領梁州刺史為前鋒又使散騎常侍徐放先行說劉裕等曰若能旋軍散甲當與之更始各授位任令不失分|{
	復扶又翻說輸芮翻更工衡翻分抶問翻}
劉裕以諸葛長民都督淮北諸軍事鎮山陽以劉敬宣為江州刺史 柔然可汗社崘從弟悦代大郍謀殺社崘|{
	崘盧昆翻從才用翻郍與那同奴何翻}
不克犇魏 燕王熙於龍騰苑起逍遥宫連房數百鑿曲光海盛夏士卒不得休息暍死者大半|{
	去年熙起龍騰苑暍於歇翻傷暑也}
西凉世子譚卒 劉毅何無忌劉道規下邳太守平昌孟懷玉帥衆自尋陽西上|{
	帥讀曰率上時掌翻}
五月癸酉與桓玄遇於峥嶸洲|{
	水經注江水東過武口又東右得李姥浦北對峥嶸洲劉毅破桓玄處在今黄州夀昌軍之間杜佑曰峥嶸洲在鄂州武昌縣峥仕耕翻嶸戶萌翻}
毅等兵不滿萬人而玄戰士數萬衆憚之欲退還尋陽道規曰不可彼衆我寡彊弱異勢今若畏懦不進必為所乘雖至尋陽豈能自固玄雖竊名雄豪内實恇怯|{
	恇曲陽翻亦怯也}
加之已經犇敗衆無固心決機兩陣將雄者克|{
	將即亮翻}
不在衆也因麾衆先進毅等從之玄常漾舸於舫側以備敗走|{
	舸古我翻}
由是衆莫有鬭心毅等乘風縱火盡鋭争先玄衆大潰燒輜重夜遁|{
	重直用翻}
郭銓詣毅降玄故將劉統馮稚等聚黨四百人襲破尋陽城|{
	降戶江翻將即亮翻}
毅遣建威將軍劉懷肅討平之懷肅懷敬之弟也|{
	劉懷敬見一百一十一卷隆安三年}
玄挾帝單舸西走留永安何皇后及王皇后於巴陵|{
	永安何皇后穆帝章皇后也王皇后帝之后也}
殷仲文時在玄艦求出别船收集散卒因叛玄奉二后犇夏口|{
	夏戶雅翻}
遂還建康己卯玄與帝入江陵馮該勸使更下戰玄不從欲犇漢中就桓希|{
	桓希時為梁州刺史}
而人情乖沮號令不行|{
	沮在呂翻}
庚辰夜中處分欲發|{
	處昌呂翻分扶問翻}
城内已亂乃與親近腹心百餘人乘馬出城西走至城門左右於闇中斫玄不中|{
	不中竹仲翻}
其徒更相殺害|{
	更工衡翻}
前後交横玄僅得至船左右分散惟卞範之在側辛巳荆州别駕王康產奉帝入南郡府舍太守王騰之帥文武為侍衛|{
	帥讀曰率下同}
玄將之漢中屯騎校尉毛脩之璩之弟子也誘玄入蜀玄從之|{
	誘音酉}
寧州刺史毛璠|{
	璠音繁}
璩之弟也卒於官璩使其兄孫祐之及參軍費恬帥數百人送璠歸江陵|{
	費扶沸翻}
壬午遇玄於枚回洲|{
	水經注江水逕江陵縣南有州曰枚回洲}
祐之恬迎擊玄矢下如雨玄嬖人丁仙期萬蓋等以身蔽玄皆死|{
	嬖卑義翻又博計翻}
益州督護漢嘉馮遷抽刀前欲擊玄玄拔頭上玉導與之|{
	魏晉以來冠幘有簪有導至尊以玉為之導引也所以引髪入冠幘之内也}
曰汝何人敢殺天子遷曰我殺天子之賊耳遂斬之又斬桓石康桓濬庾賾之|{
	賾土革翻}
執桓昇送江陵斬於市乘輿返正於江陵以毛脩之為驍騎將軍|{
	乘繩證翻驍堅堯翻騎奇寄翻}
甲寅大赦諸以畏逼從逆者一無所問戊寅奉神主于太廟|{
	納尋陽所奉送宗廟主祏也}
劉毅等傳送玄首梟于大桁|{
	傳株戀翻梟堅堯翻}
毅等既戰勝以為大事已定不急追躡又遇風船未能進玄死幾一旬|{
	幾居希翻又音祈}
諸軍猶未至時桓謙匿於沮中|{
	沿沮水上下為沮中臨沮上黄二縣皆其地也沮干余翻}
揚武將軍桓振匿於華容浦|{
	華容縣自漢以來屬南郡水經注江水左迆為中夏口右則中郎浦出焉華容縣今在監利縣界晉書振傳曰匿於華容之涌中左傳閻敖游涌而逸杜預註云涌水在南郡華容縣涌音勇}
玄故將王稚徽戍巴陵遣人報振云桓歆已克京邑馮稚復克尋陽|{
	將即亮翻復扶又翻下復陷同}
劉毅諸軍並中路敗退振大喜聚黨得二百人襲江陵桓謙亦聚衆應之閏月己丑復陷江陵殺王康產王騰之振見帝於行宫躍馬奮戈直至階下問桓昇所在聞其已死瞋目謂帝曰|{
	瞋七人翻}
臣門戶何負國家而屠滅若是琅邪王德文下牀謂曰此豈我兄弟意邪振欲殺帝謙苦禁之乃下馬歛容致拜而出壬辰振為玄舉哀立喪庭諡曰武悼皇帝|{
	為于偽翻}
癸巳謙等帥羣臣奉璽綬於帝曰|{
	帥讀曰率}
主上法堯禪舜今楚祚不終百姓之心復歸於晉矣以琅邪王德文領徐州刺史振為都督八郡諸軍事荆州刺史謙復為侍中衛將軍加江豫二州刺史帝侍御左右皆振之腹心振少薄行玄不以子妷齒之|{
	行下孟翻以年叙長幼為齒又齒列也言不使預子姪之列}
至是嘆曰公昔不早用我遂致此敗若使公在我為前鋒天下不足定也今獨作此安歸乎遂縱意酒色肆行誅殺謙勸振引兵下戰已守江陵振素輕謙不從其言劉毅至巴陵誅王稚徽何無忌劉道規進攻桓謙於馬頭|{
	馬頭岸在大江南岸北對江津口}
桓蔚於龍泉|{
	水經註靈溪之東有龍陂廣員二百餘步水至淵深有龍見于其中故曰龍陂}
皆破之蔚祕之子也|{
	桓祕見一百三卷孝武寧康元年}
無忌欲乘勝直趣江陵|{
	趣七喻翻}
道規曰兵法屈申有時不可苟進諸桓世居西楚羣小皆為竭力振勇冠三軍|{
	為于偽翻冠古玩翻}
難與争鋒且可息兵養鋭徐以計策縻之不憂不克無忌不從振逆戰於靈溪|{
	水經注江水自江陵縣南東逕燕尾洲北合靈溪水江溪之會有靈溪戍背阿面江西帶靈溪}
馮該以兵會之無忌等大敗死者千餘人退還尋陽與劉毅等上牋請罪|{
	上時掌翻}
劉裕以毅節度諸軍免其青州刺史桓振以桓蔚為雍州刺史鎮襄陽|{
	雍於用翻}
柳約之羅述甄季之聞桓玄死自白帝進軍至枝江|{
	枝江縣自漢以來屬南郡我朝省為鎮屬松滋縣甄之人翻}
聞何無忌等敗於靈溪亦引兵退俄而述季之皆病約之詣桓振偽降欲謀襲振事泄振殺之約之司馬時延祖|{
	時姓也}
涪陵太守文處茂收其餘衆保涪陵|{
	處昌呂翻涪音浮}
六月毛璩遣將攻漢中斬桓希璩自領梁州 秋七月戊申永安皇后何氏崩 燕苻昭儀有疾龍城人王榮自言能療之昭儀卒|{
	卒子恤翻}
燕王熙立榮於公車門支解而焚之 八月癸酉葬穆章皇后于永平陵 魏置六謁官凖古六卿 九月刁騁謀反伏誅刁氏遂亡|{
	刁逵之誅惟赦騁而雍得逃走投北騁又誅則江南之刁氏亡矣}
刁氏素富奴客縱横|{
	横尸孟翻}
專固山澤為京口之患劉裕散其資蓄令民稱力而取之|{
	稱尺證翻}
彌日不盡時州郡飢弊民賴之以濟乞伏乾歸及楊盛戰于竹嶺|{
	上邽西南有南山竹嶺}
為盛所敗|{
	敗補邁翻}
西凉公暠立子歆為世子|{
	暠古老翻}
魏主珪臨昭陽殿改補百官引朝臣文武親加銓擇隨才授任|{
	朝直遙翻}
列爵四等王封大郡公封小郡侯封大縣伯封小縣其品第一至第四舊臣有功無爵者追封之宗室疏遠及異姓襲封者降爵有差又置散官五等其品第五至第九|{
	散悉亶翻}
文官造士才能秀異武官堪為將帥者其品亦比第五至第九|{
	將即亮翻帥所類翻}
百官有闕則取於其中以補之其官名多不用漢魏之舊傚上古龍官鳥官|{
	左傳郯子曰昔太皥氏以龍紀故為龍師而龍名我高祖少皥摯之立也鳳鳥適至故為鳥師而鳥名鳳鳥氏歷正也玄烏氏司分者也伯趙氏司至者也青烏氏司啓者也丹鳥氏司閉者也祝鳩氏司徒也鵙鳩氏司馬也鳴鳩氏司空也爽鳩氏司寇也鶻鳩氏司事也五鳩鳩民者也五雉為五工正九扈為九農正杜預註曰太皥氏有龍瑞故以龍名官應劭曰以龍紀其官長春官為青龍夏官為赤龍秋官為白龍冬官為黑龍中官為黄龍張晏曰庖犧將興神龍負圖而至因以名官與師也}
謂諸曹之使為鳬鴨|{
	魏書官氏志作諸曹走使}
取其飛之迅疾也謂候官伺察者為白鷺取其延頸遠望也餘皆類此 盧循寇南海攻番禺|{
	番音潘禺音愚}
廣州刺史濮陽吳隱之拒守百餘日|{
	濮博木翻}
冬十月壬戌循夜襲城而陷之燒府舍民室俱盡執吳隱之循自稱平南將軍攝廣州事聚燒骨為共冡葬於洲上得髑髏三萬餘枚|{
	髑徒谷翻髏郎侯翻說文曰髑髏頂也}
又使徐道覆改始興|{
	吳孫皓甘露元年分桂陽南部都尉立始興郡唐為韶州}
執始興相阮腆之|{
	相息亮翻腆它典翻}
劉裕領青州刺史|{
	劉毅免青州裕自領之}
劉敬宣在尋陽聚糧繕船未嘗無備故何無忌等雖敗退賴以復振|{
	復扶又翻下復自同}
桓玄兄子亮自稱江州刺史寇豫章敬宣擊破之劉毅何無忌劉道規復自尋陽西上至夏口|{
	夏戶雅翻}
桓振遣鎮東將軍馮該守東岸揚武將軍孟山圖據魯山城輔國將軍桓仙客守偃月壘衆合萬人水陸相援毅攻魯山城道規攻偃月壘無忌遏中流自辰至午二城俱潰|{
	漢水與江會于魯山西南漢水之左有却月城亦曰偃月壘故曲陵縣也後更為沙羨縣治}
生禽山圖仙客該走石城|{
	走音奏竟陵縣古石城戍也郢州圖經曰子城三面墉基皆天造正西絶壁下臨漢江石城之名蓋本於此}
辛巳魏大赦改元天賜築西宫十一月魏主珪如西宫命宗室置宗師八國置大師小師州郡亦各置師以辨宗黨舉才行如魏晉中正之職|{
	魏書官氏志曰以八國姓族難分故國立大師小師令辨其宗黨品舉人才自八國以外郡各立師職分如八國比今之中正也宗室立宗宗師亦如州郡八國之職行下孟翻}
燕王熙與苻后遊畋北登白鹿山東踰青嶺|{
	水經註大遼水東南過遼東郡房縣西又右會白狼水水出右北平白狼縣東南北屈逕白鹿山西即白狼山也青嶺即青陘在龍城東南四百餘里魏收地形志建德郡石城縣有白鹿山祠}
南臨滄海而還|{
	滄海在遼西郡海陽縣南還從宣翻又如字}
士卒為虎狼所殺及凍死者五千餘人 十二月劉毅等進克巴陵毅號令嚴整所過百姓安悦劉裕復以毅為兖州刺史桓振以桓放之為益州刺史屯西陵文處茂擊破之放之走還江陵 高句麗侵燕|{
	句如字又音駒麗力知翻}
戊辰魏主珪如豺山宫 是歲晉民避亂襁負之淮北者道路相屬|{
	襁居兩翻屬之欲翻}


資治通鑑卷一百十三
