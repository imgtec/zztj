<!DOCTYPE html PUBLIC "-//W3C//DTD XHTML 1.0 Transitional//EN" "http://www.w3.org/TR/xhtml1/DTD/xhtml1-transitional.dtd">
<html xmlns="http://www.w3.org/1999/xhtml">
<head>
<meta http-equiv="Content-Type" content="text/html; charset=utf-8" />
<meta http-equiv="X-UA-Compatible" content="IE=Edge,chrome=1">
<title>資治通鑒_286-資治通鑑卷二百八十五_286-資治通鑑卷二百八十五</title>
<meta name="Keywords" content="資治通鑒_286-資治通鑑卷二百八十五_286-資治通鑑卷二百八十五">
<meta name="Description" content="資治通鑒_286-資治通鑑卷二百八十五_286-資治通鑑卷二百八十五">
<meta http-equiv="Cache-Control" content="no-transform" />
<meta http-equiv="Cache-Control" content="no-siteapp" />
<link href="/img/style.css" rel="stylesheet" type="text/css" />
<script src="/img/m.js?2020"></script> 
</head>
<body>
 <div class="ClassNavi">
<a  href="/24shi/">二十四史</a> | <a href="/SiKuQuanShu/">四库全书</a> | <a href="http://www.guoxuedashi.com/gjtsjc/"><font  color="#FF0000">古今图书集成</font></a> | <a href="/renwu/">历史人物</a> | <a href="/ShuoWenJieZi/"><font  color="#FF0000">说文解字</a></font> | <a href="/chengyu/">成语词典</a> | <a  target="_blank"  href="http://www.guoxuedashi.com/jgwhj/"><font  color="#FF0000">甲骨文合集</font></a> | <a href="/yzjwjc/"><font  color="#FF0000">殷周金文集成</font></a> | <a href="/xiangxingzi/"><font color="#0000FF">象形字典</font></a> | <a href="/13jing/"><font  color="#FF0000">十三经索引</font></a> | <a href="/zixing/"><font  color="#FF0000">字体转换器</font></a> | <a href="/zidian/xz/"><font color="#0000FF">篆书识别</font></a> | <a href="/jinfanyi/">近义反义词</a> | <a href="/duilian/">对联大全</a> | <a href="/jiapu/"><font  color="#0000FF">家谱族谱查询</font></a> | <a href="http://www.guoxuemi.com/hafo/" target="_blank" ><font color="#FF0000">哈佛古籍</font></a> 
</div>

 <!-- 头部导航开始 -->
<div class="w1180 head clearfix">
  <div class="head_logo l"><a title="国学大师官网" href="http://www.guoxuedashi.com" target="_blank"></a></div>
  <div class="head_sr l">
  <div id="head1">
  
  <a href="http://www.guoxuedashi.com/zidian/bujian/" target="_blank" ><img src="http://www.guoxuedashi.com/img/top1.gif" width="88" height="60" border="0" title="部件查字,支持20万汉字"></a>


<a href="http://www.guoxuedashi.com/help/yingpan.php" target="_blank"><img src="http://www.guoxuedashi.com/img/top230.gif" width="600" height="62" border="0" ></a>


  </div>
  <div id="head3"><a href="javascript:" onClick="javascript:window.external.AddFavorite(window.location.href,document.title);">添加收藏</a>
  <br><a href="/help/setie.php">搜索引擎</a>
  <br><a href="/help/zanzhu.php">赞助本站</a></div>
  <div id="head2">
 <a href="http://www.guoxuemi.com/" target="_blank"><img src="http://www.guoxuedashi.com/img/guoxuemi.gif" width="95" height="62" border="0" style="margin-left:2px;" title="国学迷"></a>
  

  </div>
</div>
  <div class="clear"></div>
  <div class="head_nav">
  <p><a href="/">首页</a> | <a href="/ShuKu/">国学书库</a> | <a href="/guji/">影印古籍</a> | <a href="/shici/">诗词宝典</a> | <a   href="/SiKuQuanShu/gxjx.php">精选</a> <b>|</b> <a href="/zidian/">汉语字典</a> | <a href="/hydcd/">汉语词典</a> | <a href="http://www.guoxuedashi.com/zidian/bujian/"><font  color="#CC0066">部件查字</font></a> | <a href="http://www.sfds.cn/"><font  color="#CC0066">书法大师</font></a> | <a href="/jgwhj/">甲骨文</a> <b>|</b> <a href="/b/4/"><font  color="#CC0066">解密</font></a> | <a href="/renwu/">历史人物</a> | <a href="/diangu/">历史典故</a> | <a href="/xingshi/">姓氏</a> | <a href="/minzu/">民族</a> <b>|</b> <a href="/mz/"><font  color="#CC0066">世界名著</font></a> | <a href="/download/">软件下载</a>
</p>
<p><a href="/b/"><font  color="#CC0066">历史</font></a> | <a href="http://skqs.guoxuedashi.com/" target="_blank">四库全书</a> |  <a href="http://www.guoxuedashi.com/search/" target="_blank"><font  color="#CC0066">全文检索</font></a> | <a href="http://www.guoxuedashi.com/shumu/">古籍书目</a> | <a   href="/24shi/">正史</a> <b>|</b> <a href="/chengyu/">成语词典</a> | <a href="/kangxi/" title="康熙字典">康熙字典</a> | <a href="/ShuoWenJieZi/">说文解字</a> | <a href="/zixing/yanbian/">字形演变</a> | <a href="/yzjwjc/">金 文</a> <b>|</b>  <a href="/shijian/nian-hao/">年号</a> | <a href="/diming/">历史地名</a> | <a href="/shijian/">历史事件</a> | <a href="/guanzhi/">官职</a> | <a href="/lishi/">知识</a> <b>|</b> <a href="/zhongyi/">中医中药</a> | <a href="http://www.guoxuedashi.com/forum/">留言反馈</a>
</p>
  </div>
</div>
<!-- 头部导航END --> 
<!-- 内容区开始 --> 
<div class="w1180 clearfix">
  <div class="info l">
   
<div class="clearfix" style="background:#f5faff;">
<script src='http://www.guoxuedashi.com/img/headersou.js'></script>

</div>
  <div class="info_tree"><a href="http://www.guoxuedashi.com">首页</a> > <a href="/SiKuQuanShu/fanti/">四库全书</a>
 > <h1>资治通鉴</h1> <!--         下载:【右键另存为】即可 --></div>
  <div class="info_content zj clearfix">
  
<div class="info_txt clearfix" id="show">
<center style="font-size:24px;">286-資治通鑑卷二百八十五</center>
    資治通鑑卷二百八十五 宋 司馬光 撰<br />
<br />
  胡三省 音註<br />
<br />
  後晉紀六【起旃蒙大荒落八月盡柔兆敦牂凡一年有奇】<br />
<br />
  齊王下<br />
<br />
  開運二年八月甲子朔日有食之 丙寅右僕射兼中書侍郎同平章事和凝罷守本官加樞密使戶部尚書馮玉中書侍郎同平章事事無大小悉以委之帝自陽城之捷謂天下無虞驕侈益甚【陽城之捷見上卷上年夫勝之不可恃也尚矣紂之百克而卒無後夫差數戰數勝終以亡國桑田之捷滅虢之兆也方城之勝破庸之基也項梁死於定陶而嬴秦墟宇文化及摧於黎陽而李密敗皆恃勝之祻也陽城之戰危而後克契丹折翅北歸蓄憤愈甚為謀愈深晉主乃偃然以為無虞石氏宗廟宜其不祀也】四方貢獻珍奇皆歸内府多造器玩廣宫室崇飾後庭近朝莫之及【近朝謂近世如梁如唐也朝直遥翻】作織錦樓以織地衣用織工數百期年乃成【期讀作朞】又賞賜優伶無度桑維翰諫曰曏者陛下親禦胡寇【謂元年澶州之戰也事見上卷】戰士重傷者賞不過帛數端今優人一談一笑稱旨【稱尺正翻】往往賜束帛萬錢錦袍銀帶【唐制帛以十端為束】彼戰士見之能不觖望曰我曹冒白刃絶筋折骨【觖古穴翻觖望怨望也冒莫北翻折而設翻】曾不如一談一笑之功乎如此則士卒解體陛下誰與衛社稷乎帝不聽馮玉每善承迎帝意由是益有寵嘗有疾在家帝謂諸宰相曰自刺史以上俟馮玉出乃得除其倚任如此【竇廣德有賢行漢文帝以其后弟恐天下議其私不敢相也馮玉何人斯晉出帝昌言於朝以昭親任之意臨亂之君各賢其臣其此謂乎】玉乘勢弄權四方賂遺輻輳其門【遺唯季翻】由是朝政益壞【史言晉亡形已成朝直遥翻】 唐兵圍建州既久【是年二月唐兵攻建州事始見上卷】建人離心或謂董思安【謂誨語之也】宜早擇去就思安曰吾世事王氏危而叛之天下其誰容我衆感其言無叛者丁亥唐先鋒橋道使上元王建封先登【上元本江寧縣唐肅宗上元間更名帶江寧府】遂克建州閩主延政降【閩自唐末王潮得福建傳審知延翰鏻昶曦至延政而亡】王忠順戰死董思安整衆奔泉州【史言泉州二將事閩主有始終】初唐兵之來建人苦王氏之亂與楊思恭之重斂【楊思恭重斂事見二百八十三卷天福八年斂力贍翻】爭伐木開道以迎之及破建州縱兵大掠焚宫室廬舍俱盡是夕寒雨凍死者相枕【枕職任翻】建人失望唐主以其有功皆不問 漢主殺韶王弘雅【弘雅漢主之弟也】 九月許文稹以汀州王繼勲以泉州王繼成以漳州皆降於唐【荀子有言兼并易也豎疑之難唐能取閩不能終有閩也為閩人叛唐張本】唐置永安軍於建州 丙申以西京留守兼侍中景延廣充北面行營副招討使 殿中監王欽祚權知恒州事【恒戶登翻】會之軍儲詔欽祚括糴民粟杜威有粟十餘萬斛在恒州欽祚舉籍以聞威大怒表稱臣有何罪欽祚籍没臣粟朝廷為之召欽祚還【杜威恒州之粟豈非前者表獻之數乎使其出於表獻之外亦掊克軍民所積者耳舉而籍之夫何過朝廷之法不行於貴近第能虐貧下以供調度國非其國矣為于偽翻還從宣翻】仍厚賜威以慰安之 戊申置威信軍於曹州遣侍衛馬步都指揮使李守貞戍澶州 乙卯遣彰德節度使張彦澤戍恒州 漢主殺劉思潮林少強林少良何昌廷【天福八年漢主使劉思潮等四人弑其兄弘度而自立事見二百八十三卷今又殺四人以除其偪少詩照翻】以左僕射王翷嘗與高祖謀立弘昌【事見二百八十三卷天福七年】出為英州刺史【英州漢桂陽郡湞陽縣之地唐以湞陽縣隸廣州漢主劉龔分湞陽縣置英州九域志廣州北至英州四百二十里】未至賜死内外皆懼不自保 冬十月癸巳置鎮安軍於陳州 唐元敬宋太后殂 王延政至金陵唐主以為羽林大將軍斬楊思恭以謝建人【以楊思恭厚斂也】以百勝節度使王崇文為永安節度使崇文治以寛簡建人遂安【撫寧荒餘其政當爾自蓋公授此法于曹參參以相齊又以相漢後人知此法者鮮矣治直之翻】初高麗王建用兵吞滅鄰國頗彊大【事見二百八十一卷高祖天福元年麗力之翻】因胡僧閥囉言於高祖曰勃海我昏姻也其王為契丹所虜請與朝廷共擊取之高祖不報及帝與契丹為仇閥囉復言之【閥望發翻囉魯何翻復扶又翻】帝欲使高麗擾契丹東邉以分其兵勢會建卒子武自稱權知國事上表告喪十一月戊戌以武為大義軍使高麗王遣通事舍人郭仁遇使其國【使疎吏翻】諭指使擊契丹【畏契丹知之不形諸詔命以詔指諭之而已】仁遇至其國見其兵極弱曏者襪囉之言特建為誇誕耳實不敢與契丹為敵【宋白曰晉天福中有西域僧襪囉來朝善火卜俄辭高祖請遊高麗王建甚禮之時契丹併勃海之地有年矣建因從容謂襪囉曰勃海本吾親戚之國具王為契丹所虜吾欲為朝廷攻而取之且欲平其舊怨師迴為言於天子當定期兩襲之襪囉還具奏高祖不報出帝與契丹交兵襪囉復奏之帝遣郭仁遇飛詔諭建深攻其地以牽脅之會建已卒武知國事與其父之大臣不叶自相魚肉内難稍平兵威未振且夷人怯懦襪囉之言皆建虛誕耳】仁遇還【還從宣翻】武更以它故為解【為說以自解】 乙卯吴越王弘佐誅内都監使杜昭達己未誅内牙上統軍使明州刺史闞璠【璠音翻】昭達建徽之孫也【杜建徽佐吴越王錢鏐有功】與璠皆好貨【好呼到翻】錢塘富人程昭悦以貨結二人得侍弘佐左右昭悦為人狡佞王悦之寵待踰於舊將璠不能平昭悦知之詣璠頓首謝罪璠責讓久之乃曰吾始者決欲殺汝今既悔過吾亦釋然昭悦懼謀去璠璠專而愎國人惡之者衆【去羌呂翻愎蒲逼翻惡烏路翻】昭悦欲出璠於外恐璠覺之私謂右統軍使胡進思曰今欲除公及璠各為本州使璠不疑可乎進思許之乃以璠為明州刺史進思為湖州刺史【闞璠明州人胡進思湖州人也】璠怒曰出我於外是弃我也進思曰老兵得大州幸矣不行何為璠乃受命既而復以它故留進思【復扶又反】内外馬步都統軍使錢仁俊母杜昭達之姑也昭悦因譛璠昭達謀奉仁俊作亂下獄鍛鍊成之【下戶駕翻】璠昭達既誅奪仁俊官幽于東府於是昭悦治闞杜之黨凡權任與已侔意所忌者誅放百餘人國人畏之側目【為弘佐誅昭悦張本治直之翻】胡進思重厚寡言昭悦以為戇故獨存之【胡進思獨存所以階錢氏廢立之禍】昭悦收仁俊故吏慎温其【慎姓也古有慎到温其名也】使證仁俊之罪拷掠備至【拷音考掠音亮】温其堅守不屈弘佐嘉之擢為國官【國官吳越國官也慎温其自藩府吏職㩴為國官】温其衢州人也 十二月乙丑加吴越王弘佐東南面兵馬都元帥 辛未以前中書舍人廣晉隂鵬為給事中樞密直學士【唐改魏州為興唐府高祖改為廣晉府】鵬馮玉之黨也朝廷每有遷除玉皆與鵬議之由是請謁賂遺充滿其門【遺惟季翻】 初帝疾未平【去年冬帝有疾見上卷】會正旦【謂今年正月朔旦】樞密使中書令桑維翰遣女僕入宫起居太后【女僕即女奴也唐人謂參候為起居今人之言猶爾】因問皇弟睿近讀書否【睿即重睿也避帝名去重字】帝聞之以告馮玉玉因譛維翰有廢立之志帝疑之【帝固忌重睿因桑維翰女僕之問已疑維翰矣馮玉又從而譛之其疑愈不可破矣】李守貞素惡維翰【惡烏路翻】馮玉李彦韜與守貞合謀排之以中書令行開封尹趙瑩柔而易制【易以䜴翻】共薦以代維翰丁亥罷維翰政事為開封尹以瑩為中書令李崧為樞密使守侍中維翰遂稱足疾希復朝謁杜絕賓客【亦所以遠猜疑也復扶又翻朝直遥翻】或謂馮玉曰桑公元老今既解其樞務縱不留之相位猶當優以大藩柰何使之尹京親猥細之務乎【猥雜也】玉曰恐其反耳【言所以不授維翰大鎮者恐其阻兵而反】曰儒生安能反玉曰縱不自反恐其教人耳【此指維翰贊成晉祖晉陽舉兵之謀】 楚湘隂處士戴偃【劉昫曰湘隂漢羅縣宋置湘隂縣唐屬岳州宋淳化四年以湘隂縣隷潭州九域志在州東北一百一十五里】為詩多譏刺楚王希範囚之天策副都軍使丁思瑾上書切諫希範削其官爵 唐齊王景達府屬謝仲宣言於景達曰宋齊丘先帝布衣之交今弃之草萊不厭衆心景達為之言於唐主曰【厭於葉翻伏也又於艷翻滿也為于偽翻】齊丘宿望勿用可也何必弃之以為名唐主乃使景達自至青陽召之【齊丘隱青陽見二百八十三卷天福八年】<br />
<br />
  三年春正月以齊丘為大傅兼中書令但奉朝請不預政事【奉朝會請召而已】以昭武節度使李建勲為右僕射兼門下侍郎與中書侍郎馮延已皆同平章事建勲練習吏事而懦怯少斷延已工文辭而狡佞喜大言多樹朋黨【斷丁亂翻喜許記翻惟世宦則練習吏事懦怯少斷則亦因練習之久而巧於避就者然也若馮延己所為廼少年書生之常態多大言而少成事樹朋黨以濟己私此二種人皆不可以相也】水部郎中高越上書指延己兄弟過惡唐主怒貶越蘄州司士初唐主置宣政院於禁中以翰林學士給事中常夢錫領之專典機密與中書侍郎嚴續皆忠直無私唐主謂夢錫曰大臣惟嚴續中立然無才恐不勝其黨卿宜左右之未幾夢錫罷宣政院【左右讀為佐佑幾居豈翻】續亦出為池州觀察使夢錫於是移疾縱酒不復預朝廷事【史言正邪雜處正終為邪所勝復扶又翻】續可求之子也【嚴可求徐温之謀主也】 二月壬戌朔日有食之 晉昌節度使兼侍中趙在禮【晉以京兆府為晉昌軍】更歷十鎮【更工衡翻趙在禮起於鄴都徙義成不行後歷横海泰寧匡國天平忠武武寧歸德晉昌凡十鎮】所至貪暴家貲為諸帥之最【帥所類翻】帝利其富三月唐申為皇子鎮寧節度使延煦娶其女【為于偽翻鎮寧軍澶州煦吁句翻】在禮自費緡錢十萬縣官之費數倍過之延煦及弟延寶皆高祖諸孫帝養以為子 唐泉州刺史王繼勲致書修好於威武節度使李弘義【好呼到翻】弘義以泉州故隸威武軍怒其抗禮【王繼勲與李弘義同事南唐弘義雖建節然比肩事主固不可修巡屬之禮李弘義以此起兵端耳】夏四月遣弟弘通將兵萬人伐之 初朔方節度使馮暉在靈州留党項酋長拓跋彦超於州下【事見二百八十二卷天福四年党底朗翻酋慈由翻長知兩翻】故諸部不敢為寇及將罷鎮而縱之前彰武節度使王令温代暉鎮朔方不存撫羌胡以中國法繩之【昔周之封衛疆以周索以其地居中國也其封晉則疆以戎索以其地近戎狄也戎狄不可繩以中國之法尚矣】羌胡怨怒競為寇鈔【鈔楚交翻】拓跋彦超石存也厮褒三族共攻靈州殺令温弟令周戊午令温上表告急 泉州都指揮使留從效謂刺史王繼勲曰李弘通兵勢甚盛士卒以使君賞罰不當【當丁浪翻】莫肯力戰使君宜避位自省【省昔景翻】乃廢繼勲歸私第【留從效立王繼勲見上卷上年】代領軍府事勒兵擊李弘通大破之表聞于唐唐主以從效為泉州刺史召繼勲還金陵遣將將兵戍泉州【為留從效遣唐戍將歸張本】徙漳州刺史王繼成為和州刺史汀州刺史許文稹為蘄州刺史【稹止忍翻】 定州西北二百里有狼山【匈奴須知狼山寨東北至易州八十里東南至廣信軍界】土人築堡於山上以避胡寇堡中有佛舍尼孫深意居之以妖術惑衆【妖於遙翻】言事頗驗遠近信奉之中山人孫方簡【歐史作孫方諫蓋孫方簡後避周太祖皇考諱遂改名方諫也 考異曰按周世祖實録云清苑人今從漢高祖實録】及弟行友自言深意之姪不飲酒食肉事深意甚謹深意卒方簡嗣行其術稱深意坐化【崇信釋氏而學其學專一而静者其死也能結趺端坐如生謂之坐化】嚴飾事之如生其徒日滋【薛史曰宋乾德中遷其尼朽骨赴京焚於北郊妖徒遂息】會晉與契丹絕好【好呼到翻】北邉賦役煩重寇盜充斥民不安其業方簡行友因帥鄉里豪健者據寺為寨以自保契丹入寇方簡帥衆邀擊【帥讀曰率】頗獲其甲兵牛馬軍資人挈家往依之者日益衆久之至千餘家遂為羣盜懼為吏所討乃歸欵朝廷朝廷亦資其禦寇署東北招討指揮使方簡時入契丹境鈔掠【鈔楚交翻】多所殺獲既而邀求不已朝廷小不副其意則舉寨降於契丹請為鄉道以入寇【邉境之上姦民如此者不特孫方簡唐人所謂兩面也降戶江翻鄉讀曰嚮道讀曰導】時河北大饑民餓死者所在以萬數兖鄆滄貝之間盜賊蠭起吏不能禁天雄節度使杜威遣元隨軍將劉延翰市馬於邉方簡執之獻於契丹延翰逃歸六月壬戌至大梁言方簡欲乘中國凶饑引契丹入寇宜為之備【為孫方簡乘中國無主契丹北歸入據定州張本】 初朔方節度使馮暉在靈武得羌胡心市馬朞年得五千匹朝廷忌之徙鎮邠州及陜州【陜先冉翻】入為侍衛步軍都指揮使領河陽節度使暉知朝廷之意悔離靈武【離力智翻】乃厚事馮玉李彦韜求復鎮靈州朝廷亦以羌胡方擾丙寅復以暉為朔方節度使將關西兵擊羌胡以威州刺史藥元福為行營馬步軍都指揮使【威州唐之安樂州也中世没於吐蕃大中三年收復更名威州梁唐弃之晉復置後周改為環州以大河環曲為名亦唐初之舊州名也趙珣聚米圖經靈州南至環州五百里按薛史天福四年五月勑靈州方渠鎮宜升為威州割寧州木波馬嶺二縣隸之後周改為環州顯德四年降為通遠軍】 乙丑定州言契丹勒兵壓境詔以天平節度使侍衛馬步都指揮使李守貞為北面行營都部署義成節度使皇甫遇副之彰德節度使張彦澤充馬軍都指揮使兼都虞候義武節度使薊人李殷充步軍都指揮使兼都排陳使【薊音計陳讀曰陣】遣護聖指揮使臨清王彦超大原白延遇以步兵十營詣邢州時馬軍都指揮使鎮安節度使李彦韜方用事【時以陳州置鎮安軍】視守貞蔑如也守貞在外所為事無大小彦韜必知之守貞外雖敬奉而内恨之【為李守貞與杜威降契丹張本】 初唐人既克建州【去年八月唐克建州】欲乘勝取福州唐主不許樞密使陳覺請自往說李弘義【說式芮翻】必令入朝宋齊丘薦覺才辯可不煩寸刃坐致弘義唐主乃拜弘義母妻皆為國夫人四弟皆遷官以覺為福州宣諭使厚賜弘義金帛【欲啖李弘義以禄利而誘致之】弘義知其謀見覺辭色甚倨待之疎薄覺不敢言入朝事而還【為陳覺興兵攻福州喪敗而還張本還從宣翻又如字】 秋七月河決楊劉西入莘縣廣四十里自朝城北流【莘縣在魏州之東朝城在魏州東南相去四十里廣古曠翻】 有自幽州來者言趙延夀有意歸國樞密使李崧馮玉信之命天雄節度使杜威致書於延夀具述朝旨啖以厚利【朝直遙翻啖徒濫翻】洺州軍將趙行實嘗事延夀遣齎書潜往遺之延夀復書言久處異域【遺惟季翻處昌呂翻】思歸中國乞發大軍應接拔身南去辭旨懇密朝廷欣然復遣行實詣延夀與為期約【晉人自此墮趙延夀計中矣復扶又翻】八月李守貞言與契丹千餘騎遇於長城北【此戰國時燕所築長城也在涿州固安縣南薛史李守貞奏大軍至望都縣相次至長城北遇虜轉鬬】轉鬭四十里斬其酋帥解里【酋慈秋翻解戶買翻】擁餘衆入水溺死者甚衆丁卯詔李守貞還屯澶州【還從宣翻】 帝既與契丹絕好數召吐谷渾酋長白承福入朝【好呼到翻數所角翻】宴賜甚厚承福從帝與契丹戰澶州【澶時連翻】又與張從恩戍滑州屬歲大熱【屬之欲翻】遣其部落還太原畜牧於嵐石之境【嵐盧含翻】部落多犯法劉知遠無所縱捨部落知朝廷微弱且畏知遠之嚴謀相與遁歸故地【吐谷渾部落既知朝廷微弱又畏劉知遠之嚴然不敢于太原作亂者憚劉知遠之威畧無所肆其姦故欲遁歸故地】有白可久者位亞承福帥所部先亡歸契丹【帥讀曰率】契丹用為雲州觀察使以誘承福【誘音酉】知遠與郭威謀曰今天下多事置此屬於太原乃腹心之疾也不如去之承福家甚富飼馬用銀槽【去先呂翻飼祥吏翻】威勸知遠誅之收其貨以贍軍知遠密表吐谷渾反覆難保請遷於内地帝遣使發其部落千九百人分置河陽及諸州知遠遣威誘承福等入居太原城中因誣承福等五族謀叛以兵圍而殺之合四百口籍没其家貲詔褒賞之吐谷渾由是遂微【五代會要曰吐谷渾酋長有赫連鐸者唐咸通中從太原節度使康承訓平徐方有功朝廷授振武節度使復盜據雲中後唐太祖逐之乃歸幽州李匡儔其部落散居蔚州界互為君長其氏不常有白承福者自同光初代為都督依中山北石門為柵莊宗賜其額為寧朔奉化兩府以都督為節度使仍賜承福姓李名紹魯其畜牧就善水草丁壯常數千人羊馬生息入市中土朝廷常存恤之潞王清泰三年白可久為寧朔奉化留後始見於史晉天福元年高祖以契丹有助立之功割雁門以北及幽州之地以賂之由是吐谷渾部族皆隸於契丹其後苦契丹之虐政復為鎮州節度使安重榮所誘乃背契丹率車帳羊馬取五臺路歸國契丹大怒以朝廷招納叛亡遣使責讓至六年正月高祖命供奉官張澄等率兵二千搜索并鎮忻代四州山谷吐渾還其舊地然亦以契丹誅求無厭心不平之命漢高祖出鎮太原潜加慰撫其年五月大首領白承福及麾下來朝九月又遣首領白可久來朝少主嗣位絶契丹之好數召其酋長入朝厚加錫賜每大讌會皆命列坐於勲臣之次至開運捍虜於澶州召承福等帥其部衆從行屬歲多暑熱部下多死復遣歸太原移帳於嵐石州然承福馭下無法多于軍令其族白可久在承福之亞因牧馬帥本帳北遁契丹授以官爵復遣潜誘承福承福亦思叛去事未果漢祖知之乃以兵還其部族擒承福與其族白鐵匱赫連海龍等五家凡四百有餘人伏誅籍其牛馬命别部長王義宗統其餘屬】濮州刺史慕容彦超坐違法科斂【斂力贍翻】擅取官麥五百斛造麴賦與部民李彦韜素與彦超有隙發其事罪應死彦韜趣馮玉使殺之【趣讀曰促】劉知遠上表論救【慕容彦超劉知遠之同產弟故救之上時掌翻】李崧曰如彦超之罪今天下藩侯皆有之若盡其法恐人人不自安甲戌勑免彦超死削官爵流房州 唐陳覺自福州還至劒州【劒州即殷主王延政所置之鐔州也南唐既克建州分延平建浦富沙三縣置劒州至宋混一天下以蜀中亦有劒州乃加南字為南劒州】恥無功【恥自詭說李弘義入朝而不能致也】矯詔使侍衛官顧忠召弘義入朝【侍衛官在人主左右直衛者也猶盛唐之侍官】自稱權福州軍府事擅發汀建撫信州兵及戍卒命建州監軍使馮延魯將之趣福州迎弘義【趣七喻翻】延魯先遺弘義書【遺惟季翻】諭以禍福弘義復書請戰遣樓船指揮使楊崇保將州師拒之【一本州師作舟師】覺以劒州刺史陳誨為緣江戰棹指揮使【建溪東流歷劒州至福州皆大江也故土人亦謂之為江】表福州孤危旦夕可克唐主以覺專命甚怒羣臣多言兵已傅城下【傅音附】不可中止當發兵助之丁丑覺延魯敗楊崇保於侯官【閩及侯官二縣皆治福州郭下此戰于侯官縣界也敗補賣翻】戊寅乘勝進攻福州西關弘義出擊大破之執唐左神威指揮使楊匡鄴唐主以永安節度使王崇文為東南面都招討使【去年十月唐置永安軍於建州】以漳泉安撫使諫議大夫魏岑為東面監軍使延魯為南面監軍使會兵攻福州克其外郭弘義固守第二城【第二重城也】 馮暉引兵過旱海至輝德【張洎曰自威州抵靈州旱海七百里斥鹵枯澤無溪澗川谷輝德地名在靈武南張舜民云今旱江平即旱海在清遠軍北趙珣聚米圖經曰鹽夏清遠軍間並係沙磧俗謂之旱海自環州出青剛川本靈州大路自此過美利寨漸入平夏經旱海中難得水泉至耀德清邉鎮入靈州】糗糧已盡【糗去久翻】拓拔彦超衆數萬為三陳扼要路據水泉以待之【陳讀曰陣下同】軍中大懼暉以賂求和於彦超彦超許之自旦至日中使者往返數四兵未解藥元福曰虜知我飢渇陽許和以困我耳若至暮則吾輩成擒矣今虜雖衆精兵不多依西山而陳者是也其餘步卒不足為患請公嚴陳以待我【嚴陳者嚴兵整陳也】我以精騎先犯西山兵小勝則舉黃旗大軍合勢擊之破之必矣乃帥騎先進用短兵力戰彦超小却元福舉黄旗暉引兵赴之彦超大敗【馮暉圈養拓拔彦超於靈武城中彦超固心知其故而懷怨暉去鎮而彦超得出彦超既得出而暉復來出柙之虎苟可以肆反噬者無所不至也非力戰而尅之馮暉之威令不可復行于朔方矣帥讀曰率】明日暉入靈州 九月契丹三萬寇河東壬辰劉知遠敗之於陽武谷【敗補邁翻】斬首七千級 漢劉思潮等既死陳道庠内不自安【陳道庠與劉思潮等同弑漢主弘度者也殺劉思潮等見去年九月】特進鄧伸遺之漢紀【按路振九國志陳道庠父璫與鄧伸有舊故然】道庠問其故伸曰憨獠【遺惟季翻憨呼談翻癡也獠盧皓翻又竹絞翻】此書有誅韓信醢彭越事宜審讀之漢主聞之族道庠及伸 李弘義自稱威武留後更名弘達奉表請命于晉【李弘義本名仁達弘義者唐所賜名也既叛唐遂更其名】甲午以弘達為威武節度使同平章事知閩國事 張彦澤奏敗契丹於定州北又敗之於泰州斬首二千級【敗補邁翻】 辛丑福州排陳使馬捷【陳讀曰陣】引唐兵自馬牧山拔寨而入至善化門橋都指揮使丁彦貞以兵百人拒之弘達退保善化門外城再重皆為唐兵所據弘達更名達【弘達更名達以吳越王名上從弘避之也重直龍翻更工衡翻】遣使奉表稱臣乞師于吳越 楚王希範知帝好奢靡【好呼到翻】屢以珍玩為獻求都元帥甲辰以希範為諸道兵馬都元帥 丙辰河決澶州臨黃【臨黄春秋衛河上之邑漢為東郡觀縣有衛宣公新臺後魏置臨黃縣唐屬澶州宋端拱元年省臨黃入觀城縣】 契丹使瀛州刺史劉延祚遺樂夀監軍王巒書請舉城内附【遺惟季翻 考異曰歐史作高牟翰按陷蕃記前云延祚詐輸誠欵後云大軍至瀛州偵知蕃將高模翰潜師而出蓋延祚為刺史模翰乃戍將耳今從陷蕃記】且云城中契丹兵不滿千人乞朝廷發輕兵襲之已為内應又今秋多雨自瓦橋以北積水無際契丹主已歸牙帳雖聞關南有變【瀛莫二州晉割屬契丹在瓦橋關南】地遠阻水不能救也巒與天雄節度使兼中書令杜威屢奏瀛莫乘此可取深州刺史慕容遷獻瀛莫圖馮玉李崧信以為然欲發大兵迎趙延夀及延祚【先是趙延夀亦詐通欵】先是侍衛馬步都指揮使天平節度使李守貞數將兵過廣晉【先音薦翻數所角翻過音戈魏州廣晉府】杜威厚待之贈金帛甲兵動以萬計守貞由是與威親善守貞入朝帝勞之曰【勞力到翻】聞卿為將常費私財以賞戰士對曰此皆杜威盡忠於國以金帛資臣臣安敢掠有其美因言陛下若它日用兵臣願與威戮力以清沙漠帝由是亦賢之及將北征帝與馮玉李崧議以威為元帥守貞副之趙瑩私謂馮李曰杜令國戚【謂尚公主也】貴為將相而所欲未厭心常慊慊【位兼將相謂居大鎮兼中書令未厭未滿所欲也慊慊亦不滿之意慊苦簟翻】豈可復假以兵權【復扶又翻】必若有事北方不若止任守貞為愈也【杜威之心迹雖趙瑩猶知之】不從冬十月辛未以威為北面行營都指揮使以守貞為兵馬都監【監古銜翻】泰寧節度使安審琦為左右廂都指揮使武寧節度使符彦卿為軍馬左廂都指揮使義成節度使皇甫遇為馬軍右廂都指揮使永清節度使梁漢璋為馬軍都排陳使前威勝節度使宋彦筠為步軍左廂都指揮使奉國左廂都指揮使王饒為步軍右廂都指揮使洺州團練使薛懷讓為先鋒都指揮使仍下勑牓曰專發大軍往平黠虜【黠下八翻】先取瀛莫安定關南次復幽燕盪平寨北又曰有擒獲虜主者除上鎮節度使賞錢萬緡絹萬匹銀萬兩【談何容易晉之君臣恃陽城之捷有輕祖契丹之心兵驕者敗自古而然】時自六月積雨至是未止軍行及饋運者甚艱苦 唐漳州將林贊堯作亂殺監軍使周承義劒州刺史陳誨泉州刺史留從效舉兵逐贊堯以泉州裨將董思安權知漳州唐主以思安為漳州刺史思安辭以父名章唐主改漳州為南州命思安及留從效將州兵會攻福州庚辰圍之福州使者至錢塘【乞師之使錢塘吳越國都】吳越王弘佐召諸將謀之皆曰道險遠難救惟内都監使臨安水丘昭劵以為當救【水丘複姓也何氏姓苑云漢有司隸校尉水丘岑今為臨安著姓】弘佐曰脣亡齒寒【古語多有之】吾為天下元帥曾不能救鄰道將安用之諸君但樂飽身安坐邪【樂音洛】壬午遣統軍張筠趙承泰將兵三萬水陸救福州【吳越救福州自㜈衢至建劒順流可至福州是時劒建已為南唐守此道不可由也自温州之平陽度海浦至福州界當由此道耳】先是募兵久無應者弘佐命糾之曰糾而為兵者糧賜減半明日應募者雲集弘佐命昭券專掌用兵昭劵憚程昭悦以用兵事讓之【程昭悦時為弘佐所寵任故水丘昭劵憚而讓之】弘佐命昭悦掌應援饋運事而以軍謀委元德昭德昭危仔倡之子也【危仔倡見二百六十三卷梁太祖開平三年】弘佐議鑄鐵錢以益將士禄賜其弟牙内都虞候弘億諫曰鑄鐵錢有八害新錢既行舊錢皆流入鄰國一也【舊錢謂銅錢】可用於吾國而不可用於它國則商賈不行百貨不通二也【賈音古】銅禁至嚴民猶盜鑄况家有鐺釜野有鏵犂犯法必多三也【鐺楚耕翻鏵戶花翻鏵鍫也】閩人鑄鐵錢而亂亡不足為法四也【閩鑄鐵錢見二百八十三卷天福七年及上七元年】國用幸豐而自示空乏五也【言鄰國聞之必將以為國用空乏而鑄鐵錢】禄賜有常而無故益之以啟無厭之心六也【厭於鹽反】法變而弊不可遽復七也錢者國姓易之不祥八也弘佐乃止 杜威李守貞會兵於廣晉而北行【李守貞引兵會杜威於魏州相與北行】威屢使公主入奏請益兵【公主者杜威妻宋國長公主帝之姑也】曰今深入虜境必資衆力由是禁軍皆在其麾下【杜威之計即趙德鈞請併范延光軍之計也德鈞不得請而威得請耳其志圖非望而敗國亡身則一也】而宿衛空虛十一月丁酉以李守貞權知幽州行府事己亥杜威等至瀛州城門洞啟寂若無人威等不敢進聞契丹將高謨翰先已引兵潜出威遣梁漢璋將二千騎追之遇契丹於南陽務敗死威等聞之引兵而南時束城等數縣請降【束城漢束州縣隋曰束城唐屬瀛州宋熙寧六年省束城縣為束城鎮屬河間縣】威等焚其廬舍掠其婦女而還【還從宣翻又如字】 己酉吳越兵至福州自罾浦南潜入州城【罾作滕翻魚網也福州之人就此罾魚因以得名】唐兵進據東武門李達與吳越兵共禦之不利自是内外斷絕城中益危唐主遣信州刺史王建封助攻福州時王崇文雖為元帥而陳覺馮延魯魏岑爭用事留從效王建封倔彊不用命【留從效起于泉州斬黃紹頗破李弘通唐人憚其威名王建封雖本唐將恃建州先登之功故皆倔彊不用命倔其勿翻彊其兩翻】各争功進退不相應由是將士皆解體故攻城不克唐主以江州觀察使杜昌業為吏部尚書判省事先是昌業自兵部尚書判省事出江州【判省事者判尚書省事】及還閱簿籍撫案歎曰未數年而所耗者半【言昌業出入之間未及數年而府庫之積已耗其半】其能久乎【言不能以支久也史言唐之府庫耗於用兵】 契丹主大舉入寇自易定趣恒州【趣七喻翻恒戶登翻】杜威等至武強【九域志武強縣在深州西四十五里宋白曰武強六國時武隧地屬趙故城在今縣東北三十里是為漢武強縣郡國縣道記云古武強縣城在今縣西南二十五里是為晉武強縣高齊移縣於後魏武邑郡故城今縣理是也】聞之將自貝冀而南彰德節度使張彦澤時在恒州【去年九月遣張彦澤戍恒州以備契丹恒戶登翻】引兵會之言契丹可破之狀威等復趣恒州【復扶又翻趣七喻翻】以彦澤為前鋒 【考異曰備史曰彦澤狼子其心密已變矣乃通欵邪律氏請為前導因促騎說威引軍沿滹沱水西援常山及至真定東垣渡與威通謀先遣步衆跨水不之救致敗將沮人心以行詭計因促監者高勲請降於虜按彦澤與威若已通款於契丹則彦澤何故猶奪橋契丹何故猶議回旋今不取】甲寅威等至中度橋【滹沱水逕恒州東南恒州之人各隨便為津渡之所此為中度者明上下流各冇度也】契丹已據橋彦澤帥騎爭之【帥讀曰率】契丹焚橋而退晉兵與契丹夾滹沱而軍始契丹見晉軍大至又爭橋不勝恐晉軍急渡滹沱與恒州合勢擊之議引兵還【還從宣翻又如字】及聞晉軍築壘為持久之計遂不去【知晉軍不敢戰也】 蜀施州刺史田行臯叛遣供奉官耿彦珣將兵討之 杜威雖以貴戚為上將性懦怯偏禆皆節度使【自李守貞至宋彦筠皆節度使也】但日相承迎置酒作樂罕議軍事磁州刺史兼北面轉運使李穀說威及李守貞曰【磁墻之翻說式芮翻】今大軍去恒州咫尺煙火相望若多以三股木置水中積薪布土其上橋可立成【三股木者用木三條交股縛之其下撑開為三足以寘水中】密約城中舉火相應夜募將士斫虜營而入表裏合勢虜必遁逃諸將皆以為然獨杜威不可遣穀南至懷孟督軍糧契丹以大軍當晉軍之前潜遣其將蕭翰通事劉重進將百騎及羸卒並西山出晉軍之後斷晉糧道及歸路【羸倫為翻並步浪翻斷音短】樵采者遇之盡為所掠有逸歸者皆稱虜衆之盛軍中忷懼翰等至欒城【忷許勇翻舊唐書地理志曰欒城縣漢常山郡之開縣也後魏於開縣古城置欒城縣屬趙州唐屬恒州九域志欒城縣在恒州南六十三里范成大北使録曰趙州三十里至欒城金人改趙州為沃州】城中戍兵千餘人不覺其至狼狽降之契丹獲晉民皆黥其面曰奉勑不殺縱之南走運夫在道遇之皆弃車驚潰翰契丹主之舅也【契丹后族皆以蕭為氏歐史曰翰契丹之大族其號阿巴翰之妹亦嫁契丹主德光而阿巴本無姓氏契丹呼翰為國舅既入汴將北歸以為宣武節度使李崧為製姓名曰蕭翰於是始姓蕭宋白曰蕭翰舒嚕阿巴之子】十二月丁巳朔李穀自書密奏具言大軍危急之勢請車駕幸滑州遣高行周符彦卿扈從及發兵守澶州河陽以備虜之奔衝遣軍將關勲走馬上之【高行周符彦卿一時名將也滑澶及河陽河津之要也使晉主能用李穀之言安得有張彦澤輕騎入汴之禍乎走馬上之急報也宋自寶元康定以前几邉鎮率有走馬承受之官從才用翻澶時連翻上時兩翻】己未帝始聞大軍屯中度【甲寅杜威等至中度己未大梁始聞之強寇深入諸軍孤危而驛報七日始達晉之為兵可知矣】是夕關勲至庚申杜威奏請益兵詔悉發守宫禁者得數百人赴之【自古以來重戰輕防未有不敗者也發數百人不足以增大軍之勢而重閉之防闕矣】又詔發河北及滑孟澤潞芻糧五十萬詣軍前【五十萬合束石之數言之】督迫嚴急所在鼎沸辛酉威又遣從者張祚等來告急【從才用翻】祚等還【還從宣翻】為契丹所獲自是朝廷與軍前聲問兩不相通時宿衛兵皆在行營人心懍懍【懔力錦翻】莫知為計開封尹桑維翰以國家危在旦夕求見帝言事【見賢遍翻】帝方在苑中調鷹【調鷹者調習之也使馴狎而附人】辭不見又詣執政言之執政不以為然【執政謂馮玉李彦韜等】退謂所親曰晉氏不血食矣【言晉必亡宗廟不祀盖晉氏之亡不獨桑維翰知之通國之人皆知之】帝欲自將北征李彦韜諫而止【將即亮翻】時符彦卿雖任行營職事帝留之使戍荆州口壬戌詔以歸德節度使高行周為北面都部署以彦卿副之共戍澶州以西京留守景延廣戌河陽且張形勢【史言三將戍河津雖張形勢而兵力甚弱】奉國都指揮使王清言於杜威曰今大軍去恒州五里守此何為營孤食盡勢將自潰請以步卒二千為前鋒奪橋開道公帥諸軍繼之得入恒州則無憂矣【帥讀曰率下同】威許諾遣清與宋彦筠俱進清戰甚鋭契丹不能支勢小却諸將請以大軍繼之威不許彦筠為契丹所敗【敗補邁翻】浮水抵岸得免清獨帥麾下陳於水北力戰互有殺傷屢請救於威威竟不遣一騎助之清謂其衆曰上將握兵【將即亮翻】坐觀吾輩困急而不救此必有異志吾輩當以死報國耳衆感其言莫有退者至暮戰不息契丹以新兵繼之清及士衆盡死【李穀為杜威畫計而不行猶可曰言之易而行之難至于王清力戰而不救則其欲賣國以圖己利心迹呈露人皆知之矣】由是諸軍皆奪氣清洺州人也甲子契丹遥以兵環晉營【環音宦】内外斷絕軍中食且盡杜威與李守貞宋彦筠謀降契丹威潜遣腹心詣契丹牙帳邀求重賞契丹主紿之曰趙延夀威望素淺恐不能帝中國汝果降者當以汝為之威喜遂定降計【趙延夀父子以是陷契丹杜威之才智未足以企延夀其墮契丹之計無足怪者覆轍相尋豈天意邪】丙寅伏甲召諸將出降表示之使署名諸將駭愕莫敢言者但唯唯聽命【唯于癸翻】威遣閤門使高勲齎詣契丹契丹主賜詔慰納之是日威悉命軍士出陳於外【陳讀曰陣】軍士皆踴躍以為且戰威親諭之曰今食盡塗窮當與汝曹共求生計因命釋甲軍士皆慟哭聲振原野【史言晉軍之心皆不欲降契丹迫于其帥而從之耳】威守貞仍於衆中揚言主上失德信任奸邪猜忌於己聞者無不切齒契丹主遣趙延夀衣赭袍至晉營慰撫士卒曰彼皆汝物也杜威以下皆迎謁於馬前亦以赭袍衣威以示晉軍其實皆戲之耳【契丹主非特戲杜威趙延夀也亦以愚晉軍彼其心知晉軍之不誠服也駕言將以華人為中國主是二人者必居一於此晉人謂喪君有君皆華人也夫是以不生心其計巧矣然契丹主巧於愚弄而入汴之後大不能制河東小不能制羣盜豈非挾數用術者有時而窮乎衣於既翻】以威為太傅李守貞為司徒威引契丹主至恒州城下諭順國節度使王周以已降之狀周亦出降戊辰契丹主入恒州遣兵襲代州刺史王暉以城降之【契丹以勝勢脅降代州而太原不為之動以劉知遠郭威在也九域志恒州西北至代州三百四十里】先是契丹屢攻易州刺史郭璘固守拒之【先悉薦翻璘離珍翻】契丹主每過城下指而歎曰吾能吞併天下而為此人所扼及杜威既降契丹主遣通事耿崇美至易州誘諭其衆【誘音酉】衆皆降璘不能制遂為崇美所殺【史言大厦之顛非一木所能支】璘邢州人也義武節度使李殷安國留後方太皆降於契丹契丹主以孫方簡為義武節度滿達勒為安國節度使【宋白曰滿達勒名解里按巴堅之從子也其父曰薩喇歸梁死于汴】以客省副使馬崇祚權知恒州事契丹翰林承旨吏部尚書張礪言於契丹主曰今大遼已得天下【高祖天福二年契丹改國號大遼事見二百八十一卷】中國將相宜用中國人為之不宜用北人及左右近習苟政令乖失則人心不服雖得之猶將失之契丹主不從【使契丹主用張礪言事未可知也】引兵自邢相而南【契丹之兵依山南下以臨晉相息亮翻】杜威將降兵以從【從才用翻或問杜威不降契丹晉可保乎曰設使杜威藉將士之力擊退契丹契丹主歸北完聚必復南來晉不能支也使其間有英雄之才奮然出力擊破契丹使之不敢南向則負震主之威挾不賞之功將士又將扶立以成簒事石氏必不能高枕大梁劉知遠亦不可得而狙伺其旁也】遣張彦澤將二千騎先取大梁且撫安吏民以通事傅珠爾為都監【監古銜翻】杜威之降也皇甫遇初不預謀契丹主欲遣遇先將兵入大梁遇辭退謂所親曰吾位為將相敗不能死忍復圖其主乎【復扶又翻下同】至平棘【平棘漢古縣唐帶趙州九域志曰平棘故城春秋棘蒲邑十三州志云戰國時改為平棘】謂從者曰吾不食累日矣何面目復南行遂扼吭而死【從才用翻吭古郎翻】張彦澤倍道疾驅夜度白馬津【張彦澤以澶孟有戍兵故從白馬津度】壬申帝始聞杜威等降是夕又聞彦澤至滑州召李崧馮玉李彦韜入禁中計事欲詔劉知遠發兵入援【太原距洛陽一千二百里洛陽至大梁又三百八十里就使劉知遠聞命投袂而起亦無及矣】癸酉未明彦澤自封丘門斬關而入李彦韜帥禁兵五百赴之不能遏【帥讀曰率】彦澤頓兵明德門外【五代會要曰明德門大梁皇城南門薛史天福三年十月改大寧宫門為明德門】城中大擾帝於宫中起火自攜劒驅後宫十餘人將赴火為親軍將薛超所持俄而彦澤自寛仁門傳契丹主與太后書慰撫之【五代會要曰大梁皇城之東門為寛仁門】且召桑維翰景延廣帝乃命滅火悉開宫城門帝坐苑中與后妃相聚而泣召翰林學士范質草降表自稱孫男臣重貴禍至神惑運盡天亡今與太后及妻馮氏舉族於郊野面縛待罪次遣男鎮寧節度使延煦威信節度使延寶奉國寶一金印三出迎【國寶即高祖天福三年所制受命寶也煦吁句翻】太后亦上表稱新婦李氏妾【臣妾之辱惟晉宋為然嗚呼痛哉上時掌翻】傅珠爾入宣契丹主命帝脫黄袍服素衫再拜受宣左右皆掩泣帝使召張彦澤欲與計事彦澤曰臣無面目見陛下帝復召之【復扶又翻】彦澤微笑不應或勸桑維翰逃去維翰曰吾大臣逃將安之坐而俟命彦澤以帝命召維翰維翰至天街【宫城正南門外之都街謂之天街經涂也】遇李崧駐馬語未畢有軍吏於馬前揖維翰赴侍衛司【揖赴侍衛司示將囚繫之也一曰時張彦澤處侍衛司署舍】維翰知不免顧謂松曰侍中當國【李崧官侍中】今日國亡反令維翰死之何也崧有愧色彦澤踞坐見維翰維翰責之曰去年拔公於罪人之中復領大鎮授以兵權【謂高祖時朝野皆請誅張彦澤自涇州罷歸宿衛去年桑維翰拔使同禦契丹復領彰國節度使帥兵戍常山】何乃負恩至此彦澤無以應遣兵守之宣徽使孟承誨素以佞巧有寵於帝至是帝召承誨欲與之謀承誨伏匿不至張彦澤捕而殺之彦澤縱兵大掠貧民乘之亦爭入富室殺人取其貨二日方止都城為之一空【為于偽翻下為主同】彦澤所居山積自謂有功於契丹【張彦澤自以疾驅入汴為功】晝夜以酒樂自娛出入騎從常數百人【從才用翻】其旗幟皆題赤心為主見者笑之軍士擒罪人至前彦澤不問所犯但瞋目豎三指即驅出斷其腰領【瞋昌真翻豎而主翻三指中指也示以中指言中斷之即腰斬也此蓋五代軍中虐帥相仍為此以示其下罪之輕重決於一指屈伸之間及漢史弘肇掌兵有抵罪者弘肇以三指示吏即腰斬之正此類也】彦澤素與閤門使高勲不協乘醉至其家殺其叔父及弟尸諸門首士民不寒而慄中書舍人李濤謂人曰吾與其逃於溝瀆而不免不若往見之乃投刺謁彦澤曰上書請殺太尉人李濤謹來請死【李濤請殺張彦澤事見二百八十三卷高祖天福七年】彦澤欣然接之謂濤曰舍人今日懼乎濤曰濤今日之懼亦猶足下昔年之懼也曏使高祖用濤言事安至此彦澤大笑命酒飲之【飲於禁翻】濤引滿而去旁若無人【李濤者回之族曾孫明辯有膽氣固自有種】甲戌張彦澤遷帝於開封府頃刻不得留宫中慟哭帝與太后皇后乘肩輿宫人宦者十餘人步從【從才用翻】見者流涕【亡國之恥言之者為之痛心矧見之者乎此程正叔所謂眞知者也天乎人乎】帝悉以内庫金珠自隨彦澤使人諷之曰契丹主至此物不可匿也帝悉歸之亦分以遺彦澤【遺惟季翻】彦澤擇取其奇貨而封其餘以待契丹彦澤遣控鶴指揮使李筠以兵守帝内外不通帝姑烏氏公主賂守門者入與帝訣歸第自經【氏音支按薛史烏氏公主高祖第十一妹也】帝與太后所上契丹主表章【上時掌翻下同】皆先示彦澤然後敢發帝使取内庫帛數段主者不與曰此非帝物也又求酒於李崧崧亦辭以它故不進又欲見李彦韜彦韜亦辭不往帝惆悵久之【當是時晉朝之臣已視出帝為路人雖惆悵亦何及矣惆丑鳩翻】馮玉佞張彦澤求自送傳國寶冀契丹復任用【亡國之臣其識正如此耳復扶又翻】楚國夫人丁氏延煦之母也有美色彦澤使人取之太后遲迴未與彦澤詬詈立載之去【詬苦候翻又許候翻詈力智翻】是夕彦澤殺桑維翰 【考異曰薛史帝思維翰在相時累貢謀畫請與虜和慮戎主到則顯彰已過欲殺維翰以滅口因令張彦澤殺之按是時彦澤豈肯復從少帝之命今不取】以帶加頸白契丹主云其自經契丹主曰吾無意殺維翰何為如是命厚撫其家高行周符彦卿皆詣契丹牙帳降【二人自澶州來降】契丹主以陽城之戰為彦卿所敗詰之【陽城之戰見上卷上年敗補賣翻詰去吉翻】彦卿曰臣當時惟知為晉主竭力今日死生惟命契丹主笑而釋之【符彦卿言直契丹主無以罪也為于偽翻】己卯延煦延寶自牙帳還【還從宣翻又如字】契丹主賜帝手詔且遣轄里謂帝曰孫勿憂必使汝有噉飯之所【噉徒濫翻】帝心稍安上表謝恩契丹以所獻傳國寶追琢非工又不與前史相應【追都回翻其文不與前史相應也】疑其非真以詔書詰帝使獻真者【李心傳曰秦璽者李斯之蟲魚篆也其圍四寸按玉璽圖以此璽為趙璧所刻璧本卞和所獻之璞藺相如所奪者是也余嘗以禮制考之璧五寸而有好則不得復刻為璽此說謬矣秦璽至漢謂之傳國璽自是迄于漢帝所寶用者秦璽也子嬰所封元后所投王憲所得赤眉所上皆是物也董卓之亂失之吳書謂孫堅得之洛陽甄官井中復為袁術所奪徐璆得而上之殆不然也若然則魏氏何不實用而自刻璽乎厥後歷世皆用其名永嘉之亂没于劉石永和之世復歸江左者晉璽也魏氏有國刻傳國璽如秦之文但秦璽讀自右魏璽讀自左耳晉有天下又自刻璽其文曰受命于天皇帝夀昌本書輿服志乃以為漢所傳秦璽實甚誤矣此璽更劉聰石勒逮石祗死其臣蔣幹求援於謝尚乃以璽送江南王彪之辯之亦不云秦璽也太元之末得自西燕更涉六朝至于隋代者慕容燕璽也晉孝武太元十九年西燕主永求救於郄恢併獻玉璽一紐方闊六寸高四寸六分文如秦璽自是歷宋齊梁皆寶之侯景既死北齊辛術得之廣陵獻之高氏後歷周隋皆誤指為秦璽後平江南知其非是乃更謂之神璽焉劉裕北伐得之關中歷晉暨陳復為隋有者姚秦璽也晉義熙十三年劉裕入關得傳國璽上之大四寸文與秦同然隱起而不深刻隋滅陳得此指為真璽遂以宇文所傳神璽為非是識者又謂古璽深刻以印泥後人隱起以印紙則隱起者非秦璽也姚氏取其文作之耳開運之亂没于邪律女真獲之以為大寶者石晉璽也唐太宗貞觀十六年刻受命璽文曰皇帝景命有德者昌後歸朱全忠及從珂自焚璽亦隨失德光入汴重貴以璽上之云先帝所刻蓋指敬瑭也蓋在唐時皆誤以為秦璽而秦璽之亡則久矣今按石袛死當作冉閔死李心傳之說與唐六典異今並存之以俟知者及周又製二寶有司所奏其說亦祖六典詳注于後詰其吉翻】帝奏頃王從珂自焚【事見二百八十卷高祖天福元年】舊傳國寶不知所在必與之俱燼此寶先帝所為【事見二百八十一卷天福三年】羣臣備知臣今日焉敢匿寶乃止【焉於䖍翻】帝聞契丹主將度河欲與太后於前途奉迎張彦澤先奏之契丹主不許有司又欲使帝衘璧牽羊大臣輿櫬迎於郊外先具儀注白契丹主契丹主曰吾遣奇兵直取大梁非受降也亦不許【降戶江翻】又詔晉文武羣臣一切如故朝廷制度並用漢禮【北方謂中國為漢】有司欲備法駕迎契丹主契丹主報曰吾方擐甲總戎【擐音宦】太常儀衛未暇施也皆却之【用太常儀衛則當改胡服而華服故言未暇】先是契丹主至相州即遣兵趣河陽捕景延廣延廣蒼猝無所逃伏【不料其遽見捕也先昔薦翻相息亮翻趣七喻翻】往見契丹主於封丘【九域志封丘縣在大梁北六十里】契丹主詰之曰致兩主失歡皆汝所為也十萬橫磨劒安在召喬榮使相辯證事凡十條延廣初不服榮以紙所記語示之【景延廣記其所言以授喬榮見二百八十二卷天福天年】乃服每服一事輒授一籌至八籌延廣但以面伏地請死乃鎖之丙戌晦百官宿於封禪寺【迎契丹主也封禪寺在大梁城東】<br />
<br />
  資治通鑑卷二百八十五<br />
<br />
<史部,編年類,資治通鑑>  <br>
   </div> 

<script src="/search/ajaxskft.js"> </script>
 <div class="clear"></div>
<br>
<br>
 <!-- a.d-->

 <!--
<div class="info_share">
</div> 
-->
 <!--info_share--></div>   <!-- end info_content-->
  </div> <!-- end l-->

<div class="r">   <!--r-->



<div class="sidebar"  style="margin-bottom:2px;">

 
<div class="sidebar_title">工具类大全</div>
<div class="sidebar_info">
<strong><a href="http://www.guoxuedashi.com/lsditu/" target="_blank">历史地图</a></strong>  
<a href="http://www.880114.com/" target="_blank">英语宝典</a>  
<a href="http://www.guoxuedashi.com/13jing/" target="_blank">十三经检索</a> 
<br><strong><a href="http://www.guoxuedashi.com/gjtsjc/" target="_blank">古今图书集成</a></strong> 
<a href="http://www.guoxuedashi.com/duilian/" target="_blank">对联大全</a> <strong><a href="http://www.guoxuedashi.com/xiangxingzi/" target="_blank">象形文字典</a></strong> 

<br><a href="http://www.guoxuedashi.com/zixing/yanbian/">字形演变</a>  <strong><a href="http://www.guoxuemi.com/hafo/" target="_blank">哈佛燕京中文善本特藏</a></strong>
<br><strong><a href="http://www.guoxuedashi.com/csfz/" target="_blank">丛书&方志检索器</a></strong> <a href="http://www.guoxuedashi.com/yqjyy/" target="_blank">一切经音义</a>  

<br><strong><a href="http://www.guoxuedashi.com/jiapu/" target="_blank">家谱族谱查询</a></strong>  <strong><a href="http://shufa.guoxuedashi.com/sfzitie/" target="_blank">书法字帖欣赏</a></strong> 
<br>

</div>
</div>


<div class="sidebar" style="margin-bottom:0px;">

<font style="font-size:22px;line-height:32px">QQ交流群9:489193090</font>


<div class="sidebar_title">手机APP 扫描或点击</div>
<div class="sidebar_info">
<table>
<tr>
	<td width=160><a href="http://m.guoxuedashi.com/app/" target="_blank"><img src="/img/gxds-sj.png" width="140"  border="0" alt="国学大师手机版"></a></td>
	<td>
<a href="http://www.guoxuedashi.com/download/" target="_blank">app软件下载专区</a><br>
<a href="http://www.guoxuedashi.com/download/gxds.php" target="_blank">《国学大师》下载</a><br>
<a href="http://www.guoxuedashi.com/download/kxzd.php" target="_blank">《汉字宝典》下载</a><br>
<a href="http://www.guoxuedashi.com/download/scqbd.php" target="_blank">《诗词曲宝典》下载</a><br>
<a href="http://www.guoxuedashi.com/SiKuQuanShu/skqs.php" target="_blank">《四库全书》下载</a><br>
</td>
</tr>
</table>

</div>
</div>


<div class="sidebar2">
<center>


</center>
</div>

<div class="sidebar"  style="margin-bottom:2px;">
<div class="sidebar_title">网站使用教程</div>
<div class="sidebar_info">
<a href="http://www.guoxuedashi.com/help/gjsearch.php" target="_blank">如何在国学大师网下载古籍?</a><br>
<a href="http://www.guoxuedashi.com/zidian/bujian/bjjc.php" target="_blank">如何使用部件查字法快速查字?</a><br>
<a href="http://www.guoxuedashi.com/search/sjc.php" target="_blank">如何在指定的书籍中全文检索?</a><br>
<a href="http://www.guoxuedashi.com/search/skjc.php" target="_blank">如何找到一句话在《四库全书》哪一页?</a><br>
</div>
</div>


<div class="sidebar">
<div class="sidebar_title">热门书籍</div>
<div class="sidebar_info">
<a href="/so.php?sokey=%E8%B5%84%E6%B2%BB%E9%80%9A%E9%89%B4&kt=1">资治通鉴</a> <a href="/24shi/"><strong>二十四史</strong></a>&nbsp; <a href="/a2694/">野史</a>&nbsp; <a href="/SiKuQuanShu/"><strong>四库全书</strong></a>&nbsp;<a href="http://www.guoxuedashi.com/SiKuQuanShu/fanti/">繁体</a>
<br><a href="/so.php?sokey=%E7%BA%A2%E6%A5%BC%E6%A2%A6&kt=1">红楼梦</a> <a href="/a/1858x/">三国演义</a> <a href="/a/1038k/">水浒传</a> <a href="/a/1046t/">西游记</a> <a href="/a/1914o/">封神演义</a>
<br>
<a href="http://www.guoxuedashi.com/so.php?sokeygx=%E4%B8%87%E6%9C%89%E6%96%87%E5%BA%93&submit=&kt=1">万有文库</a> <a href="/a/780t/">古文观止</a> <a href="/a/1024l/">文心雕龙</a> <a href="/a/1704n/">全唐诗</a> <a href="/a/1705h/">全宋词</a>
<br><a href="http://www.guoxuedashi.com/so.php?sokeygx=%E7%99%BE%E8%A1%B2%E6%9C%AC%E4%BA%8C%E5%8D%81%E5%9B%9B%E5%8F%B2&submit=&kt=1"><strong>百衲本二十四史</strong></a>  <a href="http://www.guoxuedashi.com/so.php?sokeygx=%E5%8F%A4%E4%BB%8A%E5%9B%BE%E4%B9%A6%E9%9B%86%E6%88%90&submit=&kt=1"><strong>古今图书集成</strong></a>
<br>

<a href="http://www.guoxuedashi.com/so.php?sokeygx=%E4%B8%9B%E4%B9%A6%E9%9B%86%E6%88%90&submit=&kt=1">丛书集成</a> 
<a href="http://www.guoxuedashi.com/so.php?sokeygx=%E5%9B%9B%E9%83%A8%E4%B8%9B%E5%88%8A&submit=&kt=1"><strong>四部丛刊</strong></a>  
<a href="http://www.guoxuedashi.com/so.php?sokeygx=%E8%AF%B4%E6%96%87%E8%A7%A3%E5%AD%97&submit=&kt=1">說文解字</a> <a href="http://www.guoxuedashi.com/so.php?sokeygx=%E5%85%A8%E4%B8%8A%E5%8F%A4&submit=&kt=1">三国六朝文</a>
<br><a href="http://www.guoxuedashi.com/so.php?sokeytm=%E6%97%A5%E6%9C%AC%E5%86%85%E9%98%81%E6%96%87%E5%BA%93&submit=&kt=1"><strong>日本内阁文库</strong></a> <a href="http://www.guoxuedashi.com/so.php?sokeytm=%E5%9B%BD%E5%9B%BE%E6%96%B9%E5%BF%97%E5%90%88%E9%9B%86&ka=100&submit=">国图方志合集</a> <a href="http://www.guoxuedashi.com/so.php?sokeytm=%E5%90%84%E5%9C%B0%E6%96%B9%E5%BF%97&submit=&kt=1"><strong>各地方志</strong></a>

</div>
</div>


<div class="sidebar2">
<center>

</center>
</div>
<div class="sidebar greenbar">
<div class="sidebar_title green">四库全书</div>
<div class="sidebar_info">

《四库全书》是中国古代最大的丛书,编撰于乾隆年间,由纪昀等360多位高官、学者编撰,3800多人抄写,费时十三年编成。丛书分经、史、子、集四部,故名四库。共有3500多种书,7.9万卷,3.6万册,约8亿字,基本上囊括了古代所有图书,故称“全书”。<a href="http://www.guoxuedashi.com/SiKuQuanShu/">详细>>
</a>

</div> 
</div>

</div>  <!--end r-->

</div>
<!-- 内容区END --> 

<!-- 页脚开始 -->
<div class="shh">

</div>

<div class="w1180" style="margin-top:8px;">
<center><script src="http://www.guoxuedashi.com/img/plus.php?id=3"></script></center>
</div>
<div class="w1180 foot">
<a href="/b/thanks.php">特别致谢</a> | <a href="javascript:window.external.AddFavorite(document.location.href,document.title);">收藏本站</a> | <a href="#">欢迎投稿</a> | <a href="http://www.guoxuedashi.com/forum/">意见建议</a> | <a href="http://www.guoxuemi.com/">国学迷</a> | <a href="http://www.shuowen.net/">说文网</a><script language="javascript" type="text/javascript" src="https://js.users.51.la/17753172.js"></script><br />
  Copyright &copy; 国学大师 古典图书集成 All Rights Reserved.<br>
  
  <span style="font-size:14px">免责声明:本站非营利性站点,以方便网友为主,仅供学习研究。<br>内容由热心网友提供和网上收集,不保留版权。若侵犯了您的权益,来信即刪。scp168@qq.com</span>
  <br />
ICP证:<a href="http://www.beian.miit.gov.cn/" target="_blank">鲁ICP备19060063号</a></div>
<!-- 页脚END --> 
<script src="http://www.guoxuedashi.com/img/plus.php?id=22"></script>
<script src="http://www.guoxuedashi.com/img/tongji.js"></script>

</body>
</html>
