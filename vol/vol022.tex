<!DOCTYPE html PUBLIC "-//W3C//DTD XHTML 1.0 Transitional//EN" "http://www.w3.org/TR/xhtml1/DTD/xhtml1-transitional.dtd">
<html xmlns="http://www.w3.org/1999/xhtml">
<head>
<meta http-equiv="Content-Type" content="text/html; charset=utf-8" />
<meta http-equiv="X-UA-Compatible" content="IE=Edge,chrome=1">
<title>資治通鑒_23-資治通鑑卷二十二_23-資治通鑑卷二十二</title>
<meta name="Keywords" content="資治通鑒_23-資治通鑑卷二十二_23-資治通鑑卷二十二">
<meta name="Description" content="資治通鑒_23-資治通鑑卷二十二_23-資治通鑑卷二十二">
<meta http-equiv="Cache-Control" content="no-transform" />
<meta http-equiv="Cache-Control" content="no-siteapp" />
<link href="/img/style.css" rel="stylesheet" type="text/css" />
<script src="/img/m.js?2020"></script> 
</head>
<body>
 <div class="ClassNavi">
<a  href="/24shi/">二十四史</a> | <a href="/SiKuQuanShu/">四库全书</a> | <a href="http://www.guoxuedashi.com/gjtsjc/"><font  color="#FF0000">古今图书集成</font></a> | <a href="/renwu/">历史人物</a> | <a href="/ShuoWenJieZi/"><font  color="#FF0000">说文解字</a></font> | <a href="/chengyu/">成语词典</a> | <a  target="_blank"  href="http://www.guoxuedashi.com/jgwhj/"><font  color="#FF0000">甲骨文合集</font></a> | <a href="/yzjwjc/"><font  color="#FF0000">殷周金文集成</font></a> | <a href="/xiangxingzi/"><font color="#0000FF">象形字典</font></a> | <a href="/13jing/"><font  color="#FF0000">十三经索引</font></a> | <a href="/zixing/"><font  color="#FF0000">字体转换器</font></a> | <a href="/zidian/xz/"><font color="#0000FF">篆书识别</font></a> | <a href="/jinfanyi/">近义反义词</a> | <a href="/duilian/">对联大全</a> | <a href="/jiapu/"><font  color="#0000FF">家谱族谱查询</font></a> | <a href="http://www.guoxuemi.com/hafo/" target="_blank" ><font color="#FF0000">哈佛古籍</font></a> 
</div>

 <!-- 头部导航开始 -->
<div class="w1180 head clearfix">
  <div class="head_logo l"><a title="国学大师官网" href="http://www.guoxuedashi.com" target="_blank"></a></div>
  <div class="head_sr l">
  <div id="head1">
  
  <a href="http://www.guoxuedashi.com/zidian/bujian/" target="_blank" ><img src="http://www.guoxuedashi.com/img/top1.gif" width="88" height="60" border="0" title="部件查字,支持20万汉字"></a>


<a href="http://www.guoxuedashi.com/help/yingpan.php" target="_blank"><img src="http://www.guoxuedashi.com/img/top230.gif" width="600" height="62" border="0" ></a>


  </div>
  <div id="head3"><a href="javascript:" onClick="javascript:window.external.AddFavorite(window.location.href,document.title);">添加收藏</a>
  <br><a href="/help/setie.php">搜索引擎</a>
  <br><a href="/help/zanzhu.php">赞助本站</a></div>
  <div id="head2">
 <a href="http://www.guoxuemi.com/" target="_blank"><img src="http://www.guoxuedashi.com/img/guoxuemi.gif" width="95" height="62" border="0" style="margin-left:2px;" title="国学迷"></a>
  

  </div>
</div>
  <div class="clear"></div>
  <div class="head_nav">
  <p><a href="/">首页</a> | <a href="/ShuKu/">国学书库</a> | <a href="/guji/">影印古籍</a> | <a href="/shici/">诗词宝典</a> | <a   href="/SiKuQuanShu/gxjx.php">精选</a> <b>|</b> <a href="/zidian/">汉语字典</a> | <a href="/hydcd/">汉语词典</a> | <a href="http://www.guoxuedashi.com/zidian/bujian/"><font  color="#CC0066">部件查字</font></a> | <a href="http://www.sfds.cn/"><font  color="#CC0066">书法大师</font></a> | <a href="/jgwhj/">甲骨文</a> <b>|</b> <a href="/b/4/"><font  color="#CC0066">解密</font></a> | <a href="/renwu/">历史人物</a> | <a href="/diangu/">历史典故</a> | <a href="/xingshi/">姓氏</a> | <a href="/minzu/">民族</a> <b>|</b> <a href="/mz/"><font  color="#CC0066">世界名著</font></a> | <a href="/download/">软件下载</a>
</p>
<p><a href="/b/"><font  color="#CC0066">历史</font></a> | <a href="http://skqs.guoxuedashi.com/" target="_blank">四库全书</a> |  <a href="http://www.guoxuedashi.com/search/" target="_blank"><font  color="#CC0066">全文检索</font></a> | <a href="http://www.guoxuedashi.com/shumu/">古籍书目</a> | <a   href="/24shi/">正史</a> <b>|</b> <a href="/chengyu/">成语词典</a> | <a href="/kangxi/" title="康熙字典">康熙字典</a> | <a href="/ShuoWenJieZi/">说文解字</a> | <a href="/zixing/yanbian/">字形演变</a> | <a href="/yzjwjc/">金 文</a> <b>|</b>  <a href="/shijian/nian-hao/">年号</a> | <a href="/diming/">历史地名</a> | <a href="/shijian/">历史事件</a> | <a href="/guanzhi/">官职</a> | <a href="/lishi/">知识</a> <b>|</b> <a href="/zhongyi/">中医中药</a> | <a href="http://www.guoxuedashi.com/forum/">留言反馈</a>
</p>
  </div>
</div>
<!-- 头部导航END --> 
<!-- 内容区开始 --> 
<div class="w1180 clearfix">
  <div class="info l">
   
<div class="clearfix" style="background:#f5faff;">
<script src='http://www.guoxuedashi.com/img/headersou.js'></script>

</div>
  <div class="info_tree"><a href="http://www.guoxuedashi.com">首页</a> > <a href="/SiKuQuanShu/fanti/">四库全书</a>
 > <h1>资治通鉴</h1> <!--         下载:【右键另存为】即可 --></div>
  <div class="info_content zj clearfix">
  
<div class="info_txt clearfix" id="show">
<center style="font-size:24px;">23-資治通鑑卷二十二</center>
    資治通鑑卷二十二    宋 司馬光 撰<br />
<br />
  胡三省 音註<br />
<br />
  漢紀十四【起昭陽恊洽盡閼逢敦牂凡十二年】<br />
<br />
  世宗孝武皇帝下之下<br />
<br />
  天漢三年春二月王卿有罪自殺以執金吾杜周為御史大夫【班表中尉掌徼循京師太初元年更名執金吾應劭曰吾禦也掌執金革以禦非常師古曰金吾鳥名主辟不祥天子出行主先導以備非常故執此鳥之象因以名官】 初榷酒酤【如淳曰榷音較應劭曰縣官自酤榷賣酒小民不復得酤也韋昭曰以木渡水曰榷謂禁民酤釀獨官開置如道路設木為榷獨取利也師古曰榷者步渡橋爾雅謂之石杜今之畧彴是也禁閉其事揔利入官而下無由以得有若渡水之榷因立名焉酤工護翻彴音酌】 三月上行幸泰山脩封祀明堂因受計還祠常山瘞玄玉【鄧展曰瘞埋也爾雅曰祭地曰瘞薶薶其物者示歸於地也瘞音於例翻】方士之祠神人入海求蓬莱者終無有驗而公孫卿猶以大人跡為解【大人跡見二十卷元封元年】天子益怠厭方士之怪迂語矣然猶覊縻不絶【師古曰覊縻牽聯之意馬絡頭曰覊牛靷曰縻】冀遇其真自此之後方士言神祠者彌衆然其效可睹矣 夏四月大旱赦天下 秋匈奴入鴈門【鴈門郡屬并州】太守坐畏愞棄市【如淳曰軍法行逗留畏愞者要斬愞加椽翻師古曰又音乃舘翻】四年春正月朝諸侯王于甘泉宫 發天下七科讁【張晏曰吏有罪一亡命二贅婿三賈人四故有市籍五父母有市籍六大父母有市籍七凡七科也】及勇敢士遣貳師將軍李廣利將騎六萬步兵七萬出朔方【朔方郡屬朔方州唐靈夏州地】彊弩都尉路博德將萬餘人與貳師會游擊將軍韓說將步兵三萬人出五原因杅將軍公孫敖將騎萬步兵三萬人出鴈門匈奴聞之悉遠其累重於余吾水北【師古曰累重謂妻子資產也累力瑞翻重直用翻余吾水在朔方北山海經曰北鮮之山鮮水出焉北流注於余吾】而單于以兵十萬待水南與貳師接戰貳師解而引歸與單于連闘十餘日 【考異曰史記匈奴傳云廣利於此降匈奴誤】游擊無所得因杅與左賢王戰不利引歸時上遣敖深入匈奴迎李陵敖軍無功還因曰捕得生口言李陵教單于為兵以備漢軍故臣無所得上於是族陵家既而聞之乃漢將降匈奴者李緒非陵也陵使人刺殺緒【降戶江翻刺七亦翻】大閼氏欲殺陵【師古曰大閼氏單于之母閼氏音煙支】單于匿之北方大閼氏死乃還單于以女妻陵【妻千細翻】立為右校王【校戶教翻】與衛律皆貴用事衛律常在單于左右陵居外有大事乃入議 夏四月立皇子髆為昌邑王【髆音博昌邑國屬兖州即山陽郡地其地在唐之宋亳單鄆四州問考異曰表云六月乙丑立今從武紀】<br />
<br />
  太始元年【應劭曰言盪滌天下與民更始故以冠元】春正月公孫敖坐妻為巫蠱要斬【巫祝也蠱厭也惑也謂使巫祠祭祝詛厭魅以蠱惑人也蠱音古孔頴逹曰蠱者損壞之名故左傳云皿蟲為蠱是蠱食器皿巫行邪術損壞於人要與腰同】 徙郡國豪桀于茂陵 夏六月赦天下 是歲匈奴且鞮侯單于死【且子余翻鞮田黎翻】有兩子長為左賢王次為左大將【匈奴二十四長左賢王位第一左大將位第五長子兩翻】左賢王未知貴人以為有病更立左大將為單于左賢王聞之不敢進左大將使人召左賢王而讓位焉左賢王辭以病左大將不聽謂曰即不幸死傳之於我左賢王許之遂立為狐鹿姑單于以左大將為左賢王數年病死其子先賢撣不得代更以為日逐王【師古曰撣音㕓日逐王居匈奴西邊以日入於西故以為名至宣帝神爵二年撣來降】單于自以其子為左賢王<br />
<br />
  二年春正月上行幸回中 杜周卒光祿大夫暴勝之為御史大夫 秋旱 趙中大夫白公奏穿渠引涇水首起谷口尾入櫟陽【班志谷口櫟陽二縣屬左馮翊師古曰谷口即今雲陽縣杜佑曰今雲陽縣治谷是又曰醴泉漢谷口縣地隋為醴泉縣谷口縣故城在縣西北櫟音藥】注渭中袤二百里【師古曰袤音茂長也】溉田四千五百餘頃因名曰白渠民得其饒<br />
<br />
  三年春正月上行幸甘泉宮二月幸東海獲赤鴈幸琅邪【東海琅邪二郡皆屬徐州琅邪唐沂密州也】禮日成山【孟康曰禮日拜日也如淳曰拜日於成山師古曰成山在東莱不夜縣斗入海】登之罘【臣瓚曰地理志東莱腄縣有之罘山師古曰罘音浮】浮大海而還【還從宣翻又如字】 是歲皇子弗陵生弗陵母曰河間趙倢伃【河間國屬冀州唐瀛莫州地帝置倢伃位視上卿爵比列侯師古曰倢言接幸於上也伃美貌倢音接伃音予】居鉤弋宮【師古曰黄圖鉤弋宮在城外漢武故事在直門内】任身十四月而生【任讀曰姙】上曰聞昔堯十四月而生今鉤弋亦然乃命其所生門曰堯母門<br />
<br />
  臣光曰為人君者動静舉措不可不慎發於中必形於外天下無不知之當是時也皇后太子皆無恙【恙余亮翻】而命鉤弋之門曰堯母非名也是以姦人逆探上意知其奇愛少子欲以為嗣【少詩照翻】遂有危皇后太子之心卒成巫蠱之禍【卒子恤翻】悲夫<br />
<br />
  趙人江充為水衡都尉【趙國屬冀州唐為冀州其地又分入深州德州界元鼎二年初置水衡都尉掌上林苑應劭曰古山林之官曰衡掌諸池苑故稱水衡張晏曰主都水及上林故稱水衡主諸官故曰都有卒徒武事故曰尉師古曰衡平也主平其税入位列九卿秩中二千石】初充為趙敬肅王客【敬肅王名彭祖薨謚敬肅】得罪於太子丹亡逃詣闕告趙太子隂事太子坐廢上召充入見【見賢遍翻】充容貌魁岸被服輕靡【師古曰魁大也岸者有廉稜如崖岸之狀被服衣服也輕輕細也靡靡麗也被皮義翻】上奇之與語政事大悦由是有寵拜為直指繡衣使者使督察貴戚近臣踰侈者充舉劾無所避【劾戶槩翻】上以為忠直所言皆中意【師古口中當也中竹仲翻】嘗從上甘泉【上時掌翻】逢太子家使乘車馬行馳道中充以屬吏【應劭曰馳道天子所行道也若今之中道也孔穎逹曰馳道正道御路也是天子馳走車馬之處故曰馳道如淳曰令乙騎乘車馬行馳道中已論者没入車馬被具師古曰家使太子遣人之甘泉請問者也使疏吏翻屬之欲翻】太子聞之使人謝充曰非愛車馬誠不欲令上聞之以教敕亡素者【師古曰言素不教敕左右古字亡與無通】唯江君寛之充不聽遂白奏上曰人臣當如是矣大見信用威震京師<br />
<br />
  四年春三月上行幸泰山壬午祀高祖于明堂以配上帝因受計癸未祀孝景皇帝于明堂甲申修封丙戌禪石閭夏四月幸不其【如淳曰其音基不其山名因以為縣應劭曰東莱縣也余據班志不其縣屬琅邪郡】五月還幸建章宫赦天下 冬十月甲寅晦日有食之 十二月上行幸雍祠五畤【雍於用翻畤音止】西至安定北地【二郡屬朔方州安定唐涇原之地北地唐邠寧環慶鹽宥州地】<br />
<br />
  征和元年【應劭曰言征伐四夷而天下和平】春正月上還幸建章宫三月趙敬肅王彭祖薨彭祖取江都易王所幸淖姬【彭祖景帝子前二年封廣川五年徙趙淖姬事見十九卷元狩二年淖奴教翻】生男號淖子時淖姬兄為漢宦者上召問淖子何如對曰為人多欲上曰多欲不宜君國子民問武始侯昌【昌亦彭祖之子班志武始縣屬魏郡】曰無咎無譽【譽音余】上曰如是可矣遣使者立昌為趙王夏大旱 上居建章宫見一男子帶劒入中龍華門疑其異人命收之男子捐劒走逐之弗獲上怒斬門候【門候掌宫門出入之禁續漢志秩六百石】冬十一月發三輔騎士大搜上林閉長安城門索【臣瓚曰搜謂索姦人也上林苑周回數百里恐姦人藏匿其中故大搜索索山客翻】十一月乃解巫蠱始起 丞相公孫賀夫人君孺衛皇后姊也賀由是有寵賀子敬聲代父為太僕驕奢不奉法擅用北軍錢千九百萬發覺下獄【下遐嫁翻下同】是時詔捕陽陵大俠朱安世甚急賀自請逐捕安世以贖敬聲罪上許之後果得安世安世笑曰丞相禍及宗矣遂從獄中上書告敬聲與陽石公主私通【陽石公主帝女也班志陽石屬北海郡上書時掌翻下且上同】上且上甘泉使巫當馳道埋偶人祝詛上有惡言【師古曰刻木為人象人之形謂之偶人偶並也對也祝職救翻詛莊助翻】<br />
<br />
  二年春正月下賀獄案驗父子死獄中家族【其家皆族誅也】以涿郡太守劉屈氂為丞相封澎侯【涿郡高帝置屬幽州唐瀛莫幽涿深祁州地屈丘勿翻氂力之翻晉灼曰澎東海縣今攷班志無之服䖍曰澎音彭】屈氂中山靖王子也【靖王勝景帝子】 夏四月大風發屋折木【折而說翻】 閏月諸邑公主陽石公主及皇后弟子長平侯伉皆坐巫蠱誅【諸琅邪縣以封公主故謂之邑與陽石公主皆衛皇后之女長平侯伉衛青子也伉音抗又音剛】 上行幸甘泉 初上年二十九乃生戾太子甚愛之及長性仁恕温謹【長知兩翻】上嫌其材能少不類已而所幸王夫人生子閎李姬生子旦胥李夫人生子髆【少詩沼翻髆音博】皇后太子寵浸衰常有不自安之意上覺之謂大將軍青曰漢家庶事草創【朱熹曰草略也創造也】加四夷侵陵中國朕不變更制度後世無法不出師征伐天下不安為此者不得不勞民【更工衡翻為於偽翻】若後世又如朕所為是襲亡秦之跡也太子敦重好静【好呼到翻】必能安天下不使朕憂欲求守文之主安有賢於太子者乎聞皇后與太子有不安之意豈有之邪可以意曉之大將軍頓首謝皇后聞之脱簪請罪【脱簪去飾也】太子每諫征伐四夷上笑曰吾當其勞以逸遺汝【遺於季翻】不亦可乎上每行幸常以後事付太子宫内付皇后有所平决還白其最【最大最也】上亦無異有時不省也【無所違異也不省不視也省悉井翻】上用灋嚴多任深刻吏太子寛厚多所平反【如淳曰反音幡幡奏使從輕也】雖得百姓心而用灋大臣皆不悦皇后恐久獲罪每戒太子宜留取上意【言留其事取上意裁决也】不應擅有所縱捨上聞之是太子而非皇后羣臣寛厚長者皆附太子而深酷用灋者皆毁之邪臣多黨與故太子譽少而毁多【譽音余少詩沼翻】衛青薨臣下無復外家為據競欲搆太子【言自衛青既薨之後姦人以太子無復外家以為憑依競欲搆成其罪】上與諸子疏【疏讀曰踈】皇后希得見【見賢遍翻】太子嘗謁皇后移日乃出【移日言日景移也】黄門蘇文告上曰【黄門屬少府以宦者為之】太子與宫人戲上益太子宫人滿二百人太子後知之心銜文文與小黄門常融王弼等常微伺太子過輒增加白之皇后切齒【切齒者怨憤之甚兩齒相摩切也】使太子白誅文等太子曰第勿為過何畏文等上聰明不信邪佞不足憂也上嘗小不平【小不平者體中微有不適也】使常融召太子融言太子有喜色上嘿然及太子至上察其貌有涕泣處而佯語笑上怪之更微問知其情乃誅融皇后亦善自防閑避嫌疑雖久無寵尚被禮遇【被皮義翻】是時方士及諸神巫多聚京師率皆左道惑衆【盧植曰左道謂邪道也地道尊右右為貴故漢書云右賢左愚右貴左賤故正道為右不正道為左若巫蠱及俗禁者】變幻無所不為女巫往來宫中教美人度厄每屋輒埋木人祭祀之因妬忌恚詈【恚於避翻】更相告訐以為祝詛上無道【更工衡翻訐居謁翻鄭玄曰詛謂祝之使沮敗也漢法有大逆無道之科祝職救翻詛莊助翻】上怒所殺後宫延及大臣死者數百人上心既以為疑嘗晝寢夢木人數千持杖欲擊上上驚寤因是體不平遂苦忽忽善忘【忘巫放翻遺忘也】江充自以與太子及衛氏有隙見上年老恐晏駕後為太子所誅因是為姦言上疾祟在巫蠱【師古曰祟謂禍咎之徵也故其字從出從示言鬼神所以示人者也音息遂翻】於是上以充為使者治巫蠱獄充將胡巫掘地求偶人捕蠱及夜祠視鬼染汗令有處輒收捕驗治燒鐵鉗灼強服之【張晏曰充捕巫蠱及夜祭祀祝詛者令胡巫視鬼詐以酒醊地令有處也師古曰捕夜祠及視鬼之人而充遣巫汗染地上為祠祭之處以誣其人又以燒鐵或鉗之或灼之強使之服鉗鑷也灼炙也汙烏故翻鉗其炎翻強其兩翻】民轉相誣以巫蠱吏輒劾以為大逆無道【劾戶槩翻】自京師三輔連及郡國坐而死者前後數萬人是時上春秋高疑左右皆為蠱祝詛有與無莫敢訟其寃者充既知上意因胡巫檀何言宮中有蠱氣不除之上終不差【差愈也】上乃使充入宫至省中壞御座掘地求蠱又使按道侯韓說御史章贛【師古曰說讀曰悦贛音貢姓譜齊人降鄣子孫去邑為章氏】黃門蘇文等助充充先治後宫希幸夫人以次及皇后太子宫掘地縱横【縱子容翻】太子皇后無復施床處充云於太子宫得木人尤多【師古曰三輔舊事云充使胡巫作桐木人而薶之】又有帛書所言不道當奏聞太子懼問少傅石德德懼為師傅并誅因謂太子曰前丞相父子兩公主及衛氏皆坐此今巫與使者掘地得徵驗不知巫置之邪將實有也無以自明可矯以節收捕充等繫獄【師古曰矯記也託詔命也】窮治其奸詐且上疾在甘泉皇后及家吏請問皆不報【蘇林曰家吏皇后吏也臣瓚曰太子稱家家吏是太子吏也師古曰既言皇后及家吏此為皇后吏及太子吏耳瓚說是也】上存亡未可知而姦臣如此太子將不念秦扶蘇事邪【事見七卷始皇三十七年】太子曰吾人子安得擅誅不如歸謝幸得無罪太子將往之甘泉而江充持太子甚急太子計不知所出遂從石德計秋七月壬午太子使客詐為使者收捕充等按道侯說疑使者有詐不肯受詔客格殺說【格古陌翻擊也】太子自臨斬充罵曰趙虜前亂乃國王父子不足邪【江充趙人故罵為趙虜乃汝也謂充前告趙太子隂事使太子見廢也】乃復亂吾父子也【復扶又翻】又炙胡巫上林中太子使舍人無且【師古曰且音子閭翻】持節夜入未央宫殿長秋門因長御倚華具白皇后【鄭氏曰長音長者之長如淳曰漢儀注女長御比侍中皇后見娙娥以下長御稱謝倚華字也師古曰倚音于綺翻】發中廏車載射士【師古曰中廄皇后車馬所在也余謂中廐者天子之内廐也秦二世時公子高曰中廐之寶馬臣得賜之非專主皇后車馬也】出武庫兵發長樂宫衛卒長安擾亂言太子反蘇文迸走得亡歸甘泉說太子無狀【迸北孟翻】上曰太子必懼又忿充等故有此變乃使使召太子使者不敢進歸報云太子反已成欲斬臣臣逃歸上大怒丞相屈氂聞變挺身逃【師古曰挺引也獨引身而逃也余謂挺拔也拔身而逃也】亡其印綬使長史乘疾置以聞【師古曰置謂所置驛也疾置急傳也】上問丞相何為對曰丞相秘之未敢發兵上怒曰事籍籍如此【師古曰籍籍猶紛紛也】何謂秘也丞相無周公之風矣周公不誅管蔡乎【屈氂於太子為兄弟故以周公之事責之】乃賜丞相璽書曰捕斬反者自有賞罰以牛車為櫓【師古曰櫓盾也遠與敵戰故以車為橧用自蔽也一說櫓望敵之樓】毋接短兵多殺傷士衆堅閉城門毋令反者得出太子宣言告令百官云帝在甘泉病困疑有變姦臣欲作亂上於是從甘泉來幸城西建章宫詔發三輔近縣兵部中二千石以下丞相兼將之太子亦遣使者矯制赦長安中都官囚徒命少傅石德及賓客張光等分將使長安囚如侯持節發長水及宣曲胡騎皆以裝會【師古曰長水宣曲並胡騎所屯今鄠縣東長水鄉即舊營校之地】侍郎馬通使長安【馬通漢書作莾通通及弟何羅以反誅明德馬皇后惡其先有反者故易其姓為莾姓譜馬本自伯益之裔趙奢封馬服君後因氏焉】因追捕如侯告胡人曰節有詐勿聽也遂斬如侯引騎入長安又發楫棹士以予大鴻臚商丘成【師古曰楫棹士主用楫及棹行船者也短曰楫長曰棹余據班表水衡都尉有楫棹令丞蓋掌楫棹士之官也太初元年改典客為大鴻臚者凡朝會使之鴻聲臚傳以贊導九賓予讀曰與臚音閭】初漢節純赤以太子持赤節故更為黄旄加上以相别【更工衡翻别彼列翻】太子立車北軍南門外召護北軍使者任安與節令發兵安拜受節入閉門不出【任音壬】太子引兵去毆四市人【二都及二京賦皆謂長安城中有几市廟記曰長安市有九各方二百六十五步六市在道西三市在道東凡四里為一市此言四市蓋以東西南北分為市也一說四市者東市西市直市柳市師古曰毆與驅同】凡數萬衆至長樂西闕下逢丞相軍合戰五日死者數萬人血流入溝中【街衢之側有溝以通水】民間皆云太子反以故衆不附太子丞相附兵寖多庚寅太子兵敗南犇覆盎城門【師古曰長安城南出東頭第一門曰覆盎城門一曰杜門三輔黄圖曰長樂宫在東直杜門故戾太子戰敗於長樂闕下南犇覆盎城門而出亡也】司直田仁部閉城門【班表元狩五年初置司直掌佐丞相舉不法秩比二千石】以為太子父子之親不欲急之太子由是得出亡丞相欲斬仁御史大夫暴勝之謂丞相曰司直吏二千石當先請柰何擅斬之丞相釋仁上聞而大怒下吏責問御史大夫曰【下遐稼翻】司直縱反者丞相斬之灋也大夫何以擅止之勝之惶恐自殺詔遣宗正劉長執金吾劉敢奉策收皇后璽綬后自殺【璽斯氏翻】上以為任安老吏見兵事起欲坐觀成敗見勝者合從之【言與之合而從之也】有兩心與田仁皆要斬【要與腰同】上以馬通獲如侯長安男子景建從通獲石德商丘成力戰獲張光【姓譜商丘衛大夫以邑為氏】封通為重合侯【班志重合侯國屬勃海郡】建為德侯【班表德侯食邑于濟南界】成為秺侯【班表秺侯國屬濟隂郡孟康曰今濟隂成武有秺亭秺音妬】諸太子賓客嘗出入宫門皆坐誅其隨太子發兵以反灋族吏士刼畧者皆徙燉煌郡【師古曰非其本心然被太子刼畧故徙之也燉咅屯】以太子在外始置屯兵長安諸城門上怒甚羣下憂懼不知所出壷關三老茂上書曰【班志壷關縣屬上黨郡苟悦漢紀茂姓令狐】臣聞父者猶天母者猶地子猶萬物也故天平地安物乃茂成父慈母愛子乃孝順今皇太子為漢適嗣承萬世之業體祖宗之重親則皇帝之宗子也【適子承大宗故謂之宗子適讀曰嫡】江充布衣之人閭閻之隸臣耳【隸賤也】陛下顯而用之銜至尊之命以迫蹵皇太子【蹵千六翻】造飾姦詐羣邪錯繆是以親戚之路鬲塞而不通【鬲與隔同塞悉則翻】太子進則不得見上退則困於亂臣獨寃結而無告不忍忿忿之心起而殺充恐懼逋逃子盗父兵以救難自免耳【難乃旦翻】臣竊以為無邪心詩曰營營青蠅止于藩愷悌君子無信讒言讒人罔極交亂四國【師古曰小雅青蝇之詩也營營往來之貌也蒲籬也愷悌易樂也言青蝇往來止於藩籬變白作黑讒人構毁間親令疏樂易之君子不當信用若讒言無極則四國亦以交亂宜深察也】往者江充讒殺趙太子天下莫不聞陛下不省察深過太子【師古曰以太子為罪過而深責之省悉景翻】發盛怒舉大兵而求之三公自將【漢丞相位三公將即亮翻】智者不敢言辯士不敢說【說式芮翻】臣竊痛之唯陛下寛心慰意少察所親【少詩沼翻】母患太子之非亟罷甲兵無令太子久亡臣不勝惓惓【勝音升師古曰惓惓忠切之意惓讀曰拳】出一旦之命待罪建章宫下書奏天子感寤然尚未敢顯言赦之也【以文理觀之不必有敢字】太子亡東至湖【湖縣屬京兆師古曰今虢州湖城閺鄉二縣皆其地】藏匿泉鳩里【師古曰泉鳩水今在閿鄉縣東南十五里見有戻太子冢冢在澗東】主人家貧常賣屨以給太子太子有故人在湖聞其富贍使人呼之而發覺八月辛亥吏圍捕太子太子自度不得脱【度徒洛翻】即入室距戶自經【孫愐曰頸在前項在後故引䋲經其頸謂之自經以刀割其頸謂之自剄】山陽男子張富昌為卒【山陽時為昌邑國】足蹋開戶新安令史李壽趨抱解太子【班志新安縣屬弘農郡續漢志縣有斗食令史】主人公遂格鬬死皇孫二人并皆遇害 【考異曰漢武故事云治隨太子反者外連郡國數十萬人壷關三老鄭茂上書上感寤赦反者拜鄭茂為宣慈校尉持節狥三輔赦太子太子欲出疑弗實吏捕太子急太子自殺按上若赦太子當詔吏弗捕此說恐妄也】上既傷太子乃封李夀為邘侯【班志河内野王縣有邘亭邘咅于】張富昌為題侯【班表題侯食邑於鉅鹿】初上為太子立博望苑【三輔黄圖曰博望苑在長安杜門外五里師古曰取其廣博觀望也為于偽翻下同】使通賓客從其所好【好呼到翻】故賓客多以異端進者臣光曰古之明王教養太子為之擇方正敦良之士【為于偽翻】以為保傅師友使朝夕與之游處【處昌呂翻】左右前後無非正人出入起居無非正道然猶有淫放邪僻而䧟於禍敗者焉今乃使太子自通賓客從其所好夫正直難親謟諛易合【易以豉翻】此固中人之常情宜太子之不終也<br />
<br />
  癸亥地震 九月商丘成為御史大夫 立趙敬肅王小子偃為平干王【平干國屬冀州本廣平也宣帝五鳳二年復舊名】匈奴入上谷五原殺掠吏民【上谷郡屬幽州庚媯州地也】<br />
<br />
  三年春正月上行幸雍至安定北地【雍於用翻】 匈奴入五原酒泉殺兩都尉三月遣李廣利將七萬人出五原商丘成將二萬人出西河馬通將四萬騎出酒泉擊匈奴夏五月赦天下 匈奴單于聞漢兵大出悉徙其輜<br />
<br />
  重北邸郅居水【重直用翻師古曰邸至也音丁體翻郅之日翻】左賢王驅其人民度余吾水六七百里居兜銜山單于自將精兵度姑且水【將即亮翻師古曰且子余翻】商丘成軍至追邪徑無所見還【師古曰從疾道而追之不見虜而還也邪音似嗟翻】匈奴使大將與李陵將三萬餘騎追漢軍轉戰九日至蒲奴水【蒲奴水又在龍勒水南】虜不利還去馬通軍至天山匈奴使大將偃渠將二萬餘騎要漢兵【要一遙翻下同】見漢兵彊引去通無所得失是時漢恐車師遮馬通軍遣開陵侯成娩將樓蘭尉犁危須等六國兵【危須國治危須城去長安七千二百九十里西至焉耆百里娩音晩又咅免】共圍車師盡得其王民衆而還貳師將軍出塞匈奴使右大都尉與衛律將五千騎要擊漢軍於夫羊句山陿【要讀曰邀服䖍曰夫羊地名也師古曰句山西山也句音鉤】貳師擊破之乘勝追北至范夫人城【應劭曰本漢將築此城將亡其妻率餘衆完保之因以為名也張晏曰范氏能胡詛者】匈奴犇走莫敢距敵初貳師之出也丞相劉屈氂為祖道【祖軷祭也崔氏云宫内之軷祭古之行神城外之軷祭山川與道路之神記曾子問諸侯適天子道而出注云祖道也聘禮曰出祖釋軷祭酒脯也注云祖始也行出國門止陳車騎釋酒脯之奠為行始也師古曰祖者送行之祭因設宴飲昔黄帝之子纍祖好遠遊而死於道故祀以為行神為于偽翻】送至渭橋廣利曰願君侯早請昌邑王為太子如立為帝君侯長何憂乎【當時列侯通呼為君侯尊稱之也】屈氂許諾昌邑王者貳師將軍女弟李夫人子也貳師女為屈氂子妻故共欲立焉會内者令郭穰【班表内者令屬少府又據昭紀内謁者令郭穰内者謁者各有令丞皆屬少府豈其時穰兼兩令乎】告丞相夫人祝詛上及與貳師共禱祠欲令昌邑王為帝按驗罪至大逆不道六月詔載屈氂㕑車以狥【師古曰㕑車載食之車狥行示也】要斬東市【要與腰同】妻子梟首華陽街【梟堅堯翻長安城中八街華陽其一也華戶化翻】貳師妻子亦收貳師聞之憂懼其掾胡亞夫亦避罪從軍說貳師曰夫人室家皆在吏若還不稱意適與獄會【掾於絹翻說式芮翻稱尺證翻】郅居以北可復得見乎【如淳曰以就誅後雖欲復降匈奴不可得復扶又翻】貳師由是狐疑深入要功【要一遥同】遂北至郅居水上虜已去貳師遣護軍將二萬騎度郅居之水逢左賢王左大將將二萬騎與漢兵合戰一日漢軍殺左大將虜死傷甚衆軍長史與决眭都尉煇渠侯謀曰【晉灼曰决眭都尉匈奴官也功臣表歸義侯僕朋子雷電以擊匈奴功封煇渠侯煇渠魯陽縣也予據班表僕朋侯煇渠食邑於魯陽雷電嗣爵雷電不自匈奴來降則决眭都尉非匈奴官也師古曰眭息隨翻煇音輝】將軍懷異心欲危衆求功恐必敗謀共執貳師貳師聞之斬長史引兵還至燕然山【據匈奴傳燕然山在匈奴中速邪烏地師古曰燕一千翻】單于知漢軍勞倦自將五萬騎遮擊貳師相殺傷甚衆夜塹漢軍前深數尺【塹七艶翻深式禁翻度深曰深】從後急擊之軍大亂貳師遂降單于素知其漢大將以女妻之【降戶江翻妻七細翻】尊寵在衛律上宗族遂滅 秋蝗 九月故城父令公孫勇【班志城父縣屬沛郡父音甫】與客胡倩等謀反【師古曰倩音千見翻】倩詐稱光禄大夫言使督盗賊淮陽太守田廣明覺知【使疏吏翻守式又翻高祖十一年置淮陽國時為郡屬兖州唐陳州地賢曰淮陽故城在今陳州宛丘縣東南】發兵捕斬焉公孫勇衣繡衣乘駟馬車至圉【師古曰陳留圉縣今據班志圉縣屬淮陽勇衣于既翻】圉守尉魏不害等誅之封不害等四人為侯【不害當塗侯江德轑陽侯蘇昌浦侯圉縣小史關内侯食邑圍之遺鄉】吏民以巫蠱相告言者案驗多不實上頗知太子惶恐無他意【言為江充所迫惶恐無以自明而起兵殺江充非有他意也】會高寢郎田千秋上急變訟太子寃【師古曰高廟衛寢之郎所告非常故云急變上時掌翻】曰子弄父兵罪當笞天子之子過誤殺人當何罪哉臣嘗夢一白頭翁教臣言上乃大感寤召見千秋謂曰父子之間人所難言也公獨明其不然此高廟神靈使公教我公當遂為吾輔佐立拜千秋為大鴻臚【師古曰當其立見而即拜之言不移時也臚陵如翻】而族滅江充家焚蘇文於横橋上【即横門外渭橋也横咅光】及泉鳩里加兵刃於太子者初為北地太守後族上憐太子無辜乃作思子宫為歸來望思之臺於湖【師古曰言已望而思之庶太子之䰟歸來也其臺在今湖城縣之西闥鄉縣之東基址猶存】天下聞而悲之四年春正月上行幸東萊臨大海欲浮海求神山羣臣諫上弗聽而大風晦冥海水沸湧上留十餘日不得御樓船乃還 二月丁酉雍縣無雲如靁者三【雍於用翻經典如而字通】隕石二黑如黳【師古曰黳烏兮翻小黑也江南人以油煎漆滓以飾物曰黳】 三月上耕于鉅定【地理志鉅定縣屬齊國水經注作巨淀縣故城在淄水北縣東南有巨澱湖蓋以水受名也】還幸泰山脩封庚寅祀于明堂癸巳禪石閭見羣臣上乃言曰朕即位以來所為狂悖【悖蒲妹翻】使天下愁苦不可追悔自今事有傷害百姓糜費天下者悉罷之田千秋曰方士言神僊者甚衆而無顯功臣請皆罷斥遣之上曰大鴻臚言是也於是悉罷諸方士候神人者是後上每對群臣自歎曏時愚惑為方士所欺天下豈有僊人盡妖妄耳【妖於遙翻】節食服藥差可少病而已【少詩沼翻】夏六月還幸甘泉 丁巳以大鴻臚田千秋為丞相封富民侯【恩澤侯表富民侯食邑於沛郡蘄縣師古曰欲百姓之殷實故取其嘉名也】千秋無它材能又無伐閱功勞【太史公曰古者人臣功有五品以德立宗廟定社稷曰勲以言曰勞角力曰功明其等曰伐積日曰閲師古曰伐積功也閲經歷也】特以一言寤意數月取宰相封侯世未嘗有也然為人敦厚有智居位自稱【師古曰言稱其職也稱尺證翻】踰於前後數公先是搜粟都尉桑弘羊與丞相御史奏【先悉薦翻】言輪臺東有溉田五千頃以上【杜佑曰輪臺渠犂地今在交河北庭界中其地相連】可遣屯田卒置校尉三人分護益種五穀張掖酒泉遣騎假司馬為斥候【斥拓也候望也言開拓道路候望也】募民壯健敢徙者詣田所益墾溉田稍築列亭連城而西以威西國輔烏孫【時烏孫王尚公主故欲屯田列亭連城以輔之】上乃下詔深陳既往之悔曰前有司奏欲益民賦三十助邊用【師古曰三十者每口轉增三十錢也】是重困老弱孤獨也【重直用翻】而今又請遣卒田輪臺輪臺西於車師千餘里前開陵侯擊車師時雖勝降其王【降戶江翻】以遼遠乏食道死者尚數千人况益西乎曩者朕之不明以軍候弘上書言匈奴縛馬前後足置城下馳言秦人我匄若馬【據漢時匈奴謂中國人為秦人至唐及國朝則謂中國為漢如漢人漢兒之類皆習故而言師古曰匄乞與也若汝也乞音氣】又漢使者久留不還故興遣貳師將軍【久留不還謂蘇武等也師古曰興遣興軍而遣之】欲以為使者威重也古者卿大夫與謀參以蓍龜不吉不行【師古曰謂共卿大夫謀事尚不專决猶雜問蓍龜也蓍筮也龜卜也孔頴達曰卜筮必用龜蓍者案劉向云蓍之言耆龜之言久龜千歲而靈蓍百年而神以其長久故能辯吉凶也說文蓍蒿屬也生千歲三百莖易以為數天子九尺諸侯七尺大夫五尺士三尺陸璣草木疏云似藾蕭青色科生洪範五行傳曰蓍生百年一本生百莖論衡云七十年生一莖七百年十莖神靈之物故生遲也史記曰滿百莖者其下必有神龜守之其上常有雲氣覆之淮南子云上有藂蓍下有伏龜卜筮實問於神龜筮能傳神命以告人故金縢告太王王季文王乃卜三龜一習吉是能傳神命也又鄭注天府云卜筮實問於鬼神筮龜能出其卦兆之占耳按白虎通稱禮三正記天子龜一尺二寸諸侯一尺大夫八寸士六寸龜隂也故其數偶蓍陽也故其數奇所以謂之卜筮者師說云卜覆也以覆審吉凶筮决也以决定其惑劉向以為卜赴也赴來者之心筮問也問筮者之事赴問互言之易繫辭云定天下之吉凶成天下之亹亹者莫大乎蓍龜又曰蓍之德圓而神又說卦云幽贊於神明而生蓍據此諸文蓍龜知靈相似傳云蓍短龜長不如從長者史蘇欲止獻公之意託云爾實無優劣也杜預鄭玄因是言以為實有長短杜預注傳云物生而後有象象而後有滋滋而後有數龜象筮數故象長數短是也象所以長者以物初生則有象去初既近且包羅萬形故為長數短者數是終去初既遠推尋事數始能求象故為短也鄭注占人云占人亦占筮掌占龜者筮短龜長主于長者是也凡卜筮天子諸侯若大事則卜筮並用皆先筮後卜大事則卜立君卜大封大祭祀出軍旅喪事及龜之八命一曰征二曰象三曰與四曰謀五曰果六曰至七曰雨八曰瘳此等皆為大事鄭注占人云將卜八事皆先以筮筮之是也若次事則惟卜不筮故表記云天子無筮小事無卜惟筮筮人掌九筮之名一曰筮更謂遷都邑也二曰筮咸咸猶僉也謂筮衆心歡不也三曰筮式謂筮作法式也四曰筮目謂事衆筮其所要當也五曰筮易謂民衆不說筮所改易也六曰筮比謂與民和比也七曰筮祠謂筮牲與日也八曰筮參謂筮御與右也九曰筮環謂筮可致師不鄭注古人不卜而徒筮者則用九筮是也僖十五年晉卜納襄王得黄帝戰阪泉之兆又筮之得大有之睽哀九年卜伐宋亦卜而後筮是大事卜筮並用也與讀曰預蓍音升脂翻】乃者以縳馬書徧視丞相御史二千石諸大夫郎為文學者【師古曰視讀曰示為文學謂學經書之人】乃至郡屬國都尉等皆以虜自縛其馬不祥甚哉或以為欲以見彊【師古曰見顯示見賢遍翻】夫不足者視人有餘【師古曰言其夸張也視亦讀曰示】公車方士太史治星望氣及太卜龜蓍皆以為吉【公車方士方士之待詔公車者太史屬太常治星習為天文之家望氣如周官之㫝祲者皆屬太史太卜屬太常有令丞治直之翻】匈奴必破時不可再得也【師古曰今便利之時後不可再得也】又曰北伐行將於鬴山必克【師古曰行將謂遣將率行也鬴山山名也將即亮翻下同鬴古釡字】卦諸將貳師最吉【卜遣諸將而於卦中貳師最為吉也】故朕親發貳師下鬴山詔之必毋深入今計謀卦兆皆反繆【師古曰言不效也繆妄也】重合候得虜侯者乃言縛馬者匈奴詛軍事也【據班史匈奴聞漢軍當來使巫埋羊牛所出諸道及水上以詛軍詛莊助翻】匈奴常言漢極大然不耐饑渴失一狼走千羊乃者貳師敗軍士死略離散【師古曰言死及被虜畧併自離散也】悲痛常在朕心今又請遠田輪臺欲起亭隧【師古曰隧者依深險之處開通行道也】是擾勞天下非所以優民也朕不忍聞大鴻臚等又議欲募囚徒送匈奴使者明封侯之賞以報忿此五伯所弗為也【盖欲使刺單于以報忿也師古曰言五伯尚耻不為况今大漢也伯讀曰霸】且匈奴得漢降者常提掖搜索【降戶江翻索山客翻提謂提挈之也掖謂兩人夾持其兩掖掖羊益翻師古曰搜索者恐其或私齎文書也余謂恐其挾兵刃】問以所聞豈得行其計乎當今務在禁苛暴止擅賦力本農脩馬復令【漢有擅賦法今止不行孟康曰先是令長吏各以秩養馬亭有牝馬名養馬者皆復不事後馬多絶乏至此復脩之也師古曰此說非也馬復因養馬以免徭賦也復方目翻】以補缺毋乏武備而已郡國二千石各上進畜馬方略補邊狀與計對【師古曰與上計者同來赴對也上時掌翻畜許六翻】由是不復出軍【復扶又翻】而封田千秋為富民侯以明休息思富養民也又以趙過為搜粟都尉過能為代田【班志一畝三甽歲代處故曰代田古法也后稷始甽田以二耜為耦廣尺深尺曰甽長終畝一畝三甽一夫三百甽而播種於三甽中師古曰代易也余謂此即周禮一易再易之田之類】其耕耘田器皆有便巧以教民用力少而得穀多民皆便之<br />
<br />
  臣光曰天下信未嘗無士也武帝好四夷之功而勇鋭輕死之士充滿朝廷闢土廣地無不如意及後息民重農而趙過之儔教民耕耘民亦被其利【好呼到翻被皮義翻】此一君之身趣好殊别而士輒應之誠使武帝兼三王之量以興商周之治【治直之翻】其無三代之臣乎<br />
<br />
  秋八月辛酉晦日有食之 【考異曰荀紀作七月漢書作八月按長歷是年九月壬戌朔言八月是也】 衛律害貳師之寵會匈奴單于母閼氏病【閼氏音煙支】律飭胡巫言先單于怒曰胡故時祠兵常言得貳師以社【師古曰飭與敕同社祠社也】何故不用於是收貳師貳師罵曰我死必滅匈奴遂屠貳師以祠<br />
<br />
  後元元年春正月上行幸甘泉郊泰畤遂幸安定 昌邑哀王髆薨【髆音博】 二月赦天下 夏六月商丘成坐祝詛自殺 【考異曰功臣表云坐為詹事祠孝文廟醉歌堂下曰出居安能鬰鬰大不敬自殺公卿表云坐祝詛按成不為詹事功臣表誤也】 初侍中僕射馬何羅與江充相善【班表侍中僕射秦官自侍中尚書郎軍屯騶宰永巷宦者皆有僕射古者重武官有主射以課督之取其領事之號沈約曰侍中本秦丞相史也使五人往來殿内東廂奏事故謂之侍中漢西京無員多至數十人入侍禁中分掌乘輿御物下至䙝器虎子之屬武帝世孔安國為侍中以其儒者特令掌御唾壺朝廷榮之久次者為僕射東京又屬少府猶無員掌侍左右贊導衆事顧問應答法駕出則多識者一人負傳國璽操斬白蛇劒參乘餘皆騎在乘輿車後光武改僕射為祭酒漢世與中官俱止禁中武帝時侍中馬何羅為逆由是侍中出禁外有事乃得入事畢即出王莽秉政侍中復入與中官俱止章帝元和中侍中郭舉與後宫通拔佩刀驚御舉伏誅侍中猶是復出外】及衛太子起兵何羅弟通以力戰封重合侯後上夷滅充宗族黨與何羅兄弟懼及【及謂及於禍也】遂謀為逆侍中駙馬都尉金日磾視其志意有非常心疑之隂獨察其動静與俱上下【師古曰上下於殿也磾丁奚翻上時掌翻下廂上同】何羅亦覺日磾意以故久不得發是時上行幸林光宫【服䖍曰甘泉一名林光師古曰秦之林光宫胡亥所造漢又於其旁起甘泉宮】日磾小疾臥廬【師古曰殿中所止曰廬】何羅與通及小弟安成矯制夜出共殺使者發兵明旦上未起何羅無何從外入【無何猶言無幾時也】日磾奏厠心動【師古曰奏向也日磾方向厠而心動】立入坐内戶下須臾何羅䄂白刃從東廂上見日磾色變走趨臥内欲入【師古曰趨讀曰趣向也臥内天子臥處】行觸寶瑟僵日磾得抱何羅因傳曰馬何羅反【傳謂傳聲而唱之】上驚起左右拔刃欲格之上恐并中日磾【中竹仲翻】止勿格日磾投何羅殿下得禽縛之窮治皆伏辜 秋七月地震 燕王旦自以次第當為太子【燕王旦元狩六年受封】上書求入宿衛上怒斬其使於北闕又坐藏匿亡命削良鄉安次文安三縣【班志良鄉縣屬涿郡安次文安屬勃海郡良鄉安次二縣唐皆屬幽州文安縣唐為莫州】上由是惡旦【惡烏路翻】旦辯慧博學其弟廣陵王胥有勇力【胥亦以元狩六年受封】而皆動作無灋度多過失故上皆不立時鉤弋夫人之子弗陵年數歲形體壯大多知【師古曰壯大者言其形體偉大】上奇愛之心欲立焉以其年穉母少【少詩沼翻下同】猶與久之【與讀曰豫】欲以大臣輔之察群臣唯奉車都尉光禄大夫霍光忠厚可任大事上乃使黄門畫周公負成王朝諸侯以賜光【師古曰黄門之署職任親近以供天子百物在焉故亦有畫工畫讀曰畫】後數日帝譴責鉤弋夫人夫人脱簪珥【珥仍吏翻耳飾也】叩頭【句斷】帝曰引持去送掖庭獄【掖庭屬少府有秘獄凡宫人有罪者下之】夫人還顧帝曰趣行【趣讀曰促】汝不得活卒賜死【卒子恤翻】頃之帝閒居問左右曰外人言云何左右對曰人言且立其子何去其母乎【去羌呂翻下同】帝曰然是非兒曹愚人之所知也往古國家所以亂由主少母壯也女主獨居驕蹇淫亂自恣莫能禁也汝不聞呂后邪故不得不先去之也<br />
<br />
  二年春正月上朝諸侯王于甘泉宫二月行幸盩厔五柞宫【班志盩厔縣屬扶風山曲曰盩水曲曰厔師古曰盩張流翻厔竹乙翻張晏曰五柞宮有五柞樹因名水經注五柞宮在長楊宫東北八里柞即各翻】 上病篤霍光涕泣問曰如有不諱【賢曰不諱謂死也死者人之常故言不諱也師古曰不諱言不可諱也】誰當嗣者上曰君未諭前畫意邪立少子君行周公之事光頓首讓曰臣不如金日磾日磾亦曰臣外國人【日磾休屠王子故云然】不如光且使匈奴輕漢矣乙丑詔立弗陵為皇太子時年八歲丙寅以光為大司馬大將軍日磾為車騎將軍太僕上官桀為左將軍受遺詔輔少主又以搜粟都尉桑弘羊為御史大夫皆拜臥内牀下光出入禁闥二十餘年出則奉車入侍左右小心謹慎未嘗有過為人沈静詳審【沈持林翻】每出入下殿門止進有常處郎僕射竊識視之【師古曰識記也音式志翻又職吏翻】不失尺寸日磾在上左右目不忤視者數十年【忤逆也五故翻】賜出宫女不敢近上欲内其女後宫【内讀曰納】不肯其篤慎如此上猶奇異之日磾長子為帝弄兒帝甚愛之其後弄兒壯大不謹自殿下與宫人戲日磾適見之惡其淫亂【惡烏路翻】遂殺弄兒上聞之大怒日磾頓首謝具言所以殺弄兒狀上甚哀為之泣【為于偽翻】已而心敬日磾上官桀始以材力得幸【桀少時為羽林期門郎從帝上甘泉天大風車不得行解蓋授桀桀奉蓋雖風常屬車雨下蓋輒御上奇其材力】為未央廄令【未央廐令屬太僕】上嘗體不安及愈見馬【師古曰見謂呈見之音胡電翻】馬多瘦上大怒曰令以我不復見馬邪欲下吏【復扶又翻下遐嫁翻】桀頓首曰臣聞聖體不安日夜憂懼意誠不在馬【師古曰誠實也】言未卒泣數行下【卒子恤翻行戶剛翻】上以為愛已由是親近【近其靳翻】為侍中稍遷至太僕三人皆上素所愛信者故特舉之授以後事丁卯帝崩于五柞宫【臣瓚曰夀七十一】入殯未央宫前殿帝聦明能斷【斷丁亂翻】善用人行灋無所假貸隆慮公主子昭平君【隆慮公主景帝女班志隆慮縣屬河内郡慮音閭】尚帝女夷安公主【班志夷安縣屬膠西國】隆慮主病困以金千斤錢千萬為昭平君豫贖死罪上許之【為于偽翻下同】隆慮主卒昭平君日驕醉殺主傅【服䖍曰主傅主之官也如淳曰禮有傳姆說者又曰傅老大夫也漢使中行說傅翁主是也師古曰傅姆是】繫獄廷尉以公主子上請【上時掌翻】左右人人為言前又入贖陛下許之上曰吾弟老有是一子死以屬我【弟謂女弟師古曰老乃有子言其晚孕育也屬咅之欲翻】於是為之垂涕歎息良久曰灋令者先帝所造也用弟故而誣先帝之灋吾何面目入高廟乎又下負萬民乃可其奏哀不能自止左右盡悲待詔東方朔前上夀【時有待詔公車者有待詔金馬門者朔時待詔宦者署】曰臣聞聖王為政賞不避仇讎誅不擇骨肉書曰不偏不黨王道蕩蕩【師古曰周書洪範之辭蕩蕩平坦貌】此二者五帝所重三王所難也陛下行之天下幸甚臣朔奉觴昧死再拜上萬夀上初怒朔既而善之以朔為中郎<br />
<br />
  班固贊曰漢承百王之弊高祖撥亂反正文景務在養民至于稽古禮文之事猶多闕焉孝武初立卓然罷黜百家表章六經【師古曰百家謂諸子雜說違背六經六經謂易詩書春秋禮樂也】遂疇咨海内舉其俊茂【師古曰疇誰也咨謀也言謀于衆人誰可為事者也】與之立功興太學修郊祀改正朔【師古曰正音之成翻】定歷數協音律作詩樂建封禪禮百神紹周後號令文章煥然可述後嗣得遵洪業而有三代之風如武帝之雄材大畧不改文景之恭儉以濟斯民雖詩書所稱何有加焉【師古曰美其雄才大畧而非其不恭儉也】<br />
<br />
  臣光曰孝武窮奢極欲繁刑重歛【歛力驗翻】内侈宫室外事四夷信惑神怪巡遊無度使百姓疲敝起為盗賊其所以異於秦始皇者無幾矣【幾居豈翻】然秦以之亡漢以之興者孝武能尊先王之道知所統守受忠直之言惡人欺蔽好賢不倦【惡烏路翻好呼到翻】誅賞嚴明晩而改過顧託得人此其所以有亡秦之失而免亡秦之禍乎<br />
<br />
  戊辰太子即皇帝位帝姊鄂邑公主共養省中【班志鄂縣屬江夏郡公主所食之邑伏儼曰蔡邕云本為禁中門閣有禁非侍御之臣不得妄入行道豹尾中亦為禁中孝元皇后父名禁避之故曰省中師古曰省察也言入此中者皆當察視不可妄也余據鄂邑公主即盖長公主鄂五各翻共居用翻養弋亮翻】霍光金日磾上官桀共領尚書事光輔幼主政自已出天下想聞其風采殿中嘗有怪一夜羣臣相驚光召尚符璽郎【續漢志本注符璽郎中二人在中主璽及虎符竹符之半者璽斯氏翻】欲收取璽【師古曰恐有變難欲收取璽】郎不肯授光欲奪之郎按劒曰臣頭可得璽不可得也光甚誼之明日詔增此郎秩二等衆庶莫不多光【多猶重也以此事為多足重也】 三月甲辰葬孝武皇帝于茂陵 夏六月赦天下 秋七月有星孛于東方【孛蒲内翻】濟北王寛坐禽獸行自殺【淮南厲王子勃徙封濟北王寛其孫也漢法内亂者為禽獸行濟子禮翻行下孟翻】 冬匈奴入朔方殺畧吏民發軍屯西河左將軍桀行北邊【行下孟翻】<br />
<br />
  資治通鑑卷二十二<br />
<br />
<史部,編年類,資治通鑑>  <br>
   </div> 

<script src="/search/ajaxskft.js"> </script>
 <div class="clear"></div>
<br>
<br>
 <!-- a.d-->

 <!--
<div class="info_share">
</div> 
-->
 <!--info_share--></div>   <!-- end info_content-->
  </div> <!-- end l-->

<div class="r">   <!--r-->



<div class="sidebar"  style="margin-bottom:2px;">

 
<div class="sidebar_title">工具类大全</div>
<div class="sidebar_info">
<strong><a href="http://www.guoxuedashi.com/lsditu/" target="_blank">历史地图</a></strong>  
<a href="http://www.880114.com/" target="_blank">英语宝典</a>  
<a href="http://www.guoxuedashi.com/13jing/" target="_blank">十三经检索</a> 
<br><strong><a href="http://www.guoxuedashi.com/gjtsjc/" target="_blank">古今图书集成</a></strong> 
<a href="http://www.guoxuedashi.com/duilian/" target="_blank">对联大全</a> <strong><a href="http://www.guoxuedashi.com/xiangxingzi/" target="_blank">象形文字典</a></strong> 

<br><a href="http://www.guoxuedashi.com/zixing/yanbian/">字形演变</a>  <strong><a href="http://www.guoxuemi.com/hafo/" target="_blank">哈佛燕京中文善本特藏</a></strong>
<br><strong><a href="http://www.guoxuedashi.com/csfz/" target="_blank">丛书&方志检索器</a></strong> <a href="http://www.guoxuedashi.com/yqjyy/" target="_blank">一切经音义</a>  

<br><strong><a href="http://www.guoxuedashi.com/jiapu/" target="_blank">家谱族谱查询</a></strong>  <strong><a href="http://shufa.guoxuedashi.com/sfzitie/" target="_blank">书法字帖欣赏</a></strong> 
<br>

</div>
</div>


<div class="sidebar" style="margin-bottom:0px;">

<font style="font-size:22px;line-height:32px">QQ交流群9:489193090</font>


<div class="sidebar_title">手机APP 扫描或点击</div>
<div class="sidebar_info">
<table>
<tr>
	<td width=160><a href="http://m.guoxuedashi.com/app/" target="_blank"><img src="/img/gxds-sj.png" width="140"  border="0" alt="国学大师手机版"></a></td>
	<td>
<a href="http://www.guoxuedashi.com/download/" target="_blank">app软件下载专区</a><br>
<a href="http://www.guoxuedashi.com/download/gxds.php" target="_blank">《国学大师》下载</a><br>
<a href="http://www.guoxuedashi.com/download/kxzd.php" target="_blank">《汉字宝典》下载</a><br>
<a href="http://www.guoxuedashi.com/download/scqbd.php" target="_blank">《诗词曲宝典》下载</a><br>
<a href="http://www.guoxuedashi.com/SiKuQuanShu/skqs.php" target="_blank">《四库全书》下载</a><br>
</td>
</tr>
</table>

</div>
</div>


<div class="sidebar2">
<center>


</center>
</div>

<div class="sidebar"  style="margin-bottom:2px;">
<div class="sidebar_title">网站使用教程</div>
<div class="sidebar_info">
<a href="http://www.guoxuedashi.com/help/gjsearch.php" target="_blank">如何在国学大师网下载古籍?</a><br>
<a href="http://www.guoxuedashi.com/zidian/bujian/bjjc.php" target="_blank">如何使用部件查字法快速查字?</a><br>
<a href="http://www.guoxuedashi.com/search/sjc.php" target="_blank">如何在指定的书籍中全文检索?</a><br>
<a href="http://www.guoxuedashi.com/search/skjc.php" target="_blank">如何找到一句话在《四库全书》哪一页?</a><br>
</div>
</div>


<div class="sidebar">
<div class="sidebar_title">热门书籍</div>
<div class="sidebar_info">
<a href="/so.php?sokey=%E8%B5%84%E6%B2%BB%E9%80%9A%E9%89%B4&kt=1">资治通鉴</a> <a href="/24shi/"><strong>二十四史</strong></a>&nbsp; <a href="/a2694/">野史</a>&nbsp; <a href="/SiKuQuanShu/"><strong>四库全书</strong></a>&nbsp;<a href="http://www.guoxuedashi.com/SiKuQuanShu/fanti/">繁体</a>
<br><a href="/so.php?sokey=%E7%BA%A2%E6%A5%BC%E6%A2%A6&kt=1">红楼梦</a> <a href="/a/1858x/">三国演义</a> <a href="/a/1038k/">水浒传</a> <a href="/a/1046t/">西游记</a> <a href="/a/1914o/">封神演义</a>
<br>
<a href="http://www.guoxuedashi.com/so.php?sokeygx=%E4%B8%87%E6%9C%89%E6%96%87%E5%BA%93&submit=&kt=1">万有文库</a> <a href="/a/780t/">古文观止</a> <a href="/a/1024l/">文心雕龙</a> <a href="/a/1704n/">全唐诗</a> <a href="/a/1705h/">全宋词</a>
<br><a href="http://www.guoxuedashi.com/so.php?sokeygx=%E7%99%BE%E8%A1%B2%E6%9C%AC%E4%BA%8C%E5%8D%81%E5%9B%9B%E5%8F%B2&submit=&kt=1"><strong>百衲本二十四史</strong></a>  <a href="http://www.guoxuedashi.com/so.php?sokeygx=%E5%8F%A4%E4%BB%8A%E5%9B%BE%E4%B9%A6%E9%9B%86%E6%88%90&submit=&kt=1"><strong>古今图书集成</strong></a>
<br>

<a href="http://www.guoxuedashi.com/so.php?sokeygx=%E4%B8%9B%E4%B9%A6%E9%9B%86%E6%88%90&submit=&kt=1">丛书集成</a> 
<a href="http://www.guoxuedashi.com/so.php?sokeygx=%E5%9B%9B%E9%83%A8%E4%B8%9B%E5%88%8A&submit=&kt=1"><strong>四部丛刊</strong></a>  
<a href="http://www.guoxuedashi.com/so.php?sokeygx=%E8%AF%B4%E6%96%87%E8%A7%A3%E5%AD%97&submit=&kt=1">說文解字</a> <a href="http://www.guoxuedashi.com/so.php?sokeygx=%E5%85%A8%E4%B8%8A%E5%8F%A4&submit=&kt=1">三国六朝文</a>
<br><a href="http://www.guoxuedashi.com/so.php?sokeytm=%E6%97%A5%E6%9C%AC%E5%86%85%E9%98%81%E6%96%87%E5%BA%93&submit=&kt=1"><strong>日本内阁文库</strong></a> <a href="http://www.guoxuedashi.com/so.php?sokeytm=%E5%9B%BD%E5%9B%BE%E6%96%B9%E5%BF%97%E5%90%88%E9%9B%86&ka=100&submit=">国图方志合集</a> <a href="http://www.guoxuedashi.com/so.php?sokeytm=%E5%90%84%E5%9C%B0%E6%96%B9%E5%BF%97&submit=&kt=1"><strong>各地方志</strong></a>

</div>
</div>


<div class="sidebar2">
<center>

</center>
</div>
<div class="sidebar greenbar">
<div class="sidebar_title green">四库全书</div>
<div class="sidebar_info">

《四库全书》是中国古代最大的丛书,编撰于乾隆年间,由纪昀等360多位高官、学者编撰,3800多人抄写,费时十三年编成。丛书分经、史、子、集四部,故名四库。共有3500多种书,7.9万卷,3.6万册,约8亿字,基本上囊括了古代所有图书,故称“全书”。<a href="http://www.guoxuedashi.com/SiKuQuanShu/">详细>>
</a>

</div> 
</div>

</div>  <!--end r-->

</div>
<!-- 内容区END --> 

<!-- 页脚开始 -->
<div class="shh">

</div>

<div class="w1180" style="margin-top:8px;">
<center><script src="http://www.guoxuedashi.com/img/plus.php?id=3"></script></center>
</div>
<div class="w1180 foot">
<a href="/b/thanks.php">特别致谢</a> | <a href="javascript:window.external.AddFavorite(document.location.href,document.title);">收藏本站</a> | <a href="#">欢迎投稿</a> | <a href="http://www.guoxuedashi.com/forum/">意见建议</a> | <a href="http://www.guoxuemi.com/">国学迷</a> | <a href="http://www.shuowen.net/">说文网</a><script language="javascript" type="text/javascript" src="https://js.users.51.la/17753172.js"></script><br />
  Copyright &copy; 国学大师 古典图书集成 All Rights Reserved.<br>
  
  <span style="font-size:14px">免责声明:本站非营利性站点,以方便网友为主,仅供学习研究。<br>内容由热心网友提供和网上收集,不保留版权。若侵犯了您的权益,来信即刪。scp168@qq.com</span>
  <br />
ICP证:<a href="http://www.beian.miit.gov.cn/" target="_blank">鲁ICP备19060063号</a></div>
<!-- 页脚END --> 
<script src="http://www.guoxuedashi.com/img/plus.php?id=22"></script>
<script src="http://www.guoxuedashi.com/img/tongji.js"></script>

</body>
</html>
