










 


 
 


 

  
  
  
  
  





  
  
  
  
  
 
  

  

  
  
  



  

 
 

  
   




  

  
  


    資治通鑑卷三十六   宋 司馬光 撰

  胡三省 音註

  漢紀二十八【起昭陽大淵獻盡著雍執徐凡六年】

  孝平皇帝下

  元始三年春太后遣長樂少府夏侯藩宗正劉宏尚書令平晏納采見女【婚有五禮納采問名納吉納徵請期陸德明曰采音七在翻擇也師古曰謂采擇其可者】還奏言公女漸漬德化【漸音沾】有窈窕之容【窈窕幽閒也王肅曰善心曰窈善容曰窕】宜承天序奉祭祀太師光大司徒宮大司空豐左將軍孫建執金吾尹賞行太常事太中大夫劉秀及太卜太史令服皮弁素積【百官表太常有太卜太史等令師古曰皮弁以鹿皮為冠形如人手之弁合也素積謂素裳也朱衣而素裳積謂襞績若今之襈為也襈音雛戀翻賢曰素積者積以為裳也言要中辟積也賈公彦曰皮弁之服十五升白布衣積素以為裳】以禮雜卜筮皆曰兆遇金水王相卦遇父母得位【兆卜也卦筮也孟康曰金水相生也張晏曰金王則水相也遇父母則泰卦乾下坤上天下於地是配享之卦師古曰王音于放翻原父曰但言父母得位安知是泰卦乎相息亮翻】所謂康彊之占逢吉之符也【洪範曰汝則從龜從筮從卿士從庶民從是之謂大同身其康彊子孫其逢吉】又以太牢策告宗廟有司奏故事聘皇后黄金二萬斤為錢二萬萬莽深辭讓受六千三百萬而以其四千三百萬分予十一媵家及九族貧者【九族上自高祖下至玄孫之親按漢書王莽傳故事聘皇后黄金二萬斤為錢二萬萬莽深辭讓受四千萬而以其三千三百萬予十一媵家羣臣復言今皇后受聘踰羣妾亡幾有詔復益二千三百萬為三千萬莽復以其千萬分予九族貧者又按杜佑通典聘后黄金二萬斤漢呂后為惠帝聘魯元公主女故事也予讀曰與】夏安漢公奏車服制度吏民養生送終嫁娶奴婢田宅器械之品立官稷【元始元年莽號安漢公至是始書以冠事表其所從來者漸矣通鑑凡書權臣例始此如淳曰郊祀志已有官社未有官稷遂立官稷於官社之後臣瓚曰漢初除秦社稷立漢社稷其後又立官社配以夏禹而不立官稷師古曰淳瓚二說皆未盡也初立官稷于官社之後是為一處今更創置建於别所不相從也】及郡國縣邑鄉聚皆置學官【張晏曰聚邑落名也師古曰聚音才喻翻】 大司徒司直陳崇使張敞孫竦草奏【師古曰草謂創立其文】盛稱安漢公功德以為宜恢公國令如周公【成王以周公有勛勞於天下封以曲阜地方七百里】建立公子令如伯禽【魯頌閟宮之詩曰王曰叔父建爾元子俾侯于魯】所賜之品亦皆如之【魯公之封于魯也賜以附庸殷民六族大路大旂封父之繁弱夏后氏之璜祝宗卜史備物典策官司彞器白牡之牲郊望之禮】諸子之封皆如六子【周公六子封於凡蔣邢茅胙祭師古曰六子伯禽之弟也祭側界翻】太后以示羣公羣公方議其事會呂寛事起初莽長子宇非莽隔絶衛氏【隔絶事見上卷元年長知兩翻】恐久後受禍即私與衛寶通書教衛后上書謝恩【上時掌翻下同】因陳丁傅舊惡冀得至京師莽白太皇太后詔有司褒賞中山孝王后益湯沐邑七千戶衛后日夜啼泣思見帝面而但益戶邑宇復教令上書求至京師【復扶又翻】莽不聽宇與師吳章及婦兄呂寛議其故章以莽不可諫而好鬼神【好呼到翻】可為變怪以驚懼之章因推類說令歸政衛氏【推類者因變怪而推言事類如洪範五行傳以說莽也說輸芮翻】宇即使寛夜持血灑莽第門吏發覺之莽執宇送獄飲藥死宇妻焉懷子【師古曰焉其名】繫獄須產子已殺之【師古曰須待也已訖也】甄邯等白太后下詔曰公居周公之位輔成王之主而行管蔡之誅不以親親害尊尊朕甚嘉之莽盡滅衛氏支屬唯衛后在吳章要斬磔尸東市門【要與腰同磔陟格翻裂也張也】初章為當世名儒【章治尚書經為博士】教授尤盛弟子千餘人莽以為惡人黨皆當禁錮不得仕宦門人盡更名他師【師古曰更以他人為師諱不言是章弟子更工衡翻】平陵云敞時為大司徒掾【姓譜䢵出自祝融之後為䢵國後去邑為云】自劾吳章弟子【劾戶槩翻】收抱章尸歸棺斂葬之【師古曰棺音工喚翻斂音力贍翻】京師稱焉莽于是因呂寛之獄遂窮治黨與連引素所惡者悉誅之【治直之翻下同惡烏路翻】元帝女弟敬武長公主素附丁傅【長知兩翻】及莽專政復非議莽【復扶又翻】紅陽侯王立莽之尊屬【立莽叔父也】平阿侯王仁素剛直【仁譚子也】莽皆以太皇太后詔遣使迫守令自殺【使疏吏翻下同令力丁翻】莽白太后主暴病薨【主言敬武公主】太后欲臨其喪莽固爭而止甄豐遣使者乘傳案治衛氏黨與【傳知戀翻】郡國豪桀及漢忠直臣不附莽皆誣以罪法而殺之何武鮑宣及王商子樂昌侯安【涿郡王商相成帝者也】辛慶忌三子護羌校尉通函谷都尉遵水衡都尉茂南郡太守辛伯皆坐死【何武不舉莽為大司馬鮑宣素冇彊項名王商與王鳳不協為所擠陷忿毒而死其子安不附王氏辛慶忌本王鳳所成莽見其三子皆能欲親厚之辛茂自以名臣子孫兄弟並顯列不宜附莽又不甚詘事甄豐甄邯伯亦辛氏之族故并及禍】凡死者數百人海内震焉北海逢萌謂友人曰三綱絶矣【莽殺其叔父又自殺其冢嫡是滅其天性也殺其君之祖姑又盡除忠直之臣是無君也故曰三綱絶矣逢皮江翻】不去禍將及人即解冠挂東都城門【萌時學於長安賢曰漢宮殿名東都門今名青門注又見昭紀】歸將家屬浮海客於遼東莽召明禮少府宗伯鳳【宗伯姓也鳳名也鳳明於禮官為少府少詩沼翻】入說為人後之誼白令公卿將軍侍中朝臣並聽【說為人後者義不得顧私親師古曰白令皆聽之朝直遥翻】欲以内厲天子而外塞百姓之議【厲諷厲也厲者磨錯垢故以就新取此義也師古曰塞止也塞悉則翻】先是秺侯金日磾子賞都成侯金安上子常皆以無子國絶【先悉薦翻秺音妬磾丁奚翻】莽以曾孫當及安上孫京兆尹欽紹其封【當日磾曾孫也】欽謂當宜為其父祖立廟【晉灼曰當是賞弟建之孫此言當自為其父及祖父建立廟也為于偽翻】而使大夫主賞祭也【如淳曰以賞故國君使大夫掌其祭事臣瓚曰當是支庶上繼大宗不得顧其私親也而欽令尊其父祖以續日磾不復為賞後而令大夫主掌祭事師古曰瓚說是】甄邯時在旁廷叱欽因劾奏欽誣祖不孝大不敬下獄自殺【劾戶槩翻下遐稼翻】邯以綱紀國體無所阿私忠孝尤著益封千戶更封安上曾孫湯為都成侯湯受封日不敢還歸家以明為人後之誼是歲尚書令潁川鍾元為大理【哀帝元夀二年復改廷尉為大理】潁川太守陵陽嚴詡【地理志陵陽縣屬丹陽郡】本以孝行為官【行下孟翻】謂掾史為師友【掾前絹翻】有過輒閉閤自責終不大言郡中亂【史言世降俗薄徒善不足以為政】王莽遣使徵詡官屬數百人為設祖道【為于偽翻】詡據地哭掾史曰明府吉徵不宜若此詡曰吾哀潁川士身豈有憂哉我以柔弱徵必選剛猛代代到將有僵仆者【師古曰僵偃也仆顚也僵音薑仆音赴】故相弔耳詡至拜為美俗使者【文穎曰宣美風化使者】徙隴西太守何並為潁川太守並到郡捕鍾元弟威及陽翟輕俠趙季李欵皆殺之郡中震栗【地理志陽翟縣屬潁川郡翟音直格翻】

  四年春正月郊祀高祖以配天宗祀孝文以配上帝【師古曰郊祀祀於郊也宗尊也祀于明堂也上帝太微五帝也一曰昊天上帝也王肅曰上帝天也馬融曰上帝泰一之神在紫微宮天之最尊者杜佑曰元氣廣大則稱昊天人之所尊莫過於帝託之於天故稱上帝】改殷紹嘉公曰宋公周承休公曰鄭公【成帝綏和元年封殷紹嘉公進周承休侯爵為公為二王後】 詔婦女非身犯法及男子年八十以上七歲已下家非坐不道詔所名捕【張晏曰名捕謂下詔所特捕也】它皆無得繫其當驗者即驗問【師古曰就其所居而問】定著令 二月丁未遣大司徒宮大司空豐等奉乘輿法駕迎皇后於安漢公第授皇后璽紱【乘繩證翻續漢志皇后綬與乘輿同四采黄赤縹紺長丈九尺九寸五百首璽蓋亦玉璽也師古曰紱所以繫璽音弗 考異曰王莽傳云四月丁未平紀云二月丁未立皇后王氏下云夏皇后見于高廟外戚傳云明年春迎皇后於安漢公第然則言四月者誤也】入未央宮大赦天下 遣太僕王惲等八人各置副假節【副副使也惲等持節其副則假之以節惲於粉翻】分行天下【行下孟翻】覽觀風俗 夏太保舜等及吏民上書者八千餘人咸請如陳崇言加賞於安漢公章下有司【下遐稼翻下同】有司請益封公以召陵新息二縣及黄郵聚新野田【新息召陵二縣屬汝南郡續漢志南陽郡新野縣有東鄉故新都王莽所封也又有黄郵聚聚才諭翻】采伊尹周公稱號加公為宰衡【伊尹曰阿衡周公位冢宰稱尺證翻】位上公三公言事稱敢言之賜公太夫人號曰功顯君【莽母也】封公子男二人安為褒新侯臨為賞都侯【莽封新都侯析其國名二字加褒賞以封其二子】加后聘三千七百萬合為一萬萬以明大禮太后臨前殿親封拜安漢公拜前二子拜後如周公故事【成王之侯伯禽于魯也周公拜前魯公拜後】莽稽首辭讓【稽音啓】出奏封事願獨受母號還安臨印韍及號位戶邑【韍即紱音弗】事下【下遐嫁翻】太師光等皆曰賞未足以直功【師古曰直當也】謙約讓公之常節終不可聽忠臣之節亦宜自屈而伸主上之義宜遣大司徒大司空持節承制詔公亟入視事詔尚書勿復受公之讓奏【復扶又翻下同】奏可莽乃起視事止減召陵黄郵新野之田而已莽復以所益納徵錢千萬遺太后左右奉共養者【遺于季翻下同白虎通曰納徵用玄纁共居用翻養弋向翻】莽雖專權然所以誑耀媚事太后下至旁側長御方故萬端賂遺以千萬數白尊太后姊妹號皆為君食湯沐邑【姊君挾為廣恩君君力為廣惠君君弟為廣施君】以故左右日夜共譽莽【譽音余】莽又知太后婦人厭居深宮中莽欲虞樂以市其權【樂音洛】乃令太后四時車駕巡狩四郊【張晏曰市權者以遊觀之樂易其權若市賈然師古曰虞與娯同邑外謂之郊近二十里也仲馮曰言郊不必二十里也】存見孤寡貞婦所至屬縣輒施恩惠賜民錢帛牛酒歲以為常太后旁弄兒病在外舍【服䖍曰官婢侍史生兒取以作弄兒也】莽自親候之其欲得太后意如此太保舜奏言天下聞公不受千乘之土【古者諸公之國地方百里出兵車千乘】辭萬金之幣【謂聘后之幣也】莫不鄉化【鄉讀曰嚮】蜀郡男子路建等輟訟慚作而雖文王卻虞芮何以加【師古曰卻退也虞芮二國名也並在河之東二國之君爭田不平聞文王之德乃往斷焉入周之境則耕者讓畔行者讓路乃相謂曰我小人也不可以履君子之庭遂相讓以其所爭為閒田而怍才各翻】宜報告天下于是孔光愈恐固稱疾辭位太后詔太師毋朝【朝直遥翻】十日一入省中置几杖賜餐十七物然後歸官屬按職如故【師古曰食具有十七種物言十日一入朝受此寵禮它日則常在家自養而其屬官依常各行職務】 莽奏起明堂辟雍靈臺【應劭曰明堂所以正四時出教化明堂上圓下方八窗四達布政之宮在國之陽上八窗法八風四達法四時九室法九州十二重法十二月三十六戶法三十六雨七十二牖法七十二風黄帝曰合宮冇虞曰總章殷曰陽館周曰明堂辟雍者象璧圜之以水象教化流行大戴禮明堂以茅蓋上圓下方天子曰靈臺諸侯曰觀臺以望氣書雲物】為學者築舍萬區【為于偽翻】制度甚盛立樂經益博士員經各五人徵天下通一藝教授十一人以上及有逸禮古書天文圖䜟【張衡曰圖䜟虛妄非聖人之法劉向父子領校祕書閱定九流亦無䜟録成哀之後乃始聞之】鍾律月令兵法史篇文字【孟康曰史籀所作十五篇古文書也師古曰周宣王太史史籀所作大篆書也籀直救翻】通知其意者皆詣公車綱羅天下異能之士前後至者千數皆令記說廷中將令正乖謬壹異說云【令各造廷中而記其說也】又徵能治河者以百數【治直之翻】其大略異者長水校尉平陵關並【姓譜闕夏大夫關龍逢之後風俗通關令尹喜之後】言河決率常於平原東郡左右其地形下而土疏惡【疏音疎】聞禹治河時本空此地以為水猥盛則放溢少稍自索【少詩沼翻師古曰猥多也索盡也音先各翻】雖時易處猶不能離此【離力智翻】上古難識近察秦漢以來河決曹衛之域【漢之濟隂定陶故曹國也東郡及魏郡黎陽古衛地也】其南北不過百八十里可空此地勿以為官亭民室而已御史臨淮韓牧以為可略於禹貢九河處穿之縱不能為九但為四五宜有益大司空掾王横言河入勃海地高於韓牧所欲穿處往者天常連雨東北風海水溢西南出寖數百里九河之地已為海所漸矣【師古曰漸寖也讀如本字又音子亷翻】禹之行河水本隨西山下東北去【師古曰行謂通流也】周譜云定王五年河徙【如淳曰譜音補世統譜諜也】則今所行非禹之所穿也又秦攻魏決河灌其都【事見七卷秦始皇二十二年】決處遂大不可復補【復扶又翻】宜卻徙完平處更開空【空音孔師古曰空猶穿】使緣西山足乘高地而東北入海乃無水災【西山謂黎陽以西諸山】司空掾沛國桓譚典其議為甄豐言【為于偽翻下同】凡此數者必有一是宜詳考驗皆可豫見計定然後舉事費不過數億萬亦可以事諸浮食無產業民【師古曰事謂役使也】空居與行役同當衣食衣食縣官而為之作乃兩便【師古曰言無產業之人端居無為及發行力役俱須衣食耳今縣官給其衣食而使修治河水是為公私兩便也】可以上繼禹功下除民疾時莽但崇空語無施行者 羣臣奏言昔周公攝政七年制度乃定今安漢公輔政四年營作二旬大功畢成宜升宰衡位在諸侯王上詔曰可仍令議九錫之法【應劭曰九錫一曰車馬二曰衣服三曰樂器四曰朱戶五曰納陛六曰虎賁百人七曰鈇鉞八曰弓矢九曰秬鬯此皆天子制度尊之故事事錫與但數少耳張晏曰九錫經本無文周禮以為九命春秋說有之臣瓚曰九錫備物覇者之盛禮齊桓晉文猶不能備鄭玄曰按九錫之名古無有也王制三公一命衮若有加則賜也不過九命孔穎達曰鄭意以為九命之外别加九賜案禮緯含文嘉上列九錫之差下云四方所瞻之成侯子所望宋均注云九錫者乃四方所共見公侯伯子男所希望孔引含文嘉所謂九錫與應劭同獨樂器曰樂則耳宋均注云進退有節行步有度賜之車馬以代其步言成文章行成法則賜之衣服以表其德長於教訓内懷至仁賜之樂則以化其民居處修理房内不渫賜之朱戶以明其别動作有禮賜之納陛以安其體勇猛勁疾執義堅彊賜之虎賁以備非常亢揚威武志在宿衛賜之斧鉞使得專殺内懷仁德執㒸不傾賜之弓矢使得專征孝慈父母賜之秬鬯以事先祖】 莽奏尊孝宣廟為中宗孝元廟為高宗又奏毁孝宣皇考廟勿修【宣帝元康元年尊悼園曰皇考】罷南陵雲陵為縣【南陵文帝母薄太后陵雲陵昭帝母趙太后陵】奏可 莽自以北化匈奴東致海外南懷黄支【莽自奏曰越裳氏重譯獻白雉黄支自三萬里貢犀東夷王度大海奉國珍匈奴單于順制去二名】唯西方未有加乃遣中郎將平憲等多持金幣誘塞外羌使獻地願内屬憲等奏言羌豪良願等種可萬二千人願為内臣【種章勇翻】獻鮮水海允谷鹽池【地理志金城郡臨羌縣西北至塞外有西海鹽池闞駰云西有卑禾羌海即獻王莽地為西海郡者也酈道元曰世謂之青海東去西平二百五十里】平地美艸皆與漢民自居險阻處為藩蔽問良願降意【降戶江翻】對曰太皇太后聖明安漢公至仁天下太平五穀成孰或禾長丈餘【長直亮翻】或一粟三米或不種自生或繭不蠶自成甘露從天下醴泉自地出鳳皇來儀神爵降集從四歲以來【謂自莽輔政以來也】羌人無所疾苦故思樂内屬宜以時處業【處謂度地以處之業謂使各有作業也樂音洛處昌呂翻】置屬國領護事下莽莽復奏【下遐稼翻】今已有東海南海北海郡請受良願等所獻地為西海郡分天下為十二州應古制奏可冬置西海郡 【考異曰王莽傳置西海郡在明年秋今從平紀】又增法五十條犯者徙之西海徙者以千萬數民始怨矣 梁王立坐與衛氏交通廢徙南鄭自殺【衛氏帝外家也】 分京師置前煇光後丞烈二郡【前煇光蓋領長安以南諸縣後丞烈蓋領長安以北諸縣也】更公卿大夫八十一元士官名位次及十二州名【更工衡翻】分界郡國所屬罷置改易天下多事吏不能紀矣

  五年春正月祫祭明堂【應劭曰禮五年而再殷祭壹禘壹祫祫祭者毁廟之主皆合食於太祖師古曰祫音合】諸侯王二十八人列侯百二十人宗室子九百餘人徵助祭禮畢皆益戶賜爵及金帛增秩補吏各有差【已封者益戶未有爵者賜爵已有爵者賜金帛已有秩者增秩未有官者補吏】 安漢公又奏復長安南北郊三十餘年間天地之祠凡五徙焉【成帝建始元年罷甘泉泰畤汾隂后土祠作長安南北郊永始三年復甘泉汾隂成帝崩皇太后詔復長安南北郊哀帝建平三年復甘泉汾隂今又復南北郊是五徙也】 詔曰宗室子自漢元至今十餘萬人其令郡國各置宗師以糾之【漢元漢初也師古曰糾謂禁察也】致教訓焉 夏四月乙未博山簡烈侯孔光薨贈賜葬送甚盛車萬餘兩【兩音亮】以馬宮為太師 吏民以莽不受新野田而上書者前後四十八萬七千五百七十二人及諸侯王公列侯宗室見者【見賢遍翻】皆叩頭言宜亟加賞於安漢公於是莽上書言諸臣民所上章下議者事皆寑勿上使臣莽得盡力畢制禮作樂事成願賜骸骨歸家避賢者路【言久處大位妨賢者進用之路避位所以避賢者路也】甄邯等白太后詔曰公每見輒流涕叩頭言願不受賞賞即加不敢當位方制作未定事須公而決故且聽公制作畢成羣公以聞究于前議【究竟也】其九錫禮儀亟奏五月策命安漢公莽以九錫莽稽首再拜受緑韍衮冕衣裳【師古曰此韍謂蔽膝也或謂韍韠韍韠音弗畢】瑒琫瑒珌【孟康曰瑒玉名也佩刀之飾上曰琫下曰珌詩云鞞琫有珌是也毛傳曰鞸容刀鞸也琫上飾珌下飾天子玉琫而珧珌諸侯璗琫而璆珌陸云鞞刀室也琫佩刀削上飾珌佩刀下飾爾雅云黄金謂之璗說文云璗金之美與玉同色者也師古曰瑒音蕩琫音布孔翻珌音必】句履【孟康曰今齊祀履頭飾也出履三寸師古曰其形岐頭句音巨俱翻】鸞路乘馬【師古曰鸞路車之施鸞者也四馬曰乘音食證翻】龍旂九旒【周禮交龍為旂爾雅有鈴曰旂旒旂之末垂也】皮弁素積戎路乘馬【師古曰戎路戎車也】彤弓矢盧弓矢【師古曰彤赤色盧黑色】左建朱鉞右建金戚【師古曰鉞戚皆斧屬】甲胄一具【胄兜鍪】秬鬯二卣【秬鬯香酒也周禮春官鬯人注云釀秬為酒秬如黑黍一稃二米陸佃埤雅曰說文鬯以秬釀鬱艸芬芳攸服以降神也舊說芬芳條暢故謂之鬯禮以鬱合鬯言鬱於中而鬯於外也又曰先鄭小毛以為鬯香艸也築而煮之為鬯秬者百穀之華鬯者百艸之英故先王煮以合鬯卣中尊也秬音巨卣音攸又音羊久翻】圭瓚三九命青玉珪二【師古曰圭瓚以圭為勺末上公九命青者春色東方生而長育萬物也】朱戶納陛【朱戶以居納陛以登孟康曰納内也謂鑿殿基際為陛不使露也師古曰孟說是也尊者不欲露而升陛故納之於霤下也】署宗官祝官卜官史官【放周公也成王之命周公祝宗卜史杜預曰太祝宗人太卜太史凡四官】虎賁三百人【孔安國曰虎賁勇士稱也若虎賁戰言其猛也賁音奔】 王惲等八人使行風俗還【惲於粉翻行下孟翻下同】言天下風俗齊同詐為郡國造歌謡頌功德凡三萬言閏月丁酉詔以羲和劉秀等四人使治明堂辟雍【治直之翻】令漢與文王靈臺周公作洛同符【詩曰經始靈臺經之營之庶民攻之不日成之周公營成周曰其作大邑其自時配皇天中乂】太僕王惲等八人使行風俗宣明德化萬國齊同皆封為列侯【四人者劉秀紅休侯平晏防鄉侯孔永寧鄉侯孫遷定鄉侯八人者王惲常鄉侯閻遷望鄉侯陳崇南鄉侯李翕邑鄉侯郝黨亭鄉侯謝殷章鄉侯逯普蒙鄉侯陳鳳盧鄉侯 考異曰恩澤侯表劉歆等十一侯皆云丁酉獨平晏云丁丑按十二人同功俱封是年閏五月甲午朔無丁丑表誤】時廣平相班穉獨不上嘉瑞及歌謡【班穉時相廣平王漢武帝征和二年於廣平置平于國宣帝五鳳二年復曰廣平上時掌翻下同】琅邪太守公孫閎言災害於公府甄豐遣屬馳至兩郡【續漢志大司空掾屬二十九人掾比三百石屬比二百石杜佑曰正曰掾副曰屬】諷吏民【師古曰遣言祥應而隱除災害】而劾閎空造不祥穉絶嘉應嫉害聖政皆不道【劾戶槩翻】穉班偼伃弟也【偼伃音接予】太后曰不宣德美宜與言災害者異罰且班穉後宮賢家我所哀也【師古曰班偼伃有賢德故哀閔其家】閎獨下獄誅【下遐稼翻】穉懼上書陳恩謝罪【陳恩者自陳述世受國恩】願歸相印入補延陵園郎【園郎掌守園寢門戶】太后許焉 莽又奏為市無二賈【師古曰言純質也賈讀曰價】官無獄訟邑無盜賊野無飢民道不拾遺男女異路之制犯者象刑【師古曰白虎通云象者其衣服象五刑也犯墨者蒙犯劓者以赭著其衣犯髕者以墨蒙其髕象而畫之犯宮者屝犯大辟者布衣無領屝草屨也髕音頻忍翻屝音扶味翻】 莽復奏言共王母丁姬前不臣妾【師古曰言不遵臣妾之道復扶又翻共音恭】冢高與元帝山齊【賈公彦曰爾雅山頂冡則山冢之冢封土為丘壠曰冡則冢墓之冢冡而隴翻】懷帝太后皇太太后璽綬以葬【師古曰懷謂挾之以自隨也璽斯氏翻綬音受】請發共王母及丁姬冢取其璽綬徙共王母歸定陶葬共王冢次太后以為既己之事不須復發【復扶又翻】莽固爭之太后詔因故棺改葬之莽奏共王母及丁姬棺皆名梓宮珠玉之衣【謂之梓宮者以香梓為之言猶生時所居宮室也珠玉之衣珠襦玉匣也】非藩妾服請更以木棺代【更工衡翻下同】去珠玉衣【去羌呂翻】葬丁姬媵妾之次奏可公卿在位皆阿莽指入錢帛遣子弟及諸生四夷凡十餘萬人操持作具【作具畚鍤之類操七刀翻】助將作掘平共王母丁姬故冢二旬間皆平莽又周棘其處以為世戒云【師古曰以棘周繞也】又隳壞共皇廟諸造議者泠褒段猶皆徙合浦【褒猶奏見三十三卷哀帝建平元年 考異曰師丹傳云復免高昌侯宏為庶人按功臣表建平四年董宏已死元夀二年子武坐父為佞邪免不得至今丹傳誤也】徵師丹詣公車賜爵關内侯食故邑數月更封丹為義陽侯【丹建平元年罷歸故邑高樂侯戶邑也恩澤侯表義陽侯國于南陽新野 考異曰恩澤侯表丹元始三年二月癸巳更為義陽侯胡旦因此并發傅太后陵徙泠褒等事俱著之三年按外戚傳云元始五年莽發共王母及丁姬冢改葬之馬宮傳莽發傅太后陵追誅前議者宮慚懼乃乞骸骨公卿表宮以今年八月壬午免然則褒等徙合浦及丹封侯皆在今年明矣按長歷二月丙申朔無癸巳日月必有誤者】月餘薨初哀帝時馬宮為光禄勲與丞相御史雜議傅太后諡曰孝元傅皇后及莽追誅前議者宮為莽所厚獨不及宮内慙懼上書言臣前議定陶共王母諡希指雷同詭經僻說【師古曰詭違也】以惑誤主上為臣不忠幸蒙洒心自新【師古曰洒音先禮翻】誠無顔復望闕庭無心復居官府無宜復食國邑【宮封扶德侯邑於琅邪贑榆復扶又翻】願上太師大司徒扶德侯印綬避賢者路【上時掌翻下同】八月壬午莽以太后詔賜宮策曰四輔之職為國維綱三公之任鼎足承君不有鮮明固守無以居位【鮮明猶言精明也】君言至誠不敢文過朕甚多之【師古曰多猶重也】不奪君之爵邑其上太師大司徒印綬使者【上印綬於使者也】以侯就第莽以皇后有子孫瑞通子午道【張晏曰時年十四始有婦人之道也子水午火也水以天一為牡火以地二為牝故火為水妃今通子午以協之按男八月生齒八歲毁齒二八十六陽道通八八六十四陽道絶女七月生齒七歲毁齒二七十四隂道通七七四十九隂道絶】從杜陵直絶南山徑漢中【師古曰子北方也午南方也言通南北道相當故謂之子午耳今京城直南山有谷通梁漢道者名子午谷又宜州西界慶州東界有山名子午嶺計南北直相當此則北山是子南山是午共為子午道仲馮曰史文自以從杜陵徑漢中為子午道耳顔說非史意也三秦記長安正南山名秦嶺谷名子午一名樊川一名御宿】 泉陵侯劉慶上書【師古曰衆陵節侯賢長沙定王子本始四年戴侯眞定嗣二十二年薨黄龍元年頃侯慶嗣此則是也莽傳及翟義傳並云泉陵地理志泉陵縣屬零陵郡而表作衆陵表為誤也】言周成王幼小周公居攝今帝富於春秋宜令安漢公行天子事如周公羣臣皆曰宜如慶言 時帝春秋益壯以衛后故怨不悦【謂衛后不得至京師其族皆死徙故不悦】冬十二月莽因臘日上椒酒【上時掌翻】置毒酒中帝有疾莽作策請命於泰畤願以身代藏策金縢置于前殿敕諸公勿敢言【師古曰詐依周公為武王請命作金縢也書曰周公納策金縢之匱中孔安國曰為請命之書藏之於匱緘之以金不欲人開之孔穎達曰縢是縛約之名】丙午帝崩于未央宮【臣瓚曰帝年九歲即位即位五年夀十四】大赦天下莽令天下吏六百石以上皆服喪三年奏尊孝成廟曰統宗孝平廟曰元宗歛孝平加元服【歛力贍翻】葬康陵【臣瓚曰康陵在長安北六十里】

  班固贊曰孝平之世政自莽出褒善顯功以自尊盛觀其文辭方外百蠻無思不服休徵嘉應頌聲並作至于變異見於上【見賢遍翻】民怨於下莽亦不能文也【如淳曰不可復文飾也】

  以長樂少府平晏為大司徒 太后與羣臣議立嗣時元帝世絶而宣帝曾孫有見王五人【王之見在者五人淮陽王縯中山王成都楚王紆信都王景東平王開明也見賢遍翻】列侯四十八人【廣戚侯顯陽興侯寄陵陽侯嘉高樂侯修平邑侯閔平纂侯況合昌侯輔伊鄉侯開就鄉侯不害膠鄉侯武宜鄉侯恢昌城侯豐樂安侯禹陶鄉侯恢釐鄉侯褒昌鄉侯且新鄉侯鯉郚鄉侯光新城侯武宜陵侯封堂鄉侯護成陵侯由成陽侯衆復昌侯休安陸侯平梧安侯譽朝鄉侯充扶鄉侯普方城侯宣當陽侯益廣城侯疌春城侯允呂鄉侯尚李鄉侯殷宛鄉侯隆夀泉侯承杏山侯遵嚴鄉侯信武平侯璜陵鄉侯曾武安侯㥅富陽侯萌西陽侯偃桃鄉侯立栗鄉侯玄成金鄉侯不害平通侯且西安侯漢湖鄉侯開重鄉侯少柏凡五十人而廣戚侯顯孺子之父栗鄉侯玄成先已免侯止四十八人耳師古曰疌音竹二翻㥅音受】莽惡其長大【惡音烏路翻長音知兩翻下同】曰兄弟不得相為後乃悉徵宣帝玄孫選立之是月前煇光謝囂奏武功長孟通浚井得白石【武功縣本屬扶風莽分屬前煇光師古曰浚抒治之也囂音許驕翻浚音峻抒音直呂翻】上圓下方有丹書著石【師古曰著音直略翻】文曰告安漢公莽為皇帝符命之起自此始矣莽使羣公以白太后太后曰此誣罔天下不可施行太保舜謂太后曰事已如此無可奈何沮之力不能止【沮慈呂翻】又莽非敢有它但欲稱攝以重其權填服天下耳【師古曰填音竹刃翻】太后心不以為可然力不能制乃聽許舜等即共令太后下詔曰孝平皇帝短命而崩已使有司徵孝宣皇帝玄孫二十三人差度宜者【師古曰差度謂擇也度音大各翻】以嗣孝平皇帝之後玄孫年在襁褓不得至德君子孰能安之安漢公莽輔政三世與周公異世同符今前煇光囂武功長通上言丹石之符朕深思厥意云為皇帝者乃攝行皇帝之事也其令安漢公居攝踐祚如周公故事【祚位也】具禮儀奏于是羣臣奏言太后聖德昭然深見天意詔令安漢公居攝臣請安漢公踐祚服天子韍冕背斧依立于戶牖之間【背蒲妹翻鄭氏曰斧依為斧文屏風師古曰依讀曰扆音於豈翻】南面朝羣臣聽政事【朝直遥翻下同】車服出入警蹕民臣稱臣妾皆如天子之制郊祀天地宗祀明堂共祀宗廟享祭羣神贊曰假皇帝【師古曰贊者謂祭祝之辭共音恭予謂此贊固主于祭祝若朝會亦有贊者所謂贊拜贊謁是也】民臣謂之攝皇帝自稱曰予平決朝事【朝直遥翻】常以皇帝之詔稱制以奉順皇天之心輔翼漢室保安孝平皇帝之幼嗣遂寄託之義【寄託謂寄以天下託以孤幼也師古曰遂成也】隆治平之化【治直吏翻】其朝見太皇太后帝皇后皆復臣節【見賢遍翻帝皇后謂平帝后也復如字反也還也】自施政教於宮家國采【宮者謂以安漢公第為宮也家者謂其家也國者謂其所封新都國也采謂以武功縣為采地名曰漢光邑也師古曰采官也以官受地故謂之采采音七在翻又音七代翻】如諸侯禮儀故事太后詔曰可

  王莽上【字巨君孝元皇后之弟子也莽父曼祖禁禁武帝繡衣御史賀之子也】

  居攝元年【莽既攝政遂改元為居攝】春正月王莽祀上帝于南郊又行迎春大射養老之禮【上無天子通鑑不得不以王莽繫年不書假皇帝而直書王莽者不與其攝也及其既簒也書莽不與其簒也呂后武后書太后其義亦然】 三月己丑立宣帝玄孫嬰為皇太子號曰孺子【亦因周公輔成王二叔流言曰公將不利於孺子而為此號】嬰廣戚侯顯之子也【楚孝王子勲封廣戚侯顯則勲之子也地理志沛郡有廣戚侯國】年二歲託以卜相最吉立之【相息亮翻】尊皇后曰皇太后 以王舜為太傅左輔甄豐為太阿右拂【師古曰拂讀曰弼】甄邯為太保後承又置四少秩皆二千石【四少少師少傅少阿少保也少詩照翻】 四月安衆侯劉崇【師古曰安衆康侯丹長沙定王子崇即丹玄孫之子也見王子侯表地理志安衆侯國屬南陽郡故宛西鄉也】與相張紹謀曰安漢公莽必危劉氏天下非之莫敢先舉此乃宗室之恥也吾帥宗族為先海内必和【帥讀曰率和戶卧翻】紹等從者百餘人遂進攻宛【宛南陽郡治所宛於元翻】不得入而敗紹從弟竦與崇族父嘉詣闕自歸莽赦弗罪【從才用翻】竦因為嘉作奏【為于偽翻】稱莽德美罪狀劉崇願為宗室倡始【師古曰倡先向翻】父子兄弟負籠荷鍤【師古曰籠所以盛土鍤鍫也荷下可翻又音何】馳之南陽豬崇宮室令如古制【古者畔逆之國既伏其罪則豬其宮室以為汙池師古曰豬謂畜水也】及崇社宜如亳社以賜諸侯用永監戒【武王勝殷分亳社以班諸侯四牆其社覆上棧下使不得通隂陽之氣所以著亡國之戒也】於是莽大說【說讀曰悦】封嘉為率禮侯嘉子七人皆賜爵關内侯後又封竦為淑德侯長安為之語曰欲求封過張伯松【師古曰伯松張竦之字】力戰鬭不如巧為奏自後謀反皆汙池云【師古曰汙下也音烏】羣臣復白劉崇等謀逆者以莽權輕也【復扶又翻】宜尊重以填海内【填竹刃翻】五月甲辰太后詔莽朝見太后稱假皇帝【朝直遥翻見賢遍翻】冬十月丙辰朔日有食之 十二月羣臣奏請以安漢公廬為攝省府為攝殿第為攝宮奏可【廬殿中止宿之舍府治事之所第所居也】 是歲西羌龎恬傅幡等【師古曰幡音敷元翻】怨莽奪其地反攻西海太守程永永犇走莽誅永遣護羌校尉竇況擊之

  二年春竇況等擊破西羌 五月更造貨【更工衡翻】錯刀一直五千契刀一直五百大錢一直五十【食貨志錯刀以黄金錯其文曰一刀直五千契刀其環如大錢身形如刀長二寸文曰契刀五百大錢徑寸二分重十二銖文曰大錢五十張晏曰按今所見契刀錯刀形質如大錢而肉好輪厚異於此大錢形如大刀環矣契刀身形圓不長二寸也其文左曰契右曰刀無五百字也錯刀則刻之作字也以黄金填其文上曰一下曰刀二刀泉甚不與志相應也似相傳差錯文字磨滅故耳師古曰張說非也王莽錢刀今並尚在形質及文與志相合無差錯也索隱曰錢本名泉以貨之流布如泉布者言貨流布刀以其利於人也】與五銖錢並行民多盜鑄者禁列侯以下不得挾黄金輸御府受直【百官表少府有御府令丞師古曰御府主天子衣服】然卒不與直【卒子恤翻】 東郡太守翟義方進之子也【翟直格翻】與姊子上蔡陳豐謀曰【地理志上蔡縣屬汝南郡】新都侯攝天子位號令天下故擇宗室幼稚者以為孺子【故意為之曰故雅直利翻】依託周公輔成王之義且以觀望【師古曰言漸試天下人心】必代漢家其漸可見方今宗室衰弱外無彊蕃天下傾首服從莫能亢扞國難【亢口浪翻禦也扞尸榦翻難乃旦翻】吾幸得備宰相子身守大郡父子受漢厚恩義當為國討賊以安社稷【為于偽翻】欲舉兵西誅不當攝者選宗室子孫輔而立之設令時命不成死國埋名【師古曰埋名謂身埋而名立】猶可以不慙於先帝今欲發之汝肯從我乎豐年十八勇壯許諾義遂與東郡都尉劉宇嚴鄉侯劉信信弟武平侯劉璜結謀【信璜皆東平煬王雲子嚴鄉武平二國蓋皆在東郡璜胡光翻】以九月都試日斬觀令【地理志觀縣屬東郡本曰畔觀應劭曰夏有觀扈世祖改為衛公國以封周後師古曰觀音工喚翻】因勒其車騎材官士募郡中勇敢部署將帥信子匡時為東平王乃并東平兵立信為天子義自號大司馬柱天大將軍移檄郡國言莽鴆殺孝平皇帝攝天子位欲絶漢室今天子已立共行天罰【師古曰共讀曰恭】郡國皆震比至山陽衆十餘萬【比必寐翻】莽聞之惶懼不能食太皇太后謂左右曰人心不相遠也【師古曰言所見者同】我雖婦人亦知莽必以此自危莽乃拜其黨親【孫建劉宏竇況莽之黨也王邑王駿王況王昌莽之親也】輕車將軍成武侯孫建為奮武將軍光禄勲成都侯王邑為虎牙將軍明義侯王駿為彊弩將軍春王城門校尉王況為震威將軍【師古曰春王長安城東出北頭第一門也本名宣平門莽改名焉予按漢城門校尉掌十二城門觀此則莽改官名十二城門各置城門校尉】宗伯忠孝侯劉宏為奮衝將軍【平帝元始四年莽更名宗正為宗伯】中少府建威侯王昌為中堅將軍【莽更少府曰共工此中少府蓋長樂少府也以職在宮中故曰中少府】中郎將震羌侯竇況為奮威將軍凡七人自擇除關西人為校尉軍吏將關東甲卒發奔命以擊義焉【校戶教翻將即亮翻】復以太僕武讓為積弩將軍屯函谷關【復扶又翻下同】將作大匠蒙鄉侯逯並為横壄將軍屯武關【師古曰逯姓也並名也逯音録又音鹿今東郡有逯姓二音並得書本逯字或作逮今河朔有逮姓自呼音徒戴翻其義兩通】羲和紅休侯劉秀為揚武將軍屯宛【宛於元翻】三輔聞翟義起自茂陵以西至汧【地理志汧縣屬右扶風音口堅翻賢曰汧故城在隴州汧源縣南】二十三縣盜賊並發槐里男子趙朋霍鴻等自稱將軍攻燒官寺殺右輔都尉及斄令【地理志右輔都尉治郿郿與斄縣皆屬扶風斄周后稷所封邑也師古曰斄與邰同音胎】相與謀曰諸將精兵悉東京師空可攻長安衆稍多至十餘萬火見未央宮前殿【見賢遍翻】莽復拜衛尉王級為虎賁將軍【賁音奔】大鴻臚望鄉侯閻遷為折衝將軍西擊朋等以常鄉侯王惲為車騎將軍屯平樂館【樂音洛】騎都尉王晏為建威將軍屯城北城門校尉趙恢為城門將軍皆勒兵自備以太保後承承陽侯甄邯為大將軍【承陽之承音烝】受鉞高廟領天下兵左杖節右把鉞屯城外王舜甄豐晝夜循行殿中【行下孟翻】莽日抱孺子禱郊廟會羣臣稱曰昔成王幼周公攝政而管蔡挾禄父以畔【師古曰禄父紂子也父讀曰甫】今翟義亦挾劉信而作亂自古大聖猶懼此況臣莽之斗筲【師古曰斗筲喻材器小也】羣臣皆曰不遭此變不章聖德冬十月甲子莽依周書作大誥【師古曰武王崩周公相成王而三監及淮夷叛周公作大誥莽自比周公故依倣其事】曰粤其聞日【孟康曰翟義反書上聞日也師古曰粤發語辭】宗室之儁有四百人【孟康曰諸劉見在者】民獻儀九萬夫【孟康曰民之表儀謂賢者也】予敬以終於此謀繼嗣圖功【師古曰我用此宗室之儁及獻儀者共圖謀國事終成其功】遣大夫桓譚等班行諭告天下以當反位孺子之意諸將東至陳留菑【孟康曰菑故戴國在梁後屬陳留今曰考城陳留風俗傳曰菑縣秦之穀縣也遭漢兵起邑多菑年故改曰菑縣章帝東巡過縣詔曰陳留菑縣其名不善其改曰考城】與翟義會戰破之斬劉璜首莽大喜復下詔先封車騎都尉孫賢等五十五人皆為列侯即軍中拜授因大赦天下于是吏士精鋭遂攻圍義於圉城十二月大破之義與劉信弃軍亡至固始界中捕得義尸磔陳都市卒不得信【地理志圉固始陳三縣皆屬淮陽國賢曰圉故城在今汴州雍丘縣東南卒子恤翻】

  初始元年【是年十一月莽始改元始初】春地震大赦天下 王邑等還京師西與王級等合擊趙朋霍鴻二月朋等殄滅諸縣悉平還師振旅莽乃置酒白虎殿勞賜將帥【勞力到翻】詔陳崇治校軍功第其高下【治直之翻校古效翻】依周制爵五等以封功臣為侯伯子男凡三百九十五人曰皆以奮怒東指西擊羌寇蠻盜反虜逆賊不得旋踵應時殄滅天下咸服之功封云其當賜爵關内侯者更名曰附城又數百人【項安世家說曰王莽封諸侯置附城漢人蓋以城解墉也古文庸即墉字後人加土以别之不成國者謂之附城猶今言支郡為屬城也予按王制不能五十里者不達於天子附於諸侯曰附庸鄭注曰小城曰附庸附庸者以國事附於大國正義曰庸城也謂小國之城不能自通以其國事附於大國故曰附庸項說本諸此更工衡翻】莽發翟義父方進及先祖冢在汝南者燒其棺柩【柩音舊】夷滅三族誅及種嗣【種章勇翻嗣祥吏翻】至皆同阬以棘五毒并葬之【如淳曰五毒野葛狼毒之屬翟方進本汝南上蔡人】又取義及趙朋霍鴻黨衆之尸聚之通路之旁濮陽無鹽圉槐里盩厔凡五所【濮陽無鹽圉義黨之尸也槐里盩厔朋鴻黨之尸也盩厔音舟窒】建表木於其上【師古曰表者所以標明也】書曰反虜逆賊䲔鯢【師古曰䲔鯢大魚為害者也以此比敵人之勇桀者崔豹古今注鯨大者長千里小者數丈一生數萬子常以五六月就岸生子至七八月導從其子還大海中鼔浪成雷噴沫成雨水族驚畏一皆逃匿莫敢當其雌曰鯢大者亦長千里蓋鯨鯢有力能驅食小魚故以喻夫強暴而凌弱者而導從數萬子跋扈大海中亦有渠魁之義䲔古鯨字音其京翻鯢五奚翻】義等既敗莽於是自謂威德日盛遂謀即眞之事矣羣臣復奏進攝皇帝子安臨爵為公【復扶又翻下同】封兄子光為衍功侯【光莽兄永之子】是時莽還歸新都國【莽既居攝故還歸新都國】羣臣復白以封莽孫宗為新都侯【宗宇子也】 九月莽母功顯君死莽自以居攝踐祚奉漢大宗之後為功顯君緦縗弁而加麻環絰如天子弔諸侯服【周禮王為諸侯緦縗弁而加環絰同姓則麻異姓則葛師古曰於弁上加環絰也謂之環者言其輕細如環之形記曰緦麻十五升去其半有事其縷無事其布曰緦賈公彦曰凡五服之絰皆兩股絞之言環絰則與絞絰有異矣謂以麻為體又以一股麻為糾而横纒之如環然故謂之環絰縗倉囘翻】凡壹弔再會而令新都侯宗為主服喪三年云 司威陳崇【莽置司威以司察百官】奏莽兄子衍功侯光私報執金吾竇況令殺人【私報者私屬之也霍顯曰少夫幸報我以事】況為收繫致其法【為于偽翻】莽大怒切責光光母曰汝自視孰與長孫中孫長孫中孫者宇及獲之字也【獲死見上卷哀帝元夀元年宇死見上平帝元始三年師古曰中讀曰仲】遂母子自殺及況皆死初莽以事母養嫂撫兄子為名【事見三十一卷成帝永始元年】及後悖虐復以示公義焉【服中曰不舍光罪為公義仲馮曰莽不服母喪亦以示公義悖蒲内翻又蒲没翻】令光子嘉嗣爵為侯 是歲廣饒侯劉京言齊郡新井【地理志齊郡有廣饒縣】車騎將軍千人扈雲言巴郡石牛【師古曰千人官名屬車騎將軍扈其姓雲其名予按百官表千人在侯司馬之下】太保屬臧鴻言扶風雍石【漢公府有掾有屬姓譜魯孝公子彄食采於臧子孫以為氏雍縣屬扶風雍於用翻】莽皆迎受十一月甲子莽奏太后曰陛下遇漢十二世三七之阸【三七二百一十年漢元至是歲二百一十四年】承天威命詔臣莽居攝廣饒侯劉京上書言七月中齊郡臨淄縣昌興亭長辛當一暮數夢曰吾天公使也【使疏吏翻】天公使我告亭長攝皇帝當為眞即不信我此亭中當有新井亭長晨起視亭中誠有新井【師古曰誠實也】入地且百尺十一月壬子直建冬至【師古曰壬子之日冬至而其日當建】巴郡石牛戊午雍石文皆到于未央宮之前殿臣與太保安陽侯舜等視天風起塵冥風止得銅符帛圖於石前文曰天告帝符獻者封侯騎都尉崔發等視說【師古曰視其文而說其意也】孔子曰畏天命畏大人畏聖人之言【師古曰論語所載孔子之言】臣莽敢不承用臣請共事神祇宗廟【師古曰共讀曰恭】奏言太皇太后孝平皇后皆稱假皇帝其號令天下天下奏言事毋言攝以居攝三年為始初元年 【考異曰莽傳作初始荀紀及韋莊美嘉號録宋庠紀元通譜皆作始初今從之】漏刻以百二十為度用應天命臣莽夙夜養育隆就孺子【師古曰隆長也成就之使其長大也】令與周之成王比德宣明太皇太后威德於萬方期於富而教之孺子加元服復子明辟如周公故事【書洛誥周公拜手稽首曰朕復子明辟孔安國注曰周公盡禮致敬言我復還明君之政於子子成王也年二十成人故必歸政】奏可衆庶知其奉符命指意羣公博議别奏以示即眞之漸矣 期門郎張充等六人謀共刧莽立楚王【楚王紆宣帝之曾孫】發覺誅死 梓潼人哀章【師古曰梓潼廣漢之縣也潼音童哀姓章名姓譜曰哀姓以謚為氏予按古人以哀為謚非一孔子弟子傳有公晳哀烈士傳有羊角哀獨不可以為出於其後乎】學問長安素無行好為大言【行下孟翻好呼到翻】見莽居攝即作銅匱為兩檢【檢居掩翻毛晃曰檢書檢印窠封題也】署其一曰天帝行璽金匱圖其一署曰赤帝璽某傳予皇帝金策書【予讀曰與】某者高皇帝名也書言王莽為眞天子皇太后如天命圖書皆書莽大臣八人又取令名王興王盛章因自竄姓名【師古曰竄謂厠著也】凡十一人皆署官爵為輔佐章聞齊井石牛事下【下遐稼翻】即日昏時衣黄衣【時衣於既翻】持匱至高廟以付僕射【高廟有令有僕射】僕射以聞戊辰莽至高廟拜受金匱神禪【師古曰言有神命使漢禪位於莽也】御王冠【王者之冠也】謁太后還坐未央宮前殿下書曰予以不德託于皇初祖考黄帝之後皇始祖考虞帝之苗裔【詳見下卷】而太皇太后之末屬皇天上帝隆顯大佑成命統序符契圖文金匱策書神明詔告屬予以天下兆民【師古曰屬委付也音之欲翻】赤帝漢氏高皇帝之靈承天命傳金策之書予甚祇畏敢不欽受以戊辰直定【師古曰以建除之次其日當定】御王冠即眞天子位定有天下之號曰新【因新都國以定號也】其改正朔易服色變犧牲殊徽幟異器制【師古曰徽幟通謂旌旗之屬也幟音式志翻】以十二月朔癸酉為始建國元年正月之朔以雞鳴為時【以十二月為正以丑時為十二時之始也】服色配德上黄犧牲應正用白【以土繼火故尚黄萬物紐牙於丑其色白故應正用白】使節之旄幡皆純黄其署曰新使五威節以承皇天上帝威命也【使疏吏翻】莽將即眞先奉諸符瑞以白太后太后大驚是時以孺子未立璽臧長樂宮【璽即傳國璽臧古藏字通】及莽即位請璽太后不肯授莽莽使安陽侯舜諭指舜素謹敕太后雅愛信之舜既見太后太后知其為莽求璽【為于偽翻】怒罵之曰而屬父子宗族【師古曰而汝也】蒙漢家力富貴累世既無以報受人孤寄乘便利時奪取其國【師古曰孤寄言以孤寄托之】不復顧恩義人如此者狗豬不食其餘【師古曰言惡賤】天下豈有而兄弟邪【言天下無此等人謂其全無人心也一曰言天下將共誅之不復有兄弟存也】且若自以金匱符命為新皇帝【師古曰若亦汝也】變更正朔服制【更工衡翻】亦當自更作璽傳之萬世何用此亡國不祥璽為而欲求之我漢家老寡婦旦暮且死欲與此璽俱葬終不可得太后因涕泣而言旁側長御以下皆垂涕【長御太后旁側常侍也】舜亦悲不能自止良久乃仰謂太后臣等已無可言者【師古曰言不可諫止】莽必欲得傳國璽太后寧能終不與邪太后聞舜語切恐莽欲脅之乃出漢傳國璽投之地以授舜曰我老已死知而兄弟今族滅也舜既得傳國璽奏之莽大說乃為太后置酒未央宮漸臺【說讀曰悦為于偽翻師古曰未央殿西南有蒼池池中有漸臺黄圖曰漸浸也言為池水所漸漸讀曰沾】大縱衆樂莽又欲改太后漢家舊號易其璽綬恐不見聽而莽疎屬王諫欲諂莽上書言皇天廢去漢而命立新室【去羌呂翻】太皇太后不宜稱尊號當隨漢廢以奉天命莽以其書白太后太后曰此言是也【師古曰恚忿之辭也】莽因曰此誖德之臣也【師古曰誖乖也音布内翻】罪當誅于是冠軍張永獻符命銅璧文【冠軍屬南陽郡服䖍曰銅璧如璧形以銅為之也冠古玩翻】言太皇太后當為新室文母太皇太后莽乃下詔從之于是鴆殺王諫而封張永為貢符子

  班彪贊曰三代以來王公失世稀不以女寵及王莽之興由孝元后歷漢四世為天下母饗國六十餘載羣小世權更持國柄【載子亥翻更工衡翻】五將十侯【師古曰五將者鳳音商根莽皆為大司馬十侯者陽平頃侯禁禁子敬侯鳳安成侯崇平阿侯譚成都侯商紅陽侯立曲陽侯根高平侯逢時安陽侯音新都侯莽也一曰鳳嗣禁為侯不當重數而十人者淳于長即其一也】卒成新都【卒子恤翻】位號已移于天下而元后卷卷猶握一璽【師古曰卷音其圓翻惓惓忠謹之意予謂此云卷卷猶眷戀也】不欲以授莽婦人之仁悲夫

  資治通鑑卷三十六


    


 


 



 

 
  







 


  
  
 
 
 


  

 















	
	









































 
  



















 





 












  
  
  

 





