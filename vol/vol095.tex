<!DOCTYPE html PUBLIC "-//W3C//DTD XHTML 1.0 Transitional//EN" "http://www.w3.org/TR/xhtml1/DTD/xhtml1-transitional.dtd">
<html xmlns="http://www.w3.org/1999/xhtml">
<head>
<meta http-equiv="Content-Type" content="text/html; charset=utf-8" />
<meta http-equiv="X-UA-Compatible" content="IE=Edge,chrome=1">
<title>資治通鑒_96-資治通鑑卷九十五_96-資治通鑑卷九十五</title>
<meta name="Keywords" content="資治通鑒_96-資治通鑑卷九十五_96-資治通鑑卷九十五">
<meta name="Description" content="資治通鑒_96-資治通鑑卷九十五_96-資治通鑑卷九十五">
<meta http-equiv="Cache-Control" content="no-transform" />
<meta http-equiv="Cache-Control" content="no-siteapp" />
<link href="/img/style.css" rel="stylesheet" type="text/css" />
<script src="/img/m.js?2020"></script> 
</head>
<body>
 <div class="ClassNavi">
<a  href="/24shi/">二十四史</a> | <a href="/SiKuQuanShu/">四库全书</a> | <a href="http://www.guoxuedashi.com/gjtsjc/"><font  color="#FF0000">古今图书集成</font></a> | <a href="/renwu/">历史人物</a> | <a href="/ShuoWenJieZi/"><font  color="#FF0000">说文解字</a></font> | <a href="/chengyu/">成语词典</a> | <a  target="_blank"  href="http://www.guoxuedashi.com/jgwhj/"><font  color="#FF0000">甲骨文合集</font></a> | <a href="/yzjwjc/"><font  color="#FF0000">殷周金文集成</font></a> | <a href="/xiangxingzi/"><font color="#0000FF">象形字典</font></a> | <a href="/13jing/"><font  color="#FF0000">十三经索引</font></a> | <a href="/zixing/"><font  color="#FF0000">字体转换器</font></a> | <a href="/zidian/xz/"><font color="#0000FF">篆书识别</font></a> | <a href="/jinfanyi/">近义反义词</a> | <a href="/duilian/">对联大全</a> | <a href="/jiapu/"><font  color="#0000FF">家谱族谱查询</font></a> | <a href="http://www.guoxuemi.com/hafo/" target="_blank" ><font color="#FF0000">哈佛古籍</font></a> 
</div>

 <!-- 头部导航开始 -->
<div class="w1180 head clearfix">
  <div class="head_logo l"><a title="国学大师官网" href="http://www.guoxuedashi.com" target="_blank"></a></div>
  <div class="head_sr l">
  <div id="head1">
  
  <a href="http://www.guoxuedashi.com/zidian/bujian/" target="_blank" ><img src="http://www.guoxuedashi.com/img/top1.gif" width="88" height="60" border="0" title="部件查字,支持20万汉字"></a>


<a href="http://www.guoxuedashi.com/help/yingpan.php" target="_blank"><img src="http://www.guoxuedashi.com/img/top230.gif" width="600" height="62" border="0" ></a>


  </div>
  <div id="head3"><a href="javascript:" onClick="javascript:window.external.AddFavorite(window.location.href,document.title);">添加收藏</a>
  <br><a href="/help/setie.php">搜索引擎</a>
  <br><a href="/help/zanzhu.php">赞助本站</a></div>
  <div id="head2">
 <a href="http://www.guoxuemi.com/" target="_blank"><img src="http://www.guoxuedashi.com/img/guoxuemi.gif" width="95" height="62" border="0" style="margin-left:2px;" title="国学迷"></a>
  

  </div>
</div>
  <div class="clear"></div>
  <div class="head_nav">
  <p><a href="/">首页</a> | <a href="/ShuKu/">国学书库</a> | <a href="/guji/">影印古籍</a> | <a href="/shici/">诗词宝典</a> | <a   href="/SiKuQuanShu/gxjx.php">精选</a> <b>|</b> <a href="/zidian/">汉语字典</a> | <a href="/hydcd/">汉语词典</a> | <a href="http://www.guoxuedashi.com/zidian/bujian/"><font  color="#CC0066">部件查字</font></a> | <a href="http://www.sfds.cn/"><font  color="#CC0066">书法大师</font></a> | <a href="/jgwhj/">甲骨文</a> <b>|</b> <a href="/b/4/"><font  color="#CC0066">解密</font></a> | <a href="/renwu/">历史人物</a> | <a href="/diangu/">历史典故</a> | <a href="/xingshi/">姓氏</a> | <a href="/minzu/">民族</a> <b>|</b> <a href="/mz/"><font  color="#CC0066">世界名著</font></a> | <a href="/download/">软件下载</a>
</p>
<p><a href="/b/"><font  color="#CC0066">历史</font></a> | <a href="http://skqs.guoxuedashi.com/" target="_blank">四库全书</a> |  <a href="http://www.guoxuedashi.com/search/" target="_blank"><font  color="#CC0066">全文检索</font></a> | <a href="http://www.guoxuedashi.com/shumu/">古籍书目</a> | <a   href="/24shi/">正史</a> <b>|</b> <a href="/chengyu/">成语词典</a> | <a href="/kangxi/" title="康熙字典">康熙字典</a> | <a href="/ShuoWenJieZi/">说文解字</a> | <a href="/zixing/yanbian/">字形演变</a> | <a href="/yzjwjc/">金 文</a> <b>|</b>  <a href="/shijian/nian-hao/">年号</a> | <a href="/diming/">历史地名</a> | <a href="/shijian/">历史事件</a> | <a href="/guanzhi/">官职</a> | <a href="/lishi/">知识</a> <b>|</b> <a href="/zhongyi/">中医中药</a> | <a href="http://www.guoxuedashi.com/forum/">留言反馈</a>
</p>
  </div>
</div>
<!-- 头部导航END --> 
<!-- 内容区开始 --> 
<div class="w1180 clearfix">
  <div class="info l">
   
<div class="clearfix" style="background:#f5faff;">
<script src='http://www.guoxuedashi.com/img/headersou.js'></script>

</div>
  <div class="info_tree"><a href="http://www.guoxuedashi.com">首页</a> > <a href="/SiKuQuanShu/fanti/">四库全书</a>
 > <h1>资治通鉴</h1> <!--         下载:【右键另存为】即可 --></div>
  <div class="info_content zj clearfix">
  
<div class="info_txt clearfix" id="show">
<center style="font-size:24px;">96-資治通鑑卷九十五</center>
    資治通鑑卷九十五   宋 司馬光 撰<br />
<br />
  胡三省 音注<br />
<br />
  晉紀十七【起玄黓執徐盡彊圉作噩凡六年】<br />
<br />
  顯宗成皇帝中之上<br />
<br />
  咸和七年春正月辛未大赦 趙主勒大饗羣臣 【考異曰晉春秋云陶侃遣使聘後趙趙王勒饗之按侃與勒必無通使之理今不取載記云勒因饗高句麗宇文屋孤使今但云饗羣臣】謂徐光曰朕可方自古何等主【方比也】對曰陛下神武謀畧過於漢高後世無可比者勒笑曰人豈不自知卿言太過朕若遇漢高祖當北面事之與韓彭比肩【戴溪曰勒豈真知高帝者特自視不如韓彭故耳】若遇光武當並驅中原未知鹿死誰手大丈夫行事宜礌礌落落如日月皎然【礌落猥翻】終不效曹孟德司馬仲達欺人孤兒寡婦狐媚以取天下也【狐妖獸也能蠱媚人石勒以此論曹馬使死者有知孟德仲達其抱愧於地下矣】羣臣皆頓首稱萬歲勒雖不學好使諸生讀書而聽之【好呼到翻】時以其意論古今得失聞者莫不悦服嘗使人讀漢書聞酈食其勸立六國後【事見十卷漢高帝三年】驚曰此法當失何以遂得天下及聞留侯諫乃曰賴有此耳 郭敬之退戍樊城也【事見上卷五年】晉人復取襄陽夏四月敬復攻拔之【敬復扶又翻】留戍而歸 趙右僕射程遐言於趙主勒曰中山王勇悍權畧羣臣莫及觀其志自陛下之外視之蔑如【蔑無也言視之若無也】加以殘賊安忍【孟子曰賊仁者謂之賊賊義者謂之殘左傳衆仲曰安忍無親】久為將帥威振内外其諸子年長皆典兵權【虎子邃宣勒皆使之典兵將即亮翻帥所類翻長知兩翻】陛下在自當無它恐非少主之臣也【少詩照翻】宜早除之以便大計勒曰今天下未安大雅冲幼宜得彊輔中山王骨肉至親有佐命之功方當委以伊霍之任何至如卿所言卿正恐不得擅帝舅之權耳吾亦當參卿顧命勿過憂也遐泣曰臣所慮者公家陛下乃以私計拒之忠言何自而入乎中山王雖為皇太后所養非陛下天屬【載記曰虎勒之從子也祖曰㔨邪父曰寇覔勒父朱幼而子虎故或稱勒弟焉】雖有微功陛下酬其父子恩榮亦足矣而其志願無極【謂虎有窺覦天位之志】豈將來有益者乎若不除之臣見宗廟不血食矣勒不聽遐退告徐光光曰中山王常切齒於吾二人恐非但危國亦將為家禍也它日光承閒言於勒曰【閒古莧翻】今國家無事而陛下神色若有不怡何也【怡悅也】勒曰吳蜀未平吾恐後世不以吾為受命之王也光曰魏承漢運劉備雖興於蜀漢豈得為不亡乎【以喻晉也】孫權在吳猶今之李氏也陛下苞括二都平蕩八州【二都長安洛陽八州冀幽并青兖豫司雍也】帝王之統不在陛下當復在誰【復扶又翻下同】且陛下不憂腹心之疾而更憂四支乎中山王藉陛下威畧所向輒克而天下皆言其英武亞於陛下且其資性不仁見利忘義父子並據權位埶傾王室而耿耿常有不滿之心近於東宮侍宴有輕皇太子之色臣恐陛下萬年之後不可復制也【復扶又翻】勒默然始命太子省可尚書奏事【省悉景翻】且以中常侍嚴震參綜可否惟征伐斷斬大事乃呈之【斷丁亂翻】于是嚴震之權過於主相【相息亮翻】中山王虎之門可設雀羅矣【漢書曰翟公為廷尉賓客填門及廢門外可設雀羅師古注云言其寂静無人行也】虎愈怏怏不悅【為後虎殺徐光程遐張本怏於兩翻】 秋趙郭敬南掠江西【江西謂邾城以東至歷陽也】太尉侃遣其子平西參軍斌【斌音彬】及南中郎將桓宣乘虚攻樊城悉俘其衆敬旋救樊宣與戰于湼水破之【水經注涅水出涅陽縣西北岐棘山東南逕涅陽縣又東南逕安衆縣又東南至新野縣東入于淯涅奴結翻】皆得其所掠侃兄子臻及竟陵太守李陽攻新野拔之敬懼遁去宣遂拔襄陽侃使宣鎮襄陽宣招懷初附簡刑罸畧威儀勸課農桑或載鉏耒於軺軒【鉏立薅所用農器也耒盧對翻手耕曲木也孔穎逹曰耒以曲木為之長六尺六寸底長尺有一寸中央直者三尺有三寸句者二尺有二寸底謂耒下向前曲接耜者頭而著耜耜金銕為之鄭玄曰耜者耒之金也廣五寸田器鎡錤之屬軺音遥使者小車駕馬者也軒曲輈也闌板曰軒】親帥民芸穫【帥讀曰率】在襄陽十餘年趙人再攻之宣以寡弱拒守趙人不能勝時人以為亞於祖逖周訪【史終言宣守襄陽之功】 成大將軍壽寇寧州以其征東將軍費黑為前鋒出廣漢鎮南將軍任回出越嶲以分寧州之兵【費扶沸翻任音壬嶲音髓】 冬十月壽黑至朱提朱提太守董炳城守【朱提音銖時】寧州刺史尹奉遣建寧太守霍彪引兵助之壽欲逆拒彪黑曰城中食少【少詩沼翻】宜縱彪入城共消其穀何為拒之壽從之城久不下壽欲急攻之黑曰南中險阻難服當以日月制之待其智勇俱困然後取之溷牢之物何足汲汲也【溷與圂同胡困翻圂厠也豕所居也牢亦犬豕所居也言城已受圍如犬豕在圂牢中不患其逸出也鄭氏曰牢閑也必有閑者防禽獸觸齧疏曰養馬者謂之閑養牛羊者謂之牢言閑見其閑衛言牢見其牢固所從言之異其實一物也】壽不從攻果不利乃悉以軍事任黑 十一月壬子朔進太尉侃為大將軍劒履上殿入朝不趨贊拜不名【上時掌翻朝直遥翻】侃固辭不受十二月庚戌帝遷于新宮【五年作新宮至是而成乃遷居之】 是歲凉州僚屬勸張駿稱凉王領秦凉二州牧置公卿百官如魏武晉文故事【魏武事見六十七卷漢獻帝建安二十一年晉文事見七十九卷魏元帝咸熙元年】駿曰此非人臣所宜言也敢言此者罪不赦然境内皆稱之為王駿立次子重華為世子【重直龍翻】<br />
<br />
  八年春正月成大將軍李壽拔朱提董炳霍彪皆降【降戶江翻下同】壽威震南中 丙子趙主勒遣使來修好【使疏吏翻好呼到翻】詔焚其幣【晉雖未能復君父之讐而焚幣一事猶足舒忠臣義士之氣】 三月寧州刺史尹奉降于成成盡有南中之地大赦以大將軍壽領寧州 夏五月甲寅遼東武宣公慕容廆卒六月世子皝以平北將軍行平州刺史督攝部内【皝字元真廆第三子廆戶罪翻皝呼廣翻】赦繫囚以長史裴開為軍諮祭酒郎中令高詡為玄菟太守皝以帶方太守王誕為左長史誕以遼東太守陽騖為才而讓之皝從之以誕為右長史【國之興也其臣推賢讓能國之衰也其臣矜已忌賢騖音務】 趙主勒寢疾中山王虎入侍禁中矯詔羣臣親戚皆不得入疾之增損外無知者又矯詔召秦王宏彭城王堪還襄國【勒以宏都督中外諸軍事蓋使之鎮鄴堪蓋在河南】勒疾小瘳見宏驚曰吾使王處藩鎮【處昌呂翻】正備今日有召王者邪將自來邪有召者當按誅之虎懼曰秦王思慕暫還耳今遣之仍留不遣數日復問之【復扶又翻】虎曰受詔即遣今已半道矣廣阿有蝗【廣阿縣前漢屬鉅鹿郡後漢晉省後魏復置廣阿縣屬南趙郡隋改為大陸縣唐武德間改為象城縣天寶初改為昭慶縣屬趙州】虎密使其子冀州刺史邃帥騎三千遊於蝗所【恐勒死有變使邃遊于蝗所若捕蝗者以為外應帥讀曰率騎奇寄翻】秋七月勒疾篤遺命曰大雅兄弟宜善相保司馬氏汝曹之前車也【前車之覆後車之戒戒其兄弟自相殘也】中山王宜深思周霍勿為將來口實【謂當如周公霍光之輔幼孤也勒謂此言可以縶虎之手足邪此數語亦徐光程遐為之耳】戊辰勒卒【年六十】中山王虎刼太子弘使臨軒收右光祿大夫程遐中書令徐光下廷尉【下遐稼翻】召邃使將兵入宿衛【將即亮翻】文武皆奔散弘大懼自陳劣弱讓位于虎虎曰君終太子立禮之常也弘涕泣固讓虎怒曰若不堪重任天下自有大義何足豫論弘乃即位【弘字大雅勒第二子】大赦殺程遐徐光【光遐固知禍之及已然亦不料如是之速】夜以勒喪潛瘞山谷莫知其處己卯備儀衛虛葬于高平陵【勒卒十一日而葬未有如是之速者也虎既潛葬勒其所以為身後之計者亦不過如此卒為女子所告果何益哉瘞於計翻】諡曰明帝廟號高祖趙將石聰及譙郡太守彭彪各遣使來降【聰時鎮譙城守式又翻降戶江翻】聰本晉人冒姓石氏朝廷遣督護喬球將兵救之未至聰等為虎所誅 慕容皝遣長史勃海王濟等來告喪【皝呼廣翻】八月趙王弘以中山王虎為丞相魏王大單于加九<br />
<br />
  錫【單音蟬】以魏郡等十三郡為國總攝百揆虎赦其境内立妻鄭氏為魏王后子邃為魏太子加使持節侍中都督中外諸軍事大將軍錄尚書事次子宣為使持節車騎大將軍冀州刺史封河間王【使疏吏翻】韜為前鋒將軍司隸校尉封樂安王遵齊王鑒封代王苞封樂平王徙平原王斌為章武王【斌音彬】勒文武舊臣皆補散任虎之府寮親屬悉署臺省要職【虎居鄴子邃都督中外諸軍宣據信都府寮親屬分領臺省弘處尊位特寄坐耳散悉亶翻】以鎮軍將軍夔安領左僕射尚書郭殷為右僕射更命太子宮曰崇訓宮【更工衡翻】太后劉氏以下皆徙居之選勒宫人及車馬服玩之美者皆入丞相府宇文乞得歸為其東部大人逸豆歸所逐走死于外<br />
<br />
  慕容皝引兵討之軍于廣安【廣安在棘城之北】逸豆歸懼而請和遂築榆隂安晉二城而還【榆隂城盖在大榆河之隂安晉城在威德城東南】成建寧牂柯二郡來降李壽復擊取之【牂牁音臧哥降戶江翻復】<br />
<br />
  【扶又翻下同】 趙劉太后謂彭城王堪曰先帝甫晏駕丞相遽相陵藉如此【陵駕也轢也藉蹈也藉慈夜翻】帝祚之亡殆不復久王將若之何堪曰先帝舊臣皆被疎斥軍旅不復由人【謂虎諸子皆握兵權也被皮義翻】宮省之内無可為者【謂宿衛及臺省要職皆虎之府寮親屬無與共謀匡正者】臣請奔兖州挾南陽王恢為盟主【恢勒少子也時鎮廪丘】據廩丘宣太后詔於牧守征鎮使各舉兵以誅暴逆庶幾猶有濟也【幾居希翻】劉氏曰事急矣當速為之九月堪微服輕騎襲兖州不克南奔譙城【騎奇寄翻】丞相虎遣其將郭太追之獲堪于城父【父音甫】送襄國炙而殺之徵南陽王恢還襄國劉氏謀泄虎廢而殺之尊弘母程氏為皇太后堪本田氏子數有功【數所角翻】趙主勒養以為子劉氏有膽畧勒每與之參决軍事佐勒建功業有呂后之風而不妬忌更過之【呂后能誅韓信彭越劉氏不能制虎殆不及也】趙河東王生鎮關中石朗鎮洛陽【劉胤之西奔也石生自洛陽鎮長安朗代生鎮洛陽】冬十月生朗皆舉兵以討丞相虎生自稱秦州刺史遣使來降【降戶江翻】氐帥蒲洪自稱雍州刺史西附張駿【咸和四年洪降于虎今以趙亂而叛帥所類翻】虎留太子邃守襄國將步騎七萬攻朗于金墉金墉潰獲朗刖而斬之【將即亮翻騎奇寄翻刖音月】進向長安以梁王挺為前鋒大都督【挺虎之干也】生遣將軍郭權帥鮮卑涉璝衆二萬為前鋒以拒之【帥讀曰率璝公回翻】生將大軍繼發軍于蒲阪權與挺戰于潼關大破之挺及丞相左長史劉隗皆死【此劉隗意即自晉奔趙者】虎還奔澠池【澠彌兖翻】枕尸三百餘里【枕職任翻】鮮卑潛與虎通謀反擊生生不知挺已死懼單騎奔長安權收餘衆退屯渭汭【孔安國曰水北曰汭杜預曰水之隈曲曰汭音如鋭翻】生遂弃長安匿於雞頭山【張守節曰括地志云雞頭山在成州上禄縣東北二十里在長安西南九百六十里酈道元云蓋大隴山異名也後漢書隗囂使王孟塞雞頭道即此也按原州平高縣西百里亦有笄頭山在長安西北八百里】將軍蔣英據長安拒守虎進兵擊英斬之生麾下斬生以降權奔隴右虎分命諸將屯汧隴【汧苦堅翻】遣將軍麻秋討蒲洪【風俗通麻齊大夫麻嬰之後】洪帥戶二萬降于虎【帥讀曰率降戶江翻】虎迎拜洪光烈將軍【前此未有光烈將軍號石虎創置也】護氐校尉【漢有護羌校尉虎以此官授洪使之監護羣氐】洪至長安說虎徙關中豪桀及氐羌以實東方【說輸芮翻】曰諸氐皆洪家部曲洪帥以從誰敢違者虎從之徙秦雍民及氐羌十餘萬戶于關東【雍於用翻】以洪為龍驤將軍流民都督使居枋頭【為苻洪子健自枋頭入關中張本驤思將翻枋音方】以羌帥姚弋仲為奮武將軍西羌大都督使帥其衆數萬徙居清河之灄頭【羌帥所類翻灄目涉翻水經注清河過廣川縣東水側有羌壘姚氏之故居也為姚弋仲父子自灄頭起兵張本】虎還襄國大赦趙主弘命虎建魏臺一如魏武王輔漢故事 慕容皝初嗣位用法嚴峻國人多不自安主簿皇甫真切諫不聽皝庶兄建武將軍翰母弟征虜將軍仁有勇畧屢立戰功得士心季弟昭有才藝皆有寵於廆皝忌之翰歎曰吾受事於先公不敢不盡力幸賴先公之靈所向有功此乃天贊吾國非人力也而人謂吾之所辦以為雄才難制吾豈可坐而待禍邪乃與其子出奔段氏段遼素聞其才冀收其用甚愛重之仁自平郭來奔喪【漢志平郭縣屬遼東郡晉省晉東夷校尉治襄平崔毖之敗慕容廆以仁鎮遼東治平郭】謂昭曰吾等素驕多無禮於嗣君嗣君剛嚴無罪猶可畏况有罪乎昭曰吾輩皆體正嫡於國有分【昭自謂與仁皆正室之子分可以得國也分扶問翻】兄素得士心我在内未為所疑伺其閒隙除之不難兄趣舉兵以來【伺相吏翻閒古莧翻趣讀曰促】我為内應事成之日與我遼東男子舉事不克則死不能效建威偷生異域也仁曰善遂還平郭閏月仁舉兵而西或以仁昭之謀告皝皝未之信遣使按驗仁兵已至黄水【黄水即潢水在棘城東北距唐營州四百里據載記黄水當在漢遼東郡險瀆縣】知事露殺使者還據平郭皝賜昭死遣軍祭酒封奕慰撫遼東以高詡為廣武將軍將兵五千與庶弟建武將軍幼稚廣威將軍軍寧遠將軍汗司馬遼東佟壽共討仁與仁戰於汶城北【佟徒冬翻姓也汶漢古縣屬遼東郡前書作文】皝兵大敗幼稚軍皆為仁所獲壽嘗為仁司馬遂降於仁【降戶江翻下同】前大農孫機等舉遼東城以應仁【孫機盖王官之避地遼東者遼東城即襄平城】封奕不得入與汗俱還東夷校尉封抽護軍平原乙逸【乙姓也逸名也姓譜商湯宇天乙支孫以為氏】遼東相太原韓矯皆弃城走於是仁盡有遼東之地段遼及鮮卑諸部皆與仁遥相應援皝追思皇甫真之言以真為平州别駕【皝領平州刺史以真為别駕】 十二月郭權據上邽遣使來降京兆新平扶風馮翊北地皆應之 初張駿欲假道於成以通表建康成主雄不許駿乃遣治中從事張淳稱藩於成以假道雄偽許之將使盜覆諸東峽【三峽在成都之東故云東峽】蜀人橋贊密以告淳淳謂雄曰寡君使小臣行無迹之地【蜀不許凉人假道則蜀地前此無凉人之迹】萬里通誠於建康者以陛下嘉尚忠義能成人之美故也若欲殺臣者當斬之都市宣示衆目曰凉州不忘舊德通使琅邪【江左自琅邪中興故以稱之使疏吏翻】主聖臣明發覺殺之如此則義聲遠播天下畏威今使盜殺之江中威刑不顯何足以示天下乎雄大驚曰安有此邪司隸校尉景騫【蜀置司隸校尉於成都】言於雄曰張淳壯士請留之雄曰壯士安肯留且試以卿意觀之騫謂淳曰卿體豐大天熱可且遣下吏小住須凉【須待也】淳曰寡君以皇輿播越梓宮未返【謂懷愍蒙塵卒之見害梓宮未返也】生民塗炭莫之振救【振舉也】故遣淳通誠上都【上都謂建康也】所論事重非下吏所能傳使下吏可了則淳亦不來矣雖火山湯海猶將赴之豈寒暑之足憚哉雄謂淳曰貴主英名蓋世土險兵彊何不亦稱帝自娯一方淳曰寡君祖考以來世篤忠貞以讐恥未雪枕戈待旦【枕職任翻】何自娯之有雄甚慙曰我之祖考本亦晉臣遭天下大亂與六郡之民避難此州【事見八十二卷惠帝元康八年難乃旦翻】為衆所推遂有今日琅邪若能中興大晉於中國者亦當帥衆輔之【帥讀曰率】厚為淳禮而遣之淳卒致命於建康【卒子恤翻致命致其君命也】長安之失守也【事見八十九卷愍帝建興四年】敦煌計吏耿訪自漢中入江東【耿訪敦煌所遣上計吏留長安未還而長安陷且河隴路絶因南入漢中自漢東下至建康敦徒門翻】屢上書請遣大使慰撫凉州【使疏吏翻下同】朝廷以訪守持書御史拜張駿鎮西大將軍選隴西賈陵等十二人配之【配侑也】訪至梁州道不通以詔書付賈陵詐為賈客以達之【賈音古】是歲陵始至凉州駿遣部曲督王豐等報謝<br />
<br />
  九年春正月趙改元延熙 詔以郭權為鎮西將軍雍州刺史【雍於用翻】 仇池王楊難敵卒子毅立自稱龍驤將軍左賢王下辨公以叔父堅頭之子盤為冠軍將軍右賢王河池公【驤思將翻辨步莧翻冠古玩翻】遣使來稱藩 二月丁卯詔遣耿訪王豐齎印綬授張駿大將軍都督陜西雍秦凉州諸軍事【陜式冉翻】自是每歲使者不絶【仇池稱藩梁凉之路通也】慕容仁以司馬翟楷領東夷校尉前平州别駕龎鑒領遼東相【龎皮江翻】 段遼遣兵襲徒河不克復遣其弟蘭與慕容翰共攻柳城【柳城縣漢屬遼西郡晉省唐為營州治所復扶又翻】柳城都尉石琮城大慕輿埿并力拒守【城大猶城主也一城之長故曰城大埿與泥同】蘭等不克而退遼怒切責蘭等必令拔之休息二旬復益兵來攻【復扶又翻】士皆重袍蒙楯【重直龍翻楯食尹翻】作飛梯【飛梯即雲梯】四面俱進晝夜不息琮埿拒守彌固殺傷千餘人卒不能拔【卒子恤翻】慕容皝遣慕容汗及司馬封奕等共救之皝戒汗曰賊氣銳勿與争鋒汗性驍果以千餘騎為前鋒【驍堅堯翻騎奇寄翻】直進封奕止之汗不從與蘭遇于牛尾谷【牛尾谷在柳城北】汗兵大敗死者大半奕整陳力戰【陳讀曰陣】故得不没蘭欲乘勝窮追慕容翰恐遂滅其國止之曰夫為將當務慎重審已量敵【量音良】非萬全不可動今雖挫其偏師未能屈其大埶皝多權詐好為潛伏【好呼到翻】若悉國中之衆自將以拒我【將即亮翻】我縣軍深入【縣讀曰懸】衆寡不敵此危道也且受命之日止求此捷若違命貪進萬一取敗功名俱喪【喪息浪翻】何以返面蘭曰此已成禽無有餘理【謂以事理策之皝必成禽無復遺餘也】卿正慮遂滅卿國耳今千年在東若進而得志吾將迎之以為國嗣終不負卿使宗廟不祀也千年者慕容仁小字也翰曰吾投身相依無復還理【復扶又翻】國之存亡於我何有但欲為大國之計且相為惜功名耳【為于偽翻】乃命所部欲獨還蘭不得已而從之【史言翰雖身在外乃心宗國】 三月成主雄分寧州置交州【成分寧州之興古永昌牂柯越嶲夜郎等郡為交州】以霍彪為寧州刺史㸑深為交州刺史趙丞相虎遣其將郭敖及章武王斌帥步騎四萬西擊郭權軍于華隂夏四月上邽豪族殺權以降【斌音彬帥讀曰率騎奇寄翻華戶化翻降戶江翻】虎徙秦州三萬餘戶于青并二州長安人陳良夫奔黑羌【羌之别種有青羌黑羌】與北羌王薄句大等侵擾北地馮翊【句音鉤】章武王斌樂安王韜合擊破之句大奔馬蘭山郭敖乘勝逐北為羌所敗【敗補邁翻】死者什七八斌等收軍還三城【魏收地形志後魏太和初分雁門之廣武朔方之沃野置編城郡治廣武縣縣有三城偏城】虎遣使誅郭敖秦王宏有怨言【以其父疾而虎矯詔召之至于失職也】虎幽之 慕容仁自稱平州刺史遼東公 長沙桓公陶侃晚年深以滿盈自懼不預朝權【朝直遥翻】屢欲告老歸國【欲歸長沙國也】佐吏等苦留之六月侃疾篤上表遜位遣左長史殷羨奉送所假節麾幢曲蓋 【按節麾者臨敵之際三軍視以為進退者也幢幡幢方言曰幢翳也楚曰翿關東西皆曰幢文選注幢以羽葆為之釋名曰幢童也其狀童童然幢傳江翻曲蓋者蓋為曲柄世說謝靈運好戴曲柄笠孔隱士曰何不能遺曲蓋之貌晉制諸公任方面者皆給節麾緹幢曲蓋】侍中貂蟬太尉章【章印章也】荆江雍梁交廣益寧八州刺史印傳棨戟【自此以上皆朝廷所授故奉送之雍於用翻傳株戀翻棨音啓】軍資器仗牛馬舟船皆有定簿封印倉庫侃自加管鑰以後事付右司馬王愆期加督護統領文武【史言陶侃綜理精密雖病不亂】甲寅輿車出臨津就船將歸長沙顧謂愆期曰老子婆娑正坐諸君【娑桑何翻婆娑肢體緩縱不取之貌侃言不得早退至于困乏如此正坐參佐苦留之也】乙卯薨於樊谿【樊谿在武昌西三里北注大江觀陶侃在西藩顛末豈有非望之圖哉晉史所記决指之事折翼之夢蓋庾亮之黨傳致之耳】侃在軍四十一年【惠帝太安二年侃擊張昌至是年凡四十一年】明毅善斷【斷丁亂翻】識察纎密人不能欺自南陵迄于白帝【南陵在宣城郡界梁置南陵郡陳置北江州於其地蓋臨江渚江州東界盡於南陵今宣州南陵縣非古之南陵戌也自南陵迄于白帝總言侃所統大界宋白曰南陵本漢春穀縣地後併于湖縣尋又屬繁昌梁武帝始置南陵縣屬南陵郡臨江有城基見存去今縣一百三十里】數千里中路不拾遺及薨尚書梅陶與親人曹識書曰【親人其所親者也】陶公機神明鑒似魏武忠順勤勞似孔明陸抗諸人不能及也謝安每言陶公雖用法而恒得法外意【史言陶侃為名流所推重如此恒戶登翻】安鯤之從子也【謝鯤見九十二卷元帝永昌元年從才用翻】 成主雄生瘍於頭【瘍余章翻頭瘡曰瘍】身素多金創【矢刃所傷為金創創初良翻】及病舊㾗皆膿潰諸子皆惡而遠之【惡烏路翻遠于願翻】獨太子班晝夜侍側不脱衣冠親為吮膿【為于偽翻吮徂兖翻】雄召大將軍建寧王壽受遺詔輔政丁卯雄卒【年六十一載記雄卒在去年】太子班即位【班字世文雄兄蕩之子也】以建寧王壽錄尚書事政事皆委於壽及司徒何點尚書王瓌【瓌古回翻】班居中行喪禮一無所預【李班豈可不謂之仁孝哉然不能包周身之防死於李越之手末俗澆漓固不可拘拘於古禮以啓奸非至于殞身亂國也】 辛未加平西將軍庾亮征西將軍假節都督江荆豫益梁雍六州諸軍事領江豫荆三州刺史鎮武昌【陶侃既没庾亮始專制上流雍於用翻】亮辟殷浩為記室參軍浩羨之子也與豫章太守褚裒【裒蒲侯翻】丹陽丞杜乂皆以識度清遠善談老易【老易老子及易也】擅名江東而浩尤為風流所宗裒䂮之孫【禇䂮見七十七卷魏元帝景元元年䂮離灼翻】乂錫之子也【杜錫見八十三卷惠帝元康九年】桓彛嘗謂裒曰季野有皮裏春秋【禇裒字季野】言其外無臧否【否音鄙】而内有褒貶也謝安曰裒雖不言而四時之氣亦備矣 秋八月王濟還遼東詔遣侍御史王齊祭遼東公廆又遣謁者徐孟策拜慕容皝鎮軍大將軍平州刺史大單于遼東公持節承制封拜一如廆故事船下馬石津【自建康出大江至于海轉料角至登州大洋東北行過大謝島龜歆島淤島烏胡島三百里北度烏湖海至馬石山東之都里鎮馬石津即此地也】皆為慕容仁所留 九月戊寅衛將軍江陵穆公陸曄卒 成主雄之子車騎將軍越屯江陽奔喪至成都以太子班非雄所生意不服與其弟安東將軍期謀作亂班弟玝勸班遣越還江陽以期為梁州刺史鎮葭萌【葭萌即晉夀之地玝阮古翻】班以未葬不忍遣推心待之無所疑閒【閒古莧翻】遣玝出屯于涪【涪音浮】冬十月癸亥朔越因班夜哭弑之於殯宮【卒如李驤之言菆塗曰殯將遷葬以賓遇之也】并殺班兄領軍將軍都矯太后任氏令罪狀班而廢之【任音壬】初期母冉氏賤任氏母養之期多才藝有令名及班死衆欲立越越奉期而立之甲子期即皇帝位【期字世運雄第四子也】諡班曰戾太子以越為相國封建寧王加大將軍壽大都督徙封漢王皆錄尚書事以兄霸為中領軍鎮南大將軍弟保為鎮西大將軍汶山太守【汶讀曰岷】從兄始為征東大將軍代越鎮江陽【據載記始特之長子於期為伯父於壽為從兄從才用翻】丙寅葬雄於安都陵諡曰武皇帝廟號太宗始欲與壽共攻期壽不敢發始怒反譖壽於期請殺之期欲藉壽以討李玝故不許遣壽將兵向涪壽先遣使告玝以去就利害開其去路玝遂來奔詔以玝為巴郡太守期以壽為梁州刺史屯涪【為李壽自涪舉兵廢李期張本】 趙主弘自齎璽綬詣魏宮【石虎為魏王其所居稱魏宮璽斯氏翻綬音受】請禪位於丞相虎虎曰帝王大業天下自當有議何為自論此邪弘流涕還宮謂太后程氏曰先帝種真無復遺矣【種章勇翻復扶又翻】于是尚書奏魏臺請依唐虞禪讓故事【此趙朝尚書奏也】虎曰弘愚暗居喪無禮便當廢之何禪讓也十一月虎遣郭殷入宮廢弘為海陽王弘安步就車容色自若謂羣臣曰庸昧不堪纂承大統夫復何言【復扶又翻】羣臣莫不流涕宮人慟哭羣臣詣魏臺勸進虎曰皇帝者盛德之號非所敢當且可稱居攝趙天王 【考異曰三十國晉春秋虎即位改元永熙陳鴻大統歷云石虎即位改建平五年為延興明年改建武按三十國晉春秋不記弘改元延熙虎之立實延熙元年也故誤云永熙弘既號延熙虎安肯稱永熙陳鴻云虎改建平五年為延興即是弘踰年不改元也恐鴻說誤】幽弘及太后程氏秦王宏南陽王恢于崇訓宮尋皆殺之【虎更太子宮曰崇訓宮弘時年二十一】西羌大都督姚弋仲稱疾不賀虎累召之乃至正色謂虎曰弋仲常謂大王命世英雄奈何把臂受託而返奪之邪虎曰吾豈樂此哉【樂音洛】顧海陽年少【少詩照翻】恐不能了家事故代之耳心雖不平然察其誠實亦不之罪虎以夔安為侍中太尉守尚書令郭殷為司空韓晞為尚書左僕射魏郡申鐘為侍中郎闓為光祿大夫【闓苦亥翻又音開】王波為中書令文武封拜各有差虎行如信都復還襄國【載記曰虎以讖文天子當從東北來于是備法駕行自信都而還以應之天子當從東北來盖謂慕容氏將從遼碣入中國也秦始皇東游以厭天子氣初不能遏止漢高之興】 慕容皝討遼東甲申至襄平遼東人王岌密信請降師進入城翟楷龎鑒單騎走居就新昌等縣皆降【居就新昌皆屬遼東郡降戶江翻下同】皝欲悉阬遼東民高詡諫曰遼東之叛實非本圖直畏仁凶威不得不從今元惡猶存【元惡謂仁也】始克此城遽加夷滅則未下之城無歸善之路矣皝乃止分徙遼東大姓於棘城以杜羣為遼東相安輯遺民 十二月趙徐州從事蘭陵朱縱斬刺史郭祥以彭城來降趙將王朗攻之縱奔淮南 慕容仁遣兵襲新昌督護新興王㝢擊走之遂徙新昌入襄平【遼東治襄平徙新昌吏民入襄平所以杜仁闚?掩襲之心㝢王矩翻】<br />
<br />
  咸康元年春正月庚午朔帝加元服【沈約禮志曰古者無天子冠禮故筮日筮賓冠於阼以著代醮於客位三加彌尊皆士禮耳漢順帝冠兼用曹褒新禮褒新禮今不存禮儀志又云乘輿初加緇布進賢次爵弁武弁次通天皆於高廟江左諸帝將冠金吾宿設百僚陪位又豫於殿上鋪大牀御府令奉冕幘簪導衮服以授侍中常侍太尉加幘太保加冕將加冕太尉跪讀祝文曰令月吉日始加元服皇帝穆穆思弘衮職欽若昊天六合是式率遵祖考永永無極眉壽無期介兹景福加冕訖侍中繋玄紞侍中脫絳紗服加襄服冠事畢太保率羣臣奉觴上壽王以下三稱萬歲乃退鄭樵曰用魏儀一加既加元服拜于太廟】大赦改元 成趙皆大赦成改元玉恒【恒戶登翻】趙改元建武 成主期立皇后閻氏以衛將軍尹奉為右丞相驃騎將軍尚書令王瓌為司徒【驃匹妙翻瓌古回翻】 趙王虎命太子邃省可尚書奏事【省悉景翻】惟祀郊廟選牧守征伐刑殺乃親之虎好治宮室鸛雀臺崩【鸛雀臺在鄴即魏武所起銅雀臺好呼到翻】殺典匠少府任汪【典匠少府即漢將作大匠之職也少詩照翻任音壬】復使修之倍於其舊邃保母劉芝封宜城君關預朝權受納賄賂求仕進者多出其門【朝直遥翻】 慕容皝置左右司馬以司馬韓矯軍祭酒封奕為之 司徒導以羸疾不堪朝會【羸倫為翻朝直遥翻】三月乙酉帝幸其府與羣臣宴于内室拜導并拜其妻曹氏侍中孔坦密表切諫以為帝初加元服動宜顧禮帝從之坦又以帝委政於導從容言曰陛下春秋已長聖敬日躋【從千容翻長知兩翻日躋猶日進也】宜博納朝臣諮諏善道【諏遵須翻】導聞而惡之【惡烏路翻】出坦為廷尉坦不得意以疾去職丹陽尹桓景為人謟巧導親愛之會熒惑守南斗經旬【晉天文志南斗六星天廟也丞相太宰之位】導謂領軍將軍陶回曰斗揚州之分【天文志斗牛女揚州九江入斗一度丹楊入斗十六度分扶問翻】吾當遜位以厭天譴【厭一葉翻】回曰公以明德作輔而與桓景造膝【造七到翻】使熒惑何以退舍導深愧之導辟太原王濛為掾【濛莫紅翻掾于絹翻】王述為中兵屬【晉公府諸曹有參軍有掾有屬】述昶之曾孫也【王昶仕魏鎮荆州以功名自見昶丑兩翻】濛不修小亷而以清約見稱與沛國劉惔齊名友善【惔徒甘翻】惔常稱濛性至通而自然有節濛曰劉君知我勝我自知當時稱風流者以惔濛為首述性沈静每坐客辯論蠭起而述處之恬如也【沈持林翻坐徂臥翻處昌呂翻】年三十尚未知名人謂之癡導以門地辟之【昶之子湛湛之子承世有高名述承子也】既見唯問江東米價【述蓋自東吳至建康】述張目不荅導曰王掾不癡人何言癡也嘗見導每發言一坐莫不贊美【坐徂臥翻】述正色曰人非堯舜何得每事盡善導改容謝之 趙王虎南遊臨江而還【還從宣翻又如字】有遊騎十餘至歷陽歷陽太守袁耽表上之【上時掌翻】不言騎多少朝廷震懼司徒導請出討之夏四月加導大司馬假黄鉞都督征討諸軍事癸丑帝觀兵廣莫門【廣莫門建康城北門也】分命諸將救歷陽及戍慈湖牛渚蕪湖司空郗鑒使廣陵相陳光將兵入衛京師俄聞趙騎至少又已去【少詩沼翻】戊午解嚴王導解大司馬袁耽坐輕妄免官趙征虜將軍石遇攻桓宣於襄陽不克 大旱會稽<br />
<br />
  餘姚米斗五百【會工外翻】 秋七月慕容皝立子雋為世子九月趙王虎遷都于鄴【趙王勒定都襄國虎遷于鄴】大赦 初趙<br />
<br />
  主勒以天竺僧佛圖澄豫言成敗數有驗敬事之及虎即位奉之尤謹衣以綾錦乘以彫輦【彫輦雕鏤以為飾數所角翻衣於既翻】朝會之日太子諸公扶翼上殿【朝直遥翻上時掌翻】主者唱大和尚【主者謂掌朝儀者也】衆坐皆起使司空李農旦夕問起居太子諸公五日一朝【諸公虎諸子也虎稱天王降諸子封王者爵為公】國人化之率多事佛澄之所在無敢向其方面涕唾者争造寺廟削髪出家虎以其真偽雜糅【糅汝救翻】或避賦役為姦宄【宄音軌】乃下詔問中書曰佛國家所奉里閭小人無爵秩者應事佛不【不讀曰否】著作郎王度等【晉職官志曰著作郎周左史之任也漢東京圖籍在東觀有其名尚未有官魏明帝太和中詔置著作郎於此始有其官隸中書省及晉受命制曰著作舊屬中書而祕書既典文籍今改中書省著作為祕書著作于是改隸祕書省自後别置省而猶隸祕書】議曰王者祭祀典禮具存佛外國之神非天子諸華所應祠奉漢氏初傳其道【事見四十五卷漢明帝永平八年】唯聽西域人立寺都邑以奉之【漢人初謂官府為寺後漢自西域白馬駞經來初止于鴻臚寺遂取寺名創置白馬寺】漢人皆不得出家魏世亦然今宜禁公卿以下毋得詣寺燒香禮拜其趙人為沙門者皆返初服【謂使還服華人之服】虎詔曰朕生自邊鄙忝君諸夏【夏戶雅翻】至于饗祀應從本俗其夷趙百姓樂事佛者特聽之【樂音洛】 趙章武王斌帥精騎二萬并秦雍二州兵以討薄句大平之【去年斌等為薄句大所敗斌音彬帥讀曰率騎奇寄翻雍於用翻句音鉤】 成太子班之舅羅演與漢王相天水上官澹【李壽封漢王相息亮翻澹徒覽翻】謀殺成主期立班子事覺期殺演澹及班母羅氏期自以得志輕諸舊臣信任尚書令景騫尚書姚華田褒中常侍許涪等【涪音浮】刑賞大政皆决於數人希復關公卿【復扶又翻】褒無他才嘗勸成主雄立期為太子故有寵由是紀綱隳紊【紊音問】雄業始衰 冬十月乙未朔日有食之 慕容仁遣王齊等南還【去年齊等為慕容仁所留】齊等自海道趣棘城【趣七喻翻】齊遇風不至十二月徐孟等至棘城慕容皝始受朝命【朝直遥翻】段氏宇文氏各遣使詣慕容仁館于平郭城外皝帳下督張英將百餘騎閒道潜行掩擊之【閒古莧翻】斬宇文氏使十餘人生擒段氏使以歸 是歲明帝母建安君荀氏卒荀氏在禁中尊重同於太后詔贈豫章郡君【荀氏元帝宮人也生明帝自以位卑每懷怨望為帝所譴漸見疎薄及明帝即位封建安君別立第宅太寧元年迎還臺内供奉隆厚及帝立尊重同於太后】 代王翳槐以賀蘭藹頭不恭【藹頭翳槐舅有擁護之功事見九十三卷咸和二年至四年逐紇那立翳槐又賀蘭部也挾親恃功所以不恭】將召而戮之諸部皆叛代王紇那自宇文部入諸部復奉之【紇那出奔見上卷咸和四年復扶又翻】翳槐奔鄴趙人厚遇之 初張軌及二子寔茂雖保據河右而軍旅之事無歲無之及張駿嗣位境内漸平駿勤修庶政總御文武咸得其用民富兵彊遠近稱之以為賢君駿遣將楊宣伐龜兹鄯善於是西域諸國焉耆于窴之屬皆詣姑臧朝貢【龜兹音丘慈鄯上扇翻窴徒賢翻又徒見翻】駿於姑臧南作五殿【駿起謙光殿四面各起一殿東曰宜陽青殿以春三月居之南曰朱陽赤殿夏三月居之西曰政刑白殿秋三月居之北曰玄武黑殿冬三月居之章服器物皆依方色】官屬皆稱臣駿有兼秦雍之志【雍於用翻】遣參軍麴護上疏以為勒雄既死虎期繼逆兆庶離主【離力智翻】漸冉經世先老消落後生不識慕戀之心日遠日忘乞敕司空鑒征西亮等汎舟江沔首尾齊舉【郗鑒時鎮京口庾亮時鎮武昌沔彌兖翻】<br />
<br />
  二年春正月辛巳彗星見于奎婁【西方奎十六星天之武庫也主以兵禁暴又主溝瀆婁三星為天獄主苑牧犧牲供給郊祀奎婁胃魯徐州分彗祥歲翻又旋芮翻又徐醉翻見賢遍翻】慕容皝將討慕容仁司馬高詡曰仁叛弃君親民神<br />
<br />
  共怒前此海未嘗凍自仁反以來連年凍者三矣且仁專備陸道天其或者欲使吾乘海氷以襲之也皝從之羣僚皆言涉氷危事不若從陸道皝曰吾計已决敢沮者斬【沮在呂翻】壬午皝帥其弟軍師將軍評等自昌黎東踐氷而進【皝呼廣翻帥讀曰率下同】凡三百餘里至歷林口【歷林口海浦之口】捨輜重輕兵趣平郭【重直用翻趣七喻翻下同】去城七里候騎以告仁【騎奇寄翻】仁狼狽出戰張英之俘二使也【事見上年使疏吏翻】仁恨不窮追及皝至仁以為皝復遣偏師輕出寇抄【復扶又翻抄楚交翻】不知皝自來謂左右曰今兹當不使其匹馬得返矣乙未仁悉衆陳于城之西北慕容軍帥所部降於皝【咸和八年軍為仁所執陳讀曰陣降戶江翻】仁衆沮動【沮在呂翻】皝從而縱擊大破之仁走其帳下皆叛遂擒之皝先為斬其帳下之叛者【為于偽翻】然後賜仁死丁衡游毅孫機等皆仁所信用也皝執而斬之王氷自殺慕容幼慕容稚佟壽郭充翟楷龎鑒皆東走幼中道而還皝兵追及楷鑒斬之壽充奔高麗【麗力知翻】自餘吏民為仁所詿誤者【詿古賣翻】皝皆赦之封高詡為汝陽侯 二月尚書僕射王彬卒 辛亥帝臨軒遣使備六禮逆故當陽侯杜乂女陵陽為皇后【婚有六禮一曰納采者將為婚必先媒通其言乃後使人納其采擇之禮用雁為贄取其隂陽往來之義也二曰問名者問名以卜其吉凶也三曰納吉者卜於廟得吉兆復使往告婚姻之事也四曰納徵用玄纁不用雁五曰請期由夫家卜得吉日使人往告之六曰親迎壻往女家御輪三周御者代之壻自乘其車而先以導婦歸】大赦羣臣畢賀 夏六月段遼遣中軍將軍李詠襲慕容皝詠趣武興【武興城在今支東】都尉張萌擊擒之遼别遣段蘭將步騎數萬屯柳城西回水【回水載記作曲水水經注陽樂水出上谷且居縣東北流逕女祁縣世謂之横水又謂之陽曲水又濡河從塞外來西北逕禦夷鎮城又東北逕孤山南又東南水流回曲謂之曲河鎮又據載記曲水當在好城西北將即亮翻騎奇寄翻】宇文逸豆歸攻安晉以為蘭聲援皝帥步騎五萬向柳城蘭不戰而遁皝引兵北趣安晉【咸安八年皝築安晉城趣七喻翻】逸豆歸棄輜重走【重直用翻】皝遣司馬封奕帥輕騎追擊大破之皝謂諸將曰二虜恥無功必將復至【復扶又翻】宜於柳城左右設伏以待之乃遣封奕帥騎數千伏於馬兜山三月段遼果將數千騎來寇抄【抄楚交翻】奕縱擊大破之斬其將榮伯保 前廷尉孔坦卒【坦先以疾解廷尉故曰前】坦疾篤庾氷省之流涕【省悉景翻】坦慨然曰大丈夫將終不問以濟國安民之術乃為兒女子相泣邪氷深謝之 九月慕容皝遣長史劉斌兼郎中令遼東陽景送徐孟等還建康【晉制王國乃有郎中令皝未為王而僭置是官斌音彬】 冬十月廣州刺史鄧岳遣督護王隨等擊夜郎興古皆克之【懷帝永嘉五年王遜分牂柯朱提建寧立夜郎郡太康地志曰蜀劉氏分建寧牂柯立興古郡】加岳督寧州 成王期以從子尚書僕射武陵公載有雋才忌之【從才用翻】誣以謀反殺之 十一月詔建威將軍司馬勲將兵安集漢中成漢王壽擊敗之【敗補邁翻】壽遂置漢中守宰戍南鄭而還 索頭郁鞠帥衆三萬降于趙【索頭鮮卑種言索頭以别於黑匿郁鞠以其辮髪故謂之索頭索昔各翻帥讀曰率降戶江翻】趙拜郁鞠等十三人為親趙王散其部衆於冀青等六州趙王虎作太武殿於襄國作東西宮於鄴【東宮以居太子邃西】<br />
<br />
  【宮虎自居之】十二月皆成太武殿基高二丈八尺縱六十五步廣七十五步甃以文石【高居號翻縱子容翻廣古曠翻甃側救翻】下穿伏室【伏室即窟室也】置衛士五百人以漆灌瓦金璫銀楹【司馬相如羽獵賦華榱璧璫注云璧璫以玉為掾頭當即所謂璇題者也三輔黄圖注云以璧飾瓦之當也又琅璫鐸也杜甫詩風動金琅璫此金璫盖以金飾瓦之當也楹柱也璫音當】珠簾玉壁窮極工巧殿上施白玉牀流蘇帳為金蓮華以冠帳頂【華讀曰花冠古玩翻】又作九殿于顯陽殿後選士民之女以實之服珠玉被綺縠者萬餘人【被皮義翻】教宮人占星氣馬步射置女太史雜伎工巧皆與外同【與外同者教宮人使執作如男子也伎渠綺翻】以女騎千人為鹵簿【車駕法從次第曰鹵簿騎奇寄翻】皆著紫綸巾【著陟畧翻陸德明曰綸繩也蓋合絲為綸其狀如䋲染紫以織巾也今鎮江金壇人能織線番羅亦合絲為線以織之】熟錦袴金銀鏤帶【鏤郎豆翻】五文織成鞾【五文五色成文也廣雅曰天竺國出細織成魏畧曰大秦國用水羊毛木皮野繭絲作織成皆好此以五采織成為鞾也鞾許戈翻】執羽儀【羽儀氅眊之屬】鳴鼓吹【鼔吹軍樂也吹尺瑞翻】遊宴以自隨於是趙大旱金一斤直粟二斗百姓嗷然而虎用兵不息百役並興使牙門張彌徙洛陽鐘虡九龍翁仲銅駞飛亷於鄴【鐘虡九龍翁仲銅駞飛亷皆魏明帝所鑄虡音巨】載以四輪纒車轍廣四尺深二尺【攷之字書無字當作輞音罔車輮也轍車輪所碾跡也廣古曠翻深式禁翻】一鐘没於河募浮没三百人【浮没在水中能浮能没者】入河繫以竹絙【絙居登翻又居鄧翻大索也】用牛百頭鹿櫨引之乃出【鹿櫨形如汲水木立兩柱横木貫柱令圓滑可轉繫絙於横木絞而引之櫨音盧】造萬斛之舟以濟之既至鄴虎大悅為之赦二歲刑【為于偽翻】賚百官穀帛賜民爵一級又用尚方令解飛之言【解姓也飛名也解戶買翻】於鄴南投石於河以作飛橋功費數千萬億橋竟不成役夫飢甚乃止使令長帥民入山澤采橡及魚以佐食復為權豪所奪【令力定翻長知兩翻帥讀曰率復扶又翻】民無所得 初日南夷帥范稚有奴曰范文【帥所類翻】常隨商賈往來中國【賈音古】後至林邑教林邑王范逸作城郭宮室器械逸愛信之【林邑國本漢象林縣馬援鑄銅柱之處也漢末縣功曹子區連殺今自立為王子孫相承其後無嗣外孫范熊繼立逸熊子也】使為將【將即亮翻】文遂譖逸諸子或徙或逃是歲逸卒文詐迎逸子於它國置毒於椰酒而殺之【椰木出交趾高數十丈葉背面相似瓊臺志曰椰子無時而生樹似檳榔葉如鳳尾實大如爪剖之其中有酒其皮可為飲器交州記曰椰子生南海狀如海梭子大如椀外有麄皮如大腹子豆蔻之類中有漿似酒飲之得醉爾雅翼椰木似檳榔無枝條高十餘尋葉在其末如束蒲實大如瓠繫在樹頭實外有皮如胡桃核裏有膚白如雪厚半寸如猪膏味美如胡桃膚裏有汁升餘清如水美如蜜飲之可以愈渴核作飲器椰以嗟翻】文自立為王於是出兵攻大岐界小岐界式僕徐狼屈都乾魯扶單等國皆滅之有衆四五萬遣使奉表入貢【使疏吏翻】 趙左校令成公段作庭燎於杠末【姓譜衛成公之後為成公氏予按春秋之時魯晉皆有成公豈獨衛成公之後得專以為氏哉杠古雙翻】高十餘丈上盤置燎【古之人君昧旦視朝故設庭燎鄭氏云在地曰燎執之曰燭又云樹之門外曰大燭於内曰庭燎皆是照衆為明今成公段懸盤於杠以置燎創意為之非有古法也燎力照翻徐又力燒翻高居傲翻】下盤置人趙王虎試而悅之<br />
<br />
  三年春正月庚辰趙太保夔安等文武五百餘人入上尊號【上時掌翻】庭燎油灌下盤死者二十餘人 【考異曰載記云七人今從三十國春秋】趙王虎惡之【惡烏路翻】腰斬成公段辛巳虎依殷周之制稱大趙天王即位於南郊大赦立其后鄭氏為天王皇后【古者稱王后稱皇帝后稱皇后未有天王皇后之稱也】太子邃為天王皇太子【古之王者其嫡長曰世子秦漢稱皇帝立皇太子未有天王皇太子之稱也】諸子為王者皆降為郡公宗室為王者降為縣侯百官封署各有差 國子祭酒袁瓌【瓌公回翻】太常馮懷以江左寖安請興學校【校乃教翻】帝從之辛卯立太學徵集生徒而士大夫習尚老莊儒術終不振瓌渙之曾孫也【漢末劉備舉袁渙茂才後仕魏行御史大夫事】 三月慕容皝於乙連城東築好城以逼乙連【乙連城段國之東境也在曲水之西】留折衝將軍蘭勃守之夏四月段遼以車數千兩輸乙連粟【兩力讓翻乘也】蘭勃擊而取之六月遼又遣其從弟揚威將軍屈雲將精騎夜襲皝子遵於興國城【興國城蓋慕容氏所築從才用翻將即亮翻騎奇寄翻】遵擊破之初北平陽裕事段疾陸眷及遼五世【疾陸眷涉復辰末柸牙遼凡五世】皆見尊禮遼數與皝相攻【數所角翻】裕諫曰親仁善鄰國之寶也【左傳陳五父之言】况慕容氏與我世婚迭為甥舅【廆皝皆娶于段氏蓋前此慕容氏亦女于段也】皝有才德而我與之搆怨戰無虛月百姓彫弊利不補害臣恐社稷之憂將由此始願兩追前失通好如初【好呼到翻下好妝同】以安國息民遼不從出裕為北平相【相息亮翻】趙太子邃素驍勇【驍堅堯翻】趙王虎愛之常謂羣臣曰司馬氏父子兄弟自相殘滅故使朕得至此如朕有殺阿鐵理否【阿鐵邃小字也阿讀從安入聲】既而邃驕淫殘忍好妝飾美姬斬其首洗血置盤上與賓客傳觀之又烹其肉共食之【好呼到翻】河間公宣樂安公韜皆有寵於虎邃疾之如讎虎荒耽酒色喜怒無常使邃省可尚書事【省悉景翻】每有所關白虎恚曰此小事何足白也時或不聞又恚曰何以不白【恚於避翻恨怒也】誚責笞棰【誚才笑翻棰止蘂翻】月至再三邃私謂中庶子李顔等曰官家難稱【稱天子為官家始見於此西漢謂天子為縣官東漢謂天子為國家故兼而稱之或曰五帝官天下三王家天下故兼稱之難稱尺證翻】吾欲行冒頓之事【事見十一卷漢高帝六年冒莫北翻】卿從我乎顔等伏不敢對秋七月邃稱疾不視事潜帥宮臣文武五百餘騎飲於李顔别舍【帥讀曰率騎奇寄翻】因謂顔等曰我欲至冀州【冀州治信都】殺河間公有不從者斬行數里騎皆逃散顔叩頭固諫邃亦昏醉而歸其母鄭氏聞之私遣中人誚讓邃邃怒殺之【誚才笑翻】佛圖澄謂虎曰陛下不宜數往東宮【數所角翻】虎將視邃疾思澄言而還既而瞋目大言曰【瞋七人翻】我為天下主父子不相信乎乃命所親信女尚書往察之邃呼前與語因抽劒擊之虎怒收李顔等詰問顔具言其狀殺顔等三十餘人幽邃于東宮既而赦之引見太武東堂【水經注曰魏武居鄴為北宮宮有文昌殿石氏於故殿處起東西太武二殿見賢遍翻】邃朝而不謝俄頃即出虎使謂之曰太子應朝中宮豈可遽去【朝直遥翻】邃徑出不顧虎大怒廢邃為庶人其夜殺邃 【考異曰燕書文明紀云咸康四年四月石虎至燕城下會鄴使至太子邃在後恣酒入宮殺害石主大恐狼狽引還又云初帳下吳胄使鄴還說四月浴佛日行像詣宮石太子邃騎出迎像往來馳騁無有儲君體王曰古者觀威儀以定禍福此子虎之副貳而輕佻無禮將不得其死然及石主東歸留邃監國荒敗内亂以致誅戮按十六國晉春秋殺邃皆在咸康三年燕書恐誤今從十六國晉春秋】及其妃張氏并男女二十六人同埋於一棺誅其宮臣支黨二百餘人廢鄭后為東海太妃立其子宣為天王皇太子宣母杜昭儀為天王皇后安定侯子光自稱佛太子云從大秦國來當王小秦<br />
<br />
  國聚衆數千人於杜南山【京兆杜陵縣之南山也】自稱大皇帝改元龍興石廣討斬之 九月鎮軍左長史封奕等【帝拜皝鎮軍大將軍皝以奕為左長史】勸慕容皝稱燕王皝從之於是備置羣司以封奕為國相【相息亮翻】韓壽為司馬裴開為奉常陽騖為司隸王㝢為太僕李洪為大理杜羣為納言令宋該劉睦石琮為常伯皇甫真陽協為冗騎常侍【納言令晉之尚書令常伯晉之侍中冗騎常侍晉之散騎常侍冗而隴翻騎奇寄翻】宋晃平熙張泓為將軍封裕為記室監洪臻之孫【李臻見八十七卷懷帝永嘉三年】晃奭之子也【宋奭見八十八卷愍帝建興元年】冬十月丁卯皝即燕王位大赦十一月甲寅追尊武宣公為武宣王【廆諡武宣公】夫人段氏曰武宣后立夫人段氏為王后世子雋為王太子如魏武晉文輔政故事 段遼數侵趙邊【數所角翻】燕王皝遣揚烈將軍宋回稱藩於趙乞師以討遼自請盡帥國中之衆以會之【帥讀曰率】并以其弟寧遠將軍汗為質【沈約志寧遠將軍晉江左置盖始於此時質音致下同】趙王虎大悅厚加慰答辭其質遣還密期以明年【為趙燕攻段遼張本】 是歲趙將李穆納拓跋翳槐於大甯其故部落多歸之【元年翳槐奔趙】代王紇那奔燕國人復奉翳槐城盛樂而居之【復扶又翻樂音洛】 仇池氐王楊毅族兄初襲殺毅并有其衆自立為仇池公稱臣于趙<br />
<br />
  資治通鑑卷九十五<br />
<br />
<史部,編年類,資治通鑑>  <br>
   </div> 

<script src="/search/ajaxskft.js"> </script>
 <div class="clear"></div>
<br>
<br>
 <!-- a.d-->

 <!--
<div class="info_share">
</div> 
-->
 <!--info_share--></div>   <!-- end info_content-->
  </div> <!-- end l-->

<div class="r">   <!--r-->



<div class="sidebar"  style="margin-bottom:2px;">

 
<div class="sidebar_title">工具类大全</div>
<div class="sidebar_info">
<strong><a href="http://www.guoxuedashi.com/lsditu/" target="_blank">历史地图</a></strong>  
<a href="http://www.880114.com/" target="_blank">英语宝典</a>  
<a href="http://www.guoxuedashi.com/13jing/" target="_blank">十三经检索</a> 
<br><strong><a href="http://www.guoxuedashi.com/gjtsjc/" target="_blank">古今图书集成</a></strong> 
<a href="http://www.guoxuedashi.com/duilian/" target="_blank">对联大全</a> <strong><a href="http://www.guoxuedashi.com/xiangxingzi/" target="_blank">象形文字典</a></strong> 

<br><a href="http://www.guoxuedashi.com/zixing/yanbian/">字形演变</a>  <strong><a href="http://www.guoxuemi.com/hafo/" target="_blank">哈佛燕京中文善本特藏</a></strong>
<br><strong><a href="http://www.guoxuedashi.com/csfz/" target="_blank">丛书&方志检索器</a></strong> <a href="http://www.guoxuedashi.com/yqjyy/" target="_blank">一切经音义</a>  

<br><strong><a href="http://www.guoxuedashi.com/jiapu/" target="_blank">家谱族谱查询</a></strong>  <strong><a href="http://shufa.guoxuedashi.com/sfzitie/" target="_blank">书法字帖欣赏</a></strong> 
<br>

</div>
</div>


<div class="sidebar" style="margin-bottom:0px;">

<font style="font-size:22px;line-height:32px">QQ交流群9:489193090</font>


<div class="sidebar_title">手机APP 扫描或点击</div>
<div class="sidebar_info">
<table>
<tr>
	<td width=160><a href="http://m.guoxuedashi.com/app/" target="_blank"><img src="/img/gxds-sj.png" width="140"  border="0" alt="国学大师手机版"></a></td>
	<td>
<a href="http://www.guoxuedashi.com/download/" target="_blank">app软件下载专区</a><br>
<a href="http://www.guoxuedashi.com/download/gxds.php" target="_blank">《国学大师》下载</a><br>
<a href="http://www.guoxuedashi.com/download/kxzd.php" target="_blank">《汉字宝典》下载</a><br>
<a href="http://www.guoxuedashi.com/download/scqbd.php" target="_blank">《诗词曲宝典》下载</a><br>
<a href="http://www.guoxuedashi.com/SiKuQuanShu/skqs.php" target="_blank">《四库全书》下载</a><br>
</td>
</tr>
</table>

</div>
</div>


<div class="sidebar2">
<center>


</center>
</div>

<div class="sidebar"  style="margin-bottom:2px;">
<div class="sidebar_title">网站使用教程</div>
<div class="sidebar_info">
<a href="http://www.guoxuedashi.com/help/gjsearch.php" target="_blank">如何在国学大师网下载古籍?</a><br>
<a href="http://www.guoxuedashi.com/zidian/bujian/bjjc.php" target="_blank">如何使用部件查字法快速查字?</a><br>
<a href="http://www.guoxuedashi.com/search/sjc.php" target="_blank">如何在指定的书籍中全文检索?</a><br>
<a href="http://www.guoxuedashi.com/search/skjc.php" target="_blank">如何找到一句话在《四库全书》哪一页?</a><br>
</div>
</div>


<div class="sidebar">
<div class="sidebar_title">热门书籍</div>
<div class="sidebar_info">
<a href="/so.php?sokey=%E8%B5%84%E6%B2%BB%E9%80%9A%E9%89%B4&kt=1">资治通鉴</a> <a href="/24shi/"><strong>二十四史</strong></a>&nbsp; <a href="/a2694/">野史</a>&nbsp; <a href="/SiKuQuanShu/"><strong>四库全书</strong></a>&nbsp;<a href="http://www.guoxuedashi.com/SiKuQuanShu/fanti/">繁体</a>
<br><a href="/so.php?sokey=%E7%BA%A2%E6%A5%BC%E6%A2%A6&kt=1">红楼梦</a> <a href="/a/1858x/">三国演义</a> <a href="/a/1038k/">水浒传</a> <a href="/a/1046t/">西游记</a> <a href="/a/1914o/">封神演义</a>
<br>
<a href="http://www.guoxuedashi.com/so.php?sokeygx=%E4%B8%87%E6%9C%89%E6%96%87%E5%BA%93&submit=&kt=1">万有文库</a> <a href="/a/780t/">古文观止</a> <a href="/a/1024l/">文心雕龙</a> <a href="/a/1704n/">全唐诗</a> <a href="/a/1705h/">全宋词</a>
<br><a href="http://www.guoxuedashi.com/so.php?sokeygx=%E7%99%BE%E8%A1%B2%E6%9C%AC%E4%BA%8C%E5%8D%81%E5%9B%9B%E5%8F%B2&submit=&kt=1"><strong>百衲本二十四史</strong></a>  <a href="http://www.guoxuedashi.com/so.php?sokeygx=%E5%8F%A4%E4%BB%8A%E5%9B%BE%E4%B9%A6%E9%9B%86%E6%88%90&submit=&kt=1"><strong>古今图书集成</strong></a>
<br>

<a href="http://www.guoxuedashi.com/so.php?sokeygx=%E4%B8%9B%E4%B9%A6%E9%9B%86%E6%88%90&submit=&kt=1">丛书集成</a> 
<a href="http://www.guoxuedashi.com/so.php?sokeygx=%E5%9B%9B%E9%83%A8%E4%B8%9B%E5%88%8A&submit=&kt=1"><strong>四部丛刊</strong></a>  
<a href="http://www.guoxuedashi.com/so.php?sokeygx=%E8%AF%B4%E6%96%87%E8%A7%A3%E5%AD%97&submit=&kt=1">說文解字</a> <a href="http://www.guoxuedashi.com/so.php?sokeygx=%E5%85%A8%E4%B8%8A%E5%8F%A4&submit=&kt=1">三国六朝文</a>
<br><a href="http://www.guoxuedashi.com/so.php?sokeytm=%E6%97%A5%E6%9C%AC%E5%86%85%E9%98%81%E6%96%87%E5%BA%93&submit=&kt=1"><strong>日本内阁文库</strong></a> <a href="http://www.guoxuedashi.com/so.php?sokeytm=%E5%9B%BD%E5%9B%BE%E6%96%B9%E5%BF%97%E5%90%88%E9%9B%86&ka=100&submit=">国图方志合集</a> <a href="http://www.guoxuedashi.com/so.php?sokeytm=%E5%90%84%E5%9C%B0%E6%96%B9%E5%BF%97&submit=&kt=1"><strong>各地方志</strong></a>

</div>
</div>


<div class="sidebar2">
<center>

</center>
</div>
<div class="sidebar greenbar">
<div class="sidebar_title green">四库全书</div>
<div class="sidebar_info">

《四库全书》是中国古代最大的丛书,编撰于乾隆年间,由纪昀等360多位高官、学者编撰,3800多人抄写,费时十三年编成。丛书分经、史、子、集四部,故名四库。共有3500多种书,7.9万卷,3.6万册,约8亿字,基本上囊括了古代所有图书,故称“全书”。<a href="http://www.guoxuedashi.com/SiKuQuanShu/">详细>>
</a>

</div> 
</div>

</div>  <!--end r-->

</div>
<!-- 内容区END --> 

<!-- 页脚开始 -->
<div class="shh">

</div>

<div class="w1180" style="margin-top:8px;">
<center><script src="http://www.guoxuedashi.com/img/plus.php?id=3"></script></center>
</div>
<div class="w1180 foot">
<a href="/b/thanks.php">特别致谢</a> | <a href="javascript:window.external.AddFavorite(document.location.href,document.title);">收藏本站</a> | <a href="#">欢迎投稿</a> | <a href="http://www.guoxuedashi.com/forum/">意见建议</a> | <a href="http://www.guoxuemi.com/">国学迷</a> | <a href="http://www.shuowen.net/">说文网</a><script language="javascript" type="text/javascript" src="https://js.users.51.la/17753172.js"></script><br />
  Copyright &copy; 国学大师 古典图书集成 All Rights Reserved.<br>
  
  <span style="font-size:14px">免责声明:本站非营利性站点,以方便网友为主,仅供学习研究。<br>内容由热心网友提供和网上收集,不保留版权。若侵犯了您的权益,来信即刪。scp168@qq.com</span>
  <br />
ICP证:<a href="http://www.beian.miit.gov.cn/" target="_blank">鲁ICP备19060063号</a></div>
<!-- 页脚END --> 
<script src="http://www.guoxuedashi.com/img/plus.php?id=22"></script>
<script src="http://www.guoxuedashi.com/img/tongji.js"></script>

</body>
</html>
