






























































資治通鑑卷一百五十七 宋 司馬光 撰

胡三省 音註

梁紀十三【起旃蒙單閼盡彊圉大荒落凡三年}


高祖武皇帝十三

大同元年春正月戊申朔大赦改元 是日魏文帝即位於城西【城西長安城西也古者天子即位御前殿魏自高歡立孝武帝復用夷禮於郊拜天而後即位}
大赦改元大統追尊父京兆王為文景皇帝妣楊氏為皇后 魏渭州刺史可朱渾道元先附侯莫陳悦悦死【悦死見上卷中大通六年}
丞相泰攻之不能克與盟而罷道元世居懷朔與東魏丞相歡善又母兄皆在鄴由是常與歡通泰欲擊之道元帥所部三千戶西北度烏蘭津抵靈州【烏蘭津在平涼西北唐分平涼之會寧鎮置會州又置烏蘭縣屬焉縣西南有烏蘭關帥讀曰率}
靈州刺史曹泥資送至雲州歡聞之遣資糧迎候拜車騎大將軍【騎奇寄翻}
道元至晉陽歡始聞孝武帝之喪【魏孝武去年十二月殂}
啟請舉哀制服東魏主使羣臣議之太學博士潘崇和以為君遇臣不以禮則無反服【記檀弓魯繆公問於子思曰為舊君反服古與子思曰古之君子進人以禮退人以禮故有舊君反服之禮也今之君子進人若將加諸膝退人若將隊諸淵毋為戎首不亦善乎又何反服之禮之有孟子齊宣王曰禮為舊君有服何如斯可為之服矣孟子曰諫行言聽膏澤下於民有故而去則君使人導之出疆又先之於其所往去三年不反然後收其田里此之謂三有禮焉如此則為之服矣今也為臣諫則不行言則不聽膏澤不下於民有故而去則君搏執之又極之於其所往去之日遂收其田里此之謂寇讎寇讎何服之有}
是以湯之民不哭桀周武之民不服紂國子博士衛既隆李同軌議以為高后於永熙離絶未彰宜為之服東魏從之【歡初立孝武納其長女以為皇后帝之西奔后留不從}
魏驍騎大將軍儀同三司李虎等招諭費也頭之衆與之共攻靈州凡四旬曹泥請降【驍堅堯翻騎奇寄翻降戶江翻}
己酉魏進丞相畧陽公泰為都督中外諸軍録尚書事大行臺封安定王泰固辭王爵及録尚書乃封安定公以尚書令斛斯椿為太保廣平王贊為司徒乙卯魏主立妃乙弗氏為皇后【乙弗之先世為吐谷渾渠帥居青海即禿}


【髪傉儃所襲者也魏平凉州后之高祖莫瓌擁部落入附其後從孝文遷洛遂家焉}
子欽為皇太子后仁恕節儉不妬忌帝甚重之【以魏文帝之重乙弗后而卒迫於彊敵使后不得其死悲夫}
稽胡劉蠡升自孝昌以來自稱天子改元神嘉居雲陽谷【李延夀曰稽胡一日步落稽蓋匈奴别種劉元海五部之苗裔也或云山戎赤狄之後自離石以西安定以東方七八百里居山谷間種荒繁熾}
魏之邊境常被其患謂之胡荒【言其本胡種侵擾漢民若在荒服之外者也被皮義翻}
壬戌東魏丞相歡襲擊大破之勃海世子澄通於歡妾鄭氏【歡封渤海王以澄為世子}
歡歸【歸自襲稽胡}
一婢告之二婢為證歡杖澄一百而幽之婁妃亦隔絶不得見歡納魏敬宗之后爾朱氏有寵生子浟【浟夷周翻}
歡欲立之澄求救於司馬子如子如入見歡偽為不知者請見婁妃歡告其故子如曰消難亦通子如妾此事正可掩覆【難乃旦翻覆敷又翻}
妃是王結髪婦常以父母家財奉王【程正叔曰古人言結髪為夫婦如言結髪事君結髪戰匈奴只言初上頭時也豈謂合髻子邪按婁妃本傳妃少見歡在城上執役慕之使婢通意又數致私財使以聘己父母不得已而許焉}
王在懷朔被杖背無完皮妃晝夜供侍後避葛賊【葛賊謂葛榮也}
同走幷州貧困妃然馬矢【然與燃同}
自作靴【隋志靴胡履也取便於事施於戎服}
恩義何可忘也夫婦相宜女配至尊【妃二女長配孝武帝次配孝静帝}
男承大業【謂澄為世子也}
且婁領軍之勳何宜搖動【妃弟昭時為領軍將軍}
一女子如草芥况婢言不必信邪歡因使子如更鞫之子如見澄尤之曰男兒何意畏威自誣因教二婢反其辭【反音翻}
脅告者自縊【縊於賜翻又於計翻}
乃啟歡曰果虛言也歡大悦召婁妃及澄妃遙見歡一步一叩頭澄且拜且進父子夫婦相泣復如初【復扶又翻}
歡置酒曰全我父子者司馬子如也賜之黄金百三十斤 甲子魏以廣陵王欣為太傅儀同三司万俟夀洛千為司空【万莫北翻俟渠之翻夀洛千即受洛千}
己巳東魏以丞相歡為相國假黄鉞殊禮固辭 東魏大行臺尚書司馬子如帥大都督竇泰太州刺史韓軌等攻潼關【按韓軌傳為秦州刺史又考魏收志東魏置秦州於河東領河東北鄉二郡史蓋誤以秦為泰緣泰之誤又以泰為太帥讀曰率}
魏丞相泰軍於霸上子如與軌囘軍從蒲津宵濟攻華州【五代志後魏置東雍州於鄭縣西魏改曰華州華戶化翻}
時脩城未畢梯倚城外比曉東魏人乘梯而入刺史王羆臥尚未起聞閤外匈匈有聲【比必利翻匈許容翻匈匈喧擾之聲}
袒身露髻徒跣持白梃大呼而出【梃待鼎翻杖也白梃即今人所謂白棓也呼火故翻}
東魏人見之驚却羆逐至東門左右稍集合戰破之子如等遂引去【兵以氣勢為用兵之勇怯恃主帥以為氣勢王羆勇於赴敵而其左右又勇於戰此其所以於不備不虞之中而能却敵也}
二月辛巳上祀明堂 壬午東魏以咸陽王坦為太傳西河王悰為太尉【悰徂宗翻}
東魏使尚書右僕射高隆之發十萬夫撤洛陽宫殿運其材入鄴 丁亥上耕籍田 東魏儀同三司婁昭等攻兖州樊子鵠使前膠州刺史嚴思達守東平【魏收志泰常中置東平郡太和末罷建義中復置治秦城屬濟州秦城在范縣界}
昭攻拔之遂引兵圍瑕丘久不下昭以水灌城己丑大野拔見子鵠計事因斬其首以降始子鵠以衆少【降戶江翻少詩沼翻}
悉驅老弱為兵子鵠死各散走諸將勸婁昭盡捕誅之昭曰此州不幸横被殘賊跂望官軍以救塗炭【横戶孟翻跂去智翻舉踵也}
今復誅之【復扶又翻下衆復同}
民將誰訴皆捨之 戊戌司州刺史陳慶之伐東魏與豫州刺史堯雄戰不利而還【東魏豫州治汝南堯姓雄名}
三月辛酉東魏以高盛為太尉高敖曹為司徒濟隂王暉業為司空【濟子禮翻}
東魏丞相歡偽與劉蠡升約和許以女妻其太子【妻七細翻下后妻同}
蠡升不設備歡舉兵襲之辛酉蠡升北部王斬蠡升首以降【降戶江翻}
餘衆復立其子南海王歡進擊擒之俘其皇后諸王公卿以下四百餘人華夷五萬餘戶壬申歡入朝於鄴【朝直遙翻}
以孝武帝后妻彭城王韶【孝武帝后歡長女也}
魏丞相泰以軍旅未息吏民勞弊命所司斟酌古今可以便時適治者為二十四條新制奏行之【治直吏翻}
泰用武功蘇綽為行臺郎中【魏收志太和十一年分扶風置武功郡屬岐州即漢魏武功縣之地}
居歲餘泰未之知也而臺中皆稱其能有疑事皆就决之【就蘇綽以决疑也此就即孟子欲有謀焉則就之之就}
泰與僕射周惠達論事惠達不能對請出議之出以告綽綽為之區處【為於偽翻處昌呂翻}
惠達入白之泰稱善曰誰與卿為此議者惠達以綽對且稱綽有王佐之才泰乃擢綽為著作郎泰與公卿如昆明池觀漁行至漢故倉池【水經志沈水枝渠至章門西飛渠引水入城東為倉池池在未央宫西池有漸臺漢兵起王莽死於此臺蘇綽傳亦云行至長安城西漢故倉池}
顧問左右莫有知者泰召綽問之具以狀對泰悦因問天地造化之始歷代興亡之迹綽應對如流泰與綽並馬徐行至池竟不設網罟而還【意在問綽不在觀漁還從宣翻又如字}
遂留綽至夜問以政事臥而聽之綽指陳為治之要【治直吏翻}
泰起整衣危坐不覺膝之前席【初臥而聽繼起而整衣危坐又不覺膝之前席蓋綽之言深有以當泰心久而愈敬也}
語遂達曙不厭【天曉為曙}
詰朝謂周惠達曰蘇綽真奇士吾方任之以政【結去吉翻}
即拜大行臺左丞參典機密自是寵遇日隆綽始制文按程式朱出墨入及計帳戶籍之法【計帳者具來歲課役之大數以報度支戶籍者戶口之籍}
後人多遵用之【世有有為之主必有能者出為之用若謂天下無才吾不信也}
東魏以封延之為青州刺史代侯淵淵既失州任而懼行及廣川【沈約曰廣川縣本屬信都後漢屬清河魏屬渤海晉還清河江左僑立廣川郡縣於濟南非舊所也魏收曰齊郡廣川縣有牛山齊桓公冢管仲冢五代志齊州長山縣舊屬廣川郡}
遂反夜襲青州南郭刼掠郡縣夏四月丞相歡使濟州刺史蔡儁討之【濟子禮翻}
淵部下多叛淵欲南奔於道為賣漿者所斬送首於鄴 元慶和攻東魏城父【魏收志陳留郡浚儀縣有城父城至隋改浚儀為城父縣屬譙郡五代志譙郡城父縣宋置浚儀縣又考沈約志陳留浚儀縣竝寄治譙郡長垣縣界則知諸志所謂浚儀非我宋朝開封府之浚儀也魏收志梁州陳留郡浚儀縣則我宋朝開封之浚儀也真宗改浚儀曰祥符所謂城父則今亳州之城父縣是也父音甫}
丞相歡遣高敖曹帥三萬人趣項竇泰帥三萬人趣城父侯景帥三萬人趣彭城【帥讀曰率趣七喻翻}
以任祥為東南道行臺僕射節度諸軍【任音壬}
五月魏加丞相泰柱國【柱國大將軍}
元慶和引兵逼東魏南兖州東魏洛州刺史韓賢拒之【東魏既遷鄴以洛陽為洛州領洛陽河隂新安中川河南陽城郡}
六月慶和攻南頓豫州刺史堯雄破之 秋七月甲戌魏以開府儀同三司念賢為太尉万俟夀洛干為司徒開府儀同三司越勒肱為司空【越勒出於越勒部因以為姓}
益州刺史鄱陽王範南梁州刺史樊文熾【五代志普安郡梁置南梁州後改為安州西魏改曰始州至唐改始州曰劒州}
合兵圍晉夀魏東益州刺史傅敬和來降【降戶江翻}
範恢之子【鄱陽王恢費妃之子上之弟也}
敬和豎眼之子也【傅豎眼著功梁益而子為降虜隤其家聲忽諸豎而庾翻}
魏下詔數高歡二十罪【數所具翻}
且曰朕將親總六軍與丞相掃除凶醜歡亦移檄於魏謂宇文黑獺斛斯椿為逆徒且言令分命諸將領兵百萬刻期西討【將即亮翻}
東魏遣行臺元晏擊元慶和 或告東魏司空濟隂王暉業與七兵尚書薛琡貳於魏【曹魏置五兵尚書謂中兵外兵騎兵别兵都兵也及晉分中兵外兵為左右與舊五兵為七曹然尚書唯置五兵而已無七兵尚書之名至後魏始有七兵尚書北齊復為五兵琡昌六翻濟子禮翻}
八月辛卯執送晉陽皆免官【時東魏丞相歡居晉陽執送二人取其裁决春秋晉為方伯執列國君臣之違命者歸之京師經猶貶之况自京師而執送晉陽乎}
甲午東魏發民七萬六千人作新宫於鄴使僕射高隆之與司空胄曹參軍辛術共營之【元魏公府有法墨田水鎧胄集士等曹皆行參軍也}
築鄴南城周二十五里術琛之子也【辛琛見一百四十七卷天監六年琛丑林翻}
趙剛自蠻中往見東魏東荆州刺史趙郡李愍勸令附魏愍從之剛由是得至長安丞相泰以剛為左光禄大夫剛說泰召賀拔勝獨孤信等於梁【剛沒蠻中勝信奔梁竝見上卷上年說式芮翻}
泰使剛來請之 九月丁巳東魏以開府儀同三司襄城王旭為司空【旭吁玉翻}
冬十月魏太師上黨文宣王長孫稚卒【長知兩翻}
魏秦

州刺史王超世丞相泰之内兄也【母黨以兄弟齒謂之内兄内弟}
驕而黷貨泰奏請加法詔賜死 十一月丁未侍中中衛將軍徐勉卒【卒子恤翻}
勉雖骨鯁不及范雲亦不阿意苟合故梁世言賢相者稱范徐云【范徐既沒專任朱异梁殆矣}
癸丑東魏主祀圜丘 甲午東魏閶闔門災門之初成也高隆之乘馬遠望謂其匠曰西南獨高一寸量之果然【高居奥翻量音良}
太府卿任忻集自矜其巧不肯改【任音壬}
隆之恨之至是譛於丞相歡曰忻集潛通西魏令人故燒之歡斬之北梁州刺史蘭欽引兵攻南鄭【梁以南鄭為北梁州蓋以欽為刺史使之}


【圖南鄭也}
魏梁州刺史元羅舉州降【降戶江翻 考異曰典畧在七月今從梁帝紀}
東魏以丞相歡之子洋為驃騎大將軍開府儀同三

司封太原公【洋音羊又音祥驃匹妙翻騎奇寄翻下同}
洋内明决而外如不慧兄弟及衆人皆嗤鄙之【嗤丑之翻}
獨歡異之謂長史薛琡曰此兒識慮過吾幼時歡嘗欲觀諸子意識使各治亂絲【治直之翻下同}
洋獨抽刀斬之曰亂者必斬又各配兵四出使都督彭樂帥甲騎偽攻之兄澄等皆怖撓【帥讀曰率怖普布翻撓奴教翻}
洋獨勒衆與樂相格樂免胄言情猶擒之以獻初大行臺右丞楊愔從兄岐州刺史幼卿以直言為孝武帝所殺【愔於今翻從才用翻}
愔同列郭秀害其能恐之曰高王欲送卿於帝所愔懼變姓名逃於田横島【五代志東莱郡即墨縣有田横島}
久之歡聞其尚在召為太原公開府司馬【為楊愔為洋所親任張本}
頃之復為大行臺右丞【復扶又翻}
十二月甲午東魏文武官量事給禄【隨任事之輕重以為給禄之差量音良}
魏以念賢為太傅河州刺史梁景叡為太尉 是歲鄱陽妖賊鮮于琛改元上願有衆萬餘人【妖於驕翻琛丑林翻}
鄱陽内史吳郡陸襄討擒之按治黨與無濫死者民歌之曰鮮于平後善惡分民無枉死賴陸君 柔然頭兵可汗求婚於東魏丞相歡以常山王妹為蘭陵公主妻之柔然數侵魏【妻七細翻數所角翻}
魏使中書舍人庫狄峙奉使至柔然與約和親【使疏吏翻}
由是柔然不復為寇【復扶又翻}


二年春正月辛亥魏祀南郊改用神元皇帝配【魏高祖太和十六年以太祖道武皇帝配南郊神元皇帝魏之先祖拓拔力微也見晉武帝紀}
甲子東魏丞相歡自將萬騎襲魏夏州【將即亮翻騎奇寄翻夏戶雅翻}
身不火食四日而至縛矟為梯【矟色角翻}
夜入其城擒刺史斛拔俄彌突因而用之留都督張瓊將兵鎮守遷其部落五千戶以歸 魏靈州刺史曹泥與其壻凉州刺史普樂劉豐復叛降東魏【魏置普樂郡屬靈州五代史志靈武郡迴樂縣後周置帶普樂郡宋白曰靈州西南至凉州九百里去年曹泥降魏今復叛樂音洛復扶又翻降戶江翻}
魏人圍之 【考異曰北齊書典畧皆云周文圍泥周書不言故但云魏人}
水灌其城不沒者四尺東魏丞相歡阿至羅三萬騎徑度靈州繞出魏師之後魏師退歡帥騎迎泥及豐拔其遺戶五千以歸【高歡豈不欲與宇文争靈州哉雖鞭之長不及馬腹也}
以豐為南汾州刺史【東魏置南汾州於定陽隋改定陽縣為吉昌縣唐為慈州治所}
東魏加丞相歡九錫固讓而止 上為文帝作皇基寺以追福【帝追尊考順之曰文皇帝為於偽翻}
命有司求良材曲阿弘氏自湘州買巨材東下南津校尉孟少卿欲求媚於上【據梁紀普通六年南州津改置校尉增加奉秩南州即今采石校戶教翻少詩照翻}
誣弘氏為刧而殺之沒其材以為寺【殺無罪之人取其材以為寺福田利益果安在哉}
二月乙亥上耕籍田 東魏勃海世子澄年十五為

大行臺并州刺史【中大通五年魏以歡為大行臺歡以授其子澄歡居晉陽并州刺史地任要重故亦以澄為之}
求入鄴輔朝政【朝直遙翻下同}
丞相歡不許丞相主簿樂安孫搴為之請【為於偽翻下為我同}
乃許之丁酉以澄為尚書令加領軍京畿大都督【考異曰魏帝紀為尚書令大行臺大都督北齊文襄紀天平元年為尚書令大行臺并州刺史入輔朝政加領軍左右京畿大都督按尚書令不在外大行臺不在内今兩捨之}
魏朝雖聞其器識猶以年少期之【少詩照翻}
既至用法嚴峻事無凝滯中外震肅引并州别駕崔暹為左丞吏部郎親任之司馬子如高季式召孫搴劇飲醉甚而卒【卒子恤翻}
丞相歡親臨其喪子如叩頭請罪歡曰卿折我右臂【折而設翻}
為我求可代者子如舉中書郎魏收歡以收為主簿收子建之子也【魏子建見一百五十卷普通五年}
他日歡謂季式曰卿飲殺我孫主簿【飲於禁翻}
魏收治文書不如我意【治直之翻}
司徒嘗稱一人謹密者為誰【時東魏以高敖曹為司徒}
季式以司徒記室廣宗陳元康對曰是能夜中闇書快吏也召之一見即授大丞相功曹掌機密 【考異曰典畧孫搴卒在大同十年四月按搴卒然後陳元康為功曹高慎叛高澄已令元康救崔暹卭山之戰元康又勸高歡追宇文泰事竝在九年北史元康傳又云草劉蠡升軍書按蠡升滅在元年孫搴二年猶存今不取然則搴卒宜置於澄入輔之下}
遷大行臺都官郎時軍國多務元康問無不知歡或出臨行留元康在後馬上有所號令九十餘條元康屈指數之盡能記憶與功曹平原趙彦深同知機密時人謂之陳趙而元康勢居趙前性又柔謹歡甚親之曰如此人誠難得天賜我也彦深名隱以字行 東魏丞相歡令阿至羅逼魏秦州刺史万俟普【万莫北翻俟渠之翻}
歡以衆應之三月戊申丹楊陶弘景卒弘景博學多藝能好養生之術仕齊為奉朝請棄官隱居茅山【茅山在今建康府句容縣南五十里山記云漢時有三茅君各乘一鶴來此故名焉卒子恤翻好呼到翻朝直遙翻}
上早與之遊及即位恩禮甚篤每得其書焚香䖍受屢以手敇招之弘景不出國家每有吉凶征討大事無不先諮之月中嘗有數信【月中謂一月之中也信使信也}
時人謂之山中宰相將沒為詩曰夷甫任散誕平叔坐論空【王衍字夷甫何晏字平叔以魏晉喻梁也誕徒旱翻}
豈悟昭陽殿遂作單于宫【弘景傳曰後侯景篡果在昭陽殿史言脩道之士有識時知數者}
時士大夫競談玄理不習武事故弘景詩及之 甲寅東魏以華山王鷙為大司馬【華戶化翻}
魏以凉州刺史李叔仁為司徒万俟洛為太宰【洛字受洛干亦曰夀洛干受夀同音洛樂亦同音按北齊有太宰之官仍晉制也西魏用周制置大冢宰無太宰}
夏四月乙未以驃騎大將軍開府同三司之儀元法僧為太尉【梁開府儀同三司之下又有開府同三司之儀}
尚書右丞考城江子四上封事極言政治得失【上時掌翻治直吏翻}
五月癸卯詔曰古人有言屋漏在上知之在下朕有過失不能自覺江子四等封事所言尚書可時加檢括於民有蠧患者宜速詳啟【江子四所上封事必不敢言帝崇信釋氏而窮兵廣地適以毒民用法寛於權貴而急於細民等事特毛舉細故而論得失耳}
戊辰東魏高盛卒【高盛東魏太尉}
魏越勒肱卒【越勒肱魏司空}
魏秦州刺史万俟普與其子太宰洛豳州刺史叱千寶樂右衛將軍破六韓常及督將三百人奔東魏【阿至羅兵近普等因之以東奔 考異曰普降東魏事北齊書帝紀在三月甲午典畧在六月北史齊紀在六月甲午周書帝紀北史魏紀齊紀在五月今從之 按考異前既引北齊書帝紀又引北史齊紀不應北史魏紀之下複出齊紀必有誤}
丞相泰輕騎追之至河北千餘里不及而還【河北龍門西河之北也還從宣翻又如字下同}
秋七月庚子東魏大赦上待魏降將賀拔勝等甚厚勝請討高歡上不許勝等思歸前荆州大都督撫寧史寧謂勝曰【按寧傳寧居撫寧鎮考魏北鎮無撫寧恐即撫冥也又按五代志雕隂郡開疆縣有後魏撫寧郡又有撫寧縣亦屬雕隂郡}
朱异言於梁主無不從請厚結之勝從之上許勝寧及盧柔皆北還【盧柔蓋去年從勝來奔}
親餞之於南苑勝懷上恩自是見禽獸南向者皆不射之【射而亦翻}
行至襄城東魏丞相歡遣侯景以輕騎邀之勝等棄舟自山路逃歸【勝等舟行蓋自淮入潁自潁入汝泝流而西入山路自三鵶取武關也}
從者凍餒道死者太半【從才用翻}
既至長安詣闕謝罪魏主執勝手歔欷曰乘輿播越天也【乘繩證翻}
非卿之咎丞相泰引盧柔為從事中郎與蘇綽對掌機密 九月壬寅東魏以定州刺史侯景兼尚書右僕射南道行臺督諸將入寇【將即亮翻}
魏以扶風王孚為司徒斛斯椿為太傅 冬十月乙亥詔大舉伐東魏東魏侯景將兵七萬寇楚州【魏收志梁置楚州治楚城領汝陽仵城城陽郡五代志汝南城陽縣梁置楚州}
虜刺史桓和進軍淮上南北司二州刺史陳慶之擊破之景棄輜重走【重直用翻}
十一月己亥罷北伐之師 魏復改始祖神元皇帝為太祖道武皇帝為烈祖【魏改二祖廟號見一百三十七卷齊武帝永明九年復扶又翻}
十二月東魏以并州刺史尉景為太保壬申東魏遣使請和【使疏吏翻}
上許之 東魏清河文宣

王亶卒 【考異曰國典云亶為高歡所酖典畧周太祖數歡罪亦云殺亶魏書北史皆無亶傳而帝紀皆云亶薨今從之}
丁丑東魏丞相歡督諸軍伐魏遣司徒高敖曹趣上洛大都督竇泰趣潼關【趣七喻翻}
癸未東魏以咸陽王坦為太師 是歲魏關中大饑人相食者什七八

三年春正月上祀南郊大赦 東魏丞相歡軍蒲坂造三浮橋欲度河魏丞相泰軍廣陽【魏收志景明元年置廣陽縣屬馮翊郡}
謂諸將曰賊掎吾三面【掎居蟻翻}
作浮橋以示必度此欲綴吾軍使竇泰得西入耳歡自起兵以來竇泰常為前鋒其下多鋭卒屢勝而驕今襲之必克克泰則歡不戰自走矣諸將皆曰賊在近捨而襲遠脱有蹉跌悔何及也【蹉七何翻跌徒結翻}
不如分兵禦之丞相泰曰歡再攻潼關吾軍不出灞上【中大通六年歡攻潼關元年歡兵又攻潼關}
今大舉而來謂吾亦當自守有輕我之心乘此襲之何患不克賊雖作浮橋未能徑度不過五日吾取竇泰必矣行臺左丞蘇綽中兵參軍代人達奚武亦以為然庚戌丞相泰還長安諸將意有異同丞相泰隱其計以問族子直事郎中深【晉武帝置直事郎在尚書諸曹郎之上}
深曰竇泰歡之驍將【驍堅堯翻將即亮翻}
今大軍攻蒲坂則歡拒守而泰攻之吾表裏受敵此危道也不如選輕鋭潛出小關【小關在潼關之左唐時謂之禁谷}
竇泰躁急必來决戰【躁則到翻}
歡持重未即救我急擊泰必可擒也擒泰則歡勢自沮【沮在呂翻}
囘師擊之可以决勝丞相泰喜曰此吾心也乃聲言欲保隴右辛亥謁魏主而潛軍東出癸丑旦至小關竇泰猝聞軍至自風陵度丞相泰出馬牧澤【水經注曰桃林之塞湖水出焉其中多野馬三秦記曰桃林塞在長安東四百里若有軍馬經過則牧華山休息林下馬牧澤蓋即此地也}
擊竇泰大破之士卒皆盡竇泰自殺傳首長安丞相歡以河氷薄不能赴救撤浮橋而退儀同代人薛孤延為殿【通志曰作蕯孤復姓殿丁練翻}
一日斫十五刀折乃得免【折即設翻}
丞相泰亦引軍還【此一段皆書丞相泰所以别竇泰也歷攷前後高歡宇文泰皆書丞相於此尤為有别}
高敖曹自商山轉鬭而進【杜佑曰商山在商州上洛縣}
所向無前遂攻上洛郡人泉岳及弟猛畧與順陽人杜窋等謀翻城應之【窋竹律翻又丁骨翻}
洛州刺史泉企知之【此魏太和中所改洛州也治上洛時屬西魏企字兩道音去智翻}
殺岳及猛畧杜窋走歸敖曹敖曹以為鄉導而攻之敖曹被流矢通中者三【郷讀曰嚮被皮義翻中作仲翻}
殞絶良久復上馬免胄巡城企固守旬餘二子元禮仲遵力戰拒之仲遵傷目不堪復戰城遂陷企見敖曹曰吾力屈非心服也敖曹以杜窋為洛州刺史敖曹創甚【復扶又翻創初良翻}
曰恨不見季式作刺史【季式敖曹弟也}
丞相歡聞之即以季式為濟州刺史敖曹欲入藍田關【唐志京兆藍田縣有藍田關故嶢關也濟于禮翻}
歡使人告曰竇泰軍沒人心恐動宜速還路險賊盛拔身可也敖曹不忍棄衆力戰全軍而還以泉企泉元禮自隨泉仲遵以傷重不行企私戒二子曰吾餘生無幾汝曹才器足以立功勿以吾在東遂虧臣節元禮於路逃還泉杜雖皆為土豪鄉人輕杜而重泉元禮仲遵隂結豪右襲窋殺之魏以元禮世襲洛州刺史 二月丁亥上耕籍田【籍秦昔翻}
己丑以尚書左僕射何敬容為中權將軍【中權將軍二百四十號之一也}
護軍將軍蕭淵藻為左僕射右僕射謝舉為右光禄大夫 魏槐里獲神璽【槐里縣漢屬扶風晉屬治平郡後魏復属扶風璽斯氏翻}
大赦 三月辛未東魏遷七帝神主入新廟【七帝道武明元太武文成獻文孝文宣武}
大赦 魏斛斯椿卒【時解斯椿為魏太傳按椿居爾朱高歡之間以智數間搆其君臣之際爾朱氏既為所夷而高歡亦不能制也及入關之後與宇文泰同列若無能為者權不在已無以舞弄其智數也}
夏五月魏以廣陵王欣為太宰賀拔勝為太師 六

月魏以扶風王孚為太保梁景叡為太傅廣平王贊為太尉開府儀同三司武川王盟為司空 東魏丞相歡遊汾陽之天池【水經注太原汾陽縣北燕京山上有大池池在山原之上世謂之天池方里餘其水澄渟鏡淨而不流}
得奇石隱起成文曰六王三川以問行臺郎中陽休之對曰六者大王之字【歡字賀六渾故云然}
王者當王天下河洛伊為三川涇渭洛亦為三川【涇渭洛之洛指關中之洛水今徑鄜坊同三州而入於渭當王於况翻}
大王若受天命終應奄有關洛歡曰世人無事常言我反况聞此乎慎勿妄言休之固之子也【陽固事魏孝文帝嘗從劉昶南伐}
行臺郎中中山杜弼承間勸歡受禪歡舉杖擊走之【高歡之志蓋如曹操所謂吾為周文王者非真無移魏鼎之心也間古莧翻}
東魏遣兼散騎常侍李諧來聘以吏部郎盧元明通直侍郎李業興副之【通直侍郎即通直散騎侍郎散悉亶翻騎奇寄翻}
諧平之孫【李平崇之從弟事孝文宣武}
元明昶之子也【盧昶盧玄之孫}
秋七月諧等至建康上引見與語【見賢遍翻}
應對如流諧等出上目送之謂左右曰朕今日遇勍敵【勍其京翻}
卿輩嘗言北間全無人物此等何自而來是時鄴下言風流者以諧及隴西李神儁范陽盧元明北海王元景弘農楊遵彦清河崔贍為首【贍而艶翻}
神儁名挺寶之孫【李寶自敦煌歸魏其後貴盛}
元景名昕憲之曾孫也【王憲猛之孫皇始中歸魏}
皆以字行贍㥄之子也【㥄力膺翻}
時南北通好【好呼到翻}
務以俊乂相誇銜命接客必盡一時之選【銜命奉使者也接客主客也}
無才地者不得與焉【與讀曰預}
每梁使至鄴【使疏吏翻下同}
鄴下為之傾動貴勝子弟盛飾聚觀禮贈優渥館門成市宴日高澄常使左右覘之一言制勝澄為之拊掌魏使至建康亦然【兩國通使各務夸矜以見所長自古然矣昭奚恤之事猶可以服覘國者之心為於偽翻覘丑廉翻又丑艶翻}
獨孤信求還北上許之信父母皆在山東【魏孝武西遷信棄父母追從之}
上問信所適信曰事君者不敢顧私親而懷貳心上以為義禮送甚厚信與楊忠皆至長安上書謝罪魏以信有定三荆之功【定三荆見上卷中大通六年}
遷驃騎大將軍加侍中開府儀同三司餘官爵如故【驃匹妙翻騎奇寄翻}
丞相泰愛楊忠之勇留置帳下 魏宇文深勸丞相泰取恒農八月丁丑泰帥李弼等十二將伐東魏以北雍州刺史于謹為前鋒攻盤豆拔之【恒戶登翻帥讀曰率將即亮翻雍於用翻五代志雍州華原縣後魏置北雍州恒農湖城閿鄉之西有皇天原原西有盤豆城}
戊子至恒農庚寅拔之擒東魏陜州刺史李徽伯【魏收志太和十一年置陜州治陜城帶恒農郡領西恒農澠池石城河北郡陜式冉翻}
俘其戰士八千時河北諸城多附東魏左丞楊檦自言父猛嘗為邵郡白水令【左丞行臺左丞也魏收志皇興四年置邵郡治白水縣五代志絳郡垣縣後魏置邵郡及白水縣裴慶孫傳邵郡治陽胡城去軹關二百餘里孔穎達曰垣縣有召亭因以名郡宋白曰絳州垣縣其地即周召分陜之所今縣東六十里有邵原祠廟與古棠樹春秋襄二十二年齊侯伐晉取朝歌入孟門登太行張武軍於熒庭戍鄲邵後魏獻文四年置邵州檦與標同}
知其豪傑請往說之以取邵郡【說式芮翻下諜說同}
泰許之檦乃與土豪王覆憐等舉兵收邵郡守程保及縣令四人斬之表覆憐為郡守【守式又翻}
遣諜說諭東魏城堡旬月之間歸附甚衆【諜達協翻}
東魏以東雍州刺史司馬恭鎮正平【正平本漢晉之臨汾縣地魏真君七年分置太平縣神䴥元年改為正平太和十八年置正平郡帶間喜縣屬東雍州杜佑曰絳州治正平縣}
司空從事中郎聞喜裴邃欲攻之恭葉城走泰以楊檦行正平郡事 上脩長干寺阿育王塔出佛爪髪舍利辛卯上幸寺設無礙食【今建康府上原縣有長干里去縣五里李白長干行所謂同居長干里乃秣陵縣東里巷江東謂山隴之間曰干僧家載國事曰佛泥湼後天人以新白緤裏佛以香花供養滿七日盛以金棺送出王宫可三里許在宫各以旃檀木為薪天人各以火燒薪歛舍利得八斛四斗諸國王各得少許齎還本國以造佛寺阿肓王起浮圖於佛泥湼處李延夀扶南傳曰長干寺塔吳時有尼居其地為小精舍孫綝毀除之吳平後諸道人復於舊處建立晉簡文咸安中造塔孝武太原九年上金相輪及承露其後有西河離石縣胡人劉蕯何遇疾暴亡七日而蘇因此出家名慧達遊行至丹陽長干里有阿育王塔掘入一丈得金函盛三舍利及佛爪髪遷舍利近北對簡文所造塔西造塔及帝開之初穿土四尺得龍窟及昔人所捨金銀釵釧鐶鑷等諸雜寶物可深九尺許至石磉之下有石函函内有鐵壺以盛銀坩坩内有金縷甖盛三舍利如粟粒大圓正光潔函内有瑠璃盌盌心得四舍利及髪爪爪有四枚竝為沈香色髪青紺色衆僧以手伸之隨手長短放之則旋屈為蠡形帝乃設無礙大會豎二利各以金甖次玉甖重盛舍利及爪髪内七寶塔内又以石函盛寶塔分入兩刹刹下}
大赦九月柔然為魏侵東魏三堆【魏收志肆州永安郡平寇縣真君七年并三堆}


【縣屬焉有三堆戍隋改平寇縣為崞縣屬雁門郡宋白曰嵐州静樂縣本漢汾陽縣地城内有堆阜三俗名三堆城為于偽翻}
丞相歡擊之柔然退走行臺郎中杜弼以文武在位多貪汙言於丞相歡請治之【治直之翻}
歡曰弼來我語爾【語牛倨翻}
天下貪汙習俗已久今督將家屬多在關西【此指言可朱渾道元万俟普劉豐生等部曲也將即亮翻}
宇文黑獺常相招誘人情去留未定【獺他達翻誘音酉}
江東復有吳翁蕭衍專事衣冠禮樂【復扶又翻}
中原士大夫望之以為正朔所在我若急正綱紀不相假借恐督將盡歸黑獺士子悉奔蕭衍人物流散何以為國爾宜少待吾不忘之【史言高歡權時施宜以凝固其衆捨小過以成大功少詩沼翻}
歡將出兵拒魏杜弼請先除内賊歡問内賊為誰弼曰諸勲貴掠奪百姓者是也歡不應使軍士皆張弓注矢舉刀按矟夾道羅列命弼冒出其間弼戰慄流汗歡乃徐諭之曰矢雖注不射刀雖舉不擊矟雖按不刺【矟色角翻射而亦翻刺七亦翻}
爾猶亡魂失膽諸勳人身犯鋒鏑百死一生雖或貪鄙所取者大豈可同之常人也弼乃頓首謝不及歡每號令軍士常令丞相屬代郡張華原宣旨其語鮮卑則曰漢民是汝奴夫為汝耕婦為汝織輸汝粟帛令汝溫飽汝何為陵之其語華人則曰鮮卑是汝作客【言汝傭作之客也語牛倨翻為于偽翻下同}
得汝一斛粟一匹絹為汝擊賊令汝安寧汝何為疾之【史言高歡雜用夷夏有撫御之術}
時鮮卑共輕華人唯憚高敖曹歡號令將士常鮮卑語敖曹在列則為之華言敖曹返自上洛歡復以為軍司大都督統七十六都督【復扶又翻}
以司空侯景為西道大行臺【使景經畧關西也}
與敖曹及行臺任祥御史中尉劉貴豫州刺史堯雄冀州刺史万俟洛同治兵於虎牢【任音壬万莫北翻俟渠之翻治直之翻下同}
敖曹與北豫州刺史鄭嚴祖握槊【魏太常中置豫州治虎牢後得汝南置豫州以虎牢為北豫州領廣武滎陽成臯郡握槊亦博塞之戲也劉禹錫觀博曰初主人執握槊之器寘於廡下曰主進者要約之既揖讓即次有博齒齒異乎古之齒其制用骨觚稜四均鏤以朱墨偶而合數取應日月視其轉止依以争道是制也行之久矣莫詳所祖以其用必投擲以博投詔之又爾朱世隆與元世儁握槊忽聞局上詨然有聲一局子盡倒立世隆甚惡之既而及禍李延夀曰握槊此盖胡戲近入中國云胡王有弟一人遇罪將殺之從獄中為此戲上之意言孤則易死也}
貴召嚴祖敖曹不時遣枷其使者使者曰枷則易脱則難【枷居牙翻易弋豉翻}
敖曹以刀就枷刎之曰又何難貴不敢校明日貴與敖曹坐外白治河役夫多溺死【溺奴狄翻}
貴曰一錢漢【言漢人之賤也}
隨之死敖曹怒拔刀斫貴貴走出還營敖曹鳴鼓會兵欲攻之侯景万俟洛共解諭久之乃止敖曹嘗詣相府門者不納敖曹引弓射之【射而亦翻}
歡知而不責 閏月甲子以武陵王紀為都督益梁等十三州諸軍事益州刺史【為後紀與湘東爭國張本}
東魏丞相歡將兵二十萬自壺口趣蒲津【班志壺口山在河東郡北屈縣東南北屈後魏改為禽昌縣屬平陽郡隋改禽昌為襄陵縣將即亮翻下同趣七喻翻}
使高敖曹將兵三萬出河南時關中饑魏丞相泰所將將士不滿萬人館穀於恒農五十餘日聞歡將濟河乃引兵入關高敖曹遂圍恒農【恒戶登翻}
歡右長史薛琡言於歡曰【琡齒肓翻}
西賊連年饑饉故冒死來入陜州欲取倉粟今敖曹已圍陜城粟不得出【陜式冉翻}
但置兵諸道勿與野戰比及麥秋【比必利翻記月令孟夏之月麥秋至}
其民自應餓死寶炬黑獺何憂不降【降戶江翻}
願勿度河侯景曰今茲舉兵形勢極大萬一不捷猝難收歛不如分為二軍相繼而進前軍若勝後軍全力前軍若敗後軍承之歡不從自蒲津濟河 【考異曰北齊帝紀十一月壬辰神武自蒲津濟魏帝紀十月壬辰敗於沙苑按長歷十月壬辰朔北齊紀誤也}
丞相泰遣使戒華州刺史王羆羆語使者曰老羆當道臥貉子那得過歡至馮翊城下【使疏吏翻華戶化翻語牛倨翻五代志馮翊郡後魏曰華州西魏後改曰同州馮翊縣後魏曰華隂貉曷各翻北方有之似狐善睡康曰貉莫白切說文云北方豸種也鄭玄曰貉子曰貆郭璞曰今江東通呼貉為余按北方豸種乃指夷貉之貉孟子所謂大貉小貉者也此乃狐貉之貉當從諸家之說}
謂羆曰何不早降【降戶江翻}
羆大呼曰【呼火故翻}
此城是王羆冢死生在此欲死者來歡知不可攻乃涉洛軍於許原西【漢志馮翊懷德縣南有荆山山下有彊梁原洛水東南入渭許原蓋在洛水之南}
泰至渭南徵諸州兵皆未會欲進擊歡諸將以衆寡不敵請待歡更西以觀其勢泰曰歡若至長安則人情大擾今及其遠來新至可擊也即造浮橋於渭令軍士齎三日糧輕騎度渭輜重自渭南夾渭而西【重直用翻}
冬十月壬辰泰至沙苑【水經注沙苑在渭北沙苑之南即漢懷德縣故城五代志馮翊縣有沙苑}
距東魏軍六十里諸將皆懼宇文深獨賀泰問其故對曰歡鎮撫河北甚得衆心以此自守未易可圖【易弋豉翻}
今懸師渡河非衆所欲獨歡恥失竇泰愎諫而來所謂忿兵【愎弼力翻戾也漢魏相曰爭恨細故不忍憤怒者謂之忿兵兵忿者敗}
可一戰擒也事理昭然何為不賀願假深一節發王羆之兵邀其走路使無遺類泰遣須昌縣公達奚武覘歡軍【覘丑亷翻又丑艶翻}
武從三騎皆效歡將士衣服日暮去營數百步下馬潛聽得其軍號因上馬歷營若警夜者有不如法往往撻之具知敵之情狀而還【還從宣翻又如字}
歡聞泰至癸巳引兵會之候騎告歡軍且至泰召諸將謀之開府儀同三司李弼曰彼衆我寡不可平地置陳此東十里有渭曲可先據以待之泰從之背水東西為陳【陳讀曰陣下陷陳魏陳同}
李弼為右拒趙貴為左拒【杜預曰拒方陳陸德明曰拒俱甫翻}
命將士皆偃戈於葦中約聞鼓聲而起晡時東魏兵至渭曲【晡奔謨翻}
都督太安斛律羌舉曰黑獺舉國而來欲一死决【言欲盡死力戰之以决勝負}
譬如猘狗或能噬人【猘漢書音義征例翻又居例翻狂犬也左傳曰國犬之瘈無不噬也}
且渭曲葦深土濘無所用力【濘乃定翻泥淖也}
不如緩與相持密分精鋭徑掩長安巢穴既傾則黑獺不戰成擒矣【使斛律羌舉之計行西魏殆哉}
歡曰縱火焚之何如侯景曰當生擒黑獺以示百姓若衆中燒死誰復信之【侯景此言固亦有恃衆輕敵之心復扶又翻下復戰可復同}
彭樂盛氣請鬭曰我衆賊寡百人擒一何憂不克歡從之東魏兵望見魏兵少爭進擊之無復行列【史言東魏將士皆恃衆輕敵故敗少詩沼翻復扶又翻行戶剛翻}
兵將交丞相泰鳴鼓士皆奮起于謹等六軍與之合戰李弼帥鐵騎横擊之【時東魏之師萃於左拒于謹等與之合戰李弼引右拒之騎兵横擊之帥讀曰率}
東魏兵中絶為二遂大破之李弼弟檦身小而勇【檦與標同}
每躍馬陷陳隱身鞍甲之中敵見皆曰避此小兒泰歎曰膽决如此何必八尺之軀征虜將軍武川耿令貴殺傷多甲裳盡赤【左傳晉楚戰于邲屈蕩搏趙旃得其甲裳杜預注曰下曰裳}
泰曰觀其甲裳足知令貴之勇何必數級【謂不必數其斬獲首級之多少}
彭樂乘醉深入魏陳魏人刺之【刺七亦翻}
腸出内之復戰丞相歡欲收兵更戰使張華原以簿歷營點兵【簿者軍之名籍}
莫有應者還白歡曰衆盡去營皆空矣歡猶未肯去阜城侯斛律金曰衆心離散不可復用宜急向河東歡據鞍未動金以鞭拂馬乃馳去夜度河船去岸遠歡跨槖駝就船乃得度喪甲士八萬人【喪息浪翻}
棄鎧仗十有八萬【鎧可亥翻}
丞相泰追歡至河上選留甲士二萬餘人餘悉縱歸都督李穆曰高歡破膽矣速追之可獲泰不聽【沙苑之戰宇文泰不敢乘勝追高歡邙山之戰歡不敢乘勝追泰蓋二人者智力相敵足以相持而不足以相斃也}
還軍渭南所徵之兵甫至【甫方也}
乃於戰所人植柳一株以旌武功侯景言於歡曰黑獺新勝而驕必不為備願得精騎二萬徑往取之歡以告婁妃妃曰設如其言景豈有還理得黑獺而失景何利之有歡乃止魏加丞相泰柱國大將軍李弼十二將皆進爵增邑有差【十二將李弼獨孤信梁禦趙貴于謹古干惠怡峯劉亮王德侯莫陳崇李遠奚達武也}
高敖曹聞歡敗釋恒農退保洛陽己酉魏行臺宫景夀等向洛陽【宫姓也左傳虞大夫有宫之奇}
東魏洛州大都督韓賢擊走之州民韓木蘭作亂賢擊破之一賊匿尸間賢自按檢收鎧仗賊欻起斫之斷脛而卒【欻許勿翻斷丁管翻下兵斷同脛胡定翻}
魏復遣行臺馮翊王季海與獨孤信將步騎二萬趣洛陽【復扶又翻趣七喻翻下同}
洛州刺史李顯趣三荆【西魏洛州治上洛}
賀拔勝李弼圍蒲坂東魏丞相歡之西伐也蒲坂民敬珍謂其從祖兄祥曰高歡迫逐乘輿【乘繩證翻}
天下忠義之士皆欲剚刃於其腹【從才用翻剚側吏翻賈公彦曰剚猶立也東齊人謂立物地中為剚}
今又稱兵西上【上時掌翻}
吾欲與兄起兵斷其歸路此千載一時也【斷丁管翻}
祥從之糾合鄉里數日有衆萬餘會歡自沙苑敗歸祥珍帥衆邀之斬獲甚衆賀拔勝李弼至河東祥珍帥猗氏等六縣十餘萬戶歸之【猗氏縣自漢以來屬河東郡丁度曰猗氏本郇國也後以猗頓居於此因為猗氏按左傳所謂郇瑕氏之地沃而近盬其後猗頓居之用盬鹽起富遂以猗氏名縣而郇瑕氏隱矣帥讀曰率}
丞相泰以珍為平陽太守祥為行臺郎中東魏秦州刺史薛崇禮守蒲坂【蒲坂縣名注見周赧王十二年}
别駕薛善崇禮之族弟也言於崇禮曰高歡有逐君之罪善與兄忝衣冠緒餘世荷國恩今大軍已臨而猶為高氏固守【荷下可翻為于偽翻}
一旦城陷函首送長安署為逆賊死有餘愧及今歸款猶為愈也崇禮猶豫不決善與族人斬關納魏師崇禮出走追獲之丞相泰進軍蒲坂畧定汾絳【五代志文城郡東魏置南汾州後周改為汾州絳郡後魏置東雍州後周改為絳州}
凡薛氏預開城之謀者皆賜五等爵善曰背逆歸順【背蒲妹翻}
臣子常節豈容闔門大小俱叨封邑與其弟慎固辭不受東魏行晉州事封祖業棄城走儀同三司薛脩義追至洪洞【杜佑曰洪洞故城在平陽北四固重複控據要險劉昫曰晉州洪洞縣古陽縣也隋義寧元年改曰洪洞取縣北洪洞嶺以名縣孫恒曰洞音同又徒弄翻}
說祖業還守【說式芮翻守手又翻下固守同}
祖業不從脩義還據晉州安集固守魏儀同三司長孫子彦引兵至城下脩義開門伏甲以待之子彦不測虛實遂退走丞相歡以脩義為晉州刺史【史言河東諸薛各行其志為東西魏宣力}
獨孤信至新安【新安縣漢屬弘農郡晉屬河南尹魏天平初置新安郡屬洛州}
高敖曹引兵北度河信逼洛陽洛州刺史廣陽王湛棄城歸鄴信遂據金墉城孝武之西遷也【西遷見上卷中大通六年}
散騎常侍河東裴寛謂諸弟曰天子既西吾不可以東附高氏帥家屬逃於大石嶺【水經注洛陽之南有新城縣縣界有大石嶺未儒之水逕其南}
獨孤信入洛乃出見之時洛陽荒廢人士流散惟河東柳虯在陽城【陽城縣漢屬潁川郡晉屬河南尹魏孝昌二年置陽城郡屬洛州隋廢郡為陽城縣唐登封元年將有事嵩山改為告成縣我宋朝屬西京登封縣界}
裴諏之在潁川信俱徵之以虯為行臺郎中諏之為開府屬【諏將侯翻又逡須翻}
東魏潁州長史賀若統執刺史田迄舉城降魏【魏收志天平初置潁州治長社城領許昌潁州陽翟郡武定七年改為鄭州賀若復姓魏書官氏志内入諸姓有賀若氏北俗謂忠貞為賀若因以為氏若人者翻}
魏都督梁迴入據其城前通直散騎侍郎鄭偉起兵陳留攻東魏梁州執其刺史鹿永吉【魏收志天平初置梁州治大梁城領陽夏開封陳留汝南潁川汝陽義陽新蔡初安襄陽城陽廣陵郡鹿姓也風俗通漢有巴郡太守鹿旗魏收官氏志阿鹿桓氏後改為鹿氏}
前大司馬從事中郎崔彦穆攻滎陽執其太守蘇淑與廣州長史劉志皆降於魏【降戶江翻}
偉先護之子也【鄭先護見一百五十二卷大通二年}
丞相泰以偉為北徐州刺史彦穆為滎陽太守十一月東魏行臺任祥帥督將堯雄趙育是云寶攻潁川【帥讀曰率將即亮翻}
丞相泰使大都督宇文貴樂陵公遼西怡峯【怡姓峯名峯遼西怡寛之子}
將步騎二千救之軍至陽翟【陽翟縣漢屬潁川郡晉屬河南尹後魏置陽翟郡九域志陽翟在長社西北九十里}
雄等軍已去潁川三十里【言雄等軍逼潁川相去三十里}
祥帥衆四萬繼其後諸將咸以為彼衆我寡不可爭鋒貴曰雄等謂吾兵少必不敢進【少詩沼翻}
彼與任祥合兵攻潁川城必危矣若賀若統陷沒吾輩坐此何為今進據潁川有城可守又出其不意破之必矣遂疾趨據潁川背城為陳以待【背蒲妹翻陳讀曰陣}
雄等至合戰大破之雄走趙育請降【降戶江翻下同}
俘其士卒萬餘人悉縱遣之任祥聞雄敗不敢進貴與怡峯乘勝逼之祥退保宛陵【宛陵縣漢屬河南尹晉屬滎陽郡魏天平初分屬廣武郡隋開皇十六年并宛陵縣入新鄭}
貴追及擊之祥軍大敗是云寶殺其陽州刺史那椿以州降魏【魏收志天平初置陽州治宜陽領宜陽金門郡那椿人姓名}
魏以貴為開府儀同三司是云寶趙育為車騎大將軍都督杜陵韋孝寛攻東魏豫州拔之執其行臺馮邕孝寛名叔裕以字行丙子東魏以驃騎大將軍儀同三司万俟普為太尉司農張樂臯等聘於東魏【司農之下恐有脱宇}
十二月魏行

臺楊白駒與東魏陽州刺史段粲戰於蓼塢【水經注蓼水出河北縣襄山蓼谷當時之人於此谷築塢因謂之蓼塢漢書音義曰襄山在潼關北十餘里}
魏師敗績魏荆州刺史郭鸞攻東魏東荆州刺史清都慕容儼

【東魏都鄴以魏郡為清都尹}
儼晝夜拒戰二百餘日乘間出擊鸞大破之【間古莧翻}
時河南諸州多失守唯東荆獲全河間邢磨納范陽盧仲禮仲禮從弟仲裕等皆起兵海隅以應魏【從才用翻}
東魏濟州刺史高季式有部曲千餘人馬八百匹鎧仗皆備【濟子禮翻鎧可亥翻}
濮陽民杜靈椿等為盜聚衆近萬人【濮博木翻近其僅翻}
攻城剽野【剽匹妙翻}
季式遣騎三百一戰擒之又擊陽平賊路文徒等悉平之於是遠近肅清或謂季式曰濮陽陽平乃畿内之郡【二郡東魏皆以屬司州故云然}
不奉詔命又不侵境【言無詔命使季式討賊而賊又不來侵濟州境}
何急而使私軍遠戰萬一失利豈不獲罪乎【季式所自養部曲不衣食於官故謂之私軍}
季式曰君何言之不忠也我與國家同安共危豈有見賊而不討乎且賊知臺軍猝不能來又不疑外州有兵擊之乘其無備破之必矣以此獲罪吾亦無恨

資治通鑑卷一百五十七
















































































































































