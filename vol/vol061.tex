










 


 
 


 

  
  
  
  
  





  
  
  
  
  
 
  

  

  
  
  



  

 
 

  
   




  

  
  


    資治通鑑卷六十一   宋 司馬光 撰

  胡三省 音註

  漢紀五十三【起閼逢閹茂盡旃蒙大淵獻凡三年】

  孝獻皇帝丙

  興平元年春正月辛酉赦天下 甲子帝加元服 二月戊寅有司奏立長秋宫詔曰皇妣宅兆未卜【孝經卜其宅兆而安厝之兆塋域也】何忍言後宫之選乎壬午三公奏改葬皇妣王夫人追上尊號曰靈懷皇后【王夫人死見五十八卷靈帝光和四年皇后紀曰改葬文昭陵上時掌翻】 陶謙告急於田楷楷與平原相劉備救之備自有兵數千人謙益以丹陽兵四千備遂去楷歸謙謙表為豫州刺史屯小沛【沛國治相縣而沛自為縣屬沛國時人謂沛縣為小沛由此時呼備為劉豫州豫州刺史本治譙備領刺史而屯小沛按此時又有豫州刺史郭貢朝命不行私相署置者也】曹操軍食亦盡引兵還 馬騰私有求於李傕【傕古岳翻】不獲而怒欲舉兵相攻帝遣使者和解之不從韓遂率衆來和騰傕既而復與騰合【遂知傕之不足與也復扶又翻】諫議大夫种邵侍中馬宇左中郎將劉範謀使騰襲長安已為内應以誅傕等壬申騰遂勒兵屯長平觀【种音冲觀古玩翻】邵等謀泄出犇槐里傕使樊稠郭汜及兄子利擊之騰遂敗走還凉州又攻槐里邵等皆死庚申詔赦騰等【傕等力不能制騰遂因下詔赦之】夏四月以騰為安狄將軍遂為安降將軍【二將軍號一時暫置耳後世不復置降戶江翻】 曹操使司馬荀彧壽張令程昱守甄城【甄城縣屬濟陰郡水經註曰沇州舊治魏武創業始於此河上之邑最為峻固甄當作鄄續漢志兖州刺史治昌邑宋白曰漢獻帝於鄄城置兖州蓋曹操以刺史始治此】復往攻陶謙【復扶又翻】遂略地至琅邪東海所過殘滅還擊破劉備於郯東謙恐欲走歸丹陽【謙丹陽人也】會陳留太守張邈叛操迎呂布操乃引軍還【還從宣翻又如字】初張邈少時好游俠【少詩照翻好呼到翻俠戶頰翻】袁紹曹操皆與之善及紹為盟主有驕色【盟主事見五十九卷初平元年】邈正議責紹紹怒使操殺之操不聽曰孟卓親友也是非當容之今天下未定奈何自相危也操之前攻陶謙【見上卷上年】志在必死敕家曰我若不還往依孟卓後還見邈垂泣相對【泣垂淚也】陳留高柔謂鄉人曰曹將軍雖據兖州本有四方之圖未得安坐守也而張府君恃陳留之資將乘間為變【間古莧翻】欲與諸君避之何如衆人皆以曹張相親柔又年少不然其言柔從兄幹自河北呼柔柔舉宗從之【高幹從袁紹在河北】呂布之捨袁紹從張楊也【事見上卷上年】過邈【過工禾翻】臨别把手共誓紹聞之大恨邈畏操終為紹殺己也【為于偽翻】心不自安前九江太守陳留邊讓嘗譏議操操聞而殺之并其妻子讓素有才名由是兖州士大夫皆恐懼陳宫性剛直壯烈内亦自疑乃與從事中郎許汜王楷及邈弟超共謀叛操宫說邈曰【說輸芮翻下同】今天下分崩雄傑並起君以千里之衆當四戰之地撫劍顧盼亦足以為人豪而反受制於人不亦鄙乎今州軍東征【謂操兵征徐州也】其處空虛呂布壯士善戰無前若權迎之共牧兖州觀天下形勢俟時事之變此亦縱横之一時也【縱子容翻】邈從之時操使宫將兵留屯東郡遂以其衆濳迎布為兖州牧布至邈乃使其黨劉翊告荀彧曰呂將軍來助曹使君擊陶謙宜亟供其軍食衆疑惑彧知邈為亂即勒兵設備急召東郡太守夏侯惇於濮陽惇來布遂據濮陽【濮博木翻】時操悉軍攻陶謙留守兵少【少詩沼翻】而督將大吏多與邈宫通謀【督將領兵大吏通掌州郡事者將即亮翻】惇至其夜誅謀叛者數十人衆乃定豫州刺史郭貢率衆數萬來至城下或言與呂布同謀衆甚懼貢求見荀彧彧將往惇等曰君一州鎮也【謂一州倚之為重也】往必危不可彧曰貢與邈等分非素結也【分扶問翻】今來速計必未定及其未定說之縱不為用可使中立【賢曰不令其有所去就也說輸芮翻下同】若先疑之彼將怒而成計貢見彧無懼意謂鄄城未易攻【易以豉翻】遂引兵去是時兖州郡縣皆應布唯鄄城范東阿不動【賢曰范縣屬東郡今濮陽縣東阿縣屬東郡今濟州縣也】布軍降者言陳宫欲自將兵取東阿又使汜嶷取范【降戶江翻汜符咸翻皇甫謐云本姓几氏遭秦亂避地於汜水因氏焉嶷鄂力翻】吏民皆恐程昱本東阿人彧謂昱曰今舉州皆叛唯有此三城宫等以重兵臨之非有以深結其心三城必動君民之望也宜往撫之昱乃歸過范說其令靳允曰【過工禾翻說輸芮翻靳居焮翻姓也戰國楚冇幸臣靳尚】聞呂布執君母弟妻子孝子誠不可為心今天下大亂英雄並起必有命世能息天下之亂者此智者所宜詳擇也得主者昌失主者亡陳宫叛迎呂布而百城皆應似能有為然以君觀之布何如人哉夫布麤中少親剛而無禮匹夫之雄耳宫等以勢假合不能相君也【相如字言不能相與定君臣之分也】兵雖衆終必無成曹使君智略不世出殆天所授君必固范我守東阿則田單之功可立也【田單事見五卷周赧王三十六年】孰與違忠從惡而母子俱亡乎唯君詳慮之允流涕曰不敢有貳心時汎嶷已在縣允乃見嶷伏兵刺殺之【刺七亦翻】歸勒兵自守

  徐衆評曰允於曹公未成君臣母至親也於義應去衛公子開方仕齊積年不返管仲以為不懷其親安能愛君【齊桓公問管仲曰開方何如對曰棄親以適君非人情難親】是以求忠臣必於孝子之門允宜先救至親徐庶母為曹公所得劉備遣庶歸北欲為天下者恕人子之情也【事見後六十五卷建安十三年】曹公亦宜遣允

  昱又遣别騎絶倉亭津【水經註河水過東阿縣北河水於范縣東北流為倉亭津述征記倉亭津在范縣界去東阿六十里】陳宫至不得渡昱至東阿東阿令潁川棗祇已率厲吏民拒城堅守【潁川文士傳棗氏本姓棘避難改焉】卒完三城以待操操還執昱手曰微子之力吾無所歸矣表昱為東平相屯范呂布攻鄄城不能下西屯濮陽曹操曰布一旦得一州不能據東平斷亢父泰山之道乘險要我【東平國當亢父泰山之道亢父本屬東平章帝元和元年分屬任城賢曰亢父故城在兖州任城縣】而乃屯濮陽吾知其無能為也乃進攻之 五月以揚武將軍郭汜為後將軍安集將軍樊稠為右將軍【安集將軍一時暫置】並開府如三公合為六府【時太傅馬日磾出使李傕以車騎將軍開府汜稠又開府與三公合為六府】皆參選舉李傕等各欲用其所舉若一違之便忿憤喜怒主者患之乃以次第用其所舉【主者蓋尚書也】先從傕起汜次之稠次之三公所舉終不見用 河西四郡以去凉州治遠隔以河寇【凉州刺史本治漢陽郡冀縣時寇賊繁興遂與河西隔絶河寇蓋羣盜阻河為寇者】上書求别置州六月丙子詔以陳留邯鄲商為雍州刺史典治之【風俗通邯鄲以國為姓余謂邯鄲非國也蓋以邑為姓左傳晉有邯鄲午時置雍州治武威治直之翻】 丁丑京師地震戊寅又震 乙酉晦日有食之 秋七月壬子太尉朱儁免戊午以太常楊彪為太尉録尚書事 甲子以鎮南將軍楊定為安西將軍開府如三公 自四月不雨至于是月糓一斛直錢五十萬長安中人相食帝令侍御史侯汶出太倉米豆為貧人作糜【汶音聞糜粥也為于偽翻】餓死者如故帝疑禀賦不實【稟給也賦與也】取米豆各五升於御前作糜得二盆乃杖汶五十於是悉得全濟【觀此則獻帝非昏蔽而無知也然終以失天下者威權去已而小惠不足以得民也】 八月馮翊羌寇屬縣郭汜樊稠等率衆破之 呂布有别屯在濮陽西曹操夜襲破之未及還會布至身自摶戰自旦至日昳【昳徒結翻日昃也】數十合相持甚急操募人陷陳【陳讀曰陳】司馬陳留興韋將應募者進當之【典姓韋名】布弓弩亂發矢至如雨韋不視謂等人曰【等人者立等以募人及等者謂之等人或曰等人一等應募之人也】虜來十步乃白之等人曰十步矣又曰五步乃白等人懼疾言虜至矣韋持戟大呼而起【呼火故翻】所抵無不應手倒者布衆退會日暮操乃得引去拜韋都尉令常將親兵數百人繞大帳左右濮陽大姓田氏為反間【間古莧翻】操得入城燒其東門示無反意及操軍敗布騎得操而不識問曰曹操何在操曰乘黄馬走者是也布騎乃釋操而追黄馬者操突火而出至營自力勞軍令軍中促為攻具進復攻之【既自力勞軍又促軍進攻者恐既敗之後士氣衰沮也勞力到翻復扶又翻】與布相守百餘日蝗蟲起百姓大餓布糧食亦盡各引去九月操還鄄城布到乘氏【乘氏縣屬濟陰郡應劭曰春秋魯敗宋師於乘丘即其地宋白曰今濟州鉅野縣西南五十七里乘氏故城是也乘繩證翻】為其縣人李進所破東屯山陽冬十月操至東阿袁紹使人說操欲使操遣家居鄴【說輸芮翻】操新失兖州軍食盡將許之程昱曰意者將軍殆臨事而懼不然何慮之不深也夫袁紹有并天下之心而智不能濟也將軍自度能為之下乎【度徒洛翻】將軍以龍虎之威可為之韓彭邪今兖州雖殘尚有三城能戰之士不下萬人以將軍之神武與文若昱等收而用之【荀彧字文若】覇王之業可成也願將軍更慮之操乃止 十二月司徒淳于嘉罷以衛尉趙温為司徒錄尚書事 馬騰之攻李傕也劉焉二子範誕皆死議郎河南龎羲素與焉善乃募將焉諸孫入蜀【龐皮江翻將如字領也攜也挾也】會天火燒城焉徙治成都【劉焉初居緜竹】疽發背而卒【說文曰疽久癰】州大吏趙韙等貪焉子璋温仁共上璋為益州刺史【韙羽鬼翻上時掌翻】詔拜潁川扈瑁為刺史【瑁音冒】璋將沈彌婁發甘寧反擊璋不勝走入荆州詔乃以璋為益州牧璋以韙為征東中郎將率衆擊劉表屯朐䏰【胊䏰縣屬巴郡師古曰朐音劬晉書音義朐音蠢䏰如允翻賢曰朐䏰故城在今夔州雲安縣西萬戶故城是也䏰音閏劉朐曰開州盛山縣漢胊䏰地余據今雲安軍漢朐䏰縣地土地下濕多朐䏰蟲故名劉禹錫曰朐䏰蚯蚓也裴松之曰䏰如振翻】 徐州牧陶謙疾篤謂别駕東海麋竺曰【姓譜楚大夫受封於南郡麋亭因以為氏或言工尹麋之後以名為氏】非劉備不能安此州也謙卒竺率州人迎備備未敢當曰袁公路近在壽春【袁術字公路】君可以州與之典農校尉下邳陳登曰【據裴松之註三國志云陶謙表登為典農校尉魏志曰曹公置典農校尉秩比二千石蓋先已有此官曹公增其秩耳】公路驕豪非治亂之主【治直之翻】今欲為使君合步騎十萬【為于偽翻】上可以匡主濟民下可以割地守境【觀登此言固未易才也】若使君不見聽許登亦未敢聽使君也北海相孔融謂備曰袁公路豈憂國忘家者邪冢中枯骨何足介意【據陳夀志備謂竺等曰袁公路近在夀春此君四世五公君可以州歸之融言冢中枯骨何足介意正為四世五公發也】今日之事百姓與能天與不取悔不可追【易曰人謀鬼謀百姓與能言百姓惟能者是與也前書曰天與不取反受其咎】備遂領徐州 初太傅馬日磾與趙岐俱奉使至壽春【磾丁奚翻】岐守志不橈【橈奴教翻】袁術憚之日磾頗有求於術術侵侮之從日磾借節視之因奪不還條軍中十餘人使促辟之日磾從術求去術留不遣又欲逼為軍師日磾病其失節嘔血而死【杜預曰病者以為已病也】 初孫堅娶錢唐吳氏生四男策權翊匡及一女堅從軍於外留家壽春策年十餘歲已交結知名舒人周瑜與策同年亦英逹夙成【夙早也】聞策聲問自舒來造焉便推結分好【造七到翻分扶問翻推分而結好也好呼到翻下同】勸策徙居舒策從之瑜乃推道旁大宅與策【推吐雷翻】升堂拜母有無通共及堅死策年十七還葬曲阿【曲阿縣屬吳郡賢曰今潤州縣余據曲阿古雲陽縣也秦時言其地有天子氣始皇鑿北阬以敗其勢截直道使阿曲故謂之曲阿杜佑曰曲阿今丹陽郡丹陽縣】已乃渡江居江都結納豪俊有復讐之志【以父堅為黄祖所殺也】丹陽太守會稽周昕與袁術相惡【會工外翻】術上策舅吳景領丹陽太守【上時掌翻】攻昕奪其郡以策從兄賁為丹陽都尉【從才用翻下賢從同】策以母弟託廣陵張紘徑到壽春見袁術涕泣言曰亡父昔從長沙入討董卓與明使君會於南陽同盟結好不幸遇難勲業不終【事見五十九卷初平元年二年難乃旦翻】策感惟先人舊恩欲自憑結願明使君埀察其誠術甚奇之然未肯還其父兵謂策曰孤用貴舅為丹陽太守賢從伯陽為都尉【舅謂吳景孫賁字伯陽】彼精兵之地【丹陽號為天下精兵處】可還依召募策遂與汝南呂範及族人孫河迎其母詣曲阿依舅氏因緣召募得數百人而為涇縣大帥祖即所襲【涇縣屬丹陽郡賢曰今宣州縣姓譜祖商祖己之後帥所類翻】幾至危殆於是復往見術【幾居希翻復扶又翻下同】術以堅餘兵千餘人還策表拜懷義校尉策騎士有罪逃入術營隱於内廐策指使人就斬之訖詣術謝【謝入術營專殺也】術曰兵人好叛當共疾之【好呼到翻】何為謝也由是軍中益畏憚之術初許以策為九江太守己而更用丹陽陳紀【更工衡翻】後術欲攻徐州從廬江太守陸康求米三萬斛康不與術大怒遣策攻康謂曰前錯用陳紀【錯誤也】每恨本意不遂今若得康廬江真卿有也策攻康拔之術復用其故吏劉勲為太守【復扶又翻】策益失望侍御史劉繇岱之弟也素有盛名詔書用為揚州刺史州舊治壽春【續漢志揚州本治歷陽蓋中世以後徙治壽春也】術已據之繇欲南渡江吳景孫賁迎置曲阿及策攻廬江繇聞之以景賁本術所置懼為袁孫所并遂搆嫌隙迫逐景賁景賁退屯歷陽【歷陽縣屬九江郡今和州】繇遣將樊能于糜屯横江張英屯當利口以拒之【横江渡在今和州正對江南之采石即今之楊林渡口當利浦在今和州東十二里】術乃自用故吏惠衢為揚州刺史【惠姓也戰國時梁有惠施】以景為督軍中郎將與賁共將兵擊英等

  二年春正月癸丑赦天下 【考異曰袁紀作癸酉按長歷是月癸卯朔無癸酉今從范書】 曹操敗呂布於定陶【敗補邁翻】 詔即拜袁紹為右將軍【即拜者就拜之也時紹在鄴就鄴拜之考異曰袁紀作後將軍今從范書】 董卓初死三輔民尚數十萬戶李傕等放兵劫掠加以饑饉二年間民相食略盡李傕郭汜樊稠各相與矜功争權欲鬬者數矣【數所角翻】賈詡每以大體責之雖内不能善外相含容樊稠之擊馬騰韓遂也李利戰不甚力稠叱之曰人欲截汝父頭【利傕兄子也故云然】何敢如此我不能斬卿邪及騰遂敗走稠追至陳倉遂語稠曰【語牛倨翻】本所爭者非私怨王家事耳與足下州里人【韓遂金城人與樊稠皆凉州人也】欲相與善語而别乃俱却騎前接馬交臂相加共語良久而别軍還李利告傕韓樊交馬語不知所道意愛甚密傕亦以稠勇而得衆忌之稠欲將兵東出關從傕索益兵【索山客翻】二月傕請稠會議便於坐殺稠【坐徂卧翻】由是諸將轉相疑貳傕數設酒請郭汜【數所角翻】或留汜止宿汜妻恐汜愛傕婢妾思有以間之【間工莧翻】會傕送饋【餉食曰饋】妻以豉為藥擿以示汜曰【豉是義翻擿他歷翻挑也】一栖不兩雄我固疑將軍信李公也【以雞為喻也一栖而兩雄必鬭】他日傕復請汜飲大醉【復扶又翻下同】汜疑其有毒絞糞汁飲之【糞汁解衆毒】於是各治兵相攻矣【治直之翻】帝使侍中尚書和傕汜傕汜不從汜謀迎帝幸其營夜有亡者告傕三月丙寅傕使兄子暹將數千兵圍宫以車三乘迎帝【暹息亷翻將即亮翻乘繩證翻下同】太尉楊彪曰自古帝王無在人家者諸君舉事奈何如是暹曰將軍計定矣於是羣臣步從乘輿以出兵即入殿中掠宫人御物帝至傕營傕又徙御府金帛置其營遂放火燒宫殿官府居民悉盡帝復使公卿和傕汜汜留楊彪及司空張喜尚書王隆光祿勲劉淵衛尉士孫瑞太僕韓融廷尉宣璠【璠孚袁翻】大鴻臚榮郃【滎姓也前書有男子榮畜姓譜周榮公之後郃曷閤翻又古合翻】大司農朱儁將作大匠梁邵屯騎校尉姜宣等於其營以為質【質音致下同】朱儁憤懣發病死【懣音悶又音滿】 夏四月甲子立貴人琅邪伏氏為皇后以后父侍中完為執金吾郭汜饗公卿議攻李傕楊彪曰羣臣共鬬一人劫天

  子一人質公卿可行乎【質音致】汜怒欲手刃之彪曰卿尚不奉國家吾豈求生邪中郎將楊密固諫汜乃止傕召羌胡數千人先以御物繒綵與之【繒慈陵翻】許以宫人婦女欲令攻郭汜汜陰與傕黨中郎將張苞等謀攻傕丙申汜將兵夜攻傕門矢及帝簾帷中又貫傕左耳苞等燒屋火不然楊奉於外拒汜汜兵退苞等因將所領兵歸汜是日傕復移乘輿幸北塢【據傕汜和後然後帝得出長安宣平門則北塢蓋在長安城中傕汜於城中各築塢而居也復扶又翻】使校尉監塢門【監工衘翻】内外隔絶侍臣皆有饑色帝求米五斗牛骨五具以賜左右傕曰朝晡上飰【上時掌翻飰與飯同】何用米為乃以臭牛骨與之帝大怒欲詰責之侍中楊琦諫曰傕自知所犯悖逆欲轉車駕幸池陽黄白城【池陽縣屬馮翊賢曰故城在今涇陽縣西北水經註曰黄白城本曲梁宫也詰去吉翻悖蒲妹翻又蒲没翻】臣願陛下忍之帝乃止司徒趙温與傕書曰公前屠陷王城殺戮大臣今争睚眥之隙【睚牛懈翻怒視也眥疾智翻目際也毛晃曰厓眥舉目相忤貌亦作眦士懈翻】以成千鈞之讎【千鈞言重也】朝廷欲令和解詔命不行而復欲轉乘輿於黄白城此誠老夫所不解也【乘繩證翻解胡買翻曉也】於易一為過再為涉三而弗改滅其頂凶【易大過上六曰過涉滅頂凶温依此而分一再三之義】不如早共和解傕大怒欲殺温其弟應諫之數日乃止【據獻帝起居註應温故掾也】傕信巫覡厭勝之術【覡奴歷翻國語在女曰巫在男曰覡厭益涉翻】常以三牲祠董卓於省門外每對帝或言明陛下或言明帝為帝說郭汜無狀【為于偽翻】帝亦隨其意應答之傕喜自謂良得天子歡心也【良信也】閏月己卯帝使謁者僕射皇甫酈和傕汜 【考異曰袁紀酈作麗今從范書】酈先詣汜汜從命又諸傕傕不肯曰郭多盜馬虜耳【英雄記曰郭汜一名多】何敢欲與吾等邪必誅之君觀吾方略士衆足辦郭多否邪郭多又劫質公卿【質音致下同】所為如是而君苟欲左右之邪【左右助也音佐佑】酈曰近者董公之彊將軍所知也呂布受恩而反圖之斯須之間身首異處此有勇而無謀也今將軍身為上將荷國寵榮【荷下可翻】汜質公卿而將軍脅主誰輕重乎張濟與汜有謀楊奉白波賊帥耳【帥所類翻】猶知將軍所為非是將軍雖寵之猶不為用也傕呵之令出酈出詣省門白傕不肯奉詔辭語不順【天子所居曰禁中亦曰省中省門即禁門也】帝恐傕聞之亟令酈去傕遣虎賁王昌呼欲殺之昌知酈忠直縱令去還答傕言追之不及 辛巳以車騎將軍李傕為大司馬在三公之右 呂布將薛蘭李封屯鉅野【鉅野縣屬山陽郡郭周於此置濟州】曹操攻之布救蘭等不勝而走操遂斬蘭等操軍乘氏【乘繩證翻】以陶謙已死欲遂取徐州還乃定布荀彧曰昔高祖保關中光武據河内【高祖取天下令蕭何守關中光武經營河北令寇恂守河内皆以為王業根本】皆深根固本以制天下進足以勝敵退足以堅守故雖有困敗而終濟大業將軍本以兖州首事平山東之難【賢曰曹操初從東郡守鮑信等遥領兖州牧遂進兵破黄巾等故能平定山東也余據此時山東猶未盡平彧誇之耳難乃旦翻】百姓無不歸心悦服且河濟天下之要地也【禹貢兖州之域孔安國曰東南據濟西北距河濟子禮翻】今雖殘壞猶易以自保【易以豉翻】是亦將軍之關中河内也不可以不先定今已破李封薛蘭若分兵東擊陳宫宫必不敢西顧以其間收熟麥約食畜穀一舉而布可破也破布然後南結揚州【謂結劉繇也】共討袁術以臨淮泗若舍布而東【舍讀作捨】多留兵則不足用少留兵則民皆保城不得樵采布乘虛寇暴民心益危唯甄城范衛可全【衛謂濮陽杜預曰濮陽古衛地甄當作鄄】其餘非己之有是無兖州也若徐州不定將軍當安所歸乎且陶謙雖死徐州未易亡也【易以豉翻】彼懲往年之敗將懼而結親【結親猶言親結也】相為表裏今東方皆已收麥必堅壁清野以待將軍攻之不拔略之無獲不出十日則十萬之衆未戰而先自困耳前討徐州威罰實行【謂多所屠戮也】其子弟念父兄之耻必人自為守無降心就能破之尚不可有也【徐州子弟既有父兄之讎必不心服於操縱破其兵猶不能有其地也降戶江翻】夫事故有棄此取彼者以大易小可也以安易危可也權一時之埶不患本之不固可也今三者莫利惟將軍熟慮之操乃止布復從東緡【東緡縣屬山陽郡春秋之緡邑也宋白曰今濟州金鄉縣本漢東緡縣復扶又翻緡眉巾翻】與陳宫將萬餘人來戰操兵皆出收麥在者不能千人屯營不固屯西有大隄其南樹木幽深操隱兵隄裏出半兵隄外布益進乃令輕兵挑戰【挑徒了翻】既合伏兵乃悉乘隄【前書音義曰乘登也】步騎並進大破之追至其營而還布夜走操復攻拔定陶分兵平諸縣布東犇劉備張邈從布使其弟超將家屬保雍丘【雍丘縣屬陳留郡故杞國也】布初見備甚尊敬之謂備曰我與卿同邊地人也【布五原人備涿郡人五原涿郡皆邊地】布見關東起兵欲誅董卓布殺卓東出關東諸將無安布者皆欲殺布耳請備於帳中坐婦牀上令婦向拜酌酒飲食名備為弟備見布語言無常外然之而内不悦李傕郭汜相攻連月死者以萬數六月傕將楊奉謀

  殺傕事泄遂將兵叛傕傕衆稍衰【果如皇甫酈之言】庚午鎮東將軍張濟自陜至【陜縣屬弘農張濟初平三年出戌焉陜式冉翻】欲和傕汜遷乘輿權幸弘農【乘繩證翻下同】帝亦思舊京【謂雒陽也】遣使宣諭十反汜傕許和欲質其愛子【質音致下同】傕妻愛其男和計未定而羌胡數來闚省門【數所角翻】曰天子在此中邪李將軍許我宫人今皆何在帝患之使侍中劉艾謂宣義將軍賈詡曰【宣義將軍亦一時暫置】卿前奉職公忠故仍升榮寵今羌胡滿路宜思方略詡乃召羌胡大帥飲食之【帥所類翻飲於禁翻食讀曰飤】許以封賞羌胡皆引去傕由此單弱於是復有言和解之計者【復扶又翻】傕乃從之各以女為質秋七月甲子車駕出宣平門【宣平門長安城東出北頭第一門】當度橋汜兵數百人遮橋曰此天子非也車不得前傕兵數百人皆持大戟在乘輿車前兵欲交侍中劉艾大呼曰是天子也使侍中楊琦高舉車帷帝曰諸君何敢迫近至尊邪【呼火故翻近其勤翻】汜兵乃却既度橋士衆皆稱萬歲夜到霸陵從者皆饑【從才用翻】張濟賦給各有差傕出屯池陽丙寅以張濟為票騎將軍開府如三公【票匹妙翻】郭汜為車騎將軍楊定為後將軍楊奉為興義將軍皆封列侯【以楊奉自白波賊帥勤王故以興義寵之】又以故牛輔部曲董承為安集將軍【蜀志曰承獻帝舅也裴松之曰承靈帝母董太后之姪於獻帝為丈人蓋古無丈人之名故謂之舅也】郭汜欲令車駕幸高陵【高陵縣屬馮翊】公卿及濟以為宜幸弘農大會議之不决帝遣使諭汜曰弘農近郊廟【近其靳翻】勿有疑也汜不從帝遂終日不食汜聞之曰可且幸近縣八月甲辰車駕幸新豐丙子郭汜復謀脅帝還都郿【復扶又翻下同】侍中种輯知之密告楊定董承楊奉令會新豐郭汜自知謀泄乃棄軍入南山【自新豐驪山西接終南謂之南山】 曹操圍雍丘張邈詣袁術求救未至為其下所殺 冬十月以曹操為兖州牧戊戌郭汜黨夏育高碩等謀脅乘輿西行【夏戶雅翻】侍中劉艾見火起不止請帝出幸一營以避火【時郭汜楊定董承楊奉各自為營艾不敢指言故請幸一將營惟帝意所向也】楊定董承將兵迎天子幸楊奉營夏育等勒兵欲止乘輿楊定董承力戰破之乃得出壬寅行幸華陰【華戶化翻】寧輯將軍段煨具服御及公卿已下資儲欲上幸其營【寧輯之號猶安集亦一時暫置也煨烏回翻】煨與楊定有隙定黨种輯左靈言煨欲反太尉楊彪司徒趙温侍中劉艾尚書梁紹皆曰段煨不反臣等敢以死保董承楊定脅弘農督郵令言郭汜來在煨營帝疑之乃露次於道南【野宿無廬舍謂之露次】丁未楊奉董承楊定將攻煨使种輯左靈請帝為詔帝曰煨罪未著奉等攻之而欲令朕有詔邪輯固請至夜半猶弗聽奉等乃輒攻煨營十餘日不下煨供給御膳稟贍百官無有二意【贍而艶翻】詔使侍中尚書告諭定等令與煨和解定等奉詔還營李傕郭汜悔令車駕東聞定攻煨相招共救之因欲劫帝而西楊定聞傕汜至欲還藍田為汜所遮單騎亡走荆州張濟與楊奉董承不相平乃復與傕汜合十二月帝幸弘農張濟李傕郭汜共追乘輿大戰於弘農東澗承奉軍敗百官士卒死者不可勝數棄御物符策典籍畧無所遺【凡乘輿服御之物皆為御物符銅虎符竹使符之類符之為言扶也兩相扶合而不差也又曰符輔也所以輔信又合也驗也策編簡為之古者誥命皆書之策漢制天子策書長二尺典籍内府圖籍及尚書中故事之類勝音升】射聲校尉沮儁被創墜馬【沮子余翻創初良翻】傕謂左右曰尚可活否儁罵之曰汝等凶逆逼劫天子使公卿被害【被皮義翻】宫人流離亂臣賊子未有如此也傕乃殺之壬申帝露次曹陽【賢曰曹陽澗名在今陜州西南七里俗謂之七里澗崔皓云自南山北通於河魏武帝改曰好陽杜佑曰陜郡西四十五里有曹陽澗以下文觀之杜佑說是】承奉乃譎傕等與連和而密遣間使至河東【譎古穴翻間古莧翻使疏吏翻】招故白波帥李樂韓暹胡才【帥所類翻暹息廉翻】及南匈奴右賢王去卑並率其衆數千騎來與承奉共擊傕等大破之斬首數千級於是董承等以新破傕等可復東引庚申車騎發東【自曹陽發而東行也】董承李樂衛乘輿胡才楊奉韓暹匈奴右賢王於後為拒傕等復來戰奉等大敗死者甚於東澗光祿鄧淵廷尉宣璠【璠孚袁翻】少府田芬大司農張義皆死司徒趙温太常王絳衛尉周忠司隸校尉管郃為傕所遮欲殺之【郃古合翻又曷閤翻】賈詡曰此皆大臣卿奈何害之乃止李樂曰事急矣陛下宜御馬上曰不可舍百官而去此何辜哉【觀帝此言發於臨危之時豈可以亡國之君待之哉特為強兵所制耳舍讀曰捨】兵相連綴四十里方得至陜【杜佑曰陜春秋虢國之地所謂北虢也】乃結營自守時殘破之餘虎賁羽林不滿百人傕汜兵繞營叫呼【呼火故翻】吏士失色各有分散之意李樂懼欲令車駕御船過砥柱出孟津【水經註河水逕大陽縣南又東過㡳柱間底柱山名也昔禹治洪水山陵當水者鑿之故破山以通河河水分流包山而過山見水中若柱然故曰底柱三穿既决水勢疎分指狀表目亦曰三門山在虢城東北大陽城東自㡳柱而下至五戶灘其間一百二十里有一十九灘水流濬急破舟船自古所患河水又東過平陰縣北又東過河陽縣南則孟津也】楊彪以為河道險難非萬乘所宜乘【萬乘繩證翻下乘輿同】乃使李樂夜渡濳具船舉火為應上與公卿步出營皇后兄伏德扶后一手挾絹十疋董承使符節令孫徽從人間斫之【百官志符節令屬少府秩六百石為符節臺率主符節事凡遣使掌授節】殺旁侍者血濺后衣【濺子賤翻】河岸高十餘丈【高居傲翻】不得下乃以絹為輦使人居前負帝餘皆匍匐而下或從上自投冠幘皆壞既至河邊士卒争赴舟董承李樂以戈擊之手指於舟中可掬【左傳晉荀林父帥師戰于邲而敗中軍與下軍争舟舟中之指可掬也】帝乃御船同濟者皇后及楊彪以下纔數十人其宫女及吏民不得渡者皆為兵所掠奪衣服俱盡髪亦被截凍死者不可勝討【勝音升】衛尉士孫瑞為傕所殺傕見河北有火遣騎候之適見上渡河呼曰汝等將天子去邪董承懼射之以被為幔【懼傕兵射之故以被為幔以禦箭幔莫半翻幕也射而亦翻】旣到大陽【賢曰大陽縣屬河東郡前書音義曰在大河之陽即今陜州河北縣是也】幸李樂營河内太守張楊使數千人負米來貢餉乙亥帝御牛車幸安邑【安邑縣屬河東郡】河東太守王邑奉獻綿帛悉賦公卿以下【賦給與也分界也】封邑為列候拜胡才為征東將軍張楊為安國將軍【安國將軍之號蓋始於此】皆假節開府其壘壁羣帥競求拜職【帥所類翻】刻印不給至乃以錐畫之乘輿居棘籬中門戶無關閉天子與羣臣會兵士伏籬上觀互相鎮壓以為笑【鎮側人翻】帝又遣太僕韓融至弘農與傕汜等連和傕乃放遣公卿百官頗歸所掠宫人及乘輿器服已而糧穀盡宫人皆食菜果乙卯張楊自野王來朝【野王縣屬河内郡隋唐為河内縣】謀以乘輿還雒陽諸將不聽楊復還野王是時長安城空四十餘日彊者四散羸者相食【羸倫為翻】二三年間關中無復人跡沮授說袁紹曰將軍累葉台輔世濟忠義今朝廷播越【播流也遷也越顚墜也走也賢曰播遷也越逸也言失其所居說輸芮翻下同】宗廟殘毁觀諸州郡雖外託義兵内實相圖未有憂存社稷卹民之意今州域粗定【州域謂冀州之域也粗坐五翻】兵強士附西迎大駕即宫鄴都【即就也】挾天子而令諸侯畜士馬以討不庭【不庭謂不朝者杜預曰下之事上皆成禮於庭中一曰庭直也不庭謂不直者】誰能禦之潁川郭圖淳于瓊曰漢室陵遲為日久矣【王肅註家語曰言若丘陵之漸逶遲】今欲興之不亦難乎且英雄並起各據州郡連徒聚衆動有萬計所謂秦失其鹿先得者王倘迎天子自近【近其靳翻】動輒表聞從之則權輕違之則拒命非計之善者也授曰今迎朝廷於義為得於時為宜若不早定必有先之者矣紹不從【紹不能從授之言果為曹操所先帝既都許乃欲移以自近不亦晩乎先悉薦翻 考異曰魏志紹傳曰天子在河東紹遣郭圖使焉圖還說紹迎天子都鄴紹不從今從范書】初丹陽朱治嘗為孫堅校尉【治從堅討長沙零桂賊表行都尉又從破董卓於陽人表行督軍校尉】見袁術政德不立勸孫策歸取江東時吳景攻樊能張英等歲餘不克策說術曰家有舊恩在東願助舅討横江横江拔因投本土召募可得三萬兵以佐明使君定天下【策本江東人故謂之本土】術知其恨【謂許以九江廬江而不用也】而以劉繇據曲阿王朗在會稽【會工外翻】謂策未必能定乃許之表策為折衝校尉將兵千餘人騎數十匹【校戶教翻將即亮翻騎奇寄翻】行收兵比至歷陽【比必寐翻】衆五六千時周瑜從父尚為丹陽太守【從才用翻】瑜將兵迎之仍助以資糧策大喜曰吾得卿諧也【諧偶也合也史言推結分好正當於此觀之又當於此别分好二字英雄相遇於草澤一見之頃靡然為之服役此豈聲音笑貌所能為哉】進攻横江當利皆拔之樊能張英敗走策渡江轉鬭所向皆破莫敢當其鋒者百姓聞孫郎至皆失䰟魄【江表傳曰策年少雖有位 號而吳人皆謂之孫郎】長吏委城郭竄伏山草【山草言深山茂草之中也李固對策曰臣伏從山草痛心傷臆則山草二字當時常談也長知兩翻】及策至軍士奉令不敢虜略雞犬菜茹一無所犯【茹亦菜也】民乃大悦競以牛酒勞軍策為人美姿顔能笑語濶逹聽受善於用人是以士民見者莫不盡心樂為致死【勞力到翻樂音洛為于偽翻 考異曰魏志袁紀皆云初平四年策受袁術使渡江漢獻帝紀吳志孫策傳皆云興平元年虞溥江表傳云策興平二年渡江按術初平四年始得壽春策傳云術欲攻徐州從陸康求米事必在劉備得徐州後也劉繇傳稱吳景攻繇歲餘不克則策渡江不應在興平元年已前今依江表傳為定】策攻劉繇牛渚營【郡國志丹陽郡秣陵縣南冇牛渚杜佑曰牛渚圻即宣城郡當塗縣采石今太平州當塗縣北三十里有牛渚山是也】盡得邸閣糧穀戰具【邸至也言所歸至也閣庋置也邸閣謂轉輸之歸至而庋置之也】時彭城相薛禮下邳相丹陽笮融依繇為盟主禮據秣陵城【沈約曰秣陵其地本名金陵本治去京邑六十里今故治村是也元豐九域志江寧府江寧縣冇秣陵鎮丁度集韻笮側格切姓也風俗通楚有笮倫】融屯縣南策皆擊破之又破繇别將於梅陵【唐書地理志宣州南陵縣有梅根鎮今冇梅根港】轉攻湖孰江乘皆下之【郡國志丹陽郡有湖孰江乘二縣元豐九域志江寧府上元縣有湖孰鎮】進擊繇於曲阿繇同郡太史慈時自東萊來省繇【大史以官為氏繇與慈皆東萊人也省悉景翻】會策至或勸繇可以慈為大將繇曰我若用子義【太史慈字子義】許子將不當笑我邪【以其覈論人品也】但使慈偵視輕重【偵丑正翻候視也】時獨與一騎卒遇策於神亭【神亭在今鎮江府丹陽縣界卒讀曰猝】策從騎十三【從才用翻】皆堅舊將遼西韓當零陵黄蓋輩也慈便前鬭正與策對策刺慈馬【刺匕亦翻】而擥得慈項上手戟【擥與攬同】慈亦得策兜鍪會兩家兵騎並各來赴於是解散【若論技擊則慈策適相當耳然慈終困於策何也】繇與策戰兵敗走丹徒 【考異曰帝紀繇敗走在興平元年今從江表傳】策入曲阿勞賜將士【勞力到翻】發恩布令告諭諸縣其劉繇笮融等故鄉部曲來降首者一無所問【首式救翻】樂從軍者一身行【樂音洛下同】復除門戶【復方目翻一人以身行除其門戶賦役也】不樂者不強【強其兩翻】旬日之間四面雲集得見兵二萬餘人【見賢遍翻】馬千餘匹威震江東丙辰袁術表策行殄寇將軍【殄寇將軍號盖始於此】策將呂範言於策曰今將軍事業日大士衆日盛而綱紀猶有不整者範願暫領都督佐將軍部分之【分扶問翻】策曰子衡旣士大夫【呂範字子衡】加手下已有大衆立功於外【範先領宛陵令破丹陽賊而還】豈宜復屈小職知軍中細事乎範曰不然今捨本土而託將軍者非為妻子也【呂範汝南人復扶又翻為于偽翻】欲濟世務也譬猶同舟涉海一事不牢卽俱受其敗此亦範計非但將軍也策笑無以答範出便釋褠著袴褶【褠居侯翻單衣也著陟畧翻褶席入翻袴褶騎服也】執鞭詣閤下啟事自稱領都督策乃授傳【傳株戀翻符傳也】委以衆事由是軍中肅睦威禁大行【老子曰盗亦有道儻無其道安能為盗哉】策以張紘為正議校尉彭城張昭為長史常令一人居守【守手又翻】一人從征討及廣陵秦松陳端等亦參與謀謨【與讀曰預】策待昭以師友之禮文武之事一以委昭昭每得北方士大夫書疏專歸美於昭策聞之歡笑曰昔管子相齊一則仲父二則仲父而桓公為覇者宗【新序曰有司請吏於齊桓公公曰以告仲父有司又請公曰以告仲父在側者曰一則告仲父二則告仲父易哉為君公曰吾未得仲父則難已得仲父曷為其不易故王者勞於求賢佚於得人】今子布賢我能用之【張昭字子布】其功名獨不在我乎【策任張昭昭何足以當管仲策之斯言蓋因北方人士書疏從而歸重耳英雄胷次可易測邪】袁術以從弟胤為丹陽太守【從才用翻】周尚周瑜皆還壽春劉繇自丹徒將犇會稽【會工外翻】許劭曰會稽富實策之所貪且窮在海隅不可往也不如豫章北連豫壤西接荆州若收合吏民遣使貢獻與曹兖州相聞雖有袁公路隔在其間其人豺狼不能久也【豫章在大江東南豫兖之壤在淮北袁術時據九江廬江之間故云隔在其中】足下受王命孟德景升必相救濟【曹操字孟德劉表字景升】繇從之 初陶謙以笮融為下邳相使督廣陵下邳彭城糧運融遂斷三郡委輸以自入【斷讀曰短委於偽翻流所聚曰委毛晃曰凡以物送之曰輸則音平聲指所送之物曰輸則音去聲委輸之委亦音去聲】大起浮屠祠課人誦讀佛經招致旁郡好佛者至五千餘戶【好呼到翻】每浴佛【釋氏謂佛以四月八日生事佛者以是日為浴佛會】輒多設飲食布席於路經數十里費以鉅億計【鉅億計言以億億計也】及曹操擊破陶謙徐土不安融乃將男女萬口走廣陵【將即亮翻】廣陵太守趙昱待以賓禮先是彭城相薛禮為陶謙所逼屯秣陵融利廣陵資貨遂乘酒酣殺昱放兵大掠因過江依禮既而復殺之【先悉薦翻復扶又翻】劉繇使豫章太守朱皓攻袁術所用太守諸葛玄玄退保西城【西城在豫章南昌縣西 考異曰袁暐獻帝春秋云劉表上玄領豫章太守范書陶謙傳亦云劉表所用而陳志諸葛亮傳云術所用按許劭勸繇依表必不攻其所用也今從亮傳】及繇泝江西上駐於彭澤【彭澤縣屬豫章郡彭蠡澤在西上時掌翻】使融助皓攻玄許劭謂繇曰笮融出軍不顧名義者也朱文明喜推誠以信人【朱皓字文明喜許記翻】宜使密防之融到果詐殺皓代領郡事繇進討融融敗走入山為民所殺詔以前太傳掾華歆為豫章太守【掾于絹翻】丹陽都尉朱治逐吳郡太守許貢而據其郡貢南依山賊嚴白虎【嚴白虎有衆萬餘人阻山屯聚在吳郡之南】 張超在雍丘曹操圍之急超曰惟臧洪當來救吾【張超先為廣陵太守請臧洪為功曹委之以政】衆曰袁曹方睦洪為袁所表用【洪為超使劉虞路梗因寓於袁紹紹表為東郡太守治東武陽】必不敗好以招禍【敗補邁翻好呼到翻招音翹又如字召也】超曰子源天下義士【臧洪字子源】終不背本【背蒲妺翻】但恐見制強力【強力謂強有力也】不相及耳洪時為東郡太守徒跣號泣從紹請兵將赴其難【號戶刀翻難乃旦翻】紹不與請自率所領以行亦不許雍丘遂潰張超自殺操夷其三族洪由是怨紹絶不與通紹興兵圍之歷年不下紹令洪邑人陳琳以書喻之洪復書曰僕小人也本乏志用中因行役蒙主人傾蓋【家語孔子之郯遇程子於塗傾蓋而語】恩深分厚遂竊大州寧樂今日自還接刃乎【分扶問翻樂音洛】當受任之初自謂究竟大事共尊王室豈悟本州被侵郡將遘戹【郡將謂張超也將即亮翻】請師見拒辭行被拘使洪故君遂至淪滅區區微節無所獲申豈得復全交友之道重虧忠孝之名乎【復扶又翻重直用翻】斯所以忍悲揮戈收淚告絶行矣孔璋足下徼利於境外【陳琳字孔璋徼一遥翻】臧洪投命於君親吾子託身於盟主【盟主謂袁紹也】臧洪策名於長安【帝在長安】子謂余身死而名滅僕亦笑子生而無聞焉紹見洪書知無降意【降戶江翻】增兵急攻城中糧穀已盡外無彊救洪自度必不免【度徒洛翻】呼將吏士民謂曰袁氏無道所圖不軌且不救洪郡將洪於大義不得不死念諸君無事空與此禍【與讀曰豫】可先城未敗將妻子出【先悉薦翻將如字領也】皆垂泣曰明府與袁氏本無怨隙今為本朝郡將之故自致殘困吏民何忍當舍明府去也初尚掘鼠煮䈥角後無可復食者【舍讀曰捨復扶又翻下同】主簿啟内厨米三升請稍以為饘粥【杜預曰饘糜也之連翻】洪歎曰何能獨甘此邪使作薄糜徧班士衆又殺其愛妾以食將士【食讀曰似】將士咸流涕無能仰視者男女七八千人相枕而死【枕職任翻】莫有離叛者城陷生執洪紹大會諸將見洪謂曰臧洪何相負若此今日服未洪據地瞋目曰諸袁事漢四世五公【自袁安至袁隗四世安為司徒子敞為司空孫湯為司空曾孫逢為司空隗為太傳凡五公瞋昌眞翻】可謂受恩今王室衰弱無扶翼之意欲因際會希冀非望多殺忠良以立姦威洪親見呼張陳留為兄【張陳留為超兄邈也】則洪府君亦宜為弟同共戮力為國除害【為于偽翻下刃為欲為舉為同】奈何擁衆觀人屠滅洪惜力劣【劣弱也】不能推刃為天下報仇【公羊傳曰事君猶事父也父受誅子復讎推刃之道推吐雷翻】何謂服乎紹本愛洪意欲令屈服原之見洪辭切知終不為己用乃殺之洪邑人陳容少親慕洪時在紹坐【少詩照翻坐徂卧翻下同】起謂紹曰將軍舉大事欲為天下除暴而先誅忠義豈合天意臧洪發舉為郡將奈何殺之紹慙使人牽出謂曰汝非臧洪儔空復爾為【爾為猶如此也】容顧曰仁義豈有常蹈之則君子背之則小人【背蒲妹翻】今日寧與臧洪同日而死不與將軍同日而生也遂復見殺【復扶又翻】在坐無不歎息竊相謂曰如何一日殺二烈士 公孫瓚既殺劉虞【事見上卷初平四年】盡有幽州之地志氣益盛恃其才力不恤百姓記過忘善睚眦必報【睚下懈翻眦士懈翻】衣冠善士名在其右者必以法害之有材秀者必抑困使在窮苦之地或問其故瓚曰衣冠皆自以職分當貴不謝人惠【分扶問翻】故所寵愛類多商販庸兒與為兄弟或結婚姻所在侵暴百姓怨之劉虞從事漁陽鮮于輔等【姓譜鮮于子姓周武王封其子於朝鮮支子仲食采於于因以鮮于為氏】合率州兵欲共報仇以燕國閻柔素有恩信推為烏桓司馬【應劭漢官曰護烏桓校尉有司馬二人秩六百石燕於賢翻】柔招誘胡漢數萬人與瓚所置漁陽太守鄒丹戰于潞北【誘音酉潞縣屬漁陽郡】斬丹等四千餘級烏桓峭王亦率種人【峭七肖翻種章勇翻】及鮮卑七千餘騎隨輔南迎虞子和與袁紹將麴義合兵十萬共攻瓚破瓚於鮑丘【鮑丘水名水經註鮑丘水從塞外來南過漁陽縣東和縣破瓚處也又南過潞縣西賢曰鮑丘水又謂之潞水俗又謂之大榆河在今幽州漁陽縣】斬首二萬餘級於是代郡廣陽上谷右北平各殺瓚所置長吏復與鮮于輔劉和兵合瓚軍屢敗先是有童謡曰【長知兩翻復扶又翻先悉薦翻】燕南垂趙北際中央不合大如礪唯有此中可避世瓚自謂易地當之遂徙鎮易為圍塹十重於塹裏築京皆高五六丈為樓其上中塹為京特高十丈【水經註易京在易城西四五里易水逕其南賢曰前書易縣屬涿郡續漢志曰屬河間瓚所居易京故城在今幽州歸義縣南十八里爾雅曰絶高謂之京非人力謂之丘重直龍翻下同高居傲翻塹七艶翻】自居焉以鐵為門斥去左右【去羌呂翻】男人七歲以上不得入門專與姬妾居其文簿書記皆汲而上之【以繩索引之而上若汲水然上時掌翻】令婦人習為大聲使聞數百步【聞音問】以傳宣教令疎遠賓客無所親信【遠于願翻】謀臣猛將稍稍乖散自此之後希復攻戰【復扶又翻】或問其故瓚曰我昔驅畔胡於塞表【事見五十九卷靈帝中平五年】埽黄巾於孟津【事見上卷初平二年】當此之時謂天下指麾可定至於今日兵革方始觀此非我所决不如休兵力耕以救凶年兵法百樓不攻今吾諸營樓樐數十重【賢曰樐即櫓字見說文釋名曰櫓露也上無覆屋】積穀三百萬斛食盡此穀足以待天下之事矣 南單于於扶羅死弟呼厨泉立居于平陽【平陽縣屬河東郡】

  資治通鑑卷六十一  
    


 


 



 

 
  







 


  
  
 
 
 


  

 















	
	









































 
  



















 





 












  
  
  

 





