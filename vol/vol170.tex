\section{資治通鑑卷一百七十}
宋 司馬光 撰

胡三省 音註

陳紀四|{
	起彊圉大淵獻盡重光單閼凡五年}


臨海王|{
	諱伯宗字奉業小字藥王文帝嫡長子也}


光大元年春正月癸酉朔日有食之 尚書左僕射袁樞卒|{
	卒子恤翻}
乙亥大赦改元|{
	改元光大}
辛卯帝祀南郊壬辰齊上皇還鄴|{
	去年八月如晉陽今還}
己亥周主耕籍田|{
	籍又尺翻}
二月壬寅朔齊主加元服大赦 初高祖為梁相|{
	高祖殺王僧辨立梁敬帝遂相之因以受禪相息亮翻下同}
用劉師知為中書舍人師知涉學工文練習儀體|{
	儀體謂朝儀國體}
歷世祖朝雖位宦不遷而委任甚重|{
	朝直遥翻下同}
與揚州刺史安城王頊尚書僕射到仲舉同受遺詔輔政師知仲舉恒居禁中參决衆事|{
	恒戶登翻}
頊與左右三百人入居尚書省師知見頊地望權勢為朝野所屬心忌之|{
	屬之欲翻}
與尚書左丞王暹等謀出頊於外|{
	暹思亷翻}
衆猶豫未敢先發東宫通事舍人殷不佞素以名節自任|{
	按蕭子顯齊志東宫職未有通事舍人五代志梁東宫官有通事守舍人典法守舍人陳因之}
又受委東宮|{
	言在東宮為上所親委}
乃馳詣相府|{
	是時尚書省為相府}
矯敕謂頊曰今四方無事王可還東府經理州務|{
	州務謂揚州事務}
頊將出中記室毛喜|{
	五代志梁制蕃王國及庶姓有持節府有中録事中記室}
馳入見頊曰陳有天下日淺國禍繼臻|{
	謂八年之間國連有大喪}
中外危懼太后深惟至計|{
	惟思也}
令王入省共康庶績今日之言必非太后之意宗社之重願王三思|{
	三息暫翻又如字}
須更聞奏無使姧人得肆其謀今出外即受制於人譬如曹爽願作富家翁其可得邪|{
	曹爽事見四十五卷魏邵陵厲公嘉平元年}
頊遣喜與領軍將軍吳明徹籌之明徹曰嗣君諒闇|{
	闇音隂}
萬機多闕殿下親實周召當輔安社稷願留中勿疑頊乃稱疾召劉師知留之與語使毛喜先入言於太后太后曰今伯宗幼弱政事並委二郎|{
	文帝居長頊居次故稱為二郎}
此非我意喜又言於帝帝曰此自師知等所為朕不知也喜出以報頊頊因囚師知自入見太后及帝|{
	見賢遍翻}
極陳師知之罪仍自草敕請畫|{
	請畫可也}
以師知付廷尉其夜於獄中賜死以到仲舉為金紫光禄大夫王暹殷不佞並付治|{
	付治付有司治罪也或作付冶付東冶使徒作也以下文不害免官言之治字為是暹息亷翻}
不佞不害之弟也少有孝行|{
	不佞少居父喪以至孝稱江陵之陷不佞母死於亂兵不佞在吳道路隔絶久不得奔赴四年之中晝夜號泣居處飲食常為居喪之禮後其兄不齊迎母喪歸葬不佞居處之節如始聞問若此者又三年身自負土手植松柏每歲時伏臘必三日不食少詩照翻行下孟翻}
頊雅重之故獨得不死免官而已王暹伏誅自是國政盡歸於頊|{
	劉師知之事大類楊愔}
右衛將軍會稽韓子高鎮領軍府在建康諸將中士馬盛|{
	會工外翻將即亮翻}
與仲舉通謀事未發毛喜請簡士馬配子高并賜鐵炭使脩器甲頊驚曰子高謀反方欲收執何為更如是邪|{
	邪音耶}
喜曰山陵始畢邉宼尚多而子高受委前朝名為杖順若收之恐不即授首或能為人患宜推心安誘|{
	朝直遥翻誘音酉}
使不自疑伺間圖之一壯士之力耳頊深然之|{
	間古莧翻考異曰陳書文沈后傳云安成王既專沈太后憂悶計無所出乃密賂宦者蔣裕令誘建安人張安國使據郡反冀因此以圖高宗安國事覺並為高宗所誅時后左右近侍頗知其事后恐連逮黨與並殺之按后欲圖高宗而令安國據建安反理不相涉且后若寔有此謀高宗既立后豈得自全今删去}
仲舉既廢歸私第心不自安子郁尚世祖妹信義長公主|{
	信義郡名五代志吳郡常熟縣梁置信義郡長知两翻}
除南康内史未之官子高亦自危求出為衡廣諸鎮郁每乘小輿蒙婦人衣與子高謀會前上虞令陸昉及子高軍主告其謀反|{
	昉分罔翻}
頊在尚書省因召文武在位議立皇太子平旦仲舉子高入省皆執之|{
	到仲舉既廢歸私第非在位者盖頊召其會議因而執之}
并郁送廷尉下詔於獄賜死 |{
	考異曰陳書子高傳死在光大元年八月按華皎傳子高誅後皎始謀叛帝紀此年五月皎已謀反又慈訓太后令先言劉師知子高誅後乃及余孝頃始興王伯茂傳師知等誅後伯茂乃進號中衛然則子高傳誤也}
餘黨一無所問 辛亥南豫州刺史余孝頃坐謀反誅 癸丑以東揚州刺史始興王伯茂為中衛大將軍開府儀同三司|{
	梁置四中將軍軍衛橅護止施於内}
伯茂帝之母弟也劉師知韓子高之謀伯茂皆預之司徒頊恐扇動内外故以為中衛專使之居禁中與帝遊處|{
	處昌呂翻}
三月甲午以尚書右僕射沈欽為侍中左僕射|{
	史言沈欽官兼两省}
夏四月癸丑齊遣散騎常侍司馬幼之來聘|{
	散悉亶翻騎奇寄翻}
湘州刺史華皎|{
	華戶化翻}
聞韓子高死内不自安|{
	皎與劉師知韓子高皆為文帝所親任二人既死故皎不自安}
繕甲聚徒撫循所部啓求廣州以卜朝廷之意司徒頊偽許之而詔書未出皎遣使潜引周兵又自歸於梁以其子玄響為質|{
	使疏吏翻質音致}
五月癸巳頊以丹陽尹吳明徹為湘州刺史 甲午齊以東平王儼為尚書令 司徒頊遣吳明徹帥舟師三萬趣郢州|{
	帥讀曰率下同趣七喻翻下同}
丙申遣征南大將軍淳于量帥舟師五萬繼之又遣冠武將軍楊文通從安成步道出茶陵|{
	梁置冠武將軍與折衝同班五代志廬陵郡安復縣舊置安成郡茶陵縣漢屬長沙郡吳分屬湘東郡隋并入衡山郡湘潭縣九域志茶陵縣屬衡州在州東三百五十五里}
巴山太守黄法慧從宜陽出澧陵|{
	宜陽即豫章郡宜春縣也晉孝武帝更名宜陽避太后諱也隋復曰宜春縣帶袁州後漢立醴陵縣屬長沙郡九域志在郡東一百六十里自宜春至醴陵二百二十里守式又翻}
共襲華皎并與江州刺史章昭逹郢州刺史程靈洗合謀進討六月壬寅以司空徐度為車騎將軍摠督建康諸軍步道趣湘州 辛亥周主尊其母叱奴氏為皇太后|{
	按魏收官氏志拓跋興於代北兼并他部以本部中别族為内姓其他諸部隨方分之北方有叱奴氏}
己未齊封皇弟仁機為西河王仁約為樂浪王|{
	樂音洛下同浪音琅}
仁儉為潁川王仁雅為安樂王仁直為丹陽王|{
	考異曰北齊書帝紀名統今從列傳統謂仁直}
仁謙為東海王|{
	皆郡王也五代志安樂郡密雲縣舊置安樂郡}
華皎使者至長安梁王亦上書言狀且乞師|{
	華戶化翻使疏吏翻上時掌翻}
周人議出師應之司會崔猷曰前歲東征死傷過半|{
	謂攻齊洛陽也事見上卷文帝天嘉五年會古外翻}
比雖循撫瘡痍未復今陳氏保境息民共敦鄰好|{
	比毗至翻好呼到翻}
豈可利其土地納其叛臣違盟約之信興無名之師乎晉公護不從閠六月戊寅遣襄州摠管衛公直督柱國陸通大將軍田弘權景宣元定等將兵助之|{
	將即亮翻又音如字領也}
辛巳齊左丞相咸陽武王斛律金卒年八十長子光為大將軍|{
	相息亮翻卒子恤翻長知两翻}
次子羨及孫武都並開府儀同三司出鎭方岳|{
	斛律羨鎮幽州武都鎮梁兖二州}
其餘子孫封侯顯貴者甚衆門中一皇后二太子妃|{
	金子光長女孝昭納為太子妃次女武成納為太子妃後主受内禪立為皇后}
三公主|{
	按後祖珽言光男尚公主蓋光子武都世雄恒伽皆尚主也}
事齊貴寵三世無比自肅宗以來禮敬尤重每朝見常聼乘步挽車至階|{
	朝直遥翻見賢遍翻步挽車不用牛馬令人步挽之}
或以羊車迎之然金不以為喜嘗謂光曰我雖不讀書聞古來外戚鮮有能保其族者|{
	鮮息淺翻}
女若有寵為諸貴所嫉無寵為天子所憎我家直以勛勞致富貴何必藉女寵也|{
	史言斛律金有識}
壬午齊以東平王儼録尚書事以左僕射趙彦深為尚書令并省尚書左僕射婁定遠為左僕射|{
	自并省入為鄴省左僕射}
中書監徐之才為右僕射定遠昭之子也|{
	昭婁后之弟}
秋七月戊申立皇子至澤為太子 八月齊以任城王湝為太師|{
	任音壬湝音皆又戶皆翻}
馮翊王潤為大司馬段韶為左丞相賀抜仁為右丞相侯莫陳相為太宰婁叡為太傅斛律光為太保韓祖念為大將軍趙郡王叡為太尉東平王為司徒儼有寵於上皇及胡后峕兼京畿大都督領軍大將軍領御史中丞魏朝故事中丞出與皇太子分路|{
	分路而行不引車避道朝直遥翻}
王公皆遥駐車去牛頓軛於地以待其過|{
	去羌呂翻軛於革翻}
其或遲違|{
	不即駐車頓軛是遲遲為違法}
則前驅以赤棒棒之|{
	棒部項翻}
自遷鄴以後此儀廢絶上皇欲尊寵儼命一遵舊制儼初從北宫出將上中丞|{
	將上時掌翻今人謂領職視事為禮上}
凡京畿步騎領軍官屬中丞威儀司徒鹵簿莫不畢從|{
	騎奇寄翻從才用翻}
上皇與胡后張幕於華林園東門外而觀之遣中使驟馬趣仗|{
	趣前導儀仗也使疏吏翻趣七喻翻}
不得入自言奉敕赤棒卒應聲碎其鞍馬驚人墜上皇大笑以為善更敕駐車勞問良久|{
	勞問也勞力到翻}
觀者傾鄴城恒在宫中坐含光殿視事|{
	恒戶登翻下同}
諸父皆拜之上皇或時如并州|{
	晉陽宫在并州}
恒居守|{
	恒戶登翻守式又翻}
每送行或半路或至晉陽乃還|{
	還從宣翻又音如字}
器玩服飾皆與齊主同所須悉官給嘗於南宫見新氷早李|{
	齊主時居鄴之南宫從上皇胡后居北宫}
還怒曰尊兄已有我何意無|{
	嘗謂齊主為尊兄}
自是齊主或先得新奇屬官及工人必獲罪儼性剛决嘗言於上皇曰尊兄懦何能帥左右上皇每稱其才有廢立意胡后亦勸之既而中止|{
	儼與齊主既定君臣之分而常以兄弟相呼又有奪嫡之意史歷言之為怙寵致禍張本帥讀曰率}
華皎遣使誘章昭逹昭逹執送建康又誘程靈洗靈

洗斬之|{
	華戶化翻使疏吏翻誘音酉下並同}
皎以武州居其心腹|{
	五代志武陵郡梁置武州}
遣使誘都督陸子隆子隆不從遣兵攻之不克巴州刺史戴僧朔等並隸於皎|{
	文帝命皎都督湘巴等四州五代志巴陵郡梁置巴州}
長沙太守曹慶等本隸皎下遂為之用|{
	湘州與長沙郡同治所以州統郡故曰本隸皎下守手又翻}
司徒頊恐上流守宰皆附之乃曲赦湘巴二州九月乙巳悉誅皎家屬梁以皎為司空遣其柱國王操將兵二萬助之周權景宣將水軍元定將陸軍衛公直摠之與皎俱下|{
	將即亮翻又音如字領也}
淳于量軍夏口直軍魯山使元定以步騎數千圍郢州|{
	騎奇寄翻 考異曰陳帝紀云步騎二萬蓋夸誕之辭今從周帝紀}
皎軍於白螺|{
	水經江水過長沙下雋縣北湘水從南來注之江水又東過彭城口又東過如山北又東過白螺山南螺盧戈翻雋辭兖翻}
與吳明徹等相持徐度楊文通由嶺路襲湘州|{
	嶺路即前所出安成宜陽步道也}
盡獲其所留軍士家屬皎自巴陵與周梁水軍順流乘風而下軍勢甚盛戰於沌口|{
	沌柱兖翻}
量明徹募軍中小艦多賞金銀令先出當西軍大艦受其拍西軍諸艦拍皆盡然後量等以大艦拍之西軍艦皆碎沒於中流|{
	戰船置拍竿之以拍敵船艦戶黯翻}
西軍又以艦載薪因風縱火俄而風轉自焚西軍大敗皎與戴僧朔單舸走|{
	舸古我翻}
過巴陵不敢岸|{
	恐當作登}
逕奔江陵衛公直亦奔江陵元定孤軍進退無路斫竹開徑且戰且引欲趣巴陵|{
	趣七喻翻}
巴陵巳為徐度等所據度等遣使偽與結盟許縱之還國定信之解仗就度度執之盡俘其衆 |{
	考異曰陳書云獲萬餘人馬四千匹亦恐夸誕今不取}
并擒梁大將軍李廣定憤恚而卒|{
	恚於避翻卒子恤翻}
皎黨曹慶等四十餘人並伏誅唯以岳陽太守章昭裕昭逹之弟桂陽太守曹宣高祖舊臣衡陽内史汝隂任忠嘗有密啓皆宥之吳明徹乘勝攻梁河東拔之|{
	守式又翻五代志巴陵郡湘隂縣梁置岳陽郡桂陽郡郴縣梁置桂陽郡長沙郡衡山縣舊置衡陽郡陳以衡陽為王國故置内史南郡松滋縣舊置河東郡任音壬}
周衛公直歸罪於梁柱國殷亮梁主知非其罪然不敢違遂誅之周與陳既交惡周沔州刺史裴寛白襄州摠管請益戍兵并遷城於羊蹄山以避水|{
	五代志沔陽郡甑山縣梁置梁安郡西魏改曰魏安郡置江州廢帝三年改曰沔州甑山有陽臺山在漢川之南三十五里土俗訛為羊蹄山}
摠管兵未至程靈洗舟師奄至城下會大雨水暴漲靈洗引大艦臨城發拍撃樓堞皆碎|{
	堞徒協翻}
矢石晝夜攻之三十餘日陳人登城寛猶帥衆執短兵拒戰|{
	帥讀曰率}
又二日乃擒之 丁巳齊上皇如晉陽山東水饑|{
	按李百藥書山東大水人饑}
僵尸滿道|{
	僵居良翻}
冬十月甲申帝享太廟 十一月戊戌朔日有食之 丙午齊大赦 癸丑周許穆公宇文貴自突厥還卒於張掖|{
	宇文貴與陳公純等如突厥逆女突厥留之貴以疾先得還厥九勿翻還從宣翻又音如字卒子恤翻掖音亦}
齊上皇還鄴 十二月周晉公護母卒|{
	卒子恤翻}
詔起令視事|{
	令力丁翻}
齊秘書監祖珽與黄門侍郎劉逖友善珽欲求宰相乃疏趙彦深元文遥和士開罪狀令逖奏之逖不敢通彦深等聞之先詣上皇自陳上皇大怒執珽詰之|{
	珽它鼎翻詰去吉翻}
珽因陳士開文遥彦深等朋黨弄權賣官鬻獄事上皇曰爾乃誹謗我珽曰臣不敢誹謗陛下取人女上皇曰我以其饑饉收養之耳珽曰何不開倉振給乃買入後宫乎上皇益怒以刀環築其口鞭杖亂下將撲殺之|{
	撲弼角翻}
珽呼曰陛下勿殺臣臣為陛下合金丹|{
	呼火故翻為于偽翻合音閤}
遂得少寛|{
	少詩沼翻}
珽曰陛下有一范增不能用|{
	漢高帝之言}
上皇又怒曰爾自比范增以我為項羽邪|{
	邪音耶}
珽曰項羽布衣帥烏合之衆五年而成覇業|{
	帥讀曰率}
陛下藉父兄之資纔得至此臣以為項羽未易可輕上皇愈怒令以土塞其口|{
	易以豉翻塞悉則翻}
珽且吐且言乃鞭二百配甲坊尋徙光州|{
	五代志東萊郡舊置光州}
敕令牢掌别駕張奉福曰牢者地牢也乃置地牢中桎梏不離身|{
	離力智翻}
夜以蕪菁子為燭眼為所熏由是失明|{
	本草曰蕪菁主明目今珽由是失明盖其子餌之則明目以之為燭則烟熏眼而失明衍義曰蕪菁今世俗謂之蔓菁夏則枯蔬圃復種之謂之鷄毛菜正在春時採擷之餘收子為油審是則菜油也東南之人多以之照夜未嘗熏眼失明}
齊七兵尚書畢義雲|{
	杜佑曰魏始置五兵尚書謂中兵外兵别兵都兵騎兵也晉分中外各為左右雖與舊為七曹唯有五兵尚書無七兵尚書之名至後魏始有七兵尚書今諸家著述或謂晉太康中置七兵尚書誤矣}
為治酷忍非人理所及|{
	治直吏翻}
於家尤甚夜為盗所殺遺其刀驗之其子善昭所佩刀也有司執善昭誅之|{
	史書此以垂戒然以情觀之善昭果弑其父必不遺刀以待驗盖盗為此計以殺其子}


二年春正月己亥安成王頊進位太傅領司徒加殊禮辛丑周主祀南郊 癸亥齊主使兼散騎常侍鄭大

護來聘 湘東忠肅公徐度卒|{
	卒子恤翻}
二月丁卯周主如武功 突厥木杆可汗貮於周|{
	厥九勿翻杆公旦翻可從刋入聲汗音寒}
更許齊人以昏留陳公純等數年不返|{
	純等逆女見上卷文帝天嘉六年}
會大雷風壞其穹廬|{
	壞音怪}
旬日不止木杆懼以為天譴即備禮送其女於周純等奉之以歸三月癸卯至長安周主行親迎之禮|{
	迎魚敬翻古者天子娶於諸侯使同姓諸侯為之主桓八年祭公來遂逆王后於紀杜預注云祭公來受命於魯是也周主行親迎與突厥為敵國之禮}
甲辰周大赦乙巳齊以東平王為大將軍南陽王綽為司徒開

府儀同三司徐顯秀為司空廣寧王孝珩為尚書令|{
	珩音行}
戊午周燕文公于謹卒|{
	燕因肩翻卒子恤翻}
謹勲高位重而事上益恭每朝參所從不過二三騎朝廷有大事多與謹謀之|{
	朝直遥翻騎奇寄翻}
謹盡忠補益於功臣中特被親信禮遇隆重始終無閒|{
	被皮義翻閒古莧翻}
教訓諸子務存静退而子孫蕃衍|{
	蕃音煩}
率皆顯逹 吳明徹乘勝進攻江陵|{
	乘沌口之勝也}
引水灌之梁主出頓紀南以避之|{
	劉昭曰江陵縣北十餘里有紀南城}
周摠管田弘從梁主副摠管高琳與梁僕射王操守江陵三城晝夜拒戰十旬梁將馬武吉徹擊明徹敗之|{
	將則亮翻敗補邁翻}
明徹退保公安梁主乃得還 夏四月辛巳周以逹奚武為太傅尉遲迥為太保齊公憲為大司馬齊上皇如晉陽 齊尚書左僕射徐之才善醫上皇有疾之才療之既愈中書監和士開欲得次遷乃出之才為兖州刺史|{
	為齊主疾作追之才不及張本五代志魯郡舊兖州治瑕丘}
五月癸卯以尚書右僕射胡長仁為左僕射士開為右僕射長仁太上皇后之兄也 庚戍周主享太廟庚申如醴泉宫|{
	醴泉宫即漢甘泉宫之舊地在漢馮翊池陽縣西後魏於此置寧夷縣隋改曰醴泉縣}
壬戍齊上皇還鄴|{
	還從宣翻又音如字}
秋七月壬寅周隨桓公楊忠卒|{
	卒子恤翻}
子堅襲爵堅為開府儀同三司小宫伯|{
	周禮宫伯屬天官中士二人下士二人鄭玄注云伯長也掌王宫宿衛之官及其政令行其秩叙作其徒役之事後周置左右宫伯掌侍衛之禁各更直於内小宫伯貳之}
晉公護欲引以為腹心堅以白忠忠曰两姑之間難為婦汝其勿往堅乃辭之|{
	史以楊忠有識因書其卒而書之楊堅始見于此}
丙午帝享太廟 戊午周主還長安壬戍封皇弟伯智為永陽王伯謀為桂陽王 八月齊請和於周周遣軍司馬陸程聘於齊九月丙申齊使侍中斛斯文略報之 冬十月癸亥周主享太廟 庚午帝享太廟 辛巳齊以廣寧王孝珩録尚書事左僕射胡長仁為尚書令右僕射和士開為左僕射中書監唐邕為右僕射 十一月壬辰朔日有食之 齊遣兼散騎常侍李諧來聘|{
	散悉亶翻騎奇寄翻}
甲辰周主如岐陽|{
	五代志太原郡雍縣有岐陽宫}
周遣開府儀同三司崔彦等聘於齊 始興王伯茂以安成王頊專政意甚不平屢肆惡言|{
	頊吁玉翻}
甲寅以太皇太后令誣帝云與劉師知華皎等通謀|{
	言以者明太皇太后令頊為之也華戶化翻}
且曰文皇知子之鑒事等帝堯傳弟之懷又符太伯今可還申曩志崇立賢君遂廢帝為臨海王以安成王入纂又下令黜伯茂為温麻侯|{
	温麻縣侯也沈約曰晉武帝以温麻船屯立縣屬晉安郡晉安隋改為建安}
寘諸别館安成王使盗邀之於道殺之車中 齊上皇疾作驛追徐之才未至辛未疾亟以後事屬和士開|{
	屬之欲翻}
握其手曰勿負我也遂殂於士開之手|{
	殂祚乎翻年三十二}
明日之才至復遣還州|{
	還兖州也復扶又翻}
士開秘喪三日不黄門侍郎馮子琮問其故士開曰神武文襄之喪皆秘不發|{
	神武卒見一百六十卷梁武帝太清元年文襄卒見一百六十二卷太清三年}
今至尊年少|{
	少詩沼翻}
恐王公有貮心者意欲盡追集於凉風堂然後與公議之士開素忌太尉録尚書事趙郡王叡及領軍婁定遠子琮恐其矯遺詔出叡於外奪定遠禁兵乃說之曰|{
	說式芮翻}
大行先已傳位於今上羣臣富貴者皆至尊父子之恩但令在内貴臣一無改易王公必無異志|{
	令力丁翻}
世異事殊豈得與覇朝相比|{
	高歡高澄未即簒魏握魏之政北齊君臣皆謂之覇朝朝直遥翻下同}
且公不出宫門已數日升遐之事|{
	鄭玄曰升上也避己也上己者若仙去云耳}
行路皆傳久而不舉|{
	謂不舉哀成服也}
恐有他變士開乃喪丙子大赦戊寅尊太上皇后為皇太后侍中尚書左僕射元文遥以馮子琮胡太后之妹夫恐其贊太后干預朝政|{
	朝直遥翻}
與趙郡王叡和士開謀出子琮為鄭州刺史世祖驕奢淫泆役繁賦重吏民苦之|{
	史言亡齊者武成}
甲申詔所在百工細作悉罷之鄴下晉陽中山宫人官口之老病者悉簡放|{
	齊有鄴宫晉陽宫中山宫官口罪人家口沒官為奴婢者}
諸家緣坐在流所者聽還|{
	緣坐謂罪非正犯緣親戚而坐罪者}
周梁州恒稜獠叛|{
	據趙文表傳恒稜地名所在險固方數百里羣獠居之文表既平獠遂置為蓬州恒戶登翻獠魯皓翻下同}
摠管長史南鄭趙文表討之|{
	趙文表為梁州摠管府長史長知两翻}
諸將欲四面進攻|{
	將即亮翻}
文表曰四面攻之獠無生路必盡死以拒我未易可克今吾示以威恩為惡者誅之從善者撫之善惡既分破之易矣|{
	易以豉翻}
遂以此意遍令軍中時有從軍熟獠多與恒稜親識即以實報之|{
	境上内附者謂之熟獠}
恒稜猶豫未决文表軍已至其境獠中先有二路一平一險有獠帥數人來見|{
	帥所類翻}
請為鄉導|{
	鄉讀曰嚮}
文表曰此路寛平不須為導卿但先行慰諭子弟使來降也|{
	降戶江翻下同}
乃遣之文表謂諸將曰獠帥謂吾從寛路而進必設伏以邀我當更出其不意乃引兵自險路入乘高而望果有伏兵獠既失計争帥衆來降|{
	降戶江翻}
文表皆慰撫之仍徵其租税無敢違者周人以文表為蓬州長史|{
	州本漢宕渠之地李勢時為獠所據蕭齊立歸化郡梁置安固縣及伏虞郡後周置蓬州因蓬山而以為名也}


高宗宣皇帝上之上|{
	諱頊字紹世小字師利始興王道譚第二子也}


大建元年春正月辛卯朔周主以齊世祖之喪罷朝會|{
	朝直遥翻}
遣司會李綸弔賻|{
	賻音附公羊傳曰貨財曰賻會古外翻}
且會葬甲午安成王即皇帝位改元大赦復太皇太后為皇太后皇太后為文皇后立妃柳氏為皇后世子叔寶為太子封皇子叔陵為始興王奉昭烈王祀|{
	文帝以子伯茂奉始興昭烈王祀帝既殺伯茂以叔陵奉祀}
乙未上謁太廟丁酉以尚書僕射沈欽為左僕射度支尚書王勱為右僕射勱份之孫也|{
	度徒洛翻勱音邁份府巾翻份奐之弟肅之叔父也}
辛丑上祀南郊 壬寅封皇子叔英為豫章王叔堅為長沙王 戊午上享太廟 齊博陵文簡王濟世祖之母弟也為定州刺史語人曰次叙當至我矣|{
	言以兄弟之次亦當為天子也語牛翻}
齊主聞之隂使人就州殺之葬贈如禮 二月乙亥上耕籍田|{
	籍秦昔翻}
甲申齊葬武成帝於永平陵廟號世祖 乙丑齊徙東平王儼為琅邪王|{
	邪音耶}
齊遣侍中叱列長乂聘於周|{
	姓纂叱列複姓出於拓跋氏西部}
齊以司空徐顯秀為太尉并省尚書令婁定遠為司空初侍中尚書右僕射和士開為世祖所親狎出入臥内無復期度|{
	復扶又翻}
遂得幸於胡后及世祖殂齊主以士開受顧託深委任之威權益盛與婁定遠及録尚書事趙彦深侍中尚書左僕射元文遥開府儀同三司唐邕領軍綦連猛高阿那肱|{
	李延夀曰綦連其先姬姓六國末避亂出塞保祈連山因以山為姓北人語訛故曰綦連魏收官氏志綦連氏出於西方諸部}
度支尚書胡長粲俱用事時號八貴|{
	樂音洛}
太尉趙郡王叡大司馬馮翊王潤安德王延宗與婁定遠元文遥皆言於齊主請出士開為外任會胡太后觴朝貴於前殿|{
	朝直遥翻}
叡面陳士開辠失云士開先帝弄臣城社鼠受納貨賂穢亂宫掖|{
	掖音亦}
臣等義無杜口冒死陳之太后曰先帝在時王等何不言今日欲欺孤寡邪|{
	邪音耶}
且飲酒勿多言叡等辭色愈厲儀同三司安吐根曰臣本商胡|{
	安吐根本安息胡人天平初柔然主使至晉陽吐根密格柔然情狀高歡因為之備柔然入掠無獲而返其後歡與柔然和親結成婚媾皆吐根為行人既而歸歡由是見親待}
得在諸貴行末|{
	行戶剛翻}
既受厚恩豈敢惜死不出士開朝野不定太后曰異日論之王等且散叡等或投冠於地或拂衣而起明日叡等復詣雲龍門|{
	復扶又翻}
令文遥入奏之三返太后不聽左丞相段韶使胡長粲傳太后言曰梓宫在殯事太怱怱欲王等更思之叡等遂皆拜謝長粲復命太后曰成妹母子家者兄之力也|{
	長粲胡太后之兄故云然}
厚賜叡等罷之太后及齊主召問士開對曰先帝於羣臣之中待臣最厚陛下諒闇始爾|{
	闇音隂}
大臣皆有覬覦|{
	覬音冀覦音俞}
今若出臣正是翦陛下羽翼宜謂叡云文遥與臣俱受先帝任用豈可一去一留並可用為州且出納如舊|{
	尚書出納帝命令且如舊領職}
待過山陵然後遣之叡等謂臣眞出心必喜之|{
	喜許記翻}
帝及太后然之告叡等如其言乃以士開為兖州刺史文遥為西兖州刺史|{
	西兖州時治滑臺}
葬畢叡等促士開就路太后欲留士開過百日|{
	古者葬日虞既三虞用剛日卒哭後人百日而卒哭至今猶然}
叡不許數日之内太后數以為言|{
	后數所角翻}
有中人知太后密旨者謂叡曰太后意既如此殿下何宜苦違叡曰吾受委不輕今嗣主幼冲豈可使邪臣在側不守之以死何面戴天遂更見太后苦言之太后令酌酒賜叡叡正色曰今論國家大事非為巵酒|{
	為于偽翻}
言訖遽出士開載美女珠簾詣婁定遠謝曰諸貴欲殺士開蒙王力|{
	武成帝封婁定遠臨淮郡王故稱之}
特全其命用為方伯今當奉别謹上二女子一珠簾|{
	上時掌翻}
定遠喜謂士開曰欲還入不|{
	不讀曰否}
士開曰在内久不自安今得出實遂本志不願更入但乞王保護長為大州刺史足矣定遠信之送至門士開曰今當遠出願得一辭覲二宫定遠許之士開由是得見太后及帝進說曰|{
	見賢遍翻說式芮翻}
先帝一旦登遐臣愧不能自死觀朝貴意勢欲以陛下為乾明|{
	乾明齊濟南王年號也事見一百六十八卷文帝天嘉元年}
臣出之後必有大變臣何面目見先帝於地下因慟哭帝太后皆泣問計安出士開曰臣已得入復何所慮正須數行詔書耳|{
	復扶又翻下將復后復復以同行戶剛翻}
於是詔出定遠為青州刺史責趙郡王叡以不臣之罪旦日叡將復入諫妻子咸止之叡曰社稷事重吾寧死事先皇不忍見朝廷顛沛至殿門又有人謂曰殿下勿入恐有變叡曰吾上不負天死亦無恨入見太后|{
	見賢遍翻}
太后復以為言叡執之彌固出至永巷遇兵執送華林園雀離佛院|{
	釋氏西域記龜兹國北四十里山上有寺名曰雀離大清浄故倣以建佛院}
令劉桃枝拉殺之|{
	拉盧合翻 考異曰北齊帝紀天統三年六月以并省尚書左僕射婁定遠為尚書左僕射五年二月殺趙郡王叡三月以并省尚書令婁定遠為司空蓋定遠既為僕射復為并省尚書令也按和士開傳先出定遠然後殺叡叡死必在定遠作司空後帝紀誤也但不知果在何時耳又士開傳云出為青州定遠傳云尋除瀛州盖先出為青州後乃除瀛州也}
叡久典朝政|{
	文宣帝時濟南以太子監國立大都督府與尚書省分理衆事以叡攝大都督府長史至武成之世拜尚書令又進攝録尚書事又進太尉朝直遥翻下同}
清正自守朝野寃惜之復以士開為侍中尚書左僕射定遠歸士開所遺|{
	遺于季翻}
加以餘珍賂之 三月齊主如晉陽夏四月甲子以并州尚書省為大基聖寺晉祠為大崇皇寺|{
	魏收志太原郡晉陽縣有晉祠}
乙丑齊主還鄴 齊主年少多嬖寵|{
	少詩沼翻}
武衛將軍高阿那肱素以謟佞為世祖及和士開所厚世祖多令在東宫侍齊主由是有寵累遷并省尚書令封淮隂王世祖簡都督二十人使侍衛東宫|{
	齊左右衛府領左右府其御仗屬官各有正副都督}
昌黎韓長鸞預焉齊主獨親愛長鸞長鸞名鳳以字行累遷侍中領軍摠知内省機密宫婢陸令萱者其夫漢陽駱超坐謀叛誅令萱配掖庭子提婆亦沒為奴齊主之在襁褓令萱保養之令萱巧黠善取媚|{
	黠戶入翻}
有寵於胡太后宫掖之中獨擅威福封為郡君和士開高阿那肱皆為之養子齊主以令萱為女侍中|{
	後魏孝文改定内官左右昭儀三夫人九嬪世婦御女之外又置内職典内司視尚書令僕作司大監女侍中三官視二品監}
令萱引提婆入侍齊主朝夕戱狎累遷至開府儀同三司武衛大將軍宫人穆舍利者斛律后之從婢也|{
	從于用翻}
有寵於齊主令萱欲附之乃為之養母薦為弘德夫人|{
	河清新令弘德崇德正德三夫人位比三公}
因令提婆冒姓穆氏然和士開用事最久諸幸臣皆依附之以固其寵齊主思祖珽|{
	齊主受内禪祖珽啓其謀故思之}
就流囚中除海州刺史|{
	祖珽去年徙光州五代志東海郡梁置南北二青州東魏改為海州}
珽乃遺陸媪弟儀同三司悉逹書曰趙彦深心腹深沈欲行伊霍事|{
	遺于季翻媪烏老翻陸媪即謂令萱沈持林翻}
儀同姊弟豈得平安何不早用智士邪和士開亦以珽有膽略欲引為謀主乃棄舊怨虛心待之與陸媪言於帝曰襄宣昭三帝之子皆不得立|{
	謂文襄文宣孝昭之子也}
今至尊獨在帝位者祖孝徵之力也|{
	祖珽字孝徵事見上卷文帝天嘉元年}
人有功不可不報孝徵心行雖薄|{
	行下孟翻}
奇略出人緩急可使且其人已盲必無反心請呼取問以籌筴齊主從之召入為秘書監加開府儀同三司士開譛尚書令隴東主胡長仁驕恣出為齊州刺史長仁怨憤謀遣刺客殺士開事覺士開與珽謀之珽引漢文帝誅薄昭故事|{
	薄昭事見十四卷漢文帝十年胡長仁帝舅也故引此事以誅之}
遂遣使就州賜死|{
	使疏吏翻}
五月庚戍周主如醴泉宫 丁巳以吏部尚書徐陵為左僕射 秋七月辛卯皇太子納妃沈氏 |{
	考異曰陳書南史沈后傳皆云太建三年拜太子妃誤也今從帝紀}
吏部尚書君理之女也 辛亥周主還長安 八月庚辰盗殺周孔城防主以其地入齊|{
	魏收志漢晉河南新城縣後魏置新城郡治孔城其地在隋河南郡伊關縣界}
九月辛卯周遣齊公憲與柱國李穆將兵趣宜陽|{
	杜佑曰宜陽郡今福昌縣}
築崇德等五城|{
	趣七喻翻}
歐陽紇在廣州十餘年|{
	武帝永定二年紇與父頠定廣州至是凡十二年紇下沒翻}
威惠著於百越自華皎之叛帝心疑之|{
	華戶化翻}
徵為左衛將軍紇恐懼其下多勸之反遂舉兵攻衡州刺史錢道戢|{
	此始興之衡州五代志南海郡始興縣梁置安遠郡及東衡州}
帝遣中書侍郎徐儉持節諭旨紇初見儉盛仗衛言辭不恭儉曰呂嘉之事誠當已遠|{
	呂嘉事見二十卷漢武帝元鼎五年六年}
將軍獨不見周廸陳寶應乎|{
	周廸陳寶應皆以叛換敗死事並見文帝紀}
轉禍為福未為晚也紇默然不應置儉於孤園寺累旬不得還紇嘗出見儉儉謂之曰將軍業已舉事儉須還報天子儉之性命雖在將軍將軍成敗不在於儉幸不見留紇乃遣儉還儉陵之子也|{
	徐陵時貴顯於陳朝}
冬十月辛未詔車騎將軍章昭逹討紇 壬午上享太廟十一月辛亥周鄫文公長孫儉卒|{
	鄫兹陵翻長知两翻卒子恤翻}


辛丑齊以斛律光為太傅馮翊王潤為太保琅邪王儼為大司馬十二月庚午以蘭陵王長恭為尚書令庚辰以中書監魏收為左僕射 周齊公憲等圍齊宜陽絶其糧道 自華皎之亂與周人絶至是周遣御正大夫杜杲來聘請復脩舊好|{
	杜杲來聘于陳故使來修舊好復扶又翻好呼到翻}
上許之遣使如周|{
	使疏吏翻}


二年春正月乙酉朔齊改元武平 齊東安王婁叡卒|{
	東安郡王五代志琅邪郡沂水縣舊置東安郡也}
丙午上享太廟 戊申齊使兼散騎常侍裴讞之來聘|{
	散悉亶翻騎奇寄翻讞魚蹇翻又魚列翻}
齊太傅斛律光將步騎三萬救宜陽屢破周軍築統關豐化二城而還|{
	還從宣翻又如字}
周軍追之光縱擊又破之獲其開府儀同三司宇文英梁景興|{
	周軍雖屢敗而攻宜陽不輟}
二月己巳齊以斛律光為右丞相并州刺史又以任城王湝為太師賀拔仁録尚書事|{
	任音壬湝音皆又古皆翻}
歐陽紇召陽春太守馮僕至南海|{
	五代志高涼郡陽春縣梁置陽春郡廣州治南海郡守式又翻}
誘與同反僕遣使告其母洗夫人夫人曰我為忠貞經今两世|{
	僕及父融為兩世誘音酉使疏吏翻洗息典翻}
不能惜汝負國遂兵拒境帥諸酋長迎章昭逹|{
	帥讀曰率酋慈秋翻長知两翻}
昭逹倍道兼行至始興紇聞昭逹奄至恇擾不知所為出頓洭口|{
	水經洭水出桂陽縣盧聚東南過含洭縣南出洭浦關右合溱水謂之洭口洭去王翻}
多聚沙石盛以竹籠|{
	盛時征翻}
置於水柵之外用遏舟艦昭逹居上流裝艦造拍|{
	艦戶黯翻下同}
令軍人銜刀潜行水中以斫籠篾皆解|{
	篾莫結翻}
因縱大艦隨流突之紇衆大敗生禽紇送之癸未斬於建康市紇之反也士人流寓在嶺南者皆惶駭前著作佐郎蕭引獨恬然曰管幼安袁曜卿亦但安坐耳|{
	管寧字幼安依公孫度度安其賢魏文帝初卒還鄉里袁渙字曜卿為呂布所拘而不為布所脅布敗歸魏武}
君子直已以行義何憂懼乎紇平上徵為金部侍郎|{
	唐六典曰漢置尚書郎四人其一人主財帛委輸盖金部郎曹之任也歷魏晉宋齊後魏北齊並有金部郎中梁陳隋為侍郎煬帝但曰郎}
引允之弟也|{
	蕭允是一百六十二卷梁武帝太清三年史言蕭允兄弟處大難而不懾}
馮僕以其母功封信都侯遷石龍太守|{
	五代志石龍縣屬高凉郡盖梁陳置郡也今為化州}
遣使持節冊命洗氏為石龍太夫人賜繡幰油絡駟馬安車一乘|{
	幰許偃翻安車加綉幰油絡也乘䋲證翻}
給鼓吹一部并麾幢旌節|{
	吹尺瑞翻幢傳江翻}
其鹵簿一如刺史之儀 三月丙申皇太后章氏殂 戊戍齊安定武王賀拔仁卒丁未大赦 夏四月甲寅周以柱國宇文盛為大宗

伯 周主如醴泉宮 辛酉齊以開府儀同三司徐之才為尚書左僕射 戊寅葬武宣皇后於萬安陵|{
	章太后謚武宣}
閠月戊申上謁太廟 五月壬午齊遣使來弔|{
	使疏故翻}
六月乙酉齊以廣寧王孝珩為司空|{
	珩音行}
甲辰齊穆夫人生子恒|{
	恒戶登翻}
齊主時未有男為之大赦|{
	為于偽翻下請為為之同}
陸令萱欲以恒為太子恐斛律后恨怒乃白齊主使斛律后母養之 己丑齊以開府儀同三司唐邕為尚書右僕射 秋七月齊立肅宗子彦基為城陽王彥忠為梁郡王甲寅以尚書令蘭陵王長恭為録尚書事中領軍和士開為尚書令賜爵淮陽王士開威權日盛朝士不知亷恥者或為之假子與富商大賈同在伯仲之列|{
	朝直遥翻賈音古}
嘗有一人士參士開疾|{
	參者候問其疾}
值毉云王傷寒極重應服黄龍湯|{
	陶弘景曰令近城寺别塞空罌口内糞倉中久年得汁甚黑而苦名為黄龍湯治温病垂死者皆差}
士開有難色人士曰此物甚易服|{
	易以豉翻}
王不須疑請為王先嘗之一舉而盡士開感其意為之強服|{
	強其两翻}
遂得愈 乙卯周主還長安 癸酉齊以華山王凝為太傅|{
	華山郡王五代志京兆郡鄭縣後魏置華山郡華戶化翻}
司空章昭逹攻梁|{
	章昭逹破歐陽紇而還遂引兵攻梁}
梁主與周摠管陸騰拒之周人於峽口南岸築安蜀城|{
	峽口西陵峽口也杜佑曰安蜀城在夷陵郡界}
横引大索於江上編葦為橋以度軍糧昭逹命軍士為長戟施於樓船上仰割其索索斷糧絶|{
	索昔洛翻}
因縱兵攻安蜀城下之梁主告急於周襄州摠管衛公直直遣大將軍李遷哲將兵救之遷哲以其所部守江陵外城自帥騎兵出南門使步出北門|{
	將即亮翻帥讀曰率步下有兵字文意較明}
首尾邀擊陳兵陳兵多死夜陳兵竊於城西以梯登城登者已數百人遷哲與陸騰力戰拒之乃退昭逹又决龍川寧朔隄|{
	水經注紀南城西南有赤坂岡岡下有瀆水東北流入城又東北出城西南注於龍陂陂在靈溪東江堤内水至淵深有龍見於其中故曰龍陂寧朔周書陸騰傳作寧邦}
引水灌江陵騰出戰於西堤昭逹兵不利乃引還|{
	還從宣翻又如字}
八月辛卯齊主如晉陽 九月乙巳齊立皇子恒為太子冬十月辛巳朔日有食之 齊以廣寧王孝珩為司

徒上洛王思宗為司空|{
	上洛古郡名}
復以梁永嘉王莊為開府儀同三司梁王|{
	復扶又翻}
許以興復竟不果及齊亡莊憤邑卒於鄴|{
	史終言之卒子恤翻}
乙酉上享太廟 己丑齊復威宗謚曰文宣皇帝廟號顯祖|{
	齊改文宣謚號見上卷文帝天嘉六年謚神至翻}
丁酉周鄭桓公逹奚武卒|{
	謚法辟土服遠曰桓卒子恤翻}
十二月

丁亥齊主還鄴 周大將軍鄭恪將兵平越嶲置西寧州|{
	西寧州後改曰嶲州嶲音髓}
周齊争宜陽久而不决勲州刺史韋孝寛謂其下曰|{
	周置勲州於玉壁}
宜陽一城之地不足損益两國争之勞師彌年彼豈無智謀之士若棄崤東|{
	宜陽在三崤之東}
來圖汾北我必失地今宜速於華谷及長秋築城以杜其意|{
	水經涑水出河内聞喜縣黍葭谷注云涑水所出俗謂之華谷又曰汾水過臨汾縣東又屈從縣南西流又西過長修縣南又西與華水合水出於北山華谷此所謂長秋即漢長修縣故墟也俗語訛以修為長秋耳}
脱其先我|{
	先音悉薦翻}
圖之實難乃畫地形具陳其狀晉公護謂使者曰韋公子孫雖多數不滿百汾北築城遣誰守之事遂不行齊斛律光果出晉州道於汾北築華谷龍門二城|{
	五代志絳郡稷山縣有後魏龍門郡}
光至汾東與孝寛相見光曰宜陽一城久勞争戰今已舍彼|{
	舍讀曰捨}
欲於汾北取償幸勿怪也孝寛曰宜陽彼之要衝汾北我之所棄我棄彼取其償安在君輔翼幼主位望隆重不撫循百姓而極武窮兵苟貪尋常之地塗炭疲弊之民竊為君不取也|{
	為于偽翻}
光進圍定陽築南汾城以逼之|{
	魏收志後魏延興四年置定陽縣及定陽郡五代志文城郡吉昌縣後魏置定陽縣及定陽郡文城郡東魏置南汾州後周改曰汾州吉昌縣後唐避李國昌諱又改為吉鄉縣九域志屬汾州宋白曰慈州吉鄉縣本漢北屈縣後魏孝文帝置定陽郡及定陽縣時會有河西定陽胡人渡河居此因以為名}
周人釋宜陽之圍以救汾北晉公護問計於齊公憲憲曰兄宜暫出同州以為聲勢憲請以精兵居前隨機攻取護從之|{
	汾扶云翻}
三年春正月乙丑以尚書右僕射徐陵為左僕射 丁巳齊使兼散騎常侍劉環儁來聘|{
	散悉亶翻騎奇寄翻}
辛酉上祀南郊辛未祀北郊 齊斛律光築十三城於西境|{
	汾北之地於鄴為西}
馬上以鞭指畫而成拓地五百里而未嘗伐功又與周韋孝寛戰於汾北破之齊公憲督諸將東拒齊師|{
	將即亮翻}
二月辛巳上祀明堂丁酉耕籍田|{
	籍秦昔翻}
壬寅齊以蘭陵王長恭為太尉趙彥深為司空和士開録尚書事徐之才為尚書令唐邕為左僕射吏部尚書馮子琮為右僕射仍攝選|{
	仍攝吏部選選須絹翻下典選}
子琮素謟附士開至是自以太后親屬且典選頗擅引用人不復啓禀由是與士開有隙|{
	為子琮勸琅邪王儼殺士開張本復扶又翻}
三月丁丑大赦 周齊公憲自龍門度河|{
	此自夏陽度汾隂也考異曰北齊書段韶傳云二月周師來寇周書帝紀云三月憲度河今從之}
斛律光退保華谷憲攻拔其新築五城齊太宰段韶蘭陵王長恭將兵禦周師攻柏谷城拔之而還|{
	此齊遣段韶等出伊洛以牽制汾北也將即亮翻下同}
夏四月戊寅朔日有食之 壬午齊以琅邪王儼為太保|{
	邪音耶}
壬辰齊遣使來聘|{
	使疏吏翻}
周陳公純取齊宜陽等九城|{
	考異曰北齊斛律光傳云周柱國紇干廣略圍宜陽今從周帝紀}
齊斛律光將步騎五

萬赴之|{
	騎奇寄翻}
五月癸亥周使納言鄭詡來聘|{
	唐六典後周天官府置御伯中大夫二人天子出入則侍於左右武帝改御伯為納言盖侍中之職也}
周晉公護使中外府參軍郭榮城於姚襄城南定陽城西|{
	姚襄城者襄為桓温所敗奔平陽所築後人因以為名杜佑曰今汾州吉昌縣西則姚襄所築城西臨黄河控帶龍門之險春秋時晉之北屈也郭榮之下有築字文意乃明}
齊段韶引兵襲周師破之六月韶圍定陽城 |{
	考異曰韶傳七月屠其外城周書北齊帝紀皆云六月陷汾州今從之}
周汾州刺史楊敷固守不下|{
	周汾州治定陽城隋改為吉昌縣汾州唐改為慈州}
韶急攻之屠其外城時韶臥病謂蘭陵王長恭曰此城三面重澗|{
	重直龍翻}
皆無走路唯慮東南一道耳賊必從此出宜簡精兵專守之此必成禽長恭乃令壯士千餘人伏於東南澗口城中粮盡齊公憲摠兵救之憚韶不敢進敷帥見兵突圍夜走|{
	帥讀曰率見賢遍翻}
伏兵擊擒之盡俘其衆|{
	考異曰周書齊王憲傳屢破齊師北齊書斛律光段韶傳屢破周師要之周失汾州齊師勝耳}
乙巳

齊取周汾州及姚襄城唯郭榮所築城獨存敷愔之族子也|{
	楊愔以死殉齊而敷為周死於城郭各盡忠於其所事也愔於今翻}
敷子素少多才藝|{
	少詩沼翻}
有大志不拘小節以其父守節䧟齊未蒙贈謚上表申理周主不許至於再三帝大怒命左右斬之素大言曰臣事無道天子死其分也|{
	分扶問翻}
帝壯其言贈敷大將軍謚曰忠壯|{
	謚法武而不遂曰壯}
以素為儀同三司漸見禮遇帝命素為詔書下筆立成詞義兼美帝曰勉之勿憂不富貴素曰但恐富貴來逼臣臣無心圖富貴也 齊斛律光與周師戰於宜陽城下取周建安等四戍捕虜千餘人而還軍未至鄴齊主敕使散兵光以軍士多有功者未得慰勞乃密通表請遣使宣旨|{
	宣慰勞之旨也勞力到翻使疏吏翻下待使同}
軍仍且進齊朝發使遲留|{
	朝直遥翻下同}
軍還將至紫陌光乃駐營待使帝聞光軍已逼心甚惡之|{
	惡烏路翻}
亟令舍人召光入見然後宣勞散兵|{
	史言齊主已有疑斛律光之心見賢遍翻勞力到翻}
齊琅邪王儼以和士開穆提婆等專横奢縱意甚不平|{
	横戶孟翻}
二人相謂曰琅邪王眼光奕奕數步射人向者暫對不覺汗出吾輩見天子奏事尚不然由是忌之乃出儼居北宫|{
	北宫在鄴之北城}
五日一朝不得無時見太后|{
	見賢遍翻}
儼之除太保也餘官悉解猶帶中丞及京畿士開等以北城有武庫欲移儼於外然後奪其兵權|{
	京畿大都督揔京畿兵}
治書侍御史王子宜|{
	治直之翻}
與儼所親開府儀同三司高舍洛中常侍劉辟彊說儼曰殿下被疏正由士開閒搆|{
	說式芮翻被皮義翻疏與踈同閒古莧翻}
何可出北宫入民間也儼謂侍中馮子琮曰士開罪重兒欲殺之何如|{
	馮子琮胡太后之妹夫故儼自稱曰兒}
子琮心欲廢帝而立儼因勸成之儼令子宜表彈士開罪請禁推|{
	請收禁而推鞫之彈徒丹翻}
子琮雜佗文書奏之帝王不審省而可之|{
	王字衍據北齊書琅邪王儼傳云後主不審省而可之通鑑就舊史刪潤以成一家言本云帝不審省而可之書吏繕寫因舊史之文遂衍主字杭本作齊主省悉并翻}
儼誑領軍庫狄伏連曰奉敕令領軍收士開|{
	魏收官氏志次南諸姓有庫狄氏誑居况翻}
伏連以告子琮且請覆奏子琮曰琅邪受勅何必更奏伏連信之京畿軍士伏於神虎門外|{
	神虎門即神武門南北國四史成於唐臣之手避唐諱凡虎字悉改為武此獨存舊}
并戒門者不聽士開入秋七月庚午旦士開依常早參|{
	依常日早入禁中朝參}
伏連前執士開手曰今有一大好事王子宜授以一函云有敕令王向臺因遣軍士護送儼遣都督馮永洛就臺斬之儼本意唯殺士開其黨因逼儼曰事既然不可中止儼遂帥京畿軍士三千餘人屯於秋門|{
	帥讀曰率}
帝使劉桃枝將禁兵八十人召儼|{
	將即亮翻}
桃枝遥拜儼命反縛將斬之禁兵散走帝又使馮子琮召儼儼辭曰士開昔來實合萬死謀廢至尊剃家家髪為尼|{
	剃他計翻}
臣為是矯詔誅之尊兄若欲殺臣不敢逃罪若赦臣願遣姊姊來迎臣即入見|{
	齊諸王皆呼嫡母為家家乳母為姊姊婦為妹妹為于偽翻見賢遍翻}
姊姊謂陸令萱也儼欲誘出殺之|{
	誘羊久翻}
令萱執刀在帝後聞之戰栗帝又使韓長鸞召儼儼將入劉辟彊牽衣諫曰若不斬穆提婆母子殿下無由得入廣寧王孝珩安德王延宗自西來|{
	珩音行}
曰何不入辟彊曰兵少|{
	少詩沼翻下謂少同}
延宗顧衆而言曰孝昭帝殺楊遵彥|{
	殺楊愔事見百六十八卷文帝天嘉元年}
止八十人今有數千何謂少帝泣啓太后曰有緣復見家家|{
	復扶又翻}
無緣永别乃急召斛律光儼亦召之光聞儼殺士開撫掌大笑曰龍子所為固自不似凡人|{
	以儼帝子故謂之龍子}
入見帝於永巷|{
	見賢遍翻}
帝帥宿衛者步騎四百授甲將出戰|{
	帥讀曰率}
光曰小兒輩弄兵與交手即亂鄙諺云奴見大家心死|{
	大家謂主也臣妾呼天子為大家亦此義}
至尊宜自至千秋門琅邪必不敢動帝從之光步道|{
	道讀曰導言步而前導也}
使人走出曰大家來儼徒駭散帝駐馬橋上遥呼之儼猶立不進光就謂曰天子弟殺一夫何所苦執其手強引以前|{
	強其两翻}
請於帝曰琅邪王年少腸肥腦滿輕為舉措稍長自不復然|{
	邪音耶少詩沼翻長謂年長音展两翻}
願寛其罪帝拔儼所帶刀鐶亂築辮頭|{
	辮頭示將斬之也}
良久乃釋之收庫狄伏連高舍洛王子宜劉辟彊都督翟顯貴|{
	翟萇伯翻又音狄}
於後園支解暴之都街帝欲盡殺儼府文武職吏光曰此皆勲貴子弟誅之恐人心不安趙彥深亦曰春秋責帥|{
	春秋左氏傳韓獻子謂中行桓子曰子為元帥帥不用命誰之罪也帥所類翻}
於是罪之各有差太后責問儼儼曰馮子琮教兒太后怒遣使就内省以弓絃絞殺子琮|{
	使疏吏翻}
使内參以庫車載尸歸其家|{
	内參宦者也庫車載庫物者也}
自是太后常置儼於宫中每食必自嘗之|{
	慮鴆毒也}
八月己亥齊主如晉陽 九月辛亥齊以任城王湝為太宰|{
	任音壬湝音皆又戶皆翻}
馮翊王潤為太師 己未齊平原忠武王段韶卒韶有謀略得將士死力|{
	卒子恤翻將即亮翻}
出摠軍旅入參幃幄功高望重而雅性温慎得宰相體事後母孝閨門雍肅齊勲貴之家無能及者 齊祖珽說陸令萱出趙彦深為兖州刺史齊主以珽為侍中陸令萱說帝曰人稱琅邪王聰明雄勇當今無敵觀其相表|{
	邪音耶說式芮翻相息亮翻}
殆非人臣自專殺以來常懷恐懼宜早為之計幸臣何洪珍等亦請殺之帝未决以食轝密迎珽問之|{
	食轝載粱肉以輸太官者也}
珽稱周公誅管叔|{
	周公使管叔監殷管叔以殷叛周公誅之}
季友酖慶父|{
	左傳魯莊公有疾問後於叔牙對曰慶父材問於季友季友曰臣以死奉般乃使鍼季酖叔牙而立般慶父使人弑般季友立閔公慶父又使人弑之季友以僖公適邾慶父奔莒季友乃入立僖公以賂求慶父於莒莒人歸之及密使公子奚斯請弗許哭而往慶父聞之曰奚斯之聲也乃縊慶父叔牙季友皆桓公之子兄弟也然而以酖死者叔牙也夫漢儒固有以經義斷獄者若祖珽者舞經義以成其獄者也不可不察}
帝乃攜儼之晉陽使右衛大將軍趙元侃誘儼執之|{
	誘音酉}
元侃曰臣昔事先帝見先帝愛王今寧就死不忍行此帝出元侃為豫州刺史庚午帝啓太后曰明旦欲與仁威早出獵|{
	儼字仁威}
夜四鼓帝召儼儼疑之陸令萱曰兄呼兒何為不去儼出至永巷劉桃枝反接其手儼呼曰乞見家家尊兄桃枝以袖塞其口|{
	呼火故翻塞悉則翻}
反袍蒙頭負出至大明宮鼻血滿面拉殺之|{
	拉盧合翻}
時年十四裹之以席埋於室内帝使啓太后太后臨哭十餘聲即擁入殿遺腹四男皆幽死冬十月罷京畿府入領軍|{
	以儼以京畿兵弄兵故罷之}
乙未周遣右武伯谷會琨等聘於齊|{
	五代志周置左右武伯掌内外衛之禁令兼六率之士左右小武伯各二人貮之谷會虜複姓}
齊胡太后出入不節與沙門統曇獻通諸僧至有戱呼曇獻為太上皇者齊主聞太后不謹而未之信後朝太后|{
	朝直遥翻}
見二尼悦而召之乃男子也於是曇獻事亦發皆伏誅己亥帝自晉陽奉太后還鄴至紫陌遇大風舍人魏僧伽習風角|{
	伽戍迦翻}
奏言即時當有暴逆事帝詐云鄴中有變彎弓纒弰馳入南城|{
	弰所交翻弓末也北齊書胡后傳作纒矟鄴都有南北城}
遣宦者鄧長顒幽太后於北宫|{
	幽母之事隐于心而未而暴風已應於上天人之際可畏哉顒魚容翻}
仍敕内外諸親皆不得與胡太后相見太后或為帝設食|{
	為于偽翻}
帝亦不敢嘗 庚戍齊遣侍中赫連子悦聘於周 十一月丁巳周主如散關|{
	散悉亶翻}
丙寅齊以徐州行臺廣寧王孝珩|{
	珩音行}
錄尚書事庚午又以為司徒癸酉以斛律光為左丞相 十二月己丑周主還長安 壬辰邵陵公章昭逹卒 是歲梁華皎將如周|{
	華戶化翻}
過襄陽說衛公直曰梁主既失江南諸郡|{
	吳明徹章昭逹再攻梁江南諸郡皆為陳所取說式芮翻}
民少國貧朝廷興亡繼絶理宜資贍望借數州以資梁國直然之遣使言狀周主詔以基平鄀三州與之|{
	五代志竟陵郡豐鄉縣西魏置基州及章山郡又竟陵郡樂鄉縣舊置武寧郡西魏置鄀州又南郡紫陵縣其城南面梁置鄀州今周以與梁者盖武寧之鄀州也當陽縣後周置平州少詩沼翻贍昌艶翻使疏吏翻鄀音若}


資治通鑑卷一百七十
