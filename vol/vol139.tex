<!DOCTYPE html PUBLIC "-//W3C//DTD XHTML 1.0 Transitional//EN" "http://www.w3.org/TR/xhtml1/DTD/xhtml1-transitional.dtd">
<html xmlns="http://www.w3.org/1999/xhtml">
<head>
<meta http-equiv="Content-Type" content="text/html; charset=utf-8" />
<meta http-equiv="X-UA-Compatible" content="IE=Edge,chrome=1">
<title>資治通鑒_140-資治通鑑卷一百三十九_140-資治通鑑卷一百三十九</title>
<meta name="Keywords" content="資治通鑒_140-資治通鑑卷一百三十九_140-資治通鑑卷一百三十九">
<meta name="Description" content="資治通鑒_140-資治通鑑卷一百三十九_140-資治通鑑卷一百三十九">
<meta http-equiv="Cache-Control" content="no-transform" />
<meta http-equiv="Cache-Control" content="no-siteapp" />
<link href="/img/style.css" rel="stylesheet" type="text/css" />
<script src="/img/m.js?2020"></script> 
</head>
<body>
 <div class="ClassNavi">
<a  href="/24shi/">二十四史</a> | <a href="/SiKuQuanShu/">四库全书</a> | <a href="http://www.guoxuedashi.com/gjtsjc/"><font  color="#FF0000">古今图书集成</font></a> | <a href="/renwu/">历史人物</a> | <a href="/ShuoWenJieZi/"><font  color="#FF0000">说文解字</a></font> | <a href="/chengyu/">成语词典</a> | <a  target="_blank"  href="http://www.guoxuedashi.com/jgwhj/"><font  color="#FF0000">甲骨文合集</font></a> | <a href="/yzjwjc/"><font  color="#FF0000">殷周金文集成</font></a> | <a href="/xiangxingzi/"><font color="#0000FF">象形字典</font></a> | <a href="/13jing/"><font  color="#FF0000">十三经索引</font></a> | <a href="/zixing/"><font  color="#FF0000">字体转换器</font></a> | <a href="/zidian/xz/"><font color="#0000FF">篆书识别</font></a> | <a href="/jinfanyi/">近义反义词</a> | <a href="/duilian/">对联大全</a> | <a href="/jiapu/"><font  color="#0000FF">家谱族谱查询</font></a> | <a href="http://www.guoxuemi.com/hafo/" target="_blank" ><font color="#FF0000">哈佛古籍</font></a> 
</div>

 <!-- 头部导航开始 -->
<div class="w1180 head clearfix">
  <div class="head_logo l"><a title="国学大师官网" href="http://www.guoxuedashi.com" target="_blank"></a></div>
  <div class="head_sr l">
  <div id="head1">
  
  <a href="http://www.guoxuedashi.com/zidian/bujian/" target="_blank" ><img src="http://www.guoxuedashi.com/img/top1.gif" width="88" height="60" border="0" title="部件查字,支持20万汉字"></a>


<a href="http://www.guoxuedashi.com/help/yingpan.php" target="_blank"><img src="http://www.guoxuedashi.com/img/top230.gif" width="600" height="62" border="0" ></a>


  </div>
  <div id="head3"><a href="javascript:" onClick="javascript:window.external.AddFavorite(window.location.href,document.title);">添加收藏</a>
  <br><a href="/help/setie.php">搜索引擎</a>
  <br><a href="/help/zanzhu.php">赞助本站</a></div>
  <div id="head2">
 <a href="http://www.guoxuemi.com/" target="_blank"><img src="http://www.guoxuedashi.com/img/guoxuemi.gif" width="95" height="62" border="0" style="margin-left:2px;" title="国学迷"></a>
  

  </div>
</div>
  <div class="clear"></div>
  <div class="head_nav">
  <p><a href="/">首页</a> | <a href="/ShuKu/">国学书库</a> | <a href="/guji/">影印古籍</a> | <a href="/shici/">诗词宝典</a> | <a   href="/SiKuQuanShu/gxjx.php">精选</a> <b>|</b> <a href="/zidian/">汉语字典</a> | <a href="/hydcd/">汉语词典</a> | <a href="http://www.guoxuedashi.com/zidian/bujian/"><font  color="#CC0066">部件查字</font></a> | <a href="http://www.sfds.cn/"><font  color="#CC0066">书法大师</font></a> | <a href="/jgwhj/">甲骨文</a> <b>|</b> <a href="/b/4/"><font  color="#CC0066">解密</font></a> | <a href="/renwu/">历史人物</a> | <a href="/diangu/">历史典故</a> | <a href="/xingshi/">姓氏</a> | <a href="/minzu/">民族</a> <b>|</b> <a href="/mz/"><font  color="#CC0066">世界名著</font></a> | <a href="/download/">软件下载</a>
</p>
<p><a href="/b/"><font  color="#CC0066">历史</font></a> | <a href="http://skqs.guoxuedashi.com/" target="_blank">四库全书</a> |  <a href="http://www.guoxuedashi.com/search/" target="_blank"><font  color="#CC0066">全文检索</font></a> | <a href="http://www.guoxuedashi.com/shumu/">古籍书目</a> | <a   href="/24shi/">正史</a> <b>|</b> <a href="/chengyu/">成语词典</a> | <a href="/kangxi/" title="康熙字典">康熙字典</a> | <a href="/ShuoWenJieZi/">说文解字</a> | <a href="/zixing/yanbian/">字形演变</a> | <a href="/yzjwjc/">金 文</a> <b>|</b>  <a href="/shijian/nian-hao/">年号</a> | <a href="/diming/">历史地名</a> | <a href="/shijian/">历史事件</a> | <a href="/guanzhi/">官职</a> | <a href="/lishi/">知识</a> <b>|</b> <a href="/zhongyi/">中医中药</a> | <a href="http://www.guoxuedashi.com/forum/">留言反馈</a>
</p>
  </div>
</div>
<!-- 头部导航END --> 
<!-- 内容区开始 --> 
<div class="w1180 clearfix">
  <div class="info l">
   
<div class="clearfix" style="background:#f5faff;">
<script src='http://www.guoxuedashi.com/img/headersou.js'></script>

</div>
  <div class="info_tree"><a href="http://www.guoxuedashi.com">首页</a> > <a href="/SiKuQuanShu/fanti/">四库全书</a>
 > <h1>资治通鉴</h1> <!--         下载:【右键另存为】即可 --></div>
  <div class="info_content zj clearfix">
  
<div class="info_txt clearfix" id="show">
<center style="font-size:24px;">140-資治通鑑卷一百三十九</center>
    資治通鑑卷一百三十九 宋 司馬光 撰<br />
<br />
  胡三省 音註<br />
<br />
  齊紀五【閼逢開茂一年】<br />
<br />
  高宗明皇帝上【諱鸞字景栖小字玄度高帝兄始安貞王道生之子】<br />
<br />
  建武元年【是年十月始改元建武】春正月丁未改元隆昌【此鬰林王改元也】大赦 雍州刺史晉安王子懋【雍於用翻】以主幼時艱密為自全之計令作部造仗【諸州各有作部主造器仗】征南大將軍陳顯逹屯襄陽【去年秋武帝以魏將入寇遣顯逹鎮樊城】子懋欲脅取以為將【將即亮翻下同】顯逹密啓西昌侯鸞鸞徵顯逹為車騎大將軍【騎奇寄翻】徙子懋為江州刺史仍令留部曲助鎮襄陽單將白直俠轂自隨【諸王有白直有夾轂隊俠讀曰夾】顯逹過襄陽【過音戈】子懋謂曰朝廷令身單身而返身是天王豈可過爾輕率【子懋自稱天王蓋謂是天家諸王也】今猶欲將二三千人自隨公意何如顯逹曰殿下若不留部曲乃是大違敕旨其事不輕且此間人亦難可收用【此間人謂襄陽人也】子懋默然顯逹因辭出即去子懋計未立乃之尋陽 西昌侯鸞將謀廢立引前鎮西諮議參軍蕭衍與同謀【隨王子隆初以鎮西將軍鎮荆州引衍為諮議參軍】荆州刺史隨王子隆性温和有文才鸞欲徵之恐其不從衍曰隨王雖有美名其實庸劣既無智謀之士爪牙唯仗司馬垣歷生武陵太守卞白龍耳二人唯利是從若啗以顯職無有不來隨王止須折簡耳鸞從之徵歷生為太子左衛率白龍為游擊將軍【啗徒濫翻折之舌翻率所律翻】二人並至續召子隆為侍中撫軍將軍【此時西昌侯已有殺諸王之心矣蕭衍由是以籌略見用】豫州刺史崔慧景高武舊將【將即亮翻】鸞疑之以蕭衍為寜朔將軍戍夀陽慧景懼白服出迎【白服若得罪而白衣領職者】衍撫安之 辛亥鬰林王祀南郊戊午拜崇安陵【鬰林王即位追尊父文惠太子曰文帝陵曰崇安廟號世宗據竟陵王子良傳陵在夾右】 癸亥魏主南巡戊辰過比干墓【水經注河内朝歌縣南有牧野有比干冢前有石銘題隸云殷大夫比干之墓不知誰所誌也】祭以太牢魏主自為祝文曰嗚呼介士胡不我臣 帝寵幸中書舍人綦毋珍之朱隆之直閤將軍曹道剛周奉叔宦者徐龍駒等【帝謂鬰林王】珍之所論薦事無不允【允信也肯也】内外要職皆先論價旬月之間家累千金擅取官物及役作不俟詔旨有司至相語云【語牛倨翻】寜拒至尊敇不可違舍人命帝以龍駒為後閤舍人【後閤禁中後閤也南史曰龍駒日夜在六宫房内】常居含章殿著黄綸㡌被貂裘【著陟略翻被皮義翻】南面向案代帝畫敇左右侍直與帝不異帝自山陵之後即與左右微服遊走市里好於世宗崇安陵隧中擲塗賭跳【好呼到翻文惠太子廟號世宗塗泥也以塗泥相擲為樂也跳躍也賭跳者以跳躍高出者為勝跳他弔翻】作諸鄙戲極意賞賜左右動至百數十萬每見錢曰我昔思汝一枚不得今日得用汝未世祖聚錢上庫五億萬齋庫亦出三億萬【上庫所儲以備軍國之用齋庫以供齋内所須人主之好用出者出三億萬數之外也】金銀布帛不可勝計【勝音升】鬰林王即位未朞歲所用垂盡入主衣庫令何后及寵姬以諸寶器相投擊破碎之用為笑樂【樂音洛】蒸於世祖幸姬霍氏更其姓曰徐【更工衡翻李延夀史以霍為文帝幸姬則世祖當作世宗】朝事大小皆决於西昌侯鸞【朝直遥翻下同】鸞數諫爭【數所角翻爭讀曰諍】帝多不從心忌鸞欲除之以尚書右僕射鄱陽王鏘為世祖所厚【世祖恐亦當作世宗】私謂鏘曰公聞鸞於法身如何【鬰林王小字法身】鏘素和謹對曰臣鸞於宗戚最長且受寄先帝臣等皆年少【長知兩翻少詩沼翻】朝廷所賴唯鸞一人願陛下無以為慮帝退謂徐龍駒曰我欲與公共計取鸞公既不同我不能獨辦且復小聽【復扶又翻下無復同言且又小時聽鸞專政也】衛尉蕭諶世祖之族子也【蕭子顯齊書曰諶於太祖為絶服族子諶氏壬翻】自世祖在郢州諶已為腹心【宋元徽末世祖在郢州欲知都下消息太祖遣諶就世祖宣傳謀計留為腹心】及即位常典宿衛機密之事無不預聞征南諮議蕭坦之諶之族人也嘗為東宫直閤為世宗所知【蕭子顯齊書曰坦之以懃直為世祖所知既為東宫直閤則從世宗為是東宫亦有直閤將軍】帝以二人祖父舊人甚親信之諶每請急出宿帝通夕不寐諶還乃安坦之得出入後宫帝䙝狎宴遊坦之皆在側帝醉後常裸袒【裸郎果翻】坦之輒扶持諫諭西昌侯鸞欲有所諫帝在後宫不出唯遣諶坦之徑進乃得聞逹何后亦淫泆【泆音逸淫泆放也】私於帝左右楊珉與同寢處如伉儷【處昌呂翻下處之同杜預曰伉敵也儷耦也伉苦浪翻儷力計翻】又與帝相愛狎故帝恣之迎后親戚入宫以耀靈殿處之齋閤通夜洞開外内淆雜無復分别【别彼列翻】西昌侯鸞遣坦之入奏誅珉何后流涕覆面【覆敷又翻】曰楊郎好年少無罪何可枉殺【少詩沼翻】坦之附耳語帝曰【語牛倨翻下每語同】外間並云楊珉與皇后有情事彰遐邇不可不誅帝不得已許之俄敇原之已行刑矣鸞又啓誅徐龍駒帝亦不能違而心忌鸞益甚蕭諶蕭坦之見帝狂縱日甚無復悛改【悛丑緣翻】恐禍及己乃更回意附鸞勸其廢立隂為鸞耳目帝不之覺也周奉叔恃勇挾勢陵轢公卿【轢郎狄翻】常翼單刀二十口自隨【翼者分列左右若兩翼然也】出入禁闥門衛不敢訶【訶虎何翻】每語人曰周郎刀不識君鸞忌之使蕭諶蕭坦之說帝出奉叔為外援【說輸芮翻下鸞說此說同】己巳以奉叔為青州刺史【蕭子顯曰宋秦始中淮北没虜徙青州治欝洲齊建元四年徙治胊山後復舊】曹道剛為中軍司馬奉叔就帝求千戶侯許之鸞以為不可封曲江縣男食三百戶奉叔大怒於衆中攘刀厲色鸞說諭之乃受【說輸芮翻下同】奉叔辭畢將之鎮部伍已出鸞與蕭諶稱敇召奉叔於省中敺殺之【省中尚書省中也敺烏口翻】啓云奉叔慢朝廷帝不獲已可其奏溧陽令錢唐杜文謙嘗為南郡王侍讀【溧陽縣自漢以來屬丹陽郡其地在建康東南帝初封南郡王溧音栗】前此說綦毋珍之曰天下事可知灰盡粉滅匪朝伊夕不早為計吾徒無類矣珍之曰計將安出文謙曰先帝舊人多見擯斥今召而使之誰不忼慨近聞王洪範【王洪範即轉言日月相者也】與宿衛將萬靈會等共語皆攘袂搥牀【將即亮翻搥傳追翻】君其密報周奉叔使萬靈會等殺蕭諶則宫内之兵皆我用也【蕭諶時以衛軍司馬兼衛尉卿掌宿衛兵】即勒兵入尚書斬蕭令【尚書省在雲龍門内】兩都伯力耳【都伯行刑者也今謂之劊子】今舉大事亦死不舉事亦死二死等耳死社稷可乎若遲疑不斷復少日錄君稱敕賜死【復扶又翻少詩沼翻少日言無多日也鸞錄尚書事故稱為錄君】父母為殉【謂皆將從坐而死也】在眼中矣珍之不能用及鸞殺奉叔并收珍之文謙殺之乙亥魏主如洛陽西宫中書侍郎韓顯宗上書陳四<br />
<br />
  事其一以為竊聞輿駕今夏不巡三齊當幸中山往冬輿駕停鄴當農隙之時猶比屋供奉不勝勞費【比毗必翻又毗至翻勝音升】况今蠶麥方急將何以堪命且六軍涉暑恐生癘疫臣願早還北京以省諸州供張之苦【北京謂平城張竹亮翻】成洛都營繕之役其二以為洛陽宫殿故基皆魏明帝所造前世已譏其奢今兹營繕宜加裁損又頃來北都富室競以第舍相尚【北都亦謂平城魏既遷洛以平城為北都】宜因遷徙為之制度及端廣衢路通利溝渠其三以為陛下之還洛陽輕將從騎【從才用翻】王者於闈闥之内【宫中門曰闈韓詩門屏間曰闥】猶施警蹕况涉履山河而不加三思乎【三息暫翻】其四以為陛下耳聽法音【法音謂雅樂也】目翫墳典【謂三墳五典書序伏羲神農黄帝之書謂之三墳言大道也少昊顓頊高辛唐虞之書謂之五典言常道也孔子序書斷自唐虞三墳五典後世不復見其全此特大槩言之】口對百辟心虞萬機景昃而食【虞度也景昃日昃也日景過中則昃昃音側】夜分而寢加以孝思之至隨時而深【謂文明太后之殂已久而帝孝思不忘也】文章之業日成篇卷雖叡明所用未足為煩然非所以嗇神養性【嗇愛也】保無疆之祚也伏願陛下垂拱司契而天下治矣【老子曰有德司契司主也契要也治直吏翻】帝頗納之顯宗麒麟之子也【韓麒麟見一百三十五卷武帝永明元年】顯宗又上言以為州郡貢察徒有秀孝之名而無秀孝之實【貢察者謂察舉秀才孝廉而貢之於朝】朝廷但檢其門望不復彈坐【復扶又翻彈坐者彈劾其違而坐之以罪】如此則可令别貢門望以叙士人何假冒秀孝之名也夫門望者乃其父祖之遺烈亦何益於皇家益於時者賢才而已苟有其才雖屠釣奴虜聖王不恥以為臣【太公屠牛於朝歌釣於渭濱又紂時箕子為奴周文王武王皆禮而用之】苟非其才雖三后之胤墜于皂隸矣【左傳申無宇曰人有十等士臣皁皁臣輿輿臣隸釋曰皁直馬者隸附屬者三后謂夏商周之王也】議者或云今世等無奇才不若取士於門此亦失矣豈可以世無周召遂廢宰相邪但當校其寸長銖重者先叙之【言其人比之衆人稍有一寸之長一銖之重則先叙用之】則賢才無遺矣又刑罸之要在於明當【當丁浪翻】不在於重苟不失有罪雖捶撻之薄人莫敢犯若容可僥幸雖參夷之嚴不足懲禁【參夷謂夷三族也捶止橤翻僥堅堯翻】今内外之官欲邀當時之名爭以深刻為無私迭相敦厲【敦迫也厲嚴以勉之】遂成風俗陛下居九重之内視人如赤子百司分萬務之任遇下如仇讐是則堯舜止一人而桀紂以千百和氣不至蓋由於此謂宜敕示百僚以惠元元之命又昔周居洛邑猶存宗周【周成王宅洛以豐為宗周存故都也】漢遷東都京兆置尹【後漢都雒陽置河南尹而長安仍置京兆尹亦存故都也】察春秋之義有宗廟曰都無曰邑况代京宗廟山陵所託王業所基其為神鄉福地實亦遠矣今便同之郡國臣竊不安謂宜建畿置尹一如故事【魏初都平城分畫甸畿置司州於平城置代尹】崇本重舊光示萬葉又古者四民異居欲其業專志定也【管仲相齊使士農工商各羣萃而州處其言曰四民者勿使雜處雜處則其言厖其事易昔聖王之處士也使就閑燕處工就官府處商就市井處農就田野長而安焉不見異物而遷焉】太祖道武皇帝創基撥亂日不暇給然猶分别士庶不令雜居工伎屠沽各有攸處【别彼列翻伎渠綺翻處昌呂翻下同處同】但不設科禁久而混殽今聞洛邑居民之制專以官位相從不分族類夫官位無常朝榮夕悴【悴秦醉翻】則是衣冠皁隸不日同處矣借使一里之内或調習歌舞或講肄詩書【肄羊至翻】縱羣兒隨其所之則必不棄歌舞而從詩書矣然則使工伎之家習士人風禮百年難成士人之子効工伎容態一朝而就是以仲尼稱里仁之美孟母勤三徙之訓【論語孔子曰里仁為美擇不處仁焉得知列女傳曰孟軻母其舍近墓孟子少嬉遊為墓間之事孟母曰此非吾所以處子也乃去舍市旁其嬉戲乃賈人衒賣之事又曰此非吾所以處子也復徙舍學宫之旁其嬉戲乃設俎豆揖遜進退孟母曰此真可以居吾子矣遂居焉】此乃風俗之原不可不察朝廷每選人士校其一㛰一宦以為升降何其密也至于度地居民則清濁連甍何其略也【度徙洛翻甍謨耕翻屋棟所以承瓦】今因遷徙之初皆是空地分别工伎在於一言有何可疑而闕盛美又南人昔有淮北之地自比中華僑置郡縣【如豫州界止於汝陽而僑置譙梁陳潁等郡縣又於青州界僑置冀州諸郡縣是也僑渠驕翻】自歸附聖化仍而不改名實交錯文書難辨宜依地理舊名一皆釐革小者并合大者分置及中州郡縣昔以戶少併省【魏初得河南止置四鎮郡縣多所併省少詩沼翻】今民口既多亦可復舊又君人者以天下為家不可有所私倉庫之儲以供軍國之用自非有功德者不可加賜在朝諸貴受祿不輕比來賜賚動以千計【朝直遥翻比毗至翻】若分以賜鰥寡孤獨之民所濟實多今直以與親近之臣殆非周急不繼富之謂也【論語孔子曰君子周急不繼富】帝覽奏甚善之 二月乙丑魏主如河隂規方澤【規度其地以立方澤】 辛卯帝祀明堂 司徒參軍劉斆等聘于魏【斆胡教翻】 丙申魏徙河南王幹為趙郡王潁川王雍為高陽王【將以河南潁川為畿甸故二王徙封】 壬寅魏主北巡癸卯濟河三月壬申至平城 【考異曰魏帝紀作閏月按魏閏二月齊歷之三月也】使羣臣更論遷都利害各言其志燕州刺史穆羆曰【魏營洛以洛為司州改平城之司州為恒州分恒州東部置燕州治昌平】今四方未定未宜遷都且征伐無馬將何以克帝曰廐牧在代何患無馬今代在恒山之北九州之外非帝王之都也【恒戶登翻】尚書于果曰臣非以代地為勝伊洛之美也但自先帝以來久居於此百姓安之一旦南遷衆情不樂【樂音洛】平陽公丕曰遷都大事當訊之卜筮帝曰昔周召聖賢乃能卜宅【書洛誥曰召公既相宅周公往營成周使來告卜曰我卜河朔黎水我又卜澗水東瀍水西惟洛食我又卜瀍水東亦惟洛食】今無其人【卜】之何益且卜以决疑不疑何卜【左傳載鬭廉之言】黄帝卜而龜焦天老曰吉黄帝從之【杜預曰龜兆不成也字書釋灼龜不兆為焦】然則至人之知未然審於龜矣王者以四海為家或南或北何常之有朕之遠祖世居北荒平文皇帝始都東木根山【拓跋欝律謚平文皇帝晉明帝大寜二年通鑑書惠帝賀傉徙居東木根山】昭成皇帝更營盛樂【拓跋什翼犍謚昭成皇帝通鑑晉成帝咸康元年烈帝翳槐城盛樂次年昭成嗣國咸康七年築盛樂新城更工衡翻】道武皇帝遷于平城【晉安帝隆安二年道武帝遷都平城】朕幸屬勝殘之運【論語孔子曰善人為邦百年亦可以勝殘去殺矣朱元晦曰勝殘謂化善人不為惡也屬之欲翻會也勝音升】而獨不得遷乎羣臣不敢復言【復扶又翻】羆夀之孫【穆夀事魏太武帝】果烈之弟也癸酉魏主臨朝堂部分遷留【分扶問翻】 夏四月庚辰魏罷西郊祭天 【考異曰魏帝紀禮志北史紀皆云三月庚辰按長歷三月丙午朔無庚辰魏閏二月齊閏四月魏三月乙亥朔齊歷之四月也故置於此】辛巳武陵昭王曅卒 戊子竟陵文宣王子良以憂卒帝常憂子良為變聞其卒甚喜【欝林但虞子良為變而不知鸞諶之謀已成矣】臣光曰孔子稱鄙夫不可與事君未得之患得之既得之患失之苟患失之無所不至【見論語】王融乘危徼倖【徼堅堯翻】謀易嗣君子良當時賢王雖素以忠慎自居不免憂死迹其所以然正由融速求富貴而已輕躁之士烏可近哉【躁則到翻近其靳翻】<br />
<br />
  己亥魏罷五月五日七月七日饗祖考【魏端午七夕之饗猶寒食之饗皆夷禮也】 魏錄尚書事廣陵王羽奏令文每歲終州鎮列屬官治狀及再考則行黜陟【治直吏翻】去十五年京官盡經考為三等【去十五年猶云昨太和十五年也】今已三載臣輒凖外考以定京官治行【欲以考州鎮屬官之法考京官載子亥翻行下孟翻】魏主曰考績事重應關朕聽不可輕且俟至秋【史言魏孝文明於君人之體不使權在臣下】 閏月丁卯鎮軍將軍鸞即本號開府儀同三司【本號鎮軍將軍也】 戊辰以新安王昭文為揚州刺史 五月甲戌朔日有食之 【考異曰齊魏書帝紀皆無此日今據齊書志南史紀】 六月己巳魏遣兼員外散騎常侍盧昶兼員外散騎侍郎王清石來聘昶度世之子也【盧度世避崔浩之禍其後自出魏太武寵任之散悉亶翻騎奇寄翻昶丑兩翻】清石世仕江南魏主謂清石曰卿勿以南人自嫌彼有知識欲見則見欲言則言凡使人以和為貴勿迭相矜夸見於辭色【使疏吏翻下同見賢遍翻】失將命之體也【將奉也奉命而行謂之將命】 秋七月乙亥魏以宋王劉昶為使持節都督吳越楚諸軍事大將軍鎮彭城【江南皆春秋時吳越楚三國之地】魏主親餞之以王肅為昶府長史昶至鎮不能撫接義故【宋蒼梧王初昶鎮彭城棄鎮奔魏故義故在焉】卒無成功【卒子恤翻】 壬午魏安定靖王休卒自卒至殯魏主三臨其第葬之如尉元之禮【尉紆勿翻】送之出郊慟哭而返 壬戌魏主北巡 西昌侯鸞既誅徐龍駒周奉叔而尼媪外入者頗傳異語【媪烏皓翻異語謂外人籍籍口語言鸞等相與有異謀也】中書令何胤以后之從叔【從才用翻】為帝所親使直殿省帝與胤謀誅鸞令胤受事胤不敢當依違諫說帝意復止乃謀出鸞於西州中敇用事不復關咨於鸞【復扶又翻】是時蕭諶蕭坦之握兵權左僕射王晏摠尚書事諶密召諸王典籖約語之不許諸王外接人物【約語者約束而語之語牛倨翻】諶親要日久衆皆憚而從之鸞以其謀告王晏晏聞之響應又告丹陽尹徐孝嗣孝嗣亦從之驃騎錄事南陽樂豫謂孝嗣曰外傳籍籍似有伊周之事君蒙武帝殊常之恩荷託付之重【徐孝嗣為王儉所薦武帝擢而用之遺詔託以尚書衆事驃匹妙翻騎奇寄翻荷下可翻】恐不得同人此舉人笑褚公至今齒冷【謂褚淵也笑則啓齒故云齒冷】孝嗣心然之而不能從帝謂蕭坦之曰人言鎮軍與王晏蕭諶欲共廢我【鸞時領鎮軍將軍故稱之】似非虛傳卿所聞云何坦之曰天下寜當有此誰樂無事廢天子邪【樂音洛】朝貴不容造此論當是諸尼姥言耳豈可信耶【朝直遥翻姥莫補翻女老稱】官若無事除此三人誰敢自保直閤將軍曹道剛疑外間有異密有處分謀未能【言曹道剛密有圖鸞等之謀而未能處昌呂翻分扶問翻】時始興内史蕭季敞南陽太守蕭頴基皆内遷諶欲待二人至藉其埶力以舉事【以二人方自外郡歸各有兵力自送為可藉也】鸞慮事變以告坦之坦之馳謂諶曰廢天子古來大事比聞曹道剛朱隆之等轉已猜疑【比毗至翻】衛尉明日若不就事無所復及【復扶又翻】弟有百歲母豈能坐聽禍敗正應作餘計耳諶惶遽從之壬辰鸞使蕭諶先入宫遇曹道剛及中書舍人朱隆之皆殺之直後徐僧亮盛怒【直後亦宿衛之官待衛於乘輿之後者也】大言於衆曰吾等荷恩【荷下可翻】今日應死報又殺之鸞引兵自尚書入雲龍門戎服加朱衣於上比入門三失履【懼而夫其常度也比必寐翻及也】王晏徐孝嗣蕭坦之陳顯逹王廣之沈文季皆隨其後帝在夀昌殿【夀昌殿武帝所起宴居常居之】聞外有變猶密為手敇呼蕭諶又使閉内殿諸房閤俄而諶引兵入夀昌閤帝走趨徐姬房拔劍自刺不入【趨七喻翻刺七亦翻】以帛纒頸輿接出延德殿諶初入殿宿衛將士皆操弓楯欲拒戰【操千高翻楯食尹翻】諶謂之曰所取自有人卿等不須動宿衛素隸服於諶皆信之及見帝出各欲自奮帝竟無一言行至西弄弑之【此延德殿之西弄也丁度集韻曰弄厦也屏也亦作㢅帝死時年二十二】輿尸出殯徐龍駒宅葬以王禮徐姬及諸嬖倖皆伏誅鸞既執帝欲作太后令徐孝嗣於袖中出而進之鸞大悦癸巳以太后令追廢帝為鬰林王又廢何后為王妃迎立新安王昭文吏部尚書謝瀹方與客圍棊左右聞有變驚走報瀹瀹每下子【子棊子也】輒云其當有意竟局乃還齋卧竟不問外事【謝瀹為此兄朏之教也】大匠卿虞悰竊歎曰王徐遂縳袴廢天子天下豈有此理邪【大匠卿即漢將作大匠之官蕭子顯曰掌宗廟土木悰徂宗翻】悰嘯父之孫也【虞嘯父虞潭之子事晉孝武帝父音甫】朝臣被召入宫【朝直遥翻被皮義翻】國子祭酒江斆至雲龍門託藥吐車中而去【吐土故翻嘔也】西昌侯鸞欲引中散大夫孫謙為腹心【散悉亶翻】使兼衛尉給甲仗百人謙不欲與之同輒散甲士鸞亦不之罪也【史言謝瀹江斆以名義自將僅能如此而已特立不懼孫謙庶幾焉】丁酉新安王即皇帝位時年十五【王諱昭文字季尚文惠太子第二子也】以西昌侯鸞為驃騎大將軍錄尚書事揚州刺史宣城郡公大赦改元延興 辛丑魏主至朔州【魏收地形志雲州舊置朔州又有朔州本漢五原郡魏為懷朔鎮孝昌中始改為朔州今此朔州當置于雲中之盛樂時置朔州於定襄故城領盛樂廣牧二郡宋白曰孝文遷洛之後於今朔州北二百八十里定襄故城置朔州後亂廢】八月甲辰以司空王敬則為太尉鄱陽王鏘為司徒車騎大將軍陳顯逹為司空【鏘千羊翻騎奇寄翻】尚書左僕射王晏為尚書令 魏主至隂山 以始安王遥光為南郡太守不之官遥光鸞之兄子也【鸞兄鳳生遥光遥欣遥光嗣始安王爵】鸞有異志遥光贊成之凡大誅賞無不預謀戊申以中書郎蕭遥欣為兖州刺史遥欣遥光之弟也鸞欲樹置親黨故用之 癸丑魏主如懷朔鎮己未如武川鎮辛酉如撫宜鎮甲子如柔玄鎮【此六鎮自西徂東之次第也水經注懷朔鎮城在漢光祿城東北考其地當在漢五原稒陽塞外杜佑曰在馬邑郡北三百餘里武川鎮城在白道中溪水上白道在隂山之北又北出大漠柔玄鎮在于延水東于延水出塞外柔玄鎮西長川城南小山東南流逕漢代郡且如縣故城南則魏柔玄鎮城在漢且如縣西北塞外也且音子閭翻撫宜鎮城未考其地若以前說六鎮自五原抵濡源分置于三千里中則撫宜當在武州柔玄之間相距各五百里據前高閭之說則相距各一百七十許里耳按北史宜當作冥】乙丑南還辛未至平城 九月壬申朔魏詔曰三載考績三考黜陟【唐虞之制三考黜陟三考九年也載子亥翻】可黜者不足為遲可進者大成賖緩朕今三載一考即行黜陟欲令愚滯無妨於賢者才能不擁於下位各令當曹考其優劣為三等其上下二等仍分為三【上等下等各又分為三等】六品已下尚書重問【重直用翻】五品已上朕將親與公卿論其善惡上上者遷之下下者黜之中者守其本任魏主之北巡也留任城王澄銓簡舊臣自公侯已下有官者以萬數澄品其優劣能否為三等人無怨者【史言任城王澄之平明】壬午魏主臨朝堂黜陟百官【朝直遥翻】謂諸尚書曰尚書樞機之任非徒摠庶務行文書而已朕之得失盡在於此卿等居官年垂再朞未嘗獻可替否進一賢退一不肖此最罪之大者又謂錄尚書事廣陵王羽曰汝為朕弟居機衡之右無勤恪之聲有阿黨之迹今黜汝錄尚書廷尉但為特進太子太保又謂尚書令陸叡曰叔翻到省之初甚有善稱比來偏頗懈怠【廣陵王羽字叔翻稱昌孕翻比毗至翻頗傍禾翻亦偏也懈居隘翻】由卿不能相導以義雖無大責宜有小罸今奪卿祿一朞又謂左僕射拓跋贊曰叔翻受黜卿應大辟【辟毗亦翻】但以咎歸一人不復重責今解卿少師削祿一朞又謂左丞公孫良右丞乞伏義受曰卿罪亦應大辟可以白衣守本官冠服祿卹【魏官本祿之外别有恤親之祿】盡從削奪若三年有成還復本任無成永歸南畝又謂尚書任城王澄曰叔神志驕傲可解少保【澄於魏主叔也】又謂長兼尚書于果曰卿不勤職事數辭以疾【數所角翻】可解長兼削祿一朞其餘守尚書尉羽盧淵等並以不軄或解任或黜官或奪祿皆面數其過而行之【尉紆勿翻數所具翻唐虞三載考績三考黜陟幽明其黜陟行於九年之後非賖緩也俗淳事簡在位者各思盡其職不為奸欺就有不稱者一考而未黜冀其能自盡也其不能盡者才力有所不逮耳再考不稱而猶未黜謂才有短長臨事有過誤前考已稱其職而今考不稱者必過誤也前考不稱而今考能稱其職者能自勉也三考皆不稱則其人信不可用矣於是乎黜之此唐虞忠厚之至也周官計羣吏之治旬終則令正日成月終則令正月要歲終則令正歲會三歲則大計羣吏之治而誅賞之是蓋無日而不考覈而誅賞則行于三年大計之時蓋俗益薄人益媮而行九年之黜陟則為賖緩觀魏孝文之考績不過慕古而務名非能行考績之實也】淵昶之兄也【昶丑兩翻】帝又謂陸叡曰北人每言北俗質魯何由知書朕聞之深用憮然【憮罔甫翻憮然者悵然失意之貌】今知書者甚衆豈皆聖人顧學與不學耳朕修百官興禮樂其志固欲移風易俗朕為天子何必居中原正欲卿等子孫漸染美俗【漸子廉翻】聞見廣博若永居恒北【恒戶登翻】復值不好文之主【復扶又翻好呼到翻】不免面墻耳【書曰不學墻面言猶正墻面而立無所睹見也】對曰誠如聖言金日磾不入仕漢朝何能七世知名【金日磾事見七十一卷漢武帝後元元年七世知名謂七世内侍也磾丁奚翻朝直遥翻】帝甚悅 鬰林王之廢也鄱陽王鏘初不知謀及宣城公鸞權埶益重中外皆知其蓄不臣之志鏘每詣鸞鸞常履履至車後迎之【言急於出迎不暇躡履至跟也】語及家國言淚俱發鏘以此信之宫臺之内皆屬意於鏘【宫臺猶言宫省也屬之欲翻】勸鏘入宫發兵輔政制局監謝粲說鏘及隨王子隆曰二王但乘油壁車入宫【李延夀恩倖傳曰武官有制局監外監皆領器仗兵役油壁車者加青油衣於車壁也王儉議曰衾書車十二乘古副車之象也榆轂輪簟子壁綠油衣說輸芮翻下之說說子因說同】出天子置朝堂夾輔號令【朝直遥翻】粲等閉城門上仗誰敢不同【上時掌翻下直上西上同】東城人正共縛送蕭令耳【東城謂東府城也按蕭子顯齊書世祖遺詔以鸞為侍中尚書令此時已進錄尚書事粲曰蕭令蓋以舊官稱之】子隆欲定計鏘以上臺兵力既悉度東府【海陵王既即位鸞出鎮東府上臺兵力悉割以自隨度過也】且慮事不捷意甚猶豫馬隊主劉巨世祖時舊人詣鏘請間叩頭勸鏘立事鏘命駕將入復還内【復扶又翻】與母陸太妃别日暮不成行典籖知其謀告之癸酉鸞遣兵二千人圍鏘第殺鏘遂殺子隆及謝粲等於時太祖諸子子隆最壯大有才能【太祖當作世祖】故鸞尤忌之江州刺史晉安王子懋聞鄱陽隨王死欲起兵謂防閤吳郡陸超之曰事成則宗廟獲安不成猶為義鬼【諸王置防閤以勇畧之士為之以防衛齋閤杜佑通典唐制親王府並給防閤庶僕白直下至州縣亦有白直】防閤丹陽董僧慧曰此州雖小宋孝武常用之【謂宋孝武帝自江州起兵誅元凶劭也】若舉兵向闕以請鬰林之罪誰能禦之子懋母阮氏在建康密遣書迎之阮氏報其同母兄于瑶之為計瑶之馳告宣城公鸞乙亥假鸞黄鉞内外纂嚴 【考異曰齊帝紀作乙未按是月壬申朔上有癸未而下有乙酉丁亥蓋癸未當作癸酉乙未當作乙亥耳】遣中護軍王玄邈討子懋又遣軍主裴叔業與于瑶之先襲尋陽聲云為郢府司馬子懋知之遣三百人守湓城叔業泝流直上【上時掌翻】至夜回襲湓城城局參軍樂賁開門納之【諸州刺史各有城局參軍掌修浚備禦】子懋聞之帥府州兵力據城自守子懋部曲多雍州人皆勇躍願奮【子懋自雍州徙為江州故部曲多雍州人勇當作踴帥讀曰率雍於用翻】叔業畏之遣于瑶之說子懋曰今還都必無過憂正當作散官不失富貴也【說輸芮翻散悉但翻】子懋既不出兵攻叔業衆情稍沮中兵參軍于琳之瑶之兄也說子懋重賂叔業可以免禍子懋使琳之往琳之因說叔業取子懋叔業遣軍主徐玄慶將四百人隨琳之入州城僚佐皆奔散【沮在呂翻說輸芮翻將即亮翻】琳之從二百人拔白刃入齋子懋罵曰小人何忍行此琳之以袖障面使人殺之王玄邈執董僧慧將殺之僧慧曰晉安舉義兵僕實預其謀得為主人死不恨矣願至大斂畢退就鼎鑊【為于偽翻斂力贍翻下殯斂同鑊戶郭翻】玄邈義之具以白鸞免死配東冶子懋子昭基九歲以方二寸絹為書參其消息并遺錢五百【遺于季翻】行金得逹僧慧視之曰郎君書也悲慟而卒【卒子恤翻】于琳之勸陸超之逃亡超之曰人皆有死此不足懼吾若逃亡非唯孤晉安之眷亦恐田横客笑人【田横客事見十一卷漢高帝五年超之守死故以此言愧琳之】玄邈等欲囚以還都超之端坐俟命超之門生謂殺超之當得賞密自後斬之頭墜而身不僵【僵居良翻】玄邈厚加殯斂門生亦助舉棺棺墜壓其首折頸而死【史言董僧慧陸超之之義烈折而設翻】鸞遣平西將軍王廣之襲南兖州刺史安陸王子敬廣之至歐陽【歐陽今真州閘即其地也】遣部將濟隂陳伯之先驅【將即亮翻濟子禮翻】伯之因城開獨入斬子敬鸞又遣徐玄慶西上害諸王【上時掌翻】臨海王昭秀為荆州刺史西中郎長史何昌㝢行州事玄慶至江陵欲以便宜從事昌㝢曰僕受朝廷意寄【意寄謂屬意寄託之】翼輔外藩殿下未有愆失君以一介之使來何容即以相付邪【使疏吏翻】若朝廷必須殿下當自啓聞更聽後旨昭秀由是得還建康 【考異曰南史明帝使裴叔業賫旨詔昌㝢令以便宜從事昌㝢拒之曰臨海王未有失寜得從君單詔邪即時自有啓聞須反更議叔業曰若爾便是拒詔拒詔軍法行事答曰能見殺者君也能拒詔者僕也叔業不敢逼而退昭秀由此得還都今從齊書】昌㝢尚之之弟子也【何昌㝢於此有周昌之節矣】鸞以吳興太守孔琇之行郢州事欲使之殺晉熙王銶【琇音秀銶音求】琇之辭不許遂不食而死琇之靖之孫也【孔靖見一百二十三卷晉安帝元興二年】裴叔業自尋陽仍進向湘州欲殺湘州刺史南平王銳防閤周伯玉大言於衆曰此非天子意今斬叔業舉兵匡社稷誰敢不從銳典籖叱左右斬之乙酉殺銳又殺郢州刺史晉熙王銶南豫州刺史宜都王鏗丁亥以廬陵王子卿為司徒桂陽王鑠為中軍將軍<br />
<br />
  開府儀同三司 冬十月丁酉解嚴【尋陽已定諸藩王已死故解嚴】以宣城公鸞為太傅領大將軍揚州牧都督中外諸軍事加殊禮進爵為王宣城王謀繼大統多引朝廷名士與參籌策侍中謝朏心不願乃求出為吳興太守至郡致酒數斛遺其弟吏部尚書瀹【朏敷尾翻遺于季翻】為書曰可力飲此勿豫人事<br />
<br />
  臣光曰臣聞衣人之衣者懷人之憂食人之食者死人之事【史記載淮陰侯答蒯徹之言衣人之衣於既翻】二謝兄弟比肩貴近安享榮祿危不預知為臣如此可謂忠乎【世多有如此而得名者】<br />
<br />
  宣城王雖專國政人情猶未服王胛上有赤誌【胛古洽翻肩背之間為胛】驃騎諮議參軍考城江祏勸王出以示人【祐音石考城前漢之甾縣也屬梁國後漢章帝改曰考城屬陳留郡晉惠帝分屬濟陽郡蕭子顯齊志南徐州南濟陽郡有考城縣皆晉氏因郡人南渡而僑置也】王以示晉夀太守王洪範曰人言此是日月相卿幸勿泄洪範曰公日月在軀如何可隱當轉言之【王洪範禁衛舊臣鸞以此覘之其言如此鸞益無所忌矣相息亮翻】王母祏之姑也 戊戌殺桂陽王鑠衡陽王鈞江夏王鋒建安王子真巴陵王子倫鑠與鄱陽王鏘齊名鏘好文章鑠好名理【好呼到翻】時人稱為鄱桂鏘死鑠不自安至東府見宣城王還謂左右曰向錄公見接慇勤【鸞以太傳錄尚書事太傅上公故稱錄公】流連不能已【流連不能相捨之意】而面有慙色此必欲殺我是夕遇害宣城王每殺諸王常夜遣兵圍其第斬關踰垣呼譟而入家貲皆封籍之江夏王鋒有才行【行下孟翻】宣城王嘗與之言遥光才力可委鋒曰遥光之於殿下猶殿下之於高皇衛宗廟安社稷實有攸寄【東昏侯之世遥光卒如鋒言】宣城王失色及殺諸王鋒遺宣城王書誚責之【遺于季翻誚才笑翻】宣城王深憚之不敢於第收鋒使兼祠官於太廟【祠官使行祭事也】夜遣兵廟中收之鋒出登車兵人欲上車【上時掌翻】鋒有力手撃數人皆仆地然後死宣城王遣典籖柯令孫殺建安王子真【姓譜柯姓也吳公子柯廬之後】子真走入牀下令孫手牽出之叩頭乞為奴不許而死又遣中書舍人茹法亮殺巴陵王子倫【茹音如】子倫性英果時為南蘭陵太守鎮琅邪城有守兵【晉置南琅邪郡於江乘蒲洲上齊徙治白下北臨江滸故有守兵】宣城王恐不肯就死以問典籖華伯茂【華戶化翻】伯茂曰公若以兵取之恐不可即辦若委伯茂一夫力耳乃手自執鴆逼之子倫正衣冠出受詔謂法亮曰先朝昔滅劉氏【見一百三十五卷高祖建元元年朝直遥翻】今日之事理數固然君是身家舊人【茹灋亮事世祖權寄甚重】今衘此使當由事不獲已【使疏吏翻】此酒非勸酬之爵因仰之而死時年十六灋亮及左右皆流涕初諸王出鎮皆置典籖主帥一方之事悉以委之【帥所類翻下同】時入奏事一歲數返時主輒與之間語【閒讀曰閑】訪以州事刺史美惡專繫其口自刺史以下莫不折節奉之恒慮弗及【恒戶登翻】於是威行州部【州部謂一州之部内也】大為姦利武陵王曅為江州性烈直不可干典籖趙渥之謂人曰今出都易刺史及見世祖盛毁之曅遂免還南海王子罕戍琅邪欲暫游東堂典籖姜秀不許子罕還泣謂母曰兒欲移五步亦不得與囚何異邵陵王子貞嘗求熊白【本草圖經曰熊形類犬豕而性輕健好攀緣上高木見人則顛倒自投而下冬多入宂而藏蟄始春而出其脂謂之熊白十一月取之須其背上者陸佃埤雅曰熊當心有白脂如玉味甚美俗呼熊白】厨人答典籖不在不敢與永明中巴東王子響殺劉寅等【事見一百三十八卷永明八年】世祖聞之謂羣臣曰子響遂反戴僧靜大言曰諸王都自應反豈唯巴東上問其故對曰大主無罪而一時被囚【被皮義翻】取一挺藕一杯漿皆諮籖帥籖帥不在則竟日忍渴諸州唯聞有籖帥不聞有刺史何得不反竟陵王子良嘗問衆曰士大夫何意詣籖帥參軍范雲曰詣長史以下皆無益詣籖帥立有倍本之價【謂所持以詣籖帥而其所得倍其所持之本也】不詣謂何子良有愧色及宣城王誅諸王皆令典籖殺之竟無一人能抗拒者孔珪聞之流涕曰齊之衡陽江夏最有意【言有意於翼輔帝室】而復害之【復扶又翻下勿復同】若不立籖帥故當不至於此【此上歷叙典籖之弊】宣城王亦深知典籖之弊乃詔自今諸州有急事當密以奏聞勿復遣典籖入都自是典籖之任浸輕矣<br />
<br />
  蕭子顯論曰帝王之子生長富厚【長知兩翻】朝出閨閫暮司方岳防驕翦逸積代常典故輔以上佐簡自帝心勞舊左右用為主帥飲食起居動應聞啓處地雖重【處昌呂翻】行已莫由威不在身恩未下及一朝艱難揔至望其釋位扶危何可得矣【左傳諸侯釋位以間王室杜預注曰間猶與也去其位與治王之政事】斯宋氏之餘風至齊室而尤弊也【諸王置典籖始於宋故云然】<br />
<br />
  癸卯以寜朔將軍蕭遥欣為豫州刺史黄門郎蕭遥昌為郢州刺史輔國將軍蕭誕為司州刺史遥昌遥欣之弟誕諶之兄也【史言宣城王用其親黨分據方面諶氏壬翻】 甲辰魏以太尉東陽王丕為太傅錄尚書事留守平城【守手又翻】 戊申魏主親告太廟使高陽王雍于烈奉遷神主于洛陽辛亥發平城 海陵王在位起居飲食皆諮宣城王而後行嘗思食蒸魚菜太官令答無錄公命竟不與辛亥皇太后令曰嗣主冲幼庶政多昧且早嬰尫疾【嬰纒也尫烏黄翻弱也杜預曰瘠疾也】弗克負荷【荷下可翻又如字】太傅宣城王胤體宣皇鍾慈太祖【蕭承之追謚宣皇帝太祖之父而鸞之祖也太祖又素愛鸞故云然】宜入承寶命帝可降封海陵王吾當歸老别館【蕭子顯齊書自此以上著於海陵王紀】且以宣城王為太祖第三子【蕭子顯齊書此語著於明帝紀】癸亥高宗即皇帝位大赦改元【此時方改元建武】以太尉王敬則為大司馬司空陳顯逹為太尉尚書令王晏加驃騎大將軍【驃匹妙翻騎奇寄翻】左僕射徐孝嗣加中軍大將軍中領軍蕭諶為領軍將軍度支尚書虞悰稱疾不陪位【悰徂宗翻】帝以悰舊人欲引參佐命使王晏齎廢立事示悰悰曰主上聖明公卿戮力寜假朽老以贊惟新乎【詩曰其命維新】不敢聞命因慟哭【史言虞悰柔而能正過謝瀹兄弟遠甚】朝議欲糾之【朝直遥翻】徐孝嗣曰此亦古之遺直乃止帝與羣臣宴會詔功臣上酒王晏等興席【上時掌翻興起也】謝瀹獨不起曰陛下受命應天順人王晏妄叨天功以為已力帝大笑解之座罷晏呼瀹共載還令省【令省謂尚書令所舍也】瀹正色曰卿巢窟在何處晏甚憚之 丁卯詔藩牧守宰或有薦獻事非任土【謂非如禹貢任土作貢也】悉加禁斷【斷音短】 己巳魏主如信都庚午詔曰比聞緣邊之蠻多竊掠南土【比毗至翻】使父子乖離室家分絶朕方蕩壹區宇子育萬姓若苟如此南人豈知朝德哉【謂江南之人將不知魏朝之德也朝直遥翻】可詔荆郢東荆三州禁勒蠻民勿有侵暴【魏初置荆州於上洛太和中徙治穰城置郢州於真陽真陽漢汝南郡之慎陽縣也置東荆州於沘陽】 十一月癸酉以始安王遥光為揚州刺史丁丑魏主如鄴 庚辰立皇子寶義為晉安王寶玄<br />
<br />
  為江夏王【夏戶雅翻】寶源為廬陵王寶寅為建安王寶融為隨郡王寶攸為南平王 甲申詔曰邑宰祿薄雖任土恒貢自今悉斷【觀此則江左之政縣邑不由郡州亦得入貢天臺矣】 乙酉追尊始安真王為景皇妃為懿后 丙戌以聞喜公遥欣為荆州刺史豐城公遥昌為豫州刺史時上長子晉安王寶義有廢疾【痼疾不可復用為廢疾長知兩翻】諸子皆弱小故以遥光居中【居中謂為揚州刺史】遥欣鎮撫上流 戊子立皇子寶卷為太子【卷讀曰捲】 魏主至洛陽欲澄清流品以尚書崔亮兼吏部郎亮道固之兄孫也【宋泰始初崔道固降魏】 魏主敇後軍宇文福行牧地福表石濟以西河内以東距河凡十里【行下孟翻牧地縱則石濟以西河内以東横則距河十里按杜佑通典衛州汲縣古牧野之地則其地宜畜牧有自來矣】魏主自代徙雜畜置其地使福掌之畜無耗失【畜許救翻】以為司衛監初世祖平統萬及秦涼【宋文帝元嘉四年魏平統萬八年赫連定滅秦定尋西奔為吐谷渾所禽秦地皆入于魏十六年魏平凉州】以河西水草豐美用為牧地畜甚蕃息【蕃讀如繁】馬至二百餘萬匹槖駞半之牛羊無數及高祖置牧場於河陽常畜戎馬十萬匹【河陽牧場即宇文福所規牧地畜許六翻】每歲自河西徙牧并州稍復南徙【復扶又翻】欲其漸習水土不至死傷而河西之牧愈更蕃滋及正光以後皆為寇盜所掠無孑遺矣【梁武帝普通元年魏改元正光史歷言魏之馬政】 永明中御史中丞沈淵表百官年七十皆令致仕【用古者七十而致事之說】並窮困私門庚子詔依舊銓叙上輔政所誅諸王皆復屬籍封其子為侯 上詐稱海陵恭王有疾數遣御師瞻視【數所角翻御師醫師也以其供御故謂之御師至于隋世尚藥局有侍御醫又有醫師】因而殞之葬禮並依漢東海恭王故事【漢東海王彊以天下讓葬用殊禮】 魏郢州刺史韋珍【韋珍先以樂陵鎮將與東荆州刺史桓誕同鎮沘陽尋為郢州刺史】在州有聲績魏主賜以駿馬穀帛珍集境内孤貧者悉散與之謂之曰天子以我能綏撫卿等故賜以穀帛吾何敢獨有之 魏主以上廢海陵王自立謀大舉入寇會邊將言雍州刺史下邳曹虎遣使請降於魏【將即亮翻雍於用翻使疏吏翻降戶江翻】十一月辛丑朔魏遣行征南將軍薛真度督四將向襄陽大將軍劉昶平南將軍王肅向義陽徐州刺史拓跋衍向鍾離平南將軍廣平劉藻向南鄭真度安都從祖弟也【從才用翻】以尚書僕射盧淵為安南將軍督襄陽前鋒諸軍淵辭以不習軍旅不許淵曰但恐曹虎為周魴耳【周魴事見七十一卷魏明帝太和二年魴符方翻】魏主欲變易舊風壬寅詔禁士民胡服國人多不悦【國人者與魏同起於北荒之子孫也】通直散騎常侍劉芳纘之族弟也【劉纘臣於齊而屢使於魏與芳皆彭城人蓋同出於楚元王之後】與給事黄門侍郎太原郭祚皆以文學為帝所親禮多引與講論及密議政事大臣貴戚皆以為踈已怏怏有不平之色【怏許兩翻】帝使給事黄門侍郎陸凱私諭之曰至尊但欲廣知古事詢訪前世法式耳終不親彼而相疎也衆意乃稍解凱馛之子也【陸馛見一百三十三卷宋明帝泰始七年馛蒲撥翻】 魏主欲自將入寇癸卯中外戒嚴戊申詔代民遷洛者復租賦三年【復方目翻】相州刺史高閭【相息亮翻】上表稱洛陽草創曹虎既不遣質任必無誠心【質音致】無宜輕舉魏主不從久之虎使竟不再來【使疏吏翻】魏主引公卿問行留之計公卿或以為宜止或以為宜行帝曰衆人紛紜莫知所從必欲盡行留之勢宜有客主共相起任城鎮南為留議【鎮南為鎮軍任音壬】朕為行論諸公坐聽得失長者從之衆皆曰諾鎮軍將軍李冲曰臣等正以遷都草創人思少安【少詩沼翻】為内應者未得審諦【諦音帝亦審也】不宜輕動帝曰彼降欵虛實誠未可知【降中江翻】若其虛也朕巡撫淮甸訪民疾苦使彼知君德之所在有北向之心若其實也今不以時應接則失乘時之機孤歸義之誠敗朕大畧矣【孤負也敗補邁翻】任城王澄曰虎無質任又使不再來其詐可知也今代都新遷之民皆有戀本之心扶老攜幼始就洛邑居無一椽之室【椽重緣翻】食無甔石之儲【應劭曰齊人名小甕為甔甔受二石甔音都濫翻】又冬月垂盡東作將起乃百堵皆興俶載南畝之時【百堵皆興謂新遷之人當作室也俶載南畝謂入春當東作也二語皆詩語俶昌六翻始也】而驅之使擐甲執兵泣當白刃殆非歌舞之師也【武王伐紂前歌後舞擐音宦】且諸軍已進非無應接若降欵有實待既平樊沔然後鑾輿順動亦何晩之有今率然輕舉【率然輕易之意】上下疲勞若空行空返恐挫損天威更成賊氣非策之得者也司空穆亮以為宜行公卿皆同之澄謂亮曰公輩在外之時見張旗授甲皆有憂色平居論議不願南征何得對上即為此語面背不同事涉欺佞豈大臣之義國士之體乎萬一傾危皆公輩所為也冲曰任城王可謂忠於社稷帝曰任城以從朕者為佞不從朕者豈必皆忠夫小忠者大忠之賊無乃似諸澄曰臣愚闇雖涉小忠要是竭誠謀國不知大忠者竟何所據帝不從辛亥發洛陽以北海王詳為尚書僕射統留臺事李冲兼僕射同守洛陽給事黄門侍郎崔休為左丞趙郡王幹都督中外諸軍事始平王勰將宗子軍宿衛左右【將即亮翻】休逞之玄孫也【魏道武伐中山崔逞降之】戊辰魏主至懸瓠己巳詔夀陽鍾離馬頭之師所掠男女皆放還南曹虎果不降【降戶江翻】魏主命盧淵攻南陽淵以軍中乏糧請先攻赭陽以取葉倉魏主許之【赭陽即漢晉之堵陽縣堵亦音者至宋時猶屬南陽郡至蕭子顯齊書赭陽葉二縣皆不見於志下言北襄城太守成公期拒魏則北襄城郡置于赭陽明矣葉式涉翻】乃與征南大將軍城陽王鸞安南將軍李佐荆州刺史韋珍共攻赭陽 【考異曰齊書作盧陽烏韋靈智按陽烏淵小字靈智珍字也】鸞長夀之子【城陽王長夀見一百三十二年宋蒼梧王元徽三年】佐寶之子也【宋文帝元嘉二十一年李寶入朝于魏】北襄城太守成公期閉城拒守薛真度軍于沙堨【堨烏葛翻堨壅也聚沙以壅水故以為地名】南陽太守房伯玉新野太守劉思忌拒之【晉武帝太康中分南陽置義陽郡惠帝又分義陽南陽置新野郡】 先是魏主遣中書監高閭治古樂【先悉薦翻治直之翻】會閭出為相州刺史是歲表薦著作郎韓顯宗太樂祭酒公孫崇參知鐘律帝從之【太樂祭酒蓋太和中初置是官】<br />
<br />
  資治通鑑卷一百三十九<br />
<br />
<史部,編年類,資治通鑑>  <br>
   </div> 

<script src="/search/ajaxskft.js"> </script>
 <div class="clear"></div>
<br>
<br>
 <!-- a.d-->

 <!--
<div class="info_share">
</div> 
-->
 <!--info_share--></div>   <!-- end info_content-->
  </div> <!-- end l-->

<div class="r">   <!--r-->



<div class="sidebar"  style="margin-bottom:2px;">

 
<div class="sidebar_title">工具类大全</div>
<div class="sidebar_info">
<strong><a href="http://www.guoxuedashi.com/lsditu/" target="_blank">历史地图</a></strong>  
<a href="http://www.880114.com/" target="_blank">英语宝典</a>  
<a href="http://www.guoxuedashi.com/13jing/" target="_blank">十三经检索</a> 
<br><strong><a href="http://www.guoxuedashi.com/gjtsjc/" target="_blank">古今图书集成</a></strong> 
<a href="http://www.guoxuedashi.com/duilian/" target="_blank">对联大全</a> <strong><a href="http://www.guoxuedashi.com/xiangxingzi/" target="_blank">象形文字典</a></strong> 

<br><a href="http://www.guoxuedashi.com/zixing/yanbian/">字形演变</a>  <strong><a href="http://www.guoxuemi.com/hafo/" target="_blank">哈佛燕京中文善本特藏</a></strong>
<br><strong><a href="http://www.guoxuedashi.com/csfz/" target="_blank">丛书&方志检索器</a></strong> <a href="http://www.guoxuedashi.com/yqjyy/" target="_blank">一切经音义</a>  

<br><strong><a href="http://www.guoxuedashi.com/jiapu/" target="_blank">家谱族谱查询</a></strong>  <strong><a href="http://shufa.guoxuedashi.com/sfzitie/" target="_blank">书法字帖欣赏</a></strong> 
<br>

</div>
</div>


<div class="sidebar" style="margin-bottom:0px;">

<font style="font-size:22px;line-height:32px">QQ交流群9:489193090</font>


<div class="sidebar_title">手机APP 扫描或点击</div>
<div class="sidebar_info">
<table>
<tr>
	<td width=160><a href="http://m.guoxuedashi.com/app/" target="_blank"><img src="/img/gxds-sj.png" width="140"  border="0" alt="国学大师手机版"></a></td>
	<td>
<a href="http://www.guoxuedashi.com/download/" target="_blank">app软件下载专区</a><br>
<a href="http://www.guoxuedashi.com/download/gxds.php" target="_blank">《国学大师》下载</a><br>
<a href="http://www.guoxuedashi.com/download/kxzd.php" target="_blank">《汉字宝典》下载</a><br>
<a href="http://www.guoxuedashi.com/download/scqbd.php" target="_blank">《诗词曲宝典》下载</a><br>
<a href="http://www.guoxuedashi.com/SiKuQuanShu/skqs.php" target="_blank">《四库全书》下载</a><br>
</td>
</tr>
</table>

</div>
</div>


<div class="sidebar2">
<center>


</center>
</div>

<div class="sidebar"  style="margin-bottom:2px;">
<div class="sidebar_title">网站使用教程</div>
<div class="sidebar_info">
<a href="http://www.guoxuedashi.com/help/gjsearch.php" target="_blank">如何在国学大师网下载古籍?</a><br>
<a href="http://www.guoxuedashi.com/zidian/bujian/bjjc.php" target="_blank">如何使用部件查字法快速查字?</a><br>
<a href="http://www.guoxuedashi.com/search/sjc.php" target="_blank">如何在指定的书籍中全文检索?</a><br>
<a href="http://www.guoxuedashi.com/search/skjc.php" target="_blank">如何找到一句话在《四库全书》哪一页?</a><br>
</div>
</div>


<div class="sidebar">
<div class="sidebar_title">热门书籍</div>
<div class="sidebar_info">
<a href="/so.php?sokey=%E8%B5%84%E6%B2%BB%E9%80%9A%E9%89%B4&kt=1">资治通鉴</a> <a href="/24shi/"><strong>二十四史</strong></a>&nbsp; <a href="/a2694/">野史</a>&nbsp; <a href="/SiKuQuanShu/"><strong>四库全书</strong></a>&nbsp;<a href="http://www.guoxuedashi.com/SiKuQuanShu/fanti/">繁体</a>
<br><a href="/so.php?sokey=%E7%BA%A2%E6%A5%BC%E6%A2%A6&kt=1">红楼梦</a> <a href="/a/1858x/">三国演义</a> <a href="/a/1038k/">水浒传</a> <a href="/a/1046t/">西游记</a> <a href="/a/1914o/">封神演义</a>
<br>
<a href="http://www.guoxuedashi.com/so.php?sokeygx=%E4%B8%87%E6%9C%89%E6%96%87%E5%BA%93&submit=&kt=1">万有文库</a> <a href="/a/780t/">古文观止</a> <a href="/a/1024l/">文心雕龙</a> <a href="/a/1704n/">全唐诗</a> <a href="/a/1705h/">全宋词</a>
<br><a href="http://www.guoxuedashi.com/so.php?sokeygx=%E7%99%BE%E8%A1%B2%E6%9C%AC%E4%BA%8C%E5%8D%81%E5%9B%9B%E5%8F%B2&submit=&kt=1"><strong>百衲本二十四史</strong></a>  <a href="http://www.guoxuedashi.com/so.php?sokeygx=%E5%8F%A4%E4%BB%8A%E5%9B%BE%E4%B9%A6%E9%9B%86%E6%88%90&submit=&kt=1"><strong>古今图书集成</strong></a>
<br>

<a href="http://www.guoxuedashi.com/so.php?sokeygx=%E4%B8%9B%E4%B9%A6%E9%9B%86%E6%88%90&submit=&kt=1">丛书集成</a> 
<a href="http://www.guoxuedashi.com/so.php?sokeygx=%E5%9B%9B%E9%83%A8%E4%B8%9B%E5%88%8A&submit=&kt=1"><strong>四部丛刊</strong></a>  
<a href="http://www.guoxuedashi.com/so.php?sokeygx=%E8%AF%B4%E6%96%87%E8%A7%A3%E5%AD%97&submit=&kt=1">說文解字</a> <a href="http://www.guoxuedashi.com/so.php?sokeygx=%E5%85%A8%E4%B8%8A%E5%8F%A4&submit=&kt=1">三国六朝文</a>
<br><a href="http://www.guoxuedashi.com/so.php?sokeytm=%E6%97%A5%E6%9C%AC%E5%86%85%E9%98%81%E6%96%87%E5%BA%93&submit=&kt=1"><strong>日本内阁文库</strong></a> <a href="http://www.guoxuedashi.com/so.php?sokeytm=%E5%9B%BD%E5%9B%BE%E6%96%B9%E5%BF%97%E5%90%88%E9%9B%86&ka=100&submit=">国图方志合集</a> <a href="http://www.guoxuedashi.com/so.php?sokeytm=%E5%90%84%E5%9C%B0%E6%96%B9%E5%BF%97&submit=&kt=1"><strong>各地方志</strong></a>

</div>
</div>


<div class="sidebar2">
<center>

</center>
</div>
<div class="sidebar greenbar">
<div class="sidebar_title green">四库全书</div>
<div class="sidebar_info">

《四库全书》是中国古代最大的丛书,编撰于乾隆年间,由纪昀等360多位高官、学者编撰,3800多人抄写,费时十三年编成。丛书分经、史、子、集四部,故名四库。共有3500多种书,7.9万卷,3.6万册,约8亿字,基本上囊括了古代所有图书,故称“全书”。<a href="http://www.guoxuedashi.com/SiKuQuanShu/">详细>>
</a>

</div> 
</div>

</div>  <!--end r-->

</div>
<!-- 内容区END --> 

<!-- 页脚开始 -->
<div class="shh">

</div>

<div class="w1180" style="margin-top:8px;">
<center><script src="http://www.guoxuedashi.com/img/plus.php?id=3"></script></center>
</div>
<div class="w1180 foot">
<a href="/b/thanks.php">特别致谢</a> | <a href="javascript:window.external.AddFavorite(document.location.href,document.title);">收藏本站</a> | <a href="#">欢迎投稿</a> | <a href="http://www.guoxuedashi.com/forum/">意见建议</a> | <a href="http://www.guoxuemi.com/">国学迷</a> | <a href="http://www.shuowen.net/">说文网</a><script language="javascript" type="text/javascript" src="https://js.users.51.la/17753172.js"></script><br />
  Copyright &copy; 国学大师 古典图书集成 All Rights Reserved.<br>
  
  <span style="font-size:14px">免责声明:本站非营利性站点,以方便网友为主,仅供学习研究。<br>内容由热心网友提供和网上收集,不保留版权。若侵犯了您的权益,来信即刪。scp168@qq.com</span>
  <br />
ICP证:<a href="http://www.beian.miit.gov.cn/" target="_blank">鲁ICP备19060063号</a></div>
<!-- 页脚END --> 
<script src="http://www.guoxuedashi.com/img/plus.php?id=22"></script>
<script src="http://www.guoxuedashi.com/img/tongji.js"></script>

</body>
</html>
