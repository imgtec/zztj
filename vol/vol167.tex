<!DOCTYPE html PUBLIC "-//W3C//DTD XHTML 1.0 Transitional//EN" "http://www.w3.org/TR/xhtml1/DTD/xhtml1-transitional.dtd">
<html xmlns="http://www.w3.org/1999/xhtml">
<head>
<meta http-equiv="Content-Type" content="text/html; charset=utf-8" />
<meta http-equiv="X-UA-Compatible" content="IE=Edge,chrome=1">
<title>資治通鑒_168-資治通鑑卷一百六十七_168-資治通鑑卷一百六十七</title>
<meta name="Keywords" content="資治通鑒_168-資治通鑑卷一百六十七_168-資治通鑑卷一百六十七">
<meta name="Description" content="資治通鑒_168-資治通鑑卷一百六十七_168-資治通鑑卷一百六十七">
<meta http-equiv="Cache-Control" content="no-transform" />
<meta http-equiv="Cache-Control" content="no-siteapp" />
<link href="/img/style.css" rel="stylesheet" type="text/css" />
<script src="/img/m.js?2020"></script> 
</head>
<body>
 <div class="ClassNavi">
<a  href="/24shi/">二十四史</a> | <a href="/SiKuQuanShu/">四库全书</a> | <a href="http://www.guoxuedashi.com/gjtsjc/"><font  color="#FF0000">古今图书集成</font></a> | <a href="/renwu/">历史人物</a> | <a href="/ShuoWenJieZi/"><font  color="#FF0000">说文解字</a></font> | <a href="/chengyu/">成语词典</a> | <a  target="_blank"  href="http://www.guoxuedashi.com/jgwhj/"><font  color="#FF0000">甲骨文合集</font></a> | <a href="/yzjwjc/"><font  color="#FF0000">殷周金文集成</font></a> | <a href="/xiangxingzi/"><font color="#0000FF">象形字典</font></a> | <a href="/13jing/"><font  color="#FF0000">十三经索引</font></a> | <a href="/zixing/"><font  color="#FF0000">字体转换器</font></a> | <a href="/zidian/xz/"><font color="#0000FF">篆书识别</font></a> | <a href="/jinfanyi/">近义反义词</a> | <a href="/duilian/">对联大全</a> | <a href="/jiapu/"><font  color="#0000FF">家谱族谱查询</font></a> | <a href="http://www.guoxuemi.com/hafo/" target="_blank" ><font color="#FF0000">哈佛古籍</font></a> 
</div>

 <!-- 头部导航开始 -->
<div class="w1180 head clearfix">
  <div class="head_logo l"><a title="国学大师官网" href="http://www.guoxuedashi.com" target="_blank"></a></div>
  <div class="head_sr l">
  <div id="head1">
  
  <a href="http://www.guoxuedashi.com/zidian/bujian/" target="_blank" ><img src="http://www.guoxuedashi.com/img/top1.gif" width="88" height="60" border="0" title="部件查字,支持20万汉字"></a>


<a href="http://www.guoxuedashi.com/help/yingpan.php" target="_blank"><img src="http://www.guoxuedashi.com/img/top230.gif" width="600" height="62" border="0" ></a>


  </div>
  <div id="head3"><a href="javascript:" onClick="javascript:window.external.AddFavorite(window.location.href,document.title);">添加收藏</a>
  <br><a href="/help/setie.php">搜索引擎</a>
  <br><a href="/help/zanzhu.php">赞助本站</a></div>
  <div id="head2">
 <a href="http://www.guoxuemi.com/" target="_blank"><img src="http://www.guoxuedashi.com/img/guoxuemi.gif" width="95" height="62" border="0" style="margin-left:2px;" title="国学迷"></a>
  

  </div>
</div>
  <div class="clear"></div>
  <div class="head_nav">
  <p><a href="/">首页</a> | <a href="/ShuKu/">国学书库</a> | <a href="/guji/">影印古籍</a> | <a href="/shici/">诗词宝典</a> | <a   href="/SiKuQuanShu/gxjx.php">精选</a> <b>|</b> <a href="/zidian/">汉语字典</a> | <a href="/hydcd/">汉语词典</a> | <a href="http://www.guoxuedashi.com/zidian/bujian/"><font  color="#CC0066">部件查字</font></a> | <a href="http://www.sfds.cn/"><font  color="#CC0066">书法大师</font></a> | <a href="/jgwhj/">甲骨文</a> <b>|</b> <a href="/b/4/"><font  color="#CC0066">解密</font></a> | <a href="/renwu/">历史人物</a> | <a href="/diangu/">历史典故</a> | <a href="/xingshi/">姓氏</a> | <a href="/minzu/">民族</a> <b>|</b> <a href="/mz/"><font  color="#CC0066">世界名著</font></a> | <a href="/download/">软件下载</a>
</p>
<p><a href="/b/"><font  color="#CC0066">历史</font></a> | <a href="http://skqs.guoxuedashi.com/" target="_blank">四库全书</a> |  <a href="http://www.guoxuedashi.com/search/" target="_blank"><font  color="#CC0066">全文检索</font></a> | <a href="http://www.guoxuedashi.com/shumu/">古籍书目</a> | <a   href="/24shi/">正史</a> <b>|</b> <a href="/chengyu/">成语词典</a> | <a href="/kangxi/" title="康熙字典">康熙字典</a> | <a href="/ShuoWenJieZi/">说文解字</a> | <a href="/zixing/yanbian/">字形演变</a> | <a href="/yzjwjc/">金 文</a> <b>|</b>  <a href="/shijian/nian-hao/">年号</a> | <a href="/diming/">历史地名</a> | <a href="/shijian/">历史事件</a> | <a href="/guanzhi/">官职</a> | <a href="/lishi/">知识</a> <b>|</b> <a href="/zhongyi/">中医中药</a> | <a href="http://www.guoxuedashi.com/forum/">留言反馈</a>
</p>
  </div>
</div>
<!-- 头部导航END --> 
<!-- 内容区开始 --> 
<div class="w1180 clearfix">
  <div class="info l">
   
<div class="clearfix" style="background:#f5faff;">
<script src='http://www.guoxuedashi.com/img/headersou.js'></script>

</div>
  <div class="info_tree"><a href="http://www.guoxuedashi.com">首页</a> > <a href="/SiKuQuanShu/fanti/">四库全书</a>
 > <h1>资治通鉴</h1> <!--         下载:【右键另存为】即可 --></div>
  <div class="info_content zj clearfix">
  
<div class="info_txt clearfix" id="show">
<center style="font-size:24px;">168-資治通鑑卷一百六十七</center>
    資治通鑑卷一百六十七 宋 司馬光 撰<br />
<br />
  胡三省 音註<br />
<br />
  陳紀一【起彊圉赤奮若盡屠維單閼凡三年 武帝既有功於梁自以為姓出於陳自吳興郡公進封陳公及受命國遂號曰東】<br />
<br />
  高祖武皇帝【諱霸先字興國小字法生姓陳氏吳興長城下若里人】<br />
<br />
  永定元年【是年十月受禪始改元永定自十月以前猶是梁太平二年】春正月辛丑周公即天王位【諱覺字陁羅尼宇文泰第三子也宇文輔政慕倣周禮泰卒覺嗣遂封周公既受命國號曰周】柴燎告天朝百官於露門【露門即古之路門路大也宇文建國率倣古制故外朝曰路門鄭玄曰外門曰臯門朝門曰應門内有路門孔頴達曰爾雅門屛之間謂之宁郭曰人主視朝所宁立處李巡曰正門外兩塾間曰宁謂天子受朝於路門之外朝於門外而宁立以待諸侯之至故云當宁而立也然門外有屛者即樹塞門是也爾雅云正門謂之應門又云屛謂之樹李巡曰常當門自蔽名曰樹郭云小牆當門中案李郭二注以推驗禮文諸侯内屛在路門之内天子外屛在路門之外而近應門周制天子三朝其一在路門内謂之燕朝太僕掌之故太僕云王眂燕朝則正其位文王世子云公族朝於内朝親之也此則王與宗人圖其嘉事及王退侯大夫之朝也其一是路門外之朝謂之治朝司士掌之故司士云正朝儀之位此是每日視朝之位其王與諸侯賓射亦與治朝同其三是臯門之内庫門之外謂之外朝朝士掌之此是詢衆庶之朝也朝直遥翻】追尊王考文公為文王妣為文后大赦封魏恭帝為宋公以木德承魏水行夏之時服色尚黑【行夏之時用寅正也服色尚黑隨水行也夏戶雅翻】以李弼為太師趙貴為太傅大冢宰獨孤信為太保大宗伯中山公護為大司馬【後周太祖初據關右宫名未改魏號及域内粗定改定章程命尚書令盧辯遠師周制置三公三孤以為論道之官次置六卿以分司庶務閔帝受禪大司馬掌兵宇文護居之以專兵要】 詔以王琳為司空驃騎大將軍【驃匹妙翻騎奇寄翻】以尚書右僕射王通為左僕射 周王祀圜丘自謂先世出於神農以神農配二丘【宇文氏自謂其先出於炎帝炎帝為黄帝所滅子孫遁去朔野其後有葛烏兔者雄武多算畧鮮卑奉以為主裔孫曰普囘因狩得玉璽三紐文曰皇帝璽以為天授其俗謂天子曰宇文遂以為氏及國號二丘圜丘方丘也】始祖獻侯配南北郊【普囘子莫那自隂山南徙始居遼西諡曰獻侯】文王配明堂廟號太祖癸卯祀方丘甲辰祭太社【五代志後周憲章姬周祭祀之式多依儀禮司量掌為壇之制圜丘三成崇一丈二尺深二丈上徑六丈十有二階每等十有二節在國陽七里之郊圓壝徑三百步内壝半之方一成下崇一丈徑六丈八尺上崇五尺方四丈八方方一階階十級級一尺方丘在國隂十里之郊丘一成八方下崇一丈方六丈八尺上崇五尺方四丈八方一階級一尺其壝八面徑百二十步内壝半之南郊為方壇於國南五里其崇一丈二尺其廣二丈其壝方百二十步内壝半之神州之壇崇一丈方四丈在北郊方丘之右其壝如方丘其祭圓丘及南郊並正月上辛右宗廟而左社稷皇帝親祀社稷冢宰亞獻宗伯終獻】除市門稅【魏末盗賊羣起國用不足稅入市門者人一錢今除之】乙巳享太廟仍用鄭玄議立太祖與二昭二穆為五廟其有德者别為祧廟不毁【記王制曰天子七廟三昭三穆與太祖之廟而七鄭元注云此周制七者太祖及文王武王之祧與親廟四决疑要注曰凡昭穆父南面故曰昭昭明也子北面故曰穆穆順也李涪曰昭本如字為漢諱昭改曰韶或曰晉文帝名昭故讀曰韶】辛亥祀南郊壬子立王后元氏后魏文帝之女晉安公主也 齊南安城主馮顯請降於周【降戶江翻】周柱國宇文貴使豐州刺史太原郭彦將兵迎之【五代志永安郡黄岡縣齊曰南安又魏收志天平初置南安郡屬襄州後陷以地考之當在五代志之頴川郡葉縣界又五代志淅陽郡武當縣舊置武當郡及始平郡後改為齊興郡梁置興州後周改為豐州隋為均州將即亮翻】遂據南安 吐谷渾為宼於周攻凉鄯河三州【武威郡凉州西平郡鄯州枹罕郡河州吐從暾入聲谷音浴鄯時戰翻】秦州都督遣渭州刺史于翼赴援【魏黄初三年始置都督諸州軍事又有都督中外諸軍其任尤重南北朝皆因之而軍行又有都督諸軍左右前後中軍大都督内而宿衛有正副都督散都督帥都督旅都督外而州郡有防城都督帳内都督都督之名雖同其位任懸絶矣此秦州都督盖都督河渭凉鄯諸州也後周九命之制都督八命其授柱國大將軍開府儀同者竝皆使持節大都督盖九命也】翼不從僚屬咸以為言翼曰攻取之術非夷俗所長此宼之來不過抄掠邊牧掠而無獲勢將自走勞師而往必無所及翼揣之已了【了明也抄楚交翻揣初委翻】幸勿復言【復扶又翻】數日間至果如翼所策 初梁世祖以始興郡為東衡州以歐陽頠為刺史久之徙頠為郢州刺史蕭勃留頠不遣【頠魚委翻】世祖以王琳代勃為廣州刺史勃遣其將孫盪監廣州盡帥所部屯始興以避之【見一百六十五卷梁元帝承聖三年將即亮翻下同監工銜翻帥讀曰率】頠别據一城不往謁閉門自守勃怒遣兵襲之盡收其貨財馬仗尋赦之使復其所與之結盟江陵陷頠遂事勃二月庚午勃起兵於廣州遣頠及其將傅泰蕭孜為前軍孜勃之從子也【從才用翻考異曰陳書南史周文育傳皆作子今從梁書帝紀】南江州刺史余孝頃以兵會之【孝頃據新吳盖就置南江州命為刺史 考異曰典畧作南康州刺史今從梁書】詔平西將軍周文育帥諸軍討之【帥讀曰率】 癸酉周王朝日於東郊【朝直遥翻下同】戊寅祭大社 周楚公趙貴衛公獨孤信故皆與太祖等夷【宇文泰廟號太祖】及晉公護專政【宇文護自中山公進封晉公】皆怏怏不服【怏於兩翻】貴謀殺護信止之開府儀同三司宇文盛告之丁亥貴入朝護執而殺之免信官【此所謂主少國疑大臣未附之時也既殺趙貴護之威權成矣】 領軍將軍徐度出東關侵齊戊子至合肥燒齊船三千艘【艘蘇遭翻】 歐陽頠等出南康頠屯豫章之苦竹灘傅泰據蹠口城【蹠之石翻周文育傳作塶】余孝頃遣其弟孝勱守郡城自出豫章據石頭【水經注贑水逕豫章郡北水之西岸有石盤謂之石頭津步之處也汪藻曰自豫章絶江而西有山屹然並江而出者石頭渚也阻江負城十里而近即殷羨投書處勱音邁】巴山太守熊曇朗誘頠共襲高州刺史黄灋又語灋約共破頠【曇徒舍翻誘音酉語牛倨翻巨俱翻】且曰事捷與我馬仗遂出軍與頠俱進至灋城下曇朗陽敗走灋乘之頠失援而走曇朗取其馬仗歸於巴山周文育軍少船【少詩沼翻】余孝頃有船在上牢文育遣軍主焦僧度襲之盡取以歸仍於豫章立柵軍中食盡諸將欲退文育不許使人閒行遺周廸書【閒古莧翻遺于季翻】約為兄弟廸得書甚喜許饋以糧於是文育分遣老弱乘故船沿流俱下燒豫章柵偽若遁去者孝頃望之大喜不復設備文育由閒道兼行據芊韶【復扶又翻閒古莧翻據姚思亷梁書芊韶在巴山界芊音千】芊韶上流則歐陽頠蕭孜下流則傅泰余孝頃營文育據其中間築城饗士頠等大駭頠退入泥溪文育遣嚴威將軍周鐵虎等襲頠癸巳擒之文育盛陳兵甲與頠乘舟而宴廵蹠口城下使其將丁法洪攻泰擒之孜孝頃退走 甲午周以于謹為太傅大宗伯侯莫陳崇為太保晉公護為大冢宰【于謹侯莫陳崇既登公位宇文護若以序遷而為大冢宰實則周之元輔也】柱國武川賀蘭祥為大司馬高陽公達奚武為大司宼 周人殺魏恭帝 三月庚子周文育送歐陽頠傅泰於建康丞相霸先與頠有舊釋而厚待之【霸先發身於嶺南故與頠有舊】 周晉公護以趙景公獨孤信名重不欲顯誅之己酉逼令自殺 甲辰以司空王琳為湘郢二州刺史 曲江侯勃在南康聞歐陽頠等敗軍中忷懼【忷許拱翻】甲寅德州刺史陳法武【五代志日南郡梁置德州】前衡州刺史譚世遠攻勃殺之 夏四月己卯鑄四柱錢一當二十【梁末有兩柱錢及鵝眼錢時人雜用其價同但兩柱重鵝眼輕至是鑄四柱錢一當細錢二十】 齊遣使請和【使疏吏翻】 壬午周王謁成陵【周太祖陵曰成陵】乙酉還宫齊以太師斛律金為右丞相前大將軍可朱渾道元<br />
<br />
  為太傅【可朱渾道元前為車騎大將軍後齊置太師太傅太保是為三師擬古上公非勲德崇者不居次有大司馬大將軍是為二大並典司武事次置太尉司徒司空是為三公皆第一品其驃騎車騎二將軍加大者在開國郡公下開國郡公從一品】開府儀同三司賀拔仁為太保尚書令常山王演為司空錄尚書事長廣王湛為尚書令右僕射楊愔為左僕射仍加開府儀同三司【愔於今翻】幷省尚書右僕射崔暹為左僕射上黨王渙錄尚書事【自高歡居晉陽幷州有行臺尚書令僕等官及齊顯祖受魏禪遂以幷州行臺為并省位任亞於鄴省】 丁亥周王享太廟 壬辰改四柱錢一當十丙申復閉細錢【閉者閉絶不使行細錢民間私鑄者也時私錢細小交易以車載錢不復計數復扶又翻下復亦同】 故曲江侯勃主帥蘭敱襲殺譚世遠軍主夏侯明徹殺敱持勃首降【帥所類翻敱五哀翻降戶江翻下同】勃故記室李寶藏奉懷安侯任據廣州【任亦蕭氏子封懷安侯何承天志鬱林郡有懷安縣】蕭孜余孝頃猶據石頭為兩城各據其一多設船艦夾水而陳【艦戶黯翻下同陳讀曰陣】丞相霸先遣平南將軍侯安都助周文育擊之戊戍安都潜師夜燒其船艦文育帥水軍安都帥步軍進攻之【帥讀曰率】蕭孜出降孝頃逃歸新吳文育等引兵還丞相霸先以歐陽頠聲著南土復以頠為衡州刺史【衡上前有東字按五代志南海郡含洭縣梁置衡州始興縣梁置東衡州此時蓋命頠鎮含洭也】使討嶺南未至其子紇已克始興頠至嶺南諸郡皆降遂克廣州嶺南悉平【為歐陽頠父子世據嶺南張本】 周儀同三司齊軌謂御正中大夫薛善曰【五代志御正中大夫屬大冢宰五命】軍國之政當歸天子何得猶在權門善以告晉公護護殺之以善為中外府司馬【中外府都督中外諸軍事府】 五月戊辰余孝頃遣使詣丞相府乞降【使疏吏翻】 王琳既不就徵大治舟艦將攻陳霸先【治直之翻】六月戊寅霸先以開府儀同三司侯安都為西道都督周文育為南道都督將舟師二萬會武昌以擊之【將即亮翻下同】 秋七月辛亥周王享太廟 河南北大蝗齊主問魏郡丞崔叔瓚曰【五代志齊制上郡丞六品中七品下八品瓚藏旱翻】何故致蝗對曰五行志土功不時蝗蟲為災【五行志言蟲蝗之災率歸之勞民動衆故叔瓚云然】今外築長城内興三臺殆以此乎齊主怒使左右敺之【敺烏口翻】擢其髮以溷沃其頭【擢拔也溷戶困翻不潔也】曳足以出叔瓚季舒之兄也 八月丁卯周人歸梁世祖之柩及諸將家屬千餘人於王琳【柩音舊】 戊辰周王祭大社 甲午進丞相霸先位太傅加黄钺殊禮贊拜不名九月辛丑進丞相為相國總百揆封陳公備九錫陳國置百司 周孝愍帝性剛果惡晉公護之專權【惡烏路翻】司會李植自太祖時為相府司錄參掌朝政【周禮司會掌聽財用之會計以詔王及冢宰後周之制司會中大夫屬大冢宰五命柱國大將軍府長史司馬司錄正七命會古外翻相府司錄總錄相府之機務朝直遥翻】軍司馬孫恒亦久居權要【唐六典周禮大司馬屬官有軍司馬下大夫蓋兵部郎中之任也後周依周置官軍司馬中大夫五命恒戶登翻】及護執政植恒恐不見容乃與宫伯乙弗鳳賀拔提等共譖之於周王【周禮宫伯掌王宫之士庶子凡在版者掌其政令行其秩叙作其徒役之事以時頒其衣裳掌其誅賞屬天官冢宰後周左宫伯中大夫五命】植恒曰護自誅趙貴以來威權日盛謀臣宿將争往附之【將即亮翻】大小之政皆决於護以臣觀之將不守臣節願陛下早圖之王以為然鳳提曰以先王之明猶委植恒以朝政【朝直遥翻】今以事付二人何患不成且護常自比周公臣聞周公攝政七年【書曰惟周公誕保文武受命惟七年】陛下安能七年邑邑如此乎【邑邑不得志之貌】王愈信之數引武士於後園講習為執縛之勢【此事何必講習耶宜其謀泄也數所角翻】植等又引宫伯張光洛同謀光洛以告護護乃出植為梁州刺史恒為潼州刺史【五代志漢川郡置梁州金山郡西魏置潼州隋開皇五年改曰綿州】欲散其謀後王思植等每欲召之護泣諫曰天下至親無過兄弟若兄弟尚相疑它人誰可信者太祖以陛下富於春秋屬臣後事【屬之欲翻】臣情兼家國【以家則兄弟之親以國則君臣之義】實願竭其股肱若陛下親覽萬機威加四海臣死之日猶生之年但恐除臣之後姦囘得志非唯不利陛下亦將傾覆社稷使臣無面目見太祖於九泉且臣既為天子之兄位至宰相尚復何求【復扶又翻】願陛下勿信讒臣之言踈棄骨肉王乃止不召而心猶疑之鳳等益懼密謀滋甚刻日召羣公入醼【醼伊甸翻合飲也】因執護誅之張光洛又以告護護乃召柱國賀蘭祥【周書祥傳曰其先與魏俱起有紇伏者為賀蘭莫何弗因以為氏】領軍尉遲綱等謀之祥等勸護廢立時綱總領禁兵護遣綱入宫召鳳等議事及至以次執送護第因罷散宿衛兵王方悟獨在内殿令宫人執兵自守護遣賀蘭祥逼王遜位幽於舊第【略陽公舊第也】悉召公卿會議廢王為略陽公迎立岐州刺史寧都公毓【五代志西城郡安康縣舊曰寧都】公卿皆曰此公之家事敢不唯命是聽乃斬鳳等於門外【宫門之外也】孫恒亦伏誅時李植父柱國大將軍遠鎮弘農【西魏之境東盡瀍洛以弘農為要地率用重將鎮之】護召遠及植還朝【召植於梁州朝直遥翻】遠疑有變沈吟久之【沈持林翻】乃曰大丈夫寜為忠鬼安可作叛臣邪遂就徵既至長安護以遠功名素重猶欲全之引與相見謂之曰公兒遂有異謀非止屠戮護身乃是傾危宗社叛臣賊子理宜同疾公可早為之所乃以植付遠遠素愛植植又口辯自陳初無此謀遠謂植信然詰朝將植謁護【詰去吉翻將如字引也】護謂植已死左右白植亦在門護大怒曰陽平公不信我【李遠封陽平公】乃召入仍命遠同坐令略陽公與植相質於遠前【質證也夫君臣無獄略陽公雖廢猶舊君也烏有與舊臣相質之理宇文護不學無識如此求良死得乎】植辭窮謂略陽曰【略陽之下當有公字】本為此謀欲安社稷利至尊耳今日至此何事云云遠聞之自投於牀曰若爾誠合萬死於是護乃害植并逼遠令自殺植弟叔詣叔謙叔讓亦死餘子以幼得免初遠弟開府儀同三司穆知植非保家之主每勸遠除之遠不能用及遠臨刑泣謂穆曰吾不用汝言以至此穆當從坐以前言獲免除名為民及其子弟亦免官植弟浙州刺史基【五代志浙陽郡西魏置浙州漢析縣地也】尚義歸公主【公主宇文泰之女也】當從坐穆請以二子代基命護兩釋之後月餘護弑略陽公【年十六】黜王后元氏為尼癸亥寧都公自岐州至長安甲子即天王位【諱毓小名統萬突安定公泰之長子也泰臨夏州帝生於統萬因以名之】大赦 冬十月戊辰進陳公爵為王辛未梁敬帝禪位於陳【梁天監元年受禪四主五十六年而亡】 癸酉周魏武公李弼卒 陳王使中書舍人劉師知引宣猛將軍沈恪勒兵入宫衛送梁主如别宫【陳受禪後國之政事並由中書省有中書舍人五人分掌二十一局各當尚書諸曹並為上司總國内機要尚書唯聽受而已劉師知陳王所親任者也宣猛將軍班第九】恪排闥見王叩頭謝曰恪身經事蕭氏【侯景圍臺城恪為右軍將軍東土山主以拒戰功封東興縣侯】今日不忍見此分受死耳【分扶問翻】决不奉命王嘉其意不復逼【復扶又翻】更以盪主王僧志代之【盪主主驍銳跳盪之兵猶北齊之直盪都督也盪徒朗翻又吐浪翻】乙亥王即皇帝位於南郊還宫大赦改元【始改元為永定】奉梁敬帝為江隂王梁太后為太妃皇后為妃以給事黄門侍郎蔡景歷為祕書監兼中書通事舍人是時政事皆由中書省置二十一局各當尚書諸曹總國機要尚書唯聽受而已【史言蔡景歷委寄之重】 丙子上幸鍾山祠蔣帝廟庚辰上出佛牙於杜姥宅【齊初僧統法獻於烏纒國得佛牙常在定林上寺梁天監末為攝山慶雲寺沙門慧興保藏慧興將終以屬弟慧志承聖末慧志密送於帝至是乃出之】設無遮大會帝親出闕前膜拜【膜拜胡禮拜也膜莫湖翻】 辛巳追尊皇考文讃為景皇帝廟號太祖皇妣董氏曰安皇后追立前夫人錢氏為昭皇后【帝先娶同郡錢仲方女早卒】世子克為孝懷太子立夫人章氏為皇后章后烏程人也 置刪定郎治律令【刪定郎自晉宋以來多置之治直之翻】 乙酉周王祀圜丘丙戌祀方丘甲午祭太社 戊子太祖神主祔太廟七廟始共用一太牢【牛羊豕具為一太牢】始祖薦首餘皆骨體 侯安都至武昌王琳將樊猛棄城走【將即亮翻下同】周文育自豫章會之安都聞上受禪歎曰吾今兹必敗戰無名矣【始者以王琳不應梁召而討之猶是挟天子以令諸侯今既受梁禪則安都之師為無名】時兩將俱行不相統攝部下交争稍不相平軍至郢州琳將潘純陀於城中遙射官軍【射而亦翻】安都怒進軍圍之未克而王琳至弇口【弇口弇水入江之口正對北岸大軍山】安都乃釋郢州悉衆詣沌口【沌柱兖翻】留沈泰一軍守漢曲【漢曲漢水之曲】安都遇風不得進琳據東岸安都據西岸相持數日乃合戰安都等大敗 【考異曰典略云乙亥安都敗陳書云是月敗續按高祖以乙亥受禪安都聞之而歎豈同日乎今從陳書】安都文育及裨將徐敬成周鐵虎程靈洗皆為琳所擒【將即亮翻下同】沈泰引軍奔歸琳引見諸將與語周鐵虎辭氣不屈琳殺鐵虎而囚安都等總以一長鏁繫之置琳所坐䑽下【鏁蘇果翻䑽音榻大船也】令所親宦者王子晉掌視之琳乃移湘州軍府就郢城又遣其將樊猛襲據江州 十一月丙申上立兄子蒨為臨川王頊為始興王弟子曇朗已死而上未知遙立為南康王【頊時在長安亦遙立也蒨倉甸翻頊呼玉翻曇徒含翻曇朗死見上卷上年】 庚子周王享太廟丁未祀圜丘十二月庚午謁成陵癸酉還宫 譙淹帥水軍七千老弱三萬自蜀江東下【譙淹自墊江東下為周所逼也言蜀江以别湘江帥讀曰率下皆帥同】欲就王琳周使開府儀同三司賀若敦叱羅暉等擊之【若人者翻叱羅虜複姓】斬淹悉俘其衆 是歲詔給事黄門侍郎蕭乾招諭閩中時熊曇朗在豫章周廸在臨川留異在東陽陳寶應在晉安共相連結閩中豪帥往往立砦以自保【帥所類翻砦柴夬翻依險立木壘石以自保守曰砦】上患之使乾諭以禍福豪帥皆帥衆請降【降戶江翻】即以乾為建安太守【按五代志閩州建安縣舊置建安郡南安縣舊曰晉安建安之地唐為建州晉安之地唐為泉州】乾子範之子也【蕭子範齊豫章王嶷之子】 初梁興州刺史席固以州降魏周太祖以固為豐州刺史【五代志浙陽郡武當縣舊置武當郡及始平郡後改為齊興郡梁置興州後周改為豐州隋改為均州降戶江翻】久之固猶習梁法不遵北方制度周人密欲代之而難其人乃以司憲中大夫令狐整權鎮豐州【唐六典曰後周秋官置司憲中大夫二人掌丞司宼之法以左右刑罰盖比御史中丞之職也】委以代固之略整廣布威恩傾身撫接數月之間化洽州府於是除整豐州刺史以固為湖州刺史【五代志舂陵郡湖陽縣後魏置淮安郡及南襄州後改為南平州西魏改曰昇州後又改曰湖州】整遷豐州於武當旬日之間城府周備遷者如歸固之去也其部曲多願留為整左右整諭以朝制弗許【朝制謂周朝之法制朝直遙翻】莫不流涕而去 齊人於長城内築重城自庫洛枝東至鳴紇戍【北史作庫洛枝塢紇戍重直龍翻】凡四百餘里 初齊有術士言亡高者黑衣故高祖每出不欲見沙門顯祖在晉陽問左右何物最黑對曰無過於漆帝以上黨王渙於兄弟第七使庫直都督破六韓伯昇之鄴徵渙渙至紫陌橋殺伯昇而逃浮河南度至濟州為人所執送鄴【黑者宇文戎衣也齊顯祖乃因此以殺其弟何異秦以亡秦者胡而伐匈奴唐太宗以代唐者武氏而殺李君羨濟子禮翻】帝之為太原公也與永安王浚皆見世宗【見賢遍翻】帝有時洟出【鼻液曰洟】浚責帝左右曰何不為二兄拭鼻【顯祖於兄弟之次第二為干偽翻】帝深銜之及即位浚為青州刺史聰明矜恕吏民悅之浚以帝嗜酒私謂親近曰二兄因酒敗德【敗補邁翻】朝臣無敢諫者【朝直遙翻下同】大敵未滅【大敵謂周也】吾甚以為憂欲乘驛至鄴面諫不知用吾不【不讀曰否】或密以白帝帝益銜之浚入朝從幸東山帝裸裎為樂【裸裎露體也裸郎果翻䄇馳成翻樂音洛】浚進諫曰此非人主所宜帝不悅浚又於屛處召楊愔【屛必郢翻屛處隱蔽之處】譏其不諫帝時不欲大臣與諸王交通愔懼奏之帝大怒曰小人由來難忍【由來猶今人言從來】遂罷酒還宫浚尋還州又上書切諫詔徵浚浚懼禍謝疾不至帝遣馳驛收浚老幼泣送者數千人至鄴與上黨王渙皆盛以鐵籠【盛時征翻】寘於北城地牢飲食溲穢共在一所【溲所鳩翻】二年春正月王琳引兵下至湓城屯於白水浦帶甲十萬琳以北江州刺史魯悉達為鎮北將軍上亦以悉達為征西將軍各送鼓吹女樂【吹尺瑞翻】悉達兩受之遷延顧望皆不就上遣安西將軍沈泰襲之不克琳欲引軍東下而悉達制其中流琳遣使說誘終不從【使疏吏翻說式芮翻下林說同誘音酉】己亥琳遣記室宗虩求援於齊【虩迄逆翻】且請納梁永嘉王莊以主梁祀【莊質齊見上卷梁敬帝紹泰元年】衡州刺史周廸欲自據南川【自南康至豫章之地謂之南川以南江所經言之也】乃總召所部八郡守宰結盟齊言入赴【廸所部八郡南康宜春安成廬陵臨川巴山豫章豫寜也姚思廉陳書作聲言入赴守式又翻】上恐其為變厚慰撫之新吳洞主余孝頃遣沙門道林說琳曰周廸黄法皆依附金陵隂窺間隙【巨俱翻間古莧翻】大軍若下必為後患不如先定南川然後東下孝頃請席卷所部以從下吏【卷讀曰捲】琳乃遣輕車將軍樊猛平南將軍李孝欽平東將軍劉廣德將兵八千赴之使孝頃總督三將屯於臨川故郡【臨川故郡周敷所屯也琳遣兵攻廸并以脅敷】徵兵糧於廸以觀其所為 以開府儀同三司侯瑱為司空衡州刺史歐陽頠為都督交廣等十九州諸軍事廣州刺史【瑱他甸翻又音鎮頠魚委翻】 周以晉公護為太師 辛丑上祀南郊大赦乙巳祀北郊 辛亥周王耕籍田 癸丑周立王后獨孤氏【后獨孤信之女】 戊午上祀明堂 二月壬申南豫州刺史沈泰奔齊【泰進不能救侯安都之覆敗退不能制魯悉達之倔彊盖懼罪而北奔也 考異曰北齊帝紀在八月今從陳帝紀】 齊北豫州刺史司馬消難以齊主昏虐滋甚隂為自全之計曲意撫循所部消難尚高祖女【難乃旦翻齊主尊其父歡廟號高祖】情好不睦公主訴之【好呼到翻】上黨王渙之亡也鄴中大擾疑其赴成臯【齊北豫州治虎牢成臯之地也】消難從弟子瑞為尚書左丞【從才用翻】與御史中丞畢義雲有隙義雲遣御史張子階詣北豫州采風聞先禁消難典籖家客等消難懼密令所親中兵參軍裴藻託以私假【假居訝翻休假也】間行入關請降於周【間古莧翻下間道同降戶江翻】三月甲午周遣柱國達奚武大將軍楊忠帥騎士五千迎消難【帥讀曰率騎奇寄翻下同】從間道馳入齊境五百里前後三遣使報消難皆不報【使疏吏翻】去虎牢三十里武疑有變欲還忠曰有進死無退生獨以千騎夜趣城下【趣七喻翻】城四面峭絶【峭七笑翻】但聞擊柝聲武親來麾數百騎西去忠勒餘騎不動俟門開而入馳遣召武齊鎮城伏敬遠勒甲士二千人據東城【鎮城即防城大都督之任東城虎牢城之東偏也北史作東陴】舉烽嚴警武憚之不欲保城乃多取財物以消難及其屬先歸忠以三千騎為殿【殿丁練翻】至洛南皆解鞍而臥齊衆來追至洛北忠謂將士曰但飽食今在死地賊必不敢度水已而果然乃徐引還武歎曰達奚武自謂天下健兒今日服矣周以消難為小司徒【唐六典曰周之地官小司徒中大夫也後周依周官杜佑通典後周地官小司徒上大夫六命 考異曰北齊帝紀四月消難叛今從周書典略】 丁酉齊主自晉陽還鄴 【考異曰北齊帝紀天保七年八月帝如晉陽不言其還八年四月帝在城東馬射敕京師婦女悉赴觀是在鄴也此月又言至自晉陽六月乙丑帝自晉陽北廵則又復在晉陽必有差互今不敢增損】 齊發兵援送梁永嘉王莊於江南冊拜王琳為梁丞相都督中外諸軍錄尚書事琳遣兄子叔寶率所部十州刺史子弟赴鄴琳奉莊即皇帝位 【考異曰北齊帝紀十一月丁巳琳遣使請立莊仍以江州内屬令莊居之十二月癸酉詔莊為梁主進居九派今從陳書及典略然陳書典略皆云立莊於郢州按琳時在湓城盖始居江州後遷郢州耳】改元天啓追諡建安公淵明曰閔皇帝【淵明廢見上卷梁敬帝紹泰元年卒見太平元年】莊以琳為侍中大將軍中書監餘依齊朝之命【朝直遙翻】 夏四月甲子上享太廟 乙丑上使人害梁敬帝立梁武林侯諮之子季卿為江隂王 己巳周以太師護為雍州牧【後周雍州牧九命雍於用翻】 甲戌周王后獨孤氏殂 辛巳齊大赦 齊主以旱祈雨於西門豹祠不應毁之幷掘其冢【戰國時魏以西門豹為鄴令鑿十二渠以利民故祠冢皆在鄴】五月癸巳余孝頃等屯二萬軍於工塘連八城以逼周廸【等下有衆字文意乃暢】廸懼請和并送兵糧樊猛等欲受盟而還【還從宣翻又如字】孝頃貪其利不許樹柵圍之由是猛等與孝頃不協 周以大司空侯莫陳崇為大宗伯 癸丑齊廣陵南城主張顯和長史張僧那各帥所部來降【帥讀曰率降戶江翻】 辛丑齊以尚書令長廣王湛錄尚書事驃騎大將軍平秦王歸彦為尚書左僕射【驃匹妙翻騎奇寄翻】甲辰以前左僕射楊愔為尚書令 辛酉上幸大莊嚴寺捨身【前車覆矣後車不知戒耳目習於事佛不知其非也】壬戌羣臣表請還宫 六月乙丑齊主北巡以太子殷監國【監工銜翻】因立大都督府與尚書省分理衆務仍開府置佐齊主特崇其選以趙郡王叡為侍中攝大都督府長史 己巳詔司空侯瑱與領軍將軍徐度帥舟師為前軍以討王琳【帥讀曰率】 齊主至祁連池【祁連池即汾陽之天池北人謂天為祁連】戊寅還晉陽 秋戊戌上幸石頭送侯瑱等【秋七月戊戌也】 高州刺史黄法吳興太守沈恪寜州刺史周敷【時盖即臨川故郡置寜州以敷為刺史】合兵救周廸敷自臨川故郡斷江口【斷音短】分兵攻余孝頃别城樊猛等不救而没劉廣德乘流先下故獲全孝頃等皆棄舟引兵步走廸追擊盡擒之送孝頃及李孝欽於建康歸樊猛於王琳 甲辰上遣吏部尚書謝哲往諭王琳哲朏之孫也【謝朏莊之子歷仕宋齊梁朏敷尾翻】 八月甲子周大赦乙丑齊主還鄴 辛未詔臨川王蒨西討【蒨倉甸翻】以舟<br />
<br />
  師五萬發建康上幸冶城寺送之 甲戌齊主如晉陽王琳在白水浦周文育侯安都徐敬成許王子晉以<br />
<br />
  厚賂子晉乃偽以小船依䑽而釣【䑽音榻】夜載之上岸入深草中步投陳軍還建康自劾上引見並宥之【上時掌翻劾戶槩翻又戶得翻見賢遍翻】戊寅復其本官【文育安都敗軍而不誅遽復其官何也二人當時名將誅之則無以為用故也】 謝哲返命王琳請還湘州詔追衆軍還癸未衆軍至自大雷 九月甲申周封少師元羅為韓國公以紹魏後【江陵之陷元羅還魏】 丁未周王如同州冬十月辛酉還長安 余孝頃之弟孝勱及子公颺猶據舊柵不下【新吳舊柵也勱音邁颺余章翻】庚午詔開府儀同三司周文育都督衆軍出豫章討之 齊三臺成更名銅爵曰金鳳金虎曰聖應氷井曰崇光【魏武築三臺於鄴城西北皆因城為之基中曰銅爵臺高十丈石虎更增二丈南曰金虎臺北曰氷井臺皆高八丈更工衡翻】十一月甲午齊主至鄴大赦齊主遊三臺戱以槊刺都督尉子輝應手而斃【槊色角翻刺七亦翻考異曰北史作子耀今從北齊書典略】常山王演以帝沈湎憂憤形於顏色帝覺之曰但令汝在我何為不縱樂【沈持林翻樂音洛】演唯涕泣拜伏竟無所言帝亦大悲抵盃於地曰汝似嫌我如是自今敢進酒者斬之因取所御盃盡壞棄【壞音怪】未幾沈湎益甚或於諸貴戚家角力批拉【幾居豈翻沈持林翻批自結翻又偏迷翻手擊也拉盧合翻】不限貴賤唯演至則内外肅然演又密撰事條將諫其友王晞以為不可【諸王宫寮有師有反】演不從因間極言【間古莧翻】遂逢大怒演性頗嚴尚書郎中剖斷有失【斷丁亂翻】輒加捶楚令史姦慝即考竟【演為尚書令故然捶止橤翻】帝乃立演於前以刀鐶擬脇【以刀鐶擬演脇示將築殺之】召被演罰者【被皮義翻】臨以白刃求演之短咸無所陳乃釋之晞昕之弟也【王昕見下昕許斤翻】帝疑演假辭於晞以諫欲殺之王私謂晞曰【王即演也通鑑因齊史修治有未純者耳】王博士明日當作一條事為欲相活【為于偽翻】亦圖自全宜深體勿怪乃於衆中杖晞二十帝尋發怒聞晞得杖以故不殺【以其得杖之故謂非敎演為之遂不殺】髠鞭配甲坊居三年演又因諫爭大被敺撻【爭讀曰諍被皮義翻敺烏口翻 考異曰北史孝昭紀云文宣賜帝魏時宫人醒而忘之謂帝擅取遂令刀鐶亂築因此致困今從北史王晞傳】閉口不食太后日夜涕泣帝不知所為曰儻小兒死奈我老母何於是數往問演疾謂曰努力彊食【數所角翻彊其兩翻下強坐同】當以王晞還汝乃釋晞令詣演演抱晞曰吾氣息惙然【惙丑捩翻類篇曰困劣也】恐不復相見【復扶又翻】晞流涕曰天道神明豈令殿下遂斃此舍【言天道福善禍淫不應使演遂死於此】至尊親為人兄尊為人主安可與計【言難與計是非也】殿下不食太后亦不食殿下縱不自惜獨不念太后乎言未卒演強坐而飯【卒子恤翻飯扶晩翻】晞由是免徒【配甲坊徒刑也由此得免】還為王友及演錄尚書事除官者皆詣演謝去必辭晞言於演曰受爵天朝拜恩私第【晉羊祜之言朝直遙翻】自古以為不可宜一切約絶演從之久之演從容謂晞曰主上起居不恒卿宜耳目所具吾豈可以前逢一怒遂爾結舌卿宜為譔諫草【從千容翻恒戶登翻為于偽翻】吾當伺便極諫【伺相吏翻】晞遂條十餘事以呈因謂演曰今朝廷所恃者惟殿下乃欲學匹夫耿介輕一朝之命狂藥令人不自覺【狂藥謂酒也】刀箭豈復識親踈【復扶又翻】一旦禍出理外將奈殿下家業何奈皇太后何演欷歔不自勝【欷音希又許氣翻歔音虛勝音升】曰乃至是乎明日見晞曰吾長夜久思今遂息意即命火對晞焚之【焚諫草也】後復承間苦諫帝使力士反接【復扶又翻間古莧翻反接兩手向後也】拔白刃注頸罵曰小子何知是誰教汝演曰天下噤口非臣誰敢有言帝趣杖亂捶之數十【噤巨禁翻趣讀曰促】會醉臥得解帝䙝黷之遊徧于宗戚所往留連【盤樂忘返謂之留連】唯至常山第多無適而去【適歡極也】尚書左僕射崔暹屢諫演謂暹曰今太后不敢言吾兄弟杜口僕射獨能犯顔内外深相感愧【常山齊之賢王文宣淫規正為多齊之史臣因其嗣祚亦多溢美觀者能於辭令之間詳其溢美者則幾矣】太子殷自幼温裕開朗禮士好學【好呼到翻】關覽時政甚有美名帝常嫌太子得漢家性質不似我欲廢之【鮮卑謂中國人為漢】帝登金鳳臺召太子使手刃囚太子惻然有難色再三不斷其首帝大怒親以馬鞭撞之【斷丁管翻撞直江翻】太子由是氣悸語吃【悸其季翻吃居乞翻言蹇也】精神昏擾帝因酣宴屢云太子性懦社稷事重終當傳位常山太子少傅魏收謂楊愔曰太子國之根本不可動搖至尊三爵之後每言傳位常山令臣下疑二若其實也當决行之此言非所以為戲恐徒使國家不安愔以收言白帝帝乃止帝既殘忍有司訊囚莫不嚴酷或燒犁耳使立其上或燒車釭【釭姑紅翻車轂中鐵也】使以臂貫之既不勝苦皆至誣伏【勝音升】唯三公郎中武強蘇瓊【三公郎自魏晉以來有之五代志後齊尚書列曹三公郎中屬殿中尚書掌五時讀時令諸曹囚帳斷罪赦日建金雞等事劉昫曰武強漢武隧縣屬河間國晉改曰武彊屬安平國後魏屬廣宗郡又置武邑郡五代志曰齊廢郡為武彊縣至隋屬信都郡】歷職中外所至皆以寛平為治【治直吏翻】時趙州及清河屢有人告謀反者前後皆付瓊推撿事多申雪尚書崔昂謂瓊曰若欲立功名當更思餘理數雪反逆【數所角翻】身命何輕瓊正色曰所雪者寃枉耳不縱反逆也昂大慙帝怒臨漳令稽曄【晉避愍帝諱改鄴為臨漳尋没於石勒復曰鄴東魏天平初分鄴併内黄斤邱肥鄉置臨漳縣】舍人李文思以賜臣下為奴中書侍郎彭城鄭頤私誘祠部尚書王昕曰自古無朝士為奴者昕曰箕子為之奴【此論語孔子之言鄭頤誘王昕使言而陷之邦無道危行言孫聖人包周身之防也如此誘音酉朝直遙翻下同】頤以白帝曰王元景比陛下於紂【王昕字元景以字行】帝銜之頃之帝與朝臣酣飲昕稱疾不至帝遣騎執之【騎奇寄翻】見方搖膝吟咏遂斬於殿前投尸漳水齊主北築長城南助蕭莊士馬死者以數十萬計重以修築臺殿【重直用翻】賜與無節府藏之積不足以供【藏徂浪翻】乃減百官之禄撤軍人常廩併省州郡縣鎮戍之職以節費用焉 十二月庚寅齊以可朱渾道元為太師尉粲為太尉冀州刺史段韶為司空常山王演為大司馬長廣王湛為司徒 壬午周大赦 齊主如北城因視永安簡平王浚上黨剛肅王渙於地牢帝臨穴謳歌令浚等和之浚等惶怖且悲不覺聲顫【和胡卧翻怖普布翻顫之膳翻】帝愴然為之下泣【為于偽翻泣淚也】將赦之長廣王湛素與浚不睦進曰猛虎安可出穴帝默然浚等聞之呼湛小字曰步落稽皇天見汝帝亦以浚與渙皆有雄略恐為後害乃自刺渙又使壯士劉桃枝就籠亂刺【刺七亦翻】槊每下【槊色角翻】浚渙輒以手拉折之號哭呼天【史言浚渙之多力折而設翻號戶高翻】於是薪火亂投燒殺之填以土石後出之皮髮皆盡尸色如炭遠近為之痛憤【為于偽翻】帝以儀同三司劉郁捷殺浚以浚妃陸氏賜之馮文洛殺渙以渙妃李氏賜之【為李妃撻馮文洛張本】二人皆帝家舊奴也陸氏尋以無寵於浚得免 高凉太守馮寶卒海隅擾亂寶妻洗氏懷集部落數州晏然其子僕生九年是歲遣僕帥諸酋長入朝【洗息典翻帥讀曰率酋慈秋翻長知兩翻】詔以僕為陽春太守【五代志高凉郡陽春縣梁置陽春郡】 後梁主遣其大將軍王操將兵略取王琳之長沙武陵南平等郡【琳兵東下故後梁得以乘虛取之操將即亮翻】<br />
<br />
  三年春正月己酉周太師護上表歸政【上時掌翻】周王始親萬機軍旅之事護猶總之初改都督州軍事為總管【諸州總管自此始】 王琳召桂州刺史淳于量【五代志始安郡梁置桂州今為静江府】量雖與琳合而濳通於陳二月辛酉以量為開府儀同三司 壬午侯瑱引兵焚齊舟艦於合肥【艦戶黯翻】 丙戌齊主於甘露寺禪居深觀【據齊紀甘露寺在遼陽此鄙語所謂獼猴坐禪也】唯軍國大事乃以聞尚書左僕射崔暹卒齊主幸其第哭之謂其妻李氏曰頗思暹乎對曰思之帝曰然則自往省之【省悉景翻】因手斬其妻擲首牆外 齊斛律光將騎一萬【將即亮翻下同騎奇寄翻】擊周開府儀同三司曹囘公斬之柏谷城主薛禹生弃城走遂取文侯鎮立戍置柵而還【還從宣翻又如字】 三月戊戌齊以高德政為尚書右僕射 吐谷渾宼周邉【吐從暾入聲谷音浴】庚戌周遣大司馬賀蘭祥擊之丙辰齊主至鄴 梁永嘉王莊至郢州遣使入貢於<br />
<br />
  齊【使疏吏翻】王琳遣其將雷文策襲後梁監利太守蔡大有殺之【沈約曰監利縣疑是吳所立晉屬南郡宋屬巴陵郡後梁置監利郡今監利在江陵府東南百八十里】 齊主之為魏相也【相息亮翻】膠州刺史定陽文肅侯杜弼為長史【五代志高密郡舊置膠州唐武德五年改曰密州五代志文城郡吉昌縣後魏曰定陽】帝將受禪弼諫止之【見一百六十三卷簡文帝大寶元年】帝問治國當用何人【治直之翻】對曰鮮卑車馬客會須用中國人帝以為譏已銜之高德政用事弼不為之下嘗於衆前面折德政德政數言其短於帝弼恃舊不自疑【折之舌翻數所角翻下同高德政讒杜弼而不知楊愔之忌已杜弼恃舊而不疑德政之讒已昏昏於利欲之場祗思害人而不知其身之受害者多矣】夏帝因飲酒積其愆失遣使就州斬之【使疏吏翻】既而悔之驛追不及 閏四月戊子周命有司更定新歷【更工衡翻】丁酉遣鎮北將軍徐度將兵城南皖口【南皖口皖水入江之口也】<br />
<br />
  【祝穆曰皖水自霍山縣流入經懷寧縣北二里又東南流三百四十里入大江吳地志皖口今舒州之山口鎮將即亮翻皖戶板翻】 齊高德政與楊愔同為相愔常忌之【高德政事齊主於初濳禪代之際又德政勸成之權利所集故為楊愔所忌】齊主酣飲德政數彊諫【數所角翻】齊主不悦謂左右曰高德政恒以精神凌逼人【恒戶登翻】德政懼稱疾欲自退帝謂楊愔曰我大憂德政病對曰陛下若用為冀州刺史病當自差【差叱駕翻病差猶言病瘳也】帝從之德政見除書即起帝大怒召德政謂曰聞爾病我為爾針【醫家按穴用針可以愈疾故云然為于偽翻】親以小刀刺之血流霑地又使曳下斬去其足【刺七亦翻去羌呂翻】劉桃枝執刀不敢下帝責桃枝曰爾頭即墮地桃枝乃斬其足之三指帝怒不解囚德政於門下【囚於門下省】其夜以氊輿送還家明旦德政妻出珍寶滿四牀欲以寄人帝奄至其宅見之怒曰我御府猶無是物詰其所從得皆諸元賂之遂曳出斬之妻出拜又斬之【高德政之禍猶宋之顏竣也詰去吉翻】并其子伯堅以司州牧彭城王浟為司徒侍中高陽王湜為尚書右僕射乙巳以浟兼太尉【浟夷周翻湜常職翻】 齊主封子紹亷為長安王 辛亥周以侯莫陳崇為大司徒達奚武為大宗伯武陽公豆盧寧為大司宼【五代志犍為郡犍為縣後周曰武陽令狐德棻曰寜之先本慕容氏前燕之枝庶也歸魏賜姓豆盧氏】柱國輔城公邕為大司空【輔城郡名在汝州郟城縣】 乙卯周詔有司無得糾赦前事唯廐庫倉廩與海内所共若有侵盗雖經赦宥免其罪徵備如法【備償也今人猶言填備】 周賀蘭祥與吐谷渾戰破之拔其洮陽洪和二城以其地為洮州【李延夀曰賀蘭之先與魏俱起有乞伏為賀蘭莫何弗因以為氏劉昫唐志洮州臨潭縣本叶谷渾之洪和城後周攻得之置美相縣唐為臨潭縣洮州治焉後移治洮陽城仍於舊洪和城置美相縣天寶中廢美相併入臨潭洮土刀翻】 五月丙辰朔日有食之 齊太史奏今年當除舊布新齊主問於特進彭城公元韶曰漢光武何故中興對曰為誅諸劉不盡【為於偽翻】於是齊主悉殺諸元以厭之【厭於協翻】癸未誅始平公元世哲等二十五家囚韶等十九家韶幽於地牢絶食㗖衣袖而死【㗖徒敢翻又徒濫翻】 周文育周廸黄法共討余公颺【公颺孝頃子也颺音揚】豫章太守熊曇朗引兵會之衆且萬人文育軍於金口【自豫章西南入象牙江至金溪口】公颺詐降謀執文育文育覺之囚送建康【降戶江翻】文育進屯三陂王琳遣其將曹慶帥二千人救余孝勱【將即亮翻下同帥讀曰率下自帥達帥同】慶分遣主帥常衆愛與文育相拒【帥所類翻】自帥其衆攻周迪及安南將軍吳明徹迪等敗文育退據金口熊曇朗因其失利謀殺文育以應衆愛監軍孫白象聞其謀勸文育先之【監工銜翻先悉薦翻】文育不從時周迪棄船走不知所在乙酉文育得迪書自齎以示曇朗曇朗殺之於坐而併其衆【坐徂卧翻】因據新淦城【新淦縣自漢至蕭齊屬豫章郡五代志屬廬陵郡唐屬吉州淦音紺又音甘】曇朗將兵萬人襲周敷敷擊破之曇朗單騎奔巴山【騎奇寄翻】 魯悉達部將梅天養等引齊軍入城【魯悉達據新蔡城】悉達帥麾下數千人濟江自歸拜平南將軍北江州刺史【五代志宣城郡南陵縣陳置北江州】 六月戊子周以霖雨詔羣臣上封事極諫左光禄大夫猗氏樂遜上言四事【猗氏縣漢以來屬河東郡古郇瑕氏之地後以猗頓居之以盬鹽致富遂改為猗氏上時掌翻】其一以為比來守令代期既促【比毘至翻守式又翻】責其成效專務威猛今關東之民淪䧟塗炭若不布政優優【詩商頌之辭毛萇曰優優和也】聞諸境外【聞音問】何以使彼勞民歸就樂土【樂音洛】其二以為頃者魏都洛陽一時殷盛貴勢之家競為侈靡終使禍亂交興天下喪敗【喪息淚翻】比來朝貴器服稍華【朝直遙翻】百工造作務盡奇巧臣誠恐物逐好移有損政俗【好呼到翻】其三以為選曹補擬宜與衆共之今州郡選置猶集鄉閭况天下銓衡不取物望既非機事何足可密【以此觀之選曹補擬皆密奏於上盖自晉山濤啟事始也】其選置之日宜令衆心明白然後呈奏其四以為高洋據有山東未易猝制譬猶碁刼相持爭行先後【奕碁有刼彼此爭行以相持以先後着决一枰之勝負易以豉翻】若一行不當【當丁浪翻】或成彼利誠應捨小營大先保封域不宜貪利邉陲輕為舉動 周處士韋夐【處昌呂翻夐休正翻】孝寛之兄也志尚夷簡魏周之際十徵不屈周太祖甚重之不奪其志世宗禮敬尤厚號曰逍遙公晉公護延之至第訪以政事護盛脩第舍夐仰視堂歎曰酣酒嗜音峻宇彫牆有一於此未或不亡【夏書五子之歌之辭】護不悦驃騎大將軍開府儀同三司宼儁【寇儁所居官後周之九命也】讚之孫也【寇讚自秦歸魏見一百一十八卷晉安帝義熙十四年】少有學行家人常賣物多得絹五匹儁於後知之曰得財失行【少詩照翻行下孟翻】吾所不取訪主還之敦睦宗族與同豐約敎訓子孫必先禮義【先悉薦翻】自大統中稱老疾不朝謁【朝直遙翻下同】世宗虛心欲見之儁不得已入見【入見賢遍翻】王引之同席而坐問以魏朝舊事載以御輿令於王前乘之以出顧謂左右曰如此之事唯積善者可以致之【史言周王禮賢】 周文育之討余孝勱也帝令南豫州刺史侯安都繼之文育死安都還遇王琳將周炅周協南歸【王琳使周炅助曹慶攻周迪自南川歸也將即亮翻炅古迥翻】與戰擒之孝勱弟孝猷帥所部四千家詣安都降【帥讀曰率降戶江翻】安都進軍至左里擊曹慶常衆愛破之衆愛奔廬山庚寅廬山民斬之傳首 詔臨川王蒨於南皖口置城【蒨倉甸翻】使東徐州刺史吳興錢道戢守之【戢則立翻】 丁酉上不豫丙午殂【年五十七】上臨戎制勝英謀獨運而為政務崇寛簡非軍旅急務不輕調發性儉素常膳不過數品私宴用瓦器蚌盤【調徒釣翻蚌盤者髹器以蚌為飾今謂之螺鈿蚌步項翻】殽核充事而已後宫無金翠之飾不設女樂時皇子昌在長安【梁元帝承聖元年徵昌為領直江陵之䧟没於長安】内無嫡嗣外有強敵宿將皆將兵在外朝無重臣【將即亮翻朝直遙翻下同】唯中領軍杜稜典宿衛兵在建康章皇后召稜及中書侍郎蔡景歷入禁中定議秘不發喪急召臨川王蒨於南皖景歷親與宦者宫人密營歛具【歛力贍翻】時天暑須治梓宫【治直之翻】恐斧斤之聲聞於外乃以蠟為袐器【蠟蜜滓也漢宫有東園祕器聞音問】文書詔敕依舊宣行侯安都軍還適至南皖與臨川王俱還朝甲寅王至建康入居中書省安都與羣臣定議奉王嗣位王謙讓不敢當皇后以昌故未肯下令羣臣猶豫不能决安都曰今四方未定何暇及遠臨川王有大功於天下【武帝既殺王僧辯使蒨平杜龕張彪等以定東土故云有大功】須共立之今日之事後應者斬即按劒上殿白皇后出璽又手解蒨髮推就喪次【上時掌翻璽斯氏翻推吐雷翻為侯安都恃定策之功以殺其身張本】遷殯大行於太極西階皇后乃下令以蒨纂承大統是日即皇帝位大赦秋七月丙辰尊皇后為皇太后辛酉以侯瑱為太尉侯安都為司空 齊顯祖將如晉陽乃盡誅諸元或祖父為王或身嘗貴顯皆斬於東市其嬰兒投於空中承之以矟【稍所角翻與槊同】前後死者凡七百二十一人悉棄尸漳水剖魚者往往得人爪甲鄴下為之久不食魚【為於偽翻】使元黄頭與諸囚自金鳳臺各乘紙鴟以飛黄頭獨能至紫陌乃墮仍付御史中丞畢義雲餓殺之【齊主每令死囚以席為翅從臺上飛下免其罪戮今欲夷諸元黄頭雖免殊死猶餓殺之】唯開府儀同三司元蠻祠部郎中元文遙等數家獲免【曹魏置祠部郎五代志齊尚書曹祠部郎掌祠祀醫藥贈賜等事】蠻繼之子常山王演之妃父【江陽王繼元又之父後徙封安定王】文遙遵之五世孫也【常山公遵佐道武帝定中原】定襄令元景安【此後漢新興郡之定襄也在隋秀容縣界】䖍之玄孫也【陳留王䖍亦佐道武帝與燕主垂戰敗而死】欲請改姓高氏其從兄景皓曰【從才用翻】安有棄其本宗而從人之姓者乎丈夫寜可玉碎何能瓦全景安以其言白帝帝收景皓誅之賜景安姓高氏 八月甲申葬武皇帝於萬安陵廟號高祖 戊戌齊封皇子紹義為廣陽王以尚書右僕射河間王孝琬為左僕射都官尚書崔昂為右僕射 周御正中大夫崔猷建議以為聖人沿革因時制宜今天子稱王不足以威天下請遵秦漢舊制稱皇帝建年號乙亥周王始稱皇帝追尊文王曰文皇帝改元武成 癸卯齊詔民間或有父祖冒姓元氏或假托攜養者不問世數遠近悉聽改復本姓 初高祖追諡兄道譚為始興昭烈王以其次子頊襲封及世祖即位頊在長安未還【頊與昌俱在長安】上以本宗乏饗【始興王道譚二子上入纂皇緒而頊留長安故本宗無主祭者世祖即上廟號】戊戌詔徙封頊為安成王皇子伯茂為始興王【以奉道譚祀】 初周太祖平蜀【見一百六十五卷梁元帝承聖二年】以其形勝之地不欲使宿將居之【將即亮翻】問諸子誰可往者皆不對少子安成公憲請行【少詩照翻】太祖以其幼不許壬子周人以憲為益州總管時年十六善於撫綏留心政術蜀人悦之九月乙卯以大將軍天水公廣為梁州總管廣導之子也【宇文導周太祖之兄子】 辛酉立皇子伯宗為太子 己巳齊主如晉陽 辛未周主封其弟輔城公邕為魯公安成公憲為齊公純為陳公盛為越公達為代公通為冀公逌為滕公【逌音由】 乙亥立太子母吳興沈妃為皇后 周少保懷寜莊公蔡祐卒【五代志蜀郡成都縣舊制懷寜等四郡】 齊顯祖嗜酒成疾不復能食【復扶又翻】自知不能久謂李后曰人生必有死何足致惜但憐正道尚幼人將奪之耳【齊太子殷字正道】又謂常山王演曰奪則任汝慎勿殺也尚書令開封王楊愔領軍大將軍平秦王歸彥侍中廣漢燕子獻【五代志滎陽郡浚儀縣舊置開封郡扶風郡雍縣後魏置平秦郡蜀郡雒縣舊曰廣漢】黄門侍郎鄭頤皆受遺詔輔政冬十月甲午殂【年三十一】癸卯發喪羣臣號哭無下泣者【以其殘暴故哭而不哀號戶刀翻】唯楊愔涕泗嗚咽太子殷即位【殷字正道小字道人文宣帝嫡子也】大赦庚戌尊皇太后為太皇太后皇后為皇太后詔諸土木金鐵雜作一切停罷 王琳聞高祖殂乃以少府卿吳郡孫瑒為郢州刺史總留任【瑒雄杏翻又音暢】奉梁永嘉王莊出屯濡須口齊揚州道行臺慕容儼帥衆臨江為之聲援【帥讀曰率】十一月乙卯琳寇大雷【五代志同安郡望江縣陳置大雷郡】詔侯瑱侯安都及儀同徐度將兵禦之【將即亮翻】安州刺史吳明徹夜襲湓城【五代志寜越郡梁置安州隋開皇十八年改曰欽州】琳遣巴陵太守任忠擊明徹大破之明徹僅以身免琳因引兵東下 齊以右丞相斛律金為左丞相常山王演為太傅長廣王湛為太尉段韶為司徒平原王淹為司空高陽王湜為尚書左僕射河間王孝琬為司州牧侍中燕子獻為右僕射 辛未齊顯祖之喪至鄴【自晉陽至鄴】 十二月戊戌齊徙上黨王紹仁為漁陽王廣陽王紹義為范陽王長樂王紹廣為隴西王【樂音洛】<br />
<br />
  資治通鑑卷一百六十七  <br>
   </div> 

<script src="/search/ajaxskft.js"> </script>
 <div class="clear"></div>
<br>
<br>
 <!-- a.d-->

 <!--
<div class="info_share">
</div> 
-->
 <!--info_share--></div>   <!-- end info_content-->
  </div> <!-- end l-->

<div class="r">   <!--r-->



<div class="sidebar"  style="margin-bottom:2px;">

 
<div class="sidebar_title">工具类大全</div>
<div class="sidebar_info">
<strong><a href="http://www.guoxuedashi.com/lsditu/" target="_blank">历史地图</a></strong>  
<a href="http://www.880114.com/" target="_blank">英语宝典</a>  
<a href="http://www.guoxuedashi.com/13jing/" target="_blank">十三经检索</a> 
<br><strong><a href="http://www.guoxuedashi.com/gjtsjc/" target="_blank">古今图书集成</a></strong> 
<a href="http://www.guoxuedashi.com/duilian/" target="_blank">对联大全</a> <strong><a href="http://www.guoxuedashi.com/xiangxingzi/" target="_blank">象形文字典</a></strong> 

<br><a href="http://www.guoxuedashi.com/zixing/yanbian/">字形演变</a>  <strong><a href="http://www.guoxuemi.com/hafo/" target="_blank">哈佛燕京中文善本特藏</a></strong>
<br><strong><a href="http://www.guoxuedashi.com/csfz/" target="_blank">丛书&方志检索器</a></strong> <a href="http://www.guoxuedashi.com/yqjyy/" target="_blank">一切经音义</a>  

<br><strong><a href="http://www.guoxuedashi.com/jiapu/" target="_blank">家谱族谱查询</a></strong>  <strong><a href="http://shufa.guoxuedashi.com/sfzitie/" target="_blank">书法字帖欣赏</a></strong> 
<br>

</div>
</div>


<div class="sidebar" style="margin-bottom:0px;">

<font style="font-size:22px;line-height:32px">QQ交流群9:489193090</font>


<div class="sidebar_title">手机APP 扫描或点击</div>
<div class="sidebar_info">
<table>
<tr>
	<td width=160><a href="http://m.guoxuedashi.com/app/" target="_blank"><img src="/img/gxds-sj.png" width="140"  border="0" alt="国学大师手机版"></a></td>
	<td>
<a href="http://www.guoxuedashi.com/download/" target="_blank">app软件下载专区</a><br>
<a href="http://www.guoxuedashi.com/download/gxds.php" target="_blank">《国学大师》下载</a><br>
<a href="http://www.guoxuedashi.com/download/kxzd.php" target="_blank">《汉字宝典》下载</a><br>
<a href="http://www.guoxuedashi.com/download/scqbd.php" target="_blank">《诗词曲宝典》下载</a><br>
<a href="http://www.guoxuedashi.com/SiKuQuanShu/skqs.php" target="_blank">《四库全书》下载</a><br>
</td>
</tr>
</table>

</div>
</div>


<div class="sidebar2">
<center>


</center>
</div>

<div class="sidebar"  style="margin-bottom:2px;">
<div class="sidebar_title">网站使用教程</div>
<div class="sidebar_info">
<a href="http://www.guoxuedashi.com/help/gjsearch.php" target="_blank">如何在国学大师网下载古籍?</a><br>
<a href="http://www.guoxuedashi.com/zidian/bujian/bjjc.php" target="_blank">如何使用部件查字法快速查字?</a><br>
<a href="http://www.guoxuedashi.com/search/sjc.php" target="_blank">如何在指定的书籍中全文检索?</a><br>
<a href="http://www.guoxuedashi.com/search/skjc.php" target="_blank">如何找到一句话在《四库全书》哪一页?</a><br>
</div>
</div>


<div class="sidebar">
<div class="sidebar_title">热门书籍</div>
<div class="sidebar_info">
<a href="/so.php?sokey=%E8%B5%84%E6%B2%BB%E9%80%9A%E9%89%B4&kt=1">资治通鉴</a> <a href="/24shi/"><strong>二十四史</strong></a>&nbsp; <a href="/a2694/">野史</a>&nbsp; <a href="/SiKuQuanShu/"><strong>四库全书</strong></a>&nbsp;<a href="http://www.guoxuedashi.com/SiKuQuanShu/fanti/">繁体</a>
<br><a href="/so.php?sokey=%E7%BA%A2%E6%A5%BC%E6%A2%A6&kt=1">红楼梦</a> <a href="/a/1858x/">三国演义</a> <a href="/a/1038k/">水浒传</a> <a href="/a/1046t/">西游记</a> <a href="/a/1914o/">封神演义</a>
<br>
<a href="http://www.guoxuedashi.com/so.php?sokeygx=%E4%B8%87%E6%9C%89%E6%96%87%E5%BA%93&submit=&kt=1">万有文库</a> <a href="/a/780t/">古文观止</a> <a href="/a/1024l/">文心雕龙</a> <a href="/a/1704n/">全唐诗</a> <a href="/a/1705h/">全宋词</a>
<br><a href="http://www.guoxuedashi.com/so.php?sokeygx=%E7%99%BE%E8%A1%B2%E6%9C%AC%E4%BA%8C%E5%8D%81%E5%9B%9B%E5%8F%B2&submit=&kt=1"><strong>百衲本二十四史</strong></a>  <a href="http://www.guoxuedashi.com/so.php?sokeygx=%E5%8F%A4%E4%BB%8A%E5%9B%BE%E4%B9%A6%E9%9B%86%E6%88%90&submit=&kt=1"><strong>古今图书集成</strong></a>
<br>

<a href="http://www.guoxuedashi.com/so.php?sokeygx=%E4%B8%9B%E4%B9%A6%E9%9B%86%E6%88%90&submit=&kt=1">丛书集成</a> 
<a href="http://www.guoxuedashi.com/so.php?sokeygx=%E5%9B%9B%E9%83%A8%E4%B8%9B%E5%88%8A&submit=&kt=1"><strong>四部丛刊</strong></a>  
<a href="http://www.guoxuedashi.com/so.php?sokeygx=%E8%AF%B4%E6%96%87%E8%A7%A3%E5%AD%97&submit=&kt=1">說文解字</a> <a href="http://www.guoxuedashi.com/so.php?sokeygx=%E5%85%A8%E4%B8%8A%E5%8F%A4&submit=&kt=1">三国六朝文</a>
<br><a href="http://www.guoxuedashi.com/so.php?sokeytm=%E6%97%A5%E6%9C%AC%E5%86%85%E9%98%81%E6%96%87%E5%BA%93&submit=&kt=1"><strong>日本内阁文库</strong></a> <a href="http://www.guoxuedashi.com/so.php?sokeytm=%E5%9B%BD%E5%9B%BE%E6%96%B9%E5%BF%97%E5%90%88%E9%9B%86&ka=100&submit=">国图方志合集</a> <a href="http://www.guoxuedashi.com/so.php?sokeytm=%E5%90%84%E5%9C%B0%E6%96%B9%E5%BF%97&submit=&kt=1"><strong>各地方志</strong></a>

</div>
</div>


<div class="sidebar2">
<center>

</center>
</div>
<div class="sidebar greenbar">
<div class="sidebar_title green">四库全书</div>
<div class="sidebar_info">

《四库全书》是中国古代最大的丛书,编撰于乾隆年间,由纪昀等360多位高官、学者编撰,3800多人抄写,费时十三年编成。丛书分经、史、子、集四部,故名四库。共有3500多种书,7.9万卷,3.6万册,约8亿字,基本上囊括了古代所有图书,故称“全书”。<a href="http://www.guoxuedashi.com/SiKuQuanShu/">详细>>
</a>

</div> 
</div>

</div>  <!--end r-->

</div>
<!-- 内容区END --> 

<!-- 页脚开始 -->
<div class="shh">

</div>

<div class="w1180" style="margin-top:8px;">
<center><script src="http://www.guoxuedashi.com/img/plus.php?id=3"></script></center>
</div>
<div class="w1180 foot">
<a href="/b/thanks.php">特别致谢</a> | <a href="javascript:window.external.AddFavorite(document.location.href,document.title);">收藏本站</a> | <a href="#">欢迎投稿</a> | <a href="http://www.guoxuedashi.com/forum/">意见建议</a> | <a href="http://www.guoxuemi.com/">国学迷</a> | <a href="http://www.shuowen.net/">说文网</a><script language="javascript" type="text/javascript" src="https://js.users.51.la/17753172.js"></script><br />
  Copyright &copy; 国学大师 古典图书集成 All Rights Reserved.<br>
  
  <span style="font-size:14px">免责声明:本站非营利性站点,以方便网友为主,仅供学习研究。<br>内容由热心网友提供和网上收集,不保留版权。若侵犯了您的权益,来信即刪。scp168@qq.com</span>
  <br />
ICP证:<a href="http://www.beian.miit.gov.cn/" target="_blank">鲁ICP备19060063号</a></div>
<!-- 页脚END --> 
<script src="http://www.guoxuedashi.com/img/plus.php?id=22"></script>
<script src="http://www.guoxuedashi.com/img/tongji.js"></script>

</body>
</html>
