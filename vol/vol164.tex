\section{資治通鑑卷一百六十四}
宋 司馬光 撰

胡三省 音註

梁紀二十|{
	起重光協洽盡玄黓涒灘凡二年}


太宗簡文皇帝下

大寶二年春正月新吳余孝頃舉兵拒侯景|{
	漢靈帝中平中立新吳縣屬豫章郡隋省新吳入建昌縣}
景遣于慶攻之不克 庚戌湘東王繹遣護軍將軍尹悦安東將軍杜幼安巴州刺史王珣將兵二萬自江夏趣武昌|{
	江夏今鄂州武昌今壽昌軍將即亮翻夏戶雅翻趣七喻翻 考異曰典略在去年十一月今從太清記}
受徐文盛節度 楊乾運攻拔劔閣楊法昌退保石門|{
	去年武陵王紀遣楊乾運討楊法琛祝穆曰大劒山在今劔門關之劔門縣大劒山西北三十里有小劒山輿地廣記曰山有小石門穿山通道六丈有餘即秦所通石牛道也又魏收志武都郡治石門縣隋為將利縣魏志東益州武興郡有石門縣仇池郡有西石門縣輿地紀勝龍州江油縣東百里有石門戌杜佑曰龍州治江油縣有石門山與氐分界蓋蜀多山險地之以石門名者多矣唐志利州景谷縣西有石門關此蓋楊法昌退保之地}
乾運據南隂平|{
	晉永嘉流寓置南隂平於緜竹之萇楊楊乾運既拔劔閣無緣棄險退據緜竹五代志南梁州有隂平縣宋之北隂平郡也九域志劍州東北五十五里有劍門縣西北百六十里有隂平縣乾運所據其北隂平歟若漢志之隂平道則今之文州是也又在西北故以此為南隂平}
辛亥齊主祀圓丘 張彪遣其將趙稜圍錢塘孫鳳圍富春侯景遣儀同三司田遷趙伯超救之稜鳳敗走 |{
	考異曰典畧去年十一月彪自圍錢塘與趙伯超戰敗於臨平死者八萬餘人走還剡伯超兄子稜在彪軍中謀殺彪偽請與彪盟引小刀披心出血自歃彪信之亦取刀刺血報之刀適至心稜以手按之刀斜入不深彪頓絶稜謂已死出外告彪諸將云彪已死當共求富貴彪左右韓武入視之彪已蘇細聲謂曰我尚沽可與手武遂誅稜彪復奉表於湘東王繹今從太清紀}
稜伯超之兄子也 癸亥齊主耕籍田乙丑享太廟 魏楊忠圍汝南李素戰死|{
	李素納邵陵王綸見上卷上年}
二月乙亥城陷執邵陵攜王綸殺之 |{
	考異曰太清紀云宇文泰遣忠襲綸詐稱來相禮接綸白服與相見執而害之今從梁書南史}
投尸江岸岳陽王詧取而葬之 或告齊太尉彭樂謀反壬辰樂坐誅 齊遣散騎常侍曹文皎使於江陵|{
	散悉亶翻騎奇寄翻使疏吏翻 考異曰典畧在正月甲午朔今從太清紀}
湘東王繹使散騎常侍王子敏報之 侯景以王克為太師宋子仙為太保元羅為太傅郭元建為太尉張化仁為司徒|{
	或曰張化仁即支化仁}
任約為司空王偉為尚書左僕射索超世為右僕射|{
	索蘇各翻}
景置三公官動以十數儀同尤多以子仙元建化仁為佐命元功偉超世為謀主于子悦彭雋主擊斷|{
	斷丁亂翻}
陳慶呂季略盧暉略丁和等為爪牙梁人為景用者則故將軍趙伯超前制局監周石珍内監嚴亶|{
	内監領内器仗}
邵陵王記室伏知命自餘王克元羅及侍中殷不害太常周弘正等景從人望加以尊位非腹心之任也 北兖州刺史蕭邕謀降魏侯景殺之 楊乾運進據平興平興者楊法琛所治也法琛退保魚石洞乾運焚平興而歸|{
	五代志義城郡景谷縣舊曰白水置平興郡唐志武德四年以利州之景谷及龍州之方維置沙州貞觀元年州廢省方維為鎮以景谷還屬利州西有石門關西北有白埧魚老二鎭城九域志文州曲水縣有方維鎭劉煦曰利州景谷縣漢白水縣地宋置平興縣隋為景谷縣}
李遷仕收衆還擊南康|{
	去年李遷仕敗走保寜都}
陳霸先遣其將杜僧明等拒之|{
	將即亮翻}
生擒遷仕斬之 |{
	考異曰太清紀在四月云遷仕追霸先於雩都縣連營相拒百餘日是月廣州刺史蕭物遣歐陽隗水步萬餘人來援隗與戰大破之斬遷仕首餘黨悉降霸先引軍前進今從陳書}
湘東王繹使霸先進兵取江州以為江州刺史三月丙午齊襄城王淯卒 庚戌魏文帝殂|{
	年四十五}
太

子欽立|{
	欽文帝之長子也母乙弗后}
乙卯徐文盛等克武昌進軍蘆洲|{
	蘆洲在武昌西替伍胥去楚出關於江上求渡漁父歌曰灼灼兮已私與子期兮蘆之漪子胥既渡解劒與之辭不受漁父遂覆舟而死即其處水經注漢邾縣故城南對蘆洲蘇軾曰武昌縣劉郎汰正與蘆洲相對伍子胥奔吳所從渡江也亦曰伍洲}
己未齊以湘東王繹為梁相國建梁臺摠百揆承制 齊司空司馬子如自求封王齊主怒庚子免子如官 任約告急侯景自帥衆西上攜太子大器從軍以為質|{
	帥讀曰率上時掌翻質音致}
留王偉居守|{
	守手又翻}
閏月景建康 |{
	考異曰梁帝紀三月丁未景京師典略云閏三月丁未按乙卯徐文盛克武昌不容丁未景已建康閏三月甲戌朔無丁未蓋字誤也}
自石頭至新林舳艫相接|{
	舳音逐艫音盧}
約分兵襲破定州刺史田龍祖於齊安|{
	上卷作田祖龍此作田龍祖必有一誤五代志黄州黄岡縣齊曰南安置齊安郡九域志黄岡縣有齊安鎭}
壬寅景軍至西陽與徐文盛夾江築壘癸卯文盛擊破之射其右丞庫狄式和墜水死|{
	射而亦翻}
景遁走還營 夏四月甲辰魏葬文帝於永陵 郢州刺史蕭方諸年十五以行事鮑泉和弱常侮易之|{
	易以豉翻}
或使伏牀騎背為馬恃徐文盛軍在近不復設備日以蒲酒為樂|{
	復扶又翻蒲酒樗蒲飲酒也樂音洛}
侯景聞江夏空虛|{
	夏戶雅翻}
乙巳使宋子仙任約帥精騎四百由淮内襲郢州|{
	自西陽至江夏百五十餘里景使宋子仙等蓋由蘆洲上流渡兵以襲之帥讀曰率騎奇寄翻下同}
丙午大風疾雨天色晦冥有登陴望見賊者告泉曰虜騎至矣泉曰徐文盛大軍在下賊何由得至當是王珣軍人還耳既而走告者稍衆始命閉門子仙等已入城方諸方踞泉腹以五色綵辮其髯|{
	辮補典翻父結也}
見子仙至方諸迎拜泉匿於牀下子仙俯窺見泉素髯間綵|{
	間古莧翻}
驚愕遂擒之及司馬虞豫送於景所景因便風中江舉帆遂越文盛等軍丁未入江夏文盛衆懼而潰與長沙王韶等逃歸江陵|{
	上甲侯韶去年繹封為長沙王}
王珣杜幼安以家在江夏遂降於景|{
	降戶江翻}
湘東王繹以王僧辯為大都督帥巴州刺史丹陽淳于量定州刺史杜龕宜州刺史王琳郴州刺史裴之横東擊景|{
	五代志巴陵郡梁置巴州夷陵郡梁置宜州柱陽郡梁置郴州龕苦含翻郴丑林翻}
徐文盛以下並受節度龕岸之兄子也戊申僧辯等軍至巴陵|{
	自江陵至巴陵四百一十里自巴陵至江夏三百五十里}
聞郢州已陷因留戍之繹遺僧辯書曰賊既乘勝必將西下|{
	遺于季翻自江夏指江陵當作西上}
不勞遠擊但守巴丘|{
	巴丘即巴陵有巴丘山}
以逸待勞無虜不克又謂將佐曰賊若水步兩道直指江陵此上策也據夏首積兵糧中策也悉力攻巴陵下策也巴陵城小而固僧辯足可委任景攻城不拔野無所掠暑疫時起食盡兵疲破之必矣|{
	湘東安能料敵如此當時作史者為之辭耳將即亮翻夏戶雅翻}
乃命羅州刺史徐嗣徽自岳陽武州刺史杜崱自武陵引兵會僧辯|{
	五代志湘隂縣梁置羅州及岳陽郡湘隂隋屬巴陵郡又梁置武州於武陵郡崱士力翻}
景使丁和將兵五千守夏首宋子仙將兵一萬為前驅趨巴陵|{
	趨七喻翻}
分遣任約直指江陵景帥大兵水步繼進於是緣江戍邏望風請服景拓邏至于隱磯|{
	拓斥開也邏遮也廵也拓開廵邏以張兵勢邏魯可翻又魯佐翻水經江水自公安而東過下雋縣北又東逕彭城磯北彭城磯北對隱磯二磯之間大江之中也}
僧辯乘城固守偃旗卧鼔安若無人壬戌景衆濟江|{
	自隱磯濟江 考異曰梁帝紀作甲子今從太清記}
遣輕騎至城下問城内為誰答曰王領軍騎曰何不早降|{
	降戶江翻}
僧辯曰大軍但向荆州此城自當非礙|{
	言兵若向荆州此城當非遮礙}
騎去頃之執王珣等至城下使說其弟琳琳曰兄受命討賊不能死難|{
	說式芮翻難乃旦翻}
曾不内慙翻欲賜誘取弓射之珣慙而退景肉薄百道攻城|{
	薄伯各翻}
城中鼓譟矢石雨下景士卒死者甚衆乃退僧辯遣輕兵出戰凡十餘返皆捷景披甲在城下督戰|{
	誘音酉射而亦翻被皮義翻}
僧辯著綬乘輿奏鼓吹廵城|{
	著陟略翻綬音受吹昌瑞翻}
景望之服其膽勇岳陽王詧聞侯景克郢州遣蔡大寶將兵一萬進據武寧|{
	五代志竟陵郡樂鄉縣舊置武寧郡劉昫曰樂鄉漢鄀縣也其地當在今郢州長壽縣西北宋白曰桓玄立武寧郡於故編縣城其屬有長林縣與郡俱立分編縣所置也}
遣使至江陵詐稱赴援|{
	使疏吏翻}
衆議欲答以侯景已破令其退軍湘東王繹曰今語以退軍|{
	語牛倨翻}
是趣之令進也|{
	趣讀曰促}
乃使謂大寶曰岳陽累啓連和不相侵犯卿那忽據武寧今當遣天門太守胡僧祐精甲二萬鐵馬五千頓湕水待時進軍詧聞之召其軍還|{
	湕居偃翻還從宣翻又如字}
僧祐南陽人也 五月魏隴西襄公李虎卒 侯景晝夜攻巴陵不克軍中食盡疾疫死傷太半湘東王繹遣晉州刺史蕭惠正將兵援巴陵惠正辭不堪舉胡僧祐自代僧祐時坐謀議忤旨繫獄|{
	據梁書胡僧祐傳時西沮蠻反使僧祐討之令盡誅其渠帥僧祐諫忤旨下獄忤五故翻}
繹即出之拜武猛將軍令赴援戒之曰賊若水戰但以大艦臨之必克|{
	艦戶黯翻}
若欲步戰自可鼓棹直就巴丘不須交鋒也僧祐至湘浦|{
	湘水入湖之口曰浦}
景遣任約帥銳卒五千據白塉以待之|{
	塉又尺翻}
僧祐由它路西上|{
	上時掌翻}
約謂其畏已急追之及於芊口|{
	據姚思亷梁書芊口在南平郡安南縣界}
呼僧祐曰吳兒何不早降|{
	降戶江翻}
走何所之僧祐不應潜引兵至赤沙亭|{
	水經注灃水過作唐縣北而東逕安南縣南又東與赤湖水會二縣皆屬南平郡巴陵志洞庭湖在巴丘西西吞赤沙南連青草横亘七八百里又有赤學城二面臨水即胡僧祐所據杜佑曰巴陵郡西華容界有赤亭城城近赤亭湖因名任約擒於此}
會信州刺史陸法和至與之合軍法和有異術隱於江陵百里洲|{
	盛弘之荆州記曰百里洲在枝江縣縣左右有數十洲槃布川中百里洲為最大}
衣食居處一如苦行沙門|{
	苦行沙門謂沙門能清苦守戒行者也處昌呂翻行下孟翻}
或豫言吉凶多中|{
	中竹仲翻}
人莫能測侯景之圍臺城也或問之曰事將何如法和曰凡人取果宜待熟時不撩自落|{
	撩落雕翻攏取物也}
固問之法和曰亦克亦不克|{
	亦克謂侯景取臺城不克謂景終於敗也法和若設為兩端之言使人莫之測而卒之言驗}
及任約向江陵法和自請擊之繹許之壬寅約至赤亭六月甲辰僧祐法和縱兵擊之約兵大潰殺溺死者甚衆|{
	溺奴狄翻}
擒約送江陵景聞之乙巳焚營宵遁以丁和為郢州刺史留宋子仙等衆號二萬戍郢城别將支化仁鎭魯山 |{
	考異曰梁帝紀作魏司徒張化仁按魏司徒安得為景守城今從典略}
范希榮行江州事 |{
	考異曰典略云江州刺史今從太清紀}
儀同三司任延和晉州刺史夏侯威生守晉州|{
	梁於晉熙郡置晉州}
景與麾下兵數千順流而下丁和以大石磕殺鮑泉及虞預沈於黄鶴磯|{
	磕古盍翻沈持林翻祝穆曰黄鶴山一名黄鵠山在江夏縣東九里近縣西北二里有黄鶴磯水經注黄鵠山東北對夏口城黄鵠磯直鸚鵡洲之下尾}
任約至江陵繹赦之徐文盛坐怨望下獄死|{
	下遐稼翻下東下同}
巴州刺史余孝頃遣兄子僧重將兵救鄱陽于慶退走|{
	余孝頃起於新吳梁授以巴州刺史考異曰長歷六月癸卯朔太清紀一日慶走二日擒任約三日景走今從梁帝紀}
繹以王僧辯

為征東將軍尚書令胡僧祐等皆進位號使引兵東下陸法和請還既至謂繹曰侯景自然平矣蜀賊將至請守險以待之|{
	法和知武陵王紀必東下}
乃引兵屯峽口|{
	巫峽之口也}
庚申王僧辯至漢口先攻魯山擒支化仁送江陵辛酉攻郢州克其羅城斬首千級宋子仙退據金城僧辯四面起土山攻之豫州刺史荀朗自巢湖出濡須邀景破其後軍|{
	荀朗起兵據巢湖帝密詔授豫州刺史使討景}
景奔歸船前後相失太子船入樅陽浦|{
	樅七容翻}
船中腹心皆勸太子因此入北太子曰自國家喪敗志不圖生主上蒙塵寧忍違離左右|{
	喪息浪翻離力智翻}
吾今若去是乃叛父非避賊也因涕泗嗚咽即命前進|{
	毛詩注自目曰涕自鼻曰泗史言哀太子之孝}
甲子宋子仙等困蹙乞輸郢城身還就景王僧辯偽許之命給船百艘以安其意|{
	艘蘇刀翻}
子仙謂為信然浮舟將僧辯命杜龕帥精勇千人攀堞而上|{
	龕苦含翻帥讀曰率下同堞達協翻上時掌翻}
鼔譟奄進水軍主宋遥帥樓船暗江雲合|{
	言樓船四合如雲江為之暗}
子仙且戰且走至自楊浦|{
	白楊浦蓋去郢城未遠}
大破之周鐵虎生擒子仙及丁和送江陵殺之 庚午齊主以司馬子如高祖之舊復以為太尉|{
	是年三月齊主免子如官復扶又翻}
江安侯圓正為西陽太守|{
	宋白曰江安縣本漢江陽縣地}
寛和好施|{
	好呼到翻施式智翻}
歸附者衆有兵一萬湘東王繹欲圖之署為平南將軍及至弗見使南平王恪與之飲醉因囚之内省分其部曲使人告其罪荆益之舋自此起矣|{
	圓正武陵王紀第二子也為紀東下攻繹張本舋許覲翻}
陳霸先引兵發南康灨石舊有二十四灘會水暴漲數丈三百里間巨石皆没霸先進頓西昌|{
	章貢圖經曰東江源於汀州界之新樂山經雩都而會于章水西江導源於南安大庾縣之聶都山與貢水合會于贛水二水合而為贛在州治後北流一百八十里至萬安縣界由萬安而上為灘十有八怪石如精鐵突兀亷厲錯峙波面自贛水而上信豐寧都俱有石磧險阻視十八灘故俚俗以為上下三百里贛石吳立西昌縣屬廬陵郡今在吉州太和縣界}
鐵勒將伐柔然突厥酋長土門邀擊破之盡降其衆

五萬餘落|{
	厥九勿翻酋慈秋翻長知兩翻降戶江翻}
土門恃其彊盛求婚於柔然柔然頭兵可汗大怒使人詈辱之曰爾我之鍛奴也|{
	突厥本柔然鐵工故云然可從刋入聲汗音寒詈力智翻}
何敢發是言土門亦怒殺其使者遂與之絶而求婚於魏魏丞相泰以長樂公主妻之|{
	使疏吏翻樂音洛妻七細翻}
秋七月乙亥湘東王繹以長沙王韶監郢州事|{
	監工衘翻}
丁亥侯景還至建康 |{
	考異曰典略作六月壬戌太清紀作七月二十日今從梁帝紀}
于慶自鄱陽還豫章侯瑱閉門拒之慶走江州據郭默城|{
	走音奏晉將郭默反時所築城也瑱它甸反又音鎮}
繹以瑱為兖州刺史|{
	據瑱傳授南兖州刺史}
景悉殺瑱子弟|{
	侯景留瑱子弟為質見上卷上年}
辛丑王僧辯乘勝下湓城|{
	下遐嫁翻}
陳霸先帥所部三萬人將會之屯于巴丘|{
	此吳所置巴丘縣也屬廬陵郡界帥讀曰率}
西軍乏食|{
	王僧辯之軍自荆州來故謂之西軍}
霸先有糧五十萬石分三十萬石以資之八月壬寅朔王僧辯前軍襲于慶慶弃郭默城走范希榮亦弃尋陽城走晉熙王僧振等起兵圍郡城僧辯遣沙州刺史丁道貴助之|{
	魏收志梁武帝置沙州治白沙關城領建寧齊安郡當在黄州黄岡縣界}
任延和等弃城走|{
	任音壬}
湘東王繹命僧辯且頓尋陽以待諸軍之集初景既克建康常言吳兒怯弱易以掩取|{
	易弋豉翻}
當須拓定中原然後為帝景尚帝女溧陽公主嬖之妨於政事|{
	溧音栗嬖卑義翻又博計翻}
王偉屢諫景景以告主主有惡言偉恐為所讒因說景除帝|{
	說式芮翻}
及景自巴陵敗歸猛將多死|{
	謂宋子仙之屬將即亮翻}
自恐不能久存欲早登大位王偉曰自古移鼎|{
	武王克商遷九鼎於洛邑故後之奪人之國者率謂之移鼎}
必須廢立既示我威權且絶彼民望景從之使前壽光殿學士謝昊為詔書以為弟姪爭立|{
	弟謂湘東王繹武陵王紀姪謂河東王譽岳陽王詧}
星辰失次皆由朕非正緒召亂致災宜禪位於豫章王棟使呂季略齎入逼帝書之棟歡之子也|{
	華容公歡昭明太子之子}
戊午景遣衛尉卿彭雋等帥兵入殿廢帝為晉安王幽於永福省悉撤内外侍衛使突騎左右守之牆垣悉布枳棘|{
	枳似橘而多刺棘似棗而多刺帥讀曰率騎奇寄翻枳諸氏翻}
庚申下詔迎豫章王棟棟時幽拘廩餼甚薄|{
	餼許氣翻}
仰蔬茹為食方與妃張氏鉏葵|{
	葵菜也}
法駕奄至棟驚不知所為泣而升輦景殺哀太子大器尋陽王大心西陽王大鈞建平王大球義安王大昕及王侯在建康者二十餘人太子神明端嶷|{
	嶷魚力翻}
於景黨未嘗屈意所親竊問之太子曰賊若於事義未須見殺|{
	事義猶言事宜也}
吾雖陵慢呵叱終不敢言若見殺時至雖一日百拜亦無所益又曰殿下今居困阨而神貌怡然不貶平日何也|{
	貶損也}
太子曰吾自度死日必在賊前|{
	度徒洛翻}
若諸叔能滅賊賊必先見殺然後就死若其不然賊亦殺我以取富貴安能以必死之命為無益之愁乎及難|{
	難乃旦翻}
太子顔色不變徐曰久知此事嗟其晩耳刑者將以衣帶絞之太子曰此不能見殺命取帳繩絞之而絶壬戌棟即帝位 |{
	考異曰典略作壬辰誤今從太清紀}
大赦改元天正太尉郭元建聞之自秦郡馳還謂景曰主上先帝太子既無愆失何得廢之景曰王偉勸吾云早除民望吾故從之以安天下元建曰吾挾天子令諸侯猶懼不濟無故廢之乃所以自危何安之有景欲迎帝復位以棟為太孫王偉曰廢立大事豈可數改邪乃止|{
	數所角翻}
乙丑景又使殺南海王大臨於吳郡南郡王大連於姑孰 |{
	考異曰太清紀云於九江今從梁書}
安陸王大春於會稽高唐王大壯於京口|{
	大臨時為吳郡太守大連時為江州刺史在姑孰大春時為東揚州刺史姚思亷梁書大壯作大莊始封高唐郡公後進封新興王時為南徐州刺史皆就殺之會工外翻}
以太子妃賜郭元建元建曰豈有皇太子妃乃為人妾乎竟不與相見聽使入道丙寅追尊昭明太子為昭明皇帝豫章安王為安皇帝|{
	華容公歡進封豫章王薨諡曰安}
金華敬妃為敬太皇太后|{
	昭明太子妃蔡氏昭明既薨武帝為妃别立金華宫供侍一同常儀薨諡曰敬按敬妃已薨只當追諡皇后以從夫曰太皇太后非也}
豫章太妃王氏為皇太后妃張氏為皇后以劉神茂為司空 九月癸巳齊主如趙定二州|{
	五代志趙郡大陸縣舊曰廣同置殷州及南鉅鹿郡後改為南趙郡改殷州為定州治中山}
遂如晉陽己亥湘東王繹以尚書令王僧辯為江州刺史江州刺史陳霸先為東揚州刺史 王偉說侯景弑太宗以絶衆心|{
	侯景尊帝廟號曰高宗元帝追諡曰簡文皇帝廟號太宗說式芮翻}
景從之冬十月壬寅夜偉與左衛將軍彭雋王修纂進酒於太宗曰丞相以陛下幽憂既久使臣等來上壽|{
	上時掌翻}
太宗笑曰已禪帝位何得言陛下此壽酒將不盡此乎|{
	言夀將盡於此酒}
於是雋等齎曲項琵琶|{
	杜佑曰傅玄琵琶賦云漢遣公主嫁烏孫念其行道思慕故使工人裁筝筑為馬上之樂今觀其器中虛外實天地象也盤圓柄直除陽叙也柱十有二配律呂也四絃法四時也以方俗語之曰琵琶取其易傳於外國也風俗通云以手琵琶因以為名釋名曰推手前曰祕引手却曰把杜縶曰秦人苦長城之役絃鼗而鼔之並未詳孰是今清樂奏琵琶俗謂之秦漢子圓頭修頸而小疑是絃鼗之遺制傅玄云體圓柄直柱十有二其他皆兌上銳下曲項形制稍大本出胡中俗傳是漢制兼用兩制者謂之秦漢據此事則南朝似無曲項琵琶劉昫曰琵琶四絃曲項琵琶五絃出胡中}
與太宗極飲太宗知將見殺因盡醉曰不圖為樂之至於斯也|{
	樂音洛}
既醉而寢偉乃出雋進土囊修纂坐其上而殂|{
	年四十九}
偉撤門扉為棺遷殯於城北酒庫中太宗自幽縶之後無復侍者及紙|{
	復扶又翻}
乃書壁及板障|{
	柱間不為壁以板為障施以丹漆因謂之板障}
為詩及文數百篇辭甚悽愴|{
	愴丑亮翻}
景諡曰明皇帝廟號高宗 侯景之逼江陵也湘東王繹求援於魏命梁秦二州刺史宜豐侯循以南鄭與魏|{
	晉志豫章郡有宜豐縣}
召循還江陵循以無故輸城非忠臣之節|{
	考異曰南史宜豐侯修今從梁書}
報曰請待改命魏太師泰遣大將軍達奚武將兵三萬取漢中|{
	將即亮翻}
又遣大將軍王雄出子午谷攻上津|{
	五代志西城郡豐利縣梁置南上洛郡西魏改郡曰豐利後周省郡入上津郡唐以上津為縣屬商州}
循遣記室參軍沛人劉璠求援於武陵王紀|{
	璠孚袁翻}
紀遣潼州刺史楊乾運救之循恢之子也|{
	鄱陽王恢武帝之弟}
王僧辯等聞太宗殂丙辰啓湘東王繹請上尊號|{
	上時掌翻下復上同考異曰典略作乙卯今從太清紀}
繹弗許 司空東道行臺劉神茂聞侯景自巴丘敗還隂謀叛景吳中士大夫咸勸之乃與儀同三司尹思合劉歸義王曅雲麾將軍元頵等據東陽以應江陵|{
	頵居筠翻又紆綸翻}
遣頵及别將李占下據建德江口|{
	五代志東陽郡金華縣隋廢建德縣入焉唐武德四年復置建德縣分為睦州治所今東陽江新安江合於州城南十里將即亮翻}
張彪攻永嘉克之新安民程靈洗起兵據郡以應神茂於是浙江以東皆附江陵湘東王繹以靈洗為譙州刺史領新安太守|{
	程靈洗字玄滌洗讀如字湘東以刺史寵靈洗實領新安太守職}
十一月乙亥王僧辯等復上表勸進|{
	復扶又翻}
湘東王繹不許戊寅繹以湘州刺史安南侯方矩為中衛將軍以自副 |{
	考異曰梁書在八月辛亥今從太清紀}
方矩方諸之弟也以南平王恪為湘州刺史侯景以趙伯超為東道行臺據錢塘以田遷為軍司據富春以李慶緒為中軍都督謝答仁為右廂都督李遵為左廂都督以討劉神茂己卯加侯景九錫漢國置丞相以下官己丑豫章王

棟禪位于景景即皇帝位于南郊還登太極殿其黨數萬皆吹脣鼓譟而上|{
	上時掌翻}
大赦改元太始封棟為淮陰王并其二弟橋樛同鎻於密室|{
	樛居蚪翻}
王偉請立七廟景曰何謂七廟偉曰天子祭七世祖考并請七世諱景曰前世吾不復記唯記吾父名標且彼在朔州那得來噉此|{
	侯景本懷朔鎭人魏改懷朔鎭為朔州}
衆咸笑之景黨有知景祖名乙羽周者自外皆王偉制其名位追尊父標為元皇帝景之作相也以西州為府文武無尊卑皆引接及居禁中非故舊不得見由是諸將多怨望|{
	將即亮翻}
景好獨乘小馬彈射飛鳥|{
	好呼到翻彈射飛鳥北俗也彈徒丹翻射而亦翻}
王偉每禁止之不許輕出景欝欝不樂|{
	樂音洛}
更成失志曰吾無事為帝與受擯不殊|{
	擯弃也擯斥者不得預人事故景以為言}
壬辰湘東王以長沙王韶為郢州刺史 益州長史劉孝勝等勸武陵王紀稱帝紀雖未許而大造乘輿車服|{
	乘輿車服天子之車服也乘繩證翻}
十二月丁未謝答仁李慶緒攻建德擒元頵李占送建康景截其手足以狥經日乃死 齊主每出入常以中山王自隨王妃太原公主恒為之飲食護視之|{
	太原公主勃海王歡之女恒戶登翻為于偽翻飲於禁翻下同食祥吏翻}
是月齊主飲公主酒使人鴆中山王殺之并其三子諡王曰魏孝静皇帝葬於鄴西漳北其後齊主忽掘其陵投梓宫於漳水齊主初受禪魏神主悉寄於七帝寺|{
	以寄魏七廟神主故謂之七帝寺}
至是亦取焚之彭城公元韶以高氏壻|{
	元韶娶魏孝武帝后高歡之女也}
寵遇異於諸元開府儀同三司美陽公元暉業|{
	魏濟陰王暉業齊受禪降封美陽公}
以位望隆重又志氣不倫尤為齊主所忌從齊主在晉陽暉業於宫門外罵韶曰爾不及一老嫗負璽與人|{
	嫗威遇翻韶奉璽於齊見上卷上年璽斯氏翻}
何不擊碎之我出此言知即死爾亦詎得幾時齊主聞而殺之及臨淮公元孝友皆鑿汾水氷沈其尸孝友彧之弟也|{
	彧魏臨淮王沈持林翻}
齊主嘗剃元韶鬢鬚加之粉黛以自隨曰吾以彭城為嬪御言其懦弱如婦人也|{
	剃它計翻嬪毗賓翻懦奴亂翻}


世祖孝元皇帝上|{
	諱繹字世誠小字七符武帝第七子也}


承聖元年|{
	是年十一月方即位改元}
春正月湘東王以南平内史王襃為吏部尚書襃騫之孫也|{
	王騫儉之孫}
齊人屢侵侯景邉地甲戌景遣郭元建帥步軍趨小峴|{
	帥讀曰率下同趣七喻翻峴戶典翻}
侯子鑒帥舟師向濡須己卯至合肥 |{
	考異曰典略二月庚子子鑒等圍合肥克其羅城今從太清紀}
齊人閉門不出乃引還 丙申齊主伐庫莫奚大破之俘獲四千人雜畜十餘萬|{
	畜許六翻}
齊主連年出塞給事中兼中書舍人唐邕練習軍書自督將以降勞効本末及四方軍士彊弱多少番代往還器械精粗|{
	將即亮翻少詩詔翻粗倉呼翻}
糧儲虚實靡不諳悉|{
	諳烏含翻}
或於帝前簡閱雖數千人不執文簿唱其姓名未嘗謬誤帝嘗曰唐邕彊幹一人當千又曰邕每有軍事手作文書口且處分|{
	處昌呂翻分扶問翻}
耳又聽受實異人也寵待賞賜羣臣莫及 魏將王雄取上津魏興東梁州刺史安康李遷哲軍敗降之|{
	五代忠西城郡安康縣舊曰寧都齊置安康郡後魏置東梁州按蕭子顯齊書安康寧都二縣皆齊所置魏收曰安康縣漢時漢中郡之安陽縣也將即亮翻降戶江翻}
突厥土門襲擊柔然大破之柔然頭兵可汗自殺|{
	厥居勿翻可從刋入聲汗音寒}
其太子菴羅辰及阿那瓌從弟登注俟利登注子庫提並帥衆奔齊餘衆復立登注次子鐵伐為王|{
	從才用翻帥讀曰率下同復扶又翻}
土門自號伊利可汗號其妻為可賀敦子弟謂之特勒 |{
	考異曰諸書或作勑勤今從劉昫舊唐書及宋祁新唐書}
别將兵者皆謂之設 湘東王命王僧辯等東擊侯景二月庚子諸軍尋陽舳艫數百里陳霸先帥甲士三萬舟艦二千自南江出湓口會僧辯於白茅灣|{
	贑水謂之南江過彭澤縣西注於彭蠡北入于江白茅灣在桑落洲西南史王僧辯傳霸先帥衆五萬出自南江前軍五千行至湓口蓋水陸俱下也帥讀曰率}
築壇歃血|{
	歃色甲翻}
共讀盟文流涕慷慨癸卯僧辯使侯瑱襲南陵鵲頭二戍克之戊申僧辯等軍于大雷丙辰鵲頭戊午侯子鑒還至戰鳥|{
	子鑒蓋自合肥還杜佑曰宣州南陵縣有鵲洲有戰鳥圻孤在江中本名孤圻昔桓温舉兵東下住此圻中宵鳥驚温謂官軍圍之既而定以羣鳥驚噪因名戰鳥唐顧况集有題靈山寺詩下注戰鳥蓋戰鳥圻後為靈山寺}
西軍奄至子鑒驚懼奔還淮南|{
	淮南今太平州當塗之地}
侯景儀同三司謝答仁攻劉神茂於東陽程靈洗張彪皆勒兵將救之神茂欲專其功不許營於下淮或謂神茂曰賊長於野戰下淮地平四面受敵不如據七里瀨|{
	七里灘在桐廬縣距建德四十餘里與嚴陵瀨相接}
賊必不能進不從神茂偏禆多北人不與神茂同心别將王曅酈通並據外營降於答仁劉歸義尹思合等弃城走神茂孤危辛未亦降於答仁|{
	降戶江翻}
答仁送之建康癸酉王僧辯等至蕪湖侯景守將張黑弃城走景聞之甚懼下詔赦湘東繹王僧辯之罪衆咸笑之侯子鑒據姑孰南洲以拒西師景遣其黨史安和等將兵二千助之三月己巳朔景下詔欲自至姑孰又遣人戒子鑒曰西人善水戰勿與爭鋒往年任約之敗良為此也|{
	去年景與徐文盛水戰亦敗走不特任約也以故畏之為于偽翻}
若得步騎一交必當可破汝但結營㟁上引船入浦以待之子鑒乃捨舟登岸閉營不出僧辯等停軍蕪湖十餘日景黨大喜告景曰西師畏吾之彊勢將遁矣不擊且失之景乃復命子鑒為水戰之備|{
	復扶又翻}
丁丑僧辯至姑孰子鑒帥步騎萬餘人度洲於岸挑戰又以鵃䑠千艘載戰士|{
	類篇曰鵃䑠船長貌玉篇曰鵃䑠小船也集韻鵃丁了翻䑠胡鳥翻按王僧辯傳鵃䑠其中載士兩邉悉八十棹棹手皆越人去來趣襲捷過風電蓋今之水哨馬即其類 考異曰典略作烏鵲舫千艘今從梁書}
僧辯麾細船皆令退縮留大艦夾泊兩岸子鑒之衆謂水軍欲退爭出趨之|{
	趨七喻翻}
大艦斷其歸路|{
	斷音短}
鼓譟大呼合戰中江|{
	呼火故翻}
子鑒大敗士卒赴水死者數千人子鑒僅以身免收散卒走還建康據東府僧辯留虎臣將軍莊丘慧達鎭姑孰引軍而前歷陽戍迎降|{
	降戶江翻}
景聞子鑒敗大懼涕下覆面|{
	覆敷又翻}
引衾而卧良久方起歎曰誤殺乃公庚辰僧辯督諸軍至張公洲|{
	張公洲即蔡州考異曰典略作戊寅今從太清紀}
辛巳乘潮入淮進至禪靈寺前|{
	禪靈寺齊武帝所建}
景召石頭津主張賓使引淮中舣䑰及海艟|{
	舣音义䑰蒲故翻方言船短而深謂之䑰艟尺庸翻}
以石縋之塞淮口|{
	縋雖偽翻塞悉則翻}
緣淮作城自石頭至于朱雀街十餘里中樓堞相接|{
	堞達協翻}
僧辯問計於陳霸先霸先曰前柳仲禮數十萬兵隔水而坐韋粲在青溪竟不度岸賊登高望之表裏俱盡故能覆我師徒|{
	事見一百六十二卷武帝太清三年}
今圍石頭須度北岸諸將若不能當鋒霸先請先往立柵壬午霸先於石頭西落星山築柵 |{
	考異曰陳書云横隴立柵今從典略}
衆軍次連八城直出石頭西北景恐西州路絶自帥侯子鑒等亦於石頭東北築五城以遏大路|{
	帥讀曰率}
景使王偉守臺城乙酉景殺湘東王世子方諸前平東將軍杜幼安 劉神茂至建康丙戌景命為大剉碓先進其足寸寸斬之以至於頭|{
	劉神茂始導侯景取夀陽及其度江又為爪牙東南之禍神茂實為之其死晚矣大剉碓者為大剉刀傳機如碓使人踏之碓都内翻}
留異外同神茂而濳通於景故得免禍 丁亥王僧辯進軍招提寺北|{
	招提寺在石頭城北}
侯景帥衆萬餘人鐵騎八百餘匹陳於西州之西|{
	陳讀曰陣下志陳衝陳陳不同}
陳霸先曰我衆賊寡應分其兵勢以彊制弱何故聚其鋒銳令致死於我乃命諸將分處置兵景衝將軍王僧志陳僧志小縮霸先遣將軍安陸徐度將弩手二千横截其後|{
	弩矢之力可以及遠横截其後箭鋒所到敵必驚却}
景兵乃却霸先與王琳杜龕等以鐵騎乘之僧辯以大兵繼之景兵敗退據其柵龕岸之兄子也|{
	杜岸死於太清三年荆雍構難之時龕岸兄岑之子龕苦含翻}
景儀同三司盧暉略守石頭城開北門降|{
	降戶江翻}
僧辯入據之景與霸先殊死戰景帥百餘騎弃矟執刀左右衝陳|{
	矟色角翻}
陳不動衆遂大潰諸軍逐北至西明門|{
	西明門建康外城西中門}
景至闕下不敢入臺召王偉責之曰爾令我為帝今日誤我偉不能對繞闕而藏景欲走偉執鞚諫曰|{
	鞚苦貢翻}
自古豈有叛天子邪宫中衛士猶足一戰弃此將欲安之景曰我昔敗賀拔勝|{
	見一百五十六卷武帝中大通六年敗補邁翻}
破葛榮|{
	見一百五十二卷大通二年}
揚名河朔度江平臺城降柳仲禮如反掌|{
	見一百六十有二卷太清三年降戶江翻}
今日天亡我也因仰觀石闕歎息久之以皮囊盛其江東所生二子|{
	盛時征翻景至建康所生子也}
掛之鞍後與房世貴等百餘騎東走欲就謝答仁於吳侯子鑒王偉陳慶奔朱方僧辯命裴之横杜龕屯杜姥宅杜崱入據臺城僧辯不戢軍士剽掠居民男女祼露|{
	姥莫補翻崱士力翻戢則立翻剽匹妙翻祼郎果翻}
自石頭至于東城號泣滿道|{
	號戶刀翻}
是夜軍士遺火焚太極殿及東西堂寶器羽儀輦輅無遺|{
	史言王僧辯御軍無法失伐罪弔民肅清禁輦之意}
戊子僧辯命侯瑱等帥精甲五千追景王克元羅等帥臺内舊臣迎僧辯於道僧辯勞克曰|{
	勞力到翻}
甚苦事夷狄之君克不能對又問璽紱何在|{
	璽斯氏翻紱音弗}
克良久曰趙平原持去|{
	侯景侍中趙思賢兼平原太守}
僧辯曰王氏百世卿族一朝而墜僧辯迎太宗梓宫升朝堂帥百官哭踊如禮|{
	朝直遥翻}
己丑僧辯等上表勸進|{
	上時掌翻 考異曰梁帝紀戊子王以賊平吿明堂大社己丑僧辯等奉表按表文云衆軍以戊子總集建康豈是日告捷即能達江陵乎盖僧辯等以己丑日表勸進耳}
且迎都建業湘東王答曰淮海長鯨雖云授首襄陽短狐未全革面|{
	禹貢曰淮海惟揚州長鯨謂侯景古者伐國取其鯨鯢以為大戮岳陽王詧據襄陽與湘東為敵故斥為短狐短狐蜮也含沙射人中之者死易曰小人革面}
太平玉燭爾乃議之|{
	爾雅春為青陽夏為朱明秋為白藏冬為玄英四時和謂之玉燭釋云此釋太平之時四時和暢以致嘉祥之事也云春為青陽者言春之氣和則青而温陽也云夏為朱明者言夏之氣和則赤而光明也云秋為白藏者言秋之氣和則白而收藏也云冬為玄英者言冬之氣和則黑而清英也四時和謂之玉燭者言四時和氣温潤明照故謂之玉燭李巡云人君德美如玉而明若燭聘義君子比德於玉焉是時人君若德輝動於内則和氣應於外統而言之謂之玉燭}
庚寅南兖州刺史郭元建秦郡戍主郭正買|{
	侯景以廣陵為南兖州因南朝之舊也}
陽平戍主魯伯和行南徐州事郭子仲並據城降|{
	降戶江翻}
僧辯之江陵也啓湘東王曰平賊之後嗣君萬福未審何以為禮王曰六門之内自極兵威|{
	臺城六門大司馬門萬者門東華門西華門太陽門承明門}
僧辯曰討賊之謀臣為己任成濟之事請舉别人|{
	成濟弑魏高貴鄉公王僧辯欲避弑君之惡名故云}
王乃密諭宣猛將軍朱買臣使為之所及景敗太宗已殂豫章王棟及二弟橋樛相扶出於密室|{
	樛居蚪翻}
逢杜崱於道為去其鎻二弟曰今日始免横死矣|{
	為于偽翻去羌呂翻横尸孟翻}
棟曰倚伏難知吾猶有懼|{
	賈誼鵩賦曰禍兮福所倚福兮禍所伏}
辛卯遇朱買臣呼之就船共飲未竟並沈於水|{
	沈持林翻}
僧辯遣陳霸先將兵向廣陵受郭元建等降又遣使者往安慰之|{
	將即亮翻降戶江翻使疏吏翻}
諸將多私使别索馬仗|{
	索山客翻}
會侯子鑒度江至廣陵謂元建等曰我曹梁之深讐何顔復見其主|{
	復扶又翻}
不若投北可得還鄉遂皆降齊霸先至歐陽齊行臺辛術已據廣陵王偉與侯子鑒相失直瀆戍主黄公喜獲之送建康|{
	孫盛晉春秋曰直瀆在方山王安石詩山蟠直瀆輸淮口陸游曰直瀆吳孫氏所開温庭筠過吳景帝陵詩虛開直瀆三千里青蓋何曾到洛陽按侯子鑒王偉已度江此非方山之直瀆也沈約志旴太守管下有直瀆今晉安帝立縣}
王僧辯問曰卿為賊相不能死節而求活草間邪偉曰廢興命也使漢帝早從偉言明公豈有今日|{
	謂臺城之破僧辯已降侯景縱還竟陵使有今日偉之此言亦以愧僧辯也}
尚書左丞虞隲嘗為偉所辱乃唾其面偉曰君不讀書不足與語隲慙而退|{
	隲之日翻唾吐卧翻}
僧辯命羅州刺史徐嗣徽鎭朱方壬辰侯景至晉陵得田遷餘兵|{
	田遷東攻劉神茂有餘兵在晉陵}
因驅掠居民東趨吳郡|{
	趨七喻翻}
夏四月齊主使大都督潘樂與郭元建將兵五萬攻陽平拔之王僧辯啓陳霸先鎭京口 |{
	考異曰陳紀高祖應接郭元建還僧辯啟高祖鎭京}


|{
	口按是時徐嗣徽為南徐州刺史蓋霸先但領兵戍京口耳未為刺史也}
益州刺史太尉武陵王紀頗有武略在蜀十七年南開寧州越雟西通資陵吐谷渾内修耕桑鹽鐵之政外通商賈遠方之利|{
	嶲音髓吐從暾入聲谷音浴賈音古}
故能殖其財用器甲殷積有馬八千匹聞侯景陷臺城湘東王將討之謂僚佐曰七官文士豈能匡濟|{
	湘東於兄弟次第七故呼為七官}
内寢柏殿柱繞節生花紀以為已瑞乙巳即皇帝位改元天正立子圓照為皇太子圓正為西陽王圓滿為竟陵王圓普為譙王圓肅為宜都王以巴西梓潼二郡太守永豐侯撝為征西大將軍益州刺史封秦郡王|{
	吳立永豐縣屬始安郡撝吁為翻}
司馬王僧略直兵參軍徐怦固諫不從|{
	怦普耕翻}
僧略僧辯之弟怦勉之從子也|{
	徐勉梁初賢相從才用翻}
初臺城之圍怦勸紀速入援紀意不欲行内銜之會蜀人費合告怦反怦有與將帥書云事事往人口具紀即以為反徵|{
	費扶沸翻將即亮翻帥所類翻徵讀曰證}
謂怦曰以卿舊情當使諸子無恙|{
	恙余亮翻}
對曰生兒悉如殿下留之何益|{
	以譏切紀不能救君父}
紀乃盡誅之梟首於市|{
	梟堅堯翻}
亦殺王僧略永豐侯撝歎曰王事不成矣善人國之基也今先殺之不亡何待紀徵宜豐侯諮議參軍劉璠為中書侍郎|{
	璠孚袁翻}
使者八反乃至紀令劉孝勝深布腹心璠苦求還中記室韋登私謂璠曰殿下忍而蓄憾足下不留將致大禍孰若共構大厦使身名俱美哉|{
	夏戶雅翻}
璠正色曰卿欲緩頰於我邪|{
	緩頰說也漢高帝謂酈食其曰緩頰往說魏王豹}
我與府侯分義已定|{
	府侯謂宜豐侯循分扶問翻}
豈以夷險易其心乎殿下方布大義於天下終不逞志於一夫紀知必不為己用乃厚禮遣之以宜豐侯循為益州刺史封隨郡王以璠為循府長史蜀郡太守 謝答仁討劉神茂還至富陽聞侯景敗走帥萬人欲北出候之|{
	帥讀曰率}
趙伯超據錢塘拒之侯景進至嘉興|{
	沈約曰嘉興本名長水秦改曰由拳吳孫權黄龍四年由拳縣生嘉禾改曰禾興孫皓避父名改曰嘉興今秀州是也北至吳郡一百五十五里}
聞伯超叛之乃退據吳己酉侯瑱追及景於松江|{
	松江在今吳縣一名笠澤在吳縣南四十里}
景猶有船二百艘衆數千人|{
	艘蘇遭翻}
瑱進擊敗之|{
	敗補邁翻}
擒彭雋田遷房世貴蔡夀樂王伯醜|{
	樂音洛}
瑱生剖雋腹抽其腸雋猶不死手自收之乃斬之景與腹心數十人單舸走推墮二子於水|{
	舸古我翻推吐雷翻}
將入海瑱遣副將焦僧度追之|{
	將即亮翻}
景納羊侃之女為小妻以其兄鵾為庫直都督待之甚厚鵾隨景東走與景所親王元禮謝葳蕤密圖之葳蕤答仁之弟也景下海欲向蒙山|{
	景自滬瀆下海魏收志東安郡新泰縣有蒙山景欲浮海趣山東復入北也葳音威蕤如住翻}
己卯景晝寢鵾語海師|{
	海師習知海道者也語牛渚翻}
此中何處有蒙山汝但聽我處分|{
	處昌呂翻分扶問翻}
遂直向京口至胡豆洲景覺大驚問岸上人云郭元建猶在廣陵景大憙|{
	覺古孝翻憙與喜同}
將依之鵾拔刀叱海師向京口 |{
	考異曰典略云舟人李横文紿景向南徐州今從梁書}
因謂景曰吾等為王効力多矣|{
	為于偽翻}
今至於此終無所成欲就乞頭以取富貴景未及答白刃交下景欲投水鵾以刀斫之景走入船中以佩刀抉船底|{
	抉船底欲入水抉一决翻}
鵾以矟刺殺之|{
	矟色角翻刺七亦翻}
尚書右僕射索超世在别船葳蕤以景命召而執之南徐州刺史徐嗣徽斬超世|{
	王偉索超世景之謀主也索蘇各翻}
以鹽内景腹中送其尸於建康僧辯傳首江陵截其手足使謝葳蕤送於齊暴景尸於市士民爭取食之并骨皆盡溧陽公主亦預食焉 |{
	考異曰典略云復烹溧陽公主今從南史}
初景之五子在北齊世宗剝其長子面而烹之幼者皆下蠶室|{
	長知兩翻下遐嫁翻}
齊顯祖即位夢獼猴坐其御牀乃盡烹之趙伯超謝答仁皆降於侯瑱瑱并田遷等送建康王僧辯斬房世貴於市送王偉呂季略周石珍嚴亶趙伯超伏知命於江陵丁巳湘東王下令解嚴乙丑葬簡文帝於莊陵廟號太宗 侯景之敗也以傳國璽自隨|{
	璽斯氏翻}
使其侍中兼平原太守趙思賢掌之曰若我死宜沈於江勿令吳兒復得之思賢自京口濟江遇盜從者棄之草間|{
	沈持林翻復扶又翻從才用翻}
至廣陵以告郭元建元建取之以與辛術壬申術送之至鄴 甲申齊以吏部尚書楊愔為右僕射以太原公主妻之|{
	愔於今翻妻七細翻}
公主即魏孝靜帝之后也 楊乾運至劒北|{
	劔北劒閣之北也}
魏達奚武逆擊之大破乾運於白馬陳其俘馘於南鄭城下且遣人辱宜豐侯循循怒出兵與戰都督楊紹伏兵擊之殺傷殆盡劉璠還至白馬西|{
	劉璠自成都還魏收志華陽郡沔陽縣有白馬城宜豐侯循傳璠見禽於嶓冢嶓冢縣亦屬華陽郡五代志漢中郡西縣舊曰嶓冢}
為武所獲送長安太師泰素聞其名待之如舊交時南鄭久不下武請屠之泰將許之璠請之於朝|{
	朝謂魏朝也朝直遥翻}
泰怒不許璠泣請不已泰曰事人當如是乃從其請 五月庚午司空南平王恪等復勸進|{
	復扶又翻}
湘東王猶不受遣侍中豐城侯泰謁山陵|{
	沈約志豫章豐城縣吳立曰富城晉武帝更名 考異曰梁書在四月官為司空太清紀在此月官太宰今從梁書}
修復廟社戊寅侯景首至江陵梟之於市三日|{
	梟堅堯翻}
煮而漆之以付武庫庚辰以南平王恪為揚州刺史甲申以王僧辯為司徒鎭衛將軍封長寧公|{
	鎭衛將軍梁定二百四十號其班穹矣僧辯元功必封長寧郡公長寧郡晉安帝置屬荆州}
陳霸先為征虜將軍開府儀同三司封長城縣侯|{
	賞平侯景之功也霸先長城縣人也杜佑曰湖州長城縣吳王闔閭遣弟夫槩築城狹而長晉武帝太康三年置}
乙酉誅侯景所署尚書僕射王偉左民尚書呂季略少府周石珍舍人嚴亶於市趙伯超伏知命餓死於獄以謝答仁不失禮於太宗特宥之王偉於獄中上五百言詩|{
	五百言詩今之五十韻詩也上時掌翻}
湘東王愛其才欲宥之有嫉之者言於王曰前日偉作檄文甚佳王求而視之檄云項羽重瞳尚有烏江之敗|{
	重直龍翻}
湘東一目寧為赤縣所歸王大怒釘其舌於柱|{
	釘丁定翻}
剜腹臠肉而殺之|{
	剜烏丸翻王偉侯景之所取計者也自圍臺城以至於移梁祚屠蕭氏以及其臣民皆偉之謀帝忘其父子兄弟之讐乃愛其才而欲宥之發怒於檄文而後誅之失刑甚矣}
丙戌齊合州刺史斛斯昭攻歷陽拔之 丁亥下令以王偉等既死自餘衣冠舊貴被逼偷生|{
	被皮義翻}
猛士勲豪和光苟免者皆不問 扶風民魯悉達糾合鄉人以保新蔡|{
	陣書魯悉達扶風人祖父至悉達皆仕於齊梁沈約宋志江州所部有南新蔡郡不言僑置之地但云去京都水行一千三百七十六里有餘以水程約言之南新蔡郡當置於今蘄州界五代志蘄州黄梅縣舊曰永興隋開皇初改曰新蔡蓋因南新蔡郡以名縣也劉昫曰黄梅縣宋分置新蔡郡}
力田蓄穀時江東飢亂餓死者什八九遺民攜老幼歸之悉達分給糧廩全濟甚衆招集晉熙等五郡盡有其地使其弟廣達將兵從王僧辯討侯景|{
	將即亮翻下同}
景平以悉達為北江州刺史|{
	魏收志梁武帝置北江州治鹿城關領義陽齊昌新昌梁安齊興郡五代志黄州木蘭縣梁置北江州唐併木蘭入黄岡縣}
齊主使其散騎常侍曹文皎等來聘湘東王使散騎常侍柳暉等報之且告平侯景亦遣舍人魏彦告于魏 齊主使潘樂郭元建將兵圍秦郡行臺尚書辛術諫曰朝廷與湘東王信使不絶|{
	使疏吏翻}
陽平侯景之土取之可也今王僧辯已遣嚴超達守秦郡於義何得復爭之|{
	復扶又翻}
且水潦方降不如班師弗從陳霸先命别將徐度引兵助秦郡固守齊衆七萬攻之甚急王僧辯使左衛將軍杜崱救之霸先亦自歐陽來會與元建大戰於士林大破之|{
	士林在六合縣界}
斬首萬餘級生擒千餘人元建收餘衆北遁猶以通好不窮追也|{
	好呼到翻}
辛術遷吏部尚書自魏遷鄴以後大選之職知名者數人|{
	吏部選為大選選須絹翻}
互有得失齊世宗少年高朗所弊者疎袁叔悳沈密謹厚所傷者細|{
	少詩照翻沈持林翻}
楊愔風流辯給取士失於浮華唯術性尚貞明取士必以才器循名責實新舊參舉管庫必擢門閥不遺|{
	閥音伐說文閥閱自序也史記明其等曰閥積其功曰閱此所謂門閥者直言世家子弟門地素高者耳又說門在左曰閥在右曰閱}
考之前後最為折衷|{
	衷陟仲翻}
魏達奚武遣尚書左丞柳帶韋入南鄭說宜豐侯循曰|{
	說式芮翻}
足下所固者險所恃者援所保者民今王旅深入所憑之險不足固也白馬破走酋豪不進|{
	謂楊乾運也酋慈秋翻}
所望之援不可恃也長圍四合所部之民不可保也且足下本朝喪亂社稷無主欲誰為為忠乎|{
	朝直遥翻喪息浪翻誰為之為于偽翻}
豈若轉禍為福使慶流子孫邪循乃請降|{
	降戶江翻}
帶韋慶之子也|{
	柳慶見一百六十一卷武帝太清二年}
開府儀同三司賀蘭德願聞城中食盡請攻之大都督赫連達曰不戰而獲城策之上者豈可利其子女貪其財貨而不愛民命乎且觀其士馬猶彊精神尚固攻之縱克則彼此俱傷如困獸猶鬪|{
	左傳吳夫槩王之言}
則成敗未可知也武曰公言是也乃受循降獲男女二萬口而還|{
	還從宣翻又如字}
於是劔北皆入於魏 六月丁未齊主還鄴乙卯復如晉陽|{
	復扶又翻}
庚寅立安南侯方矩為王太子 齊遣散騎常侍謝季卿來賀平侯景衡州刺史王懷明作亂廣州刺史蕭勃討平之 齊

政煩賦重江北之民不樂屬齊其豪傑數請兵於王僧辯僧辯以與齊通好皆不許|{
	樂音洛數所角翻好呼到翻}
秋七月廣陵僑人朱盛等|{
	僑人本非廣陵人而僑居廣陵者}
濳聚黨數千人謀襲殺齊刺史温仲邕遣使求援於陳霸先|{
	使疏吏翻下使末同}
云已克其外城霸先使告僧辯僧辯曰人之情偽未易可測|{
	易弋䜴翻}
若審克外城亟須應援如其不爾無煩進軍使未報霸先已濟江僧辯乃命武州刺史杜崱等助之會盛等謀泄霸先因進軍圍廣陵 八月魏安康人黄衆寶反攻魏興執太守柳檜進圍東梁州|{
	梁置南梁州於西城郡西魏改曰東梁州西城古魏興郡治所魏收志後魏太延五年别置魏興郡於上洛郡界帶陽亭縣屬洛州洛州後改為商州}
令檜誘說城中|{
	誘音酉說式芮翻}
檜不從而死檜虯之弟也|{
	柳虯柳慶之兄見二百五十一卷武帝大同三年}
太師泰遣王雄與驃騎大將軍武川宇文虯討之|{
	驃匹妙翻騎奇寄翻}
武陵王紀舉兵由外水東下以永豐侯撝為益州刺史守成都使其子宜都王圓肅副之 九月甲戌司空南平王恪卒甲申以王僧辯為揚州刺史 齊主使吿王僧辯陳霸先曰請釋廣陵之圍必歸廣陵歷陽兩城霸先引兵還京口江北之民從霸先濟江者萬餘口湘東王以霸先為征北大將軍開府儀同三司南徐州刺史徵霸先世子昌|{
	覇先封長城縣侯昌為世子}
及兄子頊詣江陵以昌為散騎常侍頊為領直|{
	梁宿衛之官有四廂領直蓋領直衛之士因以名官為昌頊陷魏張本}
宜豐侯循之降魏也丞相泰許其南還久而未遣從容問劉璠曰|{
	從千容翻}
我於古誰比對曰璠常以公為湯武今日所見曾桓文之不如泰曰我安敢比湯武庶幾望伊周何至不如桓文對曰齊桓存三亡國|{
	左傳宋司馬子魚曰齊桓存三亡國以屬諸侯杜預注曰三亡國魯衛邢}
晉文公不失信於伐原語未竟泰撫掌曰我解爾意|{
	解戶買翻}
欲激我耳乃謂循曰王欲之荆為之益|{
	武陵王紀封循為隨郡王故以王稱之}
循請還江陵泰厚禮遣之 |{
	考異曰典略云十月乙未朔太祖謂循云云按太清紀是月循至江陵今從之}
循以文武千家自隨湘東王疑之遣使覘察相望於道|{
	使疏吏翻覘丑廉翻又丑艶翻}
始至之夕命劫竊其財及旦循啓輸馬仗王乃安之引入對泣以循為侍中驃騎將軍開府儀同三司 冬十月齊主自晉陽如離石自黄櫨嶺起長城北至社平戍四百餘里置三十六戍|{
	此長城蓋起於唐石州北抵武州之境櫨音盧社平齊紀作社子按斛律金傳黄櫨嶺在烏突戍東}
戊申湘東王執湘州刺史王琳於殿中殺其副將殷晏琳本會稽兵家|{
	將即亮翻會工外翻}
其姊妹皆入王宫故琳少在王左右琳好勇王以為將帥琳傾身下士|{
	少詩沼翻好呼到翻帥所類翻下遐嫁翻下王下同}
所得賞賜不以入家麾下萬人多江淮羣盗從王僧辯平侯景與杜龕功居第一|{
	龕苦含翻}
在建康恃寵縱暴僧辯不能禁僧辯以宫殿之燒|{
	謂軍人遺火焚太極殿及東西堂也}
恐得罪欲以琳塞責|{
	塞悉則翻}
乃密啓王請誅琳王以琳為湘州琳自疑及禍使長史陸納帥部曲赴湘州|{
	帥讀曰率}
身詣江陵陳謝謂納等曰吾若不返子將安之|{
	之往也}
咸曰請死之相泣而别至江陵王下琳吏辛酉以王子方略為湘州刺史又以廷尉黄羅漢為長史使與太舟卿張載至巴陵先據琳軍|{
	五代志太舟卿梁初為都水臺使者天監七年改焉位視中書郎列卿之最末者也主舟航堤渠}
載有寵於王而御下峻刻荆州人疾之如讐羅漢等至琳軍陸納及士卒並哭不肯受命執羅漢及載王遣宦者陳旻往諭之納對旻刳載腹抽腸以繫馬足使繞而走腸盡氣絶又臠割出其心向之抃舞|{
	抃舞拊手而舞}
焚其餘骨以黄羅漢清謹而免之納與諸將引兵襲湘州時州中無主納遂據之 公卿藩鎭數勸進於湘東王|{
	數所角翻}
十一月丙子世祖即皇帝位於江陵改元大赦|{
	改太清為承聖}
是日帝不升正殿公卿陪列而已|{
	史言帝不能正其始}
丁丑以宜豐侯循為湘州刺史己卯立王太子方矩為皇太子更名元良|{
	更工行翻}
皇子方智為晉安王方略為始安王方等之子莊為永嘉王|{
	方等伐湘州戰敗而死}
追尊母阮修容為文宣皇后|{
	修容魏文帝所制自晉以來位列九嬪}
侯景之亂州郡太半入魏自巴陵以下至建康以長江為限荆州界北盡武寧西拒硤口|{
	北盡武寧與岳陽王詧分界西拒硤口與武陵王紀分界硤當作峽}
嶺南復為蕭勃所據|{
	復扶又翻}
詔令所行千里而近民戶著籍不盈三萬而已|{
	著直略翻史言荆州管内蕭條}
陸納襲擊衡州刺史丁道貴於淥口破之|{
	衡州治衡陽縣縣東二十里有酃湖其水湛然緑色取以釀酒甘美謂之酃淥淥口即酃湖口也唐志潭州有淥口戌丁度曰湘東有淥水張舜民郴行錄嘉魚縣口舟行七十餘里至淥溪口南北對境圖自岳州沿江東北下過侯敬港神林港象湖港新打口石頭口得淥溪口按郴行録對境圖之淥溪口皆非丁道貴敗處唯唐志之潭州淥口戍為是淥音綠}
道貴奔零陵其衆悉降於納|{
	降戶江翻}
上聞之遣使徵司徒王僧辯右衛將軍杜崱平北將軍裴之横與宜豐侯循共討納循軍巴陵以待之侯景之亂零陵人李洪雅據其郡上即以為營州刺史|{
	營陽郡亦漢零陵郡之地故因置營州隋為永州}
洪雅請討陸納上許之丁道貴收餘衆與之俱納遣其將吳藏襲擊破之洪雅等退保空雲城|{
	姚思廉梁書作空靈灘水經注長沙建寧縣故城南有空泠峽湘水所經也驚浪奔雷迅同三峽張舜民郴行錄曰自醴陵江口南行十餘里到空靈㟁 考異曰典略作空零城今從梁書 余謂空雲蓋空靈之誤也}
藏引兵圍之頃之納請降求送妻子|{
	送妻子為質以示誠欵}
上遣陳旻至納所納衆皆泣曰王郎被囚|{
	被皮義翻}
故我曹逃罪於湘州非有他志也乃出妻子付旻旻至巴陵循曰此詐也必將襲我乃密為之備納果夜以輕兵繼旻後約至城下鼓譟十二月壬午晨去巴陵十里|{
	言納兵夜行至晨距巴陵相去十里}
衆謂已至即鼓譟軍中皆驚循坐胡牀於壘門望之納乘水來攻矢下如雨循方食甘蔗略無懼色|{
	蔗之夜翻}
徐部分將士擊之|{
	分扶問翻}
獲其一艦|{
	艦戶黯翻}
納退保長沙 壬午齊主還鄴戊午復如晉陽|{
	復扶又翻}


資治通鑑卷一百六十四
