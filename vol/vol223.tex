\section{資治通鑑卷二百二十三}
宋 司馬光 撰

胡三省 音註

唐紀三十九|{
	始癸卯七月終乙巳十月凡二年零四月起昭陽單閼七月盡旃蒙大荒落十月凡二年有奇}


代宗睿文孝武皇帝上之下

廣德元年秋七月壬寅羣臣上尊號曰寶應元聖文武孝皇帝|{
	以楚州所獻十三寶為上登極之符應也上時掌翻}
壬子赦天下改元|{
	方改元廣德}
諸將討史朝義者進官階加爵邑有差|{
	將即亮翻}
冊回紇可汗為頡咄登密施合俱錄英義建功毗伽可汗可敦為娑墨光親麗華毗伽可敦|{
	頡咄華言社稷用登密施華言到竟合俱錄華言婁羅娑墨華言得憐毗伽華言足意智紇下沒翻可從刋入聲汗音寒頡奚結翻咄當沒翻伽求迦翻娑蘇何翻}
左右殺以下皆加封賞|{
	左殺封雄朔王右殺封寜朔王胡禄都督封金河王拔覧將軍封静漢王諸都督十一人並封國公}
戊辰楊綰上貢舉條目秀才問經義二十條對策五道國子監舉人令博士薦於祭酒祭酒試通者升之於省如鄉貢法|{
	令力丁翻唐取士之科由學館曰生徒由州縣者曰鄉貢凡明經秀才俊士進士明於理體為鄉里稱者縣考試州長重覆送之尚書省既至省皆疏名列到結欵通保及所居始由戶部集閲而關於禮部試之今楊綰所上國子監舉人略如鄉貢法}
明法委刑部考試|{
	明法律學也}
或以為明經進士行之已久不可遽改事雖不行識者是之 以僕固瑒為朔方行營節度使 吐蕃入大震關陷蘭廓河鄯洮岷秦成渭等州盡取河西隴右之地|{
	蘭廓秦渭等州即河西隴右之地先已為吐蕃所陷史因其入大震關而備言之蘭州漢金城郡隋置蘭州因臯蘭山為名廓州漢西平郡南界前凉以其地為湟河郡後魏置洮河郡周建德五年取河南地置廓州取廓清之義為名河州漢枹罕縣前凉張駿分置河州鄯州漢破羌允吾縣地唐平薛舉置鄯州洮州治漢洮陽城周保定初置岷州秦臨洮縣地後魏大統十年置岷州以南有岷山名秦州治成紀顯親川因魏晋舊州名成州古西戎地後千畝戎姜氏居之又後為白馬氏國漢為武都郡晉為仇池郡後魏改為南秦州西魏改成州渭州治漢襄武縣後魏置}
唐自武德以來開拓邊境地連西域皆置都督府州縣開元中置朔方隴右河西安西北庭諸節度使以統之歲發山東丁壯為戍卒繒帛為軍資開屯田供糗糧|{
	繒慈陵翻糗去久翻}
設監牧畜馬牛軍城戍邏萬里相望|{
	畜吁玉翻邏郎佐翻}
及安祿山反邉兵精銳者皆徵發入援謂之行營所留兵單弱胡虜稍蠶食之數年間西北數十州相繼淪沒自鳳翔以西邠州以北皆為左袵矣|{
	史言唐所以失河隴}
初僕固懷恩受詔與回紇可汗相見於太原河東節度使辛雲京以可汗乃懷恩壻恐其合謀襲軍府閉城自守亦不犒師|{
	使疏吏翻犒苦到翻}
及史朝義既平詔懷恩送可汗出塞往來過太原|{
	朝直遥翻過古禾翻又古臥翻}
雲京亦閉城不與相聞懷恩怒具表其狀不報懷恩將朔方兵數萬屯汾州使其子御史大夫瑒將萬人屯榆次禆將李光逸等屯祁縣|{
	將即亮翻下禆將同又音如字禆彼迷翻榆次祈皆漢古縣属太原}
李懷光等屯晉州張維嶽等屯沁州|{
	沁七浸翻 考異曰邠志作張如岳今從實錄唐歷}
懷光本勃海靺鞨也|{
	靺鞨音末曷}
姓茹|{
	茹音如}
為朔方將以功賜姓中使駱奉仙至太原雲京厚結之為言懷恩與回紇連謀反狀已露|{
	使疏吏翻為于偽翻}
奉仙還過懷恩懷恩與飲於母前母數讓奉仙曰|{
	還音旋又音如字數所角翻}
汝與吾兒約為兄弟今又親雲京何兩面也|{
	唐人謂反覆者為兩面貞元以後劒南西山白狗等羌内附賜牛糧治生業差賜官祿皆得世襲然隂附吐蕃世謂之兩面羌此其證也}
酒酣懷恩起舞奉仙贈以纒頭綵|{
	唐人宴集酒酣為人舞當此禮者以綵物為贈謂之纒頭倡伎當筵舞者亦有纒頭喝賜杜甫詩所謂舞罷錦纒頭者也酣戶甘翻}
懷恩欲酬之曰來日端午當更樂飲一日|{
	樂音洛}
奉仙固請行懷恩匿其馬奉仙謂左右曰朝來責我又匿我馬將殺我也夜踰垣而走懷恩驚遽以其馬追還之八月癸未奉仙至長安奏懷恩謀反 |{
	考異曰實錄癸未懷恩旋師次于汾州逗遛不進監軍使駱奉仙以聞上以功高不之罪優詔慰勞之又曰懷恩頓軍汾上監軍使駱奉仙因公宴言有所指懷恩已萌二心肆口酬對奉仙不告而出乘傳上聞上以功高容之叱奉仙出待懷恩如舊懷恩惮奉仙益不自安邠志曰寶應二年河朔既平詔太原節度辛雲京及僕固懷恩各以其軍送回紇還蕃既出晋關辛公率其輕兵先入太原懷恩怒其不告曰辛君有虞於我也回紇至辛公館于城外致牛酒以犒之懷恩欲因回紇規其城壁隂導回紇請觀佛寺辛公許之既入城見羅兵於諸街蕃人大驚辟易而去今從舊懷恩傳}
懷恩亦具奏其狀請誅雲京奉仙上兩無所問優詔和解之懷恩自以兵興以來|{
	謂自祿山反朔方起兵討之以至平賊時也}
所在力戰一門死王事者四十六人女嫁絶域|{
	謂嫁回紇可汗也}
說諭回紇|{
	說式芮翻紇下沒翻}
再收兩京平定河南北功無與比而為人搆陷憤怨殊深上書自訟以為臣昨奉詔送可汗歸國傾竭家貲俾之上道行至山北|{
	上時兩翻可從刋入聲汗音寒懷恩屯汾州謂太原之地為山北}
雲京奉仙閉城不出祗迎仍令潜行竊盗回紇怨怒亟欲縱兵臣力為彌縫方得出塞雲京奉仙恐臣先有奏論遂復妄稱設備|{
	令力丁翻為于偽翻復扶又翻下並同}
與李抱玉共相組織臣静而思之其罪有六昔同羅叛亂臣為先帝掃清河曲一也臣男玢為同羅所虜得間亡歸臣斬之以令衆士二也|{
	二事並見二百十八卷肅宗至德元載玢悲巾翻間古莧翻}
臣有二女遠嫁外夷為國和親蕩平寇敵三也臣與男瑒不顧死亡為國効命四也河北新附節度使|{
	謂田承嗣李寶臣李懷仙等}
皆握強兵臣撫綏以安反側五也臣說諭回紇使赴急難天下既平送之歸國六也|{
	說式芮翻難乃旦翻}
臣既負六罪誠合萬誅惟當吞恨九泉銜寃千古復何訴哉臣受恩深重夙夜思奉天顔|{
	言欲入朝也}
但以來瑱受誅|{
	瑱它甸翻事見上卷正月}
朝廷不示其罪諸道節度誰不疑懼近聞詔追數人|{
	唐人率謂召為追觀考異所引諸家雜史可見}
盡皆不至實畏中官讒口虚受陛下誅夷豈惟羣臣不忠正為回邪在側|{
	當時君臣情事誠如懷恩之言}
且臣前後所奏駱奉仙詞情非不摭實|{
	摭之實翻}
陛下竟無處置|{
	處昌呂翻}
寵任彌深皆由同類比周|{
	比毗至翻}
蒙蔽聖聽竊聞四方遣人奏事陛下皆云與驃騎議之|{
	驃騎謂程元振也驃匹妙翻騎奇計翻}
曾不委宰相可否或稽留數月不還遠近益加疑阻|{
	代宗省懷恩書至此豈不為之動心邪曾才登翻}
如臣朔方將士功効最高為先帝中興主人乃陛下蒙塵故吏曾不别加優奬反信讒嫉之詞子儀先已被猜臣今又遭詆毁|{
	將即亮翻被皮義翻}
弓藏鳥盡信匪虚言|{
	古語云高鳥盡良弓藏狡兎死走狗烹敵國破謀臣亡言以今事凖之非虚言也}
陛下信其矯誣|{
	矯託也託言謀反以厚誣}
何殊指鹿為馬|{
	引趙高事以况諸閹}
儻不納愚懇且貴因循臣實不敢保家陛下豈能安國|{
	懷恩心跡於此可見}
忠言利行|{
	忠言逆耳利於行古語也}
惟陛下圖之臣欲公然入朝|{
	朝直遥翻下同}
恐將士留沮今託廵晉絳於彼遷延乞陛下特遣一介至絳州問臣臣即與之同發九月壬戍上遣裴遵慶詣懷恩諭旨且察其去就懷恩見遵慶抱其足號泣訴寃|{
	號戶刀翻}
遵慶為言聖恩優厚諷令入朝懷恩許諾副將范志誠以為不可曰公信其甘言入則為來瑱不復還矣明日懷恩見遵慶以懼死為辭請令一子入朝志誠又以為不可遵慶乃還|{
	還音旋又如字下同}
御史大夫王翊使回紇還懷恩先與可汗往來恐翊洩其事遂留之 吐蕃之入寇也邉將告急程元振皆不以聞冬十月吐蕃寇涇州刺史高暉以城降之遂為之鄉導|{
	降戶江翻鄉讀曰嚮 考異曰汾陽家傳八月吐蕃次涇寧州遣感激軍使高暉禦之戰敗執暉九月至便橋實録十月庚午吐蕃寇涇州辛未犯奉天武功按今邠州東去涇州三程邠州南去奉天二程不應庚午寇邠州辛未已至奉天盖史官據奏到日書之耳段公家傳九月二十日吐蕃寇涇原節度使高暉降之十一月一日陷邠州節度使張藴琦棄城遁舊本紀九月己丑吐蕃寇涇州刺史高暉以城降因為吐蕃鄉導十月辛未犯京畿新本紀九月乙丑涇州刺史高暉叛附於吐蕃十月庚午吐蕃陷邠州辛未寇奉天武功今月從實録而不取其日}
引吐蕃深入過邠州上始聞之辛未寇奉天武功京師震駭詔以雍王适為關内元帥郭子儀為副元帥出鎮咸陽以禦之子儀閒廢日久|{
	寶應元年八月郭子儀自河東入朝遂留京師}
部曲離散至是召募得二十騎而行至咸陽吐蕃帥吐谷渾党項氐羌二十餘萬衆彌漫數十里已自司竹園度渭|{
	鳳翔府盩厔縣有司竹園漢書王莽傳所謂霍鴻負倚芒竹即此地也蘇軾曰盩厔有官竹園臨水數十里不絶所謂司竹也}
循山而東子儀使判官中書舍人王延昌入奏請益兵程元振遏之竟不召見|{
	見賢遍翻}
癸酉渭北行營兵馬使呂月將將精卒二千破吐蕃於盩厔之西|{
	將將上如字下即亮翻}
乙亥吐蕃寇盩厔月將復與力戰兵盡為虜所擒|{
	復扶又翻}
上方治兵|{
	治直之翻}
而吐蕃已度便橋倉猝不知所為丙子出幸陜州|{
	陜失冉翻}
官吏藏竄六軍逃散郭子儀聞之遽自咸陽歸長安比至|{
	比必利翻及也}
車駕已去上纔出苑門度滻水|{
	按唐禁苑包大明宫之北東距滻水考雍錄長安志諸書禁苑東面出滻水無其門盖出光泰門耳}
射生將王獻忠擁四百騎叛還長安|{
	將即亮翻騎奇計翻}
脅豐王珙等十王西迎吐蕃|{
	珙玄宗子音居勇翻}
遇子儀於開遠門内|{
	開遠門長安城西面兆頭第一門}
子儀叱之獻忠下馬謂子儀曰今主上東遷社稷無主令公身為元帥|{
	郭子儀為中書令故稱爲令公}
廢立在一言耳子儀未應珙越次言曰公何不言子儀責讓之以兵援送行在丁丑車駕至華州官吏奔散無復供擬|{
	復扶又翻}
扈從將士不免凍餒會觀軍容使魚朝恩將神策軍自陜來迎上乃幸朝恩營|{
	從才用翻將即亮翻使疏吏翻朝直遥翻恩將音同上陜失冉翻}
豐王珙見上於潼關上不之責退至幕中有不遜語羣臣奏請誅之乃賜死戊寅吐蕃入長安高暉與吐蕃大將馬重英等立故邠王守禮之孫承宏為帝|{
	珙居勇翻吐從暾入聲重直龍翻邠直頻翻邠王守禮章懷太子之子}
改元置百官以前翰林學士于可封等為相吐蕃剽掠府庫市里焚閭舍長安中蕭然一空|{
	相息亮翻剽匹妙翻下同}
苖晉卿病臥家遣人輿入廹脅之晉卿閉口不言虜不敢殺於是六軍散者所在剽掠士民避亂皆入山谷辛巳上至陜百官稍有至者郭子儀引三十騎自御宿川循山而東|{
	騎奇計翻三輔黄圖曰御宿川在長安城南漢武帝為離宫别館禁禦人不得往來游觀上宿其中故曰御宿程大昌曰御宿川即樊川在萬年縣南二十五里}
謂王延昌曰六軍將士逃潰者多在商州今速往收之|{
	商州治上洛縣至京師二百八十一里}
并發武關防兵數日間北出藍田以向長安吐蕃必遁過藍田遇元帥都虞侯臧希讓鳳翔節度使高昇得兵近千人|{
	武關在商州西北雍州藍田縣南吐從暾入聲過古禾翻又古臥翻使疏吏翻近其靳翻}
子儀與延昌謀曰潰兵至商州官吏必逃匿而人亂使延昌自直徑入商州撫諭之諸將方縱兵暴掠聞子儀至皆大喜聽命|{
	將即亮翻}
子儀恐吐蕃逼乘輿留軍七盤|{
	杜佑曰七盤即王莽所謂繞霤之固南當荆楚者也繞霤者言四面塞阨屈曲水回繞而霤今謂之七盤十二䋫乘承正翻吐從暾入聲}
三日乃行比至商州|{
	比必利翻}
行收兵并武關防兵合四千人軍勢稍振子儀乃泣諭將士以共雪國恥取長安皆感激受約束子儀請太子賓客第五琦為糧料使給軍食|{
	將即亮翻糧料使主給行營軍食我宋朝隨軍轉運使即其任}
上賜子儀詔恐吐蕃東出潼關徵子儀詣行在子儀表稱臣不收京城無以見陛下若出兵藍田虜必不敢東向上許之鄜延節度判官段秀實說節度使白孝德引兵赴難|{
	上元元年置渭北鄜坊節度使領鄜坊丹延四州治坊州吐從暾入聲潼音同說式芮翻難乃旦翻}
孝德即日大舉南趣京畿|{
	李嗣業自西域赴難於至德之初白孝德自鄜坊赴廣德之難皆段秀實之其忠義盖天性也}
與蒲陜商華合勢進擊|{
	陜失冉翻華戶化翻}
吐蕃既立廣武王承宏欲掠城中士女百工整衆歸國子儀使左羽林大將軍長孫全緒將二百騎出藍田觀虜勢令第五琦攝京兆尹與之偕行又令寶應軍使張知節將兵繼之|{
	長知兩翻將即亮翻又音如字騎奇計翻令力丁翻琦音奇上以射生軍入禁中清難賜名寶應功臣故射生軍亦號寶應軍}
全緒至韓公堆晝則擊鼔張旗幟夜則多然火以疑吐蕃前光禄卿殷仲卿聚衆近千人保藍田與全緒相表裏帥二百餘騎直度滻水|{
	近其靳翻帥讀曰率滻音產}
吐蕃懼百姓又紿之曰郭令公自商州將大軍不知其數至矣|{
	紿湯亥翻將即亮翻又音如字}
虜以為然稍稍引軍去全緒又使射生將王甫入城隂結少年數百夜擊鼔大呼於朱雀街|{
	將即亮翻少始照翻呼火故翻唐太極宫正南出朱雀門自朱雀門南出至明德門皆名朱雀街}
吐蕃惶駭庚寅悉衆遁去 |{
	考異曰舊吐蕃傳曰子儀帥部曲數百人及其妻子僕從南入牛心谷駝馬車牛數百兩子儀遲留未知所適行軍判官中書舍人王延昌監察御史李萼謂子儀曰令公身為元帥主上蒙塵於外今吐蕃之勢日逼豈可懷安于谷中何不南趨商州漸赴行在子儀遽從之延昌曰吐蕃知令公南行必分兵來逼若當大路事即危矣不如取王山路而去出其不意子儀又從之子儀之隊千餘人山谷束隘連延百餘里人不得馳延昌與萼恐狹徑被追前後不相救至倒迴口遂與子儀别行踰絶澗登七盤趨于商州先是六軍將張知節與麾下數百人自京城奔於商州大掠避難朝官士庶及居人資財己有日矣延昌與萼既至說知節曰將軍身掌禁兵軍敗而不赴行在又恣其下虜掠何所歸乎今郭令公元帥也已欲至洛南將軍若整頓士卒諭以禍福請令公來撫之圖收長安此則將軍非常之功也知節大悦其時諸軍臧希讓高昇彭體盈李惟詵等數人各有部曲家兵數十騎相次而至又從其計皆相率為軍約不侵暴延昌留于軍中主約萼以數騎往迎子儀去洛南十餘里及之遂與子儀迴至商州諸將大喜皆遵其約束吐蕃將入京師也前光祿卿殷仲卿逃難而出至藍田糾合敗兵及諸驍勇願從者百餘人南保藍田以拒吐蕃其衆漸振至於千人子儀既至商州募人往探賊勢羽林將軍長孫全緒請行全緒至韓公堆仲卿得官軍其勢益壯遂相為表裏仲卿帥二百餘騎遊奕直度滻水吐蕃懼問百姓百姓皆紿之曰郭令公大軍不知其數賊以為然遂抽軍而還汾陽家傳曰公以三十騎循御宿川略山而東公西望國門涕不自勝謂延昌曰為舍人計何以復國延昌歔欷不能對公謂曰料諸將散卒必逃商於若速行伩合散卒兼武關兵數日之内却出藍田設疑兵為斾屯於韓公堆吐蕃必懼我而退乃相與速驅之過藍田公與延昌議曰散兵至商州必官吏不守則兵亂而人潰使延昌間道中宿至商州果如所議延昌以公之言廵撫之亂乃止潰乃復今從之}
高暉聞之帥麾下三百餘騎東走至潼關守將李日越擒而殺之|{
	帥讀曰率將即亮翻下同 考異曰新魚朝恩傳曰朝恩遣劉德信討斬之今從實錄}
壬辰詔以元載判元帥行軍司馬以第五琦為京兆尹癸巳以郭子儀為西京留守甲午子儀發商州己亥以魚朝恩部將皇甫溫為陜州刺史周智光為華州刺史|{
	為周智光以華州跋扈張本}
驃騎大將軍判元帥行軍司馬程元振專權自恣人畏之甚於李輔國諸將有大功者元振皆忌嫉欲害之吐蕃入寇元振不以時奏致上狼狽出幸上發詔徵諸道兵李光弼等皆忌元振居中莫有至者中外咸切齒而莫敢發言太常博士柳伉上疏|{
	伉苦浪翻}
以為犬戎犯關度隴不血刃而入京師劫宫闈焚陵寢武士無一人力戰者此將帥叛陛下也陛下疏元功委近習|{
	疏與疎同}
日引月長以成大禍羣臣在廷無一人犯顔回慮者此公卿叛陛下也陛下始出都百姓填然奪府庫相殺戮此三輔叛陛下也自十月朔召諸道兵盡四十日無隻輪入關此四方叛陛下也内外離叛陛下以今日之勢為安邪危邪若以為危豈得高枕|{
	枕職任翻}
不為天下討罪人乎|{
	為于偽翻}
臣聞良醫療疾當病飲藥藥不當病猶無益也陛下視今日之病何繇至此乎必欲存宗廟社稷獨斬元振首馳告天下悉出内使隸諸州|{
	言悉出諸宦官隸諸州羈管也時宦官皆為内諸司使故曰内使}
持神策兵付大臣|{
	時魚朝恩領神策軍}
然後削尊號下詔引咎曰天下其許朕自新改過宜即募士西赴朝廷若以朕惡不悛|{
	悛丑緣翻}
則帝王大器敢妨聖賢其聽天下所往如此而兵不至人不感天下不服臣請闔門寸斬以謝陛下上以元振嘗有保護功|{
	保護事見上卷寶應元年}
十一月辛丑削元振官爵放歸田里王甫自稱京兆尹聚衆二千餘人署置官屬暴横長

安中|{
	横戶孟翻}
壬寅郭子儀至滻水西|{
	郭子儀至滻水西則已度滻水近京城矣}
甫按兵不出或謂子儀城不可入子儀不聽引三十騎徐進使人傳呼召甫甫失據出迎拜伏子儀斬之 |{
	考異曰實録云有武將王甫等誘長安惡少數百人集六街鼔於朱雀街大鼔之吐蕃聞之震懾乘夜而遁汾陽家傳曰射生將王撫猛而多力自稱御史大夫領五百騎二千步卒兼補官屬以謀作亂甲午公商州冬十一月壬寅公次滻水之右王撫知公之來也於城中堅列行陣戈矛若林指揮其間按甲不出人勸公必不可入公以三十騎徐進曾不少懼令傳呼王撫撫應聲伏烏合之徒一時而潰邠志曰郭公屯商州十二月一日率諸軍五萬餘人出藍田去城百里而軍城中相傳言大軍將至西戎懼焉三日馬家小兒張小君李酒盞射生官王甫等五百餘人夜半聚六街鼔入於子城擂擊天門街中仍分其衆建旗諸門吐蕃以為大軍夜至相帥遁去小君使報郭公七日郭公全師入于京師繫小君酒盞王甫等責之曰吾大軍未至汝設詐以畏吐蕃吐蕃知之怒汝燔燒宫闕從容而去豈不由汝乎命斬之遂以破賊收城聞舊子儀傳曰全緒遣禁軍舊將王甫入長安隂結豪俠為内應一日齊擊鼔於朱雀街蕃軍惶駭而去又曰射生將王甫自署為京兆尹聚兵二千人擾亂京城子儀召撫殺之詔子儀權京城留守吐蕃傳吐蕃餘衆尚在城軍將王撫及御史大夫王仲昇領兵自苑中入椎鼔大呼仲卿之兵又入城吐蕃皆奔走若如邠志所言是子儀殺撫而攘其功計子儀必不為也子儀勲業今古推高凌準作書多攻其短疑有宿嫌不可盡信今從汾陽家傳及子儀舊傳}
其兵盡散白孝德與邠寜節度使張藴琦將兵屯畿縣|{
	京兆府管二十縣萬年長安為赤縣餘縣皆為畿縣}
子儀召之入城京畿遂安 宦官廣州市舶使呂太一發兵作亂|{
	唐置市舶使於廣州以收商舶之利時以宦者為之舶音白}
節度使張休棄城奔端州|{
	舊志廣州西至端州二百四十里}
太一縱兵焚掠官軍討平之 吐蕃還至鳳翔節度使孫志直閉城拒守吐蕃圍之數日鎮西節度使馬璘聞車駕幸陜將精騎千餘自河西入赴難|{
	難乃旦翻}
轉鬬至鳳翔值吐蕃圍城璘帥衆持滿外向突入城中不解甲背城出戰|{
	帥讀曰率下同背蒲妹翻}
單騎先士卒奮擊俘斬千計而歸明日虜復逼城請戰|{
	先昔薦翻俘方無翻復扶又翻}
璘開懸門以待之|{
	杜預曰懸門施於内城門按今邉城之門設扉以啟閉而懸門者設於門闑之外常懸而不下寇至則下之以塞門以為重閉之固}
虜引退曰此將軍不惜死宜避之遂去居於原會成渭之地|{
	原州高平郡會州會寜郡成州同谷郡皆據河隴之勝以臨唐境}
十二月丁亥車駕發陜州|{
	陜失冉翻}
左丞顔真卿請上先謁陵廟然後還宫元載不從真卿怒曰朝廷豈堪相公再壞邪|{
	還從宣翻又音如字載祖亥翻又音如字朝直遥翻相昔亮翻邪音耶懷音怪}
載由是銜之甲午上至長安郭子儀帥城中百官及諸軍迎於滻水東伏地待罪上勞之曰用卿不早故及於此|{
	銜其銜翻勞力到翻滻音產}
以魚朝恩為天下觀軍容宣慰處置使總禁兵權寵

無比|{
	魚朝恩以陜州迎扈之勞過承權寄恩寵去程得魚所謂去虺得虎也處昌呂翻使疏吏翻朝直遥翻}
築城於鄠縣及中渭橋屯兵以備吐蕃以駱奉仙為鄠縣築城使遂將其兵|{
	鄠音戶吐從暾入聲將即亮翻又音如字}
乙未以苖晉卿為太保裴遵慶為太子少傅並罷政事以宗正卿李峴為黄門侍郎同平章事遵慶既去元載權益盛以貨結内侍董秀使主書卓英倩潜與往來上意所屬|{
	主書省吏也峴戶典翻屬之欲翻}
載必先知之承意探微|{
	探吐南翻}
言無不合上以是益愛之英倩金州人也|{
	金州安康郡}
吐蕃既去廣武王承宏逃匿草野上赦不誅丙申放之於華州|{
	華州華隂郡去京師一百八十里}
程元振既得罪歸三原聞上還宫衣婦人服|{
	還從宣翻又音如字衣於既翻}
私入長安復規任用京兆府擒之以聞|{
	復扶又翻規圖也 考異曰實錄如此仍云將圖進取舊傳元振服縗麻于車中入京城以規任用與御史大夫王仲昇飲酒為御史所彈今從實錄参以舊傳}
吐蕃陷松維保三州及雲山新築二城|{
	松州交川郡以郡界有甘松嶺名州開元二十八年以維州之定亷置奉州雲山郡天寶八年徙治天保軍更曰天保郡是年沒吐蕃至乾元元年嗣歸誠王董嘉俊以郡來歸始更名保州又按天寶八年分定亷置雲山縣時盖於縣新築二城也}
西川節度使高適不能救於是劒南西山諸州亦入於吐蕃矣|{
	使疏吏翻吐從暾入聲}
二年春正月壬寅勑稱程元振變服潜行將圖不軌長流溱州上念元振之功|{
	復念其保護之功也}
尋復令於江陵安置|{
	復扶又翻溱緇詵翻溱州治榮懿縣貞觀開山洞所置也江陵府上元置府又置南都則善地也令力丁翻}
癸卯合劒南東西川為一道以黃門侍郎嚴武為節

度使|{
	分劒南為東西川見二百二十卷肅宗至德元載考異曰舊傳武為京兆少尹以史思明阻兵不之官出為綿州刺史遷東川節度使上皇誥兩川合為一道拜武劒南節度使新傳武為少尹坐房琯貶巴州久之遷東川餘同舊傳按思明阻兵河洛京兆少尹何妨之官此年始合東西川為一道豈上皇誥所合新舊傳皆誤}
丙午遣檢校刑部尚書顔真卿宣慰朔方行營上之在陜也顔真卿請奉詔召僕固懷恩上不許|{
	校古孝翻尚辰羊翻陜失冉翻}
至是上命真卿說諭懷恩入朝對曰陛下在陜臣往以忠義責之使之赴難|{
	說式芮翻朝直遥翻難乃旦翻}
彼猶有可來之理今陛下還宫|{
	還從宣翻又音如字}
彼進不成勤王退不能釋衆召之庸肯至乎且言懷恩反者獨辛雲京駱奉仙李抱玉魚朝恩四人耳|{
	朝直遥翻}
自餘羣臣皆言其枉陛下不若以郭子儀代懷恩可不戰而服也時汾州别駕李抱真抱玉之從父弟也|{
	從才用翻}
知懷恩有異志脫身歸京師上方以懷恩為憂召見抱真問計對曰此不足憂也朔方將士思郭子儀如子弟之思父兄懷恩欺其衆云郭子儀已為魚朝恩所殺衆信之故為其用耳|{
	將即亮翻朝直遥翻}
陛下誠以子儀領朔方彼皆不召而來耳上然之甲寅禮儀使杜鴻漸|{
	唐會要曰高祖禪代之際温大雅與竇威陳叔達参定禮儀自}


|{
	後至開元初参定禮儀者並不入銜無由檢校開元九年韋縚除國子司業仍知太常禮儀事自此至二十三年凡四改官至太常卿並帶知禮儀事至天寶九載始置禮儀使以太子左庶子韋述為之使疏史翻}
奏自今祀圓丘方丘請以太祖配祈穀以高祖配太雩以太宗配明堂以肅宗配從之|{
	唐制冬至祀圓丘夏至祀方丘孟春祈穀孟夏雩祀李秋大享明堂以肅宗配嚴父之義也}
乙卯立雍王适為皇太子|{
	雍於用翻适古活翻}
吐蕃之入長安也諸軍亡卒及鄉曲無賴子弟相聚為盗吐蕃既去猶竄伏南山子午等五谷|{
	吐從暾入聲長安之南山西接岐州界東抵虢州界其谷之大者有五子午谷斜谷駱谷藍田谷衡嶺谷也}
所在為患丁巳以太子賓客薛景仙為南山五谷防禦使以討之 魏博節度使田承嗣奏名所管曰天雄軍從之|{
	使疏吏翻嗣祥吏翻}
僕固懷恩既不為朝廷所用遂與河東都將李竭誠潜謀取太原|{
	朝直遥翻將即亮翻都將都知兵馬使}
辛雲京覺之殺竭誠乘城設備懷恩使其子瑒將兵攻之雲京出與戰瑒大敗而還遂引兵圍榆次|{
	去年七月懷恩使瑒屯榆次瑒音暢又雉杏翻還從宣翻又音如字}
上謂郭子儀曰懷恩父子負朕實深|{
	代宗心事惟與子儀言之}
聞朔方將士思公如枯旱之望雨公為朕鎮撫河東|{
	為于偽翻}
汾上之師必不為變|{
	汾上謂汾州時朔方軍多在焉}
戊午以子儀為關内河東副元帥河中節度等使懷恩將士聞之皆曰吾輩從懷恩為不義何面目見汾陽王|{
	帥所類翻}
癸亥以劉晏為太子賓客李峴為詹事並罷政事晏坐與程元振交通元振獲罪峴有力焉由是為宦官所疾|{
	李峴相肅宗不為李輔國所容宦官之媢疾非一日矣峴戶典翻}
故與晏皆罷以右散騎常侍王縉為黃門侍郎太常卿杜鴻漸為兵部侍郎並同平章事|{
	散昔亶翻騎奇寄翻縉音晋為王縉黨附元載與之俱誅張本}
丁卯以郭子儀為朔方節度大使二月子儀至河中雲南子弟萬人戍河中將貪卒暴為一府患子儀斬十四人杖三十人府中遂安 癸酉上朝獻太清宫|{
	太清宫玄宗所建朝音直遥翻}
甲戍享太廟乙亥祀昊天上帝於圜丘 僕固瑒圍榆次旬餘不拔遣使急發祁縣兵李光逸盡與之|{
	李光逸屯祁縣事始上年七月}
士卒未食行不能前十將白玉焦暉以鳴鏑射其後者|{
	後者行不能及衆者也射而亦翻下乃射射之同}
軍士曰將軍何乃射人玉曰今從人反終不免死死一也射之何傷|{
	白玉焦暉欲殺瑒先激其衆}
至榆次瑒責其遲胡人曰我乘馬|{
	乘音承}
乃漢卒不行耳瑒捶漢卒卒皆怨怒曰節度使黨胡人其夕焦暉白玉帥衆攻瑒殺之|{
	帥讀曰率}
僕固懷恩聞之入告其母母曰吾語汝勿反|{
	語牛倨翻}
國家待汝不薄今衆心既變禍必及我將如之何懷恩不對再拜而出母提刀逐之曰吾為國家殺此賊|{
	為于偽翻}
取其心以謝三軍懷恩疾走得免遂與麾下三百度河北走|{
	自汾州西度河投北趣靈州}
時朔方將渾釋之守靈州懷恩檄至云全軍歸鎮釋之曰不然此必衆潰矣將拒之其甥張韶曰彼或翻然改圖以衆歸鎮何可不納也釋之疑未决懷恩行速先候者而至|{
	先昔薦翻}
釋之不得已納之張韶以其謀告懷恩懷恩以韶為間殺釋之而收其軍|{
	史言渾釋之以臨事不决自禍間古莧翻}
使韶主之既而曰釋之舅也彼尚負之安有忠於我哉他日以事杖之折其脛置於彌峩城而死|{
	張韶之死宜矣折而設翻脛音胡定翻脛脚莖也}
都虞侯張維嶽在沁州聞懷恩去乘傳至汾州|{
	傳張戀翻驛馬也}
撫定其衆殺焦暉白玉而竊其功以告郭子儀子儀使牙官盧諒至汾州|{
	節鎮州府皆有牙官行官牙官給牙前驅使行官使之行役出四方自五季以後詬詈武臣率曰牙官}
維嶽賂諒使實其言子儀奏維嶽殺瑒|{
	考異曰汾陽家傳曰開府盧昴公先使汾州慰諭及還惡不比於己者好賂於己者公捶殺之邠志曰郭公}


|{
	使牙官盧諒之軍如岳賂諒使信其言郭公以如岳殺瑒聞詔優之諸將云云郭公乃理諒罪棒殺之今参考二書諒職名從邠志}
傳首詣闕羣臣入賀上惨然不悦曰朕信不及人致勲臣顛越深用為愧又何賀焉命輦懷恩母至長安給待優厚月餘以壽終|{
	史言夀終明非殺之也}
以禮葬之功臣皆感歎戊寅郭子儀如汾州 |{
	考異曰實錄廣德元年十二月丁酉僕固瑒為帳下張維嶽所殺以其衆歸郭子儀懷恩聞之棄營脫身遁走北蕃按朔方兵所以不附僕固氏者以子儀為之帥也縱不在子儀領朔方節度使之後亦當在領河東副元帥之後而實録二年正月丁卯子儀為朔方節度使汾陽家傳二年正月子儀充河東副元帥河中節度使癸亥代宗三殿宴送二十六日上都二月至河中兼朔方節度大使戊寅往汾州甲申還至河中邠志二年正月二十日詔郭公加河中節度河東副元帥二十九日加朔方節度二月僕固瑒帥軍攻榆次逾甸不拔云云然則瑒死决不在去年十二月今因子儀如汾州并言之}
懷恩之衆悉歸之咸鼔舞涕泣喜其來而悲其晩也|{
	卒如顔真卿李抱真之言}
子儀知盧諒之詐杖殺之上以李抱真言有驗遷殿中少監 上之幸陜也李光弼竟遷延不至上恐遂成嫌隙其母在河中數遣中使存問之|{
	數所角翻}
吐蕃退除光弼東都留守以察其去就光弼辭以就江淮糧運引兵歸徐州上迎其母至長安厚加供給使其弟光進掌禁兵遇之加厚|{
	皆所以懷來光弼}
自喪亂以來|{
	喪息浪翻}
汴水堙廢漕運者自江漢抵梁洋迂險勞費|{
	自安祿山作亂關洛路阻漕運泝江入漢抵梁洋故汴渠堙廢不治}
三月已酉以太子賓客劉晏為河南江淮以東轉運使議開汴水庚戍又命晏與諸道節度使均節賦役聽便宜行畢以聞時兵火之後中外艱食關中米斗千錢百姓挼穗以給禁軍|{
	挼奴禾翻}
宫厨無兼時之積|{
	宫厨所以奉上及宫中食膳}
晏乃疏浚汴水遺元載書具陳漕運利病|{
	時元載為相故遺書言漕運事遺唯季翻}
令中外相應自是每歲運米數十萬石以給關中唐世推漕運之能者推晏為首後來者皆遵其法度云 甲子盛王琦薨|{
	琦玄宗子也}
党項寇同州郭子儀使開府儀同三司李國臣擊之曰虜得間則出掠|{
	間古莧翻}
官軍至則逃入山宜使嬴師居前以誘之勁騎居後以覆之國臣與戰於澄城北|{
	澄城春秋左傳北徵之地漢為徵縣属馮翊音澄後魏為澄城縣并置澄城郡隋廢郡存縣唐属同州九域志縣在州北九十里}
大破之斬首捕虜千餘人 夏五月癸丑初行五紀歷|{
	寶應元年六月望戊夜月蝕三之一官歷加時在日出後有交不署蝕代宗以至德歷不與天合詔司天臺官属郭獻之等復用麟德元紀更立歲差增損遲疾交會及五星差數以寫大衍舊術上元七曜起虚四度帝為製序題曰五紀歷}
庚申禮部侍郎楊綰奏歲貢孝弟力田無實狀及童子科皆僥倖悉罷之|{
	孝弟力田之科始於漢}
郭子儀以安史昔據洛陽故諸道置節度使以制其

要衝今大盗已平而所在聚兵耗蠧百姓表請罷之仍自河中為始|{
	子儀時鎮河中表光罷河中節度以示諸鎮君子惜其有安國家尊朝廷之心而時君不能盡用之也}
六月勑罷河中節度及耀德軍|{
	乾元二年置耀德軍於河中}
子儀復請罷關内副元帥|{
	復扶又翻下衆復同}
不許 僕固懷恩至靈武收合散亡其衆復振上厚撫其家癸未下詔稱其勲勞著於帝室及於天下疑隙之端起自羣小察其深衷本無它志君臣之義情實如初但以河北既平朔方已有所屬宜解河北副元帥朔方節度等使|{
	優詔解其職任使河北諸帥不復禀其約束朔方將士心歸子儀}
其太保兼中書令大寜郡王如故但當詣闕更勿有疑懷恩竟不從 秋七月庚子稅天下青苖錢以給百官俸|{
	乾元以來天下用兵京司百寮俸錢減耗上即位推恩庶寮下議公卿或言稅畝有苖者公私咸濟乃分遣憲官稅天下地青苖錢充百司課料宋白曰大歷五年五月詔京兆府應徵青苖地頭錢等承前青苖錢每畝徵十五文地頭錢每畝徵二十五文自今以後宜一切以青苖錢為名每畝減五文徵三十五文随徵夏稅時據數徵納八年每畝率十五文俸方用翻}
太尉兼侍中河南副元帥臨淮武穆王李光弼治軍嚴整|{
	冶直之翻}
指顧號令諸將莫敢仰視謀定而後戰能以少制衆與郭子儀齊名及在徐州擁兵不朝諸將田神功等不復禀畏|{
	少始照翻復扶又翻李光弼處危疑之地其迹若無君者而諸將亦不復禀畏光弼節義天下之大義非虚語也}
光弼愧恨成疾己酉薨|{
	史言李光弼不能以功名自終}
八月丙寅以王縉代光弼都統河南淮西山南東道諸行營|{
	統他綜翻俗從上聲}
郭子儀自河中入朝會涇原奏僕固懷恩引回紇吐蕃十萬衆將入寇|{
	朝直遥翻紇下沒翻吐從暾入聲 考異曰舊子儀傳云數十萬衆懷恩傳云誘吐蕃十萬衆按汾陽家傳實不過十萬}
京師震駭詔子儀帥諸將出鎮奉天|{
	帥讀曰率奉天縣属京兆宋白曰本醴泉縣地武后分置奉天縣以奉乾陵在京兆西北百五十里}
上召問方略對曰懷恩無能為也上曰何故對曰懷恩勇而少恩|{
	少詩沼翻}
士心不附所以能入寇者因思歸之士耳|{
	以此觀之則懷恩將士盖有關内河東人}
懷恩本臣偏禆|{
	禆頻眉翻}
其麾下皆臣部曲必不忍以鋒刃相向以此知其無能為也辛巳子儀發赴奉天 甲午加王縉東都留守|{
	縉音晋守式又翻}
河中尹兼節度副使崔㝢|{
	使疏吏翻㝢于矩翻 考異曰五月已罷河中節度今猶有副使者盖言其前官也}
發鎮兵西禦吐蕃為法不一九月丙申鎮兵作亂掠官府及居民終夕乃定|{
	吐從暾入聲}
丙午加河東節度使辛雲京同平章事 辛亥以郭

子儀充北道邠寜涇原河西以來通和吐蕃使以陳鄭澤潞節度使李抱玉充南道通和吐蕃使|{
	託通和以緩吐蕃之兵吐從暾入聲}
子儀聞吐蕃逼邠州甲寅遣其子朔方兵馬使晞將兵萬人救之|{
	邠卑旻翻將即亮翻又音如字}
己未劒南節度使嚴武破吐蕃七萬衆拔當狗城|{
	當狗城當白狗羌之路故以名城}
關中蟲蝗霖雨米斗千餘錢 僕固懷恩前軍至宜禄郭子儀使右兵馬使李國臣將兵為郭晞後繼邠寜節度使白孝德敗吐蕃于宜禄|{
	將即亮翻又音如字領也敗補賣翻 考異曰實錄癸巳孝德敗吐蕃一千餘衆於宜祿生擒蕃將數人按汾陽家傳二十六日賊先軍次宜祿然則前八日孝德豈得已敗吐蕃於宜祿乎實錄誤也}
冬十月懷恩引回紇吐蕃至邠州白孝德郭晞閉城拒守 |{
	考異曰汾陽家傳晞屢破吐蕃今從實錄舊子儀傳曰虜寇邠州子儀在涇陽子儀令長男朔方兵馬使曜率師拒之與孝德閉城拒守按實錄及晞傳皆云晞拒懷恩破之子儀傳云曜誤也}
庚午嚴武拔吐蕃鹽川城|{
	鹽川城在當狗城西北維州舊有鹽溪縣永徽初省入保州定亷縣}
僕固懷恩與回紇吐蕃進逼奉天京師戒嚴諸將請戰郭子儀不許曰虜深入吾地利於速戰吾堅壁以待之彼以我為怯必不戒乃可破也若遽戰而不利則衆心離矣敢言戰者斬辛未夜子儀出陳於乾陵之南|{
	陳讀曰陣}
壬申未明虜衆大至虜始以子儀為無備欲襲之忽見大軍驚愕遂不戰而退子儀使禆將李懷光等將五千騎追虜至麻亭而還虜至邠州丁丑攻之不克乙酉虜涉涇而遁 |{
	考異曰實錄十月辛未夜郭晞遣馬步三千人於邠州西斬賊營殺千餘人生擒八十三人俘大將四人十一月乙未懷恩及吐蕃等自潰京師解嚴汾陽家傳曰十月七日公誓師曰明日有寇爾其備之及夜出兵數萬陣于西門之外廣布旗幟如十萬軍未曙懷恩吐蕃回紇吐渾等已陣於乾陵北長二十里懷恩等初謂無備欲襲之既見陣兩蕃大駭不敢戰而懷恩頃為公所馭懾公之威又遁初軍中偶語夜中出兵與鬼闘耳及未曙寇已至矣軍中所以服公之先知也賊至於邠州營于北原十三日攻其東門不剋十四日横陣於南原請戰晞等與之連戰大破之追奔數十里二十一日涉涇而還邠志懷恩寇邠涇十七日衆度邠州郭晞率衆禦之戰于邠郭我師敗之懷恩覆其陣泣曰此等昔為我兒我教其射反為它人致死於我惜哉明日引軍南出舊郭晞傳曰懷恩誘虜再寇邠州陣于涇北晞乘其半濟而擊之大破獯虜斬首五千級連戰皆捷吐蕃傳曰郭鋒於邠州西三十里令精騎斫懷恩營破五千衆斬首千餘級生擒八十五人降其大將四人諸書載邠寜戰守勝敗事各不同今從汾陽家傳以實録参之}
懷恩之南寇也河西節度使楊志烈發卒五千謂監軍柏文達曰河西銳卒盡於此矣君將之以攻靈武|{
	使疏吏翻監古銜翻將即亮翻下同又音如字}
則懷恩有返顧之慮此亦救京師之一奇也文達遂將衆擊摧砂堡靈武縣皆下之|{
	摧砂堡在原州西北靈武後周置建安縣後又置歷城郡隋開皇三年廢郡十八年改建安為廣閏仁夀九年又改曰靈武属涇州}
進攻靈州懷恩聞之自永夀遽歸|{
	宋白曰永夀縣属邠州武德元年於永壽原置縣因原立名}
使蕃渾二千騎夜襲文達大破之士卒死者殆半文達將餘衆歸凉州哭而入志烈迎之曰此行有安京室之功卒死何傷士卒怨其言未幾|{
	幾居豈翻}
吐蕃圍涼州士卒不為用志烈奔甘州為沙陀所殺|{
	舊志凉州西北至甘州六百里}
沙陀姓朱耶世居沙陀磧因以為名|{
	沙陀始見于此}
十一月丁未郭子儀自行營入朝郭晞在邠州縱士卒為暴節度使白孝德患之以子儀故不敢言涇州刺史段秀實自請補都虞候|{
	虞候古候奄之職虞防虞也候候望也}
孝德從之既署一月晞軍士十七人入市取酒以刃刺酒翁|{
	酒翁釀酒者也今人呼為酒大工刺七亦翻}
壞釀器|{
	壞古聵翻}
秀實列卒取十七人首注槊上植市門|{
	植直吏翻又時力翻 考異曰此出柳宗元段太尉逸事狀段公家傳曰廣德二年正月白孝德授邠寜節度使七月大軍西還頗有俘掠又以邠土經寇未暇耕耘乃謀頓軍奉天取給畿内時倉廪匱竭史人潜竄軍士公行掘兼施捶訊閭里怨苦遠近彰聞孝德知之力不能制公戲謂賓朋曰若使余為軍候不令至是行軍司馬王稷以其言啟於白孝德即日以公為都虞候兼權知奉天縣事浹旬而軍不犯禁逾月而路不拾遺永泰元年孝德奉詔歸邠州表公進封張掖郡王北庭行軍邠寜都虞侯據實錄時晞官為左常侍宗元云尚書誤也又按實錄廣德二年十月吐蕃寇邠州孝德晞閉城拒守汾陽家傳其年九月公使陳回光與孝德議邉事於汾州則孝德不以永泰元年始歸汾州陳翃誤也逸事狀又云先是太尉在涇州為營田官涇大將焦令諶取人田自占數十頃給輿農曰且熟歸我半是歲大旱野無草農以告諶曰我知人數而已不知旱也督責益急且飢死無以償即告太尉太尉判狀辭甚巽使人求諭諶諶盛怒召農者曰我畏段某耶何敢言我取判鋪背上以大杖擊二十垂死輿來庭中太尉大泣曰乃我困汝即自取水冼去血裂裳衣瘡手注善藥旦夕自哺農者然後食取騎馬賣市穀代償使勿知淮西寓軍帥尹少榮剛直士也入見諶大罵曰汝誠人邪涇州野如赭人且飢死而必得穀又用大杖擊無罪者段公仁信大人也而汝不知敬今段公唯一馬賤賣市穀入汝又取不恥凡為人傲天災犯大人擊無罪者又取仁者穀使主人出無馬汝將何以視天地尚不愧奴隸耶諶雖暴抗然聞言則大愧流汗不能食曰吾終不可以見段公一昔自恨死按段公别傳大歷八年焦令諶猶存盖宗元得於傳聞其實令諶不死也}
晞一營大譟盡甲孝德震恐召秀實曰奈何秀實曰無傷也請往解之孝德使數十人從行秀實盡辭去選老躄者一人|{
	譟則竈翻躄俾亦翻跛之甚者也}
持馬至晞門下甲者出秀實笑且入曰殺一老卒何甲也吾戴吾頭來矣|{
	老卒秀實自謂也言戴頭而來聽其斬之也}
甲者愕因諭曰常侍負若屬邪|{
	邪音耶晞時帶左散騎常侍故稱之}
副元帥負若屬邪|{
	副元帥謂子儀帥所類翻}
奈何欲以亂敗郭氏|{
	敗補賣翻}
晞出秀實讓之曰副元帥勲塞天地|{
	塞昔則翻}
當念始終今常侍恣卒為暴行且致亂亂則罪及副元帥亂由常侍出然則郭氏功名其存者幾何|{
	幾居豈翻}
言未畢晞再拜曰公幸教晞以道|{
	句斷}
恩甚大敢不從命顧叱左右皆解甲散還火伍中敢譁者死|{
	還從宣翻又音如字唐制兵五人為伍十人為火}
秀實因留宿軍中晞通夕不解衣戒候卒擊柝衛秀實旦俱至孝德所謝不能|{
	句斷}
請改|{
	請改過也}
邠州由是無患 五谷防禦使薛景仙討南山羣盗連月不克上命李抱玉討之賊帥高玉最強抱玉遣兵馬使李崇客將四百騎自洋州入襲之於桃虢川大破之玉走成固|{
	使疎吏翻騎奇計翻洋音祥又如字走音奏將即亮翻又如字洋州古成固縣唐志洋州治興道縣即古成固地}
庚申山南西道節度使張獻誠擒玉獻之餘盗皆平 十二月乙丑加郭子儀尚書令子儀以為自太宗為此官累聖不復置|{
	復扶又翻}
近皇太子亦嘗為之非微臣所宜當固辭不受還鎮河中 是歲戶部奏戶二百九十餘萬口一千六百九十餘萬|{
	史言喪亂之後戶口減於承平什七八}
上遣于闐王勝還國勝固請留宿衛以國授其弟曜|{
	勝令弟曜攝國自將兵入援見二百十九卷至德元載闐徒賢翻又徒見翻}
上許之加勝開府儀同三司賜爵武都王永泰元年春正月癸卯朔改元赦天下 戊申加陳鄭澤潞節度使李抱玉鳳翔隴右節度使|{
	李抱玉時以陳鄭澤潞行營兵屯京西故加鳳翔隴右節度使}
以其從弟殿中少監抱真為澤潞節度副使|{
	從才用翻少始照翻}
抱真以山東有變上黨為兵衝而荒亂之餘土瘠民困無以贍軍乃籍民每三丁選一壯者免其租徭給弓矢使農隙習戰歲暮都試|{
	史炤曰謂摠閲試習武備也霍光傳都肄郎孟康曰都試也漢制郡國以八月都試閱武備}
行其賞罰比三年得精兵二萬|{
	比必利翻及也}
既不費廪給府庫充實遂雄視山東由是天下稱澤潞步兵為諸道最|{
	為李抱真以澤潞兵横制諸叛張本}
二月戊寅党項寇富平焚定陵殿|{
	党底朗翻定陵中宗陵也在雍州富平縣西北凡陵有寢有殿後曰寢前曰殿}
庚辰儀王璲薨|{
	璲玄宗子也音遂}
三月壬辰朔命左僕射装冕右僕射郭英乂等文武之臣十三人於集賢殿待制|{
	水徽中命弘文館學士二人日待制於武德殿西門文明元年詔京官五品以上清官日一人待制於章善明福門先天末又命朝集使六品以上二人随仗待制時勲臣罷節制無職事令待制於集賢殿門宋白曰是年詔左僕射裴冕右僕射郭英乂太子少傳裴遵慶檢校太子少保兼御史大夫元志直太子詹事兼御史大夫臧希讓左散騎常侍暢璀檢校刑部尚書王昂高昇檢校工部尚書知省事崔涣吏部侍郎李季師王延昌禮部侍郎賈至涇王傳吴令瑶集賢待制以勲臣罷節制無職事乃於禁門書院待制間以文儒寵之也}
左拾遺洛陽獨孤及上疏曰陛下召冕等待制以備詢問此五帝盛德也|{
	上時掌翻疏所句翻}
頃者陛下雖容其直而不録其言有容下之名無聽諫之實遂使諫者稍稍鉗口飽食相招為禄仕此忠鯁之人所以竊歎而臣亦恥之今師興不息十年矣|{
	鉗其亷翻玄宗天寶十四年安禄山反至是十年}
人之生產空於杼軸擁兵者第館亘街陌奴婢厭酒肉而貧人羸餓就役剥膚及髄長安城中白晝椎剽|{
	羸倫為翻椎直追翻剽匹妙翻}
吏不敢詰官亂職廢將墮卒暴百揆隳刺如沸粥紛麻|{
	唐虞有百揆之官孔安國曰揆度也度百事摠百官此所謂百揆盖言百官之事也詰去吉翻將即亮翻刺來達翻}
民不敢訴於有司有司不敢聞於陛下茹毒飲痛窮而無告陛下不以此時思所以救之之術臣實懼焉今天下惟朔方隴西有吐蕃僕固之虞邠涇鳳翔之兵足以當之矣自此而往東洎海南至番禺|{
	吐從暾入聲邠卑巾翻洎其計翻番禺音潘愚}
西盡巴蜀無鼠竊之盗而兵不為解|{
	為于偽翻}
傾天下之貨竭天下之穀以給不用之軍臣不知其故假令居安思危自可扼要害之地俾置屯禦|{
	令力丁翻扼與阨同}
悉休其餘以糧儲屝屨之資|{
	屝扶沸翻草僑也黄帝臣於則所造}
充疲人貢賦歲可減國租之半陛下豈可持疑於改作使率土之患日甚一日乎上不能用 丙午以李抱玉同平章事鎮鳳翔如故 庚戍吐蕃遣使請和詔元載杜鴻漸與盟於興唐寺|{
	程大昌曰安國寺在朱雀街東第四街之長樂坊興唐寺别在向南一坊開元八年造會要興唐寺在大寜坊神龍元年太平公主為天后立為罔極寺開元二十年改為興唐寺使疏吏翻載祖亥翻又音如字}
上問郭子儀吐蕃請盟何如對曰吐蕃利我不虞若不虞而來國不可守矣|{
	不虞猶不備也}
乃相繼遣河中兵戍奉天又遣兵廵涇原以覘之|{
	覘丑廉翻又丑艶翻}
是春不雨米斗千錢 夏四月丁丑命御史大夫王翊充諸道稅錢使河東道租庸鹽鐵使裴諝入奏事|{
	諝私呂翻}
上問榷酤之利歲入幾何諝久之不對上復問之|{
	復扶又翻}
對曰臣自河東來所過見菽粟未種農夫愁怨臣以為陛下見臣必先問人之疾苦乃責臣以營利臣是以未敢對也上謝之拜左司郎中諝寛之子也|{
	裴寛事玄宗}
辛卯劒南節度使嚴武薨武三鎮劒南|{
	按至德二載收長安以武}


|{
	為劒南東川節度使上皇誥以劒南兩川合為一道拜武成都尹充劒南節度使既召入朝去年復以武鎮劒南凡再鎮劍南前後三受命耳廣德二年考異謂東西川非上皇誥所合者盖至德二載分劍南為東西川是年上皇還京師則合東西川為一道必非上皇誥通鑑乾元元年書嚴武貶巴州寶應元年書以兵部侍郎嚴武為西川節度使廣德二年書合東西川以黄門侍郎嚴武為節度使據通鑑所書武盖再鎮劍南今曰三鎮劒南則是先嘗除東川乃可言三通鑑既不取新舊二書宜不書除東川一節然言武三鎮劒南更須博考按下卷書武用崔旰事亦只再鎮劒南耳唐書盖因杜甫詩有主恩前後三持節之語致有此誤}
厚賦歛以窮奢侈梓州刺史章彛小不副意召而杖殺之然吐蕃畏之不敢犯其境母數戒其驕暴|{
	歛力贍翻吐從暾入聲數所角翻}
武不從及死母曰吾今始免為官婢矣|{
	言武驕暴以悖逆致罪禍必及母今其死乃知免}
五月癸丑以右僕射郭英乂為劒南節度使|{
	為郭英乂為崔旰所殺張本射寅謝翻使疏吏翻}
畿内麥稔京兆尹第五琦請稅百姓田十畝收其一曰此古什一之法也|{
	史言第五琦傅會古法以欺君琦音奇}
上從之 平盧節度使侯希逸鎮淄青|{
	侯希逸鎮淄青始上卷上元二年使疏吏翻下同}
好游畋營塔寺|{
	好呼到翻}
軍州苦之兵馬使|{
	使疏吏翻}
李懷玉得衆心希逸忌之因事解其軍職希逸與巫宿於城外軍士閉門不納奉懷玉為帥希逸奔滑州上表待罪|{
	帥所類翻}
詔赦之召還京師|{
	還從宣翻又音如字}
秋七月壬辰以鄭王邈為平盧淄青節度大使|{
	邈皇子也}
以懷玉知留後賜名正已時承德節度使李寶臣魏博節度使田承嗣相衛節度使薛嵩盧龍節度使李懷仙收安史餘黨各擁勁卒數萬治兵完城|{
	相昔亮翻嗣祥史翻治直之翻}
自署文武將吏不供貢賦與山南東道節度使梁崇義及正已皆結為昏姻互相表裏朝廷專事姑息不能復制雖名藩臣羈縻而已|{
	史因李正已逐侯希逸究言藩鎮之横復扶又翻將即亮翻朝直遥翻}
甲午以上女昇平公主嫁郭子儀之子曖|{
	曖烏代翻}
太子母沈氏吳興人也|{
	吳興湖州}
安禄山之陷長安也|{
	見二百十八卷肅宗至德元載}
掠送洛陽宫上克洛陽|{
	見二百二十一卷乾元二年}
見之未及迎歸長安會史思明再陷洛陽|{
	見一百二十一卷乾元二年}
遂失所在上即位遣使散求之不獲己亥夀州崇善寺尼廣澄詐稱太子母按驗乃故少陽院乳母也|{
	大明宫有少陽院太子居之尼女夷翻少詩照翻}
鞭殺之 九月庚寅朔置百高座於資聖西明兩寺|{
	據百高座百尺高座也唐會要資聖寺在崇仁坊本長孫無忌宅龍朔三年為文德皇后資福立為尼寺咸亨四年復為僧寺西明寺在延康坊本越國公楊素宅貞觀中賜濮王泰泰死乃立為寺}
講仁王經|{
	所謂護國仁王經}
内出經二寶輿以人為菩薩鬼神之狀導以音樂鹵簿百官迎於光順門外從至寺|{
	菩音蒲薩桑葛翻}
僕固懷恩誘回紇吐蕃吐谷渾党項奴刺|{
	誘羊久翻吐從暾入聲紇下沒翻谷音浴党音底朗翻刺盧達翻}
數十萬衆俱入寇令吐蕃大將尚結悉贊摩馬重英等自北道趣奉天党項帥任敷鄭庭郝德等自東道趣同州|{
	宋白曰任敷朔方舊將趣七喻翻下同}
吐谷渾奴刺之衆自西道趣盩厔回紇繼吐蕃之後懷恩又以朔方兵繼之郭子儀使行軍司馬趙復入奏曰虜皆騎兵其來如飛不可易也|{
	易以䜴翻輕也}
請使諸道節度使鳳翔李抱玉滑濮李光庭邠寜白孝德|{
	李光庭恐當作李光進}
鎮西馬璘河南郝庭玉淮西李忠臣各出兵以扼其衝要|{
	此時李光庭郝庭玉李忠臣各在本道餘皆分屯京西}
上從之諸道多不時出兵李忠臣方與諸將擊毬得詔亟命治行|{
	治直之翻}
諸將及監軍皆曰師行必擇日忠臣怒曰父母有急豈可擇日而後救耶即日勒兵就道懷恩中途遇暴疾而歸丁酉死於鳴沙 |{
	考異曰舊懷恩傳曰懷恩領回紇及朔方之衆繼進行至鳴沙縣遇疾舁歸九月九日死于靈武按長歷九月庚寅朔丁酉八日也唐歷邠志皆云九月八日懷恩死於靈州今從實錄}
大將張韶代領其衆别將徐璜玉殺之范志誠又殺璜玉而領其衆懷恩拒命三年再引胡寇為國大患|{
	去年及今年之寇}
上猶為之隱|{
	為于偽翻}
前後勑制未嘗言其反及聞其死憫然曰懷恩不反為左右所誤耳吐蕃至邠州白孝德嬰城自守甲辰上命宰相及諸司長宫|{
	六曹有尚書寺有卿監有監皆為諸司長官}
於西明寺行香設素饌奏樂|{
	徼福於佛也}
是日吐蕃十萬衆至奉天京城震恐朔方兵馬使渾瑊|{
	渾戶昆翻又戶本翻瑊古咸翻}
討擊使白元光先戍奉天虜始列營瑊帥驍騎二百衝之身先士卒虜衆披靡|{
	先昔見翻披普佊翻}
瑊挾虜將一人躍馬而還從騎無中鋒鏑者|{
	從才用翻中竹仲翻}
城上士卒望之勇氣始振乙巳吐蕃進攻之虜死傷甚衆數日歛衆還營瑊夜引兵襲之殺千餘人前後與虜戰二百餘合斬首五千級 丙午罷百高座講召郭子儀於河中使屯涇陽己酉命李忠臣屯東渭橋李光進屯雲陽馬璘郝庭玉屯便橋李抱玉屯鳳翔内侍駱奉仙將軍李日越屯盩厔同華節度使周智光屯同州鄜坊節度使杜冕屯坊州上自將六軍屯苑中庚戍下制親征辛亥魚朝恩請索城中括士民私馬令城中男子皆衣皁|{
	索山客翻衣於既翻}
團結為兵城門皆塞二開一|{
	塞悉則翻}
士民大駭踰垣鑿竇而逃者甚衆吏不能禁朝恩欲奉上幸河中以避吐蕃恐羣臣議論不一一旦百官入朝立班久之閤門不開|{
	閤門謂東西上閤門也}
朝恩忽從禁軍十餘人操白刃而出宣言吐蕃數犯郊畿|{
	操七刀翻數所角翻}
車駕欲幸河中何如公卿皆錯愕不知所對有劉給事者獨出班抗聲曰勑使反邪|{
	唐人謂宦官為勑使使踈吏翻}
今屯軍如雲不勠力扞寇而遽欲脅天子棄宗廟社稷而去非反而何朝恩驚沮而退事遂寢|{
	劉給事立朝守正不可奪如此且兩省官也而史失其名唐置史館何為哉考異曰新魚朝恩傳云僕固瑒攻絳州使姚良據温誘回紇䧟河陽朝恩遣李忠臣討瑒以霍文場監之王景岑討良王希遷監之敗瑒於萬泉生擒良高暉等引吐蕃入寇遣劉德信討斬之故朝恩因麾下數克獲竊以自高是時郭子儀有定天下功居人臣第一心媢之乘相州敗醜為詆譛肅宗不内其語然猶罷子儀兵留京師代宗立與程元振一口加毀帝未及寤子儀憂甚俄而吐蕃陷京師卒用其力王室再安朝恩内慙乃勸帝徙洛陽欲遠戎狄百僚在庭朝恩從十餘人持兵出曰虜犯都甸欲幸洛云何宰相未對有近臣折曰勑使反邪今屯兵足以扞寇何遽脅天子弃宗廟為朝恩色沮而子儀亦謂不可乃止李肇國史補曰代宗朝百僚立班良久閤門不開魚朝恩忽擁白刃十餘人而出宣言曰西蕃頻犯郊圻欲幸河中何如宰臣以下蒼黄不知所對給事中劉不記其名出班抗聲曰勑使反邪云云由此罷遷幸之議按僕固瑒攻榆次不聞攻絳州高暉為李日越所擒不聞劉德信所斬朝恩欲幸河中不聞欲幸洛既云頻犯郊圻必是吐蕃後入寇時也新書所云不知據何書今從國史補}
自丙午至甲寅大雨不止故虜不能進吐蕃移兵攻醴泉党項西掠白水|{
	白水縣漢粟邑之地後魏太和二年置白水縣及白水郡隋廢郡存縣唐属同州}
東侵蒲津丁巳吐蕃大掠男女數萬而去所過焚廬舍蹂禾稼殆盡周智光引兵邀擊破之于澄城北因逐北至鄜州|{
	宋白曰鄜州漢上郡雕隂之地後魏太和十一年置東秦州孝昌二年又改為北華州廢帝二年改為鄜州因鄜畤為名九域志鄜州東南至同州四百一十四里澄城縣在同州北九十里坊州漢渠搜縣中部都尉理所後魏属鄜州管内周天和七年元皇帝作牧鄜州於此置馬坊唐高祖因置坊州取馬坊為名九域志坊州北至鄜州一百一十里鄜音夫}
智光素與杜冕不協遂殺鄜州刺史張麟阬冕家屬八十一人焚坊州廬舍三千餘家冬十月己未復講經於資聖寺吐蕃退至邠州遇回紇復相與入寇|{
	復扶又翻下復至同}
辛酉至奉天|{
	考異曰邠志云八月懷恩以諸戎入寇九月詔郭公討之師于涇陽回紇屯涇北去我十里朝恩請擊回紇郭公曰我昔與回紇情契頗至今兹為寇必將有故吾方導而問之可不戰而下也朝恩流言謂郭公與懷恩為應隂率諸軍列營渭上郭公章疏逾旬不達郭公諸子在長安聞之使小將強羽以物議告郭公郭公間道入覲且以衆議聞上曰無是即日令赴涇陽懷恩驚曰郭公真長者吾比疑之誠小人也按回紇九月未至涇陽十月辛酉始至奉天丙寅圍涇陽丁卯子儀已與之盟首尾纔七日豈容有章疏逾旬不逹之事子儀為元帥與強敵對壘豈可棄軍入朝汾陽家傳此際亦無入朝事今不取}
癸亥党項焚同州官廨民居而去|{
	廨居隘翻}
丙寅回紇吐蕃合兵圍涇陽子儀命諸將嚴設守備而不戰及暮二虜退屯北原|{
	涇陽之北原也}
丁卯復至城下是時回紇與吐蕃聞僕固懷恩死已争長不相睦|{
	長知兩翻}
分營而居子儀知之回紇在城西子儀使牙將李光瓚等往說之|{
	牙將者牙前將領統元帥親兵說式芮翻下往說同}
欲與之共擊吐蕃回紇不信曰郭公固在此乎汝紿我耳若果在此可得見乎光瓚還報子儀曰今衆寡不敵難以力勝昔與回紇契約甚厚不若挺身往說之可不戰而下也諸將請選鐵騎五百為衛從|{
	從才用翻}
子儀曰此適足為害也郭晞扣馬諫曰彼虎狼也大人國之元帥奈何以身為虜餌子儀曰今戰則父子俱死而國家危往以至誠與之言或幸而見從則四海之福也不然則身沒而家全|{
	子儀之審處利害而權其輕重者如此}
以鞭擊其手曰去遂與數騎開門而出使人傳呼曰令公來|{
	子儀時為中書令故傳呼令公}
回紇大驚其大帥合胡禄都督藥葛羅可汗之弟也執弓注矢立於陣前子儀免胄釋甲投槍而進回紇諸酋長相顧曰是也|{
	酋慈由翻}
皆下馬羅拜子儀亦下馬前執藥葛羅手讓之曰汝回紇有大功於唐|{
	謂舉兵助唐平安史也}
唐之報汝亦不薄奈何負約深入吾地侵逼畿縣|{
	唐京都属縣附城之縣為赤為次赤如昭應奉天醴泉等縣為次赤餘為畿縣}
棄前功結怨仇背恩德而助叛臣何其愚也|{
	背蒲妹翻}
且懷恩叛君棄母|{
	謂懷恩阻兵汾絳既而叛歸靈武棄母於汾州也}
於汝國何有今吾挺身而來聽汝執我殺之我之將士必致死與汝戰矣藥葛羅曰懷恩欺我言天可汗已晏駕令公亦捐館中國無主我是以敢與之來今知天可汗在上都|{
	自貞觀中四夷君長謂太宗為天可汗是後夷人率謂天子為天可汗上都長安也}
令公復摠兵於此懷恩又為天所殺我曹豈肯與令公戰乎子儀因說之曰吐蕃無道乘我國有亂不顧舅甥之親|{
	吐蕃尚唐公主為舅甥之國復扶又翻說式芮翻}
吞噬我邊鄙焚蕩我畿甸其所掠之財不可勝載|{
	勝音升}
馬牛雜畜長數百里|{
	畜許救翻長直亮翻}
彌漫在野此天以賜汝也全師而繼好|{
	漫音萬又莫官翻好呼到翻}
破敵以取富為汝計孰便於此不可失也藥葛羅曰吾為懷恩所誤負公誠深今請為公盡力擊吐蕃以謝過|{
	為于偽翻}
然懷恩之子可敦兄弟也願捨之勿殺子儀許之回紇觀者為兩翼稍前子儀麾下亦進子儀揮手却之因取酒與其酋長共飲藥葛羅使子儀先執酒為誓子儀酹地曰|{
	紇下沒翻酋慈由翻長知兩翻酹盧對翻以酒沃於地曰酹}
大唐天子萬歲回紇可汗亦萬歲兩國將相亦萬歲有負約者身殞陳前家族滅絶|{
	可從刋入聲汗音寒將即亮翻相息亮翻陳讀曰陣}
盃至藥葛羅亦酹地曰如令公誓於是諸酋長皆大喜曰曏以二巫師從軍巫言此行甚安隱|{
	隐讀曰穩}
不與唐戰見一大人而還今果然矣子儀遺之綵三千匹|{
	遺唯季翻}
酋長分以賞巫子儀竟與定約而還吐蕃聞之夜引兵遁去回紇遣其酋長石野那等六人入見天子|{
	見賢遍翻}
藥葛羅帥衆追吐蕃子儀使白元光帥精騎與之俱癸酉戰於靈臺西原大破之|{
	靈臺漢鶉觚縣地天寶元年更名靈臺九域志靈臺縣在涇州東九十里舊史破吐蕃處在靈臺縣西五十里地名赤山嶺}
殺吐蕃萬計得所掠士女四千人丙子又破之於涇州東 |{
	考異曰實錄云十月吐蕃退至邠州與回紇相遇復合從為寇辛酉寇奉天乙亥回紇以懷恩死貳於吐蕃丁丑郭子儀单騎詣回紇軍免胄與回紇大將語責以負約遂與之盟己卯回紇首領石野那等六人來朝庚辰子儀遣白元光帥精銳會囘紇兵數千人大破吐蕃十餘萬衆于靈臺縣之西原汾陽家傳曰十月八日吐蕃回紇合圍涇陽屯于北原其夜公使方面各除道二詰朝將戰明日寇又至兵甲益盛公使衙前將李光瓚等出諭之亦不受請决戰公以虜騎勁亦以衆寡不敵孤軍無救使闢軍門躍一騎而出兵部郎中馬錫主客員外郎陳翃時以一騎從回紇合胡祿都督藥葛羅宰相立于陣前持滿相向公前叱之云云藥葛羅等惘然懷慙伏而請罪因與之盟吐蕃聞之夜半抽兵而逸回紇藥葛羅等遽追之公使白元光等繼之十五日至靈臺破尚結息一十萬衆十八日於涇州東又破之舊子儀傳曰子儀自河中至屯於涇陽而虜騎已合子儀一軍萬餘人而雜虜圍之數重子儀使李國臣高昇拒其東魏楚王當其南陳迴光當其西朱元琮當其北子儀帥甲騎二十出沒于左右前後虜見而問曰此誰也報曰郭令公也囘紇曰令公存乎僕固懷恩言天可汗已弃四海令公亦謝世中國無主故從其來今令公存天可汗存乎報之曰皇帝萬夀無彊回紇皆曰懷恩欺我子儀又使諭之云云回紇曰謂令公亡矣不然何以至此令公誠存安得而見之子儀將出諸將諫子儀曰今力固不敵且至誠感神况虜輩乎回紇傳曰吐蕃將馬重英等十月初引退取邠州舊路而歸回紇首領羅達干等帥其衆二千餘騎詣涇陽請降子儀許之率衆被甲持滿數千人回紇譯曰此來非惡心要見令公子儀曰我令公也回紇曰請去甲子儀便脫兜鍪槍甲竦馬挺身而前回紇酋長相顧曰是也便下馬羅拜子儀亦下馬執回紇大將合胡祿都督藥葛羅等手責讓之曰國家知回紇冇功報汝大厚汝何負約犯我王畿我須與汝戰何乃降為我一身挺入汝營任心拘縶我下將士須與汝戰回紇又譯曰懷恩負心來報可汗云唐國天子今已向冮淮令公亦不主兵我是以敢來今知天可汗見在上都令公為將懷恩天又殺之今請追殺吐蕃收其羊馬以報國恩邠志曰十月二十四日回紇逼涇陽陣於郭西使漢語者曰城中誰將軍吏對曰郭令公也虜曰郭令公亡矣紿我也郭公聞之獨與家童五六人常服相詣其子晞等扣馬止之公撾其手曰去使人告虜按轡就之回紇熟視曰是也下馬皆拜曰始者不知令公尚在今日降可乎郭公入其衆取酒飲之虜又請曰恐不見信願擊吐蕃以自効郭公從之回紇擊吐蕃逐之三十日敗蕃衆于靈臺殺萬餘人而去按長歷十月己未朔三日辛酉十九日丁丑如實録所言豈有回紇吐蕃數十萬衆入京畿留十七日而寂無攻戰之一事乎當是時陳翃在子儀軍中所記月日近得其實今二虜圍涇陽及子儀與回紇盟及破吐蕃月日皆從汾陽家傳事則兼採衆書擇其可信者取之}
丁丑僕固懷恩將張休藏等降辛巳詔罷親征京城解嚴 初肅宗以陜西節度使郭英乂領神策軍使内侍魚朝恩監其軍英乂入為僕射朝恩專將之|{
	將即亮翻下同}
及上幸陜朝恩舉在陜兵與神策軍迎扈悉號神策軍天子幸其營及京師平朝恩遂以軍歸禁中自將之然尚未得與北軍齒|{
	北軍北門六軍也}
至是朝恩以神策軍從上屯苑中其勢寖盛分為左右廂居北軍之右矣|{
	史言神策軍雄盛之所由始}
郭子儀以僕固名臣李建忠等皆懷恩驍將恐逃入外夷請招之名臣懷恩之姪也時在回紇營上勑并舊將有功者皆赦其罪令回紇送之壬午名臣以千餘騎來降子儀使開府儀同三司慕容休貞以書諭党項帥鄭庭郝德等皆詣鳳翔降 甲申周智光詣闕獻捷再宿歸鎮智光負專殺之罪未治|{
	謂殺張麟及杜冕家属之罪治直之翻}
上既遣而悔之乙酉回紇胡禄都督等二百餘人入見前後贈賚繒

帛十萬匹府藏空竭稅百官俸以給之|{
	見賢遍翻藏徂浪翻}


資治通鑑卷二百二十三
