<!DOCTYPE html PUBLIC "-//W3C//DTD XHTML 1.0 Transitional//EN" "http://www.w3.org/TR/xhtml1/DTD/xhtml1-transitional.dtd">
<html xmlns="http://www.w3.org/1999/xhtml">
<head>
<meta http-equiv="Content-Type" content="text/html; charset=utf-8" />
<meta http-equiv="X-UA-Compatible" content="IE=Edge,chrome=1">
<title>資治通鑒_257-資治通鑑卷二百五十六_257-資治通鑑卷二百五十六</title>
<meta name="Keywords" content="資治通鑒_257-資治通鑑卷二百五十六_257-資治通鑑卷二百五十六">
<meta name="Description" content="資治通鑒_257-資治通鑑卷二百五十六_257-資治通鑑卷二百五十六">
<meta http-equiv="Cache-Control" content="no-transform" />
<meta http-equiv="Cache-Control" content="no-siteapp" />
<link href="/img/style.css" rel="stylesheet" type="text/css" />
<script src="/img/m.js?2020"></script> 
</head>
<body>
 <div class="ClassNavi">
<a  href="/24shi/">二十四史</a> | <a href="/SiKuQuanShu/">四库全书</a> | <a href="http://www.guoxuedashi.com/gjtsjc/"><font  color="#FF0000">古今图书集成</font></a> | <a href="/renwu/">历史人物</a> | <a href="/ShuoWenJieZi/"><font  color="#FF0000">说文解字</a></font> | <a href="/chengyu/">成语词典</a> | <a  target="_blank"  href="http://www.guoxuedashi.com/jgwhj/"><font  color="#FF0000">甲骨文合集</font></a> | <a href="/yzjwjc/"><font  color="#FF0000">殷周金文集成</font></a> | <a href="/xiangxingzi/"><font color="#0000FF">象形字典</font></a> | <a href="/13jing/"><font  color="#FF0000">十三经索引</font></a> | <a href="/zixing/"><font  color="#FF0000">字体转换器</font></a> | <a href="/zidian/xz/"><font color="#0000FF">篆书识别</font></a> | <a href="/jinfanyi/">近义反义词</a> | <a href="/duilian/">对联大全</a> | <a href="/jiapu/"><font  color="#0000FF">家谱族谱查询</font></a> | <a href="http://www.guoxuemi.com/hafo/" target="_blank" ><font color="#FF0000">哈佛古籍</font></a> 
</div>

 <!-- 头部导航开始 -->
<div class="w1180 head clearfix">
  <div class="head_logo l"><a title="国学大师官网" href="http://www.guoxuedashi.com" target="_blank"></a></div>
  <div class="head_sr l">
  <div id="head1">
  
  <a href="http://www.guoxuedashi.com/zidian/bujian/" target="_blank" ><img src="http://www.guoxuedashi.com/img/top1.gif" width="88" height="60" border="0" title="部件查字,支持20万汉字"></a>


<a href="http://www.guoxuedashi.com/help/yingpan.php" target="_blank"><img src="http://www.guoxuedashi.com/img/top230.gif" width="600" height="62" border="0" ></a>


  </div>
  <div id="head3"><a href="javascript:" onClick="javascript:window.external.AddFavorite(window.location.href,document.title);">添加收藏</a>
  <br><a href="/help/setie.php">搜索引擎</a>
  <br><a href="/help/zanzhu.php">赞助本站</a></div>
  <div id="head2">
 <a href="http://www.guoxuemi.com/" target="_blank"><img src="http://www.guoxuedashi.com/img/guoxuemi.gif" width="95" height="62" border="0" style="margin-left:2px;" title="国学迷"></a>
  

  </div>
</div>
  <div class="clear"></div>
  <div class="head_nav">
  <p><a href="/">首页</a> | <a href="/ShuKu/">国学书库</a> | <a href="/guji/">影印古籍</a> | <a href="/shici/">诗词宝典</a> | <a   href="/SiKuQuanShu/gxjx.php">精选</a> <b>|</b> <a href="/zidian/">汉语字典</a> | <a href="/hydcd/">汉语词典</a> | <a href="http://www.guoxuedashi.com/zidian/bujian/"><font  color="#CC0066">部件查字</font></a> | <a href="http://www.sfds.cn/"><font  color="#CC0066">书法大师</font></a> | <a href="/jgwhj/">甲骨文</a> <b>|</b> <a href="/b/4/"><font  color="#CC0066">解密</font></a> | <a href="/renwu/">历史人物</a> | <a href="/diangu/">历史典故</a> | <a href="/xingshi/">姓氏</a> | <a href="/minzu/">民族</a> <b>|</b> <a href="/mz/"><font  color="#CC0066">世界名著</font></a> | <a href="/download/">软件下载</a>
</p>
<p><a href="/b/"><font  color="#CC0066">历史</font></a> | <a href="http://skqs.guoxuedashi.com/" target="_blank">四库全书</a> |  <a href="http://www.guoxuedashi.com/search/" target="_blank"><font  color="#CC0066">全文检索</font></a> | <a href="http://www.guoxuedashi.com/shumu/">古籍书目</a> | <a   href="/24shi/">正史</a> <b>|</b> <a href="/chengyu/">成语词典</a> | <a href="/kangxi/" title="康熙字典">康熙字典</a> | <a href="/ShuoWenJieZi/">说文解字</a> | <a href="/zixing/yanbian/">字形演变</a> | <a href="/yzjwjc/">金 文</a> <b>|</b>  <a href="/shijian/nian-hao/">年号</a> | <a href="/diming/">历史地名</a> | <a href="/shijian/">历史事件</a> | <a href="/guanzhi/">官职</a> | <a href="/lishi/">知识</a> <b>|</b> <a href="/zhongyi/">中医中药</a> | <a href="http://www.guoxuedashi.com/forum/">留言反馈</a>
</p>
  </div>
</div>
<!-- 头部导航END --> 
<!-- 内容区开始 --> 
<div class="w1180 clearfix">
  <div class="info l">
   
<div class="clearfix" style="background:#f5faff;">
<script src='http://www.guoxuedashi.com/img/headersou.js'></script>

</div>
  <div class="info_tree"><a href="http://www.guoxuedashi.com">首页</a> > <a href="/SiKuQuanShu/fanti/">四库全书</a>
 > <h1>资治通鉴</h1> <!--         下载:【右键另存为】即可 --></div>
  <div class="info_content zj clearfix">
  
<div class="info_txt clearfix" id="show">
<center style="font-size:24px;">257-資治通鑑卷二百五十六</center>
    資治通鑑卷二百五十六 宋 司馬光 撰<br />
<br />
  胡三省 音註<br />
<br />
  唐紀七十二【起閼逢執徐六月盡彊圉協洽三月凡三年有奇】<br />
<br />
  僖宗惠聖恭定孝皇帝下之上<br />
<br />
  中和四年六月壬辰東川留後高仁厚奏鄭君雄斬楊師立出降仁厚圍梓州久不下乃為書射城中道其將士曰仁厚不忍城中玉石俱焚為諸君緩師十日【射而亦翻道讀曰導為于偽翻】使諸君自成其功若十日不送師立首當分見兵為五番【見賢遍翻】番分晝夜以攻之於此甚逸於彼必困矣五日不下四面俱進克之必矣諸君圖之數日君雄大呼於衆曰【呼火故翻】天子所誅者元惡耳他人無預也衆呼萬歲大譟突入府中師立自殺君雄挈其首出降【考異曰張耆舊傳四年七月一日高僕射羽檄入城云云師立自殺七月三日張鄭二將持師立首級出】<br />
<br />
  【降七月七日高僕射上東川句延慶傳曰三年五月高公進軍東川城下飛檄入城師立自刎七月辛酉師立首級至成都實録六月丙申高仁厚奏東川都將鄭君雄梟斬楊師立傳首於行在是日詔以仁厚為東川節度使續寶運録二月梓州觀察使楊師立反勑差蜀將高仁厚等討平六月三日收得梓州并楊師立首級至駕前新紀七月辛酉楊師立伏誅今日從續寶運録事從實録】仁厚獻其首及妻子于行在陳敬瑄釘其子於城北【釘丁定翻】敬瑄三子出觀之釘者呼曰【呼火故翻】兹事行及汝曹汝曹於後努力領取三子走馬而返以高仁厚為東川節度使 甲辰武寜將李師悅與尚讓追黄巢至瑕丘敗之【宋白曰春秋以邾子益來囚諸負瑕杜預注云魯邑也高平郡南平陽縣西北有瑕丘城漢為瑕丘縣敗補邁翻】巢衆殆盡走至狼虎谷【狼虎谷在泰山東南萊蕪界】丙午巢甥林言斬巢兄弟妻子首將詣時溥遇沙陀博野軍奪之并斬言首以獻於溥【黄巢乾符三年起兵為盜至是凡十年而滅 考異曰續寶運録曰尚讓降徐州黄巢走至碣山路被諸軍趁逼甚乃謂外甥朱彦之云云外甥再三不忍下手黄巢乃自刎過與外甥外甥將至路被沙陀博野奪却兼外甥首級一時送都統軍中舊紀七月癸酉賊將林言斬黄巢黄揆黄秉三人首級降舊傳巢入長安徐帥時溥遣將張友與尚讓之衆掩捕之至狼虎谷巢將林言斬巢及二弟鄴揆等七人首并妻子函送徐州新紀七月壬午黄巢伏誅新傳巢計蹙謂林言曰汝取吾首獻天子可得富貴母為他人利言巢甥也不忍巢乃自刎不殊言因斬之函首將詣時溥而太原博野軍殺言與巢首俱上今從新傳】蔡州節度使秦宗權縱兵四出侵噬鄰道天平節度使朱瑄有衆三萬從父弟瑾勇冠軍中【瑄荀緣翻當作宣瑾渠吝翻冠古玩翻】宣武節度使朱全忠為宗權所攻勢甚窘【窘渠翻】求救於瑄瑄遣瑾將兵救之敗宗權於合鄉【敗補邁翻】全忠德之與瑄約為兄弟【朱全忠反覆小人也兵勢單弱則與朱瑄為兄弟兵勢既彊則反眼為仇敵必誅屠以快其志而後已如斯人可與共功名哉】 秋七月壬午時溥遣使獻黄巢及家人首并姬妾上御大玄樓受之【大玄樓成都羅城正南門樓高駢之築成都羅城既訖功以周易筮之得大畜駢曰畜者養也濟以剛健篤實輝光日新吉孰大焉文宜去下存上因名大玄城】宣問姬妾汝曹皆勲貴子女世受國恩何為從賊其居首者對曰狂賊凶逆國家以百萬之衆失守宗祧播遷巴蜀【祧他彫翻】今陛下以不能拒賊責一女子置公卿將帥於何地乎上不復問皆戮之於市【復扶又翻】人爭與之酒其餘皆悲怖昏醉居首者獨不飲不泣至於就刑神色肅然 【考異曰張耆舊傳中和三年五月二十日北路軍前進到黄巢首級妻男今不取其年月而取其事】 朱全忠撃秦宗權敗宗權於溵水【敗補邁翻】 李克用至晉陽大治甲兵【治直之翻】遣榆次鎮將雁門李承嗣奉表詣行在自陳有破黄巢大功為朱全忠所圖僅能自免將佐已下從行者三百餘人并牌印皆没不返【古者授官賜印綬常佩之於身至解官則解印綬至唐始置職印任其職者傳而用之其印盛之以匣當官者寘之臥内别為一牌使吏掌之以謹出入印出而牌入牌出則印入故謂之牌印】全忠仍牓東都陝孟云臣已死行營兵潰令所在邀遮屠翦勿令漏失將士皆號泣寃訴【號戶刀翻】請復仇讐臣以朝廷至公當俟詔命拊循抑止復歸本道乞遣使按問發兵誅討臣遣弟克勤將萬騎在河中俟命時朝廷以大寇初平方務姑息得克用表大恐但遣中使賜優詔和解之克用前後凡八表稱全忠妬功疾能陰狡禍賊異日必為國患惟乞下詔削其官爵臣自帥本道兵討之不用度支糧餉【唐舊制諸鎮兵出境征討皆仰給度支帥讀曰率】上累遣楊復恭等諭指稱吾深知卿寃方事之殷【杜預曰殷盛也余謂殷衆也言方事之衆多也】姑存大體克用終鬱鬱不平時藩鎮相攻者朝廷不復為之辯曲直【復扶又翻為于偽翻】由是互相吞噬惟力是視皆無所稟畏矣 八月李克用奏請割麟州隸河東【麟州本屬振武節度 考異曰新方鎮表中和二年河東節度增領麟州誤也今從唐末見聞録】又請以弟克修為昭義節度使皆許之由是昭義分為二鎮【澤潞為一鎮邢洺磁為一鎮】進克用爵隴西郡王克用奏罷雲蔚防禦使依舊隸河東【武宗會昌三年分河東雲蔚朔三州置大同軍都團練使次年升為都防禦使】從之 九月己未加朱全忠同平章事 以右僕射大明宫留守王徽知京兆尹事上以長安宫室焚毁故久留蜀未歸徽招撫流散戶口稍歸復繕治宫室百司粗有緒【治直之翻粗坐五翻】冬十月關東藩鎮表請車駕還京師 朱全忠之降也義成節度使王鐸為都統承制除官【事見上卷二年降戶江翻】全忠初鎮大梁事鐸禮甚恭鐸依以為援【汴滑鄰道而鐸於全忠有恩故欲依以為援】而全忠兵寖彊益驕倨鐸知不足恃表請還朝【朝直遙翻】徙鐸為義昌節度使鹿晏弘之去河中王建韓建張造晉暉李師泰各帥其衆與之俱【見上卷本年帥讀曰率】及據興元以建等為巡内刺史不遣之官晏弘猜忌衆心不附王建韓建素相親善晏弘尤忌之數引入臥内【數所角翻】待之加厚二建相謂曰僕射甘言厚意疑我也禍將至矣田令孜密遣人以厚利誘之十一月二建與張造晉暉李師泰帥衆數千逃奔行在【誘音酉 考異曰實録九月山南西道節度使鹿晏弘為禁軍所討弃城奔許州晏弘大將韓建王建張造晉暉李師泰各帥本軍降田令孜以建等楊復光故將薄其賞皆除諸衛將軍十一月戊午朔建等以軍三千至行在田令孜録為假子統以舊軍號隨駕五都按建等既降始遣禁軍討晏弘實録云九月晏弘弃城去太早十一月又云建等降重複上云賞薄下云為假子自相違新傳帝還晏弘懼見討引兵走許州王建帥義勇四軍迎帝西縣按帝尚在成都云迎帝西縣亦誤也今月從實録事從薛居正五代史王建韓建傳】令孜皆養為假子賜與巨萬拜諸衛將軍使各將其衆號隨駕五都【田令孜先已募新軍五十四都分隸兩神策軍今得王建韓建張造晉暉李師泰五將之兵不敢分其衆隸兩軍别號隨駕五都】又遣禁兵討晏弘晏弘弃興元走【鹿晏弘得興元未朞年而弃之】初宦者曹知慤本華原富家子有膽略黄巢陷長安知慤歸鄉里集壯士據嵳峩山南為堡自固【嵳峩山在京兆雲陽縣北十五里】巢黨不敢近【近其靳翻】知慤數遣壯士變衣服語言效巢黨夜入長安攻賊營【數所角翻】賊驚以為鬼神又疑其下有叛者由是心不自安朝廷聞而嘉之就除内常侍賜金紫知慤聞車駕將還謂人曰吾施小術使諸軍得成大功【曹知慤自言賊衆病於己之宵攻已無固志諸鎮大軍臨之因得成收復京城之功】從駕羣臣但平步往來俟至大散關當閱其可歸者納之【從才用翻】行在聞之恐其為變田令孜尤惡之【惡烏路翻】密以勑旨諭邠寜節度使王行瑜使誅之【按光啓二年王行瑜斬朱玫三年始命為邠寜節度使此時蓋為邠寜將也】行瑜潛師自嵯峨山北乘高攻之知慤不為備舉營盡殪令孜益驕横禁制天子不得有所主斷上患其專時語左右而流涕【殪壹計翻横戶孟翻斷丁亂翻語牛倨翻】 鹿晏弘引兵東出襄州秦宗權遣其將秦誥趙德諲將兵會之【諲伊真翻】共攻襄州陷之山南東道節度使劉巨容奔成都【劉巨容不肯追滅黄巢欲養寇以自資自以襄陽為菟裘也而地奪於趙德諲身死於田令孜之手玩寇而邀君果何益哉 考異曰實録光啓元年四月蔡賊攻陷襄州劉巨容死焉新傳晏弘引麾下東出襄鄧宗權遣趙德諲合晏弘兵攻襄州巨容不能守奔成都龍紀元年田令孜殺之按晏弘中和四年十一月已據許州又巨容所以奔成都以天子在蜀故也今從新傳】德諲蔡州人也晏弘引兵轉掠襄鄧均房廬夀復還許州【鹿晏弘自許州從楊復光勤王見二百五十四卷中和元年宋白曰均州漢武當縣地齊永明七年於今鄖鄉縣置齊興郡西魏置興州尋改豐州周武成元年自今鄖鄉城移延岑城今郡理是也隋改均州因均水為名】忠武節度使周岌聞其至弃鎮走晏弘遂據許州【考異曰實録鹿晏弘陷許州殺節度使周岌據其鎮又曰初晏弘據有興元部將王建等帥衆歸行在乃詔禁兵討之晏弘懼弃城歸鄉里周岌聞其至遁去晏弘自稱留後朝廷因以節旄命之始云殺後云遁去自相違今從其後】自稱留後朝廷不能討因以為忠武節度使 十二月己丑陳敬瑄表辭三川都指揮招討制置安撫等使從之【去年以楊師立舉兵敬瑄兼三川都指揮等使師立既死故辭之】 初黄巢轉掠福建【見二百五十三卷乾符五年】建州人陳巖聚衆數千保鄉里號九龍軍福建觀察使鄭鎰奏為團練副使泉州刺史左廂都虞候李連有罪亡入溪洞巖撃敗之【敗補邁翻】鎰畏巖之逼表巖自代壬寅以巖為福建觀察使巖為治有威惠閩人安之【治直吏翻 考異曰實録七月泉州刺史陳巖逐福建觀察使鄭鎰自知使務又曰十二月壬寅以巖為福建觀察使巖既逐鎰逼鎰薦已為代朝廷因命之按巖既逐鎰則鎰不在福州巖安能逼之薦已新王潮傳亦曰黄巢將竊有福州王師不能下建人陳巖帥衆拔之又逐觀察使鄭鎰自領州詔即授刺史按劉恕閩録黄巢陷閩越巖聚衆千餘人號九龍軍福建觀察使鄭鎰奏為團練副史左廂都虞候李連驕慢不法縱其徒為郡人患巖將按誅之連奔溪洞中合衆攻福州巖撃破之鎰表巖自代拜觀察使今從之】 義昌節度使兼中書令王鐸厚於奉養過魏州侍妾成列服御鮮華如承平之態魏博節度使樂彦禎之子從訓伏卒數百於漳南高雞泊圍而殺之及賓僚從者三百餘人皆死掠其資裝侍妾而還【史言王鐸以承平之態處亂世至於喪身亡家誨盜誨淫自取之也從才用翻還從宣翻又如字】彦禎奏云為盜所殺朝廷不能詰 賜邠寜軍號曰靜難【難乃旦翻】 是歲餘杭鎮使陳晟逐睦州刺史柳超潁州都知兵馬使汝陰王敬蕘逐其刺史【汝陰漢縣唐帶潁州蕘如招翻】各領州事朝廷因命為刺史 均州賊帥孫喜聚衆數千人謀攻州城刺史呂曄不知所為都將武當馮行襲伏兵江南【武當漢縣唐帶均州江南漢江之南也帥所類翻將即亮翻】自乘小舟迎喜謂曰州人得良牧無不歸心然公所從之卒太多州人懼於剽掠【剽匹妙翻】尚以為疑不若置軍江北獨與腹心輕騎俱進行襲請為前道【道讀為導一讀如字以請為前道告諭為一句言先路告諭均州之人也請為于偽翻】告諭州人無不服者矣喜以為然從之既度江軍吏迎謁伏兵發行襲手撃喜斬之從喜者皆死【從才用翻】江北軍望之俱潰山南東道節度使上其功【上時掌翻 考異曰薛居正五代史行襲傳曰洋州節度使葛佐奏辟為行軍司馬請將兵鎮谷口通秦蜀道由是益知名新傳曰行襲乘勝逐呂曄據均州劉巨容因表為刺史武定節度使楊守忠表為行軍司馬使領兵搤谷口以通秦蜀新紀光啓元年四月武當賊馮行襲陷均州逐刺史呂曄在劉巨容奔成都後行襲傳云巨容以功上誤也今從薛史按若以薛史為據當言洋州節度使上其功】詔以行襲為均州刺史州西有長山當襄鄧入蜀之道羣盜據之抄掠貢賦【抄楚交翻】行襲討誅之蜀道以通 鳳翔節度使李昌言病表弟昌符知留後昌言薨制以昌符為鳳翔節度使 【考異曰諸書皆無昌言卒年月惟實録於李昌符傳中云李昌言病請昌符權留後昌言死詔除節度使按實録中和三年五月昌言加檢校司徒光啓元年二月昌符始見故以昌言薨附於中和四年之末】 時黄巢雖平秦宗權復熾【復扶又翻】令將出兵寇掠鄰道陳彦侵淮南秦賢侵江南秦誥陷襄唐鄧孫儒陷東都孟陝虢張晊陷汝鄭盧瑭攻汴宋【自孫儒以下事皆在是年之後史槩言之】所至屠翦焚蕩殆無孑遺【晊之日翻孑吉列翻毛萇曰孑遺孑然遺失也按孑單也孤也無孑遺者言無孤單之遺餘也】其殘暴又甚於巢軍行未始轉糧車載鹽尸以從【以死人尸實之以鹽以供軍糧從才用翻】北至衛滑西及關輔東盡青齊南出江淮州鎮存者僅保一城極目千里無復煙火上將還長安畏宗權為患<br />
<br />
  光啓元年【是年三月改元】春正月戊午下詔招撫之 己卯車駕發成都陳敬瑄送至漢州而還 荆南監軍朱敬玫所募忠勇軍暴横陳儒患之鄭紹業之鎮荆南也【廣明元年朱敬玫募忠勇軍鄭紹業鎮荆南亦是年也事並見上卷横戶孟翻】遣大將申屠琮將兵五千擊黄巢於長安軍還儒告琮使除之忠勇將程君從聞之帥其衆奔朗州【奔雷滿也帥讀曰率】琮追擊之殺百餘人自是琮復專軍政【復扶又翻】雷滿屢攻掠荆南儒重賂以却之淮南將張瓌韓師德叛高駢據復岳二州自稱刺史儒請瓌攝行軍司馬師德攝節度副使將兵擊雷滿師德引兵上峽大掠【上時掌翻峽巫峽也】歸於岳州瓌還兵逐儒而代之儒將奔行在瓌刼還囚之【中和二年陳儒代鄭紹業至是而敗】瓌渭州人性貪暴荆南舊將夷滅殆盡先是朱敬玫屢殺大將及富商以致富【先悉薦翻】朝廷遣中使楊玄晦代之敬玫留居荆南嘗曝衣瓌見而欲之遣卒夜攻之殺敬玫盡取其財瓌惡牙將郭禹慓悍【惡烏路翻慓匹妙翻悍下罕翻又戾旰翻】欲殺之禹結黨千人亡去庚申襲歸州據之自稱刺史禹青州人成汭也因殺人亡命更其姓名【禹先為盜詣陳儒降以為將更工衡翻按薛史成汭少年任俠乘醉殺人為讐家所稱因落髮為僧冒姓郭氏】 南康賊帥盧光稠陷䖍州自稱刺史以其里人譚全播為謀主【南康漢南野縣地吳分南野置南安縣晉改為南康唐屬䖍州九域志在州西八十里 考異曰歐陽修五代史曰盧光稠譚全播皆南康人光稠狀貌雄偉無他材能而全播勇敢有識略然全播常奇光稠為人唐末羣盜起全播聚衆立光稠為帥是時王潮攻陷領南全播攻潮取其䖍韶二州十國紀年全播推光稠為之謀主所向克捷光啓初據䖍州光稠自稱刺史天復中陷韶州使其子延昌守之按新紀光啓元年正月光稠陷䖍州天復二年陷韶州歐陽修以為同時取䖍韶二州誤也今從新紀】 秦宗權責租賦於光州刺史王緒緒不能給宗權怒發兵撃之緒懼悉舉光夀兵五千人驅吏民渡江以劉行全為前鋒轉掠江洪䖍州是月陷汀漳二州然皆不能守也【王緒之兵自此入閩為王潮兄弟割據之資】 秦宗權寇潁亳朱全忠敗之於焦夷【焦夷在亳州城父縣界按薛史梁紀龍德元年改亳州焦夷縣為夷父則焦夷時已為縣敗補邁翻夷父當作城父】 二月丙申車駕至鳳翔三月丁卯至京師荆棘滿城狐兔縱横【縱子容翻】上凄然不樂【樂音洛】己巳赦天下改元【改元光啓】時朝廷號令所行惟河西山南劒南嶺南數十州而已 秦宗權稱帝置百官 【考異曰舊宗權傳但云巢賊既誅僭稱帝號實録明年襄王即位宗權已稱帝不從新舊紀皆無之不知宗權以何年月稱帝今因時溥為都統書之】詔以武寜節度使時溥為蔡州四面行營兵馬都統以討之 盧龍節度使李可舉成德節度使王鎔惡李克用之彊【惡烏路翻考異曰太祖紀年録薛居正五代史作王景崇誤也今從舊紀】而義武節度使王處存與克用親善為姪鄴娶克用女【為于偽翻】又河北諸鎮惟義武尚屬朝廷可舉等恐其窺伺山東【此山東謂恒山以東伺相吏翻】終為已患乃相與謀曰易定燕趙之餘也【易州之地本燕南界中山本屬趙國故曰燕趙之餘】約共滅處存而分其地又說雲中節度使赫連鐸使攻克用之背【說式芮翻】可舉遣其將李全忠將兵六萬攻易州鎔遣將將兵攻無極【無極漢古縣因無極山而名唐屬定州九域志在州南九十里】處存告急於克用克用遣其將康君立等將兵救之 閏月秦宗權遣其弟宗言寇荆南 初田令孜在蜀募新軍五十四都每都千人分隸兩神策為十軍以統之又南牙北司官共萬餘員是時藩鎮各專租稅河南北江淮無復上供三司轉運無調發之所度支惟收京畿同華鳳翔等數州租稅不能贍【調徒弔翻度徒洛翻華戶化翻】賞賚不時士卒有怨言令孜患之不知所出先是安邑解縣兩池鹽皆隸鹽鐵置官榷之【宋白曰兩池鹽務舊隸度支其職是諸道巡院貞元十六年史牟以金部郎中主池務遂奏置榷鹽使解戶買翻榷訖岳翻】中和以來河中節度使王重榮專之【天子幸蜀内外百司各失其官守王重榮竊據河中得專鹽池之利】歲獻三千車以供國用令孜奏復如舊制隸鹽鐵夏四月令孜自兼兩池榷鹽使【唐會要元和十五年改河北稅鹽使為榷鹽使其後復失河北止於安邑解縣兩池置榷鹽使】收其利以贍軍重榮上章論訴不已【論盧昆翻說也辯也】遣中使往諭之重榮不可時令孜多遣親信覘藩鎮【覘丑亷翻】有不附已者輒圖之令孜養子匡祐使河中【使疏吏翻】重榮待之甚厚而匡祐傲甚舉軍皆憤怒重榮乃數令孜罪惡【數所具翻】責其無禮監軍為講解【為于偽翻】僅得脫去匡祐歸以告令孜勸圖之五月令孜徙重榮為泰寜節度使以泰寜節度使齊克讓為義武節度使以義武節度使王處存為河中節度使仍詔李克用以河東兵援處存赴鎮【為李克用王重榮連兵犯闕張本】 盧龍兵攻易州禆將劉仁恭穴地入城遂克之仁恭深州人也李克用自將救無極敗成德兵【敗補邁翻】成德兵退保新城克用復進擊大破之【復扶又翻下同】拔新城成德兵走追至九門斬首萬餘級盧龍兵既得易州驕怠王處存夜遣卒三千蒙羊皮造城下【造七到翻】盧龍兵以為羊也爭出掠之處存奮擊大破之復取易州李全忠走 加陝虢節度使王重盈同平章事 李全忠既喪師【喪息浪翻】恐獲罪收餘衆還襲幽州六月李可舉窘急舉族登樓自焚死【乾符二年李茂勲得幽州二世十一年而滅】全忠自為留後 東都留守李罕之與秦宗權將孫儒相拒數月罕之兵少食盡弃城西保澠池宗權陷東都【九域志澠池縣在都城西一百五十六里澠彌兖翻孫儒陷東都而曰宗權者儒宗權將也】秋十月以李全忠為盧龍留後 乙巳右補闕常濬<br />
<br />
  上疏以為陛下姑息藩鎮太甚是非功過駢首並足【言齊是非一功過無所差别也】致天下紛紛若此猶未之寤豈可不念駱谷之艱危復懷西顧之計乎【復扶又翻下同】宜稍振典刑以威四方田令孜之黨言於上曰此疏傳於藩鎮豈不致其猜忿庚戌貶濬萬州司戶尋賜死【宋白曰萬州春秋夔國之地秦漢為朐䏰縣地後魏分朐䏰之地置安鄉及魚泉縣後周置萬州郡兼立南州唐置浦州貞觀初改萬州以舊萬州郡為稱 考異曰實録不言令孜黨為誰按蕭遘等請誅令孜表云韋昭度無致君許國之心多醜正比頑之迹令孜黨蓋謂昭度也續寶運録曰七月三日表入上覽之不悅顧謂侍臣曰藩鎮若見此表深為忿恨自此猜間其何可堪至二十八日勑貶濬為萬州司戶疑三日脫誤當為二十三日今從實録】 滄州軍亂逐節度使楊全玫立牙將盧彦威為留後全玫奔幽州以保鑾都將曹誠為義昌節度使【保鑾神策五十四都之一也】以彦威為德州刺史 孫儒據東都月餘燒宫室官寺民居大掠席卷而去【卷讀曰捲】城中寂無雞犬李罕之復引其衆入東都築壘於市西而居之【城大難守且無居人故築壘以自保聚】 王重榮自以有復京城功【見上卷中和三年】為田令孜所擯不肯之兖州累表論令孜離間君臣【間古莧翻】數令孜十罪【數所具翻】令孜結邠寜節度使朱玫鳳翔節度使李昌符以抗之王處存亦上言幽鎮兵新退臣未敢離易定【幽鎮兵謂李可舉王鎔之兵離力智翻】且王重榮無罪有大功於國不宜輕有改易詔趣其上道【趣讀曰促上時賞翻】八月處存引軍至晉州刺史冀君武閉城不内而還【河中節度使統晉絳慈隰等州君武重榮巡屬冀晉大夫冀芮之後以采邑為姓還從宣翻又如字】 洺州刺史馬爽與昭義行軍司馬奚忠信不叶起兵屯邢州南脅孟方立請誅忠信既而衆潰爽奔魏州忠信使人賂樂彦禎而殺之 秦宗權攻鄰道二十餘州陷之惟陳州距蔡百餘里兵力甚弱刺史趙犨日與宗權戰宗權不能屈詔以犨為蔡州節度使犨德朱全忠之援【自中和三年以來黄巢攻陳州後為秦宗權所攻逼惟倚朱全忠為援】與全忠結昏凡全忠所調發無不立至【調力釣翻奉全忠者趙犨也蹙梁祚者趙犨子孫也】 王緒至漳州以道險糧少【少詩沼翻】令軍中無得以老弱自隨犯者斬惟王潮兄弟扶其母董氏崎嶇從軍【崎丘奇翻嶇音區】緒召潮等責之曰軍皆有法未有無法之軍汝違吾令而不誅是無法也三子曰【王潮兄弟三人從緒】人皆有母未有無母之人將軍奈何使人弃其母緒怒命斬其母三子曰潮等事母如事將軍既殺其母安用其子請先母死【先悉薦翻】將士皆為之請乃捨之【為于偽翻下竊為為之同】有望氣者謂緒曰軍中有王者氣於是緒見將卒有勇略踰已及氣質魁岸者皆殺之劉行全亦死衆皆自危曰行全親也【行全緒妹夫也故云然】且軍鋒之冠猶不免况吾屬乎行至南安【冠古玩翻吳置東安縣晉武帝更名晉安隋改曰南安唐屬泉州九域志南安在州西一十二里】王潮說其前鋒將曰【說式芮翻】吾屬違墳墓捐妻子羈旅外鄉為羣盜【謂弃光夀而入閩也】豈所欲哉乃為緒所迫脅故也今緒猜刻不仁妄殺無辜軍中孑孑者受誅且盡【孑孑特立之貌】子須眉若神騎射絶倫又為前鋒吾竊為子危之【竊為于偽翻】前鋒將執潮手泣問計安出潮為之謀伏壯士數十人於篁竹中伺緒至挺劒大呼躍出【挺劒拔劒也呼火故翻】就馬上擒之反縛以徇軍中皆呼萬歲【中和元年王緒起兵為盜至是為王潮所囚按新書王潮傳縛王緒者即劉行全也與此小異通鑑所書本之路振九國志】潮推前鋒將為主前鋒將曰吾屬今日不為魚肉皆王君力也天以王君為主誰敢先之【先悉薦翻】相推讓數四【推吐雷翻】卒奉潮為將軍【卒子恤翻】緒歎曰此子在吾網中不能殺豈非天哉潮引兵將還光州約其屬所過秋毫無犯行及沙縣【永徽六年分建安置沙縣屬汀州九域志在南劒州西一百二十四里宋白曰沙縣古南平餘跡也晉為延平縣太元四年改為沙戍唐武德初立為沙縣】泉州人張延魯等以刺史廖彦若貪暴【廖力救翻今俗音力弔翻姓也】帥耆老奉牛酒遮道請潮留為州將【帥讀曰率將即亮翻】潮乃引兵圍泉州 九月戊申以陳敬瑄為三川及峽内諸州都指揮制置等使【唐分三川各自為一鎮峽内諸州歸峽屬荆南節度今陳敬瑄皆指揮制置之田令孜右之也】 蔡軍圍荆南【蔡軍秦宗權所遣秦宗言之軍也】馬步使趙匡謀奉前節度使陳儒以出【是年正月張瓌囚陳儒】留後張瓌覺之殺匡及儒 冬十月癸丑秦宗權敗朱全忠于八角【九域志汴州浚儀縣有八角鎮敗補邁翻】 王重榮求救於李克用 【考異曰太祖紀年録曰朱玫李昌符每連衡入覲於天子指陳利害規畫方略不祐太祖黨庇逆温太祖抝怒滋甚時田令孜惡太祖與河中膠固奏云王重榮北引太原其心可見不可處之近輔定州王處存忠孝盡心請授以蒲帥移重榮於定州天子從之重榮憤憤不悦告於太祖曰主上新返正大臣播弃此際無辜遽被斥逐明公當鑑其深心今日使僕安歸會太祖憤怒朱玫輩即報曰當與公提鼔出汜水關誅逆賊之後則去此鼠輩如疾風之去鴻毛耳重榮曰吾地迫邠岐公若東出關二兇必傳吾城下不若先滅一兇去其君側歐陽修五代史重榮使人紿克用曰天子詔重榮俟克用至與處存共誅之因偽為詔書示克用曰此是朱全忠之謀也克用信之按時朝廷疎忌重榮克用亦知之恐無是事今從紀年録】克用方怨朝廷不罪朱全忠【朱全忠攻克用於上源驛朝廷不能治其罪故克用以為怨】選兵市馬聚結諸胡議攻汴州報曰待吾先滅全忠還掃鼠輩如秋葉耳重榮曰待公自關東還吾為虜矣不若先除君側之惡退擒全忠易矣【易以豉翻】時朱玫李昌符亦陰附朱全忠克用乃上言玫昌符與全忠相表裏欲共滅臣臣不得不自救已集蕃漢兵十五萬決以來年濟河自渭北討二鎮不近京城保無驚擾【近其靳翻】既誅二鎮乃旋師滅全忠以雪讐恥上遣使者諭釋【釋解也】冠蓋相望朱玫欲朝廷討克用數遣人潛入京城燒積聚【數所角翻積子賜翻聚從遇翻又皆如子】或刺殺近侍【刺七亦翻】聲云克用所為於是京師震恐日有訛言令孜遣玫昌符將本軍及神策鄜延靈夏等軍各三萬人【刺七亦翻按是時諸鎮分裂如鄜如延以一州為一鎮使掃境出師一鎮亦恐不及三萬人之數田令孜張大言之耳】屯沙苑以討王重榮 【考異曰新令孜傳云令孜自將討重榮帥玫等兵三萬壁沙苑今從實録】重榮發兵拒之告急於李克用克用引兵赴之十一月重榮遣兵攻同州刺史郭璋出戰敗死重榮與玫等相守月餘克用兵至與重榮俱壁沙苑表請誅令孜及玫昌符詔和解之克用不聽十二月癸酉合戰玫昌符大敗 【考異曰新傳曰克用上書請誅令孜玫帝和之不從大戰沙苑王師敗玫走還邠州與昌符皆恥為令孜用還與重榮合神策兵潰克用逼京師令孜計窮乃刼帝夜啓開遠門出奔自賊破長安火宫室廬舍什七後京兆王徽葺復粗完至是令孜唱曰王重榮反命火宫城惟昭陽蓬萊三宫僅存按令孜奉車駕幸近藩避亂其志亦俟兵退復還何為火宫城殆必不然實録六月令孜遣邠岐討重榮九月邠岐始屯沙苑重榮求援於克用十一月克用重榮對壘於沙苑表請誅令孜朱玫十二月重榮合戰朱致敗走太祖紀年録十一月重榮遣使乞師且言二鎮欲加兵於已太祖欲先討朱温重榮請先滅二鎮太祖表言二鎮黨庇朱温請自渭北討之亦不言其附令孜攻河中也又言重榮與邠鳳兵對壘月餘十二月太祖度河與朱玫戰朱玫敗走若自九月至十二月非止月餘矣疑實録遣邠岐討河中及邠岐屯沙苑太近前今並因十二月戰沙苑見之】各走還本鎮【玫還邠州昌符還鳳翔】潰軍所過焚掠克用進逼京城乙亥夜令孜奉天子自開遠門出幸鳳翔【開遠門長安城西面北來第一門】初黄巢焚長安宫室而去諸道兵入城縱掠焚府寺民居什六七王徽累年補葺僅完一二【自中和三年黄巢東走王徽即補葺長安宫室葺七入翻】至是復為亂兵焚掠無孑遺矣【復扶又翻】 是歲賜河中軍號護國<br />
<br />
  二年春正月鎮海牙將張郁作亂攻陷常州 【考異曰皮光業見聞録曰郁潤州小將也周寶差郁押兵士三百人戍於海次因正旦酗酒殺使府安慰軍將度不免禍遂作亂潤州差拓拔從領兵討之郁自常熟縣取江陰而入常州刺史劉革到任方一月親執牌印於戟門而降新紀曰正月辛巳郁陷常州按皮録但言郁以正旦殺安慰軍將耳非當日即陷常州新紀誤也】 李克用還軍河中與王重榮同表請大駕還宫因罪狀田令孜請誅之上復以飛龍使楊復恭為樞密使【田令孜擯斥楊復恭見上卷中和三年】戊子令孜請上幸興元上不從是夜令孜引兵入宫【此宫謂行宫也】刼上幸寶雞黄門衛士從者纔數百人【從才用翻】宰相朝臣皆不知翰林學士承旨杜讓能宿直禁中【天子行幸所至宿次之地宿衛將士外設環衛近臣宿直各有其次與宫禁無異故行宫内亦謂之禁中】聞之步追乘輿【乘繩證翻】出城十餘里得人所遺馬【遺馬弃而不及收者】無羈勒解帶繋頸而乘之獨追及上於寶雞【九域志寶雞縣在鳳翔西南六十五里】明日乃有太子少保孔緯等數人繼至讓能審權之子【杜審權見二百四十九卷宣宗大中十三年】緯戣之孫也【孔戣見憲宗紀】宗正奉太廟神主至鄠【鄠音戶九域志鄠縣在長安南六十里】遇盜皆失之朝士追乘輿者至盩厔【九域志盩厔在鳳翔東南二百里音舟窒】為亂兵所掠衣裝殆盡庚寅上以孔緯為御史大夫使還召百官上留寶雞以待之時田令孜弄權再致播遷【帝始焉避黄巢而奔蜀今又避并蒲之兵而出再致播遷其禍皆本於田令孜弄權】天下共忿疾之朱玫李昌符亦恥為之用且憚李克用王重榮之彊更與之合蕭遘因邠寜奏事判官李松年至鳳翔【唐末藩鎮遣其屬奏事皆謂之奏事官判官幕府右職也朱玫遣之奏事行在所故曰奏事判官以别於尋常奏事官蕭遘為相天子播越而不扈從惡得無罪】遣召朱玫亟迎車駕【朱玫尋有異圖蕭遘既不能制又不能死為法受惡基於此矣】癸巳玫引步騎五千至鳳翔孔緯詣宰相欲宣詔召之蕭遘裴澈以令孜在上側不欲往辭疾不見緯令臺吏趣百官詣行在【趣讀曰促】皆辭以無袍笏緯召三院御史【唐志御史大夫之屬有三院一曰臺院侍御史屬焉二曰殿院殿中侍御史屬焉三曰察院監察御史屬焉】泣謂布衣親舊有急猶當赴之豈有天子蒙塵為人臣子累召而不往者御史請辦裝數日而行緯拂衣起曰吾妻病垂死且不顧諸君善自為謀請從此辭乃詣李昌符請騎衛送至行在昌符義之贈裝錢遣騎送之邠寜鳳翔兵追逼乘輿敗神策指揮使楊晟於潘氏鉦鼔之聲聞於行宫【敗補邁翻聞音問】田令孜奉上發寶雞留禁兵守石鼻為後拒【潘氏在寶雞東北石鼻在寶雞西南亦曰靈壁蘇軾曰鳳翔府寶雞縣武城鎮即俗所謂石鼻寨也諸葛武侯所築城去寶雞三十里】置感義軍於興鳳二州以楊晟為節度使守散關【興州漢武都郡沮縣地自晉及宋魏為武興藩王楊氏之國魏滅楊氏為武興鎮尋改東益州唐為興州今州城即古武興城也鳳州漢武都郡故道河池二縣之地後魏為仇池鎮孝昌中置南岐州廢帝三年改為鳳州以西界有鳳凰山而名】時軍民雜糅鋒鏑縱横【糅女救翻縱子容翻】以神策軍使王建晋暉為清道斬斫使建以長劒五百前驅奮撃乘輿乃得前 【考異曰毛文錫王建紀事云光啓二年正月辛巳車駕次陳倉二月辛亥朱玫遣兵攻逼行在庚申陷虢縣二月甲午將移幸梁洋以上為清道斬斫使戊戌邠師至石鼻己亥石鼻不守庚子寇逼寶雞辛丑車駕南引今但取其事不取其月日】上以傳國寶授建負之以從登大散嶺【從才用翻大散嶺在鳳州梁泉縣松陵堡南】李昌符焚閣道丈餘將摧折【折而設翻】王建扶掖上自煙焰中躍過夜宿板下上枕建膝而寢既覺始進食【枕之酖翻覺居效翻】解御袍賜建曰以其有淚痕故也車駕纔入散關朱玫已圍寶雞石鼻軍潰玫長驅攻散關不克嗣襄王煴肅宗之玄孫也【煴肅宗子襄王僙之曾孫音於云翻又於問翻】有疾從上不及留遵塗驛【據煴傳遵塗驛在石鼻亦謂之石鼻驛】為玫所得與俱還鳳翔庚戌李克用還太原 二月王重榮朱玫李昌符復上表請誅田令孜【復扶又翻】 以前東都留守鄭從讜為守太傅兼侍中 【考異曰新宰相表從讜入三公門不為真相按新傳拜司空復秉政進太傅兼侍中從至興元以太子太保還第新誤也】 朱玫李昌符使山南西道節度使石君涉柵絶險要燒郵驛上由他道以進山谷崎嶇邠軍迫其後【邠軍朱玫之軍】危殆者數四僅得達山南三月壬午石君涉弃鎮逃歸朱玫【石君涉黨於邠岐車駕猝至故弃鎮而逃】癸未鳳翔百官蕭遘等罪狀田令孜及其黨韋昭度請誅之初昭度因供奉僧澈結宦官得為相【昭度為相見二百五十四卷廣明元年】澈師知玄鄙澈所為昭度每與同列詣知玄皆拜之知玄揖使詣澈啜茶山南西道監軍馮翊嚴遵美迎上於西縣【節度使既逃故監軍自迎車駕後魏分漢沔陽縣置嶓冢縣隋大業初改曰西縣唐屬興元府九域志縣在府西一百里宋白曰西縣本名白馬城又曰濜江城宋於此城僑立華陽郡後魏置嶓冢縣隋大業三年改為西縣】丙申車駕至興元 【考異曰皮光業見聞録正月乙酉車駕次寶雞王建紀事正月辛巳次陳倉二月辛亥朱玫將跌師瑀逼行在破楊晟於潘氏庚申陷虢縣三月甲午僖宗將移幸梁洋戊戌邠師至石鼻己亥石鼻不守庚子寇逼寶雞辛丑車駕南引四月庚申達褒中舊紀正月戊子田令孜迫乘輿幸興元庚寅次寶雞癸巳朱玫至鳳翔令孜聞邠軍至奉帝入散關三月丙申車駕至興元唐年補録三月十七日車駕至興元即丙申也寶録正月乙酉車駕次寶雞戊子癸巳三月丙申與舊紀同新紀正月戊子如興元癸巳朱玫叛寇鳳翔三月丙申次興元諸書月日不同如此若依新舊紀實錄則離寶雞六十四日乃至興元似太緩若依紀事則寶雞危逼之地車駕留彼八十日似太久要之僖宗以棧道燒絶自他道崎嶇至山南容有六十日之久至於留寶雞八十日必無此理今從新舊紀】戊戌以御史大夫孔緯翰林學士承旨兵部尚書杜讓能並為兵部侍郎同平章事保鑾都將李鋋等敗邠軍於鳳州【鋋音蟬敗補邁翻】詔加王重榮應接糧料使調本道穀十五萬斛以濟國用【調徒釣翻】重榮表稱令孜未誅不奉詔以尚書左丞盧渥為戶部尚書充山南西道留後以嚴遵美為内樞密使遣王建帥部兵戍三泉【武德四年分利州之綿谷置三泉縣時屬興元府宋白曰三泉縣本漢葭萌縣地後魏正始中分置三泉縣以界内三泉山為名九域志在府西南二百一十里帥讀曰率】晉暉及神策軍使張造帥四都兵屯黑水【從駕五都王建以一都戍三泉暉造以四都屯黑水黑水在興元成固縣西北太白山南流入漢諸葛亮牋所謂朝發南鄭夕宿黑水者也】修棧道以通往來以建遙領壁州刺史將帥遙領州鎮自此始 陳敬瑄疑東川節度使高仁厚欲去之【去羌呂翻下同】遂州刺史鄭君立起兵攻陷漢州進向成都敬瑄遣其將李順之逆戰君立敗死敬瑄又發維茂羌軍撃仁厚殺之 【考異曰張耆舊傳不言仁厚所終惟數敬瑄六錯云太師殺高仁厚一錯又云高僕射權謀智勇累有大功於太師又極忠孝若在王司徒不過梓潼昭宗實録文德元年八月仁厚楊師立羅元杲王師本俱贈官云皆先朝以疑似獲罪今從新紀新傳參以二書自他仁厚事更無所見】 朱玫以田令孜在天子左右終不可去言於蕭遘曰主上播遷六年中原將士冒矢石百姓供饋餉戰死餓死什减七八僅得復京城天下方喜車駕還宫主上更以勤王之功為勑使之榮【勤王之功楊復光實預有之田令孜以其出於北司眩惑人主以為已榮】委以大權使墮綱紀騷擾藩鎮召亂生禍【墮讀曰隳言田令孜易置王重榮以召亂】玫昨奉尊命來迎大駕【言遘召玫使迎車駕】不蒙信察反類脅君吾輩報國之心極矣戰賊之力殫矣安能垂頭弭耳受制於閹寺之手哉李氏孫尚多相公盍改圖以利社稷乎遘曰主上踐阼十餘年無大過惡止以令孜專權肘腋致坐不安席上每言之流涕不已近日上初無行意令孜陳兵帳前迫脅以行不容俟旦罪皆在令孜人誰不知足下盡心王室止有引兵還鎮拜表迎鑾廢立重事伊霍所難遘不敢聞命玫出宣言曰我立李氏一王敢異議者斬夏四月壬子玫逼鳳翔百官奉襄王煴權監軍國事承制封拜指揮仍遣大臣入蜀迎駕盟百官於石鼻驛玫使蕭遘為冊文遘辭以文思荒落【思相史翻】乃使兵部侍郎判戶部鄭昌圖為之乙卯煴受冊玫自兼左右神策十軍使 【考異曰實錄玫自補大丞相按唐無此官又下五月玫自加侍中蓋唐末著小說者謂平章事或侍中為大丞相耳實録因其文而誤也】帥百官奉煴還京師【事至於此蕭遘無所逃於天地之間帥讀曰率】以鄭昌圖同平章事判度支鹽鐵戶部各置副使三司之事一以委焉河中百官崔安潛等上襄王牋賀受冊【上之出長安百官不扈從而奔河中者謂之河中百官】 田令孜自知不為天下所容乃薦樞密使楊復恭為左神策中尉觀軍容使自除西川監軍使 【考異曰舊紀實録皆云二月以令孜為西川監軍舊傳云令孜懼引楊復恭代已從幸梁州求為西川監軍新傳云令孜留不去及帝病乃赴成都表解官求醫蓋取張之說耳按王建紀事四月庚申達褒中令孜以罪舋貫盈且慮禍及於是自授西川監軍使以避指斥復規與敬瑄為巢窟今從之】往依陳敬瑄【為敬瑄令孜併命張本】復恭斥令孜之黨出王建為利州刺史晉暉為集州刺史張造為萬州刺史李師泰為忠州刺史【王建等歸田令孜見上中和四年卜一月】 五月朱玫以中書侍郎同平章事蕭遘為太子太保自加侍中諸道鹽鐵轉運等使加裴澈判度支鄭昌圖判戶部以淮南節度使高駢兼中書令充江淮鹽鐵轉運等使諸道行營兵馬都統淮南右都押牙和州刺史呂用之為嶺南東道節度使大行封拜以悅藩鎮遣吏部侍郎夏侯譚宣諭河北戶部侍郎楊陟宣諭江淮諸藩鎮受其命者什六七高駢仍奉牋勸進【史言僖宗再幸山南天下已絶望矣其得還者幸也】呂用之建牙開幕一與駢同凡駢之腹心及將校能任事者皆逼以從已諸所施為不復咨稟【復扶又翻】駢頗疑之陰欲奪其權而根蔕已固無如之何用之知之甚懼訪於其黨前度支巡官鄭杞前知廬州事董瑾杞曰此固為晚矣【言駢早不知覺】用之問策安出杞曰曹孟德有言寜我負人無人負我【後漢末曹操避董卓之難間行東歸過故人呂伯奢伯奢出五子備賓主禮操聞食器聲以為圖已手劒殺八人而去既而悽愴曰寜我負人無人負我孟德曹操字也鄭杞蓋勸用之圖駢】明日與瑾共為書一緘授用之其語祕人莫有知者【杞瑾謀見下卷光啓三年】蕭遘稱疾歸永樂【按新書遘弟蘧為永樂令遘往從之永樂縣屬河中府武德初置宋白曰永樂縣本漢河北縣地周武帝武成二年改為永樂保定二年省以地屬芮城唐武德二年分芮城復置樂音洛】初鳳翔節度使李昌符與朱玫同謀立襄王既而玫自為宰相專權昌符怒不受其官更通表興元詔加昌符檢校司徒朱玫遣其將王行瑜將邠寜河西兵五萬追乘輿【自代宗時河西没於吐蕃宣宗復河湟張義潮收涼州河西復羈屬於唐】感義節度使楊晟戰數却【數所角翻】弃散關走行瑜進屯鳳州是時諸道貢賦多之長安不之興元【之往也】從官衛士皆乏食【從才用翻】上涕泣不知為計杜讓能言於上曰楊復光與王重榮同破黄巢復京城相親善【事見上卷中和二年】復恭其兄也若遣重臣往諭以大義且致復恭之意宜有回慮歸國之理上從之遣右諫議大夫劉崇望使於河中齎詔諭重榮重榮即聽命遣使表獻絹十萬匹且請討朱玫以自贖戊戌襄王煴遣使者至晉陽賜李克用詔言上至半塗六軍變擾蒼黄晏駕吾為藩鎮所推今已受冊朱玫亦與克用書克用聞其謀皆出於玫大怒大將蓋寓說克用曰鑾輿播遷天下皆歸咎於我【蓋古盍翻說式芮翻寓言上之播越由克用與王重榮兵逼京城為天下之所歸咎】今不誅玫黜李煴無以自湔洗【湔則前翻考異曰實録楊復恭兄弟於李克用王重榮有破賊連衡之舊乃奏遣劉崇望齎詔宣諭兼達復恭之意重】<br />
<br />
  【榮克用皆聽命按後唐太祖紀年録偽使至太原太祖詰其事狀曰皆朱玫所為將斬之以徇大將蓋寓等言云云太祖燔偽詔械其使馳檄喻諸鎮曰今月二十日得襄王偽詔及朱玫文字云田令孜脅遷鑾駕播越梁洋行至半塗六軍變擾遂至蒼黄而晏駕不知弑逆者何人永念丕基不可無主昨四鎮藩后推朕纂承已於正殿受冊畢改元大赦者李煴出自贅疣名汙藩邸智昏菽麥識昧機權李符虜之以塞辭朱玫賣之以為利呂不韋之奇貨可見奸邪蕭世誠之上囊期於匪夕近者當道徑差健步奉表起居行朝見駐巴梁宿衛比無騷動而朱玫脅其孤騃自號台衡敢首亂階明言晏駕熒惑藩鎮凌弱廟朝云云按舊復恭崇望傳及諸家五代史亦不言克用因復恭崇望而推戴僖宗今不取又於時煴未即位改元偽詔亦恐非也編遺録二年春正月壬午唐室有襄王之亂僖宗駐蹕梁洋襄王遂下偽命以檢校太傅令邸吏左環賫所授偽官告一通左環至具事以聞上怒切責環將加其罪久乃赦之遂令焚毁於庭按正月朱玫未立襄王編遺録亦誤也今從薛居正五代史梁紀】克用從之燔詔書囚使者移檄鄰道稱玫敢欺藩方明言晏駕當道已發蕃漢三萬兵進討凶逆當共立大功寓蔚州人也【蔚紆勿翻】 秦賢寇宋汴朱全忠敗之於尉氏南【敗補邁翻】癸巳遣都將郭言將步騎三萬擊蔡州 六月以扈蹕都將楊守亮為金商節度京畿制置使【扈蹕都亦神策五十四都之一】將兵二萬出金州與王重榮李克用共討朱玫守亮本姓訾名亮【訾即移翻漢書功臣表有樓虚侯訾順】曹州人與弟信皆為楊復光假子更名守亮守信【更工衡翻】李克用遣使奉表稱方發兵濟河除逆黨迎車駕願詔諸道與臣協力先是山南之人皆言克用與朱玫合【先悉薦翻】人情恟懼表至上出示從官并諭山南諸鎮由是帖然然克用表猶以朱全忠為言上使楊復恭以書諭之云俟三輔事寜【漢以京兆馮翊扶風為三輔唐京畿之地是也】别有進止 衡州刺史周岳發兵攻潭州欽化節度使閔朂招淮西將黄皓入城共守【淮西將秦宗權將也】皓遂殺朂【中和元年閔勗據潭州至是而敗】岳攻拔州城擒皓殺之鎮海節度使周寶遣牙將丁從實襲常州 【考異曰新紀武寜軍將丁從實陷常州今從皮氏見聞録】逐張郁郁奔海陵【是年正月張郁陷常州】依鎮遏使南昌高霸霸高駢將也鎮海陵有民五萬戶兵三萬人 秋七月秦宗權陷許州殺節度使鹿晏弘【中和四年晏弘許州至是敗亡】據王行瑜進攻興州感義節度使楊晟弃鎮走據文州詔保鑾都將李鋋扈蹕都將李茂貞陳佩屯大唐峰以拒之茂貞博野人本姓宋名文通以功賜姓名【李茂貞始此】 更名欽化軍曰武安【湖南觀察升欽化軍見上卷中和三年更工衡翻】以衡州刺史周岳為節度使八月盧龍節度使李全忠薨以其子匡威為留後 王潮抜泉州殺廖彦若【去年八月王潮圍泉州至是乃拔之 考異曰新紀八月王潮陷泉州刺史劉彦若死之按諸書皆云廖彦若新紀作劉恐誤】潮聞福建觀察陳巖威名不敢犯福州境遣使降之【使疏吏翻降戶江翻】巖表潮為泉州刺史潮沈勇有智略【沈持林翻】既得泉州招懷離散均賦繕兵吏民悅服幽王緒於别館緒慙自殺 九月朱玫將張行實攻大唐峰李鋋等擊却之金吾將軍滿存與邠軍戰破之復取興州【復扶又翻】進守萬仞寨 李克修攻孟方立甲午擒其將呂臻於焦岡拔故鎮武安臨洺邯鄲沙河【九域志洺州武安縣有固鎮鎮】以大將安金俊為邢州刺史 長安百官太子太師裴璩等勸進於襄王煴【璩其於翻】冬十月煴即皇帝位改元建貞遙尊上為太上元皇聖帝 董昌謂錢鏐曰汝能取越州吾以杭州授汝 【考異曰實録辛未以杭州刺史董昌為浙東觀察使按此年十一月鏐始拔越州十二月擒漢宏昌始自稱知浙東軍府事實録誤也】鏐曰然不取終為後患遂將兵自諸暨趨平水【趨七喻翻】鑿山開道五百里出曹娥埭【九域志越州會稽縣有平水鎮曹娥鎮平水今在越州東南四十餘里自此南踰山出小江沿剡溪而東二十里至曹娥埭埭徒耐翻】浙東將鮑君福帥衆降之【帥讀曰率降戶江翻】鏐與浙東軍戰屢破之進屯豐山感化牙將張雄馮弘鐸得罪於節度使時溥【徐州本號武寜軍自咸通罷節鎮之後尋復節鎮改為感化軍中間有書武寜者誤也是後時溥既死朱梁始復徐州為武寜軍】聚衆三百走渡江襲蘇州據之雄自稱刺史稍聚兵至五萬戰艦千餘自號天成軍 河陽節度使諸葛爽薨大將劉經張全義立爽子仲方為留後全義臨濮人也【武德四年分雷澤縣置臨濮縣屬濮州九域志在州南六十里濮博木翻】 李克修攻邢州不克而還 【考異曰太祖紀年録邢人出戰又敗之孟方立求救於鎮州王鎔出兵三萬赴援我軍乃退舊鎔傳是時天子蒙塵九有沸河東李克用虎視山東方謀吞據鎔以重賂結納以修和好晉軍討孟方立於邢州鎔常奉以芻糧据此則鎔助克用攻邢州也未知孰是今皆不取】 十一月丙戌錢鏐克越州劉漢宏奔台州 【考異曰實録漢宏被殺在董昌除浙東前据范坰吳越備史漢宏敗走至十二月死皆有日今從之】 義成節度使安師儒委政於兩廂都虞候夏侯晏杜標二人驕恣軍中忿之小校張驍潛出聚衆二千攻州城師儒斬晏標首諭之軍中稍息天平節度使朱瑄謀取滑州遣濮州刺史朱裕將兵誘張驍殺之朱全忠先遣其將朱珍李唐賓襲滑州入境遇大雪珍等一夕馳至壁下百梯並升遂克之虜師儒以歸 【考異曰實録告於行在命全忠兼領義成節度使按大順元年始以全忠兼宣義節度使全忠猶辭以授胡真此際未也實録誤】全忠以牙將江陵胡真知義成留後【義成自此屬朱全忠】 田令孜至成都請尋醫許之【解西川監軍使】十二月戊寅諸軍拔鳳州以滿存為鳳州防禦使 楊復恭傳檄關中稱得朱玫首者以静難節度使賞之【以朱玫職任授之也難乃旦翻】王行瑜戰數敗【屢為李鋋滿存等所破數所角翻】恐獲罪於玫與其下謀曰今無功歸亦死曷若與汝曹斬玫首迎大駕取邠寜節鉞乎衆從之甲寅行瑜自鳳州擅引兵歸京師【此諸軍所以於戊寅得敗鳳州】玫方視事聞之怒召行瑜責之曰汝擅歸欲反邪行瑜曰吾不反欲誅反者朱玫耳遂擒斬之并殺其黨數百人諸軍大亂焚掠京城士民無衣凍死者蔽地裴澈鄭昌圖帥百官二百餘人奉襄王奔河中【帥讀曰率】王重榮詐為迎奉執煴殺之【襄王煴自監國至竊號涉八月而敗】囚澈昌圖百官死者殆半 台州刺史杜雄誘劉漢宏執送董昌斬之【廣明元年劉漢宏得浙東至是而亡 考異曰十國紀年十一月丙午杜雄執漢宏按十二月丙子朔無丙午紀年誤】昌圖鎮越州自稱知浙東軍府事以錢鏐知杭州事【為錢鏐以杭州跨有二浙張本】 王重榮函襄王煴首至行在刑部請御興元城南樓獻馘百官畢賀太常博士殷盈孫議以為煴為賊臣所逼止以不能死節為罪耳禮公族罪在大辟君為之素服不舉【記文王世子公族其有死罪者有司讞于公曰某之罪在大辟公三宥之有司不對走出致刑于甸人公又使人追之曰雖然必赦之對曰無及也反命于公公素服不舉為之變如其倫之喪無服新哭之為于偽翻】今煴已就誅宜廢為庶人令所在葬其首其獻馘稱賀之禮請俟朱玫首至而行之從之盈孫侑之孫也【殷侑見二百四十二卷文宗大和二年】 河陽大將劉經畏李罕之難制自引兵鎮洛陽襲罕之於澠池為罕之所敗【敗補邁翻下同】經弃洛陽走罕之追殺殆盡罕之軍于鞏【鞏漢古縣唐屬河南府九域志在府東一百一十里】將渡河經遣張全義將兵拒之時諸葛仲方幼弱政在劉經諸將多不附全義遂與罕之合兵攻河陽為經所敗罕之全義走保懷州 初忠武決勝指揮使孫儒與龍驤指揮使朗山劉建鋒戍蔡州拒黄巢扶溝馬殷隸軍中以材勇聞【扶溝漢縣中廢隋復置唐屬許州陳留風俗傳小扶亭有洧水之溝因以名縣九域志縣在汴州南一百九十里馬殷始此】及秦宗權叛儒等皆屬焉宗權遣儒攻陷鄭州刺史李璠奔大梁【璠孚袁翻】儒進陷河陽留後諸葛仲方奔大梁【廣明元年諸葛爽得河陽及子而敗】儒自稱節度使張全義據懷州李罕之據澤州以拒之初長安人張佶為宣州幕僚惡觀察使秦彦之為人弃官去過蔡州宗權留以為行軍司馬佶謂劉建鋒曰秦公剛鷙而猜忌亡無日矣吾屬何以自免建鋒方自危遂與佶善【佶其吉翻惡烏路翻為劉建鋒張佶協力取湖南張本】 夀州刺史張翺 【考異曰妖亂志作張敖吳録作張滶今從十國紀年】遣其將魏䖍將萬人寇廬州廬州刺史楊行愍遣其將田頵李神福張訓拒之敗䖍於禇城【敗補邁翻】滁州刺史許勍襲舒州刺史陶雅奔廬州【中和四年行愍使雅取舒州】高駢命行愍更名行密【更工衡翻】 是歲天平牙將朱瑾逐泰寜節度使齊克讓 【考異曰薛居正五代史云虜克讓今從舊傳】自稱留後瑾將襲兖州求昏於克讓乃自鄆盛飾車服私藏兵甲以赴之親迎之夕甲士竊發逐克讓而代之【迎魚敬翻】朝廷因以瑾為泰寜節度使 安陸賊帥周通攻鄂州路審中亡去【中和四年路審中據鄂州帥所類翻下同】岳州刺史杜洪乘虚入鄂自稱武昌留後朝廷因以授之湘陰賊帥鄧進思復乘虚陷岳州【湘陰漢羅縣地宋分置湘陰縣唐武德八年省羅縣入焉屬岳州九域志在州西南二百七十里復扶又翻】 秦宗言圍荆南二年【去年九月圍荆南】張瓌嬰城自守城中米斗直錢四十緍食甲鼓皆盡擊門扉以警夜死者相枕【枕職任翻】宗言竟不能克而去<br />
<br />
  三年春正月以邠州都將王行瑜為静難軍節度使【以朱玫之官賞之難乃旦翻】扈蹕都頭李茂貞領武定節度使【據舊紀以洋州為武定節鎮】扈蹕都頭楊守宗為金商節度使右衛大將軍顧彦朗為東川節度使金商節度使楊守亮為山南西道節度使彦朗豐縣人也 辛巳以董昌為浙東觀察使錢鏐為杭州刺史 秦宗權自以兵力十倍於朱全忠而數為所敗恥之【數所角翻敗補邁翻】欲悉力以攻汴州全忠患兵少二月以諸軍都指揮使朱珍為淄州刺史募兵於東道【淄州本平廬巡屬全忠欲募兵於東方輒以刺史授珍】期以初夏而還【薛居正五代史曰使朱珍募兵於東道懼蔡人暴其麥期以夏首迴歸】 戊辰削奪三川都監田令孜官爵長流端州然令孜依陳敬瑄竟不行【考異曰實録載勑曰令孜雖已削奪在身官爵宜剝服色配端州長流百姓新傳曰削官爵流儋州然猶依敬瑄不行張耆舊傳曰大駕廣明二年春孟到蜀叟嘗接識北司諸官子弟有光啓承旨似先大夫為叟言去年黄巢凌犯聖上蒼忙就路諸王多是徒行夀王至斜谷行不得襪一足跣一足偃卧磻石上田軍容在後收拾驅夀王夀王起告軍容行不得與箇馬騎軍容云山谷間何處得馬以鞭一抶之令行雖迴首無言衷心深銜此恨爾後經今八年僖宗皇帝在行宫寢疾月餘彌留臣下皆知不起于疾内外屬望在於夀王夀王仁孝大度弘寛有斷衆所歸心軍容聞大恐就御寢問識臣否帝目瞪不語軍容大驚尋時矯制除西川監軍使仍馳驛赴任遂將拱宸奉鑾兩都自衛星夜倍程軍容才到西川僖宗已崩國朝果夀王登極皇帝位於是積年怨恨今日逞其志矣新令孜傳取之据實録令孜光啓二年為西川監軍此月流端州在昭宗即位前自為楊復恭所擯耳十國紀年曰三月僖宗東還詔流令孜儋州敬瑄端州皆拒朝命此据張耆舊傳致誤耳今從實録】 代北節度使李國昌薨 【考異曰薛居正五代史武皇紀國昌中和三年薨唐末見聞録中和三年十月老司徒薨舊書中和三年十月國昌卒後唐獻祖紀年録光啓中薨於位新沙陀傳光啓三年國昌卒太祖紀年録光啓三年正月云是歲獻祖文皇帝之喪太祖哀毁行服不獲專征實録置此年二月今從之】 三月癸未詔偽宰相蕭遘鄭昌圖裴澈於所在集衆斬之皆死於岐山【岐山在鳳翔東四十里按舊書帝紀河中械送偽宰相裴澈鄭昌圖命斬之於岐山縣太子少師致仕蕭遘賜死於永樂縣與此不同】時朝士受煴官者甚衆法司皆處以極法【法司謂刑部處昌呂翻】杜讓能力爭之免者什七八 壬辰車駕至鳳翔節度使李昌符恐車駕還京雖不治前過【前過謂與朱玫迫逐乘輿也治直之翻】恩賞必疎乃以宫室未完固請駐蹕府舍從之 太傅兼侍中鄭從讜罷為太子太保 鎮海節度使周寶募親軍千人號後樓兵稟給倍於鎮海軍鎮海軍皆怨而後樓兵浸驕不可制寶溺於聲色不親政事築羅城二十餘里建東第人苦其役寶與僚屬宴後樓有言鎮海軍怨望者寶曰亂則殺之度支催勘使薛朗以其言告所善鎮海軍將劉浩戒之使戢士卒浩曰惟反可以免死耳是夕寶醉方寢浩帥其黨作亂【帥讀曰卒下同】攻府舍而焚之寶驚起徒跣叩芙蓉門呼後樓兵後樓兵亦反矣寶帥家人步走出青陽門遂奔常州 【考異曰實録寶被逐在四月恐四月奏到耳吳越備史三月壬辰新紀癸巳今從之】依刺史丁從實浩殺諸僚佐癸巳迎薛朗入府推為留後【為錢鏐誅薛朗張本】寶先兼租庸副使城中貨財山積是日盡於亂兵之手高駢聞寶敗列牙受賀遣使饋以韲粉【駢與寶為仇故幸其敗為仇事見二百五十四卷中和元年韲子西翻細切為韲碎䃺為粉】寶怒擲之地曰汝有呂用之在他日未可知也揚州連歲饑城中餒死者日數千人坊市為之寥落災異數見駢悉以為周寶當之【史言高駢阽於死亡而不悟為于偽翻數所角翻見賢遍翻】 山南西道節度使楊守亮忌利州刺史王建驍勇屢召之建懼不往【利州山南西道巡屬也建懼為守亮所殺故不敢往】前龍州司倉周庠【路振九國志作周博雅】說建曰唐祚將終藩鎮互相吞噬皆無雄才遠略不能戡濟多難【說式芮翻下同難乃旦翻】公勇而有謀得士卒心立大功者非公而誰然葭萌四戰之地【利州古葭萌之地世傳古蜀王封其弟葭萌於此因以名邑】難以久安閬州地僻人富楊茂實陳田之腹心不修職貢若表其罪興兵討之可不戰而擒也建從之召募溪洞酋豪有衆八千沿嘉陵江而下襲閬州【西漢水出秦州嘉陵谷亦謂之嘉陵水東南過葭萌又東南過閬中閬州東州巡屬酋慈由翻】逐其刺史楊茂實而據之自稱防禦使招納亡命軍勢益盛守亮不能制部將張䖍裕說建曰公乘天子微弱專據方州若唐室復興公無種矣【種章勇翻】宜遣使奉表天子杖大義以行師蔑不濟矣部將綦毋諫復說建養士愛民以觀天下之變【綦母姓也毋音無】建從之庠䖍裕諫皆許州人也【汝穎多奇士自古然也史言英雄角逐天必生人才以羽翼之】初建與東川節度使顧彦朗俱在神策軍同討賊建既據閬州彦朗畏其侵暴數遣使問遺【數所角翻遺惟季翻】饋以軍食建由是不犯東川【豺狼不噬要非仁也力未及耳觀後顧彦暉之事可見已】 初周寶聞淮南六合鎮遏使徐約兵精誘之使撃蘇州【為下卷徐約逐張雄始事】<br />
<br />
  資治通鑑卷二百五十六  <br>
   </div> 

<script src="/search/ajaxskft.js"> </script>
 <div class="clear"></div>
<br>
<br>
 <!-- a.d-->

 <!--
<div class="info_share">
</div> 
-->
 <!--info_share--></div>   <!-- end info_content-->
  </div> <!-- end l-->

<div class="r">   <!--r-->



<div class="sidebar"  style="margin-bottom:2px;">

 
<div class="sidebar_title">工具类大全</div>
<div class="sidebar_info">
<strong><a href="http://www.guoxuedashi.com/lsditu/" target="_blank">历史地图</a></strong>  
<a href="http://www.880114.com/" target="_blank">英语宝典</a>  
<a href="http://www.guoxuedashi.com/13jing/" target="_blank">十三经检索</a> 
<br><strong><a href="http://www.guoxuedashi.com/gjtsjc/" target="_blank">古今图书集成</a></strong> 
<a href="http://www.guoxuedashi.com/duilian/" target="_blank">对联大全</a> <strong><a href="http://www.guoxuedashi.com/xiangxingzi/" target="_blank">象形文字典</a></strong> 

<br><a href="http://www.guoxuedashi.com/zixing/yanbian/">字形演变</a>  <strong><a href="http://www.guoxuemi.com/hafo/" target="_blank">哈佛燕京中文善本特藏</a></strong>
<br><strong><a href="http://www.guoxuedashi.com/csfz/" target="_blank">丛书&方志检索器</a></strong> <a href="http://www.guoxuedashi.com/yqjyy/" target="_blank">一切经音义</a>  

<br><strong><a href="http://www.guoxuedashi.com/jiapu/" target="_blank">家谱族谱查询</a></strong>  <strong><a href="http://shufa.guoxuedashi.com/sfzitie/" target="_blank">书法字帖欣赏</a></strong> 
<br>

</div>
</div>


<div class="sidebar" style="margin-bottom:0px;">

<font style="font-size:22px;line-height:32px">QQ交流群9:489193090</font>


<div class="sidebar_title">手机APP 扫描或点击</div>
<div class="sidebar_info">
<table>
<tr>
	<td width=160><a href="http://m.guoxuedashi.com/app/" target="_blank"><img src="/img/gxds-sj.png" width="140"  border="0" alt="国学大师手机版"></a></td>
	<td>
<a href="http://www.guoxuedashi.com/download/" target="_blank">app软件下载专区</a><br>
<a href="http://www.guoxuedashi.com/download/gxds.php" target="_blank">《国学大师》下载</a><br>
<a href="http://www.guoxuedashi.com/download/kxzd.php" target="_blank">《汉字宝典》下载</a><br>
<a href="http://www.guoxuedashi.com/download/scqbd.php" target="_blank">《诗词曲宝典》下载</a><br>
<a href="http://www.guoxuedashi.com/SiKuQuanShu/skqs.php" target="_blank">《四库全书》下载</a><br>
</td>
</tr>
</table>

</div>
</div>


<div class="sidebar2">
<center>


</center>
</div>

<div class="sidebar"  style="margin-bottom:2px;">
<div class="sidebar_title">网站使用教程</div>
<div class="sidebar_info">
<a href="http://www.guoxuedashi.com/help/gjsearch.php" target="_blank">如何在国学大师网下载古籍?</a><br>
<a href="http://www.guoxuedashi.com/zidian/bujian/bjjc.php" target="_blank">如何使用部件查字法快速查字?</a><br>
<a href="http://www.guoxuedashi.com/search/sjc.php" target="_blank">如何在指定的书籍中全文检索?</a><br>
<a href="http://www.guoxuedashi.com/search/skjc.php" target="_blank">如何找到一句话在《四库全书》哪一页?</a><br>
</div>
</div>


<div class="sidebar">
<div class="sidebar_title">热门书籍</div>
<div class="sidebar_info">
<a href="/so.php?sokey=%E8%B5%84%E6%B2%BB%E9%80%9A%E9%89%B4&kt=1">资治通鉴</a> <a href="/24shi/"><strong>二十四史</strong></a>&nbsp; <a href="/a2694/">野史</a>&nbsp; <a href="/SiKuQuanShu/"><strong>四库全书</strong></a>&nbsp;<a href="http://www.guoxuedashi.com/SiKuQuanShu/fanti/">繁体</a>
<br><a href="/so.php?sokey=%E7%BA%A2%E6%A5%BC%E6%A2%A6&kt=1">红楼梦</a> <a href="/a/1858x/">三国演义</a> <a href="/a/1038k/">水浒传</a> <a href="/a/1046t/">西游记</a> <a href="/a/1914o/">封神演义</a>
<br>
<a href="http://www.guoxuedashi.com/so.php?sokeygx=%E4%B8%87%E6%9C%89%E6%96%87%E5%BA%93&submit=&kt=1">万有文库</a> <a href="/a/780t/">古文观止</a> <a href="/a/1024l/">文心雕龙</a> <a href="/a/1704n/">全唐诗</a> <a href="/a/1705h/">全宋词</a>
<br><a href="http://www.guoxuedashi.com/so.php?sokeygx=%E7%99%BE%E8%A1%B2%E6%9C%AC%E4%BA%8C%E5%8D%81%E5%9B%9B%E5%8F%B2&submit=&kt=1"><strong>百衲本二十四史</strong></a>  <a href="http://www.guoxuedashi.com/so.php?sokeygx=%E5%8F%A4%E4%BB%8A%E5%9B%BE%E4%B9%A6%E9%9B%86%E6%88%90&submit=&kt=1"><strong>古今图书集成</strong></a>
<br>

<a href="http://www.guoxuedashi.com/so.php?sokeygx=%E4%B8%9B%E4%B9%A6%E9%9B%86%E6%88%90&submit=&kt=1">丛书集成</a> 
<a href="http://www.guoxuedashi.com/so.php?sokeygx=%E5%9B%9B%E9%83%A8%E4%B8%9B%E5%88%8A&submit=&kt=1"><strong>四部丛刊</strong></a>  
<a href="http://www.guoxuedashi.com/so.php?sokeygx=%E8%AF%B4%E6%96%87%E8%A7%A3%E5%AD%97&submit=&kt=1">說文解字</a> <a href="http://www.guoxuedashi.com/so.php?sokeygx=%E5%85%A8%E4%B8%8A%E5%8F%A4&submit=&kt=1">三国六朝文</a>
<br><a href="http://www.guoxuedashi.com/so.php?sokeytm=%E6%97%A5%E6%9C%AC%E5%86%85%E9%98%81%E6%96%87%E5%BA%93&submit=&kt=1"><strong>日本内阁文库</strong></a> <a href="http://www.guoxuedashi.com/so.php?sokeytm=%E5%9B%BD%E5%9B%BE%E6%96%B9%E5%BF%97%E5%90%88%E9%9B%86&ka=100&submit=">国图方志合集</a> <a href="http://www.guoxuedashi.com/so.php?sokeytm=%E5%90%84%E5%9C%B0%E6%96%B9%E5%BF%97&submit=&kt=1"><strong>各地方志</strong></a>

</div>
</div>


<div class="sidebar2">
<center>

</center>
</div>
<div class="sidebar greenbar">
<div class="sidebar_title green">四库全书</div>
<div class="sidebar_info">

《四库全书》是中国古代最大的丛书,编撰于乾隆年间,由纪昀等360多位高官、学者编撰,3800多人抄写,费时十三年编成。丛书分经、史、子、集四部,故名四库。共有3500多种书,7.9万卷,3.6万册,约8亿字,基本上囊括了古代所有图书,故称“全书”。<a href="http://www.guoxuedashi.com/SiKuQuanShu/">详细>>
</a>

</div> 
</div>

</div>  <!--end r-->

</div>
<!-- 内容区END --> 

<!-- 页脚开始 -->
<div class="shh">

</div>

<div class="w1180" style="margin-top:8px;">
<center><script src="http://www.guoxuedashi.com/img/plus.php?id=3"></script></center>
</div>
<div class="w1180 foot">
<a href="/b/thanks.php">特别致谢</a> | <a href="javascript:window.external.AddFavorite(document.location.href,document.title);">收藏本站</a> | <a href="#">欢迎投稿</a> | <a href="http://www.guoxuedashi.com/forum/">意见建议</a> | <a href="http://www.guoxuemi.com/">国学迷</a> | <a href="http://www.shuowen.net/">说文网</a><script language="javascript" type="text/javascript" src="https://js.users.51.la/17753172.js"></script><br />
  Copyright &copy; 国学大师 古典图书集成 All Rights Reserved.<br>
  
  <span style="font-size:14px">免责声明:本站非营利性站点,以方便网友为主,仅供学习研究。<br>内容由热心网友提供和网上收集,不保留版权。若侵犯了您的权益,来信即刪。scp168@qq.com</span>
  <br />
ICP证:<a href="http://www.beian.miit.gov.cn/" target="_blank">鲁ICP备19060063号</a></div>
<!-- 页脚END --> 
<script src="http://www.guoxuedashi.com/img/plus.php?id=22"></script>
<script src="http://www.guoxuedashi.com/img/tongji.js"></script>

</body>
</html>
