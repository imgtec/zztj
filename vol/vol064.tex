<!DOCTYPE html PUBLIC "-//W3C//DTD XHTML 1.0 Transitional//EN" "http://www.w3.org/TR/xhtml1/DTD/xhtml1-transitional.dtd">
<html xmlns="http://www.w3.org/1999/xhtml">
<head>
<meta http-equiv="Content-Type" content="text/html; charset=utf-8" />
<meta http-equiv="X-UA-Compatible" content="IE=Edge,chrome=1">
<title>資治通鑒_65-資治通鑑卷六十四_65-資治通鑑卷六十四</title>
<meta name="Keywords" content="資治通鑒_65-資治通鑑卷六十四_65-資治通鑑卷六十四">
<meta name="Description" content="資治通鑒_65-資治通鑑卷六十四_65-資治通鑑卷六十四">
<meta http-equiv="Cache-Control" content="no-transform" />
<meta http-equiv="Cache-Control" content="no-siteapp" />
<link href="/img/style.css" rel="stylesheet" type="text/css" />
<script src="/img/m.js?2020"></script> 
</head>
<body>
 <div class="ClassNavi">
<a  href="/24shi/">二十四史</a> | <a href="/SiKuQuanShu/">四库全书</a> | <a href="http://www.guoxuedashi.com/gjtsjc/"><font  color="#FF0000">古今图书集成</font></a> | <a href="/renwu/">历史人物</a> | <a href="/ShuoWenJieZi/"><font  color="#FF0000">说文解字</a></font> | <a href="/chengyu/">成语词典</a> | <a  target="_blank"  href="http://www.guoxuedashi.com/jgwhj/"><font  color="#FF0000">甲骨文合集</font></a> | <a href="/yzjwjc/"><font  color="#FF0000">殷周金文集成</font></a> | <a href="/xiangxingzi/"><font color="#0000FF">象形字典</font></a> | <a href="/13jing/"><font  color="#FF0000">十三经索引</font></a> | <a href="/zixing/"><font  color="#FF0000">字体转换器</font></a> | <a href="/zidian/xz/"><font color="#0000FF">篆书识别</font></a> | <a href="/jinfanyi/">近义反义词</a> | <a href="/duilian/">对联大全</a> | <a href="/jiapu/"><font  color="#0000FF">家谱族谱查询</font></a> | <a href="http://www.guoxuemi.com/hafo/" target="_blank" ><font color="#FF0000">哈佛古籍</font></a> 
</div>

 <!-- 头部导航开始 -->
<div class="w1180 head clearfix">
  <div class="head_logo l"><a title="国学大师官网" href="http://www.guoxuedashi.com" target="_blank"></a></div>
  <div class="head_sr l">
  <div id="head1">
  
  <a href="http://www.guoxuedashi.com/zidian/bujian/" target="_blank" ><img src="http://www.guoxuedashi.com/img/top1.gif" width="88" height="60" border="0" title="部件查字,支持20万汉字"></a>


<a href="http://www.guoxuedashi.com/help/yingpan.php" target="_blank"><img src="http://www.guoxuedashi.com/img/top230.gif" width="600" height="62" border="0" ></a>


  </div>
  <div id="head3"><a href="javascript:" onClick="javascript:window.external.AddFavorite(window.location.href,document.title);">添加收藏</a>
  <br><a href="/help/setie.php">搜索引擎</a>
  <br><a href="/help/zanzhu.php">赞助本站</a></div>
  <div id="head2">
 <a href="http://www.guoxuemi.com/" target="_blank"><img src="http://www.guoxuedashi.com/img/guoxuemi.gif" width="95" height="62" border="0" style="margin-left:2px;" title="国学迷"></a>
  

  </div>
</div>
  <div class="clear"></div>
  <div class="head_nav">
  <p><a href="/">首页</a> | <a href="/ShuKu/">国学书库</a> | <a href="/guji/">影印古籍</a> | <a href="/shici/">诗词宝典</a> | <a   href="/SiKuQuanShu/gxjx.php">精选</a> <b>|</b> <a href="/zidian/">汉语字典</a> | <a href="/hydcd/">汉语词典</a> | <a href="http://www.guoxuedashi.com/zidian/bujian/"><font  color="#CC0066">部件查字</font></a> | <a href="http://www.sfds.cn/"><font  color="#CC0066">书法大师</font></a> | <a href="/jgwhj/">甲骨文</a> <b>|</b> <a href="/b/4/"><font  color="#CC0066">解密</font></a> | <a href="/renwu/">历史人物</a> | <a href="/diangu/">历史典故</a> | <a href="/xingshi/">姓氏</a> | <a href="/minzu/">民族</a> <b>|</b> <a href="/mz/"><font  color="#CC0066">世界名著</font></a> | <a href="/download/">软件下载</a>
</p>
<p><a href="/b/"><font  color="#CC0066">历史</font></a> | <a href="http://skqs.guoxuedashi.com/" target="_blank">四库全书</a> |  <a href="http://www.guoxuedashi.com/search/" target="_blank"><font  color="#CC0066">全文检索</font></a> | <a href="http://www.guoxuedashi.com/shumu/">古籍书目</a> | <a   href="/24shi/">正史</a> <b>|</b> <a href="/chengyu/">成语词典</a> | <a href="/kangxi/" title="康熙字典">康熙字典</a> | <a href="/ShuoWenJieZi/">说文解字</a> | <a href="/zixing/yanbian/">字形演变</a> | <a href="/yzjwjc/">金 文</a> <b>|</b>  <a href="/shijian/nian-hao/">年号</a> | <a href="/diming/">历史地名</a> | <a href="/shijian/">历史事件</a> | <a href="/guanzhi/">官职</a> | <a href="/lishi/">知识</a> <b>|</b> <a href="/zhongyi/">中医中药</a> | <a href="http://www.guoxuedashi.com/forum/">留言反馈</a>
</p>
  </div>
</div>
<!-- 头部导航END --> 
<!-- 内容区开始 --> 
<div class="w1180 clearfix">
  <div class="info l">
   
<div class="clearfix" style="background:#f5faff;">
<script src='http://www.guoxuedashi.com/img/headersou.js'></script>

</div>
  <div class="info_tree"><a href="http://www.guoxuedashi.com">首页</a> > <a href="/SiKuQuanShu/fanti/">四库全书</a>
 > <h1>资治通鉴</h1> <!--         下载:【右键另存为】即可 --></div>
  <div class="info_content zj clearfix">
  
<div class="info_txt clearfix" id="show">
<center style="font-size:24px;">65-資治通鑑卷六十四</center>
    資治通鑑卷六十四   宋 司馬光 撰<br />
<br />
  胡三省 音註<br />
<br />
  漢紀五十六【起重光大荒落盡旃蒙作噩凡五年】<br />
<br />
  孝獻皇帝已<br />
<br />
  建安六年春三月丁卯朔日有食之 曹操就穀於安民【據水經東平夀張縣西界有安民亭亭在濟水東亭北對安民山洪氏隸釋曰濟水逕須句城西水西有安民山趙明誠金石録曰按地理志須句城即今中都縣】以袁紹新破欲以其間擊劉表【間古莧翻】荀彧曰紹既新敗其衆離心宜乘其困遂定之而欲遠師江漢若紹收其餘燼乘虛以出人後則公事去矣操乃止夏四月操揚兵河上擊袁紹倉亭軍破之【紹盖遣軍屯倉亭津】秋九月操還許 操自擊劉備於汝南備奔劉表龔都等皆散【備合龔都事見上卷上年】表聞備至自出郊迎以上賓禮待之益其兵使屯新野【水經注新野縣在安衆縣東南】備在荆州數年嘗於表坐起至厠慨然流涕表怪問備備曰平常身不離鞍【坐徂臥翻離力智翻】髀肉皆消今不復騎髀裏肉生日月如流老將至矣而功業不建是以悲耳【史言備志氣不衰所以能成三分之業復扶又翻】 曹操遣夏侯淵張遼圍昌豨於東海【豨叛操事見上卷三年豨許豈翻又音希】數月糧盡議引軍還遼謂淵曰數日以來每行諸圍豨輒屬目視遼【行下孟翻屬之欲翻】又其射矢更希此必豨計猶豫故不力戰遼欲挑與語【射而亦翻挑徒了翻】儻可誘也【儻或然之辭誘音酉】乃使謂豨曰公有命使遼傳之豨果下與遼語遼為說操神武方以德懷四方先附者受大賞【為于偽翻】豨乃許降【降戶江翻】遼遂單身上三公山【上時掌翻】入豨家拜妻子豨歡喜隨遼詣操操遣豨還 趙韙圍劉璋於成都東州人恐見誅滅相與力戰韙遂敗追至江州【賢曰江州縣屬巴郡今渝州巴縣】殺之【趙韙隨劉焉入蜀將以圖富貴而卒以殺身行險以徼幸不如居易以俟命也】龎羲懼遣吏程祁宣旨於其父漢昌令畿【漢昌縣屬巴郡漢末分宕渠置】索賨兵【索山客翻賨祖宗翻】畿曰郡合部曲本不為亂縱有讒諛要在盡誠若遂懷異志不敢聞命羲更使祁說之畿曰我受牧恩當為盡節【說輸芮翻為于偽翻】汝為郡吏自宜效力【謂父子當各盡節於所事也】不義之事有死不為羲怒使人謂畿曰不從太守禍將及家畿曰樂羊食子非無父子之恩大義然也【樂羊注見四十三卷光武建武十二年】今雖羮祁以賜畿畿啜之矣羲乃厚謝於璋璋擢畿為江陽太守【劉璋分犍為為江陽郡宋白曰瀘州之瀘川江安縣本江陽地】朝廷聞益州亂以五官中郎將牛亶為益州刺史徵璋為卿不至【卿九卿也】張魯以鬼道教民使病者自首其過【首拭救翻】為之請禱實無益於治病然小人昏愚競共事之犯法者三原然後乃行刑【治直之翻原赦也】不置長吏皆以祭酒為治【魯以鬼道教民其來學者初名為鬼卒後號祭酒祭酒各領部衆長知兩翻治直吏翻】民夷便樂之【樂音洛】流移寄在其地者不敢不奉其道後遂襲取巴郡朝廷力不能征遂就寵魯為鎮民中郎將領漢寧太守【袁山松書曰建安二十年分漢中之安陽置漢寧郡】通貢獻而已民有地中得玉印者羣下欲尊魯為漢寧王功曹巴西閻圃諫曰【譙周巴記曰初平六年趙韙分巴為二郡欲得巴舊名以墊江為治安漢以下為永寧郡建安六年劉璋分巴以永寧為巴東郡墊江為巴郡閬中為巴西郡】漢川之民戶出十萬財富土沃四面險固上匡天子則為桓文次及竇融不失富貴今承制署置埶足斬斷【斷丁亂翻】不煩於王願且不稱勿為禍先魯從之七年春正月曹操軍譙【譙縣屬沛國操之鄉里】遂至浚儀治睢陽渠【浚儀縣屬陳留郡睢水於此縣首受莨蕩渠水東過睢陽縣故謂之睢陽渠睢音雖治直之翻】遣使以太牢祀橋玄【玄識操於微時故祀之】進軍官渡 袁紹自軍敗慙憤發病嘔血夏五月薨初紹有三子譚熙尚紹後妻劉氏愛尚數稱於紹【數所角翻】紹欲以為後而未顯言之乃以譚繼兄後【紹本司空逢之孽子出後伯父成成盖先有子死而紹後之紹欲廢譚立尚故以譚繼兄後】出為青州刺史沮授諫曰【沮子余翻】世稱萬人逐兎一人獲之貪者悉止分定故也【慎子曰兎走於街百人逐之貪心俱存人莫之非者以兎為未定分也積兎在市過而不顧非不欲兎也分定之後雖鄙不爭分扶問翻】譚長子當為嗣而斥使居外禍其始此矣【譚尚之爭沮受固知之矣長知兩翻下同】紹曰吾欲令諸子各據一州以視其能於是以中子熙為幽州刺史【中讀曰仲】外甥高幹為并州刺史【此皆前事史因紹死而譚尚爭書之以先事】逢紀審配素為譚所疾【逢皮江翻】辛評郭圖皆附於譚而與配紀有隙及紹薨衆以譚長欲立之配等恐譚立而評等為害遂矯紹遺命奉尚為嗣譚至不得立自稱車騎將軍【袁紹初起兵自稱車騎將軍故譚亦稱之】屯黎陽尚少與之兵【少詩沼翻】而使逢紀隨之譚求益兵審配等又議不與譚怒殺逢紀秋九月曹操渡河攻譚譚告急於尚尚留審配守鄴自將助譚與操相拒連戰譚尚數敗退而固守【數所角翻】尚遣所置河東太守郭援與高幹匈奴南單于共攻河東發使與關中諸將馬騰等連兵【使疏吏翻】騰等隂許之援所經城邑皆下河東郡吏賈逵守絳【絳縣屬河東郡春秋晉所都也】援攻之急城將潰父老與援約不害逵乃降【降戶江翻】援許之援欲使逵為將【將即亮翻】以兵刼之逵不動左右引逵使叩頭逵叱之曰安有國家長吏為賊叩頭【逵郡吏非長吏也以守絳故自謂縣長吏為于偽翻】援怒將斬之或伏其上以救之絳吏民聞將殺逵皆乘城呼曰【呼火故翻】負約殺我賢君寧俱死耳乃囚於壺關著土窖中【壺關縣屬上黨郡著陟略翻窖居效翻掘地以藏粟之所】盖以車輪逵謂守者曰此間無健兒邪而使義士死此中乎有祝公道者適聞其言乃夜往盜引出逵折械遣去不語其姓名【語牛倨翻】曹操使司隸校尉鍾繇圍南單于於平陽【平陽縣屬河東郡時南單于呼㕑泉居之】未拔而援至繇使新豐令馮翊張既說馬騰【新豐縣屬京兆太守說輸苪翻】為言利害【為于偽翻】騰疑未决傅幹說騰曰古人有言順道者昌逆德者亡【新城三老董公之言】曹公奉天子誅暴亂法明政治上下用命可謂順道矣【治直吏翻】袁氏恃其彊大背弃王命【背蒲妹翻】驅胡虜以陵中國可謂逆德矣今將軍既事有道隂懷兩端【謂既附曹公又與袁氏通也】欲以坐觀成敗吾恐成敗既定奉辭責罪將軍先為誅首矣於是騰懼幹因曰智者轉禍為福今曹公與袁氏相持而高幹郭援合攻河東曹公雖有萬全之計不能禁河東之不危也將軍誠能引兵討援内外擊之【謂河東之兵擊之於内而馬騰之兵擊之於外也】其埶必舉是將軍一舉斷袁氏之臂【斷丁管翻】解一方之急曹公必重德將軍將軍功名無與比矣騰乃遣子超將兵萬餘人與繇會初諸將以郭援衆盛欲釋平陽去鍾繇曰袁氏方彊援之來關中隂與之通所以未悉叛者顧吾威名故耳若弃而去示之以弱所在之民誰非寇讐縱吾欲歸其得至乎此為未戰先自敗也【言若退師避援則關中諸將必叛雖欲歸司隸治所亦不得而至也】且援剛愎好勝必易吾軍【易輕也愎平逼翻好呼到翻易以䜴翻】若渡汾為營【水經注汾水南過平陽縣東】及其未濟擊之可大克也援至果徑前渡汾衆止之不從濟水未半繇擊大破之戰罷衆人皆言援死而不得其首援繇之甥也晚後馬超校尉南安龎德於鞬中出一頭【秦川記曰靈帝中平五年分漢陽置南安郡領䝠道新興中陶三縣鞬居言翻盛弓矢器】繇見之而哭德謝繇繇曰援雖我甥乃國賊也卿何謝之有南單于遂降【降戶江翻 考異曰魏志張既傳曰高幹及單于皆降非也】 劉表使劉備北侵至葉【葉縣屬南陽郡春秋楚葉公子高之邑也葉之涉翻】曹操遣夏侯惇于禁等拒之備一旦燒屯去惇等追之禆將軍鉅鹿李典曰【禆將軍在偏將軍之下禆頻彌翻】賊無故疑必有伏南道窄狹【窄側格翻】屮木深不可追也惇等不聽使典留守而追之果入伏裏兵大敗典往救之備乃退 曹操下書責孫權任子【任質任也操盖以此覘孫權而觀其所以應之】權召羣僚會議張昭秦松等猶豫不决權引周瑜詣吳夫人前定議【吴夫人權母也】瑜曰昔楚國初封不滿百里之地繼嗣賢能廣土開境遂據荆揚傳業延祚九百餘年【周成王封熊繹於楚以子男之田國於丹陽漢南郡枝江縣是也其後浸強至若敖蚡冒封畛於汝武王文王奄有江漢之間莊王以後與中國爭盟威王破越至于南海及秦而滅凡九百餘年】今將軍承父兄餘資兼六郡之衆【父謂孫堅兄謂孫策六郡會稽吴丹陽豫章廬陵廬江也】兵精糧多將士用命鑄山為銅煮海為鹽境内富饒人不思亂有何逼迫而欲送質【質音致下同】質一入不得不與曹氏相首尾與相首尾則命召不得不往如此便見制於人也極不過一侯印僕從十餘人車數乘馬數匹豈與南面稱孤同哉【建安十三年操自荆州東下約孫權會獵時周瑜未至魯肅說權其意亦與此同從才用翻乘繩證翻】不如勿遣徐觀其變若曹氏能率義以正天下將軍事之未晚若圖為暴亂彼自亡之不暇焉能害人【此數語所謂相時而動也然瑜之言不悖於大義魯肅呂蒙輩不能及也焉於䖍翻】吳夫人曰公瑾議是也公瑾與伯符同年小一月耳【周瑜字公瑾孫策字伯符瑾渠吝翻】我視之如子也汝其兄事之遂不送質<br />
<br />
  八月春二月曹操攻黎陽 【考異曰魏志武紀作三月今從范書袁紹傳又魏志紹傳云譚尚與太祖相拒黎陽自二月至九月當云自九月至二月】與袁譚袁尚戰於城下譚尚敗走還鄴夏四月操追至鄴收其麥 【考異曰范書紹傳曰尚逆擊破操軍今從魏志紹傳 予謂此諸葛孔明所謂偪于黎陽時也必有破操軍事魏人諱而不書耳】諸將欲乘勝遂攻之郭嘉曰袁紹愛此二子莫適立也【適丁歷翻主也】今權力相侔各有黨與【謂辛評郭圖等附譚審配等附尚也】急之則相保緩之則爭心生不如南向荆州【荆州劉表】以待其變變成而後擊之可一舉定也操曰善五月操還許留其將賈信屯黎陽譚謂尚曰我鎧甲不精故前為曹操所敗【鎧可亥翻敗補邁翻】今操軍退人懷歸志及其未濟出兵掩之可令大潰此筴不可失也尚疑之既不益兵又不易甲譚大怒郭圖辛評因謂譚曰使先公出將軍為兄後者皆審配之謀也譚遂引兵攻尚戰于門外【鄴城門外也】譚敗引兵還南皮【南皮縣屬勃海郡賢曰今滄州縣章武冇北皮亭故此曰南皮宋白曰縣道記云景州之南皮在郡東六十里南皮縣北有迎河河之北有故皮城是後漢勃海郡所理與郡理城南北非遠中隔迎河故瀆】别駕北海王修率吏民自青州往救譚【漢青州刺史治臨菑】譚欲更還攻尚修曰兄弟者左右手也譬人將鬭而斷其右手【斷丁管翻】曰我必勝其可乎夫弃兄弟而不親天下其誰親之彼讒人離間骨肉以求一朝之利願塞耳勿聽也【間古莧翻塞悉則翻】若斬佞臣數人復相親睦以御四方可横行於天下譚不從譚將劉詢起兵漯隂以叛譚【漯隂縣属平原郡應劭曰漯水出東武陽東北入海賢曰縣在漯水之南故城在今齊州臨邑縣西師古曰漯音他答翻】諸城皆應之譚歎曰今舉州皆叛豈孤之不德邪王修曰東萊太守管統雖在海表此人不反必來後十餘日統果弃其妻子來赴譚妻子為賊所殺譚更以統為樂安太守【漢末樂安國除為郡】 秋八月操擊劉表軍于西平【西平縣屬汝南郡從郭嘉之謀也】 袁尚自將攻袁譚大破之【將即亮翻】譚奔平原嬰城固守【前書音義曰嬰謂以城自繞也】尚圍之急譚遣辛評弟毗詣曹操請救劉表以書諫譚曰君子違難不適讐國【左傳公山不狃之言難乃旦翻】交絶不出惡聲【史記樂毅答燕惠王書之言】况忘先人之讐棄親戚之好而為萬世之戒遺同盟之耻哉【表與袁紹同盟好呼到翻遺于季翻下同】若冀州有不弟之傲【左傳曰段不弟書曰象傲尚據冀州故稱之】仁君當降志辱身以濟事為務事定之後使天下平其曲直不亦為高義邪又與尚書曰金木水火以剛柔相濟然後克得其和能為民用【金能勝木然執柯伐柯非木無以成金斵削之利水能勝火然水在火上非火無以成水烹餁之功此類非一可以槩推也】青州天性峭急【峭七笑翻譚據青州故稱之】迷於曲直仁君度數弘廣綽然有餘當以大包小以優容劣先除曹操以卒先公之恨【卒子恤翻】事定之後乃議曲直之計不亦善乎若迷而不反則胡夷將有譏誚之言【誚才笑翻】况我同盟復能勠力為君之役哉此韓盧東郭自困於前而遺田父之獲者也【淳于髠說齊威王曰韓盧者天下之狻犬也東郭狻者天下之狡兎也韓盧逐東郭狻騰山者五環山者三兎極於前犬疲於後犬兎俱疲各死其處田父見而獲之無勞苦而擅其功今齊魏相持頓兵敝衆恐秦楚乘其後而有田父之功也】譚尚皆不從辛毗至西平見曹操致譚意羣下多以為劉表彊宜先平之譚尚不足憂也荀攸曰天下方有事而劉表坐保江漢之間其無四方之志可知矣袁氏據四州之地帶甲數十萬紹以寬厚得衆心使二子和睦以守其成業則天下之難未息也【謂能為曹操患也難乃旦翻】今兄弟遘惡【遘當作搆或曰遘遇也謂以惡相遇也】其埶不兩全若有所并則力專力專則難圖也【謂譚尚若并於一則能專力以禦操其埶難圖】及其亂而取之天下定矣此時不可失也操從之後數日操更欲先平荆州使譚尚自相敝辛毗望操色知有變以語郭嘉【語牛倨翻】嘉白操操謂毗曰譚必可信尚必可克不【不讀曰否】毗對曰明公無問信與詐也直當論其埶耳袁氏本兄弟相伐非謂它人能間其間乃謂天下可定於己也【能間工莧翻言袁氏兄弟相攻其初計不謂它人能乘其間乃謂并青冀為一則可乘埶以定天下耳】今一旦求救於明公此可知也【言其埶窮】顯甫見顯思困而不能取【譚字顯思尚字顯甫】此力竭也兵革敗於外謀臣誅於内【謂逢紀田豐等死也】兄弟讒鬩【䦧馨激翻鬬也狠也戾也】國分為二連年戰伐介胄生蟣蝨加以旱蝗飢饉並臻天灾應於上人事困於下民無愚智皆知土崩瓦解此乃天亡尚之時也今往攻鄴尚不還救即不能自守還救即譚踵其後以明公之威應困窮之敵擊疲敝之寇無異迅風之振秋葉矣【秋葉易隕况遇迅風乎】天以尚與明公明公不取而伐荆州荆州豐樂【樂音洛】國未有釁仲虺有言取亂侮亡【見尚書孔安國註曰亂則取之冇亡形則侮之】方今二袁不務遠略而内相圖可謂亂矣居者無食行者無糧可謂亡矣朝不謀夕民命靡繼而不綏之欲待他年他年或登【歲熟曰登】又自知亡而改修厥德失所以用兵之要矣今因其請救而撫之利莫大焉且四方之寇莫大於河北河北平則六軍盛而天下震矣【觀毗之言非為譚請救也勸操以取河北也】操曰善乃許譚平冬十月操至黎陽尚聞操渡河乃釋平原還鄴尚將呂曠高翔畔歸曹操譚復隂刻將軍印以假曠翔操知譚詐乃為子整聘譚女以安之而引軍還【操本有伐尚因而取譚之心况復有誘曠翔之事乎聘其女為子婦以安之所謂將欲取之必姑與之也復扶又翻下同為于偽翻】 孫權西伐黄祖破其舟軍惟城未克而山寇復動【丹陽豫章廬陵皆有山越】權還過豫章使征虜中郎將呂範平鄱陽會稽【呂範傳止云鄱陽孫權傳則有會稽二字以地理攷之會稽二字衍】盪寇中郎將程普討樂安【晉志及宋志鄱陽郡冇樂安縣吴立建安十五年孫權始分豫章立鄱陽郡盪寇中郎將權所置也】建昌都尉太史慈領海昏【和帝永元十六年分海昏立建昌縣屬豫章郡孫策分海昏建昌六縣以太史慈為建昌都尉治海昏】以别部司馬黄蓋韓當周泰呂蒙等守劇縣令長【劇艱也甚也言其地當山越之要最為艱劇之甚者也】討山越悉平之建安漢興南平民作亂聚衆各萬餘人【建安本治縣地會稽南部都尉治焉建安中分東侯官置建安縣用漢年號也今建寧府地漢興縣沈約曰漢末立吴更名吴興南平縣亦漢末立晉武平吴改曰延平今南劒州地時皆屬南部都尉】權使南部都尉會稽賀齊進討皆平之復立縣邑料出兵萬人拜齊平東校尉【會工外翻復如字料音聊校戶教翻】<br />
<br />
  九年春正月曹操濟河遏淇水入白溝以通糧道【袁尚在鄴操將攻之故通糧道班志曰淇水至黎陽入河曹操於水口下大枋木以成堰遏淇水東入白溝水經注曰淇水東過内黄縣南為白溝】二月袁尚復攻袁譚於平原【復扶又翻】留其將審配蘇由守鄴曹操進軍至洹水【水經洹水出上黨泫氏縣東過隆慮縣北又東北出山徑鄴縣南洹于元翻又音桓】蘇由欲為内應謀泄出犇操操進至鄴為土山地道以攻之尚武安長尹楷屯毛城以通上黨糧道【武安縣屬魏郡唐洛州地長知兩翻下同】夏四月操留曹洪攻鄴自將擊楷破之而還又擊尚將沮鵠於邯鄲拔之【裴松之曰沮音葅河朔間今猶有此姓鵠沮授子也沮子余翻邯鄲音寒丹】易陽令韓範涉長梁岐皆舉縣降【易陽縣屬趙國涉縣蓋漢末分土黨之潞縣置魏後置廣平郡二縣皆屬焉北齊廢涉縣入刈陵縣隋唐復置涉縣宋白曰涉縣因縣南涉河為名磁州昭義縣理故涉城永泰元年改名昭義】徐晃言于操曰二袁未破諸城未下者傾耳而聽宜旌賞二縣以示諸城操從之範岐皆賜爵關内侯黑山賊帥張燕遣使求助操拜平北將軍【晉志曰四平止于喪亂時以河北未平授以此號及晉以後征鎮安平以次進號帥所類翻】五月操毁土山地道鑿塹圍城周回四十里【土山地道急攻也知非急攻可拔故鑿塹圍城絶其内外以久困之塹七艷翻】初令淺示若可越配望見笑之不出爭利操一夜濬之廣深二丈【廣古曠翻深悉禁翻度之廣深也後放此】引漳水以灌之【水經注漳水過鄴縣西魏武堨以圍鄴】城中餓死者過半秋七月尚將兵萬餘人還救鄴未到欲令審配知外動止先使主簿鉅鹿李孚入城孚斫問事杖繫著馬邊【問事卒也主行杖猶伍伯之類問事杖問事所執杖也著直略翻】自著平上幘【幘有顔題其顔却摞施巾連題却覆之平上幘者其上平也晉志引漢注曰冠惠文者宜短耳今平上幘也冠進賢者宜長耳今介幘也文吏服介幘武吏服平上幘著涉略翻】將三騎投暮詣鄴下自稱都督歷北圍循表而東【表圍城所立標表也騎奇寄翻】步步呵責守圍將士隨輕重行其罰遂歷操營前至南圍當章門【鄴城有七門正南曰章門亦曰中陽門】復責怒守圍者收縛之因開其圍馳到城下呼城上人城上人以繩引孚得入【不先經操營前則守圍者必疑不可得而收縛圍亦不可開矣孚之來也其定計固指從章門入也復扶又翻下同】配等見孚悲喜鼓譟稱萬歲守圍者以狀聞操笑曰此非徒得入也方且復出【操知其復出非不欲嚴為之防也審孚所以得入之由服其多智有不可得而防之也】孚知外圍益急不可復冒乃請配悉出城中老弱以省穀夜間别數千人皆使持白幡從三門並出降【鄴城南面三門曰鳳陽門中陽門廣陽門簡别彼列翻降戶江翻下同】孚復將三騎作降人服隨輩夜出突圍得去尚兵既至諸將皆以為此歸師人自為戰不如避之【兵法曰歸師勿遏】操曰尚從大道來當避之若循西山來者此成禽耳【從大道來則人懷救根本不顧勝敗有必死之志循山而來則其戰可前可卻人有依險自全之心無同力致命之意操所以料尚者如此兵法所謂觀敵之動者也】尚果循西山來東至陽平亭去鄴十七里臨滏水為營【郡國志鄴有滏水左思魏都賦曰北臨漳滏則冬夏異沼注云鄴北有滏水水熱故名滏口】夜舉火以示城中城中亦舉火相應配出兵城北欲與尚對决圍操逆擊之敗還尚亦破走依曲漳為營【賢曰漳水之曲也】操遂圍之未合尚愳遣使求降操不聽圍之益急尚夜遁保祁山【陳壽魏武紀作祁山袁紹傳作濫口范史袁紹傳作藍口賢注曰相州安陽縣界有藍嵯山與鄴相近盖藍山之口 考異曰魏志紹傳還走濫口范書作藍口今從魏武紀】操復進圍之【復扶又翻】尚將馬延張顗等臨陳降衆大潰尚奔中山盡收其輜重【陳讀曰陣重直用翻】得尚印綬節鉞及衣物以示城中城中崩沮【沮在呂翻】審配令士卒曰堅守死戰操軍疲矣幽州方至【幽州謂袁熙也】何憂無主【配以此安衆心可謂忠勇矣】操出行圍【巡行長圍也行下更翻】配伏弩射之幾中【射而亦翻幾居希翻中竹仲翻】配兄子榮為東門校尉【鄴城東門口建春門七門之名盖皆石氏所命也】八月戊寅榮夜開門内操兵【内讀曰納】配拒戰城中操兵生獲之辛評家繫鄴獄辛毗馳往欲解之已悉為配所殺操兵縛配詣帳下毗逆以馬鞭擊其頭罵之曰奴汝今日真死矣配顧曰狗輩正由汝曹破我冀州恨不得殺汝也且汝今日能殺生我邪【言殺生由曹操不由辛毗】有頃操引見謂配曰曩日孤之行圍何弩之多也配曰猶恨其少【謂射操不中也少詩沼翻】操曰卿忠於袁氏亦自不得不爾意欲活之配意氣壯烈終無橈辭【橈奴教翻曲也】而辛毗等號哭不已【號戶刀翻】遂斬之冀州人張子謙先降素與配不善笑謂配曰正南【審配字正南】卿竟何如我配厲聲曰汝為降虜審配為忠臣雖死豈羨汝生邪臨行刑叱持兵者令北向曰我君在北也【袁紹下士能盡死以效節者審配一人而已我君在北謂袁尚已奔北也】操乃臨祀紹墓哭之流涕慰勞紹妻還其家人寶物賜襍繒絮稟食之【勞力到翻繒慈陵翻食讀曰飤】初袁紹與操共起兵紹問操曰若事不輯則方面何所可據【輯猶集也集成也觀紹此言則起兵之時固無勤王之心而有割據之志矣】操曰足下意以為何如紹曰吾南據河北阻燕代兼戎狄之衆南向以爭天下庶可以濟乎操曰吾任天下之智力以道御之無所不可九月詔以操領冀州牧操讓還兖州【當時政自操出領則真領而讓非真讓也】初袁尚遣從事安平牽招至上黨督軍糧【牽姓招名】未還尚走中山招說高幹以并州迎尚并力觀變【說輸芮翻】幹不從招乃東詣曹操操復以為冀州從事又辟崔琰為别駕操謂琰曰昨案戶籍可得三十萬衆故為大州也琰對曰今九州幅裂二袁兄弟親尋干戈【左傳子產曰昔高辛氏有二子伯曰閼伯季曰實沉居于曠林不相能也日尋干戈以相征討杜預注曰尋用也】冀方蒸庶暴骨原野未聞王師存問風俗救其塗炭而校計甲兵唯此為先斯豈鄙州士女所望於明公哉操改容謝之【此操之所以重崔琰而亦不能不害崔琰也】許攸恃功驕嫚【烏巢之捷計出于攸故恃其功】嘗於衆坐呼操小字曰某甲【裴松之曰操一名吉利小字阿瞞曰某甲者史隱其辭坐徂卧翻】卿非我不得冀州也操笑曰汝言是也然内不樂【樂音洛】後竟殺之 冬十月有星孛于東井【晉天文志南方東井八星天之南門黄道所經天之亭候主水衡事法令所取平也孛蒲内翻】 高幹以并州降操復以幹為并州刺史【為幹復叛張本降戶江翻復扶又翻】 曹操之圍鄴也袁譚復背之【復扶又翻下同】略取甘陵安平勃海河間攻袁尚於中山尚敗走故安【故安縣屬涿郡賢曰故城在今易州易縣南】從袁熙譚悉收其衆還屯龍湊操與譚書責以負約與之絶婚女還然後進討【袁尚破走操於是始討譚】十二月操軍其門譚拔平原走保南皮臨清河而屯【水經清河過南皮縣西】操入平原略定諸縣 曹操表公孫度為武威將軍封永寧鄉侯度曰我王遼東何永寧也【王于况翻】藏印綬於武庫【遼東郡之武庫也】是歲度卒子康嗣位以永寧鄉侯封其弟恭操以牽招嘗為袁氏領烏桓【牽姓招名袁紹先嘗辟招為督軍從事兼領烏桓突騎】遣詣柳城撫慰烏桓值峭王嚴五千騎欲助袁譚又公孫康遣使韓忠假峭王單于印綬峭王大會羣長【烏桓部落各有君長峭七笑翻使疏吏翻長知兩翻】忠亦在坐【坐才卧翻下同】峭王問招昔袁公言受天子之命假我為單于今曹公復言當更白天子假我真單于遼東復持印綬來【復扶又翻】如此誰當為正招答曰昔袁公承制得有所拜假中間違錯天子命【違異也背也錯乖也】曹公代之言當白天子更假真單于遼東下郡何得擅稱拜假也忠曰我遼東在滄海之東擁兵百餘萬又有扶餘濊貊之用【濊音穢貊莫百翻】當今之埶彊者為右曹操何得獨為是也招呵忠曰曹公允恭明哲【孔安國尚書注曰允信也】翼戴天子伐叛柔服寧靜四海汝君臣頑嚚【嚚魚巾翻左傳曰不道忠信之言為嚚】今恃險遠背違王命【背蒲妹翻】欲擅拜假侮弄神器【威福帝王之神器】方當屠勠何敢慢易咎毁大人【大人謂曹公易音以䜴翻】便捉忠頭頓築拔刀欲斬之峭王驚怖【怖普布翻】徒跣抱招以救請忠左右失色招乃還坐為峭王等說成敗之效禍福所歸皆下席跪伏敬受敕教【敕戒也為于偽翻】便辭遼東之使罷所嚴騎丹陽大都督媯覽郡丞戴員殺太守孫翊將軍孫河<br />
<br />
  屯京城馳赴宛陵【京城即漢吳郡丹徒縣也孫權自吳徙居之命曰京城亦曰京口予謂此京取爾雅丘絶高曰京之義宛陵丹陽郡治所媯覽戴員盛憲之黨也媯俱為翻姓也舜居媯汭其後因以為氏員音云】覽員復殺之【復扶又翻】遣人迎揚州刺史劉馥【馥曹操所用也】令住歷陽以丹陽應之【歷陽與丹陽隔江使馥來屯以為聲援】覽入居軍府中欲逼取翊妻徐氏徐氏紿之曰乞須晦日【月終為晦隂之盡也紿蕩亥翻】設祭除服然後聽命覽許之徐氏潛使所親語翊親近舊將孫高傅嬰等與共圖覽【語牛倨翻】高嬰涕泣許諾密呼翊時侍養者二十餘人與盟誓合謀【侍養謂侍翊左右而厚蒙給養者】到晦日設祭徐氏哭泣盡哀畢乃除服薰香沐浴言笑懽悦大小悽愴【悽悲也痛也愴傷也音初亮翻】怪其如此覽密覘無復疑意【覘丑亷翻又丑艷翻復扶又翻】徐氏呼高嬰置戶内使人召覽入徐氏出戶拜覽適得一拜徐大呼二君可起【呼火故翻】高嬰俱出共殺覽餘人即就外殺員徐氏乃還縗絰【復著縗絰也縗倉回翻】奉覽員首以祭翊墓舉軍震駭孫權聞亂從椒丘還【椒丘在豫章】至丹陽悉族誅覽員餘黨擢高嬰為牙門【牙門將也】其餘賞賜有差河子韶年十七收河餘衆屯京城權引軍歸吳夜至京城下營試攻驚之兵皆乘城傳檄備警讙聲動地【讙許元翻】頗射外人權使曉喻乃止明日見韶拜承烈校尉統河部曲【史言孫權能用人以保江東射而亦翻】十年春正月曹操攻南皮袁譚出戰士卒多死操欲緩之議郎曹純曰今縣師深入【純仁之弟也縣讀作懸】難以持久若進不能克退必喪威【喪息浪翻】乃自執桴鼓以率攻者【桴音膚】遂克之譚出走追斬之李孚自稱冀州主簿求見操曰今城中彊弱相陵人心擾亂以為宜令新降為内所識信者宣傳明教【降戶江翻】操即使孚往入城告諭吏民使各安故業不得相侵城中乃安【李孚小才也挾才以求知非懷才以待聘者也】操于是斬郭圖等及其妻子【郭圖審配各有黨附交鬭譚尚使尋干戈以貽曹氏之驅除譚尚既敗二人亦誅禍福之報為不爽矣】袁譚使王修運糧於安樂聞譚急將所領兵往赴之至高密聞譚死下馬號哭曰無君焉歸【號戶刀翻焉於䖍翻】遂詣曹操乞收葬譚尸操許之復使修還樂安督軍糧譚所部諸城皆服唯樂安太守管統不下操命修取統首【使還運糧就取統首也】修以統亡國忠臣解其縛使詣操操悦而赦之辟修為司空掾郭嘉說操多辟青冀幽并名士以為掾屬使人心歸附操從之官渡之戰袁紹使陳琳為檄書數操罪惡連及家世極其醜詆及袁氏敗琳歸操操曰卿昔為本初移書【說輸芮翻下同數所具翻為于偽翻下同】但可罪狀孤身何乃上及父祖邪【案文選琳為紹檄豫州盖帝都許許屬潁川郡豫州部屬也故選專以檄豫州為言琳檄略曰操祖父騰與左悺徐璜並作妖孽饕餮放横傷化害人父嵩乞匄攜養因臧買位竊盜鼎司操奸閹遺醜僄狡鋒俠好亂樂禍又數其殘賢害善專制朝政發掘墳陵之罪文多不載】琳謝罪操釋之使與陳留阮瑀俱管記室【漢公府有記室令史主上章表報書記】先是漁陽王松據涿郡【先悉薦翻】郡人劉放說松以地歸操操辟放參司空軍事【為劉放因此管魏機密以亂魏張本】袁熙為其將焦觸張南所攻與尚俱犇遼西烏桓【遼西烏桓其酋曰蹋頓】觸自號幽州刺史驅率諸郡太守令長背袁向曹【長知兩翻背蒲妹翻】陳兵數萬殺白馬而盟令曰敢違者斬衆莫敢仰視各以次歃【歃色洽翻】别駕代郡韓珩曰【珩音行】吾受袁公父子厚恩今其破亡智不能救勇不能死於義闕矣若乃北面曹氏所不能為也一坐為珩失色【坐徂臥翻】觸曰夫舉大事當立大義事之濟否不待一人可卒珩志以厲事君【卒子恤翻】乃捨之觸等遂降曹操皆封為列侯【降戶江翻】 夏四月黑山賊帥張燕率其衆十餘萬降【帥所類翻】封安國亭侯 故安趙犢霍奴等殺幽州刺史及涿郡太守三郡烏桓攻鮮于輔於獷平【三郡烏桓遼西蹋頓遼東蘇僕延右北平烏延也獷平縣屬漁陽郡服䖍曰獷音鞏師古曰音九勇翻又音鑛】秋八月操討犢等斬之乃渡潞水救獷平烏垣走出塞 冬十月高幹聞操討烏桓復以并州叛【復扶又翻】執上黨太守舉兵守壺關口【賢曰潞州上黨縣有壺山口因其險而置關焉二漢志壺關縣屬上黨郡】操遣其將樂進李典擊之河内張晟衆萬餘人寇崤澠間【晟成正翻澠彌兖翻】弘農張琰起兵以應之河東太守王邑被徵【被皮義翻】郡掾衛固及中郎將范先等詣司隸校尉鍾繇請留之【掾俞絹翻】繇不許固等外以請邑為名而内實與高幹通謀曹操謂荀彧曰關西諸將外服内貳張晟寇亂崤澠南通劉表固等因之將為深害當今河東天下之要地也【高幹據并州馬騰韓遂等據關中往來交通皆由河東故曰要地】君為我舉賢才以鎮之【為于偽翻】彧曰西平太守京兆杜畿【漢末分金城置西平郡】勇足以當難【難乃旦翻下同】智足以應變操乃以畿為河東太守鍾繇促王邑交符【交郡符也】邑佩印綬徑從河北諸許自歸【河北縣屬河東郡宋白曰陜州平陸縣本漢大陽縣地後漢改為河北縣】衛固等使兵數千人絶陜津【水經注河水東過陜縣北河北對茅城謂之茅津亦謂之陜津陜式冉翻】杜畿至數月不得渡操遣夏侯惇討固等未至畿曰河東有三萬戶非皆欲為亂也今兵迫之急欲為善者無主必懼而聽於固固等埶專討之不勝為難未已討之而勝是殘一郡之民也且固等未顯絶王命外以請故君為名必不害新君吾單車直往出其不意固為人多計而無斷【斷丁亂翻】必偽受吾吾得居郡一月以計縻之足矣遂詭道從郖津度【水經注河水東逕湖縣故城北又東合柏谷水又東右合門水河水於此有郖津之名郖音竇】范先欲殺畿以威衆且觀畿去就於門下斬殺主簿以下三十餘人畿舉動自若于是固曰殺之無損徒有惡名且制之在我遂奉之畿謂固先曰衛范河東之望也吾仰成而已【仰牛向翻】然君臣有定義成敗同之大事當共平議以固為都督行丞事領功曹【既已為都督又令行郡丞事又領功曹也都督掌兵丞貳太守於郡事無所不關功曹掌選署功勞陽以郡權悉與之也】將校吏兵三千餘人皆范先督之【將即亮翻校戶教翻】固等喜雖陽事畿不以為意固欲大發兵畿患之說固曰今大發兵衆情必擾不如徐以貲募兵固以為然從之得兵甚少【以貲募兵則郡計不足以繼故得兵甚少】畿又喻固等曰人情顧家諸將掾史可分遣休息【掾俞絹翻】急緩召之不難固等惡逆衆心【惡烏路翻】又從之於是善人在外隂為已援惡人分散各還其家會白騎攻東垣【白騎張白騎之衆相聚為賊者也垣縣屬河東郡東字衍續漢志垣縣注云山在東狀如垣盖此時已有東垣之名騎奇寄翻】高幹入濩澤【濩澤縣屬河東郡賢曰今澤州縣師古曰濩音烏虢翻】畿知諸縣附已乃出單將數十騎赴堅壁而守之【將即亮翻堅壁壁壘之最堅者】吏民多舉城助畿者【舉城謂舉屬縣城也】比數十日【比必寐翻】得四千餘人固等與高幹張晟共攻畿不下略諸縣無所得曹操使議郎張既西徵關中諸將馬騰等皆引兵會擊晟等破之斬固琰等首其餘黨與皆赦之於是杜畿治河東務崇寬惠【治音直之翻】民有辭訟畿為陳義理遣歸諦思之【為音于偽翻諦音丁計翻審也】父老皆自相責怒不敢訟勸耕桑課畜牧百姓家家豐實然後興學校【校戶教翻】舉孝弟【弟讀曰悌】修戎事講武備河東遂安畿在河東十六年常為天下最【為曹操因河東資實以平關中張本杜畿之子為杜恕恕之子為杜預其守河東觀其方略固未易才也予竊謂杜氏仕于魏晉累世貴盛必有家傳史因而書之固有過其實者】 秘書監侍中荀悦【桓帝延熹二年置秘書監秩六百石】作申鑒五篇奏之悦爽之兄子也時政在曹氏天子恭己【言恭己南面而已政事無所預也孔子曰無為而治者其舜也歟夫何為哉恭己正南面而已後世遂以政在強臣已無所預為恭己舜之恭己果如是哉】悦志在獻替【獻可替否】而謀無所用故作是書其大畧曰為政之術先屏四患【屏必郢翻】乃崇五政偽亂俗私壞法放越軌奢敗制【壞音怪敗補邁翻】四者不除則政末由行矣是謂四患興農桑以養其生審好惡以正其俗【好呼到翻惡烏路翻】宣文教以章其化立武備以秉其威明賞罰以統其法是謂五政人不畏死不可懼以罪人不樂生不可勸以善故在上者先豐民財以定其志是謂養生【此說萬世不可易也樂音洛】善惡要乎功罪毁譽効於凖驗【書云無稽之言勿聽】聽言責事舉名察實無或詐偽以蕩衆心【蕩謂動之也以詐偽動之則人之心亦必動於詐偽以應其上】故俗無姦怪民無淫風是謂正俗榮辱者賞罰之精華也故禮教榮辱以加君子化其情也桎梏鞭撲以加小人化其形也【桎之日翻梏工沃翻撲普卜翻】若教化之廢推中人而墜於小人之域【推吐雷翻】教化之行引中人而納於君子之塗是謂章化在上者必有武備以戒不虞安居則寄之内政【國語管仲相齊桓公作内政以寄軍令】有事則用之軍旅是謂秉威賞罰政之柄也人主不妄賞非愛其財也賞妄行則善不勸矣不妄罸非矜其人也罰妄行則惡不懲矣賞不勸謂之止善罸不懲謂之縱惡在上者能不止下為善不縱下為惡則國法立矣是謂統法四患既蠲五政又立行之以誠守之以固簡而不怠疎而不失垂拱揖讓而海内平矣【荀悦申鑒其立論精切關於國家興亡之大致過於彧攸至于揣摩天下之勢應敵設變以制一時之勝悦未必能也曹操奸雄親信彧攸而悦乃在天子左右悦非比於彧攸而操不之忌盖知悦但能持論其才必不能辦也嗚呼東都之季荀淑以名德稱而彧攸以智略濟荀悦蓋得其祖父之彷彿耳其才不足以用世其言僅見于此書後之有天下國家者尚論其世深味其言則知悦之忠于漢室而有補于天下國家也蠲吉玄翻】<br />
<br />
  資治通鑑卷六十四  <br>
   </div> 

<script src="/search/ajaxskft.js"> </script>
 <div class="clear"></div>
<br>
<br>
 <!-- a.d-->

 <!--
<div class="info_share">
</div> 
-->
 <!--info_share--></div>   <!-- end info_content-->
  </div> <!-- end l-->

<div class="r">   <!--r-->



<div class="sidebar"  style="margin-bottom:2px;">

 
<div class="sidebar_title">工具类大全</div>
<div class="sidebar_info">
<strong><a href="http://www.guoxuedashi.com/lsditu/" target="_blank">历史地图</a></strong>  
<a href="http://www.880114.com/" target="_blank">英语宝典</a>  
<a href="http://www.guoxuedashi.com/13jing/" target="_blank">十三经检索</a> 
<br><strong><a href="http://www.guoxuedashi.com/gjtsjc/" target="_blank">古今图书集成</a></strong> 
<a href="http://www.guoxuedashi.com/duilian/" target="_blank">对联大全</a> <strong><a href="http://www.guoxuedashi.com/xiangxingzi/" target="_blank">象形文字典</a></strong> 

<br><a href="http://www.guoxuedashi.com/zixing/yanbian/">字形演变</a>  <strong><a href="http://www.guoxuemi.com/hafo/" target="_blank">哈佛燕京中文善本特藏</a></strong>
<br><strong><a href="http://www.guoxuedashi.com/csfz/" target="_blank">丛书&方志检索器</a></strong> <a href="http://www.guoxuedashi.com/yqjyy/" target="_blank">一切经音义</a>  

<br><strong><a href="http://www.guoxuedashi.com/jiapu/" target="_blank">家谱族谱查询</a></strong>  <strong><a href="http://shufa.guoxuedashi.com/sfzitie/" target="_blank">书法字帖欣赏</a></strong> 
<br>

</div>
</div>


<div class="sidebar" style="margin-bottom:0px;">

<font style="font-size:22px;line-height:32px">QQ交流群9:489193090</font>


<div class="sidebar_title">手机APP 扫描或点击</div>
<div class="sidebar_info">
<table>
<tr>
	<td width=160><a href="http://m.guoxuedashi.com/app/" target="_blank"><img src="/img/gxds-sj.png" width="140"  border="0" alt="国学大师手机版"></a></td>
	<td>
<a href="http://www.guoxuedashi.com/download/" target="_blank">app软件下载专区</a><br>
<a href="http://www.guoxuedashi.com/download/gxds.php" target="_blank">《国学大师》下载</a><br>
<a href="http://www.guoxuedashi.com/download/kxzd.php" target="_blank">《汉字宝典》下载</a><br>
<a href="http://www.guoxuedashi.com/download/scqbd.php" target="_blank">《诗词曲宝典》下载</a><br>
<a href="http://www.guoxuedashi.com/SiKuQuanShu/skqs.php" target="_blank">《四库全书》下载</a><br>
</td>
</tr>
</table>

</div>
</div>


<div class="sidebar2">
<center>


</center>
</div>

<div class="sidebar"  style="margin-bottom:2px;">
<div class="sidebar_title">网站使用教程</div>
<div class="sidebar_info">
<a href="http://www.guoxuedashi.com/help/gjsearch.php" target="_blank">如何在国学大师网下载古籍?</a><br>
<a href="http://www.guoxuedashi.com/zidian/bujian/bjjc.php" target="_blank">如何使用部件查字法快速查字?</a><br>
<a href="http://www.guoxuedashi.com/search/sjc.php" target="_blank">如何在指定的书籍中全文检索?</a><br>
<a href="http://www.guoxuedashi.com/search/skjc.php" target="_blank">如何找到一句话在《四库全书》哪一页?</a><br>
</div>
</div>


<div class="sidebar">
<div class="sidebar_title">热门书籍</div>
<div class="sidebar_info">
<a href="/so.php?sokey=%E8%B5%84%E6%B2%BB%E9%80%9A%E9%89%B4&kt=1">资治通鉴</a> <a href="/24shi/"><strong>二十四史</strong></a>&nbsp; <a href="/a2694/">野史</a>&nbsp; <a href="/SiKuQuanShu/"><strong>四库全书</strong></a>&nbsp;<a href="http://www.guoxuedashi.com/SiKuQuanShu/fanti/">繁体</a>
<br><a href="/so.php?sokey=%E7%BA%A2%E6%A5%BC%E6%A2%A6&kt=1">红楼梦</a> <a href="/a/1858x/">三国演义</a> <a href="/a/1038k/">水浒传</a> <a href="/a/1046t/">西游记</a> <a href="/a/1914o/">封神演义</a>
<br>
<a href="http://www.guoxuedashi.com/so.php?sokeygx=%E4%B8%87%E6%9C%89%E6%96%87%E5%BA%93&submit=&kt=1">万有文库</a> <a href="/a/780t/">古文观止</a> <a href="/a/1024l/">文心雕龙</a> <a href="/a/1704n/">全唐诗</a> <a href="/a/1705h/">全宋词</a>
<br><a href="http://www.guoxuedashi.com/so.php?sokeygx=%E7%99%BE%E8%A1%B2%E6%9C%AC%E4%BA%8C%E5%8D%81%E5%9B%9B%E5%8F%B2&submit=&kt=1"><strong>百衲本二十四史</strong></a>  <a href="http://www.guoxuedashi.com/so.php?sokeygx=%E5%8F%A4%E4%BB%8A%E5%9B%BE%E4%B9%A6%E9%9B%86%E6%88%90&submit=&kt=1"><strong>古今图书集成</strong></a>
<br>

<a href="http://www.guoxuedashi.com/so.php?sokeygx=%E4%B8%9B%E4%B9%A6%E9%9B%86%E6%88%90&submit=&kt=1">丛书集成</a> 
<a href="http://www.guoxuedashi.com/so.php?sokeygx=%E5%9B%9B%E9%83%A8%E4%B8%9B%E5%88%8A&submit=&kt=1"><strong>四部丛刊</strong></a>  
<a href="http://www.guoxuedashi.com/so.php?sokeygx=%E8%AF%B4%E6%96%87%E8%A7%A3%E5%AD%97&submit=&kt=1">說文解字</a> <a href="http://www.guoxuedashi.com/so.php?sokeygx=%E5%85%A8%E4%B8%8A%E5%8F%A4&submit=&kt=1">三国六朝文</a>
<br><a href="http://www.guoxuedashi.com/so.php?sokeytm=%E6%97%A5%E6%9C%AC%E5%86%85%E9%98%81%E6%96%87%E5%BA%93&submit=&kt=1"><strong>日本内阁文库</strong></a> <a href="http://www.guoxuedashi.com/so.php?sokeytm=%E5%9B%BD%E5%9B%BE%E6%96%B9%E5%BF%97%E5%90%88%E9%9B%86&ka=100&submit=">国图方志合集</a> <a href="http://www.guoxuedashi.com/so.php?sokeytm=%E5%90%84%E5%9C%B0%E6%96%B9%E5%BF%97&submit=&kt=1"><strong>各地方志</strong></a>

</div>
</div>


<div class="sidebar2">
<center>

</center>
</div>
<div class="sidebar greenbar">
<div class="sidebar_title green">四库全书</div>
<div class="sidebar_info">

《四库全书》是中国古代最大的丛书,编撰于乾隆年间,由纪昀等360多位高官、学者编撰,3800多人抄写,费时十三年编成。丛书分经、史、子、集四部,故名四库。共有3500多种书,7.9万卷,3.6万册,约8亿字,基本上囊括了古代所有图书,故称“全书”。<a href="http://www.guoxuedashi.com/SiKuQuanShu/">详细>>
</a>

</div> 
</div>

</div>  <!--end r-->

</div>
<!-- 内容区END --> 

<!-- 页脚开始 -->
<div class="shh">

</div>

<div class="w1180" style="margin-top:8px;">
<center><script src="http://www.guoxuedashi.com/img/plus.php?id=3"></script></center>
</div>
<div class="w1180 foot">
<a href="/b/thanks.php">特别致谢</a> | <a href="javascript:window.external.AddFavorite(document.location.href,document.title);">收藏本站</a> | <a href="#">欢迎投稿</a> | <a href="http://www.guoxuedashi.com/forum/">意见建议</a> | <a href="http://www.guoxuemi.com/">国学迷</a> | <a href="http://www.shuowen.net/">说文网</a><script language="javascript" type="text/javascript" src="https://js.users.51.la/17753172.js"></script><br />
  Copyright &copy; 国学大师 古典图书集成 All Rights Reserved.<br>
  
  <span style="font-size:14px">免责声明:本站非营利性站点,以方便网友为主,仅供学习研究。<br>内容由热心网友提供和网上收集,不保留版权。若侵犯了您的权益,来信即刪。scp168@qq.com</span>
  <br />
ICP证:<a href="http://www.beian.miit.gov.cn/" target="_blank">鲁ICP备19060063号</a></div>
<!-- 页脚END --> 
<script src="http://www.guoxuedashi.com/img/plus.php?id=22"></script>
<script src="http://www.guoxuedashi.com/img/tongji.js"></script>

</body>
</html>
