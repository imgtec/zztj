<!DOCTYPE html PUBLIC "-//W3C//DTD XHTML 1.0 Transitional//EN" "http://www.w3.org/TR/xhtml1/DTD/xhtml1-transitional.dtd">
<html xmlns="http://www.w3.org/1999/xhtml">
<head>
<meta http-equiv="Content-Type" content="text/html; charset=utf-8" />
<meta http-equiv="X-UA-Compatible" content="IE=Edge,chrome=1">
<title>資治通鑒_16-資治通鑑卷十五_16-資治通鑑卷十五</title>
<meta name="Keywords" content="資治通鑒_16-資治通鑑卷十五_16-資治通鑑卷十五">
<meta name="Description" content="資治通鑒_16-資治通鑑卷十五_16-資治通鑑卷十五">
<meta http-equiv="Cache-Control" content="no-transform" />
<meta http-equiv="Cache-Control" content="no-siteapp" />
<link href="/img/style.css" rel="stylesheet" type="text/css" />
<script src="/img/m.js?2020"></script> 
</head>
<body>
 <div class="ClassNavi">
<a  href="/24shi/">二十四史</a> | <a href="/SiKuQuanShu/">四库全书</a> | <a href="http://www.guoxuedashi.com/gjtsjc/"><font  color="#FF0000">古今图书集成</font></a> | <a href="/renwu/">历史人物</a> | <a href="/ShuoWenJieZi/"><font  color="#FF0000">说文解字</a></font> | <a href="/chengyu/">成语词典</a> | <a  target="_blank"  href="http://www.guoxuedashi.com/jgwhj/"><font  color="#FF0000">甲骨文合集</font></a> | <a href="/yzjwjc/"><font  color="#FF0000">殷周金文集成</font></a> | <a href="/xiangxingzi/"><font color="#0000FF">象形字典</font></a> | <a href="/13jing/"><font  color="#FF0000">十三经索引</font></a> | <a href="/zixing/"><font  color="#FF0000">字体转换器</font></a> | <a href="/zidian/xz/"><font color="#0000FF">篆书识别</font></a> | <a href="/jinfanyi/">近义反义词</a> | <a href="/duilian/">对联大全</a> | <a href="/jiapu/"><font  color="#0000FF">家谱族谱查询</font></a> | <a href="http://www.guoxuemi.com/hafo/" target="_blank" ><font color="#FF0000">哈佛古籍</font></a> 
</div>

 <!-- 头部导航开始 -->
<div class="w1180 head clearfix">
  <div class="head_logo l"><a title="国学大师官网" href="http://www.guoxuedashi.com" target="_blank"></a></div>
  <div class="head_sr l">
  <div id="head1">
  
  <a href="http://www.guoxuedashi.com/zidian/bujian/" target="_blank" ><img src="http://www.guoxuedashi.com/img/top1.gif" width="88" height="60" border="0" title="部件查字,支持20万汉字"></a>


<a href="http://www.guoxuedashi.com/help/yingpan.php" target="_blank"><img src="http://www.guoxuedashi.com/img/top230.gif" width="600" height="62" border="0" ></a>


  </div>
  <div id="head3"><a href="javascript:" onClick="javascript:window.external.AddFavorite(window.location.href,document.title);">添加收藏</a>
  <br><a href="/help/setie.php">搜索引擎</a>
  <br><a href="/help/zanzhu.php">赞助本站</a></div>
  <div id="head2">
 <a href="http://www.guoxuemi.com/" target="_blank"><img src="http://www.guoxuedashi.com/img/guoxuemi.gif" width="95" height="62" border="0" style="margin-left:2px;" title="国学迷"></a>
  

  </div>
</div>
  <div class="clear"></div>
  <div class="head_nav">
  <p><a href="/">首页</a> | <a href="/ShuKu/">国学书库</a> | <a href="/guji/">影印古籍</a> | <a href="/shici/">诗词宝典</a> | <a   href="/SiKuQuanShu/gxjx.php">精选</a> <b>|</b> <a href="/zidian/">汉语字典</a> | <a href="/hydcd/">汉语词典</a> | <a href="http://www.guoxuedashi.com/zidian/bujian/"><font  color="#CC0066">部件查字</font></a> | <a href="http://www.sfds.cn/"><font  color="#CC0066">书法大师</font></a> | <a href="/jgwhj/">甲骨文</a> <b>|</b> <a href="/b/4/"><font  color="#CC0066">解密</font></a> | <a href="/renwu/">历史人物</a> | <a href="/diangu/">历史典故</a> | <a href="/xingshi/">姓氏</a> | <a href="/minzu/">民族</a> <b>|</b> <a href="/mz/"><font  color="#CC0066">世界名著</font></a> | <a href="/download/">软件下载</a>
</p>
<p><a href="/b/"><font  color="#CC0066">历史</font></a> | <a href="http://skqs.guoxuedashi.com/" target="_blank">四库全书</a> |  <a href="http://www.guoxuedashi.com/search/" target="_blank"><font  color="#CC0066">全文检索</font></a> | <a href="http://www.guoxuedashi.com/shumu/">古籍书目</a> | <a   href="/24shi/">正史</a> <b>|</b> <a href="/chengyu/">成语词典</a> | <a href="/kangxi/" title="康熙字典">康熙字典</a> | <a href="/ShuoWenJieZi/">说文解字</a> | <a href="/zixing/yanbian/">字形演变</a> | <a href="/yzjwjc/">金 文</a> <b>|</b>  <a href="/shijian/nian-hao/">年号</a> | <a href="/diming/">历史地名</a> | <a href="/shijian/">历史事件</a> | <a href="/guanzhi/">官职</a> | <a href="/lishi/">知识</a> <b>|</b> <a href="/zhongyi/">中医中药</a> | <a href="http://www.guoxuedashi.com/forum/">留言反馈</a>
</p>
  </div>
</div>
<!-- 头部导航END --> 
<!-- 内容区开始 --> 
<div class="w1180 clearfix">
  <div class="info l">
   
<div class="clearfix" style="background:#f5faff;">
<script src='http://www.guoxuedashi.com/img/headersou.js'></script>

</div>
  <div class="info_tree"><a href="http://www.guoxuedashi.com">首页</a> > <a href="/SiKuQuanShu/fanti/">四库全书</a>
 > <h1>资治通鉴</h1> <!--         下载:【右键另存为】即可 --></div>
  <div class="info_content zj clearfix">
  
<div class="info_txt clearfix" id="show">
<center style="font-size:24px;">16-資治通鑑卷十五</center>
    資治通鑑卷十五    宋 司馬光 撰<br />
<br />
  胡三省 音註<br />
<br />
  漢紀七【起玄黓涒灘盡柔兆閹茂凡十五年】<br />
<br />
  太宗孝文皇帝下<br />
<br />
  前十一年冬十一月上行幸代春正月自代還夏六月梁懷王揖薨【揖受封事見十三卷二年】無子賈誼復上疏曰【復扶又翻】陛下即不定制如今之埶不過一傳再傳【服䖍曰一二傳世也】諸侯猶且人恣而不制【言人人自恣而不可制也】豪植而大強【言其矜豪自植立太過於強也】漢法不得行矣陛下所以為藩扞及皇太子之所恃者唯淮陽代二國耳【淮陽王武代王參帝之子而太子之弟也故云所恃唯此二國】代北邊匈奴與強敵為鄰能自完則足矣而淮陽之比大諸侯廑如黑子之著面【廑與僅同師古曰黑子今所謂黶子也著則畧翻下北著同】適足以餌大國【言國小如魚餌適足為所吞食】而不足以有所禁禦方今制在陛下制國而令子適足以為餌豈可謂工哉臣之愚計願舉淮南地以益淮陽而為梁王立後【為于偽翻】割淮陽北邊二三列城與東郡以益梁不可者可徙代王而都睢陽【睢陽故宋國微子所封班志屬梁國括地志宋州宋城縣在州南二里外城中本漢之睢陽縣也漢文帝封子武於大梁以其北卑濕徙睢陽故改曰梁睢音雖】梁起於新郪而北著之河【班志新郪縣屬汝南郡應劭曰秦為郪丘漢興為新郪師古曰潁川縣郪千移翻】淮陽包陳而南揵之江【陳即謂古陳國之地也晉灼曰包取也如淳曰揵謂立封界也或曰揵接也師古曰揵巨偃翻】則大諸侯之有異心者破膽而不敢謀梁足以扞齊趙淮陽足以禁吳楚陛下高枕終無山東之憂矣【枕職任翻】此二世之利也【如淳曰從誼言可二世安耳師古曰言帝身及太子嗣位之時】當今恬然適遇諸侯之皆少【師古曰恬安也少謂年少少時照翻】數歲之後陛下且見之矣夫秦日夜苦心勞力以除六國之旤今陛下力制天下頤指如意【如淳曰但動頤指麾則所欲皆如意仲馮曰頤指兩事】高拱以成六國之旤難以言智苟身無事畜亂宿旤【畜讀曰蓄】孰視而不定【孰古熟字通】萬年之後傳之老母弱子將使不寧不可謂仁帝於是從誼計徙淮陽王武為梁王北界太山西至高陽得大縣四十餘城後歲餘賈誼亦死死時年三十三矣 徙城陽王喜為淮南王【喜城陽王章之子齊悼惠王肥之孫】 匈奴寇狄道【狄道縣為隴西郡治所師古曰其地有狄種故曰狄道】時匈奴數為邊患【數所角翻】太子家令潁川鼂錯上言兵事【太子家令屬詹事張晏曰太子稱家故曰家令臣瓚曰茂陵中書太子家令秩八百石潁川本韓國秦置郡漢因之鼂與朝同風俗通衛大夫史鼂之後姓譜王子朝之後錯倉故翻音錯雜之錯者非】曰兵法曰有必勝之將無必勝之民繇此觀之安邊境立功名在於良將不可不擇也【將即亮翻下同】臣又聞用兵臨戰合刃之急者三一曰得地形二曰卒服習三曰器用利兵法步兵車騎弓弩長戟矛鋋劒楯之地【師古曰鋋銕杷短矛也孔頴達曰方言云矛吳揚江淮南楚五湖之間謂之鉇或謂之鋋或謂之鏦其柄謂之矜鉇音蛇晉陳安執丈八蛇矛盖蛇即方言之所謂鉇也鋋上延翻楯食尹翻】各有所宜不得其宜者或十不當一士不選練卒不服習起居不精動靜不集趨利弗及避難不畢【趨七喻翻難乃旦翻】前擊後解與金鼓之指相失【師古曰金金鉦鼔所以進衆金所以止衆指當作音】此不習勒卒之過也百不當十兵不完利與空手同甲不堅密與袒裼同【應劭曰袒裼肉袒裼音錫】弩不可以及遠與短兵同射不能中與無矢同中不能入與無鏃同【中竹仲翻師古曰鏃矢鋒也鏃子木翻】此將不省兵之禍也【師古曰省視也悉井翻】五不當一故兵法曰器械不利以其卒予敵也卒不可用以其將予敵也將不知兵以其主予敵也君不擇將以其國予敵也【予讀曰與】四者兵之至要也臣又聞小大異形彊弱異埶險易異備【師古曰易平勢也易以豉翻下同】夫卑身以事彊小國之形也合小以攻大敵國之形也【師古曰彼我之力不能相勝則須連結外援共制之也】以蠻夷攻蠻夷中國之形也【師古曰不煩華夏之兵使其同類自相攻擊也】今匈奴地形技藝與中國異上下山阪出入溪澗中國之馬弗與也【弗與猶言不如也技渠綺翻下同】險道傾仄【仄古側字】且馳且射中國之騎弗與也風雨罷勞【罷讀曰疲】飢渴不困中國之人弗與也此匈奴之長技也若夫平原易地輕車突騎則匈奴之衆易橈亂也【師古曰突騎言其驍鋭可用衝突敵人也橈攪也音火高翻其字從手一曰橈曲也弱也音女教翻其字從木】勁弩長戟射疏及遠【師古曰疏亦濶遠也仲馮曰長戟恐誤或者勁弩如今九牛大弩以槍為矢歟故可射疏及遠也然戟有鈎又不可射予謂文意各有所屬勁弩所以射疏長戟所以及遠也】則匈奴之弓弗能格也堅甲利刃長短相雜遊弩往來什伍俱前【師古曰五人為伍十人為什】則匈奴之兵弗能當也材官騶發矢道同的【如淳曰騶矢也處平易之地可以矢相射也臣瓚曰材官騎射之官也射者騶發其用矢者同中一的言其工妙也師古曰騶矢之善者春秋傳作菆其音同耳材官有材力者騶發發騶矢以射也手工矢善故中則同的的謂所射之凖臬也騶側鳩翻】則匈奴之革笥木薦弗能支也【孟康曰革笥以皮作如鎧者被之木薦以木板作如楯一曰革笥木薦之以當人心也師古曰一說非也笥音息嗣翻】下馬地鬬劒戟相接去就相薄【薄伯各翻師古曰迫也】則匈奴之足弗能給也【師古曰給謂相連及】此中國之長技也以此觀之匈奴之長技三中國之長技五陛下又興數十萬之衆以誅數萬之匈奴衆寡之計以一擊十之術也雖然兵凶器戰危事也故以大為小以彊為弱在俛仰之間耳【師古曰言不知其術則雖大必小雖彊必弱俛亦俯字予謂俛音免亦通】夫以人之死爭勝跌而不振【服䖍曰蹉跌不可復起也師古曰跌足失據也跌徒結翻】則悔之無及也帝王之道出於萬全今降胡義渠蠻夷之屬來歸誼者其衆數千飲食長技與匈奴同賜之堅甲絮衣勁弓利矢益以邊郡之良騎令明將能知其習俗和輯其心者【師古曰輯與集同】以陛下之明約將之即有險阻以此當之平地通道則以輕車材官制之兩軍相為表裏各用其長技衡加之以衆【衡與横同】此萬全之術也帝嘉之賜錯書寵荅焉錯又上言曰臣聞秦起兵而攻胡粤者非以衛邊地而救民死也貪戾而欲廣大也故功未立而天下亂且夫起兵而不知其埶戰則為人禽屯則卒積死夫胡貉之人其性耐寒揚粤之人其性耐暑秦之戌卒不耐其水土戍者死於邊輸者僨於道【耐乃代翻服䖍曰僨仆也如淳曰僨音奮】秦民見行如往棄市因以讁發之名曰讁戌先發吏有讁及贅壻賈人後以嘗有市籍者又後以大父母父母嘗有市籍者後入閭取其左【應劭曰秦以讁發戌先自吏有過至于大父母父母嘗有市籍者曹輩盡復入閭取其左者發之未及取右而秦亡孟康曰秦時復除者居閭之左後發役不供復役之也師古從應說閭里門也居閭之左者一切發之】發之不順行者憤怨有萬死之害而亡銖兩之報【亡古無字通】死事之後不得一算之復【漢律人出一算算百二十錢】天下明知禍烈及己也【師古曰猛火曰烈取以喻耳】陳勝行戍至於大澤為天下先倡【事見七卷二世元年】天下從之如流水者秦以威刼而行之之敝也胡人衣食之業不著於地【著直畧翻】其埶易以擾亂邊境【易以豉翻】往來轉徙時至時去此胡人之生業而中國之所以離南畮也【師古曰南畮所以耕種處也離力智翻】今胡人數轉牧行獵於塞下【數所角翻】以候備塞之卒卒少則入陛下不救則邊民絶望而有降敵之心救之少發則不足多發遠縣纔至則胡又已去【師古曰纔淺也猶言僅至也他皆類此】聚而不罷為費甚大罷之則胡復入【復扶又翻】如此連年則中國貧苦而民不安矣陛下幸憂邉境遣將吏發卒以治塞甚大惠也【治直之翻】然今遠方之卒守塞一歲而更【歲更見十三卷高后五年更工衡翻】不知胡人之能不如選常居者家室田作且以備之以便為之高城深塹【因山川地形之便而為之城塹】要害之處通川之道調立城邑毋下千家【師古曰調謂筭度之也總計城邑之中令有千家以上也調徒釣翻】先為室屋具田器乃募民免罪拜爵【謂有罪者免其罪無罪者拜爵以勸其徙】復其家【謂民之欲往者復除其家征役復方目翻】予冬夏衣稟食能自給而止【師古曰初徙之時縣官且稟給其衣食於後能自供贍乃止也予讀曰與下同】塞下之民禄利不厚不可使久居危難之地【難乃旦翻】胡人入驅而能止其所驅者以其半予之【孟康曰謂胡入為寇驅收中國能奪得之者以半予之師古曰孟說非也言胡人入為寇驅畧漢人及畜產也人能止得其所驅者令其本主以半賞之】縣官為贖【張晏曰得漢人官為贖也師古曰張說非也此承上句之言謂官為備價贖之耳為于偽翻下同】其民如是則邑里相救助赴胡不避死非以德上也【師古曰言非以此事欲立德義於主上也】欲全親戚而利其財也此與東方之戍卒不習地埶而心畏胡者功相萬也【言其功萬倍于東方之戍卒也】以陛下之時徙民實邊使遠方無屯戍之事塞下之民父子相保無係虜之患利施後世名稱聖明其與秦之行怨民相去遠矣【師古曰行怨民言發怨恨之民使行戍役也】上從其言募民徙塞下錯復言陛下幸募民徙以實塞下使屯戍之事益省輸將之費益寡【如淳曰將送也或曰資也復扶又翻】甚大惠也下吏誠能稱厚惠【稱尺證翻】奉明法存卹所徙之老弱善遇其壯士和輯其心而勿侵刻使先至者安樂而不思故鄉【樂音洛】則貧民相慕而勸往矣臣聞古之徙民者相其隂陽之和【相息亮翻】嘗其水泉之味然後營邑立城製里割宅先為築室家置器物焉民至有所居作有所用此民所以輕去故鄉而勸之新邑也【之往也】為置醫巫以救疾病以修祭祀男女有昏【師古曰昏謂婚姻配合也】生死相卹墳墓相從種樹畜長【師古曰種樹謂桑果之屬張晏曰畜長六畜也貢父曰所種所樹畜積長茂予謂畜長當從張說畜許六翻長知兩翻】室屋完安此所以使民樂其處而有長居之心也【樂音洛】臣又聞古之制邊縣以備敵也使五家為伍伍有長【長知兩翻】十長一里里有假士四里一連連有假五百十連一邑邑有假候【服䖍曰假音假借之假五百帥名也師古曰假大也工雅翻仲馮曰假服說是古者戌皆有期代則不置故曰假謂其權設猶假司馬之類亦非常置也予謂五百即後所謂伍伯也賈公彦曰伍伯者漢制五人為伍伯長也沈約曰舊說古者君行師從卿行旅從旅者五百人也今諸官府至郡各置五百四以象師從旅從依古義也候即軍候也】皆擇其邑之賢材有護【師古曰有保護之能者也】習地形知民心者居則習民於射法出則教民於應敵故卒伍成於内則軍政定於外服習以成勿令遷徙【師古曰各守其業也】幼則同遊長則共事【長知兩翻】夜戰聲相知則足以相救晝戰目相見則足以相識驩愛之心足以相死如此而勸以厚賞威以重罰則前死不還踵矣【師古曰還踵囘旋其足也還音旋】所徙之民非壯有材者但費衣糧不可用也雖有材力不得良吏猶亡功也【亡古無字通】陛下絶匈奴不與和親臣竊意其冬來南也【師古曰意儗也】壹大治則終身創矣【師古曰創懲艾也初亮翻】欲立威者始於折膠【蘇林曰秋氣至膠可折弓弩可用匈奴常以為候而出軍折而設翻】來而不能困使得氣去【師古曰使之得勝逞志氣而去】後未易服也【易以豉翻】錯為人陗直刻深【師古曰陗與峭同陗謂峻陿也章笑翻韋昭曰岸高曰峭臣瓚曰陗峻陗】以其辯得幸太子太子家號曰智囊【師古曰言其一身所有皆是智算若囊槖之盛物也】<br />
<br />
  十二年冬十二月河决酸棗東潰金隄東郡大興卒塞之【班志酸棗縣屬陳留郡師古曰金隄在東郡白馬界今滑州括地志金隄一名千里隄在白馬縣東五里余據河隄自汴口以東緣河積石為堰通河古口咸曰金隄又水經注濮陽縣故城在河南與衛縣分水城北十里有瓠河口有金隄塞悉則翻】 春三月除關無用傳【張晏曰傳信也若今過所也如淳曰兩行書繒帛分持其一出入關合之乃得過謂之傳也李奇曰傳棨也師古曰張說是也古者或用棨或用繒帛棨者刻木為合符也康曰傳以木為之長尺五書符于上為信傳張戀翻】 鼂錯言於上曰聖王在上而民不凍飢者非能耕而食之織而衣之也為開其資財之道也【食祥吏翻衣於既翻為于偽翻】故堯有九年之水湯有七年之旱而國亡捐瘠者【孟康曰肉腐為瘠捐骨不埋者或曰捐謂有飢相弃捐者或謂貧乞者為捐蘇林曰瘠音漬師古曰瘠瘦病也言無相弃捐而瘦病者耳不當音漬也貧乞之釋尤疏僻焉亡古無字通】以畜積多而備先具也今海内為一土地人民之衆不減湯禹加以無天災數年之水旱而畜積未及者何也地有遺利民有餘力生穀之土未盡墾山澤之利未盡出游食之民未盡歸農也夫寒之於衣不待輕暖【師古曰苟禦風霜不求美麗也】飢之於食不待甘旨【師古曰旨美也】飢寒至身不顧廉恥人情一日不再食則飢終歲不製衣則寒夫腹飢不得食膚寒不得衣雖慈父不能保其子君安能以有其民哉明主知其然也故務民于農桑薄賦斂【斂力贍翻】廣畜積以實倉廩備水旱故民可得而有也民者在上所以牧之民之趨利如水走下四方無擇也【趨七喻翻走音奏】夫珠玉金銀飢不可食寒不可衣然而衆貴之者以上用之故也其為物輕微易藏【易以豉翻下同】在於把握可以周海内而無飢寒之患【師古曰周謂周遍而遊行】此令臣輕背其主【背蒲妹翻】而民易去其鄉盜賊有所勸亡逃者得輕資也粟米布帛生于地長於時聚於力非可一日成也【長知兩翻下同】數石之重中人弗勝【師古曰中人者處強弱之中也勝音升】不為姧邪所利一日弗得而飢寒至是故明君貴五穀而賤金玉今農夫五口之家其服役者不下二人【師古曰服事也服公事之役也】其能耕者不過百畮百畮之收不過百石春耕夏耘秋穫冬藏伐薪樵治官府給繇役【治直之翻繇與徭同後以義推】春不得避風塵夏不得避暑熱秋不得避隂雨冬不得避寒凍四時之間無日休息又私自送往迎來弔死問疾養孤長幼在其中勤苦如此尚復被水旱之災【復扶又翻被皮義翻】急政暴賦賦斂不時朝令而暮改【斂力贍翻】有者半賈而賣【師古曰本直千錢止得五百也賈讀曰價】無者取倍稱之息【如淳曰取一償二為倍稱師古曰稱舉也今俗所謂舉錢者也予謂如說是稱尺證翻】于是有賣田宅鬻妻子以償責者矣而商賈大者積貯倍息小者坐列販賣【師古曰行賣曰商坐販曰賈列市列也若今市中賣物行也賈音古貯丁呂翻】操其奇贏日游都市【操千高翻師古曰奇贏謂有餘財而蓄聚奇異之物也一說奇謂殘餘物也居宜翻】乘上之急所賣必倍【師古曰上所急求則其價倍貴】故其男不耕耘女不蠶織衣必文采食必粱肉【師古曰粱好粟也即今之粱米】無農夫之苦有仟伯之得【師古曰仟謂千錢伯謂百錢也伯莫白翻今俗猶謂百錢為一伯】因其富厚交通王侯力過吏埶以利相傾千里游敖冠盖相朢乘堅策肥履絲曳縞【乘堅車策肥馬師古曰堅謂好車也縞皓素也繒之精白者也】此商人所以兼并農人農人所以流亡者也方今之務莫若使民務農而已矣欲民務農在于貴粟貴粟之道在于使民以粟為賞罰今募天下入粟縣官得以拜爵得以除罪如此富人有爵農民有錢粟有所渫【師古曰渫散也先列翻】夫能入粟以受爵皆有餘者也取於有餘以供上用則貧民之賦可損【師古曰損減也】所謂損有餘補不足令出而民利者也今令民有車騎馬一匹者復卒三人【如淳曰復三卒之算錢也或曰除三夫不作甲卒也師古曰當為卒者免其三人不為卒者復其錢復方目翻】車騎者天下武備也故為復卒【為于偽翻】神農之教曰有石城十仞【應劭曰仞六尺五寸也師古曰此說非也八尺曰仞取人伸臂之一尋也】湯池百步【師古曰池城邊池也以沸湯為池不可輒近言嚴固之甚】帶甲百萬而無粟弗能守也以是觀之粟者王者大用政之本務今民入粟受爵至五大夫以上【師古曰五大夫第九爵】乃復一人耳此其與騎馬之功相去遠矣爵者上之所擅【師古曰擅專也】出於口而無窮粟者民之所種生於地而不乏夫得高爵與免罪人之所甚欲也使天下人入粟于邊以受爵免罪不過三歲塞下之粟必多矣帝從之令民入粟於邊拜爵各以多少級數為差【時令入粟六百石爵上造稍增至四千石為五大夫萬二千石為大庶長】錯復奏言陛下幸使天下入粟塞下以拜爵甚大惠也【復扶又翻下同】竊恐塞卒之食不足用大渫天下粟邊食足以支五歲可令入粟郡縣矣【師古曰入諸郡縣以備凶災也】郡縣足支一歲以上可時赦勿收農民租如此德澤加於萬民民愈勤農大富樂矣【樂音洛】上復從其言詔曰道民之路在於務本【道讀曰導】朕親率天下農十年于今而野不加辟【師古曰辟讀曰闢開也】歲一不登民有飢色【師古曰登成也言五穀一歲不成則衆庶飢餒是無蓄積故也】是從事焉尚寡【師古曰從事謂從農事也】而吏未加務吾詔書數下歲勸民種樹而功未興是吏奉吾詔不勤而勸民不明也且吾農民甚苦而吏莫之省將何以勸焉【數所角翻省悉井翻】其賜農民今年租税之半<br />
<br />
  十三年春二月甲寅詔曰朕親率天下農耕以供粢盛【稷曰明粢在器曰盛盛時征翻】皇后親桑以供祭服其具禮儀 初秦時祝官有祕祝【應劭曰祕祝之官移過于下國家諱之故曰祕也】即有災祥輒移過於下夏詔曰盖聞天道禍自怨起而福繇德興百官之非宜由朕躬今祕祝之官移過於下以彰吾之不德朕甚弗取其除之 齊太倉令淳于意有罪【太倉令齊王國官也姓譜淳于出于姜姓州公之後】當刑詔獄逮繫長安【師古曰逮及也辭之所及則追捕之故謂之逮一曰逮者在道將送防禦不絶若今之傳送囚】其少女緹縈上書曰【師古曰緹他弟翻索隱音啼縈於營翻】妾父為吏齊中皆稱其廉平今坐法當刑妾傷夫死者不可復生刑者不可復屬【夫音扶復扶又翻下同師古曰屬聨也之欲翻】雖後欲改過自新其道無繇也【繇古由字通用】妾願没入為官婢【漢制永巷令典官婢】以贖父刑罪使得自新天子憐悲其意五月詔曰詩曰愷弟君子民之父母【師古曰大雅泂酌之詩也言君子有和樂簡易之德則其下尊之如父親之如母也】今人有過教未施而刑已加焉或欲改行為善而道無繇至朕甚憐之夫刑至斷支體刻肌膚終身不息【行下孟翻斷端管翻師古曰息生也】何其刑之痛而不德也豈為民父母之意哉其除肉刑有以易之及令罪人各以輕重不亡逃有年而免【孟康曰其不亡逃者滿其年數得免為庶人】具為令【師古曰使更為條例】丞相張蒼御史大夫馮敬奏請定律曰諸當髠者為城旦舂【髠也謂去其髮及其耏髩應劭曰城旦者旦起行治城舂者婦人不豫外徭但舂作米皆四歲刑也】當黥髠者鉗為城旦舂【鉗者以鐵束其頸】當劓者笞三百當斬左止者笞五百當斬右止及殺人先自吿及吏坐受賕枉法守縣官財物而即盜之已論而復有笞罪者皆棄市【師古曰止足也當斬右足者以其罪次重故從棄市也殺人先自告謂殺人先自首得免罪者也吏受賕枉法謂受賂而曲公法者也守縣官財物而即盜之即今律所謂主守自盜者也殺人害重受賕盜物贓汙之身故此三罪已被論而又犯笞亦皆棄市】罪人獄已决為城旦舂者各有歲數以免【城旦舂滿三歲為鬼薪白粲鬼薪白粲一歲為隸臣妾隸臣妾一歲免為庶人隸臣妾滿二歲為司寇司寇一歲及作如司寇二歲皆免為庶人】制曰可是時上既躬修玄默而將相皆舊功臣少文多質懲惡亡秦之政【惡烏路翻】論議務在寛厚恥言人之過失化行天下吿訐之俗易【自下告上曰訐師古曰面相斥罪也居謁翻】吏安其官民樂其業畜積歲增戶口寖息風流篤厚禁罔疏闊【疏與疎同】罪疑者予民【師古曰從輕斷予讀曰與】是以刑罰大省至于斷獄四百【師古曰謂普天之下重罪者也斷丁亂翻】有刑錯之風焉【應劭曰錯置也民不犯法無所刑也錯千故翻】 六月詔曰農天下之本務莫大焉今勤身從事而有租税之賦是為本末者無以異也【李奇曰本農也末賈也言農與賈俱出租無異也故除田租】其於勸農之道未備其除田之租税<br />
<br />
  十四年冬匈奴老上單于十四萬騎入朝那蕭關【班志朝那縣屬安定郡應劭曰史記故戎郡邑也蕭關在朝那界唐屬原州之境後置蕭關縣為武州治所史記正義曰蕭關今古隴山關在原州平凉縣界】殺北地都尉卬【徐廣曰卬姓段師古曰非也姓孫卬五郎翻】虜人民畜產甚多遂至彭陽使奇兵入燒囘中宫候騎至雍甘泉【班志彭陽縣屬安定郡師古曰即今彭原縣括地志彭陽縣故城在今涇州臨涇縣東二十里彭原寧州雍縣班志屬扶風騎奇寄翻下同】帝以中尉周舍郎中令張武為將軍發車千乘【乘繩證翻】騎卒十萬軍長安旁以備胡寇而拜昌侯盧卿為上郡將軍甯侯魏遫為北地將軍隆慮侯周竈為隴西將軍屯三郡【昌侯盧卿功臣表作旅卿古字借用也姓譜姜姓之後封于盧以國為氏與甯侯隆慮侯皆高祖功臣昌侯國屬琅邪郡甯侯國在河内修武縣界隆慮侯國亦屬河内郡三人分屯三郡故各以郡為將軍號遫古速字】上親勞軍勒兵申教令賜吏卒自欲征匈奴羣臣諫不聽皇太后固要上乃止【勞力到翻文穎曰要刼也哀痛祝誓之言予謂固要力止也要讀曰邀康力笑翻非也】于是以東陽侯張相如為大將軍成侯董赤【成侯董赤高帝功臣董渫之子成侯國屬涿郡赤史記正義音赫】内史欒布皆為將軍擊匈奴單于留塞内月餘乃去漢逐出塞即還不能有所殺上輦過郎署問郎署長馮唐曰【署郎舍也長知兩翻】父家何在對曰臣大父趙人父徙代上曰吾居代時吾尚食監高袪【尚食監主膳食之官袪音區】數為我言趙將李齊之賢戰于鉅鹿下【當是秦將王離圍鉅鹿時數所角翻為于偽翻】今吾每飯意未嘗不在鉅鹿也【每食時念高袪所言其心未嘗不在鉅鹿】父知之乎唐對曰尚不如廉頗李牧之為將也上摶髀曰【搏拊也左傳曰摶膺而踊髀音陛】嗟乎吾獨不得廉頗李牧為將吾豈憂匈奴哉唐曰陛下雖得廉頗李牧弗能用也上怒起入禁中良久召唐讓曰公奈何衆辱我獨無間處乎【師古曰何不於隙間之處而言】唐謝曰鄙人不知忌諱上方以胡寇為意乃卒復問唐曰【卒子恤翻】公何以知吾不能用廉頗李牧也唐對曰臣聞上古王者之遣將也跪而推轂曰閫以内者寡人制之閫以外者將軍制之【推吐雷翻閫苦本翻門橛也】軍功爵賞皆决於外歸而奏之此非虚言也臣大父言李牧為趙將居邊軍市之租【素隱曰軍中立市市有税税即租也】皆自用饗士賞賜决于外不從中覆也【師古曰覆謂覆白之也一說不從中覆校其所用之數亦通】委任而責成功故李牧乃得盡其智能選車千三百乘【乘繩證翻】彀騎萬三千百金之士十萬【弓弩引滿為彀謂騎兵能射者服䖍曰良士直百金晉灼曰百金喻貴重也彀古候翻騎寄奇翻】是以北逐單于破東胡滅澹林【澹林即襜襤澹丁甘翻】西抑彊秦南支韓魏當是之時趙幾霸【幾居依翻】其後會趙王遷立用郭開讒卒誅李牧【事見六卷始皇八年卒子恤翻】 令顔聚代之是以兵破士北為秦所禽滅今臣竊聞魏尚為雲中守【守式又翻】其軍市租盡以饗士卒私養錢【服䖍曰私廩假錢索隱曰按漢市肆租税之入為私奉養服䖍曰私廩假錢是也或云官所别給也予謂當從漢書以私養錢屬下句】五日一椎牛自饗賓客軍吏舍人是以匈奴遠避不近雲中之塞【近其靳翻】虜曾一入尚率車騎擊之所殺甚衆夫士卒盡家人子【師古曰家人子謂庶人家之子也】起田中從軍安知尺籍伍符【李奇曰尺籍所以書軍令伍符軍士伍伍相保之符信也如淳曰漢軍法曰吏卒斬首以尺籍書下縣移郡令人故行不行奪勞二歲伍符亦什五之符要節度也或曰以尺簡書故曰尺籍也索隱曰按尺籍者謂書其斬首之功於一尺之板伍符者今軍人伍伍相保不容奸詐也】終日力戰斬首捕虜上功幕府【上時掌翻】一言不相應【索隱曰應一陵翻謂數不同也予謂相應之應當從去聲】文吏以法繩之其賞不行而吏奉法必用臣愚以為陛下賞太輕罸太重且雲中守魏尚坐上功首虜差六級陛下下之吏削其爵罰作之【蘇林曰一歲刑為罰作下之遐嫁翻】由此言之陛下雖得廉頗李牧弗能用也上說【說讀曰悦】是日令唐持節赦魏尚復以為雲中守而拜唐為車騎都尉【詳考班表漢無車騎都尉官時使唐主中尉及郡國車士】 春詔廣增諸祀壇塲珪幣【師古曰築土為壇除土為塲珪幣所以薦神】且曰吾聞祠官祝釐【如淳曰釐福也師古曰釐本作禧假借用耳音禧祝職救翻】皆歸福於朕躬不為百姓【為于偽翻】朕甚愧之夫以朕之不德而專饗獨美其福百姓不與焉是重吾不德也【與讀曰預師古曰重直用翻】其令祠官致敬無有所祈 是歲河間文王辟彊薨 初丞相張蒼以為漢得水德魯人公孫臣以為漢當土德其應黄龍見蒼以為非罷之【公孫臣上書曰始秦得水德推終始傳漢當土德土德之應黄龍見宜改正朔服色尚黄張蒼以為漢乃水德河決金隄其符也公孫臣言非是罷之見賢遍翻】<br />
<br />
  十五年春黄龍見成紀【班志成紀縣屬天水郡庖犧所生處見賢遍翻】帝召公孫臣拜為博士與諸生申明土德草改歷服色事【師古曰草謂創造之】張蒼由此自絀 夏四月上始幸雍郊見五帝【秦立白帝赤帝黄帝青帝畤于雍漢高帝又立黑帝畤故雍有五帝畤雍于用翻見賢遍翻】赦天下九月詔諸侯王公卿郡守舉賢良能直言極諫者【守式又翻】上親策之太子家令鼂錯對策高第擢為中大夫錯又上言宜削諸侯及法令可更定者【更工衡翻】書凡三十篇上雖不盡聽然奇其材 是歲齊文王則河間哀王福皆薨無子國除【齊王則哀王襄之子悼惠王肥之孫河間王福辟彊之子趙幽王子遂之孫】 趙人新垣平以望氣見上言長安東北有神氣成五采于是作渭陽五帝廟【韋昭曰在渭城師古曰郊祀志云在長安東北非渭城也韋說謬矣予據水北為陽長安在渭南渭城在渭北五帝廟或在渭城界韋說未可非也括地志渭陽五帝廟在雍州咸陽縣東三十里】<br />
<br />
  十六年夏四月上郊祀五帝于渭陽五帝廟于是貴新垣平至上大夫【周官有上大夫漢官有太中大夫中大夫諫大夫爵十九級有大夫五大夫而上大夫不見於表】賜累千金而使博士諸生刺六經中作王制【師古曰刺采取也七賜翻即今禮記王制篇是也】謀議巡狩封禪事又於長門道北立五帝壇【如淳曰長門亭名在長安城東南括地志長門故亭在雍州萬年縣東北苑中】徙淮南王喜復為城陽王又分齊為六國丙寅立齊悼惠王子在者六人楊虚侯將閭為齊王安都侯志為濟北王武城侯賢為菑川王白石侯雄渠為膠東王平昌侯卬為膠西王扐侯辟光為濟南王淮南厲王子在者三人阜陵侯安為淮南王安陽侯勃為衡山王陽周侯賜為廬江王【十一年徙城陽王喜王淮南今復其舊將復以淮南地分王厲王三子安勃賜也楊虛據水經河水過楊虚縣註引地理志曰楊虚平原之隸縣也城在高唐之西南而班志無此縣不知酈道元所謂志者何志也史記正義曰安都故城在瀛州高陽縣西南三十九里濟北王都盧武成史記作武城索隱曰武城縣屬平原正義曰具州縣菑川王都劇班志金城郡有白石縣正義曰白石故城在德州安德縣北二十里膠東王都即墨班志平昌侯國屬平原郡膠西王都高苑扐侯國屬平原郡濟南王都東平陵阜陵縣屬九江郡淮南王都夀春安陽屬汝南郡衡山王都六陽周縣屬上郡廬江王都江南濟子禮翻扐音力】秋九月新垣平使人持玉杯上書闕下獻之平言上曰闕下有寶玉氣來者已視之果有獻玉杯者刻曰人主延夀平又言臣候日再中居頃之日卻復中於是始更以十七年為元年【復扶又翻】令天下大酺【漢律三人無故羣飲罸金四兩今詔横賜得會聚飲食師古曰酺布也言王德布于天下而合聚飲食為酺周禮族師春秋祭酺注酺者為人烖害之神也有馬酺有蝝螟之酺與人思之酺亦為壇位如雩禜族長無飲酒之禮因祭酺而與其民以長幼相獻酬焉正義曰古者祭酺聚錢飲酒故後世聽民聚飲皆謂之酺漢書每有嘉慶令民大酺是其事也彼注云因祭酺而與其民長幼相酬鄭注所謂祭酺合醵也酺音蒲】平言曰周鼎亡在泗水中今河決通於泗臣望東北汾隂直有金寶氣【班志汾隂縣屬河東郡師古曰直謂正當汾隂也宋白曰蒲州寶鼎縣古綸氏地夏少康所邑也汾水南流過縣漢置汾隂縣今縣北九十里汾隂故城是也】意周鼎其出乎兆見不迎則不至于是上使使治廟汾隂南臨河欲祠出周鼎【見賢遍翻】<br />
<br />
  後元年冬十月人有上書吿新垣平所言皆詐也下吏治誅夷平【師古曰夷者平也謂盡平除其家室宗族下遐嫁翻】是後上亦怠於改正服鬼神之事【師古曰正正朔也服服色也正之成翻】而渭陽長門五帝使祠官領以時致禮不往焉 春三月孝惠皇后張氏薨【孝惠皇后張敖之女諸呂之誅徙居北宫張晏曰后黨于呂氏故不曰崩】 詔曰間者數年不登又有水旱疾疫之災朕甚憂之愚而不明未達其咎意者朕之政有所失而行有過與【與與歟同下同】乃天道有不順地利或不得人事多失和鬼神廢不享與何以致此將百官之奉養或廢無用之事或多與何其民食之寡乏也夫度田非益寡【師古曰度謂量計之度徒各翻】而計民未加益以口量地【量音良】其於古猶有餘而食之甚不足者其咎安在無乃百姓之從事于末以害農者蕃【師古曰蕃多也扶元翻】為酒醪以靡穀者多【師古曰醪汁滓酒也靡散也醪來高翻靡音糜】六畜之食焉者衆與【六畜馬牛羊犬豕雞畜許救翻】細大之義吾未得其中【師古曰中竹仲翻】其與丞相列侯吏二千石博士議之有可以佐百姓者率意遠思無有所隱<br />
<br />
  二年夏上行幸雍棫陽宫【黄圖曰棫陽宫秦昭王所起括地志在岐州扶風縣東北棫音域】 六月代孝王參薨【參前二年封于太原三年徙代】 匈奴連歲入邊殺畧人民畜產甚多雲中遼東最甚【遼東戰國時燕之東北境秦置郡】郡萬餘人上患之乃使使遺匈奴書【遺于季翻】單于亦使當戶報謝【匈奴官自左右賢王至左右大當戶凡二十四長】復與匈奴和親八月戊戌丞相張蒼免帝以皇后弟竇廣國賢有行<br />
<br />
  欲相之【相息亮翻行下孟翻】曰恐天下以吾私廣國久念不可而高帝時大臣餘見無可者【謂高帝大臣薨逝之餘其見存之臣無可相者見賢遍翻】御史大夫梁國申屠嘉【莊子有申徒狄夏之賢人也一曰申徒楚官號姓譜申侯之後支子居安定屠原因為申屠氏】故以材官蹶張從高帝【梁國本秦碭郡漢為梁國如淳曰材官之多力者能脚蹋強弩張之故曰蹶張律有蹶張士師古曰今之弩以手張者為擘張以足蹋者為蹶張蹶音厥】封關内侯庚午以嘉為丞相封故安侯【班志故安縣屬涿郡括地志今易州界武陽城中東南隅故城是也】嘉為人廉直門不受私謁是時太中大夫鄧通方愛幸賞賜累鉅萬帝嘗燕飲通家其寵幸無比嘉嘗入朝而通居上旁有怠慢之禮嘉奏事畢因言曰陛下幸愛羣臣則富貴之至于朝廷之禮不可以不肅【師古曰肅敬也】上曰君勿言吾私之【師古曰言欲私告戒之】罷朝坐府中【風俗通府聚也公卿牧守道德之所聚也又舍也】嘉為檄召通詣丞相府【師古曰檄木書也長二尺】不來且斬通通恐入言上上曰汝第往吾今使人召若通詣丞相府免冠徒跣頓首謝嘉嘉坐自如弗為禮責曰夫朝廷者高帝之朝廷也通小臣戲殿上大不敬當斬吏今行斬之【如淳曰嘉語其吏曰今便行斬之】通頓首首盡出血不解上度丞相已困通【度徒洛翻】使使持節召通而謝丞相此吾弄臣君釋之鄧通既至為上泣曰丞相幾殺臣【為于偽翻幾居希翻】<br />
<br />
  三年春二月上行幸代 是歲匈奴老上單于死子軍臣單于立<br />
<br />
  四年夏四月丙寅晦日有食之【月末為晦天文書晦則日月相沓月在日後則光體伏矣】 五月赦天下 上行幸雍<br />
<br />
  五年春正月上行幸隴西三月行幸雍秋七月行幸代六年冬匈奴三萬騎入上郡三萬騎入雲中所殺畧甚衆烽火通於甘泉長安【文穎曰邊方備胡寇作高土櫓上作桔橰桔橰頭兜零以薪置其中常低之有寇即然火舉之以相告曰烽又多積薪寇至即然之以望其烟曰燧】以中大夫令免為車騎將軍屯飛狐【師古曰中大夫官名其人姓令名免耳此諸將軍下至徐厲皆書姓而徐廣以為中大夫令是官名此說非也據百官表景帝初改衛尉為中大夫令文帝時無此官而中大夫是郎中令屬官秩比二千石索隱曰據風俗通令姓楚令尹子文之後虞世南曰中大夫令是史家追書耳】故楚相蘇意為將軍屯句注【句音鈎】將軍張武屯北地【秦滅義渠置北地郡】河内太守周亞夫為將軍次細柳【項羽以河内郡為殷國高帝滅殷復置河内郡服䖍曰細柳在長安西北如淳曰長安細柳倉在渭北近石徼張揖曰在昆明池南今有柳市是也臣瓚曰一宿曰宿再宿曰信過信為次師古曰匈奴傳云置三將軍軍長安西細柳渭北棘門覇上此則細柳不在渭北揖說是也索隱曰按三輔故事細柳在直城門外阿房宫西北維舊唐書肅宗母元獻楊后葬細柳原】宗正劉禮為將軍次覇上祝兹侯徐厲為將軍次棘門【宗正秦官掌親屬漢因之徐厲高祖功臣呂后四年封祝兹侯史記表作松滋班志松滋縣屬廬江郡孟康曰棘門在長安北秦時宫門也如淳曰棘門在横門外横門長安城北出西頭第一門】以備胡上自勞軍至覇上及棘門軍直馳入將以下騎送迎【勞力到翻將即亮翻下其將同騎奇寄翻】已而之細柳軍【之往也】軍士吏被甲鋭兵刃彀弓弩持滿【被皮義翻彀古候翻】天子先驅至不得入先驅曰【師古曰先驅導駕若今之武侯隊矣】天子且至軍門都尉曰將軍令曰軍中聞將軍令不聞天子之詔居亡何上至又不得入于是上乃使使持節詔將軍吾欲入營勞軍亞夫乃傳言開壁門壁門士請車騎曰將軍約軍中不得馳驅于是天子乃按轡徐行至營將軍亞夫持兵揖曰介胄之士不拜請以軍禮見【禮介者不拜見賢遍翻】天子為動改容式車【為于偽翻】使人稱謝皇帝敬勞將軍成禮而去既出軍門羣臣皆驚上曰嗟乎此真將軍矣曩者覇上棘門軍若兒戲耳其將固可襲而虜也至于亞夫可得而犯邪稱善者久之月餘漢兵至邊匈奴亦遠塞【遠于願翻】漢兵亦罷乃拜周亞夫為中尉【為以亞夫屬太子張本】 夏四月大旱蝗【師古曰蝗即螽也食苗為災今俗呼為簸蝩說文曰一曰蝝一曰蝗蝗戶光翻蝩音鍾】令諸侯無入貢弛山澤【師古曰弛解也解而不禁與衆庶同其利】減諸服御損郎吏員發倉庾以振民【應劭曰水漕倉曰庾胡公曰在邑曰倉在野曰庾康曰凡倉無屋曰庾】民得賣爵<br />
<br />
  七年夏六月已亥帝崩于未央宫【臣瓚曰夀四十六】遺詔曰朕聞之盖天下萬物之萌生靡不有死死者天地之理萬物之自然奚可甚哀當今之世咸嘉生而惡死【惡烏路翻】厚葬以破業重服以傷生吾甚不取且朕既不德無以佐百姓今崩又使重服久臨【師古曰臨哭也力禁翻下出臨服臨當臨夕臨哭臨音同】以罹寒暑之數【師古曰罹音離遭也】哀人父子傷長老之志損其飲食絶鬼神之祭祀以重吾不德謂天下何朕獲保宗廟以眇眇之身【師古曰眇眇猶言細末也】託于天下君王之上二十有餘年矣賴天之靈社稷之福方内安寧【方内四方之内也】靡有兵革朕既不敏常懼過行以羞先帝之遺德【師古曰過行行有過失也羞謂忝辱也行下孟翻】惟年之久長懼于不終今乃幸以天年得復供養于高廟其奚哀念之有【帝自謙以謂得終其天年以從先帝幸矣奚哀念之有乎供居用翻養羊亮翻】其令天下吏民令到出臨三日皆釋服毋禁取婦嫁女祠祀飲酒食肉【取讀曰娶】自當給喪事服臨者皆無跣【跣先典翻足親地也】絰帶毋過三寸毋布車及兵器【應劭曰毋以布衣車及兵器也服䖍曰不施輕車介士也師古曰應說是也】毋發民哭臨宫殿中殿中當臨者皆以旦夕各十五舉音禮畢罷非旦夕臨時禁毋得擅哭臨已下棺服大功十五日小功十四日纎七日釋服【喪禮大功之服七升八升九升小功十升十一升十二升再朞而大祥踰月而禫禫而纎無所不佩鄭注云大祥除衰杖黑經白緯曰纎舊說纎冠者采纓也無所不佩者紛帨之屬如平常也孔氏正義曰禫而纎者禫祭之時玄冠朝服禫祭既訖而首著纎冠身著素端黄裳以至吉祭無所不佩者吉祭之時身尋常吉服平常所服之物無不佩也服䖍曰大功小功布也纎細布衣也應劭曰凡三十六日而釋服矣此以日易月也師古曰此喪制者文帝自率己意創而為之非有取于周禮也何為以日易月乎三年之喪其實二十七月豈有三十六月之文禫又無七月也應氏既失之於前近代學者因循繆說未之思也貢父曰文帝制此喪服斷自已葬之後其未葬之前則服斬衰漢諸帝自崩至葬有百餘日者未葬則服不除矣翟方進傳後母終既葬三十六日起視事其證也說者遂以日易月又不通計葬之日皆大謬也攷之文帝意既葬除重服制大功小功所以漸即吉耳賈公彦曰布之精麄斬衰三升齊衰有三等或四升或五升或六升小功大功如前說緦麻十五升抽去半朝服十五升】它不在令中者皆以此令比類從事【師古曰言此詔中無文者皆以類比而行事】布告天下使明知朕意覇陵山川因其故毋有所改【應劭曰因山為藏不復起墳山下川流不遏絶就其水名以為陵號耳師古曰覇陵在長安東南】歸夫人以下至少使【應劭曰夫人已下有美人良人八子七子長使少使皆遣歸家重絶人類】乙巳葬覇陵帝即位二十三年宫室苑囿車騎服御無所增益有不便輒弛以利民嘗欲作露臺召匠計之直百金上曰百金中人十家之產也吾奉先帝宫室嘗恐羞之何以臺為【師古曰中謂不富不貧今新豐縣南驪山之頂有露臺鄉極為高顯猶有文帝所欲作臺之處】身衣弋綈【如淳曰弋皁也師古曰弋黑色衣於既翻】所幸慎夫人衣不曳地帷帳無文繡以示敦朴為天下先治霸陵皆瓦器不得以金銀銅錫為飾因其山不起墳【古者墓而不墳墳者聚土使之高大也皇甫謚曰漢長陵高十三丈陽陵高十四丈安陵三十餘丈則不度甚矣治直之翻】吳王詐病不朝賜以几杖羣臣袁盎等諫說雖切常假借納用焉張武等受賂金錢覺更加賞賜以愧其心專務以德化民是以海内安寧家給人足後世鮮能及之【鮮息淺翻】 丁未太子即皇帝位【鄭樵曰漢大斂畢三公奏尚書顧命太子即日即天子位于柩前請太子即皇帝位皇后為皇太后奏可羣臣皆出吉服入會如儀太尉升自阼階當柩御坐北面稽首讀冊畢以傳國玉璽綬東面跪授皇太子即皇帝位告令羣臣羣臣皆伏稱萬歲或大赦天下羣臣百僚罷入成喪服如禮】尊皇太后薄氏曰太皇太后皇后曰皇太后【帝祖母曰太皇太后帝母曰皇太后】 九月有星孛于西方【孛蒲内翻】是歲長沙王吳著薨無子國除【高帝封吳芮為長沙王傳成王臣共王囘共王右至著而絶著漢書作差】初高祖賢文王芮制詔御史長沙王忠其定著令【鄧展曰漢約非劉氏不王而芮王故著令使特王也或曰以芮至忠故著令也仲馮曰兼用鄧二說乃著令之意也貢父曰長沙王忠其定著令定著令者謂於令著長沙王車服土地之類也】至孝惠高后時封芮庶子二人為列侯傳國數世絶<br />
<br />
  孝景皇帝上【荀悦曰諱啓之字曰開文帝長子也應劭曰禮諡法布義行剛曰景】<br />
<br />
  元年冬十月丞相嘉等奏功莫大於高皇帝德莫盛於孝文皇帝高皇帝廟宜為帝者太祖之廟孝文皇帝廟宜為帝者太宗之廟天子宜世世獻祖宗之廟郡國諸侯宜各為孝文皇帝立太宗之廟【應劭曰始取天下者曰祖高帝稱高祖是也始治天下者曰宗文帝稱太宗是也師古曰應說非也祖始也始受命也宗尊也有德可尊貢父曰顔說非也始受命稱太祖耳有功亦稱祖商祖甲是也】制曰可 夏四月乙卯赦天下遣御史大夫青至代下與匈奴和親【開封侯陶青高祖功臣陶舍之】<br />
<br />
  【子】 五月復收民田半租【文帝十二年賜民田租之半次年盡除田之租税今復收半租】三十而税一 初文帝除肉刑【事見文帝十三年】外有輕刑之名内實殺人斬右止者又當死斬左止者笞五百當劓者笞三百率多死是歲下詔曰加笞與重罪無異【孟康曰重罪謂死刑】幸而不死不可為人【師古曰謂不能自起居也】其定律笞五百曰三百笞三百曰二百以太中大夫周仁為郎中令張歐為廷尉【孟康曰歐音驅索隱曰於后翻】楚元王子平陸侯禮為宗正【平陸戰國時齊邑班志東平國有東平陸縣又西河郡有平陸縣意禮所封者齊地】中大夫鼂錯為左内史仁始為太子舍人【内史掌治京邑武帝建元六年始分左右内史疑左字衍續漢志太子舍人更直宿衛如三置郎中鼂音朝直遥翻】以廉謹得幸張歐亦事帝於太子宫雖治刑名家為人長者【治直之翻長知兩翻】帝由是重之用為九卿歐為吏未嘗言按人專以誠長者處官官屬以為長者亦不敢大欺【處昌呂翻】<br />
<br />
  二年冬十二月有星孛于西南【孛蒲内翻】令天下男子年二十始傅【師古曰舊制二十三而傅今此二十更為異制也傅讀曰附】 春三月甲寅立皇子德為河間王閼為臨江王餘為淮陽王非為汝南王彭祖為廣川王發為長沙王【河間王都樂成臨江王都江陵淮陽王都陳汝南王都平輿廣川王都信都長沙王都長沙閼一曷翻】 夏四月壬午太皇太后薄氏崩【薄太皇文帝母也】 六月丞相申屠嘉薨時内史鼂錯數請間言事輒聽寵幸傾九卿【漢正卿九奉常郎中令衛尉太僕廷尉典客宗正治粟内史少府是也數所角翻】法令多所更定【更工衡翻】丞相嘉自絀所言不用疾錯錯為内史東出不便更穿一門南出南出者太上皇廟堧垣也【三輔黄圖太上皇廟在長安香室街南馮翊府北武帝分内史為左右後又改左内史為左馮翊括地志漢太上皇廟在雍州長安縣西北長安故城中酒池之北服䖍曰堧垣宫外垣餘地也師古曰内垣之外餘地也堧而緣翻】嘉聞錯穿宗廟垣為奏請誅錯客有語錯【語牛倨翻】錯恐夜入宫上謁自歸上【上謁時掌翻】至朝【朝直遥翻下同】嘉請誅内史錯上曰錯所穿非真廟垣乃外堧垣故冗官居其中【師古曰冗謂散輩也如今之散官冗如隴翻】且又我使為之錯無罪丞相嘉謝罷朝嘉謂長史曰吾悔不先斬錯乃請之為錯所賣至舍因歐血而死【歐於后翻】錯以此愈貴 秋與匈奴和親 八月丁未以御史大夫開封侯陶青為丞相【班志開封縣屬河南郡姓譜陶陶唐氏之後】丁巳以内史鼂錯為御史大夫 彗星出東北【彗祥歲翻又徐醉翻又旋芮翻】秋衡山雨雹大者五寸深者二尺【大戴禮曰孔會子云陽之專氣為霰隂之專氣為雹盛陽之氣在雨水則温暖而為雨隂氣薄而脅之不相入則摶而為雹也盛隂之氣在雨水則凝滯而為雪陽氣薄而脅之不相入則消散而下因水而為霰雨于具翻】 熒惑逆行守北辰月出北辰間歲星逆行天廷中【熒惑火星北辰中宫天極星也月有九行黑道二出黄道北自立冬冬至行之青道二出黄道東立春春分行之赤道二出黄道南立夏夏至行之白道二出黄道西立秋秋分行之其去極有遠近終不能出北辰之間出北辰間失其行也歲星木星也太微為天廷摠天文志北極及太微人君之位或守之或出之或逆行經之皆變也又石氏星傳曰龍左角為天田右角為天廷孔穎達曰春秋緯文紫微宫為大帝太微為天廷中有五帝座】 梁孝王以竇太后少子故有寵王四十餘城【少詩沼翻王于况翻】居天下膏腴地賞賜不可勝道府庫金錢且百鉅萬【鉅萬萬萬也勝音升】珠玉寶器多於京師築東苑方三百餘里廣睢陽城七十里【唐宋州治宋城縣即漢睢陽】大治宫室【治直之翻】為複道自宫連屬於平臺三十餘里【如淳曰平臺在梁東北離宫所在師古曰今其城東二十里所有故臺基其處寛博俗云平臺也屬之欲翻】招延四方豪俊之士如吳人枚乘嚴忌齊人羊勝公孫詭鄒陽蜀人司馬相如之屬皆從之遊【姓譜枚姓也六國有賢人枚被嚴忌本姓莊漢書避明帝諱改為嚴羊晉羊舌大夫之後鄒以國為氏】每入朝上使使持節以乘輿駟馬迎梁王於關下既至寵幸無比入則侍上同輦出則同車射獵上林中因上疏請留且半歲梁侍中郎謁者著籍引出入天子殿門【史記正義曰籍謂名簿也若今通引出入門也著竹畧翻】與漢宦官無異<br />
<br />
  資治通鑑卷十五  <br>
   </div> 

<script src="/search/ajaxskft.js"> </script>
 <div class="clear"></div>
<br>
<br>
 <!-- a.d-->

 <!--
<div class="info_share">
</div> 
-->
 <!--info_share--></div>   <!-- end info_content-->
  </div> <!-- end l-->

<div class="r">   <!--r-->



<div class="sidebar"  style="margin-bottom:2px;">

 
<div class="sidebar_title">工具类大全</div>
<div class="sidebar_info">
<strong><a href="http://www.guoxuedashi.com/lsditu/" target="_blank">历史地图</a></strong>  
<a href="http://www.880114.com/" target="_blank">英语宝典</a>  
<a href="http://www.guoxuedashi.com/13jing/" target="_blank">十三经检索</a> 
<br><strong><a href="http://www.guoxuedashi.com/gjtsjc/" target="_blank">古今图书集成</a></strong> 
<a href="http://www.guoxuedashi.com/duilian/" target="_blank">对联大全</a> <strong><a href="http://www.guoxuedashi.com/xiangxingzi/" target="_blank">象形文字典</a></strong> 

<br><a href="http://www.guoxuedashi.com/zixing/yanbian/">字形演变</a>  <strong><a href="http://www.guoxuemi.com/hafo/" target="_blank">哈佛燕京中文善本特藏</a></strong>
<br><strong><a href="http://www.guoxuedashi.com/csfz/" target="_blank">丛书&方志检索器</a></strong> <a href="http://www.guoxuedashi.com/yqjyy/" target="_blank">一切经音义</a>  

<br><strong><a href="http://www.guoxuedashi.com/jiapu/" target="_blank">家谱族谱查询</a></strong>  <strong><a href="http://shufa.guoxuedashi.com/sfzitie/" target="_blank">书法字帖欣赏</a></strong> 
<br>

</div>
</div>


<div class="sidebar" style="margin-bottom:0px;">

<font style="font-size:22px;line-height:32px">QQ交流群9:489193090</font>


<div class="sidebar_title">手机APP 扫描或点击</div>
<div class="sidebar_info">
<table>
<tr>
	<td width=160><a href="http://m.guoxuedashi.com/app/" target="_blank"><img src="/img/gxds-sj.png" width="140"  border="0" alt="国学大师手机版"></a></td>
	<td>
<a href="http://www.guoxuedashi.com/download/" target="_blank">app软件下载专区</a><br>
<a href="http://www.guoxuedashi.com/download/gxds.php" target="_blank">《国学大师》下载</a><br>
<a href="http://www.guoxuedashi.com/download/kxzd.php" target="_blank">《汉字宝典》下载</a><br>
<a href="http://www.guoxuedashi.com/download/scqbd.php" target="_blank">《诗词曲宝典》下载</a><br>
<a href="http://www.guoxuedashi.com/SiKuQuanShu/skqs.php" target="_blank">《四库全书》下载</a><br>
</td>
</tr>
</table>

</div>
</div>


<div class="sidebar2">
<center>


</center>
</div>

<div class="sidebar"  style="margin-bottom:2px;">
<div class="sidebar_title">网站使用教程</div>
<div class="sidebar_info">
<a href="http://www.guoxuedashi.com/help/gjsearch.php" target="_blank">如何在国学大师网下载古籍?</a><br>
<a href="http://www.guoxuedashi.com/zidian/bujian/bjjc.php" target="_blank">如何使用部件查字法快速查字?</a><br>
<a href="http://www.guoxuedashi.com/search/sjc.php" target="_blank">如何在指定的书籍中全文检索?</a><br>
<a href="http://www.guoxuedashi.com/search/skjc.php" target="_blank">如何找到一句话在《四库全书》哪一页?</a><br>
</div>
</div>


<div class="sidebar">
<div class="sidebar_title">热门书籍</div>
<div class="sidebar_info">
<a href="/so.php?sokey=%E8%B5%84%E6%B2%BB%E9%80%9A%E9%89%B4&kt=1">资治通鉴</a> <a href="/24shi/"><strong>二十四史</strong></a>&nbsp; <a href="/a2694/">野史</a>&nbsp; <a href="/SiKuQuanShu/"><strong>四库全书</strong></a>&nbsp;<a href="http://www.guoxuedashi.com/SiKuQuanShu/fanti/">繁体</a>
<br><a href="/so.php?sokey=%E7%BA%A2%E6%A5%BC%E6%A2%A6&kt=1">红楼梦</a> <a href="/a/1858x/">三国演义</a> <a href="/a/1038k/">水浒传</a> <a href="/a/1046t/">西游记</a> <a href="/a/1914o/">封神演义</a>
<br>
<a href="http://www.guoxuedashi.com/so.php?sokeygx=%E4%B8%87%E6%9C%89%E6%96%87%E5%BA%93&submit=&kt=1">万有文库</a> <a href="/a/780t/">古文观止</a> <a href="/a/1024l/">文心雕龙</a> <a href="/a/1704n/">全唐诗</a> <a href="/a/1705h/">全宋词</a>
<br><a href="http://www.guoxuedashi.com/so.php?sokeygx=%E7%99%BE%E8%A1%B2%E6%9C%AC%E4%BA%8C%E5%8D%81%E5%9B%9B%E5%8F%B2&submit=&kt=1"><strong>百衲本二十四史</strong></a>  <a href="http://www.guoxuedashi.com/so.php?sokeygx=%E5%8F%A4%E4%BB%8A%E5%9B%BE%E4%B9%A6%E9%9B%86%E6%88%90&submit=&kt=1"><strong>古今图书集成</strong></a>
<br>

<a href="http://www.guoxuedashi.com/so.php?sokeygx=%E4%B8%9B%E4%B9%A6%E9%9B%86%E6%88%90&submit=&kt=1">丛书集成</a> 
<a href="http://www.guoxuedashi.com/so.php?sokeygx=%E5%9B%9B%E9%83%A8%E4%B8%9B%E5%88%8A&submit=&kt=1"><strong>四部丛刊</strong></a>  
<a href="http://www.guoxuedashi.com/so.php?sokeygx=%E8%AF%B4%E6%96%87%E8%A7%A3%E5%AD%97&submit=&kt=1">說文解字</a> <a href="http://www.guoxuedashi.com/so.php?sokeygx=%E5%85%A8%E4%B8%8A%E5%8F%A4&submit=&kt=1">三国六朝文</a>
<br><a href="http://www.guoxuedashi.com/so.php?sokeytm=%E6%97%A5%E6%9C%AC%E5%86%85%E9%98%81%E6%96%87%E5%BA%93&submit=&kt=1"><strong>日本内阁文库</strong></a> <a href="http://www.guoxuedashi.com/so.php?sokeytm=%E5%9B%BD%E5%9B%BE%E6%96%B9%E5%BF%97%E5%90%88%E9%9B%86&ka=100&submit=">国图方志合集</a> <a href="http://www.guoxuedashi.com/so.php?sokeytm=%E5%90%84%E5%9C%B0%E6%96%B9%E5%BF%97&submit=&kt=1"><strong>各地方志</strong></a>

</div>
</div>


<div class="sidebar2">
<center>

</center>
</div>
<div class="sidebar greenbar">
<div class="sidebar_title green">四库全书</div>
<div class="sidebar_info">

《四库全书》是中国古代最大的丛书,编撰于乾隆年间,由纪昀等360多位高官、学者编撰,3800多人抄写,费时十三年编成。丛书分经、史、子、集四部,故名四库。共有3500多种书,7.9万卷,3.6万册,约8亿字,基本上囊括了古代所有图书,故称“全书”。<a href="http://www.guoxuedashi.com/SiKuQuanShu/">详细>>
</a>

</div> 
</div>

</div>  <!--end r-->

</div>
<!-- 内容区END --> 

<!-- 页脚开始 -->
<div class="shh">

</div>

<div class="w1180" style="margin-top:8px;">
<center><script src="http://www.guoxuedashi.com/img/plus.php?id=3"></script></center>
</div>
<div class="w1180 foot">
<a href="/b/thanks.php">特别致谢</a> | <a href="javascript:window.external.AddFavorite(document.location.href,document.title);">收藏本站</a> | <a href="#">欢迎投稿</a> | <a href="http://www.guoxuedashi.com/forum/">意见建议</a> | <a href="http://www.guoxuemi.com/">国学迷</a> | <a href="http://www.shuowen.net/">说文网</a><script language="javascript" type="text/javascript" src="https://js.users.51.la/17753172.js"></script><br />
  Copyright &copy; 国学大师 古典图书集成 All Rights Reserved.<br>
  
  <span style="font-size:14px">免责声明:本站非营利性站点,以方便网友为主,仅供学习研究。<br>内容由热心网友提供和网上收集,不保留版权。若侵犯了您的权益,来信即刪。scp168@qq.com</span>
  <br />
ICP证:<a href="http://www.beian.miit.gov.cn/" target="_blank">鲁ICP备19060063号</a></div>
<!-- 页脚END --> 
<script src="http://www.guoxuedashi.com/img/plus.php?id=22"></script>
<script src="http://www.guoxuedashi.com/img/tongji.js"></script>

</body>
</html>
