










 


 
 


 

  
  
  
  
  





  
  
  
  
  
 
  

  

  
  
  



  

 
 

  
   




  

  
  


    資治通鑑卷一百五十九 宋 司馬光 撰

  胡三省 音註

  梁紀十五【起旃蒙赤奮若盡柔兆攝提格凡二年】

  高祖武皇帝十五

  大同十一年春正月丙申東魏遣兼散騎常侍李奨來聘【散悉亶翻騎奇寄翻】 東魏儀同爾朱文暢與丞相司馬任胄都督鄭仲禮等謀因正月望後觀打簇戲作亂【北史曰魏氏舊俗以正月十五夜為打簇戲能中者即時賞帛按魏書孝静天平四年春正月禁打簇相偷戲盖此禁尋弛也任音壬】殺丞相歡奉文暢為主事泄皆死文暢榮之子也其妹敬宗之后及仲禮姊大車皆為歡妾有寵故其兄弟皆不坐歡上書言并州軍器所聚動須女功請置宫以處配沒之口【處昌呂翻】又納吐谷渾之女以招懷之【吐谷渾國于西魏西南高歡越境納其女以招懷之盖欲借其力以侵擾西魏吐從暾入聲谷音浴】丁未置晉陽宫二月庚申東魏主納吐谷渾可汗從妹為容華【容華前漢内職舊號可從刋入聲汗音寒從才用翻】 魏丞相泰遣酒泉胡安諾槃陀始通使於突厥突厥本西方小國姓阿史那氏世居金山之陽為柔然鐵工【使疏吏翻下同李延夀曰突厥其先居西海之右獨為部落盖匈奴之别種也姓阿史那氏後為鄰國所破盡滅其俗有一兒年且十歲兵人見其小不忍殺之乃刖足斷臂棄澤中有牝狼以肉飼之及長與狼交合遂有孕焉彼王聞此兒尚在復遣殺之使者見在狼側并欲殺狼於時若有神物投狼于西海之東落高昌國西北山狼匿其中遂生十男男長外託妻孕其後各為一姓阿史那其一也最賢遂為君長故牙門建狼頭蠧示不忘本也或云突厥本平凉雜胡姓阿史那氏魏太武滅沮渠氏阿史那以五百家奔柔然世居金山之陽為柔然鐵工金山形似兜鍪借號兜鍪突厥突厥因以為號又曰突厥之先出於索國在匈奴之北其部落大人曰阿謗步兄弟七十人其一曰伊質泥師都狼所生也此說雖殊終狼種也程大昌曰金山形如兜鍪其俗謂兜鍪為突厥因以為號厥九勿翻】至其酋長土門始彊大【酋慈秋翻長知兩翻】頗侵魏西邊安諾槃陀至其國人皆喜曰【其國之下當更有國字屬下句】大國使者至吾國其將興矣 三月乙未東魏丞相歡入朝於鄴百姓迎於紫陌【朝直遥翻鄴都記紫陌在鄴城西北五里】歡握崔暹手而勞之曰往日朝廷豈無灋官莫肯舉劾【勞力到翻劾戶槩翻又戶得翻】中尉盡心徇國不避豪彊遂使遠邇肅清衝鋒陷陳大有其人【陳讀曰陣】當官正色今始見之【言聞之古人有當官正色者今始見崔暹也】富貴乃中尉自取高歡父子無以相報賜暹良馬暹拜馬驚走歡親擁之授以轡東魏主晏於華林園【鄴都倣京洛之制亦有華林園】使歡擇朝廷公直者勸之酒歡降階跪曰唯暹一人可勸并請以臣所射賜物千段賜之【時於華林園宴射賜歡物千段歡請回以賜暹】高澄退謂暹曰我尚畏羨何况餘人然暹中懷頗挟巧詐初魏高陽王斌有庶妹玉儀不為其家所齒為孫騰妓【斌音彬妓渠綺翻】騰又棄之高澄遇諸塗悦而納之遂有殊寵【白居易詩云天下無正色悦目即為姝誠有是事盖玉儀所乏者非色必妖媚善蠱惑故所如衆女謠諑而不見容】封琅琊公主澄謂崔季舒曰崔暹必造直諫【造如字作也】我亦有以待之及暹諮事澄不復假以顔色【復扶又翻】居三日暹懷刺墜之於前【續世說古者未有紙削竹木以書姓名謂之刺後以紙書謂之名紙唐李德裕貴盛人務加禮改具衘候起居之狀謂之門狀】澄問何用此為暹悚然曰未得通公主澄大悦把暹臂入見之季舒語人曰【語牛倨翻】崔暹常忿吾佞在大將軍前每言叔父可殺及其自作乃過於吾夏五月甲辰東魏大赦 魏王盟卒【九年魏以王盟為太傅】

  晉氏以來文章競為浮華魏丞相泰欲革其弊六月丁巳魏主饗太廟泰命大行臺度支尚書領著作蘇綽【度徒洛翻】作大誥宣示羣臣戒以政事仍命自今文章皆依此體【宇文泰令蘇綽倣尚書作大誥今其文尚在使當時文章皆依此體亦非所以崇雅黜浮也】 上遣交州刺史楊㬓討李賁【㬓匹妙翻】以陳霸先為司馬命定州刺史蕭勃會㬓於西江【五代志鬱林郡梁置定州】勃知軍士憚遠役因詭說留㬓【說式芮翻】㬓集諸將問計霸先曰交趾叛换罪由宗室【謂李賁之叛由武林侯諮也事見上卷七年】遂使溷亂數年逋誅累歲定州欲偷安目前不顧大計節下奉辭伐罪當死生以之豈可逗撓不進長寇沮衆也【撓奴教翻長知兩翻沮在呂翻】遂勒兵先發㬓以霸先為前鋒至交州 【考異曰典畧作十二月癸丑至交州姚思亷陳書帝紀在六月今從之】賁帥衆三萬拒之【帥讀曰率】敗於朱鳶【朱鳶縣自漢以來屬交趾郡五代志朱鳶縣舊置武平郡鳶音緣】又敗於蘇歷江口賁奔嘉寧城【沈約志吳孫皓建衡三年分交趾立新興郡并立嘉寧縣晉武帝太康三年更郡曰新昌五代志交趾郡嘉寧縣舊置興州新昌縣隋改曰峯州】諸軍圍之勃昺之子也【吴平侯昺帝從父弟也昺音丙】 魏與柔然頭兵可汗謀連兵伐東魏丞相歡患之遣行臺郎中杜弼使於柔然為世子澄求婚【使疏吏翻為于偽翻】頭兵曰高王自娶則可歡猶豫未决婁妃曰國家大計願勿疑也世子澄尉景亦勸之歡乃遣鎮南將軍慕容儼往聘之號曰蠕蠕公主【魏明元帝命柔然曰蠕蠕謂其蠕動無知識也阿那瓌曰蠕蠕王雖曰以為國號猶鄙賤之也至高歡納其女號曰蠕蠕公主則徑以為國號不復以為鄙賤矣蠕人兖翻】秋八月歡親迎於下舘【據北史彭城太妃傳下舘當在木井北宋白曰木井城今并州陽曲縣理又曰代州即古隂舘城有上舘下舘】公主至婁妃避正室以處之歡跪而拜謝【婁妃歡微時之妻正室也處昌呂翻】妃曰彼將覺之願絶勿顧【史言婁妃為國家計有趙姬使叔隗為内子而已下之之意】頭兵使其弟秃突佳來送女且報聘【或云作聘】仍戒曰待見外孫乃歸公主性嚴毅終身不肯華言歡嘗病不得往秃突佳怨恚【恚於避翻】歡輿疾就之 冬十月己未詔有罪者復聽入贖【天監三年除贖罪科見一百四十五卷復扶又翻】 東魏遣中書舍人尉瑾來聘 乙未東魏丞相歡請釋邙山俘囚桎梏配以民間寡婦【此邙山之捷所獲西魏之兵也捷事見上卷九年桎之日翻梏古沃翻】 十二月東魏以侯景為司徒中書令韓軌為司空戊子以孫騰録尚書事 魏築圓丘於城南【長安城南也】 散騎常侍賀琛啟陳四事其一以為今北邊稽服【稽音啟謂東魏通和也】正是生聚教訓之時【用伍子胥越十年生聚十年教訓之言】而天下戶口減落關外彌甚【謂淮汝潼泗新復州郡在邊關之外者】郡不堪州之控總縣不堪郡之裒削【裒薄侯翻】更相呼擾【更工衡翻】惟事徵斂【斂力贍翻】民不堪命各務流移此豈非牧守之過歟【守式又翻下同】東境戶口空虚【東境謂三吴之地】皆由使命繁數【使疏吏翻下同數所角翻】窮幽極遠無不皆至每有一使所屬搔擾駑困守宰則拱手聽其漁獵桀黠長吏又因之重為貪殘【黠下八翻長知兩翻重直用翻】縱有亷平郡猶掣肘如此雖年降復業之詔屢下蠲賦之恩而民不得反其居也其二以為今天下所以貪殘良由風俗侈靡使之然也今之燕喜【詩魯頌曰魯侯燕喜鄭氏箋云燕飲也】相競誇豪積果如丘陵列肴同綺繡露臺之產不周一燕之資【露臺之產謂百金也露臺事見十五卷漢文帝後七年】而賓主之間裁取滿腹未及下堂已同臭腐又畜妓之夫無有等秩【畜吁玉翻妓渠綺翻】為吏牧民者致貲巨億【巨億者億億也】罷歸之日不支數年率皆盡於燕飲之物歌謡之具所費事等丘山為歡止在俄頃乃更追恨向所取之少【少詩沼翻】如復傅翼增其搏噬【復扶又翻傅讀曰附言罷官家食之人復出為官猶不能奮飛之鳥復傅之羽翼也】一何悖哉【悖蒲内翻下同】其餘淫侈著之凡百【言時人凡百所為皆事淫侈也】習以成俗日見滋甚欲使人守亷白安可得邪誠宜嚴為禁制道以節儉【道讀曰導】糾奏浮華變其耳目夫不節之嗟亦民所自患止耻不能及羣故勉彊而為之【易曰不節若則嗟若無咎象曰不節之嗟又誰咎也琛引用之以發已意此論誠切中人情彊其兩翻】苟以純素為先足正彫流之弊矣其三以為陛下憂念四海不憚勤勞至於百司莫不奏事但斗筲之人既得伏奏帷扆【扆於豈翻】便欲詭競求進不論國之大體心存明恕惟務吹毛求疵擘肌分理【吹毛以求其疵瘢擘肌以分其肉理言其苛細】以深刻為能以繩逐為務【繩逐者繩糾其過失而斥逐之也】迹雖似於奉公事更成其威福犯罪者多巧避滋甚長弊增姦寔由於此【長知兩翻古寔實同】誠願責其公平之效黜其讒慝之心則下安上謐無徼倖之患矣【徼堅堯翻】其四以為今天下無事而猶日不暇給宜省事息費事省則民養費息則財聚應内省職掌各檢所部凡京師治署邸肆及國容戎備【治理事之所署舍止之所邸諸王列第及諸郡朝宿之區肆市列也國容禮樂車服旗章也戎備用兵之器備也】四方屯傳邸治【屯軍屯也傳驛傳也傳張戀翻】有所宜除除之有所宜減減之興造有非急者徵求有可緩者皆宜停省以息費休民故畜其財者所以大用之也養其民者所以大役之也若言小事不足害財則終年不息矣以小役不足妨民則終年不止矣【此亦確論也】如此則難可以語富彊而圖遠大矣啟奏上大怒召主書於前口授敇書以責琛【蕭子顯曰自齊建武以來詔命不關中書專出舍人省四省謂之四戶其下有主書令史舊用武官末改文吏人數無員莫非左右要密】大指以為朕有天下四十餘年公車讜言日關聽覽【讜多曩翻讜言善言也直言也】所陳之事與卿不異每苦【康董翻偬作孔翻困苦也不暇給也】更增惛惑卿不宜自同闒茸【闒吐盍翻茸而隴翻闒茸不肖也劣也】止取名字宣之行路言我能上事恨朝廷之不用何不分别顯言某刺史横暴【上時掌翻别彼列翻横戶孟翻】某太守貪殘【守式又翻】尚書蘭臺某人姦猾使者漁獵並何姓名【使疏吏翻】取與者誰明言其事得以誅責更擇材良又士民飲食過差若加嚴禁密房曲屋云何可知儻家家搜檢恐益增苛擾若指朝廷我無此事昔之牲牢久不宰殺【周禮王繕用六牲謂牛馬羊豕犬鷄也又曰王日一舉鼎十有二注曰殺牲盛饌曰舉鼎十有二牢鼎九陪鼎三帝事佛乃不宰殺】朝中會同菜蔬而已若復減此必有蟋蟀之譏【詩唐蟋蟀刺晉僖公儉不中禮朝直遥翻復扶又翻】若以為功德事者【帝以供佛供僧設無遮無碍會為功德事】皆是園中之物變一瓜為數十種【種章勇翻】治一菜為數十味【治直之翻下同】以變故多何損於事我自非公宴不食國家之食多歷年所乃至宫人亦不食國家之食【帝奄有東南凡其所食自其身以及六宫不由佛營不由神造又不由西天竺國來有不出於東南民力者乎惟不出於公賦遂以為不食國家之食誠如此則國家者果誰之國家邪】凡所營造不關材官及以國匠【此自文其營造塔寺之過耳材官將軍屬少府卿國匠者官給其俸廪以供國家之用者大匠卿掌土木之工】皆資雇借以成其事勇怯不同貪亷各用亦非朝廷為之傅翼【為于偽翻得讀曰附】卿以朝廷為悖乃自甘之當思致悖所以【悖蒲妹翻】卿云宜導之以節儉朕絶房室三十餘年至於居處不過一牀之地彫飾之物不入於宫受生不飲酒不好音聲所以朝中曲宴未嘗奏樂此羣賢之所見也朕三更出治事随事多少事少午前得竟【孔穎達曰雜比曰音單出曰聲竟畢其事也處昌呂翻好呼到翻更工衡翻朝直遥翻少詩沼翻】事多日昃方食日常一日若晝若夜昔要腹過於十圍【要讀曰腰】今之瘦削纔二尺餘舊帶猶存非為妄說為誰為之救物故也【為誰之為于偽翻下手為同】卿又曰百司莫不奏事詭競求進今不使外人呈事誰尸其任【尸主也】專委之人云何可得古人云專聽生姦獨任成亂【漢鄒陽之言】二世之委趙高元后之付王莽【趙高事見秦紀王莽事見漢紀】呼鹿為馬又可法歟卿云吹毛求疵復是何人擘肌分理復是何事【復扶又翻下當復復見敢復同】治署邸肆等何者宜除何者宜減何處興造非急何處徵求可緩各出其事具以奏聞富國彊兵之術息民省役之宜並宜具列若不具列則是欺罔朝廷倚聞重奏【倚側也側者傾待之義如側耳側身側席之類重直龍翻】當復省覽付之尚書班下海内【省悉景翻下遐嫁翻】庶惟新之美復見今日琛但謝過而已不敢復言上為人孝慈恭儉博學能文隂陽卜筮騎射聲律草隸圍碁無不精妙【騎奇寄翻】勤於政務冬月四更竟【夜分五更每更至五點而竟】即起視事執筆觸寒手為皴裂【皴七倫翻皮細起也】自天監中用釋氏法長齋斷魚肉【斷音短】日止一食惟菜羮糲飯而已【糲盧達翻糲者麤而不鑿也】或遇事繁日移中則口以過【日移中日過中也當作漱滌口也音先奏翻過謂度日也】身衣布衣木緜皁帳【木緜江南多有之以春三三月之晦下子種之既生須一月三蓐其四旁失時不薅則為草所荒穢輒萎死入夏漸茂至秋生黄花結實及熟時其皮四裂其中綻出如緜土人以鐵鋌碾去其核取如綿者以竹為小弓長尺四五寸許牽弦以彈綿令其勻細卷為小筩就車紡之自然抽緒如繅絲狀不勞紐緝織以為布自閩廣來者尤為麗密方勺曰閩廣多種木緜樹高七八尺葉如柞結實如大菱而色青秋深即開露白綿茸然土人摘取去殻以鐵杖捍盡黑子徐以小弓彈令紛起然後紡繢為布名曰吉具今所貨木綿特其細緊者耳當以花多為勝横數之得百二十花此最上品海南蠻人織為巾上出細字雜花卉尤工巧即古所謂白疊巾身衣於既翻】一冠三載一衾二年【載子亥翻亦年也】後宫貴妃以下衣不曳地性不飲酒非宗廟祭祀大饗宴及諸法事未嘗作樂【法事謂奉佛為梵唄】雖居暗室恒理衣冠小坐盛暑未嘗褰袒【小坐宫中便坐也恒戶登翻坐徂卧翻】對内䜿小臣如遇大賓然優假士人太過牧守多侵漁百姓使者干擾郡縣又好親任小人【守式又翻使疏吏翻好呼到翻】頗傷苛察多造塔廟公私費損江南久安風俗奢靡故琛啟及之上惡其觸實【惡烏路翻】故怒臣光曰梁高祖之不終也宜哉夫人君聽納之失在於叢脞【孔安國曰叢脞細碎無大畧馬融曰叢總也脞小也陸德明曰脞倉果翻徐音鎻】人臣獻替之病在於煩碎是以明主守要道以御萬機之本忠臣陳大體以格君心之非故身不勞而收功遠言至約而為益大也觀夫賀琛之諫未至於切直而高祖已赫然震怒護其所短矜其所長詰貪暴之主名【詰去吉翻】問勞費之條目困以難對之狀責以必窮之辭自以蔬食之儉為盛德日昃之勤為至治【昃徂力翻治直吏翻】君道已備無復可加【復扶又翻】羣臣箴規舉不足聽如此則自餘切直之言過於琛者誰敢進哉由是姦佞居前而不見【謂朱异周石珍輩也】大謀顛錯而不知【謂納侯景復與東魏和也】名辱身危覆邦絶祀為千古所閔笑豈不哀哉

  上敦尚文雅疎簡刑法自公卿大臣咸不以鞫獄為意姦吏招權弄法貨賂成市枉濫者多大率二歲刑已上歲至五千人徒居作者具五任【任謂其人巧力所任也五任謂任攻木者則役之攻木任攻金者則役之攻金任攻皮者則役之攻皮任設色者則役之設色任搏埴者則役之搏埴任音壬】其無任者著升械【魏武帝定甲子科犯釱左右趾者易以升械是時乏鐵故易以木焉著陟畧翻】若疾病權解之是後囚徒或有優劇【言囚徒有力足以行賂者則守吏詭言疾病權解其械而得優寛其無力以賂吏者則雖實罹疾病亦不得解械更增苦劇也】時王侯子弟多驕淫不法上年老厭於萬幾【幾居希翻】又專精佛戒每斷重罪則終日不懌【梁武帝斷重罪則終日不懌此好生惡殺之意也夷考帝之終身自襄陽舉兵以至下建康猶曰事關家國伐罪救民洛口之敗死者凡幾何人浮山之役死者凡幾何人寒山之敗死者又幾何人其間争城以戰殺人盈城争地以戰殺人盈野南北之人交相為死者不可以數計也至于侯景之亂東極吳會西抵江郢死于兵死于飢者自典午南渡之後未始見也驅無辜之人而就死地不惟儒道之所不許乃佛教之罪人而斷一重罪乃終日不懌吾誰欺欺天乎斷丁亂翻】或謀反逆事覺亦泣而宥之【如臨賀王正德父子是也】由是王侯益横【横戶孟翻】或白晝殺人於都街或暮夜公行剽刼【剽匹妙翻】有罪亡命者匿於王家有司不敢搜捕上深知其弊溺於慈愛不能禁也 魏東陽王榮為瓜州刺史【五代志敦煌郡舊置瓜州】與其壻鄧彦偕行榮卒瓜州首望表榮子康為刺史【各州之大姓是為望族首望者又一州望族之首】彦殺康而奪其位魏不能討因以彦為刺史屢徵不至又南通吐谷渾【吐谷渾立國在敦煌之南隔大河吐從暾入聲谷音浴】丞相泰以道遠難於動衆欲以計取之以給事黄門侍郎申徽為河西大使密令圖彦徽以五十騎行既至止於賓館彦見徽單使【兵從不多故曰單使文選李陵荅蘇武書所謂單車之使者也使疏吏翻騎奇寄翻】不以為疑徽遣人微勸彦歸朝【朝直遥翻】彦不從徽又使贊成其留計【贊其留敦煌之計】彦信之遂來至舘徽先與州主簿敦煌令狐整等密謀【令狐整瓜州之望也姓譜令狐本自畢萬之後晉大夫令狐文子即魏顆也敦徒門翻令音零】執彦於坐【坐徂卧翻】責而縛之因宣詔慰諭吏民且云大軍續至城中無敢動者【鄧彦久在瓜州豈無黨與威之以大軍繼至故懼而不敢動】遂送彦於長安泰以徽為都官尚書

  中大同元年【是年夏四月方改元為中大同】春正月癸丑楊㬓等克嘉寧城【㬓匹妙翻 考異曰典畧作乙未今從梁帝紀】李賁奔新昌獠中諸軍頓於江口【江口即蘇歷江入海之口獠魯晧翻】 二月魏以義州刺史史寧【先是東西魏争義州史寧先入城據之西魏因以為刺史】為凉州刺史前刺史宇文仲和據州不受代瓜州民張保殺刺史成慶以應之晉昌民呂興殺太守郭肆以郡應保【劉昫唐志瓜州晉昌縣漢敦煌郡之寘安縣舊置晉昌郡及寘安縣因改晉陽為永興隋改為瓜州改寘安為常樂武德七年復為晉昌唐又有常樂縣則漢之廣至縣地也又按五代志瓜州常樂縣後魏置常樂郡後周併凉興廣至寘安閠泉合為凉興縣隋廢郡改縣為常樂參而考之則晉昌郡當置于隋常樂縣界】丞相泰遣太子太保獨孤信開府儀同三司怡峯與史寧討之 三月乙巳大赦庚戍上幸同泰寺遂停寺省【同泰寺有便省】講三慧經 【考異曰】

  【典畧云癸卯詔以今月八日於同泰寺設無遮大會捨朕身及以宫人并所王境土供養三寶四月丙戊公卿以錢億萬奉贖按韓愈佛骨表云三度捨身為寺家奴若并此則四矣今從梁書】夏四月丙戌解講大赦改元是夜同泰寺浮圖災上曰此魔也宜廣為法事羣臣皆稱善乃下詔曰道高魔盛行善鄣生【魔鬼魔鄣鄣礙魔眉波翻行下孟翻】當窮兹土木倍增往日遂起十二層浮圖將成值侯景亂而止 魏史寧暁諭凉州吏民率皆歸附獨宇文仲和據城不下五月獨孤信使諸將夜攻其東北自帥壮士襲其西南遲明克之【將即亮翻帥讀曰率下同遲直二翻】遂擒仲和初張保欲殺州主簿令狐整以其人望恐失衆心雖外相敬内甚忌之整陽為親附因使人說保曰今東軍漸逼凉州【東軍謂獨孤信之軍東自長安來說式芮翻】彼勢孤危恐不能敵宜急分精鋭以救之然成敗在於將領【將即亮翻下同】令狐延保兼資文武【令狐整字延保】使將兵以往蔑不濟矣保從之整行及玉門【玉門縣漢晉屬酒泉郡師古曰闞駰云漢罷玉門關屯徙其人於此五代志瓜州玉門縣後魏置會稽郡又有玉門郡】召豪傑述保罪狀馳還襲之先克晉昌斬呂興進擊瓜州州人素信服整皆棄保來降保奔吐谷渾【降戶江翻吐從暾入聲谷音浴】衆議推整為刺史整曰吾屬以張保逆亂恐闔州之人俱陷不義故相與討誅之今復見推是效尤也【左傳曰尤而效之罪又甚焉復扶又翻】乃推魏所遣使波斯者張道義行州事【使疏吏翻】具以狀聞丞相泰以申徽為瓜州刺史召整為夀昌太守【五代志西城郡石泉縣舊曰永樂置晉昌郡西魏改為夀昌郡又改永樂為石泉守式又翻】封襄武男整帥宗族鄉里三千餘人入朝從泰征討累遷驃騎大將軍開府儀同三司加侍中【令狐整以忠順貴顯於魏史終言之朝直遥翻驃匹妙翻騎奇寄翻】 六月庚子東魏以司徒侯景為河南大將軍大行臺 秋七月壬寅東魏遣散騎常侍元廓來聘【散悉亶翻】 甲子詔犯罪非大逆父母祖父母不坐 先是江東唯建康及三吴荆郢江湘梁益用錢【先悉薦翻】其餘州郡雜以穀帛交廣專以金銀為貨上自鑄五銖及女錢二品並行【杜佑曰梁武帝鑄錢肉好周郭文曰五銖重二銖三絫二黍其百文則重一斤二兩又别鑄除其肉郭謂之公式女錢徑一寸文曰五銖重如新鑄五銖二錢並行及其末也又有兩杜錢】禁諸古錢普通中更鑄鐵錢由是民私鑄者多物價騰踊交易者至以車載錢不復計數【更工衡翻復扶又翻】又自破嶺以東八十為百名曰東錢【破嶺在今鎮江府丹陽縣秦始皇所鑿即破岡也】江郢以上七十為百名曰西錢建康以九十為百名曰長錢丙寅詔曰朝四暮三衆狙皆喜名實未虧而喜怒為用【莊子曰狙公賦芋曰朝三而暮四衆狙皆怒曰朝四而暮三衆狙皆喜狙千余翻】頃聞外間多用九陌錢陌減則物貴陌足則物賤非物有貴賤乃心有顛倒至於遠方日更滋甚徒亂王制無益民財自今可通用足陌錢令書行役百日為期若猶有犯男子謫運女子質作並同三年【謫運者以謫發之轉運質作質其身使居作皆沒之三年此古所謂三歲刑也】詔下而人不從錢陌益少【少詩沼翻】至於季年遂以三十五為百云 上年高諸子心不相下【下遐嫁翻】邵陵王綸為丹陽尹湘東王繹在江州武陵王紀在益州皆權侔人主太子綱惡之常選精兵以衛東宫【為帝諸子皆不終張本惡烏路翻】八月以綸為南徐州刺史 東魏丞相歡如鄴【自晉陽朝于鄴而書如鄴言其威權陵上若列國然】高澄遷洛陽石經五十二碑於鄴【石經見五十七卷漢靈帝熹平四年】魏徙并州刺史王思政為荆州刺史使之舉諸將可

  代鎮玉壁者【西魏置并州刺史僑治玉壁將即亮翻】思政舉晉州刺史韋孝寛【晉州屬東魏韋孝寛遥領刺史耳】丞相泰從之東魏丞相歡悉舉山東之衆將伐魏癸巳自鄴會兵於晉陽九月至玉壁圍之以挑西師【挑徒了翻】西師不出 李賁復帥衆二萬自獠中出屯典澈湖【復扶又翻湖亦當在新昌郡界 考異曰典畧云渡武平江據新安村今從陳帝紀】大造船艦充塞湖中【艦戶黯翻塞悉則翻】衆軍憚之頓湖口不敢進陳霸先謂諸將曰我師已老【楊㬓等自去年夏五月出師至是幾一年半故自謂師老】將士疲勞且孤軍無援入人心腹若一戰不捷豈望生全今藉其屢奔人情未固夷獠烏合易為摧殄【獠魯皓翻下同易弋豉翻】正當共出百死决力取之無故停留時事去矣諸將皆默然莫應【諸將心不欲戰故默而莫敢應】是夜江水暴起七丈注湖中霸先勒所部兵乘流先進衆軍鼓譟俱前賁衆大潰竄入屈獠洞中 冬十月乙亥以前東揚州刺史岳陽王詧為雍州刺史【雍于用翻】上捨詧兄弟而立太子綱【事見一百五十五卷中大通三年】内嘗愧之寵亞諸子【言詧被寵亞於諸子帝固知詧之才器足以自立矣】以會稽人物殷阜【會工外翻】故用詧兄弟迭為東揚州以慰其心詧兄弟亦内懷不平詧以上衰老朝多秕政【朝直遥翻秕卑履翻不成粟也書曰若粟之有秕後漢書安帝贊曰秕我王度注曰秕諭穢也】遂蓄聚貨財折節下士【折而設翻下遐嫁翻】招募勇敢左右至數千人以襄陽形勝之地梁業所基【謂帝自襄陽起兵以得天下】遇亂可以圖大功乃克已為政撫循士民數施恩惠延納規諫所部稱治【為詧據襄陽張本數所角翻治直吏翻】 東魏丞相歡攻玉壁晝夜不息魏韋孝寛随機拒之城中無水汲於汾歡使移汾一夕而畢【於汾水上流决而移之不使近城】歡於城南起土山欲乘之以入城上先有二樓【先悉薦翻】孝寛縛木接之令常高於土山以禦之歡使告之曰雖爾縛樓至天我當穿地取爾乃鑿地為十道又用術士李業興孤虚法【漢書藝文志有風后孤虚二十卷史記日者傳曰日辰不全故有孤虚注云甲乙謂之日子丑謂之辰六甲孤虚法甲子旬中無戌亥戌亥即為孤辰巳即為虚甲戍旬中無申酉申酉為孤寅卯為虛甲申旬中無午未午未為孤子丑為虚甲午旬中無辰巳辰巳為孤戍亥為虛甲辰旬中無寅卯寅卯為孤申酉為虛甲寅旬中無子丑子丑為孤午未為虚賢曰對孤為虚玄女謂黄帝曰戰陳之法避孤擊虚】聚攻其北北天險也【天險自然之險也天設地造不假人力者也易曰天險不可升也】孝寛掘長塹邀其地道選戰士屯塹上每穿至塹戰士輒禽殺之又於塹外積柴貯火【塹七艶翻貯丁呂翻】敵有在地道内者塞柴投火以皮排吹之【塞悉則翻排讀與鞲同音步拜翻韋囊也所以吹火】一鼓皆焦爛【鼓排吹之火氣入地道故敵人在其中者皆焦爛】敵以攻車撞城【撞直江翻】車之所及莫不摧毁無能禦者孝寛縫布為幔【幔莫半翻】随其所向張之布既懸空車不能壞【壞音怪】敵又縛松麻於竿【松薪麻骨之燥者燒之易然故敵用之】灌油加火以燒布并欲焚樓孝寛作長鉤利其刃【此所謂鉤刀也杜佑曰鉤竿如槍兩旁有曲刃可以鉤物】火竿將至以鉤遥割之松麻俱落敵又於城四面穿地為二十道其中施梁柱縱火燒之柱折城崩【高歡嘗用此術攻鄴以擒劉誕故復用之於玉壁折而設翻】孝寛於崩處竪木柵以扞之【竪而主翻】敵不得入城外盡攻擊之術而城中守禦有餘孝寛又奪據其土山歡無如之何乃使倉曹參軍祖珽說之曰【珽他鼎翻說式芮翻】君獨守孤城而西方無救恐終不能全何不降也【降戶江翻下同】孝寛報曰我城池嚴固兵食有餘攻者自勞守者常逸豈有旬朔之間已須救援【浹日為旬改月為朔】適憂爾衆有不返之危孝寛關西男子必不為降將軍也【自謂男子言决不怯懦如婦人】珽復謂城中人曰韋城主受彼榮禄或復可爾【可爾猶言可如此也復扶又翻】自外軍民何事相随入湯火中乃射募格於城中【募格者立賞格以募人射而亦翻下同】云能斬城主降者拜太尉封開國郡公賞帛萬匹孝寛手題書背返射城外云能斬高歡者準此珽瑩之子也【祖瑩見一百五十卷普通六年】東魏苦攻凡五十日士卒戰及病死者共七萬人 【考異曰北史韋孝寛傳云苦戰六旬傷及病死者什四五今從北齊書】共為一冢歡智力皆困因而發疾有星墜歡營中士卒驚懼十一月庚子解圍去先是歡别使侯景將兵趣齊子嶺【河内郡王屋縣舊名長平有齊子嶺有軹關杜佑曰按齊子嶺在今王屋縣東二十里周齊分界處先悉薦翻將即亮翻趣七喻翻】魏建州刺史楊鎮車廂恐其寇邵郡【先是取建州已而退還邵郡西魏因授以建州刺史車廂當在随唐之絳州垣縣界宋白曰絳州絳縣本理車廂城随移縣理於城北十里與標同】帥騎禦之【帥讀曰率騎奇寄翻】景聞至斫木斷路六十餘里【斷音短】猶驚而不安遂還河陽【楊常才耳侯景何至懼之如此史之所言容有過其實者】庚戌歡使段韶從太原公洋鎮鄴辛亥徵世子澄會晉陽魏以韋孝寛為驃騎大將軍開府儀同三司進爵建忠公【賞守玉壁之功也建忠公建忠郡公五代志京兆郡三原縣後周置建忠郡】時人以王思政為知人十一月己卯歡以無功表解都督中外諸軍東魏主許之歡之自玉壁歸也軍中訛言韋孝寛以定功弩射殺丞相【射而亦翻】魏人聞之因下令曰勁弩一發凶身自隕歡聞之勉坐見諸貴使斛律金作敇勒歌【斛律金敕勒部人也故使作敕勒歌洪邁曰斛律金唱敕律歌本鮮卑語按古樂府有其辭云敇勒川隂山下天似穹廬籠罩四野天蒼蒼野茫茫風吹草低見牛羊余謂此後人妄為之耳敇勒與鮮卑殊種斛律金出於敇勒故使之作敇勒歌若高歡則習鮮卑之俗者也】歡自和之哀感流涕【和胡卧翻史言高歡將死故當樂而哀不能自揜】 魏大行臺度支尚書司農卿蘇綽性忠儉常以喪亂未平為己任【度徒洛翻喪息浪翻】紀綱庶政丞相泰推心任之人莫能間【間古莧翻】或出遊常預署空紙以授綽有須處分【處昌呂翻分扶問翻】随事施行及還啟知而已綽常謂為國之道當愛人如慈父訓人如嚴師每與公卿論議自晝達夜事無巨細若指諸掌積勞成疾而卒【卒子恤翻】泰深痛惜之謂公卿曰蘇尚書平生廉讓吾欲全其素志恐悠悠之徒有所未達如厚加贈諡又乖宿昔相知之心何為而可尚書令史麻瑶【尚書令史自東漢有之唐六典曰魏晉以來令史之任用人常輕齊梁後魏北齊雖預品秩抑又微矣】越次進曰儉約所以彰其美也泰從之歸葬武功【蘇綽武功人歸葬鄉里】載以布車一乘【乘繩證翻】泰與羣公步送出同州郭外【五代志馮翊郡後魏置華州西魏改曰同州孫愐曰馮翊有九龍泉泉有九源同為一流因以名州】泰於車後酹酒【酹盧對翻餟祭以酒沃地也】言曰尚書平生為事妻子兄弟所不知者吾皆知之唯爾知吾心吾知爾志方與共定天下遽捨吾去奈何因舉聲慟哭不覺巵落於手 東魏司徒河南大將軍大行臺侯景右足偏短弓馬非其長而多謀筭諸將高敖曹彭樂等皆勇冠一時【冠古玩翻】景常輕之曰此屬皆如豕突勢何所至【言其勇而無謀也】景嘗言於丞相歡願得兵三萬横行天下要須濟江縛取蕭衍老公以為太平寺主【史言侯景夙有取江南之志太平寺盖在鄴】歡使將兵十萬專制河南杖任若已之半體【杖直兩翻憑也】景素輕高澄嘗謂司馬子如曰高王在吾不敢有異王沒吾不能與鮮卑小兒共事子如掩其口及歡疾篤澄詐為歡書以召景先是景與歡約曰今握兵在遠人易為詐【先悉薦翻易弋豉翻】所賜書皆請加微點歡從之景得書無點辭不至又聞歡疾篤用其行臺郎潁川王偉計遂擁兵自固歡謂澄曰我雖病汝面更有餘憂何也【言澄當以得盡總内外大權為喜不應更有餘憂】澄未及對歡曰豈非憂侯景叛耶對曰然歡曰景專制河南十四年矣【東魏天平元年歡使景取荆州後遂委以河南至是十三年歡此語當在來春垂没之時】常有飛揚跋扈之志顧我能畜養【畜許竹翻】非汝所能駕御也今四方未定勿遽發哀庫狄干鮮卑老公斛律金敇勒老公並性遒直終不負汝【遒慈秋翻健也固也】可朱渾道元劉豐生遠來投我必無異心【可朱渾道元奔東魏見一百五十七卷大同元年劉豐生奔東魏見二年】潘相樂本作道人心和厚汝兄弟當得其力韓軌少戅宜寛借之【少詩沼翻戅陟降翻】彭樂心腹難得宜防護之【終以邙山事衘之為後彭樂被誅張本】堪敵侯景者惟有慕容紹宗我故不貴之留以遺汝【使澄厚以官爵結紹宗之心遺於季翻下患遺同】又曰段孝先忠亮仁厚【段韶字孝先】智勇兼備親戚之中惟有此子軍旅大事宜共籌之又曰邙山之戰吾不用陳元康之言【事見上卷大同九年】留患遺汝死不瞑目【瞑目定翻】相樂廣寧人也

  資治通鑑卷一百五十九


    


 


 



 

 
  







 


  
  
 
 
 


  

 















	
	









































 
  



















 





 












  
  
  

 





