<!DOCTYPE html PUBLIC "-//W3C//DTD XHTML 1.0 Transitional//EN" "http://www.w3.org/TR/xhtml1/DTD/xhtml1-transitional.dtd">
<html xmlns="http://www.w3.org/1999/xhtml">
<head>
<meta http-equiv="Content-Type" content="text/html; charset=utf-8" />
<meta http-equiv="X-UA-Compatible" content="IE=Edge,chrome=1">
<title>資治通鑒_217-資治通鑑卷二百十六_217-資治通鑑卷二百十六</title>
<meta name="Keywords" content="資治通鑒_217-資治通鑑卷二百十六_217-資治通鑑卷二百十六">
<meta name="Description" content="資治通鑒_217-資治通鑑卷二百十六_217-資治通鑑卷二百十六">
<meta http-equiv="Cache-Control" content="no-transform" />
<meta http-equiv="Cache-Control" content="no-siteapp" />
<link href="/img/style.css" rel="stylesheet" type="text/css" />
<script src="/img/m.js?2020"></script> 
</head>
<body>
 <div class="ClassNavi">
<a  href="/24shi/">二十四史</a> | <a href="/SiKuQuanShu/">四库全书</a> | <a href="http://www.guoxuedashi.com/gjtsjc/"><font  color="#FF0000">古今图书集成</font></a> | <a href="/renwu/">历史人物</a> | <a href="/ShuoWenJieZi/"><font  color="#FF0000">说文解字</a></font> | <a href="/chengyu/">成语词典</a> | <a  target="_blank"  href="http://www.guoxuedashi.com/jgwhj/"><font  color="#FF0000">甲骨文合集</font></a> | <a href="/yzjwjc/"><font  color="#FF0000">殷周金文集成</font></a> | <a href="/xiangxingzi/"><font color="#0000FF">象形字典</font></a> | <a href="/13jing/"><font  color="#FF0000">十三经索引</font></a> | <a href="/zixing/"><font  color="#FF0000">字体转换器</font></a> | <a href="/zidian/xz/"><font color="#0000FF">篆书识别</font></a> | <a href="/jinfanyi/">近义反义词</a> | <a href="/duilian/">对联大全</a> | <a href="/jiapu/"><font  color="#0000FF">家谱族谱查询</font></a> | <a href="http://www.guoxuemi.com/hafo/" target="_blank" ><font color="#FF0000">哈佛古籍</font></a> 
</div>

 <!-- 头部导航开始 -->
<div class="w1180 head clearfix">
  <div class="head_logo l"><a title="国学大师官网" href="http://www.guoxuedashi.com" target="_blank"></a></div>
  <div class="head_sr l">
  <div id="head1">
  
  <a href="http://www.guoxuedashi.com/zidian/bujian/" target="_blank" ><img src="http://www.guoxuedashi.com/img/top1.gif" width="88" height="60" border="0" title="部件查字,支持20万汉字"></a>


<a href="http://www.guoxuedashi.com/help/yingpan.php" target="_blank"><img src="http://www.guoxuedashi.com/img/top230.gif" width="600" height="62" border="0" ></a>


  </div>
  <div id="head3"><a href="javascript:" onClick="javascript:window.external.AddFavorite(window.location.href,document.title);">添加收藏</a>
  <br><a href="/help/setie.php">搜索引擎</a>
  <br><a href="/help/zanzhu.php">赞助本站</a></div>
  <div id="head2">
 <a href="http://www.guoxuemi.com/" target="_blank"><img src="http://www.guoxuedashi.com/img/guoxuemi.gif" width="95" height="62" border="0" style="margin-left:2px;" title="国学迷"></a>
  

  </div>
</div>
  <div class="clear"></div>
  <div class="head_nav">
  <p><a href="/">首页</a> | <a href="/ShuKu/">国学书库</a> | <a href="/guji/">影印古籍</a> | <a href="/shici/">诗词宝典</a> | <a   href="/SiKuQuanShu/gxjx.php">精选</a> <b>|</b> <a href="/zidian/">汉语字典</a> | <a href="/hydcd/">汉语词典</a> | <a href="http://www.guoxuedashi.com/zidian/bujian/"><font  color="#CC0066">部件查字</font></a> | <a href="http://www.sfds.cn/"><font  color="#CC0066">书法大师</font></a> | <a href="/jgwhj/">甲骨文</a> <b>|</b> <a href="/b/4/"><font  color="#CC0066">解密</font></a> | <a href="/renwu/">历史人物</a> | <a href="/diangu/">历史典故</a> | <a href="/xingshi/">姓氏</a> | <a href="/minzu/">民族</a> <b>|</b> <a href="/mz/"><font  color="#CC0066">世界名著</font></a> | <a href="/download/">软件下载</a>
</p>
<p><a href="/b/"><font  color="#CC0066">历史</font></a> | <a href="http://skqs.guoxuedashi.com/" target="_blank">四库全书</a> |  <a href="http://www.guoxuedashi.com/search/" target="_blank"><font  color="#CC0066">全文检索</font></a> | <a href="http://www.guoxuedashi.com/shumu/">古籍书目</a> | <a   href="/24shi/">正史</a> <b>|</b> <a href="/chengyu/">成语词典</a> | <a href="/kangxi/" title="康熙字典">康熙字典</a> | <a href="/ShuoWenJieZi/">说文解字</a> | <a href="/zixing/yanbian/">字形演变</a> | <a href="/yzjwjc/">金 文</a> <b>|</b>  <a href="/shijian/nian-hao/">年号</a> | <a href="/diming/">历史地名</a> | <a href="/shijian/">历史事件</a> | <a href="/guanzhi/">官职</a> | <a href="/lishi/">知识</a> <b>|</b> <a href="/zhongyi/">中医中药</a> | <a href="http://www.guoxuedashi.com/forum/">留言反馈</a>
</p>
  </div>
</div>
<!-- 头部导航END --> 
<!-- 内容区开始 --> 
<div class="w1180 clearfix">
  <div class="info l">
   
<div class="clearfix" style="background:#f5faff;">
<script src='http://www.guoxuedashi.com/img/headersou.js'></script>

</div>
  <div class="info_tree"><a href="http://www.guoxuedashi.com">首页</a> > <a href="/SiKuQuanShu/fanti/">四库全书</a>
 > <h1>资治通鉴</h1> <!--         下载:【右键另存为】即可 --></div>
  <div class="info_content zj clearfix">
  
<div class="info_txt clearfix" id="show">
<center style="font-size:24px;">217-資治通鑑卷二百十六</center>
    資治通鑑卷二百十六  宋 司馬光 撰<br />
<br />
  胡三省 音註<br />
<br />
  唐紀三十二【起彊圉大淵獻十二月盡昭陽大荒落凡六年有奇始丁亥十二月終癸巳凡六年零一月】<br />
<br />
  玄宗至道大聖大明孝皇帝下之上<br />
<br />
  天寶六載【載子亥翻】十二月己巳上以仙芝為安西四鎭節度使徵霛詧入朝【使疏吏翻朝直遥翻】霛詧大愳仙芝見霛詧趨走如故霛詧益愳副都護京兆程千里押牙畢思琛及行官王滔等【押牙者盡管節度使牙内之事行官主將命往來京師及鄰道及廵内郡縣琛丑林翻】皆平日搆仙芝於霛詧者也仙芝面責千里思琛曰公面如男子心如婦人何也又捽滔等欲笞之【捽才沒翻笞丑之翻】既而皆釋之謂曰吾素所恨於汝者欲不言恐汝懷憂今既言之則無事矣軍中乃安初仙芝為都知兵馬使猗氏人封常清少孤貧細瘦顙目【少詩照翻顙盧對翻】一足偏短求為仙芝傔不納常清日候仙芝出入不離其門凡數十日【傔苦念翻離力智翻下離席同】仙芝不得已留之會達奚部叛夫蒙靈詧使仙芝追之斬獲略盡常清私作捷書以示仙芝皆仙芝心所欲言者由是一府奇之仙芝為節度使即署常清判官仙芝出征常為留後【唐諸使之屬判官位次副使盡總府事又節度使或出征或入朝或死而未有代皆有知留後事其後遂以節度留後為稱至我朝遂以留後為承宣使資序未應建節者為之】仙芝乳母子鄭德詮為郎將【將即亮翻】仙芝遇之如兄弟使典家事威行軍中常清嘗出德詮自後走馬突之而過常清至使院【使院留後治事之所節度使便坐治事亦或就使院使疏吏翻】使召德詮每過一門輒闔之既至常清離席謂曰常清本出寒微郎將所知今日中丞命為留後【離力智翻中丞謂高仙芝唐邉鎮諸帥或帶御史中丞大夫時隨其所帶官稱之】郎將何得於衆中相陵突因叱之曰郎將須蹔死以肅軍政【蹔與暫同】遂杖之六十面仆地曳出仙芝妻及乳母於門外號哭救之不及因以狀白仙芝仙芝覽之驚曰已死邪及見常清遂不復言【號戶高翻復扶又翻】常清亦不之謝軍中畏之惕息【史言封常清能治軍政亦緣高仙芝不以私親橈法惕他歷翻】自唐興以來邊帥皆用忠厚名臣不久任不遥領不兼統功名著者往往入為宰相【如李靖李勣劉仁軌婁師德之類是也開元以來薛訥郭元振張嘉貞王晙張說杜濯蕭嵩李適之等亦皆自邊帥入相帥所類翻】其四夷之將雖才略如阿史那社爾契苾何力猶不專大將之任皆以大臣為使以制之【社爾討高昌侯君集為元帥何力討高麗李勣為元帥將即亮翻契欺紇翻苾毗必翻使疏吏翻】及開元中天子有吞四夷之志為邊將者十餘年不易始久任矣【王晙郭知運張守珪之類是也】皇子則慶忠諸王宰相則蕭嵩牛仙客始遥領矣【請王事見二百十三卷開元十五年蕭嵩事見十七年牛仙客事見二百十四卷二十四年】蓋嘉運王忠嗣專制數道始兼統矣【蓋嘉運事見二百十四卷開元二十八年王忠嗣事見上卷五載蓋古合翻】李林甫欲杜邊帥入相之路以胡人不知書乃奏言文臣為將怯當矢石不若用寒畯胡人【寒謂卑賤畯嘗有事農耕者也畯音俊】胡人則勇决習戰寒族則孤立無黨陛下誠以恩洽其心彼必能為朝廷盡死【為于偽翻】上悦其言始用安祿山至是諸道節度盡用胡人【安祿山安思順哥舒翰高仙芝皆胡人也】精兵咸戍北邊天下之埶偏重卒使禄山傾覆天下皆出於林甫專寵固位之謀也【卒子恤翻】<br />
<br />
  七載夏四月辛丑左監門大將軍知内侍省事高力士加驃騎大將軍【知内侍省事自此始唐制勲階二十九驃騎大將軍為之首從一品監古衘翻驃匹妙翻騎奇寄翻】力士承恩歲久中外畏之太子亦呼之為兄諸王公呼之為翁駙馬輩直謂之爺【爺以遮翻俗呼父為爺】自李林甫安祿山輩皆因之以取將相其家富厚不貲【將即亮翻相悉亮翻貲即移翻不貲言不可筭計也】於西京作寶夀寺寺鐘成力士作齋以慶之舉朝畢集【宋璟若在必不預斯集矣朝直遥翻】擊鐘一杵施錢百緡【施式䜴翻】有求媚者至二十杵少者不減十杵然性和謹少過善觀時俯仰不敢驕横故天子終親任之士大夫亦不疾惡也【少詩照翻横戶孟翻惡烏路翻】 五月壬午羣臣上尊號曰開元天寶聖文神武應道皇帝【上時掌翻】赦天下免百姓來載租庸擇後魏子孫一人為三恪【三恪二王後注已見前杜佑曰周得天下封夏殷二王後又封舜後謂之恪恪敬也義取王之所敬故曰三恪天寶十三載公卿議曰三恪二王之議有三說焉一曰二王之前更立三代之後為三恪此據樂記武王克商未及下車封黄帝堯舜之後下車封夏殷之後而言一曰二王之前但在一代通二王為三恪此據左傳但封胡公以備三恪明王者所敬先王有二更封一代以備三恪三云二王之後為一恪妻之父母為二恪夷狄之君為三恪此據王有不臣者三而言之梁崔靈恩云三說以初為長按二王三恪經無正文靈恩據禮記遂以為通存五代竊恐未安記云尊賢不過二代第三代者雖遠難師法豈得不錄其後故亦存之示敬其道而已因謂之三恪故左傳云封胡公以備三恪足知其無五代也况歷代至今皆以三代為三恪焉以此攷之蓋以後魏子孫與周隋子孫為三恪也明年尋罷魏後注又見後】 六月庚子賜安祿山鐵劵 度支郎中兼侍御史楊釗善窺上意所愛惡而迎之以聚歛驟遷歲中領十五餘使【釗音昭歛力贍翻惡烏路翻使疏吏翻洪邁隨筆曰楊國忠為度支郎領十五餘使至宰相凡領四十餘使新舊唐史皆不詳載其職案其拜相制前衘云御史大夫判度支權知太府卿事兼蜀郡長使劍南節度支度營田等副大使本道兼山南西道采訪處置使兩京太府出納監倉祠祭木炭宫市長春九成宫等使關内道及京畿采訪處置使拜右相兼吏部尚書集賢殿崇玄館學士修國史太清紫微宫使自餘所領又有管當租庸鑄錢等使以是觀之概可見矣】甲辰遷給事中兼御史中丞專判度支事【度徒洛翻】恩幸日隆<br />
<br />
  蘇冕論曰設官分職各有司存政有恒而易守【恒戶登翻易以䜴翻】事歸本而難失經遠之理捨此奚據洎姧臣廣言利以邀恩多立使以示寵【洎其冀翻使疏吏翻】刻下民以厚歛張虚數以獻狀上心蕩而益奢人望怨而成禍使天子有司守其位而無其事受厚祿而虚其用宇文融首唱其端楊慎矜王鉷繼遵其軌【鉷戶公翻】楊國忠終成其亂仲尼云寜有盜臣而無聚歛之臣【記大學百乘之家不畜聚歛之臣與其有聚歛之臣寜有盜臣】誠哉是言前車既覆後轍未改求達化本不亦難乎<br />
<br />
  冬十月庚戌上幸華清宫 十一月癸未以貴妃姊適崔氏者為韓國夫人適裴氏者為虢國夫人適柳氏者為秦國夫人三人皆有才色上呼之為姨出入宫掖並承恩澤埶傾天下每命婦入見【姊蒋兕翻掖音亦命婦外命婦也見賢遍翻】玉真公主等皆讓不敢就位【玉眞公主睿宗之女】三姊與銛錡五家凡有請託府縣承迎峻於制敕四方賂遺【銛丑亷翻錡魚倚翻遺于季翻下獻遺同】輻湊其門惟恐居後朝夕如市十宅諸王及百孫院昏嫁【十王宅百孫院見二百十三卷開元十五年】皆以錢千緡賂韓虢使請無不如志上所賜與及四方獻遺五家如一競開第舍極其壯麗一堂之費動踰千萬既成見它人有勝己者輒毁而改為虢國尤為豪蕩一旦帥工徒突入韋嗣立宅即撤去舊屋【緡彌賓翻帥讀曰率去羌呂翻】自為新第但授韋氏以隙地十畝而已中堂既成召工圬墁【圬音烏墁謨官翻】約錢二百萬復求賞技【復扶又翻技巨綺翻】虢國以絳羅五百段賞之嗤而不顧曰請取螻蟻蜥蜴【嗤丑之翻蜥先撃翻蜴羊益翻師古曰爾雅云蠑螈蜥蜴蝘蜒守宫是則一類耳揚雄方言云在澤中者謂之蜥蜴】記其數置堂中苟失一物不敢受直 十二月戊戌或言玄元皇帝降于朝元閣【上於華清宫中起老君殿殿之北為朝元閣以或言老君降于此改曰降聖閣】制改會昌縣曰昭應廢新豐入昭應辛酉上還宫【自温泉宫還從宣翻又音如字】哥舒翰築神威軍於青海上吐蕃至翰擊破之又築城於青海中龍駒島【吐從暾入聲青海周八九百里中有山須氷合遊牝馬其上明年生駒號龍種故謂之龍駒島】謂之應龍城吐蕃屏迹不敢近青海【屏必郢翻近其靳翻】 是歲雲南王歸義卒子閤羅鳳嗣以其子鳳迦異為揚瓜州刺史【卒子恤翻嗣祥吏翻南詔王父子相繼其子必以父號下一字冠於己所號之上歸義本號皮邏閤帝賜名歸義其子號閤羅鳳是以閤字冠其號之上也閤羅鳳之子號鳳迦異是以鳳字冠其號之上也其後至豐祐乃革其舊開元二十六年考異不取此說然二百四十三卷穆宗之長慶四年則又書豐祐不與父連名事】<br />
<br />
  八載春二月戊申引百官觀左藏【唐六典曰周禮有外府中土主泉藏之在外者掌邦布之入出以供百物而侍邦用者也又有職幣上士中士主貨幣之入者也並今左藏之職至秦漢則分在司農少府後漢少府屬官有中藏府令丞掌中藏幣帛金銀貨物魏氏因之晉少府屬官有左右藏令東晉御史九人各掌一曹有庫曹御史後復分庫曹置外左庫内左庫宋文帝省外左庫而内左庫直曰左庫齊梁陳有右藏庫而無左藏北齊太府寺統左右藏令丞後周有外府上士中士隋有左右藏署令丞唐左藏有東庫西庫朝堂庫又有東都庫余按雍錄太極宫中東左藏庫西左藏庫東庫在恭禮門之東西庫在安仁門之西大明宫中有左藏庫在麟德殿之左又有右藏署令掌邦國寶貨雜物而天下賦調之正數錢物則皆歸左藏也藏徂浪翻下帑藏同】賜帛有差是時州縣殷富倉庫積粟帛動以萬計楊釗奏請所在糶變為輕貨及徵丁租地税皆變布帛輸京師屢奏帑藏充牣古今罕儔故上帥羣臣觀之【釗音昭帥讀曰率】賜釗紫衣金魚以賞之上以國用豐衍故視金帛如糞壤賞賜貴寵之家無有限極 三月朔方節度等使張齊丘於中受降城西北五百餘里木刺山築横塞軍以振遠軍使鄭人郭子儀為横塞軍使【横塞軍本名可敦城按宋白續通典横塞軍初置在飛狐後移蔚州開元六年張嘉貞移於古代郡大安城南以為九姓之援天寶十二年改為天德軍參考諸書横塞軍即横野軍天寶元年書河東節度統横野軍開元六年所移者也此築横塞軍在可敦城者也振遠軍在單于府界鄭縣漢屬京兆後魏置東雍州并華山縣西魏改華州隋開皇初廢郡大業初廢州復置鄭縣屬京兆唐屬華州使疏吏翻降戶江翻刺盧逹翻】 夏四月咸寜太守趙奉璋【咸寜郡本丹州丹陽郡元年更郡名守式又翻】告李林甫罪二十餘條狀未達林甫知之諷御史逮捕以為妖言杖殺之【妖於喬翻】 先是折衝府皆有木契銅魚朝廷徵發下敕書契魚【唐制銅魚符所以起軍旅王畿之内左三右一王畿之外左五右一左者在外行用之日從第一為首後事須用依次發之周而復始木契之制若太子監國則王畿之内左右各三王畿之外左右各五庶官鎮守則左右各十唐六典符寶郎掌符節曰木契者所以重鎮守慎出納車駕巡幸皇太子監國有兵馬受處分者為木契并行軍所及領軍五百人馬五百匹以上征討皆給木契三公以下兩京留守及諸州有兵馬受處分亦給木契曰銅魚符者所以起軍旅易守長兩京留守若諸州諸軍折衝府諸處捉兵鎮守之所及宫摠監皆給魚符程大昌演繁露曰唐世左魚之外又有敕牒將之故兼名魚書先悉薦翻下遐嫁翻】都督郡府參驗皆合然後遣之自募置彍騎【彍騎見二百一十二卷開元十三年彍音郭又音霍】府兵日益墮壞【墮讀曰隳】死及逃亡者有司不復點補【復扶又翻】其六馱馬牛器械糗糧耗散略盡【馱徒何翻糗去九翻】府兵入宿衛者謂之侍官言其為天子侍衛也其後本衛多以假人役使如奴隸長安人羞之至以相詬病【記儒行曰今衆人以儒相詬病注以儒相靳故相戲詬病猶恥辱也杜預云戲而相愧為靳詬音遘又呼候翻】其戍邊者又多為邊將苦使利其死而沒其財【將即亮翻】由是應為府兵者皆逃匿至是無兵可交五月癸酉李林甫奏停折衝府上下魚書是後府兵徒有官吏而已【唐府兵之制十人為火火有長火備六馱馬凡火具烏布幕鐵馬盂布糟鍤钁鑿碓筐斧鉗鋸皆一甲狀二鎌二五十人為隊隊具火鑽一胸馬繩一首羈足絆皆三人具弓一矢三十胡祿横刀礪石大觹氈帽氈裝行縢皆一麥飯九斗米二升皆自備其介胄戎具藏于庫有所征行則視其入而出給之其番上宿衛者惟給弓矢横刀而已今皆耗廢非其舊矣】其折衝果毅又歷年不遷士大夫亦恥為之其彍騎之法天寶已後稍亦變廢應募者皆市井負販無賴子弟【孔穎達曰白虎通云因井為市故曰市井應劭通俗文云市恃也養瞻老小恃以不匱也俗說市井謂至市者於井上洗濯其物香潔及自嚴飾乃到市也謹按古者二十畝為一井因為市交易故稠市井然則本由井田之中交易為市故國都之市亦因曰市井案禮制九夫為市應劭二十畝為井者劭依漢書食貨志一井八家家有私田百畝公田十畝餘二十畝以為井竈廬舍故言二十畝耳因井為市或如劭言】未嘗習兵時承平日久議者多謂中國兵可銷于是民間挾兵器者有禁子弟為武官父兄擯不齒猛將精兵皆聚於西北中國無武備矣 太白山人李渾等上言【水經注曰武功縣有太一山古文以為終南亦曰中南亦曰太白山在武功縣南去長安二百里不知其高幾何俗云武功太白去天三百杜彦遠曰太白山南連武功山於諸山最為秀傑冬夏積雪望之皓然隋志曰太一山在盩厔縣界新志曰太白山在郿縣】見神人言金星洞有玉板石記聖主福夀之符命御史中丞王鉷入仙遊谷求而獲之上以符瑞相繼皆祖宗休烈六月戊申上聖祖號曰大道玄元皇帝上高祖諡曰神堯大聖皇帝太宗諡曰文武大聖皇帝高宗諡曰天皇大聖皇帝中宗諡曰孝和大聖皇帝睿宗諡曰玄真大聖皇帝竇太后以下皆加諡曰順聖皇后 辛亥刑部尚書京兆尹蕭炅坐贓左遷汝陰太守【汝陰郡穎州】 上命隴右節度使哥舒翰帥隴右河西及突厥阿布思兵益以朔方河東兵凡六萬三千攻吐蕃石堡城【厥九勿翻吐從暾入聲】其城三面險絶惟一徑可上【上時掌翻】吐蕃但以數百人守之多貯糧食積擂木及石【貯丁呂翻擂盧對翻】唐兵前後屢攻之不能克翰進攻數日不拔召裨將高秀巖張守瑜欲斬之【高秀巖後為安祿山守大同蓋二人朔方河東將也】二人請三日期可克如期拔之獲吐蕃鐵刃悉諾羅等四百人唐士卒死者數萬果如王忠嗣之言【王忠嗣言見上卷六載】頃之翰又遣兵於赤嶺西開屯田以謫卒二千戍龍駒島冬冰合吐蕃大集戍者盡没【深入虜境聲援不接兵法曰遺人獲也】 閏月乙丑以石堡城為神武軍又于劒南西山索磨川置保寜都護府【置保寜都護府以領牂柯吐蕃】 丙寅上謁太清宫【天寶元年正月得靈符起玄元皇帝廟於西京大寜坊東京置於東宫積善坊臨淄舊邸天下諸郡各置玄元像於開元觀至二年三月十二日改在京玄元宫為太清宫東京為太極宫天下諸郡為紫極宫】丁卯群臣上尊號曰開元天地大寶聖文神武應道皇帝赦天下禘祫自今於太清宫聖祖前設位序正秋七月冊突騎施移撥為十姓河汗【騎奇寄翻可從刋入聲汗音寒】八月乙亥護密王羅眞入朝【朝直遥翻】請留宿衛許之<br />
<br />
  拜左武衛將軍冬十月乙丑上幸華清宫 十一月乙未吐火羅葉護失里怛伽羅遣使表稱朅師王親附吐蕃【怛當割翻伽求加翻使疏吏翻朅丘竭翻又去謁翻】困苦小勃律鎭軍阻其糧道臣思破凶徒望發安西兵以來歲正月至小勃律六月至大勃律【朅師亦曰羯師胡種也與吐火羅鄰接大勃律或曰布露直吐蕃西其北即小勃律也】上許之<br />
<br />
  九載春正月己亥上還宫 群臣屢表請封西嶽許之二月楊貴妃復忤旨送歸私第【楊妃初忤旨見上卷五載復扶又翻忤五】<br />
<br />
  【故翻】戶部郎中吉温因宦官言於上曰婦人識慮不遠違忤聖心陛下何愛宫中一席之地不使之就死豈忍辱之於外舍邪【吉温此言欲以自結于楊妃耳邪音耶】上亦悔之遣中使賜以御膳妃對使者涕泣曰妾罪當死陛下幸不殺而歸之今當永離掖庭【使疏吏翻下同離力智翻掖音亦】金玉珍玩皆陛下所賜不足為獻惟髪者父母所與敢以薦誠乃翦髪一繚而獻之【繚力照翻】上遽使高力士召還寵待益深【還從宣翻又如字婦人女子最為難養以忤旨而出之若棄咳唾可也既出而復召還則彼之怙寵悍悖將無所不至明皇其可再乎古之所謂英主如漢武之于鈎弋庶乎】時諸貴戚競以進食相尚上命宦官姚思藝為檢校進食使水陸珍羞數千盤一盤費中人十家之產中書舍人竇華嘗退朝【使疏吏翻朝直遥翻】值公主進食列于中衢傳呼按轡出其間宫苑小兒數百奮梃於前【宫苑小兒宫苑使領之梃待鼎翻】華僅以身免 安西節度使高仙芝破朅師虜其王勃特没【朅丘竭翻 考異曰實錄去載十一月吐火羅葉護請使安西兵討朅師上許之不見出師今載三月庚子冊朅師國王勃特没兄素迦為王冊曰頃勃特没於卿不孝于國不忠不言朅師為誰所破按十載正月高仙芝擒朅師王來獻然則朅師為仙芝所破也】三月庚子立勃特没之兄素迦為朅師王【迦音加】 上命御史大夫王鉷鑿華山路設壇場于其上【鉷戶公翻華戶化翻】是春關中旱辛亥嶽祠災制罷封西嶽 夏四月己巳御史大夫宋渾坐贓巨萬流潮陽【潮陽郡本潮州義安郡元年更郡名】初吉温因李林甫得進【天寶四載吉温鞫兵部之獄自是得進】及兵部侍郎兼御史中丞楊釗恩遇浸深温遂去林甫而附之為釗畫代林甫執政之策【釗音昭為于偽翻】蕭炅及渾皆林甫所厚也【炅火迥翻】求得其罪使釗奏而逐之以翦其心腹林甫不能救也 五月乙卯賜安祿山爵東平郡王唐將帥封王自此始【將即亮翻帥所類翻】 秋七月乙亥置廣文館於國子監以教諸生習進士者【時廣文館置博士二員助教一員】 八月丁巳以安祿山兼河北道采訪處置使 朔方節度使張齊丘給糧失宜軍士怒敺其判官【敺烏口翻】兵馬使郭子儀以身捍齊丘乃得免【世皆知郭子儀得衆然後能捍免張齊丘而不知當此之時唐之軍政果安在也】癸亥齊丘左遷濟陰太守【濟陰郡曹州濟子禮翻】以河西節度使安思順權知朔方節度事 辛卯處士崔昌上言【處昌呂翻上時掌翻】國家宜承周漢以土代火周隋皆閏位不當以其子孫為二王後事下公卿集議【下遐嫁翻】集賢殿學士衛包上言集議之夜四星聚於尾天意昭然上乃命求殷周漢後為三恪廢韓介公【韓元魏後介後周後隋後戶圭翻孔穎達曰古春秋左氏說周家封夏殷二王之後以為上公封黃帝堯舜之後謂之三恪鄭氏云所存二王之後者命使郊天以天子之禮祭其始祖受命之王自行其正朔服色恪者敬也敬其先聖而封其後與諸侯無殊異不得比夏殷之後】以昌為左贊善大夫包為虞部員外郎 冬十月庚申上幸華清宫 太白山人王玄翼上言見玄元皇帝言寶仙洞有妙寶眞符 【考異曰舊志王鉷奏玄翼見玄元于寶仙洞中遣鉷與張均王倕王濟王翼王嶽靈于洞中得玉石函上清護國經寶劵紀籙等獻之今從實錄云】命刑部尚書張均等往求得之時上尊道教慕長生故所在爭言符瑞羣臣表賀無虚月李林甫等皆請舍宅為觀【觀古玩翻】以祝聖夀上悦 安祿山屢誘奚契丹為設會飲以莨菪酒【本草曰莨菪子生海邊川谷今處處有之苖莖高二三尺許葉與地黄紅藍等而二指濶四月開花紫色苗夾莖有白毛五月結實有殻作罌子狀如小石榴房中子至細青白如米粒毒甚煮一二日而芽方生以釀酒其毒尤甚為于偽翻下先為同飲于鴆翻莨音浪菪音蕩】醉而阬之動數千人函其酋長之首以獻前後數四至是請入朝上命有司先為起第於昭應【時王公皆私置第于昭應獨祿山以承恩命有司起第酋慈由翻長知兩翻朝直遥翻】祿山至戲水【戲許宜翻】楊釗兄弟姊妹皆往迎之冠蓋蔽野上自幸望春宫以待之辛未祿山獻奚俘八千人上命考課之日書上上考前此聽禄山於上谷鑄錢五壚祿山乃獻錢様千緡【緡彌頻翻】 楊釗張易之之甥也【釗音昭 考異曰鄭審天寶故事云楊國忠本張易之之子天授中張易之恩幸莫比每歸私第詔令居樓上仍去其梯母恐張氏絶嗣乃密令女奴蠙珠上樓遂有娠而生國忠其說曖昧無稽今不取】奏乞昭雪易之兄弟【易之兄弟誅見二百七卷中宗神龍元年】庚辰制引易之兄弟迎中宗於房陵之功【迎中宗非特二張倡其議也事見二百六卷武后聖歷元年】復其官爵 【考異曰唐歷在七月二十五日今從實錄】仍賜一子官釗以圖䜟有金刀請更名【䜟楚譛翻更工衡翻】上賜名國忠 十二月乙亥上還宫 關西遊奕使王難得擊吐蕃克五橋拔樹敦城以難得為白水軍使【使疏吏翻吐從暾入聲樹敦城以古犬戎王樹惇名城隋在吐谷渾界唐在吐蕃界】 安西四鎮節度使高仙芝偽與石國約和引兵襲之虜其王及部衆以歸悉殺其老弱仙芝性貪掠得瑟瑟十餘斛【使疏吏翻張揖廣雅曰瑟瑟碧珠也】黄金五六槖駞其餘口馬雜貨稱是皆入其家【為石國王子弟誘諸國以覆仙芝之師張本稱尺證翻】楊國忠德鮮于仲通【鮮于仲通資給國忠又推轂之事見上卷四載】薦為劍南節度使仲通性急失蠻夷心【補典翻】故事南詔常與妻子俱謁都督過雲南雲南太守張䖍陀皆私之【過古禾翻雲南郡姚州守式又翻】又多所徵求南詔王閤羅鳳不應䖍陀遣人詈辱之仍密奏其罪閤羅鳳忿怨是歲發兵反攻陷雲南殺䖍陀取夷州三十二【夷州在西南夷附化羈縻之州也】<br />
<br />
  十載春正月壬辰上朝獻太清宫癸巳朝享太廟【載祖亥翻朝直遥翻】甲子合祭天地於南郊赦天下免天下今載地稅【丄書壬辰癸巳下書丁酉則甲子當作甲午】丁酉命李林甫遥領朔方節度使以戶部侍郎李暐知留後事 庚子楊氏五宅夜遊【楊銛錡及韓虢秦三夫人為五宅】與廣平公主從者爭西市門【廣平新書作廣寜公主上女也從才用翻】楊氏奴揮鞭及公主衣公主墜馬駙馬程昌裔下扶之亦被數鞭公主泣訴於上上為之杖殺楊氏奴【被皮義翻為于偽翻下司為使為遽為同】明日免昌裔官不聽朝謁上命有司為安祿山治第於親仁坊【治直之翻】敕令但窮壯麗不限財力既成具幄帟器皿充牣其中有帖白檀床二皆長丈濶六尺【本草圖經曰檀香木如檀生南海種有黄白紫之異帟音亦長直亮翻】銀平脱屏風帳方丈六尺於厨廐之物皆飾以金銀金飯甖二【甖於耕翻缶也】銀淘盆二【淘盆所以淅米】皆受五斗織銀絲筐及笟篱各一【筐去王翻所以瀝米笟側絞翻篱音離笟籬所以轑釜取食物】它物稱是【稱尺證翻】雖禁中服御之物殆不及也上每令中使為祿山護役【護監護也使疏吏翻下同】築第及造儲偫賜物常戒之曰胡眼大勿令笑我【偫直里翻大唐佐翻】祿山入新第置酒乞降墨敕請宰相至第是日上欲於樓下擊毬遽為罷戲命宰相赴之日遣諸楊與之選勝遊宴侑以棃園教坊樂【棃園皇帝棃園弟子也教坊内教坊也】上每食一物稍美或後苑校獵獲鮮禽輒遣中使走馬賜之絡繹於路甲辰祿山生日上及貴妃賜衣服寶器酒饌甚厚【饌雛戀翻又雛皖翻】後三日召祿山入禁中貴妃以錦繡為大襁褓裹禄山使宫人以綵輿舁之【襁居兩翻舁羊茹翻】上聞後宫歡笑問其故左右以貴妃三日洗祿兒對上自往觀之喜賜貴妃洗兒金銀錢復厚賜祿山盡歡而罷【復扶又翻】自是祿山出入宫掖不禁或與貴妃對食或通宵不出頗有醜聲聞於外上亦不疑也【聞音問觀明皇所以待祿山者昏庸之主所不為殆天奪之魄也 考異曰祿山事迹正月二十日祿山生日玄宗及太眞賜祿山器皿衣服件目甚多後三日召祿山入内貴妃以錦繡繃縳祿山令内人以綵輿舁之宫中歡呼動地玄宗使人問之報云貴妃與祿兒作三日洗兒玄宗就觀之大悦因賜貴妃洗兒金銀錢物極歡而罷自是宫中皆呼祿山為祿兒不禁其出入温畬天寶亂離西幸記祿山謟約楊妃誓為子母自虢國已下次及諸玉皆戲祿兒與之促膝娛宴上時聞後宫三千合處喧笑密偵則祿山果在其内貴戚猱雜未之前聞凡曰釵鬟皆㗖厚利或通宵禁掖暱狎嬪嬙和士開之出入臥内方此為疎薊城侯之獲厠刑餘又奚足尚王仁裕天寶遺事云祿山常與妃子同食無所不至帝恐外人以酒毒之遂賜金牌子繫于臂上每有王公召宴欲沃以巨即祿山以金牌示之云凖敕戒酒今略取之】 安西節度使高仙芝入朝獻所擒突騎施可汗吐蕃酋長石國王朅師王加仙芝開府儀同三司尋以仙芝為河西節度使代安思順思順諷羣胡割耳剺面請留已制復留思順於河西【剺力之翻復扶又翻】安祿山求兼河東節度二月丙辰以河東節度使韓<br />
<br />
  休珉為左羽林將軍以祿山代之戶部郎中吉温見祿山有寵又附之約為兄弟說祿山曰李右丞相雖以時事親三兄【天寶元年改侍中為左相中書令為右相李林甫時為右相中書令之職也丞字衍安祿山第三說式芮翻】不必肯以兄為相温雖蒙驅使終不得超擢兄若薦温於上温即奏兄堪大任共排林甫出之為相必矣祿山悦其言數稱温才於上上亦忘曩日之言【數所角翻忘巫放翻言見上卷四載】會祿山領河東因奏温為節度副使知留後以大理司直張通儒為留後判官【大理司直從六品上通儒帶司直而為河東留後判官是後節鎭有帶六曹尚書有帶三省長官有帶三公三師其屬亦率帶六品以下朝職謂之帶職黄琮曰外官帶職有憲銜冇檢校憲銜自監察御史至御史大夫檢校自國子祭酒至三公唐及五代之制也】河東事悉以委之是時楊國忠為御史中丞方承恩用事祿山登降殿階國忠常扶掖之【掖音亦】祿山與王鉷俱為大夫鉷權任亞於李林甫祿山見林甫禮貌頗倨林甫陽以它事召王大夫鉷至趨拜甚謹祿山不覺自失容貌益恭林甫與祿山語每揣知其情先言之【揣初委翻】祿山驚服祿山於公卿皆慢侮之獨憚林甫每見雖盛冬常汗沾衣林甫乃引與坐於中書廳【廳他經翻中庭日聽事言受事察訟于是也漢晉皆作聽事六朝以來乃始加广而徑曰廳】撫以温言自解披袍以覆之祿山忻荷言無不盡【覆敷又翻荷下可翻】謂林甫為十郎【林甫第十】既歸范陽劉駱谷每自長安來必問十郎何言得美言則喜或但云語安大夫【語牛倨翻】須好檢校輒反手據牀曰噫嘻我死矣禄山既兼領三鎭賞刑已出日益驕恣自以曩時不拜太子【事見上卷六載】見上春秋高頗内懼又見武備墮㢮【墮讀曰隳】有輕中國之心孔目官嚴莊【孔目官衙前吏職也唐世始有此名言凡使司之事一孔一目皆須經由其手也】掌書記高尚【掌書記位判官下古記室參軍之任】因為之解圖讖勸之作亂【為于偽翻讖楚譛翻】祿山養同羅奚契丹降者八千餘人謂之曳落河【契欺訖翻又音喫降戶江翻考異曰祿山事迹云養為己子按養子必無八千之數今不取】曳落河者胡言壯士也及家僮百餘人皆驍勇善戰一可當百又畜戰馬數萬匹【驍堅堯翻畜許六翻】多聚兵仗分遣商胡詣諸道販鬻歲輸珍貨數百萬【輸舂遇翻】私作緋紫袍魚袋以百萬計以高尚嚴莊張通儒及將軍孫孝哲為心腹史思明安守忠李歸仁蔡希德牛廷玠向潤容【向式亮翻姓也】李庭望崔乾祐尹子奇何千年武令珣能元皓【能奴代翻何氏姓苑云能姓出自長廣】田承嗣田乾真阿史那承慶為爪牙尚雍奴人【雍奴天寶元年更名武清屬范陽郡此因舊縣名書之】本名不危頗有辭學薄遊河朔貧困不得志常歎曰高不危當舉大事而死豈能齧艸根求活邪祿山引置幕府出入卧内尚典牋奏莊治簿書【治直之翻】通儒萬歲之子【張萬歲唐初掌廐牧通儒必非其子或者其孫也否則别又有一張萬歲】孝哲契丹也承嗣世為盧龍小校祿山以為前鋒兵馬使嘗大雪祿山按行諸營【校戶教翻使疏吏翻行下孟翻】至承嗣營寂若無人入閲士卒無一人不在者祿山以是重之 夏四月壬午劍南節度使鮮于仲通討南詔蠻大敗於瀘南【瀘水之南也武后垂拱元年置長城縣屬姚州天寶初更名瀘南縣 考異曰楊國忠傳南蠻質子閤羅鳳亡歸不獲帝怒欲討之國忠薦閬州人鮮于仲通為益州長史令帥精兵八萬討南蠻按南詔傳七年蒙歸義死詔閤羅鳳襲雲南王不云嘗為質子亡歸也九年姚州自以張䖍陀侵之故反時鮮于仲通已為益州長史國忠傳與南詔傳相違新舊書皆如此恐誤】時仲通將兵八萬分二道出戎雟州至曲州靖州【分為二道一道出戎州一道出嶲州也自戎州開邊縣西行七十里至曲州自嶲州西南行八百餘里渡瀘水曲州本隋之恭州古朱提之地武德八年更名曲州靖州隋屬恊州古夜郎地武德初分協州置靖州嶲音髓】南詔王閤羅鳳謝罪請還所俘掠城雲南而去【去年南詔攻䧟雲南城必有夷毁處故請城之以謝罪】且曰今吐蕃大兵壓境若不許我我將歸命吐蕃雲南非唐有也仲通不許囚其使進軍至西洱河與閤羅鳳戰軍大敗【使疏吏翻洱乃吏翻】士卒死者六萬人仲通僅以身免楊國忠掩其敗狀仍叙其戰功 【考異曰唐歷云令仲通白衣領節度事舊傳無之按既掩敗叙功豈得復白衣領職】閤羅鳳歛戰尸築為京觀【觀古玩翻】遂北臣於吐蕃蠻語謂弟為鍾吐蕃命閤羅鳳為贊普鍾號曰東帝給以金印閤羅鳳刻碑於國門言己不得已而叛唐且曰我世世事唐受其封爵後世容復歸唐當指碑以示唐使者知吾之叛非本心也【其後德宗之世異牟尋卒復歸唐復扶又翻】制大募兩京及河南北兵以擊南詔人聞雲南多瘴癘未戰士卒死者什八九莫肯應募楊國忠遣御史分道捕人連枷送詣軍所舊制百姓有勲者免征役時調兵既多【調徒弔翻】國忠奏先取高勲於是行者愁怨父母妻子送之所在哭聲振野 高仙芝之虜石國王也石國王子逃詣諸胡具告仙芝欺誘貪暴之狀【誘音酉】諸胡皆怒濳引大食欲共攻四鎭仙芝聞之將蕃漢三萬衆撃大食【將即亮翻 考異曰馬宇段秀實别傳云蕃漢六萬衆今從唐歷】深入七百餘里至恒羅斯城【或作怛羅斯城】與大食遇相持五日葛羅祿部衆叛【葛羅禄即葛邏禄】與大食夾攻唐軍仙芝大敗士卒死亡略盡所餘纔數千人右威衛將軍李嗣業勸仙芝宵遁道路阻隘拔汗那部衆在前人畜塞路【拔汗那時從仙芝擊大食塞悉則翻】嗣業前驅奮大梃擊之人馬俱斃仙芝乃得過將士相失别將汧陽段秀實【汧陽郡本隴州隴東郡元年改郡名有汧陽縣蓋元魏所置梃待鼎翻將即亮翻下同汧口堅翻】聞嗣業之聲詬曰【詬苦候翻】避敵先奔無勇也全已棄衆不仁也幸而得達獨無愧乎嗣業執其手謝之留拒追兵收散卒得俱免還至安西言於仙芝以秀實兼都知兵馬使為己判官 八月丙辰武庫火燒兵器三十七萬 【考異曰唐歷云四十七年事今從實錄】 安祿山將三道兵六萬【幽州平盧河東三道】以付契丹以奚騎二千為鄉導【騎奇寄翻下同郷讀曰嚮】過平盧千餘里至土護眞水遇雨【自雄武軍東北渡灤河有古盧龍鎭有斗陘嶺自古盧龍北經九荆嶺受米城張洪隘度石嶺至奚王帳六百里又東北傍吐護真河五百里至奚契丹牙帳又出檀州燕樂縣東北百八十五里至長城口又北八百里有吐護真河奚王牙帳也】祿山引兵晝夜兼行三百餘里至契丹牙帳契丹大駭時久雨弓弩筋膠皆㢮大將何思德言於祿山曰吾兵雖多遠來疲弊實不可用不如按甲息兵以臨之不過三日虜必降【將即亮翻下同降戶降翻】祿山怒欲斬之思德請前驅效死思德貌類祿山虜爭擊殺之以為已得祿山勇氣增倍奚復叛與契丹合夾擊唐兵殺傷殆盡射祿山中鞍折冠簪失屨獨與麾下二十騎走會夜追騎解得入師州【貞觀三年以室韋部落置師州治營州之廢陽師鎮復扶又翻射而亦翻中竹仲翻折而設翻】歸罪於左賢王哥解【哥解蓋自突厥來降者解戶買翻】河東兵馬使魚承仙而斬之平盧兵馬使史思明懼逃入山谷近二旬【近其靳翻】收散卒得七百人平盧守將史定方將精兵二千救祿山契丹引去祿山乃得免至平盧麾下皆亡不知所出史思明出見祿山祿山喜起執其手曰吾得汝復何憂【復扶又翻】思明退謂人曰曏使早出已與哥解并斬矣【史言史思明之智數過于安祿山】契丹圍師州祿山使思明擊却之 冬十月壬子上幸華清宫 楊國忠使鮮于仲通表請已遥領劒南十一月丙午以國忠領劒南節度使<br />
<br />
  十一載春正月丁亥上還宫 二月庚午命有司出粟帛及庫錢數十萬緡於兩市易惡錢 【考異曰舊紀唐㦄皆作癸酉今從實録】先是江淮多惡錢貴戚大商往往以良錢一易惡錢五載入長安市井不勝其弊【先悉薦翻勝音升】故李林甫奏請禁之官為易取期一月不輸官者罪之于是商賈囂然不以為便【賈音古囂許驕翻又五刀翻】衆共遮楊國忠馬自言國忠為之言於上乃更命非鈆錫所鑄及穿穴者皆聽用之如故【為于偽翻更工衡翻】 三月安祿山發蕃漢步騎二十萬擊契丹欲以雪去秋之恥初突厥阿布思來降【事見上卷元年降戶江翻】上厚禮之賜姓名李獻忠累遷朔方節度副使賜爵奉信王獻忠有才略不為安祿山下祿山恨之至是奏請獻忠帥同羅數萬騎與俱擊契丹【帥讀曰率下同】獻忠恐為祿山所害白留後張暐請奏留不行暐不許【安祿山領河東而張暐為留後暐知附祿山而已豈肯從阿布思請哉】獻忠乃率所部大掠倉庫叛歸漠北祿山遂頓兵不進 乙巳改吏部為文部兵部為武部刑部為憲部 戶部侍郎兼御史大夫京兆尹王鉷權寵日盛領二十餘使宅旁為使院文案盈積吏求署一字累日不得前中使賜賚不絶於門【使疏吏翻賚來代翻】雖李林甫亦畏避之林甫子岫為將作監鉷子凖為衛尉少卿俱供奉禁中凖陵侮岫岫常下之【下遐嫁翻】然鉷事林甫謹林甫雖忌其寵不忍害也凖嘗帥其徒過駙馬都尉王繇【王繇同皎之子也帥讀曰率過工禾翻】繇望塵拜伏凖挾彈命中於繇冠折其玉簪以為戲笑【彈徒旦翻命中者先命其處而後中之中竹仲翻折而設翻】既而繇延凖置酒繇所尚永穆公主上之愛女也【黄琮曰皇女有美名公主或以德或以容或以福壽設名最多又有郡公主小國中國等】為凖親執刀【刀匕宰夫之職記杜蕢曰蕢也宰夫也惟刀匕是共為于偽翻下同】凖去或謂繇曰鼠雖挾其父勢【稱之為鼠微之也】君乃使公主為之具食【為于偽翻】有如上聞無乃非宜繇曰上雖怒無害至于七郎【王鉷第七】死生所繫不敢不爾鉷弟戶部郎中銲凶險不法【銲何旦翻】召術士任海川【任音壬姓也】問我有王者之相否【相悉亮翻】海川懼亡匿鉷恐事洩捕得託以它事杖殺之王府司馬韋會定安公主之子王繇之同產也【定安公主中宗女下嫁王司徒生繇又嫁韋濯生會】話之私庭鉷使長安尉賈季鄰收會繫獄縊殺之【縊於計翻】繇不敢言銲所善邢縡【縡作代翻】與龍武萬騎謀殺龍武將軍以其兵作亂殺李林甫陳希烈楊國忠前期二日有告之者夏四月乙酉上臨朝【朝直遙翻】以告狀面授鉷使捕之鉷意銲在縡所先使人召之日晏乃命賈季鄰等捕縡縡居金城坊【金城坊朱雀街西第四街之北來第二坊漢顧成廟傳望苑戾園皆在焉】季鄰等至門縡帥其黨數十人持弓刀格鬭突出【帥讀曰率】鉷與楊國忠引兵繼至縡黨曰勿傷大夫人【言勿傷鉷所部人也大夫稱鉷之官】國忠之傔密謂國忠曰賊有號不可戰也【今人謂私記為號言賊私為記號以相識别傔苦念翻】縡鬭且走至皇城西南隅【京城之内有皇城皇城之内有宫城】會高力士引飛龍禁軍四百至【飛龍禁軍乘飛龍廐馬者也武后置仗内六閑一曰飛龍以中官為内飛龍使】擊斬縡捕其黨皆擒之國忠以狀白上曰鉷必預謀上以鉷任遇深不應與同逆李林甫亦為之辯解【為于偽翻】上乃特命原銲不問然意欲鉷表請罪之使國忠諷之鉷不忍上怒會陳希烈極言鉷大逆當誅戊子敕希烈與國忠鞫之仍以國忠兼京兆尹於是任海川韋會等事皆發獄具鉷賜自盡銲杖死於朝堂【朝直遥翻】鉷子凖偁流嶺南【偁齒繩翻】尋殺之有司籍其第舍數日不能徧鉷賓佐莫敢窺其門獨采訪判官裴冕收其尸葬之【王鉷蓋兼京畿采訪使】 初李林甫以陳希烈易制引為相【事見上卷五載易以䜴翻】政事常隨林甫左右晩節遂與林甫為敵林甫懼會李獻忠叛林甫乃請解朔方節制且薦河西節度使安思順自代庚子以思順為朔方節度使【使疏吏翻】 五月戊申慶王琮薨贈靖德太子【琮徂宗翻贈字之下逸謚字既曰贈矣無謚字亦可】 丙辰京兆尹楊國忠加御史大夫京畿關内采訪等使凡王鉷所綰使務悉歸國忠【鉷戶公翻使疏吏翻】初李林甫以國忠微才且貴妃之族故善遇之國忠與王鉷俱為中丞鉷用林甫薦為大夫故國忠不悦遂深探邢縡獄【探吐南翻縡作代翻】令引林甫交私鉷兄弟及阿布思事狀【令力丁翻】陳希烈哥舒翰從而證之上由是疎林甫國忠貴震天下始與林甫為仇敵矣 六月甲子楊國忠奏吐蕃兵六十萬救南詔【吐從暾入聲】劒南兵擊破之於雲南克故隰州等三城 【考異曰實錄兵部侍郎兼御史中丞劒南節度使楊國忠破吐蕃于雲南拔故隰州等三城獻俘于朝唐歷國忠上言破吐蕃于雲南拔故洪州等三城按國忠時在長安蓋劒南破吐蕃以國忠領節制故使之上表獻俘耳時國忠已為大夫云中丞誤也隰州從實錄】捕虜六千三百以道遠簡壯者千餘人及酋長降者獻之【酋慈由翻長知兩翻降戶江翻】 秋八月乙丑上復幸左藏賜羣臣帛【蜀本作己丑當從之八載已嘗幸左藏賜群臣帛矣故此書復復扶又翻藏音徂浪翻】癸巳楊國忠奏有鳳皇見左藏庫屋出納判官魏仲犀言鳳集庫西通訓門【左藏舊有令丞而已出納判官蓋帝置也是時分立諸使舊來司存之官備員莫得舉其職楊國忠方承恩遇領使最多蓋兼領左藏出納使而以魏仲犀為判官也宋白曰天寶二年始命張瑄充太府出納使閣本太極宫西左藏庫之西則通明門見賢遍翻】九月阿布思入寇圍永清柵【永清柵亦曰永濟柵在中受降城之西二百里大同川】柵使張元軌拒却之【使疏吏翻】 冬十月戊寅上幸華清宫 己亥改通訓門曰鳳集門魏仲犀遷殿中侍御史楊國忠屬吏率以鳳皇優得調【調徒釣翻】 南詔數寇邊蜀人請楊國忠赴鎮【去年楊國忠領劒南蜀人困于兵故請之數所角翻】左僕射兼右相李林甫奏遣之國忠將行泣辭上言必為林甫所害貴妃亦為之請上謂國忠曰卿蹔到蜀區處軍事朕屈指待卿還當入相【屈指計日以待之亦為于偽翻處昌呂翻相息亮翻】林甫時已有疾憂懣不知所為【懣莫困翻又莫緩翻中煩也】巫言一見上可小愈上欲就視之左右固諫上乃令林甫出庭中【林甫時蓋卧疾昭應私第】上登降聖閣遥望【天寶七載十二月以玄元皇帝見於朝元閣改為降聖閣在華清宫中】以紅巾招之【今富貴之家帨巾率以臙脂染之為真紅色唐之遺俗也】林甫不能拜使人代拜國忠比至蜀【比必利翻及也】上遣中使召還【中使疏吏翻】至昭應謁林甫拜於牀下林甫流涕謂曰林甫死矣公必為相以後事累公【累力瑞翻】國忠謝不敢當汗出覆面【國忠素憚林甫故然覆敷又翻】十一月丁卯林甫薨上晩年自恃承平以為天下無復可憂【復扶又翻】遂深居禁中專以聲色自娛悉委政事於林甫林甫媚事左右迎合上意以固其寵杜絶言路掩蔽聰明以成其姦妬賢疾能排抑勝己以保其位屢起大獄誅逐貴臣以張其埶【張知亮翻】自皇太子以下畏之側足凡在相位十九年【開元二十二年始相林甫至是年凡十九年】養成天下之亂而上不之寤也 庚申以楊國忠為右相兼文部尚書【右相即中書令文部即吏部】其判使並如故【判如判度支之類使謂諸使使疏吏翻】國忠為人彊辯而輕躁無威儀既為相以天下為己任裁決機務果敢不疑居朝廷攘袂扼腕【躁則到翻相息亮翻朝直遙翻腕烏貫翻】公卿以下頤指氣使莫不震慴【慴之涉翻】自侍御史至為相【楊國忠兼侍御史在六載七載之間】凡領四十餘使【楊國忠為度支郎領十五餘使至宰相凡領四十餘使新舊唐史皆不詳載其職案其拜相制前衘云御史大夫判度支權知太府卿事兼蜀郡長史劒南節度支度營田等副大使本道兼山南西道采訪處置使兩京太府司農出納監倉祀祭木炭宫市長春九成宫等使關内道及京畿采訪處置使拜右相兼吏部尚書集賢殿崇玄館學士修國史太清太微宫使自餘所領又有管當租庸鑄錢等使以是觀之概可見】臺省官有才行時名【行下孟翻】不為己用者皆出之或勸陜郡進士張彖謁國忠【陜郡本陜州弘農郡天寶元年更郡名陜失冉翻】曰見之富貴立可圖彖曰君輩倚楊右相如泰山吾以為冰山耳若皎日既出君輩得無失所恃乎遂隱居嵩山國忠以司勲員外郎崔圓為劒南留後徵魏郡太守吉温為御史中丞充京畿關内採訪等使【魏郡魏州京畿關内先置兩採訪使今令温兼充】温詣范陽辭安祿山【魏郡屬河北道采訪使時祿山兼采訪使故温往辭】祿山令其子慶緒送至境為温控馬出驛數十步【為于偽翻】温至長安凡朝廷動靜輒報祿山信宿而達 十二月楊國忠欲收人望建議文部選人無問賢不肖選深者留之依資據闕注官【選須絹翻 考異曰唐歷此敕在十月二十七日統紀在七月舊紀十二月甲戌國忠奏請兩京選人銓日便定留放無長名按國忠作相始兼文部尚書七月未也今從舊紀】滯淹者翕然稱之國忠凡所施置皆曲盡人所欲故頗得衆譽 甲申以平盧兵馬使史思明兼北平太守充盧龍軍使【使疏吏翻】丁亥上還宫【還自華清宫 考異曰本紀唐歷皆云己亥還京今從實錄】 丁酉以安西行軍司馬封常清為安西四鎭節度使【唐制行軍司馬位節度副使之上天寶以後節鎭以為儲帥】 哥舒翰素與安祿山安思順不協上常和解之使為兄弟是冬三人俱入朝【朝直遥翻】上使高力士宴之於城東祿山謂翰曰我父胡母突厥公父突厥母胡族類頗同何得不相親翰曰古人云狐向窟嘷不祥為其忘本故也【厥九勿翻窟苦骨翻嘷戶刀翻為于偽翻】兄苟見親翰敢不盡心祿山以為譏其胡也大怒罵翰曰突厥敢爾翰欲應之力士目翰翰乃止陽醉而散自是為怨愈深 棣王琰有二孺人爭寵【曲禮大夫之妻曰孺人注孺之言屬正義曰孺屬也言其為親屬唐制親王有孺人二人視正五品孺而樹翻】其一使巫書符置琰履中以求媚琰與監院宦者有隙【時諸皇子列宿禁城之北宦者監之監古衘翻】宦者知之密奏琰祝詛上上使人掩其履而獲之大怒琰頓首謝臣實不知有符上使鞫之果孺人所為上猶疑琰知之囚于鷹狗坊【鷹狗坊屬閑廏使】絶朝請【朝有遥翻】憂憤而薨故事兵吏部尚書知政事者【知政事即宰相之職】選事悉委侍<br />
<br />
  郎以下【選須絹翻】三注三唱仍過門下省審自春及夏其事乃畢【唐制六品以下赴選始集而試觀其書判己試而銓察其身言已銓而注詢其便利而擬已注而唱不厭者得反通其辭三唱而不厭者聽冬集厭者為甲上于僕射乃上門下省給事中讀之黄門侍郎省之侍中審之然後以聞主者受旨而奉行焉謂之奏受省悉景翻】及楊國忠以宰相領文部尚書欲自示精敏乃遣令史先於私第密定名闕十二載春正月壬戌國忠召左相陳希烈及給事中諸司長官皆集尚書都堂【尚書都堂尚書都省之堂也長知兩翻】唱注選人一日而畢曰今左相給事中俱在座已過門下矣【左相即侍中與給事皆門下省官】其間資格差繆甚衆無敢言者於是門下不復過官【復扶又翻】侍郎但掌試判而已侍郎韋見素張倚趨走門庭與主事無異【吏部主事四人吏職也】見素湊之子也【韋凑見二百十卷睿宗景雲元年】京兆尹鮮于仲通諷選人請為國忠刻頌立於省門制仲通撰其辭上為改定數字【撰士免翻為于偽翻】仲通以金填之 楊國忠使人說安祿山【說式芮翻】誣李林甫與阿布思謀反祿山使阿布思部落降者詣闕誣告林甫與阿布思約為父子上信之下吏按問【降戶江翻下戶嫁翻】林甫壻諫議大夫楊齊宣懼為所累【累力瑞翻】附國忠意證成之時林甫尚未葬二月癸未制削林甫官爵子孫有官者除名流嶺南及黔中【黔音琴】給隨身衣及糧食自餘貲產並没官近親及黨與坐貶者五十餘人剖林甫棺抉取含珠禠金紫【抉於穴翻含戶紺翻褫敕豸翻】更以小棺如庶人禮葬之【更工衡翻】己亥賜陳希烈爵許國公楊國忠爵魏國公賞其成林甫之獄也 夏五月己酉復以魏周隋後為三恪【改三恪見上九載】楊國忠欲攻李林甫之短也衛包以助邪貶夜郎尉【夜郎縣屬溱州貞觀十六年開山洞置】崔冒貶烏雷尉【烏雷縣帶陸州】 阿布思為囘紇所破安祿山誘其部落而降之【誘音酉降戶江翻】由是祿山精兵天下莫及 壬辰以左武衛大將軍何復光將嶺南五府兵【五府廣桂邕容交將即亮翻】擊南詔安祿山以李林甫狡猾踰已故畏服之及楊國忠為相禄山視之蔑如也【蔑無也言覔之若無也】由是有隙國忠屢言祿山有反狀上不聽隴右節度使哥舒翰擊吐蕃拔洪濟太漠門等城悉收九曲部落【吐蕃得九曲地見二百十卷睿宗景雲元年廓州西南百四十里冇洪濟橋】 初高麗人王思禮與翰俱為押牙事王忠嗣翰為節度使思禮為兵馬使兼河源軍使【麗力知翻使疏吏翻】翰擊九曲思禮後期翰將斬之既而復召釋之思禮徐曰斬則遂斬復召何為【復扶又翻】楊國忠欲厚結翰共排安祿山奏以翰兼河西節度使秋八月戊戌賜翰爵西平郡王翰表侍御史裴冕為河西行軍司馬是時中國盛彊自安遠門西盡唐境萬二千里【長安城西面北來第一門曰安遠門本隋之開遠門也西盡唐境萬二千里併西域内屬諸國言之】閭閻相望桑麻翳野天下稱富庶者無如隴右翰每遣使入奏常乘白槖駞日馳五百里【使疏吏翻】 九月甲辰以突騎施黑姓可汗登里伊羅蜜施為突騎施可汗 北庭都護程千里追阿布思至磧西以書諭葛邏禄使相應【磧七迹翻邏郎佐翻】阿布思窮迫歸葛邏禄葛邏禄葉護執之并其妻子麾下數千人送之甲寅加葛邏禄葉護頓毗伽開府儀同三司賜爵金山王 冬十月戊寅上幸華清宫 【考異曰舊紀唐歷皆作戊申按長歷是月無戊申今從實錄然實錄在辛巳後蓋誤】楊國忠與虢國夫人居第相鄰【虢國居宣陽坊國忠居第在其西】晝夜往來無復期度或並轡走馬入朝不施鄣幕【婦人出必有障幕以自蔽復扶又翻朝直遥翻下同】道路為之掩目三夫人將從車駕幸華清宫【三夫人韓虢秦也為于偽翻】會於國忠第車馬僕從充溢數坊【從才用翻】錦繡珠玉鮮華奪目國忠謂客曰吾本寒家一旦緣椒房至此未知税駕之所然念終不能致令名不若且極樂耳楊氏五家隊各為一色衣以相别【樂音洛别彼列翻】五家合隊粲若雲錦【合音閤】國忠仍以劒南旌節引于其前國忠子暄舉明經學業荒陋不及格禮部侍郎逹奚珣畏國忠權勢遣其子昭應尉撫先白之撫伺國忠入朝上馬【伺相吏翻上時掌翻】趨至馬下國忠意其子必中選有喜色撫曰大人白相公郎君所試不中程式然亦未敢落也【落謂黜落也中竹仲翻】國忠怒曰我子何患不富貴乃令鼠輩相賣策馬不顧而去撫惶遽書白其父曰彼恃挾貴勢令人慘嗟安可復與論曲直【復扶又翻】遂置暄上第及暄為戶部侍郎珣始自禮部遷吏部暄與所親言猶歎已之淹囘珣之迅疾國忠既居要地中外餉遺輻湊【遺于季翻】積縑至三千萬匹 上在華清宫欲夜出遊龍武大將軍陳玄禮諫曰宫外即曠野安可不備不虞陛下必欲夜遊請歸城闕上為之引還【為于偽翻還從宣翻又如字下同】 是歲安西節度使封常清擊大勃律至菩薩勞城【新舊書並作賀菩勞城菩薄胡翻薩桑葛翻】前鋒屢捷常清乘勝逐之斥候府果毅段秀實諫曰【新書作隴州大堆府果獲此從舊書】虜兵羸而屢北誘我也【羸倫為翻誘音酉】請搜左右山林常清從之果獲伏兵遂大破之受降而還 中書舍人宋昱知選事前進士廣平劉迺以選法未善【廣平郡本洛州武安都天寶元年更名選須絹翻】上書於昱以為禹稷臯陶同居舜朝猶曰載采有九德考績以九載【書臯陶曰亦行有九德亦言其人有德乃言曰載采采禹曰何臯陶曰寛而栗柔而立愿而恭亂而敬擾而毅直而温簡而亷剛而塞彊而義彰厥有常吉哉又舜典曰三載考績三考黜陟幽明三考九載也上時掌翻陶余招翻九載子亥翻】近代主司察言于一幅之判觀行于一揖之間何古今遲速不侔之甚哉借使周公孔子今處銓廷【銓廷謂吏部銓量選人之所處昌呂翻】考其辭華則不及徐庾【徐陵庾信唐正元大歷以前皆尚其文】觀其利口則不若嗇夫【嗇夫事見十四卷漢文帝三年】何暇論聖賢之事業乎<br />
<br />
  資治通鑑卷二百十六<br />
<br />
<史部,編年類,資治通鑑>  <br>
   </div> 

<script src="/search/ajaxskft.js"> </script>
 <div class="clear"></div>
<br>
<br>
 <!-- a.d-->

 <!--
<div class="info_share">
</div> 
-->
 <!--info_share--></div>   <!-- end info_content-->
  </div> <!-- end l-->

<div class="r">   <!--r-->



<div class="sidebar"  style="margin-bottom:2px;">

 
<div class="sidebar_title">工具类大全</div>
<div class="sidebar_info">
<strong><a href="http://www.guoxuedashi.com/lsditu/" target="_blank">历史地图</a></strong>  
<a href="http://www.880114.com/" target="_blank">英语宝典</a>  
<a href="http://www.guoxuedashi.com/13jing/" target="_blank">十三经检索</a> 
<br><strong><a href="http://www.guoxuedashi.com/gjtsjc/" target="_blank">古今图书集成</a></strong> 
<a href="http://www.guoxuedashi.com/duilian/" target="_blank">对联大全</a> <strong><a href="http://www.guoxuedashi.com/xiangxingzi/" target="_blank">象形文字典</a></strong> 

<br><a href="http://www.guoxuedashi.com/zixing/yanbian/">字形演变</a>  <strong><a href="http://www.guoxuemi.com/hafo/" target="_blank">哈佛燕京中文善本特藏</a></strong>
<br><strong><a href="http://www.guoxuedashi.com/csfz/" target="_blank">丛书&方志检索器</a></strong> <a href="http://www.guoxuedashi.com/yqjyy/" target="_blank">一切经音义</a>  

<br><strong><a href="http://www.guoxuedashi.com/jiapu/" target="_blank">家谱族谱查询</a></strong>  <strong><a href="http://shufa.guoxuedashi.com/sfzitie/" target="_blank">书法字帖欣赏</a></strong> 
<br>

</div>
</div>


<div class="sidebar" style="margin-bottom:0px;">

<font style="font-size:22px;line-height:32px">QQ交流群9:489193090</font>


<div class="sidebar_title">手机APP 扫描或点击</div>
<div class="sidebar_info">
<table>
<tr>
	<td width=160><a href="http://m.guoxuedashi.com/app/" target="_blank"><img src="/img/gxds-sj.png" width="140"  border="0" alt="国学大师手机版"></a></td>
	<td>
<a href="http://www.guoxuedashi.com/download/" target="_blank">app软件下载专区</a><br>
<a href="http://www.guoxuedashi.com/download/gxds.php" target="_blank">《国学大师》下载</a><br>
<a href="http://www.guoxuedashi.com/download/kxzd.php" target="_blank">《汉字宝典》下载</a><br>
<a href="http://www.guoxuedashi.com/download/scqbd.php" target="_blank">《诗词曲宝典》下载</a><br>
<a href="http://www.guoxuedashi.com/SiKuQuanShu/skqs.php" target="_blank">《四库全书》下载</a><br>
</td>
</tr>
</table>

</div>
</div>


<div class="sidebar2">
<center>


</center>
</div>

<div class="sidebar"  style="margin-bottom:2px;">
<div class="sidebar_title">网站使用教程</div>
<div class="sidebar_info">
<a href="http://www.guoxuedashi.com/help/gjsearch.php" target="_blank">如何在国学大师网下载古籍?</a><br>
<a href="http://www.guoxuedashi.com/zidian/bujian/bjjc.php" target="_blank">如何使用部件查字法快速查字?</a><br>
<a href="http://www.guoxuedashi.com/search/sjc.php" target="_blank">如何在指定的书籍中全文检索?</a><br>
<a href="http://www.guoxuedashi.com/search/skjc.php" target="_blank">如何找到一句话在《四库全书》哪一页?</a><br>
</div>
</div>


<div class="sidebar">
<div class="sidebar_title">热门书籍</div>
<div class="sidebar_info">
<a href="/so.php?sokey=%E8%B5%84%E6%B2%BB%E9%80%9A%E9%89%B4&kt=1">资治通鉴</a> <a href="/24shi/"><strong>二十四史</strong></a>&nbsp; <a href="/a2694/">野史</a>&nbsp; <a href="/SiKuQuanShu/"><strong>四库全书</strong></a>&nbsp;<a href="http://www.guoxuedashi.com/SiKuQuanShu/fanti/">繁体</a>
<br><a href="/so.php?sokey=%E7%BA%A2%E6%A5%BC%E6%A2%A6&kt=1">红楼梦</a> <a href="/a/1858x/">三国演义</a> <a href="/a/1038k/">水浒传</a> <a href="/a/1046t/">西游记</a> <a href="/a/1914o/">封神演义</a>
<br>
<a href="http://www.guoxuedashi.com/so.php?sokeygx=%E4%B8%87%E6%9C%89%E6%96%87%E5%BA%93&submit=&kt=1">万有文库</a> <a href="/a/780t/">古文观止</a> <a href="/a/1024l/">文心雕龙</a> <a href="/a/1704n/">全唐诗</a> <a href="/a/1705h/">全宋词</a>
<br><a href="http://www.guoxuedashi.com/so.php?sokeygx=%E7%99%BE%E8%A1%B2%E6%9C%AC%E4%BA%8C%E5%8D%81%E5%9B%9B%E5%8F%B2&submit=&kt=1"><strong>百衲本二十四史</strong></a>  <a href="http://www.guoxuedashi.com/so.php?sokeygx=%E5%8F%A4%E4%BB%8A%E5%9B%BE%E4%B9%A6%E9%9B%86%E6%88%90&submit=&kt=1"><strong>古今图书集成</strong></a>
<br>

<a href="http://www.guoxuedashi.com/so.php?sokeygx=%E4%B8%9B%E4%B9%A6%E9%9B%86%E6%88%90&submit=&kt=1">丛书集成</a> 
<a href="http://www.guoxuedashi.com/so.php?sokeygx=%E5%9B%9B%E9%83%A8%E4%B8%9B%E5%88%8A&submit=&kt=1"><strong>四部丛刊</strong></a>  
<a href="http://www.guoxuedashi.com/so.php?sokeygx=%E8%AF%B4%E6%96%87%E8%A7%A3%E5%AD%97&submit=&kt=1">說文解字</a> <a href="http://www.guoxuedashi.com/so.php?sokeygx=%E5%85%A8%E4%B8%8A%E5%8F%A4&submit=&kt=1">三国六朝文</a>
<br><a href="http://www.guoxuedashi.com/so.php?sokeytm=%E6%97%A5%E6%9C%AC%E5%86%85%E9%98%81%E6%96%87%E5%BA%93&submit=&kt=1"><strong>日本内阁文库</strong></a> <a href="http://www.guoxuedashi.com/so.php?sokeytm=%E5%9B%BD%E5%9B%BE%E6%96%B9%E5%BF%97%E5%90%88%E9%9B%86&ka=100&submit=">国图方志合集</a> <a href="http://www.guoxuedashi.com/so.php?sokeytm=%E5%90%84%E5%9C%B0%E6%96%B9%E5%BF%97&submit=&kt=1"><strong>各地方志</strong></a>

</div>
</div>


<div class="sidebar2">
<center>

</center>
</div>
<div class="sidebar greenbar">
<div class="sidebar_title green">四库全书</div>
<div class="sidebar_info">

《四库全书》是中国古代最大的丛书,编撰于乾隆年间,由纪昀等360多位高官、学者编撰,3800多人抄写,费时十三年编成。丛书分经、史、子、集四部,故名四库。共有3500多种书,7.9万卷,3.6万册,约8亿字,基本上囊括了古代所有图书,故称“全书”。<a href="http://www.guoxuedashi.com/SiKuQuanShu/">详细>>
</a>

</div> 
</div>

</div>  <!--end r-->

</div>
<!-- 内容区END --> 

<!-- 页脚开始 -->
<div class="shh">

</div>

<div class="w1180" style="margin-top:8px;">
<center><script src="http://www.guoxuedashi.com/img/plus.php?id=3"></script></center>
</div>
<div class="w1180 foot">
<a href="/b/thanks.php">特别致谢</a> | <a href="javascript:window.external.AddFavorite(document.location.href,document.title);">收藏本站</a> | <a href="#">欢迎投稿</a> | <a href="http://www.guoxuedashi.com/forum/">意见建议</a> | <a href="http://www.guoxuemi.com/">国学迷</a> | <a href="http://www.shuowen.net/">说文网</a><script language="javascript" type="text/javascript" src="https://js.users.51.la/17753172.js"></script><br />
  Copyright &copy; 国学大师 古典图书集成 All Rights Reserved.<br>
  
  <span style="font-size:14px">免责声明:本站非营利性站点,以方便网友为主,仅供学习研究。<br>内容由热心网友提供和网上收集,不保留版权。若侵犯了您的权益,来信即刪。scp168@qq.com</span>
  <br />
ICP证:<a href="http://www.beian.miit.gov.cn/" target="_blank">鲁ICP备19060063号</a></div>
<!-- 页脚END --> 
<script src="http://www.guoxuedashi.com/img/plus.php?id=22"></script>
<script src="http://www.guoxuedashi.com/img/tongji.js"></script>

</body>
</html>
