\chapter{資治通鑑卷二百二十一}
宋 司馬光 撰

胡三省 音註

唐紀三十七|{
	起屠維大淵獻盡上章困敦凡二年}


肅宗文明武德大聖大宣孝皇帝下之上

乾元二年春正月己巳朔史思明築壇於魏州城北自稱大聖燕王以周摯為行軍司馬 |{
	考異曰河洛春秋作周萬至邠志作周至舊傳作周贄今從實錄}
李光弼曰思明得魏州而按兵不進此欲使我懈惰而以精銳掩吾不備也請與朔方軍同逼魏城求與之戰彼懲嘉山之敗|{
	嘉山之敗事見二百十八卷至德元載}
必不敢輕出得曠日引久則鄴城必拔矣慶緒已死彼則無辭以用其衆也魚朝恩以為不可乃止|{
	使用光弼之計安有水之潰乎朝直遥翻}
戊寅上祀九宫貴神|{
	李心傳曰九宫貴神者太一攝提權主招揺天符青龍咸池太隂天一宋白曰九宫貴神其說本之黄帝九宫經蕭吉五行大義}
用王嶼之言也乙卯耕藉田|{
	乙卯當作乙酉}
鎮西節度使李嗣業攻鄴城為流矢所中|{
	中竹仲翻}
丙申薨兵馬使荔非元禮代將其衆|{
	將即亮翻}
初嗣業表段秀實為懷州長史知留後事|{
	李嗣業以鎮西北庭兵屯懷州會師攻鄴以段秀實知留後事}
時諸軍屯戍日久財竭糧盡秀實獨運芻粟募兵市馬以奉鎮西行營相繼於道 二月壬子月食既|{
	春秋之法書日食不書月食日君象也此因張后之專横而書月食記曰男教不脩陽事不得謫見於天日為之食婦順不脩隂事不得謫見於天月為之食是故日食則天子素服而脩六官之職蕩天下之陽事月食則后素服而脩六宫之職蕩天下之隂事故天子之與后猶日之與月隂之與陽相須而成者也是後月食皆書於目録上方}
先是百官請加皇后尊號曰輔聖|{
	先悉薦翻 考異曰舊紀作翊聖今從實録}
上以問中書舍人李揆對曰自古皇后無尊號惟韋后有之|{
	韋后事見二百八卷中宗景龍元年}
豈足為法上驚曰庸人幾誤我會月食事遂寢后與李輔國相表裏横於禁中|{
	幾居依翻横戶孟翻}
干預政事請託無窮上頗不悦而無如之何 郭子儀等九節度使圍鄴城築壘再重穿塹三重|{
	重直龍翻}
壅漳水灌之城中井泉皆溢搆棧而居自冬涉春安慶緒堅守以待史思明食盡一鼠直錢四千淘牆䴰及馬矢以食馬|{
	䴰與職翻先以麥䴰雜土築牆今圍急乏芻故淘䴰以飼馬食祥吏翻}
人皆以為克在朝夕而諸軍既無統帥|{
	帥所類翻}
進退無所禀|{
	禀禀令也禀必錦翻行軍進退必稟令於主帥今諸軍無所禀也}
城中人欲降者礙水深不得出|{
	降戶江翻}
城久不下上下解體|{
	師老勢屈故解體}
思明乃自魏州引兵趣鄴|{
	果如李光弼之言趣七喻翻}
使諸將去城各五十里為營每營擊鼓三百面遥脅之乂每營選精騎五百日於城下抄掠|{
	騎奇寄翻抄楚交翻}
官軍出輒散歸其營諸軍人馬牛車日有所失樵採甚艱晝備之則夜至夜備之則晝至時天下饑饉轉餉者南自江淮西自并汾舟車相繼思明多遣壯士竊官軍装號督趣運者|{
	趣讀曰促}
責其稽緩妄殺戮人運者駭懼舟車所聚則密縱火焚之往復聚散自相辨識而官軍邏捕不能察也|{
	邏郎佐翻}
由是諸軍乏食人思自潰思明乃引大軍直抵城下|{
	觀史思明用兵所謂盗亦有道焉}
官軍與之刻日决戰三月壬申官軍步騎六十萬陳於安陽河北|{
	陳讀曰陣下布陳同水逕安陽縣而東流謂之安陽河}
思明自將精兵五萬敵之|{
	將即亮翻}
諸軍望之以為遊軍未介意思明直前奮擊李光弼王思禮許叔冀魯炅先與之戰殺傷相半魯炅中流矢|{
	中竹仲翻}
郭子儀承其後未及布陳大風忽起吹沙拔木天地晝晦咫尺不相辨兩軍大驚官軍潰而南賊潰而北棄甲仗輜重委積於路|{
	史言水之戰天未悔禍非戰之罪使皆如李光弼王思禮在亂能整則其失亡不至於甚重直用翻}
子儀以朔方軍斷河陽橋保東京|{
	斷音短}
戰馬萬匹惟存三千甲仗十萬遺棄殆盡東京士民驚駭散奔山谷留守崔圓河南尹蘇震等官吏南奔襄鄧|{
	守式又翻襄鄧二州屬山南東道}
諸節度各潰歸本鎮士卒所過剽掠|{
	剽匹妙翻}
吏不能止旬日方定惟李光弼王思禮整勒部伍全軍以歸 |{
	考異曰邠志曰史思明自稱燕王牙前兵馬使吳思禮曰思明果反盖蕃將也安肯盡節於國家因目左武鋒使僕固懷恩懷恩色變隂恨之三月六日史思明輕兵抵相州郭公率諸軍禦之戰于萬金驛賊分馬軍並而西郭公使僕固懷恩以蕃渾馬軍邀擊破之還遇吳思禮於陣射殺之呼曰吳思禮陣没其夕收軍郭公疑懷恩為變遂脫身先去諸軍相繼潰于城下今從實録}
子儀至河陽將謀城守師人相驚又奔缺門|{
	水經注穀水出弘農澠池縣南又東逕新安縣故城南又東逕千秋亭南又東逕缺門山山阜之不接者里餘故得是名}
諸將繼至衆及數萬議捐東京退保蒲陜|{
	將即亮翻捐于專翻陜失冉翻蒲陜二州夾河潼關控其險可以禦敵故議退保之}
都虞侯張用濟曰蒲陜荐飢不如守河陽賊至併力拒之子儀從之使都游奕使靈武韓遊瓌將五百騎前趣河陽|{
	使疏吏翻瓌古回翻騎奇寄翻趣七喻翻}
用濟以步卒五千繼之周摯引兵争河陽後至不得入而去用濟役所部兵築南北兩城而守之|{
	是後李光弼雖斬張用濟而守河陽則實張用濟定計於其先也}
段秀實帥將士妻子及公私輜重自野戍渡河待命於河清之南岸|{
	野戍即野水渡置戍守之因謂之野戍河清縣本屬河南尹本大基縣武德二年置八年省咸亨四年復分河南洛陽新安王屋濟源河陽置大基先天元年更名河清帥讀曰率將即亮翻輜莊持翻重直用翻}
荔非元禮至而軍焉諸將各上表謝罪|{
	上時掌翻}
上皆不問惟削崔圓階封|{
	崔圓先封趙國公實封戶五百國公從一品階比開府儀同三司}
貶蘇震為濟王府長史削銀青階|{
	濟王環上弟也濟子禮翻長知兩翻}
史思明審知官軍潰去自沙河收整士衆還屯鄴城南|{
	史思明之兵潰而北去至沙河知官軍的去乃收整其衆而南使官軍於水驚潰之後各能收兵還營堅壁而圍守鄴城思明未敢南也沙河縣隋分龍岡縣置唐屬邢州在鄴城西北二百餘里還音旋又音如字}
安慶緒收子儀營中糧得六七萬石與孫孝哲崔乾祐謀閉門更拒思明諸將曰今日豈可復背史王乎|{
	復扶又翻背蒲妹翻}
思明不與慶緒相聞又不南追官軍但日於軍中饗士張通儒高尚等言於慶緒曰史王遠來臣等皆應迎謝|{
	應乙陵翻}
慶緒曰任公蹔往思明見之涕泣厚禮而歸之經三日慶緒不至思明密召安太清令誘之|{
	蹔與暫同令力丁翻誘音酉}
慶緒窘蹙不知所為乃遣太清上表稱臣於思明請待解甲入城奉上璽綬|{
	窘巨隕翻上時掌翻璽斯氏翻綬音受}
思明省表曰何至如此因出表徧示將士咸稱萬歲|{
	省昔景翻思明出慶緒表徧示將士以觀其情向背}
乃手疏唁慶緒|{
	疏所據翻唁魚戰翻弔生曰唁}
而不稱臣且曰願為兄弟之國更作藩籬之援鼎足而立猶或庶幾北面之禮固不敢受并封表還之慶緒大悦因請歃血同盟思明許之慶緒以三百騎詣思明營思明令軍士擐甲執兵以待之|{
	幾居希翻歃色甲翻騎奇計翻擐音宦}
引慶緒及諸弟入至庭下慶緒再拜稽首曰臣不克荷負|{
	稽音啓荷下可翻又如字}
棄失兩都久陷重圍|{
	重直龍翻}
不意大王以太上皇之故|{
	慶緒尊禄山為太上皇見二百十九卷至德元載}
遠垂救援使臣應死復生|{
	復扶又翻又如字}
摩頂至踵無以報德思明忽震怒曰棄失兩都亦何足言爾為人子殺父奪其位天地所不容吾為太上皇討賊|{
	吾為音于偽翻}
豈受爾佞媚乎即命左右牽出并其四弟及高尚孫孝哲崔乾祐皆殺之張通儒李庭望等悉授以官思明勒兵入鄴城收其士馬以府庫賞將士慶緒先所有州縣及兵皆歸於思明遣安太清將兵五千取懷州因留鎮之思明欲遂西略慮根本未固乃留其子朝義守相州|{
	朝直遥翻}
引兵還范陽 甲申回紇骨啜特勒帝德等十五人自相州奔還西京上宴之於紫宸殿|{
	宋敏求長安志宣政殿北曰紫宸門門内有紫宸殿即内衙之正殿}
賞賜有差庚寅骨啜特勒等辭還行營 辛卯以荔非元禮為懷州刺史權知鎮西北庭行營節度使元禮復以段秀實為節度判官|{
	復扶又翻}
甲午以兵部侍郎呂諲同平章事乙未以中書侍郎同平章事苖晉卿為太子太傳王嶼為刑部尚書皆罷政事以京兆尹李峴行吏部尚書中書舍人兼禮部侍郎李揆為中書侍郎及戶部侍郎第五琦並同平章事上於峴恩意尤厚峴亦以經濟為己任軍國大事皆獨决於峴|{
	為李輔國忌峴不得久於相位張本}
於是京師多盗李輔國請選羽林騎士五百以備廵邏|{
	羅郎佐翻}
李揆上疏曰昔西漢以南北軍相制故周勃因南軍入北軍遂安劉氏|{
	周勃安劉事見漢高后紀李揆謂勃囚南軍入北軍考其本末恐不如此}
皇朝置南北牙文武區分以相伺察今以羽林代金吾警夜忽有非常之變將何以制之乃止|{
	金吾衛屬南牙羽林衛屬北牙金吾掌廵徼李輔國欲以羽林軍奪其職故李揆以為言朝直遥翻}
丙申以郭子儀為東畿山東河東諸道元帥權知東京留守|{
	東畿謂東京畿山東謂河南河北河東自蒲絳北至并代}
以河西節度使來瑱行陜州刺史充陜虢華州節度使|{
	來瑱徙河西未行而相州師潰因使之鎮陜以守關然瑱尋徙襄陽華戶化翻}
夏四月庚子澤潞節度使王思禮破史思明將楊旻於潞城東|{
	潞城縣屬潞州隋開皇十六年置春秋潞子所邑也九域志潞城在潞州東北四十里}
太子詹事李輔國自上在靈武判元帥行軍司馬事侍直帷幄宣傳詔命四方文奏寶印符契晨夕軍號一以委之及還京師專掌禁兵常居内宅|{
	内宅蓋在禁中輔國止宿之署舍也}
制勑必經輔國押署然後施行宰相百司非時奏事皆因輔國關白承旨常於銀臺門决天下事|{
	雍錄按六典制大明宫圖有左右銀臺門左銀臺門直紫宸殿之東右銀臺門直紫宸殿之西又考閤本大明宫圖右銀臺門内即翰林院麟德殿又東歷内侍别省延英殿光順門而後至紫宸殿自左銀臺門西入歷温室浴堂殿綾綺殿而後至紫宸殿紫宸殿在宣政殿後當大明宫正中右銀臺門在宫城西面左銀臺門在宫城東面以地望準之正直紫宸東西耳}
事無大小輔國口為制勑寫付外施行事畢聞奏又置察事數十人潛令於人間聽察細事即行推按有所追索諸司無敢拒者御史臺大理寺重囚或推斷未畢輔國追詣銀臺一時縱之|{
	索山客翻斷丁亂翻}
三司府縣鞫獄皆先詣輔國咨禀輕重隨意稱制勑行之莫敢違者宦官不敢斥其官皆謂之五郎李揆山東甲族見輔國執子弟禮謂之五父|{
	李揆裔出隴西其先客居滎陽遂為山東甲族李輔國第五}
及李峴為相於上前叩頭論制勑皆應由中書出具陳輔國專權亂政之狀上感悟賞其正直|{
	峴戶典翻相息亮翻}
輔國行事多所變更|{
	更工衛翻}
罷其察事輔國由是讓行軍司馬請歸本官|{
	本官太子詹事}
上不許制比緣軍國務殷或宣口勑處分|{
	比毗至翻處昌呂翻分扶問翻}
諸色取索及杖配囚徒自今一切並停如非正宣並不得行|{
	索山客翻正宣宣命凡出宣命有㡳在中書可以檢覆謂之正宣}
中外諸務各歸有司英武軍虞候及六軍諸使諸司等比來或因論競懸自追攝|{
	英武軍殿前射生手也置虞候以統之六軍北門六軍也諸使内諸使也諸司内諸司也使疏吏翻論盧昆翻}
自今須一切經臺府|{
	臺御史臺府京兆府}
如所由處斷不平|{
	處昌呂翻斷丁亂翻}
聽具狀奏聞諸律令除十惡殺人姦盗造偽外餘煩冗一切刪除仍委中書門下與法官詳定聞奏輔國由是忌峴 |{
	考異曰實録李峴傳曰時李輔國專典禁中兵權詔旨或不由中書而出峴切陳其狀肅宗甚嘉之即日下詔如峴奏由是挫輔國威權輔國頗忌之盖即此詔也}
甲辰置陳鄭亳節度使以鄧州刺史魯炅為之以徐州刺史尚衡為青密七州節度使|{
	七州青密登萊淄沂海炅火迥翻}
以興平軍節度使李奐兼豫許汝三州節度使仍各於境上守捉防禦|{
	陳鄭亳前此未嘗置節鎮魯炅自南陽為之青密等七州尚衡自彭城升統之興平軍本置于雍州始平縣李奐時在行營使統豫許汝三州此皆臨時分鎮非有一定規模也}
九節度之潰於相州也魯炅所部兵剽掠尤甚|{
	剽匹妙翻}
聞郭子儀退屯河上李光弼還太原炅慙懼飲藥而死|{
	還從宣翻又音如字}
史思明自稱大燕皇帝改元順天|{
	燕因肩翻 考異作應天皇帝注曰河洛春秋曰上元三年春二月思明懷西侵之謀慮北地之變乃令男朝義留守相城自領士馬歸范陽因僭號後燕改元順天元年按實録此年正月一日思明稱燕王立年號實錄舊傳皆不載所改年名紀年通譜此年即思明順天元年柳璨正閏位歷思明有順天應天二號按薊門紀亂思明既殺烏承恩不稱國家正朔亦不受慶緒指麾境内但稱某月而已乾元二年四月癸酉思明僭位於范陽建元順天國號大燕立妻辛氏為皇后次子朝興為皇太子長子朝義為懷王六月於開元寺造塔改寺名為順天上元二年正月癸卯思明大赦改元應天實録云正月立年號河洛春秋云上元三年僭號薊門紀亂云立朝興為太子按思明欲立少子為太子左右泄其謀故朝義弑之紀亂云於時已立為太子誤也按長歷四月丁酉朔無癸酉}
立其妻辛氏為皇后子朝義為懷王以周摯為相李歸仁為將|{
	朝直遥翻相息亮翻將即亮翻}
改范陽為燕京諸州為郡 戊申以鴻臚卿李抱玉為鄭陳潁亳節度使|{
	臚凌如翻使疏吏翻}
抱玉安興貴之後也|{
	安興貴見一百八十七卷高祖武德二年}
為李光弼禆將屢有戰功自陳恥與安祿山同姓故賜姓李氏 回紇毗伽闕可汗卒長子葉護先遇殺國人立其少子是為登里可汗|{
	紇下没翻伽求迦翻長知兩翻少始照翻可從刋入聲汗音寒卒子恤翻}
回紇欲以寜國公主為殉公主曰回紇慕中國之俗故娶中國女為婦若欲從其本俗何必結昏萬里之外邪|{
	邪音耶}
然亦為之剺面而哭|{
	漠北之俗死者停屍於帳子孫及親屬男女各殺牛馬陳於帳前祭之遶帳走馬七匝詣帳門以刀剺面且哭血淚俱流如此者七度乃止為于偽翻剺里之翻}
鳳翔馬坊押官為劫|{
	押官者管押馬坊之官}
天興尉謝夷甫捕殺之|{
	天興縣本古雍縣至德二載改曰鳳翔仍分置天興縣帶鳳翔府}
其妻訟寃李輔國素出飛龍廐|{
	李輔國本飛龍小兒}
勑監察御史孫鎣鞫之無寃|{
	監古銜翻鎣余傾翻又烏定翻}
又使御史中丞崔伯陽刑部侍郎李曄大理卿權獻鞫之|{
	此唐制所謂小三司也}
與鎣同猶不服又使侍御史太平毛若虚鞫之|{
	太平縣屬絳州魏太武帝置泰平縣周改為太平因太平關城為名}
若虚傾巧士希輔國意歸罪夷甫伯陽怒召若虚詰責欲劾奏之|{
	詰去吉翻劾戶槩翻又戶得翻}
若虚先自歸於上上匿若虚於簾下伯陽尋至言若虚附會中人鞫獄不直上怒叱出之伯陽貶高要尉獻貶桂陽尉|{
	桂陽漢縣隋唐帶連州}
曄與鳳翔尹嚴向皆貶嶺下尉|{
	嶺下謂度嶺南下諸縣史失曄向所貶縣名故云皆貶嶺下尉}
鎣除名長流播州吏部尚書同平章事李峴奏伯陽無罪責之太重上以為朋黨五月辛巳貶峴蜀州刺史|{
	尚辰羊翻峴戶典翻 考異曰代宗實錄云屬有盗發鳳翔管在北軍者詔遣御史訊鞫盗已伏罪李輔國執奏重覆殿中侍御史毛若虚奏覆與輔國協肅宗大怒下三司推鞫之峴以若虛不直陳於上前及三司覆奏與峴理協肅宗以為朋黨會同列李揆希旨遂貶峴為通州刺史三司大臣皆貶官今從肅宗實録舊紀傳}
右散騎常侍韓擇木入對|{
	散悉亶翻騎奇寄翻}
上謂之曰李峴欲專權今貶蜀州朕自覺用法太寛對曰李峴言直非專權陛下寛之秪益聖德耳若虚尋除御史中丞威振朝廷|{
	朝直遥翻}
壬午以滑濮節度使許叔冀為汴州刺史充滑汴等七州節度使|{
	新書方鎮表汴滑節度使治滑州領州五滑濮汴曹宋}
以試汝州刺史劉展為滑州刺史充副使 六月丁巳分朔方置邠寜等九州節度使|{
	方鎮表開元九年置朔方節度使領單于大都護府夏鹽綏銀豐勝六州定遠豐安二軍三受降城十年增領魯麗契三州二十二年兼關内道采訪處置使增涇原寧慶隴鄜坊丹延會宥麟十二州以匡長二州隸慶州安樂長樂二州隸原州天寶元年增領邠州乾元元年分鎮北大都護府麟勝二州置振武節度使是年廢關内節度使罷領單于大都護以涇原寧慶坊鄜丹延隸邠寧節度邠州本州開元十三年以字類幽改曰邠}
觀軍容使魚朝恩惡郭子儀|{
	惡烏路翻}
因其敗短之於上秋七月上召子儀還京師以李光弼代為朔方節度使兵馬元帥 |{
	考異曰邠志云四月肅宗使丞相張公鎬東都慰勉諸軍郭公陳饌於軍張公不坐而去軍中不悦朋肆流議居十日有中使追郭公汾陽家傳曰六月公朝于京師三讓元帥上許之乃詔李光弼代公為副段公别傳曰五月李光弼代子儀為副元帥守東都今因實錄七月除趙王係為元帥并言之}
士卒涕泣遮中使請留子儀子儀紿之曰我餞中使耳未行也因躍馬而去光弼願得親王為之副辛巳以趙王係為天下兵馬元帥光弼副之 |{
	考異曰舊傳思明縱兵河南加光弼太尉兼中書令代郭子儀為朔方節度兵馬副元帥以東師委之新傳云帝貸諸將罪以光弼兼幽州大都督府長史知諸道節度行營事又代子儀為朔方節度使未幾為天下兵馬副元帥按實録光弼加太尉中書令在上元元年破史思明後為幽州都督在此年八月其代子儀節制朔方實録無月日制辭云宜副出車之命仍踐分麾之寵盖只在此時耳}
仍以光弼知諸節度行營光弼以河東騎五百馳赴東都夜入其軍光弼治軍嚴整始至號令一施士卒壁壘旌旗精采皆變|{
	史言光弼入朔方軍部分皆因子儀之舊但號令加嚴整耳治直之翻}
是時朔方將士樂子儀之寛憚光弼之嚴|{
	樂音洛}
左廂兵馬使張用濟屯河陽光弼以檄召之用濟曰朔方非叛軍也乘夜而入何見疑之甚邪與諸將謀以精銳突入東京逐光弼請子儀命其士皆被甲上馬銜枚以待|{
	被皮義翻上時掌翻}
都知兵馬使僕固懷恩曰鄴城之潰郭公先去|{
	觀懷恩此言則邠志所云亦可以傳信}
朝廷責帥故罷其兵柄今逐李公而彊請之是反也其可乎|{
	帥所類翻彊其兩翻又音如字}
右武鋒使康元寶曰君以兵請郭公朝廷必疑郭公諷君為之是破其家也郭公百口何負於君乎用濟乃止|{
	懷恩此言與康元寶之言皆是也使諸將從張用濟於惡史思明之兵復至唐事殆矣}
光弼以數千騎東出汜水用濟單騎來謁光弼責用濟召不時至斬之命部將辛京杲代領其衆 |{
	考異曰舊傳曰用濟承子儀之寛懼光弼之令與諸將頗有異議欲逗留其衆光弼以數千騎出次汜水縣用濟單騎迎謁即斬於轅門諸將懾伏以辛京杲代之復追都兵馬使僕固懷恩懷恩懼先期而至邠志曰五月二十三日詔河東節度使李公代子儀兼統諸軍李公既受命以河東馬軍五百騎至東都夜入其軍張用濟在河陽聞之曰朔方軍非叛人也何見疑之甚欲率精騎突入東都逐李公請郭公李公知之遂留東都表請濟師于河陽冬十月思明引衆渡河李公曰思明渡河必圖洛城我當守武牢關揚兵于廣武以待之遂引兵東出師汜水縣檄追河陽諸將用濟後至李公數其罪而戮之以辛京杲代領其職明日引軍入河陽按實録此月光弼為副元帥九月始移軍河陽耳}
僕固懷恩繼至光弼引坐與語|{
	史言李光弼待僕固懷恩有加於諸將}
須臾閽者白蕃渾五百騎至矣|{
	蕃渾謂諸蕃種及渾種}
光弼變色懷恩走出召麾下將陽責之曰語汝勿來何得故違|{
	將即亮翻語牛倨翻}
光弼曰士卒随將亦復何罪命給牛酒|{
	史言懷恩成備而後見光弼光弼雖知其情而容忍不發復音扶又翻}
以潞沁節度使王思禮|{
	王思禮節度澤潞沁三州史或稱澤潞或稱潞沁沁七鴆翻}
兼太原尹充北京留守河東節度使|{
	代李光弼也}
初潼關之敗|{
	事見一百十八卷至德元載}
思禮馬中矢而斃有騎卒盩厔張光晟下馬授之|{
	中竹仲翻盩音輈厔音窒晟丞正翻}
問其姓名不告而去思禮隂識其狀貌|{
	識音誌}
求之不獲及至河東或譖代州刺史河西辛雲京|{
	雲京蘭州金城人屬河西路}
思禮怒之雲京懼不知所出光晟時在雲京麾下曰光晟嘗有德於王公從來不敢言者恥以此取賞耳今使君有急光晟請往見王公必為使君解之|{
	為于偽翻下特為同}
雲京喜而遣之光晟謁思禮未及言思禮識之曰噫子非吾故人乎何相見之晚邪光晟以實告思禮大喜執其手流涕曰吾之有今日皆子力也|{
	思禮言光晟授己以馬脫己於兵得有今日}
吾求子久矣引與同榻坐約為兄弟光晟因從容言雲京之寃|{
	從千容翻}
思禮曰雲京過亦不細今日特為故人捨之即日擢光晟為兵馬使贈金帛田宅甚厚|{
	張光晟於王思禮可謂君子矣其後事德宗以失職怨望遂委身於朱泚何前後之相違也}
辛卯以朔方節度副使殿中監僕固懷恩兼太常卿進爵大寜郡王懷恩從郭子儀為前鋒勇冠三軍|{
	冠古玩翻}
前後戰功居多故賞之 八月乙巳襄州將康楚元張嘉延據州作亂刺史王政奔荆州楚元自稱南楚覇王 回紇以寜國公主無子聽歸丙辰至京師|{
	公主嫁回紇見上卷上年}
戊午上使將軍曹日昇往襄州慰諭康楚元貶王政為饒州長史以司農少卿張光奇為襄州刺史楚元不從 壬戌以李光弼為幽州長史河北節度等使|{
	使之收復河北及幽燕也}
九月甲午張嘉延襲破荆州荆南節度使杜鴻漸棄城走澧朗郢峽歸等州官吏聞之争潛竄山谷|{
	時荆南節度使領荆澧朗郢復夔峽忠萬歸中州}
戊辰更令絳州鑄乾元重寶大錢|{
	唐世鑄錢大凡天下諸鑪九十九而絳州之鑪三十其餘諸鑪或隔江嶺或没寇虜故當時鑄錢率倚絳州}
加以重輪一當五十|{
	大錢徑一寸二分文亦曰乾元重寶背之外郭為重輪每緡重十二斤號重稜錢重直龍翻}
在京百官先以軍旅皆無俸祿宜以新錢給其冬料|{
	冬料各官冬季所當得俸料錢也}
丁亥以太子少保崔光遠為荆襄招討使充山南東道處置兵馬都使|{
	處昌呂翻}
以陳潁亳申節度使王仲昇為申沔等五州節度使知淮南西道行營兵馬|{
	時淮西節度使領申光夀安沔五州}
史思明使其子朝清守范陽命諸郡太守各將兵三千從己向河南分為四道使其將令狐彰將兵五千自黎陽濟河取滑州思明自濮陽史朝義自白臯周摯自胡良濟河|{
	白臯胡良皆河津濟度之要在滑州西北岸良或作梁濮音卜}
會於汴州李光弼方廵河上諸營聞之還入汴州謂汴滑節度使許叔冀曰大夫能守汴州十五日我則將兵來救叔冀許諾光弼還東京思明至汴州叔冀與戰不勝遂與濮州刺史董秦及其將梁浦劉從諫田神功等降之|{
	許叔冀卒如張鎬之言}
思明以叔冀為中書令與其將李詳守汴州厚待董秦收其妻子置長蘆為質|{
	長蘆漢參戶縣地後周更名長蘆縣時属滄州質音致}
使其將南德信與梁浦劉從諫田神功等數十人徇江淮神功南宫人也|{
	南宫漢古縣屬冀州}
思明以為平盧兵馬使頃之神功襲德信斬之從諫脫身走神功將其衆來降思明乘勝西攻鄭州|{
	鄭州滎陽郡}
光弼整衆徐行至洛陽謂留守韋陟曰賊乘勝而來利在按兵不利速戰洛城不可守於公計何如陟請留兵於陜退守潼關據險以挫其銳|{
	守式又翻陜失冉翻潼音同}
光弼曰兩敵相當貴進忌退今無故棄五百里地則賊勢益張矣|{
	張知亮翻又如字}
不若移軍河陽北連澤潞利則進取不利則退守表裏相應使賊不敢西侵此猿臂之勢也|{
	猿臂可伸而長可縮而短故以為喻}
夫辨朝廷之禮光弼不如公|{
	夫音扶朝直遥翻}
論軍旅之事公不如光弼陟無以應判官韋損曰東京帝宅侍中奈何不守|{
	按李光弼至德之初己為司空乾元元年為侍中故韋損以此呼之}
光弼曰守之則氾水崿嶺龍門皆應置兵|{
	汜水有成臯之險崿嶺在登封縣龍門則伊闕汜音祀崿逆各翻}
子為兵馬判官能守之乎遂移牒留守韋陟使帥東京官属西入關牒河南尹李若幽使帥吏民出城避賊空其城光弼帥軍士運油鐵諸物詣河陽為守備光弼以五百騎殿|{
	帥讀曰率殿丁練翻}
時思明遊兵已至石橋諸將請曰今自洛城而北乎當石橋而進乎光弼曰當石橋而進|{
	水經注穀水東逕洛陽廣莫門北漢之穀門也東逕建春門石橋下即上東門也此言漢晉洛城諸門非隋唐所徙洛城也上東門之地唐為鎮}
及日暮光弼秉炬徐行部曲堅重賊引兵躡之不敢逼|{
	躡之者欲其兇懼而自潰不敢逼者以其嚴整而難犯}
光弼夜至河陽有兵二萬|{
	郭子儀自水退守河陽衆及數萬及李光弼至河陽有兵二萬何衆寡之相懸乎蓋張用濟之死朔方士卒畏威而迯散者多也}
糧纔支十日光弼按閱守備部分士卒無不嚴辦|{
	分扶問翻 考異曰實録光弼謂韋陟曰洛城無糧不可守按河陽糧纔支十日亦非糧多也今不取}
庚寅思明入洛陽城空無所得畏光弼掎其後|{
	掎居綺翻}
不敢入宫退屯白馬寺南築月城於河陽南以拒光弼|{
	史思明乘銳勝以攻河陽乃先築月城者恐戰有邂逅也}
於是鄭滑等州相繼陷沒|{
	思明既至洛陽則鄭滑等州已陷沒矣通鑑因史家成文失於刪修也}
韋陟李若幽皆寓治於陜 冬十月丁酉下制親征史思明羣臣上表諫乃止 史思明引兵攻河陽使驍將劉龍仙詣城下挑戰|{
	驍堅堯翻挑徒了翻}
龍仙恃勇舉右足加馬鬛上慢罵光弼光弼顧諸將曰誰能取彼者僕固懷恩請行光弼曰此非大將所為|{
	光弼之言得體懷恩固心服矣}
左右言禆將白孝德可往光弼召問之孝德請行光弼問須幾何兵對曰請挺身取之光弼壯其志然固問所須對曰願選五十騎出壘門為後繼兼請大軍助鼔譟以增氣光弼撫其背而遣之|{
	既賞其勇而尤賞其有取敵之方略}
孝德挾二矛策馬亂流而進|{
	横絶流曰亂}
半涉懷恩賀曰克矣光弼曰鋒未交何以知之懷恩曰觀其攬轡安閑知其萬全龍仙見其獨來甚易之|{
	易以䜴翻}
稍近將動孝德揺手示之若非來為敵者龍仙不測而止去之十步乃與之言龍仙慢罵如初孝德息馬良久|{
	息馬者使馬力完復而後戰}
因瞋目謂曰賊識我乎龍仙曰誰也曰我白孝德也龍仙曰是何狗彘孝德大呼|{
	呼火故翻瞋昌真翻}
運矛躍馬之城上鼔譟五十騎繼進龍仙矢不及發環走隄上孝德追及斬首攜之以歸|{
	龍仙恃勇輕敵而孝德出其不意之故勝}
賊衆大駭孝德本安西胡人也思明有良馬千餘匹每日出於河南渚浴之循環不休以示多光弼命索軍中牝馬得五百匹|{
	索山客翻}
縶其駒於城内俟思明馬至水際盡出之馬嘶不已思明馬悉浮渡河一時驅之入城|{
	牡馬慕牝一時渡河此小術耳思明不能制阻河水也}
思明怒列戰船數百艘泛火船於前而隨之欲乘流燒浮橋光弼先貯百尺長竿數百枚|{
	艘蘇遭翻貯丁呂翻}
以巨木承其根氈裹鐵叉置其首以迎火船而叉之船不得進須臾自焚盡又以叉拒戰船於橋上發礮石擊之中者皆沉沒賊不勝而去思明見兵於河清|{
	礮匹貌翻中竹仲翻見賢遍翻杜佑曰河清縣南臨黄河}
欲絶光弼糧道光弼軍於野水渡以備之既夕還河陽留兵千人使部將雍希顥守其柵|{
	雍於用翻}
曰賊將高庭暉李日越喻文景皆萬人敵也|{
	喻姓也姓譜東晉有喻歸撰河西記}
思明必使一人來劫我我且去之汝待於此若賊至勿與之戰降則與之俱來諸將莫諭其意皆竊笑之既而思明果謂李日越曰李光弼長於憑城今出在野此成擒矣汝以鐵騎宵濟為我取之|{
	為于偽翻}
不得則勿返日越將五百騎晨至柵下希顥阻壕休卒吟嘯相視日越怪之|{
	怪其無戰意也}
問曰司空在乎|{
	李光弼加司空侍中故稱之}
曰夜去矣兵幾何曰千人將誰曰雍希顥日越默計久之謂其下曰今失李光弼得希顥而歸吾死必矣不如降也遂請降希顥與之俱見光弼光弼厚待之任以心腹高庭暉聞之亦降或問光弼降二將何易也|{
	易弋豉翻}
光弼曰此人情耳思明常恨不得野戰聞我在外以為必可取日越不獲我勢不敢歸庭暉才勇過於日越聞日越被寵任必思奪之矣|{
	此謂之善用其所短孫臏有言善戰者因其勢而利導之}
庭暉時為五臺府果毅|{
	代州有五臺府}
己亥以庭暉為右武衛大將軍|{
	唐諸府果毅品秩猶卑諸衛大將軍則三品矣 考異曰新傳曰上元元年光弼降賊將高暉李日越按此月己亥高庭暉授特進疑即高暉也丁巳李日越又授特進是此月皆已降新傳誤邠志曰二年三月思明引衆南去使其子朝義圍河陽四月一日思明陷洛城上元元年思明耀兵于河清宣言曰我且渡河絶彼餉道三城食盡不攻自下李公聞之師于野水渡既夕還軍與實録亦相違今從實録}
思明復攻河陽|{
	復扶又翻下日復同}
光弼謂鄭陳節度使李抱玉曰|{
	方鎮表乾元二年置鄭陳節度使領鄭陳亳潁四州然此時鄭州已沒於史思明矣}
將軍能為我守南城二日乎|{
	為于偽翻}
抱玉曰過期何如光弼曰過期救不至任棄之抱玉許諾勒兵拒守城且陷抱玉紿之曰吾糧盡明旦當降賊喜歛軍以待之抱玉繕完城備明日復請戰賊怒急攻之抱玉出奇兵表裏夾擊殺傷甚衆董秦從思明寇河陽夜帥其衆五百拔柵突圍降于光弼|{
	帥讀曰率下同}
時光弼自將屯中潬城外置柵柵外穿塹深廣二丈|{
	中河起石潬築城以衛河橋潬蕩旱翻爾雅潬沙出深式禁翻廣古曠翻}
乙巳賊將周摯捨南城併力攻中潬光弼命荔非元禮出勁卒於羊馬城以拒賊|{
	城外别築短垣高纔及肩謂之羊馬城}
光弼自於城東北隅建小朱旗以望賊賊恃其衆直進逼城以車載攻具自隨督衆填塹三面各八道以過兵又開柵為門光弼望賊逼城使問元禮曰中丞視賊填塹開柵過兵晏然不動何也元禮曰司空欲守乎戰乎光弼曰欲戰元禮曰欲戰則賊為吾填塹|{
	為于偽翻下保為同}
何為禁之光弼曰善吾所不及勉之|{
	雖賞其敢戰而戰危事也故曰勉之}
元禮俟柵開帥敢死士突出擊賊却走數百步元禮度賊陳堅未易摧陷|{
	度徒洛翻易弋豉翻}
乃復引退|{
	復扶又翻下摯復同}
須其怠而擊之光弼望元禮退怒遣左右召欲斬之元禮曰戰正急召何為乃退入柵中賊亦不敢逼良久鼔譟出柵門奮擊破之周摯復收兵趣北城|{
	趣七喻翻}
光弼遽帥衆入北城登城望賊曰賊兵雖多囂而不整不足畏也不過日中保為諸軍破之乃命諸將出戰及期不决召諸將問曰向來賊陳|{
	陳讀曰陣}
何方最堅曰西北隅光弼命其將郝庭玉當之|{
	廷玉光弼之愛將也}
廷玉請精兵五百與之三百又問其次堅者曰東南隅光弼命其將論惟貞當之|{
	論姓也諸論自吐蕃來降}
惟貞請鐵騎三百與之二百光弼令諸將曰爾曹望吾旗而戰吾颭旗緩任爾擇利而戰吾急颭旗三至地|{
	颭占琰翻}
則萬衆齊入死生以之少退者斬又以短刀置鞾中|{
	鞾與靴同釋名曰鞾本胡服趙武靈王所作實録曰胡履也趙武靈王好胡服常短䩓以黄皮為之後漸以長䩓軍戎通服唐馬周殺其䩓加以靴氈開元中裴叔通以羊為之隱麖加以帶子裝束故事胡虜之服不許着入殿省至馬周加飾乃許之}
曰戰危事吾國之三公不可死賊手萬一戰不利諸君前死於敵我自剄於此不令諸君獨死也諸將出戰頃之廷玉奔還光弼望之驚曰廷玉退吾事危矣命左右取廷玉首廷玉曰馬中箭非敢退也使者馳報光弼令易馬遣之僕固懷恩及其子開府儀同三司瑒戰小却|{
	瑒音暢又雉杏翻}
光弼又命取其首懷恩父子顧見使者提刀馳來更前决戰光弼連颭其旗諸將齊進致死呼聲動天地|{
	呼火故翻}
賊衆大潰|{
	史言河陽之戰真為確鬭非李光弼督諸將致死不足以决勝}
斬首千餘級捕虜五百人溺死者千餘人周摯以數騎遁去擒其大將徐璜玉李秦授其河南節度使安太清走保懷州 |{
	考異曰舊傳斬萬餘級生擒八千餘人擒其大將徐璜玉李秦授周摯等按李秦授上元元年四月乃見擒周摯二年三月為史朝義所殺今從實録實録云擒偽懷州節度使安太凊并男朝俊偽貝州刺史徐璜玉按太清上元元年九月拔懷州始擒之今從舊傳 予按通鑑書擒徐璜玉李秦授盖從舊傳而以舊傳擒周摯為誤實録所云擒安太清朝俊通鑑皆不取而考異謂之今從實録此四字不可曉若參取二書又考本末則此時只當書擒徐璜玉如李秦授亦未當書擒}
思明不知摯敗尚攻南城光弼驅俘囚臨河示之乃遁丁巳以李日越為右金吾大將軍 卭簡嘉眉瀘戎等州蠻反|{
	簡州漢牛鞞廣都之地後魏於牛鞞置陽安縣及武康郡隋廢郡以縣屬蜀郡仁夀初分置簡州餘注見前卭渠容翻瀘龍都翻}
十一月甲子以殿中監董秦為陜西神策兩軍兵馬使|{
	此殿中監所謂帶職以寄禄也}
賜姓李名忠臣 康楚元等衆至萬餘人商州刺史充荆襄等道租庸使韋倫發兵討之駐於鄧之境招諭降者厚撫之|{
	降戶江翻}
俟其稍怠進軍擊之生擒楚元其衆遂潰得其所掠租庸二百萬緡荆襄皆平倫見素之從弟也|{
	韋見素相天寶以迨至德從才用翻}
發安西北庭兵屯陜以備史思明 第五琦作乾元錢重輪錢與開元錢三品並行|{
	重直龍翻}
民争盗鑄貨輕物重糓價騰踊餓殍相望上言者皆歸咎於琦庚午貶琦忠州長史|{
	忠州漢臨江墊江枳縣地梁置臨江郡後周置臨山至隋廢郡及州以縣屬巴東郡唐初分置忠州地邊巴徼心懷忠信為名殍皮表翻上時兩翻長知兩翻忠州南賓郡}
御史大夫賀蘭進明貶溱州員外司馬坐琦黨也 十二月甲午呂諲領度支使乙巳韋倫送康楚元詣闕斬之 史思明遣其將李

歸仁將鐵騎五千寇陜州神策兵馬使衛伯玉以數百騎擊破之於礓子阪得馬六百匹歸仁走以伯玉為鎮西四鎮行營節度使李忠臣與歸仁等戰於永寜莎柵之間屢破之|{
	礓子阪在河南永寧縣西永寧漢宜陽縣西界之地後周置同軌郡及熊耳縣崤縣隋廢郡及崤縣義寧元年改為永寧縣礓居良翻宋白曰永寧縣本漢澠池縣之西境後魏大統十年於今縣東黄蘆城置北宜陽縣廢帝二年改為熊耳後周移於劉塢隋開皇三年移於同軌城義寧三年移於永固因苻堅舊城置縣以永寧為名武德三年移理同軌貞觀十四年移理莎柵十七年又移理鹿橋}


上元元年|{
	是年閏四月始改元}
春正月辛巳以李光弼為太尉兼中書令餘如故 丙戍以于闐王勝之弟曜同四鎮節度副使權知本國事|{
	于闐王與四鎮節度使皆在行營故令其弟與節度副使同權國事}
党項等羌吞噬邊鄙將逼京畿乃分邠寜等州節度

為鄜坊丹延節度亦謂之渭北節度|{
	邠寧節度領州九分四州為渭北節度鄜音膚}
以邠州刺史桑如珪領邠寧鄜州刺史杜冕領鄜坊節度副使分道招討戊子以郭子儀領兩道節度使|{
	兩道邠寧鄜坊也}
留京師假其威名以鎮之 上祀九宫貴神二月李光弼攻懷州史思明救之癸卯光弼逆戰於

沁水之上破之斬首三千餘級|{
	沁七鴆翻}
忠州長史第五琦既行或告琦受人金二百兩遣御史劉期光追按之琦曰琦備位宰相二百兩金不可手挈若付受有憑請準律科罪期光即奏琦已服罪|{
	史言劉期光不能審克閲實而妄奏}
庚戍琦坐除名長流夷州|{
	宋白曰夷州之地歷代恃險不聞臣附隋大業七年始招慰置綏陽縣唐武德四年置夷州舊志京師南四千三百八十七里至洛陽三千八百八十里}
三月甲申改蒲州為河中府 庚寅李光弼破安太清於懷州城下夏四月壬辰破史思明于河陽西渚斬首千五百餘級 襄州將張維瑾曹玠殺節度使史翽據州反|{
	翽呼外翻}
制以隴州刺史韋倫為山南東道節度使時李輔國用事節度使皆出其門倫既朝廷所除又不謁輔國尋改秦州防禦使己未以陜西節度使來瑱為山南東道節度使|{
	至德二載廢南陽節度使升襄陽防禦使為山南東道節度使領襄鄧隨唐安均房金商九州治襄州}
瑱至襄州張維瑾等皆降|{
	降戶江翻}
閏月丁卯加河東節度使王思禮為司空自武德以來思禮始不為宰相而拜三公 甲戍徙趙王係為越王 己卯赦天下改元|{
	改元上元}
追諡太公望為武成王選歷代名將為亞聖十哲|{
	開元十九年始置太公尚父廟以留侯張良配中春中秋上戊祭之牲樂之制如文宣王出師命將發日引辭于廟仍以古名將十人為十哲配享是年尊為武成王以歷代良將為十哲像坐侍秦武安侯白起漢淮隂侯韓信蜀丞相諸葛亮唐尚書右僕射衛國公李靖司空英國公李勣列于左漢太子少傅張良齊大司馬田穰苴吳將軍孫武魏西河守吳起燕昌國君樂毅列於右}
其中祀下祀并雜祀一切並停|{
	旱故也唐六典昊天上帝五方帝皇地祇神州宗廟為大祀日月星辰社稷先代帝王岳鎮海瀆帝社先蠶孔宣父齊太公諸太子廟為中祀司中司命風師雨師衆星山林川澤五龍祠等及州縣杜稷釋奠為小祀雜祀盖小鬼之神若漢志所謂杜將軍寶雞之類}
是日史思明入東京 |{
	考異曰按去年九月思明已入東京實錄至此復云爾者蓋當時城空李光弼在河陽思明還屯白馬寺不入宫闕今始移軍入城耳}
五月丙午以太子太傳苖晉卿行侍中晉卿練達吏事而謹身固位時人比之胡廣 宦者馬上言受賂為人求官於兵部侍郎同中書門下三品呂諲諲為之補官|{
	為于偽翻}
事覺上言杖死壬子諲罷為太子賓客 癸丑以京兆尹南華劉晏為戶部侍郎充度支鑄錢鹽鐵等使|{
	南華本漢離狐縣歷代不更名天寶元年更名南華縣屬曹州鹽鐵使乾元元年以命第五琦會要開元二十五年監察御史羅文信充諸道鑄錢使其後楊慎矜楊國忠相繼為之}
晏善治財利故用之|{
	治直之翻}
六月甲子桂州經略使邢濟奏破西原蠻二十萬衆斬其帥黄乾曜等|{
	西原蠻居廣容之南邕桂之西有甯氏相承為豪又有黄氏居黄橙洞其屬也其地西接南詔天寶初黄氏彊與韋氏周氏儂氏相脣齒為寇害據十餘州又逐韋周于海濱緜地數千里帥所類翻}
三品錢行寖久|{
	開元錢與乾元當十錢重輪錢為三品}
屬歲荒米斗至七千錢人相食京兆尹鄭叔清捕私鑄錢者數月間榜死者八百餘人不能禁|{
	屬之欲翻榜音彭}
乃勑京畿開元錢與乾元小錢皆當十其重輪錢當三十諸州更俟進止是時史思明亦鑄順天得一錢|{
	史思明鑄得一元寶錢徑一寸四分既而惡得一非長祚之兆改其文曰順天元寶}
一當開元錢百賊中物價尤貴 甲申興王佋薨佋張后長子也幼曰定王侗|{
	佋音韶侗音通又音同}
張后以故數欲危太子|{
	數所角翻}
太子常以恭遜取容會佋薨侗尚幼太子位遂定乙酉鳳翔節度使崔光遠破党項於普潤|{
	普潤縣屬鳳翔府漢}


|{
	杜陽縣之地隋作仁壽宫大業初置普潤縣宋白曰普潤縣本漢安定鶉狐二縣之地在漢又為漆縣隋大業元年於細川谷置普潤縣蓋以杜漆岐三水灌溉民田民獲濟利以為縣名麟遊縣則漢杜陽之地有隋仁壽宫}
平盧兵馬使田神功奏破史思明之兵于鄭州上皇愛興慶宫自蜀歸即居之|{
	事見上卷至德二載}
上時自夾城往起居|{
	夾城開元二十年築}
上皇亦間至大明宫|{
	間古莧翻}
左龍武大將軍陳玄禮内侍監高力士久侍衛上皇上又命玉真公主如仙媛 |{
	考異曰常侍言旨作九仙媛唐歷作九公主女媛今從新舊傳盖舊宫人也}
内侍王承恩魏悦及梨園弟子常娯侍左右上皇多御長慶樓|{
	長慶樓南臨大道上皇每御之裴徊觀覽}
父老過者往往瞻拜呼萬歲上皇常於樓下置酒食賜之又嘗召將軍郭英乂等上樓賜宴有劒南奏事官過樓下拜舞|{
	諸道遣官入京師奏事者謂之奏事官}
上皇命玉真公主如仙媛為之作主人|{
	為于偽翻}
李輔國素微賤雖暴貴用事上皇左右皆輕之輔國意恨且欲立奇功以固其寵乃言於上曰上皇居興慶宫日與外人交通陳玄禮高力士謀不利於陛下今六軍將士盡靈武勲臣皆反仄不安臣曉諭不能解不敢不以聞|{
	李輔國此言是臨肅宗以兵也}
上泣曰聖皇慈仁豈容有此|{
	帝上上皇尊號曰聖皇天帝}
對曰上皇固無此意其如羣小何陛下為天下主當為社稷大計消亂於未萌豈得徇匹夫之孝且興慶宫與閻閭相参垣墉淺露非至尊所宜居大内深嚴奉迎居之與彼何殊又得杜絶小人熒惑聖聽如此上皇享萬歲之安陛下有三朝之樂|{
	記曰文王之為世子也朝於王季日三朝直遙翻}
庸何傷乎上不聽興慶宫先有馬三百匹輔國矯勑取之|{
	矯勑猶言矯詔也}
纔留十匹上皇謂高力士曰吾兒為輔國所惑不得終孝矣輔國又令六軍將士號哭叩頭請迎上皇居西内|{
	唐以大明宫為東内太極宫為西内興慶宫為南内號戶刀翻}
上泣不應輔國懼會上不豫秋七月丁未輔國矯稱上語迎上皇遊西内至睿武門輔國將射生五百騎露刃遮道奏曰皇帝以興慶宫湫隘|{
	湫下也隘小也狹也陸德明音義湫子小翻徐音秋}
迎上皇遷居大内上皇驚幾墜|{
	幾居依翻}
高力士曰李輔國何得無禮叱令下馬輔國不得已而下力士因宣上皇誥曰諸將士各好在|{
	以將士露刃遮道震驚上皇殊無善狀令其好在好在猶今人言好生言不得以兵干乘輿也}
將士皆納刃再拜萬歲力士又叱輔國與己共執上皇馬鞚侍衛如西内居甘露殿|{
	西内以兩儀殿為内朝兩儀殿北有甘露門甘露門内為甘露殿如往也}
輔國帥衆而退|{
	帥讀曰率下同}
所留侍衛兵纔尪老數十人|{
	尪烏光翻}
陳玄禮高力士及舊宫人皆不得留左右上皇曰興慶宫吾之王地|{
	王于况翻事見二百九卷睿宗景雲元年}
吾數以讓皇帝皇帝不受今日之徙亦吾志也是日輔國與六軍大將素服見上請罪|{
	北門六軍也數所角翻見賢遍翻}
上又廹於諸將乃勞之曰南宫西内亦復何殊|{
	南宫即謂興慶宫取語便順或言南宫或言南内勞利到翻復扶又翻}
卿等恐小人熒惑防微杜漸以安社稷何所懼也刑部尚書顔真卿首率百寮上表請問上皇起居輔國惡之奏貶蓬州長史|{
	梁以漢宕渠縣界置安國縣後周置蓬州隋廢州以縣屬清化郡唐復分置蓬州惡烏路翻宋白曰因蓬山為名至京師二千三百六十里東都二千五百八十二里}
癸丑勑天下重陵錢皆當三十如畿内|{
	重稜錢即重輪錢重直龍翻}
丙辰高力士流巫州王承恩流播州魏悦流溱州陳玄禮勒致仕置如仙媛于歸州|{
	貞觀八年分辰州龍標縣置巫州京師南三千一百五十八里至東都三千八百三十三里播州秦夜郎郡之南境隋牂柯郡之牂柯縣貞觀九年置郎州十一年置播州京師南四千四百五十里至東都四千九百六十里貞觀十六年開山洞置溱州至京師三千四百八十里東都四千二百里歸州漢秭歸縣地後周置秭歸郡隋廢郡以縣屬巴東郡唐武德二年分秭歸巴東二縣置歸州京師南二千三百六十八里至東都一千八百四十三里}
玉真公主出居玉真觀|{
	玉真觀睿宗為主所起}
上更選後宫百餘人置西内備灑掃|{
	灑所賣翻掃素報翻又皆如字}
令萬安咸宜二公主視服膳|{
	萬安咸宜二公主皆上皇女}
四方所獻珍異先薦上皇然上皇日以不懌因不茹葷辟穀寖以成疾上初猶往問安既而上亦有疾但遣人起居其後上稍悔寤惡輔國欲誅之|{
	惡烏路翻}
畏其握兵竟猶豫不能决 初哥舒翰破吐蕃於臨洮西關磨環川於其地置神策軍|{
	會要天寶十三載哥舒翰以前年收九曲請以其地置洮陽郡郡内置神策軍去臨洮郡二百里}
及安祿山反軍使成如璆|{
	璆與球同}
遣其將衛伯玉將千人赴難|{
	難乃旦翻}
既而軍地淪入吐蕃伯玉留屯於陜累官至右羽林大將軍八月庚午以伯玉為神策軍節度使|{
	為神策軍彊盛張本}
丁亥贈謚興王佋曰恭懿太子 九月甲午置南都於荆州以荆州為江陵府仍置永平軍團練兵三千人以扼吳蜀之衝從節度使呂諲之請也 或上言天下未平不宜置郭子儀於散地|{
	散昔亶翻}
乙未命子儀出鎮邠州党項遁去|{
	畏子儀也}
戊申制子儀統諸道兵自朔方直取范陽還定河北發射生英武等禁軍|{
	射生號英武軍見上卷至德二載十二月}
及朔方鄜坊邠寜涇原諸道蕃漢兵共七萬人皆受子儀節度|{
	鄜音夫}
制下旬日復為魚朝恩所沮事竟不行|{
	使郭子儀杲緫兵向范陽則史思明有内顧之憂李光弼成夾攻之勢必無邙山之敗矣郭李成功則又必無樹置河北諸帥之禍矣復扶又翻}
冬十月丙子置青沂等五州節度使|{
	詳考通鑑所書乾元二年四月}


|{
	甲辰以尚衡為青密節度使上元二年四月乙亥青密節度使尚衡破史朝義兵如此則是年尚衡尚鎮青密安得又置青沂等州節度使邪新書方鎮表上元二年置淄沂節度使領淄沂滄德棣五州侯希逸自平盧引兵保青州授青密節度使遂廢淄沂節度并所管五州號淄青平盧節度通鑑書侯希逸為平盧淄青節度在寶應元年五月蓋新表與通鑑各以所見書為據故參錯不同如此}
十一月壬辰涇州破党項 御史中丞李銑宋州刺史劉展皆領淮西節度副使銑貪暴不法展剛彊自用故為其上者多惡之|{
	惡烏路翻}
節度使王仲昇先奏銑罪而誅之時有謡言曰手執金刀起東方仲昇使監軍使内左常侍邢延恩入奏|{
	唐中人出監方鎮軍品秩高者為監軍使其下為監軍監古衘翻}
展倔彊不受命姓名應謡䜟|{
	此句當屬上句倔其勿翻彊其兩翻䜟謂金刀之謡應劉姓也}
請除之延恩因說上曰展與李銑一體之人今銑誅展不自安苟不去之恐其為亂然展方握強兵宜以計去之|{
	說式芮翻去羌呂翻}
請除展江淮都統代李峘|{
	峘戶登翻}
俟其釋兵赴鎮中道執之此一夫力耳上從之以展為都統淮南東江南西浙西三道節度使 |{
	考異曰沈既濟劉展亂紀云淮南東道浙江西道凡二十三州置都統節度下云以展為都統江南淮南節度使下又云三道皆發吏申圖籍按舊李峘傳峘都統淮南江南江西節度展既代峘其所統亦三道耳淮南者東道楊楚滁和舒廬濠夀八州也江南者昇潤常蘇湖杭睦七州也江西者洪䖍江吉袁信撫七州也凡二十二州亂紀誤以二為三又脫江南西道字耳}
密勑舊都統李峘及淮南東道節度使鄧景山圖之|{
	李峘為浙東節度兼淮南見上卷元年按唐會要乾元元年十二月李峘除都統淮南江東江西節度宣慰觀察處置等使都統之名起於此通鑑但書以浙東兼淮東與會要少異}
延恩以制書授展展疑之曰展自陳留參軍數年至刺史可謂暴貴矣江淮租賦所出今之重任展無勲勞又非親賢一旦恩命寵擢如此得非有讒人間之乎|{
	間古莧翻}
因泣下延恩懼曰公素有才望主上以江淮為憂故不次用公公反以為疑何哉展曰事苟不欺印節可先得乎延恩曰可乃馳詣廣陵與峘謀解峘印節以授展展得印節乃上表謝恩牒追江淮親舊置之心膂三道官屬遣使迎賀申圖籍相望於道展悉舉宋州兵七千趣廣陵|{
	趣七喻翻}
延恩知展已得其情還奔廣陵|{
	書云作偽心勞日拙邢延恩之謂矣}
與李峘鄧景山兵拒之移檄州縣言展反展亦移檄言峘反州縣莫知所從峘引兵度江與副使潤州刺史韋儇|{
	儇許緣翻}
浙西節度使侯令儀屯京口鄧景山將萬人屯徐城|{
	徐城縣屬泗州漢徐縣地隋置徐城縣於大徐城開元三十五年移就臨淮縣}
展素有威名御軍嚴整江淮人望風畏之展倍道先期至使人問景山曰吾奉詔書赴鎮此何兵也景山不應展使人呼於陳前曰|{
	呼火故翻}
汝曹皆吾民也勿干吾旗鼔使其將孫待封張灋雷擊之景山衆潰與延恩奔夀州展引兵入廣陵遣其將屈突孝標將兵三千徇濠楚王暅將兵四千略淮西|{
	暅古鄧翻}
李峘闢北固為兵場|{
	北固山在京口梁武帝所登即其地}
挿木以塞江口展軍於白沙設疑兵於瓜洲|{
	今揚州江都縣南三十里有瓜洲鎮正對京口北固山塞昔則翻}
多張火鼔|{
	張火及鼓以為疑兵}
若將趣北固者如是累日峘悉銳兵守京口以待之展乃自上流濟襲下蜀|{
	此自白沙濟江也昇州東北九十里至句容縣有下蜀戍在句容縣北近江津}
峘軍聞之自潰峘奔宣城|{
	宣城漢宛陵縣地晉置宣城縣隋平陳廢郡改宛陵為宣城縣帶宣州李峘奔宣城就鄭炅也}
甲午展陷潤州|{
	考異曰十一月壬子淮南節度奏展反鄧景山李峘戰敗八日展陷潤州十日陷昇州按八日甲午十日丙申壬子二十六日乃奏到日也唐歷壬子淮南奏宋州刺史劉展赴鎮揚州長史淮南節度鄧景山都統尚書李峘承詔拒之兵敗奔于夀州乙未劉展陷揚州丙申陷潤州丁酉陷昇州壬子在前盖因實録也今從劉展亂紀及新書本紀}
昇州軍士萬五千人謀應展攻金陵城|{
	昇州治金陵}
不克而遁侯令儀懼以後事授兵馬使姜昌羣棄城走昌羣遣其將宗犀詣展降丙申展陷昇州以宗犀為潤州司馬丹楊軍使|{
	乾元二年置丹楊軍于潤州}
使昌羣領昇州以從子伯瑛佐之|{
	從才用翻}
李光弼攻懷州百餘日乃拔之生擒安太清 |{
	考異曰舊傳云擒安太清周摯楊希文等送於闕下按周摯於時不在懷州城中明年為史朝義所殺非光弼所擒也}
史思明遣其將田承嗣將兵五千徇淮西王同芝將兵三千人徇陳許敬江將二千人徇兖鄆薛鄂將五千人徇曹州|{
	新書江作缸鄂作萼}
十二月丙子党項寇美原同官大掠而去|{
	後魏景明元年分漢富平縣置土門縣屬新平郡因土門山為名隋廢土門縣入華原咸亨二年分京兆之富平華原及同州之蒲城以故土門縣置美原縣同官本漢銅官之地後因謂之銅官川後魏真君七年置銅官縣屬北地郡隋為銅官至唐二縣並屬京兆宋白曰同官縣漢祋祤地晉為頻陽地苻堅於祋祤城東北銅官川置銅官護軍後魏真君七年罷軍為縣後周除金作此同字}
賊帥郭愔等引諸羌胡敗秦隴防禦使韋倫|{
	帥所類翻敗補賣翻}
殺監軍使 兖鄆節度使能元皓|{
	方鎮表乾元二年升鄆齊兖三州都防禦使為節度使是年以齊州隸青密而兖鄆增領徐州能奴代翻}
擊史思明兵破之李峘之去潤州也副使李藏用謂峘曰處人尊位食

人重祿臨難而逃之非忠也|{
	處昌呂翻難乃旦翻}
以數十州之兵食三江五湖之險固|{
	韋昭曰三江謂吳淞江錢唐江浦陽江也吳地記曰松江東北行七十里得三江口東北入海為婁江東南入江為東江并松江為三江五湖注已見晉安帝紀}
不發一矢而棄之非勇也失忠與勇何以事君藏用請收餘兵竭力以拒之峘乃悉以後事授藏用藏用收散卒得七百人東至蘇州募壯士得二千人立柵以拒劉展展遣其將傅子昂宗犀攻宣州宣歙節度使鄭炅之棄城走|{
	宣歙節度使領宣歙饒三州歙書涉翻}
李峘奔洪州李藏用與展將張景超孫待封戰於郁墅兵敗奔杭州景超遂據蘇州待封進陷湖州|{
	湖州本漢烏程縣地吳置吳興郡隋平陳廢郡置湖州大業初廢州以縣屬吳郡唐武德四年復置湖州}
展以其將許嶧為潤州刺史|{
	嶧音亦}
李可封為常州刺史楊持璧蘇州刺史待封領湖州事景超進逼杭州藏用使其將温晁屯餘杭|{
	餘杭漢縣時屬杭州在州西四十五里晁直遥翻}
展以李晃為泗州刺史宗犀為宣州刺史傳子昂屯南陵|{
	南陵漢春穀縣地梁置南陵縣及南陵郡隋廢郡以縣屬宣州舊治赭圻城長安四年移治青陽城}
將下江州徇江西|{
	江西謂江南西道}
於是屈突孝標陷濠楚州|{
	屈居勿翻}
王暅陷舒和滁廬等州所向無不摧靡聚兵萬人騎三千横行江淮間壽州刺史崔昭發兵拒之由是暅不得西止屯廬州初上命平盧兵馬使田神功將所部精兵三千屯任城|{
	任音壬}
鄧景山既敗與邢延恩奏乞勑神功救淮南未報景山遣人趣之|{
	趣讀曰促}
且許以淮南金帛子女為賂神功及所部皆喜悉衆南下及彭城勑神功討展|{
	田神功至彭城勑方下}
展聞之始有懼色自廣陵將兵八千拒之選精兵二千度淮擊神功於都梁山展敗走至天長|{
	天寶元年分江都六合高郵置千秋縣七載更名天長屬揚州}
以五百騎據橋拒戰又敗展獨與一騎亡度江神功入廣陵 |{
	考異曰劉展亂紀云二年春神功舉兵東下實錄唐勅神功入揚州在此月今從之}
及楚州|{
	當屬上句蓋先入楚州而後入廣陵}
大掠殺商胡以千數城中地穿掘略徧|{
	穿掘以求人所窖藏者掘其月翻}
是歲吐蕃陷廓州

資治通鑑卷二百二十一
