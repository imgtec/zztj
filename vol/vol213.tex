






























































資治通鑑卷二百十三  宋 司馬光 撰

胡三省 音注

唐紀二十九【起柔兆攝提格盡昭傷作噩凡八年}


玄宗至道大聖大明孝皇帝中之上

開元十四年春正月癸未更立契丹松漠王李邵固為廣化王奚饒樂王李魯蘇為奉誠王【契欺訖翻樂音洛}
以上從甥陳氏為東華公主妻邵固【從才用翻妻七細翻下同考異曰東華出降實録在三月壬子於此終言之}
以成安公主之女韋氏為東光公主【成安公主中宗之女下嫁韋捷}
妻魯蘇 張說奏今之五禮貞觀顯慶兩曾修纂【說讀曰悦觀古玩翻}
前後頗有不同其中或未折衷【衷竹仲翻宋白曰折斷也中當正也若折斷其物與度相中當也}
望與學士等討論古今删改施行制從之 邕州封陵獠梁大海等據賓横州反【封陵本山峒唐世以漸開拓乾无後始置為縣賓州漢領方縣地屬鬱林郡梁置領方郡隋廢郡為縣屬鬰州唐初屬南方州貞觀五年分置賓州横州漢廣鬰高梁縣地江左置寜浦郡隋廢郡為縣屬鬰州唐初分置簡州貞觀八年曰横州}
二月巳酉遣内侍楊思勗發兵討之 【考異曰舊紀作庚戌朔今從實録}
上召河南尹崔隱甫欲用之中書令張說薄其無文奏擬金吾大將軍前殿中監崔日知素與說善說薦為御史大夫上不從丙辰以日知為左羽林大將軍丁巳以隱甫為御史大夫隱甫由是與說有隙說有才智而好賄百官白事有不合者好面折之至於叱罵惡御史中丞宇文融之為人【好呼到翻折之舌翻惡烏路翻}
且患其權重【宇文融既居風憲之地又貳戶部故患其權重}
融所建白多抑之中書舍人張九齡言於說曰宇文融承恩用事辨給多權數不可不備說曰鼠輩何能為夏四月壬子隱甫融及御史中丞李林甫共奏彈說引術士占星徇私僭侈受納賄賂【彈徒丹翻賄呼罪翻}
敕源乾曜及刑部尚書韋抗大理少卿明珪與隱甫等同於御史臺鞫之林甫叔良之曾孫【長平王叔良高祖從父弟}
抗安石之從父兄子也【韋安石歷事武后中宗貶死於開元之初從才用翻}
丁巳以戶部侍郎李元紘為中書侍郎同平章事元紘以清儉著故上用為相 源乾曜等鞫張說事頗有狀上使高力士視說力士還奏說蓬首垢面席藁食以瓦器惶懼待罪上意憐之力士因言說有功於國上以為然庚申但罷說中書令餘如故【說讀曰悦}
丁卯太子太傳岐王範薨贈諡惠文太子上為之徹膳累旬【為于偽翻}
百官上表固請【上時掌翻}
然後復常 丁亥太原尹張孝嵩奏有李子嶠者自稱皇子云生於潞州母曰趙妃上命杖殺之 辛丑於定恒莫易滄五州置軍以備突厥【定州置北平軍恒州置恒陽軍莫州置唐興軍易州置高陽軍滄州置横海軍恒戶登翻}
上欲以武惠妃為皇后或上言武氏乃不戴天之讐豈可以為國母人間盛言張說欲取立后之功更圖入相之計【上時掌翻相息亮翻}
且太子非惠妃所生惠妃復自有子若登宸極太子必危上乃止【復扶又翻 考異曰唐會要云侍御史潘好禮聞上欲以惠妃為皇后進疏諫曰臣聞禮記曰父母之讐不可共戴天公羊傳曰子不復父讐不子也昔齊襄公復九世之讐丁蘭報木母之怨陛下豈得欲以武氏為國母當何以見天下之人乎不亦取笑於天下乎又惠妃再從叔三思再從父延秀等並干紀亂常遞窺神器豺狼同穴梟獍共林且匹夫匹婦欲結髪為夫妻者尚相揀擇况陛下是累聖之貴天子之尊乎伏願詳察古今鑒戒成敗慎擇華族之女必在禮義之家稱神祇之心允億兆之望又見人間盛言尚書右丞相張說自被停知政事之後每諂附惠妃欲取立后之功更圖入相之計伏願杜之於將漸不可悔之於巳成且太子本非惠妃所生惠妃復自有子若惠妃一登宸極則儲位實恐不安古人所以諫其漸者良為是也昔商山四皓雖不食漢庭之禄尚能輔翊太子况臣愚昧職忝憲府蘇冕駮曰此表非潘好禮所作且好禮先天元年為侍御史開元十二年為温州刺史致仕表是十四年獻而云職忝憲府若題年恐錯則武惠妃先天元年始年十四王皇后有寵未衰張說又未為右丞相竟未知此表是誰獻之今除其名}
然宫中禮秩一如皇后 五月癸卯戶部奏今歲戶七百六萬九千五百六十五口四千一百四十一萬九千七百一十二 秋七月河南北大水溺死者以千計【溺奴狄翻}
八月丙午朔魏州言河溢九月巳丑以安西副大都護磧西節度使杜暹同平章事【磧七迹翻暹息亷翻}
自王孝傑克復四鎮【復四鎮見二百五卷武后長夀元年}
復於龜兹置安西都護府【復扶又翻龜兹音丘慈}
以唐兵三萬戍之百姓苦其役為都護者惟田揚名郭元振張嵩及暹皆有善政為人所稱 冬十月庚申上幸汝州廣成湯 【考異曰令狐峘代宗實録云上以開元十四年十月十三日生時玄宗幸汝州之温湯有望氣者云宫中有天子氣玄宗即日還宫是夜代宗降誕按玄宗實録此月十六日庚申始幸温湯己巳乃還宫與代宗實録不同舊紀云十二月十三日生舊后妃傳章敬皇后吳氏坐父事沒入掖庭開元二十三年玄宗幸忠王邸見王服御蕭然傍無勝侍命將軍高力士選掖庭宫人以賜之而吴后在籍中明年生代宗皇帝十八年薨按代宗此年生而云二十三年以吴后賜忠王十八年薨蓋誤以十三年為二十三年也以柳氏舊聞肅宗在東宫為李林甫所構勢幾危者數矣無何鬚鬂斑白嘗早朝上見之愀然曰汝歸第吾當幸汝及上至顧見宫庭殿宇皆不灑掃而樂器塵埃左右使令無有妓女上為之動色使力士詔掖庭閱視得三人乃以賜太子而章敬吴皇后在選中生代宗按開元二十三年李林甫初為相二十五年廢太子瑛二十六年乃立肅宗為太子天寶五年李林甫始構韋堅之獄舊聞所記事皆虚誕年月不合新書后妃傳全取之今皆不取 按漢廣成苑在唐汝州梁縣界其地有湯泉}
己酉還宫 十二月丁巳上幸夀安獵於方秀川壬戌還宫 楊思勗討反獠【獠魯皓翻}
生禽梁大海等三千餘人斬首二萬級而還【還從宣翻又如字}
是歲黑水靺鞨遣使入見【黑水靺鞨在流鬼國西南女貞即其遣種也靺鞨音末曷見賢遍翻}
上以其國為黑水州仍為置長史以鎮之【長史恐當作長吏仍為于偽翻}
勃海靺鞨王武藝曰黑水入唐道由我境往者請吐屯於突厥【突厥置吐屯以領諸附從之國厥九勿翻}
先告我與我偕行今不告我而請吏於唐是必與唐合謀欲腹背攻我也遣其母弟門藝與其舅任雅將兵擊黑水【將即亮翻下同}
門藝嘗為質子于唐【質音致}
諫曰黑水請吏于唐而我以其故擊之是叛唐也唐大國也昔高麗全盛之時彊兵三十餘萬不遵唐命掃地無遺【掃地無遺言國亡無遺育也事見太宗高宗紀麗力知翻}
况我兵不及高麗什之一二一旦與唐為怨此亡國之執也武藝不從彊遣之【彊其兩翻}
門藝至境上復以書力諫武藝怒遣其從兄大壹夏代之將兵召欲殺之門藝弃衆閒道來奔【復扶又翻夏戶雅翻從才用翻閒古莧翻}
制以為左驍衛將軍武藝遣使上表罪狀門藝請殺之【驍堅堯翻使疏吏翻上時掌翻}
上密遣門藝詣安西留其使者别遣報云已流門藝於嶺南武藝知之上表稱大國當示人以信豈得為此欺誑【誑居况翻}
固請殺門藝上以鴻臚少卿李道邃源復不能督察官屬致有漏泄皆坐左遷【唐九寺皆有少卿二人鴻臚掌四夷之客故以泄漏為罪臚陵如翻少始照翻}
暫遣門藝詣嶺南以報之臣光曰王者所以服四夷威信而已門藝以忠獲罪自歸天子天子當察其枉直賞門藝而罰武藝為政之體也縱不能討猶當正以門藝之無辠告之今明皇威不能服武藝恩不能庇門藝顧效小人為欺誑之語以取困於小國乃辠鴻臚之漏洩不亦可羞哉

杜暹為安西都護突騎施交河公主遣牙官以馬千匹詣安西互市使者宣公主教【暹息廉翻騎奇寄翻使疏吏翻下同}
暹怒曰阿史那女【交河公主阿史那懷道之女}
何得宣教於我杖其使者留不遣馬經雪死盡突騎施可汗蘇禄大怒發兵寇四鎮會暹入朝【使疏吏翻朝直遥翻下同}
趙頤貞代為安西都護嬰城自守四鎮人畜儲積皆為蘇禄所掠【畜許救翻}
安西僅存既而蘇禄聞暹入相稍引退【相息亮翻}
尋遣使入貢

十五年春正月辛丑凉州都督王君㚟破吐蕃於青海之西【㚟丑略翻吐從暾入聲}
初吐蕃自恃其彊致書用敵國禮【事見二百十一卷二年}
辭指悖慢【悖蒲内翻又蒲沒翻}
上意常怒之返自東封張說言於上曰吐蕃無禮誠宜誅夷但連兵十餘年甘凉河鄯不勝其弊【鄯時戰翻又音善勝音升}
雖師屢捷所得不償所亡聞其悔過求和願聽其欵服以紓邊人上曰俟吾與王君㚟議之說退謂源乾曜曰君㚟勇而無謀常思僥幸【僥堅堯翻}
若二國和親何以為功吾言必不用矣及君㚟入朝果請深入討之去冬吐蕃大將悉諾邏寇大斗谷【將即亮翻邏郎仇翻}
進攻甘州焚掠而去君㚟度其兵疲勒兵躡其後【度徒洛翻躡尼輒翻 考異曰吐蕃傳云君㚟畏其鋒不敢出今從君㚟傳}
會大雪虜凍死者甚衆自積石軍西歸【廓州逹化縣西有積石軍本靜邉鎮儀鳳二年為軍東有黄沙戍}
君㚟先遣人間道入虜境燒道旁草悉諾邏至大非川欲休士馬而埜草皆盡馬死過半君㚟與秦州都督張景順追之及於青海之西乘氷而度悉諾邏已去破其後軍獲其輜重羊馬萬計而還【間古莧翻重直用翻 考異曰君㚟傳曰十六年冬吐蕃大將悉諾邏帥衆入寇大斗谷又移攻甘州焚燒市里而去君㚟襲其後敗之於青海之西據實録及吐蕃傳入寇在十四年冬此云十六年冬誤也}
君㚟以功遷左羽林大將軍拜其父夀為少府監致仕上由是益事邉功初洛陽人劉宗器上言請塞汜水舊汴口更於熒澤引河入汴【隋開皇四年分滎陽置廣武縣仁夀元年更名熒澤屬鄯州上時掌翻塞悉則翻下同汜音祀汴皮變翻}
擢宗器為左衛率府胄曹【率所律翻}
至是新渠填塞不通貶宗器為循州安懷戌主命將作大匠范安及發河南懷鄭汴滑衛三萬人疏舊渠旬日而畢 御史大夫崔隱甫中丞宇文融恐右丞相張說復用數奏毁之各為朋黨上惡之【復扶又翻數所角翻惡烏路翻}
二月乙巳制說致仕隱甫免官侍母融出為魏州刺史 乙卯制諸州逃戶先經勸農使括定按比後復有逃來者隨到凖白丁例輸當年租庸有征役者先差【使疏吏翻復扶又翻下不復同差初佳翻}
夏五月癸酉上悉以諸子慶王潭等領州牧刺史都督節度大使大都護經略使【使疏吏翻}
實不出外初太宗愛晉王【晉王治是為高宗}
不使出閤豫王亦以武后少子不出閤及自皇嗣為相王始出閤中宗之世譙王失愛謫居外州温王年十七猶居禁中【譙王重福温王重茂}
上即位附苑城為十王宅【朱雀街東第五街有安國寺寺東附苑城為大宅分處十王十王謂慶忠棣鄂榮儀台潁永濟也後盛夀陳豐常凉六王又就封入内宅是為十六宅}
以居皇子宦官押之就夾城參起居自是不復出閤雖開府置官屬及領藩鎮惟侍讀時入授書【歐陽修曰唐王府侍讀無定員}
自餘王府官屬但歲時通名起居其藩鎮官屬亦不通名及諸孫浸多又置百孫院太子亦不居東宫常在乘輿所幸之别院【乘䋲證翻}
上命妃嬪以下宫中育蠶欲使之知女功丁酉夏至賜貴近絲人一綟【杜佑曰唐令綿六兩為屯絲五兩為絇麻三斤為綟未知絲綟輕重何如綟郎計翻}
秋七月戊寅冀州河溢 己卯禮部尚書許文憲公蘇頲薨【頲他鼎翻}
九月丙子吐蕃大將悉諾邏恭禄及燭龍莽布支攻陷瓜州執刺史田元獻及河西節度使王君㚟之父進攻玉門軍【按王君㚟之父夀以少府監致仕居鄉里玉門軍在肅州之西二百里宋白曰肅州西門縣漢罷玉門關屯徙其人於此故曰玉門縣石門周匝山間經二十里衆流北入延興海也}
縱所虜僧使歸凉州謂君㚟曰將軍常以忠勇許國何不一戰君㚟登城西望而泣竟不敢出兵莽布支别攻常樂縣【宋白曰常樂縣屬瓜州即魏之宜禾郡前凉之凉興縣地凉武昭王於三危山東置常樂鎮唐武德五年改置常樂縣}
縣令賈師順帥衆拒守【樂音洛帥讀曰率}
及瓜州陷悉諾邏悉兵會攻之旬餘日吐蕃力盡不能克使人說降之【說式芮翻降戶江翻}
不從吐蕃曰明府既不降宜歛城中財相贈吾當退師順請脱士卒衣悉諾邏知無財乃引去毁瓜州城師順遽開門收器械修守備虜果復遣精騎還覘城中知有備乃去【田元獻不能守瓜州而賈師順能守常樂固圉固存乎其人也復扶又翻覘丑亷翻又丑艶翻}
師順岐州人也 初突厥默啜之彊也迫奪鐵勒之地故囘紇契苾思結渾四部度磧徙居甘凉之間以避之【啜叱劣翻紇下涘翻契欺訖翻苾毗必翻}
王君㚟微時往來四部為其所輕及為河西節度使以法繩之四部耻怨密遣使詣東都自訴君㚟遽發驛奏四部難制潜有叛計上遣中使往察之【使疏吏翻}
諸部竟不得直於是瀚海大都督囘紇承宗流瀼州【瀼如羊翻杜佑曰而章翻}
渾大德流吉州賀蘭都督契苾承明流藤州【藤州漢蒼梧猛陵縣地晉置永平郡隋置藤州}
盧山都督思結歸國流瓊州以囘紇伏帝難為瀚海大都督己卯貶右散騎常侍李令問為撫州别駕【舊志撫州京師東南三千三百一十二里}
坐其子與承宗交游故也 丙戌突厥毗伽可汗遣其大臣梅録啜入貢吐蕃之寇瓜州也遺毗伽書欲與之俱入寇【遺子季翻}
毗伽并獻其書上嘉之聽於西受降城為互市【降戶江翻}
每歲齎縑帛數十萬匹就市戎馬以助軍旅且為監牧之種【種章勇翻}
由是國馬益壯焉 閏月庚子吐蕃贊普與突騎施蘇禄圍安西城安西副大都護趙頤貞擊破之 囘紇承宗族子瀚海司馬護輸糾合黨衆為承宗報仇會吐蕃遣使間道詣突厥王君㚟帥精騎邀之於肅州【宋白曰隋仁夀元年分甘州福禄縣置肅州東南置甘州赤柳間二百里西南至瓜州界守樂烽三百四十里}
還至甘州南鞏筆驛【為于偽翻間古莧翻甘州張掖縣西南有鞏筆驛}
護輸伏兵突起奪君㚟旌節先殺其判官宋貞剖其心曰始謀者汝也君㚟帥左右數十人力戰【帥讀曰率}
自朝至晡左右盡死護輸殺君㚟載其尸奔吐蕃凉州兵追及之護輸棄尸而走 【考異曰舊傳云囘紇既殺君㚟上命郭知運討逐按知運九年已卒君㚟代鎮凉州舊傳誤也}
庚申車駕發東都冬己卯至西京【冬字下逸十月二字}
辛巳以左金吾衛大將軍信安王禕為朔方節度等副大使禕恪之孫也【吴王恪太宗之子禕吁韋翻}
以朔方節度使蕭嵩為河西節度等副大使時王君㚟新敗河隴震駭嵩引刑部員外郎裴寛為判官與君㚟判官牛仙客俱掌軍政人心浸安寛漼之從弟也【漼取猥翻從才用翻}
仙客本鶉觚小吏【鶉觚縣前漢屬北地郡後漢晉屬安定郡後魏置趙平郡後周廢郡以縣屬涇州劉昫曰節度使置判官二人末九品秩鶉如倫翻觚攻于翻}
以才幹軍功累遷至河西節度判官為君㚟腹心嵩又奏以建康軍使河北張守珪為瓜州刺史【甘州西北百九十里祁連山有建康軍河北縣屬陜州}
帥餘衆築故城板榦裁立【詩云縮板以載縮板兩旁内土其中而築之幹亦板也孔安國曰旁曰幹帥讀曰率}
吐蕃猝至城中相顧失色莫有鬬志守珪曰彼衆我寡又瘡痍之餘不可以矢刃相持當以奇計取勝乃於城上置酒作樂虜疑其有備不敢攻而退守珪縱兵擊之虜敗走守珪乃修復城市收合流散皆復舊業朝廷嘉其功以瓜州為都督府以守珪為都督悉諾邏威名甚盛蕭嵩縱反閒於吐蕃【閒古莧翻}
云與中國通謀贊普召而誅之吐蕃由是少衰【少詩沼翻}
十二月戊寅制以吐蕃為邊患令隴右道及諸軍團兵五萬六千人河西道及諸軍團兵四萬人【府兵廢行一切之法團結民兵謂之團兵}
又徵關中兵萬人集臨洮朔方兵萬人集會州防秋至冬初無寇而罷伺虜入寇【洮士力翻伺相吏翻}
互出兵腹背擊之 乙亥上幸驪山温泉丙戌還宫

十六年春正月壬寅安西副大都護趙頤貞敗吐蕃于曲子城【敗蒲邁翻}
甲寅以魏州刺史宇文融為戶部侍郎兼魏州刺史充河北道宣撫使【宣撫使始此使疏吏翻下同}
乙卯春瀧等州獠陳行範廣州獠馮璘何遊魯反【瀧閭江翻獠魯皓翻 考異曰本紀作馮仁智今從楊思勗傳}
陷四十餘城行範稱帝遊魯稱定國大將軍璘稱南越王欲據嶺表命内侍楊思勗發桂州及嶺北近道兵討之 丙寅以魏州刺史宇文融檢校汴州刺史充河南北溝渠堤堰决九河使【校古孝翻汴皮變翻堰於扇翻使疏吏翻}
融請用禹貢九河故道開稻田并囘易陸運錢官收其利興役不息事多不就 二月壬申以尚書右丞相致仕張說兼集賢殿學士說雖罷政事專文史之任朝廷每有大事上常遣中使訪之【史言張說寵顧不衰尚辰羊翻相息亮翻說讀為悦使疏吏翻}
壬辰改彍騎為左右羽林軍飛騎【彍騎見上卷十三年彍虚郭翻又古郭翻騎奇寄翻}
秋七月吐蕃大將悉末朗寇瓜州【吐從暾入聲將即亮翻}
都督張守珪擊走之乙巳河西節度使蕭嵩隴右節度使張忠亮大破吐蕃於渇波谷【據新書吐蕃傳渇波谷當在青海西 考異曰實録唐歷蕭嵩傳作張志亮今從舊本紀吐蕃傳}
忠亮追之拔其大莫門城【大莫門城在九曲}
擒獲甚衆焚其駱駝橋而還【還從宣翻又如字}
八月乙巳特進張說上開元大衍歷行之【僧一行推大衍數立術以應氣朔及日食以造新歷故曰大衍歷上時掌翻}
辛卯左金吾將軍杜賓客破吐蕃于祁連城下【祁連城在甘州張掖縣祁連山}
時吐蕃復入寇【復扶又翻}
蕭嵩遣賓客將彊弩四千擊之【將即亮翻又音如字}
戰自辰至暮吐蕃大潰獲其大將一人【將即亮翻又音如字}
虜散走投山哭聲四合 冬十月己卯上幸驪山温泉己丑還宫 【考異曰實録十二月丁卯又去幸温泉宫不言其還唐歷丁卯幸温泉丁丑還宫按此月已幸温泉恐重複不取}
十一月癸巳以河西節度副大使蕭嵩為兵部尚書同平章事 十二月丙寅敕長征兵無有還期人情難堪宜分五番歲遣一番還家洗沐五年酬勲五轉 是歲制戶籍三歲一定分為九等 楊思勗討陳行範至瀧州破之擒何遊魯馮璘行範逃於雲際盤遼二洞思勗追捕竟生擒斬之凡斬首六萬思勗為人嚴偏禆白事者不敢仰視故用兵所向有功然性忍酷所得俘虜或生剥面皮或以刀□髪際掣去頭皮蠻夷憚之【剺里之翻掣昌列翻去羌呂翻}


十七年春二月丁卯巂州都督張守素破西南蠻拔昆明及鹽城【昆明縣屬巂州漢定莋縣地後周置定莋鎮武德二年改置昆明縣以其地接昆明故也縣有鹽有鐵築城以衛之故又有鹽城巂音髓}
殺獲萬人 三月瓜州都督張守珪沙州刺史賈師順擊吐蕃大同軍大破之 甲寅朔方節度使信安王禕攻吐蕃石堡城拔之初吐蕃陷石堡城留兵據之侵擾河右上命禕與河西隴右同議攻取諸將咸以為石堡據險而道遠攻之不克將無以自還且宜按兵觀舋【舋許覲翻}
禕不聽引兵深入急攻拔之乃分兵據守要害令虜不得前自是河隴諸軍遊奕拓境千餘里上聞大悦更命石堡城曰振武軍【自鄯州鄯城縣河源軍西行百二十里至白水軍又西南六十里至定戎城又南隔澗七里有石堡城本吐蕃䥫仞城也宋白曰石堡城在龍支縣西四面懸崖數千仭石路盤屈長三四里西至赤嶺三十里更工衡翻}
丙辰國子祭酒楊瑒上言【瑒徒杏翻又音暢上時掌翻}
以為省司奏限天下明經進士及第每年不過百人竊見流外出身每歲二千餘人而明經進士不能居其什一則是服勤道業之士不如胥史之得仕也臣恐儒風浸墜亷耻日衰若以出身人太多則應諸色裁損不應獨抑明經進士也又奏諸司帖試明經不務求述作大指專取難知問以孤經絶句或年月日請自今並帖平文【唐取士之科有進士有明經凡明經先帖文然後口試經問大義十條答時務策三道以文理粗通為上上上中上下中上凡四等為及第凡進士試時務策五道帖一大經經策全通為甲第策通四帖過四以上為乙第通典曰唐制帖經者以所習經掩其兩端其間惟開一行裁紙為帖凡帖三字隨時增損可否不一或得四得五得六者為通}
上甚然之夏四月庚午禘于太廟唐初祫則序昭穆禘則各祀

於其室【昭讀曰佋音時遥翻}
至是太常少卿韋縚等奏如此禘與常饗不異請禘祫皆序昭穆從之縚安石之兄子也【縚上刀翻}
五月壬辰復置十道及京都兩畿按察使【雍同華商岐邠為京畿洛汝為都畿十二年停諸道按察使今復置復扶又翻又如字}
初張說張嘉貞李元紘杜暹相繼為相用事源乾曜以清謹自守常讓事於說等唯諾署名而已元紘暹議事多異同遂有隙更相奏列【唯于癸翻更工衡翻}
上不悦六月甲戌貶黄門侍郎同平章事杜暹荆州長史中書侍郎同平章事李元紘曹州刺史【舊志曹州京師東北一千四百五十三里}
罷乾曜兼侍中止為左丞相【開元初改尚書左右僕射為左右丞相唐初僕射之職無所不統是正丞相也至中宗神龍元年豆盧欽望專為僕射不敢預政事是後專拜僕射者不復知政事雖有丞相之名非復唐初丞相之職矣今源乾曜止為左丞相是止為尚書左僕射不復預政事也}
以戶部侍郎宇文融為黄門侍郎兵部侍郎裴光庭為中書侍郎並同平章事蕭嵩兼中書令遥領河西【遥領河西節度使}
開府王毛仲與龍武將軍葛福順為昏毛仲為上所信任言無不從故北門諸將多附之進退唯其指使吏部侍郎齊澣乘間言於上曰【間古莧翻下離間同}
福順典禁兵【葛福順所典萬騎也故云然}
不宜與毛仲為昏毛仲小人寵過則生姦不早為之所恐成後患上悦曰知卿忠誠朕徐思其宜澣曰君不密則失臣【易大傳之言}
願陛下密之會大理丞麻察坐事左遷興州别駕【舊志興州至京師九百四十八里}
澣素與察善出城餞之因道禁中諫語察性輕險遽奏之上怒召澣責之曰卿疑朕不密而以語麻察詎為密邪且察素無行【語牛倨翻邪音耶行下孟翻}
卿豈不知邪澣頓首謝秋七月丁巳下制澣察交構將相離間君臣【將即亮翻相息亮翻間古莧翻}
澣可高州良德丞察可潯州皇化尉【良德亦漢合浦縣地吳置高凉郡陳分置務德縣後改為良德潯州漢布山阿林之地梁於布山地置桂平郡隋廢郡為縣又於阿林地置皇化縣隋廢入桂平貞觀七年置潯州治桂平復置皇化縣屬焉}
八月癸亥上以生日宴百官於花萼樓下 【考異曰實録云癸亥朔按長歷是月己未朔癸亥五日也顧况歌曰八月五夜佳氣新昭成太后生聖人實録誤也}
左丞相乾曜右丞相說帥百官上表【帥讀曰率上時掌翻}
請以每歲八月五日為千秋節布于天下咸令宴樂【聖節錫宴自此始後改千秋節為天長節德順憲穆不置節名令力丁翻樂音洛}
尋又移社就千秋節【自古以來社用戊日}
庚辰工部尚書張嘉貞薨嘉貞不營家產有勸其市田宅者嘉貞曰吾貴為將相何憂寒餒若其獲罪雖有田宅亦無所用比見朝士廣占良田身沒之日適足為無賴子弟酒色之資【尚張羊翻將即亮翻相息亮翻比毗至翻朝直遥翻}
吾不取也聞者是之 辛巳敕以人間多盗鑄錢始禁私賣銅鉛錫及以銅為器皿其采銅鉛錫者官為市取【為于偽翻}
宇文融性精敏應對辨給以治財賦得幸於上始廣置諸使競為聚歛【治直之翻使疏吏翻歛力瞻翻}
由是百官浸失其職而上心益侈【史言唐玄宗時開利孔自宇文融始}
百姓皆怨苦之為人踈躁多言【躁則到翻}
好自矜伐【好呼到翻}
在相位謂人曰使吾居此數月則海内無事矣【相息亮翻下同}
信安王禕以軍功有寵於上【以平石堡城之功也}
融疾之禕入朝【朝直遥翻}
融使御史李寅彈之【彈徒丹翻}
泄於所親禕聞之先以白上明日寅奏果入上怒九月壬子融坐貶汝州刺史 【考異曰舊傳曰殿中侍御史李宙驛召禕將下獄禕既申訴得理融坐阿黨李宙貶今從唐歷}
凡為相百日而罷【六月甲戍至九月壬子九十九日耳}
是後言財利以取貴仕者皆祖於融 冬十月戊午朔日有食之不盡如鉤 宇文融既得罪國用不足上復思之【復扶又翻}
謂裴光庭曰卿等皆言融之惡朕既黜之矣今國用不足將若之何卿等何以佐朕光庭等懼不能對會有飛狀告融贓賄事又貶平樂尉【平樂縣漢蒼梧郡荔浦之地晉置平樂縣屬始安郡唐分置昭州有平樂水樂音洛考異曰唐歷云裴光庭等諷有司劾之積其贓鉅萬計舊傳曰裴光庭時兼御史大夫又彈融交遊朋黨及男受贓等事今從實録統紀又唐歷云十月乙未按長歷十月戊午朔無乙未今從統紀}
至嶺外歲餘司農少卿蔣岑奏融在汴州隱沒官錢鉅萬計制窮治其事融坐流巖州【高宗調露二年分横貴二州置巖州以巖岡之北因名治直之翻}
道卒【卒子恤翻}
十一月辛卯上行謁橋定獻昭乾五陵【行謁五陵以車駕經行近遠先後為次}
戊申還宫赦天下百姓今年地税悉蠲其半【蠲古玄翻}
十二月辛酉上幸新豐温泉壬申還宫【新豐温泉即驪山温泉驪山在新豐縣}
十八年春正月 【考異曰實録云癸酉上御含元殿受朝賀按長歷是月甲戌朔無癸酉實録此年事與本紀唐歷統紀皆不同正月甲子全差誤疑本書闕亡後人附益之新紀止據舊紀全不取此年實録又云丁巳新迎氣于東郊下制十八年正月五日以前天下囚徒常赦所不免者咸赦放之按是月無丁巳諸書及會要皆無十八年親迎氣事唐歷在二十六年正月七日丙子統紀在二十六年正月實録二十六年正月丁丑又載迎氣大赦其制文推恩大略與此年相似或者實録誤重出於此今不取}
辛卯以裴光庭為侍中 二月癸酉初令百官於春月旬休選勝行樂【令尋選勝地行遊而宴樂也}
自宰相至員外郎凡十二筵各賜錢五千緡上或御花萼樓邀其歸騎留飲迭使起舞盡歡而去【騎奇寄翻}
三月丁酉復給京官職田【收職田見上卷十年}
夏四月 【考異曰實録云乙巳駕幸温泉宫丁未至自温泉宮按長歷是月乙卯朔無乙巳丁未舊紀唐歷亦無幸温泉事今不取}
丁卯築西京外郭九旬而畢 乙丑以裴光庭兼吏部尚書先是選司注官惟視其人之能否【先悉薦翻選須絹翻}
或不次超遷或老於下位有出身二十餘年不得禄者又州縣亦無等級或自大入小或初久後遠皆無定制光庭始奏用循資格各以罷官若干選而集【謂罷官之後經選凡幾各以多少為次而集于吏部}
官高者選少【少詩沼翻}
卑者選多無問能否選滿即注限年躡級毋得踰越非負譴者皆有升無降【此即後魏崔亮之停年格循而行之至今猶然才俊之士老於常調者多矣}
其庸愚沈滯者皆喜【沈持林翻}
謂之聖書而才俊之士無不怨歎宋璟爭之不能得光庭又令流外行署亦過門下省審【省悉景翻}
五月吐蕃遣使致書于境上求和【使疏吏翻}
初契丹王李邵固遣可突干入貢同平章事李元紘不禮焉左丞相張說謂人曰奚契丹必叛可突干狡而狠專其國政久矣人心附之【此謂契丹國人之心也契欺訖翻又音喫狼戶墾翻}
今失其心必不來矣己酉可突干弑邵固帥其國人并脅奚衆叛降突厥奚王李魯蘇及其妻韋氏邵固妻陳氏皆來犇【史言張說之言之驗韋陳皆中國以為公主嫁兩蕃事見上十四年帥讀曰率降戶江翻厥九勿翻}
制幽州長史趙含章討之又命中書舍人裴寛給事中薛侃等於關内河東河南北分道募勇士六月 【考異曰唐朝年代記云初裴光庭娶武三思女高力士私焉光庭有吏材力士為之推轂因以入相時彦鄙之宋璟王晙酒後舞囘波樂以為戲謔光庭患之乃奏天下三十餘州缺刺史昇平日久人皆不樂外官請重臣兼外官領刺史以雄其望于是擬璟揚州晙魏州陸象先荆州凡十餘人蕭嵩執奏天下務重實賴舊臣宿德訪其得失今盡失之則朝廷空矣上乃悟遂止按實録是歲閏六月以太子少保陸象先兼荆州長史璟晙未嘗除外官今不取}
丙子以單于大都護忠王浚領河北道行軍元帥【單音蟬帥所類翻}
以御史大夫李朝隱京兆尹裴伷先副之帥十八總管以討奚契丹【朝直遥翻伷與胄同帥讀曰率}
命浚與百官相見於光順門張說退謂學士孫逖韋述曰【此集賢書院學士也}
吾嘗觀太宗畫像雅類忠王此社稷之福也可突干寇平盧先鋒使張掖烏承玼破之於捺禄山【開元初置平盧軍於營州玼且禮翻又音此捺奴葛翻 考異曰韓愈烏氏先廟碑云尚書諱承洽開元中營平盧先鋒軍屢破奚契丹從戰捺禄走可突干新傳云承玼開元中與族兄承恩皆為平盧先鋒沈勇而决號轅門二龍据此則承玼承洽一人也今從新書}
壬午洛水溢溺東都千餘家【溺奴狄翻}
秋九月丁巳以忠王浚兼河東道元帥然竟不行 吐蕃兵數敗而懼乃求和親【數所角翻}
忠王友皇甫惟明【唐諸王友從五品上掌陪侍規諷}
因奏事從容言和親之利上曰贊普嘗遺吾書悖慢【吐蕃請用敵國禮見二百十一卷二年從千容翻遺于季翻悖蒲内翻又蒲沒翻}
此何可捨對曰贊普當開元之初年尚幼穉【武后長安三年贊普立方七歲至開元初猶是幼年也穉直利翻}
安能為此書殆邊將詐為之欲以激怒陛下耳夫邊境有事則將吏得以因緣盗匿官物妄述功狀以取勲爵【將即亮翻下同}
此皆姦臣之利非國家之福也兵連不解日費千金【兵法曰興師十萬日費千金}
河西隴右由兹困敝陛下誠命一使往視公主【謂金城公主也使疏吏翻下方使遣使同}
因與贊普面相約結使之稽顙稱臣【稽音啟}
永息邊患豈非御夷狄之長策乎上悦命惟明與内侍張元方使于吐蕃贊普大喜悉出貞觀以來所得敕書以示惟明 冬十月遣其大臣論名悉獵隨惟明入貢 【考異曰實録十九年七月癸巳吐蕃遣其大臣名悉獵來朝請固和好之約且獻書云云按長歷十九年七月丁未朔無癸巳今從唐歷舊本紀吐蕃傳}
表稱甥世尚公主義同一家中間張玄表等先興兵寇鈔【武后時張玄表為安西都護與吐蕃互相侵掠鈔楚交翻}
遂使二境交惡甥深識尊卑安敢失禮正為邊將交構致獲罪於舅屢遣使者入朝【朝直遥翻}
皆為邊將所遏今蒙遠降使臣來視公主甥不勝喜荷【勝音升荷下可翻}
儻使復修舊好死無所恨自是吐蕃復欵附【復扶又翻好呼到翻}
庚寅上幸鳳泉湯癸卯還京師【岐州郡縣有鳳泉府}
甲寅護密王羅真檀入朝留宿衛【護密或曰逹摩悉鐵帝或曰鑊偘元魏所謂鉢和者亦吐火羅故地東北直京師九千里而嬴皆臨烏滸河當四鎮入吐火羅道}
十一月丁卯上幸驪山温泉丁丑還宫 是歲天下奏死罪止二十四人 突騎施遣使入貢上宴之於丹鳳樓【丹鳳門樓也東内大明宫正門曰丹鳳門}
突厥使者預焉二使爭長突厥曰突騎施小國本突厥之臣不可居我上突騎施曰今日之宴為我設也我不可以居其下【長知兩翻為于偽翻}
上乃命設東西幕突厥在東突騎施在西 開府儀同三司内外閑廐監牧都使霍國公王毛仲【内外十二閑八坊四十八監及沙苑等監及諸牧皆屬之故曰都使}
恃寵驕恣日甚上每優容之毛仲與左領軍大將軍葛福順左監門將軍唐地文左武衛將軍李守德右威衛將軍王景耀高廣濟親善福順等倚其埶多為不法毛仲求兵部尚書不得怏怏形於辭色【監占銜翻怏於兩翻}
上由是不悦是時上頗寵任宦官往往為三品將軍【楊思勗高力士之徒是也}
門施棨戟【棨音啟項安世家說曰棨戟殳也以赤油韜之亦曰油戟}
奉使過諸州官吏奉之惟恐不及所得賂遺少者不減千緡【使疏吏翻遺于季翻}
由是京城郊畿田園參半皆宦官矣【參半者或居三分之一或居其半}
楊思勗高力士尤貴幸思勗屢將兵征討【楊思勗屢出征嶺南皆有功明皇不以閹人殿國師為辱而又寵秩之將即亮翻}
力士常居中侍衛而毛仲視宦官貴近者若無人甚卑品者【甚當作其}
小忤意輒詈辱如僮僕【忤五故翻詈力智翻}
力士等皆害其寵而未敢言會毛仲妻產子三日上命力士賜之酒饌金帛甚厚【饌雛戀翻又雛晥翻}
且授其兒五品官力士還上問毛仲喜乎對曰毛仲抱其襁中兒示臣曰此兒豈不堪作三品邪【襁居兩翻}
上大怒曰昔誅韋氏此賊心持兩端【事見二百九卷睿宗景雲元年}
朕不欲言之今日乃敢以赤子怨我力士因言北門奴官太盛【王毛仲李守德皆帝奴也又葛福順等皆出於萬騎中宗以戶奴補萬騎故云然}
相與一心不早除之必生大患上恐其黨驚懼為變十九年春正月壬戌下制但述毛仲不忠怨望貶瀼州别駕【瀼如羊翻又而章翻宋白曰瀼州臨潭郡隋將劉方始開此路貞觀十二年尋劉方故道行逹交趾開拓夷獠置瀼州州在欝林之西南交趾之東北有瀼水以為州名 考異曰實録十八年六月乙丑王毛仲貶瀼州按唐歷統紀舊紀毛仲貶皆在十九年正月今從之}
福順地文守德景耀廣濟皆貶遠州别駕毛仲四子皆貶遠州參軍連坐者數十人毛仲行至永州追賜死【舊志永州京師南三千二百七十四里}
自是宦官勢益盛高力士尤為上所寵信嘗曰力士上直【上直掌翻}
吾寢則安故力士多留禁中稀至外第四方表奏皆先呈力士然後奏御小者力士即决之勢傾内外金吾大將軍程伯獻少府監馮紹正與力士約為兄弟力士母麥氏卒伯獻等被髪受弔擗踴哭泣過於己親【被皮義翻擗毘亦翻撫心也}
力士娶瀛州呂玄晤女為妻擢玄晤為少卿子弟皆王傅【唐諸王傳從三品掌輔相贊導匡其過失}
呂氏卒朝野爭致祭【朝直遥翻}
自第至墓車馬不絶然力士小心㳟恪故上終親任之 辛未遣鴻臚卿崔琳使于吐蕃琳神慶之子也【崔神慶進用於武后之時臚陵如翻使疏吏翻下同}
吐蕃使者稱公主求毛詩春秋禮記正字于休烈上疏【上時掌翻疏所去翻 考異曰實録十一年七月壬申敇遣崔琳充入吐蕃使癸未命有司寫毛詩禮記等賜金城公主于休烈諫丁亥以崔琳為御史大夫八月辛卯降書與吐蕃按吐蕃傳此年十月論名悉獵至京師本紀唐歷皆同十九年正月辛未乃遣崔琳報使二月甲午以琳為御史大夫三月乙酉琳享于吐蕃金城公主因名悉獵請書于休烈乃諫實録皆誤在前年七月八月按七月癸丑朔亦無丁亥}
以為東平王漢之懿親求史記諸子漢猶不與【漢成帝弟東平王宇來朝上疏求諸子及太史公書上以問大將軍王鳳鳳曰諸子書或反經術非聖人或明鬼神信物怪太史公書有戰國縱横權譎之謀漢興之初謀臣奇策天官災異地形阨塞皆不宜在諸侯王不可與遂不與}
况吐蕃國之寇讐今資之以書使知用兵權略愈生變詐非中國之利也事下中書門下議之【事下遐嫁翻}
裴光庭等奏吐蕃聾昧頑嚚久叛新服因其有請賜以詩書庶使之漸陶聲教化流無外休烈徒知書有權略變詐之語不知忠信禮義皆從書出也上曰善遂與之休烈志寧之玄孫也【于志寧事太宗高宗得罪於武后}
丙子上躬耕於興慶宫側盡三百步 三月突厥左賢王闕特勒卒賜書弔之【闕特勒殺默啜之子而立毗伽威得行於其國故賜書弔之}
丙申初令兩京諸州各置太公廟以張良配享選古名將以備十哲【張良配饗齊大司馬田穰苴吴將軍孫武魏西河太守吴起燕昌國君樂毅秦武安君白起漢淮隂侯韓信蜀丞相諸葛亮尚書右僕射衛國公李靖司空英國公李勣將即亮翻}
以二八月上戊致祭如孔子禮【祠武成王自此始}


臣光曰經緯天地之謂文戡定禍亂之謂武自古不兼斯二者而稱聖人未之有也故黄帝堯舜禹湯文武伊尹周公莫不有征伐之功孔子雖不試猶能兵萊夷却費人曰我戰則克【魯定公與齊會于夾谷孔子相齊使萊夷以兵刼魯公孔子曰士兵之兩君合好而裔夷以兵亂之非齊君所以命諸侯也齊侯聞之遽辟之及攝行相事將墮三都於是叔孫氏墮郈季氏將墮費公山不狃叔孫輒帥費人以襲魯仲尼命申句須樂頎伐之費人北國人追之敗諸姑蔑二子奔齊遂墮費又孔子曰我戰則克祭則受福費音秘}
豈孔子專文而太公專武乎孔子所以祀於學者禮有先聖先師故也自生民以來未有如孔子者豈太公得與之抗衡哉古者有發則命大司徒教士以車甲臝股肱决射御【記王制之言有發謂有軍師發卒教以乘兵車衣甲之儀羸股肱决射御謂擐衣出其臂脛使之射御决勝負見勇力羸力果翻}
受成獻馘莫不在學【詩魯頌泮水曰矯矯虎臣在泮獻馘淑問如臯陶在泮獻囚受成謂受獄辭之成也}
所以然者欲其先禮義而後勇力也【先悉薦翻後戶搆翻}
君子有勇而無義為亂小人有勇而無義為盗【論語載孔子之言}
若專訓之以勇力而不使之知禮義奚所不為矣自孫吳以降皆以勇力相勝狙詐相高豈足以數於聖賢之門而謂之武哉乃復誣引以偶十哲之目為後世學者之師【復扶又翻}
使太公有神必羞與之同食矣

五月壬戌初立五嶽真君祠【程大昌演繁露曰開元十九年司馬承禎言今五嶽神祠是山林之神非正真之神也敕各置真君祠一所杜佑曰開元九年天台道士司馬承禎言今五嶽神祠是山林之神非正真之神五嶽皆有洞府有上清真人降任其職山川風雨隂陽氣序是所理焉冠冕服章佐從神仙皆有名數請别立齋祠之所上奇其說因敕五岳各置真君祠}
秋九月辛未吐蕃遣其相論尚它硉入見請於赤嶺為互市許之【相悉亮翻硉郎兀翻見賢遍翻石堡城西二十里至赤嶺}
冬十月丙申上幸東都 或告巂州都督解人張審素贓汚【解縣屬河中府元魏分解縣置虞鄉縣貞覲十七年省解縣併入虞鄉二十年復置解縣省虞鄉天授二年復分解縣置虞鄉縣定為兩縣巂音髓解戶買翻}
制遣監察御史楊汪按之總管董元禮將兵七百圍汪殺告者【監古銜翻將即亮翻}
謂汪曰善奏審素則生不然則死會救兵至擊斬之汪奏審素謀反十二月審素坐斬籍沒其家【為後審素二子復讐張本}
浚苑中洛水六旬而罷

二十年春正月乙卯以朔方節度副大使信安王禕為河東河北行軍副大總管將兵擊奚契丹壬申以戶部侍郎裴耀卿為副總管 二月癸酉朔日有食之 上思右驍衛將軍安金藏忠烈【金藏事見二百五卷武后長夀二年驍堅堯翻}
三月賜爵代國公仍於東西岳立碑以銘其功金藏竟以夀終 信安王禕帥裴耀卿及幽州節度使趙含章分道擊契丹【帥讀曰率下同}
含章與虜遇虜望風遁去平盧先鋒將烏承玼言於含章曰二虜劇賊也前日遁去非畏我乃誘我也【將即亮翻玼且禮翻又音此誘音酉}
宜按兵以觀其變含章不從與虜戰於白山【白山後漢時烏桓所居在五阮關外大荒中}
果大敗承玼别引兵出其右擊虜破之己巳禕等大破奚契丹俘斬甚衆 【考異曰唐歷作庚午今從實録}
可突干帥麾下遠遁餘黨潜竄山谷奚酋李詩瑣高帥五千餘帳來降【酋慈由翻降戶江翻}
禕引兵還賜李詩爵歸義王充歸義州都督徙其部落置幽州境内【高宗總章中以新羅降戶置歸義州於良鄉縣廣陽城後廢今復置以處李詩部落}
夏四月乙亥宴百官於上陽東洲【上陽宫南臨洛水引洛水為中洲於宫之東}
醉者賜以衾褥肩輿以歸相屬于路【屬之欲翻}
六月丁丑加信安王禕開府儀同三司上命裴耀卿齎絹二十萬匹分賜立功奚官耀卿謂其徒曰戎狄貪婪【婪盧含翻}
今齎重貨深入其境不可不備乃命先期而往【先悉薦翻}
分道並進一日給之俱畢突厥室韋果發兵邀隘道欲掠之比至耀卿已還【比必利翻}
趙含章坐贓巨萬杖于朝堂【朝直遥翻}
流瀼州道死 秋七月蕭嵩奏自祠后土以來屢獲豐年宜因還京賽祠上從之【祠后土見上卷十一年還京謂還西京也賽先代翻}
敕裴光庭蕭嵩分押左右廂兵【此分押南牙左右廂兵也}
八月辛未朔日有食之 初上命張說與諸學士刋定五禮說薨蕭嵩繼之起居舍人王仲丘請依明慶禮【明慶即顯慶也避中宗諱改曰明慶}
祈穀大雩明堂皆祀昊天上帝嵩又請依上元敕父在為母齊衰三年皆從之【禮父在為母服朞開元之初禇無量固嘗以為言矣為于偽翻齊音咨衰倉囘翻}
以高祖配圓丘方丘太宗配雩祀及神州地祇睿宗配明堂九月乙巳新禮成上之【上時掌翻}
號曰開元禮 勃海靺鞨王武藝遣其將張文休帥海賊寇登州【靺鞨音末曷將即亮翻帥讀曰率}
殺刺史韋俊上命右領軍將軍葛福順發兵討之【去年春葛福順方以黨附王毛仲貶今則仍為宿衛蓋毛仲既誅福順等復叙用也開元九年貶王晙梓州已而復為尚書復居邉任事亦類此}
壬子河西節度使牛仙客加六階初蕭嵩在河西委軍政於仙客仙客亷勤善於其職嵩屢薦之竟代嵩為節度使 冬十月壬午上發東都辛卯幸潞州辛丑至北都十一月庚申祀后土于汾隂【蕭嵩之言也}
赦天下十二月辛未還西京是歲以幽州節度使兼河北采訪處置使增領衛相洺貝冀魏深趙恒定邢德博棣營鄚十六州及安東都護府【德州漢安德廣川平昌之地舊置平原郡時置德州安東都護府時治平州處昌呂翻使疏吏翻恒戶登翻鄚音莫}
天下戶七百八十六萬一千二百三十六口四千五百四十三萬一千二百六十五

二十一年春正月乙巳祔肅明皇后于太廟毁儀坤廟【肅明留祀儀坤見二百十一卷四年}
丁巳上幸驪山温泉 上遣大門藝詣幽州發兵以討勃海王武藝 【考異曰新書烏承玼傳云可突干殺其王邵固降突厥而奚亦亂是歲奚契丹入寇詔承玼擊之破於捺禄山又云勃海大武藝行兵至馬都山屠城邑承玼窒要路塹以大石亘四百里於是流人得還土少休脱鎧而耕歲省度支運錢按韓愈為烏重作廟碑叙重胤父承洽云屢破契丹從戰捺禄走可突干勃海上至馬都山吏民逃徙失業尚書領所部兵塞其道塹原累石綿四百里深高皆三丈寇不得進民還其居歲罷錢三千萬疑新書約此碑作承玼傳按新舊帝紀及勃海傳皆無武藝入寇至馬都山事或者韓碑云走可突干勃海上至馬都山謂破走可突干勃海上追之至馬都山耳二十一年郭英傑與可突干戰都山然則都山蓋契丹之地也吏民逃徙失業蓋因可突干入寇而然與上止是一事新書承之致誤然未知新書承玼傳中餘事别據何書}
庚申命太僕員外卿金思蘭使于新羅【思蘭新羅王之侍子留京師官為太僕卿員外置}
發兵擊其南鄙會大雪丈餘山路阻隘士卒死者過半無功而還武藝怨門藝不已密遣客刺門藝於天津橋南不死上命河南搜捕賊黨盡殺之【河南謂河南府刺七亦翻}
二月丁酉金城公主請立碑於赤嶺以分唐與吐蕃之境許之【為後絶吐蕃和親仆赤嶺碑張本}
三月乙巳侍中裴光庭薨太常博士孫琬議光庭用循資格失勸奬之道請諡曰克其子稹訟之【稹止忍翻}
上賜諡忠獻上問蕭嵩可以代光庭者嵩與右散騎常侍王丘善將薦之固讓於右丞韓休嵩言休於上【蕭嵩既能用王丘之言而薦韓休使能與之和衷則丘之善乃嵩之善也}
甲寅以休為黄門侍郎同平章事休為人峭直【峭七肖翻峻也}
不干榮利及為相甚允時望始嵩以休恬和謂其易制【易以䜴翻}
故引之及與共事休守正不阿嵩漸惡之宋璟歎曰不意韓休乃能如是上或宫中宴樂及後苑遊獵小有過差輒謂左右曰韓休知否言終諫疏已至上嘗臨鏡默然不樂左右曰韓休為相陛下殊瘦於舊何不逐之上歎曰吾貌雖瘦天下必肥蕭嵩奏事常順指既退吾寢不安韓休常力爭既退吾寢乃安吾用韓休為社稷耳非為身也【惡烏路翻樂音洛為于偽翻明皇待韓休如此而不能久任之何也}
有供奉侏儒名黄㼐【㼐部田翻}
性警黠【黠下八翻慧也}
上常馮之以行謂之肉几寵賜甚厚一日晩入上恠之對曰臣曏入宫道逢捕盗官與臣爭道臣掀之墜馬故晩【掀虚言翻}
因下階叩頭上曰但使外無章奏汝亦無憂有頃京兆奏其狀上即叱出付有司杖殺之 閏月癸酉幽州道副總管郭英傑與契丹戰于都山敗死時節度薛楚玉遣英傑將精騎一萬及降奚擊契丹屯於榆關之外【榆當作渝此渝關在營平之間古所謂臨渝之險者也漢書音義渝音喻又唐勝州界有榆關隋之榆林郡界二關有渝榆之異史家傳寫混淆無别故詳辯之將即亮翻騎奇寄翻降戶江翻下同}
可突干引突厥之衆來合戰奚持兩端散走保險唐兵不利英傑戰死餘衆六千餘人猶力戰不已虜以英傑首示之竟不降盡為虜所殺楚玉訥之弟也 夏六月癸亥制自今選人有才業操行委吏部臨時擢用流外奏用不復引過門下【操七到翻行下孟翻復扶又翻}
雖有此制而有司以循資格便於己猶踵行之【史言裴光庭之弊法後人循襲莫之能革}
是時官自三師以下一萬七千六百八十六員【唐制太師太傅太保為三師}
吏自佐史以上五萬七千四百一十六員而入仕之塗甚多不可勝紀【勝音升}
秋七月乙丑朔日有食之 九月壬午立皇子沔為信王泚為義王漼為陳王澄為豐王潓為恒王漎為梁王【沔彌兖翻泚且禮翻又音此潓胡桂翻漎徂聰翻又徂宗翻又將容翻又之戎翻恒戶登翻}
滔為汴王【汴皮變翻}
關中久雨穀貴上將幸東都召京兆尹裴耀卿謀之對曰關中帝業所興當百代不易但以地狹穀少故乘輿時幸東都以寛之【少詩沼翻乘繩證翻}
臣聞貞觀永徽之際禄廪不多歲漕關東一二十萬石足以周贍【贍而艶翻}
乘輿得以安居今用度寖廣運數倍於前猶不能給故使陛下數冒寒暑以恤西人【下數所角翻}
今若使司農租米悉輸東都自都轉漕稍實關中苟關中有數年之儲則不憂水旱矣且吳人不習河漕所在停留日月既久遂生隱盗臣請於河口置倉【河口汴水逹河之口也河口倉謂之武牢倉}
使吳船至彼即輸米而去官自雇載分入河洛又於三門東西各置一倉【禹鑿底柱二石見於水中若柱然故曰底柱河水至此分為三派流出其間故亦謂之三門時於三門東置集津倉西置鹽倉}
至者貯納【貯丁呂翻}
水險則止水通則下或開山路車運而過【時於三門旁側鑿山路十八里以陸運以避底柱之險}
則無復留滯省費鉅萬矣河渭之濱皆有漢隋舊倉葺之非難也上深然其言 冬十月庚戌上幸驪山温泉己未還宫 戊子左丞相宋璟致仕歸東都 韓休數與蕭嵩爭論於上前面折嵩短【數所角翻折之舌翻}
上頗不悦嵩因乞骸骨上曰朕未厭卿卿何為遽去對曰臣蒙厚恩待罪宰相富貴已極及陛下未厭臣故臣得從容引去【從千容翻}
若已厭臣臣首領且不保安能自遂因泣下【蕭嵩為乞憐之態既以自保寵禄亦所以傾韓休也}
上為之動容【為于偽翻}
曰卿且歸朕徐思之丁巳嵩罷為左丞相休罷為工部尚書以京兆尹裴耀卿為黄門侍郎前中書侍郎張九齡時居母喪起復中書侍郎並同平章事 是歲分天下為京畿都畿關内河南河東河北隴右山南東道山南西道劒南淮南江南東道江南西道黔中嶺南凡十五道各置采訪使以六條檢察非法兩畿以中丞領之餘皆擇賢刺史領之【京畿采訪使治西京城内都畿治東都關内采訪使以京官領之河南采訪使治汴州河東治蒲州河北治魏州隴右治鄯州山南東道治襄州西道治梁州淮南治揚州江南東道治蘇州西道治洪州黔中治黔州劒南治益州嶺南治廣州其後有以邉鎮節度領采訪使者則關中道固不拘京官而諸道采訪使治所亦難槩拘以定所也}
非官有遷免則使無廢更【使疏吏翻更工衡翻}
惟變革舊章乃須報可自餘聽便宜從事先行後聞 太府卿楊崇禮政道之子也【楊政道隋煬帝之孫齊王暕之子}
在太府二十餘年前後為太府者莫能及時承平日久財貨山積嘗經楊卿者無不精美每歲句駮省便出錢數百萬緡【句音鈎句者句考其出入或多或少駮者按文籍有並緣欺弊則駮異之省者節其冗濫之費便者貿遷各隨其便以取贏駮北角翻}
是歲以戶部尚書致仕年九十餘矣上問宰相崇禮諸子誰能繼其父者對曰崇禮三子慎餘慎矜慎名皆亷勤有才而慎矜為優上乃擢慎矜自汝陽令為監察御史知太府出納慎名攝監察御史知含嘉倉出納【含嘉倉在東都監工衘翻}
亦皆稱職上甚悦之【稱尺證翻}
慎矜奏諸州所輸布帛有漬汚穿破者【漬疾智翻汚烏故翻}
皆下本州徵折估錢轉市輕貨徵調始繁矣【估音古下遐嫁翻調徒弔翻}


資治通鑑卷二百十三














































































































































