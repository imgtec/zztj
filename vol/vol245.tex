<!DOCTYPE html PUBLIC "-//W3C//DTD XHTML 1.0 Transitional//EN" "http://www.w3.org/TR/xhtml1/DTD/xhtml1-transitional.dtd">
<html xmlns="http://www.w3.org/1999/xhtml">
<head>
<meta http-equiv="Content-Type" content="text/html; charset=utf-8" />
<meta http-equiv="X-UA-Compatible" content="IE=Edge,chrome=1">
<title>資治通鑒_246-資治通鑑卷二百四十五_246-資治通鑑卷二百四十五</title>
<meta name="Keywords" content="資治通鑒_246-資治通鑑卷二百四十五_246-資治通鑑卷二百四十五">
<meta name="Description" content="資治通鑒_246-資治通鑑卷二百四十五_246-資治通鑑卷二百四十五">
<meta http-equiv="Cache-Control" content="no-transform" />
<meta http-equiv="Cache-Control" content="no-siteapp" />
<link href="/img/style.css" rel="stylesheet" type="text/css" />
<script src="/img/m.js?2020"></script> 
</head>
<body>
 <div class="ClassNavi">
<a  href="/24shi/">二十四史</a> | <a href="/SiKuQuanShu/">四库全书</a> | <a href="http://www.guoxuedashi.com/gjtsjc/"><font  color="#FF0000">古今图书集成</font></a> | <a href="/renwu/">历史人物</a> | <a href="/ShuoWenJieZi/"><font  color="#FF0000">说文解字</a></font> | <a href="/chengyu/">成语词典</a> | <a  target="_blank"  href="http://www.guoxuedashi.com/jgwhj/"><font  color="#FF0000">甲骨文合集</font></a> | <a href="/yzjwjc/"><font  color="#FF0000">殷周金文集成</font></a> | <a href="/xiangxingzi/"><font color="#0000FF">象形字典</font></a> | <a href="/13jing/"><font  color="#FF0000">十三经索引</font></a> | <a href="/zixing/"><font  color="#FF0000">字体转换器</font></a> | <a href="/zidian/xz/"><font color="#0000FF">篆书识别</font></a> | <a href="/jinfanyi/">近义反义词</a> | <a href="/duilian/">对联大全</a> | <a href="/jiapu/"><font  color="#0000FF">家谱族谱查询</font></a> | <a href="http://www.guoxuemi.com/hafo/" target="_blank" ><font color="#FF0000">哈佛古籍</font></a> 
</div>

 <!-- 头部导航开始 -->
<div class="w1180 head clearfix">
  <div class="head_logo l"><a title="国学大师官网" href="http://www.guoxuedashi.com" target="_blank"></a></div>
  <div class="head_sr l">
  <div id="head1">
  
  <a href="http://www.guoxuedashi.com/zidian/bujian/" target="_blank" ><img src="http://www.guoxuedashi.com/img/top1.gif" width="88" height="60" border="0" title="部件查字,支持20万汉字"></a>


<a href="http://www.guoxuedashi.com/help/yingpan.php" target="_blank"><img src="http://www.guoxuedashi.com/img/top230.gif" width="600" height="62" border="0" ></a>


  </div>
  <div id="head3"><a href="javascript:" onClick="javascript:window.external.AddFavorite(window.location.href,document.title);">添加收藏</a>
  <br><a href="/help/setie.php">搜索引擎</a>
  <br><a href="/help/zanzhu.php">赞助本站</a></div>
  <div id="head2">
 <a href="http://www.guoxuemi.com/" target="_blank"><img src="http://www.guoxuedashi.com/img/guoxuemi.gif" width="95" height="62" border="0" style="margin-left:2px;" title="国学迷"></a>
  

  </div>
</div>
  <div class="clear"></div>
  <div class="head_nav">
  <p><a href="/">首页</a> | <a href="/ShuKu/">国学书库</a> | <a href="/guji/">影印古籍</a> | <a href="/shici/">诗词宝典</a> | <a   href="/SiKuQuanShu/gxjx.php">精选</a> <b>|</b> <a href="/zidian/">汉语字典</a> | <a href="/hydcd/">汉语词典</a> | <a href="http://www.guoxuedashi.com/zidian/bujian/"><font  color="#CC0066">部件查字</font></a> | <a href="http://www.sfds.cn/"><font  color="#CC0066">书法大师</font></a> | <a href="/jgwhj/">甲骨文</a> <b>|</b> <a href="/b/4/"><font  color="#CC0066">解密</font></a> | <a href="/renwu/">历史人物</a> | <a href="/diangu/">历史典故</a> | <a href="/xingshi/">姓氏</a> | <a href="/minzu/">民族</a> <b>|</b> <a href="/mz/"><font  color="#CC0066">世界名著</font></a> | <a href="/download/">软件下载</a>
</p>
<p><a href="/b/"><font  color="#CC0066">历史</font></a> | <a href="http://skqs.guoxuedashi.com/" target="_blank">四库全书</a> |  <a href="http://www.guoxuedashi.com/search/" target="_blank"><font  color="#CC0066">全文检索</font></a> | <a href="http://www.guoxuedashi.com/shumu/">古籍书目</a> | <a   href="/24shi/">正史</a> <b>|</b> <a href="/chengyu/">成语词典</a> | <a href="/kangxi/" title="康熙字典">康熙字典</a> | <a href="/ShuoWenJieZi/">说文解字</a> | <a href="/zixing/yanbian/">字形演变</a> | <a href="/yzjwjc/">金 文</a> <b>|</b>  <a href="/shijian/nian-hao/">年号</a> | <a href="/diming/">历史地名</a> | <a href="/shijian/">历史事件</a> | <a href="/guanzhi/">官职</a> | <a href="/lishi/">知识</a> <b>|</b> <a href="/zhongyi/">中医中药</a> | <a href="http://www.guoxuedashi.com/forum/">留言反馈</a>
</p>
  </div>
</div>
<!-- 头部导航END --> 
<!-- 内容区开始 --> 
<div class="w1180 clearfix">
  <div class="info l">
   
<div class="clearfix" style="background:#f5faff;">
<script src='http://www.guoxuedashi.com/img/headersou.js'></script>

</div>
  <div class="info_tree"><a href="http://www.guoxuedashi.com">首页</a> > <a href="/SiKuQuanShu/fanti/">四库全书</a>
 > <h1>资治通鉴</h1> <!--         下载:【右键另存为】即可 --></div>
  <div class="info_content zj clearfix">
  
<div class="info_txt clearfix" id="show">
<center style="font-size:24px;">246-資治通鑑卷二百四十五</center>
    資治通鑑卷二百四十五 宋 司馬光 撰<br />
<br />
  胡三省 音注<br />
<br />
  唐紀六十一【起閼逢攝提格盡彊圉大荒落凡四年】<br />
<br />
  文宗元聖昭獻孝皇帝中<br />
<br />
  大和八年春正月上疾小瘳丁巳御太和殿【按閣本大明宫圖入左銀臺門稍北即太和殿又西即清思殿】見近臣然神識耗減不能復故二月壬午朔日有食之 夏六月丙戌莒王紓薨【紓順宗子紓山於翻】 上以久旱詔求致雨之方司門員外郎李中敏上表以為仍歲大旱非聖德不至直以宋申錫之寃濫【宋申錫事見上卷五年】鄭注之姦邪今致雨之方莫若斬注而雪申錫表留中中敏謝病歸東都 【考異曰新舊中敏傳皆云六年夏上此疏今據開成紀事大和摧兇記皆云八年六月又中敏疏言申錫臨終按申錫去年七月卒若六年則申錫尚在今從開成紀事】 郯王經薨【經亦順宗子】 初李仲言流象州【事見二百四十三卷敬宗寶歷元年】遇赦還東都會留守李逢吉思復入相【復扶又翻】仲言自言與鄭注善逢吉使仲言厚賂之注引仲言見王守澄守澄薦於上云仲言善易上召見之時仲言有母服難入禁中乃使衣民服【衣於既翻】號王山人仲言儀狀秀偉倜儻尚氣【倜他歷翻倜儻不覊也史炤曰卓異貌】頗工文辭有口辨多權數上見之大悦以為奇士待遇日隆 【考異曰舊傳李訓初名仲言居洛中李逢吉為留守思入相訓揣知其意即以奇計動之自言與鄭注善逢吉遺訓金帛珍寶數百萬令持入長安以賂注又曰初注搆宋申錫事帝深惡之欲令京兆尹杖殺至是以藥稍效始善遇之獻替記曰先是上惡鄭注極甚嘗謂樞密使曰卿知有善和端公無歎京兆尹懦弱不能斃於枯木開成紀事曰訓除名流象州會恩歸於東洛投謁諸處困乏逢吉叱之不顧會鄭注賓副上黨路經東都于道投之廣以古今義烈披述衷欵注本兇邪趨而附之自此豁然相然諾情契稠疊及注徵赴闕訓隨而到京别第安置注因陳奏言訓文學優盛無比上納之太和八年三月以布衣在翰林注之援也甘露記曰訓為人長大美貌口辯無前常以英雄自任會鄭注介工黨出洛陽訓慨然太息曰當世操權力者握齪苛細無足與言吾聞鄭注為人好義而求奇士且通於内官易為因緣乃往說之注見訓大驚如舊相識遂結為死交及注赴闕請訓行京師為卜居供給日夕往來乘間奏於上按實錄去年九月李欵彈鄭注云前邠州行軍司馬今年九月庚申王守澄宣召鄭注對於浴堂門獻替記八年春暮上對宰臣歎天下無名醫便及鄭注精於服食或欲置於技術或欲令為神策判官注皆不願此職守澄遂託從諫奏為行軍司馬又云去歲春夏李仲言猶喪母已潛入城稱王山人兩度對於含元殿今年八月十三日欲與諫官至九月三日鄭注自絳州至便於宣徽對然則訓自去年已因注謁守澄得見上注今年暮春方從昭義辟然則訓舊與注善去春已入長安見上非注赴昭義時始定交亦非去年十一月徵注於潞州又非訓隨注到京也今從實錄獻替記】仲言既除服秋八月辛卯上欲以仲言為諫官寘之翰林李德裕曰仲言曏所為計陛下必盡知之豈宜寘之近侍【兩省官皆近侍也】上曰然豈不容其改過對曰臣聞惟顔囘能不貳過彼聖賢之過但思慮不至或失中道耳至於仲言之惡著於心本安能悛改邪【著直畧翻悛丑緣翻心本猶言心根也】上曰李逢吉薦之朕不欲食言對曰逢吉身為宰相乃薦姦邪以誤國亦罪人也上曰然則别除一官對曰亦不可上顧王涯涯對曰可德裕揮手止之上囘顧適見色殊不懌而罷始涯聞上欲用仲言草諫疏極憤激既而見上意堅且畏其黨盛遂中變尋以仲言為四門助教【四門助教從八品】給事中鄭肅韓佽封還敕書【佽七四翻】德裕將出中書謂涯曰且喜給事中封敕涯即召肅佽謂曰李公適留語令二閣老不用封敕【留語謂將出之時所留下言語也兩省官相呼曰閣老】二人即行下【書牘而行下之也行戶稼翻】明日以白德裕德裕驚曰德裕不欲封還當面聞何必使人傳言且有司封駮【駮比角翻】豈復禀宰相意邪【復扶又翻】二人悵恨而去九月辛亥徵昭義節度副使鄭注至京師【去年鄭注出佐昭義軍事見上卷】王守澄李仲言鄭注皆惡李德裕以山南西道節度使李宗閔與德裕不相悦引宗閔以敵之壬戌詔徵宗閔於興元【惡烏路翻李宗閔出帥興元見上卷元年興元府至京師一千二百二十三里】 冬十月辛巳幽州軍亂逐節度使楊志誠及監軍李懷仵【仵疑古翻】推兵馬使史元忠主留務 庚寅以李宗閔為中書侍郎同平章事甲午以中書侍郎同平章事李德裕同平章事充山南西道節度使是日以李仲言為翰林侍講學士給事中高銖鄭肅韓佽諫議大夫郭承嘏中書舍人權璩等争之不能得承嘏晞之孫【晞郭子儀之子】璩德輿之子也【權德輿元和初為相璩求於翻】 乙巳貢院奏進士復試詩賦從之【唐尚書省在朱雀門北正街之東自占一坊六部附麗其旁省前一坊别有禮部南院即貢院也罷詩賦見上卷上年李德裕罷相故復之】 李德裕見上自陳請留京師丙午以德裕為兵部尚書 楊志誠過太原李載義自毆擊欲殺之【楊志誠逐載義見上卷五年毆烏口翻】幕僚諫救得免殺其妻子及從行將卒朝廷以載義有功不問【李載義有平倉景之功將即亮翻】載義母兄葬幽州志誠取其財載義奏乞取志誠心以祭母不許十一月成德節度使王庭湊薨軍中奉其子都知兵<br />
<br />
  馬使元逵知留後元逵改父所為事朝廷禮甚謹 史元忠獻楊志誠所造衮衣及諸僭物丁卯流志誠於嶺南道殺之 李宗閔言李德裕制命己行不宜自便【以德裕自請留京師也】乙亥復以德裕為鎭海節度使不復兼平章事【復扶又翻】時德裕宗閔各有朋黨互相擠援【非其黨則相擠同黨則相援擠子西翻又子細翻援于元翻又于春翻】上患之每歎曰去河北賊易去朝廷朋黨難【去羌呂翻下同】<br />
<br />
  臣光曰夫君子小人之不相容猶氷炭之不可同器而處也故君子得位則斥小人小人得勢則排君子此自然之理也然君子進賢退不肖其處心也公其指事也實小人譽其所好毁其所惡【處昌呂翻譽音余好呼到翻惡烏路翻】其處心也私其指事也誣公且實者謂之正直私且誣者謂之朋黨在人主所以辨之耳是以明主在上度德而叙位量能而授官【荀卿子之言度徒洛翻量音良】有功者賞有罪者刑奸不能惑佞不能移夫如是則朋黨何自而生哉彼昏主則不然明不能燭彊不能斷【斷丁亂翻】邪正並進毁譽交至取捨不在於己威福潛移於人於是讒慝得志而朋黨之議興矣夫木腐而蠧生醯酸而蜹集【蜹而鋭翻】故朝廷有朋黨則人主當自咎而不當以咎羣臣也文宗苟患羣臣之朋黨何不察其所毁譽者為實為誣【譽音余】所進退者為賢為不肖其心為公為私其人為君子為小人苟實也賢也公也君子也匪徒用其言又當進之誣也不肖也私也小人也匪徒棄其言又當刑之如是雖驅之使為朋黨孰敢哉釋是不為乃怨羣臣之難治【治直之翻】是猶不種不芸而怨田之蕪也朝中之黨且不能去况河北賊乎【温公此論為熙豐也】<br />
<br />
  丙子李仲言請改名訓 幽州奏莫州軍亂刺史張元汎不知所在 十二月己卯以昭義節度副使鄭注為太僕卿郭承嘏累上疏言其不可上不聽于是注詐上表固辭上遣中使再以告身賜之不受【史極言鄭注之姦狀】 癸未以史元忠為盧龍留後 【考異曰實録十一月鎭州奏幽州留後史元忠為莫三軍逐出不知所在後不言元忠復歸幽州而至此有新命蓋因莫州軍亂鎭州承傳聞之誤而奏之耳】初宋申錫與御史中丞宇文鼎受密詔誅鄭注使京兆尹王璠掩捕之璠密以堂帖示王守澄【帖由政事堂出故謂之堂帖璠孚袁翻】注由是得免深德璠璠又與李訓善于是訓注共薦之自浙西觀察使徵為尚書左丞【王璠之險躁自可以得禍史言其預甘露之難亦有所自來】<br />
<br />
  九年春正月乙卯以王元逵為成德節度使 巢公湊薨追贈齊王【漳王湊貶巢公事見上卷五年】 鄭注上言秦地有災宜興役以禳之辛卯左右神策千五百人浚曲江及昆明池【雍錄唐曲江本秦隑州至漢為樂遊苑基地最高四望寛敞隋營京城宇文愷以其地在京城東南隅地高不便故闕此地不為居人坊巷而鑿為池以厭勝之又會黄渠水自城外南來故隋世遂從城外包之入城為芙蓉池且為芙蓉園也漢武帝時池周囘六里餘唐周七里占地三千頃又加展拓矣其地在城東南昇道坊龍華寺之南昆明池漢武帝所鑿在長安西南周囘四十里三輔故事曰池周三百二十頃長安志曰今為民田夫既可以為民田則非有水之地矣然則漢於何取水也長安志引水經曰交水西至石堨武帝穿昆明池所造有石闥堰在縣西南三十二里則昆明之周三百餘頃者用此堰之水也昆明基高故其下流尚可壅激以為都城之用於是並城疏别三?城内外皆賴之此池仍在括地志曰豐鎬二水皆已堰入昆明池無復流派括地志作於太宗之世則唐初仍自壅堰不廢至文宗而猶嘗加濬也然則圖經之作當在文宗後故竭而為田也】 三月冀王絿薨【緑順宗子】 丙辰以史元忠為盧龍節度使 初李德裕為浙西觀察使漳王傅母杜仲陽坐宋申錫事放歸金陵詔德裕存處之會德裕已離浙西【傅母女師也處昌呂翻離力智翻】牒留後李蟾使如詔旨【德裕自浙西徵見上卷三年鎭蜀見四年宋申錫事見五年繋年差殊當考】至是左丞王璠戶部侍郎李漢奏德裕厚賂仲陽隂結漳王圖為不軌上怒甚召宰相及璠漢鄭注等面質之璠漢等極口誣之路隋曰德裕不至有此果如所言臣亦應得罪言者稍息夏四月以德裕為賓客分司 癸巳以鄭注守太僕卿兼御史大夫注始受之仍舉倉部員外郎李欵自代曰加臣之罪雖於理而無辜在欵之誠乃事君而盡節【欵奏注見上卷上年 考異記曰時論或云欵外沽直名而隂事注按欵彈注之文皆訐其隱慝豈有於人如此而能隂與之合乎此皆當時庸人見注舉欵自代遂有此疑耳今不取】時人皆哂之【笑不壞顔為哂】丙申以門下侍郎同平章事路隋充鎭海節度使趣之赴鎭【趣讀曰促】不得面辭坐救李德裕故也 【考異曰舊隋傳曰德裕貶袁州長史隋不署奏狀始為鄭注所忌出鎮浙西按實錄隋出鎮在德裕貶前四日今不取】 初京兆尹河南賈餗【餗蘇谷翻】性褊躁輕率與李德裕有隙而善於李宗閔鄭注上已賜百官宴於曲江【古者上巳正用三月之上巳日自魏以後但用三月三日不復用已唐貞元間置三令節使百官選勝行樂三月三日其一也】故事尹於外門下馬揖御史餗恃其貴勢乘馬直入殿中侍御史楊儉蘇特與之争餗罵曰黄面兒敢爾坐罰俸餗耻之求出詔以為浙西觀察使尚未行戊戌以餗為中書侍郎同平章事 庚子制以曏日上初得疾【謂七年冬也】王涯呼李德裕奔問起居德裕竟不至又在西蜀徵逋懸錢三十萬緍百姓愁困貶德裕袁州長史 初宋申錫獲辠【事見上卷五年】宦官益横【横戶孟翻】上外雖包容内不能堪李訓鄭注既得幸揣知上意訓因進講數以微言動上【揣初委翻數所角翻】上見其才辨意訓可與謀大事且以訓注皆因王守澄以進冀宦官不之疑遂密以誠告之訓注遂以誅宦官為己任 【考異曰舊傳以為上出易義以示羣臣之時已與訓有誅宦官之謀按補國史云許康佐進新注春秋列國經傳六十卷上問閽弑吳子餘祭事康佐託以春秋義奥臣窮究未精不敢容易解陳後上以問李仲言仲言乃精為上言之工曰朕左右刑臣多矣餘祭之禍安得不慮仲言曰陛下留意於未萌臣願遵聖謀實録今年四月癸亥許康佐進纂集左氏傳三十卷五月乙巳朔以御集左氏列國經傳三十卷宣付史館然則上與訓謀誅宦官必在此際矣然文宗與訓語時宦官必盈左右恐亦未敢班班顯言如補國史所云也】二人相挾朝夕計議所言於上無不從聲勢炟赫【炟當割翻一作烜况遠翻】注多在禁中或時休沐賓客填門賂遺山積【遺唯季翻】外人但知訓注倚宦官擅作威福不知其與上有密謀也上之立也右領軍將軍興寜仇士良有功【興寜漢龍川縣地江左置興寜縣唐屬循州】王守澄抑之由是有隙訓注為上謀【為于偽翻】進擢士良以分守澄之權五月乙丑以士良為左神策中尉【出韋元素以士良代之】守澄不悦 戊辰以左丞王璠為戶部尚書判度支 京城訛言鄭注為上合金丹【合音閤】須小兒心肝民間驚懼上聞而惡之【惡烏路翻下同】鄭注素惡京兆尹楊虞卿與李訓共搆之云此語出於虞卿家人上怒六月下虞卿御史獄【下戶嫁翻】注求為兩省官中書侍郎同平章事李宗閔不許注毁之於上會宗閔救楊虞卿上怒叱出之壬寅貶明州刺史【明州後漢鄮縣地唐開元二十六年置明州京師東南四千三百里】 左神策中尉韋元素樞密使楊承和王踐言居中用事與王守澄争權不叶李訓鄭注因之出承和於西川元素於淮南踐言於河東皆為監軍秋七月甲辰朔貶楊虞卿䖍州司馬【䖍州漢贑縣晉置南康郡隋為】<br />
<br />
  【䖍州京師東南四千一十七里】 庚戌作紫雲樓於曲江【紫雲樓在曲江之南洊經喪亂頹圯不修今再作之】 辛亥以御史大夫李固言為門下侍郎同平章事李訓鄭注為上畫太平之策【為于偽翻】以為當先除宦官次復河湟次清河北開陳方畧如指諸掌上以為信然寵任日隆初李宗閔為吏部侍郎因駙馬都尉沈結女學士宋若憲知樞密楊承和得為相【宜寄翻宋若憲姊妹皆善屬文德宗召入宫不以妾侍命之呼學士】及貶明州鄭注其事壬子再貶處州長史【代宗大歷十四年改括州為處州京師東南四千二百七十八里】著作郎分司舒元輿與李訓善訓用事召為右司郎中兼侍御史知雜鞫楊虞卿獄【唐制侍御史六人以久次者一人知雜事謂之知雜】癸丑擢為御史中丞元輿元褒之兄也【舒元褒見上卷五年】貶吏部侍郎李漢為汾州刺史刑部侍郎蕭澣為遂州刺史【汾州漢文帝封代王都中都即其地去京師一千二百六里遂州本漢德陽縣之舊壘東晉置遂寜郡後周置遂州去京師二千三百二十九里】皆坐李宗閔之黨是時李訓鄭注連逐三相【三相李德裕路隋李宗閔】威震天下於是生平絲恩髪怨無不報者 李訓奏僧尼猥多耗蠧公私丁巳詔所在試僧尼誦經不中格者皆勒歸俗【中竹仲翻】禁置寺及私度人時人皆言鄭注朝夕且為相侍御史李甘揚言於朝曰白麻出我必壞之於庭【壞音怪】癸亥貶甘封州司馬 【考異曰舊傳曰鄭注入翰林侍講舒元輿既作相注亦求入中書甘昌言於朝云云貶封州按是時元輿未作相舊傳誤也】然李訓亦忌注不欲使為相事竟寢 甲子以國子博士李訓為兵部郎中知制誥依前侍講學士 貶左金吾大將軍沈為邵州刺史八月丙子又貶李宗閔潮州司戶賜宋若憲死 丁丑以太僕卿鄭注為工部尚書充翰林侍講學士注好服鹿裘以隱淪自處【處昌呂翻】上以師友待之注之初得幸上嘗問翰林學士戶部侍郎李珏曰卿知有鄭注乎亦嘗與之言乎對曰臣豈特知其姓名兼深知其為人其人奸邪陛下寵之恐無益聖德臣忝在近密安敢與此人交通戊寅貶珏江州刺史再貶沈柳州司戶【江州京師東南二千九百四十八里】 丙申詔以楊承和庇護宋申錫韋元素王踐言與李宗閔李德裕中外連結受其賂遺【遺唯季翻】承和可驩州安置元素可象州安置踐言可恩州安置令所在錮送【錮送者枷錮而防送之象州至京師四千九百八十九里恩州至京師六千五百里】楊虞卿李漢蕭澣為朋黨之首貶虞卿䖍州司戶漢汾州司馬澣遂州司馬尋遣使追賜承和元素踐言死【韋元素卒如李弘楚之言】時崔潭峻已卒亦剖棺鞭尸己亥以前廬州刺史羅立言為司農少卿立言贓吏以賂結鄭注而得之鄭注之入翰林也中書舍人高元裕草制言以醫藥奉君親注銜之奏元裕嘗出郊送李宗閔壬寅貶元裕閬州刺史【閬州古巴子國秦為閬中縣西魏為隆州唐先天中避諱改閬州至京師一千九百二十五里】元裕士廉之六世孫也【高士廉長孫無忌之舅事高祖太宗】時注與李訓所惡朝士皆指目為二李之黨【惡烏路翻二李謂德裕宗閔】貶逐無虚日班列殆空廷中洶洶上亦知之訓注恐為人所揺九月癸卯朔勸上下詔應與德裕宗閔親舊及門生故吏今日以前貶黜之外餘皆不問人情稍安 鹽鐵使王涯奏改江淮嶺南茶法增其税【德宗貞元九年初税茶於出茶州縣及茶山外商人要路委所由定三等時估每十税一長慶元年鹽鐵使王播奏茶税一百增之五十今又改法而增其税愈重矣】 庚申以鳳翔節度使李聽為忠武節度使代杜悰 憲宗之崩也人皆言宦官陳弘志所為【見二百四十一卷元和十五年】時弘志為山南東道監軍李訓為上謀召之至青泥驛【訓為于偽翻青泥驛在嶢關南】癸亥封杖殺之 【考異曰舊傳曰李訓既秉權衡即謀誅内豎陳弘慶自元和末負弑逆之名遣人封杖决殺按此時李訓未為相今從實錄】 鄭注求為鳳翔節度使門下侍郎同平章事李固言不可丁卯以固言為山南西道節度使 【考異曰宋敏求宣宗實錄曰固言性狷急無重望時訓注用事雖相之中實惡與宗閔為黨乃出為興元節度按固言鍛鍊楊虞卿獄宗閔由是罷相而固言代之豈得為宗閔黨也今從開成紀事】 注為鳳翔節度使 【考異曰開成紀事注引舒元輿李訓俱擢相庭注自詣宰臣李固言求鳳翔節度使固言剛勁不許惟王涯賈餗贊從其事九月二十五日紀事誤今從實錄】李訓雖因注得進及埶位俱盛心頗忌注謀欲中外協埶以誅宦官故出注於鳳翔其實俟既誅宦官并圖注也注欲取名家才望之士為參佐請禮部員外郎韋温為副使【節度副使也】温不可或曰拒之必為患温曰擇禍莫若輕拒之止於遠貶從之有不測之禍卒辭之【卒子恤翻】 戊辰以右神策中尉行右衛上將軍知内侍省事王守澄為左右神策觀軍容使兼十二衛統軍【唐因隋制置十六衛以十二衛統諸府之兵曰左右衛曰左右驍騎衛曰左右武衛曰左右威衛曰左右領軍衛曰左右候衛至開元間府兵之法寖壞乃募彍騎十二萬分隸十二衛每衛萬人其後洊更喪亂十二衛之軍無復承平之舊】李訓鄭注為上謀以虚名尊守澄實奪之權也【為于偽翻下同】 己巳以御史中丞兼刑部侍郎舒元輿為刑部侍郎兵部郎中知制誥充翰林侍講學士李訓為禮部侍郎並同平章事仍命訓二三日一入翰林講易元輿為中丞凡訓注所惡者則為之弹擊【惡烏路翻】由是得為相又上懲李宗閔李德裕多朋黨以賈餗及元輿皆孤寒新進【餗少孤客江淮間元輿地寒不與士齒】故擢為相庶其無黨耳訓起流人期年致位宰相【期讀曰朞】天子傾意任之訓或在中書或在翰林天下事皆决於訓王涯輩承順其風指惟恐不逮自中尉樞密禁衛諸將見訓皆震慴迎拜叩首【慴之涉翻】壬申以刑部郎中兼御史知雜李孝本權知御史中丞孝本宗室之子依訓注得進 李聽自恃勲舊不禮於鄭注注代聽鎮鳳翔先遣牙將丹駿至軍中慰勞【丹姓駿名姓譜丹朱之後勞力到翻】誣奏聽在鎮貪虐冬十月乙亥以聽為太子太保分司復以杜悰為忠武節度使鄭注每自負經濟之畧上問以富人之術注無以對乃請榷茶於是以王涯兼榷茶使【榷古岳翻】涯知不可而不敢違人甚苦之 鄭注欲收僧尼之譽固請罷沙汰從之【是年七月李訓乞沙汰僧尼】 李訓鄭注密言於上請除王守澄辛巳遣中使李好古就第賜酖殺之【好呼到翻】贈楊州大都督訓注本因守澄進【注事見二百二十三卷穆宗長慶三年訓事見上八年】卒謀而殺之【卒子恤翻】人皆快守澄之受佞而疾訓注之隂狡於是元和之逆黨畧盡矣乙酉鄭注赴鎮庚子以東都留守司徒兼侍中裴度兼中書令餘如<br />
<br />
  故李訓所奬拔率皆狂險之士然亦時取天下重望以順人心如裴度令狐楚鄭覃皆累朝耆俊久為當路所軋【朝直遙翻軋乙轄翻】置之散地【散悉但翻】訓皆引居崇秩由是士大夫亦有望其真能致太平者不惟天子惑之也然識者見其横甚【横戶孟翻】知將敗矣 十一月丙午以大理卿郭行餘為邠寜節度使癸丑以河東節度使同平章事李載義兼侍中丁巳以戶部尚書判度支王璠為河東節度使戊午以京兆尹李石為戶部侍郎判度支以京兆少尹羅立言權知府事石神符之五世孫也【襄邑王神符淮安王神通之弟】己未以太府卿韓約為左金吾衛大將軍始鄭注與李訓謀至鎮選壯士數百皆持白棓懷其斧以為親兵【棓蒲項翻白棓猶言白梃也】是月戊辰王守澄葬於滻水【雍録滻水源出藍田縣境之西稍北行至白鹿原西即趨京城王守澄蓋葬於白鹿原西南】注奏請入護葬事因以親兵自隨仍奏令内臣中尉以下盡集滻水送葬注因闔門令親兵斧之使無遺類約既定訓與其黨謀如此事成則注專有其功不若使行餘璠以赴鎮為名多募壯士為部曲并用金吾臺府吏卒先期誅宦者【先悉薦翻】已而并注去之【去羌呂翻】行餘璠立言約及中丞李孝本皆訓素所厚也故列置要地獨與是數人及舒元輿謀之他人皆莫之知也壬戌上御紫宸殿百官班定韓約不報平安【唐制凡朝皇帝既升御座金吾將軍奏左右廂内外平安】奏稱左金吾聽事後石榴夜有甘露臣逓門奏訖【言夜中聞奏禁門已扁於隔門逓入以奏也】因蹈舞再拜宰相亦帥百官稱賀【帥讀曰率下同】訓元輿勸上親往觀之以承天貺上許之百官退班於含元殿【紫宸内殿也含元前殿也上欲往觀甘露故百官自紫宸退而出立班於含元殿以左右金吾仗在含元殿前左右也】日加辰上乘軟輿出紫宸門【軟輿蓋以裀褥積而為之下施掆令人舉之】升含元殿先命宰相及兩省官詣左仗視之良久而還【還音旋又如字】訓奏臣與衆人驗之殆非真甘露未可遽宣布 【考異曰按訓與韓約共謀詐為甘露而自言非真瑞者蓋欲使宦官盡往金吾覆視因伏兵誅之耳故二十二日令狐楚所草制書亦云兇渠仍請共覆視今從實録】恐天下稱賀上曰豈有是邪顧左右中尉仇士良魚志弘帥諸宦者往視之【帥讀曰率】宦者既去訓遽召郭行餘王璠曰來受敕旨璠股栗不敢前獨行餘拜殿下時二人部曲數百皆執兵立丹鳳門外訓已先使入召之令入受敕獨東兵入【河東兵也東上逸河字】邠寜兵竟不至仇士良等至左仗視甘露韓約變色流汗士良怪之曰將軍何為如是俄風吹幕起見執兵者甚衆又聞兵仗聲士良等驚駭走出門者欲閉之士良叱之關不得上【關門牡也上時掌翻下來上同】士良等奔詣上告變訓見之遽呼金吾衛士來上殿衛乘輿者人賞錢百緍宦者曰事急矣請陛下還宫即舉軟輿迎上扶升輿决殿後罘罳疾趨北出【唐宫殿中罘罳以絲為之狀如網以捍燕雀非如漢宫闕之罘罳也今諸宦者能决之而出則可知矣程大昌曰罘罳者鏤木為之其中疏通可以透明或為方空或為連鎻其狀扶疏故曰罘罳讀如浮思猶曰䯱髵也因其形似而想其本狀自可見矣罘罳之名既立於是隨其所施而附著以為之名其在宮闕則為闕上罘罳臣朝於君至闕下復思所奏是也在陵垣則為陵上罘罳王莽斫去陵上罘罳而曰使人無復思漢者是也却而求之上古則禮記疏屏亦其物也疏者刻為雲氣而中空玲瓏也又有網戶刻為連文逓為綴屬其形如網也宋玉曰網戶朱綴刻方連是也既曰刻則是雕木為之其狀如網耳後人因此遂有直織絲網而張之簷窗以護禽雀者文宗甘露之變出殿北門裂斷罘罳而去是真網也此又沿放楚辭而施網焉者也元微之謂承旨時詩曰蘂珠深處少人知網索西臨太液池浴殿曉開天語後步廊騎馬笑相隨自注云網索在太液池上學士候對歇於此予按網索乃是無壁或有窗處以索掛網遮護飛雀故云網索猶掛鈴之索為鈴索也宋元獻喜子京召還為學士詩曰網索軒窗邃鑾坡羽衛重用微之句也若並今世俗語求之則門屏鏤明格子其制與青鎻同類顧所施之地不同而名亦隨異耳】訓攀輿呼曰【呼火故翻】臣奏事未竟陛下不可入宫金吾兵已登殿羅立言帥京兆邏卒三百餘自東來【邏郎佐翻】李孝本帥御史臺從人二百餘自西來【從才用翻】皆登殿縱擊宦官流血呼寃死傷者十餘人乘輿迤邐入宣政門【迤移爾翻邐力爾翻宣政門宣政殿門也】訓攀輿呼益急上叱之宦者郗志榮奮拳毆其胷偃於地【郗丑之翻毆烏口翻偃者偃仰而仆也】乘輿既入門隨闔宦者皆呼萬歲百官駭愕散出訓知事不濟脫從吏緑衫衣之【衣於既翻】走馬而出揚言於道曰我何罪而竄謫人不之疑王涯賈餗舒元輿還中書相謂曰上且開延英召吾屬議之兩省官詣宰相請其故皆曰不知何事諸公各自便士良等知上豫其謀怨憤出不遜語上慙懼不復言士良等命左右神策副使劉泰倫魏仲卿等各帥禁兵五百人露刃出閤門討賊【復扶又翻帥讀曰率】王涯等將會食【諸宰相每日會食於政事堂】吏白有兵自内出逢人輒殺涯等狼狽步走兩省及金吾吏卒千餘人填門争出門尋闔其不得出者六百餘人皆死士良等分兵閉宫門索諸司捕賊黨【索下客翻下同】諸司吏卒及民酤販在中者皆死死者又千餘人横尸流血狼藉塗地諸司印及圖籍帷幕器皿俱盡又遣騎各千餘出城追亡者又遣兵大索城中舒元輿易服单騎出安化門【安化門長安南面西頭第一門】禁兵追擒之王涯徒步至永昌里茶肆禁兵擒入左軍涯時年七十餘被以桎梏掠治不勝苦【被皮義翻桎職日翻梏古沃翻掠音亮治直之翻勝音升】自誣服稱與李訓謀行大逆尊立鄭注王璠歸長興里私第閉門以其兵自防【河東節度之兵也】神策將至門呼曰王涯等謀反欲起尚書為相魚護軍令致意【魚弘志時為左神策護軍中尉將即亮翻】璠喜出見之將趨賀再三【將即亮翻】璠知見紿涕泣而行至左軍見王涯曰二十兄自反胡為見引涯曰五弟昔為京兆尹不漏言於王守澄【王涯第二十王璠第五漏言事見工卷五年】豈有今日邪璠俛首不言又收羅立言於太平里及涯等親屬奴婢皆入兩軍繫之戶部員外郎李元臯訓之再從弟也訓實與之無恩亦執而殺之故嶺南節度使胡証家鉅富【証音正】禁兵利其財託以搜賈餗入其家執其子溵殺之【溵音殷】又入左常侍羅讓詹事渾鐬翰林學士黎埴等家【左常侍左散騎常侍也鐬火外翻】掠其貲財掃地無遺鐬瑊之子也坊市惡少年因之報私仇殺人剽掠百貨【剽匹妙翻】互相攻刼塵埃蔽天癸亥百官入朝【朝直遙翻】日出始開建福門【建福門在大明宫丹鳳門之右】惟聽以從者一人自隨【從才用翻】禁兵露刃夾道至宣政門尚未開時無宰相御史知班百官無復班列【新書儀衛志曰朝日殿上設黼扆躡席熏爐香案御史大夫領屬官至殿西廡從官朱衣傳呼促百官就列文武班於兩觀監察御史二人立於東西朝堂甎道以涖之平明傳點畢内門開監察御史領百官入夾堦監門校尉二人執門籍曰唱籍既視籍曰在入畢而止次門亦如之序班於通乾觀象門南武班居文班之次入宣政門文班自東門而入武班自西門而入至閤門亦如之夾堦校尉十人同唱入畢而止宰相兩省官對班於香案前百官班於殿庭左右廵使二人分涖於鼔鍾樓下先一品班次二品班次三品班次四品班次五品班每班尚書省官為首武班供奉者立於横街之北次千牛中郎將次千牛將軍次過狀中郎將一人次接狀中郎將一人次押柱中郎將一人次排階中郎將一人次押散手仗中郎將一人次左右金吾衛大將軍凡殿中省監少監尚夜尚舍尚輦奉御分左右隨繖扇而立東宫官居上臺之次王府官又次之唯三太三少賓客庶子王傳隨本品侍中奏外辦皇帝步出西序門索扇扇合皇帝升御座扇開左右留扇各三左右金吾將軍一人奏左右廂内外平安通事舍人贊宰相兩省官再拜升殿朝罷皇帝步入東序門觀此可以知甘露之亂蕩無朝儀矣】上御紫宸殿問宰相何為不來仇士良曰王涯等謀反繫獄因以涯手狀呈上召左僕射令狐楚右僕射鄭覃等升殿示之上悲憤不自勝【勝音升】謂楚等曰是涯手書乎對曰是也誠如此罪不容誅因命楚覃留宿中書參决機務使楚草制宣告中外楚叙王涯賈餗反事浮汎【其叙事浮汎蓋以王涯等非實反也】仇士良等不悦由是不得為相時坊市剽掠者猶未止命左右神策將楊鎮靳遂良等各將五百人分屯通衢【靳居焮翻】擊鼓以警之斬十餘人然後定賈餗變服潛民間經宿自知無所逃素服乘驢詣興安門自言我宰相賈餗也為奸人所汚【興安門大明宫南面西來第一門汚烏故翻】可送我詣兩軍門者執送西軍【西軍右神策軍也在大明宫西西内苑中】李孝本改衣緑【衣於既翻】猶服金帶以帽障面單騎奔鳳翔【欲依鄭注也】至咸陽西追擒之甲子以右僕射鄭覃同平章事李訓素與終南僧宗密善往投之宗密欲剃其髪而匿之其徒不可訓出山【剃他計翻山即謂終南山】將奔鳳翔為盩厔鎮遏使宋楚所擒【盩厔音舟窒】械送京師至昆明池訓恐至軍中更受酷辱謂送者曰得我則富貴矣聞禁兵所在搜捕汝必為所奪不若取我首送之送者從之斬其首以來乙丑以戶部侍郎判度支李石同平章事仍判度支前河東節度使李載義復舊任【王璠得罪故載義復舊任】左神策出兵三百人以李訓首引王涯王璠羅立言郭行餘右神策出兵三百人擁賈餗舒元輿李孝本獻於廟社狥於兩市【唐太廟在朱雀街東第一街之東北來第二坊太社在街西第一街之西北來第二坊兩市長安城中東市西市也】命百官臨視腰斬於獨柳之下梟其首於興安門外親屬無問親踈皆死孩穉無遺【穉直利翻】妻女不死者没為官婢百姓觀者怨王涯榷茶或詬詈或投瓦礫擊之【詬許候翻又古侯翻詈力智翻礫郎狄翻】臣光曰論者皆謂涯餗有文學聲名初不知訓注之謀横罹覆族之禍【横戶孟翻】臣獨以為不然夫顛危不扶焉用彼相【論語載孔子之言焉於䖍翻】涯餗安高位飽重禄訓注小人窮奸究險【究極也】力取將相涯餗與之比肩不以為恥國家危殆不以為憂偷合苟容日復一日【復扶又翻】自謂得保身之良策莫我如也若使人人如此而無禍則奸臣孰不願之哉一旦禍生不虞足折刑剭【易曰鼎折足覆公餗其刑剭凶剭音屋剭者誅殺不於市周制誅大臣適甸師謂之剭折而設翻】蓋天誅之也士良安能族之哉<br />
<br />
  王涯有再從弟沐【從才用翻】家於江南老且貧聞涯為相跨驢詣之欲求一簿尉留長安二歲餘始得一見涯待之殊落莫【落冷落也莫薄也落莫唐人常語】久之沐因嬖奴以道所欲【嬖卑義翻又博計翻】涯許以微官自是旦夕造涯之門以俟命【造七到翻】及涯家被收【被皮義翻】沐適在其第與涯俱腰斬舒元輿有族子守謙愿而敏元輿愛之從元輿者十年一旦忽以非罪怒之日加譴責奴婢輩亦薄之守謙不自安求歸江南元輿亦不留守謙悲歎而去夕至昭應聞元輿收族守謙獨免【王沐之并命躁之禍也舒守謙之幸免愿之餘福也禍福之應天豈爽哉】是日以令狐楚為鹽鐵轉運使左散騎常侍張仲方權知京兆尹 【考異曰實錄乙丑閤門使馬元贄已宣授仲方京兆尹至此又言者蓋當時止是口宣至此乃降敕耳】時數日之間殺生除拜皆决於兩中尉 【考異曰皮光業見聞錄曰崔慎由以元和元年登第至開成已入翰林因寓直之夕二更以來有中使宣召引入數重門至一處堂宇華煥簾幕俱垂見左右二廣燃蠟而坐謂慎由曰上不豫來已數日兼自登極後聖政多虧今奉太后中旨命學士草廢立令慎由大驚曰某有中外親族數千口列在搢紳長行兄弟甥姪僅三百人一旦聞此覆族之言寜死不敢承命况聖上高明之德覆於八荒豈可輕議二廣默然無以為對良久啟後戶引慎由至一小殿見文宗坐於殿上二廣徑登階而疏文宗過惡上唯俛首又曰不為此抝木枕措大不合更在此坐矣街談以好抝為抝木枕仍戒慎由曰事泄即是此措大也於是二廣自執炬送慎由出邃殿門復令中使送至本院慎由尋以疾出翰林遂金縢其事付胤故胤切於勦絶北司者由此也誅北司後胤方彰其事新傳曰慎由記其事藏箱枕間將没以授其子胤故胤惡中官終討除之按舊傳崔慎由大中初始入朝為右拾遺員外郎知制誥文宗時未為翰林學士蓋崔胤欲重宦官之罪而誣之新傳承皮錄之誤也】上不豫知初王守澄惡宦者田全操劉行深周元稹薛士幹似先義逸劉英誗等【惡烏路翻似先姓義逸名誗直嚴翻】李訓鄭注因之遣分詣鹽州靈武涇原夏州振武鳳翔廵邉【夏戶雅翻】命翰林學士顧師邕為詔書賜六道使殺之會訓敗六道得詔皆廢不行【六道即謂鹽靈夏涇原振武鳳翔也】丙寅以師邕為矯詔下御史獄【下遐稼翻】先是鄭注將親兵五百已鳳翔至扶風【宋白曰扶風縣本漢美陽縣地今京兆府武功縣北美陽故城是也隋開皇十六年於今岐陽縣置岐山縣武德三年分岐山縣於闈川城置圍川縣貞觀八年改扶風縣九域志鳳翔府東至扶風八十里先悉薦翻】扶風令韓遼知其謀不供具攜印及吏卒奔武功注知訓已敗復還鳳翔仇士良等使人齎密敕授鳳翔監軍張仲清令取注仲清惶惑不知所為押牙李叔和說仲清曰叔和為公以好召注【說式芮翻為于偽翻好如字以好召之言示之以無惡意也】屏其從兵於坐取之【屛必郢翻又畢正翻從才用翻坐徂卧翻】事立定矣仲清從之伏甲以待注注恃其兵衛遂詣仲清叔和稍引其從兵享之於外注獨與數人入既啜茶【啜樞悦翻飲也】叔和抽刀斬注因閉外門悉誅其親兵乃出密敕宣示將士遂滅注家并殺副使錢可復節度判官盧簡能觀察判官蕭傑掌書記盧弘茂等及其枝黨死者千餘人可復徽之子【錢徽見二百四十一卷穆宗長慶元年】簡能綸之子【盧綸與吉仲孚韓翃錢起司空曙苗崔峒耿緯夏侯審李端皆以詩齊名號大歷十才子】傑俛之弟也【蕭俛事憲穆位至宰相史言錢可復等皆名家子以託身非人併命】朝廷未知注死丁卯詔削奪注官爵令鄰道按兵觀變以左神策大將軍陳君奕為鳳翔節度使戊辰夜張仲清遣李叔和等以注首入獻 【考異曰據實錄甲子以傳注首而開成紀事二十六日方下詔削官爵云鄭注初誅京師尚未知李潛用乙卯記亦云丁卯張仲清誘注而殺之與開成紀事同但開成紀事注傳云二十六日奏朝覲恐誤乙卯記注庚申日覲十九日也至扶風聞訓敗乃還似近之實錄恐太在前新本紀云戊辰張仲清殺注今不書日以傳疑】梟於興安門人情稍安京師諸軍始各還營詔將士討賊有功及娖隊者官爵賜賚各有差【娖則角翻】右神策軍獲韓約於崇義坊己巳斬之仇士良等各進階遷官有差自是天下事皆决於北司宰相行文書而已宦官氣益盛迫脅天子下視宰相陵暴朝士如草芥每延英議事士良等動引訓注折宰相【折之舌翻】鄭覃李石曰訓注誠為亂首但不知訓注始因何人得進宦者稍屈搢紳賴之時中書惟有空垣破屋百物皆闕江西湖南獻衣糧百二十分充宰相召募從人【分扶問翻從才用翻下導從同】辛未李石上言宰相若忠正無邪神靈所祐縱遇盗賊亦不能傷若内懷奸罔雖兵衛甚設鬼得而誅之臣願竭赤心以報國止循故事以金吾卒導從足矣【從才用翻】其兩道所獻衣粮並乞停寢從之十二月壬申朔顧師邕流儋州至商山賜死【儋都甘翻儋州漢儋耳郡至京師七千四百四十二里商山即商嶺也所謂繞霤七盤是也貞元七年刺史李西華患此路之險自藍田至内鄉開新道七百餘里迴山取塗人不病涉謂之偏路行旅便之】 榷茶使令狐楚奏罷榷茶從之【王涯誅乃罷榷茶】 度支奏籍鄭注家貲得絹百餘萬匹他物稱是【稱尺證翻】庚辰上問宰相坊市安未李石對曰漸安然比日寒冽特甚【比毗至翻】蓋刑殺太過所致鄭覃曰罪人周親前已皆死【周親孔安國曰周至也】其餘殆不足問時宦官深怨李訓等凡與之有瓜葛親【瓜葛有所附麗言非至親或羣從中表相附麗以叙親好若瓜葛然】或蹔蒙奬引者誅貶不已故二相言之李訓鄭注既誅召六道廵邉使田全操追忿訓注之謀在道揚言我入城凡儒服者無貴賤當盡殺之癸未全操等乘驛疾驅入金光門【金光門長安城西面北來第二門】京城訛言有寇至士民驚譟縱横走【縱子容翻】塵埃四起兩省諸司官聞之皆奔散有不及束帶韈而乘馬者【韈勿翻】鄭覃李石在中書顧吏卒稍稍逃去覃謂石曰耳目頗異宜且出避之石曰宰相位尊望重人心所屬【屬之欲翻】不可輕也今事虚實未可知堅坐鎮之庶幾可定若宰相亦走則中外亂矣且果有禍亂避亦不免覃然之石坐視文案沛然自若敕使相繼傳呼閉皇城諸司門【六典唐都城三重外一重名京城内一重名皇城又内一重名宫城亦名子城】左金吾大將軍陳君賞帥其衆立望仙門下【大明宫城南面五門望仙門在丹鳳門之左帥讀曰率】謂敕使曰賊至閉門未晩請徐觀其變不宜示弱至晡後乃定是日坊市惡少年皆衣緋皂【衣於既翻皂在早翻】持弓刀北望見皇城門閉即欲剽掠非石與君賞鎮之京城幾再亂矣【剽匹妙翻幾居衣翻】時兩省官應入直者皆與其家人辭訣 甲申敕罷修曲江亭館【以鄭注之言而修之注誅乃罷】 丁亥詔逆人親黨自非前已就戮及指名收捕者餘一切不問諸司官雖為所脅從涉於詿誤【詿古賣翻又戶卦翻】皆赦之他人無得相告言及相恐愒見亡匿者勿復追捕【愒許葛翻見賢遍翻復扶又翻】三日内各聽自歸本司時禁軍暴横【横戶孟翻】京兆尹張仲方不敢詰宰相以其不勝任【勝音升】出為華州刺史【華戶化翻】以司農卿薛元賞代之元賞常詣李石第聞石方坐聽事與一人争辯甚喧元賞使覘之【覘丑廉翻】云有神策軍將訴事元賞趨入責石曰相公輔佐天子紀綱四海今近不能制一軍將使無禮如此何以鎮服四夷即趨出上馬命左右擒軍將俟於下馬橋【閣本大明宫圖下馬橋在建福門北】元賞至則已解衣跽之矣【跽其几翻】其黨訴于仇士良士良遣宦者召之曰中尉屈大尹元賞曰屬有公事【屬之欲翻】行當繼至遂杖殺之 【考異曰開成紀事以祕書少監王會為京兆尹按薛元賞已為京兆尹紀事誤】乃白服見士良【白服即待罪之素服】士良曰癡書生何敢杖殺禁軍大將元賞曰中尉大臣也宰相亦大臣也宰相之人若無禮於中尉如之何中尉之人無禮於宰相庸可恕乎中尉與國同體當為國惜法【為于偽翻】元賞已囚服而來惟中尉死生之士良知軍將已死無可如何乃呼酒與元賞歡飲而罷初武元衡之死詔出内庫弓矢陌刀給金吾仗使衛從宰相【事見二百三十九卷憲宗元和十年從才用翻】至建福門而退至是悉罷之<br />
<br />
  開成元年春正月辛丑朔上御宣政殿赦天下改元仇士良請以神策仗衛殿門諫議大夫馮定言其不可【南牙十六衛之兵至此雖名存實亡然以北軍衛南牙則外朝亦將聽命於北司既紊太宗之紀綱又增宦官之勢焰故馮定言其不可】乃止定宿之弟也【馮宿穆宗長慶初知制誥】 二月癸未上與宰相語患四方表奏華而不典李石對曰古人因事為文今人以文害事 昭義節度使劉從諫上表請王涯等罪名且言涯等儒生荷國榮寵【荷下可翻】咸欲保身全族安肯搆逆訓等實欲討除内臣兩中尉自為救死之謀遂致相殺誣以反逆誠恐非辜設若宰相實有異圖當委之有司正其刑典豈有内臣擅領甲兵恣行剽刼延及士庶横被殺傷【剽匹妙翻横戶孟翻】流血千門【漢武帝起建章宫度為千門萬戶後世遂謂宫門為千門】僵尸萬計搜羅枝蔓中外恫疑【恫音通痛也又勅動翻】臣欲身詣闕庭面陳臧否恐并陷孥戮【否音鄙孥音奴子也孥戮戮及子也】事亦無成謹當修飾封疆訓練士卒内為陛下心腹外為陛下藩垣如奸臣難制誓以死清君側丙申加從諫檢校司徒 天德軍奏吐谷渾三千帳詣豐州降 三月壬寅以袁州長史李德裕為滁州刺史【袁州漢宜春縣地隋置袁州京師東南三千五百八十里滁州二千五百六十四里】 左僕射令狐楚從容奏王涯等既伏辜【從千容翻】其家夷滅遺骨棄捐請官為收瘞以順陽和之氣【為于偽翻瘞於計翻月令孟春掩骼埋胔以死氣逆生也】上慘然久之命京兆收葬涯等十一人於城西各賜衣一襲【考異曰開成紀事云京兆尹薛元賞於城西張村葬涯等七人今從新傳】仇士良潛使人之棄骨於渭水 丁未皇城留守郭皎【按舊制車駕行幸則京城置留守今天子在上京而皇城置留守當考觀下奏則知置皇城留守宦官之意也】奏諸司儀仗有鋒刃者請皆輸軍器使【軍器使即軍器庫使内諸司使之一也】遇立仗别給儀刀從之【儀刀以木為之以銀裝之具刀之儀而已】 劉從諫復遣牙將焦楚長上表讓官【讓檢校司徒復扶又翻】稱臣之所陳繫國大體可聽則涯等宜蒙湔洗【湔則前翻】不可聽則賞典不宜妄加安有死寃不申而生者荷禄【荷下可翻】因暴揚仇士良等罪惡辛酉上召見楚長慰諭遣之時士良等恣横【横戶孟翻】朝臣日憂破家及從諫表至士良等憚之由是鄭覃李石粗能秉政【粗坐伍翻】天子倚之亦差以自強 夏四月乙卯以潮州司戶李宗閔為衡州司馬凡李訓指為李德裕宗閔黨者稍收復之 淄王協薨【協憲宗子】 甲午以山南西道節度使李固言為門下侍郎同平章事以左僕射令狐楚代之 戊戌上與宰相從容論詩之工拙【從千容翻】鄭覃曰詩之工者無若三百篇皆國人作之以刺美時政王者采之以觀風俗耳不聞王者為詩也後代辭人之詩華而不實無補於事陳後主隋煬帝皆工於詩不免亡國陛下何取焉【史言鄭覃能守經學以輔其君】覃篤於經術上甚重之 己酉上御紫宸殿宰相因奏事拜謝外間因訛言天子欲令宰相掌禁兵已拜恩矣由是中外復有猜阻【復扶又翻】人情恟忷士民不敢解衣寢者數日乙丑李石奏請召仇士良等面釋其疑上為召士良等出【為于偽翻】上及石等共諭釋之使無疑懼然後事解 閏月乙酉以太子太保分司李聽為河中節度使上嘗歎曰付之兵不疑置之散地不怨【散蘇旱翻】惟聽為可以然 乙未李固言薦崔球為起居舍人鄭覃再三以為不可上曰公事勿相違覃曰若宰相盡同則事必有欺陛下者矣李孝本二女配没右軍【右軍右神策軍】上取之入宫秋七月<br />
<br />
  右拾遺魏謩上疏以為陛下不邇聲色屢出宫女以配鰥夫竊聞數月以來教坊選試以百數莊宅收市猶未巳【唐内諸司有教坊使莊宅使皆宦者為之】又召李孝本女入宫不避宗姓大興物論臣竊惜之昔漢光武一顧列女屏風宋弘猶正色抗言光武即撤之【光武時宋弘為大司空嘗讌見御座新屏風圖畫列女帝數顧視之弘正容言曰未見好德如好色者帝即為撤之笑謂弘曰聞義則服可乎對曰陛下進德臣不勝其喜】陛下豈可不思宋弘之言欲居光武之下乎上即出孝本女 【考異曰實錄上云取孝本女二人入内下魏謩疏云取孝本次女一人入内所以如此不同者蓋孝本二女皆籍沒在右軍先取長女入内謩不之知又取次女謩乃知之上疏故也】擢謩為補闕曰朕選市女子以賜諸王耳憐孝本女髫齓孤露【髫于翻小兒垂髪也齓初觀翻小兒毁齒也】故收養宫中謩於疑似之間皆能盡言可謂愛我不忝厥祖矣命中書優為制辭以賞之謩徵之五世孫也【魏徵以直事大宗】 鄜坊節度使蕭洪詐稱太后弟【事見二百四十三卷太和二年】事覺八月甲辰流驩州於道賜死趙縝呂璋等皆流嶺南【縝止忍翻】初李訓知洪之詐洪懼辟訓兄仲京置幕府先是自神策軍出為節度使者軍中皆資其行裝至鎮三倍償之有自左軍出鎮鄜坊【左軍左神策軍】未償而死者軍中徵之於洪洪恃訓之勢不與又徵於死者之子洪教其子遮宰相自言訓判絶之仇士良由是恨洪太后有異母弟在閩中孱弱不能自逹【孱鉏山翻】有閩人蕭本從之得其内外族諱因士良進逹於上且洪之詐洪由是得罪上以本為真太后弟戊申擢為右贊善大夫 九月丁丑李石為上言宋申錫忠直【為于偽翻】為讒人所誣竄死遐荒未蒙昭雪上俛首久之既而流涕泫然【俛美辨翻俯也泫胡犬翻】曰兹事朕久知其誤奸人逼我以社稷大計兄弟幾不能保【謂漳王湊也幾居衣翻】况申錫僅全腰領耳非獨内臣外廷亦有助之者皆由朕之不明曏使遇漢昭帝必無此寃矣【謂漢昭帝知燕蓋上官之詐也】鄭覃李固言亦共言其寃上深痛恨有慙色庚辰詔悉復申錫官爵以其子慎微為成固尉【成固縣屬興元府】 李石用金部員外郎韓益判度支桉【桉與案同文案也句斷】益坐三千餘緡繫獄石曰臣始以益頗曉錢穀故用之不知其貪乃如是上曰宰相但知人則用有過則懲如此則人易得【易以䜴翻】卿所用人不掩其惡可謂至公從前宰相用人好曲蔽其過不欲人彈劾此大病也冬十一月丁巳貶益梧州司戶【梧州因蒼梧郡而名至京師五千五百里好呼到翻劾戶槩翻又戶得翻】 上自甘露之變意忽忽不樂兩軍毬鞠之會十減六七【樂音洛史炤曰鞠以皮為之今通謂之毬】雖宴享音伎雜遝盈庭【遝逹合翻】未嘗解顔閒居或徘徊眺望【眺它弔翻】或獨語歎息壬午上於延英謂宰相曰朕每與卿等論天下事則不免愁對曰為理者不可以速成【為理猶言為治也】上曰朕每讀書恥為凡主李石曰方今内外之臣其間小人尚多疑阻願陛下更以寛御之彼有公清奉法如劉弘逸薛季稜者陛下亦宜褒賞以勸為善甲申上復謂宰相曰【復扶又翻】我與卿等論天下事有勢未得行者退但飲醇酒求醉耳對曰此皆臣等之罪也 有司以左藏積弊日久【藏但浪翻】請行檢勘且言官典罪在赦前者請宥之上許之既而果得繒帛妄稱漬汚者【漬疾智翻汚烏故翻】敕赦之給事中狄兼謩封還敕書曰官典犯理不可赦上諭之曰有司請檢之初朕既許之矣與其失信寜失罪人卿能奉軄朕甚嘉之 十二月庚戌以華州刺史盧鈞為嶺南節度使李石言於上曰盧鈞除嶺南朝士皆相賀以為嶺南富饒之地近歲皆厚賂北司而得之今北司不撓朝權【撓奴巧翻又奴教翻】陛下亦宜有以褒之庶幾内外奉法此致理之本也上從之鈞至鎮以清惠著名 己未淑王縱薨【縱順宗子】<br />
<br />
  二年春二月己未上謂宰相薦人勿問親踈朕聞竇易直為相【竇易直為相於長慶寶歷】未嘗用親故若親故果才避嫌而棄之是亦不為至公也 均王緯薨【緯順宗子】 三月有彗星出於張【彗祥歲翻又旋芮翻又徐醉翻】長八丈餘【長直亮翻】壬申詔撤樂減膳以一日之膳分充十日 夏四月甲辰上對中書舍人翰林學士兼侍書柳公權於便殿【柳公權先除翰林侍書學士今以翰林學士兼侍書】上舉衫袖示之曰此衣已三澣矣【澣戶管翻】衆皆美上之儉德公權獨無言上問其故對曰陛下貴為天子富有四海當進賢退不肖納諫諍明賞罰乃可以致雍熙服澣濯之衣乃末節耳上曰朕知舍人不應復為諫議【杜佑通典曰中書舍人文士之極任朝廷之盛選諸官莫得比】以卿有諍臣風采須屈卿為之乙巳以公權為諫議大夫餘如故 戊戌以翰林學士工部侍郎陳夷行同平章事 六月河陽軍亂節度使李泳奔懷州軍士焚府署殺泳二子大掠數日乃止泳長安市人寓籍禁軍以賂得方鎮所至恃所交結貪殘不法其下不堪命故作亂丁未貶泳澧州長史【澧州京師東南一千八百九十三里澧音禮】戊申以左金吾將軍李執方為河陽節度使 秋七月癸亥振武奏党項三百餘帳剽掠逃去【剽匹妙翻下同】 給事中韋温為太子侍讀晨詣東宫日中乃得見温諫曰太子當鷄鳴而起問安視膳【記文王之為世子鷄初鳴而衣服至於寢門外問内䜿之御者曰今日安否何如内䜿曰安文王乃喜其有不安節則内䜿以告文王文王色憂行不能正履至於復初然後亦復初食上必在視寒暖之節史炤曰熱食曰饔具食曰膳膳之為言善也】不宜專事晏安太子不能用其言【為太子不令終張本】温乃辭侍讀辛未罷守本官 【考異曰舊傳曰兼太子侍讀每晨至少陽院午見太子温云云太子不能行其言温稱疾上不悦改太常少卿未幾拜給事中按温已為給事中乃兼太子侍讀舊傳誤今從新傳】 振武突厥百五十帳叛剽掠營田戊寅節度使劉沔擊破之 八月庚戌以昭儀王氏為德妃昭容楊氏為賢妃【唐因隋制有貴妃淑妃德妃賢妃各一人為夫人正一品】立敬宗之子休復為梁王執中為襄王言楊為王成美為陳王癸丑立皇子宗儉為蔣王【蔣古國名左傳凡蔣邢茅胙祭】河陽軍士既逐李泳日相扇欲為亂九月李執方索得首亂者七十餘人【索山客翻】悉斬之餘黨分隸外鎮然後定 冬十月國子監石經成【劉昫曰時上好文鄭覃以宰臣判國子祭酒依後漢蔡邕刋碑列於太學創立石壁九經諸儒校正訛謬上又令翰林勒字官校字體又乖師法故石經立後數十年名儒皆不窺之以為蕪累】 福建奏晉江百姓蕭弘稱太后族人【晉冮故晉安郡晉安縣地吳置東安縣晉改曰晉安隋改曰南安開元八年分南安置晉江縣帶泉州】詔御史臺按之 戊申以門下侍郎同平章事李固言同平章事充西川節度使 甲寅御史臺奏蕭弘詐妄詔逓歸鄉里【令所過給食而逓之也】不之罪冀得其真<br />
<br />
  資治通鑑卷二百四十五<br />
<br />
<史部,編年類,資治通鑑>  <br>
   </div> 

<script src="/search/ajaxskft.js"> </script>
 <div class="clear"></div>
<br>
<br>
 <!-- a.d-->

 <!--
<div class="info_share">
</div> 
-->
 <!--info_share--></div>   <!-- end info_content-->
  </div> <!-- end l-->

<div class="r">   <!--r-->



<div class="sidebar"  style="margin-bottom:2px;">

 
<div class="sidebar_title">工具类大全</div>
<div class="sidebar_info">
<strong><a href="http://www.guoxuedashi.com/lsditu/" target="_blank">历史地图</a></strong>  
<a href="http://www.880114.com/" target="_blank">英语宝典</a>  
<a href="http://www.guoxuedashi.com/13jing/" target="_blank">十三经检索</a> 
<br><strong><a href="http://www.guoxuedashi.com/gjtsjc/" target="_blank">古今图书集成</a></strong> 
<a href="http://www.guoxuedashi.com/duilian/" target="_blank">对联大全</a> <strong><a href="http://www.guoxuedashi.com/xiangxingzi/" target="_blank">象形文字典</a></strong> 

<br><a href="http://www.guoxuedashi.com/zixing/yanbian/">字形演变</a>  <strong><a href="http://www.guoxuemi.com/hafo/" target="_blank">哈佛燕京中文善本特藏</a></strong>
<br><strong><a href="http://www.guoxuedashi.com/csfz/" target="_blank">丛书&方志检索器</a></strong> <a href="http://www.guoxuedashi.com/yqjyy/" target="_blank">一切经音义</a>  

<br><strong><a href="http://www.guoxuedashi.com/jiapu/" target="_blank">家谱族谱查询</a></strong>  <strong><a href="http://shufa.guoxuedashi.com/sfzitie/" target="_blank">书法字帖欣赏</a></strong> 
<br>

</div>
</div>


<div class="sidebar" style="margin-bottom:0px;">

<font style="font-size:22px;line-height:32px">QQ交流群9:489193090</font>


<div class="sidebar_title">手机APP 扫描或点击</div>
<div class="sidebar_info">
<table>
<tr>
	<td width=160><a href="http://m.guoxuedashi.com/app/" target="_blank"><img src="/img/gxds-sj.png" width="140"  border="0" alt="国学大师手机版"></a></td>
	<td>
<a href="http://www.guoxuedashi.com/download/" target="_blank">app软件下载专区</a><br>
<a href="http://www.guoxuedashi.com/download/gxds.php" target="_blank">《国学大师》下载</a><br>
<a href="http://www.guoxuedashi.com/download/kxzd.php" target="_blank">《汉字宝典》下载</a><br>
<a href="http://www.guoxuedashi.com/download/scqbd.php" target="_blank">《诗词曲宝典》下载</a><br>
<a href="http://www.guoxuedashi.com/SiKuQuanShu/skqs.php" target="_blank">《四库全书》下载</a><br>
</td>
</tr>
</table>

</div>
</div>


<div class="sidebar2">
<center>


</center>
</div>

<div class="sidebar"  style="margin-bottom:2px;">
<div class="sidebar_title">网站使用教程</div>
<div class="sidebar_info">
<a href="http://www.guoxuedashi.com/help/gjsearch.php" target="_blank">如何在国学大师网下载古籍?</a><br>
<a href="http://www.guoxuedashi.com/zidian/bujian/bjjc.php" target="_blank">如何使用部件查字法快速查字?</a><br>
<a href="http://www.guoxuedashi.com/search/sjc.php" target="_blank">如何在指定的书籍中全文检索?</a><br>
<a href="http://www.guoxuedashi.com/search/skjc.php" target="_blank">如何找到一句话在《四库全书》哪一页?</a><br>
</div>
</div>


<div class="sidebar">
<div class="sidebar_title">热门书籍</div>
<div class="sidebar_info">
<a href="/so.php?sokey=%E8%B5%84%E6%B2%BB%E9%80%9A%E9%89%B4&kt=1">资治通鉴</a> <a href="/24shi/"><strong>二十四史</strong></a>&nbsp; <a href="/a2694/">野史</a>&nbsp; <a href="/SiKuQuanShu/"><strong>四库全书</strong></a>&nbsp;<a href="http://www.guoxuedashi.com/SiKuQuanShu/fanti/">繁体</a>
<br><a href="/so.php?sokey=%E7%BA%A2%E6%A5%BC%E6%A2%A6&kt=1">红楼梦</a> <a href="/a/1858x/">三国演义</a> <a href="/a/1038k/">水浒传</a> <a href="/a/1046t/">西游记</a> <a href="/a/1914o/">封神演义</a>
<br>
<a href="http://www.guoxuedashi.com/so.php?sokeygx=%E4%B8%87%E6%9C%89%E6%96%87%E5%BA%93&submit=&kt=1">万有文库</a> <a href="/a/780t/">古文观止</a> <a href="/a/1024l/">文心雕龙</a> <a href="/a/1704n/">全唐诗</a> <a href="/a/1705h/">全宋词</a>
<br><a href="http://www.guoxuedashi.com/so.php?sokeygx=%E7%99%BE%E8%A1%B2%E6%9C%AC%E4%BA%8C%E5%8D%81%E5%9B%9B%E5%8F%B2&submit=&kt=1"><strong>百衲本二十四史</strong></a>  <a href="http://www.guoxuedashi.com/so.php?sokeygx=%E5%8F%A4%E4%BB%8A%E5%9B%BE%E4%B9%A6%E9%9B%86%E6%88%90&submit=&kt=1"><strong>古今图书集成</strong></a>
<br>

<a href="http://www.guoxuedashi.com/so.php?sokeygx=%E4%B8%9B%E4%B9%A6%E9%9B%86%E6%88%90&submit=&kt=1">丛书集成</a> 
<a href="http://www.guoxuedashi.com/so.php?sokeygx=%E5%9B%9B%E9%83%A8%E4%B8%9B%E5%88%8A&submit=&kt=1"><strong>四部丛刊</strong></a>  
<a href="http://www.guoxuedashi.com/so.php?sokeygx=%E8%AF%B4%E6%96%87%E8%A7%A3%E5%AD%97&submit=&kt=1">說文解字</a> <a href="http://www.guoxuedashi.com/so.php?sokeygx=%E5%85%A8%E4%B8%8A%E5%8F%A4&submit=&kt=1">三国六朝文</a>
<br><a href="http://www.guoxuedashi.com/so.php?sokeytm=%E6%97%A5%E6%9C%AC%E5%86%85%E9%98%81%E6%96%87%E5%BA%93&submit=&kt=1"><strong>日本内阁文库</strong></a> <a href="http://www.guoxuedashi.com/so.php?sokeytm=%E5%9B%BD%E5%9B%BE%E6%96%B9%E5%BF%97%E5%90%88%E9%9B%86&ka=100&submit=">国图方志合集</a> <a href="http://www.guoxuedashi.com/so.php?sokeytm=%E5%90%84%E5%9C%B0%E6%96%B9%E5%BF%97&submit=&kt=1"><strong>各地方志</strong></a>

</div>
</div>


<div class="sidebar2">
<center>

</center>
</div>
<div class="sidebar greenbar">
<div class="sidebar_title green">四库全书</div>
<div class="sidebar_info">

《四库全书》是中国古代最大的丛书,编撰于乾隆年间,由纪昀等360多位高官、学者编撰,3800多人抄写,费时十三年编成。丛书分经、史、子、集四部,故名四库。共有3500多种书,7.9万卷,3.6万册,约8亿字,基本上囊括了古代所有图书,故称“全书”。<a href="http://www.guoxuedashi.com/SiKuQuanShu/">详细>>
</a>

</div> 
</div>

</div>  <!--end r-->

</div>
<!-- 内容区END --> 

<!-- 页脚开始 -->
<div class="shh">

</div>

<div class="w1180" style="margin-top:8px;">
<center><script src="http://www.guoxuedashi.com/img/plus.php?id=3"></script></center>
</div>
<div class="w1180 foot">
<a href="/b/thanks.php">特别致谢</a> | <a href="javascript:window.external.AddFavorite(document.location.href,document.title);">收藏本站</a> | <a href="#">欢迎投稿</a> | <a href="http://www.guoxuedashi.com/forum/">意见建议</a> | <a href="http://www.guoxuemi.com/">国学迷</a> | <a href="http://www.shuowen.net/">说文网</a><script language="javascript" type="text/javascript" src="https://js.users.51.la/17753172.js"></script><br />
  Copyright &copy; 国学大师 古典图书集成 All Rights Reserved.<br>
  
  <span style="font-size:14px">免责声明:本站非营利性站点,以方便网友为主,仅供学习研究。<br>内容由热心网友提供和网上收集,不保留版权。若侵犯了您的权益,来信即刪。scp168@qq.com</span>
  <br />
ICP证:<a href="http://www.beian.miit.gov.cn/" target="_blank">鲁ICP备19060063号</a></div>
<!-- 页脚END --> 
<script src="http://www.guoxuedashi.com/img/plus.php?id=22"></script>
<script src="http://www.guoxuedashi.com/img/tongji.js"></script>

</body>
</html>
