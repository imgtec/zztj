資治通鑑卷一百八十六
宋 司馬光 撰

胡三省 音註

唐紀二|{
	起著雍攝提格八月盡十二月不滿一年}


高祖神堯大聖光孝皇帝上之中

武德元年八月薛舉遣其子仁果進圍寜州|{
	西魏置寧州於定安置豳州於新平隋志并定安新平二縣皆屬北地郡大業初慶新平之豳州改定安之寧州為州唐初折北地之新平三水置州而以北地郡為寜州治定安}
刺史胡演撃却之郝瑗言於舉曰|{
	郝呼各翻瑗于眷翻}
今唐兵新破關中騷動宜乘勝直取長安舉然之會有疾而止辛巳舉卒|{
	卒子恤翻}
太子仁果立居於折墌城|{
	新志涇州保定縣有折墌故城折杜佑作析音思歷翻墌章恕翻}
諡舉曰武帝 上欲與李軌共圖秦隴|{
	薛舉父子時據秦隴}
遣使濳詣凉州|{
	復武威郡為凉州宋白曰凉州之地本月氐居之後為匈奴右地漢武帝置凉州兼統河隴之地而河西之地列置武威酒泉敦煌張掖四郡東都之季河西諸郡以去州隔遠自求立州為立雍州晉惠帝末張軌為凉州刺史治姑臧為會府後分置諸州而武威始專凉州之名使疏吏翻下同}
招撫之與之書謂之從弟|{
	從才用翻}
軌大喜遣其弟懋入貢上以懋為大將軍命鴻臚少卿張俟德冊拜軌為凉州總管封凉王|{
	臚陵如翻少詩照翻}
初朝廷以安陽令呂珉為相州刺史更以相州刺史王德仁為巖州刺史|{
	是年五月王德仁來降先受朝命德仁未能有相州也六月呂珉以相州來降故正投之新志以林慮縣置巖州正德仁所據地朝直遥翻相息亮翻}
德仁由是怨憤甲申誘山東大使宇文明逹入林慮山而殺之|{
	誘音酉慮音廬}
叛歸王世充 己丑以秦王世民為元帥|{
	帥所類翻}
撃薛仁果 丁酉臨洮等四郡來降|{
	後周武帝逐吐谷渾以置洮陽郡尋置九州大業初改州為臨九郡洮土刀翻}
隋江郡太守陳稜求得煬帝之柩取宇文化及所留輦路鼓吹粗備天子儀衛|{
	守式乂翻柩音舊吹昌瑞翻粗坐五翻}
改葬於江都宫西吳公臺下|{
	今揚州城西北有雷塘塘西有吳公臺相傳以為陳吳明徹攻廣陵所築弩臺以射城中}
其王公以下皆列瘞於帝塋之側|{
	瘞於計翻塋音營}
宇文化及之發江都也|{
	是年四月化及發江都}
以杜伏威為歷陽太守|{
	義寧元年春伏威據歷陽}
伏威不受仍上表於隋皇泰主拜伏威為東道大總管封楚王沈法興亦上表於皇泰主自稱大司馬録尚書事天門公|{
	上時掌翻}
承制置百官以陳杲仁為司徒|{
	新書竹陳果仁}
孫士漢為司空蔣元超為左僕射殷芊為左丞徐令言為右丞劉子翼為選部侍郎李百藥為府椽百藥德林之子也|{
	李德林歷事齊周隋選宣絹翻椽於絹翻}
九月隋襄國通守陳君賓來降拜邢州刺史|{
	復以襄國郡為邢州宋白曰邢州禹貢衡漳之地春秋邢侯之國邢遷於夷儀即其地秦兼天下於此置信都郡項羽改曰襄國盖以趙襄子謚名之也石氏置襄國郡隋置邢州取古邢國爲名守式又翻降戶江翻}
君賓伯山之子也|{
	伯山陳文帝之子}
虞州刺史韋義節|{
	義寧元年以安邑虞郷夏三縣置安邑郡武德元年曰虞州}
攻隋河東通守堯君素久不下軍數不利|{
	數所角翻}
壬子以工部尚書獨孤懷恩代之 初李密既殺翟讓|{
	見一百八十四卷義寧元年十一月翟萇伯翻}
頗自驕矜不恤士衆倉粟雖多無府庫錢帛戰士有功無以為賞又厚撫初附之人衆心頗怨徐世勣嘗因宴會刺譏其短密不懌使世勣出鎮黎陽雖名委任實亦踈之|{
	此叙密致敗之由非一時之事}
密開洛口倉散米無防守典當者|{
	當主當也當丁浪翻}
又無文劵取之者隨意多少或離倉之後|{
	離力智翻}
力不能致委棄衢路自倉城至郭門|{
	郭郛郭也}
米厚數寸|{
	厚戶豆翻}
為車馬所轥踐|{
	轥良刃翻踐慈演翻}
羣盗來就食者并家屬近百萬口|{
	近其靳翻}
無甕盎織荆筐淘米洛水十里兩岸之間望之皆如白沙密喜謂賈閏甫曰此可謂足食矣閏甫對曰國以民為本民以食為天今民所以襁負如流而至者以所天在此故也|{
	襁居兩翻}
而有司曾無愛吝屑越如此|{
	吝惜也屑越猶言狼籍而弃之也荀子曰貨財粟米者彼將日月棲遲薛越之中野我今將畜積并聚之於倉廩}
竊恐一旦米盡民散明公孰與成大業哉密謝之即以閏甫判司倉參軍事密以東都兵數敗微弱而將相自相屠滅謂旦夕可平王世充既專大權厚賞將士繕治器械亦隂圖取密時隋軍乏食而密軍少衣|{
	數所角翻將即亮翻治直之翻少詩沼翻}
世充請交易密難之長史邴元真等各求私利|{
	邴即古丙姓長知兩翻}
勸密許之先是東都人歸密者日以百數|{
	先悉薦翻}
既得食降者益少密悔而止密破宇文化及還|{
	還從宣翻}
其勁卒良馬多死士卒疲病世充欲乘其弊撃之恐人心不壹乃詐稱左軍衛士張永通三夢周公令宣意於世充當勒兵相助擊賊乃為周公立廟|{
	周公作洛世充假之以作士氣令力丁翻下同為於偽翻}
每出兵輒先祈禱世充令巫宣言周公欲令僕射急討李密當有大功不即兵皆疫死|{
	不讀曰否}
世充兵多楚人信妖言皆請戰|{
	妖於驕翻}
世充簡練精鋭得二萬餘人馬二千餘匹壬子出師撃密旗幡之上皆書永通字軍容甚盛|{
	以張永通宣周公之意故旗幡書永通字以表神助}
癸丑至偃師營於通濟渠南作三橋於渠上|{
	通濟渠大業元年所開}
密留王伯當守金墉自引精兵出偃師阻邙山以待之密召諸將會議|{
	將即亮翻}
裴仁基曰世充悉衆而至洛下必虚可分兵守其要路令不得東簡精兵三萬傍河西出以逼東都|{
	傍步浪翻}
世充還|{
	還從宣翻}
我且按甲世充再出我又逼之如此則我有餘力彼勞奔命破之必矣密言公言大善今東都兵有三不可當兵仗精鋭一也决計深入二也食盡求戰三也我但乘城固守畜力以待之彼欲鬬不得求走無路不過十日世充之頭可致麾下陳智略樊文超單雄信皆曰計世充戰卒甚少屢經摧破悉已喪膽|{
	少詩沼翻喪息浪翻}
兵法曰倍則戰况不啻倍哉且江淮新附之士望因此機展其勲效及其鋒而用之可以得志於是諸將諠然欲戰者什七八密惑於衆議而從之|{
	將即亮翻}
仁基苦爭不能得撃地歎曰公後必悔之魏徵言於長史鄭頲曰魏公雖驟勝而驍將鋭卒多死|{
	長知兩翻頲它鼎翻驍堅堯翻將即亮翻下同}
戰士心怠此二者難以應敵且世充乏食志在死戰難與争鋒未若深溝高壘以拒之不過旬月世充糧盡必自退追而撃之蔑不勝矣頲曰此老生之常談耳徵曰此乃奇策何謂常談拂衣而起程知節將内馬軍與密同營在北邙山上單雄信將外馬軍營於偃師城北|{
	單慈淺翻}
世充遣數百騎度通濟渠攻雄信營密遣裴行儼與知節助之行儼先馳赴敵中流矢墜於地|{
	騎奇寄翻下同中竹仲翻}
知節救之殺數人世充軍披靡|{
	披普彼翻}
乃抱行儼重騎而還|{
	重直龍翻二人共騎一馬曰重騎還從宣翻又如字}
為世充騎所逐刺槊洞過知節迴身捩折其槊|{
	刺七亦翻槊色角翻捩練結翻拗捩也折而設翻}
兼斬追者與行儼俱免會日暮各歛兵還營密騎將孫長樂等千餘人皆被重創|{
	驍堅堯翻樂音洛被皮義翻創初良翻}
密新破宇文化及有輕世充之心不設壁壘世充夜遣二百餘騎潜入北山|{
	北山即北邙山}
伏谿谷中命軍士皆秣馬蓐食甲寅旦將戰世充誓衆曰今日之戰非直爭勝負死生之分在此一舉若其捷也富貴固所不論若其不捷必無一人獲免所爭者死非獨為國|{
	為于偽翻}
各宜免之遲明|{
	遲直二翻}
引兵薄密密出兵應之未及成列世充縱兵撃之世充士卒皆江淮剽勇|{
	剽匹妙翻}
出入如飛世充先索得一人貌類密者縳而匿之|{
	索山客翻}
戰方酣使牽以過陳前|{
	陳讀曰陣}
譟曰已獲李密矣 |{
	考異曰革命記曰世充先於衆中覔得一人眉目狀似李密者隂畜之而不令出師至偃師城下與李密未大相接遽令數十騎馳將所畜人頭來云殺得李密充佯不信遣衆共看咸言是密頭也遂於城下勒兵擲頭與城中人城中人亦言是密頭也遂以城降今從壺關録}
士卒皆呼萬歲其伏兵發乘高而下馳壓密營縱火焚其廬舍密衆大潰其將張童仁陳智略皆降|{
	壓於甲翻將即亮翻降戶江翻下同}
密與萬餘人馳向洛口世充夜圍偃師鄭頲守偃師其部下翻城納世充初世充家屬在江都隨宇文化及至滑臺又隨王軌入李密密留於偃師欲以招世充及偃師破世充得其兄世偉子玄應䖍恕瓊等又獲密將佐裴仁基鄭頲祖君彦等數十人世充於是整兵向洛口得邴元真妻子鄭䖍象母及密諸將子弟皆撫慰之令濳呼其父兄|{
	令力丁翻}
初邴元真為縣吏坐贓亡命從翟讓於瓦岡|{
	翟萇伯翻}
讓以其嘗為吏使掌書記及密開幕府妙選時英讓薦元真為長史密不得已用之|{
	此義寜元年春二月事長知兩翻}
行軍謀畫未嘗參預密西拒世充留元真守洛口倉元真性貪鄙宇文温謂密曰不殺元真必為公患密不應元真知之隂謀叛密楊慶聞之以告密密固疑焉至是密將入洛口城元真已遣人濳引世充矣密知而不發因與衆謀待世充兵半濟洛水然後擊之世充軍至密候騎不時覺比將出戰|{
	比必寐翻}
世充軍悉已濟矣單雄信等又勒兵自據密自度不能支|{
	度徒洛翻}
帥麾下輕騎奔虎牢元真遂以城降|{
	降戶江翻}
初雄信驍捷善用馬槊名冠諸軍|{
	帥讀曰率下同冠古玩翻}
軍中號曰飛將|{
	將即亮翻}
彦藻以雄信輕於去就勸密除之|{
	彦藻房彦藻也是年二月彦藻死此亦叙日前事}
密愛其才不忍也及密失利雄信遂以所部降世充|{
	史叙邴元眞單雄信事皆言李密當斷不斷反受其亂}
密將如黎陽或曰殺翟讓之際徐世勣幾死|{
	事見一百八十四卷義寜元年十一月幾居希翻}
今失利而就之安可保乎時王伯當棄金墉保河陽密自虎牢歸之引諸將共議密欲南阻河北守太行東連黎陽以圖進取|{
	將即亮翻行戶剛翻}
諸將皆曰今兵新失利衆心危懼若更停留恐叛亡不日而盡又人情不願難以成功密曰孤所恃者衆也衆既不願孤道窮矣欲自刎以謝衆|{
	刎扶粉翻}
伯當抱密號絶|{
	號戶刀翻}
衆皆悲泣密復曰|{
	復扶又翻}
諸君幸不相棄當共歸關中密身雖無功諸君當保富貴府掾柳爕曰明公與唐公同族兼有疇昔之好|{
	謂自唐公起與之連和也掾于絹翻好呼到翻}
雖不陪起兵然阻東都斷隋歸路|{
	斷丁管翻}
使唐公不戰而據長安此亦公之功也衆咸曰然密又謂王伯當曰將軍室家重大豈復與孤俱行哉伯當曰昔蕭何盡帥子弟以從漢王|{
	漢王與項羽相拒蕭何悉遣子弟詣軍天下既定論功行封上曰何舉宗數十人隨我復扶又翻下同}
伯當恨不兄弟俱從|{
	從才用翻}
豈以公今日失利遂輕去就乎縱身分原野亦所甘心左右莫不感激從密入關者凡二萬人於是密之將帥州縣多降於隋朱粲亦遣使降隋|{
	將即亮翻帥所類翻降戶江翻使疏吏翻}
皇泰主以粲為楚王甲寅秦州總管竇軌撃薛仁果不利驃騎將軍劉感鎮涇州|{
	宋白曰魏黄初中分隴右為秦州因秦初封也與州同理冀城冀城改為隴城縣時復以隴西郡為秦州安定郡為涇州驃匹妙翻騎奇寄翻}
仁果圍之城中糧盡感殺所乘馬以分將士感一無所噉唯煮馬骨取汁和木屑食之城垂䧟者數矣|{
	數所角翻}
會長平王叔良將士至涇州|{
	士當作兵}
仁果乃揚言食盡引兵南去乙卯又遣高墌人偽以城降|{
	墌章恕翻 考異曰寶録云乙卯宇文歆攻高墌城下之今從劉感傳}
叔良遣感帥衆赴之|{
	帥讀曰率下同}
己未至城下扣城中人曰賊已去可踰城入感命燒其門城上下水灌之感知其詐遣步兵先還自帥精兵為殿|{
	帥讀曰率殿丁練翻}
俄而城上舉三烽仁果兵自南原大下戰於百里細川唐軍大敗感為仁果所擒仁果復圍涇州令感語城中云|{
	語牛倨翻}
援軍已敗不如早降|{
	降戶江翻}
感許之至城下大呼曰|{
	呼火故翻}
逆賊饑餒亡在旦夕秦王帥數十萬衆四面俱集城中勿憂勉之仁果怒執感於城旁埋之至膝馳騎射之至死|{
	帥讀曰率騎奇寄翻射而亦翻}
聲色逾厲叔良嬰城固守僅能自全感豐生之孫也|{
	劉豐生高齊將死於穎川}
庚申隴州刺史陜人常逹|{
	隋志扶風汧源縣西魏之東秦州也後改為隴州大業三年廢州併入扶風郡義寧二年析扶風郡之汧源汧陽南由安定郡之華亭置隴東郡唐受禪改為隴州陜失冉翻}
擊薛仁果於宜祿川|{
	宜祿川在涇二州間貞觀二年析之新平及涇之保定靈臺置宜祿縣}
斬首千餘級 上遣從子襄武公琛|{
	從才用翻琛丑林翻}
太常卿鄭元璹|{
	璹殊玉翻}
以女妓遺始畢可汗|{
	妓渠綺翻遺于季翻可從刋入聲汗音寒}
壬戌始畢復遣骨咄祿特勒來|{
	上受禪之後骨咄祿嘗來使復扶又翻咄當没翻}
癸亥白馬道士傅仁均|{
	白馬縣帶滑州}
造戊寅歷成|{
	唐受禪建國歲在戊寅故以名歷}
奏上行之|{
	上時掌翻}
薛仁果屢攻常逹不能克乃遣其將仵士政以數百人詐降|{
	作姓也姓苑仵姓望出襄陽急就章有仵終古將即亮翻仵音疑古翻降戶江翻下同}
逹厚撫之乙丑士政伺隙以其徒劫逹擁城中二千人降於仁果|{
	伺相吏翻 考異曰新舊唐書皆云薛舉遣仵士政偽降逹士政劫逹以見舉據實錄薛舉前已死此月逹再擊仁果及士政劫逹皆有日月今從實錄}
逹見仁果詞色不屈仁果壯而釋之奴賊帥張貴謂逹曰汝識我乎|{
	帥所類翻}
逹曰汝逃死奴賊耳貴怒欲殺之人救之得免 辛未追諡隋太上皇為煬帝|{
	諡神至翻}
宇文化及至魏縣張愷等謀去之事覺化及殺之腹心稍盡兵勢日蹙兄弟更無他計但相聚酣宴奏女樂化及醉尤智及曰|{
	酣戶甘翻尤責過也}
我初不知由汝為計強來立我|{
	智及創謀弑逆故尤之強其兩翻}
今所向無成士馬日散負弑君之名天下所不容今者滅族豈不由汝乎持其兩子而泣智及怒曰事捷之日初不賜尤及其將敗乃欲歸罪何不殺我以降竇建德數相鬭鬩|{
	降戶江翻數所角翻鬩許激翻恨也戾也}
言無長幼醒而復飲以此為恒|{
	恒常也長知兩翻復扶又翻恒戶登翻}
其衆多亡化及自知必敗嘆曰人生固當死豈不一日為帝乎於是鴆殺秦王浩即皇帝位於魏縣國號許|{
	宇文化及襲封許公因以為國號}
改元天壽署置百官 冬十月壬申朔日有食之 戊寅宴突厥骨咄祿引骨咄祿升御坐以寵之|{
	史言突厥驕倨唐祖欲以結其心適以滋其慢厥九勿翻咄當没翻坐徂卧翻}
李密將至上遣使迎勞相望於道|{
	使疏吏翻下同勞力到翻}
密大喜謂其徒曰我擁衆百萬一朝解甲歸唐山東連城數百知我在此遣使招之亦當盡至比於竇融功亦不細|{
	竇融以河西歸漢光武李密自謂過之}
豈不以一台司見處乎|{
	處昌呂翻}
己卯至長安有司供待稍薄所部兵累日不得食衆心頗怨既而以密為光祿卿上柱國賜爵邢國公密既不滿望朝臣又多輕之|{
	朝直遥翻}
執政者或來求賄意甚不平|{
	為李密復叛去張本}
獨上親禮之常呼為弟以舅子獨孤氏妻之|{
	妻七細翻}
庚辰詔右翊衛大將軍淮安王神通為山東道安撫大使山東諸軍並受節度以黄門侍郎崔民幹為副|{
	崔民幹山東望族故使副神通以招撫諸郡縣使疏吏翻}
鄧州刺史呂子臧與撫慰使馬元規擊朱粲破之子臧言於元規曰粲新敗上下危懼請併力擊之一舉可滅若復遷延|{
	復扶又翻下同}
其徒稍集力彊食盡致死於我為患方深元規不從子臧請獨以所部兵擊之元規不許既而粲收集餘衆兵復大振自稱楚帝於冠軍|{
	復扶又翻又音如字冠古玩翻}
改元昌逹進攻鄧州子臧撫膺|{
	膺胷也}
謂元規曰老夫今坐公死矣粲圍南陽|{
	南陽即鄧州}
會霖雨城壞所親勸子臧降子臧曰安有天子方伯降賊者乎帥麾下赴敵而死|{
	降戶江翻帥讀曰率}
俄而城陷元規亦死|{
	是年二月遣馬元規今死}
癸未王世充收李密美人珍寶及將卒十餘萬人還東都陳於闕下|{
	將即亮翻}
乙酉皇泰主大赦丙戌以世充為太尉尚書令内外諸軍事|{
	内外諸軍事之上當有總督二字}
仍使之開太尉府備置官屬妙選人物|{
	史言王世充簒形已成}
世充以裴仁基父子驍勇深禮之|{
	驍堅堯翻}
徐文遠復入東都見世充必先拜或問曰君倨見李密|{
	見上七月}
而敬王公何也文遠曰魏公君子也能容賢士王公小人也能殺故人吾何敢不拜 李密總管李育德以武陟來降|{
	隋志河内郡修武縣開皇十六年折置武陟郡劉昫曰武陟漢懷縣地故城在今縣西降戶江翻下同 考異曰舊唐書高季輔傳云與李厚德來降按以武陟來降者乃育德非厚德也}
拜陟州刺史|{
	新舊志皆云武德二年李厚德以修武縣東北濁鹿城歸順因置陟州通鑑二年書厚德逐王世充殷州刺史以獲嘉來降以厚德刺殷州二志皆云四年置殷州差殊如此當考}
育德諤之孫也|{
	李諤見一百七十六卷陳長城公至德三年}
其餘將佐劉德威賈閏甫高季輔等或以城邑或帥衆相繼來降|{
	將即亮翻帥讀曰率}
初北海賊帥綦公順|{
	大業初以青州為北海郡綦姓也帥所類翻綦音其}
帥其徒三萬攻郡城已克其外郭進攻子城城中食盡公順自謂克在旦夕不為備明經劉蘭成糾合城中驍健百餘人襲擊之|{
	劉蘭成蓋嘗應明經科因稱之新唐志曰唐制取士之科多因隋舊則明經科起於隋也帥讀曰率驍堅堯翻下同}
城中見兵繼之|{
	見賢遍翻}
公順大敗棄營走郡城獲全於是郡官及望族分城中民為六軍各將之蘭成亦將一軍|{
	將即亮翻}
有宋書佐者離間諸軍曰|{
	煬帝改郡諸曹參軍為書佐間古莧翻}
蘭成得衆心必為諸人不利不如殺之衆不忍殺但奪其兵以授宋書佐蘭成恐終及禍亡奔公順公順軍中喜譟欲奉以為主固辭乃以為長史軍事咸聽焉居五十餘日蘭成簡軍中驍健者百五十人往抄北海|{
	抄楚交翻下同}
距城四十里留十人使多芟草分為百餘積|{
	芟所銜翻}
二十里又留二十人各執大旗五六里又留三十人伏險要蘭成自將十人夜距城一里許潛伏餘八十人分置便處約聞鼓聲即抄取人畜亟去|{
	抄楚交翻亟紀力翻}
仍一時焚積草明晨城中遠望無烟塵皆出樵牧日向中蘭成以十人直抵城門城上鉦鼓亂發伏兵四出抄掠雜畜千餘頭及樵牧者而去|{
	畜許又翻}
蘭成度抄者已遠徐步而還|{
	度徒洛翻還音旋又如字}
城中雖出兵恐有伏兵不敢急追又見前有旌旗烟火遂不敢進而還既而城中知蘭成前者衆少悔不窮追|{
	少詩沼翻下同}
居月餘蘭成謀取郡城更以二十人直抵城門城中人競出逐之行未十里公順將大兵總至|{
	將即亮翻下自將主將同}
郡兵奔馳還城公順進兵圍之蘭成一言招諭城中人爭出降|{
	降戶江翻}
蘭成撫存老幼禮遇郡官見宋書佐亦禮之如舊仍資送出境内外安堵時海陵賊帥臧君相|{
	隋志海陵縣屬江都郡帥所類翻相息亮翻}
聞公順據北海帥其衆五萬來爭之|{
	帥讀曰率}
公順衆少聞之大懼蘭成為公順畫策曰|{
	為于偽翻}
君相今去此尚遠必不為備請將軍倍道襲擊其營公順從之自將驍勇五千人齎熟食倍道襲之|{
	驍堅堯翻}
將至蘭成與敢死士二十人前行距君相營五十里見其抄者負擔向營|{
	擔丁濫翻}
蘭成亦與其徒負擔蔬米燒器|{
	擔亦負也都甘翻燒器鍋釜之屬}
詐為抄者擇空而行聽察得其號|{
	號戶告翻軍號也}
及主將姓名至暮與賊比肩而入負擔巡營知其虚實得其更號|{
	更號持更之號更工衡翻}
乃於空地燃火營食至三鼓忽於主將幕前交刀亂下殺百餘人賊衆驚擾公順兵亦至急攻之君相僅以身免 |{
	考異曰舊書作劉蘭云頗涉經史善言成敗然性多兇狡見隋末將亂交通不逞于時北海完富蘭利其子女玉帛與羣盜相應破其鄉城邑武德中淮安王神通為山東道安撫大使蘭率宗黨歸之革命記序其事頗詳今從之}
俘斬數千收其資糧甲仗以還|{
	還從宣翻}
由是公順黨衆大盛及李密據洛口公順以衆附之密敗亦來降|{
	降戶江翻}
隋末羣盜起冠軍司兵李襲譽|{
	按新書李襲譽傳仕隋為冠軍府司兵考之隋志冠軍輔國將軍從六品耳其府司兵當在流外小官也冠古玩翻}
說西京留守隂世師|{
	守式又翻}
遣兵據永豐倉發粟以賑貧乏出庫物賞戰士移檄郡縣同心討賊世師不能用|{
	說輸芮翻}
乃求募兵山南世師許之上克長安自漢中召還為太府少卿|{
	隋避諱以漢中為漢川郡唐復曰漢中仍改郡曰梁州梁洋等州皆在長安南山之南少始照翻}
乙未附襲譽籍於宗正|{
	李襲譽之先亦出於隴西故附之屬籍以親之宗正寺掌天子族親屬籍以别昭穆}
襲譽襲志之弟也|{
	李襲志時事蕭銑}
丙申朱粲寇淅州|{
	舊志鄧州内鄉縣漢淅縣地後周改為中鄉隋改為内鄉武德二年置浙州又按隋志浙陽郡西魏置浙州治南鄉縣朱粲所寇蓋南鄉之淅州浙音折}
遣太常卿鄭元璹帥步騎一萬擊之|{
	璹殊玉翻帥讀曰率騎奇寄翻}
是月納言竇抗罷為左武候大將軍 十一月乙巳凉王李軌即皇帝位改元安樂|{
	樂音洛 考異曰按軌傳云軌稱凉王即改元安樂今據實錄}
戊申王軌以滑州來降|{
	李密既敗王軌來降降戶江翻下同}
薛仁果之為太子也|{
	去年秋七月薛舉稱帝仁果為太子}
與諸將多有

隙及即位衆心猜懼郝瑗哭舉得疾遂不起由是國勢浸弱秦王世民至高墌仁果使宗羅㬋將兵拒之羅㬋數挑戰|{
	將即亮翻郝呼各翻瑗于眷翻塘章恕翻數所角翻挑徒了翻}
世民堅壁不出諸將咸請戰世民曰我軍新敗士氣沮喪|{
	謂是年七月淺水原之敗也沮在呂翻喪息浪翻}
賊恃勝而驕有輕我心宜閉壘以待之彼驕我奮可一戰而克也乃令軍中曰敢言戰者斬相持六十餘日仁果糧盡其將梁胡郎等帥所部來降|{
	將即亮翻下同帥讀曰率下同}
世民知仁果將士離心命行軍總管梁實營於淺水原以誘之|{
	誘音酉}
羅㬋大喜盡銳攻之梁實守險不出營中無水人馬不飲者數日羅㬋攻之甚急世民度賊已疲|{
	度徒洛翻}
謂諸將曰可以戰矣遟明|{
	遟直利翻}
使右武候大將軍龎玉陳於淺水原|{
	陳讀曰陣下同}
羅㬋併兵擊之玉戰幾不能支|{
	幾居依翻}
世民引大軍自原北出其不意羅㬋引兵還戰世民帥驍騎數十先陷陳唐兵表裏奮擊呼聲動地|{
	帥讀曰率驍堅堯翻騎奇寄翻下同陳讀曰陣呼火故翻}
羅㬋士卒大潰斬首數千級世民帥二千餘騎追之|{
	騎奇寄翻}
竇軌叩馬苦諫曰仁果猶據堅城雖破羅㬋未可輕進請且按兵以觀之|{
	㬋音侯}
世民曰吾慮之久矣破竹之勢不可失也|{
	杜預曰兵威已振譬如破竹數節之後迎刃而解}
舅勿復言|{
	世民竇氏之出呼軌為舅復扶又翻}
遂進仁果陳於城下世民據涇水臨之仁果驍將渾幹等數人臨陳來降|{
	驍堅堯翻將即亮翻渾戶昆翻姓苑記其所出者多左傳鄭有大夫渾罕衛有渾良夫唐渾瑊祖渾邪王吐谷渾後為渾其音戶本翻以為出於渾沌氏者謬也註又見後降戶江翻}
仁果懼引兵入城拒守日向暮大軍繼至遂圍之夜半守城者爭自投下仁果計窮己酉出降得其精兵萬餘人男女五萬口諸將皆賀因問曰大王一戰而勝遽捨步兵又無攻具輕騎直造城下|{
	造七到翻}
衆皆以為不克而卒取之何也|{
	卒子恤翻}
世民曰羅㬋所將皆隴外之人將驍卒悍|{
	將即亮翻又音如字領也悍戶罕翻又侯旰翻}
吾特出其不意而破之斬獲不多若緩之則皆入城仁果撫而用之未易克也|{
	易以豉翻}
急之則散歸隴外折墌虛弱仁果破膽不暇為謀此吾所以克也|{
	折當作析音思歷翻墌章恕翻}
衆皆悅服世民所得降卒悉使仁果兄弟及宗羅㬋翟長孫等將之|{
	翟萇伯翻長知兩翻}
與之射獵無所疑間|{
	間古莧翻}
賊畏威銜恩皆願效死世民聞禇亮名求訪獲之禮遇甚厚引為王府文學|{
	自隋時親王府有文學}
上遣使謂世民曰|{
	使疏吏翻}
薛舉父子多殺我士卒必盡誅其黨以謝寃䰟李密諫曰薛舉虐殺無辜此其所以亡也陛下何怨焉懷服之民不可不撫乃命戮其謀首餘皆赦之上使李密迎秦王世民於州密自恃智略功名見上猶有傲色及見世民不覺驚服|{
	此豈獨相表服之哉威靈氣燄足以服之也}
私謂殷開山曰真英主也不如是何以定禍亂乎詔以員外散騎常侍姜謩為秦州刺史|{
	曹魏末置員外散騎常侍散悉亶翻騎奇寄翻}
謩撫以恩信盜賊悉歸首|{
	首手又翻}
士民安之 徐世勣據李密舊境未有所屬魏徵隨密至長安乃自請安集山東上以為祕書丞|{
	漢獻帝建安中魏武為魏王置祕書令及二丞}
乘傳至黎陽遺徐世勣書|{
	傳殊戀翻遺于季翻}
勸之早降世勣遂决計西向謂長史陽翟郭孝恪曰|{
	隋志陽翟縣屬襄城郡降戶江翻長知兩翻}
此民衆土地皆魏公有也|{
	李密建國稱魏公}
吾若上表獻之|{
	上時掌翻}
是利主之敗自為功以邀富貴也吾實恥之今宜籍郡縣戶口士馬之數以啟魏公使自獻之乃遣孝恪詣長安又運糧以餉淮安王神通|{
	神通時安撫山東}
上聞世勣使者至無表止有啟與密甚怪之孝恪具言世勣意上乃嘆曰徐世勣不背德不邀功|{
	使疏吏翻背蒲妹翻}
真純臣也賜姓李|{
	時授世勣黎州總管封英國公通鑑書於明年閏二月}
以孝恪為宋州刺史|{
	復以梁郡為宋州此時唐未能有宋州也}
使與世勣經畧虎牢以東所得州縣委之選補|{
	委之選補官吏也}
癸丑獨孤懷恩攻堯君素於蒲反|{
	漢志作蒲反後始作蒲坂}
行軍總管趙慈景尚帝女桂陽公主為君素所擒梟首城外以示無降意|{
	梟堅堯翻降戶江翻}
癸亥秦王世民至長安斬薛仁果於市賜常達帛三百段|{
	賞其不屈也唐制凡賜十段其率絹三匹布三端綿四斤若雜綵十段則絲布二匹紬二匹綾二匹縵四匹}
贈劉感平原郡公諡忠壯|{
	以其死節也}
撲殺仵士政於殿庭|{
	撲弼角翻擊也}
以張貴尤淫暴腰斬之上享勞將士因謂羣臣曰諸公共相翊戴以成帝業若天下承平可共保富貴使王世充得志公等豈有種乎|{
	因薛仁果君臣以相戒勞力到翻種章勇翻}
如薛仁果君臣豈可不以為前鑑也己巳以劉文靜為戶部尚書領陜東道行臺左僕射復殷開山爵位|{
	先是劉文靜殷開山皆以淺水原之敗除名}
李密驕貴日久又自負歸國之功朝廷待之不副本望|{
	事見上}
鬱鬱不樂|{
	樂音洛}
嘗遇大朝會密為光祿卿當進食|{
	六典光祿卿之職掌邦國酒醴膳羞之事總太官珍羞良醖掌醢四署之官屬朝會燕饗則節其等差量其豐約以供焉故當進食朝直遥翻}
深以為恥退以告左武衛大將軍王伯當伯當心亦怏怏|{
	怏於兩翻}
因謂密曰天下事在公度内耳今東海公在黎陽|{
	密封徐世勣為東海公}
襄陽公在羅口|{
	襄陽公未知為誰按密將張善相時為伊州刺史據襄城自襄城北出則羅口蓋李密封善相為襄城公伯當指言之也襄陽公疑當作襄城公}
河南兵馬屈指可計豈得久如此也密大喜乃獻策於上曰臣虛蒙榮寵安坐京師曾無報效山東之衆皆臣故時麾下請往收而撫之憑籍國威取王世充如拾地芥耳|{
	顏師古曰地芥謂草芥之横在地上者俯而拾之言易而必得也}
上聞密故將士多不附世充亦欲遣密往收之羣臣多諫曰李密狡猾好反|{
	將即亮翻好呼到翻}
今遣之如投魚於泉放虎於山必不反矣上曰帝王自有天命非小子所能取借使叛去如以蒿箭射蒿中耳|{
	蒿蓬蕭之屬叢生於地人皆賤其無用剡蒿為箭射之蒿中言其無用而不足惜也北齊源文宗曰國家視淮南同於蒿箭蓋蒿箭之言尚矣射而亦翻}
今使二賊交鬭吾可以坐收其弊辛未遣密詣山東收其餘衆之未下者密請與賈閏甫偕行上許之命密及閏甫同升御榻賜食傳飲巵酒曰吾三人同飲是酒以明同心善建功名以副朕意丈夫一言許人千金不易有人確執不欲弟行|{
	上呼李密為弟}
朕推赤心於弟非他人所能間也|{
	間古莧翻}
密閏甫再拜受命上又以王伯當為密副而遣之 |{
	考異曰高祖實錄未幾聞其下兵皆不附王世充今密收集餘衆以圖洛陽密言於高祖曰臣入朝日淺不願違離又在朝公卿未甚委信願得陛下腹心左右與臣同去高祖曰朕推赤心於人終無疑阻但有益國利人即當專决今從蒲山公傳}
有大鳥五集于樂壽|{
	樂音洛}
羣鳥數萬從之經日乃去竇建德以為已瑞改元五鳳宗城人有得玄圭獻於建德者宋正本及景城丞會稽孔德紹皆曰此天所以賜大禹也|{
	隋志宗城縣屬清河郡舊曰廣宗仁壽元年改焉避煬帝諱也景城縣屬河間郡舊曰成平開皇十八年改隋改越州為會稽郡禹平水土錫玄圭告厥成功蓋堯錫之也宋正本等引為天瑞以謟建德過矣隋縣置令丞會古外翻}
請改國號曰夏|{
	竇建德初稱長樂王夏戶雅翻}
建德從之以正本為納言德紹為内史侍郎初王須拔掠幽州中流矢死|{
	中竹仲翻 考異曰革命記云須拔衆散奔突厥突厥以為南面可汗今從唐書}
其將魏刀兒代領其衆據深澤掠冀定之間|{
	隋志深澤縣屬博陵郡劉昫曰治滹沱河北宋白曰以界内水澤深廣名縣時復信都郡為冀州博陵郡為定州將即亮翻}
衆至十萬自稱魏帝建德偽與連和刀兒弛備建德襲擊破之遂圍深澤其徒執刀兒降建德斬之盡幷其衆|{
	降戶江翻}
易定等州皆降唯冀州刺史麴稜不下|{
	麴稜時附於唐}
稜婿崔履行暹之孫也|{
	崔暹事齊高氏父子以不畏彊禦用}
自言有奇術可使攻者自敗稜信之履行命守城者皆坐毋得妄鬭曰賊雖登城汝曹勿怖|{
	怖普布翻}
吾將使賊自縛於是為壇夜設章醮然後自衣衰絰|{
	衣於既翻衰倉回翻絰徒結翻}
杖竹登北樓慟哭又令婦女升屋四面振裙建德攻之急稜將戰履行固止之俄而城陷履行哭猶未已|{
	自古以來信妖人之言以喪師亡城者多矣然後世之人猶有信而不悟者若高駢李守貞之徒是也}
建德見稜曰卿忠臣也厚禮之以為内史令 十二月壬申詔以秦王世民為太尉使持節陜東道大行臺|{
	使疏吏翻}
其蒲州河北諸府兵馬並受節度|{
	復以河東郡為蒲州河北謂大河以北黎相之地諸府諸總管府}
癸酉西突厥曷娑那可汗|{
	厥九勿翻娑蘇何翻可從刋入聲汗音寒}
自宇文化及所來降|{
	隋煬帝以曷婆那自從煬帝弑從化及降戶江翻}
隋將堯君素守河東|{
	將即亮翻}
上遣呂紹宗韋義節獨孤

懷恩相繼攻之俱不下|{
	義寧元年九月屈突通留堯君素守河東呂紹宗攻之不克以韋義節代之又不克武德元年九月以獨孤懷恩代之仍不下}
時外圍嚴急君素為木鵝置表於頸具論事勢浮之於河河陽守者得之達於東都皇泰主見而歎息拜君素金紫光祿大夫龎玉皇甫無逸自東都來降上悉遣詣城下為陳利害|{
	為于偽翻}
君素不從 |{
	考異曰高祖實錄令宇文士及為陳利害按宇文化及為竇建德所擒士及乃自歸於唐實錄誤也今從隋書}
又賜金劵許以不死其妻又至城下謂之曰隋室已亡君何自苦君素曰天下名義非婦人所知引弓射之應弦而倒|{
	射而亦翻 考異曰實錄云妻號慟而去今從隋書}
君素亦自知不濟然志在守死每言及國家未嘗不歔欷|{
	歔音虚欷音希又許既翻}
謂將士曰吾昔事主上於藩邸|{
	隋書堯君素傳煬帝為晉王君素以左右從}
大義不得不死必若隋祚永終天命有屬|{
	屬之欲翻}
自當斷頭以付諸君聽君等持取富貴|{
	斷丁管翻}
今城池甚固倉儲豐備大事猶未可知不可横生心也|{
	横戶孟翻}
君素性嚴明善御衆下莫敢叛久之倉粟盡人相食又獲外人微知江都傾覆丙子君素左右薛宗李楚客殺君素以降傳首長安|{
	降戶江翻下同}
君素遣朝散大夫解人王行本將精兵七百在他所|{
	解漢古縣也後魏曰安定西魏改曰南解又改曰綏化又曰虞鄉武德元年更名解縣别置虞鄉縣並屬蒲州朝直遥翻散悉亶翻解戶買翻將即亮翻}
聞之赴救不及因捕殺君素者黨與數百人悉誅之復乘城拒守|{
	復扶又翻又音如字}
獨孤懷恩引兵圍之 丁酉隋襄平太守鄧暠以柳城北平二郡來降以暠為營州總管|{
	隋置襄平柳城郡皆在遼西郡柳城縣界北平郡即平州盧龍之地時復以西郡為營州守式又翻暠工老翻}
辛巳太常卿鄭元璹擊朱粲於商州破之|{
	復以上洛郡為商州璹殊玉翻}
初宇文化及遣使招羅藝藝曰我隋臣也斬其使者為煬帝發喪臨三日|{
	使疏吏翻為于偽翻臨力鴆翻}
竇建德高開道各遣使招之藝曰建德開道皆劇賊耳吾聞唐公已定關中人望歸之此真吾主也吾將從之敢沮議者斬|{
	沮在呂翻}
會張道源慰撫山東藝遂奉表與漁陽上谷等諸郡皆來降癸未詔以藝為幽州總管|{
	隋大業初置漁陽郡於無終唐復以涿郡為幽州 考異曰創業注藝以武德元年二月降舊云三年新書云二年皆誤也今從實錄}
薛萬均世雄之子也|{
	薛世雄死見一百八十四卷義寧元年}
與弟萬徹俱以勇畧為藝所親待詔以萬均為上柱國永安郡公萬徹為車騎將軍武安縣公|{
	唐制上柱國郡公皆正二品縣公從二品車騎將軍則諸衛郎將之職也正五品騎奇寄翻下同}
竇建德既克冀州兵威益盛帥衆十萬寇幽州|{
	帥讀曰率下同}
藝將逆戰萬均曰彼衆我寡出戰必敗不若使羸兵背城阻水為陳|{
	羸倫為翻背蒲妹翻陳讀曰陣}
彼必度水擊我萬均請以精騎百人伏於城旁俟其半度擊之蔑不勝矣藝從之建德果引兵度水萬均邀擊大破之建德竟不能至其城下乃分兵掠霍堡及雍奴等縣|{
	霍堡蓋世亂霍氏宗黨築堡以自固因以為名雍奴漢古縣唐志屬幽州天寶改為武清縣}
藝復邀擊敗之|{
	復扶又翻敗補邁翻}
凡相距百餘日建德不能克乃還樂壽|{
	樂音洛}
藝得隋通直謁者温彥博以為司馬|{
	隋煬帝置謁者臺有司朝謁者通事謁者通直謁者將事謁者}
藝以幽州歸國彥博贊成之詔以彦博為幽州總管府長史未幾徵為中書侍郎|{
	長知兩翻幾居豈翻}
兄大雅時為黄門侍郎與彦博對居近密|{
	黄門侍郎居門下省謂之東省中書侍郎居中書省謂之西省故曰對居近密}
時人榮之 以西突厥曷娑那可汗為歸義王|{
	厥九勿翻娑素那翻可從刋入聲汗音寒}
曷娑那獻大珠上曰珠誠至寶然朕寶王赤心珠無所用竟還之 乙酉車駕幸周氏陂過故墅|{
	水經注白渠尾入櫟陽而東南注于渭故渠逕漢丞相周勃冢南冢北有弱夫冢故渠東南有周氏曲渠又南逕漢景帝陵南又東南注于渭周氏曲即周氏陂也在高陵縣界故墅在高陵縣西十里店上舊所居也武德六年名龍躍宫}
初羌豪旁企地以所部附薛舉|{
	旁步光翻羌姓也}
及薛仁果敗企地來降留長安企地不樂|{
	企去智翻樂音洛}
帥其衆數千叛入南山出漢川|{
	帥讀曰率此自長安南山諸谷出漢川漢川即漢中}
所過殺掠武候大將軍龎玉擊之為企地所敗|{
	敗補邁翻}
行至始州|{
	普安漢梓潼縣廣漢郡治焉宋置南安郡梁置南梁州後改安州西魏改為始州大業初改為普安郡唐復為始州先天二年改為劒州}
掠女子王氏與俱醉卧野外王氏拔其佩刀斬首送梁州|{
	唐改漢川郡為梁州}
其衆遂潰詔賜王氏號為崇義夫人 壬辰王世充帥衆三萬圍穀州|{
	新安縣後周置中州及東垣縣州尋廢開皇十六年置穀州仁壽四年州廢又廢新安入東垣大業初改名新安縣屬河南郡義寧二年破段達置新安郡武德元年改穀州取穀水為名帥讀曰率}
刺史任瓌拒却之|{
	任音壬瓌古回翻}
上使李密分其麾下之半留華州|{
	周宣王封其弟友於鄭自漢以來為鄭縣後魏置東雍州及華山郡西魏改曰華州開皇初州廢大業初郡廢為鄭縣屬京兆郡義寧元年析京兆之鄭華隂置華隂郡尋改華州華戶化翻}
將其半出關|{
	將即亮翻}
長史張寶德預在行中恐密亡去罪相及上封事言其必叛|{
	長知兩翻上時掌翻}
上意乃中變又恐密驚駭乃降敕書勞來|{
	勞力到翻來力代翻}
令密留所部徐行單騎入朝更受節度|{
	騎奇寄翻朝直遥翻}
密至稠桑得敕謂賈閏甫曰敕遣我去無故復召我還|{
	稠直留翻復扶又翻}
天子曏云有人確執不許此譖行矣吾今若還無復生理不若破桃林縣|{
	開皇十六年分閿鄉陜置桃林縣取古桃林之塞以名縣也在陜西四十五里}
收其兵糧北走度河比信達熊州|{
	宜陽縣後魏置宜陽郡東魏置陽州後周改曰熊州開皇初郡廢大業初州廢屬河南郡義寧二年破段達置宜陽郡武德元年置熊州取熊耳山以名州杜佑曰熊州今福昌縣比必寐翻}
吾已遠矣苟得至黎陽大事必成|{
	言欲就徐世勣也}
公意如何閏甫曰主上待明公甚厚况國家姓名著在圖䜟|{
	䜟楚譛翻}
天下終當一統明公既已委質|{
	質職日翻}
復生異圖|{
	復扶又翻下同}
任瓌史萬寶據熊穀二州|{
	任音壬瓌古回翻}
此事朝舉彼兵夕至雖克桃林兵豈暇集一稱叛逆誰復容人為明公計不若且應朝命|{
	朝直遥翻}
以明元無異心自然浸潤不行|{
	論語曰浸潤之譖不行焉閏甫引此}
更欲出就山東徐思其便可也密怒曰唐使吾與絳灌同列何以堪之|{
	言不得如韓彭割地而王使與周勃灌嬰同列}
且䜟文之應彼我所共今不殺我聽使東行足明王者不死縱使唐遂定關中山東終為我有天與不取乃欲束手投人公吾之心腹何意如是若不同心當斬而後行閏甫泣曰明公雖云應䜟近察天人稍已相違今海内分崩人思自擅強者為雄明公奔亡甫爾誰相聽受且自翟讓受戮之後|{
	翟萇伯翻}
人皆謂明公棄恩忘本今日誰肯復以所有之兵束手委公乎彼必慮公見奪逆相拒抗一朝失勢豈有容足之地哉自非荷恩殊厚者詎肯深言不諱乎|{
	荷下可翻}
願明公熟思之但恐大福不再|{
	楚靈王之言}
苟明公有所措身閏甫亦何辭就戮密大怒揮刃欲擊之王伯當等固請乃釋之閏甫奔熊州伯當亦止密以為未可密不從伯當乃曰義士之志不以存亡易心公必不聽伯當與公同死耳然恐終無益也密因執使者斬之庚子旦密紿桃林縣官曰奉詔蹔還京師|{
	使疏吏翻紿徒亥翻蹔與暫同}
家人請寄縣舍乃簡驍勇數十人著婦人衣戴羃䍦|{
	驍堅堯翻著側畧翻羃莫狄翻䍦音離}
藏刀裙下詐為妻妾自帥之入縣舍|{
	帥讀曰率下同}
須臾變服突出因據縣城驅掠徒衆直趣南山|{
	趣七喻翻又逡須翻}
乘險而東遣人馳告故將伊州刺史襄城張善相令以兵應接|{
	五代志襄城郡東魏置北荆州後周改曰和州開皇初改曰伊州大業初改曰汝州尋改為郡李密復開皇舊州名杜佑曰伊州今陸渾縣將即亮翻相息亮翻}
右翊衛將軍史萬寶鎮熊州謂行軍總管盛彦師曰李密驍賊也|{
	驍堅堯翻}
又輔以王伯當今决策而叛殆不可當也彦師笑曰請以數千之衆邀之必梟其首|{
	梟古堯翻}
萬寶曰公以何策能爾彦師曰兵法尚詐不可為公言之|{
	為于偽翻}
即帥衆踰熊耳山|{
	熊耳山在熊州南}
南據要道|{
	南字當屬上句}
令弓弩夾路乘高刀楯伏於溪谷|{
	楯食尹翻}
令之曰俟賊半度一時俱發或問曰聞李密欲向洛州而公入山何也|{
	洛州謂洛陽}
彦師曰密聲言向洛實欲出人不意走襄城就張善相耳|{
	走音奏相息亮翻}
若賊入谷口我自後追之山路險隘無所施力一夫殿後必不能制|{
	殿丁練翻}
今吾先得入谷擒之必矣李密既度陜以為餘不足慮|{
	陜州之兵既不能邀密密自以為踰山而南他無邀阻不足慮也陜失冉翻}
遂擁衆徐行果踰山南出彦師擊之密衆首尾斷絶不得相救遂斬密及伯當俱傳首長安 |{
	考異曰河洛記密因執驛使者斬之曉入桃林詐縣官翻據縣城中驚悸莫敢當者驅掠畜產趨南山時左翊衛將軍上柱國太平公史萬寶在熊州既聞密叛遣將劉善武領兵追躡善武兄善績往在洛口為密所屠善武因此發憤志在取密十日十夜倍道兼行百方羅捕無暫休息追至陸渾縣南七十里與密相及連戰轉鬬一步一前驅密於邢公山與王伯當死之今從實錄及舊書}
彦師以功賜爵葛國公仍領熊州|{
	領當依舊書作鎮}
李世勣在黎陽上遣使以密首示之告以反狀世勣北面拜伏號慟表請收葬|{
	使疏吏翻號戶刀翻}
詔歸其尸世勣為之行服備君臣之禮|{
	世勣以此受知於太宗為于偽翻}
大具儀衛舉軍縞素葬密于黎陽山南|{
	縞工老翻}
密素得士心哭者多歐血 隋右武衛大將軍李景守北平高開道圍之歲餘不能克遼西太守鄧暠將兵救之景帥其衆遷于柳城|{
	守式又翻暠古老翻將即亮翻帥讀曰率下同}
後將還幽州於道為盜所殺開道遂取北平進陷漁陽郡有馬數千匹衆且萬自稱燕王改元始興 |{
	考異曰實錄唐書皆無開道年號柳璨注正閏位歷六年號天成李昉歷代年號亦如之宋庠紀年通譜武德元年開道年號始興云出歷代紀要錄此號未知孰是今從紀要}
都漁陽懷戎沙門高曇晟|{
	劉昫曰懷戎後漢之潘縣屬上谷郡北齊改為懷戎縣隋屬幽州涿郡曇徒含翻晟承正翻}
因縣令設齋士民大集曇晟與僧五千人擁齋衆而反殺縣令及鎮將|{
	將即亮翻}
自稱大乘皇帝立尼靜宣為邪輸皇后|{
	釋氏以人之性識根業各差故有大乘小乘之說}
改元法輪遣使招開道立為齊王開道帥衆五千人歸之|{
	使疏吏翻帥讀曰率}
居數月襲殺曇晟悉幷其衆 有犯法不至死者上特命殺之監察御史李素立諫曰|{
	六典監察御史蓋取秦監郡御史以名官晉孝武太元中創置檢校御史後周秋官府有司憲旅下士隋初改為監察御史從八品上監工銜翻}
三尺法王者所與天下共也|{
	漢書客謂杜周曰君為廷尉不循三尺法孟康注云以三尺竹簡書法律也}
法一動揺人無所措手足陛下甫創洪業奈何棄法臣忝法司不敢奉詔上從之自是特承恩遇命所司授以七品清要官所司擬雍州司戶|{
	雍於用翻}
上曰此官要而不清又擬祕書郎上曰此官清而不要遂擢授侍御史|{
	六典侍御史從六品上杜佑曰唐侍御史之職有四謂推彈公廨雜事推者掌推鞠彈者掌彈舉公廨知公廨事雜事臺事總悉判之}
素立義深之曾孫也|{
	李義深趙郡著姓事高齊史云人位兼美}
上以舞胡安比奴為散騎侍郎|{
	散悉亶翻騎奇寄翻}
禮部尚書李綱諫曰古者樂工不與士齒雖賢如子野師襄皆世不易其業|{
	子野晉樂師曠字襄魯樂師}
唯齊末封曹妙達為王安馬駒為開府有國家者以為殷鑑|{
	齊後主亡國亦此之由詩云殷監不遠在夏后之世}
今天下新定建義功臣行賞未遍高才碩學猶滯草萊而先擢舞胡為五品使鳴玉曳組趨翔廊廟|{
	組則古翻}
非所以規模後世也上不從曰吾業已授之不可追也

陳嶽論曰受命之主發號出令為子孫法一不中理則為厲階|{
	中竹仲翻}
今高祖曰業已授之不可追苟授之而是則已授之而非胡不可追歟君人之道不得不以業已授之為誡哉

李軌吏部尚書梁碩有智畧軌常倚之以為謀主碩見諸胡浸盛隂勸軌宜加防察由是與戶部尚書安修仁有隙|{
	其後安修仁兄弟縛軌以歸於唐卒如梁碩所慮}
軌子仲琰嘗詣碩碩不為禮乃與修仁共譖碩於軌誣以謀反軌酖碩殺之有胡巫謂軌曰上帝當遣玉女自天而降軌信之發民築臺以候玉女勞費其廣河右飢人相食軌傾家財以賑之不足欲發倉粟|{
	賑津忍翻}
召羣臣議之曹珍等皆曰國以民為本豈可愛倉粟而坐視其死乎謝統師等皆故隋官心終不服|{
	謝統師等為軌所執見一百八十四卷義寧元年七月}
密與羣胡為黨排軌故人乃詬珍曰|{
	詬苦候翻}
百姓餓者自是羸弱勇壯之士終不至此國家倉粟以備不虞豈可散之以飼羸弱|{
	羸倫為翻伺祥吏翻}
僕射苟悅人情不為國計非忠臣也軌以為然由是士民離怨|{
	為李軌敗亡張本}


資治通鑑卷一百八十六
