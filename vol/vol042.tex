資治通鑑卷四十二   宋 司馬光 撰

胡三省 音註

漢紀三十四|{
	起上章攝提格盡旃蒙協洽凡六年}


世祖光武皇帝中之止

建武六年春正月丙辰以舂陵郷為章陵縣世世復徭役比豐沛|{
	復方目翻}
吳漢等拔胊斬董憲龎萌江淮山東悉平|{
	據范紀是年馬成等拔舒獲李憲吳漢等拔朐斬董憲龐萌蓋獲李憲則江淮平斬董憲龎萌則山東平也拔胊之上逸拔舒事}
諸將還京師置酒賞賜|{
	還從宣翻又如字}
帝積苦兵間以隗囂遣子内侍公孫述遠據邊垂乃謂諸將曰且當置此兩子於度外耳因休諸將於雒陽分軍士於河内數騰書隴蜀告示禍福|{
	說文曰騰傳也數所角翻}
公孫述屢移書中國自陳符命冀以惑衆帝與述書曰圖讖言公孫即宣帝也|{
	宣帝有公孫病已之符}
代漢者姓當塗其名高君豈高之身邪乃復以掌文為瑞|{
	述刻其掌文曰公孫帝自言手文有奇復扶又翻}
王莽何足效乎|{
	王莽自陳符命遣五威將帥班之天下}
君非吾賊臣亂子倉卒時人皆欲為君事耳|{
	卒讀曰猝}
君日月已逝|{
	謂已老也}
妻子弱小當早為定計天下神器不可力爭宜留三思署曰公孫皇帝述不答其騎都尉平陵荆邯說述曰漢高祖起於行陳之中兵破身困者數矣然軍敗復合瘡愈復戰|{
	邯下甘翻說輸芮翻行戶剛翻陳讀曰陣數所角翻復扶又翻下同}
何則前死而成功愈於却就於滅亡也隗囂遭遇運會割有雍州兵彊士附威加山東|{
	賢曰隴西天水皆雍州之地故言割有囂傳曰名震西州流聞山東是威加也雍於用翻}
遇更始政亂復失天下衆庶引領四方瓦解囂不及此時推危乘勝|{
	推吐雷翻}
以爭天命而退欲為西伯之事尊師章句賓友處士|{
	處昌呂翻}
偃武息戈卑辭事漢喟然自以文王復出也令漢帝釋關隴之憂|{
	賢曰以囂居西無東之意故置之度外而不為憂}
專精東伐四分天下而有其三發間使召攜貳|{
	賢曰間使謂來歙馬援等也攜貳謂王遵鄭興杜林牛邯等相次而歸光武間古莧翻}
使西州豪桀咸居心於山東則五分而有其四若舉兵天水必至沮潰|{
	沮在呂翻}
天水旣定則九分而有其八陛下以梁州之地|{
	益州禹貢梁州之域也}
内奉萬乘外給三軍百姓愁困不堪上命將有王氏自潰之變矣|{
	賢曰王氏即王莽也}
臣之愚計以為宜及天下之望未絶豪桀尚可招誘|{
	誘音酉}
急以此時發國内精兵令田戎據江陵臨江南之會倚巫山之固|{
	瓚曰巫山在今夔州巫山縣東}
築壘堅守傳檄吳楚長沙以南必隨風而靡令延岑出漢中定三輔天水隴西拱手自服如此海内震搖冀有大利述以問羣臣博士吳柱曰武王伐殷八百諸侯不期同辭然猶還師以待天命|{
	武王伐紂至于孟津諸侯不期而會者八百皆曰紂可伐矣武王曰汝未知天命乃還}
未聞無左右之助而欲出師千里之外者也邯曰今東帝無尺土之柄|{
	東帝謂光武}
驅烏合之衆跨馬陷敵所向輒平不亟乘時與之分功而坐談武王之說是復效隗囂欲為西伯也述然邯言欲悉發北軍屯士及山東客兵|{
	述倣漢制亦置北軍山東之人僑寓於蜀者述以為兵故曰客兵}
使延岑田戎分出兩道與漢中諸將合兵并埶蜀人及其弟光以為不宜空國千里之外決成敗於一舉固爭之述乃止延岑田戎亦數請兵立功述終疑不聽唯公孫氏得任事述廢銅錢置鐵錢貨幣不行百姓苦之為政苛細察於小事如為清水令時而已好改易郡縣官名少嘗為郎|{
	哀帝時述以父任為郎好呼到翻少詩沼翻}
習漢家故事出入灋駕鸞旗旄騎又立其兩子為王食犍為廣漢各數縣|{
	犍居言翻}
或諫曰成敗未可知戎士暴露而先王愛子|{
	先王于況翻}
示無大志也述不從由此大臣皆怨|{
	為述亡國張本}
馮異自長安入朝帝謂公卿曰是我起兵時主簿也|{
	帝起兵徇潁川異降以為主簿}
為吾披荆棘定關中|{
	為于偽翻}
旣罷賜珍寶錢帛詔曰倉卒蕪蔞亭豆粥虖沱河麥飯|{
	事見三十九卷更始二年卒與猝同}
厚意久不報異稽首謝曰臣聞管仲謂桓公曰願君無忘射鉤臣無忘檻車齊國賴之|{
	史記管仲射桓公中鉤後魯桎梏管仲而送於齊公以為相說苑曰管仲桎梏檻車中非無愧也自裁也新序曰齊桓公與管仲飲酣管仲上夀曰願君無忘出奔於莒也臣亦無忘束縛於魯也此云射鉤檻車義亦通射而亦翻}
臣今亦願國家無忘河北之難|{
	東都臣子率謂天子為國家難乃旦翻}
小臣不敢忘巾車之恩|{
	事見三十九卷更始元年}
留十餘日令與妻子還西 申屠剛杜林自隗囂所來 |{
	考異曰本傳云七年徵剛按明年囂已臣公孫述必不用詔書當在此年}
帝皆拜侍御史以鄭興為太中大夫 三月公孫述使田戎出江關|{
	地理志江關都尉治巴郡魚復縣賢曰華陽國志曰巴楚相攻故置江關舊在赤甲城後移在江州南岸對白帝城故基在今夔州魚復縣南}
招其故衆欲以取荆州不克帝乃詔隗囂欲從天水伐蜀囂上言白水險阻棧閣敗絶|{
	賢曰白水縣有關屬廣漢郡棧閣者山路懸險棧木為閣道又公孫述傳註曰白水關在漢陽西縣梁州記曰關城西南有白水關余據水經白水出隴西臨洮縣西南西傾山東南流入隂平又東南經廣漢白水縣臨洮與西縣接界故天水之西縣有白水關而廣漢之白水縣亦有白水關自源徂流同一白水也賢曰梁州記曰關城西南百八十里有白水關故關城在今梁州金牛縣西}
述性嚴酷上下相患須其罪惡孰著而攻之此大呼響應之埶也|{
	須待也孰古熟字通用人大呼則響必應言俟其上下乖離而攻之必有為内應者呼火故翻}
帝知其終不為用乃謀討之 夏四月丙子上行幸長安|{
	郡國志長安在雒陽西九百五十里}
謁園陵遣耿弇蓋延等七將軍從隴道伐蜀先使中郎將來歙奉璽書賜囂諭旨囂復多設疑故|{
	歙許及翻璽斯氏翻疑疑難故事故也復扶又翻}
事久冘豫不決|{
	賢曰冘豫不定之意也說文曰冘冘行貌也音淫余按冘讀與猶同毛晃曰冘字從犬曲其足古與尤字同唐史以冘豫之冘音淫者誤也}
歙遂發憤質責囂曰|{
	賢曰質正也}
國家以君知臧否曉廢興|{
	否音鄙}
故以手書暢意足下推忠誠旣遣伯春委質|{
	囂子恂字伯春質職日翻賢曰委質猶屈膝也又音摯}
而反欲用佞惑之言為族滅之計邪因欲前刺囂|{
	刺七亦翻}
囂起入部勒兵將殺歙歙徐杖節就車而去囂使牛邯將兵圍守之|{
	邯下甘翻}
囂將王遵諫曰君叔雖單車遠使而陛下之外兄也|{
	來歙字君叔賢曰光武之姑子故曰外兄使疏吏翻}
殺之無損於漢而隨以族滅昔宋執楚使遂有析骸易子之禍|{
	左傳楚使申舟聘齊不假道於宋華元曰過我而不假道鄙我也乃殺之楚子聞之遂圍宋宋人易子而食析骸而㸑}
小國猶不可辱況於萬乘之主重以伯春之命哉|{
	重直用翻}
歙為人有信義言行不違及往來游說皆可按覆|{
	行下孟翻說輸芮翻}
西州士大夫皆信重之多為其言|{
	為于偽翻}
故得免而東歸 五月己未車駕至自長安隗囂遂發兵反使王元據隴坻|{
	師古曰坻音丁計翻又音底}
伐木塞道|{
	塞悉則翻}
諸將因與囂戰大敗各引兵下隴囂追之急馬武選精騎為後拒殺數千人諸軍乃得還|{
	還從宣翻又如字}
六月辛卯詔曰夫張官置吏所以為民也|{
	為于偽翻}
今百姓遭難戶口耗少|{
	難乃旦翻少詩沼翻}
而縣官吏職所置尚繁其令司隸州牧各實所部|{
	所部郡縣名考覈其實也}
省減吏員縣國不足置長吏者并之於是并省四百餘縣吏職減損十置其一 九月丙寅晦日有食之執金吾朱浮上疏曰昔堯舜之盛猶加三考|{
	賢曰考謂考其功最也尚書舜典曰三載考績三考黜陟幽明}
大漢之興亦累功效吏皆積久至長子孫|{
	如倉氏庫氏之類是也長知兩翻下同}
當時吏職何能悉治論議之徒豈不喧譁蓋以為天地之功不可倉卒艱難之業當累日也而間者守宰數見換易|{
	卒與猝同數所角翻}
迎新相代疲勞道路尋其視事日淺未足昭見其職旣加嚴切人不自保廹於舉劾|{
	劾戶槩翻}
懼於刺譏故爭飾詐偽以希虚譽斯所以致日月失行之應也夫物暴長者必夭折功卒成者必亟壞|{
	夭於紹翻卒讀曰猝}
如摧長久之業而造速成之功非陛下之福也願陛下遊意於經年之外望治於一世之後|{
	孔子曰如有王者必世而後仁三十年為一世治直吏翻}
天下幸甚帝采其言自是牧守代易頗簡十二月壬辰大司空宋弘免 癸巳詔曰頃者師旅未解用度不足故行十一之税|{
	謂十分而税其一也}
今糧儲差積其令郡國收見田租三十税一如舊制|{
	賢曰景帝二年令田租三十而税一今依景帝故云舊制見賢遍翻}
諸將之下隴也帝詔耿弇軍漆|{
	賢曰漆縣名屬右扶風故城在今州新平縣漆水在西}
馮異軍栒邑祭遵軍汧|{
	賢曰汧水名因以名縣屬右扶風故城在今隴州汧城縣南汧苦堅翻}
吳漢等還屯長安馮異引軍未至栒邑隗囂乘勝使王元行廵將二萬餘人下隴|{
	行姓也姓譜周有大行人之官其後氏焉}
分遣廵取栒邑異即馳兵欲先據之諸將曰虜兵盛而乘勝不可與爭鋒宜止軍便地徐思方畧異曰虜兵臨境忸小利|{
	賢曰忸猶慣習也謂慣習前事而復為之爾雅曰忸復也郭景純曰謂慣忸復為之也忸尼丑翻音逝}
遂欲深入若得栒邑三輔動搖夫攻者不足守者有餘|{
	孫武子之言}
今先據城以逸待勞非所以爭也濳往閉城偃旗鼔行廵不知馳赴之異乘其不意卒擊鼓建旗而出|{
	卒讀曰猝}
廵軍驚亂奔走追擊大破之祭遵亦破王元於汧於是北地諸豪長耿定等悉畔隗囂降|{
	長知兩翻}
詔異進軍義渠|{
	義渠縣屬北地郡古義渠戌地也}
擊破盧芳將賈覽匈奴奥鞬日逐王北地上郡安定皆降|{
	奧音郁鞬居言翻}
竇融復遣其弟友上書曰臣幸得託先后末屬|{
	謂孝文竇皇后之親屬也復扶又翻}
累世二千石臣復假歷將帥守持一隅|{
	復扶又翻}
故遣劉鈞口陳肝膽|{
	事見上卷上年}
自以底裏上露長無纖介|{
	賢曰底裏皆露言無藏隱}
而璽書盛稱蜀漢二主三分鼎足之權任囂尉佗之謀竊自痛傷臣融雖無識無知利害之際順逆之分豈可背眞舊之主事姦偽之人|{
	分扶問翻背蒲妹翻}
廢忠貞之節為傾覆之事棄已成之基求無冀之利此三者雖問狂夫猶知去就而臣獨何以用心謹遣弟友詣闕口陳至誠友至高平|{
	賢曰高平縣屬安定後改為平高今原州縣}
會隗囂反道不通乃遣司馬席封間道通書|{
	姓譜席其先姓籍避項羽諱改姓席氏}
帝復遣封賜融友書所以尉藉之甚厚|{
	尉與慰同尉安也藉薦也尉以安於身上藉以安於身下}
融乃與隗囂書曰將軍親遇厄會之際國家不利之時|{
	賢曰謂漢遭王莽簒奪也}
守節不囘承事本朝融等所以欣服高義願從役於將軍者良為此也|{
	為于偽翻}
而忿悁之間|{
	悁恚也吉縣翻躁急也}
改節易圖委成功造難就|{
	委棄也就成也}
百年累之一朝毁之豈不惜乎殆執事者貪功建謀以至於此|{
	言隗囂執政事者貪有其功而立此逆謀也}
當今西州地埶局廹民兵離散易以輔人|{
	易以豉翻下同}
難以自建計若失路不反聞道猶迷不南合子陽則北入文伯耳夫負虚交而易彊禦|{
	負恃也易輕也}
恃遠救而輕近敵未見其利也自兵起以來城郭皆為丘墟生民轉於溝壑幸賴天運少還而將軍復重其難|{
	復扶又翻下同難乃旦翻}
是使積痾不得遂瘳幼孤將復流離言之可為酸鼻庸人且猶不忍況仁者乎融聞為忠甚易得宜實難憂人太過以德取怨|{
	謂憂之之過而言之甚切將以為德而反以取怨也}
知且以言獲罪也囂不納融乃與五郡太守共砥厲兵馬上疏請師期帝深嘉美之融即與諸郡守將兵入金城擊囂黨先零羌封何等大破之|{
	更始時先零羌封何諸種殺金城太守據其郡囂賂遺封何與結盟欲發其衆零音憐}
因並河揚威武|{
	賢曰並蒲浪翻}
伺候車駕時大兵未進融乃引還帝以融信效著明益嘉之脩理融父墳墓祠以太牢|{
	融祖父墳墓在扶風}
數馳輕使致遺四方珍羞|{
	遺以四方珍羞既以厚融且示四方來服能致遠物也數所角翻遺于季翻}
梁統猶恐衆心疑惑乃使人刺殺張玄|{
	張玄隗囂使刺七亦翻}
遂與隗囂絶皆解所假將軍印綬先是馬援聞隗囂欲貳於漢|{
	先悉薦翻}
數以書責譬之囂得書增怒及囂發兵反援乃上書曰臣與隗囂本實交友初遣臣東謂臣曰本欲為漢|{
	為于偽翻}
願足下往觀之於汝意可即專心矣及臣還反報以赤心實欲導之於善非敢譎以非義而囂自挟姦心盜憎主人|{
	左傳晉伯宗妻曰盜憎主人民惡其上}
怨毒之情遂歸於臣臣欲不言則無以上聞願聽詣行在所極陳滅囂之術帝乃召之援具言謀畫帝因使援將突騎五千往來游說囂將高峻任禹之屬下及羌豪為陳禍福|{
	說輸芮翻為于偽翻說客單車往使足矣光武遣馬援將突騎五千欲耀兵威以示隴右諸將使訹而來}
以離囂支黨援又為書與囂將楊廣使曉勸於囂曰援竊見四海已定兆民同情而季孟閉拒背畔為天下表的|{
	隗囂字季孟賢曰表猶標也言為標表的謂射的也言背畔之罪為天下所指射也背蒲妹翻}
常懼海内切齒思相屠裂故遺書戀戀以致惻隱之計|{
	遺于季翻}
乃聞季孟歸罪於援而納王游翁諂邪之說|{
	賢曰王元字游翁據隗囂傳元字惠孟游翁盖其别字也}
因自謂函谷以西舉足可定|{
	所謂以九泥封函谷関也}
以今而觀竟何如邪援間至河内|{
	間古莧翻}
過存伯春|{
	存存問也時囚囂子恂於河内伯春恂字也}
見其奴吉從西方還說伯春小弟仲舒望見吉欲問伯春無它否竟不能言曉夕號泣|{
	號戶刀翻}
又說其家悲愁之狀不可言也夫怨讎可刺不可毁援聞之不自知泣下也援素知季孟孝愛曾閔不過夫孝於其親豈不慈於其子可有子抱三木而跳梁妄作自同分羹之事乎|{
	賢曰三木者謂桎梏及械也分羹謂樂羊也余謂此正引高帝答項羽之事}
季孟平生自言所以擁兵衆者欲以保全父母之國而完墳墓也又言苟厚士大夫而已|{
	即其所常言以感人悟物者而窮其本情}
而今所欲全者將破亡之所欲完者將傷毁之所欲厚者將反薄之季孟嘗折隗子陽而不受其爵|{
	事見上卷四年賢曰愧猶辱也}
今更共陸陸往附之|{
	賢曰陸陸猶碌碌也}
將難為顔乎|{
	言將有慙色也}
若復責以重質當安從得子主給是哉|{
	言蜀若復責質子當何從得子以為質也復扶又翻質音致}
往時子陽獨欲以王相待而春卿拒之今者歸老更欲低頭與小兒曹共槽而食併肩側身於怨家之朝乎|{
	歸入也言其年已入老境也字林曰併音卑正翻朝直遥翻}
今國家待春卿意深宜使牛孺卿與諸耆老大人共說季孟|{
	牛邯字孺卿說輸芮翻}
若計畫不從真可引領去矣前披輿地圖見天下郡國百有六所奈何欲以區區二邦以當諸夏百有四乎|{
	二邦謂隴西天水夏戶雅翻}
春卿事季孟外有君臣之義内有朋友之道言君臣邪固當諫爭語朋友邪應有切磋|{
	賢曰骨曰切象曰磋言朋友之道如切磋以成器也}
豈有知其無成而但萎腇咋舌义手從族乎|{
	賢曰萎腇耎弱也萎音於罪翻腇音乃罪翻咋吐格翻齧也}
及今成計殊尚善也過是欲少味矣|{
	賢曰以食為喻少詩沼翻}
且來君叔天下信士朝廷重之其意依依常獨為西州言|{
	為于偽翻}
援商朝廷尤欲立信於此|{
	商度也}
必不負約援不得久留願急賜報廣竟不答諸將每有疑議更請呼援咸敬重焉|{
	更工衡翻}
隗囂上疏謝曰吏民聞大兵卒至|{
	卒讀曰猝}
驚恐自救臣囂不能禁止兵有大利不敢廢臣子之節親自追還|{
	此因王元隴坻之捷而有嫚書也}
昔虞舜事父大杖則走小杖則受|{
	賢曰家語孔子謂曾子之辭}
臣雖不敏敢忘斯義今臣之事在於本朝賜死則死加刑則刑如更得洗心死骨不朽有司以囂言慢請誅其子帝不忍復使來歙至汧|{
	復扶又翻汧苦堅翻}
賜囂書曰昔柴將軍云陛下寛仁諸侯雖有亡叛而後歸輒復位號不誅也|{
	高帝時柴武與韓王信書之言}
今若束手復遣恂弟歸闕庭者則爵禄獲全有浩大之福矣吾年垂四十在兵中十歲厭浮語虚辭即不欲勿報囂知帝審其詐遂遣使稱臣於公孫述|{
	使疏吏翻}
匈奴與盧芳為寇不息帝令歸德侯颯使匈奴以脩舊好|{
	颯使匈奴見三十九卷更始二年颯音立好呼到翻}
單于驕倨雖遣使報命而寇暴如故

七年春三月罷郡國輕車騎士材官令還復民伍|{
	漢官儀曰高祖命天下郡國選能引關蹷張材力武猛者以為輕車騎士材官平地用車騎山阻用材官}
公孫述立隗囂為朔寧王|{
	賢曰欲其寧靜北邉也}
遣兵往來為之援埶|{
	張形埶以為之援也}
癸亥晦日有食之詔百僚各上封事其上書者不得言聖|{
	上時掌翻}
太中大夫鄭興上疏曰夫國無善政則謫見日月|{
	賢曰謫責也音直革翻見賢遍翻}
要在因人之心擇人處位|{
	處昌呂翻}
今公卿大夫多舉漁陽太守郭伋可大司空者而不以時定道路流言咸曰朝廷欲用功臣功臣用則人位謬矣|{
	人不稱其位位不宜其人也}
願陛下屈已從衆以濟羣臣讓善之功|{
	賢曰濟成也}
頃年日食多在晦先時而合皆月行疾也日君象而月臣象君亢急而臣下促廹故月行疾|{
	亢苦浪翻}
今陛下高明而羣臣惶促宜留思柔克之政垂意洪範之灋|{
	賢曰克能也柔克謂和柔而能立事也尚書洪範曰高明柔克}
帝躬勤政事頗傷嚴急故興奏及之 夏四月壬午大赦 五月戊戌以前將軍李通為大司空 大司農江馮上言宜令司隸校尉督察三公司空掾陳元上疏曰臣聞師臣者帝賓臣者霸|{
	元王莽厭難將軍陳欽之子賢曰言以臣為師以臣為賓也}
故武王以太公為師齊桓以夷吾為仲父近則高帝優相國之禮太宗假宰輔之權|{
	賢曰蕭何為相國高祖賜劍履上殿入朝不趨太宗孝文也申屠嘉召責鄧通孝文令人謝嘉故曰假權也}
及亡新王莽遭漢中衰專操國柄以偷天下|{
	操于高翻}
況已自喻不信羣臣奪公輔之任損宰相之威以刺舉為明激訐為直至乃陪僕告其君長子弟變其父兄|{
	王莽時開吏告其將奴婢告其主變者上變告之也陪僕猶左傳所謂陪臺也毛晃曰陪臺臣也盖古者家臣謂之陪臣故家之臣僕謂之陪僕長知兩翻}
罔密灋峻大臣無所措手足然不能禁董忠之謀|{
	事見三十九卷更始元年}
身為世戮方今四方尚擾天下未一百姓觀聽咸張耳目陛下宜修文武之聖典襲祖宗之遺德勞心下士屈節待賢誠不宜使有伺察公輔之名帝從之 酒泉太守竺曾以弟報怨殺人|{
	東觀記曰曾弟嬰報怨殺屬國侯王胤等}
自免去郡竇融承制拜曾武鋒將軍更以辛肜為酒泉太守|{
	更工衡翻肜余中翻}
秋隗囂將步騎三萬侵安定至隂槃|{
	賢曰隂槃縣名屬安定郡今涇州縣宋白曰滑州潘原縣漢隂槃縣地}
馮異率諸將拒之囂又令别將下隴攻祭遵於汧並無利而還|{
	考異曰帝紀六年冬隗囂將行廵寇扶風馮異拒破之馮異傳六年夏諸將上隴為隗囂所敗乃詔異軍栒邑未及至囂乘勝使王元行廵將二萬人下隴分遣廵取栒邑異即先據栒邑破廵又云祭遵亦破王元於汧隗囂傳侵三輔事亦同按此文勢緣諸將才敗還隗囂即遣二將追之故得云乘勝又云馮異未及至栒邑也然則馮異祭遵之破王元行廵實在六年明矣至十年八月紀又有隗囂寇安定馮異祭遵擊却之此即隗囂傳所書秋囂侵安定至隂槃馮異拒之又令别將攻祭遵於汧兵並無利者也據此是囂兩歲各嘗攻馮異祭遵矣故遵傳亦云數挫隗囂也而袁紀不載六年事併在七年秋紀之且傳云囂乘勝若事已一年安可云乘勝又馮異何緣稽緩爾久不至栒邑故知袁紀誤矣}
帝將自征隗囂先戒竇融師期會遇雨道斷且囂兵已退乃止帝令來歙以書招王遵遵來降|{
	降戶江翻下同}
拜太中大夫封向義侯 冬盧芳以事誅其五原太守李興兄弟其朔方太守田颯|{
	颯音立守式又翻下同}
雲中太守喬扈各舉郡降|{
	前代錄匈奴貴姓喬氏代為輔相}
帝令領職如故 帝好圖讖|{
	讖楚譛翻}
與鄭興議郊祀事曰吾欲以讖斷之|{
	好呼到翻斷下亂翻}
何如對曰臣不為讖帝怒曰卿不為讖非之邪興惶恐曰臣於書有所未學而無所非也帝意乃解 南陽太守杜詩|{
	郡國志南陽郡在雒陽南七百里}
政治清平|{
	治直吏翻}
興利除害百姓便之又修治陂池|{
	治直之翻}
廣拓土田郡内比室殷足|{
	比薄必翻又毗至翻}
時人方於召信臣|{
	方比也召信臣事見二十九卷元帝竟寧元年召讀曰邵}
南陽為之語曰前有召父後有杜母

八年春來歙將二千餘人伐山開道從番須囘中徑襲畧陽|{
	賢曰畧陽縣名屬天水郡故城在今秦州隴城縣西北番音盤宋白曰畧陽道在隴城縣東六十里即故冀城魏黄初中改為隴城時隗囂居冀以地理考之當從宋說}
斬隗囂守將金梁|{
	姓譜金古金天氏之後又漢金日磾本匈奴休屠王子以祭天金人為金氏}
囂大驚曰何其神也帝聞得畧陽甚喜曰畧陽囂所依阻心腹已壞則制其支體易矣|{
	易以䜴翻}
吳漢等諸將聞歙據畧陽爭馳赴之上以為囂失所恃亡其要城埶必悉以精鋭來攻曠日久圍而城不拔士卒頓敝乃可乘危而進皆追漢等還隗囂果使王元拒隴坻行廵守番須口王孟塞雞頭道|{
	賢曰雞頭山道也一名崆峝山在原州西塞悉則翻}
牛邯軍瓦亭|{
	賢曰安定烏氏縣有瓦亭故關有瓦亭川水在今原州南杜佑曰瓦亭關在唐原州之蕭關蕭關漢朝那縣地邯下甘翻}
囂自悉其大衆數萬人圍畧陽公孫述遣將李育田弇助之斬山築堤激水灌城來歙與將士固死堅守矢盡發屋斷木以為兵|{
	斷丁管翻下同}
囂盡鋭攻之累月不能下夏閏四月帝自將征隗囂光禄勲汝南郭憲諫曰東方初定車駕未可遠征乃當車拔佩刀以斷車靷|{
	靷在馬胷音胤}
帝不從西至漆|{
	漆縣屬右扶風以漆水名縣杜佑曰新平漢漆縣地}
諸將多以王師之重不宜遠入險阻計冘豫未決|{
	冘與猶同}
帝召馬援問之援因說隗囂將帥有土崩之埶兵進有必破之狀|{
	說如字}
又於帝前聚米為山谷指畫形執開示衆軍所從道徑往來分析昭然可曉帝曰虜在吾目中矣明旦遂進軍至高平第一|{
	郡國志高平縣有第一城}
竇融率五郡太守及羌虜小月氏等步騎數萬|{
	月氏為匈奴所破餘種西踰葱嶺其不能去者保南山號小月氏氏音支}
輕重五千餘兩|{
	重直用翻兩音亮}
與大軍會是時軍旅草創諸將朝會禮容多不肅|{
	朝直遥翻}
融先遣從事問會見儀適|{
	賢曰猶言儀注余謂適當也會見之儀各有當也見賢遍翻}
帝聞而善之以宣告百僚乃置酒高會待融等以殊禮|{
	殊異也絶也謂待之之禮異絶於羣臣也}
遂共進軍數道上隴|{
	上時掌翻}
使王遵以書招牛邯下之拜邯太中大夫於是囂大將十三人屬縣十六|{
	地理志天水郡十六縣}
衆十餘萬皆降囂將妻子犇西城從楊廣|{
	賢曰西城縣名屬漢陽郡一名始昌城在今秦州上邽縣西南余據地理志西縣本屬隴西郡後乃改屬漢陽西城者西縣城也以西城為縣名誤矣明帝永平十七年方改天水為漢陽}
而田弇李育保上邽|{
	上邽縣屬天水郡弇古含翻}
畧陽圍解帝勞賜來歙|{
	勞力到翻}
班坐絶席在諸將之右|{
	專席而坐於諸將之上不與諸坐者並也}
賜歙妻縑千匹|{
	毛晃曰縑并絲繒又絹也}
進幸上邽詔告隗囂曰若束手自詣父子相見保無他也若遂欲為黥布者亦自任也|{
	謂必不歸降如黥布云欲為帝亦任之也}
囂終不降於是誅其子恂使吳漢岑彭圍西城耿弇蓋延圍上邽以四縣封竇融為安豐侯|{
	融封安豐陽泉蓼安風四縣皆屬廬江郡}
弟友為顯親侯|{
	郡國志漢陽郡有顯親縣賢曰故城在今秦州成紀縣東南帝置顯親縣以封友褒顯竇氏有孝文皇后之親也}
及五郡太守皆封列侯|{
	竺曾助義侯梁統成義侯史苞褒義侯庫鈞輔義侯辛肜扶義侯}
遣西還所鎮融以久專方面懼不自安數上書求代|{
	數所角翻下同}
詔報曰吾與將軍如左右手耳數執謙退何不曉人意勉循士民|{
	循撫循也順也}
無擅離部曲|{
	離智力翻}
潁川盜賊羣起寇沒屬縣河東守兵亦叛京師騷動|{
	郡國志潁川郡在雒陽東南五百里河東郡在雒陽西北五百里}
帝聞之曰吾悔不用郭子横之言|{
	郭憲字子横}
秋八月帝自上邽晨夜東馳賜岑彭等書曰兩城若下便可將兵南擊蜀虜人苦不知足旣平隴復望蜀|{
	復扶又翻下同}
每一發兵頭須為白|{
	言苦心於軍事也須與鬚同古字通用}
九月乙卯車駕還宫帝謂執金吾寇恂曰潁川迫近京師|{
	近其靳翻}
當以時定惟念獨卿能平之耳從九卿復出以憂國可也對曰潁川聞陛下有事隴蜀故狂狡乘間相詿誤耳|{
	賢曰狡猾也間古莧翻說文曰詿亦誤也音卦}
如聞乘與南向|{
	乘䋲證翻}
賊必惶怖歸死|{
	怖普布翻}
臣願執鋭前驅帝從之庚申車駕南征潁川盜賊悉降寇恂竟不拜郡百姓遮道曰願從陛下復借寇君一年|{
	恂前為潁川太守故云復借也}
乃留恂長社|{
	長社縣屬潁川郡應劭曰宋人圍長葛是也其社中樹暴長更名長社師古曰長讀如字}
鎮撫吏民受納餘降|{
	降戶江翻}
東郡濟隂盜賊亦起|{
	郡國志東郡去雒陽八百餘里濟隂郡在雒陽東八百里濟子禮翻}
帝遣李通王常擊之以東光侯耿純嘗為東郡太守|{
	東光縣屬渤海郡賢曰今滄州縣}
威信著於衛地|{
	東郡衛地也}
遣使拜太中大夫使與大兵會東郡東郡聞純入界盜賊九千餘人皆詣純降大兵不戰而還璽書復以純為東郡太守|{
	璽斯氏翻}
戊寅車駕還自潁川 安丘侯張步將妻子逃犇臨淮與弟弘藍欲招其故衆乘船入海琅邪太守陳俊追討斬之 冬十月丙午上行幸懷十一月乙丑還雒陽 楊廣死隗囂窮困其大將王捷别在戎丘|{
	水經註戎丘城在西城西北戎溪水逕其南}
登城呼漢軍曰為隗王城守者皆必死無二心|{
	為于偽翻}
願諸軍亟罷請自殺以明之遂自刎死|{
	刎扶粉翻}
初帝敕吳漢曰諸郡甲卒但坐費糧食若有逃亡則沮敗衆心|{
	沮在呂翻敗蒲邁翻}
宜悉罷之漢等貪并力攻囂遂不能遣糧食日少吏士疲役逃亡者多岑彭壅谷水灌西城城未沒丈餘會王元行廵周宗將蜀救兵五千餘人乘高卒至|{
	卒讀曰猝}
鼔譟大呼曰百萬之衆方至漢軍大驚未及成陳|{
	呼火故翻陳讀曰陣}
元等决圍殊死戰遂得入城迎囂歸冀吳漢軍食盡乃燒輜重引兵下隴蓋延耿弇亦相隨而退|{
	重直用翻蓋古盍翻}
囂出兵尾擊諸營|{
	尾撃謂尋其後而擊之也}
岑彭為後拒諸將乃得全軍東歸唯祭遵屯汧不退|{
	汧口堅翻}
吳漢等復屯長安岑彭還津鄉於是安定北地天水隴西復反為囂|{
	為于偽翻}
校尉太原温序為囂將苟宇所獲|{
	姓譜唐叔虞之子受封於河内温因以命族又郤至食采於温號温季因以為族據序傳序為護羌校尉行部至襄武為苟宇所獲 考異曰按序傳及袁紀皆稱序為護羌校尉檢西羌傳九年方置此官牛邯為之又云邯卒職省則序無緣作護羌今但云校尉}
宇曉譬數四欲降之序大怒叱宇等曰虜何敢迫脅漢將因以節撾殺數人|{
	撾職瓜翻擊也}
宇衆爭欲殺之宇止之曰此義士死節可賜以劍序受劍銜須於口顧左右曰旣為賊所殺無令須汙土|{
	汙烏故翻}
遂伏劒而死從事王忠持其喪歸雒陽詔賜以冢地拜三子為郎 十二月高句麗王遣使朝貢帝復其王號|{
	王莽貶高句麗為侯今復其王號句音如字又音駒又巨俱翻}
是歲大水九年春正月潁陽成侯祭遵薨於軍|{
	潁陽縣屬潁川郡}
詔馮異并將其營遵為人亷約小心克己奉公賞賜盡與士卒約束嚴整所在吏民不知有軍取士皆用儒術對酒設樂必雅歌投壺|{
	賢曰雅歌謂歌雅詩也禮記投壺經曰壺頸脩七寸腹脩五寸口徑二寸半容斗五升壺中實小豆馬為其矢之躍而出也矢以柘若棘長二尺八寸無去其皮取其堅而重投之勝者飲不勝者以為優劣也}
臨終遺戒薄葬問以家事終無所言帝愍悼之尤甚遵喪至河南車駕素服臨之望哭哀慟還幸城門閲過喪車涕泣不能已喪禮成復親祠以太牢|{
	復扶又翻下同}
詔大長秋謁者河南尹護喪事大司農給費|{
	皇后卿曰將行秦官也景帝中六年更名大長秋師古曰秋者收成之時長者恒久之義故以為皇后官名西都或用中人或用士人東都之後純用閹人矣}
至葬車駕復臨之旣葬又臨其墳存見夫人室家其後朝會帝每歎曰安得憂國奉公如祭征虜者乎|{
	遵為征虜將軍}
衛尉銚期曰陛下至仁哀念祭遵不已羣臣各懷慚懼|{
	言帝念祭遵屢以為言羣臣愧不如遵各懷懼也銚音姚}
帝乃止 隗囂病且餓餐糗糒|{
	鄭康成曰糗熬大豆與米也糒乾飯糗去久翻又丘救翻糒音備}
恚憤而卒|{
	恚於避翻卒子恤翻}
王元周宗立囂少子純為王總兵據冀公孫述遣將趙匡田弇助純帝使馮異擊之 公孫述遣其翼江王田戎大司徒任滿南郡太守程汎將數萬人下江關|{
	任音壬}
擊破馮駿等軍遂拔巫及夷道夷陵|{
	五年岑彭留馮駿軍江州分屯夷道夷陵巫縣亦屬南郡}
因據荆門虎牙|{
	水經註曰江水東歷荆門虎牙之間荆門山在南上合下開其狀似門虎牙山在北石壁色紅間有白文類牙故以名也此二山楚之西塞也賢曰在今峡州夷陵縣東南宜都縣西北今猶有故城基址在山上}
横江水起浮橋關樓立櫕柱以絶水道|{
	關樓范書作闘僂猶今城上敵樓也櫕徂官翻叢木為柱曰櫕柱又作管翻}
結營跨山以塞陸路|{
	塞悉則翻}
拒漢兵夏六月丙戌帝幸緱氏登轘轅|{
	緱氏縣屬河南尹縣有緱氏山轘轅山轘轅坂並在雒陽之東南緱工侯翻轘音環}
吳漢率王常等四將軍兵五萬餘人擊盧芳將賈覽閔堪於高柳|{
	高柳縣屬代郡賢曰故城在今雲州定襄縣水經註曰高柳在代中其山重巒疊巘霞舉雲高連山隱隱東出遼塞}
匈奴救之漢軍不利於是匈奴轉盛鈔暴日增|{
	鈔楚交翻}
詔朱祜屯常山王常屯涿郡破姦將軍侯進屯漁陽以討虜將軍王霸為上谷太守以備匈奴 帝使來歙悉監護諸將屯長安|{
	監古銜翻}
太中大夫馬援為之副歙上書曰公孫述以隴西天水為藩蔽故得延命假息|{
	息氣息也}
今二郡平蕩則述智計窮矣宜益選兵馬儲積資糧今西州新破兵人疲饉若招以財穀則其衆可集臣知國家所給非一用度不足然有不得已也帝然之於是詔於汧積穀六萬斛秋八月來歙率馮異等五將軍討隗純於天水 驃騎將軍杜茂與賈覽戰於繁畤|{
	賢曰繁畤縣屬鴈門郡今代州縣畤音止余按唐代州繁畤雖存漢縣名然非古繁畤也}
茂軍敗績 諸羌自王莽末入居塞内金城屬縣多為所有隗囂不能討因就慰納發其衆與漢相拒司徒掾班彪上言|{
	續漢志司徒掾屬三十一人掾比三百石屬比二百石}
今涼州部皆有降羌|{
	降戶江翻}
羌胡被髪左祍而與漢人雜處習俗旣異言語不通數為小吏黠人所見侵奪窮恚無聊故致反叛夫蠻夷寇亂皆為此也|{
	被皮義翻處昌呂翻數所角翻黠下八翻為于偽翻}
舊制益州部置蠻夷騎都尉|{
	武帝開西南夷置一都尉}
幽州部置領烏桓校尉涼州部置護羌校尉皆持節領護|{
	應劭曰漢官護烏桓護羌校尉比二千石擁節長史一人司馬二人皆六百石校戶教翻}
治其怨結|{
	治直之翻}
歲時廵行|{
	行下孟翻}
問所疾苦又數遣使譯通導動静使塞外羌夷為吏耳目州郡因此可得警備今宜復如舊以明威防帝從之以牛邯為護羌校尉 盜殺隂貴人母鄧氏及弟訢|{
	訢許靳翻}
帝甚傷之封貴人弟就為宣恩侯|{
	帝追爵貴人父陸為宣恩哀侯以就嗣哀侯後漢舊制惟皇后父封侯貴人未正位中宫而追爵其父非舊也}
復召就兄侍中興欲封之置印綬於前興固讓曰臣未有先登陷陳之功|{
	復扶又翻陳讀曰陣}
而一家數人竝蒙爵土令天下觖望|{
	賢曰觖音羌志翻前書音義曰觖猶冀也一音决猶望之也}
誠所不願帝嘉之不奪其志貴人問其故興曰夫外戚家苦不知謙退嫁女欲配侯王取婦眄睨公主|{
	取讀曰娶}
愚心實不安也富貴有極人當知足夸奢益為觀聽所譏貴人感其言深自降挹|{
	以器俯而取水曰挹人之謙下者亦曰挹}
卒不為宗親求位|{
	卒子恤翻為于偽翻}
帝召寇恂還以漁陽太守郭伋為潁川太守伋招降山賊趙宏召吳等數百人皆遣歸附農|{
	附農者附於農籍也召讀曰邵}
因自劾專命|{
	賢曰謂擅放降賊也劾戶槩翻}
帝不以咎之後宏吳等黨與聞伋威信遠自江南或從幽冀不期俱降駱驛不絶 莎車王康卒弟賢立攻殺拘彌西夜王|{
	拘彌即前漢之扞罙唐曰寧彌西夜國去雒陽萬四千四百里}
而使康兩子王之|{
	王于况翻}


十年春正月吳漢復率捕虜將軍王霸等四將軍六萬人出高柳擊賈覽|{
	復扶又翻}
匈奴數千騎救之連戰於平城下|{
	平城縣屬鴈門郡}
破走之 夏陽節侯馮異等|{
	馮異傳云封異陽夏侯賢曰夏音賈馬武傳末列二十八將官位姓名曰夏陽侯馮異陽夏縣屬淮陽郡夏陽縣屬左馮翊未知孰是夏陽之夏戶雅翻}
與趙匡田弇戰且一年皆斬之隗純未下諸將欲且還休兵異固持不動共攻落門|{
	天水冀縣有落門聚有落門山賢曰在今渭州隴西縣東南}
未拔夏異薨於軍 秋八月己亥上幸長安 初隗囂將安定高峻擁兵據高平第一|{
	帝之上隴也遣馬援招降峻及吳漢等軍退峻亡歸故營復助囂拒隴坻}
建威大將軍耿弇等圍之一歲不拔帝自將征之寇恂諫曰長安道里居中|{
	賢曰從雒陽至高平長安為中}
應接近便安定隴西必懷震懼此從容一處可以制四方也|{
	從千容翻}
今士馬疲倦方履險阻非萬乘之固也前年潁川可為至戒帝不從進幸汧峻猶不下帝遣寇恂往降之|{
	降戶江翻下同}
恂奉璽書至第一峻遣軍師皇甫文出謁辭禮不屈恂怒將誅之諸將諫曰高峻精兵萬人率多彊弩西遮隴道連年不下今欲降之而反戮其使|{
	使疏吏翻}
無乃不可乎恂不應遂斬之遣其副歸告峻曰軍師無禮已戮之矣欲降急降不欲固守峻惶恐即日開城門降諸將皆賀因曰敢問殺其使而降其城何也恂曰皇甫文峻之腹心其所取計者也今來辭意不屈必無降心全之則文得其計殺之亡其膽|{
	謂文死則峻亡其膽也}
是以降耳諸將皆曰非所及也 冬十月來歙與諸將攻破落門周宗行廵苟宇趙恢等將隗純降王元犇蜀徙諸隗於京師以東|{
	隗純降而徙其族以其西州彊宗恐其後復能為變也}
後隗純與賓客亡入胡至武威捕得誅之 先零羌與諸種寇金城隴西|{
	零音憐種章勇翻}
來歙率蓋延等進擊大破之|{
	蓋古盍翻}
斬首虜數千人於是開倉廪以賑飢乏隴右遂安而涼州流通焉|{
	涼州諸郡至京師皆須度隴隴右安則涼州之路流通}
庚寅車駕還宫

十一年春三月己酉帝幸南陽 |{
	考異曰帝紀己酉幸南陽庚午車駕還宫上有二月己卯袁紀三月己酉幸南陽以長歷考之二月壬申朔己卯八日也己酉庚午皆在三月盖帝紀己酉上脱三月字今從袁紀}
還幸章陵庚午車駕還宫 岑彭屯津鄉數攻田戎等不克|{
	數所角翻}
帝遣吳漢率誅虜將軍劉隆等三將發荆州兵凡六萬餘人騎五千匹與彭會荆門彭裝戰船數千艘|{
	艘蘇遭翻}
吳漢以諸郡棹卒多費糧穀欲罷之|{
	棹卒持棹行船者也}
彭以為蜀兵盛不可遣上書言狀帝報彭曰大司馬習用步騎不曉水戰荆門之事一由征南公為重而已|{
	彭為征南大將軍故稱為征南公}
閏月岑彭令軍中募攻浮橋先登者上賞於是偏將軍魯奇應募而前時東風狂急魯奇船逆流而上直衝浮橋|{
	上時掌翻}
而櫕柱有反杷鉤|{
	反杷鉤者既鉤住敵船使不得退又逆拒之使不得進也}
奇船不得去奇乘埶殊死戰因飛炬焚之風怒火盛橋樓崩燒岑彭悉軍順風並進所向無前蜀兵大亂溺死者數千人斬任滿生獲程汎而田戎走保江州彭上劉隆為南郡太守|{
	先以隆守南郡而上奏也上時掌翻}
自率輔威將軍臧宫驍騎將軍劉歆長驅入江關|{
	華陽國志巴楚相攻故置江關舊在赤甲城後移在江川南岸對白帝城故城在今夔州魚復縣南即古捍關也杜佑曰巴山縣古扞關如此則别是一處}
令軍中無得虜掠所過百姓皆奉牛酒迎勞彭復讓不受|{
	勞力到翻復扶又翻}
百姓大喜争開門降|{
	降戶江翻}
詔彭守益州牧所下郡輒行太守事彭若出界即以太守號付後將軍|{
	後將軍者將兵繼彭後而進者也}
選官屬守州中長吏彭到江州以其城固糧多難卒拔|{
	卒讀曰猝}
留馮駿守之自引兵乘利直指墊江攻破平曲|{
	賢曰墊江縣名屬巴郡今忠州縣也按宋白續通典忠州墊江縣本後漢臨江縣地後魏恭帝分臨江置墊江縣合州石鏡縣本漢墊江縣凡合州管下諸縣皆漢墊江地也墊音徒協翻平曲地闕}
收其米數十萬石吳漢留夷陵裝露橈繼進|{
	爾雅曰檝謂之橈露橈謂露檝在外人在船中橈音饒}
夏先零羌寇臨洮|{
	臨洮縣屬隴西郡零音憐洮音韜}
來歙薦馬援為隴西太守|{
	郡國志隴西郡在雒陽西二千二百二十里}
擊先零羌大破之 公孫述以王元為將軍使與領軍環安拒河池|{
	姓譜環姓也楚環列尹之後又楚有賢者環淵河池縣屬武都郡}
六月來歙與蓋延等進攻元安大破之遂克下辨|{
	辨皮莧翻}
乘勝遂進蜀人大懼使刺客刺歙未殊|{
	未殊謂未絶也客刺七亦翻}
馳召蓋延延見歙因伏悲哀不能仰視歙叱延曰虎牙何敢然|{
	延為虎牙大將軍故以虎牙稱之}
今使者中刺客無以報國|{
	中竹仲翻下同}
故呼巨卿欲相屬以軍事|{
	蓋延字巨卿屬之欲翻}
而反效兒女子涕泣乎刃雖在身不能勒兵斬公邪延牧淚強起受所誡|{
	強其兩翻}
歙自書表曰臣夜人定後|{
	日入而羣動息故甲夜謂之人定}
為何人所賊傷中臣要害|{
	何人謂不知何人也}
臣不敢自惜誠恨奉職不稱以為朝廷羞|{
	稱尺證翻}
夫理國以得賢為本太中大夫段襄骨鯁可任|{
	賢曰骨鯁謂正直也說文曰鯁魚骨也食骨留咽中為鯁}
願陛下裁察又臣兄弟不肖終恐被罪陛下哀憐數賜教督|{
	被皮義翻數所角翻}
投筆抽刃而絶|{
	凡為人所刺者刃在身猶未死抽刃則氣絶矣}
帝聞大驚省書攬涕|{
	省悉景翻}
以揚武將軍馬成守中郎將代之歙喪還洛陽乘輿縞素臨弔送葬|{
	乘䋲證翻}
趙王良從帝送歙喪還入夏城門|{
	雒陽十二城門夏門位在亥}
與中郎將張邯爭道叱邯旋車又詰責門候|{
	百官志城門校尉掌雒陽十二城門每門候一人邯下甘翻}
使前走數千步司隸校尉鮑永劾奏良無藩臣禮大不敬良尊戚貴重而永劾之|{
	劾戶槩翻又戶得翻}
朝廷肅然永辟扶風鮑恢為都官從事|{
	百官志司隸校尉從事吏十二人都官從事主察舉百官犯法者蔡質漢儀曰都官主雒陽朝會與三府掾同}
恢亦抗直不避彊禦帝嘗曰貴戚且歛手以避二鮑永行縣到霸陵|{
	司隸校尉主三河三輔弘農霸陵縣屬京兆行下孟翻}
路經更始墓下拜哭盡哀而去西至扶風椎牛上苟諫冢|{
	苟諫保護鮑永事見三十六卷更始二年上時掌翻}
帝聞之意不平問公卿曰奉使如此何如|{
	武帝置十三州刺史皆部使者也司隸今出所部故言奉使使疏吏翻}
太中大夫張湛對曰仁者行之宗忠者義之主也仁不遺舊忠不忘君行之高者也|{
	行下孟翻}
帝意乃釋 帝自將征公孫述秋七月次長安公孫述使其將延岑呂鮪王元公孫恢悉兵拒廣漢

及資中|{
	廣漢縣屬廣漢郡賢曰資中縣名屬犍為郡其地在今資州資陽縣宋白曰資州諸縣皆漢資中地磐石縣資州治所漢資中故城也}
又遣將侯丹率二萬餘人拒黄石|{
	賢曰即黄石灘也水經注曰江水自涪陵東出百里而屇于黄石在今涪州涪陵縣杜佑曰今謂之横石灘}
岑彭使臧宮將降卒五萬從涪水上平曲拒延岑|{
	涪音浮杜佑音符水經涪水出廣漢屬國剛氏道徼外東南流逕涪縣北又東南逕綿竹縣北即臧宫遡涪至平陽鄉之地涪水又東南與建始水合水發平洛郡西溪西南流屈而東西流意此即平曲也上時掌翻}
自分兵浮江下還江州泝都江而上|{
	賢曰都江成都江也宋白曰郫江一名都江一名成都江}
襲擊侯丹大破之因晨夜倍道兼行二千餘里徑拔武陽|{
	賢曰武陽縣屬犍為郡故城在今隆州隆山縣東也又曰故城在今眉州劉昫曰唐陵州仁夀縣漢武陽縣地或曰今眉州眉山彭山縣本漢武陽縣地杜佑曰漢武陽縣故城在嘉州綏山縣東}
使精騎馳擊廣都去成都數十里|{
	賢曰廣都縣名屬蜀郡故城在今益州成都縣東南宋白曰蜀志漢元朔二年置廣都縣隋仁夀元年避煬帝諱改為雙流唐龍朔三年析雙流縣又置廣都縣於舊縣南一十二里}
埶若風雨所至皆犇散初述聞漢兵在平曲故遣大兵逆之及彭至武陽繞出延岑軍後蜀地震駭述大驚以杖擊地曰是何神也延岑盛兵於沅水|{
	帝紀作沈水此作沅承臧宫傳之誤也賢曰水經註曰沈水出廣漢縣下入涪水本或作況水及沅水者並非余據今潼川府通泉縣北有沉水}
臧宫衆多食少轉輸不至降者皆欲散畔郡邑復更保聚觀望成敗|{
	復扶又翻}
宫欲引還恐為|{
	音去聲}
所反|{
	賢曰反音翻}
會帝遣謁者將兵詣岑彭有馬七百匹宫矯制取以自益晨夜進兵多張旗幟登山鼓譟右步左騎|{
	幟昌志翻騎奇寄翻}
挾船而引呼聲動山谷岑不意漢軍卒至|{
	呼火故翻卒讀曰猝}
登山望之大震恐宫因縱擊大破之斬首溺死者萬餘人水為之濁|{
	為于偽翻}
延岑犇成都其衆悉降|{
	降戶江翻}
盡獲其兵馬珍寶自是乘勝追北|{
	賢曰人好陽而惡隂北方幽隂之地故軍敗者皆謂之北史記樂書曰北者敗也近代音北為背失其指矣}
降者以十萬數軍至陽鄉|{
	臧宫傳作平陽郷此逸平字水經註曰臧宫泝涪至平陽公孫述將王元降遂拔綿竹涪水經綿竹縣北則平陽郷當在綿竹縣界}
王元舉衆降帝與公孫述書陳言禍福示以丹青之信述省書太息以示所親太常常少光禄勲張隆皆勸述曰廢興命也豈有降天子哉左右莫敢復言少隆皆以憂死|{
	省悉景翻少詩照翻復扶又翻}
帝還自長安 冬十月公孫述使刺客詐為亡奴降岑彭夜刺殺彭|{
	刺殺之刺七亦翻}
太中大夫監軍鄭興領其營以俟吳漢至而授之彭持軍整齊秋豪無犯卭穀王任貴聞彭威信數千里遣使迎降|{
	任貴降述事見四十卷元年卭渠恭翻任音壬}
會彭已被害|{
	被皮義翻}
帝盡以任貴所獻賜彭妻子蜀人為立廟祠之|{
	為于偽翻}
馬成等破河池遂平武都|{
	郡國志武都郡在雒陽西一千九百里}
先零諸種羌數萬人屯聚寇鈔拒浩亹隘|{
	零音憐種章勇翻鈔楚交翻浩亹音告門}
成與馬援深入討擊大破之徙降羌置天水隴西扶風是時朝臣以金城破羌之西|{
	破羌縣屬金城郡賢曰故城在今鄯州湟水縣西宋白曰湟水縣本漢破羌縣地後魏得羌地於此置西都縣隋改為湟水}
塗遠多寇議欲棄之馬援上言破羌以西城多堅牢易可依固|{
	易以䜴翻}
其田土肥壤|{
	賢曰無塊曰壤}
灌漑流通如令羌在湟中則為害不休不可棄也帝從之民歸者三千餘口援為置長吏繕城郭|{
	為于偽翻長知兩翻}
起塢候|{
	字林曰塢小障也字或作嗚一古翻}
開溝洫|{
	洫況域翻}
勸以耕牧郡中樂業|{
	樂音洛}
又招撫塞外氐羌皆來降附援奏復其侯王君長帝悉從之乃罷馬成軍 十二月吳漢自夷陵將三萬人泝江而上伐公孫述|{
	上時掌翻}
郭伋為并州牧過京師|{
	過古禾翻}
帝問以得失伋曰選補衆職當簡天下賢俊不宜專用南陽人是時在位多鄉曲故舊故伋言及之

資治通鑑卷四十二
