\chapter{資治通鑑卷八十六}
宋 司馬光 撰

胡三省 音註

晉紀八|{
	起旃蒙赤奮若盡著雍執徐凡四年}


孝惠皇帝下

永興二年夏四月張方廢羊后 游楷等攻皇甫重累年不能克|{
	游楷等自太安二年攻皇甫重至是首尾二年}
重遣其養子昌求救於外昌詣司空越越以太宰顒新與山東連和|{
	事見上卷上年}
不肯出兵昌乃與故殿中人楊篇|{
	故殿中人舊屬二衛部曲者}
詐稱越命迎羊后於金墉城入宫以后令發兵討張方奉迎大駕|{
	是年四月張方廢羊后其時方已奉帝入關蓋以威令遥脇留臺百官使廢羊后耳今皇甫昌迎后入宫欲發兵討方特以是起兵非因方在洛而討之也}
事起倉猝百官初皆從之俄知其詐相與誅昌顒請遣御史宣詔諭重令降|{
	降戶江翻下同}
重不奉詔先是城中不知長沙厲王及皇甫商已死|{
	長沙厲王死見上卷上年皇甫商死見上卷太安二年先悉薦翻}
重獲御史騶人|{
	因御史來宣詔獲其騶人騶側鳩翻廐御也晉制諸公給騶八人下至御史各有差齊王融曰車前無八騶何得稱丈夫則騶蓋辟車之卒}
問曰我弟將兵來欲至未騶人曰已為河間王所害重失色立殺騶人於是城中知無外救共殺重以降顒以馮翊太守張輔為秦州刺史|{
	顒以破劉沈之功用張輔}
六月甲子安豐元侯王戎薨于郟|{
	王戎犇郟見上卷上年郟音夾}
張輔至秦州殺天水太守封尚欲以立威又召隴西太守韓稚|{
	守式又翻}
稚子朴勒兵擊輔輔軍敗死涼州司馬楊胤言於張軌曰韓稚擅殺刺史明公杖鉞一方不可不討軌從之遣中督護汜瑗帥衆二萬討稚稚詣軌降|{
	中督護中軍督護也汜音凡瑗于眷翻帥讀曰率下同}
未幾|{
	幾居豈翻}
鮮卑若羅抜能寇涼州軌遣司馬宋配擊之斬抜能俘十餘萬口威名大振|{
	史言張軌能尊主攘夷以致彊盛}
漢王淵攻東嬴公騰騰復乞師於拓跋猗㐌|{
	復扶又翻㐌徒河翻}
衛操勸猗㐌助之猗㐌帥輕騎數千救騰斬漢將綦母豚|{
	母音無綦母複姓北狄傳匈奴國人有綦母氏勒氏皆勇健好反叛 考異曰後魏書桓帝紀及劉淵傳皆云淵南走蒲子按晉載記淵無走蒲子事下云自離石遷黎亭蓋後魏書夸誕妄言耳}
詔假猗㐌大單于|{
	單音蟬}
加操右將軍甲申猗㐌卒子普根代立 東海中尉劉洽以張方刼遷車駕|{
	晉諸王國有郎中令中尉大農為三卿張方劫遷車駕事見上卷上年}
勸司空越起兵討之秋七月越傳檄山東征鎮州郡云欲糾帥義旅奉迎天子還復舊都|{
	舊都謂洛陽}
東平王楙聞之懼長史王脩說楙曰東海宗室重望今興義兵公宜舉徐州以授之則免於難|{
	說輸芮翻難乃旦翻}
且有克讓之美矣楙從之越乃以司空領徐州都督楙自為兖州刺史詔即遣使者劉䖍授之|{
	楙督徐州始八十四卷永寧元年去年范陽王虓以苟晞行兖州晞留許昌未及至州而楙自領之}
是時越兄弟並據方任|{
	越弟略都督青州模都督冀州}
於是范陽王虓|{
	虓虛交翻}
及王浚等共推越為盟主越輒選置刺史以下|{
	輒專也}
朝士多赴之|{
	朝士赴越者不從帝在長安者也朝直遥翻}
成都王頴既廢|{
	頴廢見上卷上年}
河北人多憐之|{
	頴鎮鄴初有時譽後雖以驕侈致禍河北之人厭亂而思舊故多憐之}
頴故將公師藩等自稱將軍起兵於趙魏衆至數萬初上黨武鄉羯人石勒有膽力善騎射|{
	武鄉縣晉置屬上黨郡後石勒分置武鄉郡劉昫曰唐潞州武鄉縣漢河東之垣縣也唐遼州榆社縣分晉武鄉縣置載記曰勒匈奴别部羌渠之胄又匈奴傳曰北狄入居塞内者有十九種羯其一也羯居謁翻}
并州大饑建威將軍閻粹說東嬴公騰|{
	說輸芮翻}
執諸胡於山東賣充軍實勒亦被掠賣為茌平人師懽奴|{
	荏平縣前漢屬東郡後漢屬濟北國晉屬平原國應劭曰在茌山之平地置縣意其地當在唐齊州博州界劉昫曰茌平縣併入唐博州聊城縣被皮義翻師古曰茌音仕疑翻}
懽奇其狀貌而免之懽家鄰於馬牧勒乃與牧帥汲桑結壯士為羣盜|{
	帥所類翻}
及公師藩起桑與勒帥數百騎赴之|{
	帥讀曰率}
桑始命勒以石為姓勒為名|{
	石勒始此}
藩攻陷郡縣殺二千石長吏|{
	長知兩翻}
轉前攻鄴平昌公模甚懼范陽王虓遣其將苟晞救鄴與廣平太守譙國丁紹共擊藩走之|{
	漢武帝置平于國宣帝改為廣平國後漢光武省屬鉅鹿郡魏文帝黄初二年復置廣平郡唐為洺州之地}
八月辛丑大赦 司空越以琅邪王睿為平東將軍監徐州諸軍事留守下邳|{
	監工衘翻}
睿請王導為司馬委以軍事 |{
	考異曰元帝鎮下邳請導為安東司馬按元帝時為平東及徙揚州乃為安東耳或者平字誤為安或後為安東司馬故但云司馬}
越帥甲士三萬西屯蕭縣|{
	蕭縣自漢以來屬沛郡唐屬徐州}
范陽王虓自許屯于滎陽|{
	許即許昌}
越承制以豫州刺史劉喬為冀州刺史以范陽王虓領豫州刺史喬以虓非天子命發兵拒之虓以劉琨為司馬越以劉蕃為淮北護軍劉輿為頴川太守|{
	輿琨蕃之子也}
喬上尚書列輿兄弟罪惡因引兵攻許|{
	豫州刺史時治項上時掌翻}
遣長子祐將兵拒越於蕭縣之靈壁|{
	長知兩翻}
越兵不能進東平王楙在兖州徵求不已郡縣不堪命范陽王虓遣苟晞還兖州|{
	虓用苟晞為兖州刺史見上卷上年}
徙楙都督青州楙不受命背山東諸侯與劉喬合|{
	背蒲妹翻}
太宰顒聞山東兵起甚懼以公師藩為成都王穎起兵壬午表穎為鎮軍大將軍都督河北諸軍事給兵千人以盧志為魏郡太守隨穎鎮鄴欲以撫安之又遣建武將軍呂朗屯洛陽顒發詔令東海王越等各就國越等不從會得劉喬上事|{
	上事者言東海王越等起兵及喬攻許拒越之事上時掌翻}
冬十月丙子下詔稱劉輿迫脅范陽王虓造搆凶逆其令鎮南大將軍劉宏|{
	劉宏都督荆州}
平南將軍彭城王釋|{
	彭城王釋蓋代羊伊屯宛}
征東大將軍劉準|{
	劉準都督揚州}
各勒所統與劉喬并力以張方為大都督統精卒十萬與呂朗共會許昌誅輿兄弟釋宣帝弟子穆王權之孫也|{
	權宣帝弟東武城侯馗之子 考異曰劉喬傳釋作繹帝紀宗室傳皆作釋蓋喬傳誤帝紀八月車騎大將軍劉宏逐平南將軍彭城王釋于宛宏釋傳及衆書皆無之宏傳但云彭城前東奔有不善之言按宏晉室純臣劉喬與范陽搆難宏猶以書和解之以安天下尊王室釋受王命鎮宛而宏肯更自逐之乎據此詔令宏釋共討劉輿疑無宏逐釋事帝紀必誤}
丁丑顒使成都王穎領將軍劉褒等前軍騎將軍石超領北中郎將王闡等據河橋為劉喬繼援進喬鎮東將軍假節劉宏遺喬及司空越書|{
	遺于季翻}
欲使之解怨釋兵同奬王室皆不聽宏又上表曰自頃兵戈紛亂猜禍鋒生疑隙搆於羣王災難延于宗子|{
	難乃旦翻}
今日為忠明日為逆翩其反而|{
	言是非反覆之易冏乂頴顒之事誠如此}
互為戎首|{
	言迭為興戎之首也}
載籍以來骨肉之禍未有如今者也臣竊悲之今邊陲無備豫之儲中華有杼軸之困|{
	詩曰小東大東杼軸其空杼持緯器軸亦作柚皆織具也}
而股肱之臣不惟國體職競尋常|{
	惟思也職主也競爭也八尺曰尋倍尋曰常言所争者尋丈之閒不足為長短也左傳曰争尋常以盡其民}
自相楚剝|{
	楚痛也}
萬一四夷乘虛為變此亦猛虎交鬬自效於卞莊者矣|{
	劉石之禍劉宏盡知之矣}
臣以為宜速發明詔詔越等令兩釋猜嫌各保分局|{
	分扶問翻}
自今以後其有不被詔書擅興兵馬者天下共伐之|{
	被皮義翻下同}
時太宰顒方拒關東倚喬為助不納其言喬乘虛襲許破之劉琨將兵救許不及遂與兄輿及范陽王虓俱犇河北琨父母為喬所執劉宏以張方殘暴知顒必敗乃遣參軍劉盤為都護|{
	盡護行營諸將為都護督護則止督一軍耳}
帥諸軍受司空越節度|{
	帥讀曰率}
時天下大亂宏專督江漢威行南服|{
	南服南方也謂之服者責以服事天子為職}
謀事有成者則曰某人之功如有負敗則曰老子之罪每有興發|{
	興發謂興師動衆調發財賦}
手書守相|{
	守手又翻相息亮翻}
丁寧欵密所以人皆感悅争赴之咸曰得劉公一紙書賢於十部從事|{
	州有部從事部管内諸郡}
前廣漢太守辛冉說宏以從横之事宏怒斬之|{
	益州之破辛冉去羅尚從劉宏冉以事尚者事宏猶將不免於誅况以從横說之邪史言劉宏忠純說輸芮翻從子容翻}
有星孛于北斗|{
	孛蒲内翻}
平昌公模遣將軍宋胄趣河橋|{
	模自鄴遣胄進兵趣七喻翻}
十一月立節將軍周權詐被檄|{
	詐言被司空越檄也}
自稱平西將軍復立羊后洛陽令何喬攻權殺之復廢羊后|{
	復扶又翻}
太宰顒矯詔以羊后屢為姦人所立遣尚書田淑敕留臺賜后死|{
	時荀藩劉暾周馥居留臺}
詔書屢至司隸校尉劉暾等上奏|{
	暾他昆翻上時掌翻}
固執以為羊庶人門戶殘破廢放空宫門禁峻密無緣得與姦人搆亂衆無愚智皆謂其寃今殺一枯窮之人而令天下傷慘何益於治|{
	治直吏翻}
顒怒遣呂朗收暾 |{
	考異曰暾傳云顒遣陳顔呂朗帥騎五千收暾按暾匹夫安用五千騎蓋朗時在洛顒敕使收暾耳說者欲大其事故云爾}
暾犇青州依高密王略然羊后亦以是得免 十二月呂朗等東屯滎陽成都王穎進據洛陽 劉琨說冀州刺史太原温羨使讓位於范陽王虓|{
	魏收曰袁紹曹操為冀州治鄴魏晉治信都杜佑曰治房子說輸芮翻}
虓領冀州遣琨詣幽州乞師於王浚浚以突騎資之|{
	突騎天下精兵燕人致梟騎助漢高祖以破項羽光武得漁陽上谷突騎以平河北 考異曰琨傳曰得突騎八百人按劉喬傳云琨率突騎五千濟河攻喬疑八百太少或因下文迎東海王之數致有此誤今闕疑}
擊王闡於河上殺之琨遂與虓引兵濟河斬石超於滎陽劉喬自考城引退|{
	考城縣屬陳留郡前漢梁國之菑縣也章帝更名晉省後魏置考陽縣及北梁郡北齊郡縣並廢為城安縣隋改曰考城縣屬梁郡至唐屬曹州}
虓遣琨及督護田徽東擊東平王楙於廩邱|{
	廩邱縣前漢屬東郡後漢屬濟隂晉屬濮陽郡為兖州刺史治所賢曰廩邱故城在今濮州雷澤縣北}
楙走還國琨徽引兵東迎越擊劉祐於譙祐敗死喬衆遂潰喬犇平氏 |{
	考異曰帝紀云喬奔南陽按地理志南陽無平氏縣武帝分南陽置義陽郡有西平氏縣或者南陽有東平氏而非縣與 今按前漢書地理志平氏縣屬南陽郡晉書地理志平氏縣屬義陽郡平氏之上有厥西縣沈約宋書地理志南義陽太守領厥西平氏二縣賢曰厥西今二漢無晉太康地志屬義陽以此證之蓋後人傳寫晉書者誤以厥西之西字聯平氏而書之其實晉義陽之平氏即漢南陽之平氏也帝紀所謂喬奔南陽以漢古郡大界言之也劉昫曰唐申州義陽縣漢南陽郡平氏縣之義陽鄉與唐州之桐柏平氏二縣皆漢南陽平氏縣也}
司空越進屯陽武|{
	陽武縣漢屬河南郡晉屬榮陽郡唐屬鄭州}
王浚遣其將祁宏帥突騎鮮卑烏桓為越先驅|{
	帥讀曰率下同}
初陳敏既克石氷|{
	事見上卷太安二年}
自謂勇略無敵有割據江東之志其父怒曰滅我門者必此兒也遂以憂卒敏以喪去職司空越起敏為右將軍前鋒都督越為劉祐所敗|{
	敗補邁翻}
敏請東歸收兵遂據歷陽叛|{
	歷陽縣漢屬九江郡魏改九江曰淮南晉因之今和州即歷陽縣之地宋白曰縣南有歷水故曰歷陽}
吳王常侍甘卓棄官東歸|{
	晉諸王國大國置左右常侍各一人考異曰卓傳云州舉茂才為吳王常侍討石氷以功賜爵都亭侯東海王越引為參軍出補離狐令棄官東歸遇陳敏敏傳云吳王常侍甘卓自洛至按卓為常侍不應討石氷為離狐令不應至洛今從敏傳}
至歷陽敏為子景娶卓女|{
	為于偽翻}
使卓假稱皇太弟令拜敏揚州刺史敏使弟恢及别將錢端等南略江州弟斌東略諸郡|{
	斌音彬}
揚州刺史劉機丹陽太守王曠皆棄城走|{
	時揚州刺史蓋與丹陽太守同治秣陵}
敏遂據有江東以顧榮為右將軍賀循為丹陽内史周玘為安豐太守|{
	安豐縣後漢屬廬江郡魏分廬江為安豐郡其地為唐之壽州安豐霍邱縣玘墟里翻}
凡江東豪傑名士咸加收禮為將軍郡守者四十餘人或有老疾就加秩命循詐為狂疾得免乃以榮領丹陽内史玘亦稱疾不之郡敏疑諸名士終不為己用欲盡誅之榮說敏曰中國喪亂胡夷内侮觀今日之勢不能復振百姓將無遺種|{
	說輸芮翻喪息浪翻種章勇翻}
江南雖經石氷之亂人物尚全榮常憂無孫劉之主|{
	孫劉謂孫權劉備}
有以存之今將軍神武不世勲效已著帶甲數萬舳艫山積|{
	漢武帝紀舳艫千里注云舳船後持柂處艫船前刺櫂處又漢律名船方長為舳艫此言山積蓋取漢律之義}
若能委信君子使各盡懷散蔕芥之嫌|{
	張晏曰蔕芥刺鯁也師古曰蔕音丑介翻}
塞讒諂之口則上方數州可傳檄而定|{
	上方數州謂揚州以西荆江豫梁益等州也塞悉則翻}
不然終不濟也敏命僚佐推己為都督江東諸軍事大司馬楚公加九錫列上尚書|{
	上時掌翻}
稱被中詔|{
	被皮義翻}
自江入沔漢奉迎鑾駕太宰顒以張光為順陽太守|{
	順陽縣前漢曰博山後漢明帝更名順陽屬南陽郡至建安中割南陽右壤為南鄉郡晉太康中立順陽郡以南鄉為縣唐鄧州之臨湍菊潭縣皆順陽郡地}
帥步騎五千詣荆州討敏劉宏遣江夏太守陶侃武陵太守苗光屯夏口|{
	夏戶雅翻}
又遣南平太守汝南應詹督水軍以繼之侃與敏同郡|{
	侃與敏皆廬江人}
又同歲舉吏|{
	同歲舉赴京師}
隨郡内史扈懷|{
	隨縣漢屬南陽郡春秋之隨國也晉武帝分南陽立義陽國後又分義陽立隨郡隋為漢東郡唐為隋州}
言於宏曰侃居大郡統彊兵脫有異志則荆州無東門矣宏曰侃之忠能吾得之已久必無是也侃聞之遣子洪及兄子臻詣宏以自固宏引為參軍資而遣之|{
	既引為參軍又以貨物資送而遣其歸}
曰賢叔征行君祖母年高便可歸也匹夫之交尚不負心况大丈夫乎敏以陳恢為荆州刺史寇武昌宏加侃前鋒督護以禦之侃以運船為戰艦|{
	艦戶黯翻}
或以為不可侃曰用官船擊官賊何為不可侃與恢戰屢破之又與皮初張光苗光共破錢端於長岐|{
	據張光傳長岐之戰光設伏於步路苗光為水軍藏舟船於沔水則長岐當在江夏郡界}
南陽太守衛展說宏曰張光太宰腹心公既與東海宜斬光以明向背|{
	苴蒲妹翻}
宏曰宰輔得失豈張光之罪危人自安君子弗為也乃表光殊勲乞加遷擢 是歲離石大饑漢王淵徙屯黎亭|{
	續漢志上黨郡壺關縣有黎亭書西伯戡黎即此}
就邸閣穀留太尉宏守離石使大司農卜豫運糧以給之

光熙元年|{
	六月帝還洛陽始改元此猶是永興三年}
春正月戊子朔日有食之 初太弟中庶子蘭陵繆播|{
	繆靡幼翻又莫六翻姓也}
有寵於司空越播從弟右衛率胤太宰顒前妃之弟也越之起兵遣播胤詣長安說顒令奉帝還洛|{
	說輸芮翻下因說復說迎說同}
約與顒分陜為伯|{
	陜失冉翻}
顒素信重播兄弟即欲從之張方自以罪重恐為誅首|{
	以剽掠都劫天子西遷也}
謂顒曰今據形勝之地國富兵彊奉天子以號令誰敢不從柰何拱手受制於人顒乃止及劉喬敗顒懼欲罷兵與山東和解恐張方不從猶豫未决方素與長安富人郅輔親善以為帳下督|{
	方傳云初方從山東來甚微賤郅輔厚相供給及貴甚親昵之}
顒參軍河間畢垣嘗為方所侮因說顒曰張方久屯霸上聞山東兵盛盤桓不進|{
	顒遣方與呂朗會劉喬攻許方屯覇上未進而劉喬敗馬融曰盤桓旋也}
宜防其未萌其親信郅輔具知其謀繆播繆胤復說顒宜急斬方以謝山東可不勞而定|{
	復扶又翻}
顒使人召輔垣迎說輔曰張方欲反人謂卿知之王若問卿何辭以對輔驚曰實不聞方反為之奈何垣曰王若問卿但言爾爾|{
	爾爾猶言如此如此也}
不然必不免禍輔入顒問之曰張方反卿知之乎輔曰爾顒曰遣卿取之可乎又曰爾顒於是使輔送書於方因殺之輔既昵於方|{
	昵尼質翻}
持刀而入守閤者不疑方火下發函輔斬其頭還報顒以輔為安定太守送方頭於越以請和越不許宋胄襲河橋樓褒西走平昌公模遣前鋒督護馮嵩會宋胄逼洛陽成都王穎西犇長安至華隂|{
	華隂縣前漢屬京兆後漢晉屬宏農郡唐屬華州華戶化翻}
聞顒已與山東和親留不敢進呂朗屯滎陽劉琨以張方首示之遂降|{
	降戶江翻}
司空越遣祁宏宋胄司馬纂帥鮮卑西迎車駕|{
	帥讀曰率}
以周馥為司隸校尉假節都督諸軍屯澠池|{
	澠彌兖翻}
三月惤令劉伯根反|{
	惤縣自漢以來屬東萊郡拓跋魏省魏收地形志東牟郡黄縣}


|{
	有惤城師古曰惤音堅}
衆以萬數自稱惤公王彌帥家僮從之|{
	帥讀曰率}
伯根以彌為長史彌從父弟桑為東中郎將|{
	從才用翻}
伯根寇臨淄|{
	青州都督治所}
青州都督高密王略使劉暾將兵拒之暾兵敗犇洛陽|{
	暾他昆翻}
略走保聊城|{
	聊城縣漢屬東郡晉屬平原郡唐為博州治所}
王浚遣將討伯根斬之|{
	將即亮翻}
王彌亡入長廣山為羣盜|{
	長廣縣前漢屬琅邪郡後漢屬東萊郡晉武帝咸寧三年置長廣郡長廣縣屬焉隋廢長廣郡及縣更名膠水縣唐屬萊州}
寧州頻歲饑疫死者以十萬計五苓夷彊盛州兵屢敗|{
	五苓夷反事始上卷太安二年苓力丁翻}
吏民流入交州者甚衆夷遂圍州城李毅疾病救援路絶乃上疏|{
	上時掌翻}
言不能式遏寇虐坐待殄斃若不垂矜恤乞降大使及臣尚存加臣重辟|{
	使疏吏翻辟毗亦翻}
若臣已死陳尸為戮朝廷不報積數年子釗自洛往省之|{
	省悉景翻}
未至毅卒毅女秀明達有父風衆推秀領寧州事秀奬厲戰士嬰城固守城中糧盡炙鼠抜草而食之伺夷稍怠輒出兵掩擊破之|{
	伺相吏翻 考異曰懷帝紀永嘉元年五月建寧郡夷攻陷寧州死者三千餘人李雄載記曰南夷李毅固守不降雄誘建寧夷使討之毅病卒城陷殺壯士三千餘人送婦女千口於成都王遜傳云李毅卒城中奉毅女固守經年華陽國志有毅卒年月及女秀守城事今從之}
范長生詣成都|{
	自青城山詣成都也}
成都王雄門迎執板拜為丞相尊之曰范賢 夏四月己巳司空越引兵屯温初太宰顒以為張方死東方兵必可解既而東方兵聞方死争入關顒悔之乃斬郅輔遣宏農太守彭隨北地太守刁默將兵拒祁宏等於湖五月壬辰宏等擊隨默大破之遂西入關又敗顒將馬瞻郭偉於霸水|{
	敗補邁翻}
顒單馬逃入太白山|{
	三秦記太白山在武功縣南去長安三百里俗云武功太白去天三百新唐書地里志太白山在鳳翔府郿縣}
宏等入長安所部鮮卑大掠殺二萬餘人百官犇散入山中拾橡實食之|{
	橡以兩翻栩實也爾雅曰柞實謂之橡賢曰橡櫟食也}
己亥宏等奉帝乘牛車東還|{
	晉志曰古之貴者不乘牛車漢武帝推恩之後諸侯寡弱至乘牛車其後稍見貴重自靈帝以來天子至士遂以為常乘夫天子出入有大駕法駕鹵簿帝自鄴奔洛則乘犢車自長安還洛則乘牛車無復出警入蹕之制矣}
以太弟太保梁柳為鎮西將軍守關中六月丙辰朔帝至洛陽復羊后 |{
	考異曰后傳曰張方首至洛陽即日復后位按方傳首已久不至今日今從帝紀}
辛未大赦改元|{
	改元光熙}
馬瞻等入長安殺梁柳與始平太守梁邁共迎太宰顒於南山|{
	武帝泰始二年分扶風置始平郡領槐里始平武功蒯城等縣南山即太白山中南太白本一山也}
宏農太守裴廙|{
	廙羊至翻又逸職翻}
秦國内史賈龕安定太守賈疋等起兵擊顒斬馬瞻梁邁疋詡之曾孫也|{
	帝即位改扶風為秦國以封秦王東龕苦含翻疋音雅賈詡生於漢末始從李傕郭汜中從張繡後歸魏}
司空越遣督護糜晃將兵擊顒 |{
	考異曰牽秀傳云顒密遣使詣東海王越求迎越遣將麋晃等迎顒今從顒傳}
至鄭顒使平北將軍牽秀屯馮翊顒長史楊騰詐稱顒命使秀罷兵騰遂殺秀關中皆服於越顒保城而已|{
	顒僅保長安城}
成都王雄即皇帝位|{
	雄字仲雋特第三子}
大赦改元曰晏平國

號大成 |{
	考異曰晉帝紀三十國晉春秋皆云永興二年六月雄即帝位華陽國志光熙元年雄即帝位後魏書序紀及李雄傳皆云昭帝十二年雄稱帝即光熙元年也十六國春秋鈔晏平元年六月雄即帝位十六國春秋目録雄年號建興二晏平五與華陽國志同今從之諸書雄改元晏平無大武年號惟晉載記改元大武無晏平年號按雄國號大成魏書雄傳云雄稱帝號大成改元晏平故三十國春秋誤云改元大成載記轉寫誤為大武今從諸書去大武之號}
追尊父特曰景皇帝廟號始祖尊王太后曰皇太后|{
	雄母羅氏尊為王太后見上卷永興元年}
以范長生為天地太師|{
	太師乃有天地之號侯景未足多怪也羣盜私立名字以相署置可勝言哉 考異曰華陽國志尊長生曰四時八節天地太師今從晉載記}
復其部曲皆不豫征税|{
	復方目翻}
諸將恃恩互争班位|{
	將即亮翻}
尚書令閻式上疏請考漢晉故事立百官制度從之 秋七月乙酉朔日有食之 八月以司空越為太傅録尚書事范陽王虓為司空鎮鄴|{
	考異曰虓傳為司徒今從帝紀}
平昌公模為鎮東大將軍鎮許昌王浚為驃騎大將軍都督東夷河北諸軍事領幽州刺史|{
	浚恃鮮卑烏桓以為羽翼故使并督東夷諸軍驃匹妙翻}
越以吏部郎庾敳為軍諮祭酒|{
	敳魚開翻漢魏之間兵興始置軍諮祭酒}
前太弟中庶子胡母輔之為從事中郎黄門侍郎郭象為主簿鴻臚丞阮脩為行參軍|{
	臚陵如翻晉列卿各置丞行參軍在參軍事之下沈約志晉太傅司馬趙府有行參軍兼行參軍後加長兼字除拜則為參軍事府板則為行參軍行參軍始於蜀丞相諸葛亮府}
謝鯤為掾|{
	掾以絹翻}
輔之薦樂安光逸於越|{
	姓譜光姓燕人田光之後秦末子孫避地因以為氏}
越亦辟之敳等皆尚虛玄不以世務嬰心縱酒放誕敳殖貨無厭|{
	厭於鹽翻}
象薄行好招權越皆以其名重於世故辟之|{
	史言越所辟置采虚名而無實用行下孟翻好呼到翻}
祁宏之入關也成都王穎自武關犇新野|{
	新野縣漢屬南陽郡晉屬義陽郡}
會新城元公劉宏卒司馬郭勱作亂欲迎穎為主郭舒奉宏子璠以討勱斬之|{
	璠孚袁翻勱莫敗翻}
詔南中郎將劉陶收穎穎北渡河犇朝歌收故將士得數百人欲赴公師藩頓邱太守馮嵩執之送鄴|{
	頓邱縣漢屬東郡武帝泰始元年分置郡}
范陽王虓不忍殺而幽之公師藩自白馬南渡河|{
	白馬縣漢屬東郡晉屬濮陽國唐為滑州治所}
兖州刺史苟晞討斬之 進東嬴公騰爵為東燕王|{
	燕於賢翻}
平昌公模為南陽王 冬十月范陽王虓薨長史劉輿以穎素為鄴人所附祕不發喪偽令人為臺使稱詔夜賜穎死|{
	使疏吏翻}
并殺其二子穎官屬先皆逃散惟盧志隨從至死不怠收而殯之|{
	從才用翻}
太傅越召志為軍諮祭酒越將召劉輿或曰輿猶膩也近則汚人|{
	膩女利翻皮膚之垢其肥滑者為膩汚烏故翻}
及至越疎之輿密視天下兵簿及倉庫牛馬器械水陸之形皆默識之|{
	識音志記也}
時軍國多事每會議自長史潘淊以下莫知所對輿應機辯畫|{
	辯者辯析事宜畫者為之區畫也}
越傾膝酬接即以為左長史軍國之務悉以委之輿說越遣其弟琨鎮并州以為北面之重|{
	說輸芮翻}
越表琨為并州刺史以東燕王騰為車騎將軍都督鄴城諸軍事鎮鄴 十一月己巳夜帝食䴵中毒|{
	䴵必郢翻麫餈也釋名䴵并也溲麫使合并也蒸䴵湯䴵之屬隨形而名食䴵中毒或云越鴆之也中竹仲翻}
庚午崩于顯陽殿|{
	年四十八}
羊后自以於太弟熾為嫂恐不得為太后將立清河王覃侍中華混諫曰太弟在東宫已久|{
	熾立為皇太弟見上卷永興元年華戶化翻}
民望素定今日寧可易乎即露板馳召太傅越召太弟入宫后已召覃至尚書閤疑變託疾而返癸酉太弟即皇帝位大赦尊皇后曰惠皇后居宏訓宫追尊母王才人曰皇太后立妃梁氏為皇后懷帝始遵舊制於東堂聽政|{
	東堂太極殿東堂也}
每至宴會輒與羣官論衆務考經籍黄門侍郎傅宣歎曰今日復見武帝之世矣|{
	復扶又翻}
十二月壬午朔日有食之 太傅越以詔書徵河間王顒為司徒顒乃就徵南陽王模遣其將梁臣邀之於新安車上扼殺之并殺其三子|{
	模越之弟也意謂殺顒父子則兄弟身安而無患矣而不知石勒趙染之禍已伏於冥冥之中矣新安縣漢屬宏農郡晉屬河南郡 考異曰三十國晉春秋云東海王越殺顒今從顒傳}
辛丑以中書監温羨為左光禄大夫領司徒尚書左僕射王衍為司空 己酉葬惠帝于太陽陵劉琨至上黨東燕王騰即自井陘東下時并州饑饉數為胡寇所掠|{
	胡寇謂劉淵之黨也數所角翻}
郡縣莫能自保州將田甄甄弟蘭任祉祁濟李惲薄盛等|{
	州將謂并州諸將也將即亮翻甄稽延翻任音壬惲於粉翻}
及吏民萬餘人悉隨騰就穀冀州號為乞活所餘之戶不滿二萬寇賊縱横道路斷塞|{
	縱于容翻塞悉則翻}
琨募兵上黨得五百人轉鬬而前至晉陽府寺焚毁|{
	府寺府舍也}
邑野蕭條|{
	聚居城市為邑散居在外為野}
琨撫循勞徠流民稍集|{
	稍力到翻徠力代翻}


孝懷皇帝上|{
	諱熾字豐度武帝第二十五子也謚法慈仁短折曰懷}


永嘉元年春正月癸丑大赦改元 吏部郎周穆太傅越之姑子也與其妹夫御史中丞諸葛玫|{
	玫謨杯翻}
說越曰主上之為太弟張方意也|{
	成都王穎之廢河間王顒立帝為皇太弟故以為張方之意}
清河王本太子|{
	清河王齊王冏立為太子經廢者數矣}
公宜立之越不許重言之|{
	重直用翻}
越怒斬之 二月王彌寇青徐二州自稱征東大將軍攻殺二千石太傅越以公車令東萊鞠羨為本郡太守|{
	晉志公車令屬衛尉}
以討彌彌擊殺之 陳敏刑政無章不為英俊所附子弟凶暴所在為患顧榮周玘等憂之廬江内史華譚遺榮等書曰陳敏盜據吳會命危朝露|{
	言若朝露之棲草上見日即晞不得久也華戶化翻遺于季翻}
諸君或剖符名郡或列為近臣而更辱身姦人之朝|{
	朝直遥翻下同}
降節叛逆之黨不亦羞乎吳武烈父子皆以英傑之才繼承大業|{
	吳謚孫堅曰武烈皇帝}
今以陳敏凶狡七弟頑宂欲躡桓王之高蹤蹈大皇之絶軌|{
	孫策追謚長沙桓王孫權謚大皇帝宂而隴翻躡尼輒翻}
遠度諸賢猶當未許也|{
	度徒洛翻}
皇輿東返|{
	謂自長安還洛陽也}
俊彦盈朝|{
	才過千人曰俊彦美士也}
將舉六師以清建業諸賢何顔復見中州之士邪|{
	復扶又翻}
榮等素有圖敏之心及得書甚慙密遣使報征東大將軍劉凖|{
	使疏吏翻}
使發兵臨江已為内應剪髪為信準遣揚州刺史劉機等出歷陽討敏敏使其弟廣武將軍昶將兵數萬屯烏江|{
	沈約志廣武將軍晉江左置蓋始於此時晉置烏江縣屬淮南郡即烏江亭長艤船待項羽之地以名縣宋白曰烏江縣漢東城縣地晉太康六年始於東城界置烏江縣昶丑兩翻}
歷陽太守宏屯牛渚敏弟處知顧榮等有貳心勸敏殺之敏不從 |{
	考異曰敏傳云弟昶勸殺榮按晉春秋敏臨死謂處曰我負卿時昶已先死今從晉春秋}
昶司馬錢廣周玘同郡人也玘密使廣殺昶宣言州下已殺敏|{
	揚州刺史治建業故謂建業為州下}
敢動者誅三族廣勒兵朱雀橋南|{
	朱雀橋在吳建業宫城之南跨秦淮水世傳晉孝武建朱雀門上有兩銅雀故橋亦以此得名余謂朱雀橋自吳以來有之蓋取前朱雀之義非晉孝武之時始有此名也朱雀橋亦曰大桁}
敏遣甘卓討廣堅甲精兵悉委之顧榮慮敏之疑故往就敏敏曰卿當四出鎮衛|{
	謂鎮安人心乃所以衛敏也}
豈得就我邪榮乃出與周玘共說甘卓曰|{
	說輸芮翻}
若江東之事可濟當共成之然卿觀兹事勢當有濟理不|{
	不讀曰否}
敏既常才政令反覆計無所定其子弟各已驕矜其敗必矣而吾等安然坐受其官禄事敗之日使江西諸軍函首送洛|{
	江西諸軍謂劉凖所遣臨江者也}
題曰逆賊顧榮甘卓之首此萬世之辱也卓遂詐稱疾迎女斷橋收船南㟁|{
	橋即朱雀橋也建業城在秦淮水北故卓收船傍南岸斷丁管翻}
與玘榮及前松滋侯相丹陽紀瞻共攻敏|{
	松滋縣屬廬江郡後漢省晉屬安豐郡劉昫曰唐夀州霍山縣漢松滋縣地今江陵府松滋縣乃是吳樂鄉之地晉氏南渡後以松滋流民僑立松滋縣非古松滋也}
敏自帥萬餘人討卓軍人隔水語敏衆曰本所以戮力陳公者正以顧丹陽周安豐耳|{
	敏以顧榮為丹陽太守周玘為安豐太守故以稱之帥讀曰率語牛倨翻}
今皆異矣汝等何為敏衆狐疑未决榮以白羽扇揮之|{
	白羽扇編白羽為之}
衆皆潰去敏單騎北走|{
	騎奇寄翻}
追獲之於江乘歎曰諸人誤我以至今日謂弟處曰我負卿卿不負我|{
	謂不用處言殺顧榮等也}
遂斬敏於建業夷三族於是會稽等郡盡殺敏諸弟|{
	會工外翻}
時平東將軍周馥代劉凖鎮壽春三月己未朔馥傳敏首至京師詔徵顧榮為侍中紀瞻為尚書郎太傅越辟周玘為參軍陸玩為掾玩機之從弟也|{
	掾以絹翻從才用翻}
榮等至徐州聞北方愈亂疑不進越與徐州刺史裴盾書曰若榮等顧望以軍禮發遣榮等懼逃歸盾楷之兄子越妃兄也|{
	楊正衡曰盾徒損翻}
西陽夷寇江夏|{
	西陽縣春秋弦子之國漢為縣屬江夏郡晉屬弋陽郡漢和帝永元末巫蠻反討降之徙置江夏西陽諸蠻是也沈約曰晉惠帝分弋陽為西陽國劉昫曰吳分江夏置蘄春郡晉改為西陽郡唐蘄州即其地宋白曰光州光山縣本漢西陽縣夏戶雅翻}
太守楊珉請督將議之諸將争獻方略騎督朱伺獨不言|{
	將即亮翻騎奇寄翻伺相吏翻}
珉曰朱將軍何以不言伺曰諸人以舌擊賊伺惟以力耳珉又問將軍前後擊賊何以常勝伺曰兩敵共對惟當忍之彼不能忍我能忍是以勝耳珉善之|{
	凡戰非有智巧以出奇取勝而以力角力者莫過於朱伺之說矣}
詔追復楊太后尊號丁卯改葬之諡曰武悼|{
	楊后遇禍見八十二卷惠帝元康元年}
庚午立清河王覃弟豫章王詮為皇太子|{
	詮且緣翻}
辛未大赦 帝觀覧大政留心庶事太傅越不悦固求出藩庚辰越出鎮許昌|{
	為越殺繆播等張本}
以高密王略為征南大將軍都督荆州諸軍事鎮襄陽南陽王模為征西大將軍都督秦雍梁益諸軍事鎮長安東燕王騰為新蔡王都督司冀二州諸軍事仍鎮鄴|{
	去年騰自并州徙鎮鄴}
公師藩既死汲桑逃還苑中|{
	茌平牧苑也桑於此起兵赴公師藩藩死逃還}
更聚衆劫掠郡縣自稱大將軍聲言為成都王報仇|{
	為于偽翻下燕為同}
以石勒為前驅所向輒克署勒討虜將軍遂進攻鄴時鄴中府庫空竭而新蔡武哀王騰資用甚饒騰性吝嗇無所振惠臨急乃賜將士米各數升帛各丈尺以是人不為用夏五月桑大破魏郡太守馮嵩長驅入鄴騰輕騎出犇為桑將李豐所殺桑出成都王穎棺|{
	穎之死也盧志收殯之今桑出而載之}
載之車中每事啟而後行遂燒鄴宫火旬日不滅|{
	袁紹據鄴始營宫室魏武帝又增而廣之至是悉為灰燼矣}
殺士民萬餘人大掠而去濟自延津南擊兖州太傅越大懼使苟晞及將軍王讚討之 秦州流民鄧定訇氐等據成固|{
	楊正衡曰訇呼宏翻余謂訇姓氏名}
寇掠漢中梁州刺史張殷遣巴西太守張燕討之鄧定等饑窘詐降於燕且賂之燕為之緩師|{
	窘渠隕翻降戶江翻為于偽翻}
定密遣訇氐求救於成成主雄遣太尉離司徒雲司空璜將兵二萬救定與燕戰大破之張殷及漢中太守杜孟治棄城走積十餘日離等引還盡徙漢中民於蜀漢中人句方白落帥吏民還守南鄭|{
	句古侯翻姓也梁州刺史漢中太守俱治南鄭杜佑曰漢漢中郡故城在唐梁州南鄭縣東北}
石勒與苟晞等相持於平原陽平間數月大小三十餘戰互有勝負秋七月己酉朔太傅越屯官渡為晞聲援 己未以琅邪王睿為安東將軍都督揚州江南諸軍事假節鎮建業|{
	時周馥鎮壽春督揚州之江北故睿督揚州之江南考異曰元帝紀曰東海王越之收兵下邳以帝都督揚州越西迎大駕留帝居守永嘉初用王導計始鎮建業按既都督揚州不當猶鎮下邳又懷帝紀明言七月己未睿都督揚州鎮建業今從之}
八月己卯朔苟晞擊汲桑於東武陽|{
	東武陽縣漢屬柬郡魏晉屬陽平郡後魏去東字為武陽縣唐貞觀初廢武陽入魏州莘縣開元十年復置改為朝城縣杜佑曰魏郡莘縣南有東武陽城}
大破之桑退保清淵|{
	清淵縣漢屬魏郡應劭曰清河在縣西北晉屬陽平郡後魏分置臨清縣後齊廢臨清縣入清淵唐避高祖諱改清淵為臨清屬貝州}
分荆州江州八郡為湘州|{
	晉志帝分荆州之衡陽長沙湘東零陵邵陽桂陽及廣州之始安始興臨賀九郡置湘州帝紀曰分荆江八郡為湘州紀志自相抵牾此從紀沈約亦曰分荆州之長沙衡陽湘東邵陵零陵營陽建昌江州之桂陽八郡立湘州}
九月戊申琅邪王睿至建業睿以安東司馬王導為謀主推心親信每事咨焉睿名論素輕吳人不附居久之士大夫莫有至者導患之會睿出觀禊導使睿乘肩輿具威儀|{
	禊胡計翻䘠除不祥也漢儀季春上已官及百姓皆褉於東流水上應劭風俗通曰按周禮女巫掌歲時以䘠除疾病禊者潔也於水上盥潔之也肩轝平肩轝也人以肩舉之而行}
導與諸名勝皆騎從|{
	騎奇寄翻從才用翻}
紀瞻顧榮等見之驚異相帥拜於道左|{
	帥讀曰率}
導因說睿曰顧榮賀循此土之望宜引之以結人心二子既至則無不來矣睿乃使導躬造循榮二人皆應命而至|{
	說輸芮翻造七到翻 考異曰導傳曰元帝鎮建業居月餘士庶莫有至者會從兄敦來朝導謂之曰琅邪王仁德雖厚而名論猶輕兄威風已振宜有以匡濟者會三月上巳帝觀禊敦導皆騎從王敦傳東海王越誅繆播後乃以敦為揚州刺史其後徵拜尚書不就周玘傳錢璯聞劉聰逼洛陽不敢進乃謀反時王敦遷尚書與璯俱西欲殺敦敦犇告元帝懷帝紀永嘉元年七月琅邪王睿鎮建業三年三月殺繆播四年二月錢璯反是時睿在建業已三年矣安得言月餘又睿名論雖輕安有為都督數年而士庶莫有至者陳敏得江東猶首用周顧以收人望導為睿佐豈得待數年然後薦之乎然則導傳所云難以盡信今刪去導語及敦名而已}
以循為吳國内史榮為軍司|{
	軍司軍司馬也}
加散騎常侍|{
	職為軍司此加官也}
凡軍府政事皆與之謀議又以紀瞻為軍祭酒卞壼為從事中郎|{
	壺若本翻}
周玘為倉曹屬琅邪劉超為舍人|{
	晉諸王國有謁者四人中大夫六人舍人十人}
張闓及魯國孔衍為參軍|{
	闓可亥翻又音開}
壼粹之子|{
	卞粹見上卷惠帝太安二年}
闓昭之曾孫也|{
	張昭輔吳為元臣}
王導說睿謙以接士儉以足用以清靜為政撫綏新舊|{
	新謂自中原來者舊謂江東人說輸芮翻}
故江東歸心焉睿初至頗以酒廢事導以為言睿命酌引觴覆之於此遂絶|{
	史言元帝能用王導所以興於江左}
苟晞追擊汲桑破其八壘死者萬餘人桑與石勒收

餘衆將犇漢冀州刺史譙國丁紹邀之於赤橋又破之桑犇馬牧|{
	茌平馬牧也}
勒犇樂平|{
	晉志陽平郡有樂平縣前漢東郡之清縣也後漢章帝改曰樂平但石勒欲奔漢則非此樂平也又并州有樂平郡武帝泰始中置唐之遼州也勒奔于此}
太傅越還許昌加苟晞撫軍將軍都督青兖諸軍事丁紹寧北將軍監冀州諸軍事皆假節晞屢破彊寇威名甚盛善治繁劇|{
	治直之翻}
用法嚴峻其從母依之晞奉養甚厚從母子求為將晞不許曰吾不以王灋貸人|{
	從才用翻將子亮翻貸他代翻}
將無後悔邪固求之晞乃以為督護後犯灋晞杖節斬之從母叩頭救之不聽既而素服哭之曰殺卿者兖州刺史哭弟者苟道將也|{
	苟晞字道將}
胡部大張㔨督馮莫突等|{
	胡人一部之長呼為部大楊正衡曰㔨音背}
擁衆數千壁于上黨石勒往從之因說㔨督等曰劉單于舉兵擊晉|{
	劉單于謂劉淵也說輸芮翻單音蟬}
部大拒而不從自度終能獨立乎|{
	度徒洛翻}
曰不能勒曰然則安可不早有所屬今部落皆已受單于賞募往往聚議欲叛部大而歸單于矣㔨督等以為然冬十月㔨督等随勒單騎歸漢|{
	騎奇寄翻}
漢王淵署㔨督為親漢王莫突為都督部大以勒為輔漢將軍平晉王以統之烏桓張伏利度有衆二千壁于樂平淵屢招不能致勒偽獲罪於淵往犇伏利度伏利度喜結為兄弟使勒帥諸胡寇掠|{
	帥讀曰率下同}
所向無前諸胡畏服勒知衆心之附已乃因會執伏利度謂諸胡曰今起大事我與伏利度誰堪為主諸胡咸推勒勒於是釋伏利度帥其衆歸漢淵加勒督山東征討諸軍事以伏利度之衆配之|{
	史言石勒之衆浸盛}
十一月戊申朔日有食之 甲寅以尚書右僕射和郁為征北將軍鎮鄴 乙亥以王衍為司徒衍說太傅越曰|{
	說輸芮翻下同}
朝廷危亂當賴方伯宜得文武兼資以任之乃以弟澄為荆州都督族弟敦為青州刺史|{
	考異曰晉春秋王衍言於太傅越以王澄為荆州敦為揚州據吳楚以為形援越從之於是澄敦同發越餞}


|{
	之敦傅自青州入為中書監東海王越誅繆播後始出為揚州播死在永嘉三年三月此年越在許昌不在洛故以晉書為定}
語之曰荆州有江漢之固青州有負海之險卿二人在外而吾居中足以為三窟矣|{
	戰國策馮煖謂孟嘗君曰狡兎有三窟厪得免其死耳其後敦澄自相魚肉衍亦死於石勒三窟安在哉是以忠臣國爾忘家蓋國安則家亦安矣陸佃埤雅曰俗云兎營窟必背立相通所謂狡兎三窟語牛倨翻}
澄至鎮以郭舒為别駕委以府事澄日夜縱酒不親庶務雖寇戎交急不以為懷舒常切諫以為宜愛民養兵保全州境澄不從|{
	為王澄不能保荆州張本}
十二月戊寅乞活田甄田蘭薄盛等起兵於新蔡王騰報讎斬汲桑于樂陵|{
	樂陵縣漢屬平原郡晉分為樂陵國唐為縣宋白曰棣州陽信縣魏屬樂陵國晉斬汲桑於此屬滄州為于偽翻}
棄成都王穎棺於故井中穎故臣收葬之 甲午以前太傅劉寔為太尉寔以老固辭不許庚子以光禄大夫高光為尚書令前北軍中侯呂雍度支校尉陳顔等|{
	度支校尉蓋當時所置以督漕運者也度徒洛翻}
謀立清河王覃為太子事覺太傅越矯詔囚覃於金墉城 初太傅越與苟晞親善引升堂結為兄弟司馬潘滔說越曰兖州衝要魏武以之創業|{
	事見六十卷六十一卷}
苟晞有大志非純臣也久令處之則患生心腹矣|{
	處昌呂翻}
若遷于青州厚其名號晞必悦公自牧兖州經緯諸夏藩衛本朝此所謂為之於未亂者也|{
	老子曰其安易持其未兆易謀其脆易破其微易散為之於未有治之於未亂緯于貴翻夏戶雅翻朝直遥翻}
越以為然癸卯越自為丞相領兖州牧都督兖豫司冀幽并諸軍事|{
	杜佑曰晉司徒與丞相通職更置迭廢未嘗並立至永嘉元年始兩置焉王衍為司徒東海王越為丞相}
以晞為征東大將軍開府儀同三司加侍中假節都督青州諸軍事領青州刺史封東平郡公越晞由是有隙|{
	為後晞馳檄罪狀越張本}
晞至青州以嚴刻立威日行斬戮州人謂之屠伯|{
	鄧展曰言殺人若屠兒之殺六畜伯長也}
頓邱太守魏植為流民所逼衆五六萬大掠兖州晞出屯無鹽以討之|{
	無鹽縣屬東平國唐屬濟州界}
以弟純領青州刑殺更甚於晞晞討植破之初陽平劉靈少貧賤|{
	少詩照翻}
力制犇牛走及犇馬時人雖異之莫能舉也靈撫膺歎曰|{
	膺胷也}
天乎何當亂也及公師藩起靈自稱將軍寇掠趙魏會王彌為苟純所敗靈亦為王讚所敗遂俱遣使降漢|{
	敗補邁翻 考異曰彌傳曰彌逼洛陽敗於七里澗乃與其黨劉靈謀歸漢按十六國春秋靈為王讚所逐彌為苟純所敗乃謀降漢今年春靈已在淵所五月彌乃如平陽然則二人先降漢已久矣彌傳誤也}
漢拜彌鎮東大將軍青徐二州牧都督緣海諸軍事封東萊公以靈為平北將軍 李釗至寧州|{
	光熙元年李毅卒釗今乃至寧州釗音昭}
州人奉釗領州事治中毛孟詣京師求刺史屢上奏不見省|{
	上時掌翻省悉景翻}
孟曰君亡親喪幽閉窮城|{
	謂李毅已死寧州受圍不解也喪息浪翻}
萬里訴哀精誠無感生不如死欲自刎|{
	刎扶粉翻}
朝廷憐之以魏興太守王遜為寧州刺史 |{
	考異曰華陽國志以廣漢太守王遜為寧州按時廣漢已為李雄所陷今從遜傳}
仍詔交州出兵救李釗交州刺史吾彦遣其子咨將兵救之|{
	將即亮翻}
慕容廆自稱鮮卑大單于|{
	廆乎罪翻}
拓跋禄官卒弟猗盧總攝三部與廆通好|{
	禄官分國為三部事見上八十二卷惠帝元康五年好呼到翻}


二年春正月丙午朔日有食之 |{
	考異曰帝紀天文志云丙子朔誤今從長歷}
丁未大赦 漢王淵遣撫軍將軍聰等十將南據太

行|{
	行戶剛翻}
輔漢將軍石勒等十將東下趙魏 |{
	考異曰石勒載記曰元海使劉聰攻壺關命勒帥所統七千為前鋒都督劉琨遣護軍黄秀等救壺關勒敗秀於白田殺之遂陷壺關事在明年今從十六國春秋}
二月辛卯太傅越殺清河王覃 庚子石勒寇常山王浚擊破之 涼州刺史張軌病風口不能言使其子茂攝州事隴西内史晉昌張越涼州大族|{
	惠帝分燉煌酒泉置晉昌郡杜佑曰晉昌漢冥安縣地}
欲逐軌而代之與其兄酒泉太守鎮及西平太守曹祛|{
	祛邱於翻 考異曰晉春秋作曹祇今從張軌傳}
謀遣使詣長安|{
	使疏吏翻}
告南陽王模稱軌廢疾請以秦州刺史賈龕代之龕將受之其兄讓龕曰張涼州一時名士威著西州汝何德以代之龕乃止|{
	龕口含翻}
鎮祛上疏更請刺史未報遂移檄廢軌以軍司杜耽攝州事使耽表越為刺史軌下教欲避位歸老宜陽|{
	軌少隱於宜陽女几山故下教欲歸老於宜陽}
長史王融參軍孟暢蹋折鎮檄|{
	蹋徒臘翻折而設翻}
排閤入言曰晉室多故明公撫寧西夏|{
	此西夏謂河西之地夏戶雅翻}
張鎮兄弟敢肆凶逆當鳴皷誅之遂出戒嚴會軌長子寔自京師還乃以寔為中督護將兵討鎮遣鎮甥太府主簿令狐亞|{
	按張軌傳有太府司馬主簿又有少府主簿蓋以都督府為太府涼州府為少府也}
先往說鎮為陳利害|{
	說輸芮翻為于偽翻}
鎮流涕曰人誤我乃詣寔歸罪寔南擊曹祛走之朝廷得鎮祛疏以待中袁瑜為涼州刺史治中楊澹馳詣長安|{
	考異曰晉春秋作張澹今從張軌傳澹徒覽翻又徒濫翻}
割耳盤上訢軌之被誣|{
	被皮義翻}
南陽王模表請停瑜武威太守張琠亦上表留軌|{
	字林琠他殄翻}
詔依模所表且命誅曹祛軌於是命寔帥步騎三萬討祛斬之|{
	帥讀曰率下同}
張越犇鄴涼州乃定 三月太傅越自許昌徙鎮鄄城|{
	鄄音絹}
王彌收集亡散兵復大振|{
	復扶又翻}
分遣諸將攻掠青徐

兖豫四州所過攻陷郡縣多殺守令有衆數萬苟晞與之連戰不能克夏四月丁亥彌入許昌太傅越遣司馬王斌帥甲士五千人入衛京師張軌亦遣督護北宫純將兵衛京師五月彌入自轘轅|{
	轘音環}
敗官軍于伊北|{
	伊水之北也敗必邁翻下同}
京師大震宫城門晝閉壬戌彌至洛陽屯于津陽門|{
	津陽門洛陽城南面東頭第二門}
詔以王衍都督征討諸軍事北宫純募勇士百餘人突陳|{
	陳讀曰陣}
彌兵大敗乙丑彌燒建春門而東衍遣左衛將軍王秉追之戰于七里澗又敗之彌走渡河與王桑自軹關如平陽|{
	軹關在河内軹縣軹音只}
漢王淵遣侍中兼御史大夫郊迎令曰孤親行將軍之館|{
	行下孟翻}
拂席洗爵敬待將軍及至拜司隸校尉加侍中特進以桑為散騎侍郎 北宫純等與漢劉聰戰於河東敗之詔封張軌西平郡公軌辭不受時州郡之使莫有至者軌獨遣使貢獻歲時不絶|{
	使疏吏翻}
秋七月甲辰漢王淵寇平陽太守宋抽棄郡走河東太守路述戰死淵徙都蒲子|{
	蒲子縣即晉公子重耳所居蒲城也漢屬河東郡晉屬平陽郡劉昫曰唐隰州治隰川縣漢蒲子縣地杜佑曰隰州隰川蒲縣漢蒲子縣地考異曰劉琨答太傅府書曰潜遣使驛離間其部落淵遂怖懼南奔蒲子雜虜歸降萬有餘落琨傳亦然按時淵彊琨弱豈因畏琨而徙都蓋琨為自大之辭史因承以為實耳}
上郡鮮卑陸逐延氐酋單徵並降於漢|{
	酋慈由翻單上演翻降戶江翻 考異曰載記作氐酋大單于徵按當時戎狄酋長皆謂之大徵即光文單后之父于衍字也}
八月丁亥太傅越自鄄城徙屯濮陽|{
	濮陽衛墟漢屬東郡晉初分置}


|{
	濮陽國唐鄄城濮陽二縣皆屬濮州鄄音絹濮音卜}
未幾又徙屯滎陽|{
	幾居豈翻}
九月漢王彌石勒寇鄴和郁棄城走詔豫州刺史裴憲屯白馬以拒彌車騎將軍王堪屯東燕以拒勒|{
	漢東郡燕縣古南燕國晉省而故城猶在曰東燕城後魏立東燕縣屬陳留郡劉昫曰唐滑州胙城縣漢南燕縣燕於賢翻}
平北將軍曹武屯大陽以備蒲子|{
	大陽縣屬河東郡地理志曰北虢也應劭曰在大河之陽唐倂入陜州河北縣界}
憲楷之子也|{
	裴楷仕武帝惠帝時}
冬十月甲戍漢王淵即皇帝位大赦改元永鳳十一月以其子和為大將軍聰為車騎大將軍族子曜為龍驤大將軍|{
	驤思將翻}
壬寅并州刺史劉琨使上黨太守劉惇帥鮮卑攻壺關|{
	杜佑曰唐潞州治上黨漢壺關縣後魏移壺關縣當羊腸阪羊頭之阨帥讀曰率下同}
漢鎮東將軍綦母達戰敗亡歸 丙午漢都督中外諸軍事領丞相右賢王宣卒 石勒劉靈帥衆三萬寇魏郡汲郡頓邱|{
	汲縣漢屬河内郡武帝泰始二年分置汲郡唐之衛州即其地}
百姓望風降附者五十餘壘|{
	降戶江翻}
皆假壘主將軍都尉印綬|{
	綬音受}
簡其彊壯五萬為軍士老弱安堵如故己酉勒執魏郡太守王粹于三臺殺之|{
	三臺注見後八十八卷永嘉六年}
十二月辛未朔大赦 乙亥漢主淵以大將軍和為大司馬封梁王尚書令歡樂為大司徒封陳留王|{
	樂音洛}
后父御史大夫呼延翼為大司空封雁門郡公宗室以親疏悉封郡縣王異姓以功伐悉封郡縣公侯 成尚書令楊褒卒 |{
	考異曰載記云丞相楊褒今從晉春秋}
褒好直言|{
	好呼到翻}
成主雄初得蜀用度不足諸將有以獻金銀得官者褒諫曰陛下設官爵當網羅天下英豪何有以官買金邪雄謝之雄嘗醉推中書令杖太官令褒進曰天子穆穆諸侯皇皇|{
	禮記曲禮之言}
安有天子而為酗也|{
	陸德明曰酗况具翻以酒為凶曰酗}
雄慙而止 成平寇將軍李鳳屯晉壽|{
	葭萌縣漢屬廣漢郡蜀改為漢壽縣屬梓潼郡晉又改漢壽曰晉壽}
屢寇漢中漢中民東走荆沔|{
	沔水自梁州入荆州界為荆沔}
詔以張光為梁州刺史荆州寇盜不禁詔起劉璠為順陽内史江漢閒翕然歸之|{
	璠父弘之喪未終起之於苫塊荆州之民懷其父故翕然歸其子璠扶元翻}


資治通鑑卷八十六
