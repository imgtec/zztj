<!DOCTYPE html PUBLIC "-//W3C//DTD XHTML 1.0 Transitional//EN" "http://www.w3.org/TR/xhtml1/DTD/xhtml1-transitional.dtd">
<html xmlns="http://www.w3.org/1999/xhtml">
<head>
<meta http-equiv="Content-Type" content="text/html; charset=utf-8" />
<meta http-equiv="X-UA-Compatible" content="IE=Edge,chrome=1">
<title>資治通鑒_109-資治通鑑卷一百八_109-資治通鑑卷一百八</title>
<meta name="Keywords" content="資治通鑒_109-資治通鑑卷一百八_109-資治通鑑卷一百八">
<meta name="Description" content="資治通鑒_109-資治通鑑卷一百八_109-資治通鑑卷一百八">
<meta http-equiv="Cache-Control" content="no-transform" />
<meta http-equiv="Cache-Control" content="no-siteapp" />
<link href="/img/style.css" rel="stylesheet" type="text/css" />
<script src="/img/m.js?2020"></script> 
</head>
<body>
 <div class="ClassNavi">
<a  href="/24shi/">二十四史</a> | <a href="/SiKuQuanShu/">四库全书</a> | <a href="http://www.guoxuedashi.com/gjtsjc/"><font  color="#FF0000">古今图书集成</font></a> | <a href="/renwu/">历史人物</a> | <a href="/ShuoWenJieZi/"><font  color="#FF0000">说文解字</a></font> | <a href="/chengyu/">成语词典</a> | <a  target="_blank"  href="http://www.guoxuedashi.com/jgwhj/"><font  color="#FF0000">甲骨文合集</font></a> | <a href="/yzjwjc/"><font  color="#FF0000">殷周金文集成</font></a> | <a href="/xiangxingzi/"><font color="#0000FF">象形字典</font></a> | <a href="/13jing/"><font  color="#FF0000">十三经索引</font></a> | <a href="/zixing/"><font  color="#FF0000">字体转换器</font></a> | <a href="/zidian/xz/"><font color="#0000FF">篆书识别</font></a> | <a href="/jinfanyi/">近义反义词</a> | <a href="/duilian/">对联大全</a> | <a href="/jiapu/"><font  color="#0000FF">家谱族谱查询</font></a> | <a href="http://www.guoxuemi.com/hafo/" target="_blank" ><font color="#FF0000">哈佛古籍</font></a> 
</div>

 <!-- 头部导航开始 -->
<div class="w1180 head clearfix">
  <div class="head_logo l"><a title="国学大师官网" href="http://www.guoxuedashi.com" target="_blank"></a></div>
  <div class="head_sr l">
  <div id="head1">
  
  <a href="http://www.guoxuedashi.com/zidian/bujian/" target="_blank" ><img src="http://www.guoxuedashi.com/img/top1.gif" width="88" height="60" border="0" title="部件查字,支持20万汉字"></a>


<a href="http://www.guoxuedashi.com/help/yingpan.php" target="_blank"><img src="http://www.guoxuedashi.com/img/top230.gif" width="600" height="62" border="0" ></a>


  </div>
  <div id="head3"><a href="javascript:" onClick="javascript:window.external.AddFavorite(window.location.href,document.title);">添加收藏</a>
  <br><a href="/help/setie.php">搜索引擎</a>
  <br><a href="/help/zanzhu.php">赞助本站</a></div>
  <div id="head2">
 <a href="http://www.guoxuemi.com/" target="_blank"><img src="http://www.guoxuedashi.com/img/guoxuemi.gif" width="95" height="62" border="0" style="margin-left:2px;" title="国学迷"></a>
  

  </div>
</div>
  <div class="clear"></div>
  <div class="head_nav">
  <p><a href="/">首页</a> | <a href="/ShuKu/">国学书库</a> | <a href="/guji/">影印古籍</a> | <a href="/shici/">诗词宝典</a> | <a   href="/SiKuQuanShu/gxjx.php">精选</a> <b>|</b> <a href="/zidian/">汉语字典</a> | <a href="/hydcd/">汉语词典</a> | <a href="http://www.guoxuedashi.com/zidian/bujian/"><font  color="#CC0066">部件查字</font></a> | <a href="http://www.sfds.cn/"><font  color="#CC0066">书法大师</font></a> | <a href="/jgwhj/">甲骨文</a> <b>|</b> <a href="/b/4/"><font  color="#CC0066">解密</font></a> | <a href="/renwu/">历史人物</a> | <a href="/diangu/">历史典故</a> | <a href="/xingshi/">姓氏</a> | <a href="/minzu/">民族</a> <b>|</b> <a href="/mz/"><font  color="#CC0066">世界名著</font></a> | <a href="/download/">软件下载</a>
</p>
<p><a href="/b/"><font  color="#CC0066">历史</font></a> | <a href="http://skqs.guoxuedashi.com/" target="_blank">四库全书</a> |  <a href="http://www.guoxuedashi.com/search/" target="_blank"><font  color="#CC0066">全文检索</font></a> | <a href="http://www.guoxuedashi.com/shumu/">古籍书目</a> | <a   href="/24shi/">正史</a> <b>|</b> <a href="/chengyu/">成语词典</a> | <a href="/kangxi/" title="康熙字典">康熙字典</a> | <a href="/ShuoWenJieZi/">说文解字</a> | <a href="/zixing/yanbian/">字形演变</a> | <a href="/yzjwjc/">金 文</a> <b>|</b>  <a href="/shijian/nian-hao/">年号</a> | <a href="/diming/">历史地名</a> | <a href="/shijian/">历史事件</a> | <a href="/guanzhi/">官职</a> | <a href="/lishi/">知识</a> <b>|</b> <a href="/zhongyi/">中医中药</a> | <a href="http://www.guoxuedashi.com/forum/">留言反馈</a>
</p>
  </div>
</div>
<!-- 头部导航END --> 
<!-- 内容区开始 --> 
<div class="w1180 clearfix">
  <div class="info l">
   
<div class="clearfix" style="background:#f5faff;">
<script src='http://www.guoxuedashi.com/img/headersou.js'></script>

</div>
  <div class="info_tree"><a href="http://www.guoxuedashi.com">首页</a> > <a href="/SiKuQuanShu/fanti/">四库全书</a>
 > <h1>资治通鉴</h1> <!--         下载:【右键另存为】即可 --></div>
  <div class="info_content zj clearfix">
  
<div class="info_txt clearfix" id="show">
<center style="font-size:24px;">109-資治通鑑卷一百八</center>
    資治通鑑卷一百八   宋 司馬光 撰<br />
<br />
  胡三省 音註<br />
<br />
  晉紀三十【起玄黓執徐盡柔兆涒灘凡五年】<br />
<br />
  烈宗孝武皇帝下<br />
<br />
  太元十七年春正月己巳朔大赦 秦主登立昭儀隴西李氏為皇后 二月壬寅燕主垂自魯口如河間勃海平原翟釗遣其將翟都侵舘陶屯蘇康壘【蘇康人姓名舘陶縣漢屬魏郡晉屬陽平郡將即亮翻下同】三月垂引兵南擊釗 秦驃騎將軍没奕干帥衆降于後秦【驃匹妙翻騎奇寄翻帥讀曰率降戶江翻】後秦以為車騎將軍封高平公 後秦主萇寢疾命姚碩德鎮李潤尹緯守長安召太子興詣行營【萇時屯安定萇音長】征南將軍姚方成言於興曰今寇敵未滅上復寢疾【復扶又翻】王統等皆有部曲終為人患宜盡除之興從之殺王統王廣苻胤徐成毛盛【皆苻氏舊臣也】萇怒曰王統兄弟吾之州里實無他志徐成等皆前朝名將【朝直遥翻】吾方用之奈何輒殺之【使萇果以殺統等為非罪當按誅始造謀者但怒而已豈真怒邪】燕主垂進逼蘇康壘夏四月翟都南走滑臺【走音奏】翟釗求救於西燕西燕主永謀於羣臣尚書郎勃海鮑遵曰使兩寇相弊吾承其後此卞莊子之策也中書侍郎太原張騰曰垂彊釗弱何弊之承不如速救之以成鼎足之勢今我引兵趨中山【趨七喻翻下趣同】晝多疑兵夜多火炬垂必懼而自救我衝其前釗躡其後此天授之機不可失也永不從【翟釗敗則西燕之亡形成矣】 燕大赦 五月丁卯朔日有食之 六月燕主垂軍黎陽臨河欲濟翟釗列兵南岸以拒之辛亥垂徙營就西津去黎陽西四十里為牛皮船百餘艘偽列兵仗泝流而上【艘蘇遭翻上時掌翻】釗亟引兵趣西津【趣七喻翻】垂潜遣中壘將軍桂林王鎮等自黎陽津夜濟營于河南比明而營成【比必寐翻及也】釗聞之亟還攻鎮等營垂命鎮等堅壁勿戰釗兵往來疲暍【暍於歇翻傷暑也】攻營不能拔將引去鎮等引兵出戰驃騎將軍農自西津濟與鎮等夾擊大破之【燕主垂用兵於河上者再温詳則引兵徑濟而取之翟釗則張疑兵於西而潜軍東渡亦以决勝視敵之堅脆何如也驃匹妙翻騎奇寄翻農燕之驃騎大將軍此逸大字】釗走還滑臺將妻子收遺衆北濟河登白鹿山【水經注河内脩武縣北有白鹿山】憑險自守燕兵不得進農曰釗無糧不能久居山中乃引兵還留騎之釗果下山還兵掩擊盡獲其衆釗單騎犇長子西燕主永以釗為車騎大將軍兖州牧封東郡王歲餘釗謀反永殺之初郝晷崔逞及清河崔宏新興張卓遼東夔騰【夔姓也石趙之臣有夔安】陽平路纂皆仕於秦避秦亂來犇詔以為冀州諸郡各將部曲營於河南【將即亮翻】既而受翟氏官爵翟氏敗皆降於燕【降戶江翻】燕主垂各隨其材而用之釗所統七郡三萬餘戶皆按堵如故以章武王宙為兖豫二州刺史鎮滑臺徙徐州民七千餘戶于黎陽以彭城王脱為徐州刺史鎮黎陽【徐州之民蓋為翟釗所掠者】脱垂之弟子也垂以崔䕃為宙司馬初陳留王紹為鎮南將軍太原王楷為征西將軍樂浪王温為征東將軍【樂浪音洛琅】垂皆以䕃為之佐䕃才幹明敏強正善規諫四王皆嚴憚之所至簡刑法輕賦役流民歸之戶口滋息秋七月垂如鄴以太原王楷為冀州牧右光祿大夫餘蔚為左僕射【蔚紆勿翻】 秦主登聞後秦主萇疾病大喜【疾甚曰病】告祠世祖神主【苻堅廟號世祖】大赦百官進位二等秣馬厲兵進逼安定去城九十餘里八月萇疾小瘳出拒之登引兵出營將逆戰萇遣安南將軍姚熙隆别攻秦營登懼而還【還從宣翻又如字】萇夜引兵旁出以躡其後旦而騎告曰【騎奇寄翻】賊諸營已空不知所向登驚曰彼為何人去令我不知來令我不覺謂其將死忽然復來【復扶又翻】朕與此羌同世何其厄哉【苻登屢為姚萇所挫故有懼萇之心蓋至于是登氣衰矣】登遂還雍【雍於用翻下同】萇亦還安定 三河王光遣其弟右將軍寶等攻金城王乾歸寶及將士死者萬餘人又遣其子虎賁中郎將纂擊南羌彭奚念纂亦敗歸光自將擊奚念於枹罕克之奚念奔甘松【甘松郡乞伏國仁所置及將即亮翻下同賁音奔枹音膚】 冬十月辛亥荆州刺史王忱卒【忱是壬翻】 雍州刺史朱序以老病求解職詔以太子右衛率郗恢為雍州刺史代序鎮襄陽恢曇之子也【郗曇見一百卷穆帝升平三年率所律翻郗丑之翻曇徒含翻】 巴蜀人在關中者皆叛後秦據弘農以附秦秦主登以竇衝為左丞相衝徙屯華隂【華戶化翻】郗恢遣將軍趙睦守金墉河南太守楊佺期帥衆軍湖城【帥讀曰率】擊衝走之 十一月癸酉以黄門郎殷仲堪為都督荆益寧三州諸軍事荆州刺史鎮江陵仲堪雖有英譽資望猶淺議者不以為允到官好行小惠【好呼到翻】綱目不舉南郡公桓玄負其才地以雄豪自處【負其才與其門地也處昌呂翻】朝廷疑而不用年二十三始拜太子洗馬【洗悉薦翻】玄嘗詣琅邪王道子值其酣醉【酣戶甘翻】張目謂衆客曰桓温晩塗欲作賊云何玄伏地流汗不能起由是益不自安常切齒於道子後出補義興太守【守式又翻】鬱鬱不得志歎曰父為九州伯兒為五湖長【虞翻曰太湖有五湖隔湖洮湖射湖貴湖及太湖為五湖並太湖之小支俱連太湖故太湖兼得五湖之名韋昭曰胥湖蠡湖洮湖隔湖就太湖而五酈善長謂長塘湖射湖貴湖隔湖與太湖而五吳中志謂貢湖遊湖胥湖梅梁湖金鼎湖為五也長知兩翻】遂棄官歸國【玄襲封南郡公】上疏自訟曰先臣勤王匡復之勲朝廷遺之臣不復計【上時掌翻復扶又翻下同】至於先帝龍飛陛下繼明【謂桓温廢海西立簡文帝而帝繼統也易曰明兩作離大人以繼明照四方】請問談者誰之由耶疏寢不報玄在江陵仲堪甚敬憚之桓氏累世臨荆州玄復豪横【横戶孟翻】士民畏之過於仲堪嘗於仲堪聽事前戲馬以矟擬仲堪【聽讀曰廳矟色角翻通俗文長丈八者謂之矟擬者舉矟向之若將刺之也】仲堪中兵參軍彭城劉邁謂玄曰【元帝謂江東置參軍十三曹有中兵外兵騎兵】馬矟有餘精理不足玄不悦仲堪為之失色【為於偽翻】玄出仲堪謂邁曰卿狂人也玄夜遣殺卿我豈能相救邪使邁下都避之【都謂建康】玄使人追之邁僅而獲免征虜將軍豫章胡藩過江陵見仲堪說之曰【說輸芮翻】桓玄志趣不常每怏怏於失職【怏於兩翻】節下崇待太過恐非將來之計也仲堪不悦藩内弟羅企生為仲堪功曹藩退謂企生曰殷侯倒戈以授人必及於禍君不早圖去就後悔無及矣【為後桓玄殺企生仲堪張本企區智翻】 庚寅立皇子德文為琅邪王徙琅邪王道子為會稽王【會王外翻】 十二月燕主垂還中山以遼西王農為都督兖豫荆徐雍五州諸軍事鎮鄴【雍於用翻】休官權千成據顯親自稱秦州牧【休官雜夷部落之名顯親縣漢光武】<br />
<br />
  【置屬漢陽郡晉改顯親為顯新復漢陽為天水郡晉書姚興載記權千成作權干成略陽豪族也】 清河人李遼上表請敕兖州脩孔子廟【孔子廟在魯魯郡前漢屬徐州後漢晉屬豫州江表始分屬兖州】給戶灑掃【灑所賣翻掃素報翻又各如字】仍立庠序收教學者曰事有如賖而寔急者【此寔義與虚實之實同】此之謂也表不見省【省悉景翻】<br />
<br />
  十八年春正月燕陽平孝王柔卒 權千成為秦所逼請降於金城王乾歸【降戶江翻】乾歸以為東秦州刺史休官大都統顯親公 夏四月庚子燕主垂加太子寶大單于以安定王庫傉官偉為太尉【單音蟬傉奴沃翻】范陽王德為司徒太原王楷為司空陳留王紹為尚書右僕射五月立子熙為河間王朗為勃海王鑒為博陵王 秦右丞相竇衝矜才尚人【尚人者陵人而出其上】自請封天水王秦主登不許六月衝自稱秦王改元元光 金城王乾歸立其子熾磐為太子【熾昌志翻】熾磐勇略明决過於其父 秋七月秦主登攻竇衝於野人堡衝求救於後秦尹緯言於後秦主萇曰太子仁厚之稱【緯于貴翻稱尺證翻名稱也】著於遠近而英略未著請使擊苻登以著之萇從之太子興將兵攻胡空堡登解衝圍以赴之興因襲平凉大獲而歸【苻登自大界之敗以平凉為根本】萇使興還鎮長安 魏王珪以薛干太悉伏不送劉勃勃【事見上卷十六年】八月襲其城屠之太悉伏犇秦 氐帥楊佛嵩叛犇後秦【帥所類翻】楊佺期趙睦追之【佺且緣翻】九月丙戌敗佛嵩於潼關後秦將姚崇救佛嵩敗晉兵【敗補邁翻】趙睦死 冬十月後秦王萇疾甚還長安燕主垂議伐西燕諸將皆曰永未有釁我連年征討士卒疲弊未可也范陽王德曰永既國之枝葉又僭舉位號惑民視聽宜先除之以壹民心士卒雖疲庸得已乎垂曰司徒意正與吾同吾比老叩囊底智足以取之【比必寐翻及也】終不復留此賊以累子孫也【垂不欲留慕容永以累子孫而不知拓拔珪已窺?於代北矣是以有國有家者不恃無敵國外患恃吾所以傳國承家者足以待之耳累力瑞翻復扶又翻】遂戒嚴十一月垂發中山步騎七萬遣鎮西將軍丹楊王纘【纘當作瓚】龍驤將軍張崇出井陘【驤思將翻陘音刑】攻西燕武鄉公友于晉陽征東將軍平規攻鎮東將軍段平于沙亭【沙亭在鄴西南】西燕主永遣其尚書令刁雲車騎將軍慕容鍾帥衆五萬守潞川【帥讀曰率】友永之弟也十二月垂至鄴 己亥後秦主萇召太尉姚旻僕射尹緯姚晃將軍姚大目尚書狄伯支入禁中受遺詔輔政萇謂太子興曰有毁此諸公者慎勿受之汝撫骨肉以恩接大臣以禮待物以信遇民以仁四者不失吾無憂矣【姚萇所以詔其子者勝于苻健】姚晃垂涕問取苻登之策萇曰今大業垂成興才智足辦奚所復問【復扶又翻】庚子萇卒【年六十四】興祕不發喪以其叔父緒鎮安定碩德鎮隂密弟崇守長安或謂碩德曰公威名素重部曲最彊今易世之際必為朝廷所疑不如且犇秦州【碩德本起兵隴上據冀城】觀望事勢碩德曰太子志度寛明必無他慮今苻登未滅而骨肉相攻是自亡也吾有死而已終不為也遂往見興興優禮而遣之興自稱大將軍以尹緯為長史狄伯支為司馬帥衆伐秦【帥讀曰率】<br />
<br />
  十九年春秦主登聞後秦主萇卒【卒子恤翻下同】喜曰姚興小兒吾折杖笞之耳乃大赦盡衆而東【輕敵者敗宜苻登所以不亡於姚萇之時而亡於姚興之初立也】留司徒安成王廣守雍【雍於用翻】太子崇守胡空堡遣使拜金城王乾歸為左丞相河南王領秦梁益凉沙五州牧加九錫【使疏吏翻下同】 初秃髪思復鞬卒【鞬居言翻】子烏孤立烏孤雄勇有大志與大將紛陁謀取凉州【欲并呂光也將即亮翻】紛陁曰公必欲得凉州宜先務農講武禮俊賢修政刑然後可也烏孤從之三河王光遣使拜烏孤冠軍大將軍河西鮮卑大都統【冠古玩翻】烏孤與其羣下謀之曰可受乎皆曰吾士馬衆多何為屬人石真若留不對烏孤曰卿畏呂光邪石真若留曰吾根本未固小大非敵若光致死於我何以待之不如受以驕之俟釁而動蔑不克矣烏孤乃受之【紛陁與石真若留皆能審宜應事者也史言秃髪烏孤所以興紛與石真蓋皆夷姓】 二月秦主登攻屠各姚奴帛蒲二堡克之【二堡在胡空堡之東屠直於翻】 燕主垂留清河公會鎮鄴發司冀青兖兵遣太原王楷出滏口【滏音釡】遼西王農出壺關垂自出沙庭以擊西燕【庭當作亭其地在鄴西南】標榜所趣軍各就頓【分處置兵以疑敵使不知所備趣七喻翻下同】西燕主永聞之嚴兵分道拒守聚粮臺壁【水經註潞縣北對故臺壁漳水出其南本潞子所立也魏收地形志襄垣郡刈陵縣漢晉之潞縣也有臺壁】遣從子征東將軍小逸豆歸【時西燕之臣有二逸豆歸故此稱小逸豆歸從才用翻】鎮東將軍王次多右將軍勒馬駒帥衆萬餘人戍之【帥讀曰率】 夏秦主登自六陌趣廢橋【趣七喻翻】後秦始平太守姚詳據馬嵬堡以拒之【嵬五囘翻】太子興遣尹緯將兵救詳【將即亮翻】緯據廢橋以待秦秦兵爭水不能得渴死者什二三因急攻緯興馳遣狄伯支謂緯曰苻登窮寇宜持重以挫之緯曰先帝登遐【鄭玄曰登上也遐已也上已者若僊去云耳上時掌翻】人情擾懼今不因思奮之力以禽敵大事去矣遂與秦戰秦兵大敗其夜秦衆潰登單騎犇雍【騎奇寄翻雍於用翻】太子崇及安成王廣聞敗皆棄城走登至無所歸乃犇平凉收集遺衆入馬毛山【平凉城在漢安定鶉隂縣界後周始置平凉郡及縣唐為原州縣赫連定之敗魏亦據馬髦嶺以禽奚斤蓋平凉之險要處也】 燕主垂頓軍鄴西南月餘不進西燕主永怪之以為太行道寛【行戶剛翻】疑垂欲詭道取之乃悉歛諸軍屯軹關【軹知氏翻】杜太行口惟留臺壁一軍甲戌垂引大軍出滏口入天井關【前漢書地理志上黨郡高都縣有天井關蔡邕曰太行山上有天井關在井北遂因名焉余按今澤州晉城縣有太行關關内有天井泉三所即天井關也】五月乙酉燕軍至臺壁永遣從兄太尉大逸豆歸救之【從才用翻】平規擊破之小逸豆歸出戰遼西王農又擊破之斬勒馬駒禽王次多遂圍臺壁永召太行軍還自將精兵五萬以拒之刁雲慕容鍾震怖帥衆降燕永誅其妻子【將即亮翻怖普布翻帥讀曰率降戶江翻】己亥垂陳于臺壁南【陳讀曰陣】遣驍騎將軍慕容國伏千騎於澗下【驍堅堯翻騎奇寄翻】庚子與永合戰垂偽退永衆追之行數里國騎從澗中出斷其後【斷丁管翻】諸軍四面俱進大破之斬首八千餘級永走歸長子晉陽守將聞之棄城走丹楊王瓚等進取晉陽【瓚藏旱翻】 後秦太子興始發喪即皇帝位于槐里【興字子畧萇之長子也槐里縣漢屬扶風晉屬始平郡宋白曰漢槐里縣故城在唐岐州興平縣東南七里興既破苻登始發喪襲位】大赦改元皇初遂如安定諡後秦主萇曰武昭皇帝廟號太祖 六月壬子追尊會稽王太妃鄭氏曰簡文宣太后【會工外翻諡法聖善周聞曰宣】羣臣謂宣太后應配食元帝太子前率徐邈曰【晉志惠帝建東宫稱中衛率泰始五年分為左右各領一軍惠帝時愍懷太子在東宫又加前後二率江左省前後二率孝武太元中又置率所類翻】宣太后平素之時不伉儷於先帝【言非正妃伉苦浪翻敵也儷力計翻並也】至於子孫豈可為祖考立配【為于偽翻】國學明教東莞臧燾曰【據晉書儒林傳元帝運鍾百六光啟中興雖尊儒勸學亟降於綸言然東序西膠未聞于絃誦明皇雅愛流畧簡文敦悦典墳乃招集學徒弘奬風烈國學明教之官當置於明帝簡文時也莞音官】今尊號既正則罔極之情申别建寢廟則嚴禰之義顯【嚴尊也禰父廟也】繫子為稱兼明貴之所由【繫子為稱簡文繫之宣太后之上也春秋傳曰母以子貴稱尺證翻】一舉而允三義不亦善乎乃立廟於太廟路西 燕主垂進軍圍長子西燕主永欲犇後秦侍中蘭英曰昔石虎伐龍都太祖堅守不去【事見九十六卷晉成帝咸康四年】卒成大燕之基【卒子恤翻】今垂七十老翁厭苦兵革終不能頓兵連歲以攻我也但當城守以疲之永從之【兵交之變其應無窮惟知彼知己者乃能百戰不殆耳慕容永欲以棘城之事自况當時與之共守長子者果能效死不去若慕容皝之諸臣乎】 秦主登遣其子汝隂王宗為質於河南王乾歸以請救【質音致】進封乾歸梁王納其妹為梁王后乾歸遣前軍將軍乞伏益州等帥騎一萬救之【帥讀曰率騎奇寄翻】秋七月登引兵出迎乾歸兵後秦主興自安定如涇陽與登戰于山南【馬髦山南也】執登殺之【年五十二】悉散其部衆使歸農業徙隂密三萬戶於長安以李后賜姚晃益州等聞之引兵還【還從宣翻又如字】秦太子崇奔湟中即帝位改元延初諡登曰高皇帝廟號太宗 後秦安南將軍強熙鎮遠將軍強多叛推竇衝為主後秦主興自將討之軍至武功多兄子良國殺多而降【強其兩翻將即亮翻降戶江翻】熙犇秦州衝犇汧川【汧川即扶風汧縣之地汧苦堅翻】汧川氐仇高執送之 三河王光以子覆為都督玉門以西諸軍事西域大都護鎮高昌命大臣子弟隨之 八月己巳尊皇太妃李氐為皇太后居崇訓宫 西燕主永困急遣其子常山公弘等求救於雍州刺史郗恢【郗丑之翻】并獻玉璽一紐【璽斯氏翻】恢上言垂若并永為患益深不如兩存之可以乘機雙斃帝以為然詔青兖二州刺史王恭豫州刺史庾楷救之楷亮之孫也【庾氏為桓温所誅楷復不能振自此微矣】永恐晉兵不出又遣其太子亮來為質【質音致】平規追亮及於高都獲之【高都縣屬上黨郡隋為澤州丹川縣唐為晉城縣】永又告急於魏魏王珪遣陳留公䖍將軍庾岳帥騎五萬東渡河屯秀容以救之【此北秀容也在漢定襄郡界後魏置秀容郡秀容縣又立秀容護軍於汾水西北六十里徙北秀容胡人居之此南秀容也劉昫曰忻州秀容縣漢汾陽縣也隋自秀容故城移於此因更名帥讀曰率騎奇寄翻】䖍紇根之子也【紇根見一百四卷元年紇戶骨翻】晉魏兵皆未至大逸豆歸部將伐勤等開門內燕兵燕人執永斬之并斬其公卿大將刁雲大逸豆歸等三十餘人【將即亮翻】得永所統八郡七萬餘戶及秦乘輿服御伎樂珍寶甚衆【乘䋲證翻】燕主垂以丹楊王瓚為并州刺史鎮晉陽宜都王鳳為雍州刺史鎮長子永尚書僕射昌黎屈遵【瓚藏旱翻雍於用翻屈居勿翻】尚書陽平王德祕書監中山李先太子詹事勃海封則黄門郎太山胡母亮中書郎張騰尚書郎燕郡公孫表皆隨才擢叙【李先公孫表後皆仕魏位通顯】九月垂自長子如鄴 冬十月秦主崇為梁王乾歸所逐犇隴西王楊定定留司馬邵彊守秦州帥衆二萬與崇共攻乾歸乾歸遣凉州牧軻彈秦州牧益州立義將軍詰歸帥騎三萬拒之【軻彈益州詰歸皆乞伏氐也凉秦二州牧乾歸所置非能有其地軻彈晉書載記作軻殫帥讀曰率下同】益州與定戰敗於平州【載記作平川當從之】軻彈詰歸皆引退軻彈司馬翟瑥奮劔怒曰【瑥音温】主上以雄武開基所向無敵威振秦蜀將軍以宗室居元帥之任【帥所類翻】當竭力致命以佐國家今秦州雖敗二軍尚全奈何望風退衂【衂女六翻】將何面以見主上乎瑥雖無任獨不能以便宜斬將軍乎軻彈謝曰向者未知衆心何如耳果能若是吾敢愛死乃帥騎進戰益州詰歸亦勒兵繼之大敗定兵【敗補邁翻】殺定及崇斬首萬七千級【穆帝永和七年秦王健改元即位歷六主四十三年而亡】乾歸於是盡有隴西之地【乞伏始得秦州】定無子其叔父佛狗之子盛先守仇池自稱征西將軍秦州刺史仇池公諡定為武王仍遣使來稱藩【使疏吏翻】秦太子宣犇盛分氐羌為二十部護軍各為鎮戍不置郡縣 燕主垂東巡陽平平原命遼西王農濟河與安南將軍尹國畧地青兖農攻廪丘國攻陽城皆拔之東平太守韋簡戰死高平泰山琅邪諸郡皆委城犇潰農進軍臨海【臨東海也】徧置守宰 柔然曷多汗棄其父與社崘率衆西走【柔然降魏見上卷十六年汗音寒崘盧昆翻】魏長孫肥追之及於上郡跋那山斬曷多汗社崘收其餘衆數百犇疋候跋疋候跋處之南鄙【處昌呂翻】社崘襲疋候跋殺之疋候跋子啟跋吳頡等皆犇魏社崘掠五原以西諸部走度漠北【柔然自此遂為魏患據載記以社崘為河西鮮卑則柔然亦鮮卑種也】十一月燕遼西王農敗辟閭渾於龍水【郭緣生述征記曰逢山在廣固南三十里洋水歷其隂而東北流世謂之石溝水出委粟山北而東注于巨洋水謂之石溝口然是水下流亦有時通塞及其春夏水泛川瀾無輟亦或謂之龍泉水敗補邁翻】遂入臨淄十二月燕主垂召農等還 秦主興遣使與燕結好【使疏吏翻好呼到翻是歲前秦滅通鑑始書後秦為秦】并送太子寶之子敏於燕燕封敏為河東公 梁王乾歸自稱秦王大赦【自此以後史以西秦别之】<br />
<br />
  二十年春正月燕主垂遣散騎常侍封則報聘于秦【散悉亶翻騎奇寄翻】遂自平原狩于廣川勃海長樂而歸【漢高祖置信都郡景帝三年為廣川國明帝更名樂成安帝改曰安平晉改曰長樂郡又别立廣川郡樂音洛】 西秦王乾歸以太子熾磐領尚書令【熾昌志翻】左長史邊芮為左僕射右長史袐宜為右僕射置官皆如魏武晉文故事然猶稱大單于大將軍【單音蟬】邊芮等領府佐如故 薛于太悉伏自長安亡歸嶺北【嶺北謂九嵕嶺北十八年太悉伏犇秦】上郡以西鮮卑雜胡皆應之 二月甲寅尚書令陸納卒 三月庚辰朔日有食之 皇太子出就東宫以丹陽尹王雅領少傅時會稽王道子專權奢縱嬖人趙牙本出倡優【少詩照翻會工外翻倡音昌】茹千秋本錢唐捕賊吏【錢唐縣前漢屬會稽郡後漢屬吳郡錢唐記曰郡議曹華信議立此塘以防海水始開募能致土一斛者即與錢一千旬月之間來者雲集塘未成而不復取於是載土石者皆委之而去塘以之成故名錢塘楊正衡曰茹音如又而據翻浙間舊有此姓】皆以謟賂得進道子以牙為魏郡太守千秋為驃騎諮議參軍【驃頻召翻今匹妙翻騎奇寄翻】牙為道子開東第築山穿池【為于偽翻】功用鉅萬帝嘗幸其第謂道子曰府内乃有山甚善然修飾太過道子無以對帝去道子謂牙曰上若知山是人力所為爾必死矣牙曰公在牙何敢死營作彌甚千秋賣官招權聚貨累億博平令吳興聞人奭上疏言之【博平縣漢屬東郡晉屬平原郡江左屬魏郡與郡皆僑置】帝益惡道子【惡烏路翻】而逼於太后不忍廢黜乃擢時望及所親幸王恭郗恢殷仲堪王珣王雅等使居内外要任以防道子道子亦引王國寶及國寶從弟琅邪内史緒以為心腹【從才用翻】由是朋黨競起無復曏時友愛之驩矣太后每和解之中書侍郎徐邈從容言於帝曰【從千容翻】漢文明主猶悔淮南世祖聰達負愧齊王【淮南事見十四卷漢文帝六年齊王事見八十一卷武帝太康四年】兄弟之際實為深慎會稽王雖有酣媟之累【媟私列翻累力瑞翻】宜加弘貸消散羣議外為國家之計内慰太后之心帝納之復委任道子如故【復扶又翻】 初楊定之死也天水姜乳襲據上邽夏四月西秦王乾歸遣乞伏益州帥騎六千討之左僕射邊芮民部尚書王松壽曰益州屢勝而驕不可專任必以輕敵取敗乾歸曰益州驍勇諸將莫及【帥讀曰率騎奇寄翻驍堅堯翻將即亮翻】當以重佐輔之耳乃以平北將軍韋䖍為長史左禁將軍務和為司馬【務姓也古有務光】至大寒嶺【大寒嶺在上邽西】益州不設部伍聽將士遊畋縱飲令曰敢言軍事者斬䖍等諫不聽乳逆擊大破之 魏王珪叛燕侵逼附塞諸部五月甲戌燕主垂遣太子寶遼西王農趙王麟帥衆八萬自五原伐魏范陽王德陳留王紹别將步騎萬八千為後繼散騎常侍高湖諫曰【散悉亶翻騎奇寄翻】魏與燕世為昏姻【代王什翼犍兩娶于慕容皆早卒哀帝隆和元年什翼犍納女于燕燕又以女妻之】彼有内難燕實存之【事見一百六卷十一年及一百七卷十二年難乃旦翻】其施德厚矣結好久矣間以求馬不獲而留其弟【事見上卷十六年好呼到翻下同】曲在於我奈何遽興兵擊之拓跋涉圭沈勇有謀【蕭子顯曰珪字涉圭沈持林翻】幼歷艱難兵精馬彊未易輕也皇太子富於春秋志果氣鋭今委之專任必小魏而易之【易以䜴翻】萬一不如所欲傷威毁重願陛下深圖之言頗激切垂怒免湖官湖泰之子也【前燕時垂為車騎將軍以泰為從事中郎】六月癸丑燕太原元王楷卒 西秦王乾歸遷于西<br />
<br />
  城【苑川西城也】 秋七月三河王光帥衆十萬伐西秦【帥讀曰率】西秦左輔密貴周左衛將軍莫者羖羝【密以國為氏姓譜漢有尚書密忠據通鑑下文則以密貴為姓莫者夷複姓】勸西秦王乾歸稱藩於光以子敕勃為質【質音致】光引兵還乾歸悔之殺周及羖羝【羖音古羝音氐】 魏張衮聞燕軍將至言於魏王珪曰燕狃於滑臺長子之捷【滑臺事見上十七年長子事見上年狃與忸同杜預曰忸忲也】竭國之資力以來有輕我之心宜羸形以驕之乃可克也【羸倫為翻】珪從之悉徙部落畜產西渡河千餘里以避之燕軍至五原降魏别部三萬餘家【降戶江翻】收穄田百餘萬斛置黑城【黑城在五原河北按魏書帝紀登國五年劉衛辰遣子直力鞮出稒陽塞侵及黑城從可知矣】進軍臨河【水經河水自新秦中屈而南流過五原西安陽成宜宣梁臨沃稒陽等縣南】造船為濟具珪遣右司馬許謙乞師於秦 秃髪烏孤擊乙弗折掘等諸部皆破降之築亷川堡而都之【乙弗折掘二部皆在禿髪氏之西亷川在湟中降戶江翻】廣武趙振少好奇畧【少詩照翻好呼到翻】聞烏孤在亷川棄家從之烏孤喜曰吾得趙生大事濟矣拜左司馬三河王光封烏孤為廣武郡公 有長星見自須女至于哭星【天文志須女四星須賤妾之稱婦職之卑者也斗牛女揚州分虛二星危三星皆主死喪哭泣墳墓四星屬危之下主死喪哭泣為墳墓也見賢遍翻】帝心惡之於華林園舉酒祝之曰【晉都建康倣洛都起華林園惡烏路翻】長星勸汝一盃酒自古何有萬歲天子邪 八月魏王珪治兵河南【治直之翻】九月進軍臨河燕太子寶列兵將濟暴風起漂其船數十艘泊南岸【漂紕招翻艘蘇遭翻】魏獲其甲士三百餘人皆釋而遣之寶之發中山也燕主垂已有疾既至五原珪使人邀中山之路伺其使者盡執之【伺相吏翻使疏吏翻】寶等數月不聞垂起居珪使所執使者臨河告之曰若父已死何不早歸寶等憂恐士卒駭動珪使陳留公䖍將五萬騎屯河東東平公儀將十萬騎屯河北【河水自金城過武威天水安定北地郡界率東北流至朔方沃野縣界始屈而東南流䖍屯河東儀屯河北皆河曲之地未渡河也北史曰儀據朔方將即亮翻下同】畧陽公遵將七萬騎塞燕軍之南遵壽鳩之子也【壽鳩見一百四卷元年】秦主興遣楊佛嵩將兵救魏燕術士靳安言於太子寶曰【靳居焮翻】天時不利燕必大敗速去可免寶不聽安退告人曰吾輩皆當棄尸草野不得歸矣燕魏相持積旬趙王麟將慕輿嵩等以垂為實死謀作亂奉麟為主事泄嵩等皆死寶麟等内自疑冬十月辛未燒船夜遁時河冰未結寶以魏兵必不能度不設斥候十一月己卯暴風冰合魏王珪引兵濟河留輜重【重直用翻】選精鋭二萬餘騎急追之燕軍至參合陂有大風黑氣如堤自軍後來臨覆軍上【覆扶九翻】沙門支曇猛【支者曇猛之俗姓曇徒含翻】言於寶曰風氣暴迅魏兵將至之候宜遣兵禦之寶以去魏軍已遠矣而不應曇猛固請不已麟怒曰以殿下神武師徒之盛足以横行沙漠索虜何敢遠來【太元十八年慕容麟已知拓跋珪之必為燕患矣今乃輕之如此豈其心自疑而欲敗寶之師邪其後寶不能守中山而麟亦不能自立同歸于亂而已矣索昔各翻】而曇猛妄言驚衆當斬以狥曇猛泣曰苻氏以百萬之師敗於淮南正由恃衆輕敵不信天道故也【事見一百四卷五卷七年八年】司徒德勸寶從曇猛言寶乃遣麟帥騎三萬居軍後以備非常【帥讀曰率騎奇寄翻下同】麟以曇猛為妄縱騎遊獵不肯設備寶遣騎還詗魏兵【詗古永翻又翾正翻】騎行十餘里即解鞍寢魏軍晨夜兼行乙酉暮至參合陂西燕軍在陂東營於蟠羊山南水上【水經註可不埿水出鴈門沃陽縣東南六十里山下西北流注沃水合流而東逕參合縣南】魏王珪夜部分諸將【分扶問翻】掩覆燕軍士卒衘枚束馬口潛進丙戌日出魏軍登山下臨燕營燕軍將東引【引而東行也】顧見之士卒大驚擾亂珪縱兵擊之燕兵走赴水人馬相騰躡壓溺死者以萬數略陽公遵以兵邀其前燕兵四五萬人一時放仗斂手就禽其遺迸去者不過數千人【迸北孟翻】太子寶等皆單騎僅免殺燕右僕射陳留悼王紹生禽魯陽王倭奴【倭烏禾翻】桂林王道成濟隂公尹國等文武將吏數千人【濟子禮翻】兵甲糧貨以鉅萬計道成垂之弟子也魏王珪擇燕臣之有才用者代郡太守廣川賈閏閏從弟驃騎長史昌黎太守彛太史郎晁崇等留之【晁直遥翻】其餘欲悉給衣糧遣還以招懷中州之人中部大人王建曰燕衆彊盛今傾國而來我幸而大捷不如悉殺之則其國空虚取之為易【易以䜴翻】且獲寇而縱之無乃不可乎乃盡阬之十二月珪還雲中之盛樂【通鑑於惠帝元康五年書定襄之盛樂故城此書雲中之盛樂蓋歷代郡縣廢徙無常前漢成樂縣屬定襄後漢成樂縣屬雲中前書定襄之盛樂此前漢之故城也此書雲中之盛樂此後漢之故城也】燕太子寶恥於參合之敗請更擊魏司徒德言於燕主垂曰虜以參合之捷有輕太子之心宜及陛下神略以服之不然將為後患垂乃以清河公會録留臺事領幽州刺史代高陽王隆鎮龍城以陽城王蘭汗為北中郎將代長樂公盛鎮薊【樂音洛薊音計】命隆盛悉引其精兵還中山期以明年大舉擊魏 是歲秦主興封其叔父緒為晉王碩德為隴西王弟崇為齊公顯為常山公<br />
<br />
  二十一年春正月燕高陽王隆引龍城之甲入中山軍容精整燕人之氣稍振【漢人有言戰勝之威士氣百倍敗軍之卒没世不復正此之謂也】 休官權萬世帥衆降西秦【前年乞伏乾歸稱秦王故稱西秦以别於姚秦帥讀曰率降戶江翻】 燕主垂遣征東將軍平規發兵冀州二月規以博陵武邑長樂三郡兵反於魯口其從子冀州刺史喜諫不聽【從才用翻】規弟海陽令翰亦起兵於遼西以應之【海陽縣自漢以來屬遼西郡平規兄弟以燕兵敗故叛之】垂遣鎮東將軍餘嵩擊規嵩敗死垂自將擊規至魯口規棄衆將妻子及平喜等數十人走渡河垂引兵還翰引兵趣龍城【趣七喻翻】清河公會遣東陽公根等擊翰破之翰走山南【白狼徐無等山之南】三月庚子燕主垂留范陽王德守中山引兵密發踰青嶺經天門【青嶺蓋即廣昌嶺在代郡廣昌縣南所謂五迴道也其南層崖刺天積石之峻壁立直上蓋即天門也】鑿山通道出魏不意直指雲中魏陳留公䖍帥部落三萬餘家鎮平城垂至獵嶺【獵嶺在夏屋山東北魏都平城常獵於此】以遼西王農高陽王隆為前鋒以襲之是時燕兵新敗皆畏魏惟龍城兵勇鋭爭先䖍素不設備閏月乙卯燕軍至平城䖍乃覺之帥麾下出戰敗死燕軍盡收其部落魏王珪震怖欲走諸部聞䖍死皆有貳心珪不知所適【䖍勇蓋代北既敗而死故諸部皆貳然天將亡燕垂繼以殞此固非人力所能為也帥讀曰率怖普布翻】垂之過參合陂也見積骸如山為之設祭【為于偽翻】軍士皆慟哭聲震山谷垂慙憤嘔血由是發疾乘馬輿而進頓平城西北三十里太子寶等聞之皆引還燕軍叛者犇告於魏云垂已死輿尸在軍魏王珪欲追之聞平城已没乃引還隂山【魏人有言死諸葛走生仲達拓跋珪聞慕容垂之死而不敢進亦類是耳】垂在平城積十日疾轉篤乃築燕昌城而還【水經燕昌城在平城北四十里】夏四月癸未卒於上谷之沮陽【賢曰沮陽縣故城在今媯州東沮音阻垂年七十一】袐不發喪丙申至中山戊戌發喪諡曰成武皇帝廟號世祖壬寅太子寶即位【寶字道祐垂第四子也】大赦改元永康五月辛亥以范陽王德為都督冀兖青徐荆豫六州諸軍事車騎大將軍冀州牧鎮鄴遼西王農為都督并雍益梁秦凉六州諸軍事并州牧鎮晉陽【雍於用翻】又以安定王庫傉官偉為太師【傉沃奴翻】夫餘王蔚為太傅【餘蔚夫餘王子也燕王皝破夫餘得之燕亡入秦秦亂復歸燕燕主垂封為扶餘王】甲寅以趙王麟領尚書左僕射高陽王隆領右僕射長樂公盛為司隸校尉宜都王鳳為冀州刺史 乙卯以散騎常侍彭城劉該為徐州刺史鎮鄄城【散悉亶翻騎奇寄翻鄄吉掾翻】 甲子以望蔡公謝琰為尚書左僕射【望蔡縣屬豫章郡沈約曰漢靈帝中平中汝南上蔡民分徙此城立縣名上蔡晉武帝太康元年更名宋白曰以上蔡人思本土故曰望蔡】 初燕主垂先段后生子令寶後段后生子朗鑒愛諸姬子麟農隆柔熙寶初為太子有美稱【稱昌孕翻名譽也】已而荒怠中外失望後段后嘗言於垂曰【燕主垂初娶段氏以可足渾后之讒而死後即位追尊為后復納段氏為后故史書後段后以别之】太子遭承平之世足為守成之主今國步艱難恐非濟世之才遼西高陽二王陛下之賢子宜擇一人付以大業趙王麟姦詐彊愎【愎弼力翻】異日必為國家之患宜早圖之寶善事垂左右左右多譽之【譽音余】故垂以為賢謂段氏曰汝欲使我為晉獻公乎【晉獻公信驪姬之讒殺太子申生】段氏泣而退告其妹范陽王妃曰太子不才天下所知吾為社稷言之【為于偽翻】主上乃以吾為驪姬何其苦哉觀太子必喪社稷【喪息浪翻】范陽王有非常器度若燕祚未盡其在王乎寶及麟聞而恨之乙丑寶使麟謂段氏曰后常謂主上不能守大業今竟能不【不讀曰否】宜早自裁以全段宗段氏怒曰汝兄弟不難逼殺其母况能守先業乎吾豈愛死但念國亡不久耳遂自殺寶議以段后謀廢適統【適讀曰嫡】無母后之道不宜成喪羣臣咸以為然中書令眭邃颺言於朝曰【眭息惟翻颺音揚大言而疾曰颺朝直遥翻】子無廢母之義漢安帝閻后親廢順帝【事見五十卷漢安帝延光三年】猶得配饗太廟况先后曖昧之言虚實未可知乎乃成喪六月癸酉魏王珪遣將軍王建等擊燕廣甯太守劉亢泥斬之【廣甯縣漢屬上谷郡晉太康中分置廣甯郡亢苦浪翻】徙其部落於平城燕上谷太守開封公詳棄郡走詳皝之曾孫也 丁亥魏賀太妃卒【魏王珪之母也】 燕主寶定士族舊籍分辨清濁校閲戶口罷軍營封䕃之戶悉屬郡縣【軍營封䕃之戶蓋諸軍庇占以為部曲者】由是士民嗟怨始有離心【斯事行之未必非也但慕容寶即位之初國師新敗又遭大喪下之懷反側者多未可遽行耳大學曰物有本末事有終始知所先後則近道矣】 三河王呂光即天王位國號大凉大赦改元龍飛【呂光字世明畧陽氏也父婆樓為苻堅佐命】備置百官以世子紹為太子封子弟為公侯者二十人以中書令王詳為尚書左僕射著作郎段業等五人為尚書光遣使者拜秃髪烏孤為征南大將軍益州牧左賢王烏孤謂使者曰呂王諸子貪淫【光諸子見於史者纂弘紹覆】三甥暴虐【光甥石聰譛殺杜進餘二人當攷】遠近愁苦吾安可違百姓之心受不義之爵乎吾當為帝王之事耳乃留其鼓吹羽儀【吹昌瑞翻】謝而遣之 平規收合餘黨據高唐【高唐縣自漢以來屬平原郡】燕主寶遣高陽王隆將兵討之【將即亮翻】東土之民素懷隆惠迎候者屬路【相屬於路也屬之欲翻】秋七月隆進軍臨河規棄高唐走隆遣建威將軍慕容進等濟河追之斬規於濟北【於濟子禮翻】平喜犇彭城 納故中書令王獻之女為太子妃獻之羲之之子也【羲之王導之從子】魏羣臣勸魏王珪稱尊號珪始建天子旌旗出警入<br />
<br />
  蹕改元皇始【珪什翼犍之嫡孫寔之子詳見一百四卷元年自苻堅淮淝之敗至是十有四年矣關河之間戎狄之長更興迭作晉人視之漠然不關乎其心拓跋珪興而南北之形定矣南北之形既定卒之南為北所并嗚呼自隋以後名稱揚于時者代北之子孫十居六七矣氏族之辨果何益哉】參軍事上谷張恂勸珪進取中原珪善之燕遼西王農悉將部曲數萬口之并州【將即亮翻之往也】并州素乏儲㣥【㣥文里翻】是歲早霜民不能供其食又遣諸部護軍分監諸胡【監工銜翻】由是民夷俱怨潜召魏軍八月己亥魏王珪大舉伐燕【兵無内應與必勝之計不可大舉】步騎四十餘萬南出馬邑踰句注【句音鉤】旌旗二千餘里鼓行而進左將軍鴈門李栗將五萬騎為前驅别遣將軍封真等從東道出軍都襲燕幽州【魏書官氏志拓跋詰汾時餘部諸姓内入者有是賁氏後改為封氏軍都縣前漢屬上谷郡後漢屬廣陽郡晉屬燕國有軍都關賢曰今幽州昌平縣有軍都山在西北】 燕征北大將軍幽平二州牧清河公會母賤而年長【長知兩翻】雄俊有器藝燕主垂愛之寶之伐魏也垂命會攝東宫事摠錄禮遇一如太子【總錄謂總錄朝政也】及垂伐魏命會鎮龍城委以東北之任國官府佐皆選一時才望垂疾篤遺言命寶以會為嗣而寶愛少子濮陽公策意不在會長樂公盛與會同年恥為之下【少詩沼翻濮博木翻樂音洛】乃與趙王麟共勸寶立策寶從之乙亥立妃段氏為皇后策為皇太子會盛皆進爵為王策年十一素憃弱【憃與戇同陟降翻愚也】會聞之心愠懟【愠於問翻懟直類翻】九月章武王宙奉燕主垂及成哀段后之喪葬于龍城宣平陵【成哀后即寶所殺後母段氏也】寶詔宙悉徙高陽王隆參佐部曲家屬還中山【隆去年自龍城還中山會實代之故令遣還其部曲參佐】會違詔多留部曲不遣宙年長屬尊【長知兩翻】會每事陵侮之見者皆知其有異志【為寶會父子相圖張本】 戊午魏軍至陽曲【陽曲縣自漢以來屬太原郡宋白曰陽曲縣故城在太原郡北四十五里後漢末所移也隋文帝改為陽直尋又改為汾陽縣】乘西山臨晉陽遣騎環城大譟而去【騎奇寄翻環音宦】燕遼西王農出戰大敗犇還晉陽司馬慕輿嵩閉門拒之【前有慕輿嵩以謀奉趙王麟為變而誅此又一人】農將妻子帥數千騎東走魏中領將軍長孫肥追之【中領將軍魏所置猶魏晉之中領軍也帥讀曰率】及於潞川獲農妻子燕軍盡没農被創獨與三騎逃歸中山【被皮義翻創初良翻】魏王珪遂取并州初建臺省置刺史太守尚書郎以下官悉用儒生為之士大夫詣軍門無少長皆引入存慰使人人盡言【少詩照翻長知兩翻】少有才用咸加擢叙【史言拓跋珪所以能取中原少詩沼翻】己未遣輔國將軍奚收畧地汾川【奚收當作奚牧】獲燕丹楊王買德及離石護軍高秀和【離石縣自漢以來屬西河郡燕置護軍以統稽胡】以中書侍郎張恂等為諸郡太守招撫離散勸課農桑燕主寶聞魏軍將至議于東堂中山尹苻謨曰今魏軍衆彊千里遠鬭乘勝氣鋭若縱之使入平土不可敵也宜杜險以拒之【苻謨見燕見一百六卷十一年】中書令眭邃曰魏多騎兵往來剽速【剽匹妙翻】馬上齎糧不過旬日宜令郡縣聚民千家為一堡深溝高壘清野以待之彼至無所掠不過六旬食盡自退尚書封懿曰今魏兵數十萬天下之勍敵也【勍渠京翻】民雖築堡不足以自固是聚兵及粮以資之也且動揺民心示之以弱不如阻關拒戰計之上也趙王麟曰魏今乘勝氣鋭其鋒不可當宜完守中山待其弊而乘之於是修城積粟為持久之備【不據險拒戰而嬰城自守此慕容寶所以敗也】命遼西王農出屯安喜【安喜前漢之安險縣也後漢章帝改曰安喜屬中山郡】軍事動靜悉以委麟【為麟叛張本】 帝嗜酒流連内殿醒治既少【言昏醉之時多醒而治事之時少也】外人罕得進見【見賢遍翻】張貴人寵冠後宫【冠古玩翻】後宫皆畏之庚申帝與後宫宴妓樂盡侍【妓渠綺翻】時貴人年近三十帝戲之曰汝以年亦當廢矣吾意更屬少者【近其靳翻屬之欲翻少詩照翻】貴人潜怒向夕帝醉寢於清暑殿【清暑殿帝所作】貴人徧飲宦者酒散遣之使婢以被蒙帝面弑之重賂左右云因魘暴崩【飲於鴆翻魘于琰翻廣韻曰睡中魘毛晃曰氣窒心懼而神亂則魘】時太子闇弱會稽王道子昏荒【會工外翻】遂不復推問【復扶又翻下同】王國寶夜叩禁門欲入為遺詔侍中王爽拒之曰大行晏駕皇太子未至敢入者斬國寶乃止爽恭之弟也辛酉太子即皇帝位大赦癸亥有司奏會稽王道子宜進位太傅揚州牧假黄鉞詔内外衆事動靜咨之安帝幼而不慧【杜預曰不慧世所謂白癡】口不能言至于寒暑飢飽亦不能辨飲食寢興皆非已出母弟琅邪王德文性㳟謹常侍左右為之節適始得其宜【節適謂事為之節以適其口體為于偽翻】初王國寶黨附會稽王道子【王國寶黨附道子事始一百五卷八年】驕縱不法屢為御史中丞褚粲所糾國寶起齋侔清暑殿孝武帝甚惡之【惡烏路翻】國寶懼遂更求媚於帝而踈道子帝復寵昵之【昵尼質翻】道子大怒嘗於内省面責國寶以劔擲之舊好盡矣【好呼到翻】及帝崩國寶復事道子與王緒共為邪謟道子更惑之倚為心腹遂參管朝權威震内外並為時之所疾王㳟入赴山陵每正色直言道子深憚之㳟罷朝歎曰榱棟雖新【朝直遥翻下同榱所追翻秦曰屋椽齊魯曰桷周曰榱】便有黍離之歎【周大夫行役過故宗廟宫室盡為禾黍故作黍離之詩】緒說國寶【說輸芮翻】因㳟入朝勸相王伏兵殺之國寶不許道子欲輯和内外乃深布腹心於㳟冀除舊惡而㳟每言及時政輒厲聲色道子知㳟不可和恊遂有相圖之志或勸㳟因入朝以兵誅國寶㳟以豫州刺史庾楷士馬甚盛黨於國寶憚之不敢發王珣謂㳟曰國寶雖終為禍亂要之罪逆未彰今遽先事而發必大失朝野之望况擁彊兵竊發於京輦誰謂非逆國寶若遂不改惡布天下然後順衆心以除之亦無憂不濟也㳟乃止既而謂珣曰比來視君一似胡廣【謂依違於權姦之間以保禄位比毗至翻近也】珣曰王陵廷爭陳平慎默但問歲晏何如耳【謂王陵以廷爭失位陳平以慎默終能安劉爭讀曰諍】冬十月甲申葬孝武帝于隆平陵王㳟還鎮將行謂道子曰主上諒闇【闇讀如隂】冢宰之任伊周所難願大王親萬幾納直言放鄭聲遠佞人【幾與機同遠於願翻】國寶等愈懼 魏王珪使冠軍將軍代人于栗磾【魏書官氏志拓跋詰汾時餘部諸姓内入者有勿忸於氏後改為于氏冠古玩翻磾丁奚翻】寧朔將軍公孫蘭帥步騎二萬潜自晉陽開韓信故道【韓信自井陘伐趙之故路也帥讀曰率騎奇寄翻】己酉珪自井陘趨中山李先降魏【去年李先自西燕歸燕趨七喻翻降戶江翻】珪以為征東左長史西秦凉州牧軻彈與秦州牧益州不平軻彈犇凉 魏王珪進攻常山拔之獲太守苟延自常山以東守宰或走或降諸郡縣皆附於魏惟中山鄴信都三城為燕守【中山燕都慕容德守鄴慕容鳳守信都皆重鎮也為于偽翻】十一月珪命東平公儀將五萬騎攻鄴冠軍將軍王建左將軍李栗攻信都戊午珪進軍中山己未攻之燕高陽王隆守南郭帥衆力戰自旦至晡殺傷數千人魏兵乃退珪謂諸將曰中山城固寶必不肯出戰急攻則傷士久圍則費粮不如先取鄴信都然後圖之丁卯珪引兵而南章武王宙自龍城還聞有魏寇馳入薊與鎮北將軍陽城王蘭乘城固守蘭垂之從弟也【從才用翻】魏别將石河頭攻之不克【魏書官氏志拓跋詰汾時餘部諸姓内入者有嗢石蘭氏後為石氏】退屯漁陽【漁陽縣漢屬漁陽郡晉省】珪軍于魯口博陵太守申永犇河南高陽太守崔宏犇海渚【海渚海中州也】珪素聞宏名遣騎追求獲之以為黄門侍郎與給事黄門侍郎張衮對掌機要創立制度【為崔宏父子貴顯於魏張本】博陵令屈遵降魏【屈居勿翻降戶江翻】珪以為中書令出納號令兼總文誥燕范陽王德使南安王青等夜擊魏軍於鄴下破之魏軍退屯新城【新城即燕主垂攻鄴所築者也】青等請追擊之别駕韓曰【音卓】古人先計而後戰魏軍不可擊者四懸軍遠客利在野戰一也深入近畿頓兵死地二也前鋒既敗後陣方固三也彼衆我寡四也官軍不宜動者三自戰其地一也【自戰其地者衆易敗散】動而不勝衆心難固二也城隍未修敵來無備三也今魏無資糧不如深壘固軍以老之德從之召青還青詳之兄也十二月魏遼西公賀賴盧帥騎二萬會東平公儀攻鄴賴盧訥之弟也【為賀賴盧降慕容德張本按魏書官氏志内入諸姓有賀賴氏北方有賀蘭氏後皆為賀氏蓋内入者為賀賴氏留北方者為賀蘭氏蘭賴語轉耳又匈奴諸種亦有賀賴氏】魏别部大人没根有膽勇魏王珪惡之【惡烏路翻】没根懼誅己丑將親兵數十人降燕燕主寶以為鎮東大將軍封鴈門公没根求還襲魏寶難與重兵給百餘騎没根效其號令夜入魏營至中仗珪乃覺之狼狽驚走没根以所從人少不能壞其大衆多獲首虜而還【史言慕容寶不能因降人為問以破魏少詩沼翻壞音怪】楊盛遣使來請命詔拜盛鎮南將軍仇池公盛表苻宣為平北將軍 是歲越質詰歸帥戶二萬叛西秦降于秦【越質詰歸降西秦見上卷十六年帥讀曰率下同】秦人處之成紀【成紀縣自漢以來屬天水郡處昌呂翻】拜鎮西將軍平襄公 秦隴西王碩德攻姜乳於上邽乳率衆降秦以碩德為秦州牧鎮上邽徵乳為尚書彊熙權千成帥衆三萬共圍上邽碩德擊破之熙犇仇池遂來犇碩德西擊千成於畧陽千成降 西燕既亡其所署河東太守柳㳟等各擁兵自守秦主興遣晉王緒攻之㳟等臨河拒守緒不得濟初永嘉之亂汾隂薛氏聚其族黨阻河自固不仕劉石及苻氏興乃以禮聘薛彊拜鎮東將軍彊引秦兵自龍門濟【魏土地記曰梁山北有龍門山大禹所鑿通孟津河口廣八十步巖際鐫迹遺功尚存梁山在馮翊夏陽縣西北】遂入蒲阪㳟等皆降興以緒為并冀二州牧鎮蒲阪<br />
<br />
  資治通鑑卷一百八<br />
<br />
<史部,編年類,資治通鑑>  <br>
   </div> 

<script src="/search/ajaxskft.js"> </script>
 <div class="clear"></div>
<br>
<br>
 <!-- a.d-->

 <!--
<div class="info_share">
</div> 
-->
 <!--info_share--></div>   <!-- end info_content-->
  </div> <!-- end l-->

<div class="r">   <!--r-->



<div class="sidebar"  style="margin-bottom:2px;">

 
<div class="sidebar_title">工具类大全</div>
<div class="sidebar_info">
<strong><a href="http://www.guoxuedashi.com/lsditu/" target="_blank">历史地图</a></strong>  
<a href="http://www.880114.com/" target="_blank">英语宝典</a>  
<a href="http://www.guoxuedashi.com/13jing/" target="_blank">十三经检索</a> 
<br><strong><a href="http://www.guoxuedashi.com/gjtsjc/" target="_blank">古今图书集成</a></strong> 
<a href="http://www.guoxuedashi.com/duilian/" target="_blank">对联大全</a> <strong><a href="http://www.guoxuedashi.com/xiangxingzi/" target="_blank">象形文字典</a></strong> 

<br><a href="http://www.guoxuedashi.com/zixing/yanbian/">字形演变</a>  <strong><a href="http://www.guoxuemi.com/hafo/" target="_blank">哈佛燕京中文善本特藏</a></strong>
<br><strong><a href="http://www.guoxuedashi.com/csfz/" target="_blank">丛书&方志检索器</a></strong> <a href="http://www.guoxuedashi.com/yqjyy/" target="_blank">一切经音义</a>  

<br><strong><a href="http://www.guoxuedashi.com/jiapu/" target="_blank">家谱族谱查询</a></strong>  <strong><a href="http://shufa.guoxuedashi.com/sfzitie/" target="_blank">书法字帖欣赏</a></strong> 
<br>

</div>
</div>


<div class="sidebar" style="margin-bottom:0px;">

<font style="font-size:22px;line-height:32px">QQ交流群9:489193090</font>


<div class="sidebar_title">手机APP 扫描或点击</div>
<div class="sidebar_info">
<table>
<tr>
	<td width=160><a href="http://m.guoxuedashi.com/app/" target="_blank"><img src="/img/gxds-sj.png" width="140"  border="0" alt="国学大师手机版"></a></td>
	<td>
<a href="http://www.guoxuedashi.com/download/" target="_blank">app软件下载专区</a><br>
<a href="http://www.guoxuedashi.com/download/gxds.php" target="_blank">《国学大师》下载</a><br>
<a href="http://www.guoxuedashi.com/download/kxzd.php" target="_blank">《汉字宝典》下载</a><br>
<a href="http://www.guoxuedashi.com/download/scqbd.php" target="_blank">《诗词曲宝典》下载</a><br>
<a href="http://www.guoxuedashi.com/SiKuQuanShu/skqs.php" target="_blank">《四库全书》下载</a><br>
</td>
</tr>
</table>

</div>
</div>


<div class="sidebar2">
<center>


</center>
</div>

<div class="sidebar"  style="margin-bottom:2px;">
<div class="sidebar_title">网站使用教程</div>
<div class="sidebar_info">
<a href="http://www.guoxuedashi.com/help/gjsearch.php" target="_blank">如何在国学大师网下载古籍?</a><br>
<a href="http://www.guoxuedashi.com/zidian/bujian/bjjc.php" target="_blank">如何使用部件查字法快速查字?</a><br>
<a href="http://www.guoxuedashi.com/search/sjc.php" target="_blank">如何在指定的书籍中全文检索?</a><br>
<a href="http://www.guoxuedashi.com/search/skjc.php" target="_blank">如何找到一句话在《四库全书》哪一页?</a><br>
</div>
</div>


<div class="sidebar">
<div class="sidebar_title">热门书籍</div>
<div class="sidebar_info">
<a href="/so.php?sokey=%E8%B5%84%E6%B2%BB%E9%80%9A%E9%89%B4&kt=1">资治通鉴</a> <a href="/24shi/"><strong>二十四史</strong></a>&nbsp; <a href="/a2694/">野史</a>&nbsp; <a href="/SiKuQuanShu/"><strong>四库全书</strong></a>&nbsp;<a href="http://www.guoxuedashi.com/SiKuQuanShu/fanti/">繁体</a>
<br><a href="/so.php?sokey=%E7%BA%A2%E6%A5%BC%E6%A2%A6&kt=1">红楼梦</a> <a href="/a/1858x/">三国演义</a> <a href="/a/1038k/">水浒传</a> <a href="/a/1046t/">西游记</a> <a href="/a/1914o/">封神演义</a>
<br>
<a href="http://www.guoxuedashi.com/so.php?sokeygx=%E4%B8%87%E6%9C%89%E6%96%87%E5%BA%93&submit=&kt=1">万有文库</a> <a href="/a/780t/">古文观止</a> <a href="/a/1024l/">文心雕龙</a> <a href="/a/1704n/">全唐诗</a> <a href="/a/1705h/">全宋词</a>
<br><a href="http://www.guoxuedashi.com/so.php?sokeygx=%E7%99%BE%E8%A1%B2%E6%9C%AC%E4%BA%8C%E5%8D%81%E5%9B%9B%E5%8F%B2&submit=&kt=1"><strong>百衲本二十四史</strong></a>  <a href="http://www.guoxuedashi.com/so.php?sokeygx=%E5%8F%A4%E4%BB%8A%E5%9B%BE%E4%B9%A6%E9%9B%86%E6%88%90&submit=&kt=1"><strong>古今图书集成</strong></a>
<br>

<a href="http://www.guoxuedashi.com/so.php?sokeygx=%E4%B8%9B%E4%B9%A6%E9%9B%86%E6%88%90&submit=&kt=1">丛书集成</a> 
<a href="http://www.guoxuedashi.com/so.php?sokeygx=%E5%9B%9B%E9%83%A8%E4%B8%9B%E5%88%8A&submit=&kt=1"><strong>四部丛刊</strong></a>  
<a href="http://www.guoxuedashi.com/so.php?sokeygx=%E8%AF%B4%E6%96%87%E8%A7%A3%E5%AD%97&submit=&kt=1">說文解字</a> <a href="http://www.guoxuedashi.com/so.php?sokeygx=%E5%85%A8%E4%B8%8A%E5%8F%A4&submit=&kt=1">三国六朝文</a>
<br><a href="http://www.guoxuedashi.com/so.php?sokeytm=%E6%97%A5%E6%9C%AC%E5%86%85%E9%98%81%E6%96%87%E5%BA%93&submit=&kt=1"><strong>日本内阁文库</strong></a> <a href="http://www.guoxuedashi.com/so.php?sokeytm=%E5%9B%BD%E5%9B%BE%E6%96%B9%E5%BF%97%E5%90%88%E9%9B%86&ka=100&submit=">国图方志合集</a> <a href="http://www.guoxuedashi.com/so.php?sokeytm=%E5%90%84%E5%9C%B0%E6%96%B9%E5%BF%97&submit=&kt=1"><strong>各地方志</strong></a>

</div>
</div>


<div class="sidebar2">
<center>

</center>
</div>
<div class="sidebar greenbar">
<div class="sidebar_title green">四库全书</div>
<div class="sidebar_info">

《四库全书》是中国古代最大的丛书,编撰于乾隆年间,由纪昀等360多位高官、学者编撰,3800多人抄写,费时十三年编成。丛书分经、史、子、集四部,故名四库。共有3500多种书,7.9万卷,3.6万册,约8亿字,基本上囊括了古代所有图书,故称“全书”。<a href="http://www.guoxuedashi.com/SiKuQuanShu/">详细>>
</a>

</div> 
</div>

</div>  <!--end r-->

</div>
<!-- 内容区END --> 

<!-- 页脚开始 -->
<div class="shh">

</div>

<div class="w1180" style="margin-top:8px;">
<center><script src="http://www.guoxuedashi.com/img/plus.php?id=3"></script></center>
</div>
<div class="w1180 foot">
<a href="/b/thanks.php">特别致谢</a> | <a href="javascript:window.external.AddFavorite(document.location.href,document.title);">收藏本站</a> | <a href="#">欢迎投稿</a> | <a href="http://www.guoxuedashi.com/forum/">意见建议</a> | <a href="http://www.guoxuemi.com/">国学迷</a> | <a href="http://www.shuowen.net/">说文网</a><script language="javascript" type="text/javascript" src="https://js.users.51.la/17753172.js"></script><br />
  Copyright &copy; 国学大师 古典图书集成 All Rights Reserved.<br>
  
  <span style="font-size:14px">免责声明:本站非营利性站点,以方便网友为主,仅供学习研究。<br>内容由热心网友提供和网上收集,不保留版权。若侵犯了您的权益,来信即刪。scp168@qq.com</span>
  <br />
ICP证:<a href="http://www.beian.miit.gov.cn/" target="_blank">鲁ICP备19060063号</a></div>
<!-- 页脚END --> 
<script src="http://www.guoxuedashi.com/img/plus.php?id=22"></script>
<script src="http://www.guoxuedashi.com/img/tongji.js"></script>

</body>
</html>
