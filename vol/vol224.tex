\chapter{資治通鑑卷二百二十四}
宋 司馬光 撰

胡三省 音註

唐紀四十|{
	起旃蒙大荒落閏月盡昭陽赤奮若凡八年有奇}


代宗睿文孝武皇帝中之上

永泰元年閏十月乙巳郭子儀入朝子儀以靈武初復|{
	僕固懷恩死始復靈武}
百姓彫弊戎落未安請以朔方軍糧使三原路嗣恭鎮之|{
	軍糧使即糧料使}
河西節度使楊志烈既死|{
	楊志烈死見上卷廣德二年}
請遣使廵撫河西及置凉甘肅瓜沙等州長史上皆從之 丁未百官請納職田充軍糧|{
	唐制一品職分田十二頃二品十頃三品九頃四品七頃五品六頃六品四頃七品三頃五十畝八品二頃五十畝九品二頃皆給百里内之地諸州都督都護親王府官二品十二頃三品十頃四品八頃五品七頃六品五頃七品四頃八品三頃九品二頃五十畝鎮戍關津岳凟官五品五頃六品三頃五十畝七品三頃八品二頃九品一頃五十畝三衛中郎將上府折衝都尉六頃中府五頃五十畝下府及郎將五頃上府果毅都尉四頃中府三頃五十畝下府三頃上府長史别將三頃中府下府二頃五十畝親王府典軍五頃五十畝副典軍四頃千牛備身左右太子千牛備身三頃折衝上府兵曹二頃中府下府一頃五十畝外列軍校尉一頃二十畝旅帥一頃隊正副八十畝親王以下又有永業田百頃職事官一品六十頃郡王職事官從一品五十頃國公職事官從二品三十五頃郡公職事官從三品二十五頃縣公職事官從三品二十頃侯職事官從四品十二頃子職事官從五品八頃男職事官從五品五頃六品七品二頃五十八畝八品九品二頃上柱國三十頃柱國二十五頃上護軍二十頃護軍十五頃上輕車都尉十頃輕車都尉七頃上騎都尉六頃騎都尉四頃驍騎飛騎尉八十畝雲騎武騎尉六十畝散官五品以上給同職事官}
許之戊申以戶部侍郎路嗣恭為朔方節度使|{
	嗣祥吏翻使疎吏翻}
嗣恭披荆棘立軍府威令大行 己酉郭子儀還河中|{
	子儀自朝京師還鎮河中還從宣翻又音如字}
初劒南節度使嚴武奏將軍崔旰為利州刺史時蜀中新亂山賊塞路|{
	塞昔則翻}
旰討平之及武再鎮劍南賂山南西道節度使張獻誠以求旰|{
	利州古苴侯邑秦漢為葭萌之地蜀漢為漢夀縣晉為晋壽梁為黎州尋又改利州天寶為益昌郡乾元復為利州山南西道廵属也旰古旦翻}
獻誠使旰移疾自解詣武武以為漢州刺史使將兵擊吐蕃於西山連拔其數城攘地數百里|{
	漢州漢雒縣什方綿竹地唐垂拱立漢州天寶為德陽郡乾元復為州}
武作七寶轝迎旰入成都以寵之武薨|{
	見上卷四月}
行軍司馬杜濟知軍府事都知兵馬使郭英幹英乂之弟也與都虞侯郭嘉琳共請英乂為節度使旰時為西山都知兵馬使與所部共請大將王崇俊為節度使會朝廷已除英乂英乂由是銜之至成都數日即誣崇俊以罪而誅之召旰還成都旰辭以備吐蕃未可歸英乂愈怒絶其餽餉以困之旰轉徙入深山英乂自將兵攻之聲言助旰拒守會大雪山谷深數尺|{
	深式禁翻}
士馬凍死者甚衆旰出兵擊之英乂大敗收餘兵纔及千人而還|{
	還從宣翻又音如字}
英乂為政嚴暴驕奢不恤士卒衆心離怨玄宗之離蜀也|{
	之離力智翻肅宗至德元載玄宗離成都}
以所居行宫為道士觀|{
	觀古玩翻}
仍鑄金為真容英乂愛其竹樹茂美奏為軍營因徙去真容自居之旰宣言英乂反不然何以徙真容自居其處於是帥所部五千餘人襲成都辛巳戰于城西英乂大敗旰遂入成都屠英乂家英乂單騎奔簡州|{
	宋白曰簡州漢牛鞞縣地隋仁夀三年分益州之陽安平泉資州之資陽置簡州州有賴簡池因名}
普州刺史韓澄殺英乂送首於旰卭州牙將栢茂琳瀘州牙將楊子琳劍州牙將李昌巙|{
	宋白曰卭州漢臨卭縣梁武陵王紀置卭州取南界卭來山為名瀘州漢江陽縣梁置瀘州取瀘水為名劍州漢廣漢之梓潼縣梁置安州西魏為始州唐先天二年改為劍州取劍閣為名巙奴刀翻 考異曰唐歷作李昌夔今從實錄}
各舉兵討旰蜀中大亂旰衛州人也 華原令顧繇上言元載子伯和等招權受賄十二月戊戍繇坐流錦州|{
	宋白曰唐垂拱二年分辰州麻陽縣地并開山洞置錦州舊志錦州至京師三千五百里}
自安史之亂國子監室堂頹壞軍士多借居之祭酒蕭昕上言學校不可遂廢

大歷元年|{
	是年十一月方改元}
春正月乙酉勑復補國子學生丙戌以戶部尚書劉晏為都畿河南淮南江南湖南荆南山南東道轉運常平鑄錢鹽鐵等使侍郎第五琦為京畿關内河東劍南山南西道轉運等使分理天下財賦 周智光至華州|{
	周智光還華州見上卷上年}
益驕横|{
	横戶孟翻}
召之不至上命杜冕從張獻誠於山南以避之智光遣兵於商山邀之不獲智光自知罪重乃聚亡命無賴子弟衆至數萬縱其剽掠以悦其心|{
	剽匹妙翻}
擅留關中所漕米二萬斛藩鎮貢獻往往殺其使者而奪之 二月丁亥朔釋奠於國子監|{
	唐制中春中秋釋奠于文宣王皆以上丁戊以祭酒司業博士三獻}
命宰相帥常參官|{
	常参官常朝日常赴朝参者也唐制文官五品以上及兩省供奉官監察御史員外郎太常博士日参號常参官武官三品以上三日一朝號九参官五品以上及新行當番者五日一朝號六参官弘文崇文館國子監學生四時参凡諸王入朝及以恩追至者日参其文武官職事九品以上及二王後則朝朔望而已帥讀曰率下同相息亮翻}
魚朝恩帥六軍諸將往聽講|{
	朝直遥翻將即亮翻}
子弟皆服朱紫為諸生朝恩既貴顯乃學講經為文僅能執筆辨章句遽自謂才兼文武人莫敢與之抗辛卯命有司修國子監 元載專權恐奏事者攻訐其私|{
	載祖亥翻又如字訐居謁翻}
乃請百官凡論事皆先白長官長官白宰相然後奏聞仍以上旨諭百官曰比日諸司奏事煩多所言多讒毁|{
	長知兩翻比毗至翻}
故委長官宰相先定其可否刑部尚書顔真卿上疏以為郎官御史陛下之耳目|{
	尚辰羊翻上時掌翻疏所據翻郎官者尚書省曹二十四司郎官御史者御史臺三院御史}
今使論事者先白宰相是自掩其耳目也陛下患羣臣之為讒何不察其言之虚實若所言果虛宜誅之果實宜賞之不務為此而使天下謂陛下厭聽覧之煩託此為辭以塞諫争之路臣竊為陛下惜之|{
	塞昔則翻争讀曰諍為于偽翻}
太宗著門司式云|{
	唐式三十三篇以尚書省諸曹及秘書太常司農光禄太府太僕少府及監門宿衛計帳為其篇目}
其無門籍人有急奏者皆令門司與仗家引奏|{
	唐制門籍流内記官爵姓名流外記年齒狀貌月一易其籍非遷解不除無門籍者有急奏則令門司與仗家引奏仗家宿衛五仗之執事者令力丁翻}
無得關碍所以防壅蔽也天寶以後李林甫為相深疾言者道路以目上意不下逮下情不上達蒙蔽喑嗚|{
	史炤曰喑嗚語喑唖不明}
卒成幸蜀之禍|{
	卒子恤翻}
陵夷至于今日其所從來者漸矣夫人主大開不諱之路羣臣猶莫敢盡言况今宰相大臣裁而抑之則陛下所聞見者不過三數人耳天下之士從此鉗口結舌陛下見無復言者以為天下無事可論是林甫復起於今日也昔林甫雖擅權羣臣有不諮宰相輒奏事者則託以他事隂中傷之|{
	復扶又翻中竹仲翻}
猶不敢明令百司奏事皆先白宰相也陛下儻不早寤漸成孤立後雖悔之亦無及矣載聞而恨之奏真卿誹謗乙未貶峽州别駕|{
	峽州夷陵郡舊志京師東南一千一百八十八里}
己亥命大理少卿楊濟修好於吐蕃|{
	少詩照翻好呼到翻吐從暾入聲}
壬子以杜鴻漸為山南西道劒南東西川副元帥劒南西川節度使以平蜀亂|{
	崔旰之亂也帥所類翻使疏吏翻}
以四鎮北庭行營節度使馬璘兼邠寜節度使璘以段秀實為三使都虞候|{
	三使四鎮一也北庭二也邠寜三也}
卒有能引弓重二百四十斤者 |{
	考異曰舊傳作能引二十四弓今從段公别傳}
犯盗當死璘欲生之秀實曰將有愛憎而法不一雖韓彭不能為理璘善其議竟殺之璘處事或不中理|{
	處昌呂翻中竹仲翻}
秀實力争之璘有時怒甚左右戰栗秀實曰秀實罪若可殺何以怒為無罪殺人恐涉非道璘拂衣起秀實徐步而出良久璘置酒召秀實謝之自是軍州事皆咨秀實而後行璘由是在邠寜聲稱殊美|{
	稱尺證翻}
癸丑以山南西道節度使張獻誠兼劍南東川節度使卭州刺史柏茂琳為卭南防禦使|{
	卭南卭水以南也卭水出嚴道卭崍山入青衣江}
以崔旰為茂州刺史充西山防禦使三月癸未獻誠與旰戰于梓州獻誠軍敗僅以身免旌節皆為旰所奪 夏五月河西節度使楊休明徙鎮沙州|{
	凉州淪䧟故也}
秋八月國子監成丁亥釋奠|{
	記文王世子凡始立學者必釋奠于先聖先師攷之鄭注凡學四時皆有釋奠釋奠者設薦饌酌奠而已無迎尸以下之事記又曰始立學者既釁器用幣然後釋菜不舞不授器注云釋菜禮輕也釋奠則舞舞則授器司馬之属司兵司戈司盾祭祀授舞者兵也周禮大胥春入學釋菜合舞注云合舞等其進退鄭司農云舍菜謂舞者皆持芬香之菜秦漢釋奠無文魏則以太常行事晋宋以學官主祭南齊武帝時有司奏釋奠先聖先師禮文又有釋菜未詳今當行何禮用何樂時從喻希議用元嘉故事設軒懸之樂六佾之舞牲牢器用悉依上公梁及北齊車駕視學皆親釋奠唐春秋釋奠三獻皆以學官太宗貞觀十四年親釋奠於國學玄宗開元十一年詔春秋釋奠用牲牢}
魚朝恩執易升高座講鼎覆餗以譏宰相|{
	易曰鼎折足覆公餗言三公鼎足承君苟非其人則折足而覆亂美實餗音桑谷翻鼎實也}
王縉怒元載怡然 |{
	考異曰是時縉留守東都而得預此會者按實錄明年二月郭子儀入朝詔元載王縉宴於其第然則雖守東都有時朝京師也}
朝恩謂人曰怒者常情笑者不可測也 杜鴻漸至蜀境聞張獻誠敗而懼使人先達意於崔旰許以萬全旰卑辭重賂以迎之鴻漸喜進至成都見旰但接以溫恭無一言責其干紀州府事悉以委旰又數薦之於朝因請以節制讓旰|{
	數所角翻}
以柏茂琳楊子琳李昌巙各為本州刺史上不得已從之|{
	杜鴻漸習知朝廷之務姑息故敢以崔旰等請}
壬寅以旰為成都尹西川節度行軍司馬|{
	崔旰遂據有西川}
甲辰以魚朝恩行内侍監判國子監事中書舍人京兆常衮上言成均之任當用名儒|{
	五帝名學曰成均垂拱元年改國子監曰成均義取此也尋復其舊常衮謂國子監為成均亦猶今人言太學為辟雍耳}
不宜以宦者領之丁未命宰相以下送朝恩上|{
	上時掌翻}
京兆尹黎幹自南山引澗水穿漕渠入長安功竟不成 冬十月乙未上生日|{
	開元十四年十月十三日上生以其日為天興聖節}
諸道節度使獻金帛器服珍玩駿馬為壽共直緡錢二十四萬常衮上言以為節度使非能男耕女織必取之於人歛怨求媚不可長也|{
	長知兩翻}
請却之上不聽京兆尹第五琦什一稅法民苦其重多流亡十一月

甲子日南至赦改元|{
	改元大歷}
悉停什一稅法|{
	行什一稅見上卷上年}
十二月癸卯周智光殺陜州監軍張志斌|{
	斌音彬}
智光

素與陜州刺史皇甫溫不恊志斌入奏事智光館之|{
	自陜州入奏事道過華州館古玩翻}
志斌責其部下不肅智光怒曰僕固懷恩不反正由汝輩激之我亦不反今日為汝反矣|{
	為于偽翻}
叱下斬之臠食其肉|{
	史炤曰臠切其肉以食也}
朝士舉選人畏智光之暴多自同州竊過|{
	選須絹翻}
智光遣將將兵邀之於路死者甚衆|{
	將即亮翻}
戊申詔加智光檢校左僕射遣中使余元仙持告身授之智光慢罵曰智光有大功於天下國家不與平章事而與僕射且同華地狹不足展材若益以陜虢商鄜坊五州庶猶可耳因歷數大臣過失|{
	數所具翻}
且曰此去長安百八十里智光夜眠不敢舒足恐踏破長安城至於挾天子令諸侯惟周智光能之元仙股慄郭子儀屢請討智光上不許 郭子儀以河中軍食常乏乃自耕百畝將校以是為差於是士卒皆不勸而耕是歲河中野無曠土軍有餘糧|{
	史言郭子儀忠勤為國}
以隴右行軍司馬陳少遊為桂管觀察使|{
	桂管領桂昭賀富藤梧潘白亷繡欽横邕融柳貴十七州劉昫曰桂管十五州在廣州西桂昭富梧蒙龔潯欎林平琴賓澄繡象柳融}
少遊博州人也為吏彊敏而好賄|{
	好呼到翻}
善結權貴以是得進既得桂州惡其道遠多瘴癘|{
	惡烏路翻}
宦官董秀掌樞密|{
	是後遂以中官為樞密使}
少遊請歲獻五萬緡又納賄於元載子仲武内外引薦|{
	董秀引薦於内元載引薦於外}
數日改宣歙觀察使|{
	宋白曰乾元元年停采訪使及諸道黜陟使置觀察處置使其年李峘除都督江淮節度宣慰觀察處置使今按李峘之任重矣陳少遊止觀察一二州或數州其任在節度使下}


二年春正月丁巳密詔郭子儀討周智光子儀命大將渾瑊李懷光軍于渭上智光麾下聞之皆有離心己未智光大將李漢惠自同州帥所部降於子儀|{
	帥讀曰率下同降戶江翻}
壬戌貶智光澧州刺史|{
	宋白曰澧州漢零陽縣地吳立天門郡隋置松州尋改澧州州在澧水之陽故名舊志州在京師東南一千八百九十三里澧音禮}
甲子華州牙將姚懷李延俊殺智光以其首來獻淮西節度使李忠臣入朝以收華州為名帥所部兵大掠自潼關至赤水|{
	水經注渭水東過鄭縣北又東與赤水合九域志華州鄭縣有赤水鎮}
二百里間財蓄殆盡官吏有衣紙|{
	衣於既翻}
或數日不食者己巳置潼關鎮兵二千人 壬申分劍南置東川觀察使鎮遂州|{
	合東西川見上卷廣德二年}
二月丙戌郭子儀入朝上命元載王縉魚朝恩等互置酒於其第一會之費至十萬緡上禮重子儀常謂之大臣而不名郭曖嘗與昇平公主争言|{
	永泰元年下嫁郭曖事見上卷}
曖曰汝倚乃父為天子邪我父薄天子不為公主恚奔車奏之上曰此非汝所知彼誠如是使彼欲為天子天下豈汝家所有邪慰諭令歸子儀聞之囚曖入待罪上曰鄙諺有之|{
	史炤曰鄙諺俚俗所傳之言也}
不痴不聾不作家翁兒女子閨房之言何足聽也子儀歸杖曖數十 夏四月庚子命宰相魚朝恩與吐蕃盟於興唐寺|{
	命宰相及魚朝恩也}
杜鴻漸請入朝奏事以崔旰知西川留後 |{
	考異曰舊鴻漸傳云鴻漸仍帥旰同入覲寜傳云鴻漸請旰為行軍司馬仍賜名寧鴻漸歸遂授寜西川節度使至十四年始入朝實錄亦無随鴻漸入朝事鴻漸傳誤也}
六月甲戌鴻漸來自成都廣為貢獻因盛陳利害薦旰才堪寄任上亦務姑息乃留鴻漸復知政事|{
	復扶又翻}
秋七月丙寅以旰為西川節度使杜濟為東川節度使旰厚歛以賂權貴|{
	歛力贍翻}
元載擢旰弟寛至御史中丞寛兄審至給事中 丁卯魚朝恩奏以先所賜莊為章敬寺|{
	據舊史章敬寺在通化門外}
以資章敬太后冥福|{
	上母吴后諡章敬}
於是窮壯極麗盡都市之財不足用奏毁曲江及華清宫館以給之|{
	財當作材都市之材謂材木積於都市者長安朱雀街東第五街皇城之東第三街昇道坊龍華尼寺南有流水屈曲謂之曲江此地在秦為宜春院隑州在漢為樂遊園開元疏鑿遂為勝境其南有紫雲樓芙蓉苑其西有杏園慈恩寺江側菰蒲䓤翠柳隂四合碧波紅蕖依映可愛華清宫見二百十五卷天寶六載}
費逾萬億|{
	孔潁達曰億之數有大小二法其小數以十為等十萬為億十億為兆也其大數以萬為等數萬至萬是萬萬為億又從億而數至萬萬億為兆}
衛州進士高郢上書略曰先太后聖德不必以一寺增輝國家永圖無寜以百姓為本捨人就寺何福之為又曰無寺猶可無人其可乎又曰陛下當卑宫室以夏禹為法而崇塔廟踵梁武之風乎又上書略曰古之明王積善以致福不費財以求福修德以消禍不勞人以禳禍今興造急促晝夜不息力不逮者隨以榜笞|{
	榜音彭}
愁痛之聲盈於道路以此望福臣恐不然又曰陛下迴正道於内心求微助於外物徇左右之過計傷皇王之大猷臣竊為陛下惜之|{
	為于偽翻}
皆寢不報 |{
	考異曰郢集前書八月二十五日後書九月十三日上今因造寺終言之}
始上好祠祀未甚重佛元載王縉杜鴻漸為相三人皆好佛縉尤甚不食葷血|{
	葷臭棐也血者殺六畜而取之好呼到翻}
與鴻漸造寺無窮上嘗問以佛言報應果為有無載等奏以國家運祚靈長非宿植福業何以致之福業已定雖時有小災終不能為害所以安史悖逆方熾而皆有子禍|{
	謂安慶緒殺祿山史朝義殺思明也}
僕固懷恩稱兵内侮出門病死回紇吐蕃大舉深入不戰而退|{
	事並見上卷上年}
此皆非人力所及豈得言無報應也上由是深信之常於禁中飯僧百餘人|{
	飯扶晚翻}
有寇至則令僧講仁王經以禳之|{
	所謂護國仁王經也}
寇去則厚加賞賜胡僧不空官至卿監爵為國公出入禁闥勢移權貴京畿良田美利多歸僧寺勑天下無得箠曳僧尼造金閣寺於五臺山|{
	忻州五臺縣有五臺山釋氏相傳以為文殊道場注詳見後}
鑄銅塗金為瓦所費鉅億|{
	鉅億者億億也此言大數之億}
縉給中書符牒令五臺僧數十人散之四方求利以營之載等每侍上從容多談佛事|{
	從千容翻}
由是中外臣民承流相化皆廢人事而奉佛政刑日紊矣|{
	紊亡運翻}
八月庚辰鳳翔等道節度使左僕射平章事李抱玉入朝|{
	李抱玉以陳鄭澤潞鳳翔等道節度使防吐蕃}
固讓僕射言辭確至|{
	確堅也固也}
上許之癸丑又讓鳳翔節度使不許 丁酉杜鴻漸飯千僧以使蜀無恙故也|{
	飯扶晩翻}
九月吐蕃衆數萬圍靈州遊騎至潘原宜禄|{
	潘原本古隂盤縣天寶元年更名潘原属涇州}
詔郭子儀自河中帥甲士三萬鎮涇陽京師戒嚴甲子子儀移鎮奉天 |{
	考異曰汾陽家傳八月十七日吐蕃至涇西二十七日詔統精卒一萬與馬璘合攻之今從實録實録甲寅寇靈州乙卯寇宜禄盖據奏到日今從唐歷}
山獠䧟桂州逐刺史李良|{
	獠魯皓翻}
冬十月戊寅朔方節度使路嗣恭破吐蕃於靈州城下|{
	考異曰唐歷九月吐蕃圍靈武戊申嗣恭破吐蕃按長歷戊申九月一日也今從實録}
斬首二千

餘級吐蕃引去 十二月庚辰盗發郭子儀父冢捕之不獲人以為魚朝恩素惡子儀疑其使之|{
	惡烏路翻}
子儀自奉天入朝朝廷憂其為變子儀見上上語及之子儀流涕曰臣久將兵不能禁暴軍士多發人冢今日及此乃天譴非人事也朝廷乃安 是歲復以鎭西為安西|{
	改鎮西見二百二十卷肅宗至德元載復扶又翻}
新羅王憲英卒子乾運立三年春正月乙丑上幸章敬寺度僧尼千人 贈建寜王倓為齊王|{
	倓死見二百十九卷至德二載}
二月癸巳商州兵馬使劉洽殺防禦使殷仲卿尋討平之 甲午郭子儀禁無故軍中走馬南陽夫人乳母之子犯禁|{
	子儀妻封南陽夫人}
都虞候杖殺之諸子泣訴於子儀且言都虞候之横|{
	宇文泰相魏置虞候都督後世因之置虞候之官横戶孟翻}
子儀叱遣之明日以事語僚佐而歎息|{
	語牛倨翻}
曰子儀諸子皆奴材也不賞父之都虞候而惜母之乳母子非奴材而何 庚子以後宫獨孤氏為貴妃 三月乙巳朔日有食之 夏四月戊寅山南西道節度使張獻誠以疾舉從父弟右羽林將軍獻恭自代|{
	從才用翻}
上許之 壬寅西川節度使崔旰入朝 初上遣中使徵李泌於衡山|{
	泌歸衡山見二百二十卷至德二載}
既至復賜金紫|{
	泌從肅宗於靈武已賜金紫既歸衡山反其初服今復賜之復扶又翻}
為之作書院於蓬萊殿側|{
	蓬萊殿在紫宸殿北蓬萊殿北有太液池池中有蓬萊山為于偽翻下為泌同}
上時衣汗衫躡屨過之|{
	汗衫宴居之常服也今通貴賤皆服之惟天子以黄為别炙轂子曰燕朝衮冕有白紗中單有明衣皆汗衫之象以行祭接神漢高祖與項羽交戰汗透中單改名汗衫貴賤通服衣音於既翻過古禾翻}
自給舍以上|{
	給舍者謂給事中中書舍人皆唐正五品官也}
及方鎮除拜軍國大事皆與之議又使魚朝恩於白花屯為泌作外院使與親舊相見上欲以泌為門下侍郎同平章事泌固辭|{
	泌毗必翻 考異曰鄴侯家傳曰固辭以讓元載按載時已為相何讓之有又曰到山四歲而二聖登遐代宗踐阼命中人手詔馹騎徵先公於衡岳先是半年前先公夜遇盗三人為其所拉而殺之於懸澗及日出乃寤下籍樹葉丈餘都無所傷緣巖攀蘿而出不敢至舊居山中人初以為仙去及中貴將至先公大懼沐浴更衣以俟命乃代宗踐阼之徵也疑盗為張后輔國所遣亦竟不知其由按玄肅登遐泌雖在山林豈容全不知如家傳所言是代宗纔立即召泌也須經幸陜泌豈得全無一言召泌必在幸陜之後李繁誤記耳}
上曰機務之煩不得晨夕相見誠不若且居密近何必署勑然後為宰相邪|{
	相息亮翻邪音耶}
後因端午王公妃主各獻服玩上謂泌曰先生何獨無所獻對曰臣居禁中自巾至履皆陛下所賜所餘惟一身耳何以為獻上曰朕所求正在此耳泌曰臣身非陛下有誰則有之上曰先帝欲以宰相屈卿而不能得|{
	見二百一十八卷肅宗至德元載七月}
自今既獻其身當惟朕所為不為卿有矣泌曰陛下欲使臣何為上曰朕欲卿食酒肉有室家受禄位為俗人泌泣曰臣絶粒二十餘年陛下何必使臣隳其志乎上曰泣復何益|{
	復扶又翻}
卿在九重之中欲何之乃命中使為泌葬二親|{
	重直龍翻使疎吏翻為于李翻下又為同}
又為泌娶盧氏女為妻 |{
	考異曰鄴侯家傳云永泰元年端午上令泌食酒結婚按下云阿足師竊氈履置紫宸上欲使内人護燈燭泌曰臣六七年在此又曰况新賜婚上即位至永泰纔四年耳又云因此得謗元載遂因魚朝恩事排出之然則結婚與朝恩誅不相遠今盖因追贈承天言之}
資費皆出縣官賜第於光福坊令泌數日宿第中數日宿蓬萊院上與泌語及齊王倓欲厚加褒贈泌請用岐薛故事贈太子|{
	令力丁翻倓徒甘翻岐王範贈惠文太子薛王業贈惠宣太子皆在玄宗朝}
上泣曰吾弟首建靈武之議成中興之業|{
	事見二百一十九卷肅宗至德元載}
岐薛豈有此功乎竭誠忠孝乃為讒人所害曏使尚存朕必以為太弟今當崇以帝號成吾夙志乙卯制追諡倓曰承天皇帝庚申葬順陵|{
	諡申至翻順陵在咸陽縣咸陽原 考異曰鄴俟家傳曰命使自彭原迎喪葬齊陵今從實錄葬順陵}
崔旰之入朝也以弟寛為留後|{
	是年四月壬寅崔旰入朝朝直遥翻}
瀘州刺史楊子琳帥精騎數千乘虚突入成都|{
	瀘音盧帥讀曰率騎奇計翻}
朝廷聞之加旰檢校工部尚書賜名寜|{
	校古孝翻尚辰羊翻 考異曰舊傳旰初為杜鴻漸行軍司馬即改名寜今從實録}
遣還鎮 六月壬辰幽州兵馬使朱希彩經略副使昌平朱泚|{
	使疎吏翻昌平本漢軍都縣後魏更名昌平縣唐属幽州泚且禮翻又音此}
泚弟滔共殺節度使李懷仙希彩自稱留後閏月成德軍節度使李寶臣遣將將兵討希彩為希彩所敗|{
	敗補邁翻}
朝廷不得已宥之庚申以王縉領盧龍節度使丁卯以希彩領幽州留後 崔寛與楊子琳戰數不利|{
	數所角翻}
秋七月崔寜妾任氏出家財數十萬募兵得數千人帥以擊子琳破之|{
	任音壬帥讀曰率考異曰實録五月子琳襲據成都即日詔寜還成都七月壬申又云子琳寇成都遂據其城寜弟寛破之盖五月奏據城七月奏破之成功雖因任氏奏時須著寛名故也}
子琳走 乙亥王縉如幽州朱希彩盛兵嚴備以逆之縉晏然而行希彩迎謁甚恭縉度終不可制|{
	度徒洛翻}
勞軍旬餘日而還|{
	勞力到翻還從宣翻又如字}
回紇可敦卒|{
	紇下沒翻卒子恤翻}
庚辰以右散騎常侍蕭昕為

弔祭使|{
	散昔亶翻騎奇計翻昕許斤翻使疎吏翻}
回紇庭詰昕曰我於唐有大功唐奈何失信市我馬不時歸其直昕曰回紇之功唐已報之矣僕固懷恩之叛回紇助之與吐蕃連兵入寇逼我郊畿|{
	詰去吉翻}
及懷恩死吐蕃走然後回紇懼而請和我唐不忘前功加惠而縱之|{
	事見上卷永泰元年}
不然匹馬不歸矣乃回紇負約豈唐失信邪回紇慙厚禮而歸之丙戍内出盂蘭盆賜章敬寺|{
	釋氏㿻蘭盆經目連比丘見其亡母在餓鬼中目連白佛言七月望日當為七代父母及父母在阨難中者具百味五果以著盆中供養十方佛然後受食夢華録曰中元買冥器綵衣以竹牀三脚如燈窩狀謂之盂蘭盆掛冥財衣服在上焚之釋氏要覧曰梵云盂蘭此云救倒懸盆陸游曰俗以七月望日具素饌享先織竹作盆盎貯紙錢盛以一竹焚之謂之㿻蘭盆嗚呼代宗為此以七廟神靈為安在邪}
設七廟神座書尊號於旛上百官迎謁於光順門|{
	閣本大明宫圖光順門在紫宸門之西光順門内則明義殿承歡殿}
自是歲以為常 八月壬戍吐蕃十萬衆寇靈武丁卯吐蕃尚贊摩二萬衆寇邠州京師戒嚴邠寜節度使馬璘擊破之|{
	吐從暾入聲邠卑旻翻使疏利翻璘離珍翻}
庚午河東節度使同平章事辛雲京薨以王縉領河東節度使餘如故 九月壬申命郭子儀將兵五萬屯奉天以備吐蕃|{
	將即亮翻又如字}
丁丑濟王環薨|{
	環宣宗子也濟子禮翻}
壬午朔方騎將白元光擊吐蕃破之壬辰元光又破吐蕃二萬衆於靈武|{
	騎奇計翻考異曰實録戊戌郭子儀奏靈州破吐蕃六萬餘衆百僚入賀京師解嚴盖即壬辰白元光所破也子儀合前後所破而奏之耳}
鳳翔節度使李抱玉使右軍都將臨洮李晟將兵五千擊吐蕃晟曰以力則五千不足用以謀則太多乃將千人出大震關至臨洮屠吐蕃定秦堡|{
	臨洮洮州吐蕃志吞秦土故築堡於洮州以定秦為名洮徒刀翻晟承正翻}
焚其積聚虜堡帥慕容谷種而還|{
	帥所類翻還從宣翻又音如字}
吐蕃聞之釋靈州之圍而去戊戍京師解嚴 頴州刺史李岵以事忤滑亳節度使令狐彰|{
	岵侯古翻忤五故翻}
彰使節度判官姚奭按行頴州|{
	行下孟翻}
因代岵領州事且曰岵不受代即殺之岵知之因激怒將士使殺奭與奭同死者百餘人岵走依河南節度使田神功於汴州|{
	汴皮面翻}
冬十月乙巳彰表言其狀岵亦上表自理上命給事中賀若察往按之|{
	亦上時兩翻 考異曰實録十月乙巳潁州刺史李岵殺本道節度判官姚奭及奭之弟岵弃州奔汴州本道節度使令狐彰以聞岵亦抗表上聞初岵以公務為彰所怒因遣奭廵按境内便留知潁州事岵聞之遂與親吏潜謀詐為奭書將為變使將士遺於路中潁州守將得之懼乃與岵同謀殺奭詔給事中賀若察使於潁按覆唐歷曰十月潁州將士怒殺亳州判官魏奭初令狐彰怒潁州刺史李岵遣奭代之且告之曰若岵不受替即殺之岵覺之以告將吏怒而殺奭併弟統紀作滑亳州判官姚奭又曰彰表先至遣給事中賀若察往滑州宣詔决李岵配流夷州尋賜自盡今姓名從實録統紀事則参取諸書}
丁卯郭子儀自奉天入朝|{
	朝直遥翻}
十一月丁亥以幽州留後朱希彩為節度使|{
	使疏吏翻}
郭子儀還河中|{
	自奉天入朝回還河中還從宣翻又音如字}
元載以吐蕃連歲入寇馬璘以四鎮兵屯邠寜力不能拒|{
	載徂亥翻又如字吐從暾入聲璘離珍翻邠音卑旻翻}
而郭子儀以朔方重兵鎮河中深居腹中無事之地乃與子儀及諸將議徙璘鎮涇州而使子儀以朔方兵鎭邠州曰若以邊土荒殘軍費不給則以内地租稅及運金帛以助之諸將皆以為然十二月己酉徙馬璘爲涇原節度使以邠寜慶三州隸朔方|{
	將即亮翻朔方兵分屯邠蒲始此考異曰實錄己酉以吐蕃歲犯西疆增修鎮守乃以邠寜節度馬璘鎮涇州仍為涇原節度使以邠寜慶等州隸朔方汾陽家傳四年五月詔集兵於邠郊六月公自河中遣一萬兵二十八日公如邠州舊子儀傳時以西蕃侵寇京師不安馬璘雖在邠州力不能拒乃以子儀兼邠寜慶節度自河中移鎮邠州徙馬璘涇原節度使邠志初吐蕃既退諸侯入覲是時馬鎮西以四鎮兼邠寜李公軍澤潞以防秋軍盩厔丞相元公載使人諷諸將使責已曰今四郊多壘中外未寜公執國柄有年矣安危大計一無所聞如之何載曰非所及也他日又言且曰得非曠職乎載莞然曰安危繫於大臣非獨宰臣也先王作兵置之四境所以禦戎狄也今内地無虞朔方兵在河中澤潞軍在盩厔遊軍伺寇不遠京室王畿之外豈假是邪必令損益須自此始故曰非所及也郭李曰宰臣但圖之載曰今若徙四鎮于涇朔方于邠澤潞于岐則内地無虞三邉有備三賢之意何如三公曰惟所指揮既而相謂曰我曹既為所冊得無行乎十二月詔馬公兼領涇原尋以鄭潁資之李公兼領山南猶以澤潞資之郭公兼領邠寜亦以河中資之三將皆如詔朔方軍自此大徙於邠郭公雖連統數道軍之精甲悉聚邠府其它子弟分居蒲靈各置守將以專其令蒲之餘卒稍遷於邠十年之間無遺甲矣段公别傳曰馬公朝於京師以公掌留事馬公懇奏請以邠寜慶三州讓副帥子儀令以朔方河中之軍鎮之自帥四鎮北庭之衆遷赴涇州將以拓西境代宗壯而許之十二月二日朝廷以馬公為涇原節度使盖三年立此議至四年子儀始遷邠今参取諸書}
璘先往城涇州以都虞候段秀實知邠州留後|{
	璘離珍翻邠卑旻翻}
初四鎭北庭兵遠赴中原之難|{
	事見二百十九卷至德元載難乃旦翻}
久羈旅數遷徙|{
	數所角翻}
四鎮歷汴虢鳳翔北庭歷懷絳鄜然後至邠頗積勞弊|{
	汴皮面翻鄜音夫}
及徙涇州衆皆怨誹刀斧兵馬使王童之謀作亂|{
	誹敷尾翻刀斧兵馬使領羣刀斧}
期以辛酉旦警嚴而發|{
	旦警嚴者將旦嚴鼔以警衆也周禮謂之晌今人謂之攅點}
前夕有告之者秀實陽召掌漏者怒之以其失節令每更來白|{
	掌漏者謂守漏之卒也令力丁翻更工行翻}
輒延之數刻遂四更而曙童之不果發秀實欲討之而亂迹未露恐軍中疑其寃告者又云今夕欲焚馬坊草因救火謀作亂中夕火果起秀實命軍中行者皆止坐者勿起各整部伍嚴守要害童之白請救火不許及旦捕童之及其黨八人皆斬之下令曰後徙者族流言者刑遂徙于涇|{
	史言段秀實能弭亂令力丁翻}
癸亥西川破吐蕃萬餘衆|{
	吐從暾入聲}
平盧行軍司馬許杲 |{
	考異曰舊傳作許果今從韓愈順宗實錄}
將卒三千人駐濠州不去有窺淮南意|{
	將即亮翻又如字濠州治鍾離漢當塗縣地隋改濠州因濠水而名}
淮南節度使崔圓令副使元城張萬福攝濠州刺史杲聞即提卒去止當塗|{
	此當塗縣宋属太平州本漢丹陽縣地按漢書地理志當塗属九江郡晉成帝時以江北之當塗縣流人過江者立當塗僑縣遂為實土}
是歲上召萬福以為和州刺史行營防禦使討杲萬福至州杲懼移軍上元|{
	按上元楚之金陵秦之秣陵吳之建業江左之建康晉分秣陵置臨江縣太康初又改為江寜至肅宗上元二年更今名属昇州使疏吏翻}
又北至楚州大掠|{
	許果既去濠州南渡江而屯當塗及張萬福至歷陽逼之又移上元又自上元渡江而北掠楚州也}
淮南節度使韋元甫命萬福追討之未至淮隂杲為其將康自勸所逐|{
	使疏吏翻將即亮翻}
自勸擁兵繼掠循淮而東萬福倍道追而殺之免者什二三元甫將厚賞將士萬福曰官健常虚費衣糧無所事|{
	兵農既分縣官費衣糧以養軍謂之官健猶言官所養健兒也按唐六典衛士之外天下諸軍有健兒舊健兒在軍皆有年限更來往頗為勞弊開元十五年勑以天下無虞宜與人休息自今已後諸軍鎮量閑劇利害置兵防健兒於諸色征行人内及客戶中召募取丁壯情愿充健兒長住邉軍者每年加常例給賜兼給永年優復其家口情願同者聽至軍州各給田地屋宅人賴其利中外獲安永無徵之役此當時言兵農已分之利而養兵之害卒不可救以至于今}
今方立小功不足過賞請用三分之一

四年春正月丙子郭子儀入朝魚朝恩邀之遊章敬寺|{
	朝直遥翻魚朝恩建章敬寺自以為功因子儀入朝請遊之以夸大其事}
元載恐其相結密使子儀軍吏告子儀曰朝恩謀不利於公子儀不聽吏亦告諸將將士請衷甲以從者三百人|{
	杜預曰衷甲謂在衣中從才用翻下同}
子儀曰我國之大臣彼無天子之命安敢害我若受命而來汝曹欲何為乃從家僮數人而往朝恩迎之驚其從者之約子儀以所聞告且曰恐煩公經營耳朝恩撫膺捧手流涕曰非公長者能無疑乎|{
	長知兩翻}
壬午流李岾於夷州|{
	岵侯古翻流岵直在令狐彰也}
乙酉郭子儀還河中|{
	朝而還也還從宣翻又如字}
辛卯賜李岵死 二月壬寅以京兆之好畤鳳翔之麟遊普潤隸神策軍|{
	畤音止}
從魚朝恩之請也 楊子琳既敗還瀘州招聚亡命得數千人沿江東下聲言入朝涪州守捉使王守仙伏兵黄草峽|{
	瀘音盧水經注涪州之西有黄葛峽山高險絶無人居意即此峽也按杜甫詩有黄草峽西船不歸之句注云黄草峽在涪州之西涪音浮}
子琳悉擒之擊守仙於忠州守仙僅以身免子琳遂殺夔州别駕張忠據其城荆南節度使衛伯玉欲結以為援以夔州許之|{
	使疎吏翻夔州荆南廵属}
為之請於朝|{
	為于偽翻}
陽曲人劉昌裔|{
	舊志前漢陽曲縣唐忻州定襄縣即其地也後漢移陽曲縣于太原界乃於陽曲古城置定襄縣而太原之陽曲隋開皇六年改為陽直十六年又改為汾陽惡陽曲之名也武德七年復改為陽曲縣}
子琳遣使詣闕請罪|{
	說式芮翻}
子琳從之乙巳以子琳為峽州團練使|{
	峽州夷陵郡}
初僕固懷恩死|{
	見上卷永泰元年}
上憐其有功置其女宫中養以為女回紇請以為可敦夏五月辛卯冊為崇徽公主嫁回紇可汗壬辰遣兵部侍郎李涵送之涵奏祠部郎中虞鄉董晉為判官|{
	盧鄉漢解縣後魏分置虞鄉縣貞觀十七年省解縣併入虞鄉二十年復置解縣而省虞鄉天授二年復分解縣置虞鄉縣属河中府宋白曰後魏太和九年於今虞鄉縣西十三里置南解縣周明帝廢南解以虞鄉縣属綏化郡今縣西三十四里綏化故城是也寶定四年改綏化為虞鄉縣周宋置解縣於今虞鄉城東於解縣西五十里别置虞鄉縣今邑是也}
六月丁酉公主辭行至回紇牙帳回紇來言曰唐約我為市馬既入而歸我賄不足我於使人乎取之涵懼不敢對視晉晉曰我非無馬而與爾為市為爾賜不既多乎爾之馬歲至吾數皮而歸資|{
	言不計其生死皆償馬值也}
邊吏請致詰也天子念爾有勞故下詔禁侵犯諸戎畏我大國之爾與也莫敢校焉爾之父子寜而畜馬蕃者|{
	畜呼玉翻蕃音煩}
非我誰使之於是其衆皆環晉拜|{
	環音宦}
既又相帥南面序拜皆舉兩手曰不敢有意大國|{
	此晋史韓愈狀晋之辭其言容有溢美帥讀曰率}
戊申王縉表讓副元帥都統行營使 辛酉郭子儀自河中遷于邠州其精兵皆自隨餘兵使禆將將之分守河中靈州|{
	將即亮翻}
軍士久家河中頗不樂徙|{
	樂音洛}
往往自邠逃歸行軍司馬嚴郢領留府悉捕得誅其渠帥衆心乃定|{
	邠卑旻翻郢以井翻帥所類翻}
秋九月吐蕃寇靈州丁丑朔方留後常謙光擊破之 河東兵馬使王無縱張奉璋等恃功驕蹇以王縉書生易之|{
	縉音晋易以䜴翻}
多違約束縉受詔發兵詣鹽州防秋|{
	鹽州漢五原郡地隋置鹽州治五原縣今州南抵慶州馬嶺縣北界即漢馬嶺縣地}
遣無縱奉璋將步騎三千赴之奉璋逗遛不進無縱托它事擅入太原城縉悉擒斬之并其黨七人諸將悍戾者殆盡|{
	將即亮翻又如字騎奇計翻悍侯旰翻}
軍府始安 冬十月常謙光奏吐蕃寇鳴沙首尾四十里|{
	吐從暾入聲鳴沙縣属靈州本漢富平縣地}
郭子儀遣兵馬使渾瑊將銳兵五千救靈州子儀自將進至慶州聞吐蕃退乃還|{
	使疎吏翻將即亮翻瑊古咸翻還從宣翻又如字吐從暾入聲}
黄門侍郎同平章事杜鴻漸以疾辭位壬申許之乙亥薨鴻漸病甚令僧削髮遺令為塔以葬|{
	令力丁翻遺令力定翻}
丙子以左僕射裴冕同平章事初元載為新平尉|{
	射寅謝翻新平漢上郡之白土縣後漢獻帝置新平郡至於後魏縣名猶不改西魏置州隋開皇四年改曰新平縣因郡以名縣也唐為邠州治所宋白曰新平漢漆縣地漢建安中分扶風置新平郡姚萇之亂屠廢不立後魏於今縣西南置白土縣属新平郡隋開皇四年改白土縣為新平縣唐武德以新平縣為州治所}
冕嘗薦之故載舉以為相亦利其老病易制|{
	易以䜴翻}
受命之際蹈舞仆地載趨而扶之代為謝詞十二月戊戍冕薨

五年春正月己巳羌酋白對蓬等各帥部落内屬|{
	酋慈由翻}
觀軍容宣慰處置使左監門衛大將軍兼神策軍使内侍監魚朝恩專典禁兵寵任無比|{
	處昌呂翻監古銜翻朝直遥翻}
上常與議軍國事勢傾朝野朝恩好於廣座恣談時政|{
	好呼到翻}
陵侮宰相元載雖彊辯亦供默不敢應神策都虞候劉希暹都知兵馬使王駕鶴皆有寵於朝恩希暹說朝恩於北軍置獄|{
	左右神策軍左右羽林軍左右龍武軍皆謂之北軍說式芮翻相息亮翻暹息廉翻}
使坊市惡少年羅告富室誣以罪惡捕繫地牢訊掠取服|{
	少詩照翻掠音亮}
籍沒其家貲入軍并分賞告捕者地在禁密人莫敢言朝恩每奏事以必允為期朝廷政事有不豫者輒怒曰天下事有不由我者邪|{
	朝直遥翻邪音耶}
上聞之由是不懌朝恩養子令徽尚幼為内給使衣緑|{
	唐制内給使無常員属内侍省凡無官品者號内給使掌諸門進物之歷宋白曰掌諸門進物出物之歷衣於既翻}
與同列忿争歸告朝恩朝恩明日見上曰臣子官卑為儕輩所陵|{
	儕士皆翻}
乞賜之紫衣上未應有司已執紫衣於前令徽服之拜謝上強笑曰兒服紫大宜稱心|{
	強其兩翻稱尺正翻}
愈不平元載測知上指乘間奏朝恩專恣不軌|{
	間古莧翻}
請除之上亦知天下共怨怒遂令載為方略朝恩每入殿常使射生將周皓將百人自衛|{
	令力丁翻將即亮翻皓將同又如字}
又使其黨陜州節度使皇甫溫握兵於外以為援|{
	陜失冉翻度使疎吏翻}
載皆以重賂結之故朝恩隂謀密語上一一聞之而朝恩不之覺也辛卯載為上謀|{
	為于偽翻}
徙李抱玉為山南西道節度使以温為鳳翔節度使外重其權實内温以自助也載又請割郿虢寶雞鄠盩厔隸抱玉|{
	鄠音戶盩厔音舟窒漢制右扶風有郿虢二縣及晉省虢縣存郿縣後魏於虢縣地置武都郡西魏置洛邑縣後周置朔州州尋廢隋開皇初廢郡大業初改洛邑縣為虢縣後魏又於郿縣置平陽周城二縣西魏改平陽為郿城後周廢入周城縣隋開皇十八年改周城曰渭濱大業二年改曰郿縣唐志二縣皆属鳳翔府}
興平武功天興扶風隸神策軍|{
	興平舊曰始平景龍元年更名金城至德二載更名興平属京兆府}
朝恩喜於得地殊不以載為虞驕横如故|{
	横戶孟翻}
壬辰加河南尹張延賞為東京留守罷河南等道副元帥以其兵屬留守|{
	守式又翻帥所類翻}
延賞嘉貞之子也|{
	張嘉貞開元中為相}
二月戊戍李抱玉徙鎮盩厔|{
	徙屯盩厔以兼統山南}
軍士憤怨大掠鳳翔坊市數日乃定 劉希暹頗覺上意異以告魚朝恩朝恩始疑懼|{
	暹息亷翻朝直遥翻}
然上每見之恩禮益隆朝恩亦以此自安皇甫温至京師元載留之未遣因與温及周皓密謀誅朝恩 |{
	考異曰邠志五年春詔以寒食召郭公豐年令節思與大臣為樂時欲誅朝恩因喻郭公朔方一軍有社稷勞宜以功卒數千人入朝朕因宴賞得以相識正月郭公以組甲三千人入覲魚朝恩請公遊章敬寺公許之丞相元公意其相得使諷邠吏請公無往邠吏自中書馳告郭公曰軍容將不利於公亦告諸將須臾朝恩使至郭公將行士之衷甲請從者三百人願備非常郭公怒曰我大臣也彼非有密旨安敢害我若天子之命爾曹胡為獨與僮僕十數人赴之朝恩候之驚曰何車騎之省公以所聞對且曰恐勞思慮耳軍容撫膺捧手嗚咽雪涕曰非公長者得無疑乎按汾陽家傳子儀五月入朝七月至汾州或是四年正月入朝時事於時未有誅朝恩之謀今不取家傳又曰三月公上言魚朝恩潜結周智光為外應久掌禁兵若不早圖禍將作矣亦不取}
既定計載白上上曰善圖之勿反受禍三月癸酉寒食|{
	荆楚歲時記冬至後一百四日一百五日一百六日斷火謂之寒食初學記曰琴操晋文公與介子綏俱亡文公復國子綏無所得作龍蛇之歌而隐文公求之不肯出乃燔左右木子綏抱木而死文公哀之令人五月五日不得舉火及周舉移書魏武明罰令陸翽業中記並云寒食斷火起於子推琴操所云子綏即推也又云五月五日與今有異皆因流俗所傳按左傳及史記並無子推被焚之事然周書司烜氏仲春以木鐸狥火禁於國中注云為季春將出火也今寒食凖節氣是仲春之末清明是三月之初然則禁火並周制也}
上置酒宴貴近於禁中載守中書省宴罷朝恩將還營上留之議事因責其異圖朝恩自辯語頗悖慢皓與左右擒而縊殺之|{
	載祖亥翻又如字朝直遥翻還從宣翻又音如字悖蒲妹翻又蒲沒翻縊於賜翻又於計翻 考異曰實録是日初詔罷朝恩觀軍容等使更加實封留于禁中朝恩既奉詔知負恩乃自縊又曰載遣腹心京兆尹崔昭等候朝恩出處會寒食宴近臣朝恩入謁有詔留之朝恩乃懼言頗悖戾上以舊恩矜貸不加嚴刑朝恩遂自縊新傳曰載用左常侍崔昭尹京兆厚以財結其黨皇甫温周皓按實録去年十一月乙卯孟皥為京兆尹今年三月辛卯為左常侍未嘗言崔昭為京兆也奉詔自縊殆非其實新傳云周皓與左右擒縊之今從之}
外無知者上下詔罷朝恩觀軍容等使内侍監如故詐云朝恩受詔乃自縊以尸還其家賜錢六百萬以葬丁丑加劉希暹王駕鶴御史中丞以慰安北軍之心丙戍赦京畿繫囚命盡釋朝恩黨與且曰北軍將士皆朕爪牙並宜仍舊|{
	使疏吏翻暹息亷翻將即亮翻}
朕今親御禁旅勿有憂懼 己丑罷度支使及關内等道轉運常平鹽鐵使其度支事委宰相領之|{
	度徒洛翻相息亮翻}
勑皇甫温還鎮於陜|{
	既誅魚朝恩不復以温鎮鳳翔陜失冉翻}
元載既誅魚朝恩上寵任益厚載遂志氣驕溢每衆中大言自謂有文武才略古今莫及弄權舞智政以賄成僭侈無度吏部侍郎楊綰典選平允|{
	選須絹翻}
性介直不附載嶺南節度使徐浩貪而佞傾南方珍貨以賂載載以綰為國子祭酒引浩代之浩越州人也載有丈人自宣州來|{
	據顔古古漢書音義大人尊老之稱盖父執也載祖亥翻又如字使疏吏翻}
從載求官載度其人不足任事|{
	度徒洛翻}
但贈河北一書而遣之丈人不悦行至幽州私發書視之書無一言惟署名而已丈人大怒不得已試謁院僚|{
	院僚使院僚属也}
判官聞有載書大驚|{
	判官節度判官}
立白節度使遣大校以箱受書館之上舍|{
	校戶教翻館工唤翻}
留宴數日辭去贈絹千匹其威權動人如此 夏四月庚子湖南兵馬使臧玠殺觀察使崔灌澧州刺史楊子琳起兵討之取賂而還|{
	澧音禮楊子琳自陜州遷澧州還從宣翻又音如字}
涇原節度使馬璘屢訴本鎮荒殘無以贍軍|{
	璘離珍翻}
上諷李抱玉以鄭潁二州讓之乙巳以璘兼鄭頴節度使庚申王縉自太原入朝|{
	縉音晋朝直遥翻}
癸未以左羽林大將軍辛京杲為湖南觀察使 荆南節度使衛伯玉遭母喪六月戊戍以殿中監王昂代之伯玉諷大將楊鉥等拒昂留己|{
	鉥十律翻}
甲寅詔起復伯玉鎮京南如故 秋七月京畿饑|{
	唐以京兆同華商邠岐為京畿}
米斗千錢 劉希暹内常自疑|{
	希暹黨附魚朝恩朝恩死故常自疑暹昔廉翻}
有不遜語王駕鶴以聞九月辛未賜希暹死 吐蕃寇永壽|{
	吐從暾入聲永夀縣属邠州古邠地漢為漆縣唐武德分新平置永夀}
冬十一月郭子儀入朝|{
	郭子儀自邠入朝}
上悉知元載所為以其任政日久欲全始終因獨見深戒之載猶不悛上由是稍惡之|{
	載祖亥翻又音如字見賢遍翻惡烏路翻}
載以李泌有寵於上忌之言泌常與親故宴於北軍與魚朝恩親善宜知其謀上曰北軍泌之故吏也|{
	李泌從肅宗自靈武至鳳翔軍謀大事泌皆預决故言北軍將校皆其故吏泌毘必翻}
故朕使之就見親故朝恩之誅泌亦預謀卿勿以為疑載與其黨攻之不已會江西觀察使魏少遊求参佐上謂泌曰元載不容卿朕今匿卿於魏少遊所|{
	少詩照翻}
俟朕决意除載當有信報卿可束裝來乃以泌為江西判官|{
	江西觀察判官}
且屬少遊使善待之|{
	屬之欲翻}


六年春二月壬寅河西隴右山南西道副元帥兼澤潞山南西道節度使李抱玉上言|{
	帥所類翻使疏吏翻上時兩翻}
凡所掌之兵當自訓練今自河隴達於扶文綿亘二千餘里撫御至難若吐蕃道岷隴俱下|{
	吐從暾入聲言蕃兵入寇分道向岷隴二州而下}
臣保固汧隴則不救梁岷進兵扶文則寇逼關輔首尾不贍進退無從願更擇能臣委以山南使臣得專備隴坻詔許之|{
	汧口監翻坻音底}
郭子儀還邠州|{
	還從宣翻又音如字}
嶺南蠻酋梁崇牽自稱平南十道大都統據容州|{
	酋慈由翻統他綜翻俗音從上聲容州冶普寜縣漢合浦縣地今州西有容山而名}
與西原蠻張侯夏永等連兵攻䧟城邑前容管經略使元結等皆寄治蒼梧|{
	容管領辨白牢欽禺湯瀼巖古等州在桂管西南武德四年分静州之蒼梧豪静置梧州酋慈由翻}
經略使王翃至藤州|{
	翃戶宏翻}
以私財募兵不數月斬賊帥歐陽珪馳詣廣州見節度使李勉請兵以復容州|{
	嶺南節度使治廣州兼統五管故詣之請兵帥所類翻}
勉以為難翃曰大夫如未暇出兵但乞移牒諸州揚言出千兵為援冀藉聲勢亦可成功勉從之翃乃與義州刺史陳仁璀|{
	宋白曰義州即漢蒼梧郡猛陵縣地隋為永熙郡永業縣唐武德四年於此置南義州天寶改為連城郡乾元後為義州璀七罪翻}
藤州刺史李曉庭等結盟討賊|{
	藤州治鐔津縣漢之猛陵縣也}
翃募得三千餘人破賊數萬衆攻容州拔之擒梁崇牽前後大小百餘戰盡復容州故地分命諸將襲西原蠻|{
	新書西原蠻居廣容之南邕桂之西北接道州武岡依阻洞宂綿地數千里將即亮翻}
復鬰林等諸州先是番禺賊帥馮崇道|{
	先昔薦翻番禺漢縣唐帶廣州番山在州東三百步禺山在北一里因以名縣番音潘}
桂州叛將朱濟時皆據險為亂䧟十餘州官軍討之連年不克李勉遣其將李觀與翃併力攻討悉斬之|{
	觀古玩翻}
三月五嶺皆平 河北旱米斗千錢 夏四月己未澧州刺史楊子琳入朝上優接之賜名猷|{
	澧音禮朝直遥翻}
庚申以典内董秀為内常侍|{
	唐百官志太子内坊局令從五品下初内坊隸東宫開元二十七年隸内侍省為局改典内曰令置丞掌坊事及導客内常侍正五品下}
吐蕃請和庚辰遣兼御史大夫吳損使於吐蕃|{
	吐從暾入聲使疎吏翻}
成都司録李少良上書言元載姦隂事|{
	少始照翻上時掌翻載祖亥翻又如字}
上置少良於客省少良以上語告友人韋頌殿中侍御史陸珽以告載載奏之上怒下少良頌珽御史臺獄御史奏少良頌珽凶險比周離間君臣|{
	下遐駕翻少時照翻珽它頂翻比毗至翻間古莧翻}
五月戊申勑付京兆皆杖死 秋七月丙午元載奏凡别勑除文武六品以下官乞令吏部兵部無得檢勘從之時載所奏擬多不遵法度恐為有司所駁故也|{
	載祖亥翻又音如字駁比角翻}
八月丁卯淮西節度使李忠臣將兵二千屯奉天防秋|{
	使疏吏翻將即亮翻又音如字秋高馬肥吐蕃數入寇唐歲調關東之兵屯京西以防之謂之防秋}
上益厭元載所為思得士大夫之不阿附者為腹心漸收載權丙子内出制書以浙西觀察使李栖筠為御史大夫宰相不知載由是稍絀|{
	相息亮翻絀勑律翻}
九月吐蕃下青石嶺軍於那城|{
	靑石嶺在原州西那城即漢朝那故城在原州花石川吐從暾入聲}
郭子儀使人諭之明日引退 是歲以尚書右丞韓滉為戶部侍郎判度支自兵興以來所在賦歛無度|{
	尚辰羊翻度徒洛翻歛力贍翻}
倉庫出入無法國用虚耗滉為人亷勤精於簿領作賦歛出入之法|{
	滉呼廣翻}
御下嚴急吏不敢欺亦值連歲豐穰邊境無寇自是倉庫蓄積始充滉休之子也|{
	韓休開元中為相有直聲而滉以彊幹聞}


七年春正月甲辰回紇使者擅出鴻臚寺|{
	唐鴻臚寺在朱雀街西第二街北來第一坊又北即西内宫城紇下沒翻臚陵奴翻}
掠人子女所司禁之敺擊所司以三百騎犯金光朱雀門|{
	騎奇計翻金光門長安城西面中門朱雀門宫城南門也}
是日宫門皆閉上遣中使劉清潭諭之乃止|{
	使疏吏翻}
三月郭子儀入朝丙午還邠州|{
	朝直遥翻還從宣翻又音如字邠卑旻翻}
夏四月吐蕃五千騎至靈州尋退|{
	吐從暾入聲騎奇計翻}
五月乙未赦天下 秋七月癸巳回紇又擅出鴻臚寺|{
	紇下沒翻臚陵如翻}
逐長安令邵說至含光門街|{
	西内宫城之外為皇城南面三門西為金光門說讀曰悦}
奪其馬說乘它馬而去弗敢争 盧龍節度使朱希彩既得位悖慢朝廷殘虐將卒孔目官李懷瑗因衆怒伺間殺之|{
	使疏吏翻悖蒲妹翻又蒲沒翻朝直遥翻將即亮翻伺相吏翻間古莧翻朱希彩殺李懷仙自立事見上卷三年瑗于絹翻}
衆未知所從經略副使朱泚營於城北其弟滔將牙内兵潜使百餘人於衆中大言曰節度使非朱副使不可衆皆從之泚遂權知留後遣使言狀|{
	泚且禮翻又音此將即亮翻}
冬十月辛未以泚為檢校左常侍幽州盧龍節度使|{
	左常侍左散騎常侍也}
十二月辛未置永平軍於滑州

八年春正月昭義節度使相州刺史薛嵩薨子平年十二將士脅以為帥平偽許之既而讓其叔父㟧|{
	相息亮翻帥所類翻㟧五各翻}
夜奉父喪逃歸鄉里壬午制以㟧知留後 二月壬申永平節度使令狐彰薨彰承滑亳離亂之後治軍勸農府廪充實|{
	治直之翻}
時藩鎮率皆跋扈獨彰貢賦未嘗闕歲遣兵三千詣京西防秋自齎糧食道路供饋皆不受所過秋毫不犯疾亟召掌書記高陽齊映|{
	高陽漢縣属涿郡唐属瀛州}
與謀後事映勸彰請代人遣子歸私第彰從之遺表稱昔魚朝恩破史朝義欲掠滑州臣不聽由是有隙及朝恩誅值臣寢疾以是未得入朝生死愧負臣今必不起倉庫畜牧先已封籍軍中將士州縣官吏按堵待命|{
	將即亮翻朝直遥翻}
伏見吏部尚書劉晏工部尚書李勉可委大事願速以代臣臣男建等今勒歸東都私第彰薨將士欲立建建誓死不從舉家西歸三月丙子以李勉為永平節度使 吏部侍郎徐浩薛邕皆元載王縉之黨浩妾弟侯莫陳怤為美原尉|{
	載祖亥翻又如字縉音晋怤芳俱翻咸亨二年分富平華原及同州之蒲城以故士門縣置美原縣}
浩屬京兆尹杜濟虚以知驛奏優|{
	奏優者言郵驛往來供給車馬薪芻粮用皆無闕乏優於餘縣也屬之欲翻}
又屬邕擬長安尉怤参臺御史大夫李栖筠劾奏其狀|{
	筠俞輪翻劾戶盖翻又戶得翻}
勑禮部侍郎萬年于劭等按之劭奏邕罪在赦前應原除上怒夏五月乙酉貶浩明州别駕邕歙州刺史丙戌貶濟杭州刺史邵桂州長史朝廷稍肅|{
	明州京師東南四千一百里歙州京師東南三千六百六十七里杭州京師東南三千五百五十六里桂州京師水陸路四千七百六十里歙音攝杭戶剛翻長知兩翻 考異曰實録云侯莫陳怤為美原尉舊李栖筠傳云華原尉侯莫陳怤以主郵傳優改長安尉又云栖筠劾奏皓等上依違未决屬月蝕上問其故對曰臣聞日蝕修德月蝕修刑今誣上行私之罪未理此天之所以儆戒於明聖由是感寤坐怤者皆貶謫自此朝綱益振百度肅然按己丑月乃食於未時也今不取}
辛卯鄭王邈薨贈昭靖太子|{
	邈上次子也}
回紇自乾元以來歲求和市每一馬易四十縑動至

數萬匹馬皆駑瘠無用朝廷苦之所市多不能盡其數回紇待遣繼至者常不絶於鴻臚至是上欲悦其意命盡市之秋七月辛丑回紇辭歸載賜遺及馬價共用車千餘乘|{
	遺惟季翻乘承正翻}
八月己未吐蕃六萬騎寇靈武踐秋稼而去 |{
	考異曰汾陽家傳八月吐蕃五千騎至靈州南士級渠公遣温儒雅後政等連兵救之九月大破之今從實録}
辛未幽州節度使朱泚遣弟滔將五千精騎詣涇州防秋自安禄山反幽州兵未嘗為用滔至上大喜勞賜甚厚|{
	勞力到翻}
壬申回紇復遣使者赤心以馬萬匹來求互市|{
	復扶又翻}
九月壬午循州刺史哥舒晃殺嶺南節度使呂崇賁據嶺南反 癸未晉州男子郇模|{
	郇須倫翻古國名後世以國為姓}
以麻辮髮持竹筐葦席哭於東市人問其故對曰願獻三十字一字為一事若言無所取請以席裹尸貯筐中棄於野京兆以聞上召見賜新衣館於客省|{
	時於右銀臺門置客省或四方奏計未遣者上書言事忤旨者及蕃客未報者皆館於其中常數百人館古玩翻}
其言團者請罷諸州團練使也監者請罷諸道監軍使也|{
	監古銜翻}
魏博節度使田承嗣為安史父子立祠堂謂之四聖|{
	為于偽翻}
且求為相上令内侍孫知古因奉使諷令毁之冬十月甲辰加承嗣同平章事以褒之 靈州破吐蕃萬餘衆吐蕃衆十萬寇涇邠郭子儀遣朔方兵馬使渾瑊將步騎五千拒之庚申戰于宜禄|{
	宋白曰宜禄本漢鶉觚縣地後魏熙平二年分鶉觚縣置東隂盤縣廢帝元年以縣南臨宜禄川改為宜禄縣九域志宜禄在邠州西六十里 考異曰實録作甲子盖奏到之日也邠志云十八日與唐歷合今從之}
瑊登黄萯原|{
	萯音倍黄萯草名盖地多黄萯草因以名原}
望虜|{
	句斷}
命據險布拒馬以備其馳突宿將史抗溫儒雅等意輕瑊不用其命瑊召使擊虜則己醉矣見拒馬曰野戰烏用此為命撤之叱騎兵衝虜陳不能入而返|{
	陳讀曰陣}
虜躡而乘之官軍大敗士卒死者什七八居民為吐蕃所掠千餘人甲子馬璘與吐蕃戰於鹽倉又敗|{
	鹽倉在涇州城西 考異曰邠志曰十月西戎寇汾涇原節度使馬公襲之郭公使其將渾瑊率步騎五千為之犄角十八日師登黄萯原望見吐蕃瑊急引其衆前據乘險仍設拒馬槍以遏馳突之勢史抗溫儒雅等宿將五六人任氣自負輕侮都將置酒高飲使人召之至則皆醉矣見拒馬槍曰野地見賊須擊設此何為命去之戎衆既陳抗等叱馬軍使馳賊既回自衝其軍吐蕃躡背而入我師大敗卒之不死者什二三汾陽家傳十月吐蕃四節度歷涇川過閣川南與渭河合軍公遣渾瑊等前後相接以待之二十四日大戰於長武城我師敗績瑊等突出乃免唐歷十八日吐蕃寇邠州瑊與戰於宜禄官軍大敗二十二日馬璘出兵擊之又敗二十七日己巳璘遣軍斫吐蕃營破之二十八日庚午詔追諸道兵屯西郊十一月一日吐蕃退段公别傳曰八年冬十月二十三日犬戎入寇大戰於鹽倉我軍與朔方兵馬使渾瑊之衆併力齊攻防秋諸軍望賊而退於是我師不利今日從邠志唐歷段公家傳事從實録舊傳兼採諸書}
璘為虜所隔逮暮未還涇原兵馬使焦令諶等與敗卒争門而入或勸行軍司馬段秀實乘城拒守秀實曰大帥未知所在當前擊虜豈得苟自全乎召令諶等讓之曰軍法失大將麾下皆死諸君忘其死邪令諶等惶懼拜請命|{
	帥所頰翻將即亮翻}
秀實乃發城中兵未戰者悉出陳於東原|{
	陳讀曰陣}
且收散兵為將力戰狀吐蕃畏之稍却既夜璘乃得還郭子儀召諸將謀曰敗軍之罪在我不在諸將然朔方兵精聞天下|{
	聞音問}
今為虜敗何策可以雪恥|{
	敗補賣翻}
莫對渾瑊曰敗軍之將不當復預議|{
	復扶又翻下同}
然願一言今日之事惟理瑊罪|{
	理治也}
不則再見任|{
	不讀曰否}
子儀赦其罪使將兵趣朝那虜既破官軍欲掠汧隴鹽州刺史李國臣曰虜乘勝必犯郊畿我掎其後虜必返顧乃引兵趣秦原|{
	括地志曰秦州清水縣有秦亭秦谷非子所封地也趣七喻翻}
鳴鼔而西虜聞之至百城返|{
	百城即涇州靈臺縣之百里城}
渾瑊邀之於隘盡復得其所掠馬璘亦出精兵襲虜輜重於潘原|{
	重直用翻}
殺數千人虜遂遁去 乙丑以江西觀察使路嗣恭討哥舒翰 初元載嘗為西州刺史知河西隴右山川形勢是時吐蕃數為寇|{
	數所角翻}
載言於上曰四鎮北庭既治涇州無險要可守隴山高峻南連秦嶺|{
	秦嶺即商嶺}
北抵大河今國家西境盡潘原而吐蕃戍摧沙堡原州居其中間當隴山之口其西皆監牧故地草肥水美平凉在其東|{
	唐原州治古高平當隴道之要漢光武取隴右先降高峻而後可以蹙隗囂赫連勃勃據高平乘間以窺隴東嶺北得以病姚興誠要害之地也平凉縣属原州西南即隴山之六盤關}
獨耕一縣可給軍食故壘尚存吐蕃弃而不居每歲盛夏吐蕃畜牧青海去塞甚遠若乘間築之|{
	間古莧翻}
二旬可畢移京西軍戌原州移郭子儀軍戍涇州為之根本分兵守石門木峽|{
	原州西南有木峽關州境又有石門關}
漸開隴右進逹安西據吐蕃腹心則朝廷可高枕矣并圖地形獻之密遣人出隴山商度功用|{
	枕職任翻度徒洛翻}
會汴宋節度使田神功入朝上問之對曰行軍料敵宿將所難陛下奈何用一書生語欲舉國從之乎載尋得罪事遂寢|{
	為後楊炎復議城原州張本}
有司以回紇赤心馬多請市千匹郭子儀以為如此逆其意太甚自請輸一歲俸為國市之|{
	為于偽翻}
上不許十一月戊子命市六千匹

資治通鑑卷二百二十四
