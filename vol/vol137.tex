\chapter{資治通鑑卷一百三十七}
宋 司馬光 撰

胡三省 音註

齊紀三|{
	起上章敦牂盡玄黓涒灘凡三年}


世祖武皇帝中

永明八年春正月詔放隔城俘二千餘人還魏|{
	拔隔城見上卷上年}
乙丑魏主如方山二月辛未如靈泉|{
	泉下當有池字}
壬申還宫 地豆干頻寇魏邊|{
	北史曰地豆干國在室韋之西千餘里}
夏四月甲戌魏征西大將軍陽平王頤擊走之頤新城之子也|{
	新城當作新成見一百二十八卷宋孝武大明元年考異曰陽平王頤帝紀作熙又作賾今從本傳}
甲午魏遣兼員外散騎常侍邢產等來聘|{
	散悉亶翻騎奇寄翻}
五月己酉庫莫奚寇魏邊|{
	隋書庫莫奚東部胡之種為慕容氏所破遺落者竄匿松漠之間其俗甚為不潔而善射獵好寇鈔後單稱為奚魏高宗皇興二年置安州治方城領密雲廣陽安樂等郡}
安州都將樓龍兒擊走之|{
	將即亮翻}
秋七月辛丑以會稽太守安陸侯緬為雍州刺史緬鸞之弟也緬留心獄訟得刼皆赦遣許以自新再犯乃加誅|{
	刼謂刼盜也會工外翻緬彌兖翻雍於用翻}
民畏而愛之 癸卯大赦 丙午魏主如方山丙辰遂如靈泉池八月丙寅朔還宫 河南王度易侯卒乙酉以其世子伏連籌為秦河二州刺史 |{
	考異曰齊書作世子休留成今從魏書}
遣振武將軍丘冠先拜授且弔之伏連籌逼冠先使拜冠先不從伏連籌推冠先墜崖而死|{
	冠古玩翻推吐雷翻}
上厚賜其子雄敕以喪委絶域不可復尋|{
	復扶又翻}
仕進無嫌 荆州刺使巴東王子響有勇力善騎射好武事自選帶仗左右六十人皆有膽幹|{
	騎奇寄翻好呼到翻帶仗左右使之帶器仗而衛左右因名}
至鎮數於内齋以牛酒犒之|{
	數所角翻犒苦到翻}
又私作錦袍絳襖欲以餉蠻交易器仗|{
	襖烏浩翻}
長史高平劉寅司馬安定席恭穆連名密啓上敕精檢|{
	言精加檢校也}
子響聞臺使至不見敇|{
	使疏吏翻}
召寅恭穆及諮議參軍江悆|{
	悆羊茹翻}
典籖吳修之魏景淵等詰之寅等祕而不言修之曰既已降敇政應方便答塞景淵曰應先檢校|{
	修之言方便答塞欲為子響道地也景淵言應先檢校欲依敕行之也塞悉則翻}
子響大怒執寅等八人於後堂殺之具以啓聞上欲赦江悆聞皆已死怒壬辰以隨王子隆為荆州刺史上欲遣淮南太守戴僧静將兵討子響|{
	將即亮翻}
僧静面啓曰巴東王年少長史執之太急忿不思難故耳|{
	少詩沼翻難乃旦翻}
天子兒過誤殺人有何大罪官忽遣軍西上|{
	上時掌翻}
人情惶懼無所不至僧静不敢奉敇上不答而心善之|{
	不答而心善其言蓋天性所在而未敢撓國法也}
乃遣衛尉胡諧之游擊將軍尹略中書舍人茹法亮帥齋仗數百人詣江陵檢捕羣小|{
	齋仗天子齋内精㐲手也茹音如帥讀曰率}
敇之曰子響若束手自歸可全其命以平南内史張欣泰為諧之副|{
	按齊書張欣泰傳時為南平内史當作南平}
欣泰謂諧之曰今段之行勝既無名負成奇恥彼凶狡相聚所以為其用者或利賞逼威無由自潰若頓軍夏口宣示禍福可不戰而擒也|{
	夏戶雅翻}
諧之不從欣泰興世之子也|{
	張興世見一百三十一卷宋明帝泰始二年}
諧之等至江津築城燕尾洲|{
	燕尾洲在江津戍西江水至此北合靈溪水}
子響白服登城頻遣使與相聞曰天下豈有兒反身不作賊直是麤踈今便單舸還闕受殺人之罪|{
	使疏吏翻舸古我翻}
何築城見捉邪尹略獨答曰誰將汝反父人共語|{
	將引也}
子響唯灑泣|{
	灑泣揮涙也}
乃殺牛具酒饌餉臺軍|{
	饌雛戀翻又雛晥翻}
略棄之江流子響呼茹法亮法亮疑畏不肯往又求見傳詔法亮亦不遣且執錄其使|{
	録收也使疏吏翻}
子響怒遣所養勇士收集州府兵二千人從靈溪西渡子響自與百餘人操萬鈞弩宿江隄上明日府州兵與臺軍戰子響於隄上發弩射之臺軍大敗尹略死諧之等單艇逃去|{
	操七刀翻射而亦翻艇待鼎翻小船也}
上又遣丹陽尹蕭順之將兵繼至|{
	將即亮翻}
子響即日將白衣左右三十人乘舴艋沿流赴建康|{
	舴艋亦小船也舴陟格翻艋莫幸翻}
太子長懋素忌子響順之之建康也太子密諭順之使早為之所勿令得還子響見順之欲自申明順之不許於射堂縊殺之|{
	縊於賜翻又於計翻 考異曰齊書曰子響部下恐懼各逃散子響乃白服出降詔賜死蓋蕭子顯為順之諱耳今從南史 按順之梁武帝之父蕭子顯者仕梁朝而作齊書故通鑑言其為順之諱}
子響臨死啓上曰臣罪踰山海分甘斧鉞|{
	分扶問翻}
敇遣諧之等至竟無宣旨便建旗入津對城南㟁築城守臣累遣書信呼法亮乞白服相見法亮終不肯羣小怖懼|{
	怖普布翻}
遂致攻戰此臣之罪也臣此月二十五日束身投軍希還天闕停宅一月|{
	希望也宅謂建康諸王宅也}
臣自取盡可使齊代無殺子之譏臣免逆父之謗既不遂心今便命盡臨啓哽塞知復何陳|{
	塞悉則翻復扶又翻}
有司奏絶子響屬籍|{
	屬籍宗屬之籍也今謂之玉牒}
削爵土易姓蛸氏|{
	蛸相邀翻與蕭音相近}
諸所連坐别下考論|{
	謂子響之黨當連坐者别行下考覈論定其罪也下戶嫁翻}
久之上遊華林園見一猨透擲悲鳴問左右|{
	句斷}
曰猨子前日墜崕死上思子響因嗚咽流涕茹法亮頗為上所責怒蕭順之慙懼發疾而卒豫章王嶷表請收葬子響不許|{
	子響先嘗出繼嶷故以舊恩請收葬}
貶為魚復侯|{
	魚復縣時屬巴東郡應劭曰復音腹}
子響之亂方鎮皆啓子響為逆兖州刺史垣榮祖曰此非所宜言正應云劉寅等孤負恩奬逼迫巴東使至於此上省之以榮祖為知言|{
	省悉景翻}
臺軍焚燒江陵府舍官曹文書一時蕩盡上以大司馬記室南陽樂藹屢為本州僚佐引見問以西事|{
	見賢遍翻}
藹應對詳敏上悅用為荆州治中敕付以修復府州事藹繕修廨舍數百區|{
	廨古隘翻}
頃之咸畢而役不及民荆部稱之 九月癸丑魏太皇太后馮氏殂高祖勺飲不入口者五日|{
	勺音酌挹抒之器也周禮考工記梓人為飲器勺一升}
哀毀過禮中部曹華隂楊椿諫曰|{
	據北史楊椿傳時為中部法曹華戶化翻}
陛下荷祖宗之業|{
	荷下可翻}
臨萬國之重豈可同匹夫之節以取僵仆|{
	僵居良翻}
羣下惶灼莫知所言|{
	惶恐也遽也灼熱也}
且聖人之禮毀不滅性|{
	孝經曰三日而食教民無以死傷生毁不滅性此聖人之政也}
縱陛下欲自賢於萬代|{
	楊椿此語說出魏孝文心事}
其若宗廟何帝感其言為之一進粥|{
	為于偽翻}
於是諸王公皆詣闕上表請時定兆域|{
	兆域謂葬地從先帝之兆}
及依漢魏故事并太皇太后終制既葬公除|{
	公除者以天下為公而除服也}
詔曰自遭禍罰慌惚如昨|{
	慌乎往翻惚音忽鄭玄曰慌惚思念益深之時也}
奉侍梓宫猶希髣髴|{
	事死如事生猶冀髣髴見之也}
山陵遷厝所未忍聞冬十月王公復上表固請|{
	復抉又翻}
詔曰山陵可依典册衰服之宜情所未忍|{
	謂未忍公除也衰讀與縗同倉回翻}
帝欲親至陵所戊辰詔諸常從之具悉可停之|{
	從才用翻}
其武衛之官防侍如法|{
	法常法也不撤武衛備不虞也}
癸酉葬文明太皇太后于永固陵|{
	陵在方山不從金陵之兆}
甲戌帝謁陵王公固請公除詔曰比當别叙在心|{
	比並也並當别叙在心之所欲言比毗至翻}
己卯又謁陵庚辰帝出至思賢門右|{
	據魏紀太和元年起朱明思賢門蓋平城宫之南門也}
與羣臣相慰勞|{
	勞力到翻}
太尉丕等進言曰臣等以老朽之年歷奉累聖國家舊事頗所知聞伏惟遠祖有大諱之日唯侍從梓宫者凶服|{
	從才用翻}
左右盡皆從吉四祖三宗因而無改|{
	四祖者高祖昭成帝太祖道武帝世祖太武帝顯祖獻文帝三宗者太宗明元帝恭宗景穆帝高宗文成帝}
陛下以至孝之性哀毁過禮伏聞所御三食不滿半溢|{
	禮喪大記曰君之喪子食粥朝一溢米暮一溢米食之無筭注云二十兩為一溢於粟米之法為米一升二十四分升之一孔頴逹曰案律歷志黄鍾之律其實一籥律歷志合籥為合則二十四銖合重兩十合為一升升重十兩二十兩則米二升與此不同者但古秤有二法說左傳百二十斤為石則一斗十二斤為兩則一百九十二兩則一升為十九兩有奇今一兩為二十四銖則二十兩為四百八十銖計十九兩有奇為一升則總有四百六十銖八參以成四百八十銖唯有十九銖二參在是為米一升二十四分升之一此大略而言之陳言曰以紹興一升得漢五升}
晝夜不釋絰帶|{
	喪服麻在首腰皆曰絰首絰象緇布冠腰絰象大帶絰之言實也衰之言摧也衰絰明中實摧痛也}
臣等叩心絶氣坐不安席願少抑至慕之情奉行先朝舊典|{
	朝直遥翻}
帝曰哀毁常事豈足關言朝夕食粥粗可支任|{
	粗坐五翻任音壬勝也堪也}
諸公何足憂怖|{
	怖普布翻}
祖宗情專武略未修文教朕今仰禀聖訓庶習古道論時比事又與先世不同太尉等國老政之所寄於典記舊式或所未悉|{
	典記謂經典傳記也}
且可知朕大意其餘古今喪禮朕且以所懷别問尚書游明根高閭等公可聽之|{
	游明根高閭時以儒鳴故帝别與之言}
帝因謂明根等曰聖人制卒哭之禮授服之變皆奪情以漸|{
	禮親始死哭無時謂朝夕哭之外哀至則哭也既葬而虞既虞而卒哭自此朝夕之間哀至不哭猶朝夕哭三年之喪服斬衰期而小祥既祥而練再朞而大祥既祥而禫又三月而除服卒子恤翻}
今則旬日之間言及即吉特成傷理對曰臣等伏尋金冊遺音|{
	蓋以文明太后遺旨書之金冊也}
踰月而葬葬而即吉故於下葬之初奏練除之事帝曰朕惟中代所以不遂三年之喪蓋由君上違世繼主初立君德未流臣義不洽故身襲衮冕行即位之禮朕誠不德在位過紀|{
	宋明帝泰始七年魏孝文受禪至是十九年此言在位過紀蓋以宋蒼梧王元徽四年顯祖方殂踰年改元太和至是十四年故云在位過紀十二年為一紀過古禾翻}
足令億兆知有君矣於此之時而不遂哀慕之心使情禮俱失深可痛恨高閭曰杜預晉之碩學論自古天子無有行三年之喪者以為漢文之制闇與古合雖叔世所行事可承踵是以臣等慺慺干請|{
	慺洛侯翻慺慺敬謹貌}
帝曰竊尋金冊之旨所以奪臣子之心令早即吉者慮廢絶政事故也羣公所請其志亦然朕今仰奉冊令俯順羣心不敢闇默不言以荒庶政|{
	闇音隂}
唯欲衰麻廢吉禮|{
	衰吐回翻下衰絰除衰從衰同}
朔朢盡哀誠情在可許故專欲行之如杜預之論於孺慕之君諒闇之主蓋亦誣矣|{
	孺慕如孺子之慕父母也}
秘書丞李彪曰|{
	曹操為魏王置秘書令丞}
漢明德馬后保養章帝母子之道無可閒然|{
	閒古莧翻}
及后之崩葬不淹旬尋已從吉|{
	漢章帝建初四年六月癸丑明德皇后崩七月壬戌葬史不書公除之日此言葬不淹旬尋已從吉以漢文三十六日釋服之制推之也}
然漢章不受譏明德不損名願陛下遵金冊遺令割哀從議帝曰朕所以眷戀衰絰不從所議者實情不能忍豈徒苟免嗤嫌而已哉|{
	衰倉回翻嗤充之翻}
今奉終儉素一已仰遵遺冊但痛慕之心事繫於予庶聖靈不奪至願耳高閭曰陛下既不除服於上臣等獨除服於下則為臣之道不足又親䘖衰麻復聽朝政|{
	復扶又翻朝直遥翻}
吉凶事雜臣竊為疑帝曰先后撫念羣下卿等哀慕猶不忍除奈何令朕獨忍之於至親乎今朕逼於遺冊唯望至朞雖不盡禮藴結差申羣臣各以親疎貴賤遠近為除服之差庶幾稍近於古易行於今高閭曰昔王孫裸葬士安去棺其子皆從而不違|{
	近其靳翻易以䜴翻去羌呂翻漢武帝時楊王孫家累千金厚自奉養生無所不致及病且終先令其子曰吾欲臝葬以反吾真必無易吾志則為布囊盛尸入地七尺既下從足引脱其囊以身親土其子不忍往問其友人祁侯祁侯與之辨難往復而王孫終守其說祁侯曰善遂臝葬晉人皇甫謐字士安著論曰生不能保七尺之軀死何故隔一棺之土然則衣衾所以穢身棺槨所以隔真吾氣絶之後便即時服幅巾故衣以籧篨裹尸擇不毛之土穿阮下尸籧篨之外便以親土若不如此則寃悲没世其子從之}
今親奉遺令而有所不從臣等所以頻煩干奏李彪曰三年不改其父之道可謂大孝|{
	引論語孔子之言}
今不遵冊令恐涉改道之嫌帝曰王孫士安皆誨子以儉及其遵也豈異今日改父之道殆與此殊縱有所涉甘受後代之譏未忍今日之請羣臣又言春秋烝嘗事難廢闕|{
	禮曰喪三年不祭言帝若行三年之喪則宗廟之祭將至廢闕也}
帝曰自先朝以來恒有司行事|{
	朝直遥翻恒戶登翻}
朕賴蒙慈訓常親致敬今昊天降罸人神喪恃|{
	詩曰無母何恃喪息浪翻}
賴宗廟之靈亦輟歆祀|{
	賴蜀本作想當從之否則賴字衍歆尹今翻}
脫行饗薦恐乖冥旨羣臣又言古者葬而即吉不必終禮此乃二漢所以經綸治道魏晉所以綱理庶政也|{
	治直吏翻下同}
帝曰既葬即吉蓋季俗多亂權宜救世耳二漢之盛魏晉之興豈由簡畧喪禮遺忘仁孝哉平日之時公卿每稱當今四海晏然禮樂日新可以參美唐虞比盛夏商|{
	夏戶雅翻}
及至今日即欲苦奪朕志使不踰於魏晉如此之意未解所由|{
	解戶買翻曉也}
李彪曰今雖治化清晏然江南有未賓之吳漠北有不臣之虜是以臣等猶懷不虞之慮|{
	虞防也}
帝曰魯公帶絰從戎|{
	據史記武王崩成王幼管蔡反淮夷徐戎亦並興魯公伯禽征之時有武王之喪故帶絰從戎也}
晉侯墨衰敗敵|{
	春秋時晉文公卒未葬襄公墨衰絰以敗秦師于殽衰倉回翻敗補邁翻}
固聖賢所許如有不虞雖越紼無嫌|{
	鄭玄曰越猶躐也紼輴車索孔穎逹曰未葬之前屬紼於輴以備火災今既祭天地社稷須越躐此紼而往祭故云越紼紼音弗輴勑倫翻索悉各翻}
而况衰麻乎豈可於晏安之辰豫念軍旅之事以廢喪紀哉古人亦有稱王者除衰而諒闇終喪者|{
	闇音隂}
若不許朕衰服則當除衰拱默委政冢宰二事之中唯公卿所擇游明根曰淵默不言則大政將曠仰順聖心請從衰服太尉丕曰臣與尉元歷事五帝|{
	明元太武文成獻文并孝文為五帝尉紆勿翻}
魏家故事尤諱之後三月|{
	尤諱猶云大諱也尤甚也死者人之所甚諱也}
必迎神於西禳惡於北具行吉禮|{
	此魏初所用夷禮也禳如羊翻}
自皇始以來未之或改|{
	皇始道武帝年號}
帝曰若能以道事神不迎自至苟失仁義雖迎不來此乃平日所不當行|{
	言不當用夷禮}
况居喪乎朕在不言之地|{
	謂居喪諒隂三年不言也}
不應如此喋喋|{
	喋徒協翻喋喋多言也便語也}
但公卿執奪朕情遂成往復追用悲絶遂號慟羣官亦哭而辭出|{
	號戶高翻}
初太后忌帝英敏恐不利於己欲廢之盛寒閉於空室絶其食三日召咸陽王禧將立之太尉東陽王丕尚書右僕射穆泰尚書李冲固諫乃止帝初無憾意唯深德丕等泰崇之玄孫也|{
	穆崇魏開國功臣}
又有宦者譖帝於太后太后杖帝數十帝默然受之不自申理及太后殂亦不復追問|{
	不復追問譖者為誰復扶又翻}
甲申魏主謁永固陵辛卯詔曰羣官以萬機事重屢求聽政但哀慕纒綿未堪自力近侍先掌機衡者皆謀猷所寄且可委之如有疑事當時與論决 交州刺史清河房法乘專好讀書常屬疾不治事|{
	好呼到翻屬之欲翻屬托也屬疾猶言託疾也治直之翻}
由是長史伏登之得擅權改易將吏不令法乘知|{
	將直亮翻下同}
錄事房季文白之法乘大怒繫登之於獄十餘日登之厚賂法乘妹夫崔景叔得出因將部曲襲州|{
	襲州治也}
執法乘謂之曰使君既有疾不宜煩勞囚之别室法乘無事復就登之求書讀之|{
	復扶又翻}
登之曰使君静處|{
	處昌呂翻}
猶恐動疾豈可看書遂不與乃啓法乘心疾動不任視事|{
	任音壬}
十一月乙卯以登之為交州刺史法乘還至嶺而卒|{
	嶺即大庾嶺也史言徒讀書而無政事者不足以當方任}
十二月己卯立皇子子建為湘東王初太祖以南方錢少更欲鑄錢建元末奉朝請孔顗上言|{
	少詩沼翻朝直遥翻顗魚豈翻 考異曰齊紀作孔覬今從齊書南史}
以為食貨相通理勢自然李悝云糴甚貴傷民甚賤傷農甚賤甚貴其傷一也|{
	李悝魏文侯之師韋昭曰民謂士工商悝苦回翻}
三吳國之關奥比歲時被水而糴不貴|{
	比毗至翻被皮義翻}
是天下錢少非穀賤此不可不察也鑄錢之弊在輕重屢變重錢患難用而難用為累輕|{
	累力瑞翻}
輕錢弊盜鑄而盜鑄為禍深民所以盜鑄嚴法不能禁者由上鑄錢惜銅愛工也惜銅愛工者意謂錢為無用之器以通交易務欲令質輕而數多使省工而易成|{
	易以䜴翻}
不詳慮其為患也夫民之趨利如水走下|{
	用漢晁錯之言趨讀曰趣走音奏}
今開其利端從以重刑是導其為非而陷之於死豈為政歟漢興鑄輕錢民巧偽者多至元狩中始懲其弊乃鑄五銖錢周郭其上下令不可磨取鋊|{
	漢初行半兩錢及莢錢一面有文一面漫民盜磨其漫面取其鋊以更鑄作錢元狩鑄五銖文漫兩面皆周匝為郭令不得磨取鋊鋊音谷銅屑也}
而計其費不能相償私鑄益少此不惜銅不愛工之效也|{
	少詩沼翻}
王者不患無銅乏工每令民不能競則盜鑄絶矣宋文帝鑄四銖至景和錢益輕雖有周郭而鎔冶不精於是盜鑄紛紜而起不可復禁此惜銅愛工之驗也|{
	復扶又翻}
凡鑄錢與其不衷寜重無輕|{
	不衷者不得輕重之中也}
自漢鑄五銖至宋文帝歷五百餘年制度世有廢興而不變五銖者明其輕重可法得貨之宜故也案今錢文率皆五銖異錢時有耳|{
	異錢謂其文非五銖者}
自文帝鑄四銖又不禁民翦鑿為禍既博鍾弊于今豈不悲哉|{
	鍾聚也}
晉氏不鑄錢後經寇戎水火耗散沈鑠|{
	沈持林翻鑠書藥翻}
所失歲多譬猶磨礲砥礪不見其損有時而盡|{
	引漢枚乘之言}
天下錢何得不竭錢竭則士農工商皆喪其業|{
	喪息浪翻}
民何以自存愚以為宜如舊制大興鎔鑄錢重五銖一依漢法若官鑄者已布於民便嚴斷剪鑿|{
	斷音短禁截也}
輕小破缺無周郭者悉不得行官錢細小者稱合銖兩|{
	稱尺證翻合音閤合少為多也}
銷以為大利貧良之民塞姦巧之路錢貨既均遠近若一百姓樂業市道無爭衣食滋殖矣|{
	塞悉則翻樂音洛}
太祖然之使諸州郡大市銅炭會晏駕事寢是歲益州行事劉悛上言|{
	悛士倫翻又丑緣翻}
蒙山下有嚴道銅山舊鑄錢處可以經略|{
	蒙山在今雅州嚴道縣南十里此即漢鄧通鑄錢舊處}
上從之遣使入蜀鑄錢|{
	使疏吏翻}
頃之以功費多而止 自太祖治黄籍至上謫巧者戍緣淮各十年百姓怨望|{
	事見上卷四年至上謂至武帝時治直之翻下同}
乃下詔自宋昇明以前皆聽復注|{
	聽復注籍也}
其有謫役邊疆各許還本此後有犯嚴加翦治長沙威王晃卒|{
	諡法勇以果毅曰威}
吏部尚書王晏陳疾自解上欲以西昌侯鸞代晏領選手敕問之晏啓曰鸞清幹有餘然不諳百氏|{
	百氏百家氏族也自魏晉以來率以門地用人選須絹翻諳烏含翻}
恐不可居此職上乃止 以百濟王牟大為鎮東大將軍百濟王 高車阿伏至羅及窮奇遣使如魏請為天子討除蠕蠕|{
	使疏吏翻為于偽翻蠕人兖翻}
魏主賜以繡袴褶及雜絲百匹|{
	褶音習}


九年春正月辛丑上祀南郊 丁卯魏主始聽政於皇信東室|{
	自居馮太后之喪至是始聽政皇信東室蓋皇信堂之東室也}
詔太廟四時之祭薦宣皇帝起麪餅鴨|{
	起麪餅今北人能為之其餅浮軟以卷肉噉之亦謂之卷餅程大昌曰起麪餅入教麪中令鬆鬆然也教俗書作酵麪莫甸翻孟詵曰音郝肉羮也}
孝皇后筍鴨卵高皇帝肉膾葅羮昭皇后茗粣炙魚|{
	茗茶也本草曰茗苦茶郭璞曰早採者為茶晩採者為茗粣類篇云色責翻糝也又側革翻粽也南史虞悰作扁米粣蓋即今之饊子是也可以供茶炙之石翻}
皆所嗜也上夢太祖謂已宋氏諸帝常在太廟從我求食可别為吾致祠|{
	為于偽翻}
乃命豫章王妃庾氏四時祠二帝二后於清溪故宅|{
	杜佑曰蕭齊之世有清溪宫後改為華林苑據卞彬傳清溪在臺城東宫又在清溪之東建康志曰吳大帝鑿通城北塹以洩玄武湖水源於鍾山接於秦淮謂之清溪}
牲牢服章皆用家人禮

臣光曰昔屈到嗜芰屈建去之以為不可以私欲干國之典|{
	屈九勿翻芰奇寄翻菱也去羌呂翻屈建屈到子也國語屈到嗜芰有疾召宗老而屬之曰祭我必以芰及祥宗老將薦芰屈建命去之曰祭典有之國君有牛享大夫有羊饋士有豚犬之奠庶人有魚炙之薦籩豆脯醢則上下共之不羞珍異不陳庶侈夫子不以其私欲干國之典遂不用}
况子為天子而以庶人之禮祭其父違禮甚矣衛成公欲祀相甯武子猶非之|{
	左傳僖三十一年狄圍衛衛遷于帝丘衛成公夢康叔謂己曰相奪予享公命祀相甯武子不可曰鬼神非其族類不歆其祀杞鄫何事相之不享於此久矣非衛之罪也不可以間成王周公之命祀請改祀命相息亮翻}
而况降祀祖考於私室使庶婦尸之乎|{
	豫章王嶷與帝同母帝為嫡故通鑑以嶷妃為庶婦尸主也}


初魏主召吐谷渾王伏連籌入朝伏連籌辭疾不至輒修洮陽泥和二城置戌兵焉|{
	後周武帝逐吐谷渾置洮陽郡唐洮州及臨潭縣所治即洮陽城也泥和即水經注所謂迷和城洮水逕其南又在洮陽城東宋白曰洮州臨洮郡郡城本名洮陽在洮水之北乃吐谷渾所築南臨洮水極峻險今謂之洪和城吐從暾入聲谷音浴洮音七刀翻}
二月乙亥魏枹罕鎮將長孫百年請擊二戍|{
	魏枹罕鎮將帶河州刺史枹音膚將音即兩翻長音知兩翻}
魏主許之 散騎常侍裴昭明散騎侍郎謝竣如魏弔|{
	散音悉亶翻騎音奇寄翻竣音七倫翻又丑緣翻}
欲以朝服行事|{
	朝音直遥翻下同}
魏主客曰弔有常禮何得以朱衣入凶庭昭明等曰受命本朝不敢輒易往返數四昭明等固執不可魏主命尚書李冲選學識之士與之言冲奏遣著作郎上谷成淹昭明等曰魏朝不聽使者朝服出何典禮淹曰吉凶不相厭|{
	厭於葉翻}
羔裘玄冠不以弔|{
	論語記孔子容止有是言}
此童稚所知也|{
	稚直利翻}
昔季孫如晉求遭喪之禮以行|{
	左傳文六年季文子將聘於晉使求遭喪之禮以行其人曰將焉用之文子曰備豫不虞古之善教也求而無之實難過求何害}
今卿自江南遠來弔魏方問出何典禮行人得失何其遠哉昭明曰二國之禮應相凖望|{
	凖揆平之物又其義擬也倣也對看為望月有弦望後漢律歷志分天之中相與為衡謂之望謂月望日月正相對其平如衡凖望之言義取諸此}
齊高皇帝之喪魏遣李彪來弔初不素服齊朝亦不以為疑|{
	帝即位之初魏遣彪來聘非弔也昭明欲以是抗止淹耳}
何至今日獨見要逼|{
	要讀曰邀}
淹曰齊不能行亮隂之禮踰月即吉彪奉使之日齊之君臣鳴玉盈庭貂璫曜目|{
	漢制侍中常侍之冠加黄金璫貂尾以飾之晉宋以後王公皆冠貂蟬使疏吏翻}
彪不得主人之命敢獨以素服厠其間乎皇帝仁孝侔於有虞執親之喪居廬食粥豈得以此方彼乎昭明曰三王不同禮孰能知其得失淹曰然則虞舜高宗皆非邪昭明竣相顧而笑曰非孝者無親|{
	孝經之言}
何可當也乃曰使人之來唯齎袴褶此既戎服不可以弔|{
	晉志曰袴褶之制未詳所起近世凡車駕親戎中外戒嚴服之服無定色使疏吏翻褶音習}
唯主人裁其弔服然違本朝之命返必獲罪淹曰使彼有君子卿將命得宜且有厚賞若無君子卿出而光國得罪何傷自當有良史書之乃以衣幍給昭明等|{
	幍苦洽翻}
使服以致命己丑引昭明等入見文武皆哭盡哀魏主嘉淹之敏遷侍郎 |{
	考異曰揚松圿談藪作朱淹又云自著作郎遷著佐郎今從魏書}
賜絹百匹昭明駰之子也|{
	裴駰松之之子注史記行于世駰音因}
始興簡王鑑卒 三月甲辰魏主謁永固陵夏四月癸亥朔設薦於太和廟|{
	太和廟據北史作太和殿水經注太和殿在太極殿東堂之東魏書帝紀太和元年起太和安昌二殿}
魏主始進蔬食追感哀哭終日不飯侍中馮誕等諫經宿乃飯甲子罷朝夕哭|{
	蓋亦不能及朞矣飯扶晩翻}
乙五復謁永固陵|{
	復扶又翻}
魏自正月不雨至于癸酉有司請祈百神帝曰成湯遭旱以至誠致雨|{
	謂湯以六事自責也}
固不在曲禱山川今普天喪恃|{
	喪息浪翻}
幽顯同哀何宜四氣未周|{
	謂一朞而四時之氣始周}
遽行祀事唯當責躬以待天譴|{
	譴去戰翻}
甲戌魏員外散騎常侍李彪等來聘為之置燕設樂

|{
	為于偽翻}
彪辭樂且曰主上孝思罔極興墜正失|{
	言行喪禮興百王之墜典而正其失也}
去三月晦朝臣始除衰絰猶以素服從事|{
	朝直遥翻衰吐回翻}
是以使臣不敢承奏樂之賜朝廷從之彪凡六奉使|{
	據魏紀上即位之初年至三年彪凡四來聘是年再聘通前凡六使疏吏翻}
上甚重之將還上親送至琅邪城命羣臣賦詩以寵之|{
	左傳晉趙武自宋還過鄭鄭伯享之于垂隴七穆皆從趙孟曰七子從君以寵武也請皆賦詩以卒君貺}
己卯魏作明堂改營太廟 五月己亥魏主更定律令於東明觀|{
	魏主太和四年起東明觀觀古玩翻更工行翻}
親决疑獄命李冲議定輕重潤色辭旨帝執筆書之李冲忠勤明斷加以慎密為帝所委情義無間|{
	斷丁亂翻}
舊臣貴戚莫不心服中外推之 乙卯魏長孫百年攻洮陽泥和二戌克之俘三千餘人丙辰魏初造五輅|{
	五輅玉金象革木也}
六月甲戌以尚書左僕射王奐為雍州刺史|{
	為後誅奐張本雍於用翻}
丁未魏濟隂王欝以貪殘賜死|{
	濟子禮翻}
秋閏七月乙丑魏主謁永固陵己卯魏主詔曰烈祖有創業之功世祖有開拓之德宜為祖宗百世不遷平文之功少於昭成而廟號太祖|{
	道武帝天興初追尊平文帝為太祖少詩沼翻}
道武之功高於平文而廟號烈祖|{
	明元帝追尊道武帝為烈祖}
於義未允朕今奉尊烈祖為太祖以世祖顯祖為二祧|{
	鄭玄曰廟之為言貌也宗廟者先祖之象貌也祧之言超也超上去意也}
餘皆以次而遷八月壬辰又詔議養老及禋于六宗之禮|{
	尚書禋于六宗而諸儒互說不同王莽以易六子遂立六宗祠王肅亦以為易六子摯虞以為月令孟冬天子祈來年于天宗天宗六宗之神也劉邵以為萬物負隂而抱陽冲氣以為和六宗者太極冲和之氣為六氣之宗者也虞書謂之六宗周書謂之天宗孔頴逹曰王肅六宗之說用家語之文以四時也寒暑也日也月也水旱也為六宗孔注尚書同之伏生與馬融以天地四時為六宗劉歆孔晁以乹坤之子六為六宗賈逵以為天宗三日月星也地宗三河海岱也今尚書歐陽夏侯說六宗者上及天下及地旁及四方中央恍惚助隂陽變化有益於人者也古尚書說天宗日月北辰岱河海也日月為隂陽宗北辰為星宗河為水宗岱為山宗海為澤宗鄭玄以星也辰也司中也司命也風師也而師也為六宗虞喜别論曰地有五色太社象之總五為一則成六六為地數推校經傳别無他祭也劉昭以為此說近得其實張髦曰父祖之廟六宗即三昭三穆也魏文帝以天皇大帝五帝為六宗杜佑取之鄭氏曰禋之言烟周人尚臭烟氣之臭聞者}
先是魏常以正月吉日於朝廷設幕中置松柏樹設五帝座而祠之|{
	先悉薦翻}
又有探策之祭帝皆以為非禮罷之戊戌移道壇於桑乾之隂改曰崇虛寺|{
	此即寇謙之道壇也探吐南翻乾音干}
乙巳帝引見羣臣|{
	見賢遍翻}
問以禘祫王鄭之義是非安在 |{
	考異曰禮志作太和十三年五月壬戌今從本紀}
尚書游明根等從鄭中書監高閭等從王詔圜丘宗廟皆有禘名從鄭禘祫并為一祭從王著之於令|{
	記大傳曰禮不王不禘王者禘其祖之所自出以其祖配之鄭氏注曰凡大祭曰禘大祭其先祖所由生謂郊祀天也王者之先皆感太微五帝之精以生皆用正歲之正月郊祭之又祭法言虞夏殷周禘郊祖宗之法鄭注云禘郊祖宗謂祭祀以配食也此禘謂祭昊天於圜丘也孔頴逹曰王肅論引賈逵說告禘于莊公禘者遞也審諦昭穆遷主遞位孫居王父之處又引禘于太廟逸禮昭尸穆尸其祝辭總稱孝子孝孫則是父子並列逸禮又云皆升合於太祖所以劉歆賈逵鄭衆馬融等皆以為然鄭不從者以公羊傳為正逸禮不可用也左氏說及杜元覬皆以為禘三年一大祭在太祖之廟傳無祫文然則祫即禘也取其序昭穆謂之禘取其合集羣祖謂之祫杜佑通典孝文帝太和十三年詔鄭玄云天子祭員丘曰禘宗廟大祭亦曰禘三年一祫五年一禘祫則毁廟羣廟之主於太祖廟合而祭之禘則增及百官配食者審諦而祭之魯禮三年喪畢而祫明年而禘員丘宗廟大祭俱稱禘祭有兩禘明也王肅又云天子諸侯皆禘於宗廟非祭天之祭郊祀后稷不稱禘禘祫一名也合祭故稱祫禘而審諦之故稱禘非兩祭之名三年一祫五年一禘總而互舉故稱五年再殷祭不言一禘一祫斷可知矣諸儒之說大略如是公卿可議其是非尚書游明根言曰鄭氏之義禘者大祭之名大祭員丘謂之禘者審諦五精星辰也大祭宗廟謂之禘者審諦其昭穆百官也員丘常合不言祫宗廟時合故言祫斯則宗廟祫禘並行員丘一禘而已宜於宗廟俱行禘祫之禮二禮異故名殊依禮春祭特礿於嘗於烝則祫嘗祫烝不於三時皆行禘祫之禮中書監高閭又言禘祭員丘與鄭義同者以為有虞氏禘黄帝黄帝非虞在廟之帝不在廟非員丘而何又大傳云祖其所自出之祖又非在廟之文論語稱禘自既灌以往據爾雅禘大祭也諸侯無禘禮惟夏祭稱禘又非宗廟之禘魯行天子之儀不敢專行員丘之禘改殷之禘取其禘名於宗廟因先有祫遂生兩名其宗廟禘祫之祭據王氏之義祫而禘禘止於一時一時者祭不欲數一歲三禘為過數詔曰明根閭等據二家之義論禘祫詳矣至於事取折衷猶有未允閭以禘祫為名義同王氏禘祭員丘事與鄭同無所間然明根以鄭氏等兩名兩祭並存並用理有未俱稱據二義一時禘祫而闕二時之禘事有難從先王制禮内緣人子之情外協尊卑之序故天子七廟數盡則毁藏主於太祖之廟三年而祫祭之世盡則毁以示有終之義三年而祫以申追遠之情禘祫既是一祭分而兩之事無所據毁廟三年一祫又有不盡四時於禮為闕七廟四時常祭祫則三年一祭而又不究四時於情為蕳王以祫為一祭於義為長鄭以員丘為禘與宗廟大祭同名義亦為當今互取鄭王二義禘祫并為一祭從王禘是祭員丘大祭之名上下同用從鄭若以數則凟五年一禘改祫從禘五年一禘則四時盡禘以稱今情則旅天禮文先禘而後時祭便即施行著之於令永為世法}
戊午又詔國家饗祀諸神凡一千二百餘處今欲减省羣祀務從簡約又詔明堂太廟配祭配享於斯備矣白登崞山雞鳴山廟唯遣有司行事|{
	明元帝永興四年立宣武廟于白登山歲一祭無常月神瑞二年帝又立宣武廟于白登西宣武帝至泰常五年始改謚道武水經注雞鳴山在廣甯郡下洛縣于延水北昔趙襄子殺代王於夏屋而并其土襄子之姊代王夫人也遂磨笄自殺代人憐之為立祠因名為磨笄山每夜有野鷄羣鳴於祠屋上故亦名為鳴鷄山文成帝保母常氏葬於是山别立寢廟太武帝保母竇氏葬崞山别立寢廟崞音郭}
馮宣王廟在長安宜敇雍州以時供祭|{
	馮宣王太后父朗也為秦雍二州刺史生后于長安後謚文宣王因立廟長安雍於用翻}
又詔先有水火之神四十餘名及城北星神今圓丘之下既祭風伯雨師司中司命|{
	鄭衆曰風師箕也雨師畢也司中三台三階也司命文昌宫星玄曰司中司命文昌第五第四星或曰中台上台也}
明堂祭門戶井竈中霤|{
	鄭氏曰中霤猶中室也古者複宂是以名室為霤音力又翻}
四十神悉可罷之甲寅詔曰近論朝日夕月|{
	三代之禮春朝朝日秋暮夕月朝直遥翻}
皆欲以二分之日於東西郊行禮然月有餘閏行無常凖若一依分日或值月於東而行禮於西序情即理不可施行昔秘書監薛謂等以為朝日以朔夕月以朏|{
	日月所會謂之合朔月生明謂之朏月之三日也胐敷尾翻}
卿等意謂朔胐二分何者為是尚書游明根等請用朔朏從之丙辰魏有司上言求卜祥日|{
	此小祥也}
詔曰筮日求吉既乖敬事之志又違永慕之心今直用晦日九月丁丑夜帝宿於廟帥羣臣哭己|{
	巳畢也帥讀曰率}
帝易服縞冠革帶黑屨侍臣易服黑介幘|{
	隋志幘尊卑貴賤皆服之文者長耳謂之介幘武者短耳謂之平上幘各稱其冠而制之縞古老翻}
白絹單衣革帶烏履遂哭盡乙夜戊子晦帝易祭服縞冠素紕|{
	紕匹毗翻又必二翻冠飾也緣也}
白布深衣|{
	記曰古者深衣蓋有制度以應規矩繩權衡短毋見膚長毋被土續衽鉤邊要縫半下袼之高下可以運肘袂之長短反詘之及肘帶下毋厭髀上毋厭脅當無骨者制十有二幅以應十二月袂圓以應規曲祫如矩以應方負繩及踝以應直下齊如權衡以應平故聖人服之先王貴之}
麻繩履侍臣去幘易幍|{
	弁缺四隅謂之幍傳子曰幍先未有岐荀文若巾觸樹成岐時人慕之因而弗改今通為慶弔之服白紗為之或單或袂去羌呂翻}
既祭出廟帝立哭久之乃還|{
	還從宣翻又如字下同}
冬十月魏明堂太廟成 庚寅魏主謁永固陵毁瘠猶甚穆亮諫曰陛下祥練已闋號慕如始|{
	古者既祥而練闋古穴翻終也說文曰闋事已也號戶刀翻如始言如初有喪}
王者為天地所子為萬民父母未有子過哀而父母不戚父母憂而子獨悅豫者也今和氣不應風旱為災願陛下襲輕服御常膳鑾輿時動咸秩百神|{
	秩者序而祭之}
庶使天人交慶詔曰孝悌之至無所不通今飄風旱氣皆誠慕未濃幽顯無感也所言過哀之咎諒為未衷|{
	衷善也正也適也}
十一月己未朔魏主禫於太和廟|{
	禫徒感翻除服之際也}
衮冕以祭既而服黑介幘素紗深衣拜陵而還癸亥冬至魏主祀圜丘遂祀明堂還至太和廟乃入甲子臨太華殿服通天冠絳紗袍以饗羣臣|{
	劉昭曰通天冠高九寸正豎頂少邪乃直下為鐵卷梁前有山展筩為述乘輿所常服也杜佑曰秦制通天冠其狀遺失漢因秦名制高九寸正豎頂少邪乃直下為鐵卷梁前有山展筩為述駮犀簪導乘輿所常服晉因漢制前加金博山述述即鷸也鷸知天雨故冠像焉前有展筩宋因之又加黑介幘東昏侯改用玉簪導梁武帝因之復加冕於其上謂之平天冕隋因之加金博山附蟬十二首施珠翠黑介幘玉簪導唐因之其纓改以翠緌}
樂縣而不作|{
	縣讀曰懸}
丁卯服衮冕辭太和廟帥百官奉神主遷于新廟|{
	新作太廟成故遷主新廟帥讀曰率}
乙亥魏大定官品戊戌考諸牧守|{
	守式又翻}
魏假通直散騎常侍李彪等來聘 魏舊制羣臣季

冬朝賀服袴褶行事謂之小歲|{
	朝直遥翻褶音習}
丙戌詔罷之十二月壬辰魏遷社於内城之西 魏以安定王休

為太傅齊郡王簡為太保 高麗王璉卒夀百餘歲|{
	麗力知翻}
魏主為之制素委貌布深衣|{
	為于偽翻委貌冠長七寸高四寸制如覆盃前高廣後卑銳所謂夏之母追殷之章甫者也本以皂絹為之今制素者以舉哀}
舉哀於東郊遣謁者僕射李安上策贈太傅諡曰康孫雲嗣立 乙酉魏主始迎春於東郊自是四時迎氣皆親之 初魏世祖克統萬及姑臧獲雅樂器服工人|{
	宋文帝元嘉四年魏克統萬十六年克姑臧晉永嘉之亂太常樂工多避地河西夏克長安獲秦雅樂故二國有其器服工人}
並存之其後累朝無留意者|{
	朝直遥翻}
樂工浸盡音制多亡高祖始命有司訪民間曉音律者議定雅樂當時無能知者然金石羽族之飾稍壯麗於往時矣辛亥詔簡置樂官使修其職又命中書監高閭參定 初晉張斐杜預共注律三十卷自泰始以來用之|{
	此晉泰始也}
律文簡約或一章之中兩家所處生殺頓異|{
	處昌呂翻}
臨時斟酌吏得為姦上留心法令詔獄官詳正舊注七年尚書刪定郎王植集定二注表奏之|{
	魏晉以來尚書諸曹無刪定郎此蓋刪定律注而置官}
詔公卿八座參議考正竟陵王子良總其事衆議異同不能壹者制旨平决是歲書成廷尉山隂孔稚珪上表以為律文雖定苟用失其平則法書徒明於袠裏|{
	袠與帙同}
寃䰟猶結於獄中竊尋古之名流多有法學今之士子莫肯為業縱有習者世議所輕將恐此書永淪走吏之手矣今若置律助教依五經例國子生有欲讀者策試高第即加擢用以補内外之官庶幾士流有所勸慕|{
	幾居希翻}
詔從其請事竟不行 初林邑王范陽邁世相承襲|{
	范陽邁見一百二十四卷宋文帝元嘉二十三年}
夷人范當根純攻奪其國遣使獻金簟等物詔以當根純為都督緣海諸軍事林邑王|{
	為下范諸農攻當根純張本使疏吏翻}
魏冀州刺史咸陽王禧入朝|{
	朝直遥翻}
有司奏冀州民三千人稱禧清明有惠政請世胙冀州魏主詔曰利建雖古未必今宜|{
	易曰利建侯}
經野由君理非下請|{
	周禮惟王建國辨方正位體國經野鄭玄注云經謂為之里數}
以禧為司州牧都督司豫等六州諸軍事 初魏文明太后寵任宦者略陽苻承祖官至侍中知都曹事|{
	知尚書都曹事也}
賜以不死之詔太后殂承祖坐應死魏主原之削職禁錮於家仍除悖義將軍封佞濁子|{
	悖蒲内翻}
月餘而卒承祖万用事親姻爭趨附以求利|{
	趨七喻翻}
其從母楊氏為姚氏婦|{
	從母即姨也從才用翻}
獨否常謂承祖之母曰姊雖有一時之榮不若妹有無憂之樂|{
	樂音洛}
姊與之衣服多不受彊與之|{
	彊其兩翻下彊使同}
則曰我夫家世貧美衣服使人不安不得已或受而埋之與之奴婢則曰我家無食不能飼也常著弊衣自執勞苦|{
	飼祥吏翻著則略翻}
承祖遣車迎之不肯起彊使人抱置車上則大哭曰爾欲殺我由是苻氏内外號為癡姨及承祖敗有司執其二姨至殿廷其一姨伏法帝見姚氏姨貧弊特赦之 李惠之誅也|{
	事見一百三十四卷宋順帝昇明二年}
思皇后之昆弟皆死|{
	魏孝文謚其母李貴人曰思皇后}
惠從弟鳳為安樂王長樂主簿長樂坐不軌誅|{
	事見一百三十五卷高帝建元元年從才用翻樂皆音洛}
鳳亦坐死鳳子安祖等四人逃匿獲免遇赦乃出既而魏主訪舅氏存者得安祖等皆封侯加將軍既而引見謂曰卿之先世再獲罪於時|{
	先世謂惠及鳳見賢遍翻}
王者設官以待賢才由外戚而舉者季世之法也卿等既無異能且可還家自今外戚無能者視此後又例降爵為伯去其軍號|{
	軍號將軍之號也去羌呂翻}
時人皆以為帝待馮氏太厚待李氏太薄太常高閭嘗以為言帝不聽及世宗尊寵外家乃以安祖弟興祖為中山太守追贈李惠開府儀同三司中山公諡曰莊

十年春正月戊午朔魏主朝饗羣臣於太華殿懸而不樂 己未魏主宗祀顯祖於明堂以配上帝遂登靈臺以觀雲物降居青陽左个布政事|{
	鄭氏曰青陽左个大寢東堂北偏}
自是每朔依以為常 散騎常侍庾蓽等聘於魏魏主使侍郎成淹引蓽等於館南瞻望行禮|{
	祀明堂登靈臺之禮}
辛酉魏始以太祖配南郊魏主命羣臣議行次|{
	五行之次也}
中書監高閭議以為帝王莫不以中原為正統不以世數為與奪善惡為是非故桀紂至虐不廢夏商之歷厲惠至昏無害周晉之錄晉承魏為金趙承晉為水燕承趙為木秦承燕為火秦之既亡魏乃稱制玄朔且魏之得姓出於軒轅|{
	魏書曰魏之先出自黄帝軒轅氏黄帝子昌意受封北國有大鮮卑山因以為號據史記以匈奴為夏后氏苗裔蓋有此理}
臣愚以為宜為土德|{
	按魏書帝紀道武天興元年羣臣奏國家承黄帝之後宜為土德高閭蓋申前議耳}
祕書丞李彪著作郎崔光等議以為神元與晉武往來通好至於桓穆志輔晉室|{
	事並見晉紀神元力微也桓帝猗㐌穆帝猗盧好呼到翻}
是則司馬祚終於郟鄏|{
	河南郡河南縣周之王城即郟鄏也郟古洽翻鄏音辱}
而拓跋受命於雲代昔秦并天下漢猶比之共工卒繼周為火德|{
	漢律歷志曰祭典曰共工氏覇九域言雖有水德在火木之間非其序也任智刑以彊故覇而不王秦以水德在周漢木火之間周人遷其行序故易不載卒子恤翻共讀曰恭}
况劉石苻氏地褊世促魏承其弊豈可捨晉而為土邪司空穆亮等皆請從彪等議壬戌詔承晉為水德祖申臘辰 |{
	考異曰禮志太和十五年正月穆亮等言云云按帝紀十六年正月壬戌詔定行次以水承金蓋志誤以六為五耳}
甲子魏罷租課|{
	租課李延壽魏紀作袒祼}
魏宗室及功臣子孫封王者衆乙丑詔自非烈祖之胄餘王皆降為公公降為侯而品如舊蠻王桓誕亦降為公唯上黨王長孫觀以其祖有大功特不降|{
	長孫道生以功封上黨王長知兩翻}
丹楊王劉昶封齊郡公加號宋王|{
	昶丑兩翻}
魏舊制四時祭廟皆用中節丙子詔始用孟月擇日而祭|{
	自漢以來宗廟成五祀四孟及臘是也魏初用中節夷禮也}
以竟陵王子良領尚書令 魏主毁太華殿為太極殿戊子徙居永樂宫|{
	魏主太和元年起永樂遊觀于平城之北苑樂音洛}
以尚書李冲領將作大匠與司空穆亮共營之 辛卯魏罷寒食饗|{
	舊傳冬至後一百五日為寒食初學記曰周舉移書魏武明罸令陸翽鄴中記並云寒食斷火起於介子推然周禮司烜氏仲春以木鐸徇火禁於國中注云為仲春將出火今寒食凖節氣是仲春之末清明是三月之初然則禁火並周制也魏先以寒食饗祖宗今以其非禮罷之}
甲午魏主始朝日于東郊自是朝日夕月皆親之|{
	朝直遥翻}
丁酉詔祀堯於平陽舜於廣甯禹於安邑周公於洛陽|{
	皆因其故都而祀之皇甫謐曰舜所都或言蒲阪或言潘潘今上谷也廣甯縣本屬上谷又據水經註潘當作瀵}
皆令牧守執事|{
	守式又翻}
其宣尼之廟祀於中書省丁未改謚宣尼曰文聖尼父帝親行拜祭魏舊制每歲祀天於西郊魏主與公卿從二千餘騎戎服遶壇謂之蹹壇|{
	騎奇寄翻蹹與踏同}
明日復戎服登壇致祀已又遶壇謂之遶天|{
	蕭子顯曰戎服遶壇魏主一周公卿七匝謂之蹹壇明日復戎服登壇祠天魏主遶三匝公卿七匝謂之遶天復扶又翻}
三月癸酉詔盡省之 卒已魏以高麗王雲為督遼海諸軍事遼東公高句麗王詔雲遣其世子入朝|{
	句如字又音駒麗力智翻}
雲辭以疾遣其從叔升干隨使者詣平城|{
	從才用翻}
夏四月丁亥朔魏班新律令大赦 辛丑豫章文獻王嶷卒|{
	嶷魚力翻}
贈假黄钺都督中外諸軍事丞相喪禮皆如漢東平獻王故事嶷性仁謹廉儉不以財賄為事齋庫失火|{
	齋庫齋内之庫}
燒荆州還資|{
	高祖建元二年嶷自荆州還為揚州}
評直三千餘萬|{
	評直論量其所直也}
主局各杖數十而已疾篤遺令諸子曰才有優劣位有通塞運有貧富此自然之理無足以相陵侮也|{
	蓋欲諸子不以位埶相陵塞悉則翻}
上哀痛特甚久之語及嶷猶欷歔流涕|{
	欷音希又許氣翻歔音虛}
嶷卒之日第庫無見錢|{
	見賢遍翻}
上敇月給嶷第錢百萬終上之世乃省五月己巳以竟陵王子良為揚州刺史 魏文明太后之喪使人告于吐谷渾吐谷渾王伏連籌拜命不恭|{
	吐從暾入聲谷音浴}
羣臣請討之魏主不許又請還其貢物帝曰貢物乃人臣之禮今而不受是棄絶之彼雖欲自新其路無由矣因命歸洮陽泥和之俘|{
	去年長孫百年所俘}
秋七月庚申吐谷渾遣其世子賀虜頭入朝于魏|{
	朝直遥翻下同考異曰魏吐谷渾傳作賀魯頭今從帝紀}
詔以伏連籌為都督西垂諸軍事西海公吐谷渾王遣兼員外散騎常侍張禮使於吐谷渾伏連籌謂禮曰曩者宕昌常自稱名而見謂為大王今忽稱僕又拘執使人欲使偏師往問何如禮曰君與宕昌皆為魏藩比輒興兵攻之殊違臣節|{
	使疏吏翻宕徒浪翻比毗至翻}
離京師之日宰輔有言以為君能自知其過則藩業可保|{
	離力智翻謂可保藩臣之業也}
若其不悛禍難將至矣|{
	悛丑緣翻難乃旦翻}
伏連籌默然 甲戌魏遣兼員外散騎常侍廣平宋弁等來聘及還魏主問弁江南何如弁曰蕭氏父子無大功於天下既以逆取不能順守政令苛碎賦役繁重朝無股肱之臣野有愁怨之民其得没身幸矣非貽厥孫謀之道也 八月乙未魏以懷朔鎮將陽平王頤鎮北大將軍陸叡皆為都督督十二將步騎十萬分為三道以擊柔然|{
	鎮將二將即亮翻騎奇寄翻 考異曰魏帝紀太和十一年八月壬申蠕蠕犯塞遣平原王陸叡討之事具蠕蠕傳十六年八月乙未詔陽平王頤左僕射陸叡討蠕蠕按蠕蠕傳無十一年犯塞及征討事唯有十六年八月頤叡出征事與紀合蓋十一年紀誤也}
中道出黑山東道趣士盧河西道趣侯延河軍過大磧大破柔然而還|{
	趣七喻翻磧七迹翻}
初柔然伏名敦可汗|{
	可從刋入聲汗音寒}
與其叔父那蓋分道擊高車阿伏至羅伏名敦屢敗那蓋屢勝國人以那蓋為得天助乃殺伏名敦而立那蓋號候其伏代庫者可汗|{
	魏收曰魏言悦樂也}
改元太安 魏司徒尉元大鴻臚卿游明根累表請老魏主許之引見|{
	尉紆勿翻臚陵六翻見賢遍翻}
賜元玄冠素衣|{
	石渠論玄冠朝服戴聖曰玄冠委貌也今此則玄冠委貌異制}
明根委貌青紗單衣及被服雜物等而遣之魏主親養三老五更於明堂己酉詔以元為三老明根為五更帝再拜三老親袒割牲執爵而饋肅拜五更|{
	周禮九拜九曰肅拜鄭司農云肅拜但俯下手今時撎是也陸德明曰撎於至翻即今之揖更工衡翻}
且乞言焉元明根勸以孝友化民又養庶老國老於階下禮畢各賜元明根以步挽車及衣服|{
	步挽車不用牛馬使人步挽之}
祿三老以上公五更以元卿|{
	元卿即上卿}
九月甲寅魏主序昭穆於明堂|{
	昭之招翻}
祀文明太后於玄室|{
	玄室北史作玄堂鄭玄曰玄堂北堂也}
辛未魏主以文明太后再朞哭於永固陵左終日不輟聲凡二日不食甲戌辭陵還永樂宫 武興氐王楊集始寇漢中至白馬梁州刺史隂智伯遣軍主桓盧奴隂冲昌等擊破之俘斬數千人集始走還武興請降于魏辛巳入朝于魏|{
	降戶江翻朝直遥翻}
魏以集始為南秦州刺史漢中郡侯武興王 冬十月甲午上殷祭太廟|{
	殷祭大祭也}
庚戌魏以安定王休為大司馬特進馮誕為司徒誕熙之子也|{
	馮熙見一百三十二卷宋順帝昇明元年熙文明后之兄也}
魏太極殿成 十二月司徒參軍蕭琛范雲聘于魏|{
	琛丑林翻}
魏主甚重齊人親與談論顧謂羣臣曰江南多好臣侍臣李元凱對曰江南多好臣歲一易主江北無好臣百年一易主魏主甚慙 上使太子家令沈約撰宋書疑立袁粲傳審之於上|{
	傳直戀翻}
上曰袁粲自是宋室忠臣|{
	此人心之公是非不可泯者}
約又多載宋世祖太宗諸鄙瀆事上曰孝武事迹不容頓爾我昔經事明帝卿可思諱惡之義|{
	春秋之義為尊者諱}
於是多所刪除 是歲林邑王范陽邁之孫諸農帥種人攻范當根純復得其國|{
	范當根純奪林邑國事見上年帥讀曰率種章勇翻復扶又翻}
詔以諸農為都督緣海諸軍事林邑王 魏南陽公鄭羲與李冲昏姻冲引為中書令出為西兖州刺史|{
	西兖州時治滑臺}
在州貪鄙文明太后為魏主納其女為嬪|{
	后為于偽翻嬪毗賓翻}
徵為祕書監及卒尚書奏謚曰宣詔曰蓋棺定謚激揚清濁故何曾雖孝良史載其繆醜|{
	事見八十卷晉武帝咸寜四年}
賈充有勞直士謂之荒公|{
	事見八十一卷晉武帝太康三年諡法昏亂紀度曰荒}
羲雖宿有文業而治闕廉清|{
	治直吏翻}
尚書何乃情違至公愆違明典依諡法博聞多見曰文不勤成名曰靈可贈以本官加謚文靈

資治通鑑卷一百三十七
