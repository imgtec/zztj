資治通鑑卷三十五   宋 司馬光 撰

胡三省 音註

漢紀二十七|{
	起屠維恊洽盡玄黓閹茂凡四年}


孝哀皇帝下

元夀元年春正月辛丑朔 |{
	考異曰荀紀云辛卯朔誤}
詔將軍中二千右舉明習兵灋者各一人因就拜孔鄉侯傅晏為大司馬衛將軍陽安侯丁明為大司馬票騎將軍|{
	用息夫躬之言也票頻妙翻}
是日日有食之上詔公卿大夫悉心陳過失又令舉賢良方正能直言者各一人大赦天下丞相嘉奏封事曰孝元皇帝奉承大業溫恭少欲|{
	少詩沼翻下同}
都内錢四十萬萬|{
	百官表大司農有都内令丞}
嘗幸上林後宮馮貴人從臨獸圈猛獸驚出貴人前當之|{
	即二十九卷建昭元年事也圈求遠翻}
元帝嘉美其義賜錢五萬|{
	師古曰言此事雖嘉其義而賞亦不多}
掖庭見親有加賞賜屬其人勿衆謝|{
	師古曰掖庭宮人有親戚來見而帝賜之者屬其家勿使於衆人中謝也見賢遍翻下見錢同屬音之欲翻余謂有見親幸者加之賞賜則屬其人勿於衆中謝也}
示平惡偏|{
	惡烏路翻}
重失人心賞賜節約是時外戚貲千萬者少耳故少府水衡見錢多也|{
	少府掌禁錢水衡都尉有鍾官辨銅令丞掌鑄錢師古曰見在之錢也}
雖遭初元永光凶年飢饉加以西羌之變|{
	永光二年隴西羌反}
外奉師旅内振貧民終無傾危之憂以府臧内充實也|{
	臧讀曰藏音徂浪翻}
孝成皇帝時諫臣多言燕出之害|{
	師古曰燕出謂微行也}
及女寵專愛耽於酒色損德傷年其言甚切然終不怨怒也寵臣淳于長張放史育育數貶退家貲不滿千萬放斥逐就國|{
	事見三十三卷成帝綏和二年數所角翻}
長榜死於獄|{
	事見三十二卷綏和元年榜音彭}
不以私愛害公義故雖多内譏朝廷安平|{
	師古曰言雖有好内之譏而不害政也}
傳業陛下陛下在國之時好詩書|{
	好呼到翻}
上儉節徵來所過道上稱誦德美此天下所以囘心也|{
	師古曰望為治也余謂囘心者囘其戴成帝之心而戴哀帝也}
初即位易帷帳去錦繡|{
	去羌呂翻}
乘輿席緣綈繒而已|{
	乘繩證翻緣俞絹翻師古曰綈厚繒也音徒奚翻繒慈陵翻}
共皇寢廟比當作|{
	共音恭比近也音毗至翻}
憂閔元元惟用度不足|{
	師古曰惟思也}
以義割恩輒且止息今始作治|{
	治直之翻下同}
而駙馬都尉董賢亦起官寺上林中又為賢治大第開門鄉北闕|{
	為于偽翻鄉讀曰嚮下獨鄉同}
引王渠灌園池|{
	蘇林曰王渠宮渠也猶今御溝也晉灼曰渠名也在城東覆盎門外師古曰晉說是}
使者護作賞賜吏卒甚於治宗廟|{
	師古曰護監視也}
賢母病長安厨給祠具|{
	師古曰長安有厨官主為官食}
道中過者皆飲食|{
	如淳曰禱於道中故行人皆得飲食余謂若據文理則飲音於禁翻食讀曰飤}
為賢治器器成奏御乃行或物好特賜其工自貢獻宗廟三宮猶不至此|{
	師古曰三宮天子太后皇后也原父曰是時太皇太后稱長信宮傅太后稱永信宮而丁姬稱中安宮故以三宮為言余按此時丁姬死矣三宮蓋謂長信永信及趙太后宮也}
賢家有賓婚及見親諸官並共|{
	師古曰見親親戚相見也並共言百官各以所掌事及財物就供之共讀曰供}
賜及倉頭奴婢人十萬錢使者護視發取市物百賈震動|{
	師古曰賈為販賣之人也言百賈者非一之稱也賈音古}
道路讙譁|{
	讙許元翻}
羣臣惶惑詔書罷苑而以賜賢二千餘頃均田之制從此墮壞|{
	孟康曰自公卿以下至于吏民名曰均田皆有頃數於品制中令均等今賜賢二千餘頃則壞其制也師古曰墮音火規翻均田見三十三卷綏和二年}
奢僭放縱變亂隂陽災異衆多百姓訛言持籌相驚|{
	師古曰言行西王母籌也}
天惑其意不能自止陛下素仁智愼事今而有此大譏孔子曰危而不持顛而不扶則將焉用彼相矣|{
	師古曰論語季氏將伐顓臾冉有季路見於孔子孔子以此言責之以其不匡諫也相息亮翻}
臣嘉幸得備位竊内悲傷不能通愚忠之信身死有益於國不敢自惜唯陛下愼己之所獨鄉察衆人之所共疑往者鄧通韓嫣驕貴失度逸豫無厭小人不勝情欲卒陷罪辜亂國亡軀不終其禄|{
	鄧通幸於文帝賜以蜀嚴道銅山景帝立人告通盜出徼外鑄錢没入其家卒以餓死韓嫣幸於武帝出入永巷不禁以奸聞賜死嫣音偃厭於鹽翻勝音升卒子恤翻}
所謂愛之適足以害之者也宜深覽前世以節賢寵全安其命上由是於嘉浸不說|{
	為嘉死不以罪張本說讀曰悦}
前涼州刺史杜鄴以方正對策曰臣聞陽尊隂卑天之道也是以男雖賤各為其家陽女雖貴猶為其國隂故禮明三從之義|{
	師古曰謂婦人在家從父既嫁從夫夫死從子}
雖有文母之德必繫於子|{
	師古曰文母文王之妃太姒也仲馮曰文母文王之母也所謂繫於子也何預太姒余謂劉說是詩曰思齊太任文王之母}
昔鄭伯隨姜氏之欲終有叔段簒國之禍|{
	左傳鄭武姜生莊公及共叔段惡莊公而愛段為之請京使居之祭仲諫曰國之害也公曰姜氏欲之焉辟害段繕甲兵具卒乘將襲鄭莊公克之}
周襄王内迫惠后之難而遭居鄭之危|{
	史記周惠王二子長襄王次叔帶惠后愛叔帶襄王既立叔帶召狄人狄人伐周王御士將禦之王曰先后其謂我何乃出居于鄭難乃旦翻}
漢興呂太后權私親屬幾危社稷|{
	事見高后紀幾居依翻}
竊見陛下約儉正身欲與天下更始|{
	更工衡翻}
然嘉瑞未應而日食地震案春秋災異以指象為言語|{
	師古曰謂天不言但以景象指意告諭人}
日食明陽為隂所臨|{
	日者陽宗隂盛陽微日為所揜而食是為隂所臨也}
坤以灋地為土為母以安静為德震不隂之效也|{
	師古曰言地當安静而今乃震是為不遵隂道也}
占象甚明臣敢不直言其事昔曾子問從令之義孔子曰是何言與|{
	師古曰曾子問從父之令可謂孝乎孔子非之事見孝經與讀曰歟}
善閔子騫守禮不苟從親所行無非理者故無可間也|{
	師古曰論語稱孔子曰孝哉閔子騫人不間於其父母兄弟之言是也間音居莧翻}
今諸外家昆弟無賢不肖並侍帷幄布在列位|{
	師古曰不問賢與不肖皆親近在位}
或典兵衛或將軍屯|{
	將即亮翻}
寵意并於一家積貴之埶世所希見所希聞也至乃並置大司馬將軍之官皇甫雖盛三桓雖隆魯為作三軍無以甚此|{
	言周以皇甫為卿士魯三桓彊盛作三軍而三分公室比丁傅無以甚也為于偽翻}
當拜之日晻然日食|{
	師古曰晻音烏感翻}
不在前後臨事而發者明陛下謙遜無專承指非一所言輒聽所欲輒隨|{
	師古曰謂皆迫於太后也}
有罪惡者不坐辜罰無功能者畢受官爵流漸積猥過在於是|{
	猥遝也言有罪惡者不誅無功能者並進其流漸至積遝也}
欲令昭昭以覺聖朝|{
	朝直遥翻}
昔詩人所刺春秋所譏指象如此殆不在它由後視前忿邑非之|{
	師古曰由從也邑於邑也}
逮身所行不自鏡見則以為可計之過者|{
	師古曰逮及也鏡鑒照也自以所行為可是計策之誤者}
願陛下加致精誠思承始初事稽諸古以厭下心|{
	師古曰每事皆考於古者厭滿也音一贍翻}
則黎庶羣生無不說喜|{
	說讀曰悦}
上帝百神收還威怒禎祥福禄何嫌不報|{
	師古曰嫌疑也}
上又徵孔光詣公車問以日食事拜為光禄大夫秩中二千石給事中位次丞相初王莽既就國|{
	建平二年莽就國}
杜門自守其中子獲殺奴|{
	中讀曰仲}
莽切責獲令自殺在國三歲吏民上書寃訟莽者百數|{
	師古曰言其合管朝政不當就國也上時掌翻}
至是賢良周護宋崇等對策復深訟莽功德|{
	復扶又翻}
上於是徵莽及平阿侯仁還京師侍太后|{
	侍太皇太后也}
董賢因日食之變以沮傅晏息夫躬之策|{
	沮在呂翻}
辛卯上收晏印綬罷就第 丁巳皇太太后傅氏崩合葬渭陵稱孝元傅皇后|{
	史記正義曰漢帝后同陵則為合葬不合陵也諸陵皆如此傅氏以側室而合葬稱孝元傳皇后太皇太后在上此心為何如邪宜其啓王莽而授之以柄也}
丞相御史奏息夫躬孫寵等辠過上乃免躬寵官遣就國又罷侍中諸曹黄門郎數十人|{
	丁傅之親黨也}
鮑宣上書曰陛下父事天母事地子養黎民即位以來父虧明母震動子訛言相驚恐今日食於三始|{
	如淳曰正月一日為歲之朝月之朝日之朝朝猶始也}
誠可畏懼小民正朔日尚恐毁敗器物|{
	敗補邁翻}
何況於日虧乎陛下深内自責避正殿舉直言求過失罷退外親及旁仄素餐之人徵拜孔光為光禄大夫發覺孫寵息夫躬過惡免官遣就國衆庶歙然莫不說喜|{
	師古曰歙音翕}
天人同心人心說則天意解矣|{
	說讀曰悦}
乃二月丙戌白虹干日|{
	虹日旁氣也白兵象干犯也}
連隂不雨此天下憂結未解民有怨望未塞者也|{
	塞悉則翻}
侍中駙馬都尉董賢本無葭莩之親但以令色諛言自進|{
	令善也諛諂也孔安國曰令色無質巧言無實}
賞賜無度竭盡府臧|{
	臧讀曰藏音徂浪翻}
并合三第尚以為小復壞㬥室|{
	師古曰時以三第總為一第賜賢猶嫌陿小復取㬥室之地以增益之也復扶又翻下同}
賢父子坐使天子使者|{
	言其驕慢也}
將作治第|{
	將作大匠掌治宮室使為之治第治直之翻}
行夜吏卒皆得賞賜|{
	師古曰為賢第上持時行夜者行音下孟翻}
上冢有會輒太官為供|{
	為之供具也為于偽翻下同}
海内貢獻當養一君今反盡之賢家豈天意與民意邪天不可久負|{
	暴殄天物以私嬖倖是為負天}
厚之如此反所以害之也誠欲哀賢宜為謝過天地解讐海内免遣就國收乘輿器物還之縣官|{
	此所謂謝過解讐也為于偽翻乘繩證翻}
可以父子終其性命不者海内之所仇未有得久安者也孫寵息夫躬不宜居國可皆免以視天下|{
	師古曰視讀曰示}
復徵何武師丹彭宣傅喜曠然使民易視以應天心建立大政興太平之端上大感異納宣言徵何武彭宣拜鮑宣為司隸 上託傅太后遺詔令太皇太后下丞相御史益封董賢二千戶|{
	賢先封千戶下遐稼翻}
賜孔鄉侯汝昌侯陽新侯國|{
	三人者先雖封侯未有國邑今賜之國邑也陽新侯即陽信侯鄭業漢書傅昭儀傳作陽信王嘉傳及恩澤侯表作陽新}
王嘉封還詔書|{
	後世給舍封駁本此師古曰還謂郤上之於天子也}
因奏封事諫曰臣聞爵禄土地天之有也書云天命有德五服五章哉|{
	師古曰虞書辠陶謨之辭也言皇天命有德者以居列位天子諸侯卿大夫士尊卑之服采章各異也}
王者代天爵人尤宜愼之裂地而封不得其宜則衆庶不服感動隂陽其害疾自深|{
	師古曰言此氣損害故令天子身有疾也}
今聖體久不平此臣嘉所内懼也高安侯賢佞幸之臣陛下傾爵位以貴之單貨財以富之|{
	師古曰單盡也}
損至尊以寵之|{
	師古曰言上意傾惑為下所窺也余謂帝為賢治第儗於宮闕乘輿器物充物其家此所謂損至尊以寵之也}
主威已黜府臧已竭唯恐不足財皆民力所為孝文欲起露臺重百金之費克己不作|{
	事見文帝紀}
今賢散公賦以施私惠一家至受千金往古以來貴臣未嘗有此流聞四方皆同怨之里諺曰千人所指無病而死臣常為之寒心|{
	為于偽翻}
今太皇太后以永信太后遺詔詔丞相御史益賢戶賜三侯國臣嘉竊惑山崩地動日食於三朝皆隂侵陽之戒也前賢已再封|{
	謂賢先封關内侯繼封高安侯也}
晏商再易邑|{
	商先嗣爵崇祖侯後改封汝昌侯晏無所考按表晏先以皇后父封三千戶又益二千戶食邑於夏丘}
業緣私横求|{
	横戶孟翻}
恩已過厚求索自恣不知厭足甚傷尊尊之義|{
	封三侯者所以尊傅太后今求濫恩不知厭足則傷尊尊之義矣索山客翻厭於鹽翻}
不可以示天下為害痛矣|{
	痛甚也}
臣驕侵罔|{
	師古曰罔謂誣蔽也}
隂陽失節氣感相動害及身體陛下寢疾久不平繼嗣未立宜思正萬事順天人之心以求福祐何乃輕身肆意|{
	師古曰肆放也}
不念高祖之勤苦垂立制度欲傳之於無窮哉臣謹封上詔書不敢露見|{
	上時掌翻見賢遍翻}
非愛死而不自灋|{
	謂不以違拒詔指之法自劾也}
恐天下聞之故不敢自劾|{
	言自劾則天下知其事也劾戶槩翻}
初廷尉梁相治東平王雲獄時冬月未盡二旬而相心疑雲寃獄有飾辭|{
	師古曰假飾之辭非其實也治直之翻}
奏欲傳之長安|{
	師古曰傳謂移其獄事也傳知戀翻}
更下公卿覆治|{
	下遐稼翻下同}
尚書令鞫譚僕射宗伯鳳以為可許|{
	師古曰鞫及宗伯皆姓也宗伯以官為氏鞫音居六翻}
天子以為相等皆見上體不平外内顧望操持兩心|{
	師古曰操音千高翻}
幸雲踰冬無討賊疾惡主讐之意|{
	謂僥幸雲獄踰冬則雲可以減死也}
免相等皆為庶人後數月大赦嘉薦相等皆有材行|{
	行下孟翻}
聖王有計功除過臣竊為朝廷惜此三人|{
	按公卿表建平元年大司農梁相為廷尉二年貶為東海都尉三年左馮翊方賞為廷尉四年徙本紀東平王雲有罪自殺在建平四年大赦天下在今年正月若以表為證則當治東平時廷尉乃方賞非梁相表言相貶不言免為庶人又今年大赦上距建平三年十二月治東平獄時已一朞有餘是大赦亦不在後數月也通鑑書王嘉薦梁相等三人全取漢書王嘉傳然傳與紀表歲月自相牴牾繫年之書可謂難矣為于偽翻下同}
書奏上不能平|{
	師古曰心怒也}
後二十餘日嘉封還益董賢戶事上乃發怒召嘉詣尚書責問以相等前坐不忠罪惡著聞君時輒已自劾今又稱譽云為朝廷惜之何也|{
	劾戶槩翻譽音余}
嘉免冠謝罪事下將軍朝者|{
	朝者當時見入朝之臣也朝直遥翻}
光禄大夫孔光等劾嘉迷國罔上不道請謁者召嘉詣廷尉詔獄議郎龔等以為嘉言事前後相違宜奪爵土免為庶人永信少府猛等以為嘉罪名雖應灋|{
	謂灋當下吏也}
大臣括髮關械裸躬就笞|{
	師古曰括結也關紐也裸露也}
非所以重國褒宗廟也上不聽詔假謁者節召丞相詣廷尉詔獄使者既到府掾史涕泣共和藥進嘉|{
	掾命絹翻和戶卧翻}
嘉不肯服主簿曰將相不對理陳寃相踵以為故事|{
	自周勃繋獄賈誼以為言文帝自此待大臣有節將相有罪皆自殺不受刑然景帝時周亞夫武帝時公孫賀劉屈氂猶下獄死相踵為故事言其槩也理獄也對理對獄也言大臣之體縱有寃不對獄而自陳也師古曰踵猶躡也}
君侯宜引決|{
	師古曰令自殺也}
使者危坐府門上|{
	師古曰以逼促嘉也}
主簿復前進藥嘉引藥杯以擊地謂官屬曰|{
	官屬謂掾史主簿等}
丞相幸得備位三公奉職負國當伏刑都市以示萬衆丞相豈兒女子邪何謂咀藥而死|{
	師古曰咀嚼也音材汝翻}
嘉遂裝出見使者再拜受詔|{
	裝出者朝服而出}
乘吏小車去蓋不冠|{
	去羌呂翻}
隨使者詣廷尉廷尉收嘉丞相新甫侯印綬|{
	恩澤侯表新甫侯國於南陽新野}
縛嘉載致都船詔獄|{
	百官表執金吾屬官有中壘寺互武庫都船四令丞如淳曰漢儀注有寺互都船獄}
上聞嘉生自詣吏大怒|{
	上欲嘉自裁而嘉詣獄故大怒}
使將軍以下與五二千石雜治|{
	漢治大臣獄率使五二千石今又使將軍同治之怒之甚也晉灼曰大臣獄重故使秩二干石者五人雜治之}
吏詰問嘉|{
	詰去吉翻}
嘉對曰案事者思得實竊見相等前治東平王獄不以雲為不當死欲關公卿示重愼|{
	關會也古者獄成命三公六卿參聽之示明謹於用刑也}
誠不見其外内顧望阿附為雲驗|{
	驗徵驗也為于偽翻下同}
復幸得蒙大赦|{
	復扶又翻}
相等皆良善吏臣竊為國惜賢不私此三人獄吏曰苟如此則君何以為罪猶當有以負國不空入獄矣|{
	吏欲文致以負國之罪故云然}
吏稍侵辱嘉嘉喟然仰天歎曰|{
	喟丘愧翻歎息之聲}
幸得充備宰相不能進賢退不肖以是負國死有餘責|{
	吏責其負國故以此對}
吏問賢不肖主名嘉曰賢故丞相孔光故大司空何武不能進惡高安侯董賢父子亂朝而不能退|{
	惡烏路翻朝直遥翻}
罪當死死無所恨嘉繫獄二十餘日不食歐血而死已而上覽其對思嘉言會御史大夫賈延免夏五月乙卯以孔光為御史大夫秋七月丙午以光為丞相復故國博山侯又以汜鄉侯何武為御史大夫上乃知孔光前免非其罪以過近臣毁短光者|{
	光免事見上卷建平二年過督過也咎之也}
曰傅嘉前為侍中毁譛仁賢誣愬大臣令俊艾者久失其位|{
	師古曰艾讀曰乂}
其免嘉為庶人歸故郡|{
	傅氏本河内溫人}
八月何武徙為前將軍辛卯光禄大夫彭宣為御史大夫 司隸鮑宣坐摧辱丞相拒閉使者無人臣禮減死髠鉗|{
	丞相孔光四時行園陵官屬行馳道中宣使吏銁止丞相掾史没入其車馬摧辱丞相事下御史中丞侍御史至司隸官欲捕從事閉門不肯内坐以拒閉使者罪}
大司馬丁明素重王嘉以其死而憐之九月乙卯册免明使就第 冬十一月壬午以故定陶太傅光禄大夫韋賞為大司馬車騎將軍己丑賞卒|{
	成帝省王國太傅更曰傅此猶曰太傅者習於舊稱未能頓從新稱也賞韋賢之孫弘之子也}
十二月庚子以侍中駙馬都尉董賢為大司馬衛將軍册曰建爾于公以為漢輔往悉爾心匡正庶事允執其中是時賢年二十二雖為三公常給事中領尚書事|{
	給事禁中而領尚書事也}
百官因賢奏事|{
	審食其以丞相而侍禁中呂后嬖之也董賢以三公侍禁中哀帝嬖之也論道經邦之任安在哉}
以父衛尉恭不宜在卿位徙為光禄大夫秩中二千石弟寛信代賢為駙馬都尉董氏親屬皆侍中諸曹奉朝請寵在丁傅之右矣|{
	朝直遥翻請才性翻又如字}
初丞相孔光為御史大夫賢父恭為御史事光及賢為大司馬與光並為三公上故令賢私過光|{
	令私往見之觀其所以接之者何如也}
光雅恭謹|{
	雅素也}
知上欲尊寵賢及聞賢當來也光警戒衣冠出門待望見賢車乃却入賢至中門光入閤既下車乃出拜謁送迎甚謹不敢以賓客鈞敵之禮|{
	此非恭而無禮者邪光能卑事董賢則必能曲狥王莽矣}
上聞之喜立拜光兩兄子為諫大夫|{
	光有三兄福捷喜未知兩兄子為誰}
常侍|{
	為諫大夫而加常侍官也}
賢自是權與人主侔矣|{
	師古曰侔等也}
是時成帝外家王氏衰廢唯平阿侯譚子去疾為侍中|{
	去羌呂翻}
弟閎為中常侍閎妻父中郎將蕭咸前將軍望之子也賢父恭慕之欲為子寛信求咸女為婦|{
	為于偽翻}
使閎言之咸惶恐不致當私謂閎曰董公為大司馬册文言允執其中此乃堯禪舜之文|{
	論語堯曰咨爾舜天之歷數在爾躬允執其中四海困窮天禄永終舜亦以命禹道統之傳此乎出也}
非三公故事|{
	言考之漢家故事册三公者未嘗有此語也}
長老見者莫不心懼|{
	長知兩翻}
此豈家人子所能堪邪|{
	師古曰家人猶言庶人也蓋咸自謂}
閎性有知略|{
	知讀曰智}
聞咸言亦悟乃還報恭深達咸自謙薄之意恭歎曰我家何用負天下而為人所畏如是意不說|{
	說讀曰悦下同}
後上置酒麒麟殿|{
	師古曰在未央宮中}
賢父子親屬宴飲侍中中常侍皆在側上有酒所|{
	師古曰言酒在體中}
從容視賢|{
	從于容翻}
笑曰吾欲灋堯禪舜何如王閎進曰天下乃高皇帝天下非陛下有也陛下承宗廟當傳子孫於亡窮統業至重天子亡戲言|{
	周成王剪桐葉為珪與小弱弟戲曰以封汝周公入賀王曰戲也周公曰王者不可戲乃封小弱弟於唐古亡無通}
上默然不說左右皆恐於是遣閎出歸郎署|{
	三署郎各有署署舍遣出不得侍禁中也 考異曰董賢傳但云遣閎不得復侍宴自歸郎署以下皆漢紀所載也荀紀無漢書外事不知此語荀悦何從得之又云閎歸郎署二十日長樂宮深為閎謝又御史大夫彭宣上封事言國安危繼嗣事上覺寤召閎按太皇太后居長信宮云長樂宮誤也 余按漢書注長信宮以長樂官中長信殿為稱亦可言長樂宮也}
久之太皇太后為閎謝復召閎還|{
	為于偽翻復扶又翻下同}
閎遂上書諫曰臣聞王者立三公灋三光|{
	三光日月星也上時掌翻}
居之者當得賢人易曰鼎折足覆公餗喻三公非其人也|{
	易鼎之九四曰鼎折足覆公餗餗音送鹿翻虞云八珍之具也馬云䭈也䭈音之然翻鄭云菜也折而設翻}
昔孝文皇帝幸鄧通不過中大夫武帝幸韓嫣賞賜而已皆不在大位|{
	武帝幸韓嫣賞賜儗鄧通位不過上大夫以罪賜死嫣於䖍翻}
今大司馬衛將軍董賢無功於漢朝又無肺腑之連復無名迹高行以矯世|{
	行戶孟翻}
昇擢數年列備鼎足典衛禁兵無功封爵父子兄弟横蒙拔擢賞賜空竭帑藏|{
	横戶孟翻帑它朗翻藏徂浪翻}
萬民誼譁偶言道路誠不當天心也昔褒神蚖變化為人實生褒姒亂周國|{
	國語曰夏之衰也有二龍降于夏庭言曰予褒之二君也夏后請其漦而藏之歷殷周莫敢發也至厲王之末發而觀之漦流於庭不可除也王使婦人不帷而譟之其神化為玄蚖入于王府府之童妾既齓而遭之既笄而孕懼而弃之鬻檿弧者收以奔褒是為褒姒褒人有獄以入於幽王王嬖之生伯服遂黜申后而立褒姒廢太子而立伯服以亂周國蚖音元又吾官翻漦似甾翻檿於琰翻}
恐陛下有過失之譏賢有小人不知進退之禍非所以垂灋後世也上雖不從閎言多其年少志彊|{
	少詩照翻下同}
亦不罪也

二年春正月匈奴單于及烏孫大昆彌伊秩靡皆來朝|{
	朝直遥翻}
漢以為榮是時西域凡五十國|{
	三十六國分為五十餘國}
自譯長至將相侯王皆佩漢印綬凡三百七十六人|{
	譯長之官西域諸國皆有之所以通其國之語言於中國長知兩翻}
而康居大月氏安息罽賓烏弋之屬|{
	烏弋山離國去長安萬二千二百里不屬都護東與罽賓西與犂軒條支接氏音支罽音計}
皆以絶遠不在數中其來貢獻則相與報不督録總領也|{
	正謂不屬都護也}
自黄龍以來單于每入朝其賞賜錦繡繒絮輒加厚於前以慰接之單于宴見|{
	見賢遍翻}
羣臣在前單于怪董賢年少以問譯|{
	師古曰譯傳語之人也}
上令譯報曰大司馬年少以大賢居位單于乃起拜賀漢得賢臣是時上以太歲厭勝所在|{
	是年太歲在申師古曰厭音一涉翻}
舍單于上林苑蒲陶宮|{
	蒲陶本出大宛武帝伐大宛采蒲陶種植之離宮宮由此得名師古曰舍止宿}
告之以加敬於單于單于知之不悦 夏四月壬辰晦日有食之五月甲子正三公官分職|{
	成帝綏和二年置三公官哀帝建平三年罷今復正}


|{
	三公官名方職謂大司馬掌兵事大司徒掌人民事大司空掌水土事分扶問翻}
大司馬衛將軍董賢為大司馬丞相孔光為大司徒彭宣為大司空封長平侯|{
	恩澤侯表長平侯國於濟南}
六月戊午帝崩于未央宮|{
	臣瓚曰帝年二十即位即位六年夀廿五師古曰即位明年乃改元夀二十六}
帝睹孝成之世禄去王室|{
	謂政在王氏也}
及即位屢誅大臣|{
	謂殺朱博王嘉等}
欲彊主威以則武宣|{
	師古曰則法也}
然而寵信讒諂|{
	謂趙昌董賢息夫躬等}
憎疾忠直|{
	謂師丹傅喜鄭崇等}
漢業由是遂衰太皇太后聞帝崩即日駕之未央宮|{
	之往也}
收取璽綬|{
	璽斯氏翻綬音受}
太后召大司馬賢引見東箱|{
	見賢遍翻}
問以喪事調度|{
	師古曰調選發也度計料也調徒弔翻}
賢内憂不能對免冠謝太后曰新都侯莽前以大司馬奉送先帝大行|{
	謂成帝之喪也}
曉習故事吾令莽佐君賢頓首幸甚太后遣使者馳召莽詔尚書諸發兵符節百官奏事中黄門期門兵皆屬莽|{
	中黄門守禁門黄闥者也期門兵守衛殿門者也}
莽以太后指使尚書劾賢帝病不親醫藥|{
	劾戶槩翻}
禁止賢不得入宮殿司馬中|{
	據漢書趙充國傳子卬入莫府司馬中亂屯兵如淳注曰司馬中律所謂營軍司馬中也余謂此宮殿司馬中蓋宮殿屯衛司馬中也}
賢不知所為詣闕免冠徒跣謝己未莽使謁者以太后詔即闕下册賢|{
	師古曰即就也}
曰賢年少未更事理|{
	少詩照翻師古曰更歷也音工衡翻}
為大司馬不合衆心其收大司馬印綬罷歸第即日賢與妻皆自殺家惶恐夜葬莽疑其詐死有司奏請發賢棺至獄診視|{
	師古曰謂發冢取其棺柩也診驗也音軫}
因埋獄中太皇太后詔公卿舉可大司馬者莽故大司馬辭位避丁傅衆庶稱以為賢|{
	事見三十三卷綏和二年}
又太皇太后近親自大司徒孔光以下舉朝皆舉莽|{
	朝直遥翻}
獨前將軍何武左將軍公孫禄二人相與謀以為往時惠昭之世外戚呂霍上官持權幾危社稷|{
	幾居希翻}
今孝成孝哀比世無嗣|{
	比頻寐翻}
方當選立近親幼主不宜令外戚大臣持權|{
	外戚大臣謂王莽也}
親疏相錯|{
	親謂外戚疏謂異姓之為將軍公卿者師古曰錯間雜也}
為國計便|{
	為國之計唯此為便}
於是武舉公孫禄可大司馬而禄亦舉武庚申太皇太后自用莽為大司馬領尚書事太皇太后與莽議立嗣安陽侯王舜莽之從弟其人修飭|{
	師古曰飭讀與敕同敕整也從才用翻}
太皇太后所信愛也莽白以舜為車騎將軍秋七月遣舜與大鴻臚左咸使持節迎中山王箕子以為嗣|{
	使持節者奉使而持節也魏晉以下遂以為官稱臚陵如翻嗣祥吏翻}
莽又白太皇太后詔有司以皇太后與女弟昭儀專寵錮寢|{
	錮塞也杜塞後宮侍寢之路不使進御也}
殘滅繼嗣|{
	詳見上卷建平元年}
貶為孝成皇后徙居北宮又以定陶共王太后與孔鄉侯宴同心合謀背恩忘本專恣不軌|{
	背蒲妹翻}
徙孝哀皇后退就桂宮傅氏丁氏皆免官爵歸故郡|{
	莽以積年閒退之久一旦得權無所不至傅氏河内人丁氏山陽人}
傅宴將妻子徙合浦獨下詔褒揚傳喜曰高武侯喜姿性端慤|{
	師古曰慤謹也音口角翻}
論議忠直雖與故定陶太后有屬終不順指從邪介然守節以故斥逐就國|{
	喜就國見上卷建平二年}
傳不云乎歲寒然後知松柏之後凋也|{
	師古曰論語載孔子之言以喻有節操之人也}
其還喜長安位特進奉朝請喜雖外見褒賞孤立憂懼後復遣就國以夀終|{
	復扶又翻}
莽又貶傅太后號為定陶共王母丁太后號曰丁姬莽又奏董賢父子驕恣奢僭請收没入財物縣官諸以賢為官者皆免父恭弟寛信與家屬徙合浦母别歸故郡鉅鹿長安中小民讙譁鄉其第哭幾獲盜之|{
	師古曰陽往哭之實欲竊盜也讙音許爰翻鄉讀曰嚮幾讀曰冀}
縣官斥賣董氏財凡四十三萬萬賢所厚吏沛朱詡自劾去大司馬府|{
	劾戶槩翻}
買棺衣收賢屍葬之莽聞之以它辠擊殺詡莽以大司徒孔光名儒相三主|{
	成哀及平帝為三主}
太后所敬天下信之於是盛尊事光引光女壻甄邯為侍中奉車都尉|{
	甄之人翻姓也陳留風俗傳曰舜陶甄河濱其後為氏邯戶甘翻}
諸素所不說者|{
	說讀曰悦}
莽皆傅致其罪|{
	師古曰傅讀曰附附益而引致之令入罪}
為請奏草令邯持與光以太后指風光|{
	師古曰草謂文書之槀草也風讀曰諷}
光素畏愼不敢不上之|{
	上時掌翻}
莽白太后輒可其奏於是劾奏何武公孫禄互相稱舉皆免官武就國又奏董宏子高昌侯武父為佞邪奪爵|{
	宏為佞邪謂請立丁姫為帝太后也}
又奏南郡太守母將隆前為冀州牧治中山馮太后獄寃陷無辜關内侯張由誣告骨肉中太僕史立泰山太守丁玄陷人入大辟|{
	事並見三十三卷建平元年辟毗亦翻}
河内太守趙昌譛害鄭崇|{
	事見上卷建平四年}
幸逢赦令皆不宜處位在中土|{
	處昌呂翻}
免為庶人徙合浦中山之獄本立玄自典考之但與隆連名奏事莽少時慕與隆交隆不甚附故因事擠之|{
	少詩照翻擠子計翻又牋西翻排也}
紅陽侯立太后親弟雖不居位莽以諸父内敬憚之畏立從容言太后令已不得肆意復令光奏立罪惡|{
	從子容翻復扶又翻下同}
前知定陵侯長犯大逆罪為言誤朝|{
	事見三十二卷成帝綏和元年為于偽翻誤朝誤朝廷也朝直遥翻}
後白以官婢楊寄私子為皇子衆言曰呂氏少帝復出|{
	謂呂后名他人子為惠帝子也事見十三卷}
紛紛為天下所疑難以示來世成襁褓之功|{
	謂難成輔立幼主之功}
請遣立就國太后不聽莽曰今漢家衰比世無嗣|{
	比毗至翻}
太后獨代幼主統政誠可畏懼力用公正先天下尚恐不從|{
	師古曰力勉力先悉薦翻}
今以私恩逆大臣議如此羣下傾邪亂從此起宜可且遣就國安後復徵召之|{
	師古曰安猶徐也余謂安定也安後猶言事定後也}
太后不得已遣立就國莽之所以脅持上下皆此類也於是附順莽者拔擢忤恨者誅滅|{
	忤五故翻}
以王舜王邑為腹心甄豐甄邯主擊斷|{
	斷丁亂翻}
平晏領機事|{
	晏當之子也}
劉秀典文章孫建為爪牙豐子尋秀子棻|{
	師古曰棻音扶云翻}
涿郡崔發|{
	姓譜齊丁公之子食采於崔因以為氏}
南陽陳崇皆以材能幸於莽莽色厲而言方|{
	師古曰外示凛厲之色而假為方直之言}
欲有所為微見風采|{
	師古曰見音胡電翻}
黨與承其指意而顯奏之莽稽首涕泣固推讓|{
	稽音啓推吐雷翻}
上以惑太后下用示信於衆庶焉 八月莽復白太皇太后廢孝成皇后孝哀皇后為庶人就其園|{
	就孝成孝哀寢廟園也復扶又翻}
是日皆自殺 |{
	考異曰漢春秋八月甲寅未知胡旦所据}
大司空彭宣以王莽專權乃上書言三公鼎足承君一足不任則覆亂美實|{
	師古曰美實謂鼎中之實也易鼎卦九四爻辭曰鼎折足覆公餗餗食也故宣引以為言任音壬}
臣資性淺薄年齒老眊|{
	師古曰眊與耄同鄭玄曰耄惽忘也}
數伏疾病|{
	數所角翻}
昏亂遺忘|{
	忘巫放翻}
願上大司空長平侯印綬|{
	上時掌翻}
乞骸骨歸鄉里竢寘溝壑|{
	竢古俟字寘與填同}
莽白太后策免宣使就國莽恨宣求退故不賜黄金安車駟馬宣居國數年薨

班固贊曰薛廣德保縣車之榮平當逡巡有恥彭宣見險而止異乎苟患失之者矣|{
	薛廣德縣車在元帝永光元年平當不敢受封在建平三年通鑑因彭宣事以班贊繫之於此縣讀曰懸}


戊午右將軍王崇為大司空光禄勲東海馬宮為右將軍|{
	按班書馬宮木姓馬矢氏宮仕學稱馬氏}
左曹中郎將甄豐為光禄勲|{
	以中郎將加左曹官}
九月辛酉中山王即皇帝位|{
	貢父曰辛酉去哀帝崩六十四日}
大赦天下平帝年九歲太皇太后臨朝大司馬莽秉政百官總已以聽於莽|{
	援古者天子諒隂百官總已以聽於冢宰之制以盗權也師古曰聚束曰總音摠朱熹曰謂各總攝已職}
莽權日盛孔光憂懼不知所出上書乞骸骨莽白太后帝幼少宜置師傅徙光為帝太傅位四輔|{
	記曰虞夏商周有師保有疑丞設四輔及三公莽倣之以位置孔光變更官名自此始矣}
給事中領宿衛供養行内|{
	師古曰行内行在所之内中猶言禁中也}
署門戶省服御食物|{
	師古曰省視也余謂行内署門戶當為一句此宿衛事也省服御食物則供養事也文理甚明師古誤斷其句因曲為之說耳行下孟翻省悉井翻}
以馬宮為大司徒甄豐為右將軍 冬十月壬寅葬孝哀皇帝於義陵|{
	臣瓚曰自崩至葬凡百五日義陵在扶風去長安四十六里 考異曰哀紀云九月壬寅葬義陵按長歷是月辛酉朔無壬寅壬寅乃十月十二日又臣瓚注曰自崩至葬凡一百五日按帝以六月戊午崩然則葬在十月審矣蓋本紀月誤也}
孝平皇帝上|{
	荀悦曰諱衎之字曰樂應劭曰謚法布綱治紀曰平余按帝本名箕子元始二年始更名衎}


元始元年春正月王莽風益州令塞外蠻夷自稱越裳氏重譯獻白雉一黑雉二|{
	越裳注已見二十八卷元帝初元二年參考諸家之說越裳之地不在益州塞外莽自以輔幼主欲以致遠人功德比周公惑衆故為此耳師古曰越裳南方遠國也譯謂傳言也道路絶遠風俗殊隔故累譯而後乃通風讀曰諷}
莽白太后下詔以白雉薦宗廟於是羣臣盛陳莽功德致周成白雉之瑞周公及身在而託號於周|{
	時羣臣言聖王之法臣有大功則生有美號因引周公事為徵}
莽宜賜號曰安漢公益戶疇爵邑|{
	張晏曰漢律非始封十減二疇者等也言不復減賢曰疇等也言功臣子孫襲封與先人等余謂此言莽進號為公宜益其邑戶使與爵等也}
太后詔尚書具其事莽上書言臣與孔光王舜甄豐甄邯共定策今願獨條光等功賞寢置臣莽勿隨輩列|{
	進息也寢舍也}
甄邯白太后下詔曰無偏無黨王道蕩蕩|{
	師古曰尚書洪範之言也蕩蕩廣平之貌也故引之}
君有安宗廟之功不可以骨肉故蔽隱不揚君其勿辭莽復上書固讓數四|{
	復扶又翻下同}
稱疾不起左右白太后宜勿奪莽意但條孔光等莽乃肯起二月丙辰太后下詔以太傅博山侯光為太師車騎將軍安陽侯舜為太保皆益封萬戶 |{
	考異曰平紀作正月事而王子侯表公卿表皆云二月丙辰今從之 余按考異所謂王子侯云二月丙辰封者謂宣帝耳孫信等也由今考之不能無疑注見下}
左將軍光禄勲豐為少傅封廣陽侯|{
	恩澤侯表廣陽侯國於南陽}
皆授四輔之職侍中奉車都尉邯封承陽侯|{
	恩澤侯表承陽侯國於汝南師古曰承音烝}
四人既受賞莽尚未起羣臣復上言莽雖克讓朝所宜章|{
	言朝廷所當章顯也朝直遥翻}
以時加賞明重元功無使百僚元元失望太后乃下詔以大司馬新都侯莽為太傅幹四輔之事號曰安漢公益封二萬八千戶於是莽為惶恐不得已而起受太傅安漢公號讓還益封事云願須百姓家給|{
	師古曰給足也家給家家自足}
然後加賞羣臣復爭太后詔曰公自期百姓家給是以聽之其令公奉賜皆倍故|{
	奉讀曰俸所食之俸也賜歲時常賜著諸令者也師古曰倍故數多於故各一倍也}
百姓家給人足大司徒大司空以聞|{
	言待家給人足二府以其事聞也}
莽復讓不受而建言褒賞宗室羣臣立故東平王雲太子開明為王|{
	哀帝建平二年雲死國除今復立其子}
又以故東平思王孫成都為中山王奉孝王後|{
	東平思王宣帝子字也帝入奉大宗故立成都以奉孝王後}
封宣帝耳孫信等三十六人皆為列侯|{
	晉灼曰耳音仍 考異曰平紀元始元年封孝宣曾孫信等三十六人莽傳在五年按王子侯表皆以元年二月丙辰封莽傳誤也 余按王子侯表陶鄉侯恢等十五人皆以二月丙辰封不及三十六人之數又無信名按恢等皆宣帝曾孫也}
太僕王惲等二十五人皆賜爵關内侯|{
	惲等以前議定陶傅太后尊號守經法不阿指從邪賜爵惲於粉翻}
又令諸侯王公列侯關内侯無子而有孫若同產子者皆得以為嗣|{
	同產子同母兄弟之子}
宗室屬未盡而以罪絶者復其屬|{
	謂袒免以上親以罪絶屬藉者復其屬藉免音問}
天下吏比二千石以上年老致仕者參分故禄以一與之終其身|{
	師古曰參三也}
下及庶民鰥寡恩澤之政無所不施莽既媚說吏民又欲專斷|{
	說讀曰悦斷丁亂翻}
知太后老厭政乃風公卿奏言往者吏以功次遷至二千石|{
	功者以勞績遷次者以資序遷}
州部所舉茂材異等吏|{
	州部即部刺史也}
率多不稱|{
	稱尺證翻}
宜皆見安漢公又太后春秋高不宜親省小事|{
	省悉井翻}
令太后下詔曰自今以來唯封爵乃以聞它事安漢公四輔平決州牧二千石及茂材吏初除奏事者輒引入至近署對安漢公考故官問新職以知其稱否|{
	考故官者考其前任有勞績與否也問新職者問其新任當何如施設也稱尺證翻}
於是莽人人延問密致恩意厚加贈送其不合指顯奏免之權與人主侔矣 置羲和官秩二千石|{
	羲和初置自為一官莽既簒改大司農曰羲和}
夏五月丁巳朔日有食之大赦天下公卿以下舉敦厚能直言者各一人 王莽恐帝外家衛氏奪其權|{
	帝中山衛姬所生也}
白太后前哀帝立背恩義|{
	背蒲妹翻下同}
自貴外家丁傅撓亂國家|{
	師古曰撓擾也音火高翻}
幾危社稷|{
	幾居希翻}
今帝以幼年復奉大宗為成帝後宜明一統之義|{
	謂既奉大宗則以子繼父一以正統相承義不得顧私親}
以戒前事為後代灋六月遣甄豐奉璽綬即拜帝母衛姬為中山孝王后賜帝舅衛寶寶弟玄爵關内侯賜帝女弟三人號曰君|{
	謁臣號修義君皮為承禮君鬲子為尊德君師古曰鬲音歷}
皆留中山不得至京師扶風功曹申屠剛以直言對策曰臣聞成王幼少周公攝政聽言下賢均權布寵動順天地舉措不失然近則召公不說遠則四國流言|{
	賢曰尚書曰周公為師相成王為左右召公不悦言周公既還政成王宜自退今復為相故不悦也四國謂管蔡商奄也成王幼小周公攝政四國流言曰公將不利於孺子說讀曰悦}
今聖主始免襁褓即位以來至親分離外戚杜隔恩不得通且漢家之制雖任英賢猶援姻戚親疏相錯杜塞間隙|{
	援于元翻塞悉則翻間古莧翻}
誠所以安宗廟重社稷也宜亟遣使者徵中山太后置之别宮令時朝見|{
	朝直遥翻見賢遍翻}
又召馮衛二族裁與冗職|{
	賢曰冗散也}
使得執戟親奉宿衛以抑患禍之端上安社稷下全保傅|{
	此保傅謂四輔也}
莽令太后下詔曰剛所言僻經妄說違背大義|{
	背布内翻}
罷歸田里丙午封魯頃公之八世孫公子寛為褒魯侯|{
	魯頃公讐秦孝}


|{
	文王元年為楚所滅恩澤侯表褒魯侯食邑於南陽郡}
奉周公祀封褒成君孔霸曾孫均為褒成侯|{
	恩澤侯表褒成侯食邑於山陽瑕丘}
奉孔子祀 詔天下女徒已論歸家出雇山錢月三百|{
	如淳曰已論罪已定也令甲女子犯罪作女徒六月雇山遣歸說以為當於山伐木聽使入錢雇功直故謂之雇山應劭曰舊刑鬼薪取薪於山以給宗廟今使女徒出錢雇薪故曰雇山也師古曰如說近之謂女徒論罪已定並放歸家不親役之但令出錢月三百以雇人也為此恩者所以行太皇太后之德施惠於婦人}
復貞婦鄉一人|{
	師古曰復方目翻鄉一人取其尤最者}
大司農部丞十三人人部一州勸農桑|{
	武帝時桑弘羊置大司農部丞數十人分部郡國主均輸鹽鐵今以十三人部十三州}
秋九月赦天下徒

二年春黄支國獻犀牛黄支在南海中去京師三萬里|{
	應劭曰黄支國在日南之南}
王莽欲燿威德故厚遺其王|{
	遺于季翻}
令遣使貢獻 越巂郡上黄龍游江中|{
	上時掌翻}
太師光大司徒宮等咸稱莽功德比周公宜告祠宗廟大司農孫寶曰周公上聖召公大賢尚猶有不相說|{
	說讀曰悦}
著於經典兩不相損今風雨未時百姓不足每有一事羣臣同聲|{
	師古曰言雷同阿附妄說福祥}
得無非其美者|{
	謂所美非美也}
時大臣皆失色甄邯即時承制罷議者會寶遣吏迎母母道病留弟家獨遣妻子司直陳崇劾奏寶事下三公即訊|{
	師古曰就問之也劾戶槩翻下遐稼翻}
寶對曰年七十誖眊恩衰共養營妻子如章|{
	師古曰誖惑也眊與耄同自言老耄心志亂惑供養之恩衰具如所奏之章也誖音布内翻共讀曰供音居用翻}
寶坐免終於家 帝更名衎|{
	衎空旱翻又虛岸翻}
三月癸酉大司空王崇謝病免以避王莽 夏四月丁酉左將軍甄豐為大司空右將軍孫建為左將軍光禄勲甄邯為右將軍 立代孝王玄孫之子如意為廣宗王江都易王孫盱台侯宮為廣川王廣川惠王曾孫倫為廣德王|{
	代孝王參孫義改封清河傳國至孫年宣帝地節四年以罪廢今封如意以奉孝王後江都易王非傳國子建武帝元狩二年謀反自殺今立宮以奉易王後廣川惠王越宣帝地節四年以其孫文紹封傳子海陽甘露四年以罪廢今立倫以奉惠王後此皆王莽為政以繼絶世惑衆盱台音吁台}
紹封漢興以來大功臣之後周共等皆為列侯及關内侯|{
	共絳侯周勃玄孫師古曰共讀曰恭}
凡百一十七人 郡國大旱蝗青州尤甚|{
	青州部平原千乘濟南齊北海東萊等郡甾州膠東高密等王國}
民流亡王莽白太后宜衣繒練|{
	師古曰繒練謂帛無文者衣於既翻}
頗損膳以示天下莽因上書願出錢百萬獻田三十頃付大司農助給貧民於是公卿皆慕效焉凡獻田宅者二百三十人以口賦貧民|{
	師古曰計口而給其田宅}
又起五里於長安城中|{
	如淳曰民居之里}
宅二百區以居貧民莽帥羣臣奏太后|{
	帥讀曰率}
言幸賴陛下德澤閒者風雨時甘露降神芝生蓂莢朱草嘉禾休徵同時並至|{
	師古曰休美也徵證也}
願陛下遵帝王之常服復太官之灋膳使臣子各得盡驩心備共養|{
	共居用翻養羊尚翻}
莽又令太后下詔不許每有水旱莽輒素食|{
	師古曰素食即菜食無肉}
左右以白太后太后遣使者詔莽曰聞公菜食憂民深矣今秋幸孰|{
	孰古熟字通}
公以時食肉愛身為國|{
	為于偽翻}
六月隕石于鉅鹿二 光禄大夫楚國龔勝太中大夫琅邪邴漢|{
	邴姓也與丙同}
以王莽專政皆乞骸骨莽令太后策詔之曰朕愍以官職之事煩大夫大夫其修身守道以終高年皆加優禮而遣之梅福知王莽必簒漢祚一朝弃妻子去不知所之其

後人有見福於會稽者變姓名為吳市門卒云|{
	會稽郡時治吳縣會工外翻}
秋九月戊申晦日有食之赦天下徒 遣執金吾侯陳茂|{
	晉灼曰百官表執金吾屬官有兩丞侯司馬}
諭說江湖賊成重等二百餘人|{
	說輸芮翻}
皆自出送家在所收事|{
	如淳曰賊雖自出得還其家而已不得復除尚當役作之也師古曰如說非也言身既自出又各送其家人詣本屬縣邑從賦役耳貢父曰賊二百餘人皆異縣人既自出故送家在所收事也余謂劉說是}
重徙雲陽|{
	服䖍曰重成重也作賊長帥故徙之也}
賜公田宅 王莽欲悦太后以威德至盛異於前乃風單于令遣王昭君女須卜居次云入侍太后所以賞賜之甚厚 車師後王國有新道通玉門關|{
	車師後王國治務塗谷去長安八千九百五十里}
往來差近戊巳校尉徐普欲開之車師後王姑句|{
	師古曰句音鉤}
以當道供給使者心不便也普欲分明其界然後奏之召姑句使證之不肯繫之其妻股紫陬|{
	師古曰陬音子侯翻}
謂姑句曰前車師前王為都護司馬所殺|{
	此言日前事也車師前王治交河城去長安八千□百五十里}
今久繫必死不如降匈奴即馳突出高昌壁入匈奴|{
	拓拔魏時闞爽始立國於高昌蓋因漢高昌壁為名杜佑曰高昌郡蓋因其地高敞人物昌盛立名或云昔漢武帝遣兵西討師旅頓弊者因住焉有漢時高昌壘故也}
又去胡來王唐兜|{
	婼羌國王號去胡來王去陽關千八百里去長安六千三百里師古曰言去胡戍來附漢也婼孟康音兒師古曰音而遮翻}
與赤水羌數相寇|{
	羌之居赤水者大種也與婼羌比近唐有黑黨項居赤水西數所角翻}
不勝告急都護都護但欽不以時救助|{
	但欽人姓名姓書但平音或上}
唐兜困急怨欽東守玉門關玉門關不内即將妻子人民千餘人亡降匈奴|{
	降戶江翻下同}
單于受置左谷蠡地|{
	左谷蠡王所居地也谷音鹿蠡盧奚翻}
遣使上書言狀曰臣謹已受詔遣中郎將韓隆等使匈奴責讓單于單于叩頭謝罪執二虜還付使者|{
	二虜姑句及唐兜也}
詔使中郎將王萌待於西域惡都奴界上|{
	服䖍曰惡都奴西域之谷名也}
單于遣使送因請其罪|{
	為二虜請於漢求釋其背叛之罪也}
使者以聞莽不聽詔會西域諸國王陳軍斬姑句唐兜以示之|{
	欲以懲後使不敢叛}
乃造設四條|{
	師古曰更新為此制也}
中國人亡入匈奴者烏孫亡降匈奴者西域諸國佩中國印綬降匈奴者烏桓降匈奴者皆不得受遣中郎將王駿王昌副校尉甄阜王尋使匈奴班四條與單于雜函封|{
	師古曰與璽書同一函而封之}
付單于令奉行因收故宣帝所為約束封函還|{
	宣帝與匈奴約長城以南漢有之長城以北匈奴有之有降者不得受今莽以約束未明故頒四條而收舊所為約束}
時莽奏令中國不得有二名|{
	公羊春秋傳譏二名故莽效之}
因使使者以風單于宜上書慕化為一名漢必加厚賞單于從之上書言幸得備藩臣竊樂太平聖制臣故名囊知牙斯今謹更名曰知莽大說|{
	樂音洛更工衡翻說讀曰悦}
白太后遣使者答諭厚賞賜焉 莽欲以女配帝為皇后以固其權奏言皇帝即位三年長秋宮未建|{
	師古曰秋者收成之時長者恒久之義故以為皇后宮名}
掖庭媵未充|{
	媵以證翻從嫁之女也古者諸侯一國嫁女九國媵之}
乃者國家之難本從無嗣配取不正請考論五經定取后禮|{
	難乃旦翻師古曰取皆讀曰娶}
正十二女之義以廣繼嗣|{
	古者天子一取九女公羊傳曰諸侯一聘九女則周之天子固有十二女之禮莽之進女也十一媵蓋通后為十二女也}
博采二王後及周公孔子世列侯在長安者適子女|{
	二王後殷周之後周公孔子世周公之後世嫡相承者師古曰適讀曰嫡嫡謂妻所生也}
事下有司上衆女名|{
	下遐稼翻上時掌翻下同}
王氏女多在選中者莽恐其與巳女爭即上言身無德子材下不宜與衆女並采太后以為至誠乃下詔曰王氏女朕之外家其勿采庶民諸生郎吏以上守闕上書者日千餘人公卿大夫或詣廷中或伏省戶下咸言安漢公盛勲堂堂若此今當立后獨奈何廢公女天下安所歸命願得公女為天下母莽遣長史以下分部曉止公卿及諸生|{
	師古曰分音扶問翻曉止開諭之使止也}
而上書者愈甚太后不得已聽公卿采莽女莽復自白宜博采衆女公卿爭曰不宜采諸女以貳正統|{
	師古曰言皇后之位當在莽女也}
莽乃白願見女

資治通鑑卷三十五
