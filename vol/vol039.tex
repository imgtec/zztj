<!DOCTYPE html PUBLIC "-//W3C//DTD XHTML 1.0 Transitional//EN" "http://www.w3.org/TR/xhtml1/DTD/xhtml1-transitional.dtd">
<html xmlns="http://www.w3.org/1999/xhtml">
<head>
<meta http-equiv="Content-Type" content="text/html; charset=utf-8" />
<meta http-equiv="X-UA-Compatible" content="IE=Edge,chrome=1">
<title>資治通鑒_40-資治通鑑卷三十九_40-資治通鑑卷三十九</title>
<meta name="Keywords" content="資治通鑒_40-資治通鑑卷三十九_40-資治通鑑卷三十九">
<meta name="Description" content="資治通鑒_40-資治通鑑卷三十九_40-資治通鑑卷三十九">
<meta http-equiv="Cache-Control" content="no-transform" />
<meta http-equiv="Cache-Control" content="no-siteapp" />
<link href="/img/style.css" rel="stylesheet" type="text/css" />
<script src="/img/m.js?2020"></script> 
</head>
<body>
 <div class="ClassNavi">
<a  href="/24shi/">二十四史</a> | <a href="/SiKuQuanShu/">四库全书</a> | <a href="http://www.guoxuedashi.com/gjtsjc/"><font  color="#FF0000">古今图书集成</font></a> | <a href="/renwu/">历史人物</a> | <a href="/ShuoWenJieZi/"><font  color="#FF0000">说文解字</a></font> | <a href="/chengyu/">成语词典</a> | <a  target="_blank"  href="http://www.guoxuedashi.com/jgwhj/"><font  color="#FF0000">甲骨文合集</font></a> | <a href="/yzjwjc/"><font  color="#FF0000">殷周金文集成</font></a> | <a href="/xiangxingzi/"><font color="#0000FF">象形字典</font></a> | <a href="/13jing/"><font  color="#FF0000">十三经索引</font></a> | <a href="/zixing/"><font  color="#FF0000">字体转换器</font></a> | <a href="/zidian/xz/"><font color="#0000FF">篆书识别</font></a> | <a href="/jinfanyi/">近义反义词</a> | <a href="/duilian/">对联大全</a> | <a href="/jiapu/"><font  color="#0000FF">家谱族谱查询</font></a> | <a href="http://www.guoxuemi.com/hafo/" target="_blank" ><font color="#FF0000">哈佛古籍</font></a> 
</div>

 <!-- 头部导航开始 -->
<div class="w1180 head clearfix">
  <div class="head_logo l"><a title="国学大师官网" href="http://www.guoxuedashi.com" target="_blank"></a></div>
  <div class="head_sr l">
  <div id="head1">
  
  <a href="http://www.guoxuedashi.com/zidian/bujian/" target="_blank" ><img src="http://www.guoxuedashi.com/img/top1.gif" width="88" height="60" border="0" title="部件查字,支持20万汉字"></a>


<a href="http://www.guoxuedashi.com/help/yingpan.php" target="_blank"><img src="http://www.guoxuedashi.com/img/top230.gif" width="600" height="62" border="0" ></a>


  </div>
  <div id="head3"><a href="javascript:" onClick="javascript:window.external.AddFavorite(window.location.href,document.title);">添加收藏</a>
  <br><a href="/help/setie.php">搜索引擎</a>
  <br><a href="/help/zanzhu.php">赞助本站</a></div>
  <div id="head2">
 <a href="http://www.guoxuemi.com/" target="_blank"><img src="http://www.guoxuedashi.com/img/guoxuemi.gif" width="95" height="62" border="0" style="margin-left:2px;" title="国学迷"></a>
  

  </div>
</div>
  <div class="clear"></div>
  <div class="head_nav">
  <p><a href="/">首页</a> | <a href="/ShuKu/">国学书库</a> | <a href="/guji/">影印古籍</a> | <a href="/shici/">诗词宝典</a> | <a   href="/SiKuQuanShu/gxjx.php">精选</a> <b>|</b> <a href="/zidian/">汉语字典</a> | <a href="/hydcd/">汉语词典</a> | <a href="http://www.guoxuedashi.com/zidian/bujian/"><font  color="#CC0066">部件查字</font></a> | <a href="http://www.sfds.cn/"><font  color="#CC0066">书法大师</font></a> | <a href="/jgwhj/">甲骨文</a> <b>|</b> <a href="/b/4/"><font  color="#CC0066">解密</font></a> | <a href="/renwu/">历史人物</a> | <a href="/diangu/">历史典故</a> | <a href="/xingshi/">姓氏</a> | <a href="/minzu/">民族</a> <b>|</b> <a href="/mz/"><font  color="#CC0066">世界名著</font></a> | <a href="/download/">软件下载</a>
</p>
<p><a href="/b/"><font  color="#CC0066">历史</font></a> | <a href="http://skqs.guoxuedashi.com/" target="_blank">四库全书</a> |  <a href="http://www.guoxuedashi.com/search/" target="_blank"><font  color="#CC0066">全文检索</font></a> | <a href="http://www.guoxuedashi.com/shumu/">古籍书目</a> | <a   href="/24shi/">正史</a> <b>|</b> <a href="/chengyu/">成语词典</a> | <a href="/kangxi/" title="康熙字典">康熙字典</a> | <a href="/ShuoWenJieZi/">说文解字</a> | <a href="/zixing/yanbian/">字形演变</a> | <a href="/yzjwjc/">金 文</a> <b>|</b>  <a href="/shijian/nian-hao/">年号</a> | <a href="/diming/">历史地名</a> | <a href="/shijian/">历史事件</a> | <a href="/guanzhi/">官职</a> | <a href="/lishi/">知识</a> <b>|</b> <a href="/zhongyi/">中医中药</a> | <a href="http://www.guoxuedashi.com/forum/">留言反馈</a>
</p>
  </div>
</div>
<!-- 头部导航END --> 
<!-- 内容区开始 --> 
<div class="w1180 clearfix">
  <div class="info l">
   
<div class="clearfix" style="background:#f5faff;">
<script src='http://www.guoxuedashi.com/img/headersou.js'></script>

</div>
  <div class="info_tree"><a href="http://www.guoxuedashi.com">首页</a> > <a href="/SiKuQuanShu/fanti/">四库全书</a>
 > <h1>资治通鉴</h1> <!--         下载:【右键另存为】即可 --></div>
  <div class="info_content zj clearfix">
  
<div class="info_txt clearfix" id="show">
<center style="font-size:24px;">40-資治通鑑卷三十九</center>
    資治通鑑卷三十九   宋 司馬光 撰<br />
<br />
  胡三省 音註<br />
<br />
  漢紀三十一【起昭陽協洽盡閼逢涒灘凡二年】<br />
<br />
  淮陽王【諱玄字聖公光武族兄也帝王世紀曰舂陵戴侯熊渠生蒼梧太守利利生子張子張生玄後敗降赤眉光武詔封為淮陽王】<br />
<br />
  更始元年【更工衡翻是年二月即位改元】春正月甲子朔漢兵與下江兵共攻甄阜梁丘賜斬之【甄之人翻】殺士卒二萬餘人王莽納言將軍嚴尤秩宗將軍陳茂引兵欲據宛【宛於元翻】劉縯與戰于淯陽下【續漢志淯陽縣屬南陽郡賢曰故城在今鄧州南陽縣南淯水之陽因名淯音育】大破之遂圍宛先是青徐賊衆雖數十萬人訖無文書號令旌旗部曲【師古曰文謂文章號謂大位號也一曰號謂號令也先悉薦翻】及漢兵起皆稱將軍攻城略地移書稱說【稱說者數莽之罪也】莽聞之始懼舂陵戴侯曾孫玄在平林兵中號更始將軍【更工衡翻】時漢兵已十餘萬諸將議以兵多而無所統一欲立劉氏以從人望南陽豪桀及王常等皆欲立劉縯而新市平林將帥樂放縱【樂音洛】憚縯威明貪玄懦弱先共定策立之然後召縯示其議縯曰諸將軍幸欲尊立宗室甚厚然今赤眉起青徐衆數十萬聞南陽立宗室恐赤眉復有所立【其後赤眉果立盆子復扶又翻】王莽未滅而宗室相攻是疑天下而自損權【言宗室爭立則天下莫知所從是疑天下之心而自損其權也】非所以破莽也舂陵去宛三百里耳遽自尊立為天下凖的使後人得承吾敝非計之善者也不如且稱王以號令王勢亦足以斬諸將若赤眉所立者賢相率而往從之必不奪吾爵位若無所立破莽降赤眉【降戶江翻】然後舉尊號亦未晩也諸將多曰善張卬拔劒擊地 【考異曰司馬彪續漢書卬作印袁宏後漢紀作斤皆誤今從范曄後漢書】曰疑事無功【戰國策肥義對趙武靈王之言】今日之議不得有二衆皆從之二月辛巳朔設壇場於淯水上沙中【水經注淯水出弘農盧氏縣攻離山東南過南陽西鄂縣西北又東過宛縣南諸將立聖公于斯水之上】玄即皇帝位南面立朝羣臣【朝直遥翻】羞愧流汗舉手不能言于是大赦改元以族父良為國三老王匡為定國上公王鳳為成國上公朱鮪為大司馬劉縯為大司徒陳牧為大司空餘皆九卿將軍【匡鳳皆位上公而加定國成國美號也九卿將軍職為九卿各帶將軍之號仍王莽之制也】由是豪桀失望多不服【豪桀欲立縯而今立玄故失望】 王莽欲外示自安乃染其須髮立杜陵史諶女為皇后【諶時壬翻】置後宫位號視公卿大夫元士者凡百二十人【三夫人視三公九嬪視九卿二十七世婦視二十七大夫八十一御妻視八十一元士】莽赦天下詔王匡哀章等討青徐盜賊嚴尤陳茂等<br />
<br />
  討前隊醜虜明告以生活丹青之信【師古曰生活謂來降者不殺之也丹青之信言明著也】復迷惑不解散【復扶又翻】將遣大司空隆新公將百萬之師劋絶之矣【師古曰劋絶也音子小翻司空隆新公王邑】 三月王鳳與大常偏將軍劉秀等狥昆陽定陵郾皆下之【賢曰昆陽定陵郾皆縣名並屬潁川郡昆陽故城在今許州葉縣北二十五里郾今豫州郾城縣也定陵在今郾城西北郾音於建翻予按舊唐書高宗咸亨二年冬校獵於許州葉縣昆水之陽】 王莽聞嚴尤陳茂敗乃遣司空王邑馳傳【傳知戀翻】與司徒王尋發兵平定山東徵諸明兵法六十三家以備軍吏以長人巨母霸為壘尉【鄭玄曰軍壁曰壘賢曰壘尉主壁壘之事】又驅諸猛獸虎豹犀象之屬以助威武邑至洛陽州郡各選精兵牧守自將【守式又翻將即亮翻】定會者四十三萬人號百萬餘在道者旌旗輜重千里不絶【賢曰周禮曰析羽為旌熊虎為旗輜車名釋名曰輜厠也謂軍粮什物雜厠載之以其累重故稱輜重重音直用翻】夏五月尋邑南出潁川與嚴尤陳茂合諸將見尋邑兵盛皆反走入昆陽惶怖憂念妻孥【賢曰孥子也音奴】欲散歸諸城劉秀曰今兵穀既少【少詩沼翻】而外寇強大并力禦之功庶可立如欲分散埶無俱全且宛城未拔【賢曰謂伯升圍宛未拔】不能相救昆陽即拔一日之間諸部亦滅矣今不同心膽共舉功名反欲守妻子財物邪諸將怒曰劉將軍何敢如是秀笑而起會候騎還言大兵且至城北軍陳數百里不見其後【陳讀曰陣】諸將素輕秀及迫急乃相謂曰更請劉將軍計之秀復為圖畫成敗【復扶又翻為于偽翻】諸將皆曰諾時城中唯有八九千人秀使王鳳與廷尉大將軍王常守昆陽夜與五威將軍李軼等十三騎【賢曰王莽置五威將軍其衣服依五方之色以威天下李軼初起猶假以為號予謂如太常偏將軍廷尉大將軍之類亦猶莽之納言大將軍秩宗大將軍是即前所云九卿將軍也】出城南門于外收兵時莽兵到城下者且十萬秀等幾不得出【幾居希翻】尋邑縱兵圍昆陽嚴尤說邑曰【說輸芮翻】昆陽城小而堅今假號者在宛【假號者謂更始也】亟進大兵彼必犇走宛敗昆陽自服邑曰吾㫺圍翟義坐不生得【翟義事見三十六卷王莽居攝二年賢曰坐才臥翻】以見責讓今將百萬之衆遇城而不能下非所以示威也當先屠此城蹀血而進【師古曰蹀音大頰翻蹀謂履涉之也】前歌後舞顧不快邪遂圍之數十重列營百數鉦鼓之聲聞數十里【重直龍翻聞音問】或為地道衝輣撞城【賢曰衝撞車也詩曰臨衝閑閑許慎曰輣樓車也輣音步耕翻撞丈江翻】積弩亂發矢下如雨城中負戶而汲王鳳等乞降不許【降戶江翻】尋邑自以為功在漏刻不以軍事為憂嚴尤曰兵法圍城為之闕【師古曰此兵法之言也闕不合也孫子曰圍師必闕曹操注云司馬法云圍其三面闕其一面所以示生路也】宜使得逸出以怖宛下【怖普布翻】邑又不聽 棘陽守長岑彭【姓譜岑古岑子國之後呂氏春秋周文王封異母弟耀之子渠為岑子其地梁國岑亭是也彭棘陽人守本縣長長知兩翻】與前隊貳嚴說【貳副也莽使說為前隊大夫甄阜之副也】共守宛城漢兵攻之數月城中人相食乃舉城降更始入都之諸將欲殺彭劉縯曰彭郡之大吏執心固守是其節也今舉大事當表義士不如封之更始乃封彭為歸德侯【賢曰歸德縣名屬北地郡宋白曰慶州華池縣本漢歸德縣地又通遠軍西北有歸德川】 劉秀至郾定陵悉發諸營兵諸將貪惜財物欲分兵守之秀曰今若破敵珍寶萬倍大功可成如為所敗【敗補邁翻】首領無餘何財物之有乃悉發之六月己卯朔秀與諸營俱進自將步騎千餘為前鋒去大軍四五里而陳【陳讀曰陣】尋邑亦遣兵數千合戰秀奔之斬首數十級【賢曰秦法斬首一賜爵一級因謂斬首為級】諸將喜曰劉將軍平生見小敵怯今見大敵勇甚可怪也且復居前請助將軍秀復進尋邑兵却諸部共乘之斬首數百千級【自數百級以至千級也復扶又翻】連勝遂前諸將膽氣益壯無不一當百秀乃與敢死者三千人從城西水上衝其中堅【賢曰敢死謂果敢而死者凡軍事中軍將軍至尊以堅銳自輔故曰中堅也予謂敢死者敢於致死者也】尋邑易之【易以豉翻】自將萬餘人行陳【帥古曰廵行軍陳也行音卜更翻陳讀曰陣下同】敕諸營皆按部毋得動獨迎與漢兵戰不利大軍不敢擅相救尋邑陳亂漢兵乘銳奔之遂殺王尋城中亦鼓譟而出中外合埶震呼動天地【呼火故翻】莽兵大潰走者相騰踐伏尸百餘里會天雷風屋瓦皆飛雨下如注滍川盛溢【水經曰滍水出南陽魯陽縣西堯山東南經昆陽城北東入汝滍音直理翻】虎豹皆股戰士卒赴水溺死者以萬數【賢曰數過於萬故以萬為數】水為不流【為于偽翻】王邑嚴尤陳茂輕騎乘死人度水逃去盡獲其軍實輜重不可勝筭【重直用翻勝音升】舉之連月不盡或燔燒其餘士卒奔走各還其郡王邑獨與所將長安勇敢數千人還洛陽關中聞之震恐於是海内豪桀翕然響應【響應若響之應聲也】皆殺其牧守自稱將軍用漢年號以待詔命旬月之間徧于天下 莽聞漢兵言莽鴆殺孝平皇帝乃會公卿於王路堂開所為平帝請命金縢之策【事見三十六卷平帝元始六年為于偽翻】泣以示羣臣 劉秀復狥潁川攻父城不下【賢曰父城縣名故城在今許州葉縣東北汝州郟城縣亦有父城復扶又翻下同】屯兵巾車鄉【賢曰巾車鄉名也在父城界】潁川郡掾馮異監五縣【掾俞絹翻監古衘翻】為漢兵所獲異曰異有老母在父城願歸據五城以效功報德秀許之異歸謂父城長苗萌曰諸將多暴横【長知兩翻横戶孟翻】獨劉將軍所到不虜略觀其言語舉止非庸人也遂與萌率五縣以降【降戶江翻】 新市平林諸將以劉縯兄弟威名益盛陰勸更始除之秀謂縯曰事欲不善【言更始欲相圖也】縯笑曰常如是耳更始大會諸將取縯寶劒視之繡衣御史申徒建隨獻玉玦【申徒即申屠賢曰玦决也令早决斷】更始不敢發縯舅樊宏謂縯曰建得無有范增之意乎【范增事見九卷高帝元年】縯不應李軼初與縯兄弟善後更諂事新貴【新貴謂朱鮪等】秀戒縯曰此人不可復信【復扶又翻】縯不從縯部將劉稷勇冠三軍【冠古玩翻】聞更始立怒曰本起兵圖大事者伯升兄弟也今更始何為者邪更始以稷為抗威將軍稷不肯拜【不肯拜受抗威之命也】更始乃與諸將陳兵數千人先收稷將誅之縯固爭李軼朱鮪因勸更始并執縯即日殺之以族兄光禄勲賜為大司徒【賜與更始同祖蒼梧太守利】秀聞之自父城馳詣宛謝【賢曰以伯升見害心不自安故謝】司徒官屬迎弔秀【縯之官屬】也秀不與交私語【遠嫌也】惟深引過而已【引過以歸己】未嘗自伐昆陽之功又不敢為縯服喪【為于偽翻】飲食言笑如平常更始以是慙拜秀為破虜大將軍封武信侯 道士西門君惠謂王莽衛將軍王涉曰讖文劉氏當復興【復扶又翻下同】國師公姓名是也涉遂與國師公劉秀大司馬董忠司中大贅孫伋謀以所部兵劫莽降漢以全宗族【涉欲全王氏之族也降戶江翻】秋七月伋以其謀告莽莽召忠詰責因格殺之使虎賁以斬馬劒剉忠收其宗族以醇醯毒藥白刃樷棘并一坎而埋之秀涉皆自殺莽以其骨肉舊臣惡其内潰故隱其誅【師古曰王涉骨肉劉歆舊臣予按莽傳涉曲陽侯根子也惡烏路翻】莽以軍師外破大臣内畔左右亡所信【古亡無字通】不能復遠念郡國乃召王邑還為大司馬以大長秋張邯為大司徒崔發為大司空司中夀容苗訢為國師【邯下甘翻訢音欣】莽憂懣不能食【懣音悶又音滿】但飲酒㗖鰒魚【師古曰此鰒海魚也音雹郭璞注三蒼曰鰒似蛤偏著石廣志曰鰒無鱗有殼一面附石細孔雜雜或七或九本草曰石决明一名鰒魚】讀軍書倦因馮几寐【師古曰馮讀曰憑】不復就枕矣 成紀隗崔隗義【成紀縣屬天水郡賢曰故城在今秦州隴城縣西北隗姓出於赤狄】上邽楊廣冀人周宗【上邽縣屬隴西郡賢曰故邽戎邑在秦州縣冀縣屬天水郡秦武公伐冀戎因縣之宋白曰秦州治隴城縣即故冀城】同起兵以應漢攻平襄殺莽鎮戎大尹李育【賢曰平襄縣名屬天水郡故城在今秦州伏羌縣西北王莽改天水郡曰鎮戎】崔兄子囂素有名好經書【好呼到翻】崔等共推為上將軍崔為白虎將軍義為左將軍【崔本自署右將軍白虎居右又起兵于西方白虎主之因改右將軍號白虎將軍】囂遣使聘平陵方望以為軍師【賢曰平陵昭帝陵因以為縣故城在今咸陽縣西北武王伐紂以太公為師尚父田單守即墨以一卒為神師韓信既破趙師事李左車皆軍師也後遂以為官稱】望說囂立高廟于邑東【平襄邑之東也說輸芮翻】己巳祠高祖太宗世宗囂等皆稱臣執事殺馬同盟以興輔劉宗移檄郡國數莽罪惡【數所具翻】勒兵十萬擊殺雍州牧陳慶安定大尹王向【莽改漢凉州曰雍州向平阿侯王譚之子也 考異曰王莽傳作卒正王旬袁紀作太守王向今從范書】分遣諸將狥隴西武都金城武威張掖酒泉燉煌【燉徒門翻】皆下之 初茂陵公孫述為清水長【賢曰清水縣名屬天水郡今秦州縣】有能名遷導江卒正治臨卭【賢曰王莽改蜀郡曰導江臨卭今卭州縣班志臨卭縣屬蜀郡卭音渠容翻】漢兵起南陽宗成商人王岑起兵狥漢中以應漢【宗成南陽人也地理志商縣屬弘農郡賢曰今商州商雒縣也】殺王莽庸部牧宋遵衆合數萬人述遣使迎成等成等至成都【地理志成都縣屬蜀郡】虜掠暴横【横戶孟翻】述召郡中豪桀謂曰天下同苦新室思劉氏久矣故聞漢將軍到馳迎道路今百姓無辜而婦子係獲此寇賊非義兵也乃使人詐稱漢使者假述輔漢將軍蜀郡太守兼益州牧印綬選精兵西擊成等殺之【按臨卭在成都西南述兵自臨卭迎擊宗成等非西向也此承范史之誤】并其衆前鍾武侯劉望起兵汝南【按王子侯表鍾武節侯度長沙定王之孫成帝元延】<br />
<br />
  【二年侯則紹封其後不見或者望其則之子歟鍾武在義陽郡界水經注師水過義陽郡城東逕鍾武故城南考異曰王莽傳作劉聖今從范書劉玄傳】嚴尤陳茂往歸之八月望即皇<br />
<br />
  帝位以尤為大司馬茂為丞相 王莽使太師王匡國將哀章守洛陽【將即亮翻 考異曰袁紀作襃章今從范書】更始遣定國上公王匡攻洛陽西屏大將軍申屠建丞相司直李松攻武關【李松通之從弟屏必郢翻】三輔震動析人鄧曄于匡起兵南鄉以應漢【師古曰析南陽之縣南鄉析縣之鄉名也宋白曰鄧州内郷縣古之析邑析音先歷翻】攻武關都尉朱萌萌降【降戶江翻】進攻右隊大夫宋綱殺之【莽改弘農郡曰右隊】西拔湖【師古曰湖弘農之縣也本屬京兆】莽愈憂不知所出崔發言古者國有大災則哭以厭之【師古曰周禮春官之屬女巫之職曰凡邦之大災歌哭以請哭者所以告哀也春秋左氏傳宣十二年楚子圍鄭旬有七日鄭人大臨守陴者皆哭故發引之以為言也厭音一葉翻】宜告天以求救莽乃率羣臣至南郊陳其符命本末仰天大哭氣盡伏而叩頭諸生小民旦夕會哭為設飱粥【為于偽翻師古曰飱音千安翻】甚悲哀者除以為郎郎至五千餘人莽拜將軍九人皆以虎為號將北軍精兵數萬人以東内其妻子宫中以為質【質音致】時省中黄金尚六十餘萬斤它財物稱是【稱尺證翻】莽愈愛之賜九虎士人四千錢衆重怨無鬬意【師古曰重音直用翻】九虎至華陰回谿【賢曰回谿今俗所謂囘阬在洛州永寧縣東北其谿長四里關二丈深二丈五尺華戶化翻】距隘自守于匡鄧曄擊之六虎敗走二虎詣闕歸死莽使使責死者安在皆自殺其四虎亡【二虎自殺者史熊王况也四虎亡者史逸其名】三虎收散卒保渭口京師倉【三虎郭欽陳翬成重也師古曰京師倉在華隂灌北渭口也】鄧曄開武關迎漢兵李松將三千餘人至湖與曄等共攻京師倉未下曄以弘農掾王憲為校尉【掾俞絹翻】將數百人北度渭入左馮翊界李松遣偏將軍韓臣等徑西至新豐擊莽波水將軍【據竇融傳莽拜融為波水將軍前書音義曰波水在長安南】追奔至長門宫王憲北至頻陽所過迎降【師古曰所過之處人皆來迎而降附也】諸縣大姓各起兵稱漢將軍率衆隨憲李松鄧曄引軍至華隂而長安旁兵四會城下又聞天水隗氏方到皆爭欲入城貪立大功鹵掠之利【言入城誅莽既立大功又得鹵掠貪二者之利也】莽赦城中囚徒皆授兵殺豨飲其血與誓曰有不為新室者社鬼記之【豨許豈翻又音希為于偽翻】使更始將軍史諶將之【諶氏壬翻】度渭橋皆散走諶空還衆兵發掘莽妻子父祖冢燒其棺椁及九廟明堂辟雍火照城中九月戊申朔兵從宣平城門入【師古曰長安城東出北頭第一門】張邯逢兵見殺王邑王林王廵䠠惲等分將兵距擊北闕下【䠠師古曰音帶又音徒盖翻䠠姓惲名惲於粉翻】會日暮官府邸第盡奔亡己酉城中少年朱弟張魚等恐見鹵掠趨讙並和【師古曰衆羣行讙而自相和也讙許元翻和音呼臥翻】燒作室門【程大昌曰作室者未央宫西北織室暴室之類黄圖謂為尚方工作之所者也作室門則工徒出入之門盖未央宫之便門也】斧敬法闥【師古曰敬法殿名也闥小門也謂斧斫之也】呼曰反虜王莽何不出降【師古曰呼音少故翻】火及掖庭承明黄皇室主所居黄皇室主曰何面目以見漢家自投火中而死莽避火宣室前殿火輒隨之莽紺袀服【師古曰紺深青而揚赤色也袀純也純為紺服也袀音均又音弋句翻】持虞帝匕首【虞帝安得有匕首蓋莽自為之以愚人】天文郎按式於前【師古曰式所以占時日天文郎今之用式者也】莽旋席隨斗柄而坐曰天生德於予漢兵其如予何【莽引孔子之言以自况】庚戌旦明羣臣扶掖莽自前殿之漸臺【此未央宫之漸臺也水經未央漸臺在滄池中建章漸臺在太液池中程大昌曰漸者漬也言臺在水中受其漸漬也凡臺之環浸于水者皆可名為漸臺漸子廉翻】公卿從官尚千餘人隨之【從才用翻】王邑晝夜戰罷極【師古曰罷讀曰疲】士死傷略盡馳入宫間關至漸臺【師古曰間關猶言崎嶇展轉也】見其子侍中睦解衣冠欲逃邑叱之令還父子共守莽軍人入殿中聞莽在漸臺衆共圍之數百重【重直龍翻】臺上猶與相射【射而亦翻】矢盡短兵接王邑父子䠠惲王巡戰死莽入室下餔時衆兵上臺【晡後謂之下晡按前書天文志旦至食時食時至日昳日昳至晡晡至下晡下晡至日入】苖訢唐尊王盛等皆死【訢音欣】商人杜吳殺莽校尉東海公賓就斬莽首【師古曰公賓姓就名也風俗通曰魯大夫公賓庚之後王莽五十一居攝五十四即真六十八誅死】軍人分莽身節解臠分爭相殺者數十人公賓就持莽首詣王憲憲自稱漢大將軍城中兵數十萬皆屬焉舍東宫【師古曰舍止宿也】妻莽後宫乘其車服癸丑李松鄧曄入長安將軍趙萌申屠建亦至以王憲得璽綬不上【璽斯氏翻綬音受上時掌翻】多挾宫女建天子鼓旗收斬之傳莽首詣宛縣于市百姓共提擊之【縣讀曰懸提音徒計翻】或切食其舌班固贊曰王莽始起外戚折節力行以要名譽【折而設翻要一遥翻】及居位輔政勤勞國家直道而行豈所謂色取仁而行違者邪【師古曰論語載孔子答子張言也不仁之人假仁者之色而行則違之行下孟翻】莽既不仁而有佞邪之材又乘四父歷世之權【四父謂王鳳王音王商王根相繼秉政皆莽諸父也】遭漢中微國統三絶【成哀平皆絶】而太后夀考為之宗主故得肆其姦慝以成簒盜之禍推是言之亦天時非人力之致矣及其竊位南面顛覆之埶險於桀紂而莽晏然自以黄虞復出也【黄帝虞舜莽祖之復扶又翻】乃始恣睢奮其威詐【師古曰睢音呼季翻】毒流諸夏亂延蠻貉猶未足以逞其欲焉是以四海之内囂然喪其樂生之心【師古曰囂然衆口愁貌也音五高翻喪息浪翻】中外憤怨遠近俱發城池不守支體分裂遂令天下城邑為虚【虚讀曰墟】害徧生民自書傳所載亂臣賊子考其禍敗未有如莽之甚者也㫺秦燔詩書以立私議莽誦六藝以文姦言【師古曰以六經之事文飾奸言】同歸殊塗俱用滅亡皆聖王之驅除云爾【蘇林曰聖王光武也為光武驅除也師古曰言驅逐蠲除以待聖王也】<br />
<br />
  定國上公王匡拔洛陽生縛莽太師王匡哀章皆斬之冬十月奮威大將軍劉信擊殺劉望於汝南【信大司徒賜兄顯之子】并誅嚴尤陳茂郡縣皆降【降音戶江翻】 更始將都洛陽以劉秀行司隸校尉使前整修宫府【司隸校尉察三輔三河弘農故使整修宫府】秀乃置僚屬作文移【東觀記曰文書移與屬縣也】從事司察一如舊章【續漢書司隸置從事史十二人秩皆百石主督促文書察舉非法】時三輔吏士東迎更始見諸將過皆冠幘而服婦人衣【漢官儀曰幘者古之卑賤不冠者之所服也方言曰覆髻謂之幘或謂之承露劉昭志曰秦雄諸侯乃加武將首飾為絳袙以表貴賤其後稍作顔題漢興續其顔却摞之施巾連題却覆之名之曰幘幘者賾也頭首嚴賾也至孝文乃高其題崇其巾為屋合後施收上下羣臣貴賤皆服之文者長耳武者短耳】莫不笑之及見司隸僚屬皆歡喜不自勝【勝音升】老吏或垂涕曰不圖今日復見漢官威儀【復音扶又翻下同】由是識者皆屬心焉【皆屬音之欲翻】更始北都洛陽分遣使者徇郡國曰先降者復爵位【降音戶江翻下同】使者至上谷【漢上谷郡治沮陽】上谷太守扶風耿况迎上印綬【上音時掌翻】使者納之一宿無還意功曹寇恂勒兵入見使者請之【姓譜蘇忿生為周武王司寇其後以官為寇氏百官志郡功曹主選署功勞在諸曹之上】使者不與曰天王使者功曹欲脅之邪恂曰非敢脅使君竊傷計之不詳也今天下初定使君建節䘖命郡國莫不延頸傾耳今始至上谷而先墮大信【賢曰墮毁也讀曰隳】將復何以號令它郡乎使者不應恂叱左右以使者命召况况至恂進取印綬帶况使者不得已乃承制詔之况受而歸宛人彭寵吳漢亡命在漁陽鄉人韓鴻為更始使徇北州承制拜寵偏將軍行漁陽太守事【為彭寵據漁陽張本】以漌為安樂令【賢曰安樂縣名屬漁陽郡故城在今幽州潞縣西北樂音洛】更始遣使降赤眉【遣使者招諭之使降而釋兵也後以意推降戶江翻】樊崇等聞漢室復興即留其兵將渠帥二十餘人隨使者至洛陽【帥所類翻】更始皆封為列侯崇等既未有國邑而留衆稍有離叛者乃復亡歸其營【崇等時營在濮陽為赤眉攻更始張本】 王莽廬江連率潁川李憲據郡自守稱淮南王【率所類翻】 故梁王立之子永詣洛陽【立死見三十六卷平帝元始四年】更始封為梁王都睢陽【為永據梁連羣盜張本睢音雖】 更始欲令親近大將徇河北大司徒賜言諸家子獨有文叔可用【諸家子謂南陽諸宗子也光武諱秀字文叔】朱鮪等以為不可更始狐疑賜深勸之更始乃以劉秀行大司馬事持節北度河鎮慰州郡【為光武自河北定天下張本】 以大司徒賜為丞相令先入關修宗廟宫室【將都長安也】 大司馬秀至河北所過郡縣考察官吏黜陟能否平遣囚徒除王莽苛政【賢曰說文苛小艸也言政令繁細】復漢官名吏民喜悦爭持牛酒迎勞【勞力到翻】秀皆不受南陽鄧禹杖策追秀及於鄴秀曰我得專封拜生遠來寧欲仕乎禹曰不願也秀曰即如是何欲為禹曰但願明公威德加于四海禹得効其尺寸垂功名于竹帛耳【漢初未有紙以竹簡及縑素書故言竹帛】秀笑因留宿間語【賢曰間私也】禹進說曰【說輸芮翻下同】今山東未安赤眉青犢之屬動以萬數更始既是常才而不自聽斷【斷丁亂翻】諸將皆庸人屈起【賢曰屈音求勿翻】志在財幣爭用威力朝夕自快而已非有忠良明智深慮遠圖欲尊主安民者也歷觀往古聖人之興二科而已天時與人事也今以天時觀之更始既立而災變方興以人事觀之帝王大業非凡夫所任【任音壬】分崩離析形埶可見明公雖建藩輔之功猶恐無所成立也况明公素有盛德大功為天下所嚮服軍政齊肅賞罰明信為今之計莫如延攬英雄務悦民心立高祖之業救萬民之命以公而慮天下不足定也【鄧禹為中興元功實本諸此】秀大悦因令禹宿止於中與定計議每任使諸將多訪於禹皆當其才秀自兄縯之死每獨居輒不御酒肉【御進也】枕席有涕泣處主簿馮異獨叩頭寛譬【馮異自父城歸光武為司隸主簿及度河為大司馬主簿寛釋也譬曉也譬曉以寛釋其哀戚之情】秀止之曰卿勿妄言異因進說曰更始政亂百姓無所依戴夫人久飢渴易為充飽【孟子曰飢者易為食渇者易為飲賢曰猶言凋殘之後易流德澤易以豉翻】今公專命方面宜分遣官屬徇行郡縣【行下孟翻】宣布惠澤秀納之騎都尉宋子耿純謁秀於邯鄲【先是李軼承制拜耿純為騎都尉賢曰宋子縣屬鉅鹿郡故城在今趙州平棘縣北三十里邯鄲縣屬趙國今洺州縣】退見官屬將兵法度不與他將同遂自結納故趙繆王子林【賢曰繆王景帝七代孫名元前書曰坐殺人為大鴻臚所奏謚曰繆音謬】<br />
<br />
  說秀决列人河水以灌赤眉【續漢書林言于秀曰赤眉可破秀問其故對曰赤眉今在河東河水從列人北流如決河水灌之可令為魚列人縣屬鉅鹿郡賢曰故城在今洛州肥鄉縣東北】秀不從去之真定【賢曰真定縣名屬真定國今恒州縣也】林素任俠於趙魏間王莽時長安中有自稱成帝子子輿者莽殺之【如淳曰相與信為任同是非為俠所謂權行州里力折公侯者也或曰俠之為言挾也以權力夾輔人者也子輿事見三十七卷王莽始建國二年】邯鄲卜者王郎緣是詐稱真子輿云母故成帝謳者嘗見黄氣從上下遂任身【任音壬】趙后欲害之偽易它人子以故得全林等信之與趙國大豪李育張參等謀共立郎會民間傳赤眉將度河林等因此宣言赤眉當立劉子輿以觀衆心百姓多信之十二月林等率車騎數百晨入邯鄲城止于王宫【賢曰故趙王之宫也邯鄲音寒丹】立郎為天子分遣將帥狥下幽冀移檄州郡趙國以北遼東以西皆望風響應<br />
<br />
  二年春正月大司馬秀以王郎新盛乃北徇薊【賢曰薊縣名屬涿郡今幽州縣也薊音計】 申屠建李松自長安迎更始遷都二月更始發洛陽初三輔豪桀假號誅莽者【謂假漢將軍號也】人人皆望封侯申屠建既斬王憲又揚言三輔兒大黠【黠下八翻桀黠也】共殺其主吏民惶恐屬縣屯聚建等不能下更始至長安乃下詔大赦非王莽子他皆除其罪於是三輔悉平時長安唯未央宫被焚其餘宫室供帳倉庫官府皆案堵如故市里不改於舊更始居長樂宫【樂音洛】升前殿郎吏以次列庭中更始羞怍俛首刮席不敢視【賢曰怍顔色變也俛俯也刮爬也怍才各翻俛音免】諸將後至者更始問虜掠得幾何左右侍官皆宫省久吏驚愕相視【給事天子左右者謂之侍官】李松與棘陽趙萌說更始宜悉王諸功臣【說輸芮翻王于况翻】朱鮪爭之以為高祖約非劉氏不王【鮪于軌翻】更始乃先封諸宗室祉為定陶王【班志定陶縣屬濟隂郡宋白曰定陶故城在曹州東北三十七里】慶為燕王【燕於賢翻】歙為元氏王【元氏縣屬常山郡闞駰曰趙公子元之封邑故曰元氏】嘉為漢中王【祉舂陵康侯敞之子太宗也慶敞之弟嘉敞之弟子歙更始之叔父歙許及翻】賜為宛王【宛縣屬南陽郡宋白曰鄧州南陽縣漢之宛縣】信為汝隂王【班志汝隂縣屬汝南郡故胡國唐潁州治所】然後立王匡為泚陽王【泚陽後漢書作比陽比陽縣屬南陽郡唐屬唐州】王鳳為宜城王【班志宜城縣屬南郡故鄢朱為大堤之地玄華山郡後魏改宜城郡唐為縣屬襄州】朱鮪為膠東王【膠東漢王國都即墨賢曰故城在今萊州膠水縣東南】王常為鄧王【鄧縣屬南陽郡故鄧國唐為鄧城縣屬襄州】申屠建為平氏王【班志平氏縣屬南陽郡有桐柏山唐為桐柏縣屬唐州】陳牧為隂平王【賢曰隂平縣屬廣漢郡宋白曰唐文州曲水縣漢隂平道也】衛尉大將軍張卬為淮陽王【淮陽本陳國漢為淮陽國賢曰淮陽故城在今陳州宛丘縣東南】執金吾大將軍廖湛為穰王【穰縣屬南陽郡師古曰今鄧州穰縣是也】尚書胡殷為隨王【隨縣屬南陽郡古隨國唐為隨州】柱天大將軍李通為西平王【賢曰西平縣屬汝南郡故城在今豫州郾城縣南】五威中郎將李軼為舞隂王【舞隂縣屬南陽郡宋白曰唐州泌陽縣本漢舞隂縣地舞陽故城在葉縣東十里】水衡大將軍成丹為襄邑王【襄邑縣屬陳留郡國稱曰襄邑宋地本承匡襄陵鄉也宋襄公所葬故曰襄陵秦始皇以承匡卑濕徙縣於襄陵故曰襄邑縣西三十里有承匡故城賢曰今襄邑縣在宋州西】驃騎大將軍宗佻為潁隂王【佻他彫翻又田聊翻班志潁隂縣屬潁川郡】尹尊為郾王【班志郾縣屬潁川郡宋白曰七國時魏之下邑今許州郾城縣是也括地志豫州褒信縣本漢郾縣地師古曰郾一戰翻】唯朱鮪辭不受乃以鮪為左大司馬宛王賜為前大司馬使與李軼等鎮撫關東又使李通鎮荆州王常行南陽太守事以李松為丞相趙萌為右大司馬共秉内任【内任謂朝廷之内】更始納趙萌女為夫人故委政於萌日夜飲讌後庭羣臣欲言事輒醉不能見時不得已乃令侍中坐帷中與語韓夫人尤嗜酒每侍飲見常侍奏事【中常侍受外朝臣奏事而奏之天子】輒怒曰帝方對我飲正用此時持事來邪起抵破書案【賢曰抵擊也】趙萌專權生殺自恣郎吏有說萌放縱者更始怒拔劒斬之自是無敢復言【復扶又翻】以至羣小膳夫皆濫授官爵長安為之語曰竈下養中郎將【公羊傳曰炊烹為養音弋亮翻】爛羊胃騎都尉爛羊頭關内侯【言以烹煮熟爛為功也】軍師將軍李淑上書諫曰陛下定業雖因下江平林之埶斯蓋臨時濟用不可施之既安唯名與器聖人所重【孔子曰唯名與器不可以假人】今加非其人望其裨益萬分猶緣木求魚升山采珠【賢曰言求之非所不可得也】海内望此有以窺度漢祚【度徒洛翻】更始怒囚之諸將在外者皆專行誅賞各置牧守州郡交錯不知所從由是關中離心四海怨叛 更始徵隗囂及其叔父崔義等囂將行方望以更始成敗未可知固止之囂不聽望以書辭謝而去囂等至長安更始以囂為右將軍崔義皆即舊號【就其舊號以授之隗囂違方望之言而從更始違馬援之言而叛光武始則幾至殺身後則終于滅族擇木之難也】耿况遣其子弇奉奏詣長安弇時年二十一【弇古含翻】行至宋子會王郎起弇從吏孫倉衛包曰劉子輿成帝正統捨此不歸遠行安之弇按劒曰子輿弊賊卒為降虜耳【從才用翻卒終也音子恤翻】我至長安與國家陳上谷漁陽兵馬歸發突騎【賢曰突騎言能衝突軍陳】以轔烏合之衆如摧枯折腐耳【賢曰轔轢也音力刃翻】觀公等不識去就族滅不久也倉包遂亡降王郎弇聞大司馬秀在盧奴【賢曰盧奴縣名屬中山國故城在今定城安喜縣水經注曰縣有黑水故池水黑曰盧不流曰奴因以為名】乃馳北上謁【上時掌翻下異上同】秀留署長史與俱北至薊【薊故燕都昭帝改燕為廣陽國亦治薊】王郎移檄購秀十萬戶秀令功曹令史潁川王霸至市中募人擊王郎【漢舊注公府令史秩百石霸時為大司馬功曹令史】市人皆大笑舉手邪揄之霸慚懅而反【賢曰說文曰歋手相笑也歋音弋支翻音踰或音由此云邪揄語輕重不同懅亦慙也音遽】秀將南歸耿弇曰今兵從南方來不可南行漁陽太守彭寵公之邑人【彭寵南陽宛人】上谷太守即弇父也發此兩郡控弦萬騎邯鄲不足慮也秀官屬腹心皆不肯曰死尚南首奈何北行入囊中【賢曰漁陽上谷北接塞垣至彼路窮如入囊中也首音式救翻】秀指弇曰是我北道主人也會故廣陽王子接起兵薊中以應郎【賢曰廣陽王名嘉武帝五代孫】城内擾亂言邯鄲使者方到二千石以下皆出迎于是秀趣駕而出【賢曰趣急也音促】至南城門門已閉攻之得出遂晨夜南馳不敢入城邑舍食道傍至蕪蔞亭【賢曰蕪蔞亭名在今饒陽東北蔞音力于翻】時天寒烈馮異上豆粥至饒陽【賢曰饒陽縣名屬安平國在饒河之陽故城在今瀛州饒陽縣東北】官屬皆乏食秀乃自稱邯鄲使者入傳舍【賢曰傳舍客館也傳音知戀翻下同】傳吏方進食從者飢爭奪之【從才用翻】傳吏疑其偽乃椎鼓數十通紿言邯鄲將軍至【賢曰椎直追翻紿言欺誑也音衣】官屬皆失色秀升車欲馳既而懼不免徐還坐曰請邯鄲將軍入久乃駕去晨夜兼行蒙犯霜雪面皆破裂至下曲陽【賢曰下曲陽縣名屬鉅鹿郡常山郡有上曲陽故此言下劉昭曰下曲陽有鼓聚故翟鼓子國宋白曰鎮州鼔城縣漢下曲陽縣地】傳聞王郎兵在後從者皆恐【從才用翻】至嘑沱河【賢曰山海經云大戲之山滹沱之水出焉在今代州繁畤縣東流經定州深澤縣東南即光武所度處今俗猶謂之危渡口臣賢按滹沱河舊在饒陽南至魏太祖曹操因饒河故瀆决令北注新溝水所以今在饒陽縣北】候吏還白河水流澌【賢曰斯音斯冰澌也】無船不可濟秀使王霸往視之霸恐驚衆欲且前阻水還即詭曰氷堅可度官屬皆喜秀笑曰候吏果妄語也遂前比至河【比必寐翻及也】河氷亦合乃令王霸護度【賢曰監護度也】未畢數騎而氷解至南宫【賢曰南宫縣名屬信都國今冀州縣也】遇大風雨秀引車入道傍空舍馮異抱薪鄧禹爇火【賢曰爇音而悦翻】秀對竈燎衣【賢曰燎炙也】馮異復進麥飯【復扶又翻】進至下慱城西【賢曰下博縣屬信都國在博水之下故曰下博故城在今冀州下博縣南】惶惑不知所之有白衣老父在道旁【賢曰蓋神人也今下博縣西有祠堂】指曰努力信都郡為長安城守【賢曰信都郡今冀州為于偽翻】去此八十里秀即馳赴之是時郡國皆已降王郎獨信都太守南陽任光和戎太守信都邳肜不肯從【東觀記曰王莽分信都為和戎居下曲陽邳肜傳作和成成字為是風俗通奚仲為夏車正封於邳其後以為氏肜余中翻】光自以孤城獨守恐不能全【賢曰獨守無援故恐】聞秀至大喜吏民皆稱萬歲邳肜亦自和戎來會議者多言可因信都兵自送西還長安邳肜曰吏民歌吟思漢久矣故更始舉尊號而天下響應三輔清宫除道以迎之今卜者王郎假名因埶驅集烏合之衆遂振燕趙之地【振舉也】無有根本之固明公奮二郡之兵以討之【二郡信都和成】何患不克今釋此而歸豈徒空失河北必更驚動三輔墮損威重【墮讀曰隳】非計之得者也若明公無復征伐之意【復扶又翻】則雖信都之兵猶難會也何者明公既西則邯鄲埶成民不肯捐父母背成主而千里送公【謂光武西歸則王郎之位號定故曰成主背蒲妹翻 考異曰范書邳肜傳邯鄲成民不肯背成主字皆作城表紀作邯鄲和城民不肯捐和城而千里送公漢春秋作邯鄲之民不能捐父母背成主按文意城皆當作成邯鄲成謂邯鄲埶成也成主謂王郎為已成之主也】其離散亡逃可必也秀乃止秀以二郡兵弱欲入城頭子路力子都軍中【爰曾起兵盧城頭曾字子路故號城頭子路 考異曰范書作力子都同編修劉攽曰力當作刁音彫】任光以為不可乃發傍縣得精兵四千人拜任光為左大將軍信都都尉李忠為右大將軍邳肜為後大將軍和戎太守如故信都令萬修為偏將軍【萬姓也孟子弟子有萬章】皆封列侯留南陽宗廣領信都太守事使任光李忠萬修將兵以從【從才用翻下使從同】邳彤將兵居前任光乃多作檄文曰大司馬劉公將城頭子路力子都兵百萬衆從東方來擊諸反虜遣騎馳至鉅鹿界中吏民得檄傳相告語【傳知戀翻語牛倨翻】秀投暮入堂陽界【賢曰堂陽縣屬鉅鹿郡在堂水之陽故城在今冀州鹿城縣南】多張騎火彌滿澤中堂陽即降又擊貰縣降之【賢曰貰縣屬鉅鹿郡音時夜翻師古音式制翻】城頭子路者東平爰曾也寇掠河濟間有衆二十餘萬【濟子禮翻】力子都有衆六七萬故秀欲依之昌城人劉植聚兵數千人據昌城迎秀【賢曰昌城縣屬信都郡故城在今冀州西北杜佑曰故城在冀州信都縣北水經注引應劭曰在堂陽縣北三十里】秀以植為驍騎將軍耿純率宗族賓客二千餘人老病者皆載木自隨【賢曰左傳曰又如是而嫁將就木焉木謂棺也老病者恐死故載以從軍】迎秀於育【賢曰育縣名故城在冀州予考兩漢志無育縣蓋貰字之誤】拜純為前將軍進攻下曲陽降之衆稍合至數萬人復北擊中山【賢曰中山國一名中人亭故城在今定州唐縣東北張曜中山記曰城中有山故曰中山也復扶又翻】耿純恐宗家懷異心乃使從弟訢宿歸燒廬舍以絶其反顧之望秀進拔盧奴【杜佑曰定州安喜縣漢盧奴也】所過發奔命兵移檄邊郡共擊邯鄲郡縣還復響應【復扶又翻下同】時真定王楊起兵附王郎衆十餘萬秀遣劉植說楊楊乃降【楊常山憲王舜六世孫舜景帝子也說輸芮翻】秀因留真定納楊甥郭氏為夫人以結之進擊元氏防子皆下之【賢曰元氏防子屬常山郡並今趙州縣也防與房古字通用】至鄗擊斬王郎將李惲【鄗縣屬常山郡賢曰今趙州高邑縣也鄗音呼各翻惲於粉翻】至柏人復破郎將李育【賢曰柏人縣名屬趙國今邢州縣故城在縣之西北】育還保城攻之不下 南鄭人延岑起兵據漢中【延姓岑名】漢中王嘉擊降之有衆數十萬校尉南陽賈復見更始政亂乃說嘉曰【說輸芮翻】今天下未定而大王安守所保所保得無不可保乎【所保謂漢中也】嘉曰卿言大非吾任也大司馬在河北必能相用乃為書薦復及長史南陽陳俊於劉秀復等見秀於柏人秀以復為破虜將軍俊為安集掾【劉玄傳玄初從陳牧等為其軍安集掾賢曰欲以安集軍衆故權以為官名予謂光武用俊之意不特安集軍衆盖為民也掾俞絹翻】秀舍中兒犯法軍市令潁川祭遵格殺之【從軍者非一處人故於軍中立市使相貿易置令以治之姓譜周公第五子祭伯其後以為氏賢曰祭音側介翻】秀怒命收遵主簿陳副諫曰明公常欲衆軍整齊今遵奉法不避是教令所行也乃貰之【賢曰貰猶赦也】以為刺姦將軍【王莽置左右刺奸使督奸猾光武因以為將軍號】謂諸將曰當備祭遵吾舍中兒犯法尚殺之必不私諸卿也 初王莽既殺鮑宣【事見三十六卷平帝元始三年】上黨都尉路平欲殺其子永太守苟諫保護之永由是得全更始徵永為尚書僕射行大將軍事將兵安集河東并州【河東郡本屬司隸令永安集河東及并州所部諸郡】得自置偏禆永至河東擊青犢大破之以馮衍為立漢將軍屯太原與上黨太守田邑等繕甲養士以扞衛并土 或說大司馬秀以守柏人不如定鉅鹿【說輸芮翻下同】秀乃引兵東北拔廣阿【賢曰廣阿縣名屬鉅鹿郡故城在今趙州象城縣西北杜佑曰趙州昭慶縣漢廣阿縣】秀披輿地圖【武帝時羣臣請王皇子御史奏輿地圖索隱曰謂地為輿者天地有覆載之德故謂天為蓋謂地為輿故地圖稱輿地圖疑自古有此名非始漢也】指示鄧禹曰天下郡國如是今始乃得其一子前言以吾慮天下不足定何也禹曰方今海内殽亂人思明君猶赤子之慕慈母古之興者在德薄厚不以大小也 薊中之亂耿弇與劉秀相失北走昌平【賢曰昌平縣屬上谷郡今幽州縣故城在縣東】就其父况因說况擊邯鄲【說輸芮翻下同】時王郎遣將徇漁陽上谷急發其兵北州疑惑多欲從之上谷功曹寇恂門下掾閔業【閔姓也魯有大夫閔馬父孔子弟子有閔子騫】說况曰邯鄲拔起【賢曰拔猝也】難可信向大司馬劉伯升母弟尊賢下士可以歸之【下遐稼翻】况曰邯鄲方盛力不能獨拒如何對曰今上谷完實控弦萬騎可以詳擇去就恂請東約漁陽齊心合衆邯鄲不足圖也况然之遣恂東約彭寵欲各發突騎二千匹步兵千人詣大司馬秀安樂令吳漢護軍蓋延狐奴令王梁亦勸寵從秀【樂音洛蓋古盍翻狐奴縣屬漁陽郡】寵以為然而官屬皆欲附王郎寵不能奪漢出止外亭【外亭城門外之亭也】遇一儒生召而食之【食讀曰飲】問以所聞生言大司馬劉公所過為郡縣所稱邯鄲舉尊號者實非劉氏漢大喜即詐為秀書移檄漁陽使生齎以詣寵令具以所聞說之【說輸芮翻】會寇恂至寵乃發步騎三千人以吳漢行長史與蓋延王梁將之南攻薊殺王郎大將趙閎寇恂還遂與上谷長史景丹及耿弇將兵俱南與漁陽軍合所過擊斬王郎大將九卿校尉以下凡斬首三萬級定涿郡中山鉅鹿清河河間凡二十二縣前及廣阿聞城中車騎甚衆丹等勒兵問曰此何兵曰大司馬劉公也諸將喜即進至城下城下初傳言二郡兵為邯鄲來【為于偽翻下同】衆皆恐劉秀自登西城樓勒兵問之耿弇拜于城下即召入具言發兵狀秀乃悉召景丹等入笑曰邯鄲將帥數言我發漁陽上谷兵【數所角翻下同】吾聊應言我亦發之【賢曰王郎將帥數云欲發二郡兵故拒光武光武聊亦應云然猶今兩軍相戲弄也孔頴達曰聊且畧之辭】何意二郡良為吾來 【考異曰袁紀作良牧為吾來今從景丹傳 韻釋良首也信也】方與士大夫共此功名耳乃以景丹寇恂耿弇蓋延吳漢王梁皆為偏將軍使還領其兵加耿况彭寵大將軍封况寵丹延皆為列侯吳漢為人質厚少文造次不能以辭自逹【少詩沼翻朱子曰造次急遽苟且之時造七到翻】然沈厚有智畧【沈持林翻】鄧禹數薦之於秀秀漸親重之更始遣尚書令謝躬率六將軍討王郎不能下秀至與之合軍東圍鉅鹿月餘未下王郎遣將攻信都大姓馬寵等開門内之更始遣兵攻破信都秀使李忠還行太守事王郎遣將倪宏劉奉率數萬人救鉅鹿秀逆戰於南不利【賢曰南縣名屬鉅鹿郡故城在今邢州柏人縣東北左傳齊國夏伐晉取欒即其地也其後南徙故加南今謂之倫城聲之轉也杜佑曰唐鉅鹿漢南地漢鉅鹿縣今平鄉也音力全翻】景丹等縱突騎擊之宏等大敗秀曰吾聞突騎天下精兵今見其戰樂可言邪【樂音洛】耿純言於秀曰久守鉅鹿士衆疲弊不如及大兵精銳進攻邯鄲若王郎已誅鉅鹿不戰自服矣秀從之夏四月留將軍鄧滿守鉅鹿進軍邯鄲連戰破之郎乃使其諫大夫杜威請降威雅稱郎實成帝遺體秀曰設使成帝復生【復扶又翻】天下不可得况詐子輿者乎威請求萬戶侯秀曰顧得全身可矣威怒而去秀急攻之二十餘日五月甲辰郎少傅李立開門内漢兵遂拔邯鄲郎夜亡走王霸追斬之秀收郎文書得吏民與郎交關謗毁者數千章【關通也】秀不省【省悉井翻】會諸將軍燒之曰令反側子自安【賢曰反側不安詩曰展轉反側】秀部分吏卒各隸諸軍【句絶】士皆言願屬大樹將軍大樹將軍者偏將軍馮異也為人謙不伐敕吏士非交戰受敵常行諸營之後每所止舍諸將並坐論功異常獨屏樹下【屏必郢翻蔽也坐樹下以自匿也】故軍中號曰大樹將軍護軍宛人朱祜言于秀曰長安政亂公有日角之相此天命也【考異曰范書表紀朱祜皆作祜按東觀記祐皆作福避安帝諱許慎說文祜字無解云上諱然則祜名當作】<br />
<br />
  【示旁古古今之古不當作左右之右也相息亮翻】秀曰召刺姦收護軍【祜事伯升為大司徒護軍光武為大司馬復以為護軍百官表護軍都尉秦官平帝元始元年更名護軍】祜乃不敢復言【復扶又翻下同】更始遣使立秀為蕭王【賢曰蕭縣屬沛郡今徐州縣也】悉令罷兵與諸將有功者詣行在所【蔡邕獨斷曰天子以四海為家故謂所居為行在所】遣苗曾為幽州牧韋順為上谷太守蔡充為漁陽太守並北之部蕭王居邯鄲宫晝臥温明殿【賢曰趙王如意之殿也故基在今洺州邯鄲縣内水經注温明殿在叢臺西】耿弇入造牀下請間【造七到翻】因說曰吏士死傷者多請歸上谷益兵蕭王曰王郎已破河北畧平復用兵何為【復扶又翻】弇曰王郎雖破天下兵革乃始耳今使者從西方來欲罷兵不可聽也銅馬赤眉之屬數十輩輩數十百萬人所向無前聖公不能辦也【賢曰辦猶成也音蒲莧翻予據史記項梁曰使公主某事不能辦即此意今人謂了事為辦事】敗必不久蕭王起坐曰卿失言我斬卿弇曰大王哀厚弇如父子故敢披赤心蕭王曰我戲卿耳何以言之弇曰百姓患苦王莽復思劉氏聞漢兵起莫不歡喜如去虎口得歸慈母今更始為天子而諸將擅命於山東貴戚縱横於都内【横戶孟翻都内謂長安】虜掠自恣元元叩心更思莽朝【朝直遥翻】是以知其必敗也公功名已著以義征伐天下可傳檄而定也天下至重公可自取毋令它姓得之蕭王乃辭以河北未平不就徵始貳於更始【賢曰貳離異也】是時諸賊銅馬大彤高湖重連鐵脛大槍尤來上江青犢五校五幡五樓富平獲索等各領部曲衆合數百萬人【賢曰諸賊或以山川土地為名或以軍容強盛為號銅馬賊帥東山荒秃上淮况等大彤渠帥樊重尤來渠帥樊崇五校賊帥高扈檀鄉賊帥董次仲五樓賊帥張文富平賊帥徐少獲索賊帥古師郎等並見東觀記脛形定翻富平縣名屬平原郡今棣州厭次縣】所在寇掠蕭王欲擊之乃拜吳漢耿弇俱為大將軍持節北發幽州十郡突騎【幽州十郡涿郡廣陽代郡上谷漁陽遼西遼東玄菟樂浪郡是也】苗曾聞之隂敕諸郡不得應調【賢曰調發也調徒弔翻】吳漢將二十騎先馳至無終【賢曰無終本山戎國也無終山名因以為國號漢為縣名屬右北平故城在今幽州漁陽縣是時苗曾蓋治無終】曾出迎於路漢即收曾斬之耿弇到上谷亦收韋順蔡充斬之北州震駭于是悉發其兵秋蕭王擊銅馬於鄡【賢曰鄡縣名屬鉅鹿郡故城在冀州鹿城縣東鄡音古堯翻】吳漢將突騎來會清陽【賢曰清陽縣名屬清河郡今貝州縣故城在州西北】士馬甚盛漢悉上兵簿於莫府【賢曰莫大也兵簿軍士之名帳上時掌翻】請所付與不敢自私王益重之王以偏將軍沛國朱浮為大將軍幽州牧使治薊城銅馬食盡夜遁蕭王追擊於館陶大破之【賢曰館陶縣屬魏郡今魏州縣】受降未盡【降戶江翻下同】而高湖重連從東南來與銅馬餘衆合蕭王復與大戰於蒲陽悉破降之【賢曰前書音義曰蒲陽山蒲水所出在今定州北平縣西北予按此乃班書地理志中山曲逆縣下分生非音義也復扶又翻】封其渠帥為列侯【帥所類翻】諸將未能信賊降者亦不自安王知其意敕令降者各歸營勒兵自乘輕騎按行部陳降者更相語曰蕭王推赤心置人腹中安得不投死乎【賢曰投死猶言致死予謂投託也託以死也行下孟翻陳讀曰陣更工衡翻】由是皆服悉以降人分配諸將衆遂數十萬赤眉别帥與青犢上江大彤銕脛五幡十餘萬衆在射犬【賢曰續漢志野王有射犬聚故城在今懷州武德縣北】蕭王引兵進擊大破之南狥河内河内太守韓歆降 初謝躬與蕭王共滅王郎數與蕭王違戾【數所角翻】常欲襲蕭王畏其兵彊而止雖俱在邯鄲遂分城而處【處昌呂翻】然蕭王有以慰安之躬勤於吏職蕭王常稱之曰謝尚書真吏也故不自疑其妻知之常戒之曰君與劉公積不相能而信其虛談終受制矣躬不納既而躬率其兵數萬還屯于鄴【鄴縣屬魏郡】及蕭王南擊青犢使躬邀擊尤來於隆慮山【地理志隆慮縣屬河内郡應邵曰隆慮山在縣北避殤帝名改曰林慮師古曰慮音廬】躬兵大敗蕭王因躬在外使吳漢與刺姧大將軍岑彭襲據鄴城躬不知輕騎還鄴漢等收斬之其衆悉降 更始遣柱功侯李寶益州刺史李忠將兵萬餘人徇蜀漢公孫述遣其弟恢擊寶忠於綿竹【賢曰綿竹縣名屬廣漢郡今益州縣也故城今在縣東】大破走之述遂自立為蜀王都成都【述先居臨卭今徙成都】民夷皆附之冬更始遣中郎將歸德侯颯大司馬護軍陳遵使匈奴授單于漢舊制璽綬【王莽簒漢易單于璽綬事見三十七卷始建國二年今復授之颯音立璽斯氏翻綬音受】因送云當餘親屬貴人從者還匈奴【天鳳五年莽脅云當至長安莽敗云當亦死所餘親屬貴人從者今送還匈奴從才用翻】單于輿驕謂遵颯曰匈奴本與漢為兄弟匈奴中亂【師古曰言中間之時也讀如本字又音竹仲翻】孝宣皇帝輔立呼韓邪單于故稱臣以尊漢今漢亦大亂為王莽所簒匈奴亦出兵擊莽空其邊境令天下騷動思漢莽卒以敗而漢復興亦我力也當復尊我遵與相牚拒【卒子恤翻復扶又翻師古曰牚謂支拄也音人庚翻又丑庚翻】單于終持此言 赤眉樊崇等將兵入潁川分其衆為二部崇與逢安為一部【東觀記曰逢音龎】徐宣謝禄楊音為一部赤眉雖數戰勝【數所角翻】而疲弊厭兵皆日夜愁泣思欲東歸崇等計議慮衆東向必散不如西攻長安於是崇安自武關宣等從陸渾關【賢曰武關在今商州上洛縣東文穎曰弘農析縣西百七十里有武關前書曰陸渾關有關在今洛州伊關縣西南地理志陸渾縣屬弘農郡師古曰渾音胡昆翻】兩道俱入更始使王匡成丹與抗威將軍劉均等分據河東弘農以拒之 蕭王將北狥燕趙度赤眉必破長安【度徒洛翻】又欲乘釁并關中而未知所寄乃拜鄧禹為前將軍中分麾下精兵二萬人遣西入關令自選偏禆以下可與俱者時朱鮪李軼田立陳僑將兵號三十萬【僑音喬】與河南太守武勃共守洛陽鮑永田邑在并州蕭王以河内險要富實【河内北有大行之險南據河津之要】欲擇諸將守河内者而難其人【賢曰非其人不可故難之】問於鄧禹鄧禹曰寇恂文武備足有牧人御衆之才非此子莫可使也乃拜恂河内太守行大將軍事 【考異曰袁紀鄧禹初見王於鄴即言欲據河内至是又云更始武隂王李軼據洛陽尚書謝躬據鄴各十餘萬衆王患焉將取河内以迫之謂鄧禹曰卿言吾之有河内猶高祖之有關中關中非蕭何誰能使一方晏然高祖無西顧之憂吳漢之能卿舉之矣復為我舉蕭何禹曰寇恂才兼文武有御衆才非恂莫可安河内也按世祖既貳更始先得河内魏郡因欲守之以比關中非本心造謀即欲指取河内也今依范書為定】蕭王謂恂曰㫺高祖留蕭何關中吾今委公以河内當給足軍糧率厲士馬防遏它兵勿令北度而已拜馮異為孟津將軍【賢曰孟地名古今以為津在河内郡河陽縣南門外】統魏郡河内兵於河上以拒洛陽蕭王親送鄧禹至野王禹既西蕭王乃復引兵而北寇恂調糇粮【復扶又翻調徒弔翻糇音侯乾食也】治器械以供軍【治直之翻】軍雖遠征未嘗乏絶 隗崔隗義謀叛歸天水隗囂恐并及旤乃告之更始誅崔義以囂為御史大夫【武帝元鼎三年置天水郡秦州記云郡前湖水冬夏無增减因以名焉】 梁王永據國起兵招諸郡豪桀沛人周建等並署為將帥攻下濟隂山陽沛楚淮陽汝南凡得二十八城【將即亮翻帥所類翻濟子禮翻】又遣使拜西防賊帥山陽佼彊為横行將軍【賢曰西防縣名故城在今宋州單父縣北佼音絞姓也周大夫原伯佼之後姓譜曰春秋絞國即佼也後改從人漢有佼彊杜佑曰佼音効予考两漢志無西防縣】東海賊帥董憲為翼漢大將軍琅邪賊帥張步為輔漢大將軍督青徐二州與之連兵遂專據東方 邔人秦豐起兵于黎丘攻得邔宜城等十餘縣有衆萬人自號楚黎王【按王莽之末秦豐已起兵矣通鑑書于上卷地皇二年邔宜城二縣屬南郡孟康曰邔音忌師古曰邔音其又劉昭曰邔有黎丘城賢曰習鑿齒襄陽記曰秦豐黎丘鄉人黎丘楚地故稱楚黎王黎丘故城在今襄州率道縣北杜佑曰襄州宜城縣舊率道也水經注黎丘在中廬縣西北沔水逕其西】 汝南田戎攻陷夷陵【賢曰夷陵縣名屬南郡有夷山故曰夷陵今陜州縣故城在今縣西北水經注吳改夷陵為西陵】自稱埽地大將軍轉寇郡縣衆數萬人<br />
<br />
  資治通鑑卷三十九  <br>
   </div> 

<script src="/search/ajaxskft.js"> </script>
 <div class="clear"></div>
<br>
<br>
 <!-- a.d-->

 <!--
<div class="info_share">
</div> 
-->
 <!--info_share--></div>   <!-- end info_content-->
  </div> <!-- end l-->

<div class="r">   <!--r-->



<div class="sidebar"  style="margin-bottom:2px;">

 
<div class="sidebar_title">工具类大全</div>
<div class="sidebar_info">
<strong><a href="http://www.guoxuedashi.com/lsditu/" target="_blank">历史地图</a></strong>  
<a href="http://www.880114.com/" target="_blank">英语宝典</a>  
<a href="http://www.guoxuedashi.com/13jing/" target="_blank">十三经检索</a> 
<br><strong><a href="http://www.guoxuedashi.com/gjtsjc/" target="_blank">古今图书集成</a></strong> 
<a href="http://www.guoxuedashi.com/duilian/" target="_blank">对联大全</a> <strong><a href="http://www.guoxuedashi.com/xiangxingzi/" target="_blank">象形文字典</a></strong> 

<br><a href="http://www.guoxuedashi.com/zixing/yanbian/">字形演变</a>  <strong><a href="http://www.guoxuemi.com/hafo/" target="_blank">哈佛燕京中文善本特藏</a></strong>
<br><strong><a href="http://www.guoxuedashi.com/csfz/" target="_blank">丛书&方志检索器</a></strong> <a href="http://www.guoxuedashi.com/yqjyy/" target="_blank">一切经音义</a>  

<br><strong><a href="http://www.guoxuedashi.com/jiapu/" target="_blank">家谱族谱查询</a></strong>  <strong><a href="http://shufa.guoxuedashi.com/sfzitie/" target="_blank">书法字帖欣赏</a></strong> 
<br>

</div>
</div>


<div class="sidebar" style="margin-bottom:0px;">

<font style="font-size:22px;line-height:32px">QQ交流群9:489193090</font>


<div class="sidebar_title">手机APP 扫描或点击</div>
<div class="sidebar_info">
<table>
<tr>
	<td width=160><a href="http://m.guoxuedashi.com/app/" target="_blank"><img src="/img/gxds-sj.png" width="140"  border="0" alt="国学大师手机版"></a></td>
	<td>
<a href="http://www.guoxuedashi.com/download/" target="_blank">app软件下载专区</a><br>
<a href="http://www.guoxuedashi.com/download/gxds.php" target="_blank">《国学大师》下载</a><br>
<a href="http://www.guoxuedashi.com/download/kxzd.php" target="_blank">《汉字宝典》下载</a><br>
<a href="http://www.guoxuedashi.com/download/scqbd.php" target="_blank">《诗词曲宝典》下载</a><br>
<a href="http://www.guoxuedashi.com/SiKuQuanShu/skqs.php" target="_blank">《四库全书》下载</a><br>
</td>
</tr>
</table>

</div>
</div>


<div class="sidebar2">
<center>


</center>
</div>

<div class="sidebar"  style="margin-bottom:2px;">
<div class="sidebar_title">网站使用教程</div>
<div class="sidebar_info">
<a href="http://www.guoxuedashi.com/help/gjsearch.php" target="_blank">如何在国学大师网下载古籍?</a><br>
<a href="http://www.guoxuedashi.com/zidian/bujian/bjjc.php" target="_blank">如何使用部件查字法快速查字?</a><br>
<a href="http://www.guoxuedashi.com/search/sjc.php" target="_blank">如何在指定的书籍中全文检索?</a><br>
<a href="http://www.guoxuedashi.com/search/skjc.php" target="_blank">如何找到一句话在《四库全书》哪一页?</a><br>
</div>
</div>


<div class="sidebar">
<div class="sidebar_title">热门书籍</div>
<div class="sidebar_info">
<a href="/so.php?sokey=%E8%B5%84%E6%B2%BB%E9%80%9A%E9%89%B4&kt=1">资治通鉴</a> <a href="/24shi/"><strong>二十四史</strong></a>&nbsp; <a href="/a2694/">野史</a>&nbsp; <a href="/SiKuQuanShu/"><strong>四库全书</strong></a>&nbsp;<a href="http://www.guoxuedashi.com/SiKuQuanShu/fanti/">繁体</a>
<br><a href="/so.php?sokey=%E7%BA%A2%E6%A5%BC%E6%A2%A6&kt=1">红楼梦</a> <a href="/a/1858x/">三国演义</a> <a href="/a/1038k/">水浒传</a> <a href="/a/1046t/">西游记</a> <a href="/a/1914o/">封神演义</a>
<br>
<a href="http://www.guoxuedashi.com/so.php?sokeygx=%E4%B8%87%E6%9C%89%E6%96%87%E5%BA%93&submit=&kt=1">万有文库</a> <a href="/a/780t/">古文观止</a> <a href="/a/1024l/">文心雕龙</a> <a href="/a/1704n/">全唐诗</a> <a href="/a/1705h/">全宋词</a>
<br><a href="http://www.guoxuedashi.com/so.php?sokeygx=%E7%99%BE%E8%A1%B2%E6%9C%AC%E4%BA%8C%E5%8D%81%E5%9B%9B%E5%8F%B2&submit=&kt=1"><strong>百衲本二十四史</strong></a>  <a href="http://www.guoxuedashi.com/so.php?sokeygx=%E5%8F%A4%E4%BB%8A%E5%9B%BE%E4%B9%A6%E9%9B%86%E6%88%90&submit=&kt=1"><strong>古今图书集成</strong></a>
<br>

<a href="http://www.guoxuedashi.com/so.php?sokeygx=%E4%B8%9B%E4%B9%A6%E9%9B%86%E6%88%90&submit=&kt=1">丛书集成</a> 
<a href="http://www.guoxuedashi.com/so.php?sokeygx=%E5%9B%9B%E9%83%A8%E4%B8%9B%E5%88%8A&submit=&kt=1"><strong>四部丛刊</strong></a>  
<a href="http://www.guoxuedashi.com/so.php?sokeygx=%E8%AF%B4%E6%96%87%E8%A7%A3%E5%AD%97&submit=&kt=1">說文解字</a> <a href="http://www.guoxuedashi.com/so.php?sokeygx=%E5%85%A8%E4%B8%8A%E5%8F%A4&submit=&kt=1">三国六朝文</a>
<br><a href="http://www.guoxuedashi.com/so.php?sokeytm=%E6%97%A5%E6%9C%AC%E5%86%85%E9%98%81%E6%96%87%E5%BA%93&submit=&kt=1"><strong>日本内阁文库</strong></a> <a href="http://www.guoxuedashi.com/so.php?sokeytm=%E5%9B%BD%E5%9B%BE%E6%96%B9%E5%BF%97%E5%90%88%E9%9B%86&ka=100&submit=">国图方志合集</a> <a href="http://www.guoxuedashi.com/so.php?sokeytm=%E5%90%84%E5%9C%B0%E6%96%B9%E5%BF%97&submit=&kt=1"><strong>各地方志</strong></a>

</div>
</div>


<div class="sidebar2">
<center>

</center>
</div>
<div class="sidebar greenbar">
<div class="sidebar_title green">四库全书</div>
<div class="sidebar_info">

《四库全书》是中国古代最大的丛书,编撰于乾隆年间,由纪昀等360多位高官、学者编撰,3800多人抄写,费时十三年编成。丛书分经、史、子、集四部,故名四库。共有3500多种书,7.9万卷,3.6万册,约8亿字,基本上囊括了古代所有图书,故称“全书”。<a href="http://www.guoxuedashi.com/SiKuQuanShu/">详细>>
</a>

</div> 
</div>

</div>  <!--end r-->

</div>
<!-- 内容区END --> 

<!-- 页脚开始 -->
<div class="shh">

</div>

<div class="w1180" style="margin-top:8px;">
<center><script src="http://www.guoxuedashi.com/img/plus.php?id=3"></script></center>
</div>
<div class="w1180 foot">
<a href="/b/thanks.php">特别致谢</a> | <a href="javascript:window.external.AddFavorite(document.location.href,document.title);">收藏本站</a> | <a href="#">欢迎投稿</a> | <a href="http://www.guoxuedashi.com/forum/">意见建议</a> | <a href="http://www.guoxuemi.com/">国学迷</a> | <a href="http://www.shuowen.net/">说文网</a><script language="javascript" type="text/javascript" src="https://js.users.51.la/17753172.js"></script><br />
  Copyright &copy; 国学大师 古典图书集成 All Rights Reserved.<br>
  
  <span style="font-size:14px">免责声明:本站非营利性站点,以方便网友为主,仅供学习研究。<br>内容由热心网友提供和网上收集,不保留版权。若侵犯了您的权益,来信即刪。scp168@qq.com</span>
  <br />
ICP证:<a href="http://www.beian.miit.gov.cn/" target="_blank">鲁ICP备19060063号</a></div>
<!-- 页脚END --> 
<script src="http://www.guoxuedashi.com/img/plus.php?id=22"></script>
<script src="http://www.guoxuedashi.com/img/tongji.js"></script>

</body>
</html>
