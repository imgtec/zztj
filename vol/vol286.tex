






























































資治通鑑卷二百八十六 宋 司馬光 撰

胡三省 音註

後漢紀一【起疆圉協洽正月盡四月不盡一年 高祖本沙陀部人居於太原及得中國自以姓劉遂言為東漢顯宗第八子淮陽王昞之後國號曰漢通鑑已前以有漢紀此以後漢紀書之}


高祖睿文聖武昭肅孝皇帝上

【姓劉名知遠乾祐元年更名暠其先沙陀部人也}


天福十二年【漢復以天福紀年詳見後}
春正月丁亥朔百官遙辭晉主於城北【大梁城之北}
乃易素服紗帽迎契丹主伏路側請罪契丹主貂帽貂裘衷甲駐馬高阜命起改服撫慰之【按歐史時晉百官迎契丹主于赤岡}
左衛上將軍安叔千獨出班胡語【按薛史安叔千沙陀三部落之種也故習胡語}
契丹主曰汝安没字耶【安叔千狀貌堂堂而不通文字所為鄙陋人謂之没字碑}
汝昔鎮邢州已累表輸誠我不忘也叔千拜謝呼躍而退【呼躍蓋夷禮猶華人舞蹈也}
晉主與太后已下迎於封丘門外契丹主辭不見 【考異曰漢高祖實録少帝帥族候于野耶律氏疏之帝指陳前事乃大臣同謀皆歷歷能對無撓屈色耶律氏亦假以顔色陷蕃記薛史帝紀五代通録云戎王不與帝相見少帝實録帝舉族待罪於野虜長面撫之遣白封禪寺今從陷蕃記}
契丹主入門民皆驚呼而走【呼火故翻}
契丹主登城樓遣通事諭之曰我亦人也汝曹勿懼會當使汝曹蘇息【氣絶而復息曰蘇氣一出入為息一曰更息曰蘇}
我無心南來漢兵引我至此耳【歸罪於杜威等}
至明德門下馬拜而後入宫以其樞密副使劉密權開封尹事【先易置京尹以彈壓華人}
日暮契丹主復出屯於赤岡【懼人心未一未敢居城中}
戊子執鄭州防禦使楊承勲至大梁責以殺父叛契

丹【楊承勲囚父以降晉事見二百八十四卷齊王開運元年}
命左右臠食之未幾【臠力兖翻幾居豈翻}
以其弟右羽林將軍承信為平盧節度使悉以其父舊兵授之【既授之以其父舊鎮復授之以其父舊兵}
高勲訴張彦澤殺其家人於契丹主【張彦澤殺高勲家見上卷上年勲為杜威奉降表者也先已為契丹主所親故得訴其事}
契丹主亦怒彦澤剽掠京城并傅珠爾鎻之【彦澤剽掠事亦見上卷上年傅珠爾監彦澤軍者也剽匹妙翻}
以彦澤之罪宣示百官問應死否皆言應死百姓亦投牒爭疏彦澤罪己丑斬彦澤珠爾於北市仍命高勲監刑彦澤前所殺士大夫子孫皆絰杖號哭隨而詬詈以杖扑之【絰徒結翻有親喪者絰杖號戶刀翻詬苦候翻又許候翻詈力智翻扑普卜翻擊也}
勲命斷腕出鎖【斷音短腕烏貫翻}
剖其心以祭死者市人爭破其腦取髓【髓悉委翻}
臠其肉而食之 契丹送景延廣歸其國庚寅宿陳橋【九域志開封府浚儀縣有陳橋鎮}
夜伺守者稍怠扼吭而死【伺相吏翻吭居郎翻人頸曰吭}
辛卯契丹以晉主為負義侯置於黃龍府黃龍府即慕容氏和龍城也【歐史曰自幽州行十餘日過平州出榆關行沙磧中七八日至錦州又行五六日過海北州又行十餘日度遼水至勃海國鐵州又行七八日過南海府遂至黃龍府按契丹後改黄龍府為隆州北至混同江一百三十里又按慕容氏之和龍城若据晉書及酈道元水經注當在漢遼西郡界今晉主陷蕃度遼水而後至黃龍府又其地近混同江疑非慕容氏之和龍城}
契丹主使謂李太后曰聞重貴不用母命以至於此可求自便勿與俱行太后曰重貴事妾甚謹所失者違先君之志絶兩國之歡耳今幸蒙大恩全生保家母不隨子欲何所歸癸巳契丹遣晉主及其家人於封禪寺遣大同節度使兼侍中【此契丹所授官}
河内崔廷勲以兵守之【宋白曰崔廷勲本河内人少陷虜}
契丹主數遣使存問【數所角翻}
晉主每聞使至舉家憂恐【恐見殺也}
時雨雪連旬外無供億【毛居正曰供儗儗有儲偫之意供億猶供儗也億度也料度其所須之物隨多少而供之以待其乏也}
上下凍餒太后使人謂寺僧曰吾嘗於此飯僧數萬【飯扶晩翻}
今日獨無一人相念邪僧辭以虜意難測不敢獻食【噫孰知緇黃變色其徒所為有甚於不敢獻食者耶有國有家者崇奉釋氏以求福田利益可以監矣}
晉主隂祈守者乃稍得食是日契丹主自赤岡引兵入宫【入晉宫}
都城諸門及宫禁門皆以契丹守衛晝夜不釋兵仗【懼有變也}
磔犬於門以竿懸羊皮於庭為厭勝【磔涉格翻厭於葉翻}
契丹主謂羣臣曰自今不修甲兵不市戰馬輕賦省役天下太平矣【談何容易斯言甫脫口而打草穀繼之矣天下果太平乎}
廢東京降開封府為汴州尹為防禦使乙未契丹主改服中國衣冠百官起居皆如舊制【史言契丹主猶知用夏變夷}
趙延夀張礪共薦李崧之才會威勝節度使馮道自鄧州入朝契丹主素聞二人名皆禮重之【二人歷唐晉位極人臣國亡不能死視其君如路人何足重哉}
未幾以崧為太子太師充樞密使道守太傅於樞密院祗候以備顧問契丹主分遣使者以詔書賜晉之藩鎮晉之藩鎮爭上表稱臣被召者無不奔馳而至【被皮義翻}
惟彰義節度使史匡威據涇州不受命匡威建瑭之子也【史建瑭事晉王克用以及莊宗皆冇戰功}
雄武節度使何重建斬契丹使者以秦階成三州降蜀【史匡威不降契丹以其地遠契丹兵威不能至也何重建則以其鎮與蜀接境遂弃遼而附蜀耳}
初杜重威既以晉軍降契丹【重威初避晉主重貴名去重單名威晉既亡國重即復舊名其忘恩背主此特末節耳}
契丹主悉收其鎧仗數百萬貯恒州【貯丁呂翻恒戶登翻}
驅馬數萬歸其國遣重威將其衆從已而南【將即亮翻}
及河契丹主以晉兵之衆恐其為變欲悉以胡騎擁而納之河流或諫曰晉兵在它所者尚多彼聞降者盡死必皆拒命不若且撫之徐思其策契丹主乃使重威以其衆屯陳橋【陳喬在陳橋門外有陳橋驛}
會久雪官無所給士卒凍餒咸怨重威相聚而泣重威每出道旁人皆罵之契丹主猶欲誅晉兵趙延夀言於契丹主曰皇帝親冒矢石以取晉國欲自有之乎將為它人取之乎【冒莫北翻為于偽翻下同趙延夀志在帝中國以此言覘契丹之意不特為晉兵發也}
契丹主變色曰朕舉國南征五年不解甲【天福八年契丹始攻晉至是五年}
僅能得之豈為它人乎【趙延夀聞契丹主此言可以絕望矣}
延夀曰晉國南有唐西有蜀常為仇敵皇帝亦知之乎曰知之延夀曰晉國東自沂密西及秦鳳延袤數千里【袤音茂}
邉於吳蜀常以兵戍之南方暑濕上國之人不能居也【時偏方割據者謂中原為上國晉奉契丹又稱契丹為上國}
它日車駕北歸以晉國如此之大無兵守之吳蜀必相與乘虚入寇如此豈非為它人取之乎契丹主曰我不知也然則奈何延夀曰陳橋降卒可分以戍南邉則吳蜀不能為患矣契丹主曰吾昔在上黨失於斷割悉以唐兵授晉【事見二百八十卷晉高祖天福元年斷丁亂翻}
既而返為寇讐北向與吾戰辛勤累年僅能勝之今幸入吾手不因此時悉除之豈可復留以為後患乎【復扶又翻}
延夀曰曏留晉兵於河南不質其妻子【質音致}
故有此憂今若悉徙其家於恒定雲朔之間每歲分番使戍南邉何憂其為變哉此上策也契丹主悦曰善惟大王所以處之【契丹封趙延夀為燕王故稱之為大王處昌呂翻}
由是陳橋兵始得免分遣還營 契丹主殺右金吾衛大將軍李彦紳宦者秦繼是以其為唐潞王殺東丹王故也【殺東丹王見二百八十卷晉高祖天福元年唐潞王之清泰三年也為於偽翻}
以其家族貲財賜東丹王之子永康王烏雲烏雲眇一目為人雄健好施【烏雲始見於此為後得國張本施式䜴翻}
癸卯晉主與李太后安太妃馮后及弟睿子延煦延寶俱北遷後宫左右從者百餘人【從才用翻}
契丹遣三百騎援送之【援送者送其行以為防援}
又遣晉中書令趙瑩樞密使馮玉馬軍都指揮使李彦韜與之俱晉主在塗供饋不繼或時與太后俱絶食舊臣無敢進謁者獨磁州刺史李穀迎謁於路相對泣下穀曰臣無狀負陛下因傾貲以獻【天下之士苟有所負者其所為必有異於人磁墻之翻}
晉主至中度橋見杜重威寨歎曰天乎吾家何負為此賊所破慟哭而去【於晉之時通國上下皆知杜重威之不可用乃違衆用之以致亡國詩云啜其泣矣何嗟及矣今至於慟庸有及乎}
癸丑蜀主以左千牛衛上將軍李繼勲為秦州宣慰使【蜀以何重建降遣使宣慰之}
契丹主以前燕京留守劉晞為西京留守【薛史曰劉晞者涿州人陷虜歷官至平章事兼侍中 考異曰實録作禧或云名稀今從陷蕃記}
永康王烏雲之弟留珪為義成節度使烏雲姊壻潘實納為横海節度使【實納舊作聿撚今改 考異曰周太祖實録實納作聿湼今從陷蕃記}
趙延夀之子匡贊為護國節度使【為趙匡贊後以河中歸漢張本}
漢將張彦超為雄武節度使史佺為彰義節度使客省副使劉晏僧為忠武節度使前護國節度使侯益為鳳翔節度使權知鳳翔府事【侯益後以鳳翔歸漢}
焦繼勲為保大節度使晞涿州人也既而何重建附蜀【秦州附蜀張彦超無所詣}
史匡威不受代【史匡威據涇州以拒史佺}
契丹勢稍沮【沮在呂翻}
晉昌節度使趙在禮入朝【自長安入朝于大梁}
其裨將留長安

者作亂節度副使建人李肅討誅之軍府以安 晉主之絶契丹也【事見二百八十三卷晉高祖天福七年}
匡國節度使劉繼勲為宣徽北院使頗預其謀契丹主入汴繼勲入朝契丹主責之時馮道在殿上繼勲急指道曰馮道為首相與景延廣實為此謀臣位卑何敢發言契丹主曰此叟非多事者勿妄引之【馮道以依阿免祻有國家者焉用彼相哉然歷事七姓皆以德望待之亦持身謹靜有以動其敬心耳}
命鎖繼勲將送黃龍府趙在禮至洛陽【舊唐書地理志自長安東至洛陽八百五十里}
謂人曰契丹主嘗言莊宗之亂由我所致【謂皇甫暉之亂也事見二百七十四卷唐明宗天成元年莊宗之同光四年也}
我此行良可憂契丹遣契丹將蘇頁【書契丹將以别漢將與勃海將}
奚王伊喇【伊耶尼翻喇盧達翻}
勃海將高謨翰戍洛陽在禮入謁拜於庭下伊喇等皆踞坐受之乙卯在禮至鄭州【九域志自洛陽東至鄭州二百六十里}
聞繼勲被鎖大驚夜自經於馬櫪間【櫪音歷馬棧也}
契丹主聞在禮死乃釋繼勲繼勲憂憤而卒劉晞在契丹嘗為樞密使同平章事至洛陽詬奚王曰【詬苦候翻又許候翻}
趙在禮漢家大臣爾北方一酋長耳【酋慈秋翻長知兩翻}
安得慢之如此立於庭下以挫之由是洛人稍安契丹主廣受四方貢獻大縱酒作樂每謂晉臣曰中國事我皆知之吾國事汝曹不知也【契丹主自謂周防之密以夸晉臣然東丹之來已胎烏雲奪國之禍雖甚愚者知之而契丹主不知也善覘國者不觀一時之強弱而觀其治亂之大致}
趙延夀請給上國兵廩食契丹主曰吾國無此法乃縱胡騎四出以牧馬為名分番剽掠【剽匹妙翻}
謂之打草穀丁壯斃於鋒刃老弱委於溝壑自東西兩畿【大梁之屬縣為東畿洛陽之屬縣為西畿此唐制也唐制兩京除赤縣外餘屬縣為畿縣}
及鄭滑曹濮數百里間財畜殆盡【鄭滑曹濮皆大梁之旁郡以及言之明上文所謂東西兩畿為畿縣濮博木翻}
契丹主謂判三司劉昫曰契丹兵三十萬既平晉國應有優賜速宜營辦時府庫空竭昫不知所出請括借都城士民錢帛【都城大梁都城}
自將相以下皆不免又分遣使者數十人詣諸州括借皆迫以嚴誅人不聊生其實無所頒給皆蓄之内庫欲輦歸其國於是内外怨憤始患苦契丹皆思逐之矣【為契丹北歸張本}
初晉主與河東節度使中書令北平王劉知遠相猜忌雖以為北面行營都統徒尊以虛名而諸軍進止實不得預聞【事見二百八十四卷晉齊王開運元年}
知遠因之廣募士卒【天福八年齊王與契丹構隙之初劉知遠已奏募兵矣事見二百八十三卷}
陽城之戰諸軍散卒歸之者數千人【陽城之戰見二百八十四卷晉齊王開運二年按陽城之戰晉師大捷無緣有散卒歸河東此必杜重威降契丹時也}
又得吐谷渾財畜【事亦見開運二年畜吁玉翻}
由是河東富疆冠諸鎮【冠古玩翻}
步騎至五萬人晉主與契丹結怨知遠知其必危而未嘗論諫契丹屢深入知遠初無邀遮入援之志【既不據險要以邀遮契丹之兵又不遣兵入援也}
及聞契丹入汴知遠分兵守四境以防侵軼【軼徒結翻}
遣客將安陽王峻【舊唐書地理志相州漢魏郡也治安陽縣安陽漢侯國故城在湯隂東曹魏時廢安陽併入鄴後周移鄴置縣於安陽故城仍為鄴縣隋又改為安陽縣州所治也若漢魏郡城則在縣之西北七里將即亮翻}
奉三表詣契丹主一賀入汴二以太原夷夏雜居戍兵所聚未敢離鎮【夏戶雅翻離力智翻}
三以應有貢物值契丹將劉九一軍自土門西入屯於南川【南川謂晉陽城南之地}
城中憂懼俟召還此軍道路始通可以入貢契丹主賜詔褒美及進畫親加兒字於知遠姓名之上仍賜以木柺胡法優禮大臣則賜之如漢賜几杖之比惟偉王以叔父之尊得之【柺乖買翻老人拄杖也歐史曰王峻持柺歸虜人望之皆避道}
知遠又遣北都副留守太原白文珂入獻奇繒名馬【繒慈陵翻}
契丹主知知遠觀望不至及文珂還【還從宣翻又如字}
使謂知遠曰汝不事南朝又不事北朝意欲何所俟邪【朝直遙翻}
蕃漢孔目官郭威言於知遠曰虜恨我深矣王峻言契丹貪殘失人心必不能久有中國或勸知遠舉兵進取知遠曰用兵有緩有急當隨時制宜今契丹新降晉兵十萬虎據京邑未有它變豈可輕動哉且觀其所利止於貨財貨財既足必將北去况氷雪已消勢難久留宜待其去然後取之可以萬全【劉知遠料之審矣所以舉兵南向契丹不能與之爭}
昭義節度使張從恩以地迫懷洛【昭義治潞州自潞州至澤州又至懷州度河則洛州河南府舊唐書地理志潞州至洛州四百七十里}
欲入朝於契丹遣使謀於知遠知遠曰我以一隅之地安敢抗天下之大君宜先行我當繼往從恩以為然判官高防諫曰公晉室懿親【按五代會要晉少帝前妃張氏天福八年進册皇后張從恩蓋后族也}
不可輕變臣節從恩不從左驍衛大將軍王守恩與從恩姻家時在上黨從恩以副使趙行遷知留後【副使者節度副使也}
牒守恩權巡檢使與高防佐之守恩建立之子也【王建立事唐明宗見親任及事晉高祖}
荆南節度使高從誨遣使入貢於契丹契丹遣使以馬賜之從誨亦遣使詣河東勸進【荆南高氏父子事大以保其國為謀大率如此}
唐主立齊王景遂為皇太弟徙燕王景達為齊王領諸道兵馬元帥徙南昌王弘冀為燕王為之副【燕於堅翻}
景遂嘗與宫僚燕集贊善大夫元城張易有所規諫【張易北人而仕江南}
景遂方與客傳玩玉杯弗之顧易怒曰殿下重寶而輕士取玉杯抵地碎之衆皆失色景遂斂容謝之待易益厚【景遂之遷善敬士亦難能也}
景達性剛直唐主與宗室近臣飲馮延已延魯魏岑陳覺輩極傾諂之態或乘酒喧笑景達屢訶責之復極言諫唐主以不宜親近佞臣【屢力主翻復扶又翻近巨靳翻}
延己以二弟立非己意欲以虚言德之嘗晏東宫陽醉撫景達背曰爾不可忘我景達大怒拂衣入禁中白唐主請斬之唐主諭解乃止【按是時陳覺馮延魯攻福州史言其侍飲極傾諂之態槩其言常時非必拘此時也}
張易謂景達曰羣小交構禍福所繫殿下力未能去數面折之【去羌呂翻數所角翻折之舌翻}
使彼懼而為備何所不至自是每遊宴景達多辭疾不預唐主遣使賀契丹滅晉且請詣長安修復諸陵【唐末喪亂諸陵多遭發掘南唐自謂纂唐之緒故請修復也}
契丹不許而遣使報之晉密州刺史皇甫暉棣州刺史王建皆避契丹帥衆奔唐【帥讀曰率}
淮北賊帥多請命於唐【帥所類翻}
唐虞部員外郎韓熙載上疏以為陛下恢復祖業今也其時若虜主北歸中原有主則未易圖也【易以䜴翻韓熙載以定中原自期僅見此疏耳自古以來多大言少成事者何可勝數}
時方連兵福州未暇北顧唐人皆以為恨唐主亦悔之【使唐無福州之役舉兵北向亦喪師而已矣}
契丹主召晉百官悉集於庭問曰吾國廣大方數萬里有君長二十七人【長知兩翻}
今中國之俗異於吾國吾欲擇一人君之如何皆曰天無二日【孟子引孔子之言}
夷夏之心皆願推戴皇帝如是者再契丹主乃曰汝曹既欲君我今兹所行何事為先對曰王者初有天下應大赦二月丁巳朔契丹主服通天冠絳紗袍登正殿設樂懸儀衛於庭百官朝賀華人皆法服胡人仍胡服立於文武班中間【文官班於東武官班於西胡人立於中間}
下制稱大遼會同十年大赦仍云自今節度使刺史母得置牙兵市戰馬【其心固虞諸鎮有與之作敵者}
趙延夀以契丹主負約心怏怏【趙延夀之求為帝不得不止此其所以終為兀欲所鎖也怏於兩翻}
令李崧言於契丹主曰漢天子所不敢望乞為皇太子崧不得已為言之【為言於偽翻下令為同}
契丹主曰我於燕王雖割吾肉有用於燕王吾無所愛然吾聞皇太子當以天子兒為之豈燕王所可為也因令為燕王遷官時契丹以恒州為中京【恒戶登翻}
翰林承旨張礪奏擬燕王中京留守大丞相録尚書事都督中外諸軍事樞密使如故契丹主取筆塗去録尚書事都督中外諸軍事而行之【去羌呂翻孰謂契丹主起于塞外而不知中國之事體哉}
壬戌蜀李繼勲與興州刺史劉景攻固鎮拔之何重建請出蜀兵與階成兵共扼散關以取鳳州【扼散關則北兵不能入鳳州可坐取也}
丙寅蜀主發山南兵三千七百赴之【山南兵興元兵也}
劉知遠聞何重建降蜀歎曰戎狄憑陵中原無主令藩鎮外附吾為方伯良可愧也【古者除王畿之外八州八伯所謂三十國而為連連有帥二百二十國以為州州有伯者也周分天下以為二伯自陜以西召伯主之自陜以東周公主之及其衰也齊桓晉文糾合諸侯以尊王室亦以方伯之任自居晉人所謂我為伯者也石晉以劉知遠為北面都統故亦自謂為方伯}
於是將佐勸知遠稱尊號以號令四方觀諸侯去就【諸侯謂當時諸藩鎮}
知遠不許聞晉主北遷聲言欲出兵井陘迎歸晉陽【陘音刑}
丁卯命武節都指揮使滎澤史弘肇【武節軍劉知遠所置見二百八十三卷晉齊王天福八年隋置滎澤縣唐屬鄭州九域志滎澤縣在鄭州西北四十五里}
集諸軍於毬場告以出軍之期軍士皆曰今契丹陷京城執天子天下無主主天下者非我王而誰【劉知遠封北平王故稱之}
宜先正位號然後出師爭呼萬歲不已知遠曰虜勢尚彊吾軍威未振當且建功業士卒何知命左右遏止之已巳行軍司馬潞城張彦威等三上牋勸進【潞古邑也隋置潞城縣唐屬潞州九域志潞城縣在潞州東北四十里}
知遠疑未決郭威與都押牙冠氏楊邠入說知遠曰【劉昫曰冠氏春秋邑名隋分館陶東界置冠氏縣唐屬魏州九域志在州東北六十里說音稅}
今遠近之心不謀而同此天意也王不乘此際取之謙讓不居恐人心且移移則反受其咎矣知遠從之 契丹以其將劉愿為保義節度副使陜人苦其暴虐【陜失冉翻}
奉國都頭王晏與指揮使趙暉都頭侯章謀曰今胡虜亂華乃吾屬奮發之秋河東劉公威德遠著【劉知遠河東帥故稱之}
吾輩若殺愿舉陜城歸之為天下唱取富貴如反掌耳【返當作反}
暉等然之宴與壯士數人夜踰牙城入府出庫兵以給衆庚午旦斬愿首懸諸府門又殺契丹監軍奉暉為留後晏徐州暉澶州章太原人也【澶時連翻}
辛未劉知遠即皇帝位自言未忍改晉又惡開運之名乃更稱天福十二年【惡烏路翻更工衡翻歐陽修曰人君即位稱元年常事爾古不以為重也孔子未修春秋其前固己如此雖暴君昏主妄庸之史其紀事先後遠近莫不以歲月一二數之乃理之自然也其謂一為元未嘗有法焉古人之語爾古謂歲之一月亦不云一而曰正月國語言六呂曰元間大呂周易列六爻曰初九大抵古人言數多不云一不獨謂年為元也及後世曲學之士始謂孔子書元年為春秋大法遂以改元為重事自漢以後又名年以建元而正偽紛雜稱號遂多不勝其紀也五代亂世也其事無法而不合於理者多矣至其年號乖錯以惑後世則不可以不明梁太祖以乾化二年遇弑明年末帝誅友珪黜其鳳歷之號稱乾化三年尚為有說至漢高祖建國黜晉出帝開運四年復稱天福十二年者何哉蓋以愛憎之私耳方出帝時漢高祖居太原常憤憤下視晉晉亦陽優禮之幸而未見其隙及契丹滅晉漢未嘗有赴難之意出帝已北遷方陽以兵聲言追之至土門而還及其即位改元而黜開運之號則其用心可知矣蓋其於出帝無復君臣之義而幸禍以為利者其素志也可勝歎哉}
壬申詔諸道為契丹括率錢帛者皆罷之【括率錢帛見上正月}
其晉臣被迫脇為使者勿問令詣行在【被皮義翻}
自餘契丹所在誅之 何重建遣宫苑使崔延琛將兵攻鳳州不克退保固鎮【何重建為蜀圖取鳳州事始見上}
甲戌帝自將東迎晉主及太后至夀陽【晉置夀陽縣後魏改曰受陽隋開皇十年改并州南受陽為文水分州東故夀陽置夀陽縣唐屬太原府}
聞已過恒州數日乃留兵戍承天軍而還【還從宣翻又如字承天軍在井陘縣娘子關西南太原府廣陽縣界宋朝太平興國四年改廣陽為平定縣置平定軍縣有承天軍寨在太原府南三百五十里}
晉主既出塞契丹無復供給從官宫女皆自采木實草葉而食之至錦州契丹令晉主及后妃拜契丹主安巴堅墓【從才用翻契丹置錦州近木葉山金人疆域圖錦州南至燕京一千四百一十五里陳元靚曰大元於錦州置臨海節度領永樂安昌興城神水四縣屬大定府路}
晉主不勝屈辱泣曰薛超誤我【勝音升謂薛超持之不令赴火也事見上卷開運三年}
馮后隂令左右求毒藥欲與晉主俱自殺不果 契丹主聞帝即位以通事耿崇美為昭義節度使高唐英為彰德節度使崔廷勲為河陽節度使以控扼要害【昭義軍潞州彰德軍相州河陽軍孟州帝自太原西南出兵潞州兵衝也自潞州東下壺關則至相州南下太行則至孟州故皆命將控扼}
初晉置鄉兵號天威軍【見二百八十四卷晉出帝開運元年}
教習歲餘村民不閑軍旅竟不可用悉罷之但令七戶輸錢十千其鎧仗悉輸官而無賴子弟不復肯復農業【不復之復扶又翻再也肯復之復讀如字反也}
山林之盜自是而繁及契丹入汴縱胡騎打草穀【事見上正月}
又多以其子弟及親信左右為節度使刺史不通政事華人之狡獪者多往依其麾下敎之妄作威福掊斂貨財民不堪命【狡古巧翻獪古外翻掊蒲候翻斂力贍翻}
於是所在相聚為盜多者數萬人少者不減千百攻陷州縣殺掠吏民滏陽賊帥梁暉有衆數百送欵晉陽求效用帝許之磁州刺史李穀密通表於帝令暉襲相州【舊唐書地理志滏陽漢武安縣地隋置滏陽縣唐屬磁州為州治所九域志滏陽南至相州六十里帥所類翻}
暉偵知高唐英未至【偵丑鄭翻}
相州積兵器無守備丁丑夜遣壯丁踰城入啟關納其衆殺契丹數百其守將突圍走暉據州自稱留後表言其狀【表言於晉陽將即亮翻}
戊寅帝還至晉陽【自承天軍還晉陽還從宣翻又如字}
議率民財以賞將士夫人李氏諫曰陛下因河東創大業未有以惠澤其民而先奪其生生之資殆非新天子所以救民之意也今宫中所有請悉出之以勞軍雖復不厚人無怨言【勞力到翻}
帝曰善即罷率民傾内府蓄積以賜將士中外聞之大悦李氏晉陽人也【婦人之智及此異乎唐莊宗之劉后矣鄙語有之福至心靈禍來神昧二人者各居一焉}
吴越内都監程昭悦多聚賓客畜兵器【畜讀曰蓄}
與術士遊吴越王弘佐欲誅之謂水丘昭劵曰汝今夕帥甲士千人圍昭悦第【帥讀曰率}
昭劵曰昭悦家臣也有罪當顯戮不宜夜興兵弘佐曰善命内牙指揮使諸温【諸姓温名漢書地理志琅邪郡有諸縣蓋以邑為氏也}
伺昭悦歸第執送東府【伺相吏翻}
己卯斬之釋錢仁俊之囚【錢仁俊之囚見上卷開運二年}
武節都指揮使史弘肇攻代州拔之斬王暉【王暉降契丹見上卷上年}
建雄留後劉在明朝於契丹以節度副使駱從朗知州事帝遣使者張宴洪等如晉州諭以己即帝位從朗皆囚之大將藥可儔殺從朗推宴洪權留後庚辰遣使以聞契丹主遣右諫議大夫趙熙使晉州括率錢帛徵督甚急從朗既死民相帥共殺熙【帥讀曰率下同}
契丹主賜趙暉詔即以為保義留後暉斬契丹使者焚其詔遣支使河間趙矩奉表詣晉陽契丹遣其將高謨翰攻暉不克【謨一本作模}
帝見矩甚喜曰子挈咽㗋之地以歸我天下不足定也【陜州據河潼之要自河東入洛汴此其咽㗋也咽因肩翻}
矩因勸帝早引兵南向以副天下之望帝善之辛巳以暉為保義節度使侯章為鎮國節度使保義軍馬步都指揮使王宴為絳州防禦使保義軍馬步副指揮使【按王宴先已為保義軍馬步都指揮使既賞其功不應為副指揮使恐誤}
高防與王守恩謀遣指揮使李萬超白晝率衆大譟入府斬趙行遷推守恩權知昭義留後守恩殺契丹使者舉鎮來降【帝既得陜又得上黨足以示契丹形制之勢之重以澶州梗其南北之路虜氣奪而心揺矣}
鎮寧節度使耶律隆鄂特性殘虐【契丹主安巴堅以其所居横帳地名為姓曰世里世里譯曰謂之錫里史囚之}
澶州人苦之賊帥王瓊帥其徒千餘人夜襲據南城北度浮航【浮航即德勝浮梁賊帥所類翻帥讀曰率航戶剛翻}
縱兵大掠圍隆鄂特於牙城【澶州牙城蓋在北城}
契丹主聞之甚懼始遣天平節度使李守貞天雄節度使杜重威還鎮【李守貞杜重威既降契丹從契丹主南入汴遂為所留}
由是無久留河南之意遣兵救澶州瓊退屯近郊【去城三十里為近郊}
遣弟超奉表來求救癸未帝厚賜超遣還【還從宣翻又如字}
瓊兵敗為契丹所殺 蜀主加雄武節度使何重建同平章事 延州録事參軍高允權萬金之子也彰武節度使周密闇【音暗}
而貪將士作亂攻之密敗保東城衆以允權家世延帥【高萬金兄弟自梁以來帥延州帥所類翻}
推為留後 【考異曰周太祖實録允權為膚施令陷蕃記云前録事參軍退居田里漢高祖實録云允權為延州令周密以允權故將之子恐與邉人締結移為州主簿密後以闇而黨下惟誅掠是務允權乘其民怨時以言間之復勸親黨潜構諸部衆心遂揺廣本云允權為延州令密徙為録事參軍今從之周太祖實録又曰契丹犯闕以周密為延帥按晉少帝實録開運三年八月辛未以右龍武統軍周密為彰武節度使非契丹所授今從漢高祖實録}
據西城【薛史曰延州有東西二城其中限以深澗}
密應州人也 丹州都指揮使高彦珣殺契丹所署刺史自領軍事 契丹舒嚕太后遣使以其國中酒饌脯果賜契丹主賀平晉國【饌徂晥翻又雛戀翻}
契丹主與羣臣宴於永福殿每舉酒立而飲之曰太后所賜不敢坐飲 唐王淑妃與郇公從益居洛陽趙延夀娶明宗女為夫人淑妃詣大梁會禮【趙延夀妻唐明宗女燕國長公主也晉高祖天福元年契丹已遣使至洛陽取之入北矣今復從延夀至大梁故王淑妃詣之會禮}
契丹主見而拜之曰吾嫂也【契丹主以唐明宗年長于齒為兄故拜王淑妃為嫂}
統軍劉遂凝因淑妃求節鉞【劉遂凝以劉鄩舊恩因王淑妃以求節鉞}
契丹主以從益為許王威信節度使遂凝為安遠節度使淑妃以從益幼辭不赴鎮復歸於洛契丹主以張礪為右僕射兼門下侍郎同平章事左僕射和凝兼中書侍郎同平章事司空兼門下侍郎同平章事劉昫以目疾辭位罷為太保 東方羣盜大起陷宋亳密三州契丹主謂左右曰我不知中國之人難制如此【中國之人困於契丹之凌暴掊克咸不聊生起而為盜烏有難制者乎盍亦反其本矣}
亟遣泰寧節度使審琦武寧節度使符彦卿等歸鎮【澶州亂而遣李守貞杜重威歸鎮宋亳密三州陷而遣安審琦符彦卿歸鎮契丹主之北歸決矣}
仍以契丹兵送之彦卿至埇橋【埇橋在宿州埇余隴翻}
賊帥李仁恕帥衆數萬急攻徐州【賊帥所類翻恕帥讀曰率下同}
彦卿與數十騎至城下揚鞭欲招諭之仁恕控彦卿馬請從相公入城【欲刧符彦卿為質以取徐州也}
彦卿子昭序自城中遣軍校陳守習縋而出呼於賊中【校戶敎翻縋馳偽翻呼火故翻}
曰相公已陷虎口聽相公助賊攻城城不可得也賊知不可劫乃相率羅拜於彦卿馬前乞赦其罪彦卿與之誓乃解去 三月丙戌朔契丹主服赭袍坐崇元殿百官行入閤禮【歐陽修曰唐故事天子日御殿見羣臣曰常參朔朢薦食諸陵寢有思慕之心不能臨前殿則御便殿見羣臣曰入閤宣政前殿也謂之衙衙有仗紫宸便殿也謂之閤其不御前殿而御紫宸也乃自正衙喚仗由閤門而入百官俟朝於衙者因隨而入見故謂之入閤然衙朝也其禮尊閤宴見也其事殺自乾符以後因亂禮缺天子不能日見羣臣而見朔朢故正衙常日廢仗而朔朢入閤有仗其後習見遂以入閤為重至出御前殿猶謂之入閤五代之時羣臣五日一入見中興殿便殿也此入閤之遺制而謂之起居朔朢一出御文明殿前殿也反謂之入閤今按五伐會要有入閤儀司天進時刻牌閤門進班齊牌皇帝自内著袍衫穿靴乘輦至常朝殿門駐輦受樞密使已下起居訖引駕至正朝殿皇帝坐定卷簾殿上添香喝控鶴官拜次鷄叫次閤門勘契次閤門承旨喚仗次閤門使引金吾將軍南班拜訖分引至位對揖次細仗相次入次執文武班簿至位對揖次宰臣南班拜訖分引至位對揖次金吾將軍奏平安次文武百官入通事舍人揖殿靸靴入沙墀兩拜立定次引宰臣及兩省官金吾將軍合班立定閤門使喝拜搢笏舞跪三拜奏聖躬萬福又引宰臣班首一人至近前又兩拜舞跪三拜引至位對揖通事舍人引宰臣於東西踏道下立次文武百官出次兩省官南班揖殿出次翰林學士南班揖殿出次執文武班簿南班揖殿出次金吾將軍南班揖殿出次細仗出次引宰臣香案前奉事訖宣徽使喝好去南班揖殿出次閤門使引待制官到位兩拜引近前奏事訖却歸位罄折宣徽使宣所奏知又兩拜舞跪三拜舍人喝好去南班揖殿出次刑法官奏事準上次監奏御史南班揖殿出次閤門承旨放仗次閤門使奏衙内無事次喝控鶴官門外祗候次下簾皇帝上輦歸内又按歐史梁太祖乾化元年九月辛巳朔御文明殿入閤則入閤儀梁所定也㫝唐之正牙朝會其儀畧而野而五代謂之行禮會要又詳載而為書則其儀為一時之上儀矣姑備録之以志朝儀之變文明殿洛陽宫之正衙殿也崇元殿汴宮之正衙殿也薛史曰梁制每月初入閤朢日延英聽政後唐制朔朢皆入閤}
戊子帝遣使以詔書安集農民保聚山谷避契丹之患者【此時務之所當急先者}
辛卯高允權奉表來降帝諭允權聽周密詣行在密

遂弃東城來奔 壬辰高彦詢以丹州來降【丹延亦歸於漢矣}
蜀翰林承旨李昊謂王處回曰敵復據固鎮則興州

道絶不復能救秦州矣【復扶又翻}
請遣山南西道節度使孫漢韶將兵急攻鳳州癸巳蜀主命漢韶詣鳳州行營契丹主復召晉百官【復扶又翻}
諭之曰天時向熱吾難久留欲暫至上國省太后【契丹自謂其國為上國中國之人亦以稱之契丹既畏暑又畏四方羣起而攻之故急欲北歸果如劉知遠所料}
當留親信一人於此為節度使百官請迎太后契丹主曰太后族大如古柏根不可移也契丹主欲盡以晉之百官自隨或曰舉國北遷恐揺人心不如稍稍遷之乃詔有職事者從行餘留大梁復以汴州為宣武軍【契丹之入大梁也降開封府為汴州防禦使今復盛唐之舊以為節鎮欲兼華夷而撫制之也}
以蕭翰為節度使翰舒嚕太后之兄子其妹復為契丹主后翰始以蕭為姓自是契丹后族皆稱蕭氏 吴越復發水軍遣其將余安將之自海道救福州己亥至白蝦浦【將即亮翻蝦當作鰕}
海岸泥淖須布竹簀乃可行唐之諸軍在城南者聚而射之簀不得施【淖奴教翻簀測革翻射而亦翻}
馮延魯曰城所以不降者恃此救也今相持不戰徒老我師不若縱其登岸盡殺之則城不攻自降矣裨將孟堅曰浙兵至此不能進退【吴越國本唐兩浙地故謂之浙兵}
求一戰而死不可得若縱其登岸彼必致死於我其鋒不可當安能盡殺乎延魯不聽曰吾自擊之吴越兵既登岸大呼奮撃【呼火故翻}
延魯不能禦弃衆而走孟堅戰死吴越兵乘勝而進城中兵亦出夾擊唐兵大破之唐城南諸軍皆遁吴越追之王崇文以牙兵三百拒之諸軍陳於崇文之後追者乃還【陳讀曰陣還從宣翻}
或言浙兵欲弃福州拔李達之衆歸錢唐東南守將劉洪進等白王建封請縱其盡出而取其城【唐兵攻福州劉洪進當東南面故書謂東南守將}
留從效不欲福州之平【泉福相為唇齒福州平則泉州為之次矣此留從效所不欲也}
建封亦忿陳覺等專横【横戶孟翻}
乃曰吾軍敗矣安能與人爭城是夕燒營而遁城北諸軍亦相顧而潰馮延魯引佩刀自刺【刺七亦翻}
親吏救之不死唐兵死者二萬餘人委弃軍資器械數十萬府庫為之耗竭【謂唐之府庫罄於奉軍為於偽翻}
余安引兵入福州李達舉所部授之【何承天姓苑余姓戎由余之後}
留從效引兵還泉州【自福州還也}
謂唐戍將曰泉州與福州世為仇敵【唐末王潮兄弟自泉州攻福州留從效先是以泉州兵擊破福州兵又會南唐兵圍福州故云然}
南接嶺海瘴癘之鄉【漳泉之地東南際海西南接潮州嶺南之境也}
地險土瘠比年軍旅屢興農桑廢業冬徵夏斂僅能自贍【秋穀成熟徵租至冬春蠶畢收斂帛於夏即謂二歲也比毘至翻斂力贍翻贍時斂翻}
豈勞大軍久戍於此置酒餞之戍將不得已引兵歸唐主不能制加從效檢校太傅【唐兵新敗自知無以制留從效遂加其官以安之留從效自此據有漳泉}
壬寅契丹主發大梁晉文武諸司從者數千人【從才用翻}
諸軍吏卒又數千人宫女宦官數百人盡載府庫之實以行所留樂器儀仗而已夕宿赤岡契丹主見村落皆空命有司發牓數百通所在招撫百姓然竟不禁胡騎剽掠【呼雞而縱狸奴雞其敢前乎剽匹妙翻}
丙午契丹自白馬渡河謂宣徽使高勲曰吾在上國以射獵為樂至此令人悒悒【契丹之下當逸主字樂音洛悒於及翻悒悒憂愁不得志也}
今得歸死無恨矣【契丹主不惟土思亦見諸鎮及羣盜舉兵者皆歸心河東恐不得正丘首也獨不見涉珪與徒河相持於中山之時乎以此言之其才識相去遠矣}
蜀孫漢韶將兵二萬攻鳳州軍于固鎮分兵扼散關以絶援路【何重建詣扼散關猶慮契丹威令行于關西能發援兵也至是契丹歸北中國無主雖出兵取岐雍可也何必扼散關乎}
張筠余安皆還錢唐吴越王弘佐遣東南安撫使鮑修讓將兵戍福州以東府安撫使錢弘倧為丞相【吴越以越州為東府為弘倧嗣國張本倧竹冬翻}
庚戌以王弟北京馬步都指揮使崇行太原尹知府事【劉崇有太原始此考異曰薛史云崇高祖從弟王保衡晉陽見聞録云仲弟歐陽史云母弟今從實録}
辛亥契丹主將攻相州梁暉請降契丹主赦之許以為防禦使暉疑其詐復乘城拒守夏四月己未未明契丹主命蕃漢諸軍急攻相州食時克之【相悉亮翻}
悉殺城中男子驅其婦女而北胡人擲嬰孩於空中舉刃接之以為樂【觀佛狸之飲江侯景之亂江南其肆毒類如此不嗜殺人然後能一天下孟子之言豈欺我哉樂音洛}
留高唐英守相州唐英閱城中遺民男女得七百餘人其後節度使王繼弘斂城中髑髏瘞之【髑徒木翻髏音婁瘞於計翻}
凡得十餘萬或告磁州刺史李穀謀舉州應漢契丹主執而詰之【詰其吉翻}
穀不服契丹主引手於車中若取所獲文書者穀知其詐因請曰必有其驗乞顯示之凡六詰穀辭氣不屈乃釋之【史言李穀有膽氣}
帝以從弟北京馬軍都指揮使信領義成節度使充侍衛馬軍都指揮使武節都指揮使史弘肇領忠武節度使充步軍都指揮使右都押牙楊邠權樞密使蕃漢兵馬都孔目官郭威權副樞密使兩使都孔目官南樂王章權三司使【兩使節度觀察也樂音洛}
癸亥立魏國夫人李氏為皇后 契丹主見所過城邑丘墟謂蕃漢羣臣曰致中國如此皆燕王之罪也【燕王謂趙延夀}
顧張礪曰爾亦有力焉【張礪隨趙延夀入北又與趙延夀俱南以殘中國契丹主猶知其罪况中國之人乎}
甲子帝以河東節度判官長安蘇逢吉觀察判官蘇

禹珪為中書侍郎同平章事禹珪密州人也振武節度使府州團練使折從遠入朝更名從阮【避帝名更遠名阮更工衡翻}
置永安軍於府州以從阮為節度使【折從阮本領振武節又就府州置節鎮以寵之薛史曰升府州為永安軍析振武之勝州并沿河五鎮以隸之}
又以河東左都押牙劉銖為河陽節度使銖陜人也【陜失冉翻}
契丹昭義節度使耿崇美屯澤州將攻潞州乙丑詔史弘肇將步騎萬人救之 丙寅以王守恩為昭義節度使高允權為彰武節度使又以岢嵐軍使鄭謙為忻州刺史領彰國節度使【彰國軍應州時屬契丹岢枮我翻}
兼忻代二州義軍都部署丁卯以緣河巡檢使閻萬進為嵐州刺史領振武節度使兼嵐憲二州義軍都制置使【憲州本樓煩監嵐州刺史領之唐貞元十五年别置監牧使昭宗龍紀元年李克用表置憲州九域志憲州治静樂縣静樂古汾陽縣地嵐憲二州相去五十里而已嵐盧舍翻}
帝聞契丹北歸欲經畧河南故以弘肇為前驅又遣閻萬進出北方以分契丹兵勢萬進并州人也契丹主以船數十艘載晉鎧仗將自汴沂河歸其國【自汴沂河自河陽取太行路以歸其國也艘蘇遭翻}
命寧國都虞候榆次武行德將士卒千餘人部送之至河隂【河隂在河南東南相去百六十二里}
行德與將士謀曰今為虜所制將遠去鄉里人生會有死安能為異域之鬼乎虜勢不能久留中國不若共逐其黨堅守河陽以俟天命之所歸者而臣之豈非長策乎衆以為然行德即以鎧仗授之相與殺契丹監軍使會契丹河陽節度使崔廷勲以兵送耿崇美之潞州行德遂乘虚入據河陽衆推行德為河陽都部署行德遣弟行友奉蠟表間道詣晉陽【作表寘之蠟丸中故謂之蠟表間古莧翻}
契丹遣武定節度使方太詣洛陽巡檢至鄭州州有戍兵共迫太為鄭王【去年方太以安國留後降契丹契丹主蓋命之領武定節度使武定軍洋州時屬蜀}
梁嗣密王朱乙逃禍為僧【梁太祖兄存之子友倫封密王乙蓋梁亡之後避禍為僧也}
嵩山賊帥張遇得之立以為天子取嵩岳神衮冕以衣之【帥所類翻下賊帥同衣於既翻}
帥衆萬餘襲鄭州太擊走之太以契丹尚彊恐事不濟說諭戍兵欲與俱西【帥讀曰率說式芮翻欲與戍兵俱西至洛陽}
衆不從太自西門逃奔洛陽戍兵既失太反譛太於契丹云脅我為亂太遣子師朗自訴於契丹契丹將滿達勒即殺之太無以自明會羣盜攻洛陽契丹留守劉晞弃城奔許州太乃入府行留守事與巡檢使潘環擊羣盜却之張遇殺朱乙請降伊闕賊帥自稱天子誓衆於南郊壇【後唐郊天壇在洛陽城南}
將入洛陽太逆擊走之 【考異曰實録方太傳云劉禧走許田復有潁陽妖巫姓朱號嗣密王誓衆於洛南郊天壇號萬餘人太帥部曲與朝士輩虚張旗幟一舉而逐之洛師遂安今從陷蕃記}
太欲自歸於晉陽武行德使人誘太曰我裨校也公舊鎮此地【由此觀之契丹嘗命方太鎮河陽史逸之也校戶敎翻}
今虚位相待太信之至河陽為行德所殺蕭翰遣高謨翰援送劉晞自許還洛陽【蕭翰時鎮大梁}
晞疑潘環構其衆逐已使謨翰殺之戊辰武行友至晉陽庚午史弘肇奏遣先鋒將馬誨擊契丹斬首千餘級時耿崇美崔廷勲至澤州聞弘肇兵已入潞州不敢進引兵而南弘肇遣誨追擊破之崇美廷勲與奚王伊喇退保懷州【崔廷勲欲歸河陽河陽己為武行德所據故保懷州以逼河陽九域志懷州南至河陽七十里}
辛未以武行德為河陽節度使契丹主聞河陽亂歎曰我有三失宜天下之叛我也諸道括錢一失也令上國人打草穀二失也不早遣諸節度使還鎮三失也【三失並見上}
唐主以矯詔敗軍皆陳覺馮延魯之罪【陳覺矯詔事見上卷晉出帝開運三年唐主之保大四年也覺延魯敗軍之罪其事見上}
壬申詔赦諸將議斬二人以謝中外御史中丞江文蔚對仗彈馮延己魏岑曰陛下踐祚以來所信任者延己延魯岑覺四人而已皆隂狡弄權壅蔽聰明排斥忠良引用羣小諫争者逐【蔚於勿翻争讀曰諍}
竊議者刑上下相蒙道路以目【言道路相遇但以目相視而不敢言}
今覺延魯雖伏辜而延己岑猶在本根未殄枝榦復生同罪異誅【復扶又翻左傳宋子罕曰同罪異罸非刑也}
人心疑惑又曰上之視聽惟在數人雖日接羣臣終成孤立又曰在外者握兵居中者當國又曰岑覺延魯更相違戾【更工衡翻}
彼前則我却彼東則我西天生五材國之利器【天生五材民並用之出左傳杜預曰五材謂金木水火土也}
一旦為小人忿爭妄動之具又曰征討之柄在岑折簡帑藏取與繫岑一言【折之舌翻帑它朗翻藏徂浪翻}
唐主以文蔚所言為太過怒貶江州司士參軍械送覺延魯至金陵宋齊丘以嘗薦覺使福州【事見上卷晉齊王開運三年}
上表待罪【上時掌翻下同}
詔流覺於蘄州延魯於舒州知制誥會稽徐鉉史館修撰韓熙載上疏曰覺延魯罪不容誅但齊丘延己為之陳請【蘄渠希翻會工外翻為于偽翻}
故陛下赦之擅興者不罪則疆場有生事者矣喪師者獲存則行陳無效死者矣【無詔旨而擅發兵謂之擅興厥罪死場音亦喪息浪翻行戶剛翻陳讀曰陣}
請行顯戮以重軍威不從中書侍郎同平章事馮延已罷為太弟少保貶魏岑為太子洗馬【洗昔薦翻}
韓熙載屢言宋齊丘黨與必為禍亂齊丘奏熙載嗜酒猖狂【猖齒良翻}
貶和州司士參軍乙亥鳳州防禦使石奉頵舉州降蜀【蜀自是盡有秦鳳階成之地頵}


【於倫翻}
奉頵晉之宗屬也 契丹主至臨城得疾及欒城病甚【臨城縣屬趙州本房子縣唐天寶元年改為臨城縣宋白曰欒城縣本漢開縣後魏太和十一年於開縣故城置欒城縣九域志古欒城晉欒氏别邑臨城縣在趙州西南一百三里欒城縣在鎮州南六十三里}
苦熱聚氷於胷腹手足且啖之【啖徒濫翻}
丙子至殺虎林而卒【殺虎林蓋以契丹主死于此時人遂以為地名宋白曰殺虎林唐天后時襲突厥羣胡死於此故名 考異曰實録云二十日乙亥卒今從陷蕃記}
國人剖其腹實鹽數斗載之北去晉人謂之帝羓【羓邦加翻}
趙延夀恨契丹主負約謂人曰我不復入龍沙矣【盧龍山後即大漠故謂之龍沙復扶又翻}
即日先引兵入恒州契丹永康王烏雲及南北二王各以所部兵相繼而入【范成大北使録自欒城至恒州六十里恒戶登翻}
延夀欲拒之恐失大援乃納之時契丹諸將已密議奉烏雲為主烏雲登鼔角樓受叔兄拜而延夀不之知自稱受契丹皇帝遺詔權知南朝軍國事仍下敎布告諸道所以供給烏雲與諸將同烏雲銜之恒州諸門管鑰及倉庫出納烏雲皆自主之延夀使人請之不與【烏雲不與諸門管鍵事可知矣趙延夀殊不知隂為之備其鎖固當}
契丹主喪至國舒嚕太后不哭曰待諸部寧壹如故則葬汝矣【咎其傾國南伐至於耗竭部落不安也}
帝之自夀陽還也【見上二月}
留兵千人戍承天軍戍兵聞契丹北還不為備契丹襲擊之戍兵驚潰契丹焚其市邑一日狼煙百餘舉【陸佃埤雅曰古之烽火用狼糞取其煙直而聚雖風吹之不斜余謂今之烽燧豈必皆用狼糞哉}
帝曰此虜將遁張虚勢也遣親將葉仁魯將步騎三千赴之【親將即亮翻}
會契丹出剽掠【剽匹妙翻}
仁魯乘虚太破之丁丑復取承天軍 冀州人殺契丹刺史何行通推牢城指揮使張廷翰知州事廷翰冀州人符習之甥也【符習成德將歷事唐莊宗及明宗}
或說趙延夀曰【說式芮翻}
契丹諸大人數日聚謀此必有變今漢兵不下萬人不若先事圖之【先悉薦翻}
延夀猶豫不決壬午延夀下令以來月朔日於待賢館上事【上事者言欲禮上以領權知南朝軍國事上時掌翻}
受文武官賀其儀宰相樞密使拜於階上節度使以下拜於階下李崧以虜意不同事理難測固請趙延夀未行此禮乃止

資治通鑑卷二百八十六  














































































































































