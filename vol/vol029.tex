<!DOCTYPE html PUBLIC "-//W3C//DTD XHTML 1.0 Transitional//EN" "http://www.w3.org/TR/xhtml1/DTD/xhtml1-transitional.dtd">
<html xmlns="http://www.w3.org/1999/xhtml">
<head>
<meta http-equiv="Content-Type" content="text/html; charset=utf-8" />
<meta http-equiv="X-UA-Compatible" content="IE=Edge,chrome=1">
<title>資治通鑒_30-資治通鑑卷二十九_30-資治通鑑卷二十九</title>
<meta name="Keywords" content="資治通鑒_30-資治通鑑卷二十九_30-資治通鑑卷二十九">
<meta name="Description" content="資治通鑒_30-資治通鑑卷二十九_30-資治通鑑卷二十九">
<meta http-equiv="Cache-Control" content="no-transform" />
<meta http-equiv="Cache-Control" content="no-siteapp" />
<link href="/img/style.css" rel="stylesheet" type="text/css" />
<script src="/img/m.js?2020"></script> 
</head>
<body>
 <div class="ClassNavi">
<a  href="/24shi/">二十四史</a> | <a href="/SiKuQuanShu/">四库全书</a> | <a href="http://www.guoxuedashi.com/gjtsjc/"><font  color="#FF0000">古今图书集成</font></a> | <a href="/renwu/">历史人物</a> | <a href="/ShuoWenJieZi/"><font  color="#FF0000">说文解字</a></font> | <a href="/chengyu/">成语词典</a> | <a  target="_blank"  href="http://www.guoxuedashi.com/jgwhj/"><font  color="#FF0000">甲骨文合集</font></a> | <a href="/yzjwjc/"><font  color="#FF0000">殷周金文集成</font></a> | <a href="/xiangxingzi/"><font color="#0000FF">象形字典</font></a> | <a href="/13jing/"><font  color="#FF0000">十三经索引</font></a> | <a href="/zixing/"><font  color="#FF0000">字体转换器</font></a> | <a href="/zidian/xz/"><font color="#0000FF">篆书识别</font></a> | <a href="/jinfanyi/">近义反义词</a> | <a href="/duilian/">对联大全</a> | <a href="/jiapu/"><font  color="#0000FF">家谱族谱查询</font></a> | <a href="http://www.guoxuemi.com/hafo/" target="_blank" ><font color="#FF0000">哈佛古籍</font></a> 
</div>

 <!-- 头部导航开始 -->
<div class="w1180 head clearfix">
  <div class="head_logo l"><a title="国学大师官网" href="http://www.guoxuedashi.com" target="_blank"></a></div>
  <div class="head_sr l">
  <div id="head1">
  
  <a href="http://www.guoxuedashi.com/zidian/bujian/" target="_blank" ><img src="http://www.guoxuedashi.com/img/top1.gif" width="88" height="60" border="0" title="部件查字,支持20万汉字"></a>


<a href="http://www.guoxuedashi.com/help/yingpan.php" target="_blank"><img src="http://www.guoxuedashi.com/img/top230.gif" width="600" height="62" border="0" ></a>


  </div>
  <div id="head3"><a href="javascript:" onClick="javascript:window.external.AddFavorite(window.location.href,document.title);">添加收藏</a>
  <br><a href="/help/setie.php">搜索引擎</a>
  <br><a href="/help/zanzhu.php">赞助本站</a></div>
  <div id="head2">
 <a href="http://www.guoxuemi.com/" target="_blank"><img src="http://www.guoxuedashi.com/img/guoxuemi.gif" width="95" height="62" border="0" style="margin-left:2px;" title="国学迷"></a>
  

  </div>
</div>
  <div class="clear"></div>
  <div class="head_nav">
  <p><a href="/">首页</a> | <a href="/ShuKu/">国学书库</a> | <a href="/guji/">影印古籍</a> | <a href="/shici/">诗词宝典</a> | <a   href="/SiKuQuanShu/gxjx.php">精选</a> <b>|</b> <a href="/zidian/">汉语字典</a> | <a href="/hydcd/">汉语词典</a> | <a href="http://www.guoxuedashi.com/zidian/bujian/"><font  color="#CC0066">部件查字</font></a> | <a href="http://www.sfds.cn/"><font  color="#CC0066">书法大师</font></a> | <a href="/jgwhj/">甲骨文</a> <b>|</b> <a href="/b/4/"><font  color="#CC0066">解密</font></a> | <a href="/renwu/">历史人物</a> | <a href="/diangu/">历史典故</a> | <a href="/xingshi/">姓氏</a> | <a href="/minzu/">民族</a> <b>|</b> <a href="/mz/"><font  color="#CC0066">世界名著</font></a> | <a href="/download/">软件下载</a>
</p>
<p><a href="/b/"><font  color="#CC0066">历史</font></a> | <a href="http://skqs.guoxuedashi.com/" target="_blank">四库全书</a> |  <a href="http://www.guoxuedashi.com/search/" target="_blank"><font  color="#CC0066">全文检索</font></a> | <a href="http://www.guoxuedashi.com/shumu/">古籍书目</a> | <a   href="/24shi/">正史</a> <b>|</b> <a href="/chengyu/">成语词典</a> | <a href="/kangxi/" title="康熙字典">康熙字典</a> | <a href="/ShuoWenJieZi/">说文解字</a> | <a href="/zixing/yanbian/">字形演变</a> | <a href="/yzjwjc/">金 文</a> <b>|</b>  <a href="/shijian/nian-hao/">年号</a> | <a href="/diming/">历史地名</a> | <a href="/shijian/">历史事件</a> | <a href="/guanzhi/">官职</a> | <a href="/lishi/">知识</a> <b>|</b> <a href="/zhongyi/">中医中药</a> | <a href="http://www.guoxuedashi.com/forum/">留言反馈</a>
</p>
  </div>
</div>
<!-- 头部导航END --> 
<!-- 内容区开始 --> 
<div class="w1180 clearfix">
  <div class="info l">
   
<div class="clearfix" style="background:#f5faff;">
<script src='http://www.guoxuedashi.com/img/headersou.js'></script>

</div>
  <div class="info_tree"><a href="http://www.guoxuedashi.com">首页</a> > <a href="/SiKuQuanShu/fanti/">四库全书</a>
 > <h1>资治通鉴</h1> <!--         下载:【右键另存为】即可 --></div>
  <div class="info_content zj clearfix">
  
<div class="info_txt clearfix" id="show">
<center style="font-size:24px;">30-資治通鑑卷二十九</center>
    資治通鑑卷二十九   宋 司馬光 撰<br />
<br />
  胡三省 音註<br />
<br />
  漢紀二十一【起上章埶徐盡著雍困敦凡九年】<br />
<br />
  孝元皇帝下<br />
<br />
  永光三年春二月馮奉世還京師更為左將軍賜爵關内侯 三月立皇子康為濟陽王【濟子禮翻】 夏四月平昌考侯王接薨【謚法大慮行方曰考】秋七月壬戌以平恩侯許嘉為大司軍車騎將軍 冬十一月己丑地震雨水 復鹽鐵官置博士弟子員千人【罷鹽鐵官博士弟子毋置員事見上卷初元五年】以用度不足民多復除【復方目翻】無以給中外繇役故也【繇古徭字通】<br />
<br />
  四年春二月赦天下 三月上行幸雍祠五畤【雍於用翻畤音止】夏六月甲戌孝宣園東闕災 戊寅晦日有食之上於是召諸前言日變在周堪張猛者責問皆稽首謝【譛堪猛事見上卷元年稽音啟】因下詔稱堪猛之美徵詣行在所拜為光禄大夫秩中二千石領尚書事猛復為太中大夫給事中中書令石顯管尚書【師古曰言管主其事】尚書五人皆其黨也【按帝紀及百官表成帝建始四年初置尚書員五人此蓋言顯與牢梁五鹿充宗伊嘉陳順五人皆典領尚書事雖未置定員實亦五人也】堪希得見【見賢遍翻】常因顯白事事决顯口會堪疾瘖不能言而卒【師古曰瘖音於今翻】顯誣譖猛令自殺於公車 初貢禹奏言孝惠孝景廟皆親盡宜毁【按貢禹傳定漢宗廟迭毁之禮未及施行而卒其後韋玄成等毁廟之議又不純用禹說觀其奏言天子七廟孝惠孝景親盡宜毁蓋以悼考廟足為七廟也】及郡國廟不應古禮宜正定【惠帝尊高帝廟為太祖廟景帝尊文帝廟為太宗廟行所嘗幸郡國各立太祖太宗廟宣帝復尊武帝為世宗廟行所廵狩亦立焉凡祖宗廟在郡國者六十八合百六十七所春秋之義王不祭於下土諸侯故以為不應古禮】天子是其議秋七月戊子罷昭靈后武哀王昭哀后衛思后戾太子戾后園皆不奉祠裁置吏卒守焉【師古曰昭靈后高祖母也武哀王高祖兄也昭哀后高祖姊也衛思后戾太子母也戾后即史良娣也】冬十月乙丑罷祖宗廟在郡國者 諸陵分屬三輔【師古曰先是諸陵總屬太常今各依其地界屬三輔】以渭城壽陵亭部原上為初陵【服䖍曰元帝所置陵也未有名故曰初】詔勿置縣邑及徙郡國民<br />
<br />
  五年春正月上行幸甘泉郊泰畤【畤音止】三月幸河東祠后土 秋潁川水流殺人民 冬上幸長楊射熊舘【師古曰長楊宮名也在盩厔縣其中有射熊舘】大獵 十二月乙酉毁太上皇孝惠皇帝寢廟園用韋玄成等之議也【玄成等奏曰祖宗之廟世世不毁繼祖以下五廟而迭毁今高皇帝為太祖孝文皇帝為太宗孝景皇帝為昭孝武皇帝為穆孝昭皇帝與孝宣皇帝俱為昭皇考廟親未盡太上皇孝惠廟皆親盡宜毁】 上好儒術文辭頗改宣帝之政言事者多進見【好呼到翻見賢遍翻】人人以為得上意又傅昭儀及子濟陽王康愛幸【外戚傳曰元帝加昭儀之號位視丞相爵比諸侯王師古曰昭顯其儀示隆重也濟子禮翻】逾於皇后太子【昭儀位次皇后今寵逾之】太子少傅匡衡上疏曰【疏所據翻條陳也】臣聞治亂安危之機在乎審所用心蓋受命之王務在創業垂統傳之無窮繼體之君心存於承宣先生之德而褒大其功昔者成王之嗣位思述文武之道以養其心休烈盛美歸之二后而不敢專其名【師古曰休亦美也烈業也后君也二君文王武王也】是以上天歆享鬼神祐焉陛下聖德天覆子愛海内【覆敷又翻】然而隂陽未和姦邪未禁者殆議者未丕揚先帝之盛功【師古曰丕大也】爭言制度不可用也務變更之所更或不可行而復復之【更工衡翻下同上復扶又翻又下復扶目翻反也】是以羣下更相是非吏民無所信臣竊恨國家釋樂成之業而虚為此紛紛也【師古曰釋廢也樂成謂已成之業人情所樂也樂音洛】願陛下詳覽統業之事留神于遵制揚功【遵先帝之法制揚先帝之功烈也】以定羣下之心詩大雅曰無念爾祖聿修厥德【師古曰大雅文王之詩也無念念也聿述也】蓋至德之本也傳曰審好惡理情性而王道畢矣【衡守詩學此必詩傳之言傳直戀翻好呼到翻惡烏路翻】治性之道必審已之所有餘而彊其所不足【治直之翻師古曰彊勉也音其兩翻】蓋聰明疏通者戒於太察寡聞少見者戒于壅蔽【少詩沼翻】勇猛剛彊者戒於太暴仁愛温良者戒於無斷【斷丁亂翻】湛靜安舒者戒於後時廣心浩大者戒於遺忘【師古曰湛讀曰沉忘巫放翻】必審已之所當戒而齊之以義然後中和之化應而巧偽之徒不敢比周而望進【比毗至翻】唯陛下戒之所以崇聖德也臣又聞室家之道修則天下之理得故詩始國風禮本冠婚始乎國風原情性以明人倫也本乎冠婚正基兆以防未然也【師古曰聞雎美后妃之德而為國風之首禮記冠義曰冠者禮之始也婚義曰婚者禮之本也冠古玩翻下同】故聖王必慎妃后之際别適長之位【師古曰適讀曰嫡下同長知兩翻】禮之於内也卑不踰尊新不先故【先悉薦翻】所以統人情而理陰氣也其尊適而卑庶也適子冠乎阼禮之用醴【師古曰阼主階也醴甘酒也貴於衆酒適讀曰嫡】衆子不得與列所以貴正體而明嫌疑也非虚加其禮文而已乃中心與之殊異故禮探其情而見之外也【探吐南翻】聖人動靜游燕所親物得其序【師古曰言凡物大小高卑皆有次序】則海内自修百姓從化如當親者疏【疏讀曰疎】當尊者卑【師古曰如若也】則佞巧之姦因時而動以亂國家故聖人慎防其端禁於未然不以私恩害公義傳曰正家而天下定矣【師古曰易家人卦之彖辭】初武帝既塞宣房【事見二十一卷武帝元封二年塞悉則翻下同】後河復北决於館陶分為屯氏河【館陶縣屬魏郡河水自此别出為屯氏河東北至勃海章武縣入海過魏郡清河信都勃海四郡行千五百里師古曰屯音大門翻而隨氏分析州縣誤以為毛氏河乃置毛州失之甚矣復扶又翻】東北入海廣深與大河等故因其自然不隄塞也是歲河决於清河靈鳴犢口而屯氏河絶【按滿洫志靈鳴犢口在清河東界師古曰清河之靈縣鳴犢河口也按唐博州高唐縣漢靈縣地鳴犢河在縣西宋白曰魏州夏津縣本漢靈縣地漢初為鄃縣故城今在德州西南五十里天寶元年改為夏津縣】<br />
<br />
  建昭元年春正月戊辰隕石于梁【據五行志隕石于梁國】 三月上行幸雍祠五畤【雍於用翻畤音止】 冬河間王元坐賊殺不辜廢遷房陵【元河間獻王德之來孫也】 罷孝文太后寢祠園【孝文太后薄氏葬霸陵之南】 上幸虎圈鬬獸【圈求阮翻下同】後宫皆坐熊逸出圈攀檻欲上殿【上時掌翻】左右貴人傅倢伃等皆驚走【傅倢伃即傅昭儀盖後進號也偼伃音接予】馮倢伃直前當熊而立左右格殺熊上問人情驚懼何故前當熊倢伃對曰猛獸得人而止妾恐熊至御座【坐徂卧翻】故以身當之帝嗟嘆倍敬重焉傳偼伃慙由是與馮倢伃有隙【為後傅太后誣殺中山馮太后張本】馮倢伃左將軍奉世之女也<br />
<br />
  二年春正月上行幸甘泉郊泰畤三月行幸河東祠后土 夏四月赦天下 六月立皇子興為信都王 【考異曰荀紀興作譽今從漢書】 東郡京房學易於梁人焦延壽【風俗通云鄭武公子段封於京其後氏焉姓譜云周武王封神農之後於焦後以國為氏又左傳云虞虢焦滑皆姬姓也董正工曰京房本姓李吹律自定為京氏洪氏隸釋云漢中黄門譙敏碑云其先故國師譙贛傳道與京君房此碑以譙贛為譙左傳楚師伐陳取焦夷注謂焦今譙縣若是則焦譙可以通用】延壽常曰得我道以亡身者京生也其說長於災變分六十卦更直日用事以風雨寒温為候【梁國名孟康曰分卦直日之法一爻各主一日六十卦為三百六十日餘四卦震離兑坎為方伯監司之官所以用震離兑坎者是二至二分用事之日又是四時各專王之氣各卦主時其占法各以其日觀其善惡也更工衡翻】各有占驗房用之尤精以孝亷為郎上疏屢言災異有驗天子說之數召見問【說讀曰悦數所角翻】房對曰古帝王以功舉賢則萬化成瑞應者【師古曰萬化萬機之事施教化者也一曰萬物之類也】末世以毁譽取人【譽音余】故功業廢而致災異宜令百官各試其功災異可息詔使房作其事房奏考功課吏法【晉灼曰令丞尉治一縣崇教化亡犯法者輒遷有盗賊滿三日不覺者則尉事也合覺之自除二尉負其罪率相准如此法】上令公卿朝臣與房會議温室皆以房言煩碎令上下相司不可許上意鄉之【師古曰鄉讀曰嚮】時部刺史奏事京師上召見諸刺史令房曉以課事刺史復以為不可行【武帝置十三州刺史各部一州故曰部刺史復扶又翻】唯御史大夫鄭弘光禄大夫周堪初言不可後善之是時中書令石顯顓權顯友人五鹿充宗為尚書令二人用事房嘗宴見【師古曰以閒宴時而入見天子見賢遍翻】問上曰幽厲之君何以危所任者何人也上曰君不明而所任者巧佞房曰知其巧佞而用之邪將以為賢也上曰賢之房曰然則今何以知其不賢也上曰以其時亂而君危知之房曰若是任賢必治任不肖必亂必然之道也【治直吏翻下同】幽厲何不覺悟而更求賢曷為卒任不肖以至於是【師古曰卒終也音子恤翻】上曰臨亂之君各賢其臣令皆覺悟天下安得危亡之君房曰齊桓公秦二世亦嘗聞此君而非笑之然則任竪刁趙高政治日亂盗賊滿山【竪刁注見十八卷武帝元光五年趙高事見秦紀】何不以幽厲卜之而覺寤乎【以龜卜所以驗吉凶以幽厲卜所以驗治亂】上曰唯有道者能以往知來耳房因免冠頓首曰春秋紀二百四十二年災異以示萬世之君今陛下即位以來日月失明星辰逆行山崩泉湧地震石隕夏霜冬靁春凋秋榮隕霜不殺水旱螟蟲民人饑疫盗賊不禁刑人滿市春秋所記災異盡備【師古曰言今皆備有之靁古雷字春秋所記隱公十一年桓十八年莊公三十二年閔公二年僖公三十三年文公十八年宣公十八年成公十八年襄公三十一年昭公三十二年定公十五年哀公十四年凡二百四十二年】陛下視今為治邪亂邪上曰亦極亂耳尚何道房曰今所任用者誰與【道言也師古曰與讀曰歟 考異曰故資政殿學士邵亢得兩淅錢王寫本漢書無亂邪二字有上曰亦極亂耳尚何道房曰今十二字今取之】上曰然幸其愈於彼又以為不在此人也【師古曰愈猶勝也言今之災異及政道猶幸勝於往日又不由所任之人】房曰夫前世之君亦皆然矣臣恐後之視今猶今之視前也上良久乃曰今為亂者誰哉房曰明主宜自知之上曰不知也如知何故用之房曰上最所信任與圖事帷幄之中【師古曰圖謀也】進退天下之士者是矣房指謂石顯上亦知之謂房曰已諭【師古曰言已曉此意】房罷出後上亦不能退顯也臣光曰人君之德不明則臣下雖欲竭忠何自而入乎觀京房所以曉孝元可謂明白切至矣而終不能寤悲夫詩曰匪面命之言提其耳匪手之言示之事又曰誨爾諄諄聽我藐藐【皆大雅抑詩之辭也鄭氏箋曰言我非但以手掣之親示以其事之是非我非但對面告語之親提撕其耳此言以教導之熟不可啟覺也藐藐然不入也我教告王口語諄諄然王聽聆之藐藐然諄之純翻又之閏翻藐美角翻爾雅云悶也】孝元之謂矣<br />
<br />
  上令房上弟子曉知考功課吏事者欲試用之房上中郎任良姚平願以為刺史試考功法臣得通籍殿中為奏事以防壅塞石顯五鹿充宗皆疾房欲遠之【上時掌翻為于偽翻塞悉則翻遠于願翻師古曰欲出之令遠去】建言宜試以房為郡守【師古曰立議云然也】帝於是以房為魏郡太守得以考功法治郡【治直之翻】房自請歲竟乘傳奏事【歲竟歲終也傳知戀翻下同】天子許焉房自知數以議論為大臣所非【數所角翻】與石顯等有隙不欲遠離左右【離力智翻】乃上封事曰臣出之後恐為用事所蔽身死而功不成故願歲盡乘傳奏事蒙哀見許【言蒙帝哀憐而許之】乃辛巳蒙氣復乘卦太陽侵色【張晏曰晉卦解卦也太陽侵色謂大壮也原父曰蒙氣起而太陽侵色則太陽指日也大壮解卦可云太陽而非所侵色也京房易傳曰蒙如塵雲臣私禄及親兹謂罔辟厥異蒙大臣厭小臣兹謂蔽蒙微日不明若解不解晉書天文志曰凡連隂十日晝不見日夜不見月亂風四起欲雨而無雨名曰蒙復扶又翻下同】此上大夫覆陽而上意疑也【據孟康注房以消息卦為辟辟君也息卦曰太陰消卦曰太陽其餘卦曰少隂少陽謂臣下也上大夫覆陽蓋以是候之師古曰覆掩蔽也音敷救翻】己卯庚辰之間必有欲隔絶臣令不得乘傳奏事者【以辛巳蒙氣占己卯庚辰二日也】房未發上令陽平侯王鳳承制詔房止無乘傳奏事【鳳陽平侯王禁之子】房意愈恐秋房去至新豐因郵上封事【師古曰郵行書者也若今傳送文書矣郵音尤】曰臣前以六月中言遯卦不效法曰道人始去寒涌水為災【法者房占候之法著之于書者也師古曰道人有道術之人也天氣寒又有水涌出也】至其七月涌水出臣弟子姚平謂臣曰房可謂知道未可謂信道也房言災異未嘗不中【中竹仲翻】涌水已出道人當逐死尚復何言臣曰陛下至仁於臣尤厚雖言而死臣猶言也【師古曰自云不避死也】平又曰房可謂小忠未可謂大忠也【小忠謂以諫殺身而無益於國大忠謂諫行言聽而身與國同休也】昔秦時趙高用事有正先者非刺高而死【孟康曰姓正名先秦博士也姓譜正姓宋上卿正考父之後】高威自此成故秦之亂正先趣之【師古曰趣讀曰促】今臣得出守郡自詭効功恐未効而死惟陛下毋使臣塞涌水之異【師古曰塞當也】當正先之死為姚平所笑房至陕【陕縣屬弘農郡周召二伯東西分治以陕為界即此地也陕音式冉翻】復上封事曰臣前白願出任良試考功臣得居内議者知如此於身不利臣不可蔽故云使弟子不若試師臣為刺史又當奏事故復云為刺史恐太守不與同心不若以為太守此其所以隔絶臣也陛下不違其言而遂聽之此乃蒙氣所以不解太陽無色者也臣去稍遠太陽侵色益甚願陛下毋難還臣而易逆天意【師古曰易輕也音弋豉翻】邪說雖安于人天氣必變【言人君雖安其邪說而不之覺天氣必為之變而失其常】故人可欺天不可欺也願陛下察焉房去月餘竟徵下獄初淮陽憲王舅張博傾巧無行【淮陽憲王欽宣帝張偼伃之子帝弟也下遐稼翻行下孟翻】多從王求金錢欲為王求入朝博從京房學以女妻房房每朝見退輒為博道其語【師古曰所與天子言者皆具說之為于偽翻妻七細翻】博因記房所說密語令房為王作求朝奏草【師古曰草謂為文之藁草也論語孔子曰為命禆諶草創之】皆持東與王以為信驗【淮陽國在關東】石顯知之告房與張博通謀非謗政治歸惡天子詿誤諸侯王【治直吏翻詿古賣翻】皆下獄棄市 【考異曰元紀及荀紀京房死皆在此年末按房傳二月末上封事去月餘徵下獄百官表八月癸亥匡衡為御史大夫房死必不在歲末也紀不知月日故繫之歲末耳】妻子徙邊鄭弘坐與房善免為庶人【免御史大夫也】 御史中丞陳咸數毁石顯【數所角翻下同】久之坐與槐里令朱雲善漏洩省中語【時丞相韋玄成言雲暴虐無狀陳咸在前聞之以語雲雲上書自訟顯以此奏咸漏洩省中語高帝三年改廢丘為槐里屬右扶風】石顯微伺知之與雲皆下獄髠為城旦石顯威權日甚公卿以下畏顯重足一迹【師古曰言極恐懼不敢自寛縱也重足龍翻重足累足也累足而立故一迹】顯與中書僕射牢梁少府五鹿充宗結為黨友【姓譜牢姓孔子弟子琴牢之後】諸附倚者皆得寵位民歌之曰牢邪石邪五鹿客邪印何纍纍綬若若邪【師古曰纍纍重積也若若長貌纍音力追翻】顯内自知擅權事柄在掌握恐天子一旦納用左右耳目以間已【間音工莧翻】乃時歸誠取一信以為驗顯嘗使至諸官有所徵發【使疏吏翻將命曰使諸官諸官府也】顯先自白恐後漏盡宮門閉請使詔吏開門上許之顯故投夜還稱詔開門入後果有上書告顯顓命矯詔開宮門天子聞之笑以其書示顯顯因泣曰陛下過私小臣屬任以事【師古曰過猶誤也屬委也屬音之欲翻】羣下無不嫉妬欲䧟害臣者事類如此非一唯獨明主知之愚臣微賤誠不能以一軀稱快萬衆【稱音尺孕翻】任天下之怨【師古曰任猶當也任音壬】臣願歸樞機職受後宮掃除之役死無所恨唯陛下哀憐財幸【師古曰財與裁同】以此全活小臣天子以為然而憐之數勞勉顯加厚賞賜【數所角翻勞力到翻】賞賜及賂遺訾一萬萬【遺于季翻師古曰賂遺謂百官羣下所遺也訾讀與貲同】初顯聞衆人匈匈言已殺前將軍蕭望之【事見上卷初元二年】恐天下學士訕已【師古曰訕謗也音所諫翻】以諫大夫貢禹明經著節乃使人致意深自結納因薦禹天子歷位九卿禮事之甚備議者於是或稱顯以為不妬譛望之矣【薦貢禹事當在譛殺京房䧟陳咸之前故以初字發語】顯之設變詐以自解免取信人主者皆此類也<br />
<br />
  荀悦曰夫佞臣之惑君主也甚矣故孔子曰遠佞人【論語孔子告顔淵之言遠于願翻下同】非但不用而已乃遠而絶之隔塞其源【塞悉則翻】戒之極也孔子曰政者正也【論語孔子答季康子之言】夫要道之本正已而已矣平直真實者正之主也故德必核其真然後授其位能必核其真然後授其事功必核其真然後授其賞罪必核其真然後授其刑行必核其真然後貴之言必核其真然後信之物必核其真然後用之事必核其真然後修之【核與覈同謂精確得其實也行下孟翻】故衆正積于上萬事實于下先王之道如斯而已矣<br />
<br />
  八月癸亥以光禄勲匡衡為御吏大夫【鄭弘坐京房免以衡代之】閏月丁酉太皇太后上官氏崩【此昭帝上官后也】 冬十一月齊楚地震【此指齊楚古國之大界】大雨雪樹折屋壞【雨于具翻折而設翻】三年夏六月甲辰扶陽共侯韋玄成薨【共音恭】 秋七月匡衡為丞相戊辰衛尉李延壽為御史大夫 冬使西域都護騎都尉北地甘延壽副校尉山陽陳湯【師古曰言延壽及湯本充西域之使故先言使而後序其官職及姓名使疏吏翻】共誅斬郅支單于於康居始郅支單于自以大國威名尊重又乘勝驕【郅支嘗破殺閏振攻破呼韓邪又殺伊利目屢破烏孫兵故乘勝氣而驕也】不為康居王禮怒殺康居王女【康居王以女妻郅支事見上卷初元五年】及貴人人民數百或支解投都賴水中【師古曰支解謂截其四支都賴郅支水名余謂都賴水在康居國郅支城旁】發民作城日作五百人二歲乃已又遣使責闔蘇大宛諸國歲遺【師古曰胡廣云康居北可一千里有國名奄蔡一名闔蘇然則闔蘇即奄蔡也歲遺者年常所獻之物遺弋季翻】不敢不予【予讀曰與】漢遣使三輩至康居求谷吉等死【殺谷吉見上卷初元五年師古曰死尸也】郅支困辱使者不肯奉詔而因都護上書言居困戹願歸計彊漢遣子入侍【師古曰故為此言以調戲也歸計謂歸附而受計策也】其驕嫚如此湯為人沈勇有大慮【沈持林翻】多策畧喜奇功【師古曰喜音許吏翻】與延壽謀曰夷狄畏服大種其天性也【種章勇翻】西域本屬匈奴【武帝雖通西域匈奴猶役屬之至宣帝時朝呼韓邪降日逐西域乃咸屬漢】今郅支單于威名遠聞【聞音問】侵陵烏孫大宛【宛於元翻】常為康居畫計欲降伏之【為于偽翻降戶江翻】如得此二國數年之間城郭諸國危矣且其人剽悍【師古曰剽輕也悍勇也剽平妙翻又匹妙翻悍胡幹翻又下罕翻】好戰伐數取勝久畜之必為西域患【好呼到翻數所角翻畜許六翻】雖所在絶遠蠻夷無金城彊弩之守如發屯田吏士【即屯田車師者】從烏孫衆兵【師古曰驅率之令隨從也敺與驅同下同】直指其城下彼亡則無所之守則不足自保【師古曰之往也保安也】千載之功可一朝而成也【載子亥翻】延壽以為然欲奏請之湯曰國家與公卿議【此時已稱天子為國家非至東都始然也】大策非凡所見事必不從【師古曰言凡庸之人不能遠見將壞其事也】延壽猶與不聽【與讀曰豫即猶豫也】會其久病湯獨矯制發城郭諸國兵車師戊巳校尉屯田吏士【戊巳校尉屯田車師】延壽聞之驚起欲止焉湯怒按劎叱延壽曰大衆已集會孺子欲沮衆邪【師古曰沮止也壞也音材汝翻】延壽遂從之部勒行陳【行戶剛翻陳讀曰陣】漢兵胡兵合四萬餘人延壽湯上疏自劾奏矯制【劾戶槩翻自劾自奏其矯制之罪也】陳言兵狀即日引軍分行别為六校【别彼列翻分也按湯傳益置陽威合騎白虎之校併副校尉戊校尉巳校尉為六校校戶教翻】其三校從南道踰蔥嶺徑大宛其三校都護自將發温宿國從北道入赤谷過烏孫【温宿國東至都護治所二千三百八十里北至烏孫赤谷六百一十里】涉康居界至闐池西而康居副王抱闐將數千騎寇赤谷城東【文頴曰闐音填】殺畧大昆彌千餘人畜產甚多從後與漢軍相及頗寇盗後重【師古曰重謂輜重也音直用翻】湯縱胡兵擊之殺四百六十人得其所畧民四百七十人還付大昆彌其馬牛羊以給軍食又捕得抱闐貴人伊奴毒入康居東界令軍不得為寇【師古曰勿抄掠也】閒呼其貴人屠墨見之【師古曰閒謂密呼也閒古莧翻】諭以威信與飲盟遣去【既與之飲又與之盟】徑引行未至單于城可六十里止營復捕得康居貴人貝色子男開牟以為導【復扶又翻】貝色子即屠墨母之弟【師古曰母之弟即謂舅者】皆怨單于由是具知郅支情明日引行未至城三十里止營單于遣使問漢兵何以來應曰單于上書言居困戹願歸計彊漢身入朝見【朝直遙翻見賢遍翻】天子哀閔單于棄大國屈意康居故使都護將軍來迎單于妻子【當此時甘延夀止為西域都護以將兵故稱將軍至光武時遂以賈復為都護將軍復之都護蓋護諸將也】恐左右驚動故未敢至城下使數往來相答報【數所角翻下同】延壽湯因讓之【師古曰讓責也】我為單于遠來【為于偽翻】而至今無名王大人見將軍受事者【師古曰名王諸王之貴者受事受教命而供事也】何單于忽大計失客主之禮也【師古曰忽忘也又輕也】兵來道遠人畜罷極食度且盡【師古曰罷讀曰疲度音大各翻】恐無以自還願單于與大臣審計策明日前至郅支城都賴水上離城三里【離力智翻】止營傅陳【師古曰傅讀曰敷布也陳讀曰陣下同】望見單于城上立五采幡幟【幟昌志翻】數百人被甲乘城【師古曰乘謂登之備守也被皮義翻下同】又出百餘騎往來馳城下步兵百餘人夾門魚鱗陳【師古曰言其相接次形若魚鱗】講習用兵城上人更招漢軍曰鬬來【師古曰更互也音工衡翻】百餘騎馳赴營營皆張弩持滿指之騎引却頗遣吏士射城門騎步兵【射而亦翻下同】騎步兵皆入延壽湯令軍聞鼓音皆薄城下【薄伯各翻】四面圍城各有所守穿塞門戶【即塹字音尺艷翻塞悉則翻】鹵楯為前戟弩為後【楯食尹翻】仰射城樓上人樓上人下走土城外有重木城【重直龍翻】從木城中射頗殺傷外人外人發薪燒木城【孔穎達曰薪樵也大樵曰薪詩曰析薪如之何匪斧不克是大故用斧也】夜數百騎欲出外迎射殺之初單于聞漢兵至欲去疑康居怨已為漢内應又聞烏孫諸國兵皆發自以無所之【言郅支自計無所往而可也】郅支已出復還【復扶又翻】曰不如堅守漢兵遠來不能久攻單于乃被甲在樓上【被皮義翻】諸閼氏夫人數十皆以弓射外人外人射中單于鼻【中竹仲翻】諸夫人頗死單于乃下夜過半木城穿中人郤入土城乘城呼【中人木城中人也師古曰呼音火故翻下同】時康居兵萬餘騎分為十餘處四面環城亦與相應和【師古曰環繞也音宦和胡臥翻】夜數犇營不利輒郤平明四面火起吏士喜大呼乘之【師古曰乘逐也余謂乘駕也乘火起之勢而駕之也】鉦鼓聲動地【鉦音征鐃也其狀似鈴杜佑曰鐲鉦也形如小鍾軍行鳴之以為鼓節周禮以金鐲節鼓近代有大銅疊懸而擊之以節鼓曰鉦】康居兵引郤漢兵四面推鹵楯【推吐雷翻】並入土城中單于男女百餘人走入大内【師古曰大内單于之内室也】漢兵縱火吏士爭入單于被創死【被皮義翻創初良翻】軍候假丞杜勲斬單于首【漢制軍行有各部校尉部下有曲曲有軍候一人又都護有副校尉秩比二千石丞一人司馬候千人各二人杜勲本為軍候而假丞也】得漢使節二及谷吉等所齎帛書諸鹵獲以畀得者【師古曰畀與也各以與所得人音必寐翻】凡斬閼氏太子名王以下千五百一十八級【閼於焉翻氏音支】生虜百四十五人降虜千餘人賦予城郭諸國所發十五王【降戶江翻師古曰賦謂班與之也所發十五王謂所發諸國之王領兵共圍郅支單于者也予讀曰與】<br />
<br />
  四年春正月郅支首至京師延壽湯上疏曰臣聞天下之大義當混為一【師古曰混同也音胡本翻余謂混為一者合四海之内同禀命於一人天下之治出於一也】昔有唐虞今有彊漢匈奴呼韓邪單于已稱北藩唯郅支單于叛逆未伏其辜大夏之西以為彊漢不能臣也【大夏西域國名在大宛西南師古曰謂漢為不能使郅支臣服也】郅支單于慘毒行於民大惡通於天臣延壽臣湯將義兵行天誅賴陛下神靈陰陽並應天氣精明䧟陳克敵【陳讀曰陣】斬郅支首及名王以下宜縣頭稾街蠻夷邸間【晉灼曰稾街黄圖在長安城門内師古曰稾街街名蠻夷邸在此街也邸若今客館也又曰蠻夷邸若今鴻臚舘崔浩以為稾當為槖槖街即銅駞街也此說失之銅駞街在雒陽西京無也縣讀曰懸下同】以示萬里明犯彊漢者雖遠必誅丞相匡衡等以為方春掩骼埋胔之時【禮記月令孟春掩骼埋胔注云謂死氣逆生氣也應劭曰禽獸之骨曰骼骼大也鳥鼠之骨曰胔胔可惡也臣瓚曰枯骨曰骼有肉曰胔師古曰瓚說是也骼音工客翻胔音才賜翻】宜勿縣詔縣十日乃埋之仍告祠郊廟赦天下羣臣上壽置酒 六月甲申中山哀王竟薨哀王者帝之少弟與太子游學相長大【游謂宴游學謂講學師古曰同處長養以至於壮大少詩照翻長知兩翻】及薨太子前弔上望見太子感念哀王悲不能自止太子既至前不哀上大恨曰安有人不慈仁而可以奉宗廟為民父母者乎是時駙馬都尉侍中史丹護太子家【護監護也】上以責謂丹【師古曰謂者告語也】丹免冠謝曰臣誠見陛下哀痛中山王至以感損【謂哀感而神氣為之耗損】向者太子當進見【見賢遍翻下同】臣竊戒屬毋涕泣感傷陛下【師古曰屬音之欲翻】罪乃在臣當死上以為然意乃解 藍田地震山崩壅霸水安陵岸崩壅涇水涇逆流【藍田縣屬京兆水經霸水出藍田縣藍田谷過覇陵縣西北流注於渭孟康曰安陵岸惠帝陵旁涇水岸也】<br />
<br />
  五年春三月赦天下 夏六月庚申復戾園 壬申晦日有食之 秋七月庚子復太上皇寢廟園原廟昭靈后武哀王昭哀后衛思后園時上寢疾久不平以為祖宗譴怒故盡復之唯郡國廟遂廢云【永光四年罷園廟】 是歲徙濟陽王康為山陽王【濟子禮翻】 匈奴呼韓邪單于聞郅支既誅且喜且懼【喜者以郅支既誅而已無後患也懼者以漢威彊懼復得罪而滅亡如郅支也】上書願入朝見【朝直遥翻】<br />
<br />
  竟寧元年【應劭曰呼韓邪單于願保塞邊竟得以安寧故以冠元也師古曰據如應說竟讀為境古之用字境竟實同但詔云長無兵革之事竟者終極之言言永永安寧也既無兵革中外安寧豈止境上若依本字而讀義更弘通也】春正月匈奴呼韓邪單于來朝自言願壻漢氏以自親【師古曰言欲取漢女而身為漢家壻】帝以後宮良家子王嬙字昭君賜單于【嬙音牆】單于驩喜上書願保塞上谷以西至燉煌【師古曰保守也自請保守之令無寇盗燉徒門翻】傳之無窮請罷邊備塞吏卒以休天子人民天子下有司議【下遐稼翻】議者皆以為便郎中侯應習邊事以為不可許上問狀應曰周秦以來匈奴暴桀寇侵邊境漢興尤被其害【被皮義翻下同】臣聞北邊塞至遼東外有陰山東西千餘里草木茂盛多禽獸本冒頓單于依阻其中治作弓矢來出為寇是其苑囿也【冒如字又莫比翻治直之翻】至孝武世出師征伐斥奪此地攘之於幕北【師古曰斥聞也攘郤也音人羊翻】建塞徼起亭隧【徼吉弔翻境也小路也循察也師古曰隧謂深開小道而行避敵抄寇也音隧】築外城設屯戍以守之然後邊境用得少安幕北地平少草木多大沙【所謂大磧也少詩沼翻下同】匈奴來寇少所蔽隱從塞以南徑深山谷往來差難邊長老言匈奴失陰山之後過之未嘗不哭也【長知兩翻】如罷備塞吏卒示夷狄之大利不可一也今聖德廣被天覆匈奴【師古曰如天之覆也被皮義翻覆敷又翻】匈奴得蒙全活之恩稽首來臣夫夷狄之情困則卑順彊則驕逆天性然也前已罷外城【事見二十四卷宣帝地節二年】省亭隧纔足以候望通烽火而已古者安不忘危不可復罷二也【復扶又翻】中國有禮義之教刑罰之誅愚民猶尚犯禁又况單于能必其衆不犯約哉三也【師古曰必極也極保之也毛晃曰必定辭也】自中國尚建關梁以制諸侯【關梁設於水陸要會之處因山陿而設塞以譏陸行者為關或立石或架木或維舟絶水以譏舟行者為梁】所以絶臣下之覬欲也【師古曰覬音冀】設塞徼置屯戍非獨為匈奴而已亦為諸屬國降民本故匈奴之人恐其思舊逃亡四也【為于偽翻】近西羌保塞與漢人交通吏民貪利侵盗其畜產妻子以此怨恨起而背畔【背蒲妹翻】今罷乘塞則生嫚易分爭之漸五也【師古曰乘塞登之而守也嫚易猶欺侮也易音弋豉翻】往者從軍多没不還者子孫貧困一旦亡出從其親戚六也又邊人奴婢愁苦欲亡者多日聞匈奴中樂【樂音洛】無奈候望急何然時有亡出塞者七也盗賊桀黠羣輩犯法如其窘急亡走北出則不可制八也【黠下八翻】起塞以來百有餘年【自武帝起塞時至此時百有餘年】非皆以土垣也或因山巖石木谿谷水門稍稍平之卒徒築治功費久遠不可勝計【治直之翻下同勝音升】臣恐議者不深慮其終始欲以壹切省繇戍【師古曰壹切者權時之事非經常也猶如以刀切物以取整齊不顧長短縱横故言壹切繇古徭字通】十年之外百歲之内卒有他變障塞破壞亭隧滅絶當更發屯繕治累歲之功不可卒復九也【師古曰卒皆讀曰猝】如罷戍卒省候望單于自以保塞守禦必深德漢【師古曰於漢自稱恩德也】請求無已小失其意則不可測開夷狄之隙虧中國之固十也非所以永持至安威制百蠻之長策也對奏天子有詔勿議罷邊塞事使車騎將軍嘉口諭單于【師古曰將軍許嘉也諭謂曉告】曰單于上書願罷北塞吏士屯戍子孫世世保塞單于鄉慕禮義【鄉讀曰嚮】所以為民計者甚厚此長久之策也朕甚嘉之中國四方皆有關梁障塞非獨以備塞外也亦以防中國姦邪放縱出為寇害故明法度以專衆心也【專壹也】敬諭單于之意【師古曰言已曉知其意也】朕無疑焉為單于怪其不罷【為于偽翻下同】故使嘉曉單于【毛晃曰曉開諭也】單于謝曰愚不知大計天子幸使大臣告語甚厚【語牛倨翻】初左伊秩訾為呼韓邪畫計歸漢【事見二十七卷宣帝甘露元年】竟以安定其後或讒伊秩訾自伐其功【師古曰伐謂矜其功力余謂此言其矜畫計定匈奴之功耳非力也】常鞅鞅呼韓邪疑之伊秩訾懼誅將其衆千餘人降漢漢以為關内侯食邑三百戶令佩其王印綬【師古曰雖於漢為關内侯而依匈奴王號與印綬】及呼韓邪來朝與伊秩訾相見謝曰王為我計甚厚令匈奴至今安寧王之力也德豈可忘我失王意使王去不復顧留【師古曰言不復顧念而留住匈奴中】皆我過也今欲白天子請王歸庭【歸單于庭也】伊秩訾曰單于賴天命自歸于漢得以安寧單于神靈天子之祐也我安得力既已降漢又復歸匈奴是兩心也願為單于侍使於漢不敢聽命【師古曰言為單于充使留侍于漢不能還匈奴使疏吏翻】單于固請不能得而歸單于號王昭君為寧胡閼氏【師古曰言胡得之國以安寧也】生一男伊屠智牙師為右日逐王 皇太子冠【冠古玩翻】 二月御史大夫李延夀卒 初石顯見馮奉世父子為公卿著名女又為昭儀在内【馮昭儀即馮倢伃進號昭儀】顯心欲附之薦言昭儀兄謁者逡脩敕宜侍幄帷【師古曰逡音千句翻敕整也】天子召見欲以為侍中【見賢遍翻】逡請間言事上聞逡言顯專權大怒罷逡歸郎官及御史大夫缺在位多舉逡兄大鴻臚野王【臚陵如翻】上使尚書選第中二千石【選第者選其有行能者而第其高下之次也】而野王行能第一【行下孟翻】上以問顯顯曰九卿無出野王者然野王親昭儀兄臣恐後世必以陛下度越衆賢【師古曰度過也】私後宮親以為三公上曰善吾不見是【師古曰言不見此理】因謂羣臣曰吾用野王為三公後世必謂我私後宮親屬以野王為比【師古曰比例也音必寐翻當音毗寐翻】三月丙寅詔曰剛彊堅固確然亡欲大鴻臚野王是也【亡古無字通】心辯善辭【言心辯於是非而善於詞令辯别也】可使四方少府五鹿充宗是也【使疏吏翻】亷潔節儉太子少傅張譚是也其以少傅為御史大夫【以詔褒之次第不用五鹿充宗而用張譚何也帝亦知充宗為石顯之黨也】 河南太守九江召信臣為少府信臣先為南陽太守後遷河南治行常第一視民如子好為民興利躬勸耕稼開通溝瀆戶口增倍吏民親愛號曰召父【師古曰召讀曰邵治直吏翻行下孟翻好呼到翻】 癸卯復孝惠皇帝寢廟園孝文太后孝昭太后寢園【永光五年毁惠園建昭元年罷孝文太后孝昭太后寢園孝昭太后孝武帝鉤弋趙倢伃也葬雲陽甘泉宮南】 初中書令石顯嘗欲以姊妻甘延壽【妻七細翻】延壽不取及破郅支還【還從宣翻又如字】丞相御史亦惡其矯制【惡烏路翻】皆不與延壽等【師古曰與猶許也】陳湯素貪所鹵獲財物入塞多不法【師古曰不法者私自取之不依軍法余謂不法者以外國財物闌入邊闈也】司隸校尉移書道上【移書所過道上郡縣也】繫吏士按驗之湯上疏言臣與吏士共誅郅支單于幸得禽滅萬里振旅【師古曰師入曰振旅】宜有使者迎勞道路【勞來到翻】今司隸反逆收繫按驗【當勞來而收繫是於事理為反也逆迎也】是為郅支報讐也【為于偽翻】上立出吏士令縣道出酒食以過軍【漢制縣有蠻夷曰道】既至論功石顯匡衡以為延壽湯擅興師矯制幸得不誅如復加爵土【復扶又翻】則後奉使者爭欲乘危徼幸生事於蠻夷為國招難【徼古堯翻又一遥翻難乃旦翻】帝内嘉延壽湯功而重違衡顯之議【師古曰重難也】久之不决故宗正劉向上疏曰【帝初即位劉向為宗正免官久矣故曰故宗正向本名更生至是改名】郅支單于囚殺使者吏士以百數事暴揚外國傷威毁重羣臣皆閔焉【師古曰閔病也】陛下赫然欲誅之意未嘗有忘西域都護延壽副校尉湯承聖指倚神靈總百蠻之君攬城郭之兵【意之所向為指師古曰攬總持之也】出百死入絶域遂蹈康居屠三重城【郅支城木城再重并土城為三重重直龍翻】搴歙侯之旗【師古曰搴拔也音騫歙許及翻】斬郅支之首縣旌萬里之外【縣讀曰懸】揚威昆山之西【昆山指言崑崙山也】埽谷吉之恥【谷吉為郅支所殺見上卷初元五年】立昭明之功【昭明謂顯功也】萬夷慴服【師古曰慴恐也音之涉翻】莫不懼震呼韓邪單于見郅支已誅且喜且懼鄉風馳義【鄉讀曰嚮師古曰馳義慕義驅馳而來也】稽首來賓願守北藩累世稱臣立千載之功【載子亥翻】建萬世之安羣臣之勲莫大焉昔周大夫方叔吉甫為宣王誅獫狁而百蠻從【為于偽翻】其詩曰嘽嘽焞焞如霆如雷顯允方叔征伐獫狁蠻荆來威【師古曰小雅采芑之詩也嘽嘽衆也焞焞盛也言車徒既衆且盛故能克定獫狁而令荆土之蠻亦畏威而來也顯明也允信也獫音虚檢翻狁音庾凖翻嘽音他丹翻焞音土回翻】易曰有嘉折首獲匪其醜【師古曰離上九爻辭也嘉善也醜類也言王者出征克勝斬首多獲非類故以為善】言美誅首惡之人而諸不順者皆來從也今延壽湯所誅震雖易之折首詩之雷霆不能及也論大功者不録小過舉大美者不疵細瑕司馬法曰軍賞不踰月欲民速得為善之利也蓋急武功重用人也吉甫之歸周厚賜之其詩曰吉甫燕喜既多受祉來歸自鎬我行永久【師古曰小雅六月之詩也鎬地名非豐鎬之鎬此鎬及方皆在周之北時獫狁侵鎬及方至于涇陽吉甫薄伐自鎬而還王以宴禮樂之多受福賜以其行役有功日月長久故也吉甫尹吉甫也】千里之鎬猶以為遠况萬里之外其勤至矣延壽湯既未獲受祉之報反屈捐命之功久挫於刀筆之前【師古曰捐其軀命言無所顧也挫屈折也刀筆吏也】非所以厲有功勸戎士也昔齊桓前有尊周之功後有滅項之罪君子以功覆過而為之諱【師古曰尊周謂伐楚責苞茅及會王世子于首止項國名也春秋僖十七年夏滅項公羊傳曰齊滅之也不言齊為桓公諱也桓嘗有繼絶存亡之功故君子為之諱覆敷救翻為于偽翻】貳師將軍李廣利捐五萬之師靡億萬之費經四年之勞而僅獲駿馬三十匹雖斬宛王毋寡之首猶不足以復費其私罪惡甚多孝武以為萬里征伐不録其過遂封拜兩侯三卿二千石百有餘人【事見二十一卷武帝太初三年四年師古曰靡散也音縻僅少也復償也音扶目翻】今康居之國彊於大宛郅支之號重於宛王殺使者罪甚於留馬而延壽湯不煩漢士不費斗糧比於貳師功德百之【師古曰百倍勝之】且常惠隨欲擊之烏孫【事見二十四卷宣帝本始三年】鄭吉迎自來之日逐【事見二十六卷宣帝神爵二年】猶且裂土受爵故言威武勤勞則大於方叔吉甫列功覆過則優於齊桓貳師【覆敷又翻】近事之功則高於安遠長羅【師古曰安遠侯鄭吉長羅侯常惠也】而大功未著小惡數布【數所角翻】臣竊痛之宜以時解縣通籍【孟康曰縣罪未竟也如言縣罰也通籍者不禁止令得出入也縣讀曰懸】除過勿治尊寵爵位以勸有功於是天子下詔赦延壽湯罪勿治令公卿議封焉議者以為宜如軍法捕斬單于令匡衡石顯以為郅支本亡逃失國竊號絶域非真單于帝取安遠侯鄭吉故事封千戶衡顯復爭【復扶又翻下同】夏四月戊辰封延壽為義成侯【地理志沛郡有義成侯國】賜湯爵關内侯食邑各三百戶加賜黄金百斤拜延壽為長水校尉湯為射聲校尉於是杜欽上疏追訟馮奉世前破莎車功【事見二十五卷宣帝元康元年】上以先帝時事不復録欽故御史大夫延年子也<br />
<br />
  荀悦論曰誠其功義足封追録前事可也春秋之義毁泉臺則惡之舍中軍則善之【春秋公羊傳文公十六年毁泉臺何以書譏何譏爾先祖為之已毁之不如勿居而已矣昭五年舍中軍舍中軍者何復古也穀梁之義略同惡烏路翻】各由其宜也夫矯制之事先王之所慎也不得已而行之若矯大而功小者罪之可也矯小而功大者賞之可也功過相敵如斯而已可也權其輕重而為之制宜焉【為于偽翻】<br />
<br />
  初太子少好經書【少詩照翻好呼到翻下同】寛博謹慎其後幸酒樂燕樂【晉灼曰幸酒好酒也師古曰樂燕樂若論語稱孔子云損者三樂樂驕樂樂逸遊樂宴樂損矣燕樂燕私之樂也上樂讀如本字又音五孝翻下樂音來各翻】上不以為能而山陽王康有才萟母傅昭儀又愛幸上以故常有意欲以山陽王為嗣上晚年多疾不親政事留好音樂【孟康曰留意於音樂】或置鼙鼓殿下天子自臨軒檻上隤銅丸以擿鼓【師古曰鼙本騎上之鼓音步迷翻檻軒闌版也隤下也擿投也隤音頹擿音持益翻一曰擿磓也音丁力翻磓音丁回翻】聲中嚴鼓之節【李奇曰莊嚴之鼔節也晉灼曰疾擊之鼔也師古曰晉說是中竹仲翻】後宮及左右習知音者莫能為而山陽王亦能之上數稱其材【數所角翻下同】史丹進曰凡所謂材者敏而好學温故知新【師古曰敏速疾也温厚也温故厚蓄故事也何晏曰温尋也尋繹故者又知新者余謂何說是】皇太子是也若乃器人於絲竹鼙鼓之間則是陳惠李微高於匡衡可相國也【如淳曰器人取人器能也陳惠李微二人好音者也服䖍曰二人皆黄門鼔吹也】於是上嘿然而笑【聞丹言而嘿然已而笑】及上寢疾傅昭儀山陽王康常在左右而皇后太子希得進見【見賢遍翻】上疾稍侵意忽忽不平【師古曰稍侵言漸篤也平和也】數問尚書以景帝時立膠東王故事【事見十六卷景帝前六年】是時太子長舅陽平侯王鳳為衛尉侍中【長知兩翻下同】與皇后太子皆憂不知所出【師古曰不知計所出】史丹以親密臣得侍視疾候上間獨寢時丹直入卧内頓首伏青蒲上【服䖍曰青緣蒲席也應劭曰以青規地曰青龍自非皇后不得至此孟康曰蒲青為席用蔽地也師古曰應說是也】涕泣而言曰皇太子以適長立積十餘年【師古曰適讀曰嫡】名號繫於百姓天下莫不歸心臣子【師古曰自託為臣子仲馮曰臣子當屬下句不當斷之余以下文大意觀之顔說為是】見山陽王雅素愛幸今者道路流言為國生意【為于偽翻】以為太子有動揺之議審若此公卿以下必以死爭不奉詔臣願先賜死以示羣臣天子素仁不忍見丹涕泣言又切至意大感寤喟然太息曰吾日困劣【劣弱也】太子兩王幼少【兩王山陽王康信都王興】意中戀戀亦何不念乎然無有此議且皇后謹慎先帝又愛太子【事見二十七卷宣帝甘露三年】吾豈可違指駙馬都尉安所受此語【師古曰安焉也余謂安何也】丹即却頓首曰愚臣妄聞罪當死【師古曰却退也離青蒲上】上因納謂丹曰吾病寖加恐不能自還【納者納其言也師古曰寖漸也不自還者言當遂至崩亡也還讀曰旋】善輔道太子毋違我意丹嘘唏而起【師古曰道讀曰導嘘音虛唏音許既翻】太子由是遂定為嗣而右將軍光禄大夫王商中書令石顯亦擁佑太子頗有力焉夏五月壬辰帝崩于未央宮【臣瓚曰帝年二十七即位即位十六年夀四十三】<br />
<br />
  班彪贊曰臣外祖兄弟為元帝侍中【應劭曰外祖金敞也】語臣曰元帝多材藝善史書【應劭曰周宣王太史史籒所作大篆語牛倨翻】鼔琴瑟吹洞簫【如淳曰簫之無底者也杜佑曰前代有洞簫今無其器】自度曲被歌聲【應劭曰自隱度作新曲因持新曲以為歌詩聲也荀悦曰被聲能播樂也臣瓚曰度曲謂歌終更授其次謂之度曲西京賦曰度曲未終雲起雪飛張衡舞賦亦曰度終復位次受三八師古曰應荀二說皆是也度音大洛翻被音皮義翻】分刌節度窮極幼眇【蘇林曰寸度也知曲之終始節度也韋昭曰刌切也謂能分切句絶為之節制也師古曰韋說是也刌音千本翻幼眇讀曰要眇】少而好儒【少詩照翻好呼到翻】及即位徵用儒生委之以政貢薛韋匡迭為宰相【師古曰貢禹薛廣德韋玄成匡衡迭互而為丞相也】而上牽制文義優游不斷孝宣之業衰焉【師古曰為文義所牽制故不斷决斷丁亂翻】然寛弘盡下出於恭儉號令温雅有古之風烈<br />
<br />
  匡衡奏言前以上體不平故復諸所罷祠卒不蒙福【師古曰卒終也卒子恤翻】案衛思后戾太子戾后園親未盡【師古曰言不當毁也】孝惠孝景廟親盡宜毁及太上皇孝文孝昭太后昭靈后昭哀后武哀王祠請悉罷勿奉奏可 六月己未太子即皇帝位謁高廟尊皇太后曰太皇太后【宣帝卬成王皇后也】皇后曰皇太后以元舅侍中衛尉陽平侯王鳳為大司馬大將軍領尚書事【王氏得權自此始】 秋七月丙戌葬孝元皇帝于渭陵【臣瓚曰自崩及葬凡五十五日渭陵在長安北五十六里】 大赦天下丞相衡上疏曰陛下秉至孝哀傷思慕不絶於心未<br />
<br />
  有游虞弋射之宴【師古曰虞與娱同】誠隆於慎終追遠無窮已也【師古曰慎終慎孝道之終也追遠不忘本也論語稱曾子曰慎終追遠民德歸厚矣故衡引之】竊願陛下雖聖性得之猶復加聖心焉【師古曰言天性已自然矣又當加意也復扶又翻】詩云焭焭在疚【師古曰周頌閔予小子之詩焭焭憂貌也疚病也焭與煢同渠營翻】言成王喪畢思慕意氣未能平也蓋所以就文武之業崇大化之本也臣又聞之師曰妃匹之際生民之始萬福之原婚姻之禮正然後品物遂而天命全【師古曰遂成也】孔子論詩以關雎為始此綱紀之首王化之端也自上世以來三代興廢未有不由此者也願陛下詳覽得失盛衰之效以定大基采有德戒聲色近嚴敬遠技能【師古曰無德之人雖有技能則斥遠之近其靳翻遠于願翻技渠綺翻】臣聞六經者聖人所以統天地之心著善惡之歸明吉凶之分【師古曰分音扶問翻】通人道之正使不悖於其本性者也【師古曰悖乖也音布内翻】及論語孝經聖人言行之要宜究其意【行下孟翻】臣又聞聖王之自為動静周旋奉天承親臨朝享臣物有節文以章人倫【師古曰物事也事事皆有節文】盖欽翼祗栗事天之容也温恭敬遜承親之禮也正躬嚴恪臨衆之儀也【師古曰嚴讀曰儼】嘉惠和說饗下之顔也【師古曰饗宴饗也說讀曰悦】舉錯動作物遵其儀故形為仁義動為法則今正月初幸路寑【路寑大寑也】臨朝賀置酒以饗萬方傳曰君子慎始願陛下留神動静之節使羣下得望盛德休光以立基楨天下幸甚上敬納其言<br />
<br />
  資治通鑑卷二十九<br />
<br />
<史部,編年類,資治通鑑>  <br>
   </div> 

<script src="/search/ajaxskft.js"> </script>
 <div class="clear"></div>
<br>
<br>
 <!-- a.d-->

 <!--
<div class="info_share">
</div> 
-->
 <!--info_share--></div>   <!-- end info_content-->
  </div> <!-- end l-->

<div class="r">   <!--r-->



<div class="sidebar"  style="margin-bottom:2px;">

 
<div class="sidebar_title">工具类大全</div>
<div class="sidebar_info">
<strong><a href="http://www.guoxuedashi.com/lsditu/" target="_blank">历史地图</a></strong>  
<a href="http://www.880114.com/" target="_blank">英语宝典</a>  
<a href="http://www.guoxuedashi.com/13jing/" target="_blank">十三经检索</a> 
<br><strong><a href="http://www.guoxuedashi.com/gjtsjc/" target="_blank">古今图书集成</a></strong> 
<a href="http://www.guoxuedashi.com/duilian/" target="_blank">对联大全</a> <strong><a href="http://www.guoxuedashi.com/xiangxingzi/" target="_blank">象形文字典</a></strong> 

<br><a href="http://www.guoxuedashi.com/zixing/yanbian/">字形演变</a>  <strong><a href="http://www.guoxuemi.com/hafo/" target="_blank">哈佛燕京中文善本特藏</a></strong>
<br><strong><a href="http://www.guoxuedashi.com/csfz/" target="_blank">丛书&方志检索器</a></strong> <a href="http://www.guoxuedashi.com/yqjyy/" target="_blank">一切经音义</a>  

<br><strong><a href="http://www.guoxuedashi.com/jiapu/" target="_blank">家谱族谱查询</a></strong>  <strong><a href="http://shufa.guoxuedashi.com/sfzitie/" target="_blank">书法字帖欣赏</a></strong> 
<br>

</div>
</div>


<div class="sidebar" style="margin-bottom:0px;">

<font style="font-size:22px;line-height:32px">QQ交流群9:489193090</font>


<div class="sidebar_title">手机APP 扫描或点击</div>
<div class="sidebar_info">
<table>
<tr>
	<td width=160><a href="http://m.guoxuedashi.com/app/" target="_blank"><img src="/img/gxds-sj.png" width="140"  border="0" alt="国学大师手机版"></a></td>
	<td>
<a href="http://www.guoxuedashi.com/download/" target="_blank">app软件下载专区</a><br>
<a href="http://www.guoxuedashi.com/download/gxds.php" target="_blank">《国学大师》下载</a><br>
<a href="http://www.guoxuedashi.com/download/kxzd.php" target="_blank">《汉字宝典》下载</a><br>
<a href="http://www.guoxuedashi.com/download/scqbd.php" target="_blank">《诗词曲宝典》下载</a><br>
<a href="http://www.guoxuedashi.com/SiKuQuanShu/skqs.php" target="_blank">《四库全书》下载</a><br>
</td>
</tr>
</table>

</div>
</div>


<div class="sidebar2">
<center>


</center>
</div>

<div class="sidebar"  style="margin-bottom:2px;">
<div class="sidebar_title">网站使用教程</div>
<div class="sidebar_info">
<a href="http://www.guoxuedashi.com/help/gjsearch.php" target="_blank">如何在国学大师网下载古籍?</a><br>
<a href="http://www.guoxuedashi.com/zidian/bujian/bjjc.php" target="_blank">如何使用部件查字法快速查字?</a><br>
<a href="http://www.guoxuedashi.com/search/sjc.php" target="_blank">如何在指定的书籍中全文检索?</a><br>
<a href="http://www.guoxuedashi.com/search/skjc.php" target="_blank">如何找到一句话在《四库全书》哪一页?</a><br>
</div>
</div>


<div class="sidebar">
<div class="sidebar_title">热门书籍</div>
<div class="sidebar_info">
<a href="/so.php?sokey=%E8%B5%84%E6%B2%BB%E9%80%9A%E9%89%B4&kt=1">资治通鉴</a> <a href="/24shi/"><strong>二十四史</strong></a>&nbsp; <a href="/a2694/">野史</a>&nbsp; <a href="/SiKuQuanShu/"><strong>四库全书</strong></a>&nbsp;<a href="http://www.guoxuedashi.com/SiKuQuanShu/fanti/">繁体</a>
<br><a href="/so.php?sokey=%E7%BA%A2%E6%A5%BC%E6%A2%A6&kt=1">红楼梦</a> <a href="/a/1858x/">三国演义</a> <a href="/a/1038k/">水浒传</a> <a href="/a/1046t/">西游记</a> <a href="/a/1914o/">封神演义</a>
<br>
<a href="http://www.guoxuedashi.com/so.php?sokeygx=%E4%B8%87%E6%9C%89%E6%96%87%E5%BA%93&submit=&kt=1">万有文库</a> <a href="/a/780t/">古文观止</a> <a href="/a/1024l/">文心雕龙</a> <a href="/a/1704n/">全唐诗</a> <a href="/a/1705h/">全宋词</a>
<br><a href="http://www.guoxuedashi.com/so.php?sokeygx=%E7%99%BE%E8%A1%B2%E6%9C%AC%E4%BA%8C%E5%8D%81%E5%9B%9B%E5%8F%B2&submit=&kt=1"><strong>百衲本二十四史</strong></a>  <a href="http://www.guoxuedashi.com/so.php?sokeygx=%E5%8F%A4%E4%BB%8A%E5%9B%BE%E4%B9%A6%E9%9B%86%E6%88%90&submit=&kt=1"><strong>古今图书集成</strong></a>
<br>

<a href="http://www.guoxuedashi.com/so.php?sokeygx=%E4%B8%9B%E4%B9%A6%E9%9B%86%E6%88%90&submit=&kt=1">丛书集成</a> 
<a href="http://www.guoxuedashi.com/so.php?sokeygx=%E5%9B%9B%E9%83%A8%E4%B8%9B%E5%88%8A&submit=&kt=1"><strong>四部丛刊</strong></a>  
<a href="http://www.guoxuedashi.com/so.php?sokeygx=%E8%AF%B4%E6%96%87%E8%A7%A3%E5%AD%97&submit=&kt=1">說文解字</a> <a href="http://www.guoxuedashi.com/so.php?sokeygx=%E5%85%A8%E4%B8%8A%E5%8F%A4&submit=&kt=1">三国六朝文</a>
<br><a href="http://www.guoxuedashi.com/so.php?sokeytm=%E6%97%A5%E6%9C%AC%E5%86%85%E9%98%81%E6%96%87%E5%BA%93&submit=&kt=1"><strong>日本内阁文库</strong></a> <a href="http://www.guoxuedashi.com/so.php?sokeytm=%E5%9B%BD%E5%9B%BE%E6%96%B9%E5%BF%97%E5%90%88%E9%9B%86&ka=100&submit=">国图方志合集</a> <a href="http://www.guoxuedashi.com/so.php?sokeytm=%E5%90%84%E5%9C%B0%E6%96%B9%E5%BF%97&submit=&kt=1"><strong>各地方志</strong></a>

</div>
</div>


<div class="sidebar2">
<center>

</center>
</div>
<div class="sidebar greenbar">
<div class="sidebar_title green">四库全书</div>
<div class="sidebar_info">

《四库全书》是中国古代最大的丛书,编撰于乾隆年间,由纪昀等360多位高官、学者编撰,3800多人抄写,费时十三年编成。丛书分经、史、子、集四部,故名四库。共有3500多种书,7.9万卷,3.6万册,约8亿字,基本上囊括了古代所有图书,故称“全书”。<a href="http://www.guoxuedashi.com/SiKuQuanShu/">详细>>
</a>

</div> 
</div>

</div>  <!--end r-->

</div>
<!-- 内容区END --> 

<!-- 页脚开始 -->
<div class="shh">

</div>

<div class="w1180" style="margin-top:8px;">
<center><script src="http://www.guoxuedashi.com/img/plus.php?id=3"></script></center>
</div>
<div class="w1180 foot">
<a href="/b/thanks.php">特别致谢</a> | <a href="javascript:window.external.AddFavorite(document.location.href,document.title);">收藏本站</a> | <a href="#">欢迎投稿</a> | <a href="http://www.guoxuedashi.com/forum/">意见建议</a> | <a href="http://www.guoxuemi.com/">国学迷</a> | <a href="http://www.shuowen.net/">说文网</a><script language="javascript" type="text/javascript" src="https://js.users.51.la/17753172.js"></script><br />
  Copyright &copy; 国学大师 古典图书集成 All Rights Reserved.<br>
  
  <span style="font-size:14px">免责声明:本站非营利性站点,以方便网友为主,仅供学习研究。<br>内容由热心网友提供和网上收集,不保留版权。若侵犯了您的权益,来信即刪。scp168@qq.com</span>
  <br />
ICP证:<a href="http://www.beian.miit.gov.cn/" target="_blank">鲁ICP备19060063号</a></div>
<!-- 页脚END --> 
<script src="http://www.guoxuedashi.com/img/plus.php?id=22"></script>
<script src="http://www.guoxuedashi.com/img/tongji.js"></script>

</body>
</html>
