<!DOCTYPE html PUBLIC "-//W3C//DTD XHTML 1.0 Transitional//EN" "http://www.w3.org/TR/xhtml1/DTD/xhtml1-transitional.dtd">
<html xmlns="http://www.w3.org/1999/xhtml">
<head>
<meta http-equiv="Content-Type" content="text/html; charset=utf-8" />
<meta http-equiv="X-UA-Compatible" content="IE=Edge,chrome=1">
<title>資治通鑒_272-資治通鑑卷二百七十一_272-資治通鑑卷二百七十一</title>
<meta name="Keywords" content="資治通鑒_272-資治通鑑卷二百七十一_272-資治通鑑卷二百七十一">
<meta name="Description" content="資治通鑒_272-資治通鑑卷二百七十一_272-資治通鑑卷二百七十一">
<meta http-equiv="Cache-Control" content="no-transform" />
<meta http-equiv="Cache-Control" content="no-siteapp" />
<link href="/img/style.css" rel="stylesheet" type="text/css" />
<script src="/img/m.js?2020"></script> 
</head>
<body>
 <div class="ClassNavi">
<a  href="/24shi/">二十四史</a> | <a href="/SiKuQuanShu/">四库全书</a> | <a href="http://www.guoxuedashi.com/gjtsjc/"><font  color="#FF0000">古今图书集成</font></a> | <a href="/renwu/">历史人物</a> | <a href="/ShuoWenJieZi/"><font  color="#FF0000">说文解字</a></font> | <a href="/chengyu/">成语词典</a> | <a  target="_blank"  href="http://www.guoxuedashi.com/jgwhj/"><font  color="#FF0000">甲骨文合集</font></a> | <a href="/yzjwjc/"><font  color="#FF0000">殷周金文集成</font></a> | <a href="/xiangxingzi/"><font color="#0000FF">象形字典</font></a> | <a href="/13jing/"><font  color="#FF0000">十三经索引</font></a> | <a href="/zixing/"><font  color="#FF0000">字体转换器</font></a> | <a href="/zidian/xz/"><font color="#0000FF">篆书识别</font></a> | <a href="/jinfanyi/">近义反义词</a> | <a href="/duilian/">对联大全</a> | <a href="/jiapu/"><font  color="#0000FF">家谱族谱查询</font></a> | <a href="http://www.guoxuemi.com/hafo/" target="_blank" ><font color="#FF0000">哈佛古籍</font></a> 
</div>

 <!-- 头部导航开始 -->
<div class="w1180 head clearfix">
  <div class="head_logo l"><a title="国学大师官网" href="http://www.guoxuedashi.com" target="_blank"></a></div>
  <div class="head_sr l">
  <div id="head1">
  
  <a href="http://www.guoxuedashi.com/zidian/bujian/" target="_blank" ><img src="http://www.guoxuedashi.com/img/top1.gif" width="88" height="60" border="0" title="部件查字,支持20万汉字"></a>


<a href="http://www.guoxuedashi.com/help/yingpan.php" target="_blank"><img src="http://www.guoxuedashi.com/img/top230.gif" width="600" height="62" border="0" ></a>


  </div>
  <div id="head3"><a href="javascript:" onClick="javascript:window.external.AddFavorite(window.location.href,document.title);">添加收藏</a>
  <br><a href="/help/setie.php">搜索引擎</a>
  <br><a href="/help/zanzhu.php">赞助本站</a></div>
  <div id="head2">
 <a href="http://www.guoxuemi.com/" target="_blank"><img src="http://www.guoxuedashi.com/img/guoxuemi.gif" width="95" height="62" border="0" style="margin-left:2px;" title="国学迷"></a>
  

  </div>
</div>
  <div class="clear"></div>
  <div class="head_nav">
  <p><a href="/">首页</a> | <a href="/ShuKu/">国学书库</a> | <a href="/guji/">影印古籍</a> | <a href="/shici/">诗词宝典</a> | <a   href="/SiKuQuanShu/gxjx.php">精选</a> <b>|</b> <a href="/zidian/">汉语字典</a> | <a href="/hydcd/">汉语词典</a> | <a href="http://www.guoxuedashi.com/zidian/bujian/"><font  color="#CC0066">部件查字</font></a> | <a href="http://www.sfds.cn/"><font  color="#CC0066">书法大师</font></a> | <a href="/jgwhj/">甲骨文</a> <b>|</b> <a href="/b/4/"><font  color="#CC0066">解密</font></a> | <a href="/renwu/">历史人物</a> | <a href="/diangu/">历史典故</a> | <a href="/xingshi/">姓氏</a> | <a href="/minzu/">民族</a> <b>|</b> <a href="/mz/"><font  color="#CC0066">世界名著</font></a> | <a href="/download/">软件下载</a>
</p>
<p><a href="/b/"><font  color="#CC0066">历史</font></a> | <a href="http://skqs.guoxuedashi.com/" target="_blank">四库全书</a> |  <a href="http://www.guoxuedashi.com/search/" target="_blank"><font  color="#CC0066">全文检索</font></a> | <a href="http://www.guoxuedashi.com/shumu/">古籍书目</a> | <a   href="/24shi/">正史</a> <b>|</b> <a href="/chengyu/">成语词典</a> | <a href="/kangxi/" title="康熙字典">康熙字典</a> | <a href="/ShuoWenJieZi/">说文解字</a> | <a href="/zixing/yanbian/">字形演变</a> | <a href="/yzjwjc/">金 文</a> <b>|</b>  <a href="/shijian/nian-hao/">年号</a> | <a href="/diming/">历史地名</a> | <a href="/shijian/">历史事件</a> | <a href="/guanzhi/">官职</a> | <a href="/lishi/">知识</a> <b>|</b> <a href="/zhongyi/">中医中药</a> | <a href="http://www.guoxuedashi.com/forum/">留言反馈</a>
</p>
  </div>
</div>
<!-- 头部导航END --> 
<!-- 内容区开始 --> 
<div class="w1180 clearfix">
  <div class="info l">
   
<div class="clearfix" style="background:#f5faff;">
<script src='http://www.guoxuedashi.com/img/headersou.js'></script>

</div>
  <div class="info_tree"><a href="http://www.guoxuedashi.com">首页</a> > <a href="/SiKuQuanShu/fanti/">四库全书</a>
 > <h1>资治通鉴</h1> <!--         下载:【右键另存为】即可 --></div>
  <div class="info_content zj clearfix">
  
<div class="info_txt clearfix" id="show">
<center style="font-size:24px;">272-資治通鑑卷二百七十一</center>
    資治通鑑卷二百七十一 宋 司馬光 撰<br />
<br />
  胡三省 音註<br />
<br />
  後梁紀六【起屠維單閼十月盡玄黓敦牂凡三年有奇】<br />
<br />
  均王下<br />
<br />
  貞明五年冬十月出濛為楚州團練使【承上卷徐温惡濛事】 晉王如魏州發徒數萬廣德勝北城日與梁人爭大小百餘戰互有勝負左射軍使石敬瑭與梁人戰于河壖【左射軍使統軍士之能左射者壖而緣翻河邊地也】梁人擊敬瑭斷其馬甲【斷丁管翻薛史曰晉高祖為梁人所襲馬甲連革斷】横衝兵馬使劉知遠以所乘馬授之自乘斷甲者徐行為殿【殿丁練翻】梁人疑有伏不敢迫俱得免敬瑭以是親愛之敬瑭知遠其先皆沙陀人敬瑭李嗣源之壻也【石敬瑭劉知遠始此】 劉鄩圍張萬進于兖州經年城中危窘【去年八月劉鄩圍兖州事見上卷窘渠隕翻】晉王方與梁人戰河上力不能救萬進遣親將劉處讓乞師于晉晉王未之許處讓于軍門截耳曰苟不得請生不如死晉王義之將為出兵【為于偽翻】會鄩已屠兖州族萬進乃止以處讓為行臺左驍衛將軍處讓滄州人也【張萬進自滄州徙兖州劉處讓蓋從之處昌呂翻驍堅堯翻】 十一月吳武寧節度使張崇寇安州 丁丑以劉鄩為泰寧節度使同平章事【劉鄩先以河朔喪師貶為團練使落平章事今以平張萬進復為使相】 辛卯王瓚引兵至戚城【戚城在德勝西即春秋時衛之戚邑也杜預曰戚河上之邑】與李嗣源戰不利 梁築壘貯糧於潘張【貯丁呂翻潘張地名蓋潘張二姓居之因以名村如楊村之類一姓而名村也其他如麻家渡趙步又皆以姓而名津步此皆載於通鑑薛史云潘張村在河曲】距楊村五十里十二月晉王自將騎兵自河南岸西上邀其餉者俘獲而還【上時掌翻還從宣翻又如字】梁人伏兵於要路晉兵大敗晉王以數騎走梁數百騎圍之李紹榮識其旗【凡行軍主將各有旗以為表識今謂之認旗】單騎奮擊救之僅免戊戌晉王復與王瓚戰於河南【復扶又翻】瓚先勝獲晉將石君立等旣而大敗乘小舟度河走保北城【楊村北城也】失亡萬計帝聞石君立勇【石君立即救晉陽者也見二百六十九卷二年】欲將之【將即亮翻】繫於獄而厚餉之使人誘之【誘音酉】君立曰我晉之敗將而為用於梁雖竭誠効死誰則信之人各有君何忍反為仇讎用哉帝猶惜之盡殺所獲晋將獨置君立晋王乘勝遂拔濮陽 【考異曰莊宗實錄天祐十五年賀瓌屯于濮州北行臺里十二月辛酉上次于臨濮賊亦捨營踵我癸亥次于胡柳明日接戰王彦章敗走濮陽甲子進攻濮陽一鼓而拔按唐地里志濮州亦謂之濮陽郡治鄄城有濮陽臨濮二縣據莊宗實錄則行臺里在臨濮東湖柳在濮陽東彦章所保莊宗所拔者皆濮陽縣非濮州也而莊宗列傳薛史閻寶傳皆云彦章騎軍已入濮州山下惟列步兵向晚皆有歸心是以濮陽即為濮州也李嗣昭傳嗣昭云賊無營壘去臨濮地遠日已晡晚皆有歸心但以精騎撓之無令夕食晡後追擊破之必矣我若收軍拔寨賊入臨濮俟彼整齊復來則勝負未決是又以濮陽即為臨濮也按薛史梁紀貞明五年四月制書攷濮州税課是濮州猶屬梁也莊宗實錄天祐十六年十二月攻下濮陽下教告諭曹濮百姓勸令歸附是濮州未屬晉也又賀瓌屯于山西晉軍在其東彦章已西入濮陽瓌豈得更東歸臨濮疑寶傳濮州嗣昭傳臨濮皆當為濮陽史氏文飾之誤也又莊宗實錄去年十二月晉已拔濮陽至此又云攻下濮陽按薛史梁紀去年十二月晉人攻濮陽陷之今年十二月又云晉人陷濮陽唐紀去冬拔濮陽今年四月追襲賀瓌至濮陽十二月無攻下濮陽事賀瓌傳貞明四年領大軍營於行臺村十二月戰敗四月退軍行臺尋卒若非實錄及梁紀重複則是去冬唐雖得濮陽弃而不守今年復攻拔之也】帝召王瓚還以天平節度使戴思遠代為北面招討使屯河上以拒晉人 己酉蜀雄武節度使兼中書令王宗朗有罪削奪官爵復其姓名曰全師朗命武定節度使兼中書令桑弘志討之 吳禁民私畜兵器盜賊益繁御史臺主簿京兆盧樞上言【唐御史臺置主薄一人掌印受事發辰覈臺務主公廨及奴婢勲散官之職】今四方分爭宜教民戰且善人畏法禁而奸民弄干戈是欲偃武而反招盜也宜團結民兵使之習戰自衛鄉里從之<br />
<br />
  六年春正月戊辰蜀桑弘志克金州執全師朗獻于成都蜀主釋之 吳張崇攻安州不克而還崇在廬州貪暴不法廬江民訟縣令受賕徐知誥遣侍御史知雜事楊廷式往按之欲以威崇廷式曰雜端推事其體至重【唐御史臺侍御史六人以久次一人知雜事謂之雜端】職業不可不行知誥曰何如廷式曰械繫張崇使吏如昇州簿責都統【簿責者一二而責之】知誥曰所按者縣令耳何至於是廷式曰縣令微官張崇使之取民財轉獻都統耳【都統謂徐温也】豈可捨大而詰小乎【詰去吉翻】知誥謝之曰固知小事不足相煩【煩勞也】以是益重之廷式泉州人也 晉王自得魏州【得魏州見二百六十九卷元年】以李建及為魏博内外牙都將將銀槍効節都【將即亮翻下同】建及為人忠壯所得賞賜悉分士卒與同甘苦故能得其死力所向立功同列疾之宦者韋令圖監建及軍譛於晉王曰建及以私財驟施【施式豉翻】此其志不小不可使將牙兵王疑之建及知之行之自若三月王罷建及軍職以為代州刺史【史言晉王不能信屬賢將李建及由是怏怏而卒】 漢楊洞濳請立學校開貢舉設銓選漢主巖從之【校戶教翻】 夏四月乙亥以尚書左丞李琪為中書侍郎同平章事琪珽之弟也【李珽始見於唐昭宗天復三年而死於梁誅友珪之時】性疎俊挾趙巖張漢傑之勢頗通賄賂蕭頃與琪同為相頃謹密而隂伺琪短【伺相吏翻】久之有以攝官求仕者琪輒改攝為守頃奏之【歐史曰琪所私吏當得試官琪改試為守為頃所發】帝大怒欲流琪遠方趙張左右之【左右讀曰佐佑】止罷為太子少保 【考異曰薛史止有琪作相月日無罷相年月故終言之】 河中節度使冀王友謙以兵襲取同州逐忠武節度使程全暉全暉奔大梁友謙以其子令德為忠武留後表求節鉞帝怒不許旣而懼友謙怨望己酉以友謙兼忠武節度使制下【下戶嫁翻】友謙已求節鉞于晉王【朱友謙自此遂歸于晉】晉王以墨制除令德忠武節度使 【考異曰莊宗列傳上令幕客王正言送節旄賜之莊宗實錄列傳薛史友謙傳皆云友謙以令德為帥請節旄不許薛史末帝紀貞明六年云陷同州以令德為留後表求節旄不允而貞明四年六月甲辰以歙州刺史朱令德為忠武留後恐是四年已䧟同州】 吳宣王重厚恭恪徐温父子專政王未嘗有不平之意形於言色温以是安之及建國稱制【見上卷上年】尤非所樂多沈飲鮮食【樂音洛沈持林翻鮮息淺翻少也】遂成寢疾五月温自金陵入朝議當為嗣者或希温意言曰蜀先主謂武侯嗣子不才君宜自取【見六十九卷魏文帝黃初三年】温正色曰吾果有意取之當在誅張顥之初【誅張顥見二百六十九卷開平二年】豈至今日邪使楊氏無男有女亦當立之敢妄言者斬乃以王命迎丹楊公溥監國 【考異曰吳錄九國志有女當立之語在誅張顥時今從薛史十國紀年王疾病大丞相温來朝議立嗣君門下侍郎嚴可求言王諸子皆不才引蜀先主顧命諸葛亮事温以告知誥知誥曰可求多知言未必誠不過賾大人意爾温曰吾若自取非止今日張顥之亂嗣王幼弱政在吾手取之易于反掌然思太祖大漸欲傳位劉威吾獨力爭太祖垂泣以後事託我安可忘也乃與内樞密使王令謀定策稱隆演命迎丹楊公溥監國己丑隆演卒六月戊申溥即王位恐可求亦不應有此言今從薛史】徙溥兄濛為舒州團練使【越濛而立溥者濛為徐温所忌也】己丑宣王殂【年二十四】六月戊申溥即吳王位【溥楊行密第四子】尊母王氏曰太妃 丁巳蜀以司徒兼門下侍郎同平章事周庠同平章事充永平節度使【唐末置永平軍於卭州歐史職方考蜀以雅州為永平節度】 帝以泰寜節度使劉鄩為河東道招討使帥感化節度使尹皓靜勝節度使温昭圖莊宅使段凝攻同州【帥讀曰率下同】 閏月庚申朔蜀主作高祖原廟于萬里橋【原廟起于漢原再也已立太廟而再立廟曰原廟萬里橋在成都寰宇記曰昔者費禕聘吳諸葛亮送之至此橋曰萬里之路始於此矣因以名橋】帥后妃百官用䙝味作鼓吹祭之【䙝味嘗御嗜好之味也記郊特牲曰禘嘗不敢用䙝味而貴多品所以交於神明之義也䙝息列翻】華陽尉張士喬上疏諫以為非禮【華陽縣本唐貞觀十七年所置蜀縣在益州郭下與成都分治乾元元年改為華陽縣華戶化翻】蜀主怒欲誅之太后以為不可乃削官爵流黎州士喬感憤赴水死 劉鄩等圍同州朱友謙求救于晉秋七月晉王遣李存審李嗣昭李建及慈州刺史李存質將兵救之 乙卯蜀主下詔北巡以禮部尚書兼成都尹長安韓昭為文思殿大學士位在翰林承旨上昭無文學以便佞得幸【便毗連翻】出入宫禁就蜀主乞通渠巴集數州刺史賣之以營居第蜀主許之識者知蜀之將亡八月戊辰蜀主發成都被金甲冠珠帽執弓矢而行【被皮義翻冠古玩翻】旌旗兵甲亘百餘里雒令段融上言【雒漢古縣唐屬漢州為州治所上時掌翻】不宜遠離都邑【離力智翻】當委大臣征討不從九月次安遠城【凡兵一宿為信過宿為次】 李存審等至河中即日濟河【自河中濟河救同州】梁人素輕河中兵每戰必窮追不置存審選精甲二百雜河中兵直壓劉鄩壘鄩出千騎逐之知晉人已至大驚【時鄩兵出逐河中兵晉騎反擊之獲梁騎兵五十梁人知其晉軍也大驚】自是不敢輕出晉人軍于朝邑【九域志朝邑在同州東三十五里】河中事梁久【唐昭宗之世朱全忠降王珂河中遂事梁】將士皆持兩端諸軍大集芻粟踊貴友謙諸子說友謙【說式芮翻】且歸欵於梁以退其師友謙曰昔晉王親赴吾急秉燭夜戰【謂與康懷貞等戰也事見二百六十八卷乾化二年】今方與梁相拒【謂相拒于河上也】又命將星行分我資糧豈可負邪晉人分兵攻華州壞其外城【將即亮翻華戶化翻壞音怪】李存審等按兵累旬乃進逼劉鄩營鄩等悉衆出戰大敗收餘衆退保羅文寨【薛史曰鄩以餘衆退保華州羅丈寨】又旬餘存審謂李嗣昭曰獸窮則搏不如開其走路然後擊之乃遣人牧馬于沙苑鄩等宵遁追擊至渭水又破之殺獲甚衆【劉鄩用兵十步九計以此得名於時至同州之役與李存審遇為所玩弄若嬰兒在人掌股之上是何也孽也蓋鳥之中傷者曰孽聞聞鳴則引而高飛力不足斯抎矣故空弓可落也劉鄩先為晉兵所破見晉兵之來氣阻而膽消矣烏能與之為敵哉】存審等移檄告諭關右引兵略地至下邽謁唐帝陵哭之而還【唐帝陵在同州奉先縣還從宣翻又如字】河中兵進攻崇州靜勝節度使温昭圖甚懼【元年温韜以義勝軍降改耀州曰崇州義勝曰靜勝韜賜今名】帝使供奉官竇維說之曰【說式芮翻】公所有者華原美原兩縣耳【唐末温韜為盜據華原縣李茂貞以華原為茂州韜為刺史尋改耀州又以美原縣為鼎州建義勝軍以韜為節度使及降梁改耀州為崇州鼎州為裕州義勝軍為靜勝是其所有者本唐兩縣也】雖名節度使實一鎮將比之雄藩豈可同日語也公有意欲之乎昭圖曰然維曰當為公圖之【為于偽翻】即教昭圖表求移鎮帝以汝州防禦使華温琪權知靜勝留後【華戶化翻】 冬十月辛酉蜀主如武定軍數日復還安遠【復扶又翻】 十一月戊子朔蜀主以兼侍中王宗儔為山南節度使西北面都招討行營安撫使天雄節度使同平章事王宗昱永寧軍使王宗晏左神勇軍使王宗信為三招討以副之將兵伐岐出故關壁於咸宜【壁者築壁壘以屯軍咸宜當在隴州汧源縣界】入良原【良原縣屬涇州九域志在州西南六十里】丁酉王宗儔攻隴州岐王自將萬五千人屯汧陽【汧陽縣屬隴州九域志在州東六十七里東距鳳翔五十五里】癸卯蜀將陳彦威出散關敗岐兵于箭筈嶺【杜佑曰岐山即今之岐山縣其山兩岐故俗呼為箭筈嶺敗補邁翻筈古活翻】蜀兵食盡引還【還從宣翻又如字】宗昱屯泰州宗儔屯上邽宗晏宗信屯威武城庚戌蜀主發安遠城十二月庚申至利州閬州團練使林思諤來朝請幸所治從之【閬中林思諤所治也九域志利州東南至閬州二百四十里】癸亥泛江而下【泛嘉陵江也】龍舟畫舸【畫與畫同舸古我翻楚人謂大船為舸】輝映江渚州縣供辦民始愁怨【此總言蜀主所經行州縣不特言閬州為然也】壬申至閬州州民何康女色美將嫁蜀主取之賜其夫家帛百匹夫一慟而卒【記諸侯不下漁色注云謂不内取於國中也内取國中為下漁色國君而内取象捕魚然中網則取之是無所擇王衍奪人之妻其為漁也殆有甚焉】癸未至梓州 趙王鎔自恃累世鎮成德得趙人心生長富貴【長知兩翻】雍容自逸治府第園沼極一時之盛【治直之翻】多事嬉遊不親政事事皆仰成於僚佐【仰牛向翻】深居府第權移左右行軍司馬李藹宦者李弘規用事於中外【外則李藹中則李弘規】宦者石希蒙尤以諂諛得幸初劉仁恭使牙將張文禮從其子守文鎮滄州守文詣幽州省其父文禮於後據城作亂滄人討之奔鎮州【此言唐末事叙張文禮之所自來省悉景翻】文禮好誇誕【好呼到翻】自言知兵趙王鎔奇之養以為子更名德明【更工衡翻】悉以軍事委之德明將行營兵從晉王【事見二百六十七卷太祖乾化元年】鎔欲寄以腹心使都指揮使符習代還以為防城使鎔晚年好事佛及求僊【好呼到翻】專講佛經受符籙廣齋醮合煉仙丹【合音閤】盛飾館宇於西山每往遊之【鎮州西山謂之房山上有西王母祠鎔欲求仙故數往遊】登山臨水數月方歸將佐士卒陪從者常不下萬人【從才用翻】往來供頓軍民皆苦之是月自西山還宿鶻營莊【鶻戶骨翻】石希蒙勸王復之他所【復扶又翻】李弘規言於王曰晉王夾河血戰【或戰河南或戰河北故曰夾河】櫛風沐雨【櫛去瑟翻】親冒矢石而王專以供軍之資奉不給之費且時方艱難人心難測王久虚府第遠出遊從萬一有姦人為變閉關相距將若之何王將歸希蒙密言於王曰弘規妄生猜間【間古莧翻】出不遜語以刼脅王專欲誇大於外長威福耳【長知兩翻】王遂留信宿無歸志【詩九罭云於女信宿毛氏傳再宿曰信與左傳師行一宿為信之義不同】弘規乃教内牙都將蘇漢衡帥親軍擐甲拔刃【帥讀曰率擐音宦】詣帳前白王曰士卒暴露已久願從王歸弘規因進言曰石希蒙勸王遊從不已且聞欲隂為叛逆請誅之以謝衆王不聽牙兵遂大譟斬希蒙首投於前王怒且懼亟歸府是夕遣其長子副大使昭祚與王德明將兵圍弘規及李藹之第族誅之連坐者數千家又殺蘇漢衡收其黨與窮治反狀親軍大恐【為張文禮嗾軍士殺王鎔張本治直之翻下同】 吳金陵城成陳彦謙上費用之籍徐温曰吾旣任公不復會計【上時掌翻復扶又翻會工外翻】悉焚之 初閩王審知承制加其從子泉州刺史延彬領平盧節度使【從才用翻】延彬治泉州十七年吏民安之會得白鹿及紫芝僧浩源以為王者之符延彬由是驕縱密遣使浮海入貢求為泉州節度使事覺審知誅浩源及其黨黜延彬歸私第 漢主巖遣使通好于蜀【好呼到翻】 吳越王鏐遣使為其子傳琇求昏於楚楚王殷許之【為于偽翻琇音秀】<br />
<br />
  龍德元年【是年五月方改元】春正月甲午蜀主還成都【去年七月蜀主出巡遊至是方還】 初蜀主之為太子高祖為聘兵部尚書高知言女為妃無寵【蜀主王建廟號高祖】及韋妃入宫尤見疎薄至是遣還家知言驚仆不食而卒【卒子恤翻】韋妃者徐耕之孫也有殊色蜀主適徐氏見而悦之太后因納於後宫蜀主不欲娶於母族託云韋昭度之孫【韋昭度唐僖宗時嘗奉制帥蜀故託言之】初為婕妤累加元妃【婕妤音接予】蜀主常列錦步障擊毬其中往往遠適而外人不知爇諸香晝夜不絶久而厭之更爇皂莢以亂其氣【更工衡翻爇如悦翻皂莢如猪牙者良爇之其氣酷烈】結繒為山及宫殿樓觀於其上或為風雨所敗則更以新者易之或樂飲繒山涉旬不下【繒慈陵翻觀工喚翻敗補邁翻樂音洛】山前穿渠通禁中或乘船夜歸令宫女秉蠟炬千餘居前船却立照之水面如晝或酣飲禁中鼓吹沸騰【吹尺睡翻】以至達旦以是為常 甲辰徙靜勝節度使温昭圖為匡國節度使鎮許昌昭圖素事趙巖故得名藩【温昭圖求徙鎮見上年靜勝梁之邊鎮且兩縣耳匡國唐之忠武軍領許陳汝三州自來為名藩趙巖以名藩授昭圖及緩急投之以託身而斬巖者昭圖也勢利之交可不戒哉】 蜀主吳主屢以書勸晉王稱帝晉王以書示僚佐曰昔王太師亦遺先王書勸以唐室已亡宜自帝一方【王太師者以唐官呼蜀主王建遺書事見二百六十七卷開平元年遺唯季翻】先王語余云【語牛倨翻】昔天子幸石門吾發兵誅賊臣【事見二百六十卷唐昭宗乾寧二年】當是之時威振天下【振動也】吾若挾天子據關中自作九錫禪文誰能禁我顧吾家世忠孝立功帝室誓死不為耳汝它日當務以復唐社稷為心慎勿效此曹所為言猶在耳此議非所敢聞也因泣旣而將佐及藩鎮勸進不已乃令有司市玉造法物【法物謂傳國八寶之類】黄巢之破長安也【見二百五十四卷唐僖宗廣明元年】魏州僧傳真之師得傳國寶藏之四十年至是傳真以為常玉將鬻之或識之曰傳國寶也傳真乃詣行臺獻之【宋白曰同光初魏州開元寺僧傳真獻國寶驗其文即受命八寶也晉王為尚書令置行臺於魏州】將佐皆奉觴稱賀張承業在晉陽聞之詣魏州諫曰吾王世世忠於唐室【言執宜國昌克用皆輸力於唐室】救其患難【難乃旦翻】所以老奴三十餘年為王捃拾財賦【唐昭宗乾寧二年張承業始監河東軍至是年二十七年捃舉藴翻又居運翻】召補兵馬誓滅逆賊復本朝宗社耳【本朝謂唐也朝直遥翻】今河北甫定朱氏尚存而王遽即大位殊非從來征伐之意天下其誰不解體乎王何不先滅朱氏復列聖之深讐然後求唐後而立之南取吳西取蜀【時楊氏據江淮國號吳王氏據梁益國號蜀】汛掃宇内合為一家當是之時雖使高祖太宗復生【復扶又翻下不復同】誰敢居王上者讓之愈久則得之愈堅矣老奴之志無它但以受先王大恩欲為王立萬年之基耳【為于偽翻下本為同】王曰此非余所願奈羣下意何承業知不可止慟哭曰諸侯血戰本為唐家【此張承業所謂從來征伐之意也】今王自取之誤老奴矣即歸晉陽邑邑成疾不復起【張承業唐之純臣也烏可以宦者待之哉 考異曰莊宗實錄上初獲玉璽諸將勸上復唐正朔承業自太原急趣謁上曰殿下父子血戰三十餘年蓋緣報國復仇為唐宗社今元凶未殄軍賦不充河朔數州弊於供億遽先大號費養兵之事力困凋弊之生靈臣以為一未可也殿下旣化家為國新創廟朝典禮制度須取太常準的方今禮院未見其人儻失舊章為人輕笑二未可也因泣下霑襟上曰余非所願奈諸將意何承業自是多病日加危篤尋卒莊宗列傳上受諸道勸進將簒帝位承業以為晉王三代有功於國先王怒賊臣簒逆匡復舊邦賊旣未平不宜輕受推戴方疾作肩輿之鄴宫見上力諫大指皆如實錄薛史唐餘錄皆與莊宗列傳同五代闕文承業謂莊宗吾王世奉唐家最為忠孝自貞觀以來王室有難未嘗不從所以老奴三十餘年為吾王捃拾財賦召補車馬者誓滅逆賊朱温復本朝宗社耳今河朔甫定朱氏尚存吾王遽即大位可乎莊宗曰奈諸將意何承業知不可諫止乃慟哭曰諸侯血戰本為李家今吾王自取之悞老奴矣即歸太原不食而死秦再思洛中紀異承業諫帝曰大王何不待誅克梁孽更平吳蜀俾天下一家且先求唐氏子孫立之復更以天下讓有功者何人輒敢當之讓一月即一月牢讓一年即一年牢設使高祖再生太原復出又胡為哉今大王一旦自立頓失從前仗義征伐之旨人情怠矣老夫是閹官不愛大王官職富貴直以受先王付囑之重欲為先王立萬年之基爾莊宗不能從乃謝病歸太原而卒歐陽史兼採闕文紀異之意按實錄等書承業止惜費多及儀物不備太似淺陋如闕文所言承業事莊宗父子數十年唐室近親已盡豈不知其欲自取之意乎褒美承業亦恐太過又按傳真以天祐十八年獻寶承業以十九年十一月卒云即歸太原不食而死亦非實也如紀異之語承業為莊宗忠謀近得其實今從之】 二月吳改元順義 趙王旣殺李弘規李藹委政于其子昭祚昭祚性驕愎【愎符逼翻】旣得大權曏時附弘規者皆族之弘規部兵五百人欲逃聚泣偶語未知所之會諸軍有給賜趙王忿親軍之殺石希蒙獨不時與衆益懼王德明素蓄異志因其懼而激之曰王命我盡阬爾曹吾念爾曹無罪併命【併命謂一時皆誅死】欲從王命則不忍不然又獲罪於王奈何衆皆感泣【感張文禮則讎趙王鎔矣】是夕親軍有宿於潭城西門者相與飲酒而謀之【潭城常山牙城北偏也歐陽公鎮陽殘杏詩云北潭跬步病不到何暇騎馬尋郊原註云北潭常山宫後池也州之勝遊惟此以有池潭故其城謂之潭城】酒酣其中驍健者曰吾曹識王太保意【王太保謂王德明謂德明所以語親軍者其意欲使之作亂】今夕富貴決矣即踰城入趙王方焚香受籙二人斷其首而出【斷音短】因焚府第軍校張友順帥衆詣德明請為留後【帥讀曰率】德明復姓名曰張文禮盡滅王氏之族【唐穆宗長慶元年王庭湊據成德軍歷四世五帥而滅】獨置昭祚之妻普寧公主以自託於梁【梁女妻昭祚見二百六十二卷唐昭宗光化二年】 三月吳人歸吳越王鏐從弟龍武統軍鎰于錢唐【錢鎰被禽見二百六十五卷唐天祐二年錢唐吳越國都從才用翻】鏐亦歸吳將李濤於廣陵【李濤被禽見二百六十八卷乾化三年廣陵吳國都史言錢楊兩釋俘囚以固和好】徐温以濤為右雄武統軍鏐以鎰為鎮海節度副使【敗軍之罰其不行也亦已久矣】 張文禮遣使告亂於晉王且奉牋勸進因求節鉞晉王方置酒作樂聞之投盃悲泣欲討之僚佐以為文禮罪誠大然吾方與梁爭不可更立敵於肘腋宜且從其請以安之王不得已夏四月遣節度判官盧質承制授文禮成德留後【晉王雖欲撫安之而張文禮不能自安也為興兵討文禮張本】 陳州刺史惠王友能反舉兵趣大梁【九域志陳州北至大梁二百四十里趣七喻翻】詔陜州留後霍彦威宣義節度使王彦章控鶴指揮使張漢傑將兵討之【陜失冉翻】友能至陳留【九域志陳留縣在大梁東五十二里】兵敗走還陳州諸軍圍之 五月丙戌朔改元【方改元龍德】 初劉鄩與朱友謙為昏鄩之受詔討友謙也【事見上年】至陜州先遣使移書諭以禍福待之月餘友謙不從然後進兵尹皓段凝素忌鄩因譛之於帝曰鄩逗遛養寇俾俟援兵【尹皓段凝與劉鄩同攻朱友謙因其諭友謙而不服遇晉兵而敗退得以譛之】帝信之鄩旣敗歸以疾請解兵柄詔聽於西都就醫【梁以洛都為西都】密令留守張宗奭酖之丁亥卒【史言梁自翦其爪牙 考異曰莊宗實錄云憂恚發病卒薛史云張宗奭承朝廷密旨逼令飲酖而卒今從之】 六月乙卯朔日有食之 秋七月惠王友能降庚子詔赦其死降封房陵侯 晉王旣許藩鎮之請求唐舊臣欲以備百官朱友謙遣前禮部尚書蘇循詣行臺【蘇循依朱友謙見二百六十六卷太祖開平元年】循至魏州入牙城望府廨即拜謂之拜殿【廨古隘翻】見王呼萬歲舞蹈泣而稱臣翌日又獻大筆三十枚謂之畫日筆【唐制敕皆天子畫日蘇循以迎合禪代之議為朱全忠所薄而李存勗乃喜之是其識見又在全忠下矣】王大喜即命循以本官為河東節度副使張承業深惡之【惡烏路翻】 張文禮雖受晉命内不自安復遣間使因盧文進求援于契丹又遣間使來告曰【復扶又翻盧文進叛晉歸契丹見二百六十九卷貞明二年三年間古莧翻】王氏為亂兵所屠公主無恙今臣已北召契丹乞朝廷發精甲萬人相助自德棣度河則晉人遁逃不暇矣帝疑未決敬翔曰陛下不乘此舋以復河北【舋許覲翻】則晉人不可復破矣【復扶又翻】宜徇其請不可失也趙張輩皆曰今彊寇近在河上盡吾兵力以拒之猶懼不支何暇分萬人以救張文禮乎且文禮坐持兩端欲以自固於我何利焉帝乃止【史言趙張慮不及遠以誤國亡家】晉人屢於塞上及河津獲文禮蠟丸絹書【塞土所獲者通契丹之書河津所獲者通梁之書】晉王皆遣使歸之文禮慙懼文禮忌趙故將多所誅滅符習將趙兵萬人從晉王在德勝文禮請召歸以它將代之且以習子蒙為都督府參軍遣人齎錢帛勞行營將士以悅之【張文禮蓋自置鎮冀深趙都督府故有參佐勞力到翻】習見晉王泣涕請留晉王曰吾與趙王同盟討賊【晉趙同盟見二百六十七卷太祖開平元年】義猶骨肉不意一旦禍生肘腋【腋羊益翻】吾誠痛之汝苟不忘舊君能為之復讐乎【為于偽翻】吾以兵糧助汝習與部將三千餘人舉身投地慟哭曰故使授習等劒使之攘除寇敵【故使謂王鎔也已死稱為故使使疏吏翻下同】自聞變故以來寃憤無訴欲引劒自剄【剄古頸翻】顧無益於死者【顧回思也死者亦謂王鎔】今大王念故使輔佐之勤【輔佐者言以兵力輔佐晉王也】許之復寃習等不敢煩霸府之兵【晉王在魏州為河北諸藩鎮盟主故稱其府曰霸府】願以所部徑前搏取凶豎以報王氏累世之恩死不恨矣八月庚申晉王以習為成德留後又命天平節度使閻寶相州刺史史建瑭將兵助之自邢洺而北文禮先病腹疽甲子晉兵拔趙州刺史王鋋降【鋋音蟬】晉王復以為刺史文禮聞之驚懼而卒其子處瑾祕不發喪與其黨韓正時謀悉力拒晉九月晉兵渡滹沱圍鎮州【范成大北使錄曰過滹沱河五里至鎮州】決漕渠以灌之獲其深州刺史張友順壬辰史建瑭中流矢卒【中竹仲翻】晉王欲自分兵攻鎮州北面招討使戴思遠聞之謀悉楊村之衆襲德勝北城晉王得梁降者知之冬十月己未晉王命李嗣源伏兵於戚城李存審屯德勝先以騎兵誘之偽示羸怯【誘音酉羸倫為翻】梁兵競進晉王嚴中軍以待之梁兵至晉王以鐵騎三千奮擊梁兵大敗思遠走趣楊村【趣七喻翻】士卒為晉兵所殺傷及自相蹈藉【藉慈夜翻】墜河陷冰失亡二萬餘人晉王以李嗣源為蕃漢内外馬步副總管同平章事 初義武節度使兼中書令王處直未有子妖人李應之得小兒劉雲郎於陘邑【陘邑本前漢苦陘縣後漢改曰漢昌曹魏改曰魏昌隋改曰隋昌唐武德四年改曰唐昌天寶元年改曰陘邑屬定州妖一遙翻陘音刑】以遺處直曰是兒有貴相使養為子名之曰都及壯便佞多詐【遺唯季翻相息亮翻便毗連翻】處直愛之置新軍使典之處直有孽子郁無寵奔晉晉王克用以女妻之【庶子為孽妻七細翻】累遷至新州團練使餘子皆幼處直以都為節度副大使欲以為嗣及晉王存勗討張文禮處直以平日鎮定相為脣齒恐鎮亡而定孤固諫以為方禦梁寇宜且赦文禮晉王荅以文禮弑君義不可赦又潛引梁兵恐於易定亦不利處直患之以新州地鄰契丹乃濳遣人語郁【新州窮邊也北接契丹語牛倨翻】使賂契丹召令犯塞務以解鎮州之圍【王郁雖不能解鎮州之圍而亦能為契丹鄉導以寇晉】其將佐多諫不聽郁素疾都冒繼其宗乃邀處直求為嗣處直許之軍府之人皆不欲召契丹都亦慮郁奪其處乃隂與書吏和昭訓謀刧處直會處直與張文禮宴於城東【按張文禮時已受兵安能至定州與王處直宴處直所與宴者必文禮使者也文禮之下當有使字】暮歸都以新軍數百伏于府第大譟刧之曰將士不欲以城召契丹請令公歸西第乃并其妻妾幽之西第【凡官府第舍以東為上西第者即安養閒之地唐末王處存帥義武兄弟相繼至是而敗】盡殺處直子孫在中山及將佐之為處直腹心者都自為留後具以狀白晉王晉王因以都代處直【為唐明宗朝王都又以中山召契丹張本】 吳徐温勸吳王祀南郊或曰禮樂未備且唐祀南郊其費巨萬今未能辦也温曰安有王者而不事天乎吾聞事天貴誠多費何為唐每郊祀啟南門灌其樞用脂百斛【以脂灌樞欲其滑而易轉且門無聲】此乃季世奢泰之弊又安足法乎【史言徐温雖不學而知先王制禮之意】甲子吳王祀南郊配以太祖【吳王尊其楊行密廟號太祖】乙丑大赦加徐知誥同平章事領江州觀察使尋以江州為奉化軍以知誥領節度使【徐知誥自團練陞觀察尋自廉軍建節】徐温聞壽州團練使崔太初苛察失民心欲徵之徐知誥曰壽州邊隅大鎮徵之恐為變不若使之入朝因留之温怒曰一崔太初不能制如它人何【史言徐温權略過於知誥】徵為右雄武大將軍 十一月晉王使李存審李嗣源守德勝自將兵攻鎮州張處瑾遣其弟處琪幕僚齊儉謝罪請服晉王不許盡鋭攻之旬日不克【晉王但知野戰決勝負於呼吸之間未知攻城之難也】處瑾使韓正時將千騎突圍出趣定州【趣七喻翻】欲求救于王處直晉兵追至行唐斬之【行唐漢南行唐縣唐屬鎮州九域志在州北五十五里】 契丹主旣許盧文進出兵【張文禮因盧文進求援於契丹事見上】王郁又說之曰【說式芮翻】鎮州美女如雲金帛如山天皇王速往則皆己物也不然為晉王所有矣契丹主以為然悉發所有之衆而南【史言契丹為利所誘而來未有取中國之心】舒嚕后諫曰吾有西樓羊馬之富其樂不可勝窮也【樂音洛勝音升】何必勞師遠出以乘危徼利乎【徼一遥翻】吾聞晉王用兵天下莫敵脱有危敗悔之何及契丹主不聽十二月辛未攻幽州李紹宏嬰城自守【貞明五年晉王令李紹宏提舉幽州軍府事】契丹長驅而南圍涿州旬日拔之擒刺史李嗣弼進攻定州【自幽州西南至涿州一百二十里自涿州至定州二百八十里】王都告急於晉晉王自鎮州將親軍五千救之遣神武都指揮使王思同將兵戍狼山之南以拒之【狼山在定州西北二百里東北至易州八十里】 高季昌遣都指揮使倪可福以卒萬人脩江陵外郭季昌行視【行下孟翻】責功程之慢杖之季昌女為可福子知進婦季昌謂其女曰歸語汝舅吾欲威衆辦事耳以白金數百兩遺之【語牛倨翻遺唯季翻】 是歲漢以尚書左丞倪曙同平章事辰溆蠻侵楚楚寧遠節度副使姚彦章討平之【太祖乾化】<br />
<br />
  【元年姚彦章已弃容州歸潭州而領寧遠節度副使如故】<br />
<br />
  二年春正月壬午朔王都省王處直於西第處直奮拳敺其胷【省悉景翻敺烏口翻】曰逆賊我何負於汝旣無兵刃將噬其鼻都掣袂獲免未幾處直憂憤而卒【掣尺列翻幾居豈翻】 甲午晉王至新城南【按魏牧地形志新城在無極縣時屬祁州】騎白契丹前鋒宿新樂【新樂古鮮虞子國漢為新市縣隋改曰新樂唐屬定州九域志在州西南五十里宋白曰新樂縣隋開皇十六年置新樂者漢成帝時中山孝王母馮昭儀隨王就國建宫於樂里在西鄉呼為西樂城後語訛呼西為新故曰新樂】涉沙河而南將士皆失色士卒有亡去者主將斬之不能止【將即亮翻】諸將皆曰虜傾國而來吾衆寡不敵又聞梁寇内侵宜且還師魏州以救根本或請釋鎮州之圍西入井陘避之晉王猶豫未決郭崇韜曰契丹為王郁所誘本利貨財而來非能救鎮州之急難也【誘音酉難乃旦翻】王新破梁兵【貞明五年破賀瓌于胡柳又破王瓚于戚城是年破戴思遠於德勝】威振夷夏【夏戶雅翻】契丹聞王至心沮氣索【沮在呂翻索昔各翻】苟挫其前鋒遁走必矣李嗣昭自潞州至亦曰今疆敵在前吾有進無退不可輕動以搖人心晉王曰帝王之興自有天命契丹其如我何吾以數萬之衆平定山東【河北之北在太行常山之東】今遇此小虜而避之何面目以臨四海乃自帥鐵騎五千先進【帥讀曰率】至新城北半出桑林契丹萬餘騎見之驚走【契丹素憚晉王不意其至故驚走】晉王分軍為二逐之行數十里獲契丹主之子時沙河橋狹冰薄契丹陷溺死者甚衆是夕晉王宿新樂契丹主車帳在定州城下【契丹主乘奚車卓氊帳覆之寢處其中謂之車帳】敗兵至契丹舉衆退保望都【望都在定州東北六十里范成大北使錄自真定府七十里過沙河至新樂縣又四十五里至定州又五十里至望都縣水經注曰望都縣東有山孤峙帝王世紀曰堯母慶都所居謂之都山張晏曰堯山在北堯母慶都山在南登堯山見都山故望都縣以為名】晉王至定州王都迎謁于馬前請以愛女妻王子繼岌【妻七細翻王都新簒義武以附于晉申之以婚姻自固也】戊戌晉王引兵趣望都【趣七喻翻】契丹逆戰晉王以親軍千騎先進遇奚酋托諾五千騎【酋慈秋翻諾弩罪翻】為其所圍晉王力戰出入數四自午至申不解李嗣昭聞之引三百騎横擊之虜退王乃得出因縱兵奮撃契丹大敗逐北至易州【九域志定州北至易州一百四十里】會大雪彌旬平地數尺契丹人馬無食死者相屬於道【屬之欲翻】契丹主舉手指天謂盧文進曰天未令我至此【旣敗而又遇雪因歸之天屬之欲翻】乃北歸晉王引兵躡之【躡尼輒翻】隨其行止見其野宿之所布藁於地【藁工老翻禾稈也】回環方正皆如編翦雖去無一枝亂者歎曰虜用法嚴乃能如是中國所不及也晉王至幽州使二百騎躡契丹之後曰虜出境即還【還從宣翻】騎恃勇追擊之悉為所擒惟兩騎自它道走免【進軍易退軍難退而能整是難能也契丹之彊其有以哉】契丹主責王郁縶之以歸【以王郁誤之入寇也縶涉立翻】自是不聽其謀晉代州刺史李嗣肱將兵定媯儒武等州【匈奴須知媯州東南距幽州二百二十里儒武又在媯州西北契丹入塞三州皆陷故李嗣肱復定之】授山北都團練使晉王之北攻鎮州也李存審謂李嗣源曰梁人聞我<br />
<br />
  在南兵少【晉王以兵北伐留李存審等守澶魏此兵之在南者也】不攻德勝必襲魏州吾二人聚於此何為不若分軍備之遂分軍屯澶州【時澶州治頓丘】戴思遠果悉楊村之衆趣魏州【趣七喻翻】嗣源引兵先之【先悉薦翻】軍於狄公祠下【唐狄仁傑刺魏州有惠政州人為之立祠】遣人告魏州使為之備思遠至魏店嗣源遣其將石萬全將騎兵挑戰【桃徒了翻】思遠知有備乃西度洹水拔成安大掠而還【還從宣翻又如字】又將兵五萬攻德勝北城重塹複壘斷其出入【重直龍翻斷音短】晝夜急攻之李存審悉力拒守晉王聞德勝勢危二月自幽州赴之五日至魏州思遠聞之燒營遁還楊村 蜀主好為微行【好呼到翻】酒肆倡家靡所不到惡人識之乃下令士民皆著大裁帽【倡音昌惡烏路翻著陟略翻】 晉天平節度使兼侍中閻寶築壘以圍鎮州決呼沱水環之【環音宦按薛史寶攻真定結營西南隅掘塹柵環之決大悲寺漕渠以浸其郛】内外斷絶城中食盡丙午遣五百餘人出求食寶縱其出欲伏兵取之其人遂攻長圍【其人總言鎮兵五百餘人也】寶輕之不為備俄數千人繼至諸軍未集鎮人遂壞長圍而出【壞音怪】縱火攻寶營寶不能拒退保趙州【九域志鎮州南至趙州一百九十里】鎮人悉毁晉之營壘取其芻粟數日不盡晉王聞之以昭義節度使兼中書令李嗣昭為北面招討使以代寶夏四月蜀軍使王承綱女將嫁蜀主取之入宫承綱請之蜀主怒流於茂州女聞父得罪自殺【蜀主取何康之女其夫以之而死取王承綱之女則承綱以之得罪女以之殺身通鑑屢書之以示戒】 甲戌張處瑾遣兵千人迎糧於九門李嗣昭設伏於故營【故營閻寶營也】邀擊之殺獲殆盡餘五人匿于牆墟間嗣昭環馬而射之鎮兵發矢中其腦【射而亦翻中竹仲翻孫策之中頰韓賢之斷脛李嗣昭之中腦皆以主將之重而逞一夫之技以喪身善將者不如是也】嗣昭箙中矢盡【箙以盛矢音房六翻】拔矢于腦以射之一發而殪會日暮還營創流血不止【殪壹計翻創初良翻】是夕卒晉王聞之不御酒肉者累日【御進也】嗣昭遺命悉以澤潞兵授判官任圜【任音壬姓也】使督諸軍攻鎮州號令如一鎮人不知嗣昭之死圜三原人也【史言任圜之才】晉王以天䧺馬步都指揮使振武節度使李存進為北面招討使命嗣昭諸子護喪歸葬晉陽其子繼能不受命帥父牙兵數千自行營擁喪歸潞州【帥讀曰率】晉王遣母弟存渥馳騎追諭之兄弟俱忿欲殺存渥【李嗣昭死守以全潞州撫養創殘葺理軍府備有勲勞身死行陳之間晉王使其護喪歸葬晉陽曾無褒死卹存之命此其所以兄弟俱忿也存渥晉王同母之弟】存渥逃歸嗣昭七子繼儔繼韜繼達繼忠繼能繼襲繼遠繼儔為澤州刺史當襲爵素懦弱繼韜凶狡囚繼儔於别室詐令士卒刧已為留後繼韜陽讓以事白晉王晉王以用兵方殷【以鎮州未下梁兵又來攻擾河上用兵之事方殷也殷盛也】不得已改昭義軍曰安義以繼韜為留後【為李繼韜叛晉附梁張本考異曰按潞州本號昭義軍今以繼韜為安義留後蓋晉王避其父諱改之耳及繼韜降梁梁亦以為匡義】<br />
<br />
  【節度使今人猶謂澤州為安義云】 閻寶慙憤【以鎮州之敗也】疽發於背甲戌卒 漢主巖用術者言遊梅口鎮避災其地近閩之西鄙【九域志梅州程鄉縣有梅口鎮與閩之汀州接境近其靳翻】閩將王延美將兵襲之未至數十里偵者告之【偵丑鄭翻】巖遁逃僅免 五月乙酉晉李存進至鎮州營于東垣渡【真定本東垣漢高帝更名真定其津渡之處猶有東垣之名】夾呼沱水為壘 晉衛州刺史李存儒本姓楊名婆兒以俳優得幸於晉王頗有膂力晉王賜姓名以為刺史專事掊斂【掊蒲侯翻斂力贍翻】防城卒皆徵月課縱歸【月徵其課錢而免其防守之勞】八月莊宅使段凝與步軍都指揮使張朗引兵夜度河襲之詰旦登城【詰去吉翻】執存儒遂克衛州戴思遠又與凝攻陷淇門共城新鄉【共城新鄉二縣皆屬衛州舊唐書地理志曰隋割汲獲嘉二縣地於古新樂城置新鄉縣共城縣漢共縣也唐為共城縣九域志衛州治汲縣熙寧六年廢新鄉縣為鎮屬汲縣汲縣又有淇門鎮共城在州西北五十五里共音恭】於是澶州之西相州之南皆為梁有【九域澶州西至衛州二百四十里相州南至衛州一百五十里】晉人失軍儲三之一梁軍復振帝以張朗為衛州刺史朗徐州人也 九月戊寅朔張處瑾使其弟處球乘李存進無備將兵七千人奄至東垣渡時晉之騎兵亦向鎮州城兩不相遇鎮兵及存進營門存進狼狽引十餘人鬬于橋上鎮兵退晉騎兵斷其後【斷音短】夾擊之鎮兵殆盡存進亦戰沒【當是時晉兵彊天下鎮號為怯晉王杖順討逆宜一鼓而下也鎮人忘王氏百年煦養之恩而為張文禮父子爭一旦之命史建瑭殞斃於前閻寶敗退于後李嗣昭李存進相繼輿尸而歸四人者皆晉之驍將也然則鎮勇而晉怯邪非也鎮人負弑君之罪知城破之日必駢首而就戮故盡死一力以抗晉晉以常勝之兵而臨必死之衆雖兵精將勇至於喪身而不能克是以古之伐罪散其枝黨罪止元惡者誠慮此也】晉王以蕃漢馬步總管李存審為北面招討使鎮州食竭力盡處瑾遣使詣行臺請降未報存審兵至城下丙午夜城中將李再豐為内應密投縋以納晉兵比明畢登【縋馳偽翻比必利翻】執處瑾兄弟家人及其黨高濛李翥齊儉送行臺趙人皆請而食之磔張文禮尸于市【翥章恕翻磔陟格翻】趙王故侍者得趙王遺骸於灰燼中晉王命祭而葬之以趙將符習為成德節度使烏震為趙州刺史趙仁貞為深州刺史李再豐為冀州刺史震信都人也符習不敢當成德辭曰故使無後而未葬習當斬衰以葬之【臣為君服斬衰衰倉回翻】俟禮畢聽命旣葬即詣行臺趙人請晉王兼領成德節度使從之晉王割相衛二州置義寧軍以習為節度使習辭曰魏博霸府不可分也願得河南一鎮習自取之【世固多有能言而不能行者符習陳義不苟而卒不能取河南一鎮是以君子貴於踐言】乃以為天平節度使東南面招討使加李存審兼侍中 十一月戊寅晉特進河東監軍使張承業卒曹太夫人詣其第為之行服如子姪之禮【張存業平李克寧存顥之難以此故曹太夫人深德之為于偽翻】晉王聞其喪不食者累日命河東留守判官何瓚代知河東軍府事【瓚藏旱翻】 十二月晉王以魏博觀察判官晉陽張憲兼鎮冀觀察判官權鎮州軍府事魏州税多逋負晉王以讓司錄濟隂趙季良【唐制諸州有司錄司士司兵司功等諸曹所謂判司也濟隂漢郡名隋置濟隂縣唐帶曹州濟子禮翻】季良曰殿下何時當平河南王怒曰汝職在督税職之不脩何敢預我軍事季良對曰殿下方謀攻取而不愛百姓一旦百姓離心恐河北亦非殿下之有况河南乎王悦謝之自是重之每預謀議 是歲契丹改元天贊 大封王躬乂性殘忍海軍統帥王建殺之【帥所類翻】自立復稱高麗王以開州為東京平壤為西京建儉約寛厚國人安之【徐兢高麗圖經曰高麗王建之先高麗大族也高氏政衰國人以建賢立為君長後唐長興二年自稱權知國事請命于明宗乃拜建大義軍使封高麗王按徐兢宣和之間使高麗進圖經紀載疎略因其國人傳聞遂謂建得國於高氏之後不知建實殺躬又而得國也詳見貞明五年考異】<br />
<br />
  資治通鑑卷二百七十一<br />
<br />
<史部,編年類,資治通鑑>  <br>
   </div> 

<script src="/search/ajaxskft.js"> </script>
 <div class="clear"></div>
<br>
<br>
 <!-- a.d-->

 <!--
<div class="info_share">
</div> 
-->
 <!--info_share--></div>   <!-- end info_content-->
  </div> <!-- end l-->

<div class="r">   <!--r-->



<div class="sidebar"  style="margin-bottom:2px;">

 
<div class="sidebar_title">工具类大全</div>
<div class="sidebar_info">
<strong><a href="http://www.guoxuedashi.com/lsditu/" target="_blank">历史地图</a></strong>  
<a href="http://www.880114.com/" target="_blank">英语宝典</a>  
<a href="http://www.guoxuedashi.com/13jing/" target="_blank">十三经检索</a> 
<br><strong><a href="http://www.guoxuedashi.com/gjtsjc/" target="_blank">古今图书集成</a></strong> 
<a href="http://www.guoxuedashi.com/duilian/" target="_blank">对联大全</a> <strong><a href="http://www.guoxuedashi.com/xiangxingzi/" target="_blank">象形文字典</a></strong> 

<br><a href="http://www.guoxuedashi.com/zixing/yanbian/">字形演变</a>  <strong><a href="http://www.guoxuemi.com/hafo/" target="_blank">哈佛燕京中文善本特藏</a></strong>
<br><strong><a href="http://www.guoxuedashi.com/csfz/" target="_blank">丛书&方志检索器</a></strong> <a href="http://www.guoxuedashi.com/yqjyy/" target="_blank">一切经音义</a>  

<br><strong><a href="http://www.guoxuedashi.com/jiapu/" target="_blank">家谱族谱查询</a></strong>  <strong><a href="http://shufa.guoxuedashi.com/sfzitie/" target="_blank">书法字帖欣赏</a></strong> 
<br>

</div>
</div>


<div class="sidebar" style="margin-bottom:0px;">

<font style="font-size:22px;line-height:32px">QQ交流群9:489193090</font>


<div class="sidebar_title">手机APP 扫描或点击</div>
<div class="sidebar_info">
<table>
<tr>
	<td width=160><a href="http://m.guoxuedashi.com/app/" target="_blank"><img src="/img/gxds-sj.png" width="140"  border="0" alt="国学大师手机版"></a></td>
	<td>
<a href="http://www.guoxuedashi.com/download/" target="_blank">app软件下载专区</a><br>
<a href="http://www.guoxuedashi.com/download/gxds.php" target="_blank">《国学大师》下载</a><br>
<a href="http://www.guoxuedashi.com/download/kxzd.php" target="_blank">《汉字宝典》下载</a><br>
<a href="http://www.guoxuedashi.com/download/scqbd.php" target="_blank">《诗词曲宝典》下载</a><br>
<a href="http://www.guoxuedashi.com/SiKuQuanShu/skqs.php" target="_blank">《四库全书》下载</a><br>
</td>
</tr>
</table>

</div>
</div>


<div class="sidebar2">
<center>


</center>
</div>

<div class="sidebar"  style="margin-bottom:2px;">
<div class="sidebar_title">网站使用教程</div>
<div class="sidebar_info">
<a href="http://www.guoxuedashi.com/help/gjsearch.php" target="_blank">如何在国学大师网下载古籍?</a><br>
<a href="http://www.guoxuedashi.com/zidian/bujian/bjjc.php" target="_blank">如何使用部件查字法快速查字?</a><br>
<a href="http://www.guoxuedashi.com/search/sjc.php" target="_blank">如何在指定的书籍中全文检索?</a><br>
<a href="http://www.guoxuedashi.com/search/skjc.php" target="_blank">如何找到一句话在《四库全书》哪一页?</a><br>
</div>
</div>


<div class="sidebar">
<div class="sidebar_title">热门书籍</div>
<div class="sidebar_info">
<a href="/so.php?sokey=%E8%B5%84%E6%B2%BB%E9%80%9A%E9%89%B4&kt=1">资治通鉴</a> <a href="/24shi/"><strong>二十四史</strong></a>&nbsp; <a href="/a2694/">野史</a>&nbsp; <a href="/SiKuQuanShu/"><strong>四库全书</strong></a>&nbsp;<a href="http://www.guoxuedashi.com/SiKuQuanShu/fanti/">繁体</a>
<br><a href="/so.php?sokey=%E7%BA%A2%E6%A5%BC%E6%A2%A6&kt=1">红楼梦</a> <a href="/a/1858x/">三国演义</a> <a href="/a/1038k/">水浒传</a> <a href="/a/1046t/">西游记</a> <a href="/a/1914o/">封神演义</a>
<br>
<a href="http://www.guoxuedashi.com/so.php?sokeygx=%E4%B8%87%E6%9C%89%E6%96%87%E5%BA%93&submit=&kt=1">万有文库</a> <a href="/a/780t/">古文观止</a> <a href="/a/1024l/">文心雕龙</a> <a href="/a/1704n/">全唐诗</a> <a href="/a/1705h/">全宋词</a>
<br><a href="http://www.guoxuedashi.com/so.php?sokeygx=%E7%99%BE%E8%A1%B2%E6%9C%AC%E4%BA%8C%E5%8D%81%E5%9B%9B%E5%8F%B2&submit=&kt=1"><strong>百衲本二十四史</strong></a>  <a href="http://www.guoxuedashi.com/so.php?sokeygx=%E5%8F%A4%E4%BB%8A%E5%9B%BE%E4%B9%A6%E9%9B%86%E6%88%90&submit=&kt=1"><strong>古今图书集成</strong></a>
<br>

<a href="http://www.guoxuedashi.com/so.php?sokeygx=%E4%B8%9B%E4%B9%A6%E9%9B%86%E6%88%90&submit=&kt=1">丛书集成</a> 
<a href="http://www.guoxuedashi.com/so.php?sokeygx=%E5%9B%9B%E9%83%A8%E4%B8%9B%E5%88%8A&submit=&kt=1"><strong>四部丛刊</strong></a>  
<a href="http://www.guoxuedashi.com/so.php?sokeygx=%E8%AF%B4%E6%96%87%E8%A7%A3%E5%AD%97&submit=&kt=1">說文解字</a> <a href="http://www.guoxuedashi.com/so.php?sokeygx=%E5%85%A8%E4%B8%8A%E5%8F%A4&submit=&kt=1">三国六朝文</a>
<br><a href="http://www.guoxuedashi.com/so.php?sokeytm=%E6%97%A5%E6%9C%AC%E5%86%85%E9%98%81%E6%96%87%E5%BA%93&submit=&kt=1"><strong>日本内阁文库</strong></a> <a href="http://www.guoxuedashi.com/so.php?sokeytm=%E5%9B%BD%E5%9B%BE%E6%96%B9%E5%BF%97%E5%90%88%E9%9B%86&ka=100&submit=">国图方志合集</a> <a href="http://www.guoxuedashi.com/so.php?sokeytm=%E5%90%84%E5%9C%B0%E6%96%B9%E5%BF%97&submit=&kt=1"><strong>各地方志</strong></a>

</div>
</div>


<div class="sidebar2">
<center>

</center>
</div>
<div class="sidebar greenbar">
<div class="sidebar_title green">四库全书</div>
<div class="sidebar_info">

《四库全书》是中国古代最大的丛书,编撰于乾隆年间,由纪昀等360多位高官、学者编撰,3800多人抄写,费时十三年编成。丛书分经、史、子、集四部,故名四库。共有3500多种书,7.9万卷,3.6万册,约8亿字,基本上囊括了古代所有图书,故称“全书”。<a href="http://www.guoxuedashi.com/SiKuQuanShu/">详细>>
</a>

</div> 
</div>

</div>  <!--end r-->

</div>
<!-- 内容区END --> 

<!-- 页脚开始 -->
<div class="shh">

</div>

<div class="w1180" style="margin-top:8px;">
<center><script src="http://www.guoxuedashi.com/img/plus.php?id=3"></script></center>
</div>
<div class="w1180 foot">
<a href="/b/thanks.php">特别致谢</a> | <a href="javascript:window.external.AddFavorite(document.location.href,document.title);">收藏本站</a> | <a href="#">欢迎投稿</a> | <a href="http://www.guoxuedashi.com/forum/">意见建议</a> | <a href="http://www.guoxuemi.com/">国学迷</a> | <a href="http://www.shuowen.net/">说文网</a><script language="javascript" type="text/javascript" src="https://js.users.51.la/17753172.js"></script><br />
  Copyright &copy; 国学大师 古典图书集成 All Rights Reserved.<br>
  
  <span style="font-size:14px">免责声明:本站非营利性站点,以方便网友为主,仅供学习研究。<br>内容由热心网友提供和网上收集,不保留版权。若侵犯了您的权益,来信即刪。scp168@qq.com</span>
  <br />
ICP证:<a href="http://www.beian.miit.gov.cn/" target="_blank">鲁ICP备19060063号</a></div>
<!-- 页脚END --> 
<script src="http://www.guoxuedashi.com/img/plus.php?id=22"></script>
<script src="http://www.guoxuedashi.com/img/tongji.js"></script>

</body>
</html>
