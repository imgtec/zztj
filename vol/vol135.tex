<!DOCTYPE html PUBLIC "-//W3C//DTD XHTML 1.0 Transitional//EN" "http://www.w3.org/TR/xhtml1/DTD/xhtml1-transitional.dtd">
<html xmlns="http://www.w3.org/1999/xhtml">
<head>
<meta http-equiv="Content-Type" content="text/html; charset=utf-8" />
<meta http-equiv="X-UA-Compatible" content="IE=Edge,chrome=1">
<title>資治通鑒_136-資治通鑑卷一百三十五_136-資治通鑑卷一百三十五</title>
<meta name="Keywords" content="資治通鑒_136-資治通鑑卷一百三十五_136-資治通鑑卷一百三十五">
<meta name="Description" content="資治通鑒_136-資治通鑑卷一百三十五_136-資治通鑑卷一百三十五">
<meta http-equiv="Cache-Control" content="no-transform" />
<meta http-equiv="Cache-Control" content="no-siteapp" />
<link href="/img/style.css" rel="stylesheet" type="text/css" />
<script src="/img/m.js?2020"></script> 
</head>
<body>
 <div class="ClassNavi">
<a  href="/24shi/">二十四史</a> | <a href="/SiKuQuanShu/">四库全书</a> | <a href="http://www.guoxuedashi.com/gjtsjc/"><font  color="#FF0000">古今图书集成</font></a> | <a href="/renwu/">历史人物</a> | <a href="/ShuoWenJieZi/"><font  color="#FF0000">说文解字</a></font> | <a href="/chengyu/">成语词典</a> | <a  target="_blank"  href="http://www.guoxuedashi.com/jgwhj/"><font  color="#FF0000">甲骨文合集</font></a> | <a href="/yzjwjc/"><font  color="#FF0000">殷周金文集成</font></a> | <a href="/xiangxingzi/"><font color="#0000FF">象形字典</font></a> | <a href="/13jing/"><font  color="#FF0000">十三经索引</font></a> | <a href="/zixing/"><font  color="#FF0000">字体转换器</font></a> | <a href="/zidian/xz/"><font color="#0000FF">篆书识别</font></a> | <a href="/jinfanyi/">近义反义词</a> | <a href="/duilian/">对联大全</a> | <a href="/jiapu/"><font  color="#0000FF">家谱族谱查询</font></a> | <a href="http://www.guoxuemi.com/hafo/" target="_blank" ><font color="#FF0000">哈佛古籍</font></a> 
</div>

 <!-- 头部导航开始 -->
<div class="w1180 head clearfix">
  <div class="head_logo l"><a title="国学大师官网" href="http://www.guoxuedashi.com" target="_blank"></a></div>
  <div class="head_sr l">
  <div id="head1">
  
  <a href="http://www.guoxuedashi.com/zidian/bujian/" target="_blank" ><img src="http://www.guoxuedashi.com/img/top1.gif" width="88" height="60" border="0" title="部件查字,支持20万汉字"></a>


<a href="http://www.guoxuedashi.com/help/yingpan.php" target="_blank"><img src="http://www.guoxuedashi.com/img/top230.gif" width="600" height="62" border="0" ></a>


  </div>
  <div id="head3"><a href="javascript:" onClick="javascript:window.external.AddFavorite(window.location.href,document.title);">添加收藏</a>
  <br><a href="/help/setie.php">搜索引擎</a>
  <br><a href="/help/zanzhu.php">赞助本站</a></div>
  <div id="head2">
 <a href="http://www.guoxuemi.com/" target="_blank"><img src="http://www.guoxuedashi.com/img/guoxuemi.gif" width="95" height="62" border="0" style="margin-left:2px;" title="国学迷"></a>
  

  </div>
</div>
  <div class="clear"></div>
  <div class="head_nav">
  <p><a href="/">首页</a> | <a href="/ShuKu/">国学书库</a> | <a href="/guji/">影印古籍</a> | <a href="/shici/">诗词宝典</a> | <a   href="/SiKuQuanShu/gxjx.php">精选</a> <b>|</b> <a href="/zidian/">汉语字典</a> | <a href="/hydcd/">汉语词典</a> | <a href="http://www.guoxuedashi.com/zidian/bujian/"><font  color="#CC0066">部件查字</font></a> | <a href="http://www.sfds.cn/"><font  color="#CC0066">书法大师</font></a> | <a href="/jgwhj/">甲骨文</a> <b>|</b> <a href="/b/4/"><font  color="#CC0066">解密</font></a> | <a href="/renwu/">历史人物</a> | <a href="/diangu/">历史典故</a> | <a href="/xingshi/">姓氏</a> | <a href="/minzu/">民族</a> <b>|</b> <a href="/mz/"><font  color="#CC0066">世界名著</font></a> | <a href="/download/">软件下载</a>
</p>
<p><a href="/b/"><font  color="#CC0066">历史</font></a> | <a href="http://skqs.guoxuedashi.com/" target="_blank">四库全书</a> |  <a href="http://www.guoxuedashi.com/search/" target="_blank"><font  color="#CC0066">全文检索</font></a> | <a href="http://www.guoxuedashi.com/shumu/">古籍书目</a> | <a   href="/24shi/">正史</a> <b>|</b> <a href="/chengyu/">成语词典</a> | <a href="/kangxi/" title="康熙字典">康熙字典</a> | <a href="/ShuoWenJieZi/">说文解字</a> | <a href="/zixing/yanbian/">字形演变</a> | <a href="/yzjwjc/">金 文</a> <b>|</b>  <a href="/shijian/nian-hao/">年号</a> | <a href="/diming/">历史地名</a> | <a href="/shijian/">历史事件</a> | <a href="/guanzhi/">官职</a> | <a href="/lishi/">知识</a> <b>|</b> <a href="/zhongyi/">中医中药</a> | <a href="http://www.guoxuedashi.com/forum/">留言反馈</a>
</p>
  </div>
</div>
<!-- 头部导航END --> 
<!-- 内容区开始 --> 
<div class="w1180 clearfix">
  <div class="info l">
   
<div class="clearfix" style="background:#f5faff;">
<script src='http://www.guoxuedashi.com/img/headersou.js'></script>

</div>
  <div class="info_tree"><a href="http://www.guoxuedashi.com">首页</a> > <a href="/SiKuQuanShu/fanti/">四库全书</a>
 > <h1>资治通鉴</h1> <!--         下载:【右键另存为】即可 --></div>
  <div class="info_content zj clearfix">
  
<div class="info_txt clearfix" id="show">
<center style="font-size:24px;">136-資治通鑑卷一百三十五</center>
    資治通鑑卷一百三十五 宋 司馬光 撰<br />
<br />
  胡三省 音註<br />
<br />
  齊紀一【起屠維協洽盡昭陽大淵獻凡五年 按蕭子顯齊書崔祖思傳宋朝初議封太祖為梁公祖思啓太祖曰䜟書云金刀利刃齊刈之今宜稱齊實應天命太祖從之遂以齊建國】太祖高皇帝【諱道成姓蕭氏字紹伯小字鬬將本居東海蘭陵縣中都鄉中都里晉惠帝時分東海為蘭陵郡故為蘭陵郡人高祖整過江居晉陵武進縣之東城里時寓居江左者皆僑置本土加以南名更為南蘭陵人整生雋雋生樂子樂子生承之承之生帝】<br />
<br />
  建元元年【是年四月受禪始改元建元】春正月甲辰以江州刺史蕭嶷為都督荆湘等八州諸軍事荆州刺史【嶷魚力翻】尚書左僕射王延之為江州刺史安南長史蕭子良為督會稽等五郡諸軍事會稽太守【去年已命蕭映蕭晃分鎮兖豫矣嶷道成次子也子良道成之孫也江左之勢莫重於上流莫富於東土故又分布子孫以居之會工外翻守式又翻】初沈攸之欲聚衆開民相告士民坐執役者甚衆嶷至鎮一日罷遣三千餘人府州儀物務存儉約輕刑薄斂【斂力瞻翻】所部大悅 辛亥以竟陵世子賾為尚書僕射進號中軍大將軍開府儀同三司【道成進爵竟陵郡公故賾為竟陵世子賾士革翻】 太傅道成以謝朏有重名必欲引參佐命以為左長史嘗置酒與論魏晉故事因曰石苞不早勸晉文死方慟哭方之馮異非知機也【晉文王薨石苞自揚州奔喪慟哭曰基業如此而以人臣終乎馮異勸漢光即尊位道成言石苞不能早勸晉文為禪代之事比之馮異勸漢光石苞非知機者也欲以此言感動謝朏耳朏敷尾翻】朏曰晉文世事魏室必將身終北面借使魏依唐虞故事亦當三讓彌高【言三以天下讓則節行彌高也】道成不悅甲寅以朏為侍中更以王儉為左長史【更工衡翻】 丙辰以給事黄門侍郎蕭長懋為雍州刺史【長懋道成嫡長孫也】 二月丙子邵陵殤王友卒 辛巳魏太皇太后及魏主如代郡温泉 甲午詔申前命命太傅贊拜不名【前命見上卷上年】己亥魏太皇太后及魏主如西宫【西宫魏太祖天賜元年所築】<br />
<br />
  三月癸卯朔日有食之 甲辰以太傅為相國總百揆封十郡為齊公【時以青州之齊郡徐州之梁郡南徐州之蘭陵魯郡琅邪東海晉陵義興揚州之吳郡會稽十郡封】加九錫其驃騎大將軍揚州牧南徐州刺史如故【驃匹妙翻騎奇寄翻】己巳詔齊國官爵禮儀並倣天朝【朝直遥翻】丙午以世子賾領南豫州刺史 楊運長去宣城郡還家齊公遣人殺之【楊運長守宣城見上卷宋昇明元年】凌源令潘智與運長厚善【蕭子顯齊志臨淮郡有凌縣應劭曰凌水出凌縣西南入淮酈道元曰凌水出凌縣東流逕其縣故城東而東南流入淮】臨川王綽義慶之孫也【義慶長沙王第二子襲道規封】綽遣腹心陳讚說智曰【說輸芮翻】君先帝舊人身是宗室近屬如此形勢豈得久全若招合内外計多有從者臺城内人常有此心苦無人建意耳智即以告齊公庚戌誅綽兄弟及其黨與 甲寅齊公受策命赦其境内以石頭為世子宮一如東宫褚淵引何曾自魏司徒為晉丞相故事求為齊官齊公不許以王儉為齊尚書右僕射領吏部儉時年二十八夏四月壬申朔進齊公爵為王增封十郡【時又增徐州之南梁陳潁川陳留南兖州之盱眙山陽秦廣陵海陵南沛等十郡】甲戌武陵王贊卒非疾也【史言齊殺之】丙戌加齊王殊禮進世子為太子辛卯宋順帝下詔禪位于齊壬辰帝當臨軒不肯出逃于佛蓋之下【自晉以來宫中有佛屋以嚴事佛像上為寶蓋以覆之宋帝逃於其下】王敬則勒兵殿庭以板輿入迎帝太后懼自帥閹人索得之【帥讀曰率閹衣廉翻索山客翻】敬則啓譬令出引令升車帝收淚謂敬則曰欲見殺乎敬則曰出居别宫耳官先取司馬家亦如此帝泣而彈指曰願後身世世勿復生帝王家【復扶又翻】宫中皆哭帝拍敬則手曰必無過慮當餉輔國十萬錢【敬則時為輔國將軍史言帝庸闇】是日百僚陪位侍中謝朏在直當解璽綬陽為不知曰有何公事傳詔云解璽綬授齊王【傳詔屬中書舍人出入宣傳詔旨又攷南史郡府謂之傳教天臺謂之傳詔璽斯氏翻綬音受】朏曰齊自應有侍中乃引枕卧傳詔懼使朏稱疾欲取兼人【欲取兼侍中者】朏曰我無疾何所道遂朝服步出東掖門仍登車還宅【朝直遥翻】乃以王儉為侍中解璽綬禮畢帝乘畫輪車【畫輪車者車輪施文畫也晉志云畫輪車上開四望綠油幢朱絲絡兩箱裏飾以金錦黄金塗五采蕭子顯曰漆畫輪車金塗校飾如輦微有减降杜佑曰晉制駕車以采漆畫輪轂上起四夾杖左右開四望綠油纁朱絲青交絡其上形如輦其下猶犢車】出東掖門就東邸【掖音亦】問今日何不奏鼓吹左右莫有應者右光祿大夫王琨華之從父弟也【王華早入宋公霸府元嘉初輔政吹尺睡翻從才用翻下同】在晉世已為郎中至是攀車獺尾慟哭【獺毛可以辟塵故懸之於車】曰人以夀為歡老臣以夀為戚既不能先驅螻蟻【謂不能早死也】乃復頻見此事嗚咽不自勝【復扶又翻勝音升】百官雨泣【言涕泣如雨也宋永初元年受晉禪歲在庚申八主六十年而亡】司空兼太保禇淵等奉璽綬帥百官詣齊宫勸進【帥讀曰率】王辭讓未受淵從弟前安成太守炤謂淵子賁曰司空今日何在賁曰奉璽綬在齊大司馬門炤曰不知汝家司空將一家物與一家亦復何謂【炤與照同之笑翻復扶又翻下乃復同賁後辭爵廬墓盖深感炤之言也】甲午王即皇帝位于南郊還宫大赦改元奉宋順帝為汝隂王優崇之禮皆倣宋初築宫丹陽【丹陽南史作丹徒丹陽為是齊史云築宫於丹陽故縣】置兵守衛之宋神主遷汝隂廟諸王皆降為公自非宣力齊室餘皆除國獨置南康華容蓱鄉三國以奉劉穆之王弘何無忌之後【王弘之後不除國以王儉佐命耳蓱與萍同萍鄉縣吳寶鼎二年置宋白曰楚昭王度江獲萍實於此今縣北有萍實里楚王臺因以名縣】除國者凡百二十人二臺官僚依任攝職【二臺謂宋臺齊臺也】名號不同員限盈長者别更詳議【長直亮翻多而有餘也】以褚淵為司徒賓客賀者滿座禇炤歎曰彦回少立名行何意披猖至此【披猖言披靡而猖蹷也披普皮翻少詩照翻行下孟翻】門戶不幸乃復有今日之拜使彦回作中書郎而死不當為一名士邪名德不昌乃復有期頤之夀【曲禮曰人生百年曰期頤鄭注云期要也頤養也不知衣服食味孝子要盡養道而已】淵固辭不拜奉朝請河東裴顗上表數帝過惡掛冠徑去帝怒殺之【奉朝請者奉朝會請召而已非有職任也裴顗在宋朝既無職任又無車犖奇節惟不食齊粟遂得垂名青史君子惡没世而名不稱正為此也朝直遥翻顗魚豈翻數所具翻】太子賾請殺謝朏帝曰殺之遂成其名正應容之度外耳久之因事廢于家帝問為政於前撫軍行參軍沛國劉瓛【瓛胡官翻】對曰政在孝經凡宋氏所以亡陛下所以得者皆是也陛下若戒前車之失加之以寛厚雖危可安若循其覆轍雖安必危矣帝歎曰儒者之言可寶萬世 丙申魏主如崞山【崞音郭】 丁酉以太子詹事張緒為中書令齊國左衛將軍陳顯達為中護軍右衛將軍李安民為中領軍緒岱之兄子也 戊戌以荆州刺史嶷為尚書令驃騎大將軍開府儀同三司揚州刺史【嶷魚力翻驃匹妙翻騎奇寄翻】南兖州刺史映為荆州刺史 帝命羣臣各言得失淮南宣城二郡太守劉善明【江左僑立淮南郡於宣城郡界故善明兼守二郡】請除宋氏大明泰始以來諸苛政細制以崇簡易【易以䜴翻】又以為交州險遠宋末政苛遂至怨叛【宋明帝泰始四年李長仁據交州而叛】今大化創始宜懷以恩德且彼土所出唯有珠寶實非聖朝所須之急【朝直遥翻】討伐之事謂宜且停給事黄門郎清河崔祖思亦上言以為人不學則不知道【禮記學記之言上時掌翻】此悖逆禍亂所由生也【悖蒲内翻又蒲没翻】今無員之官空受祿力彫耗民財【無員之官員外官也下所謂限外之人是也祿者所食之祿力者所役之人】宜開文武二學課臺府州國限外之人各從所樂依方習業【漢書賈山傳使皆務其方而高其節註云方道也樂音洛】若有廢惰者遣還故郡經藝優殊者待以不次又今陛下雖躬履節儉而羣下猶安習侈靡宜褒進朝士之約素清修者【朝直遥翻】貶退其驕奢荒淫者則風俗可移矣宋元嘉之世凡事皆責成郡縣世祖徵求急速以郡縣遟緩始遣臺使督之【使疏吏翻下同】自是使者所在旁午競作威福營私納賂公私勞擾會稽太守聞喜公子良上表極陳其弊【會工外翻】以為臺有求須但明下詔敕為之期會則人思自竭若有稽遟自依糾坐之科今雖臺使盈湊會取正屬所辦【謂使者雖多亦當取辦於所屬也】徒相疑憤反更淹懈宜悉停臺使【懈居隘翻】員外散騎郎劉思效上言【散悉亶翻騎奇寄翻】宋自大明以來漸見凋弊徵賦有加而天府尤貧【天府謂天子之府藏也】小民嗷嗷殆無生意而貴族富室以侈麗相高乃至山澤之民不敢采食其水草陛下宜一新王度【王度王法也】革正其失上皆加褒賞或以表付外使有司詳擇所宜奏行之己亥詔二宫諸王悉不得營立屯邸封畧山湖【二宫謂上宫及東宮上宫諸王皇子也東宮諸王皇孫也杜預曰不以道取曰畧又曰畧封畧也】 魏主還平城【還從宣翻又如字】 魏秦州刺史尉洛侯雍州刺史宜都王目辰長安鎮將陳提等皆坐貪殘不法洛侯目辰伏誅提徙邊【尉紆勿翻雍於用翻將即亮翻】又詔以候官千數【魏太祖置候官以伺察内外】重罪受賕不列輕罪吹毛發舉【言吹毛求疵也】宜悉罷之更置謹直者數百人使防邏街術【更工衡翻邏郎佐翻術讀曰遂又食聿翻說文曰術邑中道】執喧鬬者而已自是吏民始得安業 自泰始以來内外多虞將帥各募部曲屯聚建康【將即亮翻帥所類翻】李安民上表以為自非淮北常備外餘軍悉皆輸遣【輸送也】若親近宜立隨身者聽限人數上從之五月辛亥詔斷衆募【斷丁管翻】 壬子上賞佐命之功禇淵王儉等進爵增戶各有差 【考異曰南史崔祖思傳曰帝將加九錫内外皆贊成之祖思獨曰公以仁恕匡社稷執股肱之義君子愛人以德不宜如此帝聞而非之曰祖思遠同荀令豈孤所望也由此不復處任職而禮貌甚重垣崇祖受密旨參訪朝臣光祿大夫垣閎曰身受宋氏厚恩復蒙明公眷接進不敢同退不敢異冠軍將軍崔文仲與崇祖意同及帝受禪閎存故爵文仲崇祖皆封侯祖思加官而已按宋朝初議封帝為梁公祖思啓高祖曰識云金刀利刃齊刈之今宜稱齊實應天命從之然則祖思安得盡誠節於宋今刪之】處士何點謂人曰【處昌呂翻】我作齊書已竟贊云淵既世族儉亦國華不賴舅氏遑恤國家點尚之孫也【何尚之仕宋貴顯於太祖世祖之時】淵母宋始安公主繼母吳郡公主又尚巴西公主儉母武康公主又尚陽羨公主故點云然 己未或走馬過汝隂王之門衛士恐有為亂者奔入殺王而以疾聞上不罪而賞之辛酉殺宋宗室隂安公燮等無少長皆死前豫州刺史劉澄之遵考之子也【少詩照翻長知兩翻劉遵考見一百二十八卷孝武孝建二年】與禇淵善淵為之固請【為于偽翻】曰澄之兄弟不武且於劉宗又踈故遵考之族獨得免【遵考弟思考有子季連亂蜀】 丙寅追尊皇考曰宣皇帝皇妣陳氏曰孝皇后 【考異曰南史在四月甲午今從齊書】丁卯封皇子鈞為衡陽王 上謂兖州刺史垣崇祖曰吾新得天下索虜必以納劉昶為辭侵犯邊鄙【南謂北為索虜以魏本索頭種也索昔各翻】夀陽當虜之衝非卿無以制此虜也乃徙崇祖為豫州刺史 六月丙子誅游擊將軍姚道和以其貳於沈攸之也【事見上卷宋順帝昇明之元年也】 甲子立王太子賾為皇太子皇子嶷為豫章王映為臨川王晃為長沙王曅為武陵王暠為安成王鏘為鄱陽王鑠為桂陽王鑑為廣陵王【曅筠輒翻暠古老翻鏘于羊翻鑠書藥翻】皇孫長懋為南郡王 乙酉葬宋順帝于遂寜陵 帝以建康居民舛雜多姦盜欲立符伍以相檢括右僕射王儉諫曰京師之地四方輻凑必也持符於事既煩理成不曠謝安所謂不爾何以為京師也乃止 初交州刺史李長仁卒從弟叔獻代領州事以號令未行遣使求刺史於宋【從才用翻使疏吏翻】宋以南海太守沈煥為交州刺史以叔獻為煥寜遠司馬武平新昌二郡太守【吳孫皓建衡三年討扶嚴夷以其地置武平郡是年又分交阯立新興郡晉武帝泰康三年更名新昌皆屬交州隋廢武平郡為隆平縣廢新昌郡為嘉寜縣並屬交阯郡唐改隆平為太平仍屬交阯以嘉寜縣為峯州】叔獻既得朝命【朝直遥翻】人情服從遂發兵守險不納煥煥停鬱林病卒秋七月丁未詔曰交阯比景獨隔書朔【言其拒命不受正朔也古者天子常以季冬頒來歲十二月之朔于諸侯諸侯受而藏之祖廟至月朔則以特羊告廟請而行之】斯乃前運方季因迷遂往宜曲赦交州即以叔獻為刺史撫安南土 魏葭蘆鎮主楊廣香請降【降戶江翻】丙辰以廣香為沙州刺史【以輿地記參考此沙州當置於唐利州景谷縣界】 八月乙亥魏主如方山【方山在平城北如渾水上魏主與馮太后將營夀陵於此故數至其地】丁丑還宫 上聞魏將入寇九月乙巳以豫章王嶷為荆湘二州刺史都督如故【是年春正月以嶷刺荆州都督八州今兼刺湘州】以臨川王映為揚州刺史 丙午以司空禇淵領尚書令 壬子魏以侍中司徒東陽王丕為太尉侍中尚書右僕射陳建為司徒侍中尚書代人苟頹為司空【魏書官氏志神元時餘部内入諸姓有若干氏後改為苟氏】 己未魏安樂厲王長樂謀反賜死【樂音洛】 庚申魏隴西宣王源賀卒 冬十月己巳朔魏大赦 癸未汝隂太妃王氏卒【即宋明帝王皇后也順帝禪位封汝隂王太后降為太妃】諡曰宋恭皇后 初晉夀民李烏奴與白水氐楊成等寇梁州【水經注白水西北出臨洮縣東南西傾山水色白濁東南入隂平界氐居水土者號白水氐】梁州刺史范柏年說降烏奴擊成破之【說輸芮翻降戶江翻】及沈攸之事起【見上卷宋順帝昇明元年】柏年遣兵出魏興聲云入援實候望形勢事平朝廷遣王玄邈代之詔柏年與烏奴俱下烏奴勸柏年不受代柏年計未决玄邈已至柏年乃留烏奴於漢中還至魏興盤桓不進左衛率豫章胡諧之嘗就柏年求馬柏年曰馬非狗也安能應無已之求待使者甚薄使者還語諧之曰柏年云胡諧之何物狗所求無厭【語牛倨翻厭於鹽翻】諧之恨之譖於上曰柏年恃險聚衆欲專據一州上使雍州刺史南郡王長懋誘柏年啓為府長史【雍於用翻誘音酉】柏年至襄陽上欲不問諧之曰見虎格得而縱上山乎【格捕也鬬也】甲午賜柏年死李烏奴叛入氐依楊文弘引氐兵千餘人寇梁州陷白馬戍【白馬戍在沔水北即陽平關也】王玄邈使人詐降誘烏奴【降戶江翻】烏奴輕兵襲州城玄邈伏兵邀擊大破之烏奴挺身復走入氐初玄邈為青州刺史【宋泰始初玄邈據盤陽以拒魏因用為青州刺史】上在淮隂為宋太宗所疑【事見一百三十二卷泰始六年】欲北附魏遣書結玄邈玄邈長史清河房叔安曰將軍居方州之重無故舉忠孝而弃之三齊之士寜蹈東海而死耳【自項羽分立諸侯王分齊地為三王後遂稱齊地為三齊猶關中稱三秦也】不敢隨將軍也玄邈乃不答上書 【考異曰南史云仍遣叔安奉表詣闕告之帝於路執之并求玄邈表叔安曰王將軍表上天子不上將軍且僕之所言利國家不利將軍無所應問荀伯玉勸帝殺之帝曰物各為主無所責也按太祖時為邉將若執叔安又不殺使應不復為宋臣齊書無此事今不取】及罷州還至淮隂嚴軍直過至建康啓太宗稱上有異志及上為驃騎【蒼梧既弑上進驃騎大將軍】引為司馬玄邈甚懼而上待之如初及破烏奴上曰玄邈果不負吾意遇也叔安為寜蜀太守【宋永初中分廣漢為寜蜀郡】上賞其忠正欲用為梁州會病卒 十一月辛亥立皇太子妃裴氏 癸丑魏遣假梁郡王嘉督二將出淮隂隴西公琛督三將出廣陵河東公薛虎子督三將出夀陽奉丹陽王劉昶入寇【景和之初昶奔魏將即亮翻下同】許昶以克復舊業世胙江南稱藩于魏【建置社稷曰胙又守社稷曰胙】蠻酋桓誕請為前驅【宋明泰豫元年誕降魏酋慈由翻】以誕為南征西道大都督義陽民謝天蓋自稱司州刺史欲以州附魏魏樂陵鎮將韋珍引兵渡淮應接【魏置樂陵鎮於比陽在今唐州界】豫章王嶷遣中兵參軍蕭惠朗將二千人助司州刺史蕭景先討天蓋韋珍畧七千餘戶而去 【考異曰齊蕭景先傳云天蓋與虜相構扇景先言于督府豫章王遣惠朗助景先討天蓋黨與虜尋遣偽南部尚書類跋屯汝南洛州刺史昌黎王馮莎屯清丘景先嚴備待敵虜退魏韋珍傳云天蓋自署司州刺史規以内附事泄為道成將崔慧景所攻圍詔珍帥在鎮士馬渡淮援接時道成聞珍將至遣將荀元賓據淮逆拒珍珍腹背奮擊破之天蓋尋為左右所殺降於慧景珍乘勝馳進又破慧景擁降民七千餘戶内徙表置城陽剛陵義陽三郡以處之按魏將無類跋馮莎而慧景亦非討天蓋之將蓋時二國之史各出傳聞互有訛謬今約取二史大槩而用之】景先上之從子也【從才用翻】南兖州刺史王敬則聞魏將濟淮委鎮還建康士民驚散既而魏竟不至上以其功臣不問上之輔宋也遣驍騎將軍王洪範使柔然約與共攻魏洪範自蜀出吐谷渾歷西域乃得逹【驍堅堯翻騎奇寄翻下同使疏吏翻吐從暾入聲谷音浴 考異曰齊書作王洪軌今從齊紀】至是柔然十餘萬騎寇魏至塞上而還【還從宣翻又如字】 是歲魏詔中書監高允議定律令允雖篤老而志識不衰詔以允家貧養薄令樂部絲竹十人五日一詣允以娛其志朝晡給膳朔望致牛酒月給衣服綿絹入見則備几杖問以政治【晡奔謨翻見賢遍翻治直吏翻】 契丹莫賀弗勿干帥部落萬餘口入附于魏居白狼水東【契丹酋帥曰莫賀弗隋書曰契丹與庫莫奚皆東胡種為慕容氏所破竄於松漠之間是時為高麗所侵求内附于魏水經注白狼水出右北平白狼縣東南東北流逕龍城西南又東南流至遼東房縣入于遼水初學記狼河附黄龍城東北下即白狼水契欺訖翻又音喫帥讀曰率】<br />
<br />
  二年春正月戊戌朔大赦 以司空褚淵為司徒尚書右僕射王儉為左僕射淵不受 【考異曰齊書建元二年正月以淵為司徒十二月戊戌以淵為司徒四年六月癸卯以司徒禇淵為司空八月癸卯司徒禇淵薨淵傳三年為司徒又固讓四年寢疾遜位改授司空及薨詔曰司徒奄至薨逝盖二年正月辭十二月受耳紀傳前後各不相顧】辛丑上祀南郊 魏隴西公琛等攻拔馬頭戍殺太守劉從【琛丑林翻】乙卯詔内外纂嚴發兵拒魏徵南郡王長懋為中軍將軍鎮石頭 魏廣川莊王略卒 魏師攻鍾離徐州刺史崔文仲擊破之文仲遣軍主崔孝伯渡淮攻魏茌眉戌主龍得侯等殺之【茌仕疑翻孫愐曰龍姓也 考異曰齊紀作龍渴侯今從齊書】文仲祖思之族人也羣蠻依阻山谷連帶荆湘雍郢司五州之境【雍於用翻】聞魏師入寇乃盡發民丁南襄城蠻秦遠乘虛寇潼陽殺縣令【蕭子顯齊志寜蠻府領郡有南襄城東襄城北襄城中襄城郡蓋因羣蠻部落分署為郡也五代志淮安郡慈丘縣有後魏襄城郡沈約宋志汶陽郡領潼陽沮陽高安三縣蓋皆宋初置也水經注東汶陽郡沮陽縣沮水出其西北東南逕汶陽郡北即高安縣界郡治鍚城縣故新城郡之下邑義熙初分新城立郡其地當在臨沮縣西蕭子顯曰桓温以臨沮西界水陸紆險道帶蠻蜓田土肥美立為汶陽郡以處流民】司州蠻引魏兵寇平昌平昌戍主荀元賓擊破之北上黄蠻文勉德寇汶陽 【考異曰齊紀作文施德今從齊書】汶陽太守戴元賓弃城奔江陵【晉武帝平吳割中廬之南鄉臨沮之北鄉立上黄縣治軨鄉屬襄陽郡晉安帝分屬長寜郡宋明帝以名與文帝陵同改為永寜郡五代志竟陵郡章山縣西魏置上黄郡今荆門軍長林縣即古之長寜縣有章山九域志曰山即禹貢所謂内方也宋白曰上黄縣隋改南漳縣】豫章王嶷遣中兵參軍劉伾緒將千人討之至當陽【章懷太子賢曰當陽縣西北即臨沮故城將即亮翻下同】勉德請降秦遠遁去【降戶江翻】魏將薛道標引兵趣夀陽上使齊郡太守劉懷慰作冠軍將軍薛淵書以招道標【趣七喻翻冠古玩翻】魏人聞之召道標還【還從宣翻又如字】使梁郡王嘉代之懷慰乘民之子也【劉乘民見一百三十一卷宋明帝泰始二年】二月丁卯朔嘉與劉昶寇夀陽將戰昶四向拜將士流涕縱横【縱子容翻】曰願同戮力以雪讎恥魏步騎號二十萬【騎奇寄翻】豫州刺史垣崇祖集文武議之欲治外城堰肥水以自固皆曰昔佛狸入寇【見一百二十五卷宋文帝元嘉二十七年治直之翻佛音弼】南平王士卒完盛數倍於今猶以郭大難守退保内城且自有肥水未嘗堰也恐勞而無益崇祖曰若弃外城虜必據之外修樓櫓内築長圍則坐成擒矣守郭築堰是吾不諫之策也【言策已先定足以制敵不為人所諫止】乃於城西北堰肥水【據水經肥水自黎漿亭北流過夀春城東此立堰於西北者西北虜衝也又因上流之勢可决以灌虜今安豐軍有小史埭即崇祖决堰處】堰北築小城周為深塹【塹七艷翻】使數千人守之曰虜見城小以為一舉可取必悉力攻之以謀破堰吾縱水衝之皆為流尸矣魏人果蟻附攻小城崇祖著白紗帽肩輿上城【著則略翻上時掌翻】晡時决堰下水魏攻城之衆漂墜塹中人馬溺死以千數【溺奴狄翻】魏師退走 謝天蓋部曲殺天蓋以降【降戶江翻】 宋自孝建以來政綱弛紊簿籍訛謬上詔黄門郎會稽虞玩之等更加檢定曰黄籍民之大紀國之治端【杜佑曰黄籍者戶口版籍也會工外翻治直吏翻下求治同】自頃巧偽日甚何以釐革玩之上表以為元嘉中故光禄大夫傅隆年出七十猶手自書籍躬加隱校【隱者痛覈其實也】今欲求治取正必在勤明令長【長知兩翻】愚謂宜以元嘉二十七年籍為正更立明科一聽首悔【首式又翻】迷而不返依制必戮若有虚昧州縣同科上從之 上以羣蠻數為叛亂【數所角翻】分荆益置巴州以鎮之壬申以三巴校尉明慧昭為巴州刺史領巴東太守【宋明帝泰始二年以三峽險隘山蠻寇賊議立三巴校尉以鎮之尋省順帝昇明二年復置校戶教翻】是時齊之境内有州二十三郡三百九十縣千四百八十五【州二十三揚南徐豫南豫南兖北兖北徐青冀江廣交越荆巴郢司雍湘梁秦益寜也郡三百九十有寄治者有新置者有俚郡獠郡荒郡左郡無屬縣者有或荒無民戶者郡縣之建置雖多而名存實亡境土蹙於宋大明之時矣】乙酉崔文仲遣軍主陳靖拔魏竹邑殺戍主白仲都崔叔延破魏睢陵殺淮陽太守梁惡【竹邑漢沛郡之竹縣也後漢晉曰竹邑後廢魏盖於故地置戍也賢曰竹故城在今徐州符離縣睢陵漢縣屬臨淮後漢晉屬下邳宋孝武大明元年度屬於濟隂時入魏置淮陽郡】 三月丁酉朔以侍中西昌侯鸞為郢州刺史鸞帝兄始安貞王道生之子也早孤為帝所養恩過諸子【為後鸞奪國殺帝子孫張本】 魏劉昶以雨水方降表請還師魏人許之丙午遣車騎大將軍馮熙將兵迎之【騎奇寄翻將即亮翻】 夏四月辛巳魏主如白登山五月丙申朔如火山【杜佑曰雲州治雲中縣縣界有白登山白登臺水經注曰白登南有武周川川東南有火山山上有火井南北六十七步廣減尺許源深不見底炎勢上升常若微雷響以草爨之則煙騰火】壬寅還平城 自晉以來建康宫之外城唯設竹籬而有六門會有發白虎樽者【晉志正旦元會設白虎樽於殿庭樽盖上施白獸若有能獻直言者則發此樽飲酒案禮白獸樽乃杜舉之遺式為白獸盖是復代所為示忌憚也白獸即白虎晉書避唐諱改曰獸】言白門三重關竹籬穿不完【重直龍翻】上感其言命改立都牆 李烏奴數乘間出寇梁州【數所角翻間古莧翻】豫章王嶷遣中兵參軍王圖南將益州兵從劒閣掩擊之梁南秦二州刺史崔慧景發梁州兵屯白馬與圖南覆背擊烏奴【嶷魚力翻將即亮翻覆當作腹】大破之烏奴走保武興 【考異曰魏書帝紀八月慧景寇武興今從慧景傳】慧景祖思之族人也秋七月辛亥魏主如火山 戊午皇太子穆妃裴氏<br />
<br />
  卒【妃卒謚曰穆】詔南郡王長懋移鎮西州 角城戍主舉城降魏【角城注見下年降戶江翻】秋八月丁酉魏遣徐州刺史梁郡王嘉迎之又遣平南將軍郎大檀等三將出胊城【魏收志琅邪朐縣有胊城胊音劬】將軍白吐頭等二將出海西【海西即漢海西縣地也宋明帝失淮北僑立青州于贑榆縣泰始七年割贑榆置鬰縣立海西郡齊明帝以為東海郡東魏武定七年改海西郡又分襄賁置海西縣】將軍元泰等二將出連口【連口漣水入淮之口也時在襄賁縣界隋改襄賁縣為漣水縣杜佑曰楚州漣水縣有連口渡應劭曰賁音肥】將軍封延等三將出角城鎮南將軍賀羅出下蔡【據班志下蔡春秋之州來國也為楚所㓕後吳取之至夫差遷蔡昭侯於此後四世侯齊竟為楚所滅故曰下蔡漢為縣屬沛郡後省東魏武定六年以梁黄城戍為下蔡郡隋為縣屬汝隂郡以下垣崇祖徙下蔡戍攷之則此戍置于淮水之西五代時周世宗徙夀春治下蔡即其地】同入寇 甲辰魏主如方山戊申遊武州山石窟寺【水經注曰武周川水東南流水側有石祗洹舍幷諸窟室比丘尼所居也其水又東轉靈巖南鑿石開山因巖結搆真容巨壯法世所稀據道元之言浮屠氏巨麗處也】庚戌還平城 崔慧景遣長史裴叔保攻李烏奴於武興為氐王楊文弘所敗【敗補邁翻】 九月甲午朔日有食之 丙午柔然遣使來聘【使疏吏翻】 汝南太守常元真龍驤將軍胡青苟降於魏【驤思將翻降戶江翻】 閏月辛巳遣領軍李安民循行清泗諸戍以備魏【行下孟翻】 魏梁郡王嘉帥衆十萬圍朐山【帥讀曰率】胊山戌主玄元度嬰城固守【孫偭曰玄姓也】青冀二州刺史范陽盧紹之遣子奐將兵助之【將郎亮翻下同】庚寅元度大破魏師臺遣軍主崔靈建等將萬餘人自淮入海夜至各舉兩炬魏師望見遁去 冬十月王儉固請解選職許之加儉侍中以太子詹事何戢領選【選須絹翻戢阻立翻又疾立翻】上以戢資重欲加常侍褚淵曰聖旨每以蟬冕不宜過多臣與王儉既已左珥若復加戢則八座遂有三貂【自漢以來侍中常侍皆左貂令僕與列曹尚書為八座據戢傳帝為領軍戢為司徒左長史相與來往數興歡宴戢蓋龍潛之舊也復扶又翻】若帖以驍游亦不為少【沈約曰驍騎將軍游擊將軍並漢襍號將軍也魏置為中軍及晉以領護左右衛驍游為六軍不少者謂其取數已多也少詩沼翻】乃以戢為吏部尚書加驍騎將軍【驍堅堯翻騎奇寄翻】 甲辰以沙州刺史楊廣香為西秦州刺史又以其子炅為武都太守【炅古迥翻又古惠翻】 丁未魏以昌黎王馮熙為西道都督與征南將軍桓誕出義陽鎮南將軍賀羅出鍾離同入寇 淮北四州民不樂屬魏【四州入魏事見一百三十三卷宋明帝奉始三年樂音洛】常思歸江南上多遣間諜誘之【間古莧翻諜從協翻誘音酉】於是徐州民桓標之 【考異曰魏書蘭陵民桓富蓋即標之也今從齊書】兖州民徐猛子等所在蠭起為寇盜聚衆保五固推司馬朗之為主魏遣淮陽王尉元平南將軍薛虎子等討之【尉紆勿翻】 十一月戊寅丹楊尹王僧䖍上言郡縣獄相承有上湯殺囚名為救疾實行寃暴【因囚有時行瘟疫宜汗遂上湯以蒸殺之上時掌翻】豈有死生大命而濳制下邑愚謂囚病必先刺郡【刺謂州刺史郡謂郡守也或曰書病囚之姓名而白之於郡曰刺】求職司與醫對共診驗【職司謂郡曹掌刑獄者】遠縣家人省視然後處治【處治謂處方治病也省悉景翻處昌呂翻治直之翻】上從之戊子以楊難當之孫後起為北秦州刺史武都王鎮武興 十二月戊戌以司空褚淵為司徒淵入朝以腰扇障日【腰扇佩之於腰今謂之摺疊扇朝直遥翻下同】征虜功曹劉祥從側過曰作如此舉止羞面見人扇障何益淵曰寒士不遜祥曰不能殺袁劉安得免寒士【謂殺袁粲劉乘也】祥穆之之孫也祥好文學而性韻剛踈撰宋書譏斥禪代王儉密以聞坐徙廣州而卒【劉穆之宋朝佐命元臣祥以是得罪於齊可謂無忝厥祖矣好呼到翻】太子宴朝臣於玄圃【東宫有玄圃】右衛率沈文季與褚淵語相失文季怒曰淵自謂忠臣不知死之日何面目見宋明帝太子笑曰沈率醉矣【史言褚淵失節人得以面斥之率所律翻】 壬子以豫章王嶷為中書監司空揚州刺史以臨川王映為都督荆雍等九州諸軍事荆州刺史【嶷魚力翻雍於用翻】 是歲魏尚書令王叡進爵中山王加鎮東大將軍置王官二十二人以中書侍郎鄭羲為傅郎中令以下皆當時名士又拜叡妻丁氏為妃【此事傳之史策可以為王叡榮邪】<br />
<br />
  三年春正月封皇子鋒為江夏王【夏戶雅翻】 魏人寇淮陽圍軍主成買於甬城【甬城當作角城水經注角城在下邳睢陵縣南臨淮水其地據濟水入淮之口後梁武帝置淮陽郡角城為縣屬焉高間曰角城去淮陽十八里杜佑曰角城晉安帝義熙中置在宿遷縣界五代志作甬城】上遣領軍將軍李安民為都督與軍主周盤龍等救之魏人緣淮大掠江北民皆驚走渡江成買力戰而死盤龍之子奉叔以二百人陷陳深入【陳讀曰陣下魏陳同】魏以萬餘騎張左右翼圍之【騎奇寄翻下同】或告盤龍云奉叔已没盤龍馳馬奮矟直突魏陳所向披靡【矟色角翻披普彼翻】奉叔已出復入求盤龍【復扶又翻】父子兩騎縈擾魏數萬之衆莫敢當者魏師遂敗殺傷萬計魏師退李安民等引兵追之戰於孫溪渚又破之【孫溪渚在淮陽之北清水之濱】 己卯魏主南廵司空苟頹留守丁亥魏主至中山 二月丁卯朔魏大赦 丁酉游擊將軍桓康復敗魏師於淮陽【復扶又翻下復如同敗蒲邁翻】進攻樊諧城拔之 【考異曰齊紀作樊階城今從齊書】 魏主自中山如信都癸卯復如中山庚戌還至肆州【魏收志曰肆州治九原天賜二年為鎮真君七年置肆州領永安秀容鴈門三郡宋白曰魏置肆州理秀容城秀容本漢陽曲縣地周武帝徙肆州於鴈門】沙門法秀以妖術惑衆謀作亂於平城【妖於遥翻】苟頹帥禁兵收掩悉擒之【帥讀曰率】魏主還平城有司囚法秀加以籠頭鐵鎻無故自解魏人穿其頸骨祝之曰若果有神當令穿肉不入遂穿以狥三日乃死議者或欲盡殺道人 【考異曰齊書魏虜傳咸陽王欲盡殺道人按咸陽王禧時尚幼太和九年始封恐非也】馮太后不可乃止 垣崇祖之敗魏師也恐魏復寇淮北【敗補邁翻復扶又翻下今復同】乃徙下蔡戍於淮東既而魏師果至欲攻下蔡聞其内徙欲夷其故城己酉崇祖引兵渡淮擊魏大破之殺獲千計 【考異曰齊書作丁卯按是月辛卯朔無丁卯今從齊紀】 晉宋之際荆州刺史多不領南蠻校尉【校戶教翻下偏校同】别以重人居之豫章王嶷為荆湘二州刺史領南蠻【嶷魚力翻】嶷罷更以侍中王奐為之奐固辭曰西土戎燼之後痍毁難復【復如字】今復割撤太府【自晉永嘉之亂張氏擅命河西以都府為太府今復扶又翻】制置偏校崇望不足助彊語實交能相弊且資力既分職司增廣衆勞務倍文案滋煩竊以為國計非允癸丑罷南蠻校尉官【晉武帝置南蠻校尉至是罷】 三月辛酉朔魏主如肆州己巳還平城 魏法秀之亂事連蘭臺御史張求等百餘人皆以反法當族尚書令王叡請誅首惡宥其餘黨下詔應誅五族者降為三族三族者門誅門誅止其身所免千餘人 夏四月己亥魏主如方山馮太后樂其山川【樂音洛】曰它日必葬我於是不必祔山陵也乃為太后作夀陵【為于偽翻】又建永固石室於山上欲以為廟【水經注曰方嶺上有文明太后陵陵之東北有高祖陵二陵之南有永固堂堂之四隅雉列榭階欄檻及扉戶梁壁椽瓦悉文石也檐前兩柱採洛陽之八風谷石為之雕鏤隱起以金銀間雲雉有若錦焉堂之内外四側結兩石扶帳青石屏風以文石為緣並隱起忠孝之容題刻貞順之名廟前錫石為碑獸碑石至佳左右列柏四周迷禽暗日南川表二石闕御路下望靈泉宫池皎若圖鏡】 桓標之等有衆數萬塞險求援庚子詔李安民督諸將往迎之【將即亮翻】又使兖州刺史周山圖自淮入清倍道應接淮北民桓磊磈破魏師於抱犢固【磊落猥翻磈口猥翻魏收志蘭陵郡承縣有抱犢山】李安民赴救遲留標之等皆為魏所滅餘衆得南歸者尚數千家魏人亦掠三萬餘口歸平城 【考異曰魏書云南征諸將擊破蕭道成游擊將軍桓康於淮陽道成豫州刺史垣榮祖寇下蔡昌黎王馮熙擊破之假梁郡王嘉大破道成將俘獲三萬餘口送平城今從齊書齊紀亦以魏書參之】 魏任城康王雲卒【任音壬】五月壬戌鄧至王像舒遣使入貢于魏鄧至者羌之别種國於宕昌之南【北史曰鄧至者白水羌也世為羌豪因地名號曰鄧至其地自街亭以東平武以西汶嶺以北宕昌以南或曰鄧至者因鄧艾所至因以為名杜佑曰鄧至今交川郡之南通化郡之北文川臨翼同昌郡之地也使疏吏翻種章勇翻】 六月壬子大赦 甲辰魏中山宣王王叡卒叡疾病太皇太后魏主屢至其家視疾及卒贈太宰立廟於平城南文士為叡作哀詩及誄者百餘人【為于偽翻哀詩起於黄鳥古者卿大夫没君命有司累其功德為文以哀之曰誄孔穎達曰誄累也累列生時行迹誄之以作謚音魯水翻】及葬自稱親姻義舊縗絰哭送者千餘人【縗倉回翻】魏主以叡子中散大夫襲代叡為尚書令領吏部曹【散悉亶翻】 戊午魏封皇叔簡為齊郡王猛為安豐王 秋七月己未朔日有食之 上使後軍參軍車僧朗使於魏【朗使疏吏翻下同】甲子僧朗至平城魏主問曰齊輔宋日淺何故遽登大位對曰虞夏登庸身陟元后【夏戶雅翻虞夏事見尚書】魏晉匡輔貽厥子孫【事見漢魏晉紀】時宜各異耳辛酉柔然别帥他稽帥衆降魏【别帥所類翻稽帥讀曰率降戶江翻】<br />
<br />
  楊文弘遣使請降詔復以為北秦州刺史【復扶又翻宋順帝昇明元年文弘降魏】先是楊廣香卒【先悉薦翻】其衆半奔文弘半奔梁州文弘遣楊後起進據白水上雖授以官爵而隂敕晉夀太守楊公則使伺便圖之【伺相吏翻】 宋昇明中遣使者殷靈誕荀昭先如魏聞上受禪靈誕謂魏典客曰【典客秦官也漢武帝泰初元年更名大鴻臚至晉大鴻臚屬官又有典客令】宋魏通好【好呼到翻】憂患是同宋今滅亡魏不相救何用和親及劉昶入寇靈誕請為昶司馬不許九月庚午魏閱武於南郊因宴羣臣置車僧朗於靈誕下僧朗不肯就席曰靈誕昔為宋使今為齊民乞魏主以禮見處【處昌呂翻】靈誕遂與相忿詈劉昶賂宋降人解奉君於會刺殺僧朗魏人收奉君誅之【解戶買翻姓也刺七亦翻】厚送僧朗之喪放靈誕等南歸及世祖即位昭先具以靈誕之語啓聞靈誕坐下獄死【史竟言其事下遐稼翻】辛未柔然主遣使來聘與上書謂上為足下自稱曰<br />
<br />
  吾遺上師子皮袴褶【遺于季翻袴褶騎服也褶寔入翻】約共伐魏 魏尉元薛虎子克五固斬司馬朗之東南諸州皆平【東南諸州謂淮北四州于魏境為東南也】尉元入為侍中都曹尚書薛虎子為彭城鎮將【將即亮翻】遷徐州刺史時州鎮戍兵資絹自隨不入公庫虎子上表以為國家欲取江東先須積穀彭城切惟在鎮之兵不減數萬資糧之絹人十二匹用度無凖未及代下【代更也下替也】不免飢寒公私損費今徐州良田十萬餘頃水陸肥沃清汴通流足以溉灌若以兵絹市牛可得萬頭興置屯田一歲之中且給官食半兵芸殖餘兵屯戍且耕且守不妨捍邊一年之收過於十倍之絹蹔時之耕足充數載之食【蹔與暫同載子亥翻】於後兵資皆貯公庫【貯丁呂翻】五稔之後穀帛俱溢非直戍卒豐飽亦有吞敵之勢魏人從之虎子為政有惠愛兵民懷之會沛郡太守邵安下邳太守張攀以汙為虎子所案【沛下邳皆徐州所統】各遣子上書告虎子與江南通魏主曰虎子必不然推案果虛詔安攀皆賜死二子各鞭一百 吐谷渾王拾寅卒【吐從暾入聲谷音浴】世子度易侯立冬十月戊子朔以度易侯為西秦河二州刺史河南王 魏中書令高閭等更定新律成【更工衡翻】凡八百三十二章門房之誅十有六大辟二百三十五雜刑三百七十七【辟毗亦翻】 初高昌王闞伯周卒【高昌建國稱王自伯周始】子義成立是歲其從兄首歸殺義成自立【從才用翻】高車王可至羅殺首歸兄弟【可至羅盖即阿伏至羅可當作阿】以敦煌張明為高昌王國人殺明立馬儒為王【敦徒門翻】<br />
<br />
  四年春正月壬戌詔置學生二百人以中書令張緒為國子祭酒【晉武帝咸寜四年初立國子學置國子祭酒】 甲戌魏大赦 三月庚申上召司徒褚淵尚書左僕射王儉受遺詔輔太子壬戌殂于臨光殿【年五十六】太子即位大赦高帝沉深有大量博學能文性清儉主衣中有玉介導【主衣主供御衣服禁中有主衣庫】上敇中書曰留此正是興長病源【長丁丈翻今知兩翻】即命擊碎仍檢按有何異物皆隨此例每曰使我治天下十年【治直之翻】當使黄金與土同價 乙丑以褚淵錄尚書事王儉為侍中尚書令車騎將軍張敬兒開府儀同三司丁卯以前將軍王奐為尚書左僕射庚午以豫章王嶷為太尉【騎奇寄翻嶷魚力翻】 庚辰魏主臨虎圈【圈求遠翻】詔曰虎狼猛暴取捕之日每多傷害既無所益損費良多從今勿復捕貢【復扶又翻】 夏四月庚寅上大行諡曰高皇帝廟號太祖丙午葬泰安陵【在晉陵武進縣上考承之先葬于此所謂武進陵也】 辛卯追尊穆妃為皇后【建元二年太子妃裴氏卒諡曰穆】六月甲申朔立南郡王長懋為皇太子丙申立太子妃王氏妃琅邪人也【妃王韶之之孫】封皇子聞喜公子良為竟陵王臨汝公子卿為廬陵王應城公子敬為安陸王【臨汝縣屬汝南郡蕭子顯齊志應城縣屬安陸郡】江陵公子懋為晉安王枝江公子隆為隨郡王子真為建安王皇孫昭業為南郡王 司徒褚淵寢疾自表遜位世祖不許【書新君廟號以别大行】淵固請懇切癸卯以淵為司空領驃騎將軍侍中錄尚書如故【驃匹妙翻騎奇寄翻】 秋七月魏發州郡五萬人治靈丘道【靈丘道自代郡靈丘南越大山至中山即古之飛狐道也治直之翻】 吏部尚書濟陽江謐【濟子禮翻】性諂躁【躁則到翻】太祖殂謐恨不豫顧命上即位謐又不遷官以此怨望誹謗會上不豫謐詣豫章王嶷請間曰【嶷魚力翻】至尊非起疾東宮又非才公今欲作何計上知之使御史中丞沈冲奏謐前後罪惡庚寅賜謐死【沈攸之之反江謐建假黄鉞之議以此位通顯既以諂躁徼幸則以諂躁致禍亦宜也】 癸卯南康文簡公褚淵卒世子侍中賁恥其父失節服除遂不仕以爵讓其弟蓁屏居墓下終身【慕側詵翻屛必郢翻】 九月丁巳以國哀罷國子學 氐王楊文弘卒諸子皆幼乃以兄子後起為嗣九月辛酉魏以後起為武都王文弘子集始為白水太守【五代史志武都郡建威縣魏置白水郡唐貞觀初省建威入將利縣】既而集始自立為王後起擊破之 魏以荆州巴氐擾亂【魏世祖泰延五年置荆州於上洛領上洛上庸魏興等郡巴與氐各是一種】以鎮西大將軍李崇為荆州刺史崇顯祖之舅子也將之鎮勑發陜秦二州兵送之【魏收地形志太和十一年置陜州是年太和七年也當考陜失冉翻】崇辭曰邉人失和本怨刺史今奉詔代之自然安靖但須一詔而已不煩發兵自防使之懷懼也魏朝從之崇遂輕將數十騎馳至上洛【朝直遥翻將即亮翻騎奇寄翻】宣詔慰諭民夷帖然崇命邊戍掠得齊人者悉還之由是齊人亦還其生口二百許人二境交和無復烽燧之警【復扶又翻】久之徙兖州刺史兖土舊多劫盜崇命村置一樓樓皆懸鼓盜發之處亂擊之旁村始聞者以一擊為節次二次三俄頃之間聲布百里皆發人守險要由是盜發無不擒獲其後諸州皆效之自崇始也 辛未以征南將軍王僧䖍為左光祿大夫開府儀同三司以尚書右僕射王奐為湘州刺史 宋故建平王景素主簿何昌㝢記室王摛及所舉秀才劉璡【摛抽知翻璡資辛翻】前後上書陳景素德美為之訟寃【景素死見一百三十四卷宋蒼梧王元徽四年為于偽翻】冬十月辛丑詔聽以士禮還葬舊塋璡瓛之弟也【瓛胡官翻】 十一月魏高祖將親祠七廟命有司具儀法依古制備牲牢器服及樂章自是四時常祀皆舉之<br />
<br />
  世祖武皇帝上之上【諱賾字宣遠高帝長子也】<br />
<br />
  永明元年春正月辛亥上祀南郊大赦改元 詔以邊境寜晏治民之官普復舊秩【宋文帝元嘉二十七年有魏師以軍興減百官奉祿淮南太守諸葛闡求減俸祿比内百官於是諸州郡縣丞尉並悉同減至明帝時軍旅不息府藏空虛内外百官並斷奉祿治直之翻下同】 以太尉豫章王嶷領太子太傅嶷不參朝務而常密獻謀畫上多從之【嶷魚力翻朝直遥翻】 壬戌立皇弟銳為南平王鏗為宜都王皇子子明為武昌王子罕為南海王 二月辛巳以征虜將軍楊炅為沙州刺史隂平王【炅楊廣香之子也炅古迥翻又古惠翻】 辛丑以宕昌王梁彌機為河凉二州刺史【宕徒浪翻】鄧至王像舒為西凉州刺史宋末以治民之官六年過久乃以三年為斷【斷丁亂翻下專】<br />
<br />
  【斷同】謂之小滿而遷換去來又不能依三年之制三月癸丑詔自今一以小滿為限有司以天文失度請禳之【禳而羊翻】上曰應天以實不以文我克己求治思隆惠政若災眚在我【治直吏翻眚所景翻】禳之何益 夏四月壬午詔袁粲劉秉沈攸之雖末節不終而始誠可錄皆命以禮改葬【三人死見一百三十四卷宋順帝昇明三年】 上之為太子也自以年長【長知兩翻】與太祖同創大業【晉安王子勛之亂帝亦起兵沈攸之反帝據湓城為衆軍節度】朝事大小率皆專斷【朝直遥翻斷丁亂翻】多違制度信任左右張景真景真驕侈被服什物僭擬乘輿【被皮義翻乘繩證翻】内外畏之莫敢言者司空諮議荀伯玉素為太祖所親厚【諮議即諮議參軍】歎曰太子所為官終不知豈得畏死蔽官耳目我不啓聞誰當啓者因太子拜陵【拜永安泰安陵也皆在武進】密以啓太祖太祖怒命檢校東宫太子拜陵還至方山晩將泊舟【建康城東北有方山埭直瀆所經也據沈瑀傳方山埭在湖熟縣界杜佑曰東晉至陳西有石頭津東有方山津各置津主一人賊曹一人直水五人以檢察禁物宋白曰丹陽記云秦始皇鑿金陵方山斷處為瀆則今淮水經城中入大江是曰秦淮】豫章王嶷自東府乘飛鷰東迎太子【飛鷰名馬也】告以上怒之意太子夜歸入宫太祖亦停門籥待之明日太祖使南郡王長懋聞喜公子良宣敇詰責【詰去吉翻】并示以景真罪狀使以太子令收景真殺之太子憂懼稱疾月餘太祖怒不解晝卧太陽殿王敬則直入叩頭啟太祖曰官有天下日淺太子無事被責【被皮義翻】人情恐懼願官往東宫解釋之太祖無言敬則因大聲宣旨裝束往東宫又勑大官設饌【饌雛戀翻又雛晚翻】呼左右索輿【索山客翻】太祖了無動意敬則索衣被太祖仍牽強登輿【被皮義翻強其兩翻】太祖不得已至東宫召諸王宴於玄圃長沙王晃捉華蓋【捉亦執也】臨川王映執雉尾扇【雉尾扇編雉尾為之以障乘輿】聞喜公子良持酒鎗【鎗楚庚翻盛酒之器按太平御覧鎗即鐺字但鐺非可持者】南郡王長懋行酒太子及豫章王嶷王敬則自捧酒饌至暮盡醉乃還【還從宣翻又如字】太祖嘉伯玉忠藎愈見親信軍國密事多委使之權動朝右【朝直遥翻】遭母憂去宅二里許冠蓋已塞路左率蕭景先侍中王晏共弔之【塞悉則翻左率左衛率也率所律翻】自旦至暮始得前比出飢乏氣息惙然【比必寐翻惙積雪翻疲乏也】憤悒形於聲貌【悒乙及翻】明日言於太祖曰臣等所見二宫門庭比荀伯玉宅可張雀羅矣【門外可設雀羅用漢書語師古注曰言其寂静無人行也】晏敬弘之從子也【王敬弘見用於元嘉中從才用翻】驍騎將軍陳胤叔先亦白景真及太子得失而語太子皆云伯玉以聞【驍堅堯翻騎奇寄翻語牛倨翻】太子由是深怨伯玉太祖隂有以豫章王嶷代太子之意而嶷事太子愈謹【嶷魚力翻】故太子友愛不衰豫州刺史垣崇祖不親附太子會崇祖破魏兵【見上太祖建元三年】太祖召還朝與之密謀【朝直遥翻】太子疑之曲加禮待謂曰世間流言我已豁懷自今以富貴相付崇祖拜謝會太祖復遣荀伯玉【復扶又翻】勑以邊事受旨夜發不得辭東宫太子以為不盡誠益銜之太祖臨終指伯玉以屬太子【屬之欲翻】上即位崇祖累遷五兵尚書伯玉累遷散騎常侍【散悉亶翻】伯玉内懷憂懼上以伯玉與崇祖善恐其為變加意撫之丁亥下詔誣崇祖招結江北荒人欲與伯玉作亂皆收殺之 庚子魏主如崞山【崞音郭】壬寅還宫 閏月癸丑魏主後宫平凉林氏生子恂【後魏分安定郡置平涼郡領鶉隂隂密二縣】大赦文明太后以恂當為太子賜林氏死自撫養恂 五月戊寅朔魏主如武州山石窟佛寺車騎將軍張敬兒好信夢【好呼到翻】初為南陽太守其妻<br />
<br />
  尚氏夢一手熱如火及為雍州夢一胛熱【雍於用翻胛音甲】為開府夢半身熱敬兒意欲無限常謂所親曰吾妻復夢舉體熱矣【復扶又翻】又自言夢舊村社樹高至天上聞而惡之【惡烏路翻】垣崇祖死敬兒内自疑會有人告敬兒遣人至蠻中貨易【貨易即貿易也以我所有易我所無】上疑其有異志會上於華林園設八關齋【釋氏之戒一不殺生二不偷盜三不邪淫四不妄語五不飲酒食肉六不著花鬘瓔珞香油塗身歌舞倡伎故往觀聼七不得坐高廣大床八不得過齋後喫食已上八戒故為八關雜錄名義云八戒者俗衆所受一日一夜戒也謂八戒一齋通謂八關齋明以禁防為義也】朝臣皆預於坐收敬兒【朝直遥翻坐徂卧翻】敬兒脫冠貂投地曰此物誤我丁酉殺敬兒并其四子敬兒弟恭兒常慮為兄禍所及居於冠軍【冠軍縣自漢以來屬南陽郡唐為鄧州臨湍縣我朝建隆初廢臨湍入禳縣冠古玩翻】未常出襄陽村落深阻牆垣重復敬兒每遣信輒上馬屬鞬【重直龍翻屬之欲翻鞬居言翻馬上盛弓矢之器】然後見之敬兒敗問至席卷入蠻【卷讀曰捲】後自出上恕之敬兒女為征北諮議參軍謝超宗子婦超宗謂丹陽尹李安民曰往年殺韓信今年殺彭越尹欲何計【用漢書薛公語激發安民使之作亂也】安民具啓之上素惡超宗輕慢【惡烏路翻】使兼御史中丞袁彖奏彈超宗丁巳收付廷尉徙越巂於道賜死【巂音髓】以彖語不刻切又使左丞王逡之奏彈彖輕文畧奏撓法容非【撓奴教翻】彖坐免官禁錮十年超宗靈運之孫【超宋靈運子鳳之子】彖顗之弟子也【衰顗死於義嘉之難顗魚豈翻】 秋七月丁丑魏主及太后如神淵池【魏太和元年起永樂遊觀殿於北苑穿神淵池】甲申如方山 魏使假員外散騎常侍頓丘李彪來聘 侍中左光祿大夫開府儀同三司王僧䖍固辭開府【去年王僧䖍除開府】謂兄子儉曰汝任重於朝行登三事【朝直遥翻】我若復有此授【復扶又翻下亦復同】乃是一門有二台司吾實懼焉累辭不拜上乃許之戊戌加僧䖍特進儉作長梁齋制度小過僧䖍視之不悅竟不入戶儉即日毁之初王弘與兄弟集會任子孫戲適僧達跳下地作虎子僧綽正坐采蠟燭珠為鳳凰僧達奪取打壞亦復不惜僧䖍累十二博棊既不墜落亦不重作【重直龍翻】弘歎曰僧達俊爽當不減人然恐終危吾家【僧達死見一百二十八卷宋世祖孝建三年】僧綽當以名義見美僧䖍必為長者位至公台【長知兩翻】已而皆如其言 八月庚申驍騎將軍王洪範自柔然還經塗三萬餘里【經塗謂所經由之路王洪範出使事見高帝建元二年還從宣翻又如字】 冬十月丙寅遣驍騎將軍劉纘聘於魏魏主客令李安世主之【主客令即典客令也】魏人出内藏之寶使賈人鬻之於市【藏徂浪翻賈音古】纘曰魏金玉大賤當由山川所出安世曰聖朝不貴金玉故賤同瓦礫【礫郎狄翻】纘初欲多市聞其言内慙而止纘屢奉使至魏馮太后遂私幸之【史言馮后淫縱使疏吏翻下同】 十二月乙巳朔日有食之癸丑魏始禁同姓為婚 王儉進號衛將軍參掌選<br />
<br />
  事【選須絹翻】 是歲省巴州【置巴州見上高帝建元二年】 魏秦州刺史于洛侯性殘酷刑人必斷腕拔舌【斷丁管翻腕烏貫翻】分懸四體合州驚駭州民王元夀等一時俱反有司劾奏之【劾戶槩翻义戶得翻】魏主遣使至州於洛侯常刑人處宣告吏民然後斬之齊州刺史韓麒麟為政尚寛從事劉普慶說麒麟曰公杖節方夏而無所誅斬何以示威麒麟曰刑罰所以止惡仁者不得已而用之今民不犯法又何誅乎若必斷斬然後可以立威當以卿應之普慶慚懼而起【說輸芮翻夏戶雅翻斷丁亂翻】<br />
<br />
  資治通鑑卷一百三十五  <br>
   </div> 

<script src="/search/ajaxskft.js"> </script>
 <div class="clear"></div>
<br>
<br>
 <!-- a.d-->

 <!--
<div class="info_share">
</div> 
-->
 <!--info_share--></div>   <!-- end info_content-->
  </div> <!-- end l-->

<div class="r">   <!--r-->



<div class="sidebar"  style="margin-bottom:2px;">

 
<div class="sidebar_title">工具类大全</div>
<div class="sidebar_info">
<strong><a href="http://www.guoxuedashi.com/lsditu/" target="_blank">历史地图</a></strong>  
<a href="http://www.880114.com/" target="_blank">英语宝典</a>  
<a href="http://www.guoxuedashi.com/13jing/" target="_blank">十三经检索</a> 
<br><strong><a href="http://www.guoxuedashi.com/gjtsjc/" target="_blank">古今图书集成</a></strong> 
<a href="http://www.guoxuedashi.com/duilian/" target="_blank">对联大全</a> <strong><a href="http://www.guoxuedashi.com/xiangxingzi/" target="_blank">象形文字典</a></strong> 

<br><a href="http://www.guoxuedashi.com/zixing/yanbian/">字形演变</a>  <strong><a href="http://www.guoxuemi.com/hafo/" target="_blank">哈佛燕京中文善本特藏</a></strong>
<br><strong><a href="http://www.guoxuedashi.com/csfz/" target="_blank">丛书&方志检索器</a></strong> <a href="http://www.guoxuedashi.com/yqjyy/" target="_blank">一切经音义</a>  

<br><strong><a href="http://www.guoxuedashi.com/jiapu/" target="_blank">家谱族谱查询</a></strong>  <strong><a href="http://shufa.guoxuedashi.com/sfzitie/" target="_blank">书法字帖欣赏</a></strong> 
<br>

</div>
</div>


<div class="sidebar" style="margin-bottom:0px;">

<font style="font-size:22px;line-height:32px">QQ交流群9:489193090</font>


<div class="sidebar_title">手机APP 扫描或点击</div>
<div class="sidebar_info">
<table>
<tr>
	<td width=160><a href="http://m.guoxuedashi.com/app/" target="_blank"><img src="/img/gxds-sj.png" width="140"  border="0" alt="国学大师手机版"></a></td>
	<td>
<a href="http://www.guoxuedashi.com/download/" target="_blank">app软件下载专区</a><br>
<a href="http://www.guoxuedashi.com/download/gxds.php" target="_blank">《国学大师》下载</a><br>
<a href="http://www.guoxuedashi.com/download/kxzd.php" target="_blank">《汉字宝典》下载</a><br>
<a href="http://www.guoxuedashi.com/download/scqbd.php" target="_blank">《诗词曲宝典》下载</a><br>
<a href="http://www.guoxuedashi.com/SiKuQuanShu/skqs.php" target="_blank">《四库全书》下载</a><br>
</td>
</tr>
</table>

</div>
</div>


<div class="sidebar2">
<center>


</center>
</div>

<div class="sidebar"  style="margin-bottom:2px;">
<div class="sidebar_title">网站使用教程</div>
<div class="sidebar_info">
<a href="http://www.guoxuedashi.com/help/gjsearch.php" target="_blank">如何在国学大师网下载古籍?</a><br>
<a href="http://www.guoxuedashi.com/zidian/bujian/bjjc.php" target="_blank">如何使用部件查字法快速查字?</a><br>
<a href="http://www.guoxuedashi.com/search/sjc.php" target="_blank">如何在指定的书籍中全文检索?</a><br>
<a href="http://www.guoxuedashi.com/search/skjc.php" target="_blank">如何找到一句话在《四库全书》哪一页?</a><br>
</div>
</div>


<div class="sidebar">
<div class="sidebar_title">热门书籍</div>
<div class="sidebar_info">
<a href="/so.php?sokey=%E8%B5%84%E6%B2%BB%E9%80%9A%E9%89%B4&kt=1">资治通鉴</a> <a href="/24shi/"><strong>二十四史</strong></a>&nbsp; <a href="/a2694/">野史</a>&nbsp; <a href="/SiKuQuanShu/"><strong>四库全书</strong></a>&nbsp;<a href="http://www.guoxuedashi.com/SiKuQuanShu/fanti/">繁体</a>
<br><a href="/so.php?sokey=%E7%BA%A2%E6%A5%BC%E6%A2%A6&kt=1">红楼梦</a> <a href="/a/1858x/">三国演义</a> <a href="/a/1038k/">水浒传</a> <a href="/a/1046t/">西游记</a> <a href="/a/1914o/">封神演义</a>
<br>
<a href="http://www.guoxuedashi.com/so.php?sokeygx=%E4%B8%87%E6%9C%89%E6%96%87%E5%BA%93&submit=&kt=1">万有文库</a> <a href="/a/780t/">古文观止</a> <a href="/a/1024l/">文心雕龙</a> <a href="/a/1704n/">全唐诗</a> <a href="/a/1705h/">全宋词</a>
<br><a href="http://www.guoxuedashi.com/so.php?sokeygx=%E7%99%BE%E8%A1%B2%E6%9C%AC%E4%BA%8C%E5%8D%81%E5%9B%9B%E5%8F%B2&submit=&kt=1"><strong>百衲本二十四史</strong></a>  <a href="http://www.guoxuedashi.com/so.php?sokeygx=%E5%8F%A4%E4%BB%8A%E5%9B%BE%E4%B9%A6%E9%9B%86%E6%88%90&submit=&kt=1"><strong>古今图书集成</strong></a>
<br>

<a href="http://www.guoxuedashi.com/so.php?sokeygx=%E4%B8%9B%E4%B9%A6%E9%9B%86%E6%88%90&submit=&kt=1">丛书集成</a> 
<a href="http://www.guoxuedashi.com/so.php?sokeygx=%E5%9B%9B%E9%83%A8%E4%B8%9B%E5%88%8A&submit=&kt=1"><strong>四部丛刊</strong></a>  
<a href="http://www.guoxuedashi.com/so.php?sokeygx=%E8%AF%B4%E6%96%87%E8%A7%A3%E5%AD%97&submit=&kt=1">說文解字</a> <a href="http://www.guoxuedashi.com/so.php?sokeygx=%E5%85%A8%E4%B8%8A%E5%8F%A4&submit=&kt=1">三国六朝文</a>
<br><a href="http://www.guoxuedashi.com/so.php?sokeytm=%E6%97%A5%E6%9C%AC%E5%86%85%E9%98%81%E6%96%87%E5%BA%93&submit=&kt=1"><strong>日本内阁文库</strong></a> <a href="http://www.guoxuedashi.com/so.php?sokeytm=%E5%9B%BD%E5%9B%BE%E6%96%B9%E5%BF%97%E5%90%88%E9%9B%86&ka=100&submit=">国图方志合集</a> <a href="http://www.guoxuedashi.com/so.php?sokeytm=%E5%90%84%E5%9C%B0%E6%96%B9%E5%BF%97&submit=&kt=1"><strong>各地方志</strong></a>

</div>
</div>


<div class="sidebar2">
<center>

</center>
</div>
<div class="sidebar greenbar">
<div class="sidebar_title green">四库全书</div>
<div class="sidebar_info">

《四库全书》是中国古代最大的丛书,编撰于乾隆年间,由纪昀等360多位高官、学者编撰,3800多人抄写,费时十三年编成。丛书分经、史、子、集四部,故名四库。共有3500多种书,7.9万卷,3.6万册,约8亿字,基本上囊括了古代所有图书,故称“全书”。<a href="http://www.guoxuedashi.com/SiKuQuanShu/">详细>>
</a>

</div> 
</div>

</div>  <!--end r-->

</div>
<!-- 内容区END --> 

<!-- 页脚开始 -->
<div class="shh">

</div>

<div class="w1180" style="margin-top:8px;">
<center><script src="http://www.guoxuedashi.com/img/plus.php?id=3"></script></center>
</div>
<div class="w1180 foot">
<a href="/b/thanks.php">特别致谢</a> | <a href="javascript:window.external.AddFavorite(document.location.href,document.title);">收藏本站</a> | <a href="#">欢迎投稿</a> | <a href="http://www.guoxuedashi.com/forum/">意见建议</a> | <a href="http://www.guoxuemi.com/">国学迷</a> | <a href="http://www.shuowen.net/">说文网</a><script language="javascript" type="text/javascript" src="https://js.users.51.la/17753172.js"></script><br />
  Copyright &copy; 国学大师 古典图书集成 All Rights Reserved.<br>
  
  <span style="font-size:14px">免责声明:本站非营利性站点,以方便网友为主,仅供学习研究。<br>内容由热心网友提供和网上收集,不保留版权。若侵犯了您的权益,来信即刪。scp168@qq.com</span>
  <br />
ICP证:<a href="http://www.beian.miit.gov.cn/" target="_blank">鲁ICP备19060063号</a></div>
<!-- 页脚END --> 
<script src="http://www.guoxuedashi.com/img/plus.php?id=22"></script>
<script src="http://www.guoxuedashi.com/img/tongji.js"></script>

</body>
</html>
