<!DOCTYPE html PUBLIC "-//W3C//DTD XHTML 1.0 Transitional//EN" "http://www.w3.org/TR/xhtml1/DTD/xhtml1-transitional.dtd">
<html xmlns="http://www.w3.org/1999/xhtml">
<head>
<meta http-equiv="Content-Type" content="text/html; charset=utf-8" />
<meta http-equiv="X-UA-Compatible" content="IE=Edge,chrome=1">
<title>資治通鑒_195-資治通鑑卷一百九十四_195-資治通鑑卷一百九十四</title>
<meta name="Keywords" content="資治通鑒_195-資治通鑑卷一百九十四_195-資治通鑑卷一百九十四">
<meta name="Description" content="資治通鑒_195-資治通鑑卷一百九十四_195-資治通鑑卷一百九十四">
<meta http-equiv="Cache-Control" content="no-transform" />
<meta http-equiv="Cache-Control" content="no-siteapp" />
<link href="/img/style.css" rel="stylesheet" type="text/css" />
<script src="/img/m.js?2020"></script> 
</head>
<body>
 <div class="ClassNavi">
<a  href="/24shi/">二十四史</a> | <a href="/SiKuQuanShu/">四库全书</a> | <a href="http://www.guoxuedashi.com/gjtsjc/"><font  color="#FF0000">古今图书集成</font></a> | <a href="/renwu/">历史人物</a> | <a href="/ShuoWenJieZi/"><font  color="#FF0000">说文解字</a></font> | <a href="/chengyu/">成语词典</a> | <a  target="_blank"  href="http://www.guoxuedashi.com/jgwhj/"><font  color="#FF0000">甲骨文合集</font></a> | <a href="/yzjwjc/"><font  color="#FF0000">殷周金文集成</font></a> | <a href="/xiangxingzi/"><font color="#0000FF">象形字典</font></a> | <a href="/13jing/"><font  color="#FF0000">十三经索引</font></a> | <a href="/zixing/"><font  color="#FF0000">字体转换器</font></a> | <a href="/zidian/xz/"><font color="#0000FF">篆书识别</font></a> | <a href="/jinfanyi/">近义反义词</a> | <a href="/duilian/">对联大全</a> | <a href="/jiapu/"><font  color="#0000FF">家谱族谱查询</font></a> | <a href="http://www.guoxuemi.com/hafo/" target="_blank" ><font color="#FF0000">哈佛古籍</font></a> 
</div>

 <!-- 头部导航开始 -->
<div class="w1180 head clearfix">
  <div class="head_logo l"><a title="国学大师官网" href="http://www.guoxuedashi.com" target="_blank"></a></div>
  <div class="head_sr l">
  <div id="head1">
  
  <a href="http://www.guoxuedashi.com/zidian/bujian/" target="_blank" ><img src="http://www.guoxuedashi.com/img/top1.gif" width="88" height="60" border="0" title="部件查字,支持20万汉字"></a>


<a href="http://www.guoxuedashi.com/help/yingpan.php" target="_blank"><img src="http://www.guoxuedashi.com/img/top230.gif" width="600" height="62" border="0" ></a>


  </div>
  <div id="head3"><a href="javascript:" onClick="javascript:window.external.AddFavorite(window.location.href,document.title);">添加收藏</a>
  <br><a href="/help/setie.php">搜索引擎</a>
  <br><a href="/help/zanzhu.php">赞助本站</a></div>
  <div id="head2">
 <a href="http://www.guoxuemi.com/" target="_blank"><img src="http://www.guoxuedashi.com/img/guoxuemi.gif" width="95" height="62" border="0" style="margin-left:2px;" title="国学迷"></a>
  

  </div>
</div>
  <div class="clear"></div>
  <div class="head_nav">
  <p><a href="/">首页</a> | <a href="/ShuKu/">国学书库</a> | <a href="/guji/">影印古籍</a> | <a href="/shici/">诗词宝典</a> | <a   href="/SiKuQuanShu/gxjx.php">精选</a> <b>|</b> <a href="/zidian/">汉语字典</a> | <a href="/hydcd/">汉语词典</a> | <a href="http://www.guoxuedashi.com/zidian/bujian/"><font  color="#CC0066">部件查字</font></a> | <a href="http://www.sfds.cn/"><font  color="#CC0066">书法大师</font></a> | <a href="/jgwhj/">甲骨文</a> <b>|</b> <a href="/b/4/"><font  color="#CC0066">解密</font></a> | <a href="/renwu/">历史人物</a> | <a href="/diangu/">历史典故</a> | <a href="/xingshi/">姓氏</a> | <a href="/minzu/">民族</a> <b>|</b> <a href="/mz/"><font  color="#CC0066">世界名著</font></a> | <a href="/download/">软件下载</a>
</p>
<p><a href="/b/"><font  color="#CC0066">历史</font></a> | <a href="http://skqs.guoxuedashi.com/" target="_blank">四库全书</a> |  <a href="http://www.guoxuedashi.com/search/" target="_blank"><font  color="#CC0066">全文检索</font></a> | <a href="http://www.guoxuedashi.com/shumu/">古籍书目</a> | <a   href="/24shi/">正史</a> <b>|</b> <a href="/chengyu/">成语词典</a> | <a href="/kangxi/" title="康熙字典">康熙字典</a> | <a href="/ShuoWenJieZi/">说文解字</a> | <a href="/zixing/yanbian/">字形演变</a> | <a href="/yzjwjc/">金 文</a> <b>|</b>  <a href="/shijian/nian-hao/">年号</a> | <a href="/diming/">历史地名</a> | <a href="/shijian/">历史事件</a> | <a href="/guanzhi/">官职</a> | <a href="/lishi/">知识</a> <b>|</b> <a href="/zhongyi/">中医中药</a> | <a href="http://www.guoxuedashi.com/forum/">留言反馈</a>
</p>
  </div>
</div>
<!-- 头部导航END --> 
<!-- 内容区开始 --> 
<div class="w1180 clearfix">
  <div class="info l">
   
<div class="clearfix" style="background:#f5faff;">
<script src='http://www.guoxuedashi.com/img/headersou.js'></script>

</div>
  <div class="info_tree"><a href="http://www.guoxuedashi.com">首页</a> > <a href="/SiKuQuanShu/fanti/">四库全书</a>
 > <h1>资治通鉴</h1> <!--         下载:【右键另存为】即可 --></div>
  <div class="info_content zj clearfix">
  
<div class="info_txt clearfix" id="show">
<center style="font-size:24px;">195-資治通鑑卷一百九十四</center>
    資治通鑑卷一百九十四 宋 司馬光 撰<br />
<br />
  胡三省 音注<br />
<br />
  唐紀十【起玄黓執徐盡彊圉作噩四月凡五年有奇】<br />
<br />
  太宗文武大聖大廣孝皇帝上之下<br />
<br />
  貞觀六年【觀古玩翻】春正月乙卯朔日有食之 癸酉靜州獠反將軍李子和討平之【獠魯皓翻】 文武官復請封禪【復扶又翻去年諸州朝集使請封禪】上曰卿輩皆以封禪為帝王盛事朕意不然若天下乂安家給人足雖不封禪庸何傷乎昔秦始皇封禪【見七卷始皇二十八年】而漢文帝不封禪後世豈以文帝之賢不及始皇邪【邪音耶】且事天掃地而祭【禮記郊特牲曰郊之祭也大報天也兆於南郊就陽位也掃地而祭於其質也】何必登泰山之巔封數尺之土然後可以展其誠敬乎羣臣猶請之不已上亦欲從之魏徵獨以為不可 【考異曰實錄唐書志及唐統紀皆以為太宗自不欲封禪而魏文貞公故事及王方慶文貞公傳錄以為太宗欲封太山徵諫而止意頗不同今兩存之】上曰公不欲朕封禪者以功未高邪曰高矣德未厚邪曰厚矣中國未安邪曰安矣四夷未服邪曰服矣年穀未豐邪曰豐矣符瑞未至邪曰至矣然則何為不可封禪對曰陛下雖有此六者然承隋末大亂之後戶口未復倉廩尚虚而車駕東巡千乘萬騎【乘繩證翻騎奇寄翻】其供頓勞費未易任也【易以豉翻任音壬】且陛下封禪則萬國咸集遠夷君長皆當扈從【長知兩翻從才用翻】今自伊洛以東至於海岱煙火尚希灌莽極目【灌木叢生也莽草深茂也】此乃引戎狄入腹中示之以虚弱也况賞賚不貲未厭遠人之望給復連年不償百姓之勞【厭於協翻復方目翻】崇虚名而受實害陛下將焉用之【焉於䖍翻】會河南北數州大水事遂寢 上將幸九成宫通直散騎常侍姚思亷諫【散悉亶翻騎奇寄翻】上曰朕有氣疾暑輒頓劇往避之耳賜思廉絹五十匹監察御史馬周上疏【監古銜翻上時掌翻】以為東宫在宫城之中而大安宫乃在宫城之西【此因大安宫在西遂謂帝所居為東宫耳】制度比於宸居尚為卑小於四方觀聽有所不足宜增修高大以稱中外之望【稱尺證翻】又太上皇春秋已高陛下宜朝夕視膳今九成宫去京師三百餘里太上皇或時思念陛下陛下何以赴之又車駕此行欲以避暑太上皇尚留暑中而陛下獨居涼處温凊之禮竊所未安【記曲禮凡為人子之禮冬温而夏凊凊音七正翻】今行計已成不可復止【復扶又翻】願速示返期以解衆惑又王長通白明達皆樂工韋槃提斛斯正止能調馬縱使技能出衆止可賚之金帛豈得超授官爵鳴玉曳履與士君子比肩而立同坐而食【技渠綺翻坐徂卧翻】臣竊恥之上深納之上以新令無三師官二月丙戌詔特置之【唐以太師太傅太保為三師正一品天子所師法無所總職】 三月戊辰上幸九成宮 庚午吐谷渾寇蘭州【吐從暾入聲谷音浴】州兵擊走之 長樂公主將出降【唐會要長樂公主下嫁長孫沖樂音洛】上以公主皇后所生特愛之勑有司資送倍於永嘉長公主【永嘉長公主高祖女下嫁竇奉節又嫁賀蘭僧伽唐制皇姑為大長公主正一品姊為長公主女為公主皆視一品長知兩翻下同】魏徵諫曰昔漢明帝欲封皇子曰我子豈得與先帝子比皆令半楚淮陽【事見四十五卷漢明帝永平十五年】今資送公主倍於長主得無異於明帝之意乎上然其言入告皇后后歎曰妾亟聞陛下稱重魏徵【亟去吏翻】不知其故今觀其引禮義以抑人主之情乃知真社稷之臣也妾與陛下結髮為夫婦曲承恩禮每言必先候顔色不敢輕犯威嚴况以人臣之疏遠乃能抗言如是陛下不可不從自請遣中使齎錢四百緍絹四百匹以賜徵【使疏吏翻 考異曰舊文德皇后傳云使齎帛五百匹詣徵第賜之魏文貞公故事云遣中使齎錢二十萬絹百匹詣公宅宣命今從舊魏徵傳】且語之曰【語牛倨翻】開公正直乃今見之故以相賞公宜常秉此心勿轉移也上嘗罷朝怒曰會須殺此田舍翁【朝直遥翻】后問為誰上曰魏徵每廷辱我后退具朝服立於庭【唐制皇后之服褘衣者受册助祭朝會大事之服也深青織成為之畫翬赤質五色十二等素紗中單黼領朱羅縠褾襈蔽膝隨裳色以緅領為緣用翟為章三等青衣革帶大帶隨衣色裨紐約佩綬如天子青襪舄加金飾首飾大小華十二樹以象衮冕之旒又有兩博鬢朝直遥翻褾彼小翻袖耑襈皺戀翻緣也緅仄鳩翻】上驚問其故后曰妾聞主明臣直今魏徵直由陛下之明故也妾敢不賀上乃悅 夏四月辛卯襄州都督鄒襄公張公謹卒【卒子恤翻】明日上出次發哀有司奏辰日忌哭【彭祖百忌辰不哭泣】上曰君之於臣猶父子也情發於衷安避辰日遂哭之 六月己亥金州刺史酆悼王元亨薨【金州西城郡梁置南梁州西魏置東梁州尋改曰金州】辛亥江王囂薨 秋七月丙辰焉耆王突騎支遣使入貢初焉耆入中國由磧路隋末閉塞道由高昌突騎支請復開磧路以便往來【騎奇寄翻使疏吏翻磧七迹翻塞悉則翻復扶又翻又音如字】上許之由是高昌恨之遣兵襲焉耆大掠而去【焉耆國東鄰高昌為討高昌張本】 辛未宴三品已上於丹霄殿上從容言曰【從千容翻】中外乂安皆公卿之力然隋煬帝威加夷夏【夏戶雅翻】頡利跨有北荒【頡奚結翻】統葉護雄據西域今皆覆亡此乃朕與公等所親見勿矜彊盛以自滿也 西突厥肆葉護可汗發兵擊薛延陁為薛延陁所敗【厥九勿翻可從刋入聲汗音寒敗補邁翻】肆葉護性猜狠信讒有乙利可汗功最多【乙利西突厥小可汗也狠戶墾翻】肆葉護以非其族類誅滅之由是諸部皆不自保肆葉護又忌莫賀設之子泥孰陰欲圖之泥孰奔焉耆設卑逵官與弩失畢二部攻之【舊傳作設卑逵官新傳作没卑逵干 考異曰今從舊傳】肆葉護輕騎奔康居尋卒【肆葉護立見上卷三年騎奇寄翻卒子恤翻】國人迎泥孰於焉耆而立之是為咄陸可汗遣使内附【咄常没翻可從刋入聲汗音寒使疏吏翻】丁酉遣鴻臚少卿劉善因立咄陸為奚利邲咄陸可汗【臚陵如翻少始照翻邲毗必翻咄陸即阿史那彌射此當參觀高宗顯慶二年考異而詳辯之考異曰舊傳册為吞阿妻狀奚利邲咄陸可汗新傳册號吞阿婁拔利邲咄陸可汗今從實錄】 閏月乙卯上宴近臣於丹霄殿長孫無忌曰王珪魏徵昔為仇讎【謂其事隱太子勸之圖帝也】不謂今日得此同宴上曰徵珪盡心所事故我用之然徵每諫我不從我與之言輒不應何也魏徵對曰臣以事為不可故諫陛下不從而臣應之則事遂施行故不敢應上曰且應而復諫庸何傷【復扶又翻】對曰昔舜戒羣臣爾無面從退有後言【書益稷之言】臣心知其非而口應陛下乃面從也豈稷契事舜之意邪【契息列翻】上大笑曰人言魏徵舉止疏慢我視之更覺娬媚【娬罔甫翻娬亦媚也】正為此耳【為於偽翻】徵起拜謝曰陛下開臣使言故臣得盡其愚若陛下拒而不受臣何敢數犯顔色乎【數所角翻】 戊辰秘書少監虞世南上聖德論【上時掌翻】上賜手詔稱卿論太高朕何敢擬上古但比近世差勝耳然卿適覩其始未知其終若朕能慎終如始則此論可傳如或不然恐徒使後世笑卿也九月己酉幸慶善宫上生時故宅也【以高祖武功舊第為慶善宫】<br />
<br />
  因與貴人宴賦詩起居郎清平呂才【清平縣屬博州劉昫曰本漢貝丘縣隋曰清平】被之管絃【被皮義翻】命曰功成慶善樂使童子八佾為九功之舞大宴會與破陳舞偕奏於庭【才有巧思故命以所賦詩被之管絃以為樂章以童子六十四人冠進德冠紫袴褶長袖漆髻屣履而舞號九功舞進蹈安徐以象文德破陳樂號七德舞擊刺往來發揚蹈厲以象武功陳讀曰陣】同州刺史尉遲敬德預宴【同州馮翊郡尉紆勿翻】有班在其上者敬德怒曰汝何功坐我上任城王道宗次其下諭解之敬德拳毆道宗目幾眇【任音壬毆烏口翻幾居希翻 考異曰唐歷云嘗因内宴於御前毆宇文士及曰汝有何功合居吾上太宗慰諭之方止今從舊傳】上不懌而罷謂敬德曰朕見漢高祖誅滅功臣意常尤之故欲與卿等共保富貴令子孫不絶【令力丁翻】然卿居官數犯法乃知韓彭葅醢非高祖之罪也國家綱紀唯賞與罰非分之恩不可數得【分扶問翻數所角翻】勉自修飾無貽後悔敬德由是始懼而自戢【戢阻立翻】冬十月乙卯車駕還京師帝侍上皇宴於大安宫帝與皇后更獻飲膳及服御之物【更工衡翻】夜久乃罷帝親為上皇捧輿至殿門【為于偽翻】上皇不許命太子代之 突厥頡利可汗鬱鬱不得意數與家人相對悲泣容貌羸憊【厥九勿翻頡奚結翻可從刋入聲汗音寒數所角翻羸倫為翻憊蒲拜翻】上見而憐之以虢州地多麋鹿【義寜元年分弘農二縣置虢州虢郡宋白曰帝王世紀云虢有三周封虢仲於西虢虢州之地也封虢叔於東虢今成臯也陜郡平陸是北虢】可以游獵乃以頡利為虢州刺史頡利辭不願往癸未復以為右衛大將軍【復扶又翻下勿復不復同又音如字】十一月辛巳契苾酋長何力帥部落六千餘家詣沙州降詔處之於甘涼之間【契欺結翻苾毗必翻酋慈由翻長知兩翻帥讀曰率降戶江翻處昌呂翻甘涼相去五百里】以何力為左領軍將軍 庚寅以左光祿大夫陳叔達為禮部尚書帝謂叔達曰卿武德中有讜言【見一百九十一卷高祖武德九年讜音黨善言直言也】故以此官相報對曰臣見隋室父子相殘以取亂亡當日之言非為陛下【為于偽翻】乃社稷之計耳 十二月癸丑帝與侍臣論安危之本中書令温彦博曰伏願陛下常如貞觀初則善矣帝曰朕比來怠於為政乎【觀古玩翻比毗至翻】魏徵曰貞觀之初陛下志在節儉求諫不倦比來營繕微多諫者頗有忤旨此其所以異耳【比毗至翻忤五故翻】帝拊掌大笑曰誠有是事 辛未帝親錄繫囚見應死者憫之縱使歸家期以來秋來就死仍敕天下死囚皆縱遣使至期來詣京師 是歲党項羌前後内屬者三十萬口【党底朗翻】 公卿以下請封禪者前後相屬【屬之欲翻】上諭以舊有氣疾恐登高增劇公等勿復言【復扶又翻】 上謂侍臣曰朕比來決事或不能皆如律令公輩以為事小不復執奏夫事無不由小而致大此乃危亡之端也【比毗至翻夫音扶】昔關龍逢忠諫而死【逢皮江翻】朕每痛之煬帝驕暴而亡公輩所親見也公輩常宜為朕思煬帝之亡朕常為公輩念關龍逢之死【為于偽翻】何患君臣不相保乎 上謂魏徵曰為官擇人不可造次【朱元晦曰造次急遽苟且之時造七到翻】用一君子則君子皆至用一小人則小人競進矣對曰然天下未定則專取其才不考其行喪亂既平【行下孟翻下同喪悤浪翻】則非才行兼備不可用也【觀此則天下已定之後可不為官擇人乎】<br />
<br />
  七年春正月更名破陳樂曰七德舞【更工衡翻左傳楚莊王曰武有七德禁暴戢兵保大定功安民和衆豐財故以為樂舞之名新志七德舞圖左圓右方先偏後伍交錯屈伸以象魚麗鵝鸛命呂才以圖教樂工百二十八人被銀甲執戟而舞凡三變每變為四陣象擊刺往來歌者和曰秦王破陳樂杜佑曰破陳樂舞圖左圓右方先偏後伍魚麗鵝鸛箕張翼舒交錯屈伸首尾囘互以象戰陳之形凡為三變每變有四陣有來往疾徐擊刺之象以應歌節發揚蹈厲聲韻慷慨陳讀曰陣】癸巳宴三品已上及州牧蠻夷酋長於玄武門奏七德九功之舞【酋慈由翻長知兩翻】太常卿蕭瑀上言七德舞形容聖功有所未盡【瑀音禹上時掌翻】請寫劉武周薛仁杲竇建德王世充等擒獲之狀上曰彼皆一時英雄今朝廷之臣往往嘗北面事之若覩其故主屈辱之狀能不傷其心乎瑀謝曰此非臣愚慮所及魏徵欲上偃武修文每侍宴見七德舞輒俛首不視見九功舞則諦觀之【俛音免諦都計翻審也】 三月戊子侍中王珪坐漏泄禁中語左遷同州刺史庚寅以祕書監魏徵為侍中 直太史雍人李淳風【雍縣屬岐州雍於用翻】奏靈臺候儀制度疏畧但有赤道請更造渾天黄道儀【更工衡翻渾戶本翻】許之癸巳成而奏之【時李淳風上言舜在璿璣玉衡以齊七政則渾天儀也周禮土圭正日景以求地中有以見日行黄道之驗也暨於周末此器乃亡漢洛下閎作渾儀其後賈逵張衡亦有之而推驗七曜並循赤道按冬至極南夏至極北而赤道常定於中國無南北之異蓋渾儀無黄道久矣上異其說因詔為之儀表裏三重下據準基上如十字末樹鼇足以張四表一曰六合儀有天經雙規金渾緯規金常規相結於四極之内列二十八宿十日十二辰經緯三百六十五度二曰三辰儀圓徑八尺有璿璣規日游規列宿所行轉於六合之内三曰四游儀圓樞為軸以連結玉衡游筩而貫約矩規又玄極北樹北辰南矩地軸傍轉於内玉衡在玄樞之間而南北游仰以觀天之辰宿下以識器之晷度皆用銅為之】夏五月癸未上幸九成宫 雅州道行軍總管張士<br />
<br />
  貴擊反獠破之【雅州漢嚴道縣地隋廢州置臨邛郡唐復為雅州獠魯皓翻】 秋八月乙丑左屯衛大將軍譙敬公周範卒上行幸常令範與房玄齡居守【卒子恤翻守式又翻】範為人忠篤嚴正疾甚不肯出外竟終於内省與玄齡相抱而訣曰所恨不獲再奉聖顔 辛未以張士貴為龔州道行軍總管使擊反獠【龔州臨江郡漢猛陵縣地隋為永平郡武林縣貞觀三年置鷰州於今州東仍於鷰州之故所置龔州】九月山東河南四十餘州水遣使賑之【使疏吏翻賑津忍翻】 去歲所縱天下死囚凡三百九十人無人督帥皆如期自詣朝堂【帥讀曰率朝直遥翻 考異曰四年實錄云天下斷死罪止二十九人今年實錄乃有二百九十九人何頓多如此事已可疑又白居易樂府云死囚四百來歸獄舊本紀統紀年代記皆云二百九十人今從新書刑法志】無一人亡匿者上皆赦之 冬十月庚申上還京師 十一月壬辰以開府儀同三司長孫無忌為司空【長知兩翻】無忌固辭曰臣忝預外戚恐天下謂陛下為私上不許曰吾為官擇人惟才是與【為于偽翻】苟或不才雖親不用襄邑王神符是也【神符少威嚴不為下所畏又足不良於行由是歸第】如其有才雖讎不棄魏徵等是也今日所舉非私親也 十二月甲寅上幸芙蓉園【景龍文舘記芙蓉園在京師羅城東南隅本隋世之離宫也青林重複綠水瀰漫帝城勝景也】丙辰校獵少陵原【少陵原在長安城南屬萬年縣界少始照翻】戊午還宫從上皇置酒故漢未央宫【漢故未央宫在長安宫城北禁苑之西偏 考異曰舊高祖紀八年閱武於城西高祖親自臨視還置酒於未央宫高祖實錄不記年月據太宗實錄八年正月頡利可汗死今從唐歷】上皇命突厥頡利可汗起舞又命南蠻酋長馮智戴詠詩【厥九勿翻頡奚結翻可從刋入聲汗音寒酋慈由翻長知兩翻】既而笑曰胡越一家自古未有也帝奉觴上夀【上時掌翻】曰今四夷入臣皆陛下教誨非臣智力所及昔漢高祖亦從太上皇置酒此宫妄自矜大【漢高祖十年置酒未央宫奉玉巵為太上皇夀曰始大人常以臣亡賴不能治產業不如仲力今某之業所就孰與仲多】臣所不取也上皇大悅殿上皆呼萬歲 帝謂左庶子于志寧右庶子杜正倫曰朕年十八猶在民間民之疾苦情偽無不知之及居大位區處世務猶有差失况太子生長深宫【處昌呂翻長知兩翻】百姓艱難耳目所未涉能無驕逸乎卿等不可不極諫太子好嬉戲頗虧禮法志寧與右庶子孔頴達數直諫【好呼到翻數所角翻】上聞而嘉之各賜金一斤帛五百匹工部尚書段綸奏徵巧工楊思齊上令試之綸使先造傀儡【傀儡木偶戲也杜佑曰窟子亦曰傀磊子作偶人以戲善歌舞本喪樂也漢末始用之於嘉會北齊高緯尤所好閭市盛行焉余按列子偃師以此伎奉周穆王其來久矣傀口猥翻儡落猥翻】上曰得巧工庶供國事卿令先造戲具豈百工相戒無作淫巧之意邪【月令孟春之月百工咸理監工日號毋或作為淫巧以蕩上心邪音耶】乃削綸階【唐制工部尚書正三品削其階則不得立於三品班中】 嘉陵州獠反【嘉州眉山郡漢犍為南安縣地陵州陵山郡漢蜀郡廣都犍為郡武陽二縣東界之地獠魯皓翻】命邗江府統軍牛進達擊破之【唐揚州有䢴江府兵䢴胡安翻】 上問魏徵曰羣臣上書可采及召對多失次何也【臣上時掌翻】對曰臣觀百司奏常事數日思之及至上前三分不能道一况諫者拂意觸忌【拂與咈同】非陛下借之辭色豈敢盡其情哉上由是接羣臣辭色愈温嘗曰煬帝多猜忌臨朝對羣臣多不語【朝直遥翻】朕則不然與羣臣相親如一體耳<br />
<br />
  八年春正月癸未突厥頡利可汗卒【厥九勿翻頡奚結翻可從刋入聲汗音寒卒子恤翻】命國人從其俗焚尸葬之 辛丑行軍總管張士貴討東西王洞反獠平之【東西王洞獠蓋在龔州界】 上欲分遣大臣為諸道黜陟大使【使疏吏翻 考異曰實錄舊本紀但云遣蕭瑀等巡省天下按時止有十道而會要統紀皆云發十六道黜陟大使據姓名止有十三人皆所未詳故但云諸道】未得其人李靖薦魏徵上曰徵箴規朕失不可一日離左右【離力智翻】乃命靖與太常卿蕭瑀等凡十三人分行天下察長吏賢不肖【行下孟翻長知兩翻】問民間疾苦禮高年賑窮乏【賑津忍翻】起淹滯俾使者所至如朕親覩 三月庚辰上幸九成宫 夏五月辛未朔日有食之 初吐谷渾可汗伏允【吐從暾入聲谷音浴可從刋入聲汗音寒 考異曰實錄十年立諾曷鉢詔稱伏允為順步薩鉢今從舊傳】遣使入貢未返大掠鄯州而去【使疏吏翻宋白曰鄯州西南至廓州廣城縣故承風嶺吐谷渾界百九十五里】上遣使讓之徵伏允入朝稱疾不至【鄯時戰翻朝直遥翻】仍為其子尊王求婚上許之令其親迎【為于偽翻迎魚敬翻】尊王又不至乃絶婚伏允又遣兵寇蘭廓二州【蘭州金城郡以臯蘭山名州】伏允年老信其臣天柱王之謀數犯邊【數所角翻】又執唐使者趙德楷上遣使諭之十返又引其使者臨軒親諭以禍福伏允終無悛心【悛丑緣翻】六月遣左驍衛大將軍段志玄為西海道行軍總管左驍衛將軍樊興為赤水道行軍總管將邊兵及契苾党項之衆以擊之【吐谷渾中有赤水城近河源驍堅堯翻將邊即亮翻契欺訖翻苾毗必翻党底朗翻 考異曰實錄六年三月吐谷渾寇蘭州不云遣志玄擊之吐谷渾寇蘭廓二州無年月新本紀此夏遣志玄實錄十月志玄破吐谷渾故參酌置此又新書本紀是夏吐谷渾寇涼州遣志玄等伐之實錄十月辛丑志玄破吐谷渾而不書遣將日月新紀亦無破吐谷渾月日實錄寇涼州在十一月今參用之】 秋七月山東河南淮海之間大水 上屢請上皇避暑九成宫上皇以隋文帝終於彼惡之【九成宫即隋之仁夀宫隋文帝仁夀四年崩於仁夀宫惡烏路翻】冬十月營大明宫【大明宫在禁苑東南西接宫城之東北隅曰東内程大昌曰大明宫地本太極宫之後苑東南面射殿也地在龍首山上龍朔二年高宗染風痺惡太極宫卑下就修大明宫改曰蓬萊宫】以為上皇清暑之所未成而上皇寢疾不果居 辛丑段志玄擊吐谷渾破之追奔八百餘里去青海三十餘里【吐谷渾中有青海闞駰曰漢金城郡臨羌縣西有卑禾羌海世謂之青海東去西平二百五十里西平唐鄯州也吐從暾入聲谷音浴】吐谷渾驅牧馬而遁 甲子上還京師右僕射李靖以疾遜位許之十一月辛未以靖為特進封爵如故祿賜吏卒並依舊給俟疾小瘳【瘳丑留翻】每三兩日至門下中書平章政事【唐初政事堂在門下省歐陽修曰平章事之名始此】甲申吐蕃贊普棄宗弄讚 【考異曰太宗實錄贊普作贊府高宗實錄棄宗作器宗今從舊傳】遣使入貢仍請昏【使疏吏翻】吐蕃在吐谷渾西南近世浸彊蠶食他國土宇廣大勝兵數十萬【勝音升】然未嘗通中國其王稱贊普俗不言姓王族皆曰論宦族皆曰尚【吐蕃本西羌屬蓋百有五十種散處河湟江岷間有發羌唐旄等未始與中國通居析支水西祖曰鶻提勃悉野健武多智稍并諸羌據其地蕃發聲近故其子孫曰吐蕃而姓勃窣野或曰南涼禿髮烏孤之後二子曰樊尼曰傉檀為乞伏熾盤所滅樊尼挈殘部降沮渠蒙遜沮渠滅樊尼率兵西濟河踰積石遂撫有羣羌云其俗謂彊雄曰贊丈夫曰普故號君長為贊普其地直長安八千里距鄯善五百里劉昫曰吐蕃禿髮氏之後語訛曰吐蕃宋白曰樊尼奔沮渠蒙遜署臨松郡丞沮渠滅建國西土改姓勃窣野時人謂丞為贊府語訛為贊普吐從暾入聲】棄宗弄讚有勇畧四鄰畏之上遣使者馮德遐往慰撫之 丁亥吐谷渾寇涼州己丑下詔大舉討吐谷渾 【考異曰舊傳云吐谷渾拘趙德楷太宗遣使宣諭十餘返竟無悛心九年詔李靖等討伐太宗實錄己丑吐谷渾拘我行人趙德楷即下此詔十二月遣李靖等今從實錄據舊傳拘德楷在前據實錄先遣使宣諭後拘德楷即下詔伐之今兩存之】上欲得李靖為將為其老重勞之【重難也以其年老難勞之以征伐之事也將即亮翻為于偽翻】靖聞之請行上大悦十二月辛丑以靖為西海道行軍大總管節度諸軍兵部尚書侯君集為積石道刑部尚書任城王道宗為鄯善道涼州都督李大亮為且末道岷州都督李道彦為赤水道利州刺史高甑生為鹽澤道行軍總管【西海鄯善且末皆隋破吐谷渾所置郡名積石山在隋河源郡赤水城亦在河源郡鹽池在西海郡任音壬鄯時戰翻且子餘翻】并突厥契苾之衆擊吐谷渾 帝聘隋通事舍人鄭仁基女為充華【充華舊有之唐六宫之職無此官】詔已行冊使將發【使疏吏翻】魏徵聞其常許嫁士人陸爽遽上表諫【上時掌翻】帝聞之大驚手詔深自克責命停册使房玄齡等奏稱許嫁陸氏無顯狀大禮既行不可中止爽亦表言初無婚姻之議帝謂徵曰羣臣或容希合爽亦自陳何也對曰彼以為陛下外雖捨之或陰加罪譴故不得不然帝笑曰外人意或當如是朕之言未能使人必信如此邪 中牟丞皇甫德參【中牟縣漢屬河南郡晉屬滎陽郡後魏屬廣武郡為治所隋開皇十年改曰郟城縣大業改曰圃田縣唐武德三年更名中牟丞貳令治縣事上縣丞從八品下中下縣各以差降一品】上言修洛陽宫勞人收地租厚斂俗好高髻蓋宫中所化【上時掌翻下上書同斂力贍翻好呼到翻下不好同】上怒謂房玄齡等曰德參欲國家不役一人不收斗租宫人皆無髮乃可其意邪欲治其謗訕之罪【治直之翻】魏徵諫曰賈誼當漢文帝時上書云可為痛哭者一可為流涕者二【見十四卷漢文帝六年】自古上書不激切不能動人主之心所謂狂夫之言聖人擇焉【漢書李左車有是言】唯陛下裁察上曰朕罪斯人則誰敢復言【復扶又翻】乃賜絹二十匹他日徵奏言陛下近日不好直言雖勉強含容非曩時之豁如【強其兩翻】上乃更加優賜拜監察御史【監古銜翻】 中書舍人高季輔上言 【考異曰貞觀政要季輔疏在三年會要在八年按舊傳季輔貞觀初拜御史累轉中書舍人故從會要置此】外官卑品猶未得祿飢寒切身難保清白今倉廩浸實宜量加優給然後可責以不貪嚴設科禁又密王元曉等皆陛下之弟比見帝子拜諸叔【量音良比毗至翻】叔皆答拜紊亂昭穆【紊音問昭時招翻】宜訓之以禮書奏上善之 西突厥咄陸可汗卒其弟同娥設立是為沙鉢羅咥利失可汗【咥徒結翻又丑栗翻】<br />
<br />
  九年春正月党項先内屬者皆叛歸吐谷渾三月庚辰洮州羌叛入吐谷渾殺刺史孔長秀【洮土刀翻】 壬辰赦天下 乙酉鹽澤道行軍總管高甑生擊叛羌破之 庚寅詔民貲分三等未盡其詳宜分九等【唐會要武德六年三月令天下戶量其資產定為三等今分九等蓋於三等各分上中下也】 上謂魏徵曰齊後主周天元皆重斂百姓厚自奉養力竭而亡譬如饞人自噉其肉肉盡而斃何其愚也【斂力贍翻饞七咸翻貪食而多取之為饞噉徒覽翻又徒濫翻】然二主孰為優劣對曰齊後主懦弱政出多門【懦乃卧翻又奴亂翻】周天元驕暴威福在己雖同為亡國齊主尤劣也 夏閏四月癸酉任城王道宗敗吐谷渾於庫山【敗補邁翻下兒敗等敗之敗同 考異曰舊道宗傳云賊聞軍至走入嶂山已行數千里諸將議欲息兵道宗固請追討李靖然之而君集不從道宗遂帥偏師并行兼道去大軍十日追及之賊據險苦戰道宗潛遣千餘騎踰山襲其後賊表裏受敵一時奔潰庫山嶂山不知其所以為同異據嶂山已行數千里今不取今即以為庫山之戰也】吐谷渾可汗伏允悉燒野草輕兵走入磧【磧七迹翻】諸將以為馬無草疲瘦未可深入侯君集曰不然曏者段志玄軍還纔及鄯州虜已至其城下蓋虜猶完實衆為之用故也今一敗之後鼠逃鳥散斥候亦絶君臣攜離父子相失取之易於拾芥【易以䜴翻】此而不乘後必悔之李靖從之 【考異曰舊道宗傳云道宗固請追討李靖然之而君集不從靖傳云軍次伏俟城吐谷渾燒去野草以餒我師退保大非川諸將咸言春草未生馬已羸瘦不可赴敵唯靖決計而進深入敵境遂踰積石山按實錄庫山之捷可汗謀將入磧以避官軍道宗復曰柏海近河源古來罕有至者賊既西走未知的處今段之行實資馬力今馬疲糧少遠入為難未若且向鄯州待馬肥之後更圖進取君集曰不然段志玄曩者纔至鄯州賊衆便到城下良由彼國尚完兇徒阻命今者一敗以後斥候亦絶君臣相失父子攜離乘其迫懼取同俯拾柏海雖遥便可鼓行而至也靖又然之道宗傳與實錄相違今從實錄】中分其軍為兩道靖與薛萬均李大亮由北道君集與任城王道宗由南道戊子靖部將薛孤兒敗吐谷渾於曼頭山斬其名王大獲雜畜以充軍食【畜許救翻】癸巳靖等敗吐谷渾於牛心堆【水經注湟水自臨羌縣東流合龍駒川水又東合晉昌川水又東合長寧川水又東合牛心川水水出西南遠山東北流逕牛心堆又東逕西平亭西東北入於湟水】又敗諸赤水源 【考異曰實錄癸巳李靖侯君集任城王道宗等破吐谷渾於赤水源按本文自庫山中分士馬為兩道靖趣北路出曼頭山踰赤水君集道宗趣南路歷破邏真谷然則赤水之戰君集道宗不在彼也今刪去其名又吐谷渾傳獲其高昌王慕容孝儁不知在何戰今亦刪去】侯君集任城王道宗引兵行無人之境二千餘里盛夏降霜經破邏真谷其地無水人齕冰馬噉雪【邏郎佐翻齕下没翻又戶結翻】五月追及伏允於烏海【隋志河源郡有烏海在漢哭山西】與戰大破之獲其名王薛萬均薛萬徹又敗天柱王於赤海【赤海蓋即赤水深廣處 考異曰舊萬徹傳作赤水源契苾何力傳作赤水川今從實錄】 上皇自去秋得風疾庚子崩於垂拱殿【舊書帝紀崩於大安宫之垂拱前殿年七十】甲辰羣臣請上準遺誥視軍國大事上不許乙巳詔太子承乾於東宫平決庶政 赤水之戰薛萬均薛萬徹輕騎先進為吐谷渾所圍兄弟皆中槍【騎奇寄翻下同中竹仲翻】失馬步鬭從騎死者什六七左領軍將軍契苾何力將數百騎救之竭力奮擊所向披靡萬均萬徹由是得免【從才用翻披丕彼翻】李大亮敗吐谷渾於蜀渾山【山在赤海西】獲其名王二十人將軍執失思力敗吐谷渾於居茹川李靖督諸軍經積石山河源至且末窮其西境聞伏允在突倫川 【考異曰吐谷渾傳云伏允西走圖倫磧蓋即突倫川虜語轉耳今從契苾何力傳】將奔于闐契苾何力欲追襲之薛萬均懲其前敗固言不可何力曰虜非有城郭隨水草遷徙若不因其聚居襲取之一朝雲散豈得復傾其巢穴邪【復扶又翻】自選驍騎千餘直趣突倫川萬均乃引兵從之【驍堅堯翻趣七喻翻考異曰吐谷渾傳云萬均率輕銳追奔入磧數百里及其餘黨破之蓋何力先進而萬均從之也】磧中<br />
<br />
  乏水將士刺馬血飲之【刺七亦翻】襲破伏允牙帳斬首數千級獲雜畜二十餘萬伏允脫身走俘其妻子侯君集等進逾星宿川至柏海還與李靖軍合【畜許救翻宿音秀考異曰吐谷渾傳柏海作柏梁今從實錄實錄及吐谷渾傳皆云君集與李靖會於大非川按十道圖大非川在青海南烏海星宿海柏海並在其西且末又在其西極遠據靖已至且末又過烏海星宿川至柏海豈得復會於大非川於事可疑故不敢著其地吐谷渾傳又云兩軍會於大非川至破邏真谷大寧王順乃降按實錄歷破邏真谷又行月餘日乃至星宿川然則破邏真谷在星宿川東甚遠矣豈得返至其處邪今從實錄】大寧王順隋氏之甥伏允之嫡子也為侍中於隋久不得歸伏允立侍子為太子及歸意常怏怏【順歸見一百八十七卷高祖武德二年怏於兩翻】會李靖破其國國人窮蹙怨天柱王順因衆心斬天柱王舉國請降伏允帥千餘騎逃磧中十餘日衆散稍盡為左右所殺【降戶江翻帥讀曰率 考異曰吐谷渾傳云自縊而死今從實錄】國人立順為可汗壬子李靖奏平吐谷渾乙卯詔復其國以慕容順為西平郡王趉故呂烏甘豆可汗【趉渠詘翻又九勿翻杜佑巨屈翻】上慮順未能服其衆仍命李大亮將精兵數千為其聲援 六月己丑羣臣復請聽政上許之其細務仍委太子太子頗能聽斷是後上每出行幸常令居守監國【復扶又翻斷丁亂翻守手又翻監工銜翻】 秋七月庚子鹽澤道行軍副總管劉德敏擊叛羌破之 丁巳詔山陵依漢長陵故事【長陵漢高祖陵也皇甫謐曰長陵東西廣百二十步高十三丈房玄齡云高九丈蓋尺度之長短有古今之異也】務存隆厚期限既促功不能及祕書監虞世南上疏以為聖人薄葬其親非不孝也深思遠慮以厚葬適足為親之累【上時掌翻累力瑞翻】故不為耳昔張釋之有言使其中有可欲雖錮南山猶有隙【見十四卷漢文帝三年】劉向言死者無終極而國家有廢興釋之之言為無窮計也【見三十一卷漢成帝永始元年】其言深切誠合至理伏惟陛下聖德度越唐虞而厚葬其親乃以秦漢為法臣竊為陛下不取雖復不藏金玉【為于偽翻復扶又翻下同】後世但見丘壟如此其大安知無金玉邪且今釋服已依霸陵【用漢文帝遺詔三十六日釋服也】而丘壟之制獨依長陵恐非所宜伏願依白虎通【班固等述白虎通義六卷】為三仞之墳器物制度率皆節損仍刻石立之陵旁别書一通藏之宗廟用為子孫永久之灋疏奏不報世南復上疏以為漢天子即位即營山陵遠者五十餘年今以數月之間為數十年之功恐於人力有所不逮上乃以世南疏授有司令詳處其宜【復扶又翻處昌呂翻】房玄齡等議以為漢長陵高九丈原陵高六丈【原陵漢光武陵也高去聲】今九丈則太崇三仭則太卑請依原陵之制從之 辛亥詔國初草創宗廟之制未備今將遷祔宜令禮官詳議諫議大夫朱子奢請立三昭三穆而虚太祖之位【昭時招翻】於是增修太廟祔弘農府君及高祖并舊神主四為六室【弘農府君諱重耳】房玄齡等議以涼武昭王為始祖【涼王李暠諡武昭】左庶子于志寧議以為武昭王非王業所因不可為始祖上從之 党項寇疊州 李靖之擊吐谷渾也厚賂党項使為鄉導【鄉讀曰嚮】党項酋長拓跋赤辭來謂諸將曰隋人無信喜暴掠我【喜許記翻】今諸軍苟無異心我請供其資糧如或不然我將據險以塞諸軍之道【塞悉則翻】諸將與之盟而遣之赤水道行軍總管李道彦行至闊水【闊水在党項羈縻闊州界】見赤辭無備襲之獲牛羊數千頭於是羣羌怨怒屯野狐峽道彦不得進赤辭擊之道彦大敗死者數萬退保松州左驍衛將軍樊興逗遛失軍期【遛音留】士卒失亡多乙卯道彦興皆坐減死徙邊上遣使勞諸將於大斗拔谷【勞力到翻】薛萬均排毁契苾何力自稱已功何力不勝忿【勝音升】拔刀起欲殺萬均諸將救止之上聞之以讓何力何力具言其狀【具言赤水之戰拔萬均兄弟於圍中及見排毁之狀也】上怒欲解萬均官以授何力何力固辭曰陛下以臣之故解萬均官羣胡無知以陛下為重胡輕漢轉相誣告馳競必多且使胡人謂諸將皆如萬均將有輕漢之心上善之而止尋令宿衛北門檢校屯營事【北門玄武門也按會要貞觀十二年於玄武門置左右屯營以諸衛將軍領之其兵名曰飛騎何力檢校屯營蓋十二年以後事史究言之】尚宗女臨洮縣主【洮土刀翻】 岷州都督鹽澤道行軍總管高甑生後軍期李靖按之甑生恨靖誣告靖謀反按驗無狀八月庚辰甑生坐減死徙邊或言甑生秦府功臣寛其罪上曰甑生違李靖節度又誣其反此而可寛法將安施且國家自起晉陽功臣多矣若甑生獲免則人人犯法安可復禁乎【復扶又翻】我於舊勲未嘗忘也為此不敢赦耳【為于偽翻】李靖自是闔門杜絶賓客雖親戚不得妄見也【以李靖事太宗然猶如此豈非功名之際難居哉】 上欲自詣園陵【園陵謂獻陵】羣臣以上哀毁羸瘠固諫而止【羸倫為翻瘠而尺翻】 冬十月乙亥處月初遣使入貢處月處密皆西突厥之别部也 庚寅葬太武皇帝於獻陵【獻陵在京兆三原縣東之十八里】廟號高祖以穆皇后祔葬【太穆皇后竇氏初葬夀安陵今祔獻陵】加號太穆皇后 十一月庚戌詔議於太原立高祖廟祕書監顔師古議以為寢廟應在京師漢世郡國立廟非禮乃止 戊午以光祿大夫蕭瑀為特進復令參預政事【蕭瑀罷預聞朝政見上卷貞觀四年復扶又翻】上曰武德六年以後高祖有廢立之心而未定我不為兄弟所容實有功高不賞之懼斯人也不可以利誘不可以死脅真社稷臣也【誘音酉】因賜瑀詩曰疾風知勁草板蕩識誠臣又謂瑀曰卿之忠直古人不過然善惡太明亦有時而失瑀再拜謝魏徵曰瑀違衆孤立唯陛下知其忠勁曏不遇聖明求免難矣 特進李靖【唐六典正二品曰特進注曰二漢及魏以為加官從本官服無吏卒品第二位次諸公在開府驃騎上江左皆兼官梁班第十七北齊特進第二品隋特進為正二品散官唐因之】上書請依遺誥御常服臨正殿弗許【上時掌翻】 吐谷渾甘豆可汗久質中國【質音致】國人不附竟為其下所殺子燕王諾曷鉢立諾曷鉢幼大臣爭權國中大亂十二月詔兵部尚書侯君集等將兵援之先遣使者諭解【將即亮翻使疏吏翻】有不奉詔者隨宜討之<br />
<br />
  十年春正月甲午上始親聽政 辛丑以突厥拓設阿史那社爾為左驍衛大將軍【驍堅堯翻】社爾處羅可汗之子也年十一以智略聞可汗以為拓設建牙於磧北與欲谷設分統敕勒諸部居官十年未嘗有所賦斂【斂力贍翻】諸設或鄙其不能為富貴社爾曰部落苟豐於我足矣諸設慙服【突厥謂子弟典兵者為設與社爾同時典兵者非一人故曰諸設】及薛延陁叛攻破欲谷設【事見一百九十二卷元年】社爾兵亦敗將其餘衆走保西陲【將即亮翻】頡利可汗既亡【見上卷四年】西突厥亦亂咄陸可汗兄弟爭國【事見同上】社爾詐往降之引兵襲破西突厥取其地幾半【降戶江翻幾居希翻】有衆十餘萬自稱答布可汗社爾乃謂諸部曰首為亂破我國者薛延陁也我當為先可汗報仇擊滅之【為于偽翻】諸部皆諫曰新得西方宜且留鎮撫今遽捨之遠去西突厥必來取其故地社爾不從擊薛延陁於磧北連兵百餘日會咥利失可汗立【見上八年】社爾之衆苦於人役多棄社爾逃歸【逃歸咥利失】薛延陀縱兵擊之社爾大敗走保高昌其舊兵在者纔萬餘家又畏西突厥之逼遂帥衆來降【帥讀曰率降戶江翻】敕處其部落於靈州之北【處昌呂翻】留社爾於長安尚皇妹南陽長公主【新舊書皆作衡陽長公主陽長如兩翻】典屯兵於苑内 癸丑徙趙王元景為荆王魯王元昌為漢王鄭王元禮為徐王徐王元嘉為韓王荆王元則為彭王滕王元懿為鄭王吳王元軌為霍王豳王元鳳為虢王陳王元慶為道王魏王靈虁為燕王【自此以上皆皇弟也】蜀王恪為吳王越王泰為魏王燕王佑為齊王梁王愔為蜀王郯王惲為蔣王漢王貞為越王申王慎為紀王【自恪以下皇子也燕因肩翻愔於今翻郯音談惲於粉翻】 二月乙丑以元景為荆州都督元昌為梁州都督元禮為徐州都督元嘉為潞州都督元則為遂州都督靈蘷為幽州都督恪為潭州都督泰為相州都督佑為齊州都督愔為益州都督惲為安州都督貞為揚州都督泰不之官以金紫光祿大夫張亮行都督事【唐制凡注官階卑而擬高者則曰守階高而擬卑則曰行今張亮行都督事乃用宋齊諸王典方面置行事之例與注官之行不同】上以泰好文學【好呼到翻】禮接士大夫特命於其府别置文學館聽自引召學士【為泰圖東宫張本】 三月丁酉吐谷渾王諾曷鉢遣使請頒歷行年號遣子弟入侍並從之【使疏吏翻】丁未以諾曷鉢為河源郡王烏地也拔勤豆可汗 癸丑諸王之藩上與之别曰兄弟之情豈不欲常共處邪但以天下之重不得不爾諸子尚可復有兄弟不可復得【處昌呂翻復扶又翻】因流涕嗚咽不能止【上之流涕嗚咽者抑思建成元吉之事乎】 夏六月壬申以温彦博為右僕射太常卿楊師道為侍中 侍中魏徵屢以目疾求為散官【散悉亶翻】上不得已以徵為特進仍知門下事【雖不居侍中之職猶令知門下事】朝章國典參議得失【朝直遥翻】徒流以上罪詳事聞奏其祿賜吏卒並同職事【特進散官也祿賜吏卒同職事官所以優賢也】 長孫皇后性仁孝儉素好讀書常與上從容商畧言事【好呼到翻從千容翻】因而獻替裨益弘多上或以非罪譴怒宫人后亦陽怒請自推鞫因命囚繫候上怒息徐為申理由是宫壼之中刑無枉濫豫章公主早喪其母后收養之慈愛逾於所生【豫章公主上女也後下嫁唐儀識為于偽翻壼苦本翻喪息浪翻】妃嬪以下有疾后親撫視輟己之藥膳以資之宫中無不愛戴訓諸子常以謙儉為先太子乳母遂安夫人【唐制太子乳母封郡夫人睦州遂安郡】嘗白后以東宫器用少請奏益之【少詩沼翻】后不許曰為太子患在德不立名不揚何患無器用邪上得疾累年不愈后侍奉晝夜不離側【離力智翻】常繫毒藥於衣帶曰若有不諱義不獨生后素有氣疾前年從上幸九成宫柴紹等中夕告變上擐甲出閣問狀后扶疾以從【擐音宦從才用翻】左右止之后曰上既震驚吾何心自安由是疾遂甚太子言於后曰醫藥備盡而疾不瘳【瘳丑留翻】請奏赦罪人及度人入道庶獲冥福后曰死生有命非智力所移若為善有福則吾不為惡如其不然妄求何益赦者國之大事不可數下【數所角翻】道釋異端之教蠧國病民皆上素所不為奈何以吾一婦人使上為所不為乎必行汝言吾不如速死太子不敢奏私以語房玄齡【語牛倨翻】玄齡白上上哀之欲為之赦【為于偽翻】后固止之及疾篤與上訣時房玄齡以譴歸第后言於帝曰玄齡事陛下久小心慎密奇謀祕計未嘗宣泄苟無大故願勿棄之妾之本宗因緣葭莩以致祿位【漢書曰非有葭莩之親張晏曰葭蘆葉也莩葉裏白皮也晉灼曰莩葭裏之白皮也皆取喻於輕薄也師古曰葭蘆也莩者其筩中白皮至薄者也葭莩喻薄莩音孚張言葉裏白皮非也】既非德舉易致顛危欲使其子孫保全慎勿處之權要但以外戚奉朝請足矣【以無忌之賢不能自保則后之所慮為深遠矣易以豉翻處昌呂翻朝直遥翻】妾生無益於人不可以死害人【用記檀弓成子高語意】願勿以丘壟勞費天下但因山為墳器用瓦木而已仍願陛下親君子遠小人納忠諫屏讒慝省作役止遊畋【遠于願翻屛必郢翻】妾雖没於九泉誠無所恨兒女輩不必令來見其悲哀徒亂人意因取衣中毒藥以視上曰妾於陛下不豫之日誓以死從乘輿不能當呂后之地耳【呂后事見漢紀乘繩證翻】己卯崩于立政殿【閣本太極宫圖東上閣門之東有萬春殿萬春殿之東有立政殿唐六典太極殿之北有兩儀殿兩儀殿之東曰萬春殿兩儀之左曰獻春門獻春門之左曰立政門其内曰立政殿】后嘗采自古婦人得失事為女則三十卷又嘗著論駮漢明德馬后以不能抑退外親使當朝貴盛徒戒其車如流水馬如龍【見四十六卷漢章帝建初二年駮北角翻朝直遥翻】是開其禍敗之源而防其末流也及崩宫司并女則奏之【唐内職有宫正糾?彤史記功書過六典尚儀局有司籍二人掌經史教學奏女則者蓋司籍也】上覽之悲慟以示近臣曰皇后此書足以垂範百世朕非不知天命而為無益之悲但入宫不復聞規諫之言【復扶又翻下今復同】失一良佐故不能忘懷耳乃召房玄齡使復其位 秋八月丙子上謂羣臣曰朕開直言之路以利國也而比來上封事者多訐人細事【比毗至翻上時掌翻訐居謁翻】自今復有為是者朕當以讒人罪之 冬十一月庚午葬文德皇后於昭陵【昭陵在京兆醴泉縣西北六十里】將軍段志玄宇文士及分統士衆出肅章門【唐六典曰西内太極殿次北曰朱明門門之左曰䖍化門右曰肅章門肅章之西曰揮政門又曰兩儀殿蓋古之内朝也承天之門東曰長樂門北入恭禮門又北入䖍化門則宫内也承天門之西曰永安門北入安仁門又北入肅章門則宫内也】帝夜使宫官至二人所士及開營内之志玄閉門不納曰軍門不可夜開使者曰此有手敕志玄曰夜中不辨真偽竟留使者至明帝聞而歎曰真將軍也帝復為文刻之石【使疏吏翻復扶又翻下復何亦復同】稱皇后節儉遺言薄葬以為盜賊之心止求珍貨既無珍貨復何所求朕之本志亦復如此王者以天下為家何必物在陵中乃為己有今因九嵕山為陵【嵕祖紅翻】鑿石之工纔百餘人數十日而畢不藏金玉人馬器皿皆用土木形具而已庶幾姦盜息心存没無累【幾居希翻累力瑞翻】當使百世子孫奉以為法上念后不已於苑中作層觀【觀古玩翻】以望昭陵嘗引魏徵同登使視之徵熟視之曰臣昏眊不能見【眊莫報翻】上指示之徵曰臣以為陛下望獻陵若昭陵則臣固見之矣上泣為之毁觀【為于偽翻】 十二月戊寅朱俱波甘棠遣使入貢朱俱波在葱嶺之北去瓜州二千八百里甘棠在大海南【朱俱波亦曰朱俱槃漢子合國也甘棠在西海之南崑崙人也二國皆在西域使疏吏翻】上曰中國既安四夷自服然朕不能無懼昔秦始皇威振胡越二世而亡唯諸公匡其不逮耳 魏王泰有寵於上或言三品以上多輕魏王上怒引三品以上作色讓之曰隋文帝時一品以下皆為諸王所顛躓【躓音致】彼豈非天子兒邪【邪音耶】朕但不聽諸子縱横耳【縱如字又子容翻横戶孟翻又如字】聞三品以上皆輕之我若縱之豈不能折辱公輩乎【折之舌翻】房玄齡等皆惶懼流汗拜謝魏徵獨正色曰臣竊計當今羣臣必無敢輕魏王者在禮臣子一也春秋王人雖微序於諸侯之上【春秋僖七年公會王人齊侯宋公衛侯許男曹伯陳世子款鄭世子華盟於洮公羊傳曰王人者何微者也曷為序乎諸侯之上先王命也】三品以上皆公卿陛下所尊禮若紀綱大壞固所不論聖明在上魏王必無頓辱羣臣之理隋文帝驕其諸子使多行無禮卒皆夷滅又足法乎【卒子恤翻】上悅曰理到之語不得不服朕以私愛忘公義曏者之忿自謂不疑及聞徵言方知理屈人主發言何得容易乎【易以豉翻】 上曰灋令不可數變數變則煩官長不能盡記【數所角翻長知兩翻】又前後差違吏得以為姦自今變灋皆宜詳慎而行之 治書侍御史權萬紀上言宣饒二州銀大發采之歲可得數百萬緡【治直之翻宋白曰饒州漢為鄱陽縣吳置鄱陽郡梁置吳州陳廢州復為郡隋平陳罷郡為饒州徐湛鄱陽記云北有堯山又以地饒衍遂加食為饒今郡圖又云以山川藴物珍奇故名饒】上曰朕貴為天子所乏者非財也但恨無嘉言可以利民耳與其多得數百萬緡何如得一賢才卿未嘗進一賢退一不肖而專言稅銀之利昔堯舜抵璧於山投珠於谷【陸賈新語曰聖人不用珠玉而寶其身故舜棄黄金於巉巖之山捐珠玉於五湖之川以杜淫邪之欲也】漢之桓靈乃聚錢為私藏【事見五十七卷漢靈帝光和元年藏徂浪翻】卿欲以桓靈俟我邪【邪音耶】是日黜萬紀使還家 是歲更命統軍為折衝都尉别將為果毅都尉【唐制上府折衝都尉正四品上中府正四品下下府正五品下上府果毅都尉從五品下中府正六品上下府從六品下更工衡翻將即亮翻】凡十道置府六百三十四而關内二百六十一皆隸諸衛及東宫六率【東宫六率者左右衛率擬上臺左右衛將軍左右宗衛率擬左右領軍將軍左右監門率擬左右監門將軍後又置左右虞候率擬左右金吾將軍左右内率擬左右千牛將軍通謂之十率】凡上府兵千二百人中府千人下府八百人三百人為團團有校尉【校戶教翻】五十人為隊隊有正十人為火火有長【長知兩翻】每人兵甲糧裝各有數皆自備輸之庫有征行則給之年二十為兵六十而免其能騎射者為越騎【越騎者言其勁勇能超越也騎奇寄翻】其餘為步兵每歲季冬折衝都尉帥其屬教戰【帥讀曰率】當給馬者官予其直市之【予讀曰與】凡當宿衛者番上兵部以遠近給番遠疎近數皆二月而更【時制五百里為五番千里七番一千五百里八番二千里十番外為十二番若簡留宿衛者五百里為七番千里八番二千里十番外為十二番上時掌翻數所角翻更工衡翻】<br />
<br />
  十一年春正月徙鄶王元裕為鄧王【鄶工外翻】譙王元名為舒王 辛卯以吳王恪為安州都督晉王治為并州都督紀王慎為秦州都督將之官上賜書戒敕曰吾欲遺汝珍玩恐益驕奢【遺于季翻】不如得此一言耳 上作飛山宫【觀明年廢明德宫及飛山宮之玄圃院以給洛人之遭水壞廬舍者則知飛山宮亦在洛陽】庚子特進魏徵上疏【上時掌翻】以為煬帝恃其富彊不虞後患窮奢極欲使百姓困窮以至身死人手社稷為墟陛下撥亂返正宜思隋之所以失我之所以得撤其峻宇安於卑宮若因基而增廣襲舊而加飾此則以亂易亂殃咎必至難得易失可不念哉【易以䜴翻】 房玄齡等先受詔定律令【先悉薦翻】以為舊法兄弟異居蔭不相及而謀反連坐皆死祖孫有蔭而止應配流據禮論情深為未愜今定律祖孫與兄弟緣坐者俱配没從之自是比古死刑除其大半天下稱賴焉玄齡等定律五百條立刑名二十等【笞刑五自十至於五十杖刑五自六十至於百徒刑五自一年至於三年流刑三自千里至於三千里死刑二絞斬】比隋律減大辟九十二條【辟毗亦翻】減流入徒者七十一條凡削煩去蠧變重為輕者不可勝紀【去羌呂翻勝音升】又定令一千五百九十餘條武德舊制釋奠於太學以周公為先聖孔子配饗玄齡等建議停祭周公以孔子為先聖顔囘配饗又刪武德以來敕格定留七百條至是頒行之又定枷杻鉗鏁杖笞皆有長短廣狹之制【械其頸曰枷械其手曰杻鉗以鉄刼束之也鏁以鉄琅當之也杖長三尺五寸削去節目訊杖大頭徑三分二釐小頭一分二釐常行杖大頭二分七釐小頭一分七釐笞杖大頭二分小頭一分有半杻女九翻】自張藴古之死【見上卷五年】法官以出罪為戒時有失入者又不加罪上嘗問大理卿劉德威曰近日刑網稍密何也對曰此在主上不在羣臣人主好寛則寛好急則急律文失入減三等失出減五等今失入無辜失出更復大罪是以吏各自免競就深文非有教使之然畏罪故耳陛下儻一斷以律則此風立變矣上悦從之由是斷獄平允【好呼到翻斷丁亂翻】 上以漢世豫作山陵免子孫倉猝勞費又志在儉葬恐子孫從俗奢靡二月丁巳自為終制因山為陵容棺而已 甲子上行幸洛陽宫 上至顯仁宫【隋志河南夀安縣有顯仁宫煬帝大業元年所起】官吏以缺儲偫有被譴者【偫直里翻被皮義翻】魏徵諫曰陛下以儲偫譴官吏臣恐承風相扇異日民不聊生殆非行幸之本意也昔煬帝諷郡縣獻食視其豐儉以為賞罰【見一百八十三卷大業十二年】故海内叛之此陛下所親見奈何欲效之乎上驚曰非公不聞此言因謂長孫無忌等曰朕昔過此買飯而食僦舍而宿【僦子就翻】今供頓如此豈得嫌不足乎 三月丙戌朔日有食之庚子上宴洛陽宫西苑泛積翠池【洛陽西苑北距北邙西至孝水南帶洛水支渠穀洛二水會於其間慮其泛溢為三陂以禦之一曰積翠二曰月陂三曰上陽苑墻周迴一百二十六里】顧謂侍臣曰煬帝作此宫苑結怨於民【築西苑見一百八十卷大業元年】今悉為我有正由宇文述虞世基裴藴之徒内為諂諛外蔽聰明故也可不戒哉【按隋煬帝大業二年令宇文愷作洛陽西苑述恐當作愷】房玄齡魏徵上所定新禮一百三十八篇【上時掌翻】丙午詔行之 以禮部尚書王珪為魏王泰師【唐初因魏晉之制諸王置師一人開元改曰傅】上謂泰曰汝事珪當如事我泰見珪輒先拜珪亦以師道自居珪子敬直尚南平公主【公主上女也】先是公主下嫁【先悉薦翻】皆不以婦禮事舅姑珪曰今主上欽明動循禮法吾受公主謁見豈為身榮所以成國家之美耳乃與其妻就席坐令公主執笲行盥饋之禮【笲音煩竹器也以盛棗栗腶脩盥音管以盤水沃洗手也婦以特豚饋士昏禮曰舅坐於阼階西面姑坐於房外南面婦執笲棗栗東面拜奠於舅席訖婦又執腶脩升進北面拜奠於姑席舅姑入於室婦盥饋特豚明婦順也右胖載之舅俎左胖載之姑俎各以南為上】是後公主始行婦禮自珪始 羣臣復請封禪【五年諸州朝集使請封禪六年文武官請今羣臣復請復扶又翻】上使秘書監顔師古等議其禮房玄齡裁定之 夏四月己卯魏徵上疏以為人主善始者多克終者寡豈取之易而守之難乎【上時掌翻易以豉翻】蓋以殷憂則竭誠以盡下【殷音隱】安逸則驕恣而輕物盡下則胡越同心輕物則六親離德雖震之以威怒亦皆貌從而心不服故也人主誠能見可欲則思知足將興繕則思知止處高危則思謙降臨滿盈則思挹損遇逸樂則思撙節【處昌呂翻樂音洛撙慈損翻】在宴安則思後患防壅蔽則思延納疾讒邪則思正己行爵賞則思因喜而僭施刑罰則思因怒而濫【僭僭差濫濫溢也】兼是十思而選賢任能固可以無為而治【治直之翻】又何必勞神苦體以代百司之任哉<br />
<br />
  資治通鑑卷一百九十四  <br>
   </div> 

<script src="/search/ajaxskft.js"> </script>
 <div class="clear"></div>
<br>
<br>
 <!-- a.d-->

 <!--
<div class="info_share">
</div> 
-->
 <!--info_share--></div>   <!-- end info_content-->
  </div> <!-- end l-->

<div class="r">   <!--r-->



<div class="sidebar"  style="margin-bottom:2px;">

 
<div class="sidebar_title">工具类大全</div>
<div class="sidebar_info">
<strong><a href="http://www.guoxuedashi.com/lsditu/" target="_blank">历史地图</a></strong>  
<a href="http://www.880114.com/" target="_blank">英语宝典</a>  
<a href="http://www.guoxuedashi.com/13jing/" target="_blank">十三经检索</a> 
<br><strong><a href="http://www.guoxuedashi.com/gjtsjc/" target="_blank">古今图书集成</a></strong> 
<a href="http://www.guoxuedashi.com/duilian/" target="_blank">对联大全</a> <strong><a href="http://www.guoxuedashi.com/xiangxingzi/" target="_blank">象形文字典</a></strong> 

<br><a href="http://www.guoxuedashi.com/zixing/yanbian/">字形演变</a>  <strong><a href="http://www.guoxuemi.com/hafo/" target="_blank">哈佛燕京中文善本特藏</a></strong>
<br><strong><a href="http://www.guoxuedashi.com/csfz/" target="_blank">丛书&方志检索器</a></strong> <a href="http://www.guoxuedashi.com/yqjyy/" target="_blank">一切经音义</a>  

<br><strong><a href="http://www.guoxuedashi.com/jiapu/" target="_blank">家谱族谱查询</a></strong>  <strong><a href="http://shufa.guoxuedashi.com/sfzitie/" target="_blank">书法字帖欣赏</a></strong> 
<br>

</div>
</div>


<div class="sidebar" style="margin-bottom:0px;">

<font style="font-size:22px;line-height:32px">QQ交流群9:489193090</font>


<div class="sidebar_title">手机APP 扫描或点击</div>
<div class="sidebar_info">
<table>
<tr>
	<td width=160><a href="http://m.guoxuedashi.com/app/" target="_blank"><img src="/img/gxds-sj.png" width="140"  border="0" alt="国学大师手机版"></a></td>
	<td>
<a href="http://www.guoxuedashi.com/download/" target="_blank">app软件下载专区</a><br>
<a href="http://www.guoxuedashi.com/download/gxds.php" target="_blank">《国学大师》下载</a><br>
<a href="http://www.guoxuedashi.com/download/kxzd.php" target="_blank">《汉字宝典》下载</a><br>
<a href="http://www.guoxuedashi.com/download/scqbd.php" target="_blank">《诗词曲宝典》下载</a><br>
<a href="http://www.guoxuedashi.com/SiKuQuanShu/skqs.php" target="_blank">《四库全书》下载</a><br>
</td>
</tr>
</table>

</div>
</div>


<div class="sidebar2">
<center>


</center>
</div>

<div class="sidebar"  style="margin-bottom:2px;">
<div class="sidebar_title">网站使用教程</div>
<div class="sidebar_info">
<a href="http://www.guoxuedashi.com/help/gjsearch.php" target="_blank">如何在国学大师网下载古籍?</a><br>
<a href="http://www.guoxuedashi.com/zidian/bujian/bjjc.php" target="_blank">如何使用部件查字法快速查字?</a><br>
<a href="http://www.guoxuedashi.com/search/sjc.php" target="_blank">如何在指定的书籍中全文检索?</a><br>
<a href="http://www.guoxuedashi.com/search/skjc.php" target="_blank">如何找到一句话在《四库全书》哪一页?</a><br>
</div>
</div>


<div class="sidebar">
<div class="sidebar_title">热门书籍</div>
<div class="sidebar_info">
<a href="/so.php?sokey=%E8%B5%84%E6%B2%BB%E9%80%9A%E9%89%B4&kt=1">资治通鉴</a> <a href="/24shi/"><strong>二十四史</strong></a>&nbsp; <a href="/a2694/">野史</a>&nbsp; <a href="/SiKuQuanShu/"><strong>四库全书</strong></a>&nbsp;<a href="http://www.guoxuedashi.com/SiKuQuanShu/fanti/">繁体</a>
<br><a href="/so.php?sokey=%E7%BA%A2%E6%A5%BC%E6%A2%A6&kt=1">红楼梦</a> <a href="/a/1858x/">三国演义</a> <a href="/a/1038k/">水浒传</a> <a href="/a/1046t/">西游记</a> <a href="/a/1914o/">封神演义</a>
<br>
<a href="http://www.guoxuedashi.com/so.php?sokeygx=%E4%B8%87%E6%9C%89%E6%96%87%E5%BA%93&submit=&kt=1">万有文库</a> <a href="/a/780t/">古文观止</a> <a href="/a/1024l/">文心雕龙</a> <a href="/a/1704n/">全唐诗</a> <a href="/a/1705h/">全宋词</a>
<br><a href="http://www.guoxuedashi.com/so.php?sokeygx=%E7%99%BE%E8%A1%B2%E6%9C%AC%E4%BA%8C%E5%8D%81%E5%9B%9B%E5%8F%B2&submit=&kt=1"><strong>百衲本二十四史</strong></a>  <a href="http://www.guoxuedashi.com/so.php?sokeygx=%E5%8F%A4%E4%BB%8A%E5%9B%BE%E4%B9%A6%E9%9B%86%E6%88%90&submit=&kt=1"><strong>古今图书集成</strong></a>
<br>

<a href="http://www.guoxuedashi.com/so.php?sokeygx=%E4%B8%9B%E4%B9%A6%E9%9B%86%E6%88%90&submit=&kt=1">丛书集成</a> 
<a href="http://www.guoxuedashi.com/so.php?sokeygx=%E5%9B%9B%E9%83%A8%E4%B8%9B%E5%88%8A&submit=&kt=1"><strong>四部丛刊</strong></a>  
<a href="http://www.guoxuedashi.com/so.php?sokeygx=%E8%AF%B4%E6%96%87%E8%A7%A3%E5%AD%97&submit=&kt=1">說文解字</a> <a href="http://www.guoxuedashi.com/so.php?sokeygx=%E5%85%A8%E4%B8%8A%E5%8F%A4&submit=&kt=1">三国六朝文</a>
<br><a href="http://www.guoxuedashi.com/so.php?sokeytm=%E6%97%A5%E6%9C%AC%E5%86%85%E9%98%81%E6%96%87%E5%BA%93&submit=&kt=1"><strong>日本内阁文库</strong></a> <a href="http://www.guoxuedashi.com/so.php?sokeytm=%E5%9B%BD%E5%9B%BE%E6%96%B9%E5%BF%97%E5%90%88%E9%9B%86&ka=100&submit=">国图方志合集</a> <a href="http://www.guoxuedashi.com/so.php?sokeytm=%E5%90%84%E5%9C%B0%E6%96%B9%E5%BF%97&submit=&kt=1"><strong>各地方志</strong></a>

</div>
</div>


<div class="sidebar2">
<center>

</center>
</div>
<div class="sidebar greenbar">
<div class="sidebar_title green">四库全书</div>
<div class="sidebar_info">

《四库全书》是中国古代最大的丛书,编撰于乾隆年间,由纪昀等360多位高官、学者编撰,3800多人抄写,费时十三年编成。丛书分经、史、子、集四部,故名四库。共有3500多种书,7.9万卷,3.6万册,约8亿字,基本上囊括了古代所有图书,故称“全书”。<a href="http://www.guoxuedashi.com/SiKuQuanShu/">详细>>
</a>

</div> 
</div>

</div>  <!--end r-->

</div>
<!-- 内容区END --> 

<!-- 页脚开始 -->
<div class="shh">

</div>

<div class="w1180" style="margin-top:8px;">
<center><script src="http://www.guoxuedashi.com/img/plus.php?id=3"></script></center>
</div>
<div class="w1180 foot">
<a href="/b/thanks.php">特别致谢</a> | <a href="javascript:window.external.AddFavorite(document.location.href,document.title);">收藏本站</a> | <a href="#">欢迎投稿</a> | <a href="http://www.guoxuedashi.com/forum/">意见建议</a> | <a href="http://www.guoxuemi.com/">国学迷</a> | <a href="http://www.shuowen.net/">说文网</a><script language="javascript" type="text/javascript" src="https://js.users.51.la/17753172.js"></script><br />
  Copyright &copy; 国学大师 古典图书集成 All Rights Reserved.<br>
  
  <span style="font-size:14px">免责声明:本站非营利性站点,以方便网友为主,仅供学习研究。<br>内容由热心网友提供和网上收集,不保留版权。若侵犯了您的权益,来信即刪。scp168@qq.com</span>
  <br />
ICP证:<a href="http://www.beian.miit.gov.cn/" target="_blank">鲁ICP备19060063号</a></div>
<!-- 页脚END --> 
<script src="http://www.guoxuedashi.com/img/plus.php?id=22"></script>
<script src="http://www.guoxuedashi.com/img/tongji.js"></script>

</body>
</html>
