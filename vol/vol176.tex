<!DOCTYPE html PUBLIC "-//W3C//DTD XHTML 1.0 Transitional//EN" "http://www.w3.org/TR/xhtml1/DTD/xhtml1-transitional.dtd">
<html xmlns="http://www.w3.org/1999/xhtml">
<head>
<meta http-equiv="Content-Type" content="text/html; charset=utf-8" />
<meta http-equiv="X-UA-Compatible" content="IE=Edge,chrome=1">
<title>資治通鑒_177-資治通鑑卷一百七十六_177-資治通鑑卷一百七十六</title>
<meta name="Keywords" content="資治通鑒_177-資治通鑑卷一百七十六_177-資治通鑑卷一百七十六">
<meta name="Description" content="資治通鑒_177-資治通鑑卷一百七十六_177-資治通鑑卷一百七十六">
<meta http-equiv="Cache-Control" content="no-transform" />
<meta http-equiv="Cache-Control" content="no-siteapp" />
<link href="/img/style.css" rel="stylesheet" type="text/css" />
<script src="/img/m.js?2020"></script> 
</head>
<body>
 <div class="ClassNavi">
<a  href="/24shi/">二十四史</a> | <a href="/SiKuQuanShu/">四库全书</a> | <a href="http://www.guoxuedashi.com/gjtsjc/"><font  color="#FF0000">古今图书集成</font></a> | <a href="/renwu/">历史人物</a> | <a href="/ShuoWenJieZi/"><font  color="#FF0000">说文解字</a></font> | <a href="/chengyu/">成语词典</a> | <a  target="_blank"  href="http://www.guoxuedashi.com/jgwhj/"><font  color="#FF0000">甲骨文合集</font></a> | <a href="/yzjwjc/"><font  color="#FF0000">殷周金文集成</font></a> | <a href="/xiangxingzi/"><font color="#0000FF">象形字典</font></a> | <a href="/13jing/"><font  color="#FF0000">十三经索引</font></a> | <a href="/zixing/"><font  color="#FF0000">字体转换器</font></a> | <a href="/zidian/xz/"><font color="#0000FF">篆书识别</font></a> | <a href="/jinfanyi/">近义反义词</a> | <a href="/duilian/">对联大全</a> | <a href="/jiapu/"><font  color="#0000FF">家谱族谱查询</font></a> | <a href="http://www.guoxuemi.com/hafo/" target="_blank" ><font color="#FF0000">哈佛古籍</font></a> 
</div>

 <!-- 头部导航开始 -->
<div class="w1180 head clearfix">
  <div class="head_logo l"><a title="国学大师官网" href="http://www.guoxuedashi.com" target="_blank"></a></div>
  <div class="head_sr l">
  <div id="head1">
  
  <a href="http://www.guoxuedashi.com/zidian/bujian/" target="_blank" ><img src="http://www.guoxuedashi.com/img/top1.gif" width="88" height="60" border="0" title="部件查字,支持20万汉字"></a>


<a href="http://www.guoxuedashi.com/help/yingpan.php" target="_blank"><img src="http://www.guoxuedashi.com/img/top230.gif" width="600" height="62" border="0" ></a>


  </div>
  <div id="head3"><a href="javascript:" onClick="javascript:window.external.AddFavorite(window.location.href,document.title);">添加收藏</a>
  <br><a href="/help/setie.php">搜索引擎</a>
  <br><a href="/help/zanzhu.php">赞助本站</a></div>
  <div id="head2">
 <a href="http://www.guoxuemi.com/" target="_blank"><img src="http://www.guoxuedashi.com/img/guoxuemi.gif" width="95" height="62" border="0" style="margin-left:2px;" title="国学迷"></a>
  

  </div>
</div>
  <div class="clear"></div>
  <div class="head_nav">
  <p><a href="/">首页</a> | <a href="/ShuKu/">国学书库</a> | <a href="/guji/">影印古籍</a> | <a href="/shici/">诗词宝典</a> | <a   href="/SiKuQuanShu/gxjx.php">精选</a> <b>|</b> <a href="/zidian/">汉语字典</a> | <a href="/hydcd/">汉语词典</a> | <a href="http://www.guoxuedashi.com/zidian/bujian/"><font  color="#CC0066">部件查字</font></a> | <a href="http://www.sfds.cn/"><font  color="#CC0066">书法大师</font></a> | <a href="/jgwhj/">甲骨文</a> <b>|</b> <a href="/b/4/"><font  color="#CC0066">解密</font></a> | <a href="/renwu/">历史人物</a> | <a href="/diangu/">历史典故</a> | <a href="/xingshi/">姓氏</a> | <a href="/minzu/">民族</a> <b>|</b> <a href="/mz/"><font  color="#CC0066">世界名著</font></a> | <a href="/download/">软件下载</a>
</p>
<p><a href="/b/"><font  color="#CC0066">历史</font></a> | <a href="http://skqs.guoxuedashi.com/" target="_blank">四库全书</a> |  <a href="http://www.guoxuedashi.com/search/" target="_blank"><font  color="#CC0066">全文检索</font></a> | <a href="http://www.guoxuedashi.com/shumu/">古籍书目</a> | <a   href="/24shi/">正史</a> <b>|</b> <a href="/chengyu/">成语词典</a> | <a href="/kangxi/" title="康熙字典">康熙字典</a> | <a href="/ShuoWenJieZi/">说文解字</a> | <a href="/zixing/yanbian/">字形演变</a> | <a href="/yzjwjc/">金 文</a> <b>|</b>  <a href="/shijian/nian-hao/">年号</a> | <a href="/diming/">历史地名</a> | <a href="/shijian/">历史事件</a> | <a href="/guanzhi/">官职</a> | <a href="/lishi/">知识</a> <b>|</b> <a href="/zhongyi/">中医中药</a> | <a href="http://www.guoxuedashi.com/forum/">留言反馈</a>
</p>
  </div>
</div>
<!-- 头部导航END --> 
<!-- 内容区开始 --> 
<div class="w1180 clearfix">
  <div class="info l">
   
<div class="clearfix" style="background:#f5faff;">
<script src='http://www.guoxuedashi.com/img/headersou.js'></script>

</div>
  <div class="info_tree"><a href="http://www.guoxuedashi.com">首页</a> > <a href="/SiKuQuanShu/fanti/">四库全书</a>
 > <h1>资治通鉴</h1> <!--         下载:【右键另存为】即可 --></div>
  <div class="info_content zj clearfix">
  
<div class="info_txt clearfix" id="show">
<center style="font-size:24px;">177-資治通鑑卷一百七十六</center>
    資治通鑑卷一百七十六 宋 司馬光 撰<br />
<br />
  胡三省 音注<br />
<br />
  陳紀十【起閼逢執徐盡著雍涒灘凡五年】<br />
<br />
  長城公下<br />
<br />
  至德二年春正月甲子日有食之 己巳隋主享太廟辛未祀南郊 壬申梁主入朝于隋【朝直遥翻下同】服通天冠絳紗袍北面受郊勞及入見於大興殿隋主服通天冠絳紗袍梁主服遠遊冠朝服君臣並拜【通天冠絳紗袍天子之服也服天子之服北面以受郊勞示臣服於隋而未至純於臣也遠遊冠朝服諸王見天子之服也入見大興殿純於臣矣大興殿隋新都正殿也唐為西内太極殿遠遊三梁冠黑介幘朝服絳紗單衣白紗内單皁領袖皁襈革帶鈎䚢假帶方心絳紗蔽膝韈舄綬劒佩君臣並拜非禮也勞力到翻見賢遍翻襈雛免翻䚢丑例翻又敕列翻】賜縑萬匹珍玩稱是【稱是言其直與萬縑稱也稱尺證翻】 隋前華州刺史張賓【華戶化翻】儀同三司劉暉等造甲子元歷成【甲子元歷其要以上元甲子己巳已來至開皇四年歲在甲辰積算起】奏之壬辰詔頒新歷 【考異曰隋律歷志云二月撰成奏上今從帝紀】 癸巳大赦 二月乙巳隋主餞梁主於灞上【梁主歸國故餞之】 突厥蘇尼部男女萬餘口降隋【厥九勿翻尼女夷翻降戶江翻下同】 庚戍隋主如隴州【五代志扶風汧源縣西魏置隴東郡及東秦州後改隴州】 突厥達頭可汗請降於隋【可從刋入聲汗音寒考異曰隋帝紀云突厥阿史那玷厥帥其衆來降按時玷厥方彊蓋文降耳】 夏四月庚子<br />
<br />
  隋以吏部尚書虞慶則為右僕射 隋上大將軍賀婁子幹發五州兵擊吐谷渾【時發河西五州兵蓋凉甘瓜鄯廓也吐從暾入聲谷音浴】殺男女萬餘口二旬而還【還音旋又如字】帝以隴西頻被寇掠而俗不設村塢【塢安古翻壁壘也說文曰小障也一曰庳城也通俗文營居曰塢被皮義翻】命子幹勒民為堡【堡音保小城也】仍營田積穀子幹上書曰隴右河西土曠民稀邊境未寧不可廣佃【上時掌翻佃徒年翻作田也】比見屯田之所【比毗至翻】獲少費多虛役人功卒逢踐暴【少詩沼翻卒子恤翻】屯田疎遠者請皆廢省但隴右之人以畜牧為事【畜許六翻】若更屯聚彌不自安但使鎭戍連接烽堠相望【堠戶遘翻】民雖散居必謂無慮帝從之以子幹曉習邊事丁巳以為榆關總管【五代志榆林郡金河縣開皇三年置榆關總管榆林郡後置勝州】五月以吏部尚書江總為僕射 隋主以渭水多沙深淺不常漕者苦之六月壬子詔太子左庶子宇文愷帥水工鑿渠引渭水【帥讀曰率】自大興城東至潼關三百餘里名曰廣通渠漕運通利關内賴之 秋七月丙寅遣兼散騎常侍謝泉等聘于隋【散悉亶翻騎奇寄翻】 八月壬寅隋鄧恭公竇熾卒【熾昌志翻卒子恤翻】 乙卯將軍夏侯苗請降于隋【夏戶雅翻降戶江翻】隋主以通和不納 九月甲戌隋主以關中饑行如洛陽 隋主不喜詞華【喜許記翻】詔天下公私文翰並宜實録泗州刺史司馬幼之【五代志下邳郡後魏置南徐州梁改東徐州陳改安州後周改泗州】文表華艷付所司治罪治書侍御史趙郡李諤亦以當時屬文體尚輕薄【治直之翻屬之欲翻】上書曰魏之三祖【上時掌翻三祖謂曹魏父子孫太祖武皇帝高祖文皇帝烈祖明皇帝】崇尚文詞忽君人之大道好雕蟲之小藝【揚子曰童子雕蟲篆刻好呼到翻】下之從上遂成風俗江左齊梁其弊彌甚競一韻之奇爭一字之巧連篇累牘不出月露之形積案盈箱唯是風雲之狀世俗以此相高朝廷據兹擢士禄利之路既開愛尚之情愈篤於是閭里童昏貴遊總丱【童昏言童幼昏蒙未有知識也鄭玄曰貴遊子弟王公之子弟遊無官司者詩總角丱兮毛傳曰總角聚兩髦也丱幼稚也丱古患翻】未窺六甲【古者八歲入小學學六甲五方書計之事六甲謂六十甲子也】先製五言【謂五言詩】至如羲皇舜禹之典伊傅周孔之說不復關心【復扶又翻】何嘗入耳以傲誕為清虛以緣情為勲績指儒素為古拙用詞賦為君子故文筆日繁其政日亂良由棄大聖之軌模搆無用以為用也今朝廷雖有是詔如聞外州遠縣仍踵弊風躬仁孝之行者【行下孟翻】擯落私門不加收齒工輕薄之藝者選充吏職舉送天朝【朝直遥翻】蓋由刺史縣令未遵風教請普加采察送臺推劾【劾戶槩翻又戶得翻】又上言士大夫矜伐干進無復亷恥乞明加罪黜以懲風軌【風軌風迹也上時掌翻】詔以諤前後所奏頒示四方 突厥沙鉢略可汗數為隋所敗【厥九勿翻可從刋入聲汗音寒數所角翻敗補邁翻】乃請和親千金公主自請改姓楊氏為隋主女隋主遣開府儀同三司徐平和使於沙鉢略更封千金公主為大義公主【千金公主宇文氏請於沙鉢略欲復讐及兵敗於外衆離於内乃請為隋主女更封以大義非嘉名也取大義滅親云爾為大義不得其死張本使疏吏翻下同更工衡翻】晉王廣請因舋乘之【舋許覲翻】隋主不許沙鉢略遣使致書曰從天生大突厥天下賢聖天子伊利居盧設莫何沙鉢略可汗致書大隋皇帝皇帝婦父乃是翁比此為女夫乃是兒例兩境雖殊情義如一自今子子孫孫乃至萬世親好不絶【好呼到翻】上天為證終不違負此國羊馬皆皇帝之畜【畜許救翻】彼之繒綵皆此國之物【繒疾陵翻】帝復書曰大隋天子貽書大突厥沙鉢略可汗得書知大有善意既為沙鉢略婦翁今日視沙鉢略與兒子不異時遣大臣往彼省女復省沙鉢略也【省悉景翻復扶又翻】於是遣尚書右僕射虞慶則使於沙鉢略驍騎將軍長孫晟副之【騎奇寄翻長知兩翻晟承正翻】沙鉢略陳兵列其珍寶坐見慶則稱病不能起且曰我諸父以來不向人拜慶則責而諭之千金公主私謂慶則曰可汗豺狼性過與爭將齧人【齧五結翻噬也】長孫晟謂沙鉢略曰突厥與隋俱大國天子可汗不起安敢違意但可賀敦為帝女則可汗是大隋女壻奈何不敬婦翁沙鉢略笑謂其達官曰須拜婦翁【突厥子弟曰特勒大臣曰葉護曰屈律啜曰阿波曰俟利發曰吐屯曰俟斤曰閻洪達曰頡利發曰達干皆達官也】乃起拜頓顙【顙桑黨翻】跪受璽書以戴於首【璽斯氏翻】既而大慙與羣下相聚慟哭慶則又遣稱臣沙鉢略謂左右曰何謂臣左右曰隋言臣猶此云奴耳沙鉢略曰得為大隋天子奴虞僕射之力也贈慶則馬千匹并以從妹妻之【從才用翻妻七細翻】 冬十一月壬戌隋主遣兼散騎常侍薛道衡等來聘【散悉亶翻騎奇寄翻】戒道衡當識朕意勿以言辭相折【折之舌翻】 是歲上於先昭殿前起臨春結綺望仙三閣各高數十丈【高古到翻】連延數十間其牕牖壁帶縣楣欄檻皆以沈檀為之【釋名曰窻聰也於内見外之聰明也牖亦窻也說文牖穿壁以木為交窻壁帶壁中横木班固西都賦金釭銜壁是為列錢賢註曰以黄金為釭其中銜壁納之於壁帶為行列歷歷如錢也懸楣横木施於前後兩楹之間下不裝構今人謂之掛楣欄檻皆所以凭也施於簷下階際者曰欄施於窻牖之間者曰檻沈檀皆香木縣讀曰懸沈持林翻】飾以金玉間以珠翠【珠珍珠翠翡翠毛間古莧翻】外施珠簾内有寶牀寶帳其服玩瑰麗近古所未有每微風暫至香聞數里【瑰工囘翻聞音問】其下積石為山引水為池雜植奇花異卉【卉百草摠名音許偉翻又音諱】上自居臨春閣張貴妃居結綺閣龔孔二貴嬪居望仙閣並複道交相往來又有王李二美人張薛二淑媛袁昭儀何婕好江修容並有寵【梁制貴妃貴嬪貴姬是為三夫人金章龜鈕紫綬八十首佩于寘玉虎頭鞶淑媛淑儀淑容昭華昭儀昭容修華修儀修容是為九嬪金章龜鈕青綬八十首虎頭鞶佩采瓄玉婕妤容華充華承徽列榮五職位亞九嬪銀印珪鈕艾綬虎頭鞶美人才人良人三職散位銅印環鈕墨綬虎頭鞶嬪毗賓翻媛于眷翻婕妤音接于】迭遊其上以宮人有文學者袁大捨等為女學士僕射江總雖為宰輔不親政務日與都官尚書孔範散騎常侍王瑳等【散悉亶翻騎奇寄翻瑳倉何翻】文士十餘人侍上遊宴後庭無復尊卑之序謂之狎客上每飲酒使諸妃嬪及女學士與狎客共賦詩互相贈答【考異曰平陳記云張貴妃等八人夾坐江總等十人預宴先令八婦人襞采牋製五言詩十客一時繼和稽緩則罰酒今從陳書南史】采其尤艷麗者被以新聲【被皮義翻】選宮女千餘人習而歌之分部迭進其曲有玉樹後庭花臨春樂等【五代志後主於清樂中造黄鸝留及玉樹後庭花金釵兩鬢垂等曲與幸臣製其歌詞綺艷相高極於輕薄男女唱和其音甚哀臨春樂者言臨春閣之樂也樂音洛】大略皆美諸妃嬪之容色君臣酣歌自夕達旦以此為常張貴妃名麗華本兵家女為龔貴嬪侍兒上見而悦之得幸生太子深貴妃髮長七尺其光可鑑【長直亮翻】性敏慧有神彩進止詳華【詳審而華麗也】每瞻視眄睞【仰視曰瞻正觀曰視斜視曰眄旁視曰睞眄莫甸翻睞洛代翻】光采溢目照映左右善候人主顔色引薦諸宮女後宮咸德之競言其善又有厭魅之術【厭魅所謂婦人媚道也厭一琰翻魅音媚】常置淫祀於宮中聚女巫鼓舞上怠於政事百司啟奏並因宦者蔡脱兒李善度進請上倚隱囊【隱囊者為囊實以細輭置諸坐側坐倦則側身曲肱以隱之隱於靳翻】置張貴妃於膝上共決之李蔡所不能記者貴妃並為條疏【疏分也為于偽翻】無所遺脱因參訪外事人間有一言一事貴妃必先知白之由是益加寵異冠絶後庭【冠古玩翻】宦官近習内外連結援引宗戚縱横不法【援于元翻横下孟翻】賣官鬻獄貨賂公行賞罰之命不出于外【言出命不由中書而出於宮掖也】大臣有不從者因而譛之於是孔張之權熏灼四方【孔張謂孔貴嬪張貴妃也】大臣執政皆從風諂附孔範與孔貴嬪結為兄妹上惡聞過失【惡烏路翻】每有惡事孔範必曲為文飾稱揚贊美由是寵遇優渥言聽計從羣臣有諫者輒以罪斥之中書舍人施文慶頗涉書史嘗事上於東宮聰敏彊記明閑吏職【閑習也】心筭口占應時條理由是大被親幸【被皮義翻】又薦所善吳興沈客卿【五代志吳郡烏程縣舊置吳興郡】陽惠朗徐哲暨慧景等【姓纂周景王封少子於陽樊因邑命氏余按春秋之時齊人遷陽子孫蓋以國為氏江南自來有暨姓吳時有暨艷暨戟乙翻】云有吏能上皆擢用之以客卿為中書舍人客卿有口辯頗知朝廷典故兼掌金帛局【陳中書省分為二十一局】舊制軍人士人並無關市之稅上盛修宮室窮極耳目府庫空虛有所興造恒苦不給客卿奏請不問士庶並責關市之征而又增重其舊於是以陽惠朗為太市令暨慧景為尚書金倉都令史【梁制太市令屬太府卿秩六百石尚書金倉都令史金部倉部都令史也梁制尚書都令史視奉朝請恒戶登翻】二人家本小吏考校簿領纎亳不差然皆不達大體督責苛碎聚斂無厭【斂力贍翻厭於鹽翻】士民嗟怨客卿總督之每歲所入過於常格數十倍【過工禾翻】上大悦益以施文慶為知人【臨亂之君各賢其臣其信然矣】尤見親重小大衆事無不委任轉相汲引【汲水者引綆期必上人臣之相汲引亦猶是也】珥貂蟬者五十人【珥市志翻】孔範自謂文武才能舉朝莫及【朝直遥翻】從容白上曰外間諸將起自行伍【從千容翻將即亮翻行戶剛翻】匹夫敵耳深見遠慮豈其所知上以問施文慶文慶畏範亦以為然司馬申復贊之【復扶又翻】自是將帥微有過失【帥所類翻】即奪其兵分配文吏奪任忠部曲以配範及蔡徵由是文武解體以至覆滅【通鑑具叙陳氏亡國之由】<br />
<br />
  三年春正月戊午朔日有食之 隋主命禮部尚書牛弘修五禮勒成百卷戊辰詔行新禮【五禮吉凶軍賓嘉】 三月戊午隋以尚書左僕射高熲為左領軍大將軍 豐州刺史章大寶昭達之子也【五代志建安郡界陳置閩州後又置豐州章昭達歷事高祖世祖高宗皆有戰功】在州貪縱朝廷以太僕卿李暈代之暈將至辛酉大寶襲殺暈舉兵反 隋大司徒郢公王誼【大司徒周之六官按王誼拜大司徒隋主未受禪也隋既受禪改周之六官司徒列於三公不應復加大字】與隋主有舊【王誼少與隋主同學】其子尚帝女蘭陵公主帝待之恩禮稍薄誼頗怨望或告誼自言名應圖䜟相表當王【相悉亮翻】公卿奏誼大逆不道壬寅賜誼死 戊申隋主還長安【去年九月隋主如洛陽今還】 章大寶遣其將楊通攻建安不克【以此觀之陳之豐州治閩縣而建安縣自别置建安郡將即亮翻】臺軍將至大寶衆潰逃入山為追兵所擒夷三族 隋度支尚書長孫平【元年隋已改度支為民部志作工部尚書長孫平度徒洛翻長知兩翻】奏令民間每秋家出粟麥一石以下貧富為差儲之當社委社司檢校以備凶年名曰義倉隋主從之五月甲申初詔郡縣置義倉時民間多妄稱老小以免賦役【隋承周制男女三歲已下為黄十歲已下為小六十者為老】山東承北齊之弊政【北齊高齊言北齊者以别蕭氏之南齊】戶口租調姦偽尤多隋主命州縣大索貌閲【調徒弔翻索山客翻貌閲者閲其貌以驗老小之實】戶口不實者里正黨長遠配【隋頒新令制人五家為保保有長保五為閭閭四為族皆有正畿外置里正比閭正黨長比族正以相檢察焉長知兩翻】大功以下皆令析籍以防容隱【堂兄弟其服大功】於是計帳得新附一百六十四萬餘口高熲請為輸籍法徧下諸州【輸籍凡民間課輸皆籍其數使州縣長吏不得以走弄出沒下戶嫁翻】帝從之自是姦無所容矣諸州調物每歲河南自潼關河北自蒲坂輸長安者相屬於路【調徒弔翻屬之欲翻】晝夜不絶者數月 梁主殂諡曰孝明皇帝廟號世宗世宗孝慈儉約境内安之太子琮嗣位【琮藏宗翻】 初突厥阿波可汗既與沙鉢略有隙【厥九勿翻可從刋入聲汗音寒有隙事始上卷元年】阿波浸彊東距都斤【都斤突厥中山名沙鉢略初立建牙於此山】西越金山龜兹鐵勒伊吾及西域諸胡悉附之【伊吾之地吐屯設主之蓋突厥所署置也龜兹音丘慈】號西突厥【突厥自是分為東西】隋主亦遣上大將軍元契使于阿波以撫之【使疏吏翻】 秋七月庚申遣散騎常侍王話等聘于隋【散悉亶翻騎奇寄翻】 突厥沙鉢略既為達頭所困【達頭資阿波以兵使攻沙鉢略是為其所困者也】又畏契丹【西既為達頭所困東又畏契丹見逼契欺訖翻又音吃】遣使告急於隋請將部落度漠南寄居白道川【欲南傍長城下倚隋為援使疏吏翻將即亮翻又如字】隋主許之命晉王廣以兵援之【晉王廣時為并省北邊皆屬焉故命以兵援沙鉢略】給以衣食賜之車服鼓吹【吹昌瑞翻】沙鉢略因西擊阿波破之【借隋兵之勢以獲勝】而阿拔國乘虛掠其妻子官軍為擊阿拔敗之【為干偽翻敗補邁翻】所獲悉與沙鉢略沙鉢略大喜乃立約以磧為界【磧七迹翻】因上表曰天無二日土無二王【此語本之孟子上時掌翻下同】大隋皇帝眞皇帝也豈敢阻兵恃險偷竊名號今感慕淳風歸心有道屈膝稽顙【稽音啟顙桑黨翻】永為藩附遣其子庫合眞入朝 【考異曰隋突厥傳作窟合眞今從帝紀朝直遥翻】八月丙戌庫合眞至長安隋主下詔曰沙鉢往雖與和【往已往也言往事也】猶是二國今作君臣便成一體因命肅告郊廟普頒遠近凡賜沙鉢略詔不稱其名宴庫合眞於内殿【突厥馮陵諸夏周齊屈體結之今沙鉢略奉表稱臣遣子入覲隋主告之郊廟布之臣庶大其事也宴之於内殿親之也】引見皇后賞勞甚厚【見賢遍翻勞力到翻】沙鉢略大悦自是歲時貢獻不絶 九月將軍湛文徹侵隋和州隋儀同三司費寶首擊擒之【費扶沸翻姓也】 丙子隋使李若等來聘 冬十月壬辰隋以上柱國楊素為信州總管【五代志巴東郡梁置信州隋置楊素於永安將使之為舟師以伐陳也】 初北地傅縡【縡作代翻】以庶子事上於東宫及即位遷祕書監右衛將軍兼中書通事舍人負才使氣人多怨之施文慶沈客卿共譛縡受高麗使金上收縡下獄【麗力知翻使疏吏翻下戶嫁翻】縡於獄中上書曰夫君人者恭事上帝子愛下民省嗜欲遠諂佞未明求衣日旰忘食【中上時掌翻夫音扶遠于願翻旰古按翻】是以澤被區宇【被皮義翻】慶流子孫陛下頃來酒色過度不䖍郊廟大神專媚淫昏之鬼【謂寵張貴妃使女巫鼓舞於宫中而淫祀也】小人在側宦豎弄權惡忠直若仇讎視生民如草芥後宫曳綺繡廐馬餘菽粟百姓流離殭尸蔽野貨賄公行帑藏損耗神怒民怨衆叛親離臣恐東南王氣自斯而盡【惡烏路翻殭居良翻帑它朗翻藏徂浪翻王于况翻又如字】書奏上大怒頃之意稍解遣使謂縡曰我欲赦卿卿能改過不【使疏吏翻不讀曰否】對曰臣心如面臣面可改則臣心可改上益怒令宦者李善慶窮治其事遂賜死獄中【治直之翻】上每當郊祀常稱疾不行故縡言及之 是歲梁大將軍戚昕以舟師襲公安不克而還【公安陳荆州治所昕許斤翻還音旋又如字】隋主徵梁主叔父太尉吳王岑入朝拜大將軍封懷義公因留不遣【朝直遥翻】復置江陵總管以監之【隋罷江陵總管見上卷陳高宗太建 十四年復扶又翻監工銜翻】梁大將軍許世武密以城召荆州刺史宜黄侯慧紀【宜黄古縣吳立屬臨川郡隋併省】謀泄梁主殺之慧紀高祖之從孫也【從才用翻】 隋主使司農少卿崔仲方發丁三萬於朔方靈武築長城東距河西至綏州【五代志雕隂郡西魏置綏州】綿歷七百里以遏胡寇<br />
<br />
  四年梁改元廣運 甲子党項羌請降於隋【隋書党項羌者三苖之後也自稱獮猴種東接臨洮西平西拒葉護南北數千里每姓别為部落党底朗翻降戶江翻】 庚午隋頒歷於突厥【頒歷則禀受正朔矣】 二月隋始令刺史上佐每歲暮更入朝上考課【上佐謂長史司馬更工衡翻上時掌翻】 丁亥隋復令崔仲方發丁十五萬於朔方以東緣邊險要築數十城【朔方郡夏州復扶又翻】 丙申立皇弟叔謨為巴東王叔顯為臨江王叔坦為新會王叔隆為新寧王【五代志歷陽郡烏江縣陳為臨江郡南海郡新會縣舊置新會郡信安郡新興縣梁置新寧郡】 庚子隋大赦 三月己未洛陽男子高德上書請隋主為太上皇傳位皇太子帝曰朕承天命撫育蒼生日旰孜孜猶恐不逮【旰古安翻】豈效近代帝王傳位於子自求逸樂者哉【言不效齊武成周天元也樂音洛】 夏四月己亥遣周磻等聘于隋【磻薄官翻】 三月丁巳立皇子莊為會稽王【會古外翻】 秋八月隋遣散騎常侍裴豪等來聘【散悉亶翻騎奇寄翻】 戊申隋申明公李穆卒【隋以李穆能知幾保身故諡曰明卒子恤翻】葬以殊禮 閏月丁卯隋太子勇鎭洛陽 隋上柱國郕公梁士彦討尉遲迥【事見一百七十四卷高宗太建十二年尉紆勿翻】所當必破代迥為相州刺史【相息亮翻】隋主忌之召還長安上柱國公宇文忻與隋主少相厚【少時照翻】善用兵有威名隋主亦忌之以譴去官【忻為右領軍大將軍】以柱國舒公劉昉皆被疎遠【被皮義翻遠于願翻以當作與】閑居無事頗懷怨望數相往來【數所角翻】隂謀不軌忻欲使士彦於蒲州起兵【蒲州蒲坂河津之要去長安三百餘里】已為内應士彦之甥裴通預其謀而告之帝隱其事以士彦為晉州刺史【晉州平陽用武之地周齊兵爭以為重鎭】欲觀其意士彦忻然謂昉等曰天也又請儀同三司薛摩兒為長史帝亦許之後與公卿朝謁【朝直遥翻下同】帝令左右執士彦忻昉於行間詰之【行戶剛翻詰去吉翻】初猶不伏捕薛摩兒適至命之庭對【於殿庭面質其事】摩兒具論始末士彦失色顧謂摩兒曰汝殺我丙子士彦忻昉皆伏誅叔姪兄弟免死除名九月辛巳隋主素服臨射殿命百官射三家資物以為誡【三人者與隋主有舊又有翼戴之功而謀為不軌故為之素服而又以誡百官】 冬十月己酉隋以兵部尚書楊尚希為禮部尚書隋主每旦臨朝日昃不倦尚希諫曰周文王以憂勤損夀武王以安樂延年【鄭玄注禮記有是言昃阻力翻樂音洛】願陛下舉大綱責成宰輔繁碎之務非人主所宜親也帝善之而不能從 癸丑隋置山南道行臺於襄州【襄州治襄陽其地在長安南山之南】以秦王俊為尚書令俊妃崔氏生男隋主喜頒賜羣官直祕書内省博陵李文博家素貧【曹魏藏書在祕書中外三閣是時祕書已有内外之分矣隋氏開獻書之路召天下工書之士補續殘缺為正副二本藏于宫中其餘以實祕書内外之閣故置直祕書内省之官博陵郡定州】人往賀之文博曰賞罰之設功過所存今王妃生男於羣官何事乃妄受賞也聞者愧之 癸亥以尚書僕射江總為尚書令吏部尚書謝伷為僕射【伷直祐翻】 十一月己卯大赦 吐谷渾可汗夸呂在位百年【夸呂隋書吐谷渾傳作呂夸】屢因喜怒廢殺太子後太子懼謀執夸呂而降【降戶江翻下同】請兵於隋邊吏秦州總管河間王弘請以兵應之【秦州天水郡河間王弘隋主從祖弟】隋主不許太子謀洩為夸呂所殺復立其少子嵬王訶為太子【復扶又翻下同少詩照翻嵬五灰翻】疊州刺史杜粲【五代志臨洮郡疊川縣後周置疊州宋白曰以其地山多重疊也】請因其舋而討之【舋許覲翻】隋主又不許是歲嵬王訶復懼誅謀帥部落萬五千戶降隋遣使詣闕請兵迎之【帥讀曰率使疏吏翻下同】隋主曰渾賊風俗特異人倫【言其去人倫與中國異俗】父既不慈子復不孝朕以德訓人何有成其惡逆乎乃謂使者曰父有過失子當諫爭【爭與諍同音則迸翻】豈可潛謀非法受不孝之名溥天之下皆朕臣妾各為善事即稱朕心【稱尺證翻】嵬王既欲歸朕唯教嵬王為臣子之灋不可遠遣兵馬助為惡事【隋主可謂有君人之言矣】嵬王訶乃止禎明元年春正月戊寅大赦改元 癸巳隋主享太廟乙未隋制諸州歲貢士三人 二月丁巳隋主朝日<br />
<br />
  于東郊【五代志禮天子以春分朝日於東郊秋分夕月於西郊漢法不俟二分於東西郊常以郊泰畤旦出竹宮東向揖日其夕西向揖月魏文譏其煩褻似家人之事而以正月朝日于東門之外前史又以為非時及明帝太和元年二月丁亥朝日于東郊八月己丑夕月於西郊始合於古後周以春分朝日於國東門外為壇如其郊用特牲青幣青圭有邸皇帝乘青輅及祀官俱青冕執事者青弁燔燎如圓丘秋分夕月於國西門外為壇於埳中燔燎禮如朝日隋開皇初於國東春明門外為壇每以春分朝日又於國西開遠門外坎中為壇每以秋分夕月牲幣與周同朝直遥翻】 遣兼散騎常侍王亨等聘于隋【散悉亶翻騎奇寄翻】 隋發丁男十萬餘人修長城二旬而罷夏四月於揚州開山陽瀆以通運【揚州治廣陵山陽縣屬焉按春秋吳城䢴溝通江淮山陽瀆通於廣陵尚矣隋特開而深廣之將以伐陳也】 突厥沙鉢略可汗遣其子入貢于隋因請獵於恒代之間【拓拔氏始都平城建為代都置司州及代都尹後遷洛陽改司州為恒州故曰恒代也厥九勿翻可從刋入聲汗音寒恒戶登翻】隋主許之仍遣人賜以酒食沙鉢略帥部落再拜受賜【帥讀曰率】沙鉢略尋卒隋為之廢朝三日【卒子恤翻為于偽翻朝直遥翻】遣太常弔祭初沙鉢略以其子雍虞閭懦弱【懦乃卧翻又奴亂翻】遺令立其弟葉護處羅侯【葉護突厥達官】雍虞閭遣使迎處羅侯將立之【使疏吏翻下同】處羅侯曰我突厥自木杆可汗以來多以弟代兄【逸可汗捨其子而立木杆木杆捨其子而立佗鉢佗鉢卒攝圖大邏便遂至爭國事並見前杆古按翻】以庶奪嫡失先祖之法不相敬畏【謂大邏便詈辱菴羅又與沙鉢略為敵達頭又從而助之也】汝當嗣位我不憚拜汝雍虞閭曰叔與我父共根連體我枝葉也豈可使根本反從枝葉叔父屈於卑幼乎且亡父之命何可廢也願叔勿疑遣使相讓者五六處羅侯竟立是為莫何可汗以雍虞閭為葉護遣使上表言狀隋使車騎將軍長孫晟持節拜之【可從刋入聲汗音寒使疏吏翻上時掌翻騎奇寄翻拜莫何為可汗也】賜以鼓吹幡旗【吹昌瑞翻】莫何勇而有謀以隋所賜旗鼓西擊阿波阿波之衆以為得隋兵助之多望風降附遂生擒阿波【降戶江翻 考異曰隋突厥傳前云沙鉢略西擊阿波破擒之後又云處羅侯生擒阿波長孫晟傳曰處羅侯因臣奏曰阿波為天所滅與五六千騎在山谷間伏聽詔旨當取之以獻按前云沙鉢略破擒之擒衍字耳處羅侯云當取以獻則是得否未可必隋安得豫議其死生乎今從突厥傳後】上書請其死生之命隋主下其議【上時掌翻下戶嫁翻莫何不敢專殺阿波而請命於隋隋之威令可謂行於突厥矣】樂安公元諧請就彼梟首【梟古堯翻】武陽公李充請生取入朝【武陽郡公隋之魏州武陽郡也朝直遥翻下同】顯戮以示百姓隋主謂長孫晟於卿何如晟對曰若突厥背誕【杜預曰背誕謂背命放誕陸德明曰背音佩誕音但按今讀從去聲亦通】須齊之以刑今其昆弟自相夷滅阿波之惡非負國家因其困窮取而為戮恐非招遠之道不如兩存之左僕射高熲曰骨肉相殘教之蠧也宜存養以示寛大隋主從之 甲戌隋遣兼散騎常侍楊同等來聘【散悉亶翻騎奇寄翻】 五月乙亥朔日有食之 秋七月己丑隋衛昭王爽卒【卒子恤翻】 八月隋主徵梁主入朝梁主帥其羣臣二百餘人發江陵【朝直遥翻帥讀曰率】庚申至長安隋主以梁主在外遣武鄉公崔弘度將兵戍江陵軍至都州【隋無都州蕭琮傳作鄀州當從之五代志竟陵郡樂鄉縣西魏置鄀州又南郡紫陵縣其城南面梁置鄀州鄀市灼翻】梁主叔父太傅安平王巖弟荆州刺史義興王瓛等【瓛戶官翻】恐弘度襲之乙丑遣都官尚書沈君公詣荆州刺史宜黄侯慧紀請降【降戶江翻】九月庚寅慧紀引兵至江陵城下辛卯巖等驅文武男女十萬口來奔隋主聞之廢梁國【梁敬帝紹泰元年後梁中宗即帝位更三主三十三年而亡】遣尚書左僕射高熲安集遺民【射寅謝翻熲居永翻】梁中宗世宗各給守冢十戶拜梁王琮上柱國賜爵莒公 甲午大赦 冬十月隋主如同州癸亥如蒲州 十一月丙子以蕭巖為開府儀同三司東揚州刺史蕭瓛為吳州刺史【五代志會稽郡梁置東揚州吳郡陳置吳州】 丁亥以豫章王叔英兼司徒 甲午隋主如馮翊親祠故社【隋主生於馮翊猶漢祀豐枌榆社之意然親祠則禮重於漢矣】戊戌還長安【還從宣翻又音如字】是行也内史令李德林以疾不從【從才用翻】隋主自同州敕書追之【追召也】與議伐陳之計及還帝馬上舉鞭南指曰待平陳之日以七寶裝嚴公使自山以東無及公者【言又將顯貴之使出於等夷李德林山東人】 初隋主受禪以來與陳鄰好甚篤每獲陳諜皆給衣馬禮遣之【好呼到翻諜徙協翻間探之人】而高宗猶不禁侵掠故太建之末隋師入寇會高宗殂隋主即命班師【事見上卷太建十四年殂祚乎翻】遣使赴弔【使疏吏翻】書稱姓名頓首帝答之益驕書末云想彼統内如宜此宇宙清泰隋主不悦以示朝臣【朝直遥翻】上柱國楊素以為主辱臣死再拜請罪隋主問取陳之策於高熲對曰江北地寒田收差晩江南水田早熟量彼收穫之際【熲居永翻量音良】微徵士馬聲言掩襲彼必屯兵守禦足得廢其農時彼既聚兵我便解甲再三若此彼以為常後更集兵彼必不信猶豫之頃我乃濟師【此濟師謂舉兵濟江】登陸而戰兵氣益倍【謂兵既登岸後限大江士無反顧之心有必死之志其氣益倍】又江南土薄舍多茅竹所有儲積皆非地窖【窖古孝翻】若密遣行人因風縱火待彼修立復更燒之【復扶又翻】不出數年自可財力俱盡隋主用其策陳人始困於是楊素賀若弼及光州刺史高勱虢州刺史崔仲方等【五代志弋陽郡梁置光州弘農郡隋置虢州若人者翻勱音邁】爭獻平江南之策仲方上書曰今唯須武昌以下蘄和滁方吳海等州【上時掌翻武昌陳為郡隋平陳廢為縣屬江夏郡五代志蘄春郡後齊置羅州後周改曰蘄州歷陽郡後齊置和州江都郡清流縣舊置南譙州改曰滁州六合縣後齊置秦州後周改曰方州江都郡本南兖州後周改曰吳州東海郡東魏海州蘄居依翻又音其】更帖精兵【帖添帖】密營度計益信襄荆基郢等州【蜀郡益州巴東郡信州襄陽郡襄州南郡荆州竟陵郡豐鄉縣西魏置基州弋陽郡定城縣舊置郢州】速造舟楫多張形勢為水戰之具蜀漢二江是其上流【蜀江出三峽過南郡漢江過襄陽竟陵沔陽而二江合流國於東南者二江其上流也】水路衝要必爭之所賊雖流頭荆門延州公安巴陵隱磯夏首蘄口湓城置船【水經注江水過夷陵而東至流頭灘其水峻激奔暴魚龞所不能游行者苦之又出西陵峽而東歷荆門虎牙之門荆門之下為延州又東過南郡而東右與油水合謂之油口油口即公安也又東過長沙下嶲縣北與湘水會匯為洞庭而得巴陵又東至彭城磯磯北對隱磯夏首即夏口以夏水入江而得名屈原哀郢過夏首而西浮江水又東過蘄春縣與蘄水會謂之蘄口又東至尋陽得湓浦有湓城皆沿江要害之地也夏戶稚翻】然終聚漢口峽口以水戰大決【漢口即夏口峽口西陵峽口】若賊必以上流有軍令精兵赴援者下流諸將即須擇便横度如擁衆自衛上江諸軍鼓行以前【將即亮翻上江諸軍謂蜀江漢江順流東下之軍也】彼雖恃九江五湖之險非德無以為固徒有三吳百越之兵非恩不能自立矣隋主以仲方為基州刺史及受蕭巖等降【降戶江翻】隋主益忿謂高熲曰我為民父母豈可限一衣帶水不拯之乎命大作戰船人請密之隋主曰吾將顯行天誅何密之有使投其柹於江【柹方廢翻斫木札也】曰若彼懼而能改吾復何求【復扶又翻】楊素在永安【蜀先主敗於秭歸退還白帝起永安宫居之故巴東有永安之名】造大艦名曰五牙【艦戶黯翻】上起樓五層高百餘尺左右前後置六拍竿【拍竿發之以拍敵船】並高五十尺【高古號翻】容戰士八百人次曰黄龍置兵百人自餘平乘舴艋各有等差【舴陟格翻艋莫幸翻】晉州刺史皇甫續將之官稽首言陳有三可滅【之往也之官往服官事也稽音啟】帝問其狀曰大吞小一也以有道伐無道二也納叛臣蕭巖於我有詞三也陛下若命將出師臣願展絲髮之効隋主勞而遣之【將即亮翻勞力到翻】時江南妖異特衆【妖於驕翻】臨平湖草久塞忽然自開【臨平湖在餘杭郡錢塘縣此湖常蓁塞故老相傳湖開則天下平塞悉則翻】帝惡之乃自賣於佛寺為奴以厭之【惡烏路翻厭於葉翻】又於建康造大皇寺起七級浮圖未畢火從中起而焚之吳興章華好學善屬文朝臣以華素無伐閱【好呼到翻屬之欲翻朝直遥翻顔師古曰伐積功也閲經歷也】競排詆之除太市令華鬱鬱不得志上書極諫略曰昔高祖南平百越【謂平盧子略李賁元景仲蘭裕蕭勃之亂上時掌翻】北誅逆虜【謂平侯景】世祖東定吳會【謂破斬杜龕張彪】西破王琳【見一百六十八卷世祖天嘉元年】高宗克復淮南辟地千里【見一百七十一卷太建五年辟讀曰闢】三祖之功勤亦至矣陛下即位于今五年不思先帝之艱難不知天命之可畏溺於嬖寵惑於酒色【溺奴狄翻嬖卑義翻又博計翻】祠七廟而不出【記天子七廟三昭三穆與太祖之廟而七】拜三妃而臨軒【三妃龔孔張也】老臣宿將【蔣即亮翻】弃之草莽諂佞讒邪升之朝廷今疆場日蹙【場音亦】隋軍壓境陛下如不改弦易張【董仲舒曰譬之琴瑟不調必改而更張之乃可鼓也】臣見麋鹿復遊於姑蘇矣【伍子胥諫吳王而不聽曰臣見麋鹿遊於姑蘇矣吳卒以亡復扶又翻】帝大怒即日斬之【古語有之殺諫臣者必亡其國豈不信哉】<br />
<br />
  二年春正月辛巳立皇子恮為東陽王【恮莊緣翻】恬為錢塘王 遣散騎常侍袁雅等聘于隋【散悉亶翻騎奇寄翻】又遣散騎常侍九江周羅㬋將兵屯峽口侵隋峽州【九江郡江南之尋陽郡江州治所也夷陵梁置宜州西魏改曰拓州後周改曰峽州將即亮翻】三月甲戌隋遣兼散騎常侍程尚賢等來聘戊寅隋主下詔曰陳叔寶據手掌之地【辛臣說田戎曰洛陽地如掌耳】恣溪壑之欲【溪壑難盈故以為喻】劫奪閻閭資產俱竭驅逼内外勞役弗已窮奢極侈俾晝作夜斬直言之客滅無罪之家欺天造惡祭鬼求恩盛粉黛而執干戈曳羅綺而呼警蹕自古昏亂罕或能比君子潜逃小人得志天災地孽【孽魚列翻】物怪人妖衣冠鉗口【鉗其亷翻】道路以目【國語周厲王監謗道路以目言道路相逢以目相視不敢有言】重以背德違言揺蕩疆場【重直用翻背蒲妹翻場音亦】晝伏夜遊鼠竊狗盜天之所覆無非朕臣【覆敷救翻】每關聽覽有懷傷惻可出師授律應機誅殄在期一舉永清吳越又送璽書暴帝二十惡【璽斯氏翻】仍散寫詔書三十萬紙遍諭江外【中原以江南為江外】太子性聰敏好文學【好呼到翻】然頗有過失詹事袁憲切諫不聽時沈后無寵而近侍左右數於東宫往來太子亦數使人至后所帝疑其怨望甚惡之【數所角翻惡烏路翻】張孔二貴妃日夜構成后及太子之短孔範之徒又於外助之帝欲立張貴妃子始安王深為嗣嘗從容言之【嗣祥吏翻從千容翻】吏部尚書蔡徵順旨稱贊袁憲厲色折之曰皇太子國家儲副億兆宅心卿是何人輕言廢立帝卒從徵議【折之舌翻宅心居心也卒子恤翻】夏五月庚子廢太子胤為吳興王立揚州刺史始安王深為太子徵景歷之子也【蔡景歷歷事陳高祖世祖高宗】深亦聰惠【惠與慧同】有志操【操七到翻】容止儼然雖左右近侍未嘗見其喜愠帝聞袁憲嘗諫胤即用憲為尚書僕射帝遇沈后素薄張貴妃專後宫之政后澹然未嘗有所忌怨【澹徒敢翻】身居儉約衣服無錦繡之飾唯尋閲經史及釋典為事【釋典佛經也】數上書諫爭【數所角翻上時掌翻爭側迸翻】帝欲廢之而立張貴妃會國亡不果 冬十月己亥立皇子蕃為吳郡王 己未隋置淮南行省於夀春【行省即行臺也】以晉王廣為尚書令帝遣兼散騎常侍王琬兼通直散騎常侍許善心聘于隋【散悉亶翻騎奇寄翻】隋人留於客館琬等屢請還不聽【還音旋又如字為隋主褒美許善心張本】甲子隋以出師有事於太廟命晉王廣秦王俊清河公楊素皆為行軍元帥【帥所類翻】廣出六合【六合本漢堂邑縣之地江左立秦郡及尉氏縣後周改秦郡為六合郡隋開皇初廢郡改尉氏縣為六合縣】俊出襄陽【秦王俊以山南道行臺鎮襄陽今自襄陽出指漢口】素出永安【素鎮永安自永安下三峽】荆州刺史劉仁恩出江陵【荆州治江陵使劉仁恩出師會楊素東下】蘄州刺史王世積出蘄春【蘄州治蘄春使王世積出師自蘄口臨江津蘄音機又音其】廬州摠管韓擒虎出廬江【廬州治廬江使韓擒虎出師自横江渡攻姑孰】吳州摠管賀若弼出廣陵【吳州治廣陵使賀若弼自瓜洲渡江攻京口若人者翻】青州摠管弘農燕榮出東海【東海郡海州青州治益都此盖使燕榮以青州之師出朐山渡海以攻南沙也燕因肩翻】凡摠管九十兵五十一萬八千皆受晉王節度東接滄海西距巴蜀旌旗舟楫横亘數千里以左僕射高熲為晉王元帥長史【帥所類翻長知兩翻】右僕射王韶為司馬軍中事皆取決焉區處支度【處昌呂翻度徒洛翻】無所凝滯十一月丁卯隋主親餞將士乙亥至定城【述征記定城去潼關三十里夾道各一城】陳師誓衆 丙子立皇弟叔榮為新昌王叔匡為太原王 隋主如河東【河東蒲州】十二月庚子還長安 突厥莫何可汗西擊鄰國中流矢而卒【厥九勿翻可從刋入聲汗音寒中竹仲翻卒子恤翻】國人立雍虞閭號頡伽施多每都藍可汗【為突厥復亂張本頡戶結翻伽求迦翻】 隋軍臨江高熲謂行臺吏部郎中薛道衡曰今兹大舉江東必可克乎道衡曰克之嘗聞郭璞有言【郭璞晉人知數之士也】江東分王三百年【王于况翻】復與中國合今此數將周一也【晉元帝南渡即王位於建康歲在丁丑是年歲在戊申凡二百七十二年】主上恭儉勤勞叔寶荒淫驕侈二也國之安危在所委任彼以江摠為相唯事詩酒拔小人施文慶委以政事蕭摩訶任蠻奴為大將皆一夫之用耳三也【任蠻奴即任忠】我有道而大彼無德而小量其甲士不過十萬【量音良】西自巫峽東至滄海分之則勢懸而力弱聚之則守此而失彼四也席卷之勢事在不疑【卷讀曰捲】熲忻然曰得君言成敗之理令人豁然本以才學相期不意籌略乃爾【爾猶言如此也】秦王俊督諸軍屯漢口為上流節度詔以散騎常侍周羅㬋都督巴峽緣江諸軍事以拒之【散悉亶翻騎奇寄翻】楊素引舟師下三峽軍至流頭灘將軍戚昕以青龍百餘艘守狼尾灘地勢險峭隋人患之【水經注江水過流頭灘又東逕古宜昌縣北又東逕狼尾灘其地猶在黄牛峽之西杜佑通典曰狼尾灘今夷陵郡宜都縣界艘蘇遭翻峭七笑翻】素曰勝負大計在此一舉若晝日下船彼見我虛實灘流迅激制不由人則吾失其便不如以夜掩之素親帥黄龍數千艘銜枚而下遣開府儀同三司王長襲引步卒自南岸擊昕别柵大將軍劉仁恩帥甲騎自北岸趣白沙遲明而至擊之昕敗走悉俘其衆勞而遣之【帥讀曰率下同艘蘇遭翻柵直革翻騎奇寄翻趣七喻翻遲直二翻勞力到翻】秋毫不犯素帥水軍東下舟艫被江【艫音盧被皮義翻】旌甲曜日素坐平乘大船容貌雄偉陳人望之皆懼曰清河公即江神也江濱鎭戍聞隋軍將至相繼奏聞施文慶沈客卿並抑而不言初上以蕭巖蕭瓛梁之宗室擁衆來奔心忌之故遠散其衆以巖為東揚州刺史瓛為吳州刺史【瓛戶官翻蕭巖蕭瓛來奔及出藩事並見上年】使領軍任忠出守吳興郡以襟帶二州【任音壬】使南平王嶷鎮江州永嘉王彦鎮南徐州【江州治尋陽南徐州治京口皆緣江重鎮也嶷魚力翻】尋召二王赴明年元會命緣江諸防船艦悉從二王還都【艦戶黯翻】為威勢以示梁人之來者由是江中無一鬬船上流諸州兵皆阻楊素軍不得至湘州刺史晉熙王叔文【湘州治長沙】在職既久大得人和上以其據有上流隂忌之自度素與羣臣少恩【度徒洛翻少詩沼翻】恐不為用無可任者乃擢施文慶為都督湘州刺史配以精兵二千欲令西上【上時掌翻】仍徵叔文還朝【朝直遥翻下同】文慶深喜其事然懼出外之後執事者持己短長因進其黨沈客卿以自代未發間二人共掌機密護軍將軍樊毅言於僕射袁憲曰京口采石俱是要地各須鋭兵五千并出金翅二百緣江上下以為防備【金翅船名】憲及驃騎將軍蕭摩訶皆以為然【驃匹妙翻騎奇寄翻】乃與文武羣臣共議請如毅策【未幾韓擒虎濟采石賀若弼拔京口二道並進而陳以亡地有所必守蓋不待智者而後知也】施文慶恐無兵從已廢其述職【孟子曰諸侯朝於天子曰述職此以出守藩方為述職】而客卿又利文慶之任【之往也任職也之任往赴所職也】已得專權【文慶與客卿時共掌機密文慶若出則客卿得專之】俱言於朝必有論議不假面陳但作文啟即為通奏【謂朝臣若必有所陳說不須面見陳主言之但文字來便為聞達為于偽翻下内為同】憲等以為然二人齎啟入白帝曰此是常事邊城將帥足以當之【將即亮翻帥所類翻】若出人船必恐驚擾及隋軍臨江間諜驟至【間古莧翻諜徙協翻】憲等殷勤奏請至于再三丈慶曰元會將逼南郊之日太子多從【陳仍梁制以間歲正月上辛祀天地於南北二郊用特牛一蓋來年正月當行此禮故施文慶云然從才用翻】今若出兵事便廢闕帝曰今且出兵若北邊無事因以水軍從郊何為不可又曰如此則聲聞鄰境【聞音問】便謂國弱後又以貨動江摠摠内為之遊說【謂衆言雜進之後文慶又以貨動江摠使之助已說輸芮翻】帝重違其意【重如字】而迫羣官之請乃令付外詳議摠又抑憲等由是議久不決帝從容謂侍臣曰王氣在此齊兵三來周師再來無不摧敗【齊師三來謂梁敬帝紹泰元年徐嗣徽任約以齊師襲建康據石頭太平元年復襲破采石與齊蕭軌同入寇逼建康世祖天嘉元年齊將劉伯球慕容恃德助王琳下蕪湖皆敗周師再來謂天嘉元年獨狐盛賀若敦入湘川臨海王光大元年宇文直元定助華皎皆敗從千容翻】彼何為者邪都官尚書孔範曰長江天塹古以為限隔南北【魏文帝伐吳臨江見江濤洶湧歎曰固天所以限南北也塹七艷翻】今日虜軍豈能飛度邪邊將欲作功勞妄言事急【將即亮翻】臣每患官卑虜若度江臣定作太尉公矣【孔範自謂兼資文武故大言自詭立功自晉宋以來率謂三公為太尉公司徒公司空公】或妄言北軍馬死範曰此是我馬何為而死【言馬若度江必不能北歸將悉為我有亦大言也】帝笑以為然故不為深備奏伎縱酒賦詩不輟【伎渠綺翻女樂也】 是歲吐谷渾禆王拓跋木彌【吐谷渾自亦有拓跋姓禆音卑吐從暾入聲谷音浴】請以千餘家降隋【降戶江翻】隋主曰普天之下皆是朕臣朕之撫育俱存仁孝渾賊惛狂妻子懷怖【怖普故翻】並思歸化自救危亡然叛夫背父不可收納【背蒲妹翻】又其本意正自避死今若違拒又復不仁若更有音信但宜慰撫任其自拔不須出兵應接其妹夫及甥欲來亦任其意不勞勸誘也【所謂叛夫背父妹夫及甥當時必皆有主名而史不詳紀隋書作名王拓跋木彌禆王亦用漢書語背蒲妹翻誘羊久翻】 河南王移兹裒卒隋主令其弟樹歸襲統其衆【移兹裒降隋見上卷高宗太建十三年裒蒲侯翻卒子恤翻】<br />
<br />
  資治通鑑卷一百七十六  <br>
   </div> 

<script src="/search/ajaxskft.js"> </script>
 <div class="clear"></div>
<br>
<br>
 <!-- a.d-->

 <!--
<div class="info_share">
</div> 
-->
 <!--info_share--></div>   <!-- end info_content-->
  </div> <!-- end l-->

<div class="r">   <!--r-->



<div class="sidebar"  style="margin-bottom:2px;">

 
<div class="sidebar_title">工具类大全</div>
<div class="sidebar_info">
<strong><a href="http://www.guoxuedashi.com/lsditu/" target="_blank">历史地图</a></strong>  
<a href="http://www.880114.com/" target="_blank">英语宝典</a>  
<a href="http://www.guoxuedashi.com/13jing/" target="_blank">十三经检索</a> 
<br><strong><a href="http://www.guoxuedashi.com/gjtsjc/" target="_blank">古今图书集成</a></strong> 
<a href="http://www.guoxuedashi.com/duilian/" target="_blank">对联大全</a> <strong><a href="http://www.guoxuedashi.com/xiangxingzi/" target="_blank">象形文字典</a></strong> 

<br><a href="http://www.guoxuedashi.com/zixing/yanbian/">字形演变</a>  <strong><a href="http://www.guoxuemi.com/hafo/" target="_blank">哈佛燕京中文善本特藏</a></strong>
<br><strong><a href="http://www.guoxuedashi.com/csfz/" target="_blank">丛书&方志检索器</a></strong> <a href="http://www.guoxuedashi.com/yqjyy/" target="_blank">一切经音义</a>  

<br><strong><a href="http://www.guoxuedashi.com/jiapu/" target="_blank">家谱族谱查询</a></strong>  <strong><a href="http://shufa.guoxuedashi.com/sfzitie/" target="_blank">书法字帖欣赏</a></strong> 
<br>

</div>
</div>


<div class="sidebar" style="margin-bottom:0px;">

<font style="font-size:22px;line-height:32px">QQ交流群9:489193090</font>


<div class="sidebar_title">手机APP 扫描或点击</div>
<div class="sidebar_info">
<table>
<tr>
	<td width=160><a href="http://m.guoxuedashi.com/app/" target="_blank"><img src="/img/gxds-sj.png" width="140"  border="0" alt="国学大师手机版"></a></td>
	<td>
<a href="http://www.guoxuedashi.com/download/" target="_blank">app软件下载专区</a><br>
<a href="http://www.guoxuedashi.com/download/gxds.php" target="_blank">《国学大师》下载</a><br>
<a href="http://www.guoxuedashi.com/download/kxzd.php" target="_blank">《汉字宝典》下载</a><br>
<a href="http://www.guoxuedashi.com/download/scqbd.php" target="_blank">《诗词曲宝典》下载</a><br>
<a href="http://www.guoxuedashi.com/SiKuQuanShu/skqs.php" target="_blank">《四库全书》下载</a><br>
</td>
</tr>
</table>

</div>
</div>


<div class="sidebar2">
<center>


</center>
</div>

<div class="sidebar"  style="margin-bottom:2px;">
<div class="sidebar_title">网站使用教程</div>
<div class="sidebar_info">
<a href="http://www.guoxuedashi.com/help/gjsearch.php" target="_blank">如何在国学大师网下载古籍?</a><br>
<a href="http://www.guoxuedashi.com/zidian/bujian/bjjc.php" target="_blank">如何使用部件查字法快速查字?</a><br>
<a href="http://www.guoxuedashi.com/search/sjc.php" target="_blank">如何在指定的书籍中全文检索?</a><br>
<a href="http://www.guoxuedashi.com/search/skjc.php" target="_blank">如何找到一句话在《四库全书》哪一页?</a><br>
</div>
</div>


<div class="sidebar">
<div class="sidebar_title">热门书籍</div>
<div class="sidebar_info">
<a href="/so.php?sokey=%E8%B5%84%E6%B2%BB%E9%80%9A%E9%89%B4&kt=1">资治通鉴</a> <a href="/24shi/"><strong>二十四史</strong></a>&nbsp; <a href="/a2694/">野史</a>&nbsp; <a href="/SiKuQuanShu/"><strong>四库全书</strong></a>&nbsp;<a href="http://www.guoxuedashi.com/SiKuQuanShu/fanti/">繁体</a>
<br><a href="/so.php?sokey=%E7%BA%A2%E6%A5%BC%E6%A2%A6&kt=1">红楼梦</a> <a href="/a/1858x/">三国演义</a> <a href="/a/1038k/">水浒传</a> <a href="/a/1046t/">西游记</a> <a href="/a/1914o/">封神演义</a>
<br>
<a href="http://www.guoxuedashi.com/so.php?sokeygx=%E4%B8%87%E6%9C%89%E6%96%87%E5%BA%93&submit=&kt=1">万有文库</a> <a href="/a/780t/">古文观止</a> <a href="/a/1024l/">文心雕龙</a> <a href="/a/1704n/">全唐诗</a> <a href="/a/1705h/">全宋词</a>
<br><a href="http://www.guoxuedashi.com/so.php?sokeygx=%E7%99%BE%E8%A1%B2%E6%9C%AC%E4%BA%8C%E5%8D%81%E5%9B%9B%E5%8F%B2&submit=&kt=1"><strong>百衲本二十四史</strong></a>  <a href="http://www.guoxuedashi.com/so.php?sokeygx=%E5%8F%A4%E4%BB%8A%E5%9B%BE%E4%B9%A6%E9%9B%86%E6%88%90&submit=&kt=1"><strong>古今图书集成</strong></a>
<br>

<a href="http://www.guoxuedashi.com/so.php?sokeygx=%E4%B8%9B%E4%B9%A6%E9%9B%86%E6%88%90&submit=&kt=1">丛书集成</a> 
<a href="http://www.guoxuedashi.com/so.php?sokeygx=%E5%9B%9B%E9%83%A8%E4%B8%9B%E5%88%8A&submit=&kt=1"><strong>四部丛刊</strong></a>  
<a href="http://www.guoxuedashi.com/so.php?sokeygx=%E8%AF%B4%E6%96%87%E8%A7%A3%E5%AD%97&submit=&kt=1">說文解字</a> <a href="http://www.guoxuedashi.com/so.php?sokeygx=%E5%85%A8%E4%B8%8A%E5%8F%A4&submit=&kt=1">三国六朝文</a>
<br><a href="http://www.guoxuedashi.com/so.php?sokeytm=%E6%97%A5%E6%9C%AC%E5%86%85%E9%98%81%E6%96%87%E5%BA%93&submit=&kt=1"><strong>日本内阁文库</strong></a> <a href="http://www.guoxuedashi.com/so.php?sokeytm=%E5%9B%BD%E5%9B%BE%E6%96%B9%E5%BF%97%E5%90%88%E9%9B%86&ka=100&submit=">国图方志合集</a> <a href="http://www.guoxuedashi.com/so.php?sokeytm=%E5%90%84%E5%9C%B0%E6%96%B9%E5%BF%97&submit=&kt=1"><strong>各地方志</strong></a>

</div>
</div>


<div class="sidebar2">
<center>

</center>
</div>
<div class="sidebar greenbar">
<div class="sidebar_title green">四库全书</div>
<div class="sidebar_info">

《四库全书》是中国古代最大的丛书,编撰于乾隆年间,由纪昀等360多位高官、学者编撰,3800多人抄写,费时十三年编成。丛书分经、史、子、集四部,故名四库。共有3500多种书,7.9万卷,3.6万册,约8亿字,基本上囊括了古代所有图书,故称“全书”。<a href="http://www.guoxuedashi.com/SiKuQuanShu/">详细>>
</a>

</div> 
</div>

</div>  <!--end r-->

</div>
<!-- 内容区END --> 

<!-- 页脚开始 -->
<div class="shh">

</div>

<div class="w1180" style="margin-top:8px;">
<center><script src="http://www.guoxuedashi.com/img/plus.php?id=3"></script></center>
</div>
<div class="w1180 foot">
<a href="/b/thanks.php">特别致谢</a> | <a href="javascript:window.external.AddFavorite(document.location.href,document.title);">收藏本站</a> | <a href="#">欢迎投稿</a> | <a href="http://www.guoxuedashi.com/forum/">意见建议</a> | <a href="http://www.guoxuemi.com/">国学迷</a> | <a href="http://www.shuowen.net/">说文网</a><script language="javascript" type="text/javascript" src="https://js.users.51.la/17753172.js"></script><br />
  Copyright &copy; 国学大师 古典图书集成 All Rights Reserved.<br>
  
  <span style="font-size:14px">免责声明:本站非营利性站点,以方便网友为主,仅供学习研究。<br>内容由热心网友提供和网上收集,不保留版权。若侵犯了您的权益,来信即刪。scp168@qq.com</span>
  <br />
ICP证:<a href="http://www.beian.miit.gov.cn/" target="_blank">鲁ICP备19060063号</a></div>
<!-- 页脚END --> 
<script src="http://www.guoxuedashi.com/img/plus.php?id=22"></script>
<script src="http://www.guoxuedashi.com/img/tongji.js"></script>

</body>
</html>
