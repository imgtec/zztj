










 


 
 


 

  
  
  
  
  





  
  
  
  
  
 
  

  

  
  
  



  

 
 

  
   




  

  
  


    資治通鑑卷八十二   宋 司馬光 撰

  胡三省 音註

  晉紀四【起屠維作噩盡著雍敦牂凡十年】

  世祖武皇帝下

  太康十年夏四月太廟成乙巳祫祭【祫大合祭也公羊傳曰大祫者何合祭也其合祭奈何毁廟之主陳于太祖未毁廟之主皆升合食於太祖祫胡夾翻】大赦 慕容廆遣使請降【降戶江翻】五月詔拜廆鮮卑都督廆謁見何龕以士大夫禮巾衣到門【魏晉間士大夫謁見尊貴以巾褠為禮褠單衣也龕口含翻】龕嚴兵以見之廆乃改服戎衣而入人問其故廆曰主人不以禮待客客何為哉龕聞之甚慙深敬異之【受降如受敵居邉之帥嚴兵以見四夷之客未為過也何必以為慙乎】時鮮卑宇文氏段氏方彊【段氏東部鮮卑也杜佑曰宇文莫槐出於遼東塞外代為鮮卑東部大人徒河段疾六眷出遼西因亂被賣為漁陽烏桓大人厙傉家奴厙傉以其健使將人衆詣遼西逐食遂招誘亡叛以至彊盛余按晉書王浚傳段疾六眷務勿塵之世子段氏自務勿塵以來強盛久矣疾六眷因亂被掠容或有之務勿塵既能為部落之帥恐不待其子招誘而後能彊盛也】數侵掠廆廆卑辭厚幣以事之段國單于階以女妻廆生皝仁昭【慕容段氏遂為婚姻之國數所角翻單音蟬妻七細翻】廆以遼東僻遠徙居徒河之青山【徒河縣前漢屬遼西後漢屬遼東屬國魏晉省併入昌黎郡界後慕容氏復置徒河縣拓跋魏太武真君八年併徒河入昌黎郡廣興縣杜佑曰徒河青山在營州郡城東百九十里】 冬十月復明堂及南郊五帝位【明堂南郊除五帝座見七十九卷泰始二年】 十一月丙辰尚書令濟北成侯荀朂卒【濟子禮翻】朂有才思【思相吏翻】善伺人主意【伺相吏翻】以是能固其寵久在中書專管機事及遷尚書甚罔悵【罔與惘同惘失志之貌悵亦恨望失志之貌】人有賀之者朂曰奪我鳳皇池諸君何賀邪 帝極意聲色遂至成疾楊駿忌汝南王亮排出之甲申以亮為侍中大司馬假黄鉞大都督督豫州諸軍事治許昌徙南陽王東為秦王都督關中諸軍事始平王瑋為楚王都督荆州諸軍事濮陽王允為淮南王都督揚江二州諸軍事【按惠帝元康元年有司奏荆揚二州疆土曠遠統理尤難於是割揚州之豫章鄱陽廬陵臨川南康建安晉安荆州之桂陽安成武昌合十郡置江州則此時未有江州也疑江二二字衍更俟博考濮博木翻】並假節之國【晉制都督諸軍事有使持節有持節有假節使持節得殺二千石以下持節殺無官位人若軍事與使持節同假節惟軍事得殺犯軍令者】立皇子乂為長沙王頴為成都王晏為吳王熾為豫章王演為代王皇孫遹為廣陵王【熾昌志翻遹以律翻】又封淮南王子迪為漢王楚王子儀為毗陵王徙扶風王暢為順陽王暢弟歆為新野公暢駿之子也【暢嗣駿爵而不居關中之任故徙封】琅邪王覲弟澹為東武公繇為東安公覲伷之子也【晉制宗室封郡公者制度如小國王澹徒覽翻又徒濫翻伷音胄】初帝以才人謝玖賜太子【才人位次美人李延壽曰晉武帝采漢魏之制三夫人九嬪之下有美人才人中才人爵視千石以下玖舉有翻】生皇孫遹宫中嘗夜失火帝登樓望之遹年五歲牽帝裾入闇中曰暮夜倉猝宜備非常不可令照見人主帝由是奇之嘗對羣臣稱遹似宣帝故天下咸歸仰之帝知太子不才然恃遹明慧故無廢立之心復用王佑之謀【佑王濟從兄也與羊祜等並事文帝帝寵任之復扶又翻下復以同】以太子母弟柬瑋允分鎮要害【要害謂雍荆揚之地】又恐楊氏之偪復以佑為北軍中候典禁兵帝為皇孫遹高選僚佐【為干偽翻】以散騎常侍劉寔志行清素命為廣陵王傅【自魏以來王國置師友晉避景帝諱改師為傅行下孟翻】寔以時俗喜進趣【喜許記翻趣讀曰趨】少亷讓【少詩沼翻】欲令初除官通謝章者必推賢讓能乃得通之一官缺則擇為人所讓最多者用之以為人情爭則欲毁已所不如讓則競推於勝已故世爭則優劣難分時讓則賢智顯出當此時也能退身脩已則讓之者多矣雖欲守貧賤不可得也馳騖進趨而欲人見讓猶却行而求前也淮南相劉頌【王國置相漢制也晉後改為内史】上疏曰陛下以法禁寛縱積之有素未可一旦以直繩御下此誠時宜也然至于矯世救弊自宜漸就清肅譬猶行舟雖不横截迅流然當漸靡而往稍向所趨然後得濟也【此引濟川為譬也濟大川者雖不横絶大川亂流而渡然必因水勢漸靡而行舟向其所趨以登陸之路然後汔濟否則為水勢所使不能制舟以向所趨不得登岸矣】自泰始以來將三十年【帝受禪改元泰始至是二十五年】凡諸事業不茂既往【言立事造業不加茂於往時也】以陛下明聖猶未反叔世之敝以成始初之隆傳之後世不無慮乎使夫異時大業或有不安其憂責猶在陛下也臣聞為社稷計莫若封建親賢然宜審量事埶可使諸侯率義而動者其力足以維帶京邑若包藏禍心其埶不足獨以有為其齊此甚難陛下宜與達古今之士深共籌之【帝之使諸王分鎮而内不足以齊之此劉頌所為深慮也】周之諸侯有罪誅放其身而國祚不泯【如周烹齊哀公而立其弟靜宣王誅魯侯伯御而立孝公之類】漢之諸侯有罪或無子者國隨以亡【見前後漢紀】今宜反漢之敝循周之舊則下固而上安矣【余謂晉之所以待藩王者其宜不在此也】天下至大萬事至衆人君至少【少始紹翻】同於天日是以聖王之化執要於已委務於下非惡勞而好逸【好呼到翻】誠以政體宜然也夫居事始以别能否甚難察也【别彼列翻】因成敗以分功罪甚易識也【易以豉翻下居易同】今陛下每精于造始而怠於考終此政功所以未善也人主誠能居易執要論功罪於成敗之後則羣下無所逃其誅賞矣古者六卿分職冢宰為師【周禮天官冢宰地官司徒春官宗伯夏官司馬秋官司寇冬官司空是為六卿而冢宰總之】秦漢已來九列執事丞相都總【此西都以前制也】今尚書制斷諸卿奉成【自漢光武以來以吏事責尚書事歸臺閣諸卿奉成而已斷丁亂翻】於古制為太重可出衆事付外寺【外寺謂諸卿寺】使得專之尚書統領大綱若丞相之為歲終課功校簿賞罰而已斯亦可矣今動皆受成於上上之所失不得復以罪下【復扶又翻】歲終事功不建不知所責也夫細過謬妄人情之所必有而悉糾以法則朝野無立人矣近世以來為監司者類大綱不振而微過必舉【御史臺官及諸州刺史皆監司也朝直遥翻監工銜翻】蓋由畏避豪強而又懼職事之曠則謹密網以羅微罪使奏劾相接狀似盡公而撓法在其中矣【劾戶槩翻又戶得翻撓奴教翻】是以聖王不善碎密之案必責凶猾之奏則害政之姦自然禽矣夫創業之勲在于立教定制使遺風繫人心餘烈匡幼弱後世憑之雖昏猶明雖愚若智乃足尚也【言法制脩明雖後嗣昏愚有所據依則其冶猶若明智之為也此言蓋指太子不能克隆堂構而帝又無典則以貽子孫也然苟非其人道不虚行以劉禪之庸而輔之以諸葛亮則昭烈雖死猶不死也孔明死則孔明治蜀之法制雖存禪不能守之矣】至夫脩飾官署凡諸作役恒傷太過【恒戶登翻】不患不舉此將來所不須於陛下而自能者也今勤所不須以傷所憑竊以為過矣帝皆不能用 詔以劉淵為匈奴北部都尉【時改匈奴五部帥為五部都尉】淵輕財好施【好呼到翻施式智翻】傾心接物五部豪桀幽冀名儒多往歸之【為劉淵得衆以移晉祚張本】奚軻男女十萬口來降【奚軻亦夷種也】

  孝惠皇帝上之上【諱衷字正度武帝第二子也謚法柔質慈民曰惠】

  永熙元年春正月辛酉朔改元太熙【太熙武帝所改至四月己酉太子即位改元永熙未踰年改元猶為非禮安有先帝初棄羣臣太子即位而遽以是日改元乎】 己巳以王渾為司徒 司空侍中尚書令衛瓘子宣尚繁昌公主宣嗜酒多過失楊駿惡瓘欲逐之【惡烏路翻】乃與黄門謀共毁宣勸武帝奪公主瓘慙懼告老遜位詔進瓘位太保以公就第【瓘封菑陽公】 劇陽康子魏舒薨 三月甲子以右光禄大夫石鑒為司空【晉志左右光禄大夫假金章紫綬及光禄大夫加金章紫綬者品秩第二】 帝疾篤未有顧命勲舊之臣多已物故侍中車騎將軍楊駿獨侍疾禁中大臣皆不得在左右駿因輒以私意改易要近樹其心腹會帝小間【間如字間者病小差也】見其新所用者正色謂駿曰何得便爾時汝南王亮尚未發【去年遣亮出督豫州】乃令中書作詔以亮與駿同輔政又欲擇朝士有聞望者數人佐之【朝直遥翻聞音問】駿從中書借詔觀之得便藏去中書監華廙恐懼【華戶化翻廙逸職翻又羊至翻】自往索之終不與會帝復迷亂【索山客翻復扶又翻】皇后奏以駿輔政帝頷之夏四月辛丑皇后召華廙及中書令何劭口宣帝旨作詔以駿為太尉太子太傅都督中外諸軍事侍中録尚書事詔成后對廙劭以呈帝帝視而無言廙歆之孫劭曾之子也【華歆仕漢魏之間何曾仕魏晉之間位皆至公二人身名相似也】遂趣汝南王亮赴鎮【趣讀曰促】帝尋小間問汝南王來未左右言未至帝遂困篤己酉崩于含章殿【年五十五坤之六三曰含章可貞坤以含宏為德后道也含章殿必在皇后宫中春秋書公薨于小寢即安也】帝宇量宏厚明達好謀【好呼到翻】容納直言未嘗失色於人太子即皇帝位大赦改元【改大熙為永熙】尊皇后曰皇太后立妃賈氏為皇后楊駿入居太極殿【前殿也】梓宫將殯六宫出辭而駿不下殿【時梓宫蓋自含章殿徙殯太極殿也】以虎賁百人自衛【賁音奔】詔石鑒與中護軍張劭監作山陵【監工銜翻】汝南王亮畏駿不敢臨喪哭於大司馬門外【亮自大司馬出鎮未行尚居府中不敢入宫臨喪而哭于大司馬府門外君父之喪哭於門外非禮也】出營城外表求過葬而行或告亮欲舉兵討駿者駿大懼白太后令帝為手詔與石鑒張劭使帥陵兵討亮劭駿甥也即帥所領趣鑒速發【帥讀曰率趣讀曰促】鑒以為不然保持之【保亮不舉兵而持討亮之兵不發也】亮問計於廷尉何朂朂曰今朝野皆歸心于公【朝直遥翻下同】公不討人而畏人討邪亮不敢發夜馳赴許昌乃得免駿弟濟及甥河南尹李斌皆勸駿留亮駿不從濟謂尚書左丞傅咸曰家兄若徵大司馬退身避之門戶庶幾可全【幾居希翻】咸曰宗室外戚相恃為安但召大司馬還共崇至公以輔政無為避也濟又使侍中石崇見駿言之駿不從五月辛未葬武帝于峻陽陵楊駿自知素無美望欲依魏明帝即位故事普進封爵以求媚于衆左軍將軍傅祗【晉志曰按魏明帝時有左軍則左軍魏官也】與駿書曰未有帝王始崩臣下論功者也駿不從祗嘏之子也【傅嘏仕魏顯于嘉平正元之間】丙子詔中外羣臣皆增位一等預喪事者增二等二千石已上皆封關中侯【按漢獻帝建安二十年魏武王置關中侯據晉書帝紀關中侯又在關内侯之下】復租調一年【復方目翻調徒弔翻】散騎常侍石崇【前書侍中石崇此書散騎常侍必有一誤蓋因舊史成文也】散騎侍郎何攀共上奏【上時掌翻】以為帝正位東宫二十餘年今承大業而班賞行爵優於泰始革命之初及諸將平吳之功輕重不稱【稱證尺翻】且大晉卜世無窮今之開制當垂于後若有爵必進則數世之後莫非公侯矣不從詔以太尉駿為太傅大都督假黄鉞録朝政百官總已以聽傅咸謂駿曰諒闇不行久矣【自漢文短喪之詔嗣君即吉聽政諒閽三年之制不行久矣闇音隂】今聖上謙冲委政於公而天下不以為善懼明公未易當也【易以豉翻下同】周公大聖猶致流言【周成王幼冲周公攝政而四國流言】况聖上春秋非成王之年乎【武帝泰始二年帝為皇太子時年九歲至是三十二歲矣】竊謂山陵既畢明公當審思進退之宜苟有以察其忠款言豈在多駿不從咸數諫駿漸不平欲出咸為郡守李斌曰斥逐正人將失人望乃止楊濟遺咸書曰【數所角翻遺于季翻】諺云生子癡了官事官事未易了也想慮破頭故具有白【慮咸以直言致禍也】咸復書曰衛公有言酒色殺人甚於作直坐酒色死人不為悔而逆畏以直致禍此由心不能正欲以苟且為明哲耳【詩曰既明且哲以保其身此言世人不能直言特以苟且為保身之計耳】自古以直致禍者當由矯枉過正或不忠篤欲以亢厲為聲【亢口浪翻】故致忿耳安有悾悾忠益而返見怨疾乎【悾苦紅翻悾悾信也包咸曰慤也】楊駿以賈后險悍多權略忌之【悍下罕翻又侯旰翻】故以其甥段廣為散騎常侍管機密張劭為中護軍典禁兵凡有詔命帝省訖【省悉景翻】入呈太后然後行之駿為政嚴碎專愎【愎弼力翻狠也】中外多惡之【惡烏路翻】馮翊太守孫楚謂駿曰公以外戚居伊霍之任當以至公誠信謙順處之【守式又翻處昌呂翻】今宗室彊盛而公不與共參萬機内懷猜忌外樹私昵禍至無日矣【昵尼質翻】駿不從楚資之孫也【孫資事魏三祖掌機密】宏訓少府蒯欽駿之姑子也數以直言犯駿他人皆為之懼【景皇后居宏訓宫置少府數所角翻為于偽翻】欽曰楊文長雖闇猶知人之無罪不可妄殺【楊駿字文長】不過疎我我得疎乃可以免不然與之俱族矣駿辟匈奴東部人王彰為司馬【匈奴東部即匈奴左部也居太原兹氏縣】彰逃避不受其友新興張宣子怪而問之【漢獻帝建安二十年省雲中定襄五原朔方郡郡置一縣領其民合為新興郡屬并州】彰曰自古一姓二后未有不敗况楊太傅昵近小人疎遠君子【近其靳翻遠于願翻】專權自恣敗無日矣吾踰海出塞以避之猶懼及禍奈何應其辟乎且武帝不惟社稷大計【惟思也】嗣子既不克負荷【荷下可翻】受遺者復非其人天下之亂可立待也【楊駿之敗人皆知之獨駿不知耳凶人吉其凶其謂是乎復扶又翻】秋八月壬午立廣陵王遹為皇太子以中書監何劭為太子太師衛尉裴楷為少師吏部尚書王戎為太傅前太常張華為少府衛將軍楊濟為太保尚書和嶠為少保【晉東宫六傳惟此時具官】拜太子母謝氏為淑媛【媛于眷翻晉志淑妃淑媛淑儀脩華脩容脩儀婕好容華充華是為九嬪銀印青綬】賈后常置謝氏於别室不聽與太子相見初和嶠嘗從容言于武帝曰【從于容翻】皇太子有淳古之風而末世多偽恐不了陛下家事武帝默然後與荀朂等同侍武帝武帝曰太子近入朝差長進【朝直遥翻長丁丈翻今知兩翻】卿可俱詣之粗及世事【粗坐五翻略也】既還朂等並稱太子明識雅度誠如明詔嶠曰聖質如初武帝不悦而起及帝即位嶠從太子遹入朝賈后使帝問曰卿昔謂我不了家事今日定如何嶠曰臣昔事先帝曾有斯言言之不效國之福也冬十月辛酉以石鑒為太尉隴西王泰為司空 以

  劉淵為建威將軍匈奴五部大都督【淵為五部大都督則左國城大單于之權輿也】

  元康元年春正月乙酉朔改元永平【永平楊駿執政所改元也駿誅改元元康】 初賈后之為太子妃也嘗以妬手殺數人又以戟擲孕妾子隨刃墯【孕以證翻】武帝大怒脩金墉城將廢之荀朂馮紞楊珧【紞丁感翻珧余招翻】及充華趙粲共營救之曰賈妃年少妬者婦人常情長自當差【少詩照翻長知兩翻差楚懈翻】楊后曰賈公閭有大勲於社稷【賈充字公閭晉之代魏充力居多】妃親其女正復妬忌豈可遽忘其先德邪【復扶目翻】妃由是得不廢后數誡厲妃【數所角翻】妃不知后之助已返以后為搆已於武帝更恨之及帝即位賈后不肯以婦道事太后又欲干預政事而為太傅駿所抑殿中中郎渤海孟觀李肇皆駿所不禮也【晉制二衛置殿中將軍中郎校尉司馬觀如字】隂構駿云將危社稷黄門董猛素給事東宫為寺人監【寺人監主東宫諸閹陸德明曰寺如字又音侍】賈后密使猛與觀肇謀誅駿廢太后又使肇報汝南王亮使舉兵討駿亮不可肇報都督荆州諸軍事楚王瑋瑋欣然許之乃求入朝駿素憚瑋勇鋭欲召之而未敢因其求朝遂聽之二月癸酉瑋及都督揚州諸軍事淮南王允來朝【朝直遥翻】三月辛卯孟觀李肇啟帝夜作詔誣駿謀反中外戒嚴遣使奉詔廢駿以侯就第【駿封臨晉侯】命東安公繇帥殿中四百人討駿【帥讀曰率】楚王瑋屯司馬門以淮南相劉頌為三公尚書【漢成帝置三公尚書主斷獄光武以三公曹主歲盡考課諸州郡事】屯衛殿中段廣跪言於帝曰楊駿孤公無子豈有反理願陛下審之【廣駿甥也使為近侍以防左右間已然終無益也】帝不答時駿居曹爽故府在武庫南聞内有變召衆官議之太傅主簿朱振說駿曰今内有變其趣可知必是閹豎為賈后設謀不利於公宜燒雲龍門以脅之索造事者首開萬春門【雲龍門洛陽宫城正南門萬春門東門也說輸芮翻為于偽翻索山客翻】引東宫及外營兵擁皇太子入宫取姦人殿内震懼必斬送之不然無以免難【難乃旦翻】駿素怯懦不决乃曰雲龍門魏明帝所造功費甚大奈何燒之侍中傅祗白駿請與尚書武茂入宫觀察事埶因謂羣僚曰宫中不宜空遂揖而下階衆皆走茂猶坐祗顧曰君非天子臣邪今内外隔絶不知國家所在【國家謂天子也自東漢以來皆然】何得安坐茂乃驚起駿黨左軍將軍劉豫陳兵在門遇右軍將軍裴頠【魏有左軍武帝又置前軍右軍泰始八年又置後軍是為四軍頠魚毁翻】問太傅所在頠紿之曰【紿徒亥翻】向於西掖門遇公乘素車從二人西出矣豫曰吾何之頠曰宜至廷尉預從頠言遂委而去【委兵而去也】尋詔頠代豫領左軍將軍屯萬春門頠秀之子也【裴秀見七十八卷魏元帝咸熙元年】皇太后題帛為書射之城外【射而亦翻下同】曰救太傅者有賞賈后因宣言太后同反尋而殿中兵出燒駿府又令弩手於閣上臨駿府而射之駿兵皆不得出駿逃于馬廄就殺之孟觀等遂收駿弟珧濟張劭李斌段廣劉預武茂及散騎常侍楊邈中書令蔣俊東夷校尉文鴦皆夷三族死者數千人珧臨刑告東安公繇曰表在石函【珧表見八十卷武帝咸寧三年作石函藏之宗廟摯虞云廟主藏于戶之外西墉之中有石函名曰宗祏函中笥以盛主】可問張華衆謂宜依鍾毓例為之申理【鍾毓例見七十八卷魏元帝咸熙元年為于偽翻】繇不聽而賈氏族黨趣使行刑【趣讀曰促】珧號叫不已【號戶刀翻】刑者以刀破其頭繇諸葛誕之外孫也故忌文鴦以為駿黨而誅之【諸葛誕文鴦事見七十七卷魏高貴鄉公甘露三年】是夜誅賞皆自繇出威振内外王戎謂繇曰大事之後宜深遠權埶繇不從【遠丁願翻】壬辰赦天下改元【改元元康】賈后矯詔使後軍將軍荀悝送太后於永寧宫【魏建永寧宫太后居之悝苦回翻】特全太后母高都君龐氏之命聽就太后居【龐皮江翻】尋復諷羣公有司奏曰皇太后隂漸姦謀【履霜者堅冰之漸言隂始凝而至于堅冰也此誣楊太后以為與駿為姦謀非一日之積也復扶又翻下可復司復同漸如字】圖危社稷飛箭繫書要募將士【要讀曰邀將即亮翻】同惡相濟自絶于天魯侯絶文姜春秋所許【文姜魯桓公之夫人也齊襄公殺桓公文姜與焉魯莊公既立夫人孫于齊穀梁傳曰不言氏姓貶之也人之於天也以道受命於人也以言受命不若于道者天絶之也不若于人者人絶之也】蓋奉祖宗任至公於天下陛下雖懷無已之情臣下不敢奉詔詔曰此大事更詳之有司又奏宜廢太后曰峻陽庶人【武帝陵曰峻陽】中書監張華議太后非得罪於先帝今黨其所親為不母於聖世宜依漢廢趙太后為孝成后故事【事見三十五卷漢哀帝元壽元年】貶皇太后之號還稱武皇后居異宫以全始終之恩左僕射荀愷與太子少師下邳王晃等議曰皇太后謀危社稷不可復配先帝宜貶尊號廢詣金墉城於是有司奏從晃等議廢太后為庶人詔可又奏楊駿造亂家屬應誅詔原其妻龎命以尉太后之心今太后廢為庶人請以龎付廷尉行刑詔不許有司復固請乃從之龐臨刑太后抱持號叫截髪稽顙上表詣賈后稱妾請全母命不見省【號戶刀翻稽音啟省悉景翻】董養遊太學【董養浚儀隱者也】升堂歎曰朝廷建斯堂將以何為乎【言庠序所以申孝弟之義今滅母子之大倫則建學果何為也】每覽國家赦書謀反大逆皆赦至於殺祖父母父母不赦者以為王法所不容故也奈何公卿處議文飾禮典乃至此乎【處昌呂翻】天人之理既滅大亂將作矣【養後與妻荷擔入蜀不知所終】有司收駿官屬欲誅之侍中傅祗啟曰昔魯芝為曹爽司馬斬關赴爽【事見七十五卷魏邵陵厲公嘉平元年】宣帝用為青州刺史駿之僚佐不可悉加罪詔赦之壬寅徵汝南王亮為太宰與太保衛瓘皆録尚書事輔政以秦王柬為大將軍東平王楙為撫軍大將軍楚王瑋為衛將軍領北軍中候下邳王晃為尚書令東安公繇為尚書左僕射進爵為王楙望之子也封董猛為武安侯三兄皆為亭侯亮欲取悦衆心論誅楊駿之功督將侯者千八十一人【將即亮翻】御史中丞傅咸遺亮書曰今封賞熏赫震動天地自古以來未之有也無功而獲賞則人莫不樂國之有禍是禍原無窮也【濫賞所以開覬幸之心其禍誠如此遺于季翻樂音洛】凡作此者由東安公人謂殿下既至當有以正之正之以道衆亦何怒衆之所怒者在於不平耳而今皆更倍論【言亮論功行賞又倍於東安公之時也】莫不失望亮頗專權埶咸復諫曰楊駿有震主之威委任親戚此天下所以諠譁今之處重宜反此失靜默頤神有大得失乃維持之自非大事一皆抑遣比過尊門冠蓋車馬填塞街衢此之翕習既宜弭息【翕衆也合也習重也因也仍也言衆人翕合相因而至也復扶又翻處昌呂翻比毗至翻塞悉則翻】又夏侯長容無功而暴擢為少府論者謂長容公之姻家【夏侯駿字長容壻家女之所因故曰姻鄭玄曰壻父曰姻夏戶雅翻】故至於此流聞四方非所以為益也亮皆不從賈后族兄車騎司馬模從舅右衛將軍郭彰【晉文帝置中衛及衛將軍武帝受命分為左右衛將軍從才用翻】女弟之子賈謐【賈后女弟賈午適韓壽生謐賈充無後以謐為後】與楚王瑋東安王繇並預國政賈后暴戾日甚繇密謀廢后賈氏憚之繇兄東武公澹素惡繇【惡烏路翻】屢譛之於太宰亮曰繇專行誅賞欲擅朝政庚戍詔免繇官又坐有悖言廢徙帶方【帶方縣漢屬樂浪郡公孫度置帶方郡杜佑曰建安中公孫康分屯有有鹽縣以南荒地置帶方郡】於是賈謐郭彰權埶愈盛賓客盈門謐雖驕奢而好學喜延士大夫【好呼到翻喜許記翻】郭彰石崇陸機機弟雲和郁及滎陽潘岳【武帝泰始二年分河南置滎陽郡】清河崔基【齊大夫崔氏之後】勃海歐陽建【姓譜越王句踐之後封於烏程歐陽子孫因以為氏】蘭陵繆徵【是年分東海置蘭陵郡】京兆杜斌摰虞【按毛詩傳摯國出於任姓子孫以國為氏】琅邪諸葛詮【詮且緣翻】宏農王粹襄城杜育【武帝泰始二年分汝南置襄城郡】南陽鄒捷齊國左思沛國劉瓌周恢安平牽秀潁川陳聄高陽許猛【泰始元年分河問涿郡置高陽國瓌姑回翻眕止忍翻】彭城劉訥中山劉輿輿弟琨皆附於謐號曰二十四友郁嶠之弟也崇與岳尤謟事謐每候謐及廣城君郭槐出皆降車路左望塵而拜太宰亮太保瓘以楚王瑋剛愎好殺惡之【愎蒲逼翻好呼到翻】

  【惡烏路翻下同】欲奪其兵權以臨海侯裴楷代瑋為北軍中候瑋怒楷聞之不敢拜【不敢拜受中侯之職】亮復與瓘謀【復扶又翻下復矯同】遣瑋與諸王之國瑋益忿怨瑋長史公孫宏舍人岐盛【姓譜古有岐伯為黄帝師又周太王居岐山文王遷豐其支庶留岐者為岐氏】皆有寵於瑋勸瑋自昵於賈后【昵尼質翻】后留瑋領太子少傅盛素善於楊駿衛瓘惡其反覆將收之盛乃與宏謀因積弩將軍李肇【武帝泰始四年罷振威陽威護軍置左右積弩將軍一說晉太康中置積射積弩營營二千五百人並以將軍領之】矯稱瑋命譛亮瓘於賈后云將謀廢立后素怨瓘【以瓘撫牀事也見八十卷武帝咸康四年】且患二公執政已不得專恣夏六月后使帝作手詔賜瑋曰太宰太保欲為伊霍之事王宜宣詔令淮南長沙成都王屯諸宫門免亮及瓘官夜使黄門齎以授瑋瑋欲覆奏黄門曰事恐漏泄非密詔本意也瑋亦欲因此復私怨遂勒本軍【本軍瑋所掌北軍也】復矯詔召三十六軍【晉洛城内外三十六軍】告以二公潜圖不軌吾今受詔都督中外諸軍諸在直衛者皆嚴加警備其在外營便相帥徑詣行府助順討逆【帥讀曰率】又矯詔亮瓘官屬一無所問皆罷遣之若不奉詔便軍法從事遣公孫宏李肇以兵圍亮府侍中清河王遐收瓘亮帳下督李龍曰外有變請拒之【晉制諸公及諸大將軍皆置帳下督及門下督】亮不聽俄而兵登牆大呼【呼火故翻】亮驚曰吾無貳心何故至此詔書其可見乎宏等不許趣兵攻之【趣讀曰促】長史劉準謂亮曰觀此必是姦謀府中俊乂如林猶可力戰又不聽遂為肇所執歎曰我之赤心可破示天下也與世子矩俱死衛瓘左右亦疑遐矯詔請拒之須自表得報就戮未晩瓘不聽初瓘為司空【武帝太康三年瓘為司空永熙元年免】帳下督榮晦有罪【姓譜榮姓周榮公之後莊子有榮啟期】斥遣之至是晦從遐收瓘輒殺瓘及子孫共九人遐不能禁岐盛說瑋【說輸芮翻下同】宜因兵埶遂誅賈郭以正王室安天下瑋猶豫未决會天明太子少傅張華使董猛說賈后曰楚王既誅二公則天下威權盡歸之矣人主何以自安宜以瑋專殺之罪誅之賈后亦欲因此除瑋深然之是時内外擾亂朝廷恟懼不知所出【恟許勇翻】張華白帝遣殿中將軍王宫齎騶虞幡出麾衆曰楚王矯詔勿聽也【晉制有白虎幡騶虞幡白虎威猛主殺故以督戰騶虞仁獸故以解兵】衆皆釋仗而走瑋左右無復一人窘迫不知所為遂執之下廷尉【下遐稼翻】乙丑斬之瑋出懷中青紙詔流涕以示監刑尚書劉頌【監工銜翻】曰幸託體先帝而受枉乃如此乎公孫宏岐盛並夷三族瑋之起兵也隴西王泰嚴兵將助瑋【泰宣帝弟子】祭酒丁綏諫曰公為宰相【泰時為司空晉公府有西東閣祭酒】不可輕動且夜中倉猝宜遣人參審定問【問音問也定問猶言實音問也】泰乃止衛瓘女與國臣書曰先公名謚未顯每怪一國蔑然無言春秋之失其咎安在【春秋公羊傳曰春秋君弑賊不討以為無臣子也子沈子曰君弑臣不討賊非臣也子不復讐非子也謚神至翻】於是太保主簿劉繇等執黄幡撾登聞鼓【古者設諫鼓立謗木所以通下情也周禮太僕建路鼓於大寢之門外以待達窮者鄭司農注云窮謂窮寃失職者來擊此鼔以達於王若今時上變事擊鼓矣此則登聞鼔之始也登聞鼓之名蓋始於魏晉之間撾陟加翻擊也】上言曰初矯詔者至公即奉送章綬單車從命【綬音受】如矯詔之文唯免公官而故給使榮晦輒收公父子及孫一時斬戮乞驗盡情偽加以明刑乃詔族誅榮晦追復亮爵位謚曰文成封瓘為蘭陵郡公謚曰成於是賈后專朝【朝直遥翻下同】委任親黨以賈模為散騎常侍加侍中賈謐與后謀以張華庶姓無逼上之嫌【據杜預左傳注庶姓非同姓】而儒雅有籌略為衆望所依欲委以朝政疑未决以問裴頠頠贊成之【廣城君郭槐頠從母也故賈氏親信頠】乃以華為侍中中書監頠為侍中又以安南將軍裴楷為中書令加侍中與右僕射王戎並管機要華盡忠帝室彌縫遺闕賈后雖凶險猶知敬重華賈模與華頠同心輔政故數年之間雖闇主在上而朝野安靜華等之功也 秋七月分荆揚十郡為江州【是時方因江水之名置江州】 八月辛未立隴西王泰世子越為東海王 九月甲午秦獻王柬薨 辛丑徵征西大將軍梁王彤為衛將軍録尚書事【肜余中翻】

  二年春二月己酉故楊太后卒於金墉城是時太后尚有侍御十餘人賈后悉奪之絶膳八日而卒【卒子恤翻】賈后恐太后有靈或訴寃於先帝乃覆而殯之仍施諸厭劾符書藥物等【厭益涉翻伏也劾胡得翻治鬼曰劾】 秋八月壬子赦天下三年夏六月宏農雨雹深三尺【雨于具翻度深曰深音式禁翻】 鮮卑宇文莫槐為其下所殺弟普撥立 拓拔綽卒子弗立四年春正月丁酉安昌元公石鑒薨 【考異曰本傳鑒封昌安縣侯今從帝紀】 夏五月匈奴郝散反攻上黨殺長吏秋八月郝散帥衆降馮翊都尉殺之【郝呼各翻郝散若自上黨帥衆向洛陽歸降當入河内界今為馮翊都尉所殺蓋自穀遠歷河東界度河至馮翊界而被殺也帥讀曰率降戶江翻】是歲大饑司隸校尉傅咸卒 【考異曰三十國晉春秋元康四年七月傅咸為司隸五年五月始親職十月卒二書附年月多差舛故以本傳為定】咸性剛簡風格峻整初為司隸校尉上言貨賂流行所宜深絶時朝政寛弛權豪放恣咸奏免河南尹澹等官【澹河南尹之名音徒濫翻又徒覽翻】京師肅然慕容廆徙居大棘城【廆自徒河之青山徙大棘城杜佑曰棘城即帝顓頊之墟在營州郡城東南一百七十里】 拓抜弗卒叔父禄官立

  五年夏六月東海雨雹深五寸【雨于具翻深式浸翻】 荆揚兖豫青徐六州大水 冬十月武庫火 【考異曰三十國晉春秋云閏月宋志五行志閏月庚寅今從晉書帝紀】焚累代之寶【華傳曰趙王倫孫秀與華有隙疾華如讐武庫火華懼因此變作列兵固守然後救之故累代之寶及漢高祖斬蛇劒王莽頭孔子屐等盡焚焉據通鑑則倫秀之隙開於明年蓋數誅大臣禍皆從中起故華懼有變而列兵固守也】及二百萬人器械十二月丙戍新作武庫大調兵器【調徒釣翻】 拓抜禄官分其國為三部一居上谷之北濡源之西自統之【水經注濡水出禦夷鎮東南鎮拓拔魏太武時所置也師古曰濡音乃官翻】一居代郡參合陂之北【參合陂在代郡參合縣後漢晉省參合縣拓抜魏復置縣屬梁城郡】使兄沙漠汗之子猗㐌統之一居定襄之盛樂故城【定襄之盛樂二漢志曰成樂後漢志屬雲中郡魏晉省拓拔魏後置盛樂郡汗音寒㐌徒河翻】使猗㐌弟猗盧統之猗盧善用兵西擊匈奴烏桓諸部皆破之代人衛操與從子雄及同郡箕澹【姓譜箕商箕子之後又晉有大夫箕鄭父從才用翻】往依拓抜氏說猗也猗盧招納晉人猗㐌悦之任以國事晉人附者稍衆【史言拓抜氏益彊當是時晉朝大臣宗室雖已自相屠而四方未為變也衛操箕澹輩何為去華就夷如是其早計也中國之人可為凜凜矣漢嚴邉關之禁懼有罪者亡命出塞耳若無威刑之迫乎其後一旦去桑梓而逐水草是必有見也邉關不之詰朝廷不之虞晉之無政亦可知矣說輸芮翻下之說同】

  六年春正月赦天下 下邳獻王晃薨以中書監張華為司空太尉隴西王泰行尚書令徙封高密王 夏郝散弟度元與馮翊北地馬蘭羌盧水胡俱反【北地有馬蘭山羌居其中因為種落之名又按馬蘭山唐時屬同州界時蓋屬馮翊北地二郡界也盧水胡居安定界】殺北地太守張損敗馮翊太守歐陽建【敗補邁翻】征西大將軍趙王倫信用嬖人琅邪孫秀與雍州刺史濟南解系爭軍事更相表奏【嬖卑義翻又博計翻濟子禮翻解戶買翻姓也春秋晉有大夫解揚更工衡翻】歐陽建亦表倫罪惡朝廷以倫撓亂關右【撓火高翻撓也擾也又音擾又女巧翻又尼交翻又女敎翻】徵倫為車騎將軍以梁王肜為征西大將軍都督雍涼二州諸軍事系與其弟御史中丞結皆表請誅秀以謝氐羌張華以告梁王肜使誅之肜許諾秀友人辛冉為之說肜曰【為于偽翻】氐羌自反非秀之罪秀由是得免倫至洛陽用秀計深交賈郭賈后大愛信之【張華使梁王肜殺秀而不遂既至洛陽獨不能明正其罪而誅之邪】倫因求録尚書事又求尚書令張華裴頠固執以為不可倫秀由是怨之【為倫秀殺華頠系張本】秋八月解系為郝度元所敗【敗禰邁翻】秦雍氐羌悉反立氐帥齊萬年為帝圍涇陽【涇陽縣前漢屬安定郡後漢晉省賢曰涇陽故城在今原州平涼縣南帥所類翻】御史中丞周處彈劾不避權戚梁王肜嘗違法處按劾之【劾戶槩翻又戶得翻】冬十月詔以處為建威將軍與振威將軍盧播【沈約志振威將軍始於東漢之時宋登為之】俱隸安西將軍夏侯駿以討齊萬年中書令陳準言於朝曰駿及梁王皆貴戚非將帥之才【景懷皇后夏侯氏也故駿為外戚夏戶雅翻朝直遥翻】進不求名退不畏罪周處吳人忠直勇果有仇無援宜詔積弩將軍孟觀以精兵萬人為處前鋒必能殄寇不然梁王當使處先驅以不救而䧟之其敗必也朝廷不從齊萬年聞處來曰周府君嘗為新平太守【袁山松曰漢獻帝興平元年分安定之鶉觚右扶風之漆置新平郡唐為邠州】有文武才若專斷而來【斷丁亂翻】不可當也或受制於人此成禽耳 關中饑疫初略陽清水氐楊駒【略陽縣漢屬天水郡後漢改天水郡為漢陽郡獻帝初平四年】

  【分漢陽上郡置永陽郡魏改為廣魏郡武帝泰始中更名略陽郡清水縣前漢屬天水郡後漢志省晉志復見】始居仇池仇池方百頃其旁平地二十餘里四面斗絶而高為羊腸蟠道三十六回而上【仇池漢書地理志所謂天池大澤在武都郡武都縣西水經注所謂瞿塘者也賢曰仇池山在今成州上禄縣南三秦記曰仇池山在倉洛二谷之間常為水所衝激故下石而上土形似覆壺仇池記曰仇池百頃周回九千四十步天形四方壁立千仞自然樓櫓却敵分置調均竦起數丈有踰人力東西二門盤道下至上凡有七里上則岡阜昂泉流交灌煮水成鹽】至其孫千萬附魏封為百頃王千萬孫飛龍浸彊盛徙居略陽飛龍以其甥令狐茂搜為子茂搜避齊萬年之亂十二月自畧陽帥部落四千家還保仇池自號輔國將軍右賢王關中人士避亂者多依之茂搜迎接撫納欲去者衛護資送之【是後楊氏遂世據仇池帥讀曰率】 是歲以揚烈將軍巴西趙廞為益州刺史發梁益兵糧助雍州討氐羌【廞許今翻為趙廞亂蜀殺耿滕陳揌以啟巴氏張本】

  七年春正月齊萬年屯梁山有衆七萬【前漢志扶風好畤縣有梁山】梁王肜夏侯駿使周處以五千兵擊之處曰軍無後繼必敗不徒亡身為國取恥彤駿不聽逼遣之癸丑處與盧播解系攻萬年於六陌【六陌在馬嵬山西】處軍士未食肜促令速進自旦戰至暮斬獲甚衆弦絶矢盡救兵不至左右勸處退處按劒曰是吾効節致命之日也遂力戰而死朝廷雖以尤肜而亦不能罪也【尤過也】 秋七月雍秦二州大旱疾疫米斛萬錢 丁丑京陵元公王渾薨九月以尚書右僕射王戎為司徒太子太師何劭為尚書左僕射戎為三公與時浮沈無所匡救委事僚寀【寀此宰翻說文曰同官為僚同地為宷爾雅曰寀僚官】輕出遊放性復貪吝園田徧天下每自執牙籌晝夜會計常若不足【復扶又翻會古外翻】家有好李賣之恐人得種【種章勇翻】常鑚其核凡所賞抜專事虚名阮咸之子瞻嘗見戎戎問曰聖人貴名敎老莊明自然其旨同異瞻曰將無同【程大昌曰不直云同而云將無同者晉人語度自爾也庾亮辟孟嘉為從事正旦大會褚裒問嘉何在亮曰但自覔之裒歷觀指嘉曰將無是乎將無者猶言殆是此人也意以為是而未敢自主也阮瞻指孔老為同亦此意】戎咨嗟良久遂辟之時人謂之三語掾【掾于絹翻】是時王衍為尚書令南陽樂廣為河南尹皆善清談宅心事外【宅居也】名重當世朝野之人爭慕效之衍與弟澄好題品人物舉世以為儀凖【朝直遥翻好呼到翻】衍神情明秀少時山濤見之嗟歎良久曰何物老嫗生寧馨兒【少詩照翻楊正衡晉書音義嫗紆遇翻馨呼刑翻嫗老婦之稱今人傳讀寧如甯武子之甯洪邁隨筆曰今吳中人語尚多用寧馨字為言猶言若何也劉夢得詩為問中華學道者幾人雄猛得寧馨蓋得其義以寧字作平聲讀】然誤天下蒼生者未必非此人也樂廣性冲約與物無競每談論以約言析理厭人之心而其所不知默如也凡論人必先稱其所長則所短不言自見【厭於叶翻伏也見賢遍翻】王澄及阮咸咸從子脩泰山胡母輔之【母音無姓譜齊宣王封母弟於母鄉其鄉本胡國因曰胡母氏漢有太史胡母恭】陳國謝鯤城陽王【晉書作尼案古仁字又音夷王字孝孫或者當讀為仁字乎然永嘉三年書河内王尼即此王也晉書曰尼城陽人或云河内人若作尼則當音女夷翻】新蔡畢卓皆以任放為達【帝分汝隂置新蔡郡任者任物之自然放者縱其心而不制】至於醉狂裸體不以為非胡母輔之嘗酣飲其子謙之闚而厲聲呼其父字曰彦國年老不得為爾輔之歡笑呼入共飲畢卓嘗為吏部郎比舍郎釀熟【比毗寐翻近也】卓因醉夜至甕間盜飲之為掌酒者所縛明旦視之乃畢吏部也樂廣聞而笑之曰名敎内自有樂地【樂音洛】何必乃爾初何晏等祖述老莊立論以為天地萬物皆以無為本無也者開物成務【易繫辭曰夫易開物成務韓康伯注曰言易通萬物之志成天下之務張氏曰物凡物也務事也開明之也成處之也事無大小不能明則何由處矣楊萬里曰開達物理成就世務余謂何晏之旨以為事事物物自無而有無者物之未生事之未形見者也故曰無者開物成務與諸儒說易之旨不同】無往而不存者也隂陽恃以化生賢者恃以成德故無之為用無爵而貴矣王衍之徒皆愛重之由是朝廷士大夫皆以浮誕為美弛廢職業裴頠著崇有論以釋其蔽曰夫利欲可損而未可絶有也事務可節而未可全無也蓋有飾為高談之具者深列有形之累盛陳空無之美形器之累有徵【累力瑞翻】空無之義難檢辯巧之文可悦似象之言足惑衆聽眩焉溺其成說雖頗有異此心者辭不獲濟屈於所習【濟通也謂虛無習以成俗崇有者辭不能通其意遂為所屈也】因謂虛無之理誠不可蓋【蓋掩也】一唱百和【和戶卧翻】往而不反遂薄綜世之務賤功利之用高浮游之業卑經實之賢【經實謂有經世之實用者】人情所徇名利從之於是文者衍其辭訥者贊其旨立言藉於虛無謂之玄妙處官不親所職謂之雅遠【處昌呂翻】奉身散其亷操謂之曠達故砥礪之風彌以陵遲【砥礪謂砥節礪行也】放者因斯或悖吉凶之禮【悖蒲内翻】忽容止之表瀆長幼之序混貴賤之級甚者至于裸裎䙝慢無所不至【裸裎露體也祼郎果翻裎馳成翻】士行又虧矣【行下孟翻】夫萬物之有形者雖生于無然生以有為己分【物之未生則有無未分既生而有則與無為己分矣】則無是有之所遺者也【遺棄也】故養既化之有非無用之所能全也治既有之衆非無為之所能脩也【治直之翻】心非事也而制事必由於心然不可謂心為無也匠非器也而制器必須於匠然不可謂匠非有也是以欲收重淵之鱗非偃息之所能獲也【重直龍翻】隕高墉之禽非靜拱之所能捷也由此而觀濟有者皆有也虚無奚益於己有之羣生哉然習俗已成頠論亦不能救也 拓跋猗㐌度漠北廵因西略諸國【既度漠北遂西行略取諸國】積五歲降附者三十餘國【降戶江翻】

  八年春三月壬戍赦天下 秋九月荆豫徐揚冀五州大水 初張魯在漢中賨人李氏自巴西宕渠往依之【宕渠縣漢屬巴郡蜀先主分置宕渠郡晉屬巴西郡唐為渠州今渠州流江縣東北七十里有古賨國城李氏之先廩君之苗裔也世居巴中秦幷天下以為黔中郡薄賦斂之口歲出錢四十巴人呼賦為賨因謂之賨人焉又按晉志劉璋分巴郡墊江置巴西郡劉備割巴郡之宕渠宣漢漢昌三縣置宕渠郡尋省以縣並屬巴西郡則宕渠之屬巴西蓋晉時也賨徂宗翻宕徒浪翻】魏武帝克漢中【事見六十八卷漢獻帝建安二十年】李氏將五百餘家歸之拜為將軍遷于畧陽北土號曰巴氐【魏分臨渭平襄略陽清水四縣置廣魏郡晉泰始中更名畧陽郡】其孫特庠流皆有材武善騎射性任俠州黨多附之【俠戶頰翻】及齊萬年反關中荐饑【荐才甸翻爾雅仍饑為荐】略陽天水六郡民流移就穀入漢川者數萬家道路有疾病窮乏者特兄弟常營護振救之由是得衆心流民至漢中上書求寄食巴蜀朝議不許遣侍御史李苾持節慰勞【朝直遥翻苾蒲必翻又蒲蔑翻勞力到翻】且監察之【監古銜翻】不令入劒閣苾至漢中受流民賂表言流民十萬餘口非漢中一郡所能振贍蜀有倉儲人復豐稔【復扶又翻】宜令就食朝廷從之由是散在梁益不可禁止李特至劒閣太息曰劉禪有如此地面縛於人豈非庸才邪聞者異之【李特事始此 考異曰帝紀元康七年關中饑八年雍州有年而華陽國志三十國晉春秋皆云八年特就穀入蜀今從之】 張華陳準以趙王梁王相繼在關中皆雍容驕貴師老無功【雍容和緩自得之貌驕貴以貴而自驕也師久不决坐自困敝為老言二王不任軍事】乃薦孟觀沈毅有文武才用【沈持林翻】使討齊萬年觀身當矢石大戰十數皆破之

  資治通鑑卷八十二


    


 


 



 

 
  







 


  
  
 
 
 


  

 















	
	









































 
  



















 





 












  
  
  

 





