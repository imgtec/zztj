<!DOCTYPE html PUBLIC "-//W3C//DTD XHTML 1.0 Transitional//EN" "http://www.w3.org/TR/xhtml1/DTD/xhtml1-transitional.dtd">
<html xmlns="http://www.w3.org/1999/xhtml">
<head>
<meta http-equiv="Content-Type" content="text/html; charset=utf-8" />
<meta http-equiv="X-UA-Compatible" content="IE=Edge,chrome=1">
<title>資治通鑒_63-資治通鑑卷六十二_63-資治通鑑卷六十二</title>
<meta name="Keywords" content="資治通鑒_63-資治通鑑卷六十二_63-資治通鑑卷六十二">
<meta name="Description" content="資治通鑒_63-資治通鑑卷六十二_63-資治通鑑卷六十二">
<meta http-equiv="Cache-Control" content="no-transform" />
<meta http-equiv="Cache-Control" content="no-siteapp" />
<link href="/img/style.css" rel="stylesheet" type="text/css" />
<script src="/img/m.js?2020"></script> 
</head>
<body>
 <div class="ClassNavi">
<a  href="/24shi/">二十四史</a> | <a href="/SiKuQuanShu/">四库全书</a> | <a href="http://www.guoxuedashi.com/gjtsjc/"><font  color="#FF0000">古今图书集成</font></a> | <a href="/renwu/">历史人物</a> | <a href="/ShuoWenJieZi/"><font  color="#FF0000">说文解字</a></font> | <a href="/chengyu/">成语词典</a> | <a  target="_blank"  href="http://www.guoxuedashi.com/jgwhj/"><font  color="#FF0000">甲骨文合集</font></a> | <a href="/yzjwjc/"><font  color="#FF0000">殷周金文集成</font></a> | <a href="/xiangxingzi/"><font color="#0000FF">象形字典</font></a> | <a href="/13jing/"><font  color="#FF0000">十三经索引</font></a> | <a href="/zixing/"><font  color="#FF0000">字体转换器</font></a> | <a href="/zidian/xz/"><font color="#0000FF">篆书识别</font></a> | <a href="/jinfanyi/">近义反义词</a> | <a href="/duilian/">对联大全</a> | <a href="/jiapu/"><font  color="#0000FF">家谱族谱查询</font></a> | <a href="http://www.guoxuemi.com/hafo/" target="_blank" ><font color="#FF0000">哈佛古籍</font></a> 
</div>

 <!-- 头部导航开始 -->
<div class="w1180 head clearfix">
  <div class="head_logo l"><a title="国学大师官网" href="http://www.guoxuedashi.com" target="_blank"></a></div>
  <div class="head_sr l">
  <div id="head1">
  
  <a href="http://www.guoxuedashi.com/zidian/bujian/" target="_blank" ><img src="http://www.guoxuedashi.com/img/top1.gif" width="88" height="60" border="0" title="部件查字,支持20万汉字"></a>


<a href="http://www.guoxuedashi.com/help/yingpan.php" target="_blank"><img src="http://www.guoxuedashi.com/img/top230.gif" width="600" height="62" border="0" ></a>


  </div>
  <div id="head3"><a href="javascript:" onClick="javascript:window.external.AddFavorite(window.location.href,document.title);">添加收藏</a>
  <br><a href="/help/setie.php">搜索引擎</a>
  <br><a href="/help/zanzhu.php">赞助本站</a></div>
  <div id="head2">
 <a href="http://www.guoxuemi.com/" target="_blank"><img src="http://www.guoxuedashi.com/img/guoxuemi.gif" width="95" height="62" border="0" style="margin-left:2px;" title="国学迷"></a>
  

  </div>
</div>
  <div class="clear"></div>
  <div class="head_nav">
  <p><a href="/">首页</a> | <a href="/ShuKu/">国学书库</a> | <a href="/guji/">影印古籍</a> | <a href="/shici/">诗词宝典</a> | <a   href="/SiKuQuanShu/gxjx.php">精选</a> <b>|</b> <a href="/zidian/">汉语字典</a> | <a href="/hydcd/">汉语词典</a> | <a href="http://www.guoxuedashi.com/zidian/bujian/"><font  color="#CC0066">部件查字</font></a> | <a href="http://www.sfds.cn/"><font  color="#CC0066">书法大师</font></a> | <a href="/jgwhj/">甲骨文</a> <b>|</b> <a href="/b/4/"><font  color="#CC0066">解密</font></a> | <a href="/renwu/">历史人物</a> | <a href="/diangu/">历史典故</a> | <a href="/xingshi/">姓氏</a> | <a href="/minzu/">民族</a> <b>|</b> <a href="/mz/"><font  color="#CC0066">世界名著</font></a> | <a href="/download/">软件下载</a>
</p>
<p><a href="/b/"><font  color="#CC0066">历史</font></a> | <a href="http://skqs.guoxuedashi.com/" target="_blank">四库全书</a> |  <a href="http://www.guoxuedashi.com/search/" target="_blank"><font  color="#CC0066">全文检索</font></a> | <a href="http://www.guoxuedashi.com/shumu/">古籍书目</a> | <a   href="/24shi/">正史</a> <b>|</b> <a href="/chengyu/">成语词典</a> | <a href="/kangxi/" title="康熙字典">康熙字典</a> | <a href="/ShuoWenJieZi/">说文解字</a> | <a href="/zixing/yanbian/">字形演变</a> | <a href="/yzjwjc/">金 文</a> <b>|</b>  <a href="/shijian/nian-hao/">年号</a> | <a href="/diming/">历史地名</a> | <a href="/shijian/">历史事件</a> | <a href="/guanzhi/">官职</a> | <a href="/lishi/">知识</a> <b>|</b> <a href="/zhongyi/">中医中药</a> | <a href="http://www.guoxuedashi.com/forum/">留言反馈</a>
</p>
  </div>
</div>
<!-- 头部导航END --> 
<!-- 内容区开始 --> 
<div class="w1180 clearfix">
  <div class="info l">
   
<div class="clearfix" style="background:#f5faff;">
<script src='http://www.guoxuedashi.com/img/headersou.js'></script>

</div>
  <div class="info_tree"><a href="http://www.guoxuedashi.com">首页</a> > <a href="/SiKuQuanShu/fanti/">四库全书</a>
 > <h1>资治通鉴</h1> <!--         下载:【右键另存为】即可 --></div>
  <div class="info_content zj clearfix">
  
<div class="info_txt clearfix" id="show">
<center style="font-size:24px;">63-資治通鑑卷六十二</center>
    資治通鑑卷六十二   宋 司馬光 撰<br />
<br />
  胡三省 音註<br />
<br />
  漢紀五十四【起柔兆困敦盡著雍攝提格凡三年】<br />
<br />
  孝獻皇帝丁<br />
<br />
  建安元年春正月癸酉大赦改元 董承張楊欲以天子還雒陽楊奉李樂不欲由是諸將更相疑貳【更工衡翻下更有同】二月韓暹攻董承承奔野王【野王張楊所屯也暹息亷翻】韓暹屯聞喜胡才楊奉之塢鄉【郡國志河南緱氏縣西南冇塢聚】胡才欲攻韓暹上使人喻止之 汝南潁川黄巾何儀等擁衆附袁術曹操擊破之 張楊使董承先繕修雒陽宫太僕趙岐為承說劉表使遣兵詣雒陽助修宫室軍資委輸前後不絶【為于偽翻說輸芮翻委于偽翻流所聚曰委毛晃曰凡以物送之曰輸則音平聲指所送之物曰輸則音去聲委輸之委亦音去聲】夏五月丙寅帝遣使至揚奉李樂韓暹營求送至雒陽奉等從詔六月乙未車駕幸聞喜袁術攻劉備以争徐州備使司馬張飛守下邳自將拒術於盱眙淮陰【郡國志盱眙淮隂二縣屬下邳國盱眙音吁怡】相持經月更有勝負【更工衡翻】下邳相曹豹陶謙故將也與張飛相失飛殺之城中乖亂袁術與呂布書勸令襲下邳許助以軍糧布大喜引軍水陸東下【布去年奔備盖屯於下邳之西】備中郎將丹陽許耽開門迎之張飛敗走布虜備妻子及將吏家口備聞之引還比至下邳【比必寐翻下比明同】兵潰備收餘兵東取廣陵與袁術戰又敗屯於海西【前漢志海西縣屬東海郡續漢志屬廣陵郡 考異曰蜀志備傳於此云楊奉韓暹寇徐揚間備邀擊盡斬之按暹奉後與呂布同破袁術於時未死也備傳為誤】飢餓困踧【踧子六翻】吏士相食從事東海糜竺以家財助軍備請降於布【降戶江翻】布亦忿袁術運糧不繼乃召備復以為豫州刺史與并埶擊術使屯小沛【賢曰高祖本泗水郡沛縣人及得天下改泗水為沛郡小沛即沛縣宋白曰郡國志云古偪陽國漢為沛縣而沛郡理相城以沛縣為小沛 考異口備傳云遣關羽守下邳此在布敗後備傳誤也】布自稱徐州牧布將河内郝萌夜攻布布科頭袒衣走詣都督高順營【科頭不冠露髻也今江東人猶謂露髻為科頭】順即嚴兵入府討之萌敗走比明萌將曹性擊斬萌 庚子楊奉韓暹奉帝東還張楊以糧迎道路秋七月甲子車駕至雒陽幸故中常侍趙忠宅丁丑大赦八月辛丑幸南宫楊安殿張楊以為已功故名其殿曰楊安楊謂諸將曰天子當與天下共之朝廷自有公卿大臣楊當出扞外難【難乃旦翻】遂還野王楊奉亦出屯梁【郡國志梁縣屬河南尹春秋之梁國也】韓暹董承並留宿衛癸卯以安國將軍張楊為大司馬楊奉為車騎將軍韓暹為大將軍領司隸校尉皆假節鉞是時宫室燒盡百官披荆棘依墻壁間州郡各擁強兵委輸不至羣僚飢乏尚書郎以下自出採稆【續漢志尚書侍郎三十六人四百石本注曰一曹有六人主作文書起草蔡質漢儀曰尚書郎初從三署詣臺試初上臺稱守尚書郎中滿歲稱尚書郎三年稱侍郎賢曰稆音呂埤蒼曰穭自生也稆與穭同】或飢死墻壁間或為兵士所殺 袁術以䜟言代漢者當塗高自云名字應之【賢曰當塗高者魏也然術自以術及路皆是塗故云應之】又以袁氏出陳為舜後以黄代赤德運之次【賢曰陳大夫轅濤塗袁氏其後也五行火生土故云以黄代赤】遂有僭逆之謀聞孫堅得傳國璽【事見五十九卷初平元年】拘堅妻而奪之及聞天子敗於曹陽【事見上卷興平二年】乃會羣下議稱尊號衆莫敢對主簿閻象進曰昔周自后稷至于文王積德累功參分天下有其二猶服事殷【國語曰后稷勤周十五代而王毛詩國風序口國君積行累功以致爵位語孔子曰三分天下有其二以服事殷】明公雖奕世克昌未若有周之盛漢室雖微未若殷紂之暴也術默然術聘處士張範【處昌呂翻】範不往使其弟承謝之術謂承曰孤以土地之廣士民之衆欲徼福齊桓擬迹高祖何如【徼一遥翻】承曰在德不在彊夫用德以同天下之欲雖由匹夫之資而興霸王之功不足為難若苟欲僭擬于時而動衆之所棄誰能興之術不悦孫策聞之與術書曰成湯討桀稱有夏多罪【尚書湯誓曰有夏多罪天命殛之夏戶雅翻】武王伐紂曰殷有重罰【史記武王徧告諸侯曰殷有重罰不可不伐】此二主者雖有聖德假使時無失道之過無由逼而取也今主上非有惡於天下徒以幼小脅於彊臣異於湯武之時也且董卓貪淫驕陵志無紀極至於廢主自興亦猶未也而天下同心疾之况效尤而甚焉者乎【左傳曰尤而效之罪又甚焉】又聞幼主明智聰敏有夙成之德【夙早也】天下雖未被其恩咸歸心焉使君五世相承【賢曰安生京京生湯湯生逢逢生術凡五代被皮義翻】為漢宰輔榮寵之盛莫與為比宜效忠守節以報王室則旦奭之美率土所望也時人多惑圖緯之言妄牽非類之文苟以悦主為美不顧成敗之計古今所慎可不孰慮【孰與熟通】忠言逆耳【前書張良曰忠言逆耳利於行】駁議致憎【賢曰駁雜也議不同也言以持異議致憎疾也駁北角翻】苟有益於尊明無所敢辭術始自以為有淮南之衆料策必與己合及得其書愁沮發疾【沮在呂翻】既不納其言策遂與之絶曹操在許【郡國志許縣屬潁川郡帝既徙都改曰許昌杜佑曰漢許昌故城在今縣南三十里】<br />
<br />
  【宋白曰在今縣西南四十里】謀迎天子衆以為山東未定韓暹楊奉負功恣睢未可卒制【睢香萃翻恣睢暴戾之貌卒讀曰猝】荀彧曰昔晉文公納周襄王而諸侯景從【賢曰左傳狐偃言於晉侯曰求諸侯莫如勤王諸侯信之且大義也晉侯以左師逆王王入于王城取太叔于温殺之于隰城遂定覇業天下服從師古曰景從言如景之從形也】漢高祖為義帝縞素而天下歸心【事見九卷高祖二年為于偽翻】自天子蒙塵【蒙冒也言播越在草莽蒙冒塵埃也】將軍首唱義兵徒以山東擾亂未遑遠赴今鑾駕旋軫【鄭玄注周禮曰軫車後横木也】東京榛蕪義士有存本之思兆民懷感舊之哀誠因此時奉主上以從人望大順也秉至公以服天下大畧也扶弘義以致英俊大德也四方雖有逆節其何能為韓暹楊奉安足恤哉若不時定使豪傑生心後雖為慮亦無及矣操乃遣揚武中郎將曹洪將兵西迎天子【西漢有中郎將東漢分置三署虎賁羽林中郎將建安之後羣雄兵争自相署置始有名號中郎將】董承等據險拒之洪不得進 【考異曰魏志此事在正月而荀彧傳迎天子在都雒後今從傳】議郎董昭以楊奉兵馬最彊而少黨援【少詩沼翻】作操書與奉曰吾與將軍聞名慕義便推赤心今將軍拔萬乘之艱難反之舊都【乘繩證翻】翼佐之功超世無疇何其休哉方今羣凶猾夏【孔安國曰猾亂也夏華夏夏戶雅翻】四海未寧神器至重事在維輔必須衆賢以清王軌誠非一人所能獨建心腹四支實相恃賴一物不備則有闕焉將軍當為内主吾為外援今吾有糧將軍有兵有無相通足以相濟死生契濶相與共之【毛萇曰契濶勤苦也此盖謂死也生也處勤苦之中相與共之也契苦結翻】奉得書喜悦語諸將軍曰兖州諸軍近在許耳有兵有粮國家所當依仰也【語牛倨翻仰牛向翻】遂共表操為鎮東將軍襲父爵費亭侯【操祖曹騰封費亭侯養子嵩襲爵今以操襲嵩爵也郡國志沛國酇縣有費亭曹騰所封也應劭曰酇音嵯師古曰王莽改酇曰贊治則此縣亦有贊音晉地道記山陽郡湖陸縣西有費亭城魏武帝初所封考異曰魏志在六月而董昭傳在都雒後今從傳】韓暹矜功專恣董承患之因濳召操操乃將兵詣雒陽既至奏韓暹張楊之罪暹懼誅單騎奔楊奉帝以暹楊有翼車駕之功詔一切勿問辛亥以曹操領司隸校尉録尚書事操于是誅尚書馮碩等三人討有罪也【袁宏紀曰誅碩及議郎侯祈侍中壺崇】封衛將軍董承等十三人為列侯賞有功也【袁宏紀曰封衛將軍董承輔國將軍伏完侍中丁种輔尚書僕射鍾繇尚書郭溥御史中丞董芬彭城相劉艾馮翊韓斌東郡太守楊衆議郎羅邵伏德趙蕤為列侯】贈射聲校尉沮儁為弘農太守矜死節也【沮儁死事見上卷興平二年沮子余翻】操引董昭並坐問曰今孤來此當施何計昭曰將軍興義兵以誅暴亂入朝天子輔翼王室此五覇之功也此下諸將人殊意異未必服從今留匡弼事埶不便惟有移駕幸許耳然朝廷播越新還舊京遠近跂望【跂渠宜翻舉足也】冀一朝獲安今復徙駕不厭衆心【復扶又翻厭於叶翻又如字】夫行非常之事乃有非常之功願將軍算其多者【凡舉事有利亦有害惟算其利多而害少者行之】操曰此本志也楊奉近在梁耳聞其兵精得無為累乎【累力偽翻下同】昭曰奉少黨援心相憑結鎮東費亭之事皆奉所定宜時遣使厚遺答謝以安其意【少詩沼翻遺于季翻】說京都無糧欲車駕暫幸魯陽【魯陽縣屬南陽郡】魯陽近許轉運稍易【近其靳翻易以豉翻】可無縣乏之憂【縣讀曰懸】奉為人勇而寡慮必不見疑比使往來【比必寐翻使疏吏翻】足以定計奉何能為累操曰善即遣使詣奉庚申車駕出轘轅而東【河南緱氏縣有轘轅關轘音環】遂遷都許巳已幸曹操營以操為大將軍封武平侯【武平縣屬陳國此取其以神武平禍亂也宋白曰亳州鹿邑縣後漢於今縣東北置武平縣隋改為鹿邑取故鹿邑城為名其古鹿邑城在縣西十三里春秋鹿鳴地也】始立宗廟社稷於許 孫策將取會稽【會工外翻】吳人嚴白虎等衆各萬餘人處處屯聚諸將欲先擊白虎等策曰白虎等羣盜非有大志此成禽耳遂引兵渡浙江【浙之舌翻】會稽功曹虞翻說太守王朗曰策善用兵不如避之朗不從發兵拒策於固陵策數渡水戰不能克策叔父靜說策曰朗阻城守難可卒拔查瀆南去此數十里宜從彼據其内【說輸芮翻數所角翻卒讀曰猝水經注浙江東逕固陵城北昔范蠡築城於浙江之濱言可以固守謂之固陵今之西陵也浙江又東逕柤塘謂之柤瀆孫策襲王朗所從出之道也裴松之曰查音柤加翻】所謂攻其無備出其不意者也策從之夜多火為疑兵分軍投查瀆道襲高遷屯【裴松之曰案今永興縣有高遷橋沈約曰永興本漢餘暨縣吳更名蔡邕嘗經會稽高遷亭取椽竹以為笛即其處也】朗大驚遣故丹陽太守周昕等帥兵逆戰【帥讀曰率】策破昕等斬之朗遁走虞翻追隨營護朗浮海至東冶【前漢志冶縣屬會稽郡師古曰故閩越地光武改曰章安晉志曰建安郡故秦閩中郡漢高祖五年以立閩越王及武帝滅之徙其人名為東冶後漢改為候官都尉及吳置建安郡洪氏隸釋據西漢志曰會稽西部都尉治錢唐南部都尉治回浦李宗諤圖經曰文帝時以山隂為都尉治元狩中徙治錢唐為西部元鼎中又立東部都尉治冶光武改回浦為章安以冶立東候官吳孫亮傳曰五鳳中以會稽東部為臨海郡孫休傳永安中以會稽南部為建安郡沈約宋志曰東陽太守本會稽西部都尉又曰臨海太守本會稽東部都尉前漢都尉治鄞後漢分會稽為吳郡疑是都尉徙治章安續漢志章安故冶光武更名晉太康記本鄞縣南之回浦鄉章帝立未詳孰是又曰司馬彪云章安是故冶然則臨海亦冶地也張勃吳録曰是句踐冶鑄之所後分為會稽東南二部都尉東部臨海是也南部建安是也杜佑通典曰後漢改冶縣為候官都尉後分冶縣為會稽東南二都尉今福州是南部台州是東部又曰二漢會稽西部都尉理婺州數說異同各有脱誤嘗參訂之自秦置會稽郡其治在今吳門至順帝分置吳郡而會稽徙郡於山隂以浙江為兩郡之境故錢唐在西漢時屬會稽所以為西部治所及會稽移於浙東則西部亦移於婺女回浦後改章安乃會稽之東部今台州蓋其地冶縣則是南部在吳屬建安郡至唐遂為福州太康記嘗云回浦本鄞之南鄉或云東部治鄞因致休文之疑然鄞及回浦皆西漢縣名謂西漢割鄞而置縣或未可知至章帝時回浦已非鄉矣太康所紀亦誤也前志註會稽之冶縣云本閩越地續志口章安故冶閩越地光武更名因脱其中數字故劉昭補注惑於太康記而休文復不能剖判也當云章安故回浦章帝更名東侯官故冶閩越地光武更名於文乃足此郡之末有東部侯國四字却是衍文侯與候相近而南部所治故文有錯亂班史註回浦為南部司馬彪謂章安是故冶張勃謂分冶為東南二都尉杜佑謂二漢西部皆在婺女圖經以冶為東部皆誤也余按洪說甚詳其言錢唐西漢時屬會稽所以為西部治所此語亦恐有未安處】策追擊大破之朗乃詣策降【降戶江翻】策自領會稽太守復命虞翻為功曹待以交友之禮策好游獵【好呼到翻】翻諫曰明府喜輕出微行從官不暇嚴【喜許記翻從才用翻】吏卒常苦之夫君人者不重則不威【重尊重威威嚴言不尊重則無威嚴】故白龍魚服困於豫且【張衡東京賦之辭注云說苑曰吳王欲從民飲酒伍子胥諫曰不可昔白龍下凊泠之淵化為魚漁者豫且射中其目白龍上訴天帝曰當是之時若安置而形白龍對曰我下清泠之淵化為魚天帝曰魚固人之所射也豫且何罪夫白龍天帝貴畜也豫且宋國之賤臣也白龍不化豫且不射今棄萬乘之位而從布衣之士飲酒臣恐其有豫且之患矣王乃止且子余翻】白蛇自放劉季害之【事見七卷秦二世元年】願少留意【少詩沼翻】策曰君言是也然不能改【為策死于輕出張本】 九月司徒淳于嘉太尉楊彪司空張喜皆罷 車駕之東遷也楊奉自梁欲邀之不及冬十月曹操征奉奉南犇袁術遂攻其梁屯拔之 詔書下袁紹責以地廣兵多而專自樹黨【下遐稼翻下之下同樹黨謂以子譚為青州刺史熙為幽州刺史外甥高幹為幷州刺史】不聞勤王之師但擅相討伐【謂與公孫瓚相攻也】紹上書深自陳愬戊辰以紹為太尉封鄴侯紹恥班在曹操下怒曰曹操當死數矣【數所角翻下同】我輒救存之【操自滎陽汴水之敗收兵從紹於河内紹表為東郡太守呂布襲取兖州紹復與操連和欲令其遣家居鄴也】今乃挾天子以令我乎表辭不受操懼請以大將軍讓紹丙戍以操為司空行車騎將軍事操以荀彧為侍中守尚書令操問彧以策謀之士彧薦其從子蜀郡太守攸【攸既免董卓之禍復辟公府舉高第遷任城相不行以蜀險固人民殷盛求為蜀郡太守道絶不得至駐荆州從才用翻下同】及潁川郭嘉操徵攸為尚書與語大悦曰公逹非常人也【荀攸字公逹】吾得與之計事天下當何憂哉以為軍師初郭嘉往見袁紹紹甚敬禮之居數十日謂紹謀臣辛評郭圖曰夫智者審於量主【量音良】故百全而功名可立袁公徒欲效周公之下士而不知用人之機多端寡要好謀無決欲與共濟天下大難定覇王之業難矣【好呼到翻大難乃旦翻】吾將更舉而求主【更工衡翻改也】子盍去乎二人曰袁氏有恩德於天下人多歸之且今最強去將何之嘉知其不寤不復言【復扶又翻】遂去之操召見與論天下事喜曰使孤成大業者必此人也嘉出亦喜曰真吾主也操表嘉為司空祭酒【陳夀三國志作司空軍祭酒此逸軍字晉志曰當塗得志剋平諸夏初置軍師祭酒参掌戎律】操以山陽滿寵為許令操從弟洪有賓客在許界數犯法寵收治之洪書報寵【報告也前書霍顯曰少夫幸報我以事數所角翻治直之翻】寵不聼洪以白操操召許主者【主者許縣主吏也】寵知將欲原客乃速殺之操喜曰當事不當爾邪北海太守孔融負其高氣志在靖難而才踈意廣訖無成功【訖竟也終也難乃旦翻】高談清教盈溢官曹辭氣清雅可玩而誦論事考實難可悉行但能張磔網羅【磔陟格翻開也】而目理甚踈造次能得人心【造七到翻】久久亦不願附也其所任用好奇取異多剽輕小才【好呼到翻剽匹妙翻輕墟正翻】至於尊事名儒鄭玄執子孫禮易其鄉名曰鄭公鄉【玄傳曰融深敬玄告高密縣為玄特立一鄉曰昔齊置士鄉越有君子軍皆異賢之意也太史公廷尉吳公謁者僕射鄧公皆漢之名臣又南山四皓有園公夏黄公世嘉其高皆悉稱公然則公者仁德之正號不必皆三事大夫也今鄭君鄉宜曰鄭公鄉】及清儁之士左承祖劉義遜等皆備在座席而已不與論政事曰此民望不可失也黄巾來寇融戰敗走保都昌【賢曰都昌縣屬北海郡故城在今青州臨朐縣東北】時袁曹公孫首尾相連融兵弱糧寡孤立一隅不與相通左承祖勸融宜自託強國融不聼而殺之劉義遜弃去青州刺史袁譚攻融自春至夏戰士所餘纔數百人流矢交集而融猶隱几讀書談笑自若【隱於靳翻賢曰隱憑也】城夜陷乃奔東山【都昌縣之東山也】妻子為譚所虜曹操與融有舊徵為將作大匠袁譚初至青州其土自河而西不過平原譚北排田楷【田楷公孫瓚用為青州刺史】東破孔融威惠甚著其後信任羣小肆志奢淫聲望遂衰 中平以來天下亂離民弃農業諸軍並起率乏糧穀無終歲之計飢則寇掠飽則弃餘瓦解流離無敵自破者不可勝數【勝音升】袁紹在河北軍人仰食桑椹【仰牛向翻椹桑實也其始生也色青熟則色黑可食椹音甚】袁術在江淮取給蒲蠃【蠃蚌屬盧戈翻】民多相食州里蕭條羽林監棗袛請建置屯田【潁川文士傳棗氏本姓棘避難改焉漢官羽林有左右監秩六百石屬光禄勲】曹操從之以袛為屯田都尉以騎都尉任峻為典農中郎將【魏志曰曹公置典農中郎將秩二千石典農都尉秩六百石或四百石典農校尉秩比二千石所主如中郎所主部分别而少為校尉】募民屯田許下得穀百萬斛于是州郡例置田官所在積穀倉廪皆滿故操征伐四方無運糧之勞遂能兼并羣雄軍國之饒起於袛而成於峻 袁術畏呂布為己害乃為子求婚布復許之【乃為于偽翻復扶又翻】術遣將紀靈等步騎三萬攻劉備備求救於布諸將謂布曰將軍常欲殺劉備今可假手於術布曰不然術若破備則北連泰山諸將【泰山諸將謂臧霸孫觀吳敦尹禮輩】吾為在術圍中不得不救也便率步騎千餘馳往赴之靈等聞布至皆歛兵而止布屯沛城西南遣鈴下請靈等【鈴下卒也在鈴閣之下有警至則掣鈴以呼之因以為名續漢志曰五百鈴下侍閭門闌部署街里走卒皆冇程品多少隨所典領程大昌續演繁露曰鈴下威儀殆今典客之吏】靈等亦請布布往就之與備共飲食布謂靈等曰玄德布弟也【劉備字玄德】為諸君所困故來救之布性不喜合鬭喜解鬭耳【言不喜合人之鬭喜解人之鬬也喜許記翻】乃令軍候植戟於營門布彎弓顧曰諸君觀布射戟小支【賢曰周禮考工記曰為戟博二寸内倍之胡參之援四之鄭注云援直刃胡其孑也小支謂胡也即今之戟旁曲支植直吏翻立也射而亦翻】中者當各解兵不中可留决鬭布即一發正中戟支【中竹仲翻下同】靈等皆驚言將軍天威也明日復歡會然後各罷備合兵得萬餘人布惡之【復扶又翻惡烏路翻】自出兵攻備備敗走歸曹操操厚遇之以為豫州牧或謂操日備有英雄之志今不早圖後必為患操以問郭嘉嘉曰有是然公起義兵為百姓除暴【為于偽翻】推誠仗信以招俊傑猶懼其未也今備有英雄名以窮歸已而害之是以害賢為名也如此則智士將自疑回心擇主公誰與定天下乎夫除一人之患以沮四海之望安危之機也不可不察【沮在呂翻 考異曰傳子以為程昱郭嘉勸操殺備今從魏書】操笑曰君得之矣遂益其兵給糧食使東至沛收散兵以圖呂布初備在豫州舉陳郡袁渙為茂才【武帝元封六年詔州郡舉茂才茂才即秀才也避光武諱史遂書為茂才】渙為呂布所留布欲使渙作書罵辱備渙不可再三彊之不許【彊其兩翻】布大怒以兵脇渙曰為之則生不為則死渙顔色不變笑而應之曰渙聞唯德可以辱人不聞以罵使彼固君子邪且不恥將軍之言彼誠小人邪將復將軍之意【言布以書罵備備君子邪固不以罵為耻其小人邪將復以書罵布也】則辱在此不在於彼且渙它日之事劉將軍猶今日之事將軍也如一旦去此復罵將軍可乎【復扶又翻】布慚而止 張濟自關中引兵入荆州界攻穰城【穰縣屬南陽郡】為流矢所中死【中竹仲翻】荆州官屬皆賀劉表曰濟以窮來主人無禮【言無郊勞授館之禮也】至于交鋒此非牧意牧受弔不受賀也使人納其衆衆聞之喜皆歸心焉濟族子建忠將軍繡代領其衆屯宛【宛於元翻】初帝既出長安宣威將軍賈詡上還印綬【上時掌翻】往依段煨于華隂【華戶化翻】詡素知名為煨軍所望煨禮奉甚備詡潜謀歸張繡或曰煨待君厚矣君去安之詡曰煨性多疑有忌詡意禮雖厚不可恃久將為所圖【詡既為煨軍所望則必為煨所忌矣久留則煨懼詡奪其軍必將圖殺之】我去必喜又望吾結大援於外必厚吾妻子繡無謀主亦願得詡則家與身必俱全矣詡遂往繡執子孫禮煨果善視其家詡說繡附於劉表【說輸芮翻】繡從之詡往見表表以客禮待之詡曰表平世三公才也不見事變多疑無决無能為也劉表愛民養士從容自保【從千容翻】境内無事關西兖豫學士歸之者以千數表乃起立學校講明經術【校戶敎翻】命故雅樂郎河南杜夔作雅樂【蔡邕曰漢樂四品一曰太子樂典郊廟上陵殿舉之樂二曰周頌雅樂典辟雍饗射六宗社稷之樂三日黄門鼓吹天子所以宴樂羣臣四曰短簫鐃歌軍樂也】樂備表欲庭觀之夔曰今將軍號不為天子合樂而庭作之無乃不可乎表乃止平原禰衡少有才辯而尚氣剛傲【禰乃禮翻姓也少詩照翻】孔融薦之於曹操衡罵辱操【操召衡為鼓吏故為衡所罵辱】操怒謂融曰禰衡豎子孤殺之猶雀鼠耳顧此人素有虛名遠近將謂孤不能容之乃送與劉表表延禮以為上賓衡稱表之美盈口而好譏貶其左右【好呼到翻】於是左右因形而譖之曰衡稱將軍之仁西伯不過也唯以為不能斷【斷丁亂翻】終不濟者必由此也其言實指表短而非衡所言也表由是怒以江夏太守黄祖性急送衡與之祖亦善待焉後衡衆辱祖祖殺之【操怒衡而送與表猶以表為寛和愛士觀其能容與否也表怒衡而送與祖知祖性急必不能容衡是直欲寘之死地耳二人皆挾數用術表則淺矣】<br />
<br />
  二年春正月曹操討張繡軍于淯水【水經注淯水出弘農盧氏縣攻離山東逕宛縣南操軍敗處也淯音育】繡舉衆降操納張濟之妻繡恨之又以金與繡驍將胡車兒繡聞而疑懼襲擊操軍殺操長子昂操中流矢敗走【降戶江翻驍堅堯翻車尺遮翻長知兩翻中竹仲翻】校尉典韋與繡力戰左右死傷略盡韋被數十創【被皮義翻創初良翻】繡兵前摶之韋雙挾兩人擊殺之瞋目大罵而死【瞋七人翻】操收散兵還住舞隂【舞隂縣屬南陽郡】繡率騎來追操擊破之繡走還穰復與劉表合【復扶又翻】是時諸軍大亂平虜校尉泰山于禁獨整衆而還道逢青州兵劫掠人禁數其辠而擊之【數所具翻】青州兵走詣操禁既至先立營壘不時謁操或謂禁青州兵已訴君矣宜促詣公辨之禁曰今賊在後追至無時不先為備何以待敵且公聰明譖訴何緣得行徐鑿塹安營訖【塹七艷翻】乃入謁具陳其狀操悦謂禁曰淯水之難【難乃旦翻】吾猶狼狽將軍在亂能整討暴堅壘【討暴謂擊刼掠者堅壘謂先鑿塹安營也】有不可動之節雖古名將何以加之于是録禁前後功封益壽亭侯操引軍還許 袁紹與操書辭語驕慢操謂荀彧郭嘉曰今將討不義而力不敵何如對曰劉項之不敵公所知也漢祖惟智勝項羽故羽雖彊終為所禽今紹有十敗公有十勝紹雖彊無能為也紹繁禮多儀公體任自然此道勝也紹以逆動公奉順以率天下【謂奉天子以率天下於理為順】此義勝也桓靈以來政失於寛紹以寛濟寛故不攝【攝整也左傳曰書於伐秦攝也杜預注曰能自攝整】公糾之以猛上下知制此治勝也【治直吏翻】紹外寛内忌用人而疑之所任唯親戚子弟公外易簡而内機明【易以豉翻】用人無疑唯才所宜不間遠近此度勝也【閒古莧翻】紹多謀少决失在後事公得策輒行應變無窮此謀勝也紹高議揖讓以收名譽士之好言飾外者多歸之【好呼到翻下同】公以至心待人不為虚美士之忠正遠見而有實者皆願為用此德勝也紹見人飢寒恤念之形於顔色其所不見慮或不及公於目前小事時有所忽至於大事與四海接恩之所加皆過其望雖所不見慮無不周此仁勝也紹大臣爭權讒言惑亂公御下以道浸潤不行此明勝也【論語浸潤之譛不行焉可謂明也已矣言譛人者如水之浸潤以漸而入也】紹是非不可知公所是進之以禮所不是正之以法此文勝也紹好為虚埶不知兵要【荀子與臨武君議兵於趙孝成王前王曰請問兵要】公以少克衆【少詩沼翻】用兵如神軍人恃之敵人畏之此武勝也操笑曰如卿所言孤何德以堪之嘉又曰紹方北擊公孫瓚【瓚藏旱翻】可因其遠征東取呂布若紹為寇布為之援此深害也彧曰不先取呂布河北未易圖也【紹攻公孫瓚而操乘間東取呂布操擊劉備而紹不能襲許此其所以敗也易以豉翻】操曰然吾所惑者又恐紹侵擾關中西亂羌胡南誘蜀漢【誘音酉】是我獨以兖豫抗天下六分之五也為將奈何彧曰關中將帥以十數【將即亮翻帥所類翻】莫能相一唯韓遂馬騰最彊彼見山東方爭必各擁衆自保今若撫以恩德遣使連和雖不能久安比公安定山東足以不動【遂騰之叛服卒如荀彧所料比必寐翻】侍中尚書僕射鍾繇有智謀若屬以西事【屬之欲翻】公無憂矣操乃表繇以侍中守司隸校尉持節督關中諸軍特使不拘科制繇至長安移書騰遂等【移猶遺也】為陳禍福【為于偽翻】騰遂各遣子入侍 袁術稱帝於壽春自稱仲家以九江太守為淮南尹置公卿百官郊祀天地沛相陳珪球弟子也少與術遊術以書召珪又劫質其子【少詩照翻質音致】期必致珪珪荅書曰曹將軍興復典刑將撥平凶慝以為足下當僇力同心匡翼漢室而隂謀不軌以身試禍欲吾營私阿附有死不能也術欲以故兖州刺史金尚為太尉尚不許而逃去術殺之【金尚奔術見六十卷初平三年】 三月詔將作大匠孔融持節拜袁紹大將軍兼督冀青幽并四州 夏五月蝗 袁術遣使者韓胤以稱帝事告呂布因求迎婦布遣女隨之陳珪恐徐揚合從為難未已【術領揚州布領徐州從子容翻難乃旦翻】往說布曰【說輸芮翻】曹公奉迎天子輔贊國政將軍宜與協同策謀共存大計今與袁術結昏必受不義之名將有累卵之危矣布亦怨術初不已受也【事見六十卷初平三年】女已在塗乃追還絶昏械送韓胤梟首許市【梟堅堯翻】陳珪欲使子登詣曹操布固不肯會詔以布為左將軍操復遺布手書深加尉納【復扶又翻遺于季翻尉與慰同安之也漢書車千秋傳尉安黎庶顔師古口尉安之字本無心】布大喜即遣登奉章謝恩并荅操書登見操因陳布勇而無謀輕於去就宜早圖之操曰布狼子野心誠難久養非卿莫䆒其情偽即增珪秩中二千石【漢制王國相秩二十石增秩中二千石則秩視九卿】拜登廣陵太守臨别操執登手曰東方之事便以相付令隂合部衆以為内應始布因登求徐州牧不得登還布怒拔戟斫几曰卿父勸吾協同曹操絶昏公路今吾所求無獲而卿父子並顯重但為卿所賣耳登不為動容【為于偽翻】徐對之曰登見曹公言養將軍譬如養虎當飽其肉不飽則將噬人公曰不如卿言譬如養鷹飢即為用飽則颺去其言如此布意乃解袁術遣其大將張勲橋蕤等與韓暹楊奉連埶步騎數萬趣下邳【趣七喻翻】七道攻布布時有兵三千馬四百匹懼其不敵謂陳珪曰今致術軍卿之由也為之奈何珪曰暹奉與術卒合之師耳【卒讀曰猝】謀無素定不能相維子登策之比於連雞埶不俱棲【戰國策秦惠王曰諸侯之不可一猶連鷄之不能俱上於棲】立可離也布用珪策與暹奉書曰二將軍親拔大駕而布手殺董卓俱立功名今奈何與袁術同為賊乎不如相與并力破術為國除害【為于偽翻】且許悉以術軍資與之暹奉大喜即回計從布布進軍去勲營百步暹奉兵同時叫呼【呼火故翻】並到勲營勲等散走布兵追擊斬其將十人首所殺傷墮水死者殆盡布因與暹奉合軍向壽春水陸並進到鍾離【鍾離縣屬九江郡距壽春二百餘里】所過虜掠還渡淮北留書辱術術自將步騎五千揚兵淮上布騎皆於水北大咍笑之而還【咍呼來翻楚人謂相啁笑曰咍】泰山賊帥臧覇襲琅邪相蕭建於莒【前漢莒縣屬城陽國後漢屬琅邪國帥所類翻】破之覇得建資實許以賂布而未送布自往求之其督將高順諫曰【將即亮翻下所將順將同】將軍威名宣播遠近所畏何求不得而自行求賂萬一不克豈不損邪布不從既至莒覇等不測往意固守拒之無獲而還順為人清白有威嚴少言辭所將七萬餘兵號令整齊每戰必克名陷陳營【少詩沼翻陳讀曰陣】布復疎順以魏續有内外之親奪其兵以與續及當攻戰則復令順將順亦終無恨意【布踈順而親續其後執順以敗布者續也將即亮翻】布性决易【易以豉翻】所為無常順每諫曰將軍舉動不肯詳思忽有失得動輒言誤誤豈可數乎【數所角翻】布知其忠而不能從 曹操遣議郎王誧【誧滂古翻又匹布翻】以詔書拜孫策為騎都尉襲爵烏程侯【策父堅以討賊功封烏程侯烏程縣屬吳郡今安吉州縣 考異曰江表傳曰建安二年夏王誧奉戊辰詔書賜策不知其何月也】領會稽太守【會工外翻】使與呂布及吳郡太守陳瑀共討袁術策欲得將軍號以自重誧便承制假策明漢將軍【明漢將軍亦權宜置此號言明於逆順知尊漢室也下輔漢同】策治嚴【嚴裝也】行到錢唐【錢唐縣前漢屬會稽郡後漢省其地當屬吳郡界錢唐記曰昔郡議曹華信議立此塘以防海募有能致一斛土者與錢一千旬月之間來者雲集塘未成而不復取于是載土石者皆委之而去塘以之成故名錢塘】瑀隂圖襲策濳結祖郎嚴白虎等使為内應策覺之遣其將呂範徐逸攻瑀於海西瑀敗單騎奔袁紹 初陳王寵有勇善弩射【寵明帝子陳敬王羨之曾孫也】黄巾賊起寵治兵自守【治直之翻】國人畏之不敢離叛國相會稽駱俊素有威恩是時王侯無復租禄而數見侵奪【數所角翻】或并日而食轉死溝壑而陳獨富彊鄰郡人多歸之有衆十餘萬及州郡兵起寵率衆屯陽夏【賢曰陽夏縣屬淮陽國夏音工雅翻】自稱輔漢大將軍袁術求糧於陳駱俊拒絶之術忿恚【恚於避翻】遣客詐殺俊及寵陳由是破敗 秋九月司空曹操東征袁術術聞操來棄軍走留其將橋蕤等於蘄陽以拒操【賢曰蘄水出江夏蘄春縣北山水經注云即蘄山也西南流逕蘄山又南對蘄陽注于大江亦謂之蘄陽口予據三國志術時侵陳操東征之術留蕤等拒操蕤等敗死術乃走渡淮則蓋戰於淮外也安得至江夏之蘄陽哉此蓋沛國之蘄縣范史衍陽字而通鑑因之耳】操擊破蕤等皆斬之 【考異曰范書呂布傳云布破張勲於下邳生擒橋蕤此又一橋蕤將蕤被獲又還也然魏志呂布傳無橋蕤事當是范書誤】術走渡淮時天旱歲荒士民凍餒術由是遂衰操辟陳國何夔為掾【掾俞絹翻】問以袁術何如對曰天之所助者順人之所助者信術無信順之實而望天人之助其可得乎操曰為國失賢則亡君不為術所用亡不亦宜乎操性嚴掾屬公事往往加杖夔常蓄毒藥誓死無辱是以終不見及沛國許褚勇力絶人聚少年及宗族數千家堅壁以禦外寇淮汝陳梁間皆畏憚之操狥淮汝褚以衆歸操操曰此吾樊噲也即日拜都尉引入宿衛諸從禇俠客皆以為虎士焉【俠戶頰翻】 故太尉楊彪與袁術昏姻【據彪傳彪子修袁術之甥彪蓋取於袁氏也】曹操惡之【惡烏路翻】誣云欲圖廢立奏收下獄劾以大逆【下遐稼翻劾戶槩翻又戶得翻】將作大匠孔融聞之不及朝服【朝直遥翻】往見操曰楊公四世清德【震秉賜彪四世以清白稱】海内所瞻周書父子兄弟罪不相及况以袁氏歸罪楊公乎操曰此國家之意【國家謂帝也】融曰假使成王殺邵公周公可得言不知邪操使許令滿寵按彪獄融與尚書令荀彧皆屬寵曰但當受辭勿加考掠【屬之欲翻掠音亮】寵一無所報考訊如法數日求見操言之曰楊彪考訊無它辭語此人有名海内若罪不明白必大失民望竊為明公惜之【為于偽翻】操即日赦出彪初彧融聞寵考掠彪皆怒及因此得出乃更善寵彪見漢室衰微政在曹氏遂稱脚攣【攣閭緣翻牽縮也】積十餘年不行由是得免於禍 馬日磾喪至京師【曰磾死見六十一卷興平元年磾丁奚翻】朝廷議欲加禮孔融曰日磾以上公之尊秉髦節之使【使疏吏翻】而曲媚奸臣為所牽率王室大臣豈得以見脇為辭聖上哀矜舊臣未忍追案不宜加禮朝廷從之金尚喪至京師詔百官弔祭拜其子瑋為郎中 冬十一月曹操復攻張繡【復扶又翻又如字】拔湖陽【湖陽縣屬南陽郡】禽劉表將鄧濟又攻舞陰下之韓暹楊奉在下邳寇掠徐揚間軍飢餓辭呂布欲詣荆州布不聽奉知劉備與布有宿憾私與備相聞欲共擊布備陽許之奉引軍詣沛備請奉入城飲食未半於座上縛奉斬之暹失奉孤特與十餘騎歸并州為杼秋令張宣所殺【杼秋縣前漢屬梁國後漢屬沛國師古曰杼食汝翻】胡才李樂留河東才為怨家所殺【怨於元翻】樂自病死郭汜為其將伍習所殺潁川杜襲趙儼繁欽避亂荆州【繁音婆左傳殷民七族有繁氏西漢冇御】<br />
<br />
  【史大夫繁延壽】劉表俱待以賓禮欽數見奇於表【數所角翻見賢遍翻下見能同】襲喻之曰吾所以與子俱來者徒欲全身以待時耳豈謂劉牧當為撥亂之主而規長者委身哉【長知兩翻】子若見能不已非吾徒也吾與子絶矣欽慨然曰請敬受命及曹操迎天子都許儼謂欽曰曹鎮東必能匡濟華夏【夏戶雅翻下同】吾知歸矣遂還詣操操以儼為朗陵長【朗陵縣屬汝南郡長知兩翻】陽安都尉江夏李通妻伯父犯法【操分汝南二縣置陽安都尉】儼收治致之大辟【治直之翻辟毗亦翻】時殺生之柄决於牧守【守式又翻】通妻子號泣以請其命【號戶刀翻】通曰方與曹公戮力義不以私廢公嘉儼執憲不阿與為親交<br />
<br />
  三年春正月曹操還許【攻張繡而還也】三月將復擊張繡【復扶又翻】荀攸曰繡與劉表相恃為彊然繡以遊軍仰食於表【仰牛向翻】表不能供也勢必乖離不如緩軍以待之可誘而致也【誘音酉】若急之其勢必相救操不從圍繡於穰 夏四月使謁者僕射裴茂【姓譜伯益之後封䓃鄉因以為氏後徙封解邑乃去邑從衣】詔關中諸將段煨等討李傕夷其三族【董卓之黨於是盡矣煨烏回翻傕古岳翻】以煨為安南將軍封閺鄉侯【閺音旻】 初袁紹每得詔書患其有不便於己者欲移天子自近使說曹操以許下埤溼【近其靳翻說輸芮翻下同埤皮弭翻又讀與卑同】雒陽殘破宜徙都鄄城以就全實【鄄音絹】操拒之田豐說紹曰徙都之計既不克從宜早圖許奉迎天子動託詔書號令海内此算之上者不爾終為人所禽雖悔無益也紹不從會紹亡卒詣操云田豐勸紹襲許操解穰圍而還【還從宣翻又如字】張繡率衆追之五月劉表遣兵救繡屯於安衆守險以絶軍後【水經注梅溪水出南陽宛縣北紫山南逕杜衍縣東土地墊下湍溪是注古人於安衆堨之令遊水是瀦謂之安衆港郡國志南陽郡有安衆侯國】操與荀彧書曰吾到安衆破繡必矣及到安衆操軍前後受敵操乃夜鑿險偽遁表繡悉軍來追操縱奇兵步騎夾攻大破之它日彧問操前策賊必破何也操曰虜遏吾歸師而與吾死地【兵法曰歸師勿遏又曰置之死地而後生】吾是以知勝矣繡之追操也賈詡止之曰不可追也追必敗繡不聽進兵交戰大敗而還詡登城謂繡曰促更追之更戰必勝繡謝曰不用公言以至於此今已敗奈何復追【復扶又翻下同】詡曰兵埶有變促追之【言兵埶無常審知其變則因敗而為勝】繡素信詡言遂收散卒更追合戰果以勝還【此亦小勝耳】乃問詡曰繡以精兵追退軍而公曰必敗以敗卒擊勝兵而公曰必克悉如公言何也詡曰此易知耳【易以䜴翻】將軍雖善用兵非曹公敵也曹公軍新退必自斷後【斷丁管翻下同】故知必敗曹公攻將軍既無失策力未盡而一朝引退必國内有故也【有故謂有變也】已破將軍必輕軍速進留諸將斷後諸將雖勇非將軍敵故雖用敗兵而戰必勝也繡乃服 呂布復與袁術通遣其中郎將高順及北地太守雁門張遼攻劉備【布以遼遥領北地太守耳】曹操遣將軍夏侯惇救之為順等所敗【敗補邁曰】秋九月順等破沛城虜備妻子備單身走曹操欲自擊布諸將皆曰劉表張繡在後而遠襲呂布其危必也荀攸曰表繡新破埶不敢動布驍猛又恃袁術若從横淮泗間【驍堅堯翻從子容翻】豪傑必應之今乘其初叛衆心未一往可破也操曰善比行泰山屯帥臧覇孫觀吳敦尹禮昌豨等皆附於布【比心寐翻帥所類翻豨許豈翻又音希史言攸料敵之審姓譜昌姓昌意之後】操與劉備遇于梁進至彭城陳宫謂布宜逆擊之以逸待勞無不克也布曰不如待其來蹙著泗水中【著直畧翻】冬十月操屠彭城廣陵太守陳登率郡兵為操先驅進至下邳布自將屢與操戰皆大敗【將即亮翻】還保城不敢出操遺布書為陳禍福【遺于季翻為于偽翻】布懼欲降【降戶江翻】陳宫曰曹操遠來埶不能久將軍若以步騎出屯於外宫將餘衆閉守於内若向將軍宫引兵而攻其背若但攻城則將軍救于外不過旬月操軍食盡擊之可破也布然之欲使宫與高順守城自將騎斷操糧道【斷丁管翻】布妻謂布曰宫順素不和將軍一出宫順必不同心共城守也如有蹉跌【蹉昌何翻跌徒結翻】將軍當於何自立乎且曹氏待公臺如赤子猶舍而歸我【陳宫字公臺歸布事見上卷興平元年舍讀曰捨】今將軍厚公臺不過曹氏而欲委全城捐妻子孤軍遠出若一旦有變妾豈得復為將軍妻哉【復扶又翻下同】布乃止濳遣其官屬許汜王楷求救於袁術【汜音祀】術曰布不與我女理自當敗何為復來汜楷曰明上今不救布為自敗耳布破明上亦破也【術時僭號故稱之為明上】術乃嚴兵為布作聲援布恐術為女不至故不遣救兵以緜纒女身縳著馬上夜自送女出與操守兵相觸格射不得過復還城【為于偽翻著直畧翻射而亦翻】河内太守張楊素與布善欲救之不能乃出兵東市【野王縣東市也】遥為之埶十一月楊將楊醜殺楊以應操别將眭固復殺醜【眭息隨翻】將其衆北合袁紹楊性仁和無威刑下人謀反發覺對之涕泣輒原不問故及於難【難乃旦翻】操掘塹圍下邳積久士卒疲敝欲還荀攸郭嘉曰呂布勇而無謀今屢戰皆北鋭氣衰矣三軍以將為主【將即亮翻】主衰則軍無奮意陳宫有智而遲今及布氣之未復宫謀之未定急攻之布可拔也乃引沂泗灌城【泗水東南流過下邳縣西沂水南流亦至下邳縣西而南入于泗故併引二水以灌城水經注沂水於下邳縣北西流分為二水一水於城北西南入泗一水逕城東屈從縣南亦注泗謂之小沂水水上有橋張良遇黄石公處也操於此處引沂泗灌城】月餘布益困廹 【考異曰范書布傳云灌其城三月魏志傳亦曰圍之三月按操以十月至下邳及殺布共在一季不可言三月今從魏志武紀】臨城謂操軍士曰卿曹無相困我我當自首於明公【首式救翻】陳宫曰逆賊曹操何等明公今日降之【降戶江翻下同】若卵投石豈可得全也布將侯成亡其名馬已而復得之諸將合禮以賀成成分酒肉先入獻布布怒曰布禁酒而卿等醖釀為欲因酒共謀布邪成忿懼十二月癸酉成與諸將宋憲魏續等共執陳宫高順率其衆降布與麾下登白門樓【水經注下邳城南門名白門宋武北征記曰下邳城有三重大城周四里呂布所守也魏武禽布於白門大城之門也宋白曰下邳中城南臨白樓門】兵圍之急布令左右取其首詣操左右不忍乃下降布見操曰今日已往天下定矣操曰何以言之布曰明公之所患不過於布今已服矣若令布將騎明公將步天下不足定也【將即亮翻騎奇寄翻】顧謂劉備曰玄德卿為坐上客【坐徂卧翻】我為降虜繩縛我急獨不可一言邪操笑曰縛虎不得不急乃命緩布縛劉備曰不可 【考異曰獻帝春秋曰太祖意欲活布命使寛縛主簿王必趨進曰布勍虜也其衆近在外不可寛也太祖曰本欲相緩主簿復不聽如之何今從范書陳志】明公不見呂布事丁建陽董太師乎【丁原字建陽董卓官至太師布皆殺之事見五十九卷靈帝中平六年及六十卷初平三年】操頷之【頷之者微動頤頷以應之】布目備曰大耳兒最叵信【備顧自見其耳故云然叵普火翻不可也洪邁曰叵為不可此以切脚稱也】操謂陳宫曰公臺平生自謂智有餘今竟何如宮指布曰是子不用宫言以至於此若其見從亦未必為禽也操曰奈卿老母何宫曰宫聞以孝治天下者不害人之親【治直之翻】老母存否在明公不在宫也操曰奈卿妻子何宫曰宫聞施仁政於天下者不絶人之祀妻子存否在明公不在宫也操未復言宫請就刑遂出不顧操為之泣涕【復扶又翻為于偽翻】幷布順皆縊殺之傳首許市操召陳宫之母養之終其身嫁宫女撫視其家皆厚於初【操厚陳宮之家而不肯存孔融之嗣必陳宫之妻子可保其無能為也】前尚書令陳紀紀子羣在布軍中操皆禮用之張遼將其衆降拜中郎將臧覇自亡匿操募索得之【索山客翻】使覇招吳敦尹禮孫觀等皆詣操降操乃分琅邪東海為城陽利城昌慮郡【城陽西漢王國光武省併入琅邪利城昌慮二縣皆屬東海此盖因諸屯帥所居而分為郡也慮師古音廬】悉以覇等為守相初操在兖州以徐翕毛暉為將及兖州亂翕暉皆叛兖州既定翕暉亡命投覇操語劉備【語牛倨翻】令覇送二首覇謂備曰覇所以能自立者以不為此也覇受主公生全之恩不敢違命然王覇之君可以義告願將軍為之辭備以覇言白操操歎息謂覇曰此古人之事而君能行之孤之願也皆以翕暉為郡守【守式又翻】陳登以功加伏波將軍 劉表與袁紹深相結約治中鄧羲諫表表曰内不失貢職外不背盟主【背蒲妺翻】此天下之逹義也治中獨何怪乎羲乃辭疾而退長沙太守張羨性屈強【屈渠勿翻強巨兩翻屈強便戾不順從貌】表不禮焉郡人桓階說羨舉長沙零陵桂陽三郡以拒表遣使附於曹操羨從之【說輸芮翻 考異曰魏志桓階傳袁曹相拒官渡而階說羨按范書劉表傳建安三年羨拒表在官渡前也】孫策遣其正議校尉張紘獻方物【正議校尉亦孫策私所署置】曹操欲撫納之表策為討逆將軍【討逆將軍亦創置也】封吳侯【由烏程徙封吳進其封也考異曰江表傳曰倍於元年所獻其年制書拜討逆封吳侯按策貢獻在二年非元年也又陳志紘傳曰建】<br />
<br />
  【安四年遣紘奉章詣許按吳書紘述策材畧忠欵曹公乃優文褒崇改號加封然則紘來在策封吳侯前本傳誤也】以弟女配策弟匡又為子彰取孫賁女【為于偽翻取讀曰娶】禮辟策弟權翊【操禮辟權翊欲其至以為質耳】以張紘為侍御史袁術以周瑜為居巢長以臨淮魯肅為東城長【居巢縣屬廬江郡東城縣前漢屬九江郡後漢省當是術復置也長知兩翻】瑜肅知術終無所成皆棄官渡江從孫策策以瑜為建威中郎將肅因家於曲阿曹操表徵王朗策遣朗還操以朗為諫議大夫參司空軍事【參軍事昉於魏晉之間位望頗重孫楚謂石苞曰天子命我參卿軍事是也自是以後位望輕矣】袁術遣間使【間古莧翻使疏吏翻】齎印綬與丹陽宗帥祖郎等【帥所類翻】使激動山越共圖孫策劉繇之奔豫章也太史慈遁於蕪湖山中自稱丹陽太守策已定宣城以東惟涇以西六縣未服慈因進住涇縣大為山越所附【蕪湖涇縣皆属丹陽郡宣城縣前漢亦屬丹陽後漢省晉大康元年分丹陽立宣城郡復置縣屬焉山越越民依阻山險而居者】于是策自將討祖郎於陵陽禽之【陵陽縣屬丹陽郡陵陽子明得仙於此縣山因名】策謂郎曰爾昔襲孤【事見上卷興平元年】斫孤馬鞍今創軍立事除棄宿恨惟取能用與天下通耳非但汝汝勿恐怖【怖普布翻】郎叩頭謝罪即破械署門下職曹又討太史慈於勇里【勇里在涇縣】禽之解縛捉其手【捉執也】曰寧識神亭時邪若卿爾時得我云何【神亭事見上卷興平二年】慈曰未可量也【量音良】策大笑曰今日之事當與卿共之聞卿有烈義天下智士也【慈東萊人少為郡奏曹史時郡與州有隙交章以闈而州章先到雒慈劫取壞之由是知名後赴孔融之急詣劉備求救此策所謂烈義也】但所託未得其人耳【謂劉繇也】孤是卿知己勿憂不如意也即署門下督軍還祖郎太史慈俱在前導軍人以為榮會劉繇卒于豫章士衆萬餘人欲奉豫章太守華歆為主歆以為因時擅命非人臣所宜衆守之連月卒謝遣之【卒子恤翻華戶化翻】其衆未有所附策命太史慈往撫安之謂慈曰劉牧往責吾為袁氏攻廬江【劉繇奉王命牧揚州故以稱之攻廬江事見上卷興平元年為于偽翻】吾先君兵數千人盡在公路許吾志在立事安得不屈意於公路而求之乎其後不遵臣節諫之不從【事見上建安元年】丈夫義交苟有大故不得不離吾交求公路及絶之本末如此恨不及其生時與共論辨也今兒子在豫章卿往視之幷宣孤意於其部曲部曲樂來者與俱來不樂來者且安慰之【樂音洛】幷觀華子魚所以牧御方規何如【華歆字子魚】卿須幾兵多少隨意慈曰慈有不赦之罪將軍量同桓文當盡死以報德今並息兵兵不宜多將數十人足矣左右皆曰慈必北去不還策曰子義捨我當復從誰【復扶又翻】餞送昌門【孫權記注曰吳西郭門曰閶門夫差作以天門通閶闔故名之後春申君改曰昌門】把腕别曰【腕烏貫翻】何時能還荅曰不過六十日慈行議者猶紛紜言遣之非計策曰諸君勿復言孤斷之詳矣【斷丁亂翻】太史子義雖氣勇有膽烈然非縱横之人【縱子容翻】其心秉道義重然諾【然是也决辭也諾應也許辭也重不輕也】一以意許知已死亡不相負諸君勿憂也慈果如期而反謂策曰華子魚良德也然無它方規自守而已又丹陽僮芝自擅廬陵【僮姓也風俗通漢有交趾刺史僮尹一曰僮即童也顓頊子老童之後或從人廬陵縣屬豫章郡】番陽民帥别立宗部言我已别立郡海昏上繚不受發召【番陽縣屬豫章郡宗部即所謂江南宗賊也帥所類翻海昏縣屬豫章郡時縣民數千家自相結聚作宗伍壁於上繚水經注僚水導源建昌縣漢元帝永光二年分海昏立僚水又東逕新吳縣漢中平中立僚水又逕海昏縣謂之上僚水繚讀曰僚】子魚但覩視之而已策拊掌大笑遂有兼幷之志袁紹連年攻公孫瓚不能克以書諭之欲相與釋憾連和瓚不答而增修守備謂長史太原關靖曰當今四方虎爭無有能坐吾城下相守經年者明矣袁本初其若我何紹于是大興兵以攻瓚先是瓚别將有為敵所圍者瓚不救【先悉薦翻】曰救一人使後將恃救不肯力戰及紹來攻瓚南界别營自度守則不能自固【度徒洛翻】又知必不見救或降或潰【降戶江翻】紹軍徑至其門【易京之門也】瓚遣子續請救於黑山諸帥【黑山諸帥張燕等也帥所類翻】而欲自將突騎出傍西山【自易京西扺故安閻鄉以西諸山連接中山之界山谷深廣皆黑山諸賊所依阻也傍步浪翻】擁黑山之衆侵掠冀州横斷紹後【斷丁管翻】關靖諫曰今將軍將士莫不懷瓦解之心所以猶能相守者顧戀其居處老少【處昌呂翻】而恃將軍為主故耳堅守曠日或可使紹自退若舍之而出【舍讀曰捨】後無鎮重易京之危可立待也瓚乃止紹漸相攻逼瓚衆日䠞【䠞子六翻】<br />
<br />
  資治通鑑卷六十二<br />
<br />
<史部,編年類,資治通鑑>  <br>
   </div> 

<script src="/search/ajaxskft.js"> </script>
 <div class="clear"></div>
<br>
<br>
 <!-- a.d-->

 <!--
<div class="info_share">
</div> 
-->
 <!--info_share--></div>   <!-- end info_content-->
  </div> <!-- end l-->

<div class="r">   <!--r-->



<div class="sidebar"  style="margin-bottom:2px;">

 
<div class="sidebar_title">工具类大全</div>
<div class="sidebar_info">
<strong><a href="http://www.guoxuedashi.com/lsditu/" target="_blank">历史地图</a></strong>  
<a href="http://www.880114.com/" target="_blank">英语宝典</a>  
<a href="http://www.guoxuedashi.com/13jing/" target="_blank">十三经检索</a> 
<br><strong><a href="http://www.guoxuedashi.com/gjtsjc/" target="_blank">古今图书集成</a></strong> 
<a href="http://www.guoxuedashi.com/duilian/" target="_blank">对联大全</a> <strong><a href="http://www.guoxuedashi.com/xiangxingzi/" target="_blank">象形文字典</a></strong> 

<br><a href="http://www.guoxuedashi.com/zixing/yanbian/">字形演变</a>  <strong><a href="http://www.guoxuemi.com/hafo/" target="_blank">哈佛燕京中文善本特藏</a></strong>
<br><strong><a href="http://www.guoxuedashi.com/csfz/" target="_blank">丛书&方志检索器</a></strong> <a href="http://www.guoxuedashi.com/yqjyy/" target="_blank">一切经音义</a>  

<br><strong><a href="http://www.guoxuedashi.com/jiapu/" target="_blank">家谱族谱查询</a></strong>  <strong><a href="http://shufa.guoxuedashi.com/sfzitie/" target="_blank">书法字帖欣赏</a></strong> 
<br>

</div>
</div>


<div class="sidebar" style="margin-bottom:0px;">

<font style="font-size:22px;line-height:32px">QQ交流群9:489193090</font>


<div class="sidebar_title">手机APP 扫描或点击</div>
<div class="sidebar_info">
<table>
<tr>
	<td width=160><a href="http://m.guoxuedashi.com/app/" target="_blank"><img src="/img/gxds-sj.png" width="140"  border="0" alt="国学大师手机版"></a></td>
	<td>
<a href="http://www.guoxuedashi.com/download/" target="_blank">app软件下载专区</a><br>
<a href="http://www.guoxuedashi.com/download/gxds.php" target="_blank">《国学大师》下载</a><br>
<a href="http://www.guoxuedashi.com/download/kxzd.php" target="_blank">《汉字宝典》下载</a><br>
<a href="http://www.guoxuedashi.com/download/scqbd.php" target="_blank">《诗词曲宝典》下载</a><br>
<a href="http://www.guoxuedashi.com/SiKuQuanShu/skqs.php" target="_blank">《四库全书》下载</a><br>
</td>
</tr>
</table>

</div>
</div>


<div class="sidebar2">
<center>


</center>
</div>

<div class="sidebar"  style="margin-bottom:2px;">
<div class="sidebar_title">网站使用教程</div>
<div class="sidebar_info">
<a href="http://www.guoxuedashi.com/help/gjsearch.php" target="_blank">如何在国学大师网下载古籍?</a><br>
<a href="http://www.guoxuedashi.com/zidian/bujian/bjjc.php" target="_blank">如何使用部件查字法快速查字?</a><br>
<a href="http://www.guoxuedashi.com/search/sjc.php" target="_blank">如何在指定的书籍中全文检索?</a><br>
<a href="http://www.guoxuedashi.com/search/skjc.php" target="_blank">如何找到一句话在《四库全书》哪一页?</a><br>
</div>
</div>


<div class="sidebar">
<div class="sidebar_title">热门书籍</div>
<div class="sidebar_info">
<a href="/so.php?sokey=%E8%B5%84%E6%B2%BB%E9%80%9A%E9%89%B4&kt=1">资治通鉴</a> <a href="/24shi/"><strong>二十四史</strong></a>&nbsp; <a href="/a2694/">野史</a>&nbsp; <a href="/SiKuQuanShu/"><strong>四库全书</strong></a>&nbsp;<a href="http://www.guoxuedashi.com/SiKuQuanShu/fanti/">繁体</a>
<br><a href="/so.php?sokey=%E7%BA%A2%E6%A5%BC%E6%A2%A6&kt=1">红楼梦</a> <a href="/a/1858x/">三国演义</a> <a href="/a/1038k/">水浒传</a> <a href="/a/1046t/">西游记</a> <a href="/a/1914o/">封神演义</a>
<br>
<a href="http://www.guoxuedashi.com/so.php?sokeygx=%E4%B8%87%E6%9C%89%E6%96%87%E5%BA%93&submit=&kt=1">万有文库</a> <a href="/a/780t/">古文观止</a> <a href="/a/1024l/">文心雕龙</a> <a href="/a/1704n/">全唐诗</a> <a href="/a/1705h/">全宋词</a>
<br><a href="http://www.guoxuedashi.com/so.php?sokeygx=%E7%99%BE%E8%A1%B2%E6%9C%AC%E4%BA%8C%E5%8D%81%E5%9B%9B%E5%8F%B2&submit=&kt=1"><strong>百衲本二十四史</strong></a>  <a href="http://www.guoxuedashi.com/so.php?sokeygx=%E5%8F%A4%E4%BB%8A%E5%9B%BE%E4%B9%A6%E9%9B%86%E6%88%90&submit=&kt=1"><strong>古今图书集成</strong></a>
<br>

<a href="http://www.guoxuedashi.com/so.php?sokeygx=%E4%B8%9B%E4%B9%A6%E9%9B%86%E6%88%90&submit=&kt=1">丛书集成</a> 
<a href="http://www.guoxuedashi.com/so.php?sokeygx=%E5%9B%9B%E9%83%A8%E4%B8%9B%E5%88%8A&submit=&kt=1"><strong>四部丛刊</strong></a>  
<a href="http://www.guoxuedashi.com/so.php?sokeygx=%E8%AF%B4%E6%96%87%E8%A7%A3%E5%AD%97&submit=&kt=1">說文解字</a> <a href="http://www.guoxuedashi.com/so.php?sokeygx=%E5%85%A8%E4%B8%8A%E5%8F%A4&submit=&kt=1">三国六朝文</a>
<br><a href="http://www.guoxuedashi.com/so.php?sokeytm=%E6%97%A5%E6%9C%AC%E5%86%85%E9%98%81%E6%96%87%E5%BA%93&submit=&kt=1"><strong>日本内阁文库</strong></a> <a href="http://www.guoxuedashi.com/so.php?sokeytm=%E5%9B%BD%E5%9B%BE%E6%96%B9%E5%BF%97%E5%90%88%E9%9B%86&ka=100&submit=">国图方志合集</a> <a href="http://www.guoxuedashi.com/so.php?sokeytm=%E5%90%84%E5%9C%B0%E6%96%B9%E5%BF%97&submit=&kt=1"><strong>各地方志</strong></a>

</div>
</div>


<div class="sidebar2">
<center>

</center>
</div>
<div class="sidebar greenbar">
<div class="sidebar_title green">四库全书</div>
<div class="sidebar_info">

《四库全书》是中国古代最大的丛书,编撰于乾隆年间,由纪昀等360多位高官、学者编撰,3800多人抄写,费时十三年编成。丛书分经、史、子、集四部,故名四库。共有3500多种书,7.9万卷,3.6万册,约8亿字,基本上囊括了古代所有图书,故称“全书”。<a href="http://www.guoxuedashi.com/SiKuQuanShu/">详细>>
</a>

</div> 
</div>

</div>  <!--end r-->

</div>
<!-- 内容区END --> 

<!-- 页脚开始 -->
<div class="shh">

</div>

<div class="w1180" style="margin-top:8px;">
<center><script src="http://www.guoxuedashi.com/img/plus.php?id=3"></script></center>
</div>
<div class="w1180 foot">
<a href="/b/thanks.php">特别致谢</a> | <a href="javascript:window.external.AddFavorite(document.location.href,document.title);">收藏本站</a> | <a href="#">欢迎投稿</a> | <a href="http://www.guoxuedashi.com/forum/">意见建议</a> | <a href="http://www.guoxuemi.com/">国学迷</a> | <a href="http://www.shuowen.net/">说文网</a><script language="javascript" type="text/javascript" src="https://js.users.51.la/17753172.js"></script><br />
  Copyright &copy; 国学大师 古典图书集成 All Rights Reserved.<br>
  
  <span style="font-size:14px">免责声明:本站非营利性站点,以方便网友为主,仅供学习研究。<br>内容由热心网友提供和网上收集,不保留版权。若侵犯了您的权益,来信即刪。scp168@qq.com</span>
  <br />
ICP证:<a href="http://www.beian.miit.gov.cn/" target="_blank">鲁ICP备19060063号</a></div>
<!-- 页脚END --> 
<script src="http://www.guoxuedashi.com/img/plus.php?id=22"></script>
<script src="http://www.guoxuedashi.com/img/tongji.js"></script>

</body>
</html>
