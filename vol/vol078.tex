<!DOCTYPE html PUBLIC "-//W3C//DTD XHTML 1.0 Transitional//EN" "http://www.w3.org/TR/xhtml1/DTD/xhtml1-transitional.dtd">
<html xmlns="http://www.w3.org/1999/xhtml">
<head>
<meta http-equiv="Content-Type" content="text/html; charset=utf-8" />
<meta http-equiv="X-UA-Compatible" content="IE=Edge,chrome=1">
<title>資治通鑒_79-資治通鑑卷七十八_79-資治通鑑卷七十八</title>
<meta name="Keywords" content="資治通鑒_79-資治通鑑卷七十八_79-資治通鑑卷七十八">
<meta name="Description" content="資治通鑒_79-資治通鑑卷七十八_79-資治通鑑卷七十八">
<meta http-equiv="Cache-Control" content="no-transform" />
<meta http-equiv="Cache-Control" content="no-siteapp" />
<link href="/img/style.css" rel="stylesheet" type="text/css" />
<script src="/img/m.js?2020"></script> 
</head>
<body>
 <div class="ClassNavi">
<a  href="/24shi/">二十四史</a> | <a href="/SiKuQuanShu/">四库全书</a> | <a href="http://www.guoxuedashi.com/gjtsjc/"><font  color="#FF0000">古今图书集成</font></a> | <a href="/renwu/">历史人物</a> | <a href="/ShuoWenJieZi/"><font  color="#FF0000">说文解字</a></font> | <a href="/chengyu/">成语词典</a> | <a  target="_blank"  href="http://www.guoxuedashi.com/jgwhj/"><font  color="#FF0000">甲骨文合集</font></a> | <a href="/yzjwjc/"><font  color="#FF0000">殷周金文集成</font></a> | <a href="/xiangxingzi/"><font color="#0000FF">象形字典</font></a> | <a href="/13jing/"><font  color="#FF0000">十三经索引</font></a> | <a href="/zixing/"><font  color="#FF0000">字体转换器</font></a> | <a href="/zidian/xz/"><font color="#0000FF">篆书识别</font></a> | <a href="/jinfanyi/">近义反义词</a> | <a href="/duilian/">对联大全</a> | <a href="/jiapu/"><font  color="#0000FF">家谱族谱查询</font></a> | <a href="http://www.guoxuemi.com/hafo/" target="_blank" ><font color="#FF0000">哈佛古籍</font></a> 
</div>

 <!-- 头部导航开始 -->
<div class="w1180 head clearfix">
  <div class="head_logo l"><a title="国学大师官网" href="http://www.guoxuedashi.com" target="_blank"></a></div>
  <div class="head_sr l">
  <div id="head1">
  
  <a href="http://www.guoxuedashi.com/zidian/bujian/" target="_blank" ><img src="http://www.guoxuedashi.com/img/top1.gif" width="88" height="60" border="0" title="部件查字,支持20万汉字"></a>


<a href="http://www.guoxuedashi.com/help/yingpan.php" target="_blank"><img src="http://www.guoxuedashi.com/img/top230.gif" width="600" height="62" border="0" ></a>


  </div>
  <div id="head3"><a href="javascript:" onClick="javascript:window.external.AddFavorite(window.location.href,document.title);">添加收藏</a>
  <br><a href="/help/setie.php">搜索引擎</a>
  <br><a href="/help/zanzhu.php">赞助本站</a></div>
  <div id="head2">
 <a href="http://www.guoxuemi.com/" target="_blank"><img src="http://www.guoxuedashi.com/img/guoxuemi.gif" width="95" height="62" border="0" style="margin-left:2px;" title="国学迷"></a>
  

  </div>
</div>
  <div class="clear"></div>
  <div class="head_nav">
  <p><a href="/">首页</a> | <a href="/ShuKu/">国学书库</a> | <a href="/guji/">影印古籍</a> | <a href="/shici/">诗词宝典</a> | <a   href="/SiKuQuanShu/gxjx.php">精选</a> <b>|</b> <a href="/zidian/">汉语字典</a> | <a href="/hydcd/">汉语词典</a> | <a href="http://www.guoxuedashi.com/zidian/bujian/"><font  color="#CC0066">部件查字</font></a> | <a href="http://www.sfds.cn/"><font  color="#CC0066">书法大师</font></a> | <a href="/jgwhj/">甲骨文</a> <b>|</b> <a href="/b/4/"><font  color="#CC0066">解密</font></a> | <a href="/renwu/">历史人物</a> | <a href="/diangu/">历史典故</a> | <a href="/xingshi/">姓氏</a> | <a href="/minzu/">民族</a> <b>|</b> <a href="/mz/"><font  color="#CC0066">世界名著</font></a> | <a href="/download/">软件下载</a>
</p>
<p><a href="/b/"><font  color="#CC0066">历史</font></a> | <a href="http://skqs.guoxuedashi.com/" target="_blank">四库全书</a> |  <a href="http://www.guoxuedashi.com/search/" target="_blank"><font  color="#CC0066">全文检索</font></a> | <a href="http://www.guoxuedashi.com/shumu/">古籍书目</a> | <a   href="/24shi/">正史</a> <b>|</b> <a href="/chengyu/">成语词典</a> | <a href="/kangxi/" title="康熙字典">康熙字典</a> | <a href="/ShuoWenJieZi/">说文解字</a> | <a href="/zixing/yanbian/">字形演变</a> | <a href="/yzjwjc/">金 文</a> <b>|</b>  <a href="/shijian/nian-hao/">年号</a> | <a href="/diming/">历史地名</a> | <a href="/shijian/">历史事件</a> | <a href="/guanzhi/">官职</a> | <a href="/lishi/">知识</a> <b>|</b> <a href="/zhongyi/">中医中药</a> | <a href="http://www.guoxuedashi.com/forum/">留言反馈</a>
</p>
  </div>
</div>
<!-- 头部导航END --> 
<!-- 内容区开始 --> 
<div class="w1180 clearfix">
  <div class="info l">
   
<div class="clearfix" style="background:#f5faff;">
<script src='http://www.guoxuedashi.com/img/headersou.js'></script>

</div>
  <div class="info_tree"><a href="http://www.guoxuedashi.com">首页</a> > <a href="/SiKuQuanShu/fanti/">四库全书</a>
 > <h1>资治通鉴</h1> <!--         下载:【右键另存为】即可 --></div>
  <div class="info_content zj clearfix">
  
<div class="info_txt clearfix" id="show">
<center style="font-size:24px;">79-資治通鑑卷七十八</center>
    資治通鑑卷七十八   宋 司馬光 撰<br />
<br />
  胡三省 音註<br />
<br />
  魏紀十【起玄黓敦牂盡閼逢涒灘凡三年涒音暾】<br />
<br />
  元皇帝下<br />
<br />
  景元三年秋八月乙酉吳主立皇后朱氏朱公主之女也戊子立子為太子【烏關翻據吳志吳主休為四子作名字音湖水灣澳之灣非先有此音也】 漢大將軍姜維將出軍右車騎將軍廖化曰兵不戢必自焚伯約之謂也【左傳魯襄仲曰兵猶火也不戢將自焚姜維字伯約廖力教翻今力弔翻】智不出敵而力小於宼用之無厭將何以存【謂較智則不出于敵人之上而較力則又弱小也厭於鹽翻】冬十月維入宼洮陽【洮陽洮水之陽也洮水之隂魏不置郡縣維渡洮而攻之也沙州記曰嵹城東北三百里有曾城臨洮水曰洮陽城杜佑曰臨洮郡城本洮陽城臨洮水洮土刀翻】鄧艾與戰於侯和破之維住沓中【水經注洮水逕洮陽城又東逕共和山南城在四山中又東逕迷和城北意侯和即此地也沓中在諸羌中即沙漒之地晉張駿據河西因前趙之亂收河南地至于狄道置武街石門侯和漒川甘松五屯護軍與後趙分境乞伏熾盤攻漒川師次沓中則侯和之地在塞内沓中之地在羌中明矣】初維以羈旅依漢【維降漢見七十一卷明帝太和元年】身受重任興兵累年功績不立黄皓用事於中與右大將軍閻宇親善隂欲廢維樹宇維知之言於漢主曰皓姦巧專恣將敗國家請殺之【敗補邁翻】漢主曰皓趨走小臣耳往董允每切齒【事見七十四卷邵陵厲公正始六年】吾常恨之君何足介意維見皓枝附葉連懼於失言遜辭而出漢主敕皓詣維陳謝維由是自疑懼【此維未出洮陽以前事也】返自洮陽因求種麥沓中不敢歸成都【司馬昭因是决計絆維於沓中而伐蜀】 吳主以濮陽興為丞相廷尉丁密光禄勲孟宗為左右御史大夫【漢成帝綏和元年罷御史大夫置大司空世祖中興因之獻帝建安十三年罷司空復置御史大夫未嘗分左右也蓋吳分之】初興為會稽太守【會工外翻守式又翻】吳主在會稽興遇之厚左將軍張布嘗為會稽王左右督將【吳主休先封琅邪王徙居會稽自會稽入立未嘗封會稽王也會稽當作琅邪將即亮翻】故吳主即位二人皆貴寵用事布典宫省興關軍國以佞巧更相表裏【更工衡翻】吳人失望吳主喜讀書【喜許記翻】欲與博士祭酒韋昭博士盛冲講論【前漢五經博士有僕射一人東漢轉為祭酒胡廣曰官名祭酒皆一位之元長也古禮賓客得主人饌老者一人舉酒以祭於地舊說以為示有先沈約志曰吳王濞為劉氏祭酒夫祭祀以酒為本長者主之故以祭酒為稱漢侍中魏散騎常侍高功者並為祭酒公府祭酒漢末有之】張布以昭冲切直恐其入侍言已隂過固諫止之吳主曰孤之涉學羣書畧徧但欲與昭等講習舊聞亦何所損君特當恐昭等道臣下姦慝故不欲令入耳如此之事孤已自備之不須昭等然後乃解也布惶恐陳謝且言懼妨政事吳主曰王務學業其流各異不相妨也【王務猶言王事也】此無所為非而君以為不宜是以孤有所及耳不圖君今日在事更行此於孤也良甚不取布拜表叩頭【據陳夀志自孤之涉學已下皆詔答之語布得詔惶恐以表陳謝重自序述吳主又面答之自王務學業以下皆面答之語也所謂今日在事更行此於孤蓋比之孫綝以綝擅權之時不使吳主親近儒生也於是布拜叩頭未嘗再上表也此表字衍在事者在官任事也】吳主曰聊相開悟耳何至叩頭乎如君之忠誠遠近所知吾今日之巍巍皆君之功也詩云靡不有初鮮克有終【詩大雅蕩之辭鮮音息淺翻】終之實難君其終之然吳主恐布疑懼卒如布意【卒音子恤翻】廢其講業不復使昭等入【復扶又翻】 譙郡嵇康【晉書曰康之先姓奚會稽上虞人以避怨徙譙郡銍縣銍有嵇山家於其側因以命氏】文辭壮麗好言老莊而尚奇任俠【俠戶頰翻】與陳留阮籍籍兄子咸【姓譜殷有阮國在岐渭之間周詩有侵阮徂共之辭子孫以國為姓後漢有已吾令阮敦】河内山濤河南向秀【向式亮翻】琅邪王戎沛國劉伶特相友善號竹林七賢皆崇尚虚無輕蔑禮法縱酒昏酣遺落世事阮籍為步兵校尉其母卒籍方與人圍碁對者求止籍留與决賭【與决勝負也】既而飲酒二斗舉聲一號吐血數升毁瘠骨立【骨立者言其瘠甚身肉俱消唯骨立也號戶刀翻吐土故翻】居喪飲酒無異平日司隸校尉何曾惡之【惡烏路翻】面質籍於司馬昭座【質正也面以正義責之也】曰卿縱情背禮敗俗之人今忠賢執政綜核名實若卿之曹不可長也【背蒲妹翻敗補邁翻長知兩翻】因謂昭曰公方以孝治天下【治直之翻】而聽阮籍以重哀飲酒食肉於公座何以訓人宜擯之四裔無令汚染華夏【汚烏故翻】昭愛籍才常擁護之【昭之讓九錫也籍代公卿為勸進牋辭甚清壮故昭愛其才】曾夔之子也【何夔見六十三卷漢獻帝建安五年】阮咸素幸姑婢姑將婢去咸方對客遽借客馬追之累騎而還【累重也兩人共馬謂之累騎還音旋又如字】劉伶嗜酒常乘鹿車【賢曰鹿車言其小僅可容鹿也】攜一壺酒使人荷鍤隨之【荷下可翻鍤側洽翻鍫也】曰死便埋我當時士大夫皆以為賢爭慕效之謂之放逹鍾會方有寵於司馬昭聞嵇康名而造之【造七到翻】康箕踞而鍜【康性巧而好鍜鍜都玩翻小冶也】不為之禮會將去康曰何所聞而來何所見而去會曰聞所聞而來見所見而去遂深衘之山濤為吏部郎【魏尚書郎有二十三員吏部其一也】舉康自代康與濤書自說不堪流俗而非薄湯武昭聞而怒之【湯武革命而康非薄之故昭聞而怒】康與東平呂安親善安兄巽誣安不孝康為證其不然【為于偽翻】會因譖康嘗欲助毋丘儉【言毋丘儉反而康欲助之毋音無】且安康有盛名於世而言論放蕩害時亂教宜因此除之昭遂殺安及康康嘗詣隱者汲郡孫登【晉泰始二年始分河内為汲郡史追書也】登曰子才多識寡難乎免於今之世矣 司馬昭患姜維數為宼官騎路遺求為刺客入蜀【官騎騶騎也數所角翻騎奇寄翻】從事中郎荀朂曰明公為天下宰宜杖正義以伐違貳【違離也背也貳攜貳也兩屬也】而以刺客除賊非所以刑于四海也【毛萇曰刑法也韓嬰曰刑正也】昭善之朂爽之曾孫也【荀爽淑之子也漢末為公】昭欲大舉伐漢朝臣多以為不可獨司隸校尉鍾會勸之昭諭衆曰自定夀春以來息役六年治兵繕甲以擬二虜【治直之翻】今吳地廣大而下濕攻之用功差難不如先定巴蜀三年之後因順流之埶水陸並進此滅虢取虞之埶也【春秋晉獻公滅虢因以滅虞此言滅蜀乘勢可以滅吳也】計蜀戰士九萬居守成都及備他境不下四萬然則餘衆不過五萬今絆姜維於沓中【絆博漫翻繫足曰絆】使不得東顧直指駱谷出其空虚之地以襲漢中以劉禪之闇而邉城外破士女内震其亡可知也乃以鍾會為鎮西將軍都督關中征西將軍鄧艾以為蜀未有釁屢陳異議【善用兵者觀釁而動此艾所以陳異議也】昭使主簿師纂為艾司馬以諭之【姓譜師古者掌樂之官因以為氏】艾乃奉命姜維表漢主聞鍾會治兵關中欲規進取宜並遣左右車騎張翼廖化【時張翼為左車騎將軍廖化為右車騎將軍】督諸軍分護陽安關口【陽安關口意即陽平關也】及隂平之橋頭【杜佑曰隂平橋頭在文州界】以防未然黄皓信巫鬼謂敵終不自致【致至也又請也送也】啓漢主寢其事羣臣莫知<br />
<br />
  四年春正月復命司馬昭進爵位如前【如元年之詔也復扶又翻】又辭不受 吳交趾太守孫諝貪暴【諝私呂翻】為百姓所患會吳主遣察戰鄧荀至交趾【裴松之曰察戰吳官號今揚都有察戰巷】荀擅調孔爵三十頭送建業【調徒弔翻】民憚遠役因謀作亂夏五月郡吏呂興等殺諝及荀遣使來請太守及兵九真日南皆應之 詔諸軍大舉伐漢遣征西將軍鄧艾督三萬餘人自狄道趣甘松沓中【甘松本生羌之地張駿置甘松護軍乞伏國仁置甘松郡後魏時白水羌朝貢置甘松縣太和六年改置扶州隋改甘松為嘉誠縣屬同昌郡唐武德初置松州取甘松嶺為名且其地產甘松也杜佑曰甘松嶺江水發源之地甘松山在今交川郡境今臨洮和政郡之南及合川郡之地新唐書曰甘松山在洮水之西吐谷渾居山之陽】以連綴姜維雍州刺史諸葛緒督三萬餘人自祁山趣武街橋頭絶維歸路【賢曰下辨縣屬武都郡今城州同谷縣舊名武街城水經注濁水逕武街城南又曰白水出臨洮縣西傾山東南逕隂平故城南又東北逕橋頭雍於用翻】鍾會統十餘萬衆分從斜谷駱谷子午谷趣漢中【斜余遮翻谷音浴趣七喻翻】以廷尉衛瓘持節監艾會軍事行鎮西軍司【鍾會時為鎮西將軍瓘既監艾會軍又行會軍司監古銜翻】瓘覬之子也【衛覬歷事武帝文帝明帝覬音冀】會過幽州刺史王雄之孫戎【王雄刺幽州遣勇士刺殺軻比能】問計將安出戎曰道家有言為而不恃【老子道經之言】非成功難保之難也或以問參相國軍事平原劉實曰鍾鄧其平蜀乎實曰破蜀必矣而皆不還客問其故實笑而不荅【鍾鄧之禍識者固知之矣】秋八月軍發洛陽大賚將士【賚來代翻賜也】陳師誓衆將軍鄧敦謂蜀未可討司馬昭斬以狥漢人聞魏兵且至乃遣廖化將兵詣沓中為姜維繼援張翼董厥等詣陽安關口為諸圍外助大赦改元炎興敕諸圍皆不得戰退保漢樂二城【用姜維之言也】城中各有兵五千人翼厥北至隂平聞諸葛緒將向建威留住月餘待之鍾會率諸軍平行至漢中九月鍾會使前將軍李輔統萬人圍王含於樂城護軍荀愷圍蒋斌於漢城【斌音彬 考異曰晉書文紀作部將易愷今從魏志】會徑過西趣陽安口遣人祭諸葛亮墓【諸葛亮葬沔陽】初漢武興督蒋舒在事無稱【宋白曰武興漢武都沮縣也元和郡國志曰興州城即古武興城也蜀以處當衝要置武興督以守之無稱言其庸庸無可稱者】漢朝令人代之【朝直遥翻】使助將軍傅僉守關口舒由是恨鍾會使護軍胡烈為前鋒攻關口舒詭謂僉曰今賊至不擊而閉城自守非良圖也僉曰受命保城惟全為功今違命出戰若喪師負國【喪息浪翻】死無益矣舒曰子以保城獲全為功我以出戰克敵為功請各行其志遂率其衆出僉謂其戰也不設備【使舒果迎戰亦未可保其必勝僉何為不設備邪關城失守僉亦有罪焉】舒率其衆迎降胡烈【降戶江翻】烈乘虚襲城僉格鬬而死僉肜之子也【傅肜死事見六十九卷文帝黄初三年肜余中翻】鍾會聞關口已下長驅而前大得庫藏積穀【藏音徂浪翻】鄧艾遣天水太守王頎直攻姜維營【前漢天水郡後漢改曰漢陽郡魏復曰天水頎渠希翻】隴西太守牽弘邀其前金城太守楊欣趣甘松維聞鍾會諸軍已入漢中引兵還欣等追躡於疆川口大戰【彊川口在嵹臺山南嵹臺山即臨洮之西傾山闕駰曰彊水出隂平西北彊山一曰彊川姜維之還也鄧艾遣王頎追敗之於彊口即是地也】維敗走聞諸葛緒已塞道屯橋頭【塞悉則翻】乃從孔函谷入北道欲出緒後緒聞之却還三十里維入北道三十餘里聞緒軍却尋還從橋頭過緒趣截維較一日不及【言較遲一日遂不及維也】維遂還至隂平合集士衆欲赴關城聞其已破退趣白水遇廖化張翼董厥等合兵守劒閣以拒會【水經注小劍戍西去大劒山三十里連山絶險飛閣通衢故謂之劒閣華陽國志曰廣漢郡德陽縣有劒閣道三十里至險祝穆曰劒門漢屬廣漢郡為葭萌縣地蜀先主以霍峻為梓潼太守有劒閣縣苻秦使徐成宼蜀攻二劒克之始有二劒之號】 安國元侯高柔卒 冬十月漢人告急於吳甲申吳主使大將軍丁奉督諸軍向夀春將軍留平就施績於南郡議兵所向將軍丁封孫異如沔中以救漢【沔中時為魏境吳兵未能至也擬其所向耳吳之巫秭歸等縣皆在江北與魏之新城接境自此行兵亦可以達沔中然亦猶激西江之水以救涸轍之魚耳】詔以征蜀諸將獻捷交至復命大將軍昭進位爵賜一如前詔【復扶又翻】昭乃受命【始受相國晉公九錫之命】昭辟任城魏舒為相國參軍【任音壬】初舒少時遲鈍不為鄉親所重【鄉里親戚也少詩沼翻】從叔父吏部郎衡有名當世【從才用翻】亦不知之使守水碓【為碓水側寘輪碓後以横木貫輪横木之兩頭復以木長二尺許交午貫之正直碓尾木激水灌輪輪轉則交午木戞擊碓尾木而自舂不煩人力謂之水碓碓都内翻】每歎曰舒堪數百戶長【謂小邑長也長知兩翻】我願畢矣舒亦不以介意不為皎厲之事【皎者求以暴白於世厲危行也】唯太原王乂謂舒曰卿終當為台輔常振其匱乏舒受而不辭年四十餘郡舉上計掾【上時掌翻掾于絹翻】察孝亷宗黨以舒無學業勸令不就可以為高舒曰若試而不中【中竹仲翻】其負在我安可虛竊不就之高以為已榮乎於是自課百日習一經因而對策升第累遷後將軍鍾毓長史毓每與參佐射【參佐參軍及諸佐吏毓余六翻】舒常為畫籌而已【射之畫籌猶投壺之釋筭也為于偽翻下徐為同】後遇朋人不足以舒滿數【射以兩人為朋射之有朋猶古射儀之有耦也周禮王以六耦射三侯諸侯以四耦射二侯卿大夫以三耦射一侯士以三侯射豻侯左傳魯襄公享范獻子射者三耦公臣不足取於家臣杜預注云二人為耦】舒容範閒雅發無不中【中竹仲翻】舉坐愕然莫有敵者【坐徂卧翻】毓歎而謝曰吾之不足以盡卿才有如此射矣豈一事哉及為相國參軍府朝碎務未嘗見是非【府朝猶言府庭也朝直遥翻見賢遍翻】至於廢興大事衆人莫能斷者【斷丁亂翻】舒徐為籌之多出衆議之表昭深器重之 癸卯立皇后卞氏昭烈將軍秉之孫也 鄧艾進至隂平簡選精鋭欲與諸葛緒自江油趣成都【水經注涪水出廣漢屬國剛氐道徼外東南流逕緜竹縣北又東南逕江油戍北鄧艾自隂平景谷步道懸兵束馬入蜀逕江油廣漢者也宋白曰龍州江油郡北踰山至文州三百三十里文州漢隂平地也鄧艾自隂平行無人之地七百里至江油即此九域志龍州北至文州四百三十里元豐九域志龍州治江油縣南至綿州二百餘里】緒以本受節度邀姜維西行非本詔遂引軍向白水【此白水關也賢曰在今梁州金牛縣西東北至關城百八十里】與鍾會合欲專軍埶密白緒畏懦不進檻車徵還軍悉屬會姜維列營守險會攻之不能克糧道險遠軍食乏欲引還鄧艾上言賊已摧折宜遂乘之若從隂平由邪徑經漢德陽亭【按前漢無德陽縣後漢志廣漢郡始有德陽縣蓋因漢故亭而置縣也自蜀分廣漢置梓潼郡之後劒閣縣屬梓潼德陽縣屬廣漢續漢志以為德陽縣有劒閣今姜維守劒閣拒鍾會而鄧艾欲從德陽亭趣涪則此時分為兩縣明矣然德陽亭亦非此時德陽縣治蓋前漢德陽亭故處也此道即所謂隂平景谷道】趣涪【趣七喻翻涪音浮】出劒閣西百里去成都三百餘里奇兵衝其腹心出其不意劒閣之守必還赴涪則會方軌而進劒閣之軍不還則應涪之兵寡矣【趣七喻翻涪音浮】遂自隂平行無人之地七百餘里鑿山通道造作橋閣【今隆慶府隂平縣北六十里有馬閣山峻峭崚嶒極為艱險鄧艾軍行至此路不得通乃懸車束馬造作棧閣始通江油因名馬閣又自文州青塘嶺至龍州百五十里自北而南者右肩不得易所負謂之左擔路亦艾伐蜀路也據鍾會傳艾自漢德陽亭入江油左擔道則德陽亭蓋當馬閣山之路】山谷高深至為艱險又糧運將匱瀕於危殆艾以氊自裹推轉而下【推吐雷翻】將士皆攀木緣崖魚貫而進【山崖險陿單行相繼而進如貫魚然】先登至江油【江油今龍州江油縣地南至綿州二百餘里綿州古涪城也】蜀守將馬邈降【降戶江翻下同】諸葛瞻督諸軍拒艾至涪停住不進【陳夀曰涪去成都三百六十里】尚書郎黄崇權之子也【黄權劉璋所用先主伐吳而敗權隔在江北遂降魏】屢勸瞻宜速行據險無令敵得入平地瞻猶豫未納崇再三言之至于流涕瞻不能從艾遂長驅而前擊破瞻前鋒瞻住緜竹【緜竹縣屬廣漢郡今緜竹縣東北至綿州百餘里】艾以書誘瞻曰若降者必表為琅邪王【諸葛氏本琅邪人故以此誘之誘音酉】瞻怒斬艾使列陳以待艾【使疏吏翻陳讀曰陣下同】艾遣子惠唐亭侯忠出其右司馬師纂等出其左忠纂戰不利並引還曰賊未可擊艾怒曰存亡之分在此一舉何不可之有叱忠纂等將斬之忠纂馳還更戰大破斬瞻及黄崇瞻子尚歎曰父子荷國重恩【荷下可翻】不早斬黄皓使敗國殄民用生何為策馬冒陳而死【杜佑曰漢州德陽縣鄧艾破諸葛瞻於此因為京觀敗補邁翻】漢人不意魏兵卒至不為城守調度【卒讀曰猝調徒弔翻】聞艾已入平土百姓擾擾皆迸山澤不可禁制【迸北孟翻】漢主使羣臣會議或以蜀之與吳本為與國宜可奔吳或以為南中七郡【南中七郡越嶲朱提牂柯雲南興古建寧永昌也】阻險斗絶易以自守【易以豉翻】宜可奔南光禄大夫譙周以為自古以來無寄它國為天子者若入吳國亦當臣服且治政不殊則大能吞小此數之自然也【治直吏翻】由此言之則魏能并吳吳不能并魏明矣等為稱臣為小孰與為大【為于偽翻】再辱之耻何與一辱【謂今降魏一辱而已若奔吳稱臣是一辱矣與吳俱亡又將臣服於魏是為再辱】且若欲犇南則當蚤為之計然後可果【果决也克也】今大敵已近禍敗將及羣小之心無一可保恐發足之日其變不測何至南之有乎【謂衆心已離既行之後中道潰散必不能至南中】或曰今艾已不遠恐不受降如之何【降戶江翻下同】周曰方今東吳未賓事埶不得不受受之不得不禮若陛下降魏魏不裂土以封陛下者周請身詣京都【京都謂洛陽魏都晉景王諱師晉人避之率謂京師為京都蜀方議降譙周已為晉人諱矣吁】以古義爭之衆人皆從周議漢主猶欲入南狐疑未决周上疏曰南方遠夷之地平常無所供為【言其民既不出税租以供上用又不出力為上有所施為】猶數反叛自丞相亮以兵威偪之窮乃率從【事見七十卷文帝黄初六年數所角翻】今若至南外當拒敵内供服御費用張廣他無所取耗損諸夷其叛必矣漢主乃遣侍中張紹等奉璽綬以降於艾【璽斯氏翻綬音受】北地王諶怒曰若理窮力屈禍敗將及便當父子君臣背城一戰同死社稷以見先帝可也【諶時壬翻背蒲妹翻】奈何降乎漢主不聽是日諶哭於昭烈之廟先殺妻子而後自殺【曾謂庸禪有子如此乎】張紹等見鄧艾於雒【雒縣屬廣漢郡西南至成都八十餘里】艾大喜報書褒納漢主遣太僕蒋顯别敕姜維使降鍾會又遣尚書郎李虎送士民簿於艾戶二十八萬口九十四萬甲士十萬二千吏四萬人艾至成都城北漢主率太子諸王及羣臣六十餘人面縛輿櫬詣軍門【杜預曰面縛縛手於後唯見其面也櫬棺也示將受死櫬初覲翻後主時年四十八】艾持節解縛焚櫬延請相見檢御將士無得虜略綏納降附使復舊業輒依鄧禹故事承制拜漢主禪行驃騎將軍太子奉車諸王駙馬都尉漢羣司各隨高下拜為王官或領艾官屬【依鄧禹承制授隗囂故事也後艾由此得罪驃匹妙翻】以師纂領益州刺史隴西太守牽弘等領蜀中諸郡艾聞黄皓姦險收閉將殺之皓賂艾左右卒以得免【卒子恤翻】姜維等聞諸葛瞻敗未知漢主所嚮乃引軍東入于巴【巴即巴中也】鍾會進軍至涪遣胡烈等追維維至郪【郪縣屬廣漢郡劉昫曰梓州飛烏縣漢郪縣地隋取飛烏山以名縣師古曰郪音妻又音千私翻】得漢主敕命乃令兵悉放仗送節傳於胡烈【傳株戀翻】自從東道與廖化張翼董厥等同詣會降將士咸怒拔刀斫石【觀此則蜀之將士豈肯下人哉其主不能用之耳】於是諸郡縣圍守皆被漢主敕罷兵降【圍守即魏延所置漢中諸圍之守兵也】鍾會厚待姜維等皆權還其印綬節蓋【漢先主以獻帝建安十九年得蜀魏文帝黄初二年即帝位傳二世四十三年而亡】 吳人聞蜀已亡乃罷丁奉等兵吳中書丞吳郡華覈【魏有中書監令無中書丞此官盖吳置也華戶化翻覈戶革翻】詣宫門上表曰伏聞成都不守臣主播越社稷傾覆失委附之土棄貢獻之國臣以草芥竊懷不寧陛下聖仁恩澤遠撫卒聞如此必垂哀悼臣不勝忡悵之情【卒讀曰猝勝音升忡丑中翻憂也】謹拜表以聞【左傳楚人滅江秦伯為之降服出次不舉過數大夫諫公曰同盟滅敢不矜乎吾自懼也蜀吳之與國蜀亡岌岌乎為吳矣吳之君臣不知懼故華覈拜表以儆之】魏之伐蜀也吳人或謂襄陽張悌曰司馬氏得政以來大難屢作【難乃旦翻謂王凌毋丘儉諸葛誕舉兵也】百姓未服今又勞力遠征敗於不暇何以能克悌曰不然曹操雖功蓋中夏【夏戶雅翻】民畏其威而不懷其德也丕叡承之刑繁役重東西驅馳無有寧歲司馬懿父子累有大功除其煩苛而布其平惠為之謀主而救其苦民心歸之亦已久矣故淮南三叛而腹心不擾【邵陵厲公嘉平元年王凌叛高貴鄉公正元元年毌丘儉叛甘露二年諸葛誕叛】曹髦之死四方不動任賢使能各盡其心其本根固矣姦計立矣今蜀閹宦專朝【朝直遥翻】國無政令而玩戎黷武民勞卒敝競於外利不修守備彼彊弱不同智筭亦勝因危而伐殆無不克噫彼之得志我之憂也吳人笑其言至是乃服 吳人以武陵五溪夷與蜀接界蜀亡懼其叛亂乃以越騎校尉鍾離牧領武陵太守魏已遣漢葭縣長郭純試守武陵太守率涪陵民入遷陵界【沈約曰漢獻帝建安六年劉璋以涪陵縣分立丹興漢葭二縣立巴東屬國都尉後為涪陵郡遷陵縣屬武陵郡吳境也長知兩翻】屯于赤沙誘動諸夷進攻酉陽【赤沙蓋在遷陵酉陽之間酉陽縣屬武陵郡縣在酉溪之陽劉昫曰黔州彭水縣漢酉陽縣地吳分酉陽置黔陽郡隋於郡置彭水縣尋為黔州九域志曰漢武陵郡酉陽縣古城在今辰州界杜佑曰思州治務川縣亦漢酉陽地】郡中震懼牧問朝吏曰【朝郡朝也朝直遥翻】西蜀傾覆邉境見侵何以禦之皆對曰今二縣山險諸夷阻兵不可以軍驚擾驚擾則諸夷盤結宜以漸安可遣恩信吏宣敎慰勞【勞力到翻】牧曰外境内侵誑誘人民【誑居况翻】當及其根柢未深而撲取之【柢典禮翻又下計翻撲普卜翻】此救火貴速之埶也敕外趣嚴【趣讀曰促嚴装也】撫夷將軍高尚謂牧曰昔潘太常督兵五萬然後討五溪夷【事見七十二卷明帝太和五年】是時劉氏連和諸夷率化今既無往日之援而郭純已據遷陵而明府欲以三千兵深入尚未見其利也牧曰非常之事何得循舊即帥所領晨夜進道【帥讀曰率】緣山險行垂二千里斬惡民懷異心者魁帥百餘人【帥所類翻】及其支黨凡千餘級純等散走五溪皆平 十二月庚戌以司徒鄭冲為太保 壬子分益州為梁州【益州統蜀犍為汶山漢嘉江陽朱提越嶲牂柯梁州統漢中梓潼廣漢涪陵巴巴西巴東梁古州也言西方金剛之氣彊梁故因名焉】 癸丑特赦益州士民復除租税之半五年【復方目翻】 乙卯以鄧艾為太尉增邑二萬戶鍾會為司徒增邑萬戶【賞平蜀之功也】 皇太后郭氏殂 鄧艾在成都頗自矜伐謂蜀士大夫諸君賴遭艾故得有今日耳如遇吳漢之徒已殄滅矣【吳漢屠成都事見四十二卷漢光武建武十二年】艾以書言於晉公昭曰兵有先聲而後實者【漢初李左車以是說韓信艾祖其說以言於晉公司馬昭既受封錫遂書其爵】今因平蜀之埶以乘吳吳人震恐席卷之時也然大舉之後將士疲勞不可便用且徐緩之留隴右兵二萬人蜀兵二萬人煮鹽興冶為軍農要用【蜀有鹽井朱提出銀嚴道卭都出銅武陽南安臨卭沔陽皆出鐵漢置鹽官鐵官艾欲復其利】並作舟船豫為順流之事然後發使告以利害吳必歸化可不征而定也【使疏吏翻】今宜厚劉禪以致孫休封禪為扶風王錫其資財供其左右郡有董卓塢【董卓築塢於扶風郿縣】為之宫舍爵其子為公侯食郡内縣以顯歸命之寵開廣陵城陽以待吳人【開廣陵城陽為王國以待孫休也廣陵屬徐州城陽屬青州蓋魏廣陵郡治淮隂故城城陽郡治莒二郡壤界實相接也】則畏威懷德望風而從矣昭使監軍衛瓘喻艾事當須報不宜輒行艾重言曰【重直用翻】衘命征行奉指授之策元惡既服至於承制拜假以安初附謂合權宜今蜀舉衆歸命地盡南海【南中之地東南帶海接于交阯】東接吳會宜早鎮定若待國命往復道途延引日月春秋之義大夫出疆有可以安社稷利國家專之可也【春秋公羊傳之言】今吳未賓埶與蜀連不可拘常以失事機兵法進不求名不避罪【孫子曰將之至任不可不察也進不求名不避罪唯人是保而利於主國之寶也】艾雖無古人之節終不自嫌以損國家計也鍾會内有異志姜維知之欲搆成擾亂乃說會曰聞君自淮南已來筭無遺策【謂平諸葛誕也說輸芮翻】晉道克昌皆君之力今復定蜀【復扶又翻】威德振世民高其功主畏其謀欲以此安歸乎何不法陶朱公汎舟絶迹全功保身邪【越大夫范蠡既與越王句踐滅吳以雪會稽之耻乃扁舟五湖汎海而止於陶欲絶其跡乃號曰陶朱公】會曰君言遠矣我不能行且為今之道或未盡於此也維曰其佗則君智力之所能無煩於老夫矣【言為亂也維之智固足以玩弄鍾會於掌股之上迫於時制於命奈之何哉】由是情好歡甚【好呼到翻】出則同轝坐則同席會因鄧艾承制專事乃與衛瓘密白艾有反狀會善效人書於劒閣要艾章表白事【要一遥翻章表上之魏朝白事白之晉公】皆易其言令辭指悖傲多自矜伐【悖蒲内翻又蒲没翻】又毁晉公昭報書手作以疑之【既以怒昭又以疑艾】<br />
<br />
  咸熙元年【是年五月始改元咸熙此猶是景元五年】春正月壬辰詔以檻車徵鄧艾晉公昭恐艾不從命敕鍾會進軍成都又遣賈充將兵入斜谷【斜昌遮翻谷音浴又古禄翻】昭自將大軍從帝幸長安【將即亮翻】以諸王公皆在鄴乃以山濤為行軍司馬鎮鄴【楚王彪之死盡録諸王公置鄴事見七十五卷邵陵厲公嘉平三年行軍司馬之號始此】初鍾會以才能見任昭夫人王氏言於昭曰【昭夫人王氏肅之女也生晉武帝齊王攸後謚文明皇后】會見利忘義好為事端【好呼到翻】寵過必亂不可大任及會將伐漢西曹屬邵悌言於晉公曰【自漢以來丞相有東西曹掾屬】今遣鍾會率十餘萬衆伐蜀愚謂會單身無任【魏制凡遣將帥皆留其家以為質任會單身無子弟故曰單身無任】不若使餘人行也晉公笑曰我寧不知此邪蜀數為邉宼師老民疲我今伐之如指掌耳【指掌言易也數所角翻】而衆言蜀不可伐夫人心豫怯則智勇並竭智勇並竭而彊使之【彊其兩翻】適所以為敵禽耳惟鍾會與人意同今遣會伐蜀蜀必可滅滅蜀之後就如卿慮何憂其不能辦邪【言會若為亂自能辦之也】夫蜀已破亡遺民震恐不足與共圖事中國將士各自思歸不肯與同也會若作惡【作為也惡不善也作惡作亂也所為不善也】秪自滅族耳卿不須憂此慎勿使人聞也及晉公將之長安悌復曰鍾會所統兵五六倍於鄧艾但可敕會取艾不須自行晉公曰卿忘前言邪【忘巫放翻】而云不須行乎雖然所言不可宣也我要自當以信意待人但人不當負我耳我豈可先人生心哉【先悉薦翻】近日賈護軍問我頗疑鍾會不【賈護軍賈充也時為中護軍不讀曰否】我答言如今遣卿行寜可復疑卿邪【復扶又翻】賈亦無以易我語也我到長安則自了矣【了辦也决也】鍾會遣衛瓘先至成都收鄧艾會以瓘兵少欲令艾殺瓘因以為艾罪瓘知其意心然不可得距【瓘監艾會軍遣之收艾是以職分使之故不可得而拒】乃夜至成都檄艾所統諸將稱奉詔收艾其餘一無所問若來赴官軍爵賞如先【謂復加爵賞如先平蜀時也】敢有不出誅及三族比至鷄鳴【比必寐翻】悉來赴瓘唯艾帳内在焉平旦開門瓘乘使者車【續漢志有大使車小使車諸使車大使車立乘駕駟赤帷持節者重導從賊曹車斧車督車功曹車皆兩大車伍伯璅弩十二人辟車四人從車四乘無節單導從者减半小使車不立乘冇騑赤屛泥油重絳帷導無斧車近小使車蘭輿赤轂白蓋赤帷從騶騎四十人此謂追捕考案有所勑取者之所乘也諸使車皆朱班輪四輻亦衡軛】徑入至艾所艾尚卧未起遂執艾父子置艾於檻車諸將圖欲刼艾整仗趣瓘營【趣七喻翻】瓘輕出迎之偽作表草將申明艾事【詭言將申明艾無反心】諸將信之而止丙子會至成都送艾赴京師會所憚惟艾艾父子既禽會獨統大衆威震西土遂决意謀反會欲使姜維將五萬人出斜谷為前驅會自將大衆隨其後既至長安令騎士從陸道步兵從水道順流浮渭入河以為五日可到孟津與騎兵會洛陽一旦天下可定也【談何容易】會得晉公書云恐鄧艾或不就徵今遣中護軍賈充將步騎萬人徑入斜谷屯樂城【諸葛亮所築成固之樂城也】吾自將十萬屯長安相見在近會得書驚呼所親語之曰但取鄧艾相國知我獨辦之【謂昭知會之足以辦取艾之事語牛倨翻】今來大重【大讀曰太】必覺我異矣【異變也】便當速發事成可得天下不成保蜀漢不失作劉備也【蜀漢謂漢蜀郡漢中郡之地】丁丑會悉請護軍郡守牙門騎督以上【此皆從會軍在成都者也】及蜀之故官為太后發哀於蜀朝堂【明元郭太后去年殂蜀都成都有朝堂朝直遥翻】矯太后遺詔使會起兵廢司馬昭皆班示坐上人【坐徂卧翻】使下議訖書版署置更使所親信代領諸軍所請羣官悉閉著益州諸曹屋中【著直略翻】城門宮門皆閉嚴兵圍守衛瓘詐稱篤出就外廨【廨古隘翻舍也】會信之無所復憚姜維欲使會盡殺北來諸將已因殺會盡坑魏兵復立漢主密書與劉禪曰願陛下忍數日之辱臣欲使社稷危而復安日月幽而復明【姜維之心始終為漢千載之下炳炳如丹陳夀孫盛干寶之譏貶皆非也】會欲從維言誅諸將猶預未决會帳下督丘建【風俗通丘魯左丘明之後又云齊太公封於營丘支孫以地為氏】本屬胡烈會愛信之建愍烈獨坐啓會使聽内一親兵出取飲食諸牙門隨例各内一人烈紿語親兵及疏與子淵曰丘建密說消息會已作大坑白棓數千【紿徒亥翻語牛倨翻棓步項翻】欲悉呼外兵入人賜白㡊【幍魏武帝所製狀如弁缺四角幍苦洽翻】拜散將【將即亮翻】以次棓殺内坑中諸牙門親兵亦咸說此語一夜轉相告皆徧己卯日中胡淵率其父兵雷鼓出門【雷盧對翻】諸軍不期皆鼓譟而出曾無督促之者而爭先赴城時會方給姜維鎧杖【杖與仗同直亮翻】白外有匈匈聲似失火者【匈許容翻又許勇翻毛晃曰匈匈喧擾之聲】有頃白兵走向城會驚謂維曰兵來似欲作惡當云何維曰但當擊之耳會遣兵悉殺所閉諸牙門郡守内人共舉机以柱門【内人謂會所閉在屋内者机舉綺翻机案也】兵斫門不能破斯須城外倚梯登城【斯此也須待也言其問無多時於此可待也】或燒城屋蟻附亂進矢下如雨牙門郡守各緣屋出與其軍士相得姜維率會左右戰手殺五六人衆格斬維爭前殺會【考異曰衛瓘傳曰會留瓘謀議乃書版云欲殺胡烈等舉以示瓘不許因相疑貳瓘如厠見胡烈故給使使宣語三軍言會反會逼瓘定議經宿不眠各横刀膝上在外諸軍已潛欲攻會瓘既不出未敢先發會使瓘慰勞諸軍瓘便下殿會悔遣之使呼瓘瓘辭眩動詐仆地比出閣數十信追之瓘至外廨服鹽湯大吐會遣所親人及醫視之皆言不起會由是無所憚及暮門閉瓘作檄宣告諸軍並已唱義陵旦共攻會殺之常璩華陽國志曰會命諸將發喪因欲誅之諸將半入而南安太守胡烈等知其謀燒成都東門以襲殺會及維今從魏志又世語曰維死時見剖膽如斗大如斗非身所能容恐當作升】會將士死者數百人殺漢太子璿及姜維妻子軍衆鈔略死喪狼籍【璿從宣翻鈔楚交翻】衛瓘部分諸將數日乃定【分扶問翻】鄧艾本營將士追出艾於檻車迎還衛瓘自以與會共䧟艾恐其為變乃遣護軍田續等將兵襲艾遇於緜竹西斬艾父子艾之入江油也田續不進艾欲斬續既而捨之及瓘遣續謂曰可以報江油之辱矣鎮西長史杜預言於衆曰伯玉其不免乎【衛瓘行鎮西軍司而杜預為鎮西長史則為同僚而軍事則瓘任之也瓘字伯玉】身為名士位望已高既無德音又不御下以正【謂激田續使報鄧艾而行其私也】將何以堪其責乎瓘聞之不候駕而謝預預恕之子也【杜恕見七十三卷明帝景初元年】鄧艾餘子在洛陽者悉伏誅徙其妻及孫於西城【西城縣屬魏興郡為晉武帝叙艾孫灼張本】鍾會兄毓嘗密言於晉公曰會挾術難保不可專任及會反毓已卒【卒子恤翻】晉公思鍾繇之勲與毓之賢【鍾繇有定關中之功】特原毓子峻辿官爵如故【裴松之曰辿勑連翻】會功曹向雄收葬會尸晉公召而責之曰往者王經之死卿哭於東市而我不問【事見上卷景元元年】鍾會躬為叛逆又輒收葬若復相容當如王法何【復扶又翻】雄曰昔先王掩骼埋胔仁流朽骨【記月令孟春之月掩骼埋胔鄭玄注曰骨枯曰骼肉腐曰胔陸德明曰露骨曰骼有肉曰胔骼江百翻胔才賜翻周文王澤及朽骨】當時豈先卜其功罪而後收葬哉今王誅既加於法已備雄感義收葬敎亦無闕法立于上敎弘于下以此訓物不亦可乎何必使雄背死違生以立于世【背蒲妹翻】明公讐對枯骨【言會已誅晉公復以枯骨為讐對不令收葬】捐之中野豈仁賢之度哉晉公悦與宴談而遣之 二月丙辰車駕還洛陽 庚申葬明元皇后 初劉禪使巴東太守襄陽羅憲將兵二千人守永安【姓譜羅本顓頊末胤受封於羅國今房州也為楚所滅子孫以為氏譙周巴記曰漢獻帝初平六年益州司馬趙韙建議分巴郡諸縣漢安以下為永寜郡建安六年劉璋改永寜為巴東郡治魚復縣蜀先主章武二年改魚復曰永安】聞成都敗吏民驚擾憲斬稱成都亂者一人百姓乃定及得禪手敕乃帥所統臨于都亭三日【帥讀曰率都亭永安之都亭也臨力鴆翻】吳聞蜀敗起兵西上【上時掌翻】外託救援内欲襲憲憲曰本朝傾覆【朝直遥翻】吳為唇齒不恤我難而背盟徼利【難乃旦翻徼一遥翻】不義甚矣且漢已亡吳何得久我寜能為吳降虜乎【降戶江翻】保城繕甲告誓將士厲以節義莫不憤激吳人聞鍾鄧敗百城無主有兼蜀之志而巴東固守兵不得過乃使撫軍步協率衆而西【協步隲子吳以為撫軍將軍】憲力弱不能禦遣參軍楊宗突圍北出告急於安東將軍陳騫又送文武印綬任子詣晉公協攻永安憲與戰大破之吳主怒復遣鎮軍陸抗等帥衆三萬人增憲之圍【時吳以陸抗為鎮軍將軍都督西陵帥讀曰率】 三月丁丑以司空王祥為太尉征北將軍何曾為司徒左僕射荀顗為司空【顗魚豈翻】 己卯進晉公爵為王增封十郡【高貴鄉公甘露三年晉公始封八郡帝景元之三年加封司州之弘農雍州之馮翊凡十郡今又增封十郡凡二十郡】王祥何曾荀顗共詣晉王【顗魚豈翻】顗謂祥曰相王尊重何侯與一朝之臣【何侯謂何曾一朝之臣謂舉魏朝之臣也朝直遥翻下同】皆已盡敬今日便當相率而拜無所疑也祥曰相國雖尊要是魏之宰相吾等魏之三公王公相去一階而已安有天子三公可輒拜人者損魏朝之望虧晉王之德君子愛人以禮我不為也及入顗遂拜而祥獨長揖王謂祥曰今日然後知君見顧之重也 劉禪舉家東遷洛陽時擾攘倉猝禪之大臣無從行者【姜維既死張翼廖化董厥必亦死於亂兵矣】惟祕書令郤正及殿中督汝南張通捨妻子單身隨禪禪賴正相導宜適舉動無闕【宜當也適亦當也禪初入洛見魏君臣其禮各有所當嗚呼使正束帶立於朝上而擯贊漢主下而與賓客言事事令宜而無闕失豈非人臣之至願哉】乃慨然歎息恨知正之晩初漢建寜太守霍弋都督南中【建寜漢益州郡也蜀後主建興元年改建寜郡治味縣】聞魏兵至欲赴成都劉禪以備敵既定不聽成都不守弋素服大臨三日【臨力鴆翻】諸將咸勸弋宜速降【降戶江翻下同】弋曰今道路隔塞【塞悉則翻】未詳主之安危去就大故不可苟也若魏以禮遇主上則保境而降不晩也若萬一危辱吾將以死拒之何論遲速邪得禪東遷之問始率六郡將守上表曰【南中七郡而此言六郡者蓋越嶲已降魏也將即亮翻守式又翻】臣聞人生在三事之如一惟難所在則致其命【無父母烏生無君烏以為生所謂人生在三也難乃旦翻】今臣國敗主附守死無所是以委質不敢有貳【質如字】晉王善之拜南中都尉委以本任丁亥封劉禪為安樂公【晉志安樂屬燕國樂音洛下間樂同】子孫及羣臣封侯者五十餘人晉王與禪宴為之作故蜀技【蜀技蜀樂也如巴渝舞之類也為于偽翻下同技與伎同渠綺翻】旁人皆為之感愴而禪喜笑自若王謂賈充曰人之無情乃至於此雖使諸葛亮在不能輔之久全况姜維邪佗曰王問禪曰頗思蜀否禪曰此間樂不思蜀也郤正聞之謂禪曰若王後問宜泣而答曰先人墳墓遠在岷蜀乃心西悲無日不思【西悲用詩東山語此儒生之搜章摘句也】因閉其目會王復問【復扶又翻】禪對如前王曰何乃似郤正語邪禪驚視曰誠如尊命左右皆笑 夏四月新附督王稚浮海入吳句章【新附督蓋以吳人新附者别為一部置督以領之句章縣屬會稽郡賢曰句章故城在今鄮縣西】略其長吏及男女二百餘口而還【長知兩翻】 五月庚申晉王奏復五等爵封騎督以上六百餘人【賞平蜀之功也周制列爵五等公侯地方百里伯七十里子男五十里秦廢五等爵漢列侯以戶為差獻帝建安二十年魏王操置名號侯以賞軍功虚封自此始矣今雖復五等爵亦虚封也騎奇寄翻】 甲戍改元【始改元咸熙】 癸未追命舞陽文宣侯懿為晉宣王忠武侯師為景王 羅憲被攻凡六月【被皮義翻】救援不到城中疾病太半或說憲棄城走【說輸芮翻下布說同】憲曰吾為城主百姓所仰危不能安急而棄之君子不為也畢命於此矣陳騫言於晉王遣荆州刺史胡烈將步騎二萬攻西陵以救憲秋七月吳師晉王使憲因仍舊任加陵江將軍【沈約志魏置陵江將軍為四十號之首言欲陵駕江流以蕩平吳會也】封萬年亭侯 晉王奏使司空荀顗定禮儀中護軍賈充正法律尚書僕射裴秀議官制太保鄭冲總而裁焉 吳分交州置廣州【漢武帝元鼎六年開百越置交趾州刺史治龍編獻帝建安八年改曰交州治蒼梧廣信縣十六年徙治南海番禺縣至是分為二州廣州治番禺交州還治龍編】吳主寢疾口不能言乃手書呼丞相濮陽興入令子出拜之【讀如彎】休把興臂指以託之癸未吳主殂謚曰景帝【年三十】羣臣尊朱皇后為皇太后吳人以蜀初亡交趾攜叛【謂呂興反也】國内恐愳欲得長君【長知兩翻】左典軍萬彧嘗為烏程令與烏程侯皓相善稱皓之才識明斷長沙桓王之儔也【孫策謚長沙桓王斷丁亂翻】又加之好學奉遵法度【好呼到翻】屢言之於丞相興左將軍布興布說朱太后欲以皓為嗣朱后曰我寡婦人安知社稷之慮苟吳國無隕宗廟有賴可矣【賴恃也利也】於是遂迎立皓改元元興大赦【皓字元宗孫和之子】 八月庚寅命中撫軍司馬炎副貳相國事【依五官將故事也】 初鍾會之伐漢也辛憲英謂其夫之從子羊祜曰會在事縱恣非持久處下之道【從才用翻處昌呂翻】吾畏其有他志也會請其子郎中琇為參軍【琇息救翻】憲英憂曰他日吾為國憂今日難至吾家矣【為于偽翻難乃旦翻】琇固請於晉王王不聽憲英謂琇曰行矣戒之軍旅之間可以濟者其惟仁恕乎琇竟以全歸詔以琇嘗諫會反賜爵關内侯【琇司馬師夫人之從父弟故以諫會為功而得封】 九月戊午以司馬炎為撫軍大將軍【晉志撫軍大將軍位從公班驃騎車騎衛伏波等將軍下】 辛未詔以呂興為安南將軍都督交州諸軍事以南中監軍霍弋遥領交州刺史得以便宜選用長吏弋表遣建寜㸑谷為交趾太守【㸑氏建寜之大姓世為耆帥至隋唐為東㸑西㸑蠻杜佑曰昆明在越嶲西南諸㸑所居】率牙門董元毛炅【炅古迥翻又古惠翻】孟幹孟通爨能李松王素等將兵助興未至興為其功曹王統所殺吳主貶朱太后為景皇后【貶其號從夫而自父其父母其母】追謚父和曰文皇帝尊母何氏為太后 冬十月丁亥詔以夀春所獲吳相國參軍事徐紹為散騎常侍水曹掾孫彧為給事黄門侍郎【水曹掾吳相府所置吳未嘗置相國魏人以晉王為相國因亦稱吳丞相參軍掾于絹翻】以使於吳【使疏吏翻】其家人在此者悉聽自隨不必使還以開廣大信【言吳不必使還以廣中國之信攜吳人之心】晉王因致書吳主諭以禍福 初晉王娶王肅之女生炎及攸以攸繼景王後【司馬師謚景王】攸性孝友多才藝清和平允名聞過於炎【聞音問】晉王愛之常曰天下者景王之天下也吾攝居相位百年之後大業宜歸攸炎立髪委地手垂過䣛【䣛與膝同】嘗從容問裴秀曰人有相否因以異相示之【從干容翻相息亮翻】秀由是歸心羊琇與炎善為炎畫策察時政所宜損益【為于偽翻】皆令炎豫記之以備晉王訪問晉王欲以攸為世子山濤曰廢長立少違禮不祥【長知兩翻少詩沼翻】賈充曰中撫軍有君人之德不可易也何曾裴秀曰中撫軍聰明神武有超世之才人望既茂天表如此固非人臣之相也【相息亮翻】晉王由是意定丙午立炎為世子【為晉武帝不能容齊王攸張本】 吳主封太子及其三弟皆為王【弟名□□音如兕觥之觥次名壾壾音如屮莽之莽次名音如襃衣下寛大之褎皆吳主休自作名字】立妃滕氏為皇后 初吳主之立發優詔恤士民開倉廪振貧乏科出宫女以配無妻者【科條也】禽獸養於苑中者皆放之當時翕然稱為明主及既得志麤暴驕盈多忌諱好酒色【好呼到翻】大小失望濮陽興張布竊悔之或譖諸吳主十一月朔興布入朝【朝直遥翻】吳主執之徙於廣州道殺之夷三族以后父滕牧為衛將軍録尚書事牧胤之族人也【滕胤為孫綝所殺】 是歲罷屯田官【置屯田官事見六十二卷漢獻帝建安元年】<br />
<br />
  資治通鑑卷七十八<br />
<br />
<史部,編年類,資治通鑑>  <br>
   </div> 

<script src="/search/ajaxskft.js"> </script>
 <div class="clear"></div>
<br>
<br>
 <!-- a.d-->

 <!--
<div class="info_share">
</div> 
-->
 <!--info_share--></div>   <!-- end info_content-->
  </div> <!-- end l-->

<div class="r">   <!--r-->



<div class="sidebar"  style="margin-bottom:2px;">

 
<div class="sidebar_title">工具类大全</div>
<div class="sidebar_info">
<strong><a href="http://www.guoxuedashi.com/lsditu/" target="_blank">历史地图</a></strong>  
<a href="http://www.880114.com/" target="_blank">英语宝典</a>  
<a href="http://www.guoxuedashi.com/13jing/" target="_blank">十三经检索</a> 
<br><strong><a href="http://www.guoxuedashi.com/gjtsjc/" target="_blank">古今图书集成</a></strong> 
<a href="http://www.guoxuedashi.com/duilian/" target="_blank">对联大全</a> <strong><a href="http://www.guoxuedashi.com/xiangxingzi/" target="_blank">象形文字典</a></strong> 

<br><a href="http://www.guoxuedashi.com/zixing/yanbian/">字形演变</a>  <strong><a href="http://www.guoxuemi.com/hafo/" target="_blank">哈佛燕京中文善本特藏</a></strong>
<br><strong><a href="http://www.guoxuedashi.com/csfz/" target="_blank">丛书&方志检索器</a></strong> <a href="http://www.guoxuedashi.com/yqjyy/" target="_blank">一切经音义</a>  

<br><strong><a href="http://www.guoxuedashi.com/jiapu/" target="_blank">家谱族谱查询</a></strong>  <strong><a href="http://shufa.guoxuedashi.com/sfzitie/" target="_blank">书法字帖欣赏</a></strong> 
<br>

</div>
</div>


<div class="sidebar" style="margin-bottom:0px;">

<font style="font-size:22px;line-height:32px">QQ交流群9:489193090</font>


<div class="sidebar_title">手机APP 扫描或点击</div>
<div class="sidebar_info">
<table>
<tr>
	<td width=160><a href="http://m.guoxuedashi.com/app/" target="_blank"><img src="/img/gxds-sj.png" width="140"  border="0" alt="国学大师手机版"></a></td>
	<td>
<a href="http://www.guoxuedashi.com/download/" target="_blank">app软件下载专区</a><br>
<a href="http://www.guoxuedashi.com/download/gxds.php" target="_blank">《国学大师》下载</a><br>
<a href="http://www.guoxuedashi.com/download/kxzd.php" target="_blank">《汉字宝典》下载</a><br>
<a href="http://www.guoxuedashi.com/download/scqbd.php" target="_blank">《诗词曲宝典》下载</a><br>
<a href="http://www.guoxuedashi.com/SiKuQuanShu/skqs.php" target="_blank">《四库全书》下载</a><br>
</td>
</tr>
</table>

</div>
</div>


<div class="sidebar2">
<center>


</center>
</div>

<div class="sidebar"  style="margin-bottom:2px;">
<div class="sidebar_title">网站使用教程</div>
<div class="sidebar_info">
<a href="http://www.guoxuedashi.com/help/gjsearch.php" target="_blank">如何在国学大师网下载古籍?</a><br>
<a href="http://www.guoxuedashi.com/zidian/bujian/bjjc.php" target="_blank">如何使用部件查字法快速查字?</a><br>
<a href="http://www.guoxuedashi.com/search/sjc.php" target="_blank">如何在指定的书籍中全文检索?</a><br>
<a href="http://www.guoxuedashi.com/search/skjc.php" target="_blank">如何找到一句话在《四库全书》哪一页?</a><br>
</div>
</div>


<div class="sidebar">
<div class="sidebar_title">热门书籍</div>
<div class="sidebar_info">
<a href="/so.php?sokey=%E8%B5%84%E6%B2%BB%E9%80%9A%E9%89%B4&kt=1">资治通鉴</a> <a href="/24shi/"><strong>二十四史</strong></a>&nbsp; <a href="/a2694/">野史</a>&nbsp; <a href="/SiKuQuanShu/"><strong>四库全书</strong></a>&nbsp;<a href="http://www.guoxuedashi.com/SiKuQuanShu/fanti/">繁体</a>
<br><a href="/so.php?sokey=%E7%BA%A2%E6%A5%BC%E6%A2%A6&kt=1">红楼梦</a> <a href="/a/1858x/">三国演义</a> <a href="/a/1038k/">水浒传</a> <a href="/a/1046t/">西游记</a> <a href="/a/1914o/">封神演义</a>
<br>
<a href="http://www.guoxuedashi.com/so.php?sokeygx=%E4%B8%87%E6%9C%89%E6%96%87%E5%BA%93&submit=&kt=1">万有文库</a> <a href="/a/780t/">古文观止</a> <a href="/a/1024l/">文心雕龙</a> <a href="/a/1704n/">全唐诗</a> <a href="/a/1705h/">全宋词</a>
<br><a href="http://www.guoxuedashi.com/so.php?sokeygx=%E7%99%BE%E8%A1%B2%E6%9C%AC%E4%BA%8C%E5%8D%81%E5%9B%9B%E5%8F%B2&submit=&kt=1"><strong>百衲本二十四史</strong></a>  <a href="http://www.guoxuedashi.com/so.php?sokeygx=%E5%8F%A4%E4%BB%8A%E5%9B%BE%E4%B9%A6%E9%9B%86%E6%88%90&submit=&kt=1"><strong>古今图书集成</strong></a>
<br>

<a href="http://www.guoxuedashi.com/so.php?sokeygx=%E4%B8%9B%E4%B9%A6%E9%9B%86%E6%88%90&submit=&kt=1">丛书集成</a> 
<a href="http://www.guoxuedashi.com/so.php?sokeygx=%E5%9B%9B%E9%83%A8%E4%B8%9B%E5%88%8A&submit=&kt=1"><strong>四部丛刊</strong></a>  
<a href="http://www.guoxuedashi.com/so.php?sokeygx=%E8%AF%B4%E6%96%87%E8%A7%A3%E5%AD%97&submit=&kt=1">說文解字</a> <a href="http://www.guoxuedashi.com/so.php?sokeygx=%E5%85%A8%E4%B8%8A%E5%8F%A4&submit=&kt=1">三国六朝文</a>
<br><a href="http://www.guoxuedashi.com/so.php?sokeytm=%E6%97%A5%E6%9C%AC%E5%86%85%E9%98%81%E6%96%87%E5%BA%93&submit=&kt=1"><strong>日本内阁文库</strong></a> <a href="http://www.guoxuedashi.com/so.php?sokeytm=%E5%9B%BD%E5%9B%BE%E6%96%B9%E5%BF%97%E5%90%88%E9%9B%86&ka=100&submit=">国图方志合集</a> <a href="http://www.guoxuedashi.com/so.php?sokeytm=%E5%90%84%E5%9C%B0%E6%96%B9%E5%BF%97&submit=&kt=1"><strong>各地方志</strong></a>

</div>
</div>


<div class="sidebar2">
<center>

</center>
</div>
<div class="sidebar greenbar">
<div class="sidebar_title green">四库全书</div>
<div class="sidebar_info">

《四库全书》是中国古代最大的丛书,编撰于乾隆年间,由纪昀等360多位高官、学者编撰,3800多人抄写,费时十三年编成。丛书分经、史、子、集四部,故名四库。共有3500多种书,7.9万卷,3.6万册,约8亿字,基本上囊括了古代所有图书,故称“全书”。<a href="http://www.guoxuedashi.com/SiKuQuanShu/">详细>>
</a>

</div> 
</div>

</div>  <!--end r-->

</div>
<!-- 内容区END --> 

<!-- 页脚开始 -->
<div class="shh">

</div>

<div class="w1180" style="margin-top:8px;">
<center><script src="http://www.guoxuedashi.com/img/plus.php?id=3"></script></center>
</div>
<div class="w1180 foot">
<a href="/b/thanks.php">特别致谢</a> | <a href="javascript:window.external.AddFavorite(document.location.href,document.title);">收藏本站</a> | <a href="#">欢迎投稿</a> | <a href="http://www.guoxuedashi.com/forum/">意见建议</a> | <a href="http://www.guoxuemi.com/">国学迷</a> | <a href="http://www.shuowen.net/">说文网</a><script language="javascript" type="text/javascript" src="https://js.users.51.la/17753172.js"></script><br />
  Copyright &copy; 国学大师 古典图书集成 All Rights Reserved.<br>
  
  <span style="font-size:14px">免责声明:本站非营利性站点,以方便网友为主,仅供学习研究。<br>内容由热心网友提供和网上收集,不保留版权。若侵犯了您的权益,来信即刪。scp168@qq.com</span>
  <br />
ICP证:<a href="http://www.beian.miit.gov.cn/" target="_blank">鲁ICP备19060063号</a></div>
<!-- 页脚END --> 
<script src="http://www.guoxuedashi.com/img/plus.php?id=22"></script>
<script src="http://www.guoxuedashi.com/img/tongji.js"></script>

</body>
</html>
