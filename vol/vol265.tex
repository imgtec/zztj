






























































資治通鑑卷二百六十五 宋 司馬光 撰

胡三省 音註

唐紀八十一|{
	起閼逢困敦五月盡柔兆攝提格凡二年有奇}


昭宗聖穆景文孝皇帝下之下

天祐元年五月丙寅加河陽節度使張漢瑜同平章事帝宴朱全忠及百官於崇勲殿|{
	時以洛陽宫前殿為貞觀殿内朝為崇勲}


|{
	殿}
既罷復召全忠宴於内殿|{
	復扶又翻}
全忠疑不入帝曰全忠不欲來可令敬翔來全忠擿翔使去曰翔亦醉矣|{
	全忠疑帝欲圖已敬翔其腹心也故亦不使之入擿它狄翻}
辛未全忠東還|{
	還從宣翻又如字}
乙亥至大梁 忠義節度使趙匡凝遣水軍上峽攻王建夔州|{
	趙匡凝以襄陽之甲窺夔門夔在三峽上游泝流攻之故曰上峽上時掌翻}
知渝州王宗阮等擊敗之萬州刺史張武作鐵絙絶江中流立柵於兩端謂之鏁峽|{
	敗補邁翻絙古恒翻鏁即鎖字}
六月李茂貞王建李繼徽傳檄合兵以討朱全忠全忠以鎮國節度使朱友裕為行營都統將步騎擊之|{
	當是時蜀兵不出朱全忠之兵力不能及也令朱友裕擊岐邠耳}
命保大節度使劉鄩弃鄜州引兵屯同州|{
	劉鄩在鄜州逼近李茂貞繼徽聲援不接故使弃鄜還屯同州與朱友裕合埶鄜音夫}
癸丑全忠引兵自大梁西討茂貞等秋七月甲子過東都入見|{
	見賢遍翻}
壬申至河中 西川諸將勸王建乘李茂貞之衰攻取鳳翔建以問節度判官馮涓|{
	涓圭淵翻}
涓曰兵者凶器殘民耗財不可窮也|{
	言不可窮兵極其兵力好戰不休是窮兵也}
今梁晉虎爭勢不兩立|{
	梁朱全忠晉李克用}
若併而為一舉兵向蜀雖諸葛亮復生不能敵矣|{
	復扶又翻}
鳳翔蜀之藩蔽不若與之和親結為婚姻無事則務農訓兵保固疆場|{
	易音亦}
有事則覘其機事觀舋而動可以萬全|{
	覘丑廉翻又丑艷翻舋許覲翻}
建曰善茂貞雖庸才然有強悍之名|{
	悍下罕翻又侯旰翻}
遠近畏之與全忠力爭則不足自守則有餘使為吾藩蔽所利多矣乃與茂貞修好|{
	王建既併山南諸州阻關而守關外倚李茂貞為藩蔽故與之修好好呼到翻}
丙子茂貞遣判官趙鍠如西川為其姪天雄節度使繼勲求昏|{
	鍠戶盲翻為于偽翻此天雄軍治秦州屬李茂貞}
建以女妻之|{
	妻七細翻}
茂貞數求貨及甲兵於建建皆與之|{
	墮軍實以厚寇讐豈王建之本心哉倚以自蔽不厭其數也數所角翻}
王建賦斂重人莫敢言馮涓因建生日獻頌先美功德後言生民之苦建愧謝曰如君忠諫功業何憂賜之金帛自是賦斂稍損|{
	史言馮涓因獻頌而進規故其諫易入斂力贍翻}
初朱全忠自鳳翔迎車駕還|{
	見二百六十三卷天復三年還從宣翻又如字}
見德王裕眉目疎秀且年齒已壯惡之|{
	惡烏路翻全忠欲簒利立庸幼德王裕貌秀而齒長立之非已之利也故惡之}
私謂崔胤曰德王嘗奸帝位|{
	謂為劉季述所立也事見二百六十二卷光化三年天復元年}
豈可復留|{
	奸音于復扶又翻}
公何不言之胤言於帝帝問全忠全忠曰陛下父子之間臣安敢竊議此崔胤賣臣耳|{
	史言朱全忠之狡猾}
帝自離長安日憂不測與皇后終日沈飲或相對涕泣|{
	離力智翻是年正月壬戌帝離長安而東沈持林翻}
全忠使樞密使蔣玄暉伺察帝動靜皆知之|{
	伺相吏翻}
帝從容謂玄暉曰德王朕之愛子全忠何故堅欲殺之因泣下齧中指血流玄暉具以語全忠|{
	史言昭宗之輕脱以速禍從千容翻齧五結翻語牛倨翻}
全忠愈不自安時李茂貞楊崇本李克用劉仁恭王建楊行密趙匡凝移檄往來皆以興復為辭全忠方引兵西討|{
	西討岐邠}
以帝有英氣恐變生於中欲立幼君易謀禪代|{
	易以豉翻}
乃遣判官李振至洛陽與玄暉及左龍武統軍朱友恭右龍武統軍氏叔琮等圖之八月壬寅帝在椒殿|{
	椒殿皇后殿也史炤曰椒殿亦猶椒房之稱}
玄暉選龍武牙官史太等百人夜叩宫門言軍前有急奏|{
	軍前謂西討行營軍前也}
欲面見帝夫人裴貞一開門見兵曰急奏何以兵為史太殺之玄暉問至尊安在昭儀李漸榮臨軒呼曰|{
	呼火故翻}
寧殺我曹勿傷大家帝方醉遽起單衣繞柱走史太追而弑之|{
	年三十八}
漸榮以身蔽帝太亦殺之又欲殺何后后求哀於玄暉乃釋之|{
	何后祈生于蔣玄暉而卒以玄暉死屈節以苟歲月之生豈若以身殉昭宗之不失節也}
癸卯蔣玄暉矯詔稱李漸榮裴貞一弑逆宜立輝王祚為皇太子更名柷|{
	更工衡翻柷昌六翻}
監軍國事|{
	監古衘翻}
又矯皇后令太子於柩前即位宫中恐懼不敢出聲哭丙午昭宣帝即位時年十三 李克用復以張承業為監軍|{
	李克用匿張承業見上卷天復三年監古衘翻}
淮南將李神福攻鄂州未下|{
	天復三年李神福始攻鄂州天祐元年又攻卾州事並見上卷}
會疾病還廣陵楊行密以舒州團練使泌陽劉存代為招討使|{
	泌陽漢湖陽縣地後魏置石馬縣後訛為上馬貞觀元年廢開元十六年復割湖陽置上馬縣天寶元年改曰泌陽屬唐州宋白曰泌陽縣本漢舞隂縣地云云同上唐改泌陽以地在泌水之陽也唐州治焉泌兵媚翻}
神福尋卒宣州觀察使臺濛卒|{
	卒子恤翻}
楊行密以其子牙内諸軍使渥為宣州觀察使右牙都指揮使徐温謂渥曰王寢疾而嫡嗣出藩此必姦臣之謀它日相召非温使者及王令書慎無亟來|{
	諸侯下令于境内謂之令書以異於天子所下制詔敕之書也亟紀力翻為徐温召渥張本}
渥泣謝而行 九月己巳尊皇后為皇太后朱全忠引兵北屯永壽南至駱谷|{
	軍永壽所以致邠兵自此而南至駱谷所以致岐兵}
鳳翔邠寧兵竟不出辛未東還 冬十月辛卯朔日有食之 朱全忠聞朱友恭等弑昭宗陽驚號哭|{
	號戶刀翻}
自投於地曰奴輩負我令我受惡名於萬代癸巳至東都|{
	自軍前東還至東都}
伏梓宫慟哭流涕又見帝自陳非已志請討賊|{
	帝昭宣帝也賊弑君之賊}
先是護駕軍士有掠米於市者|{
	先悉薦翻}
甲午全忠奏朱友恭氏叔琮不戢士卒侵擾市肆|{
	以此為二人罪猶不敢昌然以弑逆之罪罪之}
友恭貶崖州司戶復姓名李彦威|{
	李彦威夀州人客汴州殖財任俠朱全忠愛而子之考異見二百五十九卷景福二年}
叔琮貶白州司戶尋皆賜自盡彦威臨刑大呼曰賣我以塞天下之謗|{
	呼火故翻塞悉則翻}
如鬼神何行事如此望有後乎丙申天平節度使張全義來朝丁酉復以全忠為宣武護國宣義天平節度使以全義為河南尹兼忠武節度使判六軍諸衛事|{
	朱全忠兼忠武張全義帥天平見上卷上年朱友恭氏叔琮既誅以全義領宿衛}
乙巳全忠辭赴鎮庚戌至大梁 鎮國節度使朱友裕薨於棃園|{
	死于梨園行營}
光州叛楊行密降朱全忠|{
	降戶江翻}
行密遣兵圍之與鄂州皆告急於全忠|{
	楊行密使其將劉存攻杜洪於鄂州}
十一月戊辰全忠自將兵五萬自潁州濟淮|{
	自潁州潁上縣取正陽濟淮}
軍于霍丘|{
	九域志霍丘縣在夀州東一百二十七里按元豐之壽州治下蔡}
分兵救鄂州淮南兵釋光州之圍還廣陵按兵不出戰全忠分命諸將大掠淮南以困之 錢鏐潜遣衢州羅城使葉讓殺刺史陳璋事泄|{
	錢鏐恨陳璋見二百六十三卷天復二年}
十二月璋斬讓而叛降于楊行密|{
	降戶江翻}
初馬殷弟賨性沈勇|{
	賨徂宗翻沈持林翻}
事孫儒為百勝指揮使|{
	以百戰百勝名軍}
儒死事楊行密屢有功遷黑雲指揮使行密嘗從容問其兄弟|{
	從千容翻}
乃知為殷之弟大驚曰吾常怪汝器度瓌偉|{
	瓌古回翻}
果非常人當遣汝歸賨泣辭曰賨淮西殘兵|{
	馬賨從秦宗權孫儒起于淮西故云然}
大王不殺而寵任之湖南地近嘗得兄聲問賨事大王久不願歸也行密固遣之是歲賨歸長沙行密親餞之郊賨至長沙殷表賨為節度副使它日殷議入貢天子賨曰楊王地廣兵彊|{
	楊行密封吳王故稱之}
與吾鄰接不若與之結好|{
	好呼到翻}
大可以為緩急之援小可通商旅之利殷作色曰楊王不事天子一旦朝廷致討罪將及吾汝置此論勿為吾禍|{
	史言馬殷畏朱全忠}
初清海節度使徐彦若遺表薦副使劉隱權留後|{
	事見二百六十二卷天復元年}
朝廷以兵部尚書崔遠為清海節度使遠至江陵聞嶺南多盜且畏隱不受代不敢前朝廷召遠還隱遣使以重賂結朱全忠乃奏以隱為清海節度使|{
	史言劉隱自託于朱全忠}


昭宣光烈孝皇帝|{
	諱祚即位更名柷昭宗第九子後唐明宗天成三年立廟于曹州四年乃追崇諡號}


天祐二年春正月朱全忠遣諸將進兵逼壽州|{
	是時壽州治壽春朱全忠自霍丘遣諸將進逼之}
潤州團練使安仁義勇決得士心故淮南將王茂章攻之踰年不克|{
	王茂章攻潤州事始上卷天復三年八月}
楊行密使謂之曰汝之功吾不忘也|{
	安仁義歸楊行密破趙鍠孫儒平宣潤皆有功}
能束身自歸當以汝為行軍副使但不掌兵耳仁義不從茂章為地道入城遂克之仁義舉族登樓衆不敢逼|{
	安仁義在淮南軍中號最善射衆憚之故不敢逼}
先是攻城諸將見仁義輒罵之|{
	先悉薦翻}
惟李德誠不然至是仁義召德誠登樓謂曰汝有禮吾今以為汝功且以愛妾贈之德誠掖之而下并其子斬于廣陵市|{
	田頵朱延壽安仁義淮南諸將中之鎗鎗者也三叛連衡不足以病楊行密朞年之餘相次禽殄行密未易才也}
兩浙兵圍陳詢于睦州|{
	陳詢叛錢鏐事始上卷天復三年}
楊行密遣西南招討使陶雅將兵救之軍中夜驚士卒多踰壘亡去左右及禆將韓球奔告之雅安臥不應須臾自定亡者皆還|{
	史言御衆之術惟静足以制動}
錢鏐遣其從弟鎰及指揮使顧全武王球禦之為雅所敗|{
	從才用翻鎰夷質翻敗補邁翻}
虜鎰及球以歸 庚午朱全忠命李振知青州事代王師範|{
	朱全忠寛西顧之虞然後命李振代王師範}
全忠圍壽州州人閉壁不出全忠乃自霍丘引歸|{
	朱全忠遣諸將逼毒州城下而留屯霍丘為後勢}
二月辛卯至大梁|{
	霍丘至大梁九百餘里}
李振至青州王師範舉族西遷至濮陽素服乘驢而進|{
	至濮陽已入朱全忠巡屬故囚服乘驢以請罪濮博木翻}
至大梁全忠客之表李振為青州留後 戊戌以安南節度使同平章事朱全昱為太師致仕全昱全忠之兄也戇樸無能|{
	戇竹巷翻}
先領安南全忠自請罷之 是日社|{
	自古以來以戊日社戊土也立春以後歷五戊則社日}
全忠使蔣玄暉邀昭宗諸子德王裕棣王祤䖍王禊沂王禋遂王禕景王祕祁王琪雅王|{
	之人翻}
瓊王祥置酒九曲池|{
	九曲池在洛苑中}
酒酣悉縊殺之投尸池中 朱全忠遣其將曹延祚將兵與杜洪共守鄂州庚子淮南將劉存攻拔之|{
	天復二年正月淮南兵攻鄂州踰兩朞而後克}
執洪延祚及汴兵千餘人送廣陵悉誅之|{
	僖宗光啓二年杜洪據鄂州至是而亡}
行密以存為鄂岳觀察使 已酉葬聖穆景文孝皇帝於和陵|{
	和陵在河南緱氏縣懊來山是年更名太平山}
廟號昭宗 三月庚午以王師範為河陽節度使 戊寅以門下侍郎同平章事獨孤損同平章事充靜海節度使|{
	罷獨孤損政事耳靜海軍治交州在嶺海之外損安得至邪}
以禮部侍郎河間張文蔚同平章事|{
	蔚紆勿翻}
甲申以門下侍郎同平章事裴樞為左僕射崔遠為右僕射並罷政事初柳璨及第不四年為宰相性傾巧輕佻|{
	佻它彫翻}
時天子左右皆朱全忠腹心璨曲意事之同列裴樞崔遠獨孤損皆朝廷宿望意輕之璨以為憾和王傅張廷範|{
	和王福亦昭宗之子}
本優人有寵于全忠奏以為太常卿樞曰廷範勲臣幸有方鎮|{
	言幸有方鎮可以處之}
何藉樂卿|{
	太常卿掌禮樂故曰樂卿言勲人處之之職當各當其分不藉樂卿以榮}
恐非元帥之旨|{
	朱全忠時為諸道元帥故稱之}
持之不下|{
	下戶嫁翻}
全忠聞之謂賓佐曰吾常以裴十四器識真純不入浮薄之黨|{
	裴樞第十四}
觀此議論本態露矣|{
	言其猶持清議也}
璨因此并遠損譖於全忠故三人皆罷以吏部侍郎楊涉同平章事涉收之孫也|{
	楊收見懿宗紀為相以罪貶死}
為人和厚恭謹聞當為相與家人相泣謂其子凝式曰此吾家之不幸也必為汝累|{
	累良瑞翻}
加清海節度使劉隱同平章事壬辰河東都押牙蓋寓卒遺書勸李克用省營繕薄賦斂求賢俊|{
	史言蓋寓垂歿不忘效忠於李克用蓋古盍翻姓也省所景翻}
夏四月庚子有彗星出西北|{
	彗祥歲翻又徐醉翻又音歲}
淮南將陶雅會衢睦兵攻婺州|{
	光化三年田頵取婺州既而頵為楊行密所攻錢鏐取婺州使沈夏守之}
錢鏐使其弟鏢將兵救之|{
	鏢匹燒翻}
五月禮院奏皇帝登位應祀南郊敕用十月甲午行之|{
	為朱全忠殺柳璨蔣玄暉張本}
乙丑彗星長竟天|{
	彗所以除舊布新易姓之徵也薛居正五代史曰是年正月甲辰有彗出于北河貫文昌其長三文餘五月乙丑復出軒轅大角及于天市垣光耀嚴猛}
柳璨恃朱全忠之勢恣為威福會有星變占者曰君臣俱災宜誅殺以應之璨因疏其素所不快者於全忠曰此曹皆聚徒横議怨望腹非|{
	横戶猛翻非亦作誹}
宜以之塞災異|{
	塞悉則翻}
李振亦言於朱全忠曰朝廷所以不理良由衣冠浮薄之徒紊亂綱紀|{
	紊音問}
且王欲圖大事|{
	謂簒奪也}
此曹皆朝廷之難制者也不若盡去之|{
	去羌呂翻}
全忠以為然癸酉貶獨孤損為棣州刺史裴樞為登州刺史崔遠為萊州刺史乙亥貶吏部尚書陸扆為濮州司戶工部尚書王溥為淄州司戶庚辰貶太子太保致仕趙崇為曹州司戶兵部侍郎王贊為濰州司戶|{
	濰音維唐武德二年分青州北海縣置濰州八年州廢以北海還屬青州此時蓋復置濰州也}
自餘或門胄高華或科第自進居三省臺閣以名檢自處|{
	處昌呂翻}
聲迹稍著者皆指為浮薄貶逐無虛日搢紳為之一空|{
	為于偽翻}
辛巳再貶裴樞為瀧州司戶|{
	劉昫曰瀧州治瀧水縣本漢端溪縣地晉分端溪立龍鄉縣隋改龍鄉為平原縣又改為瀧水唐平蕭銑置瀧州瀧閭江翻}
獨孤損為瓊州司戶崔遠為白州司戶 甲申忠義節度使趙匡凝遣使修好於王建|{
	趙匡凝東結淮南西通巴蜀欲交隣以抗朱全忠也適以動朱全忠之兵好呼到翻}
六月戊子朔敕裴樞獨孤損崔遠陸扆王溥趙崇王贊等並所在賜自盡時全忠聚樞等及朝士貶官者三十餘人於白馬驛|{
	白馬驛在滑州白馬縣}
一夕盡殺之投尸于河初李振屢舉進士竟不中第|{
	中竹仲翻}
故深疾搢紳之士言於全忠曰此輩常自謂清流宜投之黄河使為濁流全忠笑而從之振每自汴至洛朝廷必有竄逐者時人謂之鴟梟|{
	朝直遥翻下同梟堅堯翻}
見朝士皆頤指氣使|{
	以頤指麾以氣使令言其怙朱全忠之勢而肆其驕豪也}
旁若無人全忠嘗與僚佐及遊客坐於大柳之下全忠獨言曰此柳宜為車轂|{
	轂古鹿翻}
衆莫應有遊客數人起應曰宜為車轂全忠勃然厲聲曰書生輩好順口玩人皆此類也|{
	好呼到翻言順口附和以玩狎人}
車轂須用夾榆|{
	說文榆白枌所謂夾榆乃今之田榆也生田塍間其皮類槐其肉理堅緻而赤鋸以為器堅而耐久車轂衆輻所湊其木宜堅緻者呂俛曰櫅榆宜作車轂爾雅云白棗也櫅相稽翻}
柳木豈可為之顧左右曰尚何待左右數十人捽言宜為車轂者悉撲殺之|{
	捽昨没翻撲弼角翻}
己丑司空致仕裴贄貶青州司戶尋賜死柳璨餘怒所注猶不啻十數張文蔚力解之乃止時士大夫避亂多不入朝壬辰敕所在州縣督遣無得稽留前司勲員外郎李延古德裕之孫也去官居平泉莊|{
	李德裕有平泉莊在河南府界德裕平泉記曰先公眺想屬注伊川吾于是有退居河洛之志於龍門得喬處士故居翦荆棘驅狐狸而為之康駢曰平泉莊去洛城三十里}
詔下未至責授衛尉寺主簿秋七月癸亥太子賓客致仕柳遜貶曹州司馬 庚午夜天雄牙將李公佺與牙軍謀亂羅紹威覺之公佺焚府舍剽掠奔滄州|{
	為羅紹威誅牙將張本佺且緣翻剽匹妙翻時劉守文據滄州}
八月王建遣前山南西道節度使王宗賀等將兵擊昭信節度使馮行襲於金州|{
	馮行襲附朱全忠}
朱全忠以趙匡凝東與楊行密交通西與王建結昏乙未遣武寧節度使楊師厚將兵撃之己亥全忠以大軍繼之 |{
	考異曰梁太祖實録薛居正五代史梁記皆云七月庚午遣楊師厚帥前軍討趙匡凝于襄州辛未帝南征唐實録七月全忠奏匡凝擅通好西川淮南又遣弟專領荆南請削奪官爵已遣都將楊師厚討之翌日全忠自帥軍以進編遺録八月壬辰先抽武寧楊師厚是日到乃議伐襄州帥趙匡凝乙未大發車徒委楊師厚總其軍政乙亥上領親從步騎繼大軍之後是夜宿尉氏今從之薛史太祖將圖禪代以匡凝兄弟並據藩鎮乃遣使先諭旨焉凝對使者流涕荅以受國恩深豈敢隨時妄有它志使者復命太祖大怒天祐二年秋七月遣楊師厚帥師討之辛未全忠南征表匡凝罪狀請削官爵按全忠劫遷昭宗於洛陽匡凝與行密等移檄諸道共討之全忠安肯以禪代問之今不取}
處州刺史盧約使其弟佶攻䧟温州|{
	佶巨乙翻 考異曰新紀正月約陷温州十國紀年在此月戊戌今從之}
張惠奔福州|{
	犬復三年張惠據温州至是而敗王審知時據福州自温州南出平陽縣渡海浦即福州界九域志温州東南至福州界三百二十里自界首至福州五百二十里}
錢鏐遣方永珍救婺州|{
	淮南兵自正月攻婺州}
初禮部員外郎知制誥司空圖弃官居虞鄉王官谷|{
	王官谷在虞鄉縣中條山}
昭宗屢徵之不起柳璨以詔書徵之圖懼詣洛陽入見|{
	見賢遍翻}
陽為衰野墜笏失儀璨乃復下詔畧曰既養高以傲代類移山以釣名又曰匪夷匪惠難居公正之朝|{
	復扶又翻柳璨言司空圖既非伯夷之清又非柳下惠之和且朝政如彼而自謂公正通鑑直叙其辭而媺惡自見}
可放還山圖臨淮人也 楊師厚攻下唐鄧復郢隨均房七州|{
	七州皆忠義軍巡屬}
朱全忠軍于漢北九月辛酉命師厚作浮梁於隂谷口|{
	襄州穀城縣有隂城鎮按舊史隂谷口在襄州西六十里}
癸亥引兵度漢|{
	浮梁成而度}
甲子趙匡凝將兵二萬陳于漢濱|{
	陳讀曰陣}
師厚與戰大破之遂傅其城下|{
	傅讀如附}
是夕匡凝焚府城帥其族及麾下士沿漢奔廣陵|{
	僖宗中和四年趙德諲據襄州傳子匡凝至是而亡楊行密時據廣陵匡凝沿漢入江順流東下而奔歸之帥讀曰率下同}
乙丑師厚入襄陽丙寅全忠繼至匡凝至廣陵楊行密戲之曰君在鎮歲以金帛輸全忠今敗乃歸我乎匡凝曰諸侯事天子歲輸貢賦乃其職也豈輸賊乎|{
	輸舂遇翻}
今日歸公正以不從賊故耳行密厚遇之 丙寅封皇弟禔為潁王|{
	禔是支翻又是兮翻}
祐為蔡王 丁卯荆南節度使趙匡明帥衆二萬弃城奔成都|{
	天復二年趙匡凝遣匡明據有荆南匡凝既敗匡明亦走}
戊辰朱全忠以楊師厚為山南東道留後引兵擊江陵|{
	荆南軍府治江陵}
至樂鄉|{
	九域志江陵府長林縣有樂鄉鎮}
荆南牙將王建武遣使迎降全忠以都將賀瓌為荆南留後|{
	降戶江翻瓌古回翻}
全忠尋表師厚為山南東道節度使 王宗賀等攻馮行襲所向皆捷丙子行襲弃金州奔均州其將全師朗以城降|{
	僖宗大順二年馮行襲取金州至是而敗行襲遂歸于朱全忠九域志金州東至均州七百里 考異曰李昊蜀書高祖紀作全行思後主紀作全行宗林思諤王宗播王承規傳作全行宗桑弘志傳作全行朗新書馮行襲傳作金行全蓋傳寫差誤不可考正按後蜀後主實録云金州招安指揮使全師郁世居金州疑是師朗昆弟族人也今從十國紀年}
王建更師朗姓名曰王宗朗|{
	更工衡翻下詔更同}
補金州觀察使割渠巴開三州以隸之|{
	宋白曰渠州春秋巴國秦滅巴置巴郡漢為宕渠縣地蜀先主分巴郡置宕渠郡梁大同三年於郡理置渠州巴州亦漢宕渠地後漢分宕渠北界置漢昌縣今州理是也後魏於漢昌縣理置大谷郡又于郡北置巴州開州漢朐䏰縣地後漢建安二年分朐䏰西北界置漢豐縣後周置開江郡隋改郡為開州}
乙酉詔更用十一月癸酉親郊 淮南將陶雅陳璋拔婺州執刺史沈夏以歸楊行密以雅為江南都招討使歙婺衢睦觀察使|{
	楊行密本用陶雅為歙州}
以璋為衢婺副招討使璋攻暨陽|{
	此暨陽即越州諸暨縣也與婺州東陽縣接境}
兩浙將方習敗之|{
	敗補邁翻}
習進攻婺州 濠州團練使劉金卒楊行密以金子仁規知濠州 楊行密長子宣州觀察使渥|{
	長知兩翻下子長同}
素無令譽|{
	令善也令力正翻}
軍府輕之行密寢疾命節度判官周隱召渥隱性憃直|{
	憃書容翻愚也又涉降翻}
對曰宣州司徒輕易信讒喜擊毬飲酒|{
	楊渥時守宣州蓋加官司徒易以䜴翻喜許記翻}
非保家之主餘子皆幼未能駕馭諸將廬州刺史劉威從王起細微必不負王不若使之權領軍府|{
	考異曰按徐温謂隱為姧人隱若欲為亂當密召劉威豈肯對其父斥渥短請以軍府授威隱乃戇直之人耳}
俟諸子長以授之|{
	長知兩翻}
行密不應左右牙指揮使徐温張顥言於行密曰王平生出萬死冒矢石為子孫立基業|{
	為于偽翻}
安可使它人有之行密曰吾死瞑目矣|{
	瞑莫定翻閉目也}
隱舒州人也它日將佐問疾行密目留幕僚嚴可求衆出可求曰王若不諱如軍府何行密曰吾命周隱召渥今忍死待之可求與徐温詣隱隱未出見牒猶在案上可求即與温取牒遣使者如宣州召之|{
	為楊渥不終張本}
可求同州人也|{
	路振九國志曰嚴可求本馮翊人父實仕唐為江淮陸運判官由是家于江都}
行密以潤州團練使王茂章為宣州觀察使|{
	楊行密以宣州地接杭州使良將居之豈知楊渥與王茂章搆怨乎為茂章奔兩浙張本}
冬十月丙戌朔以朱全忠為諸道兵馬元帥别開幕

府|{
	别開元帥府}
是日全忠部署將士將歸大梁|{
	將自襄陽歸大梁}
忽變計欲乘勝擊淮南|{
	洛陽以丙戌除全忠諸道元帥全忠猶在行營以是日變計欲攻淮南}
敬翔諫曰今出師未踰月平兩大鎮|{
	謂荆襄兩鎮}
闢地數千里遠近聞之莫不震懾|{
	懾之涉翻}
此威望可惜不若且歸息兵俟舋而動|{
	敬翔知淮南之不可攻舋許觀翻}
不聽 改昭信軍為戎昭軍|{
	昭信軍本置于金州時已為王建所取}
辛卯朱全忠發襄州壬辰至棗陽|{
	棗陽縣屬隨州自襄陽至棗陽一百三十餘里}
遇大雨自申州抵光州|{
	宋白曰申州春秋之申國漢置平氏縣魏文帝立義陽郡宋立司州入魏改為郢州周武帝改郢州為申州光州春秋弦國漢為西陽縣魏置弋陽郡梁末于光城置光州北齊置南郢州後周為淮南郡隋復為光州九域志自申州東南至光州三百五十五里 考異曰梁太祖實録十月壬申上御大軍發自襄州由安黄涉申光暨夀春之霍丘駐焉十國紀年十月朱全忠自襄州帥衆二十萬趨光夀按十月丙戌朔無壬申梁實録誤今從編遺録}
道險狹塗潦人馬疲乏士卒尚未冬服多逃亡全忠使人謂光州刺史柴再用曰下我以汝為蔡州刺史|{
	柴再用汝陽人也故以衣錦啗之}
不下且屠城再用嚴設守備戎服登城見全忠拜伏甚恭曰光州城小兵弱不足以辱王之威怒王苟先下壽州敢不從命全忠留其城東旬日而去 起居郎蘇楷禮部尚書循之子也|{
	裴樞等既死而蘇循等進矣奉唐璽綬而輸之梁者此輩也}
素無才行|{
	行下孟翻}
乾寧中登進士第昭宗覆試黜之仍永不聽入科場|{
	洪邁隨筆曰昭宗乾寧三年試進士刑部尚書崔凝以下二十五人放榜詔于武德殿前覆試但放十五人自狀頭張貽範以下重落其六人許再入舉場四人所試最下不許再入蘇楷其一也唐人謂貢院為科場亦謂之場屋言由此而決科進取爭名之場也}
甲午楷帥同列上言諡號美惡臣子不得而私先帝諡號多溢美乞更詳議|{
	按舊書帝紀楷時帥起居郎羅衮起居舍人盧鼎上駮議楷目不知書僅能執筆其文羅衮作也帥讀曰率上時掌翻}
事下大常|{
	下戶嫁翻}
丁酉張廷範奏改諡恭靈莊愍孝皇帝廟號襄宗詔從之 楊渥至廣陵|{
	渥自宣州至廣陵}
辛丑楊行密承制以渥為淮南留後 戊申朱全忠發光州迷失道百餘里又遇雨比及壽州|{
	九域志光州東至壽州三百五十里比必利翻}
壽人堅壁清野以待之全忠欲圍之無林木可為柵乃退屯正陽|{
	淮水流出潁壽之間夾淮有正陽鎮東正陽屬壽州安豐縣界西正陽屬潁州潁上縣界}
癸丑更名成德軍曰武順|{
	以朱全忠父名誠故改成德為武順更工衡翻}
十一月丙辰朱全忠度淮而北柴再用抄其後軍|{
	抄楚交翻}
斬首三千級獲輜重萬計全忠悔之|{
	悔不用敬翔之言也}
躁忿尤甚|{
	躁則到翻}
丁卯至大梁先是全忠急於傳禪|{
	先悉薦翻}
密使蔣玄暉等謀之玄暉與柳璨等議以魏晉以來皆先封大國加九錫殊禮然後受禪當次第行之乃先除全忠諸道元帥以示有漸仍以刑部尚書裴迪為送官告使全忠大怒宣徽副使王殷趙殷衡疾玄暉權寵欲得其處|{
	蔣玄暉時為樞密使内專朝廷之權外結朱全忠之寵}
因譛之於全忠曰玄暉璨等欲延唐祚故逗遛其事以須變|{
	須待也}
玄暉聞之懼自至壽春具言其狀|{
	言朱全忠在壽春行營蔣玄暉懼罪故自往言狀}
全忠曰汝曹巧述閒事以沮我|{
	沮在呂翻止也}
借使我不受九錫豈不能作天子邪|{
	禪代之事先封大國次加九錫殊禮此王莽創為之也魏晉踵而行之諱其名而受其實魏文帝所謂舜禹之事吾知之矣其言雖不至如朱全忠之凶暴其欲簒之心則一也}
玄暉曰唐祚已盡天命歸王愚智皆知之玄暉與柳璨等非敢有背德|{
	背蒲妹翻}
但以今兹晉燕岐蜀皆吾勍敵|{
	晉李克用燕劉仁恭岐李茂貞蜀王建勍渠京翻}
王遽受禪彼心未服不可不曲盡義理然後取之欲為王創萬代之業耳|{
	為于偽翻}
全忠叱之曰奴果反矣玄暉惶遽辭歸與璨議行九錫時天子將郊祀百官既習儀|{
	唐制大祀百官皆先習儀受誓戒散齋致齋而後行事}
裴迪自大梁還|{
	裴迪先至壽春行營從朱全忠還大梁自大梁還洛陽}
言全忠怒曰柳璨蔣玄暉等欲延唐祚乃郊天也璨等懼庚午敕改用來年正月上辛殷衡本姓孔名循為全忠家乳母養子故冒姓趙後漸貴復其姓名 壬申趙匡明至成都|{
	正月丁卯弃荆南至是方至成都}
王建以客禮遇之昭宗之喪朝廷遣告哀使司馬卿宣諭王建至是始入蜀境西川掌書記韋莊為建謀|{
	為于偽翻下思為同}
使武定節度使王宗綰諭卿曰|{
	武定節度使治洋州蜀之東北鄙也故使諭卿}
蜀之將士世受唐恩去歲聞乘輿東遷凡上二十表|{
	乘繩證翻上時掌翻}
皆不報尋有士卒自汴來聞先帝已罹朱全忠弑逆蜀之將士方日夕枕戈|{
	枕職任翻}
思為先帝報仇不知今兹使來以何事宣諭舍人宜自圖進退|{
	司馬卿抑時為中書舍人歟否則唐制中書通事舍人掌受四方章奏及宣傳詔命今以卿將命出使故稱之歟}
卿乃還|{
	還從宣翻}
庚辰吳武忠王楊行密薨|{
	年五十四 考異曰十國紀年注吳録唐烈祖實録及吳史官王振撰楊本紀皆云天祐二年十一月庚辰行密卒敬翔梁編遺録云天祐三年三月潁州獲河東諜者言去年十一月持李克用絹書往淮南十二月至楊州方知楊行密已死與莊宗功臣列傳行密傳所載畧同沈顔行密神道碑殷文圭行密墓誌游恭渥墓誌皆云天祐三年丙寅二月十三日丙申卒薛居正五代史行密傳亦云天祐三年卒行密之亡嗣君幼弱不由朝命承襲或始死未敢發喪赴以明年二月疑沈顔等從而書之墓誌云十一月吳王寢疾付渥後事授淮南使或本紀等誤以此月為行密卒王振沈顔殷文圭游恭皆仕吳而記録差異固不可考今從舊史而存碑誌年月以廣傳聞}
將佐共請宣諭使李儼承制授楊渥淮南節度使東南諸道行營都統兼侍中弘農郡王|{
	楊行密請李儼承制見二百六十三卷天復二年渥字承天楊行密長子}
柳璨蔣玄暉等議加朱全忠九錫朝士多竊懷憤邑|{
	朝直遥翻}
禮部尚書蘇循獨揚言曰梁王功業顯大歷數有歸朝廷速宜揖讓朝士無敢違者辛巳以全忠為相國總百揆以宣武宣義天平護國天雄武順佑國河陽義武昭義保義戎昭武定泰寧平盧忠武匡國鎮國武寧忠義荆南等二十一道為魏國|{
	宣武領汴宋亳單宣義領汝鄭滑天平領鄆曹濮濟護國領河中晉絳慈隰天雄領魏博貝衛澶相武順領鎮冀深趙佑國領京兆商華河陽領孟懷義武領定祁易昭義軍領潞澤保義領邢洺磁戎昭領金均房武定領洋泰寧領兖沂密平盧領青淄齊棣登萊忠武領陳許匡國領同鎮國領陜虢武寧領徐宿忠義領襄鄧隨郢唐復安荆南領荆歸峽}
進封魏王仍加九錫全忠怒其稽緩讓不受十二月戊子命樞密使蔣玄暉齎手詔詣全忠諭指癸巳玄暉自大梁還言全忠怒不解甲午柳璨奏稱人望歸梁王陛下釋重負今其時也即日遣璨詣大梁逹傳禪之意全忠拒之初璨陷害朝士過多|{
	謂白馬之禍也}
全忠亦惡之|{
	惡烏路翻}
璨與蔣玄暉張廷範朝夕宴聚深相結為全忠謀禪代事|{
	為于偽翻}
何太后泣遣宫人阿䖍阿秋達意玄暉語以它日傳禪之後求子母生全|{
	阿烏葛翻語牛倨翻帝及德王裕皆何太后子也昭宗已弑裕與諸弟稍長相繼而死事已至此后之母子能獨全乎后素號多智臨難乃爾蓋當時以能隨時上下以全生者為智也}
王殷趙殷衡譛玄暉云與柳璨張廷範於積善堂夜宴對太后焚香為誓期興復唐祚|{
	何太后時居積善宫}
全忠信之乙未收玄暉及豐德庫使應頊御㕑使朱建武繫河南獄|{
	河南府獄}
以王殷權知樞密趙殷衡權判宣徽院事全忠三表辭魏王九錫之命丁酉詔許之|{
	朱全忠憤怒正欲殺蔣玄暉等乃復行魏晉之事表辭者敬翔教之也詔許之者王殷等承朱全忠之風指也}
更以為天下兵馬元帥然全忠已修大梁府舍為宫闕矣|{
	史誅其心迹以示天下後世}
是日斬蔣玄暉杖殺應頊朱建武庚子省樞密使及宣徽南院使獨置宣徽使一員以王殷為之趙殷衡為副使辛丑敕罷宫人宣傳詔命|{
	天復三年誅宦官以内夫人宣傳詔命考異見前}
及參隨視朝|{
	開元禮疏曰晉康獻禇后臨朝不坐則宫人傳命百僚周隋相因國家承之不改唐六典曰宫嬪司贊掌朝會贊相之事凡朝引客立于殿庭至天祐三年詔曰宫嬪女職本備内任今後遇延英坐日只令小黄門祇候引從宫人不得出内正是年詔敕也}
追削蔣玄暉為凶逆百姓令河南掲尸於都門外聚衆焚之|{
	河南河南府也揭其謁翻舉也}
玄暉既死王殷趙殷衡又誣玄暉私侍何太后令阿秋阿䖍通導往來己酉全忠密令殷殷衡害太后于積善宫敕追廢太后為庶人|{
	子而廢母是復晉峻陽之事也}
阿秋阿䖍皆於殿前撲殺|{
	撲弼角翻}
庚戌以皇太后喪廢朝三日|{
	既廢母為庶人又廢朝三日廢為庶人天性滅矣廢朝三日既非喪母之禮又不足以塞天性之傷唐之臣子非唐之臣子也朝直遥翻}
辛亥敕以宫禁内亂罷來年正月上辛謁郊廟禮|{
	唐不復郊矣}
癸丑守司空兼門下侍郎同平章事柳璨貶登州刺史太常卿張廷範貶萊州司戶甲寅斬璨於上東門外車裂張廷範於都市|{
	自罷謁郊廟以下皆朱全忠之夙心}
璨臨刑呼曰負國賊柳璨死其宜矣|{
	呼火故翻}
西川將王宗朗不能守金州焚其城邑奔成都|{
	王宗朗守金州纔三月耳}
戎昭節度使馮行襲復取金州奏請金州荒殘乞徙理均州從之更以行襲領武安軍 |{
	考異曰實録云改為武寧軍新表云改為武定軍按武寧乃徐州軍額武定乃洋州軍額不應同名續寶運録注云天復七年秋汴軍都頭號馮青面改姓朱授全忠印綬為洋州刺史授當作受洋州自景福元年刺史楊守佐歸順鳳翔後被朱全忠除此年秋蜀第二指揮使王宗綰收獲金州都押衙全貴帥衆降賜姓王名宗朗拜金州刺史又編遺録天祐三年二月云行襲巳于均州建節因署韓恭知金州事請朝廷落下防禦使并不建戎昭軍以此諸書參驗似是今者以行襲兼領洋州節制非改戎昭為武定軍實録新表皆誤續寶運録天復七年亦誤也 按考異則武安軍當作武定軍參考新舊書亦然}
陳詢不能守睦州奔于廣陵|{
	為兩浙兵所逼也僖宗中和四年陳晟據睦州至詢而敗}
淮南招討使陶雅入據其城 楊渥之去宣州也欲取其幄幕及親兵以行觀察使王茂章不與渥怒既襲位遣馬步都指揮使李簡等將兵襲之|{
	楊渥襲位曾幾何時而修怨于一州將其褊量如此固不足以君國子民}
湖南兵寇淮南淮南牙内指揮使楊彪擊却之

三年春正月壬戌靈武節度使韓遜奏吐蕃七千餘騎營於宗高谷將擊嗢未及取涼州|{
	趙珣聚米圖經曰靈武自賀蘭山路過西至涼州九百里}
李簡兵奄至宣州王茂章度不能守|{
	度徒洛翻}
帥衆奔兩浙|{
	帥讀曰率}
親兵上蔡刁彦能辭以母老不從行登城諭衆曰王府命我招諭汝曹|{
	楊渥父子皆以王爵鎮廣陵故稱淮南軍府為王府}
大兵行至矣衆由是定陶雅畏茂章斷其歸路|{
	斷音短}
引兵還歙州錢鏐復取睦州|{
	睦州自此屬錢氏楊氏不能爭歙書涉翻}
鏐以茂章為鎮東節度副使更名景仁|{
	更工衡翻}
乙丑加靜海節度使曲承裕同平章事|{
	曲承裕乘亂據有安南}
初田承嗣鎮魏博選募六州驍勇之士五千人為牙軍|{
	事見二百二十二卷代宗廣德元年}
厚其給賜以自衛為腹心自是父子相繼親黨膠固歲久益驕横|{
	横戶孟翻}
小不如意輒族舊帥而易之|{
	帥所類翻}
自史憲誠以來皆立於其手|{
	穆宗長慶二年立史憲誠文宗大和三年立何進滔懿宗咸通十一年立韓允中僖宗中和三年立樂彦禎文德元年立趙文㺹尋立羅弘信}
天雄節度使羅紹威心惡之力不能制|{
	惡烏路翻}
朱全忠之圍鳳翔也|{
	圍鳳翔見昭宗天復元年二年三年}
紹威遣軍將楊利言密以情告全忠欲借其兵以誅之全忠以事方急未暇如其請隂許之及李公佺作亂|{
	去年七月李公佺亂見上}
紹威益懼復遣牙將臧延範趣全忠|{
	趣讀曰促}
全忠乃發河南諸鎮兵七萬遣其將李思安將之會魏鎮兵屯深州樂城|{
	魏鎮魏博鎮冀兩鎮樂城恐當作樂夀}
聲言擊滄州討其納李公佺也會全忠女適紹威子廷規者卒|{
	卒子恤翻}
全忠遣客將馬嗣勲實甲兵於槖中|{
	有底曰囊無底曰槖槖撻各翻}
選長直兵千人為擔夫|{
	長直兵蓋選驍勇之士長使之直衛不以番代者也擔都濫翻}
帥之入魏|{
	帥讀曰率下同}
詐云會葬全忠自以大軍繼其後云赴行營牙軍皆不之疑庚午紹威潜遣人入庫斷弓弦甲襻|{
	斷音短襻普患翻}
是夕紹威帥其奴客數百與嗣勲合擊牙軍牙軍欲戰而弓甲皆不可用遂闔營殪之|{
	殪一計翻}
凡八千家嬰孺無遺詰旦全忠引兵入城|{
	詰去吉翻}
辛未以權知寧遠留後龎巨昭嶺南西道留後葉廣

畧並為節度使|{
	二人皆能保據本道因而命之}
庚辰錢鏐如睦州|{
	九域志杭州西南至睦州三百一十五里}
西川將王宗阮攻歸州獲其將韓從實|{
	歸州屬荆南}
陳璋聞陶雅歸歙自婺州退保衢州兩浙將方永珍等取婺州進攻衢州|{
	去年九月淮南兵取婺州陳璋本以衢州附淮南今自婺州退保之}
楊渥遣先鋒指揮使陳知新攻湖南三月乙丑知新拔岳州逐刺史許德勲|{
	昭宗天復三年湖南將許德勲取岳州今弃之}
渥以知新為岳州刺史|{
	為陳知新等覆軍張本}
戊寅以朱全忠為鹽鐵度支戶部三司都制置使三司之名始于此全忠辭不受 夏四月癸未朔日有食之 羅紹威既誅牙軍魏之諸軍皆懼紹威雖數撫諭之|{
	數所角翻}
而猜怨益甚朱全忠營於魏州城東數旬將北巡行營會天雄牙將史仁遇作亂聚衆數萬據高唐|{
	高唐漢古縣唐屬博州九域志在州東北一百一十里}
自稱留後天雄巡内諸縣多應之全忠移軍入城遣使召行營兵還攻高唐至歷亭|{
	歷亭縣屬貝州九域志在州東九十里宋白曰歷亭縣之地自後魏至高齊其地屬鄃縣隋開皇十六年于永濟渠南置歷亭縣遥取漢歷城縣為名按漢地理志歷城屬信都郡在蓚縣界王莽改曰歷亭唐萬歲登封元年移就盤河置在古歷城西七十里}
魏兵在行營者作亂與仁遇相應元帥府左司馬李周彞右司馬符道昭擊之所殺殆半進攻高唐克之城中兵民無少長皆死|{
	少詩照翻長知兩翻}
擒史仁遇鋸殺之先是仁遇求救於河東及滄州|{
	先悉薦翻}
李克用遣其將李嗣昭將三千騎攻邢州以救之時邢州兵纔二百團練使牛存節守之嗣昭攻七日不克全忠遣右長直都將張筠將數千騎助存節守城筠伏兵於馬嶺擊嗣昭敗之|{
	敗襧邁翻}
嗣昭遁去義昌節度使劉守文遣兵萬人攻貝州又攻冀州拔蓚縣進攻阜城|{
	蓚阜城並漢古縣唐屬冀州九域志蓚縣在州東北一百五十里阜城在州東一百六十里蓚音條}
時鎮州大將王釗攻魏州叛將李重霸於宗城|{
	宗城縣屬魏州九域志在州西北一百七十里}
全忠遣歸救冀州滄州兵去|{
	滄州兵即劉守文所遣}
丙午重霸弃城走汴將胡規追斬之 鎮南節度使鍾傳以養子延規為江州刺史傳薨軍中立其子匡時為留後延規恨不得立遣使降淮南 |{
	考異曰實録初鍾傳養上藍院僧為子曰延圭補江州刺史傳卒遂召淮師陷其城今從十國紀年吳史}
五月丁巳朱全忠如洺州遂巡北邊視戎備還入于魏 丙子廢戎昭軍并均房隸忠義軍|{
	併屬山南東道}
以武定節度使馮行襲為匡國節度使|{
	馮行襲自均州徙同州}
楊渥以昇州刺史秦裴為西南行營都招討使將兵擊鍾匡時於江西|{
	鍾延規啓之也}
六月甲申復以忠義軍為山南東道|{
	僖宗文德元年以山南東道為忠義軍}
朱全忠以長安鄰於邠岐數有戰爭|{
	九域志長安西北至邠州二百七十五里西至鳳翔三百九里數所角翻}
奏徙佑國節度使韓建於淄青|{
	韓建本與李茂貞連結者也朱全忠恐其復然故徙之}
以淄青節度使長社王重師為佑國節度使 秋七月朱全忠克相州時魏之亂兵散據貝博澶相衛州|{
	澶時連翻相息亮翻}
全忠分命諸將攻討至是悉平之引兵南還 |{
	考異曰實録在六月今從編遺録唐太祖紀年録編遺録七月癸未上起兵離魏都按是月壬子朔無癸未編遺録誤也}
全忠留魏半歲|{
	自正月入魏至是半歲}
羅紹威供億所殺牛羊豕近七十萬資糧稱是所賂遺又近百萬比去蓄積為之一空|{
	近其靳翻稱尺證翻遺唯季翻比必利翻為于偽翻}
紹威雖去其逼|{
	去羌呂翻}
而魏兵自是衰弱紹威悔之謂人曰合六州四十三縣鐵不能為此錯也|{
	魏州領貴鄉元城魏館陶冠氏莘朝城昌樂臨河洹水成安内黄宗城永濟十四縣博州領聊城博平武水清平堂邑高唐六縣相州領安陽鄴湯隂林慮堯城臨漳六縣衛州領汲衛共城新鄉黎陽五縣貝州領清河清陽武城經城臨清漳南歷亭夏津八縣澶州領頓丘清豐觀城臨黄四縣錯鑢也鑄為之又釋錯為誤羅以殺牙兵之誤取鑄錯為喻}
壬申全忠至大梁 |{
	考異曰編遺録云壬辰亦誤}
秦裴至洪州軍于蓼洲諸將請阻水立寨裴不從鍾匡時果遣其將劉楚據之諸將以咎裴裴曰匡時驍將獨楚一人耳若帥衆守城不可猝拔|{
	帥讀曰率下同}
吾故以要害誘致之耳|{
	誘音酉}
未幾裴破寨執楚|{
	楚居豈翻}
遂圍洪州饒州刺史唐寶請降|{
	降戶江翻}
八月乙酉李茂貞遣其子侃為質於西川|{
	質音致}
王建以侃知彭州 朱全忠以幽滄相首尾為魏患|{
	幽劉仁恭滄劉守文父子相為首尾}
欲先取滄州甲辰引兵發大梁 兩浙兵圍衢州|{
	此即方永珍之兵也}
衢州刺史陳璋告急於淮南楊渥遣左廂馬步都虞候周本將兵迎璋本至衢州浙人解圍陳於城下璋帥衆歸于本兩浙兵取衢州|{
	淮南與浙人爭婺睦衢三州至是方悉歸于錢氏陳讀曰陣帥讀曰率}
呂師造曰浙人近我而不動輕我也請擊之|{
	呂師造狃於青山之捷氣陵浙人近其靳翻}
本曰吾受命迎陳使君今至矣何為復戰彼必有以待我也遂引兵還本為之殿浙人躡之|{
	復扶又翻殿丁練翻躡尼輒翻}
本中道設伏大破之 九月辛亥朔朱全忠自白馬渡河丁卯至滄州軍於長蘆|{
	杜佑曰滄州長蘆縣漢參蘆縣地宋廢縣為長蘆鎮屬清池縣}
滄人不出羅紹威饋運自魏至長蘆五百里不絶於路又建元帥府舍於魏所過驛亭供酒饌幄幕什器|{
	饌雛皖翻又雛戀翻}
上下數十萬人無一不備|{
	羅紹威厚奉朱全忠不惟以報德亦懼因伐虢之便而取虞也}
秦裴拔洪州虜鍾匡時等五千人以歸|{
	僖宗中和二年鍾傳據洪州至匡時而亡}
楊渥自兼鎮南節度使以裴為洪州制置使|{
	淮南楊氏遂兼有江西之地}
靜難節度使楊崇本以鳳翔保塞彰義保義之兵攻夏州|{
	保義當作保大蓋保義軍領邢洺磁在山東而保大軍領鄜坊與邠岐等鎮皆在關西也難乃旦翻夏戶雅翻}
匡國節度使劉知俊邀擊坊州之兵斬首三千餘級擒坊州刺史劉彦暉|{
	坊州保大軍巡屬也以此證上文保義其誤明矣}
劉仁恭救滄州戰屢敗乃下令境内男子十五以上七十以下悉自備兵糧詣行營軍發之後有一人在閭里刑無赦或諫曰今老弱悉行婦人不能轉餉此令必行濫刑者衆矣乃命勝執兵者盡行|{
	勝音升}
文其面曰定霸都士人則文其腕或臂|{
	腕烏貫翻}
曰一心事主於是境内士民穉孺之外無不文者|{
	穉直利翻}
得兵十萬軍于瓦橋時汴軍築壘圍滄州鳥鼠不能通仁恭畏其強不敢戰城中食盡丸土而食或互相掠啖|{
	啖徒濫翻}
朱全忠使人說劉守文曰|{
	說式芮翻}
援兵勢不相及何不早降守文登城應之曰僕於幽州父子也梁王方以大義服天下若子叛父而來將安用之全忠愧其辭直為之緩攻|{
	為于偽翻}
冬十月丙戌王建始立行臺於蜀|{
	考異曰續寶運録曰天復六年十月六日行下此牓帖則是此年十月也}
建東向舞蹈號

慟|{
	號戶刀翻}
稱自大駕東遷|{
	謂昭宗遷洛也}
制命不通請權立行臺用李晟鄭畋故事承制封拜|{
	按李晟討朱泚屯東渭橋但請假禆佐趙光銑唐良臣張彧為洋利劒三州刺史以通蜀漢喉衿上不暇從也其後假張彧京兆少尹以調畿内芻米表李懷光降將孟涉段威勇以要官未嘗承制封拜也鄭畋便宜從事見二百五十四卷僖宗廣明元年}
仍以牓帖告諭所部藩鎮州縣 劉仁恭求救于河東前後百餘輩李克用恨仁恭返覆|{
	劉仁恭以幽州叛李克用又約朱全忠共攻之此克用之所深恨也}
竟未之許其子存朂諫曰今天下之勢歸朱温者什七八雖強大如魏博鎮定莫不附之自河以北能為温患者獨我與幽滄耳今幽滄為温所困我不與之併力拒之非我之利也夫為天下者不顧小怨且彼嘗困我而我救其急以德懷之乃一舉而名實附也此乃吾復振之時不可失也|{
	史言李存朂智識能輔其父所不逮}
克用以為然與將佐謀召幽州兵與攻潞州曰於彼可以解圍於我可以拓境乃許仁恭和召其兵仁恭遣都指揮使李溥將兵三萬詣晉陽克用遣其將周德威李嗣昭將兵與之共攻潞州 夏州告急於朱全忠戊戌全忠遣劉知俊及其將康懷英救之楊崇本將六鎮之兵五萬軍于美原|{
	據上文則楊崇本所將者五鎮之兵耳蓋併將秦隴之兵為六鎮}
知俊等擊之崇本大敗歸于邠州 武貞節度使雷彦威屢寇荆南留後賀瓌閉城自守|{
	去年九月汴將賀瓌守荆南}
朱全忠以為怯以潁州防禦使高季昌代之|{
	高季昌自此遂據有荆南}
又遣駕前指揮使倪可福將兵五千戍荆南以備吳蜀|{
	倪可福自此遂委質於高季昌}
朗兵引去|{
	朗兵雷彦威之兵也}
十一月劉知俊康懷貞|{
	以此觀之上文誤作懷英}
乘勝攻鄜延等五州下之加知俊同平章事以懷貞為保義節度使|{
	恐即命康懷頁以鄜畤保義當作保大以通鑑明年書保平節度使康懷貞證之又恐自是保義}
西軍自是不振|{
	西軍謂邠岐軍也}
湖州刺史高彦卒子澧代之|{
	澧音禮}
十二月乙酉錢鏐表薦行軍司馬王景仁詔以景仁領寧國節度使|{
	王景仁即王茂章是年正月弃宣州歸錢鏐 考異曰薛居正五代史鏐辟為兩府行軍司馬具以狀聞太祖復命遥領宣州節度使同平章事歐陽修五代史曰鏐表景仁領宣州節度使今從之}
朱全忠分步騎數萬遣行軍司馬李周彞將之自河陽救潞州 閏月乙丑廢鎮國軍興德府復為華州隸匡國節度割金商州隸佑國軍|{
	併同華為一鎮割金商以隸佑國皆欲厚其資力以扞邠岐}
初昭宗凶訃至潞州|{
	訃音赴告喪曰訃}
昭義節度使丁會帥將士縞素流涕久之|{
	帥讀曰率}
及李嗣昭攻潞州會舉軍降於河東 |{
	考異曰唐太祖紀年録丁酉丁會開門迎降閏十二月太祖以李嗣昭為潞帥薛居正五代史梁紀在閏月後唐紀在十二月今從新舊唐紀薛史梁紀及編遺録}
李克用以嗣昭為昭義留後會見克用泣曰會非力不能守也梁王陵虐唐室會雖受其舉拔之恩誠不忍其所為故來歸命耳|{
	無是非之心非人也丁會其有是非之心者乎}
克用厚待之位於諸將之上己巳朱全忠命諸軍治攻其將攻滄州|{
	治直之翻}
壬申聞潞州不守甲戌引兵還先是調河南北芻糧水陸輸軍前|{
	先悉薦翻調徒吊翻輸舂遇翻}
諸營山積全忠將還悉命焚之煙炎數里在舟中者鑿而沉之|{
	炎讀曰燄沈持林翻朱全忠舉兩河之兵力以攻劉守文滄州孤城破在旦夕遽以潞州内叛燒營而退者豈不知功壞于垂成哉蓋潞州天下之脊而河東之兵全忠之所素憚者也自潞州而南下太行直抵懷孟之郊可以進據洛都一正唐室全忠之簒事不成矣此其所以狼狽而返}
劉守文使遺全忠書曰王以百姓之故赦僕之罪解圍而去王之惠也城中數萬口不食數月矣與其焚之為煙沈之為泥願乞其餘以救之全忠為之留數囷以遺之|{
	劉守文之辭卑而情可矜故全忠之凶暴亦為之感動遺惟季翻乞音氣為于偽翻}
滄人賴以濟河東兵進攻澤州不克而退 吉州刺史彭玕遣使請降於湖南|{
	鍾氏既亡故彭玕請降于馬氏玕音千路振九國志作玕}
玕本赤石洞蠻酋鍾傳用為吉州刺史|{
	酋慈由翻}


資治通鑑卷二百六十五














































































































































