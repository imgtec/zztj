資治通鑑卷一百四十四
宋 司馬光 撰

胡三省 音註

齊紀十|{
	重光大荒落一年}


和皇帝|{
	諱寶融字智昭明帝第八子也}


中興元年|{
	是年三月始改元}
春正月丁酉東昏侯以晉安王寶義為司徒建安王寶寅為車騎將軍開府儀同三司|{
	東昏侯以永元三年紀年騎奇寄翻}
乙巳南康王寶融始稱相國|{
	相悉亮翻下同}
大赦以蕭頴胄為左長史蕭衍為征東將軍楊公則為湘州刺史|{
	去年楊公則取長沙因就用為湘州刺史}
戊申蕭衍發襄陽 |{
	考異曰梁高祖紀云二月戊申襄陽按戊申正月十三日梁紀誤也}
留弟偉總府州事憺守壘城|{
	壘城者築壘附近大城猶今堡寨也憺徒敢翻又徒濫翻}
府司馬莊丘黑守樊城|{
	莊丘黑盖為征東府司馬}
衍既行州中兵及儲偫皆虛|{
	偫直里翻積物以待用謂之偫}
魏興太守裴師仁齊興太守顏僧都並不受衍命舉兵欲襲襄陽偉憺遣兵邀擊於始平大破之|{
	齊分魏興郡東界鄖鄉錫二縣地為齊興郡沈約曰江左僑立始平郡治武當五代志曰浙陽郡武當縣舊僑置始平郡又置齊興郡則二郡皆置於今均州界宋白曰齊永明七年置齊興郡於均州鄖鄉縣守式又翻}
雍州乃安|{
	雍於用翻}
魏咸陽王禧為上相|{
	禧以太尉輔政位居羣臣之上故曰上相}
不親政務驕奢貪淫多為不法魏主頗惡之|{
	惡烏路翻}
禧遣奴就領軍于烈求舊羽林虎賁執仗出入|{
	舊字衍執仗出入每出入欲使之執兵翊衛賁音奔}
烈曰天子諒闇事歸宰輔|{
	闇音陰}
領軍但知典掌宿衛非有詔不敢違理從私禧奴惘然而返|{
	惘然失志貌惘音罔}
禧復遣謂烈曰|{
	復扶又翻}
我天子之叔父身為元輔有所求須|{
	意之所欲為須}
與詔何異烈厲色曰烈非不知王之貴也奈何使私奴索天子羽林|{
	索山客翻}
烈頭可得羽林不可得禧怒以烈為恒州刺史|{
	恒戶登翻}
烈不願出外固辭不許遂稱疾不出|{
	卧私第不出也}
烈子左中郎將忠領直閤|{
	北齊左右衛有直閤屬官有朱衣直閤直閤將軍直寢直齋直後之屬}
常在魏主左右烈使忠言於魏主曰諸王專恣意不可測宜早罷之自攬權綱北海王詳亦密以禧過惡白帝且言彭城王勰大得人情不宜久輔政|{
	勰音協}
帝然之時將礿祭|{
	宗廟之祭春曰礿礿余若翻薄也春物始生其祭尚薄}
王公並齋於廟東坊帝夜使于忠語烈明旦入見當有處分質明烈至|{
	語牛倨翻見賢遍翻處昌呂翻分扶問翻質正也質明天正明也}
帝命烈將直閤六十餘人宣旨召禧勰詳衛送至帝所|{
	將即亮翻}
禧等入見於光極殿|{
	光極殿魏孝文帝太和十九年所起以引見羣臣見賢遍翻}
帝曰恪雖寡昧沗承寶歷比纒尫疢|{
	魏主名恪見諸父自稱其名示謙挹也比毘至翻近也尫烏光翻弱也疢丑刃翻疾也}
實憑諸父苟延視息奄涉三齡諸父歸遜殷勤今便親攝百揆且還府司當别處分|{
	還府司謂各歸公府司存之所}
又謂勰曰頃來南北務殷不容仰遂冲操|{
	南北務殷謂使勰北鎮中山南取壽陽因而守之也冲謙也虚也冲操謙虚之操}
恪是何人而敢久違先敕|{
	先敕謂高祖遺敕見一百四十二卷東昏侯永元元年}
令遂叔父高蹈之意勰謝曰陛下孝恭仰遵先詔上成睿明之美下遂微臣之志感今惟往悲喜交深|{
	惟思也}
庚戌詔勰以王歸第禧進位太保|{
	進其位而奪之權}
詳為大將軍録尚書事|{
	為詳以專恣得罪張本}
尚書清河張彞邢巒聞處分非常亡走出洛陽城為御史中尉中山甄琛所彈|{
	甄之人翻彈徒丹翻}
詔書切責之復以于烈為領軍仍加車騎大將軍|{
	復扶又翻又如字}
自是長直禁中軍國大事皆得參焉魏主時年十六不能親决庶務委之左右於是倖臣茹皓|{
	茹音如}
趙郡王仲興上谷宼猛趙郡趙脩南陽趙邕及外戚高肇等始用事魏政浸衰趙脩尤親幸旬月間累遷至光禄卿每遷官帝親至其宅設宴王公百官皆從|{
	為後趙脩誅張本從才用翻}
辛亥東昏侯祀南郊大赦 丁巳魏主引見羣臣於太極前殿告以親政之意|{
	見賢遍翻}
壬戌以咸陽王禧領太尉廣陵王羽為司徒魏主引羽入内面授之羽固辭曰彥和本自不願而陛下強與之|{
	彭城王勰字彥和事見上卷上年強其兩翻}
今新去此官而以臣代之必招物議乃以為司空 二月乙丑南康王以冠軍長史王茂為江州刺史|{
	冠古玩翻}
竟陵太守曹景宗為郢州刺史邵陵王寶攸為荆州刺史甲戌魏大赦 壬午東昏侯遣羽林兵擊雍州中外纂嚴 甲申蕭衍至竟陵命王茂曹景宗為前軍以中兵參軍張法安守竟陵城茂等至漢口諸將議欲併兵圍郢分兵襲西陽武昌|{
	將即亮翻下同}
衍曰漢口不闊一里箭道交至|{
	謂船自中流而下敵人夾岸射之其箭交至也}
房僧寄以重兵固守與郢城為掎角|{
	掎居蟻翻}
若悉衆前進僧寄必絶我軍後悔無所及不若遣王曹諸軍濟江與荆州軍合以逼郢城吾自圍魯山以通沔漢|{
	沔即漢也一水二名}
使鄖城竟陵之粟方舟而下|{
	安陸春秋鄖子之國故曰鄖城鄖音云杜預曰江夏雲杜縣東南有鄖城劉昫曰郢州長壽縣古竟陵也方泭也舟船也詩云就其深矣方之舟之泭音桴}
江陵湘中之兵相繼而至兵多食足何憂兩城之不拔天下之事可以卧取之耳|{
	卧而取之言不煩力戰也}
乃使茂等帥衆濟江頓九里|{
	其地去郢城九里因以為名帥讀曰率}
張冲遣中兵參軍陳光靜開門迎戰茂等擊破之光靜死冲嬰城自守景宗遂據石橋浦連軍相續下至加湖|{
	加湖在江夏灄陽縣界湖水自北南注江去郢城三十里}
荆州遣冠軍將軍鄧元起軍主王世興田安之將數千人會雍州兵於夏首|{
	雍於用翻夏戶雅翻}
衍築漢口城以守魯山命水軍主義陽張惠紹等遊遏江中絶郢魯二城信使|{
	使疏吏翻}
楊公則舉湘州之衆會於夏口蕭頴胄命荆州諸軍皆受公則節度雖蕭頴逹亦隸焉府朝議欲遣人行湘州事而難其人|{
	南康王開相國府故曰府朝朝直遥翻}
西中郎中兵參軍劉坦謂衆曰湘土人情易擾難信|{
	易以豉翻}
用武士則侵漁百姓用文士則威略不振必欲鎮靜一州軍民足食無踰老夫乃以坦為輔國長史長沙太守行湘州事坦先嘗在湘州多舊恩迎者屬路|{
	劉坦傳先嘗在湘州盖客游也屬之欲翻}
下車選堪事吏分詣十郡|{
	湘州領長沙桂陽零陵衡陽營陽湘東邵陵始興臨賀始安十郡}
民運租米三十餘萬斛以助荆雍之軍由是資糧不乏三月蕭衍使鄧元起進據南堂西渚|{
	南堂在郢城南北盖射堂西近江渚}
田安之頓城北王世興頓曲水故城|{
	曲水故城盖郢府官僚袚禊之地在城東}
丁酉張冲病卒驍騎將軍薛元嗣與冲子孜及征虜長史江夏内史程茂共守郢城|{
	張冲自輔國將軍進征虜將軍以程茂為長史驍堅堯翻騎奇寄翻}
乙巳南康王即皇帝位於江陵 |{
	考異曰東昏紀云丁未南康王諱即皇帝位盖是日建康始聞之耳今從和帝紀及梁武帝紀}
改元大赦|{
	始改元為中興元年}
立宗廟南北郊州府城門悉依建康宫置尚書五省以南郡太守為尹以蕭頴胄為尚書令蕭衍為左僕射晉安王寶義為司空廬陵王寶源為車騎將軍開府儀同三司建安王寶寅為徐州刺史|{
	寶義寶源寶寅皆在建康遥授之耳}
散騎常侍夏侯詳為中領軍冠軍將軍蕭偉為雍州刺史丙午詔封庶人寶卷為涪陵王|{
	涪音浮}
乙酉以尚書令蕭頴胄行荆州刺史加蕭衍征東大將軍都督征討諸軍事假黄鉞時衍次楊口和帝遣御史中丞宗夬勞軍|{
	夬古賣翻勞力到翻}
寧朔將軍新城庾域諷夬曰黄鉞未加非所以總帥侯伯|{
	武王伐紂諸侯畢會至于牧野王左杖黄鉞右秉白旄以麾後世自魏武以下率加黄鉞孔安國曰黄鉞以黄金飾斧帥讀曰率下同}
夬返西臺|{
	江陵在西故曰西臺}
遂有是命薛元嗣遣軍主沈難當帥輕舸數千亂流來戰張惠紹等擊擒之|{
	横絶流而渡曰亂詩云涉渭為亂舸苦我翻}
癸丑東昏侯以豫州刺史陳伯之為江州刺史假節都督前鋒諸軍事西擊荆雍夏四月蕭衍出沔命王茂蕭頴逹等進軍逼郢城薛元嗣不敢出|{
	不敢出戰也}
諸將欲攻之衍不許|{
	衍欲持久以全力弊郢魯二城}
魏廣陵惠王羽通於員外郎馮俊興妻夜往為俊興所擊而匿之五月壬子卒 魏主既親政事嬖倖擅權王公希得進見|{
	見賢遍翻}
齊帥劉小苟屢言於禧云|{
	帥所類翻}
聞天子左右人言欲誅禧禧益懼乃與妃兄給事黄門侍郎李伯尚氐主楊集始楊靈祐乞伏馬居等謀反會帝出獵北邙禧與其黨會城西小宅欲發兵襲帝使長子通竊入河内舉兵相應|{
	長知兩翻}
乞伏馬居說禧還入洛城勒兵閉門|{
	說式芮翻}
天子必北走桑乾|{
	謂北歸平城也平城魏故都乾音干}
殿下可斷河橋為河南天子|{
	斷丁管翻}
衆情前却不壹禧心更緩自旦至晡猶豫不决遂約不泄而散楊集始既出即馳至北邙告之直寢符承祖薛魏孫與禧通謀|{
	當是時馮太后所幸宦者符承祖已死此又别一符承祖後魏孝文帝太和九年初置直齋直寢}
是日帝寢於浮圖之隂魏孫欲弑帝承祖曰吾聞殺天子者身當病癩|{
	癩音賴惡疾也}
魏孫乃止俄而帝寤集始亦至帝左右皆四出逐禽直衛無幾|{
	幾居豈翻}
倉猝不知所出左中郎將于忠曰臣父領軍留守京城|{
	守式又翻}
計防遏有備必無所慮帝遣忠馳騎觀之|{
	騎奇寄翻}
于烈已分兵嚴備使忠還奏曰臣雖老心力猶可用此屬猖狂不足為慮願陛下清蹕徐還以安物望帝甚悦自華林園還宫|{
	華林園魏明帝所築芳林園也後避齊王芳諱改曰華林園還從宣翻又如字}
撫于忠之背曰卿差彊人意禧不知事露與姬妾及左右宿洪池别墅|{
	洪池即漢之鴻池在洛陽東二十里田廬曰墅今人謂之别業晉人以來往往治池館觀游於其中墅承與翻}
遣劉小苟奉啟云檢行田收|{
	行下孟翻}
小苟至北邙已逢軍人怪小苟赤衣欲殺之小苟困廹言欲告反乃緩之或謂禧曰殿下集衆圖事見意而停|{
	言意趣已見而中止也見賢遍翻}
恐必漏泄今夕何宜自寛禧曰吾有此身應知自惜豈待人言又曰殿下長子已濟河兩不相知豈不可慮禧曰吾已遣人追之計今應還時通已入河内列兵仗放囚徒矣于烈遣直閤叔孫侯將虎賁三百人收禧|{
	將即亮翻賁音奔}
禧聞之自洪池東南走僮僕不過數人濟洛至柏谷塢追兵至擒之送華林都亭|{
	華林都亭盖在華林園門外}
帝面詰其反狀|{
	詰去吉翻}
壬戌賜死於私第同謀伏誅者十餘人諸子皆絶屬籍微給貲產奴婢自餘家財悉分賜高肇及趙脩之家其餘賜内外百官逮于流外|{
	雜色補官不入品者謂之流外官}
多者百餘匹下至十匹禧諸子乏衣食獨彭城王勰屢賑給之河内太守陸琇聞禧敗斬送禧子通首魏朝以琇於禧未敗之前不收捕通責其通情徵詣廷尉死獄中|{
	陸馛以傅孝文於受内禪之初福澤及其子至是其子敗矣勰音協賑津忍翻琇音秀朝直遥翻}
帝以禧無故而反由是益疎忌宗室 巴西太守魯休烈巴東太守蕭惠訓不從蕭頴胄之命惠訓遣子璝將兵擊頴胄|{
	璝古回翻}
頴胄遣汶陽太守劉孝慶屯峽口|{
	此西陵峽口也在宜都夷陵界夷陵今峽州也}
與巴東太守任漾之等拒之|{
	任音壬}
東昏侯遣軍主吳子陽陳虎牙等十三軍救郢州進屯巴口|{
	水經注巴水出廬江雩婁縣之下靈山亦曰巴山南流注于江謂之巴口今黄州之巴河口是也}
虎牙伯之之子也六月西臺遣衛尉席闡文勞蕭衍軍|{
	勞力到翻}
齎蕭頴胄等議謂衍曰今頓兵兩岸不併軍圍郢定西陽武昌取江州此機已失莫若請救於魏與北連和猶為上策衍曰漢口路通荆雍控引秦梁|{
	泝漢水而上至漢中秦梁二州刺史所治也故可以控引}
糧運資儲仰此氣息所以兵壓漢口連結數州今若併軍圍郢又分兵前進魯山必阻沔路搤吾咽㗋|{
	搤於革翻咽因肩翻}
若糧運不通自然離散何謂持久鄧元起近欲以三千兵往取尋陽彼若懽然知機一說士足矣|{
	說輸芮翻}
脱距王師|{
	脱或也脱者未可必之辭}
固非三千兵所能下也進退無據未見其可西陽武昌取之即得然既得之即應鎮守欲守兩城不減萬人糧儲稱是卒無所出|{
	稱尺證翻卒讀曰猝}
脱東軍有上者|{
	上時掌翻}
以萬人攻兩城兩城勢不得相救若我分兵應援則首尾俱弱如其不遣孤城必陷一城既没諸城相次土崩天下大事去矣若郢州既拔席卷沿流|{
	卷讀曰捲}
西陽武昌自然風靡何遽分兵散衆自貽憂患乎且丈夫舉事欲清天步|{
	天步天路也詩云天步艱難}
况擁數州之兵以誅羣小懸河注火奚有不滅豈容北面請救戎狄以示弱於天下彼未必能信徒取醜聲此乃下計何謂上策|{
	蕭衍以計可謂有英雄之略矣}
卿為我輩白鎮軍前途攻取但以見付事在目中無患不捷但借鎮軍靖鎮之耳|{
	蕭頴胄時為西臺尚書令盖加號鎮軍將軍為于偽翻下佑為同}
吳子陽等進軍武口|{
	武口武湖水出江之口水上通安陸之延頭今謂之沙武口張舜民曰武口在陽羅洑西北十餘里距汴京纔十八驛二廣湖湘皆由此而濟}
衍命軍主梁天惠等屯漁湖城唐脩期等屯白陽壘|{
	時築壘於白陽浦}
夾岸待之子陽進軍加湖 |{
	考異曰梁韋叡傳作茄湖今從齊梁帝紀}
去郢三十里傍山帶水築壘自固子陽舉烽城内亦舉火應之而内外各自保不能相救會房僧寄病卒衆復推助防張樂祖代守魯山|{
	復扶又翻下篹復祐復同助防者使之助城主防守因以為稱樂祖即去年張冲所遣助房僧寄者參考前後張當作孫}
蕭頴胄之初起也弟頴孚自建康出亡廬陵民脩靈祐為之聚兵得二千人襲廬陵克之内史謝篹奔豫章|{
	篹蘇管翻}
頴胄遣寧朔將軍范僧簡自湘州赴之僧簡拔安成|{
	吳孫皓寶鼎二年分豫章廬陵長沙立安成郡時屬江州劉昫曰吉州安福縣吳置安成郡九域志安福縣在吉州西一百二十里}
頴胄以僧簡為安成太守以頴孚為廬陵内史東昏侯遣軍主劉希祖將三千人擊之南康太守王丹以郡應希祖|{
	南康今之贛州}
頴孚敗奔長沙尋病卒謝篹復還郡希祖攻拔安成殺范僧簡東昏侯以希祖為安成内史脩靈祐復合餘衆攻謝篹篹敗走 東昏侯作芳樂苑|{
	樂音洛}
山石皆塗以五采望民家有好樹美竹則毁牆撤屋而徙之時方盛暑隨即枯萎朝暮相繼|{
	言徙樹竹者朝夕相繼也}
又於苑中立市使宮人宦者共為禆販|{
	禆益也買賤賣貴以自禆益故曰禆販}
以潘貴妃為市令東昏侯自為市録事小有得失妃則予杖|{
	予讀曰與}
乃敕虎賁不得進大荆實中荻|{
	大荆牡荆也俗謂之黄荆以為箠杖荻之實中者以箠人則重而痛楚虛中者差輕賁音奔}
又開渠立埭身自引船|{
	埭徒耐翻}
或坐而屠肉又好巫覡|{
	好呼到翻覡戶狄翻}
左右朱光尚詐云見鬼東昏入樂遊苑人馬忽驚以問光尚對曰曏見先帝大嗔|{
	樂音洛先時為曏嗔昌真翻怒也}
不許數出|{
	數所角翻}
東昏大怒拔刀與光尚尋之既不見乃縛菰為高宗形|{
	菰音孤雕胡也一名蔣江南人呼為茭草}
北向斬之縣首苑門|{
	縣讀曰懸}
崔慧景之敗也巴陵王昭胄永新侯昭頴出投臺軍各以王侯還第心不自安|{
	昭胄昭頴投慧景事見上卷上年永新縣屬安成郡吳立}
竟陵王子良故防閤桑偃為梅蟲兒軍副與前巴西太守蕭寅謀立昭胄昭胄許事克用寅為尚書左僕射護軍|{
	許之以僕射領護軍將軍}
時軍主胡松將兵屯新亭寅遣人說之曰須昏人出|{
	須待也以帝昏狂斥指為昏人說式芮翻}
寅等將兵奉昭胄入臺閉城號令|{
	將即亮翻}
昏人必還就將軍但閉壘不應則三公不足得也松許諾會東昏新作芳樂苑經月不出遊偃等議募健兒百餘人從萬春門入突取之昭胄以為不可偃同黨王山沙慮事久無成以事告御刀徐僧重寅遣人殺山沙於路吏於麝幐中得其事|{
	幐徒登翻囊可帶者曰幐山沙以盛麝香故曰麝幐猶今之香袋}
昭胄兄弟與偃等皆伏誅雍州刺史張欣泰與弟前始安内史欣時密謀結胡松及前南譙太守王靈秀直閤將軍鴻選等誅諸嬖倖廢東昏|{
	晉孝武帝僑立南譙郡於淮南五代志江都郡清流縣舊置南譙郡鴻姓也姓譜帝鴻氏之後或曰大鴻之後左傳衛有鴻聊魋雍於用翻嬖卑義翻又博計翻}
東昏遣中書舍人馮元嗣監軍救郢|{
	監工銜翻}
秋七月甲午茹法珍梅蟲兒及太子右率李居士制局監楊明泰送之中興堂|{
	宋孝武帝即位於新亭改新亭曰中興堂茹音如率所律翻}
欣泰等使人懷刀於座斫元嗣頭墜果柈中|{
	柈蒲官翻柈以盛果及魚肉}
又斫明泰破其腹蟲兒傷數創手指皆墮居士法珍等散走還臺靈秀詣石頭迎建康王寶寅|{
	建康王當作建安王}
帥城中將吏見力|{
	見力見在兵力也帥讀曰率見賢遍翻}
去車輪載寶寅文武數百唱警蹕向臺城百姓數千人皆空手隨之欣泰聞事作馳馬入宮冀法珍等在外東昏盡以城中處分見委表裏相應既而法珍得返處分閉門上仗不配欣泰兵鴻選在殿内亦不敢發寶寅至杜姥宅日已暝城門閉城上人射外人外人棄寶寅潰去寶寅亦逃三日乃戎服詣草市尉|{
	臺城六門之外各有草市置草市尉司察之去羌呂翻處昌呂翻分扶問翻上仗時掌翻射而亦翻}
尉馳以啟東昏東昏召寶寅入宮問之寶寅涕泣稱爾日不知何人逼使上車仍將去制不自由|{
	爾日猶言其日也上時掌翻}
東昏笑復其爵位張欣泰等事覺與胡松皆伏誅 蕭衍使征虜將軍王茂軍主曹仲宗等乘水漲以舟師襲加湖鼓譟攻之丁酉加湖潰吳子陽等走免將士殺溺死者萬計|{
	將即亮翻}
俘其餘衆而還|{
	還從宣翻}
於是郢魯二城相視奪氣乙巳柔然犯魏邊 魯山乏糧軍人於磯頭捕細魚

供食|{
	磯居希翻沙聚成磧水所漸浸曰磯}
密治輕船將奔夏口|{
	治直之翻夏戶雅翻}
蕭衍遣偏軍斷其走路|{
	斷音短}
丁巳孫樂祖窘迫以城降|{
	降戶江翻下同}
己未東昏侯以程茂為郢州刺史薛元嗣為雍州刺史|{
	雍於用翻}
是日茂元嗣以郢城降郢城之初圍也士民男女近十萬口閉戶二百餘日疾疫流腫|{
	流腫言毒氣流注而浮腫也近其郢翻}
死者什七八 |{
	考異曰齊張冲傳云死者七八百家按死者不可以家數今從梁高祖紀及韋叡傳}
積尸牀下而寢其上比屋皆滿|{
	比毗至翻周禮五家為比取其相連比而居也又毗必翻次也}
茂元嗣等議出降|{
	降戶江翻}
使張孜為書與衍張沖故吏青州治中房長瑜|{
	明帝時張沖為青冀二州刺史以房長瑜為治中}
謂孜曰前使君忠貫昊天郎君但當坐守畫一以荷析薪|{
	畫一用漢書語蕭何為法較若畫一曹參代之守而勿失此取守而勿失之義左傳曰其父析薪其子不克負荷荷下可翻又如字}
若天運不與當幅巾待命下從使君今從諸人之計非唯郢州士女失高山之望亦恐彼所不取也|{
	詩曰高山仰止注云有高德則慕而仰之彼謂蕭衍}
孜不能用蕭衍以韋叡為江夏太守行郢府事|{
	夏戶雅翻}
收瘞死者而撫其生者|{
	瘞於計翻}
郢人遂安諸將欲頓軍夏口衍以為宜乘勝直指建康車騎諮議參軍張弘策寧遠將軍庾域亦以為然衍命衆軍即日上道緣江至建康凡磯浦村落軍行宿次立頓處所弘策逆為圖畫如在目中|{
	郢魯未克蕭衍則違衆議駐兵漢口而不輕進圖萬全也郢魯既克衍遽督諸軍直指建康乘勝勢也逆為圖畫者畫緣江可立頓及次宿之地為圖使諸將按之以為進止上時掌翻}
辛酉魏大赦 魏安國宣簡侯王肅卒於壽陽|{
	安國縣漢屬中山國晉魏屬博陵郡}
贈侍中司空初肅以父死非命|{
	王奐死見一百三十八卷武帝永明十一年}
四年不除喪高祖曰三年之喪賢者不敢過|{
	記檀弓子夏既除喪而見予之琴和之而不和彈之而不成聲作而曰哀未忘也先王制禮而弗敢過也}
命肅以祥禫之禮除喪|{
	朞而小祥再朞而大祥大祥之後中月而禫鄭氏曰祥吉也禫澹澹然平安之意禫徒感翻釋曰除服祭名}
然肅猶素服不聽樂終身 汝南民胡文超起兵於灄陽|{
	沈約曰汝南大沙羨土晉末汝南郡民流寓夏口因立為汝南縣為江夏太守治所宋白曰晉汝南郡人流寓夏口因僑立汝南郡在潼口又為汝南縣晉末改為江夏縣荆湘記云金水北岸有汝南舊城是也晉惠帝世立灄陽縣晉書朱伺傳曰張昌之亂安陸人多附昌唯伺令其鄉人討之昌既滅伺部曲以逆順有嫌求别立縣遂從之分安陸東界立灄陽縣屬江夏郡灄書涉翻時汝南之地已入於魏蕭子顯齊志司州汝南郡寄治義陽}
以應蕭衍求取義陽安陸等郡以自効衍又遣軍主唐脩期攻隨郡|{
	晉武帝分南陽義陽立隨郡屬荆州宋孝武帝度屬郢州前廢帝永光元年改屬雍州明帝泰始五年改為隨陽郡還屬郢州後廢帝元徽四年度屬司州齊曰隨郡五代志隨州隨縣舊置隨郡}
皆克之司州刺史王僧景遣子為質於衍司部悉平|{
	司部謂司州所部諸郡質音致}
崔慧景之死也其少子偃為始安内史逃潜得免|{
	少詩照翻}
及西臺建以偃為寧朔將軍偃詣公車門上書曰臣竊惟高宗之孝子忠臣而昏主之亂臣賊子者江夏王與陛下先臣與鎮軍是也|{
	慧景既死江夏王寶玄併誅事見上卷上年夏戶雅翻}
雖成敗異術而所由同方陛下初登至尊與天合符天下纎芥之屈尚望陛下申之况先帝之子陛下之兄所行之道即陛下所由哉此尚不恤其餘何冀今不可幸小民之無識而罔之|{
	以非道欺人謂之罔}
若使曉然知其情節相帥而逃陛下將何以應之哉|{
	帥讀曰率}
事寢不報偃又上疏曰近冒陳江夏之寃非敢以父子之親而傷至公之義誠不曉聖朝所以然之意若以狂主雖狂實是天子江夏雖賢實是人臣先臣奉人臣逆人君為不可未審今之嚴兵勁卒直指象魏者|{
	象魏闕也}
其故何哉臣所以不死苟存視息|{
	人目不能視氣不復息則死矣}
非有他故所以待皇運之開泰申忠魂之枉屈今皇運已開泰矣而死社禝者反為賊臣臣何用此生於陛下之世矣臣謹案鎮軍將軍臣頴胄中領軍臣詳皆社禝之臣也同知先臣股肱江夏匡濟王室天命未遂主亡與亡而不為陛下瞥然一言|{
	瞥普蔑翻暫見也為于偽翻}
知而不言不忠不知而不言不智也如以先臣遣使江夏斬之則征東之驛使何為見戮陛下斬征東之使實詐山陽|{
	斬王天虎以詐山陽事見上卷上年使疏吏翻}
江夏違先臣之請實謀孔矜|{
	事亦見上卷上年}
天命有歸故事業不遂耳臣所言畢矣乞就湯鑊然臣雖萬没猶願陛下必申先臣何則惻愴而申之則天下伏不惻愴而申之則天下叛先臣之忠有識所知南董之筆千載可期|{
	南董謂齊南史晉董狐也崔杼弑齊莊公太史書曰崔杼弑其君崔子殺之其弟嗣書而死者二人其弟又書乃舍之南史氏聞太史盡死執簡以往聞既書矣乃還晉趙盾弟穿弑靈公董狐以盾不討賊書曰趙盾弑其君以示於朝孔子曰董狐古之良史也書法不隱載子亥翻}
亦何待陛下屈申而為褒貶然小臣惓惓之愚為陛下計耳詔報曰具知卿惋切之懷今當顯加贈諡偃尋下獄死|{
	惓逵員翻惋烏貫翻下遐嫁翻}
八月丁卯東昏侯以輔國將軍申胄監豫州事辛未以光禄大夫張瓌鎮石頭|{
	監工銜翻瓌古回翻}
初東昏侯遣陳伯之鎮江州以為吳子陽等聲援子陽等既敗蕭衍謂諸將曰用兵未必須實力所聽威聲耳今陳虎牙狼狽奔歸尋陽人情理當恟懼|{
	恟許拱翻}
可傳檄而定也乃命搜俘囚得伯之幢主蘇隆之|{
	幢傳江翻}
厚加賜與使說伯之許即用為安東將軍江州刺史|{
	即就也說式芮翻下同}
伯之遣隆之返命雖許歸附而云大軍未須遽下衍曰伯之此言意懷首鼠|{
	漢書田蚡曰首鼠兩端服䖍注云首鼠一前一卻也}
及其猶豫急往逼之計無所出勢不得不降|{
	降戶江翻}
乃命鄧元起引兵先下楊公則徑掩柴桑|{
	柴桑漢縣屬豫章郡晉屬武昌郡晉惠帝立尋陽郡冶柴桑五代志日江州湓城縣舊曰柴桑杜佑曰今尋陽縣南楚城驛舊柴桑縣也}
衍與諸將以次進路元起將至尋陽|{
	今江州德化縣六朝之尋陽也}
伯之收兵退保湖口|{
	湖口彭蠡湖入江之口也今江州湖口縣即其地}
留陳虎牙守湓城選曹郎吳興沈瑀說伯之迎衍|{
	選須絹翻瑀音禹}
伯之泣曰余子在都不能不愛瑀曰不然人情匈匈|{
	毛晃曰匈匈喧擾之意漢書高帝紀天下匈匈勞苦又匈匈讙議之聲荀子居子不為小人之匈匈而易其行匈匈漢書無音荀子有平去二音}
皆思改計若不早圖衆散難合丙子衍至尋陽伯之束甲請罪初新蔡太守席謙|{
	蕭子顯齊志江州有南新蔡郡豫州有北新蔡郡以五代志考之北新蔡當置於今光州界}
父恭祖為鎮西司馬為魚復侯子響所殺|{
	事見一百三十七卷武帝永明八年復音腹}
謙從伯之鎮尋陽聞衍東下曰我家世忠貞有殞不二伯之殺之乙卯以伯之為江州刺史虎牙為徐州刺史 魯休烈蕭璝破劉孝慶等於峽口任漾之戰死休烈等進至上明江陵大震蕭頴胄恐馳告蕭衍令遣楊公則還援根本衍曰公則今泝流上江陵雖至何能及事休烈等烏合之衆尋自退散正須少時持重耳|{
	上時掌翻少詩照翻}
良須兵力|{
	良信也}
兩弟在雍|{
	謂蕭偉總雍州事憺守壘城也雍於用翻}
指遣往徵|{
	指謂上指徵徵兵也}
不為難至頴胄乃遣蔡道恭假節屯上明以拒蕭璝 辛巳東昏侯以太子左率李居士總督西討諸軍事屯新亭|{
	左率左衛率也}
九月乙未詔蕭衍若定京邑得以便宜從事衍留驍騎將軍鄭紹叔守尋陽|{
	驍堅堯翻騎奇寄翻}
與陳伯之引兵東下謂紹叔曰卿吾之蕭何寇恂也|{
	漢高帝委蕭何以關中光武任寇恂以河内使給餽餉事並見漢紀}
前途不捷我當其咎糧運不繼卿任其責紹叔流涕拜辭比克建康|{
	比必利翻及也}
紹叔督江湘糧運未嘗乏絶 魏司州牧廣陽王嘉請築洛陽三百二十三坊各方三百步曰雖有暫勞姦盜永息丁酉詔畿内夫五萬人築之四旬而罷 己亥魏立皇后于氏后征虜將軍勁之女勁烈之弟也自祖父栗磾以來|{
	于栗磾魏開國功臣磾丁奚翻}
累世貴盛一皇后四贈公三領軍二尚書令三開國公 甲申東昏侯以李居士為江州刺史冠軍將軍王珍國為雍州刺史建安王寶寅為荆州刺史輔國將軍申胄監郢州龍驤將軍扶風馬仙琕監豫州|{
	冠古玩翻雍於用翻驤思將翻琕部田翻}
驍騎將軍徐元稱監徐州軍事珍國廣之之子也|{
	王廣之歷事高武明三帝}
是日蕭衍前軍至蕪湖申胄軍二萬人棄姑孰走衍進軍據之戊申東昏侯以後軍參軍蕭璝為司州刺史前輔國將軍魯休烈為益州刺史 蕭衍之克江郢也東昏遊騁如舊謂茹法珍曰須來至白門前當一决|{
	騁丑郢翻茹音如須待也白門建康城西門也西方色白故以為稱一决言一戰以决勝負也}
衍至近道乃聚兵為固守之計簡二尚方二冶囚徒以配軍|{
	建康有左右二尚方東西二冶}
其不可活者於朱雀門内日斬百餘人衍遣曹景宗等進頓江寧|{
	沈約曰晉武帝太康元年分秣陵立臨江縣二年更名江寧其治所盖臨江濱金陵覽古云新亭去江寧十里}
丙辰李居士自新亭選精騎一千至江寧|{
	騎奇寄翻}
景宗始至營壘未立且師行日久器甲穿弊居士望而輕之鼓譟直前薄之景宗奮擊破之因乘勝而前徑至皁莢橋於是王茂鄧元起呂僧珍進據赤鼻邏|{
	邏即佐翻}
新亭城主江道林引兵出戰衆軍擒之於陳|{
	陳讀曰陣}
衍至新林命王茂進據越城鄧元起據道士墩|{
	墩音敦}
陳伯之據籬門|{
	陳伯之盖據西籬門}
呂僧珍據白板橋|{
	據陶弘景書板橋時屬江寧縣界按板橋市今在建康府城之西江寧鎮北}
李居士覘知僧珍衆少帥鋭卒萬人直來薄壘|{
	覘丑亷翻又丑艶翻少詩照翻帥讀曰率}
僧珍曰吾衆少不可逆戰可勿遥射須至塹裏當併力破之俄而皆越塹拔柵|{
	塹七艶翻}
僧珍分人上城矢石俱自帥馬步三百人出其後城上復踰城而下内外奮擊居士敗走獲其器甲不可勝計|{
	上時掌翻復扶又翻下遐嫁翻勝音升}
居士請於東昏侯燒南岸邑屋以開戰場自大航以西新亭以北皆盡衍諸弟皆自建康自拔赴軍|{
	衍諸弟亡匿於建康里巷事見上卷上年}
冬十月甲戌東昏侯遣征虜將軍王珍國軍主胡虎牙將精兵十萬餘人陳於朱雀航南宦官王寶孫持白虎旛督戰開航背水以絶歸路衍軍小却王茂下馬單刀直前其甥韋欣慶執鐵纒矟以翼之|{
	鐵纒矟以鐵線纒矟把齊武陵王晃有銀纒矟將即亮翻陳讀曰陣下突陳同背蒲妹翻纒直彥翻矟色角翻}
衝擊東軍應時而陷曹景宗縱兵乘之呂僧珍縱火焚其營將士皆殊死戰鼓譟震天地珍國等衆軍不能抗王寶孫切罵諸將帥|{
	帥所類翻}
直閤將軍席豪發憤突陣而死豪驍將也既死士卒土崩赴淮死者無數積尸與航等後至者乘之而濟於是東昏侯諸軍望之皆潰|{
	據齊書云朱爵諸軍望之皆潰盖東昏侯自登朱爵門督戰也}
衍軍長驅至宣陽門諸將移營稍前陳伯之屯西明門|{
	西明門建康城西門也}
每城中有降人出伯之輒呼與耳語|{
	耳語附耳而語也降戶江翻下同}
衍恐其復懷翻覆密語伯之曰聞城中甚忿卿舉江州降欲遣刺客中卿宜以為慮|{
	復扶又翻語牛倨翻中竹仲翻}
伯之未之信會東昏侯將鄭伯倫來降衍使伯倫過伯之謂曰城中甚忿卿欲遣信誘卿以封賞|{
	誘音酉}
須卿復降當生割卿手足卿若不降復欲遣刺客殺卿宜深為備伯之懼自是始無異志|{
	蕭衍之使鄭伯倫此孫子五間所謂因間也須待也復扶又翻}
戊寅東昏寧朔將軍徐元瑜以東府城降青冀二州刺史桓和入援屯東宮己卯和詐東昏云出戰因以其衆來降光禄大夫張瓌棄石頭還宮李居士以新亭降於衍琅邪城主張木亦降壬午衍鎮石頭命諸軍攻六門東昏燒門内營署官府驅逼士民悉入宮城閉門自守衍命諸軍築長圍守之|{
	守式又翻 考異曰齊帝紀與梁帝紀叙此事先後多不同按齊紀皆有甲子今用梁紀事以齊紀甲子次之}
楊公則屯領軍府壘北樓與南掖門相對嘗登樓望戰城中遥見麾盖以神鋒弩射之矢貫胡牀左右失色公則曰幾中吾脚談笑如初|{
	射而亦翻幾居依翻中竹仲翻}
東昏夜選勇士攻公則柵軍中驚擾公則堅臥不起徐命擊之東昏兵乃退公則所領皆湘州人素號怯懦城中輕之每出盪輒先犯公則壘公則奬厲軍士克獲更多先是東昏遣軍主左僧慶屯京口常僧景屯廣陵李叔獻屯瓜步及申胄自姑孰奔歸使屯破墩|{
	據梁書鄱陽王恢傳破墩即破岡在曲阿界秦始皇所鑿也先悉薦翻墩音敦}
以為東北聲援至是衍遣使曉諭皆帥其衆來降|{
	史言東昏唯孤城自守使疏吏翻帥讀曰率}
衍遣弟輔國將軍秀鎮京口輔國將軍恢鎮破墩從弟寧朔將軍景鎮廣陵|{
	景本名昺李延壽作南史避唐廟諱改昺為景通鑑因之}
十一月丙申魏以驃騎大將軍穆亮為司空|{
	驃匹妙翻騎奇寄翻}
丁酉以北海王詳為太傅領司徒初詳欲奪彭城王勰司徒|{
	勰音協}
故譛而黜之既而畏人議已故但為大將軍至是乃居之詳貴盛翕赫將作大匠王遇多隨詳所欲私以官物給之|{
	李延壽曰王遇本馮翊李潤鎮羌其先為羌中強族自云姓王後改為鉗耳氏至魏宣武時復改為王坐事腐刑累遷吏部尚書爵宕昌公}
司徒長史于忠責遇於詳前曰殿下國之周公阿衡王室|{
	阿衡謂如伊尹也鄭玄曰阿倚也衡平也伊尹湯所依倚以取平故以為官名}
所須材用自應關旨|{
	關旨謂關上旨也}
何至阿諛附勢損公惠私也遇既踧踖|{
	踧昌六翻踖資昔翻踧踖恭而不自安之貌}
詳亦慙謝忠每以鯁直為詳所忿嘗罵忠曰我憂在前見爾死不憂爾見我死時也忠曰人生於世自有定分|{
	分扶問翻}
若應死於王手避亦不免若其不爾王不能殺忠以討咸陽王禧功封魏郡公遷散騎常侍兼武衛將軍|{
	散悉亶翻騎奇寄翻}
詳因忠表讓之際密勸魏主以忠為列卿令解左右|{
	常侍武衛之職常在天子左右}
聽其讓爵於是詔停其封優進太府卿|{
	詳能以計疎于忠而不知高肇已制其後矣}
巴東獻武公蕭頴胄以蕭璝與蔡道恭相持不决憂憤成疾|{
	蕭頴胄以蕭衍東伐所向戰克而已輔南康居江陵近不能制蕭璝外無以服姦雄之心而内有肘掖之宼此其所以憂憤成疾璝古回翻}
壬午卒|{
	卒子恤翻}
夏侯詳祕之使似其書者假為敎命密報蕭衍衍亦祕之詳徵兵雍州蕭偉遣蕭憺將兵赴之璝等聞建康已危衆懼而潰璝及魯休烈皆降乃發頴胄喪贈侍中丞相於是衆望盡歸於衍夏侯詳請與蕭憺共參軍國詔以詳為侍中尚書右僕射尋除使持節撫軍將軍荆州刺史詳固讓于憺乃以憺行荆州府州軍|{
	豈特衆望歸衍哉西臺之權又歸於憺矣憺徒敢翻又徒濫翻使疏吏翻}
魏改築圜丘於伊水之陽|{
	齊明帝建武二年魏孝文定圜丘於委粟山今改之水經伊水出南陽縣西荀渠山東北過陸渾新城縣又東北過伊闕中又東北至洛陽縣南而北入于洛魏盖立圜丘于洛陽之南伊水之北}
乙卯始祀於其上 魏鎮南將軍元英上書曰蕭寶卷荒縱日甚虐害無辜|{
	卷讀曰捲}
其雍州刺史蕭衍東伐秣陵掃土興兵順流而下唯有孤城更無重衛|{
	此謂襄陽空虛也}
乃皇天授我之日曠世一逢之秋此而不乘將欲何待臣乞躬帥步騎三萬直指沔隂|{
	襄陽在沔南水南為隂帥讀曰率}
據襄陽之城斷黑水之路|{
	斷丁管翻水經注黑水出南鄭北山南流入漢諸葛亮牋云朝南鄭暮宿黑水四五十里英盖謂得襄陽則梁州之路斷也}
昏虐君臣自相魚肉我居上流威震遐邇長驅南出進拔江陵則三楚之地一朝可收|{
	太史公曰楚有三俗自淮沛陳汝南南郡此西楚也彭城以東東海吳廣陵此東楚也衡山九江江南豫章長沙此南楚也}
岷蜀之道自成斷絶|{
	若取荆湘則岷蜀趣建康之道亦絶矣}
又命揚徐二州聲言俱舉|{
	魏揚州治壽陽徐州治彭城}
建業窮蹙魚游釡中可以齊文軌而大同混天地而為一伏惟陛下獨决聖心無取疑議此期脱爽|{
	爽差也}
并吞無日事寢不報車騎大將軍源懷上言蕭衍内侮寶卷孤危廣陵淮隂等戍皆觀望得失斯實天啟之期并吞之會宜東西齊舉以成席卷之勢若使蕭衍克濟上下同心豈唯後圖之難亦恐揚州危逼何則壽春之去建康纔七百里|{
	魏置揚州於壽春見上卷上年}
山川水陸皆彼所諳|{
	諳烏含翻}
彼若内外無虞君臣分定乘舟藉水倏忽而至未易當也|{
	分扶問翻易弋豉翻}
今寶卷都邑有土崩之憂邊城無繼援之望廓清江表正在今日魏主乃以任城王澄為都督淮南諸軍事鎮南大將軍開府儀同三司揚州刺史使為經略既而不果|{
	使魏從二臣之計畫江為境不待侯景之亂也任音壬}
懷賀之子也|{
	源賀秃髮傉檀之子入魏賜姓源氏}
東豫州刺史田益宗上表曰蕭氏亂常君臣交爭江外州鎮中分為兩|{
	謂西陽以西盡歸蕭衍歷陽以下猶屬建康也}
東西抗峙已淹歲時民庶窮於轉輸甲兵疲於戰鬪事救於目前力盡於麾下無暇外維州鎮綱紀庶方藩城棊立孤存而已不乘機電掃廓彼蠻疆恐後之經畧未易於此|{
	易以豉翻}
且壽春雖平三面仍梗鎮守之宜實須豫設義陽差近淮源利涉津要朝廷行師必由此道|{
	水經淮水出南陽平氏縣胎簪山東北過桐柏山又東逕義陽縣故曰義陽差近淮源淮源淺狹魏人行師以此地為利涉津要}
若江南一平有事淮外|{
	謂若蕭衍平定江南勢必用兵淮外}
須乘夏水汎長列舟長淮|{
	此謂江南用兵之常勢汎長知兩翻}
師赴壽春須從義陽之北|{
	此謂魏師赴壽春之路}
便是居我㗋要|{
	謂義陽也}
在慮彌深義陽之滅今實時矣度彼不過須精卒一萬二千|{
	度徒洛翻}
然行師之法貴張形勢請使兩荆之衆西擬隨雍|{
	兩荆謂魏置荆州於穰城東荆州於泚陽也隨雍謂隨郡襄陽也雍於用翻}
揚州之卒頓于建安得捍三關之援然後二豫之軍直據南關對抗延頭|{
	二豫謂魏置豫州於汝南東豫州於新息也南關謂隂山關延頭在安陸界}
遣一都督總諸軍節度季冬進師迄于春末不過十旬克之必矣 |{
	考異曰益宗傳曰世宗納之遣元英攻義陽按英攻義陽在景明四年八月此表言蕭氏君臣交爭則是梁武攻東昏時盖益宗建策於今日而行於後年耳}
元英又奏稱今寶卷骨肉相殘藩鎮鼎立義陽孤絶密邇王土内無兵儲之固外無糧援之期此乃欲焚之鳥不可去薪|{
	去羌呂翻}
授首之寇豈容緩斧若失此不取豈唯後舉難圖亦恐更為深患今豫州刺史司馬悦已戒嚴垂發東豫州刺史田益宗兵守三關請遣軍司為之節度魏主乃遣直寢羊靈引為軍司|{
	直寢因直寢殿以為官稱}
益宗遂入寇建寧太守黄天賜與益宗戰于赤亭|{
	宋有建寧左郡孝武大明八年省建寧左郡為建寧縣屬西陽郡後復為郡隋志黄州麻城縣舊置建寧郡又宋文帝元嘉二十五年以豫部蠻民置二十八縣赤亭其一也水經注舉水自湖陂城南流逕赤亭下謂之赤亭水西陽五水蠻赤亭其一也}
天賜敗績 |{
	考異曰魏帝紀七月乙未田益宗破蕭寶卷將黄天賜於赤亭田益光傳景明初蕭衍遣軍主吳子陽帥衆寇三關益宗遣光城太守梅與之據長風城逆擊子陽大破之斬獲千餘級按吳子陽乃東昏將非衍將也且衍方與東昏相拒何暇寇魏三關此必益宗傳誤益光傳當作益宗傳}
崔慧景之逼建康也東昏侯拜蔣子文為假黄钺使持節相國太宰大將軍録尚書事揚州牧鍾山王|{
	使疏吏翻}
及衍至又尊子文為靈帝迎神像入後堂使巫禱祀求福及城閉城中軍事悉委王珍國兖州刺史張稷入衛京師以稷為珍國之副稷瓌之弟也|{
	張瓌時為光禄大夫}
時城中實甲猶七萬人東昏素好軍陳與黄門刀敕及宮人於華光殿前習戰鬪詐作被創勢使人以板掆去用為猒勝|{
	好呼到翻陳讀曰陣被皮義翻創初良翻掆音岡猒於叶翻又於琰翻}
常於殿中戎服騎馬出入以金銀為鎧胄具裝飾以孔翠|{
	孔翠孔雀翡翠也鎧苦亥翻}
晝眠夜起一如平常聞外鼓叫聲被大紅袍登景陽樓屋上望之|{
	今建康法寶寺景陽樓故基也被皮義翻}
弩幾中之|{
	幾居依翻中竹仲翻}
始東昏與左右謀以為陳顯逹一戰即敗崔慧景圍城尋走謂衍兵亦然敕太官辦樵米為百日調而已|{
	調徒釣翻筭度也}
及大桁之敗衆情兇懼|{
	兇凶勇翻}
茹法珍等恐士民逃潰故閉城不復出兵|{
	復扶又翻}
既而長圍已立塹柵嚴固然後出盪屢戰不捷|{
	塹工艶翻}
東昏尤惜金錢不肯賞賜法珍叩頭請之東昏曰賊來獨取我耶何為就我求物後堂儲數百具榜|{
	榜比朗翻木片也}
啟為城防東昏欲留作殿竟不與又督御府作三百人精仗待圍解以擬屏除|{
	屏必郢翻}
金銀雕鏤雜物倍急于常衆皆怨怠不為致力外圍既久城中皆思早亡莫敢先發茹法珍梅蟲兒說東昏曰|{
	鏤盧侯翻為于偽翻說式芮翻}
大臣不留意使圍不解宜悉誅之王珍國張稷懼禍珍國密遣所親獻明鏡於蕭衍衍斷金以報之|{
	鏡所以照物獻鏡者欲衍照其心也易大傳曰二人同心其利斷金故衍取以為報斷丁亂翻王肅丁管翻}
兖州中兵參軍張齊稷之腹心也珍國因齊密與稷謀同弑東昏齊夜引珍國就稷造膝定計|{
	造七到翻造至也對席而坐兩下促席俱前至膝以定密謀故曰造膝定計}
齊自執燭又以計告後閤舍人錢強|{
	後閤舍人盖江左所置使主殿後閤者也按後閤舍人常在宮中觀徐龍駒事可見}
十二月丙寅夜強密令人開雲龍門珍國稷引兵入殿御刀豐勇之為内應|{
	豐勇之右衛軍人為東昏所委任姓譜豐姓鄭七穆子豐之後}
東昏在含德殿作笙歌寢未熟聞兵入趨出北戶欲還後宮門已閉宦者黄泰平刀傷其膝仆地張齊斬之|{
	東昏時年十九}
稷召尚書右僕射王亮等列坐殿前西鍾下令百僚署牋以黄油裹東昏首|{
	黄絹施油可以禦雨謂之黄油以黄油裹物表可見裏盖欲蕭衍易於審視也}
遣國子博士范雲等送詣石頭 |{
	考異曰南史王亮傳曰張稷等議立湘東嗣王寶晊領軍王瑩曰城閉已久人情離解征東在近何不諮問按時和帝已立稷等知建康不可守故弑東昏豈敢復議立寶晊今從齊紀}
右衛將軍王志嘆曰冠雖弊何可加足取庭中樹葉挼服之|{
	挼奴禾翻兩手相切摩也今俗語云挼莎}
偽悶不署名衍覽牋無志名心嘉之亮瑩之從弟志僧虔之子也|{
	王瑩蕭衍引為相國左長史王僧虔齊初位登台司}
衍與范雲有舊|{
	衍與雲同遊竟陵西邸見一百三十六卷武帝永明二年}
即留參帷幄王亮在東昏朝以依違取容蕭衍至新林百僚皆間道送欵|{
	朝直遥翻間古莧翻}
亮獨不遣東昏敗亮出見衍衍曰顛而不扶安用彼相亮曰若其可扶明公豈有今日之舉|{
	論語孔子曰危而不持顛而不扶則將焉用彼相矣衍引以詰王亮}
城中出者或被刧剥楊公則親帥麾下陳於東掖門衛送公卿士民故出者多由公則營焉|{
	被皮義翻帥讀曰率}
衍使張弘策先入清宮封府庫及圖籍於時城内珍寶委積弘策禁勒部曲秋毫無犯收潘妃及嬖臣茹法珍梅蟲兒王喧之等四十一人皆屬吏|{
	陳讀曰陣茹音如喧况晚翻屬之欲翻}
初海陵王之廢也|{
	事見一百三十九卷明帝建武元年}
王太后出居鄱陽王故第號宣德宮乙巳蕭衍以宣德太后令追廢涪陵王為東昏侯|{
	涪音浮}
禇后及太子誦並為庶人以衍為中書監大司馬録尚書事驃騎大將軍揚州刺史封建安郡公依晉武陵王遵承制故事百僚致敬|{
	不待西臺詔命而以宣德太后令高自署置蕭衍之心路人所知也豈必待范雲沈約其端哉武陵王遵事見一百一十三卷晉安帝元興三年}
以王亮為長史壬申更封建安王寶寅為鄱陽王|{
	更工衡翻}
癸酉以司徒揚州刺史晉安王寶義為太尉領司徒己卯衍入屯閲武堂下令大赦又下令凡昏制謬賦淫刑濫役外可詳檢前原悉皆除盪|{
	原南史作源前源謂日前興事之源也盪字作蕩音徒朗翻}
其主守散失諸所損耗精立科條咸從原例|{
	原赦也守式又翻}
又下令通檢尚書衆曹東昏時諸諍訟失理|{
	諍讀曰爭}
及主者淹停不時施行者精加訊辯依事議奏|{
	訊問也王制三訊然後制刑辯别白也左傳曰子辭君必辯焉辯兵免翻}
又下令收葬義師瘞逆徒之死亡者|{
	瘞一計翻}
潘妃有國色衍欲留之以問侍中領軍將軍王茂茂曰亡齊者此物留之恐貽外議乃縊殺於獄并誅嬖臣茹法珍等|{
	縊於賜翻又於計翻嬖卑義翻又博計翻茹音如}
以宮女二千分賚將士|{
	賚洛代翻}
乙酉以輔國將軍蕭宏為中護軍衍之東下也豫州刺史馬仙琕擁兵不附衍|{
	琕部田翻}
衍使其故人姚仲賓說之|{
	說輸芮翻}
仙琕先為設酒乃斬於軍門以徇衍又遣其族叔懷遠說之仙琕曰大義滅親又欲斬之軍中為請乃得免|{
	說式芮翻為于偽翻}
衍至新林仙琕猶於江西日抄運船|{
	豫州治歷陽在大江之西抄楚交翻}
衍圍宮城州郡皆遣使請降|{
	使疏吏翻降戶江翻}
吳興太守袁昂獨拒境不受命昂顗之子也|{
	袁顗死於義嘉之難}
衍使駕部郎考城江革|{
	曹魏置二十三郎駕部其一也杜佑曰宋齊駕部屬左民尚書}
為書與昂曰根本既傾枝葉安附今竭力昏主未足為忠家門屠滅非所謂孝豈若翻然改圖自招多福昂復書曰三吳内地非用兵之所况以偏隅一郡何能為役自承麾斾届止莫不膝袒軍門|{
	膝袒謂膝行肉袒也}
唯僕一人敢後至者政以内揆庸素文武無施雖欲獻心不增大師之勇置其愚默寧沮衆軍之威|{
	沮在呂翻}
幸藉將軍含弘之大可得從容以禮竊以一餐微施尚復投殞|{
	從千容翻施式豉翻投殞言投命殞身也復扶又翻}
况食人之禄而頓忘一旦非唯物議不可亦恐明公鄙之所以躊躇未遑薦璧|{
	薦璧謂銜璧而降也薦進也}
昂問時事於武康令北地傅暎|{
	吳分烏程餘杭立永安縣晉武帝太康元年更名武康屬吳興郡}
暎曰昔元嘉之末開闢未有故太尉殺身以明節|{
	袁淑贈太尉淑死見一百二十七卷宋文帝元嘉三十年}
司徒當寄託之重理無苟全所以不顧夷險以徇名義|{
	司徒謂昂父顗也顗死見一百三十一卷宋明帝泰始二年}
今嗣主昏虐曾無悛改荆雍協舉乘據上流|{
	悛丑緣翻雍於用翻}
天人之意可知願明府深慮無取後悔及建康平衍使豫州刺史李元履廵撫東土敕元履曰袁昂道素之門世有忠節|{
	即謂淑顗也}
天下須共容之勿以兵威陵辱元履至吳興宣衍旨昂亦不請降開門撤備而已仙琕聞臺城不守號泣謂將士曰我受人任寄義不容降君等皆有父母我為忠臣君為孝子不亦可乎乃悉遣城内兵出降餘壯士數十閉門獨守俄而兵入圍之數十重|{
	號戶刀翻重直用翻}
仙琕令士皆持滿兵不敢近|{
	近其靳翻}
日暮仙琕乃投弓曰諸君但來見取我義不降乃檻送石頭衍釋之使待袁昂至俱入曰令天下見二義士衍謂仙琕曰射鉤斬袪昔人所美卿勿以殺使斷運自嫌|{
	斷音短}
仙琕謝曰小人如失主犬後主飼之則復為用矣|{
	飼祥吏翻復扶又翻又如字}
衍笑皆厚遇之丙戌蕭衍入鎮殿中 劉希祖既克安成移檄湘部始興内史王僧粲應之僧粲自稱湘州刺史引兵襲長沙去城百餘里於是湘州郡縣兵皆蜂起以應僧粲唯臨湘湘隂瀏陽羅四縣尚全|{
	臨湘羅二縣自漢以來屬長沙郡吳立瀏陽縣亦屬長沙宋蒼梧王元徽二年分益陽湘西羅及巴峽流民立湘隂縣屬湘東郡隋改臨湘為長沙縣潭州治所也唐廢羅縣入湘隂屬岳州瀏陽今仍屬潭州瀏音留又音柳}
長沙人皆欲汎舟走行事劉坦悉聚其舟焚之遣軍主尹法略拒僧粲戰數不利|{
	數所角翻}
前湘州鎮軍鍾玄紹|{
	按當時州府官屬無鎮軍之稱此必梁書之誤}
潜結士民數百人刻日翻城應僧粲坦聞其謀陽為不知因理訟至夜而城門遂不閉以疑之玄紹未明旦詣坦問其故坦久留與語密遣親兵收其家書玄紹在坐而收兵已報具得其文書本末玄紹即首服|{
	坐徂臥翻下於坐同首手又翻}
於坐斬之焚其文書餘黨悉無所問衆愧且服州郡遂安法略與僧粲相持累月建康城平楊公則還州僧粲等散走王丹為郡人所殺|{
	王丹先以南康應劉希祖}
劉希祖亦舉郡降公則克己亷慎輕刑薄賦頃之湘州戶口幾復其舊|{
	幾居依翻}


資治通鑑卷一百四十四
