資治通鑑卷二百四十二
宋 司馬光 撰

胡三省 音註

唐紀五十八|{
	起重光赤奮若七月盡玄黓攝提格凡一年有奇}


穆宗睿聖文惠孝皇帝中

長慶元年秋七月甲辰韋雍出逢小將策馬衝其前導雍命曳下欲於街中杖之河朔軍士不貫受杖不服|{
	韋雍欲以柳公綽治京兆之體治豳燕然公綽行之則可肅清輦轂韋雍行之則召禍興戎所居之地不同也貫讀曰慣}
雍以白弘靖弘靖命軍虞侯繫治之|{
	治直之翻}
是夕士卒連營呼譟作亂將校不能制遂入府舍掠弘靖貨財婦女囚弘靖於薊門館|{
	薊門館幽州驛館也}
殺幕僚韋雍張宗元|{
	考異曰舊傳作張宗厚今從實録}
崔仲卿鄭塤|{
	塤許元翻}
都虞侯劉操押牙張抱元明日軍士稍稍自悔悉詣館謝弘靖請改心事之凡三請弘靖不應軍士乃相謂曰相公無言是不赦吾曹軍中豈可一日無帥乃相與迎舊將朱洄奉以為留後|{
	帥所類翻將即亮翻}
洄克融之父也時以疾廢臥家自辭老病請使克融為之衆從之|{
	或問當亂軍相率詣館謝弘靖之時弘靖若能以任迪簡行於中山者行之可以弭亂乎曰否迪簡能與其下同甘苦弘靖驕貴簡默弘靖婦女為兵所掠僚佐為兵所殺使燕人果能改心以事弘靖亦徒建節帥空名於悍將兇卒之上耳悍兇憑陵無所不至祗重辱而已}
衆以判官張徹長者不殺徹罵曰汝何敢反行且族滅衆共殺之 |{
	考異曰實録徹到職纔數日軍人不之殺與弘靖同館處之後數日軍人恐徹與弘靖為謀將移之他所徹自疑就戮因抗聲大罵復遇害舊傳曰續有張徹者自遠使迴軍人以其無過不欲加害將引置館中徹不知其心遂索弘靖所在大罵軍人亦為亂兵所殺韓愈徹墓誌曰徹累官至范陽府監察御史長慶二年今牛宰相為中丞奏君為御史其府惜不敢留遣之而密奏臣始至孤怯須彊佐乃濟半道有詔以君還之至數日軍亂怨其府從事盡殺之而囚其帥且相約張御史長者無庸殺置之帥所居月餘聞有中貴人自京師至君謂其帥公無負此土人上使至可因請見自辯幸得脱免歸即推門求出守者以告其魁魁與其徒皆駭曰張御史忠義必為其帥告此餘人不如遷之别館即以衆出君君出門罵衆曰汝何敢反前日吳元濟斬東市昨日李師道斬於軍中同惡者父母妻子皆屠死肉餧狗鼠鴟鵶汝何敢反行且罵衆畏惡其言不忍聞且虞生變即撃君以死君抵死口不絶罵衆皆曰義士義士或收瘞之以俟據舊傳徹以弘靖囚時被殺實録云後數日墓誌云居月餘三書各不同按此月丁巳弘靖已貶官月餘則離幽州今從實録參以墓誌 余謂韓愈墓誌能紀張徹所以罵賊之言實録及舊傳能原張徹所以罵賊之心若其月日則考異已有所去取矣}
壬子羣臣上尊號曰文武孝德皇帝赦天下 甲寅幽州監軍奏軍亂丁巳貶張弘靖為賓客分司|{
	貶為太子賓客分司東都也}
己未再貶吉州刺史 |{
	考異曰舊傳貶撫州刺史按明年乃改撫州今從實録}
庚申以昭義節度使劉悟為盧龍節度使悟以朱克融方彊奏請且授克融節钺徐圖之乃復以悟為昭義節度使 辛酉太和公主長安 初田弘正受詔鎮成德自以久與鎮人戰有父兄之仇|{
	憲宗之世田弘正兩出兵攻鎮冀}
乃以魏兵二千從赴鎮因留以自衛奏請度支供其糧賜|{
	舊制諸鎮兵出境度支給其衣糧}
戶部侍郎判度支崔倰性剛褊無遠慮|{
	倰力曾翻}
以為魏鎮各自有兵恐開事例不肯給弘正四上表不報不得已遣魏兵歸 |{
	考異曰舊弘正傳七月歸卒於魏州王庭湊傳云六月魏兵還鎮崔倰傳曰遣魏卒還鎮不數日而鎮州亂今從之}
倰沔之孫也|{
	崔沔開元初名臣}
弘正厚於骨肉兄弟子姪在兩都者數十人競為侈靡|{
	弘正兄弟子姪皆仕於朝分居東西兩都}
日費約二十萬弘正輦魏鎮之貨以供之相屬於道|{
	屬之欲翻}
河北將士頗不平詔以錢百萬緡賜成德軍度支輦運不時至軍士益不悦都知兵馬使王庭湊本回鶻阿布思之種也|{
	庭湊曾祖五哥之驍果善鬭王武俊養以為子故冒姓王氏阿布思者天寶中以反誅種章勇翻}
性果悍隂狡|{
	悍下罕翻又侯旰翻}
潜謀作亂每抉其細故以激怒之|{
	抉一決翻挑也}
尚以魏兵故不敢發及魏兵去壬戌夜庭湊結牙兵譟於府署殺弘正及僚佐元從將吏并家屬三百餘人|{
	從才用翻下再從同}
廷湊自稱留後逼監軍宋惟澄奏求節鉞八月癸巳惟澄以聞朝廷震駭崔倰於崔植為再從兄故時人莫敢言其罪初朝廷易置魏鎮帥臣左金吾將軍楊元卿上言以為非便又詣宰相深陳利害及鎮州亂上賜元卿白玉帶辛未以元卿為涇原節度使|{
	楊元卿以言驗受賞然無救於鎮州之亂者古之明君不徒賞言者而已其言可行必先從而行之}
瀛莫將士家屬多在幽州壬申莫州都虞侯張良佐潜引朱克融兵入城刺史吳暉不知所在|{
	莫州北接幽薊故先䧟}
癸酉王庭湊遣人殺冀州刺史王進岌分兵據其州魏博節度使李愬聞田弘正遇害素服令將士曰魏人所以得通聖化至今安寧富樂者|{
	樂音洛}
田公之力也今鎮人不道輒敢害之是輕魏以為無人也諸君受田公恩宜如何報之衆皆慟哭深州刺史牛元翼成德良將也愬使以寶劒玉帶遺之|{
	遺唯季翻}
曰昔吾先人以此劒立大勲|{
	謂平朱泚也}
吾又以之平蔡州今以授公努力翦庭湊元翼以劒帶徇于軍報曰願盡死愬將出兵會疾作不果元翼趙州人也乙亥起復前涇原節度使田布為魏博節度使令乘驛之鎮布固辭不獲與妻子賓客訣曰吾不還矣悉屏去旌節導從而行|{
	屏必郢翻又卑正翻從才用翻}
未至魏州三十里被髮徒跣號哭而入居于堊室|{
	被皮義翻號戶刀翻堊遏各翻白埴也按間傳父母之喪居倚廬齊衰之喪居堊室扎穎達正義曰斬衰居倚廬齊衰居堊室論其正耳亦有斬衰不居倚廬者則雜記云大夫居倚廬士居堊室是士服斬衰而居堊室田布父為鎮人所殺寢苫枕戈之時也今居堊室蓋用士禮也}
月俸千緡一無所取賣舊產得錢十餘萬緡皆以頒士卒舊將老者兄事之|{
	以田布所為宜可以得魏卒之心而卒不濟者人心已揺而布之威略不振也}
丙子瀛州軍亂執觀察使盧士玫及監軍僚佐送幽州囚於客館王庭湊遣其將王立攻深州不克丁丑詔魏博横海昭義河東義武諸軍各出兵臨成德之境若王庭湊執迷不復宜即進討成德大將王儉等五人謀殺王庭湊事泄并部兵三千人皆死己卯以深州刺史牛元翼為深冀節度使|{
	深州南至冀州八十五里}
丁亥以殿中侍御史温造為起居舍人充鎮州四面諸軍宣慰使歷澤潞河東魏博横海深冀易定等道諭以軍期造大雅之五世孫也|{
	高祖起兵温大雅掌書翰}
己丑以裴度為幽鎮兩道招撫使癸巳王庭湊引幽州兵圍深州 九月乙巳相州軍亂殺刺史邢濋|{
	濋音楚}
吐蕃遣其禮部尚書論訥羅來求盟庚戌以大理卿劉元鼎為吐蕃會盟使 壬子朱克融焚掠易州淶水遂城滿城|{
	淶水漢涿郡遒縣地隋開皇元年以范陽為遒更置范陽縣於此地六年改范陽曰固安八年廢十年又置永陽縣十八年又改為淶水周官職方其浸淶易蓋因淶水以名縣也淶音來遂城漢北新城縣地屬中山國後魏置南營州於其地置五郡十都後省併為昌黎一郡領永樂新昌二縣隋廢郡因舊有武遂縣置遂城縣唐屬易州宋以遂城縣置威虜軍金以縣置遂州以滿城縣屬保州}
自定兩税以來|{
	定兩税見二百二十六卷德宗建中元年}
錢日重物日輕民所輸三倍其初詔百官議革其弊戶部尚書楊於陵以為錢者所以權百貨貿遷有無所宜流散|{
	貿音茂流散謂錢流布於天下}
不應蓄聚今税百姓錢藏之公府又開元中天下鑄錢七十餘爐歲入百萬|{
	新志云天寶末天下罏九十九絳州三十楊潤宣鄂蔚皆十益郴皆五洋州三定州一蓋天寶末又加多於開元矣}
今纔十餘爐歲入十五萬又積於商賈之室|{
	賈音古}
及流入四夷又大歷以前淄青太原魏博貿易雜用鉛鐵嶺南雜用金銀丹砂象齒今一用錢如此則錢焉得不重物焉得不輕|{
	焉於䖍翻}
今宜使天下輸税課者皆用穀帛廣鑄錢而禁滯積|{
	積子賜翻}
及出塞者|{
	錢出邊關則流入於夷狄}
則錢日滋矣朝廷從之始令兩税皆輸布絲纊獨鹽酒課用錢 冬十月丙寅以鹽鐵轉運使刑部尚書王播為中書侍郎同平章事使職如故播為相專以承迎為事未嘗言國家安危 以裴度為鎮州四面行營都招討使左領軍大將軍杜叔良以善事權倖得進時幽鎮兵勢方盛諸道兵未敢進上欲功速成宦官薦叔良以為深州諸道行營節度使|{
	為杜叔良喪師張本}
以牛元翼為成德節度使 癸酉命宰相及大臣凡十七人與吐蕃論訥羅盟于城西遣劉元鼎與訥羅入吐蕃亦與其宰相以下盟|{
	吐蕃國有大相副相史因亦以宰相書之}
乙亥以沂州刺史王智興為武寧節度副使先是副使皆以文吏為之|{
	先悉薦翻}
上聞智興有勇略欲用之於河北故以是寵之|{
	為王智興逐其帥崔羣張本}
丁丑裴度自將兵出承天軍故關以討王庭湊|{
	承天軍當在遼州界故關即孃子關也宋朝廢遼州以平城和順二縣為鎮以并州之樂平平定二縣為平定軍二鎮屬焉以承天軍為寨屬平定縣平定唐之廣陽縣也按沈存中筆談鎮州通河東有兩路飛狐路在大茂山之西大茂山恒山之岑也自銀治寨北出倒馬關却自石門子今水鋪入缾形梅回兩寨之間至代州自石晉割地與契丹以大茂山分脊為界此路已不通惟北寨西出承天關路可至河東然路極峭峽宋白曰承天軍太原東鄙土門路所衝也}
朱克融遣兵寇蔚州|{
	媯州西南至蔚州二百四十里蔚紆勿翻}
戊寅王庭湊遣兵寇貝州己卯易州刺史柳公濟敗幽州兵於白石嶺|{
	敗補邁翻}
殺千餘人 庚辰横海軍節度使烏重奏敗成德兵於饒陽 辛巳魏博節度使田布將全軍三萬人討王庭湊屯於南宫之南拔其二柵 翰林學士元稹與知樞密魏弘簡深相結求為宰相由是有寵於上每事咨訪焉|{
	元稹交結大閹變其素守憲宗之過也稹止忍翻}
稹無怨於裴度但以度先達重望恐其復有功大用|{
	復扶人翻}
妨已進取故度所奏畫軍事多與弘簡從中沮壞之度乃上表極陳其朋比姦蠧之狀|{
	沮在呂翻瓌音怪比毗至翻}
以為逆豎搆亂震驚山東|{
	逆豎指王庭湊等}
姦臣作朋撓敗國政|{
	姦臣指元稹等撓奴教翻敗補邁翻屈也}
陛下欲掃蕩幽鎮先宜肅清朝廷何者為患有大小議事有先後河朔逆賊祗亂山東禁闈姧臣必亂天下是則河朔患小禁闈患大小者臣與諸將必能翦滅大者非陛下覺悟制斷無以驅除|{
	斷丁亂翻}
今文武百寮中外萬品有心者無不憤忿|{
	憤懣也忿怒也}
有口者無不咨嗟直以奬用方深不敢抵觸恐事未行而禍已及不為國計且為身謀臣自兵興以來所陳章疏事皆要切所奉書詔多有參差|{
	參楚簪翻差楚宜翻參差不齊也}
蒙陛下委付之意不輕遭姧臣抑損之事不少臣素與佞倖亦無讐嫌正以臣前請乘傳詣闕面陳軍事|{
	傳株戀翻乘傳乘驛馬也}
姦臣最所畏憚恐臣其過百計止臣臣又請與諸軍齊進隨便攻討姦臣恐臣或有成功曲加阻礙逗遛日時進退皆受羈牽|{
	羈馬絡頭也牽牛紖也諭以牛馬動為人所制}
意見悉遭蔽塞|{
	塞悉則翻}
但欲令臣失所使臣無成則天下理亂山東勝負悉不顧矣為臣事君一至于此若朝中姦臣盡去則河朔逆賊不討自平若朝中姦臣尚存則逆賊縱平無益陛下儻未信臣言乞出臣表使百官集議彼不受責臣當伏辜表三上|{
	上時掌翻}
上雖不悦以度大臣不得已癸未以弘簡為弓箭庫使稹為工部侍郎稹雖解翰林恩遇如故|{
	為相稹及于方張本}
宿州刺史李直臣坐當死宦官受其賂為之請|{
	為于偽翻}
御史中丞牛僧孺固請誅之上曰直臣有才可惜僧孺對曰彼不才者無過温衣飽食以足妻子安足慮本設法令所以擒制有才之人安禄山朱泚皆才過於人法不能制者也上從之 横海節度使烏重將全軍救深州|{
	時王庭湊圍牛元翼於深州}
諸軍倚重獨當幽鎮東南|{
	横海當鎮州之東幽州之南}
重宿將知賊未可破按兵觀釁上怒以杜叔良為横海節度使徙重為山南西道節度使 靈武節度使李進誠奏敗吐蕃三千騎於大石山下|{
	敗補邁翻大石山在魯州東南魯州六胡州之一也在靈夏西河曲之地}
十一月辛酉淄青節度使薛平奏突將馬廷崟作亂伏誅|{
	崟魚音翻}
時幽鎮兵攻棣州平遣大將李叔佐將兵救之刺史王稷供饋稍薄軍士怨怒宵潰推廷崟為主行且收兵至七千餘人徑逼青州城中兵少不敵平悉府庫及家財召募得精兵二千人逆戰大破之斬廷崟其黨死者數千人 |{
	考異曰河南記曰韓國公之節制青州也長慶元年詔徵數道兵馬且問罪於常山平盧二千餘人駐于無棣臨當回戈青州所駐兵部内隊長有馬士端者殺其首領遂驅所部士卒兼招召迫脅比到博昌已萬餘人便謀入青州有日矣韓公聞之便議除討大將等進計曰彼賊者兇頑一卒無經遠之謀可令紿以尚書已赴闕庭三軍將吏皆延頸以待留後賊必信之懈然無備可伏甲而虜之韓公大然其策於是賊心不復疑貳翌日引兵而來遂於城北三十餘里三面伏兵賊衆果陷於我圍信旗一麾步騎雲合賊衆驚擾不知所為悉皆降伏遂令投戈釋甲驅入青州矯令還家待以不死遂條其數目明立簿書三千二千各屯一處霜刀齊蟻衆湯消二萬餘人同命一日賊帥馬士端潰圍奔走尋於鄒平渡口追獲磔於城北於是具列其狀以上聞旋除左僕射據實録作馬廷崟舊傳作馬狼兒河南記作馬士端今名從實録事從舊傳明年二月平加僕射舊傳云封魏國公河南記作韓公恐誤}
横海節度使杜叔良將諸道兵與鎮人戰遇敵輒北鎮人知其無勇常先犯之十二月庚午監軍謝良通奏叔良大敗於博野|{
	博野漢涿郡蠡吾縣之地後漢分置博陵縣後魏改為博野唐屬深州宋為永寜軍治所宋白曰雍熙四年於博野縣置寧邊軍}
失亡七千餘人叔良脱身還營喪其旌節|{
	喪息浪翻}
丁丑義武節度使陳楚奏敗朱克融兵於望都及北平|{
	望都漢縣屬中山郡張晏曰都山在縣南堯母慶都所居堯山在縣北登堯山望見都山故以望都為名北齊併望都入北平唐武德四年復置望都縣屬定州九域志縣在州東北六十里北平亦漢古縣唐屬定州九域志在州北九十里宋白曰定州北平縣漢曲逆縣地後漢改蒲隂後魏孝昌中於今縣東北二十里置北平郡於北平城唐為北平縣按漢志北平縣屬中山國敗補邁翻}
斬獲萬餘人 戊寅以鳳翔節度使李光顔為忠武節度使兼深州行營節度使代杜叔良 自憲宗征伐四方國用已虚上即位賞賜左右及宿衛諸軍無節及幽鎮用兵久無功府藏空竭勢不能支|{
	藏徂浪翻支持也當也}
執政乃議王庭湊殺田弘正而朱克融全張弘靖罪有重輕請赦克融專討庭湊上從之乙酉以朱克融為平盧節度使|{
	平盧當作盧龍}
戊子義武奏破莫州清源等三柵斬獲千餘人|{
	柵側革翻}


二年春正月丁酉幽州兵陷弓高先是弓高守備甚嚴|{
	弓高縣宋朝為永静軍地先悉薦翻}
有中使夜至守將不内旦乃得入中使大詬怒|{
	詬許侯翻又古侯翻}
賊諜知之|{
	諜達協翻}
他日偽遣人為中使投夜至城下守將遽内之賊衆隨之遂陷弓高|{
	史言唐宦者陵轢守禦捍敵之臣使之失守}
又圍下博中書舍人白居易上言以為自幽鎮逆命朝廷徵諸道兵計十七八萬 |{
	考異曰白集作七八十萬計無此數恐是十七八萬誤耳}
四面攻圍已踰半年王師無功賊勢猶盛弓高既陷糧道不通下博深州飢窮日急|{
	深州西南皆逼於王庭湊惟恃弓高以通横海之餫弓高既陷糧道遂梗九域志弓高東至滄州一百二十里西北至深州二百里}
蓋由節將太衆其心不齊莫肯率先遞相顧望又朝廷賞罰近日不行未立功者或已拜官已敗衂者不聞得罪|{
	衂女六翻}
既無懲勸以至遷延若不改張|{
	改張猶更張也董仲舒曰譬如琴瑟不調必改絃而更張之乃可鼔也}
必無所望請令李光顔將諸道勁兵約三四萬人從東速進開弓高糧路解深邢重圍|{
	深邢當作深州重直龍翻}
與元翼合勢令裴度將太原全軍兼招討舊職西面壓境|{
	壓鎮州之境}
觀釁而動若乘虛得便即令同力翦除若戰勝賊窮亦許受降納欵|{
	降戶江翻}
如此則夾攻以分其力招諭以動其心必未及誅夷自生變故|{
	謂賊之麾下將有誅逆而效順者}
又請詔光顔選諸道兵精鋭者留之其餘不可用者悉遣歸本道自守土疆蓋兵多而不精豈唯虛費衣糧兼恐撓敗軍陳故也|{
	撓奴巧翻敗補邁翻陳讀曰陣}
今既祗留東西二帥|{
	謂令裴度居西李光顔居東}
請各置都監一人諸道監軍一時停罷如此則衆齊令一必有成功又朝廷本用田布令報父讐|{
	令報王庭湊殺弘正之讐}
今領全師出界供給度支|{
	言仰供給於度支}
數月已來都不進討非田布固欲如此抑有其由聞魏博一軍屢經優賞|{
	自田弘正舉魏博一軍歸朝其後代恒平蔡平鄆朝廷犒賞優厚}
兵驕將富莫肯為用况其軍一月之費計實錢二十八萬緡若更遷延將何供給此尤宜早令退軍者也若兩道止共留兵六萬所費無多|{
	兩道謂河東横海}
既易支持|{
	易以豉翻}
自然豐足今事宜日急其間變故遠不可知苟兵數不抽軍費不減食既不足衆何以安不安之中何事不有况有司迫於供軍百端歛率不許即用度交闕盡許則人心無憀|{
	指言將有建中之禍而微其辭憀落蕭翻無憀賴也}
自古安危皆繫於此伏乞聖慮察而念之疏奏不省|{
	白居易之論事李絳之流亞歟顧憲穆有用不用耳省悉景翻}
己亥度支饋滄州糧車六百乘至下博盡為成德軍所掠時諸軍匱乏供軍院所運衣糧往往不得至院|{
	此時供軍院置於行營者謂之北供軍院度支自南供軍院運以給之乘繩證翻}
在塗為諸軍邀奪其懸軍深入者皆凍餒無所得初田布從其父弘正在魏善視牙將史憲誠屢稱薦至右職及為節度使遂寄以腹心以為先鋒兵馬使軍中精鋭悉以委之憲誠之先奚人也世為魏將魏與幽鎮本相表裏及幽鎮叛魏人固揺心布以魏兵討鎮軍於南宫上屢遣中使督戰而將士驕惰無鬭志又屬大雪|{
	屬之欲翻}
度支饋運不繼布發六州租賦以供軍|{
	魏博貝衛澶相六州也}
將士不悦曰故事軍出境皆給朝廷|{
	言仰給於朝廷也}
今尚書刮六州肌肉以奉軍雖尚書瘠已肥國六州之人何罪乎憲誠隂蓄異志因衆心不悦離間鼓扇之|{
	以衆情諭大火本有熾烈之性鼓鞴以吹之揺扇以扇之則愈熾烈矣間古莧翻}
會有詔分魏博軍與李光顔使救深州庚子布軍大潰多歸憲誠布獨與中軍八千人還魏壬寅至魏州癸卯布復召諸將議出兵|{
	復扶又翻}
諸將益偃蹇曰尚書能行河朔舊事則死生以之|{
	謂行田承嗣李寶臣之事也}
若使復戰則不能也布無如之何歎曰功不成矣即日作遺表具其狀略曰臣觀衆意終負國恩臣既無功敢忘即死|{
	即就也}
伏願陛下速救光顔元翼不然者忠臣義士皆為河朔屠害矣奉表號哭|{
	號戶刀翻}
拜授幕僚李石乃入啟父靈|{
	孝子之喪其親也設几筵朝夕具盥洗上飲食事之如生俗謂之靈筵}
抽刀而言曰上以謝君父下以示三軍遂刺心而死|{
	刺七亦翻}
憲誠聞布已死乃諭其衆遵河北故事衆悦擁憲誠還魏奉為留後戊申魏州奏布自殺己酉以憲誠為魏博節度使憲誠雖喜得旄鉞外奉朝廷然内實與幽鎮連結 庚戌以德州刺史王日簡為横海節度使日簡本成德牙將也壬子貶杜叔良為歸州刺史王庭湊圍牛元翼於深州官軍三面救之|{
	裴度以河東軍臨其西李光顔以横每諸軍營其東陳楚以易定軍逼其北是三面救之}
皆以乏糧不能進雖李光顔亦閉壁自守而已軍士自采薪芻日給不過陳米一勺|{
	陳舊也經年之米為陳米勺職略翻又時灼翻周禮梓人為飲器勺一升按一升之勺乃飲器也非以量米凡量十勺為合十合為升十升為斗以量言之則一人日給一勺之陳米有餒死而已作史者蓋極言其匱乏猶武成血流漂杵之語}
深州圍益急朝廷不得已二月甲子以庭湊為成德節度使軍中將士官爵皆復其舊以兵部侍郎韓愈為宣慰使上之初即位也兩河略定蕭俛段文昌以為天下已太平漸宜消兵請密詔天下軍鎮有兵處每歲百人之中限八人逃死|{
	或以逃或以死除其籍俛音免}
上方荒宴不以國事為意遂可其奏軍士落籍者衆皆聚山澤為盜及朱克融王庭湊作亂一呼而亡卒皆集|{
	呼火故翻}
詔徵諸道兵討之諸道兵既少|{
	少詩沼翻}
皆臨時召募烏合之衆又諸節度既有監軍其領偏軍者亦置中使監陳|{
	監古銜翻陳讀曰陣}
主將不得專號令戰小勝則飛驛奏捷自以為功不勝則迫脅主將以罪歸之悉擇軍中驍勇以自衛遣羸懦者就戰故每戰多敗又凡用兵舉動皆自禁中授以方略朝令夕改不知所從不度可否|{
	度徒洛翻}
惟督令速戰中使道路如織驛馬不足掠行人馬以繼之人不敢由驛路行|{
	取間道而行由驛路則馬為所掠故也}
故雖以諸道十五萬之衆裴度元臣宿望烏重胤李光顔皆當時名將討幽鎮萬餘之衆屯守踰年竟無成功財竭力盡崔植杜元穎為相皆庸才無遠略史憲誠既逼殺田布朝廷不能討遂并朱克融王庭湊以節授之由是再失河朔迄于唐亡不能復取|{
	史極言唐再失河朔之由若以三叛得節之時言之須有先後復扶又翻}
朱克融既得旌節乃出張弘靖及盧士玫|{
	去年七月朱克融囚張弘靖八月囚盧士玫}
丙寅以牛元翼為山南東道節度使以左神策行營樂壽鎮兵馬使清河傅良弼為沂州刺史|{
	樂夀鎮即置於深州樂夀縣樂音洛}
以瀛州博野鎮遏使李寰為忻州刺史良弼寰所戍在幽鎮之間朱克融王庭湊互加誘脅良弼寰不從各以其衆堅壁賊竟不能取故賞之|{
	誘音酉}
丙子賜横海節度使王日簡姓名為李全略 辛巳中書侍郎同平章事崔植罷為刑部尚書以工部侍郎元稹同平章事 |{
	考異曰實録以御史中丞牛僧孺為戶部侍郎翰林學士李德裕為御史中丞舊李德裕傳元和初用兵伐叛始於杜黄裳誅蜀吉甫經畫欲定兩河方欲出師而卒繼之元衡裴度而韋貫之李逢吉沮議深以用兵為非而韋李相次罷相故逢吉常怒吉甫裴度而德裕於元和時久之不調逢吉僧孺宗閔以私怨恒排擯之時德裕與李紳元稹俱在翰林以學識才名相類情頗款密逢吉之黨深惡之其月自學士出為御史中丞按德裕元和中剔歷清要非為不調此際元稹入相逢吉在淮南豈能排擯德裕蓋出於德裕黨人之語耳今不取}
癸未加李光顔横海節度滄景觀察使其忠武深州行營節度如故以横海節度使李全略為德棣節度使時朝廷以光顔懸軍深入饋運難通故割滄景以隸之王庭湊雖受旌節不解深州之圍丙戌以知制誥東陽馮宿為山南東道節度副使權知留後|{
	垂拱二年分烏傷縣置東陽縣取舊郡名以名縣也屬婺州九域志在州東一百五十五里}
仍遣中使入深州督牛元翼赴鎮裴度亦與幽鎮書責以大義朱克融即解圍去王庭湊雖引兵少退猶守之不去元稹怨裴度欲解其兵柄故勸上雪庭湊而罷兵丁亥以度為司空東都留守平章事如故 |{
	考異曰舊紀傳皆云度守司徒為東都留守實録此云司徒後領淮南及拜相皆云司空新書度自檢校司空為守司空東都留守及領淮南乃為司徒蓋實録此月誤紀傳遂因之新傳後云司徒亦誤今据實録除淮南及拜相制書自此至罷相止是守司空舊裴度傳又曰元稹為相請上罷兵洗雪廷湊克融解深州之圍蓋欲罷度兵柄故也按此月甲子雪廷湊辛巳稹為相蓋稹未為相時勸上也}
諫官争上言時未偃兵度有將相全才不宜置之散地|{
	散蘇但翻}
上乃命度入朝然後赴東都以靈武節度使李聽為河東節度使初聽為羽林將軍有良馬上為太子遣左右諷求之聽以職總親軍不敢獻及河東缺帥|{
	帥所類翻}
上曰李聽不與朕馬是必可任遂用之 昭義監軍劉承偕恃恩|{
	憲宗之崩也劉承偕預有援立穆宗之功故恃恩}
陵轢節度使劉悟|{
	轢郎狄翻}
數衆辱之|{
	數所角翻衆辱者於衆中慢辱之也}
又縱其下亂法隂與磁州刺史張汶謀縛悟送闕下以汶代之悟知之諷其軍士作亂殺汶圍承偕欲殺之|{
	汶音問}
幕僚賈直言入責悟曰公所為如是欲効李司空邪此軍中安知無如公者|{
	李師道為司空賈直言舊僚屬也故猶稱其官言李師道悖逆劉悟倒戈取師道而得節鉞今悟効師道所為昭義軍中亦將有効悟所為而取節鉞者}
使李司空有知得無笑公於地下乎悟遂謝直言救免承偕囚之府舍 |{
	考異曰實録監軍劉承偕頗恃恩侵權嘗對衆辱悟又縱其下亂法悟不能平異日有中使至承偕宴之請悟悟欲往左右皆曰往必為其困辱矣軍衆因亂悟不止之遂擒承偕殺其二傔欲并害承偕悟救之獲免新劉悟傳曰承偕與都將張問謀縛悟送京師以問代節度事悟知之以兵圍監軍殺小使其屬賈直言質責悟悟即撝兵退匿承偕囚之新直言傳張問作張汶杜牧上李司徒書亦云其軍大亂殺磁州刺史張汶又云汶既因依承偕謀殺悟自取軍人忌怒遂至大亂蓋軍士圍承偕必出於悟志及奏朝廷則云軍衆所為耳今承偕名從實録汶名從杜書}
初上在東宫聞天下厭苦憲宗用兵故即位務優假將卒以求姑息三月壬辰詔神策六軍使及南牙常參武官|{
	南牙常參武官十六衛上將軍大將軍將軍也}
具由歷功績牒送中書量加奨擢|{
	由者得官之由歷者所歷職任量音良}
其諸道大將久次及有功者悉奏聞與除官應天下諸軍各委本道據守舊額不得輒有減省於是商賈胥吏|{
	賈音古}
争賂藩鎮牒補列將而薦之即升朝籍|{
	朝直遥翻唐末藩鎮列將帶朝銜者著之朝籍}
奏章委積士大夫皆扼腕歎息|{
	腕烏貫翻}
武寧節度副使王智興將軍中精兵三千討幽鎮節度使崔羣忌之奏請即用智興為節度使不則召詣闕除以他官|{
	不讀曰否}
事未報智興亦自疑會有詔赦王庭湊諸道皆罷兵智興引兵先期入境羣懼遣使迎勞|{
	先悉薦翻勞力到翻}
且使軍士釋甲而入智興不從乙巳引兵直進徐人開門待之智興殺不同己者十餘人乃入府牙見羣及監軍|{
	見賢遍翻}
拜伏曰軍衆之情不可如何為羣及判官從吏具人馬及治裝|{
	為于偽翻從才用翻下同治直之翻}
皆素所辦也遣兵衛從羣至埇橋而返|{
	埇余隴翻 考異曰實録羣累表請追智興授以他官事未行詔班師智興帥衆斬關而入舊智興傳亦同舊羣傳則曰羣以智興早得士心表請因授智興旄鉞寢不報智興回戈城内皆是父兄開關延入今兼取之}
遂掠鹽鐵院錢帛|{
	埇橋有鹽鐵院}
及諸道進奉在汴中者|{
	謂諸道進奉船在汴河中者}
并商旅之物皆三分取二|{
	史言唐下陵上慢無復紀綱}
丙午加朱克融王庭湊檢校工部尚書上聞其解深州之圍故褒之然庭湊之兵實猶在深州城下韓愈既行衆皆危之詔愈至境更觀事勢勿遽入愈曰止君之仁死臣之義|{
	言止之勿使遽入鎮者君之仁不畏死而徑往致命者臣之義也}
遂往至鎮庭湊拔刃弦弓以逆之及館甲士羅於庭庭湊言曰所以紛紛者乃此曹所為非庭湊心愈厲聲曰天子以尚書有將帥材故賜之節鉞不知尚書乃不能與健兒語邪甲士前曰先太師為國擊走朱滔|{
	王武俊贈太師擊走朱滔見二百三十二卷德宗興元元年為于偽翻}
血衣猶在此軍何負朝廷乃以為賊乎愈曰汝曹尚能記先太師則善矣夫逆順之為禍福豈遠邪自禄山思明以來至元濟師道其子孫有今尚存仕宦者乎田令公以魏博歸朝廷子孫雖在孩提皆為美官|{
	田弘正之徙成德也進兼中書令子孫為美官見上卷憲宗元和十四年}
王承元以此軍歸朝廷弱冠為節度使|{
	見上卷元和十五年冠古玩翻}
劉悟李祐今皆為節度使汝曹亦聞之乎庭湊恐衆心動麾之使出|{
	恐其衆聞愈言而心動有如劉悟李祐者}
謂愈曰侍郎來|{
	韓愈時為兵部侍郎故稱之}
欲使庭湊何為愈曰神策六軍之將如牛元翼者不少|{
	少詩沼翻}
但朝廷顧大體不可弃之耳尚書何為圍之不置庭湊曰即當出之因與愈宴禮而歸之未幾牛元翼將十騎突圍出|{
	幾居豈翻}
深州大將臧平等舉城降庭湊責其久堅守殺平等將吏百八十餘人 戊申裴度至長安見上謝討賊無功先是上詔劉悟送劉承偕詣京師悟託以軍情不時奉詔上問度宜如何處置|{
	處昌呂翻下同}
度對曰承偕在昭義驕縱不法臣盡知之悟在行營|{
	謂討王承宗在行營時}
與臣書具論其事時有中使趙弘亮在軍中持悟書去云欲自奏之不知嘗奏不|{
	奏不讀曰否}
上曰朕殊不知也且悟大臣何不自奏對曰悟武臣不知事體然今事狀籍籍如此|{
	顔師古曰籍籍猶紛紛也}
臣等面論陛下猶不能決况悟當日單辭豈能動聖聽哉|{
	單辭一人之言}
上曰前事勿論直言此時如何處置對曰陛下必欲收天下心止應下半紙詔書具陳承偕驕縱之罪令悟集將士斬之則藩鎮之臣孰不思為陛下効死|{
	為于偽翻}
非獨悟也上俛首良久曰|{
	俛音免}
朕不惜承偕然太后以為養子今兹囚縶太后尚未知之况殺之乎卿更思其次度乃與王播等奏請流承偕於遠州必得出|{
	言既明底其罪則悟必釋承偕}
上從之後月餘悟乃釋承偕 李光顔所將兵聞當留滄景皆大呼西走|{
	呼火故翻西走欲歸許州}
光顔不能制因驚懼成疾己酉上表固辭横海節乞歸許州許之|{
	李光顔本忠武節度使許州忠武軍治所 考異曰舊光顔傳曰光顔以朝廷制置乖方賊帥連結未可朝夕平定事若差跌即前功盡弃乃懇辭兼鎮尋以疾作表祈歸鎮朝廷果以討賊無功而赦庭湊今從實録}
壬子以裴度為淮南節度使餘如故|{
	餘官如故也}
加劉悟檢校司徒餘如故自是悟浸驕欲效河北三鎮|{
	魏鎮幽為河北三鎮}
招聚不逞|{
	不逞者欲為非而不得逞志者也}
章表多不遜 裴度之討幽鎮也回鶻請以兵從|{
	從才用翻}
朝議以為不可遣中使止之回鶻遣其臣李義節將三千人已至豐州北却之不從詔繒帛七萬匹以賜之甲寅始還|{
	還音旋又如字}
王智興遣輕兵二千襲濠州丙辰刺史侯弘度弃城奔壽州 言事者皆謂裴度不宜出外上亦自重之戊午制留度輔政以中書侍郎同平章事王播同平章事代度鎮淮南仍兼諸道鹽鐵轉運使李寰帥其衆三千出博野|{
	帥讀曰率}
王庭湊遣兵追之寰

與戰殺三百餘人庭湊兵乃還餘衆二千猶固守博野朝廷以新罷兵力不能討徐州己未以王智興為武

寧節度使 復以德棣節度使李全略為横海節度使|{
	李光顔既還許州故全略復鎮横海}
夏四月丁酉朔日有食之 甲戌以傅良弼李寰為神策都知兵馬使 戶部侍郎判度支張平叔上言官自糶鹽|{
	糶他弔翻}
可以獲利一倍又請令所由將鹽就村糶易|{
	所由綰掌官物之吏也事必經由其手故謂之所由}
又乞令宰相領鹽鐵使又請以糶鹽多少為刺史縣令殿最|{
	殿丁練翻}
又乞檢責所在實戶據口團保|{
	團保者團結戶口使之互相保識}
給一年鹽使其四季輸價又行此策後富商大賈或行財賄邀截喧訴其為首者所在杖殺連狀人皆杖脊|{
	連狀人謂連名告狀者也}
詔百官議其可否兵部侍郎韓愈上言以為城郭之外少有見錢|{
	少詩沼翻見賢遍翻下同}
糴鹽|{
	當屬上句}
多用雜物貿易鹽商則無物不取或賖貸徐還|{
	鬻物而緩取直曰賖貸借也}
用此取濟兩得利便今令吏人坐鋪自糶|{
	列物而鬻之謂之鋪鋪普故翻}
非得見錢必不敢授如此貧者無從得鹽自然坐失常課如何更有倍利又若令人吏將鹽家至而戶糶必索百姓供應|{
	索山客翻供應言各供其物以應官吏所須也}
騷擾極多又刺史縣令職在分憂|{
	人君憂民有不得其生者故置守令以撫字之是其職在分憂也}
豈可惟以鹽利多少為之升黜不復考其理行|{
	復扶又翻理行猶言治行也行戶孟翻}
又貧家食鹽至少或有淡食動經旬月若據戶給鹽依時徵價官吏畏罪必用威刑臣恐因此所在不安此尤不可之大者也中書舍人韋處厚議以為宰相處論道之地|{
	處昌呂翻書曰三公論道經邦}
雜以鹺務|{
	鹺才何翻記曲禮曰鹽曰鹹鹺}
實非所宜竇參皇甫鎛皆以錢穀為相名利難兼卒蹈禍敗|{
	竇參事見德宗紀皇甫鎛事見憲宗紀卒子恤翻}
又欲以重法禁人喧訴|{
	謂為首告訴者杖殺連名者杖脊也}
夫強人之所不能事必不立|{
	強其兩翻}
禁人之所必犯法必不行矣事遂寢 |{
	考異曰實録因三月壬寅平叔遷戶部侍郎事遂言變鹽法及處厚駮議按韓愈時奉使鎮州猶未還又壬寅三月十一日愈論鹽法狀云奉今月九日敕不知其何月也今附於四月之末}
平叔又奏徵遠年逋欠江州刺史李渤上言度支徵當州貞元二年逃戶所欠錢四千餘緡當州今歲旱灾田損什九|{
	刺史自以所守州為當州}
陛下奈何於大旱中徵三十六年前逋負詔悉免之 邕州人不樂屬容管|{
	廢邕管入容管見上卷元和十五年樂音洛}
刺史李元宗以吏人狀授御史使奏之容管經略使嚴公素聞之遣吏按元宗擅以羅陽縣歸蠻酋黄少度|{
	羅陽當在西原羈縻縣也蓋裴行立攻黄洞時得之而元宗擅以歸之也酋慈由翻}
五月壬寅元宗將兵百人并州印奔黄洞王庭湊之圍牛元翼也和王傅于方欲以奇策干進|{
	和王綺順宗子}
言於元稹請遣客王昭于友明 |{
	考異曰實録作于友明後作于啟明舊元稹傳作王友明今從實録之初及新書}
間說賊黨使出元翼|{
	間古莧翻說式芮翻}
仍賂兵吏部令史偽出告身二十通|{
	文官告身賂吏部令史偽為之武官告身賂兵部令史偽為之}
令以便宜給賜稹皆然之|{
	元稹方圖進取而先與兵吏部令史為偽曾是以為相業乎}
有李賞者知其謀乃告裴度云方為稹結客刺度|{
	為于偽翻}
度隱而不賞詣左神策告其事 |{
	考異曰舊裴度傳曰初度與李逢吉素不協度自太原入朝而惡度者以逢吉善於隂計足能構度乃自襄陽召逢吉入朝為兵部尚書度既復知政事而魏弘簡劉承偕之黨在禁中逢吉用族子仲言之謀因毉人鄭注與中尉王守澄交結内官皆為之助五月左神策軍奏告事人李賞稱于方受元稹所使結客欲刺裴度按惡度者不過元稹與宦官彼欲害度其術甚多何必召逢吉又如所謀則稹當獲罪非所以害度也又逢吉若使李賞告之下御史按鞫賞急必連引逢吉非所以自謀也蓋賞自告耳非逢吉教令也}
丁巳詔左僕射韓臯等鞫之 戊午幽州節度使朱克融進馬萬匹羊十萬口而表云先請其直充犒賞|{
	史言朱克融玩弄朝廷}
三司按于方刺裴度事皆無驗六月甲子度及元稹皆罷相度為右僕射稹為同州刺史以兵部尚書李逢吉為門下侍郎同平章事 党項寇靈州渭北掠官馬|{
	先寇靈州遂及渭北也}
諫官上言裴度無罪不當免相元稹與于方為邪謀責之太輕上不得已壬申削稹長春宫使|{
	長春宫在同州元稹以出刺兼使今削之}
吐蕃寇靈武 庚辰鹽州奏党項都督抜跋萬誠請降|{
	党底朗翻拔跋當作拓跋降戶江翻}
壬午吐蕃寇鹽州戊子復置邕管經略使|{
	復扶又翻}
初張弘靖為宣武節度使|{
	弘靖代韓弘見上卷憲宗元和十四年}
屢賞以悦軍士府庫虚竭李愿繼之性奢侈賞勞既薄於弘靖時|{
	勞力到翻}
又峻威刑軍士不悦愿以其妻弟竇瑗典宿直兵瑗驕貪軍中惡之|{
	惡烏路翻}
牙將李臣則等作亂秋七月壬辰夜即帳中斬瑗頭因大呼|{
	呼火故翻}
府中響應愿與一子踰城奔鄭州|{
	汴州西至鄭州一百五十里}
亂兵殺其妻推都押牙李㝏為留後|{
	㝏古拜翻考異曰實録戊戌汴州監軍使奏六月四日夜軍亂節度使李愿踰城以遁新紀亦云六月癸亥李㝏反逐李愿按李愿若以六月四日夜被逐不應至此月十日方奏到疑實録十字誤為六舊紀止用此奏到日今從愿傳七月四日}
丙申宋王結薨|{
	結順宗子}
戊戌宣武監軍奏軍亂庚子李㝏自奏已權知留後 乙巳詔三省官與宰相議汴州事|{
	三省官自遺補舍人丞郎以上}
皆以為宜如河北故事授李㝏節李逢吉曰河北之事蓋非獲已今若并汴州弃之則是江淮以南皆非國家有也杜元穎張平叔争之曰奈何惜數尺之節不愛一方之死乎議未決會宋亳穎三州各上奏請别命帥|{
	三州皆宣武巡屬帥所類翻}
上大喜以逢吉議為然遣中使詣三州宣慰逢吉因請以將軍徵㝏入朝以義成節度使韓充鎮宣武充弘之弟素寛厚得衆心|{
	韓弘鎮宣武二十餘年將士懷之其弟又以寛厚得衆故逢吉請以代㝏}
脱㝏旅拒則命徐許兩軍攻其左右而滑軍蹙其北|{
	徐帥王智興許帥李光顔}
充必得入矣上皆從之丙午貶李愿為隨州刺史|{
	隨州古隨國漢為隨縣江左為隨郡西魏置隨州京師東南一千三百八十八里}
以韓充為宣武節度兼義成節度使徵李㝏為右金吾將軍㝏不奉詔宋州刺史高承簡斬其使者㝏遣兵二千攻之陷寧陵襄邑|{
	宋州西至汴州二百八十五里寧陵州西四十五里襄邑州西微北}
宋州有三城賊已陷其南城承簡保北二城與賊十餘戰癸丑忠武節度使李光顔將兵二萬五千討李㝏屯尉氏|{
	尉氏在汴州西南許州東北}
兖海節度使曹華聞㝏作亂不俟詔即兵討之㝏遣兵三千人攻宋州適至城下丙辰華逆擊破之丁巳李光顔敗宣武兵於尉氏|{
	敗補邁翻下同}
斬獲二千餘人八月辛酉大理卿劉元鼎自吐蕃還|{
	元鼎去年使吐蕃}
甲子韓充入汴境軍于千塔|{
	千塔當在汴州北}
武寧節度使王智興與高承簡共破宣武兵斬首千餘級餘衆遁去壬申韓充敗宣武兵於郭橋|{
	九域志汴州祥符縣有郭橋鎮}
斬首千餘級進軍萬勝|{
	九域志汴州中牟縣有萬勝鎮}
初李㝏既為留後以都知兵馬使李質為腹心及㝏除將軍不奉詔質屢諫不聽會㝏疽於首遣李臣則等將兵拒李光顔於尉氏既而官軍四集兵屢敗㝏疾甚悉以軍事屬李質|{
	屬之欲翻}
臥於家丙子質與監軍姚文壽擒㝏殺之詐為㝏牒追臣則等至皆斬之執㝏四子送京師韓充未至質權知軍務時牙兵三千人日給酒食物力不能支質曰若韓公始至而罷之則人情大去矣不可留此弊以遺吾帥|{
	遺唯李翻帥所類翻}
即命罷給而後迎充丁丑充入汴癸未以韓充專為宣武節度使以曹華為義成節度使高承簡為兖海沂密節度使加李光顔兼侍中以李質為右金吾將軍韓充既視事人心粗定乃密籍軍中為惡者千餘人一朝并父母妻子悉逐之曰敢少留境内者斬於是軍政大治|{
	除亂而去其根則亂無從生矣治直史翻}
九月戊子朔浙西觀察使京兆竇易直|{
	易弋豉翻}
奏大將王國清作亂伏誅初易直聞汴州亂而懼欲散金帛以賞軍士或曰賞之無名恐益生疑乃止而外已有知之者故國清作亂易直討擒之并殺其黨二百餘人 |{
	考異曰舊易直傳曰時江淮旱水淺轉運司錢帛委積不能漕國清指以為賞激諷州兵謀亂先事有告者乃收國清下獄其黨數千人大呼入獄中篡取國清而出之因欲大剽易直登樓謂將吏曰能誅為亂者每獲一人賞千萬衆喜倒戈擊亂黨擒國清等三百餘人皆斬之今從實録}
德州刺史王稷承父鍔餘貲家富厚横海節度使李景略利其財|{
	李景略當作李全略}
丙申密敎軍士殺稷屠其家納其女為妾以軍亂聞|{
	象有齒而焚其身賄也王鍔僅能免其身而禍鍾其子君子是以知守富之難}
朝廷之討李㝏也遣司門郎中韋文恪宣慰魏博史憲誠表請授㝏旌節又於黎陽築馬頭為度河之勢|{
	附河岸築土植木夾之至水次以便兵馬入船謂之馬頭}
見文恪辭禮倨慢及聞㝏死辭禮頓恭曰憲誠胡人譬如狗雖被捶擊終不離主耳|{
	捶比蘂翻離力智翻}
冬十一月庚午皇太后幸華清宫辛未上自複道幸

華清宫遂畋于驪山即日還宫太后數日乃返 丙子集王緗薨|{
	緗順宗子}
庚辰上與宦者擊毬於禁中有宦者墜馬上驚因得風疾不能履地自是人不聞上起居宰相屢乞入見不報裴度三上疏請立太子且請入見|{
	見賢遍翻}
十二月辛卯上見羣臣於紫宸殿御大繩牀|{
	程大昌演繁露曰今之交牀制本自虜來始名胡牀隋以䜟有胡改名交牀唐穆宗於紫宸殿御大繩牀見羣臣則又名繩牀矣余按交牀繩牀今人家有之然二物也交牀以木交午為足足前後皆施横木平其底使錯之地而安足之上端其前後亦施横木而平其上横木列竅以穿繩條使之可坐足交午處復為圓穿貫之以鐵歛之可挾放之可坐以其足交故曰交牀繩牀以板為之人坐其上其廣前可容膝後有靠背左右有托手可以閤臂其下四足著地}
悉去左右衛官|{
	去羌呂翻}
獨宦者十餘人侍側人情稍安李逢吉進言景王已長請立為太子裴度請速下詔副天下望既而兩省官亦繼有請立太子者癸巳詔立景王湛為皇太子 |{
	考異曰劉軻牛羊日歷曰穆宗不愈宰臣議立敬宗為皇太子時牛僧孺獨懷異圖欲立諸子僧孺乃昌言於朝曰梁守謙王守澄將不利於上又使楊虡卿漢公輩宣言于外曰王守澄欲謀廢立又令其徒於街衢門牆上施牓每於穆宗行幸處路傍或苑内草間削白而書之冀謀大亂其兇險如此此出于朋黨之言不足信也}
上疾浸瘳 是歲初行宣明歷|{
	憲宗即位司天徐昂上新歷曰觀象起元和二年用之然無蔀章之數至於歛啟閉之候循用舊法測驗不合上立以累世纘緒必更歷紀乃詔歷官改撰歷名曰宣明其氣朔斂日躔月離皆因大衍舊術晷漏交會則稍增損之}


資治通鑑卷二百四十二
