






























































資治通鑑卷二百六   宋 司馬光 撰

胡三省 音注

唐紀二十二【起強圉作噩盡上章困敦六月凡三年有奇}


則天順聖皇后中之下

神功元年【時以契丹破滅九鼎就成以九月大享改元為神功}
正月己亥朔太后享通天宫 突厥默啜寇靈州以許欽明自隨【欽明為默啜所禽見上卷三年厥九勿翻}
欽明至城下大呼求美醤粱米及墨【汜勝之曰梁是秫粟陶弘景曰凡曰梁米皆是粟類惟其牙頭色異為分别耳有青黄白三種青梁味短色惡不如黄白梁呼大故翻}
意欲城中選良將引精兵夜襲虜營【將即亮翻}
而城中無諭其意者 箕州刺史劉思禮學相人於術士張憬藏憬藏謂思禮當歷箕州位至太師思禮念太師人臣極貴非佐命無以致之乃與洛州録事參軍綦連耀謀反【相悉亮翻下相術同憬居永翻唐京都録事參軍正七品綦連虜姓也魏收官氏志西方諸姓有綦連氏}
隂結朝士【朝直遥翻}
託相術許人富貴俟其意悦因說以綦連耀有天命【說輸芮翻}
公必因之以得富貴鳳閣舍人王勮兼天官侍郎事【勮其據翻}
用思禮為箕州刺史明堂尉吉頊聞其謀以告合宫尉來俊臣【高宗總章元年分西京萬年縣為明堂縣永昌元年改東都河南縣為合宫縣宋白曰明堂縣理京兆城中永樂坊}
使上變告之【上時掌翻下同}
太后使河内王武懿宗推之懿宗令思禮廣引朝士許免其死凡小忤意皆引之【忤五故翻}
於是思禮引鳳閣侍郎同平章事李元素夏官侍郎同平章事孫元亨知天官侍郎事石抱忠劉奇給事中周譒【譒補過翻}
及王勮兄涇州刺史勔弟監察御史助等【勔彌兖翻監古衘翻}
凡三十六家皆海内名士窮楚毒以成其獄壬戌皆族誅之親黨連坐流竄者千餘人初懿宗寛思禮於外使誣引諸人諸人旣誅然後收思禮思禮悔之懿宗自天授以來太后數使之鞫獄喜誣䧟人【數所角翻喜許記翻}
時人以為周來之亞來俊臣欲擅其功復羅告吉頊【復扶又翻下是復宗復同}
頊上變得召見僅免【見賢遍翻}
俊臣由是復用而頊亦以此得進俊臣黨人羅告司刑府史樊惎謀反誅之【唐制大理寺有府二十八人史五十六人惎渠記翻}
惎子訟寃於朝堂【朝直遥翻}
無敢理者乃援刀自刳其腹【援于元翻}
秋官侍郎上邽劉如璿見之【上邽縣漢屬隴西郡古邽戎邑也後漢屬漢陽郡後魏諱珪改名上封屬天水郡隋復舊唐屬秦州璿似宣翻}
竊嘆而泣俊臣奏如璿黨惡逆下獄處以絞刑【下遐嫁翻處昌呂翻}
制流瀼州 尚乘奉御張易之行成之族孫也【張行成事太宗}
年少美姿容善音律【少詩沼翻}
太平公主薦易之弟昌宗入侍禁中昌宗復薦易之兄弟皆得幸於太后常傳朱粉衣錦繡昌宗累遷散騎常侍【散悉亶翻騎奇寄翻}
易之為司衛少卿【龍朔改衛尉為司衛光宅因之}
拜其母臧氏韋氏為太夫人賞賜不可勝紀【勝音升}
仍勅鳳閣侍郎李迥秀為臧氏私夫迥秀大亮之族孫也【李大亮歷事高祖太宗}
武承嗣三思懿宗宗楚客晋卿皆候易之門庭爭執鞭轡謂易之為五郎昌宗為六郎 癸亥突厥默啜寇勝州平狄軍副使安道買擊破之【代州北有大武軍調露元年改曰神武軍天授二年改曰平狄軍使疏吏翻}
甲子以原州司馬婁師德守鳳閣侍郎同平章事 春三月戊申清邊道總管王孝傑蘇宏暉等將兵十七萬與孫萬榮戰於東硤石谷唐兵大敗孝傑死之孝傑遇契丹帥精兵為前鋒【將即亮翻帥讀曰率}
力戰契丹引退【契欺訖翻又音喫}
孝傑追之行背懸崕【背蒲妹翻}
契丹回兵薄之【薄音伯各翻}
宏暉先遁孝傑墜崕死將士死亡殆盡 【考異曰朝野僉載云孝傑將四十萬衆被賊誘退逼就懸崖漸漸挨排一一落澗坑深萬丈尸與崖平匹馬無歸單兵莫返張鷟語事多過其實今不盡取}
管記洛陽張說馳奏其事太后贈孝傑官爵遣使斬宏暉以狥使者未至宏暉以立功得免【說讀曰悦使疏吏翻下同}
武攸宜軍漁陽【漁陽秦右北平郡所治也隋為漁陽縣屬幽州在幽州東二百一十里}
聞孝傑等敗沒軍中震恐不敢進契丹乘勝寇幽州攻䧟城邑剽掠吏民攸宜遣將撃之不克【剽匹妙翻將即亮翻}
閻知微田歸道同使突厥冊默啜為可汗【可從刋入聲汗音寒}
知微中道遇突厥使者輒與之緋袍銀帶且上言虜使至都宜大為供張【上時掌翻下同供他用翻張知亮翻}
歸道上言突厥背誕積年方今悔過宜待聖恩寛宥今知微擅與之袍帶使朝廷無以復加【背蒲妹翻朝直遥翻下同復扶又翻}
宜令反初服以俟朝恩【令力丁翻初服突厥遣來所被之服}
又小虜使臣不足大為供張太后然之知微見默啜舞蹈吮其靴鼻【吮如兖翻}
歸道長揖不拜默啜囚歸道將殺之歸道辭色不撓責其無厭【撓奴教翻厭於塩翻}
為陳禍福【為干偽翻}
阿波逹干元珍曰【突厥官二十八等自設至逹干皆世其官此即阿史德元珍}
大國使者不可殺也默啜怒稍解但拘留不遣初咸亨中突厥有降者皆處之豐勝靈夏朔代六州至是默啜求六州降戶及單于都護府之地并穀種繒帛農器鐵【降戶江翻處昌呂翻夏戶雅翻單音蟬種章勇翻繒慈陵翻}
太后不許默啜怒言辭悖慢【悖蒲内翻又蒲沒翻}
姚璹楊再思以契丹未平請依默啜所求給之麟臺少監知鳳閣侍郎贊皇李嶠曰【麟臺少監即祕書少監贊皇縣隋置屬趙州取贊皇山以為名少詩沼翻}
戎狄貪而無信此所謂借寇兵資盗粮也【秦李斯之言}
不如治兵以備之【治直之翻}
璹再思固請與之乃悉驅六州降戶數千帳以與默啜并給穀種四萬斛雜綵五萬段農器三千事鐵四萬斤并許其昏默啜由是益強田歸道始得還與閻知微爭論於太后前歸道以為默啜必負約不可恃和親宜為之備知微以為和親必可保【考異曰舊歸道傳云聖歷初默啜請和遣閻知微冊為立功報國可汗知微擅與使者緋袍歸道上言不可}


【及默啜將至單于都護府乃令歸道攝司賓卿迎勞之默啜請六胡州不許遂拘縶歸道突厥傳云李盡忠孫萬榮䧟營府默啜請為國討契丹許之默啜部衆漸盛則天遣使冊為立功報國可汗朝野僉載云歸道為知微副見默啜不拜默啜倒懸待殺之元珍諫乃放之按神功元年姚璹左遷益州長史則與之穀帛必在此前非聖歷初也實録萬歲通天元年九月丁卯以默啜不同契丹之逆遣閻知微冊為遷善可汗則於時未為立功報國可汗也冊拜此號實録無之不知的在何時今因契丹未平姚璹未出附見於此歸道在朝為左衛郎將何得預論默啜盖在道見知微所為而上所言耳其事則兼采諸書可信者存之}
夏四月鑄九鼎成徙置通天宫豫州鼎高丈八尺受千八百石餘州高丈四尺受千二百石【豫州鼎獨高大神都畿也高古犒翻}
各圖山川物產於其上共用銅五十六萬七百餘斤太后欲以黄金千兩塗之姚璹曰九鼎神器貴於天質自然且臣觀其五采煥炳相雜不待金色以為炫燿【炫熒絹翻}
太后從之自玄武門曳入令宰相諸王帥南北牙宿衛兵十餘萬人并仗内大牛白象共曳之【帥讀曰率}
前益州長史王及善已致仕會契丹作亂山東不安起為滑州刺史太后召見【見賢遍翻}
問以朝廷得失及善陳治亂之要十餘條【治直吏翻}
太后曰外州末事此為根本卿不可出癸酉留為内史 癸未以右金吾衛大將軍武懿宗為神兵道行軍大總管與右豹韜衛將軍何迦密將兵擊契丹【迦古牙翻又居伽翻將即亮翻}
五月癸卯又以婁師德為清邊道副大總管右武威衛將軍沙吒忠義為前軍總管【沙吒虜姓吒初加翻}
將兵二十萬撃契丹先是有朱前疑者【先悉薦翻}
上書云臣夢陛下夀滿八百即拜拾遺又自言夢陛下髪白再玄齒落更生遷駕部郎中【唐駕部郎掌邦國輿輦車乘傳驛廐牧官司馬牛雜畜簿籍辯其出入司其名數上時掌翻下同}
出使還上書曰聞嵩山呼萬歲賜以緋筭袋【唐初職事官三品以上賜金装刀礪石一品以下則冇手中筭袋開元以後百官朔望朝参外官衙日則佩筭袋各隨其所服之色餘日則否使疏吏翻}
時未五品於緑衫上佩之會發兵討契丹敕京官出馬一匹供軍酧以五品前疑買馬輸之屢抗表求進階太后惡其貪鄙【惡烏路翻}
六月乙丑敕還其馬斥歸田里 右司郎中馮翊喬知之有美妾曰碧玉知之為之不昏【為于偽翻}
武承嗣借以敎諸姫遂留不還知之作緑珠怨以寄之【晋石崇有愛妾曰緑珠事見八十三卷晋惠帝永康二年}
碧玉赴井死承嗣得詩於裙帶大怒諷酷吏羅告族之【考異曰唐歷天授元年十月誅喬知之新本紀八月壬戌殺右司郎中喬知之盧藏用陳氏别傳趙儋陳子昂旌德碑皆云契丹以營州叛建安郡王武攸宜親總戎律特詔右補闕喬知之及公参謀幃幕及軍罷以父年老表乞歸侍攸宜討契丹在萬歲通天元年明年平契丹子昂集冇西還至散關荅喬補闕詩云昔君事戎馬余得奉戎旃攜手同沙塞關河緬幽燕嘆此南歸日猶聞北戍邉疑知之之死在神功年後但唐歷統紀新紀殺知之皆在天授元年今據子昂詩必無誤者然猶聞比戍邉則軍未罷也又武后云來俊臣死後不聞有反者故置於此據朝野僉載知之以婢碧玉事為武承嗣諷人羅告之斬於市南破家籍沒此時知之在邉盖承嗣先衘之至此乃殺之耳}
司僕少卿來俊臣【光宅改太僕為司僕}
倚埶貪淫士民妻妾有美者百方取之或使人羅告其辠矯稱敕以取其妻前後羅織誅人不可勝計【勝音升}
自宰相以下籍其姓名而取之【考異曰朝野僉載云俊臣嘗以三月三日萃其黨於龍門竪石題朝士姓名以卜之令投石遥擊倒者則先令告至暮投李昭德不中今不取}
自言才比石勒監察御史李昭德素惡俊臣【惡烏路翻}
又嘗庭辱秋官侍郎皇甫文備二人共誣昭德謀反下獄【下遐嫁翻下不下乃下同}
俊臣欲羅告武氏諸王及太平公主又欲誣皇嗣及廬陵王與南北牙同反冀因此盗國權河東人衛遂忠告之諸武及太平公主恐懼共發其罪繋獄有司處以極刑【處昌呂翻}
太后欲赦之奏上三日不出【上時掌翻}
王及善曰俊臣凶狡貪暴國之元惡不去之必動揺朝廷【去羌呂翻朝直遥翻}
太后遊苑中吉頊執轡太后問以外事對曰外人唯怪來俊臣奏不下太后曰俊臣有功於國朕方思之頊曰于安遠告虺貞反既而果反【貞事見上卷垂拱四年}
今止為成州司馬俊臣聚結不逞誣構良善贓賄如山寃魂塞路【塞悉則翻}
國之賊也何足惜哉太后乃下其奏丁卯昭德俊臣同弃市時人無不痛眧德而快俊臣仇家爭噉俊臣之肉斯須而盡抉眼剝面披腹出心騰蹋成泥【噉徒濫翻又徒覧翻抉於决翻}
太后知天下惡之乃下制數其罪惡【惡烏路翻數所具翻}
且曰宜加赤族之誅以雪蒼生之憤可凖法籍沒其家士民皆相賀於路曰自今眠者背始帖席矣俊臣以告綦連耀功賞奴婢十人俊臣問司農婢無可者【唐六興司農丞掌凡官戶奴婢男女成人先以本色嫓偶若給賜許其妻子相隨若犯籍沒以其所能各配諸司婦人巧者入掖庭}
以西突厥可汗斛瑟羅家有細婢善歌舞欲得以為賞口乃使人誣告斛瑟羅反諸酋長詣闕割耳剺面訟寃者數千人【酋慈由翻長知兩翻剺里之翻}
會俊臣誅乃得免俊臣方用事選司受其屬請不次除官者每銓數百人俊臣敗侍郎皆自首【選須絹翻屬之欲翻首式又翻}
太后責之對曰臣負陛下死罪臣亂國家法辠止一身違俊臣語立見滅族太后乃赦之上林令侯敏【唐司農之屬有上林署令從七品下掌苑囿之事凡植果樹蔬以供朝會祭祀及季冬藏氷皆主之}
素謟事俊臣其妻董氏諫之曰俊臣國賊指日將敗君宜遠之【遠于願翻}
敏從之俊臣怒出為武龍令【武龍縣屬田州開蠻洞置舊書作武籠云失廢置年月又涪州有武龍縣武德二年分涪陵置}
敏欲不往妻曰速去勿留俊臣敗其黨皆流嶺南敏獨得免太后徵于安遠為尚食奉御擢吉頊為右肅政中丞 以檢校夏官侍郎宗楚客同平章事 武懿宗軍至趙州聞契丹將駱務整數千騎將至冀州【將即亮翻下同騎奇寄翻下同}
懿宗懼欲南遁或曰虜無輜重【重直用翻}
以抄掠為資【抄楚交翻}
若按兵拒守埶必離散從而撃之可有大功懿宗不從據相州【相悉亮翻}
委弃軍資器仗甚衆契丹遂屠趙州甲午孫萬榮為奴所殺萬榮之破王孝傑也於柳城西北四百里依險築城留其老弱婦女所獲器仗資財使妹夫乙寃羽守之引精兵寇幽州恐突厥默啜襲其後遣五人至黑沙語默啜曰【黑沙突厥庭語牛倨翻}
我已破王孝傑百萬之衆唐人破膽請與可汗乘勝共取幽州三人先至默啜喜賜以緋袍二人後至默啜怒其稽緩將殺之二人曰請一言而死默啜問其故二人以契丹之情告默啜乃殺前三人而賜二人緋使為鄉道【鄉讀曰嚮}
發兵取契丹新城殺所獲凉州都督許欽明以祭天圍新城三日克之【新城即前契丹所築在柳城西北者}
盡俘以歸使乙寃羽馳報萬榮時萬榮方與唐兵相持軍中聞之忷懼【忷許勇翻}
奚人叛萬榮神兵道總管楊玄基撃其前奚兵擊其後獲其將何阿小萬榮軍大潰【阿烏葛翻考異曰朝野僉載突厥破萬榮新城郡賊聞之失色衆皆潰散不云為玄基所破實録但云為玄基及奚所破不云突厥取新城要之契丹聞新城破衆心己離唐與奚人撃之遂潰耳今兩存之}
帥輕騎數千東走【帥讀曰率}
前軍總管張九節遣兵邀之於道萬榮窮蹙與其奴逃至潞水東【鮑丘水從塞外來南過幽州潞縣謂之潞水}
息於林下嘆曰今欲歸唐罪已大歸突厥亦死歸新羅亦死將安之乎奴斬其首以降【降戶江翻下同}
梟之四方舘門【漢有藁街蠻夷邸後魏置諸國使邸其後又作四舘以處四方來降者事見一百四十九卷梁武帝普通元年至隋焬帝置四方舘於建國門外以待四方使客各掌其方國及互市事屬鴻臚寺唐以四方舘隸中書省通事舍人主之梟堅堯翻}
其餘衆及奚霫皆降於突厥【霫而立翻}
戊子特進武承嗣春官尚書武三思並同鳳閣鸞臺三品辛卯制以契丹初平命河内王武懿宗婁師德及魏州刺史狄仁傑分道安撫河北懿宗所至殘酷民有為契丹所脇從復來歸者【復扶又翻}
懿宗皆以為反生刳取其膽先是何阿小嗜殺人【先悉薦翻}
河北人為之語曰唯此兩何殺人最多【武懿宗封河内王與何阿小為兩何}
秋七月丁酉昆明内附置竇州 武承嗣武三思並罷政事 庚午武攸宜自幽州凱旋武懿宗奏河北百姓從賊者請盡族之左拾遺王求禮庭折之曰【折之舌翻}
此屬素無武備力不勝賊苟從之以求生豈有叛國之心懿宗擁彊兵數十萬望風走賊徒滋蔓又欲委罪於草野詿誤之人【蔓音萬詿戶卦翻}
為臣不忠請先斬懿宗以謝河北懿宗不能對司刑卿杜景儉亦奏此皆脅從之人請悉原之太后從之 八月丙戌納言姚璹坐事左遷益州長史以太子宫尹豆盧欽望為文昌右相鳳閣鸞臺三品【天授中改太子詹事為太子宫尹鳳閣之上當有同字 考異曰新表庚子狄仁傑兼納言武三思檢校内史欽望為文昌右相同三品舊紀傳及新紀皆無之此月無庚子仁傑三思除命在明年新表誤重複}
九月壬辰大享通天宫大赦改元【改元神功}
庚戌婁師德守納言 甲寅太后謂侍臣曰頃者周興來俊臣按獄多連引朝臣【朝直遥翻}
云其謀反國有常法朕安敢違中間疑其不實使近臣就獄引問得其手狀皆自承服朕不以為疑自興俊臣死不復聞有反者【復扶又翻下無復后復同}
然則前死者不有寃邪夏官侍郎姚元崇對曰自垂拱以來坐謀反死者率皆興等羅織自以為功陛下使近臣問之近臣亦不自保何敢動揺所問者若有翻覆懼遭惨毒不若速死賴天啓聖心興等伏誅臣以百口為陛下保自今内外之臣無復反者【為于偽翻下多為同}
若微有實狀臣請受知而不告之罪太后悦曰曏時宰相皆順成其事䧟朕為淫刑之主聞卿所言深合朕心賜元崇錢千緡時人多為魏元忠訟寃者太后復召為肅政中丞元忠前後坐弃市流竄者四【考異曰舊傳云三被流今從御史臺記按新書元忠為洛陽令䧟周興獄當死以平楊楚功得流歲餘為來俊臣所構將就刑太后使王隱客宣詔赦之此為二事通鑑書王隱客宣赦事於永昌元年至長夀元年又下獄貶此為三事及後長安三年又貶高要尉此為四事未知御史臺記所書如何也}
嘗侍宴太后問曰卿往者數負謗何也【數所角翻}
對曰臣猶鹿耳羅織之徒欲得臣肉為羹臣安所避之 冬閏十月甲寅以幽州都督狄仁傑為鸞臺侍郎司刑卿杜景儉為鳳閣侍郎並同平章事仁傑上疏【上時掌翻}
以為天生四夷皆在先王封畧之外故東距滄海西阻流沙北横大漠南阻五嶺此天所以限夷狄而隔中外也自典籍所紀聲教所及三代不能至者國家盡兼之矣詩人矜薄伐於太原美化行於江漢【詩六月宣王北伐也其詩云薄伐玁狁至於太原又廣漢之詩美文王之道被于南國美化行乎江漢之域}
則三代之遠裔皆國家之域中也若乃用武方外邀功絶域竭府庫之實以爭不毛之地得其人不足增賦獲其土不可耕織苟求冠帶遠夷之稱【冠古玩翻稱尺證翻}
不務固本安人之術此秦皇漢武之所行非五帝三王之事業也始皇窮兵極武務求廣地死者如麻致天下潰叛【事見秦紀}
漢武征伐四夷百姓困窮盗賊蜂起末年悔悟息兵罷役故能為天所祐【事見漢武帝紀}
近者國家頻歲出師所費滋廣西戍四鎮東戍安東調發日加【調徒釣翻}
百姓虚弊今關東飢饉蜀漢逃亡江淮已南徵求不息人不復業相率為盗本根一揺憂患不淺其所以然者皆以爭蠻貊不毛之地乖子養蒼生之道也【貊莫百翻}
昔漢元納賈捐之之謀而罷朱崖郡【事見二十八卷初元二年}
宣帝用魏相之策而弃車師之田【事見二十五卷元康二年}
豈不欲慕尚虚名盖憚勞人力也近貞觀中克平九姓立李思摩為可汗使統諸部者【見一百九十卷貞觀十三年}
盖以夷狄叛則伐之降則撫之得推亡固存之義【書仲虺之語曰推亡固存邦乃其昌推吐雷翻}
無遠戍勞人之役此近日之令典經邊之故事也竊謂宜立阿史那斛瑟羅為可汗委之四鎮繼高氏絶國【謂高麗也}
使守安東省軍費於遠方并甲兵於塞上使夷狄無侵侮之患則可矣何必窮其窟宂與螻蟻校長短哉但當敕邊兵謹守備遠斥候聚資粮待其自致然後撃之以逸待勞則戰士力倍以主禦客則我得其便堅壁清野則寇無所得自然二賊深入則有顛躓之慮淺入必無寇獲之益如此數年可使二虜不撃而服矣【二賊二虜皆謂突厥吐蕃}
事雖不行識者是之 鳳閣舍人李嶠知天官選事【選須絹翻}
始置員外官數千人 先是歷官以是月為正月以臘月為閏【先悉薦翻}
太后欲正月甲子朔冬至乃下制以為去晦仍見月有爽天經【去晦謂前月晦也}
可以今月為閏月來月為正月

聖歷元年正月甲子朔冬至太后享通天宫 【考異曰實録云正月壬戍享通天宫按長歷此年一月壬戍朔實録誤也今從唐歷統紀新本紀}
赦天下改元夏官侍郎宗楚客罷政事 春二月乙未文昌右相同鳳閣鸞臺三品豆盧欽望罷為太子賓客 武承嗣三思營求為太子數使人說太后曰自古天子未有以異姓為嗣者太后意未决狄仁傑每從容言於太后曰文皇帝櫛風沐雨親冒鋒以定天下傳之子孫【數所角翻說輪芮翻從千容翻太宗謚文皇帝}
大帝以二子託陛下【高宗謚天皇大帝二子謂廬陵王及皇嗣也}
陛下今乃欲移之他族無乃非天意乎且姑姪之與母子孰親【太后之於承嗣三思姑姪也於廬陵王皇嗣母子也}
陛下立子則千秋萬歲後配食太廟承繼無窮立姪則未聞姪為天子而祔姑於廟者也太后曰此朕家事卿勿預知仁傑曰王者以四海為家四海之内孰非臣妾何者不為陛下家事君為元首臣為股肱義同一體况臣備位宰相豈得不預知乎又勸太后召還廬陵王【廬陵王光宅元年遷均州垂拱元年遷房州}
王方慶王及善亦勸之太后意稍寤它日又謂仁傑曰朕夢大鸚鵡兩翼皆折何也【折而設翻}
對曰武者陛下之姓兩翼二子也陛下起二子則兩翼振矣太后由是無立承嗣三思之意孫萬榮之圍幽州也移檄朝廷曰何不歸我廬陵王吉頊與張易之昌宗皆為控鶴監供奉【是年置拱鶴監以處近倖}
易之兄弟親狎之頊從容說二人曰公兄弟貴寵如此非以德業取之也天下側目切齒多矣不有大功於天下何以自全竊為公憂之【為于偽翻下屢為復為同}
二人懼流涕問計頊曰天下士庶未忘唐德咸復思廬陵王【復扶又翻}
主上春秋高大業須有所付武氏諸王非所屬意【屬之欲翻}
公何不從容勸上立廬陵王以繋蒼生之望如此非徒免禍亦可以長保富貴矣二人以為然承間屢為太后言之【間古莧翻}
太后知謀出於頊乃召問之頊復為太后具陳利害太后意乃定 【考異曰世有狄梁公傳云李邕撰其辭鄙誕殆非邕所為其言曰后納諸武之議將移宗社擬立武三思為儲副遷廬陵於房陵諸武隂計日夜獻謀曰陛下姓武合立武氏未冇天子而取别姓將為後者也天后既已許禮問羣臣曰朕年齒將衰國無儲主今欲擇善誰可當之朕雖得人終在群議諸宰臣多聞計定言皆希旨仁傑獨立無一言天后問曰卿獨無言當有異見公曰有之臣上觀乾象無易主之文中察人心實未厭唐德天后曰卿何以知之公曰頃者匈奴犯邉陛下使梁王三思於都市召募一月之外不滿千人後廬陵王踵之未經二旬數盈五萬以此觀之人心未去陛下將欲繼統非廬陵王餘實非臣所知天后震怒命左右扶而去之按廬陵王為河北元帥在立為太子後且當是時睿宗為皇嗣若仁傑請以廬陵王繼統則是勸太后廢立也此固未可信或者仁傑以廬陵母子至親而幽囚房陵勸召還左右則有之矣談賓録曰聖歷二年臘月張易之兄弟貴寵逾分懼不全請計於天官侍郎吉頊頊曰公兄弟承恩深矣非有大功於天下自古罕冇全者唯有一策苟能行之豈止全家亦當享茅土之封耳除此之外非頊所謀易之兄弟泣請之頊曰天下思唐德久矣主上春秋高武氏諸王殊非所屬意公何不從容請立廬陵以繋生人之望易之乃乘間屢言之則天意乃易既知頊首謀乃召問頊頊曰廬陵相王皆陛下之子高宗切託於陛下唯陛下裁之則天意乃定御史臺記曰則天置控鶴府頊與易之昌宗同於府供奉與昌宗親狎昌宗自以貴寵踰分懼不全問計於頊頊云云如談賓録盖太后寵信諸武誅鉏李氏雖己子廬陵亦廢徙房陵故仁傑勸召還左右以強李氏抑諸武耳張吉之能為唐社稷謀也欲求己利耳若仍立皇嗣則己有何功故勸太后立廬陵為太子而太后從之然則欲召還廬陵者仁傑之志也立為太子張吉之謀也談賓言聖歷二年及以頊為天官侍郎臺記謂睿宗為相王則皆誤也新狄仁傑傳云張易之嘗從容問自安計仁傑曰唯勸迎廬陵王可以免祸計仁傑亦安肯與易之深言此事狄梁公傳又云後經旬召公入曰朕昨夜夢與人雙陸頻不見勝何也對曰雙陸不勝盖為宫中無子此是上天之意假此以示陛下安可久虚儲位哉天后曰是朕家事斷在胸中卿豈合預焉仁傑對曰臣聞王者以天下為家四海之内悉為臣妾何者不為陛下家事君為元首臣為股肱臣安得不預焉又命扶出竟不納按於時王嗣在宫中不得言無子及久虚儲位也朝野僉載云則天曾夢一鸚鵡羽毛甚偉兩翅俱折以問宰臣群公默然内史狄仁傑曰鵡者陛下姓也兩翅折者陛下二子廬陵相王也陛下起此二子兩翅全也魏王承嗣武三思連頃皆赤後契丹圍幽州檄朝廷曰還我廬陵相王來則天乃憶狄公之言謂之曰卿曾為我占夢今乃應矣朕欲立太子何者為得仁傑曰陛下内有賢子外有賢姪取舍詳擇斷在宸衷則天曰我自有聖子承嗣三思是何疥癬承嗣等懼掩耳而走即降敕追廬陵河内王等奏不許入城龍門安置賊徒轉盛䧟沒冀州則天急乃立廬陵王為太子充元帥初募兵無有應者聞太子行比邙山頭兵滿無容人處賊自退散按是時睿宗未為相王又仁傑若言内有賢子外有賢姪乃是懷兩端也今採衆說之可信者存之}
三月己巳託言廬陵王有疾遣職方員外郎瑕丘徐彦伯【瑕丘故春秋魯之瑕邑晋宋置兖州於此隋開皇十三年置瑕丘縣帶兖州}
召廬陵王及其妃諸子詣行在療疾戊子廬陵王至神都 【考異曰統紀云癸丑遣職方員外郎徐彦伯往房州召廬陵王男女入都醫療狄梁公傳曰後潜發内人十人至房州宣敕云我兒在此令内人就看州縣長吏仰數出數入無令混雜隂令内人一人以代廬陵王令廬陵王衣内人衣服以舊數還州縣不悟數日逹京朝廷百僚一無知者舊傳曰廬陵王自房陵還宫太后匿之帳中又召狄仁傑以廬陵為言仁傑慷慨敷奏言發涕流遽出廬陵謂仁傑曰還卿儲君仁傑降階泣賀既已奏曰太子還宫人無知者物議安審是非則天以為然乃復置中宗於龍門具禮迎歸人情感悦狄梁公傳曰天后御一小殿垂簾於後左右隱蔽外不能知乃命公坐於階下曰前者所議事實非小寤寐反覆思卿所言彌覺理非甚乖朕意忠臣事主豈在多違今日之間須易前見以天下之位在卿一言可朕意即兩全逆朕心即俱斃公從容言曰陛下所言天子之位可得專之以臣所知是太宗文武皇帝之位陛下豈得而自有也太宗身䧟鋒鏑經綸四海所以不告勞者盖為子孫豈為武三思邪陛下身是大帝皇后大帝寢疾權使陛下監國大帝崩後合歸冢嫡陛下遂奄有神器十有餘年今議纘承豈可更易且姑與母孰親子與姪孰近云云太后於是歔欷流涕命左右褰簾手撫公背大叫曰卿非朕之臣是唐社稷之臣回謂廬陵王曰拜國老今日國老與爾天子公免冠頓首涕血灑地左右扶策久不能起天后曰即具所言宣付中外擇日禮冊公揮涕而言曰自古以來豈有偷人作天子廬陵王留在房州天下所悉知今日在内臣亦不知臣欲奉詔若同衛太子之變陛下何以明臣天后曰安可却向房陵只於石像驛安置具法駕陳百僚就迎之於是大呼萬歲儲位乃定按武后若密召廬陵王宫人十人既知其謀洛陽至房陵往來道路甚遠豈得外人都不知乎又實録豈能搆虛立徐彦伯往迎之事及冇廬陵王至自房州之日又於時若儲位已定豈可自三月來九月始立為太子盖廬陵既至太后以長幼之次欲立之皇嗣亦以此遜位故遷延半載今皆取實録為正}
夏四月庚寅朔太后祀太廟 辛丑以婁師德充隴右諸軍大使仍檢校營田事【使疏吏翻}
六月甲午命淮陽王武延秀入突厥納默啜女為妃豹韜衛大將軍閻知微攝春官尚書右武衛郎將楊齊莊攝司賓卿【考異曰實録作楊鸞莊今從僉載舊傳}
齎金帛巨億以送之延秀承嗣之

子也鳳閣舍人襄陽張柬之諫曰自古未有中國親王娶夷狄女者由是忤旨【忤五故翻}
出為合州刺史【襄陽縣漢屬南郡獻帝建安十三年置襄陽郡晋為荆州治所宋齊梁為雍州西魏為襄州合州漢墊江縣地南齊置東宕渠郡西魏改墊江郡置石鏡縣尋置合州隋改涪州唐復為合州舊志合州京師南二千四百五十里至東三千三百}
秋七月鳳閣侍郎同平章事杜景儉罷為秋官尚書

八月戊子武延秀至黑沙南庭突厥默啜謂閻知微等曰我欲以女嫁李氏安用武氏兒邪此豈天子之子乎我突厥世受李氏恩聞李氏盡滅唯兩兒在我今將兵輔立之【將即亮翻}
乃拘延秀於别所以知微為南面可汗言欲使之主唐民也遂發兵襲静難平狄清夷等軍【垂拱中置清夷軍於媯州界杜佑曰在城内南去范陽二百十里難乃旦翻}
静難軍使慕容玄崱以兵五千降之【使疏吏翻崱士力翻降戶江翻}
虜埶大振進寇媯檀等州【媯居為翻}
前從閻知微入突厥者默啜皆賜之五品三品之服太后悉奪之默啜移書數朝廷曰【數所具翻}
與我蒸穀種種之不生一也金銀器皆行濫非真物二也【穀種章勇翻行戶剛翻市列為行市列造金銀器販賣率殽它物以求贏俗謂之行作濫惡也開元八年頒租庸調法於天下好不過精惡不過濫濫者惡之極者也}
我與使者緋紫皆奪之三也繒帛皆踈惡四也我可汗女當嫁天子兒武氏小姓門戶不敵罔冒為昏五也我為此起兵欲取河北耳【為于偽翻}
監察御史裴懷古從閻知微入突厥默啜欲官之不受囚將殺之逃歸抵晉陽形容羸悴【監古衘翻羸倫為翻悴秦醉翻}
突騎譟聚以為間諜欲取其首以求功有果毅嘗為人所枉懷古按直之大呼曰裴御史也救之得全至都引見遷祠部員外郎【間古莧翻諜逹恊翻呼火故翻見賢遍翻唐祠部郎掌祠祀享祭天文漏刻國忌廟諱卜筮醫藥僧尼之事屬禮部}
時諸州聞突厥入寇方秋爭發民修城衛州刺史太平敬暉【後魏分漢臨汾縣地置太平縣隋唐屬絳州}
謂屬曰吾聞金湯非粟不守柰何捨牧穫而事城郭乎悉罷之使歸田百姓大悦 甲午鸞臺侍郎同平章事王方慶罷為麟臺監 太子太保魏宣王武承嗣恨不得為太子意怏怏戊戌病薨【怏於兩翻}
庚子以春官尚書武三思檢校内史狄仁傑兼納言太后命宰相各舉尚書郎一人仁傑舉其子司府丞光嗣【光宅改大府曰司府}
拜地官員外郎已而稱職太后喜曰卿足繼祁奚矣【左傳晉中軍尉祁奚請老晉侯問嗣焉稱解狐其讎也將立之而卒又問之曰午也可於是以祁午為中軍尉君子謂祁奚能舉其善矣稱其讎不為謟立其子不為比稱尺證翻}
通事舍人河南元行冲【唐六典曰通事舍人即秦之謁者晉武帝省謁者僕射置舍人通事各一人隸中書東晉令舍人通事兼謁者之任通事舍人之名始此也唐通事舍人十六人掌朝見引納及辭謝者於殿庭通奏凡近臣文武就列則引以進退而告其拜起出入之節凡四方通表華夷納貢皆受而進之}
博學多通仁傑重之行冲數規諫仁傑且曰凡為家者必有儲蓄脯醢以適口參术以攻疾【數所角翻參所今翻人參也}
僕竊計明公之門珍味多矣行冲請備藥物之末仁傑笑曰吾藥籠中物何可一日無也【籠力董翻}
行冲名澹以字行 以司屬卿武重規為天兵中道大總管【光宅改宗正為司屬緣此後置天兵軍於并州城中}
右武衛將軍沙吒忠義為天兵西道總管【吒初加翻}
幽州都督下邽張仁愿為天兵東道總管【秦武公代邽戎置下邽縣隴西有上邽故此加下字漢屬京北晉屬馮翊後魏置延夀郡隋廢郡以下邽屬同州垂拱元年屬華州}
將兵三十萬以討突厥默啜【將即亮翻}
又以左羽林衛大將軍閻敬容為天兵西道後軍總管將兵十五萬為後援癸丑默啜寇飛狐【漢代郡廣昌縣有飛狐口隋改廣昌為飛狐縣屬易州唐屬蔚州}
乙卯陷定州殺刺史孫彦高及吏民數千人【考異曰朝野僉載曰文昌左丞孫彦高無它識用性惟頑愚出為定州刺史歲餘默啜賊至圍其郛部彦高}


【却璅宅門不敢請聽事文按須徵發者於小牕内接入通判仍簡郭下精健自援其家賊既乘城四面並入彦高乃謂奴曰牢關門戶莫與鑰匙其愚怯也皆此類俄而陷沒刺史之宅先殱焉又曰彦高被突厥圍城數重彦高乃入匱中藏令奴曰牢掌鑰匙賊來索慎勿與恐不至此今不取}
九月甲子以夏官尚書武攸寧同鳳閣鸞臺三品 改默啜為斬啜默啜使閻知微招諭趙州知微與虜連手蹋萬歲樂於城下將軍陳令英在城上謂曰尚書位任非輕乃為虜蹋歌獨無慙乎【為于偽翻蹋歌者連手而歌蹋地以為節萬歲樂歌曲之名樂音洛}
知微微吟曰不得已萬歲樂戊辰默啜圍趙州長史唐般若翻城應之【長知兩翻般北末翻若人者翻}
刺史高叡與妻秦氏仰藥詐死虜輿之詣默啜默啜以金獅子帶紫袍示之曰降則拜官不降則死【降戶江翻}
叡顧其妻妻曰酧報國恩正在今日遂俱閉目不言經再宿虜知不可屈乃殺之虜退唐般若族誅贈叡冬官尚書謚曰節叡熲之孫也【冬官工部尚辰羊翻謚神至翻高熲隋初佐命}
皇嗣固請遜位於廬陵王太后許之壬申立廬陵王哲為皇太子復名顕【嗣祥吏翻復扶又翻又音如字}
赦天下甲戌命太子為河北道元帥以討突厥【行軍元帥起於周隋至唐唯親王及太子為元帥帥所類翻 考異曰實録云丙子據唐歷甲戌皇太子顕充河北道行軍大元帥狄梁公傳亦云皇太子為元帥以公為副是先立為太子後為元帥也今從新本紀}
先是募人月餘不滿千人【先悉薦翻}
及聞太子為元帥應募者雲集未幾數盈五萬【幾居豈翻}
戊寅以狄仁傑為河北道行軍副元帥右丞宋元爽為長史右臺中丞崔獻為司馬左臺中丞吉頊為監軍使【后分御史臺為左右肅政臺各置中丞侍御史等官頊呼玉翻監古衘翻使疏吏翻}
時太子不行命仁傑知元帥事太后親送之藍田令薛訥仁貴之子也【藍田畿縣屬雍州薛仁貴健將也事太宗高宗}
太后擢為左威衛將軍安東道經略將行言於太后曰太子雖立外議猶疑未定苟此命不易醜虜不足平也太后深然之王及善請太子赴外朝以慰人心從之【朝直遥翻考異曰實録辛巳皇太子朝見或作廟見盖睿宗為皇嗣時止於宫中朝謁不出外朝今及善始請太子與群臣俱於外庭朝謁耳}
以天官侍郎蘇味道為鳳閣侍郎同平章事味道前後在相位數歲依阿取容嘗謂人曰處事不宜明白但摸稜持兩端可矣時人謂之蘇摸稜【天官吏部相悉亮翻處昌呂翻摸音莫}
癸未突厥默啜盡殺所掠趙定等州男女萬餘人自

五回道去【厥九勿翻啜叱列翻水經注代郡廣昌縣東南有大嶺世謂之廣昌嶺嶺高四十餘里二十里中委折五回方得逹其上嶺故嶺有五回之名時屬易州易縣界至開元二十三年分易縣置五回縣於五回山下}
所過殺掠不可勝紀沙吒忠義等但引兵躡之不敢逼【勝音升吒初加翻躡泥輒翻考異曰舊突厥傳云默啜盡寇掠趙定等州男女八九萬人統紀云河北積年豐熟人畜被野默啜虜趙定恒易等州財帛億萬子女羊馬而去河朔諸州怖其兵威不敢追躡今從實録}
狄仁傑將兵十萬追之無所及【將即亮翻又音如字}
默啜還漠北擁兵四十萬據地萬里西北諸夷皆附之甚有輕中國之心 冬十月制都下屯兵命河内王武懿宗九江王武攸歸領之 癸卯以狄仁傑為河北道安撫大使時北人為突厥所驅逼者虜退懼誅往往亡匿仁傑上疏以為朝廷議者皆罪契丹突厥所脅從之人言其迹雖不同心則無别【使疏吏翻上時掌翻别彼列翻}
誠以山東近緣軍機調發傷重【調徒弔翻}
家道悉破或至逃亡重以官典侵漁【重以直用翻}
因事而起枷杖之下痛切肌膚事廹情危不循禮義愁苦之地不樂其生有利則歸且圖賖死此乃君子之愧辱小人之常行也【樂音洛行下孟翻}
又諸城入偽【入偽謂降賊者}
或待天兵將士求功皆云攻得臣憂濫賞亦恐非辜【以攻取之賞賞將士則為濫賞以從虜之罪罪士民則為非辜}
以經與賊同是為惡地至於汚辱妻子【汚烏故翻}
刼掠貨財兵士信知不仁簪笏未能以免【簪笏謂士大夫當官而行者也}
乃是賊平之後為惡更深且賊務招擕秋毫不犯【言除賊務在招撫攜貳秋毫無所侵犯也}
今之歸正即是平人翻被破傷豈不悲痛【被皮義翻}
夫人猶水也壅之則為泉疏之則為川通塞隨流【塞悉則翻}
豈有常性今負罪之伍必不在家露宿草行潜竄山澤赦之則出不赦則狂山東羣盗緣兹聚結臣以邊塵蹔起不足為憂【蹔與暫同}
中土不安此為大事罪之則衆情恐懼恕之則反側自安伏願曲赦河北諸州一無所問制從之仁傑於是撫慰百姓得突厥所驅掠者悉逓還本貫散粮運以賑貧乏修郵驛以濟旋師恐諸將及使者妄求供頓乃自食疏糲【郵音尤將即亮翻使疏吏翻疏麤也糲脱粟也一斛粟得六斗米為糲糲郎葛翻}
禁其下無得侵擾百姓犯者必斬河北遂安 以夏官侍郎姚元崇祕書少監李嶠並同平章事 突厥默啜離趙州【離力智翻}
乃縱閻知微使還太后命磔於天津橋南【磔陟格翻張也開也}
使百官共射之既乃咼其肉【射而亦翻下既射同咼古瓦翻剔人肉至骨也}
剉其骨夷其三族疏親有先未相識而同死者 【考異曰朝野僉載云則天磔知微於西市命百官射之河内王懿宗去七步射一發皆不中怯懦如此知微身上箭如蝟毛剉其骨肉夷其九族小兒年七八歲嫗抱向西市百姓哀之擲餅果與者仍相爭奪以為戲笑監刑御史不忍害奏捨之今從實録}
褒公段瓚志玄之子也【段志玄從起晉陽征伐有功瓚藏旱翻}
先沒於突厥突厥在趙州瓚邀楊齊莊與之俱逃齊莊畏懦不敢發【懦乃臥翻又奴亂翻}
瓚先歸太后賞之齊莊尋至敕河内王武懿宗鞫之懿宗以為齊莊意懷猶豫遂與閻知微同誅既射之如蝟氣殜殜未死【殜余懾翻}
乃决其腹割心投於地猶趌趌然躍不止【趌起逸翻}
擢田歸道為夏官侍郎甚見親委 蜀州每歲遣兵五百人戍姚州【蜀州漢江源武陽之地李雄置江源郡晉為晉原縣隋廢郡以縣屬益州垂拱二年分置蜀州}
路險遠死亡者多蜀州刺史張柬之上言以為姚州本哀牢之國【哀牢夷見四十五卷漢明帝永平十二年}
荒外絶域山高水深國家開以為州【武德四年以漢益州郡之雲南縣地置姚州以地人多姓姚故也舊志至京師四千九百里麟德元年移治弄棟川}
未嘗得其鹽布之税甲兵之用而空竭府庫驅率平人受役蠻夷肝腦塗地臣竊為國家惜之【竊為于偽翻}
請廢姚州以隸嶲州歲時朝覲同之蕃國【嶲音髓朝直遥翻}
瀘南諸鎮亦皆廢省於瀘北置關【瀘音盧}
百姓非奉使無得交通往來【使疏吏翻}
疏奏不納二年正月丁卯朔告朔於通天宫【告古沃翻又如字}
壬戌以皇嗣為相王領太子右衛率【相息亮翻率所律翻}
甲子置控鶴監丞主簿等官【先已置控鶴監今方備官}
率皆嬖寵之人【嬖卑義翻又博計翻}
頗用才能文學之士以參之以司衛卿張易之為控鶴監銀青光禄大夫張昌宗左臺中丞吉頊殿中監田歸道夏官侍郎李迥秀鳳閣舍人薛稷正諫大夫臨汾員半千【臨汾縣帶晉州本平陽縣隋更名半千本彭城劉氏十世祖凝之事宋及齊受禅奔魏以忠烈自比五員因自姓員員音云}
皆為控鶴監内供奉稷元超之從子也【薛元超事高宗從才用翻}
半千以古無此官且所聚多輕薄之士上疏請罷之由是忤旨【上時掌翻忤五故翻}
左遷水部郎中 臘月戊子以左臺中丞吉頊為天官侍郎右臺中丞魏元忠為鳳閣侍郎並同平章事 文昌左丞宗楚客與弟司農卿晉卿坐贓賄滿萬餘緡及第舍過度楚客貶播州司馬晉卿流峰州【峯州漢交趾麊冷縣地吴置新興郡晉改新昌郡齊置興州隋初改華州十八年改峯州大業廢州併入交趾為嘉寧縣唐武德四年復置峯州舊志播州去京師四千五百三十里東都四千九百六十里峰州至京師七千七百一十里}
太平公主觀其第歎曰見其居處【處昌呂翻}
吾輩乃虚生耳 辛亥賜太子姓武氏赦天下 太后生重眉成八字【重直龍翻}
百官皆賀 河南北置武騎團以備突厥【騎奇寄翻}
春一月庚申夏官尚書同鳳閣鸞臺三品武攸寧罷為冬官尚書 二月己丑太后幸嵩山過緱氏謁升仙太子廟【緱氏縣屬洛州升仙太子周王子晉也世傳晉升仙後桓良遇之於嵩山曰七月七日待我於緱氏山頭果乘白鶴駐山頂舉手謝時人而去後人因為立祠后加號升仙太子杜祐曰緱氏縣古滑國緱工侯翻}
壬辰太后不豫遣給事中欒城閻朝隱禱少室山朝隱自為犧牲【朝直遥翻}
沐浴伏俎上請代太后命太后疾小愈厚賞之丁酉自緱氏還 初吐蕃贊普器弩悉弄尚幼論欽陵兄弟用事皆有勇略諸胡畏之欽陵居中秉政諸弟握兵分據方面贊婆常居東邊為中國患者三十餘年器弩悉弄浸長【長知兩翻}
隂與大臣論巖謀誅之會欽陵出外贊普詐云出畋集兵執欽陵親黨二千餘人殺之遣使召欽陵兄弟欽陵等舉兵不受命贊普將兵討之欽陵兵潰自殺夏四月贊婆帥所部千餘人來降【使疏吏翻將即亮翻帥讀曰率降戶江翻 考異曰實録贊婆及其兄弟莽布支等來降以莽布支為左羽林衛員外大將軍封安國公按贊婆弟名悉多於敷論明年吐蕃將莽布支寇凉州與唐休璟戰未詳實録所云今刪去}
太后命左武衛鎧曹參軍郭元振與河源軍大使夫蒙令卿將騎迎之【夫蒙姓也姓譜夫蒙羌複姓後秦有建威將軍夫蒙大羌}
以贊婆為特進歸德王欽陵子弓仁以所統吐谷渾七千帳來降拜左玉鈐衛將軍酒泉郡公壬辰以魏元忠檢校并州長史充天兵軍大總管以

備突厥婁師德為天兵軍副大總管仍充隴右諸軍大使專掌懷撫吐蕃降者 太后春秋高慮身後太子與諸武不相容壬寅命太子相王太平公主與武攸暨等為誓文告天地於明堂銘之鐵劵藏於史舘 秋七月命建安王武攸宜留守西京代會稽王武攸望【守式又翻會工外翻}
丙辰吐谷渾部落一千四百帳内附【吐從瞰入聲谷音浴}
八月癸巳突騎施烏質勒遣其子遮弩入見【西突厥既敗突騎施始盛突騎施烏質勒者西突厥之别種也初隸斛瑟羅下號莫賀逹於後斛瑟羅入朝其地為烏質勒所併騎奇寄翻見賢遍翻}
遣侍御史元城解琬安撫烏質勒及十姓部落【解戶買翻}
制州縣長吏非奉有敕旨毋得擅立碑【長知兩翻}
内史王及善雖無學術然清正難奪有大臣之節張

易之兄弟每侍内宴無復人臣禮【復扶又翻}
及善屢奏以為不可太后不悦謂及善曰卿既年高不宜更侍遊宴但檢校閤中可也【閤謂省閤也}
及善因稱病謁假月餘【假古訝翻}
太后不問及善嘆曰豈有中書令而天子可一日不見乎事可知矣乃上疏乞骸骨【上時掌翻疏所去翻}
太后不許庚子以及善為文昌左相太子宫尹豆盧欽望為文昌右相仍並同鳳閣鸞臺三品【相悉亮翻下同 考異曰新紀表及善同平章事今從實録朝野僉載曰王及善才行庸猥風神鈍濁為内史時人號為鳩集鳳池俄遷文昌右相無它政但不許令史奴驢入臺終日廹逐無時蹔捨時人號驅驢宰相此盖張文成惡及善毁之耳今從舊傳}
鸞臺侍郎同平章事楊再思罷為左臺大夫【即左御史大夫}
丁未相王兼檢校安北大都護以天官侍郎陸元方為鸞臺侍郎同平章事 納言隴右諸軍大使婁師德薨【使疏吏翻}
師德在河隴前後四十餘年㳟勤不怠民夷安之性沈厚寛恕狄仁傑之入相也師德實薦之而仁傑不知意頗輕師德數擠之於外【沈持林翻數所角翻擠子西翻又子細翻}
太后覺之嘗問仁傑曰師德賢乎對曰為將能謹守邊陲【將即亮翻下同}
賢則臣不知又曰師德知人乎對曰臣嘗同僚未聞其知人也太后曰朕之知卿乃師德所薦也亦可謂知人矣仁傑既出歎曰婁公盛德我為其所包容久矣吾不得窺其際也是時羅織紛紜師德久為將相獨能以功名終人以是重之 戊申以武三思為内史 九月乙亥太后幸福昌【福昌縣屬東都本宜陽縣武德二年更名因隋福昌宫以名縣也}
戊寅還神都 庚子邢貞公王及善薨 河溢漂濟源百姓廬舍千餘家【濟源本春秋時原邑漢屬河東垣縣界隋開皇十六年置濟源縣屬懷州濟子禮翻}
冬十月丁亥論贊婆至都太后寵待賞賜甚厚以為右衛大將軍使將其衆守洪源谷【洪源谷在凉州昌松縣界使將即亮翻}
太子相王諸子復出閤【相王諸子幽宫中見二百四卷天授二年復扶又翻下不復同}
太后自稱制以來多以武氏諸王及駙馬都尉為成均祭酒博士助敎亦多非儒士又因郊丘明堂拜洛封嵩【郊丘祭圜丘於南郊也享萬象神宫及享通天宫皆明堂也垂拱四年拜洛萬歲通天元年封嵩山}
取弘文國子生為齋郎【齋郎者執豆籩奉樽彛罍洗以供祭祀之事}
因得選補由是學生不復習業二十年間學校殆廢而曏時酷吏所誣䧟者其親友流離未獲原宥鳳閣舍人韋嗣立上疏【上時掌翻疏所去翻}
以為時俗浸輕儒學先王之道弛廢不講宜令王公以下子弟皆入國學不聽以它岐仕進又自揚豫以來【謂徐敬業起兵於揚州越王貞起兵於豫州也}
制獄漸繁酷吏乘閒專欲殺人以求進【間古莧翻}
賴陛下聖明周丘王來相繼誅殛【天授二年周興流死丘神勣誅延載元年王弘義誅神功元年來俊臣誅}
朝野慶泰若再覩陽和【朝直遥翻}
至如仁傑元忠往遭案鞫亦皆自誣非陛下明察則以為葅醢矣今陛下升而用之皆為良輔何乃前非而後是哉誠由枉陷與甄明耳【甄稽延翻}
臣恐曏之負寃得罪者甚衆亦皆如是伏望陛下弘天地之仁廣雷雨之施【施式志翻}
自垂拱以來罪無輕重一皆昭洗死者追復官爵生者聽還鄉里如此則天下知昔之枉濫非陛下之意皆獄吏之辜幽明歡欣感通和氣太后不能從嗣立承慶之異母弟也母王氏遇承慶甚酷每放承慶嗣立必解衣請代母不許輒私自杖母乃為之漸寛【為于偽翻}
承慶為鳳閣舍人以疾去職嗣立時為萊蕪令【萊蕪縣漢屬泰山郡晉廢後魏於古城置嬴縣唐貞觀初廢入傳城縣后復於廢嬴縣置蕪萊縣屬兖州}
太后召謂曰卿父嘗言臣有兩兒堪事陛下卿兄弟在官誠如父言朕今以卿代兄更不用它人即日拜鳳閣舍人 是歲突厥默啜立其弟咄悉匐為左廂察【咄當沒翻匐蒲北翻下同}
骨篤禄子默矩為右廂察各主兵二萬餘人其子匐俱為小可汗位在兩察上主處木昆等十姓兵四萬餘人又號為拓西可汗【處木昆十姓西突厥所部也故號拓西}


久視元年【是年五月始改元}
正月戊寅内史武三思罷為特進太子少保 【考異曰新紀表皆云戊午貶吉頊為琰川尉壬申三思罷中間未嘗復入相明年十一月壬申又云三思罷日及官皆同盖誤重複耳今從實録}
天官侍郎同平章事吉頊貶安固尉 【考異曰實録但云坐事貶流僉載新書皆云貶琰川尉今從御史臺記}
太后以頊有幹略故委以腹心頊與武懿宗爭趙州之功於太后前頊魁岸辯口懿宗短小傴僂【傴於庾翻僂力主翻}
頊視懿宗聲氣陵厲太后由是不悦曰頊在朕前猶卑我諸武况異時詎可倚邪它日頊奏事方援古引今太后怒曰卿所言朕飫聞之【飫於據翻}
無多言太宗有馬名師子驄肥逸無能調馭者朕為宫女侍側言於太宗曰妾能制之然須三物一鐵鞭二鐵撾三匕首鐵鞭撃之不服則以撾撾其首又不服則以首斷其喉太宗壯朕之志今日卿豈足汚朕首邪【撾側瓜翻斷音短汚烏故翻}
頊惶懼流汗拜伏求生乃止諸武怨其附太子共發其弟冒官事由是坐貶辭日得召見【見賢遍翻下再見同}
涕泣言曰臣今遠離闕庭【離力智翻}
永無再見之期願陳一言太后命之坐問之頊曰合水土為泥有爭乎【合音閤}
太后曰無之又曰分半為佛半為天尊有爭乎曰有爭矣頊頓首曰宗室外戚各當其分則天下安【分扶問翻}
今太子已立而外戚猶為王此陛下驅之使它日必爭兩不得安也太后曰朕亦知之然業已如是不可何如【觀太后使二子與諸武立誓則誠知勢有所必至而出此下策耳}
臘月辛巳立故太孫重潤為邵王其弟重茂為北海王太后問鸞臺侍郎陸元方以外事對曰臣備位宰相

有大事不敢以不聞人間細事不足煩聖聽由是忤旨【忤五故翻}
庚寅罷為司禮卿【光宅改太常卿為司禮卿}
元方為人清謹再為宰相太后每有遷除多訪之元方密封以進未嘗漏露臨終悉取奏藳焚之曰吾於人多隂德子孫其未衰乎 以西突厥竭忠事主可汗斛瑟羅為平西軍大總管鎮碎葉 丁酉以狄仁傑為内史 庚子以文昌左丞韋巨源為納言 【考異曰新紀表庚子文昌左相韋巨源為納言十月丁巳罷先時不言巨源為左相舊紀傳皆無之盖左丞誤為左相耳}
乙巳太后幸嵩山春一月丁卯幸汝州之温湯戊寅遷神都作三陽宫於告成之石淙【三陽宫去洛城一百六十里萬歲登封元年改東都陽城縣曰告成以祀神嶽告成也淙藏宗翻又士江翻}
二月乙未同鳳閣鸞臺三品豆盧欽望罷為太子賓客 三月以吐谷渾青海王宣超為烏地也抜勤忠可汗【宣超諾曷鉢之孫也}
夏四月戊申太后幸三陽宫避暑有胡僧邀車駕觀葬舍利太后許之狄仁傑跪於馬前曰佛者夷狄之神不足以屈天下之主彼胡僧詭譎直欲邀致萬乘以惑遠近之人耳山路險狹不容侍衛非萬乘所宜臨也【譎之穴翻乘繩證翻}
太后中道而還曰以成吾直臣之氣 五月己酉朔日有食之 太后使洪州僧胡超合長生藥【合音閤}
三年而成所費巨萬太后服之疾小瘳【瘳丑留翻}
癸丑赦天下改元久視去天冊金輪大聖之號【去羌呂翻}
六月改控鶴為奉宸府以張易之為奉宸令太后每内殿曲宴輒引諸武易之及弟秘書監昌宗飲博嘲謔【嘲陟交翻謔訖郤翻}
太后欲掩其迹乃命易之昌宗與文學之士李嶠等修三敎珠英於内殿【三敎儒釋道}
武三思奏昌宗乃王子晉後身太后命昌宗衣羽衣吹笙乘木鶴於庭中文士皆賦詩以美之【宗衣於既翻}
太后又多選美少年為奉宸内供奉【少詩照翻}
右補闕朱敬則諫曰陛下内寵有易之昌宗足矣近聞右監門衛長史侯祥等【唐諸衛府各有長史從六品上各掌判其府諸曹之事監古銜翻}
明自媒衒【衒熒絹翻}
醜慢不恥求為奉宸内供奉無禮無儀溢於朝聽臣軄在諫諍不敢不奏太后勞之曰【勞力到翻}
非卿直言朕不知此賜綵百段易之昌宗競以豪侈相勝弟昌儀為洛陽令請屬無不從【屬之欲翻}
嘗早朝【朝直遥翻下同}
有選人姓薛以金五十兩并狀邀其馬而賂之【選須絹翻}
昌儀受金至朝堂以狀授天官侍郎張錫數日錫失其狀以問昌儀昌儀罵曰不了事人我亦不記但姓薛者即與之錫懼退索在銓姓薛者六十餘人悉留注官【索山客翻}
錫文瓘之凡子也【張文瓘見二百一卷高宗乾封二年}
初契丹將李楷固善用䌈索及騎射舞槊每陷陳如鶻入烏羣所向披靡【將即亮翻騎奇寄翻槊色角翻陳讀曰陣披普彼翻}
黄麞之戰張玄遇麻仁節皆為所䌈【事見上卷萬歲通天元年}
又有駱務整者亦為契丹將屢敗唐兵【敗補邁翻}
及孫萬榮死二人皆來降【降戶江翻}
有司責其後至奏請族之狄仁傑曰楷固等並驍勇絶倫【驍堅堯翻}
能盡力於所事必能盡力於我若撫之以德皆為我用矣奏請赦之所親皆止之仁傑曰苟利於國豈為身謀太后用其言赦之又請與之官太后以楷固為左鈐衛將軍務整為右武威衛將軍使將兵擊契丹餘黨悉平之

資治通鑑卷二百六
















































































































































