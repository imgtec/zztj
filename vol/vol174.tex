\section{資治通鑑卷一百七十四}
宋 司馬光 撰

胡三省 音註

陳紀八|{
	上章困敦一年}


高宗宣皇帝下之上

太建十二年春正月癸巳周天元祠太廟|{
	史言周天元既以朝政授其子而猶主祭祀}
戊戌以左衛將軍任忠為南豫州刺史|{
	此時南豫州治宣城任音壬}
督緣江軍防事 乙卯周税入市者人一錢 二月丁巳周天元幸露門學釋奠|{
	周露門學在露門左右塾古者仲春仲秋皆以上丁釋奠于先聖先師鄭玄曰釋奠者設薦饌酌奠而已}
戊午突厥入貢干周且迎千金公主|{
	周許以千金公主妻突厥事始上卷上年厥九勿翻}
乙丑周天元改制為天制敕為天敕|{
	制者大賞罰大除授赦宥慮囚慰勞用之敕者廢置州縣增減官吏除免官爵授六品以上官兵施行百官奏請戒約臣下皆用之皆宣署申覆而後行}
壬午尊天元皇太后為天元上皇太后天皇太后為天元聖皇太后|{
	二太后天元嫡母阿史那氏所生母李氏也}
癸未詔楊后與三后皆稱太皇后|{
	三后朱后元后陳后也}
司馬后直稱皇后|{
	司馬后正陽宫皇后也}
行軍總管公亮天元之從祖兄也|{
	天元從祖宇文泰之兄弟也從才用翻}
其子西陽公温妻尉遲氏蜀公迥之孫|{
	尉紆勿翻}
有美色以宗婦入朝天元飲之酒逼而淫之|{
	朝直遥翻飲干禁翻}
亮聞之懼三月軍還至豫州|{
	自淮南還軍豫州治汝南還音旋又音如字}
密謀襲韋孝寛并其衆|{
	韋孝寛征南行軍元帥帥所類翻}
推諸父為主|{
	諸父謂趙王招兄弟}
鼔行而西亮國官茹寛知其謀|{
	諸國公各有國官茹姓也後魏書普六茹後改為茹氏余按太和之時南齊有茹法亮盖南國自有茹氏今茹寛既仕於北國恐出於普六茹氏茹音如}
先告孝寛孝寛潜設備亮夜將數百騎襲孝寛營|{
	將即亮翻又音如字領也騎奇寄翻}
不克而走戊子孝寛追斬之温亦坐誅天元即召其妻入宫拜長貴妃|{
	長知兩翻}
辛卯立亮弟永昌公椿為公 周天元如同州增候正前驅式道候為三百六十重|{
	候正主候望前驅先驅也式道候在大駕前重直龍翻}
自應門至於赤岸澤|{
	鄭玄曰天子五門臯庫雉應路詩云乃立應門應門將將赤岸澤在長安北同州南道里盖適中}
數十里間幡旗相蔽音樂俱作又令虎賁持鈒馬上稱警蹕|{
	賁音奔鈒色立翻戟也鋋也警戒也戒人以車駕將至也蹕蹕止行人也}
乙未改同州宫為成天宫庚子還長安|{
	還從宣翻又音如字}
詔天臺侍衛之官皆著五色及紅紫緑衣以雜色為緣名曰品色衣|{
	著則略翻緣以絹翻五代志以錦綺繢繡為緣}
有大事與公服間服之|{
	五代志後周之制諸命秩之服曰公服其餘常服曰私衣隋唐以下有朝服有公服朝服曰具服公服曰從省服間古莧翻}
壬寅詔内外命婦皆執笏|{
	後周之制内外命婦各有命服未嘗執笏也}
其拜宗廟及天臺皆俛伏如男子|{
	俛音免}
天元將立五皇后以問小宗伯狄道辛彦之|{
	五代志狄道縣屬金城郡}
對曰皇后與天子敵體不宜有五太學博士西城何妥曰昔帝嚳四妃虞舜二妃先代之數何常之有|{
	博士秦官漢置五經博士即太學博士也晉武帝立國子學置博士一人遂有國子博士太學博士之分西城郡時置金州帝王紀云帝嚳四妃元妃有邰氏女曰姜嫄次妃有娀氏女曰簡狄次妃陳豐氏女曰慶都次妃娵訾氏曰常義列女傳云舜二妃堯之二女長曰娥皇次曰女英史言何妥曲學以阿世不可以訓嚳苦沃翻嫄音原娵遵須翻訾子斯翻妥它果翻}
帝大悦免彦之官甲辰詔曰坤儀比德土數惟五四太皇后外可增置天中太皇后一人於是以陳氏為天中太皇后尉遲妃為天左太皇后|{
	陳氏山提之女尉遲氏字文温之妻尉紆勿翻}
又造下帳五使五皇后各居其一實宗廟祭器於前自讀祝版而祭之|{
	下帳山陵中便房所用此所謂下帳盖周天元以自所居者為上帳五皇后所居者為下帳也祝版所以祝鬼神}
又以五輅載婦人自帥左右步從|{
	按古制有五輅後周之制皇帝之輅十有二等皇后之車十有二等亦曰輅下至三妃三公三公夫人之輅皆九至上媛婦中大夫孺人其輅五天元雖淫侈無道何至以古之五輅載婦人其實用媛婦以下所乘五輅耳五輅謂玄輅夏篆夏縵墨車輚車也帥讀曰率從才用翻}
又好倒懸雞及碎瓦於車上觀其號呼以為樂|{
	好呼到翻號戶高翻樂音洛}
夏四月癸亥尚書左僕射陸繕卒|{
	卒子恤翻}
己巳周天元祠太廟己卯大雩壬午幸仲山祈雨|{
	顔師古曰仲山即今九嵏山之東仲山是也括地志仲山在雍州雲陽縣西十五里}
甲申還宫令京城士女於衢巷作樂迎候|{
	還從宣翻又音如字令力丁翻}
五月癸巳以尚書右僕射晉安王伯恭為僕射 周楊后性柔婉不妬忌四皇后及嬪御等咸愛而仰之|{
	嬪内官九嬪也嬪婦也能行婦道者也御侍也進也進御於君者也嬪毗賓翻}
天元昏暴滋甚喜怒乖度嘗譴后欲加之罪后進止詳閑辭色不撓|{
	譴衍戰翻詳審也諦也閑暇也習也撓曲也又動亂也撓女巧翻又女教翻}
天元大怒遂賜后死逼令引訣|{
	漢書多作引决謂引分自裁也訣别也令力丁翻}
后母獨孤氏|{
	獨孤氏信之女也}
詣閤陳謝叩頭流血然後得免后父大前疑堅位望隆重天元忌之嘗因忿謂后曰必族滅爾家因召堅謂左右曰色動即殺之堅至神色自若乃止内史上大夫鄭譯與堅少同學|{
	少詩照翻}
奇堅相表|{
	相息亮翻}
傾心相結堅既為帝所忌情不自安嘗在永巷|{
	永巷宫中長巷}
私於譯曰|{
	身事不敢昌言之故曰私}
久願出藩公所悉也|{
	出藩謂出補外藩悉諳究也}
願少留意|{
	少詩沼翻}
譯曰以公德望天下歸心欲求多福豈敢忘也|{
	鄭譯被寵於天元為如何天元無恙而與楊堅於宫中私言至及于此小人傾覆何可託邪}
謹即言之天元將遣譯入寇譯請元帥|{
	帥所類翻}
天元曰卿意如何對曰若定江東自非懿戚重臣|{
	懿專久而美也大也}
無以鎮撫可令隨公行|{
	以楊爵稱之}
且為壽陽摠管以督軍事天元從之己丑以堅為揚州摠管使譯兵會壽陽|{
	壽陽屬南則為豫州屬北則為揚州}
將行會堅暴有足疾不果行|{
	先無此疾而忽有此疾曰暴暴猝暴也}
甲午夜天元備法駕幸天興宫乙未不豫而還|{
	還音旋又如字}
小御正博陵劉昉素以狡諂得幸於天元|{
	杜佑曰周御正屬天官御正中大夫五命小御正下大夫四命昉分罔翻狡古巧翻猾也}
與御正中大夫顔之儀並見親信天元召昉之儀入卧内欲屬以後事天元瘖不復能言|{
	寢室謂之臥内屬之欲翻瘖於今翻復如字又扶又翻}
昉見静帝幼冲|{
	冲亦幼也周成王率稱冲人冲子}
以楊堅后父有重名遂與領内史鄭譯|{
	鄭譯以内史上大夫領内史}
御飾大夫柳裘|{
	周置御飾大夫掌御飾其御服又置司服掌之}
内史大夫杜陵韋謩|{
	杜陵漢晉皆屬京兆後隋併入京兆大興縣其地在隋唐長安城南謩與謨同}
御正下士朝那皇甫績|{
	朝那縣屬安定郡後周下士二命}
謀引堅輔政堅固辭不敢當昉曰公若為速為之不為昉自為也堅乃從之稱受詔居中侍疾裘惔之孫也|{
	柳惔柳元景之從孫世隆之子世仕江南江陵陷柳氏入關中遂臣於周惔徒甘翻}
是日帝殂|{
	年二十二殂祚乎翻}
祕不喪昉譯矯詔以堅總知中外兵馬事 |{
	考異曰周帝紀乙未帝不豫還宫詔堅入侍疾丁未追五王入朝己酉大漸昉譯矯制以堅受遺輔政是日帝崩按堅以變起倉猝故得矯命當國若自乙未至己酉凡十五日事安得不泄今從隋帝紀}
顔之儀知非帝指拒而不從昉等草詔署訖逼之儀連署之儀厲聲曰主上升遐|{
	記曲禮告喪曰天王登假鄭玄曰登上也假己也上己者言若仙去云耳登猶升也假與遐同音霞}
嗣子冲幼|{
	静帝時年八歲}
阿衡之任|{
	商相伊尹輔太甲稱阿衡孔安國傳曰阿倚衡平}
宜在宗英|{
	才過人曰英宗英宗室之中其才過人者}
方今趙王最長以親以德合膺重寄|{
	趙王謂趙王招於静帝諸大父行中其年最長長知兩翻膺當也}
公等備受朝恩|{
	朝直遥翻}
當思盡忠報國奈何一旦欲以神器假人|{
	老子天下神器為者敗之執者失之註云大寶之位是天地神明之器故不可以力為也又曰國之利器不可以授人}
之儀有死而已不能誣罔先帝昉等知不可屈乃代之儀署而行之諸衛既受勑並受堅節度|{
	周自左右宫伯至左右羽林游擊皆諸衛官也}
堅恐諸王在外生變以千金公主將適突厥為辭徵趙陳越代滕五王入朝|{
	五王就國見上卷上年厥九勿翻朝直遥翻}
堅索符璽|{
	符謂兵符璽謂天子六璽索山客翻璽斯氏翻}
顔之儀正色曰此天子之物自有主者宰相何故索之|{
	索山客翻}
堅大怒命引出將殺之以其民望出為西邉郡守|{
	西邊恐當作西疆五代志臨洮郡合川縣後周置仍立西疆郡 考異曰北史鄭譯傳之儀與宦者謀引大將軍宇文仲輔政仲已至御坐譯知之遽率開府楊惠及劉昉皇甫績柳裘俱入仲與之儀見譯等愕然逡廵欲出隋文因執之於是矯詔復以譯為内史上大夫明日隋文為丞相拜譯柱國府長史按之儀若爾豈復得全今從之儀傳}
丁未喪靜帝入居天臺罷正陽宫|{
	置正陽宫見上卷上年}
大赦停洛陽宫作|{
	治洛陽宫見上卷上年二月}
庚戌尊阿史那太后為太皇太后李太后為太帝太后|{
	靜帝祖母也}
楊后為皇太后朱后為帝太后|{
	靜帝嫡母生母也}
其陳后元后尉遲后並為尼|{
	皆不以德選以色進者也尼女夷翻}
以漢王贊為上柱國右大丞相|{
	贊静帝叔父也周人上右相息亮翻}
尊以虚名實無所綜理以楊堅為假黄钺左大丞相秦王贄為上柱國百官摠己以聽於左丞相|{
	孔子曰君薨百官摠己以聽於冢宰三年朱熹曰各摠攝已職以聽也余謂若此者必有伊周之臣而後可}
堅初受顧命|{
	顧命始於周成王孔安國曰臨終之命曰顧命余謂顧命者言天子登遐若囘顧而有所言也陸德明曰顧工戶翻}
使䢴國公楊惠|{
	邗音寒又古寒翻}
謂御正下大夫李德林曰朝廷賜令摠文武事經國任重今欲與公共事必不得辭德林曰願以死奉公堅大喜始劉昉鄭譯議以堅為大冢宰譯自攝大司馬昉又求小冢宰|{
	後周置小冢宰上大夫也六命按爾雅冢大也鄭玄曰冢大之上也冢宰之上不宜加小字故周官止曰小宰昉分罔翻冢知隴翻}
堅私問德林曰欲何以見處|{
	處昌呂翻}
德林曰宜作大丞相假黄鉞都督中外諸軍事不爾無以壓衆心|{
	如昉譯之言大冢宰雖六官之長然猶與諸公等夷德林所言則宇文泰所以輔魏者也不爾猶言不如此也相息亮翻壓於甲翻}
及喪即依此行之以正陽宫為丞相府時衆情未壹|{
	言周之朝臣未盡歸心於堅}
堅引司武上士盧賁置左右|{
	賁扶分翻}
將之東宫|{
	正陽宫本東宫也}
百官皆不知所從堅潜令賁部伍仗衛|{
	仗衛執仗而宿衛之兵也盧賁以司武上士統之楊堅潜令賁此舉為如何}
因召公卿謂曰欲求富貴者宜相隨|{
	觀堅此言則其夙心可知矣}
往往偶語欲有去就賁嚴兵而至衆莫敢動出崇陽門|{
	崇陽門周宫城之東門}
至東宫門者拒不納賁諭之不去瞋目叱之|{
	瞋昌真翻}
門者遂却堅入賁遂典丞相府宿衛|{
	盧賁遂為楊堅私人矣}
賁辯之弟子也|{
	盧辯與蘇綽共定後周官制者也}
以鄭譯為丞相府長史|{
	長知兩翻}
劉昉為司馬李德林為府屬|{
	丞相府有掾有屬}
二人由是怨德林内史下大夫勃海高熲|{
	按隋書高熲自六勃海蓨人熲古迥翻}
明敏有器局習兵事多計畧堅欲引之入府|{
	引之入丞相府為官屬}
遣楊惠諭意|{
	楊惠堅族子也堅初秉周政欲引時才故率使之諭意堅既受禪封觀王改名雍}
熲承旨欣然曰願受驅馳縱令公事不成熲亦不辭滅族乃以為相府司録|{
	司録摠録一府之事令力丁翻相息亮翻}
時漢王贊居禁中每與静帝同帳而坐劉昉飾美妓進贊|{
	妓渠綺翻女樂也}
贊甚悦之昉因說贊曰大王先帝之弟時望所歸孺子幼冲豈堪大事|{
	說輸芮翻孺子謂靜帝}
今先帝初崩人情尚擾王且歸第待事寧後入為天子此萬全計也贊年少|{
	少詩照翻}
性識庸下以為信然遂從之堅革宣帝苛酷之政更為寛大删略舊律作刑書要制奏而行之躬履節儉中外悦之|{
	賈誼曰寒者利裋褐飢者甘糟糠天下之嗷嗷新主之資也古之得天下必先有以得天下之心雖奸雄挾數用術不能外此也更工衡翻}
堅夜召太史中大夫庾季才|{
	太史掌天文歷數周制太史中大夫屬春官五命}
問曰吾以庸虚|{
	庸言身無所能虚言胷中無所有謙辭也}
受茲顧命天時人事卿以為何如季才曰天道精微難可意察竊以人事卜之符兆已定|{
	符籤也證也驗也兆龜坼文也又人事之兆朕也}
季才縱言不可公豈復得為箕潁之事乎|{
	司馬貞曰堯讓天下於許由由遂逃於箕山洗耳于潁水復扶又翻又音如字}
堅默然久之曰誠如君言獨孤夫人亦謂堅曰大事已然騎虎之勢必不得下勉之|{
	獨孤夫人堅妃也騎虎而下必為所噬}
堅以相州摠管尉遲迥位望素重恐有異圖|{
	相息亮翻尉紆勿翻}
使迥子魏安公惇奉詔書召之會葬|{
	魏安郡公五代志武威郡昌松縣有後魏魏安郡注詳見後}
壬子以上柱國韋孝寛為相州摠管又以小司徒叱列長乂為相州刺史|{
	叱列虜複姓出於拓拔氏西部後為周之戚里}
先令赴鄴孝寛續進|{
	鄴相州摠管治所}
陳王純時鎮齊州|{
	純就國於濟南濟南郡齊州也}
堅使門正上士崔彭徵之|{
	門正掌門關啓閉之節及出入門者}
彭以兩騎往止傳舍|{
	騎奇寄翻傳張戀翻}
遣人召純純至彭請屏左右密有所道|{
	屏必郢翻道言也}
遂執而鎻之因大言曰陳王有罪詔徵入朝左右不得輒動其從者愕然而去|{
	朝直遥翻從才用翻}
彭楷之孫也|{
	崔楷死職見一百五十一卷梁武帝大通元年}
六月五王皆至長安 庚申周復行佛道二教|{
	周禁二教見一百七十一卷六年復扶又翻又音如字}
舊沙門道士精志者簡令入道|{
	簡分别也}
周尉遲迥知丞相堅將不利於帝室謀舉兵討之|{
	尉紆}


|{
	勿翻相息亮翻}
韋孝寛至朝歌|{
	五代志汲郡衛縣舊曰朝歌}
迥遣其大都督賀蘭貴|{
	周書賀蘭其先與魏俱起有紇伏者為賀蘭莫何弗因以為氏}
齎書候韋孝寛|{
	齎相稽翻}
孝寛留貴與語以審之疑其有變遂稱疾徐行又使人至相州求醫藥密以伺之|{
	伺相吏翻}
孝寛兄子藝為魏郡守|{
	守式又翻}
迥遣藝迎孝寛孝寛問迥所為藝黨於迥不以實對|{
	魏郡守與相州搃管府同治鄴職事有聨且迥忠于帝室宜其黨于所事守手又翻}
孝寛怒將斬之藝懼悉以迥謀語孝寛|{
	語牛倨翻}
孝寛擕藝西走每至亭驛|{
	亭郵亭也即置驛之所}
盡驅其傳馬而去|{
	傳馬即驛馬傳張戀翻}
謂驛司曰|{
	驛司掌驛之吏}
蜀公將至宜速具酒食|{
	尉遲迥封蜀公故稱之}
迥尋遣儀同大將軍梁子康將數百騎追孝寛追者至驛輒逢盛饌|{
	康將即亮翻又音如字領也騎奇寄翻饌雛皖翻又雛戀翻食也}
又無馬遂遲留不進孝寛與藝由是得免|{
	史言韋孝寛機數過人}
堅又令候正破六韓裒詣迥諭旨|{
	破六韓虜三字姓諭譬也告也暁也旨意向也裒薄侯翻}
密與摠管府長史晉昶等書|{
	姓苑晉本唐叔虞之後以國為氏長知兩翻昶音敞}
令為之備迥聞之殺昶及裒集文武士民|{
	文武謂摠管府及州郡文武官屬也令力丁翻}
登城北樓令之曰楊堅藉后父之勢挾幼主以作威福不臣之迹暴於行路|{
	令力定翻藉慈夜翻暴步卜翻顯示也又如字顯露也}
吾與國舅甥|{
	尉遲迥宇文泰之甥}
任兼將相先帝處吾於此|{
	將即亮翻相息亮翻處昌呂翻}
本欲寄以安危今欲與卿等糾合義勇|{
	糾渠黝翻䋲三合為糾言糾合者義取諸此}
以匡國庇民何如衆咸從命迥乃自稱大摠管承制置官司|{
	稱大摠管者欲以統攝諸州摠管署置官司而隔於權臣未得以聞於天子故曰承制}
時趙王招入朝留少子在國|{
	趙王招國於襄國襄國屬相州摠管府朝直遥翻少詩照翻}
迥奉以號令甲子堅關中兵以韋孝寛為行軍元帥|{
	帥所類翻}
郕公梁士彦樂安公元諧化政公宇文忻濮陽公武川宇文述武鄉公崔弘度清河公楊素隴西公李詢等皆為行軍摠管以討迥|{
	梁士彦國公郕古國名自元諧以下皆郡公五代志北海郡千乘縣舊置樂安郡地形志夏州有化政郡參考五代志當在夏州巖録縣界東平郡郵城縣舊置濮陽郡述傳曰述代郡武川人志馬邑郡善陽縣有代郡上黨郡鄉縣石勒置武鄉郡五代志周改馮翊華隂縣為武鄉郡清河郡武城縣舊置清河郡隴西古郡也後魏領襄武首陽縣樂音洛濮博木翻}
弘度楷之孫詢穆之兄子也|{
	李穆時為并州刺史}
初宣帝使計部中大夫楊尚希撫慰山東|{
	後周置計部盖主計會之簿書若周官之司書杜佑曰計部屬天官}
至相州聞宣帝殂|{
	殂祚乎翻}
與尉遲迥喪|{
	尉紆勿翻}
尚希出謂左右曰蜀公哭不哀而視不安將有他計吾不去懼及於難|{
	難乃旦翻}
遂夜從捷徑而遁|{
	捷徑者不由正路捷出取徑直而行}
遲明|{
	遲直利翻待也}
迥覺追之不及遂歸長安堅遣尚希督宗兵三千人鎮潼關|{
	楊尚希弘農人弘農華隂諸楊自東漢至後魏為名族魏分東西弘農又為兵衝故楊氏有宗兵}
雍州牧畢刺王賢|{
	雍於用翻刺盧逹翻諡法愎狠遂過曰刺又不思忘愛曰刺暴慢無親曰刺楊堅加賢以惡諡耳}
與五王謀殺堅事洩|{
	洩與泄同}
堅殺賢并其三子掩五王之謀不問|{
	堅豈真不問哉山東有變内復相圖姑以安反側耳}
以秦王贄為大冢宰杞公椿為大司徒庚子以柱國梁睿為益州摠管睿禦之子也|{
	梁禦見一百五十六卷梁武帝中大通六年}
周遣汝南公神慶司衛上士長孫晟|{
	汝南古郡名晟丞正翻}
送千金公主於突厥|{
	厥九勿翻}
晟幼之曾孫也|{
	按隋長孫晟傳及唐宰相世系表晟長孫稚之五世孫稚字幼卿生子裕子裕生紹遠紹遠生覧覧生敞敞生熾熾生晟非曾孫也若書稚字幼下亦闕卿字}
又遣建威侯賀若誼|{
	建威縣侯五代志建威縣屬武都郡若人者翻}
賂佗鉢可汙|{
	可從刋入聲汗音寒}
且說之以求高紹義|{
	說輸芮翻}
佗鉢偽與紹義獵於南境使誼執之誼敦之弟也|{
	賀若敦弼之父也}
秋七月甲申紹義至長安徙之蜀久之病死於蜀 周青州摠管尉遲勤迥之弟子也|{
	尉紆勿翻}
初得迥書表送之尋亦從迥迥所統相衛黎洺貝趙冀瀛滄|{
	五代志汲郡東魏置義州後周為衛州黎陽縣後魏黎陽郡後置黎州武安郡後周置洺州清河郡後周置貝州趙郡大陸縣舊曰廣阿置殷州後改趙州信都郡舊置冀州河間郡河間縣舊置瀛州勃海郡饒安縣舊置滄州考異曰周書迥傳又有毛州按迥滅後隋高祖始置毛州迥傳誤也}
勤所統青齊膠光

莒等州|{
	五代志北海郡置青州齊郡舊曰齊州高密郡舊置膠州東莱郡舊置光州琅邪郡沂水縣舊置南青州後周改為莒州}
皆從之衆數十萬榮州刺史邵公胄|{
	五代志榮陽郡汜水縣古虎牢也後魏置東中府東魏置北豫州後周置榮州邵公胄周宗室也封邵郡公}
申州刺史李惠|{
	五代志義陽郡江左置司州後魏改曰郢州後周改曰申州}
東楚州刺史費也利進|{
	五代志琅邪郡後魏置南徐州梁改為東徐州東魏改東楚州陳改安州後周改泗州豈史家以舊州名書之邪費也虜複姓盖即費也頭種費扶沸翻}
潼州刺史曹孝遠|{
	按五代志後周亦無潼州但云下邳郡夏丘縣梁置潼州後齊改曰睢州尋廢夏丘宿豫相去不遠宿豫舊東楚治所意此時尚有此二州而志逸之也}
各據本州徐州摠管司録席毗羅據兖州|{
	五代志魯郡瑕丘縣舊置兖州姓苑席姓其先姓籍避項羽諱改姓席氏}
前東平郡守畢義緒據蘭陵|{
	五代志蘭陵縣舊曰承置蘭陵郡守式又翻}
皆應迥懷縣永橋鎮將紇豆陵惠以城降迥|{
	五代志懷縣屬河内郡隋大業初廢入安昌縣安昌本州縣紇豆陵虜三字姓魏收官氏志次南諸姓有紇豆陵氏將即亮翻}
迥使其所署大將軍石遜攻建州建州刺史宇文弁以州降之|{
	五代志長平郡舊曰建州降戶江翻}
又遣西道行臺韓長業攻拔潞州|{
	五代志上黨郡後周置潞州}
執刺史趙威署城人郭子勝為刺史紇豆陵惠襲陷鉅鹿|{
	鉅鹿古郡也隋為縣屬襄國郡}
遂圍恒州|{
	五代志恒山郡後周置恒州恒戶登翻}
上大將軍宇文威攻汴州|{
	五代志榮陽郡浚儀縣東魏置梁州後周改曰汴州汴皮變翻}
莒州刺史烏丸尼等帥青齊之衆圍沂州|{
	五代志琅邪郡舊置北徐州後周改曰沂州帥讀曰率}
大將軍檀讓攻拔曹亳二州|{
	五代志濟隂郡後魏置西兖州後周改曰曹州譙郡後魏置南兖州後周改毫州亳旁各翻}
屯兵梁郡|{
	梁郡治睢陽}
席毗羅衆號八萬軍於蕃城攻陷昌慮下邑|{
	五代志彭城郡滕縣舊曰蕃置蕃郡隋改曰滕昌慮漢古縣後魏屬蘭陵郡下邑亦漢古縣五代志屬梁郡漢書蕃音皮慮音廬}
李惠自申州攻永州拔之|{
	五代志汝南郡城陽縣梁置楚州東魏置西楚州後齊曰永州城陽前漢侯國其地在義陽東北}
迥遣使招大左輔并州刺史李穆穆鎻其使封上其書|{
	使疏吏翻上時掌翻}
穆子士榮以穆所居天下精兵處|{
	并州用武之地士健馬多故曰天下精兵處}
隂勸穆從迥穆深拒之堅使内史大夫柳裘詣穆為陳利害|{
	為于偽翻}
又使穆子左侍上士渾往布腹心|{
	布腹心者陳其至誠非貌言也}
穆使渾奉尉斗於堅曰願執威柄以尉安天下|{
	尉斗今之熨斗毛晃曰火斗熨器篆文作叞从从火从又又偏旁手字持火所以申繒也今文作尉俗加火作熨尉紆胄翻又紆勿翻}
又以十三鐶金帶遺堅十三鐶金帶者天子之服也|{
	五代志曰革帶案禮博二寸禮圖曰璫綴於革帶阮諶以為有章印則於革帶佩之東觀記曰楊賜拜太常詔賜自所著革帶故知形制尊卑不别今博三寸半加金鏤䚢螳螂鈎以相鈎帶自大裘至於小朝服皆用之天子以十三鐶金帶為異後周制也鐶戶關翻遺于季翻}
堅大悦遣渾詣韋孝寛述穆意|{
	使述穆意以堅孝寛附己之心}
穆兄子崇為懷州刺史|{
	五代志河内郡舊制懷州}
初欲應迥後知穆附堅慨然太息曰闔家富貴者數十人值國有難竟不能扶傾繼絶復何面目處天地間乎|{
	難乃旦翻復扶又翻處昌呂翻}
不得已亦附於堅迥子誼為朔州刺史|{
	按五代志後齊朔州治桑乾隋併入馬邑郡善陽縣置摠管府}
穆執送長安又遣兵討郭子勝擒之迥招徐州摠管源雄東郡守于仲文皆不從雄賀之曾孫仲文謹之孫也|{
	東郡治白馬源賀本出禿髪氏歸魏改姓源于謹事宇文有大功守手又翻}
迥遣宇文胄自石濟宇文威自白馬濟河|{
	石濟在白馬西}
二道攻仲文仲文棄郡走還長安迥殺其妻子迥遣檀讓徇地河南丞相堅以仲文為河南道行軍摠管使詣洛陽發兵討讓命楊素討宇文胄丁未周以丞相堅都督中外諸軍事鄖州摠管司馬消難亦舉兵應迥|{
	按五代志漢東郡唐城縣本梁之下溠城後魏之㵐西縣也後魏立肆州尋改唐州後周省均欵溳歸四州入如此則鄖州已併省今有鄖州摠管而志逸置摠管府之地此考史之所以難也春秋鄖子之國杜預謂在江夏雲杜縣東南有鄖城章懷太子賢曰雲杜故城在復州沔陽縣西北周盖因古國名置鄖州于沔陽也鄖音云難乃旦翻}
己酉周以柱國王誼為行軍元帥以討消難|{
	帥所類翻}
廣州刺史于顗仲文之兄也與摠管趙文表不協詐得心疾誘文表手殺之|{
	誘音酉}
因唱言文表與尉遲迥通謀堅以迥未平因勞勉之即拜吳州摠管|{
	按隋書于顗傳顗時為東廣州刺史五代志江都郡梁置南兖州後齊改為東廣州陳復曰南兖後周改曰吳州東廣州盖因廣陵以名州觀此則此時東廣州刺史與吳州摠管並治廣陵也廣上逸東字顗魚豈翻勞力到翻}
趙僭王招謀殺堅|{
	楊堅亦以趙王招謀殺已而加惡謚}
邀堅過其第|{
	過古禾翻}
堅齎酒殽就之|{
	齎則兮翻}
招引入寢室招子員貫及妃弟魯封等皆在左右佩刀而立又藏刃於帷席之間伏壯士於室後堅左右皆不得從唯從祖弟開府太將軍弘大將軍元胄坐于戶側胄順之孫也|{
	元順以伉直得名於孝昌之間從才用翻}
弘胄皆有勇力為堅腹心酒酣招以佩刀刺瓜連啗堅欲因而刺之|{
	酣戶甘翻刺七賜翻又七迹翻啗徒敢翻又徒濫翻}
元胄進曰相府有事不可久留招訶之曰我與丞相言汝何為者叱之使却胄瞋目憤氣扣刀入衛|{
	訶虎何翻瞋昌真翻}
招賜之酒曰吾豈有不善之意邪卿何猜警如是|{
	猜疑也警戒也猜警言疑而加戒慎也邪音耶}
招偽吐將入後閤胄恐其為變扶令上坐如此再三招偽稱㗋乾|{
	吐土故翻嘔也上時掌翻坐徂卧翻下同乾音干}
命胄就厨取飲胄不動會滕王逌後至|{
	逌音由}
堅降階迎之胄耳語曰|{
	附耳而語}
事勢大異可速去堅曰彼無兵馬何能為胄曰兵馬皆彼物彼若先發大事去矣胄不辭死恐死無益堅復入坐|{
	復扶又翻又音如字}
胄聞室後有被甲聲遽請曰相府事殷|{
	被皮義翻相息亮翻殷衆也}
公何得如此因扶堅下牀趨去招將追之胄以身蔽戶招不得出堅及門胄自後至招恨不時彈指出血壬子堅誣招與越野王盛謀反皆殺之|{
	野亦惡謚也}
及其諸子賞賜元胄不可勝計|{
	勝音升}
周室諸王數欲伺隙殺堅|{
	數所角翻伺相吏翻}
堅都督臨涇李圓通常保護之|{
	按隋書李圓通傳作京兆涇陽人涇陽縣固屬京兆若以為臨涇則屬安定圓通少給使堅家}
由是得免 癸丑周主封其弟衍為葉王|{
	大建五年六月周皇孫衍生武帝建德二年也大建九年周封皇子衍為道王武帝建德之六年也今静帝又封其弟衍為葉王李延夀又謂静帝本名衍改名闡互有背馳當考葉式涉翻}
術為郢王 周豫荆襄三州蠻反|{
	豫州汝南郡荆州南郡襄州襄陽郡此蠻即所謂山蠻自荆襄至于汝漢皆有之}
攻破郡縣 周韋孝寛軍至永橋城諸將請先攻之孝寛曰城小而固若攻而不拔損我兵威今破其大軍此何能為於是引軍壁於武陟|{
	武陟地名按五代志在河内郡修武縣界至隋析置武陟縣將即亮翻}
尉遲迥遣其子魏安公惇帥衆十萬入武德軍於沁東|{
	魏安縣公五代志宕渠郡墊江縣後周為魏安縣又沔陽郡甑山縣梁置梁安郡西魏改曰魏安郡注又見前河内郡安昌縣舊曰州縣置武德郡尉於勿翻惇都昆翻帥讀曰率沁七鴆翻}
會沁水漲孝寛與迥隔水相持不進|{
	此與迥兵相持耳}
孝寛長史李詢密啓丞相堅云梁士彦宇文忻崔弘度並受尉遲迥饟金|{
	長知兩翻相息亮翻饟即亮翻饋也}
軍中慅慅|{
	慅采早翻慅慅憂愁不安也}
人情大異堅深以為憂與内史上大夫鄭譯謀代此三人者|{
	後周之制上大夫六命}
李德林曰公與諸將皆國家貴臣未相服從今正以挾令之威控御之耳|{
	將即亮翻挾令謂挾天子以令諸將也}
前所遣者疑其乖異後所遣者又安知能盡其腹心邪|{
	邪音耶}
又取金之事虚實難明今一旦代之或懼罪逃逸若加縻縶則自鄖公以下莫不驚疑|{
	縻靡為翻縶陟立翻皆謂繫縛也韋孝寛封鄖國公}
且臨敵易將此燕趙之所以敗也|{
	燕惠王信讒用騎刼代樂毅而敗于田單趙惠文王聽間用趙括代亷頗以敗于白起臨敵易將之禍也將即亮翻燕因肩翻將即亮翻下同}
如愚所見但遣公一腹心明于智略素為諸將所信服者速至軍所使觀其情偽縱有異意必不敢動動亦能制之矣堅大悟曰公不發此言幾敗大事|{
	幾居依翻敗補邁翻}
乃命少内史崔仲方往監諸軍為之節度仲方猷之子也|{
	以杜佑通典攷之少内史當作小内史崔猷見一百六十二卷梁武帝太清三年仲方有文武才幹與堅少相欵密故欲用之監工銜翻}
辭以父在山東又命劉昉鄭譯昉辭以未嘗為將|{
	將即亮翻}
譯辭以母老堅不悦府司録高熲請行堅喜遣之熲受命亟|{
	亟紀力翻}
遣人辭母而已自是堅措置軍事皆與李德林謀之時軍書日以百數德林口授數人文意百端不加治點|{
	治脩改也點塗點也不加治點不加塗改也治直之翻}
司馬消難以鄖隨温應土順沔儇岳九州及魯山等八鎮來降|{
	五代志漢東郡西魏置并州後改曰隨州安陸郡京山縣舊曰新陽梁置新州西魏改曰温州應山縣梁置應州漢東郡土山縣梁置土州順義縣梁置順州沔陽郡後周置復州後改沔州安陸郡吉陽縣後周置澴州孝昌縣西魏置岳州魯山在沔陽郡漢陽縣界臨江齊梁以來為重鎮儇當作澴音戶關翻難乃旦翻}
遣其子為質以求援|{
	質音致}
八月己未詔以消難為大都督摠督九州八鎮諸軍事司空賜爵隨公庚申詔鎮西將軍樊毅進督沔漢諸軍事|{
	沔即漢也}
南豫州刺史任忠帥衆趣歷陽超武將軍陳慧紀為前軍都督趣南兖州|{
	超武將軍梁置與宣猛將軍同班任音壬帥讀曰率趣七喻翻}
周益州摠管王謙|{
	周益州摠管府治成都}
亦不附丞相堅起巴蜀之兵以攻始州|{
	此巴蜀謂漢巴郡蜀郡大界五代志普安郡梁置南梁州後改曰安州西魏改曰始州}
梁睿至漢川不得進|{
	堅以梁睿代王謙謙舉兵故睿不得進漢川即漢中隋避諱改曰漢川}
堅即以睿為行軍元帥以討謙|{
	帥所類翻}
戊辰詔以司馬消難為大都督水陸諸軍事庚午通直散騎常侍淳于陵克臨江郡|{
	五代志歷陽郡烏江縣梁置江都郡後齊改為齊江郡陳改為臨江郡散悉亶翻騎奇寄翻}
梁世宗使中書舍人柳莊奉書入周丞相堅執莊手曰孤昔開府從役江陵深蒙梁主殊眷今主幼時艱猥蒙顧託梁主奕葉委誠朝廷|{
	奕累也奕葉累世也朝直遥翻}
當相與共保歲寒|{
	孔子曰歲寒然後知松栢之後彫何晏注曰大寒之歲衆木皆死然後知松栢不彫傷平歲衆木亦有不死者故須歲寒而後别之喻凡人處治世亦自能修整與君子同在濁世然後知君子之不苟容後之言保歲寒者義取諸此}
時諸將競勸梁主舉兵與尉遲迥連謀以為進可以盡節|{
	將即亮翻尉紆勿翻}
周氏退可以席卷山南|{
	漢沔之地在中南太華諸山之南卷讀曰捲}
梁主疑未决會莊至具道堅語且曰昔袁紹劉表王凌諸葛誕皆一時雄傑據要地擁彊兵然功業莫就禍不旋踵者良由魏晋挾天子保京都仗大順以為名故也|{
	袁紹事始六十三卷漢獻帝建安四年終六十四卷十年劉表事見六十五卷十二年十三年王凌事見七十五卷魏邵陵厲公嘉平元年終三年諸葛誕事見七十七卷高貴鄉公甘露二年三年}
今尉遲迥雖曰舊將昏耄已甚|{
	將即亮翻耄莫到翻}
司馬消難王謙常人之下者非有匡合之才|{
	匡合用管仲相齊桓公九合諸侯一匡天下事}
周朝將相多為身計競効節於楊氏|{
	朝直遥翻將即亮翻相息亮翻}
以臣料之迥等終當覆滅隨公必移周祚|{
	祚福也禄也位也}
未若保境息民以觀其變梁主深然之衆議遂止高熲至軍為橋於沁水尉遲惇於上流縱火栰|{
	大曰栰小曰桴縛本為栰寘火積薪於上流放之欲順流而下以焚橋栰房越翻}
熲豫為土狗以禦之|{
	盖積土于水中前鋭後廣前高後庳其狀如坐狗分居上流以礙火栰使不得下逼橋邉也 考異曰隋書作木栰木狗今從北史}
惇布陳二十餘里|{
	陳讀曰陣下同}
麾兵少却|{
	少詩沼翻}
欲待孝寛軍半度而擊之孝寛因其却鳴鼔齊進軍既度熲命焚橋以絶士卒反顧之心惇兵大敗單騎走|{
	騎奇寄翻}
孝寛乘勝進追至鄴庚午迥與惇及惇弟西都公祐|{
	西都縣公五代志西平郡湟水縣舊曰西都}
悉將其卒十三萬陳於城南迥别統萬人皆緑巾錦襖號黄龍兵|{
	將即亮翻襖烏浩翻袍襖}
迥弟勤帥衆五萬|{
	帥讀曰率}
自青州赴迥以三千騎先至迥素習軍旅老猶被甲臨陳|{
	被皮義翻}
其麾下皆關中人為之力戰|{
	關中人不顧父母妻子為迥力戰言其得士心為於偽翻}
孝寛等軍不利而却鄴中士民觀戰者數萬人行軍摠管宇文忻曰事急矣吾當以詭道破之乃先射觀者 |{
	考異曰隋書云高熲與李詢先犯勸者今從北史}
觀者皆走轉相騰藉聲如雷霆|{
	人衆而囂故其聲如雷霆射食亦翻下同藉慈夜翻}
忻乃傳呼曰賊敗矣衆復振因其擾而乘之迥軍大敗走保鄴城孝寛縱兵圍之李詢及思安伯代人賀婁子幹先登|{
	思安縣伯五代志河池郡河池縣後魏置思安縣魏書官氏志神元皇帝時諸部内入者有賀樓氏盖虜複姓}
崔弘度妹先適迥子為妻及鄴城破迥窘廹升樓|{
	窘巨隕翻}
弘度直上龍尾追之|{
	築道陂陀以上城其道下附於地若龍垂尾然故曰龍尾上時掌翻}
迥彎弓將射弘度弘度脱兜鍪謂迥曰頗相識不|{
	鍪莫侯翻不讀曰否}
今日各圖國事不得顧私以親戚之情謹遏亂兵不許侵辱事勢如此早為身計何所待也迥擲弓於地罵左丞相極口而自殺|{
	楊堅時為左右丞相}
弘度顧其弟弘升曰汝可取迥頭弘升斬之軍士在小城中者孝寛盡阬之|{
	以其從迥為之拒戰也}
勤惇祐東走青州|{
	走音奏又音如字}
未至開府儀同大將軍郭衍追獲之丞相堅以勤初有誠欵|{
	以勤初表送迥書也相息亮翻}
特不之罪李惠先自縛歸罪|{
	李惠自申州舉兵應迥既而知迥事不成先自歸}
堅復其官爵迥末年衰耄|{
	記五十始衰謂精力消耗八十九十曰耄注耄惛忘也耄莫報翻復扶又翻又音如字}
及起兵以小御正崔逹挐為長史逹挐暹之子也|{
	崔暹見用於高澄挐奴加翻長知兩翻暹息廉翻}
文士無籌略舉措多失凡六十八日而敗于仲文軍至蓼隄去梁郡七里|{
	九域志蓼隄梁孝王築至睢陽三百里按此則九域志所謂睢陽非漢舊城之地蓼盧鳥翻或音六非}
檀讓擁衆數萬仲文以羸師挑戰而偽北|{
	羸倫為翻挑徒了翻}
讓不設備仲文還擊大破之生獲五千餘人斬首七百級|{
	還從宣翻又音如字}
進攻梁郡|{
	梁郡治睢陽}
迥守將劉子寛弃城走|{
	將即亮翻下同}
仲文進擊曹州獲迥所署刺史李仲康檀讓以餘衆屯成武仲文襲擊破之遂拔成武|{
	五代志濟隂郡成武縣時為永昌郡}
迥將席毗羅衆十萬屯沛縣|{
	五代志沛縣屬彭城郡將即亮翻}
將攻徐州|{
	徐州彭城郡沛縣在州西北一百四十里}
其妻子在金鄉|{
	五代志金鄉縣屬曹州濟隂郡}
仲文遣人詐為毗羅使者|{
	使疏吏翻}
謂金鄉城主徐善淨曰檀讓明日午時至金鄉宣蜀公令賞賜將士金鄉人皆喜仲文簡精兵偽建迥旗幟倍道而進|{
	簡分揀也幟昌志翻}
善淨望見以為檀讓出迎謁仲文執之遂取金鄉諸將多勸屠其城仲文曰此城乃毗羅起兵之所當寛其妻子其兵自歸如即屠之彼望絶矣衆皆稱善於是毗羅恃衆來薄官軍|{
	薄廹也}
仲文設伏擊之毗羅衆大潰爭投洙水死水為之不流|{
	洙水詳見辯誤為千偽翻}
獲檀讓檻送京師|{
	京師謂長安}
斬毗羅傳首|{
	亦傳首於長安}
韋孝寛分兵討關東叛者悉平之堅徙相州於安陽毁鄴城及邑居|{
	劉昫曰楊堅令韋孝寛討尉遲迥平之焚燒鄴城徙其居人南遷四十五里以安陽城為相州理所仍為鄴縣隋又改為安陽縣漢魏郡城在縣西北七里煬帝於鄴故都大慈寺置鄴縣相息亮翻尉紆勿翻}
分相州置毛州魏州|{
	五代志武陽郡後周置魏州館陶縣置毛州顔師古曰漢武帝時河北決于館陶分為屯氏河屯音大門翻而隨室分析州縣誤以為毛氏河乃置毛州失之甚矣}
梁主聞迥敗謂柳莊曰若從衆人之言社稷已不守矣丞相堅之初得政也待黄公劉昉沛公鄭譯甚厚|{
	黄古國名沛本縣名以漢高祖初為沛公故亦為國昉分罔翻}
賞賜不可勝計|{
	勝音升}
委以心膂|{
	膂力舉翻字林膂脊骨也人之一身思慮之所以運者心腰背之所以強者膂故以為喻}
朝野傾屬|{
	朝直遥翻屬之欲翻}
稱為黄沛二人皆恃功驕恣|{
	恃其汲引之功也}
溺於財利不親職務及辭監軍堅始疎之恩禮漸薄|{
	監工銜翻}
高熲自軍所還|{
	言自鄴還還音旋又如字}
寵遇日隆時王謙司馬消難未平|{
	難乃旦翻}
堅憂之忘寢與食而昉逸游縱酒相府事多遺落|{
	遺失也落墜也相息亮翻}
堅乃以高熲代昉為司馬不忍廢譯隂勑官屬不得白事於譯|{
	勑戒也}
譯猶坐聽事|{
	聽事丞相府長史聽事也聽讀曰廳}
無所關預|{
	要會之處為關又聨絡也預參預也又干也}
惶懼頓首求解職堅猶以恩禮慰勉之 癸酉智武將軍魯廣逹克周之郭默城|{
	梁置五德將軍智武其一也郭默城當在今蘄黄二州界}
丙子淳于陵克祐州城|{
	祐州城地闕}
周以漢王贊為太師申公李穆為太傅宋王實為大

前疑秦王贄為大右弼燕公于寔為大左輔|{
	燕因肩翻}
寔仲文之父也 周王誼帥四摠管至鄖州司馬消難擁其衆以魯山甑山二鎮來降|{
	五代志甑山縣後周置屬沔陽郡帥讀曰率鄖音云難乃旦翻甑子孕翻降戶江翻}
初消難遣上開府儀同大將軍段珣將兵圍順州|{
	將即亮翻又如字領也}
順州刺史周法尚不能拒弃城走消難虜其母弟而南樊毅救消難不及周亳州摠管元景山撃之毅掠居民而去景山與南徐州刺史宇文㢸追之|{
	按隋書宇文㢸傳㢸時為南司州刺史與元景山共追樊毅又五代志安陸郡吉陽縣梁置義陽郡西魏改為南司州其地近澴順諸州南徐當作南司㢸古弼字}
與毅戰於漳口|{
	此漳非左傳所謂江漢沮漳之漳今安陸西五十里有漳水沈括筆談曰清濁相揉者為漳章文也别有雲夢之漳與溳合流色理如螮蝀數十里方混}
一日三戰三捷毅退保甑山鎮城邑為消難所據者景山皆復取之鄖州巴蠻多叛|{
	按王誼傳於時北至商洛南拒江淮東西二千餘里巴蠻多叛是則晉宋以來所謂山蠻也南朝諸史所謂荆雍州蠻者也以其先出於巴種故謂之巴蠻}
共推渠帥蘭雒州為主以附消難|{
	渠大也渠帥者大帥也帥所類翻下氐帥同}
王誼遣諸將分討之旬月皆平陳紀蕭摩訶攻廣陵|{
	陳紀即陳慧紀}
周吳州摠管于顗擊破之|{
	顗魚豈翻}
沙州氐帥楊永安聚衆應王謙大將軍樂寧公逹奚儒討之|{
	按隋書逹奚長儒傳沙氐楊永安扇動利興武文沙龍等六州以應謙參考五代志獨不載沙州盖沙氐所居之地就置沙州以授其渠帥也又長儒襲父慶爵樂安公志云北海郡博昌縣舊曰樂安寧當作安儒上逸長字}
楊素破宇文胄於石濟斬之 周以神武公竇毅為大司馬|{
	竇毅即前紇豆陵毅}
齊公于智為大司空九月以小宗伯竟陵公楊惠為大宗伯|{
	竟陵縣公五代志沔陽郡竟陵縣舊曰霄城置竟陵郡後周廢郡改縣曰竟陵}
丁亥周將王延貴帥衆援歷陽任忠擊破之生擒延貴|{
	將即亮翻下同帥讀曰率下同任音壬}
壬辰周廢皇后司馬氏為庶人|{
	以司馬后父消難起兵而南叛也}
庚戍以隨世子勇為洛州摠管東京小冢宰摠統舊齊之地|{
	自關以東河汾以北皆舊齊之地東京小冢宰此洛州所置六府官也}
壬子以左丞相堅為大丞相罷左右丞相之官冬十月甲寅日有食之 周丞相堅殺陳惑王純及其子|{
	純周五王之一也故楊堅加之惡謚}
周梁睿將步騎二十萬討王謙|{
	騎奇寄翻}
謙分命諸將據險拒守睿奮擊屢破之蜀人大駭謙遣其將逹奚惎高阿那肱乙弗䖍等帥衆十萬攻利州|{
	五代志義城郡後魏立益州世號小益州梁曰黎州西魏復曰益州又改曰利州惎渠記翻}
堰江水以灌之|{
	嘉陵江在利州城西}
城中戰士不過二千摠管昌黎豆盧勣晝夜拒守|{
	隋書豆盧勣傳勣昌黎徒河人本姓慕容燕北地王精之後也中山敗歸魏北人謂歸義為豆盧因氏焉}
凡四旬時出奇兵撃惎等破之會梁睿至惎等遁去睿自劒閣入進逼成都謙令逹奚惎乙弗䖍城守親帥精兵五萬背城結陳睿擊之謙戰敗將入城惎䖍以城降|{
	守手又翻帥讀曰率背蒲昧翻陳讀曰陣降戶江翻}
謙將麾下三十騎走新都|{
	新都縣屬蜀郡九域志新都縣在成都北四十五里將即亮翻騎奇寄翻走音奏}
新都令王寶執之戊寅睿斬謙及高阿那肱劒南平|{
	蜀地在劒閣之南故曰劒南}
十一月甲辰周逹奚儒破楊永安沙州平|{
	奚下儒上脫長字}
丁未周鄖襄公韋孝寛卒|{
	謚法辟地有德曰襄又甲胄有勞曰襄卒子恤翻}
孝寛久在邊境|{
	梁武帝中大同元年韋孝寛鎮玉壁宇文與高氏兵爭倚為蕃扞有年數矣}
屢抗彊敵所經略布置人初莫之解|{
	解戶買翻暁也}
見其成事方乃驚服雖在軍中篤意文史敦睦宗族|{
	敦厚也}
所得俸禄|{
	俸扶用翻}
不入私室人以此稱之 十二月庚辰河東康簡王叔獻卒|{
	謚法温柔好樂曰康一德不懈曰簡}
癸亥周詔諸改姓者宜悉復舊|{
	宇文泰以諸將補九十九姓見一百六十五卷梁元帝承聖三年上書十二月庚辰此書癸亥自庚辰至癸亥四十四日庚辰必誤按長歷周陳十二月皆壬子朔恐是丙辰}
甲子周以大丞相堅為相國摠百揆去都督中外大冢宰之號|{
	去羌呂翻}
進爵為王以安陸等二十郡為隨國|{
	按隋書帝紀時以隋州之崇業鄖州之安陸城陽温州之宜人應州之平靖上明順州之淮南土州之永川昌州之廣昌安昌申州之義陽淮安息州之新蔡建安豫州之汝南臨潁廣寧初安蔡州之蔡陽郢州之漢東二十郡為隨國}
贊拜不名備九錫之禮|{
	是時九錫之禮一大輅戎輅各一玄牡二駟二衮冕之服赤舄副焉三軒懸之樂六佾之舞四朱戶以居五納陛以登六虎賁三百人七鈇钺各一八彤弓一彤矢百盧弓十盧矢千九秬鬯一卣圭瓚副焉}
堅受王爵十郡而已辛未殺代奰王逹滕聞王逌及其子|{
	既殺二王亦皆加以惡謚謚法不醉而怒曰奰色取行違曰聞奰平祕翻}
壬申以小冢宰元孝規為大司徒 是歲周境内有州二百一十一郡五百八

資治通鑑卷一百七十四
