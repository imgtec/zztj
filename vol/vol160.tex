






























































資治通鑑卷一百六十  宋 司馬光 撰

胡三省 音註

梁紀十六【彊圉單閼一年}


高祖武皇帝十六

太清元年【是年四月始改元太清}
春正月朔日有食之不盡如鈎壬寅荆州刺史廬陵威王續卒【諡法猛以彊果曰威}
以湘東王

繹為都督荆雍等九州諸軍事荆州刺史【雍於用翻}
續素貪婪【婪盧南翻}
臨終有啟遣中録事參軍謝宣融獻金銀器千餘件【中録事參軍盖使之録問中事在左右親任者也件其輦翻}
上方知其富因問宣融曰王之金盡此乎宣融曰此之謂多安可加也大王之過如日月之食欲令陛下知之故終而不隱【終謂卒也}
上意乃解初湘東王繹為荆州刺史有微過續代之以狀聞【按繹在荆州有宫人李桃兒者以才慧得進及還以李氏行時得營戶禁重續具狀以聞繹對使者泣訴於太子綱太子和之不得繹懼送李氏還荆州}
自此二王不通音問繹聞其死入閤而躍屧為之破【屧蘇協翻屐也又履中薦也史言繹續生無友于之情死則從而忻快為于偽翻}
丙午東魏勃海獻武王歡卒【年五十二}
歡性深密終日儼然人不能測機權之際變化若神制馭軍旅法令嚴肅聽斷明察【斷丁亂翻}
不可欺犯擢人受任【受當作授}
在於得才苟其所堪無問厮養【厮音斯養余亮翻}
有虚聲無實者皆不任用雅尚儉素刀劒鞍勒無金玉之飾少能劇飲自當大任不過三爵知人好士全護勲舊【如尉景司馬子如孫騰諸人是也少詩照翻好呼到翻}
每獲敵國盡節之臣多不之罪【如泉企裴讓之是也}
由是文武樂為之用【樂音洛}
世子澄秘不發喪【用歡遺言也}
唯行臺左丞陳元康知之侯景自念已與高氏有隙内不自安辛亥據河南叛歸於魏潁州刺史司馬世雲以城應之景誘執豫州刺史高元成襄州刺史李密廣州刺史懷朔暴顯等【誘音酉}
遣軍士二百人載仗暮入西兖州欲襲取之刺史邢子才覺之掩捕盡獲之因散檄東方諸州各為之備由是景不能取【侯景之變當時覺之而能發其姦者邢子才一人耳孰謂文士不可以當藩翰哉}
諸將皆以景之叛由崔暹【崔暹糾劾權貴諸將恨之故以景叛為暹罪將即亮翻下同}
澄不得已欲殺暹以謝景陳元康諫曰今雖四海未清綱紀已定若以數將在外苟悦其心枉殺無辜虧廢刑典豈直上負天神何以下安黎庶晁錯前事願公慎之【晁錯事見十六卷漢景帝三年}
澄乃止遣司空韓軌督諸軍討景 辛酉上祀南郊大赦甲子祀明堂三月魏詔自今應宫刑者直沒官勿刑 魏以開府

儀同三司若干惠為司空侯景為太傅河南道行臺上谷公庚辰景又遣其行臺郎中丁和來上表言臣與高澄有隙請舉函谷以東瑕丘以西豫廣郢荆襄兖南兖濟東豫洛陽北荆北揚等十三州内附【洛陽二州名注已見前魏收志武定二年置北荆州領伊陽新城汝北郡五代志河南郡陸渾縣有東魏北荆州淮陽郡項城縣東魏置北揚州及丹陽郡秣陵郡濟子禮翻 考異曰梁書景傳云與豫州刺史高成廣州刺史暴顯潁州刺史司馬世雲荆州刺史郎椿襄州刺史李密兖州刺史邢子才南兖州刺史石長宣濟州刺史許季良東豫州刺史丘元征洛州刺史爾朱渾願揚州刺史樂恂北荆州刺史梅季昌北揚州刺史元神和等隂結私圖剋相影會蕭韶太清紀又有兖州刺史胡延豫州刺史傳士哲揚州刺史可足渾洛無邢子才典略有荆州刺史庫狄暢無高成暴顯許季良爾朱渾願樂恂梅季昌今依梁書而太清紀有兩豫州盖前官也}
惟青徐數州僅須折簡且黄河以南皆臣所職易同反掌【易弋豉翻}
若齊宋一平【齊謂青州宋謂徐州}
徐事燕趙【燕趙謂河北之地}
上召羣臣廷議尚書僕射謝舉等皆曰頃歲與魏通和【大同二年東魏請和自是交聘使命不絶}
邊境無事今納其叛臣竊謂非宜上曰雖然得景則塞北可清機會難得豈宜膠柱【謂不能圓轉如膠柱鼓瑟}
是歲正月乙卯上夢中原牧守皆以其地來降舉朝稱慶【守式又翻降戶江翻朝直遥翻 考異曰典畧云去年十二月夜夢今從梁書}
旦見中書舍人朱异告之且曰吾為人少夢【少詩沼翻}
若有夢必實异曰此乃宇宙混壹之兆也及丁和至稱景定計以正月乙卯上愈神之【帝不能自治其國而妖夢是踐其亡宜矣}
然意猶未决嘗獨言我國家如金甌無一傷缺今忽受景地詎是事宜脱致紛紜悔之何及【獨言者宴閑之時非因與侍臣問答獨言其事盖帝欲受景地念兹在兹而不能自己於言也}
朱异揣知上意對曰聖明御宇南北歸仰正以事無機會未達其心今侯景分魏土之半以來自非天誘其衷【杜預曰衷中也揣初委翻誘音酉}
人贊其謀何以至此若拒而不内恐絶後來之望此誠易見【易弋豉翻}
願陛下無疑上乃定議納景壬午以景為大將軍封河南王都督河南北諸軍事大行臺承制如鄧禹故事平西諮議參軍周弘正善占候前此謂人曰國家數年後當有兵起及聞納景曰亂階在此矣【為侯景亂梁張本}
丁亥上耕藉田【藉在亦翻}
三月庚子上幸同泰寺捨身如大通故事【大通元年帝捨身之始也事見一百五十一卷}
甲辰遣司州刺史羊鴉仁督兖州刺史桓和【梁紀作上州刺}


【史桓和五代志漢東郡上山縣梁曰龍巢置上州及東西二永寜真陽三郡}
仁州刺史湛海珍等【魏收志梁置仁州治赤坎城帶臨淮郡領已吾義城縣已吾之下注云州郡治五代志彭城穀陽縣有已吾義城二縣後齊併以為臨淮縣}
將兵三萬趣懸瓠【將即亮翻趣七喻翻}
運糧食應接侯景 魏大赦 東魏高澄慮諸州有變乃自出巡撫留段韶守晉陽委以軍事以丞相功曹趙彦深為大行臺都官郎中使陳元康豫作丞相歡條教數十紙付韶及彦深在後以次行之臨發握彦深手泣曰以母弟相託幸明此心夏四月壬申澄入朝于鄴【朝直遥翻}
東魏主與之宴澄起舞識者知其不終【昔周景王喪太子及后以喪賓宴晉叔向曰王其不終乎吾聞之所樂必卒焉今王樂憂若卒以憂不可謂終景王之喪伉儷及冢適也既葬而宴賢者非之高澄則喪父也袐喪不發死肉未寒忘雞斯徒跣之哀縱蹮蹮僛僛之樂尚為有人心乎是故榮錡之禍猶輕柏堂之禍為慘蒼蒼之報應固不爽也雞斯讀為笄纚}
丙子羣臣奉贖【自庚子捨身至丙子奉贖凡三十七日萬機之事不可一日曠廢而荒於佛若是帝忘天下矣三十七日之間天下不知為無君天下亦忘君矣}
丁亥上還宫【丁亥當作丁丑}
大赦改元如大通故事 甲午東魏遣兼散騎常侍李系來聘系繪之弟也【李繪見一百五十八卷大同八年按考異曰魏帝紀作李緯今從本傳}
五月丁酉朔東魏大赦 戊戌東魏以襄城王旭為太尉【旭吁玉翻}
高澄遣武衛將軍元柱等將數萬衆晝夜兼行以襲侯景【將即亮翻}
遇景於潁川北柱等大敗景以羊鴉仁等軍猶未至乃退保潁川【侯景不敢乘勝北向者盖以高歡雖死高澄猶能用其衆也}
甲辰東魏以開府儀同三司庫狄干為太師録尚書事孫騰為太傅汾州刺史賀拔仁為太保司徒高隆之録尚書事司空韓軌為司徒青州刺史尉景為大司馬領軍將軍可朱渾道元為司空僕射高洋為尚書令領中書監徐州刺史慕容紹宗為尚書左僕射高陽王斌為右僕射【斌盖因玉儀而進用斌音彬}
戊午尉景卒 韓軌等圍侯景於潁川景懼割東荆北兖州魯陽長社四城賂魏以求救【東魏東荆州治北陽城荆州治魯陽潁州治長社時無北兖州唯北荆州治伊陽與西魏接境豈史家誤以荆為兖邪}
尚書左僕射于謹曰景少習兵姦詐難測【少詩沼翻}
不如厚其爵位以觀其變未可遣兵也荆州刺史王思政以為若不因機進取後悔無及即以荆州步騎萬餘從魯陽關向陽翟【先是王思政盖自恒農遷刺荆州陽翟縣漢屬潁川郡晉屬河南尹魏收志興和元年分置陽翟郡屬潁川}
丞相泰聞之加景大將軍兼尚書令遣太尉李弼儀同三司趙貴將兵一萬赴潁川【按趙貴開府儀同三司此逸開府二字}
景恐上責之遣中兵參軍柳昕奉啟於上以為王旅未接【謂羊鴉仁等軍未至也昕許斤翻}
死亡交急遂求援關中自救目前臣既不安於高氏豈見容於宇文但螫手解腕【蝮蛇螫手壮士解腕螫音釋腕烏貫翻}
事不得已本圖為國【為于偽翻}
願不賜咎臣獲其力不容即弃今以四州之地為餌敵之資已令宇文遣人入守自豫州以東齊海以西悉臣控壓見有之地盡歸聖朝【見賢遍翻朝直遥翻}
懸瓠項城徐州南兖事須迎納願陛下速敇境上各置重兵與臣影響【言若影之随形響之應聲彼此相應不失機會也}
不使差互上報之曰大夫出境尚有所專【春秋之義大夫出彊專之可也上引此義欲以綏懷侯景不知狼子野心之難馴擾也}
况始創奇謀將建大業理須適事而行随方以應卿誠心有本何假詞費【上報此詔已為侯景所窺矣}
魏以開府儀同三司獨孤信為大司馬 六月戊辰以鄱陽王範為征北將軍總督漢北征討諸軍事撃穰城【使範撃魏荆州欲以應接侯景穰如羊翻}
東魏韓軌等圍潁川聞魏李弼趙貴等將至乙巳引兵還鄴【考異曰周書帝紀三月李弼救侯景今從典略}
侯景欲因會執弼與貴奪其

軍貴疑之不往貴欲誘景入營而執之弼止之【李弼之計以為執侯景不能猝兼河南之地徒為東魏去疾故止貴誘音酉}
羊鴉仁遣長史鄧鴻將兵至汝水弼引兵還長安【東魏之師已退而梁之援兵始來弼若不還師則梁魏之兵必浪戰于汝潁之間矣引兵而還則禍集于梁}
王思政入據潁川景陽稱略地引兵出屯懸瓠【景引兵出潁川以城與魏為王思政守潁川沒於東魏張本}
景復乞兵於魏【復扶又翻}
丞相泰使同軌防主韋法保及都督賀蘭願德等將兵助之【五代志河南宜陽縣後周分置熊耳縣同軌郡周齊以宜陽為界以同軌名郡者言將自此出兵以混壹東西使天下車同軌也}
大行臺左丞藍田王悦言於泰曰侯景之於高歡始敦鄉黨之情終定君臣之契【高歡侯景皆懷朔鎮人少相友善中間同事爾朱歡滅爾朱景遂委質於歡}
任居上將位重台司今歡始死景遽外叛盖所圖甚大終不為人下故也且彼能背德於高氏【將即亮翻背蒲妹翻}
豈肯盡節於朝廷今益之以勢援之以兵竊恐貽笑將來也【史言西魏多智士宇文泰能善用謀侯景之姦詐不得逞而其禍移于梁矣}
泰乃召景入朝【朝直遥翻下同}
景隂謀叛魏事計未成厚撫韋法保等冀為已用外示親密無猜閒【閒古莧翻}
每往來諸軍間侍從至少魏軍中名將皆身自造詣【從才用翻少詩沼翻將即亮翻造七到翻}
同軌防長史裴寛謂法保曰侯景狡詐必不肯入關【言其不肯應召而入朝也}
欲託欵於公恐未可信若伏兵斬之此亦一時之功也如其不爾即應深為之防不得信其誑誘自貽後悔【誑居况翻誘音酉}
法保深然之不敢圖景但自為備而已尋辭還所鎮【辭景而還同軌也}
王思政亦覺其詐密召賀蘭願德等還分布諸軍據景七州十二鎮景果辭不入朝遺丞相泰書曰吾恥與高澄鴈行安能比肩大弟【記王制父之齒随行兄之齒鴈行鴈行言如鴈並飛而進也景知泰覺其情且知梁之可侮弄也故以書絶泰而决意附梁遺丁季翻行戶剛翻}
泰乃遣行臺郎中趙士憲悉召前後所遣諸軍援景者景遂决意來降魏將任約以所部千餘人降於景【史言西魏諸將唯任約為侯景所誘降戶江翻任音壬}
泰以所授景使持節太傅大將軍兼尚書令河南大行臺都督河南諸軍事回授王思政思政並讓不授頻使敦諭【使疏吏翻下同}
唯受都督河南諸軍事 高澄將如晉陽以弟洋為京畿大都督留守於鄴使黄門侍郎高德政佐之德政顥之子也【高顥見一百四十七卷天監七年 考異曰北史作德正今從北齊書}
丁丑澄還晉陽始發喪 秋七月魏長樂武烈公若干惠卒【若干惠魏司空樂音洛}
丁酉東魏主為丞相歡舉哀服緦縗【紀間傳緦麻之縗十五升去其半有事其縷無事其布曰緦為于偽翻縗倉回翻}
凶禮依漢霍光故事【凶禮猶言喪禮也}
贈相國齊王備九錫殊禮戊戌以高澄為使持節大丞相都督中外諸軍録尚書事大行臺勃海王澄啟辭爵位壬寅詔太原公洋攝理軍國遣中使敦諭澄 庚申羊鴉仁入懸瓠城甲子詔更以懸瓠為豫州夀春為南豫州改合肥為合州【後漢豫州治譙魏治汝南安成晉治陳國晉氏南度石氏強盛祖約自譙城退屯夀春始僑立豫州於夀春是後庾亮以豫州刺史鎮蕪湖毛寶治邾城趙治牛渚謝尚鎮歷陽又進馬頭柏冲戌姑孰盖不常厥居也宋武帝欲開拓河南綏定豫土割揚州大江以西悉屬豫州豫州基趾因此而立永初二年分淮東為南豫州治歷陽淮西為豫州然猶治夀春也大明以後豫州治懸瓠常珍奇歸北懸瓠入魏豫州復治夀陽齊東昏之時裴叔業又以夀陽附魏遂以歷陽為豫州至帝天監中韋叡克合肥以為豫州復以歷陽為南豫州後復夀陽又徙豫州復舊治今得懸瓠復宋之舊為豫州以夀陽為南豫以合肥為合州南北兵争疆場之間一彼一此易置州郡類如是矣}
以鴉仁為司豫二州刺史鎮懸瓠西陽太守羊思達為殷州刺史鎮項城【改東魏之北楊州為殷州}
八月乙丑下詔大舉伐東魏遣南豫州刺史貞陽侯淵明南兖州刺史南康王會理分督諸將【將即亮翻}
淵明懿之子會理續之子也始上欲以鄱陽王範為元帥朱异取急在外【謂取休假在外舍也帥所類翻异羊至翻}
聞之遽入曰鄱陽雄豪盖世得人死力然所至殘暴非弔民之材且陛下昔登北顧亭以望謂江右有反氣骨肉為戎首【登北顧亭謂幸京口時也江郢揚南徐之地為江左豫南豫南兖之地為江右朱异告帝以防鄱陽而不知防臨賀帝知江右有反氣而不料侯景自夀陽舉兵天邪人邪}
今日之事尤宜詳擇上默然曰會理何如對曰陛下得之矣會理懦而無謀所乘襻輿【襻普患翻襻輿者輿掆施襻人以肩舉之}
施版屋冠以牛皮【冠古玩翻}
上聞不悦貞陽侯淵明時鎮夀陽屢請行上許之會理自以皇孫復為都督【言既以皇孫之貴自高又以都督之尊自處復扶又翻}
自淵明已下殆不對接淵明與諸將密告朱异追會理還遂以淵明為都督辛未高澄入朝於鄴固辭大丞相【以通鑑書法言之辛未之下當有東魏二字朝直遥翻}
詔為大將軍如故餘如前命甲申虚葬齊獻武王於漳水之西潜鑿成安鼓山石窟佛寺之旁為穴【魏收志魏郡臨漳縣有皷山成安縣後齊分臨漳置宋白曰成安縣本漢斥丘縣地春秋時乾侯邑也土地斥鹵故曰斥丘其地在鄴北齊分鄴置成安縣按臨漳縣亦分鄴縣所置}
納其柩而塞之【柩音舊塞悉則翻}
殺其羣匠及齊之亡也一匠之子知之發石取金而逃【史言潛葬之無益}
戊子武州刺史蕭弄璋攻東魏磧泉呂梁二戍拔之【五代志下邳郡下邳縣梁曰歸政置武州魏收志彭城郡呂縣有呂梁城水經注曰泗水之上有石梁焉故曰呂梁}
或告東魏大將軍澄云侯景有北歸之志會景將蔡道遵北歸言景頗知悔過景母及妻子皆在鄴澄乃以書諭之語以闔門無恙若還許以豫州刺史終其身還其寵妻愛子所部文武更不追攝【語牛倨翻攝收也}
景使王偉復書曰今已引二邦【二邦謂梁及西魏也}
揚旌北討熊豹齊奮克復中原幸自取之何勞恩賜昔王陵附漢母在不歸【事見九卷漢高帝元年}
太上囚楚乞羮自若【事見十卷高帝四年}
矧伊妻子而可介意脱謂誅之有益欲止不能殺之無損徒復阬戮家累在君何關僕也【復扶又翻累力瑞翻}
戊子詔以景録行臺尚書事東魏静帝美容儀旅力過人【旅與膂同脊骨也}
能挟石師子踰宫牆射無不中好文學從容沈雅【中竹仲翻好呼到翻從千容翻沈持林翻}
時人以為有孝文風烈大將軍澄深忌之始獻武王自病逐君之醜【謂逐孝武帝使入關也}
事静帝禮甚恭事無大小必以聞可否聽旨【言不敢專决也}
每侍晏俯伏上夀帝設法會乘輦行香歡執香爐步從【上時掌翻從才用翻}
鞠躬屏氣【屏必郢翻}
承望顔色故其下奉帝莫敢不恭及澄當國倨慢頓甚使中書黄門郎崔季舒察帝動静大小皆令季舒知之【晉書職官志曹魏黄初初中書既置監令又置通事郎次黄門郎及晉改曰中書侍郎環濟要畧漢置中書掌密詔有令僕丞郎漢舊儀云置中書領尚書事魏黄初中書置監令又置通事郎次黄門郎即中書侍郎之任也按二書皆謂黄門中書通為一官而五代志紀北齊之制黄門侍郎屬門下省中書侍郎屬中書省分為二官高澄以崔季舒為中書黄門郎者盖澄欲使季舒伺察静帝以為黄門郎則侍從左右以為中書郎則典掌詔命故兼領二職也}
澄與季舒書曰癡人比復何似【比毗至翻復扶又翻}
癡勢小差未【差楚懈翻本作瘥疾稍愈謂之差}
宜用心檢校帝常獵于鄴東馳逐如飛監衛都督烏那羅受工伐從後呼曰天子勿走馬大將軍嗔【監工衘翻監衛都督高氏置此官以監宿衛所以防制其君者也烏那羅虜三字姓呼火故翻嗔昌真翻怒也}
澄嘗侍飲酒舉大觴屬帝曰臣澄勸陛下酒【屬之欲翻舉酒相屬如儕輩然無復君臣之義}
帝不勝忿曰自古無不亡之國朕亦何用此生為澄怒曰朕朕狗脚朕使崔季舒敺帝三拳奮衣而出明日澄使季舒入勞帝【勝音升敺烏口翻勞力到翻}
帝亦謝焉賜季舒絹百匹帝不堪憂辱【徐知訓陵侮其主與高澄異世同轍皆不能保其身詩云人而無禮胡不遄死諒哉}
詠謝靈運詩曰韓亡子房奮秦帝仲連恥本自江海人忠義感君子【謝靈運作詩事見一百二十二卷宋文帝元嘉八年}
常侍侍講潁川荀濟知帝意【荀濟以散騎常侍侍講}
乃與祠部郎中元瑾長秋卿劉思逸華山王大器淮南王宣洪濟北王徽等謀誅澄大器鷙之子也【東魏華山王鷙卒于大同六年華戶化翻濟子禮翻}
帝謬為敇問濟曰欲以何日開講乃詐於宫中作土山開地道向北城至千秋門門者覺地下響以告澄澄勒兵入宫見帝不拜而坐曰陛下何意反臣父子功存社稷何負陛下邪此必左右妃嬪輩所為欲殺胡夫人及李嬪帝正色曰自古惟聞臣反君不聞君反臣王自欲反何乃責我我殺王則社稷安不殺則滅亡無日我身且不暇惜况於妃嬪必欲弑逆緩速在王澄乃下牀叩頭大啼謝罪【高澄雖悖逆不能不屈於静帝之言理所在也}
於是酣飲夜久乃出居三日幽帝於含章堂【含章堂盖取坤卦含章可貞之義必在鄴宫之内殿左右幽者閉帝於内不使出而專殺於外也}
壬辰烹濟等於市初濟少居江東【少詩照翻}
博學能文與上有布衣之舊知上有大志然負氣不服常謂人曰會於盾鼻上磨墨檄之【言上若有非常之舉亦當起兵於盾鼻上磨墨作檄以聲其罪}
上甚不平及即位或薦之於上上曰人雖有才亂俗好反不可用也濟上書諫上崇信佛法為塔寺奢費上大怒欲集朝衆斬之【朝衆即謂在朝百官好呼到翻朝直遥翻}
朱异密告之濟逃奔東魏澄為中書監【大同十年東魏以高澄領中書監}
欲用濟為侍讀獻武王曰我愛濟欲全之故不用濟濟入宫必敗澄固請乃許之【史言高歡識鑒非澄所及}
及敗侍中楊遵彦謂之曰【楊愔字遵彦}
衰暮何苦復爾【復扶又翻}
濟曰壮氣在耳【言年雖衰而氣不衰也}
因下辨曰【辨獄辭也}
自傷年紀摧頹功名不立故欲挟天子誅權臣澄欲宥其死親問之曰荀公何意反濟曰奉詔誅高澄何謂反有司以濟老病鹿車載詣東市并焚之【章懷太子賢曰鹿車小車僅容一鹿也}
澄疑諮議温子昇【子昇盖為大將軍府諮議參軍}
知瑾等謀方使之作獻武王碑既成餓於晉陽獄食弊襦而死棄尸路隅沒其家口【沒其家口為官奴婢填晉陽宫}
太尉長史宋遊道收葬之澄謂遊道曰吾近書與京師諸貴【諸貴謂司馬子如孫騰等}
論及朝士以卿僻於朋黨將為一病今乃知卿真是重故舊尚節義之人天下人代卿怖者是不知吾心也【史言士之狥義者固不計身之死亡亦未必死也怖普布翻}
九月辛丑澄還晉陽 上命蕭淵明堰泗水於寒山以灌彭城俟得彭城乃進軍與侯景犄角【左傳曰譬如捕鹿晉人角之諸戎犄之角者當其前犄者亢其下犄居綺翻}
癸卯淵明軍于寒山去彭城十八里斷流立堰【斷音短}
侍中羊侃監作堰再旬而成【監工衘翻}
東魏徐州刺史太原王則嬰城固守侃勸淵明乘水攻彭城不從諸將與淵明議軍事淵明不能對但云臨時制宜 冬十一月魏丞相泰從魏主狩于岐陽【岐陽岐山之陽也五代志扶風雍縣有岐陽宫}
東魏大將軍澄使大都督高岳救彭城欲以金門郡公潘樂為副【五代志河南郡宜陽縣有東魏所置金門郡因金門山以名郡}
陳元康曰樂緩於機變不如慕容紹宗且先王之命也【高歡令澄用慕容紹宗以敵侯景見上卷上年}
公但推赤心於斯人景不足憂也時紹宗在外澄欲召見之恐其驚叛元康曰紹宗知元康特蒙顧待新使人來餉金【近時之事謂之新}
元康欲安其意受之而厚荅其書保無異也【言保紹宗必無所違異}
乙酉以紹宗為東南道行臺與岳樂偕行初景聞韓軌來曰噉猪腸兒何能為【噉吐濫翻}
聞高岳來曰兵精人凡諸將無不為所輕者及聞紹宗來叩鞍有懼色曰誰教鮮卑兒解遣紹宗來【解胡買翻}
若然【若然猶今人言若如此也}
高王定未死邪澄以廷尉卿杜弼為軍司攝行臺左丞臨發問以政事之要【杜弼臨發從軍澄方問以政事之要盖弼在歡府夙有聲稱故問之也}
可為戒者録一二條弼請口陳之曰天下大務莫過賞罰賞一人使天下之人喜罰一人使天下之人懼苟二事不失自然盡美澄大悦曰言雖不多于理甚要紹宗帥衆十萬據槖駝峴【帥讀曰率峴戶典翻}
羊侃勸貞陽侯淵明乘其遠來擊之不從旦日又勸出戰亦不從侃乃帥所領出屯堰上【羊侃知淵明必敗故出屯堰上欲全所領而退若以行兵之節制言之則安營次舍皆當聽命於元帥豈有擅移屯之理哉}
丙午紹宗至城下引步騎萬人攻潼州刺史郭鳳營【魏收志梁置潼州武定七年改曰睢州治取慮城領淮陽穀陽睢南南濟隂臨潼郡五代志下邳郡夏丘縣東魏置臨潼郡梁置潼州}
矢下如雨淵明醉不能起命諸將救之皆不敢出北兖州刺史胡貴孫謂譙州刺史趙伯超曰【魏收志景明中置譙郡於過陽城孝昌中䧟領南譙汴龍亢蘄城下蔡臨涣蒙郡五代志譙郡山桑縣後魏置渦州渦陽郡東魏改曰譙郡}
吾屬將兵而來【將即亮翻下同}
本欲何為今遇敵而不戰乎伯超不能對貴孫獨帥麾下與東魏戰斬首二百級伯超擁衆數千不敢救謂其下曰虜盛如此與戰必敗不如全軍早歸皆曰善遂遁還初侯景常戒梁人曰逐北不過二里紹宗將戰以梁人輕悍【悍侯旰翻又下罕翻}
恐其衆不支一一引將卒謂之曰我當陽退誘吴兒使前【誘音酉}
爾擊其背東魏兵實敗走梁人不用景言乘勝深入魏將卒以紹宗之言為信争共掩擊之梁兵大敗貞陽侯淵明及胡貴孫趙伯超等皆為東魏所虜失亡士卒數萬人羊侃結陳徐還【陳讀曰陣}
上方晝寢宦者張僧胤白朱异啟事上駭之【非時啟事故駭}
遽起升輿至文德殿閣【文德殿建康宫前殿也}
异曰韓山失律【韓山即寒山}
上聞之怳然將墜牀【怳呼廣翻}
僧胤扶而就坐【坐徂卧翻}
乃歎曰吾得無復為晉家乎【謂為夷狄所取也史言帝危亡將至神不守舍復扶又翻}
郭鳳退保潼州慕容紹宗進圍之十二月甲子朔鳳弃城走東魏使軍司杜弼作檄移梁朝【朝直遥翻下同}
曰皇家垂統光配彼天唯彼吴越獨阻聲教元首懷止戈之心上宰薄兵車之命【元首謂東魏主上宰謂高歡}
遂解縶南冠【左傳楚伐鄭鄭人軍楚師囚鄖公鍾儀獻諸晉晉人囚諸軍府晉侯觀于軍府見鍾儀問曰南冠而縶者誰也有司對曰鄭人所獻楚囚也命税之使歸合晉楚之成}
喻以好睦【大同三年梁初與東魏通和好呼到翻下同}
雖嘉謀長筭爰自我始罷戰息民彼獲其利侯景豎子自生猜貳遠託關隴依憑姦偽逆主定君臣之分偽相結兄弟之親【謂侯景先降西魏也分扶問翻相息亮翻}
豈曰無恩終成難養俄而易慮親尋干戈釁暴惡盈側首無託【謂侯景不見容于西魏也}
以金陵逋逃之藪江南流寓之地甘辭卑禮進孰圖身【此以下皆言侯景歸梁之心迹孰古熟字通言進軟熟之辭於梁以為容身之圖}
詭言浮說抑可知矣而偽朝大小幸災忘義主荒於上臣蔽於下連結姦惡斷絶鄰好徵兵保境縱盗侵國盖物無定方事無定勢或乘利而受害或因得而更失是以吳侵齊境遂得句踐之師【左傳吴伐齊敗齊師於艾陵遂與晉侯會于黄池越子句踐乘虚伐吴獲其太子遂入吴吴王歸及越平其後越遂伐吴滅之句音鉤}
趙納韓地終有長平之役【事見五卷周赧王五十三年至五十五年}
矧乃鞭撻疲民侵軼徐部築壘擁川舍舟徼利【軼徒結翻又音逸杜預曰軼突也擁當作壅舍讀曰捨徼一遥翻}
是以援枹秉麾之將拔距投石之士【師古曰拔距者有人連坐相把據地以為堅而能拔取之投石者以石投人皆言其有勇力也援于元翻枹音膚將即亮翻}
含怒作色如赴私讐彼連營擁衆依山傍水【傍步浪翻}
舉螳螂之斧被蛣蜣之甲【螳螂舉臂以捍物微有鋒利故以諭斧蛣蜣蜣蜋也翼在甲下故以諭甲言梁兵之輕弱也蛣音詰}
當窮轍以待輪【古語云螳蜋怒臂以當車轍陸佃曰螳蜋有斧蟲也兖人謂之拒斧奮之當轍不避釋蟲不蟷蠰其子螵蛸舍人云不名蟷蠰今之螳蜋也方言云譚魯以南謂之蟷蠰三河之域謂之螳蜋燕趙之際謂之食厖齊以東謂之馬穀然名其子同云螵蛸也}
坐積薪而候燎及鋒刃纔交埃塵且接已亡戟弃戈土崩瓦解掬指舟中衿甲鼓下【左傳晉荀林父帥師及楚子戰於邲楚乘晉師林父不知所為皷于軍中曰先濟者有賞中軍與下軍争舟舟中之指可掬也又晉伐齊齊師夜遁晉師從之夙沙衛連大車塞隧以殿殖綽郭最曰子殿齊師國之辱也乃代之殿衛殺馬于隘以塞道晉州綽及之射殖綽中肩弛弓而自後縛之其右具丙亦舍兵而縛郭最皆衿甲面縛坐于中軍之皷下衿其鴆翻}
同宗異姓縲紲相望曲直既殊彊弱不等獲一人而失一國【左傳宋猛獲與南宫萬弑其君宋討之猛獲奔衛宋人請之衛人欲弗許石祁子曰天下之惡一也惡於宋而保於我保之何補得一夫而失一國與惡而棄好非謀也衛人歸之}
見黄雀而忘深穽【穽疾正翻}
智者所不為仁者所不向誠既往之難逮猶將來之可追【逮及也此二語以誘梁欲再與講和以携侯景}
侯景以鄙俚之夫遭風雲之會位班三事邑啟萬家揣身量分久當止足而周章向背離披不已【周章怔營貌離披分散不可收束之意揣初委翻量音良分扶問翻背蒲妹翻}
夫豈徒然意亦可見彼乃授之以利器誨之以慢藏【老子曰國之利器不可以授人易曰慢藏誨盗藏徂浪翻}
使其勢得容姦時堪乘便今見南風不競【左傳晉圍齊楚乘其間伐鄭晉人聞之師曠曰不害吾驟歌南風又歌北風南風不競多死聲楚必無功果如其言}
天亡有徵【徵讀曰證}
老賊姦謀將復作矣然推堅強者難為功【復扶又翻推吐雷翻}
摧枯朽者易為力計其雖非孫吳猛將燕趙精兵猶是久涉行陳【將即亮翻燕因肩翻易弋䜴翻行戶剛翻陳讀曰陣}
曾習軍旅豈同剽輕之師【漢張良曰楚兵剽輕剽匹妙翻輕牽正翻}
不比危脆之衆【脆此芮翻}
拒此則作氣不足攻彼則為勢有餘終恐尾大於身踵麤于股倔彊不掉【倔其勿翻彊其兩翻}
狼戾難馴【狼當作狠}
呼之則反速而舋小【舋許覲翻}
不徵則叛遲而禍大會應遥望廷尉不肯為臣【用蘇峻事見九十三卷晉成帝咸和二年}
自據淮南亦欲稱帝【用黥布事見十二卷漢高帝十一年}
但恐楚國亡猨禍延林木城門失火殃及池魚【池魚人姓名風俗通有池仲魚城門失火仲魚燒死故諺曰城門失火殃及池魚一曰城門失火汲城下之池水以救之池涸則魚受其殃}
横使江淮士子荆揚人物死亡矢石之下夭折霧露之中【横戶孟翻夭於紹翻折而設翻又之舌翻}
彼梁主者操行無聞輕險有素射雀論功蕩舟稱力【國語晉平公射鷃不死使䜿襄之失公怒將殺之以告叔向叔向曰君必殺之昔先君唐叔射兕於徒林以為大甲以封於晉今君嗣先君唐叔射鷃不死摶之不得是揚吾君之恥者也必殺之君忸怩顔乃赦之鷃扈小鳥即鷃雀也左傳齊桓公與蔡姬乘舟于囿蕩公杜預注曰蕩揺也操七到翻行下孟翻射而亦翻}
年既老矣耄又及之政散民流禮崩樂壞加以用舍乖方廢立失所【用舍乖方謂免周捨責顧琛而用朱异廢立失所謂衘昭明而不立世適孫乃立太子綱也舍讀曰捨}
矯情動俗飾智驚愚毒螫滿懷妄敦戒業躁競盈胸謬治清净【此數語曲盡帝之心事螫音釋躁則到翻治直之翻}
災異降於上怨讟興於下人人厭苦家家思亂履霜有漸堅冰且至【易坤卦初六爻辭曰履霜堅冰至象曰履霜堅冰隂始凝也馴致其道至堅冰也文言曰臣弑其君子弑其父非一朝一夕之故其所由來者漸矣由辯之不早辯也}
傳險躁之風俗任輕薄之子孫朋黨路開兵權在外必將禍生骨肉舋起腹心彊弩衝城長戈指闕徒探雀鷇無救府藏之虚【探雀鷇趙武靈王事見四卷周赧王二十年探吐南翻藏徂浪翻}
空請熊蹯詎延晷刻之命【左傳楚世子商臣圍其父成王王請食熊蹯而死不許乃縊杜預注曰熊蹯難熟冀久將有外救蹯音煩}
外崩中潰今實其時鷸蚌相持我乘其弊【戰國策趙且伐燕蘇代為燕謂惠王曰今者臣來過易水蚌方出曝而鷸啄其肉蚌合而拑其喙鷸曰今日不雨明日不雨必有死蚌蚌亦謂鷸曰今日不出明日不出即有死鷸兩者不肯相舍漁父得而并禽之今趙且伐燕燕趙久相支以弊大衆臣恐彊秦之為漁父也}
方使駿騎追風精甲輝日四七並列【漢光武用二十八將以定天下後人贊之曰授鉞四七}
百萬為羣以轉石之形【孫子曰任勢者其戰人也如轉木石木石之性安則静危則動方則止圓則行故善戰人之勢如轉圓石於千仭之山者勢也}
為破竹之勢【破竹杜預之言見八十一卷晉武帝太康元年}
當使鍾山渡江青盖入洛荆棘生於建業之宫麋鹿遊於姑蘇之館【青盖入洛事見七十九卷晉武帝泰始八年漢淮南王安隂有邪謀伍被諫曰昔子胥諫吳王吳王不用乃曰臣今見麋鹿遊姑蘇之臺也今臣亦見宫中將生荆棘露霑衣也}
但恐革車之所轥轢【轥力刃翻踐也轢來各翻碾也}
劎騎之所蹂踐梓於焉傾折竹箭以此摧殘【杞梓竹箭東南之嘉產也蹂人九翻踐息淺翻折而設翻}
若吳之王孫蜀之公子【晉左思設為東吴王孫西蜀公子以賦三都弼引用之}
歸欵軍門委命下吏當即授客卿之秩特加驃騎之號凡百君子勉求多福【李斯自楚入秦為客卿孫秀自吳奔晉為驃騎將軍弼以此誘南人要亦書檄之常談耳}
其後梁室禍敗皆如弼言侯景圍譙城不下退攻城父拔之壬申遣其行臺左丞王偉等詣建康說上曰【說式芮翻}
鄴中文武合謀召臣共討高澄事泄澄幽元善見於金墉殺諸元六十餘人河北物情俱念其主請立元氏一人以從人望如此則陛下有繼絶之名臣景有立功之效河之南北為聖朝之邾莒【言為小國以附于大國朝直遥翻}
國之男女為大梁之臣妾上以為然【此杜弼所謂進孰圖身者也帝早在兵間曾不見此盖天奪其鑒也}
乙亥下詔以太子舍人元貞為咸陽王 【考異曰梁紀作戊辰遣貞今從典略}
資以兵力使還北主魏須渡江許即位【須待也}
儀衛以乘輿之副給之【乘繩證翻}
貞樹之子也【元樹來奔中大通四年為樊子鵠所禽}
蕭淵明至鄴東魏主升閶闔門受俘讓而釋之送於晉陽大將軍澄待之甚厚【為澄因淵明約和以間侯景張本}
慕容紹宗引軍擊侯景景輜重數千兩馬數千匹士卒四萬人退保渦陽【輜重之上當有景字文意乃明重直用翻兩音亮渦音戈}
紹宗士卒十萬旗甲耀日鳴鼓長驅而進景使謂之曰公等為欲送客為欲定雌雄邪紹宗曰欲與公决勝負遂順風布陳【陳讀曰陣}
景閉壘俟風止乃出【戰不逆風故景俟風止乃出}
紹宗曰侯景多詭計好乘人背使備之果如其言景命戰士皆被短甲【好呼到翻被皮義翻}
執短刀入東魏陳但低視斫人脛馬足東魏兵遂敗【被短甲執短刀入敵陳力戰此必死之兵也紹宗之敗不亦宜乎其後景用此以敵陳霸先亦此術耳惟陳堅不可破是以一敗不能復振卒以走死}
紹宗墜馬儀同三司劉豐生被傷顯州刺史張遵業為景所擒【魏收志永安中置顯州治汾州六壁城領定戎建平真君郡}
紹宗豐生俱奔譙城禆將斛律光張恃顯尤之【尤之者責過之也將即亮翻}
紹宗曰吾戰多矣未見如景之難克者也君輩試犯之光等被甲將出紹宗戒之曰勿度渦水二人軍於水北光輕騎射之【被皮義翻渦工禾翻射而亦翻下為射遷射同}
景臨渦水謂光曰爾求勲而來我懼死而去我汝之父友【光父斛律金與景同事爾朱高歡故自謂父友}
何為射我汝豈自解不度水南【解戶買翻}
慕容紹宗教汝也光無以應景使其徒田遷射光馬洞胷光易馬隱樹又中之【中竹仲翻}
退入於軍景擒恃顯既而捨之光走入譙城紹宗曰今定何如而尤我也光金之子也開府儀同三司段韶夹渦而軍潛於上風縱火景帥騎入水出而却走草濕火不復然【斛律光之勇雖不利段韶之智雖不獲逞然東魏之士氣未衰也故慕容紹宗乘機而運其巧得以成功觀史者若祗以一時勝負論人非有識畧者也帥讀曰率復扶又翻}
魏岐州久經喪亂【喪息浪翻}
刺史鄭穆初到有戶三千穆撫循安集數年之間至四萬餘戶考績為諸州之最丞相泰擢穆為京兆尹 侯景與東魏慕容紹宗相持數月景食盡司馬世雲降於紹宗【至是則侯景潰敗之形成矣}


資治通鑑卷一百六十














































































































































