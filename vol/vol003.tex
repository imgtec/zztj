<!DOCTYPE html PUBLIC "-//W3C//DTD XHTML 1.0 Transitional//EN" "http://www.w3.org/TR/xhtml1/DTD/xhtml1-transitional.dtd">
<html xmlns="http://www.w3.org/1999/xhtml">
<head>
<meta http-equiv="Content-Type" content="text/html; charset=utf-8" />
<meta http-equiv="X-UA-Compatible" content="IE=Edge,chrome=1">
<title>資治通鑒_4-資治通鑑卷三_4-資治通鑑卷三</title>
<meta name="Keywords" content="資治通鑒_4-資治通鑑卷三_4-資治通鑑卷三">
<meta name="Description" content="資治通鑒_4-資治通鑑卷三_4-資治通鑑卷三">
<meta http-equiv="Cache-Control" content="no-transform" />
<meta http-equiv="Cache-Control" content="no-siteapp" />
<link href="/img/style.css" rel="stylesheet" type="text/css" />
<script src="/img/m.js?2020"></script> 
</head>
<body>
 <div class="ClassNavi">
<a  href="/24shi/">二十四史</a> | <a href="/SiKuQuanShu/">四库全书</a> | <a href="http://www.guoxuedashi.com/gjtsjc/"><font  color="#FF0000">古今图书集成</font></a> | <a href="/renwu/">历史人物</a> | <a href="/ShuoWenJieZi/"><font  color="#FF0000">说文解字</a></font> | <a href="/chengyu/">成语词典</a> | <a  target="_blank"  href="http://www.guoxuedashi.com/jgwhj/"><font  color="#FF0000">甲骨文合集</font></a> | <a href="/yzjwjc/"><font  color="#FF0000">殷周金文集成</font></a> | <a href="/xiangxingzi/"><font color="#0000FF">象形字典</font></a> | <a href="/13jing/"><font  color="#FF0000">十三经索引</font></a> | <a href="/zixing/"><font  color="#FF0000">字体转换器</font></a> | <a href="/zidian/xz/"><font color="#0000FF">篆书识别</font></a> | <a href="/jinfanyi/">近义反义词</a> | <a href="/duilian/">对联大全</a> | <a href="/jiapu/"><font  color="#0000FF">家谱族谱查询</font></a> | <a href="http://www.guoxuemi.com/hafo/" target="_blank" ><font color="#FF0000">哈佛古籍</font></a> 
</div>

 <!-- 头部导航开始 -->
<div class="w1180 head clearfix">
  <div class="head_logo l"><a title="国学大师官网" href="http://www.guoxuedashi.com" target="_blank"></a></div>
  <div class="head_sr l">
  <div id="head1">
  
  <a href="http://www.guoxuedashi.com/zidian/bujian/" target="_blank" ><img src="http://www.guoxuedashi.com/img/top1.gif" width="88" height="60" border="0" title="部件查字,支持20万汉字"></a>


<a href="http://www.guoxuedashi.com/help/yingpan.php" target="_blank"><img src="http://www.guoxuedashi.com/img/top230.gif" width="600" height="62" border="0" ></a>


  </div>
  <div id="head3"><a href="javascript:" onClick="javascript:window.external.AddFavorite(window.location.href,document.title);">添加收藏</a>
  <br><a href="/help/setie.php">搜索引擎</a>
  <br><a href="/help/zanzhu.php">赞助本站</a></div>
  <div id="head2">
 <a href="http://www.guoxuemi.com/" target="_blank"><img src="http://www.guoxuedashi.com/img/guoxuemi.gif" width="95" height="62" border="0" style="margin-left:2px;" title="国学迷"></a>
  

  </div>
</div>
  <div class="clear"></div>
  <div class="head_nav">
  <p><a href="/">首页</a> | <a href="/ShuKu/">国学书库</a> | <a href="/guji/">影印古籍</a> | <a href="/shici/">诗词宝典</a> | <a   href="/SiKuQuanShu/gxjx.php">精选</a> <b>|</b> <a href="/zidian/">汉语字典</a> | <a href="/hydcd/">汉语词典</a> | <a href="http://www.guoxuedashi.com/zidian/bujian/"><font  color="#CC0066">部件查字</font></a> | <a href="http://www.sfds.cn/"><font  color="#CC0066">书法大师</font></a> | <a href="/jgwhj/">甲骨文</a> <b>|</b> <a href="/b/4/"><font  color="#CC0066">解密</font></a> | <a href="/renwu/">历史人物</a> | <a href="/diangu/">历史典故</a> | <a href="/xingshi/">姓氏</a> | <a href="/minzu/">民族</a> <b>|</b> <a href="/mz/"><font  color="#CC0066">世界名著</font></a> | <a href="/download/">软件下载</a>
</p>
<p><a href="/b/"><font  color="#CC0066">历史</font></a> | <a href="http://skqs.guoxuedashi.com/" target="_blank">四库全书</a> |  <a href="http://www.guoxuedashi.com/search/" target="_blank"><font  color="#CC0066">全文检索</font></a> | <a href="http://www.guoxuedashi.com/shumu/">古籍书目</a> | <a   href="/24shi/">正史</a> <b>|</b> <a href="/chengyu/">成语词典</a> | <a href="/kangxi/" title="康熙字典">康熙字典</a> | <a href="/ShuoWenJieZi/">说文解字</a> | <a href="/zixing/yanbian/">字形演变</a> | <a href="/yzjwjc/">金 文</a> <b>|</b>  <a href="/shijian/nian-hao/">年号</a> | <a href="/diming/">历史地名</a> | <a href="/shijian/">历史事件</a> | <a href="/guanzhi/">官职</a> | <a href="/lishi/">知识</a> <b>|</b> <a href="/zhongyi/">中医中药</a> | <a href="http://www.guoxuedashi.com/forum/">留言反馈</a>
</p>
  </div>
</div>
<!-- 头部导航END --> 
<!-- 内容区开始 --> 
<div class="w1180 clearfix">
  <div class="info l">
   
<div class="clearfix" style="background:#f5faff;">
<script src='http://www.guoxuedashi.com/img/headersou.js'></script>

</div>
  <div class="info_tree"><a href="http://www.guoxuedashi.com">首页</a> > <a href="/SiKuQuanShu/fanti/">四库全书</a>
 > <h1>资治通鉴</h1> <!--         下载:【右键另存为】即可 --></div>
  <div class="info_content zj clearfix">
  
<div class="info_txt clearfix" id="show">
<center style="font-size:24px;">4-資治通鑑卷三</center>
    資治通鑑卷三     宋 司馬光 撰<br />
<br />
  胡三省 音註<br />
<br />
  周紀三【起重光赤奮若盡昭陽大淵獻凡二十有三年起辛丑盡癸亥也】<br />
<br />
  愼靚王【諱定顯王之子也此複詞也諱法敏以敬曰愼柔德安衆曰靚靚疾正翻】<br />
<br />
  元年衛更貶號曰君【顯王二十三年衛已貶號曰侯介於秦魏之間國日以削弱因更貶其號曰君更居孟翻貶悲檢翻】<br />
<br />
  二年秦伐韓取鄢【春秋晉敗楚師于鄢陵即此鄢也班志作傿陵屬潁川郡鄢音謁晚翻又於建翻師古音偃史記正義曰許州鄢陵縣西北十五里有鄢陵故城】 魏惠王薨子襄王立【索隱曰系本曰襄王名嗣今按系本即世本司馬貞避唐諱改世為系 考異曰史記魏世家云惠王三十六年卒子襄王立襄王十六年卒子哀王立哀王二十三年卒子昭王立六國表惠王元辛亥終丙戌襄王元丁亥終壬寅哀王元癸卯終乙丑按杜預春秋後序云太康初汲縣有發舊冢者大得古書其紀年篇起自夏殷周皆三代王事無諸國别也惟特記晉國起自殤叔次文侯昭侯以至曲沃莊伯皆用夏正編年相次晉國滅獨記魏事下至魏哀王之二十年蓋魏國之史記也哀王於史記襄王之子惠王之孫也古書紀年篇惠王三十六年改元從一年始至十六年而稱惠成王卒即惠王也疑史記誤分惠成之世以為後王年也哀王二十三年乃卒故特不稱諡謂之今王裴駰魏世家註引和嶠云紀年起自黄帝終於魏之今王今王者魏惠成王子按太史公書惠成王但言惠王惠王子曰襄王襄王子曰哀王惠王三十六年卒襄王立十六年卒并惠襄為五十二年今按古文惠成王立三十六年改元稱一年改元後十七年卒太史公書為誤分惠成之世以為二王之年數也世本惠王生襄王而無哀王然則今王者魏襄王也彼既魏史所書魏事必得其眞今從之】孟子入見而出語人曰望之不似人君就之而不見所畏焉【入見賢遍翻語牛倨翻】卒然問曰天下惡乎定【卒七没翻惡音烏何也】吾對曰定于一孰能一之【此一語魏襄王以問孟子】對曰不嗜殺人者能一之孰能與之【此語亦襄王問】對曰天下莫不與也王知夫苖乎【夫音扶】七八月之間旱則苖槁矣【孟子此言用周正也周七八月夏五六月也槁音考乾枯也夏戶雅翻乾音干】天油然作雲沛然下雨則苖浡然興之矣【油然雲盛貌沛然雨盛貌浡然興起貌沛普蓋翻浡音勃】其如是孰能禦之<br />
<br />
  三年楚趙魏韓燕同伐秦攻函谷關【燕因肩翻註已見上宋白曰函谷關在弘農地理志注云謂道形如函孫卿子所謂秦有松柏之塞是也】秦人出兵逆之五國之師皆敗走 宋初稱王<br />
<br />
  四年秦敗韓師于修魚斬首八萬級虜其將䱸申差于濁澤【敗補邁翻索隱曰修魚地名䱸申差二將名索山客翻將即亮翻䱸音瘦人踈鳩翻濁澤年表作觀澤括地志觀澤在魏州頓邱縣東十八里】諸侯振恐 齊大夫與蘇秦爭寵使人刺秦殺之【刺七亦翻】張儀說魏襄王曰梁地方不至千里卒不過三十萬地四平無名山大川之限卒戍楚韓齊趙之境【戍舂遇翻字從人從戈人荷戈所以戍也梁地南接楚西接韓東接齊北接趙】守亭障者不過十萬【說文亭民所安定也道路所舍也障堡障也隔也塞也所以隔塞敵人也】梁之地勢固戰場也夫諸侯之約從盟於洹水之上結為兄弟以相堅也【事見上卷顯王二十六年夫音扶從子容翻洹于元翻】今親兄弟同父母尚有爭錢財相殺傷而欲恃反覆蘇秦之餘謀其不可成亦明矣大王不事秦秦下兵攻河外據卷衍酸棗【後漢志卷縣屬河南郡酸棗縣屬陳留郡水經注河水逕卷縣北又東至酸棗延津二邑皆河津之要也卷逵員翻衍以善翻】刼衛取陽晉則趙不南趙不南則梁不北梁不北則從道絶從道絶則大王之國欲毋危不可得也【從道謂約從之路也從子容翻】故願大王審定計議且賜骸骨【人臣委身以事君身非我之有矣故於其乞退也謂之乞骸骨骸戶皆翻】魏王乃倍從約【倍蒲妹翻】而因儀以請成于秦張儀歸復相秦【儀罷秦相相魏見上卷顯王四十七年相息亮翻】 魯景公薨子平公旅立【謚法由義而濟曰景布義行剛曰景】五年巴蜀相攻擊【巴春秋巴子之國蜀蠶樷魚鳬之後華陽國志曰昔蜀王封其弟于漢中號曰苴侯因命其巴曰葭萌苴侯與巴王為好後巴與蜀為讐蜀王怒伐苴侯苴侯奔巴巴求救于秦秦伐蜀蜀王敗死秦滅蜀因遂滅巴苴置巴蜀二郡史記正義曰巴子城在合州石鏡縣南五里故墊江縣也宋白曰巴子後理閬中揚雄蜀本紀曰蜀王本治廣都樊鄉徙居城都華戶化翻苴子余翻葭音家萌謨耕翻墊音疊閬音浪】俱告急於秦秦惠王欲伐蜀以為道險陿難至【陿與狹同漢書趙充國傳註山附而夾水曰陿】而韓又來侵猶豫未能决【說文猶玃屬居山中聞人聲豫登术無人乃下世謂不决曰猶豫一說隴西謂犬子為猶犬導人行忽先忽後故曰猶豫又一說猶豫犬也犬為人行好先行却住以俟其人百步之間如是者數四先者豫也遂曰猶豫猶夷周翻又余救翻玃厥縛翻為于偽翻好呼到翻】司馬錯請伐蜀【史記重黎之後至周宣王時為程伯休父為司馬氏錯七客翻又七故翻重直龍翻父音甫】張儀曰不如伐韓王曰請聞其說儀曰親魏善楚下兵三川攻新城宜陽【伊水洛水河水為三川秦後置三川郡漢改為河南郡班志新城縣屬河南郡括地志洛州伊闕縣本漢新城縣在州南七十里隋文帝改新城為伊闕取伊闕山為名】以臨二周之郊【周分為東西故曰二周】據九鼎【昔夏禹貢金九牧鑄鼎象物桀有昏德鼎遷于商商紂暴虐鼎遷于周成王定鼎于郟鄏寶之以為三代共器夏戶雅翻郟音夾鄏音辱】按圖籍【圖籍謂天下之圖籍周官職方氏所掌是也】挾天子以令于天下天下莫敢不聽此王業也臣聞爭名者于朝爭利者于市今三川周室天下之朝市也【朝直遥翻周禮大宗伯注云朝猶朝也欲其來之早也人君昕旦親政貴早聲轉為朝猶朝陟遥翻】而王不爭焉顧爭於戎翟去王業遠矣【翟與狄同】司馬錯曰不然臣聞欲富國者務廣其地欲彊兵者務富其民欲王者務博其德【欲王于况翻又如字】三資者備而王隨之矣今王地小民貧故臣願先從事於易【易弋䜴翻】夫蜀西僻之國而戎翟之長也【夫音扶長知丈翻】有桀紂之亂以秦攻之譬如使豺狼逐羣羊【豺狙齋翻】得其地足以廣國取其財足以富民繕兵不傷衆而彼已服焉【彼謂蜀也】拔一國而天下不以為暴利盡四海而天下不以為貪是我一舉而名實附也而又有禁暴止亂之名今攻韓刼天子惡名也而未必利也又有不義之名而攻天下所不欲危矣臣請論其故周天下之宗室也【周室為天下所宗故謂之宗室】齊韓之與國也【鄰國相親睦者謂之與國】周自知失九鼎韓自知亡三川將二國并力合謀以因乎齊趙而求解乎楚魏【并必正翻求解者先與之搆怨隙而今求和解也】以鼎與楚以地與魏王弗能止也此臣之所謂危也不如伐蜀完【完全也言以兵伐蜀十全必取也】王從錯計【錯七各翻又七故翻】起兵伐蜀十月取之【取言易也易弋䜴翻】貶蜀王更號為侯【貶悲檢翻更工衡翻】而使陳莊相蜀【相息亮翻】蜀旣屬秦秦以益彊富厚輕諸侯 蘇秦既死【三年蘇秦死于齊】秦弟代厲亦以遊說顯於諸侯【說式芮翻】燕相子之與蘇代婚欲得燕權蘇代使於齊而還【燕因肩翻相息亮翻使疏吏翻還從宣翻】燕王噲問曰齊王其覇乎【噲苦夬翻】對曰不能王曰何故對曰不信其臣于是燕王專任子之鹿毛夀謂燕王曰【劉伯莊曰鹿毛夀人姓名又曰潘夀春秋後語作唐毛夀徐廣曰一作庴毛如徐廣一作之說當作庴庴音秦昔翻清河有庴縣】人之謂堯賢者以其能讓天下也今王以國讓子之是王與堯同名也燕王因屬國於子之【屬之欲翻付也託也】子之大重或曰禹薦益而以啓人為吏【孟子曰禹薦益於天禹崩天下之人不之益而之啓曰吾君之子也索隱曰人猶臣也謂以啓臣為益吏索山客翻】及老而以啓為不足任天下【任音壬】傳之於益啓與交黨攻益奪之天下謂禹名傳天下於益而實令啓自取之【按或曰一段事與師春紀伊尹放太甲潜出自 桐殺伊尹事頗相類古書雜記固多也】今王言屬國于子之而吏無非太子人者是名屬子之而實太子用事也王因收印綬自三百石吏已上而效之子之【後漢書輿服志曰三王俗化雕文詐偽漸生始有印綬以檢姦萌周禮掌節有璽節鄭氏注云今之印章也綬組綬古者佩玉以綬貫之漢承秦制乘輿璽綬諸王以下印以金銀銅為差綬以赤紫青黑黄為差印信也刻文合信也綬受也轉相授受也三百石吏銅印黑綬或黄綬王制諸侯大國之卿食祿以田計之為三十二夫之入戰國之卿食祿萬鍾其僭差不度甚矣漢制三公秩萬石至千斗食佐吏凡十六等三百石吏第十等其奉月四十斛綬音受璽斯氏翻組祖五翻乘䋲證翻奉與俸同音扶用翻】子之南面行王事而噲老不聽政顧為臣【顧反也噲苦夬翻】國事皆决于子之【為後燕亂張本】六年王崩子赧王延立<br />
<br />
  赧王上【劉伯莊曰赧慙之甚也輕微危弱寄住東西足為慙赧故號之曰赧諡法本無赧字也赧音奴版翻】<br />
<br />
  元年秦人侵義渠得二十五城【義渠戎國名按上卷顯王四十二年秦縣義渠以其君為臣是已得義渠矣今又侵得二十五城何也蓋先此秦以義渠為縣君為臣雖臣屬于秦義渠之國未滅也秦稍蠶食侵其地今得二十五城義渠之國所餘無幾矣蓋秦兼并諸侯不盡其國不止也左傳有鐘鼔曰伐無曰侵穀梁傳苞人民驅牛馬曰侵斬樹木壤宫室曰伐無幾居豈翻傳直戀翻壞音怪】 魏人叛秦秦人伐魏取曲沃而歸其人又敗韓于岸門【續漢志潁川郡穎隂縣有岸亭註引徐廣云岸亭即岸門括地志岸門在今許州長社縣東北二十八里今名長亭敗補邁翻】韓太子倉入質于秦以和【質音致】 燕子之為王三年國内大亂將軍市被與太子平謀攻子之齊王令人謂太子曰【令廬經翻】寡人聞太子將飭君臣之義【飭整也修也治也治直之翻飭君臣之義言太子平將治子之僭王之罪也明父子之位言太子平當繼其父噲之位也】明父子之位寡人之國唯太子所以令之【令力政翻命令也號令也】太子因要黨聚衆【要一遥翻要結也】使市被攻子之不克市被反攻太子搆難數月【難乃旦翻】死者數萬人百姓恫恐【恫它紅翻痛也】齊王令章子將五都之兵因北地之衆以伐燕【將即亮翻又音如字領也邑有先王之廟曰都或曰都邑之大者北地齊之北境也蓋漢千乘清河渤海之地燕因肩翻下同乘䋲證翻】燕士卒不戰城門不閉齊人取子之醢之【醢呼改翻肉醤也】遂殺燕王噲【噲苦夬翻】齊王問孟子曰或謂寡人勿取燕或謂寡人取之以萬乘之國伐萬乘之國【古者天子之地方千里出兵車萬乘七國兼并以彊大於時皆為萬乘之國乘繩證翻】五旬而舉之【十日為旬五旬五十日也】人力不至于此不取必有天殃【殃咎也旤也】取之何如孟子對曰取之而燕民悦則取之古之人有行之者武王是也取之而燕民不悦則勿取古之人有行之者文王是也以萬乘之國伐萬乘之國簞食壺漿以迎王師【簞竹器也圓曰簞方曰笥簞音丹食祥吏翻熟食也漿水也酢漿也笥相吏翻酢倉故翻】豈有他哉避水火也如水益深如火益熱亦運而已矣【運轉也言燕之民將轉而之他國也】諸侯將謀救燕齊王謂孟子曰諸侯多謀伐寡人者何以待之對曰臣聞七十里為政於天下者湯是也未聞以千里畏人者也書曰徯我后后來其蘇【書仲虺之誥之辭徯戶禮翻待也后君也】今燕虐其民王往而征之民以為將拯己於水火之中也【拯上舉也援也救也助也音之凌翻】簞食壺漿以迎王師若殺其父兄係累其子弟【趙岐曰係累縛結也系戶計翻累力追翻】毁其宗廟遷其重器【重器國之寶鎭】如之何其可也天下固畏齊之彊也今又倍地【齊并燕則地倍其舊燕因肩翻】而不行仁政是動天下之兵也王速出令反其旄倪【令力政翻趙岐曰旄老旄倪弱小陸德明曰倪謂翳倪小兒也記曲禮曰八十九十曰耄註云耄惛忘也旄讀曰耄倪五兮翻翳與繄同音烟兮翻】止其重器謀於燕衆置君而後去之則猶可及止也齊王不聽已而燕人叛【是時燕人雖未立太子平固已相帥叛齊矣】王曰吾甚慚於孟子陳賈曰王無患焉乃見孟子曰周公何人也曰古聖人也陳賈曰周公使管叔監商【古殷商通稱商者以始封為國號殷者以都亳為國號按孟子陳賈只云監殷今通鑑云監商避宋廟諱也監古衘翻】管叔以商畔也周公知其將畔而使之與【畔與叛同與讀曰歟下同】曰不知也陳賈曰然則聖人亦有過與曰周公弟也管叔兄也周公之過不亦宜乎且古之君子過則改之今之君子過則順之古之君子其過也如日月之食民皆見之及其更也【更工衡翻更改】民皆仰之今之君子豈徒順之又從為之辭 是歲齊宣王薨子湣王地立【湣讀曰閔】<br />
<br />
  二年秦右更疾伐趙【右更秦爵第十四師古曰左右中更皆主領更卒而部其役使也更工衡翻】拔藺虜其將莊豹【莊姓有出于宋者左傳所謂戴武莊之族是也有出于楚者楚莊王之後莊蹻是也齊之莊暴楚之莊辛蒙之莊周與此莊豹其時適相先後莫能審其所自出】 秦王欲伐齊患齊楚之從親【從子容翻】乃使張儀至楚說楚王曰大王誠能聽臣閉關絶約於齊【說式芮翻閉關者古之列國各置關尹敵國賓至關尹以告則行理以節逆之閉關則距絶其使不為通也使疏吏翻】臣請獻商於之地六百里使秦女得為大王箕帚之妾【於如字箕帚之妾猶言備洒掃也帚止酉翻篲也】秦楚嫁女娶婦長為兄弟之國楚王說而許之【說讀曰悦】羣臣皆賀陳軫獨弔【陳姓出於舜周武王封舜後于陳子孫以國為氏】王怒曰寡人不興師而得六百里地何弔也對曰不然以臣觀之商於之地不可得而齊秦合齊秦合則患必至矣王曰有說乎對曰夫秦之所以重楚者以其有齊也【夫音扶發語辭】今閉關絶約于齊則楚孤秦奚貪夫孤國而與之商於之地六百里張儀至秦必負王是王北絶齊交西生患於秦也【楚東北接齊西接秦】兩國之兵必俱至為王計者不若隂合而陽絶於齊使人隨張儀苟與吾地絶齊未晚也王曰願陳子閉口毋復言以待寡人得地【毋音無毋者禁止之辭復扶又翻再又也】乃以相印授張儀【相息亮翻】厚賜之遂閉關絶約于齊使一將軍隨張儀至秦【班固百官表將軍周末官秦漢因之】張儀詳墮車【詳讀曰佯詐也】不朝三月【朝直遥翻】楚王聞之曰儀以寡人絶齊未甚邪【邪余遮翻邪疑辭】乃使勇士宋遺借宋之符北罵齊王【旣閉關絶約則齊楚之信使不通故使宋遺借宋符以至齊宋姓也周武王封微子於宋子孫以國為氏】齊王大怒折節以事秦【折而設翻】齊秦之交合【儀歸而詐疾待齊秦之交合乃朝】張儀乃朝見楚使者曰子何不受地從某至某廣袤六里【朝直遥翻使疏吏翻東西曰廣南北曰袤廣古曠翻又讀如字袤音茂】使者怒還報楚王【還從宣翻又音如字】楚王大怒欲發兵而攻秦陳軫曰軫可口言乎攻之不如因賂之以一名都與之并力而攻齊是我亡地於秦取償於齊也【償辰羊翻報也】今王已絶于齊而責欺于秦是吾合齊秦之交而來天下之兵也國必大傷矣楚王不聽使屈匄帥師伐秦【屈姓也音九勿翻匄居大翻帥讀曰率】秦亦發兵使庶長章擊之【長知丈翻按史記樗里子傳庶長章姓魏】<br />
<br />
  三年春秦師及楚戰于丹陽【索隱曰此丹陽在漢中劉昭曰南郡枝江縣有丹陽聚即秦破楚處李輿地紀勝曰丹陽在今歸州秭歸縣東八里屈沱楚王城是也余按楚遣屈匄伐秦秦發兵逆擊之枝江之丹陽則距郢逼近秭歸之丹陽則不當秦楚之路索隱因下文遂取漢中即謂丹陽在漢中皆非也此丹陽謂丹水之陽班志丹水出上洛冢嶺山東至析入鈞水其水蓋在弘農丹水析兩縣之間武關之外也秦楚交戰當在此水之陽楚師既敗秦師乘勝取上庸路西入以收漢中其勢易矣索山客翻與埴同屈九勿翻冡知隴翻易弋䜴翻】楚師大敗斬甲士八萬虜屈匄及列侯執珪七十餘人【執珪楚爵也執珪而朝者也】遂取漢中郡【自沔陽成固至新城上庸時皆漢中郡之地釋名曰郡羣也人所羣聚也黄義仲十三州記曰郡之言君也改公侯之封而言君者至尊也今郡字君在其左邑在其右君為元首邑以載民故取名於君謂之郡】楚王悉發國内兵以復襲秦【復扶又翻】戰于藍田【班志藍田縣屬京兆秦孝公置史記正義曰藍田縣在雍州東南八十里從藍田關入藍田縣時楚襲秦深入】楚師大敗韓魏聞楚之困南襲楚至鄧【鄧春秋鄧國之地班志鄧縣屬南陽郡杜預曰潁川召陵縣西有鄧城括地志曰故鄧城在豫州郾陵縣東三十五里所謂在古召陵西十里者也召讀曰邵】楚人聞之乃引兵歸割兩城以請平于秦 燕人共立太子平是為昭王【燕因肩翻】昭王於破燕之後【言燕國為齊所破已承其後也】弔死問孤與百姓同甘苦卑身厚幣以招賢者謂郭隗曰齊因孤之國亂而襲破燕孤極知燕小力少【臧文仲曰列國有凶稱孤禮也杜預曰列國諸侯無凶則稱寡人郭姓出於周之虢公世亦謂虢公為郭公隗五罪翻少始紹翻】不足以報然誠得賢士與共國以雪先王之恥【謂燕王噲破國之恥噲苦夬翻】孤之願也先生視可者得身事之郭隗曰古之人君有以千金使涓人求千里馬者【春秋以來諸侯之國有㳙人秦漢之間有中㳙師古曰㳙潔也言其在中主知潔清洒掃之事蓋王之親舊左右也應劭曰㳙人如謁者㳙古玄翻洒所賣翻掃所報翻又皆音如字】馬已死買其骨五百金而返君大怒涓人曰死馬且買之况生者乎馬今至矣不期年千里之馬至者三【期讀曰朞】今王必欲致士先從隗始况賢于隗者豈遠千里哉【言燕王若加禮於郭隗則四方之賢士聞之將不以千里為遠而來】於是昭王為隗攻築宫而師事之于是士爭趣燕【為于偽翻趣七喻翻】樂毅自魏往劇辛自趙往【劇竭戟翻劇姓辛名劇姓莫知其所自出班志北海郡有劇縣蓋其先以縣為姓也】眧王以樂毅為亞卿任以國政【為燕用樂毅破齊張本】 韓宣惠王薨子襄王倉立<br />
<br />
  四年蜀相殺蜀侯【相息亮翻蜀相陳莊也】 秦惠王使人告楚懷王請以武關之外易黔中地【武關左傳之少習地在漢弘農郡析縣西百七十里道通南陽晉太康地志曰武關當冠軍西括地志曰武關在商州上洛縣東武關之外蓋秦丹析商於之地黔音琹少始照翻冠工玩翻於音如字】楚王曰不願易地願得張儀而獻黔中地張儀聞之請行王曰楚將甘心于子【楚王以墮張儀之詐故欲甘心焉】奈何行張儀曰秦彊楚弱大王在楚不宜敢取臣且臣善其嬖臣靳尚【嬖匹計翻又卑義翻靳居焮翻姓也】靳尚得事幸姬鄭袖【鄭以國為氏袖戰國策作褏古字也】袖之言王無不聽者遂往楚王囚將殺之靳尚謂鄭袖曰秦王甚愛張儀將以上庸六縣及美女贖之【上庸春秋庸國班志上庸縣屬漢中郡史記正義上庸縣今房州宋白曰今房州竹山縣古城即漢上庸縣】王重地尊秦秦女必貴而夫人斥矣於是鄭袖日夜泣於楚王曰臣各為其主耳【為于偽翻】今殺張儀秦必大怒妾請子母俱遷江南毋為秦所魚肉也王乃赦張儀而厚禮之張儀因說楚王曰夫為從者無以異於驅羣羊而攻猛虎不格明矣【說式芮翻夫音扶從子容翻格當也劉伯莊曰格各額翻其字宜從手予據字書格擊也鬬也從木亦通】今王不事秦秦刼韓驅梁而攻楚則楚危矣秦西有巴蜀治船積粟浮岷江而下【治直之翻江水出蜀郡湔氐道之岷山故謂之岷江釋名曰江共也小流入其中所公共也】一日行五百餘里不至十日而拒扞關【徐廣曰巴郡魚復縣有扞關史記正義曰在峽州巴山縣界扞寒旦翻】扞關驚則從境以東盡城守矣【境楚境也扞關楚之西境從境以東謂扞關以東也】黔中巫郡非王之有【黔巨今翻班志巫縣屬南郡酈道元曰縣故楚之巫郡杜佑曰今歸州巴東縣是也】秦舉甲出武關則北地絶【北地楚北境之地陳蔡汝潁是也】秦兵之攻楚也危難在三月之内【難乃旦翻】而楚待諸侯之救在半歲之外夫待弱國之救忘彊秦之禍此臣所為大王患也【夫音扶為于偽翻】大王誠能聽臣臣請令秦楚長為兄弟之國無相攻伐【令力丁翻】楚王已得張儀而重出黔中地【重難也以地為重意難割弃之】乃許之張儀遂之韓說韓王曰韓地險惡山居【之如也自楚如韓也韓有宜陽成臯南盡魯陽皆山險之地說式芮翻】五穀所生非菽而麥【菽式竹翻豆也】國無二歲之食見卒不過二十萬【見卒見在之兵見賢遍翻】秦被甲百餘萬【被皮義翻】山東之士被甲蒙胄以會戰秦人捐甲徒裼以趨敵【冑今謂之兜鍪捐與專翻弃也徒徒行也裼音錫袒也趨七喻翻鍪音牟】左挈人頭右挾生虜夫戰孟賁烏獲之士以攻不服之弱國【挾戶頰翻孟賁烏獲古之勇士賁音奔】無異埀千鈞之重於鳥卵之上必無幸矣【三十斤為鈞必無幸矣言無幸而獲全之理】大王不事秦秦下甲據宜陽塞成臯【下遐稼翻塞悉則翻】則王之國分矣鴻臺之宫桑林之苑非王之有也為大王計莫如事秦以攻楚以轉禍而悦秦計無便於此者韓王許之張儀歸報秦王封以六邑號武信君復使東說齊王曰從人說大王者【復扶又翻從人合從之人也從子容翻說式芮翻】必曰齊蔽於三晉地廣民衆兵彊士勇雖有百秦將無奈齊何大王賢其說而不計其實今秦楚嫁女娶婦為昆弟之國韓獻宜陽梁效河外【河外秦蓋以河東為河外梁則以河西為河外張儀以秦言之也】趙王入朝割河間以事秦【朝直遥翻河間趙地漢文帝二年分為河間國應劭曰在兩河之間唐為瀛州】大王不事秦秦驅韓梁攻齊之南地【漢泰山城陽齊南境之地也】悉趙兵度清河指博關臨菑即墨非王之有也【博關在濟州西界之博陵史記正義曰博關在博州趙兵從貝州度清河指博關則漯河以南臨菑即墨危矣濟子禮翻漯託合翻】國一日見攻雖欲事秦不可得也齊王許張儀張儀去西說趙王曰大王收率天下以擯秦秦兵不敢出函谷關十五年【擯必刃翻事見上卷顯王三十六年】大王之威行於山東敝邑恐懼【春秋以來列國交聘行人率自稱其國曰敝邑】繕甲厲兵力田積粟愁居懾處不敢動揺【懾之涉翻怖也心伏也失常也失氣也處昌呂翻】唯大王有意督過之也【師古曰督過視責也索隱曰督者正其事而責之督過是深責其過也】今以大王之力舉巴蜀【事見愼靚王五年】并漢中【事見上三年】包兩周【元年服韓魏則包兩周矣】守白馬之津【班志白馬縣屬東郡水經註白馬津在白馬城之西北白馬城唐為滑州治所開山圖曰白馬津東可二十許里有白馬山山上常有白馬羣行悲鳴則河决馳走則山崩後人因以名縣及津按通鑑不語怪今此註亦近於怪姑以廣異間耳】秦雖僻遠然而心忿含怒之日久矣今秦有敝甲凋兵軍於澠池【敝敗惡也凋瘁也半傷也敗甲凋兵謙其辭言軍於澠池則張其勢以臨趙矣康曰澠池趙邑予據趙與韓魏接境韓有野王上黨魏有河東河内而澠池則秦地也漢為縣屬弘農郡趙安能越韓魏而有之康說非是澠莫善翻又莫忍翻】願渡河踰漳據番吾【言欲自澠池北度河又自此東踰漳水而進據番吾此亦張聲勢以臨趙也番吾即漢常山郡之蒲吾縣也劉昭注曰史記番吾君杜預云晉之蒲邑也此說非括地志番吾故城在恒州房山縣東二十里番音婆又音盤】會邯鄲之下願以甲子合戰正殷紂之事【武王伐紂癸亥陳于商郊甲子昧爽紂帥其旅若林會于牧野前徒倒戈攻其後以北遂以勝殷殺紂張儀引以懼趙其有所侮而動亦已甚矣邯鄲趙都音寒丹】謹使使臣先聞左右【使臣上疏吏翻】今楚與秦為昆弟之國而韓梁稱東藩之臣齊獻魚鹽之地【齊東瀕于海海濱廣斥魚鹽所出也此時齊未嘗獻地于秦張儀駕說以恐動趙耳】此斷趙之右肩也夫斷右肩而與人鬭【夫音扶斷丁管翻】失其黨而孤居求欲毋危得乎今秦發三將軍其一軍塞午道【索隱曰午道當在趙之東齊之西午道地名也鄭玄云一縱一横為午謂交道也塞悉則翻】告齊使度清河軍於邯鄲之東【邯鄲音寒丹】一軍軍成臯驅韓梁軍於河外【史記正義曰河外謂鄭滑州北臨河予謂此河外亦因趙而言之】一軍軍於澠池約四國為一以攻趙趙服必四分其地【言秦約齊韓魏四分趙地】臣竊為大王計莫如與秦王面相約而口相結常為兄弟之國也趙王許之【當時趙于山東最強且主從約張儀說之亦費辭矣】張儀乃北之燕【燕因肩翻】說燕王曰今趙王已入朝効河間以事秦【張儀自趙至燕借此氣勢而為是虛言以動燕耳朝直遥翻】大王不事秦秦下甲雲中九原【虞氏記曰趙自五原河曲築長城東至陰山又於河西造大城一箱崩不就乃改卜陰山河曲而禱焉晝見羣鵠遊于雲中徘徊經日見火光在其下乃即其處築城今雲中城是也予謂此亦語怪酈道元為後魏書之耳宋白曰勝州榆林縣界有雲中古城趙武侯所築秦置雲中郡唐為單于都護府班志九原縣屬五原郡漢之五原即秦之九原郡也唐為豐鹽等州之地宋白曰唐豐州治九原縣按雲中九原皆在燕之西秦自上郡朔方下兵則可至史記正義曰古雲中九原郡皆在勝州雲中郡故城在榆林東北四十里九原郡故城在勝州西界今連谷縣是下遐稼翻元為干偽翻】驅趙而攻燕則易水長城非大王之有也【水經注易水出涿郡故安縣閻鄉西山東届關城西南即燕長城門也易水又歷長城而東過范陽容城安次泉州縣南而東入海】且今時齊趙之于秦猶郡縣也不敢妄舉師以攻伐今王事秦長無齊趙之患矣【以利動之】燕王請獻常山之尾五城以和【常山即北岳恒山也漢文帝諱恒改曰常山置常山郡班志常山在常郡上曲陽縣西北其尾則燕之西南界也】張儀歸報未至咸陽秦惠王薨子武王立【索隱曰武王名蕩】武王自為太子時不說張儀【說讀曰悦】及即位羣臣多毁短之【毀短訾毁而數其短也】諸侯聞儀與秦王有隙【隙乞逆翻怨隙也釁隙也物之有鏬釁者為有隙人之與人有怨者亦為有隙】皆畔衡復合從【衡讀曰横從子容翻以此觀之此時六國之勢利在合從而從張儀連衡者畏秦而揺于儀之說耳】<br />
<br />
  五年張儀說秦武王曰為王計者東方有變【韓魏皆在秦之東說式芮翻】然後王可以多割得地也臣聞齊王甚憎臣臣之所在齊必伐之臣願乞其不肖之身以之梁【不肖謙言無所肖似也魏都大梁】齊必伐梁齊梁交兵而不能相去【言兵交不解各欲去而不能也】王以其間伐韓【間居莧翻間隙也又居閑翻中間也】入三川挾天子案圖籍此王業也【張儀欲傾周而為秦始終以此說為主挾戶頰翻】王許之齊王果伐梁梁王恐張儀曰王勿患也【言勿以為患】請令齊罷兵【令盧經翻使也下同】乃使其舍人之楚借使謂齊王曰【之往也如也不敢徑遣人使齊而往楚借使借使言借楚人以為使借子夜翻康資昔切使疏吏翻】甚矣王之託儀於秦也齊王曰何故楚使者曰張儀之去秦也固與秦王謀矣欲齊梁相攻而令秦取三川也今王果伐梁是王内罷國而外伐與國【罷讀曰疲】而信儀于秦王也齊王乃解兵還【還從宣翻又如字】張儀相魏一歲卒【相息亮翻卒子恤翻】<br />
<br />
  儀與蘇秦皆以縱横之術遊諸侯致位富貴天下爭慕效之【縱子容翻】又有魏人公孫衍者號曰犀首亦以談說顯名【說式芮翻】其餘蘇代蘇厲周最樓緩之徒紛紜徧於天下務以辯詐相高不可勝紀【姓譜曰周姓本自周平王子别封汝川人謂之周家因氏焉一云以赧王為秦所滅黜為庶人百姓稱為周家因氏焉予按商有太史周任謂為周姓所自出夫豈不可又赧王於時未滅不可謂周最出於赧王樓姓夏少康之裔周封為東樓公子孫因氏焉師古曰紛紜興作貌又物多而亂貌勝音升赧奴版翻夏戶雅翻少始照翻裔苖裔】而儀秦衍最著【著者顯著于時】孟子論之曰或謂公孫衍張儀豈不大丈夫哉一怒而諸侯懼安居而天下熄【熄滅也火滅為熄此言天下兵革之事熄滅也】孟子曰是惡足為大丈夫哉【惡音烏】君子立天下之正位行天下之正道得志則與民由之不得志則獨行其道富貴不能淫貧賤不能移威武不能詘【詘與屈同】是之謂大丈夫揚子法言曰或問儀秦學乎鬼谷術而習乎縱横言安中國者各十餘年是夫【夫音扶】曰詐人也聖人惡諸【惡烏路翻】曰孔子讀而儀秦行【謂讀孔子之言而行儀秦之事】何如也曰甚矣鳳鳴而鷙翰也【翰侯旰翻又侯安翻羽翰】然則子貢不為歟曰亂而不解子貢耻諸【太史公曰子貢一出存魯亂齊破吳彊晉而覇越温公曰考其年與事皆不合蓋六國遊說之士託為之辭太史公不加考訂因而記之楊子雲亦據太史公書發此語也說式芮翻】說而不富貴儀秦耻諸【說式芮翻】或曰儀秦其才矣乎跡不蹈已【宋咸曰蹈踐也言儀秦之才術超卓自然不踐循舊人之跡踐慈演翻】曰㫺在任人帝而難之【書舜典而難任人孔安國注云任佞也難拒也言佞人則斥遠之任音壬難乃旦翻】不以才乎才乎才非吾徒之才也<br />
<br />
  秦王使甘茂誅蜀相莊【四年蜀相殺蜀侯秦武王故誅之史記莊作壯秦紀案秦既得蜀使陳莊相蜀從莊為足】 秦王魏王會于臨晉【班志臨晉縣屬馮翊故大荔也秦取之更名臨晉應劭曰臨晉水故名臣瓚曰晉水在河之東北縣在河之西不得臨晉水舊說秦築高壘以臨晉國故曰臨晉章懷太子賢曰臨晉故城在今同州朝邑縣西南予按唐書地理志蒲州有臨晉縣宋白曰漢臨晉縣在今臨晉縣東南十八里故解城是也後魏改為北解縣周省隋分猗氏縣置桑泉縣唐天寶十二載改臨晉縣天寶之改縣必有所據則應劭臨晉水之說未可厚非秦之臨晉在河西臣瓚章懷之說皆是也更工衡翻應乙陵翻瓚藏旱翻朝直遥翻解戶買翻載祖亥翻】 趙武靈王納吳廣之女孟姚【吳姓以國為氏】有寵是為惠后【孔穎達曰后後也言其後于天子亦以廣後胤也戰國諸侯僭王亦稱其夫人為后】生子何【為立何而長子章爭國張本長知兩翻】<br />
<br />
  六年秦初置丞相以樗里疾為右丞相【應劭曰丞者承也相者助也荀悦曰秦本次國命卿二人故置左右丞相無三公官樗里疾秦惠王之弟也高誘曰疾居渭南之陰鄉其里有大樗樹故號樗里子相息亮翻樗丑於翻誘羊久翻】<br />
<br />
  七年秦魏會于應【左傳曰邘晉應韓武之穆也杜預註云應國在襄陽城父縣西予按襄陽無城父縣後漢志潁川父城縣西南有應鄉古應國也括地志曰故應城因應山為名古之應國在汝州魯山縣東三十里應乙陵翻邘音于】 秦王使甘茂約魏以伐韓而令向壽輔行甘茂令向壽還謂王曰魏聼臣矣【令盧經翻使也向式讓翻姓也姓譜向姓本自宋文公枝子向文旰旰孫戌以王父字為氏予按左傳向戌本出于宋桓公孟子為齊卿出弔于滕王使王驩為輔行趙岐註曰輔行副使也旰音榦戌音恤傳直戀翻使疏吏翻】然願王勿伐王迎甘茂於息壤而問其故【柳宗元曰地長隆然而起夷之而益高者為息壤異書有云鯀竊帝之息壤以堙洪水意者此所謂息壤蓋以地長得名長音知兩翻】對曰宜陽大縣其實郡也【杜佑曰春秋時列國相滅多以其地為縣則縣大而郡小故趙鞅曰上大夫受縣下大夫受郡至于戰國則郡大而縣小矣故甘茂曰宜陽大縣其實郡也漢官儀曰凡郡或以列國陳魯齊吳是也或以舊邑長沙丹陽是也或以山陵太山山陽是也或以川原西河河東是也或以所出金城城下得金酒泉泉味如酒豫章樟樹生庭雁門雁之所育是也或以號令夏禹合諸侯大計東冶之山會計因名會稽是也令力正翻名會古外翻】今王倍數險行千里攻之難【倍與背同音蒲昧翻數險謂函谷及三崤之險】魯人有與曾參同姓名者殺人人告其母其母織自若也【參所金翻一音七南翻】及三人告之其母投杼下機踰墻而走【杼直呂翻說文曰杼機之持緯者盖今所謂梭梭蘇禾翻】臣之賢不若曾參王之信臣又不如其母疑臣者非特三人臣恐大王之投杼也魏文侯令樂羊將而攻中山三年而拔之【事見一卷威烈王二十三年令音盧經翻將即亮翻】反而論功文侯示之謗書一篋【謗訕也毀也篋竹笥也音古頬翻】樂羊再拜稽首曰此非臣之功君之力也【稽首首至地也稽音啟】今臣覉旅之臣也【甘茂楚下蔡人故云然覉居宜翻寄也旅客也】樗里子公孫奭挾韓而議之王必聼之【樗且于翻奭施隻翻挾戶頰翻】是王欺魏王而臣受公仲侈之怨也【公仲侈韓相也】王曰寡人弗聼也請與子盟乃盟于息壤秋甘茂庶長封帥師伐宜陽【長知丈翻帥讀曰率】<br />
<br />
  八年甘茂攻宜陽五月而不拔樗里子公孫奭果爭之秦王召甘茂欲罷兵甘茂曰息壤在彼【徵前盟也】王曰有之因大悉起兵以佐甘茂【佐助也】斬首六萬遂拔宜陽韓公仲侈入謝於秦以請平【請平猶請和也】 秦武王好以力戲【好呼到翻】力士任鄙烏獲孟說皆至大官【烏姓也春秋時齊有大夫烏枝鳴姓譜孟姓魯桓公之子仲孫之胤仲孫為三桓之孟故曰孟氏任音壬說讀曰悦】八月王與孟說舉鼎絶脈而薨【脈莫獲翻脈者係絡臟腑其血理分行干支體之間人舉重而力不能勝故脈絶而死按史記甘茂傳云武王至周而卒于周盖舉鼎者舉九鼎也世家以為龍文赤鼎史記脈作臏】族孟說【族者誅夷其族】武王無子異母弟稷為質于燕【質音致燕因肩翻】國人逆而立之【逆迎也】是為昭襄王昭襄王母芈八子【楚姓也漢因秦制嫡稱皇后次稱夫人又有美人良人八子七子長使少使之號美人爵視二千石比少上造八子視千石比中更亡氏翻】楚女也寔宣太后 趙武靈王北略中山之地【略地之師速而疾杜預曰略者摠攝巡行之名也】至房子【班志房子縣屬常山郡史記正義曰房子今趙州縣宋白曰天寶元年改曰臨城】遂至代北至無窮【自代北出塞外大漠數千里故曰無窮戰國策武靈王曰昔先君襄王與代交地城境封之名曰無窮之門所以詔後而期遠也】西至河登黄華之上【史記正義曰黄華蓋黄河側之山名】與肥義謀胡服騎射以教百姓【騎奇寄翻】曰愚者所笑賢者察焉雖驅世以笑我胡地中山吾必有之遂胡服國人皆不欲公子成稱疾不朝【朝直遥翻】王使人請之曰家聼于親【親謂父母也】國聼于君今寡人作教易服而公叔不服吾恐天下議己也制國有常利民為本從政有經令行為上【令力政翻】明德先論於賤而從政先信於貴【德欲其下及故先論於賤卑賤者感其德則德廣所及可知矣法行自貴近始故先信於貴貴近者奉法則法之必行可知矣】故願慕公叔之義以成胡服之功也公子成再拜稽首曰【稽音啟】臣聞中國者聖賢之所教也禮樂之所用也遠方之所親赴也蠻夷之所則效也今王舍此而襲遠方之服【則法也舍讀曰捨襲重衣也】變古之道逆人之心臣願王孰圖之也【孰古熟字通】使者以報王自往請之【使疏吏翻】曰吾國東有齊中山【按趙都邯鄲東接于齊中山在其東北故史記趙世家載武靈王之言曰吾國東有河薄落之水與齊中山同之盖河薄落之水在趙之東與齊中山同此地險也】北有燕東胡西有樓煩秦韓之邊【史記正義曰營州之境即東胡烏丸之地林胡樓煩即嵐勝之北也班志雁門郡樓煩縣應劭注云故樓煩胡地嵐勝以南石州離石藺等七國時趙邉也與秦隔河晉洺潞澤等州皆七國時韓地趙之西邉也燕因肩翻】今無騎射之備則何以守之哉【騎奇寄翻】先時中山負齊之彊兵侵暴吾地係累吾民【先悉薦翻累力追翻】引水圍鄗微社稷之神靈則鄗幾于不守也【鄗趙邑漢光武改為高邑隋唐為柏鄉縣地唐屬趙州鄗呼各翻幾居衣翻】先君醜之【以為趙國之醜】故寡人變服騎射欲以備四境之難【難乃旦翻】報中山之怨而叔順中國之俗惡變服之名【惡烏路翻】以忘鄗事之醜非寡人之所望也公子成聽命乃賜胡服明日服而朝【朝直遥翻】于是始出胡服令而招騎射焉九年秦昭王使向壽平宜陽【平正也和也正宜陽之彊界而和其民人也向式亮翻】而使樗里子甘茂伐魏甘茂言于王以武遂復歸之韓【史記正義曰武遂本屬韓近平陽楚世家云韓先王之墓在平陽武遂去之七十里去年秦拔宜陽因涉河城武遂今復歸之韓復音如字】向壽公孫奭爭之不能得【向式讓翻】由此怨讒甘茂茂懼輟伐魏蒲阪亡去【班志蒲阪縣屬河東郡舊曰蒲應劭曰秦始皇東巡見長阪因加阪云括地志蒲阪故城在蒲州河東縣南五里阪音反】樗里子與魏講而罷兵【講和也】甘茂奔齊 趙王略中山地至寧葭【水經注衡漳水東北歷下博城西又西逕樂鄉縣故城南又東引葭水注之葭音加】西略胡地至榆中【水經注諸次水出上郡諸次山其水東逕榆林塞世又謂之榆林山即漢書所謂榆溪舊塞者自溪西去悉榆柳之藪緣歷沙陵届㫣兹縣西出故云廣長榆也王恢曰樹榆為塞謂此蘇林以為榆中在上郡非也按始皇本紀西北逐匈奴自榆中並河以東屬之陰山然榆中在金城東五十許里陰山在朔方東以此推之不得在上郡予謂蘇林之說固未為盡而道元所謂榆中在金城東五十許里亦非也據衛青取河南地案榆溪舊塞正在唐麟勝二州界其西則接古上郡之境况諸次水出上郡逕榆林塞入河則榆中在上郡之東明矣諸次水無西流至金城榆中之理夷考其故道元特以班志金城郡有榆中縣遂牽合以為說不知此一節之誤尤甚于蘇林也史記正義曰榆中勝州此河北岸也杜佑曰勝州榆林郡南即秦榆林塞】林胡王獻馬【如淳曰林胡即儋林予謂此胡種落依阻林薄因曰林胡儋都甘翻種章勇翻】歸使樓緩之秦仇液之韓王賁之楚【歸謂趙王自略中山歸也仇姓也春秋時宋有大夫仇牧液音亦之往也如也賁音奔康曰離之父翦之子予按離父翦子秦將也此王賁乃趙人康說非是將即亮翻】富丅之魏【富姓也春秋時周有大夫富辰】趙爵之齊代相趙固主胡致其兵【相息亮翻致者使之至者】 楚王與齊韓合從【楚與齊韓合從尋即倍之適足致齊韓之兵耳從子容翻】<br />
<br />
  十年彗星見【彗星出所謂埽星本類星末類彗小者數寸長或竟天見則兵起主埽除除舊布新唐史臣曰彗體無光傅日以為光故夕見則東指晨見則西指或長或短光芒所及則為災乂曰孛星彗之屬也偏指曰彗氣四出口孛孛者孛孛非常惡氣之所生災甚于彗天文書謂五星之精為妖歲星流為蒼彗熒或填星散為赤彗黄彗太白辰星變為白彗黑彗彗祥歲翻又徐醉翻又旋芮翻見賢遍翻埽所報翻傅讀曰附孛蒲内翻妖于遥翻塡讀曰鎭】 趙王伐中山取丹邱爽陽鴻之塞又取鄗石邑封龍東垣【史記正義曰丹邱邢州縣予按隋唐志邢州有内邱縣漢之中邱縣也未嘗有丹邱不知其何據爽陽鴻之塞史記作華陽䲭之塞括地志曰北岳别名曰華陽臺即常山也在定州恒陽縣北百四十里徐廣曰䲭作鴻鴻上故關今名汝城在定州唐縣東北六十里又有鴻上水出唐縣北葛洪山山接北岳恒山皆在定州界班志石邑縣屬常山郡井陘山在西括地志石邑故城在恒州鹿泉縣南三十五里封龍山一名飛龍山在恒州鹿泉縣南四十五里邑盖因山為名洪氏隸釋載後漢所立白石碑云常山國元氏縣界有封龍山東垣即漢眞定國之眞定縣漢高帝更名史記正義曰趙之東垣在恒州眞定縣南八里故常山城是也鄗呼各翻垣于元翻華戶化翻恒戶登翻䲭丑之翻陘音刑更工衡翻】中山獻四邑以和 秦宣太后異父弟曰穰侯魏冉同父弟曰華陽君芊戎王之同母弟曰高陵君涇陽君魏冉最賢【秦封穰侯于陶陶即范蠡所居陶邑孟康曰陶即定陶班志定陶縣屬濟陰郡下云封于穰與陶穰縣屬南陽郡去定陶差遠水經注曰穰侯封于穰益封于陶其免相也出之陶而卒陶有穰侯冢穰音人羊翻華陽即武王歸馬之地水經註洛水自上洛縣東北分為二水枝渠東北出為門水水東北歷楊華之山即華陽也華音戶化翻眉婢翻相息亮翻卒子恤翻冢知隴翻班志高陵縣屬馮翊涇陽縣屬安定杜佑曰京兆涇陽縣乃秦封涇陽君之地漢涇陽縣在今平凉郡界涇陽故城是也宋白曰雍州涇陽本秦舊縣與杜佑同索隱曰高陵君名顯涇陽君名悝索山客翻悝苦回翻】自惠王武王時任職用事武王薨諸弟争立唯魏冉力能立昭王【惠王即惠文王昭王即昭襄王】昭王即位以魏冉為將軍衛咸陽是歲庶長壯及大臣諸公子謀作亂【長知丈翻】魏冉誅之及惠文后皆不得良死【惠文后昭王嫡母也死于正命曰良死】悼武王后出居于魏【悼武王后即秦武王后昭王嫂也】王兄弟不善者魏冉皆滅之王少宣太后自治事任魏冉為政威震秦國【少詩照翻治直之翻為范睢間魏冉張本】<br />
<br />
  十一年秦王楚王盟于黄棘【史記正義曰黄棘盖在房襄二州予按班志南陽郡有棘陽縣應劭曰縣在棘水之陽】秦復與楚上庸【三年秦敗楚師虜屈匄取楚上庸】十二年彗星見【彗祥歲翻乂徐醉翻旋芮翻見賢遍翻】 秦取魏蒲阪晉陽封陵【晉陽史記世家作陽晉其地當在蒲阪之東風陵之西大河之陽且本晉地也故謂之陽晉蘇秦所謂衛陽晉之道盖以魏境有陽晉故在衛境者稱衛陽晉以别之括地志曰晉陽故城今名晉城在蒲州虞鄉縣西水經注函谷關直北隔河有崇阜巍然獨秀世謂之風陵酈道元所謂函谷則潼關也史記正義曰封陵在蒲州唐志河中府河東縣南有風陵關今若據括地志則晉陽亦通】又取韓武遂【九年秦歸韓武遂】 齊韓魏以楚負其從親【九年楚與齊韓合從盖即負之也從子容翻】合兵伐楚楚王使太子横為質于秦以請救【質音致】秦客卿通將兵救楚三國引兵去【將即亮翻又音如字領也】<br />
<br />
  十三年秦王魏王韓太子嬰會于臨晉韓太子至咸陽而歸秦復與魏蒲阪【阪音反去年秦取魏蒲阪】 秦大夫有私與楚太子鬭者太子殺之亡歸【楚太子質秦而亡歸復質于齊秦以為言而誘䧟其父齊乘其父出而要之以利】<br />
<br />
  十四年日有食之既 秦人取韓穰【班志穰縣屬南陽郡以時考之當屬楚然韓得潁川之地與南陽接境七國兵爭疆場之間一彼一此或者此時穰屬韓歟穰之羊翻】蜀守煇叛秦秦司馬錯往誅之【蜀守蜀郡守也史記秦紀作蜀侯華陽國志曰秦封王子煇為蜀侯蜀侯祭歸胙于王後母疾之加毒以進王大怒使司馬錯賜煇劍守音狩煇索隱音暉】 秦庶長奐會韓魏齊兵伐楚【修楚太子亡歸之怨長知丈翻】敗其師于重邱殺其將唐昧遂取重邱【唐姓本于唐堯春秋之時有二重邱衛孫蒯飲馬于重邱杜預曰曹邑諸侯同盟于重邱杜預曰齊地時楚之境皆不至此呂氏春秋曰齊令章子與韓魏攻荆荆使唐薎將兵應之夾泚而軍章子夜襲之斬薎于是水之上水經注曰泚水又西澳水注之水北出茈邱山南入于泚水意者重邱即茈邱也敗補邁翻將即亮翻昧荀子作蔑楊倞註曰與昩同語音相近當音末索隱音莫葛翻重直龍翻茈才支翻】 趙王伐中山中山君犇齊<br />
<br />
  十五年秦涇陽君為質于齊【質音致】 秦華陽君伐楚大破楚師斬首三萬殺其將景缺取楚襄城【班志襄城縣屬潁川郡有西不羮楚靈王所謂大城陳蔡不羮賦皆千乘是也將即亮翻陸德明曰不羮舊音郎漢書地理志作更字乘䋲證翻】楚王恐使太子為質于齊以請平【為楚懷王入秦而卒齊留太子以邀楚張本】 秦樗里疾卒以趙人樓緩為丞相【樗丑于翻卒子恤翻相息亮翻】 趙武靈王愛少子何欲及其生而立之【少詩照翻及其生者及其生而親見之】<br />
<br />
  十六年五月戊申大朝東宫【朝直遥翻】傳國於何王廟見禮畢出臨朝【廟見始即位而見祖廟也見賢遍翻】大夫悉為臣肥義為相國并傳王【相國之官始此秦漢因之漢魏以降其位望尊於丞相相息亮翻】武靈王自號主父【主父言爲國之主之父也一曰言其子主國而已則父也】主父欲使子治國身胡服將士大夫西北畧胡地【治直之翻將即亮翻又如字】將自雲中九原南襲咸陽于是詐自為使者入秦【使疏吏翻】欲以觀秦地形及秦王之為人秦王不知已而怪其狀甚偉非人臣之度【賓主相見交際之禮已方怪其非人臣】使人逐之主父行已脱關矣審問之乃主父也【謂已脱身出秦關也】秦人大驚 齊王魏王會于韓 秦人伐楚取八城秦王遺楚王書曰始寡人與王約為兄弟盟于黄棘【見上十一年遺于季翻】太子入質至驩也【質音致見十二年】太子陵殺寡人之重臣不謝而亡去【見十三年】寡人誠不勝怒【勝音升】使兵侵君王之邊【謂戰重邱取襄城】今聞君王乃令太子質于齊以求平【見十五年】寡人與楚接境婚姻相親【妻父曰婚壻父曰姻字書婚昏也禮娶以昏時婦人隂也故曰婚壻家女之所因故曰姻字林婚婦家姻壻家賈公彦曰各據男女身則男曰昏女曰姻若以親言之則女之父曰婚壻之父曰姻予按張儀言秦楚嫁女娶婦為昆弟之國考之于史自赧王四年至是年秦楚未嘗嫁娶也至十九年楚懷王死于秦至二十三年楚襄王逆婦于秦蓋先己約親其後襄王終喪始逆婦成婚姻】而今秦楚不驩則無以令諸侯【令力政翻】寡人願與君王會武關面相約結盟而去寡人之願也楚王患之欲往恐見欺欲不往恐秦益怒昭睢曰毋行而發兵自守耳【睢息遺翻又七余翻】秦虎狼也有并諸侯之心不可信也懷王之子蘭勸王行王乃入秦秦王令一將軍詐為王伏兵武關楚王至則閉關刼之與俱西至咸陽朝章臺如藩臣禮【朝直遥翻秦章臺宫在渭南漢張敞走馬章臺街孟康曰在長安中臣瓚曰街在章臺下漢長安在渭南以此言之章臺宫在渭南明矣瓚藏旱翻】要以割巫黔中郡楚王欲盟秦王欲先得地楚王怒曰秦詐我而又彊要我以地【要一遥翻黔其今翻彊其良翻又其兩翻】因不復許秦人留之【復扶又翻】楚大臣患之乃相與謀曰吾王在秦不得還要以割地而太子為質于齊【還從宣翻又音如字要一遥翻質音致】齊秦合謀則楚無國矣欲立王子之在國者昭睢曰王與太子俱困於諸侯今又倍王命而立其庶子不宜【睢息隨翻倍蒲妹翻】乃詐赴于齊【詐言楚王薨而請太子還王楚】齊湣王召羣臣謀之或曰不若留太子以求楚之淮北【湣讀曰閔楚滅陳蔡封畛于汝滅越取吳故地并有古徐夷之地皆在淮北即楚所謂下東國】齊相曰不可郢中立王【郢楚都班志南郡江陵縣故楚郢都楚文王自丹陽徙此後九世平王城之又後十世秦拔之東徙夀春亦名曰郢水經江水東逕江陵縣故城南又東逕郢城南注云今江陵城楚船官地春秋之者宫郢城即子囊遺言所城者劉昫曰故楚都之郢城今江陵縣北十五里紀南城是也相息亮翻】是吾抱空質而行不義於天下也【質音致】其人曰不然郢中立王因與其新王市曰予我下東國吾為王殺太子【市謂相要以利如市道也予讀曰與為于偽翻】不然將與三國共立之【三國謂齊韓魏】齊王卒用其相計而歸楚太子【卒子恤翻】楚人立之 秦王聞孟嘗君之賢使涇陽君為質於齊以請孟嘗君來入秦秦王以為丞相【質音致】<br />
<br />
  十七年或謂秦王曰孟嘗君相秦必先齊而後秦【先後皆去聲】秦其危哉秦王乃以樓緩為相囚孟嘗君欲殺之孟嘗君使人求解於秦王幸姬姬曰願得君狐白裘【狐白裘緝狐掖之皮為之所謂千金之裘非一狐之掖者也】孟嘗君有狐白裘已獻之秦王無以應姬求客有善為狗盗者入秦藏中【物之所藏曰藏音徂浪翻】盗狐白裘以獻姬姬乃為之言于王而遣之【為于偽翻】王後悔使追之孟嘗君至關關法雞鳴而出客時尚蚤【蚤古早字通】追者將及客有善為鷄鳴者野鷄聞之皆鳴孟嘗君乃得脱歸 楚人告于秦曰賴社稷神靈國有王矣秦王怒發兵出武關撃楚斬首五萬取十六城 趙王封其弟為平原君【班志平原縣屬平原郡勝封于東武城號平原君非封于平原也東武城屬清河郡杜佑曰今貝州武城縣是也盖定襄有武城時同屬趙故此加東也】平原君好士【好呼到翻】食客嘗數千人有公孫龍者善為堅白同異之辨【漢書藝文志公孫龍子十四篇註云即為堅白同異之辨者成元英莊子疏云公孫龍著守白論行于世堅白即守白也言堅執其說如墨子墨守之義自堅白之論起辨者互執是非不勝異說公孫龍能合衆異而為同故謂之同異史記注曰晉太康地記云汝南西平縣有龍淵水可用淬刀劒極堅利故有堅白之論云黄所以為堅也白所以為利也或曰黄所以為不堅白所以為不利二說未知孰是勝音升淬取内翻】平原君客之孔穿自魯適趙【按孔叢子孔穿孔子之後孫愐曰孔姓殷湯之後本自帝嚳元妃簡狄吞乙卵生契賜姓子氏至湯以其祖感乙而生故名履字天乙後代以子加乙始為孔氏至宋孔父遭華督之難其子奔魯故孔子生于魯愐彌兖翻嚳苦沃翻華戶化翻難乃旦翻】與公孫龍論臧三耳【三耳如莊子所載鷄三足之說莊子疏謂數起于一一與一為二二與一為三三名雖立寔無定體故鷄可以為三足則兩耳三耳其說亦猶是耳一說耳主聽兩耳形也兼聽而言可得為三臧臧獲之臧臧獲奴婢也】龍甚辯析【辯别也析分也言分别甚精微也】子高弗應俄而辭出明日復見平原君【子高孔穿字也復扶又翻】平原君曰疇昔公孫之言信辯也【毛晃曰疇曩也昔夕也疇昔曩夕也】先生以為何如對曰然幾能令臧三耳矣【毛晃曰然如也是也語决辭幾居依翻令使也音力丁翻】雖然寔難僕願得又問于君今謂三耳甚難而寔非也謂兩耳甚易而寔是也不知君將從易而是者乎其亦從難而非者乎平原君無以應明日謂公孫龍曰公無復與孔子高辯事也【易弋豉翻】其人理勝于辭公辭勝于理終必受詘鄒衍過趙【過古禾翻】平原君使與公孫龍論白馬非馬之說【此亦莊子所謂狗非犬之說疏云狗之與犬一實兩名名實合則此為狗彼為犬名實離則狗異于犬又墨子曰狗犬也然狗非狗犬也大指與白馬非馬之說同】鄒子曰不可夫辯者别殊類使不相害序異端使不相亂抒意通指【夫音扶别彼列翻索隱曰抒音墅抒者舒也乂常恕翻康曰亦音舒】明其所謂使人與知焉不務相迷也【與音如字又讀曰預】故勝者不失其所守不勝者得其所求【辯以求是辨雖不勝而得審其是所謂得其所求也】若是故辯可為也及至煩文以相假飾辭以相惇【惇都昆翻廹也詆也誰何也】巧譬以相移引人使不得及其意如此害大道夫繳紛爭言而競後息【索隱繳音糾康吉弔切非言其言戾紛然而爭欲人先屈務在人後方止也】不能無害君子衍不為也座皆稱善【言一座之人皆稱衍言為善】公孫龍由是遂絀【通鑑書此言小辨終不足破大道絀音敕律翻說文曰絀貶下也又讀與屈同】<br />
<br />
  資治通鑑卷三  <br>
   </div> 

<script src="/search/ajaxskft.js"> </script>
 <div class="clear"></div>
<br>
<br>
 <!-- a.d-->

 <!--
<div class="info_share">
</div> 
-->
 <!--info_share--></div>   <!-- end info_content-->
  </div> <!-- end l-->

<div class="r">   <!--r-->



<div class="sidebar"  style="margin-bottom:2px;">

 
<div class="sidebar_title">工具类大全</div>
<div class="sidebar_info">
<strong><a href="http://www.guoxuedashi.com/lsditu/" target="_blank">历史地图</a></strong>  
<a href="http://www.880114.com/" target="_blank">英语宝典</a>  
<a href="http://www.guoxuedashi.com/13jing/" target="_blank">十三经检索</a> 
<br><strong><a href="http://www.guoxuedashi.com/gjtsjc/" target="_blank">古今图书集成</a></strong> 
<a href="http://www.guoxuedashi.com/duilian/" target="_blank">对联大全</a> <strong><a href="http://www.guoxuedashi.com/xiangxingzi/" target="_blank">象形文字典</a></strong> 

<br><a href="http://www.guoxuedashi.com/zixing/yanbian/">字形演变</a>  <strong><a href="http://www.guoxuemi.com/hafo/" target="_blank">哈佛燕京中文善本特藏</a></strong>
<br><strong><a href="http://www.guoxuedashi.com/csfz/" target="_blank">丛书&方志检索器</a></strong> <a href="http://www.guoxuedashi.com/yqjyy/" target="_blank">一切经音义</a>  

<br><strong><a href="http://www.guoxuedashi.com/jiapu/" target="_blank">家谱族谱查询</a></strong>  <strong><a href="http://shufa.guoxuedashi.com/sfzitie/" target="_blank">书法字帖欣赏</a></strong> 
<br>

</div>
</div>


<div class="sidebar" style="margin-bottom:0px;">

<font style="font-size:22px;line-height:32px">QQ交流群9:489193090</font>


<div class="sidebar_title">手机APP 扫描或点击</div>
<div class="sidebar_info">
<table>
<tr>
	<td width=160><a href="http://m.guoxuedashi.com/app/" target="_blank"><img src="/img/gxds-sj.png" width="140"  border="0" alt="国学大师手机版"></a></td>
	<td>
<a href="http://www.guoxuedashi.com/download/" target="_blank">app软件下载专区</a><br>
<a href="http://www.guoxuedashi.com/download/gxds.php" target="_blank">《国学大师》下载</a><br>
<a href="http://www.guoxuedashi.com/download/kxzd.php" target="_blank">《汉字宝典》下载</a><br>
<a href="http://www.guoxuedashi.com/download/scqbd.php" target="_blank">《诗词曲宝典》下载</a><br>
<a href="http://www.guoxuedashi.com/SiKuQuanShu/skqs.php" target="_blank">《四库全书》下载</a><br>
</td>
</tr>
</table>

</div>
</div>


<div class="sidebar2">
<center>


</center>
</div>

<div class="sidebar"  style="margin-bottom:2px;">
<div class="sidebar_title">网站使用教程</div>
<div class="sidebar_info">
<a href="http://www.guoxuedashi.com/help/gjsearch.php" target="_blank">如何在国学大师网下载古籍?</a><br>
<a href="http://www.guoxuedashi.com/zidian/bujian/bjjc.php" target="_blank">如何使用部件查字法快速查字?</a><br>
<a href="http://www.guoxuedashi.com/search/sjc.php" target="_blank">如何在指定的书籍中全文检索?</a><br>
<a href="http://www.guoxuedashi.com/search/skjc.php" target="_blank">如何找到一句话在《四库全书》哪一页?</a><br>
</div>
</div>


<div class="sidebar">
<div class="sidebar_title">热门书籍</div>
<div class="sidebar_info">
<a href="/so.php?sokey=%E8%B5%84%E6%B2%BB%E9%80%9A%E9%89%B4&kt=1">资治通鉴</a> <a href="/24shi/"><strong>二十四史</strong></a>&nbsp; <a href="/a2694/">野史</a>&nbsp; <a href="/SiKuQuanShu/"><strong>四库全书</strong></a>&nbsp;<a href="http://www.guoxuedashi.com/SiKuQuanShu/fanti/">繁体</a>
<br><a href="/so.php?sokey=%E7%BA%A2%E6%A5%BC%E6%A2%A6&kt=1">红楼梦</a> <a href="/a/1858x/">三国演义</a> <a href="/a/1038k/">水浒传</a> <a href="/a/1046t/">西游记</a> <a href="/a/1914o/">封神演义</a>
<br>
<a href="http://www.guoxuedashi.com/so.php?sokeygx=%E4%B8%87%E6%9C%89%E6%96%87%E5%BA%93&submit=&kt=1">万有文库</a> <a href="/a/780t/">古文观止</a> <a href="/a/1024l/">文心雕龙</a> <a href="/a/1704n/">全唐诗</a> <a href="/a/1705h/">全宋词</a>
<br><a href="http://www.guoxuedashi.com/so.php?sokeygx=%E7%99%BE%E8%A1%B2%E6%9C%AC%E4%BA%8C%E5%8D%81%E5%9B%9B%E5%8F%B2&submit=&kt=1"><strong>百衲本二十四史</strong></a>  <a href="http://www.guoxuedashi.com/so.php?sokeygx=%E5%8F%A4%E4%BB%8A%E5%9B%BE%E4%B9%A6%E9%9B%86%E6%88%90&submit=&kt=1"><strong>古今图书集成</strong></a>
<br>

<a href="http://www.guoxuedashi.com/so.php?sokeygx=%E4%B8%9B%E4%B9%A6%E9%9B%86%E6%88%90&submit=&kt=1">丛书集成</a> 
<a href="http://www.guoxuedashi.com/so.php?sokeygx=%E5%9B%9B%E9%83%A8%E4%B8%9B%E5%88%8A&submit=&kt=1"><strong>四部丛刊</strong></a>  
<a href="http://www.guoxuedashi.com/so.php?sokeygx=%E8%AF%B4%E6%96%87%E8%A7%A3%E5%AD%97&submit=&kt=1">說文解字</a> <a href="http://www.guoxuedashi.com/so.php?sokeygx=%E5%85%A8%E4%B8%8A%E5%8F%A4&submit=&kt=1">三国六朝文</a>
<br><a href="http://www.guoxuedashi.com/so.php?sokeytm=%E6%97%A5%E6%9C%AC%E5%86%85%E9%98%81%E6%96%87%E5%BA%93&submit=&kt=1"><strong>日本内阁文库</strong></a> <a href="http://www.guoxuedashi.com/so.php?sokeytm=%E5%9B%BD%E5%9B%BE%E6%96%B9%E5%BF%97%E5%90%88%E9%9B%86&ka=100&submit=">国图方志合集</a> <a href="http://www.guoxuedashi.com/so.php?sokeytm=%E5%90%84%E5%9C%B0%E6%96%B9%E5%BF%97&submit=&kt=1"><strong>各地方志</strong></a>

</div>
</div>


<div class="sidebar2">
<center>

</center>
</div>
<div class="sidebar greenbar">
<div class="sidebar_title green">四库全书</div>
<div class="sidebar_info">

《四库全书》是中国古代最大的丛书,编撰于乾隆年间,由纪昀等360多位高官、学者编撰,3800多人抄写,费时十三年编成。丛书分经、史、子、集四部,故名四库。共有3500多种书,7.9万卷,3.6万册,约8亿字,基本上囊括了古代所有图书,故称“全书”。<a href="http://www.guoxuedashi.com/SiKuQuanShu/">详细>>
</a>

</div> 
</div>

</div>  <!--end r-->

</div>
<!-- 内容区END --> 

<!-- 页脚开始 -->
<div class="shh">

</div>

<div class="w1180" style="margin-top:8px;">
<center><script src="http://www.guoxuedashi.com/img/plus.php?id=3"></script></center>
</div>
<div class="w1180 foot">
<a href="/b/thanks.php">特别致谢</a> | <a href="javascript:window.external.AddFavorite(document.location.href,document.title);">收藏本站</a> | <a href="#">欢迎投稿</a> | <a href="http://www.guoxuedashi.com/forum/">意见建议</a> | <a href="http://www.guoxuemi.com/">国学迷</a> | <a href="http://www.shuowen.net/">说文网</a><script language="javascript" type="text/javascript" src="https://js.users.51.la/17753172.js"></script><br />
  Copyright &copy; 国学大师 古典图书集成 All Rights Reserved.<br>
  
  <span style="font-size:14px">免责声明:本站非营利性站点,以方便网友为主,仅供学习研究。<br>内容由热心网友提供和网上收集,不保留版权。若侵犯了您的权益,来信即刪。scp168@qq.com</span>
  <br />
ICP证:<a href="http://www.beian.miit.gov.cn/" target="_blank">鲁ICP备19060063号</a></div>
<!-- 页脚END --> 
<script src="http://www.guoxuedashi.com/img/plus.php?id=22"></script>
<script src="http://www.guoxuedashi.com/img/tongji.js"></script>

</body>
</html>
