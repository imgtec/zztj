周紀三 (起重光赤奮若(辛丑),盡昭陽大淵獻(癸亥),凡二十有三年。)

  愼靚王|{
	諱定,顯王之子也。此複諡也。以諡法言之,諡法:敏以敬曰慎;柔德安眾曰靖。靚,疾正翻。
	}

  元年(辛醜,西元前三二零年)

  ①衛更貶號曰君。|{
	顯王二十三年,衛已貶號曰侯;介於秦、魏之間,國日以削弱,因更貶其號曰君。更,居孟翻。貶,悲檢翻。
	}

  二年(壬寅,西元前三一九年)

  ①秦伐韓,取鄢。|{
	春秋"晉敗楚師于鄢陵",既此鄢也。班志作"傿陵",屬潁川郡。鄢,音謁晚翻,又於建翻,師古音偃。史記正義曰:許州鄢陵縣西北十五裡有鄢陵古城。
	}

  ②魏惠王薨,子襄王立。|{
	索隱曰:系本曰:襄王,名嗣。今按系本即世本,司馬貞避唐諱,改"世"為"系"。考異曰;史記魏世家云:惠王三十六年卒,子襄王立。襄王十六年卒,子哀王立。哀王二十三年卒,子昭王立。六國表,惠王元辛亥,終丙戌;襄王元丁亥,終壬寅;哀王元癸卯,終乙丑。按杜預春秋後序云:太康初,汲縣有發舊塚者,大得古書,其紀年篇起自夏、殷、周,皆三代王事,無諸國別也;惟特記晉國,起自殤叔,次文侯、昭侯,以至曲沃莊伯,皆用夏正,編年相次;晉國滅,獨記魏事,下至魏哀王之二十年:蓋魏國之史記也。哀王于史記,襄王之子,惠王之孫也。古書紀年篇,惠王三十六年改元,從一年始,至十六年而稱惠成王卒,即惠王也;疑史記誤分惠成之世以為後王年也。哀王二十三年乃卒,故特不稱諡,謂之"今王"。裴駰魏世家注引和嶠云:紀年起自黃帝,終於魏之今王;今王者,魏惠成王子。按太史公書,惠成王但言惠王,惠王子曰襄王,襄王子曰哀王。惠王三十六年卒,襄王立十六年卒,並惠、襄為五十二年。今按古文惠成王立三十六年,改元,稱一年,改元後十七年卒。太史公書為誤分惠成之世以為二王之年數也。世本,惠王生襄王而無哀王,然則"今王"者,魏襄王也。彼既魏史,所書魏事必得其真,今從之。
	}

  孟子入見而出,語人曰:"望之不似人君,就之而不見所畏焉。|{
	入見,賢遍翻。語,牛倨翻。
	}
卒然問曰:『天下惡乎定?』|{
	卒,七沒翻。惡,音烏,何也。
	}
吾對曰:『定於一。』『孰能一之?』|{
	此一語,魏襄王以問孟子。
	}
對曰:『不嗜殺人者能一之。』『孰能與之?』|{
	此語亦襄王問。
	}
對曰:『天下莫不與也。王知夫苗乎?|{
	夫,音扶。
	}
七、八月之間旱,則苗槁矣。|{
	孟子此言,用周正也。周七、八月,夏五、六月也。槁,音考,乾枯也。夏,戶雅翻。幹,音幹。
	}
天油然作云,沛然下雨,則苗浡然興之矣。|{
	油然,云盛貌。沛然,雨盛貌。浡然,興起貌。沛,普蓋翻。浡,音勃。
	}
其如是,孰能禦之?』"

  三年(癸卯,西元前三一八年)

  ①楚、趙、魏、韓、燕同伐秦,攻函谷關。|{
	燕,因肩翻,注已見上。宋白曰:函谷關在弘農。地理志注云:謂道形如函,孫卿子所謂"秦有松柏之塞"是也。
	}
秦人出兵逆之,五國之師皆敗走。

  ②宋初稱王。

  四年(甲辰,西元前三一七年)

  ①秦敗韓師于修魚,斬首八萬級,虜其將〈魚叟〉、申差於濁澤。|{
	敗,補邁翻。索隱曰:修魚,地名。〈魚叟〉、申差,二將名。索,山客翻。將,即亮翻。〈魚叟〉,音瘦,又疏鳩翻。"濁澤",年表作"觀澤"。括地志,觀澤在魏州頓丘縣東十八裡。
	}
諸侯振恐。

  ②齊大夫與蘇秦爭寵,使人刺秦,殺之。|{
	刺,七亦翻。
	}

  ③張儀說魏襄王曰:"梁地方不至千里,卒不過三十萬,地四平,無名山大川之限,卒戍楚、韓、齊、趙之境,|{
	戍,舂遇翻;字從"人",從"戈",人荷戈,所以戍也。梁地南接楚,西接韓,東接齊,北接趙。
	}
守亭、障者不過【章:十二行本"過"作"下";乙十一行本同;孔本同;張校同。】十萬,|{
	說文:亭,民所安定也,道路所舍也。障,堡障也,隔也,塞也,所以隔塞敵人也。
	}
梁之地勢固戰場也。夫諸侯之約從,盟于洹水之上,結為兄弟以相堅也。|{
	事見上卷顯王二十六年。夫,音扶。從,子容翻。洹,於元翻。
	}
今親兄弟同父母,尚有爭錢財相殺傷,而欲恃反覆蘇秦之餘謀,其不可成亦明矣?大王不事秦,秦下兵攻河外,據卷衍、酸棗,|{
	後漢志:卷縣屬河南郡,酸棗縣屬陳留郡。水經注:河水逕卷縣北,又東至酸棗、延津,二邑皆河津之要也。卷,逵員翻。衍,以善翻。
	}
劫衛,取陽晉,則趙不南,趙不南則梁不北,梁不北則從道絕,從道絕則大王之國欲毋危不可得也。|{
	從道,謂約從之路也。從,子容翻。
	}
故願大王審定計議,且賜骸骨。" |{
	人臣委身以事君,身非我之有矣,故於其乞退也,謂之乞骸骨。骸,戶皆翻。
	}
魏王乃倍從約,|{
	倍,蒲妹翻。
	}
而因儀以請成于秦。張儀歸,複相秦。|{
	儀罷秦相相魏,見上卷顯王四十七年。相,息亮翻。
	}

  ④魯景公薨,子平公旅立。|{
	諡法:由義而濟曰景;布義行剛曰景。
	}

  五年(乙巳,西元前三一六年)

  ①巴、蜀相攻擊,|{
	巴,春秋巴子之國。蜀,蠶叢、魚鳧之後。華陽國志曰:昔蜀王封其弟于漢中,號曰苴侯,因命其邑曰葭萌。苴侯與巴王為好。後巴與蜀為讎,蜀王怒,伐苴侯,苴侯奔巴。巴求救于秦,秦伐蜀,蜀王敗死。秦滅蜀,因遂滅巴、苴,置巴、蜀二郡。史記正義曰:巴子城在合州石鏡縣南五裡,故墊江縣也。宋白曰:巴子後理閬中。揚雄蜀本紀曰:蜀王本治廣都樊鄉,徙居成都。華,戶化翻。苴,子餘翻。葭,音家。萌,謨耕翻。墊,音疊。閬,音浪。
	}
俱告急于秦。秦惠王欲伐蜀,以為道險陿難至,|{
	陿與狹同。漢書趙充國傳注:山附而夾水曰陿。
	}
而韓又來侵,猶豫未能決。|{
	說文:猶,玃屬,居山中;聞人聲,豫登木,無人乃下。世謂不決曰猶豫。一說,隴西謂犬子為猶,犬導人行,忽先忽後,故曰猶豫。又一說,猶豫,犬也,犬為人行,好先行,卻住以俟其人,百步之間,如是者數四;先者,豫也,遂曰猶豫。猶,夷周翻,又餘救翻。玃,厥縛翻。為,於偽翻。好,呼到翻。
	}
司馬錯請伐蜀。|{
	史記:重、黎之後,至周宣王時為程伯休父,為司馬氏。錯,七各翻,又七故翻。重,直龍翻。父,音甫。
	}
張儀曰:"不如伐韓。"王曰:"請聞其說。"儀曰:"親魏,善楚,下兵三川,攻新城、宜陽,|{
	伊水、洛水、河水為三川。秦後置三川郡,漢改為河南郡。班志,新城縣屬河南郡。括地志:洛州伊闕縣本漢新城縣,在州南七十裡。隋文帝改新城為伊闕,取伊闕山為名。
	}
以臨二周之郊,|{
	周分為東、西,故曰二周。
	}
據九鼎,|{
	昔夏禹貢金九牧,鑄鼎象物,桀有昏德,鼎遷于商;商紂暴虐,鼎遷于周;成王定鼎於郟鄏,寶之,以為三代共器。夏,戶雅翻。郟,音夾。鄏,音辱。
	}
按圖籍,|{
	圖籍,謂天下之圖籍,周官職方氏所掌是也。
	}
挾天子以令于天下,天下莫敢不聽,此王業也。臣聞爭名者于朝,爭利者于市。今三川、周室,天下之朝市也,|{
	朝,直遙翻。周禮大宗伯注云:朝,猶朝也,欲其來之早也。人君昕旦親政貴早,聲轉為朝。猶朝,陟遙翻。
	}
而王不爭焉,顧爭于戎翟,去王業遠矣。"|{
	翟,與狄同。
	}
司馬錯曰:"不然。臣聞欲富國者務廣其地,欲強兵者務富其民,欲王者務博其德,|{
	欲王,於況翻,又如字。
	}
三資者備而王隨之矣。今王地小民貧,故臣願先從事于易。|{
	易,弋豉翻。
	}
夫蜀,西僻之國而戎翟之長也,|{
	夫,音扶。長,知丈翻。
	}
有桀、紂之亂;以秦攻之,譬如使豺狼逐群羊;|{
	豺,徂齋翻。
	}
得其地足以廣國,取其財足以富民,繕兵不傷眾而彼已服焉。|{
	彼,謂蜀也。
	}
拔一國而天下不以為暴,利盡四【章:十二行本"四"作"西";乙十一行本同。】海而天下不以為貪,是我一舉而名實附也,而又有禁暴止亂之名。今攻韓,劫天子,惡名也,而未必利也;又有不義之名,而攻天下所不欲,危矣。臣請論其故:周,天下之宗室也。|{
	周室為天下所宗,故謂之宗室。
	}
齊,韓之與國也。|{
	鄰國相親睦者,謂之與國。
	}
周自知失九鼎,韓自知亡三川,將二國並力合謀,以因乎齊、趙而求解乎楚、魏,|{
	並,必正翻。求解者,先與之構怨隙而今求和解也。
	}
以鼎與楚,以地與魏,王弗能止也。此臣之所謂危也。不如伐蜀完。"|{
	完,全也。言以兵伐蜀,十全必取也。
	}
王從錯計,|{
	錯,七各翻,又七故翻。
	}
起兵伐蜀:十月取之。|{
	取,言易也。易,弋豉翻。
	}
貶蜀王,更號為侯;|{
	貶,悲檢翻。更,工衡翻。
	}
而使陳莊相蜀。|{
	相,息亮翻。
	}
蜀既屬秦,秦以益強,富厚,輕諸侯。

  ②蘇秦既死,|{
	三年,蘇秦死于齊。
	}
秦弟代、厲亦以遊說顯於諸侯。|{
	說,式芮翻。
	}
燕相子之與蘇代婚,欲得燕權。蘇代使於齊而還,|{
	燕,因肩翻。相,息亮翻。使,疏吏翻。還,從宣翻。
	}
燕王噲問曰:"齊王其霸乎?"|{
	噲,苦夬翻。
	}
對曰:"不能。"王曰:"何故?"對曰:"不信其臣。"於是燕王專任子之。鹿毛壽謂燕王曰:|{
	劉伯莊曰:鹿毛壽,人姓名;又曰潘壽。春秋後語作"唐毛壽"。徐廣曰:一作"厝毛"。如徐廣一作之說,當作"厝"。厝,音秦昔翻。清河有厝縣。
	}
"人之謂堯賢者,以其能讓天下也。今王以國讓子之,是王與堯同名也。"燕王因屬國於子之,|{
	屬,之欲翻,付也,托也。
	}
子之大重。或曰:"禹薦益而以啟人為吏,|{
	孟子曰:禹薦益于天,禹崩,天下之人不之益而之啟,曰:"吾君之子也。"索隱曰:人,猶臣也。謂以啟臣為益吏。索,山客翻。
	}
及老而以啓為不足任天下,|{
	任,音壬。
	}
傳之於益。啓與交黨攻益,奪之,天下謂禹名傳天下於益而實令啓自取之。|{
	按或曰一段事,與師春紀伊尹放太甲,潜出自桐,殺伊尹,事頗相類,古書雜記固多也。
	}
今王言屬國於子之而吏無非太子人者,是名屬子之而實太子用事也。"王因收印綬,自三百石吏已上而效之子之。|{
	後漢書輿服志曰:三王俗化雕文,詐偽漸生,始有印綬,以檢奸萌。周禮掌節有璽節,鄭氏注云:今之印章也。綬,組綬。古者佩玉以綬貫之。漢承秦制,乘輿璽綬;諸王以下,印以金、銀、銅為差,綬以赤、紫、青、黑、黃為差。印,信也,刻文合信也。綬,受也,轉相授受也。三百石吏,銅印,黑綬或黃綬。王制:諸侯大國之卿,食祿以田計之,為三十二夫之入。戰國之卿,食祿萬鐘,其僭差不度甚矣。漢制:三公秩萬石,至於鬥食佐吏,凡十六等。三百石吏,第十等,奉月四十斛。綬,音受。璽,斯氏翻。組,祖五翻。乘,繩證翻。奉,與俸同,音扶用翻。
	}
子之南面行王事,而噲老,不聽政,顧為臣,|{
	顧,反也。噲,苦夬翻。
	}
國事皆决于子之。|{
	為後燕亂張本。
	}

  六年(丙午,西元前三一五年)

  ①王崩,子赧王延立。

  赧王上|{
	劉伯莊曰:赧,慚之甚也。輕微危弱,寄住東、西,足為慚赧,故號之曰赧;諡法本無赧字也。赧,音奴版翻。
	}

  元年(丁未,西元前三一四年)

  ①秦人侵義渠,得二十五城。|{
	義渠,戎國名。按上卷顯王四十二年,秦縣義渠,以其君為臣,是已得義渠矣。今又侵得二十五城,何也?蓋先此秦以義渠為縣,君為臣,雖臣屬於秦,義渠之國未滅也,秦稍蠶食侵其地。今得二十五城,義渠之國所餘無幾矣。蓋秦兼併諸侯,不盡其國不止也。左傳:有鐘鼓曰伐,無曰侵。谷梁傳:苞人民、驅牛馬曰侵。斬樹木、壞宮室曰伐。無幾,居豈翻。傳,直戀翻。壞,音怪。
	}

  ②魏人叛秦,秦人伐魏,取曲沃而歸其人。又敗韓於岸門,|{
	續漢志,潁川郡穎隂縣有岸亭。註引徐廣云;岸亭,即岸門。括地志:岸門在今許州長社縣東北二十八裡,今名長武亭。敗,補邁翻。
	}
韓太子倉入質于秦以和。|{
	質,音致。
	}

  ③燕子之為王三年,國内大亂。將軍市被與太子平謀攻子之。齊王令人謂太子曰:|{
	令,廬經翻。
	}
“寡人聞太子將飭君臣之義,

明父子之位,寡人之國【章:十二行本 "國"下有"雖小"二字;乙十一行本同;孔本同;退齋校同。】唯太子所以令之。"|{
	飭,整也,修也,治也。治,直之翻。飭君臣之義,言太子平將治子之僭王之罪也。明父子之位,言太子平當繼其父噲之位也。令,力政翻,命令也,號令也。
	}
太子因要黨聚衆,|{
	要,一遥翻,要結也。
	}
使市被攻子之,不克。市被反攻太子。搆難數月,|{
	難,乃旦翻。
	}
死者數萬人,百姓恫恐。|{
	恫,它紅翻,痛也。
	}
齊王令章子將五都之兵,因北地之眾以伐燕。|{
	將,即亮翻,又音如字,領也。邑有先王之廟曰都。或曰:都,邑之大者。北地,齊之北境也,蓋漢千乘、清河、勃海之地。燕,因肩翻;下同。乘,繩證翻。
	}
燕士卒不戰,城門不閉。齊人取子之,醢之,|{
	醢,呼改翻,肉醬也。
	}
遂殺燕王噲。|{
	噲,苦夬翻。
	}

  齊王問孟子曰:"或謂寡人勿取燕,或謂寡人取之。以萬乘之國伐萬乘之國,|{
	古者天子之地方千里,出兵車萬乘。七國兼併以強大,于時皆為萬乘之國。乘,繩證翻。
	}
五旬而舉之,|{
	十日為旬,五旬,五十日。
	}
人力不至於此;不取,必有天殃。|{
	殃,咎也,禍也。
	}
取之何如?"孟子對曰:"取之而燕民悅則取之,古之人有行之者,武王是也。取之而燕民不悅則勿取,古之人有行之者,文王是也。以萬乘之國伐萬乘之國,簞食壺漿以迎王師,|{
	簞,竹器也;圓曰簞,方曰笥。簞,音丹。食,祥吏翻,熟食也。漿,水也,酢漿也。笥,相吏翻。酢,倉故翻。
	}
豈有他哉?避水火也。如水益深,如火益熱,亦運而已矣!"|{
	運,轉也。言燕之民將轉而之他國也。
	}

  諸侯將謀救燕,齊王謂孟子曰:"諸侯多謀伐寡人者,何以待之?"對曰:"臣聞七十裡為政於天下者,湯是也;未聞以千里畏人者也。書曰:『徯我後,後來其蘇。』|{
	書仲虺之誥之辭。徯,戶禮翻,待也。後,君也。
	}
今燕虐其民,王往而征之,民以為將拯己於水火之中也,|{
	拯,上舉也,援也,救也,助也,音之淩翻。
	}
簞食壺漿以迎王師。若殺其父兄,系累其子弟,|{
	趙岐曰:系累,縛結 也。系戶計翻。累,力追翻。
	}
毀其宗廟,遷其重器。|{
	重器,國之鎮寶。
	}
如之何其可也!天下固畏齊之強也,今又倍地|{
	齊並燕則地倍其舊。燕,因肩翻。
	}
而不行仁政,是動天下之兵也。王速出令。反其旄倪,|{
	令,力政翻。趙岐曰:旄,老旄;倪,弱小。陸德明曰:倪,謂翳倪小兒也。記曲禮曰:八十、九十曰耄,注云:耄,惛忘也。旄,讀曰耄。倪,五兮翻。翳,與繄同,音煙兮翻。
	}
止其重器,謀于燕眾,置君而後去之,則猶可及止也。"齊王不聽。

  已而燕人叛。|{
	是時燕人雖未立太子平,固已相帥叛齊矣。
	}
王曰:"吾甚慚於孟子。"陳賈曰:"王無患焉。"乃見孟子,曰:"周公何人也?"曰:"古聖人也。"陳賈曰:"周公使管叔監商,|{
	古殷,商通稱,商者,以始封為國號,殷者,以都亳為國號。按孟子,陳賈只云"監殷",今通鑒云"監商",避宋廟諱也。監,古銜翻。
	}
管叔以商畔也。周公知其將畔而使之與?"|{
	畔,與叛同。與,讀曰歟,下同。
	}
曰:"不知也。"陳賈曰:"然則聖人亦有過與?"曰:"周公,弟也,管叔,兄也,周公之過不亦宜乎!且古之君子,過則改之;今之君子,過則順之。古之君子,其過也如日月之食,民皆見之;及其更也,|{
	更,工衡翻,更改。
	}
民皆仰之。今之君子,豈徒順之,又從為之辭!"

  ④是歲,齊宣王薨,子閔王地立。|{
	閔,讀曰閔。
	}

  二年(戊申,西元前三一三年)

  ①秦右更疾伐趙,|{
	右更,秦爵第十四。師古曰:左、右、中更,皆主領更卒而部其役使也。更,工衡翻。
	}
拔藺,虜其將莊豹。|{
	莊姓有出於宋者,左傳所謂戴、武、莊之族是也;有出於楚者,楚莊王之後,莊蹻是也。齊之莊暴,楚之莊辛,蒙之莊周,與此莊豹,其時適相先後,莫能審其所自出。
	}

  ②秦王欲伐齊,患齊、楚之從親,|{
	從,子容翻。
	}
乃使張儀至楚,說楚王曰:"大王誠能聽臣,閉關絕約于齊,|{
	說,式芮翻。閉關者,古之列國各置關尹,敵國賓至,關尹以告,則行理以節逆之。閉關則距絕其使,不為通也。使,疏吏翻。
	}
臣請獻商於之地六百里,使秦女得為大王箕帚之妾,|{
	於,如字。箕帚之妾,猶言備灑掃也。帚,止酉翻,彗也。
	}
秦、楚嫁女娶婦,長為兄弟之國。"楚王說而許之。|{
	說,讀曰悅。
	}
群臣皆賀,陳軫獨吊。|{
	陳姓出於舜,周武王封舜後於陳,子孫以國為氏。
	}
王怒曰:"寡人不興師而得六百里地,何吊也?"對曰:"不然。以臣觀之,商於之地不可得而齊、秦合,齊、秦合則患必至矣。" 王曰:"有說乎?"對曰:"夫秦之所以重楚者,以其有齊也。|{
	夫,音扶,發語辭。
	}
今閉關絕約于齊則楚孤,秦奚貪夫孤國而與之商於之地六百里!張儀至秦,必負王。是王北絕齊交,西生患于秦也,|{
	楚東北接齊,西接秦。
	}
兩國之兵必俱至。為王計者,不若陰合而陽絕于齊,使人隨張儀,苟與吾地,絕齊未晚也。"王曰:"願陳子閉口,毋複言,以待寡人得地!"|{
	毋,音無,毋者,禁止之辭。複,扶又翻,再又也。
	}
乃以相印授張儀,|{
	相,息亮翻。
	}
厚賜之。遂閉關絕約于齊,使一將軍隨張儀至秦。|{
	班固百官表:將軍,週末官,秦、漢因之。
	}

  張儀詳墮車,|{
	詳,讀曰佯,詐也。
	}
【章:乙十一行本正作"佯"。】不朝三月。|{
	朝,直遙翻。
	}
楚王聞之,曰:"儀以寡人絕齊未甚邪?"|{
	邪,餘遮翻。邪,疑辭也。
	}
乃使勇士宋遺借宋之符,北罵齊王。|{
	既閉關絕約,則齊、楚之信使不通,故使宋遺借宋符以至齊。宋,姓也。周武王封微子于宋,子孫以國為氏。
	}
齊王大怒,折節以事秦,|{
	折,而設翻。
	}
齊、秦之交合。|{
	儀歸而詐疾,待齊、秦之交合乃朝。
	}
張儀乃朝,見楚使者曰:"子何不受地?從某至某,廣袤六裡。"|{
	朝,直遙翻。使,疏吏翻。東西曰廣,南北曰袤。廣,古曠翻,又讀如字。袤,音茂。
	}
使者怒,還報楚王。|{
	還,從宣翻,又音如字。
	}
楚王大怒,欲發兵而攻秦。陳軫曰:"軫可發口言乎?攻之不如因賂之以一名都,與之並力【章:十二行本"力"作"兵";乙十一行本同;孔本同。】而攻齊,是我亡地于秦,取償于齊也。|{
	償,辰羊翻,報也。
	}
今王已絕于齊而責欺于秦,是吾合齊、秦之交而來天下之兵也,國必大傷矣!"楚王不聽,使屈匄帥師伐秦。|{
	屈,姓也,音九勿翻,匄,居大翻。帥,讀曰率。
	}
秦亦發兵使庶長章擊之。|{
	長,知丈翻。按史記樗裡子傳,庶長章,姓魏。
	}

  三年(己酉,西元前三一二年)

  ①春,秦師及楚戰於丹陽,|{
	索隱曰:此丹陽在漢中。劉昭曰:南郡枝江縣有丹陽聚,即秦破楚處。李輿地紀勝曰:丹陽在今歸州秭歸縣東八裡屈沱楚王城是也。余按楚遺屈匄伐秦,秦發兵逆擊之,枝江之丹陽則距郢逼近,秭歸之丹陽則不當秦、楚之路。索隱因下文遂取漢中,即謂丹陽在漢中,皆非也。此丹陽謂丹水之陽。班志:丹水出上洛塚嶺山,東至析入鈞水,其水蓋在弘農丹水、析兩縣之間,武關之外也。秦、楚交戰當在此水之陽。楚師既敗,秦師乘勝取上庸路西入以收漢中,其勢易矣。索,山客翻。與埴同。屈,九勿翻。塚,知隴翻。易,弋豉翻。
	}
楚師大敗,斬甲士八萬,虜屈匄及列侯、執圭七十餘人,|{
	執圭,楚爵也,執圭而朝者也。
	}
遂取漢中郡。|{
	自沔陽、成固至新城、上庸,時皆漢中郡之地。釋名曰:郡,群也,人所群聚也。黃義仲十三州記曰:郡之言君也。改公侯之封而言君者,至尊也。今郡字,"君"在其左,"邑"在其右,君為元首,邑以載民,故取名於君,謂之郡。
	}
楚王悉發國內兵以複襲秦,|{
	複,扶又翻。
	}
戰于藍田,|{
	班志,藍田縣屬京兆,秦孝公置。史記正義曰:藍田縣在雍州東南八十裡。從藍田關入藍田縣,時楚襲秦深入。
	}
楚師大敗。韓、魏聞楚之困,南襲楚,至鄧。|{
	鄧,春秋鄧國之地。班志,鄧縣屬南陽郡。杜預曰:潁川召陵縣西有鄧城。括地志曰:故鄧城在豫州郾陵縣東三十五裡,所謂在古召陵西十裡者也。召,讀曰邵。
	}
楚人聞之,乃引兵歸,割兩城以請平于秦。

  ②燕人共立太子平,是為昭王。|{
	燕,因肩翻。
	}
昭王于破燕之後。【章:十二行本"後"下有"即位"二字;乙十一行本同;孔本同;張校同。】|{
	言燕國為齊所破,已承其後也。
	}
吊死問孤,與百姓同甘苦,卑身厚幣以招賢者。謂郭隗曰:"齊因孤之國亂而襲破燕,孤極知燕小力少,|{
	臧文仲曰:列國有凶稱孤,禮也。杜預曰:列國諸侯無凶則稱寡人。郭姓出於周之虢公,世亦謂虢公為郭公。隗,五罪翻。少,始紹翻。
	}
不足以報;然誠得賢士與共國,以雪先王之恥,|{
	謂燕王噲破國之恥。噲,苦夬翻。
	}
孤之願也。先生視可者,得身事之!"郭隗曰:"古之人君有以千金使涓人求千里馬者,|{
	春秋以來,諸侯之國有涓人,秦、漢之間有中涓。師古曰:涓,潔也。言其在中主知潔清灑掃之事,蓋王之親舊左右也。應劭曰:涓人如謁者。涓,古玄翻。灑,所賣翻;掃,所報翻;又皆音如字。
	}
馬已死,買其首五百金而返。君大怒,涓人曰:『死馬且買之,況生者乎!馬今至矣。』不期年,千里之馬至者三。|{
	期,讀曰朞。
	}
今王必欲致士,先從隗始,況賢於隗者,豈遠千里哉!"|{
	言燕王若加禮于郭隗,則四方之賢士聞之,將不以千里為遠而來。
	}
於是昭王為隗改築宮而師事之。於是士爭趣燕:|{
	為,於偽翻。趣,七喻翻。
	}
樂毅自魏往,劇辛自趙往。|{
	劇,竭戟翻。劇,姓;辛,名。劇姓莫知其所自出。班志,北海郡有劇縣,蓋其先以縣為姓也。
	}
昭王以樂毅為亞卿,任以國政。|{
	為燕用樂毅破齊張本。
	}

  ③韓宣惠王薨,子襄王倉立。

  四年(庚戌,西元前三一一年)

  ①蜀相殺蜀侯。|{
	相,息亮翻。蜀相,蓋陳莊也。
	}

  ②秦惠王使人告楚懷王,請以武關之外易黔中地。|{
	武關,左傳之少習,地在漢弘農郡析縣西百七十裡,道通南陽。晉太康地志曰:武關當冠軍西。括地志曰:武關在商州上洛縣東。武關之外,蓋秦丹、析、商於之地。黔,音琴。少,始照翻。冠,工玩翻。於,音如字。
	}
楚王曰:"不願易地,願得張儀而獻黔中地。"張儀聞之,請行。王曰:"楚將甘心於子,|{
	楚王以墮張儀之詐,故欲甘心焉。
	}
柰何行?"張儀曰:"秦強楚弱,大王在,楚不宜敢取臣。且臣善其嬖臣靳尚,|{
	嬖,匹計翻,又卑義翻。靳,居焮翻,姓也。
	}
靳尚得事幸姬鄭袖,|{
	鄭,以國為氏。"袖",戰國策作"袖",古字也。
	}
袖之言,王無不聽者。"遂往。楚王囚,將殺之。靳尚謂鄭袖曰:"秦王甚愛張儀,將以上庸六縣及美女贖之。|{
	上庸,春秋庸國。班志,上庸縣屬漢中郡。史記正義:上庸縣,今房州。宋白曰:今房州竹山縣古城,即漢上庸縣。
	}
王重地尊秦,秦女必貴而夫人斥矣。"於是鄭袖日夜泣于楚王曰:"臣各為其主耳。|{
	為,幹偽翻。
	}
今殺張儀,秦必大怒。妾請子母俱遷江南,毋為秦所魚肉也。"王乃赦張儀而厚禮之。張儀因說楚王曰:"夫為從者無以異於驅群羊而攻猛虎,不格明矣。|{
	說,式芮翻。夫,音扶。從,子容翻。格,當也。劉伯莊曰:格,各額翻,其字宜從"手"。餘據字書,格,擊也,鬭也,從"木"亦通。
	}
今王不事秦,秦劫韓驅梁而攻楚,則楚危矣。秦西有巴、蜀,治船積粟,浮岷江而下,|{
	治,直之翻。江水出蜀郡湔氐道之岷山,故謂之岷江。釋名曰:江,共也;小流入其中,所公共也。
	}
一日行五百餘裡,不至十日而拒【章:十二行本"拒"作"距";乙十一行本同。】捍關,|{
	徐廣曰:巴郡魚複縣有捍關。史記正義曰:在峽州巴山縣界。捍,寒旦翻。
	}
捍關驚則從境以東盡城守矣,|{
	境,楚境也。捍關,楚之西境,從境以東,謂捍關以東也。
	}
黔中、巫郡非王之有。|{
	黔,巨今翻。班志,巫縣屬南郡。酈道元曰:縣故楚之巫郡。杜佑曰:今歸州巴東縣是也。
	}
秦舉甲出武關,則北地絕。|{
	北地,楚北境之地,陳、蔡、汝、潁是也。
	}
秦兵之攻楚也,危難在三月之內。|{
	難,乃旦翻。
	}
而楚待諸侯之救在半歲之外,夫待弱國之救,忘強秦之禍,此臣所為大王患也。|{
	夫,音扶。為,於偽翻。
	}
大王誠能聽臣,臣【章:十二行本不重"臣"字;乙十一行本同;孔本同。】請令秦、楚長為兄弟之國,無相攻伐。"|{
	令,力丁翻。
	}
楚王已得張儀而重出黔中地,|{
	重,難也。以地為重,意難割棄之。
	}
乃許之。

  張儀遂之韓,說韓王曰:"韓地險惡山居,|{
	之,如也,自楚如韓也。韓有宜陽、成皋,南盡魯陽,皆山險之地。說,式芮翻。
	}
五穀所生,非菽而麥,|{
	菽,式竹翻,豆也。
	}
國無二歲之食,見卒不過二十萬。|{
	見卒,見在之兵。見,賢遍翻。
	}
秦被甲百余萬。|{
	被,皮義翻。
	}
山東之士被甲蒙胄以會戰,秦人捐甲徒裼以趨敵,|{
	胄,今謂之兜鍪。捐,與專翻,棄也。徒,徒行也。裼,音錫,袒也。趨,七喻翻。鍪,音牟。
	}
左挈人頭,右挾生虜。夫戰孟賁、烏獲之士以攻不服之弱國,|{
	挾,戶頰翻。孟賁、烏獲,古之勇士。賁,音奔。
	}
無異垂千鈞之重於鳥卵之上,必無幸矣。|{
	三十斤為鈞。必無幸矣,言無幸而獲全之理。
	}
大王不事秦,秦下甲據宜陽,塞成皋,|{
	下,遐稼翻。塞,悉則翻。
	}
則王之國分矣,鴻台之宮,桑林之苑,非王之有也。為大王計,莫如事秦以攻楚,以轉禍而悅秦,計無便於此者。"韓王許之。

  張儀歸報,秦王封以六邑,號武信君。複使東說齊王曰:"從人說大王者|{
	複,扶又翻。從人,合從之人也。從,子容翻。說,式芮翻。
	}
必曰:『齊蔽于三晉,地廣民眾,兵強士勇,雖有百秦,將無柰齊何。』大王賢其說而不計其實。今秦、楚嫁女娶婦,為昆弟之國;韓獻宜陽;梁效河外;|{
	河外,秦蓋以河東為河外,梁則以河西為河外,張儀以秦言之也。
	}
趙王入朝,割河間以事秦。|{
	朝,直遙翻。河間,趙地。漢文帝二年,分為河間國。應劭曰:在兩河之間。唐為瀛州。
	}
大王不事秦,秦驅韓、梁攻齊之南地,|{
	漢泰山、城陽,齊南境之地也。
	}
悉趙兵,渡清河,指博關,臨菑、即墨非王之有也!|{
	博關在濟州西界之博陵。史記正義曰:博關在博州。趙兵從貝州渡清河指博關,則漯河以南臨菑、即墨危矣。濟,子禮翻。漯,托合翻。
	}
國一日見攻,雖欲事秦,不可得也!"齊王許張儀。

  張儀去,西說趙王曰:"大王收率天下以擯秦,秦兵不敢出函谷關十五年。|{
	擯,必刃翻。事見上卷顯王三十六年。
	}
大王之威行于山東,敝邑恐懼,|{
	春秋以來,列國交聘,行人率自稱其國曰敝邑。
	}
繕甲厲兵,力田積粟,愁居懾處,不敢動搖,|{
	懾,之涉翻,怖也,心伏也,失常也,失氣也。處,昌呂翻。
	}
唯大王有意督過之也。|{
	師古曰:督過,視責也。索隱曰:督者,正其事而責之;督過,是深責其過也。
	}
今以大王之力,舉巴、蜀,|{
	事見慎靚王五年。
	}
並漢中,|{
	事見上二年。
	}
包兩周,|{
	元年服韓、魏,則包兩周矣。
	}
守白馬之津。|{
	班志,白馬縣屬東郡。水經注:白馬津在白馬城之西北。白馬城,唐為滑州治所。開山圖曰:白馬津東可二十許裡,有白馬山,山上常有白馬群行,悲鳴則河決,馳走則山崩,後人因以名縣及津。按通鑒不語怪,今此注亦近于怪,姑以廣異聞耳。
	}
秦雖僻遠,然而心忿含怒之日久矣。今秦有敝甲凋兵軍于澠池,|{
	敝,敗惡也,凋,瘁也,半傷也。敗甲凋兵,謙其辭,言軍于澠池,則張其勢以臨趙矣。康曰:澠池,趙邑。余據趙與韓、魏接境,韓有野王、上党,魏有河東、河內,而澠池則秦地也,漢為縣,屬弘農郡,趙安能越韓、魏而有之!康說非是。澠,莫善翻;又莫忍翻。
	}
願渡河,逾漳,據番吾,|{
	言欲自澠池北渡河,又自此東逾漳水而進據番吾,此亦張聲勢以臨趙也。番吾,即漢常山郡之蒲吾縣也。劉昭注曰:史記番吾君,杜預云:晉之蒲邑也。此說非。括地志:番吾故城,在恒州房山縣東二十裡。番,音婆,又音盤。
	}
會邯鄲之下,願以甲子合戰,正殷紂之事。|{
	武王伐紂,癸亥陳于商郊,甲子昧爽,紂帥其旅若林,會於牧野,前徒倒戈,攻其後以北,遂以勝殷殺紂。張儀引以懼趙,其有所侮而動,亦已甚矣。邯鄲,趙都,音寒丹。
	}
謹使使臣先聞左右。|{
	使臣,上疏吏翻。
	}
今楚與秦為昆弟之國,而韓、梁稱東藩之臣,齊獻魚鹽之地,|{
	齊東瀕於海,海濱廣斥,魚鹽所出也。此時齊未嘗獻地于秦,張儀駕說以恐動趙耳。
	}
此斷趙之右肩也。夫斷右肩而與人鬥,|{
	夫,音扶。斷,丁管翻。
	}
失其党而孤居,求欲毋危得乎!今秦發三將軍,其一軍塞午道,|{
	索隱曰:午道當在趙之東,齊之西。午道,地名也。鄭玄云:一縱一橫為午,謂交道也。塞,悉則翻。
	}
告齊使渡清河,軍於邯鄲之東,|{
	邯鄲,音寒丹。
	}
一軍軍成皋,驅韓、梁軍於河外,|{
	史記正義曰:河外,謂鄭滑州,北臨河。餘謂此河外,亦因趙而言之。
	}
一軍軍于澠池,約四國為一以攻趙,趙服必四分其地。|{
	言秦約齊、韓、魏四分趙地。
	}
臣竊為大王計,莫如與秦王面相約而口相結,常為兄弟之國也。"趙王許之。|{
	當時趙于山東最強,且主從約,張儀說之,亦費辭矣。
	}

  張儀乃北之燕,|{
	燕,因肩翻。
	}
說燕王曰:"今趙王已入朝,效河間以事秦。|{
	張儀自趙至燕,借此氣勢而為是虛言以動燕耳。朝,直遙翻。
	}
大王不事秦,秦下甲云中、九原。|{
	虞氏記曰:趙自五原河曲築長城,東至陰山,又於河西造大城,一箱崩不就,乃改卜陰山河曲而禱焉,晝見群鵠游於云中,徘徊經日,見大光在其下,乃即其處築城,今云中城是也。餘謂此亦語怪,酈道元為後魏書之耳。宋白曰:勝州榆林縣界有云中古城,趙武侯所築,秦置云中郡,唐為單于都護府。班志:九原縣屬五原郡。漢之五原,即秦之九原郡也。唐為豐、鹽等州之地。宋白曰:唐豐州治九原縣。按云中九原,皆在燕之西,秦自上郡朔方下兵則可至。史記正義曰:古云中、九原郡皆在勝州。云中郡故城在榆林東北四十裡。九原郡故城在勝州西界,今連穀縣是。下,遐稼翻。元為,於偽翻。
	}
驅趙而攻燕。則易水、長城非大王之有也!|{
	水經注:易水出涿郡故安縣閻鄉西山,東屆關城西南,即燕長城門也。易水又曆長城而東過范陽、容城、安次、泉州縣南而東入海。
	}
且今時齊、趙之于秦,猶郡縣也,不敢妄舉師以攻伐。今王事秦,長無齊、趙之患矣。"|{
	以利動之。
	}
燕王請獻常山之尾五城以和。|{
	常山,即北嶽恒山也。漢文帝諱恒,改曰常山,置常山郡。班志,常山在常山郡上曲陽縣西北,其尾則燕之西南界。
	}

  張儀歸報,未至咸陽,秦惠王薨,子武王立。|{
	索隱曰:武王,名蕩。
	}
武王自為太子時,不說張儀;|{
	說,讀曰悅。
	}
及即位,群臣多毀短之。|{
	毀短,訾毀而數其短也。
	}
諸侯聞儀與秦王有隙,|{
	隙,乞逆翻,怨隙也,釁隙也。物之有罅釁者為有隙,人之與人有怨者亦為有隙。
	}
皆畔衡,複合從。|{
	衡,讀曰橫。從,子容翻。以此觀之,此時六國之勢,利在合從,而從張儀連衡者,畏秦而搖於儀之說耳。
	}

  五年(辛亥,西元前三一零年)

  ①張儀說秦武王曰:"為王計者,東方有變,|{
	韓、魏皆在秦之東。說,式芮翻。
	}
然後王可以多割得地也。臣聞齊王甚憎臣,臣之所在,齊必伐之。臣願乞其不肖之身以之梁,|{
	不肖,謙言無所肖似也。魏都大樑。
	}
齊必伐梁,齊、梁交兵而不能相去,|{
	言兵交不解,各欲去而不能也。
	}
王以其間伐韓,|{
	間,居莧翻,間隙也,又居閑翻,中間也。
	}
入三川,挾天子,案圖籍,此王業也!"|{
	張儀欲傾周而為秦;始終以此說為主。挾,戶頰翻。
	}
王許之。齊王果伐梁,梁王恐。張儀曰: "王勿患也!|{
	言勿以為患。
	}
請令齊罷兵。"|{
	令,盧經翻,使也;下同。
	}
乃使其舍人之楚,借使謂齊王曰:|{
	之,往也,如也。不敢徑遣人使齊,而往楚借使,借使,言借楚人以為使。借,子夜翻;康資昔切。使,疏吏翻。
	}
"甚矣王之托儀于秦也!"齊王曰:"何故?"楚使者曰:"張儀之去秦也固與秦王謀矣,欲齊、梁相攻而令秦取三川也。今王果伐梁,是王內罷國而外伐與國,|{
	罷,讀曰疲。
	}
而信儀于秦王也。"齊王乃解兵還。|{
	還,從宣翻,又如字。
	}
張儀相魏一歲,卒。|{
	相,息亮翻。卒,子恤翻。
	}

  儀與蘇秦皆以縱橫之術游諸侯,致位富貴,天下爭慕效之。|{
	縱,子容翻。
	}
又有魏人公孫衍者,號曰犀首,亦以談說顯名。|{
	說,式芮翻。
	}
其餘蘇代、蘇厲、周最、樓緩之徒,紛紜徧於天下,務以辯詐相高,不可勝紀,|{
	姓譜曰:周姓本自周平王子,別封汝川,人謂之周家,因氏焉。一云:以赧王為秦所滅,黜為庶人,百姓稱為周家,因氏焉。余按商有太史周任,謂為周姓所自出,夫豈不可!又赧王于時未滅,不可謂周最出於赧王。樓姓,夏少康之裔,周封為東樓公,子孫因氏焉。師古曰:紛紜,興作貌,又物多而亂貌。勝,音升。赧,奴版翻。夏,戶雅翻。少,始照翻。裔,苗裔。
	}
而儀、秦、衍最著。|{
	著者,顯著于時。
	}

  孟子論之曰:或謂:"公孫衍張儀豈不大丈夫哉,一怒而諸侯懼,安居而天下熄?"|{
	熄,滅也,火滅為熄。此言天下兵革之事熄滅也。
	}
孟子曰:"是惡足為大丈夫哉!|{
	惡,音烏。
	}
君子立天下之正位,行天下之正道,得志則與民由之,不得志則獨行其道,富貴不能淫,貧賤不能移,威武不能詘,|{
	詘,與屈同。
	}
是之謂大丈夫。"

  揚子法言曰:或問:"儀、秦學乎鬼谷術而習乎縱橫言,安中國者各十餘年,是夫?"|{
	夫,音扶。
	}
曰:"詐人也,聖人惡諸。"|{
	惡,烏路翻。
	}
曰:"孔子讀而儀、秦行,|{
	謂讀孔子之言而行儀、秦之事。
	}
何如也?"曰:"甚矣鳳鳴而鷙翰也!"|{
	翰,侯旰翻,又侯安翻,羽翰。
	}
"然則子貢不為歟?"曰:"亂而不解,子貢恥諸。"|{
	太史公曰:子貢一出,存魯,亂齊,破吳,強晉而霸越。溫公曰:考其年與事皆不合,蓋六國遊說之士托為之辭,太史公不加考訂,因而記之;揚子云亦據太史公書發此語也。說,式芮翻。
	}
說而不富貴,儀、秦恥諸。"|{
	說,式芮翻。
	}
或曰:"儀、秦其才矣乎,跡不蹈已?"|{
	宋鹹曰:蹈,踐也;言儀、秦之才術超卓,自然不踐循舊人之跡。踐,慈演翻。
	}
曰:"昔在任人,帝而難之。|{
	書舜典:而難任人。孔安國注云:任,佞也;難,拒也;言佞人則斥遠之。任,音壬。難,乃旦翻。
	}
不以才乎?才乎才,非吾徒之才也!"

  ②秦王使甘茂誅蜀相莊。|{
	四年,蜀相殺蜀侯,秦武王故誅之。史記"莊"作"壯"。案秦紀,秦既得蜀,使陳莊相蜀;從"莊"為是。
	}

  ③秦王、魏王會于臨晉。|{
	班志,臨晉縣屬馮翊,故大荔也,秦取之,更名臨晉。應劭曰:臨晉水,故名。臣瓚曰:晉水在河之東,此縣在河之西,不得臨晉水。舊說,秦築高壘以臨晉國,故曰臨晉。章懷太子賢曰:臨晉故城,在今同州朝邑縣西南。余按唐書地理志,蒲州有臨晉縣。宋白曰:漢臨晉縣在今臨晉縣東南十八裡,故解城是也。後魏改為北解縣。周省。隋分猗氏縣,置桑泉縣。唐天寶十二載,改臨晉縣。天寶之改縣,必有所據,則應劭臨晉水之說,無可厚非。秦之臨晉在河西,臣瓚、章懷之說皆是也。更,工衡翻。應,乙陵翻。瓚,藏旱翻。朝,直遙翻。解,戶買翻。載,祖亥翻。
	}

  ④趙武靈王納吳廣之女孟姚,|{
	吳姓,以國為氏。
	}
有寵,是為惠後。|{
	孔穎達曰:後,後也,言其後于天子,亦以廣後胤也。戰國諸侯僭王,亦稱其夫人為後。
	}
生子何。|{
	為立何而長子章爭國張本。長,知兩翻。
	}

  六年(壬子,西元前三零九年)

  ①秦初置丞相,以樗裡疾為右丞相。|{
	應劭曰:丞者,承也;相者,助也。荀悅曰:秦本次國,命卿二人,故置左右丞相,無三公官。樗裡疾,秦惠王之弟也。高誘曰:疾居渭南之陰鄉,其裡有大樗樹,故號樗裡子。相,息亮翻。樗,醜於翻。誘,羊久翻。
	}

  七年(癸醜,西元前三零八年)

  ①秦、魏會于應。|{
	左傳曰:邘、晉、應、韓,武之穆也。杜預注云:應國在襄陽城父縣西。余按襄陽無城父縣。後漢志,潁川父城縣西南有應鄉,古應國也。括地志曰:故應城因應山為名。古之應國在汝州魯山縣東三十裡。應,乙陵翻。邘,音於。
	}

  ②秦王使甘茂約魏以伐韓,而令向壽輔行。甘茂【章:十二行本"茂"下有"至魏"二字;乙十一行本同;孔本同;張校同;退齋校同。】令向壽還,謂王曰:"魏聽臣矣,|{
	令,盧經翻,使也。向,式讓翻,姓也。姓譜:向姓本自宋文公枝子向文旰,旰孫戌以王父字為氏。余按左傳,向戌本出於宋桓公。孟子為齊卿,出吊于滕,王使王驩為輔行。趙岐注曰:輔行,副使也。旰,音幹。戌,音恤。傳,直戀翻。使,疏吏翻。
	}
然願王勿伐!"王迎甘茂於息壤而問其故。|{
	柳宗元曰:地長隆然而起,夷之而益高者為息壤。異書有云:鯀竊帝之息壤以堙洪水。意者此所謂息壤,蓋以地長得名。長,知兩翻。
	}
對曰:"宜陽大縣,其實郡也。|{
	杜佑曰:春秋時列國相滅,多以其地為縣,則縣大而郡小,故趙鞅曰:"上大夫受縣,下大夫受郡。"至於戰國,則郡大而縣小矣,故甘茂曰:"宜陽大縣。其實郡也。"漢官儀曰:凡郡:或以列國,陳、魯、齊、吳是也;或以舊邑,長沙、丹陽是也;或以山陵,泰山,山陽是也;或以川原,西河,河東是也;或以所出,金城城下得金,酒泉泉味如酒、豫章樟樹生庭、雁門雁之所育是也;或以號令,夏禹合諸侯,大計東冶之山會計,因名會稽是也。令,力正翻。名會,古外翻。
	}
今王倍數險,行千里,攻之難。|{
	倍,與背同,音蒲妹翻。數險,謂函穀及三崤之險。
	}
魯人有與曾參同姓名者殺人,人告其母,其母織自若也。|{
	參,所金翻,一音七南翻。
	}
及三人告之,其母投杼下機,逾牆而走。|{
	杼,直呂翻。說文曰:杼,機之持緯者,蓋今所謂梭。梭,蘇禾翻。
	}
臣之賢不若曾參,王之信臣又不如其母,疑臣者非特三人,臣恐大王之投杼也。魏文侯令樂羊將而攻中山,三年而拔之。|{
	事見一卷威烈王二十三年。令,音盧經翻。將,即亮翻。
	}
反而論功,文侯示之謗書一篋。|{
	謗,訕也,毀也。篋,竹笥也,音古頰翻。
	}
樂羊再拜稽首曰:『此非臣之功,君之力也!』|{
	稽首,首至地也。稽,音啟。
	}
今臣,羇旅之臣也,|{
	甘茂,楚下蔡人,故云然。羇,居宜翻,寄也。旅,客也。
	}
樗裡子、公孫奭挾韓而議之,王必聽之,|{
	樗,醜於翻。奭,施只翻。挾,戶頰翻。
	}
是王欺魏王而臣受公仲侈之怨也。"|{
	公仲侈,韓相也。
	}
王曰:"寡人弗聽也,請與子盟!"乃盟於息壤。秋,甘茂、庶長封帥師伐宜陽。|{
	長,知丈翻。帥,讀曰率。
	}

  八年(甲寅,西元前三零七年)

  ①甘茂攻宜陽,五月而不拔。樗裡子、公孫奭果爭之。秦王召甘茂,欲罷兵。甘茂曰:"息壤在彼。"|{
	征前盟也。
	}
王曰:"有之。"因大悉起兵以佐甘茂,|{
	佐,助也。
	}
斬首六萬,遂拔宜陽。韓公仲侈入謝于秦以請平。|{
	請平,猶請和也。
	}

  ②秦武王好以力戲,|{
	好,呼到翻。
	}
力士任鄙、烏獲、孟說皆至大官。|{
	烏,姓也。春秋時,齊有大夫烏枝鳴。姓譜:孟姓,魯桓公之子仲孫之胤,仲孫為三桓之孟,故曰孟氏。任,音壬。說,讀曰悅。
	}
八月,王與孟說舉鼎,絕脈而薨;|{
	脈,莫獲翻。脈者,系絡臟腑, 其血理分行于支體之間,人舉重而力不能勝,故脈絕而死。按史記甘茂傳云:武王至周而卒于周。蓋舉鼎者,舉九鼎也。世家以為龍文赤鼎。史記"脈"作 "臏"。
	}
族孟說。|{
	族者,誅夷其族。
	}
武王無子,異母弟稷為質于燕,|{
	質,音致。燕,因肩翻。
	}
國人逆而立之,|{
	逆,迎也。
	}
是為昭襄王。昭襄王母羋八子,|{
	羋,楚姓也。漢因秦制,嫡稱皇后,次稱夫人,又有美人、良人、八子、七子、長使、少使之號。美人爵視二千石,比少上造。八子視千石,比中更。羋,亡氏翻。
	}
楚女也,實宣太后。

  ③趙武靈王北略中山之地,|{
	略地之師速而疾。杜預曰:略者,總攝巡行之名也。
	}
至房子,|{
	班志,房子縣屬常山郡。史記正義曰:房子,今趙州縣。宋白曰:天寶元年改曰臨城。
	}
遂至【章:十二行本"至"作"之";乙十一行本同;孔本同;張校同;退齋校同。】代,北至無窮,|{
	自代北出塞外,大漠數千里,故日無窮。戰國策,武靈王曰:"昔先君襄王與代交地,城境封之,名曰無窮之門,所以詔後而期遠也。"
	}
西至河,登黃華之上。|{
	史記正義曰:黃華,蓋黃河側之山名。
	}
與肥義謀胡服騎射以教百姓,|{
	騎,奇寄翻。
	}
曰:"愚者所笑,賢者察焉。雖驅世以笑我,胡地、中山,吾必有之!"遂胡服。

  國人皆不欲,公子成稱疾不朝。|{
	朝,直遙翻。
	}
王使人請之曰:"家聽於親,|{
	親,謂父母。
	}
國聽於君。今寡人作教易服而公叔不服,吾恐天下議己【章:十二行本"己"作"之";乙十一行本同;孔本同;張校同,退齋校同。】也。制國有常,利民為本;從政有經,令行為上。|{
	令,力政翻。
	}
明德先論於賤,而從政先信於貴,|{
	德欲其下及,故先論於賤;卑賤者感其德,則德廣所及可知矣。法行自貴近始,故先信於貴;貴近者奉法,則法之必行可知矣。
	}
故願慕公叔之義以成胡服之功也。"公子成再拜稽首曰:|{
	稽,音啟。
	}
"臣聞中國者,聖賢之所教也,禮樂之所用也,遠方之所觀赴也,蠻夷之所則效也。今王舍此而襲遠方之服,|{
	則,法也。舍,讀曰舍。襲,重衣也。
	}
變古之道,逆人之心,臣願王孰圖之也!"|{
	孰,古熟字,通。
	}
【章:十二行本正作"熟";乙十一行本同;孔本同。】使者以報。王自往請之,|{
	使,疏吏翻。
	}
曰:"吾國東有齊、中山,|{
	按趙都邯鄲,東接于齊,中山在其東北,故史記趙世家載武靈王之言曰:"吾國東有河薄落之水,與齊、中山同之。"蓋河、薄落之水在趙之東,與齊、中山同此地險也。
	}
北有燕、東胡,西有樓煩、秦、韓之邊。|{
	史記正義曰:營州之境,即東胡、烏丸之地。林胡、樓煩,即嵐、勝之北也。班志:雁門郡樓煩縣。應劭注云:故樓煩胡地。嵐、勝以南,石州離石、藺等,七國時趙邊也,與秦隔河。晉、洺、潞、澤等州,皆七國時韓地,趙之西邊地。燕,因肩翻。
	}
今無騎射之備,則何以守之哉?|{
	騎,奇寄翻
	}
先時中山負齊之強兵,侵暴吾地,系累吾民,|{
	先,悉薦翻。累,力追翻。
	}
引水圍鄗,微社稷之神靈,則鄗幾於不守也。|{
	鄗,趙邑,漢光武改為高邑,隋、唐為柏鄉縣地,唐屬趙州。鄗,呼各翻。幾,居衣翻。
	}
先君醜之,|{
	以為趙國之醜。
	}
故寡人變服騎射,欲以備四境之難,|{
	難,乃旦翻。
	}
報中山之怨。而叔順中國之俗,惡變服之名,|{
	惡,烏路翻。
	}
以忘鄗事之醜,非寡人之所望也!"公子成聽命,乃賜胡服;明日服而朝。|{
	朝,直遙翻。
	}
於是始出胡服令|{
	令,力政翻。
	}
而招騎射焉。

  九年(乙卯,西元前三零六年)

  ①秦昭王使向壽平宜陽,|{
	平,正也,和也。正宜陽之疆界而和其民人也。向,式亮翻。
	}
而使樗裡子、甘茂伐魏。甘茂言于王,以武遂複歸之韓。|{
	史記正義曰:武遂本屬韓,近平陽。楚世家云:韓先王之墓在平陽,武遂去之七十裡。去年秦拔宜陽,因涉河城武遂,今複歸之韓。複,音如字。
	}
向壽、公孫奭爭之,不能得,|{
	向,式讓翻。
	}
由此怨讒甘茂。茂懼,輟伐魏蒲阪,亡去。|{
	班志,蒲阪縣屬河東郡,舊曰蒲。應劭曰:秦始皇東巡,見長阪,因加"阪"云。括地志:蒲阪故城,在蒲州河東縣南五裡。阪,音反。
	}
樗裡子與魏講而罷兵,|{
	講,和也。
	}
甘茂奔齊。

  ②趙王略中山地,至寧葭;|{
	水經注:衡漳水東北曆下博城西,又西逕樂鄉縣故城南,又東引葭水注之。葭,音加。
	}
西略胡地,至榆中。|{
	水經注:諸次水出上郡諸次山,其水東逕榆林塞,世又謂之榆林山,即漢書所謂"榆溪舊塞"者。自溪西去,悉榆柳之藪,緣曆沙陵,屆龜茲縣西出,故云廣長榆也。王恢曰"樹榆為塞",謂此。蘇林以為榆中在上郡,非也。按始皇本紀:西北逐匈奴,自榆中並河以東,屬之陰山。然榆中在金城東五十許裡,陰山在朔方東,以此推之,不得在上郡。余謂蘇林之說固未為盡,而道元所謂榆中在金城東五十許裡亦非也。據衛青取河南地,案榆溪舊塞,正在唐麟、勝二州界,其西則接古上郡之境。況諸次水出上郡,逕榆林塞入河,則榆中在上郡之東明矣,諸次水無西流至金城、榆中之理。夷考其故,道元特以班志金城郡有榆中縣,遂牽合以為說,不知此一節之誤尤甚于蘇林也。史記正義曰:榆中,勝州北河北岸也。杜佑曰:勝州榆林郡南即秦榆林塞。
	}
林胡王獻馬,|{
	如淳曰:林胡,即儋林。余謂此胡種落依阻林薄,因曰林胡。儋,都甘翻。種,章勇翻。
	}
歸,使樓緩之秦,仇液之韓。王賁之楚,|{
	歸,謂趙王自略中山歸也。仇,姓也。春秋時,宋有大夫仇牧。液,音亦。之,往也,如也。賁,音奔;康曰:離之父,翦之子。余按離父、翦子,秦將也;此王賁乃趙人,康說非是。將,即亮翻。
	}
富丁之魏,|{
	富,姓也。春秋時,周有大夫富辰。
	}
趙爵之齊;代相趙固主胡,致其兵。|{
	相,息亮翻。致者,使之至也。
	}

  ③楚王與齊、韓合從。|{
	楚與齊、韓合從,尋即倍之,適足致齊、韓之兵耳。從,子容翻。
	}

  十年(丙辰,西元前三零五年)

  ①彗星見,|{
	彗星,世所謂掃星,本類星,末類彗,小者數寸,長或竟天,見則兵起,主掃除,除舊佈新。唐史臣曰:彗體無光,傅日以為光,故夕見則東指,晨見則西指,或長或短,光芒所及則為災。又曰:孛星,彗之屬也,偏指曰彗,氣四出曰孛。孛者孛孛,非常惡氣之所生,災甚於彗。天文書謂五星之精為妖,歲星流為蒼彗,熒惑、填星散為赤彗、黃彗,太白、辰星變為白彗、黑彗。彗,祥歲翻,又徐醉翻,又旋芮翻。見,賢遍翻。掃,所報翻。傅,讀曰附。孛,蒲內翻。妖,於遙翻。填,讀曰鎮。
	}

  ②趙王伐中山,取丹丘、爽陽、鴻之塞,又取鄗、石邑、封龍、東垣。|{
	史記正義曰:丹丘,邢州縣。余按隋、唐志,邢州有內丘縣,漢之中丘縣也,未嘗有丹丘,不知其何據。"爽陽、鴻之塞,"史記作"華陽、鴟之塞"。括地志曰:北嶽別名曰華陽臺,即常山也,在定州恒陽縣北百四十裡。徐廣曰:"鴟"作"鴻",鴻上故關,今名汝城,在定州唐縣東北六十裡。又有鴻上水,出唐縣北葛洪山,山接北嶽恒山,皆在定州界。班志,石邑縣屬常山郡,井陘山在西。括地志:石邑故城,在恒州鹿泉縣南三十五裡。封龍山,一名飛龍山,在恒州鹿泉縣南四十五裡,邑蓋因山為名。洪氏隸釋載後漢所立白石碑云:常山國元氏縣界有封龍山。東垣,即漢真定國之真定縣,漢高帝更名。史記正義曰:趙之東垣,在恒州真定縣南八裡,故常山城是也。鄗,呼各翻。垣,於元翻。華,戶化翻。恒,戶登翻,鴟,醜之翻。陘,音刑。更,工衡翻。
	}
中山獻四邑以和。

  ③秦宣太后異父弟曰穰侯魏冉,同父弟曰華陽君羋戎;王之同母弟曰高陵君、涇陽君。魏冉最賢,|{
	秦封穰侯于陶,陶即范蠡所居陶邑。孟康曰:陶即定陶。班志,定陶縣屬濟陰郡。下云封于穰與陶;穰縣屬南陽郡,去定陶差遠。水經注曰:穰侯封於穰,益封于陶,其免相也,出之陶而卒,陶有穰侯塚。穰,音人羊翻。華陽,即武王歸馬之地。水經注:洛水自上洛縣東北分為二水,枝渠東北出為門水,水東北曆陽華之山,即華陽也。華,音戶化翻。羋,眉婢翻。相,息亮翻。卒,子恤翻。塚,知隴翻。班志,高陵縣屬馮翊,涇陽縣屬安定。杜佑曰:京兆涇陽縣乃秦封涇陽君之地。漢涇陽縣在今平涼郡界涇陽故城是也。宋白曰:雍州涇陽本秦舊縣。與杜佑同。索隱曰:高陵君,名顯,涇陽君,名悝。索,山客翻。悝,苦回翻。
	}
自惠王、武王時,任職用事,武王薨,諸弟爭立,唯魏冉力能立昭王。|{
	惠王,即惠文王。昭王,即昭襄王
	}
昭王即位,以魏冉為將軍,衛咸陽。是歲,庶長壯及大臣、諸公子謀作亂。|{
	長,知丈翻。
	}
魏冉誅之;及惠文後皆不得良死,|{
	惠文後,昭王嫡母也。死於正命曰良死。
	}
悼武王后出居【章:十二行本"居"作"歸";乙十一行本同;孔本同;退齋校同。】于魏,|{
	悼武王后,即秦武王后,昭王嫂也。
	}
王兄弟不善者,魏冉皆滅之。王少,宣太后自治事,任魏冉為政,威震秦國。|{
	少,詩照翻。治,直之翻。為范睢間魏冉張本。
	}

  十一年(丁巳,西元前三零四年)

  ①秦王、楚王盟于黃棘。|{
	史記正義曰:黃棘蓋在房、襄二州。余按班志,南陽郡有棘陽縣,應劭曰:縣在棘水之陽。
	}
秦複與楚上庸。|{
	三年,秦敗楚師,虜屈匄,取楚上庸。
	}

  十二年(戊午,西元前三零三年)

  ①彗星見。|{
	彗,祥歲翻,又徐醉翻,旋芮翻。見,賢遍翻
	}

  ②秦取魏蒲阪、晉陽、封陵,|{
	"晉陽",按史記世家作"陽晉",其地當在蒲阪之東,風陵之西,大河之陽,且本晉地也,故謂之陽晉,蘇秦所謂"衛陽晉之道",蓋以魏境有陽晉,故在衛境者稱"衛陽晉"以別之。括地志曰:晉陽故城,今名晉城,在蒲州虞鄉縣西。水經注:函谷關直北隔河有崇阜,巍然獨秀,世謂之風陵。酈道元所謂函穀,則潼關也。史記正義曰:封陵在蒲州。唐志:河中府河東縣南有風陵關。今若據括地志,則晉陽亦通
	}
又取韓武遂。|{
	九年,秦歸韓武遂。
	}

  ③齊、韓、魏以楚負其從親,|{
	九年,楚與齊、韓合從,蓋即負之也。從,子容翻。
	}
合兵伐楚。楚王使太子橫為質于秦以請救。|{
	質,音致。
	}
秦客卿通將兵救楚,三國引兵去。|{
	將,即亮翻。又音如字,領也。
	}

  十三年(己未,西元前三零二年)

  ①秦王、魏王、韓太子嬰會于臨晉,韓太子至咸陽而歸,秦複與魏蒲阪。|{
	阪,音反。去年秦取魏蒲阪。
	}

  ②秦大夫有私與楚太子鬥者。太子殺之,亡歸。|{
	楚太子質秦而亡歸,復質于齊;秦以為言而誘陷其父,齊乘其父出而要之以利。
	}

  十四年(庚申,西元前三零一年)

  ①日有食之,既。

  ②秦人取韓穰。|{
	班志,穰縣屬南陽郡。以時考之,當屬楚。然韓得潁川之地,與南陽接境,七國兵爭,疆埸之間,一彼一此,或者此時穰屬韓歟?穰,人羊翻。
	}

  ③蜀守煇叛秦,秦司馬錯往誅之。|{
	蜀守,蜀郡守也。史記秦紀作"蜀侯"。華陽國志曰:秦封王子煇為蜀侯。蜀侯祭,歸胙于王;後母疾之,加毒以進。王大怒,使司馬錯賜煇劍。守,音狩。煇,索隱音暉。
	}

  ④秦庶長奐會韓、魏、齊兵伐楚,|{
	修楚太子亡歸之怨。長,知丈翻。
	}
敗其師於重丘,殺其將唐昧;遂取重丘。|{
	唐姓本于唐堯。春秋之時,有二重丘:衛孫蒯飲馬於重丘,杜預曰:曹邑;諸侯同盟于重丘,杜預曰:齊地。時楚之境皆不至此。呂氏春秋曰:齊令章子與韓、魏攻荊,荊使唐薎將兵應之,夾泚而軍;章子夜襲之,斬薎於是水之上。水經注曰:泚水又西,澳水注之。水北出茈丘山,南入于泚水。意者重丘即茈丘也。敗,補邁翻。將,即亮翻。"昧",荀子作"蔑",楊倞注曰:與"昧"同,語音相近,當音末。索隱音莫葛翻。重,直龍翻。茈,才支翻。
	}

  ⑤趙王伐中山,中山君奔齊。

  十五年(辛酉,西元前三零零年)

  ①秦涇陽君為質于齊。|{
	質,音致。
	}

  ②秦華陽君伐楚,大破楚師,斬首三萬,殺其將景缺,取楚襄城。|{
	班志,襄城縣屬潁川郡,有西不羹,楚靈王所謂"大城陳、蔡、不羹,賦皆千乘"是也。將,即亮翻。陸德明曰:不羹,舊音郎;漢書地理志作"更"字,乘,繩證翻。
	}
楚王恐,使太子為質于齊以請平。|{
	為楚懷王入秦而卒、齊留太子以邀楚張本。
	}

  ③秦樗裡疾卒,以趙人樓緩為丞相|{
	樗,醜於翻。卒,子恤翻。相,息亮翻。
	}

  ④趙武靈王愛少子何,欲及其生而立之。|{
	少,詩照翻。及其生者,及其生而親見之。
	}

  十六年(壬戌,西元前二九九年)

  ①五月戊申,大朝東宮,|{
	朝,直遙翻。
	}
傳國於何,王廟見禮畢,出臨朝,|{
	廟見,始即位而見祖廟也。見,賢遍翻。
	}
大夫悉為臣。肥義為相國,並傅王。|{
	相國之官始此,秦、漢因之;漢、魏以降,其位望尊于丞相。相,息亮翻。
	}
武靈王自號"主父"。|{
	主父,言為國之主之父也。一曰,言其子主國而己則父也。
	}
主父欲使子治國,身胡服,將士大夫西北略胡地。|{
	治,直之翻。將,即亮翻,又如字。
	}
將自云中、九原南襲咸陽,於是詐自為使者,入秦。|{
	使,疏吏翻。
	}
欲以觀秦地形及秦王之為人,秦王不知,已而怪其狀甚偉,非人臣之度,|{
	賓主相見,交際之禮已,方怪其非人臣。
	}
使人逐之;主父行已脫關矣,審問之,乃主父也。|{
	謂已脫身出秦關也。
	}
秦人大驚。

  ②齊王、魏王會于韓。

  ③秦人伐楚,取八城。秦王遺楚王書曰:"始寡人與王約為兄弟,盟于黃棘,|{
	見上十一年。遺,于季翻。
	}
太子入質,至驩也。|{
	質,音致。見十二年。
	}
太子陵殺寡人之重臣,不謝而亡去。|{
	見十三年。
	}
寡人誠不勝怒,|{
	勝,音升。
	}
使兵侵君王之邊。|{
	謂戰重丘,取襄城。
	}
今聞君王乃令太子質于齊以求平。|{
	見十五年。
	}
寡人與楚接境,婚姻相親。|{
	妻父曰婚,婿父曰姻。字書:婚,昏也,禮娶以昏時,婦人陰也,故曰婚。婿家女之所因,故曰姻。字林:婚,婦家;姻,婿家。賈公彥曰:各據男女身,則男曰昏,女曰姻;若以親言之,則女之父曰婚, 婿之父曰姻。余按張儀言秦、楚嫁女娶婦為昆弟之國;考之于史,自赧王四年至是年,秦、楚未嘗嫁娶也。至十九年,楚懷王死于秦。至二十三年,楚襄王逆婦于秦。蓋先已約親,其後襄王終喪,始逆婦成婚姻。
	}
而今秦、楚不驩,則無以令諸侯。|{
	令,力政翻。
	}
寡人願與君王會武關,面相約,結盟而去,寡人之願也。"

  楚王患之,欲往恐見欺,欲不往恐秦益怒。昭睢曰:"毋行而發兵自守耳!|{
	睢,,息遺翻。又七餘翻。
	}
秦,虎狼也,有並諸侯之心,不可信也!"懷王之子【章:十二行本"子"下有"子"字;乙十一行本同;孔本同。】蘭勸王行,王乃入秦。秦王令一將軍詐為王,伏兵武關,楚王至則閉關劫之,與俱西,至咸陽,朝章台,如藩臣禮,|{
	朝,直遙翻。秦章台宮在渭南。漢張敞走馬章台街,孟康曰:在長安中;臣瓚曰:街在章台下。漢長安在渭南,以此言之,章台宮在渭南明矣。瓚,藏旱翻。
	}
要以割巫、黔中郡。楚王欲盟,秦王欲先得地。楚王怒曰:"秦詐我,而又強要我以地!"|{
	要,一遙翻。黔,其今翻。強,其良翻,又其兩翻。
	}
因不復許。秦人留之。|{
	複,扶又翻。
	}

  楚大臣患之,乃相與謀曰:"吾王在秦不得還,要以割地,而太子為質于齊,|{
	還,從宣翻,又音如字。要,一遙翻。質,音致。
	}
齊、秦合謀,則楚無國矣。"欲立王子之在國者。昭睢曰:"王與太子俱困于諸侯,今又倍王命而立其庶子,不宜!"|{
	睢,息隨翻。倍,蒲妹翻。
	}
乃詐赴于齊。|{
	詐,言楚王薨而請太子還王楚。
	}
齊閔王召群臣謀之,或曰:"不若留太子以求楚之淮北。"|{
	閔,讀曰閔。楚滅陳、蔡,封畛於汝,滅越,取吳故地,並有古徐夷之地,皆在淮北,即楚所謂"下東國"。
	}
齊相曰:"不可!郢中立王,|{
	郢,楚都。班志:南郡江陵縣,故楚郢都,楚文王自丹陽徙此;後九世,平王城之;又後十世,秦拔之,東徙壽春,亦名曰郢。水經:江水東逕江陵縣故城南,又東逕郢城南。注云:今江陵城,楚船官地,春秋之渚宮。郢城即子囊遺言所城者。劉昫曰:故楚都之郢城,今江陵縣北十五裡紀南城是也。相,息亮翻。
	}
是吾抱空質而行不義於天下也。"|{
	質,音致。
	}
其人曰:"不然,郢中立王,因與其新王市曰:『予我下東國,吾為王殺太子。|{
	巿,謂相要以利,如巿道也。予,讀曰與。為,於偽翻。
	}
不然,將與三國共立之。』"|{
	三國,謂齊、韓、魏。
	}
齊王卒用其相計而歸楚太子。|{
	卒,子恤翻。
	}
楚人立之。

  ④秦王聞孟嘗君之賢,使涇陽君為質于齊以請。孟嘗君來入秦,秦王以為丞相。|{
	質,音致。
	}

  十七年(癸亥,西元前二九八年)

  ①或謂秦王曰:"孟嘗君相秦,必先齊而後秦;|{
	先、後,皆去聲。
	}
秦其危哉!"秦王乃以樓緩為相,囚孟嘗君,欲殺之。孟嘗君使人求解于秦王幸姬,姬曰:"願得君狐白裘。"|{
	狐白裘,緝狐掖之皮為之,所謂千金之裘非一狐之掖者也。
	}
孟嘗君有狐白裘,已獻之秦王,無以應姬求。客有善為狗盜者,入秦藏中,|{
	物之所藏曰藏,音徂浪翻。
	}
盜狐白裘以獻姬。姬乃為之言于王而遣之。|{
	為,於偽翻。
	}
王後悔,使追之。孟嘗君至關,關法,雞鳴而出客,時尚蚤,|{
	蚤,古早字通。
	}
追者將至,客有善為雞鳴者,野雞聞之皆鳴,孟嘗君乃得脫歸。

  ②楚人告于秦曰:"賴社稷神靈,國有王矣!"秦王怒,發兵出武關擊楚,斬首五萬,取十六城。

  ③趙王封其弟【章:十二行本"弟"下有"勝"字;乙十一行本同。】為平原君。|{
	班志,平原縣屬平原郡。勝封于東武城,號平原君,非封于平原也。東武城屬清河郡,杜佑曰:今貝州武城縣是也。蓋定襄有武城,時同屬趙,故此加"東"也。
	}
平原君好士,|{
	好,呼到翻。
	}
食客嘗【章:十二行本"嘗"作"常";乙十一行本同。】數千人。有公孫龍者,善為堅白同異之辯,|{
	漢書藝文志:公孫龍子十四篇。注云:即為堅白同異之辯者。成玄英莊子疏云:公孫龍著守白論,行於世。堅白,即守白也,言堅執其說,如墨子墨守之義。自堅白之論起,辯者互執是非,不勝異說。公孫龍能合眾異而為同,故謂之同異。史記注曰:晉太康地記云:汝南西平縣有龍淵,水可用淬刀劍,極堅利,故有堅白之論云:黃,所以為堅也;白,所以為利也。或曰:黃所以為不堅,白所以為不利。二說未知孰是。勝,音升。淬,取內翻。
	}
平原君客之。孔穿自魯適趙,|{
	按孔叢子,孔穿,孔子之後。孫愐曰:孔姓,殷湯之後,本自帝嚳元妃簡狄,吞乙卵生契,賜姓子氏;至湯,以其祖感乙而生,故名履,字天乙;後代以"子"加"乙",始為孔氏。至宋孔父遭華督之難,其子奔魯,故孔子生於魯。愐,彌兗翻。嚳,苦沃翻。華,戶化翻。難,乃旦翻。
	}
與公孫龍論臧三耳,|{
	三耳,如莊子所載雞三足之說。莊子疏謂數起於一,一與一為二,二與一為三,三名雖立,實無定體,故雞可以為三足,則兩耳、三耳,其說亦猶是耳。一說,耳主聽,兩耳,形也,兼聽而言,可得為三。臧,臧獲之臧。臧獲,奴婢也。
	}
龍甚辯析。|{
	辯,別也;析,分也;言分別甚精微也。
	}
子高弗應,俄而辭出,明日複見平原君。|{
	子高,孔穿字也。複,扶又翻。
	}
平原君曰:"疇昔公孫之言信辯也,|{
	毛晃曰:疇,曩也;昔,夕也;疇昔,曩夕也。
	}
先生以為何如?"對曰:"然。幾能令臧三耳矣。|{
	毛晃曰:然,如也,是也,語決辭。幾,居依翻。令,使也,音力丁翻。
	}
雖然,實難!僕願得又問於君:今謂三耳甚難而實非也,謂兩耳甚易而實是也,不知君將從易而是者乎,其亦從難而非者乎?"平原君無以應。明日,謂公孫龍曰:"公無複與孔子高辯事也!|{
	易,弋豉翻。
	}
其人理勝於辭;公辭勝於理,終必受詘。"

  鄒衍過趙,|{
	過,古禾翻。
	}
平原君使與公孫龍論白馬非馬之說。|{
	此亦莊子所謂狗非犬之說。疏云:狗之與犬,一實兩名:名實合,則此為狗,彼為犬;名實離,則狗異於犬。又墨子曰:狗,犬也。然|{
	殺
	}
狗非狗|{
	殺
	}
犬也。大指與白馬非馬之說同。
}
鄒子曰:"不可。夫辯者,別殊類使不相害,序異端使不相亂。抒意通指,|{
	夫,音扶。別,彼列翻。索隱曰:抒,音墅,抒者舒也;又常恕翻。康曰:亦音舒。
	}
明其所謂,使人與知焉,不務相迷也。|{
	與,音如字;又讀曰預。
	}
故勝者不失其所守,不勝者得其所求。|{
	辯以求是,辯雖不勝而得審其是,所謂得其所求也。
	}
若是,故辯可為也。及至煩文以相假,飾辭以相敦,|{
	敦,都昆翻,迫也,詆也,誰何也。
	}
巧譬以相移,引人使不得及其意,如此害大道。夫繳紉【章:十二行本"紉"作"紛";乙十一行本同;孔本同;熊校同。】爭言而競後息,|{
	索隱:繳,音糾;康吉吊切,非。言其言戾,紛然而爭,欲人先屈,務在人後方止也。
	}
不能無害君子,衍不為也。"座皆稱善。|{
	言一座之人皆稱衍言為善。
	}
公孫龍由是遂詘。|{
	通鑒書此,言小辯終不足破大道。絀,音敕律翻。說文曰:絀,貶下也。又讀與屈同。
	}
