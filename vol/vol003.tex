\chapter{資治通鑑卷三}

宋 司馬光 撰

胡三省 音註

周紀三 (起重光赤奮若(辛丑),盡昭陽大淵獻(癸亥),凡二十有三年。)

  愼靚王|{
	諱定,顯王之子也。
	此複諡也。
	以諡法言之,
	諡法:敏以敬曰愼;柔德安衆曰靖。
	靚,疾正翻。
}


  元年(辛丑,西元前三二零年)

  1 衛更貶號曰君。|{
	顯王二十三年,衛已貶號曰侯;介於秦、魏之間,國日以削弱,因更貶其號曰君。
	更,居孟翻。
	貶,悲檢翻。
}

  二年(壬寅,西元前三一九年)

1 秦伐韓,取鄢。|{
	春秋"晉敗楚師于鄢陵",旣此鄢也。
	班志作"傿陵",屬潁川郡。
	鄢,音謁晚翻,又於建翻,師古音偃。
	史記正義曰:許州鄢陵縣西北十五裡有鄢陵古城。
}

2 魏惠王薨,子襄王立。|{
	索隱曰:系本曰:襄王,名嗣。
	今按系本卽世本,司馬貞避唐諱,改"世"爲"系"。
	考異曰;史記魏世家云:惠王三十六年卒,子襄王立。
	襄王十六年卒,子哀王立。
	哀王二十三年卒,子昭王立。
	六國表,惠王元辛亥,終丙戌;襄王元丁亥,終壬寅;哀王元癸卯,終乙丑。
	按杜預春秋後序云:太康初,汲縣有發舊塚者,大得古書,其紀年篇起自夏、殷、周,皆三代王事,無諸國別也;
	惟特記晉國,起自殤叔,次文侯、昭侯,以至曲沃莊伯,皆用夏正,編年相次;
	晉國滅,獨記魏事,下至魏哀王之二十年:蓋魏國之史記也。
	哀王于史記,襄王之子,惠王之孫也。
	古書紀年篇,惠王三十六年改元,從一年始,至十六年而稱惠成王卒,卽惠王也;疑史記誤分惠成之世以爲後王年也。
	哀王二十三年乃卒,故特不稱諡,謂之"今王"。
	裴駰魏世家注引和嶠云:紀年起自黃帝,終於魏之今王;今王者,魏惠成王子。
	按太史公書,惠成王但言惠王,惠王子曰襄王,襄王子曰哀王。
	惠王三十六年卒,襄王立十六年卒,並惠、襄爲五十二年。
	今按古文惠成王立三十六年,改元,稱一年,改元後十七年卒。
	太史公書爲誤分惠成之世以爲二王之年數也。
	世本,惠王生襄王而無哀王,然則"今王"者,魏襄王也。
	彼旣魏史,所書魏事必得其真,今從之。
}

	孟子入見而出,語人曰:"望之不似人君,就之而不見所畏焉。|{
	入見,賢遍翻。
	語,牛倨翻。
}
卒然問曰:『天下惡乎定?』|{
	卒,七沒翻。
	惡,音烏,何也。
}
吾對曰:『定于一。』
『孰能一之?』|{
	此一語,魏襄王以問孟子。
}
對曰:『不嗜殺人者能一之。』
『孰能與之?』|{
	此語亦襄王問。
}
對曰:『天下莫不與也。王知夫苗乎?|{
	夫,音扶。
}
七、八月之間旱,則苗槁矣。|{
	孟子此言,用周正也。
	周七、八月,夏五、六月也。
	槁,音考,乾枯也。
	夏,戶雅翻。
	乾,音干。
}
天油然作雲,沛然下雨,則苗浡然興之矣。|{
	油然,雲盛貌。
	沛然,雨盛貌。
	浡然,興起貌。
	沛,普蓋翻。
	浡,音勃。
}
其如是,孰能禦之?』"

  三年(癸卯,西元前三一八年)

  1 楚、趙、魏、韓、燕同伐秦,攻函谷關。|{
	燕,因肩翻,注已見上。
	宋白曰:函谷關在弘農。
	地理志注云:謂道形如函,孫卿子所謂“秦有松柏之塞”是也。
}
秦人出兵逆之,五國之師皆敗走。


  2 宋初稱王。


  四年(甲辰,西元前三一七年)

  1 秦敗韓師于脩魚,斬首八萬級,虜其將䱸、申差于濁澤。|{
	敗,補邁翻。
	索隱曰:脩魚,地名。
	䱸、申差,二將名。
	索,山客翻。
	將,卽亮翻。
	䱸,音瘦,又疏鳩翻。"濁澤",年表作"觀澤"。
	括地志,觀澤在魏州頓丘縣東十八里。
}
諸侯振恐。


  2 齊大夫與蘇秦爭寵,使人刺秦,殺之。|{
	刺,七亦翻。
}

  3 張儀說魏襄王曰:"梁地方不至千里,卒不過三十萬,地四平,無名山大川之限,卒戍楚、韓、齊、趙之境,|{
	戍,舂遇翻;字從"人",從"戈",人荷戈,所以戍也。
	梁地南接楚,西接韓,東接齊,北接趙。
}
守亭、障者不過十萬,|{
	說文:亭,民所安定也,道路所舍也。
	障,堡障也,隔也,塞也,所以隔塞敵人也。
}
梁之地勢固戰場也。
	夫諸侯之約從,盟於洹水之上,結爲兄弟以相堅也。|{
	事見上卷顯王二十六年。
	夫,音扶。
	從,子容翻。
	洹,于元翻。
}
今親兄弟同父母,尚有爭錢財相殺傷,而欲恃反覆蘇秦之餘謀,其不可成亦明矣?
大王不事秦,秦下兵攻河外,據卷衍、酸棗,|{
	後漢志:卷縣屬河南郡,酸棗縣屬陳留郡。
	水經注:河水逕卷縣北,又東至酸棗、延津,二邑皆河津之要也。
	卷,逵員翻。
	衍,以善翻。
}
劫衛,取陽晉,則趙不南,趙不南則梁不北,梁不北則從道絕,從道絕則大王之國欲毋危不可得也。|{
	從道,謂約從之路也。
	從,子容翻。
}
故願大王審定計議,且賜骸骨。" |{
	人臣委身以事君,身非我之有矣,故於其乞退也,謂之乞骸骨。
	骸,戶皆翻。
}
魏王乃倍從約,|{
	倍,蒲妹翻。
}
而因儀以請成于秦。
	張儀歸,復相秦。|{
	儀罷秦相相魏,見上卷顯王四十七年。
	相,息亮翻。
}

  4 魯景公薨,子平公旅立。|{
	諡法:由義而濟曰景;布義行剛曰景。
}

  五年(乙巳,西元前三一六年)

  1 巴、蜀相攻擊,|{
	巴,春秋巴子之國。
	蜀,蠶叢、魚鳧之後。
	華陽國志曰:昔蜀王封其弟于漢中,號曰苴侯,因命其邑曰葭萌。
	苴侯與巴王爲好。
	後巴與蜀爲讎,蜀王怒,伐苴侯,苴侯奔巴。
	巴求救于秦,秦伐蜀,蜀王敗死。
	秦滅蜀,因遂滅巴、苴,置巴、蜀二郡。
	史記正義曰:巴子城在合州石鏡縣南五裡,故墊江縣也。
	宋白曰:巴子後理閬中。
	揚雄蜀本紀曰:蜀王本治廣都樊鄉,徙居成都。
	華,戶化翻。
	苴,子餘翻。
	葭,音家。
	萌,謨耕翻。
	墊,音疊。
	閬,音浪。
}
俱告急于秦。
	秦惠王欲伐蜀,以爲道險陿難至,|{
	陿與狹同。
	漢書趙充國傳註:山附而夾水曰陿。
}
而韓又來侵,猶豫未能決。|{
	說文:猶,玃屬,居山中;聞人聲,豫登木,無人乃下。
	世謂不決曰猶豫。
	一說,隴西謂犬子爲猶,犬導人行,忽先忽後,故曰猶豫。
	又一說,猶豫,犬也,犬爲人行,好先行,卻住以俟其人,百步之間,如是者數四;先者,豫也,遂曰猶豫。
	猶,夷周翻,又餘救翻。
	玃,厥縛翻。
	爲,於偽翻。
	好,呼到翻。
}
司馬錯請伐蜀。|{
	史記:重、黎之後,至周宣王時爲程伯休父,爲司馬氏。
	錯,七各翻,又七故翻。
	重,直龍翻。
	父,音甫。
}
張儀曰:"不如伐韓。"王曰:"請聞其說。"儀曰:"親魏,善楚,下兵三川,攻新城、宜陽,|{
	伊水、洛水、河水爲三川。
	秦後置三川郡,漢改爲河南郡。
	班志,新城縣屬河南郡。
	括地志:洛州伊闕縣本漢新城縣,在州南七十裡。
	隋文帝改新城爲伊闕,取伊闕山爲名。
}
以臨二周之郊,|{
	周分爲東、西,故曰二周。
}
據九鼎,|{
	昔夏禹貢金九牧,鑄鼎象物,桀有昏德,鼎遷于商;商紂暴虐,鼎遷于周;成王定鼎於郟鄏,寶之,以爲三代共器。
	夏,戶雅翻。
	郟,音夾。
	鄏,音辱。
}
按圖籍,|{
	圖籍,謂天下之圖籍,周官職方氏所掌是也。
}
挾天子以令于天下,天下莫敢不聽,此王業也。
	臣聞爭名者于朝,爭利者于市。
	今三川、周室,天下之朝市也,|{
	朝,直遙翻。
	周禮大宗伯注云:朝,猶朝也,欲其來之早也。
	人君昕旦親政貴早,聲轉爲朝。
	猶朝,陟遙翻。
}
而王不爭焉,顧爭于戎翟,去王業遠矣。"|{
	翟,與狄同。
}
司馬錯曰:"不然。
	臣聞欲富國者務廣其地,欲強兵者務富其民,欲王者務博其德,|{
	欲王,於況翻,又如字。
}
三資者備而王隨之矣。
	今王地小民貧,故臣願先從事于易。|{
	易,弋豉翻。
}
夫蜀,西僻之國而戎翟之長也,|{
	夫,音扶。
	長,知丈翻。
}
有桀、紂之亂;以秦攻之,譬如使豺狼逐羣羊;|{
	豺,徂齋翻。
}
得其地足以廣國,取其財足以富民,繕兵不傷衆而彼已服焉。|{
	彼,謂蜀也。
}
拔一國而天下不以爲暴,
利盡四海而天下不以爲貪,
是我一舉而名實附也,
而又有禁暴止亂之名。
今攻韓,劫天子,惡名也,而未必利也;
又有不義之名,而攻天下所不欲,危矣。
臣請論其故:周,天下之宗室也。|{
	周室爲天下所宗,故謂之宗室。
}
齊,韓之與國也。|{
	鄰國相親睦者,謂之與國。
}
周自知失九鼎,韓自知亡三川,將二國並力合謀,以因乎齊、趙而求解乎楚、魏,|{
	並,必正翻。
	求解者,先與之構怨隙而今求和解也。
}
以鼎與楚,以地與魏,王弗能止也。
	此臣之所謂危也。
	不如伐蜀完。"|{
	完,全也。
	言以兵伐蜀,十全必取也。
}
王從錯計,|{
	錯,七各翻,又七故翻。
}
起兵伐蜀:十月取之。|{
	取,言易也。
	易,弋豉翻。
}
貶蜀王,更號爲侯;|{
	貶,悲檢翻。
	更,工衡翻。
}
而使陳莊相蜀。|{
	相,息亮翻。
}
蜀旣屬秦,秦以益強,富厚,輕諸侯。


  2 蘇秦旣死,|{
	三年,蘇秦死于齊。
}
秦弟代、厲亦以遊說顯於諸侯。|{
	說,式芮翻。
}
燕相子之與蘇代婚,欲得燕權。
	蘇代使於齊而還,|{
	燕,因肩翻。
	相,息亮翻。
	使,疏吏翻。
	還,從宣翻。
}
燕王噲問曰:"齊王其霸乎?"|{
	噲,苦夬翻。
}
對曰:"不能。"王曰:"何故?"對曰:"不信其臣。"於是燕王專任子之。
	鹿毛壽謂燕王曰:|{
	劉伯莊曰:鹿毛壽,人姓名;又曰潘壽。
	春秋後語作"唐毛壽"。
	徐廣曰:一作"厝毛"。
	如徐廣一作之說,當作"厝"。
	厝,音秦昔翻。
	清河有厝縣。
}
"人之謂堯賢者,以其能讓天下也。
	今王以國讓子之,是王與堯同名也。"燕王因屬國於子之,|{
	屬,之欲翻,付也,托也。
}
子之大重。
	或曰:"禹薦益而以啟人爲吏,|{
	孟子曰:禹薦益于天,禹崩,天下之人不之益而之啟,
	曰:"吾君之子也。"
	索隱曰:人,猶臣也。
	謂以啟臣爲益吏。
	索,山客翻。
}
及老而以啓爲不足任天下,|{
	任,音壬。
}
傳之於益。
	啓與交黨攻益,奪之,天下謂禹名傳天下於益而實令啓自取之。|{
	按或曰一段事,與師春紀伊尹放太甲,潜出自桐,殺伊尹,事頗相類,古書雜記固多也。
}
今王言屬國於子之而吏無非太子人者,是名屬子之而實太子用事也。"王因收印綬,自三百石吏已上而效之子之。|{
	後漢書輿服志曰:三王俗化雕文,詐偽漸生,始有印綬,以檢奸萌。
	周禮掌節有璽節,鄭氏注云:今之印章也。
	綬,組綬。
	古者佩玉以綬貫之。
	漢承秦制,乘輿璽綬;諸王以下,印以金、銀、銅爲差,綬以赤、紫、青、黑、黃爲差。
	印,信也,刻文合信也。
	綬,受也,轉相授受也。
	三百石吏,銅印,黑綬或黃綬。
	王制:諸侯大國之卿,食祿以田計之,爲三十二夫之入。
	戰國之卿,食祿萬鐘,其僭差不度甚矣。
	漢制:三公秩萬石,至於鬥食佐吏,凡十六等。
	三百石吏,第十等,奉月四十斛。
	綬,音受。
	璽,斯氏翻。
	組,祖五翻。
	乘,繩證翻。
	奉,與俸同,音扶用翻。
}
子之南面行王事,而噲老,不聽政,顧爲臣,|{
	顧,反也。
	噲,苦夬翻。
}
國事皆决于子之。|{
	爲後燕亂張本。
}

  六年(丙午,西元前三一五年)

  1 王崩,子赧王延立。


  赧王上|{
	劉伯莊曰:赧,慚之甚也。
	輕微危弱,寄住東、西,足爲慚赧,故號之曰赧;諡法本無赧字也。
	赧,音奴版翻。
}

  元年(丁未,西元前三一四年)

  1 秦人侵義渠,得二十五城。|{
	義渠,戎國名。
	按上卷顯王四十二年,
	秦縣義渠,以其君爲臣,是已得義渠矣。
	今又侵得二十五城,何也?
	蓋先此秦以義渠爲縣,君爲臣,
	雖臣屬於秦,義渠之國未滅也,秦稍蠶食侵其地。
	今得二十五城,義渠之國所餘無幾矣。
	蓋秦兼併諸侯,不盡其國不止也。
	左傳:有鐘鼓曰伐,無曰侵。
	谷梁傳:苞人民、驅牛馬曰侵。
	斬樹木、壞宮室曰伐。
	無幾,居豈翻。
	傳,直戀翻。
	壞,音怪。
}

  2 魏人叛秦,秦人伐魏,取曲沃而歸其人。
	又敗韓於岸門,|{
	續漢志,潁川郡穎隂縣有岸亭。
	註引徐廣云;岸亭,卽岸門。
	括地志:岸門在今許州長社縣東北二十八裡,今名長武亭。
	敗,補邁翻。
}
韓太子倉入質于秦以和。|{
	質,音致。
}

3 燕子之爲王三年,國内大亂。
  將軍市被與太子平謀攻子之。
  齊王令人謂太子曰:|{
	令,廬經翻。
}
“寡人聞太子將飭君臣之義,
明父子之位,寡人之國\fbox{雖小}唯太子所以令之。"|{
	飭,整也,脩也,治也。
	治,直之翻。
	飭君臣之義,言太子平將治子之僭王之罪也。
	明父子之位,言太子平當繼其父噲之位也。
	令,力政翻,命令也,號令也。
}
太子因要黨聚衆,|{
	要,一遥翻,要結也。
}
使市被攻子之,不克。市被反攻太子。搆難數月,|{
	難,乃旦翻。
}
死者數萬人,百姓恫恐。|{
	恫,它紅翻,痛也。
}
齊王令章子將五都之兵,因北地之衆以伐燕。|{
	將,卽亮翻,又音如字,領也。
	邑有先王之廟曰都。
	或曰:都,邑之大者。
	北地,齊之北境也,蓋漢千乘、清河、勃海之地。
	燕,因肩翻;下同。
	乘,繩證翻。
}
燕士卒不戰,城門不閉。
	齊人取子之,醢之,|{
	醢,呼改翻,肉醬也。
}
遂殺燕王噲。|{
	噲,苦夬翻。
}

齊王問孟子曰:
“或謂寡人勿取燕,或謂寡人取之。
	以萬乘之國伐萬乘之國,|{
	古者天子之地方千里,出兵車萬乘。
	七國兼併以強大,于時皆爲萬乘之國。
	乘,繩證翻。
}
五旬而舉之,|{
	十日爲旬,五旬,五十日。
}
人力不至於此;
不取,必有天殃。|{
	殃,咎也,禍也。
}
取之何如?”
孟子對曰:
“取之而燕民悅則取之,古之人有行之者,武王是也。
	取之而燕民不悅則勿取,古之人有行之者,文王是也。
	以萬乘之國伐萬乘之國,簞食壺漿以迎王師,|{
	簞,竹器也;圓曰簞,方曰笥。
	簞,音丹。
	食,祥吏翻,熟食也。
	漿,水也,酢漿也。
	笥,相吏翻。
	酢,倉故翻。
}
豈有他哉?
避水火也。
如水益深,如火益熱,亦運而已矣!"|{
	運,轉也。
	言燕之民將轉而之他國也。
}

  諸侯將謀救燕,齊王謂孟子曰:"諸侯多謀伐寡人者,何以待之?"對曰:"臣聞七十裡爲政於天下者,湯是也;未聞以千里畏人者也。
	書曰:『徯我後,後來其蘇。』|{
	書仲虺之誥之辭。
	徯,戶禮翻,待也。
	後,君也。
}
今燕虐其民,王往而征之,民以爲將拯己於水火之中也,|{
	拯,上舉也,援也,救也,助也,音之淩翻。
}
簞食壺漿以迎王師。
	若殺其父兄,系累其子弟,|{
	趙岐曰:系累,縛結 也。
	系戶計翻。
	累,力追翻。
}
毀其宗廟,遷其重器。|{
	重器,國之鎮寶。
}
如之何其可也!天下固畏齊之強也,今又倍地|{
	齊並燕則地倍其舊。
	燕,因肩翻。
}
而不行仁政,是動天下之兵也。
	王速出令。
	反其旄倪,|{
	令,力政翻。
	趙岐曰:旄,老旄;倪,弱小。
	陸德明曰:倪,謂翳倪小兒也。
	記曲禮曰:八十、九十曰耄,注云:耄,惛忘也。
	旄,讀曰耄。
	倪,五兮翻。
	翳,與繄同,音煙兮翻。
}
止其重器,謀于燕衆,置君而後去之,則猶可及止也。"齊王不聽。


  已而燕人叛。|{
	是時燕人雖未立太子平,固已相帥叛齊矣。
}
王曰:"吾甚慚於孟子。"
陳賈曰:"王無患焉。"
乃見孟子,曰:"周公何人也?"
曰:"古聖人也。"
陳賈曰:"周公使管叔監商,|{
	古殷,商通稱,商者,以始封爲國號,殷者,以都亳爲國號。
	按孟子,陳賈只云"監殷",今通鑒云"監商",避宋廟諱也。
	監,古銜翻。
}
管叔以商畔也。
	周公知其將畔而使之與?"|{
	畔,與叛同。
	與,讀曰歟,下同。
}
曰:“不知也。”
陳賈曰:"然則聖人亦有過與?"
曰:“周公,弟也,管叔,兄也,
	周公之過不亦宜乎!
	且古之君子,過則改之;
	今之君子,過則順之。
	古之君子,其過也如日月之食,民皆見之;
	及其更也,|{
	更,工衡翻,更改。
}
	民皆仰之。
	今之君子,豈徒順之,又從爲之辭!"

  4 是歲,齊宣王薨,子閔王地立。|{
	閔,讀曰閔。
}

  二年(戊申,西元前三一三年)

  1 秦右更疾伐趙,|{
	右更,秦爵第十四。
	師古曰:左、右、中更,皆主領更卒而部其役使也。
	更,工衡翻。
}
拔藺,虜其將莊豹。|{
	莊姓有出於宋者,左傳所謂戴、武、莊之族是也;有出於楚者,楚莊王之後,莊蹻是也。
	齊之莊暴,楚之莊辛,蒙之莊周,與此莊豹,其時適相先後,莫能審其所自出。
}

  2 秦王欲伐齊,患齊、楚之從親,|{
	從,子容翻。
}
乃使張儀至楚,說楚王曰:“大王誠能聽臣,閉關絕約于齊,|{
	說,式芮翻。
	閉關者,古之列國各置關尹,敵國賓至,關尹以告,則行理以節逆之。
	閉關則距絕其使,不爲通也。
	使,疏吏翻。
}
臣請獻商於之地六百里,使秦女得爲大王箕帚之妾,|{
	於,如字。
	箕帚之妾,猶言備灑掃也。
	帚,止酉翻,彗也。
}
秦、楚嫁女娶婦,長爲兄弟之國。”
楚王說而許之。|{
	說,讀曰悅。
}
羣臣皆賀,陳軫獨吊。|{
	陳姓出於舜,周武王封舜後於陳,子孫以國爲氏。
}
王\textbf{\large{怒}}曰:"寡人不興師而得六百里地,何弔也?"
對曰:"不然。
	以臣觀之,商於之地不可得而齊、秦合,齊、秦合則患必至矣。" 王曰:"有說乎?"對曰:"夫秦之所以重楚者,以其有齊也。|{
	夫,音扶,發語辭。
}
今閉關絕約于齊則楚孤,秦奚貪夫孤國而與之商於之地六百里!張儀至秦,必負王。
	是王北絕齊交,西生患于秦也,|{
	楚東北接齊,西接秦。
}
兩國之兵必俱至。
	爲王計者,不若陰合而陽絕于齊,使人隨張儀,苟與吾地,絕齊未晚也。”
	王曰:"願陳子閉口,毋複言,以待寡人得地!"|{
	毋,音無,毋者,禁止之辭。
	複,扶又翻,再又也。
}
乃以相印授張儀,|{
	相,息亮翻。
}
厚賜之。
遂閉關絕約于齊,使一將軍隨張儀至秦。|{
	班固百官表:將軍,週末官,秦、漢因之。
}

  張儀詳墮車,|{
	詳,讀曰佯,詐也。
}
【章:乙十一行本正作"佯"。】不朝三月。|{
	朝,直遙翻。
}
楚王聞之,曰:"儀以寡人絕齊未甚邪?"|{
	邪,餘遮翻。
	邪,疑辭也。
}
乃使勇士宋遺借宋之符,北罵齊王。|{
	旣閉關絕約,則齊、楚之信使不通,故使宋遺借宋符以至齊。
	宋,姓也。
	周武王封微子于宋,子孫以國爲氏。
}
齊王大怒,折節以事秦,|{
	折,而設翻。
}
齊、秦之交合。|{
	儀歸而詐疾,待齊、秦之交合乃朝。
}
張儀乃朝,見楚使者曰:"子何不受地?從某至某,廣袤六裡。"|{
	朝,直遙翻。
	使,疏吏翻。
	東西曰廣,南北曰袤。
	廣,古曠翻,又讀如字。
	袤,音茂。
}
使者怒,還報楚王。|{
	還,從宣翻,又音如字。
}
楚王大怒,欲發兵而攻秦。
陳軫曰:
	“軫可發口言乎?
	攻之不如因賂之以一名都,與之並力\fbox{\xout{\small{}兵}};
	而攻齊,是我亡地于秦,取償于齊也。|{
	償,辰羊翻,報也。
}
今王已絕于齊而責欺于秦,是吾合齊、秦之交而來天下之兵也,國必大傷矣!"楚王不聽,使屈匄帥師伐秦。|{
	屈,姓也,音九勿翻,匄,居大翻。
	帥,讀曰率。
}
秦亦發兵使庶長章擊之。|{
	長,知丈翻。
	按史記樗裡子傳,庶長章,姓魏。
}


  三年(己酉,西元前三一二年)

  1 春,秦師及楚戰於丹陽,|{
	索隱曰:此丹陽在漢中。
	劉昭曰:南郡枝江縣有丹陽聚,卽秦破楚處。
	李輿地紀勝曰:丹陽在今歸州秭歸縣東八裡屈沱楚王城是也。
	余按楚遺屈匄伐秦,秦發兵逆擊之,枝江之丹陽則距郢逼近,秭歸之丹陽則不當秦、楚之路。
	索隱因下文遂取漢中,卽謂丹陽在漢中,皆非也。
	此丹陽謂丹水之陽。
	班志:丹水出上洛塚嶺山,東至析入鈞水,其水蓋在弘農丹水、析兩縣之間,武關之外也。
	秦、楚交戰當在此水之陽。
	楚師旣敗,秦師乘勝取上庸路西入以收漢中,其勢易矣。
	索,山客翻。與埴同。
	屈,九勿翻。
	塚,知隴翻。
	易,弋豉翻。
}
楚師大敗,斬甲士八萬,虜屈匄及列侯、執圭七十餘人,|{
	執圭,楚爵也,執圭而朝者也。
}
遂取漢中郡。|{
	自沔陽、成固至新城、上庸,時皆漢中郡之地。
	釋名曰:郡,羣也,人所羣聚也。
	黃義仲十三州記曰:郡之言君也。
	改公侯之封而言君者,至尊也。
	今郡字,“君”在其左,“邑”在其右,君爲元首,邑以載民,故取名於君,謂之郡。
}
楚王悉發國內兵以複襲秦,|{
	複,扶又翻。
}
戰于藍田,|{
	班志,藍田縣屬京兆,秦孝公置。
	史記正義曰:藍田縣在雍州東南八十裡。
	從藍田關入藍田縣,時楚襲秦深入。
}
楚師大敗。
	韓、魏聞楚之困,南襲楚,至鄧。|{
	鄧,春秋鄧國之地。
	班志,鄧縣屬南陽郡。
	杜預曰:潁川召陵縣西有鄧城。
	括地志曰:故鄧城在豫州郾陵縣東三十五裡,所謂在古召陵西十裡者也。
	召,讀曰邵。
}
楚人聞之,乃引兵歸,割兩城以請平于秦。


  2 燕人共立太子平,是爲昭王。|{
	燕,因肩翻。
}
昭王于破燕之後\fbox{\xout{卽位}}。|{
	言燕國爲齊所破,已承其後也。
}
吊死問孤,與百姓同甘苦,卑身厚幣以招賢者。
	謂郭隗曰:"齊因孤之國亂而襲破燕,孤極知燕小力少,|{
	臧文仲曰:列國有凶稱孤,禮也。
	杜預曰:列國諸侯無凶則稱寡人。
	郭姓出於周之虢公,世亦謂虢公爲郭公。
	隗,五罪翻。
	少,始紹翻。
}
不足以報;然誠得賢士與共國,以雪先王之恥,|{
	謂燕王噲破國之恥。
	噲,苦夬翻。
}
孤之願也。
	先生視可者,得身事之!"郭隗曰:"古之人君有以千金使涓人求千里馬者,|{
	春秋以來,諸侯之國有涓人,秦、漢之間有中涓。
	師古曰:涓,潔也。
	言其在中主知潔清灑掃之事,蓋王之親舊左右也。
	應劭曰:涓人如謁者。
	涓,古玄翻。
	灑,所賣翻;掃,所報翻;又皆音如字。
}
馬已死,買其首五百金而返。
	君大怒,涓人曰:『死馬且買之,況生者乎!馬今至矣。』不期年,千里之馬至者三。|{
	期,讀曰朞。
}
今王必欲致士,先從隗始,況賢於隗者,豈遠千里哉!"|{
	言燕王若加禮于郭隗,則四方之賢士聞之,將不以千里爲遠而來。
}
於是昭王爲隗改築宮而師事之。
	於是士爭趣燕:|{
	爲,於偽翻。
	趣,七喻翻。
}
樂毅自魏往,劇辛自趙往。|{
	劇,竭戟翻。
	劇,姓;辛,名。
	劇姓莫知其所自出。
	班志,北海郡有劇縣,蓋其先以縣爲姓也。
}
昭王以樂毅爲亞卿,任以國政。|{
	爲燕用樂毅破齊張本。
}

  3 韓宣惠王薨,子襄王倉立。


  四年(庚戌,西元前三一一年)

  1 蜀相殺蜀侯。|{
	相,息亮翻。
	蜀相,蓋陳莊也。
}

  2 秦惠王使人告楚懷王,請以武關之外易黔中地。|{
	武關,左傳之少習,地在漢弘農郡析縣西百七十裡,道通南陽。
	晉太康地志曰:武關當冠軍西。
	括地志曰:武關在商州上洛縣東。
	武關之外,蓋秦丹、析、商於之地。
	黔,音琴。
	少,始照翻。
	冠,工玩翻。
	於,音如字。
}
楚王曰:"不願易地,願得張儀而獻黔中地。"
張儀聞之,請行。
王曰:"楚將甘心於子,|{
	楚王以墮張儀之詐,故欲甘心焉。
}
柰何行?"張儀曰:"秦強楚弱,大王在,楚不宜敢取臣。
	且臣善其嬖臣靳尚,|{
	嬖,匹計翻,又卑義翻。
	靳,居焮翻,姓也。
}
靳尚得事幸姬鄭袖,|{
	鄭,以國爲氏。
	"袖",戰國策作"袖",古字也。
}
袖之言,王無不聽者。"
遂往。
楚王囚,將殺之。
靳尚謂鄭袖曰:“秦王甚愛張儀,將以上庸六縣及美女贖之。|{
	上庸,春秋庸國。
	班志,上庸縣屬漢中郡。
	史記正義:上庸縣,今房州。
	宋白曰:今房州竹山縣古城,卽漢上庸縣。
}
王重地尊秦,秦女必貴而夫人斥矣。”
於是鄭袖日夜泣于楚王曰:“臣各爲其主耳。|{
	爲,幹偽翻。
}
今殺張儀,秦必大怒。
	妾請子母俱遷江南,毋爲秦所魚肉也。"王乃赦張儀而厚禮之。
	張儀因說楚王曰:"夫爲從者無以異於驅羣羊而攻猛虎,不格明矣。|{
	說,式芮翻。
	夫,音扶。
	從,子容翻。
	格,當也。
	劉伯莊曰:格,各額翻,其字宜從"手"。
	餘據字書,格,擊也,鬭也,從"木"亦通。
}
今王不事秦,秦劫韓驅梁而攻楚,則楚危矣。
秦西有巴、蜀,治船積粟,浮岷江而下,|{
	治,直之翻。
	江水出蜀郡湔氐道之岷山,故謂之岷江。
	釋名曰:江,共也;小流入其中,所公共也。
}
一日行五百餘裡,不至十日而拒【章:十二行本"拒"作"距";乙十一行本同。】捍關,|{
	徐廣曰:巴郡魚複縣有捍關。
	史記正義曰:在峽州巴山縣界。
	捍,寒旦翻。
}
捍關驚則從境以東盡城守矣,|{
	境,楚境也。
	捍關,楚之西境,從境以東,謂捍關以東也。
}
黔中、巫郡非王之有。|{
	黔,巨今翻。
	班志,巫縣屬南郡。
	酈道元曰:縣故楚之巫郡。
	杜佑曰:今歸州巴東縣是也。
}
秦舉甲出武關,則北地絕。|{
	北地,楚北境之地,陳、蔡、汝、潁是也。
}
秦兵之攻楚也,危難在三月之內。|{
	難,乃旦翻。
}
而楚待諸侯之救在半歲之外,夫待弱國之救,忘強秦之禍,此臣所爲大王患也。|{
	夫,音扶。
	爲,於偽翻。
}
大王誠能聽臣,臣請令秦、楚長爲兄弟之國,無相攻伐。”|{
	令,力丁翻。
}
楚王已得張儀而重出黔中地,|{
	重,難也。
	以地爲重,意難割棄之。
}
乃許之。


張儀遂之韓,說韓王曰:“韓地險惡山居,|{
	之,如也,自楚如韓也。
	韓有宜陽、成皋,南盡魯陽,皆山險之地。
	說,式芮翻。
}
五穀所生,非菽而麥,|{
	菽,式竹翻,豆也。
}
國無二歲之食,見卒不過二十萬。|{
	見卒,見在之兵。
	見,賢遍翻。
}
秦被甲百余萬。|{
	被,皮義翻。
}
山東之士被甲蒙胄以會戰,秦人捐甲徒裼以趨敵,|{
	胄,今謂之兜鍪。
	捐,與專翻,棄也。
	徒,徒行也。
	裼,音錫,袒也。
	趨,七喻翻。
	鍪,音牟。
}
左挈人頭,右挾生虜。
	夫戰孟賁、烏獲之士以攻不服之弱國,|{
	挾,戶頰翻。
	孟賁、烏獲,古之勇士。
	賁,音奔。
}
無異垂千鈞之重於鳥卵之上,必無幸矣。|{
	三十斤爲鈞。
	必無幸矣,言無幸而獲全之理。
}
大王不事秦,秦下甲據宜陽,塞成皋,|{
	下,遐稼翻。
	塞,悉則翻。
}
則王之國分矣,鴻台之宮,桑林之苑,非王之有也。
爲大王計,莫如事秦以攻楚,以轉禍而悅秦,計無便於此者。”
韓王許之。


張儀歸報,秦王封以六邑,號武信君。
複使東說齊王曰:“從人說大王者|{
	複,扶又翻。
	從人,合從之人也。
	從,子容翻。
	說,式芮翻。
}
必曰:『齊蔽于三晉,地廣民衆,兵強士勇,雖有百秦,將無柰齊何。』
大王賢其說而不計其實。
今秦、楚嫁女娶婦,爲昆弟之國;
韓獻宜陽;梁效河外;|{
	河外,秦蓋以河東爲河外,梁則以河西爲河外,張儀以秦言之也。
}
趙王入朝,割河間以事秦。|{
	朝,直遙翻。
	河間,趙地。
	漢文帝二年,分爲河間國。
	應劭曰:在兩河之間。
	唐爲瀛州。
}
大王不事秦,秦驅韓、梁攻齊之南地,|{
	漢泰山、城陽,齊南境之地也。
}
悉趙兵,渡清河,指博關,臨菑、卽墨非王之有也!|{
	博關在濟州西界之博陵。
	史記正義曰:博關在博州。
	趙兵從貝州渡清河指博關,則漯河以南臨菑、卽墨危矣。
	濟,子禮翻。
	漯,托合翻。
}
國一日見攻,雖欲事秦,不可得也!”
齊王許張儀。

張儀去,西說趙王曰:“大王收率天下以擯秦,秦兵不敢出函谷關十五年。|{
	擯,必刃翻。
	事見上卷顯王三十六年。
}
大王之威行于山東,敝邑恐懼,|{
	春秋以來,列國交聘,行人率自稱其國曰敝邑。
}
繕甲厲兵,力田積粟,愁居懾處,不敢動搖,|{
	懾,之涉翻,怖也,心伏也,失常也,失氣也。
	處,昌呂翻。
}
唯大王有意督過之也。|{
	師古曰:督過,視責也。
	索隱曰:督者,正其事而責之;督過,是深責其過也。
}
今以大王之力,舉巴、蜀,|{
	事見慎靚王五年。
}
並漢中,|{
	事見上二年。
}
包兩周,|{
	元年服韓、魏,則包兩周矣。
}
守白馬之津。|{
	班志,白馬縣屬東郡。
	水經注:白馬津在白馬城之西北。
	白馬城,唐爲滑州治所。
	開山圖曰:白馬津東可二十許裡,有白馬山,山上常有白馬羣行,悲鳴則河決,馳走則山崩,後人因以名縣及津。
	按通鑒不語怪,今此注亦近于怪,姑以廣異聞耳。
}
秦雖僻遠,然而心忿含怒之日久矣。
	今秦有敝甲凋兵軍于澠池,|{
	敝,敗惡也,凋,瘁也,半傷也。
	敗甲凋兵,謙其辭,言軍于澠池,則張其勢以臨趙矣。
	康曰:澠池,趙邑。
	余據趙與韓、魏接境,韓有野王、上党,魏有河東、河內,而澠池則秦地也,漢爲縣,屬弘農郡,趙安能越韓、魏而有之!康說非是。
	澠,莫善翻;又莫忍翻。
}
願渡河,逾漳,據番吾,|{
	言欲自澠池北渡河,又自此東逾漳水而進據番吾,此亦張聲勢以臨趙也。
	番吾,卽漢常山郡之蒲吾縣也。
	劉昭注曰:史記番吾君,杜預云:晉之蒲邑也。
	此說非。
	括地志:番吾故城,在恒州房山縣東二十裡。
	番,音婆,又音盤。
}
會邯鄲之下,願以甲子合戰,正殷紂之事。|{
	武王伐紂,癸亥陳于商郊,甲子昧爽,紂帥其旅若林,會於牧野,前徒倒戈,攻其後以北,遂以勝殷殺紂。
	張儀引以懼趙,其有所侮而動,亦已甚矣。
	邯鄲,趙都,音寒丹。
}
謹使使臣先聞左右。|{
	使臣,上疏吏翻。
}
今楚與秦爲昆弟之國,而韓、梁稱東藩之臣,齊獻魚鹽之地,|{
	齊東瀕於海,海濱廣斥,魚鹽所出也。
	此時齊未嘗獻地于秦,張儀駕說以恐動趙耳。
}
此斷趙之右肩也。
	夫斷右肩而與人鬥,|{
	夫,音扶。
	斷,丁管翻。
}
失其党而孤居,求欲毋危得乎!今秦發三將軍,其一軍塞午道,|{
	索隱曰:午道當在趙之東,齊之西。
	午道,地名也。
	鄭玄云:一縱一橫爲午,謂交道也。
	塞,悉則翻。
}
告齊使渡清河,軍於邯鄲之東,|{
	邯鄲,音寒丹。
}
一軍軍成皋,驅韓、梁軍於河外,|{
	史記正義曰:河外,謂鄭滑州,北臨河。
	餘謂此河外,亦因趙而言之。
}
一軍軍于澠池,約四國爲一以攻趙,趙服必四分其地。|{
	言秦約齊、韓、魏四分趙地。
}
臣竊爲大王計,莫如與秦王面相約而口相結,常爲兄弟之國也。"趙王許之。|{
	當時趙于山東最強,且主從約,張儀說之,亦費辭矣。
}

  張儀乃北之燕,|{
	燕,因肩翻。
}
說燕王曰:"今趙王已入朝,效河間以事秦。|{
	張儀自趙至燕,借此氣勢而爲是虛言以動燕耳。
	朝,直遙翻。
}
大王不事秦,秦下甲云中、九原。|{
	虞氏記曰:趙自五原河曲築長城,東至陰山,又於河西造大城,一箱崩不就,乃改卜陰山河曲而禱焉,晝見羣鵠游於云中,徘徊經日,見大光在其下,乃卽其處築城,今云中城是也。
	餘謂此亦語怪,酈道元爲後魏書之耳。
	宋白曰:勝州榆林縣界有云中古城,趙武侯所築,秦置云中郡,唐爲單于都護府。
	班志:九原縣屬五原郡。
	漢之五原,卽秦之九原郡也。
	唐爲豐、鹽等州之地。
	宋白曰:唐豐州治九原縣。
	按云中九原,皆在燕之西,秦自上郡朔方下兵則可至。
	史記正義曰:古云中、九原郡皆在勝州。
	云中郡故城在榆林東北四十裡。
	九原郡故城在勝州西界,今連穀縣是。
	下,遐稼翻。
	元爲,於偽翻。
}
驅趙而攻燕。
	則易水、長城非大王之有也!|{
	水經注:易水出涿郡故安縣閻鄉西山,東屆關城西南,卽燕長城門也。
	易水又曆長城而東過范陽、容城、安次、泉州縣南而東入海。
}
且今時齊、趙之于秦,猶郡縣也,不敢妄舉師以攻伐。
	今王事秦,長無齊、趙之患矣。"|{
	以利動之。
}
燕王請獻常山之尾五城以和。|{
	常山,卽北嶽恒山也。
	漢文帝諱恒,改曰常山,置常山郡。
	班志,常山在常山郡上曲陽縣西北,其尾則燕之西南界。
}

  張儀歸報,未至咸陽,秦惠王薨,子武王立。|{
	索隱曰:武王,名蕩。
}
武王自爲太子時,不說張儀;|{
	說,讀曰悅。
}
及卽位,羣臣多毀短之。|{
	毀短,訾毀而數其短也。
}
諸侯聞儀與秦王有隙,|{
	隙,乞逆翻,怨隙也,釁隙也。
	物之有罅釁者爲有隙,人之與人有怨者亦爲有隙。
}
皆畔衡,複合從。|{
	衡,讀曰橫。
	從,子容翻。
	以此觀之,此時六國之勢,利在合從,而從張儀連衡者,畏秦而搖於儀之說耳。
}

  五年(辛亥,西元前三一零年)

  1 張儀說秦武王曰:"爲王計者,東方有變,|{
	韓、魏皆在秦之東。
	說,式芮翻。
}
然後王可以多割得地也。
	臣聞齊王甚憎臣,臣之所在,齊必伐之。
	臣願乞其不肖之身以之梁,|{
	不肖,謙言無所肖似也。
	魏都大樑。
}
齊必伐梁,齊、梁交兵而不能相去,|{
	言兵交不解,各欲去而不能也。
}
王以其間伐韓,|{
	間,居莧翻,間隙也,又居閑翻,中間也。
}
入三川,挾天子,案圖籍,此王業也!"|{
	張儀欲傾周而爲秦;始終以此說爲主。
	挾,戶頰翻。
}
王許之。
	齊王果伐梁,梁王恐。
	張儀曰: "王勿患也!|{
	言勿以爲患。
}
請令齊罷兵。"|{
	令,盧經翻,使也;下同。
}
乃使其舍人之楚,借使謂齊王曰:|{
	之,往也,如也。
	不敢徑遣人使齊,而往楚借使,借使,言借楚人以爲使。
	借,子夜翻;康資昔切。
	使,疏吏翻。
}
"甚矣王之託儀于秦也!"齊王曰:"何故?"楚使者曰:"張儀之去秦也固與秦王謀矣,欲齊、梁相攻而令秦取三川也。
	今王果伐梁,是王內罷國而外伐與國,|{
	罷,讀曰疲。
}
而信儀于秦王也。"齊王乃解兵還。|{
	還,從宣翻,又如字。
}
張儀相魏一歲,卒。|{
	相,息亮翻。
	卒,子恤翻。
}

  儀與蘇秦皆以縱橫之術游諸侯,致位富貴,天下爭慕效之。|{
	縱,子容翻。
}
又有魏人公孫衍者,號曰犀首,亦以談說顯名。|{
	說,式芮翻。
}
其餘蘇代、蘇厲、周最、樓緩之徒,紛紜徧於天下,務以辯詐相高,不可勝紀,|{
	姓譜曰:周姓本自周平王子,別封汝川,人謂之周家,因氏焉。
	一云:以赧王爲秦所滅,黜爲庶人,百姓稱爲周家,因氏焉。
	余按商有太史周任,謂爲周姓所自出,夫豈不可!又赧王于時未滅,不可謂周最出於赧王。
	樓姓,夏少康之裔,周封爲東樓公,子孫因氏焉。
	師古曰:紛紜,興作貌,又物多而亂貌。
	勝,音升。
	赧,奴版翻。
	夏,戶雅翻。
	少,始照翻。
	裔,苗裔。
}
而儀、秦、衍最著。|{
	著者,顯著于時。
}

  孟子論之曰:或謂:"公孫衍張儀豈不大丈夫哉,一怒而諸侯懼,安居而天下熄?"|{
	熄,滅也,火滅爲熄。
	此言天下兵革之事熄滅也。
}
孟子曰:"是惡足爲大丈夫哉!|{
	惡,音烏。
}
君子立天下之正位,行天下之正道,得志則與民由之,不得志則獨行其道,富貴不能淫,貧賤不能移,威武不能詘,|{
	詘,與屈同。
}
是之謂大丈夫。"


揚子法言曰:或問:“儀、秦學乎鬼谷術而習乎縱橫言,安中國者各十餘年,是夫?”|{
	夫,音扶。
}
曰:“詐人也,聖人惡諸。”|{
	惡,烏路翻。
}
曰:“孔子讀而儀、秦行,|{
	謂讀孔子之言而行儀、秦之事。
}
何如也?”
曰:“甚矣鳳鳴而鷙翰也!”|{
	翰,侯旰翻,又侯安翻,羽翰。
}
“然則子貢不爲歟?”
曰:“亂而不解,子貢恥諸。|{
	太史公曰:子貢一出,存魯,亂齊,破吳,強晉而霸越。
	溫公曰:考其年與事皆不合,蓋六國遊說之士托爲之辭,太史公不加考訂,因而記之;
	\underline{揚子雲}亦據太史公書發此語也。
	說,式芮翻。
}
說而不富貴,儀、秦恥諸。”|{
	說,式芮翻。
}
或曰:“儀、秦其才矣乎,跡不蹈已?”|{
	宋鹹曰:蹈,踐也;言儀、秦之才術超卓,自然不踐循舊人之跡。
	踐,慈演翻。
}
曰:“昔在任人,帝而難之。|{
	書舜典:而難任人。
	孔安國注云:任,佞也;難,拒也;言佞人則斥遠之。
	任,音壬。
	難,乃旦翻。
}
不以才乎?
才乎才,非吾徒之才也!”

2 秦王使甘茂誅蜀相莊。|{
	四年,蜀相殺蜀侯,秦武王故誅之。
	史記"莊"作"壯"。
	案秦紀,秦旣得蜀,使陳莊相蜀;從"莊"爲是。
}

3 秦王、魏王會于臨晉。|{
	班志,臨晉縣屬馮翊,故大荔也,秦取之,更名臨晉。
	應劭曰:臨晉水,故名。
	臣瓚曰:晉水在河之東,此縣在河之西,不得臨晉水。
	舊說,秦築高壘以臨晉國,故曰臨晉。
	章懷太子賢曰:臨晉故城,在今同州朝邑縣西南。
	余按唐書地理志,蒲州有臨晉縣。
	宋白曰:漢臨晉縣在今臨晉縣東南十八裡,故解城是也。
	後魏改爲北解縣。
	周省。
	隋分猗氏縣,置桑泉縣。
	唐天寶十二載,改臨晉縣。
	天寶之改縣,必有所據,則應劭臨晉水之說,無可厚非。
	秦之臨晉在河西,臣瓚、章懷之說皆是也。
	更,工衡翻。
	應,乙陵翻。
	瓚,藏旱翻。
	朝,直遙翻。
	解,戶買翻。
	載,祖亥翻。
}

4 趙武靈王納吳廣之女孟姚,|{
	吳姓,以國爲氏。
}
有寵,是爲惠后。|{
	孔穎達曰:后,後也,言其後于天子,亦以廣後胤也。
	戰國諸侯僭王,亦稱其夫人爲后。
}
生子何。|{
	爲立何而長子章爭國張本。
	長,知兩翻。
}

  六年(壬子,西元前三零九年)

1 秦初置丞相,以樗裡疾爲右丞相。|{
	應劭曰:丞者,承也;相者,助也。
	荀悅曰:秦本次國,命卿二人,故置左右丞相,無三公官。
	樗裡疾,秦惠王之弟也。
	高誘曰:疾居渭南之陰鄉,其裡有大樗樹,故號樗里子。
	相,息亮翻。
	樗,丑於翻。
	誘,羊久翻。
}

七年(癸丑,西元前三零八年)

1 秦、魏會于應。|{
	左傳曰:邘、晉、應、韓,武之穆也。
	杜預注云:應國在襄陽城父縣西。
	余按襄陽無城父縣。
	後漢志,潁川父城縣西南有應鄉,古應國也。
	括地志曰:故應城因應山爲名。
	古之應國在汝州魯山縣東三十裡。
	應,乙陵翻。
	邘,音於。
}

2 秦王使甘茂約魏以伐韓,而令向壽輔行。
甘茂\fbox{至魏}令向壽還,謂王曰:“魏聽臣矣,|{
	令,盧經翻,使也。
	向,式讓翻,姓也。
	姓譜:向姓本自宋文公枝子向文旰,旰孫戌以王父字爲氏。
	余按左傳,向戌本出於宋桓公。
	孟子爲齊卿,出吊于滕,王使王驩爲輔行。
	趙岐注曰:輔行,副使也。
	旰,音幹。
	戌,音恤。
	傳,直戀翻。
	使,疏吏翻。
}
然願王勿伐!”
王迎甘茂於息壤而問其故。|{
	柳宗元曰:地長隆然而起,夷之而益高者爲息壤。
	異書有云:鯀竊帝之息壤以堙洪水。
	意者此所謂息壤,蓋以地長得名。
	長,知兩翻。
}
對曰:“宜陽大縣,其實郡也。|{
	杜佑曰:春秋時列國相滅,多以其地爲縣,則縣大而郡小,
	故趙鞅曰:“上大夫受縣,下大夫受郡。”至於戰國,則郡大而縣小矣,
	故甘茂曰:“宜陽大縣。其實郡也。”
	漢官儀曰:凡郡:或以列國,陳、魯、齊、吳是也;
	或以舊邑,長沙、丹陽是也;
	或以山陵,泰山,山陽是也;
	或以川原,西河,河東是也;
	或以所出,金城城下得金,酒泉泉味如酒、豫章樟樹生庭、雁門雁之所育是也;
	或以號令,夏禹合諸侯,大計東冶之山會計,因名會稽是也。
	令,力正翻。
	名會,古外翻。
}
今王倍數險,行千里,攻之難。|{
	倍,與背同,音蒲妹翻。
	數險,謂函穀及三崤之險。
}
魯人有與曾參同姓名者殺人,人告其母,其母織自若也。|{
	參,所金翻,一音七南翻。
}
及三人告之,其母投杼下機,逾牆而走。|{
	杼,直呂翻。
	說文曰:杼,機之持緯者,蓋今所謂梭。
	梭,蘇禾翻。
}
臣之賢不若曾參,王之信臣又不如其母,疑臣者非特三人,臣恐大王之投杼也。
魏文侯令樂羊將而攻中山,三年而拔之。|{
	事見一卷威烈王二十三年。
	令,音盧經翻。
	將,卽亮翻。
}
反而論功,文侯示之謗書一篋。|{
	謗,訕也,毀也。
	篋,竹笥也,音古頰翻。
}
樂羊再拜稽首曰:‘此非臣之功,君之力也!’|{
	稽首,首至地也。
	稽,音啟。
}
今臣,羇旅之臣也,|{
	甘茂,楚下蔡人,故云然。
	羇,居宜翻,寄也。
	旅,客也。
}
樗裡子、公孫奭挾韓而議之,王必聽之,|{
	樗,丑於翻。
	奭,施只翻。
	挾,戶頰翻。
}
是王欺魏王而臣受公仲侈之怨也。”|{
	公仲侈,韓相也。
}
王曰:"寡人弗聽也,請與子盟!"乃盟於息壤。
	秋,甘茂、庶長封帥師伐宜陽。|{
	長,知丈翻。
	帥,讀曰率。
}

  八年(甲寅,西元前三零七年)

  1 甘茂攻宜陽,五月而不拔。
	樗裡子、公孫奭果爭之。
	秦王召甘茂,欲罷兵。
	甘茂曰:"息壤在彼。"|{
	征前盟也。
}
王曰:"有之。"因大悉起兵以佐甘茂,|{
	佐,助也。
}
斬首六萬,遂拔宜陽。
	韓公仲侈入謝于秦以請平。|{
	請平,猶請和也。
}

  2 秦武王好以力戲,|{
	好,呼到翻。
}
力士任鄙、烏獲、孟說皆至大官。|{
	烏,姓也。
	春秋時,齊有大夫烏枝鳴。
	姓譜:孟姓,魯桓公之子仲孫之胤,仲孫爲三桓之孟,故曰孟氏。
	任,音壬。
	說,讀曰悅。
}
八月,王與孟說舉鼎,絕脈而薨;|{
	脈,莫獲翻。
	脈者,系絡臟腑, 其血理分行于支體之間,人舉重而力不能勝,故脈絕而死。
	按史記甘茂傳云:武王至周而卒于周。
	蓋舉鼎者,舉九鼎也。
	世家以爲龍文赤鼎。
	史記"脈"作 "臏"。
}
族孟說。|{
	族者,誅夷其族。
}
武王無子,異母弟稷爲質于燕,|{
	質,音致。
	燕,因肩翻。
}
國人逆而立之,|{
	逆,迎也。
}
是爲昭襄王。
	昭襄王母羋八子,|{
	羋,楚姓也。
	漢因秦制,嫡稱皇后,次稱夫人,又有美人、良人、八子、七子、長使、少使之號。
	美人爵視二千石,比少上造。
	八子視千石,比中更。
	羋,亡氏翻。
}
楚女也,實宣太后。


  3 趙武靈王北略中山之地,|{
	略地之師速而疾。
	杜預曰:略者,總攝巡行之名也。
}
至房子,|{
	班志,房子縣屬常山郡。
	史記正義曰:房子,今趙州縣。
	宋白曰:天寶元年改曰臨城。
}
遂至【章:十二行本"至"作"之";乙十一行本同;孔本同;張校同;退齋校同。】代,北至無窮,|{
	自代北出塞外,大漠數千里,故日無窮。
	戰國策,武靈王曰:"昔先君襄王與代交地,城境封之,名曰無窮之門,所以詔後而期遠也。"
	}
西至河,登黃華之上。|{
	史記正義曰:黃華,蓋黃河側之山名。
}
與肥義謀胡服騎射以教百姓,|{
	騎,奇寄翻。
}
曰:"愚者所笑,賢者察焉。
	雖驅世以笑我,胡地、中山,吾必有之!"遂胡服。


  國人皆不欲,公子成稱疾不朝。|{
	朝,直遙翻。
}
王使人請之曰:"家聽於親,|{
	親,謂父母。
}
國聽於君。
	今寡人作教易服而公叔不服,吾恐天下議己【章:十二行本"己"作"之";乙十一行本同;孔本同;張校同,退齋校同。】也。
	制國有常,利民爲本;從政有經,令行爲上。|{
	令,力政翻。
}
明德先論於賤,而從政先信於貴,|{
	德欲其下及,故先論於賤;卑賤者感其德,則德廣所及可知矣。
	法行自貴近始,故先信於貴;貴近者奉法,則法之必行可知矣。
}
故願慕公叔之義以成胡服之功也。"公子成再拜稽首曰:|{
	稽,音啟。
}
"臣聞中國者,聖賢之所教也,禮樂之所用也,遠方之所觀赴也,蠻夷之所則效也。
	今王舍此而襲遠方之服,|{
	則,法也。
	舍,讀曰舍。
	襲,重衣也。
}
變古之道,逆人之心,臣願王孰圖之也!"|{
	孰,古熟字,通。
}
【章:十二行本正作"熟";乙十一行本同;孔本同。】使者以報。
	王自往請之,|{
	使,疏吏翻。
}
曰:"吾國東有齊、中山,|{
	按趙都邯鄲,東接于齊,中山在其東北,故史記趙世家載武靈王之言曰:"吾國東有河薄落之水,與齊、中山同之。"蓋河、薄落之水在趙之東,與齊、中山同此地險也。
}
北有燕、東胡,西有樓煩、秦、韓之邊。|{
	史記正義曰:營州之境,卽東胡、烏丸之地。
	林胡、樓煩,卽嵐、勝之北也。
	班志:雁門郡樓煩縣。
	應劭注云:故樓煩胡地。
	嵐、勝以南,石州離石、藺等,七國時趙邊也,與秦隔河。
	晉、洺、潞、澤等州,皆七國時韓地,趙之西邊地。
	燕,因肩翻。
}
今無騎射之備,則何以守之哉?|{
	騎,奇寄翻
	}
先時中山負齊之強兵,侵暴吾地,系累吾民,|{
	先,悉薦翻。
	累,力追翻。
}
引水圍鄗,微社稷之神靈,則鄗幾於不守也。|{
	鄗,趙邑,漢光武改爲高邑,隋、唐爲柏鄉縣地,唐屬趙州。
	鄗,呼各翻。
	幾,居衣翻。
}
先君醜之,|{
	以爲趙國之醜。
}
故寡人變服騎射,欲以備四境之難,|{
	難,乃旦翻。
}
報中山之怨。
	而叔順中國之俗,惡變服之名,|{
	惡,烏路翻。
}
以忘鄗事之醜,非寡人之所望也!"公子成聽命,乃賜胡服;明日服而朝。|{
	朝,直遙翻。
}
於是始出胡服令|{
	令,力政翻。
}
而招騎射焉。


  九年(乙卯,西元前三零六年)

  1 秦昭王使向壽平宜陽,|{
	平,正也,和也。
	正宜陽之疆界而和其民人也。
	向,式亮翻。
}
而使樗裡子、甘茂伐魏。
	甘茂言于王,以武遂複歸之韓。|{
	史記正義曰:武遂本屬韓,近平陽。
	楚世家云:韓先王之墓在平陽,武遂去之七十裡。
	去年秦拔宜陽,因涉河城武遂,今複歸之韓。
	複,音如字。
}
向壽、公孫奭爭之,不能得,|{
	向,式讓翻。
}
由此怨讒甘茂。
	茂懼,輟伐魏蒲阪,亡去。|{
	班志,蒲阪縣屬河東郡,舊曰蒲。
	應劭曰:秦始皇東巡,見長阪,因加"阪"云。
	括地志:蒲阪故城,在蒲州河東縣南五裡。
	阪,音反。
}
樗裡子與魏講而罷兵,|{
	講,和也。
}
甘茂奔齊。


  2 趙王略中山地,至寧葭;|{
	水經注:衡漳水東北曆下博城西,又西逕樂鄉縣故城南,又東引葭水注之。
	葭,音加。
}
西略胡地,至榆中。|{
	水經注:諸次水出上郡諸次山,其水東逕榆林塞,世又謂之榆林山,卽漢書所謂"榆溪舊塞"者。
	自溪西去,悉榆柳之藪,緣曆沙陵,屆龜茲縣西出,故云廣長榆也。
	王恢曰"樹榆爲塞",謂此。
	蘇林以爲榆中在上郡,非也。
	按始皇本紀:西北逐匈奴,自榆中並河以東,屬之陰山。
	然榆中在金城東五十許裡,陰山在朔方東,以此推之,不得在上郡。
	余謂蘇林之說固未爲盡,而道元所謂榆中在金城東五十許裡亦非也。
	據衛青取河南地,案榆溪舊塞,正在唐麟、勝二州界,其西則接古上郡之境。
	況諸次水出上郡,逕榆林塞入河,則榆中在上郡之東明矣,諸次水無西流至金城、榆中之理。
	夷考其故,道元特以班志金城郡有榆中縣,遂牽合以爲說,不知此一節之誤尤甚于蘇林也。
	史記正義曰:榆中,勝州北河北岸也。
	杜佑曰:勝州榆林郡南卽秦榆林塞。
}
林胡王獻馬,|{
	如淳曰:林胡,卽儋林。
	余謂此胡種落依阻林薄,因曰林胡。
	儋,都甘翻。
	種,章勇翻。
}
歸,使樓緩之秦,仇液之韓。
	王賁之楚,|{
	歸,謂趙王自略中山歸也。
	仇,姓也。
	春秋時,宋有大夫仇牧。
	液,音亦。
	之,往也,如也。
	賁,音奔;康曰:離之父,翦之子。
	余按離父、翦子,秦將也;此王賁乃趙人,康說非是。
	將,卽亮翻。
}
富丁之魏,|{
	富,姓也。
	春秋時,周有大夫富辰。
}
趙爵之齊;代相趙固主胡,致其兵。|{
	相,息亮翻。
	致者,使之至也。
}

  3 楚王與齊、韓合從。|{
	楚與齊、韓合從,尋卽倍之,適足致齊、韓之兵耳。
	從,子容翻。
}

  十年(丙辰,西元前三零五年)

  1 彗星見,|{
	彗星,世所謂掃星,本類星,末類彗,小者數寸,長或竟天,見則兵起,主掃除,除舊佈新。
	唐史臣曰:彗體無光,傅日以爲光,故夕見則東指,晨見則西指,或長或短,光芒所及則爲災。
	又曰:孛星,彗之屬也,偏指曰彗,氣四出曰孛。
	孛者孛孛,非常惡氣之所生,災甚於彗。
	天文書謂五星之精爲妖,歲星流爲蒼彗,熒惑、填星散爲赤彗、黃彗,太白、辰星變爲白彗、黑彗。
	彗,祥歲翻,又徐醉翻,又旋芮翻。
	見,賢遍翻。
	掃,所報翻。
	傅,讀曰附。
	孛,蒲內翻。
	妖,於遙翻。
	填,讀曰鎮。
}

  2 趙王伐中山,取丹丘、爽陽、鴻之塞,又取鄗、石邑、封龍、東垣。|{
	史記正義曰:丹丘,邢州縣。
	余按隋、唐志,邢州有內丘縣,漢之中丘縣也,未嘗有丹丘,不知其何據。"爽陽、鴻之塞,"史記作"華陽、鴟之塞"。
	括地志曰:北嶽別名曰華陽臺,卽常山也,在定州恒陽縣北百四十裡。
	徐廣曰:"鴟"作"鴻",鴻上故關,今名汝城,在定州唐縣東北六十裡。
	又有鴻上水,出唐縣北葛洪山,山接北嶽恒山,皆在定州界。
	班志,石邑縣屬常山郡,井陘山在西。
	括地志:石邑故城,在恒州鹿泉縣南三十五裡。
	封龍山,一名飛龍山,在恒州鹿泉縣南四十五裡,邑蓋因山爲名。
	洪氏隸釋載後漢所立白石碑云:常山國元氏縣界有封龍山。
	東垣,卽漢真定國之真定縣,漢高帝更名。
	史記正義曰:趙之東垣,在恒州真定縣南八裡,故常山城是也。
	鄗,呼各翻。
	垣,於元翻。
	華,戶化翻。
	恒,戶登翻,鴟,醜之翻。
	陘,音刑。
	更,工衡翻。
}
中山獻四邑以和。


  3 秦宣太后異父弟曰穰侯魏冉,同父弟曰華陽君羋戎;王之同母弟曰高陵君、涇陽君。
	魏冉最賢,|{
	秦封穰侯于陶,陶卽范蠡所居陶邑。
	孟康曰:陶卽定陶。
	班志,定陶縣屬濟陰郡。
	下云封于穰與陶;穰縣屬南陽郡,去定陶差遠。
	水經注曰:穰侯封於穰,益封于陶,其免相也,出之陶而卒,陶有穰侯塚。
	穰,音人羊翻。
	華陽,卽武王歸馬之地。
	水經注:洛水自上洛縣東北分爲二水,枝渠東北出爲門水,水東北曆陽華之山,卽華陽也。
	華,音戶化翻。
	羋,眉婢翻。
	相,息亮翻。
	卒,子恤翻。
	塚,知隴翻。
	班志,高陵縣屬馮翊,涇陽縣屬安定。
	杜佑曰:京兆涇陽縣乃秦封涇陽君之地。
	漢涇陽縣在今平涼郡界涇陽故城是也。
	宋白曰:雍州涇陽本秦舊縣。
	與杜佑同。
	索隱曰:高陵君,名顯,涇陽君,名悝。
	索,山客翻。
	悝,苦回翻。
}
自惠王、武王時,任職用事,武王薨,諸弟爭立,唯魏冉力能立昭王。|{
	惠王,卽惠文王。
	昭王,卽昭襄王
	}
昭王卽位,以魏冉爲將軍,衛咸陽。
	是歲,庶長壯及大臣、諸公子謀作亂。|{
	長,知丈翻。
}
魏冉誅之;及惠文后皆不得良死,|{
	惠文后,昭王嫡母也。
	死於正命曰良死。
}
悼武王后出居【章:十二行本"居"作"歸";乙十一行本同;孔本同;退齋校同。】于魏,|{
	悼武王后,卽秦武王后,昭王嫂也。
}
王兄弟不善者,魏冉皆滅之。
	王少,宣太后自治事,任魏冉爲政,威震秦國。|{
	少,詩照翻。
	治,直之翻。
	爲范睢間魏冉張本。
}

  十一年(丁巳,西元前三零四年)

  1 秦王、楚王盟于黃棘。|{
	史記正義曰:黃棘蓋在房、襄二州。
	余按班志,南陽郡有棘陽縣,應劭曰:縣在棘水之陽。
}
秦複與楚上庸。|{
	三年,秦敗楚師,虜屈匄,取楚上庸。
}

  十二年(戊午,西元前三零三年)

  1 彗星見。|{
	彗,祥歲翻,又徐醉翻,旋芮翻。
	見,賢遍翻
	}

  2 秦取魏蒲阪、晉陽、封陵,|{
	"晉陽",按史記世家作"陽晉",其地當在蒲阪之東,風陵之西,大河之陽,且本晉地也,故謂之陽晉,蘇秦所謂"衛陽晉之道",蓋以魏境有陽晉,故在衛境者稱"衛陽晉"以別之。
	括地志曰:晉陽故城,今名晉城,在蒲州虞鄉縣西。
	水經注:函谷關直北隔河有崇阜,巍然獨秀,世謂之風陵。
	酈道元所謂函穀,則潼關也。
	史記正義曰:封陵在蒲州。
	唐志:河中府河東縣南有風陵關。
	今若據括地志,則晉陽亦通
	}
又取韓武遂。|{
	九年,秦歸韓武遂。
}

  3 齊、韓、魏以楚負其從親,|{
	九年,楚與齊、韓合從,蓋卽負之也。
	從,子容翻。
}
合兵伐楚。
	楚王使太子橫爲質于秦以請救。|{
	質,音致。
}
秦客卿通將兵救楚,三國引兵去。|{
	將,卽亮翻。
	又音如字,領也。
}

  十三年(己未,西元前三零二年)

  1 秦王、魏王、韓太子嬰會于臨晉,韓太子至咸陽而歸,秦複與魏蒲阪。|{
	阪,音反。
	去年秦取魏蒲阪。
}

  2 秦大夫有私與楚太子鬥者。
	太子殺之,亡歸。|{
	楚太子質秦而亡歸,復質于齊;秦以爲言而誘陷其父,齊乘其父出而要之以利。
}

  十四年(庚申,西元前三零一年)

  1 日有食之,旣。


  2 秦人取韓穰。|{
	班志,穰縣屬南陽郡。
	以時考之,當屬楚。
	然韓得潁川之地,與南陽接境,七國兵爭,疆埸之間,一彼一此,或者此時穰屬韓歟?穰,人羊翻。
}

  3 蜀守煇叛秦,秦司馬錯往誅之。|{
	蜀守,蜀郡守也。
	史記秦紀作"蜀侯"。
	華陽國志曰:秦封王子煇爲蜀侯。
	蜀侯祭,歸胙于王;後母疾之,加毒以進。
	王大怒,使司馬錯賜煇劍。
	守,音狩。
	煇,索隱音暉。
}

  4 秦庶長奐會韓、魏、齊兵伐楚,|{
	脩楚太子亡歸之怨。
	長,知丈翻。
}
敗其師於重丘,殺其將唐昧;遂取重丘。|{
	唐姓本于唐堯。
	春秋之時,有二重丘:衛孫蒯飲馬於重丘,杜預曰:曹邑;諸侯同盟于重丘,杜預曰:齊地。
	時楚之境皆不至此。
	呂氏春秋曰:齊令章子與韓、魏攻荊,荊使唐薎將兵應之,夾泚而軍;章子夜襲之,斬薎於是水之上。
	水經注曰:泚水又西,澳水注之。
	水北出茈丘山,南入于泚水。
	意者重丘卽茈丘也。
	敗,補邁翻。
	將,卽亮翻。"昧",荀子作"蔑",楊倞注曰:與"昧"同,語音相近,當音末。
	索隱音莫葛翻。
	重,直龍翻。
	茈,才支翻。
}

  ⑤趙王伐中山,中山君奔齊。


  十五年(辛酉,西元前三零零年)

  1 秦涇陽君爲質于齊。|{
	質,音致。
}

  2 秦華陽君伐楚,大破楚師,斬首三萬,殺其將景缺,取楚襄城。|{
	班志,襄城縣屬潁川郡,有西不羹,楚靈王所謂"大城陳、蔡、不羹,賦皆千乘"是也。
	將,卽亮翻。
	陸德明曰:不羹,舊音郎;漢書地理志作"更"字,乘,繩證翻。
}
楚王恐,使太子爲質于齊以請平。|{
	爲楚懷王入秦而卒、齊留太子以邀楚張本。
}

  3 秦樗裡疾卒,以趙人樓緩爲丞相|{
	樗,丑於翻。
	卒,子恤翻。
	相,息亮翻。
}

  4 趙武靈王愛少子何,欲及其生而立之。|{
	少,詩照翻。
	及其生者,及其生而親見之。
}

  十六年(壬戌,西元前二九九年)

  1 五月戊申,大朝東宮,|{
	朝,直遙翻。
}
傳國於何,王廟見禮畢,出臨朝,|{
	廟見,始卽位而見祖廟也。
	見,賢遍翻。
}
大夫悉爲臣。
	肥義爲相國,並傅王。|{
	相國之官始此,秦、漢因之;漢、魏以降,其位望尊于丞相。
	相,息亮翻。
}
武靈王自號"主父"。|{
	主父,言爲國之主之父也。
	一曰,言其子主國而己則父也。
}
主父欲使子治國,身胡服,將士大夫西北略胡地。|{
	治,直之翻。
	將,卽亮翻,又如字。
}
將自云中、九原南襲咸陽,於是詐自爲使者,入秦。|{
	使,疏吏翻。
}
欲以觀秦地形及秦王之爲人,秦王不知,已而怪其狀甚偉,非人臣之度,|{
	賓主相見,交際之禮已,方怪其非人臣。
}
使人逐之;主父行已脫關矣,審問之,乃主父也。|{
	謂已脫身出秦關也。
}
秦人大驚。


  2 齊王、魏王會于韓。


  3 秦人伐楚,取八城。
	秦王遺楚王書曰:"始寡人與王約爲兄弟,盟于黃棘,|{
	見上十一年。
	遺,于季翻。
}
太子入質,至驩也。|{
	質,音致。
	見十二年。
}
太子陵殺寡人之重臣,不謝而亡去。|{
	見十三年。
}
寡人誠不勝怒,|{
	勝,音升。
}
使兵侵君王之邊。|{
	謂戰重丘,取襄城。
}
今聞君王乃令太子質于齊以求平。|{
	見十五年。
}
寡人與楚接境,婚姻相親。|{
	妻父曰婚,婿父曰姻。
	字書:婚,昏也,禮娶以昏時,婦人陰也,故曰婚。
	婿家女之所因,故曰姻。
	字林:婚,婦家;姻,婿家。
	賈公彥曰:各據男女身,則男曰昏,女曰姻;若以親言之,則女之父曰婚, 婿之父曰姻。
	余按張儀言秦、楚嫁女娶婦爲昆弟之國;考之于史,自赧王四年至是年,秦、楚未嘗嫁娶也。
	至十九年,楚懷王死于秦。
	至二十三年,楚襄王逆婦于秦。
	蓋先已約親,其後襄王終喪,始逆婦成婚姻。
}
而今秦、楚不驩,則無以令諸侯。|{
	令,力政翻。
}
寡人願與君王會武關,面相約,結盟而去,寡人之願也。"

  楚王患之,欲往恐見欺,欲不往恐秦益怒。
	昭睢曰:"毋行而發兵自守耳!|{
	睢,,息遺翻。
	又七餘翻。
}
秦,虎狼也,有並諸侯之心,不可信也!"懷王之子【章:十二行本"子"下有"子"字;乙十一行本同;孔本同。】蘭勸王行,王乃入秦。
	秦王令一將軍詐爲王,伏兵武關,楚王至則閉關劫之,與俱西,至咸陽,朝章台,如藩臣禮,|{
	朝,直遙翻。
	秦章台宮在渭南。
	漢張敞走馬章台街,孟康曰:在長安中;臣瓚曰:街在章台下。
	漢長安在渭南,以此言之,章台宮在渭南明矣。
	瓚,藏旱翻。
}
要以割巫、黔中郡。
	楚王欲盟,秦王欲先得地。
	楚王怒曰:"秦詐我,而又強要我以地!"|{
	要,一遙翻。
	黔,其今翻。
	強,其良翻,又其兩翻。
}
因不復許。
	秦人留之。|{
	複,扶又翻。
}

  楚大臣患之,乃相與謀曰:"吾王在秦不得還,要以割地,而太子爲質于齊,|{
	還,從宣翻,又音如字。
	要,一遙翻。
	質,音致。
}
齊、秦合謀,則楚無國矣。"欲立王子之在國者。
	昭睢曰:"王與太子俱困于諸侯,今又倍王命而立其庶子,不宜!"|{
	睢,息隨翻。
	倍,蒲妹翻。
}
乃詐赴于齊。|{
	詐,言楚王薨而請太子還王楚。
}
齊閔王召羣臣謀之,或曰:"不若留太子以求楚之淮北。"|{
	閔,讀曰閔。
	楚滅陳、蔡,封畛於汝,滅越,取吳故地,並有古徐夷之地,皆在淮北,卽楚所謂"下東國"。
}
齊相曰:"不可!郢中立王,|{
	郢,楚都。
	班志:南郡江陵縣,故楚郢都,楚文王自丹陽徙此;後九世,平王城之;又後十世,秦拔之,東徙壽春,亦名曰郢。
	水經:江水東逕江陵縣故城南,又東逕郢城南。
	注云:今江陵城,楚船官地,春秋之渚宮。
	郢城卽子囊遺言所城者。
	劉昫曰:故楚都之郢城,今江陵縣北十五裡紀南城是也。
	相,息亮翻。
}
是吾抱空質而行不義於天下也。"|{
	質,音致。
}
其人曰:"不然,郢中立王,因與其新王市曰:『予我下東國,吾爲王殺太子。|{
	巿,謂相要以利,如巿道也。
	予,讀曰與。
	爲,於偽翻。
}
不然,將與三國共立之。』"|{
	三國,謂齊、韓、魏。
}
齊王卒用其相計而歸楚太子。|{
	卒,子恤翻。
}
楚人立之。


  4 秦王聞孟嘗君之賢,使涇陽君爲質于齊以請。
	孟嘗君來入秦,秦王以爲丞相。|{
	質,音致。
}

  十七年(癸亥,西元前二九八年)

  1 或謂秦王曰:"孟嘗君相秦,必先齊而後秦;|{
	先、後,皆去聲。
}
秦其危哉!"秦王乃以樓緩爲相,囚孟嘗君,欲殺之。
	孟嘗君使人求解于秦王幸姬,姬曰:"願得君狐白裘。"|{
	狐白裘,緝狐掖之皮爲之,所謂千金之裘非一狐之掖者也。
}
孟嘗君有狐白裘,已獻之秦王,無以應姬求。
	客有善爲狗盜者,入秦藏中,|{
	物之所藏曰藏,音徂浪翻。
}
盜狐白裘以獻姬。
	姬乃爲之言于王而遣之。|{
	爲,於偽翻。
}
王後悔,使追之。
	孟嘗君至關,關法,雞鳴而出客,時尚蚤,|{
	蚤,古早字通。
}
追者將至,客有善爲雞鳴者,野雞聞之皆鳴,孟嘗君乃得脫歸。


  2 楚人告于秦曰:"賴社稷神靈,國有王矣!"秦王怒,發兵出武關擊楚,斬首五萬,取十六城。


  3 趙王封其弟【章:十二行本"弟"下有"勝"字;乙十一行本同。】爲平原君。|{
	班志,平原縣屬平原郡。
	勝封于東武城,號平原君,非封于平原也。
	東武城屬清河郡,杜佑曰:今貝州武城縣是也。
	蓋定襄有武城,時同屬趙,故此加"東"也。
}
平原君好士,|{
	好,呼到翻。
}
食客嘗【章:十二行本"嘗"作"常";乙十一行本同。】數千人。
	有公孫龍者,善爲堅白同異之辯,|{
	漢書藝文志:公孫龍子十四篇。
	注云:卽爲堅白同異之辯者。
	成玄英莊子疏云:公孫龍著守白論,行於世。
	堅白,卽守白也,言堅執其說,如墨子墨守之義。
	自堅白之論起,辯者互執是非,不勝異說。
	公孫龍能合衆異而爲同,故謂之同異。
	史記注曰:晉太康地記云:汝南西平縣有龍淵,水可用淬刀劍,極堅利,故有堅白之論云:黃,所以爲堅也;白,所以爲利也。
	或曰:黃所以爲不堅,白所以爲不利。
	二說未知孰是。
	勝,音升。
	淬,取內翻。
}
平原君客之。
	孔穿自魯適趙,|{
	按孔叢子,孔穿,孔子之後。
	孫愐曰:孔姓,殷湯之後,本自帝嚳元妃簡狄,吞乙卵生契,賜姓子氏;至湯,以其祖感乙而生,故名履,字天乙;後代以"子"加"乙",始爲孔氏。
	至宋孔父遭華督之難,其子奔魯,故孔子生於魯。
	愐,彌兗翻。
	嚳,苦沃翻。
	華,戶化翻。
	難,乃旦翻。
}
與公孫龍論臧三耳,|{
	三耳,如莊子所載雞三足之說。
	莊子疏謂數起於一,一與一爲二,二與一爲三,三名雖立,實無定體,故雞可以爲三足,則兩耳、三耳,其說亦猶是耳。
	一說,耳主聽,兩耳,形也,兼聽而言,可得爲三。
	臧,臧獲之臧。
	臧獲,奴婢也。
}
龍甚辯析。|{
	辯,別也;析,分也;言分別甚精微也。
}
子高弗應,俄而辭出,明日複見平原君。|{
	子高,孔穿字也。
	複,扶又翻。
}
平原君曰:"疇昔公孫之言信辯也,|{
	毛晃曰:疇,曩也;昔,夕也;疇昔,曩夕也。
}
先生以爲何如?"對曰:"然。
	幾能令臧三耳矣。|{
	毛晃曰:然,如也,是也,語決辭。
	幾,居依翻。
	令,使也,音力丁翻。
	雖然,實難!
	僕願得又問於君:今謂三耳甚難而實非也,謂兩耳甚易而實是也,不知君將從易而是者乎,其亦從難而非者乎?"平原君無以應。
	明日,謂公孫龍曰:"公無複與孔子高辯事也!|{
	易,弋豉翻。
}
其人理勝於辭;公辭勝於理,終必受詘。"


鄒衍過趙,|{
	過,古禾翻。
}
平原君使與公孫龍論白馬非馬之說。|{
	此亦莊子所謂狗非犬之說。
	疏云:狗之與犬,一實兩名:
	名實合,則此爲狗,彼爲犬;
	名實離,則狗異於犬。
	又墨子曰:狗,犬也。
	然[殺]狗~
	非狗[殺]犬也。
	大指與白馬非馬之說同。
}
鄒子曰:“不可。\footnote{鄒子谈论语言用途,公孙龙论证语言本身,风马牛不相及。语言标准化,消歧义,才不致时代变迁,地域不同而致不可读。}
	夫辯者,別殊類使不相害,序異端使不相亂。
抒意通指,|{
	夫,音扶。
	別,彼列翻。
	索隱曰:抒,音墅,抒者舒也;又常恕翻。
	康曰:亦音舒。
}
明其所謂,使人與知焉,不務相迷也。|{
	與,音如字;又讀曰預。
}
故勝者不失其所守,不勝者得其所求。|{
	辯以求是,辯雖不勝而得審其是,所謂得其所求也。
}
若是,故辯可爲也。
及至煩文以相假,飾辭以相敦,|{
	敦,都昆翻,迫也,詆也,誰何也。
}
巧譬以相移,引人使不得及其意,如此害大道。
夫繳紉【紛】爭言而競後息,|{
	索隱:繳,音糾;康吉弔切,非。
	言其言戾,紛然而爭,欲人先屈,務在人後方止也。
}
不能無害君子,衍不爲也。”
座皆稱善。|{
	言一座之人皆稱衍言爲善。
}
公孫龍由是遂詘。|{
	通鑑書此,言小辯終不足破大道。
	絀,音敕律翻。
	說文曰:絀,貶下也。
	又讀與屈同。
}


資治通鑑卷三

