






























































資治通鑑卷二百三   宋 司馬光 撰

胡三省 音註

唐紀十九【起玄黓敦牂盡柔兆閹茂凡五年}


高宗天皇大聖大弘孝皇帝下

永淳元年【時以皇孫重照生改元}
春二月作萬泉宫於藍田【藍田縣漢屬京兆後魏置藍田郡隋廢郡為縣復屬京兆}
癸未改元赦天下 戊午立皇孫重照為皇太孫上欲令開府置官屬問吏部郎中王方慶【吏部掌考天下之文吏之班秩階品}
對曰晉及齊皆嘗立太孫【晉惠帝立太孫臧齊武帝立大孫昭業}
其太子官屬即為太孫官屬未聞太子在東宫而更立太孫者也上曰自我作古可乎對曰三王不相襲禮【叔孫通之言}
何為不可乃奏置師傅等官既而上疑其非法竟不補授方慶裒之曾孫也【方慶梁王裒之曾孫江陵䧟裒徙入關遂為咸陽人裒當作褒}
名綝以字行【綝丑林翻}
西突厥阿史那車薄帥十姓反【厥九勿翻帥讀曰率下同}
夏四月甲子朔日有食之 上以關中飢饉米斗三百將幸東都丙寅發京師留太子監國【監古衘翻下同}
使劉仁軌裴炎薛元超輔之時出幸倉猝扈從之士有餓死於中道者【從才用翻下以從同}
上慮道路多草竊命監察御史魏元忠檢校車駕前後元忠受詔即閱視赤縣獄【西京以長安萬年為赤縣}
得盗一人神采語言異於衆命釋桎梏【桎職日翻梏工沃翻}
襲冠帶乘驛以從【從才用翻}
與之共食宿【既與之共食又與之共宿}
託以詰盗【詰去吉翻}
其人笑許諾比及東都【比必利翻}
士馬萬數不亡一錢 辛未以禮部尚書聞喜憲公裴行儉為金牙道行軍大摠管【此指西突厥之金牙山也}
帥右金吾將軍閻懷旦等三摠管分道討西突厥師未行行儉薨行儉有知人之鑒初為吏部侍郎前進士王勮【勮其劇翻}
咸陽尉欒城蘇味道【劉昫曰欒城漢開縣後魏於漢開縣古城置欒城縣屬趙州余考漢書地理志常山郡有關縣又考宋白續通典鎮州欒城縣本漢關縣魏太和十一年於關縣故城置欒城縣則劉昫誤作開縣明矣}
皆未知名行儉一見謂之曰二君後當相次掌銓衡僕有弱息願以為託【弱息弱子也}
是時勮弟勃與華隂楊烱范陽盧照鄰【范陽漢涿縣地魏文帝改為范陽郡至隋廢郡復為涿縣屬幽州唐武德七年改為范陽縣華戶化翻烱古迴翻}
義烏駱賓王【義烏漢烏傷縣地後漢分烏傷置長山縣晉以長山為東陽郡治所烏傷別為縣武德七年改烏傷為義烏縣屬婺州}
皆以文章有盛名司列少常伯李敬玄尤重之【少詩沼翻}
以為必顯逹行儉曰士之致遠當先器識而後才藝勃等雖有文華而浮躁淺露豈享爵禄之器邪楊子稍沈静【躁則到翻沈持林翻}
應至令長餘得令終幸矣既而勃度海墮水烱終於盈川令【黔州彭水縣漢酉陽縣地武德二年分彭水於巴江西置盈隆縣先天元年避太子名改曰盈川非此也衢州龍丘縣武后如意元年分置盈川縣縣西有刑溪陳時土人留異惡刑字改曰盈川因為縣名長知兩翻}
照鄰惡疾不愈赴水死賓王反誅【謂同徐敬業反}
勮味道皆典選如行儉言【選須絹翻}
行儉為將帥所引偏禆如程務挺張䖍勗王方翼劉敬同李多祚黑齒常之後多為名將【將即亮翻帥所類翻禆賓彌翻}
行儉常命左右取犀角麝香而失之又勑賜馬及鞍令史輒馳驟馬倒鞍破【此禮部令史也}
二人皆逃去行儉使人召還謂曰爾曹皆誤耳何相輕之甚邪【謂懼罪責而逃是以常人見待相輕之甚也}
待之如故破阿史那都支【見上卷調露元年}
得馬腦盤廣二尺餘【馬腦文石也琢以為盤廣古曠翻}
以示將士軍吏王休烈捧盤升階跌而碎之【跌徒結翻}
惶恐叩頭流血行儉笑曰爾非故為何至於是不復有追惜之色詔賜都支等資產金器三千餘物襍畜稱是【復扶又翻畜許救翻稱尺證翻}
並分給親故及偏禆數日而盡 阿史那車薄圍弓月城安西都護王方翼引軍救之破虜衆於伊麗水【自弓月城過思渾川蟄失密城渡伊麗河至碎葉界}
斬首千餘級俄而三姓咽麫與車薄合兵拒方翼方翼與戰於熱海【碎葉城東有熱海地寒不凍咽於甸翻麫眡見翻}
流矢貫方翼臂方翼以佩刀截之左右不知所將胡兵謀執方翼以應車薄方翼知之悉召會議陽出軍資賜之以次引出斬之會大風方翼振金鼔以亂其聲誅七十餘人其徒莫之覺既而分遣禆將襲車薄咽麫大破之擒其酋長三百人【酋慈由翻長知兩翻}
西突厥遂平閻懷旦竟不行方翼尋遷夏州都督徵入議邊事上見方翼衣有血漬【夏戶雅翻漬疾智翻}
問之方翼具對熱海苦戰之狀上視瘡歎息竟以廢后近屬不得用而歸【廢后方翼從祖女弟也歸者復歸夏州}
乙酉車駕至東都 丁亥以黄門侍郎潁川郭待舉【隋改長社為穎川縣武德四年復曰長社屬許州}
兵部侍郎岑長倩祕書員外少監檢校中書侍郎鼓城郭正一吏部侍郎鼔城魏玄同【鼓城漢臨平下曲陽兩縣之地屬鉅鹿郡隋分槀城於下曲陽故城東五里置昔陽縣尋改為鼓城時屬定州}
並與中書門下同承受進止平章事上欲用待舉等謂韋知温曰待舉等資任尚淺且令預聞政事未可與卿等同名自是外司四品已下知政事者始以平章事為名長倩文本之兄子也【岑文本輔太宗}
先是玄同為吏部侍郎【先悉薦翻}
上言銓選之弊【上時掌翻}
以為人君之體當委任而責成功所委者當則所用者自精矣【者當丁浪翻}
故周穆王命伯冏為太僕正曰慎簡乃僚【見書冏命}
是使羣司各求其小者而天子命其大者也乃至漢氏得人皆自州縣補署五府辟召然後升於天朝【見後漢紀朝直遥翻}
自魏晉以來始專委選部【選須絹翻}
夫以天下之大士人之衆而委之數人之手用刀筆以量才案簿書而察行【量音良行下孟翻}
借使平如權衡明如水鏡猶力有所極照有所窮况所委非人而有愚闇阿私之弊乎願略依周漢之規以救魏晉之失疏奏不納 五月東都霖雨乙卯洛水溢溺民居千餘家關中先水後旱蝗繼以疾疫米斗四百兩京間死者相枕於路【枕之任翻}
人相食 上既封泰山欲遍封五嶽秋七月作奉天宫於嵩山南【奉天宫在洛州嵩陽縣}
監察御史裏行李善感諫曰【裏行者資序未至未正除監察御史令於監察御史班裏行也監古衘翻}
陛下封泰山告太平致羣瑞與三皇五帝比隆矣數年以來菽粟不稔餓殍相望四夷交侵兵車歲駕陛下宜恭默思道以禳災譴【禳加羊翻}
乃更廣營宫室勞役不休天下莫不失望臣忝備國家耳目竊以此為憂上雖不納亦優容之自褚遂良韓瑗之死【見二百卷顯慶三年四年瑗于眷翻}
中外以言為諱無敢逆意直諫幾二十年及善感始諫天下皆喜謂之鳳鳴朝陽【詩卷阿曰鳳皇鳴矣于彼高岡梧桐生矣于彼朝陽注云梧桐柔木也山東曰朝陽梧桐不生山岡太平而後生朝陽幾居依翻}
上遣宦者緣江徙異竹欲植苑中宦者科舟載竹所在縱暴過荆州荆州長史蘇良嗣囚之上疏切諫【上時掌翻下同}
以為致遠方異物煩擾道路恐非聖人愛人之意又小人竊弄威福虧損皇明上謂天后曰吾約束不嚴果為良嗣所怪手詔慰諭良嗣令弃竹江中良嗣世長之子也【蘇世長見一百八十八卷高祖武德四年}
黔州都督謝祐希天后意逼零陵王明令自殺【明徙黔州見上卷永隆元年黔音琴}
上深惜之黔府官屬皆坐免官祐後寑於平閣與婢妾十餘人共處【處昌呂翻}
夜失其首垂拱中明子零陵王俊黎國公傑為天后所殺有司籍其家得祐首漆為穢器題云謝祐乃知明子使刺客取之也太子留守京師頗事遊畋薛元超上疏規諫上聞之

遣使者慰勞元超【使疏吏翻勞力到翻}
仍召赴東都 吐蕃將論欽陵寇柘松翼等州【顯慶三年開置柘州蓬山郡屬松州都督府宋白作拓曰以開拓為稱今按新舊書皆作柘翼州本漢蠶陵縣地故城在州西有蠶陵山隋為翼斜縣唐武德元年置翼州隋縣名唐州取州南翼水為名}
詔左驍衛郎將李孝逸右衛郎將衛蒲山發秦渭等州兵分道禦之【驍堅堯翻將即亮翻}
冬十月丙寅黄門侍郎劉景先同中書門下平章事 是歲突厥餘黨阿史那骨篤禄【骨篤禄亦曰骨咄禄頡利族人也雲中都督舍利元英之部酋世襲吐屯}
阿史德元珍等招集亡散據黑沙城反【杜佑曰阿史德元珍習知中國風俗邊塞虚實在單于府檢校降戶部落坐事為單于長史王本立所拘縶會骨咄禄入寇元珍請依舊檢校部落本立許之因便投骨咄禄骨咄禄得之甚喜以為阿波大逹于令專統兵馬事}
入寇并州及單于府之北境【單音蟬}
殺嵐州刺史王德茂右領軍衛將軍檢校代州都督薛仁貴將兵擊元珍於雲州虜問唐大將為誰應之曰薛仁貴虜曰吾聞仁貴流象州【仁貴以大非川之敗除名起為雞林道總管復坐事貶象州}
死久矣何以紿我【紿蕩亥翻}
仁貴免胄示之面虜相顧失色下馬列拜稍稍引去仁貴因奮擊大破之斬首萬餘級捕虜二萬餘人 吐蕃入寇河源軍軍使婁師德將兵擊之於白水澗【白水澗有白水軍注見後使疏吏翻下同將即亮翻下同}
八戰八捷上以師德為比部員外郎左驍衛郎將河源軍經略副使曰卿有文武材勿辭也【比音毗驍堅堯翻}


弘道元年【是年十二月改元}
春正月甲午朔上行幸奉天宫二月庚午突厥寇定州刺史霍王元軌擊却之乙亥復寇媯州【復扶又翻下可復同媯居為翻}
三月庚寅阿史那骨篤禄阿史德元珍圍單于都護府執司馬張行師殺之遣勝州都督王本立夏州都督李崇義將兵分道救之 太子右庶子同中書門下三品李義琰改葬父母使其舅氏遷舊墓上聞之怒曰義琰倚埶乃陵其舅家不可復知政事義琰聞之不自安以足疾乞骸骨庚子以義琰為銀青光禄大夫致仕 癸丑守中書令崔知温薨【舊制凡九品已上職事官皆帶散位謂之本品職事則隨才叙用或去閑入劇或去高就卑遷徙出入參差不定散位則一切以門䕃結品然後以勞考進叙武德令職事解散官欠一階不至為兼職事卑者不解散官貞觀令以職事高者為守職事卑者為行仍帶散位其欠一階仍舊為兼或帶散官或為守參而用之其兩職事亦為兼頗相錯亂咸亨二年始一切為守其欠一階之兼古念翻其兩職事之兼古悟翻字同音異耳}
夏四月己未車駕還東都 綏州步落稽白鐵余【步落稽稽胡也}
埋銅佛於地中久之艸生其上紿其鄉人曰吾於此數見佛光【紿蕩亥翻數所角翻}
擇日集衆掘地果得之因曰得見聖佛者百疾皆愈遠近赴之鐵余以雜色囊盛之數十重得厚施乃去一囊【盛時征翻重直龍翻施式豉翻去羌呂翻}
數年間歸信者衆遂謀作亂據城平縣自稱光明聖皇帝置百官進攻綏德大斌二縣【城平及二縣皆屬綏州西魏所置也宋白曰二縣皆漢膚施縣地魏神龜元年置城中縣隋避諱改為城平大斌縣時理城平縣界魏平故城綏德縣亦膚施地魏大統十二年分上郡南界丘尼谷置縣歐陽脩曰大斌者取稽胡懷化文武雜半以為名}
殺官吏焚民居遣右武衛將軍程務挺與夏州都督王方翼討之甲申攻拔其城擒鐵余餘黨悉平 【考異曰僉載云延州稽胡又云自號月光王又云儀鳳中務挺斬平之今從實録}
五月庚寅上幸芳桂宫【儀鳳二年營紫桂宫於澠池縣西五里調露二年改曰避暑宫永淳元年又改曰芳桂宫}
至合璧宫遇大雨而還 乙巳突厥阿史那骨篤禄等寇蔚州殺刺史李思儉【蔚州時為忠順軍節度}
豐州都督崔智辨將兵邀之於朝那山北【朝丁度集韻音與邾同牛頭朝那山在豐州河北}
兵敗為虜所擒朝議欲廢豐州遷其百姓於靈夏豐州司馬唐休璟【都督府司馬也唐制下都督府長史司馬從五品上}
上言以為豐州阻河為固居賊衝要自秦漢已來列為郡縣土宜耕牧隋季喪亂【上時掌翻喪息浪翻}
遷百姓於寧慶二州致胡虜深侵以靈夏為邊境貞觀之末募人實之西北始安今廢之則河濱之地復為賊有【復扶又翻又音如字}
靈夏等州人不安業非國家之利也乃止 六月突厥别部寇掠嵐州偏將楊玄基擊走之【厥九勿翻將即亮翻}
秋七月己丑立皇孫重福為唐昌王【重直龍翻}
庚辰詔以今年十月有事於嵩山尋以上不豫改用來年正月 甲辰徙相王輪為豫王更名旦【相息亮翻更工衡翻}
中書令兼太子左庶子薛元超病瘖乞骸骨許之【瘖於今翻}
八月己丑以將封嵩山召太子赴東都留唐昌王重福守京師以劉仁軌為之副冬十月己卯太子至東都 癸亥車駕幸奉天宫十一月丙戌詔罷來年封嵩山上疾甚故也上苦頭

重不能視召侍醫秦鳴鶴診之【殿中省尚藥局有侍御醫四人從六品上診止忍翻}
鳴鶴請刺頭出血可愈【刺七亦翻}
天后在簾中不欲上疾愈怒曰此可斬也乃欲於天子頭刺血鳴鶴叩頭請命上曰但刺之未必不佳乃刺百會腦戶二穴【鍼炙經百會一名三陽五會在前頂後寸半頂中央旋毛中可容豆鍼二分得氣即瀉腦戶一名合顱在骨上強後寸半禁鍼鍼令人瘂舊傳鳴鶴鍼微出血頭疼立止}
上曰吾目似明矣后舉手加額曰天賜也自負綵百匹以賜鳴鶴 戊戌以右武衛將軍程務挺為單于道安撫大使招討阿史那骨篤禄等詔太子監國【監古衘翻}
以裴炎劉景先郭正一同東宫平章事 上自奉天宫疾甚宰相皆不得見丁未還東都百官見於天津橋南【見賢遍翻}
十二月丁巳改元赦天下上欲御則天門樓宣赦氣逆不能乘馬乃召百姓入殿前宣之是夜召裴炎入受遺詔輔政上崩於貞觀殿【年五十六觀古玩翻}
遺詔太子柩前即位【柩音舊}
軍國大事有不决者兼取天后進止廢萬泉芳桂奉天等宫庚申裴炎奏太子未即位未應宣敇有要速處分【處昌呂翻分扶問翻}
望宣天后令於中書門下施行甲子中宗即位尊天后為皇太后政事咸取决焉太后以澤州刺史韓王元嘉等地尊望重【澤州漢高都端氏泫氏之地西燕慕容永置建興郡後魏置建州隋改澤州大業廢州為長平郡唐復曰澤州宋白曰取濩澤為名}
恐其為變並加三公等官以慰其心 甲戌以劉仁軌為左僕射裴炎為中書令戊寅以劉景先為侍中故事宰相於門下省議事謂之政事堂故長孫無忌為司空房玄齡為僕射魏徵為太子太師皆知門下省事及裴炎遷中書令始遷政事堂於中書省 壬午遣左威衛將軍王果左監門將軍令狐智通右金吾將軍楊玄儉右千牛將軍郭齊宗分往并益荆揚四大都督府與府司相知鎮守【以國有大故備不虞也監古衘翻并卑經翻}
中書侍郎同平章事郭正一為國子祭酒罷政事

則天順聖皇后上之上【后姓武氏諱曌并州文水人后自製曌字讀與照同音之笑翻天寶八載追上尊號曰則天順聖皇后}


光宅元年【是年九月方改元光宅}
春正月甲申朔改元嗣聖【此太子即位踰年所改之元也}
赦天下 立太子妃韋氏為皇后擢后父玄貞自普州參軍為豫州刺史【此豫州本春秋沈蔡二國之地漢為汝南郡宋文帝立司州治懸瓠城以為重鎮魏改豫州唐因之後避代宗諱改為蔡州}
癸巳以左散騎常侍杜陵韋弘敏為太府卿同中書門下三品【自漢宣帝起杜陵邑至後漢為縣屬京兆隋遷京城始并杜陵入大興縣唐改大興曰萬年散悉亶翻騎奇寄翻}
中宗欲以韋玄貞為侍中又欲授乳母之子五品官裴炎固爭中宗怒曰我以天下與韋玄貞何不可而惜侍中邪炎愳白太后密謀廢立二月戊午太后集百官於乾元殿裴炎與中書侍郎劉禕之羽林將軍程務挺張䖍朂【漢置南北軍掌衛京師南軍若唐諸衛也北軍若唐羽林軍也漢武帝名羽林曰建章營騎屬光禄勲後更名羽林騎取六郡良家子及死事之孤為之後漢置羽林監南朝因之後魏周曰羽林率隨左右屯衛所領兵名曰羽林貞觀中置北衙七營兵選才力驍勇者充龍朔二年曰左右羽林軍置大將軍各一員將軍各二員品同諸衛統領北衙禁兵之法今而督攝左右廂飛騎之儀仗以統諸曹之職取府兵越騎步射以為羽林軍士大朝會則執仗以衛階陛行幸則夾馳道為内仗邪音耶禕吁韋翻}
勒兵入宫宣太后令廢中宗為廬陵王扶下殿【下遐嫁翻}
中宗曰我何辠太后曰汝欲以天下與韋玄貞何得無辠乃幽于別所己未立雍州牧豫王旦為皇帝【雍於用翻}
政事决於太后居睿宗於别殿不得有所預立豫王妃劉氏為皇后后德威之孫也【劉德威審禮之父}
有飛騎十餘人飲於坊曲【置飛騎見一百九十五卷貞觀十二年騎寄奇翻}
一人言曏知别無勲賞不若奉廬陵一人起出詣北門告之【北門玄武門也}
座未散皆捕得繫羽林獄言者斬餘以知反不告皆絞告者除五品官告密之端自此興矣 壬子以永平郡王成器為皇太子睿宗之長子也赦天下改元文明【改嗣聖為文明}
庚申廢皇太孫重照為庶人命劉仁軌專知西京留守事流韋玄貞於欽州【舊志欽州至京師五千二百五十一里}
太后與劉仁軌書曰昔漢以關中事委蕭何【見漢高帝紀}
今託公亦猶是矣仁軌上疏辭以衰老不堪居守【守式又翻}
因陳呂后禍敗事以申規戒【呂氏禍敗事見漢高后紀}
太后使祕書監武承嗣齎璽書慰諭之曰今以皇帝諒闇不言【璽斯民翻闇音隂}
眇身且代親政遠勞勸戒復辭衰疾【復扶又翻}
又云呂氏見嗤於後代禄產貽禍於漢朝【朝直遥翻下同}
引喻良深愧慰交集公忠貞之操終始不渝勁直之風古今罕比初聞此語能不罔然静而思之是為龜鏡况公先朝舊德遐邇具瞻願以匡救為懷無以暮年致請辛酉太后命左金吾將軍丘神勣詣巴州檢校故太

子賢宅以備外虞其實風使殺之【風讀曰諷}
神勣行恭之子也【丘行恭為將歷事高祖太宗}
甲子太后御武成殿【唐六典洛陽宫南三門中曰應天左曰興教右曰光政光政之内曰廣運其北曰明福明福之東曰武成門其内曰武成殿}
皇帝帥王公以下上尊號【帥讀曰率上時掌翻}
丁卯太后臨軒遣禮部尚書武承嗣冊嗣皇帝自是太后常御紫宸殿【唐六典洛陽宫不載紫宸殿以西京大明宫凖之紫宸殿内朝也其位置當在乾元殿後}
施黲紫帳以視朝【紫色之淺者為慘紫朝直遥翻}
丁丑以太常卿檢校豫王府長史王德真為侍中【句斷}
中書侍郎檢校豫王府司馬劉禕之同中書門下三品 三月丁亥徙杞王上金為畢王鄱陽王素節為葛王 丘神勣至巴州幽故太子賢於别室逼令自殺 【考異曰則天實録賢死在二月丘神勣往巴州下舊本紀在三月唐歷遣神勣舉哀追封皆有日今從之}
太后乃歸辠於神勣戊戍舉哀於顯福門【顯福門意即明福門六典避中宗諱改顯為明耳}
貶神勣為疊州刺史己亥追封賢為雍王【雍於用翻}
神勣尋復入為左金吾將軍【復扶又翻}
夏四月開府儀同三司梁州都督滕王元嬰薨 辛酉徙畢王上金為澤王拜蘇州刺史葛王素節為許王拜絳州刺史 癸酉遷廬陵王于房州丁丑又遷于均州故濮王宅【即貞觀末濮王泰遷均州所居故宅濮博木翻}
五月丙申高宗靈駕西還 閏月以禮部尚書武承嗣為太常卿同中書門下三品 秋七月戊午廣州都督路元叡為崑崙所殺【崑崙國在林邑南去交趾海行三百餘日習俗文字與婆羅門同崑盧昆翻}
元叡闇懦僚屬恣横【横戶孟翻}
有商舶至【舶音白}
僚屬侵漁不已商胡訴於元叡元叡索枷欲繫治之【索山客翻枷音加}
羣胡怒有崑崙䄂劒直登聽事【聽讀曰廳}
殺元叡及左右十餘人而去無敢近者【近其靳翻}
登州入海追之不及 温州大水【後漢分章安之東甌鄉置永寧縣屬會稽郡晉分為永嘉郡隋廢郡為永嘉縣屬栝州武德五年復於永嘉置嘉州貞觀五年廢嘉州以縣屬枯州上元二年分置温州}
流四千餘家 突厥阿史那骨篤禄等寇朔州 八月庚寅葬天皇大帝于乾陵【乾陵在奉天縣北五里梁山}
廟號高宗 初尚書左丞馮元常為高宗所委高宗晩年多疾每曰朕體中不佳可與元常平章以聞元常嘗密言中宫威權太重宜稍抑損高宗雖不能用深以其言為然及太后稱制四方爭言符瑞嵩陽令樊文獻瑞石太后命於朝堂示百官【朝直遥翻}
元常奏狀涉謟詐不可誣罔天下太后不悦出為隴州刺史【舊志隴州京師西四百九十六里至東都一千一百三十二里}
元常子琮之曾孫也【馮子琮仕於高齊}
丙午太常卿同中書門下三品武承嗣罷為禮部尚書 栝州大水流二千餘家 九月甲寅赦天下改元【改元光宅}
旗幟皆從金色【幟昌志翻}
八品以下舊服青者更服碧【青色之深者為碧更工衡翻}
改東都為神都宫名太初又改尚書省為文昌臺左右僕射為左右相六曹為天地四時六官門下省為鸞臺中書省為鳳閣侍中為納言中書令為内史御史臺為左肅政臺增置右肅政臺【左臺專知京師百官及監諸軍旅并承詔出使右臺專知諸州按察杜佑曰武后置左右肅政臺左以察朝廷右以澄郡縣後廢右臺以其官隸左臺左臺本御史臺也右臺地今太僕寺是也}
其餘省寺監率之名【祕書殿中二省九卿寺少府將作國子軍器等監東宮十率}
悉以義類改之 以左武衛大將軍程務挺為單于道安撫大使【單音蟬使疏吏翻}
以備突厥 武承嗣請太后追王其祖【王于况翻}
立武氏七廟太后從之裴炎諫曰太后母臨天下當示至公不可私於所親獨不見呂氏之敗乎太后曰呂后以權委生者故及於敗今吾追尊亡者何傷乎對曰事當防微杜漸不可長耳【長知兩翻}
太后不從己巳追尊太后五代祖克己為魯靖公妣為夫人高祖居常為太尉北平恭肅王曾祖儉為太尉金城義康王祖華為太尉太原安成王考士彠為太師魏定王【彠一虢翻}
祖妣皆為妃裴炎由是得罪又作五代祠堂於文水【文水縣舊受陽隋開皇十一年更名屬并州}
時諸武用事唐宗室人人自危衆心憤惋【惋烏貫翻}
會眉州刺史英公李敬業及弟盩厔令敬猷【漢武帝置盩厔縣屬扶風後漢晉省後魏復置後周置周南郡隋廢郡以盩厔縣屬京兆唐置岐州盩厔音舟窒}
給事中唐之奇長安主簿駱賓王【唐赤縣主簿從八品上}
詹事司直杜求仁【唐詹事司直正九品上掌彈劾宫僚糾舉職事}
皆坐事敬業貶柳州司馬敬猷免官之奇貶栝蒼令【漢會稽曰浦縣後漢更名章安光武分章安縣之南鄉置松陽縣隋分松陽之東界置栝蒼縣帶栝州以栝蒼山名縣}
賓王貶臨海丞【吳分章安置臨海縣屬臨海郡隋廢郡以縣屬栝州唐分帶台州}
求仁貶黟令【黟縣漢屬丹陽郡吳分屬新安郡隋唐屬歙州黟師古音伊劉昫音䃜}
求仁正倫之姪也【杜正倫事太宗高宗}
盩厔尉魏思温嘗為御史復被黜【復扶又翻}
皆會於揚州【舊志揚州京師東南二千七百五十三里至東都一千七百四十九里}
各自以失職怨望乃謀作亂以匡復廬陵王為辭思温為之謀主使其黨監察御史薛仲璋求奉使江都【江都縣帶揚州監古衘翻使疏吏翻}
令雍州人韋超詣仲璋告變云揚州長史陳敬之謀反仲璋收敬之繫獄居數日敬業乘傳而至【雍於用翻傳知戀翻}
矯稱揚州司馬來之官云奉密旨以高州酋長馮子猷謀反【酋慈由翻長知兩翻}
發兵討之於是開府庫令士曹參軍李宗臣就錢坊驅囚徒工匠授以甲斬敬之於繫所【考異曰實録作薛璋御史臺記云薛仲璋矯使揚州與徐敬業等謀反夜與江都令韋知止子茂道計議倉}


【曹參軍閻識微發之長史陳敬之不察抑識微令遜謝仲璋佯事竟還出郭門羣官畢從其黨韋超遮道告密復留繫問遂斬敬之今事從實録仲璋從臺記}
録事參軍孫處行拒之亦斬以徇【處音昌呂翻}
僚吏無敢動者遂起一州之兵復稱嗣聖元年【復扶又翻又如字}
開三府一曰匡復府二曰英公府三曰揚州大都督府敬業自稱匡復府上將領揚州大都督【將即亮翻}
以之奇求仁為左右長史宗臣仲璋為左右司馬思温為軍師賓王為記室旬日間得勝兵十餘萬【勝音升}
移檄州縣略曰偽臨朝武氏者【朝直遥翻}
人非温順地實寒微昔充太宗下陳【陳列也戰國策曰美人充下陳}
嘗以更衣入侍【衛子夫以更衣得幸漢武帝賓王用此事更工衡翻}
洎乎晩節穢亂春宫【東宫亦謂之春宫洎其冀翻}
密隱先帝之私隂圖後庭之嬖踐元后於翬翟【翬翟后服也翬音暉}
陷吾君於聚麀【記曰夫惟禽獸無禮故父子聚麀麀於求翻}
又曰殺姊屠兄【姊謂韓國夫人兄謂元爽元慶事見二百一卷高宗乾封元年}
弑君鴆母【此以高宗晏駕及太原王妃之死為后罪}
人神之所同嫉天地之所不容又曰包藏禍心竊窺神器君之愛子幽之於别宫【謂居睿宗於别殿}
賊之宗盟委之以重任【謂用武承嗣等}
又曰一抔之土未乾【抔蒲侯翻乾音干}
六尺之孤安在又曰試觀今日之域中竟是誰家之天下太后見檄問曰誰所為或對曰駱賓王太后曰宰相之過也人有如此才而使之流落不偶乎敬業求得人貌類故太子賢者紿衆云賢不死亡在此城中令吾屬舉兵【紿蕩亥翻}
因奉以號令楚州司馬李崇福帥所部三縣應敬業【楚州本漢射陽鹽瀆縣地晉置山陽郡隋開皇初罷郡十二年置楚州大業初州廢唐初復置帥讀曰率所部三縣山陽鹽城安宜也}
盱眙人劉行舉獨據縣不從敬業遣其將尉遲昭攻盱眙【盱眙縣漢屬臨淮郡後漢屬下邳國晉安帝分置盱眙郡陳置北譙州隋廢為縣屬江都郡唐屬楚州盱眙音吁怡其將即亮翻尉紆勿翻}
詔以行舉為遊擊將軍以其弟行實為楚州刺史甲申以左玉鈐衛大將軍李孝逸為揚州道大摠管【是年改左右領軍衛為左右玉鈐衛}
將兵三十萬以將軍李知十馬敬臣為之副以討李敬業武承嗣與其從父弟右衛將軍三思以韓王元嘉魯王靈夔屬尊位重【從才用翻二王皆高祖子}
屢勸太后因事誅之太后謀於執政劉禕之韋思謙皆無言【禕吁韋翻}
内史裴炎獨固爭太后愈不悦三思元慶之子也及李敬業舉兵薛仲璋炎之甥也炎欲示閒暇不汲汲議誅討太后問計於炎對曰皇帝年長【長知兩翻}
不親政事故豎子得以為辭若太后返政則不討自平矣監察御史藍田崔詧聞之上言【監古衘翻上時掌翻}
炎受顧託大權在己若無異圖何故請太后歸政太后命左肅政大夫金城騫味道【左肅政大夫左御史大夫也蘭州五泉縣本漢金城縣隋更名高宗咸亨二年復為金城縣風俗通騫姓閔子騫後}
侍御史櫟陽魚承曄鞫之【漢高帝改櫟陽縣為萬年縣後世因之至隋並屬京兆唐改隋大興縣曰萬年以漢萬年縣復曰櫟陽屬華州櫟音藥}
收炎下獄【下遐嫁翻 考異曰新傳云炎謀乘太后出遊龍門以兵執之還政天子會久雨太后不出而止若炎實有此謀則太后殺之宜矣且炎有此謀必有同黨當炎下獄崔詧李景諶輩無事猶欲䧟之况有此迹其同黨能不首告乎又朝野僉載裴炎為中書令時徐敬業欲反令駱賓王畫計取裴炎同起事賓王足踏壁静思食頃乃為謡曰一片火兩片火緋衣小兒當殿坐教炎莊上小兒誦之并都下小兒皆唱炎乃訪學者令解之召賓王數啖以寶物錦綺皆不言又賂以音樂妓女駿馬亦不語乃將古忠臣烈士圖共觀之見司馬宣王賓王欻然起曰此英雄丈夫也即說自古大臣執政多移社稷炎大喜賓王曰但不知謡識如何耳炎告以謡言片火緋衣之事賓王即下北面而拜曰此真人矣遂與敬業等合謀揚州兵起炎從内應書與敬業等合謀唯有青鵝字人有告者朝臣莫之能解則天曰此青字者十二月鵝字者我自與也遂誅炎此皆當時構䧟炎者所言耳非其實也}
炎被收辭氣不屈或勸炎遜辭以免炎曰宰相下獄安有全理【下遐嫁翻下同}
鳳閣舍人李景諶證炎必反【鳳閣舍人中書舍人也諶氏壬翻}
劉景先及鳳閣侍郎義陽胡元範【義陽舊曰平陽隋開皇初改曰義陽劉昫曰義陽漢平氏縣之義陽鄉也魏分南陽置義陽郡晉自石城徙居仁順今申州理所是也}
皆曰炎社稷元臣有功於國悉心奉上天下所知臣敢明其不反太后曰炎反有端顧卿不知耳對曰若裴炎為反則臣等亦反也太后曰朕知裴炎反知卿等不反文武間證炎不反者甚衆太后皆不聽俄并景先元範下獄丁亥以騫味道檢校内史同鳳閣鸞臺三品李景諶同鳳閣鸞臺平章事 魏思温說李敬業曰【說輸芮翻}
明公以匡復為辭宜帥大衆鼓行而進直指洛陽【帥讀曰率}
則天下知公志在勤王四面響應矣薛仲璋曰金陵有王氣且大江天險足以為固不如先取常潤【潤州江左為京口重鎮隋為延陵縣屬江都郡唐武德三年置潤州取潤浦以為州名}
為定霸之基然後北向以圖中原進無不利退有所歸此良策也思温曰山東豪傑以武氏專制憤惋不平【惋烏貫翻}
聞公舉事皆自蒸麥飯為糧伸鋤為兵以俟南軍之至不乘此埶以立大功乃更蓄縮自謀巢穴遠近聞之其誰不解體敬業不從使唐之奇守江都將兵度江攻潤州【按舊志揚州至潤州四十八里潤州古朱方之地漢為丹徒縣吳為京口置京督以鎮又為徐陵督爾雅絶高曰京其城因山為壘緣江為境因謂之京口晉為南徐州隋置潤州取州東潤浦為名尋廢州唐復置}
思温謂杜求仁曰兵埶合則彊分則弱敬業不并力度淮收山東之衆以取洛陽敗在眼中矣壬辰敬業陷潤州執刺史李思文 【考異曰唐紀云李思文拒守四十餘日而䧟按敬業九月丁丑起兵十一月庚申敗纔四十四日耳今不取}
以李宗臣代之思文敬業之叔父也知敬業之謀先遣使間道上變【使疏吏翻間古莧翻上時掌翻}
為敬業所攻拒守久之力屈而陷思温請斬以徇敬業不許謂思文曰叔黨於武氏宜改姓武潤州司馬劉延嗣不降【江戶降翻}
敬業將斬之思温救之得免與思文皆囚於獄中延嗣審禮從父弟也【劉審禮戰沒於青海從才用翻}
曲阿令河間尹元貞引兵救潤州【曲阿縣本雲陽秦始皇改曰曲阿前漢屬會稽郡後漢屬吳郡晉屬晉陵郡隋屬江都郡唐屬潤州河間漢州鄉縣地屬涿郡隋為河間縣屬瀛州}
戰敗為敬業所擒臨以白刃不屈而死 丙申斬裴炎于都亭【洛陽都亭}
炎將死顧兄弟曰兄弟官皆自致炎無分毫之力今坐炎流竄不亦悲乎籍没其家無甔石之儲【甔都濫翻}
劉景先貶普州刺史胡元範流瓊州而死【舊志瓊州至兩京與崖州道里相類}
裴炎弟子太僕寺丞伷先【伷直又翻}
年十七上封事請見言事太后召見詰之曰汝伯父謀反尚何言伷先曰臣為陛下畫計耳【上時掌翻見賢遍翻詰去吉翻為于偽翻}
安敢訴寃陛下為李氏婦先帝棄天下遽攬朝政【朝直遥翻}
變易嗣子踈斥李氏封崇諸武臣伯父忠於社稷反誣以辠戮及子孫陛下所為如是臣實惜之陛下早宜復子明辟高枕深居則宗族可全不然天下一變不可復救矣【枕之任翻復扶又翻}
太后怒曰胡白【胡何也白陳也言何等陳白也}
小子敢發此言命引出伷先反顧曰今用臣言猶未晩如是者三太后命於朝堂杖之一百長流瀼州【貞觀十二年李弘節遣欽州首領甯師古尋劉方故道行逹交趾開拓夷獠置瀼州取瀼水以名州也舊志曰瀼州無兩京地里北至容州二百八十二里容州至京師五千九百一十里至東都五千四百八十五里瀼而章翻}
炎之下獄也郎將姜嗣宗使至長安劉仁軌問以東都事嗣宗曰嗣宗覺裴炎有異於常久矣仁軌曰使人覺之邪嗣宗曰然仁軌曰仁軌有奏事願附使人以聞【下遐嫁翻將即亮翻使疏吏翻邪音耶}
嗣宗曰諾明日受仁軌表而還表言嗣宗知裴炎反不言太后覽之命拉嗣宗於殿庭絞於都亭【先拉其幹而後絞殺之還從宣翻又音如字拉盧合翻}
丁酉追削李敬業祖考官爵發冢斵棺復姓徐氏 李景諶罷為司賓少卿【是年改鴻臚為司賓諶氏壬翻少始照翻}
以右史武康沈君諒著作郎崔詧為正諫大夫同平章事徐敬業聞李孝逸將至自潤州回軍拒之屯高郵之下阿溪【高郵縣漢屬廣陵國魏省晉武帝復置梁置廣業郡隋廢郡以高郵縣屬江都郡唐屬揚州九域志在州西北一百里宋白曰揚州天長縣本廣陵縣地唐開元二十九年於下阿置千秋縣天寶五年改天長梁曾於石梁置涇州以此言之蓋下阿溪即今石梁河也}
使徐敬猷逼淮隂【淮隂縣漢屬臨淮郡晉屬廣陵郡後魏置淮隂郡隋廢入山陽縣乾封元年分山陽復置屬楚州}
别將韋超尉遲昭屯都梁山【盱眙縣有都梁山將即亮翻下同尉紆勿翻}
李孝逸軍至臨淮【臨淮泗州}
偏將雷仁智與敬業戰不利孝逸懼按兵不進殿中侍御史魏元忠謂孝逸曰天下安危在兹一舉四方承平日久忽聞狂狡注心傾耳以俟其誅今大軍久留不進遠近失望萬一朝廷更命它將以代將軍【將即亮翻更工衡翻}
將軍何辭以逃逗撓之辠乎【逗音豆撓奴教翻}
孝逸乃引軍而前壬寅馬敬臣擊斬尉遲昭於都梁山十一月辛亥以左鷹揚大將軍黑齒常之為江南道大摠管討敬業【是年改左右武衛為左右鷹揚衛}
韋超擁衆據都梁山諸將皆曰超憑險自固士無所施其勇騎無所展其足且窮寇死戰攻之多殺士卒不如分兵守之大軍直趣江都覆其巢穴支度使薛克揚曰【唐制凡天下邊軍有支度使以計軍資糧仗之用所費皆申度支會計以長行旨為準趣七喻翻使疏吏翻}
超雖據險其衆非多今多留兵則前軍勢分少留兵則終為後患【少詩沼翻}
不如先擊之其埶必舉舉都梁則淮隂高郵望風瓦解矣魏元忠請先擊徐敬猷諸將曰不如先攻敬業敬業敗則敬猷不戰自擒矣若擊敬猷則敬業引兵救之是腹背受敵也元忠曰不然賊之精兵盡在下阿烏合而來利在一决萬一失利大事去矣敬猷出於博徒不習軍事其衆單弱人情易揺【易以豉翻}
大軍臨之駐馬可克敬業雖欲救之計程必不能及我克敬猷乘勝而進雖有韓白不能當其鋒矣今不先取弱者而遽攻其彊非計也孝逸從之引兵擊超超夜遁進擊敬猷敬猷脱身走庚申敬業勒兵阻溪拒守後軍摠管蘇孝祥夜將五千人以小舟度溪先擊之兵敗孝祥死士卒赴溪溺死者過半【將即亮翻溺奴狹翻過古禾翻}
左豹韜衛果毅漁陽成三朗為敬業所擒【是年改左右威衛為左右豹韜衛唐制諸府各有果毅都尉上府從五品下中府正六品上下府從六品下}
唐之奇紿其衆曰【紿蕩亥翻}
此李孝逸也將斬之三朗大呼曰【呼火故翻}
我果毅成三朗非李將軍也官軍今大至矣爾曹破在朝夕我死妻子受榮爾死妻子籍没爾終不及我也遂斬之孝逸等諸軍繼至戰數不利孝逸懼欲引退魏元忠與行軍管記劉知柔言於孝逸曰風順荻乾【數所角翻乾音干}
此火攻之利固請决戰敬業置陳既久士卒多疲倦顧望陳不能整孝逸進擊之因風縱火敬業大敗斬首七千級溺死者不可勝紀【陳讀曰陣勝音升}
敬業等輕騎走入江都挈妻子奔潤州將入海奔高麗【麗力知翻}
孝逸進屯江都分遣諸將追之乙丑敬業至海陵界阻風【海陵縣漢屬臨淮後漢晉屬廣陵梁置海陵郡隋廢郡為縣屬江都郡唐属揚州今為泰州九域志揚州東至海陵界九十八里又自海陵東至海一百七里}
其將王那相斬敬業敬猷及駱賓王首來降【考異曰唐紀初官軍逆風不利俄而風回甚勁孝逸縱火賊愳燒而潰敬業猷之奇求仁賓王走歸江都焚}


【簿書擕妻子潜蒜山下手書召宗臣敬業初與宗臣木契為約時亡其契宗臣疑而不赴或云宗臣已歸順敬業入海欲奔東夷至五陵界阻風偽將王那相斬之來降餘黨赴水死今從實録唐統紀}
餘黨唐之奇魏思温皆捕得傳首神都揚潤楚三州平

陳嶽論曰敬業苟能用魏思温之策直指河洛專以匡復為事縱軍敗身戮亦忠義在焉而妄希金陵王氣是真為叛逆不敗何待

敬業之起也使敬猷將兵五千循江西上【將即亮翻上時掌翻}
略地和州前弘文館學士歷陽高子貢帥鄉里數百人拒之敬猷不能西以功拜朝散大夫成均助教【歷陽縣漢属九江郡晉置歷陽郡暨至北齊與梁通和置和州隋唐因之后改國子監為成均監按唐六典弘文館以五品以上為學士國子助教則從六品上耳掌佐博士分經以教授朝散大夫從五品下帥讀曰率朝直遥翻散悉亶翻}
丁卯郭待舉罷為左庶子以鸞臺侍郎韋方質為鳳

閣侍郎同平章事方質雲起之孫也【韋雲起仕隋唐之間}
十二月劉景先又貶吉州員外長史郭待舉貶岳州刺史【岳州京師東南二千二百三十七里至東都一千八百一十六里}
初裴炎下獄單于道安撫大使左武衛大將軍程務挺密表申理由是忤旨【下遐嫁翻單音蟬使疏吏翻忤五故翻}
務挺素與唐之奇杜求仁善或譛之曰務挺與裴炎徐敬業通謀癸卯遣左鷹揚將軍裴紹業即軍中斬之 【考異曰唐統紀曰既而太后震怒召羣臣謂曰朕於天下無負羣臣皆知之乎羣臣曰唯太后曰朕事先帝二十餘年憂天下至矣公卿富貴皆朕與之天下安樂朕長養之及先帝弃羣臣以天下託顧於朕不愛身而愛百姓今為戎首皆出於將相羣臣何負朕之深也且卿輩有受遺老臣倔彊難制過裴炎者乎有將門貴種能糾合亡命過徐敬業者乎有握兵宿將攻戰必勝過程務挺者乎此三人者人望也不利於朕朕能戮之卿等有能過此三者當即為之不然須革心事朕無為天下笑羣臣頓首不敢仰視曰唯太后所使恐武后亦不至輕淺如此今不取}
籍没其家突厥聞務挺死所在宴飲相慶又為務挺立祠【為于偽翻}
每出師必禱之太后以夏州都督王方翼與務挺連職素相親善且廢后近屬徵下獄【夏戶雅翻下遐嫁翻}
流崖州而死【舊志崖州至京師七千四百六十里至東都六千三百里}


垂拱元年春正月丁未朔赦天下改元 太后以徐思文為忠特免緣坐拜司僕少卿【緣坐者緣親黨而坐罪也光宅改太僕為司僕}
謂曰敬業改卿姓武朕今不復奪也【復扶又翻 考異曰實録云思文表請改姓武許之蓋太后有此言思文因請之也今從唐紀}
庚戍以騫味道守内史【内史中書令}
戊辰文昌左相同鳳閣鸞臺三品樂城文獻公劉仁軌薨【文昌左相即尚書左僕射}
二月癸未制朝堂所置登聞鼓及肺石【登聞鼓在西朝堂肺石在東朝堂朝直遥翻}
不須防守有撾鼓立石者令御史受狀以聞【撾則瓜翻}
乙巳以春官尚書武承嗣秋官尚書裴居道【光宅以禮部為春官刑部為秋官尚辰羊翻嗣祥吏翻}
右肅政大夫韋思謙【右肅政大夫右御史大夫}
並同鳳閣鸞臺三品突厥那史那骨篤禄等數寇邊以左玉鈐衛中郎將淳于處平為陽曲道行軍摠管擊之【厥九勿翻數所角翻鈐其廉翻將即亮翻處昌呂翻陽曲縣自漢以來屬太原郡隋惡其名改曰陽直武德三年分置汾陽縣七年省陽直縣改汾陽為陽曲縣仍移治陽直}
正諫大夫同平章事沈君諒罷 三月正諫大夫同平章事崔詧罷 丙辰遷廬陵王于房州【舊志房州京師南一千一百九十五里至東都一千一百八十五里杜佑曰房州古麇庸二國之地春秋楚子敗麇師於房渚即此曹魏為新城郡竹山縣則古庸國秦漢之上庸縣也}
辛酉武承嗣罷辛未頒垂拱格 朝士有左遷詣宰相自訴者内史騫味道曰此太后處分【朝直遥翻處昌呂翻分扶問翻}
同中書門下三品劉禕之曰緣坐改官由臣下奏請太后聞之夏四月丙子貶味道為青州刺史加禕之太中大夫【太中大夫從四品上劉禕之本職豫王府司馬王府司馬從四品下}
謂侍臣曰君臣同體豈得歸惡於君引善自取乎 癸未突厥寇代州淳于處平引兵救之至忻州為突厥所敗【敗補邁翻}
死者五千餘人 丙午以裴居道為内史納言王德真流象州【象州至京師四千九百八十九里}
己酉以冬官尚書蘇良嗣為納言【光宅改工部為冬官}
壬戌

制内外九品以上及百姓咸令自舉【令有才者咸得自言以求進用令力丁翻}
壬申韋方質同鳳閣鸞臺三品 六月天官尚書韋待價同鳳閣鸞臺三品【光宅改吏部為天官}
待價萬石之兄也同羅僕固等諸部叛遣左豹韜衛將軍劉敬同發河

西騎士出居延海以討之【甘州刪丹縣北渡張掖河西北行出合黎山峽口傍河東壖屈曲東北行千里有寧寇軍軍東北有居延海騎奇寄翻}
同羅僕固等皆敗散敇僑置安北都護府於同城以納降者【同城即刪丹之同城守捉天寶二載改為寧寇軍降戶江翻}
秋七月己酉以文昌左丞魏玄同為鸞臺侍郎同鳳閣鸞臺三品【文昌左丞即尚書左丞}
詔自今祀天地高祖太宗高宗皆配坐【坐徂臥翻}
用鳳閣舍人元萬頃等之議也 九月丁卯廣州都督王果討反獠平之【獠魯皓翻}
冬十一月癸卯命天官尚書韋待價為燕然道行軍

大摠管以討吐蕃【燕因肩翻吐從暾入聲}
初西突厥興㫺亡繼往絶可汗既死十姓無主部落多散亡太后乃擢興㫺亡之子左豹韜衛翊府中郎將元慶【唐諸衛皆有翊府中郎將郎將將即亮翻}
為左玉鈐衛將軍兼崑陵都護襲興昔亡可汗押五咄陸部落【鈐其廉翻可從刋入聲汗音寒咄當沒翻}
麟臺正字射洪陳子昂【光宅改秘書省為麟臺正字正九品下掌刋正文字射洪縣屬梓州漢郪縣地後魏分置射江縣以婁縷灘東六里有射江西魏訛為射洪}
上疏以為朝廷遣使廵察四方不可任非其人【上時掌翻使疏吏翻下同}
及刺史縣令不可不擇比年百姓疲於軍旅不可不安【比毗至翻}
其略曰夫使不擇人則黜陟不明刑罰不中【夫音扶中竹仲翻}
朋黨者進貞直者退徒使百姓修飾道路送往迎來無所益也諺曰欲知其人觀其所使不可不慎也又曰宰相陛下之腹心刺史縣令陛下之手足未有無腹心手足而能獨理者也又曰天下有危機禍福因之而生機静則有福機動則有禍百姓是也百姓安則樂其生【樂音洛}
不安則輕其死輕其死則無所不至祅逆乘釁天下亂矣【祅於喬翻}
又曰隋煬帝不知天下有危機而信貪佞之臣冀收夷狄之利卒以滅亡【卒子恤翻}
其為殷鑒豈不大哉 太后修故白馬寺以僧懷義為寺主【姚思廉曰漢明帝時西域以白馬負佛經送洛因立白馬寺魏收曰漢立白馬寺於洛城雍關西按此故洛城也唐之洛城乃隋所遷}
懷義鄠人【鄠音戶}
本姓馮名小寶賣藥洛陽市因千金公主以進【千金公主高祖女}
得幸於太后太后欲令出入禁中乃度為僧名懷義又以其家寒微令與駙馬都尉薛紹合族命紹以季父事之【薛紹尚后女太平公主}
出入乘御馬宦者十餘人侍從【從才用翻}
士民遇之者皆奔避有近之者【近其靳翻}
輒檛其首流血【檛其瓜翻}
委之而去任其生死見道士則極意之仍髠其髮而去朝貴皆匍匐禮謁【敺烏口翻朝直遥翻匍薄乎翻匐蒲比翻}
武承嗣武三思皆執僮僕之禮以事之為之執轡【為于偽翻}
懷義視之若無人多聚無賴少年度為僧縱横犯法【少詩照翻横下孟翻}
人莫敢言右臺御史馮思勗屢以法繩之【右臺右肅政臺也}
懷義遇思朂於途令從者之幾死【幾居依翻}


二年春正月太后下詔復政於皇帝睿宗知太后非誠心奉表固讓太后復臨朝稱制【復扶又翻朝直遥翻}
辛酉赦天下二月辛未朔日有食之 右衛大將軍李孝逸既克

徐敬業聲望甚重武承嗣等惡之數譛於太后左遷施州刺史【惡烏路翻數所角翻施州漢巫縣地吳分巫立沙渠縣後周於縣置施州隋廢州為清江郡唐復置施州在京師南二千七百九里至東都二千八百一十里}
三月戊申太后命鑄銅為匭【匭居洧翻}
其東曰延恩獻賦頌求仕進者投之南曰招諫言政得失者投之西曰伸寃有寃抑者投之北曰通玄言天象災變及軍機祕計者投之【四匭各依其方色}
命正諫補闕拾遺一人掌之【正諫即諫議大夫也垂拱元年置左右補闕各一人從七品上左右拾遺各一人從八品上掌供奉諷諫行立次左右史之下左属門下省右属中書省}
先責識官【識官猶今之保識}
乃聽投表疏【疏所去翻}
徐敬業之反也侍御史魚承曄之子保家教敬業作刀車及弩敬業敗僅得免太后欲周知人間事保家上書請鑄銅為匭以受天下密奏【上時掌翻}
其器共為一室中有四隔上各有竅以受表疏可入不可出太后善之 【考異曰統紀唐歷皆云八月作銅匭今從實録舊本紀又朝野僉載作魚思咺云上欲作匭召工匠無人作得者思咺應制為之甚合規矩遂用之今從御史臺記}
未幾其怨家投匭【怨於元翻幾居豈翻}
告保家為敬業作兵器殺傷官軍甚衆遂伏誅【為于偽翻}
太后自徐敬業之反疑天下人多圖已又自以久專國事且内行不正【行下孟翻}
知宗室大臣怨望心不服欲大誅殺以威之乃盛開告密之門有告密者臣下不得問皆給驛馬【唐制乘傳日四驛乘驛日六驛凡給馬者一品八匹二品六匹三品五匹四品五品四匹六品三匹七品以下二匹給傳乘者一品十馬二品九馬三品八馬四品五品四馬六品七品二馬八品九品一馬三品已上敕召者給四馬五品三馬六品已下有差一驛三十里}
供五品食【唐六典四品五品常食料七盤每日細米二升麫二升三合酒一升半羊肉三分瓜兩顆鹽豉葱薑葵韭之属各有差新唐志五品食料雜用錢月六百}
使詣行在雖農夫樵人皆得召見廩於客舘【客館屬鴻臚寺典客令廩者廩給之見賢遍翻}
所言或稱旨則不次除官【稱尺證翻}
無實者不問於是四方告密者蜂起人皆重足屏息【重直龍翻屏必郢翻}
有胡人索元禮【索蘇各翻}
知太后意因告密召見擢為游擊將軍令案制獄【見賢遍翻令力丁翻}
元禮性殘忍推一人必令引數十百人太后數召見賞賜以張其權【數所角翻張知亮翻}
於是尚書都事長安周興【唐尚書都省有都事管諸司主事令史尚辰羊翻}
萬年人來俊臣之徒效之紛紛繼起興累遷至秋官侍郎俊臣累遷至御史中丞相與私畜無賴數百人【畜吁玉翻}
專以告密為事欲陷一人輒令數處俱告事狀如一俊臣與司刑評事洛陽萬國俊【光宅改大理為司刑評事從八品掌出使推劾}
共撰羅織經數千言教其徒網羅無辜織成反狀構造布置皆有支節太后得告密者輒令元禮等推之競為訊囚酷法有定百脉突地吼死豬愁求破家反是實等名號或以椽關手足而轉之謂之鳳皇曬翅或以物絆其腰引枷向前謂之驢駒拔撅【椽重緣翻曬所賣翻絆博慢翻撅其月翻}
或使跪捧枷累甓其上謂之仙人獻果或使立高木引枷尾向後謂之玉女登梯或倒懸石縋其首或以醋灌鼻或以鐵圈轂其首而加楔【枷音加甓扶歷翻縋馳偽翻圈丘員翻轂呼角翻急束也楔先結翻}
至有腦裂髓出者每得囚輒先陳其械具以示之皆戰栗流汗望風自誣每有赦令俊臣輒令獄卒先殺重囚然後宣示太后以為忠益寵任之中外畏此數人甚於虎狼麟臺正字陳子昂上疏【上時掌翻疏所據翻}
以為執事者疾徐敬業首亂唱禍將息姦源窮其黨與遂使陛下大開詔獄重設嚴刑有迹涉嫌疑辭相逮引莫不窮捕考案至有姦人熒惑乘險相誣糾告疑似冀圖爵賞恐非伐罪弔人之意也臣竊觀當今天下百姓思安久矣故揚州搆逆殆有五旬而海内晏然纎塵不動陛下不務玄默以救疲人而反任威刑以失其望臣愚暗昧竊有大惑伏見諸方告密囚累百千輩及其窮竟百無一實陛下仁恕又屈法容之遂使奸惡之黨快意相讐睚眦之嫌即稱有密一人被訟【被皮義翻}
百人滿獄使者推捕冠蓋如市或謂陛下愛一人而害百人天下喁喁【喁魚容翻}
莫知寧所臣聞隋之末代天下猶平楊玄感作亂不踰月而敗天下之弊未至土崩蒸人之心猶望樂業【蒸人猶蒸民也避太宗諱改民為人樂音洛}
煬帝不悟遂使兵部尚書樊子蓋專行屠戮大窮黨與海内豪士無不罹殃遂至殺人如麻流血成澤【事見一百八十二卷大業九年}
天下靡然始思為亂於是雄傑並起而隋族亡矣夫大獄一起不能無濫寃人吁嗟感傷和氣羣生癘疫水旱隨之人既失業則禍亂之心怵然而生矣古者明王重慎刑法蓋懼此也昔漢武帝時巫蠱獄起使太子奔走兵交宫闕無辜被害者以千萬數宗廟幾覆賴武帝得壺關三老書廓然感悟夷江充三族【事見二十二卷漢武帝征和二年三年幾居依翻}
餘獄不論天下以安爾古人云前事之不忘後事之師【史記太史公之言}
伏願陛下念之太后不聽 夏四月太后鑄大儀置北闕【北闕蓋在玄武門外}
以岑長倩為内史六月辛未以蘇良嗣為左相同鳳閣鸞臺三品韋待價為右相己卯以韋思謙為納言蘇良嗣遇僧懷義於朝堂懷義偃蹇不為禮良嗣大怒命左右捽曳批其頰數十【捽昨沒翻批蒲列翻擊也又匹迷翻}
懷義訴於太后太后曰阿師當於北門出入【阿烏葛翻}
南牙宰相所往來勿犯也太后託言懷義有巧思【思相吏翻}
故使入禁中營造補闕長社王求禮上表【長社漢縣隋改曰穎川武德四年復舊帶許州上時掌翻}
以為太宗時有羅黑黑善彈琵琶太宗閹為給使使教宫人陛下若以懷義有巧性欲宫中驅使者臣請閹之庶不亂宫闈表寢不出 秋九月丁未以西突厥繼往絶可汗之子斛瑟羅為右玉鈐衛將軍襲繼往絶可汗押五弩失畢部落 己巳雍州言新豐縣東南有山踊出【雍於用翻 考異曰統紀在十二月今從實録 程大昌曰武后改新豐為慶山縣其說曰時因雷雨踊出一山故取以為名而其何以輒湧也不言其以也此即在位小人共加傳會也至兩京道里志則言其詳矣曰慶山踊出初時六七尺漸高至三百尺則非一旦驟為三百尺也自六七尺日日纍增至三百尺是積力為之非一夜雷雨頓能突兀如許也此為人力所成大不難見}
改新豐為慶山縣【新豐自漢以來屬京兆}
四方畢賀江陵人俞文俊上書【江陵縣帶荆州}
天氣不和而寒暑併人氣不和而疣贅生地氣不和而塠阜出【疣音尤贅之芮翻塠都回翻}
今陛下以女主處陽位反易剛柔故地氣塞隔【塞悉則翻處昌呂翻}
而山變為災陛下謂之慶山臣以為非慶也臣愚以為宜側身修德以答天譴不然殃禍至矣太后怒流於嶺外後為六道使所殺【六道使見後二百五卷長夀二年使疏吏翻}
突厥入寇左鷹揚衛大將軍黑齒常之拒之至兩井遇突厥三千餘人見唐兵皆下馬擐甲常之以二百餘騎衝之【擐音宦騎奇寄翻}
皆弃甲走日暮突厥大至常之令營中然火東南又有火起虜疑有兵相應遂夜遁 狄仁傑為寧州刺史右臺監察御史晉陵郭翰廵察隴右所至多所按劾【監古銜翻劾戶槩翻又戶得翻}
入寧州境耆老歌刺史德美者盈路翰薦之於朝【朝直遥翻}
徵為冬官侍郎

資治通鑑卷二百三
















































































































































