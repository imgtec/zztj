






























































資治通鑑卷二百四十一 宋 司馬光 撰

胡三省 音注

唐紀五十七【起屠維大淵獻二月盡重光赤奮若六月凡二年有奇}


憲宗昭文章武大聖至神孝皇帝下

元和十四年二月李聽襲海州克東海朐山懷仁等縣【海州治朐山本漢胊縣後人加山字東海漢贛榆縣地後齊置東海縣屬東海郡隋廢郡及縣入廣饒縣隋仁壽元年改廣饒曰東海避太子諱也唐屬海州九域志在州東一十里懷仁縣梁置南北二青州東魏廢州置義塘郡及懷仁縣隋廢郡以縣屬海州九域志在州北八十里宋白曰海州懷仁縣本漢贛餘縣地按漢贛餘今縣東北三十里贛餘古城是也梁於此置黄郭戍後魏置義塘郡理黄郭城領義唐歸義懷仁三縣高齊移義唐郡及懷仁縣並理今密州莒縣界隋開皇廢郡移懷仁縣理此今縣理是也}
李愬敗平盧兵於沂州拔丞縣【丞漢縣後魏置蘭陵郡隋廢郡為蘭陵縣武德四年改曰丞縣後屬沂州九域志在州西南一百八十里宋白曰丞漢舊縣春秋時鄫國也晉置蘭陵郡理丞城按前此丞縣理在今縣西一里漢丞縣故城是也隋開皇十六年置鄫州及丞縣尋廢州及縣仍移蘭陵縣置於廢鄫州故城中唐又改蘭陵為丞縣縣西北有丞水敗補邁翻丞時證翻}
李師道聞官軍侵逼民治鄆州城塹修守備【治直之翻塹七豔翻}
役及婦人民益懼且怨都知兵馬使劉悟正臣之孫也【劉正臣見二百一十七卷肅宗至德元載}
師道使之將兵萬餘人屯陽穀以拒官軍悟務為寛惠使士卒人人自便軍中號曰劉父及田弘正度河悟軍無備戰又數敗【數所角翻}
或謂師道曰劉悟不修軍法專收衆心恐有他志宜早圖之師道召悟計事欲殺之或諫曰今官軍四合悟無逆狀用一人言殺之諸將誰肯為用是自脱其爪牙也師道留悟旬日復遣之厚贈金帛以安其意悟知之還營隂為之備師道以悟將兵在外署悟子從諫門下别奏【門下别奏者使厠員牙門下俟别奏補官也唐六典凡諸軍鎮大使三品已上傔二十五人别奏十人副使傔二十人别奏八人總管三品已上傔十八人别奏六人子總管四品已上傔十一人别奏三人若討擊防禦遊奕使副傔準品各减三人别奏各減二人總管及子總管傔準品各減二人别奏各减一人若鎮守已下無副使或隸屬大軍鎮者使已下傔奏並四分減一所補傔奏皆令自召以充}
從諫與師道諸奴日遊戲頗得其隂謀密疏以白父又有謂師道者曰劉悟終為患不如早除之丙辰師道潜遣二使齎帖授行營兵馬副使張暹令斬悟首獻之勒暹權領行營時悟方據高丘張幕置酒去營二三里二使至營密以帖授暹暹素與悟善陽與使者謀曰悟自使府還【還音旋又如字}
頗為備不可怱怱暹請先往白之云司空遣使存問將士兼有賜物請都頭速歸【軍中稱都將為都頭}
同受傳語【傳語謂師道遣使者所傳言語也}
如此則彼不疑乃可圖也使者然之暹懷帖走詣悟屏人示之【屏必郢翻又卑正翻}
悟潜遣人先執二使殺之時已向暮悟按轡徐行還營坐帳下嚴兵自衛召諸將厲色謂之曰悟與公等不顧死亡以抗官軍誠無負於司空今司空信讒言來取悟首悟死諸公其次矣且天子所欲誅者獨司空一人今軍勢日蹙吾曹何為隨之族滅欲與諸公卷旗束甲【卷與捲同}
還入鄆州奉行天子之命【言奉行詔旨以誅李師道}
豈徒免危亡富貴可圖也諸公以為何如兵馬使趙垂棘立於衆首良久對曰事果濟否悟應聲罵曰汝與司空合謀邪立斬之徧問其㳄有遲疑未言者悉斬之并斬軍中素為衆所惡者【惡烏路翻}
凡三十餘尸於帳前餘皆股栗曰惟都頭命願盡死乃令士卒曰入鄆人賞錢百緡惟不得近軍帑【近其靳翻帑他朗翻}
其使宅及逆黨家財任自掠取【使宅謂節度使所居也}
有仇者報之使士皆飽食執兵夜半聽鼔三聲絶即行人銜枚馬縛口遇行人執留之【恐行人遇兵走還城報師道令執留之}
人無知者距城數里天未明悟駐軍使聽城上柝聲絶【天明則柝聲絶}
使十人前行宣言劉都頭奉帖追入城【主帥文書下諸將謂之帖}
門者請俟寫簡白使【古者聨竹為簡策以寫書後世因謂書為簡白使謂白節度使使疏吏翻}
十人拔刃擬之皆竄匿悟引大軍繼至城中譟譁動地比至【比必利翻及也}
子城已洞開惟牙城拒守【凡大城謂之羅城小城謂之子城又有第三重城以衛節度使居宅謂之牙城}
尋縱火斧其門而入牙中兵不過數百始猶有弓矢者俄知力不支皆投於地悟勒兵升聽事使捕索師道【索山客翻}
師道與二子伏厠牀下索得之【索山客翻}
悟命置牙門外隙地使人謂曰悟奉密詔送司空歸闕然司空亦何顔復見天子【復扶又翻}
師道猶有幸生之意其子弘方仰曰事已至此速死為幸尋皆斬之【代宗永泰元年李正已得淄青四世五十四年而滅}
自卯至午悟乃命兩都虞侯巡坊市禁掠者即時皆定大集兵民於毬場親乘馬巡繞慰安之斬贊師道逆謀者二十餘家文武將吏且懼且喜悟見李公度執手歔欷出賈直言於獄【直言被囚見上卷上}
年置之幕府悟之自陽穀還兵趨鄆也【趨七喻翻}
潛使人以其謀告田弘正事成當舉烽相白萬一城中有備不能入願公引兵為助功成之日皆歸於公悟何敢有之且使弘正進據已營弘正見烽知得城遣使往賀悟函師道父子三首遣使送弘正營弘正大喜露布以聞淄青等十二州皆平弘正初得師道首疑其非真召夏侯澄使識之澄熟視其面長號隕絶者久之乃抱其首舐其目中塵垢復慟哭弘正為之改容義而不責【識如字辨識也號戶刀翻舐直氏翻復扶又翻為于偽翻夏侯澄禽見上卷上年}
壬戌田弘正捷奏至乙丑命戶部侍郎楊於陵為淄青宣撫使己巳李師道首函至自廣德以來垂六十年藩鎮跋扈河南北三十餘州自除官吏不供貢賦至是盡遵朝廷約束【嗚呼兼并易也堅凝之難讀史至此盍亦知其所以得鑒其所以失則知資治通鑑一書不苟作矣}
上命楊於陵分李師道地於陵按圖籍視土地遠邇計士馬衆寡校倉庫虚實分為三道使之適均【於音烏}
以鄆曹濮為一道【鄆音運濮音卜}
淄青齊登萊為一道兖海沂密為一道上從之劉悟以初討李師道詔云部將有能殺師道以衆降者師道官爵悉以與之意謂盡得十二州之地遂補署文武將佐更易州縣長吏【更工衡翻}
謂其下曰軍府之政一切循舊自今但與諸公抱子弄孫夫復何憂【復扶又翻下復須同}
上欲移悟他鎮恐悟不受代復須用兵密詔田弘正察之弘正日遣使者詣悟託言修好實觀其所為悟多力好手【好呼到翻}
得鄆州三日則敎軍中壯士手與魏博使者庭觀之自揺肩攘臂離坐以助其勢【離力智翻坐徂臥翻}
弘正聞之笑曰是聞除改【除改謂除書改授他鎮}
登即行矣【言登時即行也}
何能為哉庚午以悟為義成節度使悟聞制下手足失墜【言驚遽失守不知所為}
明日遂行弘正已將數道比至城西二里與悟相見於客亭【客亭驛亭送迎使客之所}
即受旌節馳詣滑州辟李公度李存郭昈賈直言以自隨悟素與李文會善既得鄆州使召之未至【李文會出登州見上卷上年}
聞將移鎮昈存謀曰文會佞人敗亂淄青一道【敗補邁翻}
滅李司空之族萬人所共讎也不乘此際誅之田相公至務施寛大將何以雪三齊之憤怨乎【自項羽分齊為三以王田市田都田安遂有三齊之名後人因而言之}
乃詐為悟帖遣使即文會所至取其首以來使者遇文會於豐齊驛斬之【據梁敬翔編遺録豐齊驛當在齊州東南三十里宋白曰齊州禹城縣有漢祝阿故城在豐齊驛東北二里}
比還【比必利翻及也還音旋又如字}
悟及昈存已去無所復命矣文會一子一亡去一死於獄家貲悉為人所掠田宅没官詔以淄青行營副使張暹為戎州刺史【劉悟奏言其功也}
癸酉加田弘正檢校司徒同平章事先是李師道將敗數月【先悉薦翻}
聞風動鳥飛皆疑有變禁鄆人親識宴聚及道路偶語犯者有刑弘正既入鄆悉除苛禁縱人遊樂【樂音洛}
寒食七晝夜不禁行人【弘正特為此示鄆人以寛大耳案寒食之說不同初學記曰周禮司烜氏仲春以木鐸徇火禁於國中注云為季春將出火也今寒食準節氣是仲春之末清明是三月之初然則禁火並周制也洪容齋曰先賢傳曰太原舊俗以介子推焚骸一月寒食鄴中記曰并州俗冬至後一百五日為子推斷火冷食三日魏武以太原上黨西河皆沍寒之地令人不得寒食此注已見前}
或諫曰鄆人久為寇敵今雖平人心未安不可不備弘正曰今為暴者既除宜施以寛惠若復為嚴察是以桀易桀也庸何愈焉【愈賢也勝也復扶又翻}
先是賊數遣人入關截陵戟焚倉場流矢飛書以震駭京師沮撓官軍【事見二百三十九卷十年數所角翻沮在呂翻撓奴巧翻}
有司督察甚嚴潼關吏至人囊篋以索之【索山客翻}
然終不能絶及田弘正入鄆閲李師道簿書有賞殺武元衡人王士元等及賞潼關蒲津吏卒案乃知曏者皆吏卒受賂於賊容其姦也【案文案也亦謂之案牘史言關津不足以禁姦乃所以容姦}
裴度纂述蔡鄆用兵以來上之憂勤機略因侍宴獻之請内印出付史官【請自禁中用印而出付史官}
上曰如此似出朕志非所欲也弗許【史言憲宗此事得為君之體}
三月戊子以華州刺史馬摠為鄆曹濮等州節度使己丑以義成節度使薛平為平盧節度淄青齊登萊等州觀察使【自是之後淄青專平盧之號而鄆尋賜號天平軍矣}
以淄青四面行營供軍使王遂為沂海兖密等州觀察使【為王遂以嚴酷召亂張本}
横海節度使烏重奏河溯蕃鎮所以能旅拒朝命

六十餘年者由諸州縣各置鎮將領事收刺史縣令之權自作威福曏使刺史各得行其職則雖有姧雄如安史必不能以一州獨反也臣所領德棣景三州已舉牒各還刺史職事應在州兵並令刺史領之夏四月丙寅詔諸道節度都團練都防禦經略等使所統支郡兵馬並令刺史領之自至德以來節度使權重所統諸州各置鎮兵以大將主之暴横為患【横戶孟翻}
故重論之其後河北諸鎮惟横海最為順命由重處之得宜故也【史言反側之地擇帥不可不詳處昌呂翻}
辛未工部侍郎同平章事程异薨裴度在相位知無不言皇甫鎛之黨隂擠之【擠子細翻又子}


【西翻 考異曰舊傳曰鎛與宰相李逢吉令狐楚合勢擠度故出鎮按逢吉時在東川楚在昭義皆不為相今不取按後昭義當作河陽}
丙子詔度以門下侍郎同平章事充河東節度使皇甫鎛專以掊克取媚【掊蒲侯翻}
人無敢言者獨諫議大夫武儒衡上疏言之鎛自訴於上上曰卿以儒衡上疏將報怨邪鎛乃不敢言儒衡元衡之從父弟也【從才用翻}
史館修撰李翺上言【貞觀三年置史館於門下省有修撰四人掌修國史}
以為定禍亂者武功也興太平者文德也今陛下既以武功定海内若遂革弊事復高祖太宗舊制用忠正而不疑屏邪佞而不邇【屏必郢翻又卑正翻}
改税法不督錢而納布帛【自建中初楊炎定兩税法不令民輸其土之所產而督錢}
絶進獻寛百姓租賦厚邊兵以制戎狄侵盜數訪問待制官以通塞蔽【數所角翻塞悉則翻}
此六者政之根本太平之所以興也陛下既已能行其難若何不為其易乎以陛下天資上聖如不惑近習容悦之辭任骨鯁正直之士與之興大化可不勞而成也若不以此為事臣恐大功之後逸欲易生【易以豉翻}
進言者必曰天下既平矣陛下可以高枕自安逸【枕職任翻}
如是則太平未可期矣 秋七月丁丑朔田弘正送殺武元衡賊王士元等十六人詔仗内京兆府御史臺徧鞫之皆款服【款誠也言吐誠而伏罪也}
京兆尹崔元略以元衡物色詢之則多異同元略問其故對曰恒鄆同謀遣客刺元衡【恒戶登翻刺七亦翻}
而士元等後期聞恒人事已成遂竊以為己功還報受賞耳今自度為罪均【度徒洛翻}
終不免死故承之上亦不欲復辨正悉殺之【復扶又翻}
戊寅宣武節度使韓弘始入朝【蔡鄆既平韓弘始入朝}
上待之甚厚弘獻馬三千絹五千雜繒三萬金銀器千【繒慈陵翻}
而汴之庫廏尚有錢百餘萬緡絹百餘萬匹馬七千匹糧三百萬解【史言韓弘善完聚}
己丑羣臣上尊號曰元和聖文神武法天應道皇帝赦天下兖海沂密觀察使王遂本錢穀吏性狷急無遠識【狷古掾翻}
時軍府草創【是年三月方分四州置觀察}
人情未安遂專以嚴酷為治【治直吏翻}
所用杖絶大於常行者【唐制凡杖皆長三尺五寸削去節目訊杖大頭徑三分二釐小頭二分二釐常行杖大頭二分七釐小頭一分七釐笞杖大頭二分小頭一分有半}
每詈將卒輒曰反虜又盛夏役士卒營府舍督責峻急將卒憤怨辛卯役卒王弁與其徒四人浴於沂水【沂州治臨沂縣以臨沂水名之也}
密謀作亂曰今服役觸罪亦死奮命立事亦死死於立事不猶愈乎明日常侍與監軍副使有宴軍將皆在告直兵多休息【常侍謂王遂也副使謂觀察副使也在告謂休假在私室也直兵直衛之兵也}
吾屬乘此際出其不意取之可以萬全四人皆以為然約事成推弁為留後壬辰遂方宴飲日過中弁等五人突入於直房前取弓刀【直房直兵之所舍之室也}
徑前射副使張敦實殺之【射而亦翻}
遂與監軍狼狽起走弁執遂數之以盛暑興役用刑刻暴【數所具翻}
立斬之傳聲勿驚監軍弁即自稱留後升廳號令與監軍抗禮召集將吏參賀衆莫敢不從監軍具以狀聞 甲午韓弘又獻絹二十五萬匹絁三萬匹【絁式支翻}
銀器二百七十左右軍中尉各獻錢萬緡自淮西用兵以來度支鹽鐵及四方争進奉謂之助軍賊平又進奉謂之賀禮後又進奉謂之助賞上加尊號又進奉亦謂之賀禮【史歷言元和進奉之弊}
丁酉以河陽節度使令狐楚為中書侍郎同平章事楚與皇甫鎛同年進士故鎛引以為相【裴度之視師也令狐楚出翰林今皇甫鎛引而相之亦所以杜度之再入}
朝廷聞沂州軍亂甲辰以棣州刺史曹華為沂海兖密觀察使 韓弘累表請留京師八月己酉以弘守司徒兼中書令癸丑以吏部尚書張弘靖同平章事充宣武節度使弘靖宰相子【弘靖張延賞次子延賞相德宗}
少有令聞【少詩照翻聞音問}
立朝簡默河東宣武闕帥【帥所類翻}
朝廷以其位望素重使鎮之弘靖承王鍔聚歛之餘韓弘嚴猛之後【王鍔鎮河東韓弘鎮宣武弘靖皆承其後歛力贍翻下同}
兩鎮喜其亷謹寛大故上下安之【張弘靖之簡貴施之并汴可也施之幽燕則敗矣}
己未田弘正入朝上待之尤厚 戊辰陳許節度使欷士美薨以庫部員外郎李渤為弔祭使渤上言臣過渭南聞長源鄉舊四百戶今纔百餘戶閺鄉縣舊三千戶今纔千戶【閺音旻}
其他州縣大率相似迹其所以然皆由以逃戶税攤於比隣【攤他干翻比音毗又毗至翻}
致驅迫俱逃此皆聚歛之臣剥下媚上【歛力贍翻}
惟思竭澤不慮無魚【呂氏春秋曰竭澤而漁豈不得魚而明年無魚}
乞降詔書絶攤逃之弊盡逃戶之產償税不足者乞免之計不數年人皆復於農矣執政見而惡之【執政謂皇甫鎛惡烏路翻}
渤遂謝病歸東都 癸酉吐蕃寇慶州【慶州隋弘化郡開皇十六年改為慶州以慶美取其嘉名漢歸德富平縣地舊志京師西北五百七十三里}
營於方渠 朝廷議興兵討王弁恐青鄆相扇繼變【青鄆與兖海沂密本一鎮也故恐其相扇而動}
乃除弁開州刺史遣中使賜以告身中使紿之曰開州計已有人迎候道路留後宜速弁即日沂州導從尚百餘人【從才用翻}
入徐州境所在減之其衆亦稍逃散遂加以杻械【杻敕久翻}
乘驢入關九月戊寅腰斬東市先是三分鄆兵以隸三鎮【此言鄆青沂分為三鎮之初先悉薦翻}
及王遂死朝廷以為師道餘黨凶態未除命曹華引棣州兵赴鎮以討之沂州將士迎候者華皆以好言撫之使先入城慰安其餘衆皆不疑華視事三日大饗將士伏甲士千人於幕下乃集衆而諭之曰天子以鄆人有遷徙之勞特加優給宜令鄆人處左沂人處右【處昌呂翻下聚處同}
既定令沂人皆出因闔門謂鄆人曰王常侍以天子之命為帥於此將士何得輒害之語未畢伏者出圍而殺之死者千二百人無一得脱者門屏間赤霧高丈餘久之方散【兵死之氣凝為赤霧}


臣光曰春秋書楚子䖍誘蔡侯般殺之于申【見昭十一年般音班}
彼列國也孔子猶深貶之惡其誘討也【惡烏路翻}
况為天子而誘匹夫乎王遂以聚歛之才殿新造之邦【殿多見翻鎮也}
用苛虐致亂王弁庸夫乘釁竊【釁隙也}
苟沂帥得人戮之易於犬豕耳【帥所類翻易以豉翻}
何必以天子詔書為誘人之餌乎且作亂者五人耳乃使曹華設詐屠千餘人不亦濫乎然則自今士卒孰不猜其將帥將帥何以令其士卒上下盻盻【盻盻恨視也說文音五計翻孫奭音五禮翻又普莧翻}
如寇讐聚處【處昌呂翻}
得間則更相魚肉【間古莧翻更工衡翻}
惟先者為雄耳禍亂何時而弭哉惜夫憲宗削平僭亂幾致升平【幾鉅依翻}
其美業所以不終由苟徇近功不敦大信故也

甲辰以田弘正兼侍中魏博節度使如故弘正三表請留上不許弘正常恐一旦物故魏人猶以故事繼襲故兄弟子姪皆仕諸朝上皆擢居顯列朱紫盈庭時人榮之 乙巳上問宰相玄宗之政先理而後亂何也崔羣對曰玄宗用姚崇宋璟盧懷慎蘇頲韓休張九齡則理用宇文融李林甫楊國忠則亂故用人得失所繫非輕人皆以天寶十四年安禄山反為亂之始臣獨以為開元二十四年罷張九齡相專任李林甫此理亂之所分也願陛下以開元初為法以天寶末為戒乃社稷無疆之福皇甫鎛深恨之【皇甫鎛自知以姦謟忝相位故深恨崔羣之言}
冬十月壬戌容管奏安南賊楊清陷都護府【安南都護府治交州}
殺都護李象古及妻子官屬部曲千餘人象古道古之兄也以貪縱苛刻失衆心清世為蠻酋象古召為牙將清鬱鬱不得志象古命清將兵三千討黄洞蠻【黄洞蠻即西原蠻其屬黄氏者謂之黄洞蠻}
清因人心怨怒引兵夜還襲府城陷之初蠻賊黄少卿自貞元以來數反覆【數所角翻}
桂管觀察使裴行立【數所角翻唐桂管管桂昭蒙富梧潯龔鬱林平琴賓澄繡象柳融等州}
容管經略使陽旻欲徼幸立功【徼堅堯翻}
争請討之上從之嶺南節度使孔戣屢諫曰此禽獸耳但可自計利害不足與論是非上不聽大江湖兵會容桂二管入討士卒被瘴癘死者不可勝計【被皮義翻勝音升}
安南乘之遂殺都護行立旻竟無功二管彫弊惟戣所部晏然【嶺南節度雖兼統五管而廣州所管自為巡屬劉昫曰廣州管韶循岡賀端新康封瀧恩春高籐義竇勤等州戣渠龜翻}
丙寅以唐州刺史桂仲武為安南都護赦楊清以為瓊州刺史 是歲吐蕃節度論三摩等將十五萬衆圍鹽州党項亦兵助之刺史李文悦竭力拒守凡二十七日吐蕃不能克靈武牙將史奉敬言於朔方節度使杜叔良請兵三千齎三十日糧深入吐蕃以解鹽州之圍叔良以二千五百人與之奉敬行旬餘無聲問朔方人以為俱没矣【以為與鹽州俱没}
無何【言無何時也}
奉敬自他道出吐蕃背吐蕃大驚潰去奉敬奮擊大破不可勝計【當曰奮擊大破之殺獲不可勝計文意乃為明暢}
奉敬與鳳翔將野詩良輔涇原將郝玭皆以勇著名於邊吐蕃憚之【新舊書皆作史敬奉}
柳泌至台州驅吏民采藥歲餘無所得而懼【泌知台州見上卷上年}
舉家逃入山中浙東觀察使捕送京師皇甫鎛李道古保護之上復使待詔翰林服其藥日加躁渴【躁則到翻}
起居舍人裴潾上言以為除天下之害者受天下之利同天下之樂者饗天下之福【樂音洛}
自黄帝至於文武享國壽考皆用此道也自去歲以來所在多薦方士轉相汲引其數浸繁借令天下真有神仙彼必深潜巖壑惟畏人知凡候伺權貴之門以大言自衒奇技驚衆者【伺相吏翻衒熒絹翻伎渠錡翻}
皆不軌徇利之人豈可信其說而餌其藥邪夫藥以愈疾非朝夕常餌之物况金石酷烈有毒又益以火氣殆非人五藏之所能勝也【藏但浪翻勝音升}
古者君飲藥臣先嘗之【記曲禮之言}
乞令獻藥者先自餌一年則真偽自可辨矣上怒十一月己亥貶潾江陵令 初羣臣議上尊號皇甫鎛欲增孝德字中書侍郎同平章事崔羣曰言聖則孝在其中矣鎛譛羣於上曰羣於陛下惜孝德二字上怒時鎛給邊軍賜與多不時得又所給多陳敗【陳舊也}
不可服用軍士怨怒流言欲為亂【流言放言也}
李光顔憂懼欲自殺【李光顔時帥邠寧}
遣人訴於上上不信京師忷懼羣具以中外人情上聞【上聞時掌翻}
鎛密言于上曰邊賜皆如舊制而人情忽如此者由羣鼓扇將以賣直歸怨於上也上以為然十二月乙卯以羣為湖南觀察使於是中外切齒於鎛矣【小人去君子以為自安之謀不知適所以自危也}
中書舍人武儒衡有氣節好直言【好呼到翻}
上器之顧待甚渥人皆言且入相令狐楚忌之思有以沮之者【沮在呂翻}
乃薦山南東道節度推官狄兼謩才行【行戶孟翻}
癸亥擢兼謩左拾遺内供奉【以資序尚淺未除正官令於左拾遺班内供奉猶監察御史裏行也}
兼謩仁傑之族曾孫也楚自草制辭盛言天后竊位姦臣擅權賴仁傑保佑中宗克復明辟【事見武后紀}
儒衡泣訴於上且言臣曾祖平一在天后朝辭榮終老【平一在武后時畏禍居嵩山修浮屠法累詔不起}
上由是薄楚之為人 十五年春正月沂海兖密觀察使曹華請徙理兖州【自沂州徙治兖州}
許之 義成節度使劉悟入朝 初左軍中尉吐突承璀謀立澧王惲為太子【惲於粉翻}
上不許及上寢疾承璀謀尚未息太子聞而憂之密遣人問計於司農卿郭釗釗曰殿下但盡孝謹以俟之勿恤其他釗太子之舅也【釗音昭}
上服金丹多躁怒左右宦官往往獲罪有死者人人自危庚子暴崩於中和殿【年四十三}
時人皆言内常侍陳弘志弑逆 【考異曰實録但云上崩於大明宫之中和殿舊紀曰時帝暴崩皆言内官陳弘志弑逆史氏諱而不書王守澄傳曰憲宗疾大漸内官陳弘慶等弑逆憲宗英武威德在人内官秘之不敢除討但云藥暴崩新傳曰守澄與内常侍陳弘志弑帝於中和殿裴廷裕東觀奏記云宣宗追恨光陵商臣之酷郭太后亦以此暴崩然兹事曖昧終不能測其虚實故但云暴崩}
其黨類諱之不敢討賊但云藥外人莫能明也中尉梁守謙與諸宦官馬進潭劉承偕韋元素王守澄等共立太子殺吐突承璀及澧王惲賜左右神策軍士錢人五十緡六軍威遠人三十緡【按新志左右龍武左右神武左右神策號六軍今神策軍賜錢既厚而復有六軍則明唐中世以後以左右羽林龍武神武為六軍也威遠别是一軍}
左右金吾人十五緡閏月丙午穆宗即位於太極殿東序是日召翰林學士段文昌等及兵部郎中薛放駕部員外郎丁公著對于思政殿【以嗣君即位于太極殿東序及下文輟西宫朝臨徵之中和殿思政殿疑皆在西内實録言憲宗崩于大明宫之中和殿則在東内}
放戎之弟【薛戎見二百三十五卷德宗貞元十六年}
公著蘇州人皆太子侍讀也上未聽政放公著常侍禁中參預機密上欲以為相二人固辭 丁未輟西宫朝臨【西宫即西内大行在殯臣子朝夕臨臨哭也朝如字音陟遥翻臨力浸翻}
集羣臣於月華門外【唐東西内皆有月華門西内則大極門内之東廂有日華門西廂有月華門東内則宣政殿東廊有日華門西廊有月華門}
貶皇甫鎛為崖州司戶市井皆相賀 上議命相令狐楚薦御史中丞蕭俛辛亥以俛及段文昌皆為中書侍郎同平章事楚俛與皇甫鎛皆同年進士上欲誅鎛【以其附吐突承璀欲立澧王也}
俛及宦官救之故得免壬子杖殺柳泌及僧大通自餘方士皆流嶺表貶左金吾將軍李道古循州司馬【以其薦柳泌且保護之也}
癸丑以薛放為工部侍郎丁公著為給事中 乙卯尊郭貴妃為皇太后 丁卯上與羣臣皆釋服從吉【用漢文帝遺制也}
二月丁丑上御丹鳳門樓赦天下事畢盛陳倡優雜戲於門内而觀之【倡音昌}
丁亥上幸左神策軍觀手搏雜戲庚寅監察御史楊虞卿上疏以為陛下宜延對羣臣周徧顧問惠以氣色使進忠若趨利【趨七喻翻}
論政若訴寃如此而不致升平者未之有也衡山人趙知微亦上疏諫上遊畋無節上雖不能用亦不罪也【吳分湘南縣置衡山縣唐初屬潭州神龍三年度屬衡州九域志在州東北二百三十里}
壬辰廢邕管命容管經略使陽旻兼領之 安南都護桂仲武至安南楊清拒境不納清用刑慘虐其黨離心仲武遣人說其酋豪【說式芮翻}
數月間降者相繼得兵七千餘人朝廷以仲武為逗遛甲午以桂管觀察使裴行立為安南都護乙未以太僕卿杜式方為桂管觀察使丙申貶仲武為安州刺史 丹王逾薨【逾代宗子}
吐蕃寇靈武 憲宗之末回鶻遣合達干來求昏尤切憲宗許之三月癸卯朔遣合達干歸國 上見夏州觀察判官柳公權書跡愛之辛酉以公權為右拾遺翰林侍書學士【使之侍書而已不使任代言之職}
上問公權卿書何能如是之善對曰用筆在心心正則筆正上默然改容知其以筆諫也公權公綽之弟也 辛未安南將士開城納桂仲武執楊清斬之裴行立至海門而卒【海門鎮在白州博白縣東南卒子恤翻}
復以仲武為安南都護 吐蕃寇鹽州 初膳部員外郎元稹為江陵士曹【憲宗元和五年元稹貶江陵士曹事見二百三十八卷}
與監軍崔潭峻善上在東宫聞宫人誦稹歌詩而善之及即位潭峻歸朝獻稹歌詩百餘篇上問稹安在對曰今為散郎【郎中謂之正郎員外郎謂之散郎散悉亶翻}
夏五月庚戌以稹為祠部郎中知制誥【唐制中書舍人六人一人知制誥開元初以他官掌詔敇誥命謂之兼知制誥}
朝論鄙之【朝直遥翻}
會同僚食瓜於閣下【中書省曰鳳閣又有紫微閣}
有青蠅集其上中書舍人武儒衡以扇揮之曰適從何來遽集於此【以蠅喻稹}
同僚皆失色儒衡意氣自若 庚申葬神聖章武孝皇帝于景陵【景陵在同州奉先縣西北二十里金熾山}
廟號憲宗【古者祖有功而宗有德商之中宗高宗是也西漢以文帝為太宗武帝為世宗宣帝為中宗猶彷彿古意東漢自明帝至桓帝廟號皆稱宗非古也唐十七宗今人所稱者三宗而已}
六月以湖南觀察使崔羣為吏部侍郎召對别殿上曰朕升儲副知卿為羽翼【事見二百三十八卷憲宗元和七年}
對曰先帝之意久屬聖明臣何力之有【崔羣之對詞氣和而正處送往事居之間當以為法}
太后居興慶宫每朔望上帥百官詣宫上壽【帥讀曰率宫上時掌翻}
上性侈所以奉養太后尤為華靡【淮西既平憲宗之政衰矣况穆宗欲有以加之耶}
秋七月乙巳以鄆曹濮節度為天平軍【鄆音運濮博木翻鄆州古須句國秦為薛郡漢為東平國隋置鄆州京師東北一千六百九十七里曹州漢濟隂國後魏置西兖州後周改曹州取古國名也京師東北一千四百五十三里濮州漢東郡鄄城縣地後魏置濮陽郡隋為濮州京師東北一千五百七十里}
門下侍郎同平章事令狐楚坐為山陵使部吏盜官物又不給工人傭直收其錢十五萬緡為羨餘獻之【羨式面翻}
怨訴盈路丁卯罷為宣歙池觀察使【以史氏所書令狐楚此事言之則罷相誠是也以宣宗之用令狐綯言之則罷楚為非矣觀史必有能辨其是非者宣州秦鄣郡地漢為丹陽郡順帝改為宣城郡隋為宣州京師東南三千五百五十一里歙州吳新都郡晉改新安郡隋為歙州京師東南三千六百六十七里池州漢石城縣地梁昭明太子以其水出魚美改名貴池唐置池州東至宣州三百五里歙書涉翻}
八月癸巳神策兵二千浚魚藻池【魚藻池在魚藻宫程大昌曰禁池中有山山中建魚藻宫王建宫詞云魚藻宫中鎖翠娥先皇幸處不曾過而今池底休鋪錦菱葉雞頭漸漸多先皇謂德宗也自東内苑玄化門入禁苑魚藻宫在其西}
戊戌以御史中丞崔植為中書侍郎同平章事 己亥再貶令狐楚衡州刺史 上甫過公除【遵漢制二十七日釋服謂之公除按此時以二十七日公除下所謂易月也}
即事遊畋聲色賜與無節九月欲以重陽大宴【九月九日謂之重陽九陽數也故云貞元五年詔以二月一日三月三日九月九日為三令節任文武百寮選勝地追賞為樂}
拾遺李珏帥其同僚上疏曰伏以元朔未改【珏古岳翻元朔未改謂未踰年也春秋書元年春王正月即位}
園陵尚新雖陛下就易月之期俯從人欲而禮經著三年之制猶服心喪【謂公除易服為天下也而三年之慕内切於心不可變也}
遵同軌之會始離京【左傳天子七月而葬同軌畢至離力智翻}
告遠夷之使未復命【唐制國有大喪遣使宣遺詔於四夷謂之告哀使}
遏密弛禁蓋為齊人【書舜典曰三載四海遏密八音孔安國注遏絶也密静也齊人猶言齊民為于偽翻}
合樂後庭事將未可上不聽 戊午加邠寧節度使李光顔武寧節度使李愬並同平章事 冬十月王承宗薨其下秘不喪子知感知信皆在朝【憲宗元和十三年王承宗以二子為質於朝事見上卷}
諸將欲取帥於屬内諸州【帥所類翻下同}
參謀崔燧以承宗祖母涼國夫人命告諭諸將及親兵【涼國夫人蓋王武俊之妻}
立承宗之弟觀察支使承元承元時年二十 【考異曰舊傳作年十八按承元太和七年卒年三十三則於今年二十矣今從實録}
將士拜之承元不受泣且拜諸將固請不已承元曰天子遣中使監軍有事當與之議及監軍至亦勸之承元曰諸公未忘先德不以承元年少【少詩照翻}
欲使之攝軍務承元請盡節以遵忠烈之志【王武俊封清河郡王諡忠烈}
諸公肯從之乎衆許諾承元乃視事於都將聽事【聽讀曰廳都將聽事都知兵馬使之聽事也}
令左右不得謂已為留後委事於參佐密表請朝廷除帥庚辰監軍奏承宗疾亟弟承元權知留後并以承元表聞 党項復引吐蕃寇涇州【復扶又翻}
連營五十里辛巳遣起居舍人柏耆詣鎮州宣慰【是年改恒州為鎮州避上名也}
壬午羣臣入閤【歐陽脩曰唐故事天子日御殿見羣臣曰常參朔望薦食諸陵寢有思慕}


【之心不能臨前殿則御便殿見羣臣曰入閤宣政前殿也謂之衙衙有仗紫宸便殿也謂之閤其不御前殿而御紫宸也乃自正衙喚仗由閤門而入百官俟朝於衙者因隨而入見故謂之入閤程大昌曰宣政之左有東上閤宣政之右有西上閤二閤在殿左右而入閤者由之而入也西内太極宫兩儀殿左右有東西閤門而兩廊下有日華月華門其曰閤者即内殿也非真有閤也又曰西内太極殿北有兩儀殿即常日視朝之所太極殿兩廡有東西二上閤則是兩閤皆有門可入已又可轉北而入兩儀按程大昌言西内二閤門後說較為明白而宣政殿入閤則東内也}
諫議大夫鄭覃崔郾等五人進言陛下宴樂過多【郾音偃樂音洛}
畋遊無度今胡寇壓境【謂吐蕃入寇也}
忽有急奏不知乘輿所在【乘繩證翻}
又晨夕與倡優狎暱【倡音昌暱尼質翻}
賜與過厚夫金帛皆百姓膏血非有功不可與雖内藏有餘【藏徂浪翻}
願陛下愛之萬一四方有事不復使有司重歛百姓【復扶又翻歛力贍翻}
時久無閤中論事者【入閤諫官論事太宗之制也}
上始甚訝之【訝驚疑也}
謂宰相曰此輩何人對曰諫官上乃使人慰勞之【勞力到翻}
曰當依卿言宰相皆賀然實不能用也 【考異曰舊崔郾傳曰上即位荒於禽酒坐朝常晚郾與同列鄭覃等延英切諫上甚嘉之畋遊稍簡杜牧郾行狀曰穆宗皇帝春秋富盛稍以畋游聲色為事公晨朝正殿揮同列進而言曰十一聖之功德四海之大萬國之衆之治之亂懸於陛下自山已東百城千里昨日得之今日失之西望戎壘距宗廟十舍百姓憔悴蓄積無有願陛下稍親政事天下幸甚誠至氣直天子為之動容斂袖慰而謝之按是時未失山東杜牧直取穆宗時事文飾以為郾諫辭耳新傳承而用之皆誤也今從實録舊傳}
覃珣瑜之子也【鄭珣瑜永貞間為相}
上嘗謂給事中丁公著曰聞外間人多宴樂【樂音洛}
此乃時和人安足用為慰公著對曰此非佳事恐漸勞聖慮上曰何故對曰自天寶以來公卿大夫競為遊宴沈酣晝夜優雜子女【沈持林翻樂記獶雜子女鄭注曰獶或為優孔穎達曰獶雜謂獮猴也言舞戲之時狀如獮猴間雜男子婦人無别也}
不愧左右如此不已則百職皆廢陛下能無獨憂勞乎願少加禁止乃天下之福也 【考異曰實録明年二月景子觀神策雜伎因云上嘗召公著問云云舊紀遂云其日上歡甚顧公著云云此誤也今因覃等諫荒事言之}
癸未涇州奏吐蕃進營距州三十里告急求救以右軍中尉梁守謙為左右神策京西北行營都監將兵四千人并八鎮全軍救之【左右神策軍分屯近畿凡八鎮長武興平好畤普閏郃陽良原定平奉天也宋白所記與此稍異}
賜將士裝錢二萬緡以郯王府長史邵同為太府少卿兼御史中丞充荅吐蕃請和好使【郯王經順宗子也將即亮翻緡彌巾翻郯音談長知兩翻少始照翻好呼到翻使疏吏翻}
初秘書少監田洎入吐蕃為弔祭使【按新書吐蕃傳帝即位遣田洎往告哀則以洎為告哀使非弔祭使}
吐蕃請與唐盟於長武城下洎恐吐蕃留之不得還唯阿而已【還音旋唯于癸翻老子曰唯之與阿相去幾何}
既而吐蕃為党項所引入寇因以為辭曰田洎許我將兵赴盟於是貶洎郴州司戶【党底朗翻洎其冀翻將即亮翻又音如字郴丑林翻}
成德軍始奏王承宗薨乙酉徙田弘正為成德節度使以王承元為義成節度使劉悟為昭義節度使李愬為魏博節度使【田弘正自魏博徙成德劉悟自義成徙昭義李愬初自武寧徙昭義尋改魏博}
又以左金吾將軍田布為河陽節度使 渭州刺史郝玭數出兵襲吐蕃營所殺甚衆【元和四年以原州之平涼縣置行渭州數所角翻}
李光顔邠寧兵救涇州 【考異曰舊傳光顔救涇州事在十四年今從實録}
邠寧兵以神策受賞厚皆愠曰人給五十緡而不識戰鬭者彼何人邪【謂上即位之賞也愠於問翻}
常額衣資不得而前冒白刃者此何人邪洶洶不可止光顔親為開陳大義以諭之【為于偽翻}
言與涕俱然後軍士感悦而行將至涇州吐蕃懼而退丙戌罷神策行營【罷梁守謙之軍也}
西川奏吐蕃寇雅州辛卯鹽州奏吐蕃營於烏白池【鹽州五原縣有烏白池唐時鹽州元管四池烏池白池瓦窑池細項池青白鹽池在鹽州北}
尋亦皆退 十一月癸卯遣諫議大夫鄭覃詣鎮州宣慰賜錢一百萬緡以賞將士王承元既請朝命諸將及鄰道争以故事勸之承元皆不聽及移鎮義成將士諠譁不受命承元與柏耆召諸將以詔旨諭之諸將號哭不從【號戶刀翻}
承元出家財以散之擇其有勞者擢之謂曰諸公以先代之故不欲承元去此意甚厚然使承元違天子之詔其罪大矣昔李師道之未敗也朝廷嘗赦其罪師道欲行諸將固留之其後殺師道者亦諸將也【事見上卷元和十三年十四年}
諸將勿使承元為師道則幸矣因涕泣不自勝且拜之【勝音升}
十將李寂等十餘人固留承元承元斬以徇軍中乃定丁未承元赴滑州 【考異曰舊承元傳曰承元與柏耆召諸將於館驛諭之斬李寂等軍中始定舊鄭覃傳曰王承元移授鄭滑鎮之三軍留承元不能赴鎮承元乞重臣宣諭乃以覃為宣諭使初鎮卒辭語不遜覃至宣詔諭以大義軍人釋然聽命按實録辛亥田弘正奏今月九日王承元領兵二千人赴滑州計覃於時猶未能到鎮州作傳者推以為覃功耳今從承元傳}
將吏或以鎮州器用財貨行承元悉命留之 上將幸華清宫戊午宰相率兩省供奉官詣延英門【兩省以中書門下言也兩省官自左右常侍以下至遺補起居郎舍人皆供奉官也延英門延英殿門}
三上表切諫且言如此臣輩當扈從【從才用翻下同}
求面對皆不聽諫官伏門下【門下謂延英門下}
至暮乃退己未未明上自複道出城幸華清宫【自複道至興慶宫因而出城不欲出皇城使百官知之而扈從也}
獨公主駙馬中尉神策六軍使帥禁兵千餘人扈從晡時還宫【帥讀曰率}
十二月己巳朔鹽州奏吐蕃千餘人圍烏白池 庚辰西川奏南詔二萬人入界請討吐蕃 癸未容管奏破黄少卿萬餘衆拔營柵三十六時少卿久未平國子祭酒韓愈上言臣去年貶嶺外【謂貶潮州也}
熟知黄家賊事其賊無城郭可居依山傍險【傍蒲浪翻}
自稱洞主尋常亦各營生急則屯聚相保比緣邕管經略使【比毗至翻}
多不得人德既不能綏懷威又不能臨制侵欺虜縳以致怨恨遂攻刼州縣侵暴平人或復私讐或貪小利或聚或散終亦不能為事【言不能為大事也}
近者征討本起裴行立陽旻【事見上十四年}
此兩人者本無遠慮深謀意在邀功求賞亦緣見賊未屯聚之時將謂單弱争獻謀計自用兵以來已經二年前後所奏殺獲計不下二萬餘人儻皆非虛賊已尋盡至今賊猶依舊足明欺罔朝廷邕容兩管經此凋弊殺傷疾疫十室九空如此不已臣恐嶺南一道未有寧息之時自南討已來賊徒亦甚傷損察其情理厭苦必深賊所處荒僻【處昌呂翻}
假如盡殺其人盡得其地在於國計不為有益若因改元大慶【謂即位踰年改元大赦天下}
赦其罪戾遣使宣諭必望風降伏仍為選擇有威信者為經略使【降戶江翻仍為于偽翻}
苟處置得宜【處昌呂翻}
自然永無侵叛之事上不能用

穆宗睿聖文惠孝皇帝上【諱恒憲宗第三子}


長慶元年春正月辛丑上祀圓丘赦天下改元河北諸道各令均定兩税【以河北諸鎮各奉圖請吏輸賦税故令均定之}
門下侍郎同平章事蕭俛介潔疾惡為相重惜官職少所引拔【俛音免少詩沼翻}
西川節度使王播大修貢奉且以賂結宦官求為相段文昌復左右之【復扶又翻左音佐右音佑}
詔徵播詣京師俛屢於延英力争言播纎邪物論沸騰不可以汚台司【汚烏故翻涴也}
上不聽俛遂辭位己未播至京師壬戌俛罷為右僕射俛固辭僕射二月癸酉改吏部尚書 盧龍節度使劉總既殺其父兄【事見二百三十八卷憲宗元和五年}
心常自疑數見父兄為祟【數所角翻祟雖遂翻}
常於府舍飯僧數百【飯扶晚翻}
使晝夜為佛事每視事退則處其中或處他室則驚悸不敢寐【處昌呂翻悸其季翻}
晩年恐懼尤甚亦見河南北皆從化己卯奏乞弃官為僧 【考異曰舊温造傳曰長慶元年奉使河朔稱旨遷殿中侍御史既而幽州劉總請以所部九州聽朝旨穆宗選可使者或薦造乃拜起居舍人充太原幽州鎮州宣諭使造初至范陽劉總具櫜鞬郊迎乃宣聖旨示以禍福總俯伏流汗若兵加於頸矣及造使還總遂移家入覲按實録長慶元年正月乙巳以造為太原鎮州等道宣慰使二月己卯劉總奏乞為僧計造奉使尚未還三月癸亥總已卒八月丁亥以殿中侍御史温造為起居舍人充鎮州四面諸軍宣慰使造前以京兆司録宣慰兩河衆推其材故有是命舊傳誤也}
仍乞賜錢百萬緡以賞將士 上面諭西川節度使王播令歸鎮播累表乞留京師會中書侍郎同平章事段文昌請退壬申以文昌同平章事充西川節度使以翰林學士杜元穎為戶部侍郎同平章事以播為刑部尚書充鹽鐵轉運使元穎淹之六世孫也【杜淹太宗朝為相}
回鶻保義可汗卒 三月癸丑以劉總兼侍中充天

平節度使以宣武節度使張弘靖為盧龍節度使 乙卯以權知京兆尹盧士玫為瀛莫觀察使【玫莫杯翻}
丁巳詔劉總兄弟子姪皆除官大將僚佐亦宜超擢百姓給復一年【復方目翻}
軍士賜錢一百萬緡 戊午立皇弟憬為鄜王悦為瓊王惸為沔王懌為婺王愔為茂王怡為光王協為淄王憺為衢王惋為澶王【憬居永翻惸渠營翻愔挹淫翻憺徒覽翻又徒濫翻惋烏貫翻澶時連翻}
皇子湛為景王涵為江王湊為漳王溶為安王瀍為穎王 劉總奏懇乞為僧且以其私第為佛寺詔賜總名大覺寺名報恩遣中使以紫僧服及天平節鉞侍中告身并賜之惟其所擇詔未至總已削髮為僧將士欲遮留之總殺其唱帥者十餘人【遮留者遮道而留行唱帥者作唱以率衆帥讀曰率}
夜以印節授留後張玘遁去【張玘與總同謀殺其父兄者也}
及明軍中始知之玘奏總不知所在 【考異曰新傳總以節付張臯臯玘之兄為涿州刺史總之妻父也按實録幽州留後張玘奏總以剃髮為僧不知所在然則不以節付臯也}
癸亥卒于定州之境【德宗貞元元年劉怦得幽州三世三十六年而滅}
翰林學士李德裕吉甫之子也以中書舍人李宗閔嘗對策譏切其父恨之【譏切事見二百三十七卷憲宗元和三年}
宗閔又與翰林學士元稹争進取有隙右補闕楊汝士與禮部侍郎錢徽掌貢舉西川節度使段文昌翰林學士李紳各以書屬所善進士於徽及牓出文昌紳所屬皆不預【屬之欲翻下屬書同牓者書取中進士姓名而揭示之}
及第者【取中進士謂之及第言其文學及等第也}
鄭朗覃之弟裴譔度之子蘇巢宗閔之壻楊殷士汝士之弟也文昌言於上曰今歲禮部殊不公【殊絶也}
所取進士皆子弟無藝【言皆公卿子弟無藝能也}
以關節得之【唐人謂相屬請為關節此語至今猶然}
上以問諸學士德裕稹紳皆曰誠如文昌言上乃命中書舍人王起等覆試【覆審也再引試取中進士以審其實才曰覆試}
夏四月丁丑詔黜朗等十人 【考異曰鄭覃傳曰朗長慶元年登進士甲科此蓋言其始者登科耳}
貶徽江州刺史宗閔劒州刺史汝士開江令【江州京師東南二千九百四十八里劒州京師南一千六百六十二里開江漢朐縣地梁置漢豐縣西魏改曰永寧縣隋改曰盛山唐代宗廣德元年改曰開江帶開州}
或勸徽奏文昌紳屬書上必悟徽曰苟無愧心得喪一致【喪息浪翻}
奈何奏人私書豈士君子所為邪取而焚之時人多之紳敬玄之曾孫【李敬玄高宗朝為相}
起播之弟也自是德裕宗閔各分朋黨更相傾軋垂四十年【更工衡翻}
丙戌冊回鶻嗣君為登囉羽録没密施句主毗伽崇德可汗【按通鑑例回鶻新可汗未嘗書嗣君唐會要曰冊回鶻可汗為君登里囉羽録密施勾主毗伽崇德可汗囉魯何翻}
五月丙申朔回鶻遣都督宰相等五百餘人來迎公主 壬子鹽鐵使王播奏約榷茶額每百錢加税五十右拾遺李珏等上疏以為榷茶近起貞元多事之際【見二百三十四卷德宗貞元九年}
今天下無虞所宜寛横歛之目【横戶孟翻歛力贍翻}
而更增之百姓何時當得息肩不從 丙辰建王恪薨【恪上之弟也}
癸亥以太和長公主嫁回鶻公主上之妹也吐蕃聞唐與回鶻婚六月辛未寇青塞堡【新書吐蕃傳作清塞堡}
鹽州刺史李文悦擊却之戊寅回鶻奏以萬騎出北庭萬騎出安西拒吐蕃以迎公主初劉總奏分所屬為三道以幽涿營為一道請除張

弘靖為節度使平薊媯檀為一道請除平盧節度使薛平為節度使瀛莫為一道請除權知京兆尹盧士玫為觀察使【釋名曰幽州在北幽昧之地故曰幽西南至涿州一百二十里營州以營室分為名幽涿接境營州治柳城道里絶遠劉總奏以為一道必有說平州西至薊州二百里薊州西北至檀州二百十七里檀州西至媯州二百五十里瀛州北至莫州一百一十里玫莫回翻}
弘靖先在河東以寛簡得衆【弘靖鎮河東見二百三十九卷憲宗元和十一年}
總與之鄰境【幽并二鎮接壤}
聞其風望以燕人桀驁日久【燕於賢翻}
故舉弘靖自代以安輯之平嵩之子【薛嵩從史思明為將代宗初來降}
知河朔風俗而盡誠於國故舉之士玫則總妻族之親也總又盡擇麾下伉健難制者都知兵馬使朱克融等送之京師【伉口浪翻無所卑屈曰伉}
乞加奬拔使燕人有慕羨朝廷禄位之志又獻征馬萬五千匹【征馬戰馬也}
然後削髮委去【委弃也}
克融滔之孫也【朱滔畔換於德宗之時}
是時上方酣宴不留意天下之務崔植杜元穎無遠略不知安危大體苟欲崇重弘靖惟割瀛莫二州以士玫領之自餘皆統於弘靖朱克融等久羈旅京師至假匄衣食日詣中書求官植元穎不之省【匄居大翻乞也省悉景翻察也}
及除弘靖幽州勒克融輩歸本軍驅使克融輩皆憤怨先是河北節度使皆親冒寒暑與士卒均勞逸【先悉薦翻}
及弘靖至雍容驕貴肩輿於萬衆之中燕人訝之【訝者見之而驚疑也燕於賢翻下同}
弘靖莊默自尊涉旬乃一出坐決事賓客將吏罕得聞其言情意不接政事多委之幕僚而所辟判官韋雍輩多年少輕薄之士嗜酒豪縱出入傳呼甚盛或夜歸燭火滿街皆燕人所不習也詔以錢百萬緡賜將士弘靖留其二十萬緡充軍府雜用雍輩復裁刻軍士糧賜【復扶又翻}
繩之以法數以反虜詬責吏卒【數所角翻詬許候翻又古候翻}
謂軍士曰今天下太平汝曹能挽兩石弓不若識一丁字由是軍中人人怨怒【撫柔荒獷宣流德化適其俗修其政者易為功駭之以其所未嘗見懼之以其所未嘗聞鮮不速禍}


資治通鑑卷二百四十一














































































































































