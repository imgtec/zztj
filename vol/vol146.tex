<!DOCTYPE html PUBLIC "-//W3C//DTD XHTML 1.0 Transitional//EN" "http://www.w3.org/TR/xhtml1/DTD/xhtml1-transitional.dtd">
<html xmlns="http://www.w3.org/1999/xhtml">
<head>
<meta http-equiv="Content-Type" content="text/html; charset=utf-8" />
<meta http-equiv="X-UA-Compatible" content="IE=Edge,chrome=1">
<title>資治通鑒_147-資治通鑑卷一百四十六_147-資治通鑑卷一百四十六</title>
<meta name="Keywords" content="資治通鑒_147-資治通鑑卷一百四十六_147-資治通鑑卷一百四十六">
<meta name="Description" content="資治通鑒_147-資治通鑑卷一百四十六_147-資治通鑑卷一百四十六">
<meta http-equiv="Cache-Control" content="no-transform" />
<meta http-equiv="Cache-Control" content="no-siteapp" />
<link href="/img/style.css" rel="stylesheet" type="text/css" />
<script src="/img/m.js?2020"></script> 
</head>
<body>
 <div class="ClassNavi">
<a  href="/24shi/">二十四史</a> | <a href="/SiKuQuanShu/">四库全书</a> | <a href="http://www.guoxuedashi.com/gjtsjc/"><font  color="#FF0000">古今图书集成</font></a> | <a href="/renwu/">历史人物</a> | <a href="/ShuoWenJieZi/"><font  color="#FF0000">说文解字</a></font> | <a href="/chengyu/">成语词典</a> | <a  target="_blank"  href="http://www.guoxuedashi.com/jgwhj/"><font  color="#FF0000">甲骨文合集</font></a> | <a href="/yzjwjc/"><font  color="#FF0000">殷周金文集成</font></a> | <a href="/xiangxingzi/"><font color="#0000FF">象形字典</font></a> | <a href="/13jing/"><font  color="#FF0000">十三经索引</font></a> | <a href="/zixing/"><font  color="#FF0000">字体转换器</font></a> | <a href="/zidian/xz/"><font color="#0000FF">篆书识别</font></a> | <a href="/jinfanyi/">近义反义词</a> | <a href="/duilian/">对联大全</a> | <a href="/jiapu/"><font  color="#0000FF">家谱族谱查询</font></a> | <a href="http://www.guoxuemi.com/hafo/" target="_blank" ><font color="#FF0000">哈佛古籍</font></a> 
</div>

 <!-- 头部导航开始 -->
<div class="w1180 head clearfix">
  <div class="head_logo l"><a title="国学大师官网" href="http://www.guoxuedashi.com" target="_blank"></a></div>
  <div class="head_sr l">
  <div id="head1">
  
  <a href="http://www.guoxuedashi.com/zidian/bujian/" target="_blank" ><img src="http://www.guoxuedashi.com/img/top1.gif" width="88" height="60" border="0" title="部件查字,支持20万汉字"></a>


<a href="http://www.guoxuedashi.com/help/yingpan.php" target="_blank"><img src="http://www.guoxuedashi.com/img/top230.gif" width="600" height="62" border="0" ></a>


  </div>
  <div id="head3"><a href="javascript:" onClick="javascript:window.external.AddFavorite(window.location.href,document.title);">添加收藏</a>
  <br><a href="/help/setie.php">搜索引擎</a>
  <br><a href="/help/zanzhu.php">赞助本站</a></div>
  <div id="head2">
 <a href="http://www.guoxuemi.com/" target="_blank"><img src="http://www.guoxuedashi.com/img/guoxuemi.gif" width="95" height="62" border="0" style="margin-left:2px;" title="国学迷"></a>
  

  </div>
</div>
  <div class="clear"></div>
  <div class="head_nav">
  <p><a href="/">首页</a> | <a href="/ShuKu/">国学书库</a> | <a href="/guji/">影印古籍</a> | <a href="/shici/">诗词宝典</a> | <a   href="/SiKuQuanShu/gxjx.php">精选</a> <b>|</b> <a href="/zidian/">汉语字典</a> | <a href="/hydcd/">汉语词典</a> | <a href="http://www.guoxuedashi.com/zidian/bujian/"><font  color="#CC0066">部件查字</font></a> | <a href="http://www.sfds.cn/"><font  color="#CC0066">书法大师</font></a> | <a href="/jgwhj/">甲骨文</a> <b>|</b> <a href="/b/4/"><font  color="#CC0066">解密</font></a> | <a href="/renwu/">历史人物</a> | <a href="/diangu/">历史典故</a> | <a href="/xingshi/">姓氏</a> | <a href="/minzu/">民族</a> <b>|</b> <a href="/mz/"><font  color="#CC0066">世界名著</font></a> | <a href="/download/">软件下载</a>
</p>
<p><a href="/b/"><font  color="#CC0066">历史</font></a> | <a href="http://skqs.guoxuedashi.com/" target="_blank">四库全书</a> |  <a href="http://www.guoxuedashi.com/search/" target="_blank"><font  color="#CC0066">全文检索</font></a> | <a href="http://www.guoxuedashi.com/shumu/">古籍书目</a> | <a   href="/24shi/">正史</a> <b>|</b> <a href="/chengyu/">成语词典</a> | <a href="/kangxi/" title="康熙字典">康熙字典</a> | <a href="/ShuoWenJieZi/">说文解字</a> | <a href="/zixing/yanbian/">字形演变</a> | <a href="/yzjwjc/">金 文</a> <b>|</b>  <a href="/shijian/nian-hao/">年号</a> | <a href="/diming/">历史地名</a> | <a href="/shijian/">历史事件</a> | <a href="/guanzhi/">官职</a> | <a href="/lishi/">知识</a> <b>|</b> <a href="/zhongyi/">中医中药</a> | <a href="http://www.guoxuedashi.com/forum/">留言反馈</a>
</p>
  </div>
</div>
<!-- 头部导航END --> 
<!-- 内容区开始 --> 
<div class="w1180 clearfix">
  <div class="info l">
   
<div class="clearfix" style="background:#f5faff;">
<script src='http://www.guoxuedashi.com/img/headersou.js'></script>

</div>
  <div class="info_tree"><a href="http://www.guoxuedashi.com">首页</a> > <a href="/SiKuQuanShu/fanti/">四库全书</a>
 > <h1>资治通鉴</h1> <!--         下载:【右键另存为】即可 --></div>
  <div class="info_content zj clearfix">
  
<div class="info_txt clearfix" id="show">
<center style="font-size:24px;">147-資治通鑑卷一百四十六</center>
    資治通鑑卷一百四十六 宋 司馬光 撰<br />
<br />
  胡三省 音註<br />
<br />
  梁紀二【起旃蒙作噩盡彊圉大淵獻凡三年】<br />
<br />
  高祖武皇帝二<br />
<br />
  天監四年春正月癸卯朔詔曰二漢登賢莫非經術服膺雅道名立行成【行下孟翻朱元晦曰服著也膺胷也奉持而著之心胷之間著則略翻】魏晉浮蕩儒教淪歇風節罔樹【樹立也歇許竭翻】抑此之由可置五經博士各一人廣開館宇招内後進於是以賀瑒及平原明山賓吳興沈峻建平嚴植之補博士各主一館館有數百生給其餼廩【瑒徒杏翻又音暢餼許既翻鄭玄曰餼廩稍食也稍所教翻】其射策通明者即除為吏【漢書音義曰作簡策難問列置案上在試者意投射取而答之謂之射策】朞年之間懷經負笈者雲會瑒循之玄孫也【笈其劫翻又楚洽翻書箱也晉氏南渡之初以賀循為儒宗】又選學生往會稽雲門山從何胤受業【胤時隱雲門山今在會稽南三十一里有雲門寺會工外翻】命胤選門徒中經明行修者【行下孟翻】具以名聞分遣博士祭酒巡州郡立學 初譙國夏侯道遷以輔國將軍從裴叔業鎮夀陽為南譙太守【案魏收地形志晉孝武置南譙郡蓋治渦陽又案蕭子顯齊志武帝永明二年割揚州宣城淮南南豫譙盧江臨江六郡置南豫州四年冠軍長史沈憲啟二豫分置以桑堁子亭為斷潁州汝陽在南譙歷陽界悉屬西豫廬江居晉熙汝隂之中屬南豫求以潁川汝陽屬南豫廬江屬西豫則齊之南譙蓋置於歷陽西界而渦陽已入於魏矣南北建置郡縣最為難考者率如此夏戶雅翻守式又翻】與叔業有隙單騎犇魏魏以道遷為驍騎將軍【騎奇寄翻驍堅堯翻】從王肅鎮夀陽使道遷守合肥肅卒【卒子恤翻下同】道遷棄戍來犇從梁秦二州刺史莊丘黑鎮南鄭以道遷為長史領漢中太守黑卒詔以都官尚書王珍國為刺史未至道遷隂與軍主考城江忱之等謀降魏【降戶江翻】先是魏仇池鎮將楊靈珍叛魏來犇【事見一百四十一卷齊明帝建武四年先悉薦翻將即亮翻】朝廷以為征虜將軍假武都王助戍漢中有部曲六百人道遷憚之上遣左右吳公之等使南鄭道遷遂殺使者兵擊靈珍父子斬之并使者首送於魏【使疏吏翻】白馬戍主尹天寶聞之引兵擊道遷敗其將龎樹【敗補邁翻】遂圍南鄭道遷求救於氐王楊紹先楊集起楊集義皆不應集義弟集朗引兵救道遷擊天寶殺之魏以道遷為平南將軍豫州刺史豐縣侯 【考異曰梁帝紀天監三年二月魏陷梁州而列傳皆無其事魏帝紀正始元年閏十二月癸卯朔蕭衍行梁州事夏侯道遷據漢中來降道遷傳具言其事按長歷梁閏二月癸卯即天監四年正月朔也故置於此】又以尚書邢巒為鎮西將軍都督征梁漢諸軍事將兵赴之道遷受平南辭豫州【辭豫州者欲得梁州也】且求公爵魏主不許 辛亥上祀南郊大赦 乙丑魏以驃騎大將軍高陽王雍為司空【驃匹妙翻騎奇寄翻】加尚書令廣陽王嘉儀同三司 二月丙子魏以宕昌世子梁彌博為宕昌王【宕徒浪翻】 上謀伐魏壬午遣衛尉卿楊公則將宿衛兵塞洛口【自漢以來衛尉與太常太僕廷尉大鴻臚宗正大司農少府為九卿而職名未帶卿字至梁分十二寺始各帶卿字水經注洛澗北逕秦虚下注淮謂之洛口塞悉則翻】 壬辰交州刺史李凱據州反長史李畟討平之【畟初力翻】 魏邢巒至漢中擊諸城戍所向摧破晉夀太守王景胤據石亭【水經注漢水自武興城北西南流逕闕城北又西逕石亭戍又逕晉夀城西】巒遣統軍李義珍擊走之魏以巒為梁秦二州刺史巴西太守龎景民據郡不下【龎皮江翻】郡民嚴玄思聚衆自稱巴州刺史附於魏攻景民斬之楊集起集義聞魏克漢中而懼閏月帥羣氐叛魏斷漢中糧道【帥讀曰率斷丁管翻】巒屢遣軍擊破之 夏四月丁巳以行宕昌王梁彌博為河凉二州刺史宕昌王 冠軍將軍孔陵等將兵二萬戍深杭【冠古玩翻將即亮翻 考異曰梁鄧元起傳魏將王景胤孔陵寇東西晉夀並遣告急按魏邢巒傳曰蕭衍晉夀太守王景胤據石亭又曰蕭衍遣其將軍孔陵等據深杭然則景胤陵皆梁將也元起傳誤】魯方達戍南安【五代志始州普安縣舊曰南安始州唐之劍州】任僧褒等戍石同以拒魏【任音壬】邢巒遣統軍王足將兵擊之所至皆捷遂入劍閣陵等退保梓潼足又進擊破之梁州十四郡地東西七百里南北千里皆入于魏【蕭子顯齊志梁州注籍者二十二郡荒郡不預焉今魏取十四郡】初益州刺史鄧元起以母老乞歸詔徵為右衛將軍以西昌侯淵藻代之淵藻懿之子也【懿死于東昏之手】夏侯道遷之叛也尹天寶馳使報元起【使疏吏翻】及魏寇晉夀王景胤等並遣告急衆勸元起急救之元起曰朝廷萬里軍不猝至若寇賊侵淫【侵淫以癰疽為喻侵毒好肉為淫肉】方須撲討董督之任非我而誰何事怱怱救之【史言鄧元起乞歸非由衷之請撲普木翻】詔假元起都督征討諸軍事救漢中而晉夀已陷蕭淵藻將至元起營還裝糧儲器械取之無遺淵藻入城恨之又求其良馬元起曰年少郎子何用馬為淵藻恚因醉殺之【元起養寇自資而卒不免於死雖淵藻以私忿殺之亦不為無罪也少詩照翻恚於避翻】元起麾下圍城哭且問故淵藻曰天子有詔衆乃散遂誣以反上疑焉元起故吏廣漢羅研詣闕訟之上曰果如我所量也使讓淵藻曰元起為汝報讐【謂協力誅東昏報其父讐也量音良為于偽翻下同】汝為讐報讐忠孝之道如何乃貶淵藻號為冠軍將軍【冠古玩翻 考異曰梁書元起傳藻以糧儲無遺甚怨望之因表元起逗留不憂軍事收付州獄自縊死按若止以逗留表元起安敢擅收前刺史付獄殺之必誣以反也今從南史梁書藻本以冠軍為益州刺史與南史異】贈元起征西將軍諡曰忠侯李延夀論曰元起勤乃胥附【毛萇曰幸下親上曰胥附】功惟闢土【謂開梁益之土也】勞之不圖禍機先陷冠軍之貶於罰已輕梁之政刑於斯為失私戚之端自斯而啟年之不永不亦宜乎<br />
<br />
  益州民焦僧護聚衆作亂蕭淵藻年未弱冠【人生二十曰弱冠冠古玩翻】集僚佐議自擊之或陳不可淵藻大怒斬于階側乃乘平肩輿巡行賊壘【平肩輿使人就掆肩之故曰平肩行下孟翻】賊弓亂射矢下如雨從者舉楯禦矢淵藻命去之【射而亦翻從才用翻去羌呂翻】由是人心大安擊僧護等皆平之 六月庚戌初立孔子廟 豫州刺史王超宗【以五代志攷之此時梁置豫州於晉熙今安慶府懷寧縣地】將兵圍魏小峴【峴戶典翻】丁卯魏揚州刺史薛真度遣兼統軍李叔仁等擊之超宗兵大敗 冠軍將軍王景胤李畎輔國將軍魯方達等與魏王足戰屢敗秋七月足進逼涪城【畎姑泫翻涪音浮】 八月壬寅魏中山王英寇雍州【雍於用翻】 庚戌秦梁二州刺史魯方達與魏王足統軍紀洪雅盧祖遷戰敗方達等十五將皆死壬子王景胤等又與祖遷戰敗景胤等二十四將皆死 楊公則至洛口與魏豫州長史石榮戰斬之甲寅將軍姜慶真與魏戰於羊石不利【羊石蓋即陳伯之所屯之陽石也】公則退屯馬頭雍州蠻沔東太守田青喜叛降魏【攷之北史青喜所據之地蓋在襄陽之東竟陵之西沔彌兖翻】 魏有芝生於太極殿之西序【殿廡曰序】魏主以示侍中崔光光上表以為此莊子所謂氣蒸成菌者也【菌巨隕翻地簟也】柔脆之物生於墟落穢濕之地不當生於殿堂高華之處今忽有之厥狀扶疎誠足異也夫野木生朝野鳥入廟古人皆以為敗亡之象故太戊中宗懼災修德殷道以昌【商王太戊之時亳有祥桑榖共生于朝一暮大拱太戊懼而修德祥桑枯死殷道復興高宗祭成湯有飛雉升鼎耳而雊祖巳曰惟先格王正厥事朝諸侯有天下猶運之於掌中宗當作高宗朝直遥翻】所謂家利而怪先國興而妖豫者也【妖於遥翻】今西南二方兵革未息郊甸之内大旱踰時民勞物悴莫此之甚【悴秦醉翻】承天育民者所宜矜恤伏願陛下側躬聳意惟新聖道節夜飲之樂養方富之年則魏祚可以永隆皇夀等於山岳矣於是魏主好宴樂【樂音洛好呼到翻】故光言及之 九月己巳楊公則等與魏揚州刺史元嵩戰公則敗績 冬十月丙午上大舉伐魏以揚州刺史臨川王宏都督北討諸軍事尚書右僕射柳惔為副【惔徒甘翻】王公以下各上國租及田穀以助軍【國租者封國所入之租田穀者職田所入之穀各上時掌翻】宏軍于洛口 楊集起集義立楊紹先為帝自皆稱王十一月戊辰朔魏遣光禄大夫楊椿將兵討之【將即亮翻】 魏王足圍涪城蜀人震恐益州城戍降魏者什二三民自上名籍者五萬餘戶【上時掌翻下西上同】邢巒表於魏主請乘勝進取蜀以為建康成都相去萬里陸行既絶【自襄陽西行遵陸可以至蜀梁州既入于魏則陸路斷矣】惟資水路水軍西上非周年不達益州外無軍援一可圖也頃經劉季連反鄧元起攻圍【事見上卷元年二年】資儲空竭吏民無復固守之志二可圖也蕭淵藻裠屐少年【復扶又翻裠渠云翻下裳也屐竭戟翻蹻也少詩沼翻】未洽治務宿昔名將多見囚戮【治直吏翻將即亮翻下同】今之所任皆左右少年三可圖也蜀之所恃唯在劍閣今既克南安已奪其險據彼竟内【竟讀曰境】三分已一自南安向涪方軌無礙前軍累敗後衆喪魄四可圖也【喪息浪翻】淵藻是蕭衍骨肉至親必無死理若克涪城淵藻安肯城中坐而受困必將望風逃去若其出鬭庸蜀士卒駑怯弓矢寡弱五可圖也【武王之伐紂也庸蜀八國皆從庸上庸之地蜀蜀郡之地】臣内省文吏不習軍旅賴將士竭力頻有薄捷既克重阻【省悉景翻重直龍翻重阻猶言重險也】民心懷服瞻望涪益【時梓潼太守治涪城益州刺史治成都】旦夕可圖止以兵少糧匱未宜前出【少詩沼翻】今若不取後圖便難况益州殷實戶口十萬比夀春義陽其利三倍【魏先此已得夀春義陽故云然】朝廷若欲進取時不可失若欲保境寧民則臣居此無事乞歸侍養【養余亮翻】魏主詔以平蜀之舉當更聽後勑寇難未夷何得以養親為辭巒又表稱昔鄧艾鍾會帥十八萬衆傾中國資儲僅能平蜀【事見七十八卷魏元帝景元四年難乃旦翻帥讀曰率】所以然者鬭實力也况臣才非古人何宜以二萬之衆而希平蜀所以敢者正以據得要險士民慕義此往則易【易以豉翻下未易同】彼來則難任力而行理有可克今王足已逼涪城脱得涪則益州乃成擒之物但得之有早晩耳且梓潼已附民戶數萬【謂已上名籍之民也】朝廷豈可不守又劍閣天險得而棄之良可惜矣【諸葛孔明相蜀以大劍小劍有隘東之路故曰劍門以閣道三十里至險乃有闍尉姜維拒鍾會於此晉以其地入梓潼郡桓温入蜀於晉夀置劍閣縣屬梁州】臣誠知戰伐危事未易可為自軍度劍閣以來鬢髮中白【中竹仲翻】日夜戰懼何可為心所以勉強者【強其兩翻】既得此地而自退不守恐負陛下之爵禄故也且臣之意算正欲先取涪城以漸而進若得涪城則中分益州之地斷水陸之衝【魏已得劍閣進取成都涪當其衝梁兵由内水而上救成都涪亦當其衝斷丁管翻】彼外無援軍孤城自守何能復持久哉【復扶又翻】臣今欲使軍軍相次聲勢連接先為萬全之計然後圖功得之則大利不得則自全又巴西南鄭相距千四百里去州迢遰【迢田聊翻遰徒計翻迢遰遠也】恒多擾動【恒戶登翻】昔在南之日以其統綰勢難曾立巴州鎮静夷獠【立巴州見一百三十五卷齊高帝建元二年省巴州見武帝永明二年獠魯皓翻下同】梁州藉利因而表罷彼土民望嚴蒲何楊非唯一族雖率居山谷而豪右甚多文學風流亦為不少但以去州既遠不獲仕進至於州綱無由厠迹【州之上佐是謂州綱少詩沼翻】是以鬱怏多生異圖【怏於兩翻】比道遷建義之始【比毗至翻】嚴玄思自號巴州刺史克城以來仍使行事巴西廣袤千里戶餘四萬若於彼立州鎮攝華獠【巴西之地華人與獠雜居故云華獠袤音茂】則大帖民情【帖静也安也伏也】從墊江已還不勞征伐自為國有【李雄譙縱取蜀東不能過墊江以苻秦兵力之盛取梁益如反掌墊江以東苻秦不能有也邢巒之圖蜀亦規墊江以西而已蓋地利足恃也我朝自紹定失蜀彭大雅遂城渝為制府支持西蜀且四十年渝古墊江之地也墊音疊】魏主不從先是魏主以王足行益州刺史【先悉薦翻】上遣天門太守張齊將兵救益州未至【將即亮翻】魏主更以梁州軍司泰山羊祉為益州刺史【更工衡翻】王足聞之不悦輒引兵還【還從宣翻又如字】遂不能定蜀久之足自魏來犇邢巒在梁州接豪右以禮撫小民以惠州人悦之巒之克巴西也使軍主李仲遷守之仲遷溺於酒色費散兵儲公事諮承無能見者巒忿之切齒仲遷懼謀叛城人斬其首以城來降【史言魏所以不能定蜀降戶江翻】 十二月庚申魏遣驃騎大將軍源懷討武興氐邢巒等並受節度【驃匹妙翻騎奇寄翻】 司徒尚書令謝朏以母憂去職【朏敷尾翻】 是歲大穰【穰豐也詩豐年穰穰】米斛三十錢<br />
<br />
  五年春正月丁卯朔魏于后生子昌大赦 楊集義圍魏關城【此即陽平關城也】邢巒遣建武將軍傅豎眼討之【豎而庾翻】集義逆戰豎眼擊破之乘勝逐北壬申克武興執楊紹先送洛陽楊集起楊集義亡走遂滅其國【晉惠帝元康六年氐王楊茂搜始據仇池百頃其後浸盛盡有漢武都郡之地北侵隴西天水南侵漢中拓跋既盛取武都仇池之地楊氏僅據武興今魏既取漢中遂滅楊氏】以為武興鎮又改為東益州【東益州領武興仇池盤頭廣長廣業梓潼洛叢郡】 乙亥以前司徒謝朏為中書監司徒【朏敷尾翻】 冀州刺史桓和擊魏南青州不克【梁青冀二州治鬱洲魏顯祖取三齊置東徐州於圂城領東安東莞郡高祖太和二十二年改為南青州五代志沂州沂水縣舊置南青州】 魏秦州屠各王法智聚衆二千【屠直於翻】推秦州主簿呂苟兒為主改元建明置百官攻逼州郡涇州民陳瞻亦聚衆稱王改元聖明【魏置涇州治臨涇城領安定隴東新平趙平平凉平源等郡】 己卯楊集起兄弟相帥降魏【帥讀曰率降戶江翻】 甲申封皇子綱為晉安王 二月丙辰魏主詔王公以下直言忠諫治書侍御史陽固上表【治直之翻上時掌翻】以為當今之務宜親宗室勤庶政貴農桑賤工賈【賈音古】絶談虛窮微之論簡桑門無用之費以救饑寒之苦時魏主委任高肇疎薄宗室好桑門之法【好呼到翻】不親政事故固言及之戊午魏遣右衛將軍元麗都督諸軍討呂苟兒麗小<br />
<br />
  新成之子也【小新成見一百二十九卷宋孝武大明五年】 乙丑徐州刺史歷陽昌義之與魏平南將軍陳伯之戰於梁城【晉孝武太元中僑立梁郡於淮南夀春界故有梁城其地在夀陽東北鍾離西南】義之敗績 將軍蕭昞將兵擊魏徐州圍淮陽【角城在淮水之陽淮陽又在角城北十八里治宿預梁後於角城置淮陽郡昞音丙】 三月丙寅朔日有食之 己卯魏荆州刺史趙怡平南將軍奚康生救淮陽 魏咸陽王禧之子翼遇赦求葬其父【禧誅見一百四十四卷齊和帝中興元年】屢泣請於魏主魏主不許癸未翼與其弟昌曄來犇上以翼為咸陽王翼以曄嫡母李妃之子也請以爵讓之上不許 輔國將軍劉思效敗魏青州刺史元繫於膠水【魏收志光州長廣郡即墨縣有膠水水經膠水出黔陬縣膠山北流過夷安縣東又東北過膠東縣城北百里注于海敗補邁翻】臨川王宏使記室吳興丘遲為書遺陳伯之曰【遺于季翻】尋君去就之際非有他故直以不能内審諸已外受流言沈迷猖獗以至於此【沈持林翻】主上屈法申恩吞舟是漏【漢懲秦法之苛禁罔疏濶時稱為漏吞舟之魚】將軍松柏不翦親戚安居高臺未傾愛妾尚在【松柏不翦謂不毁夷其先世墳墓也親戚安居謂其親戚在江南者皆不以叛黨連坐安居自若也高臺未傾謂居第未嘗汙瀦池臺如故也愛妾尚在謂其婢妾猶守其家不沒於官及流落於他家也昔雍門子見孟嘗君吟曰高臺既已傾曲池既已平墳墓生荆棘牧豎游其上孟嘗君亦如是乎孟嘗君為之喟然嘆息】而將軍魚游於沸鼎之中鷰巢於飛幕之上【魚游釡中古人多有是言言將必至於焦爛左傳吳季札謂孫林父曰夫子之居此也猶燕之巢於幕上杜預註曰言至危也】不亦惑乎想早勵良圖自求多福庚寅伯之自夀陽梁城擁衆八千來降【伯之元年奔魏今復還降戶江翻】魏人殺其子虎牙詔復以伯之為西豫州刺史未之任復以為通直散騎常侍【不使之出當邊鎮恐其復叛也復扶又翻】久之卒於家 初魏御史中尉甄琛【甄七人翻琛丑林翻】表稱周禮山林川澤有虞衡之官為之厲禁蓋取之以時不使戕賊而已故雖置有司實為民守之也【周禮山虞掌山林之政令物為之厲而為之守禁令萬民時斬材有期日凡竊木者有刑罰林衡掌林麓之禁令而平其守以時計林麓而賞罰之川衡掌巡川澤之禁令而平其守以時舍其守犯禁者執而誅罰之澤虞掌國澤之政令為之厲禁使其地之人守其財物以時入于王府頒其餘于萬民為于偽翻下專為同】夫一家之長必惠養子孫【長知兩翻】天下之君必惠養兆民未有為人父母而吝其醯醢富有羣主而榷其一物者也【榷古岳翻】今縣官鄣護河東鹽池而收其利是專奉口腹而不及四體也蓋天子富有四海何患於貧乞弛鹽禁與民共之録尚書事勰尚書邢巒奏【勰彭城王勰也音協】以為琛之所陳坐談則理高行之則事闕竊惟古之善治民者必汚隆随時豐儉稱事【治直之翻稱尺證翻】役養消息以成其性命若任其自生随其飲啄乃是芻狗萬物【老子曰天地不仁以萬物為芻狗註云天施地化不以仁恩天地生萬物視之如芻草狗畜任自然也】何以君為是故聖人斂山澤之貨以寛田疇之賦收關市之税以助什一之儲【此謂田疇什一之賦不足以供國用故斂山澤税關市以助之也】取此與彼皆非為身【為于偽翻下同】所謂資天地之產惠天地之民也今鹽池之禁為日已久積而散之以濟軍國非專為供大官之膳羞給後宫之服玩既利不在己則彼我一也然自禁鹽以來有司多慢出納之間或不如法是使細民嗟怨負販輕議此乃用之者無方非作之者有失也一旦罷之恐乖本旨一行一改法若奕棊【左傳曰奕者舉棊不定不勝其耦今此以喻一行一改無定法也】參論理要宜如舊式【自此以上合載於一百四十三卷齊東昏永明二年】魏主卒從琛議【琛議既行於景明初年随格於景明四年今復罷鹽禁是卒從其議也卒子恤翻】夏四月乙未罷鹽池禁【復收鹽利見上卷二年】 庚戌魏以中山王英為征南將軍都督揚徐二州諸軍事帥衆十餘萬以拒梁軍【帥讀曰率】指授諸節度所至以便宜從事江州刺史王茂將兵數萬侵魏荆州誘魏邊民及諸蠻更立宛州【將即亮翻誘音酉更魏荆州為宛州也更工衡翻宛於元翻】遣其所署宛州刺史雷豹狼等襲取魏河南城【蕭子顯齊志雍州有河南郡所領五縣惟棘陽為實土則河南郡當在南陽棘陽縣界五代志鄧州新野縣舊曰棘陽】魏遣平南將軍楊大眼都督諸軍擊茂辛酉茂戰敗失亡二千餘人 【考異曰大眼傳云俘馘七千有餘今從魏帝紀】大眼進攻河南城茂逃還大眼追至漢水攻拔五城魏征虜將軍宇文福寇司州俘千餘口而去五月辛未太子右衛率張惠紹等侵魏徐州拔宿預執城主馬成龍【晉安帝立宿預縣屬淮陽郡魏高祖以為南徐州治所】乙亥北徐州刺史昌義之拔梁城【南徐治京口故以鍾離為北徐】豫州刺史韋叡遣長史王超等攻小峴未拔叡行圍柵【行下孟翻】魏出數百人陳於門外【陳讀曰陣】叡欲擊之諸將皆曰【將即亮翻下同】向者輕來未有戰備徐還授甲乃可進耳叡曰不然魏城中二千餘人足以固守今無故出人於外必其驍勇者也【驍堅堯翻】苟能挫之其城自拔衆猶遲疑叡指其節曰朝廷授此非以為飾韋叡法不可犯也遂進擊之士皆殊死戰魏兵敗走因急攻之中宿而拔【中讀曰仲又竹使翻 考異曰魏帝紀六月辛丑陷小峴戍今從叡傳】遂至合肥先是右軍司馬胡景略等攻合肥久未下【先悉薦翻】叡按山川夜帥衆堰肥水頃之堰成水通舟艦繼至【帥讀曰率艦戶黯翻】魏築東西小城夾合肥叡先攻二城魏將楊靈胤帥衆五萬奄至衆懼不敵請奏益兵叡笑曰賊至城下方求益兵將何所及且吾求益兵彼亦益兵兵貴用奇豈在衆也遂擊靈胤破之叡使軍主王懷静築城於岸以守堰魏攻拔之城中千餘人皆沒魏人乘勝至堤下兵勢甚盛諸將欲退還漅湖或欲保三义 【考異曰南史作三丈今從梁書蓋漅湖之水於此分三汊故名退保於此利於入船故衆欲之】叡怒曰寧有此邪命取繖扇麾幢樹之堤下示無動志【繖蘇旱翻又蘇盱翻幢傳江翻】魏人來鑿堤叡親與之争魏兵卻因築壘於堤以自固叡起鬬艦高與合肥城等四面臨之城中人皆哭守將杜元倫登城督戰中弩死【將即亮翻中竹仲翻】辛巳城潰俘斬萬餘級獲牛羊以萬數叡體素羸未嘗跨馬【羸倫為翻】每戰常乘板輿督厲將士勇氣無敵晝接賓旅夜半起筭軍書張燈逹曙撫循其衆常如不及故投募之士争歸之所至頓舍館宇藩牆皆應準繩諸軍進至東陵【水經注廬江金蘭縣西北東陵鄉大蘇山灌水之所出也攷之諸志無金蘭縣未知何世所置】有詔班師【班師之詔必在洛口師潰之後史因書叡事而終言之】去魏城既近【據姚思廉梁書時魏守甓城去東陵二十里】諸將恐其追躡叡悉遣輜重居前身乘小輿殿後【重直用翻殿丁練翻】魏人服叡威名望之不敢逼全軍而還於是遷豫州治合肥【豫州自晉熙還合肥】壬午魏遣尚書元遥南拒梁兵 癸未魏遣征西將軍于勁節度秦隴諸軍 丁亥廬江太守聞喜裴邃克魏羊石城庚寅又克霍丘城【水經注曹魏安豐都尉治安豐津南後以其故城立霍丘戍隋立霍丘縣今在夀春東百餘里杜佑曰霍丘漢松滋縣地 考異曰梁裴邃傳云五年征邵陽洲魏人為長橋以濟邃築壘逼橋密作沒突艦會淮水暴漲邃乘艦徑造橋側魏衆驚潰邃乘勝追擊大破之進克羊石霍丘城平小峴攻合肥魏帝紀辛巳衍將陷合肥己丑又陷羊石霍丘案韋叡傳叡攻邵陽洲方使邃乘艦焚橋事在克合肥後又梁帝紀辛巳叡克合肥丁亥邃克羊石庚寅克霍丘今從之邃傳載取二城在破邵陽洲後誤也】六月庚子青冀二州刺史桓和克朐山城【胊音劬】 乙巳魏安西將軍元麗擊王法智破之斬首六千級 張惠紹與假徐州刺史宋黑水陸俱進趣彭城圍高塚戍【水經注彭城同孝山隂有楚元王冢高十許丈廣百許步意者魏立戍於此乎趣七喻翻】魏武衛將軍奚康生將兵救之【將即亮翻】丁未惠紹兵不利黑戰死 太子統生五歲能遍誦五經庚戌始自禁中出居東宫 丁巳魏以度支尚書邢巒都督東討諸軍事【度徒洛翻】 魏驃騎大將軍馮翊惠公源懷卒懷性寛簡不喜煩碎【驃匹妙翻騎奇寄翻喜許記翻】常曰為貴人當舉綱維何必事事詳細譬如為屋但外望高顯楹棟平正基壁完牢足矣斧斤不平斵削不密非屋之病也 秋七月丙寅桓和擊魏兖州拔固城【固城疑即抱犢固城也抱犢固在蘭陵界】 呂苟兒率衆十餘萬屯孤山圍逼秦州【此孤山當在上邽左右魏秦州治上邽領天水略陽漢陽郡】元麗進擊大破之行秦州事李韶掩擊孤山獲其父母妻子庚辰苟兒帥其徒詣麗降【帥讀曰率降戶江翻】兼太僕卿楊椿别討陳瞻瞻據險拒守諸將或請伏兵山蹊斷其出入【斷丁管翻】待糧盡而攻之或欲斬木焚山然後進討椿曰皆非計也自官軍之至所向輒克賊所以深竄正避死耳今約勒諸軍勿更侵掠賊必謂我見險不前待其無備然後奮擊可一舉平也乃止屯不進賊果出抄掠椿復以馬畜餌之【抄楚交翻復扶又翻】不加討逐久之隂簡精卒銜枚夜襲之斬瞻傳首秦涇二州皆平 戊子徐州刺史王伯敖與魏中山王英戰於隂陵【隂陵縣漢屬九江郡晉屬淮南郡梁北譙郡治隂陵城隋改北譙郡為全椒縣屬江都郡唐全椒縣屬滁州】伯敖兵敗失亡五千餘人己丑魏發定冀瀛相并肆六州十萬人以益南行之兵【相息亮翻】上遣將軍角念將兵一萬屯蒙山招納兖州之民降者甚衆【魏收志南青州東安郡新泰縣東南有蒙山蓋蒙山即古所謂東蒙也與固城孤山皆近魏兖州東界故梁連兵據之以招兖州之民北史邢巒傳謂是時梁人侵軼徐兖是矣角姓也姓苑漢有角善叔將即亮翻降戶江翻】是時將軍蕭及屯固城桓和屯孤山【魏收志蘭陵郡蘭陵縣有石孤山又昌慮縣有孤山】魏邢巒遣統軍樊魯攻和别將元恒攻及【恒戶澄翻】統軍畢祖朽攻念壬寅魯大破和於孤山恒拔固城祖朽擊念走之己酉魏詔平南將軍安樂王詮督後發諸軍赴淮南詮長樂之子也【安樂王長樂見一百三十三卷宋蒼梧王元徽三年樂音洛詮且緣翻】將軍藍懷恭與魏邢巒戰于睢口【姓譜藍魯甘翻姓也戰國策有中山大夫藍諸水經注睢水過睢陵縣故城北而東南流逕下相縣故城南又東南流入于泗謂之睢口睢音雖】懷恭敗績巒進圍宿預懷恭復于清南築城【清南清水之南也復扶又翻】巒與平南將軍楊大眼合攻之九月癸酉拔之斬懷恭殺獲萬計張惠紹棄宿預【此與後張惠紹聞洛口敗引兵退本一事耳解見後】蕭昞棄淮陽遁還臨川王宏以帝弟將兵【將即亮翻下同】器械精新軍容甚盛北人以為百數十年所未之有軍次洛口【水經注洛澗在西曲陽縣北劉牢之斬秦將梁成處北歷秦墟下注淮謂之洛口】前軍克梁城【即謂昌義之克梁城也】諸將欲乘勝深入宏性懦怯部分乖方【分扶問翻】魏詔邢巒引兵度淮與中山王英合攻梁城宏聞之懼召諸將議旋師呂僧珍曰知難而退不亦善乎宏曰我亦以為然柳惔曰自我大衆所臨何城不服何謂難乎裴邃曰是行也固敵是求何難之避馬仙琕曰王安得亡國之言天子掃境内以屬王【惔徒甘翻琕部田翻屬之欲翻】有前死一尺無却生一寸昌義之怒須髮盡磔【磔陟格翻張開也】曰呂僧珍可斬也豈有百萬之師出未逢敵望風遽退何面目得見聖主乎朱僧勇胡辛生拔劍而退【退據南史宏傳當作起】曰欲退自退下官當前向取死議者罷出僧珍謝諸將曰殿下昨來風動【謂宏心風動也】意不在軍深恐大致沮喪故欲全師而返耳【沮在呂翻喪息浪翻】宏不敢遽違羣議停軍不前魏人知其不武遺以巾幗【遺于季翻幗古獲翻】且歌之曰不畏蕭娘與呂姥【言其怯懦如婦人女子也姥莫補翻】但畏合肥有韋虎虎謂韋叡也僧珍歎曰使始興吳平為帥而佐之豈有為敵人所侮如是乎【始興王憺吳平侯昺帥所類翻】欲遣裴邃分軍取夀陽大衆停洛口宏固執不聽令軍中曰人馬有前行者斬於是將士人懷憤怒魏奚康生馳遣楊大眼謂中山王英曰梁人自克梁城已後久不進軍其勢可見必畏我也王若進據洛水彼自犇敗英曰蕭臨川雖騃其下有良將韋裴之屬未可輕也【騃古駭翻將即亮翻】宜且觀形勢勿與交鋒張惠紹號令嚴明所至獨克軍于下邳【前已言張惠紹棄宿預遁還矣宿預在下邳東南百餘里此言軍于下邳是未棄宿預之前事李延夀以此事載之臨川王宏傳通鑑因亦連而書之】下邳人多欲降者惠紹諭之曰我若得城諸卿皆是國人【國人猶言王民也降戶江翻下同】若不能克徒使諸卿失鄉里非朝廷弔民之意也今且安堵復業勿妄自辛苦降人咸悦己丑夜洛口暴風雨軍中驚臨川王宏與數騎逃去將士求宏不得皆散歸 【考異曰梁書宏傳云會征役久有詔班師殊為不實今從南史】棄甲投戈填滿水陸捐棄病者及羸老死者近五萬人【羸倫為翻近其靳翻】宏乘小船濟江夜至白石壘叩城門求入臨汝侯淵猷登城謂曰百萬之師一朝鳥散國之存亡未可知也恐姦人乘間為變【間古莧翻】城不可夜開宏無以對乃縋食饋之【縋馳偽翻】淵猷淵藻之弟時昌義之軍梁城聞洛口敗與張惠紹皆引兵退【此即張惠紹棄宿預一事也通鑑因南史臨川王宏傳所載者書之遂致複出】魏主詔中山王英乘勝平蕩東南逐北至馬頭攻拔之城中糧儲魏悉遷之歸北議者咸曰魏運米北歸當不復南向【復扶又翻】上曰不然此必欲進兵為詐計耳乃命修鍾離城勑昌義之為戰守之備【馬頭城在鍾離之西馬頭既陷魏必東攻鍾離故預為之備】冬十月英進圍鍾離魏主詔邢巒引兵會之巒上表以為南軍雖野戰非敵而城守有餘今盡鋭攻鍾離得之則所利無幾不得則虧損甚大且介在淮外借使束手歸順猶恐無糧難守况殺士卒以攻之乎又征南士卒從戎二時【從戎二時謂兵連不解自夏迄秋也】疲弊死傷不問可知雖有乘勝之資懼無可用之力若臣愚見謂宜修復舊戍撫循諸州以俟後舉江東之釁不患其無詔曰濟淮掎角事如前敕【釁許覲翻掎居蟻翻】何容猶爾盤桓【盤桓不進貌】方有此請可速進軍巒又表以為今中山進軍鍾離實所未解【解戶買翻曉也】若為得失之計【謂為一切之計或得或失未可必也】不顧萬全直襲廣陵出其不備或未可知若止欲以八十日糧取鍾離城者臣未之前聞也【英期以八十日糧取鍾離故巒云然】彼堅城自守不與人戰城塹水深非可填塞【塞悉則翻】空坐至春士卒自弊若遣臣赴彼從何致糧夏來之兵不齎冬服脱遇氷雪何方取濟臣寧荷怯懦不進之責不受敗損空行之罪【荷下可翻】鍾離天險朝貴所具【謂朝之貴臣所具知也朝直遥翻】若有内應則所不知如其無也必無克狀【言必無可克之狀】若信臣言願賜臣停若謂臣憚行求還臣所領兵乞盡付中山任其處分【處昌呂翻分扶問翻】臣止以單騎随之東西臣屢更為將【騎奇寄翻更工衡翻將即亮翻】頗知可否臣既謂難何容強遣【強其兩翻】乃召巒還更命鎮東將軍蕭寶寅與英同圍鍾離侍中盧昶素惡巒【更工衡翻惡烏路翻】與侍中領右衛將軍元暉共譖之使御史中尉崔亮彈巒在漢中掠人為奴婢【巒傳云巒初至漢中接豪右以禮撫衆以惠歲餘之後頗因其去就誅滅百姓籍為奴婢者二百餘人彈徒丹翻】巒以漢中所得美女賂暉暉言於魏主曰巒新有大功不當以赦前小事案之【謂是年正月生皇子赦也】魏主以為然遂不問暉與盧昶皆有寵於魏主而貪縱時人謂之餓虎將軍饑鷹侍中暉尋遷吏部尚書用官皆有定價大郡二千匹次郡下郡遞減其半餘官各有等差選者謂之市曹【以選曹貨賂為市因謂之市曹選須絹翻】 丁酉梁兵圍義陽者夜遁【聞洛口師潰故亦遁】魏郢州刺史婁悦追擊破之 柔然庫者可汗卒子伏圖立號佗汗可汗【佗汗魏言緒也可從刋入聲汗音寒佗徒河翻】改元始平戊申佗汗遣使者紇奚勿六跋如魏請和魏主不報其使謂勿六跋曰蠕蠕遠祖社崘乃魏之叛臣【事見一百八卷晉孝武太元十九年使疏吏翻蠕人兖翻崘盧昆翻】往者包容蹔聽通使【事見一百三十六卷齊武帝永明五年蹔與暫同】今蠕蠕衰微不及疇昔大魏之德方隆周漢正以江南未平少寛北略【少詩沼翻】通和之事未容相許若修藩禮欵誠昭著者當不爾孤也【孤負也】魏京兆王愉廣平王懷國臣多驕縱公行屬請【屬之欲翻】<br />
<br />
  魏主詔中尉崔亮窮治之【治直之翻】坐死者三十餘人其不死者悉除名為民惟廣平右常侍楊昱文學崔楷以忠諫獲免昱椿之子也【自晉以來王國置師友文學各一人左古常侍各一人楊椿見一百三十七卷齊武帝永明八年】 十一月乙丑大赦詔右衛將軍曹景宗都督諸軍二十萬救鍾離上勑景宗頓道人洲【道人洲在邵陽洲之東】俟衆軍齊集俱進景宗固啟求先據邵陽洲尾上不許景宗欲專其功違詔而進值暴風猝起頗有溺者【溺奴狄翻】復還守先頓【謂還守道人洲也復扶又翻】上聞之曰景宗不進蓋天意也若孤軍獨往城不時立必致狼狽今破賊必矣 初漢歸義侯勢之末羣獠始出北自漢中南至邛笮布滿山谷【事見九十七卷晉孝宗永和二年獠魯皓翻邛渠容翻笮音昨】勢既亡蜀民多東徙山谷空地皆為獠所據其近郡縣與華民雜居者頗輸租賦遠在深山者郡縣不能制梁益二州歲伐獠以自潤公私利之及邢巒為梁州獠近者皆安堵樂業【樂音洛】遠者不敢為寇巒既罷去魏以羊祉為梁州刺史傅豎眼為益州刺史【去年魏得晉夀置益州豎而庾翻】祉性酷虐不得物情獠王趙清荆引梁兵入州境為寇祉遣兵擊破之豎眼施恩布信大得獠和 十二月癸卯都亭靖侯謝朏卒【朏敷尾翻】 魏人議樂久不决【三年魏命議樂事見上卷】六年春正月公孫崇請委衛軍將軍尚書右僕射高肇監其事【監工衡翻】魏主知肇不學詔太常卿劉芳佐之 魏中山王英與平東將軍楊大眼等衆數十萬攻鍾離鍾離城北阻淮水魏人於邵陽洲兩岸為橋樹柵數百步跨淮通道英據南岸攻城大眼據北岸立城以通糧運城中衆纔三千人昌義之督帥將士随方抗禦魏人以車載土填塹使其衆負土随之嚴騎蹙其後人有未及回者因以土迮之【帥讀曰率將即亮翻塹七艷翻騎奇寄翻迮側百翻迫也】俄而塹滿衝車所撞【撞傳江翻】城土輒頹義之用泥補之衝車雖入而不能壞魏人晝夜苦攻分番相代墜而復升【復扶又翻下同】莫有退者一日戰數十合前後殺傷萬計魏人死者與城平二月魏主召英使還英表稱臣志殄逋寇而月初已來【已以字通】霖雨不止若三月晴霽城必克願少賜寛假【少詩沼翻】魏主復詔曰彼土蒸濕無宜久淹勢雖必取乃將軍之深計兵久力殆亦朝廷之所憂也英猶表稱必克魏主遣步兵校尉范紹詣英議攻取形勢紹見鍾離城堅勸英引還英不從【元英違衆議志在必克鍾離恃義陽之勝而驕也兵法曰常勝之家難與慮敵又曰兵驕者敗其謂是歟校戶教翻】上命豫州刺史韋叡將兵救鍾離【將即亮翻下同】受曹景宗節度叡自合肥取直道由隂陵大澤行【水經注濠水出隂陵縣之陽亭東北流逕鍾離城下而注于淮隂陵蓋有在鍾離西南合肥東北也】值澗谷輒飛橋以濟師人畏魏兵盛多勸叡緩行叡曰鍾離今鑿穴而處【處昌呂翻】負戶而汲車馳卒奔猶恐其後而况緩乎魏人已墮吾腹中卿曹勿憂也旬日至邵陽上豫勑曹景宗曰韋叡卿之鄉望【曹景宗新野人韋叡以京兆著姓居襄陽既同州鄉而韋為望族】宜善敬之景宗見叡禮甚謹上聞之曰二將和師必濟矣景宗與叡進頓邵陽洲叡於景宗營前二十里夜掘長塹樹鹿角截洲為城去魏城百餘步南梁太守馮道根能走馬步地計馬足以賦功比曉而營立【賦市也給與也功力也計一夫之力所任作謂之功杜佑通典曰凡築城下濶與高倍上濶與下倍城高五丈下闊二丈五尺上闊一丈二尺五寸高下闊狹以此為準料功上闊下加濶得三丈七尺五寸半之得一丈八尺七寸五分以高五丈乘之一尺之城積數得九十三丈七尺五寸每一功日築土二尺計功約四十七人一步五尺之城計役二百三十五人一百步計役二萬三千五百人率一里則十里可知其出土負簣並計之大功之内城濠面濶二丈深一丈㡳闊一丈以面濶加底積數大半之得數一丈五尺以深一丈乘之鑿濠一尺得數一十五丈每一人計功日出三丈計功五人一步五尺計功二十五人十步計功二百五十人一里計功七萬五百人以此為數則百里可知比必利翻及也】魏中山王英大驚以杖擊地曰是何神也景宗等器甲精新軍容甚盛魏人望之奪氣景宗慮城中危懼募軍士言文達等【言姓孔門言偃吳人今吳人猶有言姓】潛行水底齎勑入城城中始知有外援勇氣百倍楊大眼勇冠軍中將萬餘騎來戰所向皆靡叡結車為陳【冠古玩翻將即亮翻騎奇寄翻陳讀曰陣】大眼聚騎圍之叡以彊弩二千一時俱洞甲穿中【中如字】殺傷甚衆矢貫大眼右臂大眼退走明旦英自帥衆來戰【帥讀曰率】叡乘素木輿執白角如意以麾軍【如意翻】一日數合英乃退魏師復夜來攻城【復扶又翻】飛矢雨集叡子黯請下城以避箭叡不許軍中驚叡於城上厲聲呵之乃定【此確鬭也兩軍營壘相逼旦暮接戰勇而無剛者不能支久韋叡於此是難能也比年襄陽之守使諸將連營而前如韋叡之畧城猶可全不至誤國矣嗚呼痛哉】牧人過淮北伐芻藁者皆為楊大眼所畧曹景宗募勇敢士千餘人於大眼城南數里築壘大眼來攻景宗擊却之壘成使别將趙草守之有抄掠者皆為草所獲是後始得縱芻牧上命景宗等豫裝高艦使與魏橋等為火攻之計【將即亮翻抄楚交翻艦戶黯翻】令景宗與叡各攻一橋叡攻其南景宗攻其北【魏於邵陽洲兩岸立橋南橋以接元英之兵北橋以接楊大眼之兵】三月淮水暴漲六七尺 【考異曰梁帝紀四月癸未景宗等破魏軍魏帝紀四月戊戌鍾離大水英敗績按曹景宗傳云三月春水生淮水暴漲梁魏二史蓋據奏到月日書之耳今從景宗傳】叡使馮道根與廬江太守裴邃秦郡太守李文釗等【沈約曰晉武帝分扶風為秦國中原亂其民南流寄居堂邑堂邑本為縣前漢屬臨淮後漢屬廣陵晉又屬臨淮惠帝永興元年以臨淮淮陵立堂邑郡安帝改堂邑為秦郡五代志揚州六合縣舊曰尉氏置秦郡】乘鬬艦競發擊魏洲上軍盡殪【殪於計翻】别以小船載草灌之以膏從而焚其橋風怒火盛煙塵晦冥敢死之士拔柵斫橋水又漂疾倏忽之間橋柵俱盡道根等皆身自搏戰軍人奮勇呼聲動天地【呼火故翻】無不一當百魏軍大潰英見橋絶脱身棄城走大眼亦燒營去諸壘相次土崩悉棄其器甲争投水死者十餘萬斬首亦如之叡遣報昌義之義之悲喜不暇答語但叫曰更生更生諸軍逐北至濊水上【魏收志睢州穀陽郡連城縣有濊水按水經注服䖍云穀水在沛國相縣界藎睢水逕穀熟而兩分穀水之名蓋因地變然則穀水即泗水也魏收又云睢州即梁之潼州治取慮城又按水經注睢水自穀熟東流逕取慮城北又東逕睢陵城北又東與潼水會參而攻之則濊水當在沛臨淮二郡界丁度集韻曰濊呼外翻一作渙音同水名在亳州是則濊水即渙水音同而字異耳】英單騎入梁城緣淮百餘里尸相枕藉生擒五萬人【枕之任翻藉慈夜翻 考異曰韋叡傳云其餘釋甲稽顙乞為囚奴者猶數十萬按魏軍共止數十萬如叡傳所言似為太過今從景宗傳】收其資糧器械山積牛馬驢騾不可勝計【騾盧戈翻勝音升】義之德景宗及叡請二人共會設錢二十萬官賭之【樗蒱賭博私相與為戲耳不設於公庭今官賭之於徐州府廨公賭之也博以取財曰賭音丁古翻】景宗擲得雉叡徐擲得盧遽取一子反之曰異事遂作塞【反讀曰翻又如字樗蒱得盧者勝反一子而作塞塞者擲采未成次擲者塞之以决勝負塞與簺同先代翻異事猶言怪事也】景宗與羣帥争先告捷【帥所類翻】叡獨居後世尤以此賢之【史言韋叡有功不伐】詔增景宗叡爵邑義之等受賞各有差 夏四月己酉以江州刺史王茂為尚書右僕射安成王秀為江州刺史秀將發主者求堅船以為齋舫【以船載齋庫物因曰齋舫舫甫妄翻並兩船曰舫】秀曰吾豈愛財而不愛士乎乃以堅者給參佐下者載齋物既而遭風齋舫遂破【時諸王並下士建安王偉與秀尤好人物時人方之四豪】 丁巳以臨川王宏為驃騎將軍開府儀同三司【驃匹妙翻騎奇寄翻】建安王偉為揚州刺史右光禄大夫沈約為尚書左僕射左僕射王瑩為中軍將軍 六月丙午馮翊等七郡叛降魏【馮翊等郡江左僑立於雍州界降戶江翻】 秋七月丁亥以尚書右僕射王茂為中軍將軍 八月戊子大赦魏有司奏中山王英經算失圖齊王蕭寶寅等守橋不固皆處以極法【處昌呂翻】己亥詔英寶寅免死除名為民楊大眼徙營州為兵【魏世祖真君五年置營州治和龍城領昌黎建德遼東樂浪冀陽郡】以中護軍李崇為征南將軍揚州刺史崇多事產業征南長史狄道辛琛屢諫不從【琛丑林翻】遂相糾舉詔並不問崇因置酒謂琛曰長史後必為刺史但不知得上佐何如人耳琛曰若萬一叨忝得一方正長史朝夕聞過是所願也崇有慙色 九月己亥魏以司空高陽王雍為太尉尚書令廣陽王嘉為司空 甲子魏開斜谷舊道【漢高祖之為漢王也從杜南入斜中張良送至褒中褒斜一谷也南谷曰褒北谷曰斜意此即斜谷舊道諸葛亮揚聲内斜谷取郿非杜南舊道也以事勢言之承平舊時自長安入蜀其取道就平易南北分争塞故道而開新路以依險今魏欲就平易以通梁益故復開舊道斜余遮翻谷音浴】 冬十月壬寅以五兵尚書徐勉為吏部尚書勉精力過人雖文案填積坐客充滿應對如流手不停筆又該綜百氏皆為避諱【為于偽翻】嘗與門人夜集客虞暠求詹事五官【太子詹事亦有五官掾嵩古老翻】勉正色曰今夕止可談風月不可及公事時人咸服其無私 閏月乙丑以臨川王宏為司徒行太子太傅尚書左僕射沈約為尚書令行太子少傅吏部尚書袁昂為右僕射 丁卯魏皇后于氏殂是時高貴嬪有寵而妬高肇勢傾中外后暴疾而殂人皆歸咎高氏宫禁事祕莫能詳也【為魏太子昌卒張本】 甲申以光禄大夫夏侯詳為尚書左僕射 乙酉魏葬順皇后于永泰陵 十二月丙辰豐城景公夏侯詳卒【沈約曰吳立富城縣晉武太康元年更名豐城屬豫章郡】 乙丑魏淮陽鎮都軍主常邕和以城來降【降戶江翻 考異曰魏帝紀十月庚午淮陽太守安樂以城南叛今從梁帝紀】<br />
<br />
  資治通鑑卷一百四十六  <br>
   </div> 

<script src="/search/ajaxskft.js"> </script>
 <div class="clear"></div>
<br>
<br>
 <!-- a.d-->

 <!--
<div class="info_share">
</div> 
-->
 <!--info_share--></div>   <!-- end info_content-->
  </div> <!-- end l-->

<div class="r">   <!--r-->



<div class="sidebar"  style="margin-bottom:2px;">

 
<div class="sidebar_title">工具类大全</div>
<div class="sidebar_info">
<strong><a href="http://www.guoxuedashi.com/lsditu/" target="_blank">历史地图</a></strong>  
<a href="http://www.880114.com/" target="_blank">英语宝典</a>  
<a href="http://www.guoxuedashi.com/13jing/" target="_blank">十三经检索</a> 
<br><strong><a href="http://www.guoxuedashi.com/gjtsjc/" target="_blank">古今图书集成</a></strong> 
<a href="http://www.guoxuedashi.com/duilian/" target="_blank">对联大全</a> <strong><a href="http://www.guoxuedashi.com/xiangxingzi/" target="_blank">象形文字典</a></strong> 

<br><a href="http://www.guoxuedashi.com/zixing/yanbian/">字形演变</a>  <strong><a href="http://www.guoxuemi.com/hafo/" target="_blank">哈佛燕京中文善本特藏</a></strong>
<br><strong><a href="http://www.guoxuedashi.com/csfz/" target="_blank">丛书&方志检索器</a></strong> <a href="http://www.guoxuedashi.com/yqjyy/" target="_blank">一切经音义</a>  

<br><strong><a href="http://www.guoxuedashi.com/jiapu/" target="_blank">家谱族谱查询</a></strong>  <strong><a href="http://shufa.guoxuedashi.com/sfzitie/" target="_blank">书法字帖欣赏</a></strong> 
<br>

</div>
</div>


<div class="sidebar" style="margin-bottom:0px;">

<font style="font-size:22px;line-height:32px">QQ交流群9:489193090</font>


<div class="sidebar_title">手机APP 扫描或点击</div>
<div class="sidebar_info">
<table>
<tr>
	<td width=160><a href="http://m.guoxuedashi.com/app/" target="_blank"><img src="/img/gxds-sj.png" width="140"  border="0" alt="国学大师手机版"></a></td>
	<td>
<a href="http://www.guoxuedashi.com/download/" target="_blank">app软件下载专区</a><br>
<a href="http://www.guoxuedashi.com/download/gxds.php" target="_blank">《国学大师》下载</a><br>
<a href="http://www.guoxuedashi.com/download/kxzd.php" target="_blank">《汉字宝典》下载</a><br>
<a href="http://www.guoxuedashi.com/download/scqbd.php" target="_blank">《诗词曲宝典》下载</a><br>
<a href="http://www.guoxuedashi.com/SiKuQuanShu/skqs.php" target="_blank">《四库全书》下载</a><br>
</td>
</tr>
</table>

</div>
</div>


<div class="sidebar2">
<center>


</center>
</div>

<div class="sidebar"  style="margin-bottom:2px;">
<div class="sidebar_title">网站使用教程</div>
<div class="sidebar_info">
<a href="http://www.guoxuedashi.com/help/gjsearch.php" target="_blank">如何在国学大师网下载古籍?</a><br>
<a href="http://www.guoxuedashi.com/zidian/bujian/bjjc.php" target="_blank">如何使用部件查字法快速查字?</a><br>
<a href="http://www.guoxuedashi.com/search/sjc.php" target="_blank">如何在指定的书籍中全文检索?</a><br>
<a href="http://www.guoxuedashi.com/search/skjc.php" target="_blank">如何找到一句话在《四库全书》哪一页?</a><br>
</div>
</div>


<div class="sidebar">
<div class="sidebar_title">热门书籍</div>
<div class="sidebar_info">
<a href="/so.php?sokey=%E8%B5%84%E6%B2%BB%E9%80%9A%E9%89%B4&kt=1">资治通鉴</a> <a href="/24shi/"><strong>二十四史</strong></a>&nbsp; <a href="/a2694/">野史</a>&nbsp; <a href="/SiKuQuanShu/"><strong>四库全书</strong></a>&nbsp;<a href="http://www.guoxuedashi.com/SiKuQuanShu/fanti/">繁体</a>
<br><a href="/so.php?sokey=%E7%BA%A2%E6%A5%BC%E6%A2%A6&kt=1">红楼梦</a> <a href="/a/1858x/">三国演义</a> <a href="/a/1038k/">水浒传</a> <a href="/a/1046t/">西游记</a> <a href="/a/1914o/">封神演义</a>
<br>
<a href="http://www.guoxuedashi.com/so.php?sokeygx=%E4%B8%87%E6%9C%89%E6%96%87%E5%BA%93&submit=&kt=1">万有文库</a> <a href="/a/780t/">古文观止</a> <a href="/a/1024l/">文心雕龙</a> <a href="/a/1704n/">全唐诗</a> <a href="/a/1705h/">全宋词</a>
<br><a href="http://www.guoxuedashi.com/so.php?sokeygx=%E7%99%BE%E8%A1%B2%E6%9C%AC%E4%BA%8C%E5%8D%81%E5%9B%9B%E5%8F%B2&submit=&kt=1"><strong>百衲本二十四史</strong></a>  <a href="http://www.guoxuedashi.com/so.php?sokeygx=%E5%8F%A4%E4%BB%8A%E5%9B%BE%E4%B9%A6%E9%9B%86%E6%88%90&submit=&kt=1"><strong>古今图书集成</strong></a>
<br>

<a href="http://www.guoxuedashi.com/so.php?sokeygx=%E4%B8%9B%E4%B9%A6%E9%9B%86%E6%88%90&submit=&kt=1">丛书集成</a> 
<a href="http://www.guoxuedashi.com/so.php?sokeygx=%E5%9B%9B%E9%83%A8%E4%B8%9B%E5%88%8A&submit=&kt=1"><strong>四部丛刊</strong></a>  
<a href="http://www.guoxuedashi.com/so.php?sokeygx=%E8%AF%B4%E6%96%87%E8%A7%A3%E5%AD%97&submit=&kt=1">說文解字</a> <a href="http://www.guoxuedashi.com/so.php?sokeygx=%E5%85%A8%E4%B8%8A%E5%8F%A4&submit=&kt=1">三国六朝文</a>
<br><a href="http://www.guoxuedashi.com/so.php?sokeytm=%E6%97%A5%E6%9C%AC%E5%86%85%E9%98%81%E6%96%87%E5%BA%93&submit=&kt=1"><strong>日本内阁文库</strong></a> <a href="http://www.guoxuedashi.com/so.php?sokeytm=%E5%9B%BD%E5%9B%BE%E6%96%B9%E5%BF%97%E5%90%88%E9%9B%86&ka=100&submit=">国图方志合集</a> <a href="http://www.guoxuedashi.com/so.php?sokeytm=%E5%90%84%E5%9C%B0%E6%96%B9%E5%BF%97&submit=&kt=1"><strong>各地方志</strong></a>

</div>
</div>


<div class="sidebar2">
<center>

</center>
</div>
<div class="sidebar greenbar">
<div class="sidebar_title green">四库全书</div>
<div class="sidebar_info">

《四库全书》是中国古代最大的丛书,编撰于乾隆年间,由纪昀等360多位高官、学者编撰,3800多人抄写,费时十三年编成。丛书分经、史、子、集四部,故名四库。共有3500多种书,7.9万卷,3.6万册,约8亿字,基本上囊括了古代所有图书,故称“全书”。<a href="http://www.guoxuedashi.com/SiKuQuanShu/">详细>>
</a>

</div> 
</div>

</div>  <!--end r-->

</div>
<!-- 内容区END --> 

<!-- 页脚开始 -->
<div class="shh">

</div>

<div class="w1180" style="margin-top:8px;">
<center><script src="http://www.guoxuedashi.com/img/plus.php?id=3"></script></center>
</div>
<div class="w1180 foot">
<a href="/b/thanks.php">特别致谢</a> | <a href="javascript:window.external.AddFavorite(document.location.href,document.title);">收藏本站</a> | <a href="#">欢迎投稿</a> | <a href="http://www.guoxuedashi.com/forum/">意见建议</a> | <a href="http://www.guoxuemi.com/">国学迷</a> | <a href="http://www.shuowen.net/">说文网</a><script language="javascript" type="text/javascript" src="https://js.users.51.la/17753172.js"></script><br />
  Copyright &copy; 国学大师 古典图书集成 All Rights Reserved.<br>
  
  <span style="font-size:14px">免责声明:本站非营利性站点,以方便网友为主,仅供学习研究。<br>内容由热心网友提供和网上收集,不保留版权。若侵犯了您的权益,来信即刪。scp168@qq.com</span>
  <br />
ICP证:<a href="http://www.beian.miit.gov.cn/" target="_blank">鲁ICP备19060063号</a></div>
<!-- 页脚END --> 
<script src="http://www.guoxuedashi.com/img/plus.php?id=22"></script>
<script src="http://www.guoxuedashi.com/img/tongji.js"></script>

</body>
</html>
