資治通鑑卷二百九十三 宋 司馬光 撰

胡三省 音註

後周紀四|{
	起柔兆執徐三月盡彊圉大荒落凡一年有奇}


世宗睿文孝武皇帝中

顯德三年三月甲午朔上行視水寨至淝橋|{
	行下孟翻淝橋於淝水上為橋也}
自取一石馬上持之至寨以供礮從官過橋者人齎一石|{
	礮與砲同音普豹翻從才用翻齎牋西翻}
太祖皇帝乘皮船入夀春壕中城上發連弩射之|{
	皮船縫牛皮為之連弩即今之划車弩也射而亦翻}
矢大如屋椽牙將館陶張瓊遽以身蔽之矢中瓊髀|{
	椽重椽翻中竹仲翻髀部禮翻}
死而復蘇鏃著骨不可出|{
	著直畧翻}
瓊飲酒一大巵令人破骨出之流血數升神色自若|{
	史言張瓊之勇後太祖登極遂以瓊總侍衛親兵}
唐主復以右僕射孫晟為司空|{
	復扶又翻}
遣與禮部尚書王崇質奉表入見稱自天祐以來|{
	見賢遍翻天祐唐昭宗年號}
海内分崩或跨據一方或遷革異代|{
	跨據一方謂四方割劇之國遷革異代謂中國數易主也}
臣紹襲先業奄有江表顧以瞻烏未定附鳳何從|{
	詩曰瞻烏爰止于誰之屋漢耿純曰攀龍鱗附鳳翼此言未見真主則無從而歸附也}
今天命有歸聲教遠被|{
	被皮義翻}
願比兩浙湖南仰奉正朔謹守土疆|{
	兩浙自錢鏐以來湖南自馬殷以來皆奕世奉事中國}
乞收薄伐之威|{
	薄迫也有鐘鼓曰伐詩云薄伐西戎又云薄伐玁狁}
赦其後服之罪首於下國俾作外臣則柔遠之德云誰不服又獻金千兩銀十萬兩羅綺二千匹晟謂馮延己曰此行當在左相|{
	唐以馮延己為左僕射位在孫晟上故晟云然}
晟若辭之則負先帝既行知不免中夜歎息謂崇質曰君家百口宜自為謀吾思之熟矣終不負永陵一培土餘無所知|{
	永陵者唐主父昇墓也培蒲枚翻唐陸龜蒙築城詞城上一培土手中千萬杵則培土以益土為義一培土猶言益一畚土也又薄口翻說文曰培塿小冢也一培土猶言一冢土也歐史作一抔土}
南漢甘泉宫使林延遇隂險多計數南漢主倚信之誅滅諸弟皆延遇之謀也|{
	南漢主誅諸弟事並見前}
乙未卒國人相賀延遇病甚薦内給事龔澄樞自代南漢主即日擢澄樞知承宣院及内侍省澄樞番禺人也|{
	龔澄樞繼林延遇用事南漢遂亡矣番音潘禺魚容翻又音愚}
光舒黄招安廵檢使行光州刺史何超以安隨申蔡四州兵數萬攻光州|{
	九域志光州西南至安州六百里隨州東至安州二百四十里東北至申州二百五十里申州東至光州二百五十五里光州北至蔡州二百五十里蓋以鄰郡之兵環而攻之}
丙申超奏唐光州刺史張紹棄城走都監張承翰以城降丁酉行舒州刺史郭令圖拔舒州唐蘄州將李福殺其知州王承雋舉州來降遣六宅使齊藏珍攻黄州|{
	九域志舒州西至蘄州二百九十八里蘄州西至黄州二百一十里三州皆瀕江}
彰武留後李彦頵性貪虐|{
	頵於倫翻}
部民與羌胡作亂攻之上召彦頵還朝|{
	自延州召還還從宣翻又如字朝直遥翻}
秦鳳之平也|{
	事見上卷上年}
上赦所俘蜀兵以隸軍籍|{
	五代會要顯德二年十二月以新收秦鳳州所擒川軍署為懷恩軍所謂隸軍籍也}
從征淮南復亡降于唐|{
	復扶又反}
癸卯唐主表獻百五十人上悉命斬之 舒州人逐郭令圖鐵騎都指揮使洛陽王審琦選輕騎夜襲舒州復取之令圖乃得歸|{
	得復歸舒州}
馬希崇及王延政之子繼沂皆在揚州招撫存之|{
	楚閩}


|{
	世事中國其後為南唐所俘囚於揚州周得揚州故撫存之}
丙午孫晟等至上所|{
	至行在所也}
庚戌上遣中使以孫晟詣夀春城下且招諭之仁贍見晟戎服拜於城上|{
	以邊帥見宰相禮拜晟}
晟謂仁贍曰君受國厚恩不可開門納寇上聞之甚怒晟曰臣為宰相豈可教節度使外叛邪上乃釋之|{
	孫晟之辭直周世宗亦何以罪之}
唐主使李德明孫晟言於上請去帝號|{
	去羌呂翻}
割夀濠泗楚光海六州之地|{
	六州之地皆瀕淮周既得之則唐失長淮之險藉使周從唐之請而罷兵江北之地它日亦不能守矣}
仍歲輸金帛百萬以求罷兵|{
	輸春遇翻}
上以淮南之地已半為周有諸將捷奏日至欲盡得江北之地不許德明見周兵日進奏稱唐主不知陛下兵力如此之盛願寛臣五日之誅得歸白唐主盡獻江北之地上乃許之晟因奏遣王崇質與德明俱歸|{
	王崇質副孫晟來使}
上遣供奉官安弘道送德明等歸金陵賜唐主書其略曰但存帝號何爽歲寒|{
	爽差也言歲寒知松柏之後彫此約不差也許唐主自帝江南}
儻堅事大之心終不廹人于險又曰俟諸郡之悉來|{
	謂江北諸郡也}
即大軍之立罷言盡於此更不煩云|{
	煩勞也言更不勞云云也}
苟曰未然請從茲絶又賜其將相書使熟議而來唐主復上表謝|{
	復扶又翻上時掌翻}
李德明盛稱上威德及甲兵之彊勸唐主割江北之地唐主不悦宋齊丘以割地為無益德明輕佻言多過實國人亦不之信|{
	佻士彫翻國人謂南唐通國之人史言誕妄之士雅不足以孚乎人不惟喪身且誤國事}
樞密使陳覺副使李徵古素惡德明與孫晟|{
	惡烏路翻}
使王崇質異其言因譖德明於唐主曰德明賣國求利唐主大怒斬德明於市|{
	為鍾謨為李德明修怨張本}
吳程攻常州破其外郭執唐常州團練使趙仁澤送于錢唐仁澤見吳越王弘俶不拜責以負約|{
	唐興吳越本通好而吳越以周之命而攻唐故責其負約}
弘俶怒决其口至耳元德昭憐其忠為傅良藥得不死|{
	唐非無忠臣也不能用耳為于偽翻}
唐主以吳越兵在常州恐其侵逼潤州|{
	九域志常州西北至潤州一百七十一里}
以宣潤大都督燕王弘冀年少|{
	少詩照翻}
恐其不習兵徵還金陵|{
	還從宣翻}
部將趙鐸言於弘冀曰大王元帥衆心所恃逆自退歸所部必亂弘冀然之辭不就徵部分諸將為戰守之備|{
	分扶問翻}
龍武都虞候柴克宏再用之子也|{
	柴再用事楊氏為將屢立戰功又及事徐温父子}
沈默好施|{
	沈持林翻好呼到翻施式豉翻}
不事家產雖典宿衛日與賓客博奕飲酒未嘗言兵時人以為非將帥才|{
	將即亮翻帥所類翻}
至是有言克宏久不遷官者唐主以為撫州刺史克宏請效死行陳其母亦表稱克宏有父風可為將苟不勝任分甘孥戮|{
	趙括之母不肯保任其子柴克宏之母自稱薦其子皆知之審也孥子也言與其子甘同戮也行戶剛翻陳與陣同勝音升分扶問翻孥音奴}
唐主乃以克宏為右武衛將軍使將兵會袁州刺史陸孟俊救常州時唐精兵悉在江北克宏所將數千人皆羸老|{
	羸倫為翻}
樞密使李徵古復以鎧仗之朽蠧者給之克宏訴於徵古徵古慢罵之衆皆憤恚|{
	恚於避翻}
克宏怡然至潤州徵古遣使召還|{
	還從宣翻又如字}
以神衛統軍朱匡業代之燕王弘冀謂克宏|{
	謂者告語之也}
君但前戰吾當論奏乃表克宏才略可以成功常州危在旦莫|{
	莫讀曰暮}
不宜中易主將克宏引兵徑趣常州|{
	趣七喻翻}
徵古復遣使召之|{
	復扶又翻}
克宏曰吾計日破賊汝來召吾必奸人也命斬之使者曰受李樞密命而來克宏曰李樞密來吾亦斬之|{
	柴克宏前日之怡然乃養成今日之勇决也}
初鮑修讓羅晟在福州與吳程有隙|{
	漢天福十二年吳越使鮑修讓戍福州是年以吳程鎮福州}
至是程抑挫之二人皆怨先是唐主遣中書舍人喬匡舜使於吳越|{
	先悉薦翻}
壬子柴克宏至常州蒙其船以幕匿甲士於其中聲言迎匡舜吳越邏者以告|{
	邏即佐翻}
程曰兵交使在其間|{
	用左傳語}
不可妄以為疑唐兵登岸徑薄吳越營羅晟不力戰縱之使趣程帳|{
	趣七喻翻}
程僅以身免克宏大破吳越兵斬首萬級朱匡業至行營克宏事之甚謹吳程至錢唐吳越王弘俶悉奪其官 甲寅蜀主以捧聖控鶴都指揮使李廷珪為左右衛聖諸軍馬步都指揮使仍分衛聖匡聖步騎為左右十軍以武定節度使呂彦琦等為使|{
	為軍使也}
廷珪總之如趙廷隱之任|{
	蜀自李仁罕之誅趙廷隱專總宿衛諸軍後為安思謙所譖罷事並見前}
初柴克宏為宣州廵檢使始至城塹不修器械皆闕吏云自田頵王茂章李遇相繼叛|{
	唐天復三年田頵以宣州叛楊行密天祐二年王茂章叛梁乾元二年李遇叛事並見前紀頵於倫翻}
後人無敢治之者|{
	治直之翻}
克宏曰時移事異安有此理悉繕完之由是路彦銖攻之不克|{
	史言宣州獲全亦柴克宏之力}
聞吳程敗乙卯引歸唐主以克宏為奉化節度使克宏復請將兵救夀州未至而卒|{
	人有身死而名全者柴克宏是也克宏敵吳越可以勝使遇周師未必能爾復扶又翻}
河陽節度使白重贊|{
	重直龍翻}
以天子南征慮北漢乘虚入寇繕完守備且請兵於西京西京留守王晏初不之與又慮事出非常乃自將兵赴之重贊以晏不奉詔而來拒不納遣人謂之曰令公昔在陜服已立大功|{
	謂天福十二年晏舉陜城降漢高祖也晏時兼中書令故稱為令公陜失冉翻}
河陽小城不煩枉駕晏慙怍而還|{
	怍疾各翻還從宣翻又如字}
孟洛之民數日驚擾|{
	以王晏出兵而白重贊拒之恐兵交而罹其禍}
唐主命諸道兵馬元帥齊王景達將兵拒周以陳覺為監軍使前武安節度使邊鎬為應援都軍使|{
	邊鎬以失潭州奪節今叙用之}
中書舍人韓熙載上書曰信莫信於親王重莫重於元帥安用監軍使為|{
	句斷}
唐主不從遣鴻臚卿潘承祐詣泉建召募驍勇承祐薦前永安節度使許文稹|{
	稹止忍翻}
靜江指揮使陳德誠建州人鄭彦華林仁肇唐主以文稹為西面行營應援使彦華仁肇皆為將|{
	將即亮翻}
仁肇仁翰之弟也|{
	林仁翰見二百八十四卷晉出帝開運元年唐主之保大二年也}
夏四月甲子以侍衛親軍都指揮使歸德節度使李重進為廬夀等州招討使以武寜節度使武行德為濠州城下都部署唐右衛將軍陸孟俊自常州將兵萬餘人趣泰州

|{
	九域志自常州北至泰州一百九十七里}
周兵遁去孟俊復取之|{
	復取泰州}
遣陳德誠戍泰州孟俊進攻掦州屯于蜀岡韓令坤棄掦州走|{
	蜀岡在掦州城西掦州城在蜀岡東南城之東南北皆平地溝澮交貫惟蜀岡諸山西接廬滁凡北兵南寇掦州率循山而來據高為壘以臨之今陸孟俊據蜀岡以斷周兵援路故韓令坤懼而走}
帝遣張永德將兵救之令坤復入掦州|{
	援兵至故復入掦州}
帝又遣太祖皇帝將兵屯六合|{
	六合縣屬掦州在州西北一百三十里劉昫曰六合漢臨淮郡之堂邑縣晉置秦郡北齊置秦州隋置方州後廢唐武德初置六合縣}
太祖皇帝令曰掦州兵有過六合者折其足|{
	自掦州西北歸須過六合故云然折而設翻}
令坤始有固守之志帝自至夀春以來命諸軍晝夜攻城久不克會大雨營中水深數尺|{
	深式浸翻}
攻具及士卒失亡頗多|{
	時周兵以方舟載礟自淝河中流擊夀春城又束巨竹數十萬竿上施板屋號曰竹龍載甲士以攻之會淝水暴漲礟舟竹龍皆漂向南岸為唐兵所焚}
糧運不繼李德明失期不至|{
	李德明歸至金陵被誅}
乃議旋師或勸帝東幸濠州聲言夀州已破從之己巳帝自夀春循淮而東乙亥至濠州|{
	九域志夀州東至濠州三百八十里}
韓令坤販唐兵於城東|{
	此掦州城東也敗補邁翻}
擒陸孟俊初孟俊之廢馬希萼立希崇也|{
	事見二百九十卷太祖廣順元年}
滅故舒州刺史楊昭惲之族而取其財|{
	惲於粉翻薛史曰楊昭惲長沙人父謚事馬殷為節度行軍司馬謚仲女為衡陽王夫人希聲襲位昭惲遷衡州刺史自以地連戚里積財貨建大第二子為牙内都將少長豪富任氣凌下士大夫惡之長沙兵亂陸孟俊怒曰楊氏怙寵滅義為國人所患久矣於是族滅楊氏舒當作衡}
楊氏有女美獻於希崇令坤入掦州希崇以楊氏遺令坤令坤嬖之|{
	遺唯季翻嬖卑義翻又必計翻愛也幸也}
既獲孟俊將械送帝所楊氏在簾下忽撫膺慟哭令坤驚問之對曰孟俊昔在潭州殺妾家二百口今日見之請復其寃令坤乃殺之|{
	史言人不可妄殺雖女子亦能復讐}
唐齊王景達將兵二萬自瓜步濟江距六合二十餘里設柵不進諸將欲擊之太祖皇帝曰彼設柵自固懼我也今吾衆不滿二千若往擊之則彼見吾衆寡矣不如俟其來而擊之破之必矣居數日唐出兵趣六合|{
	趣七喻翻}
太祖皇帝奮擊大破之殺獲近五千人|{
	近其靳翻}
餘衆尚萬餘爭舟走度江溺死者甚衆於是唐之精卒盡矣是戰也士卒有不致力者太祖皇帝陽為督戰以劔斫其皮笠明日徧閲其皮笠有劒跡者數十人皆斬之由是部兵莫敢不盡死先是唐主聞掦州失守命四旁發兵取之|{
	先悉薦翻}
己卯韓令坤奏敗掦州兵萬餘人於灣頭堰|{
	九域志掦州江都縣有灣頭鎮在今掦州城北十五里敗補邁翻}
獲漣州刺史秦進崇|{
	唐蓋置漣州於漣水縣九域志漣水西南至楚州六十里}
張永德奏敗泗州萬餘人於曲溪堰|{
	曲溪在盱眙縣西南十里按招信圖經曲溪堰亦謂之新河堰}
丙戌以宣徽南院使向訓為淮南節度使兼沿江招討使渦口奏新作浮梁成丁亥帝自濠州如渦口|{
	渦口渦水入淮之口郡縣志渦口城東南至濠州九十里渦音戈}
帝鋭於進取欲自至掦州范質等以兵疲食少泣諫而止又嘗怒翰林學士竇儀欲殺之范質入救之帝望見識其意即起避之質趨前伏地叩頭諫曰儀罪不至死臣為宰相致陛下枉殺近臣罪皆在臣繼之以泣帝意解乃釋之 北漢葬神武帝於交城北山|{
	隋分晉陽縣置交城縣取縣西北古交城為名初治交山唐天授元年移治却波村九域志在陽曲縣西南一百里宋白曰大通監本古交城之地管東西二冶烹鐵務東冶在綿上縣西冶在交城縣北山}
廟號世祖 五月丙辰朔以渦口為鎮淮軍 丙申唐永安節度使陳誨敗福州兵於南臺江|{
	今福州南九里有釣龍臺山臨江南臺江當即是此地薛史地理志福州福唐縣晉天福初改為南臺縣蓋以江名縣也後復舊}
俘斬千餘級唐主更命永安曰忠義軍|{
	晉開運二年唐克建州置永安軍更工衡翻}
誨德誠之父也 戊戌帝留侍衛親軍都指揮使李重進等圍夀州自渦口北歸乙卯至大梁|{
	自渦口至大梁七百四十里}
六月壬申赦淮南諸州繫囚除李氏非理賦役事有不便於民者委長吏以聞|{
	長知兩翻}
侍衛步軍都指揮使彰信節度使李繼勲營於夀州城南唐劉仁贍伺繼勲無備|{
	伺相吏翻}
出兵擊之殺士卒數百人焚其攻具 唐駕部員外郎朱元因奏事論用兵方略唐主以為能命將兵復江北諸州|{
	將即亮翻下同}
秋七月辛卯朔以周行逢為武平節度使制置武安靜江等軍事行逢既兼總湖湘乃矯前人之弊留心民事悉除馬氏横賦|{
	横戶孟翻下驕横同馬氏自希範以來始加賦於境内}
貪吏猾民為民害者皆去之|{
	去羌呂翻}
擇亷平吏為刺史縣令朗州民夷雜居劉言王逵舊將多驕横行逢壹以法治之無所寛假衆怨懟且懼|{
	治直之翻懟直類翻}
有大將與其黨十餘人謀作亂行逢知之大會諸將於座中擒之數曰吾惡衣糲食充實府庫正為汝曹|{
	數所具翻責數也糲盧達翻為于偽翻}
何負而反今日之會與汝訣也立撾殺之座上股栗行逢曰諸君無罪皆宜自安樂飲而罷|{
	撾則瓜翻樂音洛}
行逢多計數善發隱伏將卒有謀亂及叛亡者行逢必先覺擒殺之所部凛然然性猜忍常散遣人密詗諸州事|{
	詗古永翻又翾正翻}
其之邵州者無事可復命但言刺史劉光委多宴飲行逢曰光委數聚飲|{
	數所角翻}
欲謀我邪即召還殺之親衛指揮使衡州刺史張文表恐獲罪求歸治所|{
	求解兵柄歸衡州也}
行逢許之文表歲時饋獻甚厚及謹事左右由是得免|{
	其後行逢臨卒謂其子保權曰吾起隴畝為團兵同時十人皆誅張文表獨存是時王逵張倣何敬真朱全琇潘叔嗣皆已死唯蒲公益宇文瓊彭萬和與文表史不言其有它此三人者必又相繼為行逢所殺而文表獨免也行逢死則文表叛矣}
行逢妻鄖國夫人鄧氏|{
	鄖音云路振九國志作嚴氏}
陋而剛决善治生|{
	治直之翻}
嘗諫行逢用法太嚴人無親附者行逢怒曰汝婦人何知鄧氏不悦因請之村墅視田園遂不復歸府舍|{
	之往也墅承與翻復扶又翻府舍朗州府舍也}
行逢屢遣人迎之不至一旦自帥僮僕來輸税|{
	帥讀曰率輸舂遇翻下同}
行逢就見之曰吾為節度使夫人何自苦如此鄧氏曰税官物也公為節度使不先輸税何以率下且獨不記為里正代人輸税以免楚撻時邪行逢欲與之歸不可|{
	不肯歸府舍也}
曰公誅殺太過常恐一旦有變村墅易為逃匿耳|{
	易以豉翻}
行逢慙怒其僚屬曰夫人言直公宜納之行逢壻唐德求補吏行逢曰汝才不堪為吏吾今私汝則可矣汝居官無狀吾不敢以法貸汝則親戚之恩絶矣與之耕牛農具而遣之行逢少時嘗坐事黥隸辰州銅阬|{
	少詩照翻黥具京翻唐文宗之世天下銅阬五十辰州不在其數辰州銅阬蓋馬氏所置也}
或說行逢公面有文恐為朝廷使者所嗤|{
	說式芮翻嗤丑之翻}
請以藥滅之行逢曰吾聞漢有黥布不害為英雄吾何耻焉|{
	黥布事見八卷秦二世二年}
自劉言王逵以來屢舉兵將吏積功及所羈縻蠻夷檢校官至三公者以千數|{
	羈縻蠻夷謂溪峒諸蠻夷}
前天策府學士徐仲雅自馬希廣之廢杜門不出|{
	馬希廣廢事見二百八十九卷漢乾祐二年}
行逢慕之署節度判官仲雅曰行逢昔趨事我奈何為之幕吏辭疾不至行逢迫脅固召之面授文牒終辭不取行逢怒放之邵州既而召還會行逢生日諸道各遣使致賀行逢有矜色謂仲雅曰自吾兼鎮三府|{
	三府武平武安靜江軍府也}
四鄰亦畏我乎仲雅曰侍中境内|{
	周行逢加侍中故徐仲雅稱之}
彌天太保徧地司空四鄰那得不畏行逢復放之邵州|{
	復扶又翻}
竟不能屈有僧仁及為行逢所信任軍府事皆預之亦加檢校司空娶數妻出入導從如王公|{
	從才用翻}
辛亥宣懿皇后符氏殂 唐將朱元取舒州刺史郭令圖棄城走李平取蘄州唐主以元為舒州團練使平為蘄州刺史元又取和州|{
	朱元李平皆李守貞所遣求救於唐者也事見二百八十八卷漢乾祐元年}
初唐人以茶鹽強民而徵其粟帛謂之博徵|{
	強其兩翻博博易也言以茶鹽博易而徵其粟帛}
又興營田於淮南民甚苦之及周師至爭奉牛酒迎勞|{
	勞力到翻}
而將帥不之恤|{
	帥所類翻}
專事俘掠視民如土芥民皆失望相聚山澤立堡壁自固操農器為兵|{
	操七刀翻}
積紙為甲時人謂之臼甲軍周兵討之屢為所敗|{
	敗補邁翻}
先所得唐諸州多復為唐有唐之援兵營於紫金山|{
	紫金山在夀春南或云即八公山}
與夀春城中烽火相應淮南節度使向訓奏請以廣陵之兵併力攻夀春俟克城更圖進取詔許之訓封府庫以授揚州主者命揚州牙將分部按行城中秋毫不犯|{
	分扶問翻行下孟翻}
揚州民感悦軍還或負糗糒以送之|{
	糗去久翻熬米麥為之糒平秘翻乾飯也}
滁州守將亦棄城去皆引兵趣夀春唐諸將請據險以邀周師宋齊丘曰如此則怨益深乃命諸將各自保守毋得擅出擊周兵由是夀春之圍益急 齊王景達軍于濠州遥為夀州聲援軍政皆出於陳覺景達署紙尾而已擁兵五萬無决戰意|{
	嗚呼比年襄陽之䧟得非援兵不進之罪}
將吏畏覺無敢言者 八月戊辰端明殿學士王朴司天少監王處訥撰顯德欽天歷上之|{
	初王處訥私造明玄歷于家因唐世所行崇玄歷而明之也帝以王朴通於歷數乃詔朴撰定以步日步月步星步發歛為四篇合為歷經併著顯德三年七政細行歷一卷以為欽天歷}
詔自來歲行之 殿前都指揮使義成節度使張永德屯下蔡唐將林仁肇以水陸軍援夀春永德與之戰仁肇以船實薪芻因風縱火欲焚下蔡浮梁俄而風回唐兵敗退永德為鐵綆千餘尺|{
	綆若杏翻}
距浮梁十餘步横絶淮流繫以巨木由是唐兵不能近|{
	近其靳翻}
九月丙午以端明殿學士左散騎常侍權知開封府事王朴為戶部侍郎充樞密副使 冬十月癸酉李重進奏唐人寇盛唐鐵騎都指揮使王彦昇等擊破之斬首三千餘級彦昇蜀人也 丙子上謂侍臣近朝徵歛穀帛多不俟收穫紡績之畢|{
	侍臣之下有曰字文意乃足近朝猶言近代也朝直遥翻下同}
乃詔三司自今夏税以六月秋税以十月起徵|{
	五代會要曰二税起徵皆以月一日}
民間便之 山南東道節度使守太尉兼中書令安審琦鎮襄州十餘年|{
	漢天福十二年安審琦鎮襄陽至是十年矣}
至是入朝除守太師遣還鎮既行上問宰相卿曹送之乎對曰送至城南審琦深感聖恩|{
	五代以來方鎮入朝者或留不遣或易置之今加官遣還鎮故感恩}
上曰近朝多不以誠信待諸侯諸侯雖有欲効忠節者其道無由王者但能毋失其信何患諸侯不歸心哉 壬午張永德奏敗唐兵于下蔡|{
	敗補邁翻}
是時唐復以水軍攻永德|{
	復扶又翻}
永德夜令善游者沒其船下縻以鐵鎖縱兵擊之船不得進退溺死者甚衆永德解金帶以賞善游者 甲申以太祖皇帝為定國節度使兼殿前都指揮使|{
	定國軍即同州匡國軍也太祖登極避御名始改為定國軍史亦因以後所改軍號書之}
太祖皇帝表渭州軍事判官趙普為節度推官張永德與李重進不相悦永德密表重進有二心帝不之信時二將各擁重兵衆心憂恐重進一日單騎詣永德營|{
	李重進時在夀州城下張永德營下蔡}
從容宴飲謂永德曰吾與公幸以肺腑俱為將帥|{
	從千容翻下同李重進太祖之甥張永德太祖之婿故云然}
奚相疑若此之深邪永德意乃解衆心亦安唐主聞之以蠟丸遺重進誘以厚利其書皆謗毁及反間之語|{
	遺唯季翻誘以久翻間古莧翻}
重進奏之初唐使者孫晟鍾謨從帝至大梁帝待之甚厚每朝會班於中書省官之後時召見飲以醇酒|{
	飲於鴆翻}
問以唐事晟但言唐主畏陛下神武事陛下無二心及得唐蠟書帝大怒召晟責以所對不實晟正色抗辭請死而已問以唐虚實默不對十一月乙巳帝命都承旨曹翰送晟於右軍廵院|{
	侍衛親軍分左右軍各有廵院以鞫繫囚}
更以帝意問之翰與之飲酒數行從容問之晟終不言翰乃謂曰有敇賜相公死|{
	以唐所授官稱之}
晟神色怡然索袍笏整衣冠南向拜曰臣謹以死報國乃就刑|{
	索山客翻孫晟可謂盡節於所事矣}
并從者百餘人皆殺之|{
	從才用翻}
貶鍾謨耀州司馬既而帝憐晟忠節悔殺之召謨拜衛尉少卿 帝召華山隱士真源陳摶|{
	真源漢古縣隋為谷陽縣唐高宗乾封元年以老子所生之地改為真源縣載初元年改為仙源縣神龍元年復曰真源屬亳州宋大中祥符七年改曰衛真縣九域志在州西六十里摶徒九翻}
問以飛升黄白之術|{
	飛升者謂羽化而升仙黄白者謂煉白金為黄金}
對曰陛下為天子當以治天下為務安用此為戊申遣還山詔州縣長吏常存問之|{
	治直之翻長知兩翻}
十二月壬申以張永德為殿前都點檢|{
	後唐以來車駕行幸及出征則置大内都點檢之官後周選驍勇之士充殿前諸班始置殿前都點檢於都指揮使之上自宋太祖皇帝以殿前都點檢登極是後不復除授}
分命中使發陳蔡宋亳潁兖曹單等州丁夫城下蔡|{
	單音善}
是歲唐主詔淮南營田害民尤甚者罷之遣兵部郎中陳處堯持重幣浮海詣契丹乞兵|{
	處昌呂翻 考異曰十國紀年作兵部郎中段處常今從晉陽聞見録}
契丹不能為之出兵|{
	為于偽翻}
而留處堯不遣處堯剛直有口辯久之忿懟數面責契丹主|{
	懟直類翻數所角翻}
契丹主亦不之罪也蜀陵榮州獠反|{
	宋白曰晉太元中益州刺史毛璩置西城戍於漢武陽縣之東境周閔帝}


|{
	元年於此置陵州因陵并為名榮州古夜郎國漢開為南安縣地蕭齊於此晉南安郡隋廢郡以其地屬資陽郡唐武德初割資州之太牢威遠二縣置榮州取境有崇德山為名獠魯皓翻}
弓箭庫使趙季文討平之|{
	職官分紀曰唐有内弓箭庫使五代去内字}
吳越王弘俶括境内民兵勞擾頗多判明州錢弘億手疏切諫罷之四年春正月己丑朔北漢大赦改元天會以翰林學士衛融為中書侍郎同平章事内客省使段恒為樞密使|{
	恒戶登翻}
宰相屢請立皇子為王上曰諸子皆幼|{
	上諸子宗訓是為恭帝次熙讓熙謹熙誨}
且功臣之子皆未加恩而獨先朕子能自安乎 周兵圍夀春連年未下|{
	前年十一月周兵始攻夀州}
城中食盡齊王景達自濠州遣應援使永安節度使許文稹都軍使邊鎬北面招討使朱元將兵數萬泝淮救之軍於紫金山列十餘寨如連珠與城中烽火晨夕相應又築甬道抵夀春|{
	甬余拱翻}
欲運糧以饋之綿亘數十里將及夀春李重進邀擊大破之死者五千人奪其二寨丁未重進以聞戊申詔以來月幸淮上劉仁贍請以邊鎬守城自帥衆决戰|{
	帥讀曰率}
齊王景達不許仁贍憤邑成疾其幼子崇諫夜泛舟度淮北為小校所執|{
	校戶教翻}
仁贍命腰斬之左右莫敢救監軍使周廷構哭於中門以救之仁贍不許廷構復使求救於夫人|{
	復扶又翻使疏吏翻}
夫人曰妾於崇諫非不愛也然軍法不可私名節不可虧若貸之則劉氏為不忠之門妾與公何面目見將士乎趣命斬之然後成喪|{
	趣讀曰促}
將士皆感泣議者以唐援兵尚彊多請罷兵帝疑之李穀寢疾在第二月丙寅帝使范質王溥就與之謀穀上疏以為夀春危困破在旦夕若鑾駕親征則將士爭奮援兵震恐城中知亡必可下矣上悦 庚午詔有司更造祭器祭玉等命國子博士聶崇義討論制度為之圖|{
	祭器樽彞簠簋籩豆之屬也祭玉蒼璧禮天黄琮禮地青圭禮東方赤璋禮南方白琥禮西方玄璜禮北方也時禮官博士凖詔議祭器祭玉制度國子祭酒尹拙引崔靈恩三禮義宗云蒼璧所以禮天其長十有二寸蓋法天之十二時又引江都集白虎通諸書所說云璧皆外圓内方又云黄琮所以禮地其長十寸以法地之數其琮外方内圓八角而有好國子博士聶崇義以為璧内外皆圓其徑九寸按阮氏鄭玄圖皆云九寸周禮玉人職又有九寸之璧及引爾雅云肉倍好謂之璧好倍肉謂之瑗肉好若一謂之環郭璞注云好孔也肉邊也而不載尺寸之數崇義又引冬官玉人云璧好三寸爾雅云肉倍好謂之璧蓋兩邊肉各三寸通好共九寸則其璧九寸明矣崇義又云黄琮八方以象地每角各剡出一寸六分共長八寸厚一寸按周禮圖及阮氏圖並無好又引冬官玉人云琮八角而無好崇義又云琮璜圭璧並是禮天地之器而爾雅唯言璧環瑗三者有好其餘琮璜諸器並不言之則黄琮八角而無好明矣時太常卿田敏已下以崇義援引周禮正文為是乃從之更工衡翻聶尼輒翻}
甲戌以王朴權東京留守兼判開封府事以三司使張美為大内都廵檢以侍衛都虞候韓通為京城内外都廵檢乙亥帝發大梁先是周與唐戰|{
	先悉薦翻}
唐水軍鋭敏周人無以敵之帝每以為恨返自夀春於大梁城西汴水側造戰艦數百艘|{
	艦戶黯翻艘蘇遭翻}
命唐降卒教北人水戰數月之後縱横出沒|{
	縱子容翻}
殆勝唐兵至是命右驍衛大將軍王環將水軍數千自閔河沿潁入淮|{
	丁度曰閔河本曰琵琶溝今名蔡河潁潁河也注詳見後卷蔡河下今按蔡河自東京戴樓門入京城出宣化水門投東南下經陳州至蔡口入潁河潁河自嵩山發源由潁昌至鹿邑界過蔡河口與蔡河合流經順昌府潁上縣至西正陽入淮河}
唐人見之大驚乙酉帝至下蔡三月己丑夜帝度淮抵夀春城下庚寅旦躬擐甲胄|{
	擐音宦}
軍於紫金山南命太祖皇帝擊唐先鋒寨及山北一寨皆破之斬獲三千餘級斷其甬道|{
	斷音短}
由是唐兵首尾不能相救至暮帝分兵守諸寨還下蔡 唐朱元恃功頗違元帥節度|{
	朱元恃其復舒和之功也}
陳覺與元有隙屢表元反覆不可將兵唐主以武昌節度使楊守忠代之守忠至濠州覺以齊王景達之命召元至濠州計事將奪其兵元聞之憤怒欲自殺門下客宋垍說元曰大丈夫何往不富貴何必為妻子死乎|{
	垍其冀翻說式芮翻為于偽翻}
辛卯夜元與先鋒壕寨使朱仁裕等舉寨萬餘人降禆將時厚卿不從元殺之帝慮其餘衆沿流東潰遽命虎捷左廂都指揮使趙晁|{
	五代會要廣順元年改侍衛馬軍曰龍捷左右軍步軍曰虎捷左右軍晁直遥翻}
將水軍數千沿淮而下壬辰旦帝軍于趙步|{
	趙步在淮河北岸水濱泊舟之地人坎岸為道以上下謂之步趙步以趙氏居其地而得名今自夀春花靨鎮沿淮東下百餘里得趙步灘又東逕梁城灘北齊梁控扼之地也淮水中有梁城灘又東二十五里至洛河口}
諸將擊唐紫金山寨大破之殺獲萬餘人擒許文稹邊鎬楊守忠餘衆果沿淮東走帝自趙步將騎數百循北岸追之諸將以步騎循南岸追之水軍自中流而下唐兵戰溺死及降者殆四萬人獲船艦糧仗以十萬數晡時帝馳至荆山港|{
	荆山在濠州鍾離縣西八十三里即梁武帝築堰之地今懷遠軍正治荆山}
距趙步二百餘里是夜宿鎮淮軍|{
	鎮淮軍時置於渦口}
癸酉從官始至|{
	從才用翻}
劉仁贍聞援兵敗扼吭歎息|{
	吭苦郎翻}
甲午發近縣丁夫城鎮淮軍為二城夾淮水徙下蔡浮梁於其間扼濠夀應援之路會淮水漲唐濠州都監彭城郭廷詔以水軍泝淮欲掩不備焚浮梁右龍武統軍趙匡贊覘知之|{
	覘丑廉翻又丑艶翻}
伏兵邀擊破之 唐齊王景達及陳覺皆自濠州奔歸金陵惟靜江指揮使陳德誠全軍而還|{
	陳德誠誨之子也還從宣翻又如字}
戊戌以淮南節度使向訓為武寧節度使淮南道行營都監將兵戍鎮淮軍己亥上自鎮淮軍復如下蔡庚子賜劉仁贍詔使自擇禍福唐主議自督諸將拒周中書舍人喬匡舜上疏切諫唐主以為沮衆流撫州|{
	沮在呂翻}
唐主問神衛統軍朱匡業劉存忠以守禦方畧匡業誦羅隱詩曰時來天地皆同力運去英雄不自由存忠以匡業言為然唐主怒貶匡業撫州副使流存忠於饒州既而竟不敢自行甲辰帝耀兵于夀春城北唐清淮節度使兼侍中劉仁贍病甚不知人丙午監軍使周廷構營田副使孫羽等作仁贍表遣使奉之來降丁未帝賜仁贍詔遣閤門使萬年張保續入城宣諭仁贍子崇讓復出謝罪戊申帝大陳甲兵受降於夀春城北廷構等舁仁贍出城|{
	舁音余又羊茹翻}
仁贍臥不能起帝慰勞賜賚|{
	勞力到翻賚來代翻}
復令入城養疾|{
	復扶又翻 考異曰實録時仁贍臥疾已亟遂翻然納欵而城中諸軍萬計皆屏息以聽其命又曰仁贍輕財重士法令嚴肅故能以一城之衆連年拒守逮其來降而其下無敢竊議者斯亦一時之名將也歐陽史三月仁贍病甚已不知人其副使孫羽詐為仁贍書以城降世宗命舁仁贍至帳前嗟嘆久之賜以玉帶御馬復使入城養疾是日制曰劉仁贍盡忠所事抗節無虧前代名臣幾人可比予之南伐得爾為多乃拜仁贍檢校太尉兼中書令天平軍節度使仁贍不能受命而卒世宗追封彭城郡王以其子崇讚為懷州刺史李景聞仁贍卒亦贈太師又曰仁贍既殺其子以自明矣豈有垂死而變節者乎今周世宗實録載仁贍降書蓋其副使孫羽等所為也當世宗時王環為蜀守秦州攻之久不下其後力屈而降世宗頗嗟其忠然止以為大將軍視世宗待二人之薄厚而考其制書乃知仁贍非降者也今從之}
庚戌徙夀州治下蔡|{
	夀州宋升為夀春府至今治下蔡縣而夀春故縣自為縣在淮水之南西北距下蔡二十五里高宗南渡復於夀州舊治夀春縣建安豐軍以為控扼之地蓋地險所在通古今不能易也宋白曰下蔡古之蔡國吴之州來左傳蔡成公遷于州來謂之下蔡是也漢為下蔡縣梁於硤石山築城以拒魏即今縣城也}
赦州境死罪以下州民受唐文書聚山林者並召令復業勿問罪有嘗為其殺傷者毋得讎訟曏日政令有不便於民者令本州條奏辛亥以劉仁贍為天平節度使兼中書令制辭略曰盡忠所事抗節無虧前代名臣幾人堪比朕之伐叛得爾為多是日卒追賜爵彭城郡王唐主聞之亦贈太師帝復以清淮軍為忠正軍|{
	楊氏以夀州置忠正軍後改清淮軍今復為忠正軍以旌劉仁贍之節按薛史唐明宗天成二年詔昇夀州為忠正軍長興二年閏五月己丑升廬州為昭順軍八月癸酉升州為昭信軍宋白續通典曰夀州後唐天成元年升為順化軍節度今並存之以俟博考}
以旌仁贍之節以右羽林統軍楊信為忠正節度使同平章事 前許州司馬韓倫侍衛馬軍都指揮使令坤之父也令坤領鎮安節度使倫居于陳州|{
	陳州鎮安軍治所}
干預政事貪汚不法為公私患為人所訟令坤屢為之泣請|{
	屢為于偽翻}
癸丑詔免倫死流沙門島|{
	登州蓬萊縣有沙門島置沙門寨}
倫後得赦還居洛陽與光禄卿致仕柴守禮及當時將相王溥王晏王彦超之父游處恃勢恣横洛陽人畏之謂之十阿父|{
	處昌呂翻下處之同横戶孟翻阿烏葛翻}
帝既為太祖嗣人無敢言守禮子者但以元舅處之優其俸給未嘗至大梁嘗以小忿殺人有司不敢詰|{
	詰去吉翻}
帝知而不問 詔開夀州倉賑饑民丙辰帝北還|{
	還從宣翻}
夏四月己巳至大梁詔修永福殿命宦官孫延希董其役丁丑帝至其所

見役徒有削柹為匕瓦中噉飯者|{
	柹方廢翻木札也匕卑履翻噉徒濫翻}
大怒斬延希於市 帝之克秦鳳也|{
	事見上卷二年}
以蜀兵數千人為懷恩軍乙亥遣懷恩指揮使蕭知遠等將士八百餘人西還|{
	既以示中國威德又欲使之言已克平淮南數千里之地以恐動蜀人}
壬午李穀扶疾入見|{
	見賢遍翻}
帝命不拜坐於御坐之側|{
	御坐徂卧翻}
穀懇辭禄位不許 甲申分江南降卒為六軍三十指揮號懷德軍 乙酉詔疏汴水北入五丈河|{
	河自都城歷曹濟及鄆其廣五丈舊名五丈河宋開寶六年詔改名廣濟河薛史曰浚五丈河東流於定陶入于濟以通齊魯運路}
由是齊魯舟楫皆達於大梁五月丁酉以太祖皇帝領義成節度使 詔以律令文古難知格敇煩雜不壹命御史知雜事張湜等|{
	唐制御史臺有侍御史六人以久次者一人知雜事謂之雜端杜佑通典曰知雜事謂之南床殿中監察不得坐凡侍御史之例不出累月遷登南省故謂之南床百官察其行止出入揖讓去就殿中以下皆禀而隨之湜丞職翻}
訓釋詳定為刑統|{
	刑統一書終宋之世行之}
唐郭廷謂將水軍斷渦口浮梁又襲敗武寧節度使武行德于定遠|{
	斷音短敗補邁翻}
行德僅以身免唐主以廷謂為滁州團練使充上淮水陸應援使|{
	上淮謂淮水之上游也}
蜀人多言左右衛聖馬步都指揮使保寧節度使同平章事李廷珪為將敗覆|{
	敗覆謂敗軍而秦鳳階成四州之地覆沒}
不應復典兵|{
	應於陵翻不應猶言不當也復扶又翻}
廷珪亦自請罷去六月乙丑蜀主加廷珪檢校太尉罷軍職李太后以典兵者多非其人謂蜀主曰吾昔見莊宗跨河與梁戰及先帝在太原平二蜀諸將非有大功無得典兵故士卒畏服|{
	李太后本唐莊宗後宫莊宗以賜蜀高祖故能言二主時事}
今王昭遠出於廝養|{
	王昭遠成都人年十三事東郭禪師智諲為童子蜀高祖嘗飯僧於府昭遠執巾履隨智諲以入高祖愛其慧黠時後主方就學令昭遠給事左右由是見親狎厮音斯養余亮翻}
伊審徵韓保貞趙崇韜皆膏粱乳臭子|{
	按路振九國志趙崇韜者廷隱之子}
素不習兵徒以舊恩寘於人上平時誰敢言者一旦疆場有事安能禦大敵乎以吾觀之惟高彦儔太原舊人終不負汝自餘無足任者蜀主不能從|{
	及孟氏之亡僅高彦儔一人能以死殉國至蜀主之死其母亦不食而卒婦人志節如此丈夫多有愧焉者}
丁丑以前華州刺史王祚為潁州團練使祚溥之父也溥為宰相祚有賓客溥常朝服侍立|{
	華戶化翻朝直遥翻}
客坐不安席祚曰㹠犬不足為起|{
	㹠與豚同足為于偽翻}
秋七月丁亥上治定遠軍及夀春城南之敗|{
	定遠縣名屬濠州軍字衍定遠之敗見上五月夀春城南之敗見去年六月}
以武寧節度使兼中書令武行德為左衛上將軍河陽節度使李繼勲為右衛大將軍 北漢主初立七廟|{
	北漢主自以承高祖隱帝之後與僭竊者不同然地狹國貧日困於兵今始能立七廟以倣天子之制}
司空兼門下侍郎同平章事李穀臥疾二年凡九表辭位八月乙亥罷守本官令每月肩輿一詣便殿議政事 以樞密副使戶部侍郎王朴檢校太保充樞密使 懷恩軍至成都|{
	是年四月遣懷恩軍西還今方至成都}
蜀主遣梓州别駕胡立等八十人東還|{
	胡立為蜀所禽見上卷二年還從宣翻}
且致書為謝請通好|{
	好呼到翻}
癸未立等至大梁帝以蜀主抗禮不之答蜀主聞之怒曰朕為天子郊祀天地時爾猶作賊何敢如是 九月中書舍人竇儼上疏請令有司討論古今禮儀作大周通禮考正鍾律作大周正樂又以為為政之本莫大擇人擇人之重莫先宰相自有唐之末輕用名器始為輔弼即兼三公僕射之官故其未得之也則以趨競為心既得之也則以容默為事但思解密勿之務守崇重之官逍遥林亭保安宗族乞今即日宰相於南宫三品兩省給舍以上各舉所知|{
	即日宰相謂見在相位者南宫謂尚書省也三品謂六部尚書也兩省謂中書門下省也給舍謂給事中中書舍人也}
若陛下素知其賢自可登庸|{
	庸用也}
若其未也且令以本官權知政事期歲之間察其職業若果能堪稱|{
	堪稱堪其任稱其職也稱尺證翻下不稱同}
其官已高則除平章事未高則稍更遷官權知如故若有不稱則罷其政事責其舉者又班行之中|{
	行戶剛翻}
有員無職者太半|{
	如諸衛將軍東宮官屬内諸使之類}
乞量其才器|{
	量音良}
授以外任試之於事還以舊官登叙考其治狀|{
	治直吏翻}
能者進之否者黜之又請令盗賊自相糾告以其所告貲產之半賞之或親戚為之首則論其徒侣而赦其所首者如此則盗不能聚矣|{
	言或親戚相與為盗其中有能自首者則赦之其徒侣則論其罪也首式又翻}
又新鄭鄉村團為義營各立將佐一戶為盗累其一村|{
	將即亮翻下同累力瑞翻}
一戶被盗罪其一將每有盗發則鳴鼓舉火丁壯雲集盗少民多無能脱者由是鄰縣充斥而一境獨清|{
	充斥獨清皆言盗也}
請令它縣皆效之亦止盗之一術也又累朝已來屢下詔書聽民多種廣耕上輸舊税|{
	朝直遥翻輸舂遇翻}
及其既種則有司履畝而增之故民皆疑懼而田不加闢夫為政之先莫如敦信信苟著矣則田無不廣田廣則穀多穀多則藏之民猶藏之官也又言陛下南征江淮一舉而得八州|{
	八州謂光黄舒蘄和楊滁泰皆取之}
再駕而平夀春|{
	事見上三月}
威靈所加前無彊敵今以衆擊寡以治伐亂勢無不克|{
	治直吏翻}
但行之貴速則彼民免俘馘之災此民息轉輸之困矣帝覽而善之儼儀之弟也 冬十月戊午設賢良方正直言極諫經學優深可為師法詳閑吏理達於教化等科|{
	此所謂制舉也時詔應天下諸色人中不限前資見任職官黄衣草澤並許應詔其逐處州府依每年貢舉人式例差官考試解送尚書吏部仍量試策論三道共三千字已上當日取文理俱優人物爽秀方得解送取來年十月集上都其登朝官亦許上表自舉}
癸亥北漢麟州刺史楊重訓舉城降|{
	太祖廣順二年楊重訓以麟州歸欵中間必又附北漢也}
以為麟州防禦使己巳以王朴為東京留守聽以便宜從事以三司使張美充大内都點檢壬申帝發大梁十一月丙戌至鎮淮軍是夜五鼓濟淮丁亥至濠州城西濠州東北十八里有灘唐人柵於其上環水自固|{
	環音宦}
謂周兵必不能涉戊子帝自攻之命内殿直康保裔帥甲士數百乘槖駝涉水|{
	帥讀曰率下同}
太祖皇帝帥騎兵繼之遂抜之李重進破濠州南關城癸巳帝自攻濠州王審琦抜其水寨唐人屯戰船數百於城北植巨木於淮水以限周兵帝命水軍攻之抜其木焚戰船七十餘艘|{
	艘蘇遭翻下同}
斬首二千餘級又攻抜其羊馬城城中震恐丙申夜唐濠州團練使郭廷謂上表言臣家在江南今若遽降恐為唐所種族|{
	種章勇翻}
請先遣使詣金陵禀命然後出降帝許之辛丑帝聞唐有戰船數百艘在渙水東|{
	渙水逕宿亳之間東南至巉石山西而南入淮}
欲救濠州自將兵夜發水陸擊之癸卯大破唐兵於洞口|{
	今濠州東九十里有浮山山下有穴名浮山洞夏潦不能及而冬不加高故人疑其山為浮洞口竊意即浮山洞口}
斬首五千餘級降卒二千餘人因鼓行而東所至皆下乙巳至泗州城下太祖皇帝先攻其南因焚城門破水寨及月城|{
	月城者臨水築城兩頭抱水形如却月}
帝居于月城樓督將士攻城 北漢主自即位以來|{
	顯德元年冬十一月北漢主即位}
方安集境内未遑外略是月契丹遣其大同節度使侍中崔勲將兵來會北漢欲同入寇北漢主遣其忠武節度使同平章事李存瓌將兵會之南侵潞州至其城下而還|{
	忠武軍許州屬周李存瓌遥領耳還從宣翻又如字}
北漢主知契丹不足恃而不敢遽與之絶贈送勲甚厚|{
	猶欲倚之以為聲援}
十二月乙卯唐泗州守將范再遇舉城降以再遇為宿州團練使上自至泗州城下禁軍中芻蕘者毋得犯民田|{
	蕘如招翻}
民皆感悦爭獻芻粟既克泗州無一卒敢擅入城者帝聞唐戰船數百艘泊洞口遣騎詗之唐兵退保清口|{
	詗古永翻又翾正翻清口即清河口也}
戊午上自將親軍自淮北進命太祖皇帝將步騎自淮南進諸將以水軍自中流進共追唐兵時淮濱久無行人葭葦如織多泥淖溝塹|{
	淖奴教翻塹七艶翻}
士卒乘勝氣苃涉爭進皆忘其勞|{
	苃蒲撥翻草行為苃水行為涉}
庚申追及唐兵且戰且行金鼓聲聞數十里|{
	聞音問}
辛酉至楚州西北大破之|{
	九域志泗州西至濠州一百七十五里東北至楚州二百二十里}
唐兵有沿淮東下者帝自追之太祖皇帝為前鋒行六十里擒其保義節度使濠泗楚海都應援使陳承昭以歸|{
	保義軍陜州屬周陳承昭遥領耳}
所獲戰船燒沉之餘|{
	沉持林翻}
得三百餘艘士卒殺溺之餘得七千餘人唐之戰船在淮上者於是盡矣 郭廷謂使者自金陵還|{
	還從宣翻又如字}
知唐不能救命録事參軍鄱陽李延鄒草降表延鄒責以忠義廷謂以兵臨之延鄒擲筆曰大丈夫終不負國為叛臣作降表|{
	為于偽翻史言李延鄒忠壯}
廷謂斬之舉濠州降得兵萬人糧數萬斛唐主賞李延鄒之子以官壬戌帝濟淮至楚州營于城西北乙丑唐雄武軍使知漣水縣事崔萬迪降丙寅以郭廷謂為亳州防禦使戊辰帝攻楚州克其月城庚午郭廷謂見於行宫|{
	見賢遍翻}
帝曰朕南征以來江南諸將敗亡相繼獨卿能斷渦口浮梁破定遠寨|{
	事見上五月}
所以報國足矣濠州小城使李璟自守能守之乎|{
	璟居永翻}
使將濠州兵攻天長帝遣鐵騎左廂都指揮使武守琦將騎數百趨揚州至高郵|{
	九域志高郵東南至揚州一百里趨七喻翻}
唐人悉焚揚州官府民居驅其人南渡江|{
	九域志揚州南至江四十五里}
後數日周兵至城中餘癃病十餘人而已|{
	癃良中翻疲病也}
癸酉守琦以聞帝聞泰州無備遣兵襲之丁丑抜泰州 南漢中書侍郎同平章事盧膺卒 南漢主聞唐屢敗憂形於色遣使入貢于周為湖南所閉|{
	閉塞也塞其道不得通也}
乃治戰艦修武備既而縱酒酣飲曰吾身得免幸矣何暇慮後世哉|{
	此所謂坐而待亡者也又古語云民主偷必死南漢主將死之徵也治直之翻艦戶黯翻}
唐使者陳處堯在契丹白契丹主請南遊太原北漢主厚禮之留數日北還竟卒於契丹|{
	去年唐主遣陳處堯如契丹乞師}


資治通鑑卷二百九十三














































































































































