






























































資治通鑑卷一百九十二 宋 司馬光 撰

胡三省 音注

唐紀八【起柔兆閹茂九月盡著雍困敦七月几二年}


高祖神堯大聖光孝皇帝下之下

武德九年九月突厥頡利獻馬三千匹羊萬口上不受【自是年八月甲子以後凡稱上者皆太宗也厥九勿翻}
但詔歸所掠中國戶口徵温彦博還朝【彦博没于突厥見上卷八年朝直遙翻}
丁未上引諸衛將卒習射於顯德殿庭【是年八月上即位於東宫顯德殿是後常御之將即亮翻下同}
諭之曰戎狄侵盜自古有之患在邊境少安則人主逸遊忘戰【少詩照翻下同}
是以寇來莫之能禦今朕不使汝曹穿池築苑專習弓矢居閒無事則為汝師突厥入寇則為汝將庶幾中國之民可以少安乎【將即亮翻幾居希翻少始紹翻}
於是日引數百人教射於殿庭上親臨試中多者賞以弓刀帛其將帥亦加上考【唐考功之法上中下皆分三等中多之中竹仲翻帥所類翻}
羣臣多諫曰於律以兵刃至御在所者絞今使卑碎之人張弓挾矢於軒陛之側陛下親在其間萬一有狂夫竊發出於不意非所以重社稷也韓州刺史封同人詐乘驛馬入朝切諫【唐舊志武德三年分同州之河西韓城郃陽置西韓州又于陜州界置南韓州封同人當是自韓城乘驛入朝也}
上皆不聽曰王者視四海如一家封域之内皆朕赤子朕一一推心置其腹中奈何宿衛之士亦加猜忌乎由是人思自勵數年之間悉為精鋭上嘗言吾自少經畧四方頗知用兵之要【少詩照翻}
每觀敵陳則知其彊弱【陳讀曰陣下其陳同}
常以吾弱當其彊彊當其弱彼乘吾弱逐奔不過數十百步吾乘其弱必出其陳後反擊之無不潰敗所以取勝多在此也 己酉上面定勲臣長孫無忌等爵邑【長知兩翻}
命陳叔達於殿下唱名示之且曰朕叙卿等勲賞或未當宜各自言【當丁浪翻}
於是諸將爭功紛紜不已【將即亮翻}
淮安王神通曰臣舉兵關西首應義旗【事見一百八十四卷隋恭帝義寧元年}
今房玄齡杜如晦等專弄刀筆功居臣上臣竊不服上曰義旗初起叔父雖首唱舉兵蓋亦自營脱禍及竇建德吞噬山東叔父全軍覆没【事見一百八十八卷武德二年}
劉黑闥再合餘燼叔父望風奔北【事見一百八十九卷四年}
玄齡等運籌帷幄坐安社稷論功行賞固宜居叔父之先叔父國之至親朕誠無所愛但不可以私恩濫與勲臣同賞耳諸將乃相謂曰陛下至公雖淮安王尚無所私吾儕何敢不安其分遂皆悦服【儕士皆翻分扶問翻}
房玄齡嘗言秦府舊人未遷官者皆嗟怨曰吾屬奉事左右幾何年矣今除官反出前宫齊府人之後上曰王者至公無私故能服天下之心朕與卿輩日所衣食皆取諸民者也故設官分職以為民也【為于偽翻}
當擇賢才而用之豈以新舊為先後哉必也新而賢舊而不肖安可捨新而取舊乎今不論其賢不肖而直言嗟怨豈為政之體乎詔民間不得妄立妖祠【妖於驕翻}
自非卜筮正術其餘雜

占悉從禁絶 上於弘文殿聚四部書二十餘萬卷【歐陽修曰歷代盛衰文章與時高下然其變態百出不可窮極何其多也自漢以來史官列其名氏篇第以為六藝七畧至唐始分為四類曰經史子集以甲乙丙丁為次謂之四庫書亦曰四部書}
置弘文館於殿側【唐會要武德四年於門下省置修文館至九年三月改為弘文館至其年九月太宗即位於弘文殿聚四部書二十餘萬卷於殿側置弘文館貞觀三年移於納義門西按閣本太極宫圖弘文館在門下省東而不載弘文殿納義門在嘉德門之西即我朝之崇文館也避宣祖諱改弘為崇}
精選天下文學之士虞世南禇亮姚思亷歐陽詢蔡允恭蕭德言等以本官兼學士令更日宿直聽朝之隙引入内殿講論前言往行商榷政事或至夜分乃罷【唐太宗以武定禍亂出入行間與之俱者皆西北驍武之士至天下既定精選弘文館學生日夕與之議論商榷者皆東南儒生也然則欲守成者捨儒何以哉更工衡翻朝直遙翻行下孟翻榷紇岳翻}
又取三品已上子孫充弘文館學士 冬十月丙辰朔日有食之 詔追封故太子建成為息王諡曰隱齊王元吉為刺王【息古國名諡法隱拂不成曰隱不思忘愛曰刺暴戾無親曰刺諡神至翻刺盧達翻}
以禮改葬葬日上哭之於宜秋門甚哀【太極宫圖宜秋門在千秋殿之西百福門之東}
魏徵王珪表請陪送至墓所 【考異曰高祖實録建成元吉傳太宗踐阼改葬加諡太宗實録及本紀皆不書葬月日唯唐歷在此年十月貞觀政要此表在二年據此年七月魏徵為諫議大夫宣慰山東王珪亦未為黄門侍郎葬建成元吉恐在後但别無月日可附今且從唐歷}
上許之命宫府舊僚皆送葬癸亥立皇子中山王承乾為太子生八年矣【生於承乾殿因}


【以名之}
庚辰初定功臣實封有差【唐爵九等一曰王食邑萬戶正一品二曰嗣王郡王食邑五千戶從一品三曰國公食邑三千戶從一品四曰開國郡公食邑二千戶正二品五曰開國縣公食邑千五百戶從二品六曰開國縣侯食邑千戶從三品七曰開國縣伯食邑七百戶正四品上八曰開國縣子食邑五百戶正五品上九曰開國縣男食邑三百戶從五品上凡封戶三丁以上為率歲租三之一入于朝廷食實封者得真戶分食諸州}
初蕭瑀薦封德彞於上皇上皇以為中書令及上即位瑀為左僕射德彞為右僕射議事已定德彞數反於上前【瑀音禹射寅謝翻數所角翻}
由是有隙時房玄齡杜如晦新用事皆疎瑀而親德彞【太宗初政之時以房杜之賢蕭瑀之直而不相親乃親封德彞者蓋以瑀之疎直難與共事於危疑之時而封德彞之狡數不與之親密則不能得其情也後之為相者其心無所權量但曰親君子遠小人未有能濟者也}
瑀不能平遂上封事論之【上時掌翻}
辭指寥落由是忤旨【忤五故翻}
會瑀與陳叔達忿爭於上前庚辰瑀叔達皆坐不敬免官 【考異曰舊傳太宗以玄齡等功高由是忤旨廢於家俄拜少師復為左僕射坐與叔達忿爭免按實録忿爭在作少師前今從之}
甲申民部尚書裴矩奏民遭突厥暴踐者【厥九勿翻踐慈演翻}
請戶給絹一匹上曰朕以誠信御下不欲虚有存恤之名而無其實戶有大小豈得雷同給賜乎於是計口為率初上皇欲彊宗室以鎮天下故皇再從三從弟【同曾祖為}


【再從兄弟同高祖為三從兄弟從才用翻}
及兄弟之子雖童孺皆為王王者數十人【封宗室為郡王見一百九十卷五年}
上從容問羣臣徧封宗子於天下利乎【從千容翻}
封德彞對曰前世唯皇子及兄弟乃為王自餘非有大功無為王者上皇敦睦九族大封宗室自兩漢以來未有如今之多者爵命既崇多給力役【力役蓋防閤庶僕白直之類}
恐非示天下以至公也上曰然朕為天子所以養百姓也豈可勞百姓以養己之宗族乎十一月庚寅降宗室郡王皆為縣公惟有功者數人不降 丙午上與羣臣論止盜或請重法以禁之上哂之【笑不壞顔為哂哂式忍翻}
曰民之所以為盜者由賦繁役重官吏貪求饑寒切身故不暇顧亷恥耳朕當去奢省費【去羌呂翻}
輕徭薄賦選用亷吏使民衣食有餘則自不為盜安用重法邪【邪音耶}
自是數年之後海内升平路不拾遺外戶不閉商旅野宿焉上又嘗謂侍臣曰君依於國國依於民刻民以奉君猶割肉以充腹腹飽而身斃君富而國亡故人君之患不自外來常由身出夫欲盛則費廣費廣則賦重賦重則民愁民愁則國危國危則君喪矣【夫音扶喪息浪翻}
朕常以此思之故不敢縱欲也 十二月己巳益州大都督竇軌奏稱獠反【是年六月廢大行臺置大都督府是後分諸州都督府為上中下三等大州都督從二品長史從三品司馬從四品中州都督正三品别駕正四品長史正五品上司馬正五品下下州都督從三品别駕長史司馬亦皆逓降一等僚魯皓翻}
請發兵討之上曰獠依阻山林時出鼠竊乃其常俗牧守苟能撫以恩信自然帥服【守式又翻帥與率同}
安可輕動干戈漁獵其民比之禽獸豈為民父母之意邪【邪音耶}
竟不許 上謂裴寂曰比多上書言事者【比毗至翻}
朕皆粘之屋壁【粘女亷翻}
得出入省覽每思治道或深夜方寢公輩亦當恪勤職業副朕此意上厲精求治數引魏徵入卧内訪以得失【省悉景翻治直吏翻數所角翻下者數同}
徵知無不言上皆欣然嘉納上遣使點兵【使疏吏翻}
封德彝奏中男雖未十八其軀幹壯大者亦可并點【唐制民年十六為中男十八始成丁二十一為丁充力役}
上從之敕出魏徵固執以為不可不肯署敕【按唐制中書舍人則署敕魏徵時為諫議大夫抑太宗亦使之連署邪}
至於數四上怒召而讓之曰中男壯大者乃姦民詐妄以避征役取之何害而卿固執至此對曰夫兵在御之得其道不在衆多陛下取其壯健以道御之足以無敵於天下何必多取細弱以增虚數乎且陛下每云吾以誠信御天下欲使臣民皆無欺詐今即位未幾失信者數矣【幾居豈翻數所角翻}
上愕然曰朕何為失信對曰陛下初即位下詔云逋負官物悉令蠲免【蠲圭淵翻}
有司以為負秦府國司者非官物徵督如故陛下以秦王升為天子國司之物非官物而何又曰關中免二年租調關外給復一年既而繼有敕云已役已輸者以來年為始散還之後方復更徵【調徒弔翻給復方目翻方復扶又翻下復點同言既散還其已輸之物而復徵之}
百姓固已不能無怪今既徵得物復點為兵何謂以來年為始乎又陛下所與共治天下者在於守宰【治直之翻守式又翻}
居常簡閲咸以委之至於點兵獨疑其詐豈所謂以誠信為治乎【治直吏翻下同}
上悦曰曏者朕以卿固執疑卿不達政事今卿論國家大體誠盡其精要夫號令不信則民不知所從天下何由而治乎【夫音扶治直吏翻下同}
朕過深矣乃不點中男賜徵金甕一上聞景州録事參軍張玄素名【景州漢平原郡鬲縣地隋置弓高縣屬觀州唐平河北分弓高置景州上州録事參軍從七品上掌勾稽省署抄目録事掌受事發辰兼勾稽失}
召見問以政道對曰隋主好自專庶務【好呼到翻}
不任羣臣羣臣恐懼唯知禀受奉行而已莫之敢違以一人之智决天下之務借使得失相半乖謬已多下諛上蔽不亡何待陛下誠能謹擇羣臣而分任以事高拱穆清而考其成敗以施刑賞何憂不治又臣觀隋末亂離其欲爭天下者不過十餘人而已其餘皆保鄉黨全妻子以待有道而歸之耳乃知百姓好亂者亦鮮但人主不能安之耳【好呼到翻鮮悉善翻}
上善其言擢為侍御史前幽州記室直中書省張藴古上大寶箴【唐諸州無記室唯王國有記室參軍從六品上藴古蓋廬江王瑗督幽州時為記室也唐制資序未至以它官入省者為直上時掌翻}
其畧曰聖人受命拯溺亨屯【屯陟倫翻}
故以一人治天下不以天下奉一人又曰壯九重於内所居不過容膝【治直之翻重直龍翻}
彼昏不知瑶其臺而瓊其室羅八珍於前所食不過適口【周禮膳夫珍用八物注云珍謂淳熬淳毋炮豚炮䍧擣珍漬熬肝膋也淳之純翻毋莫胡翻一音武由翻䍧作郎翻膋力彫翻}
惟狂罔念丘其糟而池其酒又曰勿没没而闇勿察察而明雖冕旒蔽目而視於未形雖黈纊塞耳而聽於無聲【冕而前旒所以蔽明黈纊充耳所以塞聰師古曰以黄綿為圜用兩組掛之於冕垂兩耳旁示不外聽也黈他口翻塞悉則翻}
上嘉之賜以束帛【唐制凡賜十段其率絹三匹布三端綿四屯若雜綵十段則絲布二匹紬二匹綾二匹縵四匹若賜蕃客錦綵率十段則錦一張綾二匹縵四匹綿四屯凡時服稱一具者全給之一副者減給之正冬之會稱賜束帛有差者五品已上五匹六品已下一匹命婦視其夫子}
除大理丞【大理丞正六品掌分判寺事}
上召傅奕賜之食謂曰汝前所奏幾為吾禍【事見上卷是年六月幾居依翻}
然凡有天變卿宜盡言皆如此勿以前事為懲也上嘗謂奕曰佛之為教玄妙可師卿何獨不悟其理對曰佛乃胡中桀黠【點戶八翻}
誑耀彼土中國邪僻之人取老莊玄談飾以妖幻之語用欺愚俗無益於民有害於國臣非不悟鄙不學也上頗然之 上患吏多受賕【枉法受賂曰賕賕音求}
密使左右試賂之有司門令史受絹一匹【司門郎屬刑部掌天下門關出入往來之籍賦而審其政有令史六人唐令布帛皆闊尺八寸長四丈為匹}
上欲殺之民部尚書裴矩諫曰為吏受賂罪誠當死但陛下使人遺之而受【遺于季翻}
乃陷人於法也恐非所謂道之以德齊之以禮【引論語孔子之言道讀曰導}
上悦召文武五品已上告之曰裴矩能當官力爭不為面從儻每事皆然何憂不治【治直吏翻}


臣光曰古人有言君明臣直裴矩佞於隋而忠於唐非其性之有變也君惡聞其過則忠化為佞君樂聞直言則佞化為忠【烏烏路翻樂音洛}
是知君者表也臣者景也表動則景隨矣

是歲進皇子長沙郡王恪為漢王宜陽郡王祐為楚王新羅百濟高麗三國有宿仇【北史曰新羅本辰韓種在高麗東南亦曰秦韓}


【相傳秦世亡人避役來適馬韓割東界居之故名秦韓始有六國稍分為十二新羅其一也或稱魏母丘儉破高麗奔沃沮後復國其留者為新羅兼有沃沮不耐韓濊之地其王本百濟人自海逃入新羅遂王其國附庸百濟後致彊盛因與百濟為敵百濟伐高麗來請救悉兵往破之自是相攻不置後獲百濟王殺之滋結怨麗力知翻}
迭相攻擊上遣國子助教朱子奢往諭指【晉武帝咸寧四年立國子學置祭酒博士各一人助教十五人以教生徒孝武太元十年損助教為十人唐助教五人從六品上掌佐博士分經教授}
三國皆上表謝罪【上時掌翻}


太宗文武大聖大廣孝皇帝上之上【諱世民高祖次子也帝初諡文皇帝廟號太宗咸亨五年追諡太宗文武聖皇帝天寶八載追尊太宗文武大聖皇帝十三載又加尊太宗文武大聖大廣孝皇帝}


貞觀元年春正月乙酉改元【觀古玩翻}
丁亥上宴羣臣奏秦王破陳樂【陳讀曰陣新志太宗為秦王破劉武周軍中相與作秦王破陳樂曲}
上曰朕昔受委專征民間遂有此曲雖非文德之雍容然功業由茲而成不敢忘本封德彞曰陛下以神武平海内豈文德之足比上曰戡亂以武守成以文文武之用各隨其時卿謂文不及武斯言過矣德彝頓首謝 己亥制自今中書門下及三品以上入閤議事皆命諫官隨之有失輒諫【程大昌曰唐西内太極殿即朔望受朝之所蓋正殿也太極之北有兩儀殿即常日視朝之所太極殿兩廡有東西二上閤則是兩閤皆有門可入已又可轉北而入兩儀也此太宗時入閤之制也至高宗以後多居東内御宣政前殿則謂之衙衙有仗御紫宸便殿則謂之入閤其不御宣政前殿而御紫宸也乃自正衙喚仗由閤門而入百官候朝于衙者因隨而入見謂之入閤}
上命吏部尚書長孫無忌等與學士法官更議定律令【長知兩翻}
寛絞刑五十條為斷右趾【斷丁管翻}
上猶嫌其慘曰肉刑廢已久宜有以易之蜀王法曹參軍裴弘獻【唐制諸王有功倉戶兵騎法士等七曹參軍正七品上}
請改為加役流徙三千里居作三年詔從之 【考異曰新舊刑法志皆云居作二年今從王溥會要}
上以兵部郎中戴胄忠清公直【兵部郎中掌判帳及天下武官之階品衛府之名數}
擢為大理少卿【少始照翻}
上以選人多詐冒資䕃敕令自首不首者死【選息絹翻下同首手又翻}
未幾有詐冒事覺者【幾居豈翻}
上欲殺之胄奏據法應流上怒曰卿欲守法而使朕失信乎對曰敕者出於一時之喜怒法者國家所以布大信於天下也陛下忿選人之多詐故欲殺之而既知其不可復斷之以法【斷丁亂翻}
此乃忍小忿而存大信也上曰卿能執法朕復何憂【復扶又翻下不復朕復何復同}
胄前後犯顔執法言如涌泉上皆從之天下無寃獄 上令封德彞舉賢久無所舉上詰之【詰去吉翻}
對曰非不盡心但於今未有奇才耳上曰君子用人如器各取所長古之致治者豈借才於異代乎【治直吏翻}
正患己不能知安可誣一世之人德彝慙而退御史大夫杜淹奏諸司文案恐有稽失請令御史就司檢校上以問封德彝對曰設官分職各有所司果有愆違御史自應糾舉若徧歷諸司搜摘疵纇【摘他狄翻纇盧對翻}
太為煩碎淹默然上問淹何故不復論執對曰天下之務當盡至公善則從之德彝所言真得大體臣誠心服不敢遂非上悦曰公等各能如是朕復何憂 右驍衛大將軍長孫順德受人餽絹事覺【長知兩翻驍堅堯翻}
上曰順德果能有益國家朕與之共有府庫耳何至貪冒如是乎【冒莫北翻}
猶惜其有功不之罪但於殿庭賜絹數十匹大理少卿胡演曰順德枉法受財罪不可赦柰何復賜之絹上曰彼有人性得絹之辱甚於受刑如不知愧一禽獸耳殺之何益 辛丑天節將軍燕郡王李藝據涇州反【宜州道為天節軍置將軍一人燕因肩翻}
藝之初入朝也【武德五年藝引兵與太子建成會討劉黑闥遂入朝朝直遙翻}
恃功驕倨秦王左右至其營藝無故敺之【敺烏口翻}
上皇怒收藝繫獄既而釋之上即位藝内不自安曹州妖巫李五戒【妖於驕翻}
謂藝曰王貴色已發勸之反藝乃詐稱奉密敕勒兵入朝遂引兵至豳州豳州治中趙慈皓馳出謁之【諸州治中即别駕}
藝入據豳州詔吏部尚書長孫無忌等為行軍總管以討之【長知兩翻}
趙慈皓聞官軍將至密與統軍楊岌圖之【岌魚及翻}
事洩藝囚慈皓岌在城外覺變勒兵攻之藝衆潰弃妻子將奔突厥至烏氏【厥九勿翻漢烏氏縣屬安定郡故城在彈箏峽東氏音支}
左右斬之傳首長安弟夀為利州都督亦坐誅 初隋末喪亂【喪息浪翻}
豪傑並起擁衆據地自相雄長唐興相帥來歸上皇為之割置州縣以寵禄之【帥讀曰率長知兩翻為于偽翻}
由是州縣之數倍於開皇大業之間上以民少吏多思革其弊【少詩沼翻}
二月命大加併省因山川形便分為十道一曰關内二曰河南三曰河東四曰河北五曰山南六曰隴右七曰淮南八曰江南九曰劒南十曰嶺南【京兆同華商岐邠隴涇原寧慶鄜坊丹延靈會鹽夏綏銀豐勝為關内道洛汝陜虢鄭滑許潁陳蔡汴宋亳徐濠宿鄆齊曹濮青淄登萊棣兖海近密為河南道蒲晉絳汾濕并南汾遼沁嵐石忻代朔蔚澤潞為河東道懷孟魏博相衛澶貝邢洺磁恒冀深趙滄景德易定幽涿瀛莫燕檀營平為河北道荆峽歸夔澧朗忠涪萬襄唐隨鄧均房郢復金梁洋利鳳興成扶文壁巴蓬通開隆果渠為山南道秦渭河鄯蘭階洮岷廓疊宕涼瓜沙甘肅為隴右道楊楚滁和夀廬舒光蘄黄安申為淮南道潤常蘇湖杭睦越衢婺括台福建泉宣歙池洪江鄂岳饒信䖍吉袁橅潭衡永道郴邵黔辰夷思僰為江南道益嘉眉邛簡資巂雅南會翼維松姚恭戎梓遂綿劒合龍普渝陵榮瀘為劒南道廣番循潮南康瀧端新封南宕春羅南石高南合崖振邕南方南簡淳欽南尹象藤桂梧賀連南昆靜樂南恭融容牢南林南扶南越南義交陸峯愛南德為嶺南道}
三月癸巳皇后帥内外命婦親蠶【内命婦宫内女官自貴妃至侍巾亦分九品外命婦有六王嗣王郡王之母妻為妃一品之國公母妻為國夫人三品以上母妻為郡夫人四品母妻為郡君五品母妻為縣君勲官四品有封者母妻為鄉君凡外命婦朝參視夫子之品唐制皇后以季春吉已享先蠶遂以親桑輿服志皇后親蠶服鞠衣黄羅為之帥讀曰率}
閏月癸丑朔日有食之 壬申上謂太子少師蕭瑀曰朕少好弓矢【少詩照翻瑀音禹好呼到翻}
得良弓十數自謂無以加近以示弓工乃曰皆非良材朕問其故工曰木心不直則脉理皆邪弓雖勁而發矢不直朕始悟曏者辯之未精也朕以弓矢定四方識之猶未能盡况天下之務其能徧知乎乃令京官五品以上【京官即在京職事官也}
更宿中書内省【更工衡翻}
數延見問以民間疾苦政事得失【數所角翻下數與同}
凉州都督長樂王幼良性麤暴【樂音洛}
左右百餘人皆無賴子弟侵暴百姓又與羌胡互市或告幼良有異志上遣中書令宇文士及馳驛代之并按其事左右懼謀劫幼良入北虜又欲殺士及據有河西復有告其謀者【復扶又翻下汗復復與同}
夏四月癸巳賜幼良死 五月苑君璋帥衆來降【帥讀曰率降戶江翻下同}
初君璋引突厥陷馬邑殺高滿政【事見一百九十卷高祖武德六年厥九勿翻}
退保恒安【隋朔州雲内縣之恒安鎮即後魏所都之平城也唐後置雲州及雲中縣恒戶登翻}
其衆皆中國人多弃君璋來降君璋懼亦降請捍北邊以贖罪上皇許之君璋請約契上皇使鴈門人元普賜之金券【鴈門縣帶代州漢廣武縣地}
頡利可汗復遣人招之【頡奚結翻可從刋入聲汗音寒}
君璋猶豫未决恒安人郭子威說君璋以恒安地險城堅【說輸芮翻}
突厥方彊且當倚之以觀變未可束手於人君璋乃執元普送突厥復與之合數與突厥入寇【數所角翻}
至是見頡利政亂知其不足恃遂帥衆來降【苑君璋與劉武周同起至是始降}
上以君璋為隰州都督芮國公【芮古國名}
有上書請去佞臣者【上時掌翻去羌呂翻}
上問佞臣為誰對曰臣居草澤不能的知其人願陛下與羣臣言或陽怒以試之彼執理不屈者直臣也畏威順旨者佞臣也上曰君源也臣流也濁其源而求其流之清不可得矣君自為詐何以責臣下之直乎朕方以至誠治天下見前世帝王好以權譎小數接其臣下者常竊恥之【治直之翻好呼到翻譎古穴翻}
卿策雖善朕不取也六月辛巳右僕射密明公封德彝薨【諡法思慮果遠曰明注云自任}


【近乎專}
壬辰復以太子少師蕭瑀為左僕射【蕭瑀去年免官復扶又翻下第復同少始照翻瑀音禹}
戊申上與侍臣論周秦修短蕭瑀對曰紂為不道武王征之周及六國無罪始皇滅之得天下雖同人心則異上曰公知其一未知其二周得天下增修仁義秦得天下益尚詐力此修短之所以殊也蓋取之或可以逆得守之不可以不順故也瑀謝不及山東大旱詔所在賑恤無出今年租賦【賑津忍翻}
秋七

月壬子以吏部尚書長孫無忌為右僕射無忌與上為布衣交加以外戚有佐命功【無忌皇后之兄以佐誅建成元吉為功長知兩翻}
上委以腹心其禮遇羣臣莫及欲用為宰相者數矣【歐陽修曰唐因隋制以三省之長尚書令侍中中書令共議國政此宰相職也後以太宗為尚書令臣下避不敢居其職由是僕射為尚書省長官與侍中中書令號為宰相其品位既崇不欲輕以授人故常以他官居宰相職而假以他名如杜淹以吏部尚書參議朝政魏徵以祕書監參預朝政或曰參議得失參知政事之類其名非一皆宰相職也數所角翻}
文德皇后固請曰妾備位椒房家之貴寵極矣誠不願兄弟復執國政呂霍上官可為切骨之戒幸陛下矜察上不聽卒用之【卒子恤翻}
初突厥性淳厚政令質畧頡利可汗得華人趙德言委用之【厥九勿翻頡奚結翻可從刋入聲汗音寒華人謂中國人也華讀如字}
德言專其威福多變更舊俗政令煩苛國人始不悦頡利又好信任諸胡而疎突厥胡人貪冒多反覆兵革歲動【數興兵討其反覆者故無寧歲更工衡翻好呼到翻冒莫北翻}
會大雪深數尺【深式鴆翻}
雜畜多死連年饑饉民皆凍餒頡利用度不給重斂諸部【畜許救翻斂力贍翻}
由是内外離怨諸部多叛兵浸弱言事者多請擊之上以問蕭瑀長孫無忌【瑀音禹長知兩翻}
曰頡利君臣昏虐危亡可必今擊之則新與之盟不擊恐失機會如何而可瑀請擊之無忌對曰虜不犯塞而弃信勞民非王者之師也上乃止 上問公卿以享國久長之策蕭瑀言三代封建而久長秦孤立而速亡上以為然於是始有封建之議 黄門侍郎王珪有密奏附侍中高士亷寢而不言上聞之八月戊戌出士亷為安州大都督 九月庚戌朔日有食之辛酉中書令宇文士及罷為殿中監御史大夫杜淹參預朝政【朝直遙翻 考異曰實録云杜淹署位不知所謂署位何也今從新書宰相表是時宰相無定名或云參預朝政或云參知機務之類甚衆不知其入銜否也如李靖三兩日一至門下中書平章政事魏徵朝章國典參議得失之類則决不入銜矣}
它官參預政事自此始淹薦刑部員外郎邸懷道【刑部郎掌貳尚書侍郎舉其典憲而辯其輕重邸丁禮翻姓也後魏有邸珍}
上問其行能【行下孟翻}
對曰煬帝將幸江都召百官問行留之計懷道為吏部主事【唐承隋志尚書諸司皆有主事從九品上}
獨言不可臣親見之上曰卿稱懷道為是何為自不正諫對曰臣爾時不居重任又知諫不從徒死無益上曰卿知煬帝不可諫何為立其朝既立其朝何得不諫卿仕隋容可云位卑後仕王世充尊顯矣何得亦不諫對曰臣於世充非不諫但不從耳上曰世充若賢而納諫不應亡國若暴而拒諫卿何得免禍淹不能對上曰今日可謂尊任矣可以諫未對曰願盡死上笑 辛未幽州都督王君廓謀叛道死君廓在州驕縱多不法徵入朝【朝直遙翻下同}
長史李玄道房玄齡從甥也【從才用翻}
憑君廓附書君廓私發之不識草書疑其告己罪行至渭南【後魏於新豐鄭縣之間置渭南郡隋廢郡為縣屬京兆尹在長安東一百一十五里}
殺驛吏而逃將奔突厥【厥九勿翻}
為野人所殺 嶺南酋長馮盎談殿等迭相攻擊【談姓殿名姓譜蜀録云晉有征東將軍談巴酋慈由翻長知兩翻}
久未入朝【朝直遙翻}
諸州奏稱盎反前後以十數上命將軍藺謩等發江嶺數十州兵討之魏徵諫曰中國初定嶺南瘴癘險遠不可以宿大兵且盎反狀未成未宜動衆上曰告者道路不絶何云反狀未成對曰盎若反必分兵據險攻掠州縣今告者已數年而兵不出境此不反明矣諸州既疑其反陛下又不遣使鎮撫【使疏吏翻下同}
彼畏死故不敢入朝若遣信臣示以至誠彼喜於免禍可不煩兵而服上乃罷兵 冬十月乙酉遣員外散騎侍郎李公掩持節慰諭之【散悉亶翻騎奇寄翻 考異曰魏文貞公故事作李公淹又有前蒲州刺史韋叔諧偕行今從實録}
盎遣其子智戴隨使者入朝上曰魏徵令我發一介之使而嶺表遂安【使疏吏翻朝直遙翻}
勝十萬之師不可不賞賜徵絹五百匹十二月壬午左僕射蕭瑀坐事免【瑀音禹}
戊申利州都督李孝常等謀反伏誅孝常因入朝留京師與右武衛將軍劉德裕及其甥統軍元弘善監門將軍長孫安業互說符命謀以宿衛兵作亂【監工銜翻長知兩翻}
安業皇后之異母兄也嗜酒無賴父晟卒【卒子恤翻}
弟無忌及后並幼安業斥還舅氏【高士亷無忌及后之舅也}
及上即位后不以舊怨為意恩禮甚厚及反事覺后涕泣為之固請曰【泣為于偽翻}
安業罪誠當萬死然不慈於妾天下知之今寘以極刑人必謂妾所為恐亦為聖朝之累【累力瑞翻}
由是得減死流巂州【巂音髓}
或告右丞魏徵私其親戚上使御史大夫温彦博按之無狀【言無其事狀}
彦博言於上曰徵不存形迹遠避嫌疑【遠於願翻}
心雖無私亦有可責上令彦博讓徵且曰自今宜存形迹它日徵入見【見賢遍翻下進見同}
言於上曰臣聞君臣同體宜相與盡誠若上下俱存形迹則國之興喪尚未可知【喪息浪翻}
臣不敢奉詔上瞿然曰吾已悔之【瞿九遇翻}
徵再拜曰臣幸得奉事陛下願使臣為良臣勿為忠臣上曰忠良有以異乎對曰稷契臯陶君臣協心俱享尊榮所謂良臣【契息列翻陶音遙}
龍逢比干面折廷爭身誅國亡所謂忠臣【逢皮江翻折之舌翻爭讀曰諍}
上悦賜絹五百匹上神采英毅羣臣進見者皆失舉措上知之每見人奏事必假以辭色冀聞規諫嘗謂公卿曰人欲自見其形必資明鏡君欲自知其過必待忠臣苟其君愎諫自賢【愎符逼翻}
其臣阿諛順旨君既失國臣豈能獨全如虞世基等謟事煬帝以保富貴煬帝既弑世基等亦誅【事見二百八十五卷高祖武德元年}
公輩宜用此為戒事有得失毋惜盡言 或上言秦府舊兵宜盡除武職追入宿衛【上時掌翻}
上謂之曰朕以天下為家惟賢是與豈舊兵之外皆無可信者乎汝之此意非所以廣朕德於天下也 上謂公卿曰昔禹鑿山治水而民無謗讟者與人同利故也【治直之翻}
秦始皇營宫室而人怨叛者病人以利己故也夫靡麗珍奇固人之所欲【夫音扶}
若縱之不已則危亡立至朕欲營一殿材用己具鑒秦而止王公已下宜體朕此意由是二十年間風俗素朴衣無錦繡公私富給 上謂黄門侍郎王珪曰國家本置中書門下以相檢察中書詔敕或有差失則門下當行駮正【中書出命門下審駮按唐制凡詔旨制敕璽書冊命皆中書舍人起草進畫既下則署行而過門下省有不便者塗竄而奏還謂之塗歸駮北角翻}
人心所見互有不同苟論難往來務求至當【難乃旦翻當丁浪翻}
捨己從人亦復何傷比來或獲己之短遂成怨隙【復扶又翻比毗至翻}
或苟避私怨知非不正【言知其非而不加駮正也}
順一人之顔情為兆民之深怨此乃亡國之政也煬帝之世内外庶官務相順從當是之時皆自謂有智禍不及身及天下大亂家國兩亡雖其間萬一有得免者亦為時論所貶終古不磨卿曹各當徇公忘私勿雷同也 上謂侍臣曰吾聞西域賈胡得美珠剖身以藏之【賈音古}
有諸侍臣曰有之上曰人皆知彼之愛珠而不愛其身也吏受賕抵法與帝王徇奢欲而亡國者何以異於彼胡之可笑邪【賕音求邪音耶}
魏徵曰昔魯哀公謂孔子曰人有好忘者徙宅而忘其妻孔子曰又有甚者桀紂乃忘其身亦猶是也【好呼到翻忘巫放翻}
上曰然朕與公輩宜戮力相輔庶免為人所笑也 青州有謀反者州縣逮捕支黨收繫滿獄詔殿中侍御史安喜崔仁師覆按之【曹魏時蘭臺遣御史二人居殿中伺察姦非遂稱殿中侍御史唐從七品下掌朝廷供奉之儀式安喜縣屬定州漢為盧奴安儉二縣地章帝改為安喜慕容垂改安喜為不連後魏復曰安喜後齊廢盧奴縣入安喜隋改曰鮮虞唐復曰安喜}
仁師至悉脱去杻械【去羌呂翻杻女九翻}
與飲食湯沐寛慰之止坐其魁首十餘人餘皆釋之還報敕使將往决之【此時敕使非宦官凡奉敕出使者則謂之敕使使疏吏翻}
大理少卿孫伏伽謂仁師曰足下平反者多【少始照翻反音飜}
人情誰不貪生恐見徒侣得免未肯甘心深為足下憂之【為于偽翻下不為竊為同}
仁師曰凡治獄當以平恕為本豈可自規免罪【規圖也治直之翻}
知其寃而不為伸邪【邪音耶}
萬一闇短誤有所縱以一身易十囚之死亦所願也伏伽慙而退及敕使至更訊諸囚皆曰崔公平恕事無枉濫請速就死無一人異辭者 上好騎射【好呼到翻下同騎奇寄翻}
孫伏伽諫以為天子居則九門【大門九重人主之門亦曰九重所謂禁衛九重虎豹九關皆言九門也}
行則警蹕非欲苟自尊嚴乃為社稷生民之計也陛下好自走馬射的以娛悦近臣此乃少年為諸王時所為【少詩沼翻}
非今日天子事業也既非所以安養聖躬又非所以儀刑後世臣竊為陛下不取上悦未幾以伏伽為諫議大夫【幾居豈翻 考異曰韓琬御史臺記伏伽武德中自萬年主簿上疏極諫太宗怒命引出斬之伏伽曰臣寧與關龍逢遊于地下不願事陛下太宗曰朕試卿耳卿能若是朕何憂杜稷命授之三品宰臣曰伏伽匡陛下之過自主簿授之三品彰陛下之過深矣請授之五品遂拜為諫議大夫按高祖實録武德元年伏伽自萬年縣法曹上書高祖詔授治書侍御史御史臺記誤也今據魏徵故事}
隋世選人十一月集至春而罷人患其期促至是吏部侍郎觀城劉林甫【觀縣古之觀國國語注曰夏啟子太康之弟所封也觀縣漢屬東郡光武改曰衛縣晉魏屬頓丘郡曰衛國縣隋開皇六年改曰觀城縣屬魏州唐屬澶州選須絹翻下同觀古玩翻}
奏四時聽選隨闕注擬人以為便唐初士大夫以亂離之後不樂仕進官員不充省符下諸州差人赴選州府及詔使【樂音洛下遐嫁翻使疏吏翻詔使即前所謂敕使}
多以赤牒補官至是盡省之勒赴省選集者七千餘人林甫隨才銓叙各得其所時人稱之詔以關中米貴始分人於洛州選上謂房玄齡曰官在得人不在員多命玄齡併省留文武總六百四十三員 隋祕書監晉陵劉子翼【晉陵縣帶常州}
有學行性剛直朋友有過常面責之李百藥常稱劉四雖復罵人【劉子翼第四唐人多以第行相呼學行下孟翻復扶又翻}
人終不恨是歲有詔徵之辭以母老不至鄃令裴仁軌【鄃縣漢晉屬清河郡中廢隋開皇十六年屬置貝州鄃音輸}
私役門

夫上怒欲斬之殿中侍御史長安李乾祐諫曰法者陛下所與天下共也非陛下所獨有也今仁軌坐輕罪而抵極刑臣恐人無所措手足上悦免仁軌死以乾祐為侍御史【唐制殿中侍御史從七品下侍御史從六品下}
上嘗語及關中山東人意有同異殿中侍御史義豐張行成跪奏曰【義豐漢中山安國縣隋開皇六年改曰義豐屬定州}
天子以四海為家不當有東西之異恐示人以隘上善其言厚賜之自是每有大政常使預議 初突厥既彊敕勒諸部分散有薛延陁迴紇都播骨科幹多濫葛同羅僕固拔野古思結渾斛薛結阿跌契苾白霫等十五部皆居磧北風俗大抵與突厥同【厥九勿翻敕勒即鐵勒也薛延陁先與薛種雜居後滅延陁部有之號薛延陁姓一利咥氏回紇先曰袁紇亦曰烏護曰烏紇至隋曰韋紇後稱回紇姓藥葛羅氏居薛延陁北娑陵水上距長安七千里都播亦曰都波其地北瀕小海西堅昆南回紇骨利幹居瀚海北多濫葛亦曰多覽葛在薛延陁東瀕同羅水同羅在薛延陁北多濫葛之東距長安七千里而贏僕固亦曰僕骨在多濫葛之東地最北拔野古一曰拔野固或為拔曳固漫散磧北地千里直僕固鄰于靺思結在延陁故牙渾在諸部最南斛薛居多濫葛北奚結在同羅北阿跌一曰訶跌或為跌契苾一曰契苾羽在焉耆西北鷹娑川多濫葛之南白霫居鮮卑故地直京師東北五千里與同羅僕固接避薛延陁保奥支水令陘山斛薛之下結之上當有奚字紇音鶻跌徒結翻契欺紇翻苾毗必翻又蒲結翻霤似入翻磧七迹翻 考異曰舊書敕勒作鐵勒新書云即元魏時高車或曰敕勒訛為鐵勒今從新書舊書多濫葛作多覽葛又作多臘葛今從實録唐統紀又舊書僕固或作僕骨按胡語難明以中國字寫之故訛謬不壹今從陳子昂集及僕固懷恩傳}
薛延陁於諸部為最彊西突厥曷薩那可汗方彊敕勒諸部皆臣之曷薩那徵税無度諸部皆怨曷薩那誅其渠帥百餘人敕勒相帥叛之【薛桑葛翻可從刋入聲汗音寒渠帥所類翻相帥讀曰率}
共推契苾哥楞為易勿真莫賀可汗居貪于山北【楞盧登翻貪于山新書作貪汚山}
又以薛延陁乙失鉢為也咥小可汗【咥徒結翻}
居燕末山北【燕因肩翻}
及射匱可汗兵復振【復扶又翻}
薛延陁契苾二部並去可汗之號以臣之【此上皆序隋時事去羌呂翻}
回紇等六部在鬱督軍山者東屬始畢可汗【鬱督軍山在大漠外直長安西北六千里}
統葉護可汗勢衰乙失鉢之孫夷男帥部落七萬餘家附于頡利可汗【帥讀曰率下同頡奚結翻 考異曰舊鐵勒傳云貞觀二年葉護可汗死其國大亂夷異始附于頡利按突厥傳元年薛延陁已叛頡利擊走其欲谷設安得二年始附頡利乎}
頡利政亂薛延陁與回紇拔野古等相帥叛之頡利遣其兄子欲谷設將十萬騎討之【新書阿史那社爾傳以欲谷設為頡利子}
回紇酋長菩薩將五千騎與戰於馬鬛山大破之【將即亮翻騎奇寄翻酋慈由翻長知兩翻菩薄乎翻薩桑葛翻}
欲谷設走菩薩追至天山部衆多為所虜回紇由是大振薛延陁又破其四設【突厥號典兵者為設四設四部帥之典兵者也}
頡利不能制頡利益衰國人離散會大雪平地數尺羊馬多死民大饑頡利恐唐乘其弊引兵入朔州境上揚言會獵實設備焉鴻臚卿鄭元璹使突厥還【周有大行人之官秦為典客漢景帝曰大行武帝曰大鴻臚梁置十二卿鴻臚為冬卿去大字唐因之掌賓客及凶儀之事璹殊玉翻臚陵如翻}
言於上曰戎狄興衰專以羊馬為候今突厥民饑畜瘦【瘦許救翻下同}
此將亡之兆也不過三年上然之羣臣多勸上乘間擊突厥上曰新與人盟而背之不信【間古莧翻背蒲妹翻}
利人之災不仁乘人之危以取勝不武縱使其種落盡叛六畜無餘朕終不擊必待有罪然後討之【種章勇翻畜許救翻}
西突厥統葉護可汗【厥九勿翻可從刋入聲汗音寒 考異曰高祖實録止云葉護舊傳作統葉護今從之}
遣真珠統俟斤與高平王道立來【高平王道立使西突厥見上卷高祖武德八年俟渠之翻}
獻萬釘寶鈿金帶馬五千匹以迎公主頡利不欲中國與之和親數遣兵入寇【數所角翻}
又遣人謂統葉護曰汝迎唐公主要須經我國中過統葉護患之未成昏二年春正月辛亥右僕射長孫無忌罷【從無忌之請也考下文可見長知兩翻}
時有密表稱無忌權寵過盛者上以表示之曰朕於卿洞然無疑若各懷所聞而不言則君臣之意有不通又召百官謂之曰朕諸子皆幼視無忌如子非它人所能間也無忌自懼滿盈固求遜位皇后又力為之請【間古莧翻為于偽翻}
上乃許之以為開府儀同三司 置六司侍郎副六尚書【六司侍郎吏部正四品上餘皆正四品下}
并置左右司郎中各一人【左右司郎中從五品上尚書左丞勾吏戶禮十二司右丞管兵刑工十二司左右司郎中各掌副十二司之事以舉正稽違省署符目}
癸丑吐谷渾寇岷州都督李道彦擊走之【吐從暾入聲谷音浴}
丁巳徙漢王恪為蜀王衛王泰為越王楚王祐為燕王【燕因肩翻}
上問魏徵曰人主何為而明何為而暗對曰兼聽則明偏信則暗昔堯清問下民故有苗之惡得以上聞【書呂刑曰皇帝清問下民鰥寡有辭於苗上時掌翻}
舜明四目達四聰故共鯀驩兜不能蔽也【舜明目達聰而難任人故四凶不能逃其罪也共音恭}
秦二世偏信趙高以成望夷之禍【事見秦紀}
梁武帝偏信朱异以取臺城之辱【事見梁紀}
隋煬帝偏信虞世基以致彭城閣之變【事見隋煬帝紀及高祖武德元年}
是故人君兼聽廣納則貴臣不得擁蔽而下情得以上通也上曰善上謂黄門侍郎王珪曰開皇十四年大旱隋文帝不許賑給而令百姓就食山東比至末年【賑津忍翻比必利翻及也}
天下儲積可供五十年煬帝恃其富饒侈心無厭【厭於鹽翻}
卒亡天下【卒子恤翻下同}
但使倉廪之積足以備凶年其餘何用哉 二月上謂侍臣曰人言天子至尊無所畏憚朕則不然上畏皇天之監臨下憚羣臣之瞻仰兢兢業業猶恐不合天意未副人望魏徵曰此誠致治之要【治直吏翻下同}
願陛下慎終如始則善矣 上謂房玄齡等曰為政莫若至公昔諸葛亮竄廖立李嚴於南夷亮卒而立嚴皆悲泣有死者【事見七十二卷魏明帝青龍二年廖力救翻又力弔翻卒子恤翻}
非至公能如是乎又高熲為隋相公平識治體隋之興亡繫熲之存没【事見隋紀}
朕既慕前世之明君卿等不可不法前世之賢相也【相息亮翻}
三月戊寅朔日有食之 壬子大理少卿胡演進每月囚帳【少始照翻囚帳具每月禁繫罪囚之姓名猶今之禁歷也}
上命自今大辟【辟毗亦翻}
皆令中書門下四品已上【自二省長貳而下至諫議大夫也}
及尚書議之庶無寃濫既而引囚至岐州刺史鄭善果上謂胡演曰善果雖復有罪【復扶又翻}
官品不卑豈可使與諸囚為伍自今三品已上犯罪不須引過聽於朝堂俟進止【太極宫承天門左右有東西朝堂朝直遙翻下同}
關内旱饑民多賣子以接衣食己巳詔出御府金帛為贖之歸其父母【為于偽翻}
庚午詔以去歲霖雨今茲旱蝗赦天下詔書畧曰若使年穀豐稔天下乂安移災朕身以存萬國是所願也甘心無吝會所在有雨民大悦 夏四月己卯詔以隋末亂離因之饑饉暴骸滿野傷人心目宜令所在官司收瘞【瘞於計翻}
初突厥突利可汗建牙直幽州之北主東偏奚霫等數十部多叛突厥來降【厥九勿翻可從刋入聲汗音寒霤而立翻降戶江翻}
頡利可汗以其失衆責之及薛延陁回紇等敗欲谷設【頡奚結翻紇下没翻敗補邁翻}
頡利遣突利討之突利兵又敗輕騎奔還【騎奇寄翻還從宣翻}
頡利怒拘之十餘日而撻之突利由是怨隂欲叛頡利頡利數徵兵於突利【數所角翻}
突利不與表請入朝上謂侍臣曰曏者突厥之彊控弦百萬憑陵中夏【夏戶雅翻}
用是驕恣以失其民今自請入朝非困窮肯如是乎朕聞之且喜且懼何則突厥衰則邊境安矣故喜然朕或失道它日亦將如突厥能無懼乎卿曹宜不惜苦諫以輔朕之不逮也頡利發兵攻突利丁亥突利遣使來求救【頡奚結翻使疏吏翻下同}
上謀於大臣曰朕與突利為兄弟有急不可不救【結兄弟事見上卷高祖武德七年}
然頡利亦與之有盟【謂渭橋之盟也見上卷武德九年}
奈何兵部尚書杜如晦曰戎狄無信終當負約今不因其亂而取之後悔無及夫取亂侮亡【書仲虺之誥之辭夫音扶}
古之道也丙申契丹酋長帥其部落來降【契欺紇翻又音喫酋慈由翻長知兩翻降戶江翻下同}
頡利遣使請以梁師都易契丹上謂使者曰契丹與突厥異類今來歸附何故索之【索山客翻}
師都中國之人盜我土地暴我百姓突厥受而庇之我興兵致討輒來救之彼如魚游釜中何患不為我有借使不得亦終不以降附之民易之也先是上知突厥政亂不能庇梁師都【先悉薦翻}
以書諭之師都不從上遣夏州都督長史劉旻司馬劉蘭成圖之【夏戶雅翻長知兩翻}
旻等數遣輕騎踐其禾稼多縱反間離其君臣其國漸虚降者相屬【數所角翻騎奇寄翻踐慈演翻間古莧翻屬之欲翻}
其名將李正寶等謀執師都事洩來奔【將即亮翻洩息列翻}
由是上下益相疑旻等知可取上表請兵【上時掌翻}
上遣右衛大將軍柴紹殿中少監薛萬均擊之【少始照翻}
又遣旻等據朔方東城以逼之【克東城見一百九十卷武德五年}
師都引突厥兵至城下劉蘭成偃旗卧鼓不出師都宵遁蘭成追擊破之突厥大發兵救師都柴紹等未至朔方數十里與突厥遇奮擊大破之遂圍朔方突厥不能救城中食盡壬寅師都從父弟洛仁殺師都以城降【梁師都隋大業末起兵至是而滅從才用翻}
以其地為夏州【夏戶雅翻}
太常少卿祖孝孫以梁陳之音多吳楚周齊之音多胡夷於是斟酌南北考以古聲【少始照翻}
作唐雅樂凡八十四調三十一曲十二和【律有七聲十二律凡八十四調隋有皇夏十四曲孝孫制十二和以法天之成數凡三十一曲十二和者一曰豫和二曰順和三曰永和四曰肅和五曰雍和六曰夀和七曰舒和八曰太和九曰昭和十曰休和十一曰正和十二曰承和調徒弔翻和如字}
詔協律郎張文收與孝孫同修定【漢協律都尉佩二千石印綬唐協律郎正八品上屬太常寺}
六月乙酉孝孫等奏新樂上曰禮樂者蓋聖人緣情以設教耳治之隆替豈由於此【治直吏翻}
御史大夫杜淹曰齊之將亡作伴侣曲【北齊之時陽俊之多作六言歌辭淫蕩而拙世俗流傳名為陽五伴侣}
陳之將亡作玉樹後庭花【杜佑曰玉樹後庭花堂堂黄鸝留金釵兩鬢垂並陳後主所造恒與宫中女學士及朝臣唱和為詩太樂令何胥採其尤輕艷者為此曲}
其聲哀思【思相吏翻}
行路聞之皆悲泣何得言治之隆替不在樂也上曰不然夫樂能感人故樂者聞之則喜【夫音扶故樂音洛}
憂者聞之則悲悲喜在人心非由樂也將亡之政民必愁苦故聞樂而悲耳今二曲具存朕為公奏之【為于偽翻}
公豈悲乎右丞魏徵曰古人稱禮云禮云玉帛云乎哉樂云樂云鍾鼓云乎哉【論語載孔子之言}
樂誠在人和不在聲音也

臣光曰臣聞垂能目制方圓心度曲直【垂古之巧人度徒洛翻}
然不能以教人其所以教人者必規矩而已矣聖人不勉而中【中竹仲翻}
不思而得然不能以授人其所以授人者必禮樂而已矣禮者聖人之所履也樂者聖人之所樂也聖人履中正而樂和平【所樂音洛下所樂哀樂同}
又思與四海共之百世傳之於是乎作禮樂焉故工人執垂之規矩而施之器是亦垂之功巳王者執五帝三王之禮樂而施之世是亦五帝三王之治已【治直吏翻下同}
五帝三王其違世已久後之人見其禮知其所履聞其樂知其所樂炳然若猶存於世焉此非禮樂之功邪【所樂音洛邪音耶}
夫禮樂有本有文【夫音扶}
中和者本也容聲者末也二者不可偏廢先王守禮樂之本未嘗須臾去於心行禮樂之文未嘗須臾遠於身興於閨門著於朝廷被於鄉遂比隣【遠於願翻被皮義翻朝直遥翻比毗至翻又音毗}
達於諸侯流於四海自祭祀軍旅至於飲食起居未嘗不在禮樂之中如此數十百年然後治化周浹鳳凰來儀也【浹即協翻}
苟無其本而徒有其末一日行之而百日捨之求以移風易俗誠亦難矣是以漢武帝置協律歌天瑞非不美也不能免哀痛之詔【見本紀}
王莽建羲和考律呂非不精也不能救漸臺之禍【王莽令劉歆考定律呂羲和掌之班固取以志律歷漸臺事見漢淮陽王紀漸子亷翻}
晉武制笛尺調金石非不詳也不能弭平陽之災【晉武帝使荀勗定鍾律平陽之災謂懷愍二帝蒙塵也}
梁武帝立四器調八音非不察也不能免臺城之辱【四器謂制四通也事見一百四十五卷天監元年臺城之辱見一百六十二卷太清三年}
然則韶夏濩武之音具存於世【舜樂曰韶禹樂曰夏湯樂曰濩周武王樂曰武夏戶雅翻濩戶故翻}
苟其餘不足以稱之【稱尺證翻}
曾不能化一夫况四海乎是猶執垂之規矩而無工與材坐而待器之成終不可得也况齊陳淫昏之主亡國之音蹔奏於庭烏能變一世之哀樂乎【蹔與暫同樂音洛}
而太宗遽云治之隆替不由於樂何發言之易【易以䜴翻}
而果於非聖人也如此夫禮非威儀之謂也【夫音扶}
然無威儀則禮不可得而行矣樂非聲音之謂也然無聲音則樂不可得而見矣譬諸山取其一土一石而謂之山則不可然土石皆去山於何在哉故曰無本不立無文不行【記禮器之言}
奈何以齊陳之音不驗於今世而謂樂無益於治亂何異暏拳石而輕泰山乎必若所言則是五帝三王之樂皆妄也君子於其不知蓋闕如也【論語載孔子之言}
惜哉

戊子上謂侍臣曰朕觀隋煬帝集文辭奥博亦知是堯舜而非桀紂然行事何其反也魏徵對曰人君雖聖哲猶當虚已以受人故智者獻其謀勇者竭其力煬帝恃其俊才驕矜自用故口誦堯舜之言而身為桀紂之行【行下孟翻}
曾不自知以至覆亡也上曰前事不遠吾屬之師也 畿内有蝗辛卯上入苑中【出玄武門北入禁苑}
見蝗掇數枚【掇丁活翻又陟劣翻拾取也}
祝之曰民以穀為命而汝食之寧食吾之肺腸舉手欲吞之左右諫曰惡物或成疾上曰朕為民受災【為于偽翻}
何疾之避遂吞之是歲蝗不為災 上曰朕每臨朝欲發一言未嘗不三思【朝直遙翻三息暫翻又如字}
恐為民害是以不多言給事中知起居事杜正倫曰臣職在記言【古者有左右史天子言則左史書之動則右史書之隋始置起居舍人貞觀二年省起居舍人移其職於門下省置起居郎二員以其它官兼者謂之知起居注知起居事}
陛下之失臣必書之豈徒有害於今亦恐貽譏於後上悦賜帛二百段上曰梁武帝君臣惟談苦空【言所談者惟苦行空寂也}
侯景之亂百官不能乘馬元帝為周師所圍猶講老子百官戎服以聽【事見一百六十五卷梁元帝承聖三年}
此深足為戒朕所好者【好呼到翻}
唯堯舜周孔之道以為如鳥有翼如魚有水失之則死不可暫無耳 以辰州刺史裴䖍通隋煬帝故人特蒙寵任而身為弑逆【事見一百八十五卷高祖武德元年按通鑑紀事各為段凡改段處率空一字别為一節此段頭既空字以字之上合有上字文乃明}
雖時移事變屢更赦令【更工衡翻}
幸免族夷不可猶使牧民乃下詔除名流驩州【貞觀元年改德州日南郡曰驩州}
䖍通常言身除隋室以啟大唐自以為功頗有觖望之色【觖窺瑞翻又於决翻怨望也}
及得罪怨憤而死 秋七月詔宇文化及之黨萊州刺史牛方裕絳州刺史薛世良廣州都督長史唐奉義隋武牙郎將元禮並除名徙邊【長知兩翻武牙郎將即虎牙郎將唐避諱改虎曰武將即亮翻}
上謂侍臣曰古語有之赦者小人之幸君子之不幸一歲再赦善人喑啞【喑于今翻啞烏下翻}
夫養稂莠者害嘉穀【夫音扶稂魯當翻莠與久翻稂莠皆惡草害稼}
赦有罪者賊良民故朕即位以來不欲數赦【數所角翻}
恐小人恃之輕犯憲章故也

資治通鑑卷一百九十二














































































































































