










 


 
 


 

  
  
  
  
  





  
  
  
  
  
 
  

  

  
  
  



  

 
 

  
   




  

  
  


    資治通鑑卷七十    宋 司馬光 撰

  胡三省 音註

  魏紀二【起昭陽單閼盡強圉協洽凡五年】

  世祖文皇帝下

  黃初四年春正月曹真使張郃擊破吳兵遂奪據江陵中洲【去年吳將孫盛據中洲郃古合翻又曷閤翻】 二月諸葛亮至永安【水經注蜀先主為吳所敗退屯白帝改白帝為永安巴東郡治也】 曹仁以步騎數萬向濡須先揚聲欲東攻羨溪【羨溪在濡須東而蜀本注以為沙羨誤矣杜佑曰羨溪在濡須東三十里】朱桓分兵赴之【去年吳王以朱桓為濡須督】既行仁以大軍徑進桓聞之追還羨溪兵兵未到而仁奄至時桓手下及所部兵在者纔五千人諸將業業各有懼心【孔安國曰業業危懼意】桓喻之曰凡兩軍交對勝負在將【將即亮翻】不在衆寡諸君聞曹仁用兵行師孰與桓邪兵法所以稱客倍而主人半者謂俱在平原無城隍之守又謂士卒勇怯齊等故耳今仁既非智勇加其士卒甚怯又千里步涉人馬罷困【罷讀曰疲】桓與諸君共據高城南臨大江北背山陵【背蒲妹翻】以逸待勞為主制客此百戰百勝之埶雖曹丕自來尚不足憂况仁等邪桓乃偃旗豉外示虚弱以誘致仁【誘音酉】仁遣其子泰攻濡須城分遣將軍常雕王雙等乘油船别襲中洲【油船蓋以牛皮為之外施油以扞水】中洲者桓部曲妻子所在也蔣濟曰賊據西岸列船上流而兵入洲中是為自内地獄【内與納同】危亡之道也仁不從自將萬人留槖皋【槖皋在廬江居巢縣春秋會吳于槖皋即其地令曰柘皋在濡須北余按班志槖皋縣屬九江郡孟康音拓姑杜預曰槖皋在淮南逡遒縣東南陸德明曰槖章夜翻又音託】為泰等後援桓遣别將擊雕等而身自拒泰泰燒營退桓遂斬常雕生虜王雙臨陳殺溺死者千餘人【陳讀曰陣】初呂蒙病篤吳王問曰卿如不起誰可代者蒙對曰朱然膽守有餘愚以為可任朱然者九真太守朱治姊子也本姓施氏治養以為子時為昭武將軍【昭武將軍吳所置也】蒙卒吳王假然節鎮江陵及曹真等圍江陵破孫盛吳王遣諸葛瑾等將兵往解圍【瑾渠吝翻】夏候尚擊却之江陵中外斷絶城中兵多腫病堪戰者裁五千人真等起土山鑿地道立樓櫓臨城弓矢雨注將士皆失色然晏如無恐意【呂蒙所謂膽守於此見之】方厲吏士伺間隙【伺相吏翻間古莧翻】攻破魏兩屯魏兵圍然凡六月江陵令姚泰領兵備城北門見外兵盛城中人少【少詩沼翻】穀食且盡懼不濟謀為内應然覺而殺之時江水淺陿【陿與狹同】夏侯尚欲乘船將步騎入渚中安屯【渚洲也即江陵之中洲也】作浮橋南北往來議者多以為城必可拔董昭上疏曰武皇帝智勇過人而用兵畏敵不敢輕之若此也【言行兵不敢履危道】夫兵好進惡退【好呼到翻惡烏路翻】常然之數平地無險猶尚艱難就當深入還道宜利兵有進退不可如意今屯渚中至深也浮橋而濟至危也一道而行至陿也三者兵家所忌而今行之賊頻攻橋誤有漏失【謂橋或為敵所斷也】渚中精銳非魏之有將轉化為吳矣臣私慼之忘寢與食【慼憂也】而議者怡然不以為憂豈不惑哉加江水向長【長知兩翻】一旦暴增何以防禦就不破賊尚當自完奈何乘危不以為懼惟陛下察之帝即詔尚等促出吳人兩頭並前魏兵一道引去不時得泄【泄去也】僅而獲濟吳將潘章已作荻筏欲以燒浮橋會尚退而止後旬日江水大漲帝謂董昭曰君論此事何其審也會天大疫帝悉召諸軍還三月丙申車駕還洛陽初帝問賈詡曰吾欲伐不從命以一天下吳蜀何先對曰攻取者先兵權建本者尚德化陛下應期受禪撫臨率土若綏之以文德而俟其變則平之不難矣吳蜀雖蕞爾小國依山阻水劉備有雄才諸葛亮善治國【蕞徂外翻治直之翻】孫權識虚實陸議見兵埶【陸議即陸遜遜傳云遜本名議】據險守要汎舟江湖皆難卒謀也【據險守要謂蜀汎舟江湖謂吳卒讀曰猝】用兵之道先勝後戰量敵論將【量音良將即亮翻】故舉無遺策臣竊料羣臣無備權對雖以天威臨之未見萬全之埶也昔舜舞干戚而有苗服【舜誕敷文德舞干羽于兩階七旬有苗格】臣以為當今宜先文後武帝不納軍竟無功 丁未陳忠侯曹仁卒 初黃元為諸葛亮所不善聞漢主疾病懼有後患故舉郡反燒臨卭城【臨卭縣漢屬蜀郡蜀既分置漢嘉郡則此時當屬漢嘉卭渠容翻】時亮東行省疾【省悉景翻】成都單虚元益無所憚益州治中從事楊洪啟太子遣將軍陳曶鄭綽討元【曶呼骨翻】衆議以為元若不能圍成都當由越嶲據南中【南中漢益州永昌二郡之地】洪曰元素性凶暴無他恩信何能辦此不過乘水東下冀主上平安面縛歸死如其有異奔吳求活耳但勑曶綽於南安峽口邀遮即便得矣元軍敗果順江東下曶綽生獲斬之【此順蜀青衣水東下也水經注青衣水出青衣縣西蒙山東至蜀郡臨卭縣與沫水合又東至犍為南安縣入于江所謂南安峽口也】 漢主病篤命丞相亮輔太子以尚書令李嚴為副漢主謂亮曰君才十倍曹丕必能安國終定大事若嗣子可輔輔之如其不才君可自取【自古託孤之主無如昭烈之明白洞達者】亮涕泣曰臣敢不竭股肱之力効忠貞之節繼之以死【用晉荀息荅獻公語意】漢主又為詔勑太子曰人五十不稱夭【夭於兆翻短折曰夭】吾年已六十有餘何所復恨【復扶又翻】但以卿兄弟為念耳勉之勉之勿以惡小而為之勿以善小而不為惟賢惟德可以服人汝父德薄不足効也【自漢以下所以詔勑嗣君者能有此言否】汝與丞相從事事之如父夏四月癸巳漢主殂於永安【年六十三】諡曰昭烈【諡法昭德有勞曰昭有功安民曰烈】丞相亮奉喪還成都以李嚴為中都護留鎮永安五月太子禪即位時年十七【蜀後主諱禪字公嗣】尊皇后曰太皇后大赦改元建興封丞相亮為武鄉侯領益州牧政事無巨細咸决於亮亮乃約官職修法制【以先主孔明君臣之相得而約官職修法制乃行於輔後主之時此易之戒浚恒也】發教與羣下曰夫參署者集衆思廣忠益也【參署謂所行之事參其同異署而行之也】若遠小嫌難相違覆曠闕損矣【違異也覆審也難於違異難於覆審則事有曠闕損矣遠于願翻】違覆而得中猶棄敝蹻而獲珠玉【蹻訖約翻屐也草履也】然人心苦不能盡惟徐元直處兹不惑又董幼宰參署七年【徐庶字元直董和字幼宰處昌呂翻】事有不至至于十反來相啟告【此所謂相違覆也】苟能慕元直之十一幼宰之勤渠有忠於國則亮可以少過矣【少詩沼翻】又曰昔初交州平【亮躬耕隴畝與崔州平徐庶等友善州平崔烈子均之弟也】屢聞得失後交元直勤見啟誨前參事於幼宰每言則盡後從事於偉度數有諫止【數所角翻】雖資性鄙暗不能悉納然與此四子終始好合【好呼到翻】亦足以明其不疑於直言也偉度者亮主簿義陽胡濟也亮嘗自校簿書主簿楊顒直入【顒魚容翻】諫曰為治有體【治直吏翻】上下不可相侵請為明公以作家譬之【為于偽翻】今有人使奴執耕稼婢典炊爨雞主司晨犬主吠盜牛負重載【載才再翻】馬涉遠路私業無曠所求皆足雍容高枕【枕職任翻】飲食而已忽一旦盡欲以身親其役不復付任【復扶又翻】勞其體力為此碎務形疲神困終無一成豈其智之不如奴婢雞狗哉失為家主之法也是故古人稱坐而論道謂之王公作而行之謂之士大夫【周官考工記之言】故丙吉不問橫道死人而憂牛喘【丙吉相漢宣帝嘗出逢清道羣鬭者死傷横道吉過之不問前行逢人逐牛牛喘吐舌吉使騎吏問逐牛行幾里矣掾史謂丞相前後失問吉曰民鬭相殺傷長安令京兆尹職也方春少陽用事未可大熱恐牛近行用暑故喘此時氣失節有所傷害三公調和隂陽職當憂是以問之掾史乃服以吉知大體】陳平不肯知錢穀之數云自有主者【事見十三卷漢文帝元年】彼誠達於位分之體也【分扶問翻】今明公為治乃躬自校簿書流汗終日不亦勞乎亮謝之及顒卒亮垂泣三日 六月甲戍任城威王彰卒【諡法猛以彊果曰威服叛定功曰威】 甲申魏夀肅侯賈詡卒【魏夀亭名諡法剛德克就曰肅執心决斷曰肅】大水 吳賀齊襲蘄春虜太守晉宗以歸【蘄春縣漢屬江夏郡】

  【吳分立蘄春郡即蘄陽也東晉避諱改焉水經蘄水出江夏蘄春縣北山注云即蘄山也西南流逕蘄山又南對蘄陽會于大江亦謂之蘄河口據賀齊傳晉宗吳將也叛降魏還為蘄春太守齊襲而虜之】 初益州郡耆帥雍闓殺太守正昂因士燮以求附於吳【耆渠伊翻長也老也今嵊剡之間猶謂閭里之長曰耆帥所類翻雍於用翻姓也闓音開又可亥翻闓自交州道求附於吳正姓也秦有正先】又執太守成都張裔以與吳吳以闓為永昌太守永昌功曹呂凱府丞王伉【伉口浪翻】率吏士閉境拒守闓不能進使郡人孟獲誘扇諸夷【誘音酉】諸夷皆從之牂柯太守朱褒越雋夷王高定皆叛應闓【牂柯音臧哥嶲音髓】諸葛亮以新遭大喪皆撫而不討務農殖穀閉關息民【閉越嶲之靈關也】民安食足而後用之 秋八月丁卯以廷尉鍾繇為太尉治書執法高柔代為廷尉【漢宣帝幸宣室齋居决事令侍御史二人治書侍側後因别置謂之治書侍御史及魏又置治書執法掌奏劾而治書侍御史掌律令二官俱置及晉唯置治書侍御史四人治直之翻】是時三公無事又希與朝政【與讀曰預】柔上疏曰公輔之臣皆國之棟梁民所具瞻【詩曰赫赫師尹民具爾瞻】而置之三事不使知政【古者謂三公為三事詩曰三事大夫謂三公也】遂各偃息養高【偃息言偃卧以自安也】鮮有進納【鮮息淺翻】誠非朝廷崇用大臣之義大臣獻可替否之謂也【左傳齊晏子曰君所謂可而有否焉臣獻其否以成其可君所謂否而有可焉臣獻其可而去其否】古者刑政有疑輒議于槐棘之下【周禮朝士掌外朝之法而三槐三公位焉左九棘孤卿大夫位焉鄭注云樹棘以為位者取其赤心而外刺象以赤心三刺也槐之言懷也懷來人於此欲與之謀王制曰成獄辭史以獄成告于正正聽之正以獄成告于大司寇大司寇聽之于棘木之下大司寇以獄之成告于王王命三公參聽之】自今之後朝有疑議及刑獄大事宜數以咨訪三公【朝直遥翻下同數所角翻】三公朝朔朢之日又可特延入講論得失博盡事情庶有補起天聽光益大化帝嘉納焉 辛未帝校獵于滎陽遂東巡九月甲辰如許昌 漢尚書義陽鄧芝言於諸葛亮曰今主上幼弱初即尊位宜遣大使重申吳好【使疏吏翻下同申亦重也所以申固盟約也重直用翻好呼到翻下同】亮曰吾思之久矣未得其人耳今日始得之芝問其人為誰亮曰即使君也乃遣芝以中郎將修好於吳冬十月芝至吳時吳王猶未與魏絶狐疑不時見芝芝乃自表請見曰臣今來亦欲為吳非但為蜀也【為于偽翻】吳王見之曰孤誠願與蜀和親然恐蜀主幼弱國小埶偪為魏所乘不自保全耳芝對曰吳蜀二國四州之地【四州荆揚梁益也】大王命世之英諸葛亮亦一時之傑也蜀有重險之固【重險謂外有斜駱子午之險内有劍閣之險也重直龍翻】吳有三江之阻【韋昭曰三江吳淞江錢塘江浦陽江也吳地記云淞江東北行七十里得三江口東北入海為婁江東南入海為東江并淞江為三江】合此二長共為唇齒進可并兼天下退可鼎足而立此理之自然也大王今若委質於魏【質如字】魏必上望大王之入朝【朝直遥翻】下求太子之内侍若不從命則奉辭伐叛蜀亦順流見可而進如此江南之地非復大王之有也吳王默然良久曰君言是也遂絶魏專與漢連和 是歲漢主立妃張氏為皇后【后張飛之女也】

  五年春二月帝自許昌還洛陽 初平以來學道廢墜夏四月初立太學置博士依漢制設五經課試之法【博士課試之法始於漢武帝事見十九卷元朔五年平帝時歲課甲科四十人為郎中乙科二十人為太子舍人丙科四十人補文學掌故東都五經立十四博士皆以家法教授古文尚書毛詩穀梁左氏春秋雖不立學官然皆擢高第為講郎給事近署順帝增甲乙之科員各十人】 吳王使輔義中郎將吳郡張温聘于漢自是吳蜀信使不絶【使疏吏翻】時事所宜吳主常令陸遜語諸葛亮【語牛倨翻】又刻印置遜所王每與漢主及諸葛亮書常過示遜【過工禾翻】輕重可否有所不安每令改定以印封之【釋名曰印信也所以封物以為驗也亦曰因也封物相因付也】漢復遣鄧芝聘于吳【復扶又翻】吳主謂之曰若天下太平二主分治不亦樂乎【樂音洛】芝對曰天無二日土無二王【孟子載孔子之言】如并魏之後大王未深識天命君各茂其德臣各盡其忠將提枹鼓則戰爭方始耳【枹音膚】吳王大笑曰君之誠欵乃當爾邪 秋七月帝東巡如許昌帝欲大興軍伐吳侍中辛毗諫曰方今天下新定土廣民稀而欲用之臣誠未見其利也先帝屢起銳師臨江而旋今六軍不增於故而復修之此未易也【修之謂修怨也左傳曰將修先君之怨復扶又翻易以豉翻】今日之計莫若養民屯田十年然後用之則役不再舉矣帝曰如卿意更當以虜遺子孫邪【遺于季翻下同】對曰昔周文王以紂遺武王惟知時也帝不從留尚書僕射司馬懿鎮許昌八月為水軍親御龍舟循蔡潁浮淮如夀春【魏收地形志陳留扶溝縣有蔡河水經蔡河自陳留浚儀東南流而入於潁潁水出潁川陽城縣少室山東南流至新陽與蔡河合又東南至慎縣東南入于淮】九月至廣陵吳安東將軍徐盛建計植木衣葦為疑城假樓自石頭至于江乘【植木於内以蘆䕠遮其外為疑城假樓今淮甸諸郡城敵樓皆以蘆䕠遮護之江乘縣屬丹陽郡吳省為興農都尉治其地在建業東北衣於既翻】聨緜相接數百里一夕而成又大浮舟艦於江【艦戶黯翻】時江水盛長【長知兩翻】帝臨望歎曰魏雖有武騎千羣無所用之未可圖也【騎奇寄翻】帝御龍舟會暴風漂蕩幾至覆沒【幾居希翻】帝問羣臣權當自來否咸曰陛下親征權恐怖必舉國而應【怖普布翻】又不敢以大衆委之臣下必當自來劉曄曰彼謂陛下欲以萬乘之重牽已而超越江湖者在於别將【乘䋲證翻將即亮翻下同】必勒兵待事未有進退也大駕停住積日吳王不至帝乃旋師是時曹休表得降賊辭孫權已在濡須口【降戶江翻下同】中領軍衛臻曰【晉百官志曰漢建安四年魏武丞相府置中領軍文帝踐祚始置領軍將軍置長史司馬江左以後資重者為領軍將軍資輕者為中領軍沈約志曰領軍掌内軍漢武帝置中壘校尉掌北軍營光武省置北軍中候監五校營魏武為丞相相府自置領軍非漢官也文帝以領軍主五校中壘武衛三營晉武帝初省使中軍將軍羊祐統二衛前後左右驍騎七軍即領軍之任也祐遷復置北軍中候懷帝永嘉中又改曰中領軍】權恃長江未敢亢衡【亢與抗同】此必畏怖偽辭耳考核降者果守將所作也 吳張温少以俊才有盛名【少詩照翻】顧雍以為當今無輩諸葛亮亦重之温薦引同郡暨豔為選部尚書【暨居乙翻姓也葉夢得石林燕語曰元豐五年黃冕仲榜唱名有暨陶者主司初以洎音呼之三呼不應蘇子容時為試官神宗顧蘇蘇曰當以入聲呼之果出應上曰何以知為入聲蘇言三國志吳有暨豔陶恐其後遂問陶鄉貫曰崇安人上喜曰果吳人也漢置四曹尚書其一曰常侍曹主丞相御史公卿事光武改常侍為吏部曹主選舉祠祀靈帝以梁鵠為選部尚書魏復改選部為吏部吳蓋循東都之制】豔好為清議【好呼到翻】彈射百僚覈奏三署【三署謂五官左右三署郎也射而亦翻】率皆貶高就下降損數等其守故者十未能一其居位貪鄙志節汙卑者皆以為軍吏置營府以處之【處昌呂翻】多揚人闇昧之失以顯其謫【謫罰也】同郡陸遜遜弟瑁及侍御史朱據皆諫止之瑁與豔書曰夫聖人嘉善矜愚【論語子游曰君子嘉善而矜不能瑁音冒】忘過記功以成美化加今王業始建將一大統此乃漢高棄瑕録用之時也【謂棄其瑕玷而録其材用】若令善惡異流貴汝潁月旦之評【漢末汝南許劭與從兄靖俱有高名好共覈論鄉黨人物每月輒更其品題故汝南俗有月旦評】誠可以厲俗明教然恐未易行也【易以豉翻】宜遠模仲尼之汎愛【論語載孔子之言曰汎愛衆而親仁】近則郭泰之容濟【郭泰善人倫而不為危言覈論奨拔士人成名者甚衆而不絶左原賈淑之險惡所謂容濟也】庶有益於大道也據謂豔曰天下未定舉清厲濁足以沮勸【沮在呂翻】若一時貶黜懼有後咎豔皆不聽於是怨憤盈路爭言豔及選曹郎徐彪專用私情憎愛不由公理豔彪皆坐自殺【坐自殺謂賜死也】温素與豔彪同意亦坐斥還本郡以給厮吏【厮音斯賤也】卒於家始温方盛用事餘姚虞俊歎曰張惠恕才多智少【餘姚縣屬會稽郡在今越州上虞縣東張温字惠恕】華而不實怨之所聚有覆家之禍吾見其兆矣無幾何而敗【幾居豈翻】 冬十月帝還許昌 十一月戊申晦日有食之 鮮卑軻比能誘步度根兄抉羅韓殺之【誘音酉】步度根由是怨軻比能更相攻擊【更工衡翻】步度根部衆稍弱將其衆萬餘落保太原雁門是歲詣闕貢獻【步度根檀石槐之孫也】而軻比能衆遂強盛出擊東部大人素利護烏丸校尉田豫乘虚掎其後【掎魚豈翻】軻比能使别帥瑣奴拒豫【帥所類翻】豫擊破之軻比能由是貳數為邊寇幽并苦之【數所角翻】六年春二月詔以陳羣為鎮軍大將軍隨車駕董督衆軍録行尚書事司馬懿為撫軍大將軍留許昌督後臺文書【魏晉之制大將軍不開府者品秩第二其禄與特進同置長史司馬主簿諸曹官屬行尚書謂尚書之隨駕者後臺謂尚書臺留許昌者也】三月帝行如召陵通討虜渠【召陵縣漢屬汝南郡晉志屬潁川郡賢曰召陵故城在今豫州郾城縣東通討虜渠以伐吳也召讀曰邵】乙巳還許昌 并州刺史梁習討軻比能大破之 漢諸葛亮率衆討雍闓參軍馬謖送之數十里【謖所六翻】亮曰雖共謀之歷年今可更惠良規謖曰南中恃其險遠不服久矣雖今日破之明日復反耳【復扶又翻下同】今公方傾國北伐以事彊賊彼知官埶内虚【漢俗謂天子為縣官亦謂為國家官勢猶言國埶也】其叛亦速若殄盡遺類以除後患既非仁者之情且又不可倉卒也【卒讀曰猝】夫用兵之道攻心為上攻城為下心戰為上兵戰為下願公服其心而已【此馬謖所以為善論軍計也】亮納其言謖良之弟也 辛未帝以舟師復征吳羣臣大議宫正鮑勛諫曰【據勛傳宫正即御史中丞也】王師屢征而未有所克者蓋以吳蜀唇齒相依憑阻山水有難拔之勢故也往年龍舟飄蕩隔在南㟁【事見上】聖躬蹈危臣下破膽此時宗廟幾至傾覆【幾居希翻】為百世之戒今又勞兵襲遠日費千金【兵法曰興師十萬日費千金】中國虚耗令黠虜玩威【國語祭公謀父曰先王耀德不觀兵夫兵戢而時動動則威觀則玩玩則無震黠下八翻】臣竊以為不可帝怒左遷勛為治書執法勛信之子也【鮑信從武帝戰死】夏五月戊申帝如譙 吳丞相北海孫劭卒初吳當置丞相衆議歸張昭吳王曰方今多事職大者責重非所以優之也及劭卒百僚復舉昭吳王曰孤豈為子布有愛乎【為于偽翻】領丞相事煩而此公性剛所言不從怨咎將興非所以益之也六月以太常顧雍為丞相平尚書事雍為人寡言舉動時當【當丁浪翻】吳王嘗歎曰顧君不言言必有中至飲宴歡樂之際【中竹仲翻樂音洛下同】左右恐有酒失而雍必見之是以不敢肆情吳王亦曰顧公在坐【坐徂卜翻】使人不樂其見憚如此初領尚書令封陽遂鄉侯拜侯還寺【寺官舍也】而家人不知後聞乃驚及為相其所選用文武將吏各隨能所任心無適莫【適音的心之所主為適心之所否為莫】時訪逮民間及政職所宜輒密以聞若見納用則歸之於上不用終不宣泄【宣明也布也泄漏也】吳王以此重之然於公朝有所陳及辭色雖順而所執者正軍國得失自非面見口未嘗言王常令中書郎【中書郎魏曰通事郎晉為中書侍郎】詣雍有所咨訪若合雍意事可施行即相與反覆究而論之為設酒食【為于偽翻下同】如不合意雍即正色改容默然不言無所施設郎退告王王曰顧公歡悦是事合宜也其不言者是事未平也孤當重思之【重直用翻】江邊諸將各欲立功自效多陳便宜有所掩襲王以訪雍雍曰臣聞兵法戒於小利此等所陳欲邀功名而為其身非為國也【為于偽翻】陛下宜禁制苟不足以曜威損敵所不宜聽也王從之 利成郡兵蔡方等反【利成縣漢屬東海郡魏武始分置利成縣】殺太守徐質推郡人唐咨為主詔屯騎校尉任福等討平之【任音壬】咨自海道亡入吳吳人以為將軍 秋七月立皇子鑒為東武陽王漢諸葛亮至南中所在戰捷亮由越嶲入【嶲音髓】斬雍闓及高定使庲降督益州李恢由益州入【裴松之曰訊之蜀人云庲降地名去蜀三千餘里時未有寧州號為南中立此職以總攝之晉泰始中始分為寧州平夷縣屬牂柯郡余據蜀志庲降督住平夷蓋僑治非庲降之本地也至馬忠為庲降督乃自平夷移住建寧味縣後遂為寧州治所】門下督巴西馬忠由牂柯入擊破諸縣復與亮合【牂柯音臧哥復如字又扶又翻】孟獲收闓餘衆以拒亮獲素為夷漢所服亮募生致之既得使觀於營陳之間【陳讀曰陣下同】問曰此軍何如獲曰向者不知虚實故敗今蒙賜觀營陳若祇如此即定易勝耳【易以豉翻下同】亮笑縱使更戰七縱七禽而亮猶遣獲獲止不去曰公天威也南人不復反矣【復扶又翻】亮遂至滇地【滇池縣屬益州郡池周回二百餘里水源深廣而末更淺狹有似倒流故謂之滇池滇音顛】益州永昌牂柯越嶲四郡皆平亮即其渠率而用之【即就也渠大也渠率大率也率與帥同音所類翻】或以諫亮亮曰若留外人則當留兵兵留則無所食一不易也加夷新傷破父兄死喪留外人而無兵者必成禍患二不易也又夷累有廢殺之罪【喪息浪翻易以豉翻下同殺讀曰弑殺其郡將是亦弑也】自嫌釁重若留外人終不相信三不易也今吾欲使不留兵不運糧而綱紀粗定夷漢粗安故耳【粗坐五翻】亮於是悉收其俊傑孟獲等以為官屬出其金銀丹漆耕牛戰馬以給軍國之用自是終亮之世夷不復反【復扶又翻】 八月帝以舟師自譙循渦入淮【水經隂溝水出河南陽武縣蒗蕩渠東南至沛為渦水渦水東逕譙郡又東南至下邳淮隂縣入于淮杜佑曰亳州治譙縣有渦水渦音戈】尚書蔣濟表言水道難通帝不從冬十月如廣陵故城【廣陵故城謂之蕪城今其處不可考】臨江觀兵戎卒十餘萬旌旗數百里有渡江之志吳人嚴兵固守時天寒氷舟不得入江帝見波濤洶涌【洶許拱翻】歎曰嗟乎固天所以限南北也遂歸孫韶遣將高夀等率敢死之士五百人於逕路夜要帝【要一遥翻】帝大驚夀等獲副車羽蓋以還【還從宣翻又如字下同】於是戰船數千皆滯不得行議者欲就留兵屯田蔣濟以為東近湖北臨淮若水盛時賊易為寇不可安屯【近其靳翻易以䜴翻】帝從之車駕即發還到精湖【據蔣濟傳精湖在山陽山陽在下邳淮隂縣界今楚州山陽縣】水稍盡盡留船付濟船連延在數百里中濟更鑿地作四五道蹴船令聚豫作土豚【日録作土塍廣韻作土坉注云以草裹土築城及填水也】遏斷湖水【斷丁管翻】皆引後船一時開遏入淮中乃得還 十一月東武陽王鑒薨 十二月吳番陽賊彭綺攻沒郡縣衆數萬人【番蒲河翻】七年春正月壬子帝還洛陽謂蔣濟曰事不可不曉吾前次謂分半燒船於山陽湖中【謂到精湖水盡船不得過欲分半船也宋白曰楚州山陽縣本射陽縣地晉義熙置山陽郡及山陽縣以境内有地名山陽因以為名戴延之西征記山陽津名】卿於後致之略與吾俱至譙又每得所陳實入吾意自今討賊計畫善思論之 漢丞相亮欲出軍漢中前將軍李嚴當知後事移屯江州留護軍陳到駐永安而統屬於嚴 吳陸遜以所在少穀【少詩沼翻】表令諸將增廣農畝吳主報曰甚善令孤父子親受田車中八牛以為四耦【耒廣五寸為伐二伐為耦漢制后稷始甽田以二耜為耦注云并兩耜而耕也】雖未及古人亦欲令與衆均等其勞也 帝之為太子也郭夫人弟有罪魏郡西部都尉鮑勛治之【漢獻帝建安十八年魏武分魏郡置東西部都尉後以東部都尉立陽平郡西部都尉立廣平郡謂之三魏皆屬司州治直之翻下同】太子請不能得由是恨勛及即位勛數直諫【數所角翻】帝益忿之帝伐吳還屯陳留界勛為治書執法太守孫邕見出過勛【見賢遍翻過古禾翻】時營壘未成但立標埓【標表也埒說文曰庳垣也又封道曰埒埓龍輟翻】邕邪行不從正道軍營令史劉曜欲推之勛以塹壘未成解止不舉【塹七艶翻】帝聞之詔曰勛指鹿作馬【用趙高事】收付廷尉廷尉法議正刑五歲【法議引法而議也正結正也五歲刑髠鉗為城旦春】三官駮依律罰金二斤【三官廷尉正監平也駮北角翻】帝大怒曰勛無活分【分扶問翻】而汝等欲縱之收三官已下付刺姦當令十鼠同穴鍾繇華歆陳羣辛毗高柔衛臻等並表勛父信有功於太祖【事見五十九卷漢獻帝初平元年六十卷二年三年華戶化翻】求請勛罪帝不許高柔固執不從詔命帝怒甚召柔詣臺【召詣尚書臺也】遣使者承指至廷尉誅勛勛死乃遣柔還寺票騎將軍都陽侯曹洪家富而性吝嗇【票匹妙翻】帝在東宫嘗從洪貸絹百匹不稱意恨之遂以舍客犯法下獄當死【稱尺證翻下遐稼翻】羣臣並救莫能得卞太后責怒帝曰梁沛之間非子亷無有今日【曹洪字子亷洪脫武帝事見五十九卷初平元年】又謂郭后曰令曹洪今日死吾明日勑帝廢后矣於是郭后泣涕屢請乃得免官削爵土 初郭后無子帝使母養平原王叡以叡母甄夫人被誅【事見上卷元年】故未建為嗣叡事后甚謹后亦愛之帝與叡獵見子母鹿帝親射殺其母命叡射其子【射而亦翻】叡泣曰陛下已殺其母臣不忍復殺其子【復扶又翻】帝即放弓矢為之惻然【為于偽翻】夏五月帝疾篤乃立叡為太子丙辰召中軍大將軍曹真鎮軍大將軍陳羣撫軍大將軍司馬懿【沈約志曰中軍將軍漢武帝以公孫敖為之時為雜號鎮軍撫軍皆始於此中鎮撫三號比四鎮晉志諸大將軍開府位從公者為武官公皆著武冠平上黑幘】並受遺詔輔政丁巳帝殂【年四十通鑑書法天子奄有四海者書崩分治者書殂惟東晉諸帝以先嘗混一書崩說文曰殂往死也】

  陳夀評曰文帝天資文藻下筆成章博聞彊識才藝兼該若加之曠大之度勵以公平之誠邁志存道克廣德心則古之賢主何遠之有哉

  太子即皇帝位尊皇太后曰太皇太后皇后曰皇太后初明帝在東宫不交朝臣不問政事惟潛思書籍【思相吏翻】即位之後羣下想聞風采居數日獨見侍中劉曄語盡日衆人側聽曄既出問何如曰秦始皇漢孝武之儔才具微不及耳帝初蒞政陳羣上疏曰夫臣下雷同是非相蔽國之大患也若不和睦則有讐黨【左傳晉卻芮曰有黨必有讐】有讐黨則毁譽無端毁譽無端則真偽失實【譽音余】此皆不可不深察也 癸未追諡甄夫人曰文昭皇后【甄之人翻】壬辰立皇弟蕤為陽平王【蕤如佳翻】 六月戊寅葬文帝

  於首陽陵【葬於洛陽東北首陽山因以名陵】 吳王聞魏有大喪秋八月自將攻江夏郡太守文聘堅守【文聘時屯石陽祝穆曰魏初定荆州屯沔陽為重鎮晉立沔陽縣江夏郡自上昶移理焉今臨嶂山在漢陽軍西六十里晉沔陽縣治也意石陽即此地夏戶雅翻】朝議欲發兵救之【朝直遥翻】帝曰權習水戰所以敢下船陸攻者冀掩不備也今已與聘相拒夫攻守埶倍終不敢久也先是朝廷遣治書侍御史荀禹慰勞邊方【先悉薦翻治直之翻勞力到翻】禹到江是發所經縣兵及所從步騎千人乘山舉火【乘登也】吳王遁走 辛巳立皇子冏為清河王吳左將軍諸葛瑾等寇襄陽司馬懿擊破之斬其部

  將張霸曹真又破其别將於尋陽【此江北之尋陽漢故縣地】 吳丹陽吳會山民復為寇【吳會吳郡會稽也會工外翻復扶又翻】攻沒屬縣吳王分三郡險地為東安郡【三郡豫章丹陽新都也吳録曰東安郡治富春或曰三郡丹陽吳會稽也項安世家說曰丹陽以多赤柳在丹陽山晉書南史並用楊字若丹陽則今江陵府枝江縣楚之始封余按二漢志丹陽郡本秦鄣郡漢武帝更名丹陽郡若丹陽縣班志注誤誠如項氏所云晉宋以後以丹陽郡為丹陽尹治秣陵二漢之丹楊郡治宛陵宛陵晉宋屬宣城郡治所既異漢魏之時自當依二漢志為丹陽郡】以綏南將軍全琮領太守【綏南將軍吳所創置】琮至明賞罰招誘降附【誘音酉降戶江翻】數年得萬餘人吳王召琮還牛渚罷東安郡 冬十月清河王冏卒 吳陸遜陳便宜勸吳王以施德緩刑寛賦息調【調徒弔翻】又云忠讜之言【讜音黨善言也】不能極陳求容小臣數以利聞【求猶乞也數所角翻】王報曰書載予違汝弼而云不敢極陳何得為忠讜哉【舜曰予違汝弼汝無面從退有後言讜音黨】於是令有司盡寫科條使郎中禇逢齎以就遜及諸葛瑾意所不安令損益之 十二月以鍾繇為太傅曹休為大司馬都督揚州如故【晉志曰黃初三年始置都督諸州軍事】曹真為大將軍華歆為太尉王朗為司徒陳羣為司空司馬懿為票騎大將軍【華戶化翻票匹妙翻】歆讓位於管寧帝不許徵寧為光禄大夫勑青州給安車吏從以禮發遣【寧北海朱虛人青州所部從才用翻】寧復不至【復扶又翻】 是歲吳交趾太守士燮卒吳主以燮子徽為安遠將軍領九真太守以校尉陳時代燮交州刺史呂岱以交趾絶遠表分海南三郡為交州以將軍戴良為刺史海東四郡為廣州岱自為刺史【海南三郡交趾九真日南也海東四郡蒼梧南海鬱林合浦也】遣良與時南入而徽自署交趾太守發宗兵拒良【自漢末之亂南方之人率宗黨相聚為兵以自衛】良留合浦交趾桓鄰燮舉吏也叩頭諫徽使迎良徽怒笞殺鄰鄰兄治合宗兵擊不克呂岱上疏請討徽督兵三千人晨夜浮海而往或謂岱曰徽藉累世之恩為一州所附未易輕也【易以豉翻】岱曰今徽雖懷逆計未知吾之卒至【卒讀曰猝】若我潛軍輕舉掩其無備破之必也稽留不速使得生心嬰城固守七郡百蠻雲合響應雖有智者誰能圖之遂行過合浦【過工禾翻】與良俱進岱以燮弟子輔為師友從事【師友從事者署為從事而待以師友之禮】遣往說徽【說輸芮翻】徽率其兄弟六人出降【降戶江翻】岱皆斬之

  孫盛論曰夫柔遠能邇莫善於信呂岱師友士輔使通信誓徽兄弟肉袒推心委命岱因滅之以要功利【要讀曰邀】君子是以知呂氏之祚不延者也【呂岱子孫無聞】

  徽大將甘醴及桓治率吏民共攻岱岱奮擊破之於是除廣州復為交州如故岱進討九真斬獲以萬數又遣從事南宣威命暨徼外扶南林邑堂明諸王各遣使入貢於吳【扶南在海大灣中北距日南七千里林邑國本漢象林縣地直交趾海行三千里堂明即道明國在真臘北徼吉弔翻】

  烈祖明皇帝上之上【諱叡字元仲文帝長子也諡法照臨四方日明】

  太和元年春吳解煩督胡綜【據綜傳劉備下白帝權以見兵少使綜料諸縣得六千人立解煩兩部督督督將也】番陽太守周魴擊彭綺生獲之【番蒲何翻魴音房】初綺自言舉義兵為魏討吳【為于偽翻】議者以為因此伐吳必有所克帝以問中書令太原孫資【沈約志魏武帝為王置祕書令典尚書奏事文帝黃初初改為中書令置監】資曰番陽宗人前後數有舉義者【數所角翻】衆弱謀淺旋輒乖散昔文皇帝嘗密論賊形埶言洞浦殺萬人得船千數數日間船人復會【事見上卷文帝黃初三年】江陵被圍歷月【被皮義翻】權裁以千數百兵住東門而其土地無崩解者是其法禁上下相維之明驗也以此推綺懼未能為權腹心大疾也至是綺果敗亡 二月立文昭皇后寢園於鄴【甄后賜死於鄴因葬焉】王朗往視園陵見百姓多貧困而帝方營修宫室朗上疏諫曰昔大禹欲拯天下之大患故先卑其宫室儉其衣食【論語孔子曰禹卑宫室菲飲食而盡力乎溝洫】句踐欲廣其禦兒之疆亦約其身以及家儉其家以施國【句音勾國語句踐既獲成於吳其地北至于禦兒非其身之所種則不食非其夫人之所織則不衣十年不收於國卒以報吳禦兒吳越分界之所今嘉興府即其地今有語兒鄉施弋智翻】漢之文景欲恢弘祖業故割意於百金之臺昭儉於弋綈之服【事見十五卷漢文帝後七年】霍去病中才之將猶以匈奴未滅不治第宅【事見十九卷漢武帝元狩四年治直之翻】明卹遠者略近事外者簡内也今建始之前足用列朝會崇華之後足用序内官華林天淵足用展遊宴【建始崇華二殿皆在洛陽北宫水經註穀水逕洛陽故城北東歷大夏門下枝分渠水東入華林園又東為天淵池世語曰魏武自漢中還洛陽起建始殿近漢濯龍祠朝直遥翻華如字】若且先成象魏【象魏觀闕也象者法象也魏者高巍也】修城池其餘一切須豐年專以勤耕農為務習戎備為事則民充兵強而寇戎賓服矣 三月蜀丞相亮率諸軍北駐漢中使長史張裔參軍蔣琬統留府事臨發上疏曰先帝創業未半而中道崩殂今天下三分益州疲敝此誠危急存亡之秋也然侍衛之臣不懈於内忠志之士忘身於外者蓋追先帝之殊遇欲報之於陛下也誠宜開張聖聽以光先帝遺德恢弘志士之氣不宜妄自菲薄引喻失義以塞忠諫之路也【塞悉則翻】宫中府中俱為一體【蜀後主建興元年命亮開府治事所謂府中蓋丞相府也】陟罰臧否不宜異同【否皮鄙翻】若有作姦犯科【科律條也】及為忠善者宜付有司論其刑賞以昭陛下平明之理不宜偏私使内外異法也【觀孔明所謂兩不宜則後主之為君可知矣】侍中侍郎郭攸之費禕董允等【時攸之禕為侍中允為黃門侍郎費父弗翻禕吁韋翻】此皆良實志慮忠純是以先帝簡拔以遺陛下【遺于季翻】愚以為宫中之事事無大小悉以咨之然後施行必能裨補闕漏有所廣益將軍向寵【向式亮翻姓也】性行淑均【行下孟翻】曉暢軍事試用於昔日先帝稱之曰能是以衆議舉寵為督愚以為營中之事悉以咨之必能使行陳和睦優劣得所【行戶剛翻陳讀曰陣】親賢臣遠小人此先漢所以興隆也親小人遠賢臣此後漢所以傾頹也【遠于願翻】先帝在時每與臣論此事未嘗不歎息痛恨於桓靈也侍中尚書長史參軍此悉端良死節之臣願陛下親之信之則漢室之隆可計日而待也臣本布衣躬耕南陽苟全性命於亂世不求聞達於諸侯先帝不以臣卑鄙猥自枉屈三顧臣於草廬之中諮臣以當世之事【事見六十五卷漢獻帝建安十二年】由是感激遂許先帝以驅馳後值傾覆受任於敗軍之際奉命於危難之間【事見上卷文帝黃初四年難乃旦翻】爾來二十有一年矣【自建安十二年至是年凡二十一年】先帝知臣謹慎故臨崩寄臣以大事也受命以來夙夜憂勤恐託付不效以傷先帝之明故五月渡瀘【水經註犍為朱提縣西八十里有瀘津水廣六七百步深十數丈多瘴氣鮮有行者益州記曰瀘水兩峯有殺氣暑月舊不行故武侯以夏渡為難賢曰瀘水一名若水出旄牛徼外經朱提至僰道入江在今嶲州南特有瘴氣三月四月經之必死五月以後行者得無害故諸葛亮表云五月渡瀘言其艱苦也】深入不毛【地不生草木為不毛】今南方已定甲兵已足當奨率三軍北定中原庶竭駑鈍【駑音奴】攘除姦凶興復漢室還於舊都此臣所以報先帝而忠陛下之職分也【分扶問翻】至於斟酌損益進盡忠言則攸之禕允之任也願陛下託臣以討賊興復之效不效則治臣之罪以告先帝之靈【治直之翻】責攸之禕允等之慢以章其咎【過也】陛下亦宜自謀以諮諏善道【諏遵須翻諮事為諏】察納雅言【雅正也】深追先帝遺詔臣不勝受恩感激今當遠離【勝音升離力智翻】臨表涕零不知所言遂行屯于沔北陽平石馬【水經註沔水逕白馬戍南謂之白馬城一名陽平關又有白馬山山石似馬望之逼真後魏分沔陽置嶓冢縣屬華陽郡隋罷郡置白馬鎮於古諸葛城縣治不改大業二年改嶓冢為西縣唐屬梁州】亮辟廣漢太守姚伷為掾【伷音胄掾丞相掾也音于絹翻】伷並進文武之士亮稱之曰忠益者莫大於進人進人者各務其所尚今姚掾並存剛柔以廣文武之用可謂博雅矣願諸掾各希此事以屬其望【希慕也鄭氏周禮注屬合也】帝聞諸葛亮在漢中欲大發兵就攻之以問散騎常侍孫資資曰昔武皇帝征南鄭取張魯陽平之役危而後濟【事見六十七卷建安二十年】又自往拔出夏侯淵軍【事見六十八卷建安二十四年】數言南鄭直為天獄中斜谷道為五百里石穴耳言其深險喜出淵軍之辭也又武皇帝聖於用兵察蜀賊棲於山巖視吳虜竄於江湖皆橈而避之【數所角翻斜余遮翻谷音浴橈奴教翻曲也屈也】不責將士之力不爭一朝之忿誠所謂見勝而戰知難而退也今若進軍就南鄭討亮道既險阻計用精兵及轉運鎮守南方四州遏禦水賊凡用十五六萬人【四州荆徐楊豫也】必當復更有所發興【復扶又翻】天下騷動費力廣大此誠陛下所宜深慮夫守戰之力力役參倍但以今日見兵【見賢遍翻】分命大將據諸要險威足以震懾強寇鎮静疆場【場音亦】將士虎睡百姓無事數年之間中國日盛吳蜀二虜必自罷敝【罷讀曰疲】帝乃止 初文帝罷五銖錢【事見六十九卷黃初元年】使以穀帛為用人間巧偽漸多競濕穀以要利薄絹以為市雖處以嚴刑不能禁也【要一遥翻處昌呂翻】司馬芝等舉朝大議【朝直遥翻】以為用錢非徒豐國亦所以省刑今不若更鑄五銖為便夏四月乙亥復行五銖錢 甲申初營宗廟於洛陽六月以司馬懿都督荆豫州諸軍事率所領鎮宛【宛於】

  【元翻】 冬十二月立貴嬪河内毛氏為皇后【后典虞工卒毛嘉之女也】初帝為平原王納河内虞氏為妃及即位虞氏不得立為后太皇卞太后慰勉焉虞氏曰曹氏自好立賤未有能以義舉者也【武帝立卞后文帝立郭后皆非正室好呼到翻】然后職内事君聽外政【禮記昏義古者天子后立六宫三夫人九嬪二十七世婦八十一御妻以聽天下之内治以明章婦順故天下内和而家理天子立六官三公九一卿二十七大夫八十一元士以聽天下之外治以明章天下之男教故外和而國治】其道相由而成苟不能以善始未有能令終者也殆必由此亡國喪祀矣【喪息浪翻】虞氏遂絀還鄴宫【絀敕律翻】 初太祖世祖皆議復肉刑以軍事不果【太祖議復肉刑事見六十六卷漢獻帝建安十八年其後文帝臨饗羣臣詔謂大理欲復肉刑此誠聖王之法公卿當共善議議未定會有軍事復寢】及帝即位太傅鍾繇上言宜如孝景之令其當棄市欲斬右趾者許之其黥劓左趾宫刑者【劓魚器翻】自如孝文易以髠笞可以歲生三千人詔公卿已下議司徒朗以為肉刑不用已來歷年數百今復行之恐所減之文未彰於萬民之目而肉刑之問已宣於寇讐之耳非所以來遠人也今可按繇所欲輕之死罪使減死髠刑嫌其輕者可倍其居作之歲數【魏制髠刑居作五歲】内有以生易死不訾之恩外無以刖易釱駭耳之聲【訾津私翻釱大計翻在頸曰鉗在足曰釱臣瓚曰漢文帝除肉刑以完易髠以笞代劓以釱左右趾代刖】議者百餘人與朗同者多帝以吳蜀未平且寢是歲吳昭武將軍韓當卒其子綜淫亂不軌懼得罪閏月將其家屬部曲來奔【為韓綜為吳所禽張本】 初孟達既為文帝所寵又與桓階夏侯尚親善及文帝殂階尚皆卒【卒子恤翻】達心不自安諸葛亮聞而誘之【誘音酉】達數與通書隂許歸蜀【數所角翻】達與魏興太守申儀有隙【魏興蜀之西城郡也文帝改曰魏興】儀密表告之達聞之惶懼欲舉兵叛司馬懿以書慰解之達猶豫未决懿乃潛軍進討諸將言達與吳漢交通宜觀望而後動懿曰達無信義此其相疑之時也當及其未定促决之乃倍道兼行八日到其城下吳漢各遣偏將向西城安橋木闌塞以救達【水經註魏興安陽縣西北有高橋溪曰文水入漢之口也漢水又東逕西城縣故城南又東逕木蘭塞南右㟁有城名陵城周回數里左㟁壘石數十行重壘數十里中謂之木蘭塞蓋吳兵向安橋而蜀兵向木蘭塞也晏類要云伎陵城在金州洵陽縣庾雍漢水記即木蘭塞蜀軍救孟達之所】懿分諸將以距之初達與亮書曰宛去洛八百里【司馬懿時屯宛】去吾一千二百里聞吾舉事當表上天子比相反覆一月間也【上時掌翻比必寐翻】則吾城已固諸軍足辦吾所在深險司馬公必不自來諸將來吾無患矣及兵到達又告亮曰吾舉事八日而兵至城下何其神速也

  資治通鑑卷七十  
    


 


 



 

 
  







 


  
  
 
 
 


  

 















	
	









































 
  



















 





 












  
  
  

 





