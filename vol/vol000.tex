欽定四庫全書     史部二

資治通鑑       編年類

提要

|{
	臣}
等謹案資治通鑑二百九十四卷宋司馬光撰胡三省音注光撰通鑑以元豐七年十二月戊辰書成與目録考異同奏上之其著述大旨具見光進書原表及馬端臨經籍考中當時劉安世所作音義十卷世已無傳南渡後有蜀史炤釋文又有偽托司馬釋文及廣都費氏註又有謝珏郭仲威直音及閩本直音羣書並行而乖謬咸甚三省本其父遺志因彚集衆說覃思以為之註元袁桷清容集載先友淵源録稱三省天台人寶祐進士賈相館之釋通鑑三十年兵難藁三失乙酉歲留袁氏塾日手抄定註己丑寇作以書藏窖中得免頗見是書顛末惟桷稱定註而今本題作音註疑出三省所自改也通鑑文繁義博貫串最難三省所釋於象緯推測地形建置制度沿革諸大端極為賅備讀通鑑者奉為圭臬真不啻左傳之有杜當陽矣又袁桷稱定註今在家尚未付梓而黄溥簡籍遺聞謂刋於臨海洪武初取藏南京國學則元末始鐫諸板今世所通行乃天啟中陳仁錫重刋本也乾隆四十五年十二月恭校上

總纂官|{
	臣}
紀昀|{
	臣}
陸錫熊|{
	臣}
孫士毅

總校官|{
	臣}
陸費墀

進資治通鑑表

臣光言先奉勅編集歷代君臣事迹又奉聖旨賜名資治通鑑今已了畢者伏念臣性識愚魯學術荒疏凡百事為皆出人下獨於前史麤嘗盡心自幼至老嗜之不厭每患遷固以來文字繁多自布衣之士讀之不徧况於人主日有萬幾何暇周覽臣常不自揆欲刪削冗長舉撮機要專取關國家興衰繫生民休戚善可為法惡可為戒者為編年一書使先後有倫精粗不雜私家力薄無由可成伏遇英宗皇帝資睿智之性敷文明之治思歷覽古事用恢張大猷爰詔下臣俾之編集臣夙昔所願一朝獲伸踊躍奉承惟懼不稱先帝仍命自選辟官屬於崇文院置局許借龍圖天章閣三館秘閣書籍賜以御書筆墨繒帛及御前錢以供果餌以内臣為承受眷遇之榮近臣莫及不幸書未進御先帝違棄羣臣陛下紹膺大統欽承先志寵以冠序錫之嘉名每開經筵常令進讀臣雖頑愚荷兩朝知待如此甚厚隕身喪元未足報塞苟智力所及豈敢有遺會差知永興軍以衰疾不任治劇乞就冗官陛下俯從所欲曲賜容養差判西京留司御史臺及提舉嵩山崇福宫前後六任仍聽以書局自隨給之禄秩不責職業臣既無他事得以研精極慮窮竭所有日力不足繼之以夜徧閲舊史旁采小說簡牘盈積浩如煙海抉擿幽隱校計毫釐上起戰國下終五代凡一千三百六十二年修成二百九十四卷又畧舉事目年經國緯以備檢尋為目録三十卷又參考羣書評其同異俾歸一塗為考異三十卷合三百五十四卷自治平開局迨今始成歲月淹久其間牴牾不敢自保罪負之重固無所逃|{
	中謝}
重念臣違離闕庭十有五年雖身處于外區區之心朝夕寤寐何嘗不在陛下之左右顧以駑蹇無施而可是以專事鉛槧用酬大恩庶竭涓塵少禆海嶽臣今筋骸癯瘁目視昏近齒牙無幾神識衰耗目前所為旋踵遺忘臣之精力盡於此書伏望陛下寛其妄作之誅察其願忠之意以清間之燕時賜省覽監前世之興衰考當今之得失嘉善矜惡取是舍非足以懋稽古之盛德躋無前之至治俾四海羣生咸蒙其福則臣雖委骨九泉志願永畢矣














































































































































