\chapter{資治通鑑卷一百二十四}
宋 司馬光 撰

胡三省 音註

宋紀六|{
	起玄黓敦牂盡柔兆閹茂凡五年}


太祖文皇帝中之中

元嘉十九年春正月甲申魏主備法駕詣道壇受符籙旗幟盡青自是每帝即位皆受籙|{
	此所受者今道士所謂法籙也隋志曰道士受道之法初受五千文籙次受三洞籙次受洞玄籙次受上清籙籙皆素書紀諸天曹官屬佐吏之名又有諸符錯在其間文章詭怪世所不識籙龍玉翻}
謙之又奏作静輪宫|{
	水經註静輪宫在道壇東北道壇在平城東灅水之左}
必令其高不聞雞犬欲以上接天神崔浩勸帝為之功費萬計經年不成太子晃諫曰天人道殊卑高定分|{
	分扶問翻}
不可相接理在必然今虚耗府庫疲弊百姓為無益之事將安用之必如謙之所言請因東山萬仞之高|{
	謂平城之東山也}
為功差易|{
	易以豉翻}
帝不從 夏四月沮渠無諱將萬餘家棄敦煌西就沮渠安周未至鄯善王比龍畏之將其衆奔且末|{
	沮子余翻將即亮翻敦徒門翻鄯上扇翻且末漢故國在鄯善西去代八千三百二十里且子余翻}
其世子降於安周|{
	降戶江翻}
無諱遂據鄯善其士卒經流沙渴死者大半李寶自伊吾帥衆二千入據敦煌|{
	帥讀曰率}
繕脩城府安集故民沮渠牧犍之亡也|{
	見上卷十六年犍居言翻}
凉州人闞爽據高昌自稱太守唐契為柔然所逼擁衆西趨高昌|{
	闞苦濫翻守手又翻趨七喻翻}
欲奪其地柔然遣其將阿若追擊之契敗死|{
	營陽王景平元年契與李寶同奔伊吾}
契弟和收餘衆奔車師前部王伊洛時沮渠安周屯横截城和攻拔之又拔高寧白力二城|{
	李延壽曰高昌國有四十六鎮交河田地高寧白刃横截等餘不具載白力當作白刃}
遣使請降於魏|{
	使疏吏翻}
甲戌上以疾愈大赦 五月裴方明等至漢中與劉真道等分兵攻武興下辯白水皆取之|{
	下辯漢書作下辨並音皮莧翻}
楊難當遣建節將軍符弘祖守蘭臯|{
	元豐九域志階州將利縣有蘭臯鎮按五代志將利縣後魏武興郡之石門縣也蕭子顯曰武興西北有蘭臯戍去仇池二百里符恐當作苻楊氏苻氏皆氐種也}
使其子撫軍大將軍和將重兵為後繼方明與弘祖戰于濁水|{
	濁水城在上禄縣東南武街城西北酈道元曰濁水即白水也武街城故下辨縣治}
大破之斬弘祖和退走追至赤亭又破之難當奔上邽獲難當兄子建節將軍保熾難當以其子虎為益州刺史守隂平聞難當走引兵還至下辯方明使其子肅之邀擊之擒虎送建康斬之仇池平以輔國司馬胡崇之為北秦州刺史鎮其地立楊保熾為楊玄後使守仇池|{
	楊難當廢玄子保宗而自立見一百二十一卷六年}
魏人遣中山王辰迎楊難當詣平城秋七月以劉真道為雍州刺史|{
	雍於用翻}
裴方明為梁南秦二州刺史方明辭不拜 |{
	考異曰真道傳此事在胡崇之没後氐胡傳崇之没在明年二月即真道傳誤}
丙寅魏主使安西將軍古弼 |{
	考異曰宋索虜傳作吐奚愛弼氐胡傳作吐奚弼蓋其舊姓今從後魏書}
督隴右諸軍及殿中虎賁|{
	賁音奔}
與武都王楊保宗自祁山南入|{
	保宗奔魏見上卷十六年}
征西將軍漁陽皮豹子與琅邪王司馬楚之督關中諸軍自散關西入俱會仇池又使譙王司馬文思督洛豫諸軍南趨襄陽|{
	營陽王景平二年魏取河南置洛州於洛陽豫州於虎牢趨七喻翻下同}
征南將軍刁雍東趨廣陵|{
	雍於容翻}
移書徐州稱為楊難當報仇|{
	為于偽翻}
甲戌晦日有食之 唐契之攻闞爽也 |{
	考異曰宋氐胡傳作闕爽今從後魏書}
爽遣使詐降于沮渠無諱欲與之共擊契|{
	使疏吏翻降戶江翻}
八月無諱將其衆趨高昌比至|{
	將即亮翻下同比必利翻及也}
契已死爽閉門拒之九月無諱將衛興奴夜襲高昌屠其城 |{
	考異曰宋書衛興奴作衛尞今從後魏書}
爽奔柔然無諱據高昌遣其常侍汜雋奉表詣建康|{
	汜音凡}
詔以無諱都督凉河沙三州諸軍事征西大將軍凉州刺史河西王 |{
	考異曰宋紀封爵在六月傳在九月末今從傳}
冬十月己卯魏立皇子伏羅為晉王翰為秦王譚為燕王建為楚王余為吳王 甲申柔然遣使詣建康 十二月辛巳魏襄城孝王盧魯元卒丙申詔魯郡脩孔子廟及學舍蠲墓側五戶課役以

供灑掃|{
	灑所賣翻又所買翻掃素報翻又蘇老翻}
李寶遣其弟懷達子承奉表詣平城魏人以寶為都督西垂諸軍事|{
	遠邉曰垂}
鎮西大將軍開府儀同三司沙州牧敦煌公|{
	敦徒門翻}
四品以下聽承制假授 雍州刺史晉安襄侯劉道產卒道產善為政民安其業小大豐贍由是民間有襄陽樂歌|{
	雍於用翻贍時艷翻卒子恤翻樂音洛}
山蠻前後不可制者皆出緣沔為村落戶口殷盛及卒蠻追送至沔口未幾羣蠻大動|{
	道產卒未幾而蠻作亂後之人不能容養之也沔彌兖翻幾居豈翻}
征西司馬朱脩之討之不

利詔建威將軍沈慶之代之殺虜萬餘人 魏主使尚書李順差次羣臣賜以爵位順受賄品第不平是歲凉州人徐桀告之魏主怒且以順保庇沮渠氏面欺誤國|{
	事見上卷十六年}
賜順死

二十年春正月魏皮豹子進擊樂鄉將軍王奐之等敗没魏軍進至下辯將軍強玄明等敗死|{
	強其兩翻}
二月胡崇之與魏戰於濁水崇之為魏所擒餘衆走還漢中將軍姜道祖兵敗降魏|{
	降戶江翻}
魏遂取仇池楊保熾走 丙午魏主如恒山之陽|{
	恒戶登翻}
三月庚申還宫 壬戌烏洛侯國遣使如魏|{
	烏洛侯國在地豆于國北去代四千五百餘里地豆于在室韋西千餘里室韋當勿吉之北勿吉在高麗之北則烏洛侯東夷也使疏吏翻}
初魏之居北荒也鑿石為廟在烏洛侯西北以祀其先高七十尺深九十步|{
	度高曰高音居號翻度深曰深音式禁翻}
及烏洛侯使者至魏言石廟具在魏主遣中書侍郎李敞詣石廟致祭刻祝文於壁而還去平城四千餘里 魏河間公齊與武都王楊保宗對鎮雒谷|{
	雒谷即駱谷北史作駱}
保宗弟文德說保宗令閉險自固以叛魏|{
	說輸芮翻}
或以告齊夏四月齊誘執保宗送平城殺之前鎮東司苻達|{
	司上當有軍字否則司下當有馬字}
征西從事中郎任朏等|{
	苻達等皆楊氏官屬也任音壬朏敷尾翻}
遂舉兵立楊文德為主據白崖|{
	今大安軍東北八十里有白崖大安軍古葭萌地也 考異曰宋氐胡傳云拓跋齊聞兵起遁走達追擊斬齊因據白崖按後魏河間公齊傳云文德求援于宋宋遣房亮之苻昭啖龍等帥衆助文德斬龍擒亮之氐遂平以功拜内都大官卒然則宋書誤也}
分兵取諸戍進圍仇池自號征西將軍秦河梁三州牧仇池公 |{
	考異曰宋書在三月魏書在四月今從之}
甲午立皇子誕為廣陵王 丁酉魏大赦 己亥魏主如隂山 五月魏古弼上邽高平岍城諸軍擊楊文德|{
	岍城意當作汧城汧口堅翻}
文德退走皮豹子督關中諸軍至下辯聞仇池解圍欲還弼遣人謂豹子曰宋人恥敗必將復來|{
	復扶又翻}
軍還之後再舉為難不如練兵蓄力以待之不出秋冬宋師必至以逸待勞無不克矣豹子從之魏以豹子為仇池鎮將楊文德遣使來求援|{
	使疏吏翻}
秋七月癸丑詔以文德為都督北秦雍二州諸軍事征西大將軍北秦州刺史武都王|{
	雍於用翻}
文德屯葭蘆城|{
	五代志武都郡盤堤縣西魏之南五部縣也魏又置武陽郡及茄蘆縣後周皆併入盤堤祝穆曰盤池山在階州福津縣東南七十里郡縣志魏將鄧艾與蜀將姜維相持于此置茄蘆戍後於此置縣}
以任朏為左司馬武都隂平氐多歸之 甲子前雍州刺史劉真道梁南秦二州刺史裴方明坐破仇池減匿金寶及善馬下獄死|{
	宋人捨功録過自戮良將宜其為魏人所窺下遐稼翻}
九月辛巳魏主如漠南甲辰捨輜重|{
	重直用翻}
以輕騎襲柔然|{
	騎奇寄翻下同}
分軍為四道樂安王範建寧王崇各統十五將出東道樂平王丕督十五將出西道魏主出中道中山王辰督十五將為後繼|{
	將即亮翻}
魏主至鹿渾谷|{
	鹿渾谷即鹿渾海之谷也本高車袁紇部所居其地直平城西北其東即弱洛水}
遇敕連可汗|{
	可從刋入聲汗音寒}
太子晃言于魏主曰賊不意大軍猝至宜掩其不備速進擊之尚書令劉絜固諫以為賊營中塵盛其衆必多出至平地恐為所圍不如須諸軍大集|{
	須待也}
然後擊之晃曰塵之盛者由軍士驚怖擾亂故也|{
	怖普布翻}
何得營上而有此塵乎魏主疑之不急擊柔然遁去追至石水不及而還|{
	石水在頞根河北還從宣翻又如字}
既而獲柔然候騎曰柔然不覺魏軍至上下惶駭引衆北走經六七日知無追者乃始徐行魏主深恨之|{
	為魏誅劉絜中山王辰等張本}
自是軍國大事皆與太子謀之司馬楚之别將兵督軍糧鎮北將軍封沓亡降柔然說柔然令擊楚之以絶軍食|{
	降戶江翻說輸芮翻}
俄而軍中有告失驢耳者諸將莫曉其故楚之曰此必賊遣姦人入營覘伺|{
	覘丑廉翻又丑艷翻伺相吏翻}
割驢耳以為信耳賊至不久宜急為之備乃伐柳為城以水灌之令凍城立而柔然至冰堅滑不可攻乃散走 十一月將軍姜道盛與楊文德合衆二萬攻魏濁水戍魏皮豹子河間公齊救之道盛敗死 甲子魏主還至朔方下詔令皇太子副理萬機總統百揆 |{
	考異曰宋索虜傳晃與大臣崔氏寇氏不睦崔寇譛之玄高道人有道術晃使祈福七日七夜佛狸夢其祖父並怒手刃向之曰汝何故信讒欲害太子佛狸驚覺下偽詔曰王者大業纂承為重儲宫嗣紹百王舊例自今以往事無巨細必經太子然後上聞事節小異今從後魏書}
且曰諸功臣勤勞日久皆當以爵歸第随時朝請饗宴朕前論道陳謨而已不宜復煩以劇職|{
	朝直遥翻復扶又翻}
更舉賢俊以備百官十二月丁卯魏主還平城|{
	自伐柔然還也}


二十一年春正月己亥帝耕藉田大赦|{
	藉秦昔翻 考異曰宋畧辛酉藉田大赦下有戊午又有辛酉誤也今從宋書}
壬寅魏太子始總百揆命侍中中書監穆夀司徒崔浩侍中張黎古弼輔太子决庶政上書者皆稱臣儀與表同古弼為人忠慎質直嘗以上谷苑囿太廣乞減大半以賜貧民入見魏主欲奏其事|{
	據北史古弼傳時上谷人上書言苑囿過度人無田業宜減大半以賜貧者蓋上谷距代都甚遠魏未嘗置苑囿于其地而道武帝起鹿苑于南臺隂北距長城東苞白登屬之西山廣輪數十里天興六年幸南平城規度灅南夏屋山背黄瓜堆以建新邑至天賜三年遂築灅南宫闕引溝穿池廣苑囿所謂太廣者此也不在上谷當以北史為正見賢遍翻}
帝方與給事中劉樹圍碁志不在弼弼侍坐良久不獲陳聞|{
	坐徂卧翻}
忽起捽樹頭|{
	捽昨没翻}
掣下牀搏其耳毆其背|{
	掣尺列翻毆烏口翻}
曰朝廷不治實爾之罪|{
	治直之翻}
帝失容捨碁曰不聽奏事朕之過也樹何罪置之弼具以狀聞帝皆可其奏弼曰為人臣無禮至此其罪大矣出詣公車免冠徒跣請罪帝召入謂曰吾聞築社之役蹇蹷而築之|{
	蜀注曰跛蹇而顛蹷也}
端冕而事之神降之福然則卿有何罪其冠履就職苟可以利社稷便百姓者竭力為之勿顧慮也太子課民稼穡使無牛者借人牛以耕種而為之芸田以償之|{
	為于偽翻}
凡耕種二十二畝而芸七畝大略以是為率使民各標姓名於田首以知其勤惰禁飲酒遊戲者於是墾田大增 戊申魏主詔王公以下至庶人有私養沙門巫覡於家者|{
	男曰巫女曰覡覡刑狄翻}
皆遣詣官曹過二月十五日不出沙門巫覡死主人門誅|{
	門誅者闔門盡誅之}
庚戌又詔王公卿大夫之子皆詣太學其百工商賈之子當各習父兄之業|{
	賈音古}
毋得私立學校|{
	校戶敎翻}
違者師死主人門誅 二月辛未魏中山王辰内都坐大官薛辨|{
	魏置中都大官外都大官都坐大官皆掌折獄謂之三都坐徂卧翻}
尚書奚眷等八將|{
	將即亮翻下同}
坐擊柔然後期斬於都南初魏尚書令劉絜久典機要|{
	宋高祖永初末魏明元帝寢疾魏主監國劉絜與古弼等選侍東宫對綜機要至是二十餘年矣}
恃寵自專魏主心惡之|{
	惡烏路翻}
及將襲柔然絜諫曰蠕蠕遷徙無常前者出師勞而無功|{
	絜之言蓋指太延四年魏主伐柔然至白阜時也蠕人兖翻}
不如廣農積穀以待其來崔浩固勸魏主行魏主從之絜恥其言不用欲敗魏師|{
	敗補邁翻}
魏主與諸將期會鹿渾谷絜矯詔易其期帝至鹿渾谷欲擊柔然絜諫止之使待諸將帝留鹿渾谷六日諸將不至柔然遂遠遁追之不及軍還經漠中糧盡士卒多死絜隂使人驚魏軍勸帝委軍輕還帝不從絜以軍出無功請治崔浩之罪|{
	治直之翻下同}
帝曰諸將失期遇賊不擊浩何罪也浩以絜矯詔事白帝帝至五原收絜囚之帝之北行也絜私謂所親曰若車駕不返吾當立樂平王絜聞尚書右丞張嵩家有圖䜟問曰劉氏應王繼國家後吾有姓名否嵩曰有姓無名帝聞之命有司窮治索嵩家得䜟書|{
	索山客翻}
事連南康公狄鄰絜嵩鄰皆夷三族死者百餘人絜在勢要好作威福|{
	好呼到翻}
諸將破敵所得財物皆與絜分之既死籍其家財巨萬帝每言之則切齒癸酉樂平戾王丕以憂卒初魏主築白臺高二百餘尺|{
	魏主嗣泰常二年秋七月乙酉起白臺於平城南高二十丈}
丕夢登其上四顧不見人命術士董道秀筮之道秀曰大吉丕默有喜色及丕卒道秀亦坐棄市高允閒之曰夫筮者皆當依附爻象勸以忠孝|{
	漢嚴君平卜筮於成都市人有邪惡非正之問則依蓍龜為言利害與人子言依於孝與人弟言依於順與人臣言依於忠各因勢道之以善高允之言祖君平之術也}
王之問道秀也道秀宜曰窮高為亢易曰亢龍有悔又曰高而無民|{
	易乾上九及文言之辭亢苦浪翻}
皆不祥也王不可以不戒如此則王安於上身全於下矣道秀反之宜其死也 庚辰魏主幸廬|{
	自南北國分治人主所至例不書幸此必誤也}
己丑江夏王義恭進位太尉領司徒|{
	夏戶雅翻}
庚寅以侍中領右衛將軍沈演之為中領軍左衛將軍范曄為太子詹事 辛卯立皇子宏為建平王 三月甲辰魏主還宫 癸丑魏主遣司空長孫道生鎮統萬|{
	長知兩翻}
夏四月乙亥魏侍中太宰陽平王杜超為帳下所殺 六月魏北部民殺立義將軍衡陽公莫孤帥五千餘落北走遣兵追擊之至漠南殺其渠帥餘徙冀相定三州為營戶|{
	杜佑曰魏道武天興中詔採漏戶令輸綸綿自後諸逃戶占為紬繭羅縠者甚衆於是雜營戶率徧於天下不隸守宰賦役不同景穆皇帝一切罷之以屬郡縣孤帥讀曰率渠帥所類翻相息亮翻}
吐谷渾王慕利延兄子緯世與魏使者謀降魏|{
	緯世即阿柴之長子緯代也北史避唐太宗諱改世為代使疏吏翻降戶江翻}
慕利延殺之是月緯世弟叱力延等八人奔魏魏以叱力延為歸義王 沮渠無諱卒|{
	沮子余翻}
弟安周代立 魏入中國以來雖頗用古禮祀天地宗廟百神而猶循其舊俗所祀胡神甚衆崔浩請存合於祀典者五十七所其餘複重及小神悉罷之|{
	量直龍翻}
魏主從之 秋七月癸卯魏東雍州刺史沮渠秉謀反伏誅|{
	隋志絳郡後魏置東雍州後周改曰絳州雍於用翻}
八月乙丑魏主畋于河西尚書令古弼留守|{
	守手又翻}
詔以肥馬給獵騎弼悉以弱者給之帝大怒曰筆頭奴敢裁量朕|{
	騎奇寄翻量音良}
朕還臺先斬此奴弼頭鋭故帝常以筆目之弼官屬惶怖恐并坐誅|{
	怖普布翻}
弼曰吾為人臣不使人主盤於遊畋|{
	盤樂也}
其罪小不備不虞乏軍國之用其罪大今蠕蠕方彊南寇未滅吾以肥馬供軍弱馬供獵為國遠慮|{
	為于偽翻}
雖死何傷且吾自為之非諸君之憂也帝聞之歎曰有臣如此國之寶也賜衣一襲|{
	衣一稱為一襲猶今言一副衣服也}
馬二匹鹿十頭他日魏主復畋于山北|{
	山北平城北山之北復扶又翻}
獲麋鹿數千頭詔尚書發車五百乘以運之|{
	觀下載弼表蓋民車也秉繩證翻}
詔使已去魏主謂左右曰筆公必不與我汝輩不如以馬運之遂還行百餘里得弼表曰今秋穀懸黄麻菽布野豬鹿竊食鳥鴈侵費風雨所耗朝夕三倍|{
	言夕之所收較于朝之所收得失三倍收穫不可以不速載麋鹿猶可緩}
乞賜矜緩使得收載帝曰果如吾言筆公可謂社稷之臣矣魏主使員外散騎常侍高濟來聘|{
	散悉亶翻騎奇寄翻}
戊辰

以荆州刺史衡陽王義季為征北大將軍開府儀同三司南兗州刺史以南譙王義宣為荆州刺史初帝以義宣不才故不用會稽公主屢以為言帝不得已用之|{
	會稽公主高祖長女帝深加禮敬家事大小必咨之會工外翻}
先賜中詔敕之曰師護以在西久|{
	詔自中出不經門下者謂之中詔今之手詔是也敕戒也義季小字師護}
比表求還|{
	比毗至翻頻也}
今欲聽許以汝代之師護雖無殊績絜已節用|{
	絜與潔同}
通懷期物不恣群下聲著西土為士庶所安論者乃未議遷之今之囬換更為汝與師護年時一輩|{
	為于偽翻}
欲各試其能汝往脱有一事減之者既於西夏交有巨礙|{
	江左六朝以荆楚為西夏夏戶雅翻}
遷代之譏必歸責于吾矣|{
	言遷代之際所任非人也}
此事亦易勉耳無為使人復生評論也|{
	易以䜴翻復扶又翻}
義宣至鎮勤自課厲事亦脩理庚辰會稽長公主卒|{
	□知兩翻卒子恤翻}
吐谷渾叱力延等請師於魏以討吐谷渾王慕利延魏主使晉王伏羅督諸軍擊之 九月甲辰以沮渠安周為都督凉河沙三州諸軍事凉州刺史河西王 丁未魏主如漠南將襲柔然柔然敕連可汗遠遁乃止敕連尋卒子吐賀真立號處羅可汗|{
	魏收曰處羅魏言唯也可從刋入聲汗音寒}
魏晉王伏羅至樂都|{
	樂音洛}
引兵從間道襲吐谷渾|{
	間古莧翻}
至大母橋吐谷渾王慕利延大驚逃奔白蘭慕利延兄子拾寅奔河西魏軍斬首五千餘級慕利延從弟伏念等帥萬三千落降於魏|{
	慕利延背阿柴折箭之誡使之招引外寇至于衆叛親離同其宜也從才用翻帥讀曰率降戶江翻}
冬十月己卯以左軍將軍徐瓊為兖州刺史大將軍參軍申恬為冀州刺史徙兖州鎮須昌|{
	沈約曰武帝定河南以兖州治滑臺文帝元嘉十三年治鄒山又寄治彭城此又自彭城徙須昌也}
冀州鎮歷下|{
	歷下即歷城}
恬謨之弟也 十二月魏主還平城 是歲沙州牧李寶入朝于魏魏人留之以為外都大官|{
	為李氏貴盛張本朝直遥翻}
太子率更令何承天撰元嘉新歷表上之|{
	更工衡翻上時掌翻下所上同}
以月食之衝知日所在|{
	日與月對衝光相揜而知之}
又以中星檢之知堯時冬至日在須女十度|{
	此以堯典日短星昴推之}
今在斗十七度又測景校二至差三日有餘|{
	此亦用周禮測日至之景之法也}
知今之南至日應在斗十三四度于是更立新法冬至徙上三日五時日之所在移舊四度又月有遲疾前歷合朔月食不在朔望|{
	月食上當有日字}
今皆以贏縮定其小餘以正朔望之日|{
	贏或作盈歷法有大餘小餘史記歷書曰大餘者日也小餘者月也周天三百六十五度四分度之一日日盈一度十二月而一周天歲十二月凡三百五十四日以六除之五六三百日餘五十四日為大餘周天三百六十五度以六甲除之六六三百六十餘五為大餘小餘即四分之一未滿日之分數也其分每滿三十二則成一日蓋奇日為大餘奇分為小餘積而成閏也}
詔付外詳之太史令錢樂之等奏皆如承天所上唯月有頻三大頻二小比舊法殊為乖異謂宜仍舊詔可

二十二年春正月辛卯朔始行新歷初漢京房以十二律中呂上生黄鐘不滿九寸更演為六十律|{
	中讀曰仲更工衡翻下同}
錢樂之復演為三百六十律|{
	復扶又翻}
日當一管何承天立議以為上下相生三分損益其一蓋古人簡易之法|{
	易以豉翻}
猶如古歷周天三百六十五度四分度之一也而京房不悟謬為六十乃更設新律林鍾長六寸一釐則從中呂還得黄鍾十二旋宫聲韻無失|{
	長直亮翻中讀曰仲律歷志黄鍾律九寸三分損一下生林鍾律六寸三分林鍾益一上生太簇三分大簇損一下生南呂三分南呂益一上生姑洗三分姑洗損一下生應鍾三分應鍾益一上生蕤賓三分蕤賓損一下生大呂三分大呂益一上生夷則三分夷則損一下生夾鍾三分夾鍾益一上生無射三分無射損一下生中呂隂陽相生自黄鍾始而左旋八八為伍孟康注曰從子數至未得八下生林鍾數未至寅得八上生太簇律上下相生皆以此為率伍耦也八八為耦然月令注中呂律長六寸萬九千六百八十三分寸之萬二千九百七十四若上生黄鍾當不止九寸故孔穎達考其同異於月令疏曰十二律有上生下生冋位異位長短分寸之别故鄭注周禮太師職云其相生則以隂陽六體黄鍾初九下生林鍾之初六林鍾又上生太簇之九二太簇又下生南呂之六二南呂又上生姑洗之九三姑洗又下生應鍾之六三應鍾又上生蕤賓之九四蕤賓又上生大呂之六四大呂又下生夷則之九五夷則又上生夾鍾之六五夾鍾又下生無射之上九無射又上生中呂之上六同位者象夫妻異位者象子母所謂律娶妻而呂生子也同位象夫妻者則黄鍾之初九下生林鍾之初六同是初位故為夫婦又是律娶妻也異位為子母者謂林鍾上生太簇林鍾是初位太簇是二位故云異位為子母又是呂生子也云五下六上者鄭注云五下六上乃一終矣謂林鍾夷則南呂無射應鍾皆被子午以東之管三分減一而下生之大呂太簇夾鍾姑洗中呂蕤賓皆被子午以西之管三分益一而上生之子午皆上生應云七上而云六上者以黄鍾為諸律之首物莫之先似若無所稟生者故不數黄鍾也其實十二律終于中呂反歸黄鍾生于中呂三分益一大略得應黄鍾九寸之數也律歷志云黄鍾為天統林鍾為地統太簇為人統故數整餘律則各有分數随其相生之次每辰各自為宫各有五聲黄鍾為第一宫下生林鍾為徵上生太簇為商下生南呂為羽上生姑洗為角林鍾為第二宫上生太簇為徵下生南呂為商上生姑洗為羽下生應鍾為角太簇為第三宫下生南呂為徵上生姑洗為商下生應鍾為羽上生蕤賓為角南呂為第四宫上生姑洗為徵下生應鍾為商上生蕤賓為羽下生大呂為角姑洗為第五宫下生應鍾為徵上生蕤賓為商上生大呂為羽下生夷則為角應鍾為第六宫上生蕤賓為徵上恐當作下上生大呂為商下生夷則為羽上生夾鍾為角蕤賓為第七宫上生大呂為徵下生夷則為商上生夾鍾為羽下生無射為角大呂為第八宫下生夷則為徵上生夾鍾為商下生無射為羽上生中呂為角夷則為第九宫上生夾鍾為徵下生無射為商上生中呂為羽上生黄鍾為角夾鍾為第十宫下生無射為徵上生中呂為商上生黄鍾為羽下生林鍾為角無射為第十一宫上生中呂為徵上生黄鍾為商下生林鍾為羽下生太簇為角中呂為第十二宫上生黄鍾為徵下生林鍾為商上生太簇為羽下生南呂為角是十二宫各有五聲凡六十聲京房六十律相生之法以上生下皆三生二以下生上皆三生四陽下生隂隂上生陽終于中呂而十二律畢矣中呂上生執始執始下生去滅上下相生終于南事而六十律畢矣夫十二律之變至于六十猶八卦之變至于六十四也六十律之名詳見續漢書補志}
壬辰以武陵王駿為雍州刺史|{
	雍於用翻}
帝欲經略關河故以駿鎮襄陽 魏主使散騎常侍宋愔來聘|{
	散悉亶翻騎奇寄翻愔於今翻}
二月魏主如上黨西至吐京|{
	酈道元曰吐京即漢西河郡吐軍縣夷夏俗音訛也後魏置吐京郡隋隰州石樓縣魏吐京郡地}
討徙叛胡出配郡縣甲戌立皇子褘為東海王昶為義陽王|{
	褘吁韋翻昶丑兩翻}
三月庚申魏主還宫 魏詔諸疑獄皆付中書以經義量决|{
	量音良}
夏四月庚戌魏主遣征西大將軍高凉王那等擊吐谷渾王慕利延於白蘭秦州刺史代人封敕文安遠將軍乙烏頭擊慕利延兄子什歸于枹罕|{
	枹音膚}
河西之亡也鄯善人以其地與魏鄰大懼|{
	鄯上扇翻}
曰通其使人知我國虛實取亡必速乃閉斷魏道|{
	閉斷魏通西域之道也使疏吏翻下同斷丁管翻}
使者往來輒鈔刼之|{
	鈔楚交翻}
由是西域不通者數年魏主使散騎常侍萬度歸凉州以西兵擊鄯善 六月壬辰魏主北廵 帝謀伐魏罷南豫州入豫州以南豫州刺史南平王鑠為豫州刺史|{
	高祖永初二年分淮東之地為南豫州治歷陽淮西為豫州或治壽陽或治汝南鑠式約翻}
秋七月己未以尚書僕射孟顗為左僕射|{
	顗魚豈翻}
中護軍何尚之為右僕射武陵王駿將之鎮時緣沔諸蠻猶為寇|{
	沔彌兖翻}
水陸梗

礙駿分軍遣撫軍中兵參軍沈慶之掩擊大破之駿至鎮蠻斷驛道|{
	斷丁管翻}
欲攻随郡随郡太守河東柳元景募得六七百人邀擊大破之遂平諸蠻獲七萬餘口溳山蠻㝡彊|{
	水經注云水出蔡陽縣東南大洪山山在随郡之西南竟陵之東北槃基所跨廣圓一百餘里溳水出于其山之隂時人以為溳水所導亦曰溳山溳音云}
沈慶之討平之獲三萬餘口徙萬餘口於建康 吐谷渾什歸聞魏軍將至棄城夜遁八月丁亥封敕文入枹罕分徙其民千家還上邽留乙烏頭守枹罕|{
	枹音膚}
萬度歸至敦煌留輜重以輕騎五千度流沙襲鄯善壬辰鄯善王真達面縛出降度歸留軍屯守與真達詣平城|{
	敦徒門翻重直用翻騎奇寄翻降戶江翻 考異曰本紀作真達興今從西域傳}
西域復通|{
	復扶又翻}
魏主如隂山之北諸州兵三分之一各於其州戒嚴以須後命|{
	須待也}
徙諸種雜民五千餘家於北邉|{
	種章勇翻}
令就北畜牧以餌柔然 壬寅魏高凉王那軍至寧頭城|{
	寧頭城當在白蓢東北}
吐谷渾王慕利延擁其部落西度流沙吐谷渾慕璝之子被囊逆戰那擊破之被囊遁走中山公杜豐帥精騎追之|{
	璝古回翻帥讀曰率}
度三危至雪山|{
	酈道元曰三危山在敦煌縣南}
生擒被囊及吐谷渾什歸乞伏熾盤之子成龍皆送平城|{
	乞伏成龍蓋因赫連定之敗没于吐谷渾}
慕利延遂西入于闐|{
	闐徒賢翻又徒見翻}
殺其王據其地死者數萬人 九月癸酉上餞衡陽王義季于武帳岡|{
	餞義季往鎮南兖杜佑曰武帳岡在廣莫門外宣武場設行宫殿便坐於其上因名}
上將行敕諸子且勿食至會所設饌日旰不至|{
	饌雛戀翻又雛晥翻旰古案翻}
有飢色上乃謂曰汝曹少長豐佚|{
	少詩沼翻長知雨翻}
不見百姓艱難今使汝曹識有飢苦知以節儉御物耳

裴子野論曰善乎太祖之訓也夫侈興于有餘儉生于不足欲其隱約莫若貧賤習其險艱利以任使達其情偽易以躬臨|{
	易以䜴翻}
太祖若能率此訓也難其志操卑其禮秩教成德立然後授以政事則無怠無荒可播之於九服矣|{
	周制九服侯服甸服男服采服衛服蠻服夷服鎮服藩服每服五百里謂之服者責以服事天子為職也}
高祖思固本枝崇樹襁褓後世遵守迭據方岳|{
	謂義真義康義恭義宣皆迭居方面襁居兩翻褓音保}
及乎泰始之初升明之季絶咽於衾衽者動數十人|{
	謂明帝殺孝武諸子而宋齊禪代之際蕭氏夷劉氏也咽音煙}
國之存亡既不是繫早肆民上|{
	左傳晉師曠曰天之愛民甚矣豈其使一人肆於民上}
非善誨也

魏民間訛言滅魏者吳盧水胡蓋吳聚衆反於杏城|{
	蓋吳蓋安定盧水胡種而分居杏城蓋古盍翻}
諸衆胡爭應之|{
	種章勇翻}
有衆十餘萬遣其黨趙綰來上表自歸冬十月戊子長安鎮副將拓跋紇帥衆討吳紇敗死|{
	上時掌翻將即亮翻紇下没翻}
吳衆愈盛民皆渡渭奔南山|{
	長安南山也}
魏主高平敕勒騎赴長安命將軍叔孫拔領攝并秦雍三州兵屯渭北|{
	騎奇寄翻雍於用翻}
十一月魏冀州民造浮橋於碻磝津 蓋吳遣别部帥白廣平|{
	帥所類翻}
西掠新平安定諸胡皆聚衆應之又分兵東掠臨晉巴東|{
	巴當作已}
將軍章直擊破之溺死于河者三萬餘人|{
	溺奴狄翻}
吳又遣兵西掠至長安將軍叔孫拔與戰於渭北大破之斬首三萬餘級河東蜀薛永宗聚衆以應吳|{
	蜀人遷居河東者謂之河東蜀居絳郡者謂之絳蜀居關中赤水者謂之赤水蜀}
襲擊聞喜|{
	聞喜縣屬河東郡春秋時晉武公所居之曲沃也秦改為左邑漢武帝於此聞南越破改曰聞喜後魏分屬絳郡}
聞喜縣無兵仗令憂惶無計縣人裴駿帥厲鄉豪擊之|{
	帥讀曰率}
永宗引去魏主命薛謹之子拔糾合宗鄉|{
	宗謂薛之宗族鄉謂鄉人}
壁於河際以斷二寇往來之路|{
	二寇謂薛永宗蓋吳斷丁管翻}
庚午魏主使殿中尚書拓跋處直等將二萬騎討薛永宗殿中尚書乙拔將三萬騎討蓋吳|{
	晉置殿中尚書與吏部五兵田曹度支左民為六曹杜佑曰後魏初有殿中樂部駕部南部北部五尚書殿中掌殿内兵馬倉庫樂部掌伎樂及角使伍伯駕部掌牛馬驢騾南部掌南邉諸州郡北部掌北邉諸州郡魏書官氏志内入諸姓乙弗氏改為乙氏處昌呂翻將即亮翻騎奇寄翻下同}
西平公寇提將萬騎討白廣平|{
	官氏志内入諸姓若口引氏改為寇氏}
吳自號天台王署置百官 辛未魏主還宫|{
	自隂山還也}
魏選六州驍騎二萬|{
	六州冀定相并幽平驍堅堯翻}
使永昌王仁高凉王那分將之為二道掠淮泗以北徙青徐之民以實河北 癸未魏主西廵 初魯國孔熙先博學文史兼通數術有縱横才志|{
	縱子容翻}
為員外散騎侍郎不為時所知憤憤不得志父默之為廣州刺史以贓獲罪大將軍彭城王義康為救解得免|{
	為于偽翻}
及義康遷豫章|{
	義康遷見上卷十七年}
熙先密懷報効且以為天文圖䜟|{
	䜟楚譛翻}
帝必以非道晏駕由骨肉相殘江州應出天子以范曄志意不滿欲引與同謀而熙先素不為曄所重太子中舍人謝綜曄之甥也|{
	大子中舍人晉咸寧四年置以舍人才學美者為之與中庶子共掌文翰職如黄門侍郎在中庶子下洗馬上}
熙先傾身事之綜引熙先與曄相識熙先家饒於財數與曄博故為拙行以物輸之|{
	數所角翻凡博弈以計數誘人謂之行拙行者偽為不能也行下孟翻}
曄既利其財又愛其文藝由是情好欵洽熙先乃從容說曄曰大將軍英斷聰敏|{
	好呼到翻從于容翻斷丁亂翻大將軍謂義康}
人神攸屬|{
	屬之欲翻}
失職南垂|{
	謂遷豫章也}
天下憤怨小人受先君遺命以死報大將軍之德頃人情騷動天文舛錯此所謂時運之至不可推移者也若順天人之心結英豪之士表裏相應於肘腋|{
	腋音亦}
然後誅除異我崇奉明聖號令天下誰敢不從小人請以七尺之軀三寸之舌立功立事而歸諸君子丈人以為何如曄甚愕然熙先曰昔毛玠竭節於魏武張温畢議於孫權彼二人者皆國之俊乂豈言行玷缺然後至於禍辱哉皆以廉直勁正不得久容|{
	毛玠見六十七卷漢獻帝建安二十一年張温事見六十九卷魏文帝黄初五年行下孟翻下内行同玷多忝翻}
丈人之於本朝不深於二主人間雅譽過于兩臣讒夫側目為日久矣比肩競逐庸可遂乎|{
	言與時貴比肩競逐榮利所在衆所共爭將不得遂其志也朝直遥翻}
近者殷鐵一言而劉班碎首|{
	見上卷十七年}
彼豈父兄之讐百世之怨乎所爭不過榮名勢利先後之間耳及其末也唯恐陷之不深發之不早戮及百口猶曰未厭是可為寒心悼懼豈書籍遠事也哉今建大勲奉賢哲圖難于易|{
	易以䜴翻}
以安易危享厚利收鴻名一旦苞舉而有之豈可棄置而不取哉曄猶疑未决熙先曰又有過於此者愚則未敢道耳曄曰何謂也熙先曰丈人弈葉清通|{
	曄曾祖汪祖寧父泰皆冇名行}
而不得連姻帝室人以犬豕相遇而丈人曾不恥之欲為之死不亦惑乎|{
	為于偽翻}
曄門無内行故熙先以此激之曄默然不應反意乃决曄與沈演之並為帝所知曄先至必待演之俱入演之先至嘗獨被引|{
	被皮義翻引引見也}
曄以此為怨曄累經義康府佐中間獲罪于義康謝綜及父述皆為義康所厚綜弟約娶義康女綜為義康記室參軍自豫章還申義康意于曄求解晚隙復敦往好|{
	復扶又翻好呼到翻}
大將軍府史仲承祖有寵于義康聞熙先有謀密相結納丹楊尹徐湛之素為義康所愛承祖因此結事湛之告以密計道人法略尼法静皆感義康舊恩並與熙先往來法静妹夫許曜領隊在臺|{
	江南謂禁中為臺}
許為内應法静之豫章熙先付以牋書陳說圖䜟於是密相署置及素所不善者並入死目|{
	條分名目凡素所不善者皆欲置之死地}
熙先又使弟休先作檄文稱賊臣趙伯符肆兵犯蹕禍流儲宰|{
	趙伯符時為領軍將軍故欲以弑逆之罪歸之言禍流儲宰蓋欲併殺太子劭}
湛之曄等投命奮戈即日斬伯符首及其黨與今遣護軍將軍臧質奉璽綬迎彭城王正位辰極|{
	北辰為天極故以帝位為辰極璽斯氏翻綬音受}
熙先以為舉大事宜須以義康之旨諭衆曄又詐作義康與湛之書令誅君側之惡宣示同黨帝之燕武帳岡也曄等謀以其日作亂許曜侍帝扣刀目曄|{
	拔刀微出削為扣刀}
曄不敢仰視俄而座散徐湛之恐事不濟密以其謀白帝帝使湛之具探取本末|{
	探吐南翻}
得其檄書選署姓名上之|{
	上時掌翻}
帝乃命有司收掩窮治|{
	治直之翻}
其夜呼曄置客省|{
	客省凡四方之客入見者居之屬典客令}
先於外收綜及熙先兄弟皆款服帝遣使詰問曄曄猶隱拒|{
	詰去吉翻}
熙先聞之笑曰凡處分符檄書疏皆范所造|{
	處昌呂翻分扶問翻}
云何於今方作如此抵蹋邪帝以曄墨迹示之乃具陳本末明日仗士送付廷尉|{
	仗士士之執兵仗者}
熙先望風吐款辭氣不橈|{
	橈奴敎翻}
上奇其才遣人慰勉之曰以卿之才而滯于集書省|{
	散騎侍郎集書省官也蕭子顯曰自散騎侍郎及通直員外給事中奉朝請駙馬都尉皆集書省職也}
理應有異志此乃我負卿也又責前吏部尚書何尚之曰使孔熙先年將三十作散騎郎那不作賊熙先於獄中上書謝恩且陳圖䜟深戒上以骨肉之禍|{
	䜟楚譛翻}
曰願勿遺弃存之中書若囚死之後或可追録庶九泉之下少塞舋責|{
	少詩沼翻塞悉則翻舋許覲翻}
曄在獄為詩曰雖無嵇生琴庶同夏侯色|{
	嵇康為晉文王所殺臨命顧視日影索琴而彈夏侯玄為晉景王所殺及赴東市顔色不變}
曄本意謂入獄即死而上窮治其獄遂經二旬曄更有生望獄吏戲之曰外傳詹事或當長繫曄聞之驚喜綜熙先笑之曰詹事疇昔攘袂瞋目躍馬顧盼自以為一世之雄今擾攘紛紜畏死乃爾|{
	曄為太子詹事故稱之瞋七人翻}
設令賜以性命人臣圖主何顔可以生存十二月乙未曄綜熙先及其子弟黨與皆伏誅曄母至市涕泣責曄以手擊曄頸曄顔色不怍|{
	怍疾各翻慙也}
妺及妓妾來别曄悲涕流漣|{
	妓渠綺翻漣泣下貌}
綜曰舅殊不及夏侯色曄收淚而止謝約不豫逆謀見兄綜與熙先遊常諫之曰此人輕事好奇不近於道果鋭無檢|{
	言無檢束也好呼到翻近其靳翻}
未可與狎綜不從而敗綜母以子弟自蹈逆亂獨不出視曄語綜曰姊今不來勝人多矣收籍曄家樂器服玩並皆珍麗妓妾不勝珠翠|{
	語牛倨翻不勝音升}
母居止單陋唯有一㕑盛樵薪|{
	盛時征翻}
弟子冬無被叔父單布衣

裴子野論曰夫有逸羣之才必思冲天之據|{
	冲與翀同上飛也古語云一飛冲天}
蓋俗之量則憤常均之下|{
	常均猶言平常也}
其能守之以道將之以禮殆為鮮乎|{
	鮮息淺翻}
劉弘仁范蔚宗|{
	劉湛字弘仁范曄字蔚宗蔚於勿翻}
皆忸志而貪權|{
	忸女九翻驕也玩也狎也}
矜才以徇逆累葉風素一朝而隕嚮之所謂智能翻為亡身之具矣

徐湛之所陳多不盡為曄等辭所連引上赦不問臧質熹之子也|{
	臧熹臧燾之弟質在戚里於帝為中表之親}
先為徐兖二州刺史與曄厚善曄敗以為義興太守有司奏削彭城王義康爵收付廷尉治罪|{
	治直之翻}
丁酉詔免義康及其男女皆為庶人絶屬籍徙付安成郡以寧朔將軍沈邵為安成相領兵防守|{
	相息亮翻}
邵璞之兄也義康在安成讀書見淮南厲王長事廢書歎曰自古有此我乃不知得罪為宜也庚戌以前豫州刺史趙伯符為護軍將軍伯符孝穆皇后之弟子也|{
	高祖母趙氏追諡孝穆皇后后弟倫之}
初江左二郊無樂宗廟雖有登歌亦無二舞是歲南郊始設登歌|{
	禮記郊特牲曰奠酬而工升歌發德也歌者在上匏竹在下貴人聲也二舞文舞武舞也}
魏安南平南府移書兖州|{
	安南平南二將軍府}
以南國僑置諸州多濫北境名號又欲遊獵具區|{
	周官職方氏揚州藪曰具區師古曰具區在吳}
兖州答移曰必若因土立州則彼立徐揚豈有其地復知欲遊獵具區|{
	復扶又翻}
觀化南國開館飾邸則有司存呼韓入漢厥儀未泯饋餼之秩每存豐厚|{
	饋餉也饋餼餉客以生食及芻米也詩傳曰牲腥曰餼餼許既翻}


二十三年春正月庚申尚書左僕射孟顗罷 戊辰魏主軍至東雍州|{
	雍於用翻}
臨薛永宗壘崔浩曰永宗未知陛下自來衆心縱弛今北風迅疾宜急擊之魏主從之庚午圍其壘永宗出戰大敗與家人皆赴汾水死|{
	據南史薛安都傳諸薛家于河東汾隂世為強族}
其族人安都先據弘農棄城來奔辛未魏主南如汾隂濟河至洛水橋|{
	此華隂之洛水史記秦孝公之元年所謂魏築長城自鄭濱洛者也}
聞蓋吳在長安北帝以渭北地無穀草欲渡渭南循渭而西以問崔浩對曰夫擊蛇者先擊其首首破則尾不能掉|{
	掉徒弔翻}
今蓋吳營去此六十里輕騎趨之|{
	騎奇寄翻趨七喻翻}
一日可到到則破之必矣破吳南向長安亦不過一日一日之乏未至有傷若從南道則吳徐入北山猝未可平帝不從自渭南向長安庚辰至戲水|{
	戲許宜翻}
吳衆聞之悉散入北地山|{
	地字衍}
軍無所獲帝悔之二月丙戌帝至長安丙申如盩厔|{
	盩厔音舟窒}
歷陳倉還如雍城|{
	雍於用翻下同}
所過誅民夷與蓋吳通謀者乙拔等諸軍大破蓋吳于杏城吳復遣使上表求援|{
	使疏吏翻下以義推}
詔以吳為都督關隴諸軍事雍州刺史北地公使雍梁二州兵屯境上為吳聲援遣使賜吳印一百二十一紐使吳随宜假授 初林邑王范陽邁雖遣使入貢而寇盜不絶使貢亦薄陋帝遣交州刺史檀和之討之南陽宗慤家世儒素|{
	慤叔父少文高尚不仕諸子羣從皆愛好墳典}
慤獨好武事|{
	好呼到翻}
常言願乘長風破萬里浪及和之伐林邑慤自奮請從軍詔以慤為振武將軍和之遣慤為前鋒陽邁聞軍出遣使請還所掠日南民輸金一萬斤銀十萬斤帝詔和之若陽邁果有款誠亦許其歸順和之至朱梧戍|{
	朱梧縣自漢以來屬日南郡時于其地置戍宋白曰漢日南郡治朱吾又南行四百餘里至林邑國}
遣府戶曹參軍姜仲基等詣陽邁|{
	府者交州刺史府}
陽邁執之和之乃進軍圍林邑將范扶龍於區粟城|{
	水經注盧容水出日南盧容縣區粟城南高山東逕區粟城北林邑兵器戰具悉在城中將即亮翻}
陽邁遣其將范毗沙達救之宗慤潛兵迎擊毗沙達破之 魏主與崔浩皆信重寇謙之奉其道浩素不喜佛法|{
	喜許記翻}
每言於魏主以為佛法虚誕為世費害宜悉除之及魏主討蓋吳至長安入佛寺沙門飲從官酒從官入其室|{
	飲於鴆翻從才用翻}
見大有兵器出以白帝帝怒曰此非沙門所用必與蓋吳通謀欲為亂耳命有司案誅闔寺沙門閲其財產大得釀具及州郡牧守富人所寄藏物以萬計|{
	守式又翻}
又為窟室以匿婦女|{
	窟苦骨翻}
浩因說帝悉誅天下沙門毁諸經像|{
	說輸芮翻}
帝從之寇謙之與浩固爭浩不從先盡誅長安沙門焚毁經像并敕留臺下四方令一用長安法|{
	魏主出征太子居守故謂平城為留臺下遐稼翻}
詔曰昔後漢荒君信惑邪偽以亂天常|{
	佛法自漢明帝時入中國楚王英㝡先好之至桓帝始事浮屠}
自古九州之中未嘗有此夸誕大言不本人情叔季之世莫不眩焉|{
	日無常主不辯白黑謂之眩}
由是政教不行禮義大壞九服之内鞠為丘墟|{
	鞠窮也}
朕承天緒欲除偽定真復羲農之治|{
	治直吏翻}
其一切盪除滅其蹤跡自今已後敢有事胡神及造形像泥人銅人者門誅有非常之人然後能行非常之事非朕孰能去此歷代之偽物|{
	去羌呂翻}
有司宣告征鎮諸軍刺史諸有浮圖形像及胡經皆擊破焚燒沙門無少長悉阮之|{
	少詩照翻長知兩翻}
太子晁素好佛法|{
	好呼到翻}
屢諫不聽乃緩宣詔書使遠近豫聞之得各為計沙門多亡匿獲免或牧藏經像唯塔廟在魏境者無復孑遺|{
	復扶又翻}
魏主徙長安工巧二千家於平城還至洛水分軍誅李閏叛羌 太原顔白鹿私入魏境|{
	太原郡本屬并州江左以郡人南徙者僑立太原郡晉安帝義熙中土斷立太原縣屬泰山郡元嘉十年割濟南泰山為太原郡境屬青州}
為魏人所得將殺之詐云青州刺史杜驥使其歸誠魏人送白鹿詣平城魏主喜曰我外家也|{
	魏主母杜氏故謂驥為外家}
使崔浩作書與驥且命永昌王仁高凉王那將兵迎驥攻冀州刺史申恬於歷城杜驥遣其府司馬夏侯祖歡等將兵救歷城魏人遂寇青兖冀三州至清東而還|{
	清柬清水之東也}
殺掠甚衆北邉騷動|{
	考異曰宋文帝紀三月索虜寇兖豫青冀刺史申恬破之魏太武紀二月永昌王仁至高平禽劉義隆將王章略金鄉方與遷其民五千家於河北高凉王那至濟南東平陵遷其民六千餘家於河北蓋宋魏各據奏到之月書之耳宋索虜傳又云虜破掠太原得四千餘口蓋魏人夸張其數故不同耳}
帝以魏寇為憂咨訪群臣御史中丞何承天上表以為凡備匈奴之策不過二科武夫盡征伐之謀儒生講和親之約今欲追蹤衛霍自非大田淮泗内實青徐使民有贏儲野有積穀然後發精卒十萬一舉蕩夷則不足為也若但欲遣軍追討報其侵暴則彼必輕騎奔走不肯會戰|{
	騎奇寄翻}
徒興巨費不損於彼報復之役將遂無已斯策之最末者也安邉固守於計為長臣竊以曹孫之霸才均智敵江淮之間不居各數百里|{
	曹孫謂曹操孫權也}
何者斥候之郊非耕牧之地故堅壁清野以俟其來整甲繕兵以乘其弊保民全境不出此塗要而歸之其策有四一曰移遠就近今青兖舊民及冀州新附在界首者三萬餘家可悉徙置大峴之南以實内地|{
	峴戶典翻}
二曰多築城邑以居新徙之家假其經用春夏佃牧|{
	佃讀曰田}
秋冬入保寇至之時一城千家堪戰之士不下二千其餘羸弱猶能登陴鼔譟足抗羣虜三萬矣|{
	羸倫為翻陴音疲}
三曰纂偶車牛以載糧械|{
	纂綜集也}
計千家之資不下五百耦牛為車五百兩|{
	兩力讓翻}
參合鉤連以衛其衆設使城不可固平行趨險|{
	趨七喻翻}
賊所不能干有急徵信宿可聚四曰計丁課仗凡戰士二千随其便能各自有仗素所服習銘刻由已還保輸之於庫出行請以自新|{
	請器仗各自磨礪使精新}
弓簳利鐵民不得者官以漸充之|{
	簳古旱翻}
數年之内軍用粗備矣|{
	粗坐五翻}
近郡之師遠屯清濟|{
	近郡謂南徐州所領諸僑郡及三吳近在邦域之中者濟子禮翻}
功費既重嗟怨亦深以臣料之未若即用彼衆之易也|{
	易弋䜴翻}
今因民所利導而帥之|{
	帥讀曰率}
兵彊而敵不戒國富而民不勞比於優復隊伍|{
	復方目翻}
坐食糧廩者不可同年而校矣 魏金城邉固天水梁會與秦益雜民萬餘戶據上邽東城反攻逼西城秦益二州刺史封敕文拒却之氐羌萬餘人休官屠各二萬餘人|{
	休官屠各二種屠直於翻}
皆起兵應固會敕文擊固斬之餘衆推會為主與敕文相攻 夏四月甲申魏主至長安 丁未大赦 仇池人李洪聚衆自言應王梁會求救於氐王楊文德文德曰兩雄不並立若須我者宜先殺洪會誘洪斬之|{
	誘音酉}
送首于文德五月癸亥魏主遣安豐公閭根帥騎赴上邽|{
	帥讀曰率騎奇寄翻}
未至會弃東城走敕文先掘重塹於外|{
	重直龍翻}
嚴兵守之格鬭從夜至旦敕文曰賊知無生路致死於我多殺傷士卒未易克也|{
	易以豉翻}
乃以白虎幡宣告會衆降者赦之|{
	降戶江翻}
會衆遂潰分兵追討悉平之略陽人王元達聚衆屯松多川|{
	水經注松多水出隴山西南流逕降隴城北又西南注秦水}
敕文又討平之 蓋吳收兵屯杏城自號秦地王聲勢復振|{
	復扶又翻}
魏主遣永昌王仁高凉王那督北道諸軍討之|{
	北道諸軍謂魏兵分屯長安以北者}
檀和之等拔區粟斬范扶龍乘勝入象浦|{
	象浦即盧容浦盧容縣即秦象郡象林縣地故亦謂之象浦}
林邑王陽邁傾國來戰以具裝被象前後無際|{
	馬甲謂之其裝被皮義翻}
宗慤曰吾聞外國有師子威服百獸|{
	師子似虎正黄有耏尾端茸毛大如斗爾雅翼曰穆天子傳狻猊日走五百里其為物最猛雖虎豹亦畏之象至以鼻捲泥自塗數尺數數噴鼻隅立師子直而殺之}
乃製其形與象相拒象果驚走林邑兵大敗和之遂克林邑|{
	水綬注林邑國都治典冲在夀泠縣阿賁浦西去海岸四十里 考異曰本紀在六月傳在五月當是六月賞檀和之等今從傳}
陽邁父子挺身走所獲未名之寶不可勝計宗慤一無所取還家之日衣櫛蕭然|{
	勝音升櫛側瑟翻梳枇總名}
六月癸未朔日有食之甲申魏冀相定三州兵二萬人|{
	相息亮翻}
屯長安南山諸谷以備蓋吳竄逸丙戌又司幽定冀四州兵十萬人築畿上塞圍|{
	魏都平城置司州於代都宋白曰唐雲州雲中郡是}
起上谷西至河廣縱千里|{
	廣古曠翻縱子容翻}
帝築北隄立玄武湖|{
	以其地在臺城之後故名玄武湖在今建康府上元縣北十里祝穆曰湖今為後軍寨}
築景陽山於華林園 秋七月辛未以散騎常侍杜坦為青州刺史坦驥之兄也初杜預之子耽避晉亂居河西仕張氏前秦克凉州子孫始還關中高祖滅後秦坦兄弟從高祖過江時江東王謝諸族方盛北人晚渡者朝廷悉以傖荒遇之|{
	傖助耕翻南人呼北人為傖荒言其自荒外來也}
雖復人才可施皆不得踐清塗上嘗與坦論金日磾曰恨今無復此輩人|{
	復扶又翻磾丁奚翻}
坦曰日磾假生今世養馬不暇豈辦見知上變色曰卿何量朝廷之薄也|{
	量音良}
坦曰請以臣言之臣本中華高族晉氏喪亂|{
	喪息浪翻}
播遷凉土世業相承不殞其舊直以南度不早便以荒傖賜隔日磾胡人身為牧圉乃超登内侍齒列名賢|{
	金日磾事見二十二卷漢武帝後元二年}
聖朝雖復拔才臣恐未必能也上默然 八月魏高凉王那等破蓋吳獲其二叔諸將欲送詣平城長安鎮將陸俟曰長安險固風俗豪忮|{
	將即亮翻忮支義翻狠也}
平時猶不可忽况承荒亂之餘乎今不斬吳則長安之變未巳也吳一身潜竄非其親信誰能獲之若停十萬之衆以追一人又非長策不如私許吳叔免其妻子使自追吳擒之必矣諸將咸曰今賊黨衆已散唯吳一身何所能至俟曰諸君不見毒蛇乎不斷其首猶能為害|{
	斷丁管翻}
吳天性凶狡今若得脱必自稱王者不死以惑愚民為患愈大諸將曰公言是也但得賊不殺而更遣之若遂往不返將何以任其罪俟曰此罪我為諸君任之|{
	為于偽翻任音壬}
高凉王那亦以俟計為然遂赦二叔與刻期而遣之及期吳叔不至諸將皆咎俟俟曰彼伺之未得其便耳必不負也後數日吳叔果以吳首來傳詣平城|{
	伺相吏翻傳知戀翻又直戀翻 考異曰宋索虜傳云屠各反吳自攻之為流矢所中死吳弟吾生率衆入木面山尋皆破散今從魏書}
永昌王仁討吳餘黨白廣平路那羅悉平之以陸俟為内都大官會安定盧水胡劉超等聚衆萬餘人反魏主以俟威恩著於關中復加俟都督秦雍二州諸軍事鎮長安|{
	復扶又翻下復選復還同}
謂俟曰關中奉化日淺|{
	魏主平夏始得關中}
恩信未洽吏民數為逆亂|{
	數所角翻}
今朕以重兵授卿則超等必同心協力據險拒守未易攻也|{
	易以豉翻}
若兵少則不能制賊|{
	少詩沼翻}
卿當自以方略取之俟乃單馬之鎮超等聞之大喜以俟為無能為也俟既至諭以成敗誘納超女與為姻戚以招之超自恃其衆猶無降意俟乃帥其帳下親往見超|{
	誘音酉降戶江翻帥讀曰率}
超使人逆謂俟曰從者過三百人|{
	從才用翻}
當以弓馬相待不及三百人當以酒食相供俟乃將二百騎詣超|{
	將即亮翻騎奇寄翻}
超設備甚嚴俟縱酒盡醉而還|{
	還而宣翻又如字}
頃之俟復選敢死士五百人出獵|{
	復扶又翻}
因詣超營約曰發機當以醉為限既飲俟陽醉上馬大呼|{
	呼火故翻}
手斬超首士卒應聲縱擊殺傷千數遂平之魏主徵俟還為外都大官 是歲吐谷渾復還舊土|{
	去年吐谷渾西奔}


資治通鑑卷一百二十四
