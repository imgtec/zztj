<!DOCTYPE html PUBLIC "-//W3C//DTD XHTML 1.0 Transitional//EN" "http://www.w3.org/TR/xhtml1/DTD/xhtml1-transitional.dtd">
<html xmlns="http://www.w3.org/1999/xhtml">
<head>
<meta http-equiv="Content-Type" content="text/html; charset=utf-8" />
<meta http-equiv="X-UA-Compatible" content="IE=Edge,chrome=1">
<title>資治通鑒_25-資治通鑑卷二十四_25-資治通鑑卷二十四</title>
<meta name="Keywords" content="資治通鑒_25-資治通鑑卷二十四_25-資治通鑑卷二十四">
<meta name="Description" content="資治通鑒_25-資治通鑑卷二十四_25-資治通鑑卷二十四">
<meta http-equiv="Cache-Control" content="no-transform" />
<meta http-equiv="Cache-Control" content="no-siteapp" />
<link href="/img/style.css" rel="stylesheet" type="text/css" />
<script src="/img/m.js?2020"></script> 
</head>
<body>
 <div class="ClassNavi">
<a  href="/24shi/">二十四史</a> | <a href="/SiKuQuanShu/">四库全书</a> | <a href="http://www.guoxuedashi.com/gjtsjc/"><font  color="#FF0000">古今图书集成</font></a> | <a href="/renwu/">历史人物</a> | <a href="/ShuoWenJieZi/"><font  color="#FF0000">说文解字</a></font> | <a href="/chengyu/">成语词典</a> | <a  target="_blank"  href="http://www.guoxuedashi.com/jgwhj/"><font  color="#FF0000">甲骨文合集</font></a> | <a href="/yzjwjc/"><font  color="#FF0000">殷周金文集成</font></a> | <a href="/xiangxingzi/"><font color="#0000FF">象形字典</font></a> | <a href="/13jing/"><font  color="#FF0000">十三经索引</font></a> | <a href="/zixing/"><font  color="#FF0000">字体转换器</font></a> | <a href="/zidian/xz/"><font color="#0000FF">篆书识别</font></a> | <a href="/jinfanyi/">近义反义词</a> | <a href="/duilian/">对联大全</a> | <a href="/jiapu/"><font  color="#0000FF">家谱族谱查询</font></a> | <a href="http://www.guoxuemi.com/hafo/" target="_blank" ><font color="#FF0000">哈佛古籍</font></a> 
</div>

 <!-- 头部导航开始 -->
<div class="w1180 head clearfix">
  <div class="head_logo l"><a title="国学大师官网" href="http://www.guoxuedashi.com" target="_blank"></a></div>
  <div class="head_sr l">
  <div id="head1">
  
  <a href="http://www.guoxuedashi.com/zidian/bujian/" target="_blank" ><img src="http://www.guoxuedashi.com/img/top1.gif" width="88" height="60" border="0" title="部件查字,支持20万汉字"></a>


<a href="http://www.guoxuedashi.com/help/yingpan.php" target="_blank"><img src="http://www.guoxuedashi.com/img/top230.gif" width="600" height="62" border="0" ></a>


  </div>
  <div id="head3"><a href="javascript:" onClick="javascript:window.external.AddFavorite(window.location.href,document.title);">添加收藏</a>
  <br><a href="/help/setie.php">搜索引擎</a>
  <br><a href="/help/zanzhu.php">赞助本站</a></div>
  <div id="head2">
 <a href="http://www.guoxuemi.com/" target="_blank"><img src="http://www.guoxuedashi.com/img/guoxuemi.gif" width="95" height="62" border="0" style="margin-left:2px;" title="国学迷"></a>
  

  </div>
</div>
  <div class="clear"></div>
  <div class="head_nav">
  <p><a href="/">首页</a> | <a href="/ShuKu/">国学书库</a> | <a href="/guji/">影印古籍</a> | <a href="/shici/">诗词宝典</a> | <a   href="/SiKuQuanShu/gxjx.php">精选</a> <b>|</b> <a href="/zidian/">汉语字典</a> | <a href="/hydcd/">汉语词典</a> | <a href="http://www.guoxuedashi.com/zidian/bujian/"><font  color="#CC0066">部件查字</font></a> | <a href="http://www.sfds.cn/"><font  color="#CC0066">书法大师</font></a> | <a href="/jgwhj/">甲骨文</a> <b>|</b> <a href="/b/4/"><font  color="#CC0066">解密</font></a> | <a href="/renwu/">历史人物</a> | <a href="/diangu/">历史典故</a> | <a href="/xingshi/">姓氏</a> | <a href="/minzu/">民族</a> <b>|</b> <a href="/mz/"><font  color="#CC0066">世界名著</font></a> | <a href="/download/">软件下载</a>
</p>
<p><a href="/b/"><font  color="#CC0066">历史</font></a> | <a href="http://skqs.guoxuedashi.com/" target="_blank">四库全书</a> |  <a href="http://www.guoxuedashi.com/search/" target="_blank"><font  color="#CC0066">全文检索</font></a> | <a href="http://www.guoxuedashi.com/shumu/">古籍书目</a> | <a   href="/24shi/">正史</a> <b>|</b> <a href="/chengyu/">成语词典</a> | <a href="/kangxi/" title="康熙字典">康熙字典</a> | <a href="/ShuoWenJieZi/">说文解字</a> | <a href="/zixing/yanbian/">字形演变</a> | <a href="/yzjwjc/">金 文</a> <b>|</b>  <a href="/shijian/nian-hao/">年号</a> | <a href="/diming/">历史地名</a> | <a href="/shijian/">历史事件</a> | <a href="/guanzhi/">官职</a> | <a href="/lishi/">知识</a> <b>|</b> <a href="/zhongyi/">中医中药</a> | <a href="http://www.guoxuedashi.com/forum/">留言反馈</a>
</p>
  </div>
</div>
<!-- 头部导航END --> 
<!-- 内容区开始 --> 
<div class="w1180 clearfix">
  <div class="info l">
   
<div class="clearfix" style="background:#f5faff;">
<script src='http://www.guoxuedashi.com/img/headersou.js'></script>

</div>
  <div class="info_tree"><a href="http://www.guoxuedashi.com">首页</a> > <a href="/SiKuQuanShu/fanti/">四库全书</a>
 > <h1>资治通鉴</h1> <!--         下载:【右键另存为】即可 --></div>
  <div class="info_content zj clearfix">
  
<div class="info_txt clearfix" id="show">
<center style="font-size:24px;">25-資治通鑑卷二十四</center>
    資治通鑑卷二十四   宋 司馬光 撰<br />
<br />
  胡三省 音註<br />
<br />
  漢紀十六【起強圉協洽盡昭陽赤奮若凡七年】<br />
<br />
  孝昭皇帝下<br />
<br />
  元平元年春二月詔減口賦錢什三【如淳曰漢儀注民年七歲至十四出口賦錢人二十三二十錢以食天子其三錢者武帝加口錢以補車騎馬】 夏四月癸未帝崩于未央宫【臣瓚曰夀二十三】無嗣時武帝子獨有廣陵王胥大將軍光與羣臣議所立咸持廣陵王王本以行失道先帝所不用光内不自安郎有上書言周太王廢太伯立王季文王舍伯邑考立武王【師古曰太伯者王季之兄伯邑考文王長子也舍讀曰捨】唯在所宜雖廢長立少可也【長知兩翻少時照翻】廣陵王不可以承宗廟言合光意光以其書示丞相敞等擢郎為九江太守【九江郡屬揚州唐濠夀廬滁和州地守式又翻】即日承皇后詔遣行大鴻臚事少府樂成宗正德光禄大夫吉中郎將利漢【樂成史樂成德劉德吉丙吉利漢不知其姓】迎昌邑王賀乘七乘傳【文帝之入立也乘六乘傳今乘七乘傳傳張戀翻】詣長安邸【諸王國皆置邸長安此謂長安之昌邑邸也】光又白皇后徙右將軍安世為車騎將軍賀昌邑哀王之子也【哀王名髆武帝子也】在國素狂縱動作無節武帝之喪賀游獵不止嘗游方與【方與縣本屬山陽郡武帝以山陽為昌邑王國方與縣屬焉方音房與音豫】不半日馳二百里中尉琅邪王吉上疏諫曰大王不好書術而樂逸游【好呼到翻樂五孝翻】馮式撙衘【馮讀曰憑臣瓚曰撙促也師古曰撙挫也音子本翻】馳騁不止口倦虖叱咤【師古曰咤亦吒字也音竹駕翻】手苦於箠轡【師古曰箠馬策】身勞虖車輿朝則冒霧露【師古曰冒莫北翻犯也】晝則被塵埃【被皮義翻】夏則為大暑之所㬥炙【暴步木翻】冬則為風寒之所匽薄【師古曰匽與偃同言遇疾風則偃靡也薄言廹也】數以耎脆之玉軆【師古曰耎柔也音而兖翻脆音此芮翻】犯勤勞之煩毒非所以全夀命之宗也又非所以進仁義之隆也【師古曰宗尊也隆高也】夫廣厦之下細旃之上【師古曰廣厦大屋也旃與氈同】明師居前勸誦在後上論唐虞之際下及殷周之盛考仁聖之風習治國之道【治直之翻】訢訢焉發憤忘食【訢與欣同】日新厥德其樂豈衘橛之間哉【樂音洛下同】休則俛仰屈伸以利形【師古曰形形體也俛音免】進退步趨以實下【如淳曰今人不行則膝以下虛弱不實】吸新吐故以練臧【師古曰藏五藏也練練其氣也臧古藏字通音徂浪翻】專意積精以適神【師古曰適和也】於以養生豈不長哉大王誠留意如此則心有堯舜之志體有喬松之夀【師古曰仙人伯喬及赤松子也】美聲廣譽登而上聞則福禄其臻【師古曰臻至也】而社稷安矣皇帝仁聖至今思慕未怠【師古曰皇帝謂昭帝也言武帝晏駕未久故尚思慕】於宫館囿池弋獵之樂未有所幸大王宜夙夜念此以承聖意諸侯骨肉莫親大王大王於屬則子也於位則臣也一身而二任之責加焉恩愛行義介有不具者【行下孟翻下同孅與纎同息亷翻】於以上聞非饗國之福也王乃下令曰寡人造行不能無惰中尉甚忠數輔吾過【數所角翻】使謁者千秋賜中尉牛肉五百斤酒五石脯五束【孔穎達曰脯訓始始作即成也修訓治治之乃成鄭注腊人云薄析曰脯棰而施薑桂曰鍜修】其後復放縱自若郎中令山陽龔遂忠厚剛毅有大節【龔姓也左傳晉有大夫龔堅】内諫爭於王外責傅相引經義陳禍福至于涕泣蹇蹇亡已【爭讀曰諍相息亮翻亡古無字通師古曰蹇蹇不阿順之意易曰王臣蹇蹇】面刺王過王至掩耳起走曰郎中令善媿人【師古曰媿古愧字也媿辱也】王嘗久與騶奴宰人遊戲飲食【騶導車而撝訶者也宰人掌膳食者也騶側鳩翻】賞賜無度遂入見王涕泣䣛行左右侍御皆出涕王曰郎中令何為哭遂曰臣痛社稷危也願賜清閒竭愚王辟左右【師古曰閒讀曰閑辟音闢】遂曰大王知膠西王所以為無道亡乎【膠西王謂于王端也】王曰不知也曰臣聞膠西王有諛臣侯得王所為儗于桀紂也【儗與擬同師古曰儗比也】得以為堯舜也王說其謟諛常與寑處【說讀曰悦處昌呂翻】唯得所言以至于是【師古曰唯用得之邪言故至亡】今大王親近羣小【近其靳翻】漸漬邪惡【漸子亷翻漬疾智翻】所習存亡之機不可不慎也臣請選郎通經有行義者與王起居坐則誦詩書立則習禮容宜有益王許之遂乃選郎中張安等十人侍王居數日王皆逐去安等【去羌呂翻下同】王嘗見大白犬頸以下似人冠方山冠而無尾【方山冠以五采縠為之前高七寸後高三寸長八寸樂舞人服之冠方之冠古玩翻 考異曰昌邑王傳云無頭五行志云無尾且云不得置後之象若頸以下似人而無頭何以辨其為犬且安所施冠盖傳誤也】以問龔遂遂曰此天戒言在側者盡冠狗也【言王左右之人皆狗而冠也】去之則存不去則亡矣後又聞人聲曰熊視而見大熊左右莫見以問遂遂曰熊山野之獸而來入宫室王獨見之此天戒大王恐宫室將空危亡象也王仰天而嘆曰不祥何為數來遂叩頭曰臣不敢隱忠數言危亡之戒大王不說夫國之存亡豈在臣言哉願王内自揆度【數所角翻下同說讀曰悅度徒洛翻】大王誦詩三百五篇人事浹【浹即恊翻洽也徹也】王道備王之所行中詩一篇何等也【中竹仲翻師古曰言王所行皆不合法度王自謂當于何詩之文也】大王位為諸侯王行汙於庶人【行下孟翻師古曰汙濁穢】以存難以亡易【易以豉翻】宜深察之後又血汙王坐席王問遂遂叫然號【汙烏故翻號戶高翻】曰宫空不久妖祥數至血者隂憂象也宜畏慎自省【省悉景翻】王終不改節及徵書至夜漏未盡一刻以火發書其日中王發晡時至定陶【定陶縣為濟隂郡治所】行百三十五里侍從者馬死相望於道【從才用翻】王吉奏書戒王曰臣聞高宗諒闇三年不言【闇讀與隂同】今大王以喪事徵宜日夜哭泣悲哀而已慎毋有所發【師古曰發謂興舉衆事】大將軍仁愛勇智忠信之德天下莫不聞事孝武皇帝二十餘年未嘗有過先帝棄羣臣屬以天下【屬之欲翻】寄幼孤焉大將軍抱持幼君襁緥之中【襁負兒衣論語曰襁負其子博物志曰織縷為之廣八寸長二尺以約小兒於背上李奇曰絡也以繒布為之絡負小兒孟康曰小兒綳師古曰孟說是緥小兒衣李奇曰緥小兒大籍又齊人名小兒被為緥繦舉兩翻緥博抱翻】布政施教海内晏然雖周公伊尹無以加也今帝崩無嗣大將軍惟思可以奉宗廟者攀援而立大王【師古曰援引也音爰】其仁厚豈有量哉臣願大王事之敬之政事壹聽之大王垂拱南面而已願留意常以為念王至濟陽【班志濟陽縣屬陳留郡杜佑曰濟陽縣故城在曹州莬旬縣西南濟子禮翻】求長鳴雞【師古曰雞之鳴聲長者也范成大曰長鳴雞自南詔諸蠻來形矮而大鳴聲圓長一鳴半刻終日啼號不絶蠻甚貴之一雞直銀一兩邕州谿洞亦有之】道買積竹杖【文頴曰合竹作杖也】過弘農使大奴善以衣車載女子【師古曰凡言大奴者謂奴之尤長大者也善其名也】至湖使者以讓相安樂【師古曰使者長安使人也讓責也安樂史逸其姓相息亮翻樂音洛】安樂告龔遂遂入問王王曰無有遂曰即無有何愛一善以毁行義請收屬吏以湔洒大王【師古曰以善付吏也湔澣也洒濯也行下孟翻屬之欲翻下同湔子顚翻洒先禮翻】即捽善屬衛士長行法【師古曰衛士長主衛之官捽持頭也音才兀翻長知兩翻】王到覇上大鴻臚郊迎【臚陵如翻】騶奉乘輿車王使夀成御【夀成人名昌邑太僕也乘䋲證翻下同】郎中令遂參乘且至廣明東都門遂曰禮犇喪望見國都哭此長安東郭門也【廣明注見上卷元鳳元年三輔黄圖宣平門長安城東出北頭第一門其外郭名東都門】王曰我嗌痛不能哭【師古曰嗌喉咽也音益】至城門遂復言【復扶又翻】王曰城門與郭門等耳且至未央宫東闕遂曰昌邑帳在是闕外馳道北【文穎曰吊哭帳也】未至帳所有南北行道馬足未至數步大王宜下車郷闕西面伏哭盡哀止【郷讀曰嚮】王曰諾到哭如儀六月丙寅王受皇帝璽綬襲尊號【璽斯氏翻綬音受】尊皇后曰皇太后 壬申葬孝昭皇帝于平陵【平陵屬右扶風在長安西北七十里自崩至葬十日】 昌邑王既立淫戲無度昌邑官屬皆徵至長安往往超擢拜官相安樂遷長樂衛尉龔遂見安樂流涕謂曰王立為天子日益驕溢諫之不復聽今哀痛未盡【師古曰謂新居喪服】日與近臣飲酒作樂鬬虎豹召皮軒車九旒【漢大駕法駕前驅有雲䍐九斿皮軒鸞旗薛綜曰雲䍐旌旗名胡廣曰皮軒以虎皮為軒郭璞曰皮軒革車即曲禮前有士師則載虎皮師古曰皮軒之上以赤皮為重蓋今此制尚存非用虎皮飾車】驅馳東西所為誖道【孔穎逹曰走馬謂之馳策馬謂之驅誖蒲内翻師古曰乖也】古制寛大臣有隐退今去不得陽狂恐知身死為世戮柰何君陛下故相宜極諫争王夢青蠅之矢積西階東可五六石以屋版瓦覆之【師古曰版瓦大瓦也覆敷又翻】以問遂遂曰陛下之詩不云乎【以昌邑王習詩故云然蘇林曰猶言陛下所讀之詩也】營營青蠅止于藩愷悌君子毋信讒言陛下左側讒人衆多如是青蠅惡矣【師古曰惡即矢也吳越春秋云越王勾踐為吳王嘗惡即其義也】宜進先帝大臣子孫親近以為左右【近其靳翻】如不忍昌邑故人【師古曰如若也】信用讒諛必有凶咎願詭禍為福【師古曰詭反也】皆放逐之臣當先逐矣王不聽太僕丞河東張敞上書諫【班表太僕有兩丞續漢志丞一人秩千石河東郡屬并州按此時河東郡當屬司隸】曰孝昭皇帝蚤崩無嗣【師古曰蚤古早字】大臣憂懼選賢聖承宗廟東迎之日唯恐屬車之行遲【師古曰不欲斥乘輿故但言屬車耳屬之欲翻】今天子以盛年初即位天下莫不拭目傾耳觀化聽風【師古曰言改易視聽欲急聞見善政化也拭音式】國輔大臣未褒而昌邑小輦先遷【李奇曰挽輦小臣也】此過之大者也王不聽大將軍光憂懣【懣母本翻又音滿又音悶煩懣也】獨以問所親故吏大司農田延年延年曰將軍為國柱石【師古曰柱者梁下之柱石承柱之礎言大臣負國重任如屋之柱及其石也】審此人不可何不建白太后【建議而白之也】更選賢而立之光曰今欲如是於古嘗有此不【師古曰光不涉學故有此問也不讀曰否】延年曰伊尹相殷廢太甲以安宗廟後世稱其忠【師古曰商書太甲篇太甲既立不明伊尹放諸桐也】將軍若能行此亦漢之伊尹也光乃引延年給事中【給事中給事禁中也西漢以為加官】隂與車騎將軍張安世圖計【師古曰圖謀也】王出遊光禄大夫魯國夏侯勝當乘輿前諫曰天久隂而不雨臣下有謀上者陛下出欲何之【之往也】王怒謂勝為祅言【祅與妖同音於驕翻】縛以屬吏【屬之欲翻】吏白霍光光不舉法光讓安世以為泄語安世實不言乃召問勝勝對言在鴻範傳曰皇之不極厥罰常隂【漢儒作洪範傳以五事應五行皇之不極是謂不建厥罰常隂時則有下人伐上之痾皇君也極中也建立也人君貌言視聽思五事皆失不得其中則不能立萬事失在眊悖故其咎也王者承天理物雲起于山而彌于天天氣亂故其罰常隂也君亂且弱人之所叛故有下人伐上之痾也】時則有下人伐上者惡察察言【惡忌諱也察察言不敢明言之也惡烏路翻】故云臣下有謀光安世大驚以此益重經術士侍中傅嘉數進諫【數所角翻】王亦縛嘉繫獄光安世既定議乃使田延年報丞相楊敞敞驚懼不知所言汗出洽背徒唯唯而已【師古曰唯唯者恭應之辭也唯於癸翻】延年起至更衣【師古曰古者延賓必有更衣之處也更工衡翻】敞夫人遽從東廂謂敞曰此國大事今大將軍議已定使九卿來報君侯君侯不疾應與大將軍同心猶與無决先事誅矣【與讀曰豫先悉薦翻】延年從更衣還敞夫人與延年參語許諾請奉大將軍教令【師古曰三人共言故曰參語】癸巳光召丞相御史將軍列侯中二千石大夫博士會議未央宫光曰昌邑王行昏亂恐危社稷如何羣臣皆驚鄂失色【師古曰凡鄂者皆謂阻礙不依順也後字作愕其義亦同】莫敢發言但唯唯而已田延年前離席按劍【離力智翻】曰先帝屬將軍以幼孤寄將軍以天下以將軍忠賢能安劉氏也今羣下鼎沸社稷將傾且漢之傳諡常為孝者以長有天下令宗廟血食也如漢家絶祀將軍雖死何面目見先帝于地下乎今日之議不得旋踵【師古曰宜速決】羣臣後應者臣請劍斬之光謝曰九卿責光是也天下匈匈不安光當受難【師古曰受其憂責也難乃旦翻】於是議者皆叩頭曰萬姓之命在於將軍唯大將軍令【師古曰言一聽之也】光即與羣臣俱見【見賢遍翻】白太后具陳昌邑王不可以承宗廟狀皇太后乃車駕幸未央承明殿【未央宫有承明殿天子於是延儒生學士武帝責莊助曰君厭承明之廬西都賦曰承明金馬著作之庭是也】詔諸禁門母内昌邑羣臣王入朝太后還乘輦欲歸温室【晉灼曰長樂宫有温室殿三輔黄圖温室殿在未央殿北武帝建余謂長樂宫固亦有温室但漢諸帝皆居未央則此當為未央之温室也】中黄門宦者各持門扇【中黄門屬少府黄門令師古曰中黄門謂奄人居禁中在黄門之内給事者也比百石】王入門閉昌邑羣臣不得入王曰何為大將軍跪曰有皇太后詔毋内昌邑羣臣【内讀曰納】王曰徐之何乃驚人如是光使盡驅出昌邑羣臣置金馬門外車騎將軍安世將羽林騎【將即亮翻騎奇寄翻】收縳二百餘人皆送廷尉詔獄令故昭帝侍中中臣侍守王光勑左右謹宿衛卒有物故自裁【師古曰卒讀曰猝物故死也自裁謂自殺也】令我負天下有殺主名王尚未自知當廢謂左右我故羣臣從官安得罪而大將軍盡繫之乎【從才用翻師古曰安焉也余謂安得罪猶言何所得罪也】頃之有太后詔召王王聞召意恐乃曰我安得罪而召我哉太后被珠襦【被皮義翻如淳曰以珠飾襦也晉灼曰貫珠以為襦形若今革襦矣師古曰晉說是也襦汝朱翻】盛服坐武帳中侍御數百人皆持兵期門武士陛戟陳列殿下【期門屬光禄勲掌執兵送從武帝為微行與勇力之士期諸殿門故曰期門】羣臣以次上殿召昌邑王伏前聽詔光與羣臣連名奏王尚書令讀奏曰丞相臣敞等【臣敞下即連名史以等字約言之】昩死言皇太后陛下孝昭皇帝早棄天下遣使徵昌邑王典喪服斬衰【師古曰典喪言為喪主也斬哀謂縗裳下不緶直斬割之而已緶步千翻】無悲哀之心廢禮誼居道上不素食【師古曰素食菜食無肉也言王在道常肉食非居喪之制也而鄭康成解素食云平常之食失之遠矣】使從官略女子載衣車内所居傳舍始至謁見【傳張戀翻見賢遍翻】立為皇太子常私買雞豚以食受皇帝信璽行璽大行前【孟康曰漢初有三璽天子之璽自佩信璽行璽在符節臺大行前昭帝柩前也韋昭曰大行不反之辭也】就次發璽不封【師古曰璽既國器常當緘封而王於大行前受之退還所次遂爾發漏更不封之令凡人皆見言不重慎】從官更持節引内昌邑從官騶宰官奴二百餘人常與居禁闥内敖戲【更工衡翻敖讀曰傲】為書曰皇帝問侍中君卿使中御府令高昌奉黄金千斤賜君卿取十妻【師古曰昌邑之侍中名君卿也】大行在前殿發樂府樂器引内昌邑樂人擊皷歌吹作俳倡【師古曰俳優諧戲也倡樂人也倡音昌】召内泰壹宗廟樂人悉奏衆樂【鄭氏曰祭泰一樂人也余據武帝祠泰一用樂舞召歌兒作二十五弦及空侯瑟又采詩夜誦有趙代秦楚之謳宗廟樂有文德昭德文始五行之舞嘉至永至登歌休成之樂房中祠樂安世樂昭容樂禮容樂其員八百二十九人】駕法駕驅馳北宫桂宫【師古曰北宫桂宫並在未央宫北三輔黄圖桂宫武帝造周回十餘里有紫房複道通未央宫三秦記未央宫漸臺西有桂宫】弄彘鬬虎召皇太后御小馬車【張晏曰皇太后所駕遊宫中輦車也漢廐有果下馬高三尺以駕輦師古曰小馬可于果下乘之故曰果下馬】使官奴騎乘遊戲掖庭中與孝昭皇帝宫人蒙等淫亂詔掖庭令敢泄言要斬【掖庭令屬少府武帝太初元年更名本永巷令也要與腰同】太后曰止【師古曰令且止讀奏也】為人臣子當悖亂如是邪王離席伏【悖蒲内翻離力智翻】尚書令復讀曰【復扶又翻】取諸侯王列侯二千石綬及墨綬黄綬以并佩昌邑郎官者免奴【續漢志諸侯王赤綬四采青黄縹紺列侯紫綬二采紫白二千石青綬三采青白紅千石六百石墨綬三采青赤紺四百石三百石二百石黄綬師古曰免奴謂奴免放為良人者】發御府金錢刀劍玉器采繒賞賜所與遊戲者與從官官奴夜飲湛沔于酒【師古曰湛讀曰沈又讀曰耽湛沔者乃荒迷之義也沔與湎同】獨夜設九賓温室【師古曰于温室中設九賓之禮也】延見姊夫昌邑關内侯祖宗廟祠未舉為璽書使使者持節以三太牢祠昌邑哀王園廟稱嗣子皇帝【師古曰時在喪服故未祠宗廟而私祭昌邑哀王也余謂賀入繼大宗不當於昌邑哀王稱嗣子皇帝既於禮悖三年不祭之義又悖為人後者為之子之義】受璽以來二十七日使者旁午【如淳曰旁午分布也師古曰一縱一横為旁午猶言交横也】持節詔諸官署徵發凡一千一百二十七事荒淫迷惑失帝王禮誼亂漢制度臣敞等數進諫不變更【數所角翻更工衡翻】日以益甚恐危社稷天下不安臣敞等謹與博士議皆曰今陛下嗣孝昭皇帝後行淫辟不軌【辟讀曰僻】五辟之屬莫大不孝【孝經孔子曰五刑之屬三千其罪莫大于不孝辟五刑之辟也辟頻亦翻】周襄王不能事母春秋曰天王出居于鄭由不孝出之絶之於天下也【僖二十四年經書天王出居於鄭公羊傳曰王者無外此其言出何不能於母也】宗廟重于君陛下不可以承天序奉祖宗廟子萬姓當廢臣請有司以一太牢具告祠高廟皇太后詔曰可光令王起拜受詔王曰聞天子有争臣七人雖亡道不失天下【引孝經孔子之言爭讀曰諍亡古無字通】光曰皇太后詔廢安得稱天子乃即持其手解脫其璽組【師古曰即就也組則古翻說文曰組綬屬續漢志乘輿黄赤綬四采黄赤紺縹長丈有九尺九寸五百首】奉上太后【上時掌翻】扶王下殿出金馬門羣臣隨送王西面拜曰愚戇不任漢事【戇涉降翻任音壬】起就乘輿副車【乘繩證翻】大將軍光送至昌邑邸光謝曰王行自絶於天臣寧負王不敢負社稷願王自愛臣長不復左右【師古曰言不復得侍見于左右】光涕泣而去羣臣奏言古者廢放之人屏于遠方【屏必郢翻又卑正翻】不及以政【師古曰言不豫政令】請徙王賀漢中房陵縣【漢中郡屬益州房陵縣唐為房州】太后詔歸賀昌邑賜湯沐邑二千戶故王家財物皆與賀及哀王女四人各賜湯沐邑千戶國除為山陽郡【昌邑國本山陽郡也今國除復為郡】昌邑羣臣坐在國時不舉奏王罪過令漢朝不聞知【朝直遥翻】又不能輔道【道讀曰導】䧟王大惡皆下獄誅殺二百餘人【下遐嫁翻】唯中尉吉郎中令遂以忠直數諫正【數所角翻】得减死髠為城旦師王式繫獄當死治事使者責問曰師何以無諫書【王式時為昌邑王師以授王詩治事使者即治獄使者也治直之翻】式對曰臣以詩三百五篇朝夕授王至于忠臣孝子之篇未嘗不為王反復誦之也【為于偽翻下同師古曰復音方目翻】至于危亡失道之君未嘗不流涕為王深陳之也臣以三百五篇諫是以無諫書使者以聞亦得減死論霍光以羣臣奏事東宫太后省政【省悉景翻】宜知經術白令夏侯勝用尚書授太后遷勝長信少府【長信宫名少府掌其宫事班表長信詹事掌皇太后宫景帝中六年更名長信少府平帝元始四年更名長樂少府張晏曰以太后所居名也居長信宫則曰長信少府居長樂宮則曰長樂少府也三輔黄圖長信殿在長樂宫太后常居之余據表長信少府後改為長樂少府則長信長樂非兩宫也張說誤】賜爵關内侯 初衛太子納魯國史良娣【姓譜史周史佚之後師古曰太子有妃有良姊有孺子凡三等】生子進【師古曰進皇孫之名也】號史皇孫皇孫納涿郡王夫人【涿郡屬幽州王夫人名翁須】生子病已【師古曰盖以夙遭屯難而多病苦故名病已欲其速差也後以為鄙更改諱詢】號皇曾孫皇曾孫生數月遭巫蠱事【見二十三卷武帝征和二年】太子三男一女及諸妻妾皆遇害獨皇曾孫在亦坐收擊郡邸獄【師古曰漢舊儀郡邸獄治天下郡國上計者屬大鴻臚此盖巫蠱獄收繫者衆故皇曾孫寄在郡邸獄】故廷尉監魯國丙吉【班表廷尉有左右監秩千石丙姓也左傳齊有丙歜功臣表有高苑侯丙倩】受詔治巫蠱獄【治直之翻】吉心知太子無事實重哀皇曾孫無辜【師古曰重音直用翻】擇謹厚女徒渭城胡組淮陽郭徵卿令乳養曾孫置閒燥處【李奇曰輕罪男子守邉一歲女子輭弱不任守復令作于官亦一歲故班史謂之女徒復作復作者復為官作滿其本罪月日班志渭城縣屬扶風師古曰閒寛净之處也燥高敞也閒讀曰閑燥蘇老翻】吉日再省視【省悉景翻】巫蠱事連歲不决武帝疾來往長楊五柞宫【師古曰二宫並在盩厔皆以木名之水經注漏水出南山赤谷東北流逕長楊宫漏水又東北耿谷水注之水發南山耿谷北流與柳泉合東北逕五柞宫】望氣者言長安獄中有天子氣於是武帝遣使者分條中都官詔獄繫者【師古曰條謂疏録之】無輕重一切皆殺之内謁者令郭穰夜到郡邸獄【班表謁者令屬少府續漢志主宫中布張諸褻物漢官云秩千石盖當時權為此使】吉閉門拒使者不納曰皇曾孫在他人無辜死者猶不可况親曾孫乎相守至天明不得入穰還以聞因劾奏吉【劾戶槩翻】武帝亦寤曰天使之也因赦天下郡邸獄繫者獨賴吉得生既而吉謂守丞誰如皇孫不當在官【孟康曰郡守丞也來詣京師邸治獄姓誰名如文穎曰不當在官不當在郡邸獄也師古曰守丞守獄官之丞耳非郡丞也誰如者其人名本作譙字言姓又非也仲馮曰守丞盖郡邸守邸之丞也與朱買臣傳守丞同】使誰如移書京兆尹遣與胡組俱送京兆尹不受復還及組日滿當去皇孫思慕吉以私錢雇組令留與郭徵卿並養數月乃遣組去後少内嗇夫白吉曰食皇孫無詔令【師古曰少内掖庭主府藏之官也食讀曰飤詔令無文無從得其廪具而食之】時吉得食米肉月月以給皇曾孫曾孫病幾不全者數焉吉數敕保養乳母加致醫藥【幾居衣翻數所角翻】視遇甚有恩惠吉聞史良娣有母貞君及兄恭乃載皇曾孫以付之貞君年老見孫孤甚哀之自養視焉後有詔掖庭養視上屬籍宗正【應劭曰掖庭宫人之官有令丞宦者為之詔敕掖庭養視之始令宗正著其屬籍】時掖庭令張賀嘗事戾太子思顧舊恩【張賀安世兄也幸於衛太子太子敗賓客皆誅安世上書為賀請得下蠶室後為掖庭令師古曰顧念也】哀曾孫奉養甚謹以私錢供給教書既壯賀欲以女孫妻之【妻千細翻下同】是時昭帝始冠【冠古玩翻】長八尺二寸【長直亮翻】賀弟安世為右將軍輔政聞賀稱譽皇曾孫欲妻以女【譽音余】怒曰曾孫乃衛太子後也幸得以庶人衣食縣官足矣勿復言予女事【復扶又翻予讀曰與】於是賀止時暴室嗇夫許廣漢有女【暴室屬掖庭令師古曰取暴曬為名盖主織作染練之署應劭曰暴室宮人獄也今曰薄室許廣漢坐法腐為宦者作嗇夫也師古又曰暴室職務既多因為置獄主治罪人故往往云暴室獄耳然非獄名嗇夫者暴室屬官亦猶縣郷嗇夫姓譜許姓出高陽本自姜姓炎帝之後太嶽之裔其後因封國為氏】賀乃置酒請廣漢酒酣為言曾孫體近下乃關内侯【師古曰言曾孫於帝為近親縱其人得下劣猶為關内侯也為于偽翻】可妻也廣漢許諾明日嫗聞之怒【嫗謂廣漢妻也說文曰嫗母也音威遇翻】廣漢重令人為介【師古曰更令人作媒結婚姻重音直用翻】遂與曾孫賀以家財聘之曾孫因依倚廣漢兄弟及祖母家史氏受詩于東海澓中翁【服䖍曰澓音福師古曰姓澓字中翁 澓房福翻中讀曰仲】高材好學【好呼到翻】然亦喜游俠【師古曰喜許吏翻】鬬雞走狗以是具知閭里姦邪吏治得失【治直吏翻】數上下諸陵【師古曰諸陵皆據高敞地為之縣即在其側帝每周游往來去則上來則下故言上下諸陵數所角翻上時掌翻】周徧三輔嘗困于蓮勺鹵中【班志連勺縣属左馮翊賢曰故城在同州下邽縣東北如淳曰為人所困辱也蓮勺縣有鹽池縱横十餘里其郷人名為鹵中師古曰鹵者鹹地今在櫟陽縣東今其郷人謂此中為鹵鹽池程大昌曰蓮勺唐下邽縣蓮音輦勺音酌】尤樂杜鄠之間【班志杜縣屬京兆鄠縣屬扶風樂音洛鄠音戶】率常在下杜【孟康曰下杜在長安南師古曰即今之杜城括地志下杜城在雍州長安縣東南九里古杜伯國】時會朝請舍長安尚冠里【文穎曰以屬弟尚親故歲時從宗室朝會也如淳曰春曰朝秋曰請師古曰尚冠者長安中里名帝會朝請之時即于此里中止息三輔黄圖曰京兆尹治尚冠里朝直遥翻舍如字請才性翻】及昌邑王廢霍光與張安世諸大臣議所立未定丙吉奏記光曰將軍事孝武皇帝受襁褓之屬任天下之寄【屬之欲翻】孝昭皇帝早崩亡嗣【亡古無字通】海内憂懼欲亟聞嗣主發喪之日以大誼立後所立非其人復以大誼廢之【師古曰雖無嫡嗣旁立支屬今宗廟有奉既而恐危社稷故廢黜之皆以大誼而行也】天下莫不服焉方今社稷宗廟羣生之命在將軍之壹舉竊伏聽於衆庶察其所言諸侯宗室在列位者未有所聞于民間也而遺詔所養武帝曾孫名病已在掖庭外家者【蘇林曰外家猶言在外人民家不在宫中晉灼曰出郡邸獄歸在外家史氏後入掖庭耳師古曰晉說是也】吉前使居郡邸時【使疏吏翻】見其幼少至今十八九矣通經術有美材行安而節和【行下孟翻】願將軍詳大義參以蓍龜豈宜【句斷言參以蓍龜卜其宜與不宜也】褒顯先使入侍【師古曰侍太后】令天下昭然知之然後决定大策天下幸甚杜延年亦知曾孫德美勸光安世立焉秋七月光坐庭中會丞相以下議所立遂復與丞相敞等上奏曰【復扶又翻上時掌翻】孝武皇帝曾孫病已年十八師受詩論語孝經躬行節儉慈仁愛人可以嗣孝昭皇帝後奉承祖宗廟子萬姓【師古曰天子以萬姓為子故云子萬姓】臣昧死以聞【昧死冒死也】皇太后詔曰可光遣宗正德至曾孫家尚冠里洗沐賜御衣太僕以軨獵車迎曾孫【文穎曰軨獵小車前有曲輿不衣近世謂之軨獵車孟康曰今之載獵車也前有曲軨特高大獵時立其中格射禽獸李奇曰蘭輿輕車也師古曰文李二說是時未備天子車駕故且取其輕便耳非取其高大也孟說失之軨音零】就齋宗正府庚申入未央宫見皇太后封為陽武侯【班志陽武縣屬河南郡師古曰先封侯者不欲立庶人為天子也見賢遍翻】已而羣臣奏上璽綬即皇帝位【癸巳廢昌邑王庚申立寅帝漢朝無君二十七日天下不揺霍光處此誠難能也上時掌翻】謁高廟尊皇太后為太皇太后侍御史嚴延年【班表侍御史屬御史大夫員十五人受公卿奏事舉劾按章此嚴非莊助之嚴自是一姓戰國時有濮陽嚴仲子】劾奏大將軍光擅廢立主無人臣禮不道奏雖寑然朝廷肅然敬憚之 八月己巳安平敬侯楊敞薨【班表安平侯食邑于汝南】 九月大赦天下 戊寅蔡義為丞相 初許廣漢女適皇曾孫一歲生子奭數月曾孫立為帝許氏為倢伃是時霍將軍有小女與皇太后親公卿議更立皇后皆心擬霍將軍女亦未有言上乃詔求微時故劒大臣知指白立許倢伃為皇后十一月壬子立皇后許氏霍光以后父廣漢刑人不宜君國歲餘乃封為昌成君 太皇太后歸長樂宫長樂宫初置屯衛【漢太后常居長樂宫太皇太后自昌邑之廢居未央宫今宣帝既立復歸長樂宮樂音洛】<br />
<br />
  中宗孝宣皇帝上之上【荀悅曰諱詢字次郷諱詢之字曰謀應劭曰諡法聖善周閒曰宣余據帝本名病已元康二年乃更名詢】<br />
<br />
  本始元年春詔有司論定策安宗廟功大將軍光益封萬七千戶與故所食凡二萬戶車騎將軍富平侯安世以下益封者十人封侯者五人賜爵關内侯者八人【昭帝始元二年霍光以捕馬何羅功封博陸侯二千三百五十戶今益封萬七千二百戶元鳳六年張安世封富平侯三千四十戶今益封萬六百戶楊敞始封安平侯七百戶今益封其子忠四千八百四十七戶蔡義始封陽平侯今益封通前凡七百戶范明友始封平陵侯今益封通前凡二千九百二十戶韓增始紹封龍雒侯今益封千戶建平侯杜延年始封二千戶今益封二千三百六十戶蒲侯蘇昌始封千二十六戶今益封王譚始紹封宜春侯今益封通前凡一千一百八戶魏聖始紹封當塗侯今益封通前凡二千二百戶屠耆堂始紹封杜侯千三百戶今益封夏侯勝始賜爵關内侯今益封千戶凡十人封田廣明為昌水侯趙充國為營平侯田延年為陽城侯樂成為爰氏侯王遷為平丘侯凡五人周德蘇武李光劉德韋賢宋畸丙吉趙廣漢八人皆賜爵關内侯】 大將軍光稽首歸政【稽音啟】上謙讓不受諸事皆先關白光然後奏御自昭帝時光子禹及兄孫雲皆為中郎將雲弟山奉車都尉侍中領胡越兵光兩女壻為東西宫衛尉昆弟諸壻外孫皆奉朝請為諸曹大夫騎都尉給事中黨親連體根據於朝廷及昌邑王廢光權益重每朝見上虛已歛容禮下之已甚【胡越兵胡騎及越騎也東西宫衛尉長樂衛尉及未央衛尉也侍中得入禁中諸曹受尚書奏事給事中給事禁中皆加官也下胡稼翻已甚言過當也】 夏四月庚午地震 五月鳳皇集膠東千乘赦天下勿收田租賦 六月詔曰故皇太子在湖未有號諡【戾太子死事見二十二卷武帝征和二年】歲時祠其議諡置園邑有司奏請禮為人後者為之子也故降其父母不得祭【師古曰謂本生之父母也】尊祖之義也陛下為孝昭帝後承祖宗之祀愚以為親諡宜曰悼【如淳曰親謂父也】母曰悼后故皇太子諡曰戾史良娣曰戾夫人【諡法不悔前過曰戾又不思念曰戾】皆改葬焉 秋七月詔立燕刺王太子建為廣陽王【燕王旦死建為庶人事見二十三卷昭帝元鳳元年廣陽國屬幽州旦死燕國除為廣陽郡今因以為國名刺音來曷翻】立廣陵王胥少子弘為高密王【封胥子弘為王加親親之恩也】初上官桀與霍光争權光既誅桀遂遵武帝法度以刑罰痛䋲羣下由是俗吏皆尚嚴酷以為能而河南太守丞淮陽黄霸獨用寛和為名上在民間時知百姓苦吏急也聞覇持法平乃召為廷尉正數决疑獄庭中稱平【廷尉正秩千石庭中漢書作廷中師古曰此廷中謂廷尉之中也余謂通鑑作庭中言漢庭之中也數所角翻】二年春大師農田延年有罪自殺昭帝之喪大司農僦民車延年詐增僦直【師古曰僦謂賃之與雇直也僦子就翻】盗取錢三千萬為怨家所告【怨於元翻】霍將軍召問延年欲為道地【師古曰為之開通道路使有安全之地也】延年抵曰【師古曰抵拒諱也抵丁禮翻】無有是事光曰既無事當窮竟【師古曰既無實事當令有司窮治盡其理】御史大夫田廣明謂太僕杜延年曰春秋之義以功覆過【公羊傳僖十七年夏滅項孰滅之齊滅之昌為不言齊滅之為桓公諱也桓公嘗有存亡繼絶之功故君子為之諱】當廢昌邑王時非田子賓之言大事不成【延年字子賓事見上昭帝元平元年】今縣官出三千萬自乞之何哉【師古曰謂自乞與之也柳宗元曰哉疑辭也何哉猶曰何如也乞音氣】願以愚言白大將軍延年言之大將軍大將軍曰誠然實勇士也當發大議時震動朝廷光因舉手自撫心曰使我至今病悸【師古曰悸心動也音揆韻畧其季翻】謝田大夫曉大司農通往就獄得公議之【師古曰曉者告白意指也通者從公家通理也光忿其拒諱故不佑之】田大夫使人語延年【語牛倨翻】延年曰幸縣官寛我耳何面目入牢獄使衆人指笑我卒徒唾吾背乎即閉閣獨居齋舍偏袒持刀東西步數日使者召延年詣廷尉聞鼓聲自刎死【晉灼曰使者至司農司農發詔書故鳴鼔也師古曰刎謂斷頸也刎武粉翻】 夏五月詔曰孝武皇帝躬仁誼厲威武功德茂盛而廟樂未稱【師古曰稱副也稱尺證翻】朕甚悼焉其與列侯二千石博士議于是羣臣大議庭中【師古曰大議總會議也此庭中謂朝廷之中】皆曰宜如詔書長信少府夏侯勝獨曰武帝雖有攘四夷廣土境之功然多殺士衆竭民財力奢泰無度天下虛耗百姓流離物故者半蝗蟲大起赤地數千里【師古曰言無五穀之苖】或人民相食畜積至今未復【畜讀曰蓄】無德澤于民不宜為立廟樂公卿共難勝曰【為于偽翻難乃旦翻】此詔書也勝曰詔書不可用也人臣之誼宜直言正論非苟阿意順旨議已出口雖死不悔於是丞相御史劾奏勝非議詔書毁先帝不道及丞相長史黄霸阿縱勝不舉劾俱下獄【劾戶槩翻下遐嫁翻】有司遂請尊孝武帝廟為世宗廟奏盛德文始五行之舞【應劭曰宣帝復采昭德之舞為盛德舞以尊世宗廟也諸帝廟皆常奏文始四時五行舞也】武帝廵狩所幸郡國皆立廟如高祖太宗焉夏侯勝黄霸既久繫霸欲從勝受尚書勝辭以罪死霸曰朝聞道夕死可矣【論語載孔子之言】勝賢其言遂授之繫再更冬【師古曰更歷也更音工衡翻】講論不怠 初烏孫公主死漢復以楚王戊之孫解憂為公主妻岑娶岑娶胡婦子泥靡尚小岑娶且死【復扶又翻妻七細翻漢書作岑陬師古曰岑音仕林翻陬音子侯翻】以國與季父大禄子翁歸靡曰泥靡大以國歸之翁歸靡既立號肥王復尚楚主生三男兩女長男曰元貴靡次曰萬年次曰大樂昭帝時公主上書言匈奴與車師共侵烏孫唯天子幸救之漢養士馬議擊匈奴會昭帝崩上遣光禄大夫常惠使烏孫烏孫公主及昆彌皆遣使上書言匈奴復連發大兵侵擊烏孫使使謂烏孫趣持公主來【趣讀曰促】欲隔絶漢昆彌願發國精兵五萬騎盡力擊匈奴唯天子出兵以救公主昆彌先是匈奴數侵漢邉【先悉薦翻】漢亦欲討之秋大發兵遣御史大夫田廣明為祁連將軍四萬餘騎出西河度遼將軍范明友三萬餘騎出張掖前將軍韓增三萬餘騎出雲中後將軍趙充國為蒲類將軍三萬餘騎出酒泉雲中太守田順為虎牙將軍三萬餘騎出五原期以出塞各二千餘里以常惠為校尉持節護烏孫兵共擊匈奴<br />
<br />
  三年春正月癸亥恭哀許皇后崩【張晏曰禮婦人從夫諡閔其見殺故兼二諡師古曰共讀曰恭余據班史自高后以下皆從夫稱之未嘗有諡也至帝諡孝武衛皇后口思亦以其不令終也至于東都如光烈明德始從夫而加二諡】時霍光夫人顯欲貴其小女成君道無從【師古曰從因也由也無由得納其女】會許后當娠病女醫淳于衍者【姓譜淳于出於姜姓周公之後】霍氏所愛嘗入宫侍皇后疾衍夫賞為掖庭戶衛【掖庭戶衛掌衛掖庭門戶戶郎主之也】謂衍可過辭霍夫人行為我求安池監【安池池名監掌池之官為于偽翻】衍如言報顯顯因心生辟左右【師古曰辟謂屛去之音闢】字謂衍曰少夫幸報我以事【如淳曰稱衍字曰少夫親之也晉灼曰報我以事謂求池監也少時照翻】我亦欲報少夫可乎【晉灼曰報少夫謀弑許后事】衍曰夫人所言何等不可者【師古曰無事而不可】顯曰將軍素愛少女成君欲奇貴之願以累少夫【師古曰累託也音力瑞翻】衍曰何謂邪顯曰婦人免乳大故十死一生【師古曰免乳謂產子也大故大事也乳音人喻翻】今皇后當免身可因投毒藥去也【師古曰去謂除去皇后也音丘呂翻】成君即為皇后矣如蒙力事成富貴與少夫共之衍曰藥雜治常先嘗安可【師古曰與衆醫共雜治之又有先嘗者何可行毒治直之翻】顯曰在少夫為之耳將軍領天下誰敢言者緩急相護但恐少夫無意耳衍良久曰願盡力即擣附子【附子與天雄烏喙同出一種有大毒】齎入長定宫皇后免身後衍取附子并合太醫大丸以飲皇后【師古曰大丸今澤蘭丸之屬合音閤飲於禁翻】有頃曰我頭岑岑也藥中得無有毒【師古曰岑岑痺悶之意】對曰無有遂加煩懣崩【師古曰懣音滿又音悶】衍出過見顯相勞問【勞力到翻】亦未敢重謝衍【師古曰恐人知覺之】後人有上書告諸醫侍疾無狀者皆收繫詔獄劾不道【劾戶槩翻】顯恐急即以狀具語光【語牛據翻】因曰既失計為之無令吏急衍光大驚欲自發舉不忍猶與【師古曰猶與不决也與讀曰豫】會奏上光署衍勿論【李奇曰光題其奏也師古曰言之於帝故解釋耳光不自署也余據霍光傳光薨後帝始聞毒許后事光于是時安敢言之於帝邪李說為是上時掌翻】顯因勸光内其女入宫 戊辰五將軍發長安匈奴聞漢兵大出老弱犇走畜產遠遁逃【師古曰與驅同】是以五將少所得【少詩沼翻】夏五月軍罷度遼將軍出塞千二百餘里至蒲離候水【自張掖出塞】斬首捕虜七百餘級前將軍出塞千二百餘里至烏員【自雲中出塞師古曰烏員地名也音云】斬首捕虜百餘級蒲類將軍出塞千八百餘里西至候山【自酒泉出塞】斬首捕虜得單于使者蒲隂王以下三百餘級聞虜已引去皆不至期還天子薄其過寛而不罪祈連將軍出塞千六百里至雞秩山【自西河出塞】斬首捕虜十九級逢漢使匈奴還者冉弘等【姓譜楚大夫叔山冉之後案夫子弟子有冉伯牛冉有使疏吏翻還從宣翻又如字】言雞秩山西有虜衆祈連即戒弘使言無虜欲還兵御史屬公孫益夀諫以為不可祈連不聽遂引兵還虎牙將軍出塞八百餘里至丹餘吾水上【自五原出塞】即止兵不進斬首捕虜千九百餘級引兵還上以虎牙將軍不至期詐增鹵獲而祈連知虜在前逗遛不進【孟康曰逗遛律語也謂軍行頓止稽留不進也師古曰逗讀與住同又音豆】皆下吏自殺擢公孫益夀為侍御史【百官表侍御史員十五人受公卿奏事舉劾案章下遐嫁翻】烏孫昆彌自將五萬騎與校尉常惠從西方入至右谷蠡王庭【谷蠡音鹿黎】獲單于父行【行胡浪翻】及嫂居次【晉灼曰匈奴女號若言公主也】名王犁汗都尉千長騎將以下四萬級【犂汗都尉犂汗王之都尉也師古曰千長千人之長長知兩翻】馬牛羊驢橐佗七十餘萬頭【師古曰橐佗言能負橐囊而馱物也佗音徒河翻 考異曰常惠傳四萬級為三萬九千人七十餘萬頭為六十餘萬頭今從烏孫傳】烏孫皆自取所虜獲上以五將皆無功獨惠奉使克獲封惠為長羅侯【長羅侯國屬陳留郡賢曰故城在今滑州匡城縣東北】次匈奴民衆傷而去者及畜產遠移死亡不可勝數【勝音升】於是匈奴遂衰耗怨烏孫上復遣常惠持金幣還賜烏孫貴人有功者【復扶又翻】惠因奏請龜兹國嘗殺校尉賴丹未伏誅請便道擊之帝不許大將軍霍光風惠以便宜從事【師古曰言至前所專命而行也風讀曰諷】惠與吏士五百人俱至烏孫還【還從宣翻又如字下同】過發西國兵二萬人【自烏孫還所過西國皆發其兵】令副使發龜兹東國二萬人烏孫兵七千人從三面攻龜兹兵未合先遣人責其王以前殺漢使狀王謝曰乃我先王時為貴人姑翼所誤耳我無罪惠曰即如此縛姑翼來吾置王【師古曰置猶放】王執姑翼諸惠惠斬之而還【龜兹殺賴丹事見上卷昭帝元鳳四年】 大旱 六月己丑陽平節侯蔡義薨【陽平屬東郡 考異曰荀紀作乙丑誤】 甲辰長信少府韋賢為丞相 大司農魏相為御史大夫冬匈奴單于自將數萬騎擊烏孫頗得老弱欲還會天大雨雪【雨于具翻】一日深丈餘【深式鴆翻】人民畜產凍死還者不能什一於是丁令乘弱攻其北【令音零】烏桓入其東烏孫擊其西凡三國所殺數萬級馬數萬匹牛羊甚衆又重以餓死【重直用翻】人民死者什三畜產什五匈奴大虚弱諸國覊屬者皆瓦解攻盗不能理其後漢遣三千餘騎為三道並入匈奴捕虜得數千人還匈奴終不敢取當【師古曰當者報其直】滋欲郷和親【師古曰滋益也郷讀曰嚮】而邉境少事矣 是歲潁川太守趙廣漢為京兆尹潁川俗豪傑相朋黨廣漢為缿筩【蘇林曰缿音項如瓶可受投書孟康曰筩竹筩也如今官受密事筩也師古曰缿若今盛錢臧瓶為小孔可入而不可出或缿或筩皆為此制而用受書令投于中也筒音同】受吏民投書使相告訐【師古曰面相斥曰訐音居又翻又音居謁翻】於是更相怨咎【更工衡翻】姦黨散落盗賊不敢發匈奴降者言匈奴中皆聞廣漢名【降戶江翻】由是入為京兆尹廣漢遇吏殷勤甚備事推功善歸之於下行之發於至誠吏咸願為用僵仆無所避【師古曰僵偃也仆頓也僵音薑仆音赴】廣漢聰明皆知其能之所宜盡力與否其或負者輒收捕之無所逃案之罪立具即時伏辜尤善為鉤距以得事情【蘇林曰鉤得其情使不得去也晉灼曰鉤致也距閉也使對者無疑若不問而自知衆莫覺所由以閉其術為鉤距也師古曰晉說是也】閭里銖兩之姧皆知之長安少年數人會窮里空舍【師古曰窮里里中之極隐處】謀共刼人坐語未訖廣漢使吏捕治具服其發姦擿伏如神【師古曰擿謂動發之也音它狄翻】京兆政清吏民稱之不容口長老傳以為自漢興治京兆者莫能及【長知兩翻治直之翻】<br />
<br />
  四年春三月乙卯立霍光女為皇后赦天下初許后起微賤登至尊日淺從官車服甚節儉及霍后立轝馬侍從益盛【從才用翻】賞賜官屬以千萬計與許后時縣絶矣【縣讀曰懸】夏四月壬寅郡國四十九同日地震或山崩壞城郭<br />
<br />
  室屋殺六千餘人北海琅邪壞祖宗廟【景帝元帝令郡國各立太祖高皇帝廟太宗文皇帝廟壞音怪】詔丞相御史與列侯中二千石博問經學之士有以應變【師古曰謂禦塞災異也】毋有所諱令三輔太常内郡國舉賢良方正各一人大赦天下上素服避正殿五日釋夏侯勝黄霸以勝為諫大夫給事中霸為揚州刺史【揚州統盧江九江會稽丹陽豫章等郡】勝為人質樸守正簡易無威儀【易以䜴翻】或時謂上為君誤相字於前【師古曰前天子之前也君前臣名不當相呼字也】上亦以是親信之【師古曰知其質樸也】嘗見出道上語【師古曰入見天子而以其言為外人道之見賢遍翻】上聞而讓勝勝曰陛下所言善臣故揚之堯言布於天下至今見誦臣以為可傳故傳耳朝廷每有大議上知勝素直謂曰先生建正言無懲前事【師古曰懲創也前事謂坐議廟樂事】勝復為長信少府後遷太子太傅年九十卒太后賜錢二百萬為勝素服五日以報師傅之恩【為于偽翻】儒者以為榮 五月鳳皇集北海安丘淳于【安丘淳于二縣皆屬北海郡安丘春秋時之渠丘淳于春秋之州國】 廣川王去坐殺其師及姬妾十餘人或銷鈆錫灌口中或支解并毒藥煮之令糜盡廢徙上庸自殺【廣川王去景帝子廣川惠王越之孫師古曰糜碎也】地節元年【應劭曰以先者地震山崩水出於是改元曰地節欲令地得其節】春正月有星孛于西方【孛蒲内翻】 楚王延夀【景帝立平陸侯禮為楚王奉元王後傳子道孫注曾孫純延夀純之子也】以廣陵王胥武帝子天下有變必得立隂附助之為其後母弟趙何齊取廣陵王女為妻因使何齊奉書遺廣陵王曰願長耳目【師古曰言常伺聼勿失幾也取讀曰娶遺於季翻長如字】毋後人有天下【師古曰方争天下勿使在人後後戶覯翻】何齊父長年上書告之事下有司考驗辭服【下遐嫁翻】冬十一月延夀自殺胥勿治 十二月癸亥晦日有食之 是歲于定國為廷尉【姓譜周武王子封於䢴子孫以國為氏其後去邑單為于】定國决疑平法務在哀鰥寡罪疑從輕加審慎之心朝廷稱之曰張釋之為廷尉天下無寃民【師古曰言决罪皆當】于定國為廷尉民自以不寃【師古曰言知其寛平皆無寃枉之慮也】<br />
<br />
  二年春霍光病篤車駕自臨問上為之涕泣【為于偽翻】光上書謝恩願分國邑三千戶以封兄孫奉車都尉山為列侯奉兄去病祀【霍去病封冠軍侯子嬗嗣封薨無後國除故光乞分國邑以封其孫】即日拜光子禹為右將軍三月庚午光薨上及皇太后親臨光喪中二千石治冢賜梓宫葬具皆如乘輿制度諡曰宣成侯發三河卒穿復土【乘䋲證翻復如字】置園邑三百家長丞奉守下詔復其後世【復方目翻】疇其爵邑【應劭曰疇等也】世世無有所與【與讀曰豫】御史大夫魏相上封事【言事而不欲宣泄重封上之故曰封事漢官曰凡章表皆啟封其言密事得用皁囊】曰國家新失大將軍宜顯明功臣以塡藩國【塡古鎭字通】毋空大位以塞争權【師古曰大臣位空則起争奪之權也塞悉則翻】宜以車騎將軍安世為大將軍毋令領光禄勲事以其子延夀為光禄勲上亦欲用之夏四月戊申以安世為大司馬車騎將軍領尚書事 【考異曰百官表地節三年四月戊申張安世為大司馬七月戊戌更為衛將軍霍禹為大司馬七月壬辰禹要斬荀紀三年四月戊辰安世為大司馬按明年四月無戊辰七月無戊戍又不當再言七月以宣紀張安世霍光傳考之安世為司馬當在今年為衛將軍當在明年十月禹死在四年七月盖年表旁行通連書之致此誤也】 鳳皇集魯羣鳥從之大赦天下 上思報大將軍德乃封光兄孫山為樂平侯使以奉車都尉領尚書事魏相因昌成君許廣漢奏封事言春秋譏世卿【公羊傳隱三年夏四月辛卯尹氏卒尹氏者何天子之大夫也其稱尹氏何貶曷為貶譏世卿世卿非禮也】惡宋三世為大夫【公羊傳曰宋三世無大夫三世内取也師古曰三世謂襄公成公昭公也内取于國之大夫也為恐當作無惡烏路翻】及魯季孫之專權【魯自季友立僖公行父逐東門氏意如逐昭公世專魯國至哀公惡季氏之偪而不能去遂孫于邾】皆危亂國家自後元以來禄去王室政由冢宰今光死子復為右將軍兄子秉樞機【謂領尚書事也賢曰樞機近要之官也春秋運斗樞曰北斗第一天樞第二璇第三機也】昆弟諸壻據權勢在兵官光夫人顯及諸女皆通籍長信宫【師古曰通籍謂禁門之中皆有名籍恣出入也應劭曰籍者謂二尺竹牒設其年紀名字物色懸之宫門案省相應乃得入也】或夜詔門出入驕奢放縱恐寖不制【師古曰寖漸也不制不可制御也】宜有以損奪其權破散隂謀以固萬世之基全功臣之世又故事諸上書者皆為二封署其一曰副領尚書者先發副封所言不善屛去不奏相復因許伯白去副封以防壅蔽【屏必郢翻去丘呂翻】帝善之詔相給事中皆從其議【漢三公九卿皆外朝今魏相給事中則得入禁中預中朝之議】帝興于閭閻【師古曰閭里門也閻里中門也言從里巷而即天位也】知民事之囏難【囏古艱字】霍光既薨始親政事厲精為治【治直吏翻下同】五日一聽事自丞相以下各奉職奏事敷奏其言考試功能【應劭曰敷陳也各自奏陳其言然後試之以官考其功德也】侍中尚書功勞當遷及有異善厚加賞賜至于子孫終不改易【師古曰言各久其職事也貢父曰至于子孫謂賞賜逮及子孫也非謂侍中尚書官至子孫不改易也】樞機周密品式備具上下相安莫有苟且之意及拜刺史守相輒親見問觀其所由退而考察所行以質其言【師古曰質正也】有名實不相應必知其所以然常稱曰庶民所以安其田里而亡歎息愁恨之心者政平訟理也【師古曰訟理言所訟見理而無寃滯也亡古無字通】與我共此者其唯良二千石乎師守【古曰謂郡諸侯相】以為太守吏民之本數變易則下不安【數所角翻】民知其將久不可欺罔乃服從其教化故二千石有治理效輒以璽書勉厲增秩賜金或爵至關内侯公卿缺則選諸所表以次用之【師古曰所表謂增秩賜金爵也】是以漢世良吏於是為盛稱中興焉 匈奴壺衍鞮單于死弟左賢王立為虚閭權渠單于以右大將軍女為大閼氏而黜前單于所幸顓渠閼氏【顓渠閼氏單于之元妃也其次為大閼氏將即亮翻閼氏音煙支】顓渠閼氏父左大且渠怨望【且子閭翻】是時漢以匈奴不能為邉寇罷塞外諸城以休百姓【師古曰外城塞外諸城也如光禄塞受降城遮虜障等城是也】單于聞之喜召貴人謀欲與漢和親左大且渠心害其事曰前漢使來兵隨其後今亦效漢發兵先使使者入乃自請與呼盧訾王各將萬騎南旁塞獵相逢俱入【訾子移翻旁步浪翻】行未到會三騎亡降漢言匈奴欲為寇【降戶江翻】於是天子詔發邉騎屯要害處使大將軍軍監治衆等四人【師古曰治衆者軍監之名余據軍監位次軍正】將五千騎分三隊出塞各數百里捕得虜各數十人而還時匈奴亡其三騎不敢入即引去是歲匈奴饑人民畜產死什六七又發兩屯各萬騎以備漢其秋匈奴前所得西嗕居左地者【孟康曰嗕音辱匈奴種師古曰嗕音奴獨翻余謂西嗕自是一種為匈奴所得使居左地耳非匈奴種也】其君長以下數千人皆驅畜產行與甌脫戰所殺傷甚衆遂南降漢<br />
<br />
  資治通鑑卷二十四  <br>
   </div> 

<script src="/search/ajaxskft.js"> </script>
 <div class="clear"></div>
<br>
<br>
 <!-- a.d-->

 <!--
<div class="info_share">
</div> 
-->
 <!--info_share--></div>   <!-- end info_content-->
  </div> <!-- end l-->

<div class="r">   <!--r-->



<div class="sidebar"  style="margin-bottom:2px;">

 
<div class="sidebar_title">工具类大全</div>
<div class="sidebar_info">
<strong><a href="http://www.guoxuedashi.com/lsditu/" target="_blank">历史地图</a></strong>  
<a href="http://www.880114.com/" target="_blank">英语宝典</a>  
<a href="http://www.guoxuedashi.com/13jing/" target="_blank">十三经检索</a> 
<br><strong><a href="http://www.guoxuedashi.com/gjtsjc/" target="_blank">古今图书集成</a></strong> 
<a href="http://www.guoxuedashi.com/duilian/" target="_blank">对联大全</a> <strong><a href="http://www.guoxuedashi.com/xiangxingzi/" target="_blank">象形文字典</a></strong> 

<br><a href="http://www.guoxuedashi.com/zixing/yanbian/">字形演变</a>  <strong><a href="http://www.guoxuemi.com/hafo/" target="_blank">哈佛燕京中文善本特藏</a></strong>
<br><strong><a href="http://www.guoxuedashi.com/csfz/" target="_blank">丛书&方志检索器</a></strong> <a href="http://www.guoxuedashi.com/yqjyy/" target="_blank">一切经音义</a>  

<br><strong><a href="http://www.guoxuedashi.com/jiapu/" target="_blank">家谱族谱查询</a></strong>  <strong><a href="http://shufa.guoxuedashi.com/sfzitie/" target="_blank">书法字帖欣赏</a></strong> 
<br>

</div>
</div>


<div class="sidebar" style="margin-bottom:0px;">

<font style="font-size:22px;line-height:32px">QQ交流群9:489193090</font>


<div class="sidebar_title">手机APP 扫描或点击</div>
<div class="sidebar_info">
<table>
<tr>
	<td width=160><a href="http://m.guoxuedashi.com/app/" target="_blank"><img src="/img/gxds-sj.png" width="140"  border="0" alt="国学大师手机版"></a></td>
	<td>
<a href="http://www.guoxuedashi.com/download/" target="_blank">app软件下载专区</a><br>
<a href="http://www.guoxuedashi.com/download/gxds.php" target="_blank">《国学大师》下载</a><br>
<a href="http://www.guoxuedashi.com/download/kxzd.php" target="_blank">《汉字宝典》下载</a><br>
<a href="http://www.guoxuedashi.com/download/scqbd.php" target="_blank">《诗词曲宝典》下载</a><br>
<a href="http://www.guoxuedashi.com/SiKuQuanShu/skqs.php" target="_blank">《四库全书》下载</a><br>
</td>
</tr>
</table>

</div>
</div>


<div class="sidebar2">
<center>


</center>
</div>

<div class="sidebar"  style="margin-bottom:2px;">
<div class="sidebar_title">网站使用教程</div>
<div class="sidebar_info">
<a href="http://www.guoxuedashi.com/help/gjsearch.php" target="_blank">如何在国学大师网下载古籍?</a><br>
<a href="http://www.guoxuedashi.com/zidian/bujian/bjjc.php" target="_blank">如何使用部件查字法快速查字?</a><br>
<a href="http://www.guoxuedashi.com/search/sjc.php" target="_blank">如何在指定的书籍中全文检索?</a><br>
<a href="http://www.guoxuedashi.com/search/skjc.php" target="_blank">如何找到一句话在《四库全书》哪一页?</a><br>
</div>
</div>


<div class="sidebar">
<div class="sidebar_title">热门书籍</div>
<div class="sidebar_info">
<a href="/so.php?sokey=%E8%B5%84%E6%B2%BB%E9%80%9A%E9%89%B4&kt=1">资治通鉴</a> <a href="/24shi/"><strong>二十四史</strong></a>&nbsp; <a href="/a2694/">野史</a>&nbsp; <a href="/SiKuQuanShu/"><strong>四库全书</strong></a>&nbsp;<a href="http://www.guoxuedashi.com/SiKuQuanShu/fanti/">繁体</a>
<br><a href="/so.php?sokey=%E7%BA%A2%E6%A5%BC%E6%A2%A6&kt=1">红楼梦</a> <a href="/a/1858x/">三国演义</a> <a href="/a/1038k/">水浒传</a> <a href="/a/1046t/">西游记</a> <a href="/a/1914o/">封神演义</a>
<br>
<a href="http://www.guoxuedashi.com/so.php?sokeygx=%E4%B8%87%E6%9C%89%E6%96%87%E5%BA%93&submit=&kt=1">万有文库</a> <a href="/a/780t/">古文观止</a> <a href="/a/1024l/">文心雕龙</a> <a href="/a/1704n/">全唐诗</a> <a href="/a/1705h/">全宋词</a>
<br><a href="http://www.guoxuedashi.com/so.php?sokeygx=%E7%99%BE%E8%A1%B2%E6%9C%AC%E4%BA%8C%E5%8D%81%E5%9B%9B%E5%8F%B2&submit=&kt=1"><strong>百衲本二十四史</strong></a>  <a href="http://www.guoxuedashi.com/so.php?sokeygx=%E5%8F%A4%E4%BB%8A%E5%9B%BE%E4%B9%A6%E9%9B%86%E6%88%90&submit=&kt=1"><strong>古今图书集成</strong></a>
<br>

<a href="http://www.guoxuedashi.com/so.php?sokeygx=%E4%B8%9B%E4%B9%A6%E9%9B%86%E6%88%90&submit=&kt=1">丛书集成</a> 
<a href="http://www.guoxuedashi.com/so.php?sokeygx=%E5%9B%9B%E9%83%A8%E4%B8%9B%E5%88%8A&submit=&kt=1"><strong>四部丛刊</strong></a>  
<a href="http://www.guoxuedashi.com/so.php?sokeygx=%E8%AF%B4%E6%96%87%E8%A7%A3%E5%AD%97&submit=&kt=1">說文解字</a> <a href="http://www.guoxuedashi.com/so.php?sokeygx=%E5%85%A8%E4%B8%8A%E5%8F%A4&submit=&kt=1">三国六朝文</a>
<br><a href="http://www.guoxuedashi.com/so.php?sokeytm=%E6%97%A5%E6%9C%AC%E5%86%85%E9%98%81%E6%96%87%E5%BA%93&submit=&kt=1"><strong>日本内阁文库</strong></a> <a href="http://www.guoxuedashi.com/so.php?sokeytm=%E5%9B%BD%E5%9B%BE%E6%96%B9%E5%BF%97%E5%90%88%E9%9B%86&ka=100&submit=">国图方志合集</a> <a href="http://www.guoxuedashi.com/so.php?sokeytm=%E5%90%84%E5%9C%B0%E6%96%B9%E5%BF%97&submit=&kt=1"><strong>各地方志</strong></a>

</div>
</div>


<div class="sidebar2">
<center>

</center>
</div>
<div class="sidebar greenbar">
<div class="sidebar_title green">四库全书</div>
<div class="sidebar_info">

《四库全书》是中国古代最大的丛书,编撰于乾隆年间,由纪昀等360多位高官、学者编撰,3800多人抄写,费时十三年编成。丛书分经、史、子、集四部,故名四库。共有3500多种书,7.9万卷,3.6万册,约8亿字,基本上囊括了古代所有图书,故称“全书”。<a href="http://www.guoxuedashi.com/SiKuQuanShu/">详细>>
</a>

</div> 
</div>

</div>  <!--end r-->

</div>
<!-- 内容区END --> 

<!-- 页脚开始 -->
<div class="shh">

</div>

<div class="w1180" style="margin-top:8px;">
<center><script src="http://www.guoxuedashi.com/img/plus.php?id=3"></script></center>
</div>
<div class="w1180 foot">
<a href="/b/thanks.php">特别致谢</a> | <a href="javascript:window.external.AddFavorite(document.location.href,document.title);">收藏本站</a> | <a href="#">欢迎投稿</a> | <a href="http://www.guoxuedashi.com/forum/">意见建议</a> | <a href="http://www.guoxuemi.com/">国学迷</a> | <a href="http://www.shuowen.net/">说文网</a><script language="javascript" type="text/javascript" src="https://js.users.51.la/17753172.js"></script><br />
  Copyright &copy; 国学大师 古典图书集成 All Rights Reserved.<br>
  
  <span style="font-size:14px">免责声明:本站非营利性站点,以方便网友为主,仅供学习研究。<br>内容由热心网友提供和网上收集,不保留版权。若侵犯了您的权益,来信即刪。scp168@qq.com</span>
  <br />
ICP证:<a href="http://www.beian.miit.gov.cn/" target="_blank">鲁ICP备19060063号</a></div>
<!-- 页脚END --> 
<script src="http://www.guoxuedashi.com/img/plus.php?id=22"></script>
<script src="http://www.guoxuedashi.com/img/tongji.js"></script>

</body>
</html>
