










 


 
 


 

  
  
  
  
  





  
  
  
  
  
 
  

  

  
  
  



  

 
 

  
   




  

  
  


    資治通鑑卷七十一   宋 司馬光 撰

  胡三省 音註

  魏紀三【起著雍涒灘盡上章閹茂凡三年}


  烈祖明皇帝上之下

  太和二年春正月司馬懿攻新城旬有六日拔之斬孟達申儀久在魏興擅承制刻印多所假授懿召而執之歸于洛陽【歸儀于京師也}
 初征西將軍夏侯淵之子楙尚太祖女清河公主【此女欲以妻丁儀文帝止之以妻楙楙音茂}
文帝少與之親善【少詩照翻}
及即位以為安西將軍都督關中鎭長安使承淵處【淵鎮長安見六十六卷漢獻帝建安十六年}
諸葛亮將入寇與羣下謀之丞相司馬魏延曰【漢丞相有長史而無司馬是時用兵故置司馬}
聞夏侯楙主壻也怯而無謀今假延精兵五千負糧五千直從褒中出循秦嶺而東當子午而北【褒中縣屬漢中郡子午道王莽所通事見三十六卷平帝元始五年安帝延光四年順帝罷子午道通褒斜路三秦記曰子午長安正南山名秦嶺谷一名樊川余按今洋川東百六十里有子午谷郡縣志曰舊子午道在金州安康縣界梁將軍王神念以緣山避水橋梁百數多有毁壞乃别開乾路更名子午道則今路是也}
不過十日可到長安楙聞延奄至必弃城逃走長安中惟御史京兆太守耳【時遣督軍御史與京兆太守共守長安晉志曰文帝受禪改漢京兆尹為太守守式又翻}
橫門邸閣與散民之穀足周食也【魏置邸閣於橫門以積粟民聞兵至必逃散可收其穀以周食橫音光}
比東方相合聚尚二十許日【比必寐翻}
而公從斜谷來【斜余遮翻谷音浴又古禄翻}
亦足以達如此則一舉而咸陽以西可定矣亮以為此危計不如安從坦道可以平取隴右十全必克而無虞故不用延計【由今觀之皆以亮不用延計為怯凡兵之動知敵之主知敵之將亮之不用延計者知魏主之明略而司馬懿輩不可輕也亮欲平取隴右且不獲如志况欲乘險僥倖盡定咸陽以西邪}
亮揚聲由斜谷道取郿【班志斜水出衙嶺山北至郿入渭脈水沿山則斜谷之路可知矣郿師古音媚郿故城陳倉縣東北十五里故郿城是}
使鎮東將軍趙雲揚武將軍鄧芝為疑兵據箕谷【今興元府褒縣北十五里有箕山鄭子真隱於此趙雲鄧芝所據即此谷也又據後漢書馮異傳箕谷當在陳倉之南漢中之北}
帝遣曹真都督關右諸軍軍郿亮身率大軍攻祁山戎陳整齊【陳讀曰陣}
號令明肅始魏以漢昭烈既死數歲寂然無聞是以略無備豫【謂不豫為之備也}
而卒聞亮出【卒讀曰猝}
朝野恐懼於是天水南安安定皆叛應亮【魏分隴右置秦州天水南安屬焉漢靈帝中平四年分漢陽之䝠道立南安郡漢陽郡至晉方改為天水史追書也安定郡屬雍州杜佑曰南安今隴西郡隴西縣}
關中響震朝臣未知計所出帝曰亮阻山為固今者自來正合兵書致人之術【兵法曰善戰者致人帝姑以此言安朝野之心耳}
破亮必也乃勒兵馬步騎五萬遣右將軍張郃督之西拒亮【郃古合翻又曷閤翻}
丁未帝行如長安【親帥師繼郃之後以張聲勢如往也}
初越嶲太守馬謖才器過人好論軍計【好呼到翻}
諸葛亮深加器異漢昭烈臨終謂亮曰馬謖言過其實不可大用君其察之亮猶謂不然以謖為參軍每引見談論自晝達夜【以孔明之明略所以待謖者如此亦足以見其善論軍計矣觀孔明南征之時謖陳攻心之論豈悠悠坐談者所能及哉}
及出軍祁山亮不用舊將魏延吳懿等為先鋒而以謖督諸軍在前與張郃戰于街亭【續漢志漢陽略陽縣有街泉亭前漢之街泉縣也省入略陽杜佑曰街泉亭在隴縣又曰平凉郡界有街泉亭馬謖為張郃所敗處又考五代史志漢川郡西縣有街亭山嶓冢山漢水則隋之西縣蓋兼得隴西之䝠道漢陽之西縣矣又按郡國縣道記梁州之西縣本名白馬城又曰濜口城後魏正始中立嶓冢縣隋始改曰西縣此非續漢志漢陽之西縣也}
謖違亮節度舉措煩擾舍水上山不下據城【郃傳言謖依阻南山舍讀曰捨上時掌翻}
張郃絶其汲道擊大破之士卒離散亮進無所據乃拔西縣千餘家還漢中【續漢志西縣前漢屬隴西郡後漢屬漢陽郡有嶓冢山西漢水}
收謖下獄殺之亮自臨祭為之流涕【下遐稼翻為于偽翻}
撫其遺孤恩若平生【殺之者王法也恩之者故人之情不忘也}
蔣琬謂亮曰昔楚殺得臣文公喜可知也【左傳晉文公及楚子玉得臣戰于城濮楚師敗績晉入楚軍三日穀文公猶有憂色曰得臣猶在憂未歇也及楚殺得臣然後喜可知也杜預曰謂喜見於顔色}
天下未定而戮智計之士豈不惜乎【觀此則蔣琬亦重謖矣}
亮流涕曰孫武所以能制勝於天下者用法明也【孫子始計篇曰法令孰行言法令行者必勝也故其教吳宫美人兵必殺吳王寵姬二人以明其法}
是以揚干亂法魏絳戮其僕【左傳晉悼公合諸侯其弟揚干亂行魏絳戮其僕悼公謂魏絳能以刑佐民使佐新軍}
四海分裂兵交方始若復廢法何用討賊邪謖之未敗也裨將軍巴西王平連規諫謖謖不能用及敗衆盡星散惟平所領千人鳴鼓自守張郃疑其有伏兵不往偪也於是平徐徐收合諸營遺迸率將士而還【據王平傳平所識不過十字觀其收馬謖敗散之兵拒曹爽猝至之師則用兵方略固不在於多識字也迸北孟翻還從宣翻又如字}
亮既誅馬謖及將軍李盛奪將軍黄襲等兵平特見崇顯加拜參軍統五部兼當營事【既總統五部兵時亮屯漢中又使之兼當營屯之事}
進位討寇將軍封亭侯【後漢之制列侯有縣侯鄉侯亭侯}
亮上疏請自貶三等漢主以亮為右將軍行丞相事是時趙雲鄧芝兵亦敗於箕谷雲歛衆固守故不大傷雲亦坐貶為鎮軍將軍【據晉書職官志鎮軍將軍在四征四鎮將軍之上今趙雲自鎮東將軍貶鎮軍將軍蓋蜀漢之制以鎮東為專鎮方面而以鎮軍為散號故為貶也}
亮問鄧芝曰街亭軍退兵將不復相録【録收拾也將即亮翻下同復扶又翻}
箕谷軍退兵將初不相失何故芝曰趙雲身自斷後【斷丁管翻}
軍資什物略無所棄兵將無緣相失雲有軍資餘絹亮使分賜將士雲曰軍事無利何為有賜其物請悉入赤㟁庫【水經注褒水西北出衙嶺山東南逕大石門歷故棧道下谷俗謂千梁無柱也諸葛亮與兄瑾書曰前趙子龍退軍燒壞赤崖閣道緣谷一百餘里其閣梁一頭入山腹一頭立柱於水中今水大而急不得安柱又云頃大水暴出赤崖以南橋閣悉壞時趙子龍與鄧伯苗一戍赤崖屯田一戍赤崖口但得緣崖與伯苗相聞而已後亮死于五丈原魏延先退而焚之即是道也赤崖即赤㟁蜀置庫於此以儲軍資}
須十月為冬賜【須待也}
亮大善之或勸亮更發兵者亮曰大軍在祁山箕谷皆多於賊而不破賊乃為賊所破此病不在兵少也在一人耳【謂兵之勝敗在將也少詩沼翻}
今欲減兵省將【將即亮翻}
明罰思過校變通之道於將來若不能然者雖兵多何益自今已後諸有忠慮於國者但勤攻吾之闕則事可定賊可死功可蹻足而待矣【蹻巨嬌翻}
於是考微勞甄壯烈【甄稽延翻察也别也}
引咎責躬布所失於境内厲兵講武以為後圖戎士簡練民忘其敗矣【善敗者不亡此之謂也姜維之敗則不可復振矣}
亮之出祁山也天水參軍姜維詣亮降【降戶江翻}
亮美維膽智辟為倉曹掾【續漢志丞相倉曹掾主倉穀事}
使典軍事【考異曰孫盛雜語曰維詣諸葛亮與母相失後得母書令求當歸維曰良田百頃不在一畝但有遠志不在當歸也按維粗知學術恐不至此今不取}
曹真討安定等三郡皆平真以諸葛亮懲於祁山後必出從陳倉乃使將軍郝昭等守陳倉治其城【杜佑曰漢陳倉故城在今縣東二十里治直之翻}
 夏四月丁酉帝還洛陽帝以燕國徐邈為凉州刺史【晉志曰凉州蓋以其地處西方常寒凉也地勢西北邪出在南山之間南隔西羌西通西域統金城西平武威張掖西郡酒泉燉煌西海等郡}
邈務農積穀立學明訓進善黜惡與羌胡從事不問小過若犯大罪先告部帥使知應死者乃斬以徇【帥所類翻}
由是服其威信州界肅清 五月大旱 吳王使鄱陽太守周魴密求山中舊族名帥為北方所聞知者【所謂山越宗帥也魴符方翻帥所類翻}
令譎挑揚州牧曹休【魏揚州止得漢之九江廬江二郡地而江津要害之地多為吳所據譎古穴翻挑徒了翻}
魴曰民帥小醜不足杖任事或漏泄不能致休乞遣親人齎牋以誘休言被譴懼誅欲以郡降北【誘音酉降戶江翻}
求兵應接吳王許之時頻有郎官詣魴詰問諸事【郎官尚書郎也詰去吉翻}
魴因詣郡門下【鄱陽郡門下}
下髪謝【吳主之詰周魴之謝皆所以譎曹休也}
休聞之率步騎十萬向皖以應魴【皖戶板翻下同}
帝又使司馬懿向江陵【懿督諸軍屯宛使向江陵}
賈逵向東關【東關即濡須口亦謂之柵江口有東西關東關之南㟁吳築城西關之北㟁魏置柵後諸葛恪於東關作大堤以遏巢湖謂之東興堤即其地也}
三道俱進秋八月吳王至皖以陸遜為大都督假黃鉞親執鞭以見之【此猶古之王者遣將跪而推轂之意也}
以朱桓全琮為左右督【琮徂宗翻}
各督三萬人以擊休休知見欺而恃其衆欲遂與吳戰朱桓言於吳王曰休本以親戚見任非智勇名將也今戰必敗敗必走走當由夾石挂車【元豐九域志舒州桐城縣北有挂車鎮有挂車嶺鎮因嶺而得名}
此兩道皆險阨若以萬兵柴路【柴路謂以柴塞路也}
則彼衆可盡休可生虜臣請將所部以斷之【斷丁管翻}
若蒙天威得以休自效便可乘勝長驅進取夀春割有淮南以規許洛【漢末都許有許昌宫魏時都洛魏略曰文帝改長安譙許昌鄴洛陽為五都立石表西界宜陽北循太行東北界陽平南循魯陽東界郯為中都之地}
此萬世一時不可失也【言歷萬世惟有此一時機會可乘耳}
權以問陸遜遜以為不可乃止尚書蔣濟上疏曰休深入虜地與權精兵對而朱然等在上流乘休後臣未見其利也前將軍滿寵上疏曰曹休雖明果而希用兵今所從道背湖旁江易進難退【背蒲妹翻旁步浪翻易以豉翻}
此兵之絓地也【絓古賣翻罥也言其地險師行由之為所罥挂進退不可也孫子地形篇曰地形有通者有挂者我可以往彼可以來曰通可以往難以返曰挂}
若入無疆口【無疆口在夾石東南}
宜深為之備寵表未報休與陸遜戰于石亭【時吳王在皖口遣遜等與休戰于石亭則其地當在今舒州懷寧桐城二縣之間}
遜自為中部令朱桓全琮為左右翼三道並進衝休㐲兵因驅走之追亡逐北徑至夾石斬獲萬餘牛馬騾驢車乘萬兩軍資器械略盡【休蓋未嘗整陳交戰而敗也兩音亮乘繩證翻}
初休表求深入以應周魴帝命賈逵引兵東與休合【按逵傳逵自豫州進兵取西陽以向東關休自夀春向皖西陽在皖之西而東關又在皖之東今與休合蓋使合兵向東關也}
逵曰賊無東關之備必并軍於皖休深入與賊戰必敗乃部署諸將水陸並進行二百里獲吳人言休戰敗吳遣兵斷夾石【斷丁管翻下同}
諸將不知所出或欲待後軍逵曰休兵敗於外路絶於内進不能戰退不得還安危之機不及終日賊以軍無後繼故至此今疾進出其不意此所謂先人以奪其心也【左傳軍志曰先人有奪人之心先悉薦翻}
賊見吾兵必走若待後軍賊已斷險兵雖多何益乃兼道進軍多設旗鼓為疑兵吳人望見逵軍驚走【驚走者斷夾石之軍耳}
休乃得還逵據夾石以兵糧給休休軍乃振初逵與休不善【逵與休不善文帝黃初中欲假逵節休曰逵性剛易侮諸將不可為督遂止}
及休敗賴逵以免 九月乙酉立皇子穆為繁陽王 長平壯侯曹休上書謝罪帝以宗室不問【敗軍者必誅烏可以宗室而不問邪}
休慙憤疽發於背庚子卒帝以滿寵都督揚州以代之 護烏桓校尉田豫擊鮮卑鬱築鞬鬱築鞬妻父軻比能救之以三萬騎圍豫於馬城【馬城縣漢屬代郡魏晉省蓋城邑殘破已弃為荒外之地矣鞬居言翻}
上谷太守閻志柔之弟也素為鮮卑所信【自漢建安時閻柔已護烏桓故其兄弟為二虜所信}
往解諭之乃解圍去 冬十一月蘭陵成侯王朗卒 漢諸葛亮聞曹休敗魏兵東下關中虚弱欲出兵擊魏羣臣多以為疑【因祁山之敗疑魏不可伐}
亮上言於漢主曰先帝深慮以漢賊不兩立王業不偏安故託臣以討賊以先帝之明量臣之才固當知臣伐賊才弱敵強然不伐賊王業亦亡惟坐而待亡孰與伐之是故託臣而弗疑也臣受命之日寢不安席食不甘味思惟北征宜先入南故五月渡瀘深入不毛【瀘魯都翻}
臣非不自惜也顧王業不可偏全於蜀都故冒危難以奉先帝之遺意也【難乃旦翻下同}
而議者以為非計今賊適疲於西又務於東【疲於西謂郿縣祁山之師務於東謂江陵東關石亭之師也}
兵法乘勞此進趨之時也謹陳其事如左高帝明並日月謀臣淵深然涉險被創【被皮義翻創初良翻}
危然後安今陛下未及高帝謀臣不如良平而欲以長計取勝坐定天下此臣之未解一也【解讀曰懈言未敢懈怠也後皆同}
劉繇王朗各據州郡論安言計動引聖人羣疑滿腹衆難塞胸今歲不戰明年不征使孫策坐大遂并江東此臣之未解二也【難乃旦翻坐大言坐致強大也策破劉繇事見六十一卷漢獻帝興平二年破王朗事見六十二卷建安元年}
曹操智計殊絶於人其用兵也髣髴孫吳【以操之善用兵亮謂之髣髴孫吳孫吳固未易才也}
然困於南陽險於烏巢危於祁連偪於黎陽幾敗伯山殆死潼關然後偽定一時耳【困於南陽謂攻穰為張繡所敗也險於烏巢謂攻袁紹將淳于瓊時也偪於黎陽謂攻袁譚兄弟時也幾敗伯山謂與烏桓戰于白狼山時也殆死潼關謂與馬超戰時也危於祁連當考或曰圍袁尚於祁山時也偽定者言雖定一時之功而有心於簒漢故曰偽幾居希翻}
况臣才弱而欲以不危定之此臣之未解三也曹操五攻昌霸不下四越巢湖不成任用李服而李服圖之委夏侯而夏侯敗亡【昌霸昌絺也操累攻不下後命于禁擊斬之四越巢湖不成謂攻孫權也李服蓋王服也與董承謀殺操被誅夏侯謂夏侯淵守漢中為先主所敗也}
先帝每稱操為能猶有此失况臣駑下何能必勝此臣之未解四也【駑下者自謙以馬為喻若駑駘下乘也}
自臣到漢中中間期年耳然喪趙雲陽羣馬玉閻芝丁立白夀劉郃鄧銅等及曲長屯將七十餘人【喪息浪翻郃古合翻又曷閤翻曲長一曲之長也軍行有部部下有曲曲各有長長丁丈翻屯將將屯者也將即亮翻}
突將無前【將即亮翻}
賨叟青羌散騎武騎一千餘人【蜀兵謂之叟賨叟巴賨之兵也青羌亦羌之一種散騎武騎當時騎兵分部之名賨藏宗翻騎奇寄翻}
皆數十年之内糾合四方之精銳非一州之所有若復數年則損三分之二【復扶又翻}
當何以圖敵此臣之未解五也【言不戰而將士耗損已如此也}
今民窮兵疲而事不可息事不可息則住與行勞費正等而不及虛圖之【亮意欲及魏與吳連兵未解乘虛而圖之也}
欲以一州之地與賊支久此臣之未解六也【支持也支久猶言持久也}
夫難平者事也昔先帝敗軍於楚當此時曹操拊手謂天下已定然後先帝東連吳越【事見六十五卷漢獻帝建安十三年拊手乘快之意發見於外者也}
西取巴蜀【事見六十七卷建安十九年}
舉兵北征夏侯授首【事見六十八卷建安二十四年}
此操之失計而漢事將成也然後吳更違盟關羽毁敗【事見六十八卷建安二十四年此兩然後之然轉語之辭與他文然後之義不同}
秭歸蹉跌曹丕稱帝【事見六十九卷黃初元年三年}
凡事如是難可逆見臣鞠躬盡瘁死而後已至於成敗利鈍非臣之明所能逆覩也【自祁山之敗亮益知魏人情偽故其所言如此}
十二月亮引兵出散關圍陳倉陳倉己有備亮不能克【曹真使郝昭先守故亮不能克此下申言昭守亮攻客主相持之事通鑑書法類如此}
亮使郝昭鄉人靳詳於城外遥說昭【靳於焮翻說輸芮翻下同}
昭於樓上應之曰魏家科法卿所練也【科條也練習也}
我之為人卿所知也我受國恩多而門戶重卿無可言者但有必死耳卿還謝諸葛便可攻也詳以昭語告亮亮又使詳重說昭【重直用翻}
言人兵不敵空自破滅昭謂詳曰前言已定矣我識卿耳箭不識也詳乃去亮自以有衆數萬而昭兵纔千餘人又度東救未能便到【魏兵救陳倉者自東來故曰東救度徒洛翻}
乃進兵攻昭起雲梯衝車以臨城昭於是以火箭逆射其梯【射而亦翻下同}
梯然梯上人皆燒死昭又以繩連石磨壓其衝車【磨莫卧翻石磑也}
衝車折【折而設翻}
亮乃更為井闌百尺以射城中【以木交構若井闌狀}
以土丸填壍【壍七艶翻}
欲直攀城昭又於内築重牆【重直用翻}
亮又為地突【地突地道也}
欲踊出於城裏昭又於城内穿地橫截之晝夜相攻拒二十餘日曹真遣將軍費耀等救之帝召張郃于方城【時郃將兵伐吳屯于方城續漢志曰葉縣南有長山曰方城屈完所謂楚國方城以為城者即此也}
使擊亮帝自幸河南城置酒送郃【河南城在洛陽城西}
問郃曰遟將軍到【遟直利翻待也}
亮得無已得陳倉乎郃知亮深入無穀屈指計曰比臣到亮已走矣【比必寐翻}
郃晨夜進道未至亮糧盡引去將軍王雙追之亮擊斬雙詔賜昭爵關内侯【攻者不足守者有餘尚論其才則全城卻敵者其才非優於攻者也客主之勢異耳故曰用兵之術攻城最下}
 初公孫康卒子晃淵等皆幼官屬立其弟恭恭劣弱不能治國淵既長【治直之翻長知兩翻}
脇奪恭位上書言狀侍中劉曄曰公孫氏漢時所用【公孫度守遼東見五十九卷獻帝初平元年}
遂世官相承【古者世爵不世官爵謂公侯伯子男官謂卿大夫也今謂之世官者以公孫氏所據之地漢遼東太守之職守耳子孫相襲是世官也}
水則由海陸則阻山外連胡夷絶遠難制而世權日久今若不誅後必生患若懷貳阻兵然後致誅於事為難不如因其新立有黨有仇【有黨故能奪恭位與之為仇者則恭之黨也}
先其不意以兵臨之【先悉薦翻}
開設賞募可不勞師而定也帝不從拜淵揚烈將軍遼東太守【為公孫淵叛魏張本}
 吳王以揚州牧呂範為大司馬印綬未下而卒【下遐稼翻}
初孫策使範典財計時吳王年少【少詩照翻}
私從有求範必關白不敢專許當時以此見望【望責望也怨望也}
吳王守陽羨長【陽羨縣前漢屬會稽郡後漢屬吳郡賢曰故城在今常州義興縣南長知兩翻}
有所私用策或料覆【料音聊覆審校也}
功曹周谷輒為傅著簿書【為于偽翻傅讀曰附著直略翻}
使無譴問王臨時悦之及後統事以範忠誠厚見信任以谷能欺更簿書不用也【周世宗之待周美我朝太祖之重竇儀事亦類此更工衡翻}
三年春漢諸葛亮遣其將陳戒攻武都隂平二郡【隂平道前漢屬廣漢郡後漢屬廣漢屬國都尉魏分置隂平郡唐為文州}
雍州刺史郭淮引兵救之【禹貢黑水西河為雍州以其四山之地故以雍名焉亦謂西北之位陽所不及隂陽雍閼周都豐鎬雍州為王畿平王東遷雍州為秦地漢武置十三州以雍州之西偏為凉州其餘並屬司隷光武都洛關中復置雍州尋罷復以司隸統三輔獻帝興平元年河西為河寇所隔置雍州以統河西諸郡至魏以河西置凉州以隴右為雍州及晉以隴右置秦州而雍州統京兆馮翊扶風安定北地新平武都隂平雍於用翻}
亮自出至建威【水經注漢水西南逕祁山軍南西流與建安川水合建安水導源建威西北山東逕建威城南又東逕西縣歷城南祝穆曰天下之大川以漢名者二班固謂之東漢西漢而黎州之漢水源於飛越嶺者不與焉固之所謂東漢則禹貢之漾漢其源出於今興元之西縣嶓冢山逕洋金房均襄郢復至漢陽入江者是也西漢則蘇代所謂漢中之甲輕舟出於巴乘夏水下漢四日而至五渚者其源出於西和州徼外逕階沔州與嘉陵水會俗謂之西漢又逕大安軍利劍閬果合與涪水會至渝州入江}
淮退亮遂拔二郡以歸漢主復策拜亮為丞相 夏四月丙申吳主即皇帝位大赦改元黃龍【時夏口武昌並言黃龍見權遂以改元}
百官畢會吳主歸功周瑜綏遠將軍張昭舉笏欲褒贊功德未及言【沈約志魏罝將軍四十號綏遠第十四}
吳主曰如張公之計今已乞食矣【歸功周瑜以能拒曹公而成三分之業也乞食謂張昭欲迎曹公也事見六十五卷漢獻帝建安十三年}
昭大慙伏地流汗吳主追尊父堅為武烈皇帝兄策為長沙桓王立子登為皇太子封長沙桓王子紹為吳侯以諸葛恪為太子左輔張休為右弼顧譚為輔正陳表為翼正都尉【輔正及翼正都尉皆吳自創置之}
而謝景范慎羊衜等皆為賓客【衜古道字}
於是東宫號為多士太子使侍中胡綜作賓客目【目者因其人之才品為之品題也}
曰英才卓越超踰倫匹則諸葛恪精識時機達幽究微則顧譚凝辯宏達言能釋結則謝景【凝堅定也宏濶遠也達明通也好辯者每不能堅定其所守故以能凝辯而證據宏遠明通者可以釋難疑之糾結也}
究學甄微游夏同科則范慎【究窮竟也甄察别也夏戶雅翻}
羊衜私駮綜曰元遜才而疏子嘿精而狠叔發辯而浮孝敬深而陿【諸葛恪字元遜顧譚字子嘿謝景字叔發范慎字孝敬狠戶墾翻陿與狹同}
衜卒以此言為恪等所惡【卒子恤翻惡烏路翻}
其後四人皆敗如衜所言吳主使以並尊二帝之議往告于漢漢人以為交之無益而名體弗順宜顯明正義絶其盟好【天無二日土無二王古今之正義也好呼到翻}
丞相亮曰權有僭逆之心久矣國家所以略其釁情者求掎角之援也【釁隙也情欲也左傳戎子駒支對范宣子曰殽之師晉禦其上戎亢其下秦師不復我諸戎實然譬如捕鹿晉人角之諸戎掎之與晉踣之杜預注曰掎其足也}
今若加顯絶讐我必深當更移兵東戍與之角力須并其土乃議中原彼賢才尚多將相輯穆未可一朝定也頓兵相守坐而須老【須待也}
使北賊得計非算之上者【北賊謂魏也}
昔孝文卑辭匈奴先帝優與吳盟【事並見前優饒也今人猶謂寛假為優饒}
皆應權通變深思遠益非若匹夫之忿者也【言所計者大也}
今議者咸以權利在鼎足不能并力且志望已滿無上㟁之情【謂孫權之志在保江不能上㟁而北向也上時掌翻}
推此皆似是而非也何者其智力不侔故限江自保權之不能越江猶魏賊之不能渡漢【言魏不能渡漢而圖江陵也此漢班志所謂東漢水也}
非力有餘而利不取也若大軍致討彼高當分裂其地以為後規下當略民廣境示武於内非端坐者也【言蜀若破魏吳亦將分功}
若就其不動而睦於我我之北伐無東顧憂河南之衆不得盡西此之為利亦已深矣【言蜀與吳和則雖傾國北伐不須東顧以備吳而魏河南之衆欲留備吳不得盡西以抗蜀兵也}
權僭逆之罪未宜明也乃遣衛尉陳震使於吳賀稱尊號吳主與漢人盟約中分天下以豫青徐幽屬吳兖冀并凉屬漢其司州之土以函谷關為界【漢武帝置司隸校尉所部三輔三河諸郡其界西得雍州之京兆扶風馮翊三郡北得冀州之河東河内二郡東得豫州之河南弘農二郡位望隆乎牧伯銀印青綬在十三部刺史之上後漢省朔方刺史以隸并州合司隸於十三部之數魏以司隸所部河東河南河内弘農并冀州之平陽合五郡置司州以三輔還屬雍州此言司州以函谷關為界以漢司隸所部分之也}
張昭以老病上還官位及所統領【上時掌翻}
更拜輔吳將軍【更工衡翻}
班亞三司改封婁侯【婁古縣也前漢屬會稽郡東漢分屬吳郡今蘇州崑山縣地吳以封昭非真國於婁而君國子民也}
食邑萬戶昭每朝見【見賢遍翻下同}
辭氣壯厲義形於色曾已直言逆旨【已當作以古已以字通}
中不進見後漢使來【使疏吏翻下同}
稱漢德美而羣臣莫能屈吳主歎曰使張公在坐【坐徂卧翻}
彼不折則廢安復自誇乎【折屈也李奇曰廢失氣也晉灼曰廢不收也復扶又翻下同}
明日遣中使勞問【勞力到翻}
因請見昭昭避席謝吳主跪止之昭坐定仰曰昔太后桓王不以老臣屬陛下而以陛下屬老臣【太后謂權母吳氏也屬之欲翻}
是以思盡臣節以報厚恩而意慮淺短違逆盛旨然臣愚心所以事國志在忠益畢命而已若乃變心易慮以偷榮取容此臣所不能也吳主辭謝焉 元城哀王禮卒 六月癸卯繁陽王穆卒戊申追尊高祖大長秋曰高皇帝【大長秋漢宦者曹騰也}
夫人

  吳氏曰高皇后 秋七月詔曰禮王后無嗣擇建支子以繼大宗【嫡子之出相承為宗子庶子之出為支子支岐出也}
則當纂正統而奉公義何得復顧私親哉漢宣繼昭帝後加悼考以皇號【事見二十五卷元康元年}
哀帝以外藩援立而董宏等稱引亡秦惑誤時朝【朝直遥翻}
既尊恭皇立廟京都又寵藩妾使比長信叙昭穆於前殿【昭讀曰佋如遥翻}
並四位於東宫僭差無度人神弗祐而非罪師丹忠正之諫用致丁傅焚如之禍【序昭穆於前殿謂定陶恭皇與元帝序昭穆也東宫謂太后宫四位謂丁傳趙后與元后並稱太后事具見三十四卷三十五卷}
自是之後相踵行之【謂漢安帝尊父清河孝王為孝德皇桓帝尊祖河間孝王為孝穆皇父蠡吾侯志為孝崇皇靈帝尊祖河間王淑為孝元皇父解瀆亭侯萇為孝仁皇其妃皆尊為后也}
昔魯文逆祀罪由夏父宋國非度譏在華元【春秋文公二年大事于太廟躋僖公逆祀也於是夏父弗忌為宗伯且明見曰吾見新鬼大舊鬼小先大後小順也躋聖賢明也君子以為失禮禮無不順祀國之大事也而逆之可謂禮乎成公二年宋文公卒始厚葬用蜃炭益車馬始用殉重器備君子謂華元於是乎不臣華戶化翻}
其令公卿有司深以前世行事為戒後嗣萬一有由諸侯入奉大統則當明為人後之義敢為佞邪導諛時君妄建非正之號以干正統謂考為皇稱妣為后則股肱大臣誅之無赦其書之金策藏之宗廟著于令典【帝無子知必以支孽為後故豫下此詔以約飭為人子為人臣者}
九月吳主遷都建業皆因故府不復增改【復扶又翻}
留太

  子登及尚書九官於武昌【九官九卿也}
使上大將軍陸遜輔太子并掌荆州及豫章三郡事董督軍國【吳於大將軍之上復置上大將軍三郡豫章鄱陽廬陵也三郡本屬揚州而地接荆州又有山越易相扇動故使遜兼掌之}
南陽劉廙嘗著先刑後禮論【廙羊職翻又羊至翻}
同郡謝景稱之於遜遜呵之曰禮之長於刑久矣【長知兩翻}
廙以細辯而詭先聖之教【詭異也戾也}
君今侍東宫宜遵仁義以彰德音若彼之談不須講也太子與西陵都督步隲書【吳保江南凡邊要之地皆置督獨西陵置都督以國之西門統攝要重也杜佑曰西陵今夷陵郡隲之日翻}
求見啟誨隲於是條于時事業在荆州界者及諸僚吏行能以報之【行下孟翻}
因上疏奬勸曰臣聞人君不親小事使百官有司各任其職故舜命九賢則無所用心不下廟堂而天下治也【舜命九官禹作司空宅百揆契作司徒棄后稷皋陶作士益作朕虞垂共工夷作秩宗龍作納言夔典樂治直吏翻}
故賢人所在折衝萬里【晏子春秋曰晉平公欲攻齊使范昭觀焉景公觴之范昭曰願請君之棄爵景公曰諾已飲晏子命徹尊更之范昭歸以報晉平公曰齊未可伐也吾欲恥其君而晏子知之仲尼聞之曰起於尊俎之間而折衝千里之外漢何武上封事曰虞有宫之奇晉獻不寐衛青在位淮南寢謀故賢人立朝折衝厭難勝於無形}
信國家之利器崇替之所由也願明太子重以經意則天下幸甚張紘還吳迎家道病卒臨困授子留牋【留牋猶今遺表也}
曰自古有國有家者咸欲修德政以比隆盛世至於其治多不馨香【書君陳曰至治馨香感于神明治直吏翻下同}
非無忠臣賢佐也由主不勝其情弗能用耳夫人情憚難而趨易好同而惡異【易以豉翻好呼到翻惡烏路翻}
與治道相反傳曰從善如登從惡如崩言善之難也人君承奕世之基據自然之勢操八柄之威【周禮天官太宰以八柄詔王馭羣臣一曰爵以馭其貴二曰禄以馭其富三曰予以馭其幸四曰置以馭其行五曰生以馭其福六曰奪以馭其貧七曰廢以馭其罪八曰誅以馭其過操千高翻}
甘易同之歡【易以豉翻}
無假取於人而忠臣挾難進之術吐逆耳之言其不合也不亦宜乎離則有釁【言納忠而不合於上則上下之情離釁隙由此而生也}
巧辯緣間【間古莧翻}
眩於小忠戀於恩愛賢愚雜錯黜陟失序其所由來情亂之也故明君寤之求賢如饑渇受諫而不厭抑情損欲以義割恩則上無偏謬之授下無希冀之望矣吳王省書為之流涕【省悉景翻為于偽翻}
冬十月改平望觀曰聽訟觀【水經注平望觀在華林園東南天淵池水逕觀南觀古玩翻}
帝常言獄者天下之性命也每斷大獄常詣觀臨聽之【斷丁亂翻}
初魏文侯師李悝著法經六篇【悝苦回翻漢藝文志法家者流李子三十二篇注云李悝相魏富國強兵今言法經六篇蓋其書有經有解若韓非子也}
商君受之以相秦蕭何定漢律益為九篇後稍增至六十篇又有令三百餘篇决事比九百六卷【師古曰比以例相比况也程大昌曰古書皆卷至唐始為葉子今書冊也}
世有增損錯糅無常【糅女救翻糴也}
後人各為章句馬鄭諸儒十有餘家【馬鄭馬融鄭玄也}
以至於魏所當用者合二萬六千二百七十二條七百七十三萬餘言覽者益難帝乃詔但用鄭氏章句尚書衛覬奏曰【覬音冀}
刑法者國家之所貴重而私議之所輕賤獄吏者百姓之所縣命【縣讀曰懸}
而選用者之所卑下王政之敝未必不由此也請置律博士帝從之【晉職官志律博士屬廷尉}
又詔司空陳羣散騎常侍劉邵等刪約漢法制新律十八篇州郡令四十五篇尚書官令軍中令合百八十餘篇【州郡令用之刺史太守尚書令用之於國軍中令用之於軍}
於正律九篇為增於旁章科令為省矣 十一月洛陽廟成【元年初營宗廟至是而成}
迎高太武文四神主于鄴【高帝漢大長秋曹騰太帝漢太尉曹嵩裴松之曰魏初唯立親廟四祀四室而已至景初元年始定七廟之制}
 十二月雍丘王植徙封東阿 漢丞相亮徙府營於南山下原上築漢城於沔陽築樂城於成固【沔陽成固二縣皆屬漢中郡水經注沔水逕白馬戌城南城即陽平關也又東逕武侯壘南諸葛武侯所居也又東逕沔陽故城南城南對定軍山又東過南鄭縣又東過成固縣南如此則漢城在南鄭西樂城在南鄭東也又南鄭縣東南百八十里有梁州山與孤雲兩角山相接大山四圍其中三十里許甚平或云古梁州治也杜佑曰樂城在梁州西縣西南杜佑曰洋州興道縣漢成固縣地蜀之興埶宋白曰興勢山名在興道縣西北二十里洋州管下西鄉縣本成固縣地}


  四年春吳主使將軍衛温諸葛直將甲士萬人浮海求夷洲亶洲【後漢書東夷傳曰會稽海外有夷洲及亶洲傳言秦始皇使徐福將童男女數千人入海求蓬萊神仙不得福懼誅不敢還遂止此洲世世相承有數萬家人民時至會稽市會稽東冶縣人有入海行遭風流移至亶洲者所在絶遠不可往來沈瑩臨海水土志曰夷洲在臨海東去郡二千里土地無霜雪草木不死四面是山谿地有銅鐵唯用鹿骼為矛以戰鬬摩厲青石以作弓矢取生魚肉雜貯大瓦器中以鹽鹵之歷月餘日仍啖食之以為上肴也今人相傳倭人即徐福止王之地其國中至今廟祀徐福}
欲俘其民以益衆陸遜全琮皆諫以為桓王創基兵不一旅今江東見衆【見賢遍翻}
自足圖事不當遠涉不毛萬里襲人風波難測又民易水土必致疾疫欲益更損欲利反害且其民猶禽獸得之不足濟事無之不足虧衆吳主不聽尚書琅邪諸葛誕中書郎南陽鄧颺等【中書郎即通事郎晉志曰}


  【魏黄初初中書既置監令又置通事郎次黃門郎黃門郎已署事過通事乃署名已署奏以入為帝省讀書可及晉改曰中書侍郎颺余章翻又余亮翻}
相與結為黨友更相題表【更工衡翻}
以散騎常侍夏侯玄等四人為四聰誕輩八人為八達玄尚之子也中書監劉放子熙中書令孫資子密吏部尚書衛臻子烈三人咸不及比以其父居勢位容之為三豫【晉職官志曰漢武帝遊晏後庭始使宦者典事尚書謂之中書謁者置令僕射成帝改中書謁者令曰中謁者令罷僕射漢東京省中謁者令而有中官謁者令非其職也魏武帝為魏王置秘書令典尚書奏事文帝黃初初改為中書置監令以秘書左丞劉放為中書監右丞孫資為中書令監令自此始魏又改漢選郡尚書曰吏部尚書比等比也音毘寐翻三豫者容三人得豫於題品之中也}
行司徒事董昭【資望輕未可為公者為行事}
上疏曰凡有天下者莫不貴尚敦樸忠信之士深疾虚偽不真之人者以其毁教亂治敗俗傷化也【治直吏翻敗補邁翻}
近魏諷伏誅建安之末曹偉斬戮黃初之始【魏諷事見六十八卷建安二十四年曹偉事見六十九卷黃初二年}
伏惟前後聖詔深疾浮偽欲以破散邪黨常用切齒而執法之吏皆畏其權勢莫能糾擿【擿他狄翻}
毁壞風俗【壞音怪}
侵欲滋甚竊見當今年少不復以學問為本【少詩照翻}
專更以交游為業國士不以孝悌清修為首乃以趨勢游利為先【趨七喻翻}
合黨連羣互相褒歎以毁訾為罰戮【訾將此翻}
用黨譽為爵賞附已者則歎之盈言【歎者嗟歎而稱其美也盈溢也歎美之過溢於言辭則為溢美之言}
不附者則為作瑕釁【玉之病曰瑕器之隙曰釁}
至乃相謂今世何憂不度邪但求人道不勤羅之不博耳【言廣布黨友則互為羽翼身安而無患可以度世也}
人何患其不已知但當吞之以藥而柔調耳【謂毁譽所加彼誠好譽而惡毁則其心柔服調順於我無忤如吞之以藥也}
又聞或有使奴客名作在職家人冒之出入往來禁奥交通書疏有所探問【謂如職在尚書出入禁省則有令史有主書有蒼頭廬兒為之給使今使奴客冒其名以出入往來為姦}
凡此諸事皆法之所不取刑之所不赦雖諷偉之罪無以加也帝善其言二月壬午詔曰世之質文隨教而變【謂殷尚質周尚文各隨教而變也}
兵亂以來經學廢絶後生進趣不由典謨【三典三謨也}
豈訓導未洽將進用者不以德顯乎其郎吏學通一經才任牧民博士課試擢其高第者亟用其浮華不務道本者罷退之【郎吏謂尚書郎也}
於是免誕颺等官 夏四月定陵成侯鍾繇卒 六月戊子太皇太后卞氏殂秋七月葬武宣皇后 大司馬曹真以漢人數入寇【數所角翻}
請由斜谷伐之【斜余遮翻谷音浴}
諸將數道並進可以大克帝從之詔大將軍司馬懿泝漢水由西城入與真會漢中諸將或由子午谷或由武威入【武威恐當作武都否則建威也}
司空陳羣諫曰太祖昔到陽平攻張魯【事見六十七卷漢獻帝建安二十年}
多收豆麥以益軍糧魯未下而食猶乏今既無所因且斜谷阻險難以進退轉運必見鈔截【鈔楚交翻}
多留兵守要則損戰士不可不熟慮也帝從羣議真復表從子午道【復扶又翻}
羣又陳其不便并言軍事用度之計詔以羣議下真真據之遂行【詔以議下真將與之商度可否也真銳於出師遂以詔為據而行下遐稼翻}
八月辛巳帝行東巡乙未如許昌 漢丞相亮聞魏兵至次于成固赤坂以待之【赤坂在今洋州東二十里龍亭山坂色正赤魏兵泝漢水及從子午道入者皆會于成固故於此待之}
召李嚴使將二萬人赴漢中表嚴子豐為江州都督督軍典嚴後事【李嚴本都督江州今赴漢中令其子為督軍以典後事}
會天大雨三十餘日棧道斷絶太尉華歆上疏曰【華戶化翻上時掌翻}
陛下以聖德當成康之隆願先留心於治道【治直吏翻}
以征伐為後事為國者以民為基民以衣食為本使中國無饑寒之患百姓無離上之心則二賊之釁可坐而待也【魏以吳蜀為二賊}
帝報曰賊憑恃山川二祖勞於前世猶不克平【二祖謂太祖武皇帝世祖文皇帝也}
朕豈敢自多謂必滅之哉諸將以為不一探取【探他含翻}
無由自敝是以觀兵以闚其釁若天時未至周武還師乃前事之鑒朕敬不忘所戒少府楊阜上疏曰昔武王白魚入舟君臣變色【史記周文王崩武王奉文王木主東觀兵于孟津武王度河中流白魚躍入王舟是時諸侯不期而會者八百皆曰紂可伐矣武王曰汝未知天命未可也乃還師}
動得吉瑞猶尚憂懼况有災異而不戰竦者哉今吳蜀未平而天屢降變諸軍始進便有天雨之患稽閡山險【閡與礙同}
已積日矣轉運之勞擔負之苦所費已多若有不繼必違本圖傳曰見可而進知難而退【左傳隨武子之言}
軍之善政也徒使六軍困於山谷之間進無所略退又不得非王兵之道也【王兵王者之兵也}
散騎常侍王肅上疏曰前志有之千里饋糧士有饑色樵蘇後爨師不宿飽【前書李左車說陳餘之言蓋前乎左車已有是言矣}
此謂平塗之行軍者也又况於深入阻險鑿路而前則其為勞必相百也今又加之以霖雨山坂峻滑衆迫而不展糧遠而難繼實行軍者之大忌也聞曹真發已踰月而行裁半谷【謂子午谷之路行纔及半也}
治道功夫戰士悉作【治直之翻}
是賊偏得以逸待勞乃兵家之所憚也言之前代則武王伐紂出關而復還【復扶又翻}
論之近事則武文征權臨江而不濟【事見漢獻帝紀及魏文帝紀}
豈非所謂順天知時通於權變者哉兆民知上聖以水雨艱劇之故休而息之後日有釁乘而用之則所謂悦以犯難民忘其死者矣【易兌卦彖辭難乃旦翻}
肅朗之子也【王朗為公於黃初之初}
九月詔曹真等班師【班還也}
 冬十月乙卯帝還洛陽時左僕射徐宣總統留事【漢成帝罷中書宦者置尚書五人一人為僕射四人分為四曹一曰常侍曹二曰二千石曹三曰民曹四曰主客曹後又置三公曹是為五曹光武改常侍曹為吏部曹又置中都官曹合為六曹并令僕二人謂之八坐後改吏部為選部魏又改選部為吏部又有左民客曹五兵度支凡五曹尚書左右二僕射一合為八坐}
帝還主者奏呈文書【尚書諸曹各有主者還從宣翻又如字下同}
帝曰吾省與僕射省何異【省悉景翻}
竟不視 十二月改葬文昭皇后于朝陽陵【帝以舊陵庳下改葬朝陽陵亦在鄴}
 吳主揚聲欲至合肥征東將軍滿寵表召兖豫諸軍皆集吳尋退還詔罷其兵寵以為今賊大舉而還非本意也此必欲偽退以罷吾兵而倒還乘虛掩不備也表不罷兵【上表言敵情請不罷兵也}
後十餘日吳果更到合肥城不克而還 漢丞相亮以蔣琬為長史亮數外出【數所角翻}
琬常足食足兵以相供給亮每言公琰託志忠雅【蔣琬字公琰}
當與吾共贊王業者也 青州人隱蕃【姓譜隱以諡為氏}
逃奔入吳上書於吳主曰臣聞紂為無道微子先出【商紂無道微子抱祭器而奔周}
高祖寛明陳平先入【事見九卷漢高帝二年}
臣年二十二委棄封域歸命有道賴蒙天靈得自全致【言蒙天之靈得自全而致身於吳也}
臣至止有日而主者同之降人未見精别【此主者謂主客之官降戶江翻别彼列翻}
使臣微言妙旨不得上達於邑三歎【於邑短氣貌讀如本字或曰於音烏邑烏合翻}
曷惟其已【用詩人語}
謹詣闕拜章乞蒙引見【見賢遍翻}
吳主即召入蕃進謝答問及陳時務甚有辭觀【言其敏於言辭美於儀觀也觀古玩翻}
侍中右領軍胡綜侍坐【吳置中領軍及左右領軍坐徂卧翻}
吳主問何如綜對曰蕃上書大語有似東方朔巧捷詭辨有似禰衡【禰乃禮翻}
而才皆不及吳主又問可堪何官綜對曰未可以治民【治直之翻}
且試都輦小職【國都在輦轂下故曰都輦}
吳主以蕃盛語刑獄用為廷尉監【自漢以來廷尉有正有監有平}
左將軍朱據廷尉郝普數稱蕃有王佐之才【數所角翻}
普尤與之親善常怨歎其屈於是蕃門車馬雲集賓客盈堂自衛將軍全琮等皆傾心接待惟羊衜及宣詔郎豫章楊迪【吳置宣詔郎掌宣傳詔命}
拒絶不與通潘濬子翥亦與蕃周旋【翥章庶翻杜預曰周旋相追逐也}
饋餉之濬聞大怒疏責翥曰吾受國厚恩志報以命【言志在致命以報國恩}
爾輩在都當念恭順親賢慕善何故與降虜交以糧餉之【降戶江翻}
在遠聞此心震面熱惆悵累旬【惆丑鳩翻}
疏到急就往使受杖一百促責所餉【濬欲布其子之罪於國中以絶後禍也使疏吏翻}
當時人咸怪之頃之蕃謀作亂於吳事覺亡走捕得伏誅吳主切責郝普普惶懼自殺朱據禁止【禁止者雖未下之獄使人守之禁其不得出入止不得與親黨交通也鄭樵通志曰禁止謂禁入殿省也符所屬行之盤洲洪氏曰魏晉以來三臺奏劾則符光禄勲加禁止解禁止亦如之禁止者身不得入殿省光禄勲主殿門故也}
歷時乃解 武陵五谿蠻夷叛吳吳主以南土清定召交州刺史呂岱還屯長沙漚口【呂岱討交州見上卷文帝黃初七年}


  資治通鑑卷七十一


    


 


 



 

 
  







 


  
  
 
 
 


  

 















	
	









































 
  



















 





 












  
  
  

 





