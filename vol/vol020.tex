






























































資治通鑑卷二十    宋 司馬光 撰

胡三省 音註

漢紀十二【起昭陽大淵獻盡重光協洽凡九年}


世宗孝武皇帝中之下

元狩五年春三月甲午丞相李蔡坐盗孝景園堧地葬其中當下吏自殺【堧而緣翻下遐嫁翻}
罷三銖錢更鑄五銖錢【去年廢半兩錢行三銖錢更工衡翻 考異曰漢書食貨志前以銷半兩錢鑄三銖錢明年以三銖錢輕更鑄五銖錢武帝元狩五年乃云罷半兩錢行五銖錢誤也}
於是民多盗鑄錢楚地尤甚上以為淮陽楚地之郊【師古曰郊謂交迫衝要之處}
乃召拜汲黯為淮陽太守【黯去年免故召拜之守式又翻}
黯伏謝不受印詔數彊予【彊其兩翻予讀曰與}
然後奉詔黯為上泣曰【為于偽翻下正為同}
臣自以為填溝壑不復見陛下【復扶又翻填大賢翻}
不意陛下復收用之臣常有狗馬病力不能任郡事【任音壬}
臣願為中郎出入禁闥補過拾遺臣之願也上曰君薄淮陽邪吾今召君矣【師古曰言後即召也}
顧淮陽吏民不相得【師古曰顧思念也言吏民不相安而失其所也}
吾徒得君之重【師古曰徒但也重威重也}
卧而治之黯既辭行過大行李息曰黯棄逐居郡不得與朝廷議矣【過古禾翻與讀曰預}
御史大夫湯智足以拒諫詐足以飾非務巧佞之語辨數之辭非肯正為天下言專阿主意主意所不欲因而毁之主意所欲因而譽之【譽音餘}
好興事舞文法【好呼到翻}
内懷詐以御主心外挾賊吏以為威重公列九卿不早言之公與之俱受其戮矣息畏湯終不敢言及湯敗上抵息罪【師古曰抵至也致之於罪也}
使黯以諸侯相秩居淮陽【如淳曰諸侯王相在郡守上秩真二千石月得百五十斛歲凡得千八百石二千石月得百二十斛歲凡得千四百四十石耳}
十歲而卒 詔徙姦猾吏民於邊 夏四月乙卯以太子少傅武彊侯莊青翟為丞相【武彊侯莊不識高祖功臣青翟其孫也班志武強縣屬廣川唐冀州武強縣是也}
天子病鼎湖甚【晉灼曰黄圖鼎湖宫名在京兆班志湖本在京北後分屬弘農索隱曰昔黄帝采首山銅鑄鼎於湖曰鼎湖即今之湖城縣也}
巫醫無所不致不愈游水發根言上郡有巫病而鬼神下之【服䖍曰游水縣名發根人名晉灼曰地理志游水水名在臨淮師古曰二說皆非也游水姓也發根名也盖因水為姓也本嘗遇病而神下之故為巫也下戶嫁翻降附也}
上召置祠之甘泉及病使人問神君神君言曰天子無憂病病少愈彊與我會甘泉【少詩沼翻彊其兩翻}
於是病愈遂起幸甘泉病良已【孟康曰良已善已謂瘉也}
置酒夀宫【帝置夀宫以奉神君臣瓚曰壽宫奉神之宫也楚辭曰蹇將澹兮壽宫括地志夀宫在雍州長安縣西北三十里長安故城中}
神君非可得見聞其言言與人音等時去時來來則風肅然居室帷中神君所言上使人受書其言命之曰畫法【孟康曰策畫之法也}
其所語世俗之所知也無絶殊者而天子心獨喜其事祕世莫知也【師古曰喜好也音計吏翻}
時上卒起幸甘泉【卒讀曰猝}
過右内史界中道多不治上怒曰義縱以我為不復行此道乎銜之【師古曰銜含也包含在心以為過也復扶又翻}


六年冬十月雨水無氷【雨于具翻}
上既下緡錢令而尊卜式【事見上卷四年}
百姓終莫分財佐縣官於是楊可告緡錢縱矣【縱放也肆也}
義縱以為此亂民部吏捕其為可使者天子以縱為廢格沮事【孟康曰武帝使楊可主告緡没入其財物縱捕其為可使者此為廢格詔書沮已成之事也格音閣沮才汝翻壞也 考異曰漢書武紀元鼎三年十一月令民告緡據義縱傳則在今冬}
棄縱市 郎中令李敢怨大將軍之恨其父【怨大將軍衛青也恨其父事見上卷四年師古曰令其父抱恨而死也}
乃擊傷大將軍大將軍匿諱之居無何【師古曰無何謂未多時也}
敢從上雍【師古曰雍之所在地形積高故曰上也上時掌翻雍於用翻}
至甘泉宫獵票騎將軍去病射殺敢【射而亦翻考異曰史記封禪書云明年天子病鼎湖甚病愈幸甘泉大赦莫知其為何年本紀皆無其事獨義縱傳有之按漢書百官公卿表義縱李敢死皆在今年敢傳云從上雍至甘泉宫雍盖衍字也平凖書云自造白金五銖錢後五歲赦按武紀元狩四年造白金元鼎元年赦首尾四年若今年更有赦則四年再赦與平凖書不合今從百官表}
去病時方貴幸上為諱云鹿觸殺之【為于偽翻}
夏四月乙巳廟立皇子閎為齊王旦為燕王胥為廣陵王初作誥策【師古曰於廟中策命之服䖍曰誥敕王如尚書諸誥李奇曰今敕封拜諸王策文起於此毛晃曰漢制天子之策長二尺釋名曰策書教令於上所以驅策於下也}
自造白金五銖錢後吏民之坐盗鑄金錢死者數十萬人其不發覺者不可勝計【勝旨升}
天下大抵無慮皆鑄金錢矣【師古曰抵歸也大歸猶言大凡也無慮亦謂大率無少計慮云耳}
犯者衆吏不能盡誅六月詔遣博士褚大徐偃等六人【姓譜宋恭公子石食采于褚其德可師號曰褚師因以命氏}
分循郡國舉兼并之徒及守相為吏有罪者【守郡守相諸侯相也}
秋九月冠軍景桓侯霍去病薨【冠古玩翻}
天子甚悼之為冢像祁連山初霍仲孺吏畢歸家【霍仲孺本河東平陽縣吏給事平陽侯家與侍者衛少兒私通而生去病吏畢言為吏畢免歸家也}
娶婦生子光去病既壮大乃自知父為霍仲孺會為票騎將軍擊匈奴道出河東遣吏迎仲孺而見之大為買田宅奴婢而去【為于偽翻}
及還因將光西至長安任以為郎稍遷至奉車都尉【任保任也帝置奉車都尉掌御乘輿車秩比二千石}
光禄大夫 是歲大農令顔異誅【景帝後元年更治粟内史為大農令 考異曰徐廣注史記平凖書云異誅在元狩四年壬戌歲廣見漢書百官公卿表其年注云大農令顔異二年坐腹非誅不思有二年字致此誤也}
初異以廉直稍遷至九卿上與張湯既造白鹿皮幣【見上卷四年}
問異異曰今王侯朝賀以蒼璧直數千而以皮薦反四十萬【時王侯朝賀以皮幣薦璧故曰皮薦朝直遥翻}
本末不相稱天子不說【稱尺證翻說讀曰悦}
張湯又與異有郤【郤讀曰隙}
及人有告異以它事下張湯治異【下遐嫁翻}
異與客語初令下有不便者【李奇曰異與客語詔令初下有不便處}
異不應微反脣【師古曰盖非也}
湯奏當異九卿見令不便不入言而腹誹論死自是之後有腹誹之法比【師古曰比則例也讀如字又頻寐翻}
而公卿大夫多謟諛取容矣

元鼎元年【應劭曰得寶鼎故因是改元 考異曰漢書武紀此年云得鼎汾水上漢紀云六月得寶鼎于河東汾水上吾丘壽王對云云按封禪書欒大封樂通侯之歲其夏六月汾隂巫錦為民祠魏脽后土營旁得鼎詔曰問者廵祭后土云云武紀元鼎四年十月幸汾隂十一月立后土祠于汾隂脽上六月得寶鼎后土祠旁禮樂志又云元鼎五年得寶鼎恩澤侯表元鼎四年四月乙巳欒大封侯然則得鼎應在四年盖武紀因今年改元而誤增此得鼎一事耳非兩曾得鼎於汾水上也封禪書天子封泰山反至甘泉有司言寶鼎出為元鼎以今年為元封元年然則元鼎年號亦如建元元光皆後來追改之耳}
夏五月赦天下濟東王彭離驕悍【彭離梁孝王子景帝中六年受封濟子禮翻悍下罕翻又侯肝翻}


昬暮與其奴亡命少年數十人行剽殺人取財物以為好【如淳曰以是為好喜之事也剽匹妙翻刼也好呼到翻}
所殺發覺者百餘人坐廢徙上庸【班志上庸縣屬漢中郡}


二年冬十一月張湯有罪自殺初御史中丞李文與湯有郤【班表御史大夫有兩丞一曰中丞在殿中蘭臺掌圖籍秘書外督部刺史内領侍御史員十五人受公卿奏事舉劾按章成帝綏和元年更名御史大夫為大司空置長史而中丞官職如故哀帝建平二年復為御史大夫元夀二年又為大司空而中丞出外為御史臺主歷漢東京至魏晉以來皆然郤讀曰隙下同}
湯所厚吏魯謁居隂使人上變告文姦事事下湯治論殺之【上時掌翻下遐嫁翻}
湯心知謁居為之上問變事蹤跡安起湯佯驚曰此殆文過人怨之【師古曰殆近也}
謁居病湯親為之摩足【為于偽翻}
趙王素怨湯上書告湯大臣乃與吏摩足疑與為大姦事下廷尉謁居病死事連其弟弟繫導官【蘇林曰漢儀注獄二十六所導官無獄也師古曰蘇說非也導擇也以主擇米故曰導官時或以諸獄皆滿故權寄此署繫之非本獄所也班表導官属少府}
湯亦治它囚導官見謁居弟欲隂為之而佯不省【囚徐尤翻為于偽翻省心景翻}
謁居弟弗知怨湯使人上書告湯與謁居謀共變告李文事下減宣【減宣人姓名減古斬翻}
宣嘗與湯有郤及得此事窮竟其事未奏也會人有盜發孝文園瘞錢【如淳曰瘞埋也埋錢於園陵以送死也瘞於計翻}
丞相青翟朝與湯約俱謝【師古曰將入朝之時為此要約朝直遥翻}
至前【至帝之前也}
湯獨不謝【湯以丞相四時行園陵當謝御史大夫不預園陵事故不謝}
上使御史案丞相湯欲致其文丞相見知【欲以見知故縱之罪罪丞相}
丞相患之丞相長史朱買臣王朝邊通皆故九卿二千石【朱買臣嘗為主爵都尉王朝至右内史邊通至濟南相陳留風俗傳邊祖于宋平公子戎字子邊予按左傳周有大夫邊伯}
仕宦絶在湯前湯數行丞相事【數所角翻}
知三長史素貴故陵折丞史遇之三長史皆怨恨欲死之【欲以死發湯之姦也}
乃與丞相謀使吏捕案賈人田信等曰湯且欲奏請信輒先知之居物致富【服䖍曰居謂儲也賈音古下同}
與湯分之事辭頗聞【師古曰聞於天子也}
上問湯曰吾所為賈人輒先知之益居其物【師古曰益多也}
是類有以吾謀告之者【師古曰類似也}
湯不謝又佯驚曰固宜有減宣亦奏謁居等事天子以湯懷詐面欺【師古曰對面欺誣也}
使趙禹切責湯湯乃為書謝因曰陷臣者三長史也遂自殺湯既死家產直不過五百金昆弟諸子欲厚葬湯湯母曰湯為天子大臣被汙惡言而死【被皮義翻汙烏故翻}
何厚葬乎載以牛車有棺無槨天子聞之乃盡案誅三長史十二月壬辰丞相青翟下獄自殺 春起柏梁臺【服䖍曰用百頭梁作臺因名焉師古曰三輔舊事云以香柏為之今書皆作柏服說非也}
作承露盤高二十丈【高居號翻}
大七圍以銅為之上有仙人掌以承露和玉屑飲之云可以長生宫室之修自此日盛 二月以太子太傅趙周為丞相 三月辛亥以太子太傅石慶為御史大夫【衛有大夫石氏}
大雨雪【雨于具翻}
夏大水關東餓死者以千數是歲孔僅為大農令而桑弘羊為大農中丞【班表大農有兩丞元狩四年以東郭咸陽及孔僅為之今置中丞其位當在兩丞上}
稍置均輸以通貨物【時置均輸官於郡國令遠方各以其物而灌輸置平凖於京師都受天下委輸貴則賣之賤則買之使富商大賈無所牟利杜佑曰漢武帝置均輸謂所當輸於官者皆令輸其土地所饒平其所在時價官更於他處賣之輸者既便而官有利}
白金稍賤民不寶用竟廢之【鑄白金見上卷元狩四年}
於是悉禁郡國無鑄錢專令上林三官鑄錢令天下非三官錢不得行【裴駰曰百官表水衡都尉掌上林苑属官有上林均輸鍾官辨銅令然則上林三官其是此三令乎}
而民之鑄錢益少計其費不能相當惟真工大姦乃盜為之 渾邪王既降漢【見上卷元狩元年}
漢兵擊逐匈奴於幕北【見元狩元年}
自鹽澤以東空無匈奴西域道可通于是張騫建言烏孫王昆莫本為匈奴臣後兵稍彊不肯復朝事匈奴匈奴攻不勝而遠之【朝直遥翻遠于願翻}
今單于新困於漢而故渾邪地空無人蠻夷俗戀故地又貪漢財物今誠以此時厚幣賂烏孫招以益東居故渾邪之地【張騫傳昆莫父難兜靡本與大月氏同在敦煌祁連間小國也大月氏攻殺難兜靡奪其地而大月氏又為匈奴所破西擊塞王而奪其國昆莫報父怨西攻破大月氏國因留居為烏孫國騫欲誘之復歸故地}
與漢結昆弟其埶宜聽聽則是斷匈奴右臂也【斷丁管翻}
既連烏孫自其西大夏之屬皆可招來而為外臣天子以為然拜騫為中郎將將三百人馬各二匹牛羊以萬數齎金幣帛直數千巨萬多持節副使【師古曰為騫之副而各令持節也}
道可便遣之它旁國【沿道有便可通使他國者即遣之}
騫既至烏孫昆莫見騫禮節甚倨騫諭指曰【師古曰以天子意指曉吿之}
烏孫能東居故地則漢遣公主為夫人結為兄弟共距匈奴匈奴不足破也烏孫自以遠漢未知其大小素服屬匈奴日久且又近之【近其靳翻}
其大臣皆畏匈奴不欲移徙騫留久之不能得其要領【要讀曰腰}
因分遣副使使大宛康居大月氏大夏安息身毒于闐及諸旁國烏孫發譯道送騫還【宛於元翻氏音支身毒音捐篤闐從賢翻師古曰道讀曰導}
使數十人馬數十匹隨騫報謝因令窺漢大小是歲騫還到拜為大行後歲餘騫所遣使通大夏之屬者皆頗與其人俱來【晉灼曰其國人}
於是西域始通於漢矣西域凡三十六國南北有大山中央有河【西域始通於漢凡三十六國其後分置五十餘國婼羌鄯善且末小宛精絶戎盧扞彌渠勒皮山烏秅西夜蒲犂子合依耐無雷難兜罽賓烏弋山離犂鞬條支安息大月氏大夏康居奄蔡大宛桃槐休循捐篤莎車疏勒尉頭烏孫姑墨温宿龜兹烏壘渠犂尉犂危須焉耆烏貪訾離卑陸卑陸後國郁立師單桓蒲類蒲類後國西且彌東且彌刼國山國狐胡車師前後王是也南北有大山者南山在于窴之南東出金城與漢南山接北山在車師之北即唐志所謂西州交河縣北柳谷金沙嶺等山是也中央有河者河有兩源一出葱嶺一出于窴南山其河北流與葱嶺河合注蒲昌海自于窴以西水皆西流逕休循罽賓大月氏安息等國而入于西海蒲昌之水潜行地下南出積石為中國河西海之水東南合于交州漲海}
東西六千餘里南北千餘里東則接漢玉門陽關【班志敦煌郡龍勒縣有玉門關陽關酒泉郡有玉門縣闞駰曰漢罷玉門關屯置其人於此括地志沙州龍勒山在縣南百六十五里玉門關在縣西北百一十八里}
西則限以葱嶺【西河舊事葱嶺其山高大上悉生葱故以名焉}
河有兩源一出葱嶺一出于窴合流東注鹽澤鹽澤去玉門陽關三百餘里自玉門陽關出西域有兩道從鄯善傍南山北循河西行至莎車為南道【鄯善亦曰樓蘭國治杆泥城去陽關千六百里鄯上扇翻傍步浪翻莎車治莎車城去長安九千九百五十里莎素河翻}
南道西踰葱嶺則出大月氏安息自車師前王廷隨北山循河西行至疏勒為北道【車師前王治交河城去長安八千一百五十里唐西州交河縣是也疏勒治疏勒城去長安九千三百五十里西當大月氏大宛康居之道}
北道西踰葱嶺則出大宛康居奄蔡焉【杜佑曰奄蔡後為肅特國}
故皆役屬匈奴匈奴西邊日逐王置僮僕都尉【匈奴盖以僮僕視西域諸國故以名官}
使領西域常居焉耆危須尉黎間【焉耆治員渠城去長安七千三百里危須治危須城在焉耆東百里去長安七千二百九十里尉犂治尉犂城去長安六千七百五十里南接鄯善且末二國}
賦税諸國取富給焉烏孫王既不肯東還漢乃於渾邪王故地置酒泉郡【應劭曰其水如酒故曰酒泉師古曰城下有金泉泉味如酒唐為肅州宋白曰東南至長安二千九百里}
稍發徙民以充實之後又分置武威郡【本匈奴休屠王所居地大初四年分置武威郡唐之凉州即其地宋白曰東南至長安二千八百里}
以絶匈奴與羌通之道天子得宛汗血馬愛之名曰天馬使者相望於道以求之諸使外國一輩大者數百少者百餘人人所齎操大放博望侯時【齎資也操持也放依也言遣使所將節幣大槩依遣博望侯時也放讀倣}
其後益習而衰少焉【師古曰以其串習故不多發人少詩沼翻}
漢率一歲中使多者十餘少者五六輩遠者八九歲近者數歲而反

三年冬徙函谷關於新安【據班史以故關為弘農縣應劭曰弘農去新安三百里述征記新安縣今猶謂之新關}
春正月戊子陽陵園火 夏四月雨雹【雨于具翻}
關東郡國十餘飢人相食 常山憲王舜薨【舜景帝子中五年受封謚法博聞多能曰憲}
子勃嗣坐憲王病不侍疾及居喪無禮廢徙房陵【班志房陵縣属漢中郡宋白曰闞駰云即春秋防渚地漢獻帝改防為房兼立房陵郡今為房州}
後月餘天子更封憲王子平為真定王【真定縣本属常山今分真定綿曼藁城肥纍四縣為王國}
以常山為郡于是五嶽皆在天子之邦矣【華山嵩高本在天子之郡南嶽霍山属廬江淮南衡山謀反國除入漢為郡元狩元年濟北王獻太山及其旁邑今又以常山為郡然後皆在天子之邦}
徙代王義為清河王【義文帝子代王參之孫王登之子清河王乘孝景之子薨無子國除徙代王王焉}
是歲匈奴伊穉斜單于死子烏維單于立

四年冬十月上行幸雍祠五畤【雍於用翻畤音止}
詔曰今上帝朕親郊而后土無祀則禮不荅也【師古曰荅對也郊天而不祀地失對偶之義一曰闕地祗之祀不為神所荅應}
其令有司議立后土祠於澤中圜丘【郊祀志有司議祠后土宜於澤中園丘為五壇}
上遂自夏陽東幸汾隂【班志夏陽縣屬左馮翊汾隂縣屬河東郡}
是時天子始廵郡國河東守不意行至不辦自殺【不意天子行幸至郡供具不能備也}
十一月甲子立后土祠於汾隂脽上【如淳曰脽者河之東㟁特堆堀長四五里廣二里餘高十餘丈汾隂縣治脽之上后上祠在縣西汾在脽之北西流與河合師古曰脽者以其形高起如人尻脽故以名云一說此臨汾水之上地本名鄈音與葵同彼郷人呼葵音如誰故轉而為脽字耳故漢舊儀曰鄈上脽音誰}
上親望拜如上帝禮禮畢行幸滎陽還至洛陽【班志滎陽洛陽並屬河南郡}
封周後姬嘉為周子南君【臣瓚曰汲冢古文謂衛將軍文子為子南彌牟其後有子南固子南勁紀年勁朝於魏後惠成王如衛命子南為侯秦并六國衛最後亡疑嘉是衛後故氏子南而稱君也師古曰子南其封邑之號以為周後故總言周子南君瓚說非也例不先言姓而後稱君且自嘉以下皆姓姬著於史傳予據恩澤侯表周子南君食邑於潁川長社}
春二月中山靖王勝薨【勝景帝子中二年受封}
樂成侯丁義【義高祖功臣丁禮之曾孫班志樂成侯國属南陽郡 考異曰漢書郊祀志作樂成侯登按史記漢書功臣表當為丁義今從史記漢書功臣表}
薦方士欒大云與文成將軍同師上方悔誅文成【誅文成見上卷元狩四年}
得欒大大說【說讀曰悦}
大先事膠東康王【康王寄上弟也}
為人長美言【師古曰善為甘美之言}
多方畧而敢為大言處之不疑【處昌呂翻}
大言曰臣常往來海中見安期羨門之屬顧以臣為賤不信臣又以為康王諸侯耳不足與方臣之師曰黄金可成而河决可塞【塞悉則翻}
不死之藥可得仙人可致也然臣恐效文成則方士皆掩口惡敢言方哉【惡音烏}
上曰文成食馬肝死耳【索隱曰論衡云氣勃而毒盛故食走馬肝馬肝殺人儒林傳食肉無食馬肝是也}
子誠能修其方我何愛乎大曰臣師非有求人人者求之陛下必欲致之則貴其使者令為親屬以客禮待之乃可使通言于神人於是上使驗小方鬬旗旗自相觸擊 【考異曰封禪書郊祀志皆作棊獨史記孝武紀作旗按漢武故事云大嘗於殿前樹旍數百枚大令旍自相擊繙繙竟庭中去地十餘文觀者皆駭然則作旗字者是也}
是時上方憂河決而黄金不就乃拜大為五利將軍又拜為天士將軍地士將軍大通將軍夏四月乙巳封大為樂通侯【恩澤侯表樂通侯食邑於安定郡高平縣}
食邑二千戶賜甲第僮千人乘輿斥車馬帷帳器物以充其家【師古曰斥不用者也}
又以衛長公主妻之【乘繩證翻長知兩翻孟康曰衛太子妹如淳曰衛太子姊也師古曰外戚傳云子夫生三女元朔三年生男據是則衛太子之姊也孟說非是妻七細翻}
齎金十萬斤天子親如五利之第使者存問共給【共讀曰供}
相屬於道【屬之欲翻}
自太主將相以下【太主帝姑竇大主也}
皆置酒其家獻遺之【遺于季翻}
天子又刻玉印曰天道將軍【據前史下文言為天子道天神則道讀曰導}
使使衣羽衣夜立白茅上五利將軍亦衣羽衣立白茅上受印以示不臣【羽衣緝羽毛為衣也今道士服被之使衣亦衣於既翻}
大見數月佩六印【五利天士地士大通天道五將軍并樂通侯為六卬}
貴震天下于是海上燕齊之間莫不搤腕自言有禁方能神仙矣【搤音厄腕烏貫翻}
六月汾隂巫錦【應劭曰錦巫名}
得大鼎於魏脽后土營旁【師古曰汾脽本魏地之墳故曰魏脽也營謂后土祠之兆域}
河東太守以聞天子使驗問巫得鼎無奸詐乃以禮祠迎鼎至甘泉從上行【如淳曰以鼎從行上甘泉}
薦之宗廟及上帝藏於甘泉宫羣臣皆上夀賀秋立常山憲王子商為泗水王【泗水統凌泗陽于三縣本屬東海郡帝分為王國}
初條侯周亞夫為丞相【周亞夫景帝前七年為相中三年罷}
趙禹為丞相史府中皆稱其廉平然亞夫弗任曰極知禹無害【漢書音義曰文無所枉害蕭何以文無害為沛主吏掾章懷太子賢曰案律有無害都吏如今言公平吏蘇林曰無害若言無比也一曰害勝也無能勝害之者師古曰傷害也無人能傷害之者貢父曰持法者或以私意陷人謂之害故貴於文無害無害者取其為人無害於行則可以為史矣}
然文深不可以居大府【應劭曰禹持文法深劾}
及禹為少府比九卿為酷急【言以當時九卿同列者比之禹為酷急也}
至晚節吏務為嚴峻而禹更名寛平中尉尹齊素以敢斬伐著名【姓譜少昊之子封于尹城子孫因以為氏按尹氏周之世卿}
及為中尉吏民益彫敝是歲齊坐不勝任抵罪【勝音升}
上乃復以王温舒為中尉趙禹為廷尉後四年禹以老貶為燕相是時吏治皆以慘刻相尚【治直吏翻}
獨左内史兒寛勸農業緩刑罰理獄訟務在得人心擇用仁厚士推情與下不求名聲吏民大信愛之收租税時裁濶狹與民相假貸【師古曰謂有貧弱及農要之時不即徵收也予謂濶謂征斂稍寛禁防疏潤之時狹謂督促迫急之時濶時不急征收假貸與民使營生業}
以故租多不入後有軍發左内史以負租課殿當免【殿丁練翻課下下曰殿}
民聞當免皆恐失之大家牛車小家擔負輸租繦屬不絶【師古曰繦索也言輸者接連不絶於道若繩索之相屬也猶今言續索矣屬之欲翻}
課更以最【課上上曰最}
上由此愈奇寛 初南越文王遣其子嬰齊入宿衛【南越王胡薨謚文王嬰齊入宿衛見十七卷建元元年}
在長安取邯鄲樛氏女【取讀曰娶邯鄲屬趙國師古曰樛居虯翻}
生子興文王薨嬰齊立乃藏其先武帝璽【趙佗自號南越武帝李奇曰去其僭號}
上書請立樛氏女為后興為嗣漢數使使者風諭嬰齊入朝【數所角翻師古曰風讀曰諷諷諭令入朝}
嬰齊尚樂擅殺生自恣懼入見要用漢法比内諸侯【樂音洛見賢遍翻下同要讀曰邀恐漢邀之以用朝廷之法如内諸侯王}
固稱病遂不入見嬰齊薨謚曰明王太子興代立其母為太后太后自未為嬰齊姬時嘗與覇陵人安國少季通【師古曰姓安國字少季少詩沼翻}
是歲上使安國少季往諭王王太后以入朝比内諸侯令辨士諫大夫終軍等宣其辭【百官表元狩五年初置諫大夫秩八百石}
勇士魏臣等輔其决【師古曰助令决策也}
衛尉路博德將兵屯桂陽【班志桂陽縣屬桂陽郡唐為連州桂陽連山二縣地}
待使者南越王年少太后中國人安國少季往復與私通國人頗知之多不附太后太后恐亂起亦欲倚漢威數勸王及羣臣求内屬【數所角翻}
即因使者上書請比内諸侯三歲一朝【朝直遥翻}
除邊關于是天子許之賜其丞相呂嘉銀印及内史中尉太傅印餘得自置除其故黥劓刑用漢法比内諸侯使者皆留填撫之【漢制諸侯王國二千石以上皆漢朝所命餘得自置今賜南越丞相内史中尉太傅印使之比内諸侯也漢自文帝除肉刑不用黥劓之法故亦令南越除之劓魚器翻又牛例翻填讀曰鎮為呂嘉反張本}
上行幸雍【雍於用翻}
且郊或曰五帝泰一之佐也宜立泰一而上親郊上疑未定齊人公孫卿曰今年得寶鼎其冬辛巳朔旦冬至與黄帝時等卿有札書【師古曰等同也札木簡之薄小者也}
曰黄帝得寶鼎是歲己酉朔旦冬至凡三百八十年黄帝仙登于天因嬖人奏之【嬖卑義翻又博計翻}
上大悦召問卿對曰受此書申公申公曰漢興復當黄帝之時漢之聖者在高祖之孫且曾孫也寶鼎出而與神通黄帝接萬靈明庭明庭者甘泉也黄帝采首山銅【班志河東蒲坂縣有首山}
鑄鼎于荆山下【班志馮翊懷德縣有荆山}
鼎既成有龍垂胡䫇下迎黄帝【師古曰胡謂頷下垂肉也䫇其毛也䫇人占翻}
黄帝上騎龍與羣臣後宫七十餘人俱登天於是天子曰嗟乎誠得如黄帝吾視去妻子如脱屣耳【師古曰屣小履也脱屣者言其便易無所顧也屣山爾翻}
拜卿為郎使東候神于太室【師古曰太室山在潁川崇高縣是為中岳}


五年冬十月上祠五畤於雍遂踰隴【隴坻也在天水郡隴縣三奉記曰其坂九曲上隴者七日乃越}
西登崆峒【唐地理志崆峒在岷州溢樂縣西岷州漢臨洮之地史記作空桐正義曰空桐山在原州平高縣西百里}
隴西守以行往卒【卒讀曰猝}
天子從官不得食惶恐自殺【從才用翻}
於是上北出蕭關從數萬騎獵新奏中以勒邊兵而歸新秦中或千里無亭徼於是誅北地大守以下【唐麟州治新秦杜佑漢新秦中地予謂唐取漢新秦中之名以名郡巨麟州不能盡有漢新秦中之地也北地與朔方接境時朔方新置郡盖使北地并力以營築亭徼也徼吉弔翻}
上又幸甘泉立泰一祠壇所用祠具如雍一畤而有加焉【雍有五畤今祠太一所用如雍一畤之祠具也有加者加醴棗脯之屬}
五帝壇環居其下四方地為醊食羣神從者及北斗云【說文醊祭酎也師古曰謂聯屬而祭也醊竹芮翻食讀曰飤從才用翻}
十一月辛巳朔冬至昧爽【昧冥也爽明也謂日尚昧昧而天色漸明也}
天子始郊拜泰一朝朝日夕夕月則揖【應劭曰天子春朝日秋夕月朝日以朝夕月以夕臣瓚曰漢儀注郊泰畤皇帝平旦出竹宮東向揖日其夕西南向揖月便用郊日不用春秋也師古曰春朝朝日秋暮夕月盖常禮郊奉畤而揖日月此又别儀朝朝下直遥翻下同}
其祠列火滿壇壇旁亨炊具【亨讀曰烹}
有司云祠上有光又云晝有黄氣上屬天【屬之欲翻}
太史令談祠官寛舒等【班表大史令屬太常劉昭志秩六百石掌天時星歷凡國祭祀喪娶之事談即司馬談祠官掌祠祀之官寛舒史逸其姓}
請三歲天子一郊見【見賢遍翻}
詔從之南越王王太后飭治行重齎【治直之翻齎讀曰資}
為入朝具

其相呂嘉年長矣相三王宗族仕宦為長吏者七十餘人男盡尚王女女盡嫁王子弟宗室及蒼梧秦王有連【孟康曰蒼梧越中王自名為秦王連親婚也晉灼曰秦王即後趙光趙本與秦同姓故曰秦王予據南越王姓趙曷為不稱南越秦王晉說未為通長知兩翻}
其居國中甚重得衆心愈於王【師古曰愈勝也}
王之上書數諫止王王弗聽有畔心數稱病不見漢使者【數所角翻}
使者皆注意嘉執未能誅王王太后亦恐嘉等先事發【先悉薦翻}
欲介漢使者權謀誅嘉等【韋昭曰恃使者為介冑也索隱曰志林云介者因也欲因使者權誅呂嘉也韋昭以介為恃介者間也以言聞恃漢使之權意即得矣然云恃為介胄則非也虞喜以介為因亦有所由介者賓主所因也}
乃置酒請使者大臣皆侍坐飲【坐徂卧翻}
嘉弟為將將卒居宫外【將即亮翻}
酒行太后謂嘉曰南越内屬國之利也而相君苦不便者何也以激怒使者使者狐疑相杖【杖直亮翻}
遂莫敢發嘉見耳目非是【師古曰言異於常也}
即起而出太后怒欲鏦嘉以矛【鏦楚江翻}
王止太后嘉遂出介其弟兵就舍【李奇曰介被也師古曰介甲也被甲以自衛也弟兵即上所云弟將卒居外者}
稱病不肯見王及使者隂與大臣謀作亂王素無意誅嘉嘉知之以故數月不發天子聞嘉不聽命王王太后孤弱不能制使者怯無决又以為王王太后已附漢獨呂嘉為亂不足以興兵欲使莊參以二千人往使【往使疏吏翻}
參曰以好往數人足矣以武往二千人無足以為也辭不可天子罷參郟壮士故濟北相韓千秋【班志郟縣屬潁川郡史記正義曰今汝州郟城縣郟音夾千秋盖相濟北成王胡也胡貞王勃之子}
奮曰以區區之越又有王王太后應獨相呂嘉為害願得勇士三百人必斬嘉以報于是天子遣千秋與王太后弟樛樂將二千人往入越境【樛居虯翻}
呂嘉等乃遂反下令國中曰王年少太后中國人也又與使者亂專欲内屬盡持先王寶器入獻天子以自媚多從人行至長安虜賣以為僮僕取自脱一時之利無顧趙氏社稷為萬世慮計之意乃與其弟將卒攻殺王王太后及漢使者遣人告蒼梧秦王及其諸郡縣立明王長男越妻子術陽侯建德為王【建德降漢始封術陽侯史盖追書也班表術陽侯食邑於東海之下邳長知兩翻}
而韓千秋兵入破數小邑其後越開直道給食【師古曰縱之令深入然後擊滅之}
未至番禺四十里【番禺南越都翻音潘}
越以兵擊千秋等遂滅之使人函封漢使者節置塞上好為謾辭謝罪【師古曰謾誰也音慢又莫連翻}
發兵守要害處春三月壬午天子聞南越反曰韓千秋雖無功亦軍鋒之冠【冠古玩翻}
封其子延年為成安侯【班表成安侯食邑於潁川郡之郟縣}
樛樂姊為王太后首願屬漢封其子廣德為龍亢侯【班志龍亢縣屬沛國亢音剛 考異曰漢書功臣表作龍侯南越傳作㰍侯晉灼曰㰍古龍字史記建元以來侯者表及南越傳皆作龍亢侯今從之}
夏四月赦天下 丁丑晦日有食之 秋遣伏波將

軍路博德【環濟要畧曰伏波將軍者船涉江海欲使波濤伏息也}
出桂陽下湟水【水經滙水出桂陽縣盧聚南出貞女峽合洭水東南過含洭縣南出洭湳關為桂水山海經以洭水為湟水徐廣曰湟水一名洭水出桂陽通四會師古曰湟音皇}
樓船將軍楊僕出豫章下湞水【應劭曰湞水出南海龍川西入秦水水經湞水逕桂陽郡之湞陽縣南而右注溱水湞鄭氏曰湞音檉孟康曰湞音貞師古曰湞丈庚翻}
歸義越侯嚴為戈船將軍出零陵下離水【張晏曰嚴故越人降為歸義侯越人於水中負人船又有蛟龍之害故置戈於船下因以為名臣瓚曰伍子胥書有戈船以載干戈因謂之戈船也師古曰以樓船之例言之非謂載干戈也此盖船下安戈以禦蛟鼉水虫之害張說近之貢父曰船下安戈既難措置又不可以行今造舟船甚多未嘗有置戈者顔北人不晚行船故信張說盖瓚說是予據表無歸義越侯嚴零陵本屬桂陽帝分置郡唐為永道二州灕水班志出零陵縣陽海山東南至廣信入欝水}
甲為下瀨將軍下蒼梧【服䖍曰甲故越人歸漢者臣瓚曰瀨湍也吳越謂之瀨中國謂之磧伍子胥書有下瀨船瀨音賴蒼梧本越地帝始置郡有灕水關唐梧賀康端封之地}
皆將罪人江淮以南樓船十萬人越馳義侯遺别將巴蜀罪人發夜郎兵下牂柯江咸會番禺齊相卜式上書請父子與齊習船者往死南越天子下詔襃美式賜爵關内侯金六十斤田十頃布告天下天下莫應是時列侯以百數皆莫求從軍擊越會九月嘗酎祭宗廟列侯以令獻金助祭少府省金金有輕及色惡者上皆令劾以不敬奪爵者百六人【如淳曰漢儀注王子為侯歲以黄金嘗酎於漢廟皇帝臨受獻金金少不如斤兩色惡王削縣侯免國余據當時失侯者列侯王子侯共一百六人盖不特王子侯有酎金也酎直又翻省悉景翻劾戶槩翻}
辛巳丞相趙周坐知列侯酎金輕下獄自殺【下遐嫁翻}
丙申以御史大夫石慶為丞相封牧丘侯【思澤侯表牧丘侯食邑平原}
時國家多事桑弘羊等致利王温舒之屬峻法而兒寛等推文學皆為九卿更進用事【更工衡翻}
事不關决於丞相丞相慶醇謹而已【師古曰醇專厚也}
五利將軍装治行東入海求其師既而不敢入海之太山祠上使人隨驗實無所見五利妄言見其師其方盡多不售【師古曰售應當也不售者無驗也}
坐誣罔腰斬樂成侯亦棄市西羌衆十萬人反與匈奴通使【使疏吏翻}
攻故安圍枹罕【故安縣属涿郡西羌之兵安能至此當作安故班志安故枹罕二縣皆属隴西郡枹罕故罕羌邑宋白曰安故故城在蘭州南枹罕今河州治所枹音膚罕如字}
匈奴入五原【五原即秦九原郡帝更名唐為鹽州宋白曰五原郡有原五所故名謂龍游原乞地千原青嶺原岢嵐真原横槽原也五原故城在今榆林縣界}
殺太守【守式又翻下同}


六年冬發卒十萬人遣將軍李息郎中令徐自為征西羌平之 樓船將軍楊僕入越地先陷尋陿【陿作陜音姚氏曰尋陿在始興西三百里近連口也陿音狹}
破石門【石門在番禺西北二十里郡國志呂嘉拒漢積石江中為門因名石門}
挫越鋒以數萬人待伏波將軍路博德至俱進樓船居前至番禺南越王建德相呂嘉城守樓船居東南面伏波居西北面會暮樓船攻敗越人縱火燒城【敗蒲邁翻}
伏波為營【師古曰設營壘以待降者}
遣使者招降者賜印綬復縱令相招【師古曰來降者即賜以侯印而放令還更相招諭復扶又翻}
樓船力攻燒敵驅而入伏波營中黎旦城中皆降建德嘉已夜亡入海伏波遣人追之校尉司馬蘇弘得建德越郎都稽得嘉【孟康曰越中所自置郎也 考異曰史記漢書表皆作孫都南越傳皆云都稽今從傳}
戈船下瀨將軍兵及馳義侯所發夜郎兵未下南越已平矣遂以其地為南海蒼梧鬰林合浦交趾九真日南珠厓儋耳九郡【南海唐廣州循州之地蒼梧注見上欝林唐桂州欝林黨繡州之地合浦唐廉雷潘州之地交趾唐安南之地杜佑曰南方夷人其足大指開廣若並足而立其指交故名交趾劉欣期交州記曰交趾之人出南定縣足骨無節身有毛卧者更扶乃得起山海經交脛國為人交脛郭璞曰脚脛曲戾相交所謂雕題交趾也九真唐愛州之地日南唐驩州之地師古曰言其在日之南所謂開北戶以向日者珠厓儋耳唐瓊管之地應劭曰二郡在大海厓岸之邊出真珠故曰珠厓儋耳者種大耳其渠率自謂王者耳尤綏下肩三寸張晏曰異物志二郡在海中東西千里南北五百里儋耳之人鏤其頰皮上連耳匡分為數支狀如羊腸累耳而下垂賢曰儋耳故城即今儋州義倫縣儋丁甘翻臣瓚曰珠厓郡治䐺都去長安七千三百二十四里儋耳去長安七千三百三十五里見茂陵書}
師還上益封伏波封樓船為將梁侯蘇弘為海常侯都稽為臨蔡侯【徐廣曰海常在東萊余以王子侯表參考之則海常侯當食邑琅邪功臣表臨蔡侯食邑河内}
及越降將蒼梧王趙光等四人皆為侯【趙光封隨桃侯史定封安道侯畢取封膫侯居翁封湘城侯 考異曰凡此等封侯者年表皆有月日為其先後難齊故盡附於立功之處後倣此}
公孫卿候神河南言見仙人跡緱氏城上【班志緱氏縣屬河南郡宋白曰漢緱氏故城在今縣東南二十五里緱工侯翻}
春天子親幸緱氏城視跡問卿得毋效文成五利乎卿曰仙者非有求人主人主者求之其道非寛假神不來言神事如迂誕【師古曰迂回遠也誕大言也}
積以歲月乃可致也上信之於是郡國各除道繕治宫觀名山神祠以望幸焉【觀古玩翻}
賽南越祠泰一后土始用樂舞【據郊祀志五年秋為伐南越告禱太一故今賽祠賽先代翻}
馳義侯發南夷兵欲以擊南越且蘭君恐遠行【且蘭亦南夷種帝開為縣屬牂柯郡且音苴子閭翻}
旁國虜其老弱乃與其衆反殺使者及犍為太守【犍渠延翻守式又翻}
漢乃發巴蜀罪人當擊南越者八校尉遣中郎將郭昌衛廣將而擊之【將即亮翻}
誅且蘭及卭君莋侯【卭君卭都之君莋侯莋都之君莋才各翻下同}
遂平南夷為牂柯郡夜郎侯始倚南越南越已滅夜郎遂入朝【朝直遥翻}
上以為夜郎王冉駹皆振恐請臣置吏乃以卭都為越嶲郡【卭渠容翻越嶲郡唐為嶲州嶲音髓}
莋都為沈黎郡【服䖍曰今蜀郡北部都尉所治本莋都臣瓚曰茂陵書沈黎治莋都去長安三千三百三十五里唐為黎州地}
冉駹為汶山郡【駹莫江翻應劭曰今蜀郡㟭山本冉駹地宣帝地節四年省㟭山郡并蜀今茂州諸羌之地是也華陽國志汶山南接漢嘉西接凉州酒泉北接隂平皆其地也唐置茂州汶山縣注云有岷山類篇汶音岷又據史記夏紀引禹貢岷嶓既藝及岷山之陽及岷山導江之岷皆作汶盖漢時古字通用也康曰汶音問非也}
廣漢西白馬為武都郡【高祖置廣漢郡唐為梓州白馬居武都仇池班志所謂天池大澤括地志隴右成州武州皆白馬氐其豪族楊氏居成州仇池山上武都郡唐階成武等州地}
初

東越王餘善上書請以卒八千人從樓船擊呂嘉兵至掲陽【班志掲陽縣屬南海郡唐為潮州韋昭曰揭其逝翻蘇林音偈師古音竭}
以海風波為解不行持兩端隂使南越【使疏吏翻}
及漢破番禺不至楊僕上書願便引兵擊東越上以士卒勞倦不許令諸校屯豫章梅嶺以待命【徐廣曰梅嶺在會稽界索隱曰徐說非也案今豫章三十里有梅嶺在供崔山當古驛道杜佑曰梅嶺在䖍州䖍化縣界括地志在䖍化縣東北一百二十八里校戶教翻}
餘善聞樓船請誅之漢兵臨境乃遂反發兵距漢道號將軍騶力等為吞漢將軍入白沙武林梅嶺【索隱曰案今豫章北二百里接番陽界地名白沙沙束南八十里有武陽亭東南三十里地名武林當閩越之京道劉昫曰武林在蒼梧猛陵縣界隋分猛陵置武林縣属永平郡唐置龔州}
殺漢三校尉是時漢使大農張成故山州侯齒將屯【齒城陽共王子坐酎金失侯故書曰故侯將即亮翻下僕將同}
弗敢擊却就便處皆坐畏懦誅餘善自稱武帝上欲復使楊僕將為其伐前勞【為于偽翻}
以書敕責之曰將軍之功獨有先破石門尋陿非有斬將搴旗之實也【師古曰搴拔取之也}
烏足以驕人哉前破番禺捕降者以為虜【降戶江翻}
掘死人以為獲是一過也使建德呂嘉得以東越為援【師古曰以僕不窮追之故令得以東越為援也}
是二過也士卒暴露連歲將軍不念其勤勞而請乘傳行塞【傳張戀翻行下孟翻}
因用歸家懷銀黄垂三組夸郷里是三過也【師古曰銀銀印也黄金印也僕為主爵都尉又為樓船將軍并將梁侯故為三組組印綬也}
失期内顧【師古曰言顧思妻妾也}
以道惡為解是四過也問君蜀刀價而陽不知【蜀刀蜀中所作刀師古曰蜀刀有環者也}
挾偽干君【師古曰干犯也}
是五過也受詔不至蘭池【蘭池宫在渭城如淳曰本出軍時欲使之蘭池宫頓而不至}
明日又不對假今將軍之吏問之不對令之不從其罪何如推此心在外江海之間可得信乎今東越深入將軍能率衆以掩過不【不讀曰否}
僕惶恐對曰願盡死贖罪上乃遣横海將軍韓說出句章【班志句章縣屬會稽郡史記正義曰句章故城在越州鄮縣西一百里}
浮海從東方往樓船將軍楊僕出武林中尉王温舒出梅嶺以越侯為戈舩下瀬將軍出若邪白沙【若邪時屬會稽山隂縣界今之若邪溪在越州東南二十五里曰五雲溪}
以擊東越 博望侯既以通西域尊貴其吏士争上書言外國奇怪利害求使天子為其絶遠非人所樂往聽其言【師古曰凡人皆不樂去故有自請為使者即聽而遣之為于偽翻樂音洛使疏史翻下同}
予節募吏民毋問所從來【師古曰不為限禁遠近雖家人私隷並許應募予讀曰與}
為具備人衆遣之【為于偽翻下同}
以廣其道來還不能毋侵盗幣物及使失指【師古曰乖天子指意}
天子為其習之輒覆按致重罪以激怒令贖【師古曰言其串習不以為難必當更求充使令立功以贖罪}
復求使使端無窮而輕犯法【復扶又翻使疏吏翻下同}
其吏卒亦輒復盛推外國所有言大者予節言小者為副【予讀曰與}
故妄言無行之徒皆争效之【行下孟翻}
其使皆貧人子私縣官齎物【師古曰言所齎官物竊自用之同於私物}
欲賤市以私其利【師古曰所市之物得利多故不盡入官也}
外國亦厭漢使人人有言輕重【服䖍曰漢使言於外國人人輕重不實}
度漢兵遠不能至而禁其食物以苦漢使【師古曰令其困苦也度徒洛翻}
漢使乏絶積怨至相攻擊而樓蘭車師小國當空道【漢出西域有兩道南道從樓蘭北道從車師故二國當漢使空道師古曰空即孔也}
攻劫漢使王恢等尤甚而匈奴奇兵又時遮擊之使者争言西域皆有城邑兵弱易擊【易以䜴翻}
于是天子遣浮沮將軍公孫賀將萬五千騎出九原二千餘里至浮沮井而還【浮沮匈奴中井名出軍時期賀至浮沮井故以為將軍之號下匈河將軍其義類此沮子餘翻}
匈河將軍趙破奴將萬餘騎出令居數千里至匈河水而還【臣瓚曰匈奴河水去今居千里}
以斥逐匈奴不使遮漢使皆不見匈奴一人乃分武威酒泉地置張掖敦煌郡【應劭曰敦大也煌盛也張掖張國臂掖也敦音屯張掖昆邪王所居地唐為甘州敦煌唐為沙州 考異曰漢書武紀元狩二年渾邪王降以其地為武威酒泉郡元鼎六年分置張掖敦煌郡而地理志云張掖酒泉郡太初元年開武威郡太初四年開敦煌郡後元元年分酒泉置今從武紀}
徙民以實之 是歲齊相卜式為御史大夫式既在位乃言郡國多不便縣官作鹽鐵器苦惡【如淳曰苦或作鹽鹽不攻嚴也臣瓚曰謂作鐵器民患苦其不好也師古曰二說非也鹽既味苦器又脆惡故總云苦惡也余謂鹽器則官與牢盆是也鐵器則官鑄鐵器是也苦惡專指鹽鐵器而言如說未可厚非}
價貴或彊令民買之而船有算【船算及鹽鐵器並見上卷四年彊其兩翻}
商者少物貴【少詩沼翻}
上由是不悦卜式 初司馬相如病且死有遺書頌功德言符瑞勸上封泰山上感其言會得寶鼎上乃與公卿諸生議封禪封禪用希曠絶莫知其儀而諸方士又言封禪者合不死之名也【漢書作古不死之名}
黄帝以上封禪皆致怪物與神通秦皇帝不得上封陛下必欲上稍上即無風雨遂上封矣【上時掌翻師古曰稍漸也}
上於是乃令諸儒采尚書周官王制之文草封禪儀數年不成上以問左内史兒寛寛曰封泰山禪梁父昭姓考瑞帝王之盛節也【父音甫}
然享薦之義不著于經【師古曰封禪之享薦也以非常禮故經無其文著竹筯翻}
臣以為封禪告成合袪於天地神祗【李奇曰袪開散合閉也開閉於天地也袪丘居翻}
唯聖主所由制定其當【師古曰當猶中也}
非羣臣之所能列今將舉大事優游數年使羣臣得人人自盡【師古曰所言不同各有執見也}
終莫能成唯天子建中和之極兼總條貫金聲而玉振之【師古曰言振揚德音如金玉之聲也}
以順成天慶垂萬世之基上乃自制儀頗采儒術以文之上為封禪祠器以示羣儒或曰不與古同於是盡罷諸儒不用上又以古者先振兵釋旅然后封禪

元封元年【應劭曰始封泰山故改元}
冬十月下詔曰南越東甌咸伏其辜西蠻北夷頗未輯睦【師古曰輯與集同集和也}
朕將廵邊垂躬秉武節置十二部將軍親帥師焉【帥讀曰率}
乃行自雲陽【班志雲陽縣属左馮翊}
北歷上郡西河五原【元朔四年置西河郡其地自汾石州西北至塞下}
出長城北登單于臺【杜佑曰單于臺在雲州雲中縣西北百餘里}
至朔方臨北河勒兵十八萬騎旌旗徑千餘里以見武節威匈奴遣使者郭吉告單于曰南越王頭已縣於漢北闕【縣古懸通}
今單于能戰天子自將待邊【將即亮翻}
不能即南面而臣於漢何徒遠走亡匿于幕北寒苦無水草之地毋為也語卒而單于大怒【卒于恤翻}
立斬主客見者【師古曰主客主接諸客者也見者謂引見郭吉於單于者}
而留郭吉遷之北海上然匈奴亦讋【讋之涉翻師古曰失氣也}
終不敢出上乃還祭黄帝冢橋山【應劭曰橋山在上郡陽周縣}
釋兵須如【須如地名 考異曰漢書作凉如今從史記}
上曰吾聞黄帝不死今有冢何也公孫卿曰黄帝已仙上天羣臣思慕葬其衣冠 【考異曰史記漢書皆云或對漢武故事云公孫卿對今取之}
上歎曰吾後升天羣臣亦當葬吾衣冠於東陵乎【東陵謂茂陵也在長安東故曰東陵}
乃還甘泉類祠太一【師古曰類祠謂以事類而祭也}
上以卜式不習文章貶秩為太子太傅以兒寛代為御史大夫 漢兵入東越境東越素發兵距險使徇北將軍守武林樓船將軍卒錢塘轅終古斬狥北將軍【班志錢塘縣属會稽郡師古曰轅姓終古名}
故越衍侯吳陽以其邑七百人反攻越軍於漢陽越建成侯敖與繇王居股殺餘善以其衆降【據東越傳吳陽先在漢漢使歸喻餘善餘善不聽及漢軍至陽以邑人攻越書故越衍侯者言其舊為越衍侯也越衍侯及建成侯皆東越所封}
上封終古為禦兒侯【孟康曰禦兒越中地今吳南亭是也國語曰吾用禦兒臨之宋祁注云禦兒越北鄙今嘉興史記正義曰禦今作語語兒鄉在蘇州嘉興縣南七十里臨官道}
陽為卯石侯居股為東成侯敖為開陵侯又封横海將軍說為桉道侯横海校尉福為繚嫈侯東越降將多軍為無錫侯【卯石侯功臣表作外石食邑於濟南東成作東城屬九江郡開陵作國屬臨淮郡桉道功臣表作安道食邑於南陽索隱曰繚嫈縣名師古曰繚音遼嫈於耕翻横海校尉福城陽共王子海常侯福也坐法失侯以今功封繚嫈侯服䖍曰嫈音瑩劉伯莊曰紆營翻無錫縣属會稽郡}
上以閩地險阻數反覆【數所角翻}
終為後世患乃詔諸將悉徙其民於江淮之間遂虛其地【虚如字康讀曰墟}
春正月上行幸緱氏 【考異曰封禪書郊祀志作三月漢書武紀及荀紀皆作正月今從之}
禮祭中嶽太室從官在山下聞若有言萬歲者三【荀悦曰萬歲神稱之也從才用翻}
詔祠官加增太室祠禁無伐其草木以山下戶三百為之奉邑【奉扶用翻}
上遂東廵海上行禮祠八神齊人之上疏言神怪奇方者以萬數乃益發船令言海中神山者數千人求蓬萊神人公孫卿持節常先行候名山至東萊【東莱春秋萊子之國高祖置萊郡唐置為登萊二州之地}
言夜見大人長數丈【長直亮翻}
就之則不見其迹甚大類禽獸云羣臣有言見一老父牽狗言吾欲見鉅公【鄭氏曰鉅公天子也張晏曰天子為天下父故曰鉅公師古曰鉅大也}
已忽不見上既見大迹未信及羣臣又言老父則大以為仙人也宿留海上【師古曰宿留謂有所須待也宿先就翻留力就翻}
與方士傳車及間使求神仙人以千數【師古曰間微也隨間隙而行也}
夏四月還至奉高【奉高泰山郡治所}
禮祠地主於梁父【地主八神之一也梁父縣屬泰山郡父音甫}
乙卯令侍中儒者皮弁搢紳射牛行事【續漢志委貌皮弁同制長七寸高四寸制如覆盆前高廣後卑鋭所謂夏之毋追殷之章甫者也委貌以皂絹為之皮弁以鹿皮為之沈約曰古者貴賤皆執笏其有事則搢之於腰帶所謂搢紳之士者搢笏而垂紳紳帶也長三尺天子有事必自射牛示親殺也今採此禮以為封禪儀}
封泰山下東方 【考異曰武紀癸卯上還登封泰山盖癸卯自海上還乙卯至泰山行事也}
如郊祠泰一之禮封廣丈二尺高九尺【廣古曠翻度廣曰廣高居號翻度高曰高}
其下則有玉牒書書祕禮畢天子獨與侍中奉車都尉霍子侯上泰山【服䖍曰子侯霍去病子也上時掌翻下同}
亦有封其事皆禁明日下隂道【山北為隂}
丙辰禪泰山下阯【師古曰阯者山之基足阯音止}
東北肅然山如祭后土禮天子皆親拜見【見賢遍翻下同}
衣尚黄而盡用樂焉江淮間茅三脊為神藉【藉才夜翻薦也}
五色土益雜封其封禪祠夜若有光晝有白雲出封中【師古曰雲出於所封之中}
天子從禪還坐明堂【班志明堂在奉高西南四里臣瓚曰郊祀志初天子封泰山泰山東北阯古時有明堂處則此所坐者也明年秋乃作明堂}
羣臣更上夀頌功德【更互也工衝翻}
詔曰朕以眇身承至尊兢兢焉惟德菲薄不明于禮樂故用事八神遭天地况施【應劭曰况賜也施與也言天地神靈乃賜我瑞應施式智翻}
著見景象屑然如有聞【臣瓚曰聞呼萬歲者三是也}
震于怪物欲止不敢遂登封泰山至于梁父然後升䄠肅然【䄠與禪同}
自新嘉與士大夫更始【更工衡翻下同}
其以十月為元封元年行所廵至博奉高蛇丘歷城梁父【博與蛇丘屬泰山郡博縣有泰山廟岱山在西北師古曰蛇音移歷城縣属濟南郡}
民田租逋賦皆貸除之無出今年筭賜天下民爵一級又以五載一廵狩用事泰山令諸侯各治邸泰山下【載子亥翻治直之翻下同}
天子既已封泰山無風雨而方士更言蓬萊諸神若將可得于是上欣然庶幾遇之復東至海上望焉【幾居衣翻復扶又翻}
上欲自浮海求蓬萊羣臣諫莫能止東方朔曰夫仙者得之自然不必躁求【躁則到翻}
若其有道不憂不得若其無道雖至蓬萊見仙人亦無益也臣願陛下第還宫静處以須之【處昌呂翻須待也}
仙人將自至上乃止會奉車霍子侯暴病一日死子侯去病子也上甚悼之乃遂去並海上【並步浪翻上時掌翻}
北至碣石廵自遼西歷北邊至九原五月乃至甘泉凡周行萬八千里云先是桑弘羊為治粟都尉領大農【原父曰大司農舊治粟内史耳弘羊為搜粟都尉也先悉薦翻}
盡管天下鹽鐵弘羊作平凖之法令遠方各以其物如異時商賈所轉販者【賈音古}
為賦而相灌輸置平凖于京師都受天下委輸【委於偽翻輸音戌}
大農諸官盡籠天下之貨物貴即賣之賤則買之欲使富商大賈無所牟大利【如淳曰牟取也}
而萬物不得騰踴至是天子廵狩郡縣所過賞賜用帛百餘萬匹錢金以巨萬計皆取足大農弘羊又請吏得入粟補官及罪人贖罪山東漕粟益歲六百萬石一歲之中太倉甘泉倉滿邊餘穀諸物均輸帛五百萬匹民不益賦而天下用饒於是弘羊賜爵左庶長黄金再百斤焉是時小旱上令官求雨卜式言曰縣官當食租衣税而已【師古曰衣於既翻}
今弘羊令吏坐市列肆販物求利烹弘羊天乃雨 秋有星孛于東井【晉天文志東井八星天之南門黄道所經又曰東井雍州分孛蒲内翻下同}
後十餘日有星孛于三台【天文志魁下六星兩兩而比曰三台}
望氣王朔言候獨見填星出如瓜食頃復入【填星土星也填讀曰鎮}
有司皆曰陛下建漢家封禪天其報德星云【師古曰德星即填星也言天以德星報於帝}
齊懷王閎薨無子國除【閎元狩六年受封}


資治通鑑卷二十
















































































































































