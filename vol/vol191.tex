










 


 
 


 

  
  
  
  
  





  
  
  
  
  
 
  

  

  
  
  



  

 
 

  
   




  

  
  


    資治通鑑卷一百九十一 宋 司馬光 撰

  胡三省 音註

  唐紀七【起閼逢涒灘六月盡柔兆閹茂八月凡二年有奇}


  高祖神堯大聖光孝皇帝下之上

  武德七年六月辛丑上幸仁智宫避暑【帝作仁智宫於宜州之宜君縣}
辛亥瀧州扶州獠作亂遣南尹州都督李光度等擊

  平之【瀧州永熙郡漢端溪縣地又瀧州信義縣武德元年分置懷德縣仍置南扶州南尹州鬰林郡漢廣鬰縣地後漢谷永為鬱林太守降烏許人十餘萬開置七縣即此地也瀧呂江翻獠魯皓翻}
 丙辰吐谷渾寇扶州【此扶州以生羌之地置注已見上吐從暾入聲谷音浴}
刺史蔣善合擊走之 壬戌慶州都督楊文幹反【慶州弘化郡漢北地馬嶺方渠縣地按宋白續通典慶州弘化郡東南三里有不窋城後魏大統十一年置朔州隋文帝改置合川鎮十六年置慶州以慶美取其嘉名今郡城名尉李城在白馬兩川交口亦曰不窋城附郭安化縣隋置合水縣武德改合川縣貞觀改弘化縣尋隨郡改縣名管下華池縣漢歸德縣地樂盤縣漢富平縣地馬領方渠則為通遠軍地矣史記正義曰漢郁郅縣今慶州弘化縣是}
初齊王元吉勸太子建成除秦王世民曰當為兄手刃之【為于偽翻下迭為復為同}
世民從上幸元吉第元吉伏護軍宇文寶於寢内欲刺世民【刺七亦翻}
建成性頗仁厚遽止之元吉愠曰為兄計耳於我何有建成擅募長安及四方驍勇二千餘人為東宫衛士【愠於問翻驍堅堯翻}
分屯左右長林號長林兵【東宫有左右長林門 考異曰舊傳云建成私召四方驍勇并募長安惡少年二千餘人畜為宫甲分屯左右長林號長林兵實録云元吉見秦王有大功每懷妬害言論醜惡譖害日甚每謂建成曰當為大哥手刃之建成性頗仁厚初止之元吉數言不已建成後亦許之元吉因令速發遂與建成各募壯士多匿罪人賞賜之圖行不軌其記室榮九思為詩以刺之曰丹青飾成慶玉帛擅專諸而弗悟也典籖裴宣儼因免官改事秦府謂泄其事又鴆之自殺斯人已後人皆振恐知其事莫有敢言後乃連結宫闈與建成俱通德妃尹氏以為内援舊傳又云厚賂中書令封德彞以為黨助由是高祖頗疎太宗而加愛元吉今但擇取其可信者書之}
又密使右虞候率可達志從燕王李藝發幽州突騎三百置宫東諸坊欲以補東宫長上【可達虜複姓騎奇寄翻燕因肩翻唐六典凡應宿衛官各從番第諸衛將軍中郎將郎將及諸衛率副率千牛備身備身左右太子千牛并上折衝果毅應宿衛者並一日上兩日下諸色長上若司階中候司戈並五日上十日下上時掌翻下上變同}
為人所告上召建成責之流可達志於巂州【巂音髓}
楊文幹嘗宿衛東宫建成與之親厚私使募壯士送長安上將幸仁智宫命建成居守世民元吉皆從【守手又翻下同從才用翻}
建成使元吉就圖世民曰安危之計决在今歲又使郎將爾朱煥校尉橋公山以甲遺文幹二人至豳州上變【豳州漢漆縣地漢末置新平郡東北有古豳亭後魏置豳州爾朱煥等至豳州言有急變豳州以聞遂得至仁智宫遺于季翻將即亮翻校戶數翻}
告太子使文幹舉兵使表裏相應 【考異曰統紀云建成遣郎將爾朱煥校尉橋公山齎甲以賜文幹令起兵煥等行至豳州懼罪告之劉餗小說云人妄告東宫今從實録}
又有寧州人杜鳳舉亦詣宫言狀上怒託他事手詔召建成令詣行在建成懼不敢赴太子舍人徐師謩勸之據城舉兵 【考異曰統記作師譽今從實録}
詹事主簿趙弘智勸之貶損軍服屏從者【屏必郢翻又卑正翻}
詣上謝罪建成乃詣仁智宫未至六十里悉留其官屬於毛鴻賓堡【後魏將毛鴻賓所築因以為名宋白曰三原縣有鴻賓柵後魏孝昌中蕭寶寅亂毛鴻賓立柵捍之其故城在縣北一十五里}
以十餘騎往見上【騎奇寄翻}
叩頭謝罪奮身自擲幾至於絶【幾居依翻}
上怒不解是夜置之幕下【鄭康成曰在上曰幕幕或在地展陳於上}
飼以麥飯【飼祥吏翻}
使殿中監陳福防守遣司農卿宇文頴馳召文幹【漢初置治粟内史景帝改曰大農武帝加司字梁置十二卿曰司農卿掌邦國倉儲委積之事}
頴至慶州以情告之文幹遂舉兵反上遣左武衛將軍錢九隴與靈州都督楊師道擊之甲子上召秦王世民謀之世民曰文幹豎子敢為狂逆計府僚已應擒戮若不爾止應遣一將討之耳【將即亮翻}
上曰不然文幹事連建成恐應之者衆汝宜自行還立汝為太子吾不能效隋文帝自誅其子當封建成為蜀王蜀兵脆弱【脆此芮翻}
他日苟能事汝汝宜全之不能事汝汝取之易耳【易以豉翻}
上以仁智宫在山中恐盜兵猝發夜帥宿衛南出山外【帥讀曰率}
行數十里東宫官屬繼至皆令三十人為隊分兵圍守之明日復還仁智宫 【考異曰實録云高祖之出山也建成憂憤卧于幕下天策兵曹杜淹請因亂襲之建成左右亦有斯請今上並拒而不納唐統紀云太宗之從内出夜經建成幕度建成侍衛左右唯有十人並來跪捧太宗足皆云今日之事一聽王旨若遣屏除今其時也太宗叱而止之既而還向府僚說其事衆僚文武並進曰文幹為儲君作逆天下共知假手宫臣正合天意太宗曰寡人始奉恩旨何忍旋踵即有所違卿與之言必無此理府僚又請終拒而不聽按是時高祖無誅建成意左右何敢輒殺之今不取}
世民既行元吉與妃嬪更迭為建成請封德彞復為之營解於外【為於偽翻}
上意遂變復遣建成還京師居守惟責以兄弟不睦歸罪於太子中允王珪左衛率韋挺【左右衛率掌東宫羽衛兵仗之政令正四品上率所律翻}
天策兵曹參軍杜淹並流於巂州【巂音髓}
挺沖之子也【韋沖事隋文帝招撫叛胡以赴長城之役又著績於南方}
初洛陽既平杜淹久不得調【調徒弔翻}
欲求事建成房玄齡以淹多狡數恐其教導建成益為世民不利乃言於世民引入天策府 突厥寇代州之武周城【武周城漢屬鴈門郡魏晉省後魏屬代郡隋廢入朔州雲内縣杜佑曰朔州馬邑郡治善陽縣有秦馬邑城武周塞厥九勿翻}
州兵擊破之 秋七月己巳苑君璋以突厥寇朔州摠管秦武通擊却之 楊文幹襲陷寧州【朱白曰寧州以安寧取稱九域志北至慶州一百二十里}
驅掠吏民出據百家堡【百家堡在慶州馬嶺縣}
秦王世民軍至寧州其黨皆潰癸酉文幹為其麾下所殺傳首京師獲宇文頴誅之 丁丑梁師都行臺白伏願來降【降戶江翻}
 戊寅突厥寇原州遣寧州刺史鹿大師救之又遣楊師道趨大木根山【大木根山在雲中河之西拓拔氏之先所居也}
庚辰突厥寇隴州遣護軍尉遲敬德擊之【尉紆勿翻}
 吐谷渾寇岷州辛巳吐谷渾党項寇松州【吐從暾入聲谷音浴}
 癸未突厥寇隂盤【隂盤縣漢屬安定晉屬京兆後魏置平凉郡隋唐屬涇州唐後改隂盤曰潘原}
 甲申扶州刺史蔣善合擊吐谷渾於松州赤磨鎮破之 己丑突厥吐利設與苑君璋寇并州 甲子車駕還京師 或說上曰【說輸芮翻}
突厥所以屢寇關中者以子女玉帛皆在長安故也【厥九勿翻}
若焚長安而不都則胡寇自息矣上以為然遣中書侍郎宇文士及踰南山至樊鄧行可居之地【踰長安南山出商州即至樊鄧行下孟翻}
將徙都之太子建成齊王元吉裴寂皆贊成其策蕭瑀等雖知其不可而不敢諫【瑀音禹}
秦王世民諫曰戎狄為患自古有之陛下以聖武龍興光宅中夏【夏戶雅翻}
精兵百萬所征無敵奈何以胡寇擾邊遽遷都以避之貽四海之羞為百世之笑乎彼霍去病漢廷一將猶志滅匈奴【霍去病曰匈奴未滅無以家為將即亮翻}
况臣忝備藩維願假數年之期請係頡利之頸致之闕下【頡奚結翻}
若其不効遷都未晚上曰善建成曰昔樊噲欲以十萬衆横行匈奴中【事見十二卷漢惠帝三年}
秦王之言得無似之世民曰形勢各異用兵不同樊噲小豎何足道乎不出十年必定漠北非虛言也【為太宗滅突厥張本}
上乃止建成與妃嬪因共譖世民曰突厥雖屢為邊患得賄即退秦王外託禦寇之名内欲總兵權成其簒奪之謀耳上校獵城南太子秦齊王皆從【從才用翻}
上命三子馳射角勝建成有胡馬肥壯而喜蹶【喜許記翻}
以授世民曰此馬甚駿能超數丈澗弟善騎【騎奇寄翻}
試乘之世民乘以逐鹿馬蹶世民躍立於數步之外馬起復乘之【復扶又翻}
如是者三顧謂宇文士及曰彼欲以此見殺死生有命庸何傷乎建成聞之因令妃嬪譖之於上【令力丁翻嬪毗賓翻}
曰秦王自言我有天命方為天下主豈有浪死上大怒先召建成元吉然後召世民入責之曰天子自有天命非智力可求汝求之一何急邪【邪音耶}
世民免冠頓首請下法司案驗【下遐嫁翻}
上怒不解會有司奏突厥入寇上乃改容勞勉世民命之冠帶與謀突厥【厥九勿翻勞力到翻冠古玩翻}
閏月己未詔世民元吉將兵出豳州以禦突厥【將即亮翻}
上餞之於蘭池【蘭池即秦始皇遇盜之地史記注曰地理志渭城縣有蘭池宫正義曰括地志蘭池陂即古之蘭池在咸陽縣界秦記曰始皇引渭水為池築為蓬瀛刻石為鯨長二百丈遇盜之處也}
上每有寇盜輒命世民討之事平之後猜嫌益甚 初隋末京兆韋仁夀為蜀郡司法書佐【按新書百官志諸州法曹司法參軍掌鞠獄麗法督盜賊知贓賄没入又有參軍事注云武德初改行書佐曰行參軍尋又改曰參軍事則書佐即參軍之任也}
所論囚至市猶西向為仁夀禮佛然後死【史言韋仁夀論刑人自以為不寃為於偽翻}
唐興爨弘達帥西南夷内附朝廷遣使撫之【帥讀曰率使疏吏翻}
類皆貪縱遠民患之有叛者仁夀時為巂州都督長史上聞其名命檢校南寧州都督寄治越巂【巂州越巂郡巂音髓長知兩翻}
使之歲一至其地慰撫之仁夀性寛厚有識度既受命將兵五百人至西洱河【將即亮翻洱仍吏翻}
周歷數千里蠻夷豪帥皆望風歸附來見仁夀仁夀承制置七州十五縣各以其豪帥為刺史縣令【按舊書地理志是年置西寧豫西平利南雲磨南寧七州志又有西平州亦是年置帥所類翻}
法令清肅蠻夷悦服將還【還從宣翻又音如字}
豪帥皆曰天子遣公都督南寧何為遽去仁夀以城池未立為辭蠻夷即相帥為仁夀築城立廨舍【帥讀曰率為於偽翻廨古隘翻}
旬日而就仁夀乃曰吾受詔但令巡撫不敢擅留蠻夷號泣送之【號戶高翻}
因各遣子弟入貢壬戌仁夀還朝【朝直遙翻}
上大悦命仁夀徙鎮南寧以兵戍之 苑君璋引突厥寇朔州【厥九勿翻}
 八月戊辰突厥寇原州己巳吐谷渾寇鄯州【鄯州西平郡秃髮氏所都之地鄯時戰翻}
 壬申突厥寇忻州丙子寇并州京師戒嚴戊寅寇綏州【綏州雕隂郡雕隂古縣漢屬上郡今延州以北横山之地也孫愐曰綏州春秋時為白狄所居秦為上郡後魏置上州又改為綏州取綏德縣為名}
刺史劉大俱擊却之是時頡利突利二可汗舉國入寇連營南上【頡奚結翻可從刋入聲汗音寒上時掌翻}
秦王世民引兵拒之會關中久雨糧運阻絶士卒疲於征役器械頓弊【頓讀曰鈍}
朝廷及軍中咸以為憂世民與虜遇於豳州勒兵將戰己卯可汗帥萬餘騎奄至城西陳於五隴阪【帥讀曰率騎奇寄翻下同陳讀曰陣下虜陳同阪音反}
將士震恐世民謂元吉曰今虜騎憑陵不可示之以怯當與之一戰汝能與我俱乎元吉懼曰虜形勢如此奈何輕出萬一失利悔可及乎世民曰汝不敢出吾當獨往汝留此觀之【世民獨出外以威示突厥内以服元吉之心}
世民乃帥騎馳詣虜陳吿之曰國家與可汗和親何為負約深入我地我秦王也可汗能鬭獨出與我鬭若以衆來我直以此百騎相當耳頡利不之測笑而不應【頡利素服秦王神武恐其以百騎挑戰而伏大兵四合以擊之故不敢應}
世民又前遣騎吿突利曰爾往與我盟有急相救今乃引兵相攻何無香火之情也【古者盟誓質諸天地山川鬼神歃血而已後世有對神立誓者有禮佛立誓者始有香火之事}
突利亦不應【秦王以此疑頡利之心突利恐因此為頡利所疑故亦不敢應}
世民又前將渡溝水頡利見世民輕出又聞香火之言疑突利與世民有謀乃遣止世民曰王不須度我無它意更欲與王申固盟約耳乃引兵稍却是後霖雨益甚世民謂諸將曰虜所恃者弓矢耳【將即亮翻}
今積雨彌時筋膠俱解弓不可用彼如飛鳥之折翼【折而設翻}
吾屋居火食刀槊犀利【犀堅也}
以逸制勞此而不乘將復何待【復扶又翻}
乃潛師夜出冒雨而進突厥大驚世民又遣說突利以利害【說輸芮翻}
突利悦聽命頡利欲戰突利不可乃遣突利與其夾畢特勒阿史那思摩來見世民請和親世民許之思摩頡利之從叔也【從才用翻}
突利因自託於世民請結為兄弟世民亦以恩意撫之與盟而去【為後突利先來降張本}
 庚寅岐州刺史柴紹破突厥於杜陽谷【杜陽山在岐州扶風縣孔穎達詩譜曰周原者岐山陽地屬杜陽地形險阻而原田肥美杜陽漢縣屬扶風有杜陽山山北有杜陽谷}
 壬申突厥阿史那思摩入見【見賢遍翻}
上引升御榻慰勞之【勞力到翻}
思摩貌類胡不類突厥故處羅疑其非阿史那種【厥九勿翻種章勇翻}
歷處羅頡利世常為夾畢特勒終不得典兵為設既入朝【處昌呂翻頡奚結翻朝直遙翻}
賜爵和順王 丁酉遣左僕射裴寂使於突厥【使疏吏翻}
 九月癸卯日南人姜子路反【日南郡德州後改驩州}
交州都督王志遠擊破之 癸卯突厥寇綏州都督劉大俱擊破之獲特勒三人 冬十月己巳突厥寇甘州辛未上校獵於鄠之南山【鄠縣屬京兆在南山下北至長安城六十里鄠音戶}
癸酉幸終南【酈道元曰武功縣太一山古文以為終南山在武功縣西南按鄠長安之西南山皆曰終南山終亦作中}
 吐谷渾及羌人寇疊州陷合川【疊州合川郡治疊川秦漢以來為諸羌保據後周武帝逐吐谷渾取羣山重疊之義置疊州合川縣後周置西疆郡隋廢為縣所治吐谷渾馬牧城唐武德三年移治交戍城吐從暾入聲谷音浴}
 丙子上幸樓觀謁老子祠【岐州盩厔縣有樓觀老子祠觀古玩翻}
癸未以太牢祭隋文帝陵 十一月丁卯上幸龍躍宫【京兆高陵縣西四十里有龍躍宫}
庚午還宫太子詹事裴矩權檢校侍中【太子詹事正三品掌東宫三寺十率府之政令唐改隋納言為侍中}


  八年春正月丙辰以夀州都督張鎮周為舒州都督【夀州淮南郡南朝曰豫州北朝曰揚州隋開皇九年曰夀州}
鎮周以舒州本其鄉里到州就故宅多市酒肴召親戚故人與之酣宴【酣戶甘翻}
散髮箕踞如為布衣時凡十日既而分贈金帛泣與之别曰今日張鎮周猶得與故人歡飲明日之後則舒州都督治百姓耳【治直之翻}
君民禮隔不得復為交遊【復扶又翻下復置同}
自是親戚故人犯法一無所縱境内肅然 丁巳遣右武衛將軍段德操徇夏州地【夏戶雅翻}
 吐谷渾寇疊州【吐從暾入聲谷音浴}
 是月突厥吐谷渾各請互市詔皆許之【厥九勿翻}
先是中國喪亂民乏耕牛至是資於戎狄雜畜被野【先悉薦翻喪息浪翻畜許救翻被皮義翻}
 夏四月乙亥党項寇渭州【党底朗翻}
 甲申上幸鄠縣校獵於甘谷【鄠縣有甘亭夏啟與有扈氏戰之地甘水出南山甘谷北流逕秦萯陽宫西又北逕甘亭西鄠音戶}
營太和宫於終南山【長安城南五十里有太和谷太和宫}
丙戌還宫 西突厥統葉護可汗遣使請昏【突厥大臣曰葉護西突厥可汗自葉護為可汗因號統葉護可汗可從刋入聲汗音寒使疏吏翻}
上謂裴矩曰西突厥道遠緩急不能相助今求昏何如對曰今北狄方彊為國家今日計且當遠交而近攻【用秦范睢之言}
臣謂宜許其昏以威頡利【頡奚結翻}
俟數年之後中國完實足抗北夷然後徐思其宜上從之 【考異曰新舊傳皆云封德彝之謀今從實録}
遣高平王道立至其國統葉護大喜道立上之從子也【從才用翻}
 初上以天下大定罷十二軍【見上卷上年}
既而突厥為寇不已辛亥復置十二軍以太常卿竇誕等為將軍簡練士馬議大舉擊突厥 甲寅涼州胡睦伽陀引突厥襲都督府【孫愐曰睦姓也伽求迦翻}
入子城長史劉君傑擊破之【長知兩翻}
 六月甲子上幸太和宫 丙子遣燕郡王李藝屯華亭縣【華亭縣隋大業初置屬安定郡義寧二年分置隴州至元和三年并入汧源縣燕因肩翻}
及彈箏峽【皆以守隴道箏音爭}
水部郎中姜行本斷石嶺道以備突厥【唐制水部郎中掌天下川瀆陂池之政令以導達溝洫堰决溝渠凡舟楫灌溉之利皆總而舉之凡諸曹郎中從五品上員外郎從六品上斷丁管翻厥九勿翻}
丙戌頡利可汗寇靈州【頡奚結翻可從刋入聲汗音寒}
丁亥以右衛大將軍張瑾為行軍總管以禦之以中書侍郎温彦博為長史先是上與突厥書用敵國禮【先悉薦翻}
秋七月甲辰上謂侍臣曰突厥貪婪無厭【婪廬南翻厭於鹽翻}
朕將征之自今勿復為書【復扶又翻}
皆用詔敕丙午車駕還宫 己酉突厥頡利可汗寇相州【相州疑當作柏州此時突厥兵不能至相州也}
 睦伽陀攻武興【蜀有武興鎮後魏置東益州梁為武興蕃王國西魏改曰興州順政郡此非睦伽陀所攻者也按晉書地理志永寧中張軌為涼州刺史鎮武威上表請合秦雍流移人於姑臧西北置武興郡睦伽陀所攻者即此武興故城}
 丙辰代州都督藺謩與突厥戰於新城不利【新城在馬邑南}
復命行軍總管張瑾屯石嶺李高遷趨大谷以禦之【大谷當作太谷舊曰陽邑隋開皇十八年更名太谷屬并州宋白曰并州太谷縣本漢陽邑縣今縣東十五里陽邑故城是也後魏大武景明二年復置陽邑縣隋開皇十八年改陽邑為太谷因縣西太谷為名復扶又翻趨七喻翻}
丁巳命秦王出屯蒲州以備突厥 【考異曰舊本紀八月六日突厥寇定州命皇太子往幽州秦王往并州以備突厥唐歷亦同今據實録七月秦王出蒲州八月無太子往幽州秦王往并州事}
 八月壬戌突厥踰石嶺寇并州癸亥寇靈州丁卯寇潞沁韓三州【沁原漢穀遠縣地後魏改名隋恭帝義寧元年置義寧郡武德元年置沁州又以潞州之襄垣黎城涉銅鞮鄉等縣置韓州沁七鴆翻}
 左武候大將軍安修仁擊睦伽陀於且渠川破之【且子余翻且渠川沮渠氏之墟也沮渠蒙遜據涼州川以是得名}
 詔安州大都督李靖出潞州道行軍總管任瓌屯太行以禦突厥【行戶剛翻厥九勿翻}
頡利可汗將兵十餘萬大掠朔州【頡奚結翻可從刋入聲汗音寒將即亮翻}
壬申并州道行軍總管張瑾與突厥戰於太谷全軍皆没瑾脱身奔李靖行軍長史温彦博為虜所執【長知兩翻}
虜以彦博職在機近【中書侍郎機近之官}
問以國家兵糧虛實彦博不對虜遷之隂山庚辰突厥寇靈武【考異曰實録統紀並云寇廣武按北邉地名無廣武下云靈州都督敗之蓋靈武字誤耳 今按舊唐志代州鴈門漢廣武縣或者寇廣武即大谷乘勝之兵歟史臣以漢古縣名稱鴈門為廣武耳}
甲申靈州都督任城王道宗擊破之【道宗所破者癸亥寇靈州之兵詳見通鑑舉要}
丙戌突厥寇綏州丁亥頡利可汗遣使請和而退【使疏吏翻}
 九月癸巳突厥没賀咄設陷并州一縣丙申代州都督藺謩擊破之 癸卯初令太府檢校諸州權量【檢校其輕重小大也唐制凡度以北方秬黍中者一黍之廣為分十分為寸十寸為尺一尺二寸為大尺十尺為丈凡量以秬黍中者容一千二百黍為籥二籥為合十合為升十升為斗三斗為大斗十斗為斛凡權衡以秬黍中者百黍之重為銖二十四銖為兩三兩為大兩十六兩為斤其量制公私又不用籥合内之分則有抄撮之細程大昌曰杜佑通典叙六朝賦税而論其總曰其度量三升當今一升秤則三兩當今一兩尺則尺二寸當今一尺註云當今謂即時即時者當佑之時也}
 丙午右領軍將軍王君廓破突厥於幽州俘斬二千餘人 突厥寇藺州【藺州當置於漢西河郡藺縣界而新舊志並不載}
冬十月壬申吐谷渾寇疊州遣扶州刺史蔣善合救之【吐從暾入聲谷音浴}
 戊寅突厥寇鄯州遣霍公柴紹救之【厥九勿翻突厥既能寇鄯州則上之藺州為蘭州未可知也鄯時戰翻}
 十一月辛卯朔上幸宜州 權檢校侍中裴矩罷判黄門侍郎 戊戌突厥寇彭州【武德元年以寧州彭原縣置彭州}
 庚子以天策司馬宇文士及權檢校侍中 辛丑徙蜀王元軌為吳王漢王元慶為陳王 癸卯加秦王世民中書令齊王元吉侍中 丙午吐谷渾寇岷州 戊申眉州山獠反【眉州通義郡本漢犍為郡南安縣地西魏置眉州因峨眉山而名獠魯皓翻}
 十二月辛酉上還至京師 庚辰上校獵於鳴犢泉辛巳還宫 以襄邑王神符檢校揚州大都督始自丹陽徙州府及居民於江北【由此廣陵專揚州之名}


  九年春正月己亥詔太常少卿祖孝孫等更定雅樂【少詩照翻更工衡翻}
 甲寅以左僕射裴寂為司空日遣員外郎一人更直其第 二月庚申以齊王元吉為司徒 丙子初令州縣祀社稷又令士民里閈相從立社【閈侯旰翻閭也里門謂之閈}
各申祈報【春夏祈而秋冬報}
用洽鄉黨之歡戊寅上祀社稷 丁亥突厥寇原州遣折威將軍楊毛擊之【折威將軍十二軍將軍之一也寧州道為折威軍}
 三月庚寅上幸昆明池壬辰還宫癸巳吐谷渾党項寇岷州 戊戌益州道行臺尚書

  郭行方擊眉州叛獠破之【獠魯皓翻}
 壬寅梁師都寇邊陷靜難鎮【難乃旦翻}
 丙午上幸周氏陂 辛亥突厥寇靈州【厥九勿翻}
 乙卯車駕還宫 癸丑南海公歐陽胤奉使在突厥帥其徒五十人謀掩襲可汗牙帳【使疏吏翻帥讀曰率可從刋入聲汗音寒 考異曰實紀云五千人按奉使安得五千人蓋十字誤作千字耳}
事泄突厥囚之丁巳突厥寇涼州都督長樂王幼良擊走之【樂音洛}


  戊午郭行方擊叛獠於洪雅二州大破之【歷考新舊志劒南有雅州無洪州或曰即眉州洪雅縣二州二字衍隋開皇十三年以西魏嘉州洪雅鎮置縣宋白曰因洪雅川為名}
俘男女五千口 夏四月丁卯突厥寇朔州庚午寇原州癸酉寇涇州 戊寅安州大都督李靖與突厥頡利可汗戰於靈州之硤石自旦至申突厥乃退 太史令傅奕上疏【唐太史令從五品下掌觀察天文稽定歷數凡日月星辰之變風雲氣色之異上時掌翻}
請除佛法曰佛在西域言妖路遠【妖於驕翻}
漢譯胡書恣其假託使不忠不孝削髮而揖君親遊手遊食易服以逃租賦偽啓三塗謬張六道【釋氏以地獄餓鬼畜生為三塗言人之為惡者必墮此也又添阿修羅天神地祇為六道}
恐愒愚夫【愒今人讀如喝呼葛翻}
詐欺庸品乃追懴既往之罪【懴楚鑒翻釋氏以自陳悔過為懴}
虛規將來之福布施萬錢希萬倍之報【施式䜴翻}
持齋一日冀百日之糧遂使愚迷妄求功德不憚科禁輕犯憲章有造為惡逆身墜刑網方乃獄中禮佛規免其罪且生死夀夭【夭於矯翻}
由於自然刑德威福關之人主貧富貴賤功業所招而愚僧矯詐皆云由佛竊人主之權擅造化之力其為害政良可悲矣降自羲農至於有漢皆無佛法君明臣忠祚長年久漢明帝始立胡神西域桑門自傳其法【事見四十五卷漢明帝永明八年}
西晉以上國有嚴科不許中國之人輒行髠髮之事洎于苻石羌胡亂華主庸臣佞政虐祚短梁武齊襄足為明鏡【謂梁武帝餓死臺城齊文襄為膳奴所弑也}
今天下僧尼數盈十萬翦刻繒綵裝束泥人競為厭魅【尼女夷翻繒慈陵翻厭於琰翻魅音媚}
迷惑萬姓請令匹配即成十萬餘戶產育男女十年長養一紀教訓可以足兵【長知兩翻}
四海免蠶食之殃百姓知威福所在則妖惑之風自革淳朴之化還興【妖於驕翻}
竊見齊朝章仇子佗表言僧尼徒衆糜損國家寺塔奢侈虛費金帛【沙門或曰桑門亦聲相近總謂之僧皆胡言也僧譯為和命衆桑門為息心比丘為乞俗人之信憑道法者男曰優婆塞女曰優婆夷其為沙門者初修十誡曰沙彌而終於二百五十則具足成大僧佛弟子收奉舍利建宫宇謂為塔亦胡言猶宗廟也故世稱塔廟}
為諸僧附會宰相對朝讒毁【言對朝廷而肆讒毁也朝直遙翻佗徒何翻}
諸尼依託妃主潛行謗讟子佗竟被囚縶刑於都市【被皮義翻}
周武平齊制封其墓臣雖不敏竊慕其蹤上詔百官議其事唯太僕卿張道源稱奕言合理【古有太僕正漢九卿有太僕梁十二卿有太僕卿唐太僕卿掌邦國廏牧車輿之政令}
蕭瑀曰佛聖人也而奕非之非聖人者無法【引孝經之言瑀音禹}
當治其罪【治直之翻}
奕曰人之大倫莫如君父佛以世嫡而叛其父以匹夫而抗天子【釋典謂佛以王太子出家故言以世嫡叛其父釋氏之法不拜君親故言以匹夫抗天子}
蕭瑀不生於空桑【昔有莘氏女採桑於伊川得嬰兒於空桑中言其母孕於伊水之濱夢神告之曰臼水出而東走母明而視之臼水出焉告其隣居而走顧望其邑咸為水矣其母化為空桑子在其中莘女取而獻之長有賢德教以為尹是為伊尹}
乃遵無父之教非孝者無親瑀之謂矣【亦以孝經之言難瑀也}
瑀不能對但合手曰地獄之設正為是人【釋氏之說謂為善者則升天堂為惡者墮地獄為于偽翻}
上亦惡沙門道士苟避征徭不守戒律皆如奕言又寺觀隣接㕓邸溷雜屠沽【惡烏路翻觀古玩翻下同}
辛巳下詔命有司沙汰天下僧尼道士女冠其精勤練行者遷居大寺觀給其衣食毋令闕乏【行下孟翻觀古喚翻}
庸猥麤穢者悉令罷道勒還鄉里京師留寺三所觀二所諸州各留一所餘皆罷之傅奕性謹密既職在占候杜絶交遊所奏災異悉焚其藁人無知者 癸未突厥寇西會州【武德二年以平凉郡之會寧鎮置西會州厥九勿翻}
 五月戊子䖍州胡成郎等殺長史叛歸梁師都【䖍州當作慶州長知兩翻}
都督劉旻追斬之 壬辰党項寇廓州【廓州澆河郡古邯川之地党底朗翻}
 戊戌突厥寇秦州 壬寅越州人盧南反殺刺史甯道明【此嶺南之越州後改亷州}
 丙午吐谷渾党項寇河州【吐從瞰入聲谷音浴}
 突厥寇蘭州【蘭州金城郡漢金城郡之枝陽縣地以臯蘭山名州}
 丙辰遣平道將軍柴紹將兵擊胡【岐州道為平道軍柴紹為將軍紹將即亮翻}
 六月丁巳太白經天【漢天文志曰太白經天天下革民更王永康註云謂出東入西出西入東也太白隂星出東當㐲東出西當伏西過午則經天晉灼云日陽也日出則星亡晝見午上為經天劉向五紀論曰太白少隂弱不得專行故以己未為界不得經天而行經天則晝見其占為兵喪為不臣為更王彊國弱小國彊}
秦王世民既與太子建成齊王元吉有隙以洛陽形勝之地恐一朝有變欲出保之乃以行臺工部尚書温大雅鎮洛陽遣秦府車騎將軍滎陽張亮將左右王保等千餘人之洛陽【騎奇寄翻亮將即亮翻之往也}
隂結納山東豪傑以俟變多出金帛恣其所用元吉告亮謀不軌下吏考驗【下遐嫁翻}
亮終無言乃釋之使還洛陽建成夜召世民飲酒而酖之世民暴心痛吐血數升【吐土故翻}
淮安王神通扶之還西宫【西宫蓋即弘義宫新書曰秦王居西宫之承乾殿}
上幸西宫問世民疾敕建成曰秦王素不能飲自今無得復夜飲【復扶又翻下可復不復事復能復同}
因謂世民曰首建大謀削平海内皆汝之功吾欲立汝為嗣汝固辭【事見前嗣祥吏翻}
且建成年長為嗣日久吾不忍奪也觀汝兄弟似不相容同處京邑必有紛競【長知兩翻處昌呂翻}
當遣汝還行臺居洛陽自陜以東皆主之【秦王時領陜東道大行臺陜失冉翻}
仍命汝建天子旌旗如漢梁孝王故事【梁孝王事見漢景帝紀}
世民涕泣辭以不欲遠離膝下【離力智翻}
上曰天下一家東西兩都道路甚邇【舊書地理志東都在西都之東八百五十里}
吾思汝即往毋煩悲也將行建成元吉相與謀曰秦王若至洛陽有土地甲兵不可復制【復扶又翻}
不如留之長安則一匹夫耳取之易矣乃密令數人上封事言秦王左右聞往洛陽無不喜躍觀其志趣恐不復來又遣近幸之臣以利害說上【易以豉翻上時掌翻說輸芮翻}
上意遂移事復中止建成元吉與後宫日夜譖訴世民於上【後宫即尹德妃張偼妤等}
上信之將罪世民陳叔達諫曰秦王有大功於天下不可黜也且性剛烈若加挫抑恐不勝憂憤或有不測之疾【勝音升}
陛下悔之何及上乃止元吉密請殺秦王上曰彼有定天下之功罪狀未著何以為辭元吉曰秦王初平東都顧望不還散錢帛以樹私恩又違敕命非反而何但應速殺何患無辭上不應秦府僚屬皆憂懼不知所出行臺考功郎中房玄齡謂比部郎中長孫無忌曰【唐制考功郎中屬吏部掌文武官吏之考課考課之法有四善二十七最比部屬刑部掌勾諸司百僚俸料公廨贓贖調斂徒役課程逋懸數物周知内外之經費而總勾之比音毗}
今嫌隙已成一旦禍機竊發豈惟府朝塗地【府朝猶言府廷也漢時郡僚謂本郡為郡朝亦此類朝直遙翻}
乃實社稷之憂莫若勸王行周公之事以安家國【謂周公誅管蔡也}
存亡之機間不容髮正在今日無忌曰吾懷此久矣不敢發口今吾子所言正合吾心謹當白之乃入言世民世民召玄齡謀之玄齡曰大王功蓋天地當承大業今日憂危乃天贊也願大王勿疑乃與府屬杜如晦共勸世民誅建成元吉建成元吉以秦府多驍將欲誘之使為己用【驍堅堯翻將即亮翻誘音酉}
密以金銀器一車贈左二副護軍尉遲敬德【時秦齊府各置右左六府護軍尉紆勿翻}
并以書招之曰願迃長者之眷以敦布衣之交【長知兩翻}
敬德辭曰敬德蓬戶甕牖之人遭隋末亂離久淪逆地罪不容誅秦王賜以更生之恩【事見一百八十八卷三年}
今又策名藩邸【左傳狐突曰策名委質貳乃辟也杜預註云名書於所臣之策}
唯當殺身以為報於殿下無功不敢謬當重賜若私交殿下乃是貳心狥利忘忠殿下亦何所用建成怒遂與之絶敬德以告世民世民曰公心如山嶽雖積金至斗【斗謂北斗唐人詩曰身後堆金柱北斗蓋時人常語也}
知公不移相遺但受何所嫌也【遺唯季翻}
且得以知其隂計豈非良策不然禍將及公既而元吉使壯士夜刺敬德敬德知之洞開重門【刺七亦翻重直龍翻}
安卧不動刺客屢至其庭終不敢入【畏其勇也}
元吉乃譖敬德於上下詔獄訊治【下遐嫁翻治直之翻}
將殺之世民固請得免又譖左一馬軍總管程知節出為康州刺史【武德元年以成州同谷縣置西康州}
知節謂世民曰大王股肱羽翼盡矣身何能久知節以死不去願早决計又以金帛誘右二護軍段志玄志玄不從【誘音酉}
建成謂元吉曰秦府智畧之士可憚者獨房玄齡杜如晦耳皆譖之於上而逐之世民腹心唯長孫無忌尚在府中與其舅雍州治中高士亷右候車騎將軍三水侯君集【長知兩翻右候車騎將軍以車騎將軍屬右候衛也三水縣漢屬安定郡隋唐屬邠州宋白曰三水縣以縣界存羅川谷三泉並流為名雍于用翻騎奇寄翻}
及尉遲敬德等【尉紆勿翻}
日夜勸世民誅建成元吉世民猶豫未决問於靈州大都督李靖靖辭問於行軍總管李世勣世勣辭世民由是重二人 【考異曰統紀云秦王懼不知所為李靖李勣數言大王以功高被疑靖等請申犬馬之力劉餗小說太宗將誅蕭牆之惡以主社稷謀於衛公靖靖辭謀於英公徐勣勣亦辭帝由是珍此二人二說未知誰得其實然劉說近厚有益風化故從之舊建成傳又云封德彝密勸太宗誅建成世民不從德彞更言於上曰秦王既有大功終不為太子之下若不立之願早為之所又說建成作亂曰夫為四海者不顧其親漢高乞羮此之謂矣按許敬宗傳云敬宗父善心及虞世南兄世基皆為宇文化及所殺封德彞時為内史舍人備見其事嘗謂人曰世基被誅世南匍匐而請代善心之死敬宗舞蹈以求生人以為口實敬宗銜之及為德彝立傳盛加其惡此亦近誣今不取}
會突厥郁射設將數萬騎屯河南入塞圍烏城【烏城蓋在鹽州五原縣烏鹽池或曰在朔方烏水上杜佑曰武威郡南二里在烏城守捉將即亮翻騎奇寄翻厥九勿翻}
建成薦元吉代世民督諸軍北征上從之命元吉督右武衛大將軍李藝天紀將軍張瑾等救烏城【關内十二軍涇州道曰天紀軍置將軍一人}
元吉請尉遲敬德程知節段志玄及秦府右三統軍秦叔寶等與之偕行簡閲秦王帳下精鋭之士以益元吉軍率更丞王晊密告世民曰【唐志太子率更寺令一人從四品上丞二人從七品上掌宗族次序禮樂刑罰及漏刻之政令更工衡翻晊之日翻}
太子語齊王今汝得秦王驍將精兵擁數萬之衆吾與秦王餞汝於昆明池使壯士拉殺之於幕下奏云暴卒主上宜無不信【語牛倨翻拉盧合翻驍堅堯翻將即亮翻 考異曰舊傳以為建成實有此言而晊告之按建成前酖秦王高祖已知之今若明使壯士拉殺而欺云暴卒高祖豈有肯信之理此說殆同兒戲今但云晊告建成等則事之虛實皆未可知所謂疑以傳疑也}
吾當使人進說令授吾國事敬德等既入汝手宜悉坑之孰敢不服世民以晊言告長孫無忌等無忌等勸世民先事圖之【先悉薦翻}
世民歎曰骨肉相殘古今大惡吾誠知禍在朝夕欲俟其發然後以義討之不亦可乎敬德曰人情誰不愛其死今衆人以死奉王乃天授也禍機垂發而王猶晏然不以為憂大王縱自輕如宗廟社稷何大王不用敬德之言敬德將竄身草澤不能留居大王左右交手受戮也無忌曰不從敬德之言事今敗矣敬德等必不為王有無忌亦當相隨而去不能復事大王矣【敬德無忌詭言逃去以激世民使之速發復扶又翻下同}
世民曰吾所言亦未可全棄公更圖之敬德曰王今處事有疑非智也臨難不决非勇也【處昌呂翻}
且大王素所畜養勇士八百餘人【畜吁玉翻}
在外者今已入宫擐甲執兵【擐音宦}
事勢已成大王安得已乎世民訪之府僚皆曰齊王凶戾終不肯事其兄比聞護軍薛實嘗謂齊王曰【比毗至翻此齊府護軍也}
大王之名合之成唐字大王終主唐祀【合音閤}
齊王喜曰但除秦王取東宫如反掌耳彼與太子謀亂未成已有取太子之心亂心無厭【厭於鹽翻}
何所不為若使二人得志恐天下非復唐有【復音扶又翻下聽復同又並音如字}
以大王之賢取二人如拾地芥耳奈何徇匹夫之節忘社稷之計乎世民猶未决衆曰大王以舜為何如人曰聖人也衆曰使舜浚井不出則為井中之泥塗廪不下則為廪上之灰安能澤被天下法施後世乎是以小杖則受大杖則走蓋所存者大故也【瞽䏂使舜浚井既入從而揜之舜穿井為匿空旁出使塗廩捐堦瞽䏂焚廩舜以兩笠自扞而下家語孔子曰舜事瞽䏂小杖則受大杖則走被皮義翻}
世民命卜之幕僚張公謹自外來取龜投地【說苑曰靈龜五色似玉似金背隂向陽上高象天下平法地易號為龜}
曰卜以决疑今事在不疑尚何卜乎卜而不吉庸得已乎於是定計【考異曰唐歷云布卦未畢張公謹適自外至諫曰夫事不可疑而疑者其禍立至今假使卜之不吉其可已乎遂折蓍秦王曰善今從舊唐書}
世民令無忌密召房玄齡等曰敕旨不聽復事王今若私謁必坐死不敢奉教【房玄齡之言亦以激發世民}
世民怒謂敬德曰玄齡如晦豈叛我邪【邪音耶}
取所佩刀授敬德曰公往觀之若無來心可斷其首以來【斷丁管翻}
敬德往與無忌共諭之曰王已决計公宜速入共謀之吾屬四人不可羣行道中乃令玄齡如晦著道士服【著陟畧同}
與無忌俱入敬德自它道亦至己未太白復經天傅奕密奏太白見秦分【見賢遍翻分扶問翻}
秦王當有天下上以其狀授世民於是世民密奏建成元吉淫亂後宫且曰臣於兄弟無絲毫負今欲殺臣似為世充建德報讐臣今枉死永違君親魂歸地下實恥見諸賊上省之愕然【為于偽翻省悉景翻}
報曰明當鞫問汝宜早參【明謂明日也參謂朝參}
庚申世民帥長孫無忌等入伏兵於玄武門【玄武門宫城北門帥讀曰率長知兩翻}
張婕妤竊知世民表意馳語建成【婕妤音接予語牛倨翻}
建成召元吉謀之元吉曰宜勒宫府兵託疾不朝以觀形勢【朝直遙翻}
建成曰兵備已嚴當與弟入參自問消息乃俱入趣玄武門【趣七喻翻}
上時已召裴寂蕭瑀陳叔達等欲按其事【瑀音禹}
建成元吉至臨湖殿覺變即跋馬東歸宫府【跋蒲掇翻跋馬者揺竦馬銜偏促一轡又以兩足揺鼓馬腹使之迴走}
世民從而呼之元吉張弓射世民再三不彀【控弦不開所以不至於彀蓋倉皇失措也射而亦翻下同}
世民射建成殺之尉遲敬德將七十騎繼至【將即亮翻騎奇寄翻下同}
左右射元吉墜馬世民馬逸入林下為木枝所絓【絓胡卦翻}
墜不能起元吉遽至奪弓將扼之敬德躍馬叱之元吉步欲趣武德殿敬德追射殺之翊衛車騎將軍馮翊馮立【太子左右衛率府所領亦有親勲翊三衛府}
聞建成死歎曰豈有生受其恩而死逃其難乎【難乃旦翻}
乃與副護軍薛萬徹屈咥直府左車騎萬年謝叔方【屈咥直即驅咥直也屬帳内府咥徒結翻又丑栗翻萬年赤縣本隋大興縣武德元年更名}
帥東宫齊府精兵二千馳趣玄武門【帥讀曰率趣七喻翻}
張公謹多力獨閉關以拒之不得入雲麾將軍敬君弘掌宿衛屯兵玄武門【雲麾將軍梁百二十五號將軍之一也唐為武散階從三品上}
挺身出戰所親止之曰事未可知且徐觀變俟兵集成列而戰未晚也君弘不從與中郎將呂世衡大呼而進皆死之【唐諸衛中郎將皆正四品下呼火故翻}
君弘顯雋之曾孫也【敬顯雋仕北齊官至尚書右僕射}
守門兵與萬徹等力戰良久萬徹鼓譟欲攻秦府將士大懼【將即亮翻}
尉遲敬德持建成元吉首示之【尉紆勿翻}
宫府兵遂潰萬徹與數十騎亡入終南山馮立既殺敬君弘謂其徒曰亦足以少報太子矣【少詩沼翻}
遂解兵逃於野上方泛舟海池【閣本太極宫圖太極宫中凡有三海池東海池在玄武門内之東近凝雲閣北海池在玄武門内之西又南有南海池近咸池殿}
世民使尉遲敬德入宿衛敬德擐甲持矛直至上所上大驚問曰今日亂者誰邪【邪音耶}
卿來此何為對曰秦王以太子齊王作亂舉兵誅之恐驚動陛下遣臣宿衛上謂裴寂等曰不圖今日乃見此事當如之何蕭瑀陳叔達曰建成元吉本不預義謀又無功於天下疾秦王功高望重共為姦謀今秦王已討而誅之秦王功蓋宇宙率土歸心陛下若處以元良【太子謂之元良瑀音禹處昌呂翻下處分處决同}
委之國事無復事矣【復扶又翻}
上曰善此吾之夙心也時宿衛及秦府兵與二宫左右戰猶未已敬德請降手敕令諸軍並受秦王處分【分扶問翻}
上從之天策府司馬宇文士及自東上閤門出宣敕【閣本太極宫圖太極殿有東上閤門西上閤門}
衆然後定上又使黄門侍郎裴矩至東宫曉諭諸將卒皆罷散【將即亮翻下同}
上乃召世民撫之曰近日以來幾有投杼之惑【投杼事見三卷周赧王七年幾居希翻}
世民跪而吮上乳號慟久之【吮徂兖翻號戶高翻}
建成子安陸王承道河東王承德武安王承訓汝南王承明鉅鹿王承義元吉子梁郡王承業漁陽王承鸞普安王承奬江夏王承裕義陽王承度皆坐誅仍絶屬籍初建成許元吉以正位之後立為太弟故元吉為之盡死【為于偽翻}
諸將欲盡誅建成元吉左右百餘人籍没其家尉遲敬德固爭曰罪在二凶既伏其誅若及支黨非所以求安也乃止是日下詔赦天下凶逆之罪止於建成元吉自餘黨與一無所問其僧尼道士女冠並宜依舊【是年四月命有司沙汰僧尼道士女冠}
國家庶事皆取秦王處分【處昌呂翻}
辛酉馮立謝叔方皆自出薛萬徹亡匿世民屢使諭之乃出世民曰此皆忠於所事義士也釋之癸亥立世民為皇太子又詔自今軍國庶事無大小悉委太子處决然後聞奏

  臣光曰立嫡以長【長知兩翻}
禮之正也然高祖所以有天下皆太宗之功隱太子以庸劣居其右地嫌勢逼必不相容曏使高祖有文王之明隱太子有泰伯之賢太宗有子臧之節【文王舍伯邑考而立武王泰伯讓國於弟王季歷子臧辭曹國而不受}
則亂何自而生矣既不能然太宗始欲俟其先發然後應之如此則事非獲已猶為愈也既而為羣下所迫遂至蹀血禁門【如淳曰殺人流血滂沱為蹀血師古曰蹀謂履涉之也蹀徒頰翻}
推刃同氣【推吐雷翻}
貽譏千古惜哉夫創業垂統之君子孫之所儀刑也【夫音扶}
彼中明肅代之傳繼得非有所指擬以為口實乎【明皇不稱廟號而稱帝號者温公避本朝諱耳中宗肅宗之季玄宗代宗並以兵清内難而後繼大統}


  戊辰以宇文士及為太子詹事長孫無忌杜如晦為左庶子高士亷房玄齡為右庶子尉遲敬德為左衛率程知節為右衛率虞世南為中舍人禇亮為舍人【尉紆勿翻率所律翻東宫門下坊左庶子二人正四品上掌侍從贊相駮正啟奏皇太子出則版奏外辨中嚴入則解嚴凡令書下則畫諾覆審留所畫以為案更寫印署注令諾送詹事府典書坊右庶子二人正四品下中舍人正五品上舍人正六品上舍人掌行令書令旨及表啟之事太子通表如人臣之禮宫臣上太子大事以牋小事以啟其封題皆曰上右春坊通事舍人開封以進其事可施行者皆下於坊舍人開庶子參詳之然後進不可者則否蓋門下坊猶上臺之門下省典書坊猶上臺之中書省唐初仍隋制也龍朔改門下坊為左春坊典書坊為右春坊}
姚思亷為洗馬【洗悉薦翻下同}
悉以齊王國司金帛什器賜敬德【唐制親王國有國司置國尉國丞掌判國司勾稽監印事}
初洗馬魏徵常勸太子建成早除秦王及建成敗世民召徵謂曰汝何為離間我兄弟衆為之危懼【問古莧翻為於偽翻}
徵舉止自若對曰先太子早從徵言必無今日之禍世民素重其才改容禮之引為詹事主簿【詹事主簿從七品上掌印檢勾稽府事}
亦召王珪韋挺於巂州【去年六月王珪等流巂州巂音髓}
皆以為諫議大夫世民命縱禁苑鷹犬罷四方貢獻聽百官各陳治道【治直吏翻}
政令簡肅中外大悦以屈突通為陜東道行臺左僕射鎮洛陽【陜失冉翻}
益州行臺僕射竇軌與行臺尚書韋雲起郭行方不協雲起弟慶儉及宗族多事太子建成建成死軌誣雲起與建成同反收斬之行方懼逃奔京師軌追之不及 吐谷渾寇岷州【吐從暾入聲谷音浴}
 突厥寇隴州辛未寇渭州遣右衛大將軍柴紹擊之【厥九勿翻左右衛大將軍掌統領宫庭警衛之法}
 廢益州大行臺置大都督府 壬申上以手詔賜裴寂等曰朕當加尊號為太上皇 辛巳幽州大都督廬州王瑗反【瑗于眷翻}
右領軍將軍王君廓殺之傳首初上以瑗懦怯非將帥才【懦乃卧翻乂奴亂翻將即亮翻帥所類翻}
使君廓佐之君廓故羣盜勇悍險詐【悍戶旰翻}
瑗推心倚仗之許為婚姻太子建成謀害秦王密與瑗相結建成死詔遣通事舍人崔敦禮馳驛召瑗【通事舍人秦謁者之官也晉置舍人通事各一人隸中書東晉曰通事舍人唐從六品上掌朝見引納及辭謝者於殿庭凡近臣入侍文武就列引以進退凡四方通表蠻夷納貢皆受而進之}
瑗心不自安謀於君廓君廓欲取瑗以為功乃說曰【說輸芮翻下涉說同}
大王若入必無全理今擁兵數萬奈何受單使之召自投罔罟乎【使疏吏翻下同}
因相與泣瑗曰我今以命託公舉事决矣乃劫敦禮問以京師機事敦禮不屈瑗囚之發驛徵兵且召燕州刺史王詵赴薊與之計事【隋于營州之境汝羅故城置遼西郡武德元年曰燕州六年自營州遷於幽州城中又於懷戎置北燕州武德六年李藝自幽州入朝王詵為長史實掌州事幽州之人素信服之瑗欲反故召之與計事燕因肩翻詵疎臻翻}
兵曹參軍王利涉說瑗曰王君廓反覆不可委以機柄宜早除去以王詵代之【去羌呂翻}
瑗不能决君廓知之往見詵詵方沐握髮而出君廓手斬之持其首告衆曰李瑗與王詵同反囚執敕使擅自徵兵今詵已誅獨有李瑗無能為也汝寧隨瑗族滅乎欲從我以取富貴乎衆皆曰願從公討賊君廓乃帥其麾下千餘人踰西城而入瑗不之覺君廓入獄出敦禮瑗始知之遽帥左右數百人被甲而出【帥讀曰率被皮義翻}
遇君廓於門外君廓謂瑗衆曰李瑗為逆汝何為隨之入湯火乎衆皆弃兵而潰唯瑗獨存罵君廓曰小人賣我行自及矣遂執瑗縊之【縊於賜翻又於計翻}
壬午以王君廓為左領軍大將軍兼幽州都督以瑗家口賜之敦禮仲方之孫也【崔仲方仕周獻平齊之策及隋獻平陳之策孝芬之孫也}
 乙酉罷天策府【置天策府見一百八十九卷四年}
 秋七月己丑柴紹破突厥於秦州斬特勒一人士卒首千餘級【厥九勿翻}
 以秦府護軍秦叔寶為左衛大將軍又以程知節為右武衛大將軍尉遲敬德為右武候大將軍【尉紆勿翻 考異曰唐歷三人除官皆在癸巳今從實録}
 壬辰以高士亷為侍中房玄齡為中書令蕭瑀為左僕射長孫無忌為吏部尚書杜如晦為兵部尚書癸巳以宇文士及為中書令封德彝為右僕射又以前天策府兵曹參軍杜淹為御史大夫中書舍人顔師古劉林甫為中書侍郎左衛副率侯君集為左衛將軍左虞候段志玄為驍衛將軍副護軍薛萬徹為右領軍將軍右内副率張公謹為右武候將軍【左虞候即東宫左虞候率也按唐書驍衛之上當有左字隋文帝置左右内率領東宫千牛備身侍奉之事副率為之貳瑀音禹長知兩翻率所律翻驍堅堯翻}
右監門率長孫安業為右監門將軍【漢魏置城門校尉唐置左右監門衛大將軍將軍掌宫禁門籍之法凡京司應入宫殿門皆有籍左將軍判入右將軍判出監古銜翻}
右内副率李客師為領左右軍將軍【領字當在左右之下左右二字亦當去其一但未知當去何字耳唐志隋置左右領軍府大業三年改左右屯衛唐因屯衛名改為左右威衛又採前代領軍名别置左右領軍衛職掌如左右衛又按新志武德五年改左右備身府為左右府或者李客師為領左右將軍左右之下亦當去軍字顯慶五年改左右府為千牛府}
安業無忌之兄客師靖之弟也 太子建成齊王元吉之黨散亡在民間雖更赦令【更工衡翻}
猶不自安徼幸者爭告捕以邀賞【徼堅堯翻}
諫議大夫王珪以啟太子丙子太子下令六月四日已前事連東宫及齊王十七日前連李瑗者並不得相告言違者反坐【瑗于眷翻反坐者反以所告罪人之罪坐之 考異曰太宗實録六月丙申唐歷脱七月而在壬辰下按六月無丙申丙申七月十日也今從唐歷}
丁酉遣諫議大夫魏徵宣慰山東聽以便宜從事徵至磁州【武德元年以相州之滏陽臨水成安置磁州以其地產磁石名州舊志磁州在京師東北一千四百八十五里磁疾之翻}
遇州縣錮送前太子千牛李志安齊王護軍李思行詣京師【械鎖而送之謂之錮送}
徵曰吾受命之日前宫齊府左右皆赦不問今復送思行等【復扶又翻}
則誰不自疑雖遣使者人誰信之【使疏吏翻}
吾不可以顧身嫌不為國慮且既蒙國士之遇敢不以國士報之乎遂皆解縱之太子聞之甚喜右衛率府鎧曹參軍唐臨出為萬泉丞【東宫十率府皆有倉兵鎧三曹參軍從八品武德元年分蒲州之稷山安邑龍門猗氏汾隂置萬泉縣屬泰州後屬絳州鎧可亥翻率所律翻}
縣有繫囚十許人會春雨臨縱之使歸耕種皆如期而返臨令則之弟子也【唐令則事隋太子勇勇廢被誅}
 八月丙辰突厥遣使請和【厥九勿翻使疏吏翻下同}
 壬戌吐谷渾遣使請和【吐從暾入聲谷音浴}
 癸亥制傳位於太子太子固辭不許甲子太宗即皇帝位於東宫顯德殿赦天下關内及蒲芮虞泰陜鼎六州免二年租調自餘給復一年【陜失冉翻調徒弔翻復方目翻}
 詔以宫女衆多幽閟可愍【閟兵媚翻}
宜簡出之各歸親戚任其適人 初稽胡酋長劉仚成帥衆降梁師都【事見一百八十九卷四年酋慈由翻長知兩翻仚許延翻帥讀曰率降戶江翻下同}
師都信讒殺之由是所部猜懼多來降者【降戶江翻}
師都浸衰弱乃朝於突厥為之畫策【朝直遙翻為於偽翻}
勸令入寇於是頡利突利二可汗合兵十餘萬寇涇州【頡奚結翻可從刋入聲汗音寒}
進至武功京師戒嚴 丙子立妃長孫氏為皇后【長知兩翻}
后少好讀書造次必循禮法【少詩沼翻好呼到翻造七到翻}
上為秦王與太子建成齊王元吉有隙后奉事高祖承順妃嬪【嬪毗賓翻}
彌縫其闕甚有内助及正位中宫務存節儉服御取給而已上深重之嘗與之議賞罰后辭曰牝雞之晨唯家之索【書牧誓引古人之言索蘇各翻盡也}
妾婦人安敢豫聞政事固問之終不對己卯突厥進寇高陵【厥九勿翻高陵縣漢屬馮翊唐屬京兆在長安東北七十里}


  辛巳涇州道行軍總管尉遲敬德與突厥戰於涇陽【涇陽縣屬京兆在長安北七十里杜佑曰京兆涇陽縣乃秦封涇陽君之地後漢及晉池陽之地漢涇陽縣在今平凉郡界涇陽故城是尉紆勿翻}
大破之獲其俟斤阿史德烏没啜【突厥官二十八等俟斤在吐屯之下阿史德别是一姓俟渠機翻}
斬首千餘級癸未頡利可汗進至渭水便橋之北【自長安出咸陽過渭水便橋}
遣其腹心執失思力入見以觀虚實【見賢遍翻}
思力盛稱頡利與突利二可汗將兵百萬今至矣【頡奚結翻可從刋入聲汗音寒將即亮翻}
上讓之曰吾與汝可汗面結和親贈遺金帛前後無算【言不可算計其數也遺於季翻}
汝可汗自負盟約引兵深入於我無愧汝雖戎狄亦有人心何得全忘大恩自誇強盛我今先斬汝矣思力懼而請命【請貸其死命也}
蕭瑀封德彞請禮遣之上曰我今遣還虜謂我畏之愈肆憑陵【瑀音禹還從宣翻又音如字}
乃囚思力於門下省上自出玄武門與高士亷房玄齡等六騎徑詣渭水上【騎奇寄翻下同}
與頡利隔水而語責以負約突厥大驚皆下馬羅拜【厥九勿翻}
俄而諸軍繼至旌甲蔽野頡利見執失思力不返而上挺身輕出軍容甚盛有懼色上麾諸軍使却而布陳【陳讀曰陣}
獨留與頡利語蕭瑀以上輕敵叩馬固諫上曰吾籌之已熟非卿所知突厥所以敢傾國而來直抵郊甸者以我國内有難【謂方有殺建成元吉之難難乃旦翻}
朕新即位謂我不能抗禦故也我若示之以弱閉門拒守虜必放兵大掠不可復制【復扶又翻}
故朕輕騎獨出示若輕之又震曜軍容使之必戰出虜不意使之失圖虜入我地既深必有懼心故與戰則克與和則固矣制服突厥在此一舉卿第觀之是日頡利來請和詔許之上即日還宫乙酉又幸城西斬白馬與頡利盟於便橋之上突厥引兵退【頡奚結翻厥九勿翻 考異曰劉餗小說武德末年突厥至渭水橋控弦四十萬太宗初親庶政驛召衛公問策時發諸州軍未到長安居人勝兵不過數萬胡人精騎騰突挑戰日數合帝怒欲擊之靖請傾府庫賂以求和潛軍邀其歸路帝從其言胡兵遂退於是據險邀之虜弃老弱而遁獲馬數萬匹金帛一無遺焉今據實録紀傳結盟而退未嘗掩襲小說所載為誤}
蕭瑀請於上曰突厥未和之時諸將爭請戰陛下不許【瑀音禹厥九勿翻將即亮翻}
臣等亦以為疑既而虜自退其策安在上曰吾觀突厥之衆雖多而不整君臣之志唯賄是求當其請和之時可汗獨在水西【謂渭水之西可從刋入聲汗音寒}
達官皆來謁我【突厥言達官猶中國言顯官也}
我若醉而縛之因襲擊其衆勢如拉朽【拉盧合翻}
又命長孫無忌李靖伏兵於幽州以待之【幽州當作豳州自渭北北歸歸路正經豳州此史書傳寫誤耳開元十三年以豳字類幽改曰邠州則當時亦病此矣}
虜若奔歸伏兵邀其前大軍躡其後覆之如反掌耳所以不戰者吾即位日淺國家未安百姓未富且當靜以撫之一與虜戰所損甚多虜結怨既深懼而修備則吾未可以得志矣故卷甲韜戈啗以金帛【卷讀曰惓啗徒濫翻}
彼既得所欲理當自退志意驕墯不復設備【復扶又翻}
然後養威伺舋一舉可滅也【舋許覲翻}
將欲取之必固與之【老子曰將欲奪之必固與之}
此之謂矣卿知之乎瑀再拜曰非所及也【言非己之智慮所能及也}


  資治通鑑卷一百九十一


    


 


 



 

 
  







 


  
  
 
 
 


  

 















	
	









































 
  



















 





 












  
  
  

 





