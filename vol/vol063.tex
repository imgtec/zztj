<!DOCTYPE html PUBLIC "-//W3C//DTD XHTML 1.0 Transitional//EN" "http://www.w3.org/TR/xhtml1/DTD/xhtml1-transitional.dtd">
<html xmlns="http://www.w3.org/1999/xhtml">
<head>
<meta http-equiv="Content-Type" content="text/html; charset=utf-8" />
<meta http-equiv="X-UA-Compatible" content="IE=Edge,chrome=1">
<title>資治通鑒_64-資治通鑑卷六十三_64-資治通鑑卷六十三</title>
<meta name="Keywords" content="資治通鑒_64-資治通鑑卷六十三_64-資治通鑑卷六十三">
<meta name="Description" content="資治通鑒_64-資治通鑑卷六十三_64-資治通鑑卷六十三">
<meta http-equiv="Cache-Control" content="no-transform" />
<meta http-equiv="Cache-Control" content="no-siteapp" />
<link href="/img/style.css" rel="stylesheet" type="text/css" />
<script src="/img/m.js?2020"></script> 
</head>
<body>
 <div class="ClassNavi">
<a  href="/24shi/">二十四史</a> | <a href="/SiKuQuanShu/">四库全书</a> | <a href="http://www.guoxuedashi.com/gjtsjc/"><font  color="#FF0000">古今图书集成</font></a> | <a href="/renwu/">历史人物</a> | <a href="/ShuoWenJieZi/"><font  color="#FF0000">说文解字</a></font> | <a href="/chengyu/">成语词典</a> | <a  target="_blank"  href="http://www.guoxuedashi.com/jgwhj/"><font  color="#FF0000">甲骨文合集</font></a> | <a href="/yzjwjc/"><font  color="#FF0000">殷周金文集成</font></a> | <a href="/xiangxingzi/"><font color="#0000FF">象形字典</font></a> | <a href="/13jing/"><font  color="#FF0000">十三经索引</font></a> | <a href="/zixing/"><font  color="#FF0000">字体转换器</font></a> | <a href="/zidian/xz/"><font color="#0000FF">篆书识别</font></a> | <a href="/jinfanyi/">近义反义词</a> | <a href="/duilian/">对联大全</a> | <a href="/jiapu/"><font  color="#0000FF">家谱族谱查询</font></a> | <a href="http://www.guoxuemi.com/hafo/" target="_blank" ><font color="#FF0000">哈佛古籍</font></a> 
</div>

 <!-- 头部导航开始 -->
<div class="w1180 head clearfix">
  <div class="head_logo l"><a title="国学大师官网" href="http://www.guoxuedashi.com" target="_blank"></a></div>
  <div class="head_sr l">
  <div id="head1">
  
  <a href="http://www.guoxuedashi.com/zidian/bujian/" target="_blank" ><img src="http://www.guoxuedashi.com/img/top1.gif" width="88" height="60" border="0" title="部件查字,支持20万汉字"></a>


<a href="http://www.guoxuedashi.com/help/yingpan.php" target="_blank"><img src="http://www.guoxuedashi.com/img/top230.gif" width="600" height="62" border="0" ></a>


  </div>
  <div id="head3"><a href="javascript:" onClick="javascript:window.external.AddFavorite(window.location.href,document.title);">添加收藏</a>
  <br><a href="/help/setie.php">搜索引擎</a>
  <br><a href="/help/zanzhu.php">赞助本站</a></div>
  <div id="head2">
 <a href="http://www.guoxuemi.com/" target="_blank"><img src="http://www.guoxuedashi.com/img/guoxuemi.gif" width="95" height="62" border="0" style="margin-left:2px;" title="国学迷"></a>
  

  </div>
</div>
  <div class="clear"></div>
  <div class="head_nav">
  <p><a href="/">首页</a> | <a href="/ShuKu/">国学书库</a> | <a href="/guji/">影印古籍</a> | <a href="/shici/">诗词宝典</a> | <a   href="/SiKuQuanShu/gxjx.php">精选</a> <b>|</b> <a href="/zidian/">汉语字典</a> | <a href="/hydcd/">汉语词典</a> | <a href="http://www.guoxuedashi.com/zidian/bujian/"><font  color="#CC0066">部件查字</font></a> | <a href="http://www.sfds.cn/"><font  color="#CC0066">书法大师</font></a> | <a href="/jgwhj/">甲骨文</a> <b>|</b> <a href="/b/4/"><font  color="#CC0066">解密</font></a> | <a href="/renwu/">历史人物</a> | <a href="/diangu/">历史典故</a> | <a href="/xingshi/">姓氏</a> | <a href="/minzu/">民族</a> <b>|</b> <a href="/mz/"><font  color="#CC0066">世界名著</font></a> | <a href="/download/">软件下载</a>
</p>
<p><a href="/b/"><font  color="#CC0066">历史</font></a> | <a href="http://skqs.guoxuedashi.com/" target="_blank">四库全书</a> |  <a href="http://www.guoxuedashi.com/search/" target="_blank"><font  color="#CC0066">全文检索</font></a> | <a href="http://www.guoxuedashi.com/shumu/">古籍书目</a> | <a   href="/24shi/">正史</a> <b>|</b> <a href="/chengyu/">成语词典</a> | <a href="/kangxi/" title="康熙字典">康熙字典</a> | <a href="/ShuoWenJieZi/">说文解字</a> | <a href="/zixing/yanbian/">字形演变</a> | <a href="/yzjwjc/">金 文</a> <b>|</b>  <a href="/shijian/nian-hao/">年号</a> | <a href="/diming/">历史地名</a> | <a href="/shijian/">历史事件</a> | <a href="/guanzhi/">官职</a> | <a href="/lishi/">知识</a> <b>|</b> <a href="/zhongyi/">中医中药</a> | <a href="http://www.guoxuedashi.com/forum/">留言反馈</a>
</p>
  </div>
</div>
<!-- 头部导航END --> 
<!-- 内容区开始 --> 
<div class="w1180 clearfix">
  <div class="info l">
   
<div class="clearfix" style="background:#f5faff;">
<script src='http://www.guoxuedashi.com/img/headersou.js'></script>

</div>
  <div class="info_tree"><a href="http://www.guoxuedashi.com">首页</a> > <a href="/SiKuQuanShu/fanti/">四库全书</a>
 > <h1>资治通鉴</h1> <!--         下载:【右键另存为】即可 --></div>
  <div class="info_content zj clearfix">
  
<div class="info_txt clearfix" id="show">
<center style="font-size:24px;">64-資治通鑑卷六十三</center>
    資治通鑑卷六十三    宋 司馬光 撰<br />
<br />
  胡三省 音註<br />
<br />
  漢紀五十五【起屠維單閼盡上章執徐凡二年】<br />
<br />
  孝獻皇帝戊<br />
<br />
  建安四年春黑山賊帥張燕與公孫續率兵十萬三道救之【帥所類翻】未至瓚密使行人齎書告續使引五千銕騎於北隰之中【賢曰下溼曰隰孔穎逹曰下濕謂土地窊下常沮洳名為隰也】起火為應瓚欲自内出戰紹候得其書如期舉火瓚以為救至遂出戰紹設伏擊之瓚大敗復還自守【復扶又翻】紹為地道穿其樓下施木柱之度足逹半便燒之樓輒傾倒稍至京中【柱拄也易之中京瓚所居也度徒洛翻】瓚自計必無全乃悉縊其姊妹妻子然後引火自焚紹趣兵登臺斬之【縊於賜賜又於計翻趣讀曰促】田楷戰死關靖歎曰前若不止將軍自行未必不濟吾聞君子陷人危必同其難【難乃旦翻】豈可以獨生乎策馬赴紹軍而死【公孫瓚之計與陳宫之計一也陳宫之計呂布不能用公孫瓚之計關靖止之是知不惟决計之難贊决者亦難也】續為屠各所殺【屠各胡也屠直於翻】漁陽田豫說太守鮮于輔曰【輔既斬鄒丹遂領漁陽太守說輸芮翻守式又翻】曹氏奉天子以令諸侯終能定天下宜早從之輔乃率其衆以奉王命詔以輔為建忠將軍都督幽州六郡初烏桓王丘力居死子樓班年少從子蹋頓有武畧代立【少詩照翻從才用翻下同賢曰蹋音大蠟翻楊正衡晋書音義蹋徒合翻】摠攝上谷大人難樓遼東大人蘇僕延右北平大人烏延等袁紹攻公孫瓚蹋頓以烏桓助之瓚滅紹承制皆賜蹋頓難樓蘇僕延烏延等單于印綬又以閻柔得烏桓心因加寵慰以安北邊其後難樓蘇僕延奉樓班為單于以蹋頓為王然蹋頓猶秉計策 眭固屯射犬【郡國志河内野王縣有射犬聚唐懷州河内縣冇漢射犬故城眭息隨翻】夏四月曹操進軍臨河使將軍史渙曹仁渡河擊之仁操從弟也固自將兵北詣袁紹求救與渙仁遇於犬城渙仁擊斬之操遂濟河圍射犬射犬降【降戶江翻】操還軍敖倉初操在兖州舉魏种孝廉【种音冲】兖州叛【張邈舉兖州附呂布事見六十一卷興平元年】操曰唯魏种且不弃孤及聞种走操怒曰种不南走越北走胡不置汝也既下射犬生禽种操曰唯其才也釋其縛而用之以為河内太守屬以河北事【屬之欲翻】 以衛將軍董承為車騎將軍 袁術既稱帝淫侈滋甚媵御數百【媵以證翻】無不兼羅紈厭粱肉自下飢困莫之收恤既而資實空盡不能自立乃燒宫室犇其部曲陳簡靁薄於山【灊縣屬廬江郡有天柱山賢曰灊縣之山也灊今壽州霍山縣也灊音潜】復為簡等所拒遂大窮士卒散走憂懣不知所為【復扶又翻懣音悶】乃遣使歸帝號於從兄紹【紹與術同祖袁湯以親則從以年則兄也】曰禄去漢室久矣袁氏受命當王符瑞炳然今君擁有四州【賢曰青冀幽并】人戶百萬謹歸大命君其興之袁譚自青州迎術欲從下邳北過曹操遣劉備及將軍清河朱靈邀之術不得過復走壽春六月至江亭坐簀床而歎曰袁術乃至是乎【賢曰簀笫也謂無茵席也】因憤慨結病歐血死術從弟胤畏曹操不敢居壽春率其部曲奉術柩及妻子奔廬江太守劉勲於皖城【皖縣屬廬江郡今舒州也師古曰皖胡管翻杜祐曰音患 考異曰吳志孫策傳曰術死長史楊弘大將張勲等將其衆欲就策廬江太守劉勲邀擊悉虜之收其珍寶以歸與諸書不同今從范書陳志術傳及江表傳】故廣陵太守徐璆得傳國璽獻之【璆渠尤翻傳國璽術拘孫堅妻所奪者璽斯氏翻】 袁紹既克公孫瓚心益驕貢御稀簡主簿耿包密白紹宜應天人稱尊號紹以包白事示軍府【白事所白之事也】僚屬皆言包妖妄宜誅【妖於驕翻】紹不得已殺包以自解紹簡精兵十萬騎萬匹欲以攻許沮授諫曰近討公孫瓚師出歷年百姓疲敝倉庫無積未可動也宜務農息民先遣使獻捷天子若不得通乃表曹操隔我王路【沮子余翻王路謂尊王之路也】然後進屯黎陽漸營河南益作舟船繕修器械分遣精騎抄其邊鄙令彼不得安我取其逸如此可坐定也【使紹能用授言曹其殆乎抄楚交翻】郭圖審配曰以明公之神武引河朔之彊衆以伐曹操易如覆手【易以䜴翻】何必乃爾授曰夫救亂誅暴謂之義兵恃衆憑彊謂之驕兵義者無敵驕者先滅【前漢魏相上書曰兵義者王兵驕者滅】曹操奉天子以令天下今舉師南向於義則違且廟勝之策不在彊弱曹操法令既行士卒精練非公孫瓚坐而受攻者也今弃萬安之術而興無名之師【前漢董公曰兵出無名事故不成】竊為公懼之【為于偽翻下為子同】圖配曰武王伐紂不為不義况兵加曹操而云無名且以公今日之彊將士思奮不及時以定大業所謂天予不取反受其咎【史記范蠡之言】此越之所以覇吳之所以滅也監軍之計在於持牢【紹使授監護諸將故稱為監軍持牢猶今南人言把穩也監古衘翻】而非見時知幾之變也【幾居衣翻】紹納圖言圖等因是譖授曰授監統内外【監古衘翻】威震三軍若其寖盛何以制之夫臣與主同者亡此黄石之所忌也【臣與主同言作威作福與主無别也黄石即張良於下邳圯上所得之書也】且御衆於外不宜知内紹乃分授所統為三都督使授及郭圖淳于瓊各典一軍騎都尉清河崔琰諫曰天子在許民望助順不可攻也紹不從許下諸將聞紹將攻許皆懼曹操曰吾知紹之為人志大而智小色厲而膽薄忌克而少威【少詩沼翻下以少翻】兵多而分畫不明將驕而政令不壹【將即亮翻】土地雖廣糧食雖豐適足以為吾奉也孔融謂荀彧曰紹地廣兵強田豐許攸智士也為之謀審配逢紀忠臣也【逢皮江翻】任其事【任音壬】顔良文醜勇將也統其兵殆難克乎彧曰紹兵雖多而法不整田豐剛而犯上許攸貪而不治審配專而無謀逢紀果而自用此數人者埶不相容必生内變顔良文醜一夫之勇耳可一戰而禽也秋八月操進軍黎陽使臧霸等將精兵入青州以扞東方【臧覇起於泰山稱雄于東方者也故使之為扞袁氏雖欲自平原而東無能為矣】留于禁屯河上九月操還許分兵守官渡【賢曰裴松之北征記曰中牟臺下臨汴水是為官渡袁紹曹操壘尚存焉在今鄭州中牟縣北據水經註汴水即莨蕩渠也杜佑曰鄭州中牟縣北十二里有中牟臺是為官渡城袁曹相持之所】袁紹遣人招張繡并與賈詡書結好繡欲許之詡於繡坐上【好呼到翻坐徂卧翻】顯謂紹使曰歸謝袁本初兄弟不能相容【謂與袁術有隙各結黨與以相圖也顯者明言之於稠人中也】而能容天下國士乎繡驚懼曰何至于此竊謂詡曰若此當何歸詡曰不如從曹公繡曰袁強曹弱又先與曹為讐【謂淯水之戰殺其子也】從之如何詡曰此乃所以宜從也夫曹公奉天子以令天下其宜從一也紹彊盛我以少衆從之【少詩沼翻下同】必不以我為重曹公衆弱其得我必喜其宜從二也夫有覇王之志者固將釋私怨以明德於四海其宜從三也願將軍無疑冬十一月繡率衆降曹操【降戶江翻】操執繡手與歡宴為子均取繡女【為于偽翻取讀曰娶】拜揚武將軍表詡為執金吾封都亭侯【凡郡國縣道治所皆有都侯】關中諸將以袁曹方争皆中立顧望凉州牧韋端使從事天水楊阜詣許阜還關右諸將問袁曹勝敗孰在阜曰袁公寛而不斷好謀而少决不斷則無威【斷丁亂翻】少决則後事今雖彊終不能成大業曹公有雄才遠畧决機無疑法一而兵精能用度外之人所任各盡其力必能濟大事者也曹操使治書侍御史河東衛覬鎭撫關中【治直之翻覬音冀】時四方大有還民關中諸將多引為部曲覬書與荀彧曰關中膏腴之地頃遭荒亂人民流入荆州者十萬餘家聞本土安寧皆企望思歸【企去智翻舉踵也】而歸者無以自業諸將各競招懷以為部曲郡縣貧弱不能與爭兵家遂彊一旦變動必有後憂夫鹽國之大寶也亂來放散宜如舊置使者監賣【監古衘翻下同】以其直益市犂牛若有歸民以供給之勤耕積粟以豐殖關中遠民聞之必日夜競還又使司隸校尉留治關中以為之主【治直之翻】則諸將日削官民日盛此彊本弱敵之利也彧以白操操從之始遣謁者僕射監鹽官【河東安邑鹽池舊有鹽官鹽之為利厚矣齊用管子鬻筴而覇晉之定都諸大夫必欲其近鹽至漢武之世斡之以佐軍興及唐安史之亂第五琦榷鹽以贍國用自此遂為經賦其利居天下歲入之半監占䘖翻】司隸校尉治弘農【時以鍾繇為司隸校尉據魏畧及三國志繇實治洛陽蓋暫治弘農以招撫關中也】關中由是服從袁紹使人求助於劉表表許之而竟不至亦不援曹操從事中郎南陽韓嵩【漢制惟司隸校尉有從事中郎至漢末則州牧亦有從事中郎矣】别駕零陵劉先說表曰【說輸芮翻】今兩雄相持天下之重在于將軍若欲有為起乘其敝可也如其不然固將擇所宜從豈可擁甲十萬坐觀成敗求援而不能助見賢而不肯歸此兩怨必集於將軍恐不得中立矣曹操善用兵賢俊多歸之其埶必舉袁紹然後移兵以向江漢恐將軍不能禦也今之勝計【勝計謂諸計之中此計為勝也】莫若舉荆州以附曹操操必重德將軍長享福祚垂之後嗣此萬全之策也蒯越亦勸之【蒯若怪翻】表狐疑不斷乃遣嵩詣許曰今天下未知所定而曹操擁天子都許君為我觀其釁【為于偽翻下同】嵩曰聖逹節次守節【左傳曹公子欣時之言】嵩守節者也夫君臣名定以死守之今策名委質【質如字】唯將軍所命雖赴湯蹈火死無辭也以嵩觀之曹公必得志於天下將軍能上順天子下歸曹公使嵩可也如其猶豫嵩至京師天子假嵩一職不獲辭命則成天子之臣將軍之故吏耳在君為君則嵩守天子之命義不得復為將軍死也惟加重思【為于偽翻重除用翻重思猶言三思也】無為負嵩表以為憚使彊之【以其憚於使許強之使行使疏吏翻】至許詔拜嵩侍中零陵太守及還盛稱朝廷曹公之德勸表遣子入侍表大怒以為懷貳大會寮屬陳兵持節將斬之【持節以示將斬猶不敢專殺存漢制也】數曰韓嵩敢懷貳邪衆皆恐欲令嵩謝嵩不為動容【數所具翻為于偽翻】徐謂表曰將軍負嵩嵩不負將軍具陳前言表妻蔡氏諫曰韓嵩楚國之望也且其言直誅之無辭表猶怒考殺從行者【從才用翻下同】知無他意乃弗誅而囚之 揚州賊帥鄭寶欲略居民以赴江表【帥所類翻下同】以淮南劉曄高族名人【曄出於漢之宗室與蔣濟胡質俱為揚州名士】欲刼之使唱此謀曄患之會曹操遣使詣州有所案問曄要與歸家【要讀曰邀】寶來候使者曄留與宴飲手刃殺之斬其首以令寶軍曰曹公有令敢有動者與寶同罪其衆數千人皆讋服【讋即涉翻失氣也】推曄為主曄以其衆與廬江太守劉勲勲怪其故曄曰寶無法制其衆素以鈔畧為利僕宿無資【謂先無名位為之資也鈔楚交翻】而整齊之必懷怨難久故以相與耳【天下殽亂之時設有不幸為衆推當以劉曄為法】勲以袁術部曲衆多不能贍遣從弟偕求米於上繚諸宗帥不能滿數【不滿其所求之數也繚讀曰僚】偕召勲使襲之孫策惡勲兵強偽卑辭以事勲曰上繚宗民數欺鄙郡【惡烏路翻數所角翻】欲擊之路不便上繚甚富實願君伐之請出兵以為外援且以珠寶葛越賂勲【文選注曰葛越布也今葛布謂之葛越白布謂之白越】勲大喜外内盡賀劉曄獨否勲問其故對曰上繚雖小城堅池深攻難守易【易以䜴翻】不可旬日而舉也兵疲於外而國内虛策乘虚襲我則後不能獨守是將軍進屈於敵退無所歸若軍必出禍今至矣勲不聽遂伐上繚至海昬宗帥知之皆空壁逃遷勲了無所得時策引兵西擊黄祖行及石城【海昬縣屬豫章郡當豫章大江之口有地名慨口永元中分海昬置建昌縣上繚在建昌界石城縣屬丹陽郡賢曰在今蘇州西南予據水經石城縣在牛渚東酈道元注又云牛渚在石城東減五百里未知孰是又據五代志宣城秋浦縣舊曰石城宋白曰池州貴池石埭二縣皆漢石城縣之地】聞勲在海昬策乃分遣從兄賁輔將八千人屯彭澤【宋白曰彭澤縣取彭蠡澤為名漢屬豫章郡今江州彭澤縣南康軍都昌縣皆漢彭澤縣地】自與領江夏太守周瑜將二萬人襲皖城克之【夏戶雅翻皖戶坂翻】得術勲妻子及部曲三萬餘人表汝南李術為廬江太守給兵三千人以守皖城【為李術不附孫氏張本】皆徙所得民東詣吳勲還至彭澤孫賁孫輔邀擊破之勲走保流沂【流沂地名近西塞西塞山在今壽昌軍東北三十里】求救於黄祖祖遣其子射率船軍五千人助勲【船軍即舟師也】策復就攻勲【復扶又翻下同】大破之勲北歸曹操射亦遁走策收得勲兵二千餘人船千艘遂進擊黄祖十二月辛亥策軍至沙羨【沙羨縣屬江夏郡晉灼曰羨音夷水經注蒲圻江中有沙陽洲沙陽縣治縣本江夏之沙羨晉太康中改曰沙陽縣】劉表遣從子虎及南陽韓睎將長矛五千來救祖【從才用翻將即亮翻】甲寅策與戰大破之斬睎祖脱身走獲其妻子及船六千艘【艘蘇刀翻】士卒殺溺死者數萬人策盛兵將徇豫章屯于椒丘【椒丘去豫章南昌縣數十里】謂功曹虞翻曰華子魚自有名字【華歆字子魚自有名字言其名聞當時也】然非吾敵也若不開門讓城金鼓一震不得無所傷害卿便在前具宣孤意翻乃往見華歆曰竊聞明府與鄙郡故王府君齊名中州海内所宗雖在東垂常懷瞻仰歆曰孤不如王會稽【王朗為會稽太守為策所破會工外翻】翻復曰不審豫章資糧器仗士民勇果孰與鄙郡【復扶又翻】歆曰大不如也翻曰明府言不如王會稽謙光之譚耳【易曰謙尊而光譚與談同】精兵不如會稽實如尊敎孫討逆智畧超世用兵如神前走劉揚州君所親見【劉揚州謂劉繇】南定鄙郡亦君所聞也【鄙郡即謂會稽】今欲守孤城自料資糧已知不足不早為計悔無及也今大軍已次椒丘僕便還去明日日中迎檄不到者與君辭矣歆曰久在江表常欲北歸孫會稽來吾便去也乃夜作檄明旦遣吏齎迎策便進軍歆葛巾迎策 【考異曰華嶠譜叙曰孫策畧有揚州盛兵徇豫章一郡大恐官屬請出郊迎歆曰無然策稍進復白發兵又不聽及策至一府皆造閤請出避之乃笑曰今將自來何遽避之有頃門下白曰孫將軍至請見乃前與歆共坐談議良久夜乃别去義士聞之皆長歎而心自服也此說太不近人情今不取】策謂歆曰府君年德名望遠近所歸策年幼稚【稚直利翻】宜修子弟之禮便向歆拜禮為上賓<br />
<br />
  孫盛曰歆既無夷皓韜邈之風又失王臣匪躬之操【夷皓謂伯夷四皓也易曰王臣蹇蹇匪躬之故言華歆不能高尚其志又失蹇蹇匪躬之節也】橈心於邪儒之說交臂於陵肆之徒位奪節墮咎孰大焉【邪儒謂虞翻陵肆謂孫策也橈奴教翻墮讀曰隳】<br />
<br />
  策分豫章為廬陵郡以孫賁為豫章太守孫輔為廬陵太守會僮芝病輔遂進取廬陵【僮芝據廬陵事見上卷上年】留周瑜鎭巴丘【裴松之曰案孫策于時始得豫章廬陵尚未能得定江夏瑜之所鎮應在今巴丘縣也與後所屯巴丘處不同予據晉地理志廬陵郡有巴丘縣沈約曰晉立今撫州崇仁縣即其地梁改巴丘曰巴山】 孫策之克皖城也撫視袁術妻子及入豫章收載劉繇喪善遇其家士大夫以是稱之會稽功曹魏騰嘗迕策意【迕五故翻】策將殺之衆憂恐計無所出策母吳夫人倚大井謂策曰汝新造江南其事未集方當優賢禮士捨過錄功魏功曹在公盡規汝今日殺之則明日人皆叛汝吾不忍見禍之及當先投此井中耳策大驚遽釋騰初吳郡太守會稽盛憲舉高岱孝亷許貢來領郡岱將憲避難於營帥許昭家烏程鄒佗錢銅及嘉興王晟等【難乃旦翻帥所類翻姓譜彭祖裔孫孚為周錢府上士因官命氏佗徒河翻沈約曰嘉興縣本名長水秦改曰由拳吳孫權黄龍四年由拳縣生嘉禾改曰禾興孫皓避父名改曰嘉興縣屬吳郡晟承正翻】各聚衆萬餘或數千人不附孫策策引兵撲討皆破之【撲音普卜翻】進攻嚴白虎白虎兵敗犇餘杭【餘杭縣前漢屬會稽郡後漢分屬吳郡】投許昭程普請擊昭策曰許昭有義於舊君有誠於故友此丈夫之志也【裴松之曰許昭有義於舊君謂濟盛憲也有誠於故友則受嚴白虎也】乃舍之【舍讀曰捨】 曹操復屯官渡【復扶又翻】操常從士徐他等謀殺操【常從士常隨從在左右者也從音才用翻他音徒何翻】入操帳見校尉許禇色變禇覺而殺之 初車騎將軍董承稱受帝衣帶中密詔與劉備謀誅曹操操從容謂備曰【從千容翻】今天下英雄惟使君與操耳本初之徒不足數也備方食失匕箸【備以操知其英雄懼將圖已故驚失匕筯也匕匙也箸挾也箸音遲助翻】值天雷震備因曰聖人云迅雷風烈必變【論語記孔子之容】良有以也遂與承及長水校尉种輯將軍吳子蘭王服等同謀會操遣備與朱靈邀袁術程昱郭嘉董昭皆諫曰備不可遣也操悔追之不及術既南走朱靈等還備遂殺徐州刺史車胄留關羽守下邳行太守事身還小沛【車尺遮翻 考異曰蜀志先叙董承謀洩誅死備乃殺車胄魏志備殺車胄後明年董承乃死袁紀備據下邳亦在承死前蜀志誤也】東海賊昌豨及郡縣多叛操為備【據蜀志昌豨即昌覇豨許豈翻又音希呂布之敗太山諸屯帥皆降於曹操獨豨反側於其間蓋自恃其才畧過於臧覇之徒也】備衆數萬人遣使與袁紹連兵操遣司空長史沛國劉岱中郎將扶風王忠擊之不克備謂岱等曰使汝百人來無如我何曹公自來未可知耳<br />
<br />
  五年春正月董承謀洩壬子曹操殺承及王服种輯皆夷三族操欲自討劉備諸將皆曰與公爭天下者袁紹也今紹方來而弃之東【言紹方來寇乃弃而不顧而東征備也】紹乘人後若何操曰劉備人傑也今不擊必為後患郭嘉曰紹性遲而多疑來必不速備新起衆心未附急擊之必敗操師遂東冀州别駕田豐說袁紹曰曹操與劉備連兵未可卒解【說輸芮翻卒讀曰猝】公舉軍而襲其後可一往而定紹辭以子疾未得行豐舉杖擊地曰嗟乎遭難遇之時而以嬰兒病失其會惜哉事去矣曹操擊劉備破之 【考異曰魏書曰備謂操與大敵連不得東而候騎卒至言曹公來備大驚然猶未信自將數十出望公車見麾旌便弃衆而走計備必不至此魏書多妄】獲其妻子進拔下邳禽關羽又擊昌豨破之備奔青州因袁譚以歸袁紹紹聞備至去鄴二百里迎之【紹遠出迎備重敬之也】駐月餘所亡士卒稍稍歸之曹操還軍官渡紹乃議攻許田豐曰曹操既破劉備則許下非復空虛【復扶又翻】且操善用兵變化無方衆雖少【少詩沼翻下同】未可輕也今不如以久持之將軍據山河之固擁四州之衆外結英雄内修農戰然後簡其精鋭分為奇兵【孫子兵法曰凡戰以正合以奇勝注曰正者當敵奇者擊其不備】乘虚迭出以擾河南救右則擊其左救左則擊其右使敵疲於奔命民不得安業我未勞而彼已困不及三年可坐克也今釋廟勝之策【定策於廟堂之上而决勝於千里之外謂之廟勝孫子曰未戰而廟勝得算多也未戰而廟不勝得算少也】而决成敗於一戰若不如志悔無及也紹不從豐彊諫忤紹紹以為沮衆械繫之【忤五故翻沮在呂翻】於是移檄州郡數操罪惡【數所具翻】二月進軍黎陽沮授臨行會其宗族散資財以與之【沮子余翻】曰埶存則威無不加埶亡則不保一身哀哉其弟宗曰曹操士馬不敵君何懼焉授曰以曹操之明畧又挾天子以為資我雖克伯珪【公孫瓚字伯珪】衆實疲敝而主驕將忲【將即亮翻忲他蓋翻侈也】軍之破敗在此舉矣楊雄有言六國蚩蚩為嬴弱姬其今之謂乎【賢曰法言之文也嬴秦姓姬周姓方言曰蚩悖也六國悖惑侵弱周室終為秦所併也為于偽翻】振威將軍程昱【沈約曰振威將軍始於後漢初宋登為之】以七百兵守鄄城【鄄音絹】曹操欲益昱兵二千昱不肯曰袁紹操十萬衆自以所向無前今見昱少兵必輕易不來【少詩沼翻下同易以䜴翻】攻若益昱兵過則不可不攻攻之必克徒兩損其埶願公無疑紹聞昱兵少果不往操謂賈詡曰程昱之膽過於賁育矣【賁音奔】袁紹遣其將顔良攻東郡太守劉延於白馬【賢曰白馬縣屬東郡今滑州縣也故城在今縣東】沮授曰良性促狹雖驍勇不可獨任紹不聽【驍堅堯翻】夏四月曹操北救劉延荀攸曰今兵少不敵必分其勢乃可公到延津【杜預曰陳留酸棗縣北有延津唐衛州新鄉縣有延津關關蓋在延津北岸曹操所向乃延津南岸】若將渡兵向其後者紹必西應之然後輕兵襲白馬掩其不備顔良可禽也操從之紹聞兵渡即分兵西邀之操乃引軍兼行趣白馬【趣七喻翻】未至十餘里良大驚來逆戰操使張遼關羽先登擊之羽望見良麾蓋【戎車大將所乘者設幢麾張蓋】策馬刺良於萬衆之中【刺七亦翻】斬其首而還【還從宣翻又如字】紹軍莫能當者遂解白馬之圍徙其民循河而西紹渡河追之沮授諫曰勝負變化不可不詳今宜留屯延津分兵官渡若其克獲還迎不晩【還迎留屯大軍也】設其有難【難乃旦翻】衆弗可還紹弗從授臨濟歎曰上盈其志下務其功悠悠黄河吾其濟乎遂以疾辭紹不許而意恨之復省其所部并屬郭圖紹軍至延津南操勒兵駐營南阪下【水經注白馬縣有神馬亭實中層崎南北二百步東西五十餘步自外耕耘墾斫削落平盡正南有陟躔陛下方軌西去白馬津可二十里南距白馬縣故城可五十里即開山圖所謂白馬山也南阪其在山之南與此時操兵循河已入酸棗界當攷】使登壘望之曰可五六百騎有頃復白騎稍多步兵不可勝數【復扶又翻下同勝音升數所具翻】操曰勿復白令騎解鞍放馬是時白馬輜重就道諸將以為敵騎多不如還保營荀攸曰此所以餌敵如何去之操顧攸而笑紹騎將文醜與劉備將五六千騎前後至諸將復白可上馬操曰未也有頃騎至稍多或分趣輜重【趣七喻翻重直用翻】操曰可矣乃皆上馬時騎不滿六百遂縱兵擊大破之斬醜醜與顔良皆紹名將也再戰悉禽之紹軍奪氣【三軍以氣為主氣奪則其軍不振】初操壯關羽之為人而察其心神無久留之意使張遼以其情問之羽嘆曰吾極知曹公待我厚然吾受劉將軍恩誓以共死不可背之【背蒲妹翻】吾終不留要當立効以報曹公乃去耳遼以羽言報操操義之及羽殺顔良操知其必去重加賞賜羽盡封其所賜拜書告辭而犇劉備於袁軍【袁紹軍也】左右欲追之操曰彼各為其主【為于偽翻】勿追也操還軍官渡閻柔遣使詣操操以柔為烏桓校尉鮮于輔身見操於官渡操以輔為右度遼將軍還鎭幽土【當是時幽州為紹所統與許隔遠而柔輔已歸心於操矣漢度遼將軍始於范明友中興之後置度遼將軍以護南匈奴屯於西河今使鮮于輔還鎮幽土故以為右度遼將軍自中國而北向以西河為左幽土為右也】 廣陵太守陳登治射陽【射陽縣前漢屬臨淮郡後漢屬廣陵郡應劭曰在射水之陽今楚州山陽縣有射陽湖即其地賢曰射陽在今楚州安宜縣東】孫策西擊黄祖登誘嚴白虎餘黨圖為後害策還擊登軍到丹徒【丹徒縣前漢屬會稽郡後漢分屬吳郡春秋之朱方也秦時望氣者云其地有天子氣始皇使赭徒二千人鑿城以敗其埶改曰丹徒 考異曰此事出江表傳據策傳云策謀襲許未發而死陳矯傳云登為孫權所圍於匡奇登令矯求救於太祖太祖遣赴救吳軍既退登設伏追奔大破之先賢行狀云登有吞滅江南之志孫策遣軍攻登於匡奇城登大破之斬虜以萬數賊忿喪軍尋復大興兵向登登使功曹陳矯求救於太祖此數者參差不同孫盛異同評云按袁紹以建安五年至黎陽策以四月遇害而志云策聞曹公與紹相拒於官渡謬矣伐登之言為有證也今從之】須待運糧初策殺吳郡太守許貢 【考異曰江表傳曰初貢上表於漢帝言策驍雄宜召還京邑若放於外必作世患候吏得表以示策策以讓貢貢辭無表策令武士絞殺之按貢先為朱治所廹已去郡依嚴白虎安能復爾盖策破白虎時殺貢耳】貢奴客濳民間欲為貢報讐策性好獵數出驅馳【為于偽翻好呼到翻數所角翻】所乘馬精駿從騎絶不能及【從才用翻】卒遇貢客三人【卒讀曰猝】射策中頰後騎尋至皆刺殺之策創甚【射而亦翻中竹仲翻刺七亦翻創初良翻】召張昭等謂曰中國方亂以吳越之衆三江之固【韋昭曰三江謂吳松江錢塘江浦陽江也吳地記云松江東北行七十里得三江口東北入海為婁江東南入海為東江幷松江為三江】足以觀成敗公等善相吾弟【相息亮翻】呼權佩以印綬謂曰舉江東之衆决機於兩陳之間【陳讀曰陣】與天下爭衡【衡所以平輕重也爭衡言分爭之世兵力所加天下大埶為之輕重也】卿不如我舉賢任能各盡其心以保江東我不如卿丙午策卒 【考異曰虞喜志林云策以四月四日死故置此陳志策傳策陰欲襲許迎漢帝密治兵部署未發為許貢客所殺郭嘉傳曰策渡江北襲許衆聞皆懼嘉料之曰策輕而無備必死於匹夫之手果為貢客所殺嘉雖先見安能知策死於未襲許之前乎蓋時人見策臨江治兵疑其襲許嘉料其不能為耳】時年二十六權悲號未視事【號戶刀翻】張昭曰孝廉此寧哭時邪【孫權先為陽羨長郡察孝廉故以稱之】乃改易權服扶令上馬使出廵軍昭率僚屬上表朝廷下移屬城中外將校各令奉職周瑜自巴丘將兵赴喪遂留吳以中護軍與張昭共掌衆事【秦置護軍都尉漢因之高祖以陳平為護軍中尉武帝復以為護軍都尉屬大司馬三國虎争始有中護軍之官東觀記曰漢大將軍出征置中護軍一人魏晉以後資輕者為中護軍資重者為護軍將軍然吳又有左右護軍則吳制自是分中左右為三部】時策雖有會稽吳郡丹陽豫章廬江廬陵然深險之地猶未盡從流寓之士皆以安危去就為意未有君臣之固而張昭周瑜等謂權可與共成大業遂委心而服事焉 秋七月立皇子馮為南陽王壬午馮薨 汝南黄巾劉辟等叛曹操應袁紹紹遣劉備將兵助辟郡縣多應之紹遣使拜陽安都尉李通為征南將軍劉表亦陰招之通皆拒焉或勸通從紹通按劍叱之曰曹公明哲必定天下紹雖彊盛終為之虜耳吾以死不貳即斬紹使【使疏吏翻】送印綬詣操通急錄戶調【調徒釣翻下同戶出綿絹謂之調錄收拾也】朗陵長趙儼見通曰方今諸郡並叛獨陽安懷附復趣收其緜絹【復扶又翻趣讀曰促】小人樂亂【樂音洛】無乃不可乎通曰公與袁紹相持甚急左右郡縣背叛乃爾【背蒲妹翻下同】若緜絹不調送觀聽者必謂我顧望有所須待也儼曰誠亦如君慮然當權其輕重小緩調當為君釋此患【為于偽翻】乃書與荀彧曰今陽安郡百姓困窮鄰城並叛易用傾蕩【易以豉翻】乃一方安危之機也且此郡人執守忠節在險不貳以為國家宜垂慰撫而更急歛緜絹【歛力贍翻】何以勸善彧即白操悉以緜絹還民上下歡喜郡内遂安通擊羣賊瞿㳟等皆破之【瞿姓也王僧孺百家譜有蒼梧瞿寶】遂定淮汝之地時操制新科下州郡頗增嚴峻而調緜絹方急長廣太守何夔【長廣縣前漢屬琅邪郡後漢屬東萊郡此盖操遣樂進入青州新收以為郡】言於操曰先王辨九服之賦以殊遠近【周官職方氏辨九服之邦國方千里曰王畿其外方五百里曰侯服又其外方五百里曰甸服又其外方五百里曰男服又其外方五百里曰采服又其外方五百里曰衛服又其外方五百里曰蠻服又其外方五百里曰夷服又其外方五百里曰鎭服又其外方五百里曰藩服】制三典之刑以平治亂【周官大司寇掌建邦之三典以佐王刑邦國一曰刑新國用輕典二曰刑平國用中典三曰刑亂國用重典治直吏翻】愚以為此郡宜依遠域新邦之典其民間小事使長吏臨時隨宜上不背正法下以順百姓之心【背蒲妹翻】比及三年【比必寐翻】民安其業然後乃可齊之以法也操從之劉備畧汝潁之間自許以南吏民不安曹操患之曹仁曰南方以大將軍方有目前急其埶不能相救劉備以彊兵臨之其背叛故宜也備新將紹兵未能得其用擊之可破也操乃使仁將騎擊備破走之【將即亮翻】盡復收諸叛縣而還備還至紹軍陰欲離紹【還從宣翻又如字離力智翻去也】乃說紹南連劉表紹遣備將本兵復至汝南【說輸芮翻復扶又翻下同】與賊龔都等合衆數千人曹操遣將蔡楊擊之為備所殺袁紹軍陽武【陽武縣屬河南尹在官渡水北】沮授說紹曰北兵雖衆而勁果不及南南軍穀少而資儲不如北南幸於急戰北利在緩師宜徐持久曠以日月紹不從八月紹進營稍前依沙塠為屯【塠都回翻】東西數十里操亦分營與相當 九月庚午朔日有食之 曹操出兵與袁紹戰不勝復還堅壁紹為高櫓【賢曰釋名曰櫓者露上無覆屋也】起土山射營中【射而亦翻】營中皆蒙楯而行【楯食尹翻賢曰今之旁排也】操乃為霹靂車【賢曰以其石聲烈震呼之為霹靂即今之砲車也張晏曰范蠡兵法飛石重十二斤為機行三百步操盖祖其遺法耳魏氏春秋曰以古有矢石又傳云旝動而鼓說曰旝石也於是造發石車車尺遮翻】發石以擊紹樓皆破紹復為地道攻操操輒於内為長塹以拒之操衆少粮盡【少詩沼翻下同】士卒疲乏百姓困於征賦多叛歸紹者操患之與荀彧書議欲還許以致紹師【賢曰致猶至也兵法善戰者致人不致於人】彧報曰紹悉衆聚官渡欲與公决勝敗公以至弱當至彊若不能制必為所乘是天下之大機也且紹布衣之雄耳能聚人而不能用以公之神武明哲而輔以大順何向而不濟今穀食雖少未若楚漢在滎陽成臯間也是時劉項莫肯先退者以為先退則埶屈也公以十分居一之衆【賢曰言與紹衆相懸也】畫地而守之【賢曰言畫地作限隔也】搤其喉而不得進已半年矣【搤於革翻】情見埶竭必將有變【見賢遍翻】此用奇之時不可失也操從之乃堅壁持之操見運者撫之曰却十五日【却後也晉人帖中多用少却字其意猶言小退也】為汝破紹不復勞汝矣【為于偽翻復扶又翻下同】紹運穀車數千乘至官渡【乘繩證翻下同】荀攸言於操曰紹運車旦暮至其將韓猛鋭而輕敵擊可破也操曰誰可使者攸曰徐晃可乃遣偏將軍河東徐晃【按沈約志曹魏置將軍四十號偏將軍禆將軍居其末】與史渙邀擊猛破走之燒其輜重【重直用翻下同】冬十月紹復遣車運穀使其將淳于瓊等將兵萬餘人送之宿紹營北四十里沮授說紹可遣蔣奇别為支軍於表【說輸芮翻支别也表外也】以絶曹操之鈔【鈔楚交翻】紹不從許攸曰曹操兵少而悉師拒我許下餘守埶必空弱若分遣輕軍星行掩襲【星行戴星而行也】許可拔也許拔則奉迎天子以討操操成禽矣如其未潰可令首尾奔命破之必也紹不從曰吾要當先取操會攸家犯法審配收繫之攸怒遂奔操【考異曰魏志武紀曰攸貪財袁紹不能足來奔今從范書紹傳】操聞攸來跣出迎之<br />
<br />
  撫掌笑曰子卿遠來吾事濟矣【許攸字子逹今呼為子卿貴之也或曰操字攸曰子遠卿來吾事濟矣於文為順】既入坐【坐徂卧翻】謂操曰袁氏軍盛何以待之今有幾糧乎操曰尚可支一歲攸曰無是更言之又曰可支半歲攸曰足下不欲破袁氏邪何言之不實也操曰向言戲之耳其實可一月為之奈何攸曰公孤軍獨守外無救援而糧穀已盡此危急之日也袁氏輜重萬餘乘在故市烏巢【據水經烏巢澤在陳留酸棗縣東南乘繩證翻】屯軍無嚴備若以輕兵襲之不意而至燔其積聚【積七賜翻聚慈喻翻】不過三日袁氏自敗也操大喜乃留曹洪荀攸守營自將步騎五千人皆用袁軍旗幟【幟赤志翻】衘枚縛馬口夜從間道出【間古莧翻】人抱束薪所歷道有問者語之曰【語牛倨翻】袁公恐曹操鈔畧後軍遣兵以益備聞者信以為然皆自若既至圍屯大放火營中驚亂會明瓊等望見操兵少出陳門外【陳讀曰陣】操急擊之瓊退保營操遂攻之紹聞操擊瓊謂其子譚曰就操破瓊吾拔其營彼固無所歸矣【就即也言即使操破淳于瓊而我攻拔其營將無所歸也】乃使其將高覽張郃等攻操營【郃曷閤翻又古盍翻】郃曰曹公精兵往必破瓊等瓊等破則事去矣請先往救之郭圖固請攻操營郃曰曹公營固攻之必不拔若瓊等見禽吾屬盡為虜矣紹但遣輕騎救瓊而以重兵攻操營不能下紹騎至烏巢操左右或言賊騎稍近請分兵拒之操怒曰賊在背後乃白士卒皆殊死戰遂大破之斬瓊等盡燔其糧穀士卒千餘人皆取其鼻牛馬割脣舌以示紹軍紹軍將士皆恟懼【恟許勇翻】郭圖慙其計之失復譖張郃於紹曰【復扶又翻】郃快軍敗郃忿懼遂與高覽焚攻具詣操營降【降戶江翻下同】曹洪疑不敢受荀攸曰郃計畫不用怒而來奔君有何疑乃受之於是紹軍驚擾大潰紹及譚等幅巾乘馬【傳子曰漢末王公多委正服以幅巾為雅是以袁紹崔豹之徒雖為將帥皆著縑巾魏太祖以天下凶荒資財乏匱擬古皮弁裁縑帛以為帢合乎簡易隨時之義以色别其貴賤于今施行可謂軍容非國容也】與八百騎渡河操追之不及盡收其輜重圖書珍寶餘衆降者操盡阬之前後所殺七萬餘人 【考異曰范書紹傳曰所殺八萬人按獻帝起居注曹公上言凡斬首七萬餘級】沮授不及紹渡為操軍所執乃大呼曰【呼火故翻】授不降也為所執耳操與之有舊迎謂曰分野殊異遂用圯絶【二十八宿布列於天各有躔度周天三百六十五度四分度之一分為十二次班固取三統歷十二次配十二野而分野之說行焉費直說周易蔡邕月令章句所言頗有先後魏太史令陳卓更言郡國所入宿度而分野之說詳矣皇甫謐曰黄帝推分星次以定律度天有十二次日月之所躔也地有十二分王侯之所國也分扶問翻圮當作否否隔也】不圖今日乃相禽也授曰冀州失策自取犇北【紹牧冀州故稱之猶劉備以牧豫州稱之為劉豫州】授知力俱困宜其見禽操曰本初無謀不相用計今喪亂未定【知讀曰智喪息浪翻】方當與君圖之授曰叔父母弟縣命袁氏【縣讀曰懸】若蒙公靈速死為福操歎曰孤早相得天下不足慮也遂赦而厚遇焉授尋謀歸袁氏操乃殺之操收紹書中得許下及軍中人書皆焚之曰當紹之彊孤猶不能自保况衆人乎【此光武安反側之意英雄處事世雖相遠若合符節】冀州城邑多降於操【降戶江翻】袁紹走至黎陽北㟁入其將軍蔣義渠營把其手曰孤以首領相付矣義渠避帳而處之【處昌呂翻】使宣號令衆聞紹在稍復歸之或謂田豐曰君必見重矣豐曰公貌寛而内忌不亮吾忠【亮信也明也】而吾數以至言迕之【數所角翻迕五故翻】若勝而喜猶能救我今戰敗而恚【恚於避翻】内忌將發吾不望生紹軍士皆拊膺泣曰向令田豐在此必不至于敗紹謂逢紀曰【逢皮江翻】冀州諸人聞吾軍敗皆當念吾惟田别駕前諫止吾與衆不同吾亦慙之紀曰豐聞將軍之退拊手大笑喜其言之中也【中竹仲翻】紹于是謂僚屬曰吾不用田豐言果為所笑遂殺之初曹操聞豐不從戎【謂紹囚之不使從軍也】喜曰紹必敗矣及紹犇遁復曰【復扶又翻下同】向使紹用其别駕計尚未可知也審配二子為操所禽紹將孟岱言於紹曰配在位專政族大兵彊且二子在南必懷反計郭圖辛評亦以為然紹遂以岱為監軍代配守鄴【監古衘翻】護軍逢紀素與配不睦紹以問之紀曰配天性烈直每慕古人之節必不以二子在南為不義也願公勿疑紹曰君不惡之邪【惡烏路翻】紀曰先所爭者私情也今所陳者國事也紹曰善乃不廢配配由是更與紀親【逢紀能為審配言而不肯救田豐之死果為國事乎】冀州城邑叛紹者紹稍復擊定之紹為人寛雅有局度喜怒不形於色而性矜愎自高【愎平逼翻戾也狠也】短於從善故至於敗 冬十月辛亥有星孛于大梁【賢曰大梁酉之分蔡邕曰自胃一度至畢六度謂之大梁之次皇甫謐曰自胃七度至畢十度曰大梁之次晉書天文志從謐孛蒲内翻】 廬江太守李術攻殺揚州刺史嚴象廬江梅乾雷緒陳蘭等各聚衆數萬在江淮間曹操表沛國劉馥為揚州刺史時揚州獨有九江【時廬江丹陽會稽吳郡豫章皆屬孫氏馥刺揚州獨有九江耳】馥單馬造合肥空城建立州治【郡國志漢揚州刺史治歷陽今馥移合肥後又移治壽春而江左揚州治建業揚州分矣造七到翻】招懷乾緒等皆貢獻相繼數年中恩化大行流民歸者以萬數於是廣屯田興陂堨【堨於葛翻以土壅水曰堨】官民有畜乃聚諸生立學校又高為城壘多積木石以修戰守之備【為孫權攻合肥不能下張本】 曹操聞孫策死欲因喪伐之侍御史張紘諫曰【三年策遣紘獻方物至許拜侍御史】乘人之喪既非古義【古不伐喪】若其不克成讐棄好【好呼到翻】不如因而厚之操即表權為討虜將軍【討虜將軍之號創置於此】領會稽太守【會工外翻】操欲令紘輔權内附乃以紘為會稽東部都尉【沈約曰臨海太守本會稽東部都尉治前漢都尉治鄞後漢分會稽為吳郡疑是都尉徙治章安也】紘至吳太夫人以權年少【少詩照翻】委紘與張昭共輔之紘思惟補察知無不為太夫人問揚武都尉會稽董襲曰江東可保不【不讀曰否】襲曰江東有山川之固而討逆明府恩德在民討虜承基大小用命【討逆策也討虜權也】張昭秉衆事襲等為爪牙此地利人和之時也萬無所憂權遣張紘之部或以紘本受北任嫌其志趣不止於此權不以介意【介間也纎微也言其意不以纎微嫌間也】魯肅將北還【肅從孫策事見上卷三年】周瑜止之 【考異曰肅傳曰劉子揚招肅往依鄭寶肅將從之瑜以權可輔止肅按劉曄殺鄭寶以其衆與劉勲勲為策所滅寶安得及權時也】因薦肅於權曰肅才宜佐時當廣求其比以成功業權即見肅與語悦之賓退獨引肅合榻對飲曰【榻床也有坐榻有卧榻今江南又呼几案之屬為卓牀卓高也以其比坐榻卧榻為高也合榻猶言合卓也】今漢室傾危孤思有桓文之功君何以佐之肅曰昔高帝欲尊事義帝而不獲者以項羽為害也今之曹操猶昔項羽將軍何由得為桓文乎肅竊料之漢室不可復興曹操不可卒除【復扶又翻卒讀曰猝】為將軍計惟有保守江東以觀天下之釁耳若因北方多務勦除黄祖進伐劉表竟長江所極據而有之此王業也【江東君臣上下本謀不過此耳】權曰今盡力一方冀以輔漢耳此言非所及也張昭毁肅年少麤疎【少詩沼翻】權益貴重之賞賜儲偫富擬其舊【魯肅家本饒富先嘗指囷以資周瑜矣偫直里翻】權料諸小將兵少而用薄者幷合之【料力條翻量也又力弔翻】别部司馬汝南呂蒙【續漢志大將軍營五部部各有校尉一人軍司馬一人其别營領属為别部司馬其兵多少各隨時宜】軍容鮮整士卒練習權大悦增其兵寵任之功曹駱統勸權尊賢接士勤求損益饗賜之日人人别進問其燥濕【人之居處避濕就燥問其燥濕者問其居處何如也】加以密意誘諭使言察其志趣權納用焉統俊之子也【駱俊見上卷二年誘音酉】廬陵太守孫輔恐權不能保江東陰遣人齎書呼曹操行人以告權悉斬輔親近分其部曲徙輔置東【置之吳東也】曹操表徵華歆為議郎參司空軍事廬江太守李術不肯事權而多納其亡叛【術本權兄策所樹置也】權以狀白曹操曰嚴刺史昔為公所用而李術害之肆其無道宜速誅滅今術必復詭說求救明公居阿衡之任【以伊尹况操復扶又翻下同】海内所瞻願敇執事勿復聽受因舉兵攻術於皖城【皖戶板翻】術求救於操操不救遂屠其城梟術首【梟堅堯翻】徙其部曲二萬餘人 劉表攻張羨連年不下【羨叛表事始上卷三年】曹操方與袁紹相拒未暇救之羨病死長沙復立其子懌表攻懌及零桂皆平之于是表地方數千里帶甲十餘萬遂不供職貢郊祀天地居處服用僭擬乘輿焉【處昌呂翻】 張魯以劉璋闇懦不復承順襲别部司馬張修殺之而幷其衆【魯初與修取漢中事見六十卷初平二年】璋怒殺魯母及弟魯遂據漢中與璋為敵璋遣中郎將龎羲擊之不克璋以羲為巴郡太守屯䦘中以禦魯【閬中縣屬巴郡】羲輒召漢昌賨民為兵【譙周巴記曰和帝永元中分宕渠之地置漢昌縣屬巴郡夷人歲入賨錢口四十謂之賨民賨徂宗翻】或構羲於璋璋疑之趙韙數諫不從亦恚恨【數所角翻】初南陽三輔民流入益州者數萬家劉焉悉收以為兵名曰東州兵璋性寛柔無威畧東州人侵暴舊民璋不能禁趙韙素得人心【趙韙從焉入蜀璋又韙所立益州之大吏也】因益州士民之怨遂作亂引兵數萬攻璋厚賂荆州【荆州劉表也】與之連和蜀郡廣漢犍為皆應之【犍居言翻】<br />
<br />
  資治通鑑卷六十三  <br>
   </div> 

<script src="/search/ajaxskft.js"> </script>
 <div class="clear"></div>
<br>
<br>
 <!-- a.d-->

 <!--
<div class="info_share">
</div> 
-->
 <!--info_share--></div>   <!-- end info_content-->
  </div> <!-- end l-->

<div class="r">   <!--r-->



<div class="sidebar"  style="margin-bottom:2px;">

 
<div class="sidebar_title">工具类大全</div>
<div class="sidebar_info">
<strong><a href="http://www.guoxuedashi.com/lsditu/" target="_blank">历史地图</a></strong>  
<a href="http://www.880114.com/" target="_blank">英语宝典</a>  
<a href="http://www.guoxuedashi.com/13jing/" target="_blank">十三经检索</a> 
<br><strong><a href="http://www.guoxuedashi.com/gjtsjc/" target="_blank">古今图书集成</a></strong> 
<a href="http://www.guoxuedashi.com/duilian/" target="_blank">对联大全</a> <strong><a href="http://www.guoxuedashi.com/xiangxingzi/" target="_blank">象形文字典</a></strong> 

<br><a href="http://www.guoxuedashi.com/zixing/yanbian/">字形演变</a>  <strong><a href="http://www.guoxuemi.com/hafo/" target="_blank">哈佛燕京中文善本特藏</a></strong>
<br><strong><a href="http://www.guoxuedashi.com/csfz/" target="_blank">丛书&方志检索器</a></strong> <a href="http://www.guoxuedashi.com/yqjyy/" target="_blank">一切经音义</a>  

<br><strong><a href="http://www.guoxuedashi.com/jiapu/" target="_blank">家谱族谱查询</a></strong>  <strong><a href="http://shufa.guoxuedashi.com/sfzitie/" target="_blank">书法字帖欣赏</a></strong> 
<br>

</div>
</div>


<div class="sidebar" style="margin-bottom:0px;">

<font style="font-size:22px;line-height:32px">QQ交流群9:489193090</font>


<div class="sidebar_title">手机APP 扫描或点击</div>
<div class="sidebar_info">
<table>
<tr>
	<td width=160><a href="http://m.guoxuedashi.com/app/" target="_blank"><img src="/img/gxds-sj.png" width="140"  border="0" alt="国学大师手机版"></a></td>
	<td>
<a href="http://www.guoxuedashi.com/download/" target="_blank">app软件下载专区</a><br>
<a href="http://www.guoxuedashi.com/download/gxds.php" target="_blank">《国学大师》下载</a><br>
<a href="http://www.guoxuedashi.com/download/kxzd.php" target="_blank">《汉字宝典》下载</a><br>
<a href="http://www.guoxuedashi.com/download/scqbd.php" target="_blank">《诗词曲宝典》下载</a><br>
<a href="http://www.guoxuedashi.com/SiKuQuanShu/skqs.php" target="_blank">《四库全书》下载</a><br>
</td>
</tr>
</table>

</div>
</div>


<div class="sidebar2">
<center>


</center>
</div>

<div class="sidebar"  style="margin-bottom:2px;">
<div class="sidebar_title">网站使用教程</div>
<div class="sidebar_info">
<a href="http://www.guoxuedashi.com/help/gjsearch.php" target="_blank">如何在国学大师网下载古籍?</a><br>
<a href="http://www.guoxuedashi.com/zidian/bujian/bjjc.php" target="_blank">如何使用部件查字法快速查字?</a><br>
<a href="http://www.guoxuedashi.com/search/sjc.php" target="_blank">如何在指定的书籍中全文检索?</a><br>
<a href="http://www.guoxuedashi.com/search/skjc.php" target="_blank">如何找到一句话在《四库全书》哪一页?</a><br>
</div>
</div>


<div class="sidebar">
<div class="sidebar_title">热门书籍</div>
<div class="sidebar_info">
<a href="/so.php?sokey=%E8%B5%84%E6%B2%BB%E9%80%9A%E9%89%B4&kt=1">资治通鉴</a> <a href="/24shi/"><strong>二十四史</strong></a>&nbsp; <a href="/a2694/">野史</a>&nbsp; <a href="/SiKuQuanShu/"><strong>四库全书</strong></a>&nbsp;<a href="http://www.guoxuedashi.com/SiKuQuanShu/fanti/">繁体</a>
<br><a href="/so.php?sokey=%E7%BA%A2%E6%A5%BC%E6%A2%A6&kt=1">红楼梦</a> <a href="/a/1858x/">三国演义</a> <a href="/a/1038k/">水浒传</a> <a href="/a/1046t/">西游记</a> <a href="/a/1914o/">封神演义</a>
<br>
<a href="http://www.guoxuedashi.com/so.php?sokeygx=%E4%B8%87%E6%9C%89%E6%96%87%E5%BA%93&submit=&kt=1">万有文库</a> <a href="/a/780t/">古文观止</a> <a href="/a/1024l/">文心雕龙</a> <a href="/a/1704n/">全唐诗</a> <a href="/a/1705h/">全宋词</a>
<br><a href="http://www.guoxuedashi.com/so.php?sokeygx=%E7%99%BE%E8%A1%B2%E6%9C%AC%E4%BA%8C%E5%8D%81%E5%9B%9B%E5%8F%B2&submit=&kt=1"><strong>百衲本二十四史</strong></a>  <a href="http://www.guoxuedashi.com/so.php?sokeygx=%E5%8F%A4%E4%BB%8A%E5%9B%BE%E4%B9%A6%E9%9B%86%E6%88%90&submit=&kt=1"><strong>古今图书集成</strong></a>
<br>

<a href="http://www.guoxuedashi.com/so.php?sokeygx=%E4%B8%9B%E4%B9%A6%E9%9B%86%E6%88%90&submit=&kt=1">丛书集成</a> 
<a href="http://www.guoxuedashi.com/so.php?sokeygx=%E5%9B%9B%E9%83%A8%E4%B8%9B%E5%88%8A&submit=&kt=1"><strong>四部丛刊</strong></a>  
<a href="http://www.guoxuedashi.com/so.php?sokeygx=%E8%AF%B4%E6%96%87%E8%A7%A3%E5%AD%97&submit=&kt=1">說文解字</a> <a href="http://www.guoxuedashi.com/so.php?sokeygx=%E5%85%A8%E4%B8%8A%E5%8F%A4&submit=&kt=1">三国六朝文</a>
<br><a href="http://www.guoxuedashi.com/so.php?sokeytm=%E6%97%A5%E6%9C%AC%E5%86%85%E9%98%81%E6%96%87%E5%BA%93&submit=&kt=1"><strong>日本内阁文库</strong></a> <a href="http://www.guoxuedashi.com/so.php?sokeytm=%E5%9B%BD%E5%9B%BE%E6%96%B9%E5%BF%97%E5%90%88%E9%9B%86&ka=100&submit=">国图方志合集</a> <a href="http://www.guoxuedashi.com/so.php?sokeytm=%E5%90%84%E5%9C%B0%E6%96%B9%E5%BF%97&submit=&kt=1"><strong>各地方志</strong></a>

</div>
</div>


<div class="sidebar2">
<center>

</center>
</div>
<div class="sidebar greenbar">
<div class="sidebar_title green">四库全书</div>
<div class="sidebar_info">

《四库全书》是中国古代最大的丛书,编撰于乾隆年间,由纪昀等360多位高官、学者编撰,3800多人抄写,费时十三年编成。丛书分经、史、子、集四部,故名四库。共有3500多种书,7.9万卷,3.6万册,约8亿字,基本上囊括了古代所有图书,故称“全书”。<a href="http://www.guoxuedashi.com/SiKuQuanShu/">详细>>
</a>

</div> 
</div>

</div>  <!--end r-->

</div>
<!-- 内容区END --> 

<!-- 页脚开始 -->
<div class="shh">

</div>

<div class="w1180" style="margin-top:8px;">
<center><script src="http://www.guoxuedashi.com/img/plus.php?id=3"></script></center>
</div>
<div class="w1180 foot">
<a href="/b/thanks.php">特别致谢</a> | <a href="javascript:window.external.AddFavorite(document.location.href,document.title);">收藏本站</a> | <a href="#">欢迎投稿</a> | <a href="http://www.guoxuedashi.com/forum/">意见建议</a> | <a href="http://www.guoxuemi.com/">国学迷</a> | <a href="http://www.shuowen.net/">说文网</a><script language="javascript" type="text/javascript" src="https://js.users.51.la/17753172.js"></script><br />
  Copyright &copy; 国学大师 古典图书集成 All Rights Reserved.<br>
  
  <span style="font-size:14px">免责声明:本站非营利性站点,以方便网友为主,仅供学习研究。<br>内容由热心网友提供和网上收集,不保留版权。若侵犯了您的权益,来信即刪。scp168@qq.com</span>
  <br />
ICP证:<a href="http://www.beian.miit.gov.cn/" target="_blank">鲁ICP备19060063号</a></div>
<!-- 页脚END --> 
<script src="http://www.guoxuedashi.com/img/plus.php?id=22"></script>
<script src="http://www.guoxuedashi.com/img/tongji.js"></script>

</body>
</html>
