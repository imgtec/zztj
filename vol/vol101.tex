\chapter{資治通鑑卷一百一}
宋 司馬光 撰

胡三省 音註

晉紀二十三|{
	起上章涒灘盡著雍執徐凡九年}


孝宗穆皇帝下

升平四年春正月癸巳燕主儁大閱于鄴欲使大司馬恪司空陽騖將之入寇|{
	騖音務將卽亮翻}
會疾篤乃召恪騖及司徒評領軍將軍慕輿根等受遺詔輔政甲午卒|{
	年四十二}
戊子太子暐卽皇帝位|{
	暐字景茂儁第三子按長歷是年正月甲戌朔今儁以甲午卒則戊子在甲午前卽位恐是戊戌}
年十一大赦改元建熙 秦王堅分司隸置雍州|{
	雍於用翻}
以河南公雙爲都督雍河涼三州諸軍事征西大將軍雍州刺史改封趙公鎮安定|{
	河涼二州非秦土也雙所督實土惟安定五郡耳爲雙以安定叛張本}
封弟忠爲河南公 仇池公楊俊卒子世立 二月燕人尊可足渾后爲皇太后以太原王恪爲太宰專録朝政|{
	録總也朝直遥翻}
上庸王評爲太傅陽騖爲太保慕輿根爲太師參輔朝政|{
	朝直遥翻下同}
根性木強|{
	師古曰木謂質直強音其兩翻}
自恃先朝勲舊|{
	自皝以來根屢有戰功}
心不服恪舉動倨傲時太后可足渾氏頗預外事根欲爲亂乃言于恪曰今主上幼冲母后干政殿下宜防意外之變思有以自全且定天下者殿下之功也兄亡弟及古今成法|{
	此殷法也非周法也}
俟畢山陵宜廢主上爲王殿下自踐尊位以爲大燕無窮之福恪曰公醉邪何言之悖也|{
	悖蒲内翻又蒲没翻}
吾與公受先帝遺詔云何而遽有此議根愧謝而退恪以告吳王垂垂勸恪誅之恪曰今新遭大喪二鄰觀釁|{
	二鄰謂晉秦也}
而宰輔自相誅夷恐乖遠近之望且可忍之祕書監皇甫真言於恪曰根本庸豎過蒙先帝厚恩引參顧命而小人無識自國哀已來驕狠日甚將成禍亂|{
	狠戶墾翻}
明公今日居周公之地當爲社稷深謀早爲之所恪不聽根又言於可足渾氏及燕主暐曰太宰太傅將謀不軌臣請帥禁兵以誅之|{
	帥讀曰率}
可足渾氏將從之暐曰二公國之親賢先帝選之託以孤嫠|{
	嫠陵之翻無夫曰嫠}
必不肯爾安知非太師欲爲亂也乃止根又思戀東土|{
	龍城在鄴城東北故曰東土}
言於可足渾氏及暐曰今天下蕭條外寇非一國大憂深不如還東恪聞之乃與太傅評謀密奏根罪狀使右衛將軍傅顔就内省誅根并其妻子黨與大赦|{
	既誅根及其妻子黨與恐衆心反側故肆赦以安之}
是時新遭大喪誅夷狼籍内外恟懼|{
	恟許拱翻}
太宰恪舉止如常人不見其有憂色每出入一人步從|{
	從才用翻}
或說以宜自嚴備|{
	說輸芮翻}
恪曰人情方懼當安重以鎮之柰何復自驚擾衆將何仰|{
	復扶又翻}
由是人心稍定恪雖綜大任而朝廷之禮兢兢嚴謹每事必與司徒評議之未嘗專決虛心待士諮詢善道量才授任|{
	量音良}
人不踰位官屬朝臣或有過失|{
	朝直遥翻}
不顯其狀随宜他叙不令失倫|{
	以叙遷爲他官不令失其倫等也}
唯以此爲貶時人以爲大愧莫敢犯者或有小過自相責曰爾復欲望宰公遷官邪|{
	恪爲太宰故稱之爲宰公復扶又翻}
朝廷初聞燕主儁卒皆以爲中原可圖桓温曰慕容恪尚在憂方大耳|{
	史言慕容恪能輔幼主桓温能料敵}
三月己卯葬燕主儁于龍陵|{
	陵在龍城因以爲名}
諡曰景昭皇帝廟號烈祖所徵郡國兵以燕朝多難|{
	難乃旦翻}
互相驚動往往擅自散歸自鄴以南道路斷塞|{
	塞悉則翻}
太宰恪以吳王垂爲使持節征南將軍都督河南諸軍事兖州牧荆州刺史鎮梁國之蠡臺|{
	使疏吏翻下同}
孫希爲并州刺史傅顔爲護軍將軍帥騎二萬觀兵河南臨淮而還境内乃安|{
	史言恪當國有大憂衆心危疑之際處之有方帥讀曰率騎奇寄翻觀古玩翻示之也觀兵曜兵以示之也}
希泳之弟也|{
	孫泳拒趙見九十六卷成帝咸康四年史書孫泳鞠彭宋燭之子弟皆貴顯于燕所以勸委質者能守死而不貳子孫必獲其福也}
匈奴劉衛辰遣使降秦|{
	降戶江翻}
請田内地春來秋返秦王堅許之夏四月雲中護軍賈雍遣司馬徐贇帥騎襲之|{
	贇於倫翻}
大獲而還堅怒曰朕方以恩信懷戎狄而汝貪小利以敗之何也|{
	敗補邁翻}
黜雍以白衣領職遣使還其所獲慰撫之衛辰於是入居塞内貢獻相尋 夏六月代王什翼犍妃慕容氏卒|{
	犍居言翻}
秋七月劉衛辰如代會葬因求婚什翼犍以女妻之|{
	妻七細翻}
八月辛丑朔日有食之既 謝安少有重名|{
	少詩照翻}
前後徵辟皆不就寓居會稽|{
	會工外翻}
以山水文籍自娯雖爲布衣時人皆以公輔期之士大夫至相謂曰安石不出當如蒼生何|{
	謝安字安石江東人士始焉所期望者殷浩浩旣無以滿衆望矣繼而所望者謝安而安卒能匡輔晉室世之論者皆優安而劣浩余謂盛名之下其實難副浩之所以敗正以與桓温齊名其心易温又值石氏之亂以爲可以立功敗于輕率也謝安當桓温擅政之時又身嘗爲之僚屬而懲浩之所以失戒温而爲之備温旣死而值秦之強兢兢焉爲自保之謀常持懼心此其所以濟也史氏謂其能矯情鎮物蓋因屐齒之折白雞之夢而知之耳}
安每遊東山|{
	東山在今紹興府上虞縣西南四十五里安故居今爲國慶禪寺}
常以妓女自随|{
	妓渠綺翻}
司徒昱聞之曰安石既與人同樂|{
	樂音洛}
必不得不與人同憂召之必至安妻劉惔之妹也見家門貴盛|{
	劉惔以清談貴顯而謝尚謝弈謝萬皆爲方伯盛于一時惔徒甘翻}
而安獨静退謂曰丈夫不如此也安掩鼻曰恐不免耳|{
	言恐亦不免如諸兄弟也}
及弟萬廢黜安始有仕進之志時已年四十餘征西大將軍桓温請爲司馬安乃赴召温大喜深禮重之 冬十月烏桓獨孤部鮮卑没弈干各帥衆數萬降秦秦王堅處之塞南|{
	帥讀曰率降戶江翻處昌呂翻下同}
陽平公融諫曰戎狄人面獸心不知仁義其稽顙内附實貪地利非懷德也|{
	稽音啟}
不敢犯邊實憚兵威非感恩也今處之塞内與民雜居彼窺郡縣虛實必爲邊患不如徙之塞外以防未然堅從之 十一月封桓温爲南郡公温弟冲爲豐城縣公子濟爲臨賀縣公 燕太宰恪欲以李績爲右僕射燕主暐不許恪屢以爲請暐曰萬機之事皆委之叔父伯陽一人暐請獨裁出爲章武太守以憂卒|{
	暐不平李績事見上卷上年}


五年春正月戊戌大赦 劉衛辰掠秦邊民五十餘口爲奴婢以獻於秦秦王堅責之使歸所掠衛辰由是叛秦專附於代|{
	史言夷狄反覆難保}
東安簡伯郗曇卒|{
	郗丑之翻曇徒含翻}
二月以東陽太守范汪都督徐兖冀青幽五州諸軍事兼徐兖二州刺史 平陽人舉郡降燕|{
	平陽時屬張平}
燕以建威將軍段剛爲太守遣督護韓苞將兵共守平陽 方士丁進有寵于燕主暐欲求媚于太宰恪說恪令殺太傅評|{
	說輸芮翻}
恪大怒奏收斬之 高昌卒|{
	三年高昌奔滎陽}
燕河内太守呂護并其衆遣使來降拜護冀州刺史護欲引晉兵以襲鄴三月燕太宰恪將兵五萬冠軍將軍皇甫真將兵萬人共討之|{
	將即亮翻冠古玩翻}
燕兵至野王護嬰城自守護軍將軍傅顔請急攻之以省大費恪曰老賊經變多矣觀其守備未易猝攻|{
	易以豉翻}
而多殺士卒頃攻黎陽多殺精銳卒不能拔|{
	事見上卷二年}
自取困辱護内無蓄積外無救援我深溝高壘坐而守之休兵養士離間其黨|{
	間古莧翻}
於我不勞而賊勢日蹙不過十旬取之必矣何爲多殺士卒以求且夕之功乎乃築長圍守之 夏四月桓温以其弟黄門郎豁都督沔中七郡諸軍事|{
	魏置中書監令又置通事郎黄門郎沔中七郡魏興新城上庸襄陽義成竟陵江夏也}
兼新野義城二郡太守|{
	城當作成}
將兵取許昌破燕將慕容塵 涼驃騎大將軍宋混疾甚|{
	驃匹妙翻騎奇寄翻}
張玄靚及其祖母馬氏往省之|{
	靚疾正翻又疾郢翻省悉景翻}
曰將軍萬一不幸寡婦孤兒將何所託欲以林宗繼將軍可乎混曰臣子林宗幼弱不堪大任殿下儻未棄臣門臣弟澄政事愈於臣但恐其儒緩機事不稱耳|{
	凡儒者多務為舒緩而不能應機以趨事赴功稱尺證翻}
殿下策勵而使之可也混戒澄及諸子曰吾家受國大恩當以死報無恃勢位以驕人又見朝臣皆戒之以忠貞|{
	朝直遥翻}
及卒行路爲之揮涕|{
	卒子恤翻爲于僞翻}
玄靚以澄爲領軍將軍輔政 五月丁巳帝崩|{
	年十九}
無嗣皇太后令曰琅邪王丕中興正統|{
	元帝明帝成帝皆正統相傳琅邪王丕成帝長子也故曰中興正統}
義望情地莫與爲比其以王奉大統於是百官備法駕迎于琅邪第庚申卽皇帝位大赦壬戌改封東海王弈爲琅邪王秋七月戊午葬穆帝于永平陵廟號孝宗 燕人圍野王數月呂護遣其將張興出戰傳顔擊斬之城中日蹙皇甫真戒部將曰護勢窮犇突必擇虛隙而投之吾所部士卒多羸器甲不精宜深爲之備乃多課櫓楯親察行夜者|{
	將卽亮翻羸倫爲翻楯食尹翻行下孟翻}
護食盡果夜悉精銳趨真所部|{
	趨七喻翻}
突圍不得出太宰恪引兵擊之護衆死傷殆盡棄妻子犇滎陽恪存撫降民給其廩食|{
	降戶江翻}
徙士人將帥于鄴自餘各随所樂|{
	帥所類翻樂音洛}
以護參軍廣平梁琛爲中書著作郎|{
	晉武帝以祕書并中書省故曰中書著作郎琛丑林翻}
九月戊申立妃王氏爲皇后后濛之女也穆帝何皇后稱穆皇后居永安宫 涼右司馬張邕惡宋澄專政|{
	惡烏路翻下同}
起兵攻澄殺之併滅其族|{
	宋澄豈特機事不稱哉遂赤其族以此知經世非儒緩者所能爲也}
張玄靚以邕爲中護軍叔父天錫爲中領軍同輔政 張平襲燕平陽殺段剛韓苞又攻鴈門殺太守單男|{
	單音善姓也}
旣而爲秦所攻平復謝罪于燕以求救|{
	復扶又翻}
燕人以平反覆弗救也平遂爲秦所滅 乙亥秦大赦 徐兖二州刺史范汪素爲桓温所惡|{
	桓温初以安西鎮上流汪爲上佐蓋惡其異已也若汪於此時能立異必知温之心迹矣惡烏路翻}
温將北伐命汪帥衆出梁國|{
	帥讀曰率}
冬十月坐失期免爲庶人遂廢卒於家|{
	卒子恤翻}
子甯好儒學|{
	好呼到翻}
性質直常謂王弼何晏之罪深于桀紂或以爲貶之太過甯曰王何蔑棄典文幽沈仁義|{
	沈持林翻}
游辭浮說波蕩後生使搢紳之徒翻然改轍以至禮壞樂崩中原傾覆遺風餘俗至今爲患桀紂縱暴一時適足以喪身覆國|{
	喪息浪翻}
爲後世戒豈能迴百姓之視聽哉故吾以爲一世之禍輕歷代之患重自喪之惡小迷衆之罪大也|{
	喪息浪翻}
呂護復叛犇燕燕人赦之以爲廣州刺史|{
	燕無廣州以刺史之名授護耳}
涼張邕驕矜淫縱樹黨專權多所刑殺國人患之張天錫所親敦煌劉肅謂天錫曰|{
	敦徒門翻}
國家事欲未静天錫曰何謂也肅曰今護軍出入有似長寧|{
	長寧侯張祚也}
天錫驚曰我固疑之未敢出口計將安出肅曰正當速除之耳天錫曰安得其人肅曰肅卽其人也肅時年未二十天錫曰汝年少|{
	少詩照翻}
更求其助肅曰趙白駒與肅二人足矣十一月天錫與邕俱入朝|{
	朝直遥翻}
肅與白駒從天錫肅斫之不中|{
	中竹仲翻}
白駒繼之又不克二人與天錫俱入宫中邕得逸走帥甲士三百餘人攻宫門天錫登屋大呼曰|{
	帥讀曰率呼火故翻}
張邕凶逆無道旣滅宋氏又欲傾覆我家汝將士世爲涼臣何忍以兵相向邪|{
	將卽亮翻}
今所取者止張邕耳它無所問於是邕兵悉散走邕自刎死盡滅其族黨|{
	刎扶粉翻}
玄靚以天錫爲使持節冠軍大將軍都督中外諸軍事輔政|{
	自張重華没後張祚張瓘宋混宋澄以及張邕張天錫遁相屠滅涼浸衰矣使疏吏翻冠古玩翻}
十二月始改建興四十九年奉升平年號|{
	涼至是方奉建康年號}
詔以玄靚爲大都督督隴右諸軍事涼州刺史護羌校尉西平公 燕大赦秦王堅命牧伯守宰各舉孝弟廉直文學政事察其所舉得人者賞之非其人者罪之由是人莫敢妄舉而請託不行士皆自勵雖宗室外戚無才能者皆弃不用當是之時内外之官率皆稱職|{
	稱尺證翻}
田疇修闢倉庫充實盜賊屏息|{
	屏必郢翻}
是歲歸義侯李勢卒|{
	永和三年李勢降至是而卒}
哀皇帝|{
	諱丕成帝長子也字千齡咸康八年封琅邪王諡法㳟仁短折曰哀}


隆和元年春正月壬子大赦改元 甲寅减田租畝收二升|{
	成帝咸和五年始度百姓田畝取十分之一率畝稅米三升今减之畝收二升}
燕豫州刺史孫興請攻洛陽曰晉將陳祐弊卒千餘介守孤城不足取也|{
	將卽亮翻介如字獨也又音戞}
燕人從其言遣寧南將軍呂護屯河隂 二月辛未以吳國内史庾希爲北中郎將徐兖二州刺史鎮下邳龍驤將軍袁真爲西中郎將監護豫司并冀四州諸軍事豫州刺史鎮汝南|{
	希真旣並假節職任宜同之矣希亦當帶並護之職史逸之也驤思將翻監工銜翻}
並假節希氷之子也|{
	庾氷秉政于咸康}
丙子拜帝母周貴人爲皇太妃儀服擬于太后燕呂護攻洛陽三月乙酉河南太守戴施犇宛|{
	永和十二}


|{
	年桓温留戴施戍洛陽宛於元翻}
陳祐告急五月丁巳桓温遣庾希及竟陵太守鄧遐帥舟師三千人助祐守洛陽|{
	帥讀曰率}
遐嶽之子也|{
	鄧嶽王敦將也敦敗後自歸著功交廣}
温上疏請遷都洛陽自永嘉之亂播流江表者一切北徙以實河南朝廷畏温不敢爲異而北土蕭條人情疑懼雖並知不可莫敢先諫散騎常侍領著作郎孫綽上疏曰|{
	晉志曰著作郎周左史之任也漢東京圖籍在東觀故使儒者著作東觀有其名尚未有官魏明帝太和中詔置著作郎於此始有其官隸中書省晉惠帝置祕書監併統著作省蓋著作雖别置省而猶隸秘書也余按班固西都賦曰承明金馬著作之廷如是則漢西都雖未置著作之官而承明金馬亦著作之所也散悉亶翻騎奇寄翻}
昔中宗龍飛|{
	元帝廟號中宗}
非惟信順協于天人|{
	易大傳曰天之所助者順也人之所助者信也}
實賴萬里長江畫而守之耳今自喪亂已來六十餘年|{
	自賈后之廢趙王倫之誅繼而諸王交兵胡羯乘之而起天下大亂至是六十餘年矣喪息浪翻}
河洛丘墟函夏蕭條|{
	函容也夏大也言平原之地所函容者大也夏戶雅翻}
士民播流江表已經數世存者老子長孫|{
	長知兩翻}
亡者丘隴成行|{
	行戶剛翻}
雖北風之思感其素心目前之哀實爲交切若遷都旋軫之日|{
	賈公彦曰若不定之辭}
中興五陵卽復緬成遐域|{
	中興五陵元帝建平陵明帝武平陵成帝興平陵康帝崇平陵穆帝永平陵皆在江南緬遠也遐亦遠也}
泰山之安既難以理保|{
	言以理觀之遷都于洛難以保泰山之安也}
烝烝之思豈不纒於聖心哉|{
	烝烝進進也言若遷洛纒心于江南陵寢孝思進進也}
温今此舉誠欲大覽始終爲國遠圖而百姓震駭同懷危懼豈不以反舊之樂賖趨死之憂促哉|{
	樂音洛趨七喻翻}
何者植根江外數十年矣|{
	中原以江南爲江外亦曰江表}
一朝頓欲拔之驅踧於窮荒之地|{
	踧昌六翻}
提挈萬里踰險浮深離墳墓|{
	離力智翻}
棄生業田宅不可復售|{
	復扶又翻}
舟車無從而得捨安樂之國適習亂之鄉將頓仆道塗飄溺江川僅有達者|{
	溺奴狄翻}
此仁者所宜哀矜國家所宜深慮也臣之愚計以爲且宜遣將帥有威名資實者先鎮洛陽|{
	將卽亮翻帥所類翻}
掃平梁許|{
	梁謂梁國許謂許昌皆當江南入洛之要路}
清壹河南運漕之路旣通開墾之積已豐豺狼遠竄中夏小康然後可徐議遷徙耳|{
	戛戶雅翻}
柰何捨百勝之長理舉天下而一擲哉綽楚之孫也|{
	孫楚仕武帝時有才名}
少慕高尚|{
	少詩照翻}
嘗著遂初賦以見志温見綽表不悅曰致意興公|{
	孫綽字興公}
何不尋君遂初賦而知人家國事邪時朝廷憂懼將遣侍中止温揚州刺史王述曰温欲以虛聲威朝廷耳非事實也但從之自無所至乃詔温曰在昔喪亂忽涉五紀|{
	孔頴達曰言在昔者自下本上之辭言昔在者從上自下爲稱喪息浪翻自惠帝永興元年劉淵始亂距是歲五十九年自懷帝永嘉五年洛陽䧟距是歲五十年}
戎狄肆暴繼襲凶迹眷言西顧慨歎盈懷知欲躬帥三軍蕩滌氛穢廓清中畿|{
	中畿王畿也周禮九畿王畿方千里其外侯甸男采衛蠻夷鎮蕃皆以五百里言之王畿在九畿之中故此曰中畿帥讀曰率}
光復舊京非夫外身徇國孰能若此諸所處分|{
	處昌呂翻分扶問翻}
委之高筭但河洛丘墟所營者廣經始之勤致勞懷也事果不行温又議移洛陽鍾虡|{
	虡音巨}
述曰永嘉不競暫都江左方當蕩平區宇旋軫舊京若其不爾宜改遷園陵不應先事鍾虡温乃止朝廷以交廣遼遠|{
	温督荆司雍益梁寧交廣八州}
改授温都督并司冀三州温表辭不受 秦王堅親臨太學考第諸生經義與博士講論自是每月一至焉 六月甲戌燕征東參軍劉拔刺殺征東將軍冀州刺史范陽王友於信都|{
	刺七亦翻}
秋七月呂護退守小平津|{
	以晉援兵至也}
中流矢而卒|{
	中竹仲翻}


燕將段崇收軍北渡屯于野王鄧遐進屯新城|{
	新城春秋戎蠻子之國也自漢以來屬河南隋改爲伊闕縣}
八月西中郎將袁真進屯汝南運米五萬斛以饋洛陽 冬十一月代王什翼犍納女于燕|{
	犍居言翻}
燕人亦以女妻之|{
	妻七細翻}
十二月戊午朔日有食之 庾希自下邳退屯山陽袁真自汝南退屯壽陽|{
	以洛陽兵解退屯而燕兵尋復至矣}


興寧元年春二月己亥大赦改元 三月壬寅皇太妃周氏薨于琅邪第癸卯帝就第治喪|{
	治直之翻}
詔司徒會稽王昱總内外衆務帝欲爲太妃服三年|{
	爲于僞翻}
僕射江虨啟於禮應服緦麻|{
	虨逋閉翻}
又欲降服朞虨曰厭屈私情所以上嚴祖考乃服緦麻|{
	周禮曰王爲諸侯緦縗弁而加環絰又禮爲人後者爲之子故爲所後服斬衰三年而降其父母朞虨以爲應服緦者蓋以帝入後大宗則周氏者琅邪之母當以服諸侯者服之也厭於葉翻嚴尊也}
夏四月燕寧東將軍慕容忠攻滎陽太守劉遠遠犇魯陽五月加征西大將軍桓温侍中大司馬都督中外諸軍録尚書事假黃鉞温以撫軍司馬王坦之爲長史坦之述之子也又以征西掾郗超爲參軍王珣爲主簿每事必與二人謀之府中爲之語曰髯參軍短主簿|{
	以超多髯而珣短也掾于絹翻郗丑之翻}
能令公喜能令公怒|{
	令力呈翻}
温氣槩高邁罕有所推與超言常自謂不能測傾身待之超亦深自結納珣導之孫也與謝玄皆爲温掾温俱重之曰謝掾年四十必擁旄杖節王掾當作黑頭公皆未易才也|{
	易以豉翻}
玄奕之子也|{
	升平二年謝奕卒}
以西中郎將袁真都督司冀并三州諸軍事北中郎將庾希都督青州諸軍事 癸卯燕人拔密城|{
	密縣漢屬河南郡晉屬滎陽郡}
劉遠犇江陵 秋八月有星孛于角亢|{
	角二星亢四星晉天文志角亢氐鄭兖州分孛蒲内翻亢居郎翻}
張玄靚祖母馬氏卒|{
	靚疾正翻又疾郢翻}
尊庶母郭氏爲太妃郭氏以張天錫專政與大臣張欽等謀誅之事泄欽等皆死玄靚懼以位讓天錫天錫不受右將軍劉肅等勸天錫自立閏月天錫使肅等夜帥兵入宫弑玄靚|{
	帥讀曰率下同 考異曰帝紀天錫殺玄靚自立在七月今從晉春秋}
宣言暴卒諡曰冲公天錫自稱使持節大都督大將軍涼州牧西平公|{
	使疏吏翻}
時年十八尊母劉美人曰太妃遣司馬綸騫奉章詣建康請命|{
	綸姓也姓譜曰魏志孫文端臣綸直}
并送御史俞歸東還|{
	穆帝永和三年歸使涼州今乃還}
癸亥大赦 冬十月燕鎮南將軍慕容塵攻陳留太守袁披于長平|{
	長平縣前漢屬汝南郡後漢晉屬陳郡賢曰長平故城在今陳州宛丘縣西北}
汝南太守朱斌乘虛襲許昌克之|{
	考異曰燕書作朱黎今從晉帝紀}
代王什翼犍擊高車大破之|{
	高車卽敕勒也俗乘高輪車故亦號高車部李延壽曰高車蓋古赤狄之餘種也初號爲狄歷北方以爲高車丁零其遷徙随水草衣皮食肉與柔然同唯車輸高大輻數至多犍居言翻}
俘獲萬餘口馬牛羊百餘萬頭 以征虜將軍桓冲爲江州刺史十一月姚襄故將張駿殺江州督護趙毗帥其徒北叛冲討斬之|{
	桓温之破姚襄獲襄將張駿楊凝等徙于尋陽}


二年春正月丙辰燕大赦 二月燕太傅評龍驤將軍李洪略地河南|{
	驤思將翻}
三月庚戍朔大閱戶口令所在土斷|{
	令西北士民僑寓東南者所在以土著爲斷也斷丁亂翻}
嚴其法制謂之庚戌制 帝信方士言斷穀餌藥以求長生|{
	斷讀曰短}
侍中高崧諫曰此非萬乘所宜爲|{
	乘繩證翻}
陛下兹事實日月之食|{
	論語子貢曰君子之過也如日月之食焉}
不聽辛未帝以藥發不能親萬幾禇太后復臨朝攝政|{
	穆帝以幼冲嗣位禇太后臨朝稱制升平元年帝加元服太后歸政帝卽位年長矣以疾不能親政太后復臨朝復扶又翻朝直遥翻}
夏四月甲辰燕李洪攻許昌汝南敗晉兵于懸瓠|{
	敗補邁翻水經注曰懸瓠城汝南郡治也城之西北汝水枝别左出西北流又屈西東轉又西南會汝形如垂瓠因以名城瓠音胡又音互}
潁川太守李福戰死汝南太守朱斌犇壽春陳郡太守朱輔退保彭城大司馬温遣西中郎將袁真等禦之|{
	去年五月加桓温督録假黃鉞至是書其官名而不姓堅冰至矣}
温帥舟師屯合肥|{
	帥讀曰率}
燕人遂拔許昌汝南陳郡徙萬餘戶於幽冀二州遣鎮南將軍慕容塵屯許昌 五月戊辰以揚州刺吏王述爲尚書令加大司馬温揚州牧録尚書事壬申使侍中召温入參朝政温辭不至王述每受職不爲虛讓其所辭必於不受及爲尚書令子坦之白述故事當讓述曰汝謂我不堪邪坦之曰非也但克讓自美事耳述曰旣謂堪之何爲復讓|{
	復扶又翻}
人言汝勝我定不及也 六月秦王堅遣大鴻臚拜張天錫爲使持節爲大將軍涼州牧西平公秋七月丁卯詔復徵大司馬温入朝八月温至赭圻

詔尚書車灌止之温遂城赭圻居之|{
	赭圻在宣城界南史沈攸之自虎檻洲進攻赭圻陶亮等自鵲頭引兵救之劉昫曰宣州南陵縣漢春穀縣地梁置南陵縣舊治赭圻城唐長安四年移治青陽城按温表云春穀縣之赭圻城在江東岸臨當濡須口上二十里距建康宫三百二十里南有聲里比有高安戌車昌遮翻圻渠希翻}
固讓内録|{
	内録謂録尚書事也}
遥領揚州牧秦汝南公騰謀反伏誅騰秦主生之弟也是時生弟晉公柳等猶有五人王猛言于堅曰不去五公終必爲患堅不從|{
	爲後柳等反張本去羌呂翻}
燕侍中慕輿龍詣龍城徙宗廟及所留百官皆詣鄴 燕太宰恪將取洛陽 |{
	考異曰帝紀慕容暐寇洛陽上云苻堅别帥侵河南按明年恪拔洛陽堅親將以備潼關是未敢與燕争河南也十六國春秋堅傳亦無此舉帝紀恐誤}
先遣人招納土民遠近諸塢皆歸之乃使司馬悅希軍于盟津|{
	盟讀曰孟}
豫州刺史孫興軍于成臯初沈充之子勁以其父死于逆亂|{
	見九十三卷明帝太寧二年}
志欲立功以雪舊恥年三十餘以刑家不得仕吳興太守王胡之爲司州刺史上疏稱勁才行|{
	行下孟翻}
請解禁錮參其府事朝廷許之會胡之以病不行及燕人逼洛陽冠軍將軍陳祐守之|{
	冠古玩翻}
衆不過二千勁自表求配祐効力詔以勁補冠軍長史令自募壯士得千餘人以行勁屢以少擊燕衆摧破之|{
	少詩沼翻}
而洛陽糧盡援絶祐自度不能守|{
	度徒洛翻}
乃以救許昌爲名九月留勁以五百人守洛陽祐帥衆而東|{
	帥讀曰率}
勁喜曰吾志欲致命|{
	論語子張曰士見危致命朱子曰致命謂委致其命猶言授命也}
今得之矣祐聞許昌已没遂犇新城燕悅希引兵略河南諸城盡取之 秦王堅命公國各置三卿|{
	晉制王國置郎中令中尉大農爲三卿秦因其制}
并餘官皆聽自采辟獨爲置郎中令|{
	爲于僞翻}
富商趙掇等車服僭侈諸公競引以爲卿|{
	掇陟劣翻又都活翻}
黃門侍郎安定程憲請治之|{
	治直之翻}
堅乃下詔稱本欲使諸公延選英儒乃更猥濫如是宜令有司推檢辟召非其人者悉降爵爲侯自今國官皆委之銓衡|{
	銓衡謂吏部尚書也}
自非命士已上不得乘車馬去京師百里内工商皁隸不得服金銀錦繡犯者棄市於是平陽平昌九江陳留安樂五公皆降爵爲侯|{
	樂音洛}
三年春正月庚申皇后王氏崩 劉衛辰復叛代|{
	劉衛辰附代見上升平五年復扶又翻}
代王什翼犍東渡河擊走之|{
	犍居言翻}
什翼犍性寛厚郎中令許謙盜絹二匹什翼犍知而匿之|{
	按北史代國俗無繒帛而謙盜之其罪在不赦而什翼犍能容之故史以此言其寛厚之一端}
謂左長史燕鳳曰吾不忍視謙之面若謙慙而自殺是吾以財殺士也嘗討西部叛者流矢中目|{
	中竹仲翻}
旣而獲射者羣臣欲臠割之什翼犍曰彼各爲其主鬬耳|{
	爲于僞翻}
何罪遂釋之 大司馬温移鎮姑孰|{
	温又自赭圻而東鎮姑孰}
二月乙未以其弟右將軍豁監荆州揚州之義城雍州之京兆諸軍事領荆州刺史|{
	義城郡置于襄陽襄陽郡屬荆州而義城郡領揚州淮南之平阿下蔡蓋桓宣先從祖約退屯淮南後鎮襄陽陶侃以其淮南部曲置義成郡於穀城蓋有揚州之民而又置揚州僑縣於穀城穀城荆州統内之地也故曰荆州揚州之義成曰義成者言以義成軍因而名郡後人又於成字旁添土失其初立郡之旨矣京兆郡屬雍州時亦僑立於襄陽雍於用翻}
加江州刺史桓沖監江州及荆豫八郡諸軍事|{
	初沖刺江州領西陽譙二郡太守今加監荆州之江夏随郡豫州之汝南西陽新蔡潁川凡六郡通所鎮尋陽爲八郡監工銜翻 考異曰帝紀云沖領南蠻校尉按江左唯荆州領南蠻沖傳亦無蓋紀因桓豁重出今不取}
並假節司徒昱聞陳祐棄洛陽會大司馬温于冽洲|{
	今姑孰江中有冽山卽其地}
共議征討丙申帝崩于西堂|{
	年二十五西堂太極殿西堂也建康太極殿有東西堂東堂以見羣臣西堂爲卽安之地}
事遂寢帝無嗣丁酉皇太后詔以琅邪王奕承大統|{
	奕當作弈}
百官奉迎于琅邪第是日卽皇帝位大赦 秦大赦改元建元 燕太宰恪吳王垂共攻洛陽恪謂諸將曰卿等常患吾不攻今洛陽城高而兵弱易克也|{
	易以䜴翻}
勿更畏懦而怠惰遂攻之三月克之執揚武將軍沈勁勁神氣自若恪將宥之中軍將軍慕輿䖍曰勁雖奇士觀其志度終不爲人用今赦之必爲後患遂殺之恪略地至崤澠|{
	崤崤谷也澠澠池也澠彌兖翻}
關中大震秦王堅自將屯陝城以備之|{
	將卽亮翻陝式冉翻}
燕人以左中郎將慕容筑爲洛州刺史鎮金墉|{
	筑張六翻}
吳王垂爲都督荆揚洛徐兖豫雍益涼秦十州諸軍事征南大將軍荆州牧配兵一萬鎮魯陽|{
	雍於用翻}
太宰恪還鄴謂僚屬曰吾前平廣固不能濟辟閭蔚|{
	見上卷穆帝永和十二年}
今定洛陽使沈勁爲戮雖皆非本情然身爲元帥實有愧于四海|{
	帥讀曰率}
朝廷嘉勁之忠贈東陽太守

臣光曰沈勁可謂能子矣恥父之惡致死以滌之變凶逆之族爲忠義之門易曰幹父之蠱用譽|{
	易蠱卦六五爻辭象曰幹父用譽承以德也}
蔡仲之命曰爾尚蓋前人之愆惟忠惟孝|{
	見尚書}
其是之謂乎

太宰恪爲將不事威嚴專用恩信撫士卒務綜大要不爲苛令使人人得便安平時營中寛縱似若可犯然警備嚴密敵至莫能近者|{
	近其靳翻}
故未嘗負敗 壬申葬哀帝及静皇后于安平陵|{
	王皇后諡曰静晉書作靖}
夏四月壬午燕太尉武平匡公封弈卒|{
	諡法貞心大度曰匡}
以司空陽騖爲太尉侍中光禄大夫皇甫真爲司空領中書監騖歷事四朝|{
	廆皝儁暐四朝朝直遥翻}
年耆望重自太宰恪以下皆拜之而騖謙恭謹厚過於少時戒束子孫雖朱紫羅列無敢違犯其法度者|{
	封弈事燕亦歷事四朝其宣勞過于陽騖子孫貴顯亦遏于陽氏豈弈之謙德有愧於騖邪或者史家因陽氏家傳書之而封氏闕然無述也少詩照翻}
六月戊子益州刺史建城襄公周撫卒|{
	諡法因事有功曰襄}
撫在益州三十餘年|{
	穆帝永和三年桓温平蜀留撫鎮之至是纔十九年蓋晉未得蜀之前置益州刺史于巴東撫先已爲刺史温既克蜀撫仍爲益州刺史鎮彭模曰在益州三十餘年者史通其鎮巴東鎮彭模之年數之也}
甚有威惠詔以其子犍爲太守楚代之|{
	犍居言翻}
秋七月己酉徙會稽王昱復爲琅邪王|{
	元帝以昱爲琅邪王奉恭王祀成帝咸和元年王生母鄭夫人薨王號慕請服重徙封會稽王是後康帝哀帝及今帝皆自琅邪入繼大統會工外翻}
壬子立妃庾氏爲皇后后冰之女也 甲申立琅邪王昱子昌明爲會稽王昱固讓猶自稱會稽王|{
	會工外翻}
匈奴右賢王曹轂左賢王劉衛辰皆叛秦轂帥衆二萬寇杏城秦王堅自將討之|{
	轂古禄翻帥讀曰率將卽亮翻}
使衛大將軍李威左僕射王猛輔太子宏留守長安八月堅擊轂破之斬轂弟活轂請降|{
	降戶江翻}
徙其豪傑六千餘戶于長安建節將軍鄧羌討衛辰擒之於木根山|{
	木根山在朔方}
九月堅如朔方廵撫諸胡冬十月征北將軍淮南公幼帥杏城之衆乘虛襲長安李威擊斬之|{
	幼亦秦主生之弟也}
鮮卑秃髪椎斤卒年一百一十子思復鞬代統其衆|{
	鞬居言翻}
椎斤樹機能從弟務丸之孫也|{
	樹機能亂涼州見晉武帝紀從才用翻}
梁州刺史司馬勲爲政酷暴治中别駕及州之豪右言語忤意卽於坐梟斬之|{
	忤五故翻坐徂臥翻梟堅堯翻}
或親射殺之|{
	射而亦翻}
常有據蜀之志憚周撫不敢發及撫卒勲遂舉兵反别駕雍端西戎司馬隗粹切諫|{
	西戎司馬西戎校尉之屬官也雍於用翻隗五罪翻}
勲皆殺之自號梁益二州牧成都王十一月勲引兵入劔閣攻涪西夷校尉母丘暐棄城走|{
	晉初置西夷校尉治汶山今蓋治涪城涪音浮}
乙卯圍益州刺史周楚于成都大司馬温表鷹揚將軍江夏相義陽朱序爲征討都護以救之|{
	夏戶雅翻相息亮翻}
秦王堅還長安以李威守太尉加侍中以曹轂爲鴈門公劉衛辰爲夏陽公|{
	夏戶雅翻}
各使統其部落 十二月戊戌以尚書王彪之爲僕射

海西公上|{
	諱弈字延齡哀帝之母弟也咸康八年封爲東海王穆帝升平五年改封琅邪王卽位後桓温廢爲海西公}


太和元年春三月荆州刺史桓豁使督護桓羆攻南鄭討司馬勲 燕太宰大司馬恪太傅司徒評稽首歸政上章綬請歸第|{
	稽音啟上時掌翻}
燕主暐不許 夏五月戊寅皇后庾氏崩 朱序周楚擊司馬勲破之擒勲及其黨送大司馬温温皆斬之傳首建康 代王什翼犍遣左長史燕鳳入貢于秦|{
	犍居言翻燕於賢翻}
秋七月癸酉葬孝皇后于敬平陵|{
	庾后諡曰孝}
秦輔國將軍王猛前將軍楊安揚武將軍姚萇等帥衆二萬宼荆州攻南鄉郡|{
	萇仲良翻帥讀曰率}
荆州刺史桓豁救之八月軍于新野秦兵掠安陽民萬餘戶而還|{
	安陽縣漢屬漢中郡魏置魏興郡安陽屬焉晉省秦攻南鄉而退安能深入山阻掠安陽之民乎載記作漢陽謂漢水之北也當從載記爲是}
九月甲午曲赦梁益二州|{
	司馬勲初平赦其支黨及脅從者}
冬十月加司徒昱丞相録尚書事入朝不趨讃拜不名劍履上殿|{
	朝直遥翻上時掌翻}
張天錫遣使至秦境上告絶於秦|{
	涼與秦通見上卷穆帝永和十二年使疏吏翻}
燕撫軍將軍下邳王厲寇兖州拔魯高平數郡置守宰而還 初隴西李儼以郡降秦旣而復通于張天錫|{
	李儼據隴西事始上卷永和十一年降戶江翻復扶又翻}
十二月羌歛岐以略陽四千家叛秦稱臣于儼|{
	載記作歛岐張天錫傳作廉岐歛羌姓也}
儼於是拜置牧守與秦涼絶 南陽督護趙億據宛城降燕太守桓澹走保新野燕人遣南中郎將趙盤自魯陽戌宛|{
	宛於元翻}
徐兖二州刺史庾希以后族故兄弟貴顯大司馬温忌之二年春正月庾希坐不能救魯高平免官 |{
	考異曰帝紀是月希有罪走入海按本傳海西廢後希始逃于海陵此時才坐免官耳}
二月燕撫軍將軍下邳王厲鎮北將軍宜都王桓襲敕勒 秦輔國將軍王猛隴西太守姜衡南安太守南安邵羌揚武將軍姚萇等帥衆萬七千討歛岐三月張天錫遣前將軍楊遹向金城征東將軍常據向左南|{
	張軌置左南縣屬晉興郡闞駰十三州志曰石城西一百四十里有左南城河水逕其南曰左南津遹音聿}
游擊將軍張統向白土|{
	晉志白土縣屬金城郡十三州志左南津西六十里有白土城城在大河之北爲緣河濟渡之地}
天錫自將三萬人屯倉松|{
	倉松縣自漢以來屬武威郡後梁呂光改曰昌松縣將卽亮翻}
以討李儼歛岐部落先屬姚弋仲聞姚萇至皆降王猛遂克略陽歛岐奔白馬|{
	白馬卽武都白馬氐之地}
秦王堅以萇爲隴東太守 夏四月燕慕容塵寇竟陵太守羅崇擊破之 張天錫攻李儼大夏武始二郡下之|{
	宋白曰張駿十八年分武始興晉廣武置大夏郡唐爲大夏縣屬河州張駿以狄道縣置武始郡今熙州卽其地夏戶雅翻}
常據敗儼兵於葵谷|{
	敗補邁翻}
天錫進屯左南儼懼退守枹罕|{
	枹音膚}
遣其兄子純謝罪於秦且請救秦王堅使前將軍楊安建威將軍王撫帥騎二萬會王猛以救儼猛遣邵羌追歛岐王撫守侯和姜衡守白石|{
	白石縣前漢屬金城郡後漢屬隴西郡賢曰白石山在今蘭州宋白曰河州鳳林縣本漢白石縣地張駿八年改爲永固縣}
猛與楊安救枹罕天錫遣楊遹逆戰于枹罕東猛大破之俘斬萬七千級與天錫相持於城下|{
	枹罕城下也}
邵羌禽歛岐於白馬送之猛遺天錫書曰|{
	遺于季翻}
吾受詔救儼不令與涼州戰今當深壁高壘以聽後詔曠日持久恐二家俱弊非良筭也|{
	二家謂秦涼也}
若將軍退舍吾執儼而東將軍徙民西旋不亦可乎天錫謂諸將曰猛書如此吾本來伐叛不來與秦戰遂引兵歸李儼猶未納秦師王猛白服乘輿從者數十人|{
	從才用翻}
請與儼相見儼開門延之未及爲備將士繼入遂執儼以立忠將軍彭越爲平西將軍涼州刺史鎮枹罕|{
	立忠將軍苻秦所創置}
張天錫之西歸也李儼將賀肫說儼曰以明公神武將士驍悍柰何束手於人王猛孤軍遠來士卒疲弊且以我請救必不設備若乘其怠而擊之可以得志儼曰求救于人以免難難旣免而擊之天下其謂我何不若固守以老之彼將自退猛責儼以不卽出迎儼以賀肫之謀告猛斬肫|{
	肫株倫翻又音豚說輸芮翻難乃旦翻}
以儼歸至長安堅以儼爲光禄勲賜爵歸安侯 燕太原桓王恪言于燕主暐曰|{
	諡法辟土服遠曰桓}
吳王垂將相之才十倍于臣先帝以長幼之次|{
	長知兩翻}
故臣得先之|{
	得先悉薦翻}
臣死之後願陛下舉國以聽吳王五月壬辰恪疾篤暐親視之問以後事恪曰臣聞報恩莫大於薦賢賢者雖在板築猶可爲相|{
	謂殷王高宗起傅說於板築之間命以爲相}
况至親乎吳王文武兼資管蕭之亞|{
	謂才亞於管仲蕭何也}
陛下若任以大政國家可安不然秦晉必有窺窬之計言終而卒|{
	窬音俞卒子恤翻}
秦王堅聞恪卒隂有圖燕之計欲覘其可否|{
	覘丑廉翻又丑艷翻}
命匈奴曹轂發使如燕朝貢|{
	曹轂匈奴右賢王也前年降於秦朝直遥翻}
以西戎主簿郭辯爲之副|{
	晉武帝置西戎校尉於長安秦蓋因之主簿其屬也 考異曰燕建熙八年皇甫真爲太尉燕書及載記真傳郭辯至燕皆在真爲太尉下晉春秋在建熙十年八月恐皆非是故附於曹轂降秦下}
燕司空皇甫真兄腆及從子奮覆皆仕秦腆爲散騎常侍|{
	皇甫真本安定人仕於燕從才用翻散悉亶翻騎奇寄翻}
辯至燕歷造公卿|{
	造七到翻}
謂真曰僕本秦人家爲秦所誅故寄命曹主貴兄常侍及奮覆兄弟並相知有素真怒曰臣無境外之交此言何以及我君似奸人得無因緣假託乎白暐請窮治之|{
	治直之翻}
太傅評不許辯還爲堅言燕朝政無綱紀實可圖也|{
	爲于僞翻朝直遥翻}
鑒機識變唯皇甫真耳堅曰以六州之衆|{
	六州幽井冀司兖豫也}
豈得不使有智士一人哉曹轂尋卒秦分其部落爲二使其二子分統之號東西曹|{
	堅分轂部落貳城以西二萬餘落使轂長子璽統之貳城以東二萬餘落使轂小子寅統之}
荆州刺史桓豁竟陵太守羅崇攻宛拔之趙億走趙盤退歸魯陽豁追擊盤於雉城擒之|{
	雉縣自漢以來屬南陽郡其地當在唐鄧州向城縣界新唐志曰向城縣北八十里有魯陽關}
留兵戌宛而還|{
	還從宣翻又如字}
秋七月燕下邳王厲等破敕勒獲馬牛數萬頭初厲兵過代地犯其穄田|{
	穄子例翻也今南人呼黍爲穄北方地寒五穀不生惟黍生之故有穄田項安世曰黍有二種正黍似粟而大以五月熟今荆人專謂之黍又謂之黍穄是也又一種尤高大稈之狀至如蘆實之狀至如薏苡荆人謂之討黍又謂之蘆穄然以秋而熟非正黍也}
代王什翼犍怒|{
	犍居言翻}
燕平北將軍武強公埿以幽州兵戌雲中八月什翼犍攻雲中埿棄城走|{
	埿與泥同}
振威將軍慕輿賀辛戰没 九月以會稽内史郗愔爲都督徐兖青幽揚州之晉陵諸軍事徐兖二州刺史鎮京口|{
	郗丑之翻愔揖淫翻沈約曰晉永嘉大亂幽冀青并兖州及徐州之淮北流民相率過淮亦有過江在晉陵界者成帝咸和四年郗鑒又徙流民之在淮南者於晉陵諸縣其徙過江南及留在江北者並立僑郡縣以司牧之徐兖二州或治江北江北又僑立幽冀青并四州}
秦淮南公幼之反也征東大將軍并州牧晉公柳征西大將軍秦州刺史趙公雙皆與之通謀秦王堅以雙母弟至親柳健之愛子隱而不問柳雙復與鎮東將軍洛州刺史魏公廋安西將軍雍州刺史燕公武謀作亂|{
	復扶又翻秦并州刺史治蒲阪秦州刺史治上邽洛州刺史治陝雍州刺史治安定廋武皆健于也廋疎鳩翻}
鎮東主簿南安姚眺|{
	眺他弔翻}
諫曰明公以周邵之親受方面之任國家有難當竭力除之况自爲難乎|{
	難乃旦翻}
廋不聽堅聞之徵柳等詣長安冬十月柳據蒲阪雙據上邽廋據陝城武據安定皆舉兵反|{
	果如王猛之言}
堅遣使諭之曰吾待卿等恩亦至矣何苦而反今止不徵卿宜罷兵各定其位一切如故各齧棃以爲信皆不從|{
	棃肉脆而齧之易入以喻親戚離叛則國力脆弱將爲敵人所乘故齧棃付使者賜柳等以爲信也使疏吏翻齧魚結翻}
代王什翼犍擊劉衛辰河冰未合什翼犍命以葦絙約流澌俄而冰合|{
	自代擊朔方西渡大河其津曰君子津經居登翻}
然猶未堅乃散葦於其上冰草相結有如浮梁代兵乘之以渡衛辰不意兵猝至與宗族西走什翼犍收其部落什六七而還衛辰犇秦秦王堅送衛辰還朔方遣兵戌之 十二月甲子燕太尉建寧敬公陽騖卒|{
	諡法合善典法曰敬夙夜警戒曰敬}
以司空皇甫真爲侍中太尉光禄大夫李洪爲司空

三年春正月秦王堅遣後將軍楊成世左將軍毛嵩分討上邽安定輔國將軍王猛建節將軍鄧羌攻蒲阪前將軍楊安廣武將軍張蚝攻陝城堅命蒲陝之軍皆距城三十里堅壁勿戰俟秦雍已平然後并力取之|{
	陝式冉翻雍於用翻}
初燕太宰恪有疾以燕主暐幼弱政不在已太傅評多猜忌恐大司馬之任不當其人謂暐兄樂安王臧曰今南有遺晉西有彊秦二國常蓄進取之志顧我未有隙耳夫國之興衰繫于輔相大司馬總統六軍不可任非其人我死之後以親疎言之當在汝及冲汝曹雖才識明敏然年少未堪多難|{
	少詩照翻難乃旦翻}
吳王天資英傑智略超世汝曹若能推大司馬以授之必能混壹四海况外寇不足憚也慎無冒利而忘害不以國家爲意也|{
	冒利而忘害者謂利在於得兵權而冒當大司馬之任而忘亡國敗家之害也}
又以語太傅評|{
	語牛倨翻}
及恪卒評不用其言二月以車騎將軍中山王冲爲大司馬冲暐之弟也以荆州刺史吳王垂爲侍中車騎大將軍儀同三司|{
	爲評垂有隙張本騎奇寄翻}
秦魏公廋以陝城降燕請兵應接秦人大懼盛兵守華隂|{
	華隂縣在陝城之西有潼關之險降戶江翻華戶化翻}
燕魏尹范陽王德|{
	燕都鄴以魏郡太守爲魏尹}
上疏以爲先帝應天受命志平六合陛下纂統當繼而成之今苻氏骨肉乖離國分爲五|{
	蒲阪陵城上邽安定與長安爲五}
投誠請援前後相尋是天以秦賜燕也天與不取反受其殃吳越之事足以觀矣|{
	國語越范蠡曰昔天以越賜吳吳不敢取今天以吳賜越越其敢逆天乎}
宜命皇甫真引并冀之衆徑趨蒲阪|{
	趨七喻翻}
吳王垂引許洛之兵馳解廋圍太傅總京師虎旅爲二軍後繼傳檄三輔示以禍福明立購賞彼必望風響應渾壹之期于此乎在矣時燕人多請救陝因圖關中者太傅評曰秦大國也今雖有難未易可圖|{
	難乃旦翻易以豉翻}
朝廷雖明未如先帝|{
	燕人謂其主爲朝廷}
吾等智畧又非太宰之比但能閉關保境足矣平秦非吾事也魏公廋遺吳王垂及皇甫真牋曰|{
	遺于季翻}
苻堅王猛皆人傑也謀爲燕患久矣今不乘機取之恐異日燕之君臣將有甬東之悔矣|{
	左傳吳入越越子保于會稽使行成於吳吳子許之伍子胥諫不聼其後越入吳請使吳王居甬東吳王曰孤老矣不能事君王也吾悔不用子胥之言自令陷此乃縊賈逵曰甬東越鄙甬江東也索隱曰今鄮縣卽其處甬余隴翻}
垂謂真曰方今爲人患者必在於秦主上富於春秋觀太傅識度豈能敵苻堅王猛乎真曰然吾雖知之如言不用何 三月丁巳朔日有食之 癸亥大赦 秦楊成世爲趙公雙將苟興所敗毛嵩亦爲燕公武所敗犇還秦王堅復遣武衛將軍王鑒寧朔將軍呂光將軍馮翊郭將翟傉等帥衆三萬討之|{
	敗補邁翻復扶又翻傉奴沃翻}
夏四月雙武乘勝至于榆眉以苟興爲前鋒王鑒欲速戰呂光曰興新得志氣勢方銳宜持重以待之彼糧盡必退退而擊之蔑不濟矣二旬而興退光曰興可擊矣遂追之興敗因擊雙武大破之斬獲萬五千級武棄安定與雙皆犇上邽鑒等進攻之晉公柳數出挑戰|{
	數所角翻挑徒了翻}
王猛不應柳以猛爲畏之五月留其世子良守蒲阪帥衆二萬西趨長安去蒲阪百餘里鄧羌帥精騎七千夜襲敗之|{
	帥讀曰率趨七喻翻敗補邁翻}
柳引軍還猛邀擊之盡俘其衆柳與數百騎入城猛羌進攻之秋七月王鑒等拔上邽斬雙武宥其妻子以左衛將軍苻雅爲秦州刺史八月以長樂公丕爲雍州刺史|{
	樂音洛雍於用翻}
九月王猛等拔蒲阪斬晉公柳及其妻子猛屯蒲阪遣鄧羌與王鑒等會攻陝城燕王公貴戚多占民爲䕃戶|{
	占之贍翻晉制官品自第一至第九各以貴賤}


|{
	占田有差而又各以品之高卑䕃其親屬多者及九族少者三世宗室國賓先賢之後及士人子孫亦如之而又得䕃人以爲衣食客及佃客}
國之戶口少於私家|{
	少所沼翻}
倉庫空竭用度不足尚書左僕射廣信公悅綰曰今三方鼎峙|{
	三方謂燕晉秦也}
各有吞併之心而國家政法不立豪貴恣横|{
	横戶孟翻}
至使民戶殫盡委輸無入|{
	委於僞翻輸書遇翻}
吏斷常俸戰士絶廩官貸粟帛以自贍給旣不可聞於隣敵且非所以爲治|{
	治直吏翻}
宜一切罷斷諸䕃戶盡還郡縣|{
	罷斷丁管翻}
燕主暐從之使綰專治其事糾擿姦伏|{
	擿他歷翻}
無敢蔽匿出戶二十餘萬舉朝怨怒|{
	朝直遥翻}
綰先有疾自力釐校戶籍疾遂亟冬十一月卒 十二月秦王猛等拔陝城獲魏公廋送長安秦王堅問其所以反對曰臣本無反心但以弟兄屢謀逆亂臣懼并死故謀反耳堅泣曰汝素長者固知非汝心也且高祖不可以無後|{
	苻健廟號高祖}
乃賜廋死原其七子以長子襲魏公餘子皆封縣公以嗣越厲王及諸弟之無後者|{
	苻生廢爲越王諡曰厲}
苟太后曰廋與雙俱反雙獨不得置後何也堅曰天下者高祖之天下高祖之子不可以無後至于仲羣不顧太后謀危宗廟|{
	苻雙字仲羣}
天下之法不可私也以范陽公抑爲征東大將軍并州刺史鎮蒲阪鄧羌爲建武將軍洛州刺史鎮陝城擢姚眺爲汲郡太守 加大司馬温殊禮位在諸侯王上 是歲以仇池公楊世爲秦州刺史世弟統爲武都太守世亦稱臣于秦秦以世爲南秦州刺史

資治通鑑卷一百一
