










 


 
 


 

  
  
  
  
  





  
  
  
  
  
 
  

  

  
  
  



  

 
 

  
   




  

  
  


    資治通鑑卷二百四十六 宋 司馬光 撰

  胡三省 音註

  唐紀六十二【起著雍敦牂盡玄黓閹茂凡五年}


  文宗元聖昭獻孝皇帝下

  開成三年春正月甲子李石入朝中塗有盜射之【射食亦翻}
微傷左右奔散石馬驚馳歸第又有盜邀擊于坊門斷其馬尾【唐諸坊之南皆有門以時啓閉斷音短}
僅而得免上聞之大驚命神策六軍遣兵防衛敕中外捕盜甚急竟無所獲乙丑百官入朝者九人而已京城數日方安 丁卯追贈故齊王湊為懷懿太子【知湊之寃也湊被枉事見二百二十四卷太和五年}
 戊申以鹽鐵轉運使戶部尚書楊嗣復戶部侍郎判戶部李珏並同平章事 【考異曰舊傳三年楊嗣復輔政薦珏以本官同平章事按珏與嗣復並命今從實錄}
判使如故【判謂判戶部使謂鹽鐵轉運使}
嗣復於陵之子也【楊於陵見二百三十七卷憲宗元和三年於音烏}
 中書侍郎同平章事李石承甘露之亂人情危懼宦官恣横【横戶孟翻}
忘身徇國故紀綱粗立仇士良深惡之【粗坐五翻惡烏路翻下同}
潛遣盜殺之不果石懼累表稱疾辭位上深知其故而無如之何丙子以石同平章事充荆南節度使 陳夷行性介直惡楊嗣復為人每議政事多相詆斥壬辰夷行以足疾辭位不許上命起居舍人魏謩獻其祖文貞公笏【魏徵諡曰文貞}
鄭覃曰在人不在笏上曰亦甘棠之比也【言周人思召公愛其甘棠而不敢翦伐今思魏徵之正直則亦當寶愛其故笏}
 楊嗣復欲援進李宗閔【復扶又翻援于元翻下同}
恐為鄭覃所沮乃先令宦官諷上上臨朝謂宰相曰宗閔積年在外宜與一官【李宗閔貶見上卷太和九年}
鄭覃曰陛下若憐宗閔之遠止可移近北數百里【近其靳翻}
不宜再用用之臣請先避位陳夷行曰宗閔曏以朋黨亂政陛下何愛此纖人【纖人猶言小人也}
楊嗣復曰事貴得中不可但徇愛憎上曰可與一州覃曰與州太優止可洪州司馬耳【洪州京師東南三千九十里}
因與嗣復互相詆訐以為黨【訐居謁翻}
上曰與一州無傷覃等退上謂起居郎周敬復舍人魏謩曰宰相諠爭如此可乎【唐制起居郎起居舍人掌録天子起居法度天子御正殿則郎居左舍人居右有命俯陛以聽每仗下天子與宰相議政事郎舍人亦分侍左右若仗在紫宸内閣則夾香案分立殿下覃等諠爭既退故上因問之}
對曰誠為不可然覃等盡忠憤激不自覺耳丁酉以衡州司馬李宗閔為杭州刺史【唐制衡州中洪州上都督府杭州上中州司馬從五品下大都督府司馬從四品下上州刺史從三品}
李固言與楊嗣復李珏善故引居大政以排鄭覃陳夷行每議政之際是非鋒起上不能决也【史言文宗明不足以燭理}
 三月牂柯寇涪州清溪鎮【牂柯蠻在涪州東九百里東距辰州二千四百里涪音浮}
鎮兵擊却之初太和之末杜悰為鳳翔節度使有詔沙汰僧尼【事見}


  【上卷太和八年}
時有五色雲見于岐山【見賢遍翻下同}
近法門寺民間訛言佛骨降祥【佛骨在法門寺故云然}
以僧尼不安之故監軍欲奏之悰曰雲物變色何常之有佛若果愛僧尼當見于京師未幾獲白兔【幾居豈翻未幾言未得幾何時也}
監軍又欲奏之曰此西方之瑞也悰曰野獸未馴且宜畜之【馴松倫翻畜吁玉翻}
旬日而斃監軍不悦以為掩蔽聖德獨畫圖獻之及鄭注代悰鎮鳳翔【按通鑑上卷太和八年九月庚申以鳳翔節度使李聽為忠武節度使代杜悰丁卯以鄭注為鳳翔節度使注誣奏聽在鳳翔貪虐冬十月乙亥以聽為太子太保分司復以杜悰為忠武節度使若如上卷所書則杜悰鎮忠武不在鳳翔}
奏紫雲見又獻白雉是歲八月有甘露降于紫宸殿前櫻桃之上上親采而嘗之百官稱賀其十一月遂有金吾甘露之變及悰為工部尚書判度支河中奏騶虞見【詩注騶虞義獸白虎黑文不食生物有至信之德則應之司馬相如封禪書曰般般之獸樂我君囿白質黑章其儀可喜師古注謂騶虞也山海經騶虞如虎五色尾長于身}
百官稱賀上謂悰曰李訓鄭注皆因瑞以售其亂乃知瑞物非國之慶卿前在鳳翔不奏白兔真先覺也對曰昔河出圖伏羲以畫八卦洛出書大禹以敘九疇皆有益於人故足尚也至于禽獸草木之瑞何時無之劉聰桀逆黄龍三見石季龍暴虐得蒼麟十六白鹿七以駕芝蓋【石虎字季龍唐避廟諱故稱其字}
以是觀之瑞豈在德玄宗嘗為潞州别駕【中宗時玄宗為潞州别駕}
及即位潞州奏十九瑞玄宗曰朕在潞州惟知勤職業此等瑞物皆不知也願陛下專以百姓富安為國慶自餘不足取也上善之他日謂宰相曰時和年豐是為上瑞嘉禾靈芝誠何益於事宰相因言春秋記災異以儆人君而不書祥瑞用此故也【意此必鄭覃之言}
夏五月乙亥詔諸道有瑞皆無得以聞亦勿申牒所司其臘饗太廟【唐制四孟及臘享于太廟唐臘用寅}
及饗太清宫【玄宗天寶二年以西京玄元皇帝廟為太清宫}
元日受朝奏祥瑞皆停【六典凡大祥瑞隨即表奏文武百寮詣闕奉賀其它並年終具表以聞有司告廟百寮詣闕奉賀又儀制令大瑞即隨表奏聞中瑞下瑞申報有司元日聞奏今皆停罷 考異曰實録初上謂宰臣曰歲豐人安豈非上瑞宰臣因言春秋不書祥瑞上深然之遂有此詔補國史以為因杜悰進言今兼取之}
 初靈武節度使王晏平自盜贓七千餘緡上以其父智興有功【王智興有討横海之功}
免死長流康州晏平密請於魏鎮幽三節度使【魏帥何進滔鎮帥王元逵幽帥史元忠}
使上表雪已上不得已六月壬寅改永州司戶 八月己亥嘉王運薨【運代宗子}
 太子永之母王德妃無寵為楊賢妃所譖而死【唐因隋制有貴妃淑妃德妃賢妃各一品為夫人正一品開元中玄宗以后妃四星一為后有后而復置四妃非典法乃置惠妃麗妃華妃以為三夫人其後復置貴妃蓋復唐初四妃之制}
太子頗好遊宴昵近小人【好呼到翻昵尼質翻近其靳翻}
賢妃日夜毁之九月壬戌上開延英召宰相及兩省御史郎官疏太子過惡議廢之曰是宜為天子乎羣臣皆言太子年少【少詩照翻下同}
容有改過國本至重豈可輕動御史中丞狄兼謩論之尤切至于涕泣給事中韋温曰陛下惟一子不教陷之至是豈獨太子之過乎癸亥翰林學士六人神策六軍軍使十六人復上表論之【復扶又翻}
上意稍解是夕太子始得歸少陽院如京使王少華等【唐置如京使以武臣為之内職也未知所職何事}
及宦官宫人坐流死者數十人義武節度使張璠在鎮十五年【穆宗長慶三年璠代陳楚鎮義武}
為幽鎮所憚及有疾請入朝朝廷未及制置疾甚戒其子元益舉族歸朝毋得效河北故事及薨軍中欲立元益觀察留後李士季不可衆殺之又殺大將十餘人壬申以易州刺史李仲遷為義武節度使義武馬軍都虞候何清朝自拔歸朝癸酉以為儀州刺史【宋白曰遼州樂平郡唐武德三年置遼州八年改為箕州先天二年以玄宗嫌名改為儀州}
 朝廷以義昌節度使李彦佐在鎮久【太和六年李彦佐代殷侑鎮義昌}
甲戌以德州刺史劉約為節度副使欲以代之 開成以來神策將吏遷官多不聞奏直牒中書令覆奏施行遷改殆無虛日【甘露之變之後宦官專横遂至于此}
癸未始詔神策將吏改官皆先奏聞狀至中書然後檢勘施行【先奏聞于上禁中以其狀付中書方與檢勘由歷而施行之}
 冬十月易定監軍奏軍中不納李仲遷請以張元益為留後 太子永猶不悛【悛丑緣翻改也}
庚子暴薨 【考異曰按文宗後見緣橦者而泣曰朕為天子不能全一子遂殺劉楚材等然則太子非良死也但宫省事祕外人莫知其詳故實錄但云終不悛過是日暴薨}
諡曰莊恪 乙巳以左金吾大將軍郭旼為邠寧節度使【旼莫貧翻 考異曰舊柳公權傳作皎按子儀子姪名皆連日旁今從實錄}
宰相議兵討易定上曰易定地狹人貧軍資半仰度支【仰半向翻}
急之則靡所不為緩之則自生變但謹備四境以俟之乃除張元益代州刺史頃之軍中果有異議乃上表以不便李仲遷為辭朝廷為之罷仲遷【為于偽翻}
十一月詔俟元益出定州其義武將士始謀立元益者皆赦不問 以義昌節度使李彦佐為天平節度使以劉約為義昌節度使 丁卯張元益出定州 【考異曰補國史曰易定張公璠卒三軍請公璠子元益繼統軍務公璠乃孝忠孫也公璠彌留之際誡元益歸闕三軍復效幽鎮魏三道自立連帥坐邀制命廟謀未决丞相衛公欲伐而克之貞穆公議未可興師且行弔贈禮追元益赴闕若拒命跋扈討之不遲上前互陳短長未行朝典貞穆公有密疏進追元益詔意云敕張元益卿太祖孝忠功列鼎彛垂于不朽卿乃祖茂昭克荷遺訓不墜義風文宗覽詔意深叶睿謀詔下定州元益拜詔慟哭焚墨衰請死于衆三軍將士南向稽首蹈舞流涕扶元益就苫廬請監軍使幕府進諸道例各知留後公璠遂全家赴闕詔以神策軍使陳君賞為帥所謂貞穆公者李珏也按實録璠定州牙將非孝忠孫又李德裕此年不為相補國史蓋傳聞之說不可據今從實録}
 庚午上問翰林學士柳公權以外議對曰郭旼除邠寧外間頗以為疑上曰旼尚父之姪【德宗以郭子儀為尚父}
太后叔父【太后即謂太皇郭太后}
在官無過自金吾作小鎮外間何尤焉對曰非謂旼不應為節度使也聞陛下近取旼二女入宫有之乎上曰然入參太皇太后耳公權曰外間不知皆云旼納女後宫故得方鎮上俛首良久曰然則奈何對曰獨有自南内遣歸其家則外議自息矣是日太皇太后遣中使送二女還旼家【太皇太后居興慶宫興慶宫謂之南内使疏吏翻還如字}
 上好詩【好呼到翻}
嘗欲置詩學士李珏曰今之詩人浮薄無益于理乃止 甲戌以蔡州刺史韓威為義武節度使【張元益既出定州乃除韓威}
 河東節度使司徒中書令裴度以疾求歸東都【裴度治第東都集賢里號綠野堂}
十二月辛丑詔度入知政事遣中使敦諭上道【上時掌翻}
 鄭覃累表辭位丙午詔三五日一入中書 是歲吐蕃彛泰贊普卒弟達磨立彛泰多病委政大臣由是僅能自守久不為邊患達磨荒淫殘虐國人不附災異相繼吐蕃益衰【按吐蕃衰回鶻衰而唐亦衰矣 考異曰彜泰卒及達磨立實録不書舊傳續會要皆無之今據補國史}


  四年春閏正月己亥裴度至京師以疾歸第【此長安平樂里第也}
不能入見【見賢遍翻}
上勞問賜賚使者旁午【勞力到翻}
三月丙戌薨諡曰文忠上怪度無遺表問其家得半藁以儲嗣未定為憂言不及私度身貌不踰中人而威望遠達四夷四夷見唐使輒問度老少用捨【少詩照翻}
以身繫國家輕重如郭子儀者二十餘年 夏四月戊辰上稱判度支杜悰之才楊嗣復李珏因請除悰戶部尚書陳夷行曰恩旨當由上出自古失其國未始不由權在臣下也珏曰陛下嘗語臣云【語牛倨翻}
人主當擇宰相不當疑宰相五月丁亥上與宰相論政事陳夷行復言不宜使威福在下【復扶又翻}
李珏曰夷行意疑宰相中有弄陛下威權者耳臣屢求退苟得王傅臣之幸也【王傅散地自宰執以下貶官者居之}
鄭覃曰陛下開成元年二年政事殊美三年四年漸不如前楊嗣復曰元年二年鄭覃夷行用事三年四年臣與李珏同之罪皆在臣因叩頭曰臣不敢更入中書【政事堂在中書省}
遂趨出上遣使召還勞之【勞力到翻}
曰鄭覃失言卿何遽爾覃起謝曰臣愚拙意亦不屬嗣復【屬之欲翻}
而遽如是乃嗣復不容臣耳嗣復曰覃言政事一年不如一年非獨臣應得罪亦上累聖德【累良瑞翻}
退三上表辭位上遣中使召出之癸巳始入朝丙申門下侍郎同平章事鄭覃罷為右僕射陳夷行罷為吏部侍郎覃性清儉夷行亦耿介故嗣復等深疾之【史言小人排君子不遺餘力}
 上以鹽鐵推官檢校禮部員外郎姚朂能鞠疑獄命權知職方員外郎右丞韋温不聽上奏稱郎官朝廷清選不宜以賞能吏上乃以朂檢校禮部郎中依前鹽鐵推官【姚朂權知職方員外郎而韋温爭之檢校禮部郎中而温不復言者蓋唐制藩鎮及諸使僚屬率帶檢校官而權知則為職事官故也}
六月丁丑上以其事問宰相楊嗣復對曰温志在澄清流品若有吏能者皆不得清流則天下之事孰為陛下理之【為于偽翻}
恐似衰晉之風然上素重温終不奪其所守秋七月癸未以張元益為左驍衛將軍以其母侯莫陳氏為趙國太夫人賜絹二百匹易定之亂侯莫陳氏說諭將士【說式芮翻}
且戒元益以順朝命故賞之 甲辰以太常卿崔鄲同中書門下平章事鄲郾之弟也【鄲多寒翻崔郾見二百四十四卷太和五年}
 八月辛亥鄜王憬薨【憬憲宗之子}
 癸酉昭義節度使劉從諫上言蕭本詐稱太后弟上下皆稱蕭弘是真以本來自左軍故弘為臺司所抑【蕭本事見上卷元年蕭弘事見二年臺司謂御史臺官吏主案驗蕭弘者}
今弘詣臣求臣上聞【上時掌翻}
乞追弘赴闕與本對推以正真偽詔三司鞫之 冬十月乙卯上就起居舍人魏謩取記注觀之【記注即起居注貞觀初以給事中諫議大夫兼知起居注或知起居事每仗下議記事起居郎一人執筆記錄于前史官隨之其後復置起居舍人分侍左右秉筆隨宰相入殿若仗在紫宸内閤則夾香案分立殿下直第二螭首和墨濡筆皆即坳處時號螭頭高宗臨朝不决事有所奏惟辭見而已許敬宗李義府為相奏請多畏人之知也命起居郎舍人對仗承旨仗下與百官皆出不敢聞機務矣長夀中宰相姚璹建議仗下後宰相一人錄軍國政要為時政記月送史館然率推美讓善事非其實未幾亦罷而起居郎因制勑稍稍筆削以廣國史之闕起居舍人本記言之職惟編詔書不及它事開元初復詔修史官非供奏者皆隨仗而入位于起居郎舍人之次及李林甫專權又廢太和九年詔起居郎舍人凡入閣日具紙筆立螭頭下復貞觀故事}
謩不可曰記注兼書善惡所以儆戒人君陛下但力為善不必觀史上曰朕曏嘗觀之對曰此曏日史官之罪也若陛下自觀史則史官必有所諱避何以取信于後上乃止 楊妃請立皇弟安王溶為嗣上謀于宰相李珏非之丙寅立敬宗少子陳王成美為皇太子【為楊妃及成美見殺張本}
丁卯上幸會寧殿作樂有童子緣橦【橦職容翻字様曰本音同今借為木橦字漢有都盧緣橦即此伎也}
一夫來往走其下如狂上怪之左右曰其父也上泫然流涕曰朕貴為天子不能全一子【泫胡犬翻以太子永死于非命也}
召教坊劉楚材等四人宫人張十十等十人責之曰構會太子皆爾曹也今更立太子【更工衡翻}
復欲爾邪【復扶又翻}
執以付吏己巳皆殺之上因是感傷舊疾遂增 十一月三司按蕭本蕭弘皆非真太后弟本除名流愛州弘流儋州【愛州漢九真郡梁置愛州至京師八千八百里}
而太后真弟在閩中終不能自達 乙亥上疾少閒【閒讀如字}
坐思政殿召當直學士周墀賜之酒因問曰朕可方前代何主對曰陛下堯舜之主也上曰朕豈敢比堯舜所以問卿者何如周赧漢獻耳【赧奴版翻}
墀驚曰彼亡國之主豈可比聖德上曰赧獻受制于彊諸侯今朕受制于家奴以此言之朕殆不如因泣下霑襟墀伏地流涕自是不復視朝 【考異曰高彦休唐闕史曰文宗開成後常鬱鬱不樂五年春風痺稍間坐思政殿問周墀云云既而龍姿掩抑淚落衣襟汝南公俯伏嗚咽再拜而退自是不復視朝以至厭代按實録明年正月朔上不康不受朝賀四日帝崩恐非五年春今從新傳仍置于此}
 是歲天下戶口四百九十九萬六千七百五十二回鶻相安允合特勒柴革謀作亂彰信可汗殺之相

  掘羅勿將兵在外以馬三百賂沙陁朱邪赤心借其兵共攻可汗可汗兵敗自殺國人立㕎馺特勒為可汗【㕎安盍翻馺先合翻 考異曰後唐獻祖紀年録曰開成四年回鶻大饑族帳離復為戛斯所逼漸過磧口至于榆林天德軍使温德彛請帝為援遂帥騎赴之時胡特勒可汗牙帳在近帝遣使說回鶻相嗢沒斯為陳利害云云嗢沒斯然之决有歸國之約俄而回鶻宰相勿篤公叛可汗將圖歸義遣人獻良馬三百以求應接帝自天德引軍至磧口援之為回鶻所薄帝一戰敗之進撃可汗牙帳胡特勒可汗勢窮自殺國昌因奏勿篤公為署颯可汗是歲開成五年也文宗崩武宗即位遣嗣澤王溶告哀于回鶻使還始知特勒可汗易代按朱邪赤心若奏勿篤公為可汗安得因溶告哀始知易代乎此則自相違矣舊傳開成初其相有安允合者與特勒柴革欲篡薩特勒可汗可汗覺殺柴革及安允合又有回鶻相掘羅勿者擁兵在外怨誅柴革安允合又殺薩特勒可汗以㕎級特勒為可汗新傳云開成四年其相掘羅勿作難引沙陁共攻可汗可汗自殺國人立㕎馺特勒為可汗今從之}
會歲疫大雪羊馬多死回鶻遂衰赤心執宜之子也

  五年春正月己卯詔立潁王瀍為皇太弟【瀍直連翻}
應軍國事權令句當【句古候翻當丁浪翻}
且言太子成美年尚冲幼未漸師資【漸子廉翻老子曰善人者不善人之師不善人者善人之資}
可復封陳王時上疾甚命知樞密劉弘逸薛季稜引楊嗣復李珏至禁中欲奉太子監國中尉仇士良魚弘志以太子之立功不在已乃言太子幼且有疾更議所立【更工衡翻}
李珏曰太子位已定豈得中變士良弘志遂矯詔立瀍為太弟 【考異曰唐闕史曰武宗皇帝王夫人者燕趙倡女也武宗為潁王獲愛幸文宗于十六宅西别建安王溶潁王瀍陀上數幸其中縱酒如家人禮及文宗晏駕後宫無子所立敬宗男陳王年幼且病未任軍國事中貴主禁掖者以安王大行親弟既賢且長遂起左右神策軍及飛龍羽林驍騎數千衆即藩邸奉迎安王中貴遥呼曰迎大者迎大者如是者數四意以安王為兄即大者也及兵仗至二王宅首兵士相語曰奉命迎大者不知安潁孰為大者王夫人竊聞之擁髻褰裙走出矯言曰大者潁王也大家左右以王魁梧頎長皆呼為大王且與中尉有死生之契汝曹或誤必赤族矣時安王心云其次第合立志少疑懦懼未敢出潁王神氣抑揚隱于屏間夫人自後聳出之衆惑其語遂扶上馬戈甲霜擁前至少陽院諸中貴知己誤無敢出言者遂羅拜馬前連呼萬歲尋下詔以潁王瀍立為皇太弟權句當軍國事新后妃傳曰武宗賢妃王氏開成末王嗣帝位妃隂為助畫故進號才人蓋亦取於闕史也按立嗣大事豈容繆誤闕史難信今不取從文宗武宗實録}
是日士良弘志將兵詣十六宅迎潁王至少陽院百官謁見于思賢殿瀍沈毅有斷喜愠不形于色【見賢遍翻沈持林翻斷丁亂翻愠於問翻}
與安王溶皆素為上所厚異于諸王辛巳上崩于太和殿【年三十三}
以楊嗣復攝冢宰癸未仇士良說太弟賜楊貴妃安王溶陳王成美死【說式芮翻 考異曰舊傳曰安王溶穆宗第八子母楊賢妃武宗即位李德裕秉政或告文宗崩時楊嗣復以與賢妃宗家欲立安王為嗣故王受禍復貶官按是時德裕未入相今從武宗實録}
勅大行以十四日殯成服 【考異曰武宗實錄裴夷直上言伏見二日敕令有司以今月十四日攢斂成服按文宗以二日崩豈得二日遽有此敕必誤也}
諫議大夫裴夷直上言期日太遠不聽時仇士良等追怨文宗【以甘露之事也}
凡樂工及内侍得幸于文宗者誅貶相繼夷直復上言陛下自藩維繼統是宜儼然在疚【記檀弓秦穆公弔公子重耳曰儼然在憂服之中詩閔予小子嬛嬛在疚注疚病也在憂病之中復扶又翻}
以哀慕為心速行喪禮早議大政以慰天下而未及數日屢誅戮先帝近臣驚率土之視聽傷先帝之神靈人情何瞻國體至重若使此輩無罪固不可刑若其有罪彼已在天網之内無所逃伏旬日之外行之何晚不聽辛卯文宗始大歛【大行十一日而始大歛非禮也歛力瞻翻}
武宗即位甲午追尊上母韋妃為皇太后 二月乙卯赦天下 丙寅諡韋太后曰宣懿 夏五月己卯門下侍郎同平章事楊嗣復罷為吏部尚書以刑部尚書崔珙同平章事兼鹽鐵轉運使【珙居竦翻}
 秋八月壬戌葬元聖昭獻孝皇帝于章陵【章陵在京兆富平縣西北二十里}
廟號文宗 庚午門下侍郎同平章事李珏坐為山陵使龍輴陷【輴敕倫翻記天子龍輴輴載柩車也畫龍于轅}
罷為太常卿貶京兆尹敬昕為郴州司馬【郴丑林翻}
 義武軍亂逐節度使陳君賞君賞募勇士數百人復入軍城誅亂者初上之立非宰相意故楊嗣復李珏相繼罷去召淮

  南節度使李德裕入朝九月甲戌朔至京師丁丑以德裕為門下侍郎同平章事庚辰德裕入謝言于上曰致理之要【致理猶言致治也}
在于辯羣臣之邪正夫邪正二者勢不相容正人指邪人為邪邪人亦指正人為邪人主辯之甚難臣以為正人如松柏特立不倚邪人如藤蘿非附他物不能自起故正人一心事君而邪人競為朋黨先帝深知朋黨之患然所用卒皆朋黨之人【卒子恤翻}
良由執心不定故奸人得乘閒而入也【閒古莧翻下疑間同}
夫宰相不能人人忠良或為欺罔主心始疑于是旁詢小臣以察執政如德宗末年所聽任者惟裴延齡輩宰相署敕而已此政事所以日亂也陛下誠能慎擇賢才以為宰相有姧罔者立黜去【去羌呂翻}
常令政事皆出中書推心委任堅定不移則天下何憂不理哉又曰先帝于大臣好為形迹【好呼到翻}
小過皆含容不言日累月積【累魯水翻}
以致禍敗兹事大誤願陛下以為戒臣等有罪陛下當面詰之【詰起吉翻}
事苟無實得以辯明若其有實辭理自窮小過則容其悛改【悛丑緣翻}
大罪則加之誅譴如此君臣之際無疑間矣上嘉納之初德裕在淮南敕召監軍楊欽義人皆言必知樞密德裕待之無加禮欽義心銜之一旦獨延欽義置酒中堂情禮極厚陳珍玩數牀罷酒皆以贈之欽義大喜過望行至汴州敕復還淮南【復扶又翻}
欽義盡以所餉歸之德裕曰此何直【言此物所直能幾何也}
卒以與之【卒子恤翻}
其後欽義竟知樞密德裕柄用欽義頗有力焉【史言李德裕亦不免由宦官以入相}
 初伊吾之西焉耆之北有戛斯部落【下八翻戛紇翻}
即古之堅昆唐初結骨也後更號黠戛斯【結骨入貢見二百九十八卷太宗貞觀二十二年 考異曰李德裕會昌一品集安撫回鶻制作紇吃斯今從會昌伐叛記杜牧集新舊傳實錄}
乾元中為回鶻所破自是隔閡不通中國【閡牛代翻}
其君長曰阿熱建牙青山去回鶻牙槖駞行四十日【青山在劒河西}
其人悍勇【悍戶罕翻又侯旰翻}
吐蕃回鶻常賂遺之【遺唯季翻}
假以官號回鶻既衰阿熱始自稱可汗回鶻遣相國將兵擊之連兵二十餘年數為戛斯所敗【數所角翻敗補邁翻}
詈回鶻曰【詈力智翻}
汝運盡矣我必取汝金帳金帳者回鶻可汗所居帳也及掘羅勿殺彰信立㕎馺【事見上年}
回鶻别將句錄莫賀引戛斯十萬騎攻回鶻大破之殺㕎馺及掘羅勿 【考異曰舊傳作句錄未賀今從新傳}
焚其牙帳蕩盡回鶻諸部逃散其相馺職特勒庬等十五部西奔葛邏祿一支奔吐蕃一支奔安西【邏郎佐翻}
可汗兄弟嗢没斯等【嗢烏没翻}
及其相赤心僕固特勒那頡啜【啜樞悦翻}
各帥其衆抵天德塞下就雜虜貿易穀食【帥讀曰率貿音茂}
且求内附冬十月丙辰天德軍使温德彝奏回鶻潰兵侵逼西城【西城朔方西受降城也}
亘六十里不見其後【亘横亘也}
邊人以回鶻猥至恐懼不安詔振武節度使劉沔屯雲迦關以備之【新志單于府有雲伽關振武節度治單于府迦古牙翻又居伽翻 考異曰新傳實錄作雲伽關今從一品集}
 魏博節度使何進滔薨軍中推其子都知兵馬使重順知留後 蕭太后徙居興慶宫積慶殿號積慶太后【蕭太后文宗之母}
 十二月癸酉朔上幸雲陽校獵 故事新天子即位兩省官同署名上之即位也諫議大夫裴夷直漏名由是出為杭州刺史 【考異曰新傳曰武宗立夷直視冊牒不肯署今從武宗實錄}
 開府儀同三司左衛上將軍兼内謁者監仇士良請以開府䕃其子為千牛【唐制千牛備身掌執御刀服花鈿繡衣緑執象笏宿衛侍從宋白曰唐制千牛進馬並係資䕃}
給事中李中敏判曰開府階誠宜䕃子【唐制從五品以上皆得䕃子開府從一品宜得䕃子}
謁者監何由有兒士良慙恚【恚于避翻}
李德裕亦以中敏為楊嗣復之黨惡之【惡烏路翻}
出為婺州刺史【婺州春秋越之西界漢為會稽郡烏傷縣地吳置東陽郡陳置縉州隋平陳為吳州其地于天文為婺女之分改婺州京師東南四千七百里婺亡遇翻}
十二月庚申以何重順知魏博留後事 立皇子峻為王

  武宗至道昭肅孝皇帝【諱瀍穆宗第五子}


  會昌元年春正月辛巳上祀圓丘赦天下改元 劉沔奏回鶻已退詔沔還鎮【自雲迦關還鎮}
 二月回鶻十三部近牙帳者立烏希特勒為烏介可汗南保錯子山【新志鸊鵜泉北十里入磧經麚鹿山鹿耳山至錯甲山據李德裕言錯子山東距釋迦泊三百里 考異曰據伐叛記烏介立在二月今從之後唐獻祖繋年錄曰王子烏希特勒者曷薩之弟胡特勒之叔為戛斯所迫帥衆來歸至錯子山乃自立為可汗二年七月冊為烏介可汗}
 三月甲戌以御史大夫陳夷行為門下侍郎同平章事 初知樞密劉弘逸薛季稜有寵于文宗仇士良惡之【惡烏路翻}
上之立非二人及宰相意故楊嗣復出為湖南觀察使李珏出為桂管觀察使士良屢譖弘逸等于上勸上除之乙未賜弘逸季稜死遣中使就潭桂州誅嗣復及珏【湖南觀察使治潭州桂管觀察使治桂州潭州古長沙郡京師南二千四百四十五里秦取陸梁地為桂林郡吳于桂林置始安郡梁置桂州至京師水陸路四千七百六十里}
戶部尚書杜悰奔馬見李德裕曰天子年少【少詩照翻}
新即位兹事不宜手滑丙申德裕與崔珙崔鄲陳夷行三上奏又邀樞密使至中書使入奏以為德宗疑劉晏動揺東宫而殺之中外咸以為寃兩河不臣者由兹恐懼得以為辭德宗後悔錄其子孫【劉晏之死見二百二十六卷德宗建中元年李正已等請晏罪見上年興元初帝寤許晏歸葬貞元五年擢晏子執經太常博士宗經秘書郎}
文宗疑宋申錫交通藩邸竄謫至死既而追悔為之出涕【宋申錫竄事見二百四十四卷太和五年追悔事見上卷開成元年為于偽翻}
嗣復珏等若有罪惡乞更加重貶必不可容亦當先行訊鞠俟罪狀著白誅之未晩今不謀于臣等遽遣使誅之人情莫不震駭願開延英賜對至晡時開延英召德裕等入德裕等泣涕極言陛下宜重慎此舉毋致後悔上曰朕不悔三命之坐德裕等曰臣等願陛下免二人於死勿使既死而衆以為寃今未奉聖旨臣等不敢坐久之上乃曰特為卿等釋之【特為于偽翻}
德裕等躍下階舞蹈上召升坐【坐徂卧翻}
歎曰朕嗣位之際宰相何嘗比數李珏季稜志在陳王【陳王成美也}
嗣復弘逸志在安王【安王溶也}
陳王猶是文宗遺意安王則專附楊妃【楊妃請立安王故云然}
嗣復仍與妃書云姑何不效則天臨朝曏使安王得志朕那復有今日【復扶又翻}
德裕等曰兹事曖昧虛實難知上曰楊妃嘗有疾文宗聽其弟玄思入侍月餘以此得通指意朕細詢内人情狀皎然非虛也遂追還二使【二使一往潭一往桂}
更貶嗣復為潮州刺史李珏為昭州刺史【昭州至京師四千四百三十六里}
裴夷直為驩州司戶 【考異曰舊紀開成五年八月十七日葬文宗于章陵知樞密劉弘逸薛季稜率禁軍護靈駕二人素為文宗奬遇仇士良惡之心不自安因是欲倒戈誅士良弘志鹵簿使王起山陵使崔鄲覺其謀先諭鹵簿諸軍是日弘逸季稜伏誅以楊嗣復為湖南觀察使李珏為桂管觀察使中丞裴夷直為杭州刺史皆坐弘逸季稜也賈緯唐年補錄曰五年八月云是月誅樞密使劉弘逸薛季稜帝即位尤忌宦官季稜弘逸深懼之及將葬文宗于章陵聚禁兵欲議廢立賴山陵使崔鄲鹵簿使王起拒而獲濟遂擒弘逸季稜殺之舊王起傳八月充山陵鹵簿使樞密使劉弘逸薛季稜懼誅欲因山陵兵士謀廢立起與山陵使知其謀密奏皆伏誅舊嗣復傳五年九月貶湖南明年誅季稜弘逸中人言二人頃附嗣復李珏不利于陛下武宗性急立命中使往湖南桂管殺嗣復與珏按去年八月若已誅弘逸季稜不當至此月始再貶嗣復等舊紀王起傳與嗣復傳自相違今從實錄實錄又曰時有再以其事動帝意者帝赫怒欲殺之中使既雖宰相亦不知之戶部尚書判度支杜悰奔馬見德裕云云舊嗣復傳曰宰相崔鄲崔珙等亟請開延英極言云云獻替記曰會昌元年三月二十四日遇假在宅向晚聞有中使一人向東一人向南處置二故相及裴夷直余遣人問鹽鐵崔相度支杜尚書京兆盧尹皆云聞有使去不知其故余遂草約奏狀二十五日早入中書崔相珙續至崔鄲次至陳相最後至己巳時矣余令三相會食自歸廳寫狀請開延英賜對進狀後更無報答至午又自寫第二狀封進兼請得樞密使至中書問有此事無樞密使對曰向者不敢言相既知只是二人嗣復李珏德裕言此事至重陛下都不訪問便遣使去物情無不驚懼請附德裕奏聖旨若疑德裕情故請先自遠貶惟此一事不可更行德裕等至夜不敢離中書請早開延英賜對至中時報開延英余邀得丞相兩省官謂曰上性剛若有一人進狀伏問必不捨矣容德裕極力救解繼以叩頭流血德裕救不得它人固不可矣及召入延英德裕率三相公立當御榻奏事嗚咽流涕云云上既捨之又令德裕召丞郎兩省官宣示今從實錄亦采獻替記 宋白曰天福六年修撰起居注賈緯奏伏覩史館唐高祖至代宗已有紀傳德宗亦存實錄武宗至濟隂廢帝凡六代惟有武宗實錄一卷餘皆闕落臣今采訪遺文及耆舊傳說編六十五卷目為唐年補遺錄以備將來史館修述詔褒美付史館}
 夏六月乙巳詔自今臣下論人罪惡並應請付御史臺按問毋得乞留中以杜讒邪 以魏博留後何重順為節度使 上命道士趙歸真等于三殿建九天道場親授法籙【道家符籙起于張道陵盛于寇謙之崇而信之則後魏世祖唐武宗也授當作受}
右拾遺王哲上疏切諫坐貶河南府士曹【考異曰實錄道士趙歸真等八十一人于三殿建九天道場帝親傳法籙右拾遺王哲上疏請不度進士明經為道士不從又上書諫求仙事詞甚切直貶河南府士曹參軍舊紀以衡山道士劉玄靜為崇玄館學士令與道士趙歸真于禁中修法籙左補闕劉彦謨切諫貶彦謨河南府戶曹實錄去年九月已命歸真建道場親受法籙哲疏言王業之始不宜崇信過篤至此又有此事與舊紀劉彦謨事相類今從實錄}
 秋八月加仇士良觀軍容使 天德軍使田牟監軍韋仲平欲擊回鶻以求功奏稱回鶻叛將嗢沒斯等侵逼【嗢烏沒翻}
塞下吐谷渾沙陀党項皆世與為仇請自出兵驅逐【党底朗翻}
上命朝臣議之議者皆以為嗢沒斯叛可汗而來不可受宜如牟等所請擊之便上以問宰相李德裕以為窮鳥入懷猶當活之况回鶻屢建大功【謂助收兩京平安史之亂也}
今為鄰國所破部落離散窮無所歸遠依天子無秋毫犯塞奈何乘其困而擊之宜遣使者鎮撫運糧食以賜之此漢宣帝所以服呼韓邪也【呼韓邪事見二十七卷漢宣帝之甘露三年}
陳夷行曰此所謂借寇兵資盜糧也【史記秦李斯之言}
不如擊之德裕曰彼吐谷渾等各有部落見利則鋭敏爭進不利則鳥驚魚散各走巢穴【走音奏}
安肯守死為國家用今天德城兵纔千餘若戰不利城陷必矣不若以恩義撫而安之必不為患縱使侵暴邊境亦須徵諸道大兵討之豈可獨使天德擊之乎時詔以鴻臚卿張賈為巡邊使使察回鶻情偽【臚陵如翻邊使疏吏翻 考異曰一品集賜嗢沒斯等詔曰天德軍遞至覽所奉表又曰方圖鎮撫己命使臣又知堅昆等五族深入陵虐可汗被害公主及新可汗播越它所特勒等相率遁逃萬里歸命又曰豈非欲討除外寇匡復本蕃又曰但緣未知指的難便聽從又曰又慮邊境守臣或懷疑沮又曰故遣張賈往安撫又曰秋熱然則詔下必在此際也}
未還上問德裕曰嗢沒斯等請降可保信乎對曰朝中之人臣不敢保况敢保數千里外戎狄之心乎然謂之叛將則恐不可若可汗在國嗢沒斯等帥衆而來則于體固不可受今聞其國敗亂無主將相逃散或奔吐蕃或奔葛邏祿惟此一支遠依大國觀其表辭危迫懇切豈可謂之叛將乎【降戶江翻朝直遙翻將即亮翻}
况嗢沒斯等自去年九月至天德今年二月始立烏介自無君臣之分【分扶聞翻}
願且詔河東振武嚴兵保境以備之俟其攻犯城鎮然後以武力驅除或于吐谷渾等部中少有抄掠聽自讎報亦未可助以官軍【先鬭之以離其交此在兵法習者不察耳抄楚交翻}
仍詔田牟仲平毋得邀功生事常令不失大信懷柔得宜彼雖戎狄必知感恩辛酉詔田牟約勒將士及雜虜【雜虜即吐谷渾沙陀党項等部落}
毋得先犯回鶻 【考異曰舊紀八月烏介遣使告故可汗死部人推為可汗今奉公主南投大國時烏介至塞上嗢沒斯與赤心相攻殺赤心率數千帳近西城田牟以聞烏介又令其相頡干迦斯表借天德城仍乞糧儲牛羊詔王會李師偃往宣慰令放公主入朝賑粟二萬石舊德裕傳曰開成末回鶻為戛斯所破部族離散烏介奉太和公主南來會昌二年二月牙于塞上遣使求助兵糧收復本國權借天德軍田牟請以沙陀退渾諸部擊之下百寮議議者多云如牟之奏德裕云云帝以為然許借米三萬石伐叛記曰會昌元年二月回鶻遠涉沙漠饑餓尤甚將金寶于塞上部落博糴糧食邊人貪其財寶生攘奪之心至其年秋城使田牟監軍韋仲平上表稱退渾党項與回鶻宿有嫌怨願出本部兵馬驅逐其時天德城内只有將士一千人職事又居其半上令宰臣商量德裕面奏云云八月二十四日請賜田牟仲平詔漢兵及番渾不得先犯回鶻語在會要集奏狀中按舊紀實錄皆采集衆書為之事前後多差互今從伐叛記一品集}
九月戊辰朔詔河東振武嚴兵以備之牟布之弟也【田布弘正之子死于史憲誠之亂}
 癸巳盧龍軍亂殺節度使史元忠推陳行泰主留務 李德裕請遣使慰撫回鶻且運糧三萬斛以賜之上以為疑閏月己亥開延英召宰相議之陳夷行于候對之所【唐自德宗以後羣臣乞對延英率于延英門請對會要曰元和十五年詔于西上閤門西廊内開便門以通宰臣自閤中赴延英路宋申錫之得罪也召諸宰相自中書入對延英}
屢言資盜糧不可德裕曰今徵兵未集天德孤危儻不以此糧噉飢虜【噉徒濫翻}
且使安靜萬一天德陷沒咎將誰歸【李德裕之本計是也至于此言特以箝陳夷行之喙耳若以用兵大勢言之固將不計一城得失也此弊自唐及宋皆然嗚呼可易言哉}
夷行至上前遂不敢言上乃許以穀二萬斛賑之【賑之忍翻考異曰伐叛記云降使賜米二萬石尋又烏介至天德按實錄十一月初猶未知公主所在遣苖縝至嗢沒斯處訪問月末始云公主遣使言烏介可汗乞冊命及降使宣慰十二月庚辰制曰公主遣使入朝已知新立可汗寓居塞下宜令王會慰問仍賑米二萬斛然則閏九月中烏介未至天德德裕但欲賑嗢沒斯等耳上雖許賜米而未遣使會聞烏介在塞下因遣王會併賜之二萬斛耳非再賜也伐叛記終言其事非以閏九月中即降使賜米也}
 以前山南東道節度使同平章事牛僧孺為太子太師先是漢水溢壞襄州民居【先悉薦翻壞音怪}
故李德裕以為僧孺罪而廢之【廢之者使居散地也史言李德裕以私怨而廢牛僧孺}
 盧龍軍復亂【復扶又翻下同}
殺陳行泰立牙將張絳 【考異曰舊紀十月幽州雄武軍使張絳遣軍吏吳仲舒入朝言行泰慘虐請以鎮軍加討許之是月誅行泰遂以絳知兵馬事二年正月以絳知留後仍賜名仲武以兩人為一人誤也今從舊仲武傳伐叛記實錄}
初陳行泰逐史元忠遣監軍傔【傔苦念翻監軍傔監軍之傔從也}
以軍中大將表來求節鉞李德裕曰河朔事勢臣所熟諳比來朝廷遣使賜詔常太速【諳烏含翻比毗至翻}
故軍情遂固若置之數月不問必自生變今請留監軍傔勿遣使以觀之既而軍中果殺行泰立張絳復求節鉞朝廷亦不問會雄武軍使張仲武起兵擊絳【雄武軍在薊州廣漢川}
且遣軍吏吳仲舒奉表詣京師稱絳慘虐請以本軍討之冬十月仲舒至京師詔宰相問狀仲舒言行泰絳皆遊客故人心不附仲武幽州舊將【仲武范陽舊將張光朝之子}
性忠義通書習戎事人心嚮之曏者張絳初殺行泰召仲武欲以留務讓之牙中一二百人不可仲武行至昌平絳復却之今計仲武纔雄武軍中已逐絳矣李德裕問雄武士卒幾何對曰軍士八百外有土團五百人【團結土人為兵故謂之土團}
德裕曰兵少何以立功對曰在得人心苟人心不從兵三萬何益德裕又問萬一不克如何對曰幽州糧食皆在媯州及北邊七鎮【媯州南至幽州二百九十里東至檀州二百五十里檀州有大王北來保要鹿固赤城邀虜石子齕七鎮媯居為翻}
萬一未能入則據居庸關【幽州昌平縣軍都陘西北三十五里有納款關即居庸故關亦謂之軍都關按今居庸關在燕京之北一百一十里}
絶其糧道幽州自困矣【李德裕因吳仲舒之言固心服張仲武之方畧矣命掌燕留務豈徒然哉}
德裕奏行泰絳皆使大將上表脅朝廷邀節鉞故不可與今仲武先自兵為朝廷討亂【為于偽翻}
與之則似有名【德裕既未敢保張仲武又恐與其初論河朔事勢者相違故然}
乃以仲武知盧龍留後仲武尋克幽州上校獵咸陽 十一月李德裕上言今回鶻破亡太和公主未知所在君不遣使訪問則戎狄必謂國家降主虜庭本非愛惜既負公主又傷虜情請遣通事舍人苖縝齎詔詣嗢沒斯【縝正忍翻}
令轉達公主兼可卜嗢沒斯逆順之情從之 上頗好田獵及武戲【武戲謂毬鞠騎射手搏等好呼到翻}
五坊小兒得出入禁中賞賜甚厚嘗謁郭太后【郭太后憲宗妃于上為祖母時居興慶宫以養}
從容問為天子之道【從千容翻}
太后勸以納諫上退悉取諫疏閱之多諫遊獵自是上出畋稍稀五坊無復横賜【横下孟翻}
 癸亥以中書侍郎同平章事崔鄲同平章事充西川節度使 初戛斯既破回鶻得太和公主自謂李陵之後【唐書曰戛斯人皆長大赤髮晳面綠瞳以黑髮者為不祥黑瞳者必曰李陵裔也}
與唐同姓遣達干十人奉公主歸之于唐回鶻烏介可汗引兵邀擊達干盡殺之質公主南度磧【質音致磧七迹翻}
屯天德軍境上【天德軍境北至磧口三百里}
公主遣使上表言可汗已立求冊命烏介又使其相頡干伽斯等上表借振武一城以居公主可汗 【考異曰新傳曰達干奉主來歸烏介怒撃達干殺之南度磧進攻天德城劉沔屯雲伽關拒却之按烏介方倚唐為援豈敢攻天德今從舊紀傳實錄}
十二月庚辰制遣右金吾大將軍王會等慰問回鶻仍賑米二萬斛又賜烏介可汗敕書諭以宜帥部衆漸復舊疆【帥讀曰率}
漂寓塞垣殊非良計又云欲借振武一城前代未有此比【比毗至翻例也}
或欲别遷善地求大國聲援亦須于漠南駐止朕當許公主入覲親問事宜儻須應接必無所吝

  二年春正月以張仲武為盧龍節度使 朝廷以回鶻屯天德振武北境以兵部郎中李拭為巡邊使察將帥能否拭鄘之子也【李鄘見二百四十卷元和十二年}
 二月淮南節度使李紳入朝丁丑以紳為中書侍郎同平章事判度支河東節度使苻澈修杷頭烽舊戍以備回鶻【杷頭烽北臨大}


  【磧東望雲朔西望振武杷蒲巴翻}
李德裕奏請增兵鎮守及修東中二受降城以壯天德形勢從之 右散騎常侍柳公權素與李德裕善崔珙奏為集賢學士判院事【玄宗開元十三年改麗正修書院為集賢殿書院五品以上為學士六品以下為直學士宰相一人為學士知院事常侍一人為副知院事又置判院一人押院中使一人元和四年集賢御書院學士直學士皆用五品如開元故事以學士一人年高者判院事}
德裕以恩非已出因事左遷公權為太子詹事【此德裕所以不能免朋黨之禍也}
 回鶻復奏求糧【復扶又翻}
及尋勘吐谷渾党項所掠又借振武城詔遣内使楊觀賜可汗書諭以城不可借餘當應接處置【處昌呂翻}
三月李拭巡邊還稱振武節度使劉沔有威畧可任大事時河東節度使苻澈疾病【疾甚曰病}
庚申以沔代之以金吾上將軍李忠順為振武節度使遣將作少監苖縝冊命烏介可汗使徐行駐于河東俟可汗位定然後進既而可汗屢侵擾邉境縝竟不行 回鶻嗢沒斯以赤心桀難知【下八翻}
先告田牟云赤心謀犯塞乃誘赤心并僕固殺之那頡啜收赤心之衆七千帳東走 【考異曰伐叛記曰赤心宰相欲謀犯塞嗢沒斯先布誠于田牟然後誘赤心同謁可汗戮于可汗帳下赤心所領兵馬遂潰散東去歸投幽州一品集幽州紀聖功碑赤心怙力負氣潛圖厲階為嗢沒斯所紿誘以俱謁可汗戮于帳下其衆大潰東逼漁陽舊傳曰回鶻相赤心者與連位相姓僕固者與特那頡啜擁部衆不賓烏介赤心欲犯塞烏介遣其屬嗢沒斯先布誠于田牟然後誘赤心同謁烏介戮赤心于可汗帳下并僕固二人那頡戰勝全占赤心下七千帳東瞰振武大同據室韋黑沙榆林東南入幽州雄武軍西北界新傳曰嗢沒斯以赤心姧桀難得要領即密約田牟誘赤心斬帳下按一品集賜可汗敕書雖云去歲嗢沒斯已至近界今可汗既立彼又降附然賜可汗書意又云嗢沒斯自本國破之初奔迸先至塞上不隨可汗公主已是二年是則嗢沒斯自有部衆雖遥降烏介身未嘗往也安得斬赤心僕固于帳下乎且赤心若不賓烏介又安肯隨嗢沒斯同謁烏介乎蓋嗢沒斯自惡赤心桀誘至已之帳下而殺之耳今從新傳又伐叛記嗢沒斯殺赤心于烏介至天德下連言之舊傳亦然新傳在召諸道兵討烏介下按一品集據回鶻到横水柵未知是那頡特下為復是可汗遣來蓋那頡特下脱勒字即那頡啜也然則虜犯横水在赤心死後故置于此}
河東奏回鶻兵至横水 【考異曰實錄苻澈奏回鶻掠横水事在正月李拭巡邊前按一品集此狀云宜密詔劉沔忠順則狀必在李忠順鎮振武之後也蓋澈在太原時奏之沔除河東後德裕方有此奏故置于此}
殺掠兵民今退屯釋迦泊東李德裕上言釋迦泊西距可汗帳三百里【烏介時移帳保錯子山}
未知此兵為那頡所部為可汗遣來宜且指此兵云不受可汗指揮擅掠邊鄙密詔劉沔仲武【仲武張仲武也}
先經畧此兵如可以討逐事亦有名摧此一支可汗必自知懼夏四月庚辰天德都防禦使田牟奏回鶻侵擾不已不俟朝旨已出兵三千拒之壬午李德裕奏田牟殊不知兵戎狄長于野戰短于攻城牟但應堅守以待諸道兵集今全軍出戰萬一失利城中空虛何以自固望亟遣中使止之如已交鋒即詔雲朔天德以來羌渾各出兵奮擊回鶻凡所虜獲並令自取回鶻羈旅二年糧食乏絶人心易動【易以䜴翻}
宜詔田牟招誘降者給糧轉致太原不可留於天德嗢沒斯情偽雖未可知然要早加官賞【考異曰一品集異域歸忠傳序云二年四月甲申回鶻大特勒嗢沒斯率其國特勒宰相等内附而此四月}


  【十八日狀已言嗢沒斯送款者蓋嗢沒斯自欲誅赤心之時已送款于田牟至二十日乃率衆至天德耳故其授左金吾大將軍制云屢獻款誠布于邊將尋執反虜不遺君親戢其餓殍之徒曾靡秋毫之犯旋觀所履大節甚明蓋回鶻亂亡嗢沒斯本與赤心等來歸唐而邊吏疑阻故赤心等怒欲犯塞而嗢沒斯先告邊吏誘赤心之衆東走而嗢沒斯帥其衆降唐也}
縱使不誠亦足為反閒【閒古莧翻}
且欲奬其忠義為討伐之名令遠近諸番知但責可汗犯順非欲盡滅回鶻石雄善戰無敵請以為天德都團練副使佐田牟用兵上皆從其言初太和中河西党項擾邊文宗召石雄于白州【雄流白州見二百二十四卷太和三年}
隸振武軍為禆將屢立戰功以王智興故未甚進擢至是德裕舉用之甲申嗢沒斯帥其國特勒宰相等二千二百餘人來降 【考異曰一品集嗢沒斯特勒等狀五月四日上實錄在五月丙申蓋據奏到之日也今從歸忠傳序}
上信任李德裕觀軍容使仇士良惡之【惡烏路翻}
會上將

  受尊號御丹鳳樓宣赦或告士良宰相與度支議草制減禁軍衣糧及馬芻粟士良揚言于衆曰如此至日軍士必于樓前諠譁德裕聞之乙酉乞開延英自訴上怒遽遣中使宣諭兩軍赦書初無此事且赦書皆出朕意非由宰相爾安得此言士良乃惶愧稱謝丁亥羣臣上尊號曰仁聖文武至神大孝皇帝赦天下 五月戊申遣鴻臚卿張賈安撫嗢沒斯等以嗢沒斯為左金吾大將軍懷化郡王其次酋長官賞有差【酋慈由翻長知丈翻}
賜其部衆米五千斛絹三千匹那頡啜帥其衆自振武大同東因室韋黑沙南趣雄武軍窺幽州【趣七喻翻}
盧龍節度使張仲武遣其弟仲至將兵三萬迎擊大破之斬首捕虜不可勝計【勝音升}
悉收降其七千帳分配諸道那頡啜走烏介可汗獲而殺之 【考異曰伐叛記曰仲武招降赤心下潰兵及可汗下部落前後三萬餘人分配諸道回鶻種族遂至寡弱新舊紀皆無仲武破回鶻事舊回紇傳曰仲武大破那頡之衆全收七千帳殺戮收擒老小共九萬人那頡中箭透駝羣潛脱烏介獲而殺之一品集幽州紀聖功碑曰公前後受降三萬人特勒二人可汗妹一人大都督外宰相四人其他禆王騎將不可備載諸書皆不言仲武破那頡啜日月故附于此}
時烏介衆雖衰減尚號十萬駐牙于大同軍北閭門山楊觀自回鶻還【還音旋}
可汗表求糧食牛羊【因楊觀之還而上表}
且請執送嗢沒斯等詔報以糧食聽自以馬價于振武糴三千石【回鶻自肅代以來以馬與中國互市隨其直而償其價}
牛稼穡之資中國禁人屠宰羊中國所鮮【鮮息淺翻}
出于北邊雜虜國家未嘗科調【調徒弔翻}
嗢沒斯自本國初破先投塞下不隨可汗已及二年慮彼猜嫌【彼謂烏介}
窮迫歸命前可汗正以猜虐無親致内離外叛今可汗失地遠客尤宜深矯前非若復骨肉相殘【復扶又翻下同}
則可汗左右信臣誰敢自保朕務在兼愛已受其降【謂受嗢沒斯降也}
于可汗不失恩慈于朝廷免虧信義豈不兩全事體深叶良圖 嗢沒斯入朝六月甲申以嗢沒斯所部為歸義軍以嗢沒斯為左金吾大將軍充軍使 門下侍郎同平章事陳夷行罷為左僕射秋七月以尚書右丞李讓夷為中書侍郎同平章事 嵐州人田滿川據州城作亂劉沔討誅之 嗢沒斯請置家太原與諸弟竭力扞邊詔劉沔存撫其家烏介可汗復遣其相上表借兵助復國又借天德城詔不許初可汗往來天德振武之間剽掠羌渾【剽正妙翻}
又屯頭烽北【宋白曰頭烽在朔州}
朝廷屢遣使諭之使還漠南可汗不奉詔李德裕以為那頡啜屯于山北烏介恐其與奚契丹連謀邀遮故不敢遠離塞下【離力智翻}
望敕張仲武諭奚契丹與回鶻共滅那頡啜使得北還及那頡啜死可汗猶不去議者又以為回鶻待馬價詔盡以馬價給之又不去八月可汗帥衆過頭烽南突入大同川驅掠河東雜虜牛馬數萬轉鬭至雲州城門【宋白曰雲州古平城之地北至長城三百里即蕃界今大元大同府治大同縣領雲中白登二縣又有雲内州領柔服蠻川二縣}
刺史張獻節閉城自守吐谷渾党項皆挈家入山避之庚午詔陳許徐汝襄陽等兵屯太原及振武天德俟來春驅逐回鶻 【考異曰實錄六月回鶻寇雲州劉沔出太原兵禦之又云劉沔救雲州為回鶻所敗七月又云烏介過天德至頭烽突入大同川驅太原部落牛馬數萬轉戰至雲州新紀正月回鶻寇横水撫畧天德振武軍三月回鶻寇雲朔六月劉沔及回鶻戰于雲州敗績按一品集奏回鶻事宜狀臣等見楊觀說緣回鶻赤心下兵馬多散在山北恐與奚契丹室韋同邀截可汗所以未敢遠去今因賜仲武詔令諭以朝旨緣回鶻曾有忠效又因殘破歸附國家朝廷事體須有存恤令奚契丹等與其同力討除赤心下散卒遣可汗漸出漢界免有滯留此狀雖無日月約須在楊觀自回鶻還赤心死那頡啜未敗前也又賜可汗書云一昨數使却回皆言可汗只待馬價及令交付之次又聞所止屢遷則是可汗邀求馬價而朝廷于此盡以給之也又七月十九日狀云望賜可汗書得嗢沒斯表稱在本國之時各有本分馬其馬價絹並合落下請充進奉以可汗本國殘破久在邉陲此已量與嗢沒斯優當其嗢沒斯以下本分馬價絹便賜可汗然則給其馬價必在七月十九日前當是時回鶻必未寇雲州敗劉沔突入大同川掠太原牛馬故朝廷曲徇所求欲其早離塞下北去尚未有攻討之意也又實錄八月壬戌朔李德裕奏請遣石雄斫營取公主擒可汗戊辰又奏斫營事令且住辛未詔陳許徐汝襄陽兵屯太原振武天德救援按一品集德裕論討襲回紇狀云臣頻奉聖旨緣回鶻漸逼杷頭烽早須討襲臣比知戎虜不解攻城只知馬上馳突臣料必無遊奕伏道又不會斫營儻令石雄以義武馬軍兼退渾馬騎精選步卒以為羽翼衘枚夜襲必易成功狀無月日實錄據七日狀云今月一日所商量石雄斫營事望且令住故置之朔日耳此時猶云漸逼杷頭烽則是尚未知過杷頭烽南也又八月七日論回鶻事宜狀云回鶻自至杷頭烽北已是數旬奏報寂然更無侵軼察其情狀只與在天德振武界首不殊臣等今月一日所商量石雄斫營事望且令住更審候事勢據此狀意則是殊未知可汗深入犯雲州也又八月十日請陳許等兵狀云臣等昨日已于延英面奏請太原振武天德各加兵備請更徵陳許徐汝襄陽等兵至河冰合時深慮可汗突出過河兼與吐蕃連結則為患不細深要防虞其所徵諸道兵恐不可停須令及冰未合各到所在然則回鶻突入大同川犯雲州必在八月之初一日七日猶未知九日始奏到故議兵守備驅逐實錄新紀皆誤今從舊紀}
丁丑賜嗢沒斯與其弟阿歷支習勿啜烏羅思皆姓李氏名思忠思貞思義思禮【嗢沒斯曰思忠阿歷支曰思貞習勿啜曰思義烏羅思曰思禮 考異曰舊紀六月嗢沒斯等至京師制以嗢沒斯充歸義軍使賜姓名李思忠以回鶻宰相受邪勿為歸義軍副使賜姓名李弘順舊回鶻傳曰二年冬三年春回鶻七部共三萬衆相次降于幽州詔配諸道有嗢沒斯受邪勿等諸部降振武皆賜姓李氏及名思忠思貞思義今從實錄}
國相愛邪勿姓愛名弘順仍以弘順為歸義軍副使上遣回鶻石誡直還其國賜可汗書 【考異曰舊紀此詔在劉沔張仲武為招討使下按一品集八月十八日狀兩日來臣等竊聞外議云石誡直久在京城事無巨細靡不諳悉昨緣收入鴻臚懼朝廷處置因求奉使意在脫身又云石誡直先有兩男逃走必是已入回鶻料其此去豈肯盡心伏望速詔劉沔所在勒迥然則遣石誡直賜可汗書必在此狀之前未知後來果曾勒回否也}
諭以自彼國為紇吃斯所破【戛斯一名紇吃斯蓋語音相近}
來投邊境撫納無所不至今可汗尚此近塞【近其靳翻}
未議還蕃或侵掠雲朔等州或鈔擊羌渾諸部【鈔楚交翻}
遥揣深意似恃姻好之情【謂質太和公主以邀中國揣初委翻好呼到翻}
每觀蹤由時懷馳突之計中外將相咸請誅翦朕情深屈己未忍幸災可汗宜速擇良圖無貽後悔上又命李德裕代劉沔答回鶻相頡干迦斯書以為回鶻遠來依投當效呼韓邪遣子入侍身自入朝【呼韓邪事見漢宣帝紀}
及令太和公主入謁太皇太后求哀乞憐則我之救卹無所愧懷【言無所愧于懷也}
而乃睥睨邊城桀驁自若【睥匹詣翻睨研計翻驁五到翻}
邀求過望如在本蕃又深入邊境侵暴不已求援繼好【好呼到翻}
豈宜如是來書又云胡人易動難安若令忿怒不可復制【復扶又翻下同}
回鶻為紇吃斯所破舉國將相遺骸弃于草莽累代可汗墳墓隔在天涯回鶻忿怒之心不施于彼【彼謂紇吃斯}
而蔑弃仁義逞志中華天地神祇豈容如此昔郅支不事大漢竟自夷滅【事見漢宣帝元帝紀}
往事之戒得不在懷戊子李德裕等上言若如前詔河東等三道嚴兵守備【三道河東盧龍振武也}
俟來春驅逐乘回鶻人困馬羸之時【羸倫為翻}
又官軍免盛寒之苦則幽州兵宜令止屯本道以俟詔命若慮河冰既合回鶻復有馳突須早驅逐則當及天時未寒決策于數月之間以河朔兵益河東兵必令收功于兩月之内今聞外議紛紜互有異同儻一不詢羣情終為浮辭所撓【撓奴教翻又奴巧翻}
望令公卿集議詔從之時議者多以為宜俟來春九月以劉沔兼招撫回鶻使如須驅逐其諸道行營兵權令指揮以張仲武為東面招撫回鶻使其當道行營兵及奚契丹室韋等並自指揮以李思忠為河西党項都將回鶻西南面招討使【此河西謂河北之西}
皆會軍于太原令沔屯鴈門關【鴈門關在代州鴈門縣即陘嶺關}
初奚契丹羈屬回鶻各有監使歲督其貢賦且詗唐事【監古衘翻使疏吏翻詗火迥翻又翾正翻}
張仲武遣牙將石公緒統二部盡殺回鶻監使等八百餘人仲武破那頡啜得室韋酋長妻子【酋慈由翻長知丈翻}
室韋以金帛羊馬贖之仲武不受曰但殺監使則歸之癸卯李德裕等奏河東奏事官孫儔適至云回鶻移營近南四十里【近其靳翻}
劉沔以為此必契丹不與之同恐為其掩襲故也據此事勢正堪驅除臣等問孫儔若與幽州合勢迫逐回鶻更須益幾兵儔言不須多益兵唯大同兵少得易定千人助之足矣上皆從之詔河東幽州振武天德各出大兵移營稍前以迫回鶻 上聞太子少傅白居易名欲相之【易以䜴翻相息亮翻}
以問李德裕德裕素惡居易【惡烏路翻}
乃言居易衰病不任朝謁【任音壬}
其從父弟左司員外郎敏中辭學不減居易且有器識甲辰以敏中為翰林學士【為敏中排德裕張本}
 李思忠請與契苾沙陀吐谷渾六千騎合勢擊回鶻乙巳以銀州刺史何清朝蔚州刺史契苾通分將河東蕃兵詣振武受李思忠指揮通何力之五世孫【契苾種帳太和中附于振武契苾何力太宗時來朝遂留宿衛蔚紆勿翻契欺訖翻}
冬十月丁卯立皇子峴為益王岐為兖王 戛斯

  遣將軍踏布合祖等至天德軍言先遣都呂施合等奉公主歸之大唐至今無聲問不知得達或為姧人所隔今出兵求索【索山客翻}
上天入地期于必得【上時掌翻}
又言將徙就合羅川居回鶻故國【回鶻舊居薛延陀北娑陵水上去長安七千里開元中破突厥徙牙烏德鞬山昆河之間南距漢高闕塞一千七百里}
兼已得安西北庭蒙古等五部落【李心傳曰蒙古之先世居斡難河靺鞨之後也靺鞨本臣高麗唐滅高麗其遺人迸入勃海惟黑水完彊及勃海盛靺鞨皆役屬後為奚契丹所攻部族分散其居混同江之上者其部落最為強盛焉其居隂山者自號為蒙古蒙古之人皆勇悍善戰其近漢地者性情多馴良尚能種秫穄以平底瓦釡煮而食之其遠者風俗多強武以射獵為生無器甲矢貫骨鏃而已余謂李心傳蜀人也安能知直北事特以所傳聞書之}
 十一月辛卯朔昭義節度使劉從諫上言請出部兵五千討回鶻詔不許 上遣使賜太和公主冬衣命李德裕為書賜公主畧曰先朝割愛降婚義寧家國謂回鶻必能禦侮安靜塞垣今回鶻所為甚不循理每馬首南向姑得不畏高祖太宗之威靈欲侵擾邊疆豈不思太皇太后之慈愛為其國母足得指揮若回鶻不能稟命則是弃絶姻好今日已後不得以姑為詞【太和公主憲宗女也于上為姑}
 上幸涇陽校獵乙卯諫議大夫高少逸鄭朗于閤中諫曰陛下比來遊獵稍頻【比毗志翻}
出城太遠侵星夜歸萬機曠廢上改容謝之少逸等出上謂宰相曰本置諫官使之論事朕欲時時聞之宰相皆賀己未以少逸為給事中朗為左諫議大夫 劉沔張仲武固稱盛寒未可進兵請待歲首【唐以建寅之月為歲首欲待來春進兵}
李忠順獨請與李思忠俱進十二月丙寅李德裕奏請遣思忠進屯保大柵從之 丁卯吐蕃遣其臣論普熱來告達磨贊普之喪【會要會昌三年贊普卒至十二月遣論贊等來告喪 考異曰實錄丁卯吐蕃贊普卒遣使告喪廢朝三日贊普立僅三十餘年有心疾不知國事委政大臣焉命將作少監李景為弔祭使據補國史彛泰卒後又有達磨贊普此年卒者達磨也文宗實錄不書彛泰贊普卒舊傳及續會要亦皆無達磨新書據補國史疑文宗實錄闊畧故它書皆因而誤彛泰以元和十一年立至此二十七年然開成三年已卒達磨立至此五年而實錄云僅三十年亦是誤以達磨為彛泰也}
命將作少監李璟為弔祭使 劉沔奏移軍雲州 李忠順奏擊回鶻破之 丙戌立皇子嶧為德王嵯為昌王【嶧音亦嵯才何翻}
 初吐蕃達磨贊普有佞幸之臣以為相達磨卒無子佞相立其妃綝氏兄尚延力之子乞離胡為贊普【綝丑林翻}
纔三歲佞相與妃共制國事吐蕃老臣數十人皆不得預政事首相結都那見乞離胡不拜曰贊普宗族甚多而立綝氏子國人誰服其令鬼神誰饗其祀國必亡矣比年災異之多乃為此也【比毗至翻為于偽翻}
老夫無權不得正其亂以報先贊普之德有死而已拔刀剺面慟哭而出佞相殺之滅其族國人憤怒又不遣使詣唐求冊立洛門川討擊使論恐熱【洛門川在渭州隴西縣東南漢來歙破隗純于落門即此 考異曰補國史曰恐熱姓末名農力吐蕃國法不呼本姓但王族則曰論官族則曰尚其中字即蕃號也熱者例皆言之如中華呼郎}
性悍忍多詐謀乃屬其徒告之曰【屬之欲翻聚會其徒也}
賊捨國族立綝氏專害忠良以脅衆臣且無大唐冊命何名贊普吾當與汝屬舉義兵入誅綝妃及用事者以正國家天道助順功無不成遂說三部落得萬騎【三部落蕃種落之分居河隴者或云吐渾党項嗢末說式芮翻}
是歲與青海節度使同盟舉兵自稱國相至渭州遇國相尚思羅屯薄寒山恐熱擊之思羅弃輜重西奔松州【王涯曰從龍州青川鎮入吐蕃界直抵故松州之城是吐蕃舊置節度之所}
恐熱遂屠渭州思羅蘇毗吐谷渾羊同等兵合八萬保洮水焚橋拒之【洮土刀翻}
恐熱至隔水語蘇毗等曰【語牛倨翻}
賊臣亂國天遣我來誅之汝曹奈何助逆我今已為宰相國内兵我皆得制之汝不從將滅汝部落蘇毗等疑不戰恐熱引驍騎涉水蘇毗等皆降思羅西走追獲殺之恐熱盡併其衆合十餘萬自渭州至松州所過殘滅尸相枕藉【枕職任翻藉慈夜翻}


  資治通鑑卷二百四十六  
    


 


 



 

 
  







 


  
  
 
 
 


  

 















	
	









































 
  



















 





 












  
  
  

 





