<!DOCTYPE html PUBLIC "-//W3C//DTD XHTML 1.0 Transitional//EN" "http://www.w3.org/TR/xhtml1/DTD/xhtml1-transitional.dtd">
<html xmlns="http://www.w3.org/1999/xhtml">
<head>
<meta http-equiv="Content-Type" content="text/html; charset=utf-8" />
<meta http-equiv="X-UA-Compatible" content="IE=Edge,chrome=1">
<title>資治通鑒_188-資治通鑑卷一百八十七_188-資治通鑑卷一百八十七</title>
<meta name="Keywords" content="資治通鑒_188-資治通鑑卷一百八十七_188-資治通鑑卷一百八十七">
<meta name="Description" content="資治通鑒_188-資治通鑑卷一百八十七_188-資治通鑑卷一百八十七">
<meta http-equiv="Cache-Control" content="no-transform" />
<meta http-equiv="Cache-Control" content="no-siteapp" />
<link href="/img/style.css" rel="stylesheet" type="text/css" />
<script src="/img/m.js?2020"></script> 
</head>
<body>
 <div class="ClassNavi">
<a  href="/24shi/">二十四史</a> | <a href="/SiKuQuanShu/">四库全书</a> | <a href="http://www.guoxuedashi.com/gjtsjc/"><font  color="#FF0000">古今图书集成</font></a> | <a href="/renwu/">历史人物</a> | <a href="/ShuoWenJieZi/"><font  color="#FF0000">说文解字</a></font> | <a href="/chengyu/">成语词典</a> | <a  target="_blank"  href="http://www.guoxuedashi.com/jgwhj/"><font  color="#FF0000">甲骨文合集</font></a> | <a href="/yzjwjc/"><font  color="#FF0000">殷周金文集成</font></a> | <a href="/xiangxingzi/"><font color="#0000FF">象形字典</font></a> | <a href="/13jing/"><font  color="#FF0000">十三经索引</font></a> | <a href="/zixing/"><font  color="#FF0000">字体转换器</font></a> | <a href="/zidian/xz/"><font color="#0000FF">篆书识别</font></a> | <a href="/jinfanyi/">近义反义词</a> | <a href="/duilian/">对联大全</a> | <a href="/jiapu/"><font  color="#0000FF">家谱族谱查询</font></a> | <a href="http://www.guoxuemi.com/hafo/" target="_blank" ><font color="#FF0000">哈佛古籍</font></a> 
</div>

 <!-- 头部导航开始 -->
<div class="w1180 head clearfix">
  <div class="head_logo l"><a title="国学大师官网" href="http://www.guoxuedashi.com" target="_blank"></a></div>
  <div class="head_sr l">
  <div id="head1">
  
  <a href="http://www.guoxuedashi.com/zidian/bujian/" target="_blank" ><img src="http://www.guoxuedashi.com/img/top1.gif" width="88" height="60" border="0" title="部件查字,支持20万汉字"></a>


<a href="http://www.guoxuedashi.com/help/yingpan.php" target="_blank"><img src="http://www.guoxuedashi.com/img/top230.gif" width="600" height="62" border="0" ></a>


  </div>
  <div id="head3"><a href="javascript:" onClick="javascript:window.external.AddFavorite(window.location.href,document.title);">添加收藏</a>
  <br><a href="/help/setie.php">搜索引擎</a>
  <br><a href="/help/zanzhu.php">赞助本站</a></div>
  <div id="head2">
 <a href="http://www.guoxuemi.com/" target="_blank"><img src="http://www.guoxuedashi.com/img/guoxuemi.gif" width="95" height="62" border="0" style="margin-left:2px;" title="国学迷"></a>
  

  </div>
</div>
  <div class="clear"></div>
  <div class="head_nav">
  <p><a href="/">首页</a> | <a href="/ShuKu/">国学书库</a> | <a href="/guji/">影印古籍</a> | <a href="/shici/">诗词宝典</a> | <a   href="/SiKuQuanShu/gxjx.php">精选</a> <b>|</b> <a href="/zidian/">汉语字典</a> | <a href="/hydcd/">汉语词典</a> | <a href="http://www.guoxuedashi.com/zidian/bujian/"><font  color="#CC0066">部件查字</font></a> | <a href="http://www.sfds.cn/"><font  color="#CC0066">书法大师</font></a> | <a href="/jgwhj/">甲骨文</a> <b>|</b> <a href="/b/4/"><font  color="#CC0066">解密</font></a> | <a href="/renwu/">历史人物</a> | <a href="/diangu/">历史典故</a> | <a href="/xingshi/">姓氏</a> | <a href="/minzu/">民族</a> <b>|</b> <a href="/mz/"><font  color="#CC0066">世界名著</font></a> | <a href="/download/">软件下载</a>
</p>
<p><a href="/b/"><font  color="#CC0066">历史</font></a> | <a href="http://skqs.guoxuedashi.com/" target="_blank">四库全书</a> |  <a href="http://www.guoxuedashi.com/search/" target="_blank"><font  color="#CC0066">全文检索</font></a> | <a href="http://www.guoxuedashi.com/shumu/">古籍书目</a> | <a   href="/24shi/">正史</a> <b>|</b> <a href="/chengyu/">成语词典</a> | <a href="/kangxi/" title="康熙字典">康熙字典</a> | <a href="/ShuoWenJieZi/">说文解字</a> | <a href="/zixing/yanbian/">字形演变</a> | <a href="/yzjwjc/">金 文</a> <b>|</b>  <a href="/shijian/nian-hao/">年号</a> | <a href="/diming/">历史地名</a> | <a href="/shijian/">历史事件</a> | <a href="/guanzhi/">官职</a> | <a href="/lishi/">知识</a> <b>|</b> <a href="/zhongyi/">中医中药</a> | <a href="http://www.guoxuedashi.com/forum/">留言反馈</a>
</p>
  </div>
</div>
<!-- 头部导航END --> 
<!-- 内容区开始 --> 
<div class="w1180 clearfix">
  <div class="info l">
   
<div class="clearfix" style="background:#f5faff;">
<script src='http://www.guoxuedashi.com/img/headersou.js'></script>

</div>
  <div class="info_tree"><a href="http://www.guoxuedashi.com">首页</a> > <a href="/SiKuQuanShu/fanti/">四库全书</a>
 > <h1>资治通鉴</h1> <!--         下载:【右键另存为】即可 --></div>
  <div class="info_content zj clearfix">
  
<div class="info_txt clearfix" id="show">
<center style="font-size:24px;">188-資治通鑑卷一百八十七</center>
    資治通鑑卷一百八十七 宋 司馬光 撰<br />
<br />
  胡三省 音註<br />
<br />
  唐紀三【起屠維單閼正月盡十月不滿一年】<br />
<br />
  高祖神堯大聖光孝皇帝上之下<br />
<br />
  武德二年春正月壬寅王世充悉取隋朝顯官名士為太尉府官屬【朝直遥翻下同】杜淹戴胄皆預焉胄安陽人也【安陽縣帶相州】隋將軍王隆帥屯衛將軍張鎮周【煬帝改左右領軍衛為左右屯衛帥讀曰率考異曰高祖實錄作鎮州今從隋書陳稜傳】都水少監蘇世長等以山南兵始至東都【義寧元年七月遣王隆會兵東都今始至少始照翻】王世充專總朝政事無大小悉關太尉府臺省監署莫不闃然【闃苦鵙翻】世充立三牌於府門外一求文學才識堪濟時務者一求武勇智畧能摧鋒陷敵者一求身有寃滯擁抑不申者於是上書陳事日有數百世充悉引見躬自省覽【省悉景翻】殷勤慰諭人人自喜以為言聽計從然終無所施行下至士卒厮養【析薪為厮炊烹為養厮音斯養羊尚翻】世充皆以甘言悅之而實無恩施【施式智翻】隋馬軍總管獨孤武都為世充所親任其從弟司隸大夫機【煬帝置司隸臺以大夫為之長掌諸巡察正四品從才用翻】與虞部郎楊恭慎【六典周禮地官有山虞澤虞蓋虞部之職也魏始有虞曹郎中晉因之梁陳為侍郎後周冬官有虞部下大夫梁陳後魏北齊並祠部尚書領之隋工部尚書領之煬帝曰虞曹郎】前勃海郡主簿孫師孝【煬帝改滄州為勃海郡】步兵總管劉孝元李儉崔孝仁謀召唐兵使孝仁說武都曰王公徒為兒女之態以悅下愚而鄙隘貪忍不顧親舊豈能成大業哉圖䜟之文應歸李氏人皆知之【說輸芮翻下說同䜟楚譖翻】唐起晉陽奄有關内兵不留行英雄景附且坦懷待物舉善責功不念舊惡據勝勢以爭天下誰能敵之吾屬託身非所坐待夷滅今任管公兵近在新安【任瓌以穀州刺史鎮新安封管國公任音壬】又吾之故人也若遣間使召之使夜造城下【間使上古莧翻下疏吏翻造七到翻】吾曹共為内應開門納之事無不集矣武都從之事泄世充皆殺之恭慎達之子也【遠隋觀德王雄之弟】癸卯命秦王世民出鎮長春宫【長春宫在同州朝邑縣後周宇文護所建】宇文化及攻魏州總管元寶藏四旬不克魏徵往說<br />
<br />
  之丁未寶藏舉州來降【魏徵本元寶藏官屬說式芮翻降戶江翻下同】 戊午淮安王神通擊宇文化及於魏縣化及不能抗東走聊城【聊城縣時屬魏州武德四年分為博州】神通拔魏縣斬獲二千餘人引兵追化及至聊城圍之 甲子以陳叔達為納言 丙寅李密所置伊州刺史張善相來降【相息亮翻降戶江翻】 朱粲有衆二十萬剽掠漢淮之間【剽匹妙翻】遷徙無常每破州縣食其積粟未盡復他適【復扶又翻下同】將去悉焚其餘資又不務稼穡民餒死者如積粲無可復掠軍中乏食乃敎士卒烹婦人嬰兒噉之【噉徒濫翻又徒覽翻】曰肉之美者無過於人但使佗國有人何憂於餒隋著作佐郎陸從典通事舍人顔愍楚【六典著作佐郎修國史宋百官春秋云常道鄉公咸熙百官名有著作佐郎三人晉制著作佐郎始到職必撰名臣傳一人宋氏之初國朝始建未有合撰者此制遂替後周春官府置著作中士即著作佐郎之任通事舍人即秦之謁者漢書百官表謁者掌賓贊受事舊儀云謁者有缺選郎中美鬚眉大音者補晉初中書置舍人通事各一人東晉令舍人通事兼謁者之任通事舍人之名由此始也隋初罷謁者官置通事舍人煬帝改通事舍人為通事謁者顏愍楚蓋大業前為舍人】謫官在南陽【南陽鄧州】粲初引為賓客其後無食闔家皆為所噉愍楚之推之子也【顔之推仕於高齊之季】又稅諸城堡細弱以供軍食諸城堡相帥叛之【帥讀曰率下同】淮安土豪楊士林田瓚起兵攻粲【後魏置東荆州於比陽西魏改為淮州梁置淮安縣於桐柏幷立上川郡隋開皇廢郡改淮安為桐柏縣改淮安郡曰顯州領比陽平氏桐柏等七縣大業改顯州為淮安郡瓚藏旱翻】諸州皆應之粲與戰于淮源【水經注淮水出平氏縣桐柏大復山山南有淮源廟唐州桐柏淮源縣碑漢延熹六年立其文曰淮出平氏始於大復潛行地中見於陽口】大敗帥餘衆數千奔菊潭【帥讀曰率菊潭舊曰酈縣開皇初改焉時屬鄧州山有菊人飲其水多壽故以名縣】士林家世蠻酋【酋慈由翻】隋末士林為鷹揚府校尉殺郡官而據其郡【校戶敎翻】既逐朱粲己巳帥漢東四郡遣使詣信州總管廬江王瑗請降【大業改隋州為漢東郡梁置信州於魚復大業改為巴東郡唐復為信州使疏吏翻瑗于眷翻】詔以為顯州道行臺【宋白曰後魏置東荆州於比陽後改淮州隋文帝改顯州取界内顯望岡為名】士林以瓚為長史 初王世充既殺元盧【元盧元文都盧楚世充殺之事見一百八十五卷元年七月】慮人情未服猶媚事皇泰主禮甚謙敬又請為劉太后假子尊號曰聖感皇太后既而漸驕横嘗賜食於禁中還家大吐【横戶孟翻吐土故翻】疑遇毒自是不復朝謁【復扶又翻朝直遥翻】皇泰主知其終不為臣而力不能制唯取内庫綵物大造幡花又出諸服玩令僧散施貧乏以求福【施式智翻】世充使其黨張績董濬守章善顯福二門【東都皇城南面三門中曰應天左曰興教右曰光政興教之内曰會昌其北曰章善光政之内曰廣運其北曰顯福】宫内雜物毫釐不得出是月世充使人獻印及劒又言河水清欲以耀衆為已符瑞云 上遣金紫光祿大夫武功靳孝謨安集邊郡【靳居焮翻】為梁師都所獲孝謨罵之極口師都殺之二月詔追賜爵武昌縣公謚曰忠初定租庸調法每丁租二石絹二匹綿三兩【租庸調之法以人丁為本梁陳齊周各有損益唐制凡授田者丁歲輸粟二斛稻三斛謂之租丁隨鄉所出歲輸絹二匹綾絶二丈布加五之一綿三兩麻三斤非蠶鄉則輸銀十四兩謂之調用人之力歲二十日閏加二日不役者日為絹三尺謂之庸有事而加役二十五日者免調三十日者租調皆免通正役不過五十日調徒釣翻下同】自兹以外不得横有調歛【横戶孟翻歛力瞻翻】 丙戍詔諸宗姓居官者在同列之上未仕者免其徭役每州置宗師一人以攝總别為團伍 張俟德至涼【去年八月遣張俟德冊拜李軌】李軌召其羣臣廷議曰唐天子吾之從兄【從才用翻下同】今已正位京邑一姓不可自爭天下吾欲去帝號受其封爵可乎【去羌呂翻】曹珍曰隋失其鹿天下共逐之稱王稱帝者奚啻一人唐帝關中凉帝河右固不相妨且已為天子奈何復自貶黜【復扶又翻】必欲以小事大請依蕭詧事魏故事【蕭詧事魏事見一百六十五卷梁元帝承聖三年】軌從之戊戌軌遣其尚書左丞鄧曉入見【見賢遍翻】奉書稱皇從弟大凉皇帝臣軌而不受官爵帝怒拘曉不遣始議興師討之初隋煬帝自征吐谷渾吐谷渾可汗伏允以數千騎奔党項【事見一百八十一卷煬帝大業五年吐從暾入聲谷音浴可從刋入聲汗音寒騎奇寄翻】煬帝立其質子順為主【質音致】使統餘衆不果入而還會中國喪亂【還從宣翻又如字喪息浪翻】伏允復還收其故地【復扶又翻】上受禪順自江都還長安【煬帝既弑順逃還長安】上遣使與伏允連和使擊李軌許以順還之伏允喜起兵擊軌數遣使入貢請順上遣之【為後太宗立順以統吐谷渾之衆張本遣使疏吏翻下同】 閏月朱粲遣使請降【降戶江翻】詔以粲為楚王聽自置官屬以便宜從事 宇文化及以珍貨誘海曲諸賊賊帥王薄帥衆從之【誘羊久翻賊帥所類翻薄帥讀曰率】與共守聊城竇建德謂其羣下曰吾為隋民隋為吾君今宇文化及弑逆乃吾讎也吾不可以不討乃引兵趣聊城【趣七喻翻又逡須翻】淮安王神通攻聊城化及糧盡請降神通不許安撫副使崔世幹勸神通許之【降戶江翻下同使疏吏翻】神通曰軍士暴露日久賊食盡計窮克在旦暮吾當攻取以示國威且散其玉帛以勞將士【勞力到翻】若受其降將何以為軍賞乎世幹曰今建德方至若化及未平内外受敵吾軍必敗夫不攻而下之為功甚易【夫音扶易以豉翻】奈何貪其玉帛而不受乎神通怒囚世幹於軍中【去年十月遣神通安撫山東書崔民幹為副今書世幹當有一誤】既而宇文士及自濟北餽之【齊北郡濟州濟子禮翻】化及軍稍振遂復拒戰【復扶又翻下同】神通督兵攻之貝州刺史趙君德攀堞先登【時復以清河郡為貝州宋白曰貝州清河郡春秋為晉東陽之地亦為齊境秦為鉅鹿郡地漢分鉅鹿郡置清河郡理清陽石趙移郡理平晉城即今博州清平縣後周平齊於清河縣置貝州清河後漢之甘陵清陽縣又兼有漢貝丘縣之地貝州以此得名堞徒協翻】神通心害其功收兵不戰君德大詬而下【詬苦候翻】遂不克建德軍且至神通引兵退建德與化及連戰大破之化及復保聊城建德縱兵四面急攻王薄開門納之建德入城生擒化及先謁隋蕭皇后語皆稱臣素服哭煬帝盡哀收傳國璽及鹵簿儀仗【璽斯氏翻】撫存隋之百官然後執逆黨宇文智及楊士覽元武達許弘仁孟景集隋官而斬之梟首軍門之外【梟堅堯翻】以檻車載化及并二子承基承趾至襄國斬之【煬帝改邢州為襄國郡杜佑曰邢州古邢國治龍岡縣秦為信都項羽改襄國隋改龍岡 考異曰隋書云載之河間斬之唐書云至大陸斬之河洛記云建德將化及并蕭后南陽公主隨軍于時襄國郡尚為隋守建德因其迴兵欲攻之營於城下遣大理官引化及出營東南二里許宣令數其罪并二子一號魏王一號蜀王同時受戮按蜀王乃士及所封今不取】化及且死更無餘言但云不負夏王【夏戶雅翻】建德每戰勝克城所得資財悉以分將士身無所取又不噉肉常食蔬茹粟飯妻曹氏不衣紈綺【將即亮翻噉徒覽翻又徒濫翻衣於既翻紈音丸綺區几翻】所役婢妾纔十許人及破化及得隋宫人千數即時散遣之以隋黄門侍郎裴矩為左僕射掌選事【選宣絹翻】兵部侍郎崔君肅為侍中【考異曰革命記作君秀今從舊建德傳】少府令何稠為工部尚書【漢書百官表少府秦官至北齊不置少府以其屬官併太府寺隋煬帝大業三年始分太府為少府監置監少監其後改監為令少監為少令少詩照翻】右司郎中柳調為左丞【六典左右司郎中前代不置煬帝三年尚書都司始置左右司郎各一人掌都省之職品同諸曹郎從五品司馬彪續漢書云尚書丞一人秦所置漢因之成帝置列曹尚書更置丞四人至光武減其二惟置左右丞各一人丞者承也言承助令僕總理臺事也】虞世南為黄門侍郎歐陽詢為太常卿詢紇之子也【歐陽紇見一百七十卷陳宣帝太建元年紇下没翻】自餘隨才授職委以政事其不願留欲詣關中及東都者亦聽之仍給資糧以兵援之出境隋驍果尚近萬人亦各縱遣任其所之【驍堅堯翻近其靳翻】又與王世充結好【好呼到翻】遣使奉表於隋皇泰主【使疏吏翻】皇泰主封為夏王建德起於羣盜雖建國未有文物法度裴矩為之定朝儀制律令【為于偽翻朝直遥翻】建德甚悦每從之諮訪典禮 甲辰上考第羣臣以李綱孫伏伽為第一【伽求加翻】因置酒高會謂裴寂等曰隋氏以主驕臣諂亡天下朕即位以來每虛心求諫然惟李綱差盡忠欵孫伏伽可謂誠直餘人猶踵敝風俛眉而已豈朕所望哉朕視卿如愛子卿當視朕如慈父有懷必盡勿自隱也因命捨君臣之敬極歡而罷 遣前御史大夫段確使於朱粲【使疏吏翻】 初上為隋殿内少監【少始照翻】宇文士及為尚輦奉御【隋尚輦局屬殿内省】上與之善士及從化及至黎陽上手詔召之士及潛遣家僮間道詣長安又因使者獻金環【金環言欲還長安問古莧翻使疏吏翻】化及至魏縣兵勢日蹙士及勸之歸唐化及不從内史令封德彝說士及於濟北徵督軍糧以觀其變【說輸芮翻濟子禮翻】化及稱帝立士及為蜀王化及死士及與德彝自濟北來降【降戶江翻】時士及妹為昭儀由是授上儀同上以封德彝隋室舊臣而諂巧不忠深誚責之【誚才笑翻】罷遣就舍德彝以祕策干上上悅尋拜内史舍人俄遷侍郎 甲寅隋夷陵郡丞安陸許紹帥黔安武陵澧陽等諸郡來降【梁置宜州於夷陵郡西魏改曰拓州後周改曰峽州宋白曰周武帝以州扼三峽之口故名隋煬帝改為夷陵郡改安州為安陸郡黔州為黔安郡武陵郡梁置武州後改曰沅州隋改曰朗州大業復為郡澧陽隋初置澧州大業改為郡帥讀曰率澧音禮 考異曰舊書傳云世充簒位乃來降按世充簒在四月實錄紹降在此今從之】紹幼與帝同學詔以紹為峽州刺史賜爵安陸公 丙辰以徐世勣為黎州總管【後魏置黎陽郡於黎陽縣後置黎州至隋州郡並廢以黎陽縣屬汲郡今復置黎州】 丁巳驃騎將軍張孝珉以勁卒百人襲王世充汜水城入其郛沈米船百五十艘【隋志汜水縣舊曰成皋即虎牢也後魏置東中府東魏置北豫州後周置滎州開皇初曰鄭州十八年改成皋曰汜水驃匹妙翻騎奇寄翻下同汜音祀沈持林翻艘蘇遭翻】己未世充寇穀州世充以秦叔寶為龍驤大將軍【驤思】<br />
<br />
  【將翻】程知節為將軍待之皆厚然二人疾世充多詐知節謂叔寶曰王公器度淺狹而多妄語好為呪誓此乃老巫嫗耳【好呼到翻呪職救翻嫗威遇翻】豈撥亂之主乎世充與唐兵戰於九曲【水經注洛水自宜陽而東逕九曲南其地十里有阪九曲穆天子傳所謂天子西征升于九阿此是也洛水又東與豪水會豪水出新安縣密山南流歷九曲東而南流入于洛舊志熊州壽安縣義寧元年移治九曲城】叔寶知節皆將兵在陳【將即亮翻陳讀曰陣】與其徒數十騎西馳百許步下馬拜世充曰僕荷公殊禮【荷下可翻】深思報効公性猜忌喜信讒言【喜許既翻】非僕託身之所今不能仰事請從此辭遂躍馬來降【降戶江翻 考異曰河洛記二月王世充將兵圍新安將軍程咬金帥其徒以歸義按新安乃穀州也而梁載言十道志九曲在壽安壽安乃熊州也或者世充亦寇熊州乎】世充不敢逼上使事秦王世民世民素聞其名厚禮之以叔寶為馬軍總管知節為左三統軍時世充驍將又有驃騎武安李君羨【隋改洺州為武安郡驍堅堯翻將即亮翻下守將同驃匹妙翻騎奇寄翻】征南將軍臨邑田留安【隋臨邑縣屬齊郡】亦惡世充之為人【惡烏路翻】帥衆來降【帥讀曰率降戶江翻】世民引君羨置左右以留安為右四統軍 王世充囚李育德之兄厚德於獲嘉【獲嘉漢縣武帝廵幸至此聞破南越獲呂嘉因以名縣隋屬河内郡】厚德與其守將趙君穎逐殷州刺史段大師以城來降以厚德為殷州刺史【隋開皇十六年於獲嘉縣置殷州大業初州廢王世充復置今因以命厚德】 竇建德陷邢州執總管陳君賓 上遣殿内監竇誕【隋煬帝置殿内監諱中改為内唐為殿中監是時蓋未改隋官名也】右衛將軍宇文歆【秦漢始置衛將軍晉武帝分為左右二衛下至隋初皆因之煬帝改左右翊衛唐復舊歆許金翻】助幷州總管齊王元吉守晉陽誕抗之子也【竇抗后兄也幷卑盈翻】尚帝女襄陽公主元吉性驕侈奴客婢妾數百人好使之被甲戲為攻戰【好呼到翻下同被皮義翻下同】前後死傷甚衆元吉亦嘗被傷其乳母陳善意苦諫元吉醉怒命壯士歐殺之【歐烏口翻】性好田獵載罔罟三十車嘗言我寧三日不食不能一日不獵常與誕遊獵蹂踐人禾稼【蹂人九翻】又縱左右奪民物當衢射人觀其避箭【射而亦翻】夜開府門宣淫他室百姓憤怨歆屢諫不納乃表言其狀壬戌元吉坐免官【為元吉失守張本】 癸亥陟州刺史李育德【按志是年育德以修武縣濁鹿城降置陟州因武陟為名】攻下王世充河内堡聚三十一所乙丑世充遣其兄子君廓侵陟州李育德擊走之靳首千餘級李厚德歸省親疾使李育德守獲嘉【李厚德蓋歸濁鹿省親而使育德守獲嘉也省悉景翻】世充併兵攻之丁卯城陷育德及弟三人皆戰死 己巳李公逸以雍丘來降拜杞州總管以其族弟善行為杞州刺史【隋志梁郡雍丘縣後魏置陽夏郡開皇廢郡置杞州大業廢州為縣李公逸因亂據之今復置州時邊要州置總管及刺史】 隋吏部侍郎楊恭仁從宇文化及至河北化及敗魏州總管元寶藏獲之己巳送長安上與之有舊拜黄門侍郎尋以為涼州總管恭仁素習邊事曉羌胡情偽民夷悅服自葱嶺已東並入朝貢【按恭仁至長安時李軌尚據河西唐未得涼州也安能遠及葱嶺乎史終言恭仁事耳是年五月安興貴執李軌方遣楊恭仁安撫河西朝直遥翻】 突厥始畢可汗將其衆度河至夏州【夏州漢朔方之地赫連所都統萬也魏滅赫連以為統萬鎮魏太和十一年置夏州因赫連故國名以名州隋大業改州為朔方郡今復以朔方郡為夏州厥九勿翻可從刋入聲汗音寒夏戶雅翻】梁師都發兵會之以五百騎授劉武周欲自句注入寇太原【騎奇寄翻句音鉤又如字又音拘】會始畢卒 【考異曰高祖實錄六月己酉始畢可汗卒疑遣使告喪月日也今從舊書本紀列傳傳】子什鉢苾幼未可立【苾毗必翻】立其弟俟利弗設為處羅可汗【俟渠之翻處昌呂翻】處羅以什鉢苾為尼步設使居東偏直幽州之北【直當也尼女夷翻】先是上遣右武候將軍高靜奉幣使於突厥【使疏吏翻】至豐州【復以五原郡為豐州豐州漢朔方臨戌縣地後周保定三年置永豐鎮隋開皇五年置豐州因鎮為名大業廢州為五原郡唐復為州大元以豐州置天德軍節度屬大同府路】聞始畢卒勑納於所在之庫突厥聞之怒欲入寇豐州總管張長遜遣高靜以幣出塞為朝廷致賻【為于偽翻賻音附貨財曰賻】突厥乃還【還從宣翻又如字】 三月庚午梁師都寇靈州長史楊則擊走之【邊要之州置總管刺史長史司馬長知兩翻】 壬申王世充寇穀州刺史史萬寶戰不利 庚辰隋北海通守鄭䖍符文登令方惠整及東海齊郡東平任城平陸壽張須昌賊帥王薄等並以其地來降【煬帝以青州為北海郡守式又翻文登縣屬東萊郡以海州為東海郡齊州為齊郡鄆州為東平郡須昌縣屬焉任城平陸二縣屬魯郡壽張縣屬濟北郡任音壬帥所類翻降戶江翻】 王世充之寇新安也外示攻取實召文武之附已者議受禪李世英深以為不可曰四方所以奔馳歸附東都者以公能中興隋室故也【中竹仲翻又如字】今九州之地未清其一遽正位號恐遠人皆思叛去矣世充曰公言是也長史韋節楊續等曰隋氏數窮在理昭然夫非常之事固不可與常人議之【夫音扶】太史令樂德融曰昔歲長星出乃除舊布新之徵【隋志大業十三年六月有星孛于太微五帝座色黄赤長三四尺許】今歲星在角亢亢鄭之分野【晉天文志自軫十二度至氐四度為壽星於辰在辰鄭之分野屬兗州陳卓范蠡鬼谷先生張良諸葛亮京房譙周張衡並云角亢氐鄭兗州亢音剛分扶問翻】若不亟順天道恐王氣衰息世充從之外兵曹參軍戴胄【外兵曹隋官無之世充取魏晉以來官制而置之耳】言於世充曰君臣猶父子也休戚同之明公莫若竭忠徇國則家國俱安矣世充詭辭稱善而遣之世充議受九錫胄復固諫【復扶又翻】世充怒出為鄭州長史使與兄子行本鎮虎牢【長知兩翻】乃使段達等言於皇泰主請加世充九錫皇泰主曰鄭公近平李密已拜太尉【事見去年九月十月】自是以來未有殊績俟天下稍平議之未晚段達曰太尉欲之皇泰主熟視達曰任公辛巳達等以皇泰主之詔命世充為相國假黄鉞總百揆進爵鄭王加九錫鄭國置丞相以下官 初宇文化及以隋大理卿鄭善果為民部尚書從至聊城為化及督戰中流矢【為化于偽翻下何為同中竹仲翻】竇建德克聊城王琮獲善果責之曰公名臣之家【鄭善果父誠討尉遟迴以力戰死由是為隋名臣家】隋室大臣奈何為弑君之賊効命苦戰傷痍至此乎善果大慙欲自殺宋正本馳往救止之建德復不為禮乃奔相州【相息亮翻】淮安王神通送之長安庚午善果至上優禮之拜左庶子檢校内史侍郎【隋室之臣若宇文士及鄭善果安可復用乎】 齊王元吉諷并州父老詣闕留己甲申復以元吉為并州總管【復扶又翻】 戊子淮南五州皆遣使來降【使疏吏翻降戶江翻】 辛卯劉武周寇并州 壬辰營州總管鄧暠擊高開道敗之【暠古浩翻敗補邁翻】 甲午王世充遣其將高毗寇義州【新志武德元年以衛州之汲新鄉置義州仍高齊舊州名也將即亮翻】 東都道士桓法嗣獻孔子閉房記於王世充言相國當代隋為天子【嗣祥吏翻相息亮翻】世充大悅以法嗣為諫議大夫世充又羅取雜鳥書帛繫頸自言符命而縱之有得鳥來獻者亦拜官爵於是段達以皇泰主命加世充殊禮世充奉表三讓百官勸進設位於都堂納言蘇威年老不任朝謁【任音壬朝直遥翻】世充以威隋氏重臣欲以眩耀士民每勸進必冠威名【冠古玩翻】及受殊禮之日扶威置百官之上然後南面正坐受之 夏四月劉武周引突厥之衆軍於黄蛇嶺【嶺在榆次縣北】兵鋒甚盛【厥九勿翻】齊王元吉使車騎將軍張達以步卒嘗寇【嘗試也騎奇寄翻】達辭以兵少不可往元吉強遣之【少詩沼翻強其兩翻】至則俱没達忿恨庚子引武周襲榆次陷之【榆次縣屬幷州漢古縣也】 散騎常侍段確性嗜酒奉詔慰勞朱粲於菊潭【是年二月段確以前御史大夫出使今書散騎常侍蓋續命之散悉亶翻騎奇寄翻勞力到翻】辛丑乘醉侮粲曰聞卿好噉人人作何味【好呼到翻噉徒濫翻又徒覽翻下同】粲曰噉醉人正如糟藏彘肉確怒罵曰狂賊入朝為一頭奴耳復得噉人乎粲於座收確及從者數十人悉烹之【朝直遥翻復扶又翻下同從才用翻】以噉左右遂屠菊潭奔王世充世充以為龍驤大將軍【驤思將翻】王世充令長史韋節楊續等及太常博士衡水孔穎達【長知兩翻衡水縣屬冀州本漢桃縣隋開皇十六年置衡水縣】造禪代儀遣段達雲定興等十餘人入奏皇泰主曰天命不常鄭王功德甚盛願陛下遵唐虞之迹皇泰主歛膝據按怒曰天下高祖之天下若隋祚未亡此言不應輒發必天命已改何煩禪讓公等或祖禰舊臣【禰乃禮翻】或台鼎高位既有斯言朕復何望顔色凛冽【言嚴冷也】在廷者皆流汗退朝泣對太后世充更使人謂之曰今海内未寧須立長君【朝直遥翻長知兩翻】俟四方安集當復子明辟必如前誓【謂去年七月禁中被髮之誓也】癸卯世充稱皇泰主命禪位于鄭遣其兄世惲幽皇泰主於含涼殿【惲於粉翻】雖有三表陳讓及敕書敦勸皇泰主皆不知也遣諸將引兵入清宫城又遣術人以桃湯葦火祓除禁省【將即亮翻下同祓敷勿翻】 隋將帥郡縣及賊帥前後繼有降者【帥所類翻降戶江翻下同】詔以王薄為齊州總管【齊州治歷城縣古歷下城也漢為歷城縣劉宋僑立冀州於此魏為濟南郡隋立濟州唐復隋初之舊】伏德為濟州總管鄭䖍符為青州總管綦公順為淮州總管【濟子禮翻宋白曰濟州古碻磝城也秦為東郡茌平地宋置碻磝戍及濟北郡後魏立濟州按綦公順本起北海新志云是年分青州之北海營丘下密置濰州盖以公順為濰州總管淮當作濰宋白曰濰州取界内濰水為名】王孝師為滄州總管【宋白曰滄州禹疏九河在此州界漢置勃海郡理浮陽後魏置滄州】 甲辰遣大理卿新樂郎楚之安撫山東【舊志曰新樂古鮮虞子國漢新市縣屬中山郡隋改為新樂縣唐屬定州宋白曰隋開皇十六年置新樂縣新樂者漢成帝時中山孝王母馮昭儀隨王就國建宫於西鄉呼為西樂城後語訛呼西為新故曰新樂樂音洛】祕書監夏侯端安撫淮左 乙巳王世充備法駕入宫即皇帝位丙午大赦改元開明 丁未隋禦衛將軍陳稜以江都來降以稜為揚州總管【揚州漢廣陵江都之地自漢以來揚州所治不常厥邑至唐廣陵始專有揚州之名】 戊申王世充立子玄應為太子玄恕為漢王餘兄弟宗族十九人皆為王奉皇泰主為潞國公以蘇威為太師段達為司徒雲定興為太尉張僅為司空楊續為納言韋節為内史【内史下當有令字】王隆為左僕射韋霽為右僕射齊王世惲為尚書令楊汪為吏部尚書杜淹為少吏部【少吏部即吏部侍郎惲於粉翻少始照翻】鄭頲為御史大夫【頲他鼎翻】世惲世充之兄也又以國子助教吳人陸德明為漢王師【晉武帝立國子學置助教掌佐博士敎授後世因之】令玄恕就其家行束修禮【論語孔子曰自行束修以上吾未嘗無誨焉朱子曰修脯也十脡為束古者相見必執贄以為禮束修其至薄者後世緣夫子之言遂以為事師之禮】德明恥之服巴豆散卧稱病【巴豆有毒能痢人散悉但翻】玄恕入跪牀下對之遺利竟不與語德明名朗以字行【陸德明過孔穎逹遠矣】世充於闕下及玄武門等數處皆設榻坐無常所親受章表或輕騎歷衢市亦不清道【天子清道而後行騎奇寄翻】民但避路而已世充按轡徐行語之曰【語牛倨翻】昔時天子深居九重【重直龍翻】在下事情無由聞徹【徹敕列翻】今世充非貪天位但欲救恤時危正如一州刺史親覽庶務當與士庶共評朝政尚恐門有禁限今於門外設坐聽朝【朝直遥翻下同坐徂卧翻】宜各盡情又令西朝堂納寃抑東朝堂納直諫於是獻策上書者日有數百條流既煩【上時掌翻條流猶言條?】省覽難遍【省悉景翻】數日後不復更出【復扶又翻】 竇建德聞王世充自立乃絶之始建天子旌旗出警入蹕下書稱詔追諡隋煬帝為閔帝齊王暕之死也【諡神至翻江都之難齊王暕亦死暕古限翻】有遺腹子政道建德立以為鄖公【鄖音云】然猶依倚突厥以壯其兵勢【厥九勿翻】隋義成公主遣使迎蕭皇后及南陽公主建德遣千餘騎送之【使疏吏翻騎奇寄翻】又傳宇文化及首以獻義成公主 丙辰劉武周圍并州齊王元吉拒却之戊午詔太常卿李仲文將兵救并州【將即亮翻】 王世充將軍丘懷義居門下内省召越王君度漢王玄恕將軍郭士衡雜妓妾飲博侍御史張藴古彈之世充大怒令散手執君度玄恕批其耳數十【散手者散手仗也凡朝會之仗三衛番上分為五仗一曰供奉仗以左右衛為之二曰親仗以親衛為之三曰勲仗以勳衛為之四曰翊仗以翊衛為之皆服鶡冠緋衫裌五曰散手仗以親勲翊衛為之服緋絁裲襠繡野馬列坐于東西廊下唐謂之衙内五衛唐蓋因隋制世充亦因隋制也妓渠綺翻彈徒丹翻散悉但翻批蒲鼈翻又普迷翻】又命引入東上閤杖之各數十【東都皇宫正殿曰乾陽殿殿左曰東上閤右曰西上閤閤各有門】懷義士衡不問賞藴古帛百段遷太子舍人君度世充之兄子也世充每聽朝殷勤誨諭言詞重複【朝直遥翻重直龍翻】千端萬緒侍衛之人不勝倦弊【勝音升】百司奏事疲於聽受御史大夫蘇良諫曰陛下語太多而無領要【領要猶漢人言要領也】計云爾即可何煩許辭也世充默然良久亦不罪良然性如是終不能改也 王世充數攻伊州【數所角翻】總管張善相拒之糧盡援兵不至癸亥城陷善相罵世充極口而死帝聞歎曰吾負善相善相不負吾也賜其子襄城郡公【相息亮翻子下當有爵字蜀本然】 五月王世充陷義州復寇西濟州【新志濟源縣武德二年王世充將丁伯德以縣來降置西濟州復扶又翻濟子禮翻】遣右驍衛大將軍劉弘基將兵救之【驍堅堯翻基將即亮翻下同】 李軌將安修仁兄興貴仕長安表請說軌諭以禍福【說式芮翻下同】上曰軌阻兵恃險連結吐谷渾突厥【吐從暾入聲谷音浴厥九勿翻】吾興兵擊之尚恐不克豈口舌所能下乎興貴曰臣家在凉州弈世豪望為民夷所附弟修仁為軌所信任子弟在機近者以十數臣往說之【說式芮翻】軌聽臣固善若其不聽圖之肘腋易矣【腋音亦易以豉翻】上乃遣之興貴至武威軌以為左右衛大將軍興貴乘閒說軌曰【閒古莧翻】凉地不過千里土薄民貧今唐起太原取函秦宰制中原戰必勝攻必取此殆天啟非人力也不若舉河西歸之則竇融之功復見於今日矣【竇融事見漢光武紀復扶又翻】軌曰吾據山河之固彼雖彊大若我何汝自唐來為唐遊說耳【為于偽翻】興貴謝曰臣聞富貴不歸故鄉如衣繡夜行【項羽之言衣於既翻】臣闔門受陛下榮祿安肯附唐但欲效其愚慮可否在陛下耳於是退與修仁隂結諸胡起兵擊軌軌出戰而敗嬰城自守興貴徇曰大唐遣我來誅李軌敢助之者夷三族城中人爭出就興貴軌計窮與妻子登玉女臺【軌築玉女臺見上卷上年】置酒為别庚辰興貴執之以聞河西悉平鄧曉在長安舞蹈稱慶上曰汝為人使臣【使疏吏翻下同】聞國亡不慼而喜以求媚於朕不忠於李軌肯為朕用乎遂廢之終身【是年二月李軌遣鄧曉入見】軌至長安并其子弟皆伏誅以安興貴為右武候大將軍上柱國凉國公賜帛萬段安修仁為左武候大將軍申國公 隋末離石胡劉龍兒【舊志云離石漢縣後周改名昌化郡隋為離石郡唐為石州離石胡匈奴種也即稽胡】擁兵數萬自號劉王以其子季真為太子虎賁郎將梁德擊斬龍兒【賁音奔將即亮翻】至是季真與弟六兒復舉兵為亂【復扶又翻】引劉武周之衆攻陷石州殺刺史王儉季真自稱突利可汗【可從刋入聲汗音寒】以六兒為拓定王六兒遣使請降【降戶冮翻】詔以為嵐州總管【以樓煩郡置嵐州宋白曰因界内岢嵐山立名嵐盧含翻】壬午以秦王世民為左武候大將軍使持節凉甘等九州諸軍事凉州總管【九州凉甘瓜鄯肅會蘭河廓皆李軌所據之地也】其太尉尚書令雍州牧陜東道行臺並如故【雍於用翻陜朱冉翻】遣黄門侍郎楊恭仁安撫河西 丙戌劉武周陷平遥【平遥縣屬汾州即漢平陶縣魏避國諱改陶為遥】 癸巳梁州總管山東道安撫副使陳政為麾下所殺攜其首奔王世充政茂之子也【隋書陳茂傳政歸唐卒於梁州總管不言死於山東通鑑當是據實錄諸書但是時山東無梁州或者政先為梁州總管後安撫山東而死也陳茂事隋文帝典機密】 王世充以禮部尚書裴仁基左輔大將軍裴行儼有威名忌之仁基父子知之亦不自安乃與尚書左丞宇文儒童儒童弟尚食直長温【隋制尚食局屬殿中省有奉御有直長長知兩翻】散騎常侍崔德本謀殺世充及其黨【散悉亶翻騎奇寄翻】復尊立皇泰主事泄皆夷三族齊王世惲言於世充曰儒童等謀反正為皇泰主尚在故也【復扶又翻惲於粉翻為于偽翻下同】不如早除之世充從之遣兄子唐王仁則及家奴梁百年酖皇泰主皇泰主曰更為請太尉以往者之言未應至此【謂世充往有復子明辟之言既不能踐今不應遽殺之也】百年欲為啟陳【為于偽翻】世惲不許又請與皇太后辭訣亦不許乃布席焚香禮佛願自今已往不復生帝王家飲藥不能絕以帛縊殺之諡曰恭皇帝【縊於賜翻又於計翻諡神至翻】世充以其兄楚王世偉為太保齊王世惲為太傅領尚書令六月庚子竇建德陷滄州【滄州隋之勃海郡】初易州賊帥宋金剛有衆萬餘與魏刀兒連結【易州上谷郡宋白曰易州六國時燕地秦并天下是為上谷郡漢置涿郡今州即涿郡故安縣地圖經云隋初自今遂城縣所理英雄城移南營州居燕之侯臺仍改名易州取州南易水為名帥所類翻】刀兒為竇建德所滅【去年十二月建德滅刀兒】金剛救之戰敗帥衆四千西奔劉武周【帥讀曰率】武周聞其善用兵得之甚喜號曰宋王委以軍事中分家貲以遺之【遺于季翻】金剛亦深自結出其故妻納武周之妺因說武周圖晉陽南向爭天下【說輸芮翻】武周以金剛為西南道大行臺使將兵三萬寇并州【將即亮翻下周將人將同】丁未武周進逼介州【義寧元年以介休平遥置介休郡武德元年曰介州】沙門道澄以佛幡縋之入城【縋他偽翻】遂陷介州詔左武衛大將軍姜寶誼行軍總管李仲文擊之武周將黄子英往來雀鼠谷【新志介休縣有雀鼠谷將即亮翻】數以輕兵挑戰【數所角翻挑徒了翻】兵纔接子英陽不勝而走如是再三寶誼仲文悉衆逐之伏兵發唐兵大敗寶誼仲文皆為所虜 【考異曰舊裴寂傳云寶誼仲文相次陷没按實錄二人敗處皆在雀鼠谷賊將黄子英陽不勝以誘之遇伏而没事迹並同必一時共戰皆被擒耳】既而俱逃歸上復使二人將兵擊武周【復扶又翻】 己酉突厥使來告始畢可汗之喪【厥九勿翻使疏吏翻下同可從刋入聲汗音寒】上舉哀于長樂門【六典長安宫城南面三門中曰承天東曰長樂西曰永安樂音洛】廢朝三曰【朝直遥翻】詔百官就館弔其使者又遣内史舍人鄭德挺弔處羅可汗賻帛三萬段【處昌呂翻賻音附】 上以劉武周入寇為憂右僕射裴寂請自行癸亥以寂為晉州道行軍總管【晉州曹魏之平陽郡後魏真君四年置東雍州孝昌中改為唐州建義元年又改為晉州隋為臨汾郡唐復為晉州】討武周聽以便宜從事 秋七月初置十二軍分關内諸府以隸焉皆取天星為名【以萬年道為參旗軍長安道為鼓旗軍富平道為玄戈軍醴泉道為井鉞軍同州道為羽林軍華州道為騎官軍寧州道為折威軍岐州道為平道軍州道為招揺軍西麟州道為苑遊軍涇州道為天紀軍宜州道為天節軍】以車騎府統之每軍將副各一人取威名素重者為之【騎奇寄翻將即亮翻下同】督以耕戰之務由是士馬精彊所向無敵 海岱賊帥徐圓朗以數州之地請降【言徐圓朗跨據數州東至海西距岱帥所類翻降戶江翻下同】拜兗州總管【兗州隋之魯郡禹貢之兗州東南據濟西北距河封域廣矣後漢以來兗州所治不常厥邑所部亦廣至是始專以魯郡為兗州】封魯國公 王世充遣其將羅士信寇穀州士信帥其衆千餘人來降【帥讀曰率下同】先是士信從李密擊世充兵敗為世充所得【先悉薦翻】世充厚禮之與同寢食既而得邴元真等待之如士信士信恥之士信有駿馬世充兄子趙王道詢欲之不與世充奪之以賜道詢士信怒故來降上聞其來甚喜遣使迎勞【使疏吏翻勞力到翻】廪食其所部【食祥吏翻】以士信為陜州道行軍總管【陜失冉翻宋白曰陜州即周二伯分陜之地後魏太和十一年置陜州】世充左龍驤將軍臨涇席辯【新志臨涇縣屬涇州驤思將翻】與同列楊䖍安李君義皆帥所部來降 丙子王世充遣其將郭士衡寇穀州刺史任瓌大破之俘斬且盡【任音壬瓌古回翻】甲申行軍總管劉弘基遣其將种如願襲王世充河陽城【种音冲】毁其河橋而還【還從宣翻又如字】乙酉西突厥統葉護可汗高昌王麴伯雅各遣使入貢初西突厥曷娑那可汗入朝于隋隋人留之國人立其叔父號射匱可汗【事見一百八十一卷煬帝大業七年厥九勿翻可從刋入聲娑素何翻朝直遥翻】射匱者達頭可汗之孫也既立拓地東至金山【按開元中以西州為金山都督府又突厥之先興於金山在高昌西北則知是山近高昌】西至海【此西海也】遂與北突厥為敵建庭於龜兹北三彌山【龜兹音丘慈】射匱卒子統葉護立統葉護勇而有謀北并鐵勒控弦數十萬據烏孫故地又移庭於石國北千泉【石國康居枝庶之分王者也治拓折城漢時大宛北鄙也】西域諸國皆臣之葉護各遣吐屯監之督其征賦【監上銜翻】 辛卯宋金剛寇浩州浹旬而退【唐初改西河郡為浩州浹子協翻】 八月丁酉酅公薨【酅奚圭翻】謚曰隋恭帝無後以族子行基嗣【嗣祥吏翻】 竇建德將兵十餘萬趣洺州【洺州隋之武安郡】淮安王神通帥諸軍退保相州己亥建德兵至洺州城下【將即亮翻下充將禮將同趣七喻翻下同洺音名帥讀曰率下同相息亮翻】 丙午將軍秦武通軍至洛陽敗王世充將葛彦璋【敗補邁翻】丁未竇建德陷洺州總管袁子幹降之【降戶江翻 考異曰實錄作甲子盖奏到之日今從革命記】乙卯引兵趣相州淮安王神通聞之帥諸軍就李世勣於黎陽 梁師都與突厥合數千騎寇延州【延州隋之延安郡騎奇寄翻厥九勿翻】行軍總管段德操兵少不敵閉壁不戰伺師都稍怠九月丙寅遣副總管梁禮將兵擊之師都與禮戰方酣德操以輕騎多張旗幟掩擊其後【少詩沼翻伺相吏翻酣戶江翻幟昌志翻】師都軍潰逐北二百里破其魏州【新志綏州成平縣置魏州因魏平關而名】虜男女二千餘口德操孝先之子也【段孝先柄用於高齊之季】 蕭銑遣其將楊道生寇峽州刺史許紹擊破之銑又遣其將陳普環帥舟師上峽規取巴蜀【將即亮翻上時掌翻】紹遣其子智仁及錄事參軍李弘節等追至西陵大破之【夷陵孫吳之西陵世謂之步闡壘唐貞觀九年峽州徙治焉隋之峽州本冶下牢戌在步闡壘西南二十八里水經江水逕夷陵縣南又東逕流頭灘狼尾灘黄牛山之黄牛灘而後逕西陵峽出峽東南流而後逕步闡壘此蓋自下牢追至西陵峽也】擒普環銑遣兵戍安蜀城及荆門城【安蜀城在公安縣界荆門城在長林縣界皆荆州西南要地】先是上遣開府李靖詣夔州經畧【巴東郡舊置信州是年改夔州杜佑曰避皇外祖獨孤信諱改之先悉薦翻】蕭銑【句斷】靖至峽州阻銑兵久不得進上怒其遲留隂敕許紹斬之【不以明詔而隂敕猶欲以宿憾殺之】紹惜其才為之奏請獲免【為于偽翻】 己巳竇建德陷相州殺刺史呂珉【相息亮翻考異曰實錄作庚辰盖亦奏到之日今從革命記】 民部尚書魯公劉文靜自<br />
<br />
  以才畧功勲在裴寂之右而位居其下意甚不平每廷議寂有所是文靜必非之數侵侮寂【數所角翻下同】由是有隙文靜與弟通直散騎常侍文起飲酒酣怨望拔刀擊柱曰會當斬裴寂首家數有妖【散悉亶翻騎奇寄翻妖於驕翻】文起召巫於星下被髮銜刀為厭勝【被皮義翻厭於葉翻】文靜有妾無寵使其兄上變告之【上時掌翻】上以文靜屬吏【屬之欲翻】遣裴寂蕭瑀問狀文靜曰建義之初忝為司馬計與長史位望畧同【長知兩翻】今寂為僕射據甲第【甲第甲於諸第也】臣官賞不異衆人東西征討老母留京師風雨無所庇實有觖望之心【觖望怨望也觖苦穴翻】因醉怨言不能自保上謂羣臣曰觀文靜此言反明白矣李綱蕭瑀皆明其不反秦王世民為之固請曰昔在晉陽文靜先定非常之策始告寂知【事見一百八十四卷隋恭帝義寧元年為于偽翻】及克京城任遇懸隔令文靜觖望則有之【令力丁翻】非敢謀反裴寂言於上曰文靜才畧實冠時人性復麤險【冠古玩翻復扶又翻】今天下未定留之必貽後患上素親寂低回久之卒用寂言【卒子恤翻 考異曰高祖實錄唐書唐歷等皆以文靜之死由於裴寂今據實錄此年六月裴寂為晉州道行軍總管討劉武周此月丁丑為宋金剛敗於介州去文靜死才七日此時不當在京師實錄曰高祖低回者久之蓋寂未行時先有此言高祖未忍殺至是乃决意耳】辛未文靜及文起坐死籍没其家 沈法興既克毗陵【克毗陵見一百八十五卷元年三月】謂江淮之南指撝可定【撝與麾同】自稱梁王都毗陵改元延康置百官性殘忍專尚威刑將士小有過即斬之由是其下離怨時杜伏威據歷陽陳稜據江都李子通據海陵俱有窺江表之心法興軍數敗會子通圍稜於江都稜送質求救於法興【將即亮翻下同數所角翻質音致】及伏威法興使其子綸將兵數萬與伏威共救之伏威軍清流綸軍揚子相去數十里【舊志清流縣漢全椒縣地梁置南譙州居桑根山之朝陽在今縣西南八十里南譙州故城是也北齊自南譙故城徙治於新昌郡城今滁州治清流縣是也隋志江都郡帶江陽縣有江都宫揚子宫唐永淳元年始分江都縣置揚子縣今真州治其地宋白曰清流縣因縣東清流水為名隋文帝更梁新昌縣為清流縣】子通納言毛文深獻策募江南人詐為綸兵夜襲伏威營伏威怒復遣兵襲綸【復扶又翻】由是二人相疑莫敢先進子通得盡銳攻江都克之稜奔伏威子通入江都因縱擊綸大破之伏威亦引去子通即皇帝位國號吳改元明政丹陽賊帥樂伯通帥衆萬餘降之【丹陽郡隋初之蔣州賊帥所類翻通帥讀曰率降戶江翻】子通以為左僕射 杜伏威請降丁丑以伏威為淮南安撫大使和州總管【和州漢歷陽縣地九江都尉治所晉氏南度置戍守以防江侯景之亂江西皆入高齊置和州於歷陽以南北通和往來之津要也大業廢州為歷陽郡今復以歷陽郡為和州使疏吏翻】 裴寂至介休【介休漢古縣因介子推介山而名時為介州】宋金剛據城拒之寂軍于度索原營中飲澗水金剛絶之士卒渴乏寂欲移營就水金剛縱兵擊之寂軍遂潰失亡畧盡寂一日一夜馳至晉州先是劉武周屢遣兵攻西河浩州刺史劉贍拒之【浩州隋之西河郡先悉薦翻贍而豔翻】李仲文引兵就之與共守西河及裴寂敗自晉州以北城鎮俱没唯西河獨存姜寶誼復為金剛所虜謀逃歸金剛殺之【復扶又翻】裴寂上表謝罪【上時掌翻】上慰諭之復使鎮撫河東【劉文靜淺水原之敗貶落不偶以至於誅裴寂度索原之敗位任如故唐高祖以賞罰馭臣上下其手矣】劉武周進逼并州齊王元吉紿其司馬劉德威曰卿以老弱守城吾以彊兵出戰辛巳元吉夜出兵攜其妻妾棄州奔還長安【紿蕩亥翻還從宣翻】元吉始去武周兵已至城下晉陽土豪薛深以城納武周上聞之大怒謂禮部尚書李綱曰元吉幼弱未習時事故遣竇誕宇文歆輔之【歆許令翻】晉陽彊兵數萬食支十年興王之基一旦棄之聞宇文歆首畫此策我當斬之綱曰王年少驕逸竇誕曾無規諫又掩覆之【少詩照翻覆敷又翻】使士民憤怨今日之敗誕之罪也歆諫王不悛尋皆聞奏【事見上二月悛丑緣翻】乃忠臣也豈可殺哉明日上召綱入升御座曰我得公遂無濫刑元吉自為不善非二人所能禁也并誕赦之衛尉少卿劉政會在太原為武周所虜政會密表論武周形勢武周據太原遣宋金剛攻晉州拔之虜右驍衛大將軍劉弘基【驍堅堯翻】弘基逃歸金剛進逼絳州陷龍門【宋白曰絳州晉都左傳所謂新田也後周武成二年置絳州龍門漢皮氏縣也後魏改為龍門縣隋唐屬蒲州】 西突厥曷娑那可汗與北突厥有怨曷娑那在長安北突厥遣使請殺之【厥九勿翻可從刋入聲汗音寒使疏吏翻】上不許羣臣皆曰保一人而失一國後必為患秦王世民曰人窮來歸我殺之不義上遲迴久之不得已丙戌引曷娑那於内殿宴飲既而送中書省縱北突厥使者使殺之 禮部尚書李綱領太子詹事【漢書百官表詹事秦官掌皇后太子家應劭云詹省也給也言給事太子家晉宋及後魏用人漸重北齊總東宫内外衆事無大小皆統之後周置太子宫正宫尹隋復置詹事唐統東宫三寺十率府之政令】太子建成始甚禮之久之太子漸昵近小人【昵尼質翻近其靳翻】疾秦王世民功高頗相猜忌綱屢諫不聽乃乞骸骨上罵之曰卿為何潘仁長史乃恥為朕尚書邪【長知兩翻邪音耶】且方使卿輔導建成而固求去何也綱頓首曰潘仁賊也每欲妄殺人臣諫之即止為其長史可以無愧【為長史事見一百八十四卷義寧元年】陛下創業明主臣不才所言如水投石【言以水投石雖沾濕而不能受水】言於太子亦然臣何敢久汚天臺辱東朝乎【天臺謂尚書省東朝謂東宫汚烏故翻朝直遥翻】上曰知公直士勉留輔吾兒戊子以綱為太子少保尚書詹事如故【少始照翻】綱復上書諫太子飲酒無節及信讒慝疎骨肉【復扶又翻下同上時掌翻慝吐得翻】太子不懌而所為如故綱鬱鬱不得志是歲固稱老病辭職詔解尚書仍為少保 淮安王神通使慰撫使張道源鎮趙州庚寅竇建德陷趙州【撫使疏吏翻宋白曰趙州周穆王封造父於趙城即此地後魏置趙郡隋大業初置趙州 考異曰實錄今年三月建德陷趙州此又云陷趙州蓋重複或三月是貝州唐統紀唯有九月陷趙州今從之】執總管張志昂及道源建德以二人及邢州刺史陳君賓不早下欲殺之國子祭酒凌敬諫曰人臣各為其主用【為于偽翻】彼堅守不下乃忠臣也今大王殺之何以勵羣下乎建德怒曰吾至城下彼猶不降力屈就擒何可捨也敬曰今大王使大將高士興拒羅藝於易水藝纔至興即降【將即亮翻降戶江翻】大王之意以為何如建德乃悟即命釋之 乙未梁師都復寇延州【復扶又翻 考異曰太宗實錄云經數月師都又來寇按丙寅九月朔寇延州乙未九月晦也今從高祖實錄】段德操擊破之斬首二千餘級師都以百餘騎遁去【騎奇寄翻】德操以功拜柱國賜爵平原郡公鄜州刺史鄜城壯公梁禮戰没【隋志上郡後魏置東秦州後改為北華州西魏為敷州大業二年改為鄜城郡後改為上郡唐為鄜州鄜音膚】 冬十月己亥就加凉州總管楊恭仁納言賜幽州總管燕公羅藝姓李氏封燕郡王【自國公進封郡王唐制國公食邑三千戶郡王食邑五千戶皆從一品燕因肩翻】 辛丑李藝破竇建德於衡水【衡水縣屬冀州宋白曰衡水縣本漢桃縣隋開皇十六年置衡水縣】 癸卯以左武候大將軍龎玉為梁州總管時集州獠反【新志武德元年析梁州之難江巴州之符陽長池白石置集州按舊志漢宕渠符陽之地也宋白曰後魏恭帝改梁之東巴州為集州以東北集川水為名獠魯皓翻】玉討之獠據險自守軍不得進糧且盡熟獠與反者皆鄰里親黨【近邊者為熟獠遠者為生獠】爭言賊不可擊請玉還玉揚言秋穀將熟百姓毋得收刈一切供軍非平賊吾不返聞者大懼曰大軍不去吾曹皆將餒死其中壯士乃入賊營與所親潛謀斬其渠帥而降【帥所類翻下同降戶江翻】餘黨皆散玉追討悉平之 劉武周將宋金剛進攻澮州陷之【新志義寧元年以絳郡之翼城絳縣置澮州因澮水而名將即亮翻下同澮古外翻】軍勢甚銳裴寂性怯無將帥之畧唯發使駱驛趣虞泰二州居民入城堡焚其積聚【駱驛相繼不絶也新志義寧元年以蒲州之安邑虞鄉夏置安邑郡武德元年曰虞州又義寧元年以蒲州之汾隂龍門置汾隂郡武德元年曰泰州使疏吏翻趣讀曰促積聚上子賜翻下慈喻翻今人多讀如字非也】民驚擾愁怨皆思為盜夏縣民呂崇茂聚衆自稱魏王以應武周【夏縣古虞公之地稷山虞坂皆在縣界隋屬河東郡時屬虞州杜佑曰夏縣漢安邑縣地夏都安邑城在縣北十五里蓋以此名縣夏戶雅翻】寂討之為所敗【敗補邁翻】詔永安王孝基獨孤懷恩陜州總管于筠内史侍郎唐儉等將兵討之【陜失冉翻將即亮翻下同】時王行本猶據蒲坂未下亦與武周相應【去年十二月隋將堯君素死王行本據蒲坂事見上卷】關中震駭上出手敕曰賊勢如此難與爭鋒宜棄大河以東謹守關西而已【所謂蒲津關以西也】秦王世民上表曰太原王業所基國之根本河東富實京邑所資若舉而棄之臣竊憤恨【上時掌翻】願假臣精兵三萬必冀平殄武周克復汾晉上於是悉發關中兵以益世民所統使擊武周乙卯幸華隂至長春宫以送之【華戶化翻】 竇建德引兵趣衛州【衛州漢汲縣地東魏立義州後周改衛州治汲宋白曰其州城隋已前謂之陣城郡國縣道記武王伐紂於此列陣因名】 建德每行軍常為三道輜重細弱居中央步騎夾左右相去三里許【趣七喻翻重直用翻騎奇寄翻下同】建德以千騎前行過黎陽三十里【黎陽縣在衛州東北百二十里】李世勣遣騎將丘孝剛將三百騎偵之【偵丑鄭翻】孝剛驍勇善馬槊【驍堅堯翻槊色角翻】與建德遇遂擊之建德敗走右方兵救之【右方用漢書語此謂建德兵之在右者也】擊斬孝剛建德怒還攻黎陽克之【考異曰實錄黎陽䧟在十一月丙子盖亦奏到之日今從革命記】虜淮安王神通李世<br />
<br />
  勣父蓋魏徵及帝妺同安公主唯李世勣以數百騎走度河數日以其父故還詣建德降衛州聞黎陽陷亦降【降戶江翻下同】建德以李世勣為左驍衛將軍使守黎陽【驍堅堯翻考異曰革命記使與其將高雅賢守新鄉按是時新鄉猶屬王世充使劉黑闥守之世勣既事建德乃為建】<br />
<br />
  【德攻下新鄉虜黑闥耳今從唐書】常以其父蓋自隨為質【質音致】以魏徵為起居舍人滑州刺史王軌奴殺軌攜其首詣建德降建德曰奴殺主大逆吾何為受之立命斬奴返其首於滑州吏民感悅即日請降於是其旁州縣及徐圓朗等皆望風歸附己未建德還洺州築萬春宫徙都之置淮安王神通於下博【下博縣時屬冀州宋白曰下博漢舊縣應劭云泰山有博縣故此云下故縣在今縣南二十里今縣後魏移於衡水北俗謂之故縣城以周建德六年又移縣於今理也今理即後漢祭遵壘北枕衡水】待以客禮 行軍總管羅士信帥勇士夜入洛陽外郭縱火焚清化里而還【帥讀曰率還從宣翻】壬戌士信拔青城堡【蓋因青城宫為堡】 王世充自將兵徇地至滑臺臨黎陽尉氏城主時德叡汴州刺史王要漢亳州刺史丁叔則遣使降之【汴州古大梁地戰國時為魏都漢為陳留郡東魏為梁州後周改汴州以城臨汴水因以為名宋白曰亳州春秋為東國之焦邑漢為譙縣魏為譙郡後周武帝置亳州遥取古南亳之名以名州將即亮翻使疏吏翻】以德叡為尉州刺史要漢伯當之兄也夏侯端至黎陽【是年四月遣夏侯端安撫淮左夏戶雅翻】李世勣發兵送之自澶淵濟河【隋開皇十六年置澶淵縣時屬黎州澶市連翻】傳檄州縣東至于海南至于淮二十餘州皆遣使來降【使疏吏翻】行至譙州【舊志亳州臨渙縣隋置譙州】會汴亳降於王世充還路遂絶端素得衆心所從二千人雖糧盡不忍委去端坐澤中殺馬以饗士因歔欷謂曰【歔音虚欷音希又許既翻】卿等鄉里皆已從賊特以共事之情未能見委【委棄也】我奉王命不可從卿卿有妻子無宜效我可斬吾首歸賊必獲富貴衆皆流涕曰公於唐室非有親屬直以忠義志不圖存某等雖賤心亦人也寧肯害公以求利乎端曰卿不忍見殺吾當自刎【刎式粉翻】衆抱持之乃復同進【復扶又翻】潛行五日餒死及為賊所擊奔潰相失者大半唯餘五十二人同走采䝁豆生食之【䝁豆野豆也音勞魯刀翻】端持節未嘗離身屢遣從者散自求生【離力智翻從才用翻】衆又不可時河南之地皆入世充唯杞州刺史李公逸為唐堅守遣兵迎端館給之【為于偽翻館古玩翻】世充遣使召端解衣遺之【使疏吏翻遺于季翻】仍送除書以端為淮南郡公尚書少吏部【少始照翻】端對使者焚書毁衣曰夏侯端天子大使豈受王世充官乎汝欲吾往唯可取吾首耳因解節旄懷之置刃於竿自山中西走無復蹊徑冒踐荆棘晝夜兼行得達宜陽【宜陽唐熊州】從者墜崖溺水為虎狼所食又喪其半【從才用翻下同喪息浪翻】其存者鬢髮禿落無復人狀端詣闕見上但謝無功初不自言艱苦上復以為祕書監郎楚之至山東亦為竇建德所獲楚之不屈竟得還【郎楚之與夏侯端同時出使史言唐之興也使於四方者皆能不辱君命還從宣翻】王世充遣其從弟世辨以徐亳之兵攻雍丘李公逸遣使求救【從才用翻使疏吏翻】上以隔賊境不能救公逸乃留其屬李善行守雍丘身帥輕騎入朝【帥讀曰率騎奇寄翻朝直遥翻】至襄城為世充伊州刺史張殷所獲世充謂曰卿越鄭臣唐其說安在公逸曰我於天下唯知有唐不知有鄭世充怒斬之善行亦没上以公逸子為襄邑公 甲子上祠華山【新志華山在華州華隂縣有岳祠華戶化翻】<br />
<br />
  資治通鑑卷一百八十七<br />
<br />
<史部,編年類,資治通鑑>  <br>
   </div> 

<script src="/search/ajaxskft.js"> </script>
 <div class="clear"></div>
<br>
<br>
 <!-- a.d-->

 <!--
<div class="info_share">
</div> 
-->
 <!--info_share--></div>   <!-- end info_content-->
  </div> <!-- end l-->

<div class="r">   <!--r-->



<div class="sidebar"  style="margin-bottom:2px;">

 
<div class="sidebar_title">工具类大全</div>
<div class="sidebar_info">
<strong><a href="http://www.guoxuedashi.com/lsditu/" target="_blank">历史地图</a></strong>  
<a href="http://www.880114.com/" target="_blank">英语宝典</a>  
<a href="http://www.guoxuedashi.com/13jing/" target="_blank">十三经检索</a> 
<br><strong><a href="http://www.guoxuedashi.com/gjtsjc/" target="_blank">古今图书集成</a></strong> 
<a href="http://www.guoxuedashi.com/duilian/" target="_blank">对联大全</a> <strong><a href="http://www.guoxuedashi.com/xiangxingzi/" target="_blank">象形文字典</a></strong> 

<br><a href="http://www.guoxuedashi.com/zixing/yanbian/">字形演变</a>  <strong><a href="http://www.guoxuemi.com/hafo/" target="_blank">哈佛燕京中文善本特藏</a></strong>
<br><strong><a href="http://www.guoxuedashi.com/csfz/" target="_blank">丛书&方志检索器</a></strong> <a href="http://www.guoxuedashi.com/yqjyy/" target="_blank">一切经音义</a>  

<br><strong><a href="http://www.guoxuedashi.com/jiapu/" target="_blank">家谱族谱查询</a></strong>  <strong><a href="http://shufa.guoxuedashi.com/sfzitie/" target="_blank">书法字帖欣赏</a></strong> 
<br>

</div>
</div>


<div class="sidebar" style="margin-bottom:0px;">

<font style="font-size:22px;line-height:32px">QQ交流群9:489193090</font>


<div class="sidebar_title">手机APP 扫描或点击</div>
<div class="sidebar_info">
<table>
<tr>
	<td width=160><a href="http://m.guoxuedashi.com/app/" target="_blank"><img src="/img/gxds-sj.png" width="140"  border="0" alt="国学大师手机版"></a></td>
	<td>
<a href="http://www.guoxuedashi.com/download/" target="_blank">app软件下载专区</a><br>
<a href="http://www.guoxuedashi.com/download/gxds.php" target="_blank">《国学大师》下载</a><br>
<a href="http://www.guoxuedashi.com/download/kxzd.php" target="_blank">《汉字宝典》下载</a><br>
<a href="http://www.guoxuedashi.com/download/scqbd.php" target="_blank">《诗词曲宝典》下载</a><br>
<a href="http://www.guoxuedashi.com/SiKuQuanShu/skqs.php" target="_blank">《四库全书》下载</a><br>
</td>
</tr>
</table>

</div>
</div>


<div class="sidebar2">
<center>


</center>
</div>

<div class="sidebar"  style="margin-bottom:2px;">
<div class="sidebar_title">网站使用教程</div>
<div class="sidebar_info">
<a href="http://www.guoxuedashi.com/help/gjsearch.php" target="_blank">如何在国学大师网下载古籍?</a><br>
<a href="http://www.guoxuedashi.com/zidian/bujian/bjjc.php" target="_blank">如何使用部件查字法快速查字?</a><br>
<a href="http://www.guoxuedashi.com/search/sjc.php" target="_blank">如何在指定的书籍中全文检索?</a><br>
<a href="http://www.guoxuedashi.com/search/skjc.php" target="_blank">如何找到一句话在《四库全书》哪一页?</a><br>
</div>
</div>


<div class="sidebar">
<div class="sidebar_title">热门书籍</div>
<div class="sidebar_info">
<a href="/so.php?sokey=%E8%B5%84%E6%B2%BB%E9%80%9A%E9%89%B4&kt=1">资治通鉴</a> <a href="/24shi/"><strong>二十四史</strong></a>&nbsp; <a href="/a2694/">野史</a>&nbsp; <a href="/SiKuQuanShu/"><strong>四库全书</strong></a>&nbsp;<a href="http://www.guoxuedashi.com/SiKuQuanShu/fanti/">繁体</a>
<br><a href="/so.php?sokey=%E7%BA%A2%E6%A5%BC%E6%A2%A6&kt=1">红楼梦</a> <a href="/a/1858x/">三国演义</a> <a href="/a/1038k/">水浒传</a> <a href="/a/1046t/">西游记</a> <a href="/a/1914o/">封神演义</a>
<br>
<a href="http://www.guoxuedashi.com/so.php?sokeygx=%E4%B8%87%E6%9C%89%E6%96%87%E5%BA%93&submit=&kt=1">万有文库</a> <a href="/a/780t/">古文观止</a> <a href="/a/1024l/">文心雕龙</a> <a href="/a/1704n/">全唐诗</a> <a href="/a/1705h/">全宋词</a>
<br><a href="http://www.guoxuedashi.com/so.php?sokeygx=%E7%99%BE%E8%A1%B2%E6%9C%AC%E4%BA%8C%E5%8D%81%E5%9B%9B%E5%8F%B2&submit=&kt=1"><strong>百衲本二十四史</strong></a>  <a href="http://www.guoxuedashi.com/so.php?sokeygx=%E5%8F%A4%E4%BB%8A%E5%9B%BE%E4%B9%A6%E9%9B%86%E6%88%90&submit=&kt=1"><strong>古今图书集成</strong></a>
<br>

<a href="http://www.guoxuedashi.com/so.php?sokeygx=%E4%B8%9B%E4%B9%A6%E9%9B%86%E6%88%90&submit=&kt=1">丛书集成</a> 
<a href="http://www.guoxuedashi.com/so.php?sokeygx=%E5%9B%9B%E9%83%A8%E4%B8%9B%E5%88%8A&submit=&kt=1"><strong>四部丛刊</strong></a>  
<a href="http://www.guoxuedashi.com/so.php?sokeygx=%E8%AF%B4%E6%96%87%E8%A7%A3%E5%AD%97&submit=&kt=1">說文解字</a> <a href="http://www.guoxuedashi.com/so.php?sokeygx=%E5%85%A8%E4%B8%8A%E5%8F%A4&submit=&kt=1">三国六朝文</a>
<br><a href="http://www.guoxuedashi.com/so.php?sokeytm=%E6%97%A5%E6%9C%AC%E5%86%85%E9%98%81%E6%96%87%E5%BA%93&submit=&kt=1"><strong>日本内阁文库</strong></a> <a href="http://www.guoxuedashi.com/so.php?sokeytm=%E5%9B%BD%E5%9B%BE%E6%96%B9%E5%BF%97%E5%90%88%E9%9B%86&ka=100&submit=">国图方志合集</a> <a href="http://www.guoxuedashi.com/so.php?sokeytm=%E5%90%84%E5%9C%B0%E6%96%B9%E5%BF%97&submit=&kt=1"><strong>各地方志</strong></a>

</div>
</div>


<div class="sidebar2">
<center>

</center>
</div>
<div class="sidebar greenbar">
<div class="sidebar_title green">四库全书</div>
<div class="sidebar_info">

《四库全书》是中国古代最大的丛书,编撰于乾隆年间,由纪昀等360多位高官、学者编撰,3800多人抄写,费时十三年编成。丛书分经、史、子、集四部,故名四库。共有3500多种书,7.9万卷,3.6万册,约8亿字,基本上囊括了古代所有图书,故称“全书”。<a href="http://www.guoxuedashi.com/SiKuQuanShu/">详细>>
</a>

</div> 
</div>

</div>  <!--end r-->

</div>
<!-- 内容区END --> 

<!-- 页脚开始 -->
<div class="shh">

</div>

<div class="w1180" style="margin-top:8px;">
<center><script src="http://www.guoxuedashi.com/img/plus.php?id=3"></script></center>
</div>
<div class="w1180 foot">
<a href="/b/thanks.php">特别致谢</a> | <a href="javascript:window.external.AddFavorite(document.location.href,document.title);">收藏本站</a> | <a href="#">欢迎投稿</a> | <a href="http://www.guoxuedashi.com/forum/">意见建议</a> | <a href="http://www.guoxuemi.com/">国学迷</a> | <a href="http://www.shuowen.net/">说文网</a><script language="javascript" type="text/javascript" src="https://js.users.51.la/17753172.js"></script><br />
  Copyright &copy; 国学大师 古典图书集成 All Rights Reserved.<br>
  
  <span style="font-size:14px">免责声明:本站非营利性站点,以方便网友为主,仅供学习研究。<br>内容由热心网友提供和网上收集,不保留版权。若侵犯了您的权益,来信即刪。scp168@qq.com</span>
  <br />
ICP证:<a href="http://www.beian.miit.gov.cn/" target="_blank">鲁ICP备19060063号</a></div>
<!-- 页脚END --> 
<script src="http://www.guoxuedashi.com/img/plus.php?id=22"></script>
<script src="http://www.guoxuedashi.com/img/tongji.js"></script>

</body>
</html>
