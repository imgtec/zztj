






























































資治通鑑卷二百三十二 宋 司馬光 撰

胡三省 音註

唐紀四十八【起旃蒙赤奮若八月盡彊圉單閼七月凡二年始乙丑八月終丁卯七月凡二年整}


德宗神武聖文皇帝七

貞元元年八月甲子詔凡不急之費及人冗食者皆罷之【冗而隴翻}
馬燧至行營與諸將謀曰長春宫不下【燧音遂將即亮翻圍長春宮事始上卷是年四月}
則懷光不可得長春宫守備甚嚴攻之曠日持久我當身往諭之遂徑造城下【造七到翻}
呼懷光守將徐庭光庭光帥將士羅拜城上【將即亮翻帥讀曰率}
燧知其心屈徐謂之曰我自朝廷來可西向受命庭光等復西向拜【復扶又翻}
燧曰汝曹自祿山已來徇國立功四十餘年【天寶十四載安祿山反郭子儀李光弼皆以朔方軍討賊立大功其後回紇吐蕃深入京畿諸鎮叛亂外禦内討亦倚朔方軍以成功至是年凡三十一年今曰四十餘年四字誤也當作三}
何忽為滅族之計從吾言非止免禍富貴可圖也衆不對燧披襟曰汝不信吾言何不射我【射而亦翻}
將士皆伏泣燧曰此皆懷光所為汝曹無罪弟堅守勿出【弟讀曰第但也}
皆曰諾壬申燧與渾瑊韓遊瓌進軍逼河中至焦籬堡【渾戶昆翻又戶本翻瑊戶咸翻瓌古回翻焦籬堡在河中府河西縣西}
守將尉珪以七百人降【尉紆勿翻本複姓尉遲後單姓尉以從便易降戶江翻下同}
是夕懷光舉火諸營不應駱元光在長春宫下使人招徐庭光庭光素輕元光遣卒罵之又為優胡于城上以侮之【駱元光本安息胡人故徐庭光為優胡以侮之}
且曰我降漢將耳元光使白燧燧還至城下庭光開門降燧以數騎入城慰撫其衆大呼曰【還從宣翻又音如字騎奇寄翻呼火故翻}
吾輩復為王人矣【復扶又翻又音如字}
渾瑊謂僚佐曰始吾謂馬公用兵不吾逮也今乃知吾不逮多矣【渾戶昆翻又戶本翻瑊古衘翻不逮不及也}
詔以庭光試殿中監兼御史大夫【此謂之試官兼官以寄祿且憲衘也}
甲戍燧帥諸軍至河西【宋白曰河西縣本同州舊朝邑之地唐上元元年以朝邑地置河西縣大歷三年復置朝邑縣仍析朝邑五鄉并割河東三鄉依舊為河西縣縣境東西十四里帥讀曰率考異曰舊燧傳曰燧帥諸軍濟河兵凡八萬陳於城下是日牛名俊斬懷光首以城降今從邠志}
河中軍士自相驚曰西城擐甲矣又曰東城娖隊矣【河中夾河為兩城西城河西縣東城河東縣河中府治焉擐音宦娖側角翻}
須臾軍士皆易其號為太平字懷光不知所為乃縊而死【縊於計翻又於賜翻}
初懷光之解奉天圍也【事見二百二十九卷建中四年}
上以其子璀為監察御史【璀七罪翻監古衘翻}
寵待甚厚及懷光屯咸陽不進【事見上卷興元元年}
璀密言于上曰臣父必負陛下願早為之備臣聞君父一也【人生在三事之如一謂君父師也}
但今日之勢陛下未能誅臣父而臣父足以危陛下陛下待臣厚胡人性直故不忍不言耳上驚曰知卿大臣愛子當為朕委曲彌縫而密奏之【為于偽翻下同言璀當委曲彌縫使君臣之間無隙不當密奏其事}
對曰臣父非不愛臣非不愛其父與宗族也顧臣力竭不能回耳上曰然則卿以何策自免對曰臣之進言非苟求生臣父敗則臣與之俱死矣復有何策哉【復扶又翻下同又音如字下同}
使臣賣父求生陛下亦安用之上曰卿勿死爲朕更至咸陽諭卿父使君臣父子俱全不亦善乎璀至咸陽而還【更古孟翻還音旋又如字}
曰無益也願陛下備之勿信人言臣今往說諭萬方【說式芮翻}
臣父言汝小子何知主上無信吾非貪富貴也直畏死耳汝豈可陷吾入死地邪【邪音耶}
及李泌赴陜【李泌赴陜見上卷是年七月泌薄必翻陜失冉翻}
上謂之曰朕所以再三欲全懷光者誠惜璀也卿至陜試為朕招之對曰陛下未幸梁洋懷光猶可降也【陜失冉翻爲于季翻洋音祥降戶江翻}
今則不然豈有人臣迫逐其君【廹逐其君謂懷光逼帝自奉天幸山南也}
而可復立於其朝乎縱彼厚顔無慙【人知愧者色見于面不知愧者謂之顔厚復扶又翻又音如字朝直遥翻下同}
陛下每視朝何心見之臣得入陜借使懷光請降臣不敢受况招之乎李璀固賢者必與父俱死矣若其不死則亦無足貴也及懷光死璀先刃其二弟乃自殺【楚令尹子南之子與李璀者皆處君臣父子大倫之變以死繼之可哀也已}
朔方將牛名俊斷懷光首出降【將即亮翻斷音短}
河中兵猶萬六千人燧斬其將閻晏等七人【閻晏勸懷光東保河中稱兵犯同州者也考異曰邠志云八人今從舊馬燧傳}
餘皆不問燧自辭行至河中平凡二十七日【戊申至甲戍二十七日史言馬燧期以一月平懷光不愆于素}
燧出高郢李鄘於獄【懷光囚郢鄘見上卷本年郢以井翻鄘余封翻}
皆奏置幕下韓遊瓌之攻懷光也楊懷賓戰甚力上命特原其子朝晟【李懷光囚楊朝晟見二百三十卷元年三月瓌古回翻朝直遥翻晟成正翻}
遊瓌遂以朝晟為都虞候【為楊朝晟後帥邠寜張本}
上使問陸贄河中既平復有何事所宜區處【處昌㠯翻}
令悉條奏【令力丁翻}
贄以河中既平慮必有希旨生事之人以為王師所向無敵請乘勝討淮西者李希烈必誘諭其所部及新附諸帥曰【新附諸帥謂李納王武俊田緒等誘音酉}
奉天息兵之旨乃因窘而言朝廷稍安必復誅伐如此則四方負罪者孰不自疑河朔青齊固當響應【窘臣隕翻復扶又翻又音如字河朔謂王武俊田緒劉怦青齊謂李納}
兵連禍結賦役繁興建中之憂行將復起乃上奏其略曰福不可以屢徼幸不可以常覬【上時掌翻徼一遥翻覬音冀}
臣姑以生禍為憂未敢以獲福為賀又曰陛下懷悔過之深誠【實心為誠}
降非常之大號【此謂興元敕書也}
所在宣之際聞者莫不涕流【敭與揚同}
假王叛換之夫削僞號以請罪【王武俊田悦李納去王號謝罪見二百二十九卷興元元年}
觀釁首鼠之將一純誠以效勤【謂馬燧韓滉陳少遊讀通鑑者因其事而觀其心迹則知之矣}
又曰曩討之而愈叛今釋之而畢來曩以百萬之師而力殫今以咫尺之詔而化洽是則聖王之敷理道服暴人【理道即治道避高宗諱改之}
任德而不任兵明矣羣帥之悖臣禮拒天誅圖活而不圖王又明矣【帥所類翻下同悖蒲内翻}
是則好生以及物者乃自生之方施安以及物者乃自安之術擠彼于死地【擠子細翻又子西翻}
而求此之久生也措彼於危地而求此之久安也從古及今未之有焉又曰一夫不率【率循也不率謂不循上之教今也}
闔境罹殃一境不寜普天致擾又曰億兆汙人【汙烏瓜翻汙下也}
四三叛帥感陛下自新之旨悦陛下盛德之言革面易辭且修臣禮其於深言密議固亦未盡坦然必當聚心而謀傾耳而聽觀陛下所行之事考陛下所誓之言若言與事符則遷善之心漸固儻事與言背則慮禍之態復興【陸贄斯言亦可以謂之深切當時事情背蒲妹翻復扶又翻}
又曰朱泚滅而懷光戮懷光戮而希烈征希烈儻平禍將次及則彼之蓄素疑而懷夙負者能不為之動心哉【爲于僞翻}
又曰今皇運中興天禍將悔以逆泚之偷居上國【泚且禮翻又音此唐都長安故謂之上國}
以懷光之竊保中畿【開元八年以河中為中都河東河西二縣為次赤縣諸縣為次畿縣}
歲末再周相次梟殄【去年六月斬朱泚今年八月平懷光梟殄謂梟其首而殄絶其類梟堅克翻}
實衆慝驚心之日【衆慝猶言衆惡也慝吐得翻}
羣生改觀之時【觀古玩翻又音如字}
威則已行惠猶未洽誠宜上副天眷下收物情布恤人之惠以濟威乘滅賊之威以行惠又曰臣所未敢保其必從唯希烈一人而已揆其私心非不願從也想其濳慮非不追悔也【興元赦文李希烈不與朱泚同科亦在肆赦之數}
但以猖狂失計已竊大號雖荷陛下全宥之恩然不能不自靦於天地之間耳【荷下可翻靦它典翻慙顔也}
縱未順命斯為獨夫【孟子曰殘賊之人謂之獨夫言人無親輔之者}
内則無辭以起兵外則無類以求助其計不過厚撫部曲偷容歲時心雖陸梁勢必不致陛下但敕諸鎮各守封疆彼既氣奪算窮是乃狴牢之類【狴邉迷翻又部禮翻狴犴牢獄所以拘囚有罪}
不有人禍則當鬼誅【陸贄論李希烈事曲盡情勢}
古之不戰而屈人之兵者此之謂歟【兵法百戰百勝不如不戰而屈人之兵}
丁卯詔以李懷光嘗有功宥其一男使續其後賜之田宅歸其首及尸使葬加馬燧兼侍中渾瑊檢校司空餘將卒賞賚各有差【燧音遂渾戶昆翻又戶本翻瑊古咸翻校古孝翻將即亮翻賚來戴翻}
諸道與淮西連接者宜各守封疆非彼侵軼不須進討【軼徒結翻突也}
李希烈若降當待以不死自餘將士百姓一無所問【行陸䞇之言也}
初李晟嘗將神策軍戍成都【盖大歷十四年敕蜀時也將即亮翻又音如字晟成正翻}
及還以營妓高自隨【還從宣翻妓渠綺翻}
西川節度使張延賞怒追而還之由是有隙【使疏吏翻}
至是劉從一有疾上召延賞入相晟表陳其過惡上重違其意【相息亮翻重難也}
以延賞為左僕射【李晟居功名之際以一婦人之故修怨于嚮用之臣且天子命相而勲臣以私怨間之其能自安乎斯不學之由也為延賞纔晟張本射寅謝翻}
駱元光將殺徐庭光謀於韓遊瓌【瓌古回翻}
曰庭光辱吾祖考【謂為優胡以戲侮之也}
吾欲殺之馬公必怒公能救其死乎遊瓌曰諾壬午遇庭光於軍門之外揖而數其罪【數所具翻又所主翻}
命左右碎斬之 【考異曰實錄甲申駱元光專殺徐庭光上令宰相諭諫官勿論邠志曰二十日駱公謀于韓公曰徐庭光見詬辱及祖父義不同天是日遂殺之按是月癸亥朔甲申二十二日盖奏到之日也今從邠志}
入見馬燧頓首請罪燧大怒曰庭光已降受朝廷官爵公不告輒殺之是無統帥也【燧音遂降戶江翻朝直遥翻統他綜翻俗從上聲帥所類翻}
欲斬之遊瓌曰元光殺禆將公猶怒如此公殺節度使天子其謂何燧默然渾瑊亦為之請【爲于偽翻}
乃捨之渾瑊鎮河中盡得李懷光之衆朔方軍自是分居邠蒲矣【自郭子儀以來朔方軍亦分屯邠蒲而統于一帥今居邠者韓遊瓌帥之居蒲者渾瑊帥之不相統屬故史言其始分渾戶昆翻又戶本翻邠卑旻翻}
盧龍節度使劉怦疾病【使疏吏翻怦普耕翻疾甚曰病}
九月己亥詔以其子行軍司馬濟權知節度事怦尋薨【薨昏胲翻}
己未中書侍郎同平章事劉從一罷為戶部尚書庚申薨【以疾罷而薨尚辰羊翻}
冬十月癸卯上祀圓丘赦天下 十二月甲戍戶部

奏今歲入貢者凡百五十州【時河朔諸鎮及淄青淮西皆不入貢河隴諸州又没于吐蕃}
于闐王曜上言兄勝讓國于臣【事見二百二十一卷肅宗上元元年闐徒賢翻又徒見翻上時掌翻}
今請復立勝子鋭上以鋭檢校光祿卿還其國勝固辭曰曜久行國事國人悅服銳生長京華【復扶又翻又音如字校古孝翻還從宣翻又音如字長知兩翻}
不習其俗不可往上嘉之以鋭為韶王諮議【韶王暹代宗子也唐制王府官諮議參軍正五品上}
二年春正月壬寅以吏部侍郎劉滋為左散騎常侍與給事中崔造中書舍人齊映並同章事滋子玄之孫也【散悉亶翻騎奇寄翻劉子玄以史筆事武后中宗}
造少居上元【少詩照翻上元縣帶昇州}
與韓會盧東美張正則為友以王佐自許時人謂之四夔【夔者唐虞之良臣時人重四人者以四夔稱之}
上以造在朝廷敢言故不次用之【朝直遥翻}
滋映多讓事於造造久在江外疾錢穀諸使罔上之弊奏罷水陸運使度支廵院江淮轉運使等諸道租賦悉委觀察使刺史遣官部送詣京師令宰相分判尚書六曹齊映判兵部李勉判刑部劉滋判吏部禮部造判戶部工部又以戶部侍郎元琇判諸道鹽鐵榷酒【使疏吏翻度徒洛翻尚辰羊翻琇音秀榷古岳翻}
吉中孚判度支兩税 李希烈將杜文朝寇襄州二月癸亥山南東道節度使樊澤擊擒之【將即亮翻朝直遥翻山南東道節度治裴州}
崔造與元琇善故使判鹽鐵韓滉奏論鹽鐵過失【為崔造元琇得罪張本滉呼廣翻}
甲戍以琇為尚書右丞陜州水陸運使李泌奏自集津至三門【泌薄必翻集津倉在三門東三門倉在三門西}
鑿山開車道十八里以避底柱之險【底柱兩山屹立河中河水分流包山而過世謂之三門車道者陸運之道舍舟而車運也}
是月道成 三月李希烈别將寇鄭州義成節度使李澄擊破之【代宗大歷七年賜滑亳節度為永平節度貞元元年永平軍節度更號義成軍節度興元元年李澄得鄭州}
希烈兵勢日蹙會有疾夏四月丙寅大將陳仙奇使醫陳山甫毒殺之【果如陸贄所料}
因以兵悉誅其兄弟妻子舉衆來降【降戶江翻 考異曰杜牧竇烈女傳曰初希烈入汴州聞戶曹參軍竇良女美使甲士至良門取桂娘以去將出門顴其父曰慎無戚必能滅賊使大人取富貴于天子桂娘以才色在希烈側復能巧曲取信凡希烈之密謀雖妻子不知者悉皆得聞希烈歸蔡州桂娘謂希烈曰忠而勇一軍莫如陳仙奇其妻竇氏仙奇寵且信之願得先往來以姊妹叙齒因徐說之使堅仙奇之心希烈然之桂娘因以姊事仙奇妻嘗問曰為賊遲晚必敗姊宜早圖遺種之地仙奇妻然之興元元年四月希烈暴死其子不發喪欲盡誅老將校以卑少者代之計未决有獻含桃者桂娘白希烈子請分遺仙奇妻且以示無事于外因為蠟帛書曰前日已死殯在後堂欲誅大臣須自為計以朱染帛丸如含桃仙奇發丸見之言于薛育育曰兩日希烈稱疾但怪樂曲雜晝夜不絶此乃有誅未定示暇于外事不疑矣明日仙奇薛育各以所部譟于牙門請見希烈希烈子迫出拜曰願去偽號一如李納仙奇曰爾父勃逆天子有命誅之因斬希烈及妻子函七首以獻暴其尸于市後兩月吳少誠殺仙奇知桂娘謀因亦殺之今從實錄及舊傳}
甲申以仙奇為淮西節度使 關中倉廩竭禁軍或自脫巾呼於道曰【呼火故翻}
拘吾于軍而不給糧吾罪人也上憂之甚會韓滉運米三萬斛至陜李泌即奏之上喜遽至東宫謂太子曰米已至陜吾父子得生矣【滉呼廣翻陜失冉翻泌薄必翻記王制曰國無六年之蓄曰急無三年之蓄曰國非其國也况日關無儲者乎日關無儲有以繼之猶可况漕運不繼朝不及夕者乎唐都關中仰給東南之餫德宗于兵荒之餘其窘乏尤不可言觀其父子相與語亦懲涇卒之變之于言語有不能以自揜者裴延齡知之故得因以排陸贄}
時禁中不釀命於坊市取酒為樂【樂音洛}
又遣中使諭神策六軍軍士皆呼萬歲時比歲饑饉【比毗至翻}
兵民率皆瘦黑至是麥始熟市有醉人當時以為嘉瑞人乍飽食死者復伍之一【復扶又翻}
數月人膚色乃復故 以横海軍使程日華爲節度使【滄州始别為節鎮以此觀之則以日華為横海軍副大使上卷衍大孚明矣}
秋七月淮西兵馬使吳少誠殺陳仙奇自為留後少誠素狡險為李希烈所寵任故為之報仇【使疏吏翻少始照翻故爲于偽翻下因為同}
己酉以䖍王諒為申光隨蔡節度大使以少誠為留後 以隴右行營節度使曲環為陳許節度使【曲環時以隴西行營兵戍陳許}
陳許荒亂之餘戶口流散曲環以勤儉率下政令寛簡賦役平均數年之間流亡復業兵食皆足 八月癸未義成節度使李澄薨其子士寜謀總軍務祕不發喪 丙戍吐蕃尚結贊大舉寇涇隴邠寜掠人畜芟禾稼西鄙騷然州縣各城守詔渾瑊將萬人駱元光將八千人屯咸陽以備之 初上與李泌議復府兵泌因爲上歷叙府兵自西魏以來興廢之由【西魏置府兵見一百六十三卷梁簡文帝大寶元年府兵廢見二百一十二卷玄宗開元十年}
且言府兵平日皆安居田畝每府有折衝領之折衝以農隙教習戰陳【陳讀曰陣}
國家有事徵發則以符契下其州及府【府者折衝果毅府下遐稼翻}
參驗發之至所期處【兵刻期所會之地}
將帥按閱有教習不精者罪其折衝甚者罪及刺史軍還則賜勲加賞便道罷之【罷兵使各隨便道歸農不必還至京師而後罷}
行者近不踰時遠不經歲高宗以劉仁軌為洮河鎮守使以圖吐蕃【見二百二卷高宗儀鳳二年}
於是始有久戍之役武后以來承平日久府兵浸墮【墮讀曰隳}
為人所賤百姓耻之至蒸熨手足以避其役又牛仙客以積財得宰相【事見二百一十四卷玄宗開元二十四年}
邉將效之山東戍卒多齎繒帛自隨邉將誘之寄於府庫晝則苦役夜縶地牢【繒慈陵翻誘音酉縶音執縛也}
利其死而没入其財故自天寶以後山東戍卒還者什無二三其殘虐如此然未嘗有外叛内侮殺帥自擅者誠以顧戀田園恐累宗族故也【累良瑞翻}
自開元之末張說始募長征兵謂之彍騎【事見一百一十二卷開元十年十三年}
其後益為六軍【六軍分左右為十二軍}
及李林甫為相奏諸軍皆募人為之【見二百一十六卷天寶八載}
兵不土著【著直畧翻}
又無宗族不自重惜忘身徇利禍亂遂生至今為梗【毛萇曰梗惡也鄭玄曰始生此禍乃至今日相梗不止}
曏使府兵之法常存不廢安有如此下陵上替之患哉【侵犯為陵偏下為替}
陛下思復府兵此乃社稷之福太平有日矣上曰俟平河中當與卿議之【因置十六衛上將軍先叙議復府兵之事}
九月丁亥詔十六衛各置上將軍以寵功臣改神策左右廂為左右神策軍殿前射生左右廂為殿前左右射生軍各置大將軍二人將軍二人【十六衛上將軍從二品神策大將軍正二品統軍從三品將軍從五品}
庚寅李克寜始發父澄之喪殺行軍司馬馬鉉墨縗出視事【墨縗自晋襄公始縗倉回翻}
增兵城門劉玄佐出師屯境上以制之且使告諭切至克寜廼不敢襲位丁酉以東都留守賈耽為義成節度使克寜悉取府庫之財夜出軍士從而剽之比明殆盡【剽匹妙翻比必利翻及也}
淄青兵數千自行營歸過滑州【自李正已以來淄青兵未嘗應調發赴行營也此必李納遣兵自戍守其境亦稱行營耳}
將佐皆曰李納雖外奉朝命内蓄兼并之志請館其兵於城外【朝直遥翻館古玩翻}
賈耽曰柰何與人鄰道而野處其將士乎【處昌呂翻}
命館於城中耽時引百騎獵於納境納聞之大喜服其度量不敢犯也【騎奇寄翻}
吐蕃遊騎及好畤【畤音止}
乙巳京城戒嚴復遣金吾將軍張獻甫屯咸陽民閒傳言上復欲出幸以避吐蕃【復扶又翻}
齊映見上言曰外閒皆言陛下已理裝具糗糧【見賢遍翻理裝治裝也糗去久翻乾飯屑也}
人情恟懼夫大福不再【左傳楚靈王之言也恟許拱翻夫音扶}
陛下柰何不與臣等熟計之因伏地流涕上亦為之動容【為于偽翻}
李晟遣其將王佖將驍勇三千伏於汧城【晟成正翻其將即亮翻佖毗必翻佖將音同上又音如字驍堅堯翻隴州之東有汧陽縣汧城在其旁汧口堅翻}
戒之曰虜過城下勿擊其首首雖敗彼全軍而至汝勿能當也不若俟前軍已過見五方旗虎豹衣【言其軍士所服之衣畫為虎豹文}
乃其中軍也出其不意擊之必大捷佖用其言尚結贊敗走軍士不識尚結贊僅而獲免尚結贊謂其徒曰唐之良將李晟馬燧渾瑊而已當以計去之【燧音遂渾戶昆翻又戶本翻瑊古咸翻為尚結贊聞李晟刼渾瑊賣馬燧張本去羌呂翻}
入鳳翔境内無所俘掠以兵二萬直抵城下曰李令公召我來【俘芳無翻李晟時為中書令故稱之為令公此尚結贊所以間晟也}
何不出犒我經宿乃引退冬十月癸亥李晟遣蕃落使野詩良輔【犒口到翻使疏吏翻野詩蕃姓也良輔其名}
與王佖將步騎五千襲吐蕃摧砂堡壬申遇吐蕃衆二萬與戰破之乘勝逐北至堡下攻拔之斬其將扈屈律悉蒙焚其蓄積而還【騎奇寄翻扈屈律蕃人三字姓還從宣翻又如字}
尚結贊引兵自寜慶北去【寜慶二州名}
癸酉軍於合水之北【合水縣屬慶州隋開皇十六年置九域志合水縣在慶州東北四十五里}
邠寜節度使韓遊瓌遣其將史履程夜襲其營殺數百人吐蕃追之遊瓌陳于平川【邠卑旻翻使疏吏翻瓌古回翻將即亮翻吐從暾入聲陳讀曰陣}
濳使人鼓于西山虜驚弃所掠而去 十一月甲午立淑妃王氏為皇后 乙未韓滉入朝【滉呼廣翻自京口入朝朝直遥翻}
丁酉皇后崩辛丑吐蕃寇鹽州【鹽州五原郡漢五原縣地}
謂刺史杜彦光曰我

欲得城聽爾率人去彦光悉衆奔鄜州【九域志慶州東至鄜州三百五十里}
吐蕃入據之 【考異曰邠志曰十二月三日吐蕃圍鹽州刺史杜彦光請委城以其衆去吐蕃許之分軍竊據今據實錄在此月}
劉玄佐在汴習鄰道故事【習淄青淮西及河朔故事}
久未入朝韓滉過汴玄佐重其才望以屬吏禮謁之【汴皮變翻朝直遥翻滉呼廣翻過古禾翻又古卧翻韓滉鎮二浙雖王室播遷而廵屬寜晏轉輸絡繹劉玄佐以是重其才滉父休以剛直致位宰輔滉所歷任皆著聲績劉玄佐以足重其望滉為江淮河南諸道轉運使玄佐賜履之地乃漕運之所經以職分言之則非屬吏也玄佐敬滉故以屬吏禮脩謁}
滉相約為兄弟請拜玄佐母其母喜置酒見之酒半滉曰弟何時入朝玄佐曰久欲入朝但力未辦耳滉曰滉力可及弟宜早入朝丈母垂白【諸父執行謂之丈人行韓滉與劉玄佐結為兄弟則視其父為丈人行故呼其母謂之丈母也}
不可使更帥諸婦女往填宫也【凡反者家屬皆沒入掖庭故云然帥讀曰率}
母悲泣不自勝【勝音升}
滉乃遺玄佐錢二十萬緡【遺唯季翻}
備行裝滉留大梁三日大出金帛賞勞【勞力到翻緡屬巾翻 考異曰柳氏叙訓云以綾二十萬匹犒軍今從國史補}
一軍為之傾動【為于偽翻}
玄佐驚服既而遣人密聽之滉問孔目吏【孔目吏今州郡皆有之謂之孔目宮亦謂之都吏言一孔一目無不總也}
今日所費幾何詰責甚細【詰去吉翻細織詳也}
玄佐笑曰吾知之矣壬寅玄佐與陳許節度使曲環俱入朝【韓滉既遺劉玄佐以入朝之資又大出賞勞以動其一軍之心玄佐雖欲不入朝得乎使疏吏翻朝直遥翻 考異曰鄴侯家傳曰韓相將入朝覲先公令人報比在闕庭已奏來則必能致大梁入朝今求所望善諭以致之十二月劉玄佐果入朝此盖李繁掠美今從柳氏叙訓}
崔造改錢穀法事多不集諸使之職行之已久中外安之【諸使謂鹽鐵轉運諸使也}
元琇既失職【謂解判鹽鐵而為右丞也琇音酉}
造憂懼成疾不視事既而江淮運米大至上嘉韓滉之功十二月丁巳以滉兼度支諸道鹽鐵轉運等使造所條奏皆改之【是年正月崔造為相改錢穀法及罷諸使今更從舊}
吐蕃又寇夏州亦令刺史托跋乾暉帥衆去遂據其城【托與拓同托拔起于鮮卑之裔自謂托天而生拔地而長故以為姓此後魏所本者也若唐時党項諸部亦自有拓拔一姓我朝西夏其後也夏兵雅翻}
又寇銀州州素無城吏民皆潰吐蕃亦棄之又陷麟州【宋白曰銀州漢為西河郡□隂縣地周武帝保定二年于縣城置銀防三年置銀州因谷為名舊有人收驄馬于此谷虜語驄馬為乞銀故名西北至夏州二百三十里北至麟州三百里}
韓滉屢短元琇于上庚申崔造罷為右庶子琇貶雷州司戶【舊志雷州至京師六千五百一十二里 考異曰實錄曰初元琇判度攴闕輔旱儉請運江淮租米以給京師上以韓滉素著威名加江淮轉運使欲令專督運務琇以滉性剛愎難與集事乃條奏令滉督運江南未至楊子凡一百十八里自楊子以北皆琇主之滉深怒于琇琇以京師錢重貨輕乃于江東監院收獲見錢四十餘萬今轉送入關滉不許誣奏以為運千錢至京師費錢萬上以問琇琇奏曰千錢之重納與一斗米均自江南水路至京師所費三二百耳上然之遣中使齎手詔令運錢滉堅執以為不可及滉總度支遂逞宿心累誣奏琇至是而貶焉舊崔造傳曰造與元琇素厚罷使之後以鹽鐵委之而韓滉以司務久行不可遽改德宗復以滉為江淮轉運使餘如造所條奏其年秋初江淮漕運大至京師德宗嘉其功以滉專領度支諸道鹽鐵轉運等使造所條奏皆改乃罷造知政事貶雷州司戶鄴侯家傳曰時元琇判度支江淮進米相次已入汴州而淄青及魏府蝗旱尤甚人皆相食李納無計欲束身入朝元琇廼支米十五萬石與之納軍遂濟三月入河運第一綱米三萬石自集津車般至三門十日而畢造入渭船亦成米至陜俄而度支牒至支充河中軍糧先公憂迫不知所為欲使人聞奏先令走馬與韓相謀之韓相報曰慎不可奏某判度支米在外勢不禁它反被它更鼓作言語待某今冬運畢當請朝覲此時面奏時蝗旱運路阻澀自四月初後有一日之内七奉手詔者皆為催米且言軍國糧儲自今月半後悉盡此米所藉公忠副朕憂屬星夜遣以濟憂勤其旨如此而不知米皆被外支盖琇及時宰忌韓相及先公運米功成而不為朝廷大計幾至再亂十月韓相以饋運功成請入朝及對見上大悦言無不從遂奏運事且言元琇支米與淄青河中臣在外與先公俱不敢奏上大驚即日貶琇為雷州司戶二說相違恐各有所私今但取其大要}
以吏部侍郎班宏為戶部侍郎度支副使【度徒洛翻使疏吏翻}
韓遊瓌奏請發兵攻鹽州吐蕃救之則使河東襲其

背丙寅詔駱元光及陳許兵馬使韓全義將步騎萬二千人會邠寧軍趣鹽州【瓌古曰翻吐從暾入聲將即亮翻又音如字騎奇寄翻邠卑旻翻趣七喻翻}
又命馬燧以河東軍擊吐蕃燧至石州河曲六胡州皆降遷於雲朔之閒【燧音遂石州昌化郡漢離石地河曲六胡州時已為宥州盖諸部酋長各以舊州名帶刺史故于時猶有六胡州之名雲州雲中郡本魏平城地朔州馬邑郡漢馬邑縣地降戶江翻}
工部侍郎張彧李晟之壻也晟在鳳翔以女嫁幕客崔樞禮重樞過於彧彧怒遂附於張延賞給事中鄭雲逵嘗為晟行軍司馬失晟意亦附延賞上亦忌晟功名會吐蕃有離間之言【彧於六翻晟成正翻過古禾翻又古卧翻吐從暾入聲間古莧翻離間之言見上}
延賞等騰謗於朝無所不至【朝直遥翻下同}
晟聞之晝夜泣目為之腫【蘇軾有言木必先蠧而後蟲生之人必先疑也而後讒入之張延賞之讒間亦因帝有忌晟之心而入之也為于偽翻}
悉遣子弟詣長安表請削髮為僧上慰諭不許辛未入朝見上自陳足疾懇辭方鎮上不許韓滉素與晟善上命滉與劉玄佐諭旨於晟使與延賞釋怨晟奉詔滉等引延賞詣晟第謝結為兄弟因宴飲盡歡又宴於滉玄佐之第亦如之滉因使晟表薦延賞為相【朝直遥翻見賢遍翻滉呼廣翻相息亮翻}


三年春正月壬寅以左僕射張延賞同平章事李晟為其子請昏於延賞【射寅謝翻爲于僞翻}
延賞不許晟謂人曰武夫性快釋怨於杯酒間則不復貯胸中矣【貯丁呂翻}
非如文士難犯外雖和解内蓄憾如故吾得無懼哉【張延賞心事李晟盖已洞見之矣}
初李希烈據淮西選騎兵尤精者為左右門槍奉國四將步兵尤精者為左右克平十將【李希烈自建中初據淮西騎奇寄翻槍千羊翻將即亮翻門槍奉國各分左右凡四將左右克平軍則分十將領之}
淮西少馬【少詩紹翻}
精兵皆乘騾謂之騾軍【騾力戈翻}
陳仙奇舉淮西降纔數月詔發其兵於京西防秋仙奇遣都知兵馬使蘇浦悉將淮西精兵五千人以行會仙奇為吳少誠所殺少誠密遣人召門槍兵馬使吳法超等使引兵歸浦不之知法超等引步騎四千自鄜州叛歸渾瑊使其將白娑勒追之【娑素和翻}
反為所敗【敗補邁翻}
丙午上急遣中使敇陜虢觀察使李泌發兵防遏勿令濟河【吳法超等自鄜州擅歸自鄜州即東北濟河下棧盖道蒲趨陜若從□華至陜則不必濟河矣}
泌遣押牙唐英岸將兵趣靈寶【九域志靈寶縣在陜州西四十五里趣七喻翻}
淮西兵已陳於河南矣【陳讀曰陣}
泌乃命靈寶給其食淮西兵亦不敢剽掠【剽匹妙翻}
明日宿陜西七里【陜西陜州之西距城七里}
泌不給其食遣將將選士四百人【選士簡選其驍勇者}
分為二隊伏於太原倉之隘道令之曰賊十隊過東伏則大呼擊之西伏亦大呼應之【呼火故翻}
勿遮道勿留行常讓以半道隨而擊之【遮道留行賊必人自為戰讓以半道隨而擊之前者得脱後者務進心不在戰此泌所以制勝}
又遣虞候集近村少年各持弓刀瓦石躡賊後聞呼亦應而追之又遣唐英岸將千五百人夜出南門陳於澗北【陳讀曰陣}
明日四鼓淮西兵起行入隘兩伏賊衆驚亂且戰且走死者四之一進遇唐英岸邀而擊之賊衆大敗擒其騾軍兵馬使張崇獻泌以賊必分兵自山路南遁又遣都將燕子楚將兵四百自炭竇谷趣長水【長水本隋弘農郡長淵縣唐初避高祖名更為長水五代志曰長淵縣後號曰南陜西魏更今名唐志長水縣屬洛州河南府宋白曰長水縣本漢盧氏縣地後魏延昌二年分盧氏東境庫谷已西沙渠谷已東為南陜縣廢帝改為長淵縣以縣洛水長淵爲名唐改長水九域志在府西二百四十里燕於䖍翻趣七喻翻下同}
賊二日不食屢戰皆敗英岸追至永寜東賊皆潰入山谷吳法超果帥其衆大半趣長水【帥讀曰率}
燕子楚擊之斬法超殺其士卒三分之二上以陜兵少神策軍步騎五千往助泌至赤水聞賊已破而還上命劉玄佐乘驛歸汴以詔書緣道誘之得百三十餘人至汴州盡殺之其潰兵在道復為村民所殺【復扶又翻}
得至蔡者纔四十七人吳少誠以其少【少詩沼翻}
悉斬之以聞且遣使以幣謝李泌爲其誅叛卒也【爲于偽翻}
泌執張崇獻等六十餘人送京師詔悉腰斬於鄜州軍門以令防秋之衆 初雲南王閤羅鳳陷嶲州【肅宗至德元載嶲州陷事見二百一十八卷}
獲西瀘令鄭回【西瀘縣屬嶲州本漢卬都縣地江左置宣化郡隋廢郡置可泉縣天寶元年改曰西瀘}
回相州人通經術閤羅鳳愛重之其子鳳迦異【迦求加翻}
及孫異牟尋曾孫尋夢湊皆師事之每授學回得撻之及異牟尋為王【大歷十四年異牟尋立見二百二十六卷}
以回為清平官清平官者蠻相也【南詔官曰坦綽曰布燮曰久贊謂之清平官所以决國事輕重猶唐宰相也}
凡有六人而國事專决於回五人者事回甚卑謹有過則回撻之雲南有衆數十萬吐蕃每入寇常以雲南為前鋒賦斂重數【歛力贍翻重數所角翻}
又奪其險要立城堡歲徵兵助防雲南苦之回因說異牟尋復自歸于唐【說式芮翻}
曰中國尚禮義有惠澤無賦役異牟尋以為然而無路自致凡十餘年及西川節度使韋臯至鎮招撫境上羣蠻異牟尋潜遣人因羣蠻求内附臯奏今吐蕃弃好【好呼到翻}
暴亂鹽夏【夏戶雅翻}
宜因雲南及八國生羌有歸化之心【八國生羌白狗君哥國君逋租君南水君弱水君悉董君凊遠君咄霸君}
招納之以離吐蕃之黨分其勢上命臯先作邉將書以諭之微觀其趣【為南詔内附張本}
張延賞與齊映有隙映在諸相中頗稱敢言上浸不悦延賞言映非宰相器壬子映貶夔州刺史劉滋罷為左散騎常侍以兵部侍郎柳渾同平章事韓滉性苛暴方為上所任言無不從它相充位而已百吏救過不贍渾雖為滉所引薦正色讓之曰先相公以褊察為相不滿歲而罷【先相公謂滉父休也罷相事見二百一十三卷開元二十一年}
今公又甚焉奈何榜吏於省中【榜音彭}
至有死者且作福作威豈人臣所宜【書洪範曰臣無有作威作福其害于而家凶于而國}
滉愧之為之少霽威嚴【為于偽翻}
二月壬戍以檢校左庶子崔澣充入吐蕃使 戊寅鎮海節度使同平章事充江淮轉運使韓滉薨滉久在二浙【大歷十四年滉觀察二浙建中二年建節}
所辟僚佐各隨其長無不得人嘗有故人子謁之考其能一無所長滉與之宴竟席未嘗左右視及與並坐交言【並坐謂並肩而坐者坐徂臥翻}
後數日署為隨軍使監庫門【監古銜翻}
其人終日危坐吏卒無敢妄出入者分浙江東西道為三浙西治潤州浙東治越州宣歙池治宣州【武德四年以宣州之秋浦南陵二縣置池州貞觀元年州廢永泰元年復分宣州之秋浦青陽饒州之至德置池州治秋浦秋浦漢石城縣地宣歙池三州屬江南東道唐初分十道江南東西道與二浙總為江南道乾元置浙江西道觀察使兼領宣歙饒三州其後罷領復領不一自分二浙為三道而宣歙池三州屬江南東道}
各置觀察使以領之上以果州刺史白志貞為浙西觀察使【果州南充郡治南充縣建中四年十二月白志貞貶思州司馬中間盖轉果州刺史今自刺史復欲用為觀察使}
柳渾曰志貞憸人【憸利于上佞人也又曰憸詖也音息廉翻}
不可復用【復扶又翻下同}
會渾疾不視事辛巳詔下用之渾疾間【間如字}
遂乞骸骨【以言不用也}
不許甲申葬昭德皇后于靖陵【王后諡昭德靖陵在奉天縣東北十里}
三月丁酉以左庶子李銛充入吐蕃使【銛思廉翻吐從暾入聲}
初吐蕃尚結贊得鹽夏州各留千餘人戍之退屯鳴沙【去年冬吐蕃留兵戍鹽夏州鳴沙縣屬靈州漢富平縣地宋白曰見後夏戶雅翻}
自冬入春羊馬多死糧運不繼又聞李晟克摧沙馬燧渾瑊等各舉兵臨之大懼【晟成正翻燧音遂渾戶昆翻又戶本翻瑊古咸翻}
屢遣使求和上未之許乃遣使卑辭厚禮求和于馬燧且請修清水之盟而歸侵地【清水盟見二百二十八卷建中四年}
使者相繼於路燧信其言留屯石州不復濟河爲之請於朝【爲于僞翻以馬燧智略功名而信尚結贊為之請使其刼盟之謀獲遂則自損功名而智略不足言}
李晟曰戎狄無信不如擊之韓遊瓌曰吐蕃弱則求盟彊則入寇今深入塞内而求盟此必詐也韓滉曰今兩河無虞若城原鄯洮渭四州使李晟劉玄佐之徒將十萬衆戍之河湟二十餘州可復也其資糧之費臣請主辦上由是不聽燧計趣使進兵燧請與吐蕃使論頰熱俱入朝論之【滉呼廣翻鄯以戰翻又音善洮土刀翻將即亮翻又音如字趣讀曰促使疏吏翻朝直遥翻 考異曰邠志作論莾熱今從實錄}
會滉薨燧延賞皆與晟有隙欲反其謀争言和親便上亦恨回紇【謂陜州之辱也}
欲與吐蕃和共擊之得二人言正會己意計遂定【史言馬燧張延賞以私隙誤國}
延賞數言晟不宜久典兵【數所角翻}
請以鄭雲逵代之上曰當令自擇代者【令力丁翻}
乃謂晟曰朕以百姓之故與吐蕃和親决矣大臣既與吐蕃有怨不可復之鳳翔【帝敬禮李晟謂之大臣之往也史言帝忌李晟因吐蕃請和將相有隙而奪其兵柄}
宜留朝廷朝夕輔朕自擇一人可代鳳翔者晟薦都虞候邢君牙君牙樂壽人也【樂壽本漢河間樂城縣故城在今縣東南十六里後魏移縣近古樂壽亭因改為樂壽唐初屬瀛州永泰中度屬深州}
丙午以君牙為鳳翔尹兼團練使丁未加晟太尉中書令勲封如故【勲上柱國封西平王}
餘悉罷之晟在鳳翔嘗謂僚佐曰魏徵好直諫【好呼到翻}
余竊慕之行軍司馬李叔度曰此乃儒者所為非勲德所宜晟歛容曰司馬失言晟任兼將相知朝廷得失不言何以為臣叔度慙而退【余謂李晟欲忠于君李叔度之言亦可謂忠于李晟}
及在朝廷上有所顧問極言無隱性沉密未嘗洩於人【朝直遥翻況持淋翻}
辛亥馬燧入朝燧既來諸軍皆閉壁不戰尚結贊遽自鳴沙引歸【宋白曰鳴沙縣屬靈州本漢富平縣地後周保定二年于此置會州建德六年立鳴沙鎮隋文帝立環州以大河環曲為名仍立鳴沙縣屬焉此地人馬行沙有聲異于餘沙故曰鳴沙}
其衆乏馬多徒行者崔澣見尚結贊責以負約尚結贊曰吐蕃破朱泚【以武亭之功邀唐事見二百三十卷元年四月泚且禮翻又音此}
未獲賞是以來而諸州各城守無由自逹鹽夏守將以城授我而遁非我取之也【夏戶雅翻將即亮翻}
今明公來欲踐修舊好【言欲踐前言以修舊好一曰欲踐前迹以修前好踐慈演翻好呼到翻}
固吐蕃之願也今吐蕃將相以下來者二十一人渾侍中嘗與之共事【言嘗與渾瑊共討朱泚}
知其忠信靈州節度使杜希全涇原節度使李觀皆信厚聞於異域請使之主盟【尚結贊欲因盟刼執二帥以取涇靈耳使疏吏翻觀古玩翻}
夏四月丙寅澣至長安辛未以澣為鴻臚卿復使入吐蕃語尚結贊曰【臚陵如翻復扶又翻下同語牛倨翻}
希全守靈不可出境李觀已改官今遣渾瑊盟於清水【清水漢故縣唐屬秦州 考異曰實錄崔澣至自鳴沙傳尚結贊言盟會之期及定界之所唯命是聽君歸奏决定當以鹽夏相還又云清水之會同盟者少是以和好輕慢不成今蕃及元帥已下凡二十一人赴盟靈州節度使杜希全稟性和善外境所知請令主此盟會涇原節度使李觀亦請同主之辛未以澣為鴻臚卿充入吐蕃使今澣報尚結贊希全職在靈州不可出境李觀又已改官遣侍中渾瑊充會盟俠約以五月二十四日復盟于清水按尚結贊本怨渾瑊故欲刼而執之然則求瑊主盟乃吐蕃意非由唐出也今從鄴侯家傳}
且令先歸鹽夏二州【令力丁翻夏戶雅翻}
五月甲申渾瑊自咸陽入朝以為清水會盟使戊子以兵部尚書崔漢衡為副使司封員外郎鄭叔矩為判官特進宋奉朝為都監【宋奉朝宦者也朝直遥翻下同}
己丑瑊將二萬餘人赴盟所乙巳尚結贊遣其屬論泣贊來言清水非吉地請盟於原州之土棃樹既盟而歸鹽夏二州上皆許之神策將馬有麟奏土棃樹多阻險恐吐蕃設伏兵不如平凉川坦夷【新唐書地理志平凉西北五里有吐蕃會盟壇}
時論泣贊已還丁未遣使追告之 申蔡留後吳少誠繕兵完城欲拒朝命【朝直遥翻}
判官鄭常大將楊冀謀逐之詐為手詔賜諸將申州刺史張伯元等事泄少誠殺常冀伯元大將宋旻曹濟奔長安 閠月己未韋臯復與東蠻和義王苴那時書【東蠻跨地二千里勿鄧豐琶兩林各有大鬼主為之長苴那時勿鄧鬼主也苴子魚翻}
使詗伺導達雲南【詗翾正翻又火迥翻詗伺刺探之人也}
庚申大省州縣官員收其祿以給戰士張延賞之謀

也時新除官千五百人而當減者千餘人怨嗟盈路初韓滉薦劉玄佐可使將兵復河湟上以問玄佐玄佐亦贊成之滉薨玄佐奏言吐蕃方彊未可與爭上遣中使勞問玄佐【勞力到翻}
玄佐臥而受命張延賞知玄佐不可用奏以河湟事委李抱真抱真亦固辭皆由延賞罷李晟兵柄故武臣皆憤怒解體不肯為用故也【史言張延賞妒功疾能之罪}
上以襄鄧扼淮西衝要癸亥以荆南節度使曹王臯為山南東道節度使以襄鄧復郢安隨唐七州隸之 渾瑊之發長安也李晟深戒之以盟所為備不可不嚴張延賞言於上曰晟不欲盟好之成【好呼到翻下同}
故戒瑊以嚴備我有疑彼之形則彼亦疑我矣盟何由成上乃召瑊切戒以推誠待虜勿自為猜貳以阻虜情瑊奏吐蕃决以辛未盟延賞集百官以瑊表稱詔示之【稱詔以渾瑊表徧示百官}
曰李太尉謂吐蕃和好必不成【李晟時加太尉故以稱之吐從暾入聲好呼到翻}
此渾侍中表也盟日定矣晟聞之泣謂所親曰吾生長西陲【李晟洮州人長事王忠嗣李抱玉皆有功名長知兩翻}
備諳虜情【諳烏含翻諳悉也}
所以論奏但耻朝廷為犬戎所侮耳上始命駱元光屯潘原韓遊瓌屯洛口【朝直遥翻自古以來謂西戎為犬戎潘原縣屬原州本隂盤也天寶更名時其地已沒于吐蕃瓌古回翻洛口即水洛口在瓦亭川東北}
以為瑊援元光謂瑊曰潘原距盟所且七十里公有急元光何從知之請與公俱瑊以詔指固止之元光不從與瑊連營相次距盟所三十餘里元光壕柵深固瑊壕柵皆可踰也【壕音豪塹也柵測革翻}
元光伏兵於營西韓遊瓌亦遣五百騎伏於其側曰若有變則汝曹西趣栢泉以分其勢【騎奇寄翻趣逡諭翻唐書地理志原州有百泉縣五代史志曰後魏分平凉置長城郡及黄石縣隋大業初改黄石為百泉宋白曰時已没蕃界}
尚結贊與瑊約各以甲士三千人列於壇之東西常服者四百人從至壇下辛未將盟尚結贊又請各遣遊騎數十更相覘索【更工衡翻覘丑廉翻索山客翻}
瑊皆許之吐蕃伏精騎數萬於壇西遊騎貫穿唐軍【穿尺絹翻}
出入無禁唐騎入虜軍悉為所擒瑊等皆不知入幕易禮服【禮服盟會之服}
虜伐鼓三聲【伐鼔擊鼓也}
大譟而至殺宋奉朝等於幕中【譟則竈翻朝直遥翻}
瑊自幕後出偶得它馬乘之伏鬛入其衘馳十餘里衘方及馬口故矢過其背而不傷唐將卒皆東走虜縱兵追擊或殺或擒之【衘古監翻過古禾翻又古臥翻將即亮翻}
死者數百人【是後劉昌為涇原帥收聚刼盟將士亡沒者骸骨具棺槥衣服葬于淺水原}
擒者千餘人崔漢衡為虜騎所擒渾瑊至其營則將卒皆遁去營空矣駱元光發伏成陳以待之【陳讀曰陣}
虜追騎愕眙【眙丑吏翻驚視也}
瑊入元光營追騎顧見邠寜軍西馳乃還【西馳者韓遊瓌所遣趣栢泉之軍也}
元光以輜重資瑊【重直用翻}
與瑊收散卒勒兵整陳而還是日上臨朝謂諸相曰今日和戎息兵社稷之福馬燧曰然柳渾曰戎狄豺狼也非盟誓可結今日之事臣竊憂之李晟曰誠如渾言上變色曰柳渾書生不知邉計大臣亦為此言邪皆伏地頓首謝因罷朝【朝直遥翻}
是夕韓遊瓌表言虜刼盟者兵臨近鎮【近鎮言邠寜之近鎮}
上大驚街遞其表以示渾【倉猝之際不及遣中使今術使遞其表以示渾}
明旦謂渾曰卿書生乃能料敵如此其審乎上欲出幸以避吐蕃大臣諫而止李晟大安園多竹復有為飛語者【復扶人翻}
云晟伏兵大安亭謀因倉猝為變晟遂伐其竹癸酉上遣中使王子恒齎詔遺尚結贊【遺唯季翻}
至吐蕃境不納而還渾瑊留屯奉天甲戍尚結贊至故原州【原州自廣德初沒于吐蕃城邑墟矣故曰故}
引見崔漢衡等曰吾飾金械欲械瑊以獻贊普今失瑊虛致公輩又謂馬燧之姪弇曰胡以馬為命吾在河曲春草未生馬不能舉足當是時侍中度河掩之吾全軍覆沒矣【在河曲謂屯鳴沙時馬燧時屯石州不度河燧加侍中故以稱之}
所以求和蒙侍中力今全軍得歸【今當作令}
柰何拘其子孫命弇與宦官俱文珍渾瑊將馬寜俱歸【獨遣弇歸尚結贊雖有此言馬燧諱之掩覆而不傳矣俱文珍歸則必言之于帝馬寧歸則必言之于渾瑊中外傳播燧不可得而掩也所以問燧者可謂巧矣}
分囚崔漢衡等於河廓鄯州上聞尚結贊之言由是惡馬燧【馬燧倍尚結贊之言而爲之請和既墮其計矣德宗又信尚結贊之間而惡馬燧又墮其計焉然德宗但知惡馬燧而不知惡張延賞又何也惡烏路翻}
六月丙戍以馬燧為司徒兼侍中罷其副元帥節度使初吐蕃尚結贊惡李晟馬燧渾瑊曰去三人則唐可圖也於是離間李晟【惡烏路翻去羌呂翻間古莧翻}
因馬燧以求和欲執渾瑊以賣燧使并獲罪因縱兵直犯長安會失渾瑊而止張延賞慚懼謝病不視事 以陜虢觀察使李泌為中書侍郎同平章事河東都虞候李自良從馬燧入朝上欲以為河東節度

使自良固辭曰臣事燧日久【馬燧初鎮河東即親任李自良}
不欲代之為帥【帥所類翻}
乃以為右龍武大將軍明日自良入謝上謂之曰卿於馬燧存軍中事分【分扶問翻}
誠為得禮然北門之任非卿不可卒以自良為河東節度使【卒于恤翻}
吐蕃之戍鹽夏者饋運不繼人多病疫思歸尚結贊遣三千騎逆之悉焚其廬舍毁其城驅其民而去靈鹽節度使杜希全遣兵分守之 韋臯以雲南頗知書壬辰自以書招諭之令趣遣使入見【趣讀曰促見賢遍翻下同}
李泌初視事【入政事堂視事也}
壬寅與李晟馬燧柳渾俱入見上謂泌曰卿昔在靈武已應為此官卿自退讓【事見二百一十九卷肅宗至德元載}
朕今用卿欲與卿有約【此亦帝猜忌見之一端也}
卿慎勿報仇有恩者朕當為卿報之【為于偽翻}
對曰臣素奉道不與人爲仇李輔國元載皆害臣者今自斃矣素所善及有恩者率已顯達或多零落臣無可報也上曰雖然有小恩者亦當報之對曰臣今日亦願與陛下為約可乎上曰何不可泌曰願陛下勿害功臣臣受陛下厚恩固無形迹李晟馬燧有大功於國聞有讒之者雖陛下必不聽然臣今日對二人言之欲其不自疑耳陛下萬一害之則宿衛之士方鎮之臣無不憤惋而反仄【惋烏貫翻}
恐中外之變不日復生也【復扶又翻}
人臣苟蒙人主愛信則幸矣官於何有臣在靈武之日未嘗有官而將相皆受臣指畫陛下以李懷光為太尉而懷光愈懼遂至於叛此皆陛下所親見也今晟燧富貴已足苟陛下坦然待之使其自保無虞國家有事則出從征伐無事則入奉朝請何樂如之【樂音洛}
故臣願陛下勿以二臣功大而忌之二臣勿以位高而自疑則天下永無事矣【惋烏貫翻樂音洛李泌不特欲使李晟馬燧無自疑之心亦以德宗猜忌聞廣而言之耳}
上曰朕始聞卿言聳然不知所謂及聽卿剖析乃知社稷之至計也朕謹當書紳二大臣亦當共保之晟燧皆起泣謝上因謂泌曰自今凡軍旅糧儲事卿主之吏禮委延賞刑法委渾泌曰不可陛下不以臣不才使待罪宰相宰相之職不可分也非如給事則有吏過兵過【吏部兵部主文武選凡奏擬皆過門下省百司奏抄侍中既審給事中讀之有違失則駮正}
舍人則有六押【唐制中書舍人六員佐宰相判案同署乃奏六典中書舍人六人分押六司}
至於宰相天下之事咸共平章若各有所主是乃有司非宰相也上笑曰朕適失辭卿言是也泌請復所減州縣官【是年閏月用張延賞之言大省州縣官}
上曰置吏以為人也【爲于偽翻下誰為第爲具爲同}
今戶口減於承平之時三分之二而吏員更增可乎對曰戶口雖減而事多於承平且十倍吏得無增乎且所減皆有職而冗官不減此所以為未當也【當丁浪翻}
至德以來置額外官敵正官三分之一若聽使計日得資然後停加兩選授同類正員官【停字句斷謂計其在官之日叙資然後隨所減員而停其官又加以文武兩選授以正員官與其元所居官同類者}
如此則不惟不怨兼使之喜矣又請諸王未出閤者不除府官【此泌所謂冗官不減者因請減而不除}
上皆從之乙卯詔先所減官並復故初張延賞在西川與東川節度使李叔明有隙上入駱谷【謂上自奉天幸山南時也}
值霖雨道途險滑衛士多亡歸朱泚叔明之子昇 【考異曰鄴侯家傳及舊叔明傳皆作昇今從實錄及舊蕭復傳}
及郭子儀之子曙令狐彰之子建等六人恐有姦人危乘輿相與齧臂為盟著行幐釘鞵【乘繩證翻著陟畧翻幐當作縢徒登翻行縢以邪幅纒足膊腸詩采菽斜幅在下傳云幅偪也所以自偈束也箋云邪幅如今行縢也偪束其脛自足至正義曰邪纒于足謂之邪幅釘鞵以皮為之外施油蠟底著鐵釘鞵戶皆翻}
更鞚上馬以至梁州【更工衡翻鞚苦貢翻}
它人皆不得近【近其靳翻}
及還長安上皆以為禁衛將軍寵遇甚厚張延賞知昇私出入郜國大長公主第【郜國肅宗之女初嫁裴徽又嫁蕭昇唐制皇姑為大長公主正一品郜古到翻長知丈翻}
密以白上上謂李泌曰郜國已老昇年少何為如是【泌薄必翻少始照翻}
殆必有故卿宜察之泌曰此必有欲動摇東宫者誰為陛下言之上曰卿勿問第為朕察之【為于偽翻}
泌曰必延賞也上曰何以知之泌具爲上言二人之隙【言延賞與昇父叔明有隙}
且曰昇承恩顧典禁兵延賞無以中傷【中竹仲翻}
而郜國乃太子蕭妃之母也故欲以此陷之耳上笑曰是也泌因請除昇它官勿令宿衛以遠嫌【遠于願翻}
秋七月以昇為詹事郜國肅宗之女也 甲子割振武之綏銀二州以右羽林將軍韓潭為夏綏銀節度使帥神策之士五千朔方河東之士三千鎮夏州【夏戶雅翻使疏吏翻帥讀曰率}
時關東防秋兵大集國用不充李泌奏自變兩税法以來【兩税事始見二百二十六卷建中元年}
藩鎮州縣多違法聚歛繼以朱泚之亂爭榷率徵罰以為軍資點募自防【歛力贍翻泚具禮翻又音此榷率者拘榷而敷率徵罰者吏民有罪罰使納錢穀以免罪而如數徵之也凡此皆州鎮以充軍資點募強壯以自防衛}
泚既平自懼違法匿不敢言請遣使以詔旨赦其罪但令革正自非於法應留使留州之外悉輸京師【令力丁翻留使者留以應本道節度觀察使徵調留州者留以給本州經用}
其官典逋負可徵者徵之難徵者釋之以示寛大敢有隱沒者重設告賞之科而罪之【重設賞格告者依格給賞而罪其隱沒者}
上喜曰卿策甚長然立法太寛恐所得無幾對曰兹事臣固熟思之寛則獲多而速急則獲少而遲盖以寛則人喜於免罪而樂輸【少始沼翻樂音洛}
急則競為蔽匿非推鞫不能得其實財不足濟今日之急而皆入於姦吏矣上曰善以度支員外郎元友直爲河南江淮南句勘兩税錢帛使【度徒洛翻句音勾使疏吏翻}
初河隴既沒於吐蕃【代宗初年河隴陷沒吐從暾入聲}
自天寶以來安西北庭奏事及西域使人在長安者歸路既絶人馬皆仰給於鴻臚禮賓委府縣供之【仰牛向翻臚陵如翻鴻臚掌四夷之客有禮賓院府縣謂京兆府及其所屬赤縣畿縣也}
於度支受直度支不時付直長安市肆不勝其弊【度徒洛翻勝音升}
李泌知胡客留長安久者或四十餘年皆有妻子買田宅舉質取利【舉者舉貸以取倍稱之利也貿者以物質錢計月而取其利也}
安居不欲歸命檢括胡客有田宅者停其給凡得四千人將停其給胡客皆詣政府訴之【政府謂相府也}
泌曰此皆從來宰相之過豈有外國朝貢使者留京師數十年不聽歸乎【朝直遥翻}
今當假道於回紇或自海道各遣歸國有不願歸當於鴻臚自陳授以職位給俸祿爲唐臣人生當乘時展用豈可終身客死邪於是胡客無一人願歸者泌皆分隸神策兩軍王子使者為散兵馬使或押牙【散悉亶翻}
餘皆為卒禁旅益壯鴻臚所給胡客纔十餘人歲省度支錢五十萬緡市人皆喜【免億故喜}
上復問泌以復府兵之策【上復扶又翻}
對曰今歲徵關東卒戍京西者十七萬人計歲食粟二百四萬斛今粟斗直百五十為錢三百六萬緡國家比遭饑亂【比毗至翻}
經費不充就使有錢亦無粟可糴未暇議復府兵也上曰然則奈何亟減戍卒歸之何如對曰陛下用臣之言可以不減戍卒不擾百姓糧食皆足粟麥日賤府兵亦成上曰苟能如是何為不用對曰此須急為之過旬日則不及矣今吐蕃久居原會之間以牛運糧糧盡牛無所用請發左藏惡繒染為綵纈【藏徂浪翻惡繒積于庫藏年深以致脆惡者纈戶結翻撮綵以線結之而後染色既染則解其結凡結處皆元色餘則入染色矣其色斑爛謂之纈}
因党項以市之每頭不過二三匹計十八萬匹可致六萬餘頭又命諸冶鑄農器糴麥種【種章勇翻下其種同}
分賜沿邉軍鎮募戍卒耕荒田而種之約明年麥熟倍償其種其餘據時價五分增一官爲糴之【為于偽翻}
來春種禾亦如之關中土沃而久荒所收必厚戍卒獲利耕者浸多邊地居人至少軍士月食官糧粟麥無所售其價必賤名為增價實比今歲所減多矣上曰善即命行之泌又言邉地官多闕請募人入粟以補之可足今歲之糧上亦從之因問曰卿言府兵亦集如何對曰戍卒因屯田致富則安于其土不復思歸【復扶又翻}
舊制戍卒三年而代及其將滿下令有願留者即以所開田為永業家人願來者本貫給長牒續食而遣之【戍兵家口發赴邉鎮者本貫為給長牒所過郡縣續食以至戍所}
據應募之數移報本道雖河朔諸帥得免更代之煩【帥所類翻更工衡翻}
亦喜聞矣【喜許記翻}
不過數番則戍卒土著【著直畧翻}
乃悉以府兵之法理之【理治也}
是變關中之疲弊為富彊也上曰如此天下無復事矣【泌所謂復府兵之策當以積漸而成帝遽謂之天下無復事是但喜其言之可聽而不察其事非旦暮之可集也}
泌曰未也臣能不用中國之兵使吐蕃自困上曰計將安出對曰臣未敢言之俟麥禾有效然後可議也上固問不對泌意欲結回紇大食雲南與共圖吐蕃令吐蕃所備者多知上素恨回紇恐聞之不悦并屯田之議不行故不肯言既而戍卒應募願耕屯田者什五六【自李泌為相觀其處置天下事姚崇以來未之有也史臣謂其出入中禁事四君數為權倖所疾常以智免好縱横大言時時讜議能寤移人主意然常持黄老鬼神說故為人所譏余謂泌以智免信如史臣言矣然其縱横大言持黄老鬼神說亦智也泌處肅代父子之間其論與復形勢言無不効及張李之間所以保右代宗者言無不行元載之讒疾卒能自免可謂智矣至其與德宗論天下事若指諸掌以肅代之信泌而泌不肯為相以德宗之猜忌而泌夷然當之亦智也嗚呼仕而得君諫行言聽則置身宰輔宜也歷事三世潔身遠害筋力向衰乃方入政事堂與新貴人伍所謂經濟之畧曏未能為肅代吐者盡為德宗吐之豈德宗之度弘于祖父邪泌盖量而後入耳彼德宗之猜忌刻薄直如蕭姜謂之輕已賣直功如李馬忌而置之散地而泌也恣言無憚彼其心以泌為祖父舊人智畧無方弘濟中興其敬信之也久矣泌之所以敢當相位者其自量亦審矣庸非智乎其持黄老鬼神說則子房欲從赤松游之故智也但子房功成後為之泌終始篤好之耳}
壬申賜駱元光姓名李元諒 左僕射同平章事張延賞薨【射寅謝翻薨呼肱翻}


資治通鑑卷二百三十二
















































































































































