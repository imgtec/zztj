<!DOCTYPE html PUBLIC "-//W3C//DTD XHTML 1.0 Transitional//EN" "http://www.w3.org/TR/xhtml1/DTD/xhtml1-transitional.dtd">
<html xmlns="http://www.w3.org/1999/xhtml">
<head>
<meta http-equiv="Content-Type" content="text/html; charset=utf-8" />
<meta http-equiv="X-UA-Compatible" content="IE=Edge,chrome=1">
<title>資治通鑒_216-資治通鑑卷二百十五_216-資治通鑑卷二百十五</title>
<meta name="Keywords" content="資治通鑒_216-資治通鑑卷二百十五_216-資治通鑑卷二百十五">
<meta name="Description" content="資治通鑒_216-資治通鑑卷二百十五_216-資治通鑑卷二百十五">
<meta http-equiv="Cache-Control" content="no-transform" />
<meta http-equiv="Cache-Control" content="no-siteapp" />
<link href="/img/style.css" rel="stylesheet" type="text/css" />
<script src="/img/m.js?2020"></script> 
</head>
<body>
 <div class="ClassNavi">
<a  href="/24shi/">二十四史</a> | <a href="/SiKuQuanShu/">四库全书</a> | <a href="http://www.guoxuedashi.com/gjtsjc/"><font  color="#FF0000">古今图书集成</font></a> | <a href="/renwu/">历史人物</a> | <a href="/ShuoWenJieZi/"><font  color="#FF0000">说文解字</a></font> | <a href="/chengyu/">成语词典</a> | <a  target="_blank"  href="http://www.guoxuedashi.com/jgwhj/"><font  color="#FF0000">甲骨文合集</font></a> | <a href="/yzjwjc/"><font  color="#FF0000">殷周金文集成</font></a> | <a href="/xiangxingzi/"><font color="#0000FF">象形字典</font></a> | <a href="/13jing/"><font  color="#FF0000">十三经索引</font></a> | <a href="/zixing/"><font  color="#FF0000">字体转换器</font></a> | <a href="/zidian/xz/"><font color="#0000FF">篆书识别</font></a> | <a href="/jinfanyi/">近义反义词</a> | <a href="/duilian/">对联大全</a> | <a href="/jiapu/"><font  color="#0000FF">家谱族谱查询</font></a> | <a href="http://www.guoxuemi.com/hafo/" target="_blank" ><font color="#FF0000">哈佛古籍</font></a> 
</div>

 <!-- 头部导航开始 -->
<div class="w1180 head clearfix">
  <div class="head_logo l"><a title="国学大师官网" href="http://www.guoxuedashi.com" target="_blank"></a></div>
  <div class="head_sr l">
  <div id="head1">
  
  <a href="http://www.guoxuedashi.com/zidian/bujian/" target="_blank" ><img src="http://www.guoxuedashi.com/img/top1.gif" width="88" height="60" border="0" title="部件查字,支持20万汉字"></a>


<a href="http://www.guoxuedashi.com/help/yingpan.php" target="_blank"><img src="http://www.guoxuedashi.com/img/top230.gif" width="600" height="62" border="0" ></a>


  </div>
  <div id="head3"><a href="javascript:" onClick="javascript:window.external.AddFavorite(window.location.href,document.title);">添加收藏</a>
  <br><a href="/help/setie.php">搜索引擎</a>
  <br><a href="/help/zanzhu.php">赞助本站</a></div>
  <div id="head2">
 <a href="http://www.guoxuemi.com/" target="_blank"><img src="http://www.guoxuedashi.com/img/guoxuemi.gif" width="95" height="62" border="0" style="margin-left:2px;" title="国学迷"></a>
  

  </div>
</div>
  <div class="clear"></div>
  <div class="head_nav">
  <p><a href="/">首页</a> | <a href="/ShuKu/">国学书库</a> | <a href="/guji/">影印古籍</a> | <a href="/shici/">诗词宝典</a> | <a   href="/SiKuQuanShu/gxjx.php">精选</a> <b>|</b> <a href="/zidian/">汉语字典</a> | <a href="/hydcd/">汉语词典</a> | <a href="http://www.guoxuedashi.com/zidian/bujian/"><font  color="#CC0066">部件查字</font></a> | <a href="http://www.sfds.cn/"><font  color="#CC0066">书法大师</font></a> | <a href="/jgwhj/">甲骨文</a> <b>|</b> <a href="/b/4/"><font  color="#CC0066">解密</font></a> | <a href="/renwu/">历史人物</a> | <a href="/diangu/">历史典故</a> | <a href="/xingshi/">姓氏</a> | <a href="/minzu/">民族</a> <b>|</b> <a href="/mz/"><font  color="#CC0066">世界名著</font></a> | <a href="/download/">软件下载</a>
</p>
<p><a href="/b/"><font  color="#CC0066">历史</font></a> | <a href="http://skqs.guoxuedashi.com/" target="_blank">四库全书</a> |  <a href="http://www.guoxuedashi.com/search/" target="_blank"><font  color="#CC0066">全文检索</font></a> | <a href="http://www.guoxuedashi.com/shumu/">古籍书目</a> | <a   href="/24shi/">正史</a> <b>|</b> <a href="/chengyu/">成语词典</a> | <a href="/kangxi/" title="康熙字典">康熙字典</a> | <a href="/ShuoWenJieZi/">说文解字</a> | <a href="/zixing/yanbian/">字形演变</a> | <a href="/yzjwjc/">金 文</a> <b>|</b>  <a href="/shijian/nian-hao/">年号</a> | <a href="/diming/">历史地名</a> | <a href="/shijian/">历史事件</a> | <a href="/guanzhi/">官职</a> | <a href="/lishi/">知识</a> <b>|</b> <a href="/zhongyi/">中医中药</a> | <a href="http://www.guoxuedashi.com/forum/">留言反馈</a>
</p>
  </div>
</div>
<!-- 头部导航END --> 
<!-- 内容区开始 --> 
<div class="w1180 clearfix">
  <div class="info l">
   
<div class="clearfix" style="background:#f5faff;">
<script src='http://www.guoxuedashi.com/img/headersou.js'></script>

</div>
  <div class="info_tree"><a href="http://www.guoxuedashi.com">首页</a> > <a href="/SiKuQuanShu/fanti/">四库全书</a>
 > <h1>资治通鉴</h1> <!--         下载:【右键另存为】即可 --></div>
  <div class="info_content zj clearfix">
  
<div class="info_txt clearfix" id="show">
<center style="font-size:24px;">216-資治通鑑卷二百十五</center>
    資治通鑑卷二百十五  宋 司馬光 撰<br />
<br />
  胡三省 音註<br />
<br />
  唐紀三十一【起玄黓敦牂盡彊圉大淵獻十一月凡五年有奇】<br />
<br />
  玄宗至道大聖大明孝皇帝中之下<br />
<br />
  天寶元年春正月丁未朔上御勤政樓【帝於興慶宫西南隅建二樓花萼相輝樓在西臨街以燕兄弟勤政務本樓在南以修政事】受朝賀【朝直遥翻】赦天下改元 壬子分平盧别為節度以安祿山為節度使【使疏吏翻下同】是時天下聲教所被之州三百二十一【被皮義翻考異曰舊紀云三百六十二按地理志開元二十八年州府三百二十八至此纔二年不應遽增三卜餘州今從唐歷會要統紀】羈縻之州八百置十節度經略使以備邊安西節度撫寜西域統龜兹焉耆于闐疎勒四鎮治龜兹城兵二萬四千【焉耆治所在安西府東八百里于闐在南二千里疎勒在西二千餘里龜兹音丘慈又音屈住闐徒賢翻又徒見翻】北庭節度防制突騎施堅昆統瀚海天山伊吾三軍屯伊西二州之境治北庭都護府兵二萬人【突騎施牙帳在北庭府西北三千餘里堅昆在北七千里瀚海軍在北庭府城内兵萬二千人天山軍在西州城内兵五千人伊吾軍在伊州西北三百里甘露川兵三千人騎奇寄翻】河西節度斷隔吐蕃突厥【斷丁管翻吐從暾入聲厥九勿翻】統赤水大斗建康寜寇玉門墨離豆盧新泉八軍張掖交城白亭三守捉屯凉肅瓜沙會五州之境治凉州兵七萬三千人【赤水軍在凉州城内兵三萬三千人大斗軍在凉州西二百餘里甘肅二州界兵七千五百人建康軍在凉州西二百里兵五千三百人寜寇軍在凉州東北千餘里兵八千五百人玉門軍在肅州西二百里管兵五千二百人墨離軍本月氐國在瓜州西北千里管兵五千人豆盧軍在沙州城内管兵四千三百人新泉軍在會州西北二百里管兵千人張掖守捉在凉州南二百里管兵五百人交城守捉在凉州西二百里管兵千人白亭守捉在凉州西北五百里管兵千七百人唐制大曰軍小曰守捉趙珣聚米圖經自甘州西至肅州五百里自肅州西至瓜州四百五十里自瓜州西至沙州二百八十里自沙州西至伊州四百里會州東至鹽州八百里西至凉州六百里南至宋鎮戎軍一百四十里北至靈州六百里】朔方節度捍禦突厥統經略豐安定遠三軍三受降城安北單于二都護府屯靈夏豐三州之境治靈州兵六萬四千七百人【經略軍在靈州城内兵二萬七百人豐安軍在靈州西黄河外百八十里兵八千人定遠軍在靈州東北二百里黄河外兵七千人西受降城在豐州北黄河外八十里兵七千人安北都護府治中受降城黄河北岸兵六千人東受降城在勝州東北二百里兵七千人振武軍在單于都護府城内兵九千人降戶江翻單音蟬夏戶雅翻】河東節度與朔方犄角以禦突厥統天兵大同横野岢嵐四軍雲中守捉屯太原府忻代嵐三州之境治太原府兵五萬五千人【天兵軍在太原城内兵三萬人大同軍在代州北三百里兵九千五百人横野軍在蔚州東北一百四十里兵三千人岢嵐軍在嵐州北百里兵一十人雲中守捉在單于府西北二百七十里兵七千七百人忻州在太原府北八十里兵七千八百人代州至太原五百里兵四千人嵐州在太原府西北二百五十里兵三千人岢拈我翻嵐盧含翻】范陽節度臨制奚契丹統經略威武清夷靜塞恒陽北平高陽唐興横海九軍屯幽薊媯檀易恒定漠滄九州之境治幽州兵九萬一千四百人【經略軍在幽州城内兵三萬人威武軍在檀州城内兵萬人清夷軍在媯州城内兵萬人靜塞軍在薊州城内兵萬六千人恒陽軍在恒州城東兵六千五百人北平軍在定州城西兵六千人高陽軍在易州城内兵六千人唐興軍在漠川城内兵六千人横海軍在滄州城内兵六千人景雲元年以嬴州鄚縣置鄚州開元十三年以鄚字類鄭字改為漠州尋又改莫州契欺訖翻又音喫恒戶登翻薊音計媯居為翻】平盧節度鎮撫室韋靺鞨統平盧盧龍二軍榆關守捉安東都護府屯營平二州之境治營州兵三萬七千五百人【平盧軍在營州城内兵萬六千人盧龍軍在平州城内兵萬人榆關守捉在營州城西四百八十里兵三千人安東都護府在營州東三百里兵八千五百人靺鞨音末曷榆當作偷注詳見上卷】隴右節度備禦吐蕃統臨洮河源白水安人振威威戎漠門寜塞積石鎮西十軍綏和合川平夷三守捉屯鄯廓洮河之境治鄯州兵七萬五千人【臨洮軍在鄯州城内兵萬五千人河源軍在鄯州西百二十里兵四千人白水軍在鄯州西北二百三十里兵四千人安人軍在鄯州界星宿川西兵萬人振威軍在鄯州西三百里兵千人威戎軍在鄯州西北三百五十里兵千人漠門軍在洮州城内兵五千五百人寜塞軍在廓州城内兵五百人積石軍在廓州西百八十里兵七千人鎮西軍在河州城内兵萬一千人綏和守捉在鄯州西南二百五十里兵千人合州守捉在鄯州南百八十里兵千人平夷守捉在河州西南四十里兵三千人洮土刀翻鄯時戰翻又音善】劒南節度西抗吐蕃南撫蠻獠統天寶平戎昆明寜遠澄川南江六軍屯益翼茂當嶲柘松維恭雅黎姚悉十三州之境治益州兵三萬九百人【團結營在成都府城内兵萬四千人天寶軍在恭州東南九十里兵千人平戎軍在恭州南八十里兵千人昆明軍在嶲州南兵五千一百人寜遠軍在嶲州西兵三百人澄州守捉在姚州東六百里管兵二千人南江軍兵三百人翼州兵五百人茂州兵三百人維州兵五百人柘州兵五百人松州兵二千八百人當州兵五百人雅州兵四百人黎州兵千人姚州兵三百人悉州兵五千人杜佑曰當州江源郡在翼州西二百七十里西北到故通軌縣二百里以西即是生羌悉州在當州南八十里黎州漢沉黎郡也東去一里高山萬重更無郡縣西南去郡一里高山萬重東北去郡五里西北去郡二里皆高山萬重茂州劉昫曰隋汶山郡武德元年改曰會州貞觀八年改曰茂州以郡界茂濕山為名松州東至茂州三百里西南至當州三百里西北至吐蕃界九十里南至翼州一百八十里恭州開元二十四年分靜州廣平縣置東至柘州一百里悉州西北至當州八十里獠魯皓翻嶲音髄】嶺南五府經略綏靜夷獠統經略清海二軍桂容邕交四營治廣州兵萬五千四百人【經略軍在廣州城内兵五千四百人清海軍在恩州城内兵二千人桂府兵千人容府兵千一百人邕府兵千七百人安南府兵四千二百人已上兵輕税本鎮以自給】此外又有長樂經畧福州領之【樂音洛】兵千五百人東萊守捉萊州領之東牟守捉登州領之兵各千人凡鎮兵四十九萬人【捉仄角翻 考異曰此兵數唐歷所載也舊紀是歲天下健兒團結彍騎等總五十七萬四千七百三十三此蓋止言邊兵彼并京畿諸州彍騎數之耳】馬八萬餘匹【安西府馬二千七百匹北庭瀚海軍馬四千二百匹天山軍馬五百匹伊吾軍馬三百匹河西赤水軍馬萬三千匹大斗軍馬二千四百匹建康軍馬五百匹寜寇玉門軍共管馬六百匹墨離軍馬四百匹豆盧軍馬四百匹朔方經略軍馬三千匹安豐軍馬千三百匹定遠軍馬二千匹西受降城馬千七百匹中受降城馬二千匹東受降城馬千七百匹振武軍馬千六百匹河東天兵軍馬五千五百匹雲中守捉馬二千匹大同軍馬五千五百匹横野軍馬千八百匹范陽經略軍馬五千四百匹威武軍馬三百匹清夷軍馬三百匹靜塞軍馬五百匹平盧軍馬四千二百匹盧龍軍馬五百匹榆關守捉馬百匹安東府馬七百匹隴右臨洮軍馬八千匹河源軍馬六百五十匹白水軍馬五百匹安人軍馬三百五十匹威戎軍馬五十匹漠門軍馬二百匹寜塞軍馬五十匹積石軍馬三百匹劒南團結營馬千八百匹昆明軍馬二百匹】開元之前每歲供邊兵衣糧費不過二百萬天寶之後邊將奏益兵浸多每歲用衣千二十萬疋糧百九十萬斛【安西衣賜六十二萬疋段北庭衣賜四十八萬疋段河西衣賜百八十萬疋段朔方衣賜二百萬疋段河東衣賜百二十六萬疋段糧五十萬石范陽衣賜八十萬疋段糧五十萬石平盧失衣糧數隴右衣賜二百五十萬疋段劒南衣賜八十萬疋段糧七十萬石將即亮翻】公私勞費民始困苦矣 甲寅陳王府參軍田同秀【皇子陳王珪府參軍也】上言見玄元皇帝於丹鳳門之空中告以我藏靈符在尹喜故宅【上時掌翻下上表同】上遣使於故函谷關尹喜臺旁求得之【列仙傳曰關令尹喜者周大夫也老子西游喜先見其氣候物色而迹之果得老子老子亦知其旨為著道德經使疏吏翻】 陜州刺史李齊物穿三門運渠辛未渠成【新書曰齊物鑿砥柱為門以通漕開其山巔為輓路沃醯而鑿之然棄石入河水益湍怒舟不能入新門候水漲以人輓舟而上天子疑之遣宦者按視齊物厚賂宦者還言其便陜失冉翻】齊物神通之曾孫也【淮安王神通】 壬辰羣臣上表以函谷靈符濳應年號先天不違【易曰先天而天不違先悉薦翻】請於尊號加天寶字從之二月辛卯上享玄元皇帝於新廟【時置玄元廟於大寜坊西南角】甲午享太廟丙申合祀天地於南郊赦天下改侍中為左相中書令為右相尚書左右丞相復為僕射【開元初改左右僕射為尚書左右丞相復扶又翻下清復同】東都北都皆為京州為郡刺史為太守改桃林縣曰靈寶【隋開皇十六年置桃林縣取古者桃林之野以為縣名屬洛州唐屬陜州今以得玄元靈符改曰靈寶守式又翻】田同秀除朝散大夫【朝直遥翻散悉亶翻】時人皆疑寶符同秀所為閒一歲清河人崔以清復言見玄元皇帝於天津橋北云藏符在武城紫微山【武城即漢之東武城縣與清河縣皆屬清河郡】敕使往求亦得之東都留守王倕知其詐按問果首服【使疏吏翻守式又翻下太守同倕音垂首式又翻】奏之上亦不深罪流之而已 三月以長安令韋堅為陜郡太守【陜郡陜州陜失冉翻】領江淮租庸轉運使【先天中李傑為陜州刺史領水陸發運使置使自傑始也裴耀卿之後命堅始以租庸使入銜使疏吏反下同】初宇文融既敗言利者稍息【事見二百十三卷開元十八年】及楊慎矜得幸【事始二百十三卷二十一年】於是韋堅王鉷之徒【鉷戶公翻】競以利進百司有利權者稍稍别置使以領之舊官充位而已【史言諸使所由始】堅太子之妃兄也為吏以幹敏稱上使之督江淮租運歲增巨萬上以為能故擢任之王鉷方翼之曾孫也【自高宗至武后朝王方翼著功名于西域】亦以善治租賦為戶部員外郎兼侍御史【治直之翻】李林甫為相凡才望功業出己右及為上所厚勢位將逼己者必百計去之【去羌呂翻】尤忌文學之士或陽與之善啗以甘言而陰䧟之世謂李林甫口有蜜【謂其言甘也㗖徒濫翻又徒覽翻】腹有劒【謂其心在害人也】上嘗陳樂於勤政樓垂簾觀之兵部侍郎盧絢謂上已起垂鞭按轡横過樓下絢風標清粹上目送之深歎其蘊藉【絢詐縣翻藴於運翻】林甫常厚以金帛賂上左右上舉動必知之乃召絢子弟謂曰尊君素望清崇今交廣藉才聖上欲以尊君為之可乎若憚遠行則當左遷不然則以賓詹分務東洛【謂以太子賓客詹事分司東都也】亦優賢之命也何如絢懼以賓詹為請林甫恐乖衆望乃除華州刺史【華戶化翻】到官未幾誣其有疾州事不理除詹事員外同正【幾居豈翻】上又嘗問林甫以嚴挺之今安在是人亦可用挺之時為絳州刺史林甫退召挺之弟損之諭以上待尊兄意甚厚盍為見上之策奏稱風疾求還京師就醫挺之從之林甫以其奏白上云挺之衰老得風疾宜且授以散秩使便醫藥上歎吒久之【散悉亶翻吒陟駕翻】夏四月壬寅以為詹事又以汴州刺史河南采訪使齊澣為少詹事【唐少詹事正四品上汴皮變翻使疏吏翻】皆員外同正於東京養疾澣亦朝廷宿望故并忌之 上發兵納十姓可汗阿史那昕於突騎施至俱蘭城【俱蘭國所都城也俱蘭或曰俱羅弩或曰屈浪拏與吐火羅接可從刋入聲汗音寒騎奇寄翻 考異曰會要作俱南城胡語不明耳】為莫賀達干所殺突騎施大纛官都摩度來降【纛徒到翻降戶江翻】六月乙未冊都摩度為三姓葉護 秋七月癸卯朔日有食之 辛未左相牛仙客薨八月丁丑以刑部尚書李適之為左相 突厥拔悉密囘紇葛邏祿三部共攻骨咄葉護殺之推拔悉密酋長為頡跌伊施可汗囘紇葛邏祿自為左右葉護【厥九勿翻紇下沒翻邏郎佐翻咄當沒翻酋慈由翻頡奚結翻跌徒結翻】突厥餘衆共立判闕特勒之子為烏蘇米施可汗以其子葛臘哆為西殺【哆昌者翻突厥以其親屬分掌東西兵號左右殺亦曰東西殺西殺右殺也】上遣使諭烏蘇令内附烏蘇不從朔方節度使王忠嗣盛兵磧口以威之【使疏吏翻磧七迹翻 考異曰新舊書忠嗣傳皆曰是歲忠嗣北伐與奚怒皆戰于桑乾河三敗之大虜其衆又曰明年再破怒皆突厥之衆自是塞外晏然按朔方不與奚相接不知所云奚怒皆何也今闕之】烏蘇懼請降而遷延不至忠嗣知其詐乃遣使說拔悉密囘紇葛邏祿使攻之烏蘇遁去忠嗣因出兵擊之取其右廂以歸【擊垂亡之虜猶不肯輕用其兵此王忠嗣所以為善將也突厥左右殺所部謂之左右廂降戶江翻說輸芮翻紇下沒翻邏郎佐翻】丁亥突厥西葉護阿布思及西殺葛臘哆默啜之孫勃德支伊然小妻毗伽登利之女帥部衆千餘帳相次來降【意此皆突厥右廂之衆也啜陟劣翻伽求迦翻帥讀曰率 考異曰實錄舊紀皆云突厥阿布思之孫登利可汗之女與其黨屬來降唐歷云烏蘇米施可汗遁逃其西葉護阿布思及毗伽可汗可敦男西殺葛臘哆率其部千餘帳來降舊王忠嗣傳云三部落攻米施可汗走之忠嗣因出兵伐之取其右廂而歸其西葉護及毗伽可敦男葛臘哆率其部落千餘帳入朝突厥傳云西殺妻子及默啜之孫勃德支特勒毗伽可汗女大洛公主伊然可汗小妻余塞匐登利可汗女余燭公主及阿布思頡利發等並帥其部衆相次來降今參取用之】突厥遂微九月辛亥上御花萼樓宴突厥降者 【考異曰本紀作辛卯按長歷是月己卯朔無辛卯唐歷云九日辛卯亦誤也】賞賜甚厚 護密先附吐蕃戊午其王頡吉里匐遣使請降【頡奚結翻匐蒲北翻】 冬十月丁酉上幸驪山温泉己巳還宫 十二月隴右節度使皇甫惟明奏破吐蕃大嶺等軍戊戌又奏破青海道莽布支營三萬餘衆斬獲五千餘級庚子河西節度使王倕奏破吐蕃漁海及遊奕等軍【史言明皇喜邊功故邊帥告捷者相繼倕音垂】 是歲天下縣一千五百二十八鄉一萬六千八百二十九戶八百五十二萬五千七百六十三口四千八百九十萬九千八百 囘紇葉護骨力裴羅遣使入貢 【考異曰舊傳云天寶初其酋長葉護頡利吐發遣使入朝封奉義王唐歷天寶三載突厥拔悉密可汗又為囘紇葛邏祿等部落襲殺之立囘紇為主是為骨咄祿毗伽闕可汗遣使立為奉義王又加懷仁可汗新突厥傳云囘紇葛邏祿殺拔悉密可汗奉囘紇骨力裴羅定其國是為骨咄祿毗伽闕可汗按奉義王懷仁可汗是一人而新突厥囘紇傳其名不同然新傳自吐迷度以來世系皆可譜今從之】賜爵奉義王【明未冊為可汗也】二年春正月安祿山入朝【朝直遥翻】上寵待甚厚謁見無時【見賢遍翻】祿山奏言去年營州蟲食苖臣焚香祝天云臣若操心不正事君不忠願使蟲食臣心若不負神祗願使蟲散即有羣鳥從北來食蟲立盡請宣付史官從之【操干高翻】 李林甫領吏部尚書日在政府【政府謂政事堂】選事悉委侍郎宋遥苖晉卿御史中丞張倚新得幸于上遥晉卿欲附之時選人集者以萬計【選須絹翻】入等者六十四人倚子奭為之首羣議沸騰前薊令蘇孝韞以告安祿山【薊縣帶幽州涿郡時改涿郡為范陽郡薊音計】祿山入言於上上悉召入等人面試之奭手持試紙終日不成一字時人謂之曳白癸亥遥貶武當太守晉卿貶安康太守倚貶淮揚太守【武當郡均州安康郡金州本西城郡元年更郡名淮陽郡陳州舊志金州京師南七百三十七里陳州一千五百二十里】同考判官禮部郎中裴朏等皆貶嶺南官晉卿壺關人也【壺關縣自漢以來屬上黨郡而唐上黨縣乃漢壺關縣隋分置上黨縣帶郡唐武德四年分隋之上黨縣置壺關縣治高望堡貞觀十七年移治進流川朏敷尾翻】 三月壬子追尊玄元皇帝父周上御大夫為先天太皇又尊臯繇為德明皇帝凉武昭王為興聖皇帝【唐虞之世臯繇為理唐以為李氏得姓之始故追尊為德明皇帝凉武昭王暠高祖之七世祖建國于瓜沙李氏由是而興故尊為興聖皇帝繇余招翻】 江淮南租庸等使韋堅引滻水抵苑東望春樓下為潭【苑禁苑也潭在長安城東九里滻音產】以聚江淮運船役夫匠通漕渠發人丘壟自江淮至京城民間蕭然愁怨二年而成丙寅上幸望春樓觀新潭堅以新船數百艘扁榜郡名各陳郡中珍貨於船背陜尉崔成甫著錦半臂缺胯緑衫以裼之【艘蘇遭翻扁補典翻陜失冉翻著陟略翻胯苦瓦翻裼先擊翻袒衣也】紅袹首【袹莫白翻袹首今人謂之抹額】居前船唱得寶歌【先是民間唱俚歌曰得體紇那邪其後得寶符于桃林成甫乃更紇體歌為得寶弘農野歌曰得寶弘農野弘農得寶耶潭裏舟船閙揚州銅器多三郎當殿坐聽唱得寶歌其俚又甚焉】使美婦百人盛飾而和之【和戶卧翻】連檣數里堅跪進諸郡輕貨仍上百牙盤食【程大昌演繁露曰唐少府監御饌器用九飣食以牙盤九枚裝食味於上置上前亦謂之看席仍上時掌翻】上置宴竟日而罷觀者山積夏四月加堅左散騎常侍【散悉亶翻騎奇寄翻】其僚屬吏卒褒賞有差名其潭曰廣運時京兆尹韓朝宗亦引渭水置潭於西街以貯材木【朝直遥翻貯丁呂翻】 丁亥皇甫惟明引軍出西平【西平郡即鄯州】擊吐蕃行千餘里攻洪濟城破之【杜佑曰廓州達化縣有洪濟鎮周武帝逐吐谷渾所築在縣西二百七十里長慶中劉元鼎為盟會使言河之上流由洪濟西行二千里水益狹冬春可涉夏秋乃勝舟其南三百里三山中高四下曰歷山直大羊同國古所謂昆侖者也虜曰□摩黎山東距長安五千里河源其間流澄緩下稍合衆流色赤行益遠他水并注則濁河源東北直莫賀延磧尾隱測其地蓋在劒南之西此劉元鼎因洪濟城而上叙河源附見于此】 上以右贊善大夫楊慎矜【龍朔二年改太子中允為贊善大夫咸亨元年復置中允而贊善大夫不廢後又分左右各置五員班左右諭德下諭德掌諭太子以道德贊善掌翊贊太子以規諷】知御史中丞事時李林甫專權公卿之進有不出其門者必以罪去之【去羌呂翻】慎矜由是固辭不敢受五月辛丑以慎矜為諫議大夫 冬十月戊寅上幸驪山温泉乙卯還宫【戊寅至乙卯三十八日史言帝耽樂而忘返驪力知翻還從宣翻又音如字 考異曰舊紀十月戊寅幸溫泉宫十一月乙卯還宫與實錄同十二月戊申又幸溫泉宫丙辰還宫實錄無按十二月丙寅朔無戊申丙辰唐歷十一月戊申幸溫泉宫丙辰還京又與實錄本紀不同今皆不取】<br />
<br />
  三載春正月丙申朔改年曰載【載子亥翻】 辛丑上幸驪山温泉二月庚午還宫【往還亦幾三旬】 辛卯太子更名亨【更工衡翻】海賊吳令光等抄掠台明【明州漢句章鄮縣之地屬會稽郡開元二十六年采訪使齊澣奏以越州之鄮縣置明州以境内有四明山因名抄楚交翻】命河南尹裴敦復將兵討之【將即亮翻】 三月己巳以平盧節度使安祿山兼范陽節度使以范陽節度使裴寛為戶部尚書【使疏吏翻尚辰羊翻】禮部尚書席建侯為河北黜陟使稱祿山公直李林甫裴寛皆順旨稱其美三人皆上所信任由是祿山之寵益固不搖矣 夏四月裴敦復破吳令光擒之 五月河西節度使夫蒙靈詧討突騎施莫賀達干斬之【按元和姓纂云夫蒙本西羌姓後秦有建威將軍夫蒙羌今蒲同二州多此姓或改姓馬氏按考異曰會要作馬靈詧今從實錄】更請立黑姓伊里底蜜施骨咄祿毗伽【咄當沒翻伽求迦翻 考異曰會要作伊地米里骨咄禄毗伽今從實錄】 六月甲辰冊拜骨咄祿毗伽為十姓可汗【可從刋入聲汗音寒】 秋八月拔悉蜜攻斬突厥烏蘇可汗傳首京師國人立其弟鶻隴匐白眉特勒是為白眉可汗於是突厥大亂敇朔方節度使王忠嗣出兵乘之至薩河内山【厥九勿翻嗣祥吏翻薩桑葛翻】破其左廂阿波達干等十一部右廂未下會囘紇葛邏祿共攻拔悉蜜頡跌伊施可汗殺之回紇骨力裴羅自立為骨咄祿毗伽闕可汗遣使言狀上冊拜裴羅為懷仁可汗於是懷仁南據突厥故地立牙帳於烏德犍山【回紇牙帳東有平野西據烏德犍山南依嗢昆水犍居言翻】舊統藥邏葛等九姓其後又并拔悉蜜葛邏祿凡十一部各置都督每戰則以二客部為先 李林甫以楊慎矜屈附於己九月甲戍復以慎矜為御史中丞充諸道鑄錢使【復扶又翻又如字】 冬十月癸巳上幸驪山溫泉十一月丁卯還宫 術士蘇嘉慶上言遯甲術有九宫貴神【九宫貴神蓋易乾鑿度所謂太一也注已見五十二卷漢順帝陽嘉三年時置九宫貴神壇其壇三成成三尺四階其上依位置九壇壇尺五寸東南曰招搖正東曰軒轅東北曰太隂正南曰天一中央曰天符正北曰太一西南曰攝提正西曰咸池西北曰青龍五為中戴九履一左三右七二四為上六八為下符於遁甲仍編於勅曰九宫貴神實司水旱功佐上帝德庇下人又黄帝九宫經一宫其神太一其星天蓬其卦坎其行水其方白二宫其神攝提其星天内其卦坤其行土其方黑三宫其神軒轅其星天衝其卦震其行木其方碧四宫其神招搖其星天輔其卦巽其行木其方緑五宫其神天符其星天禽其卦離其行土其方黄六宫其神青龍其星天心其卦乾其行金其方白七宫其神咸池其星天柱其卦兌其行金其方赤八宫其神太陰其星天任其卦艮其行土其方白九宫其神天一其星天英其卦離其行火其方紫上時掌翻】典司水旱請立壇於東郊祀以四孟月從之禮在昊天上帝下太清宫太廟上所用牲玉皆侔天地 十二月癸巳置會昌縣於溫泉宫下【時分新豐萬年置會昌縣雍錄温湯在臨潼縣南一百五十步在驪山西北十道志曰泉有三所其一處即皇堂石井後周宇文護所造隋文帝又修屋宇種松栢千餘株唐貞觀十八年詔閻立德營建宫殿御湯名湯泉宫咸亨三年名温泉宫元和志則曰開元十一年置温泉宫天寶六載改為華清宫於驪山上益治湯井為池臺殿環列山谷自開元來每歲十月臨幸歲盡乃歸以新豐縣去泉稍遠即於湯所置會昌縣又置百司及公卿邸第焉臨潼縣唐之新豐慶山皆其地也按通鑑開元十一年書作温泉宫與元和志合】 戶部尚書裴寛素為上所重李林甫恐其入相忌之【相息亮翻】刑部尚書裴敦復擊海賊還受請託廣序軍功寛微奏其事林甫以告敦復敦復言寛亦嘗以親故屬敦復【屬之欲翻】林甫曰君速奏之勿後於人敦復乃以五百金賂女官楊太眞之姊使言於上甲午寛坐貶睢陽太守【睢陽郡宋州本梁郡天寶元年更郡名睢音雖】初武惠妃薨【開元二十五年惠妃薨】上悼念不已後宫數千無當意者或言夀王妃楊氏之美絶世無雙上見而悦之乃令妃自以其意乞為女官號太眞更為夀王娶左衛郎將韋昭訓女【更工衡翻將即亮翻】濳内太真宫中太真肌態豐豔曉音律性警穎善承迎上意不朞歲寵遇如惠妃宫中號曰娘子凡儀體皆如皇后 癸卯以宗女為和義公主嫁寜遠奉化王阿悉爛達干【帝以拔汗那助平吐火仙冊其王為奉化王改其國曰寜遠】 癸丑上祀九宫貴神赦天下 初令百姓十八為中二十三成丁 初上自東都還李林甫知上厭巡幸乃與牛仙客謀增近道粟賦及和糴以實關中數年蓄積稍豐上從容謂高力士曰朕不出長安近十年【開元二十四年上自東都還自是不復東幸從千容翻近其靳翻】天下無事朕欲高居無為悉以政事委林甫何如對曰天子巡狩古之制也且天下大柄不可假人彼威勢既成誰敢復議之者上不悦力士頓首自陳臣狂疾發妄言罪當死上乃為力士置酒【復扶又翻為于偽翻下為百同】左右皆呼萬歲力士自是不敢深言天下事矣【力士之不敢言以李林甫機穽可畏也】<br />
<br />
  四載春正月庚午上謂宰相曰朕比以甲子日【比毗至翻】於宫中為壇為百姓祈福朕自草黄素置案上俄飛升天聞空中語云聖夀延長又朕於嵩山鍊藥成亦置壇上及夜左右欲收之又聞空中語云藥未須收此自守護達曙乃收之太子諸王宰相皆上表賀【史言唐之君誕妄而臣謟諛】囘紇懷仁可汗擊突厥白眉可汗殺之傳首京師突<br />
<br />
  厥毗伽可敦帥衆來降【可從刋入聲帥讀曰率降戶江翻】於是北邊晏然烽燧無警矣囘紇斥地愈廣東際室韋西抵金山南跨大漠盡有突厥故地【史言囘紇至此彊盛】懷仁卒子磨延啜立號葛勒可汗二月己酉以朔方節度使王忠嗣兼河東節度使忠嗣少以勇敢自負【少詩照翻】及鎮方面專以持重安邊為務常曰太平之將但當撫循訓練士卒而已不可疲國中之力以邀功名【將即亮翻】有漆弓百五十斤常貯之櫜中以示不用【貯丁呂翻】軍中日夜思戰忠嗣多遣諜人伺其間隙【諜達協翻伺相吏翻間古莧翻】見可勝然後興師故出必有功既兼兩道節制自朔方至雲中邊陲數千里【陲音垂】要害之地悉列置城堡斥地各數百里邊人以為自張仁亶之後將帥皆不及【張仁愿本名仁亶以睿宗諱旦音近亶避之改名仁愿將即亮翻帥所類翻】三月壬申上以外孫獨孤氏為靜樂公主嫁契丹王李懷節【樂音洛】甥楊氏為宜芳公主嫁奚王李延寵【宜芳縣屬嵐州】 乙巳以刑部尚書裴敦復充嶺南五府經略等使五月壬申敦復坐逗留不之官貶淄川太守【淄川郡淄州舊志淄州京師東北二千一百三十三里守式又翻】以光祿少卿彭杲代之上嘉敦復平海賊之功故李林甫陷之 李適之與李林甫爭權有隙適之領兵部尚書駙馬張垍為侍郎【垍其冀翻】林甫亦惡之【惡烏路翻】使人發兵部銓曹姦利事收吏六十餘人付京兆與御史對鞫之數日竟不得其情京兆尹蕭炅使法曹吉温鞫之【法曹司法參軍事掌鞫獄麗法知贓賄沒入炅火迥翻】温入院置兵部吏於外先於後廳取二重囚訊之或杖或壓號呼之聲所不忍聞【號戶高翻】皆曰苟存餘生乞紙盡荅兵部吏素聞温之慘酷引入皆自誣服無敢違温意者頃刻而獄成驗囚無榜掠之迹【榜音彭掠音亮】六月辛亥敕誚責前後知銓侍郎及判南曹郎官而宥之【文宗開成二年宰相李石奏定長定選格吏部請加置南曹郎中一人别置卬以新置南曹之印為文蓋吏部先以郎官判南曹開成間因置南曹郎也宋白曰南曹起於總章二年司列常伯李敬玄奏置未置已前銓中自勘責故事兩轉廳至建中元年侍郎邵說奏挾闕替南曹郎中王鋗已後遂不轉廳貞元十一年侍郎杜黄裳請凖舊例轉廳後云云同上誚才笑翻】垍均之兄温頊之弟子也【吉頊進用於武后之朝】温始為新豐丞太子文學薛嶷薦温才【唐六典曰魏置太子文學魏武為丞相命司馬宣王為文學椽甚為世子所信與吳質朱鑠陳羣為太子四友自晉之後不置至後周建德三年置太子文學十人後廢唐顯慶中始置太子文學二人屬司經局掌分知經籍侍奉文章總緝經籍繕寫裝染之功筆札給用之數皆料度之嶷魚力翻】上召見【見賢遍翻】顧嶷曰是一不良人朕不用也蕭炅為河南尹嘗坐事西臺遣温往按之【西臺西京御史臺也】温治炅甚急【治直之翻下同】及温為萬年丞未幾炅為京兆尹【幾居豈翻】温素與高力士相結力士自禁中歸温度炅必往謝官【度徒洛翻】乃先詣力士與之談謔握手甚歡【謔迄郤翻】炅後至温陽為驚避力士呼曰吉七不須避【吉温第七】謂炅曰此亦吾故人也召還與炅坐炅接之甚恭不敢以前事為怨它日温謁炅曰曩者温不敢隳國家法自今請洗心事公炅遂與盡歡引為法曹 【考異曰唐歷云温聯按大獄附邪以出入人命者凡十餘年性巧詆忍而不忌失意眉睫間必引而陷之其欲膠固之雖王公大人立可親也初蕭炅以贓下獄温深竟其罪後為萬年縣丞炅拜京兆尹温見炅於高力士第乃與之相結為膠漆之交引為法曹而薦于林甫温之進也反以炅力舊傳云炅為河南尹有事京臺差温推詰堅執不捨及温選炅已為京兆尹一唱萬年尉即就其官人為危之今參取二書用之】及林甫欲除不附己者求治獄吏炅薦温於林甫林甫得之大喜温常曰若遇知己南山白額虎不足縛也時又有杭州人羅希奭為吏深刻林甫引之自御史臺主簿再遷殿中侍御史【唐御史臺主簿從七品上掌印及受事發辰勾檢稽失兼知官厨及黄卷】二人皆隨林甫所欲深淺鍛鍊成獄無能自脱者時人謂之羅鉗吉網【以鐵劫束物曰鉗鉗其亷翻】秋七月壬午冊韋昭訓女為夀王妃八月壬寅冊楊太真為貴妃 【考異曰統紀八月冊女道士楊氏為貴妃本紀甲辰唐歷甲寅今據實錄壬寅贈太真妃父玄琰等官甲辰甲寅皆在後恐冊妃在贈官前新本紀亦云八月壬寅立太真為貴妃今從之】贈其父玄琰兵部尚書以其叔父玄珪為光祿卿從兄銛為殿中少監錡為駙馬都尉【從才用翻下之從同銛息亷翻錡渠綺翻又魚綺翻又音奇】癸卯冊武惠妃女為太華公主命錡尚之 【考異曰實錄舊傳皆以銛錡為再從兄國忠為從祖兄然則從祖亦再從也推恩之時何以及銛錡而不及國忠新傳謂之宗兄唐歷以銛為玄琰之子借使非子比于國忠必應稍親今但謂之從兄舊傳云錡為侍御史今從實錄】及貴妃三姊皆賜第京師寵貴赫然楊釗貴妃之從祖兄也不學無行【釗音昭行下孟翻】為宗黨所鄙從軍於蜀得新都尉考滿家貧不能自歸新政富民鮮于仲通常資給之【新都縣漢屬廣漢郡梁置始康郡西魏廢郡隋開皇十八年改新都曰興樂尋廢縣唐初復置屬蜀郡武德四年分南部相如二縣置新城縣尋避隱太子名改曰新政時屬閬中郡】楊玄琰卒於蜀【卒子恤翻】釗往來其家遂與其中女通【中讀曰仲】鮮于仲通名向以字行頗讀書有材智劒南節度使章仇兼瓊引為采訪支使 【考異曰唐歷云為節度巡官按顔真卿所作仲通碑見存云為采訪支使今從之 唐采訪節度等使幕屬有判官有支使有掌書記推官巡官衙推等宋朝始定制書記支使不得並置有出身者為書記無出身者為支使】委以心腹嘗從容謂仲通曰今吾獨為上所厚苟無内援必為李林甫所危聞楊妃新得幸人未敢附之子能為我至長安與其家相結吾無患矣仲通曰仲通蜀人未嘗遊上國恐敗公事今為公更求得一人因言釗本末兼瓊引見釗儀觀豐偉言辭敏給【從千容翻為于偽翻敗補邁翻觀古玩翻】兼瓊大喜即辟為推官往來浸親密乃使人獻春綵於京師將别謂曰有少物在郫【少詩沼翻郫縣自漢以來屬蜀郡九域志耶縣在成都府西四十五里師古曰郫音疲】以具一日之糧子過可取之釗至郫兼瓊使親信大齎蜀貨精美者遺之【遺于季翻】可直萬緡釗大喜過望晝夜兼行至長安歷抵諸妹以蜀貨遺之曰此章仇公所贈也時中女新寡釗遂館於其室中分蜀貨以與之於是諸楊日夜譽兼瓊且言釗善摴蒱引之見上【館古玩翻譽音余見賢遍翻】得隨供奉官出入禁中【唐制中書門下省官皆供奉官也外官得隨朝士入見者謂之仗内供奉隨翰林院官班者謂之翰林供奉宦官謂之内供奉又有朝士供奉禁中者】改金吾兵曹參軍 九月癸未以陜郡太守江淮租庸轉運使韋堅為刑部尚書罷其諸使以御史中丞楊慎矜代之【陜失冉翻使疏吏翻 考異曰舊食貨志三載以楊釗為水陸運使誤也今從實錄】堅妻姜氏皎之女林甫之舅子也故林甫昵之【昵尼質翻】及堅以通漕有寵于上遂有入相之志【相息亮翻】又與李適之善林甫由是惡之【惡烏路翻】故遷以美官實奪之權也 安祿山欲以邊功市寵數侵掠奚契丹奚契丹各殺公主以叛【數所角翻下欲數數徵同所殺者蓋即靜樂宜芳也】祿山討破之 隴右節度使皇甫惟明與吐蕃戰于石堡城為虜所敗副將褚誗戰死【敗補邁翻誗直亷翻 考異曰新傳作諸葛誗今從實錄】 冬十月甲午安祿山奏臣討契丹至北平郡【北平郡平州】夢先朝名將李靖李勣從臣求食【朝直遥翻將即亮翻下同】遂命立廟又奏薦奠之日廟梁產芝【通鑑不語怪而書安祿山飛鳥食蝗廟梁產芝之事以著祿山之欺罔明皇之昏蔽】 丁酉上幸驪山温泉 上以戶部郎中王鉷為戶口色役使敕賜百姓復除【鉷戶公翻使疏吏翻復方目翻】鉷奏徵其輦運之費廣張錢數又使市本郡輕貨百姓所輸乃甚於不復除舊制戍邊者免其租庸六歲而更【更工衡翻】時邊將恥敗士卒死者皆不申牒貫籍不除【貫籍本貫之籍也】王鉷志在聚歛【斂力贍翻】以有籍無人者皆為避課按籍戍邊六歲之外悉徵其租庸有併徵三十年者民無所訴上在位久用度日侈後宫賞賜無節不欲數于左右藏取之【唐有左藏右藏藏徂浪翻】鉷探知上指歲貢額外錢百億萬貯於内庫以供宫中宴賜曰此皆不出於租庸調無預經費【貯丁呂翻調徒弔翻】上以鉷為能富國益厚遇之鉷務為割剝以求媚中外嗟怨丙子以鉷為御史中丞京畿采訪使楊釗侍宴禁中專掌摴蒱文簿鈎校精密上賞其彊明曰好度支郎【唐度支郎掌判天下租賦多少之數物產豐約之宜水陸道塗之利每歲計其所出而度其所用轉運徵斂送納皆凖程而節其遲速凡和糴和市皆量其貴賤均天下之貨以利于人凡金銀寶貨綾羅之屬皆折庸調以造凡天下舟車水陸載運皆具為脚直輕重貴賤平易險澁而為之制凡天下邊軍有支度使以計軍資糧仗之用每歲所費皆申度支會計以長行旨為凖度徒洛翻】諸楊數徵此言於上【徵讀曰證】又以屬王鉷鉷因奏充判官【屬之欲翻】 十二月戊戌上還宫【還自温泉宫還從宣翻又音如字】<br />
<br />
  五載春正月乙丑以隴右節度使皇甫惟明兼河西節度使李適之性疎率李林甫嘗謂適之曰華山有金礦【華戶化翻西山記曰太華之山削成而四方其高五十仭其廣十里礦古猛翻】采之可以富國主上未之知也它日適之因奏事言之上以問林甫對曰臣久知之但華山陛下本命王氣所在【帝製華嶽碑曰予小子之生也歲景戌月仲秋膺少皥之盛德協太華之本命故常寤寐靈嶽肹蠁神文林甫知此旨故以誤適之而陷之王干况翻】鑿之非宜故不敢言上以林甫為愛已薄適之慮事不熟謂曰自今奏事宜先與林甫議之無得輕脱適之由是束手矣適之既失恩韋堅失權益相親密林甫愈惡之【惡烏路翻】初太子之立非林甫意【事見二百十卷開元二十六年】林甫恐異日為己禍常有動搖東宫之志而堅又太子之妃兄也皇甫惟明嘗為忠王友【見二百十三卷開元十八年】時破吐蕃入獻捷見林甫專權意頗不平時因見上乘間微勸上去林甫【吐從暾入聲因見賢遍翻間古莧翻去羌呂翻】林甫知之使楊慎矜密伺其所為【伺相吏翻】會正月望夜太子出遊與堅相見堅又與惟明會於景龍觀道士之室【景龍觀在長安城中崇仁坊申公高士亷宅西北左金吾衛神龍元年併為長寜公主宅韋庶人敗後遂立為觀仍以中宗年號為名觀古玩翻】慎矜發其事以為堅戚里不應與邊將狎暱林甫因奏堅與惟明結謀欲共立太子堅惟明下獄【將即亮翻暱尼質翻下遐嫁翻】林甫使慎矜與御史中丞王鉷京兆府法曹吉温共鞫之 【考異曰舊林甫傳云林甫濳令慎矜伺堅隙奏上慎矜傳云鉷推堅慎矜引身中立以候望鉷恨之林甫亦憾焉二傳自相矛楯今從唐歷】上亦疑堅與惟明有謀而不顯其罪癸酉下制責堅以干進不已貶縉雲太守【縉雲郡本括州永嘉郡元年更郡名 考異曰舊紀貶括蒼太守今從實錄及舊傳】惟明以離間君臣【間古莧翻】貶播川太守【播川郡播州】仍别下制戒百官 以王忠嗣為河西隴右節度使兼知朔方河東節度事忠嗣始在朔方河東每互市高估馬價諸胡聞之爭賣馬於唐忠嗣皆買之由是胡馬少【少詩沼翻】唐兵益壯及徙隴右河西復請分朔方河東馬九千匹以實之【復扶又翻】其軍亦壯忠嗣仗四節控制萬里天下勁兵重鎮皆在掌握與吐蕃戰於青海積石皆大捷又討吐谷渾於墨離軍虜其全部而歸【吐從暾入聲谷音浴】 夏四月癸未立奚酋娑固為昭信王契丹酋楷洛為恭仁王【酋慈由翻娑素禾翻】 己亥制自今四孟月皆擇吉日祀天地九宫 韋堅等既貶左相李適之懼自求散地【散悉但翻】庚寅以適之為太子少保罷政事其子衛尉少卿霅嘗盛饌召客【霅文甲翻饌雛皖翻又雛戀翻】客畏李林甫竟日無一人敢往者 以門下侍郎崇玄館大學士陳希烈同平章事【後魏置崇玄署掌僧尼道士女冠隋以崇玄署隸鴻臚唐置諸寺觀監隸鴻臚每寺觀有監一人貞觀中廢寺觀監上元二年置漆園監尋廢開元二十五年置崇玄學於玄元皇帝廟天寶元年兩京置博士助教各一員二載改崇玄學曰崇玄館博士曰學士助教曰直學士置大學士以宰相為之領兩京玄元宫及道院】希烈宋州人以講老莊得進專用神仙符瑞取媚於上李林甫以希烈為上所愛且柔佞易制【易以䜴翻】故引以為相凡政事一決於林甫希烈但給唯諾【唯于偽翻】故事宰相午後六刻乃出林甫奏今太平無事巳時即還第軍國機務皆決于私家主書抱成案詣希烈書名而已 五月壬子朔日有食之 乙亥以劒南節度使章仇兼瓊為戶部尚書諸楊引之也 秋七月丙辰敕流貶人多在道逗留自今左降官日馳十驛以上【上時掌翻】是後流貶者多不全矣 楊貴妃方有寵每乘馬則高力士執轡授鞭織繡之工專供貴妃院者七百人中外爭獻服器珍玩嶺南經略使張九章廣陵長史王翼【廣陵郡楊州長知兩翻】以所獻精美九章加三品翼入為戶部侍郎天下從風而靡民間歌之曰生男勿喜女勿悲君今看女作門楣【凡人作室自外至者見其門楣宏敞則為壯觀言楊家因生女而宗門崇顯也或曰門以楣而撐拄言生女能撐拄門戶也】妃欲得生荔支歲命嶺南馳驛致之【自蘇軾諸人皆云此時荔支自涪州致之非嶺南也】比至長安色味不變【白居易曰荔支生巴峽間樹形團團如帷蓋葉如冬青華如橘春榮實如丹夏熟朶如蒲萄核如枇杷殻如紅繒膜如紫綃瓤肉潔白如氷雪漿液甘酸如醴酪大略如彼其實過之若離本枝一日而色變二日而香變四五日外色香味盡去矣】至是妃以妬悍不遜上怒命送歸兄銛之第【悍戶罕翻又戶旰翻】是日上不懌比日中猶未食左右動不稱旨横被棰撻【比必利翻及也稱尺證翻棰止蘂翻横戶孟翻】高力士欲嘗上意請悉載院中儲偫送貴妃凡百餘車上自分御膳以賜之【偫直里翻】及夜力士伏奏請迎貴妃歸院遂開禁門而入【唐六典城門郎掌京城皇城宫殿諸門開闔之節承天門擊曉鼔聽擊鐘後一刻鼔聲絶皇城門開第一鼕鼕鼔聲絶宫城門及左右延明門乾化門開第二鼕鼕鼔聲絶宫殿門開夜第一鼕鼕鼔聲絶宫殿門閉第二鼕鼕鼓聲絶宫城門及左右延明門皇城門閉其京城門開閉與皇城門同刻承天門擊鼓皆聽漏刻契契至乃擊待漏刻所牌到鼔聲乃絶凡皇城闔門之鑰先酉而出後戌而入開門之鑰後丑而出夜盡而入京城闔門之鑰後中而出先子而入開門之鑰後子而出先卯而入若非其時而有命啓閉則詣閤覆奏奉旨合符而開闔之殿門及城門若有勑夜開受勑人具錄須開之門宣送中書門下】自是恩遇愈隆後官莫得進矣 將作少匠韋蘭兵部員外郎韋芝為其兄堅訟寃【少詩照翻為于偽翻】且引太子為言上益怒太子懼表請與妃離昏乞不以親廢法丙子再貶堅江夏别駕【江夏郡鄂州舊志鄂州京師東南二千三百四十六里夏戶雅翻】蘭芝皆貶嶺南然上素知太子孝謹故譴怒不及李林甫因言堅與李適之等為朋黨後數日堅長流臨封適之貶宜春太守太常少卿韋斌貶巴陵太守嗣薛王琄貶夷陵别駕睢陽太守裴寛貶安陸别駕河南尹李齊物貶竟陵太守【臨封郡木封州廣信郡元年更郡名宜春郡袁州巴陵郡岳州夷陵郡峽州安陸郡安州竟陵郡本復州沔陽郡元年更郡名舊志封州至京師水陸四千五百一十里岳州二千二百三十七里峽州一千八百八十八里安州京師東南二千五十一里復州一千八百里斌音彬琄胡畎翻】凡堅親黨坐流貶者數十人斌安石之子琄業之子堅之甥也琄母亦令隨琄之官【韋安石事武后中睿三朝業上之弟也】 冬十月戊戌上幸驪山温泉十一月乙巳還宫 贊善大夫杜有鄰【唐贊善大夫正五品上掌諷誦規諫太子】女為太子良娣【唐太子内官良娣正三品娣待計翻】良娣之姊為左驍衛兵曹柳勣妻【驍堅堯翻】勣性狂疎好功名喜交結豪俊【好呼到翻喜許記翻】淄川太守裴敦復薦於北海太守李邕【北海郡青州】邕與之定交勣至京師與著作郎王曾等為友皆當時名士也勣與妻族不協欲䧟之為飛語告有鄰妄稱圖讖交構東宫指斥乘輿【乘䋲證翻】林甫令京兆士曹吉温與御史鞫之【士曹司士參軍事掌津梁舟車舍宅工藝】乃勣首謀也温令勣連引曾等入臺十二月甲戌有鄰勣及曾等皆杖死積尸大理 【考異曰舊紀唐歷皆作辛未今從實錄實錄云勣與其黨並伏法詔書則曰猶寛極刑俾從杖罪其王曾等各決重杖一百杜有鄰柳勣念以微親特寛殊死決一頓貶嶺南新興尉吉温傳則云勣等杖死積尸於大理寺蓋詔雖與杖其實皆死杖下也】妻子流遠方中外震慄嗣虢王巨貶義陽司馬【義陽郡申州舊志申州至京師一千七百九十六里】巨邕之子也【高祖之子虢王鳳鳳嫡孫曰嗣虢王邕】别遣監察御史羅希奭往按李邕【監古銜翻】太子亦出良娣為庶人乙亥鄴郡太守王琚坐贓貶江華司馬【鄴郡本相州魏郡元年更名冮華郡道州】琚性豪侈與李邕皆自謂耆舊久在外意怏怏李林甫惡其負材使氣故因事除之【王琚事上於東宫贊決誅太平公主惡烏路翻】<br />
<br />
  六載春正月辛巳李邕裴敦復皆杖死邕才藝出衆盧藏用常語之曰君如干將莫邪【語牛倨翻干將莫邪吳王所鑄寶劍】難與爭鋒然終虞缺折耳【折而設翻】邕不能用林甫又奏分遣御史即貶所賜皇甫惟明韋堅兄弟等死羅希奭自青州如嶺南所過殺遷謫者【希奭既殺李邕於青州遂如嶺南也】郡縣惶駭排馬諜至宜春【御史所過沿路郡縣給驛馬故未至先有排馬牒】李適之憂懼仰藥自殺至江華王琚仰藥不死聞希奭已至即自縊希奭又过路過安陸欲怖殺裴寛【怖普布翻】寛向希奭叩頭祈生希奭不宿而過乃得免李適之子霅迎父喪至東京李林甫令人誣告霅杖死於河南府給事中房琯至與適之善貶宜春太守琯融之子也【房融見二百七卷武后長安四年】林甫恨韋堅不已遣使於循河及江淮州縣求堅罪【使疏吏翻】收繫綱典船夫溢於牢獄【十船為一綱以吏為綱典船夫輓船及駕船之夫也】徵剝逋負延及鄰伍皆裸露死於公府【裸郎果翻】至林甫薨乃止 丁亥上享太廟戊子合祭天地於南郊赦天下制免百姓今載田租又令削絞斬條上慕好生之名故令應絞斬者皆重杖流嶺南其實有司率杖殺之又令天下為嫁母服三載【載子亥翻好呼到翻為于偽翻下即為同】上欲廣求天下之士命通一藝以上皆詣京師李林甫恐草野之士對策斥言其姦惡建言舉人多卑賤愚聵【聵五怪翻】恐有俚言汚濁聖聽乃令郡縣長官精加試練灼然超絶者具名送省委尚書覆試御史中丞監之【監古銜翻】取名實相副者聞奏既而至者皆試以詩賦論遂無一人及第者林甫乃上表賀野無遺賢戊寅以范陽平盧節度使安祿山兼御史大夫祿山體充肥腹垂過膝嘗自稱腹重三百斤外若癡直内實狡黠常令其將劉駱谷留京師詗朝廷指趣動靜皆報之【黠下八翻詗翾正翻有所候伺也】或應有牋表者駱谷即為代作通之歲獻俘虜雜畜奇禽異獸珍玩之物【畜許救翻】不絶於路郡縣疲於逓運祿山在上前應對敏給雜以詼諧上嘗戲指其腹曰此胡腹中何所有其大乃爾對曰更無餘物止有赤心耳上悦又嘗命見太子祿山不拜左右趣之拜【趣讀曰促】祿山拱立曰臣胡人不習朝儀不知太子者何官【朝直遥翻】上曰此儲君也朕千秋萬歲後代朕君汝者也祿山曰臣愚曏者惟知有陛下一人不知乃更有儲君不得已然後拜上以為信然益愛之上嘗宴勤政樓百官列坐樓下獨為祿山於御座東間設金雞障【障坐障也畫金雞為飾為于偽翻】置榻使坐其前仍命卷簾以示榮寵【卷讀曰捲】命楊銛楊錡貴妃三姊皆與祿山叙兄弟祿山得出入禁中因請為貴妃兒上與貴妃共坐祿山先拜貴妃上問何故對曰胡人先母而後父上悦李林甫以王忠嗣功名日盛恐其入相【相息亮翻】忌之安<br />
<br />
  祿山潛蓄異志託以禦寇築雄武城【薊州廣漢川有雄武軍】大貯兵器【貯丁呂翻】請忠嗣助役因欲留其兵忠嗣先期而往不見祿山而還數上言祿山必反林甫益惡之【為王忠嗣得罪張本先悉薦翻數所角翻上時掌翻惡烏路翻】 夏四月忠嗣固辭兼河東朔方節度許之 冬十月己酉上幸驪山温泉 【考異曰舊紀唐歷皆作戊申今從之】改温泉宫曰華清宫 河西隴右節度使王忠嗣以部將哥舒翰為大斗軍副使李光弼為河西兵馬使充赤水軍使【兵馬使節鎮衙前軍職也總兵權任甚重至德以後都知兵馬使率為藩鎮儲帥將即亮翻使疏吏翻】翰父祖本突騎施别部酋長【西突厥五弩失畢有哥舒闕俟斤騎奇寄翻酋慈由翻長知兩翻】光弼契丹王楷洛之子也【開元初李楷洛封為契丹王】皆以勇略為忠嗣所重忠嗣使翰擊吐蕃有同列為之副倨慢不為用翰檛殺之軍中股慄【檛則瓜翻】累功至隴右節度副使每歲積石軍麥熟吐蕃輒來穫之【穫戶郭翻】無能禦者邊人謂之吐蕃麥莊翰先伏兵於其側虜至斷其後夾擊之【斷音短】無一人得返者自是不敢復來【復扶又翻】上欲使王忠嗣攻吐蕃石堡城【石堡城陷見上卷開元二十九年】忠嗣上言石堡險固吐蕃舉國守之今頓兵其下非殺數萬人不能克臣恐所得不如所亡不如且厲兵秣馬俟其有釁然後取之上意不快將軍董延光自請將兵取石堡城【請將即亮翻】上命忠嗣分兵助之忠嗣不得已奉詔而不盡副延光所欲延光怨之李光弼言於忠嗣曰大夫以愛士卒之故不欲成延光之功【唐中世以前率呼將帥為大夫白居易詩所謂武官稱大夫是也】雖廹於制書實奪其謀也何以知之今以數萬衆授之而不立重賞士卒安肯為之盡力乎【為于偽翻】然此天子意也彼無功必歸罪於大夫大夫軍府充牣何愛數萬段帛不以杜其讒口乎忠嗣曰今以數萬之衆爭一城得之未足以制敵不得亦無害於國故忠嗣不欲為之忠嗣今受責天子不過以金吾羽林一將軍歸宿衛其次不過黔中上佐【黔中一道皆溪峒蠻徭雜居貶謫而不過嶺者處之上佐長史司馬也黔音琴】忠嗣豈以數萬人之命易一官乎李將軍子誠愛我矣然吾志決矣子勿復言光弼曰曏者恐為大夫之累【復扶又翻累力瑞翻】故不敢不言今大夫能行古人之事非光弼所及也遂趨出延光過期不克言忠嗣沮撓軍計【沮在呂翻撓女教翻】上怒李林甫因使濟陽别駕魏林告忠嗣嘗自言我幼養宫中與忠王相愛狎【武德四年分東平之盧縣置濟州隋之濟北郡也天寶元年改曰濟陽郡忠嗣年九歲父海賓戰死于渭源長城堡帝養忠嗣宫中太子時為忠王與之遊處魏林先為朔州刺史忠嗣節度河東朔州其巡屬也故使林譖之以示言冇所自來濟子禮翻】欲擁兵以尊奉太子敕徵忠嗣入朝【朝直遥翻】委三司鞫之上聞哥舒翰名召見華清宫【見賢遍翻】與語悦之十一月辛卯以翰判西平太守充隴右節度使以朔方節度使安思順判武威郡事充河西節度使【西平郡鄯州武威郡凉州】戶部侍郎兼御史中丞楊慎矜為上所厚李林甫浸忌之慎矜與王鉷父晉中表兄弟也少與鉷狎【少詩照翻】鉷之入臺頗因慎矜推引及鉷遷中丞慎矜與語猶名之鉷自恃與林甫善意稍不平慎矜奪鉷職田【俱為中丞因併鉷職田奪而有之】鉷母本賤慎矜嘗以語人【語牛倨翻】鉷深銜之慎矜猶以故意待之嘗與之私語讖書【讖楚譖翻】慎矜與術士史敬忠善敬忠言天下將亂勸慎矜於臨汝山中買莊為避亂之所【臨汝郡本伊州襄城郡貞觀八年更伊州曰汝州天寶元年更郡名為臨汝郡】會慎矜父墓田中草木皆流血慎矜惡之【惡烏路翻】以問敬忠敬忠請禳之【禳如羊翻】設道塲於後園慎矜退朝輒躶貫桎梏坐其中【朝直遙翻躶郎果翻】旬日血止慎矜德之慎矜有侍婢明珠色美敬忠屢目之慎矜即以遺敬忠車載過貴妃姊柳氏樓下姊邀敬忠上樓求車中美人敬忠不敢拒明日姊入宫以明珠自隨上見而異之問所從來明珠具以實對上以慎矜與術士為妖法惡之含怒未發楊釗以告鉷鉷心喜因侮慢慎矜慎矜怒林甫知鉷與慎矜有隙密誘使圖之【遺于季翻上時掌翻惡烏路翻誘音酉考異曰明皇雜錄曰慎矜父墓封域之内草木流血慎矜大懼問術者史敬思敬思曰禳之可以免於慎矜後園大陳法事令貫桎梏坐於叢林間以厭之唐歷云敬思本胡人出家還俗涉獵書傳陰陽玄象慎矜與之善每言天下將亂居於臨汝山中亦勸慎矜於臨汝買得山莊良田數十頃嘗於慎矜第夜坐談宴怒婢春草將杖殺之敬思曰七郎何須虚殺却十頭壯牛慎矜曰何謂也敬思曰賣却買牛每年耕田十頃慎矜雅厚敬思曰任公收取明曰至市賣與大真柳氏姊得錢百二十千文買牛以歸柳氏數將春草來往宫中玄宗見其狀貌壯大應對分明數目之謂柳曰幾錢買得此婢以實對遂留之玄宗曾晝寢問春草曰汝木何人何以得至柳家春草曰本楊慎矜婢賣與柳家玄宗曰慎矜豈少錢而賣你春草曰不是要錢木將殺某敬思救得不殺所以賣之玄宗素聞敬思名因詰問春草以實對曰每夜坐中庭或說天文遙指宿曜某亦盡知其言玄宗怒變色良久後王鉷因奏事言引慎矜玄宗勃然曰慎矜與卿有親更不須相往來鉷初内怨慎矜凌已常隱忍不泄至是覺上意異楊釗先知之以告鉷鉷心喜數悖慢似侵之慎矜尤怒明皇雜録又曰慎矜之婢有美者字明珠敬思數目之慎矜即以遺之兼以囊槖甚厚以車送之敬思乘馬隨之路經貴妃妹八姨樓下方登樓張樂姨素與敬思相識固邀敬思登樓乃曰車中美人請以見遺敬思不敢拒姨明日入宫婢從上見而異之問所從來明珠曰本楊慎矜家人也近贈史敬思上曰敬思何人而慎矜輒贈以婢明珠乃具言厭勝之事上大怒曰彼為妖乎遂告林甫林甫素忌慎矜才恐其作相以告中丞吉温温險害亦有憾于慎矜因構成其事今參取書之】鉷乃遣人以飛語告慎矜隋煬帝孫【慎矜隋煬帝之玄孫】與凶人往來家有讖書謀復祖業上大怒收愼矜繫獄命刑部大理與侍御史楊釗殿中侍御史盧鉉同鞫之太府少卿張瑄慎矜所薦也盧鉉誣瑄嘗與愼矜論讖栲掠百端瑄不肯荅辯【瑄音宣掠音亮辯者鞫問之辭今人謂之問頭】乃以木綴其足使人引其枷柄向前挽之身加長數尺腰細欲絶眼鼻出血瑄竟不荅又使吉温捕史敬忠於汝州敬忠與温父素善温之幼也敬忠常抱撫之及捕獲温不與交言鎖其頸以布蒙首驅之馬前至戲水【戲水在新豐東戲許宜翻】温使吏誘之曰楊慎矜已欵服惟須子一辯若解人意則生【解戶買翻】不然必死前至温湯則求首不獲矣【温湯即謂會昌時置會昌縣於温泉宫下首式又翻謂自首其事或曰首如字】敬忠顧謂温曰七郎求一紙温陽不應去温湯十餘里敬忠祈請哀切乃于桑下令荅三紙辯皆如温意温徐謂曰丈人且勿恠因起拜之至會昌【天寶元年改驪山曰會昌山三載以新豐縣去華清宫遠分新豐萬年置會昌縣是年改温泉曰華清宫治湯井為池環山列宫室又築羅城置百司及十宅】始鞫愼矜以敬忠為證愼矜皆引服惟搜讖書不獲林甫危之使盧鉉入長安搜愼矜家鉉袖讖書入闇中詬而出曰【詬古候翻】逆賊深藏秘記至會昌以示慎矜愼矜歎曰吾不蓄讖書此何從在吾家哉吾應死而已丁酉賜愼矜及兄少府少監愼餘洛陽令愼名自盡敬忠杖百妻子皆流嶺南瑄杖六十流臨封死於會昌嗣虢王巨雖不預謀坐與敬忠相識解官南賓安置【南賓郡忠州本巴郡之臨江縣隋義寜二年置臨州貞觀八年改忠州天寶元年改為郡舊志忠州京師南二千一百二十二里】自餘連坐者數十人慎名聞敕神色不變為書别姊愼餘合掌指天而縊 三司按王忠嗣上曰吾兒居深宮安得與外人通謀此必妄也 【考異曰新傳李林甫屢白太子宜有謀上云云按林甫雖志欲害太子亦未肯自言之今不取】但劾忠嗣沮撓軍功【劾戶槩翻又戶得翻沮在呂翻撓奴巧翻又奴教翻】哥舒翰之入朝也或勸多齎金帛以救忠嗣翰曰若直道尚存王公必不寃死如其將喪【喪息浪翻】多賂何為遂單囊而行三司奏忠嗣罪當死翰始遇知於上力陳忠嗣之寃且請以己官爵贖忠嗣罪上起入禁中翰叩頭隨之言與淚俱上感寤己亥貶忠嗣漢陽太守【漢陽郡沔州】 李林甫屢起大獄别置推事院於長安以楊釗有掖庭之親【掖音亦】出入禁闥所言多聽乃引以為援擢為御史事有微涉東宫者皆指擿使之奏劾付羅希奭吉温鞫之釗因得逞其私志所擠䧟誅夷者數百家皆釗發之【擿他歷翻擠子西翻又子細翻】幸太子仁孝謹靜張垍高力士常保護於上前故林甫終不能間也【間古莧翻 考異曰明皇雜錄云上與李林甫議立太子意屬忠王林甫從容言于上曰古者建立儲君必推賢德苟非有大勲於社稷則惟元子上默然曰朕長子琮往年因獵苑中所傷面目尤甚林甫曰破面不猶愈於破國乎陛下其圖之上微感其言徐思之林甫亦素知其有疾意欲動搖肅宗而託附武惠妃因以夀王瑁為請竟以肅宗孝友聰明中外所屬故姦邪之計莫得行焉按是時忠王若未為太子上用林甫之言則琮為太子矣安能及瑁新書林甫傳云林甫數危太子未得志一日從容曰古者立儲君非冇大勲于社稷則莫若元子帝久之曰慶王獵為豽傷面甚荅曰破面不愈於破國乎帝頗惑曰朕徐思之此則情理似近然新書此事必出于雜錄若太子已立則不當云止與林甫議立太子意屬忠王也今雜錄本於所傷字上脱為豽兩字别本必有之按說文豽獸名無前足此非常有之物或者豹字誤為豽字耳事既可疑今不取】十二月壬戌發馮翊華陰民夫築會昌城置百司【華陰郡華州馮翊郡同州華戶化翻】王公各置第舍土畝直千金癸亥上還宫丙寅命百官閲天下歲貢物於尚書省既而悉以車<br />
<br />
  載賜李林甫家上或時不視朝【朝直遥翻】百司悉集林甫第門臺省為空陳希烈雖坐府無一人入謁者林甫子岫為將作監【唐初曰將作人匠龍朔改曰繕工監光宅改曰營繕監神龍復曰將作監】頗以滿盈為懼嘗從林甫遊後園指役夫言於林甫曰大人久處鈞軸怨仇滿天下一朝禍至欲為此得乎林甫不樂曰勢已如此將若之何先是宰相皆以德度自處不事威勢騶從不過數人士民或不之避林甫自以多結怨常虞刺客出則步騎百餘人為左右翼金吾靜街前驅在數百步外公卿走避居則重關複壁【處昌呂翻樂音洛先悉薦翻騶則尤翻從才用翻騎奇寄翻下同重直龍翻】以石甃地【甃則又翻】牆中置板如防大敵一夕屢徙牀雖家人莫知其處宰相騶從之盛自林甫始 初將軍高仙芝本高麗人【麗力知翻】從軍安西仙芝驍勇善騎射【驍堅堯翻】節度使夫蒙靈詧屢薦至安西副都護都知兵馬使充四鎭節度副使吐蕃以女妻小勃律王【小勃律去長安九千里而嬴距吐蕃贊普牙三千里妻七細翻】及其旁二十餘國皆附吐蕃貢獻不入前後節度使討之皆不能克制以仙芝為行營節度使將萬騎討之【將即亮翻】自安西行百餘日乃至特勒滿川分軍為三道【特勒滿川即五識匿國所居三道一由北谷道一由赤佛道仙芝自由護密道自護密勒城南至小勃律國都五百里】期以七月十三日會吐蕃連雲堡下【連雲堡南依山北據娑勒川以為固】有兵近萬人【近其靳翻】不意唐兵猝至大驚依山拒戰礟櫑如雨仙芝以郎將高陵李嗣業為陌刀將【礟匹貌翻櫑盧對翻礟即砲石杜佑曰櫑木長五尺徑一尺小至六七寸唐六典武庫分掌兵器辨其名數以備國用刀之制有四曰儀刀曰障刀曰横刀曰陌刀儀刀蓋古斑劍之類宋晉以來謂之御刀後魏曰長刀皆施龍鳳環至隋謂之儀刀装飾以金銀羽儀所執障刀蓋用以障身以禦敵横刀佩刀也兵士所佩名亦起於隋陌刀長刀也步兵所持蓋古之斬馬劍】令之曰不及日中決須破虜嗣業執一旗引陌刀緣險先登力戰自辰至巳大破之 【考異曰舊嗣業傳云天寶七載今從實錄及封常清傳】斬首五千級捕虜千餘人餘皆逃潰中使邊令誠以入虜境已深懼不敢進【邉令誠時為監軍使疏吏翻】仙芝乃使令誠以羸弱三千守其城【羸倫為翻】復進三日至坦駒嶺下峻阪四十餘里【復扶又翻】前有阿弩越城仙芝恐士卒憚險不肯下先令人胡服詐為阿弩越城守者迎降【降戶江翻】云阿弩越赤心歸唐娑夷水籐橋已斫斷矣娑夷水即弱水也【小勃律王居孽多城臨娑夷水娑素禾翻】其水不能勝草芥【勝音升】籐橋者通吐蕃之路也仙芝陽喜士卒乃下又三日阿弩越城迎者果至明日仙芝入阿弩越城遣將軍席元慶將千騎前行謂曰小勃律間大軍至其君臣百姓必走山谷【走音奏】第呼出取繒帛稱敕賜之【繒慈陵翻】大臣至盡縛之以待我元慶如其言悉縛諸大臣王及吐蕃公主逃入石窟取不可得仙芝至斬其附吐蕃者大臣數人籐橋去城猶六十里仙芝急遣元慶往斫之甫畢吐蕃兵大至已無及矣籐橋闊盡一矢力修之朞年乃成八月仙芝虜小勃律王及吐蕃公主而還九月至連雲堡與邊令誠俱月末至播密川遣使奏狀【奏捷狀於京師使疏吏翻】至河西【此河西白馬河西也自安西西出柘厥關度白馬河】夫蒙靈詧怒仙芝不先言已而遽發奏一不迎勞【一猶言一切也勞力到翻】罵仙芝曰噉狗糞高麗奴【噉徒濫翻又徒覧翻】汝官皆因誰得而不待我處分【處昌呂翻分扶問翻】擅奏捷書高麗奴汝罪當斬但以汝新有功不忍耳仙芝但謝罪邊令誠奏仙芝深入萬里立奇功今旦夕憂死<br />
<br />
  資治通鑑卷二百十五  <br>
   </div> 

<script src="/search/ajaxskft.js"> </script>
 <div class="clear"></div>
<br>
<br>
 <!-- a.d-->

 <!--
<div class="info_share">
</div> 
-->
 <!--info_share--></div>   <!-- end info_content-->
  </div> <!-- end l-->

<div class="r">   <!--r-->



<div class="sidebar"  style="margin-bottom:2px;">

 
<div class="sidebar_title">工具类大全</div>
<div class="sidebar_info">
<strong><a href="http://www.guoxuedashi.com/lsditu/" target="_blank">历史地图</a></strong>  
<a href="http://www.880114.com/" target="_blank">英语宝典</a>  
<a href="http://www.guoxuedashi.com/13jing/" target="_blank">十三经检索</a> 
<br><strong><a href="http://www.guoxuedashi.com/gjtsjc/" target="_blank">古今图书集成</a></strong> 
<a href="http://www.guoxuedashi.com/duilian/" target="_blank">对联大全</a> <strong><a href="http://www.guoxuedashi.com/xiangxingzi/" target="_blank">象形文字典</a></strong> 

<br><a href="http://www.guoxuedashi.com/zixing/yanbian/">字形演变</a>  <strong><a href="http://www.guoxuemi.com/hafo/" target="_blank">哈佛燕京中文善本特藏</a></strong>
<br><strong><a href="http://www.guoxuedashi.com/csfz/" target="_blank">丛书&方志检索器</a></strong> <a href="http://www.guoxuedashi.com/yqjyy/" target="_blank">一切经音义</a>  

<br><strong><a href="http://www.guoxuedashi.com/jiapu/" target="_blank">家谱族谱查询</a></strong>  <strong><a href="http://shufa.guoxuedashi.com/sfzitie/" target="_blank">书法字帖欣赏</a></strong> 
<br>

</div>
</div>


<div class="sidebar" style="margin-bottom:0px;">

<font style="font-size:22px;line-height:32px">QQ交流群9:489193090</font>


<div class="sidebar_title">手机APP 扫描或点击</div>
<div class="sidebar_info">
<table>
<tr>
	<td width=160><a href="http://m.guoxuedashi.com/app/" target="_blank"><img src="/img/gxds-sj.png" width="140"  border="0" alt="国学大师手机版"></a></td>
	<td>
<a href="http://www.guoxuedashi.com/download/" target="_blank">app软件下载专区</a><br>
<a href="http://www.guoxuedashi.com/download/gxds.php" target="_blank">《国学大师》下载</a><br>
<a href="http://www.guoxuedashi.com/download/kxzd.php" target="_blank">《汉字宝典》下载</a><br>
<a href="http://www.guoxuedashi.com/download/scqbd.php" target="_blank">《诗词曲宝典》下载</a><br>
<a href="http://www.guoxuedashi.com/SiKuQuanShu/skqs.php" target="_blank">《四库全书》下载</a><br>
</td>
</tr>
</table>

</div>
</div>


<div class="sidebar2">
<center>


</center>
</div>

<div class="sidebar"  style="margin-bottom:2px;">
<div class="sidebar_title">网站使用教程</div>
<div class="sidebar_info">
<a href="http://www.guoxuedashi.com/help/gjsearch.php" target="_blank">如何在国学大师网下载古籍?</a><br>
<a href="http://www.guoxuedashi.com/zidian/bujian/bjjc.php" target="_blank">如何使用部件查字法快速查字?</a><br>
<a href="http://www.guoxuedashi.com/search/sjc.php" target="_blank">如何在指定的书籍中全文检索?</a><br>
<a href="http://www.guoxuedashi.com/search/skjc.php" target="_blank">如何找到一句话在《四库全书》哪一页?</a><br>
</div>
</div>


<div class="sidebar">
<div class="sidebar_title">热门书籍</div>
<div class="sidebar_info">
<a href="/so.php?sokey=%E8%B5%84%E6%B2%BB%E9%80%9A%E9%89%B4&kt=1">资治通鉴</a> <a href="/24shi/"><strong>二十四史</strong></a>&nbsp; <a href="/a2694/">野史</a>&nbsp; <a href="/SiKuQuanShu/"><strong>四库全书</strong></a>&nbsp;<a href="http://www.guoxuedashi.com/SiKuQuanShu/fanti/">繁体</a>
<br><a href="/so.php?sokey=%E7%BA%A2%E6%A5%BC%E6%A2%A6&kt=1">红楼梦</a> <a href="/a/1858x/">三国演义</a> <a href="/a/1038k/">水浒传</a> <a href="/a/1046t/">西游记</a> <a href="/a/1914o/">封神演义</a>
<br>
<a href="http://www.guoxuedashi.com/so.php?sokeygx=%E4%B8%87%E6%9C%89%E6%96%87%E5%BA%93&submit=&kt=1">万有文库</a> <a href="/a/780t/">古文观止</a> <a href="/a/1024l/">文心雕龙</a> <a href="/a/1704n/">全唐诗</a> <a href="/a/1705h/">全宋词</a>
<br><a href="http://www.guoxuedashi.com/so.php?sokeygx=%E7%99%BE%E8%A1%B2%E6%9C%AC%E4%BA%8C%E5%8D%81%E5%9B%9B%E5%8F%B2&submit=&kt=1"><strong>百衲本二十四史</strong></a>  <a href="http://www.guoxuedashi.com/so.php?sokeygx=%E5%8F%A4%E4%BB%8A%E5%9B%BE%E4%B9%A6%E9%9B%86%E6%88%90&submit=&kt=1"><strong>古今图书集成</strong></a>
<br>

<a href="http://www.guoxuedashi.com/so.php?sokeygx=%E4%B8%9B%E4%B9%A6%E9%9B%86%E6%88%90&submit=&kt=1">丛书集成</a> 
<a href="http://www.guoxuedashi.com/so.php?sokeygx=%E5%9B%9B%E9%83%A8%E4%B8%9B%E5%88%8A&submit=&kt=1"><strong>四部丛刊</strong></a>  
<a href="http://www.guoxuedashi.com/so.php?sokeygx=%E8%AF%B4%E6%96%87%E8%A7%A3%E5%AD%97&submit=&kt=1">說文解字</a> <a href="http://www.guoxuedashi.com/so.php?sokeygx=%E5%85%A8%E4%B8%8A%E5%8F%A4&submit=&kt=1">三国六朝文</a>
<br><a href="http://www.guoxuedashi.com/so.php?sokeytm=%E6%97%A5%E6%9C%AC%E5%86%85%E9%98%81%E6%96%87%E5%BA%93&submit=&kt=1"><strong>日本内阁文库</strong></a> <a href="http://www.guoxuedashi.com/so.php?sokeytm=%E5%9B%BD%E5%9B%BE%E6%96%B9%E5%BF%97%E5%90%88%E9%9B%86&ka=100&submit=">国图方志合集</a> <a href="http://www.guoxuedashi.com/so.php?sokeytm=%E5%90%84%E5%9C%B0%E6%96%B9%E5%BF%97&submit=&kt=1"><strong>各地方志</strong></a>

</div>
</div>


<div class="sidebar2">
<center>

</center>
</div>
<div class="sidebar greenbar">
<div class="sidebar_title green">四库全书</div>
<div class="sidebar_info">

《四库全书》是中国古代最大的丛书,编撰于乾隆年间,由纪昀等360多位高官、学者编撰,3800多人抄写,费时十三年编成。丛书分经、史、子、集四部,故名四库。共有3500多种书,7.9万卷,3.6万册,约8亿字,基本上囊括了古代所有图书,故称“全书”。<a href="http://www.guoxuedashi.com/SiKuQuanShu/">详细>>
</a>

</div> 
</div>

</div>  <!--end r-->

</div>
<!-- 内容区END --> 

<!-- 页脚开始 -->
<div class="shh">

</div>

<div class="w1180" style="margin-top:8px;">
<center><script src="http://www.guoxuedashi.com/img/plus.php?id=3"></script></center>
</div>
<div class="w1180 foot">
<a href="/b/thanks.php">特别致谢</a> | <a href="javascript:window.external.AddFavorite(document.location.href,document.title);">收藏本站</a> | <a href="#">欢迎投稿</a> | <a href="http://www.guoxuedashi.com/forum/">意见建议</a> | <a href="http://www.guoxuemi.com/">国学迷</a> | <a href="http://www.shuowen.net/">说文网</a><script language="javascript" type="text/javascript" src="https://js.users.51.la/17753172.js"></script><br />
  Copyright &copy; 国学大师 古典图书集成 All Rights Reserved.<br>
  
  <span style="font-size:14px">免责声明:本站非营利性站点,以方便网友为主,仅供学习研究。<br>内容由热心网友提供和网上收集,不保留版权。若侵犯了您的权益,来信即刪。scp168@qq.com</span>
  <br />
ICP证:<a href="http://www.beian.miit.gov.cn/" target="_blank">鲁ICP备19060063号</a></div>
<!-- 页脚END --> 
<script src="http://www.guoxuedashi.com/img/plus.php?id=22"></script>
<script src="http://www.guoxuedashi.com/img/tongji.js"></script>

</body>
</html>
