\section{資治通鑑卷二百五十三}
宋 司馬光 撰

胡三省 音註

唐紀六十九|{
	起彊圉作噩盡上章困敦十月凡三年有奇}


僖宗惠聖恭定孝皇帝上之下

乾符四年春正月王郢誘魯寔入舟中執之|{
	王郢因魯寔請降事見上卷上年}
將士從寔者皆奔潰朝廷聞之以右龍武大將軍宋皓為江南諸道招討使先徵諸道兵外更忠武宣武感化三道|{
	陳許忠武軍汴宋宣武軍徐州感化軍}
宣泗二州兵新舊合萬五千餘人並受皓節度二月郢攻陷望海鎮掠明州又攻台州陷之刺史王葆退守唐興|{
	唐興即今天台縣在台州西一百一十里}
詔二浙福建各出舟師以討之 王仙芝陷鄂州黄巢陷鄆州殺節度使薛崇 南詔酋龍嗣立以來

為邊患殆二十年|{
	宣宗大中十三年酋龍立酋慈由翻}
中國為之虛耗|{
	為于偽翻}
而其國中亦疲弊酋龍卒諡曰景莊皇帝子法立改元貞明承智大同國號鶴拓亦號大封人 |{
	考異曰除雲䖍南詔録曰南詔别名鶴拓其後亦自稱大封人是以封為國號也}
法好畋獵酣飲|{
	好呼到翻}
委國事於大臣閏月嶺南西道節度使辛讜奏南詔遣陁西段瑳寶等來請和|{
	南詔官有陁西猶中國判官也瑳七何翻又七可翻}
且言諸道兵戍邕州歲久餽餉之費疲弊中國請許其和使羸瘵息肩|{
	羸倫為翻瘵側介翻}
詔許之讜遣大將杜弘等齎書幣送瑳寶還南詔但留荆南宣歙數軍戍邕州|{
	歙書涉翻}
自餘諸道兵什减其七 王郢横行浙西鎮海節度使裴璩嚴兵設備|{
	璩求於翻}
不與之戰密招其黨朱實降之|{
	降戶江翻}
散其徒六七千人輸器械二十餘萬舟航粟帛稱是|{
	稱尺證翻}
勑以實為金吾將軍於是郢黨離散郢收餘衆東至明州甬橋鎮遏使劉巨容以筒箭射殺之|{
	劉巨容以宿州甬橋鎮遏使將兵討王郢筒箭長纔尺餘内之竹筒注之弦上繫竹筒於手腕彀弓既豁筒向後激矢射敵皆洞貫詳見辯誤乾符二年王郢反至是而平射而亦翻}
餘黨皆平璩諝之從曾孫也|{
	裴諝見二百六卷代宗大歷十四年從才用翻}
三月黄巢陷沂州 夏四月壬申朔日有食之 賊帥柳彦璋剽掠江西|{
	帥所類翻剽匹妙翻}
陜州軍亂逐觀察使崔碣貶碣懷州司馬|{
	陜失冉翻碣其謁翻}
黄巢與尚讓合兵保查牙山 |{
	考異曰舊紀四年三月巢陷鄆州七月入查牙山與王仙芝合五年二月君長仙芝皆死尚讓以兄遇害大掠淮南舊傳五年八月王鐸斬王仙芝先是尚君長弟讓以兄奉使見誅師部衆入查牙山黄巢黄揆昆仲八人率盜數千依讓按實録乾符二年仙芝陷曹濮巢已起兵應之三年十二月招討副都監楊復光奏草賊尚讓據查牙山官軍退保鄧州四年四月黄巢引其衆保查牙山其年冬君長乃死驚聽録巢與仙芝俱入蘄州以仙芝獨受官而怒歐仙芝傷面由是分去時君長亦在座非仙芝死後巢方依讓也又按舊紀仙芝死後王鐸始為都統討賊而舊傳云王鐸斬仙芝又先云殺張璘乃陷廣州先云陷華州方攻潼關敘事顛錯不倫今從實録}
五月甲子以給事中楊損為陜虢觀察使損至官誅首亂者損嗣復之子也|{
	楊嗣復事文宗}
初桂管觀察使李瓚失政支使薛堅石屢規正之瓚不能從及瓚被逐|{
	李瓚被逐見上卷上年被皮義翻}
堅石攝留務移牒鄰道禁遏亂兵一方以安詔擢堅石為國子博士六月柳彦璋襲陷江州執刺史陶祥使祥上表彦璋亦自附降狀|{
	上時掌翻降戶江翻}
勑以彦璋為右監門將軍令散衆赴京師以左武衛將軍劉秉仁為江州刺史彦璋不從以戰艦百餘固湓江為水寨|{
	湓江在江州城外接于大江故謂之湓江湓蒲奔翻}
剽掠如故|{
	剽匹妙翻}
忠武都將李可封戍邊還至邠州迫脅主師索舊欠糧鹽|{
	帥所類翻索山客翻}
留止四日闔境震驚秋七月還至許州節度使崔安潜悉按誅之 庚申王仙芝黄巢攻宋州三道兵與戰不利|{
	三道兵平盧宣武忠武也}
賊遂圍宋威於宋州甲寅左威衛上將軍張自勉將忠武兵七千救宋州殺賊二千餘人賊解圍遁去王鐸盧攜欲使張自勉以所將兵受宋威節度鄭畋以為威與自勉已有疑忿若在麾下必為所殺不肯署奏八月辛未鐸攜訴於上求罷免庚辰畋請歸滻川養疾|{
	滻川在長安東滻音產}
上皆不許|{
	史言僖宗不能定國是}
王仙芝陷安州 鹽州軍亂逐刺史王承顔詔高品牛從珪往慰諭之貶承顔象州司戶承顔及崔碣素有政聲以嚴肅為驕卒所逐朝廷與貪暴致亂者同貶時人惜之|{
	史言唐末賞罰失當且言主昏政亂能吏不惟不得展其才亦不免於罪}
從珪自鹽州還軍中請以大將王宗誠為刺史詔宗誠詣闕將士皆釋罪仍加優結 乙卯王仙芝陷隨州執刺史崔休徵山南東道節度使李福遣其子將兵救隨州戰死福奏求援兵遣左武衛大將軍李昌言將鳳翔五百騎赴之仙芝遂轉掠復郢忠武大將張貫等四千人與宣武兵援襄州自申蔡間道逃歸|{
	間古莧翻}
詔忠武節度使崔安潜宣武節度使穆仁裕遣人約還|{
	約還者戒約將士使還赴援也}
冬十月邠寧節度使李侃奏遣兵討王宗誠斬之餘黨悉平|{
	逐王承顔之黨也}
鄭畋與王鐸盧爭論用兵於上前畋不勝退復上奏|{
	復扶又翻下同}
以為自王仙芝俶擾|{
	按孔安國尚書注俶始也擾亂也俶尺六翻}
崔安潜首請會兵討之繼士卒罄竭資糧|{
	言竭本道所有以供征行士卒資糧}
賊往來千里塗炭諸州獨不敢犯其境又以本道兵授張自勉解宋州圍使江淮漕運流通不輸寇手今蒙盡以自勉所將七千兵令張貫將之|{
	將即亮翻下同}
隸宋威|{
	句斷}
自勉獨歸許州威復奏加誣毀因功受辱臣竊痛之安潜出師前後克捷非一一旦彊兵盡付他人良將空還若勍敵忽至|{
	勍渠京翻}
何以枝梧臣請以忠武四千人授威餘三千人使自勉將之守衛其境既不侵宋威之功又免使安潜愧恥時盧不以為然上不能決畋復上言宋威欺罔朝廷敗衂狼籍|{
	衂女六翻籍秦昔翻}
又聞王仙芝七狀請降威不為聞奏|{
	為于偽翻}
朝野切齒以為宜正軍法迹狀如此不應復典兵權願與内大臣參酌|{
	内大臣謂兩中尉兩樞密也}
早行罷黜不從河中軍亂逐節度使劉侔縱兵焚掠以京兆尹竇璟

為河中宣慰制置使|{
	璟俱永翻}
黄巢寇掠蘄黄|{
	蘄黄相去一百六十五里}
曾元裕擊破之斬首四千級巢遁去 十一月己酉以竇璟為河中節度使 招討副都監楊復光遣人說諭王仙芝仙芝遣尚君長等請降於復光|{
	監古銜翻說輸芮翻降戶江翻楊復光時屯鄧州}
宋威遣兵於道中刼取君長等十二月威奏與君長等戰於頴州西南生擒以獻復光奏君長等實降非威所擒詔侍御史歸仁紹等鞫之|{
	姓譜曰左傳胡子國姓歸為楚所滅子孫或以國為氏或以姓為氏}
竟不能明斬君長等於狗脊嶺黄巢陷匡城遂陷濮州|{
	匡城縣屬滑州本漢長垣縣宋白曰隋開皇於婦姑城置匡城縣以縣南有故匡城為名即孔子所畏之所濮博木翻}
詔潁州刺史張自勉將諸道兵擊之 江州刺史劉秉仁乘驛之官單舟入柳彦璋水寨賊出不意即迎拜秉仁斬彦璋散其衆|{
	柳彦璋為盜九月而敗}
王仙芝寇荆南節度使楊知温知至之兄也|{
	楊知至見上卷懿宗咸通十一年}
以文學進不知兵或告賊至知温以為妄不設備時漢水淺狹賊自賈塹度|{
	九域志郢州長壽縣有賈塹鎮塹七艶翻}


五年春正月丁酉朔大雪知温方受賀|{
	凡元旦冬至諸州鎮皆受將吏牙賀下至縣邑亦然}
賊已至城下遂陷羅城將佐共治子城而守之|{
	治直之翻}
及暮知温猶不出將佐請知温出撫士卒知温紗帽皁裘而行將佐請知温擐甲以備流矢|{
	皁不早翻擐音宦}
知温見士卒拒戰猶賦詩示幕僚遣使告急於山南東道節度使李福福悉其衆自將救之時有沙陀五百在襄陽福與之俱至荆門遇賊|{
	晉分編縣置長林縣德宗貞元二十一年又分長林置荆門縣屬江陵府九域志在府北一百六十許里}
沙陀縱騎奮擊破之仙芝聞之焚掠江陵而去江陵城下舊三十萬戶至是死者什三四 壬寅招討副使曾元裕大破王仙芝於申州東所殺萬人招降散遣者亦萬人敕以宋威久病罷招討使還青州|{
	宋威本平盧帥罷招討使還鎮}
以曾元裕為招討使潁州刺史張自勉為副使 庚戍以西川節度使高駢為荆南節度使兼鹽鐵轉運使 振武節度使李國昌之子克用為沙陀副兵馬使戍蔚州|{
	宋白曰蔚州秦趙間亦為代郡之地後魏置懷荒禦夷二鎮於此東魏於此置北靈丘郡後周大象二年置蔚州唐開元初移郡治於靈丘西南一百三十里西至朔州三百八十里}
時河南盜賊蠭起|{
	謂王仙芝黄巢等也}
雲州沙陀兵馬使李盡忠與牙將康君立薛志勤程懷信李存璋等謀曰今天下大亂朝廷號令不復行於四方|{
	復扶又翻}
此乃英雄立功名富貴之秋也吾屬雖各擁兵衆然李振武功大官高名聞天下|{
	言李國昌討平龎勛於當時功為大帥振武於諸將官為高聞音問}
其子勇冠諸軍|{
	冠古玩翻}
若輔以舉事代北不足平也衆以為然君立興唐人|{
	隋分靈丘縣置安邊縣中廢唐開元十二年復置治横野軍至德三載更名興唐縣屬蔚州}
存璋雲州人志勤奉誠人也|{
	貞觀二十二年以内屬奚可度者部落置饒樂都督府開元二十三年更名奉誠都督府薛志勤其府人也}
會大同防禦使段文楚兼水陸運使|{
	宋白曰朔州馬邑縣貞觀已來已為大同軍城開元五年分鄯陽縣之東三十里置大同軍以戍邊後於軍内置馬邑徵漢舊名也建中間馬燧自鄯陽移朔州於馬邑縣}
代北荐飢|{
	荐才甸翻再也再歲五穀不熟曰荐飢}
漕運不繼文楚頗减軍士衣米又用法稍峻軍士怨怒盡忠遣君立濳詣蔚州說克用起兵除文楚而代之|{
	說式芮翻}
克用曰吾父在振武俟我禀之君立曰今機事已泄緩則生變何暇千里禀命乎|{
	言道里遼遠使命往返則事必泄而害成}
於是盡忠夜帥牙兵攻牙城|{
	攻雲州牙城也帥讀曰率下同}
執文楚及判官柳漢璋繫獄自知軍州事遣召克用克用帥其衆趣雲州|{
	趣七喻翻}
行收兵二月庚午至城下衆且萬人屯於鬭雞臺下壬申盡忠遣使送符印請克用為防禦留後癸酉盡忠械文楚等五人送鬭雞臺下克用令軍士冎而食之以騎踐其骸|{
	冎古瓦翻踐慈演翻}
甲戍克用入府舍視事 |{
	考異曰趙鳳後唐太祖紀年録曰乾符三年河南水災盜寇蜂起朝廷以段文楚為代北水陸運雲州防禦使以代支謨時歲荐飢文楚削軍人衣米諸軍咸怨太祖為雲中防邊督將部下爭訴以軍食不充請具聞奏邊校程懷信康君立等十餘帳日譁於太祖之門請共除虐帥以謝邊人衆因大譟擁太祖上馬比及雲中衆且萬人城中械文楚出以應太祖後唐閔帝時史官張昭遠撰莊宗功臣列傳曰康君立為雲中牙校事防禦使段文楚時天下將亂代北仍歲阻飢諸部豪傑咸有嘯聚邀功之志文楚法令稍峻軍食轉餉不給戍兵咨怨雲州沙陀兵馬使李盡忠私謂君立等曰段公儒者難與共事方今四方雲擾皇威不振丈夫不能於此時立功立事非人豪也吾等雖擁部衆然以雄勁聞於時者莫若李振武父子官高功大勇冠諸軍吾等合勢推之則代北之地旬月可定功名富貴事無不濟也時武皇為沙陀三部落副兵馬使在蔚州盡忠令君立私往圖之曰方今天下大亂天子付將臣以邊事歲偶飢荒便削儲給我等邊人焉能守死公家父子素以威惠及五部當共除虐帥以謝邊人武皇曰予家尊在振武萬一相逼俟予稟命君立曰事機已泄遲則變生咸通十三年十二月盡忠夜帥牙兵攻牙城執文楚及判官柳漢璋陳韜等繫之於獄遂自知軍州事遣君立召太祖於蔚州是月太祖與退渾突厥三部落衆萬人趨雲中十四年正月六日至鬭雞臺盡忠遣監軍判官符印請太祖知留後事七日盡忠械文楚漢璋等五人送門雞臺軍人亂食其肉九日太祖權知留後府牙受上三軍表請授太祖太同防禦使懿宗不悦時已除盧簡方代文楚未至而文楚被害實録乾符元年十二月李克用殺大同軍防禦使段文楚自稱防禦留後塞下之亂自兹始矣薛居正五代史君立傳皆與莊宗列傳同惟削去李盡忠名但云君立與薛鐵山程懷信王行審李存璋等謀悉以盡忠語為君立之語云君立等乃夜謁武皇言曰方今天下大亂云云衆因聚譟擁武皇比及雲州衆且萬人師營鬭雞臺城中械文楚以應武皇之軍既收城推武皇為大同防禦留後衆狀以聞舊紀咸通十三年十二月李國昌小男李克用殺雲州防禦使段文楚㨿雲州自稱防禦留後乾符五年正月沙陀首領李盡忠陷遮虜軍竇澣遣康傳圭率土團二千屯代州將求賞呼譟殺馬步軍使鄧䖍有唐末三朝見聞録者不著撰人姓名專記晉陽事其書云乾符五年戊戍竇澣自前守京兆尹拜河東節度使在任便值大同軍變殺防禦使段文楚正月二十六日軍於石窯二十七日到白泊二十九日至静邊軍三十日築却四面城門二月一日在城將士三人共賞絹一匹監軍使差仇判官聞奏李盡忠等准詔各賞馬一匹銀鞍轡一副銀三鋌銀椀一枚絹一束錦二匹紫羅三匹諸軍將銀椀絹等三日李盡忠却入四日兩面馬步五萬餘人城四面下營五日又賞土團牛酒六日監軍使送牌印與李九郎七日城南門樓上繫縛下段尚書柳漢璋雍侍御陳韜等四人尋分付軍兵於鬬雞臺西剮却又令馬軍踐踏却骸骨八日李九郎被上團馬步軍約一千人持弓刀送上與舊紀五年事微合實録亦頗采之云五年正月壬戍竇澣奏沙陀首領李盡忠寇石窯白泊至静邊軍二月奏李盡忠求賞詔賞馬一匹銀鞍勒綿絹等按莊宗列傳舊紀克用殺文楚在咸通十三年十二月歐陽修五代史記取之太祖紀年録在乾符三年薛居正五代史新沙陀傳取之見聞録在乾符五年二月新記取之惟實録在乾符元年不知其所据何書也克用既殺文楚豈肯晏然安處必更侵擾邊陲朝廷亦須兵征討而自乾符四年以前皆不見其事唐末見聞録叙日月今從之}
令將士表求勑命朝廷不許李國昌上言乞朝廷速除大同防禦使若克用違命臣請帥本道兵討之終不愛一子以負國家朝廷方欲使國昌諭克用會得其奏乃以司農卿支詳為大同軍宣慰使詔國昌語克用令迎候如常儀除克用官必令稱愜|{
	李克用始此語牛倨翻稱尺證翻愜詰叶翻}
又以太僕卿盧簡方為大同防禦使 |{
	考異曰舊紀咸通十三年七月以前義昌節度使盧簡方為太僕卿十二月以振武節度使李國昌為雲州刺史大同軍防禦等使國昌稱病辭軍務乃以太僕卿盧簡方為雲州刺史充大同軍防禦等使上召簡方於思政殿謂之曰卿以滄州節制屈居大同然朕以沙陀退渾撓亂邊鄙以卿曾在雲中惠及部落且忍屈為朕此行具逹朕旨安慰國昌勿令有所猜嫌也十四年正月辛未以雲朔暴亂代北騷動賜盧簡方詔曰近知大同軍不安殺害段文楚李國昌小男克用主領兵權又曰若克用暫勿主兵務束手待朝廷除人則事出權宜不足猜慮若便圖軍柄欲奄大同則患繫久長故難依允料國昌輸忠效節必當已有指揮簡方準詔諭之國昌不奉詔乃詔太原節度使崔彦昭幽州節度使張公素出師討之三月以簡方為振武節度使至嵐州卒實録乾符元年十二月簡方除大同二年正月賜詔亦不云使彦昭公素討之蓋舊記實録各随段文楚死之後載除簡方及詔書使事相接續耳恐皆未足据也舊紀所云太原幽州討之蓋因叙後來事實録所以不取者方知招諭未必攻討也唐末見聞録又云五年四月勑除簡方振武節度使五月卒實録亦在五年而云六月卒蓋約奏到之月耳今從三朝聞見録}
貶楊知温為郴州司馬|{
	以王仙芝犯江陵城幾失守士民多為所殺畧也}
曾元裕奏大破王仙芝於黄梅|{
	黄梅縣屬蘄州宋白曰宋分江夏郡置南新蔡郡隋開皇十八年改為黄梅縣界内有黄梅山因名}
殺五萬餘人追斬仙芝傳首餘黨散去黄巢方攻亳州未下尚讓帥仙芝餘衆歸之|{
	帥讀曰率 考異曰實録元裕奏大破仙芝於黄梅縣殺戮五萬餘人追至曹州南華縣斬仙芝傳首京師舊紀二月王仙芝餘黨攻江西招討使宋威出軍屢敗之仍宣詔書諭仙芝仙芝致書於威求節钺威偽許之仙芝令其大將尚君長蔡温玉奉表入朝威乃斬君長温玉以狗仙芝怒急攻洪州陷其郛宋威赴援與賊戰大敗之殺仙芝傳首京師君長弟讓與黄巢大掠淮南舊傳曰齊克讓為兖州節度使以本軍討仙芝仙芝懼引衆歷陳許襄鄧無少長皆虜之衆號三十萬三年七月陷江陵又遣將徐唐莒陷洪州時仙芝表請符節不允以宋威為荆南節度招討使楊復光為監軍復光遣判官吳彦宏諭以朝旨釋罪别加官爵仙芝乃令尚君長蔡温玉楚彦威相次詣闕請罪且求恩命時宋威害復光之功並擒送闕勑於狗脊嶺斬之賊怒悉精銳擊官軍威軍大敗復光收其餘衆以統之朝廷以王鐸代為招討五年八月收復荆州斬仙芝首獻于闕下新傳黄巢自蘄州與仙芝分其衆尚君長入陳蔡巢北掠齊魯衆萬人入鄆州殺節度使薛崇進陷沂州由潁蔡保查岈山引兵復與仙芝合圍宋州會自勉救兵至仙芝解而南渡漢攻荆南陷之賊不能守巢攻和州未克仙芝自圍洪州取之使徐唐莒守進破朗岳遂圍潭州觀察使崔瑾拒却之乃向浙西擾宣潤不能得所欲身留江西趣别部還入河南帝詔崔安潜歸忠武復起宋威曾元裕以招討使還之而楊復光監軍復光以詔諭賊仙芝遣尚君長等詣闕請罪又遺威書求節度威陽許之上言與君長戰擒之復光固言其降命侍御史與中人即訊不能明卒斬之仙芝怒還攻洪州入其郛威自將往救敗仙芝於黄梅斬五萬級獲仙芝傳首京帥當此時巢方圍亳州未下君長弟讓帥仙芝潰黨歸巢新舊傳叙賊所經歷皆不同又云宋威殺仙芝今皆從實録}
推巢為王號衝天大將軍改元王覇 |{
	考異曰續寶運録乾符元年黄巢聚衆於會稽反建元曰王覇元年舊傳先是尚君長弟讓以兄見誅率衆入查牙山黄巢黄揆昆仲八人率盜數千依讓月餘衆至數萬陷汝州虜刺史王鐐大掠關東官軍加討屢為所敗其衆十餘萬尚讓乃與羣盜推巢為主曰衝天大將軍仍署官屬藩鎮不能制新傳曰尚君長弟讓率仙芝潰黨歸巢推巢為王號衝天大將軍署拜官屬驅河南山南之民十餘萬掠淮南建元王覇今從之}
署官屬巢襲陷沂州濮州既而屢為官軍所敗乃遺天平節度使張裼書|{
	敗補邁翻遺于季翻裼先擊翻又徒計翻}
請奏之詔以巢為右衛將軍令就鄆州解甲巢竟不至|{
	考異曰舊傳及王仙芝敗巢東攻亳州不下乃襲破沂州據之仙芝餘黨悉附焉實録巢自稱黄王建元王覇連為王師所敗詣天平乞降除右衛將軍復叛去自是兵不能制新傳曰曾元裕敗賊於申州死者萬人帝以宋威殺尚君長非是且討賊無功詔還青州以元裕為招討使張自勉為副巢破考城取濮州元裕軍荆襄援兵阻更拜自勉東北面行營招討使督諸軍急捕巢巢方掠襄邑雍丘詔滑州節度使李嶧壁原武巢寇葉陽翟欲窺東都會左神武大將軍劉景仁以兵五千援東都河陽節度使鄭延休兵三千壁河隂巢兵在江西者為鎮海節度使高駢所破寇鄭郟襄城陽翟者為崔安潜逐走在浙西者為節度使裴璩斬二長死者甚衆巢大沮畏乃詣天平軍乞降詔授巢右衛將軍巢度藩鎮不一未足制已即叛去轉寇浙東執觀察使崔璆與實録先後不同今從實録}
加山南東道節度使李福同平章事賞救荆南之功也 三月羣盜陷朗州岳州|{
	朗岳相去五百五十里}
曾元裕屯荆襄|{
	荆襄相去三百四十里}
黄巢自滑州畧宋汴|{
	滑州南至汴州二百一十里汴州東至宋州三百五十里}
乃以副使張自勉充東南面行營招討使黄巢攻衛南|{
	隋置楚丘縣於古楚丘城以曹有楚丘改曰衛南唐時屬滑州}
遂攻葉陽翟詔河陽兵千人赴東都與宣武昭義兵二千人共衛宫闕|{
	衛東都宫闕也}
以左神武大將軍劉景仁充東都應援防遏使并將三鎮兵|{
	三鎮河陽宣武昭義}
仍聽於東都募兵二千人景仁昌之孫也|{
	劉昌見德宗紀}
又詔曾元裕將兵徑還東都義成兵三千守轘轅伊闕河隂武牢|{
	河南氏縣北有轘轅故關伊闕縣北有伊闕故關孟州汜水縣有乕牢關唐避先諱以乕為武}
王仙芝餘黨王重隐陷洪州江西觀察使高湘奔湖口|{
	江州東北六十里有湖口鎮當彭蠡湖入江之口宋朝置湖口縣}
賊轉掠湖南别將曹師雄掠宣潤詔曾元裕楊復光引兵救宣潤 湖南軍亂都將高傑逐觀察使崔瑾瑾郾之子也|{
	崔郾見二百四十四卷文宗太和五年郾音偃}
黄巢引兵渡江攻陷䖍吉饒信等州 朝廷以李充

用據雲中夏四月以前大同軍防禦使盧簡方為振武節度使以振武節度使李國昌為大同節度使以為克用必無以拒也|{
	是時朝議謂使李國昌以父臨子子必無以拒豈知李國昌父子欲兼據兩鎮考異曰唐末聞見録遮虜軍及代州告急竇尚書差回鶻五百騎邊界廵檢至四月三日進發至五里北}


|{
	副將康叔譚恃酒叛逆射損都將趙歸義斫損將判官閻建弘擒縛入府尚書令下於衙南門全家處斬使司差副兵馬使趙元掠領馬軍進閻建弘逓送海西當月内有勑送節到除前大同軍防禦使盧簡方充振武節度使除振武節度使李尚書充大同軍節度使實録云戊辰以簡方為振武國昌為大同蓋誤以康叔譚作亂之日為簡方等建節之日也新沙陀傳曰李克用既殺段文楚諸校共丐克用為大同防禦留後不許諸道兵進捕諸道不甚力而黄巢方引兵度江朝廷度未能制乃赦之以國昌為大同軍防禦使國昌不受命詔河東節度使崔彦昭幽州張公素共擊之無功据此則是大同防禦使非節度使也薛居正五代史紀曰武皇殺段文楚諸將列狀以聞請授武皇旄鉞朝廷不允徵諸道兵以討之乾符五年黄巢度江其勢滋蔓天子乃悟其事以武皇為大同軍節度使檢校工部尚書是克用為大同軍節度使非國昌實録國昌傳及獻祖紀年録舊唐本紀俱不言國昌為大同節度使獨實録於此言之下五月又云國昌殺監軍不肯代必有所据蓋國昌父子俱不肯受代朝廷以為用國昌代克用必無違命故徙國昌為大同軍節度使而以盧簡方鎮振武二人竟不受命故簡方不得赴鎮而死於嵐州國昌亦未嘗赴大同也}
詔以東都軍儲不足貸商旅富人錢穀以供數月之費仍賜空名殿中侍御史告身五通監察御史告身十通有能出家財助國稍多者賜之時連歲旱蝗寇盜充斥耕桑半廢租賦不足内藏虛竭無所佽助|{
	藏徂浪翻佽音次亦助也}
兵部侍郎判度支楊嚴三表自陳才短不能濟辦辭極哀切詔不許|{
	人見美官誰不欲之乃有辭而不獲者可以觀世道矣}
曹師雄寇湖州|{
	曹師雄自宣潤進寇湖州}
鎮海節度使裴璩遣兵擊破之王重隱死其將徐唐莒據洪州|{
	曹師雄王重隐皆王仙芝之黨}
饒州將彭幼璋合義營兵克復饒州|{
	饒州比為黄巢所陷義營兵饒州之起義者也}
南詔遣其酋望趙宗政來請和親|{
	南詔官有酋望在大將之下久贊之上亦清平官也酋慈由翻}
無表但令督爽牒中書請為弟而不稱臣詔百僚議之禮部侍郎崔澹等以為南詔驕僭無禮高駢不識大體反因一僧呫囁卑辭誘致其使|{
	呫他恊翻囁而涉翻又之涉翻僧謂景仙也景仙使南詔見上卷上年}
若從其請恐垂笑後代 |{
	考異曰實録置澹議於二月至四月又云南詔遣酋望趙宗政來朝且議和好今因盧鄭爭蠻事置此}
尚駢聞之上表與澹爭辯詔諭解之澹璵之子也|{
	璵珙之弟也珙相武宗}
五月丙申朔鄭畋盧議蠻事欲與之和親畋固爭以為不可怒拂衣起袂罥硯墮地破之|{
	罥古泫翻繫取也掛也釋名曰硯研也研墨使和濡也}
上聞之曰大臣相詬|{
	詬古候翻又許候翻}
何以儀刑四海丁酉畋皆罷為太子賓客分司 |{
	考異曰舊紀六年五月賊圍廣州與李苕崔璆書求天平節鉞畋爭論於中書辭語不遜俱罷分司畋傳曰五年黄巢東渡江淮衆百萬所經屢陷郡邑六年陷安南府據之致書與浙東觀察使崔璆求鄆州節钺璆言賊勢難圖宜因授之以絶北顧之患天子下百僚議初黄巢之起也宰相盧以浙西觀察使高駢素有軍功奏為淮南節度使令扼賊衝尋以駢為諸道行營都統及崔璆之奏朝臣議之有請假節以紓患者畋採衆議欲以南海節制縻之以始用高駢欲其立功以圖勝曰高駢將畧無雙淮土甲兵甚銳今諸道之師方集蕞爾纎寇不足平殄何事捨之示怯而令諸軍解體邪畋曰巢賊之亂本因飢歲人以利合乃至寔繁江淮以南荐食殆半國家久不用兵皆忘戰所在節將閉門自守尚不能枝不如釋咎包容權降恩澤彼本以飢年利合一遇豐歲孰不懷思鄉土其衆一離則巢賊几上肉耳若此際不以計攻全恃兵力恐天下之憂未艾也羣議然之而左僕射于琮曰南海有市舶之利歲貢珠璣如令為妖賊所有國藏漸當廢竭上亦望駢成功乃依議及中書商量制勑畋曰妖賊百萬横行天下高公遷延玩寇無意翦除又從而保之彼得計矣國祚安危在我輩三四人畫度公倚淮南用兵吾不知稅駕之所矣怒拂衣而起染袂於硯因投之僖宗聞之怒曰大臣相詬何以表儀四海二人俱罷政事傳曰五年黄巢陷荆南江西外郛及饒吉䖍信等州自浙東陷福建遂至嶺南陷廣州殺節度使李岧遂抗表求節鉞初王仙芝起河南舉宋威齊克讓曾衮等有將畧用為招討使及宋威殺尚君長致賊充斥朝廷遂以宰臣王鐸為都統深不悦浙帥崔璆等上表請假黄巢廣州節旄上令宰臣議以王鐸為統帥欲激怒黄巢堅言不可假賊節制止授率府率而已與同列鄭畋爭論投硯於地由是兩罷之實録五年五月丙申朔是日宰臣鄭畋盧議南蠻事請降公主通和畋固爭以為不可抗論是非怒拂衣而起袂染於硯因投碎之丁酉以畋並為太子賓客分司注云舊史洎雜說皆云畋議黄巢節制忿爭賜罷而鄭延昌撰畋行狀乃云議蠻事無可證之然當時所述恐不謬又畋傳曰時黄巢攻陷江浙上表乞節鉞畋與同列盧謀議攻討及拔用將帥事多異同又南詔蠻請降公主和好畋固爭以為不可遂抗論之乃與俱罷相又傳曰人質甚陋語亦不正與鄭畋俱李翺之外孫及同輔政議論不協初王仙芝起河南舉宋威齊克讓曾衮等有將畧用為招討使討賊皆無功致賊充斥又主高駢之請欲以公主和南詔蠻鄭畋執之以為不可帝前忿爭由是兩罷之舊紀六年五月賊圍廣州仍與廣南節度使李苕浙東觀察使崔璆書求保薦乞天平節璆岧上表論之宰相鄭畋盧爭論於中書詞語不遜俱罷為太子賓客分司東都按新舊傳舊紀皆以畋罷相在六年實録新紀表在此年五月實録新書皆自相矛楯然宋氏多書知二人罷在五月必有所据今從之}
以翰林學士承旨戶部侍郎豆盧瑑為兵部侍郎吏部侍郎崔沆為戶部侍郎並同平章事時宰相有好施者常使人以布囊貯錢自隨|{
	施式豉翻貯丁呂翻}
行施匄者每出襤褸盈路|{
	襤力三翻衣無緣也褸力主翻衣醜敝也}
有朝士以書規之曰今百姓疲弊寇盜充斥相公宜舉賢任能紀綱庶務捐不急之費杜私謁之門使萬物各得其所則家給人足自無貧者何必如此行小惠乎宰相大怒 邕州大將杜弘送段瑳寶至南詔踰年而還|{
	杜弘去年閏二月送段瑳寶還從宣翻}
甲辰辛讜復遣攝廵官賈宏大將左瑜曹朗使於南詔|{
	復扶又翻}
李國昌欲父子并據兩鎮得大同制書毁之殺監軍不受代與李克用合兵陷遮虜軍|{
	遮虜軍在洪谷東北亦曰遮虜平}
進擊寧武及岢嵐軍|{
	媯州懷戎縣西有寧武軍非此此當在遮虜平南嵐州嵐谷縣有岢嵐軍按宋白續通典雲州東取寧武媯州路至幽州七百里朔州西至岢嵐軍二百二十里此李國昌合雲朔之兵東西攻掠既陷遮虜東擊寧武西擊岢嵐也此即媯州之西寧武軍岢嵐軍在嵐州東北百里岢枯我翻嵐盧含翻}
盧簡方赴振武至嵐州而薨丁巳河東節度使竇澣民塹晉陽己未以都押牙康傳圭為代州刺史又土團千人赴代州土團至城北娖隊不|{
	娖側角翻言娖整其隊而不行也}
求優賞時府庫空竭澣遣馬步都虞候鄧䖍往慰諭之土團冎䖍|{
	冎古瓦翻}
牀舁其尸入府|{
	舁羊茹翻}
澣與監軍自出慰諭人給錢三百布一端衆乃定押牙田公鍔給亂軍錢布衆遂刼之以為都將赴代州澣借商人錢五萬緍以助軍 |{
	考異曰唐末聞見録五月振武損却别勑不受除替李尚書收却遮虜軍進打寧武及岢嵐軍代州告急二十二日指揮在府三城排門差夫一人齊掘四面壕塹盧尚書赴振武至嵐州身薨二十四日拜都押牙康傳圭充代州刺史又太原晉陽兩縣點到土團子弟一千人往代州屯駐至城北卓隊不索出軍優賞差馬步都虞候鄧䖍安慰尋被冎却牀舁尸柩入府尚書監軍自出安慰定每人各給錢三百文布一端差押牙田公鍔給散不放却囘便被請將充都將赴軍前使司有榜借商人助軍錢五萬貫文實録五月李國昌殺監軍使不肯受代起兵進打寧武及岢嵐軍代州出兵禦之始國昌遣克用以兵襲大同三軍表克用為留後朝廷不允乃以國昌命之欲以其子無能拒也時國昌貪其土地欲父子分統故拒命焉實録六月乙丑朔嵐州奏新除振武節度使盧簡方卒以太原府都押衙康傳圭為代州刺史太原晉陽土團千人戍代州至城北卓隊不索優賞馬步都虞候鄧䖍安慰為其衆殺之節度使竇澣自出撫慰乃定初太原府帑空竭每有賞賚必科民家至是尤窘迫乃牓借商人助軍錢五萬此皆約唐末見聞録為之而後其月日以象奏到之時唐末見聞録又云六月十一日左散騎常侍支謨奉勅到府充大同軍制置使兼攝河東節度副使軍前同指揮使此謂到府之日而實録云甲戍以謨為制置使甲戍乃六月十日亦誤也}
朝廷以澣為不才六月以前昭義節度使曹翔為河東節度使 王仙芝餘黨剽掠浙西|{
	剽匹妙翻下同}
朝廷以荆南節度使高駢先在天平有威名|{
	高駢之威名以破蠻於交趾而徙鎮天平鄆人遂畏之耳}
仙芝黨多鄆人乃徙駢為鎮海節度使 沙陀焚唐林崞縣入忻州境|{
	武后證聖元年分五臺崞置武延縣唐隆元年更名唐林崞漢古縣也時並屬代州宋白曰唐林本漢廣武縣地九域志崞在州西南五十里崞音郭}
秋七月曹翔至晉陽己亥捕土團殺鄧䖍者十三人殺之義武兵至晉陽不解甲讙譟求優賞|{
	讙與諠同}
翔斬其十將一人乃定義成忠武昭義河陽兵會于晉陽以禦沙陀八月戊寅曹翔引兵救忻州沙陀攻岢嵐軍陷其羅城敗官軍于洪谷|{
	洪谷在岢嵐軍南敗補邁翻}
晉陽閉門城守 黄巢寇宣州宣歙觀察使王凝拒之敗於南陵|{
	九域志南陵縣在宣州西一百五里}
巢攻宣州不克乃引兵攻浙東開山路七百里攻剽福建諸州|{
	按九域志自婺州至衢州界首一百九十里衢州治所至建州七百五里此路豈黄巢始開之邪剽匹妙翻}
九月平盧軍奏節度使宋威薨|{
	老病而死固其宜也史書威死以為握兵玩寇不能報國之戒}
辛丑以諸道行營招討使曾元裕領平盧節度使 壬寅曹翔暴薨丙午昭義兵大掠晉陽坊市民自共擊之殺千餘人乃潰 中書侍郎同平章事李蔚罷為東都留守|{
	蔚紆勿翻守手又翻}
以吏部尚書鄭從讜為中書侍郎同平章事從讜餘慶之孫也|{
	鄭餘慶始見二百三十五卷德宗貞元十四年}
以戶部尚書判戶部事李都同平章事兼河中節度使 冬十月詔昭義節度使李鈞幽州節度使李可舉與吐谷渾酋長赫連鐸白義誠沙陀酋長安慶薩葛酋長米海萬合兵討李國昌父子於蔚州|{
	參考新舊書安慶薩葛皆部落之名更以後廣明元年安慶都督史敬存證之可見酋慈由翻長知丈翻薩桑葛翻}
十一月岢嵐軍翻城應沙陀丁未以河東宣慰使崔季康為河東節度代北行營招討使沙陀攻石州庚戍崔季康救之 十二月甲戌黄巢陷福州觀察使韋岫弃城走 南詔使者趙宗政還其國|{
	是年四月趙宗政來講和}
中書不答督爽牒但作西川節度使崔安濳書意使安濳答之 崔季康及昭義節度使李鈞與李克用戰於洪谷兩鎮兵敗鈞戰死昭義兵還至代州士卒剽掠|{
	剽匹妙翻}
代州民殺之殆盡餘衆自鵶鳴谷走歸上黨|{
	鵶鳴谷在忻州秀容縣東北 考異曰舊紀河東節度使崔季康與北面行營招討使李鈞與沙陀李克用戰于岢嵐軍之洪谷王師大敗鈞中流矢而卒戊戍至代州昭義軍亂為代州百姓所殺殆盡此年實録畧同廣明元年八月實録河東奏昭義節度使李鈞為猛虎軍所殺又曰詔統本道兵由雁門出討雲州與賊戰敗歸為其下殺之新紀庚辰崔季康李鈞及李克用戰於洪谷敗續薛居正五代史紀曰乾符六年春朝廷以昭義節度使李鈞充北面招討使將上黨太原之師過石嶺關屯于代州與幽州李可舉會赫連鐸同攻蔚州獻祖以一軍禦之武皇以一軍南抵遮虜城以拒李鈞是冬大雪弓弩弦絶南軍苦寒臨陣大敗奔歸代州李鈞中流矢而卒唐末見聞録曰十九日崔尚書往岢嵐軍請别勑賈敬嗣大夫權兵馬留後觀察判官李劭權觀察留後昭義節度使李鈞領本道兵馬到代州軍變被代州殺戮並盡捉到李鈞殘軍潰散取鵶鳴谷各歸本道按眧義軍變必非李鈞所為代州百姓捉到李鈞不知如何處之今從舊紀}
王郢之亂|{
	事始上卷二年終本卷四年}
臨安人董昌以土團討賊有功補石鏡鎮將|{
	垂拱四年分餘杭於濳地以故臨水城置臨安縣屬杭州有石鏡山石鏡鎮九域志臨安縣在州西一百二十里臨安志石鏡山在臨安縣南一里錢鏐改為衣錦山}
是歲曹師雄寇二浙杭州募諸縣鄉兵各千人以討之昌與錢塘劉孟安阮結富陽聞人宇鹽官徐及新城杜稜餘杭凌文舉臨平曹信各為之都將號杭州八都|{
	錢唐餘杭皆漢縣富陽漢富春縣晉避諱改為富陽新城吳縣鹽官漢海鹽縣地有鹽官吳遂名縣臨平鎮在錢唐北隋之餘杭縣置杭州後移州治錢唐後又移於柳浦西今州城是九域志富陽在州西南七十三里鹽官在州東一百二十九里新城在州西南一百三十里餘杭在州西北七十二里將即亮翻下同}
昌為之長|{
	長知兩翻}
其後宇卒錢塘人成及代之|{
	卒子恤翻}
臨安人錢鏐以驍勇事昌以功為石鏡都知兵馬使|{
	錢鏐事始此鏐力求翻}


六年春正月魏王佾薨|{
	佾懿宗子}
鎮海節度使高駢遣其將張璘梁纘 |{
	考異曰舊紀張璘作張麟新紀傳實録作潾今從舊高駢黄巢傳及唐年補録妖亂志唐補紀續寶運録舊紀梁纘作梁績今從衆書}
分道擊黄巢屢破之降其將秦彦畢師鐸李罕之許勍等數十人|{
	降戶江翻勍渠京翻爲後秦彦畢師鐸反攻高駢張本 考異曰郭延誨妖亂志曰初黄巢將蹂踐淮甸委師鐸為先鋒攻脇天長累日不克師鐸之志沮焉及巢北向師鐸遂降北海按舊師鐸傳駢敗巢於浙西皆師鐸之效故置於此}
巢遂趣廣南|{
	趣七喻翻}
彦徐州人師鐸寃句人罕之項城人也 賈宏等未至南詔相繼卒於道中|{
	去年五月辛讜使賈宏使南詔卒子恤翻}
從者死亦大半|{
	從才用翻}
時辛讜已病風痺|{
	痺必至翻脚冷濕病也}
召攝巡官徐雲䖍執其手曰讜已奏朝廷使入南詔而使者相繼物故奈何|{
	使疏吏翻下同}
吾子既仕則思狥國能為此行乎讜恨風痺不能拜耳因嗚咽流涕雲䖍曰士為知己死|{
	古語云士為知己者死女為悦己者容為于偽翻}
明公見辟恨無以報德敢不承命讜喜厚具資裝而遣之|{
	史言辛讜垂死不忘國事}
二月丙寅雲䖍至善闡城驃信見大使抗禮受副使已下拜己巳驃信使慈雙羽楊宗就館謂雲䖍曰貴府牒欲使驃信稱臣奉表貢方物驃信已遣人自西川入唐與唐約為兄弟不則舅甥|{
	不讀曰否}
夫兄弟舅甥書幣而已何表貢之有雲䖍曰驃信既欲為弟為甥驃信景莊之子景莊豈無兄弟|{
	酋龍諡景莊皇帝}
於驃信為諸父驃信為君則諸父皆稱臣况弟與甥乎且驃信之先由大唐之命得合六詔為一|{
	事見二百一十四卷玄宗開元二十六年}
恩德深厚中間小忿罪在邊鄙|{
	謂南詔與西川爭恨細故以致興戎}
今驃信欲修舊好|{
	好呼到翻}
豈可違祖宗之故事乎順祖考孝也事大國義也息戰爭仁也審名分禮也|{
	分扶問翻}
四者皆令德也可不勉乎驃信待雲䖍甚厚雲䖍留善闡十七日而還驃信以木夾二授雲䖍其一上中書門下|{
	上時掌翻}
其一牒嶺南西道然猶未肯奉表稱貢 辛未河東軍至静樂|{
	静樂漢汾陽縣地齊周之際改曰岢嵐隋開皇十八年改曰汾源大業四年改曰静樂唐屬嵐州九域志在州東北五十里}
士卒作亂殺孔目官石裕等壬申崔季康逃歸晉陽甲戍都頭張鍇郭昢帥行營兵攻東陽門|{
	鍇器駭翻昢滂佩翻又普罪翻又普沒翻帥讀曰率}
入府殺季康辛巳以陜虢觀察使高潯為昭義節度使|{
	陜失冉翻}
以邠寧節度使李侃為河東節度使 |{
	考異曰唐末見聞録二十日宣慰使到府李侃除河東節度使實録因云庚寅除侃誤也}
三月天平軍節度使張裼薨牙將崔君裕自知州事淄州刺史曹全晸討誅之|{
	晸知領翻}
夏四月庚申朔日有食之 西川節度使崔安濳到官不詰盜蜀人怪之|{
	詰去吉翻}
安濳曰盜非所由通容則不能為|{
	所由謂捕盜官吏}
今窮覈則應坐者衆|{
	覈下格翻}
搜捕則徒為煩擾甲子出庫錢千五百緍分置三市|{
	成都城中鬻花果蠶器於一所號蠶市鬻香藥於一所號藥市鬻器用者號七寶市}
置牓其上曰有能告捕一盜賞錢五百緍盜不能獨為必有侶侶者告捕釋其罪賞同平人未幾|{
	幾居豈翻}
有捕盜而至者盜不服曰汝與我同為盜十七年贓皆平分汝安能捕我我與汝同死耳安濳曰汝既知吾有牓何不捕彼以來則彼應死汝受賞矣汝既為所先死復何辭|{
	先悉薦翻復扶又翻}
立命給捕者錢使盜視之然後冎盜於市并㓕其家於是諸盜與其侶互相疑無地容足夜不及旦散逃出境境内遂無一人之盜安潜以蜀兵怯弱奏遣大將齎牒詣陳許募壯士與蜀人相雜訓練用之得三千人分為三軍亦戴黄帽號黄頭軍|{
	襲忠武黄頭軍之名也}
又奏乞洪州弩手教蜀人用弩走丸而射之選得千人號神機弩營蜀兵由是浸彊|{
	余嘗謂兵之強弱在將不在兵以秦之兵強天下而漢高祖以蜀兵定三秦自唐以來蜀兵號為懦怯然韋臯用以制吐蕃而有餘未嘗借工於他道也至李德裕始募工於他道以治器械崔安潜蓋倣李德裕之故智耳諸葛孔明治蜀作木牛連弩之法自晉以下倣而為之宋自女真侵噬吳玠兄弟畫境而守蜀東南以西路兵為天下最夫豈借工於别路哉射而亦翻}
涼王侹薨|{
	侹懿宗子音他頂翻}
上以羣盜為憂王鐸曰臣為宰相之長|{
	長知兩翻}
在朝不足分陛下之憂請自督諸將討之乃以鐸守司徒兼侍中充荆南節度使南面行營招討都統 |{
	考異曰舊紀五年二月鐸自請督衆討賊天子以宋威失策殺尚君長乃以鐸檢校司徒兼侍中門下侍郎江陵尹荆南節度使充諸道兵馬都統舊傳四年賊陷江陵楊知温失守宋威破賊失策朝議統帥盧攜稱高駢累立戰功宜付軍柄物議未允鐸廷奏臣願自率諸軍盪滌羣盜朝議然之五年以鐸守司徒門下侍郎同平章事兼江陵尹荆南節度使充諸道行營兵馬都統今從實録及新紀表}
五月辛卯勅賜河東軍士銀牙將賀公雅所部士卒作亂焚掠三城|{
	新書地理志唐自高祖起兵於晉陽自天授已來以為北都而晉陽宫仍隋不廢宫在都之西北宫城周二千五百二十步崇四丈八尺都城左汾右晉濳丘在中長四千三百二十一步廣三千一百二十步周萬五千一百五十三步其崇四丈汾東曰東城貞觀十一年長史李勣築兩城之間有中城武后時築以合東城宫南有大明城故宫城也}
執孔目官王敬送馬步司節度使李侃與監軍自出慰諭為之斬敬於牙門乃定 泰寧節度使李係晟之曾孫也有口才而實無勇畧王鐸以其家世良將奏為行營副都統兼湖南觀察使|{
	官人以世而不考其才古今之通患也為鐸係失守殄民張本}
使將精兵五萬并土團屯潭州以塞嶺北之路拒黄巢|{
	塞悉則翻}
河東都虞候每夜密捕賀公雅部卒族㓕之丁巳餘黨近百人稱報寃將|{
	近其靳翻}
大掠三城焚馬步都虞候張鍇府城都虞候郭昢家節度使李侃下令以軍府不安曲順軍情收鍇昢斬於牙門并逐其家以賀公雅為馬步都虞候鍇昢臨刑泣言於衆曰所殺皆捕盜司密申今日寃死獨無烈士相救乎於是軍士復大譟簒取鍇昢歸都虞候司|{
	復扶又翻下巢復同}
尋下令復其舊職并召還其家收捕盜司元義宗等三十餘家誅㓕之己未以馬步都教練使朱玫等為三城斬斫使|{
	玫莫杯翻}
將兵分捕報寃將悉斬之軍城始定 黄巢與浙東觀察使崔璆嶺南東道節度使李迢書求天平節度使|{
	璆渠幽翻}
二人為之奏聞|{
	為于偽翻}
朝廷不許巢復上表求廣州節度使 |{
	考異曰續寶運録曰黄巢先求廣府兼使相朝廷不與黄巢夏初兵屯廣南屢候勑旨不下遂恣行攻劫黄巢夏六月上表稱義軍百萬都統兼韶廣等州觀察處置等使末云六月十五日表秋遣内侍仇公度并廣南邕府安南安東等道節度使指揮觀察使開國公食邑五百戶官告六通又賜節度將吏空名尚書僕射官告五十通九月二十日仇公度到廣州至十月一日巢與公度雜匹段藥物等五馱表函并所賜官告並却付公度表末云廣明元年十月一日上表公度等其年十月二十九日至京如寶運録所言則是廣明元年十月一日巢猶在廣州也按其月巢已入長安今從舊紀}
上命大臣議之左僕射于琮以為廣州市舶寶貨所聚|{
	舶薄陌翻大舟也唐置市舶司於廣州以招來海中蕃舶}
豈可令賊得之亦不許乃議别除官六月宰相請除巢府率從之 |{
	考異曰賊圍廣州仍與廣南節度使李迢浙東觀察使崔璆書求保薦乞天平節鉞迢璆上表論之實録迢璆上表論請詞甚懇激乃詔公卿集議巢又自表乞廣州節度安南都護巢自春夏其衆大疫死者什三四欲㨿有嶺表永為巢穴乃繼有是請右僕射于琮議云云時朝廷倚高駢成功不允其奏乃議除官或云以正員將軍縻之宰相亦沮其議乃除率府率舊巢傳曰時高駢鎮淮南表請招討賊許之議加都統巢乃渡淮偽降于駢駢遣將張潾帥兵受降于天長鎮巢禽潾殺之因虜其衆尋南陷湖湘遂據交廣託崔璆奏乞天平節度朝議不允又乞除官時宰臣鄭畋與樞密使楊復恭欲請授同正員將軍盧攜駁其議請授率府率如其不受請以高駢討之新巢傳曰有詔高駢為諸道行營都統巢進寇廣州詒李迢書求表為天平節度使又脅崔璆言于朝宰相鄭畋欲許之盧攜田令孜執不可巢又乞安南都護廣州節度使書聞右僕射于琮議云云乃拜巢率府率舊盧傳亦皆以為議授巢率府率按此時已罷相今從實録}
河東節度使李侃以軍府數有亂|{
	數所角翻}
稱疾請尋醫勑以代州刺史康傳圭為河東行軍司馬徵侃詣京師秋八月甲子侃晉陽尋以東都留守李蔚同平章事充河東節度使|{
	蔚音欝}
鎮海節度使高駢奏請以權舒州刺史郎幼復充留後守浙西遣都知兵馬使張璘將兵五千於郴州守險兵馬留後王重任將兵八千於循潮二州邀遮臣將萬人自大庾嶺趣廣州擊黄巢|{
	趣七喻翻}
巢聞臣往必當遁逃乞勑王鐸以所部兵三萬於梧桂昭永四州守險詔不許九月黄巢得率府率告身大怒詬執政急攻廣州即

日陷之執節度使李迢轉掠嶺南州縣巢使迢草表述其所懷迢曰予代受國恩親戚滿朝腕可斷表不可草|{
	朝直遥翻腕烏貫翻}
巢殺之 |{
	考異曰驚聽録曰擁李迢在寇復併爇海隅又陷桂州次攻湖南屯衡州方知王仙芝已山東沒陣又尚君長生送咸京遂召李迢怒而躓害新紀十一月辛酉黄巢陷江陵殺李迢新傳曰其十月巢據荆南脅李迢草表報天子迢不可巢怒殺之北夢瑣言曰黄巢入廣州執李佋隨軍至荆州令佋草表述其所懷佋曰某骨肉滿朝世受國恩腕即可斷表終不為領於江津害之今從實録}
冬十月以鎮海節度使高駢為淮南節度使充鹽鐵轉運使以涇原節度使周寶為鎮海節度使|{
	為駢寶鬭䦧張本}
以山南東道行軍司馬劉巨容為節度使寶平州人也 黄巢在嶺南士卒罹瘴疫死者什三四其徒勸之北還以圖大事巢從之自桂州編大栰數十乘暴水沿湘江而下歷衡永州癸未抵潭州城下李係嬰城不敢出戰巢急攻一日陷之係奔朗州|{
	九域志自潭州至朗州三百八十餘里}
巢盡殺戍兵流尸蔽江而下尚讓乘勝進逼江陵衆號五十萬時諸道兵未集江陵兵不滿萬人王鐸留其將劉漢宏守江陵自帥衆趣襄陽|{
	九域志自江陵至襄陽四百四十里帥讀曰率趣七喻翻}
云欲會劉巨容之師鐸既去漢宏大掠江陵 |{
	考異曰舊紀廣明元年二月巢陷潭州王鐸弃江陵奔襄陽漢宏大掠寶録閏月湖南奏黄巢賊衆自衡永州下十月二十七日攻陷潭州新巢傳曰廣明初賊自嶺南寇湖南諸郡攻潭州陷之舊巢傳巢欲據南海之地坐邀朝命是歲自春及夏其衆大疫死者十三四衆勸請北歸以圖大利巢不得已廣明元年北踰五嶺犯湖湘江浙按舊紀傳皆云廣明元年敗王鐸今月日從實録事從舊紀又据舊紀傳則劉漢宏本王鐸將鐸去而漢宏留江陵大掠遂為盜也實録用之而於鐸奔襄陽下添先是字若鐸在江陵漢宏時為羣盜安能入其城大掠借使漢宏先曾寇掠江陵與黄巢事了不相干何必言後半月餘賊衆乃據其城也吳越備史云漢宏本兖州小吏領本州兵禦巢寇遂殺將首劫輜重而叛後命前濠州刺史崔鍇招降之据此則漢宏本羣盜也新傳用之而云鐸招降之或者漢宏本羣盜中間降鐸為部將鐸去江陵漢宏復大掠為盜其後又降於崔鍇遂為唐臣也}
焚蕩殆盡士民逃竄山谷會大雪僵尸滿野|{
	僵居良翻}
後旬餘賊乃至漢宏兖州人也帥其衆北歸為羣盜|{
	帥讀曰率}
閏月丁亥朔河東節度使李蔚有疾以供軍副使李邵權觀察留後監軍李奉臯權兵馬留後己丑蔚薨都虞候張鍇郭昢署狀絀邵|{
	狀奏狀鍇昢因署狀黜李邵而進丁球絀丑律翻}
以少尹丁球知觀察留後 十一月戊午以定州已來制置使萬年王處存為義武節度使河東行軍司馬雁門關已來制置使康傳圭為河東節度使|{
	四朝志宣宗大中五年以白敏中充招討党項行營都統制置等使制置使之名始此宋朝初不常置掌經畫邊鄙軍旅之事政和中熙秦用兵以内侍童貫為之迄南渡之後江淮荆蜀皆置制置使其任重矣}
黄巢北趣襄陽|{
	趣七喻翻}
劉巨容與江西招討使淄州刺史曹全晸合兵屯荆門以拒之|{
	九域志襄陽南至荆門二百七十餘里}
賊至巨容伏兵林中全晸以輕騎逆戰陽不勝而走賊追至伏大破賊衆乘勝逐北比至江陵|{
	比必利翻及也九域志荆門南至江陵一百六十五里}
俘斬其什七八巢與尚讓收餘衆渡江東走或勸巨容窮追賊可盡也巨容曰國家喜負人|{
	喜許記翻}
有急則撫存將士不愛官賞事寧則弃之或更得罪|{
	唐末之政誠如劉巨容之言}
不若留賊以為富貴之資衆乃止全晸度江追賊會朝廷以泰寧都將段彦謨代為招討使全晸亦止由是賊勢復振|{
	復扶又翻}
攻鄂州陷其外郭轉掠饒信池宣歙杭十五州衆至二十萬 康傳圭自代州赴晉陽庚辰至烏城驛張鍇郭昢出迎亂刀斫殺之至府又族其家 十二月以王鐸為太子賓客分司|{
	以江陵之敗也}
初兵部尚書盧嘗薦高駢可為都統至是駢將張璘等屢破黄巢乃復以為門下侍郎平章事凡關東節度使王鐸鄭畋所除者多易置之|{
	為盧倚高駢以誤國張本}
是歲桂陽賊陳彦謙陷柳州殺刺史董岳

廣明元年春正月乙卯朔改元 沙陀入雁門關寇忻代二月庚戍沙陀二萬餘人逼晉陽辛亥陷太谷|{
	宋白曰太谷縣本漢陽邑縣隋開皇十八年改名太谷因縣西太谷為名}
遣汝州防禦使博昌諸葛爽帥東都防禦兵救河東|{
	博昌漢古縣名後魏置樂安郡及樂安縣隋時改樂安縣為博昌縣唐時屬青州帥讀曰率}
河東節度使康傳圭專事威刑多復仇怨強取富人財遣前遮虜軍使蘇弘軫擊沙陀於太谷至秦城遇沙陀戰不利而還傳圭怒斬弘軫時沙陀已還代北傳圭遣都教練使張彦球將兵三千追之壬戍至百井|{
	百井鎮在太原陽曲縣}
軍變還趣晉陽|{
	趣七喻翻}
傳圭閉城拒之亂兵自西明門入殺傳圭監軍周從寓自出慰諭乃定以彦球為府城都虞候朝廷聞之遣使宣慰曰所殺節度使事出一時各宜自安勿復憂懼|{
	復扶又翻}
左拾遺侯昌業以盜賊滿關東而上不親政事專務遊戲賞賜無度田令孜專權無上天文變異社稷將危上疏極諫上大怒召昌業至内侍省賜死 |{
	考異曰續寶運録云司天少監侯昌業上疏其畧曰陛下不納李蔚杜希教之諫又曰臣乃明祈五道暗祝冥官悚息於班列之中願早過於閻浮之世又曰受爵不逢於有德之君立戟每佐於無道之主又曰不望堯舜之年得同先帝之日又曰明取尹希復指揮暗策王士成進狀強奪波斯之寶貝抑取茶店之珠珍渾取匱坊全城般運又曰莫是唐家合盡之歲為復是陛下壽足之年又曰伏惟陛下蹔停戲賞救接蒼生於殿内立掲諦道場以無私財帛供養諸佛用資世禄共力攘災表奏聖上龍威震怒侍臣驚悸宣徽使宣云侯昌業付内侍省侯進止翌日午時又内養劉季遠宣口敕云侯昌業出自寒門擢居清近不能修慎妄奏閒詞訕謗萬乘君王毁斥百辟卿士在我彛典是不能容其侯昌業宜賜自盡北夢瑣言曰唐自廣明後閹人擅權置南北廢置使軍容田令孜有囘天之力中外側目而王仙芝黄巢剽掠江淮朝廷憂之左拾遺侯昌業上疏極言時病留中不出命于仗内戮之後有傳侯昌業疏詞不合事體其末云請開掲諦道場以銷兵厲似為庸僧偽作也必若侯昌業以此識見犯上宜其死也今從之}
上好騎射劒槊法筭|{
	唐國子監有筭學博士掌教九章海島孫子五曹張丘陽周髀五經綴術緝古為專業皆法筭也好呼倒翻下同}
至於音律蒲博無不精妙好蹴鞠鬭雞與諸王賭鵝鵝一頭至五十緡 |{
	考異曰新田令孜傳帝冲騃喜鬬鵝一鵝至直五十萬錢按鵝非可鬭之物至直五十萬錢亦恐失實新傳誤也今從續寶運録}
尤善擊毬嘗謂優人石野猪曰朕若應擊毬進士舉須為狀元對曰若遇堯舜作禮部侍郎恐陛下不免駮放|{
	駮糾駮也放黜也駮放者糾駮其非是而放黜之也駮北角翻}
上笑而已 度支以用度不足奏借富戶及胡商貨財敕借其半鹽鐵轉運使高駢上言天下盜賊蜂起皆出於饑寒獨富戶胡商未耳乃止 高駢奏改楊子院為運使|{
	楊子院舊置留後今改為運使宋朝江淮運使本此}
三月庚午以左金吾大將軍陳敬瑄為西川節度使敬瑄許州人田令孜之兄也|{
	田令孜本姓陳咸通中隨義父入内侍省為宦者遂冒田姓}
初崔安潜鎮許昌|{
	許昌許州也忠武節度治所}
令孜為敬瑄求兵馬使安潜不許敬瑄因令孜得隸左神策軍數歲累遷至大將軍令孜見關東羣盜日熾隂為幸蜀之計奏以敬瑄及其腹心左神策大將軍楊師立牛朂羅元杲鎮三川上令四人擊毬賭三川敬瑄得第一籌|{
	凡擊毬立毬門於毬場設賞格天子按轡入毬場諸將迎拜天子入講武榭升御座諸將羅拜于下各立馬于毬場之兩偏以俟命神策軍吏讀賞格訖都教練使放毬於場中諸將皆駷馬趨之以先得毬而擊過毬門者為勝先勝者得第一籌其餘諸將再入場擊毬其勝者得第二籌焉}
即以為西川節度使代安潜 辛未以門下侍郎同平章事鄭從讜同平章事充河東節度使康傳圭既死河東兵益驕故以宰相鎮之使自擇參佐從讜奏以長安令王調為節度副使前兵部員外郎史館修撰劉崇龜為節度判官前司勲員外郎史館修撰趙崇為觀察判官前進士劉崇魯為推官|{
	進士及第而於時無官謂之前進士}
時人謂之小朝廷言名士之多也崇龜崇魯政會之七世孫也|{
	劉政會唐初功臣}
時承晉陽新亂之後日有殺掠從讜貌温而氣勁多謀而善斷|{
	斷丁亂翻}
將士欲為惡者從讜輒先覺誅之姧軌惕息|{
	惕他歷翻}
為善者撫待無疑知張彦球有方畧百井之變非其本心獨推首亂者殺之召彦球慰諭悉以兵柄委之軍中由是遂安彦球為從讜盡死力|{
	為于偽翻}
卒獲其用|{
	卒子恤翻}
淮南節度使高駢遣其將張璘等擊黄巢屢捷盧奏以駢為諸道行營都統 |{
	考異曰續寶運録載駢上表及荅詔云今以卿為諸道行營都統應行營將士兵馬悉受指揮詔旨未到之間朝廷猜貳續敕却不許行軍只令固守封疆不得擅行征討於是高駢乃引淮水繞江都城三里坐甲不討黄巢自此轉盛舊紀傳王鐸出鎮荆南亦為諸道行營都統而實録及新紀表皆云為南面行營都統舊紀乾符四年六月以駢為鎮海節度使江西招討使六年十月以駢為淮南節度使江南行營招討使廣明元年三月朝廷以鐸統衆無功乃授駢諸道行營兵馬都統駢傳四年為鎮海節度使尋授諸道兵馬都統六年冬徙淮南節度使兵馬都統如故盧傳曰及王鐸失守罷都統以高駢代之實録五年六月駢移鎮海六年正月以駢為諸道行營兵馬都統仍賜詔如寶運録所載者八月駢上表亦如之十月駢徙淮南依前充都統按駢表請追郎幼復備守浙西則是在鎮海時也詔云周旋六鎮則是駢已移淮南後也六鎮謂安南天平西川荆南鎮海淮南也又詔云今以卿為諸道都統則似移淮南後方為都統也疑駢在浙西方為招討使既數破巢軍乃以滅巢為己任上表請布置諸軍自攻巢于廣州及王鐸敗盧遂以駢代之欲重其權故為諸道都統若駢先為諸道都統鐸但為南面都統則鐸已在駢統下可以指揮不須云乞降敕指揮鐸也且鐸自宰相都統諸將討賊故立都統之名不應同時有兩都統也其在浙西領江西招討使者時黄巢方掠䖍吉饒信故也今從舊紀及盧傳}
駢乃傳檄徵天下兵且廣召募得土客之兵共七萬|{
	土兵謂淮南之兵也客兵謂諸道之兵也}
威望大振朝廷深倚之|{
	為朝廷為駢所誤張本}
安南軍亂節度使曾衮出城避之諸道兵戍邕管者往往自歸夏四月丁酉以太僕卿李琢為蔚朔等州招討都統行營節度使琢聽之子也|{
	聽李晟之子 考異曰琢作瑑者誤也}
張璘度江擊賊帥王重覇降之|{
	帥所類翻重直龍翻}
屢破黄巢軍巢退保饒州别將常宏以其衆數萬降|{
	降戶江翻}
璘攻饒州克之巢走時江淮諸軍屢奏破賊率皆不實宰相已下表賀朝廷差以自安|{
	賈誼有言厝火積薪之下火未及然因謂之安唐則薪已然矣尚可以自安邪}
以李琢為蔚朔節度使仍充都統 以楊師立為東川節度使牛朂為山南西道節度使|{
	田令孜之志也}
以諸葛爽為北面行營副招討 初劉巨容既還襄陽|{
	還襄陽見上年}
荆南監軍楊復光以忠武都將宋浩權知府事泰寧都將段彦謩以兵守其城詔以浩為荆南安撫使彦謩耻居其下浩禁軍士翦伐街中槐柳彦謩部卒犯令浩杖其背彦謩怒挾刃馳入并其二子殺之復光奏浩殘酷為衆所誅詔以彦謩為朗州刺史以工部侍郎鄭紹業為荆南節度使 五月丁巳以汝州防禦使諸葛爽為振武節度使 劉漢宏之黨浸盛侵掠宋兖甲子徵東方諸道兵討之|{
	東方諸道宜武忠武義成天平泰寧平盧感化也}
黄巢屯信州遇疾疫卒徒多死張璘急擊之巢以金啗璘|{
	啗徒濫翻}
且致書請降於高駢求保奏駢欲誘致之許為之求節鉞|{
	降戶江翻誘音酉為于偽翻}
時昭義感化義武等軍皆至淮南駢恐分其功乃奏賊不日當平不煩諸道兵請悉遣歸朝廷許之賊詗知諸道兵已北度淮|{
	詗翾正翻又火迥翻}
乃告絶於駢且請戰駢怒令璘擊之兵敗璘死巢埶復振|{
	復扶又翻考異曰舊紀是歲春末賊在信州疫癘其徒多喪淮南將張璘急擊之賊懼以金啗璘仍致書高駢乞保明歸國駢信之許求節鉞時昭義武寧義武等軍兵馬數萬赴淮南駢欲收功於己乃奏賊已將殄滅不假諸道之師並遣還淮北賊知諸軍已退以求節鉞不獲暴怒與駢絶請戰駢怒令張璘整軍擊之為賊所敗臨陣殺璘賊遂乘勝度江攻天長六合等縣駢不能拒但自固而已朝廷聞賊復振大恐高駢傳曰廣明元年夏黄巢自嶺表北趨江淮由採石渡江璘勒兵天長欲擊之黄巢傳曰巢乃渡淮偽降於駢駢遣將張璘率兵受降于天長鎮巢擒璘殺之實録五月璘已為巢所殺七月巢乃過江其言璘所以死與舊紀同新紀傳皆與實録同据舊傳則璘死在江北也舊紀及實録新紀傳璘死在江南也按璘已死巢又陷睦州婺州宣州然後度江璘死在江南是也}
乙亥以樞密使西門思恭為鳳翔監軍丙子以宣徽

使李順融為樞密使皆降白麻於閤門出案與將相同|{
	唐制凡拜將相先一日中書納案遲明降麻於閤門出案會要凡將相翰林學士草制謂之白麻韋執誼翰林故事曰故事中書省用黄白二麻為綸命重輕之辯近者所出獨得黄麻其白麻皆在翰林院自非國之重事拜授將相德音赦宥則不得出于斯史言唐末宦官恣横監軍與樞密使恩數埒於將相程大昌曰凡欲降白麻若商量於中書門下省皆前一日進文書然後付翰林草麻制注已前見}
西川節度使陳敬瑄素微賤報至蜀蜀人皆驚莫知為誰有青城妖人乘其聲埶帥其黨詐稱陳僕射|{
	青城縣漢江源縣地南齊置齊基縣後周改為青城以縣西北三十二里有青城山也唐屬蜀州九域志縣在州北五十里帥讀曰率}
馬步使瞿大夫覺其妄|{
	馬步使掌馬步軍蓋唐末節度牙前職也}
執之沃以狗血即引服悉誅之六月庚寅敬瑄至成都 |{
	考異曰錦里耆舊傳云敬瑄九月二十五日上任按實録敬瑄除西川在三月庚午又雲南事狀敬瑄與市爕以下牒云某謬膺朝寄獲授藩條以六月八日到鎮上訖今從之}
黄巢别將陷睦州婺州|{
	睦婺相去一百八十里}
盧病風不能行謁告|{
	謁告謂請假居私第養疾也}
己亥始入對

敕勿拜遣二黄門掖之攜内挾田令孜外倚高駢上寵遇甚厚由是專制朝政高下在心|{
	朝直遥翻}
既病精神不完事之可否決於親吏楊温李修貨賂公行豆盧瑑無他材專附會崔沆時有啓陳常為所沮|{
	沮在呂翻}
庚子李琢奏沙陀二千來降琢時將兵萬人屯代州|{
	降戶江翻將即亮翻下同}
與盧龍節度使李可舉吐谷渾都督赫連鐸共討沙陀李克用遣大將高文集守朔州自將其衆拒可舉於雄武軍鐸遣人說文集歸國|{
	說式芮翻}
文集執克用將傅文達與沙陀酋長李友金薩葛都督米海萬安慶都督史敬存皆降於琢開門迎官軍|{
	酋慈由翻長知兩翻薩桑葛翻 考異曰實録六月云國昌遣文逹守蔚州七月云李琢赫連鐸奏破沙陀於蔚州降傅文達等薛居正五代史記武皇令軍使傅文逹起兵於蔚州高文集縛送李瑑按國昌時在蔚州何必令文達守之今從薛史}
友金克用之族父也 庚戍黄巢攻宣州陷之 劉漢宏南掠申光趙宗政之還南詔也西川節度使崔安濳表以崔澹

之說為是|{
	崔澹議見上五年}
且曰南詔小蠻本雲南一郡之地|{
	劉蜀分建寧永昌置雲南郡}
今遣使與和彼謂中國為怯復求尚主|{
	復扶又翻}
何以拒之上命宰相議之盧豆盧瑑上言大中之末府庫充實自咸通以來蠻兩陷安南邕管一入黔中四犯西川|{
	咸通元年蠻陷安南二年陷邕州四年又陷安南進逼邕管明年又圍邕州十四年寇黔中咸通二年寇嶲州四年寇西川六年陷嶲州十五年寇西川明年逼成都乾符元年寇西川事並見前紀}
徵兵運糧天下疲弊踰十五年租賦大半不入京師三使内庫由茲空竭|{
	度支戶部鹽鐵謂之三使}
戰士死於瘴癘百姓困為盜賊致中原榛皆蠻故也前歲冬蠻不為寇由趙宗政未歸去歲冬蠻不為寇由徐雲䖍復命蠻尚有覬望今安南子城為叛卒所據節度使攻之未下|{
	節度使謂曾衮}
自餘戍卒多已自歸|{
	事見上三月}
邕管客軍又減其半冬期且至儻蠻寇侵軼何以枝梧不若且遣使臣報復縱未得其稱臣奉貢且不使之懷怨益深堅決犯邊則可矣乃作詔賜陳敬瑄許其和親不稱臣 |{
	考異曰實録六月丙申陳敬瑄奏請遣使和蠻丁酉中書奏請令百官集議甲辰百官議定壬子中書奏遣使按敬瑄此月八日上丙申乃十四日也奏報豈能遽至今不取新傳先是南詔知蜀強故襲安南陷之會西川節度使陳敬瑄申和親議時盧復輔政與豆盧瑑皆厚高駢乃議通和今從雲南事狀雲南事狀又曰中書奏玄宗冊蒙歸義為雲南王其子閻羅鳳和於吐蕃其孫異牟尋却歸朝廷自請改雲南王賜號南詔德宗從之至曾孫蒙豐祐杜悰奏以入朝人多減之後索質子漸為侮慢卷末載陳敬瑄與雲南書牒或稱鶴拓或稱大封人雲南事狀不著撰人姓名似盧奏草也}
令敬瑄録詔白并移書與之仍增賜金帛以嗣曹王龜年為宗正少卿充使以徐雲䖍為副使别遣内使共齎詣南詔|{
	内使即中使}
秋七月黄巢自采石度江圍天長六合|{
	采石戍在宣州當塗縣西北渡江即和州界天寶元年分江都六合高郵三縣地置千秋縣天寶七載改為天長六合漢堂邑縣地東晉屬秦郡北齊改秦州後周改方州隋為六合縣唐並屬揚州宋白曰六合縣春秋時楚之棠邑秦滅楚以棠邑為縣九域志天長在揚州西一百一十里六合在真州西北七十里}
兵勢甚盛淮南將畢師鐸言於高駢曰朝廷倚公為安危今賊數十萬衆乘勝長驅|{
	謂乘殺張璘之勝勢也}
若涉無人之境不據險要之地以擊之使踰長淮不可復制|{
	復扶又翻下同}
必為中原大患駢以諸道兵已散張璘復死自度力不能制|{
	度徒洛翻}
畏怯不敢出兵但命諸將嚴備自保而已且上表告急稱賊六十餘萬屯天長去臣城無五十里先是盧謂駢有文武長才|{
	先悉薦翻}
若悉委以兵柄黄巢不足平朝野雖有謂駢不足恃者然猶庶幾望之|{
	幾居希翻}
及駢表至上下失望人情大駭詔書責駢散遣諸道兵致賊乘無備度江駢上表言臣奏聞遣歸亦非自專今臣竭力保衛一方必能濟辦但恐賊迤邐過淮|{
	迤移尔翻邐力紙翻}
宜急勅東道將士善為禦備|{
	東道謂關東諸道}
遂稱風痺不復出戰|{
	痺必至翻 考異曰舊駢傳駢怨朝議有不附己者欲賊縱横河洛令朝廷聳振則從而誅之大將畢師鐸說駢云云駢駭然曰君言是也即令出軍有愛將呂用之以左道媚駢駢頗用其言用之懼師鐸等立功即奪已權從容謂駢曰相公勲業高矣妖賊未殄朝廷已有間言賊若盪平則威望震主功居不賞公安稅駕邪為公良畫莫若觀釁自求多福駢深然之乃止諸將但握兵保境而已驚聽録朝廷議駢以文以武國之名將今此黄巢必無喪於淮海也尋淮南表至云今大寇忽至入臣封廵未肯綿伏狼必能晦沉大衆但以山東兵士屯駐揚州各思故鄉臣遂放去亦具聞奏非臣自專今奉詔書責臣無備不合放囘武勇又告城危致勞徵兵勞於往返臣今以寡擊衆然白武經與賊交鋒已當數陣粗成勝捷不落姧謀固護一方臣必能了但慮寇設深計支梧官軍迤邐過淮彼岸無敵即東道將士以至蕃臣繫朝廷速下明詔上委中書門下速與商量表至中書咸有異議遂京國士庶浮謗日興云淮南與巢衷私通連自固城池放賊過淮也妖亂志曰廣明元年七月黄巢自采石北度直抵天長時城内土客諸軍尚十餘萬皆良將勁兵議者有狂寇奔犯關防之患悉願盡力死戰用之慮其立功之後侵奪已權謂勃海曰黄巢起於羣盜遂至横行所在雄藩望風瓦解天時人事斷然可知令公既統強兵又居重地只得坐觀成敗不可更與爭鋒若稍損威名則大事去矣勃海深以為然竟不議出軍巢遂至北焉初巢寇廣陵也江東諸侯以勃海屯數道勁卒居將相重任巢江海一逋逃耳固可掉折箠而擒之及聞安然度淮由是方鎮莫不解體按駢宿將豈不知賊過淮之後不可復制若怨朝議不附己者則尤欲破賊立功以間執讒慝之口若縱賊過淮乃適足實議者之言非所以消謗也借使駢實有意使賊震驚朝廷從而誅之則賊入汝洛之後當晨夜追擊以爭功名豈得返坐守淮南數年逗留不出兵乎又舊傳呂用之云恐成功不賞妖亂志云恐敗衂稍損威名夫大功既成則有不賞之懼豈有未戰不知勝負豫憂威望震主乎駢為都統控扼江淮而擁兵縱賊使安然北度其於威名獨無損乎雖用之淺謀無所不至駢自無參酌一至此邪蓋駢好驕矜大言自恃累有戰功謂巢烏合疲弊之衆可以節鉞誘致淮南坐而取之不意巢初無降心反為所欺張璘驍將一戰敗死巢奄濟采石諸軍北去見兵不多狼狽惴恐自保不暇故歛兵退縮任賊過淮非故欲縱之實不能制也盧闇於知人致中原覆沒駢先銳後怯致京邑丘墟呂用之妖妄姧囘致廣陵塗炭皆人所深疾故衆惡歸焉未必實然也又唐末見聞録廣明二年十二月五日黄巢傾陷京國轉牒諸軍据牒云屯軍淮甸牧馬潁陂則似在淮南時非入長安後又續寶運録云王仙芝既叛自稱天補均平大將軍兼海内諸豪帥都統傳檄諸道其文與此畧同末云願垂聽知謹告乾符二年正月三日此蓋當時不逞之士偽作此文託於仙芝及巢以譏斥時病未必二人實有此檄牒也}
詔河南諸道兵屯溵水泰寧節度使齊克讓屯汝州以備黄巢 辛酉以淄州刺史曹全晸為天平節度使兼東面副都統 劉漢宏請降戊辰以為宿州刺史 |{
	考異曰實録漢宏寇擾荆襄王鐸遣前濠州刺史崔鍇招之至是始歸降辛未漢宏奏請於濠州倒戈歸降優詔褒之按鐸奔襄陽漢宏始掠江陵叛去鐸尋分司蓋未分司時遣鍇招之又戊辰漢宏除宿州云至是始降是已降也辛未又云請於濠州歸降者朝廷聞其降戊辰已除官而辛未漢宏表方至也}
李克用自雄武軍引兵還擊高文集於朔州李可舉遣行軍司馬韓玄紹邀之於藥兒嶺|{
	藥兒嶺在雄武軍西}
大破之殺七千餘人李盡忠程懷信皆死|{
	盡忠懷信與克用同起兵於蔚朔者也}
又敗之於雄武軍之境殺萬人|{
	敗補邁翻}
李琢赫連鐸進攻蔚州李國昌戰敗部衆皆潰獨與克用及宗族北入達靼|{
	宋白曰達靼者本東北方之夷蓋靺鞨之部也貞元元和之後奚契丹漸盛多為攻劫部衆分散或投屬契丹或依於勃海漸流徙于隂山其俗語訛因謂之逹靼唐咸通末有首領每相温于越相温部帳于漠南隨草畜牧李克用為吐渾所困嘗往依焉達靼善待之及授雁門節度使二相温帥族帳以從克用收復長安逐黄巢於河南皆從戰有功由是俾牙于雲代之間恣其畜牧}
詔以鐸為雲州刺史大同軍防禦使吐谷渾白義成為蔚州刺史|{
	白義成一作白義誠}
薩葛米海萬為朔州刺史加李可舉兼侍中達靼本靺鞨之别部也居于隂山|{
	歐陽修曰靺鞨本在奚契丹東北後為契丹所攻部族分散居隂山者自號達靼洪景盧曰蕃語以華言譯之皆得其近似耳天竺語轉而為捐篤身毒秃髪語轉而為吐蕃逹靼乃靺鞨也契丹之讀如喫惟新唐書有音冒頓讀如墨突惟晉書音義有之}
後數月赫連鐸隂賂達靼使取李國昌父子李克用知之時與其豪帥遊獵置馬鞭木葉或懸針射之無不中|{
	帥所類翻下同射而亦翻中竹仲翻}
豪帥心服又置酒與飲酒酣克用言曰吾得罪天子願效忠而不得今聞黄巢北來必為中原患一旦天子若赦吾罪得與公輩南向共立大功不亦快乎人生幾何誰能老死沙磧邪逹靼知無留意乃止|{
	赫連鐸蓋說誘達靼豪帥以李克用父子才勇久留逹靼必將并有其部落故使殺之而克用與其豪帥言欲與之南向勤王達靼豪帥知其志大決不肯久居隂山圖并其部落彼既無圖我之心我何苦殺之於是遂止}
八月甲午以前西川節度使崔安濳為太子賓客分司|{
	盧惡之也}
九月東都奏汝州所募軍李光庭等五百人自代州還過東都燒安喜門焚掠市肆由長夏門去|{
	燒洛城東北門由東南門去}
黄巢衆號十五萬曹全晸以其衆六千與之戰頗有殺獲以衆寡不敵退屯泗上|{
	泗上即泗州}
以俟諸軍至併力擊之而高駢竟不之救賊遂擊全晸破之 徐州遣兵三千赴溵水過許昌徐卒素名凶悖|{
	悖蒲妹翻又蒲沒翻}
節度使薛能自謂前鎮彭城|{
	乾符初能鎮徐州今鎮許}
有恩信於徐人舘之毬場|{
	館古玩翻}
及暮徐卒大譟能登子城樓問之對以供備踈闕慰勞久之方定|{
	勞力到翻}
許人大懼時忠武亦遣大將周岌詣溵水|{
	岌逆及翻}
行未遠聞之夜引兵還比明入城|{
	比必利翻}
襲擊徐卒盡殺之且怨能之厚徐卒也遂逐之能將奔襄陽亂兵追殺之并其家岌自稱留後汝鄭把截制置使齊克讓恐為岌所襲引兵還兖州|{
	齊克讓本泰寧節度使引兵還鎮}
諸道屯溵水者皆散黄巢遂悉衆度淮所過不虜掠惟取丁壯以益兵|{
	志在攻長安}
先是徵振武節度使吳師泰為左金吾大將軍以諸葛爽代之|{
	先悉薦翻}
師泰見朝廷多故使軍民上表留已冬十月復以師泰為振武節度使以爽為夏綏節度使|{
	夏戶雅翻}
黄巢陷申州遂入潁宋徐兖之境所至吏民逃潰 羣盜陷灃州殺刺史李詢判官皇甫鎮鎮舉進士二十三上不中第|{
	上禮部者二十三而不中第可謂老於塲屋矣上時掌翻中竹仲翻}
詢辟之賊至城陷鎮走問人曰使君免乎曰賊執之矣鎮曰吾受知若此去將何之遂還詣賊竟與同死|{
	士為知己死皇甫鎮有焉科舉之設烏足以盡天下之士哉}


資治通鑑卷二百五十三
