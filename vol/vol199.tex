<!DOCTYPE html PUBLIC "-//W3C//DTD XHTML 1.0 Transitional//EN" "http://www.w3.org/TR/xhtml1/DTD/xhtml1-transitional.dtd">
<html xmlns="http://www.w3.org/1999/xhtml">
<head>
<meta http-equiv="Content-Type" content="text/html; charset=utf-8" />
<meta http-equiv="X-UA-Compatible" content="IE=Edge,chrome=1">
<title>資治通鑒_200-資治通鑑卷一百九十九_200-資治通鑑卷一百九十九</title>
<meta name="Keywords" content="資治通鑒_200-資治通鑑卷一百九十九_200-資治通鑑卷一百九十九">
<meta name="Description" content="資治通鑒_200-資治通鑑卷一百九十九_200-資治通鑑卷一百九十九">
<meta http-equiv="Cache-Control" content="no-transform" />
<meta http-equiv="Cache-Control" content="no-siteapp" />
<link href="/img/style.css" rel="stylesheet" type="text/css" />
<script src="/img/m.js?2020"></script> 
</head>
<body>
 <div class="ClassNavi">
<a  href="/24shi/">二十四史</a> | <a href="/SiKuQuanShu/">四库全书</a> | <a href="http://www.guoxuedashi.com/gjtsjc/"><font  color="#FF0000">古今图书集成</font></a> | <a href="/renwu/">历史人物</a> | <a href="/ShuoWenJieZi/"><font  color="#FF0000">说文解字</a></font> | <a href="/chengyu/">成语词典</a> | <a  target="_blank"  href="http://www.guoxuedashi.com/jgwhj/"><font  color="#FF0000">甲骨文合集</font></a> | <a href="/yzjwjc/"><font  color="#FF0000">殷周金文集成</font></a> | <a href="/xiangxingzi/"><font color="#0000FF">象形字典</font></a> | <a href="/13jing/"><font  color="#FF0000">十三经索引</font></a> | <a href="/zixing/"><font  color="#FF0000">字体转换器</font></a> | <a href="/zidian/xz/"><font color="#0000FF">篆书识别</font></a> | <a href="/jinfanyi/">近义反义词</a> | <a href="/duilian/">对联大全</a> | <a href="/jiapu/"><font  color="#0000FF">家谱族谱查询</font></a> | <a href="http://www.guoxuemi.com/hafo/" target="_blank" ><font color="#FF0000">哈佛古籍</font></a> 
</div>

 <!-- 头部导航开始 -->
<div class="w1180 head clearfix">
  <div class="head_logo l"><a title="国学大师官网" href="http://www.guoxuedashi.com" target="_blank"></a></div>
  <div class="head_sr l">
  <div id="head1">
  
  <a href="http://www.guoxuedashi.com/zidian/bujian/" target="_blank" ><img src="http://www.guoxuedashi.com/img/top1.gif" width="88" height="60" border="0" title="部件查字,支持20万汉字"></a>


<a href="http://www.guoxuedashi.com/help/yingpan.php" target="_blank"><img src="http://www.guoxuedashi.com/img/top230.gif" width="600" height="62" border="0" ></a>


  </div>
  <div id="head3"><a href="javascript:" onClick="javascript:window.external.AddFavorite(window.location.href,document.title);">添加收藏</a>
  <br><a href="/help/setie.php">搜索引擎</a>
  <br><a href="/help/zanzhu.php">赞助本站</a></div>
  <div id="head2">
 <a href="http://www.guoxuemi.com/" target="_blank"><img src="http://www.guoxuedashi.com/img/guoxuemi.gif" width="95" height="62" border="0" style="margin-left:2px;" title="国学迷"></a>
  

  </div>
</div>
  <div class="clear"></div>
  <div class="head_nav">
  <p><a href="/">首页</a> | <a href="/ShuKu/">国学书库</a> | <a href="/guji/">影印古籍</a> | <a href="/shici/">诗词宝典</a> | <a   href="/SiKuQuanShu/gxjx.php">精选</a> <b>|</b> <a href="/zidian/">汉语字典</a> | <a href="/hydcd/">汉语词典</a> | <a href="http://www.guoxuedashi.com/zidian/bujian/"><font  color="#CC0066">部件查字</font></a> | <a href="http://www.sfds.cn/"><font  color="#CC0066">书法大师</font></a> | <a href="/jgwhj/">甲骨文</a> <b>|</b> <a href="/b/4/"><font  color="#CC0066">解密</font></a> | <a href="/renwu/">历史人物</a> | <a href="/diangu/">历史典故</a> | <a href="/xingshi/">姓氏</a> | <a href="/minzu/">民族</a> <b>|</b> <a href="/mz/"><font  color="#CC0066">世界名著</font></a> | <a href="/download/">软件下载</a>
</p>
<p><a href="/b/"><font  color="#CC0066">历史</font></a> | <a href="http://skqs.guoxuedashi.com/" target="_blank">四库全书</a> |  <a href="http://www.guoxuedashi.com/search/" target="_blank"><font  color="#CC0066">全文检索</font></a> | <a href="http://www.guoxuedashi.com/shumu/">古籍书目</a> | <a   href="/24shi/">正史</a> <b>|</b> <a href="/chengyu/">成语词典</a> | <a href="/kangxi/" title="康熙字典">康熙字典</a> | <a href="/ShuoWenJieZi/">说文解字</a> | <a href="/zixing/yanbian/">字形演变</a> | <a href="/yzjwjc/">金 文</a> <b>|</b>  <a href="/shijian/nian-hao/">年号</a> | <a href="/diming/">历史地名</a> | <a href="/shijian/">历史事件</a> | <a href="/guanzhi/">官职</a> | <a href="/lishi/">知识</a> <b>|</b> <a href="/zhongyi/">中医中药</a> | <a href="http://www.guoxuedashi.com/forum/">留言反馈</a>
</p>
  </div>
</div>
<!-- 头部导航END --> 
<!-- 内容区开始 --> 
<div class="w1180 clearfix">
  <div class="info l">
   
<div class="clearfix" style="background:#f5faff;">
<script src='http://www.guoxuedashi.com/img/headersou.js'></script>

</div>
  <div class="info_tree"><a href="http://www.guoxuedashi.com">首页</a> > <a href="/SiKuQuanShu/fanti/">四库全书</a>
 > <h1>资治通鉴</h1> <!--         下载:【右键另存为】即可 --></div>
  <div class="info_content zj clearfix">
  
<div class="info_txt clearfix" id="show">
<center style="font-size:24px;">200-資治通鑑卷一百九十九</center>
    資治通鑑卷一百九十九 宋 司馬光 撰<br />
<br />
  胡三省 音註<br />
<br />
  唐紀十五【起著雍涒灘四月盡旃蒙單閼九月凡七年有奇】<br />
<br />
  太宗文武大聖大廣孝皇帝下之下<br />
<br />
  貞觀二十二年【觀古玩翻】夏四月丁巳右武候將軍梁建方擊松外蠻破之【松外諸蠻依阻山谷亦屬古南中之地蓋以其在松州之外而得名也新志松外蠻在嶲州昌明縣徼外】初嶲州都督劉伯英上言松外諸蠻蹔降復叛請出師討之以通西洱天竺之道【此即漢武帝欲通之道而為昆明所蔽者也嶲州漢邛都夷之地武帝開置越嶲郡後周武帝置嚴州唐為嶲州嶲音髓上時掌翻蹔與暫同降戶江翻復扶又翻洱乃吏翻】勑建方發巴蜀十二州兵討之【十二州益眉榮梓利綿遂巴盧渠逹集渝也】蠻酋雙舍帥衆拒戰【酋慈由翻帥讀曰率下同】建方擊敗之【敗補邁翻】殺獲千餘人羣蠻震懾亡竄山谷建方分遣使者諭以利害【懾之涉翻使疏吏翻】皆來歸附前後至者七十部戶十萬九千三百建方署其酋長蒙和等為縣令【長知兩翻下同】各統所部莫不感悦因遣使詣西洱河【新書曰西洱河蠻道由郎州走三千里時建方自嶲州道千五百里遣奇兵奄至其地】其帥楊盛大駭具船將遁使者曉諭以威信盛遂請降【帥所類翻降戶江翻】其地有楊李趙董等數十姓各據一州大者六百小者二三百戶無大君長不相統壹語雖小訛其生業風俗大略與中國同自云本皆華人其所異者以十二月為歲首 己未契丹辱紇主曲據帥衆内附【奚契丹酋領皆稱為辱紇主契欺訖翻又音喫帥讀曰率】以其地置玄州以曲據為刺史隸營州都督府甲子烏胡鎮將古神感【烏胡鎮當置於海中烏胡島自登州東北海行過大謝島龜歆島淤島而後至烏胡島又三百里北渡烏湖海姓譜周太王去邠適岐稱古公因氏焉】將兵浮海擊高麗遇高麗步騎五千戰於易山破之【易山新書作曷山將即亮翻麗力知翻騎奇寄翻】其夜高麗萬餘人襲神感船神感設伏又破之而還【還從宣翻又如字】 初西突厥乙毘咄陸可汗【厥九勿翻咄當沒翻可從刋入聲汗音寒】以阿史那賀魯為葉護居多邏斯水在西州北千五百里【邏郎佐翻】統處月處密始蘇歌邏禄失畢五姓之衆乙毘咄陸奔吐火羅【見一百九十六卷十六年】乙毘射匱可汗遣兵迫逐之部落亡散乙亥賀魯帥其餘衆數千帳内屬詔處之於庭州莫賀城【庭州西延城西六十里有沙鉢城守捉蓋即莫賀城也以賀魯後立為沙鉢羅葉護可汗故改城名也處昌呂翻】拜左驍衛將軍【驍堅堯翻】賀魯聞唐兵討龜茲請為鄉導【龜茲音丘慈鄉讀曰嚮】仍從數十騎入朝【朝直遥翻】上以為崑丘道行軍總管厚宴賜而遣之【為賀魯後叛張本】 五月庚子右衛率長史王玄策擊帝那伏帝王阿羅那順大破之【東宫十率府各有長史正七品上新書作那伏帝阿羅那順無王字率所律翻】初中天竺王尸羅逸多兵最彊四天竺皆臣之【天竺國漢身毒國也或曰摩伽陀或曰婆羅門去京師九千六百里居蔥嶺南幅員三萬里分東西南北中五天竺南天竺瀕海北天竺距雪山東天竺際海與扶南林邑接西天竺與罽賓波斯接中天竺在四天竺之會都城曰茶餺和羅城杜佑曰天竺塞種也顔師古曰塞釋也】玄策奉使至天竺諸國皆遣使入貢會尸羅逸多卒國中大亂其臣阿羅那順自立胡兵攻玄策玄策帥從者三十人與戰【使疏吏翻卒子恤翻帥讀曰率從才用翻】力不敵悉為所擒阿羅那順盡掠諸國貢物玄策脱身宵遁抵吐蕃西境以書徵鄰國兵吐蕃遣精鋭千二百人泥婆國遣七千餘騎赴之【泥婆羅國直吐蕃之西樂陵川臣於吐蕃吐從暾入聲騎奇寄翻】玄策與其副蔣師仁帥二國之兵進至中天竺所居茶餺和羅城【帥讀曰率餺音博新書曰茶餺和羅城濱伽毘黎河】連戰三日大破之斬首三千餘級赴水溺死者且萬人【溺奴狄翻】阿羅那順棄城走更收餘衆還與師仁戰又破之擒阿羅那順餘衆奉其妃及王子阻乾陀衛江【水經注曰崑崙山釋氏曰阿耨逹山河水出其東北陬屈從其東南流注于蒲昌海自蒲昌海潛行地下南出積石而為中國河其崑崙山西有大水出焉曰新頭河西南流逕烏長國又東南流逕中天竺國亦曰恒河又西逕四大塔北又西逕陀衛國北所謂乾陀衛江蓋即此也】師仁進擊之衆潰獲其妃及王子虜男女萬二千人於是天竺響震城邑聚落降者五百八十餘所【降戶江翻】俘阿羅那順以歸以玄策為朝散大夫【唐制文散階朝散大夫從五品下朝直遥翻散悉亶翻】 六月乙丑以白霫部為居延州【霫而立翻】 癸酉特進宋公蕭瑀卒太常議諡曰德尚書議諡曰肅【周公諡法剛德克就曰肅諡時利翻】上曰諡者行之迹當得其實【行下孟翻】可諡曰貞褊公【賀琛諡法直道不橈曰貞儉嗇無德曰褊心隘政急曰褊】子鋭嗣尚上女襄城公主上欲為之營第【為于偽翻】公主固辭曰婦事舅姑當朝夕侍側若居别第所闕多矣上乃命即瑀第而營之 上以高麗困弊議以明年發三十萬衆一舉滅之或以為大軍東征須備經歲之糧非畜乘所能載宜具舟艦為水運隋末劒南獨無宼盗屬者遼東之役劒南復不預及【畜許救翻乘繩證翻艦戶黯翻屬之欲翻復扶又翻】其百姓富庶宜使之造舟艦上從之秋七月遣右領左右府長史強偉【領左右府亦分為左右各有長史此即左右千牛府也強其兩翻姓也】於劒南道伐木造舟艦大者或長百尺其廣半之别遣使行水道【長直亮翻行下孟翻】自巫峽抵江揚趣萊州【趣七喻翻】庚寅西突厥相屈利啜請帥所部從討龜茲【相息亮翻屈居勿翻啜陟劣翻帥讀曰率】 初左武衛將軍武連縣公武安李君羨直玄武門【武連縣時屬始州始州後改劒州武安縣漢屬魏郡晉屬廣平郡後周隋屬洺州左右武衛將軍乃南牙諸衛將軍直玄武門則掌北門宿衛】時太白屢晝見太史占云女主昌民間又傳秘記云唐三世之後女主武王代有天下上惡之【見賢遍翻惡烏路翻下深惡同】會與諸武臣宴宫中行酒令【行酒令者一人為令伯餘人以次行之下文使各言小名即酒令也】使各言小名君羨自言名五娘上愕然因笑曰何物女子乃爾勇健又以君羨官稱封邑皆有武字深惡之後出為華州刺史【華戶化翻】有布衣員道信自言能絶粒曉佛法君羨深敬信之數相從屛人語【員音運姓也數所角翻屏必郢翻】御史奏君羨與妖人交通謀不軌【妖於喬翻】壬辰君羨坐誅籍沒其家上密問太史令李淳風秘記所云信有之乎對曰臣仰稽天象俯察歷數其人已在陛下宫中為親屬自今不過三十年當王天下【王于况翻】殺唐子孫殆盡其兆既成矣上曰疑似者盡殺之何如對曰天之所命人不能違也王者不死徒多殺無辜且自今以往三十年其人已老庶幾頗有慈心為禍或淺【幾居希翻】今借使得而殺之天或生壯者肆其怨毒恐陛下子孫無遺類矣上乃止 司空梁文昭公房玄齡留守京師【守手又翻】疾篤上徵赴玉華宫肩輿入殿至御座側乃下相對流涕因留宫下聞其小愈則喜形於色加劇則憂悴【悴秦醉翻】玄齡謂諸子曰吾受主上厚恩今天下無事唯東征未已羣臣莫敢諫吾知而不言死有餘責乃上表諫【上時掌翻】以為老子曰知足不辱知止不殆陛下功名威德亦可足矣拓地開疆亦可止矣且陛下每决一重囚必令三覆五奏進素膳止音樂者【見一百九十三卷五年】重人命也今驅無罪之士卒委之鋒刃之下使肝腦塗地獨不足愍乎【明謹用刑重人命也踴躍用兵則忘人命之為重矣引彼形此玄齡之言可謂深切著明】向使高麗違失臣節誅之可也侵擾百姓滅之可也它日能為中國患除之可也今無此三條而坐煩中國内為前代雪恥外為新羅報讎豈非所存者小所損者大乎【說到此分明見得高麗不必征當時在朝之臣諫東征者未有能及此者也此是忠誠懇切中流出為于偽翻】願陛下許高麗自新焚陵波之船罷應募之衆自然華夷慶賴遠肅邇安臣且夕入地儻蒙録此哀鳴【論語曾子有疾孟敬子問之曾子言曰鳥之將死其鳴也哀人之將死其言也善】死且不朽玄齡子遺愛尚上女高陽公主上謂公主曰彼病篤如此尚能憂我國家上自臨視握手與訣悲不自勝【勝音升】癸卯薨<br />
<br />
  柳芳曰玄齡佐太宗定天下及終相位凡三十二年天下號為賢相【相息亮翻】然無跡可尋德亦至矣故太宗定禍亂而房杜不言功王魏善諫諍而房杜讓其賢英衛善將兵而房杜行其道【新贊作房杜濟以文將即亮翻】理致太平善歸人主為唐宗臣宜哉<br />
<br />
  八月己酉朔日有食之 丁丑勑越州都督府及婺洪等州造海船及雙舫千一百艘【東陽郡隋平陳置婺州舫甫妄翻艘蘇遭翻】辛未遣左領軍大將軍執失思力出金山道擊薛延<br />
<br />
  陁餘宼 九月庚辰崑丘道行軍大總管阿史那社爾擊處月處密破之餘衆悉降【降戶江翻】 癸未薛萬徹等伐高麗還【還從宣翻又如字】萬徹在軍中使氣陵物裴行方奏其怨望坐除名流象州【裴行方副萬徹東伐見上卷上年象州漢潭中中溜縣之地隋為始安郡桂林縣唐武德四年置象州桂林郡以象山名州】 己丑新羅奏為百濟所攻破其十三城 己亥以黄門侍郎褚遂良為中書令強偉等發民造船役及山獠雅邛眉三州獠友【強其兩翻】<br />
<br />
  【卭渠容翻獠魯皓翻】壬寅遣茂州都督張士貴右衛將軍梁建方發隴右峽中兵二萬餘人以擊之蜀人苦造船之役或乞輸直雇潭州人造船上許之州縣督迫嚴急民至賣田宅鬻子女不能供穀價踊貴劒外騷然【自劒門關以南謂之劒外内京師而外諸夏也】上聞之遣司農少卿長孫知人馳驛往視之知人奏稱蜀人脆弱不耐勞劇【脆此芮翻】大船一艘庸絹二千二百三十六匹山谷已伐之木挽曳未畢復徵船庸【艘蘇遭翻復扶又翻】二事倂集民不能堪宜加存養上乃勑潭州船庸皆從官給 冬十月癸丑車駕還京師 回紇吐迷度兄子烏紇蒸其叔母【紇下沒翻】烏紇與俱陸莫賀逹官俱羅勃皆突厥車鼻可汗之壻也相與謀殺吐迷度以歸車鼻烏紇夜引十餘騎襲吐迷度殺之燕然副都護元禮臣使人誘烏紇許奏以為瀚海都督烏紇輕騎詣禮臣謝禮臣執而斬之以聞【燕因肩翻誘音酉騎奇寄翻】上恐回紇部落離散遣兵部尚書崔敦禮往安撫之久之俱羅勃入見上留之不遣【回紇由是又微見賢遍翻】 阿史那社爾既破處月處密引兵自焉耆之西趣龜茲北境【趣七喻翻】分兵為五道出其不意焉耆王薛婆阿那支棄城奔龜茲保其東境社爾遣兵追擊擒而斬之【十六年郭孝恪破焉耆立栗婆凖為王而阿那支殺之今也罪人斯得】立其從父弟先那凖為焉耆王【新書曰立突騎支弟婆伽利為王此從舊書從才用翻】使修職貢龜茲大震守將多棄城走社爾進屯磧口去其都城三百里【磧口新舊書作磧石龜茲都伊邏盧城北倚白山亦曰阿羯田山將即亮翻磧七迹翻】遣伊州刺史韓威帥千餘騎為前鋒【帥讀曰率下同騎奇寄翻】右衛將軍曹繼叔次之至多褐城龜茲王訶利布失畢其相那利羯獵顛帥衆五萬拒戰【相息亮翻羯居謁翻】鋒刃甫接威引兵偽遁龜茲悉衆追之行三十里與繼叔軍合龜茲懼將却繼叔乘之龜茲大敗逐北八十里 甲戍以回紇吐迷度子前左屯衛大將軍婆閏為左驍衛大將軍大俟利發瀚海都督【驍堅堯翻俟渠之翻 考異曰舊回紇傳云詔西突厥可汗阿史那賀魯統五啜五俟斤二十餘部居多羅斯水南去西州馬行十五日程回紇不肯西屬突厥按賀魯時為將軍自多羅斯水入居庭州永徽二年乃西遁自稱可汗所統咄陸五啜弩失畢五俟斤唐末嘗以回紇隸之也今不取】 十一月庚子契丹帥窟哥奚帥可度者並帥所部内屬【帥所類翻下别帥同並帥讀曰率】以契丹部為松漠府【杜佑曰松漠之地在柳城郡之北】以窟哥為都督又以其别帥逹稽等部為峭落等九州各以其辱紇主為刺史【峭落州無逢州羽陵州白連州徒何州萬丹州疋黎州赤山州并松漠府為九州峭七笑翻】以奚部為饒樂府以可度者為都督【樂音洛】又以其别帥阿會等部為弱水等五州亦各以其辱紇主為刺史【弱水州祁黎州洛壞州太魯州渇野州】 辛丑置東夷校尉官於營州【校戶教翻】 十二月庚午太子為文德皇后作大慈恩寺成【西京雜記西京外城朱雀街東第三橋皇城之東第一街進業坊隋無漏寺之故基太子即其地建寺為文德皇后祈福竹木森邃為京城觀游之最雍録曰慈恩寺在朱雀街東第三街自北次南第十五坊名曰進昌坊寺南臨黄渠竹木森邃為于偽翻】社龜兹王布失畢既敗走保都城阿史那社爾進軍逼之布失畢輕騎西走社爾拔其城使安西都護郭孝恪守之沙州刺史蘇海政尚輦奉御薛萬備帥精騎追布失畢行六百里布失畢窘急保撥換城【自安西府西出柘厥關渡白馬河四百餘里至撥換城騎奇寄翻帥讀曰率】 爾進軍攻之四旬閏月丁丑拔之擒布失畢及羯獵顛那利脱身走濳引西突厥之衆并其國兵萬餘人襲擊孝恪孝恪營於城外龜茲人或告之孝恪不以為意那利奄至孝恪帥所部千餘人將入城那利之衆已登城矣城中降胡與之相應【降戶江翻下同】共擊孝恪矢刃如雨孝恪不能敵將復出【復扶又翻下利復同】死於西門城中大擾倉部郎中崔義超【倉部郎掌判天下倉儲受納租税出給禄廩之事屬戶部義超以是官從軍】召募得二百人衛軍資財物與龜茲戰於城中曹繼叔韓威亦營於城外自城西北隅擊之那利經宿乃退斬首三千餘級城中始定後旬餘日那利復引山北龜兹萬餘人趣都城【山北蓋白山之北也趣七喻翻】繼叔逆擊大破之斬首八千級那利單騎走龜茲人執之以詣軍門阿史那社爾前後破其大城五遣左衛郎將權袛甫詣諸城開示禍福皆相帥請降【帥讀曰率降戶江翻】凡得七百餘城虜男女數萬口社爾乃召其父老宣國威靈諭以伐罪之意立其王之弟葉護為王龜茲人大喜西域震駭西突厥于闐安國爭饋駝馬軍糧【闐徒賢翻又徒見翻】社爾勒石紀功而還【還從宣翻又如字】 戊寅以崑丘道行軍總管左驍衛將軍阿史那賀魯為泥伏沙鉢羅葉護賜以鼓纛使招討西突厥之未服者【假賀魯以羽翼正速其叛耳驍堅堯翻纛徒到翻】 癸未新羅相金春秋及其子文王入見【相息亮翻見賢遍翻】春秋真德之弟也上以春秋為特進文王為左武衛將軍春秋請改章服從中國内出冬服賜之<br />
<br />
  二十三年春正月辛亥龜茲王布失畢及其相那利等至京師上責讓而釋之以布失畢為左武衛中郎將【龜茲音丘慈又音屈佳將即亮翻 考異曰實録云左武衛翊衛中郎將舊傳為武翊衛中郎將按會要武德五年改左右翊衛為左右衛然則於時已無翊衛之名且布失畢必不獨兼兩衛之官今去翊衛字 按唐六典左右衛有親勲翊三衛中郎將其餘諸衛府各有翊衛中郎將翊衛二字恐不可去】西南徒莫袛等蠻内附以其地為傍望覽丘四州隸郎州都督府【徒莫袛蠻在爨蠻之西郎州當作朗州武德元年開南中仍舊置南寧州貞觀八年改為郎州以其地本夜郎國也】 上以突厥車鼻可汗不入朝遣右驍衛郎將高侃回紇僕骨等兵襲擊之兵入其境諸部落相繼來降拔悉密吐屯肥羅察降以其地置新黎州【舊書云車鼻長子羯漫陀先統拔悉密部遣其子菴鑠入朝帝嘉之為置新黎州朝直遥翻降戶江翻 考異曰高宗實録云初突厥車鼻可汗遣其子車鉢羅入貢太宗遣使徵之不至太宗大怒遣右驍衛郎將高侃引回紇僕骨等兵襲擊之其下諸部落相次歸降其子羯漫陀先統拔悉密部泣諫其父請歸國車鼻不聼羯漫陀遂背父來降以其地為新黎州舊傳云二十三年遣右驍衛郎將高侃潛引回紇僕骨等兵衆襲撃之其酋長歌邏禄泥執闕俟利乃拔塞匐處木昆莫賀咄俟斤等帥部落背車鼻相繼來降車鼻長子羯漫陀先統拔悉密部車鼻未敗前遣其子菴鑠入朝太宗嘉之拜左屯衛將軍更置新黎州以統其衆今從太宗實録】 二月丙戍置瑶池都督府【此因穆天子傳西王母觴天子於瑶池之上而名之也】隸安西都護戊子以左衛將軍阿史那賀魯為瑶池都督 三月丙辰置豐州都督府使燕然都護李素立兼都督 去冬旱至是始雨辛酉上力疾至顯道門外赦天下丁卯勑太子於金液門聼政【按唐六典城門郎掌京城皇城宫殿諸門明德等門為京城門朱雀等門為皇城門承天等門為宫城門嘉德等門為宫門太極等門為殿門通内諸門並同上閤門顯道金液其亦通内諸門之門歟圖志不能盡載耳】 夏四月乙亥上行幸翠微宫上謂太子曰李世勣才智有餘然汝與之無恩恐不能懷服我今黜之若其即行俟我死汝於後用為僕射新任之若徘徊顧望當殺之耳五月戊午以同中書門下三品李世勣為疊州都督世勣受詔不至家而去【史言太宗以機數御李世勣世勣亦以機心而事君杜佑曰疊州去京師千三百四十里孫愐曰疊州自秦至魏諸羌據焉周武帝逐諸羌乃置疊州蓋以山重疊名之】 辛酉開府儀同三司衛景武公李靖薨 上苦利增劇太子晝夜不離側【離力智翻】或累日不食髪有變白者上泣曰汝能孝愛如此吾死何恨丁卯疾篤召長孫無忌入含風殿【含風殿在翠微宫】上臥引手捫無忌頤無忌哭悲不自勝【勝音升】上竟不得有所言因令無忌出己巳復召無忌及褚遂良入臥内【復扶又翻】謂之曰朕今悉以後事付公輩太子仁孝公輩所知善輔導之謂太子曰無忌遂良在汝勿憂天下又謂遂良曰無忌盡忠於我我有天下多其力也我死勿令讒人間之【武許之間二臣玉几之命猶在高宗之耳何遽忘之邪間古莧翻】仍令遂良草遺詔有頃上崩【年五十有三】太子擁無忌頸號慟將絶無忌攬涕請處分衆事以安内外太子哀號不已【號戶高翻處昌呂翻分扶問翻】無忌曰主上以宗廟社稷付殿下豈得效匹夫唯哭泣乎乃秘不發喪庚午無忌等請太子先還飛騎勁兵及舊將皆從【騎奇寄翻將即亮翻從才用翻】辛未太子入京城大行御馬輿侍衛如平日繼太子而至頓於兩儀殿以太子左庶子于志寧為侍中少詹事張行成兼侍中以檢校刑部尚書右庶子兼吏部侍郎高季輔兼中書令壬申喪於太極殿宣遺詔太子即位【太極殿西内正朝於此喪太子於柩前即位】軍國大事不可停闕平常細務委之有司諸王為都督刺史者並聼奔喪濮王泰不在來限罷遼東之役及諸土木之功四夷之人入仕於朝及來朝貢者數百人聞喪皆慟哭翦髪剺面割耳流血灑地【朝直遥翻剺里之翻】 六月甲戌朔高宗即位赦天下 丁丑以疊州都督李勣為特進檢校洛州刺史洛陽宫留守【李世勣去世字避太宗二名也守手又翻】 先是太宗二名令天下不連言者勿避【先悉薦翻】至是始改官名犯先帝諱者【孔頴逹曰曲禮卒哭乃諱注云敬鬼神之名也諱避也生者不相避名衛候名惡大夫有名惡君臣同名春秋不非按昭七年衛候惡卒穀梁傳云昭元年有衛齊惡今衛候惡所謂君臣同名也君子不奪人親所名也】 癸未以長孫無忌為太尉兼檢校中書令知尚書門下二省事無忌固辭知尚書省事帝許之仍令以太尉同中書門下三品【唐制三公正一品無忌既為太尉而令同中書門下三品當時朝議之失也】 癸巳以李勣為開府儀同三司同中書門下三品 阿史那社爾之破龜茲也行軍長史薛萬備請因兵威說于闐王伏闍信入朝【說輪芮翻闍視遮翻朝直遥翻】社爾從之秋七月己酉伏闍信隨萬備入朝詔入謁梓宫 八月癸酉夜地震晉州尤甚壓殺五千餘人庚寅葬文皇帝于昭陵【昭陵在京兆醴泉縣西北六十里九嵕山】廟號太宗【自唐太宗後為臣子者率稱其君之廟號豈非子孫臣民亦病其謚號太多非實而古者袓有功宗有德之義微乎】 阿史那社爾契苾何力請殺身殉葬上遣人諭以先旨不許蠻夷君長為先帝所擒服者頡利等十四人皆琢石為其像刻名列於北司馬門内 丁酉禮部尚書許敬宗奏弘農府君廟應毁【弘農府君魏弘農太守重耳也於高宗為七世祖親盡應毁】請藏主於西夾室從之【太廟有東西夾室夾太室兩旁故謂之夾室】九月乙卯以李勣為左僕射【行先帝之治命也】 冬十月以<br />
<br />
  突厥諸部置舍利等五州隸雲中都督府【五州舍利州思辟州阿史那州綽州白登州】蘇農等六州隸定襄都督府【史只載蘇農州阿德州執失州拔延州餘二州逸】 乙亥上問大理卿唐臨繋囚之數對曰見囚五十餘人【見賢遍翻】唯二人應死上悦上嘗録繫囚前卿所處者多號呼稱寃臨所處者獨無言上怪問其故囚曰唐卿所處本自無寃【號戶高翻處昌呂翻】上歎息良久曰治獄者不當如是邪【治直之翻】 上以吐蕃贊普弄讃為駙馬都尉【漢武帝置三都尉曰奉車都尉曰駙馬都尉曰騎都尉唐以騎都尉為勲官駙馬都尉以授尚主者奉車都尉不復除授】封西海郡王贊普致書于長孫無忌等云天子初即位臣下有不忠者當勒兵赴國討除之【吐蕃以太宗晏駕固有輕中國之心矣】 十二月詔濮王泰開府置僚屬車服珍膳特加優異<br />
<br />
  高宗天皇大聖大弘孝皇帝上之上【諱治字為善小字雉奴太宗第九子也文明元年諡曰天皇大帝廟號高宗天寶八年加尊號高宗天皇大聖皇帝十三載加尊號高宗天皇大聖大弘孝皇帝】<br />
<br />
  永徽元年春正月辛丑朔改元 丙午立妃王氏為皇后后思政之孫也【王思政為西魏守頴川沒於東魏】以后父仁祐為特進魏國公 己未以張行成為侍中 辛酉上召朝集使【朝直遥翻使疏吏翻】謂曰朕初即位事有不便於百姓者悉宜陳不盡者更封奏自是日引刺史十人入閤問以百姓疾苦及其政治【治直吏翻】有洛陽人李弘泰誣告長孫無忌謀反上命立斬之無忌與褚遂良同心輔政上亦尊禮二人恭已以聼之【以帝之尊任二人如此武后譛而去之雖墜諸淵不悔也哲婦之為鴟梟也尚矣】故永徽之政百姓阜安有貞觀之遺風【觀古玩翻】 太宗女衡山公主應適長孫氏有司以為服既公除欲以今秋成昏于志寧上言漢文立制本為天下百姓公主服本斬衰【上時掌翻為于偽翻衰倉回翻】縱使服隨例除豈可情隨例改請俟三年喪畢成昏上從之 二月辛卯立皇子孝為許王上金為杞王素節為雍王【帝後宫鄭生孝楊生上金蕭淑妃生素節雍於用翻】 夏五月壬戍吐蕃贊普弄讃卒【卒子恤翻】其嫡子早死立其孫為贊普贊普幼弱政事皆决於國相禄東贊【相息亮翻】禄東贊性明逹嚴重行兵有法吐蕃所以彊大威服氐羌皆其謀也 六月高侃擊突厥至阿息山車鼻可汗召諸部兵皆不赴與數百騎遁去侃帥精騎追至金山擒之以歸其衆皆降【騎奇寄翻帥讀曰率降戶江翻】 初阿史那爾虜龜茲王布失畢立其弟為王【事見太宗貞觀二十六年】唐兵<br />
<br />
  既還其酋長爭立更相攻擊【酋慈由翻長知兩翻更工衡翻】秋八月壬午詔復以布失畢為龜茲王【復扶又翻】遣歸國撫其衆 九月庚子高侃執車鼻可汗至京師釋之拜左武衛將軍處其餘衆於鬱督軍山【處昌呂翻】置狼山都督府以統之以高侃為衛將軍【唐無衛將軍衛字之上須有脱字】於是突厥盡為封内之臣分置單于瀚海二都護府單于領狼山雲中桑乾三都督蘇農等一十四州瀚海領瀚海金徽新黎等七都督仙萼等八州各以其酋長為刺史都督【新書作蘇農二十四州舊書作一十四州又攷是後調露元年温傅奉職二部反二十四州皆叛應之則二字為是然單于都護府所領見於史者蘇農等四州舍利等五州及桑乾府所領郁射藝失畢失叱略等四州呼延府所領賀魯葛邏跌等三州財十九州耳其五州逸無所攷又有定襄呼延二都督而無狼山都督是其廢置離合不可詳也狼山府顯慶三年廢為州金徽當作金微瀚海都護府領瀚海金微新黎幽陵龜林堅昆六都督府其一逸仙萼浚稽余吾稽落居延寘顔榆溪渾河燭龍凡八州宋白曰振武軍舊為單于都護府即漢定襄郡之盛樂縣也在隂山之陽黄河之北西南至東受降城百二十里瀚海都護後移於回紇本部乾音干】 癸亥上出畋遇雨問諫議大夫昌樂谷那律曰油衣若為則不漏【炙轂子曰惟絹油之製及㡌油陳始有之樂音洛】對曰以瓦為之必不漏上悦為之罷獵【悦為之為于偽翻 考異曰舊書那律傳云嘗從太宗出獵在塗遇雨有此語意欲太宗不為畋獵太宗悦賜帛二百段唐録政要高宗出獵有此月日唐統紀亦在此年今從之】 李勣固求解職冬十月戊辰解勣左僕射以開府儀同三司同中書門下三品 己未監察御史陽武韋思謙【陽武縣漢屬河南郡自晉以來屬榮陽郡監工銜翻】劾奏中書令褚遂良抑買中書譯語人地【中書掌受四方朝貢及通表疏故有譯語人劾戶槩翻又戶得翻】大理少卿張叡冊以為准估無罪思謙奏曰佑價之設備國家所須臣下交易豈得准佑為定【估音古】叡冊舞文附下罔上罪當誅是日左遷遂良為同州刺史叡冊循州刺史思謙名仁約以字行 十二月庚午梓州都督謝萬歲兖州都督謝法興與黔州都督李孟嘗討琰州叛獠【梓州當作牂州武德三年牂柯蠻酋謝龍羽降以其地置牂州兖州當作充州武德三年以牂柯蠻别部置琰州亦蠻州貞觀四年置皆屬黔州都督府黔音琴獠魯皓翻】萬歲法興入洞招慰為獠所殺<br />
<br />
  二年春正月乙巳以黄門侍郎宇文節中書侍郎柳奭並同中書門下三品奭亨之兄子【柳亨西魏尚書左僕射慶之孫竇誕之壻也亨妻即襄陽公主之女】王皇后之舅也 左驍衛將軍瑶池都督阿史那賀魯【驍堅堯翻】招集離散廬帳漸盛聞太宗崩謀襲取西庭二州庭州刺史駱弘義知其謀表言之上遣通事舍人橋寶明馳往慰撫寶明說賀魯令長子咥運入宿衛授右驍衛中郎將尋復遣歸咥運乃說其父擁衆西走【說輸芮翻復扶又翻】擊破乙毘射匱可汗倂其衆建牙于雙河及千泉【自雙河西南抵賀魯牙帳二百里千泉屬石國界又在賀魯牙帳西南新書曰素葉城西四百里至千泉地贏二百里南雪山三垂平陸多泉地因名之】自號沙鉢羅可汗咄陸五啜弩失畢五俟斤皆歸之勝兵數十萬【咄當沒翻啜步劣翻俟渠之翻勝音升】與乙毘咄陸可汗連兵處月處密及西域諸國多附之以咥運為莫賀咄葉護【咥徒結翻】焉耆王婆伽利卒國人表請復立故王突騎支【率子恤翻復扶又翻騎奇寄翻】夏四月詔加突騎支右武衛將軍遣還國 金州刺史滕王元嬰驕奢縱逸居亮隂中畋遊無節數夜開城門勞擾百姓或引彈彈人或埋人雪中以戲笑【數所角翻引彈徒旦翻】上賜書切讓之且曰取適之方亦應多緒晉靈荒君何足為則【左傳晉靈公不君從臺上彈人以觀其避丸】朕以王至親不能致王於法今書王下上考以愧王心元嬰與蔣王惲皆好聚斂【惲於粉翻好呼到翻斂力贍翻】上嘗賜諸王帛各五百段獨不及二王勑曰滕叔蔣兄自能經紀不須賜物給麻兩車以為錢貫二王大慙 秋七月西突厥沙鉢羅可汗寇庭州攻陷金嶺城及蒲類縣【西州交河縣北行八十里入谷又百三十里經柳谷渡金沙嶺百六十里至庭州蒲類縣屬西州後屬庭州又改為後庭縣】殺略數千人詔左武候大將軍梁建方右驍衛大將軍契苾何力為弓月道行軍總管【弓月城在庭州西千有餘里】右驍衛將軍高德逸右武候將軍薛孤吳仁為副發秦成岐雍府兵三萬人【成州漢武都上禄下辨之地後魏置仇池郡漢陽郡南秦州西魏改曰成州雍州京兆郡雍於用翻】及回紇五萬騎以討之【紇下沒翻】 癸巳詔諸禮官學士議明堂制度以高袓配五天帝太宗配五人帝【五天帝注已見七十九卷晉武帝泰始二年五人帝東方帝太皥西方帝少皥南方帝炎帝北方帝顓頊中央帝黄帝】 八月己巳以于志寧為左僕射張行成為右僕射高季輔為侍中志寧行成仍同中書門下三品 己卯郎州白水蠻反寇麻州【白水蠻與青蛉弄棟接隸郎州麻州貞觀二十二年分郎州置】遣左領軍將軍趙孝袓等兵討之 九月癸巳廢玉華宫為佛寺戊戌更命九成宫為萬年宫【更工衡翻】 庚戌左武候引駕盧文操踰牆盗左藏物上以引駕職在糾繩【左右武候掌宫中及京城晝夜巡警之法以執禦非違有引駕仗三衛六十人引駕快飛六十六人左右藏晉始有之唐因而不改各有令一人宋白曰唐制左右金吾有引駕仗百四十人以三衛人數充左藏掌邦國庫藏右藏掌國寶貨藏徂浪翻】乃自為盗命誅之諫議大夫蕭鈞諫曰文操情實難原然法不至死上乃免文操死顧侍臣曰此真諫議也 閏月長孫無忌等上所刪定律令式【上時掌翻】甲戌詔頒之四方 上謂宰相曰聞所在官司行事猶互觀顔面多不盡公長孫無忌對曰此豈敢言無然肆情曲法實亦不敢至於小小收取人情恐陛下尚不能免無忌以元舅輔政凡有所言上無不嘉納 冬十有一月辛酉上祀南郊 癸酉詔自今京官及外州有獻鷹隼及犬馬者罪之【隼息允翻】戊寅特浪羌酋董悉奉求辟惠羌酋卜檐莫各帥種<br />
<br />
  落萬餘戶詣茂州内附【特浪辟惠皆生羌也是年以其地置蓬魯等三十二州屬茂州都督府酋慈由翻檐余亷翻帥讀曰率種章勇翻】竇州義州蠻酋李寶誠等反【義州漢猛陵縣地梁置永業郡隋改為懷德縣屬瀧州唐武德五年置南義州貞觀二年曰義州】桂州都督劉伯英討平之 郎州道總管趙孝祖討白水蠻蠻酋秃磨蒲及儉彌于帥衆據險拒戰孝祖皆擊斬之會大雪蠻飢凍死亡略盡孝祖奏言貞觀中討昆州烏蠻始開青蛉弄棟為州縣【昆州漢益州郡地隋置昆州以亂廢唐武德初開南中復置柘東兩㸑蠻自曲州靖州西南昆川曲軛晉寧喻獻安寧距龍和城通謂之西㸑白蠻自彌鹿升麻二川南至步頭謂之東㸑烏蠻青蛉漢武帝開為縣屬越嶲郡弄棟縣屬益州郡晉並屬雲南郡後屬興寧郡隋亂與中國絶唐以青蛉地置髳州弄棟地置裒川】弄棟之西有小勃弄大勃弄二川【勃弄屬漢永昌郡界唐武德七年置南雲州貞觀八年更名匡州】恒扇誘弄棟欲使之反【恒戶登翻】其勃弄以西與黄瓜葉榆西洱河相接【葉榆亦漢武帝開為縣有葉榆澤屬益州郡後漢屬永昌郡晉屬雲南郡後分屬東河陽郡】人衆殷實多於蜀川無大酋長好結讎怨【好呼到翻】今因破白水之兵請隨便西討撫而安之勑許之 十二月壬子處月朱邪孤注殺招慰使單道惠【邪音耶單音善】與突厥賀魯相結 是歲百濟遣使入貢上戒之使勿與新羅高麗相攻不然吾將兵討汝矣<br />
<br />
  三年春正月己未朔吐谷渾新羅高麗百濟並遣使入貢 癸亥梁建方契苾何力等大破處月朱邪孤注於牢山【新書牢山亦曰賭蒲東北距烏德犍山度馬行十五日】孤注夜遁建方使副總管高德逸輕騎追之【騎奇寄翻】行五百餘里生擒孤注斬首九千級軍還御史劾奏梁建方兵力足以追討而逗留不進高德逸勑令市馬自取駿者【劾戶槩翻又戶得翻】上以建方等有功釋不問大理卿李道裕奏言德逸所取之馬筋力異常請實中廐【中廐猶言内廐也】上謂侍臣曰道裕法官進馬非其本職妄希我意豈朕行事不為臣下所信邪朕方自咎故不復黜道裕耳【復扶又翻】 己巳以同州刺史褚遂良為吏部尚書同中書門下三品 丙子上饗太廟丁亥饗先農躬耕籍田【漢儀天子正月親耕籍田告祠先農先農即神農也祠以太牢百官皆從唐制天子以孟冬吉亥享先農而遂以耕籍】 二月甲寅上御安褔門樓【唐六典長安皇城西面二門北曰安褔南曰順義安福西直京城之開遠門】觀百戲乙卯上謂侍臣曰昨登樓欲以觀人情及風俗奢儉非為聲樂【為于偽翻】朕聞胡人善為擊鞠之戲【鞠以韋為之實以柔物今謂之毬子】嘗一觀之昨初升樓即有羣胡擊鞠意謂朕篤好之也【好呼到翻】帝王所為豈宜容易【易以䜴翻】朕已焚此鞠冀杜胡人窺望之情亦因以為誡 三月辛巳以宇文節為侍中柳奭為中書令以兵部侍郎三原韓瑗守黄門侍郎同中書門下三品【瑗于眷翻】 夏四月趙孝祖大破西南蠻斬小勃弄酋長歿盛擒大勃弄酋長楊承顛自餘皆屯聚保險大者有衆數萬小者數千人孝祖皆破降之【降戶江翻】西南蠻遂定 甲午澧州刺史彭思王元則薨【澧音禮】六月戊申遣兵部尚書崔敦禮等將并汾步騎萬人往茂州【茂州考之新舊志無之當置之於薛延陀故地將即亮翻】薛延陀餘衆渡河置祁連州以處之 秋七月丁巳立陳王忠為皇太子赦天下王皇后無子柳奭為后謀【為于偽翻】以忠母劉氏微賤勸后立忠為太子冀其親已外則諷長孫無忌等使請於上上從之乙丑以于志寧兼太子少師張行成兼少傳高季輔兼少保 丁丑上問戶部尚書高履行去年進戶多少【戶部尚書即民部尚書避太宗諱改焉進戶新增進之戶也少詩沼翻】履行奏去年進戶總一十五萬因問隋代及今日見戶【見賢遍翻】履行奏隋開皇中戶八百七十萬即今戶三百八十萬【即今猶言當今也唐人多有此語】履行士亷之子也 九月守中書侍郎來濟同中書門下三品 冬十一月庚寅弘化長公主自吐谷渾來朝【弘化公主貞觀十三年降吐谷渾慕容諾曷鉢長知兩翻】 癸巳濮王泰薨於均州【濮博木翻】 散騎常侍房遺愛尚太宗女高陽公主【散悉亶翻騎奇寄翻】公主驕恣甚房玄齡薨公主教遺愛與兄遺直異財既而反譖遺直遺直自言太宗深責讓主由是寵衰主怏怏不悦【怏於兩翻】會御史劾盗得浮屠辯機寶枕【浮屠正號曰佛陀與浮屠音聲相近皆西方言其來轉為二音華言譯之則謂之諍覺言滅穢成明道為聖悟劾戶槩翻又戶得翻】云主所賜主與辯機私通餉遺億計【餉遺唯季翻】更以二女子侍遺愛太宗怒腰斬辯機殺奴婢十餘人主益怨望太宗崩無戚容上即位主又令遺愛與遺直更相訟【直更工衡翻】遺愛坐出為房州刺史【房州古房陵上庸地西魏置光遷國後周改曰遷州随改曰房州尋廢州為房陵郡唐復曰房州】遺直為隰州刺史又浮屠智朂等數人私侍主主使掖庭令陳玄運伺宫省禨祥【掖庭局令從七品下官者為之屬内侍省掌宫禁女工之事凡宫人名籍司其除附禨居希翻又其既翻】先是駙馬都尉薛萬徹【高祖女丹陽公主下嫁薛萬徹先悉薦翻】坐事除名徙寧州刺史入朝與遺愛欵昵【朝直遥翻昵尼質翻】對遺愛有怨望語且曰今雖病足坐置京師鼠輩猶不敢動因與遺愛謀若國家有變當奉司徒荆王元景為主元景女適遺愛弟遺則由是與遺愛往來元景嘗自言夢手把日月駙馬都尉柴令武紹之子也【柴紹尚高祖女平陽公主】尚巴陵公主【巴陵公主太宗之女】除衛州刺史託以主疾留京師求醫因與遺愛謀議相結高陽公主謀黜遺直奪其封爵使人誣告遺直無禮於己遺直亦言遺愛及主罪云罪盈惡稔恐累臣私門【累力瑞翻】上令長孫無忌鞫之【令力丁翻長知兩翻】更獲遺愛及主反狀司空安州都督吳王恪母隋焬帝女也恪有文武才太宗常以為類已欲立為太子無忌固爭而止【事見一百九十七卷貞觀十七年】由是與無忌相惡恪名望素高為物情所向無忌深忌之欲因事誅恪以絶衆望遺愛知之因言與恪同謀冀如紇干承基得免死【事見一百九十六卷一百九十七卷貞觀十七年】<br />
<br />
  四年春二月甲申詔遺愛萬徹令武皆斬元景恪高陽巴陵公主並賜自盡上泣謂侍臣曰荆王朕之叔父吳王朕兄欲匄其死可乎【匄居大翻】兵部尚書崔敦 以為不可乃殺之萬徹臨刑大言曰薛萬徹大健兒留為國家効死力豈不佳【尚辰羊翻為于偽翻】乃坐房遺愛殺之乎吳王恪且死罵曰長孫無忌竊弄威權構害良善宗社有靈當族滅不久乙酉侍中兼太子詹事宇文節特進太常卿江夏王道宗左驍衛大將軍駙馬都尉執失思力【高祖女九江公主下嫁執失思力夏戶雅翻驍堅堯翻】並坐與房遺愛交通流嶺表節與遺愛親善及遺愛下獄節頗左右之【下遐嫁翻左右讀曰佐佑】江夏王道宗素與長孫無忌褚遂良不恊故皆得罪戊子廢恪母弟蜀王愔為庶人置巴州【愔於今翻】房遺直貶春州銅陵尉【銅陵縣漢允吾縣地屬合浦郡宋置瀧潭縣屬新寧郡隋改為銅陵縣屬端州唐初屬春州】萬徹弟萬備流交州罷房玄齡配饗【鄭樵曰盤庚云兹予大享于先王爾袓其從與享之周制凡有功者祭于大蒸漢制祭功臣於庭生時侍燕於庭死則降在庭位謂之配饗】開府儀同三司李勣為司空 初林邑王范頭利卒【卒子恤翻】子真龍立大臣伽獨弑之盡滅范氏伽獨自立國人弗從乃立頭利之壻婆羅門為王國人咸思范氏復罷婆羅門【復扶又翻】立頭利之女為王女不能治國【治直之翻】有諸葛地者頭利之姑子也父為頭利所殺南奔真臘【真臘一名吉蔑本扶南屬國去京師二萬七百里東距車渠西屬驃南瀕海北與道明接東北抵驩州貞觀初并扶南有其地】大臣可倫翁定遣使迎而立之【使疏吏翻下同】妻以女王【妻七細翻】衆然後定夏四月戊子遣使入貢 秋九月壬戍右僕射北平定公張行成薨【諡法純行不爽曰定】甲戍以褚遂良為右僕射同中書門下三品如故仍知選事【選須絹翻】 冬十月庚子上幸驪山温湯乙巳還宫 初睦州女子陳碩真【吳孫權分丹陽立新安郡隋仁壽三年置睦州大業初廢州為遂安郡唐復為睦州】以妖言惑衆【妖於喬翻下同】與妹夫章叔胤舉兵反自稱文佳皇帝以叔胤為僕射甲子夜叔胤帥衆攻桐廬陷之【吳分富春立桐廬縣屬吳郡隋唐屬睦州九域志縣在州東一百五里頃安世曰桐廬縣魏黄初四年吳置以桐溪側有大桐樹垂條偃蓋旁䕃數畝遠望如廬因謂之桐廬帥讀曰率】碩真撞鍾焚香【撞直江翻】引兵二千攻陷睦州及於潛【於潛縣漢屬丹楊郡晉宋屬吳興郡梁陳屬錢唐郡隋唐屬杭州宋白曰吳越春秋秦徙大越鳥語之人寘之替闞駰十三州志替讀為潛吳録地理志縣西有替山舊替字無水至隋加水於如字】進攻歙州不克【歙音懾】敕揚州刺史房仁裕兵討之碩真遣其黨童文寶將四千人寇婺州【將即亮翻下同】刺史崔義玄發兵拒之民間訛言碩真有神犯其兵者必滅族士衆兇懼【兇許勇翻】司功參軍崔玄籍曰【功倉戶兵法士參軍所謂州判司也】起兵仗順猶且無成况憑妖妄其能久乎義玄以玄籍為前鋒自將州兵繼之至下淮戍遇賊與戰左右以楯蔽義玄【楯食尹翻】義玄曰刺史避箭人誰致死命撤之於是士卒齊奮賊衆大潰斬首數千級聼其餘衆歸首【歸首式又翻】進至睦州境降者萬計【降戶江翻】十一月庚戍房仁裕軍合獲碩真叔胤斬之餘黨悉平義玄以功拜御史大夫【御史大夫天子耳目官也非以賞功厥後崔義玄承中宫旨繩長孫無忌等豈不忝厥官哉】癸丑以兵部尚書崔敦禮為侍中 十二月庚子侍中蓨憲公高季輔薨【諡法博聞多能曰憲蓚音條】 是歲西突厥乙毗咄陸可汗卒其子頡苾逹度設號真珠葉護始與沙鉢羅可汗有隙與五弩失畢共撃沙鉢羅破之斬首千餘級<br />
<br />
  五年春正月壬戍羌酋凍就内附以其地置劒州【凍就特浪生羌卜樓大首領也劒州羈縻屬松州都督府】 三月戊午上行幸萬年宫【考異曰實録戊午以下皆為二月按長歷二月丁丑朔無戊午戊午三月十二日也】 庚申加贈武德功臣屈突通等十三人官初王皇后無子蕭淑妃有寵 【考異曰新舊唐書或作蕭淑妃或作蕭良娣實録皆作良娣廢王后詔亦曰良娣蕭氏按當時後宫位號無良娣名唯漢世太子宫有良娣疑高宗在東宫時蕭為良娣及即位拜淑妃也】王后疾之上之為太子也入侍太宗見才人武氏而悦之【才人晉武帝所制爵視千石以下宋齊之時以為散職梁於九嬪之下置五職三職才人位列三職比駙馬都尉唐承隋制才人五人正五品】太宗崩武氏隨衆感業寺為尼【長安志曰貞觀二十三年五月太宗上僊其年即以安業坊濟度尼寺為靈寶寺盡度太宗嬪御為尼以處之程大昌曰以通鑑及長安志及呂大坊長安圖參定通鑑言武氏在感業寺長安志在安業寺惟此差不同然志能言寺之位置及始末則安業者是也】忌日上詣寺行香見之武氏泣上亦泣王后聞之隂令武氏長髪【長知兩翻】勸上内之後宫欲以間淑妃之寵【閒古莧翻】武氏巧慧多權數初入宫卑辭屈體以事后后愛之數稱其美於上【數所角翻】未幾大幸【幾居豈翻】拜為昭儀后及淑妃寵皆衰更相與共譛之上皆不納昭儀欲追贈其父而無名故託以褒賞功臣而武士彠預焉【為廢皇后淑妃張本彠一虢翻】 乙丑上幸鳳泉湯【鳳泉湯在岐州郿縣】乙巳還萬年宫夏四月大食發兵擊波斯【波斯國居逹遏水之西距京師萬五千里而嬴東與吐火羅康接北鄰突厥可薩部西南皆瀕海其先波斯匿王大月氏别裔王因以姓又以為國號杜佑曰波斯國即條支之故地大月氏之别種其先有波斯匿王因以為號大食本波斯地隋大業中有波斯國人牧于俱紛摩地山有獸言曰山西三穴有利兵黑質而白文得之者王走視如言石文言當反乃詭衆裒亡命於恒曷水劫商旅保西鄙自王移黑石寶之國人往討皆大敗而還於是遂強】殺波斯王伊嗣侯伊嗣侯之子卑路斯奔吐火羅大食兵去吐火羅發兵立卑路斯為波斯王而還 閏月丙子以處月部置金滿州【其地近古輪臺屬北庭都護府】 丁丑夜大雨山水漲溢衝玄武門【此萬年宫之玄武門也唐離宫諸門蓋略倣宫城之制】宿衛士皆散走右領軍郎將薛仁貴曰【唐制自左右衛至左右金吾衛其屬各有左右中郎將府有中郎將及左郎將右郎將將即亮翻】安有宿衛之士天子有急而敢畏死乎乃登門桄大呼以警宫内【桄枯黄翻門前横木也呼火故翻】上遽出乘高俄而水入寢殿水溺衛士及麟遊居人【隋文帝於岐州之北置仁壽宫太業初置普潤縣義寧二年於宫獲白麟因分普潤於宫置麟遊縣仁壽宫唐改為九成宫又改曰萬年宫溺奴狄翻】死者三千餘人 壬辰新羅女王金真德卒詔立其弟春秋為新羅王 六月丙午恒州大水呼沱溢漂溺五千三百家【恒戶登翻沱徒河翻】 中書令柳奭以王皇后寵衰内不自安請解政事癸亥罷為吏部尚書 秋七月丁酉車駕至京師 戊戍上謂五品以上曰頃在先帝左右見五品以上論事或仗下面陳【唐制常朝諸衛皆立仗仗下宰執諫官奏事】或退上封事終日不絶【上時掌翻】豈今日獨無事邪何公等皆不言也 冬十月雇雍州四萬一千人築長安外郭三旬而畢【雇者以錢若物酬其功事不徒役其力也雍於同翻】癸丑雍州參軍薛景宣上封事言漢惠帝城長安尋晏駕【事見十二卷漢惠帝三年五年上時掌翻下同】今復城之【復扶又翻】必有大咎于志寧等以景宣言涉不順請誅之上曰景宣雖狂妄若因上封事得罪恐絶言路遂赦之 高麗遣其將安固將高麗靺鞨兵擊契丹【麗力知翻將即亮翻靺鞨音末曷契欺訖翻又音喫】松漠都督李窟哥禦之大敗高麗於新城【窟苦骨翻敗補邁翻】 是歲大稔洛州粟米斗兩錢半秔米斗十一錢【秔音庚稻之不黎者】 王皇后蕭淑妃與武昭儀更相譛訴【更工衡翻】上不信后淑妃之語獨信昭儀后不能曲事上左右母魏國夫人柳氏及舅中書令柳奭入見六宫又不為禮武昭儀伺后所不敬者【伺相吏翻】必傾心與相結所得賞賜分與之由是后及淑妃動静昭儀必知之皆以聞於上后寵雖衰然上未有意廢也會昭儀生女后憐而弄之后出昭儀潛扼殺之覆之以被【覆敷又翻】上至昭儀陽歡笑發被觀之女已死矣即驚啼問左右左右皆曰皇后適來此上大怒曰后殺吾女昭儀因泣數其罪【數所具翻】后無以自明上由是有廢立之志又畏大臣不從乃與昭儀幸太尉長孫無忌第酣飲極驩席上拜無忌寵姬子三人皆為朝散大夫【朝直遥翻散悉亶翻】仍載金寶繒錦十車以賜無忌上因從容言皇后無子以諷無忌【從才容翻】無忌對以佗語竟不順旨上及昭儀皆不悦而罷昭儀又令母楊氏詣無忌第屢有祈請無忌終不許禮部尚書許敬宗亦數勸無忌無忌厲色折之【上於無忌官及庶孽又有横賜意可知矣無忌欲格其非心則辭而不受可也為無忌得罪張本數所角翻折之舌翻】<br />
<br />
  六年春正月壬申朔上謁昭陵甲戌還宫 己丑嶲州道行軍總管曹繼叔破胡叢顯養車魯等蠻於斜山拔十餘城【胡叢劒山招討使所領五部落之一也顯養車魯亦各蠻種車魯新書作東魯嶲音髓】庚寅立皇子弘為代王賢為潞王 高麗與百濟靺鞨連兵侵新羅北境取三十三城新羅王春秋遣使求援【使疏吏翻】二月乙丑遣營州都督程名振左衛中郎將蘇定方發兵擊高麗【將即亮翻】 夏五月壬午名振等度遼水高麗見其兵少開門度貴端水逆戰【按舊書程名振傳貴端水當在新城西南少詩沼翻】名振等奮擊大破之殺獲千餘人焚其外郭及村落而還 癸未以右屯衛大將軍程知節為蔥山道行軍大總管【蔥山即蔥嶺】以討西突厥沙鉢羅可汗 壬辰以韓瑗為侍中【瑗于眷翻】來濟為中書令 六月武昭儀誣王后與其母魏國夫人柳氏為厭勝【厭於葉翻又一琰翻考異曰舊傳云后懼不自安密與母柳氏求巫祝厭勝事故廢今從實録】敕禁后母柳氏不得入宫秋七月戊寅貶吏部尚書柳奭為遂州刺史奭行至扶風【武德元年分岐山置準川縣取準水為名貞觀八年更名扶風屬岐州九域志縣在州東八十里】岐州長史于承素希旨奏奭漏洩禁中語復貶榮州刺史【榮州漢南安江陽之地隋為大牢縣屬資州武德元年分置榮州復扶又翻】唐因隋置後宫有貴妃淑妃德妃賢妃皆視一品上欲特置宸妃以武昭儀為之韓瑗來濟諫以為故事無之乃止 【考異曰唐歷在此年四月今據實録四月韓瑗來濟未為侍中中書令唐歷又云瑗濟諫帝不從按立武后詔書猶云昭儀武氏然則未嘗為宸妃也今從會要】中書舍人饒陽李義府為長孫無忌所惡【惡烏路翻】左遷壁州司馬【武德八年析巴州始寧縣地置壁州】敕未至門下義府密知之問計於中書舍人幽州王德儉德儉曰上欲立武昭儀為后猶豫未决者直恐宰臣異議耳君能建策立之則轉禍為福矣義府然之是日代德儉直宿叩閤上表請廢皇后王氏立武昭儀以厭兆庶之心【厭於葉翻】上悦召見與語賜珠一斗留居舊職昭儀又密遣使勞勉之【使疏吏翻勞力到翻】尋超拜中書侍郎 【考異曰舊傳云高宗將立武后義府密申叶贊擢拜中書侍郎同中書門下三品監修國史賜爵廣平縣男新書本紀年表皆云是歲七月義府為中書侍郎參知政事實録但云超拜中書侍郎宰輔圖十一月自中書侍郎參知政事今從之】於是衛尉卿許敬宗御史大夫崔義玄中丞袁公瑜皆潛布腹心於武昭儀矣 乙酉以侍中崔敦禮為中書令八月尚藥奉御蔣孝璋員外特置仍同正員【尚藥局奉御員二人掌合和御藥及診候方胍之事】員外同正自孝璋始 長安令裴行儉聞將立武昭儀為后以國家之禮必自此始與長孫無忌褚遂良私議其事袁公瑜聞之以告昭儀母楊氏行儉坐左遷西州都督府長史【唐制長安萬年河南洛陽太原晉陽六縣謂之京縣京縣令正五品上西州中都督府中都督府長史亦正五品上但從輦轂下出佐邊州故謂左遷】行儉仁基之子也【裴仁基隋將歸李密為王世充所殺】 九月戊辰以許敬宗為禮部尚書上一日退朝【朝直遥翻】召長孫無忌李勣于志寧褚遂良入内殿遂良曰今日之召多為中宫上意既决逆之必死太尉元舅司空功臣不可使上有殺元舅及功臣之名遂良起於草茅無汗馬之勞致位至此且受顧託不以死爭之何以下見先帝勣稱疾不入無忌等至内殿上顧謂無忌曰皇后無子武昭儀有子今欲立昭儀為后何如遂良對曰皇后名家先帝為陛下所娶【為于偽翻】先帝臨崩執陛下手謂臣曰朕佳兒佳婦今以付卿此陛下所聞言猶在耳皇后未聞有過豈可輕廢臣不敢曲從陛下上違先帝之命上不悦而罷明日又言之遂良曰陛下必欲易皇后伏請妙擇天下令族何必武氏武氏經事先帝衆所具知天下耳目安可蔽也萬代之後謂陛下為如何願留三思臣今忤陛下罪當死【忤五故翻】因置笏於殿階解巾叩頭流血曰還陛下笏乞放歸田里上大怒命引出昭儀在簾中大言曰何不撲殺此獠無忌曰遂良受先朝顧命有罪不可加刑【撲弼角翻又普木翻獠魯皓翻朝直遥翻顧音古 考異曰唐歷云無忌等將入遂良曰今者多為中宫事遂良欲諫何如無忌曰公但極言無忌接公及入上三顧無忌曰莫大之罪無過絶嗣皇后無子今欲廢之立武士彠女何如無忌曰自貞觀二十三年後先朝託付遂良望陛下問其可否按如此則是無忌賣遂良也今不取】于志寧不敢言韓瑗因間奏事【間古莧翻】涕泣極諫上不納明日又諫悲不自勝【勝音升】上命引出瑗又上疏諫曰匹夫匹婦猶相選擇况天子乎皇后母儀萬國善惡由之故嫫母輔佐黄帝【漢書古今人表幠母黄帝妃生倉林師古曰幠音謨即嫫母也何承天纂文曰嫫母醜人也黄帝愛幸之嫫音謨】妲己傾覆殷王【妲己有蘇氏之美女紂愛之唯妲己之言是從卒以亡殷妲當割翻】詩云赫赫宗周褒姒滅之【詩小雅正月之辭韓瑗之意謂嫫毋以醜而佐黄帝有天下妲己褒姒以美豔而亡殷周女在德不在色也】每覽前古常興歎息不謂今日塵黷聖代作而不法後嗣何觀【左傳曹劌諫魯莊公之辭】願陛下詳之無為後人所笑使臣有以益國葅醢之戮臣之分也【分扶問翻】昔吳王不用子胥之言而麋鹿遊於姑蘇【漢伍被曰昔子胥諫吳王吳王不用迺曰臣今見麋鹿遊於姑蘇之臺也師古曰姑蘇因山為臺名西南去吳國二十五里范成大吳郡志曰姑蘇臺在姑蘇山舊圖經云在吳縣西三十里續圖經云三十五里史記正義曰在吳縣西南三十里横山西北麓姑蘇山上】臣恐海内失望荆棘生於闕庭宗廟不血食期有日矣來濟上表諫曰王者立后上法乾坤必擇禮教名家幽閑令淑副四海之望稱神祗之意是故周文造舟以迎太姒而興關雎之化百姓蒙祚【太姒文王之妃也詩云文定厥祥親迎于渭造舟為梁不顯其光太姒佐文王以興王業故關雎美其德稱尺證翻造七到翻】孝成縱欲以婢為后使皇統亡絶社稷傾淪【事見漢成帝紀】有周之隆既如彼大漢之禍又如此惟陛下詳察上皆不納【褚遂良韓瑗來濟言皆痛切此時去貞觀未遠士大夫敢言之氣未衰自三人者得罪在朝之臣唯承武后風旨安能言人所難言哉】它日李勣入見【見賢遍翻】上問之曰朕欲立武昭儀為后遂良固執以為不可遂良既顧命大臣事當且己乎對曰此陛下家事何必更問外人【自李勣有是言李林甫襲取之以成明皇殺三子之禍德宗舒王之議亦袓此說微李泌東宫殆哉】上意遂决許敬宗宣言於朝曰田舍翁多收十斛麥尚欲易婦况天子欲立后何豫諸人事而妄生異議乎【以田舍翁况天子許敬宗之事君不敬莫大乎是朝直遥翻】昭儀令左右以聞庚午貶遂良為潭州都督【潭州在京師南三千四百四十五里】<br />
<br />
  資治通鑑卷一百九十九<br />
<br />
<史部,編年類,資治通鑑>  <br>
   </div> 

<script src="/search/ajaxskft.js"> </script>
 <div class="clear"></div>
<br>
<br>
 <!-- a.d-->

 <!--
<div class="info_share">
</div> 
-->
 <!--info_share--></div>   <!-- end info_content-->
  </div> <!-- end l-->

<div class="r">   <!--r-->



<div class="sidebar"  style="margin-bottom:2px;">

 
<div class="sidebar_title">工具类大全</div>
<div class="sidebar_info">
<strong><a href="http://www.guoxuedashi.com/lsditu/" target="_blank">历史地图</a></strong>  
<a href="http://www.880114.com/" target="_blank">英语宝典</a>  
<a href="http://www.guoxuedashi.com/13jing/" target="_blank">十三经检索</a> 
<br><strong><a href="http://www.guoxuedashi.com/gjtsjc/" target="_blank">古今图书集成</a></strong> 
<a href="http://www.guoxuedashi.com/duilian/" target="_blank">对联大全</a> <strong><a href="http://www.guoxuedashi.com/xiangxingzi/" target="_blank">象形文字典</a></strong> 

<br><a href="http://www.guoxuedashi.com/zixing/yanbian/">字形演变</a>  <strong><a href="http://www.guoxuemi.com/hafo/" target="_blank">哈佛燕京中文善本特藏</a></strong>
<br><strong><a href="http://www.guoxuedashi.com/csfz/" target="_blank">丛书&方志检索器</a></strong> <a href="http://www.guoxuedashi.com/yqjyy/" target="_blank">一切经音义</a>  

<br><strong><a href="http://www.guoxuedashi.com/jiapu/" target="_blank">家谱族谱查询</a></strong>  <strong><a href="http://shufa.guoxuedashi.com/sfzitie/" target="_blank">书法字帖欣赏</a></strong> 
<br>

</div>
</div>


<div class="sidebar" style="margin-bottom:0px;">

<font style="font-size:22px;line-height:32px">QQ交流群9:489193090</font>


<div class="sidebar_title">手机APP 扫描或点击</div>
<div class="sidebar_info">
<table>
<tr>
	<td width=160><a href="http://m.guoxuedashi.com/app/" target="_blank"><img src="/img/gxds-sj.png" width="140"  border="0" alt="国学大师手机版"></a></td>
	<td>
<a href="http://www.guoxuedashi.com/download/" target="_blank">app软件下载专区</a><br>
<a href="http://www.guoxuedashi.com/download/gxds.php" target="_blank">《国学大师》下载</a><br>
<a href="http://www.guoxuedashi.com/download/kxzd.php" target="_blank">《汉字宝典》下载</a><br>
<a href="http://www.guoxuedashi.com/download/scqbd.php" target="_blank">《诗词曲宝典》下载</a><br>
<a href="http://www.guoxuedashi.com/SiKuQuanShu/skqs.php" target="_blank">《四库全书》下载</a><br>
</td>
</tr>
</table>

</div>
</div>


<div class="sidebar2">
<center>


</center>
</div>

<div class="sidebar"  style="margin-bottom:2px;">
<div class="sidebar_title">网站使用教程</div>
<div class="sidebar_info">
<a href="http://www.guoxuedashi.com/help/gjsearch.php" target="_blank">如何在国学大师网下载古籍?</a><br>
<a href="http://www.guoxuedashi.com/zidian/bujian/bjjc.php" target="_blank">如何使用部件查字法快速查字?</a><br>
<a href="http://www.guoxuedashi.com/search/sjc.php" target="_blank">如何在指定的书籍中全文检索?</a><br>
<a href="http://www.guoxuedashi.com/search/skjc.php" target="_blank">如何找到一句话在《四库全书》哪一页?</a><br>
</div>
</div>


<div class="sidebar">
<div class="sidebar_title">热门书籍</div>
<div class="sidebar_info">
<a href="/so.php?sokey=%E8%B5%84%E6%B2%BB%E9%80%9A%E9%89%B4&kt=1">资治通鉴</a> <a href="/24shi/"><strong>二十四史</strong></a>&nbsp; <a href="/a2694/">野史</a>&nbsp; <a href="/SiKuQuanShu/"><strong>四库全书</strong></a>&nbsp;<a href="http://www.guoxuedashi.com/SiKuQuanShu/fanti/">繁体</a>
<br><a href="/so.php?sokey=%E7%BA%A2%E6%A5%BC%E6%A2%A6&kt=1">红楼梦</a> <a href="/a/1858x/">三国演义</a> <a href="/a/1038k/">水浒传</a> <a href="/a/1046t/">西游记</a> <a href="/a/1914o/">封神演义</a>
<br>
<a href="http://www.guoxuedashi.com/so.php?sokeygx=%E4%B8%87%E6%9C%89%E6%96%87%E5%BA%93&submit=&kt=1">万有文库</a> <a href="/a/780t/">古文观止</a> <a href="/a/1024l/">文心雕龙</a> <a href="/a/1704n/">全唐诗</a> <a href="/a/1705h/">全宋词</a>
<br><a href="http://www.guoxuedashi.com/so.php?sokeygx=%E7%99%BE%E8%A1%B2%E6%9C%AC%E4%BA%8C%E5%8D%81%E5%9B%9B%E5%8F%B2&submit=&kt=1"><strong>百衲本二十四史</strong></a>  <a href="http://www.guoxuedashi.com/so.php?sokeygx=%E5%8F%A4%E4%BB%8A%E5%9B%BE%E4%B9%A6%E9%9B%86%E6%88%90&submit=&kt=1"><strong>古今图书集成</strong></a>
<br>

<a href="http://www.guoxuedashi.com/so.php?sokeygx=%E4%B8%9B%E4%B9%A6%E9%9B%86%E6%88%90&submit=&kt=1">丛书集成</a> 
<a href="http://www.guoxuedashi.com/so.php?sokeygx=%E5%9B%9B%E9%83%A8%E4%B8%9B%E5%88%8A&submit=&kt=1"><strong>四部丛刊</strong></a>  
<a href="http://www.guoxuedashi.com/so.php?sokeygx=%E8%AF%B4%E6%96%87%E8%A7%A3%E5%AD%97&submit=&kt=1">說文解字</a> <a href="http://www.guoxuedashi.com/so.php?sokeygx=%E5%85%A8%E4%B8%8A%E5%8F%A4&submit=&kt=1">三国六朝文</a>
<br><a href="http://www.guoxuedashi.com/so.php?sokeytm=%E6%97%A5%E6%9C%AC%E5%86%85%E9%98%81%E6%96%87%E5%BA%93&submit=&kt=1"><strong>日本内阁文库</strong></a> <a href="http://www.guoxuedashi.com/so.php?sokeytm=%E5%9B%BD%E5%9B%BE%E6%96%B9%E5%BF%97%E5%90%88%E9%9B%86&ka=100&submit=">国图方志合集</a> <a href="http://www.guoxuedashi.com/so.php?sokeytm=%E5%90%84%E5%9C%B0%E6%96%B9%E5%BF%97&submit=&kt=1"><strong>各地方志</strong></a>

</div>
</div>


<div class="sidebar2">
<center>

</center>
</div>
<div class="sidebar greenbar">
<div class="sidebar_title green">四库全书</div>
<div class="sidebar_info">

《四库全书》是中国古代最大的丛书,编撰于乾隆年间,由纪昀等360多位高官、学者编撰,3800多人抄写,费时十三年编成。丛书分经、史、子、集四部,故名四库。共有3500多种书,7.9万卷,3.6万册,约8亿字,基本上囊括了古代所有图书,故称“全书”。<a href="http://www.guoxuedashi.com/SiKuQuanShu/">详细>>
</a>

</div> 
</div>

</div>  <!--end r-->

</div>
<!-- 内容区END --> 

<!-- 页脚开始 -->
<div class="shh">

</div>

<div class="w1180" style="margin-top:8px;">
<center><script src="http://www.guoxuedashi.com/img/plus.php?id=3"></script></center>
</div>
<div class="w1180 foot">
<a href="/b/thanks.php">特别致谢</a> | <a href="javascript:window.external.AddFavorite(document.location.href,document.title);">收藏本站</a> | <a href="#">欢迎投稿</a> | <a href="http://www.guoxuedashi.com/forum/">意见建议</a> | <a href="http://www.guoxuemi.com/">国学迷</a> | <a href="http://www.shuowen.net/">说文网</a><script language="javascript" type="text/javascript" src="https://js.users.51.la/17753172.js"></script><br />
  Copyright &copy; 国学大师 古典图书集成 All Rights Reserved.<br>
  
  <span style="font-size:14px">免责声明:本站非营利性站点,以方便网友为主,仅供学习研究。<br>内容由热心网友提供和网上收集,不保留版权。若侵犯了您的权益,来信即刪。scp168@qq.com</span>
  <br />
ICP证:<a href="http://www.beian.miit.gov.cn/" target="_blank">鲁ICP备19060063号</a></div>
<!-- 页脚END --> 
<script src="http://www.guoxuedashi.com/img/plus.php?id=22"></script>
<script src="http://www.guoxuedashi.com/img/tongji.js"></script>

</body>
</html>
