<!DOCTYPE html PUBLIC "-//W3C//DTD XHTML 1.0 Transitional//EN" "http://www.w3.org/TR/xhtml1/DTD/xhtml1-transitional.dtd">
<html xmlns="http://www.w3.org/1999/xhtml">
<head>
<meta http-equiv="Content-Type" content="text/html; charset=utf-8" />
<meta http-equiv="X-UA-Compatible" content="IE=Edge,chrome=1">
<title>資治通鑒_156-資治通鑑卷一百五十五_156-資治通鑑卷一百五十五</title>
<meta name="Keywords" content="資治通鑒_156-資治通鑑卷一百五十五_156-資治通鑑卷一百五十五">
<meta name="Description" content="資治通鑒_156-資治通鑑卷一百五十五_156-資治通鑑卷一百五十五">
<meta http-equiv="Cache-Control" content="no-transform" />
<meta http-equiv="Cache-Control" content="no-siteapp" />
<link href="/img/style.css" rel="stylesheet" type="text/css" />
<script src="/img/m.js?2020"></script> 
</head>
<body>
 <div class="ClassNavi">
<a  href="/24shi/">二十四史</a> | <a href="/SiKuQuanShu/">四库全书</a> | <a href="http://www.guoxuedashi.com/gjtsjc/"><font  color="#FF0000">古今图书集成</font></a> | <a href="/renwu/">历史人物</a> | <a href="/ShuoWenJieZi/"><font  color="#FF0000">说文解字</a></font> | <a href="/chengyu/">成语词典</a> | <a  target="_blank"  href="http://www.guoxuedashi.com/jgwhj/"><font  color="#FF0000">甲骨文合集</font></a> | <a href="/yzjwjc/"><font  color="#FF0000">殷周金文集成</font></a> | <a href="/xiangxingzi/"><font color="#0000FF">象形字典</font></a> | <a href="/13jing/"><font  color="#FF0000">十三经索引</font></a> | <a href="/zixing/"><font  color="#FF0000">字体转换器</font></a> | <a href="/zidian/xz/"><font color="#0000FF">篆书识别</font></a> | <a href="/jinfanyi/">近义反义词</a> | <a href="/duilian/">对联大全</a> | <a href="/jiapu/"><font  color="#0000FF">家谱族谱查询</font></a> | <a href="http://www.guoxuemi.com/hafo/" target="_blank" ><font color="#FF0000">哈佛古籍</font></a> 
</div>

 <!-- 头部导航开始 -->
<div class="w1180 head clearfix">
  <div class="head_logo l"><a title="国学大师官网" href="http://www.guoxuedashi.com" target="_blank"></a></div>
  <div class="head_sr l">
  <div id="head1">
  
  <a href="http://www.guoxuedashi.com/zidian/bujian/" target="_blank" ><img src="http://www.guoxuedashi.com/img/top1.gif" width="88" height="60" border="0" title="部件查字,支持20万汉字"></a>


<a href="http://www.guoxuedashi.com/help/yingpan.php" target="_blank"><img src="http://www.guoxuedashi.com/img/top230.gif" width="600" height="62" border="0" ></a>


  </div>
  <div id="head3"><a href="javascript:" onClick="javascript:window.external.AddFavorite(window.location.href,document.title);">添加收藏</a>
  <br><a href="/help/setie.php">搜索引擎</a>
  <br><a href="/help/zanzhu.php">赞助本站</a></div>
  <div id="head2">
 <a href="http://www.guoxuemi.com/" target="_blank"><img src="http://www.guoxuedashi.com/img/guoxuemi.gif" width="95" height="62" border="0" style="margin-left:2px;" title="国学迷"></a>
  

  </div>
</div>
  <div class="clear"></div>
  <div class="head_nav">
  <p><a href="/">首页</a> | <a href="/ShuKu/">国学书库</a> | <a href="/guji/">影印古籍</a> | <a href="/shici/">诗词宝典</a> | <a   href="/SiKuQuanShu/gxjx.php">精选</a> <b>|</b> <a href="/zidian/">汉语字典</a> | <a href="/hydcd/">汉语词典</a> | <a href="http://www.guoxuedashi.com/zidian/bujian/"><font  color="#CC0066">部件查字</font></a> | <a href="http://www.sfds.cn/"><font  color="#CC0066">书法大师</font></a> | <a href="/jgwhj/">甲骨文</a> <b>|</b> <a href="/b/4/"><font  color="#CC0066">解密</font></a> | <a href="/renwu/">历史人物</a> | <a href="/diangu/">历史典故</a> | <a href="/xingshi/">姓氏</a> | <a href="/minzu/">民族</a> <b>|</b> <a href="/mz/"><font  color="#CC0066">世界名著</font></a> | <a href="/download/">软件下载</a>
</p>
<p><a href="/b/"><font  color="#CC0066">历史</font></a> | <a href="http://skqs.guoxuedashi.com/" target="_blank">四库全书</a> |  <a href="http://www.guoxuedashi.com/search/" target="_blank"><font  color="#CC0066">全文检索</font></a> | <a href="http://www.guoxuedashi.com/shumu/">古籍书目</a> | <a   href="/24shi/">正史</a> <b>|</b> <a href="/chengyu/">成语词典</a> | <a href="/kangxi/" title="康熙字典">康熙字典</a> | <a href="/ShuoWenJieZi/">说文解字</a> | <a href="/zixing/yanbian/">字形演变</a> | <a href="/yzjwjc/">金 文</a> <b>|</b>  <a href="/shijian/nian-hao/">年号</a> | <a href="/diming/">历史地名</a> | <a href="/shijian/">历史事件</a> | <a href="/guanzhi/">官职</a> | <a href="/lishi/">知识</a> <b>|</b> <a href="/zhongyi/">中医中药</a> | <a href="http://www.guoxuedashi.com/forum/">留言反馈</a>
</p>
  </div>
</div>
<!-- 头部导航END --> 
<!-- 内容区开始 --> 
<div class="w1180 clearfix">
  <div class="info l">
   
<div class="clearfix" style="background:#f5faff;">
<script src='http://www.guoxuedashi.com/img/headersou.js'></script>

</div>
  <div class="info_tree"><a href="http://www.guoxuedashi.com">首页</a> > <a href="/SiKuQuanShu/fanti/">四库全书</a>
 > <h1>资治通鉴</h1> <!--         下载:【右键另存为】即可 --></div>
  <div class="info_content zj clearfix">
  
<div class="info_txt clearfix" id="show">
<center style="font-size:24px;">156-資治通鑑卷一百五十五</center>
    資治通鑑卷一百五十五 宋 司馬光 撰<br />
<br />
  胡三省 音註<br />
<br />
  梁紀十一【起重光大淵獻盡玄黓困敦凡二年】<br />
<br />
  高祖武皇帝十一<br />
<br />
  中大通三年春正月辛巳上祀南郊大赦 魏尚書右僕射鄭先護聞洛陽不守士衆逃散遂來奔【去年魏敬宗遣鄭先護討東郡】丙申以先護為征北大將軍二月辛丑上祀明堂魏自敬宗被囚宫室空近百日【被皮義翻近其靳翻去年十二月壬寅爾朱兆渡河囚敬宗甲寅遷晉陽是月己巳節閔帝即位始入宫】爾朱世隆鎮洛陽商旅流通盜賊不作世隆兄弟密議以長廣王疎遠又無人望欲更立近親【更工衡翻】儀同三司廣陵王恭羽之子也【廣陵王羽魏孝文帝之弟】好學有志度正光中領給事黃門侍郎以元乂擅權託瘖病居龍華佛寺【好呼到翻瘖於今翻華讀曰花】無所交通永安末有白敬宗言王陽瘖將有異志 【考異曰伽藍記云莊帝疑恭姦詐夜遣人盜掠衣物拔刀劒欲殺之恭張口以手拈舌竟乃不言莊帝信其真放令歸第今從魏書】恭懼逃於上洛山【上洛山在洛州上洛郡上洛縣界】洛州刺史執送之【魏洛州刺史治上洛】繫治久之以無狀獲免【無狀者無反狀也治直之翻】關西大行臺郎中薛孝通說爾朱天光曰廣陵王高祖猶子【禮曰兄弟之子猶子也廣陵王羽高祖之弟恭則猶子也高祖孝文廟號說式芮翻】夙有令望沈晦不言【沈持林翻】多歷年所若奉以為主必天人允叶天光與世隆等謀之疑其實瘖使爾朱彦伯潛往敦諭且脅之恭乃曰天何言哉【用論語孔子之言】世隆等大喜孝通聰之子也【薛聰見一百四十卷齊明帝建武元年】己巳長廣王至邙山南世隆等為之作禪文【為於偽翻】使泰山太守遼西竇瑗執鞭獨入【守式又翻瑗於脊翻】啓長廣王曰天人之望皆在廣陵願行堯舜之事遂署禪文廣陵王奉表三讓然後即位【帝諱恭字脩業廣陵惠王羽之子也】大赦改元普泰黃門侍郎邢子才為赦文叙敬宗枉殺太原王榮之狀節閔帝曰永安手翦彊臣非為失德直以天未厭亂故逢成濟之禍耳【成濟弑高貴鄉公事見七十七卷魏元帝景元元年】因顧左右取筆自作赦文直言門下【魏晉以來出命皆由門下省故其端必曰勑門下】朕以寡德運屬樂推思與億兆同兹大慶肆眚之科一依常式【屬之欲翻樂音洛春秋莊二十二年肆大眚杜預注曰赦有罪也易稱赦過宥罪書稱眚災肆赦傳稱肆眚圍鄭皆放赦罪人蕩滌衆故以新其心】帝閉口八年至是乃言中外欣然以為明主望至太平【至一作致】庚午詔以三皇稱皇五帝稱帝三代稱王蓋遞為冲挹【謂皇降稱帝帝降稱王蓋遞為謙下也】自秦以來競稱皇帝予今但稱帝亦已褒矣加爾朱世隆儀同三司贈爾朱榮相國晉王加九錫世隆使百官議榮配饗司直劉季明曰【社佑通典曰後魏永安三年高道穆奏廷尉置司直十人位在正監上不署曹事唯覆理御史檢校事】若配世宗於時無功【宣武帝廟號世宗】若配孝明親害其母【謂殺胡后也】若配莊帝為臣不終【孝武帝始改謚敬宗曰莊帝此時當稱為懷帝】以此論之無所可配世隆怒曰汝應死季明曰下官既為議首依禮而言不合聖心翦戮唯命世隆亦不之罪以榮配高祖廟庭又為榮立廟於首陽山【為於偽翻】因周公舊廟而為之以為榮功可比周公廟成尋為火所焚【因周公舊廟而祀爾朱榮周公豈以奪余饗為嫌哉天人之心固不許也】爾朱兆以不預廢立之謀大怒欲攻世隆世隆使爾朱彦伯往諭之乃止初敬宗使安東將軍史仵龍平北將軍陽文義各領兵三千守太行嶺侍中源子恭鎮河内及爾朱兆南向仵龍文義帥衆先降由是子恭之軍望風亦潰兆遂乘勝直入洛陽【事見上卷上年仵疑古翻行戶剛翻帥讀曰率降下江翻】至是爾朱世隆論仵龍文義之功各封千戶侯魏主曰仵龍文義於王有功於國無勲竟不許爾朱仲遠鎮滑臺表用其下都督為西兖州刺史【魏收志西兖州領沛濟隂郡】先用後表詔荅曰已能近補何勞遠聞爾朱天光之滅万俟醜奴也【事見上卷上年万莫北翻俟渠夷翻】始獲波斯所獻師子送洛陽【波斯獻師子見一百五十二卷大通二年】及節閔帝即位詔曰禽獸囚之則違其性命送歸本國使者以波斯道遠不可達於路殺之而返有司劾違旨帝曰豈可以獸而罪人遂赦之【史言節閔帝賢明而不終者制於彊臣也使疏吏翻劾戶槩翻又戶得翻】 魏鎮遠將軍清河崔祖螭等聚青州七郡之衆圍東陽【青州領齊北海樂安勃海高陽河間樂陵七郡治東陽】旬日之間衆十餘萬刺史東莱王貴平帥城民固守【帥音率】使太傅諮議參軍崔光伯出城慰勞其兄光韶曰城民陵縱日久【盖言東陽之民挾州家之勢陵暴屬郡恣縱之日久矣勞力到翻】衆怒甚盛非慰諭所能解家弟往必不全貴平彊之【彊其兩翻】既出外人射殺之【射而亦翻】 幽安營幷四州行臺劉靈助自謂方術可以動人又推算知爾朱氏將衰乃起兵自稱燕王開府儀同三司大行臺聲言為敬宗復讎【燕因肩翻為於偽翻】且妄述圖䜟云劉氏當王【䜟七譛翻】由是幽瀛滄冀之民多從之【魏熙平二年分瀛冀二州置滄州治饒安城領浮陽樂陵安德三郡】從之者夜舉火為號不舉火者諸村共屠之引兵南至博陵之安國城【魏收志博陵郡安國縣有安國城北平蒲隂縣亦有安國城故稱博陵以别之】爾朱兆遣監軍孫白鷂至冀州【監工咸翻鷂弋照翻】託言調民馬【調徒釣翻】欲俟高乾兄弟送馬而收之乾等知之與前河内太守封隆之等合謀潛部勒壯士襲據信都殺白鷂 【考異曰北史作白鷄今從北齊書】執刺史元嶷【嶷魚力翻】乾等欲推其父翼行州事翼曰和集鄉里我不如封皮【皮封隆之小字也】乃奉隆之行州事為敬宗舉哀【為于偽翻】將士皆縞素【將即亮翻下同縞古勞翻】升壇誓衆移檄州郡共討爾朱氏仍受劉靈助節度隆之磨奴之族孫也【封磨奴見一百一十九卷宋高祖永初元年】殷州刺史爾朱羽生將五千人襲信都高敖曹不暇擐甲將十餘騎馳【擐音宦騎奇計翻】擊之乾在城中繩下五百人追救未及敖曹已交兵羽生敗走敖曹馬矟絶世【矟色角翻】左右無不一當百時人比之項籍高歡屯壺關大王山【魏收地形志上黨郡屯留縣有鳳凰山一名大王山按魏太平真君九年二月詔於壺關東北大王山累石為三封又斬其北鳳凰山南足以斷之以其有王氣也後高歡果屯兵於其地】六旬乃引兵東出聲言討信都信都人皆懼高乾曰吾聞高晉州雄略蓋世【爾朱榮用歡為晉州刺史故稱之】其志不居人下且爾朱無道弑君虐民正是英雄立功之會今日之來必有深謀吾當輕馬迎之密參意旨【參也】諸君勿懼也乃將十餘騎與封隆之子子繪潛謁歡於滏口說歡曰爾朱酷逆痛結人神凡曰有知孰不思奮明公威德素著天下傾心若兵以義立則屈彊之徒不足為明公敵矣【屈與倔同其勿翻彊其兩翻屈彊之徒指爾朱氏之黨也】鄙州雖小戶口不下十萬穀秸之税足濟軍資【秸工八翻藁也】願公熟思其計乾辭氣慷慨歡大悦與之同帳寢初河南太守趙郡李顯甫喜豪俠【喜許既翻】集諸李數千家於殷州西山方五六十里居之【殷州西山廣阿之西山也】顯甫卒子元忠繼之家素富多出貸求利元忠悉焚劵免責鄉人甚敬之【契約也即古所謂劵也免責不責其償也】時盜賊蠭起清河有五百人西戍還經趙郡以路梗共投元忠【梗塞也】元忠遣奴為導曰若逢賊但道李元忠遣 言賊皆舍避【舍讀曰捨】及葛榮起元忠帥宗黨作壘以自保【帥讀曰率】坐大槲樹下【槲胡谷翻】前後斬違命者凡三百人賊至元忠輒擊却之葛榮曰我自中山至此連為趙李所破【李氏趙郡之大姓時號為趙李】何以能成大事乃悉衆攻圍執元忠以隨軍賊平就拜南趙郡太守【此言爾朱榮平葛榮時事魏太和之十一年分趙郡之平鄉柏人中丘鉅鹿之南蠻鉅鹿廣阿為南鉅鹿郡後改為南趙郡屬殷州】好酒無政蹟【好呼到翻】及爾朱兆弑敬宗元忠棄官歸謀舉兵討之會高歡東出元忠乘露車載素箏濁酒以奉迎歡聞其酒客未即見之元忠下車獨坐酌酒擘脯食之謂門者曰本言公招延儁傑今聞國士到門不吐哺輟洗其人可知【以周公漢祖之事責歡也洗先典翻】還吾刺勿通也門者以告歡遽見之引入觴再行元忠車上取箏皷之長歌慷慨歌闋【闋苦穴翻歌終也】謂歡曰天下形勢可見明公猶事爾朱邪歡曰富貴皆因彼所致安敢不盡節元忠曰非英雄也高乾邕兄弟來未【高乾字乾邕】時乾己見歡歡紿之曰從叔輩麤何肯來【歡與乾兄弟同出於勃海故稱從叔紿待亥翻從才用翻】元忠曰雖麤竝解事【解胡買翻曉也】歡曰趙郡醉矣使人扶出元忠不肯起孫騰進曰此君天遣來不可違也歡乃復留與語【復扶又翻】元忠慷慨流涕歡亦悲不自勝【勝音升】元忠因進策曰殷州小無糧仗不足以濟大事若向冀州高乾邕兄弟必為明公主人【魏冀州治信都高乾邕兄弟據之故云然】殷州便以賜委冀殷既合滄瀛幽定自然弭服唯劉誕黠胡或當乖拒【劉誕亦契胡種也時為相州刺史鎮鄴黠下八翻】然非明公之敵歡急握元忠手而謝焉歡至山東約勒士卒絲毫之物不聽侵犯每過麥地歡輒步牽馬遠近聞之皆稱高儀同將兵整肅益歸心焉【爾朱氏加歡儀同三司故當時以稱之史言高歡能收衆心以傾爾朱將即亮翻】歡求糧於相州刺史劉誕誕不與【相息亮翻】有車營租米【車營北齊紀作軍營】歡掠取之進至信都封隆之高乾等開門納之高敖曹時在外略地聞之以乾為婦人遺以布裙【裙渠云翻婦人下裳也遺於季翻】歡使世子澄以子孫禮見之敖曹乃與俱來【敖曹以歡叙羣從子侄之禮乃來孰謂其麤也】 癸酉魏封長廣王曄為東海王以青州刺史魯郡王肅為太師淮陽王欣為太傅爾朱世隆為太保長孫稚為太尉趙郡王諶為司空【諶氏壬翻】徐州刺史爾朱仲遠雍州刺史爾朱天光竝為大將軍幷州刺史爾朱兆為天柱大將軍賜高歡爵渤海王徵使入朝【高歡之先本渤海人爾朱氏爵之為本郡王欲以誘致之朝直遙翻】長孫稚固辭太尉【世衷難作故辭】乃以為驃騎大將軍開府儀同三司【驃匹妙翻騎奇計翻】爾朱兆辭天柱曰此叔父所終之官我何敢受固辭不拜尋加都督十州諸軍事【十州南盡汾晉北極雲朔】世襲幷州刺史高歡辭不就徵爾朱仲遠徙鎮大梁復加兗州刺史【大梁兖州統内故加兖州復扶又翻下無復不復同】爾朱世隆之初為僕射也【大通二年魏爾朱榮入洛以世隆為尚書僕射】畏爾朱榮之威嚴深自刻厲留心几案【案亦几屬應文書皆陳於几案而省覽之留心几案謂留心於尚書省文書也又案據也凡官文書留以為據者亦謂之案】應接賓客有開敏之名及榮死無所顧憚為尚書令家居視事坐符臺省事無大小不先白世隆有司不敢行使尚書郎宋遊道邢昕在其聽事東西别坐受納辭訟稱命施行【稱命者稱世隆之命也昕許斤翻聽讀曰廳】公為貪淫生殺自恣又欲收軍士之意汛加階級皆為將軍無復員限自是勳賞之官大致猥濫人不復貴【猥雜也】是時天光專制關右兆奄有幷汾仲遠擅命徐兖世隆居中用事競為貪暴而仲遠尤甚所部富室大族多誣以謀反籍沒其婦女財物入私家【私家謂仲遠私家也】投其男子於河如是者不可勝數【勝音升】自滎陽已東租税悉入其軍不送洛陽東南州郡自牧守以下至士民畏仲遠如豺狼由是四方之人皆惡爾朱氏而憚其彊莫敢違也【為爾朱氏敗張本守式又翻惡烏路翻】 己丑魏以涇州刺史賀拔岳為岐州刺史渭州刺史侯莫陳悦為秦州刺史竝加儀同三司【涇渭荒殘而秦岐差完故以内遷為進律】 魏使大都督侯淵驃騎大將軍代人叱列延慶討劉靈助至固城【叱列虜複姓固城當在中山城東北安國城西南】淵畏其衆欲引兵西入據關拒險以待其變延慶曰靈助庸人假妖術以惑衆【妖於驕翻】大兵一臨彼皆恃其符厭【厭一恊翻謂劉靈助書為符勑以厭勝也】豈肯戮力致死與吾兵爭勝負哉不如出營城外詐言西歸靈助聞之必自寛縱然後潜軍擊之往則成擒矣淵從之出頓城西聲云欲還丙申簡精騎一千夜直抵靈助壘靈助戰敗斬之傳首洛陽初靈助起兵自占勝負曰三月之末我必入定州爾朱氏不久當滅及靈助首函入定州果以是月之末【史言用兵不可徒信占驗而無方略】 夏四月乙巳昭明太子統卒太子自加元服【天監十四年太子加元服】上即使省録朝政【省悉井翻朝直遙翻】百司進事填委於前太子辯析詐謬秋毫必睹但令改正不加案劾【劾戶槩翻又戶得翻】平斷灋獄多所全宥寛和容衆喜慍不形於色好讀書屬文【斷丁亂翻好呼報翻屬之欲翻】引接才俊賞愛無倦出宫二十餘年【言自禁中出居東宫也】不畜聲樂每霖雨積雪遣左右周行閭巷視貧者賑之【行下孟翻賑之忍翻此所謂好行小惠也】天性孝謹在東宫雖燕居坐起恒西向【必西向者不敢背上臺也恒戶登翻謹居忍翻】或宿被召當入【隔夜為宿被皮義翻】危坐達旦及寢疾恐貽帝憂勑參問輒自力手書【言帝出勑候問太子輒力疾手書自為奏荅】及卒朝野惋愕建康男女奔走宫門號泣道路【卒子卹翻朝直遙翻惋烏貫翻愕五各翻奔甫門翻走音奏號戶到翻】癸丑魏以高歡為大都督東道大行臺冀州刺史【東道謂太行恒山以東也】又以安定王爾朱智虎為肆州刺史 魏爾朱天光出夏州遣將討宿勤明達癸亥擒明達送洛陽斬之【爾朱天光既擒万俟醜奴又擒宿勤明達河隴平矣不知乃以為宇文泰之資也夏戶雅翻將即亮翻】 丙寅魏以侍中驃騎大將軍爾朱彦伯為司徒 魏詔有司不得復稱偽梁【魏不競於梁故也復扶又翻下同】 五月丙子魏荆州城民斬趙修延復推李琰之行州事【趙修延執李琰之見上卷上年】魏爾朱仲遠使都督魏僧勗等討崔祖螭於東陽斬<br />
<br />
  之 【考異曰北齊李渾傳普泰中崔社客反於海岱攻圍青州以渾為征東將軍都官尚書行臺赴援而社客宿將多謀諸城各自保固堅壁清野諸將議有異同渾曰社客賊之根本若簡練驍勇銜枚夜襲徑趨營下出其不意咄嗟之間便可擒殄如社客就擒則諸郡可傳檄而定諸將遲疑渾乃速行未明達城下賊徒驚散擒社客斬首送洛陽按其年時事迹與祖螭略同未知社客即祖螭為别一人也今從魏帝紀】 初昭明太子葬其母丁貴嬪【普通七年丁貴嬪卒】遣人求墓地之吉者或賂宦者俞三副求賣地云若得錢三百萬以百萬與之三副密啓上言太子所得地不如今地於上為吉【今地謂求賣之地也】上年老多忌即命市之葬畢有道士云此地不利長子若厭之【長子亮翻厭一恊翻又於琰翻下厭禱同】或可申延【申寛也】乃為蠟鵝及諸物埋於墓側長子位宫監鮑邈之魏雅初皆有寵於太子【五代志梁制東宫有外監殿局内監宫監者即唐内直局之職也龍朔二年改監曰内直郎】邈之晩見疎於雅乃密啟上云雅為太子厭禱上遣檢掘果得鵝物大驚將窮其事徐勉固諫而止但誅道士由是太子終身慙憤不能自明及卒上徵其長子南徐州刺史華容公歡至建康欲立以為嗣衘其前事猶豫久之卒不立【卒子恤翻】庚寅遣還鎮【史因帝不立孫究言事始嗚呼帝於豫章王綜臨賀王正德雖犯惡逆猶容忍之至於昭明被讒則終身衘其事蓋天奪其魄也為昭明子詧仇視諸父張本】<br />
<br />
  臣光曰君子之於正道不可少頃離也【少詩沼翻離力智翻】不可跬步失也【跬窺婢翻】以昭明太子之仁孝武帝之慈愛一染嫌疑之迹身以憂死罪及後昆求吉得凶不可湔滌可不戒哉【湔將先翻】是以詭誕之士奇邪之術君子遠之【奇居宜翻異也遠于願翻】<br />
<br />
  丙申立太子母弟晉安王綱為皇太子朝野多以為不順【立世適孫為順朝直遙翻】司議侍郎周弘正【按陳書周弘正傳普通中初置司文義郎直夀光省以弘正為司義侍郎議當作義】嘗為晉安王主簿乃奏記曰謙讓道廢多歷年所伏惟明大王殿下天挺將聖【論語太宰問于子貢曰夫子聖者歟子貢曰固天縱之將聖朱元晦注曰將殆也謙若不敢知之辭或曰將大也】四海歸仁是以皇上德音以大王為儲副意者願聞殿下抗目夷上仁之義【左傳宋桓公疾太子兹父固請曰目夷長且仁君其立之公命子魚子魚辭曰能以國讓仁孰大焉臣弗及也遂走而退子魚目夷字也】執子臧大賢之節【左傳成十三年諸侯伐秦曹宣公卒于師曹人使公子負芻守公子欣時逆曹伯之喪負芻殺太子而自立既葬子臧將亡國人皆將從之成公乃懼告罪且請焉子臧乃反諸侯討曹成公執而歸諸京師將見子臧于王而立之辭曰聖達節次守節為君非吾節也雖不能聖敢失守乎遂逃奔宋負芻立是為成公子臧欣時是也】逃玉輿而弗乘【玉輿當作王輿莊子讓王篇曰越人三世弑其君王子搜患之逃乎丹宂而越國無君求王子搜不得從之丹宂王子搜不出越人董之以艾乘以王輿王子搜援綏登車仰天而呼曰君乎君獨不可以捨我乎】棄萬乘如脱屣【孟子曰舜視棄天下猶敝屣也乘䋲正翻屣所是翻】庶改澆競之俗以大吳國之風【謂太伯以天下讓逃而君吳也澆堅堯翻薄也】古有其人今聞其語能行之者非殿下而誰使無為之化復生於遂古【復扶又翻】讓王之道不墜於來葉【遂往也莊子外篇有讓王述堯舜以天下相讓來葉來世也】豈不盛歟王不能從弘正捨之兄子也【周捨柄用于天監之初】太子以侍讀東海徐摛為家令【晉王紹宗遷秘書少監仍侍皇太子讀書此侍讀之始也】兼管記尋帶領直【管記職同公府記室梁制上臺東宫皆有領直領直者領直衛兵也】摛文體輕麗春坊盡學之【東宫謂之春宫宫坊謂之春坊】侍人謂之宫體上聞之怒召摛欲加誚責及見【誚才笑翻見賢遍翻】應對明敏辭義可觀意更釋然因問經史及釋敎摛商較從橫【較古搉翻從子容翻】應對如響上甚加歎異【上崇信釋氏意謂徐摛業儒但知經史而已扣擊之餘及于釋敎商較從横應對如響遂加歎異殊不思上有好者下必有甚者焉釋敎盛行可以媒富貴利達江東人士孰不從風而靡乎】寵遇曰隆領軍朱异不悦【异羊至翻】謂所親曰徐叟出入兩宫漸來見逼我須早為之所遂乘間白上曰摛年老又愛泉石意在一郡自養上謂摛真欲之乃召摛謂曰新安大好山水遂出為新安太守【史言朱异固寵忌前間古莧翻】六月癸丑立華容公歡為豫章王其弟枝江公譽為河東王曲阿公詧為岳陽王上以人言不息【謂不順也】故封歡兄弟以大郡用慰其心久之鮑邈之坐誘掠人【誘音酉】罪不至死太子綱追思眧明之寃揮淚誅之 魏高歡將起兵討爾朱氏鎮南大將軍斛律金軍主善無庫狄千【善無縣前漢屬鴈門郡後漢屬定襄郡魏晉省後魏天平二年始以善無為郡】與歡妻弟婁昭妻之姊夫段榮皆勸成之歡乃詐為書稱爾朱兆將以六鎮人配契胡為部曲衆皆憂懼【契欺訖翻】又為幷州符徵兵討步落稽【爾朱兆擅命幷汾此亦高歡偽為兆符也步落稽即稽胡】萬人將遣之孫騰與都督尉景為請留五日如此者再【孫騰尉景既為鎮人請留必又因其願留之情扇動之於下此當以意會也為於偽翻】歡親送之郊雪涕執别衆皆號慟聲震原野歡乃諭之【先感動其心面後諭之號戶刀翻】曰與爾俱為失鄉客義同一家【高歡亦鎮戶故云然】不意在上徵發乃爾今直西向已當死【自信都赴幷汾為西向】後軍期又當死配國人又當死【爾朱契胡種也故謂契胡為國人】奈何衆曰唯有反耳歡曰反乃急計然當推一人為主誰可者衆共推歡歡曰爾鄉里難制不見葛榮乎雖有百萬之衆曾無灋度終自敗滅今以吾為主當與前異毋得陵漢人犯軍令生死任吾則可不然不能為天下笑【高歡先立法制以齊其衆故能成大事史言盜亦有道】衆皆頓顙曰死生唯命歡乃椎牛饗士庚申起兵於信都 【考異曰魏書帝紀起兵於庚申北齊書帝紀在庚子北史魏紀齊紀亦然今從魏書紀】亦未敢顯言叛爾朱氏也會李元忠舉兵逼殷州歡令高乾帥衆救之【高乾預歡密謀而使之救殷州此不過使之誘禽爾朱羽生耳】乾輕騎入見爾朱羽生與指畫軍計【騎奇計翻】羽生與乾俱出因擒斬之持羽生首謁歡歡撫膺曰今日反决矣【高歡反謀非一日矣及爾朱羽生授首方言反决蓋其初猶有疑李元忠高乾邕之心元忠既舉兵逼殷州乾邕又斬羽生歡於是深悉二人之心而冀殷之勢已合於是决反】乃以元忠為殷州刺史鎮廣阿歡於是抗表罪狀爾朱氏爾朱世隆匿之不通【世隆為尚書令故得匿歡表】 魏楊播及弟椿津皆有名德播剛毅椿津謙恭家世孝友緦服同㸑【凡三從之服服緦麻】男女百口人無間言【間古莧翻異也】椿津皆至三公一門七郡太守三十二州刺史【史言楊氏貴盛】敬宗之誅爾朱滎也播子侃預其謀【事見上卷上年】城陽王徽李彧皆其姻戚也爾朱兆入洛侃逃歸華隂【華戶化翻】爾朱天光使侃婦父韋義遠招之與盟許貰其罪【貰貸也音始制翻】侃曰彼雖食言死者不過一人猶冀全百口乃出應之天光殺之時椿致仕與其子昱在華隂椿弟冀州刺史順司空津順子東雍州刺史辨正平太守仲宣皆在洛【雍於用翻】秋七月爾朱世隆誣奏楊氏謀反請收治之【治直之翻】魏主不許世隆苦請帝不得己命有司檢案以聞壬申夜世隆遣兵圍津第天光亦遣兵掩椿家於華隂東西之族無少長皆殺之【世隆天光先已約同夷楊氏故東西一時俱居華隂者為西族居洛者為東族少詩照翻長知兩翻】籍沒其家世隆奏云楊氏實反與收兵相拒已皆格殺帝惋悵久之不言而已【惋烏貫翻悵丑亮翻】朝野聞之無不痛憤津子逸為光州刺史爾朱仲遠遣使就殺之唯津子愔於被收時適出在外逃匿獲免往見高歡於信都【使疏吏翻愔於今翻被皮義翻 考異曰北齊書愔傳云愔父津為幷州刺史愔隨之任俄而孝莊幽崩愔時適欲還都行達邯鄲過津義從楊寛家為寛所執至相州見刺史劉誕以愔名家盛德甚相哀念遣隊主鞏榮貴防禁送都至安陽亭榮貴遂與俱逃乃投高昂兄弟潜竄累載屬齊神武至信都遂投刺轅門即署行臺郎中按時齊神武已在信都言潜竄累載誤矣又云孝莊幽崩而愔欲還都見執皆非也】泣訴家禍因為言討爾朱氏之策【為於偽翻】歡甚重之即署行臺郎中【楊愔門地既高又有幹用高歡起兵之初藉人望以為重藉才幹以為用所以擢而用之士無賢不肖入朝見嫉田横島之逃實基於此】乙亥上臨軒策拜太子大赦 丙戍魏司徒爾朱彦伯以<br />
<br />
  旱遜位戊子以彦伯為侍中開府儀同三司彦伯於兄弟中差無過惡爾朱世隆固讓太保魏主特置儀同三司之官位次上公之下【太師太傅太保為三司位上公】庚寅以世隆為之斛斯椿譖朱瑞於世隆世隆殺之【以朱瑞為敬宗所親遇也】庚寅凡宗戚有服屬者【有服屬諸緦麻以上】竝可賜湯沐食鄉亭侯【婦人賜湯沐邑男子食鄉侯亭侯也】隨遠近為差【隨服屬之遠近以為等差】 壬辰以吏部尚書何敬容為尚書右僕射敬容昌㝢之子也【何昌寓尚之之弟子】 魏爾朱仲遠度律等聞高歡起兵恃其彊不以為慮獨爾朱世隆憂之爾朱兆將步騎二萬出井陘趣殷州【將即亮翻騎奇計翻陘音刑趣七喻翻】李元忠棄城奔信都八月丙午爾朱仲遠度律將兵討高歡九月己卯魏以仲遠為太宰庚辰以爾朱天光為大司馬 癸巳魏主追尊父廣陵惠王為先帝【支子入繼大宗尊所生父為皇自漢哀帝始尊】<br />
<br />
  【之為帝自吳孫皓始】母王氏為先太妃封弟永業為高密王子恕為勃海王 冬十月己酉上幸同泰寺升灋坐講涅槃經【涅奴結翻】七日而罷 樂山侯正則先有罪徙鬱林【五代志樂山縣屬鬱林郡鬱林漢古郡唐為鬱林郡】招誘亡命欲攻番禺【誘音酉番禺音潘愚】廣州刺史元仲景討斬之【仲景當作景仲】正則正德之弟也【史言臨川王宏諸子皆兇悖】 孫騰說高歡曰今朝廷隔絶號令無所稟【說式芮翻受命曰稟音必錦翻】不權有所立則衆將沮散【沮在呂翻】歡疑之騰再三固請乃立勃海太守元朗為帝朗融之子也【章武王融為葛榮所殺】壬寅朗即位於信都城西改元中興【廢帝諱朗字仲哲章武王融第三子也】以歡為侍中丞相都督中外諸軍事大將軍録尚書事大行臺高乾為侍中司空高敖曹為驃騎大將軍儀同三司冀州刺史【時廢帝除昂為冀州刺史終其身驃匹妙翻騎奇計翻】孫騰為尚書左僕射河北行臺魏蘭根為右僕射【去年敬宗以魏蘭根為河北行臺】己酉爾朱仲遠度律與驃騎大將軍斛斯椿車騎大將軍儀同三司賀拔勝車騎大將軍賈顯智軍於陽平【此陽平縣也漢時屬東郡魏晉以來屬陽平郡唐魏州莘縣陽平之地也】顯智名智以字行顯度之弟也爾朱兆出井陘軍於廣阿衆號十萬高歡縱反間云世隆兄弟謀殺兆復云兆與歡同謀殺仲遠等【間古莧翻復扶又翻】由是迭相猜貳徘徊不進仲遠等屢使斛斯椿賀拔勝往諭兆兆帥輕騎三百來就仲遠【帥讀曰率】同坐幕下意色不平手舞馬鞭長嘯凝望【鄭玄曰嘯蹙口而出聲】疑仲遠等有變遂趨出馳還仲遠遣椿勝等追曉說之兆執椿勝還營仲遠度律大懼引兵南遁兆數勝罪將斬之【說式芮翻數所具翻】曰爾殺衛可孤罪一也【事見一百五十卷普通五年】天柱薨爾不與世隆等俱來而東征仲遠罪二也【事見上卷上年】我欲殺爾久矣今復何言【復扶又翻】勝曰可孤為國巨患勝父子誅之其功不小反以為罪乎天柱被戮以君誅臣【被皮義翻】勝寧負王不負朝廷今日之事生死在王但寇賊密邇骨肉構隙自古及今未有如是而不亡者勝不憚死恐王失策兆乃捨之高歡將與兆戰而畏其衆彊以問親信都督段韶【親信都督魏末諸將擅兵始置是官以領親兵】韶曰所謂衆者得衆人之死【言得衆人之死力也】所謂彊者得天下之心爾朱氏上弑天子中屠公卿下暴百姓王以順討逆如湯沃雪何衆彊之有歡曰雖然吾以小敵大恐無天命不能濟也韶曰韶聞小能敵大小道大淫皇天無親惟德是輔【小能敵大小道大淫左傳記隨大夫季梁之言也皇天無親惟德是輔書蔡仲之命之辭也段韶父子起於北邊以騎射為工安能作書語魏收以其於北齊為勳戚宗門彊盛從而為之辭耳孟子曰盡信書不如無書信哉】爾朱氏外亂天下内失英雄心智者不為謀勇者不為鬭【為於偽翻】人心已去天意安有不從者哉韶榮之子也【段榮與高歡親善】辛亥歡大破兆於廣阿俘其甲卒五千餘人 十一月乙未上幸同泰寺講般若經七日而罷【般北末翻若人者翻】 庚辰魏高歡引兵攻鄴相州刺史劉誕嬰城固守【相息亮翻】 是歲魏南兖州城民王乞得刼刺史劉世明舉州來降【魏正光中置南兖州治譙城領陳留梁譙沛下蔡北梁馬頭等郡降戶江翻】世明芳之族子也【劉芳以儒學用於孝文宣武二帝】上以侍中元樹為鎮北將軍都督北討諸軍事鎮譙城以世明為征西大將軍郢州刺史加儀同三司世明不受固請北歸上許之世明至洛陽奉送所持節歸鄉里不仕而卒【陳力就列不能者止劉世明有焉劉氏世居彭城】<br />
<br />
  四年春正月丙寅以南平王偉為大司馬元灋僧為太尉袁昂為司空 立西豐侯正德為臨賀王正德自結於朱异上既封昭明諸子异言正德失職【言帝嘗養正德為子既而還本爵秩不得與諸子齒也异音羊至翻】故王之 以太子右衛率薛灋護為司州牧衛送魏王悦入洛 庚午立太子綱之長子大器為宣城王【長知兩翻】 魏高歡攻鄴為地道施柱而焚之城䧟入地【宂城下為地道而未成恐其土頹落而不得䆒功故施柱地道既成乃焚其柱故城陷入地】壬午拔鄴擒劉誕以楊愔為行臺右丞時軍國多事文檄教令皆出於愔及開府諮議參軍崔㥄㥄逞之五世孫也【㥄力應翻崔逞去燕歸魏為道武帝所殺】二月以太尉元灋僧為東魏王【上既以元悦為魏王使自西道入又使元法僧從東道入故謂之東魏王】欲遣還北兖州刺史羊侃為軍司馬與灋僧偕行 揚州刺史邵陵王綸遣人就市賖買錦綵絲布數百匹市人皆閉邸店不出少府丞何智通依事啟聞綸被責還弟【被皮義翻弟與第同下於弟同】乃遣防閤戴子高等以槊刺智通於都巷【都巷猶前言京巷也槊色角翻刺七亦翻】刃出於背智通識子高取其血以指畫車壁為邵陵字乃絶由是事覺庚戌綸坐免為庶人鎻之於第經二旬乃脱鎻頃之復封爵 辛亥魏安定王追謚敬宗曰武懷皇帝【朗既禪位孝武帝封為安定王】甲子以高歡為丞相柱國大將軍太師三月丙寅以高澄為驃騎大將軍丁丑安定王帥百官入居於鄴【帥讀曰率】爾朱兆與爾朱世隆等互相猜阻世隆卑辭厚禮諭兆欲使之赴洛唯其所欲又請節閔帝納兆女為后兆乃悦幷與天光度律更立誓約復相親睦斛斯椿隂謂賀拔勝曰天下皆怨毒爾朱而吾等為之用亡無日矣不如圖之勝曰天光與兆各據一方欲盡去之甚難去之不盡必為後患柰何椿曰此易致耳乃說世隆追天光等赴洛共討高歡【復扶又翻去羌呂翻說式芮翻】世隆屢徵天光天光不至使椿自往邀之曰高歡作亂非王不能定豈可坐視宗族夷滅邪天光不得已將東出問策於雍州刺史賀拔岳【賀拔岳自岐州遷刺雍州雍於用翻】岳曰王家跨據三方【兆北據幷汾天光西奄關隴仲遠擅命徐兖是跨據三方】士馬殷盛高歡烏合之衆豈能為敵但能同心戮力往無不捷若骨肉相疑則圖存之不暇安能制人如下官所見莫若且鎮關中以固根本分遣鋭師與衆軍合勢進可以克敵退可以自全天光不從閏月壬寅天光自長安兆自晉陽度律自洛陽仲遠自東郡皆會於鄴衆號二十萬夾洹水而軍【水經注洹水逕鄴城南洹於元翻】節閔帝以長孫稚為大行臺總督之高歡令吏部尚書封隆之守鄴癸丑出頓紫陌【水經注漳水東出山過鄴又北逕祭陌西戰國之時俗巫為河伯娶婦祭於此陌田融以為紫陌趙建武十一年造紫陌浮橋於水上】大都督高敖曹將鄉里部曲王桃湯等三千人以從【將即亮翻下同從才用翻】歡曰高都督所將皆漢兵恐不足集事欲割鮮卑兵千餘人相雜用之何如敖曹曰敖曹所將練習已久前後格鬭不減鮮卑今若雜之情不相洽勝則爭功退則推罪【推吐雷翻】不煩更配也庚申爾朱兆帥輕騎三千夜襲鄴城叩西門不克而退壬戌歡將戰馬不滿二千步兵不滿三萬衆寡不敵乃於韓陵為圓陳【五代志鄴縣有韓陵山杜佑曰在相州安陽縣東北陳讀曰陣】連繫牛驢以塞歸道【塞息則翻】於是將士皆有死志【將即亮翻】兆望見歡遙責歡以叛己歡曰本所以勠力者共輔帝室今天子何在兆曰永安枉害天柱我報讎耳【敬宗年號永安故以稱之】歡曰我昔聞天柱計汝在戶前立豈得言不反邪【對兩軍其隂謀以正爾朱之罪】且以君殺臣何報之有今日義絶矣遂戰歡將中軍高敖曹將左軍歡從父弟岳將右軍【將即亮翻從才用翻】歡戰不利兆等乘之岳以五百騎衝其前别將斛律敦收散卒躡其後敖曹以千騎自栗園出横擊之兆等大敗賀拔勝與徐州刺史杜德於陳降歡【陳音陣同】兆對慕容紹宗撫膺曰不用公言以至於此【謂紹宗諫兆使歡統州鎮兵而兆不用也】欲輕騎西走【自鄴西走歸晉陽】紹宗反旗鳴角【徐廣車服儀制曰角前代書記所不載或云本出羌胡以驚中國之馬杜佑曰大角即後魏簸邏迴是也】收散卒成軍而去兆還晉陽仲遠奔東郡【秦置東郡晉改為濮陽國後復曰東郡治滑臺城】爾朱彦伯聞度律等敗欲自將兵守河橋【將即亮翻】世隆不從度律天光將之洛陽大都督斛斯椿謂都督賈顯度賈顯智曰今不先執爾朱氏吾屬死無類矣乃夜於桑下盟約倍道先還世隆使其外兵參軍陽叔淵馳赴北中【北中即北中郎府城在河橋北岸】簡閲敗卒以次内之【内讀曰納】椿至不得入城乃詭說叔淵曰【說式芮翻】天光部下皆是西人聞欲大掠洛邑遷都長安宜先内我以為之備叔淵信之夏四月甲子朔椿等入據河橋盡殺爾朱氏之黨度律天光欲攻之會大雨晝夜不止士馬疲頓弓矢不可施遂西走至灅陂津為人所擒送於椿所【灅陂津在河橋西亦曰雷波即爾朱兆犯洛帥騎踏淺涉度之處灅力水翻】椿使行臺長孫稚詣洛陽奏狀别遣賈顯智張歡帥騎掩襲世隆執之【帥讀曰率騎奇計翻】彦伯時在禁直長孫稚於神虎門啟陳高歡義功既振請誅爾朱氏節閔帝使舍人郭崇報彦伯彦伯狼狽走出為人所執與世隆俱斬於閶闔門外送其首幷度律天光於高歡節閔帝使中書含人盧辯勞歡於鄴【勞力到翻】歡使之見安定王辯抗辭不從歡不能奪乃捨之辯同之兄子也【盧同始附元乂以進】辛未驃騎大將軍行濟州事侯景降於安定王以景為尚書僕射南道大行臺濟州刺史【濟子禮翻降戶江翻下同】爾朱仲遠來奔仲遠帳下都督喬寧張子期自滑臺詣歡降歡責之曰汝事仲遠擅其榮利盟契百重許同生死【契約也重直龍翻】前仲遠自徐州為逆汝為戎首【謂前年仲遠舉兵向洛時也】今仲遠南走汝復叛之【復扶又翻】事天子則不忠事仲遠則無信犬馬尚識飼之者【飼祥吏翻】汝曾犬馬之不如遂斬之爾朱天光之東下也留其弟顯夀鎮長安召秦州刺史侯莫陳悦欲與之俱東賀拔岳知天光必敗欲留悦共圖顯夀以應高歡計未有所出字文泰謂岳曰今天光尚近悦未必有貳心若以此告之恐其驚懼然悦雖為主將不能制物【將即亮翻】若先說其衆必人有留心悦進失爾朱之期退恐人情變動乘此說悦事無不遂岳大喜即令泰入悦軍說之悦遂與岳俱襲長安【岳為雍州刺史本治長安蓋天光東下使之出捍西北也】泰帥輕騎為前驅顯夀棄城走追至華隂擒之【帥讀曰率騎奇計翻華戶化翻】歡以岳為關西大行臺 【考異曰北史薛孝通為中書郎以關中險固秦漢舊都須預謀防遏以為後計縱河北失利猶足據之節閔帝深以為然問誰可任者孝通與賀拔岳同事天光又與周文帝有舊二人竝先在關右竝推薦之乃超授岳督岐華秦雍諸軍事關西大行臺雍州牧周文帝為左丞孝通為右丞齎詔書馳驛入關授岳等同鎮長安後天光敗於韓陵節閔遂不得入關為齊神武幽廢按天光尚在節閔安敢除岳鎮關中今從魏書】岳以泰為行臺左丞領府司馬事無巨細皆委之爾朱世隆之拒高歡也使齊州行臺尚書房謨募兵趣四瀆【四瀆津名在臨濟縣水經注河水東北流四瀆津津西側岸臨河有四瀆祠以其自河入濟自泗入淮自淮達江水徑周流故有四瀆之名趣七喻翻】又使其弟青州刺史弼趣亂城揚聲北渡為掎角之勢【掎居蟻翻】及韓陵既敗弼還東陽聞世隆等死欲來奔數與左右割臂為盟帳下都督馮紹隆素為弼所信待說弼曰今方同契闊【數所角翻說式芮翻詩擊鼓曰死生契闊毛萇注曰契闊勤苦也契苦結翻】宜更割心前之血以盟衆弼從之大集部下披胸令紹隆割之紹隆因推刃殺之【推吐雷翻】傳首洛陽丙子安東將軍辛永以建州降於安定王辛巳安定王至邙山高歡以安定王疎遠【章武王太洛文成之子獻文之季弟也太洛生融融生安定王於孝明帝緦麻親也故以為疎遠魏收書章武王太洛景穆之子以彬為後彬子融審爾則愈疎遠矣】使僕射魏蘭根慰諭洛邑且觀節閔帝之為人欲復奉之【復扶又翻】蘭根以帝神采高明恐於後難制與高乾兄弟及黄門侍郎崔㥄共勸歡廢之【㥄力膺翻】歡集百官問所宜立莫有應者太僕代人綦母儁盛稱節閔帝賢明宜主社稷歡欣然是之㥄作色曰若言賢明自可待我高王徐登大位廣陵既為逆胡所立【節閔帝本廣陵王】何得猶為天子若從儁言王師何名義舉歡遂幽節閔帝於崇訓佛寺歡入洛陽斛斯椿謂賀拔勝曰今天下事在吾與君耳若不先制人將為人所制高歡初至圖之不難勝曰彼有功於時害之不祥比數夜與歡同宿【比毘至翻】且序往昔之懷兼荷兄恩意甚多【荷下可翻古以儋負為義故以受任為荷受恩為荷而感恩者亦曰荷】何苦憚之椿乃止【史言賀拔勝有才武而無遠識高歡能以姦詐玩弄時輩而悦其心斛斯椿者小有才反覆人也其圖歡之志固在孝武帝未立之前矣】歡以汝南王悦高祖之子召欲立之聞其狂暴無常乃止 【考異曰魏書悦傳云神武令人示意悦既至清狂如故動為罪失不可扶立乃止按悦時猶在梁境比召至洛往返幾日蓋神武聞其所為而止耳】時諸王多逃匿尚書左僕射平陽王修懷之子也【廣平王懷高祖之子修於孝明帝從兄弟也】匿於田舍歡欲立之使斛斯椿求之椿見修所親員外散騎侍郎太原王思政問王所在思政曰須知問意椿曰欲立為天子思政乃言之椿從思政見修修色變謂思政曰得無賣我邪曰不也曰敢保之乎曰變態百端何可保也椿馳報歡歡遣四百騎迎修入氈帳【騎奇計翻氈帳胡夷酋帥所居漢人謂之穹廬】陳誠泣下霑襟修讓以寡德歡再拜修亦拜歡出備服御進湯沐達夜嚴警【嚴為警備也】昧爽【孔安國曰昧冥爽明早旦馬曰昧未旦也陸德明曰爽謂早旦也】文武執鞭以朝【軍中不能備朝服故執鞭以為敬朝直遙翻】使斛斯椿奉勸進表椿入帷門磬折延首而不敢前【張守節曰磬折謂曲體揖之若石磬之形曲折也磬形皆中屈垂兩頭言人屈腰則似也】修令思政取表視之曰便不得不稱朕矣【書曰天位艱哉又曰毋安厥位惟危雖天人樂推神器歸屬賢君處此之時慓慓乎懼其不勝也平陽王視勸進表而此言驕滿之氣溢出於肝鬲之上君子以是知其不能終】乃為安定王作詔策而禪位焉【為于偽翻】戊子孝武帝即位於東郭之外【帝諱修字孝則廣平武穆王懷之第三子也東郭洛城東郭也】用代都舊制以黑氈蒙七人歡居其一帝於氈上西向拜天畢入御太極殿【魏自孝文帝用夏變夷宣武孝明即位皆用漢魏之制今復用夷禮】羣臣朝賀【朝直遙翻下同】升閶闔門大赦改元太昌以高歡為大丞相天柱大將軍太師世襲定州刺史庚寅加高澄侍中開府儀同三司初歡起兵信都爾朱世隆知司馬子如與歡有舊自侍中驃騎大將軍出為南岐州刺史歡入洛召子如為大行臺尚書朝夕左右參知軍國廣州刺史廣寧韓賢【魏收志廣寧郡屬朔州領石門中川二縣五代志馬邑郡善陽縣後齊置朔州及廣寧郡】素為歡所善歡入洛凡爾朱氏所除官爵例皆削奪唯賢如故以前御史中尉樊子鵠兼尚書左僕射為東南道大行臺與徐州刺史杜德追爾朱仲遠仲遠已出境逐攻元樹於譙丞相歡徵賀拔岳為冀州刺史岳畏歡欲單馬入朝行臺右丞薛孝通說【說式芮翻】岳曰高王以數千鮮卑破爾朱百萬之衆誠亦難敵然諸將或素居其上或與之等夷屈首從之勢非獲已今或在京師或據州鎮高王除之則失人望留之則為腹心之疾且吐萬人雖復敗走【爾朱兆字吐萬人】猶在幷州高王方内撫羣雄外抗勍敵【勍其京翻】安能去其巢穴與公爭關中之地乎今關中豪俊皆屬心於公【屬之欲翻】願效其智力公以華山為城黄河為塹【華戶化翻塹七艷翻】進可以兼山東退可以封函谷【後漢王元說隗囂曰元請以一丸泥為大王東封函谷關】柰何欲東手受制於人乎言未卒【卒子恤翻】岳執孝通手曰君言是也乃遜辭為啟而不就徵壬辰丞相歡還鄴送爾朱度律天光於洛陽斬之 五月丙申魏主酖節閔帝於門下外省【年三十五西魏謚帝曰節閔】詔百司會喪葬用殊禮【加九旒鑾輅黄屋左纛班劒百二十人蓋其禮特異于諸王之喪耳】以沛郡王欣為太師趙郡王諶為太保【諶世壬翻】南陽王寶炬為太尉長孫稚為太傅寶炬愉之子也【京兆王愉亦孝文帝之子】丞相歡固辭天柱大將軍戊戌許之己酉清河王亶為司徒侍中河南高隆之本徐氏養子丞相歡命以為弟恃歡勢驕公卿南陽王寶炬敺之曰鎮兵何敢爾【魏遷洛陽北人留居北鎮者率隸尺籍故詈之曰鎮兵敺烏口翻】魏主以歡故六月丁卯黜寶炬為驃騎大將軍歸第 魏主避廣平武穆王之諱改謚武懷皇帝曰孝莊皇帝【謚法武而不遂曰莊死於原野曰莊兵甲亟作曰莊】廟號敬宗秋七月庚子魏復以南陽王寶炬為太尉 壬寅魏<br />
<br />
  丞相歡引兵入滏口大都督庫狄干入井陘擊爾朱兆【滏音父陘音刑】庚戌魏主使驃騎大將軍儀同三司高隆之帥步騎十萬會丞相歡於太原因以隆之為丞相軍司歡軍於武鄉【晉置武鄉縣屬上黨郡石勒分置武鄉郡唐為武鄉縣屬潞州我朝屬威勝軍】爾朱兆大掠晉陽北走秀容幷州平【走音奏】歡以晉陽四塞【太原郡之地東阻太行常山西有蒙山南有霍太山高壁嶺北阨東陘西陘關故亦以為四塞之地】乃建大丞相府而居之【自此至於高齊建國遂以晉陽為陪都】 魏夏州遷民郭遷據青州反【郭遷自夏州遷青州必叛黨也】刺史元嶷棄城走【嶷魚力翻】詔行臺侯景等討之拔其城遷來奔 魏東南道大行臺樊子鵠圍元樹於譙城分兵攻取蒙縣等五城【去年梁遣元樹鎮譙城蒙縣漢晉屬梁國魏屬譙郡隋并入山桑縣唐改山桑為蒙城縣屬亳州】以絶援兵之路樹請帥衆南歸以地還魏【帥讀曰率】子鵠等許之與之誓約樹衆半出子鵠擊之擒樹及譙州刺史朱文開以歸羊侃行至官竹【水經注睢水自睢陽東南流歷竹圃水次綠竹䕃渚菁菁彌望世人謂之梁王竹園官收其利因曰官竹】聞樹敗而還【還音旋】九月樹至洛陽久之復欲南奔魏人殺之【復扶又翻】 乙巳以司空袁昂領尚書令冬十一月丁酉日南至【夏至之日日北至冬至之日日南至杜預曰冬至之日日】<br />
<br />
  【南極班志日行光道夏至至於東井北近極故晷短冬至至於牽牛遠極故晷長司馬彪志曰日道南去極彌遠其景彌長遠長乃極冬乃至日道歛北去極彌近其景彌短近短乃極夏乃至焉】魏主祀圜丘【古者因天事天故祭天於圜丘其圜象天】 甲辰魏殺安定王朗東海王曄【二王皆嘗擁立雖已廢退居嫌疑之地故見殺】己酉以汝南王悦為侍中大司馬 魏葬靈太后胡氏【爾朱榮沈靈后於河今乃克葬】 上聞魏室已定十二月庚辰復以太尉元灋僧為郢州刺史【是年春以元法僧為東魏王】 魏主以汝南王悦屬近地尊【悦魏王之叔父也】丁亥殺之 魏大赦改元永興以與太宗同號【永興魏太宗即位之初元也】復改永熙【復扶又翻】 魏主納丞相歡女為后命太常卿李元忠納幣於晉陽歡與之宴論及舊事元忠曰昔日建義轟轟大樂【樂音洛】比來寂寂無人問【比毗至翻】歡撫掌笑曰此人逼我起兵元忠戲曰若不與侍中當更求建義處歡曰建義不慮無止畏如此老翁不可遇耳元忠曰止為此翁難遇所以不去因捋歡須大笑【為于偽翻捋盧括翻】歡悉其雅意深重之【悉諳究也知也雅素也】 爾朱兆既至秀容分守險隘出入寇抄【抄楚交翻】魏丞相歡揚聲討之師出復止者數四兆意怠歡揣其歲首當宴會【復扶又翻揣初委翻】遣都督竇泰以精騎馳之一日一夜行三百里歡以大軍繼之【為明年竇泰破爾朱兆張本騎奇計翻】<br />
<br />
  資治通鑑卷一百五十五<br />
<br />
<史部,編年類,資治通鑑>  <br>
   </div> 

<script src="/search/ajaxskft.js"> </script>
 <div class="clear"></div>
<br>
<br>
 <!-- a.d-->

 <!--
<div class="info_share">
</div> 
-->
 <!--info_share--></div>   <!-- end info_content-->
  </div> <!-- end l-->

<div class="r">   <!--r-->



<div class="sidebar"  style="margin-bottom:2px;">

 
<div class="sidebar_title">工具类大全</div>
<div class="sidebar_info">
<strong><a href="http://www.guoxuedashi.com/lsditu/" target="_blank">历史地图</a></strong>  
<a href="http://www.880114.com/" target="_blank">英语宝典</a>  
<a href="http://www.guoxuedashi.com/13jing/" target="_blank">十三经检索</a> 
<br><strong><a href="http://www.guoxuedashi.com/gjtsjc/" target="_blank">古今图书集成</a></strong> 
<a href="http://www.guoxuedashi.com/duilian/" target="_blank">对联大全</a> <strong><a href="http://www.guoxuedashi.com/xiangxingzi/" target="_blank">象形文字典</a></strong> 

<br><a href="http://www.guoxuedashi.com/zixing/yanbian/">字形演变</a>  <strong><a href="http://www.guoxuemi.com/hafo/" target="_blank">哈佛燕京中文善本特藏</a></strong>
<br><strong><a href="http://www.guoxuedashi.com/csfz/" target="_blank">丛书&方志检索器</a></strong> <a href="http://www.guoxuedashi.com/yqjyy/" target="_blank">一切经音义</a>  

<br><strong><a href="http://www.guoxuedashi.com/jiapu/" target="_blank">家谱族谱查询</a></strong>  <strong><a href="http://shufa.guoxuedashi.com/sfzitie/" target="_blank">书法字帖欣赏</a></strong> 
<br>

</div>
</div>


<div class="sidebar" style="margin-bottom:0px;">

<font style="font-size:22px;line-height:32px">QQ交流群9:489193090</font>


<div class="sidebar_title">手机APP 扫描或点击</div>
<div class="sidebar_info">
<table>
<tr>
	<td width=160><a href="http://m.guoxuedashi.com/app/" target="_blank"><img src="/img/gxds-sj.png" width="140"  border="0" alt="国学大师手机版"></a></td>
	<td>
<a href="http://www.guoxuedashi.com/download/" target="_blank">app软件下载专区</a><br>
<a href="http://www.guoxuedashi.com/download/gxds.php" target="_blank">《国学大师》下载</a><br>
<a href="http://www.guoxuedashi.com/download/kxzd.php" target="_blank">《汉字宝典》下载</a><br>
<a href="http://www.guoxuedashi.com/download/scqbd.php" target="_blank">《诗词曲宝典》下载</a><br>
<a href="http://www.guoxuedashi.com/SiKuQuanShu/skqs.php" target="_blank">《四库全书》下载</a><br>
</td>
</tr>
</table>

</div>
</div>


<div class="sidebar2">
<center>


</center>
</div>

<div class="sidebar"  style="margin-bottom:2px;">
<div class="sidebar_title">网站使用教程</div>
<div class="sidebar_info">
<a href="http://www.guoxuedashi.com/help/gjsearch.php" target="_blank">如何在国学大师网下载古籍?</a><br>
<a href="http://www.guoxuedashi.com/zidian/bujian/bjjc.php" target="_blank">如何使用部件查字法快速查字?</a><br>
<a href="http://www.guoxuedashi.com/search/sjc.php" target="_blank">如何在指定的书籍中全文检索?</a><br>
<a href="http://www.guoxuedashi.com/search/skjc.php" target="_blank">如何找到一句话在《四库全书》哪一页?</a><br>
</div>
</div>


<div class="sidebar">
<div class="sidebar_title">热门书籍</div>
<div class="sidebar_info">
<a href="/so.php?sokey=%E8%B5%84%E6%B2%BB%E9%80%9A%E9%89%B4&kt=1">资治通鉴</a> <a href="/24shi/"><strong>二十四史</strong></a>&nbsp; <a href="/a2694/">野史</a>&nbsp; <a href="/SiKuQuanShu/"><strong>四库全书</strong></a>&nbsp;<a href="http://www.guoxuedashi.com/SiKuQuanShu/fanti/">繁体</a>
<br><a href="/so.php?sokey=%E7%BA%A2%E6%A5%BC%E6%A2%A6&kt=1">红楼梦</a> <a href="/a/1858x/">三国演义</a> <a href="/a/1038k/">水浒传</a> <a href="/a/1046t/">西游记</a> <a href="/a/1914o/">封神演义</a>
<br>
<a href="http://www.guoxuedashi.com/so.php?sokeygx=%E4%B8%87%E6%9C%89%E6%96%87%E5%BA%93&submit=&kt=1">万有文库</a> <a href="/a/780t/">古文观止</a> <a href="/a/1024l/">文心雕龙</a> <a href="/a/1704n/">全唐诗</a> <a href="/a/1705h/">全宋词</a>
<br><a href="http://www.guoxuedashi.com/so.php?sokeygx=%E7%99%BE%E8%A1%B2%E6%9C%AC%E4%BA%8C%E5%8D%81%E5%9B%9B%E5%8F%B2&submit=&kt=1"><strong>百衲本二十四史</strong></a>  <a href="http://www.guoxuedashi.com/so.php?sokeygx=%E5%8F%A4%E4%BB%8A%E5%9B%BE%E4%B9%A6%E9%9B%86%E6%88%90&submit=&kt=1"><strong>古今图书集成</strong></a>
<br>

<a href="http://www.guoxuedashi.com/so.php?sokeygx=%E4%B8%9B%E4%B9%A6%E9%9B%86%E6%88%90&submit=&kt=1">丛书集成</a> 
<a href="http://www.guoxuedashi.com/so.php?sokeygx=%E5%9B%9B%E9%83%A8%E4%B8%9B%E5%88%8A&submit=&kt=1"><strong>四部丛刊</strong></a>  
<a href="http://www.guoxuedashi.com/so.php?sokeygx=%E8%AF%B4%E6%96%87%E8%A7%A3%E5%AD%97&submit=&kt=1">說文解字</a> <a href="http://www.guoxuedashi.com/so.php?sokeygx=%E5%85%A8%E4%B8%8A%E5%8F%A4&submit=&kt=1">三国六朝文</a>
<br><a href="http://www.guoxuedashi.com/so.php?sokeytm=%E6%97%A5%E6%9C%AC%E5%86%85%E9%98%81%E6%96%87%E5%BA%93&submit=&kt=1"><strong>日本内阁文库</strong></a> <a href="http://www.guoxuedashi.com/so.php?sokeytm=%E5%9B%BD%E5%9B%BE%E6%96%B9%E5%BF%97%E5%90%88%E9%9B%86&ka=100&submit=">国图方志合集</a> <a href="http://www.guoxuedashi.com/so.php?sokeytm=%E5%90%84%E5%9C%B0%E6%96%B9%E5%BF%97&submit=&kt=1"><strong>各地方志</strong></a>

</div>
</div>


<div class="sidebar2">
<center>

</center>
</div>
<div class="sidebar greenbar">
<div class="sidebar_title green">四库全书</div>
<div class="sidebar_info">

《四库全书》是中国古代最大的丛书,编撰于乾隆年间,由纪昀等360多位高官、学者编撰,3800多人抄写,费时十三年编成。丛书分经、史、子、集四部,故名四库。共有3500多种书,7.9万卷,3.6万册,约8亿字,基本上囊括了古代所有图书,故称“全书”。<a href="http://www.guoxuedashi.com/SiKuQuanShu/">详细>>
</a>

</div> 
</div>

</div>  <!--end r-->

</div>
<!-- 内容区END --> 

<!-- 页脚开始 -->
<div class="shh">

</div>

<div class="w1180" style="margin-top:8px;">
<center><script src="http://www.guoxuedashi.com/img/plus.php?id=3"></script></center>
</div>
<div class="w1180 foot">
<a href="/b/thanks.php">特别致谢</a> | <a href="javascript:window.external.AddFavorite(document.location.href,document.title);">收藏本站</a> | <a href="#">欢迎投稿</a> | <a href="http://www.guoxuedashi.com/forum/">意见建议</a> | <a href="http://www.guoxuemi.com/">国学迷</a> | <a href="http://www.shuowen.net/">说文网</a><script language="javascript" type="text/javascript" src="https://js.users.51.la/17753172.js"></script><br />
  Copyright &copy; 国学大师 古典图书集成 All Rights Reserved.<br>
  
  <span style="font-size:14px">免责声明:本站非营利性站点,以方便网友为主,仅供学习研究。<br>内容由热心网友提供和网上收集,不保留版权。若侵犯了您的权益,来信即刪。scp168@qq.com</span>
  <br />
ICP证:<a href="http://www.beian.miit.gov.cn/" target="_blank">鲁ICP备19060063号</a></div>
<!-- 页脚END --> 
<script src="http://www.guoxuedashi.com/img/plus.php?id=22"></script>
<script src="http://www.guoxuedashi.com/img/tongji.js"></script>

</body>
</html>
