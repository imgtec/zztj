資治通鑑卷二百   宋 司馬光 撰

胡三省 音注

唐紀十六|{
	起旃蒙單閼十月盡玄黓閹茂七月凡六年有奇}


高宗天皇大聖大弘孝皇帝上之下

永徽六年冬十月己酉下詔稱王皇后蕭淑妃謀行鴆毒廢爲庶人母及兄弟並除名流嶺南許敬宗奏故特進贈司空王仁祐告身尚存使逆亂餘孽猶得爲䕃|{
	唐制凡受官者皆給以符謂之告身司空正一品凡三品以上䕃及曾孫}
並請除削從之乙卯百官上表請立中宫|{
	上時掌翻下同}
乃下詔曰武氏門著勲庸地華纓黻往以才行選入後庭|{
	行下孟翻}
譽重椒闈德光蘭掖朕昔在儲貳特荷先慈常得侍從弗離朝夕|{
	荷下可翻從才用翻離力智翻}
宫壼之内恒自飭躬|{
	恒戶登翻}
嬪嬙之間未嘗迕目|{
	嬙慈良翻婦官也迕五故翻逆而視之謂之迕目}
聖情鑒悉每垂賞歎遂以武氏賜朕事同政君|{
	政君事見二十七卷漢宣帝甘露三年}
可立爲皇后丁巳赦天下是日皇后上表稱陛下前以妾爲宸妃韓瑗來濟面折庭爭|{
	事見上卷上年瑗于眷翻折之舌翻爭讀曰諍}
此既事之極難豈非深情爲國|{
	爲于僞翻}
乞加褒賞上以表示瑗等瑗等彌憂懼屢請去位上不許十一月丁卯朔臨軒命司空李勣齎璽綬冊皇后武氏|{
	璽斯氏翻綬音受}
是日百官朝皇后於肅義門故后王氏故淑妃蕭氏並囚於别院上嘗念之閒行至其所|{
	間古莧翻}
見其室封閉極密惟竅壁以通食器惻然傷之呼曰皇后淑妃安在王氏泣對曰妾等得罪爲宫婢何得更有尊稱|{
	稱尺證翻}
又曰至尊若念疇昔使妾等再見日月乞名此院爲囘心院上曰朕即有處置|{
	處昌呂翻}
武后聞之大怒遣人杖王氏及蕭氏各一百斷去手足捉酒甕中曰令二嫗骨醉|{
	斷丁管翻去羌呂翻嫗威遇翻}
數日而死又斬之王氏初聞宣敕再拜曰願大家萬歲昭儀承恩死自吾分淑妃罵曰阿武妖猾乃至於此願他生我爲猫阿武爲鼠生生扼其喉由是宫中不畜猫|{
	分扶問翻妖於喬翻畜吁玉翻}
尋又改王氏姓爲蟒氏|{
	蟒莫朗翻蛇最大者曰蟒}
蕭氏爲梟氏|{
	梟古堯翻}
武后數見王蕭爲祟被髪瀝血如死時狀後徙居蓬萊宫復見之|{
	數所角翻祟雖遂翻復扶又翻大明宫接西内宫城之東北曰東内本永安宫貞觀八年置九月更名大明宫以備太上皇清暑後高宗以風痺厭西内湫濕龍朔二年始大興葺曰蓬萊宫}
故多在洛陽終身不歸長安己巳許敬宗奏曰永徽爰始國本未生權引彗星越升明兩|{
	易離卦大象曰明兩作離大人以繼明照四方彗祥歲翻}
近者元妃載誕正胤降神|{
	言代王弘武后之子當立}
重光日融爝暉宜息|{
	崔豹古今注曰漢文帝為太子樂人歌四章以贊太子之德一曰日重光二曰月重輪三曰星重暉四曰海重潤莊子曰日月出矣而爝火不息其於光也不亦難乎重直龍翻爝即略翻}
安可反植枝幹久易位於天庭倒襲裳衣使違方於震位|{
	震長子也以守社稷宗廟為祭主也}
又父子之際人所難言|{
	漢武帝語田千秋之辭}
事或犯鱗必嬰嚴憲|{
	驪龍頷下有逆鱗徑尺嬰之則死喻人主之威不可犯也}
煎膏染鼎臣亦甘心上召見問之對曰皇太子國之本也本猶未正萬國無所係心且在東宫者所出本微今知國家已有正嫡必不自安竊位而懷自疑恐非宗廟之福願陛下熟計之上曰忠已自讓對曰能為太伯願速從之 西突厥頡苾達度設數遣使請兵討沙鉢羅可汗|{
	厥九勿翻數所角翻使疏吏翻可從刋入聲汗音寒}
甲戍遣豐州都督元禮臣冊拜頡苾達度設為可汗禮臣至碎葉城|{
	自弓月城過思渾川度伊麗河至碎葉界又西行千里至碎葉城屬焉耆都督府界}
沙鉢羅發兵拒之不得前頡苾逹度設部落多為沙鉢羅所併餘衆寡弱不為諸姓所附禮臣竟不冊拜而歸 中書侍郎李義府參知政事義府容貌温恭與人語必嬉怡微笑而狡險忌克故時人謂義府笑中有刀又以其柔而害物謂之李猫顯慶元年春正月辛未以皇太子忠為梁王梁州刺史立皇后子代王弘為皇太子生四年矣忠既廢官屬皆懼罪亡匿無敢見者右庶子李安仁獨侯忠泣涕拜辭而去安仁綱之孫也|{
	李綱著節於隋唐之間}
壬申赦天下改元二月辛亥贈武士彠司徒賜爵周國公|{
	彠一虢翻}
三月

以度支侍郎杜正倫為黄門侍郎同三品|{
	顯慶元年改戶部為度支度徒洛翻}
夏四月壬子矩州人謝無靈舉兵反|{
	矩州諸蠻亦東謝蠻之種落武德四年置矩州}
黔州都督李子和討平之|{
	黔音琴}
己未上謂侍臣曰朕思養人之道未得其要公等為朕陳之|{
	為于偽翻}
來濟對曰昔齊桓公出遊見老而飢寒者命賜之食老人曰願賜一國之飢者賜之衣曰願賜一國之寒者公曰寡人之廩府安足以周一國之飢寒老人曰君不奪農時則國人皆有餘食矣不奪蠶桑則國人皆有餘衣矣故人君之養人在省其征役而已今山東役丁歲别數萬役之則人大勞取庸則人大費臣願陛下量公家所須外餘悉免之|{
	量音良}
上從之 六月辛亥禮官奏停太祖世祖配祀|{
	高祖受禪追尊祖虎曰景皇帝廟號太祖考昺曰元皇帝廟號世祖}
以高祖配昊天於圓丘太宗配五帝於明堂|{
	武德初立圓丘壇於明德門外道東二里壇制四成各廣八尺一寸下成廣二十丈再成廣十五丈三成廣十丈四成廣五丈每祀則昊天上帝及配帝設位于平座籍用藳秸器用陶匏五方上帝日月内官中官外官及衆星並皆從祀其五方帝及日月七座在壇之第二等内五星以下官五十五座在壇之第三等二十八宿已下官一百三十五座在壇之第四等外官一百二十二座在壇下外壝之内衆星三百六十座在外壝之外以景帝配圓丘元帝配明堂}
從之 秋七月乙丑西洱蠻酋長楊棟附顯和蠻酋長王郎祁郎昆棃盤四州酋長王伽衝等帥衆内附|{
	棃州本西寧州武德七年分南寧州二縣置貞觀八年更名梨州其地北接昆州晉梁水郡地也盤州本西平州武德四年置貞觀八年更名晉興古郡地也洱乃吏翻酋慈由翻帥讀曰率}
癸未以中書令崔敦禮爲太子少師同中書門下三品八月丙申固安昭公崔敦禮薨|{
	諡法容儀恭美曰昭昭德有勞曰昭}
辛丑蔥山道行軍總管程知節擊西突厥與歌邏處月二部戰於榆慕谷|{
	處月處密姑蘇歌邏祿弩失畢五姓之衆賀魯爲葉護時所統也據新書歌邏祿即葛邏祿也榆慕谷舊書本紀作榆幕谷}
大破之斬首千餘級副總管周智度攻突騎施處木昆等部於咽城拔之|{
	西突厥咄陸五啜處木昆律突騎施皆一啜也據新書咽城即處木昆所居處昌呂翻}
斬首三萬級 乙巳龜兹王布失畢入朝|{
	龜兹音丘慈朝直遙翻}
李義府恃寵用事洛州婦人淳于氏美色繫大理獄義府屬大理寺丞畢正義枉法出之|{
	屬之欲翻}
將納爲妾大理卿段寶玄疑而奏之上命給事中劉仁軌等鞫之義府恐事洩逼正義自縊於獄中|{
	縊於計翻}
上知之原義府罪不問侍御史漣水王義方欲奏彈之|{
	漣水舊曰襄賁置東海郡東魏改曰海安郡隋開皇初廢郡改襄賁曰漣水屬海州唐屬泗水漣音連}
先白其母曰義方爲御史視姦臣不糾則不忠糾之則身危而憂及於親爲不孝二者不能自决奈何母曰昔王陵之母殺身以成子之名|{
	事見九卷漢高帝元年}
汝能盡忠以事君吾死不恨義方乃奏義府於輦轂之下擅殺六品寺丞|{
	唐六典大理寺丞從六品上}
就云正義自殺亦由畏義府威殺身以滅口如此則生殺之威不由上出漸不可長請更加勘當|{
	長知兩翻當丁浪翻}
於是對仗叱義府令下義府顧望不退義方三叱上既無言義府始趨出義方乃讀彈文上釋義府不問而謂義方毀辱大臣言辭不遜貶萊州司戶 九月括州暴風海溢溺四千餘家|{
	新志處州本括州永嘉郡時兼有永嘉之地上元元年始析置温州}
冬十一月丙寅生羌酋長浪我利波等帥衆内附以其地置柘栱二州|{
	柘州蓬山郡栱州以鉢南伏浪恐部置皆屬松州都督府宋白曰柘州以開拓爲稱音達各翻}
十二月程知節引軍至鷹娑川遇西突厥二萬騎别部鼠尼施等二萬餘騎繼至|{
	鼠尼施咄陸五啜之一也居鷹娑川後置鷹娑都督府娑素何翻}
前軍總管蘇定方帥五百騎馳往擊之西突厥大敗追奔二十里殺獲千五百餘人獲馬及器械綿亘山野不可勝計|{
	勝音升}
副大總管王文度害其功言於知節曰今兹雖云破賊官軍亦有死傷乘危輕脱乃成敗之法耳何急而爲此自今常結方陳置輜重在内|{
	陳讀曰陣重直用翻}
遇賊則戰此萬全策也又矯稱别得旨以知節恃勇輕敵委文度爲之節制遂收軍不許深入士卒終日跨馬被甲結陳不勝疲頓|{
	被皮義翻下同陳讀曰陣勝音升}
馬多瘦死定方言於知節曰出師欲以討賊今乃自守坐自困敝若遇賊必敗懦怯如此何以立功且主上以公爲大將|{
	將即亮翻}
豈可更遣軍副專其號令事必不然請囚文度飛表以聞知節不從至恒篤城|{
	新書作坦篤城}
有羣胡歸附文度曰此屬伺我旋師還復爲賊|{
	伺相吏翻復扶又翻}
不如盡殺之取其資財定方曰如此乃自爲賊耳何名伐叛文度竟殺之分其財獨定方不受師旋文度坐矯詔當死特除名知節亦坐逗遛追賊不及減死免官 是歲以太常卿駙馬都尉高履行爲益州長史|{
	高履行尚太宗女東陽公主}
韓瑗上疏爲禇遂良訟寃曰|{
	上時掌翻爲于僞翻}
遂良體國忘家捐身狥物風霜其操鐵石其心社稷之舊臣陛下之賢佐無聞罪狀斥去朝廷内外甿黎咸嗟舉措|{
	論語孔子曰舉直錯諸枉則民服舉枉錯諸直則民不服}
臣聞晉武弘裕不貽劉毅之誅|{
	事見八十一卷太康三年}
漢祖深仁無恚周昌之直|{
	注已見前恚於避翻}
而遂良被遷已經寒暑違忤陛下其罸塞焉|{
	忤五故翻塞悉則翻}
伏願緬鑒無辜|{
	緬遠也}
稍寛非罪俯矜微欵以順人情上謂瑗曰遂良之情朕亦知之然其悖戾好犯上|{
	悖蒲内翻又蒲没翻好呼到翻}
故以此責之卿何言之深也對曰遂良社稷忠臣爲讒諛所毀昔微子去而殷國以亡|{
	殷紂暴虐日甚微子抱樂器以奔周武王乃告諸侯曰殷有重罪不可不伐遂伐紂滅之}
張華存而綱紀不亂|{
	事見八十二卷至八十三卷}
陛下無故棄逐舊臣恐非國家之福上不納瑗以言不用乞歸田里上不許 劉洎之子訟其父寃稱貞觀之末爲禇遂良所譛而死|{
	事見一百九十八卷貞觀十九年洎其冀翻}
李義府復助之|{
	復扶又翻}
上以問近臣衆希義府之旨皆言其枉給事中長安樂彥瑋獨曰劉洎大臣人主暫有不豫豈得遽自比伊霍今雪洎之罪謂先帝用刑不當乎|{
	當丁浪翻}
上然其言遂寢其事

二年春正月癸巳分哥邏祿部置陰山大漠二都督府|{
	以謀落部置隂山府以熾俟部置大漠府俱屬北庭都護府邏郎佐翻}
閏月壬寅上行幸洛陽 庚戌以左屯衛將軍蘇定方爲伊麗道行軍總管|{
	伊麗河一名帝帝河}
帥燕然都護渭南任雅相|{
	燕然都護府在黄河北北至隂山七十里至回紇界七百里去京師二千七百里龍朔三年改曰瀚海都督府總章二年改爲安北大都護府杜佑曰後爲中受降城南去朔方千三百餘里後魏於渭南置渭南郡隋廢爲縣屬京兆 帥讀曰率燕因肩翻任音壬相息亮翻}
副都護蕭嗣業發回紇等兵自北道討西突厥沙鉢羅可汗嗣業鉅之子也|{
	蕭鉅見一百八十一卷隋煬帝大業六年}
初右衛大將軍阿史那彌射及族兄左屯衛大將軍步眞皆西突厥酋長|{
	酋慈由翻長知兩翻}
太宗之世帥衆來降|{
	彌射室點密可汗五世孫世爲莫賀咄葉護貞觀中遣使立爲可汗族兄步眞謀殺彌射而自立彌射不能國即入朝步眞遂自立爲咄陸葉護衆不厭去之因亦與族人入朝帥讀曰率降戶江翻}
至是詔以彌射步眞爲流沙安撫大使 |{
	考異曰舊西突厥咄陸傳咄陸可汗泥熟父莫賀設貞觀七年遣鴻臚少卿劉善因冊爲吐谷妻狀奚利苾咄陸可汗明年泥熟卒弟同娥設立爲咥利失可汗彌射傳云彌射者室點密可汗五代孫也世統十姓部落在本蕃爲莫賀咄葉護貞觀六年詔遣鴻臚少卿劉善因就蕃立爲奚利邲咄陸可汗其族兄步眞欲自立謀殺彌射彌射既與步眞有隙以貞觀十三年率所部處月處密部落入朝其後步眞遂自立爲咄陸葉護部落不服步眞復携家屬入朝彌射後從太宗征高麗有功封平襄縣伯顯慶二年轉右武衛大將軍新傳略同今欲以咄陸彌射爲二人則事多相類以爲一人則事又相違疑不能明故但云西突厥酋長 余按彌射爲咄陸可汗唐所冊也步眞爲咄陸葉護自稱也咄陸之號雖同而可汗葉護位之尊卑有異不必泥咄陸之號而傳疑而彌射步眞實二人也余前注所引者新傳也其辭略考異所引者舊傳也其辭詳大略同也又參考新舊書劉善因冊可汗事與通鑑有六年七年之差而新舊書可汗號有婁抜妻狀之差舊書又多一奚字而貞觀中立彌射爲奚利邲咄陸可汗則新舊書同詳而考之劉善因冊泥孰爲奚利邲咄陸可汗明年而泥孰死弟同娥設立爲沙鉢羅咥利失可汗又三年而咥利失不爲衆所歸西部又立欲谷設爲乙毗咄陸可汗二可汗兵爭咄利失乙毗相繼走死他國而射匱寔承之太宗崩賀魯反而射匱爲賀魯所并西突厥世次曉然可考而新舊書於彌射傳皆云貞觀中遣劉善因立彌射爲奚利邲咄陸可汗以泥敦傳觀之則善因所立者泥孰也以彌射傳觀之則善因所立者彌射也考異所疑當以此耳}
自南道招集舊衆 二月辛酉車駕至洛陽宫 庚午立皇子顯爲周王壬申徙雍王素節爲郇王|{
	雍於用翻郇音荀}
三月甲辰以潭州都督禇遂良爲桂州都督|{
	桂州至京師水陸路四千七百六十里}
癸丑以李義府兼中書令 夏五月丙申上幸明德宫避暑上自即位每日視事庚子宰相奏天下無虞請隔日視事許之 秋七月丁亥朔上還洛陽宫 王玄策之破天竺也|{
	見上卷貞觀二十二年}
得方士那羅邇娑婆寐以歸|{
	娑素何翻}
自言有長生之術太宗頗信之深加禮敬使合長生藥|{
	太宗令娑婆寐於金飊門合延年藥合音閤}
發使四方求奇藥異石又發使詣婆羅門諸國采藥|{
	使疏吏翻}
其言率皆迂誕無實苟欲以延歲月藥竟不就乃放還上即位復詣長安|{
	復扶又翻}
又遣歸玄策時爲道王友|{
	道王元慶高祖之子唐諸王府置友一人從五品下掌陪侍規諷}
辛亥奏言此婆羅門實能合長年藥自謂必成今遣歸可惜失之玄策退上謂侍臣曰自古安有神仙秦始皇漢武帝求之疲弊生民卒無所成|{
	卒子恤翻}
果有不死之人今皆安在李勣對曰誠如聖言此婆羅門今兹再來容髪衰白已改於前何能長生陛下遣之内外皆喜娑婆寐竟死於長安 許敬宗李義府希皇后旨誣奏侍中韓瑗中書令來濟與禇遂良潛謀不軌以桂州用武之地授遂良桂州都督欲以爲外援八月丁卯瑗坐貶振州刺史濟貶台州刺史終身不聽朝覲|{
	台州漢回浦縣地光武改回浦爲章安縣吳孫亮分會稽東部都尉爲臨海郡治章安江左皆因之隋平陳廢爲臨海縣屬永嘉郡唐武德四年分置台州諸州刺史有朝集故禁絶二人不得至京師振州至京師八千六百六里台州在京師東南四千一百七十七里朝直遙翻}
又貶禇遂良爲愛州刺史榮州刺史柳奭爲象州刺史|{
	榮州至京師二千九百七十二里象州至京師四千九百八十九里 考異曰唐歷三月甲辰貶遂良爲桂州都督奭愛州刺史據實錄奭坐韓瑗貶象州新舊書唐歷皆云愛州誤也今從實錄}
遂良至愛州上表自陳|{
	上時掌翻}
往者濮王承乾交爭之際臣不顧死亡歸心陛下時岑文本劉洎奏稱承乾惡狀已彰身在别所其於東宮不可少時虛曠|{
	少詩沼翻}
請且遣濮王往居東宫臣又抗言固爭皆陛下所見卒與無忌等四人共定大策|{
	事見一百九十七卷貞觀十七年卒子恤翻}
及先朝大漸獨臣與無忌同受遺詔|{
	見上卷貞觀二十三年朝直遙翻}
陛下在草土之辰不勝哀慟臣以社稷寛譬陛下手抱臣頸臣與無忌區處衆事咸無廢闕數日之間内外寜謐力小任重動罹愆過螻蟻餘齒乞陛下哀憐表奏不省|{
	勝音升處昌呂翻省悉景翻}
己巳禮官奏四郊迎氣存太微五帝之祀南郊明堂廢緯書六天之義其方丘祭地之外别有神州亦請合爲一祀從之|{
	歐陽修曰禮曰以禋祀祀昊天上帝此天也鄭玄以爲天皇大帝者北辰耀魄寶也又曰兆五帝於四郊此五行精氣之神也玄以爲青帝靈威仰赤帝赤熛怒黄帝含樞紐白帝白招矩黑帝叶光紀者五天也由是有六天之說唐初貞觀禮冬至祀昊天上帝於圓丘正月辛日祀感生帝靈威仰於南郊以祈穀而孟夏雩於南郊季春大享於明堂皆祀五天帝至高宗時禮官以謂太史圓丘圖昊天上帝在壇上而耀魄寶在壇第一等則昊天上帝非耀魄寶可知許敬宗與禮官議曰六天出於緯書而南郊圓丘一也玄以爲二物郊及明堂本以祭天而玄皆以爲祭太微五帝傳曰凡祀啓蟄而郊郊而後耕故郊祀后稷以祈農事而玄謂周祭感帝靈威仰配以后稷因而祈穀皆繆說也由是盡黜玄說又武德中冬至及孟夏雩祭皇地祗於方丘神州地祗於北郊今亦合爲一祀}
辛未以禮部尚書許敬宗爲侍中兼度支尚書杜正倫爲兼中書令 冬十月戊戌上行幸許州|{
	許州漢頴川郡地東魏立南鄭州後周改許州因古許國以名州也至京師一千三百里至東都四百里}
乙巳畋于滍水之南壬子至汜水曲|{
	滍直凡翻汜水曲在鄭州新鄭縣界師古曰汜舊音凡今俗讀爲祀}
十二月乙卯朔車駕還洛陽宫 蘇定方擊西突厥沙鉢羅可汗|{
	厥九勿翻可從刋入聲汗音寒}
至金山北先擊處木昆部大破之其俟斤嬾獨祿等帥萬餘帳來降|{
	俟渠之翻帥讀曰率下同降戶江翻}
定方撫之發其千騎與俱右領軍郎將薛仁貴上言泥孰部素不伏賀魯|{
	泥孰部弩失畢五俟斤之一也騎奇寄翻下同將即亮翻下同上時掌翻}
爲賀魯所破虜其妻子今唐兵有破賀魯諸部得泥孰妻子者宜歸之仍加賜賚使彼明知賀魯爲賊而大唐爲之父母則人致其死不遺力矣上從之泥孰喜請從軍共擊賀魯定方至曳咥河西|{
	曳咥河在伊麗河東}
沙鉢羅帥十姓兵且十萬來拒戰|{
	咄陸五啜弩失畢五俟斤是爲西突厥十姓}
定方將唐兵及回紇萬餘人擊之沙鉢羅輕定方兵少直進圍之|{
	紇下没翻少詩沼翻}
定方令步兵據南原攅稍外向|{
	稍色角翻}
自將騎兵陳於北原|{
	陳讀曰陣下布陳同}
沙鉢羅先攻步軍三衝不動定方引騎兵擊之沙鉢羅大敗追奔三十里斬獲數萬人明日勒兵復進|{
	復扶又翻下可復同}
於是胡禄屋等五弩失畢悉衆來降沙鉢羅獨與處木昆屈律啜數百騎西走|{
	降戶江翻處昌呂翻啜陟劣翻騎奇寄翻}
時阿史那步眞出南道五咄陸部落聞沙鉢羅敗皆詣步眞降定方乃命蕭嗣業回紇婆閏將胡兵趨邪羅斯川|{
	舊書賀魯居多邏斯川在西州直北一千五百里此邪羅斯川當在伊麗水之西咄當没翻嗣祥吏翻紇下没翻趨讀曰趣音七喻翻邪讀曰耶}
追沙鉢羅定方與任雅相將新附之衆繼之|{
	任音壬相息亮翻將即亮翻又音如字}
會大雪平地二尺軍中咸請俟晴而行定方曰虜恃雪深謂我不能進必休息士馬亟追之可及若緩之彼遁逃浸遠不可復追省日兼功在此時矣乃蹋雪晝夜兼行所過收其部衆至雙河與彌射步眞合去沙鉢羅所居二百里布陳長驅徑至其牙帳|{
	賀魯牙帳在金牙山直石國東北復扶又翻陳讀曰陣}
沙鉢羅與其徒將獵定方掩其不備縱兵擊之斬獲數萬人得其䜴纛|{
	纛徒到翻又徒沃翻}
沙鉢羅與其子咥運壻閻啜等脱走趣石國|{
	咥徒結翻啜陟劣翻趣七喻翻}
定方於是息兵諸部各歸所居通道路置郵驛掩骸骨問疾苦畫疆場復生業凡爲沙鉢羅所掠者悉括還之十姓安堵如故乃命蕭嗣業將兵追沙鉢羅定方引軍還沙鉢羅至石國西北蘇咄城|{
	場音亦嗣祥吏翻還從宣翻又音如字咄當没翻}
人馬飢乏遣人齎珍寶入城市馬城主伊沮達官|{
	沮子余翻}
詐以酒食出迎誘之入|{
	誘音酉}
閉門執之送於石國蕭嗣業至石國石國人以沙鉢羅授之乙丑分西突厥地置濛池崑陵二都護府|{
	濛池都護府居碎葉川西崑陵都護府居碎葉川東考異曰舊書賀魯傳云定方行至曳咥河西賀魯率胡祿居闕啜等二萬餘騎列陳而待定方率任雅相等與之交戰賊衆大敗斬大首領都搭達官等二百餘人賀魯及闕啜輕騎奔竄渡伊西麗河兵馬溺死者甚衆彌射進軍至伊麗水處月處密等部各帥衆來降彌射又進次雙河賀魯先使步失達官鳩集散卒據柵拒戰彌射步眞攻之大潰又與蘇定方攻賀魯於碎葉水大破之舊書本紀三年二月定方平賀魯甲寅西域平以其地置濛池崑陵二都督府據實錄擒賀魯置二都督皆在此月本紀又非奏到月日今從實錄}
以阿史那彌射爲左衛大將軍崑陵都護興昔亡可汗押五咄陸部落阿史那步眞爲右衛大將軍濛池都護繼往絶可汗押五弩失畢部落遣光祿卿盧承慶持節冊命仍命彌射步眞與承慶據諸姓降者凖其部落大小位望高下授刺史以下官丁卯以洛陽宫爲東都|{
	唐六典洛陽宫在東都皇城之北東西四里一百八十步南北二里八十五步周回十三里二百四十一步}
洛州官吏員品並如雍州|{
	雍於用翻}
是歲詔自今僧尼不得受父母及尊者禮拜|{
	尼女夷翻}
所

司明有法制禁斷|{
	有當作爲斷音短}
以吏部侍郎劉祥道爲黄門侍郎仍知吏部選事|{
	選須絹翻下同}
祥道以爲今選司取士傷濫每年入流之數過一千四百雜色入流曾不銓簡|{
	雜色補官者謂之流外官入流内叙品謂之入流}
即日内外文武官一品至九品凡萬三千四百六十五員約凖三十年則萬三千餘人略盡矣|{
	即日者即今日也}
若年别入流者五百人|{
	別彼列翻}
足充所須之數望有釐革既而杜正倫亦言入流人太多上命正倫與祥道詳議而大臣憚於改作事遂寢祥道林甫之子也|{
	劉林甫貞觀初爲吏部侍郎請四時聽選}


三年春正月戊子長孫無忌等上所修新禮詔中外行之|{
	上時掌翻}
先是議者謂貞觀禮節文未備|{
	先悉薦翻}
故命無忌等修之時許敬宗李義府用事所損益多希旨學者非之太常博士蕭楚材等以爲豫備凶事非臣子所宜言敬宗義府深然之遂焚國恤一篇由是凶禮遂闕|{
	唐制太常博士從七品上掌六禮之儀式本先王之法制適變隨時而損益焉六禮既闕凶禮遂爲五禮焉}
初龜兹王布失畢妻阿史那氏與其相那利私通布失畢不能禁|{
	布失畢歸國見上卷永徽元年龜兹音丘慈又音屈隹}
由是君臣猜阻各有黨與互來告難|{
	難乃旦翻}
上兩召之既至囚那利遣左領軍郎將雷文成送布失畢歸國|{
	十四衛郎將正五品上}
至龜兹東境泥師城龜兹大將羯獵顚發衆拒之仍遣使降於西突厥沙鉢羅可汗|{
	使疏吏翻}
布失畢據城自守不敢進詔左屯衛大將軍楊胄發兵討之會布失畢病卒胄與羯獵顚戰大破之擒羯獵顚及其黨盡誅之乃以其地爲龜兹都督府戊申立布失畢之子素稽爲龜兹王兼都督二月丁巳上發東都甲戌至京師 夏五月癸未徙

安西都護府於龜兹以舊安西復爲西州都督府鎮高昌故地|{
	貞觀十四年平高昌置安西都護府於交河城今徙於龜兹}
六月營州都督兼東夷都護程名振右領軍中郎將薛仁貴將兵攻高麗之赤烽鎮拔之斬首四百餘級捕虜百餘人高麗遣其大將豆方婁帥衆三萬拒之名振以契丹逆擊大破之斬首二千五百級 |{
	考異曰舊書仁貴傳云顯慶二年副程名振經略遼東破高麗於貴端城斬首三千級今從實錄}
秋八月甲寅播羅哀獠酋長多胡桑等帥衆内附|{
	播羅哀羅竇生獠部落之名獠魯皓翻酋慈由翻長知兩翻}
冬十月庚申吐蕃贊普來請婚 中書令李義府有寵於上諸子孩抱者並列清貴而義府貪冒無厭|{
	冒莫北翻厭於鹽翻}
母妻及諸子女壻賣官鬻獄其門如市多樹朋黨傾動朝野|{
	朝直遙翻}
中書令杜正倫每以先進自處|{
	處昌呂翻}
義府恃恩不爲之下由是有隙與義府訟於上前上以大臣不和兩責之十一月乙酉貶正倫横州刺史義府普州刺史正倫尋卒於横州|{
	横州漢廣鬱高梁之地晉武帝太康七年置寜浦郡梁分置簡陽郡隋置簡州大業廢爲寜浦縣屬鬱林郡唐武德初復置南簡州貞觀八年更名横州至京師五千五百三十九里東都四千七百五里普州漢牛鞞墊江資中三縣地後周置安岳縣并置普州隋廢州以縣屬資陽郡唐武德二年復置普州至京師二千三百六十里至東都三千二百三里卒子恤翻}
阿史那賀魯既被擒謂蕭嗣業曰我本亡虜爲先帝所存|{
	事見上卷貞觀二十二年被皮義翻}
先帝遇我厚而我負之今日之敗天所怒也吾聞中國刑人必於市願刑我於昭陵之前以謝先帝上聞而憐之賀魯至京師甲午獻於昭陵敕免其死分其種落爲六都督府|{
	以處木昆部爲匐延都督府突騎施索葛莫賀部爲嗢鹿都督府胡祿屋闕部爲鹽洎都督府攝舍提暾部爲雙河都督府鼠尼施處半部爲鷹娑都督府突騎施阿利施部爲潔山都督府種章勇翻}
其所役屬諸國皆置州府西盡波斯並隸安西都護府|{
	四鎮都督府州三十四西域都督府十六州七十二}
賀魯尋死葬於頡利墓側 戊戌以許敬宗爲中書令大理卿辛茂將爲兼侍中 開府儀同三司鄂忠武公尉遲敬德薨|{
	尉紆勿翻}
敬德晩年閑居學延年術修飾池臺奏清商樂以自奉養不交通賓客凡十六年年七十四以病終朝廷恩禮甚厚是歲愛州刺史禇遂良卒 雍州司士許禕與來濟

善|{
	唐雍州士曹司士參軍事正七品下掌津梁舟車舍宅工藝雍於用翻禕吁韋翻}
侍御史張倫與李義府有怨吏部尚書唐臨奏以禕爲江南道巡察使倫爲劒南道巡察使|{
	使疏吏翻}
是時義府雖在外皇后常保護之以臨爲挾私選授

四年春二月乙丑免臨官 三月壬午西突厥興昔亡可汗與眞珠葉護戰於雙河斬眞珠葉護|{
	眞珠葉護事始上卷永徽四年}
夏四月丙辰以于志寧爲太子太師同中書門下三品乙丑以黄門侍郎許圉師參知政事 |{
	考異曰舊傳云二年同中書門下三品新傳無年今從實錄 按考異所謂舊傳斬傳皆許圉師傳也}
武后以太尉趙公長孫無忌受重賜而不助已|{
	事見上卷永徽五年}
深怨之及議廢王后燕公于志寧中立不言|{
	事見上卷永徽六年燕因肩翻}
武后亦不悅許敬宗屢以利害說無忌無忌每面折之敬宗亦怨|{
	說輸芮翻折之舌翻}
武后既立無忌内不自安后令敬宗伺其隙而陷之|{
	伺相吏翻}
會洛陽人李奉節告太子洗馬韋季方監察御史李巢朋黨事|{
	洗悉薦翻監古銜翻}
敕敬宗與辛茂將鞫之敬宗按之急季方自刺不死|{
	刺七亦翻}
敬宗因誣奏季方欲與無忌構陷忠臣近戚使權歸無忌伺隙謀反今事覺故自殺上驚曰豈有此邪舅爲小人所間|{
	間古莧翻}
小生疑阻則有之何至於反敬宗曰臣始末推究反狀已露陛下猶以爲疑恐非社稷之福 |{
	考異曰實録洛陽人李奉節上封事告太子洗馬韋季方監察御史李巢交通朝貴有朋黨之事詔敬宗與辛茂將鞫之敬宗按之甚急季方事迫自刺不死又搜奉節得私書有題與趙師者遂奏言趙師即無忌也隂爲隱語欲陷忠良伺隙謀反上驚曰豈當有此或容惡人間構小生疑阻至於即反猶恐不然敬宗奏曰臣始末推勘自奉節有趙師之言又得僞書是季方所作即疑無忌欲反使其潛行構間斥除忠臣近戚此計若行自然權歸無忌蹤跡已露陛下猶有所疑恐非社稷之福舊無忌傳云敬宗使人上封事稱監察御史李巢與無忌交通謀反詔敬宗與茂將鞫之唐歷統紀與實錄略同按奉節乃告事之人推鞫者豈得反搜奉節之家且與趙師者誰之私書若是季方書安得在奉節家若在奉節家奉節當執以興訟何待搜而後得又既云趙師是無忌乃是實與無忌書何得謂之僞書實錄叙此事殊鹵莾首尾差舛不可知其詳實故略取大意而已舊傳所云雖爲簡徑然高宗初無疑無忌之心故李弘泰告無忌反高宗立斬之何至奉節而獨令敬宗鞫之也且實錄在前而詳列傳在後而略故亦未可據也}
上泣曰我家不幸親戚間屢有異志往年高陽公主與房遺愛謀反|{
	事見上卷永徽三年}
今元舅復然|{
	復扶又翻}
使朕慙見天下之人兹事若實如之何對曰遺愛乳臭兒與一女子謀反勢何所成無忌與先帝謀取天下天下服其智爲宰相三十年|{
	無忌自貞觀初爲相至是三十餘年}
天下畏其威若一旦竊發陛下遣誰當之今賴宗廟之靈皇天疾惡因按小事乃得大姦實天下之慶也臣竊恐無忌知季方自刺窘急發謀攘袂一呼|{
	呼火故翻}
同惡雲集必爲宗廟之憂臣昔見宇文化及父述爲煬帝所親任結以婚姻委以朝政述卒化及復典禁兵|{
	卒子恤翻復扶又翻下宗復同}
一夕於江都作亂先殺不附己者臣家亦豫其禍於是大臣蘇威裴矩之徒皆舞蹈馬首唯恐不及黎明遂傾隋室|{
	事見一百八十六卷高祖武德元年}
前事不遠願陛下速决之上命敬宗更加審察明日敬宗復奏曰昨夜季方已承與無忌同反臣又問季方無忌與國至親累朝寵任何恨而反季方答云韓瑗嘗語無忌云|{
	語牛倨翻}
柳奭禇遂良勸公立梁王爲太子今梁王既廢上亦疑公故出高履行於外|{
	履行無忌舅子也去年出爲益州長史}
自此無忌憂恐漸爲自安之計後見長孫祥又出韓瑗得罪日夜與季方等謀反臣參驗辭狀咸相符合請收捕準法上又泣曰舅若果爾朕决不忍殺之天下將謂朕何後世將謂朕何敬宗對曰薄昭漢文帝之舅也文帝從代來昭亦有功所坐止於殺人文帝使百官素服哭而殺之|{
	事見漢文帝紀}
至今天下以文帝爲明主今無忌忘兩朝之大恩謀移社稷其罪與薄昭不可同年而語也幸而姦狀自發逆徒引服陛下何疑猶不早决古人有言當斷不斷反受其亂|{
	道家之言斷丁亂翻}
安危之機間不容髮無忌今之姦雄王莽司馬懿之流也陛下少更遷延|{
	少詩沼翻}
臣恐變生肘腋悔無及矣上以爲然竟不引問無忌戊辰下詔削無忌太尉及封邑以爲楊州都督於黔州安置凖一品供給|{
	唐六典膳部郎中一品食料每日細白米二升粳米梁米各一斗五升粉一升油五升鹽一升半醋三升蜜三合粟一斗棃七顆蘇一合乾棗一升木橦十根炭十斤䓤韭䜴䔉薑椒之類各有差每月給羊二十口猪肉六十斤魚三十頭各一尺酒九斗黔音琴}
祥無忌之從父兄子也|{
	從才用翻}
前此自工部尚書出爲荆州長史故敬宗以此誣之敬宗又奏無忌謀逆由禇遂良柳奭韓瑗構扇而成奭仍潛通宫掖謀行鴆毒于志寜亦黨附無忌於是詔追削遂良官爵除奭瑗名免志寜官|{
	于志寜欲以緘默免禍而卒不免不若禇遂良無愧於昭陵也}
遣使發道次兵援送無忌詣黔州|{
	使疏吏翻}
無忌子秘書監駙馬都尉冲等皆除名流嶺表|{
	長孫冲尚太宗女長樂公主}
遂良子彥甫彥冲流愛州於道殺之益州長史高履行累貶洪州都督|{
	自大都督府長史爲遠州都督爲貶益州在京師西南二千三百七十九里東都三千二百一十六里洪州在京師東南三千九十里至東都二千二百一十一里 考異曰舊傳云三年誤也今從唐歷}
五月丙申兵部尚書任雅相度支尚書盧承慶並參知政事承慶思道之孫也|{
	盧思道仕於高齊以文稱任音壬相息亮翻度徒洛翻}
凉州刺史趙持滿多力善射喜任俠|{
	喜許記翻}
其從母爲韓瑗妻|{
	從才用翻下之從同}
其舅駙馬都尉長孫銓無忌之族弟也銓坐無忌流嶲州|{
	嶲音髓}
許敬宗恐持滿作難|{
	難乃旦翻}
誣云無忌同反|{
	誣云之下恐脱與字}
驛召至京師下獄訊掠備至終無異辭曰身可殺也辭不可更|{
	下遐嫁翻掠音亮更工衡翻}
吏無如之何乃代爲獄辭結奏|{
	結奏結其罪而奏之}
戊戌誅之尸於城西親戚莫敢視友人王方翼歎曰欒布哭彭越義也|{
	事見十二卷漢高帝十一年}
文王葬枯骨仁也下不失義上不失仁不亦可乎乃收而葬之上聞之不罪也方翼廢后之從祖兄也長孫銓至流所縣令希旨杖殺之 六月丁卯詔改氏族志爲姓氏錄初太宗命高士亷等修氏族志|{
	事見一百九十五卷貞觀十二年}
升降去取時稱允當|{
	當丁浪翻}
至是許敬宗等以其書不叙武氏本望奏請改之乃命禮部郎中孔志約等比類升降以后族爲第一等其餘悉以仕唐官品高下爲凖凡九等於是士卒以軍功致位五品豫士流時人謂之勲格 許敬宗議封禪儀己巳奏請以高祖太宗俱配昊天上帝太穆文德二皇后俱配皇地祗從之 秋七月命御史往高州追長孫恩象州追柳奭振州追韓瑗並枷鎻詣京師仍命州縣簿錄其家恩無忌之族弟也壬寅命李勣許敬宗辛茂將與任雅相盧承慶更共覆按無忌事|{
	考異曰唐歷是日以台州刺史來濟爲庭州刺史來濟與韓瑗事同一體瑗方下獄濟豈得移官舊書云五年徙庭州近是}
許敬宗又遣中書舍人袁公瑜等詣黔州再鞫無忌反狀|{
	黔音琴}
至則逼無忌令自縊詔柳奭韓瑗所至斬决使者殺柳奭於象州 |{
	考異曰舊傳云奭累貶愛州刺史高宗就愛州殺之今從實錄}
韓瑗已死發驗而還|{
	發棺而驗其尸還從宣翻又如字 考異曰舊瑗傳云四年卒官明年長孫無忌死遣使殺之使至瑗已死禇遂良傳三年卒官後二歲追削官爵實錄或因無忌徙黔州終言之然諸書多在此月蓋因實錄年代記云七月辛未遣使逼無忌自縊按長歷七月丙子朔無辛未不可據也}
籍没三家近親皆流嶺南爲奴婢常州刺史長孫祥坐與無忌通書處絞|{
	常州在京師東南二千八百四十三里至東都一千九百八十三里處昌呂翻}
長孫恩流檀州|{
	檀州漢漁陽郡蹏奚縣地後魏置安州後周改曰玄州隋開皇十六年改檀州大業初廢州爲安樂郡唐復爲檀州在京師東北二千五百六十七里至東都一千八百四十四里考異曰唐統紀唐歷皆云長孫恩新書云族弟恩統紀唐歷長孫詮流嶲州縣令希旨殺之在此下實録詮}


|{
	流嶲州許敬宗懼其甥趙持滿作難遂殺持滿是詮流嶲州在前今從之}
八月壬子以普州刺史李義府兼吏部尚書同中書門下三品義府既貴自言本出趙郡與諸李叙昭穆|{
	昭市招翻}
無賴之徒藉其權勢拜伏爲兄叔者甚衆給事中李崇德初與同譜及義府出爲普州即除之義府聞而銜之及復爲相|{
	復扶又翻}
使人誣構其罪下獄自殺|{
	下遐稼翻}
乙卯長孫氏柳氏緣無忌奭貶降者十三人高履行貶永州刺史|{
	永州舊零陵郡隋平陳置永州在京師南三千二百七十四里至東都三千六百六十五里}
于志寜貶榮州刺史于氏貶者九人自是政歸中宫矣 九月詔以石米史大安小安曹拔汗那悒怛疎勒朱駒半等國置州縣府百二十七|{
	米國或曰彌末或曰弭秣賀北百里距康國其王治鉢息德城大安一曰布豁又曰捕喝元魏謂忸密者東北至小安四百里西瀕烏滸河治河謐城即康居小君長罽王故地小安一曰東安曰喝汗在那密水之陽東距河二百里許治喝汗城曹亦有東西中國東曹居波悉山之陰漢貳師城地也北至石西至康東北寜遠皆四百里西曹者隋時曹國也南接史及波覽治瑟底痕城中曹治迦底眞城抜汗那即寧遠或曰鏺汗元魏所謂破洛那去京師八千里居西鞬城在真珠河之北後分爲二一治呼悶城一治遏塞城悒怛國漢大月氏之種大月氏為烏孫所奪西過大宛擊大夏臣之大夏即吐火羅也嚈噠王姓也後世以姓為國訛為悒怛杜佑曰嚈噠或云高車之别種或云大月氏之别種悒怛亦大月氏别種如佑所云則嚈噠悒怛似是兩國疎勒一名佉沙距長安九千里而贏悒音邑怛當割翻}
冬十月丙午太子加元服赦天下|{
	太子加元服其儀備見於新書禮志}
初太宗疾山東士人自矜門地昏姻多責資財命修氏族志例降一等王妃主壻皆取勲臣家不議山東之族而魏徵房玄齡李勣家皆盛與爲昏常左右之|{
	左右讀曰佐佑}
由是舊望不減或一姓之中更分某房某眷高下懸隔李義府爲其子求昏不獲恨之|{
	爲丁偽翻}
故以先帝之旨勸上矯其弊壬戌詔後魏隴西李寶太原王瓊榮陽鄭温范陽盧子遷盧渾盧輔清河崔宗伯崔元孫前燕博陵崔懿晉趙郡李楷等子孫不得自爲昏姻|{
	燕因肩翻}
仍定天下嫁女受財之數毋得受陪門財|{
	陪門財者女家門望素高而議姻之家非耦令其納財以陪門望}
然族望爲時所尚終不能禁或載女竊送夫家或女老不嫁終不與異姓爲昏其衰宗落譜昭穆所不齒者往往反自稱禁昏家益增厚價|{
	厚取陪門之財也昭市招翻}
閏月戊寅上發京師令太子監國太子思慕不已|{
	人少則慕父母}
上聞之遽召赴行在戊戌車駕至東都|{
	監古銜翻}
十一月丙午以許圉師爲散騎常侍檢校侍中|{
	散悉亶翻騎奇寄翻}
戊午侍中兼左庶子辛茂將薨 思結俟斤都曼帥疎勒朱俱波謁般陁三國反|{
	新書作喝盤陀或曰漢陀曰渇館檀亦謂渇羅陀由疎勒西南入劒末谷不忍嶺六百里則其國也距瓜州四千五百里直朱俱波西南距懸度山俟渠之翻帥讀曰率}
擊破于闐癸亥以左驍衛大將軍蘇定方爲安撫大使以討之|{
	驍堅堯翻使疏吏翻}
以盧承慶同中書門下三品 右領軍中郎將薛仁貴等與高麗將温沙門戰於横山破之|{
	將即亮翻麗力知翻}
蘇定方軍至業葉水|{
	自庭州輪臺縣西行三百許里至業葉河}
思結保馬頭川定方選精兵萬人騎三千匹馳往襲之一日一夜行三百里詰旦至城下|{
	騎奇寄翻詰去吉翻}
都曼大驚戰於城外都曼敗退保其城及暮諸軍繼至遂圍之都曼懼而出降|{
	降戶江翻}


五年春正月定方獻俘於乾陽殿|{
	乾陽殿在洛陽宫}
法司請誅都曼定方請曰臣許以不死故都曼出降願匄其餘生|{
	匄古太翻}
上曰朕屈法以全卿之信乃免之 甲子上發東都二月辛巳至并州|{
	東都至并州八百八里}
三月丙午皇后宴親戚故舊鄰里於朝堂婦人於内殿班賜有差|{
	武后并州文水縣人天子行幸所至皆有朝堂太宗伐高麗張受降幕於朝堂之側是也皇后所居爲内殿朝直遥翻}
詔并州婦人年八十以上皆版授郡君|{
	郡君有正四品從四品正五品之差}
百濟恃高麗之援數侵新羅|{
	數所角翻}
新羅王春秋上表求救辛亥以左武衛大將軍蘇定方爲神丘道行軍大總管|{
	新書作神兵道}
帥左驍衛將軍劉伯英等|{
	帥讀曰率驍堅堯翻}
水陸十萬以伐百濟 |{
	考異曰舊書定方傳新羅傳皆云定方爲熊津道大總管實錄定方傳亦同今從此年實録新唐書本紀又舊本紀唐歷皆云四年十二月癸亥以定方爲神丘道大總管劉伯英為嵎夷道行軍總管按定方時討都曼未為神丘道總管舊書唐歷皆誤今從實録}
以春秋爲嵎夷道行軍總管|{
	因堯典宅嵎夷曰暘谷而命之}
將新羅之衆與之合勢|{
	將即亮翻}
夏四月丙寅上發并州癸巳至東都五月作合璧宫

|{
	時改八關宫爲合璧宫在東都苑内}
壬戌上幸合璧宫 戊辰以定襄都督阿史德樞賓左武候將軍延陁梯眞|{
	梯眞薛延陁之種故以為姓}
居延州都督李合珠並爲冷岍道行軍總管|{
	岍與同即令徑山奚與契丹依阻此山以自固其地在潢水之南黄龍之北}
各將所部兵以討叛奚仍命尚書右丞崔餘慶充使總護三處兵奚尋遣使降|{
	將即亮翻下同使疏吏翻降戶江翻}
更以樞賓等爲沙磚道行軍總管以討契丹擒契丹松漠都督阿卜固送東都|{
	磚職緣翻}
六月庚午朔日有食之 甲午車駕還洛陽宫 房州刺史梁王忠年浸長頗不自安或私衣婦人服以備刺客|{
	長知兩翻衣於既翻}
又數自占吉凶|{
	數所角翻}
或告其事秋七月乙巳廢忠爲庶人徙黔州囚於承乾故宅|{
	太宗貞觀十七年徙太子承乾於黔州黔音琴}
丁卯度支尚書同中書門下三品盧承慶坐科調失

所免官|{
	度支尚書凡徭賦職貢之方經費賙給之算藏貨贏縮之準悉以咨之今科調不得其所為不任其職故免所居官調徒弔翻}
八月吐蕃禄東贊遣其子起政將兵擊吐谷渾以吐谷渾内附故也|{
	吐從暾入聲谷音浴}
蘇定方引兵自成山濟海百濟據熊津江口以拒之定方進擊破之百濟死者數千人餘皆潰走定方水陸齊進直趣其都城|{
	北史百濟都俱抜城亦曰固麻城其外更有五方中方曰古沙城東方曰得安城南方曰久知下城西方曰刀先城北方曰熊津城趣七喻翻}
未至二十餘里百濟傾國來戰大破之殺萬餘人追奔入其郭百濟王義慈及太子隆逃於北境定方進圍其城義慈次子泰自立爲王帥衆固守隆子文思曰王與太子皆在而叔遽擁兵自王|{
	帥讀曰率下同王于况翻}
借使能却唐兵我父子必不全矣遂帥左右踰城來降百姓皆從之泰不能止定方命軍士登城立幟泰窘迫開門請命於是義慈隆及諸城主皆降|{
	降戶江翻幟昌志翻}
百濟故有五部分統三十七郡二百城七十六萬戶詔以其地置熊津五都督府|{
	熊津馬韓東明金連德安五都督府}
以其酋長爲都督刺史|{
	酋慈由翻長知兩翻}
壬午左武衛大將軍鄭仁泰將兵討思結抜也固僕骨同羅四部|{
	拔也固即拔野古僕骨即僕固將即亮翻}
三戰皆捷追奔百餘里斬其酋長而還 冬十月上初苦風眩頭重目不能視百司奏事上或使皇后决之后性明敏涉獵文史處事皆稱旨由是始委以政事權與人主侔矣|{
	史言后移唐祚至是而勢成處昌呂翻稱尺證翻}
十一月戊戌朔上御則天門樓|{
	唐六典東都宫城南面三門中曰應天後以武后號則天遂更曰應天也}
受百濟俘自其王義慈以下皆釋之蘇定方前後滅三國皆生擒其主|{
	謂賀魯都曼義慈也}
赦天下 甲寅上幸許州十二月辛未畋於長社|{
	長社漢古縣屬頴川郡隋改長社曰頴川唐復舊帶許州}
己卯還東都 壬午以左驍衛大將軍契苾何力爲浿江道行軍大總管|{
	浿水在高麗國中驍堅堯翻契欺訖翻浿普蓋翻}
左武衛大將軍蘇定方爲遼東道行軍大總管左驍衛將軍劉伯英爲平壤道行軍大總管蒲州刺史程名振爲鏤方道總管將兵分道擊高麗青州刺史劉仁軌坐督海運覆船以白衣從軍自効 |{
	考異曰舊傳云監統水軍征遼以後期坐免官按仁軌從軍乃在百濟非征遼也今從張鷟朝野僉載}


龍朔元年春正月乙卯募河南北淮南六十七州兵得四萬四千餘人詣平壤鏤方行營戊午以鴻臚卿蕭嗣業爲扶餘道行軍總管帥回紇等諸部兵詣平壤|{
	臚陵如翻帥讀曰率紇下没翻}
二月乙未晦改元 三月丙申朔上與羣臣及外夷宴於洛城門|{
	唐六典洛陽宫城西北出曰洛城西門其内曰德昌殿德昌殿南出曰延慶門又南曰韶暉門西南曰洛城南門其内曰洛城殿}
觀屯營新教之舞謂之一戎大定樂|{
	取一戎衣天下大定之義舞者百四十人被五采甲持槊而舞劉昫曰大定樂出自破陳樂自破陳舞以下皆雷大鼓雜以龜兹之樂聲振百里動蕩山谷大定樂加金鉦象平遼東而邊隅大定也杜佑曰大定樂歌云八紘同軌樂}
時上欲親征高麗以象用武之勢也 初蘇定方既平百濟留郎將劉仁願鎮守百濟府城又以左衛中郎將王文度爲熊津都督撫其餘衆文度濟海而卒|{
	卒子恤翻}
百濟僧道琛故將福信聚衆據周留城|{
	將即亮翻}
迎故王子豐於倭國而立之|{
	倭烏禾翻}
引兵圍仁願於府城詔起劉仁軌檢校帶方州刺史|{
	帶方州置於百濟界因古地名以名州 考異曰僉載云劉仁願以仁軌檢校帶方州刺史今從本傳}
將王文度之衆便道發新羅兵以救仁願|{
	將即亮翻}
仁軌喜曰天將富貴此翁矣於州司請唐歷及廟諱以行|{
	按劉仁軌自青州刺史白衣從軍此蓋於青州州司請之也}
曰吾欲掃平東夷頒大唐正朔於海表仁軌御軍嚴整轉鬬而前所向皆下百濟立兩柵於熊津江口仁軌與新羅兵合擊破之殺溺死者萬餘人|{
	溺奴狄翻}
道琛乃釋府城之圍退保任存城|{
	任存城在百濟西部任存山 考異曰實録或作任孝城未知孰是今從其多者}
新羅糧盡引還道琛自稱領軍將軍福信自稱霜岑將軍招集徒衆其勢益張|{
	張知亮翻}
仁軌衆少與仁願合軍休息士卒|{
	少詩沼翻}
上詔新羅出兵新羅王春秋奉詔遣其將金欽將兵救仁軌等至古泗福信邀擊敗之|{
	將即亮翻敗補邁翻}
欽自葛嶺道遁還新羅不敢復出|{
	復扶又翻}
福信尋殺道琛專總國兵 夏四月丁卯上幸合璧宫庚辰以任雅相爲浿江道行軍總管契苾何力爲遼

東道行軍總管蘇定方爲平壤道行軍總管與蕭嗣業及諸胡兵凡三十五軍水陸分道並進上欲自將大軍繼之癸巳皇后抗表諫親征高麗|{
	麗力知翻}
詔從之 六月癸未以吐火羅嚈噠罽賓波斯等十六國|{
	罽賓隋漕國也居蔥嶺南距長安萬二千里而嬴四國及訶逹羅支國解蘇國骨咄施國帆延國石汗那國護時犍國恒没國烏拉喝國多勒建國俱蜜國護蜜多國久越得犍國凡十六嚈益涉翻噠當割翻又宅軋翻罽音計}
置都督府八州七十六 |{
	考異曰唐歷云置州二十六今從統紀今按新書地理志時自于闐以西波斯以東凡十六國各以其王都爲都督府吐火羅國都爲月氏都督府領州二十五嚈噠國都爲大汗都督府領州十五訶達羅支國都爲條支都督府領州九解蘇國都爲天馬都督府領州二骨咄施國都為高附都督府領州二罽賓國都為脩鮮都督府領州十帆延國都為寫鳳都督府領州四石汗那國都爲悅般州都督府領州一護時犍國都爲奇沙州都督府領州二怛没國都爲姑墨州都督府領州一烏拉喝國都爲旅州都督府多勒建國都爲崑墟州都督府俱蜜國都爲至拔州都督府護蜜多國都爲鳥飛州都督府領州一久越得犍國都爲王庭州都督府波斯國都爲波斯都督府通鑑言置都督府八者蓋謂月氏大汗條支天馬高附脩鮮寫鳳波斯八都督府餘悅般等八州都督府不預也新志所載領州七十二其數亦與通鑑所引統紀不合}
縣一百一十軍府一百二十六並隸安西都護府 秋七月甲戌蘇定方破高麗於浿江屢戰皆捷遂圍平壤城 九月癸巳朔特進新羅王春秋卒以其子法敏爲樂浪郡王新羅王|{
	卒子恤翻樂浪音洛琅}
壬子徙潞王賢爲沛王賢聞王勃善屬文|{
	屬之欲翻}
召爲修撰|{
	撰士免翻}
勃通之孫也|{
	王通隋末大儒謚文中子}
時諸王鬬雞勃戲爲檄周王雞文 |{
	考異曰舊傳云檄英王雞按中宗爲英王時沛王賢已爲太子當云周王}
上見之怒曰此乃交構之漸斥勃出沛府 高麗蓋蘇文遣其子男生以精兵數萬守鴨緑水諸軍不得度契苾何力至值氷大合何力引衆乘氷度水鼓譟而進高麗大潰追奔數十里斬首三萬級餘衆悉降|{
	降戶江翻}
男生僅以身免會有詔班師乃還 冬十月丁卯上畋於陸渾|{
	陸渾古伊川春秋時秦晉遷陸渾戎於此漢因以名縣屬弘農郡後魏置伊川郡隋廢郡改縣曰伏流大業初復曰陸渾屬洛州}
戊申又畋於非山癸酉還宫 回紇酋長婆閏卒姪比粟毒代領其衆|{
	紇下没翻酋慈由翻長知兩翻卒子恤翻 考異曰新書傳云婆閏卒子比粟嗣今從舊傳}
與同羅僕固犯邊詔左武衛大將軍鄭仁泰爲鐵勒道行軍大總管燕然都護劉審禮左武衛將軍薛仁貴爲副|{
	燕因肩翻}
鴻臚卿蕭嗣業爲仙萼道行軍總管|{
	磧北有仙萼河}
右屯衛將軍孫仁師爲副將兵討之|{
	將即亮翻}
審禮德威之子也|{
	劉德威見一百五十四卷太宗貞觀十一年}


二年春正月辛亥立波斯都督卑路斯爲波斯王 二月甲子改百官名以門下省爲東臺中書省爲西臺尚書省爲中臺侍中爲左相中書令爲右相僕射爲匡政左右丞爲肅機尚書爲太常伯侍郎爲少常伯其餘二十四司御史臺九寺七監十六衛並以義訓更其名而職任如故|{
	改吏部為司列司勲司封如故考功為司績戶部為司元度支為司度金部為司珍倉部為司庾禮部為司禮祠部為司禋主客為司蕃膳部為司膳兵部為司戎職方為司城駕部為司輿庫部為司庫刑部為司刑都官為司僕比部為司計司門為司關工部為司平屯田為司田虞部為司虞水部為司川凡二十四司郎中皆改為大夫改御史臺曰憲臺大夫曰大司憲中丞曰司憲大夫改太常寺曰奉常寺光禄寺曰司宰寺衛尉寺曰司衛寺宗正寺曰司宗寺太僕寺曰司馭寺大理寺曰詳刑寺鴻臚寺曰同文寺司農寺曰司稼寺太府寺曰外府寺凡九寺卿皆曰正卿少卿皆曰大夫改秘書省曰蘭臺監監曰太史少監曰侍郎丞曰大夫殿中省為中御府監監曰大監國子監為司成館祭酒曰大司成司業曰少司成少府監為内府監將作監為繕工監大匠曰大監少匠曰少監都水監為司津監凡七監左右衛府驍衛府武衛府皆省府字左右威衛曰左右武威衛左右領軍衛曰左右戎衛左右候衛曰左右金吾衛左右監門府曰左右監門衛左右千牛府曰左右奉宸衛凡十六衛}
甲戌浿江道大總管任雅相薨於軍雅相爲將|{
	浿普大翻任音壬相息亮翻將即亮翻}
未嘗奏親戚故吏從軍皆移所司補授謂人曰官無大小皆國家公器豈可苟便其私由是軍中賞罸皆平人服其公 戊寅左驍衛將軍白州刺史沃沮道總管龎孝泰|{
	白州本漢合浦縣地武德四年置南州六年改白州沮子余翻}
與高麗戰於蛇水之上軍敗與其子十三人皆戰死蘇定方圍平壤久不下會大雪解圍而還三月鄭仁泰等敗鐵勒於天山|{
	敗補邁翻下所敗翻}
鐵勒九姓

聞唐兵將至合衆十餘萬以拒之選驍健者數十人挑戰薛仁貴發三矢殺三人餘皆下馬請降仁貴悉阬之度磧北擊其餘衆獲葉護兄弟三人而還|{
	驍堅堯翻挑徒了翻降戶江翻磧七亦翻}
軍中歌之曰將軍三箭定天山壯士長歌入漢關思結多濫葛等部落先保天山聞仁泰等將至皆迎降仁泰等縱兵擊之掠其家以賞軍虜相帥遠遁|{
	帥讀曰率}
將軍楊志追之爲虜所敗候騎告仁泰虜輜重在近往可取也|{
	騎奇寄翻下同重直用翻}
仁泰將輕騎萬四千倍道赴之遂踰大磧至仙萼河|{
	新書回鶻牙北六百里至仙娥河}
不見虜糧盡而還值大雪士卒飢凍棄捐甲兵殺馬食之馬盡人自相食比入塞|{
	比必寐翻及也}
餘兵纔八百人軍還司憲大夫楊德裔劾奏仁泰等|{
	漢御史臺有二丞掌殿内秘書謂之中丞漢末改為御史長史後漢復為中丞後魏改為中尉正北齊復曰中丞後周曰司憲中大夫隋諱中改為治書御史唐因之貞觀末避高宗名改為中丞是年改為司憲大夫正五品上掌貳大司憲持邦國憲章以肅正朝廷劾戶槩翻又戶得翻}
誅殺已降|{
	降戶江翻}
使虜逃散不撫士卒不計資糧遂使骸骨蔽野棄甲資宼自聖朝開創以來未有如今日之喪敗者仁貴於所監臨貪淫自恣雖矜所得不補所喪並請付法司推科|{
	喪息浪翻監古銜翻推科者推問而科處其罪}
詔以功贖罪皆釋之以右驍衛大將軍契苾何力為鐵勒道安撫使左衛將軍姜恪副之以安輯其餘衆何力簡精騎五百馳入九姓中虜大驚何力乃謂曰國家知汝皆脅從赦汝之罪罪在酋長得之則己|{
	酋慈由翻}
其部落大喜共執其葉護及設特勒等二百餘人以授何力何力數其罪而斬之|{
	數所具翻}
九姓遂定 甲午車駕發東都辛亥幸蒲州夏四月庚申朔至京師 辛巳作蓬萊宫|{
	蓬萊宫即大明宫亦曰東内程大昌曰大明宫地本太極宫之後苑東北面射殿之地在龍首山上太宗初於其地營永安宫以備太上皇清暑雖嘗改名大明宫而太上皇仍居大安宫不曾徙入龍朔二年高宗苦風痺惡太極宫卑下故就修大明宫改名蓬萊宫取殿後蓬萊池以為名作營造也}
五月丙申以許圉師爲左相|{
	相息亮翻}
六月乙丑初令僧尼道士女官致敬父母|{
	尼女夷翻}
秋七月戊子朔赦天下 丁巳熊津都督劉仁願帶方州刺史劉仁軌大破百濟於熊津之東拔眞峴城初仁願仁軌等屯熊津城 |{
	考異曰去歲道琛福信圍仁願於百濟府城今云尚在熊津城或者共是一城不則圍解之後徙屯熊津城耳}
上與之敕書以平壤軍回一城不可獨固宜拔就新羅若金法敏藉卿留鎭宜且停彼若其不須即宜泛海還也將士咸欲西歸仁軌曰人臣徇公家之利有死無貳豈得先念其私主上欲滅高麗故先誅百濟留兵守之制其心腹雖餘宼充斥而守備甚嚴宜礪兵秣馬擊其不意理無不克既捷之後士卒心安然後分兵據險開張形勢飛表以聞更求益兵朝廷知其有成必命將出師聲援纔接凶醜自殱|{
	將即亮翻殱息亷翻}
非直不棄成功實亦永清海表今平壤之軍既還熊津又抜|{
	抜謂抜軍就新羅或抜軍西還也}
則百濟餘燼不日更興高麗逋宼何時可滅且今以一城之地居敵中央苟或動足即爲擒虜縱入新羅亦爲羇客脱不如意悔不可追况福信凶悖殘虐君臣猜離行相屠戮正宜堅守觀變乘便取之不可動也衆從之時百濟王豐與福信等以仁願等孤城無援遣使謂之曰大使等何時西還當遣相送|{
	使疏吏翻下同}
仁願仁軌知其無備忽出擊之抜其支羅城及尹城大山沙井等柵殺獲甚衆分兵守之福信等以眞峴城險要加兵守之仁軌伺其稍懈引新羅兵夜傳城下攀草而上比明入據其城|{
	伺相吏翻懈古隘翻傳音附上時掌翻比必寐翻}
遂通新羅運糧之路仁願乃奏請益兵詔發淄青萊海之兵七千人以赴熊津|{
	史言劉仁軌能堅忍伺間待援兵以盡平百濟}
福信專權與百濟王豐浸相猜忌福信稱疾卧於窟室欲俟豐問疾而殺之豐知之帥親信襲殺福信|{
	果如劉仁軌所料帥讀曰率}
遣使詣高麗倭國乞師以拒唐兵|{
	倭烏禾翻}


資治通鑑卷二百
