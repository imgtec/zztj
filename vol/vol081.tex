\chapter{資治通鑑卷八十一}
宋 司馬光 撰

胡三省 音註

晉紀三|{
	起上章困敦盡著雍涒灘凡九年}


世祖武皇帝中

太康元年|{
	是年四月改元}
春正月吳大赦 杜預向江陵王渾出横江攻吳鎮戍所向皆克二月戊午王濬唐彬擊破丹陽監盛紀|{
	丹陽城在秭歸縣東八里昔周武王封熊繹於荆丹陽之地即此今謂之屈沱楚王城}
吳人於江磧要害之處|{
	磧七逆翻水渚有沙石曰磧}
並以鐵鎖横截之又作鐵錐長丈餘暗置江中以逆拒舟艦|{
	長直亮翻艦戶黯翻}
濬作大筏數十方百餘步縛草為人被甲持仗令善水者以筏先行遇鐵錐錐輒著筏而去|{
	筏音伐被皮義翻著陟略翻後著手同}
又作大炬長十餘丈|{
	長直亮翻}
大數十圍灌以麻油在船前遇鎖然炬燒之須臾融液斷絶於是船無所礙|{
	以人力設險而不以人力守之無益也}
庚申濬克西陵殺吳都督留憲等壬戍克荆門夷道二城|{
	荆門在西陵之東夷道之西}
殺夷道監陸晏杜預遣牙門周旨等帥奇兵八百汎舟夜渡江襲樂鄉|{
	帥讀曰率}
多張旗幟起火巴山|{
	巴山在今江陵府松滋縣有巴復村幟昌志翻}
吳都督孫歆懼與江陵督伍延書曰北來諸軍乃飛渡江也旨等伏兵樂鄉城外歆遣軍出拒王濬大敗而還旨等發伏兵隨歆軍而入歆不覺直至帳下虜歆而還乙丑王濬擊殺吴水軍都督陸景 |{
	考異曰武紀壬戌濬克夷道樂鄉城殺陸景陸抗傳壬戌殺晏癸亥殺景王濬傳壬戊克夷道獲晏乙丑克樂鄉獲景今從濬傳}
杜預進攻江陵甲戍克之斬伍延於是沅湘以南接于交廣州郡皆望風送印綬|{
	水經沅水出䍧牱且蘭縣東北過臨沅縣又東至長沙下雋縣西北入於江湘水出零陵始安縣陽海山東北過洮陽泉陵重安酃隂山澧陵臨湘羅下雋等縣又北至巴丘山入于江沅音元}
預杖節稱詔而綏撫之凡所斬獲吳都督監軍十四牙門郡守百二十餘人胡奮克江安|{
	江安即公安吳南郡治焉杜預既定江南改曰江安縣為南平郡治所}
乙亥詔王濬唐彬既定巴丘與胡奮王戎共平夏口武昌順流長騖直造秣陵|{
	夏戶雅翻造到翻下徑造同}
杜預當鎮静零桂懷輯衡陽|{
	零陵桂陽漢古郡衡陽吳主亮大平二年分長沙西部都尉立}
大兵既過荆州南境固當傳檄而定|{
	謂重鎮既破其餘當望風而靡也}
預等各分兵以益濬彬太尉充移屯項|{
	以荆州已定不復使賈充南屯襄陽移屯項為諸軍節度}
王戎遣參軍襄陽羅尚南陽劉喬將兵與王濬合攻武昌吴江夏太守劉朗督武昌諸軍虞昺皆降|{
	夏戶雅翻降戶江翻}
昺翻之子也杜預與衆軍會議或曰百年之寇未可盡克方春水生難於久駐 |{
	考異曰杜預傳曰今向暑水潦方降疾疫將起按時未暑今依三十國春秋}
宜俟來冬更為大舉預曰昔樂毅藉濟西一戰以并彊齊|{
	事見四卷周赧王三十一年}
今兵威已振譬如破竹數節之後皆迎刃而解無復著手處也|{
	復扶又翻下可復所復同著陟畧翻}
遂指授羣帥方畧徑造建業|{
	帥所類翻}
吴主聞王渾南下使丞相張悌督丹陽太守沈瑩護軍孫震副軍師諸葛靚帥衆三萬渡江逆戰|{
	靚疾正翻帥讀曰率下同}
至牛渚沈瑩曰晉治水軍於蜀久矣|{
	治直之翻}
上流諸軍素無戒備名將皆死幼小當任|{
	謂陸晏陸景留憲孫歆等}
恐不能禦也晉之水軍必至於此宜畜衆力以待其來與之一戰若幸而勝之江西自清|{
	大江北流自建業言之歷陽皖城皆為江西}
今渡江與晉大軍戰不幸而敗則大事去矣悌曰吴之將亡賢愚所知非今日也吾恐蜀兵至此衆心駭懼不可復整|{
	復扶又翻下同}
及今渡江猶可决戰若其敗喪|{
	喪息浪翻}
同死社稷無所復恨若其克捷北敵犇走兵埶萬倍便當乘勝南上|{
	上時掌翻}
逆之中道不憂不破也若如子計恐士衆散盡坐待敵到君臣俱降無一人死難者不亦辱乎|{
	如悌之言吴人至此為計窮矣然悌之志節亦可憐也難乃旦翻}
三月悌等濟江圍渾部將城陽都尉張喬於楊荷|{
	水經注淮水自江夏平春縣北東北流逕汝南城陽縣故城南漢高帝十二年封定侯奚竟為侯國王莽之新利也魏置城陽郡按干寶晉紀楊荷橋名今按水經注之城陽郡乃元魏所置張喬蓋以渾部將領青州之城陽都尉也}
喬衆纔七千閉柵請降諸葛靚欲屠之悌曰彊敵在前不宜先事其小且殺降不祥靚曰此屬以救兵未至力少不敵故且偽降以緩我非真伏也|{
	降戶江翻伏屈伏也或曰伏當作服}
若捨之而前必為後患悌不從撫之而進悌與揚州刺史汝南周浚結陳相對|{
	陳讀曰陣}
沈瑩帥丹陽鋭卒刀楯五十三衝晉兵不動|{
	楯食尹翻}
瑩引退其衆亂將軍薛勝蔣班因其亂而乘之吳兵以次奔潰將帥不能止張喬自後擊之大敗吳兵于版橋|{
	敗補邁翻}
諸葛靚帥數百人遁去使過迎張悌悌不肯去靚自往牽之曰存亡自有大數非卿一人所支奈何故自取死悌垂涕曰仲思|{
	諸葛靚字仲思}
今日是我死日也且我為兒童時便為卿家丞相所識拔|{
	丞相謂諸葛亮也或曰謂諸葛瑾余謂張悌襄陽人蓋亮在荆州識之於童幼也}
常恐不得其死負名賢知顧今以身徇社稷復何道邪|{
	道言也復扶又翻}
靚再三牽之不動乃流淚放去行百餘步顧之已為晉兵所殺并斬孫震沈瑩等七千八百級吳人大震初詔書使王濬下建平受杜預節度至建業受王渾節度預至江陵謂諸將曰若濬得建平則順流長驅威名已著不宜令受制于我若不能克則無緣得施節度濬至西陵預與之書曰足下既摧其西藩便當徑取建業討累世之逋寇釋吳人於塗炭振旅還都亦曠世一事也|{
	言歷世所曠見之事}
濬大悦表陳預書及張悌敗死揚州别駕何惲|{
	惲委粉翻}
謂周浚曰張悌舉全吳精兵殄滅於此吳之朝野莫不震懾|{
	朝直遥翻懾之涉翻}
今王龍驤既破武昌|{
	王濬為龍驤將軍驤思將翻}
乘勝東下所向輒克土崩之勢見矣|{
	見賢遍翻}
謂宜速引兵渡江直指建業大軍猝至奪其膽氣可不戰禽也浚善其謀使白王渾惲曰渾闇於事機而欲慎已免咎必不我從浚固使白之渾果曰受詔但令屯江北以抗吳軍不使輕進貴州雖武豈能獨平江東乎今者違命勝不足多若其不勝為罪已重且詔令龍驤受我節度但當具君舟檝一時俱濟耳惲曰龍驤克萬里之寇以既成之功來受節度未之聞也且明公為上將|{
	將即亮翻}
見可而進豈得一 一須詔令乎|{
	須待也}
今乘此渡江十全必克何疑何慮而淹留不進此鄙州上下所以恨恨也|{
	此所謂恨恨悵望不滿之意}
渾不聽王濬自武昌順流徑趣建業|{
	趣七喻翻}
吳主遣遊擊將軍張象帥舟師萬人禦之象衆望旗而降濬兵甲滿江旌旗燭天威埶甚盛吳人大懼吳主之嬖臣岑昏以傾險諛佞致位九列|{
	九列九卿也}
好興功役|{
	好呼到翻}
為衆患苦及晉兵將至殿中親近數百人叩頭請於吳主曰北軍日近而兵不舉刃陛下將如之何吳主曰何故對曰正坐岑昏耳吳主獨言若爾當以奴謝百姓|{
	獨言謂其言止此耳}
衆因曰唯|{
	唯于癸翻諾也}
遂並起收昏吳主駱驛追止|{
	駱驛言相繼遣人不絶也}
已屠之矣陶濬將討郭馬至武昌聞晉兵大入引兵東還至建業吳主引見問水軍消息|{
	見賢遍翻}
對曰蜀船皆小|{
	陶濬蓋以尋常蜀船言之諜不明亦可見矣}
今得二萬兵乘大船以戰自足破之於是合衆授濬節鉞明日當發其夜衆悉逃潰時王渾王濬及琅邪王伷皆臨近境|{
	伷音胄}
吳司徒何植建威將軍孫晏|{
	漢光武命耿弇為建威大將軍建威之號自此始}
悉送印節詣渾降吳主用光禄勲薛瑩中書令胡冲等計分遣使者奉書於渾濬伷以請降又遺其羣臣書|{
	遺于季翻}
深自咎責且曰今大晉平治四海是英俊展節之秋勿以移朝改朔用損厥志|{
	治直之翻朝直遥翻}
使者先送璽綬於琅邪王伷壬寅王濬舟師過三山|{
	三山在今建康府上元縣西南四十五里又西即江寧夾陸游曰三山磯在烈洲下凡山臨江皆曰磯三山距金陵財五十餘里}
王渾遣信要濬蹔過論事|{
	信即信使要讀曰邀蹔與暫同}
濬舉帆直指建業報曰風利不得泊也是日濬戎卒八萬方舟百里|{
	詩云就其深矣方之舟之注方泭也舟船也爾雅方木置水曰泭音夫}
鼓譟入于石頭吳主皓面縛輿櫬詣軍門降濬解縛焚櫬延請相見|{
	櫬初覲翻}
收其圖籍克州四郡四十二戶五十二萬三千兵二十三萬|{
	吳有荆揚交廣四州漢獻帝興平二年孫策始取江東魏文帝黄初三年吳王孫權始稱帝傳四主五十七年而亡}
朝廷聞吳已平羣臣皆賀上壽帝執爵流涕曰此羊太傅之功也|{
	異義韓詩一升曰爵爵盡也足也羊祜贈太傅}
票騎將軍孫秀不賀|{
	孫秀來奔見七十九卷泰始六年票匹妙翻}
南向流涕曰昔討逆弱冠以一校尉創業|{
	討逆孫策也起兵之初袁術表為懷義校尉冠古玩翻}
今後主舉江南而棄之宗廟山陵於此為墟悠悠蒼天此何人哉|{
	詩黍離之辭}
吳之未下也大臣皆以為未可輕進獨張華堅執以為必克賈充上表稱吳地未可悉定方夏江淮下濕疾疫必起宜召諸軍還以為後圖雖腰斬張華不足以謝天下帝曰此是吾意華但與吾同耳荀朂復奏宜如充表帝不從|{
	復扶又翻}
杜預聞充奏乞罷兵馳表固爭使至轘轅而吳已降|{
	使疏吏翻轘音環}
充慙懼詣闕請罪帝撫而不問夏四月甲申詔賜孫皓爵歸命侯乙酉大赦改元|{
	改元太康自此以前係咸寧六年事}
大酺五日|{
	酺薄乎翻}
遣使者分詣荆揚撫慰吳牧守已下皆不更易|{
	守式又翻更工衡翻}
除其苛政悉從簡易|{
	易以豉翻}
滕脩討郭馬未克|{
	去年吴主皓遣滕脩討郭馬}
聞晉伐吴帥衆赴難|{
	帥讀曰率難乃旦翻}
至巴丘聞吴亡縞素流涕還與廣州刺史閭豐|{
	閭姓豐名此與後魏閭大肥不同所自出閭大肥出於柔然郁久閭氏左傳楚平王之子啟字子閭其後以為氏}
蒼梧太守王毅各送印綬請降孫皓遣陶璜之子融持手書諭璜璜流涕數日亦送印綬降帝皆復其本職|{
	綬音受}
王濬之東下也吳城戍皆望風欵附獨建平太守吾彦嬰城不下聞吳亡乃降帝以彦為金城太守初朝廷尊寵孫秀孫楷|{
	楷降見上卷咸寧二年}
欲以招來吳人及吳亡降秀為伏波將軍楷為渡遼將軍琅邪王伷遣使送孫皓及其宗族詣洛陽五月丁亥朔皓至 |{
	考異曰皓傳天紀四年三月丙寅殺岑昏戊辰陶濬從武昌還壬申王濬到受皓降五月丁亥集于京邑四月甲申封歸命侯晉武紀太康元年二月王濬等破武昌王渾斬張悌三月壬申濬下石頭皓降乙酉大赦改元四月遣朱震等慰撫五月辛亥封歸命侯丙寅引皓升殿庚午詔士卒六十歸家庚辰以濬為輔國將軍王濬傳二月庚申克西陵又云壬寅濬入石頭而無月又上書曰臣十四日至牛渚十五日至秣陵亦無月又曰去二月武昌失守皓左右皆得寶散走三十國春秋四月甲子王渾斬張悌丙寅殺岑昏與何楨書庚午送降書壬申濬入石頭甲申封歸命侯五月丁亥至洛陽晉春秋畧與之同按長歷去年閏七月今年二月戊午朔三月戊子朔四月丁巳朔五月丁亥朔六月丙辰朔然則三月無戊辰丙寅壬申五月無庚午庚辰與吳志晉書不合若依三十國春秋月日雖合然二月武昌失守皓左右離散不容四月十六日王濬乃至秣陵而皓降又皓以四月十六日降舉家西上至五月一日未能至洛今事之先後並依吴志晉書但削去其日之不與歷合者}
與其太子瑾等泥頭面縛詣東陽門|{
	晉志洛陽城東有建春東陽清明三門泥頭者以泥塗其首也瑾渠吝翻}
詔遣謁者解其縛賜衣服車乘田三十頃歲給錢穀綿絹甚厚|{
	武王伐紂斬其首懸於太白之旗如孫皓之凶暴斬之以謝吳人可也乘繩證翻}
拜瑾為中郎諸子為王者皆為郎中吳之舊望隨才擢叙孫氏將吏渡江者復十年百姓復二十年|{
	將即亮翻復方目翻}
庚寅帝臨軒大會文武有位及四方使者國子學生皆預焉引見歸命侯皓及吳降人皓登殿稽顙|{
	見賢遍翻稽顙周之喪拜顙額也稽顙額觸地無容稽音啟}
帝謂皓曰朕設此座以待卿久矣皓曰臣於南方亦設此座以待陛下賈充謂皓曰聞君在南方鑿人目剥人面皮此何等刑也皓曰人臣有弑其君及姦回不忠者則加此刑耳|{
	斥充世受魏恩而姦回附晉弑高貴鄉公也}
充默然甚愧而皓顔色無怍|{
	怍疾各翻慙也}
帝從容問散騎常侍薛瑩孫皓所以亡對曰皓昵近小人|{
	從千容翻近其靳翻}
刑罰放濫大臣諸將人不自保此其所以亡也它日又問吾彦對曰吳主英俊宰輔賢明帝笑曰若是何故亡彦曰天禄永終歷數有屬故為陛下禽耳帝善之|{
	有學而無識此薛瑩所以不及吾彦也屬之欲翻}
王濬之入建業也其明日王渾乃濟江以濬不待已至先受孫皓降意甚愧忿將攻濬何攀勸濬送皓與渾由是事得解何惲以渾與濬爭功與周浚牋曰書貴克讓易大謙光|{
	書曰允㳟克讓易曰謙尊而光}
前破張悌吳人失氣龍驤因之䧟其區宇論其前後我實緩師既失機會不及於事而今方競其功|{
	競爭也}
彼既不吞聲將虧雍穆之宏興矜爭之鄙|{
	雍穆和也書曰汝惟不矜天下莫與汝爭能}
斯實愚情之所不取也浚得牋即諫止渾渾不納表濬違詔不受節度誣以罪狀渾子濟尚常山公主|{
	公主帝女也}
宗黨彊盛有司奏請檻車徵濬帝弗許但以詔書責讓濬以不從渾命違制昧利濬上書自理曰前被詔書令臣直造秣陵|{
	被皮義翻下同造七到翻下同}
又令受太尉充節度臣以十五日至三山見渾軍在北岸遣書邀臣臣水軍風發徑造賊城無緣迴船過渾|{
	過工禾翻}
臣以日中至秣陵暮乃被渾所下當受節度之符|{
	被皮義翻下遐稼翻}
欲令臣明十六日悉將所領還圍石頭|{
	十六日者十五之明日故曰明十六日將即亮翻}
又索蜀兵及鎮南諸軍人名定見|{
	鎮南諸軍杜預所統蓋分以隨濬東下者也定見謂軍人在行定數索山客翻}
臣以為皓已來降無緣空圍石頭又兵人定見不可倉猝得就皆非當今之急不可承用非敢忽弃明制也皓衆叛親離匹夫獨坐雀鼠貪生苟乞一活耳而江北諸軍不知虛實不早縛取自為小誤臣至便得更見怨恚|{
	恚於避翻}
並云守賊百日而令他人得之臣愚以為事君之道苟利社稷死生以之若其顧嫌疑以避咎責此是人臣不忠之利實非明主社稷之福也渾又騰周浚書云濬軍得吳寶物|{
	騰其書使上聞}
又云濬牙門將李高放火燒皓偽宫濬復表曰|{
	復扶又翻}
臣孤根獨立結恨彊宗夫犯上干主其罪可救乖忤貴臣禍在不測|{
	忤五故翻}
偽中郎將孔攄說去二月武昌失守水軍行至|{
	二月已過故云去二月行至猶言行將至也攄抽居翻}
皓案行石頭還|{
	行下孟翻}
左右人皆跳刀大呼|{
	揚正衡曰跳大么翻呼火故翻}
云要當為陛下一死戰决之|{
	為于偽翻}
皓意大喜意必能然便盡出金寶以賜與之小人無狀得便馳走皓懼乃圖降首|{
	首式救翻}
降使適去|{
	降戶江翻使疏吏翻}
左右刼奪財物略取妻妾放火燒宫皓逃身竄首恐不脫死臣至遣參軍主者救斷其火耳|{
	斷丁管翻}
周浚先入皓宫渾又先登皓舟臣之入觀皆在其後皓宫之中乃無席可坐若有遺寶則浚與渾先得之矣浚等云臣屯聚蜀人不時送皓欲有反狀又恐動吳人言臣皆當誅殺取其妻子冀其作亂得騁私忿|{
	騁丑郢翻}
謀反大逆尚以見加其餘謗?|{
	?語相惡也音達合翻}
故其宜耳今年平吳誠為大慶於臣之身更受咎累|{
	累力瑞翻}
濬至京師有司奏濬違詔大不敬請付廷尉科罪詔不許|{
	科斷也}
又奏濬赦後燒賊船百三十五艘輒敕付廷尉禁推|{
	此皆王渾親黨使為之艘蘇刀翻}
詔勿推渾濬爭功不已帝命守廷尉廣陵劉頌校其事以渾為上功濬為中功帝以頌折法失理|{
	折法猶折獄之折折斷也}
左遷京兆太守|{
	魏文帝受禪改京兆尹為太守夷於列郡}
庚辰增賈充邑八千戶以王濬為輔國大將軍封襄陽縣侯杜預為當陽縣侯王戎為安豐縣侯封琅邪王伷二子為亭侯增京陵侯王渾邑八千戶進爵為公尚書關内侯張華進封廣武縣侯增邑萬戶|{
	王渾除京陵舊食邑之外增八千戶張華則增廣武侯邑為萬戶}
荀朂以專典詔命功封一子為亭侯|{
	朂為中書監專典詔命}
其餘諸將及公卿以下賞賜各有差帝以平吳功策告羊祜廟乃封其夫人夏侯氏為萬歲鄉君食邑五千戶|{
	夏戶雅翻}
王濬自以功大而為渾父子及黨與所挫抑每進見|{
	見賢遍翻}
陳其功伐之勞及見枉之狀或不勝忿憤徑出不辭帝每容恕之|{
	晉武之量宏於隋文勝音升}
益州護軍范通謂濬曰卿功則美矣然恨所以居美者未盡善也卿旋斾之日角巾私第|{
	晉志曰巾以葛為之形如幍而横著之古者尊卑共服之余謂幅巾以横幅為之角巾則巾之有角者郭林宗遇雨巾一角墊則角巾也}
口不言平吳之事若有問者則曰聖人之德羣帥之力老夫何力之有此藺生所以屈亷頗也|{
	事見四卷周赧王三十六年帥所類翻}
王渾能無愧乎濬曰吾始懲鄧艾之事|{
	鄧艾之死以鍾會所蔽艾情不得上通也}
懼禍及身不得無言其終不能遣諸胷中是吾褊也|{
	自知數陳其功及為渾所枉為褊褊補辨翻}
時人咸以濬功重報輕為之憤邑|{
	為于偽翻}
博士秦秀等並上表訟濬之屈帝乃遷濬鎮軍大將軍 |{
	考異曰濬傳云領步兵校尉舊校唯五置此營自濬始也按職官志屯騎步兵長水越騎射聲校尉是為五校並漢官也然則步兵之名非自濬始武帝紀是年六月丁丑初置翊軍校尉官疑濬所領者翊軍也}
王渾嘗詣濬濬嚴設備衛然後見之|{
	周勃就國絳及河東吏至常令家人被甲持兵以見之亦猶王濬之嚴設備衛以見王渾也此二人者力足以定天下之難智足以取一國而其所以包周身之防乃爾可笑也哉}
杜預還襄陽以為天下雖安忘戰必危乃勤於講武申嚴戍守又引滍淯水以浸田萬餘頃|{
	水經注滍水出南陽魯山縣西堯山東逕犨縣又東南逕昆陽縣又東北逕潁川定陵縣東入于汝淯水出宏農盧氏縣攻離山東南逕南陽西鄂縣宛縣而屈南過淯陽縣又南過新野縣西過鄧縣南入于沔滍音丈几翻淯音育}
開揚口通零桂之漕|{
	水經注揚水上承江陵縣赤湖東北流逕郢城南又東北與三湖水會三湖者合為一水東通荒谷東岸有冶父城春秋傳曰莫敖縊于荒谷羣帥囚於冶父謂此處也春夏水盛則南通大江否則南迄江隄揚水又東入華容縣又東北與柞溪水合又北逕竟陵縣又北注于沔謂之揚口預傳曰舊水道惟沔漢達江陵千數百里北無通路預乃開揚口起夏水達巴陵千餘里内瀉長江之險外通零桂之漕杜佑曰夏水揚口在今江陵郡江陵縣界}
公私賴之預身不跨馬射不穿札|{
	札甲札也左傳潘尫之黨與養由基蹲甲而射之徹七札焉}
而用兵制勝諸將莫及預在鎮數餉遺洛中貴要|{
	數所角翻遺于季翻}
或問其故預曰吾但恐為害不求益也王渾遷征東大將軍復鎮壽陽諸葛靚逃竄不出|{
	靚入吴見七十七卷魏高貴鄉公甘露二年}
帝與靚有舊靚姊為琅邪王妃|{
	琅邪王仙}
帝知靚在姊間因就見焉靚逃于厠帝又逼見之謂曰不謂今日復得相見靚流涕曰臣不能漆身破面|{
	自謂不能如豫讓聶政也}
復覩聖顔誠為慙恨詔以為侍中固辭不拜歸于鄉里終身不向朝廷而坐|{
	諸葛氏之子皆有志節}
六月復封丹水侯睦為高陽王|{
	睦貶爵見上卷咸寧三年}
秋八月己未封皇弟延祚為樂平王尋薨 九月庚寅賈充等以天下一統屢請封禪帝不許 冬十月前將軍青州刺史淮南胡威卒|{
	帝以左右前後四將軍為四軍}
威為尚書嘗諫時政之寛帝曰尚書郎以下吾無所假借威曰臣之所陳豈在丞郎令史正謂如臣等輩始可以肅化明法耳|{
	威質之子也}
是歲以司隸所統郡置司州凡州十九 |{
	考異曰宋書州郡志太康元年天下一統凡十六州後又分雍梁為秦分荆揚為江分益為寧分幽為平而為二十矣按杜佑通典平吴分十九州司兖豫冀并青徐荆揚梁雍秦益梁寧幽平交廣今從之 杜佑曰司州治洛陽兖治廩丘今濮陽郡雷澤縣豫治項今淮陽郡項城縣冀治房子今趙郡縣并治晉陽青治臨甾徐治彭城荆初治襄陽後治江陵楊治夀春後治建業涼治武威分三輔為雍治京兆分隴山之西為秦治上邽益治成都分巴漢之地為梁治南鄭分雲南為寧治雲南幽治涿分遼東為平治昌黎交治龍編分合浦之北為廣治番禺}
郡國一百七十三戶二百四十五萬九千八百四十 詔曰昔自漢末四海分崩刺史内親民事外領兵馬今天下為一當韜戢干戈刺史分職皆如漢氏故事|{
	察舉郡縣長吏而已}
悉去州郡兵大郡置武吏百人小郡五十人交州牧陶璜上言交廣東西數千里|{
	交州統合浦交趾新昌武平九真九德日南廣州統南海臨賀始安始興蒼梧欎林桂林高涼高興寧浦郡去羌呂翻下宜去同}
不賓屬者六萬餘戶至於服從官役纔五千餘家二州脣齒唯兵是鎮又寧州諸夷接據上流水陸並通|{
	僕水葉榆水勞水橋水皆出寧州界入交廣界又霍弋自寧州遣楊稷等經略交廣是水陸並通也}
州兵未宜約損以示單虛僕射山濤亦言不宜去州郡武備帝不聽及永寧以後盜賊蜂起州郡無備不能禽制天下遂大亂如濤所言然其後刺史復兼兵民之政州鎮愈重矣 漢魏以來羌胡鮮卑降者|{
	降戶江翻}
多處之塞内諸郡其後數因忿恨殺害長吏深為民患侍御史西河郭欽上疏曰戎狄彊獷|{
	處昌呂翻數所角翻獷古猛翻麤惡貌}
歷古為患魏初民少|{
	少詩沼翻}
西北諸郡皆為戎居内及京兆魏郡宏農往往有之今雖服從若百年之後有風塵之警胡騎自平陽上黨不三日而至孟津北地西河太原馮翊安定上郡盡為狄庭矣宜及平吴之威謀臣猛將之略漸徙内郡雜胡於邉地峻四夷出入之防明先王荒服之制|{
	禹貢五服相距方五千里荒服内距甸服二千里}
此萬世之長策也帝不聽|{
	為後諸胡亂華張本}


二年春三月詔選孫皓宫人五千人入宫帝既平吳頗事遊宴怠於政事掖庭殆將萬人常乘羊車|{
	晉志曰羊車一名輦車上如軺伏兔箱漆畫輪軛}
恣其所之至便宴寑宫人競以竹葉插戶鹽汁灑地以引帝車|{
	羊嗜竹葉而喜鹹故以二者引帝車}
而后父楊駿及弟珧濟始用事|{
	珧余招翻}
交通請謁埶傾内外時人謂之三楊舊臣多被疎退山濤數有規諷|{
	數所角翻下同}
帝雖知而不能改 初鮮卑莫護跋始自塞外入居遼西棘城之北|{
	棘城在昌黎縣界是後慕容氏置棘城縣拓跋魏太武真君八年併棘城入昌黎郡龍城縣載記曰莫護跋從宣帝伐公孫氏有功拜率義王始建國于棘城之北}
號曰慕容部|{
	魏書曰漢桓帝時鮮卑檀石槐分其地為東中西三部中部大人曰柯最闕居慕容尋為大帥是則慕容部之始也載記曰莫護跋國于棘城之北時燕代多冠步揺冠莫護跋見而好之乃斂髪襲冠諸部因呼之為步揺其後音訛遂為慕容或云慕二儀之德繼三光之容遂以慕容為氏余謂步揺之說誕或云之說慕容氏既得中國其臣子從而為之辭}
莫護跋生木延木延生涉歸遷於遼東之北世附中國數從征討有功拜大單于|{
	單音蟬}
冬十月涉歸始寇昌黎|{
	昌黎漢之交黎縣屬遼西郡東漢屬遼東屬國都尉魏正始五年鮮卑内附復置遼東屬國立昌黎縣以居之後立昌黎郡慕容氏始此 考異曰帝紀云慕容廆按范亨燕書武宣紀廆泰始五年生年十五父單于涉歸卒太康四年也此年入寇當是涉歸}
十一月壬寅高平武公陳騫薨|{
	考異曰帝紀云大司馬按騫以咸寧三年辭位以高平公還第}
是歲揚州刺史周浚移鎮秣陵|{
	魏揚州治夀春晉平吴乃移治秣陵揚者江南之氣躁勁厥性輕揚亦曰州界多水水波揚也統丹陽宣城淮南廬陵廬江毗陵吳吳興會稽東陽新安臨海建安晉安豫章臨川鄱陽南康凡十八郡}
吳民之未服者屢為寇亂浚皆討平之賓禮故老搜求俊乂威惠並行吳人悦服

三年春正月丁丑朔帝親祀南郊禮畢喟然問司隸校尉劉毅曰朕可方漢之何帝對曰桓靈帝曰何至於此對曰桓靈賣官錢入官庫陛下賣官錢入私門以此言之殆不如也帝大笑曰桓靈之世不聞此言今朕有直臣固為勝之 |{
	考異曰地理志太康元年省司隸置司州毅傳毅為司隸校尉帝嘗南郊禮畢問毅而無年月晉春秋問毅在此月而不言毅官按毅傳六年自司隸遷左僕射或者此年尚未改為司州也今從毅傳}
毅為司隸糾繩豪貴無所顧忌|{
	繩彈正也糾督也}
皇太子鼓吹入東掖門|{
	臣子至宫掖門屛儀導下車而入太子鼓吹入掖門為不敬吹昌瑞翻}
毅劾奏之|{
	劾戶槩翻又戶得翻}
中護軍散騎常侍羊琇與帝有舊恩|{
	事見七十八卷魏元帝咸熙元年琇音秀}
典禁兵豫機密十餘年恃寵驕侈數犯法|{
	數所角翻}
毅劾奏琇罪當死帝遣齊王攸私請琇於毅毅許之都官從事廣平程衛徑馳入護軍營收琇屬吏|{
	屬之欲翻}
考問隂私先奏琇所犯狼籍然後言於毅帝不得已免琇官未幾復使以白衣領職|{
	幾居豈翻}
琇景獻皇后之從父弟也後將軍王愷文明皇后之弟也|{
	景帝羊后謚景獻文帝王后謚文明從才用翻}
散騎常侍石崇苞之子也三人皆富於財競以奢侈相高愷以澳釡|{
	盈之翻?也說文白米糵煎也一曰濡弱者為澳於到翻今台明謂以水沃釡為澳鑊又乙六翻}
崇以蠟代薪|{
	蠟蜜滓也}
愷作紫絲步障四十里崇作錦步障五十里|{
	步障夾道設之以障蔽若今之罘罳}
崇塗屋以椒|{
	椒性温而芬馥}
愷用赤石脂|{
	本草圖經曰赤石脂出濟南射陽及太山之隂蘇恭云濟南太山不聞出者惟虢州盧氏縣澤州陵川縣慈州呂鄉縣並有及宜州諸山亦出今出潞州以色理鮮膩者為勝}
帝每助愷嘗以珊瑚樹賜之|{
	本草珊瑚生海㡳柯枝明潤如紅玉}
高二尺許愷以示石崇崇便以鐵如意碎之|{
	鐵如意手撾也以鐵為之若今之骨朶子}
愷怒以為疾已之寶崇曰不足多恨今還卿乃命左右悉取其家珊瑚樹高三四尺者六七株如愷比者甚衆愷怳然自失|{
	怳虎晃翻自失不得意貌}
車騎司馬傳咸上書曰|{
	晉志曰驃騎以下及諸大將軍不開府非特節都督者置長史司馬各一人}
先王之治天下|{
	治直之翻}
食肉衣帛皆有其制|{
	古者黎民五十而後食肉六十而後衣帛衣於既翻}
竊謂奢侈之費甚於天災古者人稠地狹而有儲蓄由於節也今者土廣人稀而患不足由於奢也欲人崇儉當詰其奢奢不見詰轉相高尚無有窮極矣|{
	詰去吉翻}
尚書張華以文學才識名重一時論者皆謂華宜為三公中書監荀朂侍中馮紞以伐吳之謀深疾之|{
	紞都感翻}
會帝問華誰可託後事者華對以明德至親莫如齊王由是忤旨|{
	忤五故翻}
朂因而譛之甲午以華都督幽州諸軍事華至鎮撫循夷夏|{
	夏戶雅翻}
譽望益振帝復欲徵之馮紞侍帝從容語及鍾會|{
	從千容翻}
紞曰會之反頗由太祖|{
	會反見七十八卷魏元帝咸熙元年文帝廟號太祖}
帝變色曰卿是何言邪紞免冠謝曰臣聞善御者必知六轡緩急之宜故孔子以仲由兼人而退之冉求退弱而進之|{
	事見論語}
漢高祖尊寵五王而夷滅|{
	事並見漢高帝紀五王兩韓信彭越英布盧綰}
光武抑損諸將而克終|{
	光武不使功臣預政事故皆保其福祿無誅譴者}
非上有仁暴之殊下有愚智之異也蓋抑揚與奪使之然耳鍾會才智有限而太祖誇奬無極居以重埶委以大兵使會自謂算無遺策功在不賞遂搆凶逆耳向令太祖録其小能節以大禮抑之以威權納之以軌則則亂心無由生矣帝曰然紞稽首曰陛下既然臣之言宜思堅冰之漸|{
	稽音啟易坤之初六曰履霜堅冰至象曰履霜堅冰隂始凝也馴致其道至堅冰也}
勿使如會之徒復致傾覆|{
	復扶又翻}
帝曰當今豈復有如會者邪紞因屏左右而言曰|{
	屛必郢翻}
陛下謀畫之臣著大功于天下據方鎮總戎馬者皆在陛下聖慮矣帝默然由是止不徵華 三月安北將軍嚴詢敗慕容涉歸於昌黎斬獲萬計|{
	敗補邁翻}
魯公賈充老病上遣皇太子省視起居|{
	省悉景翻}
充自憂諡傳|{
	充自知姦回弑逆後當加惡諡且不能逃良史之筆誅傳柱戀翻}
從子模曰是非久自見不可掩也|{
	從才用翻見賢遍翻}
夏四月庚午充薨世子黎民早卒無嗣妻郭槐欲以充外孫韓謐為世孫|{
	韓謐充壻韓壽之子世孫謂嫡孫承祖父之世者}
郎中令韓咸中尉曹軫諫曰|{
	晉制諸王及諸郡公國有郎中令中尉大農為三卿}
禮無異姓為後之文今而行之是使先公受譏於後世而懷愧於地下也槐不聽咸等上書求改立嗣事寑不報槐遂表陳之云充遺意帝許之仍詔自非功如太宰始封無後者皆不得以為比及太常議謚博士秦秀曰充悖禮溺情以亂大倫|{
	悖蒲内翻}
昔鄫養外孫莒公子為後春秋書莒人滅鄫|{
	春秋襄六年莒人滅鄫公羊傳曰取後於莒也莒女有為鄫夫人者立其出也穀梁傳曰莒人滅鄫非滅也立異姓以涖祭祀滅亡之道也}
絶父祖之血食開朝廷之亂原案諡法昏亂紀度曰荒請謚荒公帝不從更諡曰武 閏月丙子廣陸成侯李胤薨 齊王攸德望日隆荀朂馮紞楊珧皆惡之|{
	惡烏路翻}
紞言於帝曰陛下詔諸侯之國宜從親者始親者莫如齊王今獨留京師可乎朂曰百僚内外皆歸心齊王陛下萬歲後太子不得立矣陛下試詔齊王之國必舉朝以為不可則臣言驗矣帝以為然冬十二月甲申詔曰古者九命作伯或入毗朝政或出御方嶽|{
	朝直遥翻周禮九命作伯鄭玄曰上公有功德者加命為二伯得征五侯九伯者也鄭司農云長諸侯為方伯}
其揆一也侍中司空齊王攸佐命立勲劬勞王室其以為大司馬都督青州諸軍事侍中如故仍加崇典禮主者詳案舊制施行以汝南王亮為太尉録尚書事領太子太傅光祿大夫山濤為司徒尚書令衛瓘為司空征東大將軍王渾上書以為攸至親盛德侔於周公宜贊皇朝與聞政事|{
	與讀曰預}
今出攸之國假以都督虚號而無典戎幹方之實|{
	典戎典兵也詩韓奕曰幹不庭方言為楨榦也}
虧友于欵篤之義懼非陛下追述先帝文明太后待攸之宿意也|{
	待攸事見上卷咸寧二年}
若以同姓寵之太厚則有吳楚逆亂之謀漢之呂霍王氏皆何人也|{
	渾之意蓋謂齊王不當疑三楊不當信也}
歷觀古今苟事之輕重所在無不為害唯當任正道而求忠良耳若以智計猜物雖親見疑至于疏者庸可保乎愚以為太子太保缺宜留攸居之與汝南王亮楊珧共幹朝事三人齊位足相持正既無偏重相傾之埶又不失親親仁覆之恩計之盡善者也|{
	覆敷又翻}
於是扶風王駿光禄大夫李憙中護軍羊琇侍中王濟甄德皆切諫|{
	憙許記翻又音熹甄之人翻}
帝並不從濟使其妻常山公主及德妻長廣公主俱入稽顙涕泣請帝留攸|{
	稽音啟}
帝怒謂侍中王戎曰兄弟至親今出齊王自是朕家事而甄德王濟連遣婦來生哭人邪乃出濟為國子祭酒德為大鴻臚|{
	自侍中出為外朝官}
羊琇與北軍中候成粲謀見楊珧手刃殺之|{
	北軍中候漢官掌北軍五營魏省泰始四年罷中軍將軍置北軍中候七年又罷中領軍併焉}
珧知之辭疾不出諷有司奏琇左遷太僕琇憤怨發病卒李憙亦以年老遜位卒於家憙在朝|{
	朝直遥翻下同}
姻親故人與之分衣共食而未嘗私以王官人以此稱之是歲散騎常侍薛瑩卒或謂吳郡陸喜曰瑩於吳士當為第一乎喜曰瑩在四五之間安得為第一夫以孫皓無道吳國之士沈默其體潜而勿用者第一也|{
	沈持林翻}
避尊居卑禄以代耕者第二也侃然體國執正不懼者第三也斟酌時宜時獻微益者第四也温恭脩慎不為謟首者第五也過此以往不足復數|{
	復扶又翻}
故彼上士多淪沒而遠悔吝中士有聲位而近禍殃觀瑩之處身本末又安得為第一乎|{
	遠于願翻近其靳翻處昌呂翻}


四年春正月甲申以尚書右僕射魏舒為左僕射下邳王晃為右僕射晃孚之子也 戊午新沓康伯山濤薨|{
	魏明帝景初三年以遼東東沓縣吏民過海居齊郡界者立為新沓縣}
帝命太常議崇錫齊王之物博士庾旉太叔廣劉暾|{
	旉讀曰敷太叔複姓鄭莊公之弟段封於京謂之京城太叔其後以為氏又衛有太叔儀暾他昆翻}
繆蔚郭頤秦秀傅珍上表曰|{
	繆靡幼翻又莫六翻姓也蔚紆勿翻}
昔周選建明德以左右王室周公康叔耼季皆入為三公|{
	左傳衛大祝子魚曰武王之母弟八人周公為太宰康叔為司寇季為司空左右讀如佐佑乃甘翻}
明股肱之任重守地之位輕也漢諸侯王位在丞相三公上其入讚朝政者乃有兼官其出之國亦不復假台司虛名為隆寵也|{
	漢諸侯王讃朝政者惟東平王蒼耳}
今使齊王賢邪則不宜以母弟之親尊居魯衛之常職不賢邪不宜大啟土宇表見東海也古禮三公無職坐而論道不聞以方任嬰之|{
	嬰縈也}
惟宣王救急朝夕然後命召穆公征淮夷故其詩曰徐方不回王曰旋歸|{
	見詩江漢常武篇}
宰相不得久在外也今天下已定六合為家將數延三事與論太平之基|{
	數色角翻}
而更出之去王城二千里違舊章矣|{
	司馬彪郡國志齊國在洛陽東千八百里}
旉純之子暾毅之子也旉既具草先以呈純純不禁事過太常鄭默博士祭酒曹志|{
	續漢志博士祭酒一人本僕射中興轉為祭酒胡廣曰官名祭酒皆一位之元長也}
志愴然歎曰安有如此之才如此之親不得樹本助化而遠出海隅晉室之隆其殆矣乎乃奏議曰古之夾輔王室同姓則周公異姓則太公皆身居朝廷五世反葬|{
	禮記檀弓曰太公封於營邱比及五世皆反葬于周古人曰狐死正邱首仁也}
及其衰也雖有五霸代興豈與周召之治同日而論哉|{
	言五霸代興以尊周室不可與周召夾輔之治同日而論也治直吏翻}
自羲皇以來豈一姓所能獨有當推至公之心與天下共其利害乃能享國久長是以秦魏欲獨擅其權而纔得沒身周漢能分其利而親疎為用此前事之明驗也志以為當如博士等議帝覽之大怒曰曹志尚不明吾心况四海乎|{
	謂曹志本魏陳思王植之子植於魏文帝兄弟也文帝之禁制植者為何如今尚不能明吾之心乎}
且謂博士不答所問而答所不問|{
	所問者崇錫齊王禮物而已不問齊王當出與不當出也}
横造異論下有司策免鄭默|{
	横下孟翻下戶嫁翻}
於是尚書朱整褚䂮奏志等侵官離局|{
	䂮離灼翻離力智翻}
迷罔朝廷崇飾惡言假託無諱請收志等付廷尉科罪詔免志官以公還第|{
	志在魏嗣爵陳王晉受禪降為鄄城縣公}
其餘皆付廷尉科罪庾純詣廷尉自首旉以議草見示愚淺聽之詔免純罪|{
	首式又翻}
廷尉劉頌奏旉等大不敬當棄市尚書奏請報聽廷尉行刑尚書夏侯駿曰官立八座正為此時|{
	六曹尚書并令僕為八座為于偽翻}
乃獨為駮議|{
	駮北角翻}
左僕射下邳王晃亦從駿議奏留中七日乃詔曰旉是議主應為戮首但旉家人自首宜并廣等七人皆丐其死命|{
	丐貸也}
並除名二月詔以濟南郡益齊國|{
	濟子禮翻}
己丑立齊王攸子長樂亭侯寔為北海王|{
	樂音洛}
命攸備物典策設軒縣之樂|{
	樂天子宫縣諸侯軒縣軒縣者缺其一面縣讀曰懸}
六佾之舞黄鉞朝車乘輿之副從焉|{
	朝直遥翻乘繩證翻}
三月辛丑朔日有食之齊獻王攸憤怒發病乞守先后陵|{
	先后謂文明皇后也}
帝不許

遣御醫診視|{
	診止忍翻候脉也}
諸醫希旨皆言無疾河南尹向雄諫曰陛下子弟雖多然有德望者少齊王卧居京邑所益實深不可不思也帝不納雄憤恚而卒|{
	恚於避翻}
攸疾轉篤帝猶催上道|{
	上時掌翻}
攸自強入辭素持容儀疾雖困尚自整厲舉止如常帝益疑其無疾辭出數日歐血而薨帝往臨喪攸子冏號踊|{
	號戶刀翻}
訴父病為醫所誣詔即誅醫以冏為嗣|{
	冏俱永翻}
初帝愛攸甚篤為荀朂馮紞等所搆欲為身後之慮故出之及薨帝哀慟不已馮紞侍側曰齊王名過其實天下歸之今自薨殞社稷之福也陛下何哀之過帝收淚而止詔攸喪禮依安平獻王故事|{
	事見七十九卷泰始八年}
攸舉動以禮鮮有過事雖帝亦敬憚之每引之同處必擇言而後發|{
	鮮息善翻處昌呂翻}
夏五月己亥琅邪武王伷薨 冬十一月以尚書左僕射魏舒為司徒河南及荆楊等六州大水|{
	荆強也言其氣躁強亦曰警也言南蠻數為寇逆其}


|{
	人有道後服無道先強當警備也又云取荆山以名州統江夏南郡襄陽南陽順陽義陽新城魏興上庸建平宜都南平武陵天門長沙衡陽湘東零陵邵陵桂陽武昌安成}
歸命侯孫皓卒 是歲鮮卑慕容涉歸卒弟刪簒立 |{
	考異曰載記刪作耐今從燕書}
將殺涉歸子廆廆亡匿于遼東徐郁家|{
	廆戶賄翻又五罪翻載記曰廆字奕洛瓖杜佑曰本名若洛廆}


五年春正月己亥有青龍二見武庫井中|{
	見賢遍翻考異曰五行志作癸卯今從帝紀}
帝觀之有喜色百官將賀尚書左僕射劉毅表曰昔龍降夏庭卒為周禍|{
	國語曰夏之衰也褒人之神化為二龍以伺于夏庭夏后卜殺之與去之與止之莫吉卜請其漦而藏之吉乃布幣而策告之龍亡而漦在櫝而藏之及殷周莫之發也及厲王之末發而觀之漦流于庭不可除也王使婦人不幃而譟之化為玄蚖以入于王府府之童妾未筓亂而遭之既笄而孕當宣王而生不夫而育故懼而棄之鬻弧服者取之以逃于褒褒人有獄以入于幽王王遂嬖之使為后生伯服欲殺太子以成伯服太子奔申申侯與犬戎伐王殺之驪山下夏戶雅翻卒子恤翻}
易稱潜龍勿用陽在下也|{
	易乾之初九爻辭}
尋案舊典無賀龍之禮帝從之 初陳羣以吏部不能審覈天下之士故令郡國各置中正州置大中正皆取本土之人任朝廷官德充才盛者為之使銓次等級以為九品|{
	事見六十九卷魏文帝黄初元年}
有言行脩著則升之|{
	行下孟翻}
道義虧缺則降之吏部憑之以補授百官行之浸久中正或非其人姦敝日滋劉毅上疏曰今立中正定九品高下任意榮辱在手操人主之威福奪天朝之權埶|{
	操千高翻朝直遥翻}
公無考校之負私無告訐之忌|{
	謂銓次高下或有不當而在公不以考校失實謂罪負發人隱慝無所不至而在私不以告訐為避忌}
用心百態營求萬端亷讓之風滅爭訟之俗成臣竊為聖朝恥之|{
	為于偽翻}
蓋中正之設於損政之道有八高下逐彊弱是非隨興衰一人之身旬日異狀上品無寒門下品無埶族一也置州都者|{
	州都謂中正}
本取州里清議咸所歸服將以鎮異同一言議也今重其任而輕其人使駁違之論横於州里|{
	駁北角翻横戶孟翻}
嫌讐之隙結於大臣二也本立格之體為九品者謂才德有優劣倫輩有首尾也今乃使優劣易地首尾倒錯三也|{
	錯千故翻}
陛下賞善罰惡無不裁之以法獨置中正委以一國之重曾無賞罰之防又禁人不得訴訟使之縱横任意|{
	縱子容翻}
無所顧憚諸受枉者抱怨積直不獲上聞四也一國之士多者千數或流徙異邦或取給殊方|{
	謂衣食有不給者客於殊方以取給也}
面猶不識况盡其才而中正知與不知皆當品狀采譽於臺府|{
	譽音余}
納毁於流言任已則有不識之蔽聽受則有彼此之偏五也凡求人才欲以治民也|{
	治直之翻}
今當官著効者或附卑品在官無績者更獲高叙是為抑功實而隆空名長浮華而廢考績六也|{
	長知兩翻}
凡官不同人事不同能今不狀其才之所宜而但第為九品以品取人或非才能之所長以狀取人則為本品之所限徒結白論|{
	白素也釋素餐者以為空餐白論猶空言也}
而品狀相妨七也九品所下不彰其罪所上不列其善各任愛憎以植其私天下之人焉得不懈德行而鋭人事八也|{
	焉於䖍翻懈古隘翻}
由此論之職名中正實為姦府事名九品而有八損古今之失莫大於此愚臣以為宜罷中正除九品棄魏氏之敝法更立一代之美制太尉汝南王亮司空衛瓘亦上疏曰魏氏承喪亂之後|{
	喪息浪翻}
人士流移考詳無地故立九品之制粗且為一時選用之本耳|{
	粗坐五翻}
今九域同規大化方始臣等以為宜皆蕩除末法咸用土斷|{
	以土著為斷也斷丁亂翻}
自公卿以下以所居為正無復縣客|{
	縣讀曰懸}
遠屬異土盡除中正九品之制使舉善進才各由鄉論則華競自息各求於已矣始平王文學江夏李重上疏|{
	自魏以來王國置師友文學各一人夏戶雅翻上時掌翻}
以為九品既除宜先開移徙聽相并就則土斷之實行矣帝雖善其言而終不能改也冬十二月庚午大赦 閏月當陽成侯杜預卒 是歲塞外匈奴胡太阿厚帥部落二萬九千三百人來降|{
	帥讀曰率降戶江翻}
帝處之塞内西河|{
	處昌呂翻}
罷寧州入益州置南夷校尉以護之|{
	置寧州見七十九卷泰始七年 考異曰地理志太康三年廢寧州置南夷校尉今從華陽國志}


六年春正月尚書左僕射劉毅致仕尋卒 |{
	考異曰晉春秋在七年十月今從本傳}
戊辰以王渾為尚書左僕射渾子濟為侍中渾主者處事不當|{
	尚書主者也處昌呂翻當丁浪翻}
濟明法繩之|{
	侍中管門下諸事故得䋲以法}
濟從兄佑素與濟不協|{
	從才容翻}
因毁濟不能容其父帝由是疎濟後坐事免官濟性豪侈帝謂侍中和嶠曰我將罵濟而後官之如何嶠曰濟俊爽恐不可屈帝召濟切讓之既而曰頗知愧不|{
	不讀曰否}
濟曰尺布斗粟之謡常為陛下愧之|{
	謂帝不能容齊王攸也為于偽翻}
他人能令親者疎臣不能令親者親|{
	謂諫而不聽也}
以此愧陛下耳帝默然嶠洽之孫也|{
	和洽見六十六卷漢獻帝建安十四年}
青梁幽冀州旱 秋八月丙戌朔日有食之 冬十二月庚子襄陽武侯王濬卒 是歲慕容刪為其下所殺部衆復迎涉歸子廆而立之涉歸與宇文部素有隙|{
	宇文部亦鮮卑種其先有大人曰普回因狩得玉璽三紐文曰皇帝璽普回以為天授其俗謂天子曰宇文故國號宇文併以為氏何氏姓苑曰宇文氏出自炎帝其後以嘗草之功鮮卑呼草為俟汾遂號為俟汾氏後世通稱俟汾蓋音訛也代為鮮卑單于}
廆請討之朝廷弗許廆怒入宼遼西殺略甚衆帝遣幽州軍討廆戰于肥如|{
	肥如縣屬遼西郡應劭曰肥子奔燕燕封於此賢曰肥如今平州}
廆衆大敗自是每歲犯邉又東擊扶餘扶餘王依慮自殺|{
	慮音閭}
子弟走保沃沮|{
	沮千余翻}
廆夷其國城驅萬餘人而歸七年春正月甲寅朔日有食之魏舒稱疾固請遜位以劇陽子罷舒所為必先行而後言遜位之際莫有知者|{
	考異曰舒遜位紀傳皆無年月本傳曰以災異遜位帝不聽後因正旦朝罷還第表送章綬按本傳又曰遜}


|{
	位之際人莫知者若今年正旦日食遜位至它年正旦乃送章綬不得云人無知者蓋止因今者正旦朝罷遂以災異遜位不復起耳}
衛瓘與舒書曰每與足下共論此事日日未果|{
	瓘言亦欲遜位與共論此事日復一日未果如言}
可謂瞻之在前忽焉在後矣|{
	用論語顔淵之言}
夏慕容廆寇遼東故扶餘王依慮子依羅求帥見人還復舊國請援于東夷校尉何龕|{
	帥讀曰率下同見賢遍翻見人謂見存之人也龕口含翻晉志曰武帝置南蠻校尉於襄陽西戎校尉於長安南夷校尉於寧州東夷校尉蓋亦帝所置治遼東}
龕遣督護賈沈將兵送之|{
	魏晉之間方鎮各置督護領兵之官也沈持林翻}
廆遣其將孫丁帥騎邀之於路|{
	騎奇寄翻}
沈力戰斬丁遂復扶餘 秋匈奴胡都大博及萎莎胡各帥種落十萬餘口詣雍州降|{
	楊正衡曰莎素和翻帥讀曰率據晉書萎莎胡北狄種蓋亦匈奴類也杜佑曰晉史云北狄各以部落為類其入居塞内者有屠各萎莎羌渠賀賴等種種章勇翻雍於用翻}
九月戊寅扶風武王駿薨冬十一月壬子以隴西王泰都督關中諸軍事泰宣帝弟馗之子也|{
	馗渠龜翻}
是歲鮮卑拓跋悉鹿卒|{
	鹿一作禄}
弟綽立|{
	自泰始以來鮮卑慕容拓跋二}


|{
	部日以彊盛故史著其世}


八年春正月戊申朔日有食之 太廟殿䧟九月改營太廟作者六萬人 是歲匈奴都督大豆得一育鞠等復帥衆落萬一千五百口來降|{
	魏既分塞内匈奴為五部矣自去年來匈奴帥種落來降者十有餘萬口史不言所以處之之地此必自塞外來北匈奴之種落也復扶又翻}


九年春正月壬申朔日有食之|{
	比三年正旦日食帝尋晏駕晉以大亂天之示戒蓋昭昭矣}
夏六月庚子朔日有食之 郡國三十二大旱 秋八月壬子星隕如雨 地震

資治通鑑卷八十一
