\chapter{資治通鑑卷一百二十二}
宋 司馬光 撰

胡三省 音註

宋紀四|{
	起重光協洽盡旃蒙大淵獻凡五年}


太祖文皇帝上之下

元嘉八年春正月壬午朔燕大赦改元大興 丙申檀道濟等自清水救滑臺魏叔孫建長孫道生拒之丁酉道濟至夀張遇魏安平公乙旃眷|{
	魏收官氏志獻帝命叔父之裔為乙旃氏}
道濟帥寧朔將軍王仲德驍騎將軍段宏奮擊大破之|{
	帥讀曰率驍堅堯翻騎奇寄翻}
轉戰至高梁亭斬魏濟州刺史悉煩庫結|{
	魏明元帝泰常八年置濟州治碻磝城濟子禮翻}
夏主擊秦將姚獻敗之|{
	將即亮翻敗補邁翻}
遂遣其叔父北平公韋代帥衆一萬攻南安|{
	去年暮末保南安}
城中大饑人相食秦侍中征虜將軍出連輔政侍中右衛將軍乞伏延祚吏部尚書乞伏跋跋踰城犇夏秦王暮末窮蹙輿櫬出降|{
	乞伏氏四主四十九年而滅櫬初覲翻降戶江翻}
并沮渠興國送于上邽|{
	興國為秦所禽見上卷六年沮子余翻}
秦太子司直焦楷犇廣寧|{
	太子司直掌糾劾宫僚及帥府兵晉志無是官當是二趙燕秦所置}
泣謂其父遺曰大人荷國寵靈|{
	荷下可翻}
居藩鎮重任今本朝顛覆豈得不率見衆唱大義以殄寇讎|{
	朝直遥翻見賢遍翻}
遺曰今主上已陷賊庭吾非愛死而忘義顧以大兵追之是趣絶其命也|{
	趣讀曰促}
不如擇王族之賢者奉以為主而伐之庶有濟也楷乃築壇誓衆二旬之間赴者萬餘人會遺病卒|{
	卒子恤翻}
楷不能獨舉事亡犇河西 二月戊午以尚書右僕射江夷為湘州刺史 檀道濟等進至濟上|{
	濟子禮翻下同}
二十餘日間前後與魏三十餘戰道濟多捷軍至歷城|{
	歷城縣自漢以來屬濟南郡宋為冀州刺史治所}
叔孫建等縱輕騎邀其前後焚燒草穀|{
	騎奇寄翻}
道濟軍乏食不能進由是安頡司馬楚之等得專力攻滑臺|{
	頡戶結翻}
魏主復使楚兵將軍王慧龍助之|{
	復扶又翻}
朱脩之堅守數月糧盡與士卒熏鼠食之辛酉魏克滑臺執脩之及東郡太守申謨|{
	東郡自漢魏以來治白馬白馬滑臺之地也}
虜獲萬餘人謨鍾之曾孫也|{
	申鍾見九十五卷晉成帝咸和五年}
癸酉魏主還平城大饗告廟將帥及百官皆受賞戰士賜復十年|{
	賞北伐柔然西伐夏南禦宋之功也將即亮翻帥所類翻復方目翻復勿事也下復境同}
於是魏南鄙大水民多饑死尚書令劉絜言於魏主曰自頃邊寇内侵戎車屢駕天贊聖明所在克殄方難既平|{
	難乃旦翻}
皆蒙優錫而郡國之民雖不征討服勤農桑以供軍國實經世之大本府庫之所資今自山以東徧遭水害應加哀矜以弘覆育|{
	覆敷又翻下米覆同}
魏主從之復境内一歲租賦 檀道濟等食盡自歷城引還軍士有亡降魏者具告之魏人追之衆忷懼將潰|{
	降戶江翻忷許拱翻}
道濟夜唱籌量沙以所餘少米覆其上|{
	量音良少詩沼翻下同}
及旦魏軍見之謂道濟資糧有餘以降者為妄而斬之時道濟兵少魏兵甚盛騎士四合道濟命軍士皆被甲|{
	被皮義翻}
已白服乘輿引兵徐出魏人以為有伏兵不敢逼稍稍引退道濟全軍而返青州刺史蕭思話聞道濟南歸欲委鎮保險|{
	宋青州治東陽城}
濟南太守蕭承之固諫不從丁丑思話棄鎮犇平昌|{
	平昌縣前漢屬琅邪後漢屬北海晉太康地志屬城陽惠帝分立平昌郡五代志密州膠西縣舊曰黔陬置平昌郡}
參軍劉振之戍下邳聞之亦委城走魏軍竟不至而東陽積聚已為百姓所焚|{
	積子智翻凡指所聚之物曰積則去聲聚才諭翻}
思話坐徵繋尚方 燕王立夫人慕容氏為王后 庚戌魏安頡等還平城魏主嘉朱脩之守節妻以宗女|{
	為脩之自北還張本妻子細翻}
初帝之遣到彦之也戒之曰若北國兵動先其未至徑前入河|{
	先悉薦翻}
若其不動留彭城勿進及安頡得宋俘魏主始聞其言謂公卿曰卿輩前謂我用崔浩計為謬驚怖固諫|{
	怖普布翻崔浩計見上卷元年}
常勝之家始皆自謂踰人至於歸終|{
	歸終謂事勢究極處}
乃不能及司馬楚之上疏以為諸方已平請大舉伐宋魏主以兵久勞不許徵楚之為散騎常侍|{
	散悉亶翻騎奇寄翻}
以王慧龍為滎陽太守|{
	魏雖置滎陽太守實以虎牢為重鎮按魏書宮氏志高宗太安三年始以諸部護軍各為太守蓋是時唯以滎陽太守命王慧龍至太安三年遂悉改之也守式又翻}
慧龍在郡十年農戰並脩大著聲績歸附者萬餘家帝縱反間于魏云慧龍自以功高位下欲引宋人入寇因執司馬楚之以叛|{
	間古莧翻楚之時屯潁川}
魏主聞之賜慧龍璽書曰劉義隆畏將軍如虎欲相中害|{
	璽斯氏翻中竹仲翻}
朕自知之風塵之言想不足介意帝復遣刺客呂玄伯刺之|{
	復扶又翻}
曰得慧龍首封二百戶男賞絹千匹玄伯詐為降人|{
	降戶江翻}
求屏人有所論慧龍疑之使人探其懷得尺刀|{
	屏必郢翻探吐南翻}
玄伯叩頭請死慧龍曰各為其主耳釋之|{
	為于偽翻}
左右諫曰宋人為謀未已不殺玄伯無以制將來慧龍曰死生有命彼亦安能害我我以仁義為扞蔽又何憂乎遂捨之|{
	史因慧龍守滎陽終言之}
夏五月庚寅魏主如雲中 六月乙丑大赦 夏主殺乞伏暮末及其宗族五百人 夏主畏魏人之逼擁秦民十餘萬口|{
	秦民所得乞伏氏之民也}
自治城濟河欲擊河西王蒙遜而奪其地吐谷渾王慕璝遣益州刺史慕利延寧州刺史拾虔|{
	璝古囬翻 考異曰十六國春秋作沒利延拾虎今從宋書}
帥騎三萬乘其半濟邀擊之執夏主定以歸|{
	赫連氏歷三主二十六年而滅自是中原及西北之地一歸于魏矣帥讀曰率騎奇寄翻}
沮渠興國被創而死|{
	沮子余翻被皮義翻創初良翻}
拾虔樹洛干之子也|{
	樹洛干卒於晉安帝義熙十三年}
魏之邊吏獲柔然邏者二十餘人魏主賜衣服而遣

之柔然感悦閏月乙未柔然敇連可汗遣使詣魏魏主厚禮之|{
	邏郎佐翻使疏吏翻}
魏主遣散騎侍郎周紹來聘|{
	散悉亶翻騎奇寄翻}
且求昏帝依違答之 荆州刺史江夏王義恭年寖長欲專政事長史劉湛每裁抑之遂與湛有隙|{
	宋制幼王臨州率以長史行府州事事皆決于行事義恭欲專之而湛不可遂有隙長知兩翻}
帝心重湛使人詰讓義恭且和解之|{
	詰去吉翻}
是時王華王曇首皆卒|{
	卒子恤翻}
領軍將軍殷景仁素與湛善白帝以時賢零落徵湛為太子詹事加給事中共參政事以雍州刺史張邵代湛為撫軍長史南蠻校尉頃之邵坐在雍州營私畜聚贓滿二百四十五萬下廷尉當死|{
	永嘉五年邵刺雍州雍於用翻}
左衛將軍謝述上表陳邵先朝舊勲|{
	武帝討桓玄邵白父敞表獻忠欵又不附劉毅}
宜蒙優貸帝手詔酬納免邵官削爵土述謂其子綜曰主上矜邵夙誠特加曲恕吾所言謬會故特見酬納耳若此迹宣布則為侵奪主恩不可之大者也使綜對前焚之帝後謂邵曰卿之獲免謝述有力焉 秋七月己酉魏主如河西 八月乙酉河西王蒙遜遣子安周入侍于魏 吐谷渾王慕璝遣侍郎謝太寧奉表于魏請送赫連定己丑魏以慕璝為大將軍西秦王|{
	璝古囬翻}
左僕射臨川王義慶固求解職甲辰以義慶為中書令丹陽尹如故 九月癸丑魏主還宫庚申加太尉長孫嵩柱國大將軍|{
	柱國大將軍始此}
以左光祿大夫崔浩為司徒征西大將軍長孫道生為司空道生性清儉一熊皮鄣泥數十年不易|{
	類篇馬障泥曰韂韂昌艷翻蜀註云遮擁泥濘也}
魏主使歌工歷頌羣臣曰智如崔浩廉若道生 魏主欲選使者詣河西崔浩薦尚書李順乃以順為太常拜河西王蒙遜為侍中都督涼州西域羌戎諸軍事太傅行征西大將軍涼州牧涼王王武威張掖敦煌酒泉西海金城西平七郡|{
	使疏吏翻王武于況翻敦徒門翻}
冊曰盛衰存亡與魏升降北盡窮髮南極庸㟭西被崐嶺東至河曲|{
	經典釋文司馬曰窮髮北極之下無毛之地也按毛草也地理書云山以草木為髪庸魏興上庸之地㟭㟭山也崑嶺謂崑崙河曲朔方之河曲也㟭與岷同被皮義翻}
王實征之以夾輔皇室置將相羣卿百官承制假授建天子旌旗出入警蹕如漢初諸侯王故事 壬申魏主詔曰今二寇摧殄將偃武修文理廢職舉逸民范陽盧玄博陵崔綽趙郡李靈河間邢穎勃海高允廣平游雅太原張偉等皆賢雋之胄冠冕周邦|{
	周當作州}
易曰我有好爵吾與爾縻之|{
	易中孚九二爻辭}
如玄之比者盡敇州郡以禮遣遂徵玄等及州郡所遣至者數百人差次叙用崔綽以母老固辭玄等皆拜中書博士玄諶之曾孫|{
	晉永嘉之後盧諶展轉于石氏之間冉閔之敗逆死於兵諶氏壬翻}
靈順之從父兄也|{
	從才用翻}
玄舅崔浩每與玄言輒歎曰對子真使我懷古之情更深|{
	盧玄字子真}
浩欲大整流品明辨姓族玄止之曰夫創制立事各有其時樂為此者詎有幾人宜加三思|{
	樂音洛三息暫翻}
浩不從由是得罪于衆 初魏昭成帝始制法令|{
	什翼犍諡昭成帝}
反逆者族其餘當死者聽入金馬贖罪殺人者聽與死家牛馬葬具以平之盜官物一備五私物一備十|{
	備陪償也今人多云陪備}
四部大人共坐王庭決辭訟無繫訊連逮之苦境内安之太祖入中原|{
	道武帝廟號太祖}
患前代律令峻密命三公郎王德刪定務崇簡易|{
	事見一百十卷晉安帝隆安二年易以䜴翻}
季年被疾刑罰濫酷|{
	事見一百十一卷晉安帝隆安四年被皮義翻}
太宗承之吏文亦深|{
	明元帝廟號太祖}
冬十月戊寅世祖命崔浩更定律令除五歲四歲刑增一年刑巫蠱者負羖羊抱犬沈諸淵|{
	羖果五翻說文夏羊壯曰羖一說羖䍽羊也沈持林翻}
初令官階九品者得以官爵除刑|{
	漢官以石秩為差魏晉始定品秩之次}
婦人當刑而孕產後百日乃決闕左懸登聞鼔以達寃人|{
	禹令有獄訟者揺鞀周禮左嘉石以平罷民皆所以達幽枉也登聞鼓令負寃者得詣闕檛鼔登時上聞也}
魏主如漠南十一月丙辰北部敇勒莫弗庫若干|{
	高車酋長謂之莫弗 考異曰後魏書北史本紀皆作敇勒鄧淵傳皆作高車按高車即敇勒别名也}
帥所部數萬騎驅鹿數百萬頭詣魏主行在|{
	帥讀曰率騎奇寄翻}
魏主大獵以賜從官|{
	從才用翻}
十二月丁丑還宫 是歲涼王改元義和林邑王范陽邁寇九德交州兵擊却之|{
	九德郡古越裳氏國隋唐為驩州}


九年春正月丙午魏主尊保太后竇氏為皇太后|{
	尊保母為毋非禮也}
立貴人赫連氏為皇后子晃為皇太子大赦改元延和 燕王立慕容后之子王仁為太子 三月庚戌衛將軍王弘進位太保加中書監丁巳征南大將軍檀道濟進位司空還鎮尋陽|{
	道濟自歷城還師至建康復使之還鎮尋陽}
壬申吐谷渾王慕璝送赫連定于魏魏人殺之慕璝上表曰臣俘擒僭逆獻捷王府爵秩雖崇而土不增廓車旗既飾而財不周賞願垂鑒察魏主下其議|{
	下戶嫁翻}
公卿以為慕璝所致唯定而已塞外之民皆為己有而貪求無厭不可許也|{
	厭於鹽翻}
魏主乃詔曰西秦王所得金城枹罕隴西之地|{
	枹音膚}
朕即與之乃是裂土何須復廓|{
	復扶又翻}
西秦欵至綿絹隨使疏數臨時增益非一賜而止也|{
	使疎吏翻下同疏與疎同數所角翻}
自是慕璝貢使至魏者稍簡 魏方士祁纎奏改代為萬年代尹為萬年尹以代令為萬年令崔浩曰昔太祖應天受命兼稱代魏以法殷商|{
	帝嚳都亳子契受封于商自契至湯八遷湯始都亳從先王居謂之亳殷故兼稱殷商}
國家積德當享年萬億不待假名以為益也纎之所聞皆非正義宜復舊號魏主從之 夏五月壬申華容文昭公王弘卒弘明敏有思致而輕率少威儀性褊隘好折辱人人以此少之|{
	思相吏翻少詩沼翻褊方緬翻好呼到翻折之舌翻}
雖貴顯不營財利及卒家無餘業帝聞之特賜錢百萬米千斛 魏主治兵于南郊謀伐燕|{
	治直之翻}
帝遣使者趙道生聘于魏 六月戊寅司徒南徐州刺史彭城王義康改領揚州刺史|{
	王弘卒義康始領揚州}
詔分青州置冀州|{
	宋冀州領廣川平原清河樂陵魏郡河間頓丘高陽勃海九郡皆僑置於河濟間}
治歷城 吐谷渾王慕璝遣其司馬趙叙入貢且來告捷|{
	告擒赫連定之捷也璝古囬翻}
庚寅魏主伐燕命太子晃錄尚書事時晃纔五歲又遣左僕射安原建寧王崇等屯漠南以備柔然 辛卯魏主遣散騎常侍鄧穎來聘 乙未以吐谷渾王慕璝為都督西秦河沙三州諸軍事征西大將軍西秦河二州刺史進爵隴西王且命慕璝悉歸南方將士先沒於夏者得百五十餘人|{
	劉義真之敗沒于夏者}
又加北秦州刺史楊難當征西將軍難當以兄子保宗為鎮南將軍鎮宕昌|{
	宕昌隋唐為宕州之地宕徒浪翻}
以其子順為秦州刺史守上邽保宗謀襲難當事泄難當囚之 壬寅以江夏王義恭為都督南兖等六州諸軍事開府儀同三司南兖州刺史臨川王義慶為都督荆雍等七州諸軍事荆州刺史|{
	雍於用翻}
竟陵王義宣為中書監衡陽王義季為南徐州刺史初高祖以荆州居上流之重土地廣遠資實兵甲居朝廷之半故遺詔令諸子居之上以義慶宗室令美且烈武王有大功于社稷|{
	臨川王道規諡烈武王}
故特用之 秋七月己未魏主至濡水|{
	水經濡水自塞外來過遼西令支肥如海陽等縣而入于海濡乃官翻}
庚申遣安東將軍奚斤幽州民及密雲丁零萬餘人|{
	魏收曰道武帝皇始二年置密雲郡密雲縣治提攜城本漢厗奚縣地孟康曰厗音題}
運攻具出南道會和龍魏主至遼西燕王遣其侍御史崔聘奉牛酒犒師己巳魏主至和龍庚午以領軍將軍殷景仁為尚書僕射太子詹事劉

湛為領軍將軍 益州刺史劉道濟粹之弟也信任長史費謙别駕張熙等聚歛興利|{
	費扶沸翻}
傷政害民立官冶禁民鼓鑄而貴賣鐵器商賈失業吁嗟滿路|{
	歛力瞻翻賈音古}
流民許穆之變姓名稱司馬飛龍自云晉室近親往依氐王楊難當難當因民之怨資飛龍以兵使侵擾益州飛龍招合蜀人得千餘人攻殺巴興令|{
	沈約曰巴興令徐志不註置立疑李氏所立屬遂寧郡宋白曰晉永初十一年置巴興縣西魏改曰長江縣唐屬遂州}
逐隂平太守|{
	晉孝武帝泰始中置隂平郡至武帝永初間又分為南隂平北隂平此南隂平也隋併南隂平入雒縣宋白曰文州古隂平也戰國氐羌所據漢為隂平道魏晉為隂平郡隂平縣永嘉末太守玉鑒以郡降李雄晉人於是悉流移於蜀漢其氐羌並屬楊茂搜此郡不復預受正朔故南史諸志悉無所錄其晉人流寓於蜀者仍於益州立南北二隂平寓于漢中者亦於梁州立南北二隂平今劍州隂平縣益州之北隂平郡也}
道濟遣軍擊斬之道濟欲以五城人帛氐奴梁顯為參軍督護|{
	孫愐曰帛姓也}
費謙固執不與氐奴等與鄉人趙廣構扇縣人詐言司馬殿下猶在陽泉山中聚衆得數千人引向廣漢|{
	沈約曰蜀分緜竹立陽泉縣屬廣漢郡隋復併入緜竹}
道濟參軍程展會治中李抗之將五百人擊之皆敗死巴西人唐頻聚衆應之趙廣等進攻涪城陷之|{
	將即亮翻涪音浮}
於是涪陵江陽遂寧諸郡守皆棄城走蜀土僑舊俱反|{
	沈約曰遂寧郡永初郡國志有之疑晉末分廣漢所立唐為遂州守手又翻下太守同}
燕石城太守李崇等十郡降于魏|{
	石城縣前漢屬右北平燕分置石城郡魏真君八年置建德郡於白狼以石城為縣屬焉降戶江翻}
魏主發其民三萬穿圍塹以守和龍崇績之子也|{
	李績見一百卷晉穆帝升平四年塹七艷翻}
八月燕王使數萬人出戰魏昌黎公丘等擊破之|{
	公丘之下當有漏字}
死者萬餘人燕尚書高紹帥萬餘家保羌胡固|{
	帥讀曰率}
辛巳魏主攻紹斬之平東將軍賀多羅攻帶方撫軍大將軍永昌王健攻建德驃騎大將軍樂平王丕攻冀陽皆拔之九月乙卯魏主引兵西還徙營丘成周遼東樂浪帶方玄菟六郡民三萬家於幽州|{
	五代志曰後魏置營州於和龍城領建德冀陽遼東樂浪營丘等郡龍城大興永樂帶方定荒石城廣都陽武襄平新昌平剛柳城富平等縣蓋燕國自慕容以來分置郡縣於遼西其後或省或併為郡為縣皆不可攷如玄菟郡亦當置於遼西也驃匹妙翻騎奇寄翻樂浪音洛琅菟同都翻}
燕尚書郭淵勸燕王送欵獻女於魏乞為附庸燕王曰負舋在前結忿已深降附取死不如守志更圖也|{
	舋許覲翻降戶江翻}
魏主之圍和龍也宿衛之士多在戰陳|{
	陳與陣同}
行宫人少雲中鎮將朱脩之謀與南人襲殺魏主因入和龍浮海南歸以告冠軍將軍毛脩之毛脩之不從乃止|{
	少詩沼翻將即亮翻冠古玩翻}
既而事泄朱脩之逃奔燕魏人數伐燕|{
	數所角翻}
燕王遣脩之南歸求救脩之汎海至東萊遂還建康拜黄門侍郎趙廣等進攻成都劉道濟嬰城自守賊衆屯聚日久

不見司馬飛龍欲散去廣懼將三千人及羽儀詣陽泉寺|{
	以羽為儀故曰羽儀}
詐云迎飛龍至則謂道人枹罕程道養曰|{
	枹音膚}
汝但自言是飛龍則坐享富貴不則斷頭|{
	斷丁管翻}
道養惶怖許諾|{
	怖普布翻}
廣乃推道養為蜀王車騎大將軍益梁二州牧改元泰始備置百官以道養弟道助為驃騎將軍長沙王鎮涪城趙廣帛氐奴梁顯及其黨張尋嚴遐皆為將軍奉道養還成都衆至十餘萬四面圍城使人謂道濟曰但送費謙張熙來我輩自解去道濟遣中兵參軍裴方明任浪之各將千餘人出戰皆敗還|{
	任音壬}
冬十一月乙巳魏主還平城 壬子以少府中山甄

法崇為益州刺史|{
	代劉道濟也甄之人翻}
初燕王嫡妃王氏生長樂公崇崇於兄弟為最長|{
	樂音洛最長知兩翻}
及即位立慕容氏為王后王氏不得立又黜崇使鎮肥如|{
	燕以幽州刺史鎮肥如遼西之地也}
崇母弟廣平公朗樂陵公邈相謂曰今國家將亡人無愚智皆知之王復受慕容后之譛|{
	復扶又翻}
吾兄弟死無日矣乃相與亡奔遼西說崇使降魏崇從之會魏主使給事郎王德招崇|{
	給事郎北史作給事中說輸芮翻降戶江翻下同}
十二月己丑崇使邈如魏請舉郡降燕王聞之使其將封羽圍崇於遼西|{
	將即亮翻下同}
魏主徵諸名士之未仕者州郡多逼遣之魏主聞之下詔令守宰以禮申諭|{
	申重也重直龍翻守式又翻}
任其進退毋得逼遣 初帝以少子紹為廬陵孝獻王嗣|{
	義真諡曰孝獻少詩照翻}
以江夏王義恭子朗為營陽王嗣庚寅封紹為廬陵王朗為南豐縣王 裴方明等復出撃程道養營破之|{
	復扶又翻下衆復豈復順復同}
焚其積聚|{
	積子賜翻聚才諭翻}
賊黨江陽楊孟子將千餘人屯城南|{
	江陽隋併入陽州隆山縣}
參軍梁雋之統南樓投書說諭孟子邀使入城見劉道濟道濟板為主簿克期討賊趙廣知其謀孟子懼將所領奔晉原晉原太守文仲興與之同拒守趙廣遣帛氐奴攻晉原破之|{
	李雄分蜀郡為漢原郡晉穆帝更名晉原郡治江原縣唐為蜀州晉原縣宋白曰晉原縣本漢江原縣地李雄立江原郡晉改為多融縣又改晉原以縣界晉原山為名}
仲興孟子皆死裴方明復出擊賊屢戰破之賊遂大潰程道養收衆得七千人還廣漢趙廣别將五千餘人還涪城先是張熙說道濟糶倉穀|{
	先悉薦翻說輸芮翻}
故自九月末圍城至十二月糧儲俱盡方明將二千人出城求食為賊所敗|{
	敗補邁翻}
單馬獨還賊衆復大集方明夜縋而上|{
	縋馳偽翻上時掌翻}
道濟為設食|{
	為于偽翻下微為同}
涕泣不能食道濟曰卿非大丈夫小敗何苦賊勢既衰臺兵垂至但令卿還何憂于賊即減左右以配之賊於城外揚言云方明已死城中大恐道濟夜列炬火出方明以示衆衆乃安道濟悉出財物於北射堂令方明募人時城中或傳道濟已死莫有應者梁携之說道濟遣左右給使三十餘人出外|{
	梁携之蓋即梁雋之携字悞也}
且告之曰吾病小損各聽歸家休息給使既出城中乃安應募者日有千餘人 初晉謝混上晉陵公主|{
	晉陵公主晉孝武之女}
混死|{
	見一百十六卷晉安帝義熙八年}
詔公主與謝氏絶婚公主悉以混家事委混從子弘微|{
	從才用翻}
混仍世宰輔僮僕千人唯有二女年數歲弘微為之紀理生業一錢尺帛有文簿|{
	為于偽翻}
九年而高祖即位公主降號東鄉君聽還謝氏入門室宇倉廩不異平日田疇墾闢有加于舊東鄉君歎曰僕射平生重此子可謂知人僕射為不亡矣|{
	混仕晉為尚書左僕射}
親舊見者為之流涕是歲東鄉君卒|{
	為于偽翻卒子恤翻}
公私咸謂貲財宜歸二女田宅僮僕應屬弘微弘微一無所取自以私祿葬東鄉君混女夫殷叡好樗蒲|{
	好呼到翻}
聞弘微不取財物乃奪其妻妹及伯母兩姑之分以還戲責|{
	分扶問翻責如字又讀曰債}
内人皆化弘微之讓一無所争或譏之曰謝氏累世財產充殷君一朝戲責理之不允莫此為大卿視而不言譬棄物江海以為廉耳設使立清名而令家内不足亦吾所不取也弘微曰親戚争財為鄙之甚今内人尚能無言豈可導之使争乎分多共少不至有乏身死之後豈復見關也 秃髮保周自涼奔魏|{
	保周奔涼見一百十六卷晉安帝義熙十年}
魏封保周為張掖公魏李順復奉使至涼|{
	復扶又翻使疏吏翻}
涼王蒙遜遣中兵校郎楊定歸謂順曰年衰多疾腰髀不隨不堪拜伏比三五日消息小差當相見|{
	比必寐翻及也}
順曰王之老疾朝廷所知豈得自安不見詔使明日蒙遜延順入至庭中蒙遜箕坐應几無動起之狀|{
	師古曰謂伸兩脚而坐其形如箕隱於靳翻}
順正色大言曰不謂此叟無禮乃至于此今不憂覆亡而敢陵侮天地䰟魄逝矣何用見之握節將出涼王使定歸追止之曰太常既雅恕衰疾|{
	雅素也}
傳聞朝廷有不拜之詔是以敢自安耳順曰齊桓公九合諸侯一匡天下周天子賜胙命無下拜桓公猶不敢失臣禮下拜登受|{
	齊桓公合諸侯于葵丘王使宰孔賜胙齊侯將下拜孔曰天子以伯舅耄老加勞賜一級無下拜對曰天威不違顔咫尺小白余敢貪天子之命無下拜恐隕越于下以為天子羞敢不下拜下拜登受}
今王雖功高未如齊桓朝廷雖相崇重未有不拜之詔而遽自偃蹇此豈社稷之福邪蒙遜乃起拜受詔使還魏主問以涼事順曰蒙遜控制河右踰三十年|{
	晉安帝隆安五年蒙遜殺段業篡有其國至是三十一年}
經涉艱難粗識機變|{
	粗坐五翻}
綏集荒裔羣下畏服雖不能貽厥孫謀猶足以終其一世然禮者德之輿敬者身之基也蒙遜無禮不敬以臣觀之不復年矣|{
	復扶又翻言其死在朝夕}
魏主曰易世之後何時當滅順曰蒙遜諸子臣畧見之皆庸才也如聞敦煌太守牧犍器性粗立|{
	敦徒門翻犍居言翻}
繼蒙遜者必此人也然比之於父皆云不及此殆天之所以資聖明也 |{
	考異曰後魏書順初奉冊拜沮渠蒙遜為涼州牧即有蒙遜不拜及順使還論牧犍事南史順冊拜蒙遜還拜都督四州長安鎮都大將徵為四部尚書加常侍延和初使涼始有不拜等事今據順云不復周矣明年蒙遜死帝曰卿言蒙遜死驗矣故從南史}
魏主曰朕方有事東方|{
	謂方圖燕也}
未暇西略如卿所言不過數年之外不為晩也初罽賓沙門曇無䜟自云能使鬼治病且有袐術|{
	北史曰曇無䜟自云能使鬼療病令婦人多子罽音計曇徒含翻䜟楚譛翻治直之翻}
涼王蒙遜甚重之謂之聖人諸女及子婦皆往受術魏主聞之使李順往徵之蒙遜留不遣仍殺之魏主由是怒涼蒙遜荒淫猜虐羣下苦之|{
	為魏滅涼張本}


十年春正月乙卯魏主遣永昌王健督諸軍救遼西|{
	以馮崇被圍也}
己未大赦 丙寅魏以樂安王範為都督秦雍等五州諸軍事衛大將軍開府儀同三司長安鎮都大將|{
	都大將又在鎮大將之上雍於用翻將即亮翻下同}
魏主以範年少|{
	少詩沼翻}
更選舊德平西將軍崔徽征北大將軍鴈門張黎為之副共鎮長安徽宏之弟也|{
	崔宏崔浩之父也}
範謙恭寛惠徽務敦大體黎清約公平政刑簡易|{
	易以豉翻}
輕傜薄賦關中遂安二月庚午魏主以馮崇為都督幽平東夷諸軍事車騎大將軍幽平二州牧封遼西王錄其國尚書事食遼西十郡承制假授尚書刺史征虜以下官|{
	自征虜以下雜號將軍皆得假授騎奇寄翻}
魏平涼休屠征西將軍金崖|{
	句斷屠直於翻}
羌涇州刺史狄子玉|{
	魏置涇州于安定郡治臨涇城}
與安定鎮將延普争權崖子玉舉兵攻普不克退保胡空谷|{
	即胡空堡之地}
魏主以虎牢鎮大將陸俟為安定鎮大將擊崖等皆擒之魏主徵陸俟為散騎常侍出為懷荒鎮大將|{
	懷荒鎮魏降高車所置六鎮之一也散悉亶翻騎奇寄翻}
未期歲高車諸莫弗訟俟嚴急無恩復請前鎮將郎孤|{
	復扶又翻下無復將復同}
魏主徵俟還以孤代之俟既至言于帝曰不過期年郎孤必敗高車必叛|{
	期讀曰朞}
帝怒切責之使以建業公歸第明年諸莫弗果殺郎孤而叛帝大驚立召俟問之曰卿何以知其然也俟曰高車不知上下之禮故臣臨之以威制之以法欲以漸訓導使知分限|{
	分扶問翻}
而諸莫弗惡臣所為|{
	惡烏路翻}
訟臣無恩稱孤之美臣以罪去孤獲還鎮悦其稱譽|{
	譽音余}
益收名聲專用寛恕待之無禮之人易生驕慢|{
	易以豉翻}
不過朞年無復上下孤所不堪必將復以法裁之如此則衆心怨懟必生禍亂矣|{
	魏裴潜去代郡而烏桓叛事亦如此懟直類翻}
帝笑曰卿身雖短思慮何長也即日復以為散騎常侍 壬午魏主如河西遣兼散騎常侍宋宣來聘且為太子晃求婚|{
	為于偽翻}
帝依違答之 劉道濟卒梁儁之裴方明等密埋其尸於齋後詐為道濟教命以答籖疏雖其母妻亦不知也程道養于毁金橋登壇郊天方明將三千人出擊之|{
	將即亮翻下同}
道養等大敗退保廣漢荆州刺史臨川王義慶以巴東太守周籍之督巴西等五郡諸軍事將二千人救成都 三月亡人司馬天助降於魏自稱晉會稽世子元顯之子|{
	降戶江翻會工外翻}
魏人以為青徐二州刺史東海公 壬子魏主還宫 趙廣等自廣漢至郫|{
	郫縣自漢以來屬蜀郡師古曰郫音疲}
連營百數周籍之與裴方明等合兵攻郫克之進擊廣等于廣漢廣等走還涪及五城|{
	涪音浮}
夏四月戊寅始發劉道濟喪 帝聞梁南秦二州刺史甄法護刑政不治|{
	甄之人翻}
失氐羌之和乃自徒中起蕭思話為梁南秦二州刺史 |{
	考異曰思話傳云楊難當寇漢中乃用思話按本紀及氏胡傳難當寇漢中皆在十一月}
法護法崇之兄也 涼王蒙遜病甚國人共議以世子菩提幼弱立菩提之兄敦煌太守牧犍為世子加中外都督大將軍錄尚書事|{
	菩薄乎翻犍居言翻 考異曰宋書十六國春秋作茂䖍後魏書紀傳作牧犍今從之}
蒙遜卒諡曰武宣王廟號太祖牧犍即河西王位大赦改元永和立子封壇為世子加撫軍大將軍錄尚書事遣使請命于魏牧犍聰頴好學|{
	使疏吏翻好呼到翻}
和雅有度量故國人立之先是魏主遣李順迎武宣王女為夫人|{
	先悉薦翻}
會卒牧犍稱先王遺意遣左丞宋繇送其妹興平公主于魏拜右昭儀|{
	李延夀曰魏主增置左右昭儀}
魏主謂李順曰卿言蒙遜死今則驗矣又言牧犍立何其妙哉朕克涼州亦當不遠於是賜絹千匹廏馬一乘|{
	乘繩證翻}
進號安西將軍寵待彌厚政事無巨細皆與之參議|{
	李順以言中見寵待而亦以為涼隱受誅為臣之不易也如此}
遣順拜牧犍都督涼沙河三州西域羌戎諸軍事車騎將軍開府儀同三司涼州刺史河西王|{
	騎奇寄翻}
以宋繇為河西王右相牧犍以無功受賞留順上表乞安平一號|{
	謂若安西將軍若平西將軍乞一號}
優詔不許牧犍尊敦煌劉昞為國師|{
	敦徒門翻}
親拜之命官屬以下皆北面受業 五月己亥魏主如山北|{
	武周山之北也}
林邑王范陽邁遣使入貢求領交州|{
	使疏吏翻}
詔答以道遠不許 裴方明進軍向涪城破張尋唐頻擒程道助斬嚴遐于是趙廣等皆奔散 六月魏永昌王健左僕射安原督諸軍擊和龍將軍樓㪍别將五千騎圍凡城|{
	魏書官氏志内入諸姓賀樓氏改為樓氏將即亮翻下同}
燕守將封羽以凡城降|{
	降戶江翻下同}
收其三千餘家而還 辛巳魏人秦雍兵一萬築小城於長安城内|{
	雍於用翻}
秋八月馮崇上表請說降其父|{
	說輸芮翻}
魏主不聽 九月益州刺史甄法崇至成都收費謙誅之|{
	費扶沸翻}
程道養張尋將二千餘家逃入郪山|{
	廣漢郪縣之山也師古曰郪音妻又音于私翻}
餘黨各擁衆藏竄山谷時出為寇不絶 戊午魏主遣兼大鴻臚崔賾持節拜氐王楊難當為征南大將軍開府儀同三司秦梁二州牧南秦王賾逞之子也|{
	崔逞自燕歸魏以侮慢為魏主珪所殺賾士革翻}
楊難當因蕭思話未至甄法護將下舉兵襲梁州破白馬獲晉昌太守張範|{
	白馬戊在沔水北即陽平關晉桓温平蜀以巴漢流人立晉昌郡于上庸之西}
敗法護參軍魯安期等|{
	敗補邁翻}
又攻葭萌獲晉夀太守范延朗|{
	晉孝武太元十五年梁州刺史周馥表分梓潼立晉夀郡古葭萌之地也葭音家}
冬十一月丁未法護棄城犇洋川之西城|{
	後魏方立洋川郡于漢中之西鄉縣此蓋因其地有洋水故謂之洋川洋音祥又如字}
難當遂有漢中之地以其司馬趙溫為梁秦二州刺史 甲寅魏主還宫 十一月己巳魏大赦 辛未魏主如隂山之北 魏寧朔將軍盧玄來聘 前祕書監謝靈運好為山澤之遊|{
	五年靈運免官故曰前好呼到翻}
窮幽極險從者數百人|{
	從才用翻}
伐木開徑百姓驚擾以為山賊會稽太守孟顗與靈運有隙|{
	會工外翻顗魚豈翻}
表其有異志兵自防靈運詣闕自陳上以為臨川内史靈運遊放自若廢棄郡事為有司所糾是歲司徒遣使隨州從事鄭望生收靈運|{
	使疏吏翻望生蓋為江州從事}
靈運執望生興兵逃逸作詩曰韓亡子房奮秦帝魯連恥|{
	靈運自以世為晉臣故賦是詩子房事見七卷秦始皇二十九年魯連事見考異}
追討擒之廷尉奏靈運率衆反叛論正斬刑上愛其才欲免官而已彭城王義康堅執謂不宜恕乃降死一等徙廣州久之或告靈運令人買兵器結健兒欲于三江口簒取之不果|{
	水經温水出䍧柯夜郎縣東至林廣縣為水灕水出陽海山南過蒼梧荔浦縣又南至廣信縣入于封水出臨賀郡馮乘縣西牛屯山西南流入廣信縣南流注于水此蓋三水所會之地謂之三江口}
詔於廣州棄市靈運恃才放逸多所陵忽故及於禍 魏立徐州於外黄以刁雍為刺史|{
	雍於用翻}


十一年春正月戊戌燕王遣使請和於魏|{
	使疏吏翻}
魏主不許 楊難當以克漢中告捷于魏送雍州流民七千家於長安|{
	雍於用翻}
蕭思話至襄陽遣横野司馬蕭承之為前驅|{
	思話時以横野將軍鎮梁州以承之為司馬}
承之緣道收兵得千人進據磝頭|{
	水經註漢水逕黄金南東流歷敖頭魏興安康縣治磝敖同音}
楊難當焚掠漢中引衆西還留趙溫守梁州又遣其魏興太守薛健據黄金山思話遣隂平太守蕭坦攻鐵城戍拔之|{
	黄金山註見七十四卷魏邵陵厲公正始五年水經註鐵城與黄金戍相對一城在山上一城在山下}
二月趙温薛健與其馮翊太守蒲甲子合攻坦營坦擊破之溫等退保西水|{
	水經註西水作酉水}
臨川王義慶遣龍驤將軍裴方明將三千人助承之|{
	驤思將翻將即亮翻下同}
拔黄金戍而據之溫棄州城退據小城健甲子退保下桃城思話繼至與承之共擊趙溫等屢破之行參軍王靈濟别將出洋川攻南城拔之|{
	沈約曰譙縱滅梁州刺史還治漢中之苞中縣所謂南城也余考前史漢中郡無苞中縣意即褒中縣蓋因語近而字遂訛也褒中縣在南鄭西南故謂之南城}
擒其守將趙英南城空無所資靈濟引兵還與承之合 魏主以西海公主妻柔然敇連可汗|{
	妻七細翻可從入聲汗音寒}
又納其妹為夫人遣潁川王提往逆之丁卯敇連遣其異母兄秃鹿傀送妹并獻馬二千匹|{
	傀公回翻}
魏主以其妹為左昭儀提曜之子也 辛卯魏主還宫三月甲寅復如河西|{
	復扶又翻}
楊難當遣其子和將兵與蒲甲子等共撃蕭承之相拒四十餘日圍承之數十重短兵接弓矢無所復施氐悉衣犀甲戈矛所不能入|{
	重直龍翻復扶又翻下可復同衣於既翻周禮考工記犀甲夀百年以牛皮為之}
承之斷矟長數尺以大斧椎之一矟輒貫數人|{
	斷丁管翻矟色角翻長直亮翻}
氐不能當燒營走據大桃閏月承之等追擊之至南城氐敗走斬獲甚衆悉收漢中故地置戍于葭萌水|{
	水經註白水出臨洮縣西南西傾山東南流至葭萌縣北因謂之葭萌水水有津關即所謂白水關也}
初桓希既敗|{
	希敗見一百十三卷晉安帝元興三年}
氐王楊盛據漢中梁州刺史范元之傅歆皆治魏興|{
	治直之翻}
唯得魏興上庸新城三郡及索邈為刺史|{
	見一百十六卷晉安帝義熙九年索昔各翻}
乃治南城至是南城為氐所焚不可復固蕭思話徙鎮南鄭|{
	自此梁州治南鄭}
甲戌赫連昌叛魏西走丙子河西候將格殺之|{
	此河西五原河西也候將斥候將也將即亮翻}
魏人并其羣弟誅之己卯魏主還宫 辛巳燕王遣尚書高顒上表稱藩請罪于魏|{
	顒魚容翻}
乞以季女充掖庭魏主乃許之徵其太子王仁入朝燕王送魏使者于什門還平城|{
	朝直遥翻使疏吏翻下同}
什門在燕二十一年不屈節|{
	什門使燕見一百十六卷晉安帝義熙十年 考異曰後魏書節義傳云什門在燕歷二十四年按後魏本紀神瑞元年八月遣于什門招諭馮跋至此年二十一年矣若二十四年乃在太延三年而太延二年馮氏亡矣}
魏主下詔褒稱以比蘇武拜治書御史|{
	治直之翻}
賜羊千口帛千匹策告宗廟頒示天下 戊子休屠金當川圍魏隂密|{
	隂密縣漢晉屬安定郡魏收志屬平涼郡括地志隂密故城在涇州鷄觚縣西屠直於翻}
夏四月乙未魏征西大將軍常山王素擊之丁未魏主行如河西壬戍獲當川斬之甄法護坐委鎮賜死于獄|{
	甄之人翻}
楊難當遣使奉表謝罪帝下詔赦之 河西王牧犍遣使上表告嗣位戊寅詔以牧犍為都督涼秦等四州諸軍事征西大將軍涼州刺史河西王|{
	犍居言翻}
六月甲辰魏主還宮 燕王不遣太子質魏|{
	質音致}
散騎常侍劉滋諫曰|{
	散悉亶翻騎奇寄翻}
昔劉禪有重山之險孫皓有長江之阻皆為晉擒|{
	劉禪事見七十八卷魏元帝景元四年孫浩事見七十九卷晉武帝太康元年重直龍翻}
何則疆弱之勢異也今吾弱於吳蜀而魏彊於晉不從其欲將有危亡之禍願亟遣太子而修政事撫百姓收離散賑饑窮勸農桑省賦役社稷猶庶幾可保燕王怒殺之辛亥魏主遣撫軍大將軍永昌王健等伐燕收其禾稼徙民而還|{
	還從宣翻又如字}
秋七月壬午魏主如美稷遂至隰城|{
	隰城縣自漢以來屬西河郡劉昫曰汾州西河縣漢美稷縣隋為隰城縣上元元年更名西河蓋二縣皆併于汾州西河縣矣}
命陽平王它督諸軍擊山胡白龍于西河|{
	山胡即稽胡一日步落稽蓋匈奴别種劉元海五部之苗裔也或云山戎赤狄之後自離石以西安定以東方七八百里居山谷間種落繁熾}
它熙之子也|{
	陽平王熙見一百十九卷武帝永初二年}
魏主輕山胡日引數十騎登山臨視之|{
	騎奇寄翻}
白龍伏壯士十餘處掩擊之魏主墜馬幾為所擒|{
	幾居希翻}
内入行長代人陳建以身扞之|{
	内入行長魏官也蓋選勇力之士入直禁中行長則其部帥也魏書官氏志次南諸姓侯莫陳氏改為陳氏行戶剛翻長知兩翻}
大呼奮擊殺胡數人身被十餘創|{
	呼火故翻被皮義翻}
魏主乃免九月戊子大破胡衆斬白龍屠其城冬十月甲午魏人破白龍餘黨于五原誅數千人以其妻子賜將士十一月魏主還宫十二月甲辰復如雲中|{
	復扶又翻}
十二年春正月己未朔日有食之 辛酉大赦 辛未上祀南郊 燕王數為魏所攻|{
	數所角翻}
遣使詣建康稱藩奉貢癸酉詔封為燕王江南謂之黄龍國|{
	以其都和龍也今北國以和龍為黄龍府}
甲申魏大赦改元太延 有老父投書於敦煌東門|{
	敦徒門翻}
求之不獲書曰涼王三十年若七年河西王牧犍以問奉常張慎對曰昔虢之將亡神降于莘|{
	左傳莊公三十二年有神降于莘虢公使祝應宗區史嚚享焉神賜之土田史嚚曰虢其亡乎吾聞之國將興聽於民將亡聽于神神聰明正直而壹者也依人而行虢多涼德其何土之能得後七年晉滅虢犍居言翻虢古百翻}
願陛下崇德脩政以享三十年之祚若盤于遊田荒于酒色臣恐七年將有大變牧犍不悦|{
	史言涼之將亡}
二月丁未魏主還宫 三月癸亥燕王遣大將湯燭入貢於魏|{
	將即亮翻}
辭以太子王仁有疾故未之遣 領軍將軍劉湛與僕射殷景仁素善湛之入也景仁實引之|{
	見上八年}
湛既至以景仁位遇本不踰已而一旦居前意甚憤憤俱被時遇以景仁專管内任謂為間已|{
	被皮義翻間古莧翻}
猜隙漸生知帝信仗景仁不可移奪時司徒義康專秉朝權|{
	朝直遙翻下同}
湛嘗為義康上佐|{
	見一百十九卷武帝永初元年}
遂委心自結欲因宰相之力以囬上意傾黜景仁獨當時務夏四月己巳帝加景仁中書令中護軍即家為府湛加太子詹事湛愈憤怒使義康毁景仁于帝帝遇之益隆景仁對親舊歎曰引之令入入便噬人乃稱疾解職表疏累上|{
	上時掌翻}
帝不許使停家養病湛議遣人若刼盜者於外殺之以為帝雖知當有以解之不能傷義康至親之愛帝微聞之遷護軍府于西掖門外使近宫禁|{
	近其靳翻}
故湛謀不行義康僚屬及諸附麗湛者潜相約勒無敢歷殷氏之門彭城王主簿沛郡劉敬文父成未悟其機詣景仁求郡敬文遽往謝湛曰老父悖耄遂就殷鐵干祿|{
	鐵景仁小字也悖蒲内翻}
由敬文闇淺上負生成闔門慙懼無地自處|{
	史言劉湛怙權時輩諂事之處昌呂翻}
唯後將軍司馬庾炳之遊二人之間皆得其歡心而密輸忠於朝廷景仁卧家不朝謁帝常使炳之銜命往來湛不疑也|{
	為後廢義康誅湛張本}
炳之登之之弟也|{
	庾登之見一百二十卷元嘉三年}
燕王遣右衛將軍孫德來乞師五月庚申魏主進宜都公穆夀爵為王汝隂公長孫

道生為上黨王宜城公奚斤為恒農王|{
	奚斤先封宜城王以罪降為公魏顯祖諱弘乃改弘農為恒農史以後來郡名書之長知兩翻恒戶登翻}
廣陵公樓伏連為廣陵王加夀征東大將軍夀辭曰臣祖父崇所以得効功前朝流福於後者由梁眷之忠也|{
	事見一百六卷晉孝武太元十年}
今眷元勲未錄而臣獨奕世受賞心實愧之魏主悦求眷後得其孫賜爵郡公夀觀之子也|{
	穆觀見一百十九卷武帝永初三年}
龜兹疏勒烏孫悦般渇槃陁鄯善焉耆車師粟持九

國入貢于魏|{
	龜兹疏勒烏孫鄯善焉耆車師漢時舊國也悦般國在烏孫西北去代一萬九百三十里其先北匈奴之部落為竇憲所破北單于度金微山西走康居其羸弱不能去住龜兹北地為悦般國涼州人猶謂之單于王渴槃陁國在蔥嶺東朱駒波西粟持當從魏書隋書作粟特粟特國在蔥嶺之西當康居西北去代一萬六千里漢之奄蔡國也龜兹音丘慈般釋典音鉢槃薄官翻陁徒河翻鄯上扇翻 考異曰後魏書皆作烏耆云漢時舊國也按漢書作焉耆今從之}
魏主以漢世雖通西域有求則卑辭而來無求則驕慢不服蓋自知去中國絶遠大兵不能至故也今報使往來徒為勞費終無所益欲不遣使|{
	使疏吏翻下同}
有司固請以為九國不憚險遠慕義入貢不宜拒絶以抑將來乃遣使者王恩生等二十輩使西域恩生等始度流沙為柔然所執恩生見敇連可汗|{
	可從刋入聲汗音寒}
持魏節不屈魏主聞之切責敇連敇連乃遣恩生等還竟不能達西域 甲戌魏主如雲中 六月甲午魏主以時和年豐嘉瑞沓臻詔大酺五日|{
	酺音蒲}
徧祭百神用答天貺 丙午高句麗王璉遣使入貢于魏|{
	句如字又音駒麗力知翻璉力展翻}
且請國諱魏主使錄帝系及諱以與之拜璉都督遼海諸軍事征東將軍遼東郡公高句麗王璉釗之曾孫也|{
	高句麗王制為燕所破見八十七卷晉成帝咸康八年}
戊申魏主命驃騎大將軍樂平王丕|{
	驃匹妙翻騎奇寄翻樂音洛}
鎮東大將軍徒河屈垣等帥騎四萬伐燕|{
	魏書官氏志内入諸姓尸突氏為屈氏}
揚州諸郡大水己酉運徐豫南兖穀以賑之|{
	賑津忍翻}
揚州西曹主簿沈亮建義以為酒糜穀而不足療饑|{
	自晉以來公府分東西曹各有掾屬主簿}
請權禁止詔從之亮林子之子也|{
	沈林子隨武帝征伐有功}
秋七月魏主畋于稒陽|{
	稒陽北出即光禄塞漢五原之北邊也師古曰稒音固}
己卯魏樂平王丕等至和龍燕王以牛酒犒軍|{
	犒苦到翻}
獻甲三千屈垣責其不送侍子掠男女六千口而還 八月丙戍魏主如河西九月甲戌還宮 魏左僕射河間公安原恃寵驕恣或告原謀為逆冬十月癸卯原坐族誅 甲辰魏主如定州十一月乙丑如冀州己巳畋于廣川丙子如鄴 魏人數伐燕|{
	數所角翻}
燕日危蹙上下憂懼太常楊㟭復勸燕王速遣太子入侍|{
	㟭音岷復扶又翻}
燕王曰吾未忍為此若事急且東依高麗以圖後舉㟭曰魏舉天下以擊一隅理無不克高麗無信始雖相親終恐為變燕王不聽密遣尚書陽伊請迎於高麗|{
	為燕王為高麗所殺張本麗力知翻}
丹陽尹蕭摹之上言佛化被于中國已歷四代|{
	四代漢魏晉宋也被皮義翻}
形像塔寺所在千數自頃以來情敬浮末不以精誠為至更以奢競為重材竹銅綵糜損無極無關神祗有累人事|{
	累力瑞翻}
不為之防流遁未息請自今欲鑄銅像及造塔寺者皆當列言須報乃得為之詔從之摹之思話從叔也|{
	從才用翻}
魏秦州刺史薛謹撃吐沒骨滅之 楊難當釋楊保宗之囚|{
	囚保宗事見上九年}
使鎮童亭|{
	水經註谷水出上邽東南注谷之山東北歷董亭下楊難當使兄子保宗鎮董亭即是亭也董童字相近 考異曰後魏書作薰亭宋書作童今從之}


資治通鑑卷一百二十二
