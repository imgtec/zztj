<!DOCTYPE html PUBLIC "-//W3C//DTD XHTML 1.0 Transitional//EN" "http://www.w3.org/TR/xhtml1/DTD/xhtml1-transitional.dtd">
<html xmlns="http://www.w3.org/1999/xhtml">
<head>
<meta http-equiv="Content-Type" content="text/html; charset=utf-8" />
<meta http-equiv="X-UA-Compatible" content="IE=Edge,chrome=1">
<title>資治通鑒_234-資治通鑑卷二百三十三_234-資治通鑑卷二百三十三</title>
<meta name="Keywords" content="資治通鑒_234-資治通鑑卷二百三十三_234-資治通鑑卷二百三十三">
<meta name="Description" content="資治通鑒_234-資治通鑑卷二百三十三_234-資治通鑑卷二百三十三">
<meta http-equiv="Cache-Control" content="no-transform" />
<meta http-equiv="Cache-Control" content="no-siteapp" />
<link href="/img/style.css" rel="stylesheet" type="text/css" />
<script src="/img/m.js?2020"></script> 
</head>
<body>
 <div class="ClassNavi">
<a  href="/24shi/">二十四史</a> | <a href="/SiKuQuanShu/">四库全书</a> | <a href="http://www.guoxuedashi.com/gjtsjc/"><font  color="#FF0000">古今图书集成</font></a> | <a href="/renwu/">历史人物</a> | <a href="/ShuoWenJieZi/"><font  color="#FF0000">说文解字</a></font> | <a href="/chengyu/">成语词典</a> | <a  target="_blank"  href="http://www.guoxuedashi.com/jgwhj/"><font  color="#FF0000">甲骨文合集</font></a> | <a href="/yzjwjc/"><font  color="#FF0000">殷周金文集成</font></a> | <a href="/xiangxingzi/"><font color="#0000FF">象形字典</font></a> | <a href="/13jing/"><font  color="#FF0000">十三经索引</font></a> | <a href="/zixing/"><font  color="#FF0000">字体转换器</font></a> | <a href="/zidian/xz/"><font color="#0000FF">篆书识别</font></a> | <a href="/jinfanyi/">近义反义词</a> | <a href="/duilian/">对联大全</a> | <a href="/jiapu/"><font  color="#0000FF">家谱族谱查询</font></a> | <a href="http://www.guoxuemi.com/hafo/" target="_blank" ><font color="#FF0000">哈佛古籍</font></a> 
</div>

 <!-- 头部导航开始 -->
<div class="w1180 head clearfix">
  <div class="head_logo l"><a title="国学大师官网" href="http://www.guoxuedashi.com" target="_blank"></a></div>
  <div class="head_sr l">
  <div id="head1">
  
  <a href="http://www.guoxuedashi.com/zidian/bujian/" target="_blank" ><img src="http://www.guoxuedashi.com/img/top1.gif" width="88" height="60" border="0" title="部件查字,支持20万汉字"></a>


<a href="http://www.guoxuedashi.com/help/yingpan.php" target="_blank"><img src="http://www.guoxuedashi.com/img/top230.gif" width="600" height="62" border="0" ></a>


  </div>
  <div id="head3"><a href="javascript:" onClick="javascript:window.external.AddFavorite(window.location.href,document.title);">添加收藏</a>
  <br><a href="/help/setie.php">搜索引擎</a>
  <br><a href="/help/zanzhu.php">赞助本站</a></div>
  <div id="head2">
 <a href="http://www.guoxuemi.com/" target="_blank"><img src="http://www.guoxuedashi.com/img/guoxuemi.gif" width="95" height="62" border="0" style="margin-left:2px;" title="国学迷"></a>
  

  </div>
</div>
  <div class="clear"></div>
  <div class="head_nav">
  <p><a href="/">首页</a> | <a href="/ShuKu/">国学书库</a> | <a href="/guji/">影印古籍</a> | <a href="/shici/">诗词宝典</a> | <a   href="/SiKuQuanShu/gxjx.php">精选</a> <b>|</b> <a href="/zidian/">汉语字典</a> | <a href="/hydcd/">汉语词典</a> | <a href="http://www.guoxuedashi.com/zidian/bujian/"><font  color="#CC0066">部件查字</font></a> | <a href="http://www.sfds.cn/"><font  color="#CC0066">书法大师</font></a> | <a href="/jgwhj/">甲骨文</a> <b>|</b> <a href="/b/4/"><font  color="#CC0066">解密</font></a> | <a href="/renwu/">历史人物</a> | <a href="/diangu/">历史典故</a> | <a href="/xingshi/">姓氏</a> | <a href="/minzu/">民族</a> <b>|</b> <a href="/mz/"><font  color="#CC0066">世界名著</font></a> | <a href="/download/">软件下载</a>
</p>
<p><a href="/b/"><font  color="#CC0066">历史</font></a> | <a href="http://skqs.guoxuedashi.com/" target="_blank">四库全书</a> |  <a href="http://www.guoxuedashi.com/search/" target="_blank"><font  color="#CC0066">全文检索</font></a> | <a href="http://www.guoxuedashi.com/shumu/">古籍书目</a> | <a   href="/24shi/">正史</a> <b>|</b> <a href="/chengyu/">成语词典</a> | <a href="/kangxi/" title="康熙字典">康熙字典</a> | <a href="/ShuoWenJieZi/">说文解字</a> | <a href="/zixing/yanbian/">字形演变</a> | <a href="/yzjwjc/">金 文</a> <b>|</b>  <a href="/shijian/nian-hao/">年号</a> | <a href="/diming/">历史地名</a> | <a href="/shijian/">历史事件</a> | <a href="/guanzhi/">官职</a> | <a href="/lishi/">知识</a> <b>|</b> <a href="/zhongyi/">中医中药</a> | <a href="http://www.guoxuedashi.com/forum/">留言反馈</a>
</p>
  </div>
</div>
<!-- 头部导航END --> 
<!-- 内容区开始 --> 
<div class="w1180 clearfix">
  <div class="info l">
   
<div class="clearfix" style="background:#f5faff;">
<script src='http://www.guoxuedashi.com/img/headersou.js'></script>

</div>
  <div class="info_tree"><a href="http://www.guoxuedashi.com">首页</a> > <a href="/SiKuQuanShu/fanti/">四库全书</a>
 > <h1>资治通鉴</h1> <!--         下载:【右键另存为】即可 --></div>
  <div class="info_content zj clearfix">
  
<div class="info_txt clearfix" id="show">
<center style="font-size:24px;">234-資治通鑑卷二百三十三</center>
    資治通鑑卷二百三十三 宋 司馬光 撰<br />
<br />
  胡三省 音註<br />
<br />
  唐紀四十九【起彊圉單閼八月盡重光恊洽凡四年有奇】<br />
<br />
  德宗神武聖文皇帝八<br />
<br />
  貞元三年八月辛巳朔日有食之 吐蕃尚結贊遣五騎送崔漢衡歸【吐從暾入聲漢衡為吐蕃所擒見上卷是年五月騎奇寄翻】且上表求和至潘原李觀語之以有詔不納吐蕃使者【上時掌翻觀古玩翻語牛倨翻使疏吏翻】受其表而却其人 初兵部侍郎同平章事柳渾與張延賞俱為相渾議事數異同【相息亮翻敷所角翻】延賞使所親謂曰相公舊德但節言于廟堂則重位可久渾曰爲吾謝張公【爲于偽翻】柳渾頭可斷【斷音短】舌不可禁【禁居吟翻】由是交惡上好文雅醖藉【好呼到翻醖紆運翻藉夜翻史炤曰醖藉有雅度之稱余謂炤說非也記禮器云禮有擯詔樂有相步温之至也鄭氏注云皆為温藉重禮也皇氏云温謂丞藉凡玉以物緼裏丞藉君子亦以威儀擯相以自丞藉温與緼同】而渾質直輕侻無威儀【侻它活翻】于上前時發俚語上不悦欲黜為王府長史李泌言渾褊直無它【俚音里長知丈翻褊補典翻】故事罷相無為長史者又欲以為王傅泌請以為常侍上曰苟得罷之無不可者【于此可以見帝之親任泌泌薄必翻】己丑渾罷為左散騎常侍【散悉亶翻】初郜國大長公主適駙馬都尉蕭升升復之從兄弟也【郜音告長知兩翻從才用翻】公主不謹詹事李昇蜀州别駕蕭鼎【武后垂拱二年分益州置蜀州漢州】彭州司馬李萬豐陽令韋恪【豐陽縣屬商州漢商縣地晉分商縣置豐陽縣以川為名舊治吉川城麟德元年移治豐陽川】皆出入主第主女為太子妃始者上恩禮甚厚主常直乘肩輿抵東宫宗戚皆疾之或告主淫亂且為厭禱【厭於琰翻又一叶翻】上大怒幽主于禁中切責太子太子不知所對請與蕭妃離昏上召李泌告之且曰舒王近已長立【長知兩翻】孝友温仁泌曰何至于是陛下惟有一子 【考異曰按德宗十一子誼謜其所生外猶有九子而泌云惟冇一子者蓋當是時小王或未生誼謜之外尚有昭靖子也】奈何一旦疑之欲廢之而立姪得無失計乎上勃然怒曰卿何得間人父子【間古莧翻】誰語卿舒王為姪者對曰陛下自言之大歷初陛下語臣【語牛倨翻】今日得數子臣請其故陛下言昭靖諸子主上令吾子之【昭靖太子上弟邈也】今陛下所生之子猶疑之何有于姪【當此之時微李泌孰能言及此者】舒王雖孝自今陛下宜努力勿復望其孝矣【囚父子天性推而言及人情利害極處以感動之復扶又翻】上曰卿不愛家族乎對曰臣惟愛家族故不敢不盡言若畏陛下盛怒而為曲從陛下明日悔之必尤臣云吾獨任汝為相不力諫使至此必復殺而子【而汝也】臣老矣餘年不足惜若寃殺臣子使臣以姪爲嗣臣未知得歆其祀乎因嗚咽流涕【人以自家真情感動之】上亦泣曰事已如此使朕如何而可對曰此大事願陛下審圖之臣始謂陛下聖德當使海外蠻夷皆戴之如父母豈謂自有子而疑之至此乎臣今盡言不敢避忌諱自古父子相疑未有不亡國覆家者陛下記昔在彭原建寜何故而誅上曰建寜叔實寃肅宗性急譛之者深耳【建寜王倓德宗之叔也倓寃死事見二百一十九卷肅宗至德元載】泌曰臣昔以建寜之故固辭官爵誓不近天子左右【近其靳翻】不幸今日復為陛下相又覩茲事【復扶又翻相息亮翻】臣在彭原承恩無比竟不敢言建寜之寃及臨辭乃言之肅宗亦悔而泣【事見二百二十卷至德二載】先帝自建寜之死常懷危懼臣亦為先帝誦黄臺瓜辭以防讒構之端【事見同上爲于僞翻】上曰朕固知之意色稍解乃曰貞觀開元皆易太子何故不亡對曰臣方欲言之昔承乾屢嘗監國【監古衘翻】託附者衆東宫甲士甚多與宰相侯君集謀反事覺太宗使其舅長孫無忌與朝臣數十人鞫之事狀顯白然後集百官而議之當時言者猶云願陛下不失為慈父使太子得終天年太宗從之并廢魏王泰【事見一百九十七卷貞觀十七年】陛下既知肅宗性急以建寜為寃臣不勝慶幸【勝音升】願陛下戒覆車之失從容三日【從千容翻】䆒其端緒而思之陛下必釋然知太子之無它矣若果有其迹當召大臣知理義者二十人與臣鞫其左右必有實狀願陛下如貞觀之法行之并廢舒王而立皇孫則百代之後有天下者猶陛下子孫也至于開元之末武惠妃譖太子瑛兄弟殺之海内寃憤【事見二百一十四卷玄宗開元二十五年】此乃百代所當戒又可法乎且陛下昔嘗令太子見臣于蓬萊池【大明宮中蓬萊殿北冇太液池池中有蓬萊山所謂蓬萊池盖即此也】觀其容表非有蠭目犲聲商臣之相也【左傳楚成王將立太子商臣令尹子上曰不可是人也蠭目而犲聲忍人也不聽卒立之商臣後果以宫甲圍成王而殺之】正恐失于柔仁耳又太子自貞元以來常居少陽院【大明宫中有少陽院在浴堂殿之東温室殿西南少詩照翻】在寢殿之側【德宗常居浴堂殿】未嘗接外人預外事安有異謀乎彼譖人者巧詐百端雖有手書如晉愍懷【事見八十三卷西晉惠帝元康九年】衷甲如太子瑛【開元二十五年楊洄復構太子瑛鄂王瑶光王琚與妃兄薛鏞有異謀武惠妃使人詭召太子二王曰宫中有賊請甲以入太子從之妃白帝曰太子二王謀反甲而來帝使中人視之如言遂並廢為庶人】猶未可信况但以妻母有罪為累乎【累良瑞翻下累汝同】幸陛下語臣【語牛倨翻】臣敢以家族保太子必不知謀曏使楊素許敬宗李林甫之徒承此旨已就舒王圖定策之功矣上曰此朕家事何豫于卿而力爭如此對曰天子以四海為家臣今獨任宰相之重四海之内一物失所責歸于臣况坐視太子寃横而不言【横戶孟翻】臣罪大矣上曰為卿遷延至明日思之【爲于僞翻】泌抽笏叩頭而泣曰如此臣知陛下父子慈孝如初矣然陛下還宮當自審思勿露此意于左右露之則彼皆欲樹功于舒王太子危矣上曰具曉卿意泌歸謂子弟曰吾本不樂富貴而命與願違今累汝曹矣【樂音洛累力瑞翻】太子遣人謝泌曰若必不可救欲先自仰藥何如【言欲飲藥而死也】泌曰必無此慮願太子起敬起孝【起敬起孝禮記之言】苟泌身不存則事不可知耳閒一日【按經典釋文間音間厠之間】上開延英殿獨召泌【宋白曰唐制内中有公事商量即降宣頭付閤門開延英閤門翻宣申中書并牓正衙門如中書有公事敷奏即宰臣入牓子奏請開延英只是宰臣赴對】流涕闌干【泣涕縱横為闌干一曰闌干淚不斷貌】撫其背曰非卿切言朕今日悔無及矣皆如卿言太子仁孝實無它也自今軍國及朕家事皆當謀于卿矣泌拜賀因曰陛下聖明察太子無罪臣報國畢矣臣前日驚悸亡魂【悸其季翻心動也】不可復用【復扶又翻】願乞骸骨上曰朕父子賴卿得全方屬子孫【屬之欲翻】使卿代代富貴以報德何為出此言乎甲午詔李萬不知避宗宜杖死【左傳齊盧蒲癸臣于慶舍有寵妻之以女慶舍之士謂盧蒲癸曰男女辨姓子不避宗何也癸曰宗不避余余獨安避之】李昇等及公主五子皆流嶺南及遠州 戊申吐蕃帥羌渾之衆寇隴州連營數十里京城震恐【帥讀曰率】九月丁卯遣神策將石季章戍武功决勝軍使唐良臣戍百里城丁巳吐蕃大掠汧陽吳山華亭【吳山縣屬隴州隋之長蛇縣地唐貞觀元年更名以縣有吳山也史炤曰華亭本屬安定郡後屬隴州垂拱二年更名曰亭川元和三年省入汧源汧口堅翻】老弱者殺之或斷手鑿目棄之而去【斷音短】驅丁壯萬餘悉送安化陜西【安化峽當在秦州清水縣界九域志平凉西南七十里冇安化縣又隴州汧陽縣有安化鎮】將分隸羌渾乃告之曰聽爾東向哭辭鄉國衆大哭赴崖谷死傷者千餘人未幾【幾居豈翻】吐蕃之衆復至圍隴州【復扶又翻】刺史韓清與神策副將蘇太平夜出兵擊却之【瀰兖翻】 上謂李泌曰每歲諸道貢獻共值錢五十萬緡今歲僅得三十萬緡【泌薄必翻緡眉巾翻】言此誠知失體然宫中用度殊不足泌曰古者天子不私求財【春秋左傳之言】今請歲宫中錢百萬緡願陛下不受諸道貢獻及罷宣索【遣中使以聖旨就有司宣取財物謂之宣索索山客翻】必有所須請降勅折税【折之舌翻】不使姦吏因緣誅剝上從之 回紇合骨咄祿可汗屢求和親且請昏上未之許會邉將告乏馬無以給之【紇下没翻咄當没翻可從刋入聲汗戶千翻將即亮翻】李泌言于上曰陛下誠用臣策數年之後馬賤于今十倍矣上曰何故對曰願陛下推至公之心屈己徇人爲社稷大計臣乃敢言上曰卿何自疑若是對曰臣願陛下北和回紇南通雲南西結大食天竺如此則吐蕃自困馬亦易致矣【吐從暾入聲易弋豉翻】上曰三國當如卿言至于回紇則不可【以陜州之辱恨回紇也】泌曰臣固知陛下如此所以不敢早言【見上卷是年七月】爲今之計當以回紇為先三國差緩耳【三國謂雲南大食天竺】上曰惟回紇卿勿言泌曰臣備位宰相事有可否在陛下何至不許臣言【相息亮翻】上曰朕于卿言皆聽之矣至于回紇宜待子孫于朕之時則固不可泌曰豈非以陜州之恥邪上曰然韋少華等以朕之故受辱而死【事見二百二十二卷寶應元年陜失冉翻邪音耶少始照翻】朕豈能忘之屬國家多難【屬之欲翻難乃旦翻】未暇報之和則决不可卿勿更言泌曰害少華者乃牟羽可汗陛下即位舉兵入寇未出其境今合骨咄祿可汗殺之然則今可汗乃有功于陛下宜受封賞又何怨邪其後張光晟殺突董九百餘人【殺牟羽殺突董事並見二百二十六卷建中元年】合骨咄祿竟不敢殺朝廷使者【見二百二十七卷建中三年】然則合骨咄祿固無罪矣上曰卿以和回紇爲是則朕固非邪對曰臣爲社稷而言【為于僞翻】若苟合取容何以見肅宗代宗於天上【凡人言死則曰見某人于地下人主之前尊君之祖父則曰見于天上言其神靈在天死則將得見之】上曰容朕徐思之自是泌凡十五餘對未嘗不論回紇事上終不許泌曰陛下既不許回紇和親願賜臣骸骨上曰朕非拒諫但欲與卿較理耳何至遽欲去朕邪對曰陛下許臣言理此固天下之福也上曰朕不惜屈己與之和但不能負少華輩對曰以臣觀之少華輩負陛下非陛下負之也上曰何故對曰昔回紇葉護將兵助討安慶緒肅宗但令臣宴勞之于元帥府先帝未嘗見也【勞力到翻討安慶緒之時代宗以廣平王為元帥】葉護固邀臣至其營肅宗猶不許及大軍將發先帝始與相見所以然者彼戎狄犲狼也舉兵入中國之腹不得不過為之防也陛下在陜富于春秋少華輩不能深慮以萬乘元子徑造其營【造七到翻】又不先與之議相見之儀使彼得肆其桀驁【驁五告翻】豈非少華輩負陛下邪死不足償責矣且香積之捷葉護欲引兵入長安先帝親拜之于馬前以止之葉護遂不敢入城【事見二百二十卷肅宗至德二載】當時觀者十萬餘人皆嘆息曰廣平真華夷主也然則先帝所屈者少所伸者多矣葉護乃牟羽之叔父也牟羽身為可汗舉全國之兵赴中原之難【難乃旦翻】故其志氣驕矜敢責禮于陛下陛下天資神武不為之屈當是之時臣不敢言其它若可汗留陛下于營中歡飲十日天下豈得不寒心哉【不敢察察言故云爾】而天威所臨豺狼馴擾【馴從也善也擾者順也】可汗母捧陛下于貂裘叱退左右親送陛下乘馬而歸陛下以香積之事觀之則屈己爲是乎不屈為是乎陛下屈于牟羽乎牟羽屈于陛下乎上謂李晟馬燧曰故舊不宜相逢朕素怨回紇今聞泌言香積之事朕自覺少理【此多少之少音詩沼翻】卿二人以為何如對曰果如泌所言則回紇似可恕上曰卿二人復不與朕【復扶又翻】朕當柰何泌曰臣以為回紇不足怨曏來宰相乃可怨耳今回紇可汗殺牟羽其國人有再復京城之勲【回紇至德二載與代宗復兩京寶應元年又與帝復東京是有再復京城之勲】夫何罪乎吐蕃幸國之災陷河隴數千里之地又引兵入京城使先帝蒙塵于陜【見二百二十三卷代宗廣德元年】此乃必報之讐况其贊普尚存【言牟羽已死則回紇為可恕贊普尚存則國讐當必復】宰相不為陛下别白言此【爲于偽翻别彼列翻】乃欲和吐蕃以攻回紇此為可怨耳上曰朕與之為怨已久又聞吐蕃刼盟今往與之和得無復拒我為夷狄之笑乎【復扶又翻】對曰不然臣曩在彭原今可汗為胡祿都督與今國相白婆帝皆從葉護而來臣待之頗親厚故聞臣為相而求和安有復相拒乎臣今請以書與之約稱臣為陛下子每使來不過二百人印馬不過千匹【唐六典冇諸監馬印凡諸監馬駒以小官字印印左膞以年辰印印右髀以監名依左右廂印印尾側若形容端正擬送尚乘者則不須印監名至二歲起脊量強弱漸以飛字印印右膞細馬次馬俱以龍形印印項左送尚乘者于尾側依左右閑印以三花其餘雜馬上乘者以風字印印左膞以飛字印印右髀經印之後簡入别所者各以新入處監名印印左頰官馬賜人者以賜字印配諸軍及充傳送驛者以出字印並印右頰諸蕃馬印隨部落各為印識回紇馬印□□□□此所謂印馬者回紇以馬來與中國為互市中國以印印之也】無得携中國人及商胡出塞五者皆能如約則主上必許和親如此威加北荒旁讋吐蕃【讋之涉翻】足以快陛下平昔之心矣上曰自至德以來與爲兄弟之國今一旦欲臣之彼安肯和乎對曰彼思與中國和親久矣其可汗國相素信臣言若其未諧但應再發一書耳上從之既而回紇可汗遣使上表稱兒及臣凡泌所與約五事一皆聽命上大喜謂泌曰回紇何畏服卿如此對曰此乃陛下威靈臣何力焉上曰回紇則既和矣所以招雲南大食天竺奈何對曰回紇和則吐蕃已不敢輕犯塞矣次招雲南則是斷吐蕃之右臂也【斷音短】雲南自漢以來臣屬中國【雲南本漢之哀牢夷後漢永平之問始臣屬中國其地在漢永昌郡界】楊國忠無故擾之使叛臣于吐蕃【事見二百一十六卷玄宗天寶九載】苦于吐蕃賦役重未嘗一日不思復為唐臣也大食在西域為最強自葱嶺盡西海地幾半天下【大食既并波斯突騎施又亡其地東盡葱嶺西南際海方萬餘里幾居衣翻】與天竺皆慕中國代與吐蕃為仇臣故知其可招也癸亥遣回紇使者合闕將軍歸許以咸安公主妻可汗【蓬州咸安郡公主上女也妻七細翻 考異曰鄴侯家傳九月泌請與回紇和親十月與回紇書十二月回紇遣聿支達干上表謝恩皆請如宰相約和親按實錄八月丁酉回紇遣默啜達干來貢方物且請和親九月癸亥遣回紇使合闕將軍歸其國初合闕將其君命請昏上許以咸安公主嫁之命見于麟德殿且令齎公主畫圖就示可汗以馬價絹五萬還之許互市而去十二月無聿支入聘之事回紇自大歷十一年以來未嘗入寇信使往來亦無不和及求和之迹盖德宗心恨回紇而外迹猶羈縻不絶今回紇請昏則拒絶不許而李泌勸與為昏耳其月數之差則恐李繁記之不詳或者聿支即默啜與合闕皆不可知也若以默啜即為請昏之使合闕即為謝恩之人又泌論回紇凡十五餘對須半月以上泌又云臣木夾中與書令朝臣遞云一月可到歲内報至自丁酉至癸亥纔二十六日耳今依實錄月日因許嫁咸安本其事而言之】歸其馬價絹五萬疋 吐蕃寇華亭及連雲堡皆陷之【連雲堡在涇州西界宋祁曰連雲堡涇要地也三垂峭絶北據高所虜進退烽火易通 考異曰鄴侯家傳曰時京西諸鎮報種麥已畢絶萬頃而皆亘野上大喜既而尚結贊來入寇諸軍閉壁候夜斫營悉捷結贊乃退歸上以十餘年來邉軍嘗被戎挫皆入踐京畿此來始敗又不能更深入且報種麥已畢而喜甚按實錄吐蕃䧟華亭及連雲堡驅掠邠涇編戶牛畜萬計悉送至彈箏峽是秋數州人無種麥者與家傳相反今從實錄】甲戍吐蕃驅二城之民數千人及邠涇人畜萬計而去寘之彈筝峽西涇州恃連雲為斥候連雲既陷西門不開門外皆為虜境樵采路絶每收穫必陳兵以扞之多失時得空穗而已【禾麥熟而不收穫其實隕落故得空穗】由是涇州常苦乏食 冬十月甲申吐蕃寇豐義城【武德二年分彭原置豐義縣屬寧州宋白曰彭陽縣後魏于縣置雲州周武保定二年廢州為防隋文帝廢防為豐義城唐武德初分彭原縣為豐義縣屬彭州貞觀廢彭州以縣屬寜州其城即後魏雲州城】前鋒至大回原邠寜節度使韓遊瓌擊却之乙酉復寇長武城【復扶又翻】又城故原州而屯之妖僧李軟奴自言本皇族見嶽瀆神【嶽謂五嶽瀆謂四瀆妖于遥翻】命己為天子結殿前射生將韓欽緒等謀作亂丙戍其黨告之上命捕送内侍省推之【推鞫也】李晟聞之遽仆于地曰晟族滅矣李泌問其故晟曰晟新罹謗毁【事見上卷本年三月】中外家人千餘若有一人在其黨中則兄亦不能救矣泌乃密奏大獄一起所連引必多外間人情恟懼請出付臺推【付御史臺推鞫之也】上從之欽緒遊瓌之子也亡抵邠州遊瓌出屯長武城留後械送京師壬辰腰斬軟奴等八人北軍之士坐死者八百餘人而朝廷之臣無連及者韓遊瓌委軍詣闕謝上遣使止之委任如初遊瓌又械送欽緒二子上亦宥之 吐蕃以苦寒不入寇而糧運不繼十一月詔渾瑊歸河中 【考異曰鄴侯家傳十一月以張獻甫為邠寜等州節度使代韓遊瓌而以渾侍中為朔方河中絳邠寜慶副元帥先公乃令獻甫修西界堡障濠塹南接涇州于是塞内始有藩籬之固尚結贊不能輕入窺邉矣按獻甫明年七月乃為邠寜節度家傳誤也】李元諒歸華州劉昌分其衆歸汴州【劉昌本汴州將也貞元三年入朝詔以汴兵八千戊涇原尋授涇原帥華戶化翻】自餘防秋兵退屯鳳翔京兆諸縣以就食 十二月韓遊瓌入朝 自興元以來是歲最為豐稔米斗直錢百五十粟八十詔所在和糴庚辰上畋于新店入民趙光奇家問百姓樂乎對曰不樂上曰今歲頗稔何為不樂【樂音洛】對曰詔令不信前云兩税之外悉無他徭今非税而誅求者殆過于税後又云和糴而實強取之【強其良翻】曾不識一錢始云所糴粟麥納于道次今則遣致京西行營動數百里車摧馬斃破產不能支愁苦如此何樂之有每有詔書優恤徒空文耳恐聖主深居九重皆未知之也上命復其家【復方目翻復除也除其家賦役也】臣光曰甚矣唐德宗之難寤也自古所患者人君之澤壅而不下逹小民之情鬱而不上通故君勤恤于上而民不懷【勤恤者切於憂民也】民愁怨于下而君不知以至于離叛危亡凡以此也德宗幸以遊獵得至民家值光奇敢言而知民疾苦此乃千載之遇也【載子亥翻】固當按有司之廢格詔書【格音閣】殘虐下民橫增賦歛【横戶孟翻歛力贍翻】盗匿公財及左右謟諛日稱民間豐樂者而誅之【樂音洛】然後洗心易慮一新其政屏浮飾【屏必郢翻又卑正翻】廢虛文謹號令敦誠信察真偽辨忠邪矜困窮伸寃滯則太平之業可致矣釋此不爲乃復光奇之家夫以四海之廣兆民之衆又安得人人自言于天子而戶戶復其徭賦乎<br />
<br />
  李泌以李軟奴之黨猶有在北軍未者請大赦以安之【恐其自疑而動于惡】<br />
<br />
  四年春正月庚戍朔赦天下詔兩税等第自今三年一定 【考異曰實錄赦云天下兩税更審定等第仍加三年一定以為常式按陸贄論兩税狀云兩税之立惟以資產為宗不以身丁為本資產少者則其税少資產多者則其税多然則當時税賦但以貧富為等第若今時坊郭十等戶鄉村五等戶臨時科配也又云額内官勿更注擬見任者三考勒停此盖用李泌之策也按鄴侯家傳泌請罷天下額外官又云陛下許復所减官員臣因請停額外官許其得資後停額外官員當正官三分之一則今年計已停一半据此則似有額内官又有額外官皆在正官之外不則内皆應作外字之誤也】李泌奏京官俸太薄請自三師以下悉倍其俸【唐以太師】<br />
<br />
  【太傅太保爲三師倍俸倍大歷十二年所增之數也泌薄必翻俸扶用翻 考異曰實錄辛巳詔以中外給用除陌錢給文武官俸料自是京官益重頗優裕馬初除陌錢隸度支至是令戶部别庫貯之給俸之餘以備它用按興元元年正月赦其所加墊陌錢税間架之類悉宜停罷今猶有除陌錢者盖當時止罷所加之數或私買賣者官不收墊陌錢官給錢猶有除陌在故也】從之壬申以宣武行營節度使劉昌爲涇原節度使甲戍以鎮國節度使李元諒爲隴右節度使【涇原節度使治涇州隴右節度使治秦州劉昌以汴兵防秋為行營節度使李元諒本鎮華州領鎮軍國節度使】昌元諒皆帥卒力田【帥讀曰率下同】數年軍食充羨【羨弋線翻】涇隴稍安 韓遊瓌之入朝也軍中以爲必不返【以其子欽緒黨逆謂當連坐也瓌古回翻朝直遙翻去年十二月遊瓌入朝】餞送甚薄遊瓌見上盛陳築豐義城可以制吐蕃上悦遣還鎮【見賢遍翻吐從暾入聲還從宣翻又音如字】軍中憂懼者衆遊瓌忌都虞候虞鄉范希朝有功名得衆心【虞鄉縣屬河中府】求其罪將殺之希朝奔鳳翔上召之寘于左神策軍遊瓌帥衆築豐義城二版而潰【城二尺為一版上下相疑故漬】 二月元友直運淮南錢帛二十萬至長安【元友直句勘東南兩税錢帛見上卷去年七月】李泌悉輸之大盈庫然上猶數有宣索【泌薄必翻數所角翻索山客翻】仍敕諸道勿令宰相知泌聞之惆悵而不敢言【相息亮翻惆丑鳩翻】<br />
<br />
  臣光曰王者以天下為家天下之財皆其有也阜天下之財以養天下之民已必豫焉或乃更為私藏此匹夫之鄙志也古人有言貧不學儉夫多財者奢欲之所自來也【夫音扶】李泌欲弭德宗之欲而豐其私財財豐則欲滋矣財不稱欲能無求乎【弭眉比翻稱尺證翻】是猶啓其門而禁其出也雖德宗之多僻亦泌所以相之者非其道故也<br />
<br />
  咸陽人或上言臣見白起令臣奏云請為國家捍禦西陲正月吐蕃必大下當爲朝廷破之以取信【上時掌翻為于偽翻】既而吐蕃入寇邉將敗之【敗補邁翻】不能深入上以爲信然欲于京城立廟贈司徒李泌曰臣聞國將興聽于人【左傳虢史嚚之言】今將帥立功而陛下褒賞白起臣恐邉臣解體矣若立廟京城盛為祈禱流聞四方將長巫風【長知兩翻言巫祝之風將由此盛】今杜郵有舊祠【白起死于杜郵故有舊祠在焉】請敇府縣葺之則不至驚人耳目矣且白起列國之將贈三公太重【唐以大尉司徒司空為三公】請贈兵部尚書可矣上笑曰卿于白起亦惜官乎對曰人神一也陛下儻不之惜則神亦不以為榮矣上從之泌自陳衰老獨任宰相精力耗竭既未聽其去乞更除一相上曰朕深知卿勞苦但未得其人耳上從容與泌論即位以來宰相【從千容翻】曰盧忠清彊介人言姦邪朕殊不覺其然泌曰人言杞姦邪而陛下獨不覺其姦邪此乃杞之所以爲姦邪也 【考異曰舊李勉傳勉對德宗已有此語與鄴侯家傳述泌語畧同未知孰是今兩存之一本泌語之下有與勉】儻陛下覺之豈有建中之亂乎杞以私隙殺楊炎【殺楊炎事見二百二十七卷建中二年】擠顔眞卿于死地【事見二百二十八卷建中二年擠七細翻又牋西翻】激李懷光使叛【事見二百二十九卷建中四年】賴陛下聖明竄逐之人心頓喜天亦悔禍不然亂何由弭上曰楊炎以童子視朕每論事朕可其奏則悦與之往復論難【難乃旦翻下難之問難同】即怒而辭位觀其意以朕爲不足與言故也以是交不可忍【交不可忍者言炎既形之辭而帝亦心懷不平】非由也建中之亂術士豫請城奉天【事見二百二十六卷建中元年】此盖天命非所能致也泌曰天命他人皆可以言之惟君相不可言盖君相所以造命也若言命則禮樂刑政皆無所用矣紂曰我生不有命在天【見書西伯戡黎篇】此商之所以亡也上曰朕好與人較量理體【好呼到翻量音良理體猶言治體也】崔祐甫性褊躁【躁則到翻】朕難之則應對失次朕常知其短而護之楊炎論事亦有可采而氣色粗傲難之輒勃然怒無復君臣之禮所以每見令人忿發餘人則不敢復言【難乃旦翻下同復扶又翻】盧小心朕所言無不從又無學不能與朕往復故朕所懷常不盡也對曰言無不從豈忠臣乎夫言而莫予違此孔子所謂一言而喪邦者也【見論語喪息浪翻】上曰惟卿則異彼三人者朕言當卿有喜色不當常有憂色【當丁浪翻】雖時有逆耳之言如曏來紂及喪邦之類朕細思之皆卿先事而言【先悉薦翻】如此則理安【理安猶言治安也】如彼則危亂言雖深切而氣色和順無楊炎之陵傲朕問難往復卿辭理不屈又無好勝之志直使朕中懷已盡屈服而不能不從此朕所以私喜于得卿也泌曰陛下所用相尚多今皆不論何也上曰彼皆非所謂相也凡相者必委以政事如玄宗時牛仙客陳希烈可以謂之相乎如肅宗代宗之任卿雖不受其名乃真相耳必以官至平章事為相則王武俊之徒皆相也【唐之使相時主未嘗不知名器之濫也】 劉昌復築連雲堡【去年九月吐蕃䧟連雲堡復扶又翻】夏四月乙未更命殿前左右射生曰神威軍【更工衡翻 考異曰實錄作神武軍今從新志】與左右羽林龍武神武神策號曰十軍神策尤盛多戍京西散屯畿甸福建觀察使吳詵【武德四年分泉州之建安縣置建州】輕其軍士脆弱苦役之軍士作亂殺詵腹心十餘人逼詵牒大將郝誡溢掌留務誡溢上表請罪上遣中使就赦以安之 乙未隴右節度使李元諒築良原故城而鎮之【良原縣隋大業初置唐屬涇州貞元二年為吐蕃所破今乃修復九域志良原在涇州西南六十里宋白曰隋分安定鶉觚置良原縣西南三十里有良原因名】 雲南王異牟尋欲内附未敢自遣使先遣其東蠻鬼主驃旁苴夢衝苴烏星入見【苴子魚翻見賢遍翻】五月乙卯宴之于麟德殿賜賚甚厚封王給印而遣之【封驃旁為和義王苴夢衝為懷化王苴烏星為順政王】 辛未以太子賓客吳湊為福建觀察使貶吳詵為涪州刺史【涪音浮】 吐蕃三萬餘騎寇涇邠寧慶鄜等州先是吐蕃常以秋冬入寇及春多病疫而退至是得唐人質其妻子【先悉薦質音致】 遣其將將之盛夏入寇諸州皆城守無敢與戰者吐蕃俘掠人畜萬計而去 夏縣人陽城以學行著聞隱居柳谷之北【夏戶雅翻柳谷在安邑縣中條山行下孟翻】李泌薦之六月徵拜諫議大夫 韓遊瓌以吐蕃犯塞自戍寧州病求代歸秋七月庚戍加渾瑊邠寜副元帥以左金吾將軍張獻甫為邠寧節度使陳許兵馬使韓全義為長武城行營節度使獻甫未至壬子夜遊瓌不告于衆輕騎歸朝戍卒裴滿等憚獻甫之嚴乘無帥之際癸丑帥其徒作亂【騎奇寄翻朝直遥翻無帥所類翻帥其讀曰率】曰張公不出本軍我必拒之【謂張獻甫本不出于朔方軍也】因剽掠城市【剽匹妙翻】圍監軍楊明義所居使奏請范希朝為節度使都虞候楊朝晟避亂出城聞之復入曰所請甚契我心我來賀也亂卒稍安朝晟潜與諸將謀晨勒兵召亂卒謂曰所請不行張公已至邠州汝輩作亂當死不可盡殺宜自推列唱帥者遂斬二百餘人帥衆迎獻甫【帥讀曰率】上聞軍衆欲得范希朝將授之希朝辭曰臣畏遊瓌之禍而來今往代之非所以防窺覦安反仄也上嘉之擢為寧州刺史以副獻甫遊瓌至京師除右龍武統軍 振武節度使唐朝臣不嚴斥候己未奚室韋寇振武【李延壽曰室韋盖契丹之在南者為契丹在北者為室韋宋祁曰室韋契丹别種東胡北邉蓋于零苗裔也地據黄龍北傍峱越河直長安東北七千里東黑水靺鞨西突厥南契丹北瀕海】執宣慰中使二人大掠人畜而去時回紇之衆逆公主者在振武朝臣遣七百騎與回紇數百騎追之回紇使者為奚室韋所殺 九月庚申吐蕃尚志董星寇寧州張獻甫擊却之吐蕃轉掠鄜坊而去 元友直句檢諸道税外物【事始見上卷上年句古侯翻】悉輸戶部遂為定制歲于税外輸百餘萬緡斛民不堪命諸道多自訴于上上意寤詔今年已入在官者輸京師未入者悉以與民明年以後悉免之于是東南之民復安其業 回紇合骨咄祿可汗得唐許昏甚喜遣其妹骨咄祿毗伽公主及大臣妻并國相跌都督【奚結翻跌徒結翻跌與回紇同出鐵勒而異種】以下千餘人來迎可敦辭禮甚恭曰昔為兄弟今為子壻半子也若吐蕃為患子當為父除之【當為于偽翻】因詈辱吐蕃使者以絶之冬十月戊子回紇至長安可汗仍表請改回紇為回鶻許之 【考異曰舊回紇傳元和四年里伽可汗遣使請改為回鶻義取回旋輕捷如鶻崔鉉續會要貞元五年七月公主至衙帳回紇使李義進請改紇字為鶻與統紀同鄴侯家傳四年七月可汗上表請改紇字為鶻與李繁北荒君長錄及新回鶻傳同按李泌明年春薨若明年七月方改家傳不應言之今從家傳君長錄新書】 吐蕃兵十萬將寇西川亦發雲南兵雲南内雖附唐外未敢叛吐蕃亦兵數萬屯于瀘北【瀘北瀘水之北瀘水即諸葛亮五月所度者】韋臯知雲南計方猶豫乃為書遺雲南王叙其叛吐蕃歸化之誠貯以銀函【遺唯季翻貯丁呂翻】使東蠻轉致吐蕃吐蕃始疑雲南遣兵二萬屯會川【會川本卭都縣高宗上元二年徒縣于會川因更名新志會川縣屬嶲州有瀘津闕在會川東南三十里】以塞雲南趣蜀之路【塞悉則翻趣逡諭翻又逡須翻】雲南怒引兵歸國由是雲南與吐蕃大相猜阻歸唐之志益堅吐蕃失雲南之助兵勢始弱矣然吐蕃業已入寇遂分兵四萬攻兩林驃旁三萬攻東蠻七千寇清溪關【清溪關在巂州界自關而南七百二十里至嶲州洪源志清溪關在黎州西南界】五千寇銅山【新志黎州有銅山要衝十一城】臯遣黎州刺史韋晉等與東蠻連兵禦之破吐蕃于清溪關外庚子册命咸安公主加回鶻可汗長夀天親可汗十<br />
<br />
  一月以刑部尚書關播為送咸安公主兼册回鶻可汗使【自此以後通鑑皆依前史書回鶻】 吐蕃恥前日之敗【謂上清溪關外之敗也】復以衆二萬寇清溪關一萬攻東蠻【復扶又翻】韋臯命韋晉鎮要衝城督諸軍以禦之嶲州經畧使劉朝彩出關連戰自乙卯至癸亥大破之 李泌言于上曰江淮漕運以甬橋為咽喉【咽音】地屬徐州鄰于李納【徐州與李納廵屬鄰境】刺史高明應年少不習事【高明應嗣鎮徐州始二百三十一卷興元元年少詩照翻】若李納一旦復有異圖【復扶又翻下同】竊據徐州是失江淮也國用何從而致請徙夀廬濠都團練使張建封鎮徐州割濠泗以隸之復以廬夀歸淮南則淄青惕息而運路常通江淮安矣及今明應幼騃可代【騃五駭翻】宜徵為金吾將軍萬一使他人得之則不可復制矣上從之以建封爲徐泗濠節度使建封為政寛厚而有綱紀不貸人以法【犯法者有誅無貸】故其下無不畏而悦之 横海節度使程【日】華薨子懷直自知留後 吐蕃屢遣人誘脅雲南【誘音】<br />
<br />
  【酉】<br />
<br />
  五年春二月丁亥韋臯遺異牟尋書稱回鶻屢請佐天子共滅吐蕃王不早定計一旦爲回鶻所先【遺唯季翻先悉薦翻】則王累代功名虛棄矣且雲南久為吐蕃屈辱今不乘此時依大國之勢以復怨雪恥後悔無及矣 戊戍以横海留後程懷直爲滄州觀察使懷直請分弓高景城爲景州【景城縣本屬滄州武德四年屬瀛州貞觀元年屬滄州大歷七年屬瀛州横海盖因朱滔之敗復得而有之後尋屬瀛州弓高漢古縣魏晉廢省隋置弓高縣于漢鬲縣地唐屬滄州】仍請朝廷除刺史上喜曰三十年無此事矣乃以員外郎徐伸為景州刺史 中書侍郎同平章事李泌屢乞更命相上<br />
<br />
  欲用戶部侍郎班宏泌言宏雖清彊而性                    【多】凝滯乃薦竇參通敏可兼度支鹽鐵董晉方正可處門下【處昌呂翻】上皆以為不可參誕之玄孫也【竇誕武德中勸齊王元吉弃并州者也】時為御史中丞兼戶部侍郎晉為太常卿至是泌疾甚復薦二人【復扶又翻】庚子以董晉為門下侍郎竇參為中書侍郎兼度支轉運使並同平章事以班宏為尚書依前度支轉運副使參為人剛果峭刻【尚辰羊翻度徒洛翻使疏吏翻峭七笑翻】無學術多權數每奏事諸相出【相息亮翻】參獨居後以奏度支事爲辭實專大政多引親黨置要地使為耳目董晉充位而已然晉為人重愼所言于上前者未嘗泄于人子弟或問之晉曰欲知宰相能否視天下安危所謀議于上前者不足道也 【考異曰韓愈作晉行狀曰在宰相位凡五年所奏于上前者皆二帝三王之道由秦漢以降未嘗言退歸未嘗言所言于上者于人子弟有私問者公曰宰相所職繫天下安危宰相之能與否可見欲知宰相之能與否如此視之其可凡所謀議于上前者不足道也故其事卒不聞愈作行狀必揚美盖惡叙其為相時事止于此則其循默充位可知然其重慎亦可稱也今略取行狀】三月甲辰李泌薨泌有謀畧而好談神仙詭誕【泌薄必翻薨呼肱翻好呼到翻】故為世所輕 【考異曰國史補曰李泌相以虛誕自任常對客教家人速灑掃今夜產先生來宿有人遺美酒一榼會冇客至乃曰麻姑送酒與君同傾傾未畢門者曰某侍郎來取榼泌令倒還略無愧色舊泌傳曰德宗初即位尤惡巫祝怪譚之士及建中末寇戎内梗桑道茂有城奉天之說上稍以時日禁忌為意而雅聞泌長于鬼道故自外徵還以至大用時論不以為惬及在相位隨時俯仰無足可稱復引顧况輩輕薄之流動爲朝士戲侮頗貽譏誚泌放曠敏辨好大言自出入中禁累為權倖忌嫉恒由智免終以言論縱横上悟聖主以躋相位初泌流放江南與柳渾顧况為人外之交吟詠自適而渾先逹故泌復得入官于朝况蘇州人按泌雖詭誕好談神仙然其智畧實有過人者至于佐肅代復兩京不受相位而去代宗順宗之在東宫皆賴泌得安此其大節可重者也舊傳毀之太過家傳出于其子雖難盡信亦豈得盡不信今擇其可信者存之】初上思李懷光之功欲宥其一子【事見二百三十二卷貞元元年】而<br />
<br />
  子孫皆已伏誅戊辰詔以懷光外孫燕八八爲懷光後【燕於䖍翻姓也】賜姓名李承緒除左衛率胄曹參軍賜錢千緡使養懷光妻王氏【率所律翻養羊尚翻】及守其墓祀 冬十月韋臯遣其將曹有道將兵與東蠻兩林蠻及吐蕃青海臘城二節度戰于嶲州臺登谷【臺登漢縣唐屬嶲州】大破之斬首二千級投崖及溺死者不可勝數殺其大兵馬使乞藏遮遮乞藏遮遮虜之驍將也既死臯所攻城栅無不下數年盡復嶲州之境 易定節度使張孝忠興兵襲蔚州【蔚紆勿翻】驅掠人畜詔書責之踰旬還鎮 瓊州自乾封中為山賊所陷【瓊州在海中大洲上中有黎母山黎人居之不輸王賦所謂山賊盖黎人也宋白曰瓊州北十五里極大海泛大船使西南風帆三日三夜到地名崖山門入江一日至新會縣或便風十日到廣州】至是嶺南節度使李復遣判官姜孟京與崖州刺史張少遷攻拔之 十二月庚午聞回鶻天親可汗薨戊寅遣鴻臚卿郭鋒册命其子為登里羅没密施俱錄忠貞毗伽可汗先是安西北庭皆假道於回鶻以奏事【為吐蕃所隔河隴之路不可通也故假道于回鶻以入奏先悉薦翻】故與之連和北庭去回鶻尤近誅求無厭【厭於鹽翻】又有沙陀六千餘帳與北庭相依【沙陀西突厥别部處月種也居金娑山之陽蒲類海之東有大磧名沙陀故自號沙陀】及三葛祿白服突厥皆附于回鶻【三葛祿葛邏祿三部也一曰謀刺二曰婆匐三曰踏實力在北庭西北金山之西白服突厥新唐書作白眼突厥】回鶻數侵掠之【數所角翻】吐蕃因葛祿白服之衆以攻北庭回鶻大相頡于迦斯將兵救之 雲南雖貳于吐蕃亦未敢顯與之絶壬辰韋臯復以書招諭之【復扶又翻】<br />
<br />
  六年春詔出岐山無憂王寺佛指骨迎置禁中又送諸寺以示衆傾都瞻禮施財巨萬【施式豉翻】二月乙亥遣中使復葬故處 初朱滔敗于貝州【見二百三十一卷興元元年】其棣州刺史趙鎬以州降于王武俊既而得罪于武俊召之不至田緒殘忍其兄朝仕李納為齊州刺史或言納欲納朝於魏緒懼判官孫光佐等為緒謀厚賂納且說納招趙鎬取棣州以悦之【爲于僞翻說式芮翻】因請送朝於京師納從之丁酉鎬以棣州降于納三月武俊使其子士真擊之不克 回鶻忠貞可汗之弟弑忠貞而自立 【考異曰新傳曰可汗為少可敦業公主所毒死可汗之弟乃自立今從實錄】其大相頡干迦斯西擊吐蕃未還夏四月次相帥國人殺簒者而立忠貞之子阿啜爲可汗年十五【相息亮翻帥讀曰率下同】 五月王武俊屯冀州將擊趙鎬鎬帥其屬奔鄆州【鄆音運】李納分兵據之田緒使孫光佐如鄆州矯詔以棣州隸納武俊怒遣其子士清伐貝州取經城等四縣 回鶻頡干迦斯與吐蕃戰不利吐蕃急攻北庭北庭人苦於回鶻誅求與沙陀酋長朱邪盡忠皆降于吐蕃【為後沙陀來降張本】節度使楊襲古帥麾下二千人奔西州六月頡干迦斯引兵還國次相恐其有廢立與可汗皆出郊迎俯伏自陳擅立之狀曰今日惟大相死生之盛陳郭鋒所齎國信悉以遺之【去年唐遣郭鋒冊忠貞可汗遺唯季翻】可汗拜且泣曰兒愚幼若幸而得立惟仰食于阿爹國政不敢豫也虜謂父為阿爹【仰牛向翻唐韵北人呼父曰阿爹爹徒可翻】頡干迎斯感其卑屈持之而哭遂執臣禮悉以所遺頒從行者已無所受國中由是稍安秋頡干迦斯悉舉國兵數萬將復北庭又為吐蕃所敗【敗補邁翻史言回鶻衰亂】死者大半襲古收餘衆數百將還西州頡干迦斯紿之曰且與我同至牙帳既而留不遣竟殺之安西由是遂絶莫知存亡【北庭既陷于吐蕃安西路絶故莫知其音問】而西州猶爲唐固守【爲于僞翻】葛祿乘勝取回鶻之浮圖川【浮圖川在烏德犍山西北】回鶻震恐悉遷西北部落于牙帳之南以避之遣逹北特勒梅錄隨郭鋒偕來告忠貞可汗之喪且求册命先是回鶻使者入中國禮容驕慢【先悉薦翻】刺史皆與之鈞禮梅錄至豐州刺史李景略欲以氣加之謂梅錄曰聞可汗新没欲申弔禮景略先據高壟而坐【壟即隴字】梅錄俯僂前哭【俯低頭也僂曲背也僂力主翻】景略撫之曰可汗棄代助爾哀慕梅錄驕容猛氣索然俱盡【索蘇各翻】自是回鶻使至皆拜景略于庭威名聞塞外【聞音問】冬十月辛亥郭鋒始自回鶻還十一月庚午上祀圓丘 上屢詔李納以棣州歸王<br />
<br />
  武俊納百方遷延請以海州易之於朝廷上不許乃請詔武俊先歸田緒四縣上從之十二月納始以棣州歸武俊<br />
<br />
  七年春正月己巳襄王僙薨【僙肅宗子音戶光翻】 二月癸卯遣鴻臚少卿庾鋋冊回鶻奉誠可汗【鋋音蟬 考異曰實錄作康鋋今從新舊傳】 戊戍詔涇原節度使劉昌築平凉故城【舊書曰城去原州一百五十里】以扼彈筝峽口浹辰而畢【浹與周禮挾日而歛之挾同鄭注云從甲至甲謂之挾此言浹辰從子至子也史炤曰自子至亥曰辰浹辰十二日】分兵戍之昌又築朝谷堡【舊唐書作胡谷堡東距平凉三十五里】甲子詔名其堡曰彰信【舊書作彰義】涇原稍安 初上還長安以神策等軍有衛從之勞【從才用翻下同】皆賜名興元元從奉天定難功臣【難乃旦翻宋白曰唐玄宗平内難賜衛士葛福順等爲唐元功臣不過十數人德宗駐蹕奉天及幸山南賜從駕立功將校爲元從奉天定難功臣谷口以來元從將士賜名元從功臣及僖昭頻年播遷功臣差多至後梁後唐徧及戍卒非賞興也】以官領之撫恤優厚禁軍恃恩驕横【横戶孟翻】侵暴百姓陵忽府縣至詬辱官吏【府謂京兆府縣謂赤縣畿縣詬呼漏翻詈也】毁裂案牘府縣官有不勝忿而刑之者【勝音升】朝笞一人夕貶萬里由是府縣雖有公嚴之官莫得舉其職市井富民往往行賂寄名軍籍則府縣不能制辛巳詔神威六軍吏士與百姓訟者委之府縣小事牒本軍奏聞若軍士陵忽府縣禁身以聞【北軍十軍左右羽林龍武神武神威神策也神策尤盛建中之前未分左右軍謂之神策六軍者指言神策軍與左右羽林龍武神武六軍也貞元二年以神策左右廂爲左右神策軍又以殿前射生左右廂爲左右射生軍四年以左右射生軍爲左右神威軍北軍遂為十軍此時神策軍既居北軍之右史家書此事又專言神策恃恩陵暴而有是詔則所謂神威六軍者亦當為神策六軍威字誤也此神策六軍提起左右神策軍以及左右龍武神武神威六軍也不及左右羽林軍者羽林置于唐初龍武等軍皆開元以來節次增置于禁衛又親近于羽林也禁身者囚禁其身】委御史臺推覆縣吏輒敢笞辱必從貶謫 癸未易定節度使張孝忠薨 安南都護高正平重賦斂【安南都護府本交州調露二年置為安南都護府斂力贍翻】夏四月羣蠻酋長杜英翰等起兵圍都護府正平以憂死羣蠻聞之皆降【史言蠻非好亂苦于貪帥而亂酋慈由翻長知兩翻】五月辛巳置柔遠軍于安南 端王遇薨【遇皇弟也】 韋臯比年致書招雲南王異牟尋【比毗至翻韋臯書招靈南事始上卷三年】終未獲報然吐蕃每發雲南兵雲南與之益少【少詩沼翻】臯知異牟尋心附于唐討擊副使段忠義本閤羅鳳使者也【閤羅鳳者異牟尋之祖】六月丙申韋臯遣忠義還雲南并致書敦諭之【敦迫也厚也】秋七月戊寅以定州刺史張昇雲為義武留後 庚<br />
<br />
  辰以䖍州刺史趙昌為安南都護羣蠻遂安 八月丙午以翰林學士陸贄爲兵部侍郎餘職皆解竇參惡之也【惡烏路翻下同】 吐蕃攻靈州爲回鶻所敗夜遁【敗袖邁翻】九月回鶻遣使來獻俘冬十二月甲午又遣使獻所獲吐蕃酋長尚結心【酋慈由翻長知兩翻】 福建觀察使吳湊為治有聲【福建皆古閩越地秦為關中郡漢為治縣後漢為候官縣吳置建安郡陳置閩州隋改衆州唐移泉州于晉江縣而閩州治閩縣及候官縣而于建安縣立建州建安吳孫策所置縣也以年號為名本亦東官之地開元十三年又改閩州為福州自此福建泉三州始不相紊治直吏翻】竇參以私憾毁之且言其病風上召至京師使之步以察之知參之誣由是始惡參【為竇參貶逐張本】丁酉以湊爲陜虢觀察使以代參黨李翼友睦王述薨【述亦皇弟】 吐蕃知韋臯使者在雲南遣使讓之雲南王異牟尋紿之曰唐使本蠻也臯聽其歸耳無它謀也因執以送吐蕃吐蕃多取其大臣之子為質【質音致】雲南愈怨勿鄧酋長苴夢衝濳通吐蕃扇誘羣蠻隔絶雲南使者【酋慈由翻長知兩翻】韋臯遣三部落揔管蘇峞將兵至琵琶川【峞牛罪翻又音嵬三部落兩林勿鄧豐琶也琵琶川在嶲州西南徼外爲下卷明年誅夢衝張本】<br />
<br />
  資治通鑑卷二百三十三  <br>
   </div> 

<script src="/search/ajaxskft.js"> </script>
 <div class="clear"></div>
<br>
<br>
 <!-- a.d-->

 <!--
<div class="info_share">
</div> 
-->
 <!--info_share--></div>   <!-- end info_content-->
  </div> <!-- end l-->

<div class="r">   <!--r-->



<div class="sidebar"  style="margin-bottom:2px;">

 
<div class="sidebar_title">工具类大全</div>
<div class="sidebar_info">
<strong><a href="http://www.guoxuedashi.com/lsditu/" target="_blank">历史地图</a></strong>  
<a href="http://www.880114.com/" target="_blank">英语宝典</a>  
<a href="http://www.guoxuedashi.com/13jing/" target="_blank">十三经检索</a> 
<br><strong><a href="http://www.guoxuedashi.com/gjtsjc/" target="_blank">古今图书集成</a></strong> 
<a href="http://www.guoxuedashi.com/duilian/" target="_blank">对联大全</a> <strong><a href="http://www.guoxuedashi.com/xiangxingzi/" target="_blank">象形文字典</a></strong> 

<br><a href="http://www.guoxuedashi.com/zixing/yanbian/">字形演变</a>  <strong><a href="http://www.guoxuemi.com/hafo/" target="_blank">哈佛燕京中文善本特藏</a></strong>
<br><strong><a href="http://www.guoxuedashi.com/csfz/" target="_blank">丛书&方志检索器</a></strong> <a href="http://www.guoxuedashi.com/yqjyy/" target="_blank">一切经音义</a>  

<br><strong><a href="http://www.guoxuedashi.com/jiapu/" target="_blank">家谱族谱查询</a></strong>  <strong><a href="http://shufa.guoxuedashi.com/sfzitie/" target="_blank">书法字帖欣赏</a></strong> 
<br>

</div>
</div>


<div class="sidebar" style="margin-bottom:0px;">

<font style="font-size:22px;line-height:32px">QQ交流群9:489193090</font>


<div class="sidebar_title">手机APP 扫描或点击</div>
<div class="sidebar_info">
<table>
<tr>
	<td width=160><a href="http://m.guoxuedashi.com/app/" target="_blank"><img src="/img/gxds-sj.png" width="140"  border="0" alt="国学大师手机版"></a></td>
	<td>
<a href="http://www.guoxuedashi.com/download/" target="_blank">app软件下载专区</a><br>
<a href="http://www.guoxuedashi.com/download/gxds.php" target="_blank">《国学大师》下载</a><br>
<a href="http://www.guoxuedashi.com/download/kxzd.php" target="_blank">《汉字宝典》下载</a><br>
<a href="http://www.guoxuedashi.com/download/scqbd.php" target="_blank">《诗词曲宝典》下载</a><br>
<a href="http://www.guoxuedashi.com/SiKuQuanShu/skqs.php" target="_blank">《四库全书》下载</a><br>
</td>
</tr>
</table>

</div>
</div>


<div class="sidebar2">
<center>


</center>
</div>

<div class="sidebar"  style="margin-bottom:2px;">
<div class="sidebar_title">网站使用教程</div>
<div class="sidebar_info">
<a href="http://www.guoxuedashi.com/help/gjsearch.php" target="_blank">如何在国学大师网下载古籍?</a><br>
<a href="http://www.guoxuedashi.com/zidian/bujian/bjjc.php" target="_blank">如何使用部件查字法快速查字?</a><br>
<a href="http://www.guoxuedashi.com/search/sjc.php" target="_blank">如何在指定的书籍中全文检索?</a><br>
<a href="http://www.guoxuedashi.com/search/skjc.php" target="_blank">如何找到一句话在《四库全书》哪一页?</a><br>
</div>
</div>


<div class="sidebar">
<div class="sidebar_title">热门书籍</div>
<div class="sidebar_info">
<a href="/so.php?sokey=%E8%B5%84%E6%B2%BB%E9%80%9A%E9%89%B4&kt=1">资治通鉴</a> <a href="/24shi/"><strong>二十四史</strong></a>&nbsp; <a href="/a2694/">野史</a>&nbsp; <a href="/SiKuQuanShu/"><strong>四库全书</strong></a>&nbsp;<a href="http://www.guoxuedashi.com/SiKuQuanShu/fanti/">繁体</a>
<br><a href="/so.php?sokey=%E7%BA%A2%E6%A5%BC%E6%A2%A6&kt=1">红楼梦</a> <a href="/a/1858x/">三国演义</a> <a href="/a/1038k/">水浒传</a> <a href="/a/1046t/">西游记</a> <a href="/a/1914o/">封神演义</a>
<br>
<a href="http://www.guoxuedashi.com/so.php?sokeygx=%E4%B8%87%E6%9C%89%E6%96%87%E5%BA%93&submit=&kt=1">万有文库</a> <a href="/a/780t/">古文观止</a> <a href="/a/1024l/">文心雕龙</a> <a href="/a/1704n/">全唐诗</a> <a href="/a/1705h/">全宋词</a>
<br><a href="http://www.guoxuedashi.com/so.php?sokeygx=%E7%99%BE%E8%A1%B2%E6%9C%AC%E4%BA%8C%E5%8D%81%E5%9B%9B%E5%8F%B2&submit=&kt=1"><strong>百衲本二十四史</strong></a>  <a href="http://www.guoxuedashi.com/so.php?sokeygx=%E5%8F%A4%E4%BB%8A%E5%9B%BE%E4%B9%A6%E9%9B%86%E6%88%90&submit=&kt=1"><strong>古今图书集成</strong></a>
<br>

<a href="http://www.guoxuedashi.com/so.php?sokeygx=%E4%B8%9B%E4%B9%A6%E9%9B%86%E6%88%90&submit=&kt=1">丛书集成</a> 
<a href="http://www.guoxuedashi.com/so.php?sokeygx=%E5%9B%9B%E9%83%A8%E4%B8%9B%E5%88%8A&submit=&kt=1"><strong>四部丛刊</strong></a>  
<a href="http://www.guoxuedashi.com/so.php?sokeygx=%E8%AF%B4%E6%96%87%E8%A7%A3%E5%AD%97&submit=&kt=1">說文解字</a> <a href="http://www.guoxuedashi.com/so.php?sokeygx=%E5%85%A8%E4%B8%8A%E5%8F%A4&submit=&kt=1">三国六朝文</a>
<br><a href="http://www.guoxuedashi.com/so.php?sokeytm=%E6%97%A5%E6%9C%AC%E5%86%85%E9%98%81%E6%96%87%E5%BA%93&submit=&kt=1"><strong>日本内阁文库</strong></a> <a href="http://www.guoxuedashi.com/so.php?sokeytm=%E5%9B%BD%E5%9B%BE%E6%96%B9%E5%BF%97%E5%90%88%E9%9B%86&ka=100&submit=">国图方志合集</a> <a href="http://www.guoxuedashi.com/so.php?sokeytm=%E5%90%84%E5%9C%B0%E6%96%B9%E5%BF%97&submit=&kt=1"><strong>各地方志</strong></a>

</div>
</div>


<div class="sidebar2">
<center>

</center>
</div>
<div class="sidebar greenbar">
<div class="sidebar_title green">四库全书</div>
<div class="sidebar_info">

《四库全书》是中国古代最大的丛书,编撰于乾隆年间,由纪昀等360多位高官、学者编撰,3800多人抄写,费时十三年编成。丛书分经、史、子、集四部,故名四库。共有3500多种书,7.9万卷,3.6万册,约8亿字,基本上囊括了古代所有图书,故称“全书”。<a href="http://www.guoxuedashi.com/SiKuQuanShu/">详细>>
</a>

</div> 
</div>

</div>  <!--end r-->

</div>
<!-- 内容区END --> 

<!-- 页脚开始 -->
<div class="shh">

</div>

<div class="w1180" style="margin-top:8px;">
<center><script src="http://www.guoxuedashi.com/img/plus.php?id=3"></script></center>
</div>
<div class="w1180 foot">
<a href="/b/thanks.php">特别致谢</a> | <a href="javascript:window.external.AddFavorite(document.location.href,document.title);">收藏本站</a> | <a href="#">欢迎投稿</a> | <a href="http://www.guoxuedashi.com/forum/">意见建议</a> | <a href="http://www.guoxuemi.com/">国学迷</a> | <a href="http://www.shuowen.net/">说文网</a><script language="javascript" type="text/javascript" src="https://js.users.51.la/17753172.js"></script><br />
  Copyright &copy; 国学大师 古典图书集成 All Rights Reserved.<br>
  
  <span style="font-size:14px">免责声明:本站非营利性站点,以方便网友为主,仅供学习研究。<br>内容由热心网友提供和网上收集,不保留版权。若侵犯了您的权益,来信即刪。scp168@qq.com</span>
  <br />
ICP证:<a href="http://www.beian.miit.gov.cn/" target="_blank">鲁ICP备19060063号</a></div>
<!-- 页脚END --> 
<script src="http://www.guoxuedashi.com/img/plus.php?id=22"></script>
<script src="http://www.guoxuedashi.com/img/tongji.js"></script>

</body>
</html>
