<!DOCTYPE html PUBLIC "-//W3C//DTD XHTML 1.0 Transitional//EN" "http://www.w3.org/TR/xhtml1/DTD/xhtml1-transitional.dtd">
<html xmlns="http://www.w3.org/1999/xhtml">
<head>
<meta http-equiv="Content-Type" content="text/html; charset=utf-8" />
<meta http-equiv="X-UA-Compatible" content="IE=Edge,chrome=1">
<title>資治通鑒_199-資治通鑑卷一百九十八_199-資治通鑑卷一百九十八</title>
<meta name="Keywords" content="資治通鑒_199-資治通鑑卷一百九十八_199-資治通鑑卷一百九十八">
<meta name="Description" content="資治通鑒_199-資治通鑑卷一百九十八_199-資治通鑑卷一百九十八">
<meta http-equiv="Cache-Control" content="no-transform" />
<meta http-equiv="Cache-Control" content="no-siteapp" />
<link href="/img/style.css" rel="stylesheet" type="text/css" />
<script src="/img/m.js?2020"></script> 
</head>
<body>
 <div class="ClassNavi">
<a  href="/24shi/">二十四史</a> | <a href="/SiKuQuanShu/">四库全书</a> | <a href="http://www.guoxuedashi.com/gjtsjc/"><font  color="#FF0000">古今图书集成</font></a> | <a href="/renwu/">历史人物</a> | <a href="/ShuoWenJieZi/"><font  color="#FF0000">说文解字</a></font> | <a href="/chengyu/">成语词典</a> | <a  target="_blank"  href="http://www.guoxuedashi.com/jgwhj/"><font  color="#FF0000">甲骨文合集</font></a> | <a href="/yzjwjc/"><font  color="#FF0000">殷周金文集成</font></a> | <a href="/xiangxingzi/"><font color="#0000FF">象形字典</font></a> | <a href="/13jing/"><font  color="#FF0000">十三经索引</font></a> | <a href="/zixing/"><font  color="#FF0000">字体转换器</font></a> | <a href="/zidian/xz/"><font color="#0000FF">篆书识别</font></a> | <a href="/jinfanyi/">近义反义词</a> | <a href="/duilian/">对联大全</a> | <a href="/jiapu/"><font  color="#0000FF">家谱族谱查询</font></a> | <a href="http://www.guoxuemi.com/hafo/" target="_blank" ><font color="#FF0000">哈佛古籍</font></a> 
</div>

 <!-- 头部导航开始 -->
<div class="w1180 head clearfix">
  <div class="head_logo l"><a title="国学大师官网" href="http://www.guoxuedashi.com" target="_blank"></a></div>
  <div class="head_sr l">
  <div id="head1">
  
  <a href="http://www.guoxuedashi.com/zidian/bujian/" target="_blank" ><img src="http://www.guoxuedashi.com/img/top1.gif" width="88" height="60" border="0" title="部件查字,支持20万汉字"></a>


<a href="http://www.guoxuedashi.com/help/yingpan.php" target="_blank"><img src="http://www.guoxuedashi.com/img/top230.gif" width="600" height="62" border="0" ></a>


  </div>
  <div id="head3"><a href="javascript:" onClick="javascript:window.external.AddFavorite(window.location.href,document.title);">添加收藏</a>
  <br><a href="/help/setie.php">搜索引擎</a>
  <br><a href="/help/zanzhu.php">赞助本站</a></div>
  <div id="head2">
 <a href="http://www.guoxuemi.com/" target="_blank"><img src="http://www.guoxuedashi.com/img/guoxuemi.gif" width="95" height="62" border="0" style="margin-left:2px;" title="国学迷"></a>
  

  </div>
</div>
  <div class="clear"></div>
  <div class="head_nav">
  <p><a href="/">首页</a> | <a href="/ShuKu/">国学书库</a> | <a href="/guji/">影印古籍</a> | <a href="/shici/">诗词宝典</a> | <a   href="/SiKuQuanShu/gxjx.php">精选</a> <b>|</b> <a href="/zidian/">汉语字典</a> | <a href="/hydcd/">汉语词典</a> | <a href="http://www.guoxuedashi.com/zidian/bujian/"><font  color="#CC0066">部件查字</font></a> | <a href="http://www.sfds.cn/"><font  color="#CC0066">书法大师</font></a> | <a href="/jgwhj/">甲骨文</a> <b>|</b> <a href="/b/4/"><font  color="#CC0066">解密</font></a> | <a href="/renwu/">历史人物</a> | <a href="/diangu/">历史典故</a> | <a href="/xingshi/">姓氏</a> | <a href="/minzu/">民族</a> <b>|</b> <a href="/mz/"><font  color="#CC0066">世界名著</font></a> | <a href="/download/">软件下载</a>
</p>
<p><a href="/b/"><font  color="#CC0066">历史</font></a> | <a href="http://skqs.guoxuedashi.com/" target="_blank">四库全书</a> |  <a href="http://www.guoxuedashi.com/search/" target="_blank"><font  color="#CC0066">全文检索</font></a> | <a href="http://www.guoxuedashi.com/shumu/">古籍书目</a> | <a   href="/24shi/">正史</a> <b>|</b> <a href="/chengyu/">成语词典</a> | <a href="/kangxi/" title="康熙字典">康熙字典</a> | <a href="/ShuoWenJieZi/">说文解字</a> | <a href="/zixing/yanbian/">字形演变</a> | <a href="/yzjwjc/">金 文</a> <b>|</b>  <a href="/shijian/nian-hao/">年号</a> | <a href="/diming/">历史地名</a> | <a href="/shijian/">历史事件</a> | <a href="/guanzhi/">官职</a> | <a href="/lishi/">知识</a> <b>|</b> <a href="/zhongyi/">中医中药</a> | <a href="http://www.guoxuedashi.com/forum/">留言反馈</a>
</p>
  </div>
</div>
<!-- 头部导航END --> 
<!-- 内容区开始 --> 
<div class="w1180 clearfix">
  <div class="info l">
   
<div class="clearfix" style="background:#f5faff;">
<script src='http://www.guoxuedashi.com/img/headersou.js'></script>

</div>
  <div class="info_tree"><a href="http://www.guoxuedashi.com">首页</a> > <a href="/SiKuQuanShu/fanti/">四库全书</a>
 > <h1>资治通鉴</h1> <!--         下载:【右键另存为】即可 --></div>
  <div class="info_content zj clearfix">
  
<div class="info_txt clearfix" id="show">
<center style="font-size:24px;">199-資治通鑑卷一百九十八</center>
    資治通鑑卷一百九十八 宋 司馬光 撰<br />
<br />
  胡三省 音註<br />
<br />
  唐紀十四【起旃蒙大荒落六月盡著雍涒灘三月凡二年有奇】<br />
<br />
  太宗文武大聖大廣孝皇帝下之上<br />
<br />
  貞觀十九年【觀古玩翻】六月丁酉李世勣攻白巖城西南上臨其西北城主孫代音濳遣腹心請降【降戶江翻下同】臨城投刀鉞為信且曰奴願降城中有不從者上以唐幟與其使【幟昌志翻使疏吏翻】曰必降者宜建之城上代音建幟城中人以為唐兵已登城皆從之上之克遼東也白巖城請降既而中悔上怒其反覆令軍中曰得城當悉以人物賞戰士【言以其男女及財物為賞也】李世勣見上將受其降帥甲士數十人請曰士卒所以爭冒矢石不顧其死者貪虜獲耳【帥讀曰率下同冒莫北翻下同】今城垂拔奈何更受其降孤戰士之心【觀世勣此言蓋少年為盗之氣習未除耳】上下馬謝曰將軍言是也然縱兵殺人而虜其妻孥【孥音奴】朕所不忍將軍麾下有功者朕以庫物賞之庶因將軍贖此一城世勣乃退得城中男女萬餘口上臨水設幄受其降仍賜之食八十以上賜帛有差他城之兵在白巖者悉慰諭給糧仗任其所之先是遼東城長史為部下所殺其省事奉妻子奔白巖【省事吏職也自後魏以來有之賀拔岳之攻尉遲菩薩也菩薩使省事傳語是也先悉薦翻省悉景翻】上憐其有義賜帛五匹為長史造靈輿歸之平壤【為于偽翻下自為彼為汝為當為同】以白巖城為巖州以孫代音為刺史契苾何力瘡重【契欺訖翻苾毘必翻】上自為傅藥推求得刺何力者高突勃付何力使自殺之何力奏稱彼為其主冒白刃刺臣乃忠勇之士也【刺七亦翻】與之初不相識非有怨讐遂捨之【怨於元翻】初莫離支遣加尸城七百人戍蓋牟城李世勣盡虜之其人請從軍自效上曰汝家皆在加尸汝為我戰莫離支必殺汝妻子得一人之力而滅一家吾不忍也戊戍皆廩賜遣之己亥以蓋牟城為蓋州丁未車駕發遼東丙辰至安市城【安市漢古縣屬遼東郡舊書薛仁貴傳作安地城】進兵攻之丁巳高麗北部耨薩延壽惠真帥高麗靺鞨兵十五萬救安市【後漢書東夷傳高句麗有五族有消奴部絶奴部順奴部灌奴部桂婁部賢曰案今高麗五部一曰内部一名黄部即桂婁部也二曰北部一名後部即絶奴部也三曰東部一名左部即順奴部也四曰南部一名前部即灌奴部也五曰西部一名右部即消奴部也據北史高麗五部各有耨薩蓋其酋長之稱也耨奴屋翻新書高麗大城置耨薩一比都督也麗力知翻靺鞨音末曷】上謂侍臣曰今為延壽策有三引兵直前連安市城為壘據高山之險食城中之粟縱靺鞨掠吾牛馬攻之不可猝下欲歸則泥潦為阻坐困吾軍上策也【若高延壽出於上策不知太宗何以應之唯有江夏王道宗之計策耳】拔城中之衆與之宵遁中策也不度智能來與吾戰下策也【度徒洛翻】卿曹觀之必出下策成擒在吾目中矣高麗有對盧年老習事【東夷傳高句麗置官有相加對盧沛者陳壽曰其置官有對盧則不置沛者有沛者則不置對盧薛居正曰高麗官其大者號大對盧比一品總知國事對盧以下官總十一級列置州縣六十餘大城置耨薩比都督小城置運使比刺史】謂延壽曰秦王内芟羣雄【芟所銜翻】外服戎狄獨立為帝此命世之材今舉海内之衆而來不可敵也為吾計者莫若頓兵不戰曠日持久分遣奇兵斷其運道【斷丁管翻】糧食既盡求戰不得欲歸無路乃可勝也【此即帝所謂上策也】延壽不從引軍直進去安市城四十里上猶恐其低徊不至命左衛大將軍阿史那社爾將突厥千騎以誘之【厥九勿翻騎奇寄翻誘音酉】兵始交而偽走高麗相謂曰易與耳競進乘之至安市城東南八里依山而陳【易以䜴翻陳讀曰陣下為陳陳於布陳其陳同】上悉召諸將問計長孫無忌對曰臣聞臨敵將戰必先觀士卒之情臣適行經諸營見士卒聞高麗至皆拔刀結斾喜形於色此必勝之兵也陛下未冠【冠古玩翻】身親行陣【行戶剛翻】凡出奇制勝皆上禀聖謀諸將奉成算而已今日之事乞陛下指蹤【以獵為喻指示獸蹤則狗得以追殺】上笑曰諸公以此見讓朕當為諸公商度【度徒洛翻】乃與無忌等從數百騎乘高望之觀山川形勢可以伏兵及出入之所高麗靺鞨合兵為陳長四十里【長直亮翻】江夏王道宗曰高麗傾國以拒王師平壤之守必弱願假臣精卒五千覆其本根則數十萬之衆可不戰而降上不應【為上悔不用道宗策張本夏戶雅翻】遣使紿延壽曰我以爾國彊臣弑其主故來問罪至於交戰非吾本心入爾境芻粟不給故取爾數城俟爾國修臣禮則所失必復矣延壽信之不復設備【使疏吏翻紿蕩亥翻復扶又翻】上夜召文武計事命李世勣將步騎萬五千陳於西嶺長孫無忌將精兵萬一千為奇兵自山北出於狹谷以衝其後上自將步騎四千挾鼓角偃旗幟登北山上敕諸軍聞鼓角齊出奮擊因命有司張受降幕於朝堂之側【降戶江翻朝直遥翻行營備宫省之制故亦有朝堂】戊午延壽等獨見李世勣布陳勒兵欲戰上望見無忌軍塵起命作鼓角舉旗幟諸軍鼓譟並進延壽等大懼欲分兵禦之而其陳已亂會有雷電【方合戰而雷電皆至】龍門人薛仁貴【龍門漢皮氏縣地後魏曰龍門縣并置龍門郡隋廢郡以縣屬蒲州唐武德初為泰州治所貞觀十七年州廢屬絳州薛仁貴自編戶應募】著奇服大呼陷陳【著陟略翻呼火故翻】所向無敵高麗兵披靡【披普彼翻】大軍乘之高麗兵大潰斬首二萬餘級上望見仁貴召拜游擊將軍【唐制武散階游擊將軍從五品下同】仁貴安都之六世孫【薛安都為將以勇聞於宋魏之間】名禮以字行延壽等將餘衆依山自固上命諸軍圍之長孫無忌悉撤橋梁斷其歸路【斷丁管翻】己未延壽惠真帥其衆三萬六千八百人請降 【考異曰實錄云李勣奏曰向若陛下不自親行臣與道宗將數萬人攻安市城未克延壽等十餘萬抽戈齊至城内兵士復應開門而出臣救首救尾旋踵即敗必為延壽等縳送向平壤為莫離支所笑今日臣敢謝陛下性命恩澤帝素狎勣笑而領之按勣後獨將兵取高麗豈必太宗親行邪此非史官虚美乃勣諛辭耳今不取】入軍門膝行而前拜伏請命上語之曰東夷少年跳梁海曲至於摧堅决勝故當不及老人自今復敢與天子戰乎【語牛倨翻少詩照翻復扶又翻下無復同】皆伏地不能對上簡耨薩以下酋長三千五百人授以戎秩遷之内地【酋慈由翻長知兩翻】餘皆縱之使還平壤皆雙舉手以顙頓地歡呼聞數十里外【聞音問】收靺鞨三千三百人悉阬之【以靺鞨犯陣也】獲馬五萬匹牛五萬頭鐵甲萬領佗器械稱是【稱尺證翻】高麗舉國大駭後黄城銀城皆自拔遁去數百里無復人煙上驛書報太子仍與高士亷等書曰朕為將如此何如【史言太宗有矜功之心將即亮翻】更名所幸山曰駐驆山【據舊史其山本名六山更工衡翻】秋七月辛未上徙營安市城東嶺己卯詔標識戰死者尸【識音志】俟軍還與之俱歸戊子以高延壽為鴻臚卿【臚陵如翻】高惠直為司農卿張亮軍過建安城下壁壘未固士卒多出樵牧高麗兵奄至軍中駭擾亮素怯踞胡床直視不言將士見之更以為勇總管張金樹等鳴鼓勒兵擊高麗破之八月甲辰候騎獲莫離支諜者高竹離反接詣軍門【反接兩手縳之也騎奇寄翻諜逹恊翻下同】上召見解縳問曰何瘦之甚對曰竊道間行【間古莧翻下同】不食數日矣命賜之食謂曰爾為謀宜速反命為我寄語莫離支【語牛倨翻下語爾同】欲知軍中消息可遣人徑詣吾所何必間行辛苦也竹離徒跣上賜屩而遣之【屩居灼翻草履也】丙午徙營於安市城南上在遼外凡置營但明斥候不為塹壘雖逼其城高麗終不敢出為寇抄【塹七艶翻】軍士單行野宿如中國焉【史言帝威懾絶域所謂善師者不陳】上之伐高麗也薛延陀遣使入貢【使疏吏翻】上謂之曰語爾可汗【可從刋入聲汗音寒】今我父子東征高麗汝能為寇宜亟來真珠可汗惶恐遣使致謝且請發兵助軍上不許及高麗敗於駐驆山莫離支使靺鞨說真珠啗以厚利真珠懾服不敢動【說輸芮翻啗徒覽翻又徒濫翻懾之涉翻 考異曰實録上謂近臣曰以我量之延陀其死矣聞者莫能測按太宗雖明安能料薛延陀之死今不取】九月壬申真珠卒【卒子恤翻】上為之發哀【為于偽翻】初真珠請以其庶長子曳莽為突利失可汗居東方統雜種【長知兩翻種章勇翻】嫡子拔灼為肆葉護可汗居西方統薛延陀詔許之皆以禮冊命曳莾性躁擾【躁則到翻】輕用兵與拔灼不恊真珠卒來會喪既葬曳莾恐拔灼圖已先還所部拔灼追襲殺之自立為頡利俱利薛沙多彌可汗【為薛延陀亂亡張本】 上之克白巖也謂李世勣曰吾聞安市城險而兵精其城主材勇莫離支之亂城守不服莫離支擊之不能下因而與之建安兵弱而糧少【少詩沼翻】若出其不意攻之必克公可先攻建安建安下則安市在吾腹中此兵法所謂城有所不攻者也【孫子兵法之言】對曰建安在南安市在北吾軍糧皆在遼東今踰安市而攻建安若賊斷吾運道將若之何【斷丁管翻】不如先攻安市安市下則鼓行而取建安耳上曰以公為將【將即亮翻】安得不用公策勿誤吾事世勣遂攻安市安市人望見上旗蓋輒乘城鼓譟上怒世勣請克城之日男女皆阬之安市人聞之益堅守攻久不下高延壽高惠真請於上曰奴既委身大國不敢不獻其誠欲天子早成大功奴得與妻子相見安市人顧惜其家人自為戰未易猝拔【易以豉翻】今奴以高麗十餘萬衆望旗沮潰【沮在呂翻】國人膽破烏骨城耨薩老耄不能堅守移兵臨之期至夕克其餘當道小城必望風奔潰然後收其資糧鼓行而前平壤必不守矣羣臣亦言張亮兵在沙城【沙城即卑沙城】召之信宿可至乘高麗兇懼【兇許拱翻】倂力拔烏骨城度鴨緑水直取平壤在此舉矣上將從之獨長孫無忌以為天子親征異於諸將不可乘危徼幸【徼古堯翻】今建安新城之虜衆猶十萬若向烏骨皆躡吾後不如先破安市取建安然後長驅而進此萬全之策也上乃止【太宗之定天下多以出奇取勝獨遼東之役欲以萬全制敵所以無功】諸軍急攻安市上聞城中雞彘聲謂李世勣曰圍城積久城中煙火日微今雞彘甚喧此必饗士欲夜出襲我宜嚴兵備之是夜高麗數百人縋城而下【縋馳偽翻】上聞之自至城下召兵急擊斬首數十級高麗退走江夏王道宗督衆築土山於城東南隅浸逼其城城中亦增高其城以拒之士卒分番交戰日六七合衝車礮石壞其樓堞【礮與砲同匹貌翻壤音怪】城中隨立木栅以塞其缺道宗傷足上親為之針【塞悉則翻為于偽翻】築山晝夜不息凡六旬用功五十萬山頂去城數丈下臨城中道宗使果毅傳伏愛將兵屯山頂以備敵山頹壓城城崩會伏愛私離所部【離力智翻】高麗數百人從城缺出戰遂奪據土山塹而守之【塹士艶翻】上怒斬伏愛以狥命諸將攻之三日不能克道宗徒跣詣旗下請罪上曰汝罪當死但朕以漢武殺王恢【見十八卷元光二年】不如秦穆用孟明【秦穆公使孟明帥師東伐再為晉師所敗穆公復用孟明孟明增修其政帥師伐晉晉人不敢出遂覇西戎】且有破蓋牟遼東之功故特赦汝耳上以遼左早寒草枯水凍士馬難久留且糧食將盡癸未勑班師先拔遼蓋二州戶口渡遼乃耀兵於安市城下而旋城中皆屏跡不出【屏必郅翻】城主登城拜辭上嘉其固守賜縑百匹【縑幷絲繒也】以勵事君命李世勣江夏王道宗將步騎四萬為殿【殿丁練翻】乙酉至遼東丙戍度遼水遼澤泥潦車馬不通命長孫無忌將萬人翦草填道水深處以車為梁上自繫薪於馬鞘以助役【將即亮翻鞘所交翻鞭鞘也按孔頴逹禮記正義曰弓頭為鞘此所謂馬鞘蓋馬鞍頭也】冬十月丙申朔上至蒲溝駐馬督填道諸軍度渤錯水【蒲溝渤錯水皆在遼澤中】暴風雪士卒沾濕多死者勑然火於道以待之凡征高麗拔玄菟横山蓋牟磨米遼東白巖卑沙麥谷銀山後黄十城【菟同都翻磨莫臥翻】徙遼蓋巖三州戶口入中國者七萬人 【考異曰實録上云徙三州戶口入内地者前後七萬人下癸丑詔書云獲戶十萬口十有八萬蓋并不徙者言之耳】新城建安駐驆三大戰斬首四萬餘級戰士死者幾二千人【幾音祁近也】戰馬死者什七八上以不能成功深悔之歎曰魏徵若在不使我有是行也命馳驛祀徵以少牢【少詔照翻】復立所製碑【踣碑見上卷十七年】召其妻子詣行在勞賜之【勞力到翻】丙午至營州【營州至洛陽二千九百一十里】詔遼東戰亡士卒骸骨並集柳城東南命有司設太牢上自作文以祭之臨哭盡哀其父母聞之曰吾兒死而天子哭之死何所恨上謂薛仁貴曰朕諸將皆老思得新進驍勇者將之【將即亮翻驍堅堯翻】無如卿者朕不喜得遼東喜得卿也丙辰上聞太子奉迎將至從飛騎三千人馳入臨渝關【漢遼西郡有臨渝縣唐志營州有渝關守捉城杜佑曰臨渝關在平州盧龍縣城東百八十里騎奇寄翻師古曰渝音喻】道逢太子上之發定州也指所御褐袍謂太子曰俟見汝乃易此袍耳在遼左雖盛暑流汗弗之易及秋穿敗左右請易之上曰軍士衣多弊吾獨御新衣可乎至是太子進新衣乃易之諸軍所虜高麗民萬四千口先集幽州將以賞軍士上愍其父子夫婦離散命有司平其直悉以錢布贖為民讙呼之聲三日不息【讙許爰翻】十一月辛未車駕至幽州高麗民迎於城東拜舞呼號【號戶高翻】宛轉於地塵埃彌望庚辰過易州境司馬陳元璹使民於地室蓄火種蔬而進之上惡其諂免元璹官【璹殊玉翻惡烏路翻】丙戍車駕至定州丁亥吏部尚書楊師道坐所署用多非其才左遷工部尚書壬辰車駕發定州十二月辛丑上病癰御步輦而行戊申至并州太子為上吮癰扶輦步從者數日【為于偽翻吮徐兖翻從才用翻】辛亥上疾瘳百官皆賀【廖丑留翻】上之征高麗也使右領軍大將軍執失思力將突厥屯夏州之北以備薛延陀【力將即亮翻下同夏戶雅翻】薛延陀多彌可汗既立以上出征未還引兵寇河南【河南者北河之南即朔方新秦之地】上遣左武候中郎將長安田仁會與思力合兵擊之思力嬴形偽退誘之深入及夏州之境整陳以待之【嬴倫為翻誘音酉陳讀曰陣】薛延陀大敗追奔六百餘里耀威磧北而還【磧七迹翻還從宣翻又如字 考異曰高宗實録云會延陀死耀威漠北而還其意指真珠為延陀也按真珠憚太宗威靈不敢入寇又死在九月而此云冬來寇必非真珠也田仁會傳作十八年亦誤也】多彌復發兵寇夏州【復扶又翻】己未敕禮部尚書江夏王道宗發朔并汾箕嵐代忻蔚雲九州兵鎮朔州【武德三年分并州之樂平遼山平城石艾置遼州樂平郡八年改曰箕州後周置蔚州於漢代郡之靈丘隋廢州以靈丘縣屬肆州唐武德六年分肆州之靈丘易州之飛狐地置蔚州雲州雲中郡貞觀十四年自朔州北定襄城徙治定襄縣其地寔隋馬邑郡之雲内縣恒安鎮即後魏所都平城也開元十八年改定襄縣為雲中縣蔚紆勿翻】右衛大將軍代州都督薛萬徹左驍衛大將軍阿史那社爾發勝夏銀綏丹延鄜坊石隰十州兵鎮勝州【勝州隋之榆林郡後魏舊有銀州隋廢為儒林縣屬綏州貞觀二年分綏州之儒林真鄉縣復置銀州銀川郡漢西河之圁隂圁陽縣地也圁音銀杜祐曰銀州春秋白狄地治儒林縣漢圁隂縣地丹州古孟門河西之地西魏置汾州義川郡後改州為丹州隋廢州及郡以義川縣屬延州義寧元年分延州之義川咸寧汾川置丹州咸寧郡坊州春秋白狄之地姚興置中部縣後魏置中部郡隋廢郡以中部縣屬敷州武德二年分鄜州置坊州中部郡以周天和七年元皇帝放牧鄜州於此置馬坊也鄜音膚】勝州都督宋君明左武候將軍薛孤吳發靈原寧鹽慶五州兵鎮靈州【西魏於五原置西安州後改為鹽川隋廢州為鹽川郡貞觀二年復置鹽州】又令執失思力發靈勝二州突厥兵與道宗等相應薛延陀至塞下知有備不敢進 初上留侍中劉洎輔皇太子於定州仍兼左庶子檢校民部尚書總吏禮戶部三尚書事【劉洎既檢校民部尚書又總吏禮是為三尚書事民部之外安得復有戶部哉唐六典貞觀二十三年始改民部為戶部洎其冀翻】上將行謂洎曰我今遠征爾輔太子安危所寄宜深識我意對曰願陛下無憂大臣有罪者臣謹即行誅上以其言妄發頗怪之戒曰卿性踈而太健必以此敗深宜慎之及上不豫洎從内出色甚悲懼謂同列曰疾勢如此聖躬可憂或譛於上曰洎言國家事不足憂但當輔幼主行伊霍故事大臣有異志者誅之自定矣上以為然【因洎於上前先有誅有罪大臣之言遂信譛者之言為然】庚申下詔稱洎與人竊議窺窬萬一謀執朝衡自處伊霍【朝直遥翻處昌呂翻】猜忌大臣皆欲夷戮宜賜自盡【賜自盡即賜死也令自盡其命 考異曰實録云黄門侍郎褚遂良誣奏之曰國家之事不足慮也正當輔少主行伊霍大臣有異志者誅之自然定矣太宗疾愈詔問其故洎以實對遂良執證之不已洎引中書令馬周以自明太宗問周周對與洎所陳不異帝以詰遂良又證周諱之洎遂及罪按此事中人所不為遂良忠直之臣且素無怨仇何至如此蓋許敬宗惡遂良故修實録時以洎死歸咎於遂良耳今不取】免其妻孥【孥音奴】中書令馬周攝吏部尚書以四時選為勞【四時選始一百九十二卷元年選須絹翻】請復以十一月選至三月畢從之【復扶又翻】 是歲右親衛中郎將裴行方【六典曰隋氏左右親衛左右勲衛左右翊衛各置開府一人武德七年改開府各置中郎將一人正四品下掌各領其屬以宿衛而各總其府事將即亮翻】討茂州叛羌黄郎弄大破之【貞觀八年改會州汶山郡曰茂州取界内茂滋山為名後書冉駹其山有六夷七羌九氏各部落】窮其餘黨西至乞習山臨弱水而歸【蜀之西山有弱水】<br />
<br />
  二十年春正月辛未夏州都督喬師望右領軍大將軍執失思力等擊薛延陀大破之虜獲二千餘人多彌可汗輕騎遁去【騎奇寄翻】部内騷然矣 丁丑遣大理卿孫伏伽等二十二人以六條廵察四方【用漢六條也】刺史縣令以下多所貶黜其人詣闕稱寃者前後相屬【屬之欲翻】上令褚遂良類狀以聞上親臨决以能進擢者二十人以罪死者七人流以下除免者數百千人 二月乙未上發并州三月己巳車駕還京師【并州至京師一千三百六十里】上謂李靖曰吾以天下之衆困於小夷何也靖曰此道宗所解【解戶買翻】上顧問江夏王道宗具陳在駐驆時乘虚取平壤之言上悵然曰當時匆匆吾不憶也【是役也不唯不用乘虚取平壤之策乘勝取烏骨之策亦不用也】 上疾未全平欲專保養庚午詔軍國機務並委皇太子處决於是太子間日聼政於東宫既罷則入侍藥膳不離左右【處昌呂翻間古莧翻離力智翻】上命太子暫出遊觀太子辭不願出上乃置别院於寢殿側使太子居之褚遂良請遣太子旬日一還東宫與師傅講道義從之上嘗幸未央宫辟仗已過【辟仗者衛士在駕前攘辟左右止行人所謂陳兵清道而後行也辟音闢】忽於草中見一人帶横刀【横刀者用皮襻帶之刀横於掖下】詰之【詰去吉翻】曰聞辟仗至懼不敢出辟仗者不見遂伏不敢動上遽引還顧謂太子茲事行之則數人當死汝於後逹縱遣之又嘗乘腰輿【腰輿令人舉之其高至腰】有三衛誤拂御衣【親衛勲衛翊衛謂之三衛】其人懼色變上曰此間無御史吾不汝罪也 陜人常惪玄告刑部尚書張亮養假子五百人與術士公孫常語云名應圖䜟【陜失冉翻䜟楚譛翻】又問術士程公頴曰吾臂有龍鱗起欲舉大事可乎上命馬周等按其事亮辭不服上曰亮有假子五百人養此輩何為正欲反耳命百官議其獄皆言亮反當誅獨將作少匠李道裕言亮反形未具【將作少匠從四品下】罪不當死上遣長孫無忌房玄齡就獄與亮訣曰法者天下之平與公共之公自不謹與凶人往還陷入於法今將奈何公好去【好去者與之决别之辭】己丑亮與公潁俱斬西市籍沒其家歲餘刑部侍郎缺上命執政妙擇其人擬數人皆不稱旨既而曰朕得其人矣往者李道裕議張亮獄云反形未具此言當矣【稱尺證翻當丁浪翻】朕雖不從至今悔之遂以道裕為刑部侍郎 閏月癸巳朔日有食之 戊戍罷遼州都督府及巖州【伐高麗所得二州】 夏四月甲子太子太保蕭瑀解太保仍同中書門下三品 五月甲寅高麗王藏及莫離支蓋金遣使謝罪【使疏吏翻下同】并獻二美女上還之金即蘇文也 六月丁卯西突厥乙毗射匱可汗遣使入貢且請昏上許之且使割龜茲于闐踈勒朱俱波蔥嶺五國以為聘禮【于闐時兼有漢戎廬扞彌渠勒皮山五國故地踈勒在蔥嶺東北判汗國治葱嶺中都城杜佑曰朱俱波亦曰朱俱槃漢子合國也去踈勒八九百里】 薛延陀多彌可汗性褊急猜忌無恩廢弃父時貴臣專用已所親昵【昵尼質翻】國人不附多彌多所誅殺人不自安回紇酋長吐迷度與僕骨同羅共擊之【紇下沒翻酋慈由翻長知兩翻】多彌大敗乙亥詔以江夏王道宗左衛大將軍阿史那社爾為瀚海安撫大使又遣右領衛大將軍執失思力將突厥兵右驍衛大將軍契苾何力將凉州及胡兵代州都督薛萬徹營州都督張儉各將所部兵分道並進以擊薛延陀上遣校尉宇文法詣烏羅護靺鞨【烏羅護直京師東北六千里一日烏羅渾即後魏之烏洛侯也東鄰靺鞨大抵風俗皆靺鞨也將即亮翻驍堅堯翻契欺訖翻苾毘必翻校戶教翻靺鞨音未曷】遇薛延陀阿波設之兵於東境法帥靺鞨擊破之薛延陀國中驚擾曰唐兵至矣諸部大亂多彌引數千騎奔阿史悳時健部落【頡利滅李靖徙突厥嬴破數百帳於雲中以阿史德為之長衆稍盛】迴紇攻而殺之并其宗族殆盡遂據其地諸俟斤互相攻擊爭遣使來歸命【俟渠之翻】薛延陀餘衆西走猶七萬餘口共立真珠可汗兄子咄摩支為伊特勿失可汗歸其故地尋去可汗之號【咄當沒翻去羌呂翻】遣使奉表請居欝督軍山之北使兵部尚書崔敦禮就安集之敕勒九姓酋長以其部落素服薛延陀種聞咄摩支來皆恐懼朝議恐其為磧北之患乃更遣李世勣與九姓勑勒共圖之上戒世勣曰降則撫之叛則討之【種草勇翻朝直遥翻磧七迹翻降戶江翻下同 考異曰舊李勣傳云詔勣以二百騎發突厥兵討擊今從鐵勒傳】己丑上手詔以薛延陀破滅其勑勒諸部或來降附或未歸服今不乘機恐貽後悔朕當自詣靈州招撫其去歲征遼東兵皆不調發【調徒釣翻】時太子當從行少詹事張行成上疏以為皇太子從幸靈州不若使之監國【上時掌翻監古銜翻】接對百寮明習庶政既為京師重鎮且示四方盛悳宜割私愛俯從公道上以為忠進位銀青光禄大夫 李世勣至欝督軍山 【考異曰勣傳作烏德揵山唐歷云即欝督軍山虜語兩音也鐵勒傳云至于天山今從唐歷】其酋長梯真逹官帥衆來降【帥讀曰率】薛延陀咄摩支南奔荒谷世勣遣通事舍人蕭嗣業往招慰咄摩支詣嗣業降其部落猶持兩端世勣縱兵追擊前後斬五千餘級虜男女三萬餘人秋七月咄摩支至京師拜右武衛大將軍 八月甲子立皇孫忠為陳王 己巳上行幸靈州 江夏王道宗兵既渡磧遇薛延陀阿波逹官衆數萬拒戰道宗擊破之斬首千餘級追奔二百里道宗與薛萬徹各遣使招諭勑勒諸部其酋長皆喜頓首請入朝【朝直遥翻】庚午車駕至浮陽【浮陽舊書作涇陽當從之涇陽縣前漢屬安定郡後漢晉省後魏屬隴東郡隋唐屬京兆杜佑曰京兆涇陽縣乃秦封涇陽君之地漢涇陽縣在今平凉郡界涇陽故城是此時車駕蓋至京兆之涇陽】迴紇拔野古同羅僕骨多濫葛思結阿跌契苾跌結渾斛薛等十一姓各遣使入貢【跌徒結翻 考異曰舊回紇鐵勒傳作多覽葛今從實録及本紀唐歷又回紇傳陳彭年唐紀作斛薩鐵勤傳作解薛今從實録實録又有契丹奚云十三姓按契丹奚本非薛延陀所統又内附已久嘗從征遼非至此乃降今從舊本紀】稱薛延陁不事大國暴虐無道不能與奴等為主自取敗死部落鳥散不知所之奴等各有分地【分扶問翻】不從薛延陀去歸命天子願賜哀憐乞置官司養育奴等上大喜辛未詔回紇等使者宴樂頒賚拜官【樂音洛】賜其酋長璽書【璽斯氏翻】遣右領軍中郎將安永壽報使【使疏吏翻】 壬申上幸漢故甘泉宫【甘泉宫在京兆雲陽縣界磨石嶺又曰磨盤嶺又曰車盤嶺元和志曰當其登山必自車箱阪而上阪在雲陽縣西北三十八里縈紆曲折單車財通上阪即平原宏敞樓觀相屬以其曲折故名】詔以戎狄與天地俱生上皇並列流殃構禍乃自運初【言戎狄之流殃構禍乃自唐興運之初也】朕聊命偏師遂擒頡利始弘廟略已滅延陁鐵勒百餘萬戶散處北溟【處昌呂翻】遠遣使人委身内屬請同編列並為州郡混元以降【太極元氣函三為一混沌未分謂之混元】殊未前聞宜備禮告廟仍頒示普天 庚辰至涇州丙戍踰隴山【隴山時屬隴州汧源縣界】至西瓦亭觀馬牧【原州平高縣南有瓦亭故關瓦亭水出隴山東北斜趣西南流經成紀略陽顯親界又東南出新陽峽入于渭故有東西瓦亭之别】九月上至靈州【靈州在京師西北千二百五十里】勑勒諸部俟斤遣使相繼詣靈州者數千人咸云願得天至尊為奴等天可汗子子孫孫常為天至尊奴死無所恨甲辰上為詩序其事曰雪耻酬百王除凶報千古公卿請勒石於靈州從之特進同中書門下三品宋公蕭瑀性狷介與同寮多不合【狷吉縣翻】嘗言於上曰房玄齡與中書門下衆臣朋黨不忠執權膠固陛下不詳知但未反耳上曰卿言得無大甚人君選賢才以為股肱心膂當推誠任之人不可以求備必捨其所短取其所長朕雖不能聰明何至頓迷臧否乃至於是【否音鄙】瑀内不自得既數忤旨【數所角翻忤五故翻】上亦銜之但以其忠直居多未忍廢也上嘗謂張亮曰卿既事佛何不出家瑀因自請出家上曰亦知公雅好桑門今不違公意【好呼到翻】瑀須臾復進曰【復扶又翻】臣適思之不能出家上以瑀對羣臣發言反覆尤不能平會稱足疾不朝或至朝堂而不入見上知瑀意終快怏冬十月手詔數其罪曰【朝直遥翻見賢遍翻怏於兩翻數所具翻】朕於佛教非意所遵求其道者未驗福於將來修其教者翻受辜於既往至若梁武窮心於釋氏簡文銳意於法門傾帑藏以給僧袛殫人力以供塔廟【帑他朗翻藏徂浪翻袛巨支翻事並見梁紀】及乎三淮沸浪【三淮本之詩淮有三洲】五嶺騰煙【謂侯景既亂而蕭勃元蘭又復亂於嶺南也】假餘息於熊蹯引殘魂於雀鷇【熊蹯楚成王事雀鷇趙武靈王事引以喻梁武餓死於臺城蹯音煩鷇苦侯翻】子孫覆亡而不暇社稷俄頃而為墟報施之徵何其謬也【施式豉翻】瑀踐覆車之餘軌襲亡國之遺風弃公就私未明隐顯之際身俗口道莫辨邪正之心修累葉之殃源祈一躬之福本上以違忤君主下則扇習浮華自請出家尋復違異【復扶又翻】一迴一惑在乎瞬息之間自可自否變於帷扆之所【帷扆之所謂天子朝羣臣之所】乖棟梁之體豈具曕之量乎朕隐忍至今瑀全無悛改【悛丑緣翻】可商州刺史【商州漢弘農上洛商縣地晉置上洛郡後魏置洛州後周改商州京師至商州二百八十一里】仍除其封 上自高麗還蓋蘇文益驕恣雖遣使奉表其言率皆詭誕又待唐使者倨慢常窺伺邊隙屢勑令勿攻新羅而侵陵不止壬申詔勿受其朝貢更議討之【使疏吏翻伺相吏翻朝直遥翻】 丙戌車駕還京師冬十月己丑上以幸靈州往還冒寒疲頓欲於歲前專事保攝十一月己丑詔祭祀表疏胡客兵馬宿衛行魚契給驛【祭祀謂郊廟社稷明堂也表疏在朝羣臣及四方所上者胡客四夷朝貢之客兵馬調遣征伐及番上宿衛者也符寶郎掌天子八寶及國之符節辨其所用有事則請之於内既事則奉而藏之藏其左而班其右以合中外之契一曰銅魚符所以起軍旅易守長二曰傳符所以給郵驛通制命三曰隨身魚符所以明貴賤應徵召四曰木契所以重鎮守慎出納五曰旌節所以委良能假賞罰魚符之制王畿之内左三右一王畿之外左五右一左者在内右者在外行用之日從第一為首後事須用以次發之周而復始大事兼敕書小事但降符函封遣使合而行之傳符之制太子監國曰雙龍符左右各十京都留守曰麟符左二十其右一十有九東方曰青龍符西方曰騶虞符南方曰朱雀符北方曰玄武符左四右三左者進内右者付外隨身符之制左二右一太子以玉親王以金庶官以銅佩以為飾刻姓名者去官而納焉不刻者傳而佩之木契之制太子監國則王畿之内左右各三王畿之外左右各五庶官鎮守則左右各十旌節之制命大將帥及遣使於四方則請而假之旌以專賞節以專殺】授五品以上官及除解决死罪皆以聞餘並取皇太子處分【處昌呂翻分扶問翻】 十二月己丑羣臣累請封禪從之詔造羽衛送洛陽宫 戊寅迴紇俟利發吐迷度僕骨俟利發歌濫拔延多濫葛俟斤末拔野古俟利發屈利失同羅俟利發時健啜思結酋長烏碎及渾斛薛奚結阿跌契苾白霫酋長皆來朝 庚辰上賜宴於芳蘭殿【按閣本大明宮圖玄武門右玄武殿後有紫蘭殿大樂宴胡客率引入玄武門今此芳蘭殿豈紫蘭殿邪】命有司每五日一會癸未上謂長孫無忌等曰今日吾生日世俗皆為樂【樂音洛下宴樂同】在朕翻成傷感今君臨天下富有四海而承歡膝下永不可得此子路所以有負米之恨也【家語子路見孔子曰㫺由事二親之時常食藜藿之實為親負米百里之外親沒之後南游於楚後車百乘積粟萬鍾累茵而坐列鼎而食願欲食藜藿為親負米不可得也子曰由也事親可謂生事盡力死事盡思者也】詩云哀哀父母生我劬勞【詩蓼莪之辭】奈何以劬勞之日更為宴樂乎因泣數行下【行戶剛翻】左右皆悲 房玄齡嘗以微譴歸第褚遂良上疏以為玄齡自義旗之始翼贊聖功【謂謁見於軍門署為記室時也上時掌翻】武悳之季冒死决策【謂誅建成元吉時也】貞觀之初選賢立政【謂遜直於王魏在朝文武隨能收叙也觀古玩翻】人臣之勤玄齡為最自非有罪在不赦搢紳同尤不可遐弃陛下若以其衰老亦當諷諭使之致仕退之以禮不可以淺鮮之過【鮮少也鮮息淺翻】弃數十年之勲舊上遽召出之頃之玄齡復避位還家【復扶又翻】久之上幸芙蓉園【芙蓉園在京城東南隅秦之隑州漢之樂遊苑唐之曲江同此地也長安志曰隋營宫城宇文愷以其地在京城東南隅地高不便故闕此地不為居人坊巷而鑿之為池以厭勝之又會黄渠水自城外南來入城為芙蓉池且為芙蓉園也劉餗小說曰園本古曲江文帝惡其名曲改曰芙蓉為其水盛而芙蓉富也】玄齡勑子弟汛掃門庭曰乘輿且至【乘繩證翻】有頃上果幸其第因載玄齡還宫<br />
<br />
  二十一年春正月開府儀同三司申文獻公高士亷疾篤辛卯上幸其第流涕與訣壬辰薨上將往哭之房玄齡以上疾新愈固諫上曰高公非徒君臣兼以故舊姻戚【高士亷長孫后之母舅也士亷識帝於龍濳因以甥女妻帝】豈得聞其喪不往哭乎公勿復言【復扶又翻】帥左右自興安門出【按六典大明宫南面五門次西曰興安門但貞觀以前人主常居太極宫高宗龍朔之後方居大明宫然此時已營永安宫永安即大明也或者帝自永安宫而出興安門歟按舊書高士亷傳上出興安門至延喜門長孫無忌迎諫馬首延喜門直皇城之東北隅而興安門直大明宫城之西南隅由大明之興安門至皇城之延喜門其路迂且遠意太極宫中别自有興安門也帥讀曰率】長孫無忌在士亷喪所聞上將至輟哭迎諫於馬首曰陛下餌金石於方不得臨喪奈何不為宗廟蒼生自重【為于偽翻】且臣舅臨終遺言深不欲以北首夷衾輒屈鑾駕【死者北首夷衾覆尸之衾鄭氏曰夷之言尸也尸之槃曰夷槃床曰夷牀衾曰夷衾移尸曰夷于堂皆依尸而為言者也首式又翻】上不聼無忌中道伏臥流涕固諫上乃還入東苑南望而哭涕下如雨及柩出横橋【長安故城横門外有橋曰横橋柩音舊横音光】上登長安故城西北樓望之慟哭丙申詔以迴紇部為瀚海府僕骨為金微府 【考異曰舊書】<br />
<br />
  【作金微今從實録唐歷】多濫葛為燕然府拔野古為幽陵府同羅為龜林府思結為盧山府【府者都督府也燕因肩翻】渾為臯蘭州斛薛為高闕州奚結為雞鹿州阿跌為雞田州契苾為榆溪州思結别部為蹛林州白霫為寘顔州【蹛音帶寘徒年翻】各以其酋長為都督刺史各賜金銀繒帛及錦袍【繒慈陵翻】勑勒大喜捧戴歡呼拜舞宛轉塵中及還上御天成殿宴設十部樂而遣之諸酋長奏稱臣等既為唐民往來天至尊所如詣父母請於迴紇以南突厥以北開一道謂之參天可汗道置六十八驛各有馬及酒肉以供過使【使疏吏翻】歲貢貂皮以充租賦仍請能屬文人【屬之欲翻】使為表疏【疏所去翻】上皆許之於是北荒悉平然迴紇吐迷度己私自稱可汗官號皆如突厥故事 丁酉詔以明年仲春有事泰山禪社首【應劭曰社首山在漢泰山郡博縣晉灼曰山在鉅平縣南十二里唐志兖州博城縣有社首山】餘並依十五年議 二月丁丑太子釋奠于國學 上將復伐高麗【復扶又翻】朝議以為高麗依山為城攻之不可猝拔【朝直遥翻】前大駕親征國人不得耕種所克之城悉收其穀繼以旱災民太半乏食今若數遣偏師更迭擾其疆場【數所角翻更工衡翻場音亦】使彼疲於奔命釋耒入堡【耒廬對翻】數年之間千里蕭條則人心自離鴨緑之北可不戰而取矣上從之三月以左武衛大將軍牛進逹為青丘道行軍大總管【相如子虚賦曰夫齊東陼鉅海觀乎成山射乎之罘秋獵乎青丘彷徨乎海外服䖍曰青丘國在海東三百里晉天文志有青丘七星在軫東南蠻夷之國也】右武候將軍李海岸副之發兵萬餘人乘樓舩自萊州汎海而入又以太子詹事李世勣為遼東道行軍大總管右武衛將軍孫貳朗等副之將兵三千人【將即亮翻】因營州都督府兵自新城道入兩軍皆選習水善戰者配之 辛卯上曰朕於戎狄所以能取古人所不能取臣古人所不能臣者皆順衆人之所欲故也㫺禹帥九州之民鑿山槎木【帥讀曰率槎士下翻逆斫木也】疏百川注之海其勞甚矣而民不怨者因人之心順地之勢與民同利故也 是月上得風疾苦京師盛暑夏四月乙丑命修終南山太和廢宫為翠微宫【楊大年曰翠微宫在驪山絶頂】 丙寅置燕然都護府統瀚海等六都督臯蘭等七州【六都督七州並見上新書曰置燕府都護府於右單于臺宋白曰在西受降城東南四十里】以揚州都督府司馬李素立為之素立撫以恩信夷落懷之共率馬牛為獻素立唯受其酒一盃餘悉還之 五月戊子上幸翠微宫冀州進士張昌齡獻翠微宫頌上愛其文命於通事舍人裏供奉【資格淺不得除正官命於通事舍人班裏供奉】初昌齡與進士王公治皆善屬文名振京師考功員外郎王師旦知貢舉【屬之欲翻唐初以考功員外郎知貢舉至開元間考功員外郎李昂為舉人詆訶帝以員外郎望輕遂移貢舉於禮部以侍郎主之禮部選士自此始】黜之舉朝莫曉其故及奏第上怪無二人名詰之【朝直遥翻詰去吉翻】師旦對曰二人雖有辭華然其體輕薄終不成令器若置之高第恐後進效之傷陛下雅道上善其言壬辰詔百司依舊啓事皇太子 庚辰上御翠微殿<br />
<br />
  【翠微宫之正殿也】問侍臣曰自古帝王雖平定中夏不能服戎狄【夏戶雅翻】朕才不逮古人而成功過之自不諭其故諸公各率意以實言之羣臣皆稱陛下功悳如天地萬物不得而名言上曰不然朕所以能及此者止由五事耳自古帝王多疾勝己者朕見人之善若己有之人之行能不能兼備【行下孟翻】朕常弃其所短取其所長人主往往進賢則欲寘諸懷退不肖則欲推諸壑【推吐雷翻】朕見賢者則敬之不肖者則憐之賢不肖各得其所人主多惡正直【惡烏路翻】隂誅顯戮無代無之朕踐祚以來正直之士比肩於朝未嘗黜責一人【朝直遥翻】自古皆貴中華賤夷狄朕獨愛之如一故其種落皆依朕如父母【種章勇翻】此五者朕所以成今日之功也顧謂褚遂良曰公嘗為史官【褚遂良嘗知起居注十八年拜黄門侍郎參縂朝政不復兼史職故曰嘗】如朕言得其實乎對曰陛下盛悳不可勝載【勝音升】獨以此五者自與蓋謙謙之志耳 李世勣軍既渡遼歷南蘇等數城【前漢書玄菟郡高句驪縣有南蘇水西北經塞外】高麗多背城拒戰【背蒲妹翻】世勣擊破其兵焚其羅郭而還【還從宣翻又如字】 六月癸亥以司徒長孫無忌領揚州都督實不之任 丁丑詔以隋末喪亂【喪息浪翻】邊民多為戎狄所掠今鐵勒歸化宜遣使詣燕然等州【使疏吏翻下同燕因肩翻】與都督相知訪求沒落之人贖以貨財給糧逓還本貫其室韋烏羅護靺鞨三部人為薛延陀所掠者亦令贈還 癸未以司農卿李緯為戶部尚書【緯于貴翻】時房玄齡留守京師【守手又翻】有自京師來者上問玄齡何言對曰玄齡聞李緯拜尚書但云李緯美髭鬢【髭郎移翻】帝遽改除緯洛州刺史 【考異曰唐歷云居無何改緯太子詹事今從舊傳】 秋七月牛進逹李海岸入高麗境凡百餘戰無不捷攻石城拔之進至積利城下高麗兵萬餘人出戰海岸擊破之斬首二千級 上以翠微宫險隘不能容百官庚子詔更營玉華宫於宜春之鳳皇谷【玉華宫在宜春縣西四十里】庚戌車駕還宫【還從宣翻又音如字】 八月壬戌詔以薛延陀新降土功屢興【降戶江翻屢力句翻又音如字】加以河北水災停明年封襌 辛未骨利幹遣使入貢丙戍以骨利幹為玄闕州拜其俟斤為刺史【使疏吏翻俟渠之翻】骨利幹於鐵勒諸部為最遠晝長夜短日沒後天色正曛煮羊脾適熟日已復出矣【骨利幹居瀚海北產良馬其地北距海至京師最遠又北渡海則晝長夜短蓋近日出處復扶又翻 考異曰實録唐歷皆作羊脾僧一行大衍歷義及舊天文志唐統紀皆作脾新天文志云胹羊髀按正言羊脾者取其易熟故也若煮羊肺及髀則雖中國通夕亦未爛矣今從大衍歷義】 己丑齊州人段志冲上封事【上時掌翻】請上致政於皇太子太子聞之憂形於色發言流涕長孫無忌等請誅志冲【長知兩翻】上手詔曰五岳陵霄四海亘地納汙藏疾無損高深【左傳云川澤納汙山林藏疾亘吉鄧翻】志冲欲以匹夫解位天子【言欲使天子解位也】朕若有罪是其直也若其無罪是其狂也譬如尺霧障天不虧於大寸雲點日何損於明 丁酉立皇子明為曹王明母楊氏巢刺王之妃也有寵於上【刺盧逹翻】文德皇后之崩也欲立為皇后魏徵諫曰陛下方比德唐虞奈何以辰嬴自累【左傳晉太子圉為質於秦秦穆公以女妻子圉將逃歸謂之曰與子歸乎嬴氏不敢從圉遂逃歸及晉公子重耳入秦秦穆公納女五人懷嬴與焉謂之辰嬴賈季曰辰嬴嬖於二君是也累力瑞翻】乃止尋以明繼元吉後 戊戍勅宋州刺史王波利等發江南十二州工人造大船數百艘欲以征高麗【十二州宣潤常蘇湖杭越台㜈括睦洪也艘蘇遭翻麗力知翻】冬十月庚辰奴刺啜匐俟友帥其所部萬餘人内附【奴刺部落居吐谷渾党項之間刺來逹翻啜陟劣翻匐蒲北翻俟渠之翻帥讀曰率下同】十一月突厥車鼻可汗遣使入貢車鼻名斛勃本突厥同族世為小可汗頡利之敗突厥餘衆欲奉以為大可汗時薛延陁力強車鼻不敢當帥其衆歸之或說薛延陁車鼻貴種有勇略【說輸芮翻種章勇翻】為衆所附恐為後患不如殺之車鼻知之逃去薛延陁遣數千騎追之【騎奇寄翻】車鼻勒兵與戰大破之乃建牙於金山之北【其地三垂斗絶惟一面可容車騎壤土夷博】自稱乙注車鼻可汗突厥餘衆稍稍歸之數年間勝兵三萬人【勝音升】時出抄掠薛延陁【抄楚交翻】及薛延陁敗車鼻埶益張【張知亮翻】遣其子沙鉢羅特勒入見【見賢遍翻】又請身自入朝【朝直遥翻下同】詔遣將軍郭廣敬徵之車鼻特為好言初無來意竟不至 【考異曰實録詔遣雲麾將軍安調遮右屯衛郎將韓華迎之車鼻徒飾其辭初無來意韓華將招歌邏禄其刦之車鼻覺其謀華與車鼻子陟苾特勒相射而死調遮亦被殺今從舊突厥傳】 癸卯徙順陽王泰為濮王【濮博木翻】 壬子上疾愈三日一視朝 十二月壬申西趙酋長趙磨帥萬餘戶内附以其地為明州【東謝之南有西趙蠻西抵昆明南即西洱河山宂阻深趙氏世為酋長酋慈由翻長知兩翻】 龜茲王伐疊卒弟訶黎布失畢立【龜茲音丘茲訶虎何翻】浸失臣禮侵漁鄰國上怒戊寅詔使持節崑丘道行軍大總管【自古相傳西域有崑崙山河源所出又爾雅曰三城為崑崙丘故曰崑丘道使疏吏翻】左驍衛大將軍阿史那社爾副大總管左驍衛大將軍契苾何力安西都護郭孝恪等將兵擊之仍命鐵勒十三州突厥吐蕃吐谷渾連兵進討【驍堅堯翻契欺訖翻苾毘必翻將即亮翻吐從暾入聲谷音浴】 高麗王使其子莫離支任武入謝罪上許之<br />
<br />
  二十二年春正月己丑上作帝範十二篇以賜太子曰君體建親求賢審官納諫去讒戒盈崇儉賞罰務農閲武崇文【去羌呂翻】且曰修身治國備在其中【治直之翻】一旦不諱更無所言矣又曰汝當更求古之哲王以為師如吾不足法也夫取法於上僅得其中取法於中不免為下吾居位已來不善多矣錦繡珠玉不絶於前宫室臺榭屢有興作犬馬鷹隼無遠不致【隼息尹翻】行遊四方供頓煩勞此皆吾之深過勿以為是而法之顧我弘濟蒼生其益多肇造區夏其功大益多損少故人不怨功大過微故業不墮【夏戶雅翻少詩沼翻墮讀曰隳】然比之盡美盡善固多愧矣汝無我之功勤而承我之富貴竭力為善則國家僅安驕惰奢縱則一身不保且成遲敗速者國也失易得難者位也可不惜哉可不慎哉【太宗自疏其所行之過差者以戒太子可謂至矣然太子病於柔弱好内乃無一言及此以警策之人莫知其子之惡信矣易以豉翻】 中書令兼右庶子馬周病上親為調藥【為于偽翻】使太子臨問庚寅薨戊戍上幸驪山温湯 己亥以中書舍人崔仁師為中書侍郎參知機務 新羅王金善悳卒以善德妹真德為柱國封樂浪郡王遣使冊命【卒子恤翻樂浪音洛琅使疏吏翻】 丙午詔以右武衛大將軍薛萬徹為青丘道行軍大總管右衛將軍裴行方副之將兵三萬餘人及樓舩戰艦【艦戶黯翻】自萊州泛海以擊高麗 長孫無忌檢校中書令知尚書門下省事【長孫無忌蓋總三省之事】 戊申上還宫 結骨自古未通中國【杜祐曰結骨在回紇西北三千里】聞鐵勒諸部皆服二月其俟利失鉢屈阿棧入朝【俟渠之翻屈居勿翻阿烏葛翻棧士限翻朝直遥翻】其國人皆長大赤髪緑睛【睛音精】有黑髪者以為不祥上宴之於天成殿謂侍臣曰㫺渭橋斬三突厥首自謂功多【謂武德九年頡利犯便橋時也】今斯人在席更不以為怪邪失鉢屈阿棧請除一官執笏而歸誠百世之幸戊午以結骨為堅昆都督府以失鉢屈阿棧為右屯衛大將軍堅昆都督隸燕然都護【燕因肩翻】又以阿史德時健俟斤部落置祈連州隸營州都督是時四夷大小君長爭遣使入獻見【長知兩翻使疏吏翻見賢遍翻下引見同】道路不絶每元正朝賀常數百千人辛酉上引見諸胡使者謂侍臣曰漢武帝窮兵三十餘年疲弊中國所獲無幾【幾居豈翻】豈如今日綏之以德使窮髪之地盡為編戶乎【陸德明經典釋文曰司馬云窮髪北極之下無毛之地也崔云北方無毛地也按毛草也地理書曰山以草木為髪】 上營玉華宫【程大昌曰玉華宫在坊州宜君縣】務令儉約惟所居殿覆以瓦【覆敷又翻】餘皆茅茨然備設太子宫百司苞山絡野所費已巨億計乙亥上行幸玉華宫己卯畋于華原【華原宜君銅官漢雲陽祋祤之地後魏於華原置北雍州西魏改為宜州又置北地郡隋開皇初郡廢大業初州廢以縣屬京兆唐初復置宜州貞觀十七年州廢而以華原復屬於京兆】 中書侍郎崔仁師坐有伏閤自訴者仁師不奏除名流連州【連州漢桂陽陽山之地梁置陽山郡隋置連州大業初廢州為熙平郡唐復為連州連州在京師南三千六百六十五里考異曰舊傳本龔州今從新舊本紀】 三月己丑分瀚海都督俱羅勃部置燭龍州 甲午上謂侍臣曰朕少長兵間頗能料敵【少詩照翻長知兩翻】今崑丘行師處月處密二部及龜茲用事者羯獵顛那利每懷首鼠必先授首弩失畢其次也【弩失畢當作布失畢龜茲王也】 庚子隋蕭后卒詔復其位號諡曰愍使三品護葬備鹵薄儀衛送至江都與煬帝合葬 充容長城徐惠【唐會要曰舊制昭儀昭容昭媛脩儀脩容脩媛充儀充容充媛各一人為九嬪正二品晉武帝太康三年分烏程立長城縣屬吴興郡今湖州長興縣是也惠徐孝德之女】以上東征高麗西討龜茲翠微玉華營繕相繼又服玩頗華靡上疏諫其略曰以有盡之農功填無窮之巨浪圖未獲之他衆喪已成之我軍【喪息浪翻下喪國同】昔秦皇并吞六國反速危亡之基晉武奄有三方翻成覆敗之業【魏蜀吴三方鼎峙至晉混一】豈非矜功恃大弃德輕邦圖利忘危肆情縱欲之所致乎是知地廣非常安之術人勞乃易亂之源也又曰雖復茅茨示約【易以豉翻復扶又翻】猶興木石之疲和雇取人不無煩擾之弊又曰珍玩伎巧乃喪國之斧斤【伎渠綺翻喪息浪翻】珠玉錦繡寔迷心之酖毒又曰作法於儉猶恐其奢作法於奢何以制後上善其言甚禮重之<br />
<br />
  資治通鑑卷一百九十八  <br>
   </div> 

<script src="/search/ajaxskft.js"> </script>
 <div class="clear"></div>
<br>
<br>
 <!-- a.d-->

 <!--
<div class="info_share">
</div> 
-->
 <!--info_share--></div>   <!-- end info_content-->
  </div> <!-- end l-->

<div class="r">   <!--r-->



<div class="sidebar"  style="margin-bottom:2px;">

 
<div class="sidebar_title">工具类大全</div>
<div class="sidebar_info">
<strong><a href="http://www.guoxuedashi.com/lsditu/" target="_blank">历史地图</a></strong>  
<a href="http://www.880114.com/" target="_blank">英语宝典</a>  
<a href="http://www.guoxuedashi.com/13jing/" target="_blank">十三经检索</a> 
<br><strong><a href="http://www.guoxuedashi.com/gjtsjc/" target="_blank">古今图书集成</a></strong> 
<a href="http://www.guoxuedashi.com/duilian/" target="_blank">对联大全</a> <strong><a href="http://www.guoxuedashi.com/xiangxingzi/" target="_blank">象形文字典</a></strong> 

<br><a href="http://www.guoxuedashi.com/zixing/yanbian/">字形演变</a>  <strong><a href="http://www.guoxuemi.com/hafo/" target="_blank">哈佛燕京中文善本特藏</a></strong>
<br><strong><a href="http://www.guoxuedashi.com/csfz/" target="_blank">丛书&方志检索器</a></strong> <a href="http://www.guoxuedashi.com/yqjyy/" target="_blank">一切经音义</a>  

<br><strong><a href="http://www.guoxuedashi.com/jiapu/" target="_blank">家谱族谱查询</a></strong>  <strong><a href="http://shufa.guoxuedashi.com/sfzitie/" target="_blank">书法字帖欣赏</a></strong> 
<br>

</div>
</div>


<div class="sidebar" style="margin-bottom:0px;">

<font style="font-size:22px;line-height:32px">QQ交流群9:489193090</font>


<div class="sidebar_title">手机APP 扫描或点击</div>
<div class="sidebar_info">
<table>
<tr>
	<td width=160><a href="http://m.guoxuedashi.com/app/" target="_blank"><img src="/img/gxds-sj.png" width="140"  border="0" alt="国学大师手机版"></a></td>
	<td>
<a href="http://www.guoxuedashi.com/download/" target="_blank">app软件下载专区</a><br>
<a href="http://www.guoxuedashi.com/download/gxds.php" target="_blank">《国学大师》下载</a><br>
<a href="http://www.guoxuedashi.com/download/kxzd.php" target="_blank">《汉字宝典》下载</a><br>
<a href="http://www.guoxuedashi.com/download/scqbd.php" target="_blank">《诗词曲宝典》下载</a><br>
<a href="http://www.guoxuedashi.com/SiKuQuanShu/skqs.php" target="_blank">《四库全书》下载</a><br>
</td>
</tr>
</table>

</div>
</div>


<div class="sidebar2">
<center>


</center>
</div>

<div class="sidebar"  style="margin-bottom:2px;">
<div class="sidebar_title">网站使用教程</div>
<div class="sidebar_info">
<a href="http://www.guoxuedashi.com/help/gjsearch.php" target="_blank">如何在国学大师网下载古籍?</a><br>
<a href="http://www.guoxuedashi.com/zidian/bujian/bjjc.php" target="_blank">如何使用部件查字法快速查字?</a><br>
<a href="http://www.guoxuedashi.com/search/sjc.php" target="_blank">如何在指定的书籍中全文检索?</a><br>
<a href="http://www.guoxuedashi.com/search/skjc.php" target="_blank">如何找到一句话在《四库全书》哪一页?</a><br>
</div>
</div>


<div class="sidebar">
<div class="sidebar_title">热门书籍</div>
<div class="sidebar_info">
<a href="/so.php?sokey=%E8%B5%84%E6%B2%BB%E9%80%9A%E9%89%B4&kt=1">资治通鉴</a> <a href="/24shi/"><strong>二十四史</strong></a>&nbsp; <a href="/a2694/">野史</a>&nbsp; <a href="/SiKuQuanShu/"><strong>四库全书</strong></a>&nbsp;<a href="http://www.guoxuedashi.com/SiKuQuanShu/fanti/">繁体</a>
<br><a href="/so.php?sokey=%E7%BA%A2%E6%A5%BC%E6%A2%A6&kt=1">红楼梦</a> <a href="/a/1858x/">三国演义</a> <a href="/a/1038k/">水浒传</a> <a href="/a/1046t/">西游记</a> <a href="/a/1914o/">封神演义</a>
<br>
<a href="http://www.guoxuedashi.com/so.php?sokeygx=%E4%B8%87%E6%9C%89%E6%96%87%E5%BA%93&submit=&kt=1">万有文库</a> <a href="/a/780t/">古文观止</a> <a href="/a/1024l/">文心雕龙</a> <a href="/a/1704n/">全唐诗</a> <a href="/a/1705h/">全宋词</a>
<br><a href="http://www.guoxuedashi.com/so.php?sokeygx=%E7%99%BE%E8%A1%B2%E6%9C%AC%E4%BA%8C%E5%8D%81%E5%9B%9B%E5%8F%B2&submit=&kt=1"><strong>百衲本二十四史</strong></a>  <a href="http://www.guoxuedashi.com/so.php?sokeygx=%E5%8F%A4%E4%BB%8A%E5%9B%BE%E4%B9%A6%E9%9B%86%E6%88%90&submit=&kt=1"><strong>古今图书集成</strong></a>
<br>

<a href="http://www.guoxuedashi.com/so.php?sokeygx=%E4%B8%9B%E4%B9%A6%E9%9B%86%E6%88%90&submit=&kt=1">丛书集成</a> 
<a href="http://www.guoxuedashi.com/so.php?sokeygx=%E5%9B%9B%E9%83%A8%E4%B8%9B%E5%88%8A&submit=&kt=1"><strong>四部丛刊</strong></a>  
<a href="http://www.guoxuedashi.com/so.php?sokeygx=%E8%AF%B4%E6%96%87%E8%A7%A3%E5%AD%97&submit=&kt=1">說文解字</a> <a href="http://www.guoxuedashi.com/so.php?sokeygx=%E5%85%A8%E4%B8%8A%E5%8F%A4&submit=&kt=1">三国六朝文</a>
<br><a href="http://www.guoxuedashi.com/so.php?sokeytm=%E6%97%A5%E6%9C%AC%E5%86%85%E9%98%81%E6%96%87%E5%BA%93&submit=&kt=1"><strong>日本内阁文库</strong></a> <a href="http://www.guoxuedashi.com/so.php?sokeytm=%E5%9B%BD%E5%9B%BE%E6%96%B9%E5%BF%97%E5%90%88%E9%9B%86&ka=100&submit=">国图方志合集</a> <a href="http://www.guoxuedashi.com/so.php?sokeytm=%E5%90%84%E5%9C%B0%E6%96%B9%E5%BF%97&submit=&kt=1"><strong>各地方志</strong></a>

</div>
</div>


<div class="sidebar2">
<center>

</center>
</div>
<div class="sidebar greenbar">
<div class="sidebar_title green">四库全书</div>
<div class="sidebar_info">

《四库全书》是中国古代最大的丛书,编撰于乾隆年间,由纪昀等360多位高官、学者编撰,3800多人抄写,费时十三年编成。丛书分经、史、子、集四部,故名四库。共有3500多种书,7.9万卷,3.6万册,约8亿字,基本上囊括了古代所有图书,故称“全书”。<a href="http://www.guoxuedashi.com/SiKuQuanShu/">详细>>
</a>

</div> 
</div>

</div>  <!--end r-->

</div>
<!-- 内容区END --> 

<!-- 页脚开始 -->
<div class="shh">

</div>

<div class="w1180" style="margin-top:8px;">
<center><script src="http://www.guoxuedashi.com/img/plus.php?id=3"></script></center>
</div>
<div class="w1180 foot">
<a href="/b/thanks.php">特别致谢</a> | <a href="javascript:window.external.AddFavorite(document.location.href,document.title);">收藏本站</a> | <a href="#">欢迎投稿</a> | <a href="http://www.guoxuedashi.com/forum/">意见建议</a> | <a href="http://www.guoxuemi.com/">国学迷</a> | <a href="http://www.shuowen.net/">说文网</a><script language="javascript" type="text/javascript" src="https://js.users.51.la/17753172.js"></script><br />
  Copyright &copy; 国学大师 古典图书集成 All Rights Reserved.<br>
  
  <span style="font-size:14px">免责声明:本站非营利性站点,以方便网友为主,仅供学习研究。<br>内容由热心网友提供和网上收集,不保留版权。若侵犯了您的权益,来信即刪。scp168@qq.com</span>
  <br />
ICP证:<a href="http://www.beian.miit.gov.cn/" target="_blank">鲁ICP备19060063号</a></div>
<!-- 页脚END --> 
<script src="http://www.guoxuedashi.com/img/plus.php?id=22"></script>
<script src="http://www.guoxuedashi.com/img/tongji.js"></script>

</body>
</html>
