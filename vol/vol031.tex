資治通鑑卷三十一   宋 司馬光 撰

胡三省 音註

漢紀二十三|{
	起屠維大淵獻盡彊圉恊洽凡九年}


孝成皇帝上之下

陽朔三年春三月壬戌隕石東郡八 夏六月潁川鐵官徒申屠聖等百八十人殺長吏盗庫兵自稱將軍經歷九郡遣丞相長史御史中丞逐捕以軍興從事|{
	長知兩翻師古曰逐捕之事須有發興皆依軍法}
皆㐲辜 秋王鳳疾天子數自臨問|{
	數所角翻}
親執其手涕泣曰將軍病如有不可言|{
	師古曰不可言謂死也不欲斥言之}
平阿侯譚次將軍矣鳳頓首泣曰譚等雖與臣至親行皆奢僭|{
	行下孟翻}
無以率導百姓不如御史大夫音謹敕|{
	敕整也正也固也理也}
臣敢以死保之及鳳且死上疏謝上復固薦音自代|{
	復扶又翻}
言譚等五人必不可用天子然之初譚倨不肯事鳳|{
	師古曰倨慢也}
而音敬鳳卑恭如子故鳳薦之八月丁巳鳳薨九月甲子以王音為大司馬車騎將軍而王譚位特進領城門兵|{
	長安十二城門皆有屯兵}
安定太守谷永以譚失職勸譚辭讓不受城門職由是譚音相與不平 冬十一月丁卯光禄勲于永為御史大夫永定國之子也

四年春二月赦天下 夏四月雨雪|{
	雨于具翻}
秋九月壬申東平思王宇薨|{
	宇宣帝之子}
少府王駿為京兆尹駿吉之子也先是京兆有趙廣漢張敞王尊王章至駿皆有能名故京師稱曰前有趙張後有三王|{
	趙廣漢張敞宣帝時尹京三王皆帝所用史言尹京者難其材先悉薦翻}
閏月壬戌于永卒 烏孫小昆彌烏就屠死子拊離代立|{
	師古曰拊讀與撫同}
為弟日貳所殺漢遣使者立拊離子安日為小昆彌日貳亡阻康居|{
	亡奔康居依阻其遠以自全}
安日使貴人姑莫匿等三人詐亡從日貳刺殺之|{
	師古曰詐畔亡而投之因得以刺殺刺七亦翻}
於是西域諸國上書願復得前都護段會宗|{
	會宗前為西域都護終更而還復扶又翻}
上從之城郭諸國聞之皆翕然親附 谷永奏言聖王不以名譽加於實效御史大夫任重職大少府宣逹於從政唯陛下留神考察上然之

鴻嘉元年春正月癸巳以薛宣為御史大夫|{
	用谷永之言也}
二月壬午上行幸初陵赦作徒|{
	師古曰徒人之在陵役作者}
以新豐之戲鄉為昌陵縣|{
	師古曰戲水之鄉也戲音許宜翻}
奉初陵 上始為微行|{
	張晏曰出入市里不復警蹕若微賤者之所為故曰微行}
從期門郎或私奴十餘人或乘小車或皆騎|{
	騎奇寄翻}
出入市里郊野遠至旁縣|{
	旁縣諸縣環長安旁者也}
甘泉長楊五柞|{
	柞才各翻}
鬭雞走馬常自稱富平侯家人富平侯者張安世四世孫放也放父臨尚敬武公主|{
	文穎曰公主成帝姊也臣瓚曰敬武公主是元帝姊也師古曰二說皆非也薛宣傳云主怒曰嫂何以取妺殺之既謂元后為嫂是即元帝妹也地理志鉅鹿郡有敬武縣}
生放放為侍中中郎將娶許皇后女弟當時寵幸無比故假稱之 三月庚戌張禹以老病罷以列侯朝朔望位特進見禮如丞相|{
	朝直遥翻}
賞賜前後數千萬 夏四月庚辰薛宣為丞相封高陽侯|{
	恩澤侯表高陽侯食邑于東莞}
京兆尹王駿為御史大夫王音既以從舅越親用事小心親職|{
	從才用翻}
上以音自

御史大夫入為將軍|{
	將軍中朝官故曰入}
不獲宰相之封|{
	自公孫弘以來為相者封侯}
六月乙巳封音為安陽侯|{
	地理志汝南郡有安陽侯國}
冬黄龍見真定|{
	見賢遍翻}
是歲匈奴復株累單于死弟且麋胥立為搜諧若鞮單于遣子左祝都韓王昫留斯侯入侍以且莫車為左賢王|{
	累力追翻單音蟬且子余翻鞮丁兮翻昫漢書作胊師古曰音許干翻}


二年春上行幸雲陽甘泉|{
	甘泉宮在雲陽縣}
三月博士行大射禮|{
	古者天子諸侯大夫士皆有大射之禮博士所行士之射禮也}
有飛雉集于庭歷階登堂而雊|{
	師古曰歷階謂以次而登也雊古豆翻}
後雉又集太常宗正丞相御史大夫車騎將軍之府又集未央宮承明殿屋上車騎將軍音待詔寵等上言|{
	師古曰以經術待詔其人名寵不記姓也}
天地之氣以類相應譴告人君甚微而著雉者聽察先聞雷聲故月令以紀氣|{
	師古曰謂季冬之月雉雊鷄乳}
經載高宗雊雉之異以明轉禍為福之驗|{
	師古曰高宗祭成湯有飛雉升鼎耳而雊祖已曰惟先假王正厥事故能禳妖而致百年之夀}
今雉以博士行禮之日歷階登堂萬衆睢睢|{
	師古曰睢睢仰目視貌音呼惟翻}
驚怪連日徑歷三公之府太常宗正典宗廟骨肉之官然後入宮其宿留告曉人具備深切|{
	師古曰宿音先就翻留音力救翻}
雖人道相戒何以過是後帝使中常侍鼂閎詔音曰|{
	鼂古朝字}
聞捕得雉毛羽頗摧折類拘執者得無人為之|{
	師古曰言人放此雉故欲為變異者折而設翻}
音復對曰陛下安得亡國之語不知誰主為佞讇之計誣亂聖德如此者左右阿諛甚衆不待臣音復讇而足|{
	復扶又翻讇古諂字師古曰足益也音子喻翻足其不足曰足}
公卿以下保位自守莫有正言如令陛下覺寤懼大禍且至身深責臣下繩以聖灋臣音當先誅豈有以自解哉今即位十五年繼嗣不立日日駕車而出失行流聞|{
	行所行也言帝所行多非道過失流布聞於遠方也行下孟翻}
海内傳之甚於京師外有微行之害内有疾病之憂皇天數見災異欲人變更|{
	數所角翻見賢遍翻更工衡翻下同}
終已不改天尚不能感動陛下臣子何望獨有極言待死命在朝暮而已如有不然老母安得處所尚何皇太后之有高祖天下當以誰屬乎|{
	如淳曰老母音之老母也當隨己受罪誅也又謂己言深切觸悟人主積恚而犯必行之誅不能復顧太后也師古曰如說非也此言摠屬於成帝耳不然者謂不如所諫而自修改也老母即帝之母太后也言帝不自修改國家危亡太后不知所處高祖天下無所付屬也屬音之欲翻}
宜謀於賢智克己復禮|{
	用論語孔子答顔淵之言}
以求天意繼嗣可立災變尚可銷也 初元帝儉約渭陵不復徙民起邑|{
	事見二十九卷元帝永光四年復扶又翻}
帝起初陵|{
	即延陵也}
數年後樂覇陵曲亭南更營之|{
	即新豐戲鄉之地關中記昌陵在霸城東二十里樂音洛}
將作大匠解萬年|{
	解尸買翻姓也姓譜自晉唐叔虞食邑於解今解縣也晉有解狐解揚}
使陳湯為奏請為初陵徙民起邑欲自以為功求重賞湯因自請先徙冀得美田宅上從其言果起昌陵邑|{
	為萬年湯得罪罷昌陵張本}
夏徙郡國豪桀貲五百萬以上五千戶于昌陵 五月癸未隕石于杜郵三 六月立中山憲王孫雲客為廣德王|{
	中山憲王福靖王勝之玄孫也地節元年福薨 子懷王修嗣五鳳三年修薨無後今立雲客}
是歲城陽哀王雲薨無子國除|{
	城陽景王章傳國十世至雲}


三年夏四月赦天下 大旱 王氏五侯争以奢侈相尚成都侯商嘗病欲避暑從上借明光宮|{
	師古曰黄圖云明光宮在城内近桂宮}
後又穿長安城引内灃水|{
	地理志灃水出鄠縣東南北流過上林苑入渭}
注第中大陂以行船立羽盖|{
	羽盖編羽為之}
張周帷|{
	周帷船之四周皆張帷}
楫棹越歌|{
	師古曰楫棹皆所以行船也令執楫棹人為越歌也楫謂棹之短者也今吴越之人謂之撓音饒越歌為越之歌}
上幸商第見穿城引水意恨内衘之未言後微行出過曲陽侯第又見園中土山漸臺象白虎殿|{
	起土山漸臺又為室屋象白虎殿也}
於是上怒以讓車騎將軍音商根兄弟欲自黥劓以謝太后|{
	劓魚器翻又牛倨翻}
上聞之大怒乃使尚書責問司隸校尉京兆尹知成都侯商等奢僭不軌藏匿姦猾皆阿縱不舉奏正灋二人頓首省戶下|{
	司隸校尉察三輔京兆尹治京邑而阿縱不舉奏故責之省戶禁門也}
又賜車騎將軍音策書曰外家何甘樂禍敗|{
	師古曰言此罪過並身自為之余謂言商等奢侈必將得罪何乃甘心為之以為樂也樂音洛}
而欲自黥劓相戮辱於太后前傷慈母之心以危亂國家外家宗族彊上一身寖弱日久今將一施之|{
	師古曰行刑罪}
君其召諸侯令待府舍|{
	諸侯指商根等師古曰令摠集音舍待詔命}
是日詔尚書奏文帝誅將軍薄昭故事|{
	見十四卷文帝前十年}
車騎將軍音藉稾請罪|{
	師古曰自坐稾上言待刑戮也}
商立根皆負斧質謝良久乃已上特欲恐之實無意誅也 秋八月乙卯孝景廟北闕災 初許皇后與斑偼伃皆有寵於上上嘗遊後庭欲與倢伃同輦載|{
	偼伃音接于下同}
偼伃辭曰觀古圖畫賢聖之君皆名臣在側三代末主乃有嬖妾今欲同輦得無近似之乎|{
	師古曰嬖愛也音必計翻又卑義翻近音巨靳翻}
上善其言而止太后聞之喜曰古有樊姬|{
	張晏曰楚王好田樊姬為不食禽獸之肉按樊姬事楚莊王}
今有斑偼伃斑偼伃進侍者李平得幸亦為偼伃賜姓曰衛其後上微行過陽阿主家|{
	師古曰陽阿平原之縣也應劭曰平原漯隂東南五十里有陽阿鄉故縣也 考異曰五行志作河陽主伶玄趙后外傳及荀紀亦作河陽外戚傳顔師古注曰陽阿平原之縣也今俗書阿字作河或為河陽皆後人所妄改耳今從之}
悦歌舞者趙飛燕|{
	師古曰以其體輕故曰飛燕}
召入宮大幸有女弟復召入|{
	復扶又翻}
姿性尤醲粹左右見之皆嘖嘖嗟賞|{
	嘖嘖衆口稱羨而作聲也音側革翻}
有宣帝時披香博士淖方城在帝後|{
	披香博士後宮女職也淖音女教翻姓也}
唾曰此禍水也滅火必矣姊弟俱為偼伃貴傾後宮許皇后斑偼伃皆失寵於是趙飛燕譛告許皇后斑偼伃挟媚道|{
	婦人挟媚道者蠱詛他人求已親媚}
祝詛後宮詈及主上|{
	祝職救翻詛莊助翻詈力智翻}
冬十一月甲寅許后廢處昭臺宮|{
	師古曰宮在上林苑中處昌呂翻}
后姊謁皆誅死親屬歸故郡|{
	后姊謁為平安剛侯夫人許氏本山陽人也}
考問斑偼伃偼伃對曰妾聞死生有命富貴在天|{
	論語載子夏答司馬牛之言}
修正尚未蒙福為邪欲以何望使鬼神有知不受不臣之愬|{
	師古曰祝詛主上是不臣也}
如其無知愬之何益故不為也上善其對赦之賜黄金百斤趙氏姊弟驕妬偼伃恐久見危乃求共養太后於長信宮|{
	師古曰共音居用翻養音弋向翻宮閣記長信殿在長樂宮太后常居之}
上許焉 廣漢男子鄭躬等六十餘人攻官寺簒囚徒盗庫兵自稱山君|{
	廣漢郡高帝分蜀郡置屬益州師古曰逆取曰簒風俗通寺司也諸官府所止皆曰寺}
四年秋勃海清河信都河水湓溢|{
	勃海唐滄景州清河唐貝州信都唐冀州師古曰湓湧也音普頓翻}
灌縣邑三十一敗官亭民舍四萬餘所|{
	敗補邁翻}
平陵李尋奏言議者常欲求索九河故迹而穿之今因其自决可且勿塞以觀水埶|{
	索山客翻塞悉則翻下同}
河欲居之當稍自成川跳出沙土然後順天心而圖之必有成功而用財力寡於是遂止不塞朝臣數言百姓可哀上遣使者處業振贍之|{
	師古曰處業謂安處之使得居業數所角翻處昌呂翻}
廣漢鄭躬黨與濅廣犯歷四縣衆且萬人州郡不能制冬以河東都尉趙護為廣漢太守發郡中及蜀郡合三萬人擊之或相捕斬除罪|{
	師古曰賊黨相捕斬赦其本罪}
旬月平遷護為執金吾賜黄金百斤 是歲平阿安侯王譚薨上悔廢譚使不輔政而薨也乃復進成都侯商|{
	復扶又翻}
以特進領城門兵置幕府得舉吏如將軍|{
	漢制列將軍置幕府得舉吏}
魏郡杜鄴時為郎素善車騎將軍音見音前與平阿侯有隙即說音曰夫戚而不見殊孰能無怨|{
	師古曰戚近也殊謂異於疏也說輸芮翻}
昔秦伯有千乘之國而不能容其母弟|{
	師古曰秦景公母弟公子鍼有寵於其父桓公景公立鍼懼而奔晉事在昭元年故經書秦伯之弟鍼出奔晉傳曰稱弟罪秦伯也}
春秋譏焉周召則不然|{
	師古曰言周公召公無私怨也余謂不然者不為秦伯之為也召讀曰邵}
忠以相輔義以相匡同己之親等已之尊不以聖德獨兼國寵又不為長專受榮任分職於陕並為弼凝|{
	師古曰分職於陕謂自陕以東周公主之自陕以西召公主之陕即今陕州縣也音式冉翻而說者妄云分陕是潁川郟縣謬矣弼凝謂左輔右弼前凝後丞也今按字書陕從兩入郏從兩人人自不考耳為于偽翻長知兩翻}
故内無感恨之隙|{
	師古曰感音胡闇翻}
外無侵侮之羞俱享天祐兩荷高名者盖以此也|{
	荷下可翻}
竊見成都侯以特進領城門兵復有詔得舉吏如五府|{
	丞相御史及車騎左右將軍府也復扶又翻}
此明詔所欲必寵也將軍宜承順聖意加異往時每事凡議必與及之發於至誠則孰不說諭|{
	師古曰言皆出於至誠彼必和說無憂乖異也說讀曰悦}
音甚嘉其言由是與成都侯商親密二人皆重鄴永始元年春正月癸丑太官凌室火|{
	師古曰凌室藏氷之室凌音力證翻又音陵}
戊午戾后園南闕火 |{
	考異曰五行志及荀紀二火皆作災今從漢書}
上欲立趙偼伃為皇后皇太后嫌其所出微甚難之太后姊子淳于長為侍中數往來通語東宮|{
	數所角翻}
歲餘乃得太后指許之夏四月乙亥上先封偼伃父臨為成陽侯|{
	恩澤侯表成陽侯食邑於汝南新息}
諫大夫河間劉輔上書|{
	漢書劉輔河間宗室}
言昔武王周公承順天地以饗魚烏之瑞然猶君臣祗懼動色相戒|{
	今文尚書泰誓曰白魚入于王舟有火復于王屋流為烏周公曰復哉復哉}
况於季世不蒙繼嗣之福屢受威怒之異者虖|{
	威怒謂皇天降威震怒也虖古乎字}
雖夙夜自責改過易行|{
	行下孟翻}
畏天命念祖業妙選有德之世考卜窈窕之女|{
	鄭玄曰考猶稽也師古曰窈窕幽閒也}
以承宗廟順神祗心塞天下望|{
	塞悉則翻}
子孫之祥猶恐晚暮今乃觸情縱欲傾於卑賤之女欲以母天下不畏於天不愧於人惑莫大焉里語曰腐木不可以為柱人婢不可以為主 |{
	考異曰劉輔傳云腐木不可以為柱卑人不可以為主荀紀柱作珪卑人作人婢今柱從漢書人婢從荀紀}
天人之所不予必有禍而無福市道皆共知之|{
	師古曰市道市中之道也一曰市人及行於道路者也予讀曰與}
朝廷莫肯壹言臣竊傷心不敢不盡死書奏上使侍御史收縛輔繫掖庭祕獄|{
	師古曰漢舊儀掖庭詔獄令丞宦者為之主理婦人女宫也}
羣臣莫知其故於是左將軍辛慶忌右將軍亷褒光禄勲琅邪師丹太中大夫谷永|{
	四人皆中朝官}
俱上書曰竊見劉輔前以縣令求見擢為諫大夫|{
	輔以襄賁令上書言得失召見擢諫大夫襄賁東海縣也賁音肥}
此其言必有卓詭切至當聖心者故得拔至於此旬月之間收下秘獄|{
	下遐稼翻}
臣等愚以為輔幸得託公族之親在諫臣之列新從下土來未知朝廷體獨觸忌諱不足深過|{
	過猶罪也}
小罪宜隱忍而已如有大惡宜暴治理官與衆共之|{
	理官謂廷尉也師古曰令衆人知其罪抶而罰之暴顯示也顯示其罪使理官治之}
今天心未豫|{
	張晏曰豫悦豫也}
災異屢降水旱迭臻方當隆寛廣問褒直盡下之時也而行慘急之誅於諫争之臣|{
	争讀曰諍}
震驚羣下失忠直心假令輔不坐直言所坐不著|{
	師古曰著明也}
天下不可戶曉|{
	師古曰言不可家家曉諭之也}
同姓近臣本以言顯其於治親養忠之義|{
	治直之翻}
誠不宜幽囚於掖庭獄公卿以下見陛下進用輔亟而折傷之暴人有懼心精鋭銷耎|{
	師古曰人人皆懼也蘇林曰耎弱也師古曰耎音乃亂翻又乳兖翻}
莫敢盡節正言非所以昭有虞之聽|{
	師古曰舜有敢諫之鼓故言有虞之聽也一曰謂達四聰也}
廣德美之風臣等竊深傷之惟陛下留神省察|{
	省悉井翻}
上乃徙輔繫共工獄|{
	蘇林曰考工也師古曰少府之屬官亦有詔獄共讀與龔同}
減死罪一等論為鬼薪|{
	應劭曰取薪給宗廟為鬼薪三歲刑也}
初太后兄弟八人獨弟曼早死不侯|{
	鳳嗣父爵陽平侯崇安成侯庶弟五人同日封謂之五侯八人之中獨曼不侯}
太后憐之曼寡婦渠供養東宮|{
	供古用翻養余亮翻}
子莽幼孤不及等比|{
	師古曰比音必寐翻余謂當音毗至翻}
其羣兄弟皆將軍五侯子乘時侈靡|{
	師古曰乘因也因富貴之時}
以輿馬聲色佚游相高|{
	師古曰佚與逸同}
莽因折節為恭儉勤身博學|{
	折而設翻}
被服如儒生|{
	師古曰被音皮義翻}
事母及寡嫂養孤兄子行甚敕備|{
	莽兄永早死有子光行下孟翻}
又外交英俊内事諸父曲有禮意大將軍鳳病莽侍疾親嘗藥|{
	鄭玄曰嘗藥度其所堪}
亂首垢面不解衣帶連月鳳且死以託太后及帝拜為黄門郎|{
	漢舊儀曰黄門郎屬黄門令日暮入對青瑣門拜名曰夕郎董巴曰禁門曰黄闥}
遷射聲校尉久之叔父成都侯商上書願分戶邑以封莽長樂少府戴崇|{
	姓譜戴宋戴公之後一曰宋滅戴子孫以國為氏}
侍中金涉中郎陳湯等皆當世名士咸為莽言|{
	為于偽翻下同}
上由是賢莽太后又數以為言|{
	數所角翻}
五月乙未封莽為新都侯|{
	莽傳以南陽新野之都鄉為新都侯國}
遷騎都尉光禄大夫侍中宿衛謹敕爵位益尊節操愈謙散輿馬車裘振施賓客|{
	師古曰振舉也施式智翻}
家無所餘收贍名士交結將相卿大夫甚衆故在位者更推薦之|{
	更工衡翻}
游者為之談說虚譽隆洽傾其諸父矣|{
	隆盛也洽漸浃也周徧也}
敢為激發之行處之不慚恧|{
	師古曰激急動恧愧也激音工歷翻行下孟翻處昌呂翻恧音女六翻}
嘗私買侍婢昆弟或頗聞知莽因曰後將軍朱子元無子|{
	朱博字子元}
莽聞此兒種宜子|{
	師古曰此兒謂所買婢也種章勇翻}
即日以婢奉朱博其匿情求名如此|{
	王莽事始此}
六月丙寅立皇后趙氏大赦天下皇后既立寵少衰而其女弟絶幸為昭儀居昭陽宮其中庭彤朱而殿上髹漆|{
	師古曰以漆漆物謂之髹音許求翻又許昭翻今關東俗器物一再著漆者謂之捎漆捎即髹聲之轉重耳髹字或作䰍音義亦與髹同今關西俗云黑髹盤朱髹盤其音如此兩義並通毛晃曰髹赤黑漆}
切皆銅沓黄金塗|{
	師古曰切門限也音千結翻沓冒其頭也塗以金塗銅上也沓音他合翻}
白玉階|{
	師古曰階所由陞殿陛也}
壁帶往往為黄金缸函藍田璧明珠翠羽飾之|{
	服䖍曰缸壁中之横帶也晉灼曰以金環飾之也師古曰壁帶壁之横木露出如帶者也於壁帶之中往往以金為缸若車缸之形也其缸中著玉璧明珠翠羽耳藍田山名出美玉缸音工流俗讀之音江非也}
自後宫未嘗有焉趙后居别館多通侍郎宫奴多子者|{
	侍郎郎之得出入禁中者宫奴有罪沒為宫奴給使宫中者}
昭儀嘗謂帝曰妾姊性剛有如為人構陷則趙氏無種矣|{
	種章勇翻}
因泣下悽惻帝信之有白后姦狀者帝輒殺之由是后公為淫恣無敢言者然卒無子|{
	卒子恤翻}
光禄大夫劉向以為王教由内及外自近者始|{
	詩大序關雎后妃之德也風之始也所以風天下而正夫婦也故曰正始之道王化之基}
於是採取詩書所載賢妃貞婦興國顯家及孽嬖亂亡者|{
	師古曰孽庶也嬖愛也}
序次為列女傳凡八篇及采傳記行事著新序說苑凡五十篇奏之數上疏言得失陳法戒|{
	數所角翻}
書數十上以助觀覽補遺闕|{
	上時掌翻}
上雖不能盡用然内嘉其言常嗟歎之 昌陵制度奢泰久而不成劉向上疏曰臣聞王者必通三統|{
	應劭曰二王之後與己為三統也孟康曰天地人之始也張晏曰一曰天統謂周以十一月建子為正天始施之端也二曰地統謂殷以十二月建丑為正地始化之端也三曰人統謂夏以十三月建寅為正人始成之端也師古曰諸家之說皆不備也言王者象天地人之三統故存三代也}
明天命所授者博非獨一姓也自古及今未有不亡之國孝文皇帝嘗美石槨之固張釋之曰使其中有可欲雖錮南山猶有隙夫死者無終極而國家有廢興故釋之之言為無窮計也|{
	釋之對詳見十四卷文帝前三年}
孝文寤焉遂薄葬棺槨之作自黄帝始|{
	古之葬者厚衣之以薪葬之中野不封不樹黄帝易之以棺槨}
黄帝堯舜禹湯文武周公丘隴皆小葬具甚微|{
	晋灼曰丘壠冢墳也}
其賢臣孝子亦承命順意而薄葬之此誠奉安君父忠孝之至也孔子葬母於防|{
	師古曰防魯邑名也杜預曰昌邑縣西有防城}
墳四尺|{
	記檀弓曰孔子既得合葬於防曰古者墓而不墳今丘也東西南北之人也不可以不識也於是封之崇四尺師古曰墳者謂積土也春秋緯天子墳高三仞樹以松諸侯半之樹以柏大夫八尺樹以藥草士四尺樹以槐庶人無墳樹以楊柳鄭玄曰孔子盖用士禮}
延陵季子葬其子封墳掩坎其高可隱|{
	孟康曰隱蔽之財可見而已臣瓚曰謂人立可隱肘也師古曰瓚說是也隱音於靳翻}
故仲尼孝子而延陵慈父舜禹忠臣周公弟弟|{
	師古曰弟弟者言弟能順理也上弟音徒計翻}
其葬君親骨肉皆微薄矣非苟為儉誠便於體也秦始皇葬於驪山之阿下錮三泉上崇山墳水銀為江海黄金為鳬鴈珎寶之臧|{
	臧古藏字通下臧椁同}
機械之變棺槨之麗宫館之盛不可勝原|{
	詳見七卷秦始皇三十七年勝音升}
天下苦其役而反之驪山之作未成而周章百萬之師至其下矣|{
	事見七卷秦二世二年}
項籍燔其宫室營宇|{
	事見九卷高帝元年}
牧兒持火照求亡羊失火燒其臧椁自古及今葬未有盛如始皇者也數年之間外被項籍之災|{
	被皮義翻}
内離牧竪之禍|{
	師古曰離遭也}
豈不哀哉是故德彌厚者葬彌薄知愈深者葬愈微無德寡知|{
	知讀曰智下賢知同}
其葬愈厚丘壠愈高宫闕甚麗發掘必速由是觀之明暗之效葬之吉凶昭然可見矣陛下即位躬親節儉始營初陵其制約小天下莫不稱賢明|{
	始營陵見上卷建始二年}
及徙昌陵增庳為高|{
	師古曰庳下也音婢}
積土為山發民墳墓積以萬數營起邑居期日廹卒|{
	師古曰卒讀曰猝}
功費大萬百餘|{
	應劭曰大萬億也大巨也}
死者恨於下生者愁於上臣甚惛焉|{
	師古曰惛謂不了言惑於此事也惛音昏一云惛古閔字憂病也余謂當從後說}
以死者為有知發人之墓其害多矣若其無知又安用大謀之賢知則不說|{
	說讀與悦同下同}
以示衆庶則苦之若苟以說愚夫淫侈之人又何為哉唯陛下上覽明聖之制以為則下觀亡秦之禍以為戒初陵之模宜從公卿大夫之議以息衆庶上感其言初解萬年自詭昌陵三年可成卒不能就|{
	卒子恤翻}
羣臣多言其不便者下有司議|{
	下遐稼翻}
皆曰昌陵因卑為高度便房猶在平地上|{
	漢書音義曰便房藏中便坐也度徒洛翻}
客土之中不保幽冥之靈淺外不固|{
	服䖍曰取它處土以增高為客土}
卒徒工庸以鉅萬數至然脂夜作取土東山且與穀同賈|{
	師古曰賈讀曰價}
作治數年天下徧被其勞|{
	治直之翻被皮義翻}
故陵因天性㨿真土處勢高敞旁近祖考|{
	初陵近渭陵又西近茂陵處昌呂翻近其靳翻}
前又已有十年功緒|{
	師古曰緒謂端次也}
宜還復故陵勿徙民便秋七月詔曰朕執德不固謀不盡下|{
	師古曰言不博謀於羣下}
過聽將作大匠萬年言昌陵三年可成|{
	師古曰過誤也萬年解萬年也}
作治五年中陵司馬殿門内尚未加功|{
	如淳曰陵中有司馬殿門如生時制也臣瓚曰天子之藏壙中無司馬殿門也此謂陵上寢殿及司馬門也時皆未作之故曰尚未加功師古曰中陵陵中正寢也司馬殿門瓚說是也}
天下虛耗百姓罷勞客土疏惡|{
	罷讀曰疲疏音疎}
終不可成朕惟其難|{
	師古曰惟思也}
怛然傷心|{
	怛當割翻驚也懼也悼也不安也}
夫過而不改是謂過矣|{
	師古曰論語載孔子之言故詔引之}
其罷昌陵及故陵勿徙吏民|{
	罷昌陵還故陵而故陵勿起陵邑徙吏民也}
令天下毋有動揺之心 初鄼侯蕭何之子嗣為侯者無子及有罪凡五絶祀高后文帝景帝武帝宣帝思何之功輒以其支庶紹封|{
	蕭何薨子祿嗣薨亡子高后乃封何夫人同為鄼侯小子延為筑陽侯孝文元年罷同更封延為鄼侯薨子遺嗣薨亡子文帝復以遺弟則嗣有罪免景帝二年封則弟嘉為武陽侯薨子勝嗣有罪免武帝元狩中復以鄼戶二千四百封何曾孫慶為鄼侯慶則子也薨子夀成嗣坐罪免宣帝封何玄孫建世為鄼侯凡五紹封}
是歲何七世孫鄼侯獲坐使奴殺人减死完為城旦|{
	獲建世孫也}
先是上詔有司訪求漢初功臣之後久未省録杜業說上曰|{
	先悉薦翻省悉井翻說輸芮翻}
唐虞三代皆封建諸侯以成太平之美是以燕齊之祀與周並傳|{
	太公封於齊至周安王二十三年始為田氏所滅召公封於燕後周而滅}
子繼弟及歷載不墮|{
	師古曰弟繼兄位謂之及載子亥翻墮毁也音火規翻}
豈無刑辟|{
	辟毗亦翻}
繇祖之竭力故支庶賴焉|{
	師古曰言國家非無刑辟而功臣子孫得不䧟罪辜而能長存者思其先人之力令有嗣續也}
迹漢功臣亦皆剖符世爵受山河之誓|{
	高帝封爵之誓曰使黄河如帶泰山若厲國以永存爰及苖裔}
百餘年間而襲封者盡朽骨孤於墓苖裔流於道生為愍隸死為轉屍|{
	應劭曰死不能葬故屍流轉在溝壑之中師古曰愍隸者言為徒隸在可哀愍之中}
以往况今|{
	師古曰况譬也}
甚可悲傷聖朝憐閔詔求其後四方忻忻靡不歸心出入數年而不省察恐議者不思大義徒設虛言則厚德掩息吝簡布章|{
	吝靳也簡略也言既詔求其後復靳而不封畧而不問若如此必布聞於天下也}
非所以示化勸後也雖難盡繼宜從尤功|{
	言漢之功臣絶世者多雖難盡繼宜取功尤重者後紹其國封也}
上納其言癸卯封蕭何六世孫南䜌長喜為鄼侯|{
	地理志南䜌縣屬鉅鹿郡孟康曰䜌音力全翻百官表縣令長皆秦官掌治其縣萬戶以上為令秩千石至六百石减萬戶為長秩五百石至三百石長知兩翻 考異曰成紀元延元年封蕭相國後喜為鄼侯荀胡皆用之按功臣表永始元年釐侯喜紹封三年薨永始四年質侯尊嗣五年薨質侯章嗣蓋本紀誤以永始為元延故也}
立城陽哀王弟俚為王|{
	鴻嘉二年哀王雲薨無後 考異曰漢紀俚作悝今從漢書}
八月丁丑太皇太后王氏崩|{
	師古曰宣帝王皇后也}
九月黑龍見東萊|{
	見賢遍翻}
丁巳晦日有食之 |{
	考異曰荀紀}


|{
	作乙巳按長歷丁巳晦荀說誤}
是歲以南陽太守陳咸為少府侍中淳于長為水衡都尉

二年春正月己丑安陽敬侯王音薨王氏唯音為修整數諫正|{
	數所角翻}
有忠直節 二月癸未夜星隕如雨繹繹未至地滅|{
	師古曰繹繹光采貌}
乙酉晦日有食之 三月丁酉以成都侯商為大司馬衛將軍紅陽侯王立位特進領城門兵 京兆尹翟方進為御史大夫|{
	翟亭歷翻又直格翻}
谷永為凉州刺史奏事京師訖當之部|{
	凉州部隴西天水武都金城安定北地武威張掖燉煌酒泉等郡漢制諸州刺史常以八月廵行所部録囚徒考殿最歲盡詣京師奏事}
上使尚書問永受所欲言|{
	師古曰永冇所言令尚書即受之}
永對曰臣聞王天下有國家者|{
	王于况翻}
患在上有危亡之事而危亡之言不得上聞如使危亡之言輒上聞|{
	師古曰如若也冇即上聞}
則商周不易姓而迭興三正不變改而更用|{
	更工衡翻}
夏商之將亡也行道之人皆知之|{
	師古曰凡在道路行者也}
晏然自以若有天日莫能危|{
	尚書大傳曰桀云天之有日猶吾之有民日冇亡哉日亡吾亦亡矣師古曰自謂如日在天而無有能傷危也}
是故惡日廣而不自知大命傾而不寤易曰危者有其安者也亡者保其存者也|{
	師古曰易下繫之辭也言安必思危存不忘亡乃得保其安存}
陛下誠垂寛明之聽無忌諱之誅使芻蕘之臣得盡所聞於前|{
	刈草曰芻采薪曰蕘文王詢于芻蕘}
羣臣之上願社稷之長福也元年九月黑龍見|{
	見賢遍翻}
其晦日有食之今年二月己未夜星隕乙酉日有食之|{
	己當作癸此承谷永傳之誤}
六月之間大異四發二二而同月三代之末春秋之亂未嘗有也臣聞三代所以隕社稷喪宗廟者皆由婦人與羣惡沈湎於酒|{
	喪息浪翻沈持林翻}
秦所以二世十六年而亡者|{
	秦始皇二十六年初幷天下三十七年崩二世三年而亡其有天下財十六年}
養生泰奢奉終泰厚也二者陛下兼而有之臣請略陳其效建始河平之際許斑之貴傾動前朝|{
	師古曰許皇后及斑偼伃之家朝直遥翻}
熏灼四方女寵至極不可上矣|{
	師古曰上猶加也}
今之後起十倍于前|{
	如淳曰謂趙李本從微賤起也}
廢先帝法度聽用其言官秩不當縱釋王誅|{
	師古曰縱放也釋解也王誅謂王法當誅者當丁浪翻}
驕其親屬假之威權從横亂政|{
	師古曰從音子用翻横音胡孟翻}
刺舉之吏莫敢奉憲又以掖庭獄大為亂阱|{
	師古曰阱穿也為阬阱以拘繫人也亂者言其非正而又多也阱音才性翻仲馮曰言設獄陷人如阱耳余謂仲說是}
榜箠㿊於炮烙|{
	師古曰㿊痛也炮烙紂所作刑也膏塗銅柱加之火上令罪人行其上輒墮炭中笑而以為樂㿊音千感翻}
絶滅人命主為趙李報德復怨|{
	師古曰復亦報也為于偽翻}
反除白罪建治正吏|{
	師古曰反讀曰幡罪之明白者反而除之吏之公正者建議劾治也}
多繫無辜掠立廹恐|{
	師古曰掠笞服之立其罪名}
至為人起責分利受謝|{
	師古曰言富賈有錢假託其名代之為主放與它人以取利息而共分之或受報謝别取財物為于偽翻}
生入死出者不可勝數|{
	勝音升}
是以日食再既以昭其辜|{
	孟康曰既盡也師古曰昭明也}
王者必先自絶然後天絶之今陛下棄萬乘之至貴樂家人之賤事|{
	師古曰謂私畜田及奴婢財物樂音洛}
厭高美之尊號好匹夫之卑字|{
	孟康曰成帝好微行更作私字以相呼如淳曰稱張放家人為卑字好呼到翻}
崇聚僄輕無義小人以為私客|{
	師古曰僄疾也音頻妙翻又匹妙翻}
數離深宫之固|{
	數所角翻離力智翻}
挺身晨夜與羣小相隨|{
	師古曰挺引也音大鼎翻}
烏集雜會醉飽吏民之家|{
	師古曰言聚散不常如烏鳥之集}
亂服共坐沈湎媟嫚溷淆無别黽勉遁樂|{
	師古曰黽勉言不息也遁流遁也言流遁為樂也沈持林翻樂音洛}
晝夜在路典門戶奉宿衛之臣執干戈而守空宫公卿百僚不知陛下所在積數年矣王者以民為基民以財為本財竭則下畔下畔則上亡是以明王愛養基本不敢窮極使民如承大祭|{
	論語孔子答仲弓之言師古曰言常畏慎}
今陛下輕奪民財不愛民力聽邪臣之計去高敞初陵改作昌陵役百乾谿費擬驪山|{
	楚靈王侈心無厭民不堪其役潰於乾谿王縊而死驪山事見秦紀師古曰擬比也言勞役之工百倍於楚靈王費財之廣比於秦始皇杜預曰乾谿在譙國城父縣南乾音干}
靡敝天下|{
	師古曰靡音武皮翻}
五年不成而後反故百姓愁恨感天饑饉仍臻|{
	師古曰仍頻也}
流散宂食餧死於道以百萬數|{
	師古曰宂亦散也餧餓也冗音人勇翻餧音乃賄翻}
公家無一年之畜|{
	師古曰畜讀曰蓄}
百姓無旬月之儲上下俱匱無以相救詩云殷鑒不遠在夏后之世|{
	師古曰大雅蕩之篇也}
願陛下追觀夏商周秦所以失之以鏡考已行|{
	師古曰鏡謂鑒照之考校也行下孟翻}
有不合者臣當伏妄言之誅|{
	師古曰言上所為違於節儉皆與永言同余謂此言帝之失行與夏殷周秦所以失者合耳}
漢興九世百九十餘載|{
	載子亥翻}
繼體之主七皆承天順道遵先祖灋度或以中興或以治安|{
	治直吏翻}
至於陛下獨違道縱欲輕身妄行當盛壮之隆無繼嗣之福有危亡之憂積失君道不合天意亦以多矣為人後嗣守人功業如此豈不負哉方今社稷宗廟禍福安危之機在於陛下陛下誠能昭然遠寤專心反道|{
	師古曰反猶還也}
舊愆畢改新德既章則赫赫大異庶幾可銷天命去就庶幾可復|{
	師古曰去就言去離無德而就有德}
社稷宗廟庶幾可保唯陛下留神反覆孰省臣言帝性寛好文辭而溺於宴樂|{
	省悉井翻好呼到翻樂音洛}
皆皇太后與諸舅夙夜所常憂至親難數言|{
	數所角翻}
故推永等使因天變而切諫勸上納用之永自知有内應展意無所依違|{
	師古曰展申也}
每言事輒見答禮|{
	師古曰如禮而答之余謂答禮者答之而又加禮也}
至上此對|{
	上時掌翻}
上大怒衛將軍商密擿永令發去|{
	師古曰擿謂發動之}
上使侍御史收永敕過交道廐者勿追|{
	晋灼曰交道廐去長安六十里近延陵}
御史不及永還上意亦解自悔|{
	悔遣侍御史收永也}
上嘗與張放及趙李諸侍中共宴飲禁中皆引滿舉白|{
	服䖍曰舉滿桮有餘白瀝者罸之也孟康曰舉白見驗飲酒盡不也師古曰謂引取滿觴而飲飲訖舉觴告白盡不也一說白者罸爵之名飲有不盡者則以此爵罰之魏文侯與大夫飲酒令曰不釂者浮以大白於是公乘不仁舉白浮君是也釂子省翻飲酒盡爵也}
談笑大噱|{
	師古曰噱笑聲也音其略翻或曰噱謂唇口之中大笑則見此說非}
時乘輿幄坐張畫屏風|{
	乘繩證翻師古曰坐音材卧翻畫占畫字通下同}
畫紂醉踞妲己作長夜之樂|{
	妲當割翻妲己有蘇氏之女樂音洛}
侍中光禄大夫班伯久疾新起|{
	姓譜班楚令尹鬬班之後斑書叙傳自以為楚令尹子文之後子文初生棄於夢中而虎乳之楚人謂乳為穀謂虎為於菟故名穀於菟楚人謂虎班其子以為號師古注曰子文之子鬪班亦為楚令尹余按左傳莊三十年申公鬪班殺令尹子元鬬穀於菟為令尹恐班非子文之子}
上顧指畫而問伯曰紂為無道至於是虖|{
	虖古乎字}
對曰書云乃用婦人之言|{
	師古曰今文尚書泰誓之辭}
何有踞肆於朝|{
	師古曰肆放也陳也朝直遥翻}
所謂衆惡歸之不如是之甚者也|{
	師古曰論語稱子貢曰紂之不善不如是之甚也是以君子惡居下流天下之惡皆歸焉}
上曰苟不若此此圖何戒對曰沈湎于酒微子所以告去也|{
	孔穎達云酒誥注云飲酒齊色曰湎然則湎者顔色湎然齊一之辭師古曰微子殷之卿士封於微爵稱子也殷紂錯亂天命微子作誥告箕子比干而去其誥曰用沈酗于酒用亂敗厥德于下我其發出狂吾家耄遜于荒事見尚書微子篇}
式號式謼大雅所以流連也|{
	師古曰大雅蕩之詩曰式號式謼俾晝作夜言醉酒號呼以晝為夜也流連言作詩之人嗟歎而泣涕流連也而說者乃以流連為荒亡盖失之矣大雅所以流連不謂飲酒之人也謼音火故翻}
詩書淫亂之戒其原皆在於酒上乃喟然歎曰吾久不見班生今日復聞讜言|{
	復扶又翻師古曰讜言善言也讜音黨}
放等不懌|{
	師古曰懌悦也音亦}
稍自引起更衣|{
	更工衡翻}
因罷出時長信庭林表適使來聞見之|{
	孟康曰長信太后宫名也庭林表宫中婦人官名也師古曰長信宮庭之林表也林表官名耳庭非官稱也使疏吏翻}
後上朝東宫|{
	朝直遥翻}
太后泣曰帝閒顔色瘦黑|{
	師古曰閒謂比日也}
班侍中本大將軍所舉|{
	大將軍謂王鳳也}
宜寵異之益求其比以輔聖德|{
	鳳初薦伯宜勸學召見親近今太后以其能諫正欲令帝寵異之也師古曰比類也音必寐翻當如字}
宜遣富平侯且就國|{
	富平侯張放}
上曰諾上諸舅聞之以風丞相御史|{
	師古曰風讀曰諷}
求放過失於是丞相宣御史大夫方進奏放驕蹇縱恣奢淫不制拒閉使者|{
	侍御史修奉使至放家逐名捕賊奴從者閉門設弓弩距使者不肯内}
賊傷無辜|{
	放知李游君欲獻女求不得使奴康等之其家賊傷三人}
從者支屬並乘權勢為暴虐|{
	從才用翻}
請免放就國 |{
	考異曰叙傳云王音以風丞相御史按放傳丞相宣御史大夫方進奏放過惡音以正月乙巳薨方進以三月丁酉為御史大夫然則風丞相御史者疑非音也放傳又云上諸舅皆害其寵故但云上諸舅}
上不得已|{
	師古曰已止也}
左遷放為北地都尉其後比年數有災變|{
	師古曰比頻也比毗至翻數所角翻}
故放久不得還璽書勞問不絶|{
	璽斯氏翻勞力到翻}
敬武公主有疾詔徵放歸第視母疾數月主有瘳後復出放為河東都尉|{
	復扶又翻}
上雖愛放然上廹太后下用大臣故常涕泣而遣之 卭成太后之崩也|{
	卭成太后孝宣王皇后也父奉光封卭成侯故書卭成太后以别孝元王皇后恩澤侯表卭成侯國於濟隂}
喪事倉卒吏賦歛以趨辦|{
	卒讀曰猝歛力贍翻師古曰趨讀曰趣言苟取辦趣與促同}
上聞之以過丞相御史|{
	過罪也}
冬十一月己丑策免丞相宣為庶人御史大夫方進左遷執金吾二十餘日丞相官缺羣臣多舉方進者上亦器其能十一月壬子擢方進為丞相封高陵侯|{
	恩澤侯表高陵侯國于琅邪考異曰方進傳丞相薛宣免方進亦左遷執金吾二十餘日遂擢為丞相而荀紀云秋八月方進貶為執金吾盖以公卿表云三月丁酉京兆尹方進為御史大夫八月貶為執金吾故致此誤也按公卿表所云者謂方進自三月為御史大夫至十一月而貶凡居官八月耳又黑龍見東萊在去年九月谷永傳著之甚明而荀悦亦載之于此年云冬黑龍見東萊盖因陳湯獲罪在今年故也漢春秋雖正黑龍之誤而方進貶官猶承荀悦之失}
以諸吏散騎光禄勲孔光為御史大夫|{
	散悉亶翻}
方進以經術進|{
	方進以射策甲科為郎舉明經遷議郎}
其為吏用灋刻深好任勢立威有所忌惡峻文深詆中傷甚多|{
	好呼到翻惡烏路翻中竹仲翻}
有言其挟私詆欺不專平者上以方進所舉應科不以為非也|{
	科律條也}
光褒成君霸之少子也|{
	霸見二十八卷元帝永光元年}
領尚書典樞機十餘年守法度修故事上有所問據經灋以心所安而對不希指苟合|{
	師古曰希指希望天子之意指也}
如或不從不敢強諫争|{
	争讀曰諍}
以是久而安時有所言輒削草藳|{
	服䖍曰言己繕書更削壞其草也}
以為章主之過以奸忠直人臣大罪也|{
	師古曰奸求也奸忠直之名也奸音干}
有所薦舉惟恐其人之聞知沐日歸休兄弟妻子燕語終不及朝省政事|{
	朝直遥翻}
或問光温室省中樹皆何木也光嘿不應更答以他語其不泄如是 上行幸雍祠五畤|{
	建始二年罷雍五畤今以久無繼嗣并甘泉秦畤皆復之雍於用翻畤音止}
衛將軍王商惡陳湯奏湯妄言昌陵且復發徙|{
	初湯請起昌陵邑既罷昌陵丞相御史請廢昌陵邑中室奏未下人以問湯第宅不徹得無復發徙湯曰縣官且順聽羣臣言猶復發徙之也惡烏路翻}
又言黑龍冬出微行數出之應|{
	東萊郡黑龍出人以問湯曰是所謂玄門開微行數出出入不時故龍以非時出也數所角翻}
廷尉奏湯非所宜言大不敬詔以湯有功|{
	有斬郅支功}
免為庶人徙邉上以趙后之立也淳于長有力焉故德之乃追顯其前白罷昌陵之功下公卿議封長|{
	下遐稼翻}
光禄勲平當以為長雖有善言不應封爵之科|{
	姓譜平齊相晏平仲之後一曰韓哀侯少子婼食采平邑因以為氏高祖之法非有功不侯}
當坐左遷鉅鹿太守上遂下詔以常侍閎衛尉長首建至策|{
	師古曰閎王閎也}
賜長閎爵關内侯將作大匠萬年佞邪不忠毒流衆庶與陳湯俱徙燉煌|{
	燉徒門翻}
初少府陳咸衛尉逢信官簿皆在翟方進之右|{
	逢皮江翻姓也古有逢蒙師古曰簿謂伐閲也簿音主簿之簿}
方進晚進為京兆尹與咸厚善及御史大夫缺三人皆名卿俱在選中而方進得之會丞相薛宣得罪與方進相連上使五二千石雜問丞相御史|{
	晉灼曰大臣獄重故以秩二千石者五人詰責之}
咸詰責方進冀得其處方進心恨|{
	詰去吉翻}
陳湯素以材能得幸於王鳳及王音咸信皆與湯善湯數稱之於鳳音所|{
	數所角翻}
以此得為九卿及王商黜逐湯方進因奏咸信附會湯以求薦舉苟得無恥皆免官|{
	考異曰咸信免官皆在明年以後因陳湯事連言之}
是歲琅邪太守朱博為左馮翊博治郡常令縣屬各用其豪桀以為大吏文武從宜|{
	師古曰各因其材而任之治直之翻}
縣有劇賊及它非常博輒移書以詭責之其盡力有效必加厚賞懷詐不稱誅罰輒行|{
	師古曰稱副也稱尺證翻}
以是豪強懾服事無不集|{
	懾之涉翻}


三年春正月己卯晦日有食之 初帝用匡衡議罷甘泉泰畤|{
	事見上卷建始元年}
其日大風壞甘泉竹宫|{
	武帝以正月上辛有事甘泉圜丘自竹宫而望拜韋昭曰以竹為宫天子居中師古曰漢舊儀竹宫去壇三里壞音怪}
折拔畤中樹木十圍以上百餘|{
	折而設翻}
帝異之以問劉向對曰家人尚不欲絶種祠|{
	師古曰家人謂庶人之家也種祠繼嗣所傳祠也}
况於國之神寶舊畤且甘泉汾隂及雍五畤始立皆有神祇感應然後營之非苟而已也|{
	武帝祠泰一於甘泉夜常有神光如流星集于祠壇汾隂男子公孫滂洋等見汾旁有光如絳上遂立后土祠於汾隂脽上文帝十四年黄龍見成紀始幸雍郊見五畤}
武宣之世奉此三神禮敬敕備神光尤著祖宗所立神祇舊位誠未易動|{
	易以豉翻}
前始納貢禹之議後人相因多所動揺|{
	元帝時貢禹建言漢家祭祀多不應古禮韋玄成匡衡等因之}
易大傳曰誣神者殃及三世恐其咎不獨止禹等上意恨之|{
	師古曰恨悔也}
又以久無繼嗣冬十月庚辰上白太后令詔有司復甘泉泰畤汾隂后土如故及雍五畤陳寶祠長安及郡國祠著明者皆復之是時上以無繼嗣頗好鬼神方術之屬|{
	好呼到翻}
上書言祭祀方術得待詔者甚衆祠祭費用頗多谷永說上曰|{
	說輸芮翻}
臣聞明於天地之性不可惑以神怪知萬物之情不可罔以非類|{
	師古曰罔猶蔽余謂罔欺也欺人以所無者罔}
諸背仁義之正道|{
	背蒲妺翻}
不遵五經之灋言而盛稱奇怪鬼神廣崇祭祀之方求報無福之祠及言世有僊人服食不終之藥遥興輕舉|{
	如淳曰遥遠也興舉也師古曰興起也謂起而遠去也}
黄冶變化之術者|{
	晋灼曰黄者鑄黄金也道家言冶丹沙令變化可鑄作黄金也}
皆姦人惑衆挟左道懷詐偽以欺罔世主|{
	師古曰左道邪僻之道非正義也王制曰執左道以亂政者殺}
聽其言洋洋滿耳若將可遇|{
	師古曰洋洋美盛之貌洋音羊又音祥}
求之盪盪如係風捕景終不可得|{
	師古曰盪盪空曠之貌也盪音蕩景影也}
是以明王距而不聽聖人絶而不語|{
	師古曰謂孔子不語怪神}
昔秦始皇使徐福發男女入海求神采藥因逃不還天下怨恨|{
	事見秦始皇紀}
漢興新垣平|{
	事見文帝紀}
齊人少翁公孫卿欒大等|{
	事見武帝紀}
皆以術窮詐得誅夷伏辜|{
	師古曰詐得謂主上得其詐偽之情}
唯陛下距絶此類毋令姦人有以窺朝者|{
	朝直遥翻}
上善其言 十一月尉氏男子樊並等十三人謀反|{
	地理志尉氏縣屬陳留郡應劭曰古獄官曰尉氏鄭之别獄也臣瓚曰鄭大夫尉氏之邑故遂以為邑名師古曰鄭大夫尉氏亦以掌獄之官故為侯耳應說是也}
殺陳留太守刼略吏民自稱將軍徒李譚稱忠鍾祖訾順共殺並以聞皆封為侯|{
	姓譜稱平聲漢功臣表有新山侯稱忠楚有鍾儀鍾建又有知音鍾子期訾即移翻何氏姓苑云今齊是本姓祭氏譚延鄉侯忠新山侯祖童鄉侯順樓虛侯 考異曰本紀云五人而功臣表止有四人盖紀誤}
十二月山陽鐵官徒蘇令等二百二十八人攻殺長吏盗庫兵自稱將軍|{
	地理志山陽郡有鐵官}
經郡國十九殺東郡太守及汝南都尉汝南太守嚴訢捕斬令等遷訢為大司農|{
	師古曰訢與欣同}
故南昌尉九江梅福上書曰|{
	地理志南昌縣屬豫章郡後漢志尉主盗賊凡有賊發主名不立則推索行尋案察姦宄以起端緒}
昔高祖納善若不及從諫如轉圜|{
	師古曰不及恐失之也轉圜者言其順易也}
聽言不求其能舉功不考其素|{
	師古曰直取其功不論其舊行及所從來也}
陳平起於亡命而為謀主韓信拔於行陳而建上將|{
	事並見高帝紀行戶剛翻陳讀曰陣}
故天下之士雲合歸漢|{
	師古曰言四面而至}
争進奇異知者竭其策|{
	知讀曰智下同}
愚者盡其慮勇士極其節怯夫勉其死合天下之知并天下之威是以舉秦如鴻毛取楚若拾遺|{
	師古曰鴻毛諭輕拾遺言其易也}
此高祖所以無敵於天下也孝武皇帝好忠諫說至言|{
	好呼到翻說讀曰悦}
出爵不待亷茂|{
	亷茂孝亷秀才也光武諱秀改為茂才}
慶賜不須顯功|{
	師古曰謂諫争合意即得爵賜不由薦舉及軍功也亷亷吏也茂茂材也}
是以天下布衣各厲志竭精以赴闕庭自衒鬻者不可勝數|{
	師古曰衒行賣也鬻亦賣也衒音州縣之縣又音工縣翻勝音升}
漢家得賢於此為盛使孝武皇帝聽用其計升平可致|{
	張晏曰民有三年之儲曰升平}
於是積尸暴骨快心胡越故淮南王安緣間而起|{
	間古莧翻下同}
所以計慮不成而謀議泄者以衆賢聚於本朝故其大臣埶陵不敢和從也|{
	事見武紀師古曰本朝謂漢朝也大臣謂淮南相内史之屬也服䖍曰臣埶陵君和戶卧翻}
方今布衣乃窺國家之隙見間而起者蜀郡是也|{
	孟康曰鴻嘉中廣漢男子鄭躬等反是也}
及山陽亡徒蘇令之羣蹈藉名都大郡|{
	賢曰前書曰十二萬戶為大郡}
求黨與索隨和而無逃匿之意|{
	李奇曰求索與己和及隨己者原父曰漢氏世寶隨和珠玉謂匹夫至欲索此物所謂與上争衡也索山谷翻}
此皆輕量大臣無所畏忌|{
	量音良}
國家之權輕故匹夫欲與上争衡也士者國之重器得士則重失士則輕詩云濟濟多士文王以寧|{
	師古曰詩大雅文王之詩也濟濟盛貌也言文王能多用賢人故邦國得以安寧也濟子禮翻}
廟堂之議非草茅所言也|{
	漢書所字下有當字}
臣誠恐身塗野草尸并卒伍故數上書求見輒報罷|{
	福去南昌歸夀春數因縣道上書求假軺傳詣行在所條對急政輒報罷數音所角翻見賢遍翻}
臣聞齊桓之時有以九九見者桓公不逆欲以致大也今臣所言非特九九也陛下距臣者三矣此天下士所以不至也昔秦武王好力任鄙叩關自鬻|{
	事見三卷周赧王七年周禮司關凡四方之賓客叩關者則為之告注曰叩關謂謁關人也疏曰叩猶至也好呼到翻}
繆公行霸由余歸德|{
	秦繆公開霸業由余自西戎歸之繆讀曰穆}
今欲致天下之士民有上書求見者輒使詣尚書問其所言言可采取者秩以升斗之禄賜以一束之帛若此則天下之士發憤懣|{
	懣音悶}
吐忠言嘉謀日聞於上天下條貫國家表裏爛然可覩矣|{
	師古曰爛然分明之貌也}
夫以四海之廣士民之數|{
	數趨玉翻}
能言之類至衆多也然其雋桀指世陳政言成文章質之先世而不繆施之當世合時務若此者亦無幾人|{
	師古曰無幾言不多也幾音居豈翻}
故爵禄束帛者天下之砥石高祖所以厲世摩鈍也|{
	師古曰砥細石也音之履翻又音祇}
孔子曰工欲善其事必先利其器|{
	師古曰論語載孔子之言也工以諭國政利器諭賢材}
至秦則不然張誹謗之罔以為漢除|{
	為于偽翻}
倒持泰阿授楚其柄|{
	師古曰太阿劒名歐冶所鑄也言秦無道令陳涉項羽乘間而發譬倒持劒以把授人也}
故誠能勿失其柄天下雖有不順莫敢觸其鋒此孝武皇帝所以辟地建功為漢世宗也|{
	師古曰辟讀曰闢}
今陛下既不納天下之言又加戮焉夫鳶鵲遭害則仁鳥增逝|{
	師古曰鳶鴟也仁鳥鸞鳳也鳶音緣}
愚者蒙戮則智士深退間者愚民上書多觸不急之灋|{
	師古曰言以其所言為不急而罪之也}
或下廷尉而死者衆|{
	下遐稼翻下同}
自陽朔以來天下以言為諱朝廷尤甚|{
	懲王章之死也師古曰防人之口法禁嚴切也}
羣臣皆承順上指莫有執正何以明其然也取民所上書陛下之所善試下之廷尉廷尉必曰非所宜言大不敬以此卜之一矣故京兆尹王章資質忠直敢面引廷争|{
	争讀曰諍}
孝元皇帝擢之以厲具臣而矯曲朝|{
	元帝初擢章為左曹中郎將師古曰具臣具位之臣無益者也矯正也朝直遥翻}
及至陛下戮及妻子|{
	事見上卷陽朔元年}
且惡惡止其身|{
	公羊傳惡惡止其身善善及子孫}
王章非有反畔之辜而殃及室家|{
	言王章妻子坐徙也孔穎達曰左傳曰男有室女有家謂男處妻之室女安夫之家夫婦共為家室故為夫婦家室之道為室家也}
折直士之節|{
	折而設翻}
結諫臣之舌羣臣皆知其非然不敢争天下以言為戒最國家之大患也願陛下循高祖之軌杜亡秦之路除不急之灋下無諱之詔博覽兼聽謀及疏賤令深者不諱遠者不塞所謂辟四門明四目也|{
	師古曰虞書舜典曰闢四門明四目言開四門以致衆賢則明視於四方也塞悉則翻辟讀曰闢}
往者不可及來者猶可追方今君命犯而主威奪|{
	師古曰君命犯者謂大臣犯君之命}
外戚之權日以益隆陛下不見其形願察其景建始以來日食地震以率言之三倍春秋水災無與比數|{
	師古曰言其極多不可比校而數也亡讀曰無}
隂盛陽微金鐵為飛此何景也|{
	張晏曰河平二年沛郡鐵官鑄鐵如星飛上去權臣用事之異也蘇林曰言之不從是為不又則金不從革景象也何象言將危亡也為于偽翻}
漢興以來社稷三危呂霍上官皆母后之家也親親之道全之為右|{
	師古曰務全安之此為上}
當與之賢師良傅教以忠孝之道今乃尊寵其位授以魁柄|{
	師古曰以斗為諭也斗身為魁}
使之驕逆至於夷滅|{
	師古曰夷平也謂平除之}
此失親親之大者也自霍光之賢不能為子孫慮故權臣易世則危書曰毋若火始庸庸|{
	師古曰周書洛誥之辭也庸庸微小貌也言火始微小不早撲滅則至熾盛大臣貴擅亦當早圖黜其權也}
埶陵於君權隆於主然後防之亦無極已|{
	師古曰已語終辭}
上不納

資治通鑑卷三十一
