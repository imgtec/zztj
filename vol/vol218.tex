<!DOCTYPE html PUBLIC "-//W3C//DTD XHTML 1.0 Transitional//EN" "http://www.w3.org/TR/xhtml1/DTD/xhtml1-transitional.dtd">
<html xmlns="http://www.w3.org/1999/xhtml">
<head>
<meta http-equiv="Content-Type" content="text/html; charset=utf-8" />
<meta http-equiv="X-UA-Compatible" content="IE=Edge,chrome=1">
<title>資治通鑒_219-資治通鑑卷二百十八_219-資治通鑑卷二百十八</title>
<meta name="Keywords" content="資治通鑒_219-資治通鑑卷二百十八_219-資治通鑑卷二百十八">
<meta name="Description" content="資治通鑒_219-資治通鑑卷二百十八_219-資治通鑑卷二百十八">
<meta http-equiv="Cache-Control" content="no-transform" />
<meta http-equiv="Cache-Control" content="no-siteapp" />
<link href="/img/style.css" rel="stylesheet" type="text/css" />
<script src="/img/m.js?2020"></script> 
</head>
<body>
 <div class="ClassNavi">
<a  href="/24shi/">二十四史</a> | <a href="/SiKuQuanShu/">四库全书</a> | <a href="http://www.guoxuedashi.com/gjtsjc/"><font  color="#FF0000">古今图书集成</font></a> | <a href="/renwu/">历史人物</a> | <a href="/ShuoWenJieZi/"><font  color="#FF0000">说文解字</a></font> | <a href="/chengyu/">成语词典</a> | <a  target="_blank"  href="http://www.guoxuedashi.com/jgwhj/"><font  color="#FF0000">甲骨文合集</font></a> | <a href="/yzjwjc/"><font  color="#FF0000">殷周金文集成</font></a> | <a href="/xiangxingzi/"><font color="#0000FF">象形字典</font></a> | <a href="/13jing/"><font  color="#FF0000">十三经索引</font></a> | <a href="/zixing/"><font  color="#FF0000">字体转换器</font></a> | <a href="/zidian/xz/"><font color="#0000FF">篆书识别</font></a> | <a href="/jinfanyi/">近义反义词</a> | <a href="/duilian/">对联大全</a> | <a href="/jiapu/"><font  color="#0000FF">家谱族谱查询</font></a> | <a href="http://www.guoxuemi.com/hafo/" target="_blank" ><font color="#FF0000">哈佛古籍</font></a> 
</div>

 <!-- 头部导航开始 -->
<div class="w1180 head clearfix">
  <div class="head_logo l"><a title="国学大师官网" href="http://www.guoxuedashi.com" target="_blank"></a></div>
  <div class="head_sr l">
  <div id="head1">
  
  <a href="http://www.guoxuedashi.com/zidian/bujian/" target="_blank" ><img src="http://www.guoxuedashi.com/img/top1.gif" width="88" height="60" border="0" title="部件查字,支持20万汉字"></a>


<a href="http://www.guoxuedashi.com/help/yingpan.php" target="_blank"><img src="http://www.guoxuedashi.com/img/top230.gif" width="600" height="62" border="0" ></a>


  </div>
  <div id="head3"><a href="javascript:" onClick="javascript:window.external.AddFavorite(window.location.href,document.title);">添加收藏</a>
  <br><a href="/help/setie.php">搜索引擎</a>
  <br><a href="/help/zanzhu.php">赞助本站</a></div>
  <div id="head2">
 <a href="http://www.guoxuemi.com/" target="_blank"><img src="http://www.guoxuedashi.com/img/guoxuemi.gif" width="95" height="62" border="0" style="margin-left:2px;" title="国学迷"></a>
  

  </div>
</div>
  <div class="clear"></div>
  <div class="head_nav">
  <p><a href="/">首页</a> | <a href="/ShuKu/">国学书库</a> | <a href="/guji/">影印古籍</a> | <a href="/shici/">诗词宝典</a> | <a   href="/SiKuQuanShu/gxjx.php">精选</a> <b>|</b> <a href="/zidian/">汉语字典</a> | <a href="/hydcd/">汉语词典</a> | <a href="http://www.guoxuedashi.com/zidian/bujian/"><font  color="#CC0066">部件查字</font></a> | <a href="http://www.sfds.cn/"><font  color="#CC0066">书法大师</font></a> | <a href="/jgwhj/">甲骨文</a> <b>|</b> <a href="/b/4/"><font  color="#CC0066">解密</font></a> | <a href="/renwu/">历史人物</a> | <a href="/diangu/">历史典故</a> | <a href="/xingshi/">姓氏</a> | <a href="/minzu/">民族</a> <b>|</b> <a href="/mz/"><font  color="#CC0066">世界名著</font></a> | <a href="/download/">软件下载</a>
</p>
<p><a href="/b/"><font  color="#CC0066">历史</font></a> | <a href="http://skqs.guoxuedashi.com/" target="_blank">四库全书</a> |  <a href="http://www.guoxuedashi.com/search/" target="_blank"><font  color="#CC0066">全文检索</font></a> | <a href="http://www.guoxuedashi.com/shumu/">古籍书目</a> | <a   href="/24shi/">正史</a> <b>|</b> <a href="/chengyu/">成语词典</a> | <a href="/kangxi/" title="康熙字典">康熙字典</a> | <a href="/ShuoWenJieZi/">说文解字</a> | <a href="/zixing/yanbian/">字形演变</a> | <a href="/yzjwjc/">金 文</a> <b>|</b>  <a href="/shijian/nian-hao/">年号</a> | <a href="/diming/">历史地名</a> | <a href="/shijian/">历史事件</a> | <a href="/guanzhi/">官职</a> | <a href="/lishi/">知识</a> <b>|</b> <a href="/zhongyi/">中医中药</a> | <a href="http://www.guoxuedashi.com/forum/">留言反馈</a>
</p>
  </div>
</div>
<!-- 头部导航END --> 
<!-- 内容区开始 --> 
<div class="w1180 clearfix">
  <div class="info l">
   
<div class="clearfix" style="background:#f5faff;">
<script src='http://www.guoxuedashi.com/img/headersou.js'></script>

</div>
  <div class="info_tree"><a href="http://www.guoxuedashi.com">首页</a> > <a href="/SiKuQuanShu/fanti/">四库全书</a>
 > <h1>资治通鉴</h1> <!--         下载:【右键另存为】即可 --></div>
  <div class="info_content zj clearfix">
  
<div class="info_txt clearfix" id="show">
<center style="font-size:24px;">219-資治通鑑卷二百十八</center>
    資治通鑑卷二百十八  宋 司馬光 撰<br />
<br />
  胡三省 音註<br />
<br />
  唐紀三十四【起柔兆涒灘五月至九月不滿一年】<br />
<br />
  肅宗文明武德大聖大宣孝皇帝上之下<br />
<br />
  至德元載【載祖亥翻】五月丁巳炅衆潰走保南陽【炅火迥翻炅不書姓承上卷安祿山將攻魯炅事也炅自穎川走保南陽考異曰玄宗實錄云炅攜百姓數千人奔順陽川今從舊傳】賊就圍之太常卿張垍薦夷陵太守虢王巨有勇略上徵吳王祇為太僕卿【垍其冀翻夷陵郡峽州守式又翻上亦謂玄宗自靈武即位後玄宗稱上皇稱肅宗為上】以巨為陳留譙郡太守河南節度使兼統嶺南節度使何履光【陳留郡汴州譙郡亳州此二郡太守也是年升五府經略討擊使為嶺南節度使領廣韶循潮康瀧端新封春勤羅潘高思雷崖瓊振儋萬安軍二十二州治廣州】黔中節度使趙國珍【趙國珍牂柯别部充州蠻酋趙君道之裔楊國忠兼劒南節度以國珍有方略授黔中都督護五溪十餘年天下方亂其所部獨寜按新書方鎮表開元二十六年黔州置五溪諸州經略使天寶十四載增領守捉使代宗大歷四年始置辰溪巫錦業五州都圍練守捉觀察處置使憲宗元和三年黔州觀察增領涪州唐末始于黔州置節鎮疑此時趙國珍未得建節至明年通鑑書置黔中節度必有所據】南陽節度使魯炅國珍本牂柯夷也【牂音臧柯音哥】戊辰巨引兵自藍田出趣南陽【趣七喻翻】賊聞之解圍走 令狐潮復引兵攻雍丘潮與張巡有舊於城下相勞苦如平生潮因說巡曰【復扶又翻勞力到翻說式芮翻】天下事去矣足下堅守危城欲誰為乎【為于偽翻】巡曰足下平生以忠義自許今日之舉忠義何在潮慙而退 郭子儀李光弼還常山【還從宣翻又音如字】史思明收散卒數萬踵其後子儀選驍騎更挑戰【驍堅堯翻騎奇寄翻更工衡翻挑徒了翻】三日至行唐【即漢南行唐縣屬常山郡九域志在郡北五十五里】賊疲乃退子儀乘之又敗之於沙河【沙河在新樂行唐二縣之間敗補邁翻】蔡希德至洛陽安禄山復使將步騎二萬人北就思明【復扶又翻將即亮翻又音如字】又使牛廷玠發范陽等郡兵萬餘人助思明合五萬餘人而同羅曳落河居五分之一子儀至恒陽思明隨至【恒戶登翻】子儀深溝高壘以待之賊來則守去則追之晝則耀兵夜斫其營賊不得休息數日子儀光弼議曰賊倦矣可以出戰 【考異曰河洛春秋以此為光弼語汾陽家傳作子儀語蓋二人共議耳】壬午戰于嘉山【據舊史安禄山傳嘉山在常山郡東魏收地形志中山郡上曲陽縣有嘉山上曲陽即唐之恒陽也 考異曰實録云六月壬午按長歷六月癸未朔壬午五月二十九日也汾陽家傳舊禄山傳亦云六月戰嘉山河洛春秋云六月二十五日光弼破賊於嘉山今從實録而改其月】大破之斬首四萬級捕虜千餘人思明墜馬露髻跣足步走至暮杖折槍歸營【折而設翻】奔于博陵光弼就圍之軍聲大振於是河北十餘郡皆殺賊守將而降【將即亮翻下同降戶江翻下同 考異曰河洛春秋云五月蔡希德從東都見禄山禄山又與馬步二萬人至邢州取堯山招慶射趙州東界効曲鼔鹿城閒渡洿池水入無極至定州牛介從幽州占歸檀幽易兼大同紇蠟共萬餘人帖思明思明軍既壯共五萬餘人其中精騎萬人悉是同羅曳落河精於馳突光弼以十五萬衆頓軍恒陽樵採往來人有難色召有策者試之時趙州司戶參軍先人亡父包處遂上書與光弼曰思明用軍惟將勁悍觀其舉措實謂無謀昔秦趙爭山先居者勝豈不為勞逸勢倍高下相懸今宜重出軍人有膂力者五萬被甲兩重陌刀各二東有高山甚大先令五千甲士於山上設伏後出二千人山東取糧賊見必追之則奔山上伏兵馬與一百面鼔應山上避賊百姓壯者亦與器械令隨大軍老弱者令居險固守遥為聲援賊必圍山攻之城内出五萬人擇將二人統之各領二萬一將於南面一將於城北門出賊營悉在山東其軍夜出長去賊三十里行廣張左右翼以天曉合圍其軍每二十五為隊每隊置旗兩口鼕鼕鼓子一具圍落纔合則動鼔子賊必不測人之多少然于城中出軍一萬人布掌㡳陳山上亦擊鼔而下齊攻之必克勝光弼尤然此計乃出朔方計會出人取糧賊果然來襲即奔山上至六月二十五日依前計大破賊於嘉山下斬首數萬餘級生擒數千思明落馬步遁至暮拄折槍歸營希德中搶索押衙劉旻斬斷而走生擒得旻至二十六日覆陣二十七日有詔至恒陽云潼關失守駕幸劒南包諝專欲歸功其父而它書皆無之今不取】漁陽路再絶【漁陽即謂范陽也范陽郡幽州其後又分置薊州漁陽郡二郡始各有分界然范陽節度盡統幽易平檀媯燕等州賊之根本實在范陽也唐人於此時多以范陽漁陽通言之白居易詩所謂漁陽鼙鼓動地來是以范陽通為漁陽也前此顏杲卿以常山返正漁陽路絶矣杲卿敗而復通今郭李破史思明故再絶】賊往來者皆輕騎竊過多為官軍所獲將士家在漁陽者無不揺心禄山大懼召高尚嚴莊詬之曰汝數年教我反以為萬全今守潼關數月不能進北路已絶諸軍四合吾所有者止汴鄭數州而已萬全何在汝自今勿來見我尚莊懼數日不敢見田乾真自關下來為尚莊說禄山曰【為于偽翻說式芮翻下密說同】自古帝王經營大業皆有勝敗豈能一舉而成今四方軍壘雖多皆新募烏合之衆未更行陳【更工衡翻行戶剛翻陳讀曰陣】豈能敵我薊北勁鋭之兵何足深憂尚莊皆佐命元勲陛下一旦絶之使諸將聞之誰不内懼若上下離心臣竊為陛下危之禄山喜曰阿浩汝能豁我心事即召尚莊置酒酣宴自為之歌以侑酒待之如初阿浩乾真小字也【為于偽翻 考異曰祿山事迹作阿法今從唐歷統紀舊傳】禄山議弃洛陽走歸范陽計未決是時天下以楊國忠驕縱召亂莫不切齒又禄山起兵以誅國忠為名王思禮密說哥舒翰使抗表請誅國忠【說式芮翻 考異曰玄宗實録云或勸翰留兵二萬守關悉以精鋭回誅楊國忠此漢挫七國之計也公以為何如翰心許之未有客泄其謀于國忠國忠大懼按翰若回兵誅國忠則正與禄山無異思禮勸翰抗表言國忠罪猶不敢况敢舉兵乎事必不然且翰雖心計它人安得知之正由翰按兵不進故國忠及其黨疑懼恐翰回兵誅之其實翰無此心也若果欲誅國忠則安肯慟哭出關乎幸蜀記云翰使王思禮至陜郡見賊偽御史中丞無敵將軍平西大使崔乾祐令傳檄與禄山數其干紀亂常背天逆理且曰若面縳而來束身歸死赦爾九族罪爾一身如更屈彊王師遲疑未決大軍一鼓玉石俱焚爾審思之悔無及矣按翰與乾祐方對壘相攻思禮軍中大將豈可使齎罵禄山之檄詣乾祐乎必無此理今不取】翰不應思禮又請以三十騎劫取以來至潼關殺之翰曰如此乃翰反非禄山也或說國忠今朝廷重兵盡在翰手翰若援旗西指【說式芮翻援于元翻】於公豈不危哉國忠大懼乃奏潼關大軍雖盛而後無繼萬一失利京師可憂請選監牧小兒三千於苑中訓練【時監牧五坊禁苑之卒率謂之小兒】上許之使劒南軍將李福德等領之又募萬人屯灞上令所親杜乾運將之【將即亮翻】名為禦賊實備翰也翰聞之亦恐為國忠所圖乃表請灞上軍隸潼關六月癸未召杜乾運詣關因事斬之國忠益懼會有告崔乾祐在陜兵不滿四千皆羸弱無備【此祿山之用閒也陜失冉翻】上遣使趣哥舒翰進兵復陜洛【趣讀曰促下以義推】翰奏曰祿山久習用兵今始為逆豈肯無備是必羸師以誘我若往正墮其計中【羸倫為翻誘羊久翻】且賊遠來利在速戰官軍據險以扼之利在堅守况賊殘虐失衆兵勢日蹙將有内變因而乘之可不戰擒也要在成功何必務速今諸道徵兵尚多未集請且待之郭子儀李光弼亦上言請引兵北取范陽覆其巢穴質賊黨妻子以招之【上時掌翻質音致】賊必内潰潼關大軍唯應固守以弊之不可輕出國忠疑翰謀已言於上以賊方無備而翰逗留將失機會上以為然續遣中使趣之項背相望翰不得已撫膺慟哭丙戍引兵出關【逗音豆使疏吏翻趣讀曰促 考異曰幸蜀記曰賊將崔乾祐於陜郡西濳鋒蓄鋭卧鼔偃旗而偵者奏云賊全無備上然之又曰玄宗久處太平不練軍事既被國忠眩惑中使相繼督責於公不得已撫膺慟哭久之乃引師出關國忠又令杜乾運領所募兵於馮翊境上濳備哥舒公公曰令軍出關勢十全矣更置乾運於側以為疑軍人心憂疑即不俟見賊吾軍潰矣必當併之以除内憂遂令衙前總管叱萬進追軍誡之曰若不受追即便斬頭來乾運果不肯赴進詐詞如欲叛哥舒竊請見乾運遂喜遽見之與語進忽抽佩刀曰奉處分取公頭乾運驚懼其左右悉新招募者悉投仗散走進遂斬乾運攜首至於軍門衆皆攝氣乃統其軍赴關按翰若擅殺乾運而奪其軍則是己反也朝廷安能趣之出關乎蓋奏乞以其軍隸潼關朝廷已許之翰召乾運受處分或有所違拒因託軍法以斬之耳凌準邠志云郭子儀李光弼將進軍聞朝廷議出潼關圖復陜洛二公議曰哥舒公老疾昏耄賊素知請軍烏合不足以戰今祿山悉鋭南馳宛洛賊之餘衆盡委思明我且破之便覆其巢質叛徒之族取禄山之首其勢必矣若潼關出師有戰必敗關城不守京室有變天下之亂何可平之乃陳利害以聞且請固關無出唐歷會偵人自陜至云崔乾祐所將衆不滿四千不足圖也上大悦舊翰傳翰既斬乾運心不自安又素有風疾至是頗甚軍中之務不復躬親委政于行軍司馬田良丘良丘復不敢專斷教令不一頗無部伍其將王思禮李承光又爭長不叶人無鬭志今兼采之】己丑遇崔乾祐之軍於靈寶西原【靈寶縣更名見二百十五卷天寶元年】乾祐據險以待之南薄山北阻河隘道七十里庚寅官軍與乾祐會戰【薄伯各翻隘烏介翻 考異曰肅宗實録乙酉翰與乾祐會戰舊傳四日次靈寶西原八日與賊交戰新傳丙戍次靈寶西原庚寅與乾祐戰按翰軍既遇賊必不留四日然後戰玄宗實録丙戍翰出關己丑遇賊庚寅戰此近是今從之幸蜀記亦然】乾祐伏兵於險翰與田良丘浮舟中流以觀軍勢見乾祐兵少趣諸軍使進王思禮等將精兵五萬居前龐忠等將餘兵十萬繼之翰以兵三萬登河北阜望之鳴鼓以助其勢【少始紹翻趣讀曰促將即亮翻又音如字】乾祐所出兵不過萬人什什伍伍散如列星或疎㦯密或前或却官軍望而笑之乾祐嚴精兵陳於其後兵既交賊偃旗如欲遁者官軍懈不為備須臾伏兵賊乘高下木石撃殺士卒甚衆道隘士卒如束槍槊不得用翰以氊車駕馬為前驅欲以衝賊日過中東風暴急乾祐以草車數十乘塞氊車之前縱火焚之【乘繩證翻塞悉則翻考異曰幸蜀記曰野中先有官草積數十堆因風焚之今從舊傳】煙焰所被【被皮義翻】官軍<br />
<br />
  不能開目妄自相殺謂賊在煙中聚弓弩而射之【射而亦翻】日暮矢盡乃知無賊乾祐遣同羅精騎自南山過出官軍之後擊之官軍首尾駭亂不知所備於是大敗或弃甲竄匿山谷或相擠排入河溺死囂聲振天地賊乘勝蹙之後軍見前軍敗皆自潰河北軍望之亦潰【河北軍翰所自將者也】翰獨與麾下數百騎走自首陽山西度河入關【首陽山當是首山衍陽字首山在蒲州河東縣界與湖城縣之荆山隔河相對】關外先為三塹皆廣二丈深丈【廣古曠翻深式浸翻】人馬墜其中須臾而滿餘衆踐之以度【踐息淺翻】士卒得入關者纔八千餘人辛卯乾祐進攻潼關克之翰至關西驛揭牓收散卒欲復守潼關【復扶又翻】蕃將火拔歸仁等以百餘騎圍驛入謂翰曰賊至矣請公上馬翰上馬出驛歸仁帥衆叩頭曰公以二十萬衆一戰棄之何面目復見天子【帥讀曰率復扶又翻】且公不見高仙芝封常清乎【謂軍敗必誅也事見上卷上年】請公東行翰不可欲下馬歸仁以毛縶其足於馬腹及諸將不從者皆執之以東【將即亮翻下同降戶江翻】會賊將田乾真已至遂降之俱送洛陽安禄山問翰曰汝常輕我【事見二百十六卷天寶十一載】今定何如翰伏地對曰臣肉眼不識聖人今天下未平李光弼在常山李祗在東平【李祗即謂吳王祇】魯炅在南陽【炅火迥翻】陛下留臣使以尺書招之不日皆下矣禄山大喜以翰為司空同平章事謂火拔歸仁曰汝叛主不忠不義執而斬之翰以書招諸將皆復書責之禄山知不效乃囚諸苑中【東都苑中池】潼關既敗於是河東華陰馮翊上洛防禦使皆弃郡走【河東郡蒲州華陰郡華州馮翊郡同州上洛郡商州華戶化翻】所在守兵皆散是日翰麾下來告急上不時召見【見賢遍翻】但遣李福德等將監牧兵赴潼關及暮平安火不至【六典唐鎮戍烽候所至大率相去三十里每日初夜放煙一炬謂之平安火時守兵已潰無人復舉火】上始懼壬辰召宰相謀之楊國忠自以身領劒南聞安祿山反即令副使崔圓陰具儲偫以備有急投之【相息亮翻令力丁翻使疏吏翻偫直里翻】至是首唱幸蜀之策上然之癸巳國忠集百官於朝堂惶懅流涕【朝直遥翻下同懅巨魚翻急也】問以策略皆唯唯不對【唯于癸翻】國忠曰人告禄山反狀已十年上不之信今日之事非宰相之過仗下【朝罷則左右三衛立仗者皆休下】士民驚擾奔走不知所之市里蕭條國忠使韓虢入宫勸上入蜀甲午百官朝者什無一二上御勤政樓下制云欲親征聞者皆莫之信以京兆尹魏方進為御史大夫兼置頓使京兆少尹靈昌崔光遠為京兆尹充西京留守將軍邉令誠掌宫闈管鑰託以劒南節度大使頴王璬將赴鎮令本道設儲偫【璬公了翻偫文里翻】是日上移仗北内【唐都長安以太極宫為西内大明宫為東内興慶宫為南内北内當在玄武門内又以地望言之則自興慶宫移仗歸大明宫興慶宫在南大明宫在北故亦謂大明宫為北内 考異曰幸蜀記上遣中使曹仙領千人撃鼔於春明門外又令燒閑廐草積煙焰燎天上將乘馬楊國忠諫以為當謹守宗祧不可輕動韋見素力爭以為賊勢逼近人心不固陛下不可不出避狄國忠暗與賊通其言不可聽往返數四上乃從見素議加魏方進御史大大充前路知頓使按賊陷潼關鑾輿將出人心已危豈有更擊鼔燒草以驚之國忠久蓄幸蜀之謀見素乃其所引豈得上前有此爭論此蓋宋巨欲歸功見素事乃近誣今不取】既夕命龍武大將軍陳玄禮整比六軍【比毗寐翻】厚賜錢帛選閑廐馬九百餘匹外人皆莫之知乙未黎明上獨與貴妃姊妹皇子妃主皇孫楊國忠韋見素魏方進陳玄禮及親近宦官宫人出延秋門【延秋門唐長安禁苑之西門也程大昌雍錄有漢唐要地參出圖唐禁苑西北包漢長安故城未央宫唐後改為通光殿西出即延秋門 考異曰幸蜀記云丙申百官尚赴朝此乙未日事宋巨誤也】妃主皇孫之在外者皆委之而去上過左藏【藏徂浪翻】楊國忠請焚之曰無為賊守上愀然曰賊來不得必更歛於百姓不如與之無重困吾赤子【史記玄宗有君人之言愀子小翻歛力瞻翻】是日百官猶有入朝者至宫門猶聞漏聲三衛立仗儼然【唐朝會之制三衛番上分為五仗號衙内五衛一曰供奉仗以左右衛為之二曰親仗以親衛為之三曰勲仗以勲衛為之四曰翊仗以翊衛為之五曰散手仗以親勲翊衛為之平明傳點畢内門開百官入立班皇帝升御坐金吾將軍一人奏左右廂内外平安通事舍人贊宰相兩省官再拜升殿内謁者承旨喚仗左右羽林將軍勘以木契自東西閣而入朝罷皇帝步入東序門然後放仗内外仗隊七刻乃下常參輟朝日六刻即下】門既啓則宫人亂出中外擾攘不知上所之於是王公士民四出逃竄山谷細民爭入宫禁及王公第舍盜取金寶或乘驢上殿又焚左藏大盈庫崔光遠邊令誠帥人救火【帥讀曰率】又募人攝府縣官分守之殺十餘人乃稍定光遠遣其子東見祿山令誠亦以管鑰獻之上過便橋楊國忠使人焚橋上曰士庶各避賊求生柰何絶其路留内侍監高力士使撲滅乃來【玄宗始置内侍監秩三品以高力士及袁思藝為之撲普卜翻】上遣宦者王洛卿前行告諭郡縣置頓食時至咸陽望賢宫【咸陽縣在京城西四十里望賢宫在縣東】洛卿與縣令俱逃中使徵召吏民莫有應者日向中上猶未食楊國忠自市胡餅以獻【胡餅今之蒸餅高似孫曰胡餅言以胡麻著之也崔鴻前趙錄石虎諱胡改胡餅曰麻餅緗素雜記曰有鬻胡餅者不曉名之所謂易其名曰爐餅以為胡人所啗故曰胡餅也】於是民爭獻糲飯【糲盧達翻麤也】雜以麥豆皇孫輩爭以手匊食之須臾而盡猶未能飽【考異曰唐歷至望賢頓御馬病上曰殺此馬拆行宫舍木煮食之衆不忍食幸蜀記至望賢宫行從皆饑上】<br />
<br />
  【入宫憩於樹下怫然若有棄海内之意高力士覺之遂抱上足嗚咽開諭上乃止肅宗實錄楊國忠自入市衣袖中盛餬餅獻上皇天寶亂離記六月十一日大駕幸蜀至望賢宫官吏奔竄迨曛黑百姓有稍稍來者上親問之卿家有飯否不擇精麤但且將來老幼於是競擔挈壺漿雜之以麥子飯送至上前先給兵士六宫及皇孫以下咸以手匊而食頃時又盡猶不能飽既乏器用又無釭燭從駕枕藉寢止長幼莫之分别賴月入戶庭上與六宫皇孫等差異焉按上九日幸蜀温畬云十一日非也餘則兼采之】上皆酬其直慰勞之【勞力到翻】衆皆哭上亦掩泣有老父郭從謹進言曰祿山包藏禍心固非一日亦有詣闕告其謀者陛下往往誅之【事見上上卷年】使得逞其姦逆致陛下播越是以先王務延訪忠良以廣聰明蓋為此也臣猶記宋璟為相數進直言天下賴以安平【為于偽翻數所角翻】自頃以來在廷之臣以言為諱惟阿諛取容是以闕門之外陛下皆不得而知草野之臣必知有今日久矣但九重嚴邃區區之心無路上達事不至此臣何由得睹陛下之面而訴之乎上曰此朕之不明悔無所及慰諭而遣之俄而尚食舉御膳而至【尚主也主御膳之官有奉御有直長而一作以】上命先賜從官【從才用翻下時從同】然後食之令軍士散詣村落求食期未時皆集而行夜將半乃至金城【金城縣屬京兆本始平縣中宗景龍二年送金城公主降吐蕃至此更名金城在京城西八十五里】縣令亦逃縣民皆脱身走飲食器皿具在士卒得以自給時從者多逃内侍監袁思藝亦亡去驛中無燈人相枕藉而寢貴賤無以復辨【枕即任翻藉慈夜翻復扶又翻】王思禮自潼關至始知哥舒翰被擒以思禮為河西隴右節度使即令赴鎮收合散卒以俟東討丙申至馬嵬驛【金人疆域圖馬嵬驛在京兆興平縣】將士饑疲皆憤怒陳玄禮以禍由楊國忠欲誅之因東宫宦者李輔國以告太子太子未决會吐蕃使者二十餘人遮國忠馬訴以無食國忠未及對軍士呼曰國忠與胡虜謀反或射之中鞍國忠走至西門内【馬嵬驛之西門也呼火故翻射而亦翻中竹仲翻】軍士追殺之屠割支體以槍揭其首於驛門外并殺其子戶部侍郎暄及韓國秦國夫人御史大夫魏方進曰汝曹何敢害宰相衆又殺之韋見素聞亂而出為亂兵所檛腦血流地衆曰勿傷韋相公救之得免軍士圍驛上聞諠譁問外何事左右以國忠反對上杖屨出驛門慰勞軍士令收隊軍士不應上使高力士問之玄禮對曰國忠謀反貴妃不宜供奉願陛下割恩正法上曰朕當自處之【處昌呂翻】入門倚杖傾首而立久之京兆司錄韋諤前言曰【京兆府司錄參軍正七品上武德初改州主簿曰錄事參軍掌正違失涖符印開元元年改曰司錄】今衆怒難犯【引左傳鄭子產之言】安危在晷刻願陛下速决因叩頭流血上曰貴妃常居深宫安知國忠反謀高力士曰貴妃誠無罪然將士已殺國忠而貴妃在陛下左右豈敢自安願陛下審思之將士安則陛下安矣【將即亮翻下同】上乃命力士引貴妃於佛堂縊殺之輿尸寘驛庭召玄禮等入視之玄禮等乃免胄釋甲頓首請罪上慰勞之【勞力到翻】令曉諭軍士玄禮等皆呼萬歲再拜而出於是始整部伍為行計諤見素之子也國忠妻裴柔【裴柔故蜀倡也】與其幼子晞及虢國夫人夫人子裴徽皆走至陳倉縣令薛景仙帥吏士追捕誅之【帥讀曰率下同】丁酉上將馬嵬朝臣惟韋見素一人乃以韋諤為御史中丞充置頓使【朝直遥翻使疏吏翻】將士皆曰國忠謀反其將吏皆在蜀不可往或請之河隴或請之靈武㦯請之太原【之往也】或言還京師上意在入蜀慮違衆心竟不言所向韋諤曰還京當有禦賊之備今兵少未易東向【易以豉翻】不如且至扶風徐圖去就 【考異曰幸蜀記曰上意將幸西蜀有中使常清奏曰國忠久在劒南又諸將吏或有連謀慮遠防微須深詳議中官陳全節奏曰太原城池固莫之比可以久處請幸北京中官郭希奏曰朔方地近被帶山河鎮遏之雄莫之與比以臣愚見不及朔方中使駱氶休奏曰姑臧一郡嘗霸中原秦隴河蘭皆足徵取且巡隴右駐蹕凉州翦彼鯨鯢事將取易左右各陳其意見者十餘輩高力士在側而無言上顧之曰以卿之意何道堪行力士曰太原雖固地與賊鄰本屬祿山人心難測朔方近塞半是蕃戎不達朝章卒難教馭西凉懸遼沙漠蕭條大駕順動人馬非少先無備擬必有闕供賊騎趣來恐見狼狽劒南雖窄土富人繁表裏江山内外險固以臣所料蜀道可行上然之即除韋諤御史中丞充置頓使今從唐歷】上詢于衆衆以為然乃從之及行父老皆遮道請留曰宫闕陛下家居陵寢陛下墳墓今捨此欲何之上為之按轡久之乃令太子於後宣慰父老父老因曰至尊既不肯留某等願帥子弟從殿下東破賊取長安【帥讀曰率】若殿下與至尊皆入蜀使中原百姓誰為之主須臾衆至數千人太子不可曰至尊遠冒險阻吾豈忍朝夕離左右【離力智翻】且吾尚未面辭當還白至尊更禀進止涕泣跋馬欲西【還從宣翻跋馬者勒馬使回轉也跋蒲撥翻】建寜王倓【倓徒甘翻】與李輔國執鞚諫曰逆胡犯闕四海分崩不因人情何以興復今殿下從至尊入蜀若賊兵燒絶棧道【鞚苦貢翻棧士限翻】則中原之地拱手授賊矣人情既離不可復合雖欲復至此其可得乎【復扶又翻又音如字】不如收西北守邊之兵召郭李於河北與之併力東討逆賊克復兩京削平四海使社稷危而復安宗廟毁而更存掃除宫禁以迎至尊豈非孝之大者乎何必區區温凊為兒女之戀乎【記曰凡為人子冬温而夏凊昏定而晨省凊七政翻 考異曰舊宦者傳李靖忠啓太子請留張良娣贊成之按太子獨還宣慰百姓良娣不在旁何以得贊成留計今不取天寶亂離記大駕至岐州上取褒斜路幸蜀儲皇取彭原路扺靈武此誤也】廣平王俶亦勸太子留【俶昌六翻】父老共擁太子馬不得行太子乃使俶馳白上上總轡待太子久不至使人偵之【偵丑鄭翻】還白狀上曰天也乃分後軍二千人及飛龍廐馬從太子【仗内六廐飛龍廐為最上乘馬】且諭將士曰太子仁孝可奉宗廟汝曹善輔佐之【將即亮翻】又諭太子曰汝勉之勿以吾為念西北諸胡吾撫之素厚汝必得其用太子南向號泣而已【上已南邁而太子留在後故南向號泣號戶刀翻】又使送東宫内人於太子【張良姊在軍中自此搆建寜之禍】且宣旨欲傳位太子不受俶倓皆太子之子也 己亥上至岐山【岐山縣在扶風郡東北後周天和四年割涇州鶉觚縣之南界置三龍縣隋開皇十六年移於岐山南十里改為岐山縣大業九年移于今縣東北八里唐武德元年移于岐陽縣界張堡壘七年移理龍尾驛城貞觀八年又移理石猪驛】或言賊前鋒且至上遽過宿扶風郡士卒濳懷去就往往流言不遜陳玄禮不能制上患之會成都貢春綵十餘萬匹至扶風上命悉陳之於庭召將士入臨軒諭之曰朕比來衰耄【比毗至翻】託任失人致逆胡亂常須遠避其鋒知卿等皆倉猝從朕不得别父母妻子苃涉至此【草行為苃水行為涉】勞苦至矣朕甚愧之蜀路阻長郡縣褊小人馬衆多或不能供今聽卿等各還家朕獨與子孫中官前行入蜀亦足自達今日與卿等訣别可共分此綵以備資糧若歸見父母及長安父老為朕致意【為于偽翻】各好自愛也因泣下霑襟衆皆哭曰臣等死生從陛下不敢有貳上良久曰去留聽卿自是流言始息【玄宗於此有楚昭王去國諭父老之意然玄宗之為是言也出于不得已】 太子既留莫知所適廣平王俶曰日漸晏此不可駐衆欲何之皆莫對建寜王倓曰殿下昔嘗為朔方節度大使【事見二百十三卷開元十五年】將吏歲時致啓倓略識其姓名今河西隴右之衆皆敗降賊【將即亮翻降戶江翻】父兄子弟多在賊中或生異圖朔方道近士馬全盛裴冕衣冠名族必無貳心【時裴冕為河西行軍司馬】賊入長安方掠虜未暇徇地乘此速往就之徐圖大舉此上策也衆皆曰善至渭濱遇潼關敗卒誤與之戰死傷甚衆己乃收餘卒擇渭水淺處乘馬涉度無馬者涕泣而返太子自奉天北上【文明元年分京兆之醴泉始平好畤武功豳州之永夀縣置奉天縣以奉乾陵在長安西北一百五十里上時掌翻】比至新平【比必寐翻及也】通夜馳三百里士卒器械失亡過半所存之衆不過數百新平太守薛羽棄郡走太子斬之是日至安定太守徐瑴亦走又斬之【新平郡豳州安定郡涇州守手又翻下同瑴訖岳翻】 庚子以劒南節度留後崔圓為劒南節度等副大使辛丑上扶風宿陳倉太子至烏氏彭原太守李遵出迎【烏氏漢縣故墟在彭原東南據舊書烏氏驛名康曰是年改烏氏曰保定余按保定縣本漢安定縣唐為涇州治所在彭原西一百二十里保定縣固是此年更名然非烏氏之地彭原郡寜州本北地郡天寶元年更郡名氏音支】獻衣及糗糧至彭原募士得數百人是日至平凉【糗去久翻平凉郡原州】閲監牧馬得數萬匹又募士得五百餘人軍勢稍振 壬寅上至散關【散關在陳倉縣西南散蘇旱翻】分扈從將士為六軍【從才用翻將即亮翻下同】使頴王璬先行詣劒南【璬公了翻 考異曰肅宗實錄七月壬寅上皇入劒門幸普安郡命頴王璬先入蜀今從玄宗實錄康駢劇談錄上至駱谷山登高望遠嗚咽流涕謂高力士曰吾昔若聽九齡語不到此命中使往韶州祭之按玄宗入蜀不自駱谷康駢誤也舊張九齡傳曰上皇在蜀思張九齡之先覺下詔贈司徒仍遣就韶州致祭案其詔乃德宗贈九齡司徒詔也張九齡事迹云建中元年七月詔舊傳誤也】夀王瑁等分將六軍以次之【瑁莫報翻將同上音又音如字】丙午上至河池郡【河池郡鳳州】崔圓奉表迎車駕具陳蜀土豐稔甲兵全盛上大悦即日以圓為中書侍郎同平章事蜀郡長史如故以隴西公瑀為漢中王梁州都督山南西道采訪防禦使瑀璡之弟也【長知兩翻瑀音禹使疏吏翻璡則鄰翻汝陽王璡寜王憲之嫡長子】 王思禮至平凉聞河西諸胡亂還詣行在初河西諸胡部落聞其都護皆從哥舒翰没於潼關故爭自立相攻擊而都護實從翰在北岸不死又不與火拔歸仁俱降賊【降戶降翻】上乃以河西兵馬使周泌為河西節度使隴右兵馬使彭元耀為隴右節度使【泌薄必翻考異曰肅宗實錄即位之日以泌為河西耀為隴右節度使或者玄宗已命以二鎮二人至靈武見肅宗又加新命乎唐歷作周秘令從玄宗實錄】與都護思結進明等俱之鎮【突厥之臯蘭州興昔府思結之蹛林州金水州賀蘭州盧山府皆覊屬河西又隴右道有突厥州三府二十七】招其部落以思禮為行在都知兵馬使 戊申扶風民康景龍等自相帥擊賊所署宣慰使薛總斬首二百餘級庚戍陳倉令薛景仙殺賊守將克扶風而守之【帥讀曰率將即亮翻下同】 安祿山不意上遽西幸遣使止崔乾祐兵留潼關凡十日乃遣孫孝哲將兵入長安 【考異曰肅宗實錄祿山事迹惟載七月丁卯己巳祿山害諸妃主諸書皆無賊入長安之日惟亂離記云六月二十三日孫孝哲等攻陷長安害諸妃主皇孫七月一日禄山遣殿中御史張通儒為西京留守此書多牴牾不足為據然以月日計之賊以六月八日破潼關其入長安必在此月内矣新傳云賊不謂天子能遽去駐兵潼關十日乃西行時已至扶風按玄宗十六日至扶風縣十七日至扶風郡若賊駐潼關十日則於時未能至長安也又云祿山使張通儒守東京田乾真為京兆尹又云祿山未至長安士人皆逃入山谷羣不逞剽在藏大盈庫百司帑藏竭乃火其餘祿山至怒乃大索三日按舊傳通儒為西京留守徧檢諸書祿山自反後未嘗至長安新傳誤也】以張通儒為西京留守崔光遠為京兆尹使安忠順將兵屯苑中以鎮關中【此西京苑中也】孝哲為祿山所寵任尤用事常與嚴莊爭權祿山使監關中諸將【監工衘翻】通儒等皆受制于孝哲孝哲豪侈果於殺戮賊黨畏之禄山命搜捕百官宦者宫女等每獲數百人輒以兵衛送洛陽王侯將相扈從車駕家留長安者誅及嬰孩【從才用翻】陳希烈以晩節失恩怨上與張均張垍等皆降於賊【陳希烈以罷相失職張均張垍恨不大用故皆降賊】禄山以希烈垍為相自餘朝士皆授以官於是賊勢大熾西脅汧隴南侵江漢北割河東之半【得扶風則西脅汧隴圍南陽則南侵江漢崔乾祐乘潼關之捷北取河東汧口堅翻】然賊將皆麤猛無遠略既克長安以為得志日夜縱酒專以聲色寶賄為事無復西出之意【復扶又翻下始復同】故上得安行入蜀太子北行亦無追迫之患 李光弼圍北陵未下聞潼關不守解圍而南史思明踵其後光弼撃却之與郭子儀皆引兵入井陘留常山太守王俌將景城河間團練兵守常山【俌音甫】平盧節度使劉正臣將襲范陽未至史思明引兵逆擊之正臣大敗棄妻子走士卒死者七千餘人初顔真卿聞河北節度使李光弼出井陘即歛軍還平原以待光弼之命聞郭李西入井陘真卿始復區處河北軍事【處昌呂翻】太子之平凉數日朔方留後杜鴻漸六城水陸運使魏少遊【朔方所統有三受降城及豐安定遠振武三城皆在黄河外】節度判官崔漪支度判官盧簡金鹽池判官李涵【靈鹽二州皆有鹽池故置判官】相與謀曰平凉散地非屯兵之所靈武兵食完富【靈武郡靈州朔方節度使治所】若迎太子至此北收諸城兵西河隴勁騎南向以定中原此萬世一時也乃使涵奉牋于太子且籍朔方士馬甲兵穀帛軍須之數以獻之涵至平凉太子大悦會河西司馬裴冕入為御史中丞至平凉見太子亦勸太子之朔方太子從之鴻漸暹之族子【杜暹開元中為相】涵道之曾孫也【道永安王孝基兄子嗣孝基後】鴻漸漪使少遊居後葺次舍庀資儲【庀卑婢翻具也】自迎太子於平凉北境說太子曰朔方天下勁兵處也今吐蕃請和囘紇内附【說輸芮翻紇下没翻】四方郡縣大抵堅守拒賊以俟興復殿下今理兵靈武按轡長驅移檄四方收攬忠義則逆賊不足屠也少遊盛治宫室帷帳皆倣禁中飲膳備水陸秋七月辛酉太子至靈武悉命撤之【史言肅宗以此成興復之功】 甲子上至普安【普安郡劒州】憲部侍郎房琯來謁見【見賢遍翻】上之長安也群臣多不知至咸陽謂高力士曰朝臣誰當來誰不來對曰張均張垍父子受陛下恩最深且連戚里【謂垍尚主也垍其冀翻】是必先來時論皆謂房琯宜為相而陛下不用【琯古緩翻相息亮翻】又禄山嘗薦之恐或不來上曰事未可知及琯至上問均兄弟對曰臣帥與偕來逗遛不進觀其意似有所蓄而不能言也【帥讀曰率逗音豆】上顧力士曰朕固知之矣即日以琯為文部侍郎同平章事【天寶十一載改刑部曰憲部吏部曰文部】初張垍尚寜親公主【寜親公主自興信徙封上女也】聽於禁中置宅寵渥無比陳希烈求解政務【事見上卷天寶十三載】上幸垍宅問可為相者垍未對上曰無若愛壻垍降階拜舞既而不用故垍懷怏怏上亦覺之【怏於兩翻】是時均垍兄弟及姚崇之子尚書右丞奕蕭嵩之子兵部侍郎華韋安石之子禮部侍郎陟太常少卿斌皆以才望至大官上嘗曰吾命相當徧舉故相子弟耳既而皆不用【自初張垍以下史皆追叙前事斌音彬】裴冕杜鴻漸等上太子牋請遵馬嵬之命即皇帝位太子不許【上時掌翻嵬五囘翻】冕等言曰將士皆關中人日夜思歸所以崎嶇從殿下遠涉沙塞者冀尺寸之功若一朝離散不可復集願殿下勉狥衆心為社稷計牋五上【將即亮翻復扶又翻上時掌翻】太子乃許之是日肅宗即位於靈武城南樓群臣舞蹈上流涕歔欷【自此以後凡書上者皆謂肅宗也歔音虚欷許既翻又音希】尊玄宗為上皇天帝赦天下改元【至是方改天寶十四載為至德元載】以杜鴻漸崔漪並知中書舍人事裴冕為中書侍郎同平章事改關内采訪使為節度使徙治安化以前蒲關防禦使呂崇賁為之【關内采訪使以京官領無治所今改為節鎮治安化領京兆同岐金啇五州安化縣本隋之弘化縣天寶元年連名併更慶州弘化郡為安化郡蒲關即蒲津關使疏吏翻】以陳倉令薛景仙為扶風太守兼防禦使隴右節度使郭英乂為天水太守兼防禦使【守式又翻天水郡秦州】時塞上精兵皆選入討賊惟餘老弱守邉文武官不滿三十人披草萊立朝廷制度艸創武人驕慢大將管崇嗣在朝堂背關而坐言笑自若監察御史李勉奏彈之【朝直遥翻將即亮翻背蒲妹翻監工衘翻】繫於有司上特原之歎曰吾有李勉朝廷始尊勉元懿之曾孫也【鄭王元懿高祖之子】旬日間歸附者漸衆張良娣性巧慧能得上意從上來朔方時從兵單寡【娣大計翻時從才用翻】良娣每寢常居上前上曰禦寇非婦人所能良娣曰倉猝之際妾以身當之殿下可從後逸去至靈武產子三日起縫戰士衣上止之對曰此非妾自養之時上以是益憐之【為良娣挾寵當權得禍張本良娣秩正三品】丁卯上皇制以太子亨充天下兵馬元帥領朔方河東河北平盧節度都使南取長安洛陽【甲子太子即位於靈武丁卯上皇下此制蓋道里相去遼遠蜀中未之知也帥所類翻使疏吏翻】以御史中丞裴冕兼左庶子隴西郡司馬劉秩試守右庶子【隴西郡渭州劉秩必房琯所薦】永王璘充山南東道嶺南黔中江南西道節度都使以少府監竇紹為之傅【璘離珍翻黔音琴少始照翻】長沙太守李峴為都副大使【節度都副大使也】盛王琦充廣陵大都督領江南東路及淮南河南等路節度都使以前江陵都督府長史劉彚為之傅廣陵郡長史李成式為都副大使【廣陵郡揚州】豐王珙充武威都督仍領河西隴右安西北庭等路節度都使以隴西太守濟陰鄧景山為之傅充都副大使【諸道各有節度使以諸王為都使以統之其不赴鎮者都副大使攝統濟子禮翻】應須士馬甲仗糧賜等並於當路自供其諸路本節度使虢王巨等並依前充使【依前為節度使也】其署置官屬及本路郡縣官並任自簡擇署訖聞奏時琦珙皆不出閤惟璘赴鎮【為璘舉兵作亂張本】置山南東道節度使領襄陽等九郡【領襄州襄陽郡鄧州南陽郡隨州漢東郡唐州淮安郡均州武當郡房州房陵郡金州安康郡商州上洛郡】升五府經略使為嶺南節度領南海等二十二郡升五溪經略使為黔中節度領黔中等諸郡【註見上年黔音琴】分江南為東西二道東道領餘杭西道領豫章等諸郡【餘杭郡杭州豫章郡洪州】先是四方聞潼關失守莫知上所之及是制下始知乘輿所在【先悉薦翻守式又翻乘䋲證翻】彚秩之弟也 安禄山使孫孝哲殺霍國長公主【霍國長公主睿宗女下嫁裴虚已長知兩翻】及王妃駙馬等於崇仁坊刳其心以祭安慶宗【安慶宗誅見上卷上年】凡楊國忠高力士之黨及禄山素所惡者皆殺之【惡烏路翻】凡八十三人或以鐵棓揭其腦蓋【棓蒲項翻人門有骨蓋其上謂之腦蓋今方書所云天靈蓋即其物】流血滿街己巳又殺皇孫及郡縣主二十餘人 庚午上皇至巴西太守崔渙迎謁【隆州巴西郡先天二年避上皇諱更名閬州天寶元年更名閬中郡更綿州金山郡曰已西郡考異曰肅宗實録作辛未今從玄宗實録次柳氏舊聞上始入斜谷天尚早煙霧甚昧知頓使給事中韋倜】<br />
<br />
  【於墅中得新熟酒一壺跪獻于馬首者數四上不為之舉倜懼乃注于他器自引滿于前上曰卿以我為疑也始吾御宇之初嘗大醉損一人吾悼之因以為戒迨今四十年矣未嘗甘酒味指力士近臣曰此皆知之非紿卿也從者聞之無不感悦幸蜀記上皇在巴西郡宰臣請高力士奏蜀中氣厚温瘴宜數進酒上皇令高力士宣旨曰朕本嗜酒斷之已久終不再飲深愧卿等意也力士因說上皇開元四年因醉怒殺一人明日都不記得猶召之左右具奏上愴然不言乃賜御庫絹五百匹用給喪事更令力士就宅宣旨致祭從兹斷酒雖下藥亦不輒飲按玄宗荒于聲色幾喪天下斷酒小善夫何足言今不取】上皇與語悦之房琯復薦之【復扶又翻】即日拜門下侍郎同平章事以韋見素為左相渙玄暐之孫也【中宗之復辟也崔玄暐之功列于五王】 初京兆李泌幼以才敏著聞玄宗使與忠王遊忠王為太子泌已長【長知兩翻】上書言事玄宗欲官之不可使與太子為布衣交太子常謂之先生楊國忠惡之奏徙蘄春【蘄春郡蘄州惡烏路翻】後得歸隱居頴陽【武后載初元年分河南伊闕嵩陽置武臨縣開元十五年更名穎陽屬河南府】上自馬嵬北行遣使召之謁見于靈武 【考異曰舊傳云謁見於彭原今從泌子繁所為鄴侯家傳云即位八九日矣見賢遍翻】上大喜出則聯轡寢則對榻如為太子時事無大小皆咨之言無不從至於進退將相亦與之議上欲以泌為右相泌固辭曰陛下待以賓友則貴於宰相矣何必屈其志上乃止 【考異曰舊傳泌稱山人固辭官秩得以散官寵之得當作特解褐拜銀青光祿大夫俾掌樞務鄴侯家傳曰初欲拜為右相恐戎事固辭爵願以客從曰陛下待以賓友則貴于宰相矣何必屈其志上無以逼今從之】 同羅突厥從安祿山反者屯長安苑中甲戍其酋長阿史那從禮帥五千騎竊廄馬二千匹逃歸朔方【帥讀曰率下同騎奇寄翻】謀邀結諸胡盜據邊地上遣使宣慰之降者甚衆 【考異曰肅宗實錄忽聞同羅突厥背祿山走投朔方與六州羣胡共圖河朔諸將皆恐上曰因之招諭當益我軍威上使宣慰果降者過半舊崔光遠傳云同羅背祿山以廐馬二千出至滻水孫孝哲安神威從而召之不得神威憂死陳翃汾陽王家傳云安禄山多譎詐更謀河曲熟蕃以為己屬使蕃將阿史那從禮領同羅突厥五千騎偽稱叛乃投朔方出塞門說九姓府六胡州悉已來矣甲兵五萬部落五十萬蟻集於經略軍北按同羅叛賊則當西出豈得復至滻水此舊傳誤也若禄山使從禮偽叛則孝哲何故召之神威何為怖死又必須先送降欵于肅宗如此則諸將當喜而不恐賊之陰計豈徒取河曲熟蕃也蓋同羅等久客思歸故叛祿山欲乘世亂結諸胡據邉地耳肅宗錄所謂共圖河朔者欲據河朔西方兩道猶言河隴也肅宗從而招之必有降者若又大半則似太多今參取諸書可信者存之】 賊遣兵寇扶風薛景仙擊却之安禄山遣其將高嵩以敕書繒綵誘河隴將士大震<br />
<br />
  關使郭英乂擒斬之【大震關在隴州汧源縣西隴山繒慈陵翻誘音酉】 同羅突厥之逃歸也長安大擾官吏竄匿獄囚自出京兆尹崔光遠以為賊且遁矣遣吏卒守孫孝哲宅孝哲以狀白禄山光遠乃與長安令蘇震帥府縣官十餘人來奔【府京兆府也縣長安萬年】己卯至靈武上以光遠為御史大夫兼京兆尹使之渭北招集吏民 【考異曰天寶亂離記祿山以張通儒為西京留守通儒素憚侍中苖公晉卿内史崔公光遠二人並偽於通儒處請分本職通儒許之由是微申存撫兩街百姓長安稍見寜帖密宣諭人主倉惶西幸之意老幼對泣悲不自勝皆感恩旨苗公乘驢間道赴蜀奔駕光遠亦潛去焉通儒素憚兩公名德内特寛之按舊苗晉卿傳濳遁山谷南投金州未嘗受賊官今不取】以震為中丞震瓌之孫也【蘇瓌事武后中睿三朝歷位台輔】禄山以田乾真為京兆尹侍御史呂諲右拾遺楊綰奉天令安平崔器相繼詣靈武以諲器為御史中丞綰為起居舍人知制誥【唐制誥皆中書舍人掌之以他官掌制誥者謂之知制誥諲音因】上命河西節度副使李嗣業將兵五千赴行在 【考異曰段秀實别傳曰詔嗣業將安西五萬衆赴行在今從舊傳】嗣業與節度使梁宰謀且緩師以觀變綏德府折衝段秀實讓嗣業曰豈有君父告急而臣子晏然不赴者乎特進常自謂大丈夫今日視之乃兒女子耳【據新書秀實自大堆府果毅遷綏德府折衝李嗣業以戰功散階轉至特進故稱之】嗣業大慙即白宰如數發兵以秀實自副將之詣行在上又徵兵於安西行軍司馬李栖筠精兵七千人勵以忠義而遣之敕改扶風為鳳翔郡 庚辰上皇至成都從官及六軍至者千三百人而已【從才用翻】令狐潮圍張廵於雍丘相守四十餘日【是年五月令狐潮攻雍丘】朝廷聲問不通潮聞玄宗已幸蜀復以書招巡【復扶又翻下後復敢復同】有大將六人官皆開府特進白巡以兵勢不敵且上存亡不可知不如降賊巡陽許諾明日堂上設天子畫像帥將士朝之【將即亮翻帥讀曰率朝直遥翻】人人皆泣巡引六將於前責以大義斬之士心益勸城中矢盡巡縳藁為人千餘被以黑衣夜縋城下潮兵爭射之【祓皮義翻縋馳偽翻射而亦翻下弩射同】久乃知其藁人得矢數十萬其後復夜縋人【復扶又翻下同】賊笑不設備乃以死士五百斫潮營潮軍大亂焚壘而遁追奔十餘里潮慙益兵圍之巡使郎將雷萬春於城上與潮相聞賊弩射之面中六矢而不動【中竹仲翻】潮疑其木人使諜問之乃大驚【諜達恊翻】遥謂巡曰向見雷將軍方知足下軍令矣然其如天道何巡謂之曰君未識人倫焉知天道【言叛君附賊未識君臣之倫也焉於乾翻】未幾出戰【幾居豈翻】擒賊將十四人斬首百餘級賊乃夜遁收兵入陳留不敢復出頃之賊步騎七千餘衆屯白沙渦【九域志開封中牟縣有白沙鎮杜預曰梁國寜陵縣北沙陽亭春秋之沙隨地也】巡夜襲撃大破之還至桃陵【司馬彪郡國志東郡燕縣有桃城燕縣唐為滑州胙城縣】遇賊救兵四百餘人悉擒之分别其衆【别彼列翻】媯檀及胡兵悉斬之滎陽陳留脅從兵皆散令歸業【媯州漢潘縣地檀州漢白檀縣地續書云白檀縣即古北平 考異曰張中丞傳自三月二日潮至雍丘城下攻守六十餘日潮大敗而走則於時已五月初矣又云未幾潮又帥衆來攻謂巡曰木朝危蹙兵不出關則是潼關未破也又巡答潮書主上緣哥舒被衂幸于西蜀孝義皇帝收河隴之馬取太原之甲蕃漢雲集不減四十萬衆前月二十七日已到土門蜀漢之兵吳楚驍勇循江而下永王申王部統已到申息之南門竊料胡虜遊䰟終不臘矣則是七月十五日丁卯以後也其曰前月二十七日兵到土門蓋圍城中傳聞之誤也又云相守四十餘日潮收兵入陳留不敢出其下乃云五月魯炅敗于葉六月哥舒翰敗于潼關上皇幸蜀皇帝北廵靈武六月九日賊將瞿伯玉據圉城十二日賊屯白沙渦十四日夜巡襲破之七月十二日潮伯玉至雍丘又破之其日月前後差舛不可考蓋李翰亦得于傳聞不能精審今但置關破以前事於五月關破以後事于七月耳】旬日間民去賊來歸者萬餘戶 河北諸郡猶為唐守【為于偽翻】常山太守王俌欲降賊諸將怒因擊毬縱馬踐殺之時信都太守烏承恩麾下有朔方兵三千人諸將遣使者宗仙運帥父老詣信都迎承恩鎮常山承恩辭以無詔命仙運說承恩曰【說式芮翻】常山地控燕薊路通河洛有井陘之險足以扼其咽喉頃屬車駕南遷【咽音煙屬之欲翻南遷謂自長安南幸蜀也蜀在長安南山之南】李大夫收軍退守晉陽【李大夫謂光弼也】王太守權統後軍欲舉城降賊衆心不從身首異處大將軍兵精氣肅遠近莫敵若以家國為念移據常山與大夫首尾相應則洪勲盛烈孰與為比若疑而不行又不設備常山既陷信都豈能獨全承恩不從仙運又曰將軍不納鄙夫之言必懼兵少故也今人不聊生咸思報國競相結聚屯據鄉村若懸賞招之不旬日十萬可致與朔方甲士三千餘人相參用之足成王事若捨要害以授人居四通而自安【言信都之地夷庚四逹非可居之以自安】譬如倒持劒戟取敗之道也承恩竟疑不決承恩承玼之族兄也【烏承玼見二百十三卷開元二十年玼音此又且禮翻 考異曰韓愈烏氏先廟碑云承恩承洽之兄今從新傳】是月史思明蔡希德將兵萬人南攻九門旬日九門偽降伏甲於城上思明登城伏兵攻之思明墜城鹿角傷其左脅夜奔博陵 顔真卿以蠟丸達表於靈武以真卿為工部尚書兼御史大夫依前河北招討采訪處置使并致赦書亦以蠟丸達之真卿頒下河北諸郡【處昌呂翻又下遐嫁翻】遣人頒於河南江淮由是諸道始知上即位於靈武徇國之心益堅矣 郭子儀等將兵五萬自河北至靈武靈武軍威始盛人有興復之望矣八月壬午朔以子儀為武部尚書靈武長史以李光弼為戶部尚書北都留守【武后天授元年以太原為北都中宗神龍元年罷開元十一年復置天寶元年曰北京是年復曰北都】並同平章事餘如故光弼以景城河間兵五千赴太原先是河東節度使王承業軍政不脩朝廷遣侍御史崔衆交其兵尋遣中使誅之衆侮易承業【先悉薦翻易式豉翻】光弼素不平至是敕交兵於光弼衆見光弼不為禮又不時交兵光弼怒收斬之軍中股栗 【考異曰肅宗實錄八月壬子子儀光弼皆於常山郡嘉山大破賊子儀等俱奉詔領士馬五萬至自河北以子儀為某官光弼為某官汾陽家傳六月八日破史思明於嘉山之下公謂光弼曰賊散矣其餘幾何可長驅而南以定天下其月發恒陽至常山中使邢延恩至奉詔取河北路席卷而南會哥舒翰敗績玄宗幸蜀肅宗如朔方公聞之獨總精兵五萬奔肅宗行在玄宗有誥以肅宗嗣皇帝位肅宗奉誥歔欷哀不自勝公諫云云跪上天子璽以七月十三日即皇帝位二十七日制可武部尚書平章事幸蜀記六月十一日玄宗追郭子儀赴京李光弼守太原河洛春秋六月二十五日大破賊于嘉山二十六日覆陳二十七日有詔至恒陽云潼關失守駕幸劒南儲君又往靈武由是拔軍入井陘口邠志六月八日敗史思明于嘉山會潼關失守二公班師唐歷七月二十八日子儀光弼並加平章事又詔子儀收軍赴朔方光弼赴太原河洛春秋又云光弼至太原殺王承恩固守晉陽舊紀與實錄同子儀傳七月肅宗即位以賊據兩京方謀收復詔子儀班師八月子儀與光弼帥步騎五萬至自河北光弼傳肅宗理兵於靈武遣中使劉智達追光弼子儀赴行在又云以景城河間之卒五千赴太原玄宗實錄六月壬午光弼子儀破史思明於嘉山舊紀六月癸未朔庚寅哥舒翰敗於靈寶其日光弼破史思明于嘉山子儀光弼傅皆云六月無日諸書言李郭事不同如此按歲朔歷六月癸未朔與舊紀同玄宗實錄云壬午誤也肅宗實錄八月壬午朔日也子儀光弼皆於嘉山大破賊領士馬至自河北以為某官某官蓋壬午乃拜官日因言已前事耳汾陽家傳邠志皆云六月八日破思明與舊紀同家傳云勸肅宗即位上璽則恐不然哥舒翰以六月八日敗亦須旬日方傳至河北肅宗七月十三日即位若六月二十七日班師七月十三日豈能便逹靈武也河洛春秋二十五日破賊與諸書皆不合恐太後也今據舊玄宗紀汾陽家傳邠志唐歷皆云六月八日破史思明宜可從幸蜀記十一日玄宗召子儀光弼事或如此但二傳皆云肅宗召之恐是二人在河北聞潼關不守已收軍赴難在道遇肅宗中使遂趨靈武今從舊傳唐歷拜相在七月二十八日汾陽家傳二十七日肅宗實錄八月一日三書皆不相遠子儀傳云八月雖無日與實錄亦略相應今從實錄據舊傳光弼亦曾到靈武疑朔方兵盡從肅宗故光弼但領河北兵赴太原耳河洛春秋月日尤疎所云殺王承恩固守晉陽必誤也】囘紇可汗吐蕃贊普相繼遣使請助國討賊宴賜而遣之癸未上皇下制赦天下 【考異曰玄宗寶錄舊紀皆云八月癸未朔肅宗寶錄唐歷舊紀長歷皆云壬午朔今從之 是時上皇尚未知太子即位於靈武】北海太守賀蘭進明遣録事參軍第五琦入蜀奏事琦言於上皇以為今方用兵財賦為急財賦所產江淮居多乞假臣一職可使軍無乏用上皇悦即以琦為監察御史江淮租庸使【開元十一年宇文融除句當租庸地税使此租庸使之始也其後韋堅楊國忠相繼為之】 史思明再攻九門辛卯克之所殺數千人引兵東圍藁城 李庭望將蕃漢二萬餘人東襲寜陵襄邑夜去雍丘城三十里置營張巡帥短兵三千掩擊大破之殺獲大半庭望收軍夜遁 癸巳靈武使者至蜀【七月甲子即位至是凡三十日使者方至蜀】上皇喜曰吾兒應天順人吾復何憂【復扶又翻下不復同】丁酉制自今改制敕為誥表疏稱太上皇四海軍國事皆先取皇帝進止仍奏朕知俟克復上京朕不復預事己亥上皇臨軒命韋見素房琯崔渙奉傳國寶玉冊詣靈武傳位 【考異曰肅宗實錄癸未上奉表至蜀玄宗實錄八月癸未朔赦天下時皇太子已至靈武七月甲子即位道路險澁表疏未達及下是詔數日北使方至具陳羣臣懇請太子辭避之旨辛卯下詔稱太上皇庚子遣韋見素等奉冊今從舊紀唐歷】 辛丑史思明陷藁城 初上皇每酺宴先設太常雅樂坐部立部繼以鼔吹胡樂教坊府縣散樂雜戲【太常雅樂唐初祖孝孫張文收所定樂也玄宗分樂為二部堂下立奏謂之立部伎堂上坐奏謂之坐部伎立部八一安舞二太平樂三破陣樂四慶善樂五大定樂六上元樂七聖夀樂八光聖樂坐部六一燕樂二長夀樂三天授樂四鳥歌萬歲樂五龍池樂六小破陣樂鼔吹鼔吹署令所掌鐃歌鼔吹曲也胡樂者龜兹疎勒高昌天竺諸部樂也教坊者内教坊及棃園法曲也府縣者京兆府及長安萬年兩赤縣散樂雜戲也酺音蒲】又以山車陸船載樂往來【山車者車上施棚閣加以綵繒為山林之狀陸船者縳竹木為船形飾以繒綵列人於中舁之以行】又出宫人舞霓裳羽衣【玄宗時河西節度使楊敬述獻霓裳羽衣曲十二遍凡曲終必遽惟霓裳羽衣曲終引聲益緩俚俗相傳以為帝遊月宫見素娥數百舞于廣庭帝記其曲歸製霓裳羽衣舞非也】又教舞馬百匹衘盃上夀【帝以馬百匹盛飾分左右施三重榻舞傾盃數十曲壯士舉榻馬不動劉昫曰帝即内廐引蹀馬三十匹為傾杯樂曲奮首鼔尾縱横應節又施三層板牀乘馬而上抃轉而舞】又引犀象入場或拜或舞【五坊使引大象入場或拜或舞動容鼔旅中於音律】安祿山見而悦之既克長安命搜捕樂工運載樂器舞衣驅舞馬犀象皆詣洛陽<br />
<br />
  臣光曰聖人以道德為麗仁義為樂【樂音洛】故雖茅茨土階惡衣菲食不恥其陋惟恐奉養之過以勞民費財明皇恃其承平不思後患殫耳目之玩窮聲技之巧【技渠綺翻】自謂帝王富貴皆不我如欲使前莫能及後無以踰非徒娛己亦以誇人豈知大盜在旁已有窺窬之心卒致鑾輿播越生民塗炭【卒子恤翻】乃知人君崇華靡以示人適足為大盜之招也<br />
<br />
  祿山宴其羣臣於凝碧池【唐六典洛陽禁苑中有芳樹金谷二亭凝碧之池】盛奏衆樂梨園弟子往往歔欷泣下【梨園弟子見二百十一卷開元二年】賊皆露刃睨之【睨五計翻衺視也】樂工雷海清不勝悲憤【勝音升】擲樂器於地西向慟哭禄山怒縛於試馬殿前支解之禄山聞嚮日百姓乘亂多盜庫物既得長安命大索三日【索山客翻】并其私財盡掠之又令府縣推按銖兩之物無不窮治【治直之翻】連引搜捕支蔓無窮民間騷然益思唐室自上離馬嵬北行【離力智翻】民間相傳太子北收兵來取長安長安民日夜望之或時相驚曰太子大軍至矣則皆走市里為空賊望見北方塵起輒驚欲走京畿豪傑往往殺賊官吏遥應官軍誅而復起【復扶又翻】相繼不絶賊不能制其始自京畿鄜坊至于岐隴皆附之至是西門之外率為敵壘【西門謂長安城西門也】賊兵力所及者南不出武關北不過雲陽【雲陽縣漢屬馮翊後魏屬北地郡隋以來屬京兆】西不過武功【武功縣漢晉屬扶風隋唐屬京兆】江淮奏請貢獻之蜀之靈武者【之往也】皆自襄陽取上津路抵扶風【上津漢漢中長利縣地梁置南洛州後魏改曰上州隋廢州為上津縣唐屬商州】道路無壅皆薛景仙之功也 九月壬子史思明圍趙郡丙辰拔之又圍常山旬日城陷殺數千人建寜王倓性英果有才略從上自馬嵬北行兵衆寡<br />
<br />
  弱屢逢寇盜倓自選驍勇居上前後血戰以衛上上或過時未食倓悲泣不自勝軍中皆屬目向之【過古禾翻又古卧翻勝音升屬之欲翻下所屬同】上欲以倓為天下兵馬元帥使統諸將東征【帥所類翻統他綜翻將即亮翻】李泌曰建寜誠元帥才然廣平兄也若建寜功成豈可使廣平為吳太伯乎上曰廣平冢嗣也【泌毗必翻嗣祥吏翻】何必以元帥為重泌曰廣平未正位東宫今天下艱難衆心所屬在於元帥若建寜大功既成陛下雖欲不以為儲副同立功者其肯已乎太宗上皇即其事也【謂皆以有定天下功承大統】上乃以廣平王俶為天下兵馬元帥諸將皆以屬焉 【考異曰鄴侯家傳曰以李光弼為元帥左廂兵馬使出井陘以攻常山圖范陽郭子儀為右廂兵馬使帥衆南取馮翊河東按汾陽家傳時郭子儀方北討同羅未向河東也鄴侯家傳又曰上召光弼子儀議征討計二人有遷延之言上大怒作色叱之二人皆仆地不畢詞而罷上告公曰二將自偏禆一年遇國家有難朕又即位於此遂至三公將相看已有驕色啇議征討欲遷延適來叱之皆倒方圖克復而將已驕朕深憂之朕今委先生戎事府中議事因示以威令使其知懼對曰陛下必欲使畏臣二人未見廣平伏望令王亦暫至府二人至時寒臣與飲酒二人必請謁王臣因為酒令約不起王至但談笑共臣同慰安酒散乃諭其脩謁於元帥則二人見元帥以帝子之尊俯從臣酒令可以知陛下方寵任臣軍中之令必行它時或失律能死生之也上稱善又奏曰伏望言於廣平知是聖意欲李郭之畏臣非臣敢恃恩然也上曰廣平於卿豈有形迹對曰帝子國儲以陛下故親臣臣何人敢不懼明日將曉王亦至及李郭至具軍容脩敬乃坐飲二人因言未見元帥乃使報王王將至執盞為令並不得起及王至先公曰適有令許二相公不起王曰寡人不敢遽就座飲李郭失色談笑皆歡先公曰二人起謝廣平曰先公能為二相公如此復何憂寡人亦盡力今者同心成宗社大計以副聖意既出李謂郭曰適來飲令非行軍意皆上旨也欲令吾徒禀令耳按肅宗温仁二公沈勇必無面叱仆地之事今不取】倓聞之謝泌曰此固倓之心也上與泌出行軍【行下孟翻】軍士指之竊言曰衣黄者聖人也衣白者山人也【衣於既翻下且衣同】上聞之以告泌曰艱難之際不敢相屈以官且衣紫袍以絶羣疑泌不得已受之服之入謝上笑曰既服此豈可無名稱【稱尺證翻】出懷中敕以泌為侍謀軍國元帥府行軍長史【創侍謀之官以處泌】泌固辭上曰朕非敢相臣以濟艱難耳俟賊平任行高志泌乃受之置元帥府於禁中俶入則泌在府泌入俶亦如之泌又言於上曰諸將畏憚天威在陛下前敷陳軍事或不能盡所懷萬一小差為害甚大乞先令與臣及廣平熟議臣與廣平從容奏聞【從千容翻】可者行之不可者已之上許之時軍旅務繁四方奏報自昏至曉無虚刻上悉使送府泌先開視有急切者及烽火重封隔門通進【重直龍翻凡宫禁官府門側置輪盤或遇夜門已閉外有急切文書納諸輪盤旋轉向内以通之】餘則待明禁門鑰契悉委俶與泌掌之【為泌請還鑰契張本】 阿史那從禮說誘九姓府六胡州諸胡數萬衆聚於經略軍北【時九姓胡皆居河曲猶各帶舊置府號按舊書李吉甫傳經略軍唐末之宥州是也天寶移經略軍於靈州城内以宥州寄治經略軍元和九年遂于經略軍故城置宥州六胡州於郭下置延恩縣宋白曰經略軍在夏州西北三百里天寶中王忠嗣奏於榆多勒城置軍今屬靈武去靈武六百餘里說式芮翻】將寇朔方上命郭子儀詣天德軍兵討之【天德軍在大同川天寶十二年安思順奏廢横塞軍請于大同城西築城置軍玄宗賜名天安軍乾元後改為天德軍東南至中受降城二百里西度河至豐州百六十里西至西受降城百八十里北至磧口三百里西北至横寨軍二百里 考異曰汾陽家傳云甲兵五萬部落五十萬今從舊子儀傳汾陽家傳又云九月十九日駕欲幸彭原命公赴天德軍伐叛蕃按寶錄戊辰行幸彭原戊辰十七日也汾陽傳誤】左武鋒使僕固懷恩之子玢别將兵與虜戰兵敗降之既而復逃歸懷恩叱而斬之【玢方貧翻復扶又翻】將士股栗無不一當百遂破同羅上雖用朔方之衆欲借兵於外夷以張軍勢【張知亮翻】以王守禮之子承宷為敦煌王與僕固懷恩使于囘紇以請兵又發拔汗那兵且使轉諭城郭諸國【北狄逐水草為行國西域諸國皆有城郭故謂之城郭諸國】許以厚賞使從安西兵入援李泌勸上且幸彭原俟西北兵將至進幸扶風以應之於時庸調亦集【調徒弔翻】可以贍軍上從之戊辰發靈武 内侍邉令誠復自賊中逃歸【復扶又翻】上斬之 丙子上至順化【上改慶州安化郡為順化郡】韋見素等至自成都奉上寶冊上不肯受曰比以中原未靖權總百官豈敢乘危遽為傳襲羣臣固請上不許寘寶冊於别殿朝夕事之如定省之禮【禮記凡為人子者昏定而晨省奉上時掌翻比毗至翻省悉景翻】上以韋見素本附楊國忠【事見上卷天寶十三載十四載】意薄之素聞房琯名虚心待之琯見上言時事辭情慷慨上為之改容【為于偽翻】由是軍國事多謀於琯琯亦以天下為己任知無不為諸相拱手避之 上皇賜張良娣七寶鞍李泌言於上曰今四海分崩當以儉約示人良娣不宜乘此請撤其珠玉付庫吏以俟有戰功者賞之良娣自閤中言曰鄉里之舊何至於是【良娣母家新豐泌居京兆故云然】上曰先生為社稷計也遽命撤之建寜王倓泣於廊下聲聞於上【聞音問】上驚召問之對曰臣比憂禍亂未已【比毗至翻】今陛下從諫如流不日當見陛下迎上皇還長安是以喜極而悲耳良娣由是惡李泌及倓【為良娣譖殺倓泌不自安張本惡烏路翻下亦惡同】上嘗從容與泌語及李林甫欲敕諸將克長安發其冢焚骨揚灰泌曰陛下方定天下柰何讎死者彼枯骨何知徒示聖德之不弘耳【李林甫動揺東宫見二百十五卷天寶五載六載從于容翻】且方今從賊者皆陛下之讎也若聞此舉恐阻其自新之心上不悦曰此賊昔日百方危朕當是時朕弗保朝夕朕之全特天幸耳林甫亦惡卿但未及害卿而死耳柰何矜之對曰臣豈不知上皇有天下向五十年太平娛樂一朝失意遠處巴蜀【惡烏路翻樂音洛處昌呂翻】南方地惡上皇春秋高聞陛下此敕意必以為用韋妃之故【廢韋妃事亦見二百十五卷天寶五載】内慙不懌萬一感憤成疾是陛下以天下之大不能安君親言未畢上流涕被面【被皮義翻】降階仰天拜曰朕不及此是天使先生言之也遂抱泌頸泣不已它夕上又謂泌曰良娣祖母昭成太后之妹也上皇所念【玄宗幼失昭成后母視良娣祖母鞠愛篤備帝即位封為鄧國夫人其子去逸生良娣泌毗必翻娣大計翻】朕欲使正位中宫以慰上皇心何如對曰陛下在靈武以群臣望尺寸之功故踐大位非私己也至於家事宜待上皇之命不過晚歲月之間耳上從之【史言李泌能引君當道】 南詔乘亂陷越嶲會同軍據清溪關【越巂郡巂州會同軍當在越巂會川縣當瀘津關要路清溪關在大定城北 考異曰唐歷是月吐蕃陷巂州新傳是歲閤羅鳳乘舋取巂州會同軍云云蓋二國兵共陷巂州也】尋傳驃國皆降之【新書尋傳蠻俗無絲纊跣履荆棘不以為苦射豪豬生食其肉戰以竹籠頭如兜鍪驃古朱波也在永昌南二千里去京師萬四千里南屬海比南詔驃所妙翻降戶江翻】<br />
<br />
  資治通鑑卷二百十八  <br>
   </div> 

<script src="/search/ajaxskft.js"> </script>
 <div class="clear"></div>
<br>
<br>
 <!-- a.d-->

 <!--
<div class="info_share">
</div> 
-->
 <!--info_share--></div>   <!-- end info_content-->
  </div> <!-- end l-->

<div class="r">   <!--r-->



<div class="sidebar"  style="margin-bottom:2px;">

 
<div class="sidebar_title">工具类大全</div>
<div class="sidebar_info">
<strong><a href="http://www.guoxuedashi.com/lsditu/" target="_blank">历史地图</a></strong>  
<a href="http://www.880114.com/" target="_blank">英语宝典</a>  
<a href="http://www.guoxuedashi.com/13jing/" target="_blank">十三经检索</a> 
<br><strong><a href="http://www.guoxuedashi.com/gjtsjc/" target="_blank">古今图书集成</a></strong> 
<a href="http://www.guoxuedashi.com/duilian/" target="_blank">对联大全</a> <strong><a href="http://www.guoxuedashi.com/xiangxingzi/" target="_blank">象形文字典</a></strong> 

<br><a href="http://www.guoxuedashi.com/zixing/yanbian/">字形演变</a>  <strong><a href="http://www.guoxuemi.com/hafo/" target="_blank">哈佛燕京中文善本特藏</a></strong>
<br><strong><a href="http://www.guoxuedashi.com/csfz/" target="_blank">丛书&方志检索器</a></strong> <a href="http://www.guoxuedashi.com/yqjyy/" target="_blank">一切经音义</a>  

<br><strong><a href="http://www.guoxuedashi.com/jiapu/" target="_blank">家谱族谱查询</a></strong>  <strong><a href="http://shufa.guoxuedashi.com/sfzitie/" target="_blank">书法字帖欣赏</a></strong> 
<br>

</div>
</div>


<div class="sidebar" style="margin-bottom:0px;">

<font style="font-size:22px;line-height:32px">QQ交流群9:489193090</font>


<div class="sidebar_title">手机APP 扫描或点击</div>
<div class="sidebar_info">
<table>
<tr>
	<td width=160><a href="http://m.guoxuedashi.com/app/" target="_blank"><img src="/img/gxds-sj.png" width="140"  border="0" alt="国学大师手机版"></a></td>
	<td>
<a href="http://www.guoxuedashi.com/download/" target="_blank">app软件下载专区</a><br>
<a href="http://www.guoxuedashi.com/download/gxds.php" target="_blank">《国学大师》下载</a><br>
<a href="http://www.guoxuedashi.com/download/kxzd.php" target="_blank">《汉字宝典》下载</a><br>
<a href="http://www.guoxuedashi.com/download/scqbd.php" target="_blank">《诗词曲宝典》下载</a><br>
<a href="http://www.guoxuedashi.com/SiKuQuanShu/skqs.php" target="_blank">《四库全书》下载</a><br>
</td>
</tr>
</table>

</div>
</div>


<div class="sidebar2">
<center>


</center>
</div>

<div class="sidebar"  style="margin-bottom:2px;">
<div class="sidebar_title">网站使用教程</div>
<div class="sidebar_info">
<a href="http://www.guoxuedashi.com/help/gjsearch.php" target="_blank">如何在国学大师网下载古籍?</a><br>
<a href="http://www.guoxuedashi.com/zidian/bujian/bjjc.php" target="_blank">如何使用部件查字法快速查字?</a><br>
<a href="http://www.guoxuedashi.com/search/sjc.php" target="_blank">如何在指定的书籍中全文检索?</a><br>
<a href="http://www.guoxuedashi.com/search/skjc.php" target="_blank">如何找到一句话在《四库全书》哪一页?</a><br>
</div>
</div>


<div class="sidebar">
<div class="sidebar_title">热门书籍</div>
<div class="sidebar_info">
<a href="/so.php?sokey=%E8%B5%84%E6%B2%BB%E9%80%9A%E9%89%B4&kt=1">资治通鉴</a> <a href="/24shi/"><strong>二十四史</strong></a>&nbsp; <a href="/a2694/">野史</a>&nbsp; <a href="/SiKuQuanShu/"><strong>四库全书</strong></a>&nbsp;<a href="http://www.guoxuedashi.com/SiKuQuanShu/fanti/">繁体</a>
<br><a href="/so.php?sokey=%E7%BA%A2%E6%A5%BC%E6%A2%A6&kt=1">红楼梦</a> <a href="/a/1858x/">三国演义</a> <a href="/a/1038k/">水浒传</a> <a href="/a/1046t/">西游记</a> <a href="/a/1914o/">封神演义</a>
<br>
<a href="http://www.guoxuedashi.com/so.php?sokeygx=%E4%B8%87%E6%9C%89%E6%96%87%E5%BA%93&submit=&kt=1">万有文库</a> <a href="/a/780t/">古文观止</a> <a href="/a/1024l/">文心雕龙</a> <a href="/a/1704n/">全唐诗</a> <a href="/a/1705h/">全宋词</a>
<br><a href="http://www.guoxuedashi.com/so.php?sokeygx=%E7%99%BE%E8%A1%B2%E6%9C%AC%E4%BA%8C%E5%8D%81%E5%9B%9B%E5%8F%B2&submit=&kt=1"><strong>百衲本二十四史</strong></a>  <a href="http://www.guoxuedashi.com/so.php?sokeygx=%E5%8F%A4%E4%BB%8A%E5%9B%BE%E4%B9%A6%E9%9B%86%E6%88%90&submit=&kt=1"><strong>古今图书集成</strong></a>
<br>

<a href="http://www.guoxuedashi.com/so.php?sokeygx=%E4%B8%9B%E4%B9%A6%E9%9B%86%E6%88%90&submit=&kt=1">丛书集成</a> 
<a href="http://www.guoxuedashi.com/so.php?sokeygx=%E5%9B%9B%E9%83%A8%E4%B8%9B%E5%88%8A&submit=&kt=1"><strong>四部丛刊</strong></a>  
<a href="http://www.guoxuedashi.com/so.php?sokeygx=%E8%AF%B4%E6%96%87%E8%A7%A3%E5%AD%97&submit=&kt=1">說文解字</a> <a href="http://www.guoxuedashi.com/so.php?sokeygx=%E5%85%A8%E4%B8%8A%E5%8F%A4&submit=&kt=1">三国六朝文</a>
<br><a href="http://www.guoxuedashi.com/so.php?sokeytm=%E6%97%A5%E6%9C%AC%E5%86%85%E9%98%81%E6%96%87%E5%BA%93&submit=&kt=1"><strong>日本内阁文库</strong></a> <a href="http://www.guoxuedashi.com/so.php?sokeytm=%E5%9B%BD%E5%9B%BE%E6%96%B9%E5%BF%97%E5%90%88%E9%9B%86&ka=100&submit=">国图方志合集</a> <a href="http://www.guoxuedashi.com/so.php?sokeytm=%E5%90%84%E5%9C%B0%E6%96%B9%E5%BF%97&submit=&kt=1"><strong>各地方志</strong></a>

</div>
</div>


<div class="sidebar2">
<center>

</center>
</div>
<div class="sidebar greenbar">
<div class="sidebar_title green">四库全书</div>
<div class="sidebar_info">

《四库全书》是中国古代最大的丛书,编撰于乾隆年间,由纪昀等360多位高官、学者编撰,3800多人抄写,费时十三年编成。丛书分经、史、子、集四部,故名四库。共有3500多种书,7.9万卷,3.6万册,约8亿字,基本上囊括了古代所有图书,故称“全书”。<a href="http://www.guoxuedashi.com/SiKuQuanShu/">详细>>
</a>

</div> 
</div>

</div>  <!--end r-->

</div>
<!-- 内容区END --> 

<!-- 页脚开始 -->
<div class="shh">

</div>

<div class="w1180" style="margin-top:8px;">
<center><script src="http://www.guoxuedashi.com/img/plus.php?id=3"></script></center>
</div>
<div class="w1180 foot">
<a href="/b/thanks.php">特别致谢</a> | <a href="javascript:window.external.AddFavorite(document.location.href,document.title);">收藏本站</a> | <a href="#">欢迎投稿</a> | <a href="http://www.guoxuedashi.com/forum/">意见建议</a> | <a href="http://www.guoxuemi.com/">国学迷</a> | <a href="http://www.shuowen.net/">说文网</a><script language="javascript" type="text/javascript" src="https://js.users.51.la/17753172.js"></script><br />
  Copyright &copy; 国学大师 古典图书集成 All Rights Reserved.<br>
  
  <span style="font-size:14px">免责声明:本站非营利性站点,以方便网友为主,仅供学习研究。<br>内容由热心网友提供和网上收集,不保留版权。若侵犯了您的权益,来信即刪。scp168@qq.com</span>
  <br />
ICP证:<a href="http://www.beian.miit.gov.cn/" target="_blank">鲁ICP备19060063号</a></div>
<!-- 页脚END --> 
<script src="http://www.guoxuedashi.com/img/plus.php?id=22"></script>
<script src="http://www.guoxuedashi.com/img/tongji.js"></script>

</body>
</html>
