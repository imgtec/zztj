<!DOCTYPE html PUBLIC "-//W3C//DTD XHTML 1.0 Transitional//EN" "http://www.w3.org/TR/xhtml1/DTD/xhtml1-transitional.dtd">
<html xmlns="http://www.w3.org/1999/xhtml">
<head>
<meta http-equiv="Content-Type" content="text/html; charset=utf-8" />
<meta http-equiv="X-UA-Compatible" content="IE=Edge,chrome=1">
<title>資治通鑒_249-資治通鑑卷二百四十八_249-資治通鑑卷二百四十八</title>
<meta name="Keywords" content="資治通鑒_249-資治通鑑卷二百四十八_249-資治通鑑卷二百四十八">
<meta name="Description" content="資治通鑒_249-資治通鑑卷二百四十八_249-資治通鑑卷二百四十八">
<meta http-equiv="Cache-Control" content="no-transform" />
<meta http-equiv="Cache-Control" content="no-siteapp" />
<link href="/img/style.css" rel="stylesheet" type="text/css" />
<script src="/img/m.js?2020"></script> 
</head>
<body>
 <div class="ClassNavi">
<a  href="/24shi/">二十四史</a> | <a href="/SiKuQuanShu/">四库全书</a> | <a href="http://www.guoxuedashi.com/gjtsjc/"><font  color="#FF0000">古今图书集成</font></a> | <a href="/renwu/">历史人物</a> | <a href="/ShuoWenJieZi/"><font  color="#FF0000">说文解字</a></font> | <a href="/chengyu/">成语词典</a> | <a  target="_blank"  href="http://www.guoxuedashi.com/jgwhj/"><font  color="#FF0000">甲骨文合集</font></a> | <a href="/yzjwjc/"><font  color="#FF0000">殷周金文集成</font></a> | <a href="/xiangxingzi/"><font color="#0000FF">象形字典</font></a> | <a href="/13jing/"><font  color="#FF0000">十三经索引</font></a> | <a href="/zixing/"><font  color="#FF0000">字体转换器</font></a> | <a href="/zidian/xz/"><font color="#0000FF">篆书识别</font></a> | <a href="/jinfanyi/">近义反义词</a> | <a href="/duilian/">对联大全</a> | <a href="/jiapu/"><font  color="#0000FF">家谱族谱查询</font></a> | <a href="http://www.guoxuemi.com/hafo/" target="_blank" ><font color="#FF0000">哈佛古籍</font></a> 
</div>

 <!-- 头部导航开始 -->
<div class="w1180 head clearfix">
  <div class="head_logo l"><a title="国学大师官网" href="http://www.guoxuedashi.com" target="_blank"></a></div>
  <div class="head_sr l">
  <div id="head1">
  
  <a href="http://www.guoxuedashi.com/zidian/bujian/" target="_blank" ><img src="http://www.guoxuedashi.com/img/top1.gif" width="88" height="60" border="0" title="部件查字,支持20万汉字"></a>


<a href="http://www.guoxuedashi.com/help/yingpan.php" target="_blank"><img src="http://www.guoxuedashi.com/img/top230.gif" width="600" height="62" border="0" ></a>


  </div>
  <div id="head3"><a href="javascript:" onClick="javascript:window.external.AddFavorite(window.location.href,document.title);">添加收藏</a>
  <br><a href="/help/setie.php">搜索引擎</a>
  <br><a href="/help/zanzhu.php">赞助本站</a></div>
  <div id="head2">
 <a href="http://www.guoxuemi.com/" target="_blank"><img src="http://www.guoxuedashi.com/img/guoxuemi.gif" width="95" height="62" border="0" style="margin-left:2px;" title="国学迷"></a>
  

  </div>
</div>
  <div class="clear"></div>
  <div class="head_nav">
  <p><a href="/">首页</a> | <a href="/ShuKu/">国学书库</a> | <a href="/guji/">影印古籍</a> | <a href="/shici/">诗词宝典</a> | <a   href="/SiKuQuanShu/gxjx.php">精选</a> <b>|</b> <a href="/zidian/">汉语字典</a> | <a href="/hydcd/">汉语词典</a> | <a href="http://www.guoxuedashi.com/zidian/bujian/"><font  color="#CC0066">部件查字</font></a> | <a href="http://www.sfds.cn/"><font  color="#CC0066">书法大师</font></a> | <a href="/jgwhj/">甲骨文</a> <b>|</b> <a href="/b/4/"><font  color="#CC0066">解密</font></a> | <a href="/renwu/">历史人物</a> | <a href="/diangu/">历史典故</a> | <a href="/xingshi/">姓氏</a> | <a href="/minzu/">民族</a> <b>|</b> <a href="/mz/"><font  color="#CC0066">世界名著</font></a> | <a href="/download/">软件下载</a>
</p>
<p><a href="/b/"><font  color="#CC0066">历史</font></a> | <a href="http://skqs.guoxuedashi.com/" target="_blank">四库全书</a> |  <a href="http://www.guoxuedashi.com/search/" target="_blank"><font  color="#CC0066">全文检索</font></a> | <a href="http://www.guoxuedashi.com/shumu/">古籍书目</a> | <a   href="/24shi/">正史</a> <b>|</b> <a href="/chengyu/">成语词典</a> | <a href="/kangxi/" title="康熙字典">康熙字典</a> | <a href="/ShuoWenJieZi/">说文解字</a> | <a href="/zixing/yanbian/">字形演变</a> | <a href="/yzjwjc/">金 文</a> <b>|</b>  <a href="/shijian/nian-hao/">年号</a> | <a href="/diming/">历史地名</a> | <a href="/shijian/">历史事件</a> | <a href="/guanzhi/">官职</a> | <a href="/lishi/">知识</a> <b>|</b> <a href="/zhongyi/">中医中药</a> | <a href="http://www.guoxuedashi.com/forum/">留言反馈</a>
</p>
  </div>
</div>
<!-- 头部导航END --> 
<!-- 内容区开始 --> 
<div class="w1180 clearfix">
  <div class="info l">
   
<div class="clearfix" style="background:#f5faff;">
<script src='http://www.guoxuedashi.com/img/headersou.js'></script>

</div>
  <div class="info_tree"><a href="http://www.guoxuedashi.com">首页</a> > <a href="/SiKuQuanShu/fanti/">四库全书</a>
 > <h1>资治通鉴</h1> <!--         下载:【右键另存为】即可 --></div>
  <div class="info_content zj clearfix">
  
<div class="info_txt clearfix" id="show">
<center style="font-size:24px;">249-資治通鑑卷二百四十八</center>
    資治通鑑卷二百四十八 宋 司馬光 撰<br />
<br />
  胡三省 音註<br />
<br />
  唐紀六十四【起閼逢困敦閏月盡屠維大荒落凡五年有奇】<br />
<br />
  武宗至道昭肅孝皇帝下<br />
<br />
  會昌四年閏月壬戌以中書侍郎同平章事李紳同平章事充淮南節度使 李德裕奏鎮州奏事官高迪【方鎮遣牙職入奏事因謂之奏事官】密陳意見二事其一以為賊中好為偷兵術【好呼到翻】潛抽諸處兵聚於一處官軍多就廹逐以致失利經一兩月又偷兵詣它處官軍須知此情自非來攻城柵慎勿與戰彼淹留不過三日須散歸舊屯如此數四空歸自然喪氣【喪息浪翻】官軍密遣諜者詗其抽兵之處乘虛襲之無不捷矣【詗翾正翻又火迥翻】其二鎮魏屯兵雖多終不能分賊勢何則下營不離故處【離力智翻】每兩三月一深入燒掠而去賊但固守城柵城外百姓賊亦不惜宜令進營據其要害以漸逼之若止如今日賊中殊不以為懼望詔諸將各使知之劉稹腹心將高文端降言賊中乏食令婦人挼穗舂之以給軍【挼奴禾翻兩手相切摩也】德裕訪文端破賊之策文端以為官軍今直攻澤州恐多殺士卒城未易得【易以豉翻】澤州兵約萬五千人賊常分兵大半潛伏山谷伺官軍攻城疲弊則四集救之官軍必失利【伺相吏翻】今請令陳許軍過乾河立寨【乾音干】自寨城連延築為夾城環繞澤州【環音宦】日遣大軍布陳於外以扞救兵【陳讀曰陣】賊見圍城將合必出大戰待其敗北然後乘勢可取德裕奏請詔示王宰文端又言固鎮寨四崖懸絶勢不可攻【九域志磁州武安縣有固鎮鎮武安西北至遼州三百餘里】然寨中無水皆飲澗水在寨東約一里許宜令王逢進兵逼之絶其水道不過三日賊必棄寨遁去官軍即可追躡前十五里至青龍寨亦四崖懸絶水在寨外可以前法取也其東十五里則沁州城【沁州治沁源縣漢上黨穀遠縣地沁七鴆翻】德裕奏請詔示王逢文端又言都頭王釗將萬兵戍洺州劉稹既族薛茂卿又誅邢洺救援兵馬使談朝義兄弟三人釗自是疑懼稹遣使召之釗不肯入士卒皆譁譟釗必不為稹用但釗及士卒家屬皆在潞州又士卒恐己降為官軍所殺招之必不肯來惟有諭意於釗使引兵入潞州取稹事成之日許除别道節度使仍厚有賜與庶幾肯從【幾居依翻】德裕奏請詔何弘敬潛遣人諭以此意劉稹年少懦弱【少詩照翻】押牙王協宅内兵馬使李士貴用事專聚貨財府庫充溢而將士有功無賞由是人心離怨劉從諫妻裴氏冕之支孫也【裴冕相肅代兩朝】憂稹將敗其弟問典兵在山東欲召之使掌軍政士貴恐問至奪已權且泄其姧狀乃曰山東之事仰成於五舅【仰牛向翻裴問第五】若召之是無三州也乃止【三州邢洺磁】王協薦王釗為洺州都知兵馬使釗得衆心而多不遵使府約束同列高元武安玉言其有貳心稹召之釗辭以到洺州未立少功實所慙恨乞留數月然後詣府許之王協請稅商人每州遣軍將一人主之名為稅商實籍編戶家貲【編戶猶言編民也將即亮翻】至於什器無所遺皆估為絹匹十分取其二率高其估民竭浮財及糗糧輸之不能充皆忷忷不安【民財非地著轉易以致利者為浮財糗去久翻忷許拱翻】軍將劉溪尤貪殘劉從諫棄不用溪厚賂王協協以邢州富商最多命溪主之裴問所將兵號夜飛多富商子弟溪至悉拘其父兄軍士訴於問問為之請【為于偽翻】溪不許以不遜語答之問怒密與麾下謀殺溪歸國并告刺史崔嘏嘏從之丙子嘏問閉城斬城中大將四人請降於王元逵時高元武在党山聞之亦降【党山恐當作堯山】先是使府賜洺州軍士布人一端尋有帖以折冬賜【先悉薦翻以前所賜布折充冬賜折之舌翻】會稅商軍將至洺州王釗因人不安謂軍士曰留後年少【少詩照翻】政非已出今倉庫充實足支十年豈可不少散之【少詩沼翻】以慰勞苦之士使帖不可用也乃擅開倉庫給士卒人絹一匹穀十二石士卒大喜釗遂閉城請降於何弘敬安玉在磁州聞二州降亦降於弘敬堯山都知兵馬使魏元談等降於王元逵元逵以其久不下皆殺之八月辛卯鎮魏奏邢洺磁三州降宰相入賀李德裕曰昭義根本盡在山東三州降則上黨不日有變矣上曰郭誼必梟劉稹以自贖德裕曰誠如聖料上曰於今所宜先處者何事【處昌呂翻】德裕請以盧弘止為三州留後 【考異曰舊紀傳皆作弘正實錄新紀傳皆作弘止今從之】曰萬一鎮魏請占三州【占之贍翻】朝廷難於可否上從之詔山南東道兼昭義節度使盧鈞乘驛赴鎮潞人聞三州降大懼郭誼王協謀殺劉稹以自贖稹再從兄中軍使匡周兼押牙【再從兄同曾祖從才用翻】誼患之言於稹曰十三郎在牙院【劉匡周第十三牙院押牙治事之所】諸將皆莫敢言事恐為十三郎所疑而獲罪以此失山東今誠得十三郎不入則諸將始敢盡言采於衆人必獲長策稹召匡周諭之使稱疾不入匡周怒曰我在院中故諸將不敢有異圖我出院家必滅矣稹固請之匡周不得已彈指而出誼令稹所親董可武說稹曰【說式芮翻】山東之叛事由五舅城中人人誰敢相保留後今欲何如【五舅謂裴問劉稹自為留後故稱之】稹曰今城中尚有五萬人且當閉門堅守耳可武曰非良策也留後不若束身歸朝如張元益【元益事見二百四十六卷文宗開成三年】不失作刺史且以郭誼為留後俟得節之日徐奉太夫人及室家金帛歸之東都不亦善乎【太夫人謂從諫妻裴氏】稹曰誼安肯如是可武曰可武已與之重誓必不負也乃引誼入稹與之密約既定乃白其母母曰歸朝誠為佳事但恨已晩吾有弟不能保【謂裴問以邢州降也】安能保郭誼汝自圖之稹乃素服出門以母命署誼都知兵馬使王協已戒諸將列於外廳誼拜謝稹己【巳猶畢也】出見諸將稹治裝於内廳【治直之翻】李士貴聞之帥後院兵數千攻誼【帥讀曰率】誼叱之曰何不自取賞物乃欲與李士貴同死乎軍士乃退共殺士貴誼易置將吏部署軍士一夕俱定明日使董可武入謁稹曰請議公事稹曰何不言之可武曰恐驚太夫人乃引稹步出牙門至北宅【北宅昭義節度使别宅也在使宅之北故曰北宅】置酒作樂酒酣乃言今日之事欲全太尉一家【劉悟贈太尉】須留後自圖去就則朝廷必垂矜閔稹曰如所言稹之心也可武遂前執其手崔玄度自後斬之因收稹宗族匡周以下至襁褓中子皆殺之【襁舉兩翻褓音保穆宗長慶初劉悟始帥昭義三世二十六年而滅】又殺劉從諫父子所厚善者張谷陳揚庭李仲京郭台王羽韓茂章茂實王渥賈庠等凡十二家并其子姪甥壻無遺仲京訓之兄台行餘之子羽涯之從孫茂章茂實約之子渥璠之子庠餗之子也甘露之亂仲京等亡歸從諫從諫撫養之【李仲京等僅脫甘露之禍卒與劉從諫之族俱屠蓋天聚而殱之也】凡軍中有小嫌者誼日有所誅流血成泥乃函稹首遣使奉表及書降於王宰首過澤州劉公直舉營慟哭亦降於宰乙未宰以狀聞丙申宰相入賀李德裕奏今不須復置邢洺磁留後【復扶又翻下同】但遣盧弘止宣慰三州及成德魏博兩道上曰郭誼宜如何處之德裕曰劉稹騃孺子耳【處昌呂翻騃五駭翻癡也】阻兵拒命皆誼為之謀主及勢孤力屈又賣稹以求賞此而不誅何以懲惡宜及諸軍在境并誼等誅之上曰朕意亦以為然乃詔石雄將七千人入潞州以應謡言【謡言見上卷三年】杜悰以饋運不給謂誼等可赦上熟視不應德裕曰今春澤潞未平太原復擾自非聖斷堅定【斷丁亂翻】二寇何由可平外議以為若在先朝赦之久矣上曰卿不知文宗心地不與卿合安能議乎罷盧鈞山南東道專為昭義節度使戊戌劉稹傳首至京師詔昭義五州給復一年【復方目翻除其賦役也】軍行所過州縣免今年秋稅昭義自劉從諫以來横增賦歛【横戶孟翻歛力贍翻】悉從蠲免所籍土團並縱遣歸農諸道將士有功者等級加賞郭誼既殺劉稹日望旌節既久不聞問乃曰必移他鎮於是閱鞍馬治行裝【治直之翻】及聞石雄將至懼失色雄至誼等參賀畢敕使張仲清曰郭都知告身來日當至【郭誼為昭義都知兵馬使故稱之】諸高班告身在此晩牙來受之【諸高班謂諸將凡方鎮及州縣率早晩兩牙將校吏卒皆集】乃以河中兵環毬場【河中兵石雄所統入潞州者環讀如宦】晩牙誼等至唱名引入凡諸將桀拒官軍者【下八翻】悉執送京師加何弘敬同平章事丁未詔劉從諫尸暴於潞州市三日石雄取其尸置毬場斬剉之戊申加李德裕太尉趙國公德裕固辭上曰恨無官賞卿耳卿若不應得朕必不與卿初李德裕以韓全義以來【德宗遣韓全義討吳少誠敗于溵水】將帥出征屢敗其弊有三一者詔令下軍前日有三四【下戶嫁翻】宰相多不與聞二者監軍各以意見指揮軍事將帥不得專進退三者每軍各有宦者為監使悉選軍中驍勇數百為牙隊其在陣戰鬭者皆怯弱之士每戰監使自有信旗【信旗者别為一旗軍中視之以為進退監古銜翻使疏吏翻】乘高立馬以牙隊自衛視軍勢小却輒引旗先走陳從而潰【陳讀曰陣】德裕乃與樞密使楊欽義劉行深議約敕監軍不得預軍政每兵千人聽監使取十人自衛有功隨例霑賞二樞密皆以為然白上行之自禦回鶻至澤潞罷兵皆守此制自非中書進詔意更無它詔自中出者號令既簡將帥得以施其謀略故所向有功【史因李德裕之事而敘之以見唐中世之所以敗武宗之所以勝】自用兵以來河北三鎮每遣使者至京師李德裕常面諭之曰河朔兵力雖彊不能自立須藉朝廷官爵威命以安軍情歸語汝使【語牛倨翻使疏吏翻】與其使大將邀宣慰敕使以求官爵何如自奮忠義立功立事結知明主使恩出朝廷不亦榮乎且以耳目所及者言之李載義在幽州為國家盡忠平滄景【為于偽翻】及為軍中所逐不失作節度使後鎮太原位至宰相楊志誠遣大將遮敕使馬求官及為軍中所逐朝廷竟不赦其罪【事並見前紀】此二人禍福足以觀矣德裕復以其言白上【復扶又翻】上曰要當如此明告之由是三鎮不敢有異志 九月詔以澤州隸河陽節度【用李德裕三年之議也】 丁巳盧鈞入潞州鈞素寛厚愛人劉稹未平鈞已領昭義節度【事見上卷三年】襄州士卒在行營者與潞人戰常對陳揚鈞之美【陳讀曰陣】及赴鎮入天井關昭義散卒歸之者鈞皆厚撫之人情大洽昭義遂安劉稹將郭誼王協劉公直安全慶李道德李佐堯劉武德董可武等至京師皆斬之<br />
<br />
  臣光曰董重質之在淮西【事見憲宗紀】郭誼之在昭義吳元濟劉稹如木偶人在伎兒之手耳【伎渠綺翻】彼二人始則勸人為亂終則賣主規利其死固有餘罪然憲宗用之於前武宗誅之於後臣愚以為皆失之何則賞姦非義也殺降非信也失義與信何以為國昔漢光武待王郎劉盆子止於不死知其非力竭則不降故也樊崇徐宣王元牛邯之徒豈非助亂之人乎而光武不殺【事並見光武紀】蓋以既受其降則不可復誅故也若既赦而復逃亡叛亂【復扶又翻下司】則其死固無辭矣如誼等免死流之遠方沒齒不還可矣殺之非也<br />
<br />
  王羽賈庠等已為誼所殺李德裕復下詔稱逆賊王涯賈餗等已就昭義誅其子孫宣告中外識者非之【王涯賈餗非為逆也設以其附麗非人害于而家凶于而國罪亦不至於殄滅而無遺育李德裕明底其罪若真假手于郭誼而致天誅者宜識者之非之也】劉從諫妻裴氏亦賜死又令昭義降將李丕高文端王釗等疏昭義將士與劉稹同惡者悉誅之死者甚衆盧鈞疑其枉濫奏請寛之不從昭義屬城有嘗無禮於王元逵者元逵推求得二十餘人斬之餘衆懼復閉城自守戊辰李德裕等奏寇孽既平盡為國家城鎮豈可令元逵窮兵攻討望遣中使賜城内將士敕招安之仍詔元逵引兵歸鎮并詔盧鈞自遣使安撫從之乙亥李德裕等請上尊號且言自古帝王成大功必告天地又宣懿太后祔廟【上初即位追諡母韋妃曰宣懿太后】陛下未嘗親謁上瞿然曰郊廟之禮誠宜亟行至於徽稱非所敢當凡五上表乃許之【瞿紀具翻瞿然失其常度之貌徽美也稱昌孕翻】李德裕奏據幽州奏事官言詗知回鶻上下離心【詗火迥翻又翾正翻】可汗欲之安西其部落言親戚皆在唐不如歸唐又與室韋已相失計其不日來降或自相殘滅望遣識事中使【欲遣識事宜者出使】賜仲武詔諭以鎮魏已平昭義惟回鶻未滅仲武猶帶北面招討使宜早思立功 李德裕怨太子太傅東都留守牛僧孺湖州刺史李宗閔言於上曰劉從諫據上黨十年太和中入朝僧孺宗閔執政不留之加宰相縱去【事見二百四十四卷文宗太和七年】以成今日之患竭天下力乃能取之皆二人之罪也德裕又使人於潞州求僧孺宗閔與從諫交通書疏無所得乃令孔目官鄭慶言從諫每得僧孺宗閔書疏皆自焚毁詔追慶下御史臺按問【下遐稼翻】中丞李回知雜鄭亞以為信然【唐制御史臺侍御史六人以久次者一人知雜事謂之雜端】河南少尹呂述與德裕書言稹破報至僧孺出聲歎恨【此希德裕意而誣僧孺也】德裕奏述書上大怒以僧孺為太子少保分司宗閔為漳州刺史戊子再貶僧孺汀州刺史宗閔漳州長史【垂拱元年分福州西南境置漳州以南有漳水為名舊志京師東南七千三百里】 上幸鄠校獵【鄠音戶】 十一月復貶牛僧孺循州長史宗閔長流封州【復扶又翻】 十二月以忠武節度使王宰為河東節度使河中節度使石雄為河陽節度使 【考異曰實錄九月盧鈞奏十七日石雄回軍赴孟州按雄于時未為河陽節度使實錄誤也】 上幸雲陽校獵<br />
<br />
  五年春正月己酉朔羣臣上尊號曰仁聖文武章天成功神德明道大孝皇帝尊號始無道字中旨令加之【是時帝崇信道士趙歸真等至親受道籙故旨令羣臣於尊號中加道字而不知其所謂道者非吾之所謂道也】庚戌上謁太廟辛亥祀昊天上帝赦天下 築望仙臺於南郊 庚申義安太后王氏崩【太和五年宰相建白以太皇太后與寶歷太后稱號未辨前代詔令不敢斥言皆以宫為稱今寶歷太后居義安殿宜曰義安太后詔可】 以秘書監盧弘宣為義武節度使弘宣性寛厚而難犯為政簡易【易以䜴翻】其下便之河北之法軍中偶語者斬弘宣至除其法【河北諸帥防其下相與聚謀以圖已故嚴軍中偶語之法以剛制之盧弘宣至中山乃除其法】詔賜粟三十萬斛在飛狐西計運致之費踰於粟價弘宣遣吏守之會春旱弘宣命軍民隨意自往取之粟皆入境約秋稔償之時成德魏博皆饑獨易定之境無害淮南節度使李紳按江都令吳湘盜用程糧錢【新書百官】<br />
<br />
  【志主客郎中主蕃客東南蕃使還者給入海程糧西北蕃使還者給度磧程糧至於官吏以公事有遠行則須計程以給糧而糧重不可遠致則以錢凖估故有程糧錢】強娶所部百姓顔悦女估其資裝為罪當死湘武陵之兄子也【吳武陵見二百三十九卷憲宗元和十年】李德裕素惡武陵【惡烏路翻】議者多言其寃諫官請覆按詔遣監察御史崔元藻李稠覆之還言湘盜程糧錢有實顔悦本衢州人嘗為青州牙推妻亦士族與前獄異德裕以為無與奪二月貶元藻端州司戶稠汀州司戶不復更推亦不付法司詳斷即如紳奏處湘死【復扶又翻斷丁亂翻處昌呂翻為德裕以吳湘獄致禍張本】諫議大夫柳仲郢敬晦皆上疏爭之不納稠晉江人【宋白曰泉州治晉江縣晉為晉安縣地隋廢郡為邑】晦昕之弟也【敬昕見上卷三年】 李德裕以柳仲郢為京兆尹素與牛僧孺善謝德裕曰不意太尉恩奬及此仰報厚德敢不如奇章公門館德裕不以為嫌【隋封牛弘為奇章公牛僧孺蓋其後也故時人亦呼之為奇章公宋白曰奇章縣屬巴州本漢葭萌縣地梁置奇章縣取縣東八里奇章山為名】夏四月壬寅以陜虢觀察使李拭為冊戛斯可汗使【陜失冉翻】 五月壬戌葬恭僖皇后于光陵柏城之外【義安太后諡曰恭僖后於穆宗非伉儷故陪葬光陵而不合】 門下侍郎同平章事杜悰罷為右僕射中書侍郎同平章事崔鉉罷為戶部尚書乙丑以戶部侍郎李回為中書侍郎同平章事判戶部如故 祠部奏括天下寺四千六百蘭若四萬僧尼二十六萬五百【祠部掌僧尼故使括之若人者翻釋氏要覽曰蘭若者梵言阿蘭若唐言無諍也四方律云空靜處智度經云遠離處大悲經云離諸忿】 戛斯可汗為宗英雄武誠明可汗 秋七月丙午朔日有食之 上惡僧尼耗蠧天下欲去之【惡烏路翻去羌呂翻】道士趙歸真等復勸之【復扶又翻】乃先毁山野招提蘭若【釋書云招提菩薩皆佛名故號寺或謂之招提增輝記曰招提者梵言拓鬬提奢唐言四方僧物後人傳寫之誤以拓為招又省去鬬奢二字只稱招提即今十方寺院是也薩波論云西天度地以四肘為一弓去村店五百弓不遠不近以閒靜為蘭若史炤曰今若以唐尺計之度二里許】敕上都東都兩街各留二寺【唐謂長安曰上都時左街留慈恩薦福右街留西明莊嚴】每寺留僧三十人天下節度觀察使治所及同華商汝州各留一寺【華戶化翻】分為三等上等留僧二十人中等留十人下等五人 【考異曰實録中書門下奏請上都東都兩街各留寺十所每寺留僧十人大藩鎮各一所僧亦依前詔敕上都東都每街各留寺兩所每寺僧各留三十人中書門下奏奉敕諸道所留僧尼數宜令更商量分為三等上至二十人中至十人下至五人今據天下諸道共五十處四十六道合配三等鎮州魏博淮南西川山南東道荆南嶺南汴宋幽州東川鄂岳浙西浙東宣歙湖南江西河南府望每道許留僧二十人山南西道河東鄭滑陳許潞磁鄆曹徐泗鳳翔兖海淄青滄景易定福建同華州望今每道許留十人夏桂邕管黔中安南汝金商州容管望每道許留五人一道河中已敕下留十三人按鎮州凡五十六州四十一道今云五十處四十六道誤也杜牧杭州南亭記曰武宗即位始去其山臺野邑四萬所冠其徒幾至十萬人後至會昌五年始命西京留佛寺四僧惟十人東都二寺天下所謂節度觀察同華汝三十四治所得留一寺僧凖西京數其他刺史州不得有寺凡除寺四千六百僧尼䈂冠二十六萬五百實録注又云按唐時石刻云兩都留寺僧各十人郡國留寺二僧各三人數皆不同今從實録前文】餘僧及尼并大秦穆護祅僧皆勒歸俗【大秦穆護又釋氏之外教如回鶻摩尼之類是時勑曰大秦穆護等祠釋教既已釐革邪法不可獨存其人並勒還俗遞歸本貫充稅戶如外國人送遠處收管祅乎煙翻胡神也唐制祠部歲再祀磧西諸州火祅而禁民祈祭官品令有祅正蓋主祅僧也】寺非應留者立期令所在毁撤仍遣御史分道督之財貨田產並沒官寺材以葺公廨驛舍【廨古隘翻】銅像鐘磬以鑄錢 以山南東道節度使鄭肅檢校右僕射同平章事 詔昭義騎兵五百步兵千五百戍振武節度使盧鈞出至裴村餞之潞卒素驕憚於遠戍乘醉回旗入城閉門大譟鈞奔潞城以避之【宋白曰潞城縣春秋潞子嬰兒之國漢為潞縣十三州志云潞水出焉後魏太武改為刈陵縣隋開皇十三年置潞城縣九域志潞城在潞州東北四十里】監軍王惟直自出曉諭亂兵擊之傷旬日而卒李德裕奏請詔河東節度使王宰以步騎一千守石會關三千自儀州路據武安以斷邢洺之路【斷音短】又令河陽節度使石雄引兵守澤州河中節度使韋恭甫發步騎千人戍晉州如此賊必無能為【分守四境使潞之亂卒不得越逸而奔他鎮】皆從之 八月李德裕等奏東都九廟神主二十六今貯於太微宫小屋【玄宗天寶二年改東都玄元皇帝廟曰太微宫劉昫曰東都太微宫本武后家廟神龍初中宗反正廢武氏廟主立太祖已下神主祔主安禄山陷洛陽以廟為馬廄棄其神主協律郎嚴郢收而藏之史思明再陷洛陽尋又散失賊平東都留守盧正已又募得之廟已焚毁乃寄主於太微宫貯丁呂翻】請以廢寺材復修太廟 壬午詔陳釋教之弊宣告中外凡天下所毁寺四千六百餘區歸俗僧尼二十六萬五百人大秦穆護祅僧二千餘人毁招提蘭若四萬餘區 【考異曰會要元和二年薛平奏請賜中條山蘭若額為大和寺蓋官賜額者為寺私造者為招提蘭若杜牧所謂山臺野邑是也】收良田數千萬頃奴婢十五萬人所留僧皆隸主客不隸祠部【時中書門下奏據大唐六典祠部掌天地宗廟大祭與僧事殊不相當又萬務根本合歸尚書省隸鴻臚寺亦未為允當又據六典主客掌朝貢之國七十餘蕃五天竺國並在數内釋氏出自天竺國今陛下以其非中國之教已有釐革僧尼名籍便令係主客不隸祠部及鴻臚寺至為允當從之】百官奉表稱賀尋又詔東都止留僧二十人諸道留二十人者減其半留十人者減三人留五人者更不留五臺僧多亡奔幽州【五臺在代州五臺縣山形五峙相傳以為文殊示現之地華嚴經疏云清涼山者即代州鴈門五臺山也以歲積堅冰夏仍飛雪曾無炎暑故曰清涼五峯聳出頂無林木有如壘土之臺故曰五臺古傳云山在長安東北一千六百餘里代州之所管山頂至州城一百餘里其山左隣恒山右接天池南屬五臺縣北至繁畤縣環基所至五百餘里靈記云五臺山有四埵去臺各一百二十里據古經所載今北臺即是中臺中臺即是南臺大黄尖即是北臺栲栳山即是西臺漫天石即是東臺惟北臺中臺古時無異東臺西臺古今無别無恤臺恒山頂是也昔趙襄子名無恤曾登此山觀代國下瞰東海西瞢□山有宫池古廟隋煬帝避暑於此而居因天池造立宫室龍樓鳳閣遍滿池邊號為西埵南繋舟山上有銅環船軸猶在昔帝堯遭水繋舟於此世傳文殊見于南臺號為南埵北有覆宿堆即夏屋山也後魏孝文皇帝避暑往復宿此下見雲州謂之北埵中臺稍近西北有太華泉有古寺二十餘處東臺去太華泉四十二里臺上遥見滄瀛諸州日出時下視大海猶陂澤焉有古寺十五處西臺去太華泉四里危嶝千雲喬林拂日有古寺十二處南臺去太華泉八十里最為幽寂有古寺九處北臺去太華泉十二里有古寺八處唐末所添寺不在其數五臺縣本漢慮虒縣慮虒音驢夷隋大業二年改為五臺縣】李德裕召進奏官謂曰汝趣白本使【趣讀曰促】五臺僧為將必不如幽州將為卒必不如幽州卒何為虛取容納之名染於人口【將即亮翻染如豔翻又而險翻】獨不見近日劉從諫招聚無筭閒人竟有何益張仲武乃封二刀付居庸關曰有游僧入境則斬之主客郎中韋博以為事不宜太過李德裕惡之【惡烏路翻】出為靈武節度副使昭義亂兵奉都將李文矩為帥【帥所類翻】文矩不從亂兵亦不敢害文矩稍以禍福諭之亂兵漸聽命乃遣人謝盧鈞於潞城鈞還入上黨復遣之戍振武行一驛乃潛選兵追之明日及於太平驛【唐制三十里一驛太平驛在潞州北六十里宋白曰太平驛東南距潞州八十里】盡殺之 【考異曰獻替記上信任宰臣無不先訪問無獨斷之事唯誅討澤潞不肯捨赴振武官健及誅翦党項此二事並禁中詔處分更不顧問振武官健回旗不肯進先害監軍傔一人監軍王惟直自出曉諭又被傷痍旬日而卒禁中兩軍樞密已下恨其不殺節將唯害中人所以激上之怒盡須勦戮上問宰臣曰我送石雄領兵至澤潞令盧鈞不誅討罪人如何德裕曰盧鈞已失律性又寛愞必恐自誅不得若便替却盧鈞亂卒罪惡轉大須興兵討伐恐不如先除替令新帥誅翦上謂德裕曰勿惜盧鈞本非材將救澤潞叛兵疑李丕報嫌往劉稹平後處置澤潞與劉稹同惡僅五千餘人皆是取得高文端王釗狀通姓名勘李丕狀同然後處分其間有三兩人或王釗狀無名並不更問足明是李丕不能逞其憾又云惟務苟安因循為政凡方鎮兵只合不出軍城嚴兵自衛於城門閱過部伍更令軍將慰安豈有自出送兵馬又令家口縱觀事同兒戲實不足惜緣大兵之後須有防虞臣不敢隱默由是中詔處分不復顧問 按盧鈞還入潞州諭戍兵使赴振武尋遣兵追擊盡殺之非上不肯捨也既云不可便替又云不如先除替語自相違上云勿惜盧鈞是上語下云臣不敢隱默乃是德裕語獻替記至此差舛尤甚不可復據又處置澤潞五千餘人太多必是五十字誤耳】具以狀聞且請罷河東河陽兵在境上者從之 九月詔修東都太廟【如李德裕所奏也】 李德裕請置備邊庫令戶部歲入錢帛十二萬緡匹度支鹽鐵歲入錢帛十三萬緡匹明年減其三之一凡諸道所進助軍財貨皆入焉以度支郎中判之 王才人寵冠後庭【冠古玩翻】上欲立以為后李德裕以才人寒族且無子恐不厭天下之望【厭益涉翻伏也合也】乃止 上餌方士金丹性加躁急喜怒不常冬十月上問李德裕以外事對曰陛下威斷不測【斷丁亂翻】外人頗驚懼曏者寇逆暴横【横戶孟翻】固宜以威制之今天下既平願陛下以寛理之但使得罪者無怨為善者不驚則為寛矣 以衡山道士劉玄靜為銀青光祿大夫崇玄館學士賜號廣成先生為之治崇玄館置吏鑄印【唐有崇玄署令掌僧道屬宗正寺又有崇玄學博士掌教玄學生玄宗天寶二年改崇玄學曰崇玄館改博士曰學士為之于偽翻治直之翻】玄靜固辭乞還山許之 李德裕秉政日久好徇愛憎【好呼到翻】人多怨之自杜悰崔鉉罷相宦官左右言其太專上亦不悦給事中韋弘質上疏言宰相權重不應更領三司錢穀德裕奏稱制置職業人主之柄弘質受人教導所謂賤人圖柄臣【傳曰下輕其上爵賤臣圖柄臣則國家動揺而人不靜】非所宜言十二月弘質坐貶官由是衆怒愈甚【史言李德裕以自專自用速禍】 上自秋冬以來覺有疾而道士以為換骨上祕其事外人但怪上希復遊獵【復扶又翻下同】宰相奏事者亦不敢久留詔罷來年正旦朝會【以有疾也】 吐蕃論恐熱復糾合諸部擊尚婢婢婢婢遣厖結藏將兵五千拒之恐熱大敗與數十騎遁去婢婢傳檄河湟數恐熱殘虐之罪【數恐所具翻】曰汝輩本唐人吐蕃無主則相與歸唐毋為恐熱所獵如狐兔也於是諸部從恐熱者稍稍引去 是時天下戶四百九十五萬五千一百五十一 朝廷雖為党項置使【帝以侍御史為使分三部招定党項以邠寧延屬崔彦曾鹽夏長凙屬李鄠靈武麟勝屬鄭賀】党項侵盜不已攻陷邠寧鹽州界城堡屯叱利寨宰相請遣使宣慰上決意討之<br />
<br />
  六年春二月庚辰以夏州節度使米暨為東北道招討党項使【米姓出於西域康居枝庶分為米國後入中國子孫遂以為姓】 上疾久未平以為漢火德改洛為雒【漢光武改洛陽為雒陽】唐土德不可以王氣勝君名三月下詔改名炎【王于况翻唐以土德王而帝名瀍瀍旁從水土勝水故言以王氣勝君名今改名炎炎從火火能生土取以君名生王氣也帝未幾而晏駕厭勝果何益哉】上自正月乙卯不視朝 【考異曰實錄作十五日按獻替記自正月十三日後至三月二十日更不開延英時見中詔處分莫得預焉今從之】宰相請見不許【見賢遍翻】中外憂懼初憲宗納李錡妾鄭氏生光王怡怡幼時宫中皆以為不慧太和以後益自韜匿羣居游處【處昌呂翻】未嘗言文宗幸十六宅宴集好誘其言以為戲笑【好呼到翻】上性豪邁尤所不禮 【考異曰韋昭度續皇王寶運錄曰宣宗即憲皇第四子自憲皇崩便合紹位乃與姪文宗文宗崩武皇慮有他謀乃密令中常侍四人擒宣宗於永巷幽之數日沈於宫厠宦者仇公武愍之乃奏武宗曰前者王子不宜久於宫厠誅之武宗曰唯唯仇公武取出於車中以糞土雜物覆之將别路歸家密養之三年後武皇宫車晏駕百官奉迎于玉宸殿立之尋擢仇公武為軍容使尉遲偓中朝故事曰敬宗文宗武宗相次即位宣皇皆叔父也武宗初登極深忌焉一日會鞠於禁苑間武宗召上遥覩瞬目於中官仇士良士良躍馬向前曰適有旨王可下馬士良命中官輿出軍中奏云落馬已不救矣尋請為僧遊行江表間會昌末中人請還京遂即位令狐澄貞陵遺事曰上在藩時嘗從駕迴而上誤墮馬人不之覺比二更方能興時天大雪四顧悄無人聲上寒甚會巡警者至大驚上曰我光王也不悟至此方困且渇若為我求水警者即於旁近得水以進遂委而去上良久起舉甌將飲顧甌中水盡為芳醪矣上獨喜自負一舉盡甌已而體微煖有力遂步歸藩邸此三事皆鄙妄無稽今不取】及上疾篤旬日不能言諸宦官密於禁中定策辛酉下詔稱皇子冲幼須選賢德光王怡可立為皇太叔 【考異曰舊紀三月一日止為皇太叔武宗實錄云壬戌宣宗實錄云辛酉按獻替記云自正月十三日後至三月二十日更不聞延英蓋二十一日則宣宗見百寮也今從宣宗實錄】更名忱【更工衡翻忱時壬翻】應軍國政事令權句當【以武宗之英逹李德裕之得君而不能定後嗣卒制命於宦豎北司掌兵且專宫禁之權也句古侯翻當丁浪翻下咸當同】太叔見百官哀戚滿容裁决庶務咸當於理人始知有隱德焉【當丁浪翻】甲子上崩【年三十三】以李德裕攝冢宰丁卯宣宗即位宣宗素惡李德裕之專【惡烏路翻】即位之日德裕奉冊既罷謂左右曰適近我者非太尉邪每顧我使我毛髮洒淅【近其靳翻洒淅肅然之意言可畏憚也】夏四月辛未朔上始聽政 尊母鄭氏為皇太后 壬申以門下侍郎同平章政事李德裕同平章事充荆南節度使 【考異曰實錄新表傳皆云德裕自守太尉檢校司徒今從舊紀又貞陵遺事曰上初即位於太極殿時宰相李德裕與行冊禮及退上謂宦侍云云聽政之二日遂出為荆門舊德裕傳曰五年武宗上尊號累表乞骸不許德裕病月餘堅請解機務乃以本官平章事兼江陵尹荆南節度使數月追復知政事宣宗即位罷相出為東都留守按舊紀新表及諸書武宗朝德裕未嘗罷免此年九月方自江陵除東都留守舊傳謬誤今從實錄】德裕秉權日久位重有功衆不謂其遽罷聞之莫不驚駭甲戌貶工部尚書判鹽鐵轉運使薛元賞為忠州刺史弟京兆少尹權知府事元龜為崖州司戶皆德裕之黨也 杖殺道士趙歸真等數人流羅浮山人軒轅集于嶺南五月乙巳赦天下上京兩街先聽留兩寺外更各增置八寺【左街先留慈恩薦福今增置興唐保夀二寺寶應寺改為資聖寺青龍寺改為護國寺菩提寺改為保唐寺清禪寺改為安國寺尼寺二所法雲寺改為唐安寺崇敬寺改為唐昌寺右街先留西明寺改為福夀寺莊嚴寺改為聖夀寺添置僧寺一所千福寺尼寺一所興聖寺依舊名化度寺改為崇福寺永泰寺改為萬夀寺清國寺改為崇聖寺經行寺改為龍興寺奉恩寺改為興福寺尼寺一所萬善寺改為延唐寺 考異曰杭州南亭記曰今天子即位天下州率與二寺用齒衰男女為其徒各止三十人兩京數倍其四五焉實録凖五日敕兩街先留寺兩所外更添八所注唐石刻云京師兩街各置十寺寺僧五十人蓋謂二年正月赦後非今赦也】僧尼依前隸功德使不隸主客【唐初天下僧尼道士女官皆隸鴻臚寺武后延載元年以僧尼隸祠部開元二十四年道士女官隸宗正寺天寶二載以道士隸司封貞元四年崇玄館罷大學士後復置左右街大功德使東都功德使修功德使總僧尼之籍及功役元和二年以道士女官隸左右街功德使會昌二年以僧尼隸主客太清宫置玄元館亦有學士至六年廢而僧尼復隸兩街功德使即是年也】所度僧尼仍令祠部給牒【改武宗之政也牒即今祠部所給僧道度牒也】 以翰林學士兵部侍郎白敏中同平章事 辛酉立皇子温為鄆王渼為雍王【渼音美】涇為雅王滋為夔王沂為慶王 六月禮儀使奏請復代宗神主於太廟【開成五年文宗升祔代宗神主以親盡祧遷今請復之】以敬宗文宗武宗同為一代於廟東增置兩室為九代十一室從之 秋七月壬寅淮南節度使李紳薨回鶻烏介可汗之衆稍稍降散及凍餒死所餘不及<br />
<br />
  三千人國相逸隱啜殺烏介於金山【烏介可汗自殺胡山之敗竄於黑車子族今為其下所殺】立其弟特勒遏捻為可汗【捻奴協翻】 八月壬申葬至道昭肅孝皇帝于端陵【端陵在京兆三原縣東十里】廟號武宗初武宗疾困顧王才人曰我死汝當如何對曰願從陛下於九泉武宗以巾授之武宗崩才人即縊【武宗之問王才人之死懲楊妃之禍也】上聞而矜之贈貴妃葬於端陵柏城之内【考異曰蔡京王貴妃傳曰帝疾亟才人久視帝而歸燕息處濃粧潔服如常日乃取所翫用物散與内家淨盡持帝所授巾至帝前已見升遐容易自縊而仆於御座下以縊為名而得卒舊紀武宗葬端陵德妃王氏祔焉李德裕獻替記自上臨御王妃有專房之寵至是以嬌妬忤旨一夕而殞羣情無不驚懼以謂上功成之後喜怒不測德裕因以進諫在五年十月與王貴妃傳不同恐獻替記誤康軿劇談録曰孟才人善歌有寵於武宗屬一旦聖體不豫召而問之曰我或不諱汝將何之對曰若陛下萬歲之後無復生為是口令於御前歌河滿子一曲聲調悽咽聞者涕零及宫車晏駕哀慟數日而殞窆於端陵之側此事恐正是王才人傳聞不同】以循州司馬牛僧孺為衡州長史封州流人李宗閔為郴州司馬恩州司馬崔珙為安州長史【安州漢安陸縣地京師東南二千五十一里】潮州刺史楊嗣復為江州刺史昭州刺史李珏為郴州刺史僧孺等五相皆武宗所貶逐【楊嗣復貶見二百四十六卷元年三年崔珙罷相崔鉉代之奏珙妄費宋滑院鹽鐵錢九十萬緡又劾與劉從諫厚數護其姦貶澧州刺史再斥恩州司馬僧孺宗閔貶見上四年】至是同日北遷宗閔未離封州而卒【離力智翻】 九月以荆南節度使李德裕為東都留守解平章事以中書侍郎同平章事鄭肅同平章事充荆南節度使 以兵部侍郎判度支盧商為中書侍郎同平章事商翰之族孫也【盧翰相德宗於興元貞元之間】 冊戛斯可汗使者以國喪未行或以為僻遠小國不足與之抗衡回鶻未平不應遽有建置詔百官集議事遂寢 蠻寇安南經畧使裴元裕帥鄰道兵討之【帥讀曰率】 以右常侍李景讓為浙西觀察使【右常侍右散騎常侍也】初景讓母鄭氏性嚴明早寡家貧居於東都諸子皆幼母自教之宅後古牆因雨隤陷【隤杜回翻下墜也】得錢盈船奴婢喜走告母母往焚香祝之曰吾聞無勞而獲身之災也天必以先君餘慶矜其貧而賜之則願諸孤他日學問有成乃其志也此不敢取遽命掩而築之三子景讓景温景莊皆舉進士及第景讓官達髮已斑白小有過不免捶楚【捶止橤翻】景讓在浙西有左都押牙迕景讓意【迕五故翻】景讓杖之而斃軍中憤怒將為變母聞之景讓方視事母出坐聽事【聽讀曰廳】立景讓於庭而責之曰天子付汝以方面國家刑法豈得以為汝喜怒之資妄殺無罪之人乎萬一致一方不寧豈惟上負朝廷使垂年之母銜羞入地【垂末垂也垂年猶言末垂之年】何以見汝之先人乎命左右褫其衣坐之【褫丑豸翻】將撻其背將佐皆為之請【為于偽翻】拜且泣久乃釋之軍中由是遂安景莊老於場屋【唐人謂貢院為場屋至今猶然】每被黜母輒撻景讓然景讓終不肯屬主司【屬之欲翻主司校文主司也禮部侍郎知貢舉者是也】曰朝廷取士自有公道豈敢效人求關節乎久之宰相謂主司曰李景莊今歲不可不收可憐彼翁每歲受撻由是始及第 冬十月禮院奏禘祭祝文於穆敬文武四室但稱嗣皇帝臣某昭告從之【太常有禮院帝於穆宗弟也於敬文武叔也】 甲申上受三洞法籙於衡山道士劉玄靜【既杖殺趙歸真而復受法籙所謂尤而效之會昌五年劉玄靜還衡山】 十二月戊辰朔日有食之<br />
<br />
  宣宗元聖至明成武獻文睿智章仁神聰懿道大孝皇帝上【諱怡即位改名忱憲宗第十三子按通鑑書唐諸帝號自玄宗以後皆以葬陵諡冊為正宣宗諡聖武獻文孝皇帝若元聖至明成武獻文睿智章仁神聰懿道大孝則咸通十三年追崇之號也】<br />
<br />
  大中元年春正月甲寅上祀圓丘赦天下改元 二月加盧龍節度使張仲武同平章事賞其破回鶻也【石雄獨非破回鶻者乎】 癸未上以旱故減膳徹樂出宫女縱鷹隼【隼聳尹翻】止營繕命中書侍郎同平章事盧商與御史中丞封敖疎理京城繫囚大理卿馬植奏稱盧商等務行寛宥凡抵極法一切免死彼官典犯贓及故殺人平日大赦所不免今因疎理而原之使貪吏無所懲畏死者衘寃無告恐非所以消旱災致和氣也昔周饑克殷而年豐【左傳甯莊子之言為討邢也】衛旱討邢而雨降是則誅罪戮姦式合天意雪寃决滯乃副聖心也乞再加裁定詔兩省五品以上議之【兩省五品以上官自給事中中書舍人以上也】 初李德裕執政引白敏中為翰林學士【見二百四十六卷會昌二年】及武宗崩德裕失勢敏中乘上下之怒竭力排之使其黨李咸訟德裕罪【考異曰實錄白敏中令狐綯在會昌中德裕不以朋黨疑之置之臺閣及德裕失勢抵掌戟手同謀斥逐而】<br />
<br />
  【崔鉉亦以會昌末罷相怨德裕大中初敏中復薦鉉在中書乃令其黨人李咸者訟德裕輔政時隂事罷德裕留守以太子少保分司東都按舊傳綯以大中二年自湖州刺史入知制誥鉉以三年自河中節度使入為相此時未也實録誤 今按通鑑所書令狐綯知制誥在是年六七月之間湖州刺史有前字】德裕由是自東都留守以太子少保分司【分司東都也】左諫議大夫張鷺等上言陛下以旱理繫囚慮有寃滯今所原死罪無寃可雪恐凶險僥倖之徒常思水旱為災宜如馬植所奏詔從之皆論如法以植為刑部侍郎充鹽鐵轉運使植素以文學政事有名於時李德裕不之重及白敏中秉政凡德裕所薄者皆不次用之以盧商為武昌節度使以刑部尚書判度支崔元式為門下侍郎翰林學士戶部侍郎韋琮為中書侍郎並同平章事 閏月敕應會昌五年所廢寺有僧能營葺者聽自居之有司毋得禁止是時君相務反會昌之政【相息亮翻】故僧尼之弊皆復其舊【觀通鑑所書則會昌大中之是非可見矣】 己酉積慶太后蕭氏崩【蕭后文宗之母也武宗時徙居積慶殿故以稱之】 五月幽州節度使張仲武大破諸奚 吐蕃論恐熱乘武宗之喪誘党項及回鶻餘衆寇河西【誘音酉】詔河東節度使王宰將代北諸軍擊之【代北諸軍謂陘嶺以北諸軍也】宰以沙陀朱邪赤心為前鋒自麟州濟河與恐熱戰於鹽州破走之 六月以鴻臚卿李業為冊戛斯英武誠明可汗使 上謂白敏中曰朕昔從憲宗之喪道遇風雨百官六宫四散避去惟山陵使長而多髯【髯如占翻】攀靈駕不去誰也對曰令狐楚上曰有子乎對曰長子緒今為隨州刺史上曰堪為相乎對曰緒少病風痺【少詩照翻痺必至翻脚冷濕病也】次子綯前湖州刺史有才器上即擢為考功郎中知制誥綯入謝上問以元和故事綯條對甚悉【綯徒刀翻悉詳也】上悦遂有大用之意【為令狐綯抦用張本】 秋八月丙申以門下侍郎同平章事李回同平章事充西川節度使 葬貞獻皇后於光陵之側【積慶蕭后諡貞獻】 上敦睦兄弟作雍和殿於十六宅【會要是年勅親親樓號雍和殿别造屋宇廊舍七百間宋白曰雍和殿在睦親院】數臨幸置酒作樂擊毬盡歡【數所角翻】諸王有疾常親至臥内存問憂形於色 突厥掠漕米及行商振武節度使史憲忠擊破之 【考異曰按突厥亡巳久蓋猶有餘種在振武之北者余謂此突厥餘種保塞内屬者也】  九月丁卯以金吾大將軍鄭光為平盧節度使光潤州人太后之弟也乙酉前永寧尉吳汝納訟其弟湘罪不至死李紳與<br />
<br />
  李德裕相表裏欺罔武宗枉殺臣弟乞召江州司戶崔元藻等對辨【吳湘死見上卷武宗會昌五年】丁亥敕御史臺鞠實以聞【鞠實窮治其實也】冬十二月庚戌御史臺奏據崔元藻所列吳湘寃狀如吳汝納之言戊午貶太子少保分司李德裕為潮州司馬 吏部奏會昌四年所減州縣官内復增三百八十三員【讀者至此以減者為是邪以於既減之後而復增者為是邪】<br />
<br />
  二年正月甲子羣臣上尊號曰聖敬文思和武光孝皇帝【思相吏翻】赦天下 初李德裕執政有薦丁柔立清直可任諫官者德裕不能用上即位柔立為右補闕德裕貶潮州柔立上疏訟其寃丙寅坐阿附貶南陽尉【史言丁柔立有是非之心南陽縣漢南陽郡所治宛縣地也隋改為南陽縣唐屬鄧州九域志在州東北一百二十里】西川節度使李回桂管觀察使鄭亞坐前不能直吳湘寃乙酉回左遷湖南觀察使亞貶循州刺史李紳追奪三任告身【李紳已薨故追奪】中書舍人崔嘏坐草李德裕制不盡言其罪己丑貶端州刺史 回鶻遏捻可汗仰給於奚王石舍朗【仰牛向翻】及張仲武大破奚衆【見去年五月】回鶻無所得食日益耗散至是所存貴人以下不滿五百人依於室韋使者入賀正【此回鶻使者也】過幽州張仲武使歸取遏捻等遏捻聞之夜與妻葛禄子特勒毒斯等九騎西走餘衆追之不及相與大哭室韋分回鶻餘衆為七七姓共分之【室韋有嶺西部山北部黄頭部如者部婆萵部訥北部駱丹部凡七姓悉居柳城東北近者三千里遠者六千里而贏】居二日戛斯遣其相阿播帥諸胡兵號七萬來取回鶻【帥讀曰率】大破室韋悉收回鶻餘衆歸磧北猶有數帳潛竄山林鈔盜諸胡【鈔楚交翻】其别部厖勒先在安西亦自稱可汗居甘州總磧西諸城種落微弱時入獻見【見賢遍翻回鶻至五季時入獻見者皆厖勒種類也種章勇翻】 二月庚子以知制誥令狐綯為翰林學士上嘗以太宗所撰金鏡【金鏡書太宗所著也】授綯使讀之至亂未嘗不任不肖至治未嘗不任忠賢【治直吏翻】上止之曰凡求致太平當以此言為首又書貞觀政要於屏風每正色拱手而讀之【觀古玩翻】上欲知百官名數令狐綯曰六品已下官卑數多皆吏部注擬五品以上則政府制授各有籍命曰具員上命宰相作具員御覽五卷上之【上之時掌翻】常寘於案上 立皇子澤為濮王上欲作五王院於大明宫以處皇子之幼者【處昌呂翻】召術士柴嶽明使相其地嶽明對曰臣庶之家遷徙不常故有自陽宅入隂宅隂宅入陽宅刑克禍福師有其說【隂陽家所謂三刑謂寅刑巳巳刑申申刑寅丑刑戌戌刑未未刑丑子刑卯卯刑子辰刑辰午刑午酉刑酉亥刑亥克謂金克木木克土土克水水克火火克金】今陛下深拱法宫【如淳曰法宫路寢正殿也】萬神擁衛隂陽書本不言帝王家上善其言賜束帛遣之 夏五月己未朔日有食之 門下侍郎同平章事崔元式罷為戶部尚書以兵部侍郎判度支戶部周墀刑部侍郎鹽鐵轉運使馬植并同平章事【并當作並】初墀為義成節度使辟韋澳為判官及為相謂澳曰力小任重何以相助澳曰願相公無權墀愕然不知所謂澳曰官賞刑罰與天下共其可否勿以己之愛憎喜怒移之天下自理何權之有墀深然之澳貫之之子也【澳烏到翻韋貫之元和中為相】 己卯太皇太后郭氏崩于興慶宫六月禮院檢討官王皥貶句容令【唐太常寺有禮院修撰檢討官各一員宋白曰貞元九年四月敕太常寺宜署禮院修撰檢討官各一員使為定額句容縣屬昇州宋白曰句容縣本漢縣以界内茅山本名句曲山因立名】初憲宗之崩上疑郭太后預其謀又鄭太后本郭太后侍兒有宿怨故上即位待郭太后禮殊薄郭太后意怏怏一日登勤政樓【即玄宗所起勤政務本之樓在興慶宫】欲自隕上聞之大怒是夕崩外人頗有異論上以鄭太后故不欲以郭后袝憲宗有司請葬景陵外園皥奏宜合葬景陵神主配憲宗室奏入上大怒白敏中召皥詰之皥曰太皇太后汾陽王之孫【郭子儀封汾陽王】憲宗在東宫為正妃逮事順宗為婦憲宗厭代之夕事出曖昧太皇太后母天下歷五朝【五朝穆敬文武宣】豈得以曖昧之事遽廢正嫡之禮乎敏中怒甚皥辭氣愈厲諸相會食周墀立於敏中之門以俟之敏中使謝曰方為一書生所苦公弟先行【弟與第同】墀入至敏中廳問其事見皥爭辨方急墀舉手加顙歎皥孤直明日皥坐貶官 【考異曰實録五月戊寅以太皇太后寢疾權不聽政宰臣帥百寮問太后起居己卯復問起居下遺令是日太后崩初上纂位以憲宗遇弑頗疑后在黨中至是暴得疾崩帝之志也甲申白敏中帥百寮上表請聽政不許乙酉又上表不許丙戌三上表乃依六月貶禮院檢討官王皥為潤州句容令舊傳曰宣宗繼統即后之諸子也思禮愈異於前朝大中年崩袝景陵后歷位七朝五居太母之尊人君行子孫之禮福夀隆貴四十餘年雖漢之馬鄧無以加焉識者以為汾陽社稷之功未泯復鍾慶于懿安焉裴延裕東觀奏記曰憲宗皇帝晏駕之夕上雖幼頗記其事追恨光陵商臣之酷即位後誅鉏惡黨無漏網者郭太后以上英察孝果且懷慙懼時居興慶宫一日與一二侍兒同升勤政樓倚衡而望便欲殞於樓下欲成上過左右急持之即聞于上上大怒其夕大后暴崩上志也又曰懿安郭太后既崩喪服許如故事禮院檢討官王皥抗疏請后合葬景陵配饗憲宗廟室既入上大怒宰臣白敏中召皥詰其事皥對云云翌日皥貶潤州句容縣令周墀亦免相按實錄所言暴崩事皆出于東觀奏記若實有此事則既云是夕暴崩何得前一日先下詔云以太后寢疾權不聽政若無此事則廷裕豈敢輒誣宣宗或者郭后實以病終而宣宗以平日疑忿之心欲黜其禮故皥爭之疑以傳疑今參取之東觀奏記又曰杜悰通貴日久門下有術士姓李悰任西川節度使馬植罷黔中赴闕至西川李術士一見植謂悰曰馬中丞非常人也相公厚遇之悰未之信術士一日密言於悰曰相公將有甚禍非馬中丞不能救乞厚結之悰始驚信日厚幣贈之仍令邸吏為植買宅生生之費無闕焉植至門方知感悰不知其旨尋除光禄卿報狀至蜀悰謂術士曰貴人到闕作光禄勲矣術士曰姑待之稍進大理卿又遷刑部侍郎充諸道鹽鐵使悰始驚憂俄而作相懿安皇太后崩後悰懿安子壻也忽一日内榜子索檢責宰相元載故事植諭旨翌日延英上前萬端營救植素辯能回上旨事遂中寢按植會昌中已自黔中入為大理卿悰今年二月始為西川節度今不取 按裴延裕後作廷裕必有一誤】 秋九月甲子再貶潮州司馬李德裕為崖州司戶湖南觀察使李回為賀州刺史【崖州去京師七千四百六十里賀州京師東南四千一百三十里】 前鳳翔節度使石雄詣政府自陳黑山烏嶺之功【政府即謂政事堂黑山烏嶺功並見上卷武宗會昌三年 考異曰此出范櫖雲谿友議彼以烏嶺為天井誤也】求一鎮以終老執政以雄李德裕所薦曰曏日之功朝廷以蒲孟岐三鎮酬之足矣【蒲河中孟河陽岐鳳岐】除左龍武統軍雄怏怏而薨 十一月庚午萬夀公主適起居郎鄭顥顥絪之孫【鄭絪為相于元和之初】登進士第為校書郎右拾遺内供奉以文雅著稱公主上之愛女故選顥尚之有司循舊制請用銀裝車上曰吾欲以儉約化天下當自親者始令依外命婦以銅裝車【唐制公主乘厭翟車外命婦一品乘白銅厭犢車】詔公主執婦禮皆如臣庶之法戒以毋得輕夫族毋得預時事又申以手詔曰苟違吾戒必有太平安樂之禍【樂音洛】顥弟顗嘗得危疾上遣使視之還問公主何在曰在慈恩寺觀戲場上怒歎曰我怪士大夫家不欲與我家為昏良有以也亟命召公主入宫立之階下不之視公主懼涕泣謝罪上責之曰豈有小郎病不往省視乃觀戲乎【自晉以來嫂謂叔為小郎省悉景翻】遣歸鄭氏由是終上之世貴戚皆兢兢守禮法如山東衣冠之族 壬午葬懿安皇后於景陵之側【非禮也憲宗不為正其始以致宣宗不為正其終】以中書侍郎同平章事韋琮為太子賓客分司 十<br />
<br />
  二月鳳翔節度使崔珙奏破吐蕃克清水清水先隸秦州【宋白曰清水漢舊縣其地即秦仲始所封九域志清水縣在秦州九十里宋白曰長興中移清水縣于上邽鎮九域志之清水長興所移也】詔以本州未復權隸鳳翔 上見憲宗朝公卿子孫多擢用之刑部員外郎杜勝次對上問其家世對曰臣父黄裳首請憲宗監國【事見二百三十六卷永貞元年朝直遥翻】即除給事中翰林學士裴諗度之子也上幸翰林面除承旨【諗式荏翻以裴度相元和之功自足以賞延于世但翰林學士承旨非賞功之官耳】吐蕃論恐熱遣其將莽羅急藏將兵二萬畧地西鄙尚婢婢遣其將拓拔懷光擊之於南谷大破之急藏降【降戶江翻】<br />
<br />
  三年春正月上與宰相論元和循吏孰為第一周墀曰臣嘗守土江西聞觀察使韋丹功德被於八州【被皮義翻八州洪江鄂岳䖍吉袁撫也】沒四十年老稚歌思【稚直吏翻】如丹尚存乙亥詔史館修撰杜牧撰丹遺愛碑以紀之仍擢其子河陽觀察判官宙為御史 二月吐蕃論恐熱軍於河州尚婢婢軍于河源軍【河源軍在鄯州東宋白曰河源軍置在湟州東西本趙充國亭堠也】婢婢諸將欲擊恐熱婢婢曰不可我軍驟勝而輕敵彼窮困而致死戰必不利諸將不從婢婢知其必敗據河橋以待之諸將果敗婢婢收餘衆焚橋歸鄯州【據河橋則兵敗而退者有歸路敗兵既度焚橋阻河則可以截論恐熱之追掩史言尚婢婢善兵】 吐蕃秦原安樂三州及石門等七關來降【高宗時吐谷渾為吐蕃所逼徙于鄯州不安其居又徙于靈州之境咸亨三年以靈州故鳴沙縣地置安樂州以居之安史之亂吐蕃取安樂州吐谷渾又徙朔方河東之境原州界冇石門驛藏制勝石峽木靖木峽六盤七關 考異曰實録涇原節度使康季榮奏吐蕃宰相論恐熱殺東道節度使奉表以三州七關來降獻祖紀年錄亦云殺東道節度使奉表國史敘論恐熱事甚詳至五年五月始來降此際未降也又不云殺東道節度使且恐熱若以三州七關來降朝廷必官賞之何故但賞邊將而不及恐熱蓋三州七關以吐蕃國亂自來降唐朝廷遣諸道應接之非恐熱帥以來實錄誤耳】以太僕卿陸耽為宣諭使詔涇原靈武鳳翔邠寧振武皆出兵應接 河東節度使王宰入朝以貨結權倖求以使相領宣武刑部尚書同平章事周墀上疏論之宰遂還鎮駙馬都尉韋讓求為京兆尹墀言京兆尹非才望不可為讓議竟寢墀又諫上開邊【開邊謂經畧河西也】由是忤旨【忤五故翻】夏四月以墀為東川節度使以御史大夫崔鉉為中書侍郎同平章事兵部侍郎判戶部魏扶同平章事 癸巳盧龍奏節度使張仲武薨軍中立其子節度押牙直方 翰林學士鄭顥言于上曰周墀以直言入相亦以直言罷相上深感悟甲午墀入謝加檢校右僕射 戊戌以張直方為盧龍留後 五月徐州軍亂逐節度使李廓廓程之子也【李程見二百四十三卷長慶四年】在鎮不治【治直之翻】右補闕鄭魯上言其狀且曰臣恐新麥未登徐師必亂速命良帥救此一方【帥所類翻】上未之省【省悉景翻】徐州果亂上思魯言擢為起居舍人以義成節度使盧弘止為武寧節度使武寧士卒素驕有銀刀都尤甚屢逐主帥弘止至鎮都虞候胡慶方復謀作亂【復扶又翻】弘止誅之撫循其餘訓以忠義軍府由是獲安 六月戊申以張直方為盧龍節度使 涇原節度使康季榮取原州【原州本治高平安史亂後沒于吐蕃】及石門驛藏木峽制勝六磐石峽六關秋七月丁巳靈武節度使朱叔明取長樂州【長樂當作安樂宋白曰安樂州置於靈州鳴沙縣樂音洛下同】甲子邠寧節度使張君緒取蕭關【蕭關縣舊志屬原州】甲戌鳳翔節度使李玭取秦州【玭蒲蠲翻凡取言易也秦州本治上邽宋白曰時治成紀在舊州南一百里】詔邠寧節度權移軍於寧州以應接河西 八月乙酉改長樂州為威州【宋白曰靈州鳴沙縣本漢富平縣地後周立會州隋立環州以大河環曲為名唐神龍中默啜寇掠移縣於廢豐安城咸通三年歸復以舊縣基置安樂州大中三年改為威州】河隴老幼千餘人詣闕 【考異曰實錄云數千人今從舊傳】己丑上御延喜門樓見之【延喜門在皇城東北角六典皇城東面二門北曰延喜南曰景風延喜門則承天門外横街東直通化門】歡呼舞躍解胡服襲冠帶觀者皆呼萬歲詔募百姓懇闢三州七關土田五年不租稅自今京城罪人應配流者皆配十處【十處三州七關也】四道將吏能於鎮戍之地營田者官給牛及種糧【四道涇原邠寜靈武鳳翔宋白曰史臣曰營田之名蓋緣邊多隙地蕃兵鎮戍課其播殖以助軍須謂之屯田其後中原兵興民戶減耗野多閒田而治財賦者如沿邊例開置名曰營田行之歲久不以兵乃招致農民強戶謂之營田戶復有主務敗闕犯法之家沒納田宅亦係于此自此諸道皆有營田務種章勇翻】温池鹽利可贍邊陲委度支制置【神龍元年置温池縣屬靈州是年度屬威州縣有鹽池】其三州七關鎮戍之卒皆倍給衣糧【言衣糧倍于其他戍卒】仍二年一代道路建置堡柵有商旅往來販易及戍卒子弟通傳家信關鎮毋得留難其山南劔南邊境有沒蕃州縣亦令量力收復【廣德以來西羌内侵山南巡内階成陷沒文州移治劒南西山諸州亦多有沒於吐蕃者按階州時為武州宋白曰階州漢武都之地後魏平武都築城於仙陵山置武都鎮西魏始置武州大歷初與秦州俱沒于吐蕃大中三年收復復立武州景歷元年改階州】 冬十月改備邊庫為延資庫【備邊庫初置見上武宗會昌五年】 西川節度使杜悰奏取維州 閏十一月丁酉宰相以克復河湟請上尊號上曰憲宗常有志復河湟【見二百三十八卷元和五年】以中原方用兵【謂方用兵於兩河也】未遂而崩今乃克成先志耳其議加順憲二廟尊諡以昭功烈 盧龍節度使張直方暴忍喜遊獵【喜許記翻】軍中將作亂直方知之託言出獵遂舉族逃歸京師軍中推牙將周綝為留後【綝丑林翻 考異曰舊紀十一月幽州軍亂逐張直方軍人推周綝為留後四年九月周綝卒軍人立張允伸為留後直方傳曰直方多不法慮為將卒所圖三年冬託以遊獵奔赴闕廷張允伸傳曰四年戎帥周綝寢疾表允伸為留後新紀四年八月幽州軍亂逐張直方張允伸自稱留後傳亦言直方出奔即以允伸為留後實錄直方赴闕亦在去年八月至九月又云張允伸知留後皆無周綝姓名今從舊傳】直方至京師拜金吾大將軍 甲戌追上順宗諡曰至德弘道大聖大安孝皇帝憲宗諡曰昭文章武大聖至神孝皇帝仍改題神主【自天寶已來加上諸帝諡號陵中玉冊及神主未嘗改題】 己未崖州司戶李德裕卒 山南西道節度使鄭涯奏取扶州【劉昫曰扶州治同昌縣歷代吐谷渾所據西魏逐吐谷渾於此置鄧州及鄧寧郡蓋以平定鄧至羌為名隋初改置扶州及同昌縣在長安西南一千六百九十里廣德後沒於吐蕃】<br />
<br />
  資治通鑑卷二百四十八  <br>
   </div> 

<script src="/search/ajaxskft.js"> </script>
 <div class="clear"></div>
<br>
<br>
 <!-- a.d-->

 <!--
<div class="info_share">
</div> 
-->
 <!--info_share--></div>   <!-- end info_content-->
  </div> <!-- end l-->

<div class="r">   <!--r-->



<div class="sidebar"  style="margin-bottom:2px;">

 
<div class="sidebar_title">工具类大全</div>
<div class="sidebar_info">
<strong><a href="http://www.guoxuedashi.com/lsditu/" target="_blank">历史地图</a></strong>  
<a href="http://www.880114.com/" target="_blank">英语宝典</a>  
<a href="http://www.guoxuedashi.com/13jing/" target="_blank">十三经检索</a> 
<br><strong><a href="http://www.guoxuedashi.com/gjtsjc/" target="_blank">古今图书集成</a></strong> 
<a href="http://www.guoxuedashi.com/duilian/" target="_blank">对联大全</a> <strong><a href="http://www.guoxuedashi.com/xiangxingzi/" target="_blank">象形文字典</a></strong> 

<br><a href="http://www.guoxuedashi.com/zixing/yanbian/">字形演变</a>  <strong><a href="http://www.guoxuemi.com/hafo/" target="_blank">哈佛燕京中文善本特藏</a></strong>
<br><strong><a href="http://www.guoxuedashi.com/csfz/" target="_blank">丛书&方志检索器</a></strong> <a href="http://www.guoxuedashi.com/yqjyy/" target="_blank">一切经音义</a>  

<br><strong><a href="http://www.guoxuedashi.com/jiapu/" target="_blank">家谱族谱查询</a></strong>  <strong><a href="http://shufa.guoxuedashi.com/sfzitie/" target="_blank">书法字帖欣赏</a></strong> 
<br>

</div>
</div>


<div class="sidebar" style="margin-bottom:0px;">

<font style="font-size:22px;line-height:32px">QQ交流群9:489193090</font>


<div class="sidebar_title">手机APP 扫描或点击</div>
<div class="sidebar_info">
<table>
<tr>
	<td width=160><a href="http://m.guoxuedashi.com/app/" target="_blank"><img src="/img/gxds-sj.png" width="140"  border="0" alt="国学大师手机版"></a></td>
	<td>
<a href="http://www.guoxuedashi.com/download/" target="_blank">app软件下载专区</a><br>
<a href="http://www.guoxuedashi.com/download/gxds.php" target="_blank">《国学大师》下载</a><br>
<a href="http://www.guoxuedashi.com/download/kxzd.php" target="_blank">《汉字宝典》下载</a><br>
<a href="http://www.guoxuedashi.com/download/scqbd.php" target="_blank">《诗词曲宝典》下载</a><br>
<a href="http://www.guoxuedashi.com/SiKuQuanShu/skqs.php" target="_blank">《四库全书》下载</a><br>
</td>
</tr>
</table>

</div>
</div>


<div class="sidebar2">
<center>


</center>
</div>

<div class="sidebar"  style="margin-bottom:2px;">
<div class="sidebar_title">网站使用教程</div>
<div class="sidebar_info">
<a href="http://www.guoxuedashi.com/help/gjsearch.php" target="_blank">如何在国学大师网下载古籍?</a><br>
<a href="http://www.guoxuedashi.com/zidian/bujian/bjjc.php" target="_blank">如何使用部件查字法快速查字?</a><br>
<a href="http://www.guoxuedashi.com/search/sjc.php" target="_blank">如何在指定的书籍中全文检索?</a><br>
<a href="http://www.guoxuedashi.com/search/skjc.php" target="_blank">如何找到一句话在《四库全书》哪一页?</a><br>
</div>
</div>


<div class="sidebar">
<div class="sidebar_title">热门书籍</div>
<div class="sidebar_info">
<a href="/so.php?sokey=%E8%B5%84%E6%B2%BB%E9%80%9A%E9%89%B4&kt=1">资治通鉴</a> <a href="/24shi/"><strong>二十四史</strong></a>&nbsp; <a href="/a2694/">野史</a>&nbsp; <a href="/SiKuQuanShu/"><strong>四库全书</strong></a>&nbsp;<a href="http://www.guoxuedashi.com/SiKuQuanShu/fanti/">繁体</a>
<br><a href="/so.php?sokey=%E7%BA%A2%E6%A5%BC%E6%A2%A6&kt=1">红楼梦</a> <a href="/a/1858x/">三国演义</a> <a href="/a/1038k/">水浒传</a> <a href="/a/1046t/">西游记</a> <a href="/a/1914o/">封神演义</a>
<br>
<a href="http://www.guoxuedashi.com/so.php?sokeygx=%E4%B8%87%E6%9C%89%E6%96%87%E5%BA%93&submit=&kt=1">万有文库</a> <a href="/a/780t/">古文观止</a> <a href="/a/1024l/">文心雕龙</a> <a href="/a/1704n/">全唐诗</a> <a href="/a/1705h/">全宋词</a>
<br><a href="http://www.guoxuedashi.com/so.php?sokeygx=%E7%99%BE%E8%A1%B2%E6%9C%AC%E4%BA%8C%E5%8D%81%E5%9B%9B%E5%8F%B2&submit=&kt=1"><strong>百衲本二十四史</strong></a>  <a href="http://www.guoxuedashi.com/so.php?sokeygx=%E5%8F%A4%E4%BB%8A%E5%9B%BE%E4%B9%A6%E9%9B%86%E6%88%90&submit=&kt=1"><strong>古今图书集成</strong></a>
<br>

<a href="http://www.guoxuedashi.com/so.php?sokeygx=%E4%B8%9B%E4%B9%A6%E9%9B%86%E6%88%90&submit=&kt=1">丛书集成</a> 
<a href="http://www.guoxuedashi.com/so.php?sokeygx=%E5%9B%9B%E9%83%A8%E4%B8%9B%E5%88%8A&submit=&kt=1"><strong>四部丛刊</strong></a>  
<a href="http://www.guoxuedashi.com/so.php?sokeygx=%E8%AF%B4%E6%96%87%E8%A7%A3%E5%AD%97&submit=&kt=1">說文解字</a> <a href="http://www.guoxuedashi.com/so.php?sokeygx=%E5%85%A8%E4%B8%8A%E5%8F%A4&submit=&kt=1">三国六朝文</a>
<br><a href="http://www.guoxuedashi.com/so.php?sokeytm=%E6%97%A5%E6%9C%AC%E5%86%85%E9%98%81%E6%96%87%E5%BA%93&submit=&kt=1"><strong>日本内阁文库</strong></a> <a href="http://www.guoxuedashi.com/so.php?sokeytm=%E5%9B%BD%E5%9B%BE%E6%96%B9%E5%BF%97%E5%90%88%E9%9B%86&ka=100&submit=">国图方志合集</a> <a href="http://www.guoxuedashi.com/so.php?sokeytm=%E5%90%84%E5%9C%B0%E6%96%B9%E5%BF%97&submit=&kt=1"><strong>各地方志</strong></a>

</div>
</div>


<div class="sidebar2">
<center>

</center>
</div>
<div class="sidebar greenbar">
<div class="sidebar_title green">四库全书</div>
<div class="sidebar_info">

《四库全书》是中国古代最大的丛书,编撰于乾隆年间,由纪昀等360多位高官、学者编撰,3800多人抄写,费时十三年编成。丛书分经、史、子、集四部,故名四库。共有3500多种书,7.9万卷,3.6万册,约8亿字,基本上囊括了古代所有图书,故称“全书”。<a href="http://www.guoxuedashi.com/SiKuQuanShu/">详细>>
</a>

</div> 
</div>

</div>  <!--end r-->

</div>
<!-- 内容区END --> 

<!-- 页脚开始 -->
<div class="shh">

</div>

<div class="w1180" style="margin-top:8px;">
<center><script src="http://www.guoxuedashi.com/img/plus.php?id=3"></script></center>
</div>
<div class="w1180 foot">
<a href="/b/thanks.php">特别致谢</a> | <a href="javascript:window.external.AddFavorite(document.location.href,document.title);">收藏本站</a> | <a href="#">欢迎投稿</a> | <a href="http://www.guoxuedashi.com/forum/">意见建议</a> | <a href="http://www.guoxuemi.com/">国学迷</a> | <a href="http://www.shuowen.net/">说文网</a><script language="javascript" type="text/javascript" src="https://js.users.51.la/17753172.js"></script><br />
  Copyright &copy; 国学大师 古典图书集成 All Rights Reserved.<br>
  
  <span style="font-size:14px">免责声明:本站非营利性站点,以方便网友为主,仅供学习研究。<br>内容由热心网友提供和网上收集,不保留版权。若侵犯了您的权益,来信即刪。scp168@qq.com</span>
  <br />
ICP证:<a href="http://www.beian.miit.gov.cn/" target="_blank">鲁ICP备19060063号</a></div>
<!-- 页脚END --> 
<script src="http://www.guoxuedashi.com/img/plus.php?id=22"></script>
<script src="http://www.guoxuedashi.com/img/tongji.js"></script>

</body>
</html>
