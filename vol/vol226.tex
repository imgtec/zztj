<!DOCTYPE html PUBLIC "-//W3C//DTD XHTML 1.0 Transitional//EN" "http://www.w3.org/TR/xhtml1/DTD/xhtml1-transitional.dtd">
<html xmlns="http://www.w3.org/1999/xhtml">
<head>
<meta http-equiv="Content-Type" content="text/html; charset=utf-8" />
<meta http-equiv="X-UA-Compatible" content="IE=Edge,chrome=1">
<title>資治通鑒_227-資治通鑑卷二百二十六_227-資治通鑑卷二百二十六</title>
<meta name="Keywords" content="資治通鑒_227-資治通鑑卷二百二十六_227-資治通鑑卷二百二十六">
<meta name="Description" content="資治通鑒_227-資治通鑑卷二百二十六_227-資治通鑑卷二百二十六">
<meta http-equiv="Cache-Control" content="no-transform" />
<meta http-equiv="Cache-Control" content="no-siteapp" />
<link href="/img/style.css" rel="stylesheet" type="text/css" />
<script src="/img/m.js?2020"></script> 
</head>
<body>
 <div class="ClassNavi">
<a  href="/24shi/">二十四史</a> | <a href="/SiKuQuanShu/">四库全书</a> | <a href="http://www.guoxuedashi.com/gjtsjc/"><font  color="#FF0000">古今图书集成</font></a> | <a href="/renwu/">历史人物</a> | <a href="/ShuoWenJieZi/"><font  color="#FF0000">说文解字</a></font> | <a href="/chengyu/">成语词典</a> | <a  target="_blank"  href="http://www.guoxuedashi.com/jgwhj/"><font  color="#FF0000">甲骨文合集</font></a> | <a href="/yzjwjc/"><font  color="#FF0000">殷周金文集成</font></a> | <a href="/xiangxingzi/"><font color="#0000FF">象形字典</font></a> | <a href="/13jing/"><font  color="#FF0000">十三经索引</font></a> | <a href="/zixing/"><font  color="#FF0000">字体转换器</font></a> | <a href="/zidian/xz/"><font color="#0000FF">篆书识别</font></a> | <a href="/jinfanyi/">近义反义词</a> | <a href="/duilian/">对联大全</a> | <a href="/jiapu/"><font  color="#0000FF">家谱族谱查询</font></a> | <a href="http://www.guoxuemi.com/hafo/" target="_blank" ><font color="#FF0000">哈佛古籍</font></a> 
</div>

 <!-- 头部导航开始 -->
<div class="w1180 head clearfix">
  <div class="head_logo l"><a title="国学大师官网" href="http://www.guoxuedashi.com" target="_blank"></a></div>
  <div class="head_sr l">
  <div id="head1">
  
  <a href="http://www.guoxuedashi.com/zidian/bujian/" target="_blank" ><img src="http://www.guoxuedashi.com/img/top1.gif" width="88" height="60" border="0" title="部件查字,支持20万汉字"></a>


<a href="http://www.guoxuedashi.com/help/yingpan.php" target="_blank"><img src="http://www.guoxuedashi.com/img/top230.gif" width="600" height="62" border="0" ></a>


  </div>
  <div id="head3"><a href="javascript:" onClick="javascript:window.external.AddFavorite(window.location.href,document.title);">添加收藏</a>
  <br><a href="/help/setie.php">搜索引擎</a>
  <br><a href="/help/zanzhu.php">赞助本站</a></div>
  <div id="head2">
 <a href="http://www.guoxuemi.com/" target="_blank"><img src="http://www.guoxuedashi.com/img/guoxuemi.gif" width="95" height="62" border="0" style="margin-left:2px;" title="国学迷"></a>
  

  </div>
</div>
  <div class="clear"></div>
  <div class="head_nav">
  <p><a href="/">首页</a> | <a href="/ShuKu/">国学书库</a> | <a href="/guji/">影印古籍</a> | <a href="/shici/">诗词宝典</a> | <a   href="/SiKuQuanShu/gxjx.php">精选</a> <b>|</b> <a href="/zidian/">汉语字典</a> | <a href="/hydcd/">汉语词典</a> | <a href="http://www.guoxuedashi.com/zidian/bujian/"><font  color="#CC0066">部件查字</font></a> | <a href="http://www.sfds.cn/"><font  color="#CC0066">书法大师</font></a> | <a href="/jgwhj/">甲骨文</a> <b>|</b> <a href="/b/4/"><font  color="#CC0066">解密</font></a> | <a href="/renwu/">历史人物</a> | <a href="/diangu/">历史典故</a> | <a href="/xingshi/">姓氏</a> | <a href="/minzu/">民族</a> <b>|</b> <a href="/mz/"><font  color="#CC0066">世界名著</font></a> | <a href="/download/">软件下载</a>
</p>
<p><a href="/b/"><font  color="#CC0066">历史</font></a> | <a href="http://skqs.guoxuedashi.com/" target="_blank">四库全书</a> |  <a href="http://www.guoxuedashi.com/search/" target="_blank"><font  color="#CC0066">全文检索</font></a> | <a href="http://www.guoxuedashi.com/shumu/">古籍书目</a> | <a   href="/24shi/">正史</a> <b>|</b> <a href="/chengyu/">成语词典</a> | <a href="/kangxi/" title="康熙字典">康熙字典</a> | <a href="/ShuoWenJieZi/">说文解字</a> | <a href="/zixing/yanbian/">字形演变</a> | <a href="/yzjwjc/">金 文</a> <b>|</b>  <a href="/shijian/nian-hao/">年号</a> | <a href="/diming/">历史地名</a> | <a href="/shijian/">历史事件</a> | <a href="/guanzhi/">官职</a> | <a href="/lishi/">知识</a> <b>|</b> <a href="/zhongyi/">中医中药</a> | <a href="http://www.guoxuedashi.com/forum/">留言反馈</a>
</p>
  </div>
</div>
<!-- 头部导航END --> 
<!-- 内容区开始 --> 
<div class="w1180 clearfix">
  <div class="info l">
   
<div class="clearfix" style="background:#f5faff;">
<script src='http://www.guoxuedashi.com/img/headersou.js'></script>

</div>
  <div class="info_tree"><a href="http://www.guoxuedashi.com">首页</a> > <a href="/SiKuQuanShu/fanti/">四库全书</a>
 > <h1>资治通鉴</h1> <!--         下载:【右键另存为】即可 --></div>
  <div class="info_content zj clearfix">
  
<div class="info_txt clearfix" id="show">
<center style="font-size:24px;">227-資治通鑑卷二百二十六</center>
    資治通鑑卷二百二十六 宋 司馬光 撰<br />
<br />
  胡三省 音注<br />
<br />
  唐紀四十二【起屠維恊洽八月盡重光作噩五月凡一年有奇始己未八月終辛酉五月凡二年零十月】<br />
<br />
  代宗睿文孝武皇帝下<br />
<br />
  大歷十四年八月甲辰以道州司馬楊炎為門下侍郎【大歷十二年楊炎以黨元載貶】懷州刺史喬琳為御史大夫並同平章事 【考異曰崔祐甫與炎皆自門下遷中書是時中書在上也憲宗以後門下在上中書在下不知何時遷改】上方勵精求治【治直吏翻】不次用人卜相於崔祐甫【相息亮翻】祐甫薦炎器業上亦素聞其名故自遷謫中用之琳太原人性粗率喜詼諧【粗讀曰麁善許記翻】無它長與張涉善涉稱其才可大用上信涉言而用之聞者無不駭愕 代宗之世吐蕃數遣使求和【吐從暾入聲數所角翻使疏吏翻下同】而寇盜不息代宗悉留其使者前後八輩有至老死不得歸者俘獲其人皆配江嶺【使疏吏翻俘方無翻江謂大江之南嶺謂五嶺之外】上欲以德懷之乙巳以隨州司馬韋倫為太常少卿使于吐蕃悉集其俘五百人各賜襲衣而遣之【少始照翻襲衣衣一襲也衣一稱為一襲】恊律郎沈既濟上選舉議【唐制恊律郎掌和律呂辯四時之氣八風五音之節】<br />
<br />
  【屬太常寺正八品上上時掌翻】以為選用之法三科而已曰德也曰才也勞也今選曹皆不及焉【選須絹翻下同】考校之法皆在書判簿歷言詞俯仰而已【唐擇人之法有四曰身言書判身取其體貌豐偉言取其言詞辯正書取其楷法遒美判取其文理優長簿歷所以著其資考殿最俯仰則觀諸身言之間】夫安行徐言非德也麗藻芳翰非才也累資積考非勞也執此以求天下之士固未盡矣今人未土著【夫音扶著直畧翻】不可本於鄉閭鑒不獨明不可專於吏部臣謹詳酌古今謂五品以上及羣司長官宜令宰臣進叙吏部兵部得參議焉【長知兩翻】其六品以下或僚佐之屬許州府辟用其牧守將帥【守手又翻將即亮翻帥所類翻】或選用非公則吏部兵部得察而舉之罪其私冒不慎舉者小加譴黜大正刑典責成授任誰敢不勉夫如是則賢者不奬而自進不肖者不抑而自退衆才並進而官無不治矣今選法皆擇才於吏部試能於州郡若才職不稱【稱尺證翻】紊亂無任【紊文運翻任音壬】責於刺史則曰命官出於吏曹不敢廢也責於侍郎則曰量書判資考而授之不保其往也【量音良】責於令史則曰按由歷出入而行之不知其它也黎庶徒弊誰任其咎若牧守自用則罪將焉逃必州郡之濫獨換一刺史則革矣如吏部之濫雖更其侍郎無益也【焉於䖍翻更工衡翻】蓋人物浩浩不可得而知法使之然非主司之過今諸道節度都團練觀察租庸等使自判官副將以下皆使自擇縱其間或有情故大舉其例十猶七全則辟吏之法已試於今但未及於州縣耳利害之理較然可觀曏令諸使僚佐盡受於選曹則安能鎮方隅之重理財賦之殷乎既濟吴人也【等使疏吏翻下諸使同將即亮翻令力丁翻選須絹翻】 初衡州刺史曹王臯有治行【治直吏翻行下孟翻衡州治衡陽縣屬湖南觀察】湖南觀察使辛京杲疾之【大歷五年辛京杲為湖南觀察使】䧟以法貶潮州刺史【度嶺為貶】時楊炎在道州知其直及入相復擢為衡州刺史【相息亮翻復扶又翻】始臯之遭誣在治【在治者謂獄吏治其事臯以囚服在列】念太妃老將驚而戚出則囚服就辯入則擁笏垂魚【唐高宗給五品以上隨身魚銀袋以防召命之詐三品以上金飾袋天授二年改佩魚為龜中宗罷龜復給以魚郡王嗣王亦佩金魚袋】即貶于潮【即就也】以遷入賀及是然後跪謝告實臯明之玄孫也【曹王明太宗之子】 朔方邠寧節度使李懷光既代郭子儀邠府宿將史抗温儒雅龎仙鶴張獻明李光逸功名素出懷光右皆怏怏不服【邠卑旻翻龐都江翻怏於兩翻】懷光發兵防秋屯長武城軍期進退不時應令監軍翟文秀勸懷光奏令宿衛既離營【監工銜翻翟萇伯翻奏令力丁翻離力智翻】使人追捕誣以它罪且曰黄萯之敗【黄萯敗事見二百二十四卷九年萯音倍】職爾之由盡殺之 九月甲戍改淮西曰淮寧 西川節度使同平章事崔寧在蜀十餘年【永泰元年崔旰入成都至是十四年矣】恃地險兵彊恣為淫侈朝廷患之而不能易至是入朝加司空兼山陵使南詔王閤羅鳳卒子鳳迦異前死孫異牟尋立【朝直遥翻使疎吏翻卒子恤翻迦音加】冬十月丁酉朔吐蕃與南詔合兵十萬三道入寇一出茂州一出扶文【吐從暾入聲文州漢隂平之地隋為曲水縣義寧三年分武都之曲水正西長松置文州扶州古鄧至地後周天和中置扶州舊本置龍涸防與隂平接界盖吐蕃出扶文南詔出黎雅也】一出黎雅【黎州之地漢屬越嶲郡界隋置漢源縣武后大足元年置黎州黎州漢沉黎縣雅州漢嚴道縣境相接也 考異曰建中實録裴垍德宗實録此月吐蕃三道入寇皆在梁益之境而來年四月乃云去冬吐蕃三道來侵一自靈武一自山南一自蜀又云贊普謂韋倫曰今靈武之師聞命輟矣而山南已入扶文蜀師已趣灌口追且不及與此自相違今不取】曰吾欲取蜀以為東府崔寧在京師所留諸將不能禦【將即亮翻】虜連䧟州縣刺史弃城走士民竄匿山谷上憂之趣寧歸鎮【趣讀曰促】寧已辭楊炎言於上曰蜀地富饒寧據有之朝廷失其外府十四年矣寧雖入朝全師尚守其後貢賦不入與無蜀同且寧本與諸將等夷因亂得位威令不行今雖遣之必恐無功若其有功則義不可奪是蜀地敗固失之勝亦不得也願陛下熟察上曰然則柰何對曰請留寧發朱泚所領范陽兵數千人雜禁兵往擊之何憂不克【泚且禮翻又音此】因而得内親兵於其腹中蜀將必不敢動然後更授他帥【將即亮翻更工衡翻帥所類翻】使千里沃壤復為國有【復扶又翻又音如字】是因小害而收大利也上曰善遂留寧初馬璘忌涇原都知兵馬使李晟功名遣入宿衛為右神策都將【璘離珍翻使疏吏翻晟成正翻將即亮翻】上發禁兵四千人使晟將之發邠隴范陽兵五千【邠隴邠寧隴右二鎮之兵也將即亮翻又音如字下同】使金吾大將軍安邑曲環將之以救蜀【史炤曰曲姓也漢有代郡太守曲謙】東川出兵自江油趨白垻【江油漢魏為無人之地晉始置平武縣唐改為江油縣帶龍州利州管下景谷縣西北有白垻鎮城垻必駕翻蜀人謂平川為垻】與山南兵合擊吐蕃南詔破之范陽兵追及於七盤【吐從暾入聲七盤縣屬巴州武后久視元年置】又破之遂克維茂二州李晟追擊於大度河外【大度河在雅州盧山縣寰宇記大度河自吐蕃界經雅州諸部落至黎州東界流入通望界於黎州為南邊要害之地】又破之吐蕃南詔饑寒隕於崖谷死者八九萬人吐蕃悔怒殺誘導使之來者異牟尋懼築苴哶城【自瀘州南渡瀘水六百五十里至羊苴哶城舊史羊苴咩城南去大和城十餘里東北至成都二千四百里去雲南城三百里誘羊久翻咩莫者翻又徐婢翻史炤曰苴音酢又徐嗟切咩音養又彌嗟切薛能聞官軍破吉浪詩越嶲通遊客苴咩閙聚蚊又西縣塗中野色生肥芉鄉儀搗散茶梯航經杜宇烽火徹苴咩】延袤十五里徙居之吐蕃封之為日東王 上用法嚴百官震悚以山陵近禁人屠宰郭子儀之隸人潜殺羊載以入城【隸人僕隸之屬】右金吾將軍裴諝奏之或謂諝曰郭公有社稷大功君獨不為之地乎諝曰此乃吾所以為之地也郭公勲高望重上新即位以為羣臣附之者衆【德宗之猜忌裴諝于其初政已窺見之諝私呂翻】吾故發其小過以明郭公威權不足畏也如此上尊天子下安大臣不亦可乎 己酉葬睿文孝武皇帝于元陵【元陵在京兆富平縣西北二十五里檀山】廟號代宗將發引上送之見輼輬車不當馳道稍指丁未之間【引羊晉翻輼音温輬音凉 考異曰按車指丁未之間則行出道外矣盖出門欲斜就道西不當道之中間行耳】問其故有司對曰陛下本命在午不敢衝也上哭曰安有枉靈駕而謀身利乎命改轅直午而行肅宗代宗皆喜隂陽鬼神【喜許切記】事無大小必謀之卜祝故王嶼黎幹皆以左道得進上雅不之信【嶼音余雅素也】山陵但取七月之期【禮天子七月而葬】事集而發不復擇日【復扶又翻下謀復同】 十一月丁丑以晉州刺史韓滉為蘇州刺史浙江東西觀察使【滉呼廣翻使疏吏翻】 喬琳衰老耳聵【聵五怪翻】上或時訪問應對失次所謀議復疎闊壬午以琳為工部尚書罷政事上由是疎張涉【喬琳涉所薦也尚辰羊翻】 楊炎既留崔寧二人由是交惡炎託以北邊須大臣鎮撫癸巳以京畿觀察使崔寧為單于鎮北大都護朔方節度使鎮坊州【單音蟬】以荆南節度使張延賞為西川節度使又以靈鹽節度都虞候醴泉杜希全知靈鹽州留後代州刺史張光晟知單于振武等城綏銀麟勝州留後 【考異曰舊傳云王雄為振武今從實録】延州刺史李建徽知鄜坊丹州留後時寧既出鎮不當更置留後炎欲奪寧權且窺其所爲令三人皆得自奏事仍諷之使伺寧過失【令力丁翻伺相吏翻 考異曰舊傳初寧代喬琳為御史大夫平章事寧以為選擇御史當出大夫不謀及宰相乃奏請以李衡于結等數人為御史楊炎大怒其狀遂寢炎又數讒毁劉晏寧又救解之因此大怒其年十月南蠻大至上遣寧還鎮炎懼怨已入蜀難制奏止之按寧為御史大夫在吐蕃南蠻寇蜀後舊傳恐悮】 十二月乙卯立宣王誦為皇太子 舊制天下金帛皆貯於左藏太府四時上其數比部覆其出入【唐制太府掌廪藏財貨出納比部掌句會蜀注曰唐制天下財賦皆納於左藏太府四時以數聞尚書比部覆校其出入貯丁呂翻藏狙浪翻上時掌翻比音毘】及第五琦為度支鹽鐵使【琦音奇度徒洛翻使疎吏翻】時京師多豪將求取無節琦不能制【將即亮翻】乃奏盡貯於大盈内庫【百寶大盈庫始於玄宗朝詳見二百二十八卷德宗建中四年十月注】使宦官掌之天子亦以取給為便故久不出由是以天下公賦為人君私藏有司不復得窺其多少校其贏縮【藏徂浪翻復扶又翻贏有餘也縮不足也】殆二十年宦官領其事者三百餘員皆蠶食其中蟠結根據牢不可動楊炎頓首於上前曰財賦者國之大本生民之命重輕安危靡不由之是以前世皆使重臣掌其事猶或耗亂不集【耗當讀曰眊或讀如字】今獨使中人出入盈虛大臣皆不得知政之蠧敝莫甚於此請出之以歸有司度宫中歲用幾何【度徒洛翻】量數奉入【量音良】不敢有乏如此然後可以為政上即日下詔凡財賦皆歸左藏一用舊式歲於數中擇精好者三五千匹進入大盈【下遐稼翻 考異曰德宗實録作三五十萬匹今從建中實録】炎以片言移人主意議者稱之 丙寅晦日有食之 湖南賊帥王國良阻山為盗【帥所類翻】上遣都官員外郎關播招撫之【唐都官郎掌俘隸簿録給衣糧醫藥而理其訴寃】辭行上問以為政之要對曰為政之本必求有道賢人與之為理上曰朕比以下詔求賢【比毘至翻近以當作己】又遣使臣廣加搜訪庶幾可以為理乎【使疎吏翻幾居希翻】對曰下詔所求及使者所薦惟得文詞干進之士耳安有有道賢人肯隨牒舉選乎上悦 崔祐甫有疾上令肩輿入中書或休假在第【令力丁翻假古訝翻】大事令中使咨决<br />
<br />
  德宗神武孝文皇帝一【諱适代宗長子也謚法諫爭不威曰德言不以威拒諫也執義揚善曰德言稱人之善也】<br />
<br />
  建中元年春正月丁卯朔改元羣臣上尊號曰聖神文武皇帝【上時掌翻】赦天下始用楊炎議命黜陟使與觀察刺史約百姓丁產定等級改作兩税法【楊炎作兩税法夏輸無過六月秋輸無過十一月視大歷十四年墾田數為定】比來新舊徵科色目一切罷之【比毘至翻比來猶云近來也】二税外輒率一錢者以枉法論唐初賦歛之法曰租庸調有田則有租有身則有庸有戶則有調玄宗之末版籍浸壞多非其實及至德兵起所在賦歛迫趣取辦【歛力贍翻調徒弔翻趣讀曰促】無復常凖【復扶又翻又音如字】賦歛之司增數而莫相統攝【統他綜翻俗從上聲】各隨意增科自立色目新故相仍不知紀極民富者丁多率為官為僧以免課役而貧者丁多無所伏匿故上戶優而下戶勞吏因緣蠶食旬輸月送不勝困弊【勝音升】率皆逃徙為浮戶其土著百無四五【著直畧翻】至是炎建議作兩税法先計州縣每歲所應費用及上供之數而賦於人量出以制入戶無主客以見居為簿人無丁中以貧富為差【州縣有主戶客戶天寶三載令民十八以上為中男二十三以上成丁量音良見賢遍翻】為行商者在所州縣税三十之一使與居者均無僥利【言居行皆無僥幸之利也僥堅堯翻】居人之税秋夏兩徵之其租庸調雜徭悉省皆總統於度支上用其言因敕令行之 初左僕射劉晏為吏部尚書楊炎為侍郎不相悦【射寅謝翻尚辰羊翻】元載之死晏有力焉【事見上卷代宗大歷十三年載祖亥翻又如字】及上即位晏久典利權衆頗疾之多上言轉運使可罷【多上時掌翻使疎吏翻】又有風言晏嘗密表勸代宗立獨孤妃為皇后者【風言謂得於風聞而言之者也】楊炎為宰相欲為元載報仇因為上流涕言晏與黎幹劉忠翼同謀【幹忠翼死於大歷十四年事見上卷為于偽翻相息亮翻】臣為宰相不能討罪當萬死崔祐甫言茲事曖昩陛下已曠然大赦不當復䆒尋虛語【曖音愛復扶又翻】炎乃建言尚書省國政之本比置諸使分奪其權【尚辰羊翻比毘至翻】今宜復舊上從之甲子【按是月無甲子恐是丙子否則戊子】詔天下錢穀皆歸金部倉部【唐志金部掌天下庫藏出納之數京市互市和市宫市交易之事倉部掌天下庫儲出納租税禄粮倉廩之事】罷晏轉運租庸青苖鹽鐵等使 【考異曰建中實録曰初大歷中上居東宫貞懿皇后方為妃有寵生韓王回帝又鍾愛故閹官劉清潬京兆尹黎幹與左右嬖幸欲立貞懿為皇后且言韓王所居獲黄蛇以為符動摇儲宫而晏附其謀冀立殊効圖為宰輔時宰臣元載獨保護上以為最長而賢且嘗有功義不當移王縉亦謂人曰晏黠者也今所圖無乃過黠乎後其議漸定貞懿卒不立上憾之至是以晏大臣而附邪為姦不去將為亂託陳奏不實謫為忠州刺史沈既濟楊炎所薦盖附炎為說今從舊傳】 二月丙申朔命黜陟使十一人分廵天下【黜陟使始置於太宗貞觀八年】先是魏博節度使田悦事朝廷猶㳟順【先悉薦翻使疏吏翻朝直遥翻】河北黜陟使洪經綸 【考異曰建中實録黜陟使十一人而無名德宗實録有十人名而無河北道及經綸名盖脱誤也】不曉時務聞悦軍七萬人符下罷其四萬令還農【下遐稼翻令力丁翻】悦陽順命如符罷之既而集應罷者激怒之曰汝曹久在軍中有父母妻子今一旦為黜陟使所罷將何資以自衣食乎衆大哭悦乃出家財以賜之使各還部伍於是軍士皆德悦而怨朝廷【為田悦連諸鎮之兵以拒命張本】崔祐甫以疾多不視事楊炎獨任大政專以復恩讐為事奏用元載遺策城原州【任音壬載祖亥翻又音如字元載策見二百二十四卷代宗大歷八年】又欲發兩京關内丁夫浚豐州陵陽渠以興屯田【陵陽渠在州九原縣】豐上遣中使詣涇原節度使段秀實訪以利害秀實以為今邊備尚虛未宜興事以召寇炎怒以為沮己徵秀實為司農卿【使疏吏翻沮在呂翻】丁未邠寧節度使李懷光兼四鎮北庭行營涇原節度使使移軍原州以四鎮北庭留後劉文喜為别駕【為劉文喜以涇州拒命張本】京兆尹嚴郢奏按朔方五城舊屯沃饒之地自喪亂以來【郢以井翻喪息浪翻】人功不及因致荒廢十不耕一若力可墾闢不俟浚渠今發兩京關輔人於豐州浚渠營田計所得不補所費而關輔之人不免流散是虛畿甸而無益軍儲也疏奏不報【疏所據翻】既而陵陽渠竟不成弃之 上用楊炎之言託以奏事不實己酉貶劉晏為忠州刺史【舊志忠州京師南二千一百二十二里】 癸丑以澤潞留後李抱眞為節度使【為李抱眞以澤潞為國藩翰張本】 楊炎欲城原州以復秦原【秦原謂秦州原州】命李懷光居前督作朱泚崔寧各將萬人翼其後【泚且禮翻又音此將即亮翻又音如字】詔下涇州為城具【下遐嫁翻為築城之具也】涇之將士怒曰吾屬為國家西門之屏十餘年矣【將即亮翻屏必郢翻蔽也】始居邠州甫營耕桑有地著之安徙屯涇州【邠卑旻翻著直畧翻徙涇州見二百二十四卷大歷三年】披荆榛立軍府坐席未暖又投之塞外【先弃原州不守故云投之塞外】吾屬何罪而至此乎李懷光始為邠寧帥即誅温儒雅等【事見上大歷十四年帥所類翻下同】軍令嚴峻及兼涇原諸將皆懼曰彼五將何罪而為戮【五將即史抗温儒雅龐仙鶴張獻明李光逸將即亮翻】今又來此吾屬能無憂乎劉文喜因衆心不安據涇州不受詔上疏復求段秀實為帥不則朱泚【上時掌翻疏所據翻復扶又翻或如字不讀曰否又讀如字泚且禮翻又音此】癸亥以朱泚兼四鎮北庭行營涇原節度使代懷光【使疎吏翻】 三月翰林學士左散騎常侍張涉受前湖南觀察使辛京杲金事覺上怒欲寘于法李忠臣以檢校司空同平章事奉朝請【散悉亶翻騎奇寄翻校古效翻朝直遥翻】言於上曰陛下貴為天子而先生以乏財犯法以臣愚觀之非先生之過也【張涉先侍讀東宫故李忠臣言以為先生】上意解辛未放涉歸田里辛京杲以私忿杖殺部曲有司奏京杲罪當死上將從之李忠臣曰京杲當死久矣上問其故忠臣曰京杲諸父兄弟皆戰死獨京杲至今尚存臣故以為當死久矣上憫然左遷京杲諸王傅忠臣乘機救人多此類 楊炎罷度支轉運使【度徒洛翻】命金部倉部代之既而省職久廢【謂尚書省諸司失其職已久】耳目不相接莫能振舉天下錢穀無所總領癸巳復以諫議大夫韓洄為戶部侍郎判度支以金部郎中萬年杜佑權江淮水陸轉運使皆如舊制【復扶又翻或如字諸杜居城南時號城南韋杜去天尺五戶貫則萬年】 劉文喜又不受詔欲自邀旌節夏四月乙未朔據涇州叛遣其子質于吐蕃以求援【質音致】上命朱泚李懷光討之【泚且禮翻又音此】又命神策軍使張巨濟將禁兵二千助之【使疏吏翻將即亮翻又音如字】 吐蕃始聞韋倫歸其俘【吐從暾入聲帝初即位欲以德懷吐蕃遣倫歸代宗朝所獲之俘】不之信及俘入境各還部落稱新天子出宫人放禽獸英威聖德洽於中國吐蕃大悦除道迎倫贊普即發使隨倫入貢且致賻贈【至代宗之賻贈也賻音附】癸卯至京師上禮接之既而蜀將上言吐蕃豺狼所獲俘不可歸【將上上即亮翻下時掌翻】上曰戎狄犯塞則擊之服則歸之擊以示威歸以示信威信不立何以懷遠悉命歸之【又悉歸劒南所獲之俘也 考異曰建中實録曰及境境上守陴者焚樓櫓弃城壁而去初吐蕃既得河湟之地土宇日廣守兵勞弊以國家始因用胡為邊將而致禍故得河隴之士約五十萬人以為非族類也無賢愚莫敢任者悉以為奴僕故其人苦之及見倫歸國皆毛裘蓬首窺覷墻隙或搥心隕泣或東向拜舞及密通章疏言蕃之虛實望王師之至若歲焉君子曰惜乎人心之可乘也若逾代之後斯人既没後生安於所習難乎哉此恐沈既濟之溢美且欲附楊炎復河隴之說耳今不取】代宗之世每元日冬至端午生日州府於常賦之外競為貢獻貢獻多者則悦之武將姦吏緣此侵漁下民【自代宗迄于五代正至端午降誕州府皆有貢獻謂之四節進奉將即亮翻】癸丑上生日【上生於天寶元年四月十九日不置節名】四方貢獻皆不受李正已田悦各獻縑三萬匹【縑并絲繒也】上悉歸之度支以代租賦【度徒洛翻】 五月戊辰以韋倫為太常卿乙酉復遣倫使吐蕃【復扶又翻】倫請上自為載書【載書盟誓之書】與吐蕃盟楊炎以為非敵請與郭子儀輩為載書以聞令上畫可而已從之【令力丁翻】 朱泚等圍劉文喜於涇州【泚且禮翻又音此】杜其出入而閉壁不與戰久之不拔天方旱徵發餽運内外騷然朝臣上書請赦文喜以蘇疲人者不可勝紀【朝直遥翻上時掌翻勝音升】上皆不聼曰微孽不除何以令天下文喜使其將劉海賓入奏【孽魚列翻將即亮翻】海賓言於上曰臣乃陛下藩邸部曲【帝初以雍王為天下兵馬元帥討史朝義凡在行營皆部曲也】豈肯附叛臣必為陛下梟其首以獻【為于偽翻梟堅堯翻】但文喜今所求者節而已願陛下姑與之文喜必怠則臣計得施矣上曰名器不可假人【孔子之言】爾能立效固善我節不可得也使海賓歸以告文喜而攻之如初減御膳以給軍士城中將士當受春服者賜予如故【予讀曰與】於是衆知上意不可移時吐蕃方睦於唐不為發兵【為于偽翻】城中勢窮庚寅海賓與諸將共殺文喜傳首【考異曰邠志曰詔李懷光朱泚并軍誅之師圍涇城數月不拔文喜使其子求救于吐蕃蕃衆將至二將議退軍以避之都遊奕使韓遊瓖争之曰西戎若來涇衆必變義不為文喜没身於戎虜秋七月西蕃遊騎登高麾涇人涇人果曰始吾為文喜求節度耳王師致討困則歸之安能赤土塗面為異方之人乎劉海賓因之殺文喜以衆降泚泚無所戮涇人德之萌泚之亂亦自此始按是時吐蕃通好無入援文喜事又此月涇州平而邠志云七月西蕃至皆相違今從建中實録】而原州竟不果城自上即位李正已内不自安遣參佐入奏事會涇州捷奏至上使觀文喜之首而歸正已益懼 六月甲午朔門下侍郎同平章事崔祐甫薨【薨呼肱翻】 術士桑道茂上言陛下不出數年暫有離宫之厄【上時掌翻離力智翻】臣望奉天有天子氣宜高大其城以備非常辛丑命京兆發丁夫數千雜六軍之士築奉天城 【考異曰舊傳云道茂待詔翰林建中初神策修奉天城道茂請高其垣墻大為制度德宗不之省及朱泚之亂帝倉猝出幸至奉天方思道茂之言時道茂已卒命祭之今從實録又及袁光庭幸奉天傳】 初囘紇風俗樸厚君臣之等不甚異故衆志專一勁健無敵【紇下没翻】及有功於唐【謂平安史也】唐賜遺甚厚【遺于季翻】登里可汗始自尊大築宫殿以居婦人有粉黛文繡之飾中國為之虛耗【可從刋入聲汗音寒為于偽翻】而虜俗亦壞及代宗崩上遣中使梁文秀往告哀登里驕不為禮九姓胡附囘紇者說登里以中國富饒今乘喪伐之可有大利【使疏吏翻說式芮翻 考異曰既云乘喪入寇當在去年今因源休冊命追叙之耳】登里從之欲舉國入寇其相頓莫賀逹千登里之從父兄也【相息亮翻從才用翻】諫曰唐大國也無負於我吾前年侵太原獲羊馬數萬可謂大捷【事見上卷代宗大歷十三年】而道遠糧乏比歸士卒多徒行者【比必利翻及也道遠糧乏士卒殺馬食之故多徒行】今舉國深入萬一不捷將安歸乎登里不聽頓莫賀乘人心之不欲南寇也舉兵擊殺之并九姓胡二千人自立為合骨咄禄毗伽可汗遣其臣聿逹千與梁文秀俱入見【咄當没翻伽求迦翻可從刋入聲汗音寒見賢遍翻】願為藩臣垂髪不翦以待詔命乙卯命京兆少尹臨漳源休册頓莫賀為武義成功可汗【少始照翻臨漳縣屬相州本鄴縣地東魏孝静帝分鄴縣于鄴城中置臨漳縣 考異曰舊傳曰休妻即吏部侍郎王翊女也因小忿而離妻族上訴下御史臺驗理休遅留不荅欵狀除名配流溱州久之移岳州建中初楊炎執政以京兆尹嚴郢威名稍著心欲傾之郢即王翊甥壻也休與王氏離絶之時炎風聞休郢有隙遂擢休自流人為京兆少尹俾令伺郢過失休既在職久與郢親善炎怒之奏令以本官兼御史中丞奉使囘紇按休奉使時囘紇方㳟順張光晟未殺董突炎安知囘紇欲殺休而遣之今不取】秋七月丙寅邵州賊帥王國良降【帥所類翻降戶江翻】國良本湖南牙將觀察使辛京杲使戍武岡【將即亮翻使疎吏翻武岡縣漢零陵郡都梁縣之地晉分都梁置武岡縣今岡東五十里有漢都梁故城是也後漢武陵蠻為漢所伐來保此岡故謂之武岡郡國志云武岡接武陵因以得名隋廢武德四年分邵陽復置武岡縣屬邵州新志曰本武攸縣武德四年更名梁夫夷縣在今武岡界】以扞西原蠻京杲貪暴國良家富京杲以死罪加之國良懼據縣叛與西原蠻合聚衆千人侵掠州縣瀕湖千里咸被其害【被皮義翻】詔荆黔洪桂諸道合兵討之【荆南節度使治荆州黔中觀察使治黔州江南西道觀察使治洪州桂管經畧觀察使治桂州黔音禽】連年不能克及曹王臯為湖南觀察使曰驅疲甿誅反仄非策之得者也乃遺國良書【遺于季翻】言將軍非敢為逆欲救死耳我與將軍俱為辛京杲所搆【曹王臯事見上大歷十四年】我已蒙聖朝湔洗何心復加兵刃於將軍乎【朝直遥翻復扶又翻又音如字】將軍遇我不速降後悔無及國良且喜且懼遣使乞降猶疑未决臯乃假為使者從一騎越五百里抵國良壁鞭其門大呼曰我曹王也【使疏吏翻騎奇寄翻呼火故翻】來受降舉軍大驚國良趨出迎拜請罪臯執其手約為兄弟盡焚攻守之具散其衆使還農詔赦國良罪賜名惟新【按新書南蠻傳西原蠻居廣容之南邕桂之西地數千里種落甚衆乾元以來累為叛亂與夷獠梁崇牽覃問西原酋長吴功曹合兵内寇䧟道州進攻永州䧟邵州辛京杲遣王國良戍武岡國良亦叛建中初城溆州以斷西原國良乃降降戶江翻】 辛巳遥尊上母沈氏為皇太后【沈氏以開元末選入代宗宫安禄山之亂玄宗避賊諸王妃妾不及從者皆為賊所得拘之東都之掖庭代宗克東都入宫得沈氏留之東都宫中史思明再䧟東都遂失所在】 荆南節度使庾準希楊炎指奏忠州刺史劉晏與朱泚書求營救辭多怨望又奏召補州兵欲拒朝命【忠州荆南廵屬也故庾凖得以誣奏劉晏使疏吏翻泚且禮翻又音此朝直遥翻】炎證成之上密遣中使就忠州縊殺之【縊于賜翻又于計翻】己丑乃下詔賜死天下寃之初安史之亂數年間天下戶口什亡八九州縣多為藩鎮所據貢賦不入朝廷府庫耗竭中國多故戎狄每歲犯邊所在宿重兵仰給縣官【朝直遥翻仰牛向翻】所費不貲皆倚辦于晏晏初為轉運使獨領陜東諸道【寶應元年劉晏充度支轉運等使代宗廣德二年始以晏為河南江淮以來轉運使乃疏浚汴水以開漕運之利陜失冉翻】陜西皆度支領之末年兼領未幾而罷【度徒洛翻大歷十四年晏兼判度支建中元年罷幾居豈翻】晏有精力多機智變通有無曲盡其妙常以厚直募善走者置遞相望覘報四方物價【覘丑亷翻】雖遠方不數日皆逹使司【使司謂轉運使司】食貨輕重之權悉制在掌握國家獲利而天下無甚貴甚賤之憂常以為辦集衆務在於得人故必擇通敏精悍亷勤之士而用之至於句檢簿書【悍侯旰翻又下罕翻句古侯翻】出納錢穀必委之士類吏惟書符牒不得輕出一言常言士陷賄則淪弃於時名重於利故士多清修吏雖潔亷終無榮顯利重于名故吏多貪汚然惟晏能行之他人效者終莫能逮其屬官雖居數千里外奉教令如在目前起居語言無敢欺紿【紿待亥翻】當時權貴或以親故屬之者【屬之欲翻】晏亦應之使俸給多少遷次緩速皆如其志然無得親職事其場院要劇之官【俸扶用翻少始紹翻場謂交場船場院為廵院】必盡一時之選故晏没之後掌財賦有聲者多晏之故吏也晏又以為戶口滋多則賦税自廣故其理財以愛民為先諸道各置知院官【知院官掌諸道廵院者也】每旬月具州縣雨雪豐歉之狀白使司豐則貴糴歉則賤糶或以穀易雜貨供官用及於豐處賣之知院官始見不稔之端先申至某月須如干蠲免某月須如干救助【使疏吏翻如干猶言若干也程大昌曰若干者設數之言也干猶箇也若箇猶言幾何枚也又設干者十干自甲至癸也亦以數言也】及期晏不俟州縣申請即奏行之應民之急未嘗失時不待其困弊流亡餓殍然後賑之也由是民得安其居業戶口蕃息【殍居表翻賑津忍翻蕃音煩】晏始為轉運使時天下見戶不過二百萬【見賢遍翻】其季年乃三百餘萬在晏所統則增非晏所統則不增也其初財賦歲入不過四百萬緡季年乃千餘萬緡晏專用榷鹽法充軍國之用時自許汝鄭鄧之西皆食河東池鹽度支主之汴滑唐蔡之東【統他綜翻俗音如字緡眉巾翻榷古岳翻代宗寶應元年更豫州為蔡州避上名也】皆食海鹽晏主之晏以為官多則民擾故但于出鹽之鄉置鹽官收鹽戶所煮之鹽轉鬻於商人任其所之自餘州縣不復置官【復扶又翻下賈復同】其江嶺閒去鹽鄉遠者轉官鹽於彼貯之或商絶鹽貴則減價鬻之謂之常平鹽【後又榷茶遂置常平茶鹽官貯丁呂翻】官獲其利而民不乏鹽其始江淮鹽利不過四十萬緡季年乃六百餘萬緡由是國用充足而民不困弊其河東鹽利不過八十萬緡而價復貴於海鹽先是運關東穀入長安者【先悉薦翻】以河流湍悍率一斛得八斗至者則為成勞受優賞晏以為江汴河渭水力不同各隨便宜造運船教漕卒江船達揚州汴船達河隂【悍下罕翻又候旴翻汴皮變翻江船達揚州入淮汴船自清口達河隂開元二十二年分汜水武涉滎澤置河隂縣屬河南府有河隂倉】河船達渭口渭口達太倉【渭口謂渭水入河之口】其閒緣水置倉轉相受給自是每歲運穀或至百餘萬斛無斗升沈覆者【沈持林翻】船十艘為一綱使軍將領之【艘蘇遭翻將即亮翻】十運無失授優勞官其人數運之後無不斑白者晏於楊子置十場造船每艘給錢千緡【艘蘇遭翻緡眉巾翻】或言所用實不及半虛費太多晏曰不然論大計者固不可惜小費凡事必為永久之慮今始置船場執事者至多當先使之私用無窘則官物堅牢矣若遽與之屑屑校計錙銖【窘巨隕翻八銖為錙十絫為銖】安能久行乎異日必有患吾所給多而減之者減半以下猶可也過此則不能運矣其後五十年有司果減其半及咸通中有司計費以給之無復羨餘【復扶又翻又音如字羨千線翻羨贏也】船益脆薄易壞【易以豉翻】漕運遂廢矣【宋白曰武德永徽之後姜行本薛大鼎禇朗皆言漕運未能通濟後監察御史王師順請運晉絳之粟于河渭之間始置渭橋倉開元初李傑為水運使始大興漕事十八年裴耀卿以言漕運拜江淮轉運使以崔希逸蕭炅為副轉運鹽鐵有副使自此始肅宗初第五琦以錢穀見始置江淮租庸使乾元初加鹽鐵使始大鹽鐵法就山海井竈收榷其鹽立監院官吏至劉晏始以鹽鐵兼漕運】晏為人勤力事無閑劇必于一日中决之不使留宿後來言財利者皆莫能及之 八月甲午振武留後張光晟殺囘紇使者董突等九百餘人董突者武義可汗之叔父也【晟成正翻紇下没翻使疏吏翻可從刋入聲汗音寒】代宗之世九姓胡常冒囘紇之名雜居京師殖貨縱暴與囘紇共為公私之患上即位命董突盡帥其徒歸國輜重甚盛【帥讀曰率輜莊持翻重直用翻】至振武留數月厚求資給日食肉千斤它物稱是【稱尺證翻】縱樵牧者暴踐禾稼振武人苦之光晟欲殺囘紇取其輜重而畏其衆彊未敢發九姓胡聞其種族為新可汗所誅多道亡【踐息淺翻種章湧翻】董突防之甚急九姓胡不得亡又不敢歸乃密獻策於光晟請殺囘紇光晟喜其黨自離許之上以陜州之辱【事見二百二十二卷寶應元年陜失冉翻】心恨囘紇光晟知上旨乃奏稱囘紇本種非多【紇下没翻晟成正翻種章勇翻】所輔以彊者羣胡耳今聞其自相魚肉頓莫賀新立移地健有孽子【登里可汗名移地健】及國相梅録各擁兵數千人相攻【國相息亮翻宋白曰梅録囘鶻將軍號柳公綽帥河東時有梅録將軍李暢入貢】國未定彼無財則不能使其衆陛下不乘此際除之乃歸其人與之財正所謂借寇兵齎盗糧者也【引李斯諫秦王逐客之言】請殺之三奏上不許光晟乃使副將過其館門故不為禮董突怒執而鞭之數十光晟勒兵掩擊并羣胡盡殺之聚為京觀獨留一胡使歸國為證曰囘紇鞭辱大將且謀襲據振武故先事誅之【將即亮翻觀古玩翻先悉薦翻】上徵光晟為右金吾將軍遣中使王嘉祥往致信幣囘紇請得專殺者以復讐上為之貶光晟為睦王傅以慰其意【睦王述上弟也為于偽翻】 加盧龍隴右涇原節度使朱泚兼中書令盧龍隴右節度使如故【使疏吏翻】以舒王謨為四鎮北庭行軍涇原節度大使【行軍當作行營】以涇州牙前兵馬使河中姚令言為留後【為後姚令言以涇原兵作亂張本 考異曰舊傳孟皥尋歸朝遂拜令言為四鎮北庭行營涇原節度使按實録建中三年八月以涇原節度留後姚令言為節度使此年必始為留後也 按姚令言傳建中元年孟皥為涇原節度留後自以文吏進身不樂軍旅頻表薦令言謹肅堪任將帥皥尋歸朝】謨邈之子也【邈代宗子大歷八年薨】早孤上子之癸丑詔贈太后父祖兄弟官及自餘宗族男女拜官封邑者告第告身【第恐當作策】凡百二十有七通中使以馬負而賜之九月壬午將作奏宣政殿廊壞十月魁岡未可修【隂陽家拘忌有天岡河魁凡魁岡之月及所繫之地忌修造史炤曰魁岡者北斗魁星之氣十月在戌為魁岡宋白曰隂陽氏書謂是歲孟冬為魁岡不利修作】上曰但不妨公害人則吉矣安問時日即命修之 大歷以前賦歛出納俸給皆無法長吏得專之重以元王秉政貨賂公行【歛力贍翻長知兩翻俸扶用翻元王謂元載王縉也】天下不按吏者殆二十年 【考異曰建中實録云三十年盖字之誤也】惟江西觀察使路嗣恭按䖍州刺史源敷翰流之上以宣歙觀察使薛邕文雅舊臣徵為左丞邕去宣州盗隱官物以巨萬計殿中侍御史員㝢發之【時以宣歙二州依山而扼江湖之要分置觀察使使疏吏翻嗣祥吏翻歙音攝員音運姓也】冬十月己亥貶連山尉【連山縣屬連州晉武帝分桂陽立廣惠縣隋改為廣澤仁壽元年改為連山縣避太子廣諱也】於是州縣始畏朝典不敢放縱上初即位疎斥宦官親任朝士而張涉以儒學入侍薛邕以文雅登朝【朝直遥翻】繼以敗宦官武將得以藉口曰南牙文臣動至巨萬而謂我曹濁亂天下豈非欺罔邪【將即亮翻邪音耶】於是上心始疑不知所倚仗矣 中書舍人高參請分遣諸沈訪求太后庚寅以睦王述為奉迎使工部尚書喬琳副之又命諸沈四人為判官與中使分行諸道求之【尚辰羊翻行下孟翻又音如字】 十一月初令待制官外更引朝集使二人訪以時政得失遠人疾苦【令力丁翻朝直遥翻使疏吏翻】 先是公主下嫁者舅姑拜之婦不荅【先悉薦翻】上命禮官定公主拜見舅姑及壻之諸父兄姊之儀舅姑坐受於中堂兄姊立受于東序如家人禮有縣主將嫁擇用丁丑是日上之從父妺卒【姊將兕翻從才用翻】命罷之有司奏供張已備【供居用翻張知亮翻】且殤服不足廢事【殤音傷說文未成人而死者為殤禮十九至十六死者為長殤十五至十二死者為中殤十一至八歲死者為下殤】上曰爾愛其費我愛其禮卒罷之【卒子恤翻】至德以來國家多事公主郡縣主多不以時嫁有華髪者雖居禁中或十年不見天子上始引見諸宗女【髪中白者曰華見賢遍翻】尊者致敬卑者存慰悉命嫁之所齎小大之物必經心目己卯庚辰二日嫁岳陽等凡十一縣主【齎則兮翻】 吐蕃見韋倫再至益喜【是年五月韋倫再使吐蕃吐從暾入聲】十二月辛卯朔倫還吐蕃遣其相論欽明思等入貢【還從宣翻又音如字相息亮翻】 是歲册太子母王氏為淑妃 天下税戶三百八萬五千七十六籍兵七十六萬八千餘人【籍兵兵之著籍者也】税錢一千八十九萬八千餘緡穀二百一十五萬七千餘斛【緡眉巾翻】<br />
<br />
  二年春正月戊辰成德節度使李寶臣薨【恒冀成德軍考異曰建中實録云二月丁已寶臣卒疑奏到之日也今從德宗實録谷况燕南記曰忠志末年惟納妖妄之人兼隂陽術數謟媚苟且之輩爭獻圖䜟稱有尊位詐作朱草靈芝鑿石上作名字又于後堂院結壇場清齋菜食置金杯玉斚銀盤云甘靈神酒自至其内又言天符下降忠志自謂命符上天將吏罔有諫者使行文牒布告州縣云靈芝朱草王者之瑞輒生壇上香滿院中靈石呈祥天符飛應甘露如蜜神酒盈杯匪我所求不期自至各牒管内郡縣宜令知委同為喜慶也既而日為妖妄者更相矯云不日當有天神下降持金箱玉印而至然後即大位為天所授也四方皆自歸伏不待征討海内坐而定矣忠志大悦多以金銀羅錦異物賞之隂陽妖妄者自知虛偽恐事泄見誅共言相公宜服甘露靈芝草湯即天神降速忠志一任妖者遂於湯中密著毒藥既飲畢便失音三日而卒舊傳亦以為然按方士妖妄必為一府所疾所憑恃者寶臣一人耳若酖殺寶臣身在府中逃無所之安能免死乎計方士雖愚必不為此盖時人見寶臣曾飲其湯遇疾而死以為方士所酖谷况承而書之耳】寶臣欲以軍府傳其子行軍司馬惟岳以其年少闇弱【少詩照翻】豫誅諸將之難制者【將即亮翻】深州刺史張獻誠等至有十餘人同日死者寶臣召易州刺史張孝忠孝忠不往使其弟孝節召之孝忠使孝節謂寶臣曰諸將何罪連頸受戮孝忠懼死不敢往亦不敢叛正如公不入朝之意耳【朝直遥翻】孝節泣曰如此孝節必死孝忠曰往則併命我在此必不敢殺汝遂歸寶臣亦不之罪也兵馬使王武俊位卑而有勇故寶臣特親愛之以女妻其子士眞【使疏吏翻妻七細翻】士眞復厚結其左右故孝忠武俊獨全【史言人不可妄殺且為孝忠武俊歸國張本復扶又翻】及薨孔目官胡震家僮王它奴勸惟岳匿喪二十餘日詐為寶臣表求令惟岳繼襲上不許【令力丁翻】遣給事中汲人斑宏往問寶臣疾且諭之惟岳厚賂宏宏不受還報惟岳乃發喪自為留後使將佐共奏求旌節上又不許【汲縣屬衛州還從宣翻又音如字將即亮翻】初寶臣與李正已田承嗣梁崇義相結【事見上卷代宗大歷十二年嗣祥吏翻】期以土地傳之子孫故承嗣之死【見上卷大歷十四年】寶臣力為之請于朝【為于偽翻下屢為同朝直遥翻】使以節授田悦代宗從之悦初襲位事朝廷禮甚恭河東節度使馬燧表其必反請先為備至是悦屢為惟岳請繼襲上欲革前弊不許或諫曰惟岳已據父業不因而命之必為亂上曰賊本無資以為亂皆藉我土地假我位號以聚其衆耳曏日因其所欲而命之多矣而亂日益滋是爵命不足以已亂而適足以長亂也【德宗鋭于削平藩鎮而發是言誠中肅代之病而終不能已亂亦以召亂所行者未能副其言也長知丈翻】然則惟岳必為亂命與不命等耳竟不許悦乃與李正已各遣使詣惟岳潜謀勒兵拒命魏博節度副使田庭玠謂悦曰爾藉伯父遺業【田承嗣者悦之伯父也】但謹事朝廷坐享富貴不亦善乎奈何無故與恒鄆共為叛臣【成德節度使治恒州淄青節度使治鄆州故以恒鄆稱之恒戶登翻鄆音運】爾觀兵興以來逆亂者誰能保其家乎必欲行爾之志可先殺我無使我見田氏之族滅也因稱病卧家悦自往謝之庭玠閉門不内竟以憂卒【卒子恤翻】成德判官邵眞聞李惟岳之謀泣諫曰先相公受國厚恩大夫衰絰之中【相悉亮翻衰倉囘翻】遽欲負國此甚不可勸惟岳執李正已使者送京師且請討之曰如此朝廷嘉大夫之忠則旄節庶幾可得【使疏吏翻朝直遥翻幾居希翻】惟岳然之使眞草奏長史畢華曰先公與二道結好二十餘年【長知兩翻好呼到翻】奈何一旦弃之且雖執其使朝廷未必見信正已忽來襲我孤軍無援何以待之惟岳又從之前定州刺史谷從政惟岳之舅也有膽略頗讀書王武俊等皆敬憚之為寶臣所忌從政乃稱病杜門惟岳亦忌之不與國事日夜獨與胡震王它奴等計議多散金帛以悦將士【將即亮翻】從政往見惟岳曰今海内無事自上國來者【時藩鎮竊據自比古諸侯謂京師為上國】皆言天子聰明英武志欲致太平深不欲諸侯子孫專地爾今首違詔命天子必遣諸道致討將士受賞皆言為大夫盡死【為于偽翻下且為同】苟一戰不勝各惜其生誰不離心大將有權者乘危伺便咸思取爾以自為功矣【伺相吏翻】且先相公所殺高斑大將殆以百數撓敗之際【撓奴教翻】其子弟欲復仇者庸可數乎又相公與幽州有隙【謂李寶臣襲朱滔也事見上卷代宗大歷之十年】朱滔兄弟常切齒于我今天子必以為將滔與吾擊柝相聞【左傳曰魯擊柝聞於邾謂接境也】計其聞命疾驅若虎狼之得獸也何以當之【是後李惟岳禍敗皆如谷從政所言】昔田承嗣從安史父子同反身經百戰凶悍聞于天下【嗣祥吏翻悍下罕翻又戶旰翻】違詔舉兵自謂無敵及盧子期就擒吴希光歸國承嗣指天垂泣身無所措賴先相公按兵不進且為之祈請先帝寛仁赦而不誅【事亦見上卷大歷十年相息亮翻為于偽翻】不然田氏豈有種乎况爾生長富貴齒髪尚少不更艱危【種章勇翻長知丈翻少詩照翻更工衡翻】乃信左右之言欲效承嗣所為乎為爾之計不若辭謝將佐使惟誠攝領軍府身自入朝乞留宿衛因言惟誠且留攝事恩命决於聖志上必悦爾忠義縱無大位不失榮禄永無憂矣不然大禍將及吾亦知爾素疎忌我顧以舅甥之情事急不得不言耳惟岳見其言切益惡之從政乃復歸杜門稱病【將即亮翻朝直遥翻惡烏路翻復扶又翻又音如字】惟誠者惟岳之庶兄也謙厚好書得衆心【好呼到翻】其母妹為李正巳子婦【母妹者惟誠同母之妹也】是日惟岳送惟誠於正已正已使復姓張遂仕淄青惟岳遣王它奴詣從政家察其起居【將殺之示之以意使自引分】從政飲藥而卒【卒子恤翻】且死曰吾不憚死哀張氏今族滅矣【李寶臣本張忠志故云然】劉文喜之死也李正已田悦等皆不自安劉晏死正已等益懼【劉文喜劉晏死皆見上年】相謂曰我輩罪惡豈得與劉晏比乎【李正巳田悦非面相告語也使人傳言有此語】會汴州城隘廣之東方人訛言上欲東封【汴皮變翻隘烏介翻東封非東封泰山之謂盖用左傳燭之武說秦伯既東封鄭又欲肆其西封之語】故城汴州正已懼兵萬人屯曹州【曹州李正已廵屬與汴州接壤】田悦亦完聚為備【杜預曰完聚者完城郭聚人民】與梁崇義李惟岳遥相應助河南士民騷然驚駭永平舊領汴宋滑亳陳潁泗七州【此平李靈曜後永平軍所領廵屬也按代宗大歷七年賜滑亳軍號永平十一年平李靈曜增領宋泗二州十四年增領汴潁二州滑亳未賜軍號之前已領陳州共七州】丙子分宋亳潁别為節度使以宋州刺史劉洽為之以泗州隸淮南又以東都留守路嗣恭為懷鄭汝陜四州河陽三城節度使【使疏吏翻守式又翻嗣祥吏翻陜失冉翻】旬日又以永平節度使李勉都統洽嗣恭二道仍割鄭州隸之選嘗為將者為諸州刺史以備正已等【統它綜翻俗又讀如字將即亮翻】 初高力士有養女嫠居東京【嫠里之翻無夫為嫠】頗能言宫中事女官李眞一意其為沈太后詣使者具言其狀【去年遣使訪求太后】上聞之驚喜時沈氏故老已盡無識太后者上遣宦官宫人往驗視之年狀頗同宦官宫人不審識太后皆言是高氏辭稱實非太后驗視者益疑之強迎入上陽宫【強其兩翻下所強同】上宫女百餘人齎乘輿服御物就上陽宫供奉【乘繩證翻輿音于】左右誘諭百方【誘音酉】高氏心動乃自言是驗視者走馬入奏上大喜二月辛卯上以偶日御殿羣臣皆入賀【唐制天子以隻日受朝賀今喜於得太后故以耦日御殿而受賀】詔有司草儀奉迎高氏弟承悦在長安恐不言久獲罪遽自言本末上命力士養孫樊景超往覆視景超見高氏居内殿以太后自處【處昌呂翻】左右侍衛甚嚴景超謂高氏曰姑何自置身於俎上【謂將以詐偽伏罪如置身俎上以俟刀也】左右叱景超使下景超抗聲曰有詔太后詐偽左右可下左右皆下殿高氏乃曰吾為人所強非己出也【強其兩翻】以牛車載還其家【還從宣翻又音如字】上恐後人不復敢言太后皆不之罪曰吾寧受百欺庶幾得之【復扶又翻幾居依翻】自是四方稱得太后者數四皆非是而眞太后竟不知所之【之往也】 御史中丞盧奕之子也【天寶十四載安禄山䧟洛陽李憕盧奕死之】貌醜色如藍有口辯上悦之丁未擢為大夫【擢為御史大夫】領京畿觀察使【句】郭子儀每見賓客姬妾不離側嘗往問疾子儀悉屏侍妾獨隱几待之【離力智翻屏必郢翻隱於靳翻】或問其故子儀曰貌陋而心險婦人輩見之必笑它日得志吾族無類矣楊炎既殺劉晏朝野側目李正已累表請晏罪譏斥朝廷炎懼遣腹心分詣諸道以宣慰為名實使之密諭節度使云晏昔朋附姦邪請立獨孤后上自惡而殺之上聞而惡之【朝直遥翻度使疎吏翻惡烏路翻】由是有誅炎之志隱而未乙巳遷炎為中書侍郎擢盧為門下侍郎並同平章事不專任炎矣蕞陋無文學【蕞徂外翻】炎輕之多託疾不與會食【唐制諸宰相日會食于政事堂】亦恨之杞隂狡欲起勢立威小不附者必欲寘之死地引太常博士裴延齡為集賢殿直學士親任之【為盧以姦邪致亂張本然為建中厲階人皆知之其引裴延齡以樹黨其禍蔓延迄于貞元之末年人未知其罪也故通鑑著言之】 丙午更汴宋軍曰宣武【按是時李勉以永平軍節度使鎮汴州盖以宋亳潁為宣武軍當從新書方鎮表更工衡翻】振武節度使彭令芳苛虐監軍劉惠光貪婪【婪盧含翻】乙卯軍士共殺之 京西防秋兵萬二千人戍關東【時吐蕃通和西邊無警而河南北諸鎮連兵拒命關東騷然故抽京西防秋之兵以戍關東】上御望春樓【望春樓在灞水之西臨廣運潭】宴勞將士【勞力到翻將即亮翻】神策軍士獨不飲上使詰之其將楊惠元對曰臣等奉天軍帥張巨濟戒之曰此行大建功名凱還之日相與為歡故不敢奉詔【按建中元年遣神策軍使張巨濟將禁兵助朱泚等討劉文喜盖涇州既平巨濟還屯奉天也詰去吉翻帥所類翻還音旋又如字】及行有司緣道設酒食獨惠元所部缾罌不【罌烏莖翻】上深歎美賜書勞之【勞力到翻】惠元平州人也【平州北平郡】 三月置溵州於郾城【魏收地形志潁川郡曲陽縣有郾城後齊置臨潁郡隋廢郡為郾城縣唐屬蔡州時分郾城臨潁陳州之溵水置溵州溵于巾翻郾于建翻】 辛巳以汾州刺史王翃為振武軍使鎮北綏銀等州留後【翃戶萌翻】遣殿中少監崔漢衡使于吐蕃【少始照翻使疏吏翻吐從暾入聲】 梁崇義雖與李正已等連結兵勢寡弱禮數最恭或勸其入朝【朝直遥翻】崇義曰來公有大功於國上元中為閹宦所讒遷延稽命【稽緩也】及代宗嗣位不俟駕入朝猶不免族誅【來公謂來瑱死于廣德元年事見二百二十四卷】吾歲久舋積何可往也淮寧節度使李希烈屢請討之【方鎮表大歷十四年淮西節度使復治蔡州賜號淮寧軍事見上】崇義懼益修武備流人郭昔告崇義為變【郭昔以告崇義得流罪史因稱流人以叙其事】崇義聞之請罪上為之杖昔遠流之【為于偽翻下為陳同】使金部員外郎李舟詣襄州諭旨以安之舟嘗奉使詣劉文喜為陳禍福文喜囚之會帳下殺文喜以降諸道跋扈者聞之謂舟能覆城殺將【將即亮翻】至襄州崇義惡之舟又勸崇義入朝言頗切直崇義益不悦及遣使宣慰諸道舟復詣襄州崇義拒境不内【惡烏路翻復扶又翻拒境者拒之于境上】上言軍中疑懼請易以它使【上時掌翻】時兩河諸鎮方猜阻上欲示恩信以安之夏四月庚寅加崇義同平章事妻子悉加封賞賜以鐵劵遣御史張著齎手詔徵之仍以其禆將藺杲為鄧州刺史【禆賓彌翻鄧州治穰縣】五月丙寅以軍興增商税為什一【楊炎定税法商賈三十税一今增之】田悦卒與李正已李惟岳定計【卒子恤翻終也竟也】連兵拒命遣兵馬使孟祐將步騎五千北助惟岳【將即亮翻又音如字騎奇寄翻】薛嵩之死也田承嗣盗據洺相二州【事見上卷大歷十年洺音名相息亮翻】朝廷獨得邢磁二州及臨洺縣【臨洺漢之易陽縣地屬趙國晉屬廣平郡後魏屬魏郡後齊廢入襄國縣置襄國郡後周改置易陽縣别置襄國縣隋開皇之六年改易陽為邯鄲十年改邯鄲為臨洺屬武安郡唐屬洺州范大成北使録臨洺縣東至洺州三十五里朝直遥翻磁牆之翻】悦欲阻山為境曰邢磁如兩眼在吾腹中不可不取乃遣兵馬使康愔將八千人圍邢州【使疏吏翻愔挹淫翻將即亮翻又音如字】别將楊朝光將五千人柵於邯鄲西北以斷昭義救兵【邯鄲縣漢屬趙國晉屬廣平郡東魏廢隋復置屬武安郡唐屬磁州余按隋開皇十年既改邯鄲為臨洺隋志不復言别置邯鄲至唐志則臨洺縣屬洺州邯鄲縣屬磁州盖邯鄲縣必復置於唐世與臨洺各為一縣史逸其置縣之歲月也宋白曰臨洺縣漢易陽縣地屬趙國魏屬魏郡晉屬廣平郡後魏省入邯鄲孝文于北中府城復置易陽縣屬廣平郡今理是也隋開皇六年改易陽為邯鄲縣十年移邯鄲理涉鄉在今邯鄲縣界仍於北中府城置臨洺縣北濱洺水為名九域志邯鄲縣在磁州東北七十里柵測革翻邯音寒鄲音單斷音短】悦自將兵數萬圍臨洺 【考異曰馬燧傳悦自將兵三萬圍邢州次臨洺燕南記悦自統馬步五千人應接今從舊傳】邢州刺史李共臨洺將張伾堅壁拒守【伾音丕】貝州刺史邢曹俊田承嗣舊將也老而有謀悦寵信牙官扈㟧而疎之【嗣徐史翻將即亮翻㟧五各翻】及攻臨洺召曹俊問計曹俊曰兵法十圍五攻【此孫子兵法之言】尚書以逆犯順勢更不侔【尚辰羊翻言以寡敵衆勢已不侔而以逆犯順更不侔也】今頓兵堅城之下糧竭卒盡自亡之道也不若置萬兵於崞口以遏西師【西師謂澤潞河東之師自西山而下崞音郭崞口當西山之下直相州之西】則河北二十四州皆為尚書有矣【河北二十四州即玄宗所謂河朔二十四郡也自至德改郡為州安史既平之後河北又有分置之州若以開元天寶河北道采訪使所統大畧言之此時河北不止二十四州邢曹俊之說盖因時俗為傳習古語耳】諸將惡其異已共毁之悦不用其策【為悦摧敗張本惡烏路翻】<br />
<br />
  資治通鑑卷二百二十六<br />
<br />
<史部,編年類,資治通鑑>  <br>
   </div> 

<script src="/search/ajaxskft.js"> </script>
 <div class="clear"></div>
<br>
<br>
 <!-- a.d-->

 <!--
<div class="info_share">
</div> 
-->
 <!--info_share--></div>   <!-- end info_content-->
  </div> <!-- end l-->

<div class="r">   <!--r-->



<div class="sidebar"  style="margin-bottom:2px;">

 
<div class="sidebar_title">工具类大全</div>
<div class="sidebar_info">
<strong><a href="http://www.guoxuedashi.com/lsditu/" target="_blank">历史地图</a></strong>  
<a href="http://www.880114.com/" target="_blank">英语宝典</a>  
<a href="http://www.guoxuedashi.com/13jing/" target="_blank">十三经检索</a> 
<br><strong><a href="http://www.guoxuedashi.com/gjtsjc/" target="_blank">古今图书集成</a></strong> 
<a href="http://www.guoxuedashi.com/duilian/" target="_blank">对联大全</a> <strong><a href="http://www.guoxuedashi.com/xiangxingzi/" target="_blank">象形文字典</a></strong> 

<br><a href="http://www.guoxuedashi.com/zixing/yanbian/">字形演变</a>  <strong><a href="http://www.guoxuemi.com/hafo/" target="_blank">哈佛燕京中文善本特藏</a></strong>
<br><strong><a href="http://www.guoxuedashi.com/csfz/" target="_blank">丛书&方志检索器</a></strong> <a href="http://www.guoxuedashi.com/yqjyy/" target="_blank">一切经音义</a>  

<br><strong><a href="http://www.guoxuedashi.com/jiapu/" target="_blank">家谱族谱查询</a></strong>  <strong><a href="http://shufa.guoxuedashi.com/sfzitie/" target="_blank">书法字帖欣赏</a></strong> 
<br>

</div>
</div>


<div class="sidebar" style="margin-bottom:0px;">

<font style="font-size:22px;line-height:32px">QQ交流群9:489193090</font>


<div class="sidebar_title">手机APP 扫描或点击</div>
<div class="sidebar_info">
<table>
<tr>
	<td width=160><a href="http://m.guoxuedashi.com/app/" target="_blank"><img src="/img/gxds-sj.png" width="140"  border="0" alt="国学大师手机版"></a></td>
	<td>
<a href="http://www.guoxuedashi.com/download/" target="_blank">app软件下载专区</a><br>
<a href="http://www.guoxuedashi.com/download/gxds.php" target="_blank">《国学大师》下载</a><br>
<a href="http://www.guoxuedashi.com/download/kxzd.php" target="_blank">《汉字宝典》下载</a><br>
<a href="http://www.guoxuedashi.com/download/scqbd.php" target="_blank">《诗词曲宝典》下载</a><br>
<a href="http://www.guoxuedashi.com/SiKuQuanShu/skqs.php" target="_blank">《四库全书》下载</a><br>
</td>
</tr>
</table>

</div>
</div>


<div class="sidebar2">
<center>


</center>
</div>

<div class="sidebar"  style="margin-bottom:2px;">
<div class="sidebar_title">网站使用教程</div>
<div class="sidebar_info">
<a href="http://www.guoxuedashi.com/help/gjsearch.php" target="_blank">如何在国学大师网下载古籍?</a><br>
<a href="http://www.guoxuedashi.com/zidian/bujian/bjjc.php" target="_blank">如何使用部件查字法快速查字?</a><br>
<a href="http://www.guoxuedashi.com/search/sjc.php" target="_blank">如何在指定的书籍中全文检索?</a><br>
<a href="http://www.guoxuedashi.com/search/skjc.php" target="_blank">如何找到一句话在《四库全书》哪一页?</a><br>
</div>
</div>


<div class="sidebar">
<div class="sidebar_title">热门书籍</div>
<div class="sidebar_info">
<a href="/so.php?sokey=%E8%B5%84%E6%B2%BB%E9%80%9A%E9%89%B4&kt=1">资治通鉴</a> <a href="/24shi/"><strong>二十四史</strong></a>&nbsp; <a href="/a2694/">野史</a>&nbsp; <a href="/SiKuQuanShu/"><strong>四库全书</strong></a>&nbsp;<a href="http://www.guoxuedashi.com/SiKuQuanShu/fanti/">繁体</a>
<br><a href="/so.php?sokey=%E7%BA%A2%E6%A5%BC%E6%A2%A6&kt=1">红楼梦</a> <a href="/a/1858x/">三国演义</a> <a href="/a/1038k/">水浒传</a> <a href="/a/1046t/">西游记</a> <a href="/a/1914o/">封神演义</a>
<br>
<a href="http://www.guoxuedashi.com/so.php?sokeygx=%E4%B8%87%E6%9C%89%E6%96%87%E5%BA%93&submit=&kt=1">万有文库</a> <a href="/a/780t/">古文观止</a> <a href="/a/1024l/">文心雕龙</a> <a href="/a/1704n/">全唐诗</a> <a href="/a/1705h/">全宋词</a>
<br><a href="http://www.guoxuedashi.com/so.php?sokeygx=%E7%99%BE%E8%A1%B2%E6%9C%AC%E4%BA%8C%E5%8D%81%E5%9B%9B%E5%8F%B2&submit=&kt=1"><strong>百衲本二十四史</strong></a>  <a href="http://www.guoxuedashi.com/so.php?sokeygx=%E5%8F%A4%E4%BB%8A%E5%9B%BE%E4%B9%A6%E9%9B%86%E6%88%90&submit=&kt=1"><strong>古今图书集成</strong></a>
<br>

<a href="http://www.guoxuedashi.com/so.php?sokeygx=%E4%B8%9B%E4%B9%A6%E9%9B%86%E6%88%90&submit=&kt=1">丛书集成</a> 
<a href="http://www.guoxuedashi.com/so.php?sokeygx=%E5%9B%9B%E9%83%A8%E4%B8%9B%E5%88%8A&submit=&kt=1"><strong>四部丛刊</strong></a>  
<a href="http://www.guoxuedashi.com/so.php?sokeygx=%E8%AF%B4%E6%96%87%E8%A7%A3%E5%AD%97&submit=&kt=1">說文解字</a> <a href="http://www.guoxuedashi.com/so.php?sokeygx=%E5%85%A8%E4%B8%8A%E5%8F%A4&submit=&kt=1">三国六朝文</a>
<br><a href="http://www.guoxuedashi.com/so.php?sokeytm=%E6%97%A5%E6%9C%AC%E5%86%85%E9%98%81%E6%96%87%E5%BA%93&submit=&kt=1"><strong>日本内阁文库</strong></a> <a href="http://www.guoxuedashi.com/so.php?sokeytm=%E5%9B%BD%E5%9B%BE%E6%96%B9%E5%BF%97%E5%90%88%E9%9B%86&ka=100&submit=">国图方志合集</a> <a href="http://www.guoxuedashi.com/so.php?sokeytm=%E5%90%84%E5%9C%B0%E6%96%B9%E5%BF%97&submit=&kt=1"><strong>各地方志</strong></a>

</div>
</div>


<div class="sidebar2">
<center>

</center>
</div>
<div class="sidebar greenbar">
<div class="sidebar_title green">四库全书</div>
<div class="sidebar_info">

《四库全书》是中国古代最大的丛书,编撰于乾隆年间,由纪昀等360多位高官、学者编撰,3800多人抄写,费时十三年编成。丛书分经、史、子、集四部,故名四库。共有3500多种书,7.9万卷,3.6万册,约8亿字,基本上囊括了古代所有图书,故称“全书”。<a href="http://www.guoxuedashi.com/SiKuQuanShu/">详细>>
</a>

</div> 
</div>

</div>  <!--end r-->

</div>
<!-- 内容区END --> 

<!-- 页脚开始 -->
<div class="shh">

</div>

<div class="w1180" style="margin-top:8px;">
<center><script src="http://www.guoxuedashi.com/img/plus.php?id=3"></script></center>
</div>
<div class="w1180 foot">
<a href="/b/thanks.php">特别致谢</a> | <a href="javascript:window.external.AddFavorite(document.location.href,document.title);">收藏本站</a> | <a href="#">欢迎投稿</a> | <a href="http://www.guoxuedashi.com/forum/">意见建议</a> | <a href="http://www.guoxuemi.com/">国学迷</a> | <a href="http://www.shuowen.net/">说文网</a><script language="javascript" type="text/javascript" src="https://js.users.51.la/17753172.js"></script><br />
  Copyright &copy; 国学大师 古典图书集成 All Rights Reserved.<br>
  
  <span style="font-size:14px">免责声明:本站非营利性站点,以方便网友为主,仅供学习研究。<br>内容由热心网友提供和网上收集,不保留版权。若侵犯了您的权益,来信即刪。scp168@qq.com</span>
  <br />
ICP证:<a href="http://www.beian.miit.gov.cn/" target="_blank">鲁ICP备19060063号</a></div>
<!-- 页脚END --> 
<script src="http://www.guoxuedashi.com/img/plus.php?id=22"></script>
<script src="http://www.guoxuedashi.com/img/tongji.js"></script>

</body>
</html>
