






























































資治通鑑卷二百六十六 宋 司馬光 撰

胡三省 音註

後梁紀一|{
	起彊圉單闋盡著雍執徐七月凡一年有奇 朱氏本碭山人碭山戰國時屬梁地太祖以宣武節度使創業宣武軍治汴州古大梁也寖益彊盛進封梁王國遂號曰梁通鑑以前紀巳有蕭梁故此稱曰後梁}


太祖神武元聖孝皇帝上|{
	姓朱氏名温宋州碭山午溝里人背黄巢歸唐賜名全忠即位改名晃}


開平元年|{
	是年四月即位始改元}
春正月辛巳梁王休兵於貝州|{
	自滄州還休兵貝州且因魏博糧餉也}
淮南節度使兼侍中東面諸道行營都統弘農郡王楊渥既得江西|{
	謂并鍾匡時也事見上卷天祐三年}
驕侈益甚謂節度判官周隱曰君賣人國家何面復相見遂殺之|{
	以隱言其不克負荷欲屬國於劉威也事見上卷天祐二年復扶又翻}
由是將佐皆不自安|{
	既逐王茂章又殺周隱宜餘人之不自安也}
黑雲都指揮使呂師周與副指揮使綦章將兵屯上高|{
	上高在洪州高安縣界宋置上高縣屬筠州在州西南九十五里宋白曰上高縣本高安縣之上鎮以地形高上故曰上高南唐昇元中立上高場保大十年升為縣}
師周與湖南戰屢有功渥忌之師周懼謀於綦章曰馬公寛厚|{
	謂馬殷也}
吾欲逃死焉可乎章曰茲事君自圖之吾舌可斷不敢泄|{
	斷都管翻}
師周遂奔湖南章縱其孥使逸去|{
	路振九國志呂師周父珂以勇敢事楊行密累有功拜黑雲都指揮使珂卒師周代之自言三代將家不可保富貴每恣為盃酌醉必起舞或擊節狂歌慷慨泣下行密聞而疑之密使人偵其動靜師周不自安乃謀於綦章而奔湖南據此則為渥所疑非行密也孥音奴子也}
師周楊州人也渥居喪|{
	居其父行密之喪也}
晝夜酣飲|{
	酣戶甘翻樂飲也湛嗜也應劭曰洽也張宴曰中酒也}
作樂|{
	句斷}
然十圍之燭以擊毬一燭費錢數萬或單騎出游從者奔走道路不知所之|{
	從才用翻之往也}
左右牙指揮使張顥徐温泣諫|{
	蜀注曰牙者旗名執牙者因以名之分左右隊故稱左右牙余謂牙兵以衛府牙}
渥怒曰汝謂我不才何不殺我自為之二人懼渥選壯士號東院馬軍廣署親信為將吏所署者恃勢驕横|{
	横戶孟翻}
陵蔑勲舊顥温潜謀作亂渥父行密之世有親軍數千營於牙城之内|{
	蜀注曰古者軍行有牙尊者所在後人因以所治為衙曰牙城即衙城也}
渥遷出於外以其地為射場顥温由是無所憚|{
	史言楊渥自去其爪牙}
渥之鎮宣州也|{
	天祐元年楊渥鎮宣州三年召為嗣}
命指揮使朱思勍范思從陳璠將親兵三千|{
	勍渠京翻璠音番}
及嗣位召歸廣陵顥温使三將從秦裴擊江西因戍洪州誣以謀叛命别將陳祐往誅之|{
	史言張顥徐温又翦去渥之爪牙}
祐間道兼行|{
	間古莧翻}
六日至洪州微服懷短兵徑入秦裴帳中裴大驚祐告之故|{
	告之以所以徑入之故}
乃召思勍等飲酒祐數思勍等罪|{
	數所具翻下因數同俗從所主翻}
執而斬之渥聞三將死益忌顥温欲誅之丙戍渥晨視事顥温帥牙兵二百露刃直入庭中|{
	帥讀曰率}
渥曰爾果欲殺我邪對曰非敢然也欲誅王左右亂政者耳因數渥親信十餘人之罪曳下以鐵檛擊殺之|{
	檛側瓜翻 考異曰歐陽史四年正月渥視事陳璠等侍側温顥擁牙兵入拽璠等下斬之渥不能止由是失政按璠已死於宣州今從十國紀年 按通鑑本文宣州當作洪州}
謂之兵諫|{
	左傳鬻拳強諫楚子不從臨之以兵懼而從之遂自刖也張顥徐温以兵諫自文鬻拳之罪人也}
諸將不與之同者顥温稍以法誅之於是軍政悉歸二人渥不能制|{
	為顥温弑渥張本}
初梁王以河北諸鎮皆服惟幽滄未下故大舉伐之欲以堅諸鎮之心既而潞州内叛王燒營而還|{
	事見上卷天祐三年還從宣翻又如字}
威望大沮|{
	沮在呂翻}
恐中外因此離心欲速受禪以鎮之丁亥王入館于魏|{
	館古玩翻}
有疾臥府中羅紹威恐王襲之入見王曰今四方稱兵為王患者皆以翼戴唐室為名王不如早滅唐以絶人望王雖不許而心德之乃亟歸|{
	亟紀力翻}
壬寅至大梁甲辰唐昭宣帝遣御史大夫薛貽矩至大梁勞王|{
	勞力到翻}
貽矩請以臣禮見|{
	見賢遍翻}
王揖之升階貽矩曰殿下功德在人三靈改卜|{
	三靈天地人之靈也言天地人之心皆已去唐室改卜君而命之}
皇帝方行舜禹之事臣安敢違乃北面拜舞於庭王側身避之貽矩還言於帝曰元帥有受禪之意矣帝乃下詔|{
	帝皆謂唐昭宣帝元帥謂梁王}
以二月禪位于梁又遣宰相以書諭王王辭 河東兵猶屯長子欲窺澤州|{
	九域志長子西南至澤州一百四十里}
王命保平節度使康懷貞悉發京兆同華之兵屯晉州以備之|{
	宋太宗皇帝太平興國元年始改保義軍為保平軍避藩邸舊名也此因史臣避廟諱而書之然觀令康懷貞發京兆同華兵屯晉州則恐自鄜州而東發兩鎮兵屯晉州蓋懷貞若自邢州發京兆同華兵道里隔涉邢州與潞州相近亦當備河東兵之來無緣使懷貞離邢州而屯晉州竊謂保平亦當作保大據歐史懷英傳亦書保義蓋以美原之捷方除保義節朱全忠急於簒唐未暇舉兵攻潞州自備而已故潞州益得以嚴備}
二月唐大臣共奏請昭宣帝遜位壬子詔宰相帥百官詣元帥府勸進|{
	梁王建元帥府于大梁相帥讀曰率}
王遣使却之於是朝臣藩鎮乃至湖南嶺南上牋勸進者相繼|{
	朝直遥翻上時掌翻}
三月癸未王以亳州刺史李思安為北路行軍都統將兵擊幽州|{
	擊劉仁恭也}
庚寅唐昭宣帝詔薛貽矩再詣大梁諭禪位之意又詔禮部尚書蘇循齎百官牋詣大梁 鎮海鎮東節度使吳王錢鏐遣其子傳璙傳瓘討盧佶於温州|{
	璙力小翻又力弔翻佶其吉翻}
甲辰唐昭宣帝降御札禪位于梁|{
	考異曰實録薛居正五代史唐餘録皆云四月唐帝御札勅宰臣張文蔚等備法駕奉迎梁朝而無日五代通録云四月丁未丁未四月一日也舊唐書云三月甲辰甲辰三月二十七日也唐年補録三月二十七日甲子降此御札四月戊辰朱全忠即位尤為差誤按此年三月戊寅朔四月丁未朔今從舊唐書}
以攝中書令張文蔚為冊禮使禮部尚書蘇循副之|{
	冊禮使奉傳禪冊寶押金吾仗衛太常鹵簿等}
攝侍中楊涉為押傳國寶使|{
	唐有傳國八寶武后惡璽字改為寶其受命傳國八寶並改雕寶字}
翰林學士張策副之御史大夫薛貽矩為押金寶使|{
	唐六典曰天子八寶其用以玉其封以泥皇后及太子之信曰寶其用以金}
尚書左丞趙光逢副之帥百官備法駕詣大梁|{
	唐六典大駕備五輅五輅皆有副車又有指南車記里鼔車白鷺車鸞旗車辟惡車皮軒車金根車安車四望車羊車黄鉞車豹尾車屬車一十有二若法駕則减五副輅白鷺辟惡安車四望車四分屬車之一帥讀曰率}
楊涉子直史館凝式|{
	貞觀三年置史館於門下省以它官兼領之或卑位有才者亦以直館稱以宰相涖修撰天寶後它官兼史職者曰史館修撰初入為直館元和元年宰臣裴垍建議登朝領史職者為修撰以官高人判館事未登朝者為直館}
言於涉曰大人為唐宰相而國家至此不可謂之無過况手持天子璽綬與人雖保富貴奈千載何盍辭之|{
	璽斯氏翻綬音受載子亥翻 考異曰陶岳五代史補曰凝式恐事泄即日佯狂時謂之風子按周世宗實録凝式本傳仕梁未嘗有疾唐同光初知制誥始以心疾罷明宗時及清泰帝末俱以心恙罷官天福初致仕在洛有風子之號非梁初佯狂也今不取}
涉大駭曰汝滅吾族神色為之不寧者數日|{
	楊涉之相也知必為凝式之累今乃駭凝式之言何邪為于偽翻}
策敦煌人|{
	敦徒門翻}
光逢隱之子也|{
	趙隱見二百五十二卷懿宗咸通之十三年}
盧龍節度使劉仁恭驕侈貪暴常慮幽州城不固築館於大安山|{
	薛史幽州西有名山曰大安山}
曰此山四面懸絶可以少制衆其棟宇壯麗擬於帝者選美女實其中與方士鍊丹藥求不死悉歛境内錢瘞於山顛令民間用堇泥為錢|{
	瘞於計翻堇几隱翻堇泥黏土也}
又禁江南茶商無得入境自采山中草木為茶鬻之仁恭有愛妾羅氏其子守光通焉仁恭杖守光而斥之不以為子數|{
	不齒之於諸子之列}
李思安引兵入其境所過焚蕩無餘夏四月己酉直扺幽州城下仁恭猶在大安山城中無備幾至不守|{
	幾居依翻}
守光自外引兵入登城拒守又出兵與思安戰思安敗退守光遂自稱節度使令部將李小喜元行欽將兵攻大安山仁恭遣兵拒戰為小喜所敗|{
	敗補邁翻}
虜仁恭以歸囚於别室仁恭將佐及左右凡守光素所惡者皆殺之|{
	惡烏路翻}
銀胡䩮都指揮使王思同帥部兵三千|{
	䩮盧谷翻胡䩮箭室也帥讀曰率}
山後八軍巡檢使李承約帥部兵二千|{
	盧龍以媯檀新武四州為山後}
奔河東|{
	奔李克用}
守光弟守奇奔契丹未幾亦奔河東|{
	幾居豈翻為劉守奇引河東兵伐燕張本}
河東節度使晉王克用以承約為匡霸指揮使思同為飛騰指揮使思同母仁恭之女也|{
	匡霸飛騰皆晉王所置軍都之號}
梁王始御金祥殿|{
	王溥五代會要梁受禪都大梁改正衙殿為崇元殿東殿為玄德殿内殿為金祥殿萬歲堂為萬歲殿門如殿名薛史曰梁自謂以金德王又以福建上獻鸚鵡諸州相繼上白烏白兎洎白蓮之合蒂者以為金行應運之兆故名殿曰金祥}
受百官稱臣|{
	此梁所自置百官也}
下書稱教令自稱曰寡人辛亥令諸牋表簿籍皆去唐年號但稱月日|{
	去羌呂翻}
丙辰張文蔚等至大梁盧佶聞錢傳璙等將至將水軍拒之於青澳|{
	青澳在温州東北海中俗謂之青澳門由青澳門而進舟則入温州其外則大洋也澳烏到翻海之隈厓曰澳}
錢傳瓘曰佶之精兵盡在於此不可與戰乃自安固捨舟間道襲温州|{
	安固後漢之章安也問古莧翻}
戊午温州潰擒佶斬之|{
	天祐二年盧佶䧟温州至是敗亡}
吳王鏐以都監使吳璋為温州制置使|{
	監古衘翻}
命傳璙等移兵討盧約於處州 壬戌梁王更名晃|{
	更工衡翻薛史曰時將受禪下教以本名二字異帝王之稱故改名}
王兄全昱聞王將即帝位謂王曰朱三爾可作天子乎甲子張文蔚楊涉乘輅自上源驛從冊寶諸司各備儀衛鹵簿前導百官從其後|{
	此唐之百官從才用翻}
至金祥殿前陳之王被衮冕|{
	被皮義翻}
即皇帝位張文蔚蘇循奉冊升殿進讀楊涉張策薛貽矩趙光逢以次奉寶升殿讀己|{
	巳者畢也}
降帥百官舞蹈稱賀|{
	帥讀曰率}
帝遂與文蔚等宴於玄德殿帝舉酒曰朕輔政未久此皆諸公推戴之力文蔚等慙懼俯伏不能對獨蘇循薛貽矩及刑部尚書張禕|{
	禕許韋翻}
盛稱帝功德宜應天順人帝復與宗戚飲博於宫中|{
	宗同姓也戚異姓之親也復扶又翻}
酒酣朱全昱忽以投瓊擊盆中迸散|{
	鮑宏博經曰楚辭琨蔽象碁有六博琨蔽玉箸也各投六箸行六棊故云六博用十二棊六棊白六棊黑所擲頭謂之瓊瓊有五采刻為一畫者謂之塞刻為兩畫者謂之白刻為三畫謂之黑不刻者五塞之間謂之五塞據歐史此所謂投瓊即骰子也迸北孟翻 考異曰王仁裕玉堂閑話曰骰子數匝廣王全昱忽駐不擲顧而白梁祖再呼朱三梁祖動容廣王曰你愛它爾許大官職久遠家族得安否於是大怒擲戲具於階下抵其盆而碎之喑嗚眦睚數日不止今從王禹偁五代史闕文}
睨帝曰朱三汝本碭山一民也從黄巢為盜天子用汝為四鎮節度使|{
	梁王始兼四鎮見二百六十二卷唐昭宗天復元年}
富貴極矣奈何一旦滅唐家三百年社稷|{
	唐武德元年受禪歲在著雍攝提格禪位于梁歲在彊圉單閼享國二百九十年}
自稱帝王行當族滅奚以博為帝不懌而罷乙丑命有司告天地宗廟社稷丁卯遣使宣諭州鎮|{
	皆言受禪於唐也}
戊辰大赦 |{
	考異曰梁實録編遺録薛史唐餘録皆不云大赦今從歐陽史}
改元|{
	改元開平}
國號大梁奉唐昭宣帝為濟隂王|{
	曹州濟隂郡濟子禮翻}
皆如前代故事唐中外舊臣官爵並如故以汴州為開封府命曰東都以故東都為西都廢故西京以京兆府為大安府置佑國軍於大安府|{
	唐以長安為西京洛陽為東京今梁都大梁在洛陽之東故以洛陽為西都大梁為東都而以長安為大安府}
更名魏博曰天雄軍|{
	通鑑二百六十四卷昭宗天祐元年四月已書更命魏博曰天雄軍蓋亦出朱全忠之意此複出也但未知更軍額的在何年更工衡翻}
遷濟隂王于曹州栫之以棘|{
	用左傳語栫在甸翻圍也}
使甲士守之 辛未以武安節度使馬殷為楚王|{
	馬殷不由郡王徑封國王即位之初特恩也}
以宣武掌書記太府卿敬翔知崇政院事|{
	梁崇政院即唐樞密院之職後遂廢樞密院入崇政院}
以備顧問參謀議於禁中承上旨宣於宰相而行之宰相非進對時有所奏請及已受旨應復請者皆具記事因崇政院以聞得旨則復宣於宰相翔為人沈深|{
	沈持林翻}
有智略在幕府三十餘年|{
	僖宗光啟間敬翔入汴幕至此時二十年史誤以二十為三十耳}
軍謀民政帝一以委之翔盡心勤勞晝夜不寐自言惟馬上乃得休息帝性暴戾難近|{
	近其靳翻}
人莫能測惟翔能識其意趣或有所不可翔未嘗顯言但微示持疑帝意己悟多為之改易|{
	為于偽翻}
禪代之際翔謀居多 追尊皇高祖考妣以來皆為帝后|{
	五代會要梁以舜臣朱虎為始祖四十二代至黯追尊肅祖宣元皇帝妃范氏諡宣僖皇后黯子茂琳諡敬祖光獻皇帝妃楊氏諡孝皇后茂琳子信諡憲祖昭武皇帝妃劉氏諡昭懿皇后信子誠}
皇考誠為烈祖文穆皇帝妣王氏為文惠皇后初帝為四鎮節度使凡倉庫之籍置建昌院以領之至是以養子宣武節度副使友文為開封尹判院事掌凡國之金穀友文本康氏子也 乙亥下制削奪李克用官爵|{
	李克用稱唐官用唐年號豈梁得而削奪之哉史姑書梁之初政耳}
是時惟河東鳳翔淮南稱天祐西川稱天復年號|{
	天復四年梁王劫唐昭宗遷洛改元曰天祐河東西川謂劫天子遷都者梁也天祐非唐號不可稱乃稱天復五年是歲梁滅唐河東稱天祐四年西川仍稱天復}
餘皆稟梁正朔稱臣奉貢蜀王與弘農王移檄諸道|{
	淮南楊渥爵弘農王}
云欲與岐王晉王會兵興復唐室卒無應者|{
	卒子恤翻}
蜀王乃謀稱帝下教諭統内吏民义遺晉王書云|{
	遺唯季翻}
請各帝一方俟朱温既平乃訪唐宗室立之退歸藩服晉王復書不許曰誓於此生靡敢失節|{
	史言李克用雖出於夷狄而終身為唐臣亦天性之忠純也}
唐末之誅宦官也詔書至河東晉王匿監軍張承業於斛律寺斬罪人以應詔|{
	見二百六十四卷唐昭宗天復三年斛律寺蓋高齊建霸府於晉陽斛律氏貴盛時所立}
至是復以為監軍待之加厚承業亦為之竭力|{
	為于偽翻}
岐王治軍甚寛待士卒簡易|{
	治直之翻易以䜴翻}
有告部將苻昭反者岐王直詣其家悉去左右熟寢經宿而還|{
	還從宣翻}
由是衆心悦服然御軍無紀律及聞唐亡以兵羸地蹙|{
	贏倫為翻}
不敢稱帝但開岐王府置百官名其所居為宫殿妻稱皇后|{
	李茂貞自為岐王而妻稱皇后妻之貴踰於其夫矣卒伍之雄乘時竊號私立名字以相署置豈可與之言禮乎哉}
將吏上書稱牋表鞭扇號令多擬帝者|{
	鞭鳴鞭扇雉尾扇也唐制天子視朝從禁中出則鳴鞭傳警既出西序門索扇扇合天子升御座扇開百官畢朝}
鎮海節度判官羅隱說吳王鏐舉兵討梁|{
	說式芮翻}
曰縱無成功猶可退保杭越自為東帝奈何交臂事賊為終古之羞乎鏐始以隱為不遇於唐必有怨心及聞其言雖不能用心甚義之 五月丁丑朔以御史大夫薛貽矩為中書侍郎同平章事加武順節度使趙王王鎔守太師天雄節度使鄴王

羅紹威守太傅義武節度使王處直兼侍中 契丹遣其臣袍笏美楞來通好|{
	好呼到翻}
帝遣太府少卿高頎報之|{
	頎渠希翻}
初契丹有八部|{
	歐陽修曰契丹君長曰達呼哩後分為八部一曰達爾扎部二曰伊斯琿部三曰舍琿部四曰諾爾威部五曰頗摩部六曰訥古濟部七曰濟勒勤部八曰實衮部部之長號大人路振九國志契丹古匈奴之種也代居遼澤之中潢水南岸南距榆關一千一百里榆關南距幽州七百里 考異曰蘇逢吉漢高祖實録曰契丹本姓大賀氏後分八族一曰利皆邸二曰乙失活邸三曰實活邸四曰納尾邸五曰頻沒邸六曰内會雞邸七曰集解邸八曰奚嗢邸管縣四十一縣有令八族之長皆號大人稱刺史常推一人為王建旗鼔以尊之每三年第其名以相代莊宗列傳曰咸通末其王曰習爾疆土稍大累來朝貢光啟中其王曰欽德乘中原多故北邊無備遂蠶食諸部達靼奚室韋之屬咸被驅役漢高祖實録唐餘録皆曰僖昭之際其王邪律安巴堅怙強恃勇距諸侯不受代自號天皇王後諸族邀之請用舊制安巴堅不得已傳旗鼔且曰我為長九年所得漢又頗衆欲以古漢城領本族率漢人守之自為一部諸部諾之俄設策復併諸族僭稱皇帝土地日廣大順中後唐武皇遣使與之連和大會於雲州東城延之帳中約為昆弟莊宗列傳又曰及欽德政衰安巴堅族盛自稱國王天祐二年大寇我雲中太祖遣使連和因與之面會於雲州東城延入帳中約為兄弟謂曰唐室為賊臣所簒吾以今冬大舉弟助我精騎二萬同收汴洛安巴堅許諾安巴堅既還欽德以國事傳之賈緯備史云武皇會保機故雲州城結以兄弟之好時列兵相去五里使人馬上持杯往來以展酬酢之禮安巴堅喜謂武皇曰我蕃中酋長舊法三年則罷若它日見公復相禮否武皇曰我受朝命鎮太原亦有遷移之制但不受代則可何憂罷乎安巴堅由此用其教不受諸族之代趙志忠虜庭雜紀云太祖諱億番名安巴堅父色勒迪太祖生而智八部落主愛其雄勇遂退其主約尼氏歸本部立太祖為王又云凡立王則衆部酋長皆集會議其有德行功業者立之或災害不生羣牧孶盛人民安堵則王更不替代苟不然其諸酋會衆部别選一名為王故王以番法亦甘心退焉不為衆所害又曰有韓知古韓潁康枚王奏事王郁皆中國人共勸太祖不受代新唐書載契丹八部名與漢高祖實録所載八部名多不同蓋年紀相遠虜語不常耳其實一也安巴堅云我為長九年則其在國不受代久矣非因武皇之教也今從漢高祖實録又唐餘録前云乾寧中劉仁恭鎮幽州安巴堅入寇仁恭擒其妻兄舒嚕阿巴由此十餘年不能犯塞下乃云大順中與武皇會於雲中按大順在乾寧前乾寧二年仁恭方為幽州節度大順中末也又武皇謂曰唐室為賊臣所簒吾以今冬大舉此非大順中事唐餘録誤也又編遺録開平二年五月契丹主安巴堅及前國王欽德貢方物然則於時七部猶在也}
部各有大人相與約推一人為王建旗鼓以號令諸部每三年則以次相代咸通末有習爾者為王土宇始大其後欽德為王乘中原多故時入盜邊及安巴堅為王尤雄勇五姓奚|{
	五姓奚一阿古部二綽哈部三伊實部四圖沁部五烏埒濟部各有和碩主為之酋領歐陽修曰奚當唐末居隂凉川在營府之西幽州之西北皆數百里分為五部一曰阿薈部二曰啜米部三曰粤質部四曰怒皆部五曰黑訖支部後徙居幽州之東北數百里宋白曰奚居隂凉川東去營府五百里西南去幽州九百里東南接海山川三千里後徙居琵琶川}
及七姓室韋|{
	室韋本有二十餘部其近契丹者七姓}
達靼咸役屬之安巴堅姓邪律氏|{
	歐史四夷附録曰安巴堅以其所居横帳地名為姓曰世里世里譯者謂之錫里}
恃其彊不肯受代久之安巴堅擊黄頭室韋還七部劫之於境上求如約|{
	如三年一代之約}
安巴堅不得已傳旗鼓且曰我為王九年得漢人多請帥種落|{
	帥讀曰率種章勇翻}
居古漢城與漢人守之别自為一部七部許之漢城故後魏滑鹽縣也|{
	漢志滑鹽縣屬漁陽郡後漢明帝改曰鹽田水經注大榆河自密雲城南東南流徑後魏安州舊漁陽郡之滑鹽縣南滑鹽世謂之斛鹽城西北去禦夷鎮三百里歐陽修曰漢城在炭山東南欒河上宋白曰契丹居遼澤之中潢水南岸遼澤去渝關一千一百三十里渝關去幽州一百七十四里其地東南接海東際遼河西包冷陘北界松陘山東西三千里地多松柳澤多蒲葦安巴堅居漢城在檀州西北五百五十里城北有龍門山山北有炭山炭山西是契丹室韋二界相連之地其地灤河上源西有鹽泊之利則後魏滑鹽縣也}
地宜五穀有鹽池之利其後安巴堅稍以兵擊滅七部復併為一國又北侵室韋女真|{
	女真肅慎氏之遺族黑水靺鞨即其地也入遼東著籍者號熟女真界外野處者號生女真極邊遠者號黄頭女真}
西取突厥故地擊奚滅之復立奚王而使契丹監其兵|{
	監古衘翻}
東北諸夷皆畏服之是歲安巴堅帥衆三十萬寇雲州晉王與之連和面會東城約為兄弟延之帳中縱酒握手盡歡約以今冬共擊梁 |{
	考異曰唐大祖紀年録太祖以安巴堅族黨稍盛召之天祐二年五月安巴堅領其部族三十萬至雲州東城帳中言事握手甚歡約為兄弟旬日而去留男果勒圖錫里首領濟必美楞為質約冬初大舉渡河反正會昭宗遇盜而止歐陽史曰梁將簒唐晉王李克用使人聘于契丹安巴堅以兵三十萬會克用於雲州東城握手約為兄弟期共舉兵擊梁按雲州之會莊宗列傳薛史皆在天祐四年而紀年録獨在天祐二年又云約今年冬同收汴洛會昭宗遇盜而止如此則應在天祐元年昭宗崩已前不應在二年也且昭宗遇盜則尤宜興兵討之何故止也唐室為賊臣所簒此乃四年語也其冬武皇寢疾蓋以此不果出兵耳今從之}
或勸晉王因其來可擒也王曰讐敵未滅而失信夷狄自亡之道也安巴堅留旬日乃去晉王贈以金繒數萬安巴堅留馬三千匹雜畜萬計以酬之安巴堅歸而背盟更附于梁|{
	繒慈陵翻畜許救翻背蒲妹翻更工衡翻遣使通好是附梁也}
晉王由是恨之|{
	通鑑於唐紀書李克用君臣之分也於梁紀書晉王敵國之體也吳蜀義例同}
己卯以河南尹兼河陽節度使張全義為魏王鎮海鎮東節度使吳王錢鏐為吳越王加清海節度使劉隱威武節度王審知兼侍中|{
	威武節度之下當有使字}
仍以隱為大彭王|{
	自宋武帝以彭城之裔興於江南後多以彭城之劉為名族劉隱封大彭王意蓋取此}
癸未以權知荆南留後高季昌為節度使荆南舊統八州|{
	荆歸硤夔忠萬澧朗共八州}
乾符以來寇亂相繼諸州皆為隣道所據獨餘江陵季昌到官城邑殘毁戶口彫耗季昌安集流散民皆復業 乙酉立兄全昱為廣王子友文為博王友珪為郢王友璋為福王友貞為均王友雍為賀王友徽為建王|{
	友文以養子居諸子之上友珪弑逆禍胎於此}
辛卯以東都舊第為建昌宫改判建昌院事為建昌宫使|{
	薛史曰初帝創業之時以四鎮兵馬倉庫籍繁總因置建昌院以領之至是改為宫蓋重其事也宋白曰是年中書門下奏改判建昌院事為建昌宫使仍請在京上舊邸為建昌宫}
壬辰命保平節度使康懷貞將兵八萬會魏博兵攻潞州|{
	攻晉將李嗣昭也}
甲午詔廢樞密院其職事皆入於崇政院以知院事敬翔為院使 |{
	考異曰實録四月辛未以翔知崇政院事五月甲午詔樞密院宜加為崇政院始命翔為院使蓋崇政院之名先已有之至是始併樞密院職事悉歸崇政院耳}
禮部尚書蘇循及其子起居郎楷自謂有功于梁|{
	唐昭宣帝天祐二年蘇循鼓成禪代之事故自以為有功}
當不次擢用循朝夕望為相帝薄其為人|{
	舊唐書帝紀昭宣帝天祐二年蘇楷上議駮昭宗諡全忠雄猜鑒物自楷駮諡後深鄙之既傳代之後父子皆斥逐不令在朝}
敬翔及殿中監李振亦鄙之翔言於帝曰蘇循唐之䲭梟賣國求利不可以立於惟新之朝|{
	朝直遥翻}
戊戍詔循及刑部尚書張禕等十五人並勒致仕楷斥歸田里循父子乃之河中依朱友謙|{
	為同光之初蘇循謟唐莊宗張本}
盧約以處州降吳越|{
	僖宗中和元年盧約據處州至是而亡降戶江翻}
弘農王以鄂岳觀察使劉存為西南面都招討使岳州刺史陳知新為岳州團練使廬州觀察使劉威為應援使别將許玄應為監軍將水軍三萬以擊楚楚王馬殷甚懼靜江軍使楊定真賀曰我軍勝矣殷問其故定真曰夫戰懼則勝驕則敗今淮南兵直趨吾城|{
	趨七喻翻}
是驕而輕敵也而王有懼色吾是以知其必勝也殷命在城都指揮使秦彦暉|{
	在城都指揮使盡統潭州在城之兵}
將水軍三萬浮江而下水軍副指揮使黄璠帥戰艦三百屯瀏陽口|{
	吳分長沙置瀏陽縣隋廢景龍二年於故城復置屬潭州九域志縣在州東北一百六十里水經注湘水北過漢臨湘縣西瀏水從縣西北流注之有瀏口戍璠孚袁翻瀏力周翻}
六月存等遇大雨引兵還至越堤北彦暉追之存數戰不利乃遺殷書詐降|{
	數所角翻遺唯季翻}
彦暉使謂殷曰此必詐也勿受存與彦暉夾水而陳|{
	陳讀曰陣}
存遥呼曰|{
	呼火故翻}
殺降不祥公獨不為子孫計耶彦暉曰賊入吾境而不擊奚顧子孫鼓譟而進存等走黄璠自瀏陽絶江與彦暉合擊大破之執存及知新 |{
	考異曰編遺録天祐四年四月湖南軍陳邵告捷淮南朗州水陸合勢奔衝其境馬殷出舟師於瀏陽口大破賊黨生擒偽鄂州節度使劉存按薛史梁紀馬殷奏破淮寇在六月十國紀年吳史劉存攻楚在五月敗在六月楚史亦然編遺録誤也}
禆將死者百餘人士卒死者以萬數獲戰艦八百艘威以餘衆遁歸彦暉遂拔岳州|{
	陳知新取岳州見上卷上年艦戶黯翻艘蘇遭翻}
殷釋存知新之縳慰諭之二人皆罵曰丈夫以死報主肯事賊乎遂斬之|{
	史言劉存陳知新忠壮}
許玄應弘農王之腹心也常預政事張顥徐温因其敗收斬之 楚王殷遣兵會吉州刺史彭玕攻洪州不克|{
	彭玕附楚見上卷唐昭宣帝天祐三年}
康懷貞至潞州晉昭義節度使李嗣昭副使李嗣弼閉城拒守懷貞晝夜攻之半月不克乃築壘穿蚰蜒塹而守之|{
	塹七艷翻}
内外斷絶晉王以蕃漢都指揮使周德威為行營都指揮使|{
	周德威盡統蕃漢之兵河東大將也}
帥馬軍都指揮使李嗣本馬步都虞候李存璋先鋒指揮使史建瑭鐵林都指揮使安元信|{
	五季之世諸鎮各有都指揮使而命官之職分有不同者如周德威蕃漢都指揮使則蕃漢之兵皆受指揮也行營都指揮使則行營兵皆受指揮也鐵林都指揮使安元信則鐵林軍都之指揮使耳讀史者宜各以其義類求之}
横衝指揮使李嗣源騎將安金全救潞州|{
	史言晉傾國救潞州帥讀曰率}
嗣弼克修之子|{
	克修晉王之弟見唐僖昭紀}
嗣本本姓張建瑭敬思之子|{
	史敬思見二百五十五卷唐僖宗中和四年}
金全代北人也 晉兵攻澤州|{
	攻澤州以擬康懷貞之後}
帝遣左神勇軍使范居實將兵救之 甲寅以平盧節度使韓建守司徒同平章事 武貞節度使雷彦恭會楚兵攻江陵荆南節度使高季昌引兵屯公安|{
	公安漢孱陵縣漢末劉備屯於此改名公安唐屬江陵府九域志在府南九十里}
絶其糧道彦恭敗楚兵亦走 劉守光既囚其父|{
	事見上四月}
自稱盧龍留後遣使請命秋七月甲午以守光為盧龍節度使同平章事 靜海節度使曲裕卒|{
	曲裕即曲承裕}
丙申以其子權知留後顥為節度使 |{
	考異曰諸書不見顥於裕何親按薛史六月丙辰裕卒七月丙申以靜海行營司馬權知留後曲顥起復為安南都護充節度使既云起復知其子也行營當作行軍}
雷彦恭攻岳州不克|{
	雷彦恭既與楚攻荆南尋又攻楚岳州可以見其反覆矣}
丙午賜河南尹張全義名宗奭|{
	帝舊名全忠故更全義名宗奭}
辛亥以吳越王鏐兼淮南節度使楚王殷兼武昌節度使各充本道招討制置使|{
	欲使兩浙湖南攻弘農王楊渥先分授以楊氏所統二鎮}
晉周德威壁於高河|{
	高河在潞州屯留縣東南}
康懷貞遣親騎都頭秦武將兵擊之武敗|{
	親騎梁之親兵馬軍也}
丁巳帝以亳州刺史李思安代懷貞為潞州行營都統黜懷貞為行營都虞候思安將河北兵西上|{
	上黨地高在河北諸鎮之西故曰西上上時掌翻}
至潞州城下更築重城|{
	重直龍翻}
内以防奔突外以拒援兵謂之夾寨調山東民饋軍糧德威日以輕騎抄之|{
	調徒弔翻抄楚交翻}
思安乃自東南山口築甬道屬於夾寨|{
	屬之欲翻}
德威與諸將互往攻之排牆填塹一晝夜間數十發梁兵疲於奔命夾寨中出芻牧者德威輒抄之於是梁兵閉壁不出 九月雷彦恭攻涔陽公安|{
	九域志江陵府公安縣有涔陽鎮涔鋤針翻}
高季昌擊敗之|{
	敗補邁翻}
彦恭貪殘類其父|{
	雷彦恭滿之子也}
專以焚掠為事荆湖間常被其患|{
	被皮義翻}
又附於淮南丙申詔削彦恭官爵命季昌與楚王殷討之 蜀王會將佐議稱帝皆曰大王雖忠於唐唐已亡矣此所謂天與不取者也馮涓獨獻議請以蜀王稱制曰朝興則未爽稱臣|{
	朝直遥翻爽乖也言若唐朝復興則為臣之節未乖也}
賊在則不同為惡王不從涓杜門不出|{
	馮涓馮宿之孫於唐室既亡之後義存故主眂韋莊張格輩有間矣}
王用安撫副使掌書記韋莊之謀帥吏民哭三日|{
	帥讀曰率}
己亥即皇帝位|{
	王建字光圖許州舞陽人 考異曰莊宗列傳太祖厭代建自帝於成都年號武成薛史唐餘録天祐五年九月建自帝於成都年號武成九國志此年七月即皇帝位明年改元宋庠紀年通譜天祐四年秋稱帝次年改元歐陽史十國紀年天復七年九月即位明年改元今從之}
國號大蜀辛丑以前東川節度使兼侍中王宗佶為中書令韋莊為左散騎常侍判中書門下事閬州防禦使唐道襲為内樞密使莊見素之孫也|{
	韋見素天寶之末為相}
蜀主雖目不知書好與書生談論|{
	好呼到翻}
粗曉其理|{
	粗坐五翻}
是時唐衣冠之族多避亂在蜀蜀主禮而用之使修舉故事故其典章文物有唐之遺風|{
	史言蜀王起於卒伍而能親用儒生}
蜀主長子校書郎宗仁幼以疾廢立其次子祕書少監宗懿為遂王 冬十月高季昌遣其將倪可福會楚將秦彦暉攻朗州雷彦恭遣使乞降於淮南且告急弘農王遣將泠業將水軍屯平江|{
	泠盧經翻姓也平江縣本漢羅縣地後漢分立漢昌縣孫吳立漢昌郡後又為吳昌縣隋省唐神龍元年分湘隂置昌江縣屬岳州五代改曰平江蓋後唐既滅梁楚人為之避廟諱昌字也九域志平江縣在岳州東南二百五十七里}
李饒將步騎屯瀏陽以救之楚王殷遣岳州刺史許德勲將兵拒之泠業進屯朗口|{
	朗水西南自辰錦州入朗州界經州城入大江謂之朗口}
德勲使善游者五十人以木枝葉覆其首|{
	覆扶又翻}
持長刀浮江而下夜犯其營且舉火業軍中驚擾德勲以大軍進擊大破之追至鹿角鎮擒業又破瀏陽寨擒李饒掠上高唐年而歸|{
	唐天寶二年開山洞置唐年縣屬鄂州}
斬業饒於長沙市 十一月甲申夾馬指揮使尹皓攻晉江猪嶺寨拔之|{
	梁西都有夾馬營江猪嶺在潞州長子縣西由北路達雕窠嶺}
義昌節度使劉守文聞其弟守光幽其父集將吏大哭曰不意吾家生此梟獍|{
	梟堅堯翻不孝鳥也食母獍讀如鏡破獍惡獸也食父}
吾生不如死誓與諸君討之乃發兵擊守光互有勝負天雄節度使鄴王紹威謂其下曰守光以窘急歸國|{
	窘巨隕翻謂上七月劉守光遣使請命也}
守文孤立無援滄州可不戰服也乃遺守文書|{
	遺唯季翻}
諭以禍福守文亦恐梁乘虚襲其後戊子遣使請降以子延祐為質帝拊手曰紹威折簡勝十萬兵|{
	質音致折之舌翻}
加守文中書令撫納之 初帝在藩鎮用法嚴將校有戰沒者所部兵悉斬之謂之跋隊斬|{
	將即亮翻校戶教翻跋卜末翻又蒲末翻}
士卒失主將者多亡逸不敢歸帝乃命凡軍士皆文其面以記軍號軍士或思鄉里逃去關津輒執之|{
	關往來必由之要處津濟度必由之要處}
送所屬無不死者其鄉里亦不敢容由是亡者皆聚山澤為盜大為州縣之患壬寅詔赦其罪自今雖文面亦聽還鄉里盜減什七八 淮南右都押牙米志誠等將兵度淮襲潁州克其外郭刺史張實據子城拒守 晉王命李存璋攻晉州以分上黨兵勢十二月壬戍詔河中陜州發兵救之|{
	陜失冉翻}
甲子詔發步騎五千救頴州米志誠等引去 丁卯晉兵寇洺州|{
	此救潞州之遊兵也}
淮南兵攻信州刺史危仔倡求救於吳越|{
	危全諷以仔倡守信州之地仔子之翻倡音昌又尺亮翻}


二年春正月癸酉朔蜀主登興義樓有僧抉一目以獻蜀主命飯萬僧人以報之|{
	抉於决翻飯扶晚翻}
翰林學士張格曰小人無故自殘赦其罪已幸矣不宜復崇奨以敗風俗|{
	復扶又翻敗補邁翻}
蜀主乃止 丁丑蜀以韋莊為門下侍郎同平章事 辛巳蜀主祀南郊壬午大赦改元武成 晉王疽發於首病篤周德威等退屯亂柳|{
	亂柳在潞州屯留縣界}
晉王命其弟内外蕃漢都知兵馬使振武節度使克寧監軍張承業大將李存璋吳珙|{
	珙居勇翻}
掌書記盧質立其子晉州刺史存朂為嗣 |{
	考異曰五代史闕文世傳武皇臨薨以三矢付莊宗曰一矢討劉仁恭汝不先下幽州河南未可圖也一矢擊契丹且曰安巴堅與吾把臂而盟結為兄弟誓復唐家社稷今背約附梁汝必伐之一矢滅朱溫汝能成善志死無恨矣莊宗藏三矢於武皇廟庭及討劉仁恭命幕吏以少牢告廟請一矢盛以錦囊使親將負之以為前驅凱旋之日隨俘馘納矢於太廟伐契丹滅朱氏亦如之按薛史契丹傳莊宗初嗣位亦遣使告哀賂以金繒求騎軍以救潞州契丹荅其使曰我與先王為兄弟兒即吾兒也寧有父不救子邪許出師會潞平而止廣本劉守光為守文所攻屢求救於晉晉王遣將部兵五千救之然則此時莊宗未與契丹及守光為仇也此蓋後人因莊宗成功撰此事以誇其英武耳 余按晉王實怨燕與契丹垂沒以屬莊宗容有此理莊宗之告哀於安巴堅與遣兵救劉守光此兵法所謂將欲取之必固與之也其心豈忘父之治命哉觀後來之事可見已}
曰此子志氣遠大必能成吾事爾曹善教導之辛卯晉王謂存朂曰嗣昭厄於重圍|{
	謂李嗣昭為梁兵圍於潞州也重直龍翻}
吾不及見矣俟葬畢汝與德威輩速竭力救之又謂克寧等曰以亞子累汝|{
	累良瑞翻}
亞子存朂小名也言終而卒|{
	年五十三}
克寧綱紀軍府中外無敢諠譁克寧久總兵柄有次立之勢|{
	兄死弟及以長幼之次有自立之勢}
時上黨圍未解軍中以存朂年少多竊議者人情忷忷|{
	少詩照翻忷許勇翻}
存朂懼以位讓克寧克寧曰汝冢嗣也且有先王之命誰敢違之將吏欲謁見存朂|{
	見賢遍翻}
存朂方哀哭未出張承業入謂存朂曰大孝在不墜基業多哭何為因扶存朂出襲位為河東節度使晉王|{
	張承業之扶李存朂出嗣位猶張昭之於孫權也}
李克寧首帥諸將拜賀|{
	帥讀曰率}
王悉以軍府事委之以李存璋為河東軍城使馬步都虞侯先王之時多寵借胡人及軍士侵擾市肆|{
	先王謂李克用}
存璋既領職執其尤暴横者戮之|{
	横戶孟翻}
旬月間城中肅然 吳越王鏐遣兵攻淮南甘露鎮以救信州|{
	牽制淮南之兵使之不得急攻危仔倡}
蜀中書令王宗佶於諸假子為最長|{
	王宗佶本姓甘王建為忠武軍卒掠得之養以為子及長為將數有功長知兩翻}
且恃其功專權驕恣唐道襲已為樞密使宗佶猶以名呼之道襲心衘之而事之逾謹宗佶多樹黨友蜀主亦惡之|{
	惡烏路翻}
二月甲辰以宗佶為太師罷政事|{
	為王宗佶見殺張本}
蜀以戶部侍郎張格為中書侍郎同平章事格為相多迎合主意有勝己者必以計排去之|{
	去羌呂翻為張格亂蜀張本}
初晉王克用多養軍中壯士為子寵遇如真子及晉王存朂立諸假子皆年長握兵心怏怏不伏|{
	長知兩翻怏於兩翻}
或託疾不出或見新王不拜李克寧權位既重人情多向之假子李存顥隂說克寧曰|{
	說式芮翻下同}
兄終弟及自古有之|{
	殷人之制兄終弟及自周以來父子相繼未有能易之者也李存顥以殷制動克寧耳}
以叔拜姪於理安乎天與不取後悔無及克寧曰吾家世以慈孝聞天下|{
	聞音問}
先王之業苟有所歸吾復何求|{
	復扶又翻}
汝勿妄言我且斬汝克寧妻孟氏素剛悍|{
	悍下罕翻又侯旰翻}
諸假子各遣其妻入說孟氏|{
	李克用義兒百餘人必不盡然獨存顥等為此耳史槩言之曰諸假子}
孟氏以為然且慮語泄及禍數以迫克寧克寧性怯朝夕惑於衆言心不能無動又與張承業李存璋相失數誚讓之|{
	數所角翻誚才笑翻}
又因事擅殺都虞侯李存質又求領大同節度使以蔚朔應州為巡屬|{
	唐末置應州領金城混源二縣蔚紆勿翻}
晉王皆聽之李存顥等為克寧謀因晉王過其第|{
	為于偽翻過音戈}
殺承業存璋奉克寧為節度使舉河東九州附於梁|{
	河東領并遼沁汾石忻代嵐憲九州}
執晉王及太夫人曹氏送大梁太原人史敬鎔少事晉王克用居帳下見親信|{
	少詩照翻}
克寧欲知府中隂事召敬鎔密以謀告之敬鎔陽許之入告太夫人太夫人大駭召張承業指晉王謂之曰先王把此兒臂授公等如聞外間謀欲負之但置吾母子有地勿送大梁自它不以累公|{
	累力瑞翻}
承業惶恐曰老奴以死奉先王之命此何言也晉王以克寧之謀告且曰至親不可自相魚肉吾苟避位則亂不作矣承業曰克寧欲投大王母子於虎口不除之豈有全理乃召李存璋吳珙及假子李存敬長直軍使朱守殷使隂為之備壬戍置酒會諸將於府舍伏甲執克寧存顥於座晉王流涕數之曰|{
	數所具翻}
兒曏以軍府讓叔父叔父不取今事已定奈何復為此謀|{
	復扶又翻下同}
忍以吾母子遺仇讐乎|{
	遺唯季翻仇讐謂梁也}
克寧曰此皆讒人交構夫復何言是日殺克寧及存顥|{
	李克寧之奉存朂初焉非不忠順其後外揺於讒口内溺於悍妻以至變節而殺其身地親而屬尊者居主少國疑之時可不戒哉}
癸亥酖殺濟隂王於曹州追謚曰唐哀皇帝|{
	年十七葬于濟隂縣之定陶鄉濟子禮翻}
甲子蜀兵入歸州|{
	歸州荆南巡屬不地曰入言入之而不能有其地}
執刺史

張瑭 辛未以韓建為侍中兼建昌宫使 李思安等攻潞州久不下士卒疲弊多逃亡晉兵猶屯余吾寨|{
	前漢書地理志上黨郡有余吾縣章懷太子賢曰余吾故城在潞州屯留縣西北}
帝疑晉王克用詐死欲召兵還恐晉人躡之乃議自至澤州應接歸師且召匡國節度使劉知俊將兵趣澤州|{
	趣七喻翻}
三月壬申朔帝發大梁丁丑次澤州辛巳劉知俊至壬午以知俊為潞州行營招討使 癸巳門下侍郎同平章事張文蔚卒|{
	蔚紆勿翻}
帝以李思安久無功亡將校四十餘人士卒以萬計更閉壁自守遣使召詣行在甲午削思安官爵勒歸本貫充役|{
	充役使充齊民之役}
斬監押楊敏貞晉李嗣昭固守踰年|{
	前年十二月李嗣昭入潞州去年五月康懷貞始攻之至夾寨破則是年五月也}
城中資用將竭嗣昭登城宴諸將作樂流矢中嗣昭足|{
	矢中竹仲翻}
嗣昭密拔之座中皆不覺|{
	李嗣昭登城宴樂示敵以餘暇也中矢而密拔之所以安衆也}
帝數遣使賜嗣昭詔諭降之|{
	數所角翻}
嗣昭焚詔書斬使者帝留澤州旬餘欲召上黨兵還遣使就與諸將議之諸將以為李克用死余吾兵且退上黨孤城無援請更留旬月以俟之|{
	夾寨之敗正坐此也}
帝從之命增運芻糧以饋其軍劉知俊將精兵萬餘人擊晉軍斬獲甚衆|{
	劉知俊之小捷所以驕梁兵而殱之也天之厭梁于此可見}
表請自留攻上黨車駕宜還京師帝以關中空虚慮岐人侵同華|{
	岐人謂李茂貞之兵}
命知俊休兵長子旬日退屯晉州俟五月歸鎮 蜀太師王宗佶既罷相怨望隂畜養死士謀作亂|{
	畜吁玉翻}
上表以為臣官預大臣親則長子|{
	長知兩翻}
國家之事休戚是同今儲貳未定必生厲階陛下若以宗懿才堪繼承宜早行冊禮以臣為元帥兼總六軍儻以時方艱難宗懿冲幼臣安敢持謙不當重事陛下既正位南面軍旅之事宜委之臣下臣請開元帥府鑄六軍印征戍徵發臣悉專行太子視膳於晨昏微臣握兵於環衛萬世基業惟陛下裁之蜀主怒隱忍未發以問唐道襲對曰宗佶威望内外懾服足以統御諸將蜀主益疑之己亥宗佶入見|{
	見賢遍翻}
辭色悖慢|{
	悖蒲妹翻又蒲沒翻}
蜀主諭之宗佶不退蜀主不堪其忿命衛士撲殺之|{
	撲弼角翻以華洪之得衆心猶不免於禍况甘佶之驕恃輕脱哉其死宜矣}
貶其黨御史中丞鄭騫為維州司戶衛尉少卿李綱為汶川尉|{
	綱古郎翻汶川漢綿虒地晉置汶川縣唐屬茂州九域志在州南一百里玉壘山石紐山皆在縣界汶讀曰岷}
皆賜死於路 初晉王克用卒周德威握重兵在外國人皆疑之晉王存朂召德威使引兵還|{
	還從宣翻又如字}
夏四月辛丑朔德威至晉陽留兵城外獨徒步而入伏先王柩哭極哀退謁嗣王禮甚恭衆心由是釋然|{
	史言周德威臨敵勇而事上敬}
癸卯門下侍郎同平章事楊涉罷為右僕射以吏部侍郎于兢為中書侍郎翰林學士承旨張策為刑部侍郎並同平章事兢琮之兄子也|{
	于琮見唐宣紀僖紀}
夾寨奏余吾晉兵已引去帝以援兵不能復來|{
	復扶又翻}


|{
	下同}
潞州必可取丙午自澤州南還壬子至大梁梁兵在夾寨者亦不復設備|{
	兵不可以無備也有備無患今梁之為兵也主驕於上將惰于下其敗宜矣}
晉王與諸將謀曰上黨河東之藩蔽無上黨是無河東也|{
	潞州上黨郡}
且朱温所憚者獨先王耳聞吾新立以為童子未閑軍旅|{
	閑習也}
必有驕怠之心若簡精兵倍道趣之|{
	趣七喻翻下同}
出其不意破之必矣取威定霸|{
	左傳晉先軫之言}
在此一舉不可失也張承業亦勸之行乃遣承業及判官王緘乞師於鳳翔|{
	岐王李茂貞據鳳翔}
又遣使賂契丹王阿保機求騎兵岐王衰老兵弱財竭竟不能應晉王大閲士卒以前昭義節度使丁會為都招討使|{
	丁會以潞州降晉見二百六十四卷唐昭宣帝天祐三年}
甲子帥周德威等發晉陽|{
	帥讀曰率}
淮南遣兵寇石首|{
	唐武德四年分華容縣置石首縣取縣北石首山而名屬江陵府九域志在府東南二百里孫鑑曰自安陸至竟陵兩驛皆平地南至大江並無丘陵之阻度江至石首始有淺山謂之竟陵陵至此而竟謂之石首石至此而首也}
襄州兵敗之於瀺港|{
	瀺士咸翻敗補邁翻下同}
又遣其將李厚將水軍萬五千趣荆南高季昌逆戰敗之於馬頭|{
	荆南治江陵在江北南岸曰馬頭岸正對沙市}
己巳晉王軍於黄碾距上黨四十五里|{
	黄碾村在潞州潞城縣碾紐善翻}
五月辛未朔晉王伏兵三垂岡下|{
	三垂岡在屯留縣東南}
詰旦大霧|{
	詰去吉翻}
進兵直抵夾寨梁軍無斥候不意晉兵之至將士尚未起軍中驚擾晉王命周德威李嗣源分兵為二道德威攻西北隅嗣源攻東北隅填塹燒寨鼓譟而入梁兵大潰南走招討使符道昭馬倒為晉人所殺失亡將校士卒以萬計|{
	校戶教翻}
委弃資粮器械山積周德威等至城下呼李嗣昭曰先王已薨今王自來破賊夾寨賊已去矣可開門嗣昭不信曰此必為賊所得使來誑我耳欲射之|{
	誑居况翻射而亦翻}
左右止之嗣昭曰王果來可見乎王自往呼之嗣昭見王白服大慟幾絶|{
	氣幾絶也幾居依翻}
城中皆哭遂開門初德威與嗣昭有隙晉王克用臨終謂晉王存朂曰進通忠孝吾愛之深今不出重圍|{
	重直龍翻}
豈德威不忘舊怨邪汝為吾以此意諭之若潞圍不解吾死不瞑目|{
	為于偽翻瞑莫定翻閉目也}
進通嗣昭小名也晉王存朂以告德威德威感泣由是戰夾寨甚力既與嗣昭相見遂歡好如初|{
	好呼到翻}
康懷貞以百餘騎自天井關遁歸帝聞夾寨不守大驚既而歎曰生子當如李亞子克用為不亡矣至如吾兒豚犬耳詔所在安集散兵周德威李存璋乘勝進趣澤州|{
	趣七喻翻}
刺史王班素失人心衆不為用龍虎統軍牛存節自西都將兵應接夾寨潰兵|{
	龍虎軍即唐龍武軍號梁受唐禪改武為虎王漙五代會要曰開平元年四月改左右長直為左右龍虎軍又梁以洛陽為西都}
至天井關謂其衆曰澤州要害地不可失也雖無詔旨當救之衆皆不欲曰晉人勝氣方鋭且衆寡不敵存節曰見危不救非義也畏敵彊而避之非勇也遂舉策引衆而前|{
	策馬策也}
至澤州城中人已縱火諠譟欲應晉王班閉牙城自守存節至乃定 |{
	考異曰歐陽史云存節從康懷貞攻潞州為行營排陳使晉兵已破夾城存節以餘兵歸行至天井關聞晉兵攻澤州而救之梁列傳澤州將䧟河南尹張宗奭召龍虎統軍牛存節謀之存節帥本軍及右神武羽林等軍往應接上黨回師至天井關即引衆前救澤州薛史亦同按存節若自夾城遁歸則先過澤州後至天井關豈得已過而返救之也今從梁列傳及薛史}
晉兵尋至緣城穿地道攻之存節晝夜拒戰凡旬有三日劉知俊自晉州引兵救之|{
	先是命劉知俊休兵晉州九域志晉州東南至澤州三百一十里}
德威焚攻具退保高平|{
	高平漢泫氏縣地後魏置高平縣唐屬澤州九域志在州東北八十三里 考異曰莊宗列傳云李存璋進攻澤州刺史王班弃城而去澤潞皆平今不取}
晉王歸晉陽休兵行賞以周德威為振武節度使同平章事命州縣舉賢才黜貪殘寛租賦撫孤窮伸寃濫禁姦盗境内大治|{
	治直吏翻}
以河東地狹兵少乃訓練士卒令騎兵不見敵無得乘馬部分已定無得相踰越及留絶以避險|{
	分扶問翻踰越謂左軍不得越右軍後部不得踰前部之類留絶謂軍行須聨屬不得或留止而中絶或避險而不整}
分道並進期會無得差晷刻|{
	後期必斬軍法也晷居洧翻日景也期以日中日晷過中而不至則為差餘以類推晝夜分百刻}
犯者必斬故能兼山東取河南由士卒精整故也初晉王克用平王行瑜|{
	見二百六十卷唐昭宗乾寧二年}
唐昭宗許其承制封拜時方鎮多行墨制王耻與之同每除吏必表聞至是晉王存朂始承制除吏晉王德張承業|{
	德其除李克寧之難}
以兄事之每至其第升堂拜母賜遺甚厚|{
	遺唯季翻}
潞州圍守歷年士民凍餒死者大半市里蕭條李嗣昭勸課農桑寛租緩刑數年之間軍城完復|{
	史究言李嗣昭鎮潞之績效}
靜江節度使同平章事李瓊卒|{
	李瓊取靜江見二百六十二卷唐昭宗光化三年}
楚王殷以其弟永州刺史存知桂州事 壬申更以許州忠武軍為匡國軍同州匡國軍為忠武軍陜州保義軍為鎮國軍|{
	更工衡翻陜失冉翻}
乙亥楚兵寇卾州淮南所署知州秦裴擊破之 淮南左牙指揮使張顥右牙指揮使徐温專制軍政弘農威王心不能平|{
	楊渥諡威王}
欲去之而未能|{
	去羌呂翻}
二人不自安共謀弑王分其地以臣於梁戊寅顥遣其黨紀祥等弑王於寢室 |{
	考異曰吳録顥使紀祥陳暉黎璠孫殷等執渥于寢室弑之不言徐温蓋徐鉉為温諱耳薛史因之而江南别録有獨用左衙兵事歐陽史云温顥共遣盗殺渥約分其地以臣於梁按温與顥分掌牙兵温若不同謀顥必不敢獨弑渥今從江南别録十團紀年張顥欲稱淮南留後送欵於梁以淮南易蔡州節制徐温曰揚州距汴州往返僅三千里軍府踰月無主必亂不若有所立然後圖之按顥稱留後則有主矣今不取}
詐云暴死|{
	年二十三}
己卯顥集將吏於府庭夾道及庭中堂上各列白刃令諸將悉去衛從然後入|{
	去羌呂翻從才用翻}
顥厲聲問曰嗣王已薨軍府誰當主之三問莫應顥氣色益怒幕僚嚴可求前密啓曰軍府至大四境多虞非公主之不可然今日則恐大速顥曰何謂速也可求曰劉威陶雅李遇李簡|{
	劉威在廬州陶雅在歙州李遇在宣州李簡在常州}
皆先王之等夷公今自立此曹肯為公下乎不若立幼主輔之諸將孰敢不從顥默然久之可求因屏左右|{
	屏必郢翻又卑正翻}
急書一紙置袖中麾同列謂使宅賀|{
	節度使所居為使宅賀者欲賀新君使疏吏翻}
衆莫測其所為既至可求跪讀之乃太夫人史氏教也|{
	按路振九國志渥母史氏封武昌郡君蓋渥嗣位後尊為太夫人}
大要言先王創業艱難|{
	此一段凡言先王皆指楊行密}
嗣王不幸早世隆演次當立諸將宜無負楊氏善輔導之辭旨明切顥氣色皆沮以其義正不敢奪遂奉威王弟隆演稱淮南留後東面諸道行營都統|{
	楊隆演字鴻源行密第三子薛史及路振九國志皆以隆演為渭}
既罷副都統朱瑾詣可求所居曰瑾年十六七即横戈躍馬衝犯大敵未嘗畏懾|{
	懾之涉翻}
今日對顥不覺流汗公面折之如無人|{
	折之舌翻}
乃知瑾匹夫之勇不及公遠矣因以兄事之顥以徐温為浙西觀察使鎮潤州嚴可求說温曰|{
	說式芮翻下同}
公捨牙兵而出外藩顥必以弑君之罪歸公温驚曰然則奈何可求曰顥剛愎而暗於事公能見聽請為公圖之|{
	復扶又翻為于偽翻}
時副使李承嗣|{
	李承嗣時為淮南行軍副使}
參預軍府之政可求又說承嗣曰顥凶威如此今出徐於外意不徒然恐亦非公之利承嗣深然之可求往見顥曰公出徐公于外人皆言公欲奪其兵權而殺之多言亦可畏也顥曰右牙欲之|{
	右牙者以官稱徐温}
非吾意也業已行矣|{
	事已成為業}
奈何可求曰止之易耳|{
	易以䜴翻}
明日可求邀顥及承嗣俱詣温可求瞋目責温曰|{
	瞋昌真翻}
古人不忘一飯之恩况公楊氏宿將今幼嗣初立多事之時乃求自安於外可乎温謝曰苟諸公見容温何敢自專由是不行顥知可求隂附温夜遣盗刺之|{
	刺七亦翻}
可求知不免請為書辭府主|{
	府主謂隆演也}
盜執刀臨之可求操筆無懼色|{
	操七刀翻}
盜能辨字見其辭旨忠壯曰公長者|{
	長知兩翻}
吾不忍殺掠其財以復命曰捕之不獲顥怒曰吾欲得可求首何用財為温與可求謀誅顥可求曰非鍾泰章不可泰章者合肥人時為左監門衛將軍 |{
	考異曰吳紀作鍾章十國紀年作鍾泰章今從之}
温使親將翟䖍告之|{
	翟直格翻姓也}
泰章聞之喜密結壯士三十人夜刺血相飲為誓|{
	刺七亦翻飲于禁翻}
丁亥旦直入斬顥於牙堂|{
	牙堂左右牙指揮使治事之所}
并其親近温始暴顥弑君之罪|{
	暴者發露其罪音如字}
轘紀祥等於市|{
	轘音患車裂也}
詣西宫白太夫人|{
	廣陵西宫楊行密妃史夫人居之}
太夫人恐懼大泣曰吾兒冲幼禍難如此|{
	難乃旦翻}
願保百口歸廬州公之惠也温曰張顥弑逆不可不誅夫人宜自安初温與顥謀弑威王温曰參用左右牙兵心必不一不若獨用吾兵顥不可温曰然則獨用公兵顥從之至是窮治逆黨皆左牙兵也由是人以温為實不知謀也|{
	原情定罪徐温宜與張顥同科而徐温得免弑君之名遂專吳國之政殆天啟之也治直之翻}
隆演以温為左右牙都指揮使軍府事咸取决焉以嚴可求為揚州司馬温性沈毅|{
	沈持林翻}
自奉簡儉雖不知書使人讀獄訟之辭而决之皆中情理|{
	中竹仲翻}
先是張顥用事|{
	先悉薦翻}
刑罰酷濫縱親兵剽奪市里|{
	剽匹妙翻}
温謂嚴可求曰大事已定吾與公輩當力行善政使人解衣而寢耳乃立法度禁彊暴舉大綱軍民安之|{
	古人有言盗亦有道然盗貨者小盗也盗國者大盗也觀徐温之盗國斯言豈欺我哉}
温以軍旅委嚴可求以財賦委支計官駱知祥|{
	支計官猶天臺度支郎之任也}
皆稱其職|{
	稱尺證翻}
淮南謂之嚴駱己丑契丹主安巴堅遣使隨高頎入貢|{
	高頎報使契丹見上卷五月}
且求冊命|{
	夷狄覘國勢而為去來彼以梁為彊則其背晉宜矣}
帝復遣司農卿渾特|{
	復扶又翻渾特人姓名渾戶昆翻又戶本翻}
賜以手詔約共滅沙陀乃行封冊 壬辰夾寨諸將詣闕待罪皆赦之|{
	夾寨以辛未敗壬辰諸將方詣闕待罪經二十二日}
帝賞牛存節全澤州之功以為六軍馬步都指揮使 雷彦恭引沅江環朗州以自守|{
	沅水逕朗州城南去城二十步環音宦}
秦彦暉頓兵月餘不戰彦恭守備稍懈|{
	懈古隘翻}
彦暉使禆將曹德昌帥壯士夜入自水竇|{
	帥讀曰率}
内外舉火相應城中驚亂彦暉鼓譟壞門而入|{
	壞音怪}
彦恭輕舟奔廣陵|{
	雷滿唐僖宗中和元年據朗州傳至彦恭而亡考異曰梁太祖實録云丁酉朗州軍前奏捷彦恭没溺于江今從紀年}
彦暉虜其弟彦雄送于大梁淮南以彦恭為節度副使先是澧州刺史向瓌與彦恭相表裏至是亦降於楚|{
	向瓌亦以中和元年據澧州}
楚始得澧朗二州|{
	其後破楚者亦澧朗之兵也}
蜀主遣將將兵會岐兵五萬攻雍州|{
	梁受禪改京兆府為雍州大安府雍於用翻}
晉張承業亦將兵應之六月壬寅以劉知俊為西路行營都招討使以拒之 金吾上將軍王師範家於洛陽朱友寧之妻泣訴於帝曰陛下化家為國宗族皆蒙榮寵妾夫獨不幸因王師範叛逆死於戰塲|{
	朱友寧死見二百六十四卷唐昭宗天復三年}
今仇讐猶在妾誠痛之帝曰朕幾忘此賊|{
	幾居依翻}
己酉遣使就洛陽族之使者先鑿阬於第側乃宣敕告之師範盛陳宴具與宗族列坐謂使者曰死者人所不免况有罪乎予不欲使積尸長幼無序酒既行命自幼及長引於阬中戮之死者凡二百人丙辰劉知俊及佑國節度使王重師大破岐兵于幕谷|{
	幕谷即漠谷}
晉蜀兵皆引歸 蜀立遂王宗懿為太子|{
	為宗懿不終張本}
帝欲自將撃潞州丁卯詔會諸道兵 湖南判官高郁請聽民自采茶賣於北客收其征以贍軍楚王殷從之秋七月殷奏於汴荆襄唐郢復州置回圖務|{
	回圖務猶今之回易場也}
運茶於河南北賣之以易繒纊戰馬而歸|{
	繒慈陵翻纊古謗翻}
仍歲貢茶二十五萬斤詔許之湖南由是富贍壬申淮南將吏請於李儼承制授楊隆演淮南節度

使東面諸道行營都統同平章事弘農王|{
	李儼承制事始二百六十三卷唐昭宗天復二年}
鍾泰章賞薄|{
	殺張顥之賞也}
泰章未嘗自言後踰年因醉與諸將爭言而及之或告徐温以泰章怨望請誅之温曰是吾過也擢為滁州刺史

資治通鑑卷二百六十六
















































































































































