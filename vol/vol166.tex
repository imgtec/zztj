<!DOCTYPE html PUBLIC "-//W3C//DTD XHTML 1.0 Transitional//EN" "http://www.w3.org/TR/xhtml1/DTD/xhtml1-transitional.dtd">
<html xmlns="http://www.w3.org/1999/xhtml">
<head>
<meta http-equiv="Content-Type" content="text/html; charset=utf-8" />
<meta http-equiv="X-UA-Compatible" content="IE=Edge,chrome=1">
<title>資治通鑒_167-資治通鑑卷一百六十六_167-資治通鑑卷一百六十六</title>
<meta name="Keywords" content="資治通鑒_167-資治通鑑卷一百六十六_167-資治通鑑卷一百六十六">
<meta name="Description" content="資治通鑒_167-資治通鑑卷一百六十六_167-資治通鑑卷一百六十六">
<meta http-equiv="Cache-Control" content="no-transform" />
<meta http-equiv="Cache-Control" content="no-siteapp" />
<link href="/img/style.css" rel="stylesheet" type="text/css" />
<script src="/img/m.js?2020"></script> 
</head>
<body>
 <div class="ClassNavi">
<a  href="/24shi/">二十四史</a> | <a href="/SiKuQuanShu/">四库全书</a> | <a href="http://www.guoxuedashi.com/gjtsjc/"><font  color="#FF0000">古今图书集成</font></a> | <a href="/renwu/">历史人物</a> | <a href="/ShuoWenJieZi/"><font  color="#FF0000">说文解字</a></font> | <a href="/chengyu/">成语词典</a> | <a  target="_blank"  href="http://www.guoxuedashi.com/jgwhj/"><font  color="#FF0000">甲骨文合集</font></a> | <a href="/yzjwjc/"><font  color="#FF0000">殷周金文集成</font></a> | <a href="/xiangxingzi/"><font color="#0000FF">象形字典</font></a> | <a href="/13jing/"><font  color="#FF0000">十三经索引</font></a> | <a href="/zixing/"><font  color="#FF0000">字体转换器</font></a> | <a href="/zidian/xz/"><font color="#0000FF">篆书识别</font></a> | <a href="/jinfanyi/">近义反义词</a> | <a href="/duilian/">对联大全</a> | <a href="/jiapu/"><font  color="#0000FF">家谱族谱查询</font></a> | <a href="http://www.guoxuemi.com/hafo/" target="_blank" ><font color="#FF0000">哈佛古籍</font></a> 
</div>

 <!-- 头部导航开始 -->
<div class="w1180 head clearfix">
  <div class="head_logo l"><a title="国学大师官网" href="http://www.guoxuedashi.com" target="_blank"></a></div>
  <div class="head_sr l">
  <div id="head1">
  
  <a href="http://www.guoxuedashi.com/zidian/bujian/" target="_blank" ><img src="http://www.guoxuedashi.com/img/top1.gif" width="88" height="60" border="0" title="部件查字,支持20万汉字"></a>


<a href="http://www.guoxuedashi.com/help/yingpan.php" target="_blank"><img src="http://www.guoxuedashi.com/img/top230.gif" width="600" height="62" border="0" ></a>


  </div>
  <div id="head3"><a href="javascript:" onClick="javascript:window.external.AddFavorite(window.location.href,document.title);">添加收藏</a>
  <br><a href="/help/setie.php">搜索引擎</a>
  <br><a href="/help/zanzhu.php">赞助本站</a></div>
  <div id="head2">
 <a href="http://www.guoxuemi.com/" target="_blank"><img src="http://www.guoxuedashi.com/img/guoxuemi.gif" width="95" height="62" border="0" style="margin-left:2px;" title="国学迷"></a>
  

  </div>
</div>
  <div class="clear"></div>
  <div class="head_nav">
  <p><a href="/">首页</a> | <a href="/ShuKu/">国学书库</a> | <a href="/guji/">影印古籍</a> | <a href="/shici/">诗词宝典</a> | <a   href="/SiKuQuanShu/gxjx.php">精选</a> <b>|</b> <a href="/zidian/">汉语字典</a> | <a href="/hydcd/">汉语词典</a> | <a href="http://www.guoxuedashi.com/zidian/bujian/"><font  color="#CC0066">部件查字</font></a> | <a href="http://www.sfds.cn/"><font  color="#CC0066">书法大师</font></a> | <a href="/jgwhj/">甲骨文</a> <b>|</b> <a href="/b/4/"><font  color="#CC0066">解密</font></a> | <a href="/renwu/">历史人物</a> | <a href="/diangu/">历史典故</a> | <a href="/xingshi/">姓氏</a> | <a href="/minzu/">民族</a> <b>|</b> <a href="/mz/"><font  color="#CC0066">世界名著</font></a> | <a href="/download/">软件下载</a>
</p>
<p><a href="/b/"><font  color="#CC0066">历史</font></a> | <a href="http://skqs.guoxuedashi.com/" target="_blank">四库全书</a> |  <a href="http://www.guoxuedashi.com/search/" target="_blank"><font  color="#CC0066">全文检索</font></a> | <a href="http://www.guoxuedashi.com/shumu/">古籍书目</a> | <a   href="/24shi/">正史</a> <b>|</b> <a href="/chengyu/">成语词典</a> | <a href="/kangxi/" title="康熙字典">康熙字典</a> | <a href="/ShuoWenJieZi/">说文解字</a> | <a href="/zixing/yanbian/">字形演变</a> | <a href="/yzjwjc/">金 文</a> <b>|</b>  <a href="/shijian/nian-hao/">年号</a> | <a href="/diming/">历史地名</a> | <a href="/shijian/">历史事件</a> | <a href="/guanzhi/">官职</a> | <a href="/lishi/">知识</a> <b>|</b> <a href="/zhongyi/">中医中药</a> | <a href="http://www.guoxuedashi.com/forum/">留言反馈</a>
</p>
  </div>
</div>
<!-- 头部导航END --> 
<!-- 内容区开始 --> 
<div class="w1180 clearfix">
  <div class="info l">
   
<div class="clearfix" style="background:#f5faff;">
<script src='http://www.guoxuedashi.com/img/headersou.js'></script>

</div>
  <div class="info_tree"><a href="http://www.guoxuedashi.com">首页</a> > <a href="/SiKuQuanShu/fanti/">四库全书</a>
 > <h1>资治通鉴</h1> <!--         下载:【右键另存为】即可 --></div>
  <div class="info_content zj clearfix">
  
<div class="info_txt clearfix" id="show">
<center style="font-size:24px;">167-資治通鑑卷一百六十六</center>
    資治通鑑卷一百六十六 宋 司馬光 撰<br />
<br />
  胡三省 音註<br />
<br />
  梁紀二十二【起旃蒙大淵獻盡柔兆困敦凡二年】<br />
<br />
  敬皇帝【諱方智字慧相小字法眞元帝第九子也諡法夙夜警戒曰敬】<br />
<br />
  紹泰元年【是年十月方改元】春正月壬午朔邵陵太守劉棻將兵援江陵【吳孫皓寶鼎元年分零陵北部都尉置邵陵郡隋廢邵陵郡為邵陽縣屬長沙郡唐為邵州棻符分翻將即亮翻】至三百里灘部曲宋文徹殺之帥其衆還據邵陵【帥讀曰率】 梁王詧即皇帝位於江陵【詧字理孫梁昭明太子之第三子也 考異曰周書詧傳云詧在位八載保定二年薨然則詧雖以甲戌年為魏所立乙亥年乃即位改元也】改元大定追尊昭明太子為昭明皇帝廟號高宗妃蔡氏為昭德皇后尊其母龔氏為皇太后立妻王氏為皇后子巋為皇太子【巋區韋翻又苦鬼翻又丘愧翻】賞刑制度並同王者唯上疏於魏則稱臣奉其正朔【上時掌翻】至於官爵其下亦依梁氏之舊其勲級則兼用柱國等名【勲級置以賞功柱國魏所置也為勲級之首】以諮議參軍蔡大寶為侍中尚書令參掌選事【選須絹翻】外兵參軍太原王操為五兵尚書大寶嚴整有智謀雅達政事【雅素也】文辭贍速後梁主推心任之以為謀主比之諸葛孔明操亦亞之追贈邵陵王綸太宰諡曰壯武【邵陵王綸死於大寶二年】河東王譽丞相諡曰武桓【河東王譽死于大寶元年】以莫男為武州刺史魏永夀為巴州刺史【武州巴州皆置於江陵之南岸二將尋為侯平所擒不能有二州也】 湘州刺史王琳將兵自小桂北下【據姚思亷陳書小桂嶺名輿地志連州桂陽縣漢屬桂陽郡所謂小桂也】至蒸城【蓋漢臨蒸縣古城也在衡州界】聞江陵已陷為世祖哀三軍縞素【為于偽翻縞古老翻】遣别將侯平帥舟師攻後梁【帥讀曰率】琳屯兵長沙傳檄州郡為進取之計長沙王韶及上游諸將皆推琳為盟主 齊主使清河王岳將兵攻魏安州【五代志安陸郡西魏置安州】以救江陵岳至義陽江陵陷因進軍臨江郢州刺史陸法和及儀同三司宋蒞舉州降之【降戶江翻 考異曰北史宋蒞作宋茝今從北齊紀又北齊紀云壬寅岳渡江克夏首送法和按典略甲午齊已召岳還今從典略】長史江夏太守王珉不從殺之【夏戶雅翻】甲午齊召岳還使儀同三司清都慕容儼戍郢州【北齊書慕容儼清都武安人皝之後也按魏收地形志東魏都鄴以魏郡置魏尹武安縣屬焉五代志齊官有清都尹蓋改魏尹為清都尹也 考異曰梁紀四月法和降齊遣侯瑱討之按齊主與王僧辯書云清和王岳今次漢口與陸居士相會然則法和先已降齊也今從典略】王僧辯遣江州刺史侯瑱攻郢州任約徐世譜宜豐侯循皆引兵會之【瑱他甸翻又音鎭任音壬】 辛丑齊立貞陽侯淵明為梁主使其上黨王渙將兵送之【寒山之敗貞陽没于齊】徐陵湛海珍等皆聽從淵明歸【武帝太清二年徐陵使魏魏禪於齊而梁又有侯景之亂是以留北湛海珍降見一百六十二卷三年】 二月癸丑晉安王至自尋陽入居朝堂【朝直遥翻】即梁王位時年十三以太尉王僧辯為中書監錄尚書驃騎大將軍都督中外諸軍事【驃匹妙翻騎奇寄翻】加陳霸先征西大將軍以南豫州刺史侯瑱為江州刺史湘州刺史蕭循為太尉廣州刺史蕭勃為司徒鎭東將軍張彪為郢州刺史 齊主先使殿中尚書邢子才馳傳詣建康與王僧辯書以為嗣主冲藐【傳張戀翻藐亡沼翻】未堪負荷【荷下可翻又如字】彼貞陽侯梁武猶子長沙之胤【貞陽雖為縲臣於齊而貞陽侯則梁爵也故於僧辯書稱彼貞陽侯淵明長沙王懿之子武帝兄子故曰猶子】以年以望堪保金陵故置為梁主納於彼國卿宜部分舟艦迎接今主【分扶問翻艦戶黯翻】并心一力善建良圖乙卯貞陽侯淵明亦與僧辯書求迎僧辯復書曰嗣主體自宸極受於乂祖【乂當作文蓋用受終於文祖事】明公儻能入朝同奬王室【朝直遥翻】伊呂之任僉曰仰歸意在主盟不敢聞命甲子齊以陸法和為都督荆雍等十州諸軍事太尉大都督西南道大行臺【雍於用翻】又以宋蒞為郢州刺史蒞弟簉為湘州刺史【簉初救翻】甲戌上黨王渙克譙郡【梁置合州于合肥立南譙郡于襄安縣界襄安漢之巢縣也梁置蘄縣隋改曰襄安唐復曰巢縣】己卯淵明又與僧辯書僧辯不從 魏以右僕射申徽為襄州刺史【魏既得梁雍州改曰襄州因襄陽以名州也】 侯平攻後梁巴武二州故劉棻主帥趙朗殺宋文徹以邵陵歸於王琳【帥所類翻】 三月貞陽侯淵明至東關散騎常侍裴之横禦之齊軍司尉瑾儀同三司蕭軌南侵城【晉熙郡懷寧縣漢之皖城也散悉亶翻騎奇寄翻皖戶板翻】晉州刺史蕭惠以州降之【降戶江翻】齊改晉熙為江州【齊晉州治平陽故此晉州改為江州】以尉瑾為刺史丙戍齊克東關斬裴之横俘數千人王僧辯大懼出屯姑孰謀納淵明 丙申齊主還鄴封世宗二子孝珩為廣寧王【珩音行】延宗為安德王 孫瑒聞江陵陷棄廣州還【瑒雉杏翻又音暢】曲江侯勃復據有之【去年蕭勃避王琳居始興復扶又翻】 魏太師泰遣王克沈烱等還江南【去年江陵陷王克等入長安】泰得庾季才厚遇之令參掌太史季才散私財購親舊之為奴婢者泰問何能如是對曰僕聞克國禮賢古之道也【武王克商釋箕子囚式商容閭封比干墓所謂禮賢也】今郢都覆没其君信有罪矣【江陵楚之故都古郢城及渚宫皆在其地】搢紳何咎皆為早隸【杜預曰皁隸賤官皁才早翻隸力計翻】鄙人羈旅不敢獻言誠竊哀之故私購之耳泰乃悟曰吾之過也微君遂失天下之望因出令免梁俘為奴婢者數千口 夏四月庚申齊主如晉陽 五月庚辰侯平等擒莫勇魏永夀江陵之陷也永嘉王莊生七年矣【莊世子方等之子元帝之孫】尼法慕匿之【尼女夷翻】王琳迎莊送之建康 庚寅齊主還鄴 王僧辯遣使奉啓於貞陽侯淵明定君臣之禮又遣别使奉表於齊【使疏吏翻】以子顯及顯母劉氏弟子世珍為質於淵明【質音致 考異曰典略三月辛卯遣廷尉張種等送質於鄴按淵明五月始入建康疑太早恐非】遣左民尚書周弘正至歷陽奉迎【晉武帝太康中置左民尚書唐六典曹魏置左民尚書晉惠帝置右戶尚書唐戶部尚書即左民右戶之任也】因求以晉安王為皇太子淵明許之淵明求度衛士三千僧辯慮其為變止受散卒千人【散蘇旱翻散卒者冗散之卒非敗散之卒也敗散之散去聲】庚子遣龍舟法駕迎之淵明與齊上黨王渙盟於江北辛丑自采石濟江 【考異曰梁紀七月辛丑淵明濟江甲辰入京師北齊紀五月蕭明入建業按典略五月庚子僧辯逆淵明辛丑濟江癸卯至建康今從之】於是梁輿南度齊師北返僧辯疑齊擁檝中流【檝與楫同櫂也所以撥水行船檝附船而不鼓則船定而不進】不敢就西岸齊侍中裴英起衛送淵明與僧辯會於江寧癸卯淵明入建康望朱雀門而哭逆者以哭對丙午即皇帝位改元天成以晉安王為皇太子王僧辯為大司馬陳霸先為侍中 六月庚戌朔齊發民一百八十萬築長城自幽州夏口西至恒州九百餘里【幽州夏口蓋即居庸下口也幽州軍都縣西北有居庸關濕餘水出上谷沮陽縣之東南流出關謂之下口夏當作下恒戶登翻】命定州刺史趙郡王叡將兵監之叡琛之子也【趙郡王琛即高永寶歡之弟也永寶琛字監工衘翻琛丑林翻】 齊慕容儼始入郢州而侯瑱等奄至城下儼隨方備禦瑱等不能克乘閒出擊瑱等軍大破之【閒古莧翻】城中食盡煮草木根葉及靴皮帶角食之【靴許戈翻】與士卒分甘共苦堅守半歲人無異志貞陽侯淵明立乃命瑱等解圍瑱還鎭豫章齊人以城在江外難守因割以還梁儼歸望齊主悲不自勝【勝音升】齊主呼前執其手脫帽看髪歎息久之 吳興太守杜龕【龕苦含翻】王僧辯之壻也僧辯以吳興為震州【因震澤以為州名】用龕為刺史又以其弟侍中僧愔為豫章太守【愔於今翻】 壬子齊主以梁國稱藩詔凡梁民悉遣南還 丁卯齊主如晉陽壬申自將擊柔然【將即亮翻】秋七月己卯至白道留輜重【重直用翻】帥輕騎五千追柔然壬午及之於懷朔鎭齊主親犯矢石頻戰大破之至於沃野獲其酋長【水經注雲中郡有白道嶺白道川帥讀曰率騎奇寄翻酋慈秋翻長知兩翻】及生口二萬餘牛羊數十萬壬申還晉陽八月辛巳王琳自蒸城還長沙 齊主還鄴以佛道二敎不同欲去其一集二家論難於前【去羌呂翻難乃旦翻】遂敇道士皆剃髪為沙門有不從者殺四人乃奉命於是齊境皆無道士【今道家有太霄琅書經云人行大道號曰道士士者何理也事也身心順理唯道是從從道為事故曰道士余按此說是道流借吾儒經解大義以演繹道士二字道家雖曰宗老子而西漢以前未嘗以道士自名至東漢始有張道陵于吉等其實與佛敎皆起于東漢之時】 初王僧辯與陳霸先共滅侯景【見一百六十四卷世祖承聖元年】情好甚篤僧辯為子頠娶霸先女【好呼到翻為于偽翻頠魚委翻】會僧辯有母喪未成昏僧辯居石頭城霸先在京口僧辯推心待之頠兄顗屢諫不聽【顗魚豈翻】及僧辯納貞陽侯淵明霸先遣使苦爭之【使疏吏翻】往返數四僧辯不從霸先竊歎謂所親曰武帝子孫甚多唯孝元能復讎雪恥【謂誅滅侯景也】其子何罪而忽廢之吾與王公並處託孤之地【處昌呂翻】而王公一旦改圖外依戎狄援立非次其志欲何所為乎【僧辯立淵明名不正而言不順故姦雄得因以為資】乃密具袍數千領及錦綵金銀為賞賜之具會有吿齊師大舉至夀春將入寇者僧辯遣記室江旰告霸先使為之備霸先因是留旰於京口【旰古汗翻】舉兵襲僧辯九月壬寅召部將侯安都周文育及安陸徐度錢塘杜稜謀之【將即亮翻下同】稜以為難霸先懼其謀泄以手巾絞稜【今人盥洗以布拭手長七八尺謂之手巾】悶絶於地因閉於别室部分將士【分扶問翻】分賜金帛以弟子著作郎曇朗鎮京口知留府事【曇朗霸先母弟休先之子曇徒含翻】使徐度侯安都帥水軍趨石頭【帥所類翻下同趨七喻翻】霸先帥馬步自江乘羅落會之【江乘羅落江乘縣之羅落橋自江乘至羅落橋京口趨建康之大路劉裕伐桓玄由此】是夜皆發召杜稜與同行知其謀者唯安都等四將外人皆以為江旰徵兵禦齊不之怪也甲辰安都引舟艦將趣石頭【艦戶黯翻趣七喻翻】霸先控馬未進安都大懼追霸先罵曰今日作賊事勢已成生死須决在後欲何所望若敗俱死後期得免斫頭邪霸先曰安都嗔我乃進【霸先控馬踟蹰以觀安都之意見安都决死前向乃進嗔昌眞翻恚怒也】安都至石頭城北棄舟登㟁石頭城北接岡阜不甚危峻安都被甲帶長刀軍人捧之投於女垣内【被皮義翻女垣城上堞也】衆隨而入進及僧辯卧室霸先兵亦自南門入僧辯方視事外白有兵俄而兵自内出僧辯遽走遇子頠與俱出閤帥左右數十人苦戰於聽事前【聽讀曰廳】力不敵走登南門樓拜請求哀霸先欲縱火焚之僧辯與頠俱下就執霸先曰我有何辜公欲與齊師賜討且曰何意全無備僧辯曰委公北門何謂無備【京口為建康北門】是夜霸先縊殺僧辯父子既而竟無齊兵亦非霸先之譎也【譎古宂翻】前青州刺史新安程靈洗帥所領救僧辯力戰於石頭西門軍敗霸先遣使招諭久之乃降【使疏吏翻降戶江翻】霸先深義之以為蘭陵太守使助防京口【守式又翻】乙巳霸先為檄布告中外列僧辯罪狀且曰資斧所指唯王僧辯父子兄弟其餘親黨一無所問丙午貞陽侯淵明遜位出就邸 【考異曰梁書九月丙午帝即皇帝位十月己巳大赦改元按長歷丙午九月二十九日己巳十月二十二日豈有即位二十四日始改元大赦乎蓋丙午復梁王位十月乃即帝位耳典略丁未廢貞陽侯出就邸今並從陳書】百僚上晉安王表勸進【上時掌翻】冬十月己酉晉安王即皇帝位大赦改元【改元紹泰】中外文武賜位一等以貞陽侯淵明為司徒封建安公告齊云僧辯隂圖簒逆故誅之仍請稱臣於齊永為藩國齊遣行臺司馬恭與梁人盟於歷陽 辛亥齊主如晉陽 壬子加陳霸先尚書令都督中外諸軍事車騎將軍揚南徐二州刺史癸丑以宜豐侯循為太保建安公淵明為太傅曲江侯勃為太尉王琳為車騎將軍開府儀同三司【帝之初為梁王也諸藩皆進官獨不及王琳抑王僧辯雅知王琳之不可制邪】 戊午尊帝所生夏貴妃為皇太后【夏戶雅翻】立妃王氏為皇后 杜龕恃王僧辯之勢【龕苦含翻】素不禮於陳霸先在吳興每以法繩其宗族霸先深怨之及將圖僧辯密使兄子蒨還長城【長城縣霸先與其宗族世居之晉太康三年分烏程立長城縣屬吳興郡今湖州長興縣是也在湖州西北七十里蒨七見翻】立柵以備龕僧辯死龕據吳興拒霸先義興太守韋載以郡應之 【考異曰典略作韋載今從梁陳書 今按典略若作韋載則與梁陳書同不須考異矣】吳郡太守王僧智僧辯之弟也亦據城拒守史【考異曰南云僧智奔任約今從典略】陳蒨至長城收兵纔數百人杜龕遣其將杜泰將精兵五千奄至將士相視失色【將即亮翻下同】蒨言笑自若部分益明【分扶問翻】衆心乃定泰晝夜苦攻數旬不克而退霸先使周文育攻義興義興屬縣卒皆霸先舊兵善用弩韋載收得數十人繫以長鎖命所親監之【監工銜翻】使射文育軍約曰十射不兩中者死【射而亦翻中竹仲翻】故每發輒斃一人文育軍稍却載因於城外據水立栅相持數旬杜龕遣其從弟北叟將兵拒戰【從才用翻下同】北叟敗歸於義興霸先聞文育軍不利辛未自表東討留高州刺史侯安都石州刺史杜稜宿衛臺省【五代志永平郡梁置石州隋後改曰藤州宋白曰藤州治鐔津縣漢猛陵縣也】甲戌軍至義興丙子拔其水柵譙秦二州刺史徐嗣徽從弟嗣先僧辯之甥也僧辯死嗣先亡就嗣徽嗣徽以州入於齊【五代志江都郡清流縣梁置新昌郡及譙州又六合縣置秦郡及秦州】及陳霸先東討義興嗣徽密結南豫州刺史任約將精兵五千乘虚襲建康是日襲據石頭遊騎至闕下侯安都閉門藏旗幟示之以弱令城中曰登陴闚賊者斬【幟昌志翻陴頻彌翻】及夕嗣徽等收兵還石頭安都夜為戰備將旦嗣徽等又至安都帥甲士三百開東西掖門出戰【臺城正南端門其左右二門曰東西掖門帥讀曰率】大破之嗣徽等奔還石頭不敢復逼臺城【復扶又翻】陳霸先遣韋載族弟翽齎書諭載【翽呼會翻】丁丑載及杜北叟皆降【降戶江翻】霸先厚撫之以翽監義興郡【監工銜翻】引載置左右與之謀議霸先卷甲還建康【卷讀曰捲 考異曰梁書十一月庚寅霸先還建康按庚寅十一月十三日太晩且庚寅以前霸先已有在建康與齊相拒事迹今從陳書】使周文育討杜龕救長城將軍黄他攻王僧智於吳郡不克霸先使寧遠將軍裴忌助之忌選所部精兵輕行倍道自錢塘直趣吳郡【按陳霸先自義興還建康遣裴忌助黄他攻吳郡自錢塘直趣吳郡非路也錢塘必誤趣七喻翻】夜至城下鼓譟薄之【薄伯各翻】僧智以為大軍至輕舟奔吳興忌入據吳郡因以忌為太守十一月己卯齊遣兵五千度江據姑孰以應徐嗣徽任約陳霸先使合州刺史徐度立柵於冶城庚寅齊又遣安州刺史翟子崇楚州刺史劉士榮淮州刺史柳達摩【五代志鍾離郡梁置北徐州齊改曰楚州管下定遠縣梁置安州江都郡山陽縣有淮隂郡東魏置淮州翟直格翻】將兵萬人於胡墅度米三萬石馬千匹入石頭【胡墅在大江北㟁對石頭城墅神與翻】霸先問計於韋載載曰齊師若分兵先據三吳之路略地東境則時事去矣今可急於淮南因侯景故壘築城以通東道轉輸【淮南秦淮之南也輸式喻翻下運輸同】分兵絶彼之糧運則齊將之首旬日可致【將即亮翻】霸先從之癸未使侯安都夜襲胡墅 【考異曰典略作己巳按長歷是月戊寅朔無己巳今從陳書】燒齊船千餘艘【艘蘇遭翻】仁威將軍周鐵虎斷齊運輸【斷音短】擒其北徐州刺史張領州【五代志琅邪郡舊置北徐州】仍遣韋載於大航築侯景故壘使杜稜守之【航戶剛翻】齊人於倉門水南立二柵【倉門石頭倉城門水南秦淮水之南】與梁兵相拒壬辰齊大都督蕭軌將兵屯江北 初齊平秦王歸彦幼孤高祖令清河昭武王岳養之【歸彦高歡族弟也歸彦父徽于歡有舊恩故歡憐其孤而命岳養之歡廟號高祖】岳情禮甚薄歸彦心銜之及顯祖即位歸彦為領軍大將軍大被寵遇【被皮義翻】岳謂其德已更倚賴之岳屢將兵立功有威名而性豪侈好酒色【好呼到翻】起第於城南【城南鄴城之南】聽事後開巷【聽讀曰廳】歸彦譖之於帝曰清河僭擬宫禁制為永巷但無闕耳帝由是惡之【惡烏路翻】帝納倡婦薛氏於後宫【倡尺良翻優也】岳先嘗因其姊迎之至第帝夜遊於薛氏家其姊為其父乞司徒【為于偽翻】帝大怒懸其姊鋸殺之讓岳以姦岳不服帝益怒乙亥使歸彦鴆岳岳自訴無罪歸彦曰飲之則家全飲之而卒葬贈如禮薛嬪有寵於帝【嬪毘賓翻】久之帝忽思其與岳通無故斬首藏之於懷出東山宴飲勸酬始合忽探出其首投於柈上【探吐南翻柈蒲官翻】支解其尸弄其髀為琵琶一座大驚帝方收取對之流涕曰佳人難再得【漢李延年歌曰北方有佳人絶世而獨立一顧傾人城再顧傾人國寧不知傾城與傾國佳人難再得】載尸以出被髪步哭而隨之【被皮義翻】 甲辰徐嗣徽等攻冶城柵陳霸先將精甲自西明門出擊之嗣徽等大敗留柳達摩等守城自往采石迎齊援 以郢州刺史宜豐侯循為太保廣州刺史曲江侯勃為司空并徵入侍循受太保而辭不入勃方謀舉兵遂不受命 鎭南將軍王琳侵魏魏大將軍豆盧寧禦之【姓氏志豆盧本姓慕容氏燕北地王精降魏北人謂歸義為豆盧因賜以為氏】十二月癸丑侯安都襲秦郡破徐嗣徽柵俘數百人<br />
<br />
  收其家得其琵琶及鷹遣使送之曰昨至弟處得此今以相還【使疏吏翻下同】嗣徽大懼丙辰陳霸先對冶城立航【航戶剛翻連舟為橋也】悉度衆軍攻其水南二柵【即倉門水南二柵】柳達摩等度淮置陳【陳讀曰陣】霸先督兵疾戰縱火燒柵齊兵大敗爭舟相擠【擠牋西翻又子細翻】溺水者以千數呼聲震天地【溺徒狄翻呼火故翻】盡收其船艦是日嗣徽與任約引齊兵水步萬餘人還據石頭霸先遣兵詣江寧據要險嗣徽等水步不敢進頓江寧浦口霸先遣侯安都將水軍襲破之嗣徽等單舸脱走【舸古我翻】盡收其軍資器械己未霸先四面攻石頭城中無水升水直絹一匹庚申達摩遣使請和於霸先且求質子【請和而求質子者恐還以無功得罪欲以質子藉手質音致】時建康虚弱糧運不繼朝臣皆欲與齊和【朝直遥翻】請以霸先從子曇朗為質【曇朗時留鎭京口從才用翻曇苦含翻】霸先曰今在位諸賢欲息肩於齊【左傳鄭成公疾子駟請息肩于晉杜預注曰以負擔諭】若違衆議謂孤愛曇朗不恤國家今决遣曇朗弃之寇庭齊人無信謂我微弱必當背盟【背蒲妹翻】齊寇若來諸君須為孤力鬬也【霸先知齊人恥於無功必增兵復至故先以此諭衆責其効死為于偽翻】乃與曇朗及永嘉王莊丹陽尹王冲之子珉為質【與當作以則文意明順】與齊人盟於城外【城外者石頭城外】將士恣其南北【徐嗣徽等南人恣其南柳達摩等北人恣其北恣其南北言唯意所適也】辛酉霸先陳兵石頭南門送齊人歸北徐嗣徽任約皆奔齊收齊馬仗船米不可勝計【勝音升】齊主誅柳達摩壬戌齊和州長史烏丸遠自南州奔還歷陽【劉昫曰齊梁通和置和州於歷陽郡烏丸蓋出于東胡烏丸之種因以為姓】江寧令陳嗣黄門侍郎曹朗據姑孰反霸先命侯安都等討平之霸先恐陳曇朗亡竄自帥步騎至京口迎之【帥讀曰率】 交州刺史劉元偃帥其屬數千人歸王琳 魏以侍中李遠為尚書左僕射 魏益州刺史宇文貴使譙淹從子子嗣誘說淹以為大將軍【從才用翻說式芮翻】淹不從斬子嗣貴怒攻之淹自東遂寧徙屯墊江【晉于德陽縣界東南置遂寧郡五代志遂寧郡方義縣梁曰小溪置東遂寧郡墊江縣漢屬巴郡梁為楚州治所隋為渝州墊音疊】 初晉安民陳羽【吳立東安縣晋武帝更名晉安太康三年分建安立晉安郡五代志建安郡南安縣舊日晉安今之泉州即其地】世為閩中豪姓其子寶應多權詐郡中畏服侯景之亂晉安太守賓化侯雲以郡讓羽羽老但治郡事令寶應典兵時東境荒饉而晉安獨豐衍寶應數自海道出寇抄臨安永嘉會稽【沈約志吳分餘杭為臨水縣晉武帝太康元年更名臨安五代志無臨安郡及臨安縣但有餘杭郡耳數所角翻抄楚交翻會工外翻】或載米粟與之貿易由是能致富彊侯景平世祖因以羽為晉安太守及陳霸先輔政羽求傳位於寶應霸先許之【為後陳寶應亂閩中張本】是歲魏宇文泰諷淮安王育上表請如古制降爵為<br />
<br />
  公於是宗室諸王皆降為公 突厥木杆可汗擊柔然鄧叔子滅之【厥九勿翻杆公旦翻可從刋入聲汗音寒】叔子收其餘燼奔魏木杆西破嚈噠【嚈益涉翻噠當割翻又宅軋翻】東走契丹北并契骨【契骨即唐之結骨唐書曰黠戞斯古堅昆國或曰居勿或曰結骨盖堅昆語訛為結骨稍號紇骨亦曰紇扢斯契丹欺訖翻又音喫契骨苦結翻】威服塞外諸國其地東自遼海西至西海長萬餘里【長直亮翻】南自沙漠以北五六千里皆屬焉木杆恃其彊請盡誅鄧叔子等於魏使者相繼於道太師泰收叔子以下三千餘人付其使者盡殺之於青門外【長安城東出南頭第一門曰霸城門民見門色青名曰青城門或曰青門秦東陵侯召平種瓜於青門外即其地使疏吏翻】 初魏太師泰以漢魏官繁命蘇綽及尚書令盧辯依周禮更定六官【更工衡翻】<br />
<br />
  太平元年【是年九月方改元太平】春正月丁丑 魏初建六官以宇文泰為太師大冢宰柱國李弼為太傅大司徒趙貴為太保大宗伯【宗伯以上以三公兼六卿之職北史盧辯傳置太師太傅太保各一人是曰三孤】獨孤信為大司馬于謹為大司寇侯莫陳崇為大司空自餘百官皆倣周禮 戊寅大赦其與任約徐嗣徽同謀者一無所問癸未陳霸先使從事中郎江旰說徐嗣徽使南歸【說式芮翻下因說同】嗣徽執旰送齊 陳蒨周文育合軍攻杜龕於吳興龕勇而無謀嗜酒常醉其將杜泰隂與蒨等通龕與蒨等戰敗泰因說龕使降【將即亮翻說式芮翻降戶江翻】龕然之其妻王氏曰【王氏僧辯女也】霸先讎隙如此何可求和因出私財賞募復擊蒨等大破之【復扶又翻】既而杜泰降於蒨龕尚醉未覺【覺古效翻又如字】蒨遣人負出於項王寺前斬之【項羽起吳下故後人為立寺于吳興 考異曰梁書太平元年正月癸未杜龕降詔賜死陳書紹泰元年十二月杜龕以城降明年正月癸未誅杜龕於吳興龕從弟北叟司馬沈孝敦並賜死典略魏恭帝二年十二月蒨命劉澄等攻龕大敗之龕乃降明年正月丁亥周鐵虎送杜龕祠項王神使力士拉龕于坐從弟北叟司馬沈孝敦並賜死今從南史】王僧智與其弟豫章太守僧愔俱奔齊【愔于今翻 考異曰梁書南史王僧辯傳僧辯既亡僧智得就任約約敗走僧智肥不能行又遇害僧智弟僧愔位譙州刺史征蕭勃及聞兄死引軍還時吳州刺史羊亮隸在僧愔下與僧愔不平密召侯瑱見擒僧愔以名義責瑱瑱乃委罪于將羊鯤斬之僧愔復得奔齊陳書南史侯瑱傳則云僧辯使其弟僧愔與瑱共討蕭勃及陳武帝誅僧辯僧愔隂欲圖瑱及奪其軍瑱知之盡收僧愔徒黨僧愔奔齊典略魏恭帝三年正月初僧愔與瑱共討曲江侯勃至是吳州刺史羊亮說僧愔襲瑱而翻以告瑱瑱攻之僧愔奔齊凡此諸說莫知孰是今約其梗槩言之】東揚州刺史張彪素為王僧辯所厚不附霸先二月庚戌陳蒨周文育輕兵襲會稽彪兵敗走入若邪山中【簡文帝大寶元年張彪起兵于若邪山邪音耶】蒨遣其將吳興章昭達追斬之【將即亮翻下同】東陽太守留異餽蒨糧食霸先以異為縉州刺史【因縉雲山而置縉州五代志處州栝蒼縣有縉雲山縉音晉】江州刺史侯瑱本事王僧辯亦擁兵據豫章及江州不附霸先霸先以周文育為南豫州刺史使將兵擊湓城庚申又遣侯安都周鐵虎將舟師立柵於梁山以備江州 癸亥徐嗣徽任約襲采石執戍主明州刺史張懷鈞送於齊【五代志日南郡交谷縣梁置明州張懷鈞蓋帶刺史而戌采石也】 後梁主擊侯平於公安平與長沙王韶引兵還長沙王琳遣平鎭巴州三月壬午詔雜用古今錢 戊戌齊遣儀同三司蕭軌庫狄伏連堯難宗東方老等與任約徐嗣徽合兵十萬入寇出柵口【柵口柵江口也在今和州歷陽縣西南百五十里與無為軍分界即古之濡須口宋白曰廬州東南至柵口今謂之新婦口三百八十四里對㟁即舊南陵縣地對㟁為繁昌縣】向梁山陳霸先帳内盪主黄叢逆擊破之【盪主主勇士以突盪敵人】齊師退保蕪湖霸先遣定州刺史沈泰等就侯安都共據梁山以禦之周文育攻湓城未克召之還夏四月丁巳霸先如梁山廵撫諸軍 乙丑齊儀同三司婁叡討魯陽蠻破之 侯安都輕兵襲齊行臺司馬恭於歷陽 【考異曰梁書云壬午安都襲恭按長歷是月乙巳朔無壬午】大破之俘獲萬計 魏太師泰尚孝武妹馮翊公主生略陽公覺姚夫人生寧都公毓毓於諸子最長【五代志西城郡安康縣舊曰寧都毓余六翻長知兩翻下同】娶大司馬獨孤信女泰將立嗣謂公卿曰孤欲立子以嫡恐大司馬有疑如何衆默然未有言者尚書左僕射李遠曰夫立子以嫡不以長【春秋公羊傳之言】略陽公為世子公何所疑若以信為嫌請先斬之遂拔刀而起泰亦起曰何至於是信又自陳解遠乃止於是羣公並從遠議遠出外拜謝信曰臨大事不得不爾信亦謝遠曰今日賴公决此大議遂立覺為世子 太師泰北廵 五月齊人召建安公淵明詐許退師 【考異曰典略云五月齊主在東山飲酒投杯赫怒召魏收于前立為制書欲自將西討長安令上黨王渙將兵伐梁于是渙南侵按梁陳北齊帝紀及渙傳皆無是事今去之】陳霸先具舟送之癸未淵明疽發背卒甲申齊兵發蕪湖庚寅入丹陽縣【此丹陽縣乃漢古縣非今鎭江府之丹楊縣也據沈約志晉武帝太康三年分丹陽縣立于湖縣于湖今太平州也丹陽縣地當在太平州東北】丙申至秣陵故治【沈約曰秣陵本治去京邑六十里今故治村是也晉安帝義熙九年移治京邑在鬭塲鬬塲猶今言敎塲晉成帝咸和中詔内外諸軍戲于南郊之塲因名戲塲亦曰鬬塲】陳霸先遣周文育屯方山【丹陽記秦始皇鑿方山其斷處為瀆則今淮水】徐度頓馬牧【馬牧牧馬之地】杜稜頓大航南以禦之 齊漢陽敬懷王洽卒【洽齊主之弟】 辛丑齊人跨淮立橋柵度兵夜至方山徐嗣徽等列艦於青墩至於七磯以斷周文育歸路【艦戶黯翻下同墩音敦斷音短】文育鼓譟而發嗣徽等不能制至旦反攻嗣徽嗣徽驍將鮑砰獨以小艦殿軍【驍堅堯翻將即亮翻下同砰普耕翻殿丁練翻】文育乘單舴艋與戰【舴陟格翻艋莫梗翻舴艋小船一舟曰單】跳入艦中【跳他弔翻】斬砰仍牽其艦而還【還從宣翻又如字】嗣徽衆大駭因留船蕪湖自丹陽步上【上時掌翻下槊上同】陳霸先追侯安都徐度皆還【追梁山之軍還建康以禦齊師】癸卯齊兵自方山進及倪塘【倪塘在臺城東】游騎至臺【騎奇寄翻】建康震駭帝總禁兵出頓長樂寺【樂音洛】内外纂嚴霸先拒嗣徽等於白城【白城當在湖熟縣界】適與周文育會將戰風急霸先曰兵不逆風文育曰事急矣何用古法抽槊上馬先進【槊色角翻】風亦尋轉殺傷數百人侯安都與嗣徽等戰於耕壇南【天子親耕籍田祭先農于田所故有耕壇宋文帝元嘉二十一年令司空司農京尹令尉度官之辰地八里之外整制千畝中開阡陌立先農壇於中阡西陌南設御耕壇于中阡東陌北】安都帥十二騎突其陳破之【騎奇寄翻陳讀曰陣】生擒齊儀同三司乞伏無勞 【考異曰南史作乞伏無芳今從陳書】霸先潜撤精卒三千配沈泰度江襲齊行臺趙彦深於瓜步獲艦百餘艘粟萬斛【艘蘇遭翻】六月甲辰齊兵潛至鍾山侯安都與齊將王敬寶戰於龍尾【鍾山之龍尾也自山趾築道陂陁以登山曰龍尾】軍主張纂戰死丁未齊師至幕府山【幕府山在今建康城西二十五里晉琅邪王初度江丞相王導建幕府其上因名】霸先遣别將錢明將水軍出江乘邀擊齊人糧運盡獲其船米齊軍乏食殺馬驢食之庚戌齊軍踰鍾山霸先與衆軍分頓樂遊苑東及覆舟山北斷其衝要【斷音短】壬子齊軍至玄武湖西北將據北郊壇【晉成帝立北郊壇於覆舟山南】衆軍自覆舟東移頓壇北與齊人相對會連日大雨平地水丈餘齊軍晝夜坐立泥中足指皆爛懸鬲以㸑【鬲音歷爾雅鼎欵足者謂之鬲說文鬲鼎屬也實五觳斗二升曰觳】而臺中及潮溝北路燥【潮溝吳孫權所開以引潮抵于秦淮】梁軍每得番易時四方壅隔糧運不至建康戶口流散徵求無所甲寅少霽【少詩沼翻】霸先將戰調市人得麥飯【調徒弔翻】分給軍士士皆飢疲會陳蒨饋米三千斛鴨千頭霸先命炊米煮鴨人人以荷葉裹飯婫以鴨肉數臠【婫公渾翻以鴨内蓋飯上曰婫今江東人猶謂以物蒙頭曰婫臠力兖翻】乙卯未明蓐食比曉【比必利翻】霸先帥麾下出幕府山侯安都謂其部將蕭摩訶曰卿驍勇有名千聞不如一見【帥讀曰率將即亮翻驍堅堯翻】摩訶對曰今日令公見之及戰安都墜馬齊人圍之摩訶單騎大呼直衝齊軍齊軍披靡安都乃免【騎奇寄翻呼火故翻披普彼翻】霸先與吳明徹沈泰等衆軍首尾齊舉縱兵大戰安都自白下引兵横出其後齊師大潰斬獲數千人相蹂踐而死者不可勝計【蹂人九翻踐慈演翻勝音升】生擒徐嗣徽及弟嗣宗斬之以徇追奔至於臨沂【晉成帝咸康元年桓温領南琅邪太守鎮江乘蒲州之金城求割丹陽之江乘縣境立郡又分江乘地立臨沂縣宋白曰臨沂山西北臨大江】其江乘攝山鍾山等諸軍相次克捷【攝山在今建康城北四十五里江乘地記曰有草可以攝生故名】虜蕭軌東方老王敬寶等將帥凡四十六人【將即亮翻帥所類翻】其軍士得竄至江者縛荻筏以濟【荻亭歷翻葦也】中江而溺流尸至京口翳水彌岸唯任約王僧愔得免丁巳衆軍出南州燒齊舟艦戊午大赦己未解嚴軍士以賞俘貿酒一人裁得一醉【貿音茂】庚申斬齊將蕭軌等齊人聞之亦殺陳曇朗霸先啓解南徐州以授侯安都【賞其功也】 侯平頻破後梁軍以王琳兵威不接更不受指麾琳遣將討之平殺巴州助防呂旬收其衆奔江州侯瑱與之結為兄弟琳軍勢益衰乙丑遣使奉表詣齊并獻馴象【安南出象處曰象山歲一捕之縛欄道旁中為大穽以雌象前行為媒遺甘蔗於地傅藥蔗上雄象來食蔗漸引入欄閉其中就穽中敎習馴擾之始甚咆哮穽深不可出牧者以言語諭之久則漸解人意使疏吏翻馴松倫翻】江陵之陷也琳妻蔡氏世子毅皆没於魏琳又獻欵於魏以求妻子亦稱臣於梁 齊丁匠三十餘萬脩廣三臺宫殿【三臺在鄴城曹操所築】 齊顯祖之初立也留心政術務存簡靖坦於任使【謂任使之際坦懷待人】人得盡力又能以法馭下或有違犯不容勲戚内外莫不肅然至於軍國機策獨决懷抱每臨行陳【行戶剛翻陳讀曰陣】親當矢石所向有功數年之後漸以功業自矜遂嗜酒淫泆【泆弋乙翻淫放也】肆行狂暴或身自歌舞盡日通宵或散髪胡服雜衣錦綵【衣於既翻】或袒露形體塗傅粉黛或乘驢牛橐駞白象不施鞍勒或令崔季舒劉桃枝負之而行擔胡皷拍之【胡皷以手拍之成聲劉昫曰腰皷大者瓦小者木皆廣首而纖腹本胡皷也擔都甘翻】勲戚之第朝夕臨幸游行市里街坐巷宿或盛夏日中暴身【暴讀曰曝】或隆冬去衣馳走從者不堪【去羌呂翻從才用翻】帝居之自若三臺構木高二十七丈【高居報翻】兩棟相距二百餘尺工匠危怯皆繫繩自防帝登脊疾走殊無怖畏【脊棟脊也怖普布翻】時復雅儛【復扶又翻儛與舞同】折旋中節【中竹仲翻】傍人見者莫不寒心嘗於道上問婦人曰天子何如曰顛顛癡癡何成天子帝殺之婁太后以帝酒狂舉杖擊之曰如此父生如此兒帝曰即當嫁此老母與胡太后大怒遂不言笑帝欲太后笑自匍匐【匍音蒲匐莫北翻】以身舉牀墜太后於地頗有所傷既醒大慚恨使積柴熾火欲入其中太后驚懼親自持挽強為之笑曰曏汝醉耳【強其兩翻為于偽翻】帝乃設地席命平秦王歸彦執杖口自責數【自責而數罪也數所具翻】脫背就罰謂歸彦曰杖不出血當斬汝太后前自抱之帝流涕苦請乃笞脚五十然後衣冠拜謝悲不自勝【勝音升】因是戒酒一旬又復如初帝幸李后家以鳴鏑射后母崔氏【射而亦翻】罵曰吾醉時尚不識太后老婢何事馬鞭亂擊一百有餘雖以楊愔為相使進厠籌以馬鞭鞭其背流血浹袍嘗欲以小刀剺其腹【愔於今翻相息亮翻浹即協翻剺力之翻劃也】崔季舒託俳言曰老小公子惡戲【託為俳諧之言】因掣刀去之【掣昌列翻去羌呂翻】又置愔於棺中載以轜車【轜音而喪車也】又嘗持槊走馬【槊色角翻】以擬左丞相斛律金之胸者三金立不動乃賜帛千段高氏婦女不論親疎多與之亂或以賜左右又多方苦辱之彭城王浟太妃爾朱氏魏敬宗之后也【浟夷周翻】帝欲蒸之不從手刃殺之故魏樂安王元昂李后之姊壻也其妻有色帝數幸之【數所角翻】欲納為昭儀召昂令伏以鳴鏑射之百餘下【射而亦翻】凝血垂將一石竟至於死后啼不食乞讓位於姊太后又以為言帝乃止又嘗於衆中召都督韓哲無罪斬之作大鑊長鋸剉碓之屬陳之於庭【鑊戶郭翻鼎大無足曰鑊】每醉輒手殺人以為戲樂【樂音洛】所殺者多令支解或焚之於火或投之於水楊愔乃簡鄴下死囚置之仗内【殿庭左右立仗】謂之供御囚帝欲殺人輒執以應命三月不殺則宥之開府參軍裴謂之上書極諫帝謂楊愔曰此愚人何敢如是對曰彼欲陛下殺之以成名於後世耳帝曰小人我且不殺爾焉得名【焉音煙】帝與左右飲酒曰樂哉都督王紘曰有大樂亦有大苦【樂音洛】帝曰何謂也對曰長夜之飲不寤國亡身隕所謂大苦帝縛紘欲斬之思其有救世宗之功乃捨之【高澄之死王紘冒刃禦賊見一百六十二卷武帝太清二年】帝遊宴東山以關隴未平投盃震怒召魏收於前立為詔書宣示遠近將事西行魏人震恐常為度隴之計【宇文泰識虚實何得因西行一詔便為度隴之計此齊史官之華言耳】然實未行一日泣謂羣臣曰黑獺不受我命奈何都督劉桃枝曰臣得三千騎【騎奇寄翻】請就長安擒之以來帝壯之賜帛千匹趙道德進曰東西兩國彊弱力均彼可擒之以來此亦可擒之以往桃枝妄言應誅陛下奈何濫賞帝曰道德言是囘絹賜之帝乘馬欲下峻岸入於漳【欲入漳水】道德攬轡回之帝怒將斬之道德曰臣死不恨當於地下啓先帝論此兒酣酗顛狂不可敎訓【酗吁句翻陸德明曰以酒為凶曰酗】帝默然而止它日帝謂道德曰我飲酒過【過謂過多】須痛杖我道德抶之【抶丑栗翻擊也】帝走道德逐之曰何物人為此舉止典御丞李集面諫【五代志曰後齊制官多循後魏之舊尚食尚藥二局皆有典御及丞尚食總知御膳事尚藥總知御藥事屬門下省】比帝於桀紂帝令縛置流中【流水中也】沈没久之【沈持林翻】復令引出【復扶又翻】謂曰吾何如桀紂集曰向來彌不及矣帝又令沈之引出更問如此數四集對如初帝大笑曰天下有如此癡人方知龍逢比干未是俊物【龍逢諫夏桀而死比干諫殷紂而死逢皮江翻】遂釋之頃之又被引入見【被皮義翻見賢遍翻】似有所諫帝令將出要斬【要讀曰腰】其或斬或赦莫能測焉内外憯憯【憯七感翻憯憯痛毒之意】各懷怨毒而素能默識彊記加以嚴斷【斷丁亂翻】羣下戰慄不敢為非又能委政楊愔愔總攝機衡百度修敇【敇理也】故時人皆言主昏於上政清於下愔風表鑒裁為朝野所重少歷屯阨【爾朱屠害楊氏唯愔得脫潜竄累載後歸高歡又以讒間逃隱海島歡訪而用之裁才代翻少詩照翻屯陟倫翻】及得志有一餐之惠者必重報之雖先嘗欲殺己者亦不問典選二十餘年以奬拔賢才為己任性復強記一見皆不忘其姓名選人魯漫漢自言猥賤獨不見識【選須絹翻復扶又翻猥鄙也】愔曰卿前在元子思坊【元子思坊鄴城中坊名魏侍中元子思居此後謀西奔被誅時人因以名坊】乘短尾牝驢見我不下以方麴障面我何為不識卿漫漢驚服 秋七月甲戌前天門太守樊毅襲武陵殺武州刺史衡陽王護王琳使司馬潘忠擊之執毅以歸護暢之孫也【暢武帝之弟】丙子以陳霸先為中書監司徒揚州刺史進爵長城<br />
<br />
  公餘如故 初余孝頃為豫章太守【姓譜余姓由余之後】侯瑱鎭豫章孝頃於新吳縣【漢靈帝中平中立新吳縣屬豫章郡今洪州奉新縣即其地】别立城栅與瑱相拒瑱使其從弟奫守豫章【從才用翻奫於倫翻】悉衆攻孝頃久不克築長圍守之癸酉侯平發兵攻奫大掠豫章焚之奔於建康瑱衆潰奔湓城依其將焦僧度僧度勸之奔齊會霸先使記室濟江蔡景歷南上【濟子禮翻自建康泝流至湓城為南上上時掌翻】說瑱令降瑱乃詣闕歸罪霸先為之誅侯平【說式芮翻降戶江翻為于偽翻】丁亥以瑱為司空南昌民熊曇朗世為郡著姓曇朗有勇力侯景之亂聚衆據豐城為柵【南昌縣帶豫章郡吳立富城縣晉武帝太康元年更名豐城縣屬豫章郡今縣在郡城南一百五十五里曇徒含翻】世祖以為巴山太守【元帝廟號世祖】江陵陷曇朗兵力浸彊侵掠鄰縣侯瑱在豫章曇朗外示服從而隂圖之及瑱敗走曇朗獲其馬仗 己亥齊大赦 魏太師泰遣安州長史鉗耳康買使于王琳【鉗耳夷姓也出於西羌鉗其亷翻孫愐曰鉗耳西羌人自云周王季之後為䖍仁氐音訛為鉗耳】琳遣長史席豁報之且請歸世祖及愍懷太子之柩泰許之【江陵之䧟魏既戕元帝遂殺愍懷太子元良柩音舊】 八月己酉鄱陽王循卒於江夏弟豐城侯泰監郢州事【夏戶雅翻監工銜翻】王琳使兖州刺史吳藏攻江夏不克而死【琳署藏領兖州耳】 魏太師泰北度河【據魏紀泰北廵度北河】 魏以王琳為大將軍長沙郡公 魏江州刺史陸騰討陵州叛獠【五代志隆山郡西魏置陵州今陵井監是也江州亦置於隆山郡之隆山縣獠魯皓翻】獠因山為城攻之難拔騰乃陳伎樂於城下一面【伎渠綺翻】獠棄兵攜妻子臨城觀之騰潜師三面俱上斬首萬五千級遂平之【唐柴紹破吐谷渾亦用此術上時掌翻】騰俟之玄孫也【陸俟事魏太武帝及文成帝之初立以子麗誅宗愛功封王】 庚申齊主將西廵百官辭於紫陌帝使矟騎圍之【騎兵執矟者為矟騎矟色角翻騎奇寄翻】曰我舉鞭即殺之日晏帝醉不能起黄門郎是連子暢曰【是連亦夷姓也魏書官氏志内入諸姓有是連氏】陛下如此羣臣不勝恐怖【勝音升怖普布翻】帝曰大怖邪若然勿殺【若然猶云若如此也】遂如晉陽 九月壬寅改元大赦以陳霸先為丞相錄尚書事鎭衛大將軍揚州牧義興公【自長城縣公進封義興郡公按陳書帝紀義當作吳】以吏部尚書王通為右僕射 突厥木杆可汗假道於凉州以襲吐谷渾魏太師泰使凉州刺史史寧帥騎隨之至番禾【番禾縣漢屬張掖郡魏分置番禾郡如淳曰番音盤隋廢番禾郡為番禾縣屬凉州唐天寶三年改為天寶縣厥九勿翻杆公旦翻可從刋入聲汗音寒吐從暾入聲谷音浴帥讀曰率騎奇寄翻】吐谷渾覺之奔南山木杆將分兵追之寧曰樹敦賀眞二城吐谷渾之巢穴也拔其本根餘衆自散木杆從之木杆從北道趣賀眞寧從南道趣樹敦【樹敦城在曼頭山北吐谷渾之舊都也周穆王時犬戎樹敦居之因以名城祭公謀父所謂犬戎樹惇能帥舊職者也趣七喻翻】吐谷渾可汗在賀眞使其征南王將數千人守樹敦【將即亮翻】木杆破賀眞獲夸呂妻子寧破樹敦虜征南王還與木杆會於青海【吐谷渾中有青海周回千餘里海中有小山每冬冰合以良牝馬置此山至來春牧之牝馬皆有孕生駒號為龍種必多駿異日行千里】木杆歎寧勇决贈遺甚厚【遺干季翻】 甲子王琳以舟師襲江夏冬十月壬申豐城侯泰以州降之 齊山東寡婦二千六百人以配軍有夫而濫奪者什二三魏安定文公宇文泰還至牽屯山而病【北廵而還也杜佑曰牽屯在平凉郡高平縣亦曰汧屯山今謂之笄頭山】驛召中山公護護至涇州見泰泰謂護曰吾諸子皆幼外寇方強天下之事屬之於汝【屬之欲翻下所屬同】宜努力以成吾志乙亥卒於雲陽【年五十雲陽縣漢屬馮翊魏收志屬北地郡後周置雲陽郡有雲陽宫】護還長安發喪泰能駕馭英豪得其力用性好質素不讓虛飾明達政事崇儒好古凡所施設皆依倣三代而為之【好呼到翻】丙子世子覺嗣位為太師柱國大冢宰出鎮同州【宇文泰輔政多居同州以其地扼關河之要齊人或來侵軼便于應接也】時年十五中山公護名位素卑雖為泰所屬而羣公各圖執政莫肯服從護問計於大司寇于謹謹曰謹早蒙先公非常之知恩深骨肉今日之事必以死爭之若對衆定策公必不得讓明日羣公會議謹曰昔帝室傾危非安定公無復今日【謂魏孝武帝為高歡所逼遁逃入關宇文泰迎而輔之以立國于關右復扶又翻】今公一旦違世嗣子雖幼中山公親其兄子兼受顧託軍國之事理須歸之辭色抗厲衆皆悚動【抗厲舉聲高亢且正色嚴厲也】護曰此乃家事護雖庸昧何敢有辭謹素與泰等夷護常拜之至是謹起而言曰公若統理軍國謹等皆有所依遂再拜羣公廹於謹亦再拜於是衆議始定護綱紀内外撫循文武人心遂安 十一月辛丑豐城侯泰奔齊齊以為永州刺史【隋方以零陵郡為永州齊以泰為永州刺史未知此永州置于何地】詔徵王琳為司空【此齊詔也】琳辭不至留其將潘純陀監郢州【監工銜翻】身還長沙魏人歸其妻子壬子齊主詔以魏末豪傑糾合鄉部因緣請託各立州郡離大合小公私煩費丁口減於疇日【疇日猶言昔日】守令倍於昔時且要荒向化【要荒引古要服荒服為言要者要結好信而從服之荒者言其來服荒忽無常也要一遥翻】舊多浮偽百室之邑遽立州名三戶之民空張郡目【此謂梁末所置州郡在江淮之間者也】循名責實事歸焉有【焉於䖍翻何也】於是併省三州一百五十三郡 【考異曰北史作五十六郡今從齊書】 詔分江州四郡置高州【四郡蓋臨川安成豫寧巴山以其地在南江之西負山面水據高臨深因名高州】軍黄法為刺史鎮巴山【巨俱翻宋白曰梁大同二年分廬陵之興平臨川之新建二縣立西寧巴山二縣合其縣立為巴山郡其郡古迹在撫州崇仁縣巴山之北】 十二月壬申以曲江侯勃為太保 甲申魏葬安定文公丁亥以岐陽之地封世子覺為周公【岐陽即扶風之地昔周興於岐周因為國號宇文輔魏倣周以立法制故魏朝之臣以周封之將禪代也】 初侯景之亂臨川民周續起兵郡中【臨川漢豫章郡南城縣之地後漢分南城北境為臨汝縣吳孫亮太平二年分豫章之東部南城臨汝二縣置臨川郡隋唐為撫州】始興王毅以郡讓之而去續部將皆郡中豪族多驕横【將即亮翻下同横戶孟翻】續裁制之諸將皆怨相與殺之續宗人廸勇冠軍中【冠古玩翻】衆推為主廸素寒微恐郡人不服以同郡周敷族望高顯折節交之【折而設翻】敷亦事廸甚謹廸據上塘【上塘下卷作工塘必有一誤按陳書周廸傳工字為是】敷據故郡朝廷以廸為衡州刺史領臨川内史【以臨川内史帶衡州刺史耳】時民遭侯景之亂皆棄農業羣聚為盗唯廸所部獨務農桑各有贏儲政敎嚴明【敎謂敎令州郡下令謂之敎】徵斂必至【歛力贍翻】餘郡乏絶者皆仰以取給【仰牛向翻】廸性質朴不事威儀居常徒跣雖外列兵衛内有女伎挼繩破篾傍若無人【伎渠綺翻挼奴禾翻篾莫結翻竹筠也】訥於言語而襟懷信實臨川人皆附之 齊自西河總秦戌築長城 【考異曰去歲六月已云築長城而地名長短不同不知與此為一事為二事比齊書北史皆然今皆存之】東至於海前後所築東西凡三千餘里率十里一戌其要害置州鎭凡二十五所 魏宇文護以周公幼弱欲早使正位以定人心庚子以魏恭帝詔禪位於周【魏道武帝以晉孝武太元二十一年改元皇始歷十二世至孝武帝永熙三年西遷魏遂分為東西西魏又歷三世凡十五世一百六十年而亡】使大宗伯趙貴持節奉冊濟北公廸致皇帝璽紱恭帝出居大司馬府【濟子禮翻璽斯氏翻紱音弗】<br />
<br />
  資治通鑑卷一百六十六<br />
<br />
<史部,編年類,資治通鑑>  <br>
   </div> 

<script src="/search/ajaxskft.js"> </script>
 <div class="clear"></div>
<br>
<br>
 <!-- a.d-->

 <!--
<div class="info_share">
</div> 
-->
 <!--info_share--></div>   <!-- end info_content-->
  </div> <!-- end l-->

<div class="r">   <!--r-->



<div class="sidebar"  style="margin-bottom:2px;">

 
<div class="sidebar_title">工具类大全</div>
<div class="sidebar_info">
<strong><a href="http://www.guoxuedashi.com/lsditu/" target="_blank">历史地图</a></strong>  
<a href="http://www.880114.com/" target="_blank">英语宝典</a>  
<a href="http://www.guoxuedashi.com/13jing/" target="_blank">十三经检索</a> 
<br><strong><a href="http://www.guoxuedashi.com/gjtsjc/" target="_blank">古今图书集成</a></strong> 
<a href="http://www.guoxuedashi.com/duilian/" target="_blank">对联大全</a> <strong><a href="http://www.guoxuedashi.com/xiangxingzi/" target="_blank">象形文字典</a></strong> 

<br><a href="http://www.guoxuedashi.com/zixing/yanbian/">字形演变</a>  <strong><a href="http://www.guoxuemi.com/hafo/" target="_blank">哈佛燕京中文善本特藏</a></strong>
<br><strong><a href="http://www.guoxuedashi.com/csfz/" target="_blank">丛书&方志检索器</a></strong> <a href="http://www.guoxuedashi.com/yqjyy/" target="_blank">一切经音义</a>  

<br><strong><a href="http://www.guoxuedashi.com/jiapu/" target="_blank">家谱族谱查询</a></strong>  <strong><a href="http://shufa.guoxuedashi.com/sfzitie/" target="_blank">书法字帖欣赏</a></strong> 
<br>

</div>
</div>


<div class="sidebar" style="margin-bottom:0px;">

<font style="font-size:22px;line-height:32px">QQ交流群9:489193090</font>


<div class="sidebar_title">手机APP 扫描或点击</div>
<div class="sidebar_info">
<table>
<tr>
	<td width=160><a href="http://m.guoxuedashi.com/app/" target="_blank"><img src="/img/gxds-sj.png" width="140"  border="0" alt="国学大师手机版"></a></td>
	<td>
<a href="http://www.guoxuedashi.com/download/" target="_blank">app软件下载专区</a><br>
<a href="http://www.guoxuedashi.com/download/gxds.php" target="_blank">《国学大师》下载</a><br>
<a href="http://www.guoxuedashi.com/download/kxzd.php" target="_blank">《汉字宝典》下载</a><br>
<a href="http://www.guoxuedashi.com/download/scqbd.php" target="_blank">《诗词曲宝典》下载</a><br>
<a href="http://www.guoxuedashi.com/SiKuQuanShu/skqs.php" target="_blank">《四库全书》下载</a><br>
</td>
</tr>
</table>

</div>
</div>


<div class="sidebar2">
<center>


</center>
</div>

<div class="sidebar"  style="margin-bottom:2px;">
<div class="sidebar_title">网站使用教程</div>
<div class="sidebar_info">
<a href="http://www.guoxuedashi.com/help/gjsearch.php" target="_blank">如何在国学大师网下载古籍?</a><br>
<a href="http://www.guoxuedashi.com/zidian/bujian/bjjc.php" target="_blank">如何使用部件查字法快速查字?</a><br>
<a href="http://www.guoxuedashi.com/search/sjc.php" target="_blank">如何在指定的书籍中全文检索?</a><br>
<a href="http://www.guoxuedashi.com/search/skjc.php" target="_blank">如何找到一句话在《四库全书》哪一页?</a><br>
</div>
</div>


<div class="sidebar">
<div class="sidebar_title">热门书籍</div>
<div class="sidebar_info">
<a href="/so.php?sokey=%E8%B5%84%E6%B2%BB%E9%80%9A%E9%89%B4&kt=1">资治通鉴</a> <a href="/24shi/"><strong>二十四史</strong></a>&nbsp; <a href="/a2694/">野史</a>&nbsp; <a href="/SiKuQuanShu/"><strong>四库全书</strong></a>&nbsp;<a href="http://www.guoxuedashi.com/SiKuQuanShu/fanti/">繁体</a>
<br><a href="/so.php?sokey=%E7%BA%A2%E6%A5%BC%E6%A2%A6&kt=1">红楼梦</a> <a href="/a/1858x/">三国演义</a> <a href="/a/1038k/">水浒传</a> <a href="/a/1046t/">西游记</a> <a href="/a/1914o/">封神演义</a>
<br>
<a href="http://www.guoxuedashi.com/so.php?sokeygx=%E4%B8%87%E6%9C%89%E6%96%87%E5%BA%93&submit=&kt=1">万有文库</a> <a href="/a/780t/">古文观止</a> <a href="/a/1024l/">文心雕龙</a> <a href="/a/1704n/">全唐诗</a> <a href="/a/1705h/">全宋词</a>
<br><a href="http://www.guoxuedashi.com/so.php?sokeygx=%E7%99%BE%E8%A1%B2%E6%9C%AC%E4%BA%8C%E5%8D%81%E5%9B%9B%E5%8F%B2&submit=&kt=1"><strong>百衲本二十四史</strong></a>  <a href="http://www.guoxuedashi.com/so.php?sokeygx=%E5%8F%A4%E4%BB%8A%E5%9B%BE%E4%B9%A6%E9%9B%86%E6%88%90&submit=&kt=1"><strong>古今图书集成</strong></a>
<br>

<a href="http://www.guoxuedashi.com/so.php?sokeygx=%E4%B8%9B%E4%B9%A6%E9%9B%86%E6%88%90&submit=&kt=1">丛书集成</a> 
<a href="http://www.guoxuedashi.com/so.php?sokeygx=%E5%9B%9B%E9%83%A8%E4%B8%9B%E5%88%8A&submit=&kt=1"><strong>四部丛刊</strong></a>  
<a href="http://www.guoxuedashi.com/so.php?sokeygx=%E8%AF%B4%E6%96%87%E8%A7%A3%E5%AD%97&submit=&kt=1">說文解字</a> <a href="http://www.guoxuedashi.com/so.php?sokeygx=%E5%85%A8%E4%B8%8A%E5%8F%A4&submit=&kt=1">三国六朝文</a>
<br><a href="http://www.guoxuedashi.com/so.php?sokeytm=%E6%97%A5%E6%9C%AC%E5%86%85%E9%98%81%E6%96%87%E5%BA%93&submit=&kt=1"><strong>日本内阁文库</strong></a> <a href="http://www.guoxuedashi.com/so.php?sokeytm=%E5%9B%BD%E5%9B%BE%E6%96%B9%E5%BF%97%E5%90%88%E9%9B%86&ka=100&submit=">国图方志合集</a> <a href="http://www.guoxuedashi.com/so.php?sokeytm=%E5%90%84%E5%9C%B0%E6%96%B9%E5%BF%97&submit=&kt=1"><strong>各地方志</strong></a>

</div>
</div>


<div class="sidebar2">
<center>

</center>
</div>
<div class="sidebar greenbar">
<div class="sidebar_title green">四库全书</div>
<div class="sidebar_info">

《四库全书》是中国古代最大的丛书,编撰于乾隆年间,由纪昀等360多位高官、学者编撰,3800多人抄写,费时十三年编成。丛书分经、史、子、集四部,故名四库。共有3500多种书,7.9万卷,3.6万册,约8亿字,基本上囊括了古代所有图书,故称“全书”。<a href="http://www.guoxuedashi.com/SiKuQuanShu/">详细>>
</a>

</div> 
</div>

</div>  <!--end r-->

</div>
<!-- 内容区END --> 

<!-- 页脚开始 -->
<div class="shh">

</div>

<div class="w1180" style="margin-top:8px;">
<center><script src="http://www.guoxuedashi.com/img/plus.php?id=3"></script></center>
</div>
<div class="w1180 foot">
<a href="/b/thanks.php">特别致谢</a> | <a href="javascript:window.external.AddFavorite(document.location.href,document.title);">收藏本站</a> | <a href="#">欢迎投稿</a> | <a href="http://www.guoxuedashi.com/forum/">意见建议</a> | <a href="http://www.guoxuemi.com/">国学迷</a> | <a href="http://www.shuowen.net/">说文网</a><script language="javascript" type="text/javascript" src="https://js.users.51.la/17753172.js"></script><br />
  Copyright &copy; 国学大师 古典图书集成 All Rights Reserved.<br>
  
  <span style="font-size:14px">免责声明:本站非营利性站点,以方便网友为主,仅供学习研究。<br>内容由热心网友提供和网上收集,不保留版权。若侵犯了您的权益,来信即刪。scp168@qq.com</span>
  <br />
ICP证:<a href="http://www.beian.miit.gov.cn/" target="_blank">鲁ICP备19060063号</a></div>
<!-- 页脚END --> 
<script src="http://www.guoxuedashi.com/img/plus.php?id=22"></script>
<script src="http://www.guoxuedashi.com/img/tongji.js"></script>

</body>
</html>
