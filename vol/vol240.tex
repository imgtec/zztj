資治通鑑卷二百四十
宋 司馬光 撰

胡三省 音註

唐紀五十六|{
	起彊圉作噩盡屠維大淵獻正月凡二年有奇}


憲宗昭文章武大聖至神孝皇帝中之下

元和十二年春正月甲申貶袁滋為撫州刺史李愬至唐州軍中承喪敗之餘|{
	喪息浪翻嚴綬慈丘之敗山南東道未分為二帥也既分為二帥而高霞寓敗於鐵城袁滋代之又敗}
士卒皆憚戰愬知之有出迓者愬謂之曰天子知愬柔懦能忍恥故使來拊循爾曹至於戰攻進取非吾事也衆信而安之愬親行視士卒|{
	行下孟翻}
傷病者存恤之不事威嚴或以軍政不肅為言愬曰吾非不知也袁尚書專以恩惠懷賊賊易之|{
	易弋豉翻輕易也}
聞吾至必增備吾故示之以不肅彼必以吾為懦而懈惰然後可圖也|{
	懈古隘翻}
淮西人自以嘗敗高袁二帥|{
	敗補邁翻帥所類翻}
輕愬名位素微遂不為備|{
	為愬乘虚取蔡張本 考異曰舊傳曰愬沈勇長筭推誠待士故能用其卑弱之勢出賊不意居半歲知人可用乃謀襲蔡表請濟師詔以河中鄜坊騎兵二千人益之鄭澥平蔡録曰正月二十四日甲申公至所部先是士卒經萬勝蕭陂鐵城之敗人心皆惴恐不敢言戰公佯曰戰爭非吾所能既而隂召大將計其事是時公以表請徑襲元濟人皆笑其說乃使觀察判官王擬請師闕下詔徵義成河中鄜坊馬步共二千以補其闕据此則是始至便請益兵又二月即擒丁士良降吳秀珠是不待半歲然後知人可用舊傳恐誤然愬密謀襲蔡豈可先洩之而云以表請襲元濟人皆笑其說則是人人知之恐非也今不取}
遣鹽鐵副使程异督財賦於江淮 回鶻屢請尚公主有司計其費近五百萬緡|{
	近其靳翻}
時中原方用兵故上未之許二月辛卯朔遣回鶻摩尼僧等歸國|{
	摩尼來見二百三十七卷元年史照曰元和初回鶻再朝獻始以摩尼至摩尼至京師歲往來西市商賈頗與囊槖為姦至是遣歸國也}
命宗正少卿李誠使回鶻諭意以緩其期 李愬謀襲蔡州表請益兵詔以昭義河中鄜坊步騎二千給之丁酉愬遣十將馬少良將十餘騎廵邏|{
	十將軍中小校也邏郎佐翻}
遇吳元濟捉生虞候丁士良與戰擒之士良元濟驍將常為東邊患|{
	言唐鄧之東邊也}
衆請刳其心愬許之既而召詰之士良無懼色愬曰真丈夫也命釋其縛士良乃自言本非淮西士貞元中隸安州與吳氏戰為其所擒自分死矣|{
	分扶問翻}
吳氏釋我而用之我因吳氏而再生故為吳氏父子竭力|{
	為于偽翻}
昨日力屈復為公所擒|{
	復扶又翻}
亦分死矣今公又生之請盡死以報德愬乃給其衣服器械署為捉生將 己亥淮西行營奏克蔡州古葛伯城|{
	漢書陳留寧陵縣孟康注曰古葛伯國今葛鄉是此必韓弘奏捷也}
丁士良言於李愬曰吳秀琳擁三千之衆據文城柵|{
	文城柵在蔡州西南一百二十里按續通典柵在吳房縣界}
為賊左臂官軍不敢近者|{
	近其靳翻}
有陳光洽為之主謀也光洽勇而輕|{
	輕牽正翻}
好自出戰請為公先擒光洽|{
	好呼到翻為于偽翻下同}
則秀琳自降矣|{
	降戶江翻}
戊申士良擒光洽以歸 鄂岳觀察使李道古引兵出穆陵關|{
	黄州麻城縣西北有穆陵關在穆陵山上}
甲寅攻申州克其外郭進攻子城城中守將夜出兵擊之道古之衆驚亂死者甚衆道古臯之子也|{
	曹成王臯歷江西山南等鎮著功名}
淮西被兵數年|{
	被皮義翻}
竭倉廩以奉戰士民多無食采菱芡魚鼈鳥獸食之亦盡|{
	芡巨險翻今謂之雞頭}
相帥歸官軍者前後五千餘戶|{
	帥讀曰率}
賊亦患其耗糧食不復禁|{
	復扶又翻}
庚申敕置行縣以處之|{
	未能得其縣故權置行縣以處來歸之民處昌呂翻}
為擇縣令使之撫養并置兵以衛之三月乙丑李愬自唐州徙屯宜陽柵 郗士美敗於

栢鄉抜營而歸士卒死者千餘人 戊辰賜程執恭名權 戊寅王承宗遣兵二萬入東光斷白橋路|{
	東光縣屬景州宋白曰東光漢舊縣也故城在縣東二十里齊天保七年移於今縣東南三十里陶氏故城隋開皇三年又移于後魏廢勃海舊城縣西四里有永濟渠渠上有橋當自縣通弓高之路白橋跨永濟渠在德州長河縣斷音短}
程權不能禦以衆歸滄州|{
	渾鎬既敗郗士美又敗程權又退歸王承宗之才非諸帥所能制也}
吳秀琳以文城柵降于李愬戊子愬引兵至文城西五里遣唐州刺史李進誠將甲士八千至城下召秀琳城中矢石如雨衆不得前進誠還報賊偽降未可信也愬曰此待我至耳即前至城下秀琳束兵投身馬足下愬撫其背慰勞之|{
	勞力到翻}
降其衆三千人秀琳將李憲有材勇愬更其名曰忠義而用之|{
	更工衡翻}
悉遷婦女於唐州|{
	質其家於唐州則文城之士心不敢懷反側}
於是唐鄧軍氣復振人有欲戰之志賊中降者相繼於道隨其所便而置之聞有父母者給粟帛遣之曰汝曹皆王人勿弃親戚衆皆感泣|{
	自此已上李愬事}
官軍與淮西兵夾溵水而軍諸軍相顧望無敢度溵水者陳許兵馬使王沛先引兵五千度溵水據要地為城於是河陽宣武河東魏博等軍相繼皆度進逼郾城丁亥李光顔敗淮西兵三萬於郾城|{
	按宋白續通典郾城在蔡州西平縣北五十里敗補邁翻}
走其將張伯良殺士卒什二三|{
	自此以上攻郾城事}
己丑李愬遣山河十將董少玢等分兵攻諸柵其日少玢下馬鞍山拔路口柵|{
	時都畿及唐鄧皆募土人之材勇者為兵以討蔡號為山河子弟置十將以領之玢府巾翻按唐蔡交兵凡境上要地處處置守所謂馬鞍山路口柵固不可盡詳其處而強為之注也}
夏四月辛卯山河十將馬少良下嵖岈山|{
	嵖鋤加翻岈虚加翻}
擒淮西將柳子野|{
	此以上又李愬事}
吳元濟以蔡人董昌齡為郾城令質其母楊氏|{
	質音致}
楊氏謂昌齡曰順死賢於逆生|{
	順死謂歸順而死逆生謂從逆而生}
汝去逆而吾死乃孝子也從逆而吾生是戮吾也會官軍圍青陵絶郾城歸路|{
	青陵在郾城西南}
郾城守將鄧懷金謀於昌齡昌齡勸之歸國懷金乃請降於李光顔曰城人之父母妻子皆在蔡州請公來攻城吾舉烽求救救兵至公逆擊之蔡兵必敗然後吾降則父母妻子庶免矣光顔從之乙未昌齡懷金舉城降光顔引兵入據之吳元濟聞郾城不守甚懼時董重質將騾軍守洄曲|{
	據新書李光顔傳洄曲即時曲蓋溵水於此回曲因以為名}
元濟悉發親近及守城卒詣重質以拒之|{
	此以上又李光顔事}
李愬山河十將媯雅田智榮下冶爐城|{
	媯居為翻姓也九域志曰蔡州冶爐城韓國鑄劍之地時當在西平界按新書冶爐城在嵖岈山東}
丙申十將閻士榮下白狗汶港二柵|{
	白狗汶港二柵皆在蔡州真陽縣界蕭梁置西淮州於真陽白狗堆後齊廢州為齊興郡尋廢郡為白狗縣隋開皇初改縣曰懷川大業初省入真陽隋志真陽有汶水}
癸卯媯雅田智榮破西平|{
	西平春秋栢國漢為西平縣屬汝南郡唐屬蔡州九域志在州西一百五里}
丙午遊奕兵馬使王義破楚城|{
	楚城在汝陽縣西南蕭梁置西楚州及汝陽郡於此}
五月辛酉李愬遣柳子野李忠義襲朗山擒其守將梁希果 六鎮討王承宗者|{
	事見上卷十一年}
兵十餘萬回環數千里既無統帥又相去遠期約難壹由是歷二年無功千里饋運牛驢死者什四五劉總既得武彊引兵出境纔五里|{
	出境謂出武彊之境}
留屯不進月給度支錢十五萬緡李逢吉及朝士多言宜併力先取淮西俟淮西平乘其勝勢回取恒冀如拾芥耳上猶豫久乃從之|{
	李逢吉等之言即韋貫之等之言也然憲宗有用不用者前此兵勢未屈今則兵勢已屈不得不從也}
丙子罷河北行營各使還鎮 丁丑李愬遣方城鎮遏使李榮宗擊青喜城拔之|{
	方城縣本漢堵陽縣地後漢改為順陽隋改為方城縣唐屬唐州九域志在州北一百六十里縣有青臺鎮此作青喜筆誤也}
愬每得降卒必親引問委曲由是賊中險易遠近虚實盡知之|{
	易弋豉翻平易也}
愬厚待吳秀琳與之謀取蔡秀琳曰公欲取蔡非李祐不可秀琳無能為也祐者淮西騎將有勇畧守興橋柵|{
	興橋柵在張柴村東}
常陵暴官軍|{
	陵者加之以氣暴者虐之以威}
庚辰祐率士卒刈麥於張柴村|{
	張柴村在文城柵東六十里帥讀曰率}
愬召廂虞候史用誠|{
	廂虞候掌左右廂之兵}
戒之曰爾以三百騎伏彼林中又使人揺幟於前|{
	幟昌志翻}
若將焚其麥積者祐素易官軍|{
	易弋豉翻輕之也}
必輕騎來逐之爾乃發騎掩之必擒之用誠如言而往生擒祐以歸將士以祐曏日多殺官軍爭請殺之愬不許釋縛待以客禮時愬欲襲蔡而更密其謀獨召祐及李忠義屏人語|{
	屏必郢翻又卑正翻}
或至夜分|{
	夜半為夜分}
它人莫得預聞諸將恐祐為變多諫愬愬待祐益厚士卒亦不悦諸軍日有牒稱祐為賊内應且言得賊諜者具言其事|{
	此行營諸軍移文之言諜徒協翻}
愬恐謗先逹於上已不及救乃持祐泣曰豈天不欲平此賊邪何吾二人相知之深而不能勝衆口也因謂衆曰諸君既以祐為疑請令歸死於天子|{
	歸死猶言致尸也左傳魏絳曰請歸死於司寇杜預注云致尸於司寇使戮之}
乃械祐送京師先密表其狀|{
	密表言與祐謀襲蔡之狀}
且曰若殺祐則無以成功詔釋之以還愬愬見之喜執其手曰爾之得全社稷之靈也|{
	李愬之期待祐者如此祐安得不力}
乃署散兵馬使|{
	散員兵馬使未得統兵散悉但翻}
令佩刀廵警出入帳中或與之同宿密語不寐逹曙有竊聽於帳外者但聞祐感泣聲時唐隨牙隊三千人|{
	牙隊者節度使牙衛從之隊猶今之簇帳部}
號六院兵馬皆山南東道之精鋭也|{
	時山南東道分為兩鎮八州精鋭盡抽選赴唐州使之攻戰}
愬又以祐為六院兵馬使舊軍令舍賊諜者屠其家|{
	舊軍令先時之軍令也舍者停藏之於家也}
愬除其令使厚待之諜反以情告愬愬益知賊中虛實乙酉愬遣兵攻朗山淮西兵救之官軍不利衆皆悵恨愬獨歡然曰此吾計也|{
	賊恃勝而不備愬則愬得以成入蔡之功其計出此}
乃募敢死士三千人號曰突將|{
	將即亮翻}
朝夕自教習之使常為行備欲以襲蔡會久雨所在積水未果 閏月己亥程异還自江淮得供軍錢百八十五萬緡|{
	是年春程异督財賦於江淮}
諫議大夫韋綬兼太子侍讀每以珍膳餉太子又悦太子以諧謔|{
	綬音受謔香畧翻}
上聞之丁未罷綬侍讀|{
	觀憲宗之罷韋綬亦知所謂諭教者矣然觀穆宗之臨政也習與性成得非所急者固在於選左右歟}
尋出為䖍州刺史|{
	舊志䖍州京師東南四千一十七里}
綬京兆人|{
	史著綬京兆人以言其生長京邑習見淫侈非能以德義經術誘掖東宫古言沃土之民不才良有以也}
吳元濟見其下數叛|{
	數所角翻}
兵勢日蹙六月壬戌上表謝罪願束身自歸上遣中使賜詔許以不死而為左右及大將董重質所制不得出|{
	史言董重質之情}
秋七月大水或平地二丈 初國子祭酒孔戣為華州刺史|{
	戣巨龜翻華戶化翻}
明州歲貢蚶蛤淡菜|{
	蚶呼甘翻魁陸也横從其理五味自充殻如瓦壠者謂之瓦壠蚶蛤葛合翻蛤小於蚶蚶殻厚其理如瓦壠蛤殻薄其文如貝呂令云雀入大水化為蛤說文云百歲燕所化又云老伏翼所化皆非也蚶蛤皆生於海瀕潮汐往來舄鹵之地淡菜狀如䗒而小黑殻唇有鬚如茸肉甘脆䗒蒲幸翻}
水陸遞夫勞費戣奏疏罷之|{
	華州京畿輔郡自東南來者水陸遞夫咸經焉故得言其勞費而罷之}
甲辰嶺南節度使崔詠薨宰相奏擬代詠者數人上皆不用曰頃有諫進蚶蛤淡菜者為誰可求其人與之庚戌以戣為嶺南節度使 諸軍討淮蔡四年不克|{
	九年冬始討淮西}
饋運疲弊民至有以驢耕者|{
	牛斃於運轉民至無以耕}
上亦病之以問宰相李逢吉等競言師老財竭意欲罷兵裴度獨無言上問之對曰臣請自往督戰乙卯上復謂度曰|{
	復扶又翻}
卿真能為朕行乎|{
	為于偽翻}
對曰臣誓不與此賊俱生臣比觀吳元濟表|{
	比毗至翻}
勢實窘蹙但諸將心不壹不併力廹之故未降耳若臣自詣行營諸將恐臣奪其功必爭進破賊矣上悦丙戌以度為門下侍郎同平章事兼彰義節度使仍充淮西宣慰招討處置使|{
	觀裴度不附羣議請身督戰則韓愈平淮西碑推功於度有以也處昌呂翻}
又以戶部侍郎崔羣為中書侍郎同平章事制下度以韓弘已為都統不欲更為詔討請但稱宣慰處置使仍奏刑部侍郎馬摠為宣慰副使右庶子韓愈為彰義行軍司馬判官書記皆朝廷之選上皆從之度將行言於上曰臣若賊滅則朝天有期賊在則歸闕無日上為之流涕|{
	為于偽翻下為卿同}
八月庚申度赴淮西上御通化門送之|{
	通化門長安城東面北來第一門}
右神武將軍張茂和茂昭弟也嘗以膽畧自衒於度|{
	衒熒絹翻}
度表為都押牙茂和辭以疾度奏請斬之上曰此忠順之門|{
	茂和父孝忠兄茂昭鎮易定比河朔諸鎮為忠順}
為卿遠貶辛酉貶茂和永州司馬以嘉王傅高承簡為都押牙|{
	高承簡為嘉王傅蓋嘉王運之子嗣為嘉王故置府官}
承簡崇文之子也李逢吉不欲討蔡翰林學士令狐楚與逢吉善度恐其合中外之勢以沮軍事|{
	翰林學士居禁中宰相在外朝恐其中外相應以上罷兵之議沮在呂翻}
乃請改制書數字且言其草制失辭壬戍罷楚為中書舍人 李光顔烏重胤與淮西戰癸亥敗於賈店 裴度過襄城南白草原淮西人以驍騎七百邀之鎮將楚丘曹華知而為備擊却之|{
	楚丘古巳氏縣隋開皇六年改曰楚丘唐屬宋州九域志在州東北七十里}
度雖辭招討名實行元帥事以郾城為治所甲申至郾城先是諸道皆有中使監陳|{
	監古銜翻陳讀曰陣}
進退不由主將勝則先使獻捷不利則陵挫百端度悉奏去之諸將始得專軍事戰多有功|{
	去羌呂翻}
九月庚子淮西兵寇溵水鎮殺三將焚芻藁而去初上為廣陵王布衣張宿以辯口得幸及即位累官

至比部員外郎|{
	唐比部郎屬刑部掌句諸司百僚俸料公廨贓贖調歛徒役課程逋懸數物以周知内外之經費而總句之比音毗}
宿招權受賂於外門下侍郎同平章事李逢吉惡之|{
	惡烏路翻}
上欲以宿為諫議大夫逢吉曰諫議重任必能可否朝政始宜為之|{
	朝直遥翻}
宿小人豈得竊賢者之位必欲用宿請去臣乃可上由是不悦逢吉又與裴度異議上方倚度以平蔡丁未罷逢吉為東川節度使 甲寅李愬將攻吳房|{
	吳房漢縣屬汝南郡孟康曰本房子國楚靈王遷房於楚吳王闔廬弟夫概奔楚楚封之于此為棠谿氏故曰吳房今吳房城棠谿亭是唐吳房縣屬蔡州平蔡後改為遂平縣}
諸將曰今日往亡|{
	隂陽家之說八月以白露後十八日為往亡九月以寒露後第二十七日為往亡}
愬曰吾兵少不足戰宜出其不意彼以往亡不吾虞正可擊也遂往克其外城斬首千餘級餘衆保子城不敢出愬引兵還以誘之|{
	還音旋又如字}
淮西將孫獻忠果以驍騎五百追擊其背衆驚將走愬下馬據胡牀|{
	胡牀今謂之交牀其制本自虜來隋以䜟有胡改曰交牀唐猶謂之胡牀}
令曰敢退者斬返斾力戰獻忠死 |{
	考異曰舊傳作孫忠憲今從平蔡録}
淮西兵乃退或勸愬乘勝攻其子城可拔也愬曰非吾計也|{
	定計入蔡不在取吳房}
引兵還營 李祐言於李愬曰蔡之精兵皆在洄曲|{
	考異曰舊元濟傳李祐曰元濟勁軍多在時曲按李光顔傳曰董重質弃洄曲軍李愬傳云分五百人斷洄曲路又云洄曲子弟歸求寒衣然則元濟傳誤當為洄曲余意洄曲蓋即時曲也}
及四境拒守|{
	守音狩}
守州城者皆羸老之卒可以乘虛直抵其城比賊將聞之|{
	比必利翻及也}
元濟已成擒矣愬然之冬十月甲子遣掌書記鄭澥至郾城密白裴度度曰兵非出奇不勝常侍良圖也|{
	澥胡買翻李愬檢校左散騎常侍鎮唐鄧隨故裴度稱之}
上竟用張宿為諫議大夫崔羣王涯固諫不聽乃請以為權知諫議大夫許之宿由是怨執政及端方之士與皇甫鎛相表裏譖去之|{
	去羌呂翻}
裴度帥僚佐觀築城於沱口|{
	九域志郾城縣有沱口鎮沱徒河翻}
董重質帥騎出五溝邀之|{
	五溝在洄曲之北帥讀曰率}
大呼而進|{
	呼火故翻}
注弩挺刃|{
	挺抜也}
勢將及度李光顔與田布力戰拒之度僅得入城賊退布扼其溝中歸路賊下馬踰溝墜壓死者千餘人辛未李愬命馬步都虞候隨州刺史史旻留鎮文城命李祐李忠義帥突將三千為前驅自與監軍將三千人為中軍命李進誠將三千人殿其後|{
	殿丁練翻}
軍出不知所之愬曰但東行行六十里夜至張柴村盡殺其戍卒及烽子|{
	唐凡烽候之所有烽帥烽副烽子蓋守烽之卒候望警急而舉烽者也杜佑曰一烽六人五人為烽子遞知更刻觀視動静一人烽率知文書符辭轉牒}
據其柵命士少休|{
	少詩沼翻}
食乾糒整羈靮|{
	糒音備乾飯也羈馬絡頭靮紖也音丁歷翻}
留義成軍五百人鎮之以斷洄曲及諸道橋梁|{
	斷音短}
復夜引兵出門|{
	復扶又翻}
諸將請所之愬曰入蔡州取吳元濟諸將皆失色監軍哭曰果落李祐姦計時大風雪旌旗裂人馬凍死者相望天隂黑自張柴村以東道路皆官軍所未嘗行人人自以為必死然畏愬莫敢違夜半雪愈甚行七十里至州城|{
	至蔡州城下也}
近城有鵝鴨池愬令擊之以混軍聲自吳少誠拒命官軍不至蔡州城下三十餘年|{
	德宗貞元二年吳少誠據蔡州至是三十二年}
故蔡人不為備壬申四鼓愬至城下無一人知者李祐李忠義钁其城為坎以先登|{
	钁居縳翻鋤也}
壯士從之守門卒方熟寐盡殺之而留擊柝者使擊柝如故遂開門納衆及裏城亦然城中皆不之覺雞鳴雪止愬入居元濟外宅|{
	節度使外宅也}
或告元濟曰官軍至矣元濟尚寢笑曰俘囚為盗耳曉當盡戮之又有告者曰城陷矣元濟曰此必洄曲子弟就吾求寒衣也起聽於廷聞愬軍號令曰常侍傳語應者近萬人|{
	近其靳翻}
元濟始懼曰何等常侍能至於此乃帥左右登牙城拒戰|{
	帥讀曰率}
時董重質擁精兵萬餘人據洄曲愬曰元濟所望者重質之救耳乃訪重質家厚撫之遣其子傳道持書諭重質重質遂單騎詣愬降愬遣李進誠攻牙城毁其外門得甲庫取器械癸酉復攻之燒其南門民爭負薪芻助之城上矢如蝟毛|{
	愬軍聚射矢集城上如蝟毛言其多也}
晡時門壞元濟於城上請罪進誠梯而下之甲戌愬以檻車送元濟詣京師|{
	德宗貞元二年吳少誠得蔡州三世三十二年而滅 考異曰舊愬傳曰其月七日使判官鄭澥告期於裴度十日夜以李祐率突將三千為先鋒愬自率中軍三千田進誠以後軍三千殿而行元濟傳曰十一月愬夜出軍令李祐為前鋒其十日夜至蔡州城下實録曰愬以十月將襲蔡州先七日使判官鄭澥告師期於裴度按先七日即是平蔡錄所云八日甲子也而愬傳誤云七日而又云十日夜師軍行亦誤元濟傳十一月愬出軍尤誤裴度傳十月十一日李愬襲破懸瓠城擒元濟亦誤按十月戊午朔韓愈平淮西碑云壬申愬用所得賊將自文城因天大雪疾馳百二十里即十五日也又曰用夜半到蔡破其門取元濟以獻即十六日也實錄己卯執元濟乃奏到日也今從平蔡錄}
且告於裴度是日申光二州及諸鎮兵二萬餘人相繼來降自元濟就擒愬不戮一人凡元濟官吏帳下厨廐之卒皆復其職使之不疑|{
	推赤心置人腹中}
然後屯於鞠場以待裴度|{
	鞠場毬場也}
以淮南節度使李鄘為門下侍郎同平章事 己卯淮西行營奏獲吳元濟光祿少卿楊元卿言於上曰淮西大有珍寶臣能知之往取必得|{
	元和九年楊元卿以淮西節度判官入奏輸誠於朝廷吳元濟屠其家今請將命往取淮西珍寶其情可知也}
上曰朕討淮西為人除害|{
	為于偽翻}
珍寶非所求也董重質之去洄曲軍也李光顔馳入其壁悉降其衆庚辰裴度遣馬摠先入蔡州慰撫辛巳度建彰義軍節將降卒萬餘人入城李愬具櫜鞬出迎拜於路左|{
	櫜姑勞翻鞬居言翻櫜以藏弓鞬以藏箭鄭玄曰道左道東也余按古者乘車尚左故迎拜於車下者皆拜於道左蓋自北而來者以道東為左自南而來者以道西為左自東西而來者亦隨車之所嚮而分左右也鄭玄舉一隅耳故孔穎逹正義曰凡言左右據南鄉西鄉為正蓋南鄉君道也西鄉主道也}
度將避之愬曰蔡人頑悖不識上下之分數十年矣|{
	悖蒲妹翻又蒲沒翻分扶問翻}
願公因而示之使知朝廷之尊度乃受之|{
	史言李愬識度非當時諸帥所及}
李愬還軍文城|{
	裴度既入蔡李愬還軍文城此皆是識體統處又非諸帥怙功欲專地為私利者比也}
諸將請曰始公敗於朗山而不憂勝於吳房而不取|{
	事並見上}
冒大風甚雪而不止孤軍深入而不懼然卒以成功|{
	卒子恤翻}
皆衆人所不諭也敢問其故愬曰朗山不利則賊輕我而不為備矣取吳房則其衆奔蔡併力固守故存之以分其兵風雪隂晦則烽火不接不知吾至孤軍深入則人皆致死戰自倍矣夫視遠者不顧近慮大者不詳細若矜小勝恤小敗先自撓矣何暇立功乎衆皆服|{
	余按李愬入蔡誠為奇功史家稱述其與諸將楊榷用兵方畧所以取勝之由遣文命意實祖史漢韓信戰井陘事所書者然愬平蔡之事猶可以發揚若唐末王式平裘甫事則又祖李家述平蔡之功者也若其所敵之堅脆所規之廣狹固不可以欺衒識者文之過實者多學者其於是察之橈奴教翻}
愬儉於奉已而豐於待士知賢不疑見可能斷|{
	斷丁亂翻}
此其所以成功也裴度以蔡卒為牙兵或諫曰蔡人反仄者尚多不可不備度笑曰吾為彰義節度使元惡既擒蔡人則吾人也又何疑焉蔡人聞之感泣|{
	裴度平蔡蔡人不復叛矣識者知其所以然乎}
先是吳氏父子阻兵|{
	吳氏父子謂少陽元濟也先悉薦翻}
禁人偶語於塗夜不然燭有以酒食相過從者罪死|{
	盗亦有道此其以法束下所以自防也過工禾翻}
度既視事下令惟禁盗賊餘皆不問往來者不限晝夜蔡人始知有生民之樂|{
	解人之束縛使得舒展四體長欠大伸豈不快哉}
甲申詔韓弘裴度條列平蔡將士功狀及蔡之將士降者皆差第以聞|{
	史炤曰謂將士有功者等差而次第之余謂當時詔旨既令弘度差第平蔡將士之功狀而蔡之將士歸降者有降於元濟未就擒之前者有降于元濟既就擒之後者有先嘗拒殺官軍勢窮力屈而降者有先通誠欵欲降而未能自致者亦令弘度差第其狀以聞史炤之說舉其一而遺其一者也}
淮西州縣百姓給復二年|{
	復方目翻除其賦役二年以優新附之民}
近賊四州免來年夏税|{
	近賊四州陳許潁唐也頻遭蔡人攻剽又供億官軍故免來年夏税亦以優之}
官軍戰亡者皆為收葬|{
	為于偽翻}
給其家衣糧五年其因戰傷殘廢者勿停衣糧|{
	死者葬其尸又贍其家殘廢者養之終身殘廢謂因戰傷折腰膂手足不復為完人堪世用者}
十一月上御興安門受俘|{
	大明宫南面五門興安門最在其西}
遂以吳元濟獻廟社斬于獨柳之下初淮西之人刼於李希烈吳少誠之威虐不能自拔久而老者衰幼者壯安於悖逆不復知有朝廷矣|{
	悖蒲内翻又蒲沒翻}
自少誠以來遣諸將出兵皆不束以法制聽各以便宜自戰故人人得盡其才韓全義之敗於溵水也|{
	事見二百三十五卷德宗貞元十六年}
於其帳中得朝貴所與問訊書少誠束以示衆曰此皆公卿屬全義書|{
	屬之欲翻託也}
云破蔡州日乞一將士妻女為婢妾由是衆皆憤怒以死為賊用雖居中土其風俗獷戾|{
	考之漢志汝南戶口為百郡之最古人謂汝頴多奇士至唐而獷戾乃爾習俗之移人也}
過於夷貊|{
	嗚呼吾恐後之視今亦猶今之視昔獷古猛翻悍也貊莫百翻}
故以三州之衆舉天下之兵環而攻之|{
	環音宦}
四年然後克之官軍之攻元濟也李師道募人通使於蔡察其形勢牙前虞候劉晏平應募出汴宋間潛行至蔡元濟大喜厚禮而遣之晏平還至鄆師道屏人而問之|{
	還音旋屏必郢翻又卑正翻}
晏平曰元濟暴兵數萬於外阽危如此|{
	阽余亷翻臨危也}
而日與僕妾遊戲博奕於内|{
	奕當作弈弈棋也}
晏然曾無憂色以愚觀之殆必亡不久矣師道素倚淮西為援聞之驚怒尋誣以他過杖殺之|{
	以劉晏平之善覘其智識必有過人者李師道不能委心歸計以求自安之術乃怒而殺之終亦必亡而已矣}
戊子以李愬為山南東道節度使賜爵凉國公加韓弘兼侍中李光顔烏重胤等各遷官有差 舊制御史一人知驛|{
	開元中令監察御史兼廵傳驛至二十五年以監察御史檢校兩京館驛大歷十四年兩京以御史一人知館驛號館驛使}
壬辰詔以宦者為館驛使左補闕裴潾諫曰|{
	潾力珍翻}
内臣外事職分各殊|{
	分扶問翻}
切在塞侵官之源|{
	塞悉則翻}
絶出位之漸事有不便必戒於初令或有妨不必在大上不聽 甲午恩王連薨|{
	連代宗子}
辛丑以唐隨兵馬使李祐為神武將軍知軍事|{
	會要乾元四年十月四日勑左右羽林左右龍武左右神武軍文武官並昇同金吾四衛唐制諸衛將軍大將軍上將軍類加以名號而不掌兵知軍事則掌兵矣唐隨謂當作唐鄧隨}
裴度以馬摠為彰義留後癸丑發蔡州上封二劍以授梁守謙使誅吳元濟舊將度至郾城遇之復與俱入蔡州量罪施刑|{
	量音良}
不盡如詔旨仍上疏言之 十二月壬戌賜裴度爵晉國公復入知政事以馬摠為淮西節度使 初吐突承璀方貴寵用事為淮南監軍李鄘為節度使性剛嚴與承璀互相敬憚故未嘗相失承璀歸|{
	吐突承璀六年出為淮南監軍九年召還}
引鄘為相|{
	是年十月相李鄘}
鄘恥由宦官進及將佐出祖|{
	出城祖道謂餞之也}
樂作鄘泣下曰吾老安外鎮宰相非吾任也戊寅鄘至京師辭疾不入見|{
	見賢遍翻}
不視事百官到門皆辭不見|{
	史言李鄘知恥}
庚辰貶淮西降將董重質為春州司戶重質為元濟謀主屢破官軍上欲殺之李愬奏先許重質以不死

十三年春正月乙酉朔赦天下 初李師道謀逆命判官高沐與同僚郭昈李公度屢諫之|{
	昈侯古翻 考異曰新傳又有郭航名按航乃牙將昈所使詣李愿者非幕僚同諫者也今從河南記}
判官李文會孔目官林英素為師道所親信涕泣言於師道曰文會等盡心為尚書憂家事|{
	心為于偽翻}
反為高沐等所疾尚書奈何不愛十二州之土地|{
	十二州鄆兖曹濮淄青齊海登萊沂密也}
以成沐等之功名乎師道由是疎沐等出沐知萊州|{
	萊州古萊子之國後魏置光州隋改萊州}
會林英入奏事令進奏吏密申師道云沐潛輸欵於朝廷文會從而構之師道殺沐并囚郭昈凡軍中勸師道效順者文會皆指為高沐之黨而囚之及淮西平師道憂懼不知所為李公度及牙將李英曇|{
	曇徒含翻}
因其懼而說之使納質獻地以自贖|{
	說式芮翻質音致}
師道從之遣使奉表請使長子入侍并獻沂密海三州上許之乙巳遣左常侍李遜詣鄆州宣慰 上命六軍修麟德殿右龍武統軍張奉國大將軍李文悦|{
	大將軍即右龍武大將軍}
以外寇初平|{
	謂淮西初平}
營繕太多白宰相冀有論諫裴度因奏事言之上怒二月丁卯以奉國為鴻臚卿壬申以文悦為右武衛大將軍|{
	既出奉國於外朝文悦又自北門諸衛遷南牙諸衛臚陵如翻}
充威遠營使|{
	威遠營亦非北軍也}
於是浚龍首池起承暉殿土木浸興矣|{
	大明宫東面有東内苑苑中有龍首殿龍首池龍首渠水自城南而注入於此池宋白曰龍首殿在右軍}
李愬奏請判官大將以下官凡百五十員上不悦謂裴度曰李愬誠有奇功然奏請過多使如李晟渾瑊又何如哉遂留中不下|{
	下戶嫁翻}
李鄘固辭相位戊戌以鄘為戶部尚書以御史大夫李夷簡為門下侍郎同平章事初渤海僖王言義卒弟簡王明忠立改元太始一歲卒從父仁秀立改元建興乙巳遣使來告喪 横海節度使程權自以世襲滄景|{
	德宗始命程日華為横海帥傳子懷直為從兄懷信所逐懷信死子權嗣為帥}
與河朔三鎮無殊内不自安己酉遣使上表請舉族入朝許之横海將士樂自擅|{
	樂音洛}
不聽權去掌書記林藴諭以禍福權乃得出詔以藴為禮部員外郎裴度之在淮西也布衣栢耆以策干韓愈曰吳元濟既就擒王承宗破膽矣願得奉丞相書往說之|{
	說式芮翻}
可不煩兵而服愈白度為書遣之承宗懼求哀於田弘正請以二子為質|{
	質音致}
及獻德棣二州輸租税請官吏弘正為之奏請|{
	為于偽翻}
上初不許弘正上表相繼|{
	上表時掌翻}
上重違弘正意乃許之夏四月甲寅朔魏博遣使送承宗子知感知信及德棣二州圖印至京師幽州大將譚忠說劉總曰|{
	說式芮翻}
自元和以來劉闢李錡田季安盧從史吳元濟阻兵馮險|{
	馮讀曰憑}
自以為深根固蔕|{
	蔕丁計翻}
天下莫能危也然顧盼之間身死家覆皆不自知此非人力所能及殆天誅也况今天子神聖威武苦身焦思|{
	思相吏翻}
縮衣節食|{
	縮歛也短也}
以養戰士此志豈須臾忘天下哉今國兵駸駸北來|{
	國兵謂王師也駸駸馬行疾貌}
趙人已獻城十二|{
	德州領安德長河平原平昌將陵安陵六縣棣州領厭次滳河陽信蒲臺渤海五縣程權之退承宗又取景州之東光今皆以歸朝廷故曰獻城十二}
忠深為公憂之|{
	為于偽翻}
總泣且拜曰聞先生言吾心定矣遂專意歸朝廷 戊辰内出廢印二紐賜左右三軍辟仗使|{
	龍武神武羽林三軍各分左右辟讀如闢}
舊制以宦官為六軍辟仗使如方鎮之監軍無印|{
	監軍有印見二百三十五卷德宗貞元十一年宋白曰舊制内官為三軍辟仗使監視刑賞奏察違謬猶方鎮之監軍使}
及張奉國得罪至是始賜印得糾繩軍政事任專逹矣 庚戌詔洗雪王承宗及成德將士復其官爵|{
	削王承宗官爵見上卷十一年}
李師道暗弱軍府大事獨與妻魏氏奴胡惟堪楊自温婢蒲氏袁氏及孔目官王再升謀之大將及幕僚莫得預焉魏氏不欲其子入質|{
	質音致}
與蒲氏袁氏言於師道曰自先司徒以來有此十二州|{
	李正已初據有十五州及李納拒命徐州入于朝廷德棣入于朱滔有十二州而已先司徒謂李納也}
奈何無故割而獻之今計境内之兵不下數十萬不獻三州不過以兵相加|{
	三州謂請獻沂密海}
若力戰不勝獻之未晩師道乃大悔欲殺李公度幕僚賈直言謂其用事奴曰今大禍將至豈非高沐寃氣所為若又殺公度軍府其危哉乃囚之遷李英曇於萊州未至縊殺之李遜至鄆州師道大陳兵迎之遜盛氣正色為陳禍福|{
	為于偽翻}
責其决語|{
	决語决為一定之說不依違持兩端}
欲白天子師道退與其黨謀之皆曰弟許之|{
	弟與第同}
它日止煩一表解紛耳師道乃謝曰曏以父子之私且廹於將士之情故遷延未遣今重煩朝使豈敢復有二三|{
	重直用翻朝直遥翻使疏吏翻復扶又翻朝使謂朝廷所遣使者}
遜察師道非實誠歸言於上曰師道頑愚反覆恐必須用兵既而師道表言軍情不聽納質割地上怒决意討之賈直言冒刃諫師道者二輿櫬諫者一又畫縛載檻車妻子係纍者以獻師道怒囚之|{
	史炤曰孟子曰係纍其子弟趙氏注云係纍縛結也}
五月丙申以忠武節度使李光顔為義成節度使|{
	李光顔自許州徙鎮滑州}
謀討師道也以淮西節度使馬摠為忠武節度使陳許溵蔡州觀察使以申州隸鄂岳光州隸淮南|{
	不復以蔡州為節鎮}
辛丑以知勃海國務大仁秀為勃海王 以河陽都知兵馬使曹華為棣州刺史詔以河陽兵送至滳河|{
	滳河漢千乘濕沃縣地隋開皇十六年置滴河縣廢濕沷入焉唐屬棣州九域志在州西南八十里漢都尉許商鑿此通海故以商河為名後人加水焉宋白曰縣南有滳河因以為名}
會縣為平盧兵所陷|{
	平盧兵李師道之兵也}
華擊却之殺二千餘人復其縣以聞詔加横海節度副使 六月癸丑朔日有食之 丁丑復以烏重胤領懷州刺史鎮河陽|{
	淮西已平故烏重胤自汝州復還鎮河陽}
秋七月癸未朔徙李愬為武寧節度使乙酉下制罪狀李師道令宣武魏博義成武寧横海兵共討之以宣歙觀察使王遂為供軍使遂方慶之孫也|{
	王方慶武后聖歷中為相歙書涉翻}
上方委裴度以用兵門下侍郎同平章事李夷簡自謂才不及度求出鎮辛丑以夷簡同平章事充淮南節度使 八月壬子朔中書侍郎同平章事王涯罷為兵部侍郎 吳元濟既平韓弘懼九月自將兵擊李師道圍曹州 淮西既平上浸驕侈戶部侍郎判度支皇甫鎛衛尉卿鹽鐵轉運程异曉其意數進羨餘以供其費|{
	史言鎛异逢君之惡數所角翻羨弋線翻}
由是有寵鎛又以厚賂結吐突承璀甲辰鎛以本官异以工部侍郎並同平章事判使如故|{
	皇甫鎛以戶部侍郎相判度支如故程异進貳起部以相鹽鐵轉運使如故}
制下朝野駭愕至於市井負販者亦嗤之|{
	下戶稼翻嗤丑之翻笑也}
裴度崔羣極陳其不可上不聽度恥與小人同列表求自退不許度復上疏|{
	復扶又翻上時掌翻}
以為鎛异皆錢穀吏佞巧小人陛下一旦寘之相位中外無不駭笑况鎛在度支專以豐取刻與為務凡中外仰給度支之人無不思食其肉|{
	仰牛向翻}
比者裁損淮西糧料|{
	比毗至翻近也謂討吳元濟時裁損淮西行營諸軍糧料}
軍士怨怒會臣至行營曉諭慰勉僅無潰亂今舊將舊兵悉向淄青|{
	謂舊所遣討蔡之將討蔡之兵悉遣之討李師道}
聞鎛入相必盡驚憂知無可訴之地矣|{
	言鎛在度支減刻糧賜軍士猶可訴之於廟堂今既為相無可訴之地矣}
程异雖人品庸下然心事和平可處煩劇不宜為相|{
	處昌呂翻下同}
至如鎛資性狡詐天下共知唯能上惑聖聰足見姦邪之極|{
	言憲宗英明且為所惑可以見其極姦邪}
臣若不退天下謂臣不知亷恥臣若不言天下謂臣有負恩寵今退既不許言又不聽臣如烈火燒心衆鏑叢體所可惜者淮西盪定河北底寧承宗歛手削地|{
	謂獻德棣二州}
韓弘輿疾討賊|{
	謂自將討李師道}
豈朝廷之力能制其命哉直以處置得宜能服其心耳陛下建升平之業十已八九何忍還自墮壞|{
	墮讀曰隳壞音怪}
使四方解體乎上以度為朋黨不之省|{
	省悉景翻}
鎛自知不為衆所與益為巧謟以自固奏減内外官俸以助國用給事中崔植封還敕書極論之乃止植祐甫之弟子也|{
	崔祐甫相德宗有可稱者}
時内出積年繒帛付度支令賣鎛悉以高價買之以給邊軍其繒帛朽敗隨手破裂邊軍聚而焚之|{
	繒慈陵翻}
度因奏事言之鎛於上前引其足曰此靴亦内庫所出臣以錢二千買之堅完可久服度言不可信上以為然|{
	引足於君前不敬大矣憲宗溺於利不惟不察其慢又且然其言}
由是鎛益無所憚|{
	為鎛得罪張本}
程异亦自知不合衆心能亷謹謙遜為相月餘不敢知印秉筆|{
	時宰相更日知印秉筆}
故終免於禍 五坊使楊朝汶|{
	汶音問}
妄捕繫人廹以考捶責其息錢遂轉相誣引所繫近千人|{
	捶止蕊翻近其靳翻}
中丞蕭俛劾奏其狀|{
	俛音免劾漢書音義戶概翻今音戶得翻}
裴度崔羣亦以為言上曰姑與卿論用兵事|{
	姑且也}
此小事朕自處之|{
	處昌呂翻}
度曰用兵事小所憂不過山東耳五坊使暴横恐亂輦轂|{
	横戶孟翻史照曰轂者輻所凑也京都四方所輻凑以輦轂取喻余按漢書京兆尹率自言待罪輦轂下謂京兆在天子輦轂之下耳}
上不悦退召朝汶責之曰以汝故令吾羞見宰相冬十月賜朝汶死盡釋繫者上晩節好神仙|{
	好呼到翻}
詔天下求方士宗正卿李道古

先為鄂岳觀察使以貪暴聞恐終獲罪思所以自媚於上乃因皇甫鎛薦山人柳泌云能合長生藥甲戌詔泌居興唐觀煉藥|{
	合音閣唐會要興唐觀本司農園地在長樂坊開元十八年造李道古薦柳泌以求媚免罪不知適足以重罪也泌既誅而道古亦貶矣為上服泌藥致疾張本泌薄必翻又兵媚翻}
十一月辛巳朔鹽州奏吐蕃寇河曲夏州靈武奏破吐蕃長樂州克其外城|{
	吐蕃長樂州當在靈州黄河外定遠城之西夏戶雅翻樂音洛}
柳泌言於上曰天台山神仙所聚|{
	新志台州唐興縣有天台山宋朝改唐興縣為天台縣天台山在縣西一百一十里臨海記天台山超然秀出山有八重視之如一高一萬八千丈周回八百里}
多靈草臣雖知之力不能致誠得為彼長吏庶幾可求上信之|{
	長知丈翻幾居希朝}
丁亥以泌權知台州刺史|{
	台州漢回浦縣地會稽東部都尉理所光武改回浦為章安縣吳分章安置臨海縣唐武德四年置海州五年改台州因天台山為名}
仍賜服金紫諫官爭論奏以為人主喜方士|{
	喜許記翻}
未有使之臨民賦政者|{
	賦布也}
上曰煩一州之力而能為人主致長生|{
	為于偽翻}
臣子亦何愛焉由是羣臣莫敢言甲午鹽州奏吐蕃遁去 壬寅以河陽節度使烏重

胤為横海節度使丁未以華州刺史令狐楚為河陽節度使重胤以河陽精兵三千赴鎮河陽兵不樂去鄉里|{
	樂音洛}
中道潰歸又不敢入城屯于城北將大掠令狐楚適至單騎出慰撫之與俱歸先是田弘正請自黎陽渡河會義成節度使李光顔討李師道|{
	先悉薦翻}
裴度曰魏博軍既渡河不可復退|{
	復扶又翻}
立須進擊方有成功既至滑州即仰給度支|{
	義成節度使治滑州魏博與滑州以河為界兵至滑州為已出界唐中世以來命藩鎮兵征討已出境芻糧皆仰給於度支惟裴度用兵於東平李德裕用兵於上黨知其弊有以制之}
徒有供餉之勞更生觀望之勢又或與李光顔互相疑阻益致遷延|{
	一棲不兩雄又有賓主之形疑阻或生何事不有其患豈止於遷延之役}
與其渡河而不進不若養威於河北宜且使之秣馬厲兵俟霜降水落自楊劉渡河|{
	楊劉鎮在鄆州東北東阿縣臨河津}
直指鄆州得至陽穀置營|{
	隋置陽穀縣以陽穀臺為名唐屬鄆州九域志在州西一百三十里宋白曰陽穀縣本漢須昌縣地今縣界有須昌故城}
則兵勢自盛賊衆揺心矣|{
	上文言得至恐兵有利鈍也此言賊衆揺心指其成效也}
上從之是月弘正將全師自楊劉渡河距鄆州四十里築壘|{
	此自楊劉直進不復迂其路至陽穀也舊史李師道傳曰距鄆州九十里田弘正傳曰四十里 考異曰河南記云營於陽穀西北今從實錄}
賊中大震 功德使上言鳳翔法門寺塔有佛指骨|{
	法門寺在鳳翔府岐山縣時功德使言法門寺有護國真身塔塔内有釋迦牟尼佛指骨一節}
相傳三十年一開開則歲豐人安來年應開請迎之十二月庚戌朔上遣中使帥僧衆迎之|{
	帥讀曰率}
戊辰以春州司戶董重質為試太子詹事委武寧軍驅使李愬請之也|{
	時徙李愬鎮武寧以討李師道}
戊寅魏博義成軍送所獲李師道都知兵馬使夏侯澄等四十七人上皆釋弗誅各付所獲行營驅使曰若有父母欲歸者優給遣之朕所誅者師道而已於是賊中聞之降者相繼|{
	降戶江翻}
初李文會與兄元規皆在李師古幕下師古薨師道立|{
	薨立見二百三十七卷元年}
元規辭去文會屬師道親黨請留|{
	屬之欲翻}
元規將行謂文會曰我去身退而安全汝留必驟貴而受禍及官軍四臨平盧兵勢日蹙將士喧然皆曰高沐郭昈李存為司空忠謀|{
	為于偽翻下不為同師道檢校司空故稱之}
李文會姧佞殺沐囚昈存以致此禍師道不得已出文會攝登州刺史召昈存還幕府 上常語宰相|{
	語牛倨翻}
人臣當力為善何乃好立朋黨朕甚惡之|{
	好呼到翻惡烏路翻}
裴度對曰方以類聚物以羣分|{
	易大傳之言}
君子小人志趣同者勢必相合君子為徒謂之同德小人為徒謂之朋黨外雖相似内實懸殊在聖主辨其所為邪正耳武寧節度使李愬與平盧兵十一戰皆捷乙卯晦進攻金鄉克之|{
	金鄉縣唐屬兖州宋白曰金鄉縣本漢東緍縣今縣理即古緍國城陳留風俗傳云東緍者故陽武戶牖鄉後漢於任城縣西南七十五里置金鄉縣因穿山得金故曰金鄉}
李師道性懦怯自官軍致討聞小敗及失城邑輒憂悸成疾|{
	悸其季翻}
由是左右皆蔽匿不以實告金鄉兖州之要地也既失之其刺史驛騎告急左右不為通|{
	為于偽翻}
師道至死竟不知也

十四年春正月辛巳韓弘拔考城殺二千餘人|{
	考城漢古縣唐屬曹州九城志在汴州東一百八十里}
丙戌師道所署沭陽令梁洞以縣降於楚州刺史李聽|{
	沭陽漢廩丘縣後魏曰沭陽以其地在沭水之陽也唐屬海州九域志在州西南一百八十里沭食聿翻}
吐蕃遣使者論短立藏等來修好未返|{
	好呼到翻}
入寇河曲上曰其國失信其使何罪庚寅遣歸國 壬辰武寧節度使李愬拔魚臺|{
	魚臺漢方與縣地唐屬兖州寶應元年改為魚臺小城北有魯公觀魚臺而名之觀魚臺即春秋魯隱公如棠觀魚之地元和四年李師道請移縣於黄臺市}
中使迎佛骨至京師上留禁中三日乃歷送諸寺王公士民瞻奉捨施惟恐弗及有竭產充施者|{
	施式智翻}
有然香臂頂供養者|{
	供居用翻養余亮翻}
刑部侍郎韓愈上表切諫以為佛者夷狄之一法耳自黄帝以至禹湯文武皆享壽考百姓安樂|{
	樂音洛}
當是時未有佛也漢明帝時始有佛法|{
	見四十五卷永平八年}
其後亂亡相繼運祚不長宋齊梁陳元魏已下事佛漸謹年代尤促惟梁武帝在位四十八年前後三捨身為寺家奴竟為侯景所逼餓死臺城國亦尋滅事佛求福乃更得禍|{
	事並見前紀}
由此觀之佛不足信亦可知矣百姓愚冥易惑難曉苟見陛下如此皆云天子猶一心敬信百姓微賤於佛豈可更惜身命佛本夷狄之人口不言先王之法言身不服先王之法服不知君臣之義父子之恩假如其身尚在奉國命來朝京師陛下容而接之不過宣政一見|{
	唐時四夷入朝貢者皆引見於宣政殿見賢遍翻}
禮賓一設|{
	唐有禮賓院凡胡客人朝設宴於此元和九年置禮賓院於長興里之北宋白曰屬鴻臚寺}
賜衣一襲衛而出之於境不令惑衆也况其身死已久枯朽之骨豈宜以入宫禁古之諸侯行弔於國尚先以桃茢祓除不祥|{
	記曰君臨臣喪以巫祝桃茢執戈惡之也注云為有凶邪之氣在側桃鬼所惡也茢萑苕可掃除不祥左傳魯襄公如楚楚康王卒楚人使公親襚公患之叔孫穆子曰祓殯而襚則布幣也乃使巫以桃茢先祓殯韓愈正引此事茢音列又音例祓敷勿翻又音廢}
今無故取朽穢之物親視之巫祝不先|{
	先悉薦翻}
桃茢不用羣臣不言其非御史不舉其罪臣實恥之乞以此骨付有司投諸水火永絶根本斷天下之疑|{
	斷丁亂翻一音短}
絶後代之惑使天下之人知大聖人之所作為出於尋常萬萬也豈不盛哉佛如有靈能作禍祟凡有殃咎宜加臣身上得表大怒出示宰相將加愈極刑|{
	殊死謂之極刑}
裴度崔羣為言愈雖狂於忠懇|{
	為于偽翻懇誠也}
宜寛容以開言路癸巳貶愈為潮州刺史自戰國之世老莊與儒者爭衡更相是非|{
	更工衡翻}
至漢末益之以佛然好者尚寡|{
	好呼到翻}
晉宋以來日益繁熾自帝王至於士民莫不尊信下者畏慕罪福高者論難空有|{
	難乃旦翻釋氏之說談空以難有}
獨愈惡其蠧財惑衆力排之|{
	惡烏路翻}
其言多矯激太過惟送文暢師序最得其要曰夫鳥俛而啄仰而四顧獸深居而簡出懼物之為已害也猶且不免焉弱之肉彊之食今吾與文暢安居而暇食優游以生死與禽獸異者寧可不知其所自邪|{
	原其所自則聖人之所以垂世立教者也}
丙申田弘正奏敗淄青兵於東阿|{
	敗蒲邁翻東阿漢古縣唐屬鄆州九域志在州西北六十里}
殺萬餘人 滄州刺史李宗奭與横海節度使鄭權不叶|{
	程權既入朝以鄭權代鎮横海}
不受其節制權奏之上遣中使追之宗奭使其軍中留已|{
	此謂滄州本州之軍也}
表稱懼亂未敢離州|{
	離力智翻}
詔以烏重胤代權將吏懼逐宗奭|{
	懼重胤討其黨惡}
宗奭奔京師辛丑斬於獨柳之下 丙午田弘正奏敗平盧兵於陽穀

資治通鑑卷二百四十
