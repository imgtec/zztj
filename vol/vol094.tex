\chapter{資治通鑑卷九十四}
宋 司馬光 撰

胡三省 音註

晉紀十六|{
	起著雍困敦盡重光單閼凡四年}


顯宗成皇帝上之下

咸和三年春正月温嶠入救建康軍于尋陽|{
	自武昌東下軍于尋陽}
韓晃襲司馬流於慈湖流素懦怯將戰食炙不知口處兵敗而死|{
	炙之夜翻燔肉也}
丁未蘇峻帥祖渙許柳等衆二萬人濟自横江登牛渚軍于陵口|{
	牛渚山在今太平州當塗縣北三十里山下有磯津渡之處與和州横江渡相對陵口當在牛渚山東北即東陵口也帥讀曰率}
臺兵禦之屢敗二月庚戌峻至蔣陵覆舟山|{
	陵阜也蔣陵蔣山之阜也覆舟山形如覆舟故名}
陶囘謂庾亮曰峻知石頭有重戍不敢直下必向小丹陽南道步來|{
	漢丹陽郡治宛陵縣武帝太康二年分丹陽置宣城郡治宛陵而丹陽移治建業建業本漢之秣陵也吳改曰建業晉復曰秣陵至太康三年分秣陵之水北為建業後避愍帝諱改曰建康元帝南渡建康置丹陽尹治于臺城西而丹陽太守舊治秣陵縣俗謂之小丹陽其路即今太平州取建康之路也}
宜伏兵邀之可一戰擒也亮不從峻果自小丹陽來迷失道夜行無復部分|{
	分扶問翻}
亮聞乃悔之朝士以京邑危逼|{
	朝直遥翻}
多遣家人入東避難|{
	建康以吳會稽為東難乃旦翻}
左衛將軍劉超獨遷妻孥入居宮内|{
	孥音奴子也}
詔以卞壼都督大桁東諸軍事|{
	壼苦本翻桁讀與航同}
與侍中鍾雅帥郭默趙胤等軍及峻戰于西陵|{
	據壼傳峻至東陵口壼與戰于陵西成帝紀作西陵}
壼等大敗死傷以千數丙辰峻攻青溪柵卞壼率諸軍拒擊不能禁峻因風縱火燒臺省及諸營寺署一時蕩盡|{
	杜佑曰宋齊有三臺五省之號三臺盖兩漢舊名五省謂尚書中書門下祕書集書省也}
壼背癰新愈創猶未合|{
	創初良翻}
力疾帥左右苦戰而死二子眕盱隨父後亦赴敵而死其母撫尸哭曰父為忠臣子為孝子夫何恨乎|{
	眕之忍翻盱凶于翻夫音扶}
丹陽尹羊曼勒兵守雲龍門與黄門侍郎周導廬江太守陶瞻皆戰死庾亮帥衆將陳于宣陽門内|{
	帥讀曰率陳讀曰陣}
未及成列士衆皆弃甲走亮與弟懌條翼及郭默趙胤俱奔尋陽|{
	依温嶠也}
將行顧謂鍾雅曰後事深以相委雅曰棟折榱崩誰之咎也|{
	折而設翻榱所追翻秦曰屋椽齊魯曰桷周曰榱}
亮曰今日之事不容復言|{
	復扶又翻}
亮乘小船亂兵相剝掠亮左右射賊誤中柂工應弦而倒|{
	柂待可翻柂以正船柂工一船之司命也射而亦翻中竹仲翻}
船上咸失色欲散亮不動徐曰此手何可使著賊|{
	言射不能殺賊而反射殺柂工自恨之辭也著直畧翻}
衆乃安峻兵入臺城司徒導謂侍中禇翜曰至尊當御正殿君可啓令速出翜即入上閤躬自抱帝登太極前殿|{
	翜所甲翻}
導及光禄大夫陸曄荀崧尚書張闓共登御床擁衛帝|{
	闓苦亥翻又音開}
以劉超為右衛將軍|{
	晉志文帝初置中衛及衛將軍武帝受命分為左右衛以羊琇為左趙序為右}
使與鍾雅禇翜侍立左右太常孔愉朝服守宗廟|{
	朝直遥翻}
時百官奔散殿省蕭然峻兵既入叱禇翜令下翜正色不動呵之曰蘇冠軍來覲至尊|{
	峻先以討沈充功進冠軍將軍故稱之冠古玩翻}
軍人豈得侵逼由是峻兵不敢上殿|{
	上時掌翻}
突入後宮宮人及太后左右侍人皆見掠奪峻兵驅役百官光祿勲王彬等皆被捶撻|{
	捶止橤翻}
令負擔登蔣山|{
	擔都藍翻又徒濫翻蔣山即鍾山在今上元縣東北十八里輿地志曰古曰金陵山縣名因此又名蔣山漢末秣陵尉蔣子文討賊戰死于此吳太帝為立廟子文祖諱鍾因改曰蔣山予謂孫權祖亦諱鍾當因是改也}
裸剝士女|{
	祼魯果翻}
皆以壞席苫草自障無草者坐地以土自覆哀號之聲震動内外|{
	苫詩亷翻覆敷救翻下同號戶刀翻}
初姑孰既陷尚書左丞孔坦謂人曰觀峻之埶必破臺城自非戰士不須戎服及臺城陷戎服者多死白衣者無他時官有布二十萬匹金銀五千斤錢億萬絹數萬匹他物稱是|{
	言他物與布金銀錢絹相稱也稱尺證翻}
峻盡費之太官惟有燒餘米數石以供御膳或謂鍾雅曰君性亮直必不容於寇讐盍早為之計雅曰國亂不能匡君危不能濟各遁逃以求免何以為臣丁巳峻稱詔大赦惟庾亮兄弟不在原例|{
	不在見赦之例}
以王導有德望猶使以本官居已之右祖約為侍中太尉尚書令峻自為驃騎將軍録尚書事|{
	驃匹妙翻}
許柳為丹陽尹馬雄為左衛將軍祖渙為驍騎將軍|{
	驍堅堯翻}
弋陽王羕詣峻稱述峻功峻復以羕為西陽王太宰録尚書事|{
	羕降爵見上卷咸和元年羕余亮翻}
峻遣兵攻吳國内史庾氷氷不能禦弃郡奔會稽|{
	時以吳郡為吳國太守為内史會工外翻}
至浙江峻購之甚急吳鈴下卒引氷入船以蘧蒢覆之吟嘯鼓枻沂流而去|{
	蘧求於翻蒢陳如翻說文曰蘧蒢竹席也予謂從艸者今蘆䕠也枻以制翻楫謂之枻泝蘇故翻逆流曰泝}
每逢邏所|{
	邏所謂津要置邏卒之所邏郎佐翻}
輒以杖叩船曰何處覓庾氷庾氷正在此人以為醉不疑之氷僅免峻以侍中蔡謨為吳國内史温嶠聞建康不守號慟|{
	號戶刀翻}
人有候之者悲哭相對庾亮至尋陽宣太后詔以嶠為驃騎將軍開府儀同三司又加徐州刺史郗鑒司空|{
	郗丑之翻}
嶠曰今日當以滅賊為急未有功而先拜官將何以示天下遂不受嶠素重亮亮雖奔敗嶠愈推奉之分兵給亮 後趙大赦改元太和 |{
	考異曰晉春秋云勒即帝位改元太和按勒建平元年始即帝位今從勒載記}
三月丙子庾太后以憂崩 蘇峻南屯于湖 夏四月後趙將石堪攻宛南陽太守王國降之|{
	宛於元翻降戶江翻}
遂進攻祖約軍于淮上約將陳光起兵攻約約左右閻秃貌類約|{
	秃吐谷翻}
光謂為約而擒之約踰垣獲免光奔後趙 壬申葬明穆皇后于武平陵 庾亮温嶠將起兵討蘇峻而道路斷絶不知建康聲聞|{
	聞音問}
會南陽范汪至尋陽言峻政令不壹貪暴縱横|{
	横戶孟翻}
滅亡已兆雖彊易弱|{
	易以䜴反}
朝廷有倒懸之急宜時進討嶠深納之亮辟汪參護軍事亮嶠互相推為盟主嶠從弟充曰|{
	從才用翻 考異曰晉春秋作從兄今從晉書嶠傳}
陶征西位重兵彊|{
	侃時為征西大將軍都督荆湘雍梁專制上流}
宜共推之嶠乃遣督護王愆期詣荆州邀陶侃與之同赴國難|{
	難乃旦翻下同}
侃猶以不豫顧命為恨|{
	事見上卷咸和元年}
答曰吾疆場外將不敢越局|{
	謂内輔外禦各有局分不敢踰越也將即亮翻}
嶠屢說不能囘|{
	說輸芮翻}
乃順侃意遣使謂之曰仁公且守|{
	漢魏以來率呼宰輔岳牧為明公今嶠呼侃為仁公盖取天下歸仁之義言晉之征鎮皆歸重於侃也使疏吏翻下同}
僕當先下使者去已二日平南參軍滎陽毛寶|{
	嶠為平南將軍以寶為參軍}
别使還聞之|{
	還從宣翻又如字}
說嶠曰凡舉大事當與天下共之師克在和不宜異同|{
	左傳楚鬭亷曰師克在和不在衆也}
假令可疑猶當外示不覺况自為攜貳邪宜急追信改書|{
	信即使也}
言必應俱進若不及前信當更遣使嶠意悟即追使者改書侃果許之遣督護龔登帥兵詣嶠|{
	帥讀曰率}
嶠有衆七千於是列上尚書|{
	以侃為盟主與亮嶠列名上之尚書也上時掌翻}
陳祖約蘇峻罪狀移告征鎮灑泣登舟陶侃復追龔登還嶠遺侃書曰大軍有進而無退可增而不可減近已移檄遠近言于盟府|{
	盟府謂侃府也侃為盟主故稱為盟府復扶又翻遺于季翻}
刻後月半大舉諸郡軍並在路次惟須仁公軍至便齊進耳仁公今召軍還疑惑遠近成敗之由將在於此僕才輕任重實憑仁公篤愛遠禀成規至于首啓戎行|{
	行戶剛翻詩元戎十乘以先啓行}
不敢有辭僕與仁公如首尾相衛脣齒相依也恐或者不達高旨將謂仁公緩於討賊此聲難追僕與仁公並受方岳之任安危休戚理既同之且自頃之顧綢繆往來情深義重|{
	綢除留翻繆莫彪翻纒綿也}
一旦有急亦望仁公悉衆見救况社稷之難乎今日之憂豈惟僕一州文武莫不翹企|{
	言翹首企足以望侃兵之來難乃旦翻}
假令此州不守約峻樹置官長於此|{
	此謂江州也長知兩翻}
荆楚西逼彊胡東接逆賊因之以飢饉將來之危乃當甚於此州之今日也仁公進當為大晉之忠臣參桓文之功當以慈父之情雪愛子之痛|{
	謂侃子瞻為峻所殺}
今約峻凶逆無道痛感天地人心齊壹咸皆切齒今之進討若以石投卵耳苟復召兵還是為敗於幾成也|{
	復扶又翻幾居希翻}
願深察所陳王愆期謂侃曰蘇峻豺狼也如得遂志四海雖廣公寧有容足之地乎侃深感悟即戎服登舟瞻喪至不臨|{
	臨力鴆翻}
晝夜兼道而進郗鑒在廣陵城孤糧少逼近胡寇|{
	近其靳翻}
人無固志得詔書即流涕誓衆入赴國難將士争奮|{
	難乃旦翻將即亮翻}
遣將軍夏侯長等間行謂温嶠曰或聞賊欲挾天子東入會稽當先立營壘屯據要害既防其越逸又斷賊糧運|{
	間古莧翻斷丁管翻}
然後清野堅壁以待賊賊攻城不拔野無所掠東道既斷糧運自絶必自潰矣嶠深以為然|{
	晉都建康粮運皆仰三吳故欲先斷東道王敦蘇峻之亂匡復之謀郗鍳為多}
五月陶侃率衆至尋陽議者咸謂侃欲誅庾亮以謝天下亮甚懼用温嶠計詣侃拜謝侃驚止之曰庾元規乃拜陶士行邪|{
	陶侃字士行}
亮引咎自責風止可觀侃不覺釋然曰君侯修石頭以擬老子|{
	見上卷咸和元年}
今日反見求邪即與之談宴終日遂與亮嶠同趣建康|{
	趣七喻翻}
戎卒四萬旌旗七百餘里鉦鼓之聲震於遠近蘇峻聞西方兵起用參軍賈寧計自姑孰還據石頭分兵以拒侃等乙未峻逼遷帝於石頭司徒導固争不從帝哀泣升車宮中慟哭時天大雨道路泥濘|{
	濘乃定翻淖也}
劉超鍾雅步侍左右峻給馬不肯乘而悲哀忼慨峻聞而惡之然未敢殺也|{
	惡烏路翻}
以其親信許方等補司馬督殿中監外託宿衛内實防禦超等峻以倉屋為帝宮日來帝前肆醜言劉超鍾雅與右光禄大夫荀崧金紫光禄大夫華恒|{
	左右光禄大夫金章紫綬光禄大夫銀章青綬加金章紫綬者謂之金紫光禄大夫華戶化翻恒戶登翻}
尚書荀邃侍中丁潭侍從不離帝側|{
	從才用翻離力智翻}
時飢饉米貴峻問遺超一無所受繾綣朝夕|{
	遺于季翻繾詰戰翻又去演翻綣區願翻繾綣反覆不相離也孔穎達曰繾綣牢固相著之意左傳曰繾綣從公毋通内外}
臣節愈恭雖居幽厄之中超猶啓帝授孝經論語峻使左光禄大夫陸曄守留臺逼迫居民盡聚之後苑使匡術守苑城尚書左丞孔坦奔陶侃侃以為長史初蘇峻遣尚書張闓權督東軍司徒導密令以太后詔諭三吳吏士|{
	漢置吳郡吳分吳郡置吳興郡晉又分吳興丹陽置義興郡是為三吳酈道元曰世謂吳郡吳興會稽為三吳杜佑曰晉宋之間以吳郡吳興丹陽為三吳}
使起義兵救天子會稽内史王舒以庾氷行奮武將軍使將兵一萬西渡浙江|{
	將即亮翻下同}
于是吳興太守虞潭吳國内史蔡謨前義興太守顧衆等皆舉兵應之潭母孫氏謂潭曰汝當捨生取義勿以吾老為累|{
	累力瑞翻}
盡遣其家僮從軍鬻其環珮以為軍資謨以庾氷當還舊任即去郡以讓氷蘇峻聞東方兵起遣其將管商張健弘徽等拒之虞潭等與戰互有勝負未能得前陶侃温嶠軍于茄子浦嶠以南兵習水蘇峻兵便步|{
	南兵謂温嶠之兵便步謂便於步戰}
令將士有上岸者死|{
	上時掌翻}
會峻送米萬斛饋祖約約遣司馬桓撫等迎之毛寶帥千人為嶠前鋒|{
	帥讀曰率下同}
告其衆曰兵法軍令有所不從豈可視賊可擊不上岸擊之邪乃擅往襲撫悉獲其米斬獲萬計約由是飢乏嶠表寶為廬江太守陶侃表王舒監浙東軍事虞潭監浙西軍事|{
	監工銜翻}
郗鑒都督揚州八郡諸軍事令舒潭皆受鑒節度鑒帥衆渡江與侃等會于茄子浦|{
	類篇茄求加翻菜名子可食茄葉似高蓼葉而青子熟於夏秋之間大如秤錘有紫色者有白色者及其熟也色正黄盖其地宜茄子人多於此樹藝因以名浦}
雍州刺史魏該亦以兵會之|{
	雍於用翻}
丙辰侃等舟師直指石頭至于蔡洲侃屯查浦|{
	蔡洲在石頭西岸查浦在大江南岸直秦淮口}
嶠屯沙門浦峻登烽火樓望見士衆之盛有懼色謂左右曰吾本知温嶠能得衆也庾亮遣督護王彰擊峻黨張曜反為所敗亮送節傳以謝侃|{
	敗補邁翻傳株戀翻}
侃答曰古人三敗|{
	謂魯將曹沬也}
君侯始二當今事急不宜數爾|{
	言不宜數數如此數所角翻}
亮司馬陳郡殷融詣侃謝曰將軍為此非融等所裁王彰至曰彰自為之將軍不知也侃曰昔殷融為君子王彰為小人今王彰為君子殷融為小人宣城内史桓彛聞京城不守慷慨流涕進屯涇縣|{
	彛自廣德進屯涇縣}
時州郡多遣使降蘇峻|{
	使疏吏翻降戶江翻}
禆惠復勸彛宜且與通使以紓交至之禍|{
	紓緩也交至之禍言州郡多降峻兵將四合而交至也復扶又翻}
彛曰吾受國厚恩義在致死焉能忍恥與逆臣通問|{
	焉於䖍翻}
如其不濟此則命也彛遣將軍俞縱守蘭石|{
	蘭石在涇縣東北}
峻遣其將韓晃攻之縱將敗左右勸縱退軍縱曰吾受桓侯厚恩當以死報吾之不可負桓侯猶桓侯之不負國也遂力戰而死晃進軍攻彛六月城陷執彛殺之諸軍初至石頭即欲决戰陶侃曰賊衆方盛難與争鋒當以歲月智計破之既而屢戰無功監軍部將李根請築白石壘|{
	是時同盟諸將無監軍事者竊意李根盖郗鑒軍部將也前史既逸郗字後人遂改鑒為監白石壘在石頭東北峻極險固杜佑曰白石壘在臺城西宋武帝大明四年為蠶所置大殿於此}
侃從之夜築壘至曉而成聞峻軍嚴聲|{
	聞峻軍擊鼔嚴隊之聲}
諸將咸愳其來攻孔坦曰不然若峻攻壘必須東北風急令我水軍不得往救今天清静賊必不來所以嚴者必遣軍出江乘掠京口以東矣已而果然侃使庾亮以二千人守白石峻帥步騎萬餘四面攻之不克|{
	帥讀曰率}
王舒虞潭等數與峻兵戰不利|{
	數所角翻}
孔坦曰本不須召郗公遂使東門無限今宜遣還雖晩猶勝不也|{
	言雖遣還之晚猶勝不遣還也}
侃乃令鑒與後將軍郭默還據京口立大業曲阿庱亭三壘以分峻之兵埶|{
	曲阿秦雲陽縣也前漢屬會稽郡後漢屬吳郡晉屬毗陵郡大業里名在曲阿北丁度曰庱亭在吳興庱丑升翻裴松之曰庱攄陵翻}
使郭默守大業壬辰魏該卒祖約遣祖渙桓撫襲湓口|{
	湓口在尋陽今江州德化縣西一里有湓浦}
陶侃聞之將自擊之毛寶曰義軍恃公公不可動寶請討之侃從之渙撫過皖因攻譙國内史桓宣|{
	宣時屯皖縣馬頭山皖戶版翻}
寶往救之為渙撫所敗|{
	敗補邁翻}
箭貫寶髀徹鞍|{
	徹敕列翻}
寶使人蹋鞍拔箭血流滿鞾|{
	鞾許戈翻}
還擊渙撫破走之宣乃得出歸于温嶠寶進攻祖約軍于東關拔合肥戍會嶠召之復歸石頭祖約諸將隂與後趙通謀許為内應後趙將石聰石堪引兵濟淮攻壽春秋七月約衆潰奔歷陽聰等虜壽春二萬餘戶而歸 後趙中山公虎帥衆四萬自軹關西入擊趙河東|{
	軹關在河内軹縣帥讀曰率}
應之者五十餘縣遂進攻蒲阪趙主曜遣河間王述發氐羌之衆屯秦州以備張駿楊難敵自將中外精銳水陸諸軍以救蒲阪自衛關北濟|{
	晉書地理志汲郡汲縣有衛關}
虎懼引退曜追之八月及於高候|{
	杜佑曰今絳州聞喜縣北有高候原}
與虎戰大破之斬石瞻枕尸二百餘里|{
	枕職鴆翻}
收其資仗億計虎奔朝歌|{
	杜佑曰衛州衛縣漢朝歌縣紂都朝歌在今縣西}
曜濟自大陽|{
	大陽屬河東郡應劭曰在大河之陽故曰大陽唐志陜州陜縣有大陽故關春秋之茅津也}
攻石生于金墉決千金堨以灌之|{
	堨烏葛翻}
分遣諸將攻汲郡河内後趙滎陽太守尹矩野王太守張進等皆降之|{
	野王縣自漢以來屬河内郡後趙始置郡也降戶江翻}
襄國大震 張駿治兵欲乘虛襲長安理曹郎中索詢諫曰|{
	理曹郎中張氏所置以掌刑獄索昔各翻}
劉曜雖東征其子胤守長安未易輕也|{
	易以䜴翻下同}
借使小有所獲彼若釋東方之圖還與我校禍難之期未可量也|{
	難乃旦翻量音良}
駿乃止蘇峻腹心路永匡術賈寧聞祖約敗恐事不濟勸峻盡誅司徒導等諸大臣更樹腹心|{
	更工衡翻}
峻雅敬導不許永等更貳於峻|{
	貳者其心攜而兩向}
導使參軍袁耽潜誘永使歸順|{
	誘音酉}
九月戊申導攜二子與永皆奔白石耽渙之曾孫也|{
	袁渙事曹操}
陶侃温嶠等與蘇峻久相持不决峻分遣諸將東西攻掠所嚮多捷人情恟懼|{
	恟許拱翻}
朝士之奔西軍者皆曰峻狡黠有膽决其徒驍勇所向無敵|{
	朝直遥翻黠下八翻驍堅堯翻}
若天討有罪則峻終滅亡止以人事言之未易除也温嶠怒曰諸君怯懦乃更譽賊|{
	譽羊諸翻稱揚之也}
及累戰不勝嶠亦憚之嶠軍食盡貸於陶侃|{
	貸他代翻借也}
侃怒曰使君前云不憂無良將及兵食惟欲得老僕為主耳今數戰皆北良將安在荆州接胡蜀二虜當備不虞若復無食|{
	復扶又翻}
僕便欲西歸更思良算徐來殄賊不為晚也嶠曰凡師克在和古之善教也光武之濟昆陽|{
	見三十九卷漢淮陽王更始元年}
曹公之拔官渡|{
	見六十二卷漢獻帝建安五年}
以寡敵衆杖義故也峻約小豎凶逆滔天何憂不滅峻驟勝而驕自謂無前今挑之戰|{
	挑徒了翻}
可一鼔而擒也奈何捨垂立之功設進退之計乎且天子幽逼社稷危殆乃四海臣子肝腦塗地之日嶠等與公並受國恩事若克濟則臣主同祚如其不捷當灰身以謝先帝耳今之事埶義無旋踵譬如騎虎安可中下哉公若違衆獨返人心必沮沮衆敗事義旗將迴指於公矣|{
	温嶠辭嚴義正所以能留陶侃共成大功沮在呂翻敗補邁翻}
毛寶言於嶠曰下官能留陶公乃往說侃曰|{
	說輸芮翻}
公本應鎮蕪湖為南北埶援前既已下埶不可還且軍政有進無退非直整齊三軍示衆必死而已亦謂退無所據終至滅亡往者杜弢非不彊盛公竟滅之|{
	見八十九卷愍帝建興三年弢士刀翻}
何至於峻獨不可破邪賊亦畏死非皆勇健公可試與寶兵使上岸斷賊資糧|{
	上時掌翻斷丁管翻}
若寶不立効然後公去人心不恨矣侃然之加寶督護而遣之竟陵太守李陽說侃曰|{
	惠帝元康九年分江夏西界立竟陵郡}
今大事若不濟公雖有粟安得而食諸侃乃分米五萬石以餉嶠軍毛寶燒峻句容湖孰積聚|{
	句容湖孰二縣屬丹陽郡}
峻軍乏食侃遂留不去張健韓晃等急攻大業壘中乏水人飲糞汁郭默懼潜突圍出外留兵守之郗鑒在京口軍士聞之皆失色參軍曹納曰大業京口之扞蔽也一旦不守則賊兵徑至不可當也請還廣陵以俟後舉鑒大會僚佐責納曰吾受先帝顧託之重正復捐軀九泉不足報塞|{
	復扶又翻塞悉則翻}
今彊寇在近衆心危逼君腹心之佐而生長異端|{
	長丁丈翻又知兩翻}
當何以帥先義衆鎮壹三軍邪|{
	帥讀曰率下同}
將斬之久乃得釋陶侃將救大業長史殷羨曰吾兵不習步戰救大業而不捷則大事去矣不如急攻石頭則大業自解|{
	謂急攻蘇峻健晃必還救之大業之兵自解}
侃從之羨融之兄也庚午侃督水軍向石頭庾亮温嶠趙胤帥步兵萬人從白石南上欲挑戰峻將八千人逆戰|{
	上時掌翻挑徒了翻將即亮翻下同}
遣其子碩及其將匡孝分兵先薄趙胤軍敗之|{
	薄迫也敗補邁翻}
峻方勞其將士|{
	勞力到翻}
乘醉望見胤走曰孝能破賊我更不如邪因舍其衆|{
	舍讀曰捨}
與數騎北下突陳不得入|{
	陳讀曰陣}
將囘趨白木陂馬躓|{
	躓陟利翻跲也白木陂在東陵東趨七喻翻下兵趨同}
侃部將彭世李千等投之以矛峻墜馬斬首臠割之焚其骨三軍皆稱萬歲餘衆大潰|{
	一鼔禽峻果如温嶠之言}
峻司馬任讓等共立峻弟逸為主閉城自守|{
	任音壬}
温嶠乃立行臺布告遠近凡故吏二千石以下皆令赴臺于是至者雲集韓晃聞峻死引兵趣石頭管商弘徽攻庱亭壘督護李閎輕車長史滕含擊破之|{
	輕車長史輕車將軍長史也}
含修之孫也商走詣庾亮降|{
	降戶江翻}
餘衆皆歸張健 冬十一月後趙王勒欲自將救洛陽|{
	將即亮翻}
僚佐程遐等固諫曰劉曜懸軍千里埶不支久大王不宜親動動無萬全勒大怒按劒叱遐等出乃赦徐光|{
	光被囚見上卷咸和元年}
召而謂之曰劉曜乘一戰之勝圍守洛陽庸人之情皆謂其鋒不可當曜帶甲十萬攻一城而百日不克師老卒怠以我初銳擊之可一戰而禽也若洛陽不守曜必送死冀州|{
	後趙都襄國冀州之地}
自河以北席卷而來|{
	卷讀曰捲下同}
吾事去矣程遐等不欲吾行卿以為何如對曰劉曜乘高候之勢不能進臨襄國更守金墉此其無能為可知也以大王威畧臨之彼必望旗奔敗平定天下在今一舉不可失也勒笑曰光言是也乃使内外戒嚴有諫者斬命石堪石聰及豫州刺史桃豹等各統見衆會滎陽|{
	見賢遍翻}
中山公虎進據石門|{
	水經注漢靈帝於敖城西北壘石為門以遏浚儀渠口謂之石門而滎瀆受河水亦有石門}
勒自統步騎四萬趣金墉濟自大堨|{
	水經注石勒襲劉曜塗出延津以河氷泮為神靈之助號是處為靈昌津騎奇寄翻趣七喻翻堨烏葛翻}
勒謂徐光曰曜盛兵成臯關上策也阻洛水其次也坐守洛陽此成禽耳十二月乙亥後趙諸軍集于成臯步卒六萬騎二萬七千勒見趙無守兵大喜舉手指天復加額曰天也|{
	復扶又翻}
卷甲銜枚詭道兼行出于鞏訾之間|{
	鞏縣屬河南郡有東訾城左傳單子取訾杜預曰在鞏縣西南晉地道記曰在縣之東訾于斯翻}
趙主曜專與嬖臣飲博不撫士卒左右或諫曜怒以為妖言斬之|{
	嬖卑義翻又必計翻妖於驕翻}
聞勒已濟河始議增滎陽戍杜黄馬關|{
	據水經黄馬坂在成臯縣河水逕其北謂之黄馬關}
俄而洛水候者與後趙前鋒交戰擒羯送之曜問大胡自來邪其衆幾何羯曰王自來軍埶甚盛|{
	羯居謁翻}
曜色變使攝金墉之圍|{
	攝馭也}
陳于洛西|{
	陳讀曰陣下揮陳同}
衆十餘萬南北十餘里勒望見益喜謂左右曰可以賀我矣勒帥步騎四萬入洛陽城|{
	帥讀曰率}
己卯中山公虎引步卒三萬自城北而西攻趙中軍石堪石聰等各以精騎八千自城西而北擊趙前鋒大戰于西陽門|{
	西陽門即洛城宣陽門也城西面南頭第一門或曰西陽門即第二門西明門也}
勒躬貫甲胄出自閶闔門夾擊之|{
	閶闔門洛城西面北頭門}
曜少而嗜酒|{
	少詩照翻}
末年尤甚將戰飲酒數斗常乘赤馬無故跼頓|{
	跼足踡曲不能伸也頓首低下不能舉也跼音局}
乃乘小馬比出復飲酒斗餘|{
	比必寐翻復扶又翻下同}
至西陽門揮陳就平石堪因而乘之趙兵大潰曜昏醉退走馬陷石渠墜于氷上被瘡十餘通中者三|{
	中竹仲翻}
為堪所執勒遂大破趙兵斬首五萬餘級下令曰所欲擒者一人耳今已獲之其勑將士抑鋒止銳縱其歸命之路曜見勒曰石王頗憶重門之盟否|{
	據水經注重門城在河内共縣故城西北二十里此盟當在懷帝永嘉四年同圍河内之時重直龍翻}
勒使徐光謂之曰今日之事天使其然復云何邪|{
	復扶又翻}
乙酉勒班師使征東將軍石邃將兵衛送曜邃虎之子也曜瘡甚載以馬輿使毉李永與同載己亥至襄國舍曜於永豐小城給其妓妾嚴兵圍守|{
	妓渠綺翻}
遣劉岳劉震等從男女盛服以見之|{
	岳被禽見上卷明帝太寧三年}
曜曰吾謂卿等久為灰土石王仁厚乃全宥至今邪我殺石佗|{
	見上卷太寧三年}
愧之多矣今日之禍自其分耳|{
	分扶問翻}
留宴終日而去勒使曜與其太子熙書諭令速降|{
	降戶江翻}
曜但敕熙與諸大臣匡維社稷勿以吾易意也勒見而惡之|{
	惡烏路翻}
久之乃殺曜 是歲成漢獻王驤卒|{
	成封李驤為漢王驤思將翻}
其子征東將軍壽以喪還成都成主雄以李玝為征北將軍梁州刺史代壽屯晉壽|{
	玝阮古翻}


四年春正月光禄大夫陸曄及弟尚書左僕射玩說匡術以苑城附于西軍|{
	說輸芮翻}
百官皆赴之推曄督宮城軍事陶侃命毛寶守南城鄧岳守西城|{
	苑城之南城西城也}
右衛將軍劉超侍中鍾雅與建康令管斾等謀奉帝出赴西軍事洩蘇逸使其將平原任讓將兵入宮收超雅|{
	將即亮翻任音壬}
帝抱持悲泣曰還我侍中右衛讓奪而殺之初讓少無行太常華恒為本州大中正|{
	華恒平原高唐人少詩照翻行下孟翻華戶化翻恒戶登翻}
黜其品及讓為蘇峻將乘埶多所誅殺見恒輒恭敬不敢縱暴及鍾劉之死蘇逸欲并殺恒讓盡心救衛恒乃得免 冠軍將軍趙胤遣部將甘苗擊祖約于歷陽戊辰約夜帥左右數百人奔後趙|{
	為後石勒殺祖約張本冠古玩翻帥讀曰率下同}
其將牽騰率衆出降|{
	降戶江翻下同}
蘇逸蘇碩韓晃并力攻臺城焚太極東堂及祕閣毛寶登城射殺數十人|{
	射而亦翻}
晃謂寶曰君名勇果何不出鬬寶曰君名健將|{
	將即亮翻}
何不入鬬晃笑而退 趙太子熙聞趙主曜被禽大懼|{
	被皮義翻}
與南陽王胤謀西保秦州尚書胡勲曰今雖喪君境土尚完將士不叛且當并力拒之力不能拒走未晚也胤怒以為沮衆斬之遂帥百官奔上邽|{
	以劉胤之才武不能守長安以抗石勒劉曜既禽胤膽破矣喪息浪翻沮在呂翻}
諸征鎮亦皆棄所守從之關中大亂將軍蔣英辛恕擁衆數十萬據長安遣使降于後趙後趙遣石生帥洛陽之衆赴之 二月丙戌諸軍攻石頭建威長史滕含擊蘇逸大破之|{
	滕含自輕車長史進建威將軍長史}
蘇碩帥驍勇數百渡淮而戰|{
	淮秦淮也驍堅堯翻}
温嶠擊斬之韓晃等懼以其衆就張健於曲阿門隘不得出更相蹈藉|{
	更工衡翻藉慈夜翻}
死者萬數西軍獲蘇逸斬之滕含部將曹據抱帝奔温嶠船羣臣見帝頓首號泣請罪殺西陽王羕并其二子播充孫崧及彭城王雄|{
	羕附蘇峻見上咸和三年雄奔峻見上卷二年}
陶侃與任讓有舊為請其死|{
	為于偽翻}
帝曰是殺吾侍中右衛者不可赦也乃殺之司徒導入石頭令取故節|{
	導自討王敦時假節其自石頭出奔也弃之}
陶侃笑曰蘇武節似不如是導有慙色|{
	導為侃所譏自愧其失節}
丁亥大赦張健疑弘徽等貳於已皆殺之帥舟師自延陵將入吳興|{
	毗陵縣前漢屬會稽郡後漢分屬吳郡晉分屬毗陵郡師古曰毗陵舊延陵漢改之晉分毗陵延陵為兩縣毗陵則今常州晉陵縣地延陵則今之潤州丹徒金壇之地宋白曰延陵縣本漢曲阿縣地晉太康二年分曲阿之延陵鄉置帥讀曰率}
乙未揚烈將軍王允之與戰大破之獲男女萬餘口健復與韓晃馬雄等西趨故鄣|{
	故鄣縣漢屬丹陽郡吳分吳郡丹陽置吳興郡故鄣屬焉其地本秦鄣郡所治故曰故鄣今湖州安吉縣故鄣之南郷也今廣德軍漢故鄣縣之地杜佑曰湖州長城縣西八十里鄣郡故城即秦鄣郡縣城也復扶又翻趨七喻翻}
郗鑒遣參軍李閎追之及於平陵山皆斬之|{
	蘇峻傳作巖山據帝紀平陵山當在溧陽界沈約曰吳分溧陽為永平縣晉武帝更名永世董覧吳地志云晉分永世為平陵縣宋文帝元嘉九年併入永世溧陽二縣}
是時宮闕灰燼以建平園為宮温嶠欲遷都豫章三吳之豪請都會稽|{
	會工外翻}
二論紛紜未决司徒導曰孫仲謀劉玄德俱言建康王者之宅|{
	見六十六卷漢獻帝建安十七年}
古之帝王不必以豐儉移都苟務本節用何憂彫敝若農事不修則樂土為墟矣|{
	樂音洛}
且北寇游魂伺我之隙|{
	伺相吏翻}
一旦示弱竄於蠻越求之望實懼非良計|{
	望者見於外者也實者有諸中者也}
今特宜鎮之以静羣情自安由是不復徙都以禇翜為丹陽尹|{
	復扶又翻翜所甲翻}
時兵火之後民物彫殘翜收集散亡京邑遂安 壬寅以湘州并荆州|{
	分湘州見八十六卷懷帝永嘉元年}
三月壬子論平蘇峻功以陶侃為侍中太尉封長沙

郡公加都督交廣寧州諸軍事|{
	侃先督荆襄雍梁四州今加都督三州}
郗鑒為侍中司空南昌縣公温嶠為驃騎將軍開府儀同三司加散騎常侍始安郡公|{
	晉制驃騎將軍位從公驃匹妙翻}
陸曄進爵江陵公自餘賜爵侯伯子男者甚衆卞壼及二子眕盱桓彛劉超鍾雅羊曼陶瞻皆加贈諡路永匡術賈寧皆蘇峻之黨也峻未敗永等去峻歸朝廷王導欲賞以官爵温嶠曰永等皆峻之腹心首為亂階罪莫大焉晚雖改悟未足以贖前罪得全首領為幸多矣豈可復褒寵之哉|{
	復扶又翻下同}
導乃止陶侃以江陵偏遠移鎮巴陵|{
	江陵偏在江北又遠建康武帝太康元年立巴陵縣屬長沙郡後置建昌郡水經注曰湘水北至巴丘山入于江石岸有巴陵故城本吳之巴丘邸閣也巴丘山名天岳山一名幕阜前有培塿曰巴蛇冡}
朝議欲留温嶠輔政|{
	朝直遥翻下同}
嶠以王導先帝所任固辭還藩又以京邑荒殘資用不給乃留資蓄具器用而後旋于武昌帝之出石頭也庾亮見帝稽顙哽咽|{
	稽音啓哽古杏翻}
詔亮與大臣俱升御座明日亮復泥首謝罪|{
	復扶又翻下同}
乞骸骨欲闔門投竄山海帝遣尚書侍中手詔慰喻曰此社稷之難|{
	難乃旦翻}
非舅之責也亮上疏自陳祖約蘇峻縱肆凶逆罪由臣發|{
	事見上卷元年}
寸斬屠戮不足以謝七廟之靈塞四海之責|{
	塞悉則翻}
朝廷復何理齒臣於人次臣亦何顔自次於人理願陛下雖垂寛宥全其首領猶宜弃之任其自存自沒則天下粗知勸戒之綱矣|{
	粗坐五翻}
優詔不許亮又欲遁逃山海自暨陽東出|{
	武帝太康二年分毗陵無錫立暨陽縣屬毘陵郡其地在今平江府常熟縣界杜佑曰江隂晉曰暨陽按暨陽今江隂軍城秦漢為暨陽鄉晉置暨陽縣城更有暨陽湖}
詔有司錄奪舟船|{
	録拘也收也}
亮乃求外鎮自效出為都督豫州揚州之江西宣城諸軍事|{
	豫州揚州之江西淮南廬江弋陽安豐歷陽等郡也宣城郡屬揚州}
豫州刺史領宣城内史鎮蕪湖陶侃温嶠之討蘇峻也移檄征鎮使各引兵入援湘州刺史益陽侯卞敦擁兵不赴又不給軍糧遣督護將數百人隨大軍而已朝野莫不怪歎|{
	不料其如此而乃如此故怪之又念其平昔為何如人而今乃為此故歎之}
及峻平陶侃奏敦沮軍顧望不赴國難請檻車收付廷尉|{
	勤王之師侃為盟主湘州又侃所督也故侃奏收敦沮在呂翻難乃旦翻}
王導以喪亂之後宜加寛宥轉敦安南將軍廣州刺史病不赴徵為光禄大夫領少府敦憂愧而卒|{
	少詩照翻卒子恤翻}
追贈本官加散騎常侍諡曰敬|{
	諡法夙夜警戒曰敬合善典法曰敬卞敦何足以當之}
臣光曰庾亮以外戚輔政首發禍機國破君危竄身苟免卞敦位列方鎮兵粮俱足朝廷顛覆坐觀勝負人臣之罪孰大於此既不能明正典刑又以寵禄報之晉室無政亦可知矣任是責者豈非王導乎

徙高密王紘為彭城王紘雄之弟也 夏四月乙未始安忠武公温嶠卒葬於豫章朝廷欲為之造大墓於元明二帝陵之北|{
	為于偽翻}
太尉侃上表曰嶠忠誠著於聖世勲義感於人神使亡而有知豈樂今日勞費之事|{
	樂音洛}
願陛下慈恩停其移葬詔從之以平南軍司劉胤為江州刺史|{
	胤本為温嶠軍司}
陶侃郗鑒皆言胤非方伯才司徒導不從或謂導子悅曰今大難之後|{
	難乃旦翻}
紀綱弛頓自江陵至于建康三千餘里流民萬計布在江州江州國之南藩要害之地而胤以忲侈之性卧而對之|{
	忲奢也忲音太又音大}
不有外變必有内患矣悦曰此温平南之意也|{
	温嶠為平南將軍}
胤秋八月趙南陽王 帥衆數萬自上邽趣長安|{
	帥讀曰率趣七喻翻}
隴東武都安定新平北地扶風始平諸郡戎夏皆起兵應之|{
	魏收地形志有隴東郡領涇陽祖厲撫夷三縣盖後趙分安定置也應劭曰祖音罝師古曰厲音賴夏戶雅翻}
胤軍于仲橋|{
	鄭國渠涇仲山渠上有橋謂之仲橋在九嵏山之東宋白曰雍州醴泉縣城即仲橋城}
石生嬰城自守後趙中山公虎帥騎二萬救之九月虎大破趙兵於義渠|{
	義渠戰國時義渠戎之地前漢為義渠縣後漢晉省}
胤犇還上邽虎乘勝追擊枕尸千里|{
	枕職鴆翻}
上邽潰虎執趙太子熙南陽王胤及其將王公卿校以下三千餘人皆殺之|{
	載記曰自劉淵至曜三世二十七年而滅將即亮翻校戶教翻}
徙其臺省文武關東流民秦雍大族九千餘人于襄國|{
	雍於用翻}
又阬五郡屠各五千餘人于洛陽進攻集木且羌于河西克之|{
	屠各匈奴種前趙之族類也五郡屠各即匈奴五部之衆集木且羌種落之名屠直於翻且子於翻}
俘獲數萬秦隴悉平氐王蒲洪羌酋姚弋仲俱降于虎虎表洪監六夷軍事|{
	酋慈由翻降戶江翻監工銜翻}
弋仲為六夷左都督徙氐羌十五萬落于司冀州 初隴西鮮卑乞伏述延居于苑川|{
	乞伏鮮卑部落之名後以為姓苑川水出天水勇士縣之子城南山東流歷子城川又北逕牧師苑故漢牧苑之地也有東西苑城相去七里西城即乞伏所都也杜佑曰苑川在蘭州五泉縣界}
侵并鄰部士馬彊盛及趙亡述延懼遷于麥田述延卒子傉大寒立傉大寒卒子司繁立|{
	水經注麥田山在安定北界山之東北有麥田城又北有麥田泉傉奴沃翻乞伏始見于此}
江州刺史劉胤矜豪日甚專務商販殖財百萬縱酒躭樂不恤政事冬十二月詔徵後將軍郭默為右軍將軍默樂為邊將不願宿衛以情愬於胤|{
	默盖自平蘇峻還至尋陽而被徵也晉志云按魏明帝時有左軍則左軍魏官也武帝時又置前軍右軍泰始八年又置後軍是為四軍皆宿衛兵也樂音洛邊將即亮翻下同}
胤曰此非小人之所及也|{
	晉以後文武之士率稱小人今西北之人猶然}
默將赴召求資於胤胤不與默由是怨胤胤長史張滿等素輕默或倮露見之|{
	倮郎果翻}
默常切齒臘日胤餉默豚酒默對信投之水中|{
	信使也}
會有司奏今朝廷空竭百官無祿惟資江州運漕而胤商旅繼路以私廢公請免胤官書下|{
	下遐稼翻}
胤不即歸罪方自申理僑人蓋肫掠人女為妻|{
	寄寓者為僑人蓋古盍翻肫徒昆翻}
張滿使還其家肫不從而謂郭默曰劉江州不受免|{
	謂胤不受免官之命也}
密有異圖與張滿等日夜計議惟忌郭侯一人欲先除之默以為然帥其徒候旦門開襲|{
	帥讀曰率}
將吏欲拒默默呵之曰我被詔有所討|{
	被皮義翻}
動者誅三族遂入至内寢牽胤下斬之出取胤僚佐張滿等誣以大逆悉斬之傳胤首于京師詐作詔書宣示内外掠 女及諸妾|{
	孔穎達曰妾之言接也聞彼有禮走而往以得接見于君子也}
并金寶還船初云下都既而停胤故府招引譙國内史桓宣宣固守不從|{
	桓宣自去年歸温嶠屯于武昌}
是歲賀蘭部及諸大人共立拓拔翳槐為代王|{
	賀蘭部擁護翳槐見上卷咸和二年}
代王紇那奔宇文部|{
	後周書言宇文之先出自炎帝炎帝為黄帝所滅其子孫遁居朔野後有大人普回因狩得玉璽文曰皇帝璽普回以為天授其俗謂天子曰宇文故國號宇文因以為氏予謂此盖宇文氏既興於關西其臣子為之緣飾耳李延夀曰宇文部出遼東塞外其先南軍于之遠屬也世為東部大人此言為得其實}
翳槐遣其弟什翼犍質於趙以請和|{
	犍居延翻質音致}
河南王吐延雄勇多猜忌羌酋姜聰刺之|{
	酋慈由翻刺七亦翻}
吐延不抽劒召其將紇扢埿|{
	紇胡骨翻又恨竭翻扢古齕翻又胡骨翻渥與泥同}
使輔其子葉延保于白蘭|{
	白蘭在吐谷渾西南其地險遠羌之别種居之西北接利摸徒南界郍卾風俗物產與宕昌畧同}
抽劒而死葉延孝而好學|{
	好呼到翻}
以為禮公孫之子得以王父字為氏乃自號其國曰吐谷渾|{
	左傳魯衆仲曰天子建德因生以賜姓胙之土而命之氏諸侯以字杜預注曰諸侯之子稱公子公子之子稱公孫公孫之子以王父字為氏}


五年春正月劉胤首至建康司徒導以郭默驍勇難制己亥大赦梟胤首于大航|{
	驍梟並音堅堯翻}
以默為江州刺史太尉侃聞之投袂起曰此必詐也即將兵討之默遣使送妓妾及絹并寫中詔呈侃|{
	妓渠綺翻}
參佐多諫曰默不被詔豈敢為此|{
	被皮義翻}
若欲進軍宜待詔報侃厲色曰國家年幼詔令不出胷懷劉胤為朝廷所禮雖方任非才何緣猥加極刑郭默恃勇所在貪暴以大難新除|{
	謂蘇峻新平也難乃旦翻}
禁綱寛簡欲因際會騁其從横耳|{
	騁丑郢翻從子容翻}
發使上表言狀|{
	使疏吏翻上時掌翻}
且與導書曰郭默殺方州即用為方州害宰相便為宰相乎導乃收胤首答侃書曰默據上流之埶加有船艦成資|{
	艦戶黯翻}
故苞含隱忍使有其地朝廷得以潛嚴|{
	潛密也潛嚴密敕諸軍嚴裝也}
俟足下軍到風發相赴|{
	風發言其速也}
豈非遵養時晦以定大事者邪侃笑曰是乃遵養時賊也豫州刺史庾亮亦請討默詔加亮征討都督帥步騎二萬往與侃會|{
	帥讀曰率}
西陽太守鄧岳武昌太守劉詡皆疑桓宣與默同豫州西曹王隨曰宣尚不附祖約|{
	事見上卷咸和二年}
豈肯同郭默邪岳詡遣隨詣宣觀之隨說宣曰明府心雖不爾|{
	不爾猶言不如此也說輸芮翻}
無以自明惟有以賢子付隨耳宣乃遣其子戎與隨俱迎陶侃侃辟戎為掾|{
	掾于眷翻}
上宣為武昌太守|{
	上時掌翻上言於天臺也}
二月後趙羣臣請後趙王勒即皇帝位勒乃稱大趙天王行皇帝事|{
	勒字世龍初名胷上黨武鄉羯人也其先匈奴别部羌渠之胄祖耶奕干父周曷朱一名乞翼加並為部落小帥}
立妃劉氏為王后世子弘為太子以其子宏為驃騎大將軍都督中外諸軍事大單于封秦王|{
	驃匹妙翻單音蟬}
斌為左衛將軍封太原王|{
	斌音彬}
恢為輔國將軍封南陽王以中山公虎為太尉尚書令進爵為王虎子邃為冀州刺史封齊王宣為左將軍挺為侍中封梁王又封石生為河東王石堪為彭城王以左長史郭敖為尚書左僕射右長史程遐為右僕射領吏部尚書左司馬夔安右司馬郭殷從事中郎李鳳前郎中令裴憲皆為尚書參軍事徐光為中書令領祕書監自餘文武封拜各有差中山王虎怒私謂齊王邃曰主上自都襄國以來|{
	懷帝永嘉六年勒據襄國}
端拱仰成|{
	仰牛向翻}
以吾身當矢石二十餘年南擒劉岳|{
	見上卷明帝太寧三年}
北走索頭|{
	見上卷咸和二年索昔各翻}
東平齊魯西定秦雍|{
	平齊魯謂滅徐龕曹嶷也見九十二卷元帝永昌元年明帝太寧元年定秦雍謂滅劉氏降苻姚也}
克十有三州成大趙之業者我也大單于當以授我今乃以與黄吻婢兒|{
	吻武粉翻口邊曰吻鳥雛始出巢者口黄未褪目之曰黄吻言少艾也}
念之令人氣塞不能寢食待主上晏駕之後不足復留種也|{
	塞悉則翻復扶又翻種章勇翻}
程遐言於勒曰天下粗定|{
	粗坐五翻}
當顯明逆順故漢高祖赦季布斬丁公|{
	事見十一卷高祖五年}
大王自起兵以來見忠於其君者輒褒之背叛不臣者輒誅之|{
	背蒲妹翻}
此天下所以歸盛德也今祖約猶存臣竊惑之安西將軍姚弋仲亦以為言勒乃收約并其親屬中外百餘人悉誅之妻妾兒女分賜諸胡初祖逖有胡奴曰王安逖甚愛之在雍丘謂安曰石勒是汝種類|{
	種章勇翻}
吾亦無在爾一人厚資送而遣之安以勇幹仕趙為左衛將軍及約之誅安歎曰豈可使祖士雅無後乎|{
	祖逖字士雅}
乃往就市觀刑逖庶子道重始十歲安竊取以歸匿之變服為沙門及石氏亡道重復歸江南 郭默欲南據豫章|{
	欲自尋陽而南據也}
會太尉侃兵至默出戰不利入城固守聚米為壘以示有餘侃築土山臨之三月庾亮兵至湓口|{
	湓口湓浦口也湓蒲奔翻}
諸軍大集夏五月乙卯默將宋侯縳默父子出降|{
	將即亮翻降戶江翻}
侃斬默于軍門傳首建康同黨死者四十人詔以侃都督江州領刺史|{
	至是侃都督八州}
以鄧岳督交廣諸軍事領廣州刺史侃還巴陵因移鎮武昌庾亮還蕪湖辭爵賞不受 趙將劉徵帥衆數千浮海抄東南諸縣殺南沙都尉許儒|{
	沈約志晉陵太守有南沙令本吳縣司鹽都尉署吳時名沙中吳平後立暨陽割屬之晉成帝咸康七年罷鹽署立以為南沙縣今平江府常熟縣地帥讀曰率抄楚交翻}
張駿因前趙之亡復收河南地至于狄道置五屯護軍與趙分境|{
	駿失河南地見上卷咸和二年五屯護軍武街石門侯和漒川甘松也}
六月趙遣鴻臚孟毅拜駿征西大將軍凉州牧加九錫|{
	臚陵如翻}
駿恥為之臣不受留毅不遣 初丁零翟斌世居康居後徙中國至是入朝於趙趙以斌為句町王|{
	朝直遥翻句音胊町音捉 考異曰晉書春秋作翟真按秦亡後慕容垂誅翟斌斌兄子真北走故知此乃斌也}
趙羣臣固請正尊號秋九月趙王勒即皇帝位|{
	考異曰載記云自襄國都臨漳即鄴也按建平二年四月勒如鄴議營新宮三年勒如鄴臨石虎第勒疾虎詐召石宏還襄國至虎建武元年九月始遷鄴是勒未嘗都鄴也}
大赦改元建平文武封進各有差立其妻劉氏為皇后太子弘為皇太子弘好屬文|{
	好呼到翻屬之欲翻}
親敬儒素勒謂徐光曰大雅愔愔|{
	弘字大雅愔愔安和貌音揖淫翻}
殊不似將家子|{
	將即亮翻}
光曰漢祖以馬上取天下孝文以玄默守之聖人之後必有勝殘去殺者天之道也|{
	論語孔子曰善人為邦百年亦可以勝殘去殺矣王氏注曰勝殘能使殘暴之人不為惡也去殺去刑殺也勝音升去羌呂翻}
勒甚悅光因說曰|{
	說輸芮翻}
皇太子仁孝温恭中山王雄暴多詐陛下一旦不諱臣恐社稷非太子所有也宜漸奪中山王權使太子早參朝政|{
	朝直遥翻}
勒心然之而未能從 趙荆州監軍郭敬寇襄陽南中郎將周撫監沔北軍事屯襄陽|{
	監工銜翻}
趙主勒以驛書敕敬退屯樊城使之偃藏旗幟寂若無人|{
	幟尺志翻}
曰彼若使人觀察則告之曰汝宜自愛堅守後七八日大騎將至|{
	騎奇寄翻}
相策|{
	相策謂相策歷也杜佑通典作相禁一曰相策屬下句策計也猶言計汝不復得走也}
不復得走矣|{
	復扶又翻下同}
敬使人浴馬于津周而復始晝夜不絶偵者還以告周撫|{
	偵丑鄭翻}
撫以為趙兵大至愳奔武昌敬入襄陽中州流民悉降于趙魏該弟遐帥其部衆自石城降敬|{
	帥讀曰率降戶江翻}
敬毁襄陽城遷其民于沔北城樊城以戍之趙以敬為荆州刺史周撫坐免官 休屠王羌叛趙|{
	休屠王羌石武之部落也屠直於翻}
趙河東王生擊破之羌奔凉州西平公駿懼遣孟毅還使其長史馬詵稱臣入貢于趙更造新宮|{
	蘇峻之亂宮闕焚毁故更造之更工衡翻}
甲辰徙樂成王欽

為河間王|{
	河間王顒之死也詔以彭城王植子融為顒嗣改封樂成縣王薨無子元帝又以彭城王釋子欽為融嗣今復其河間舊封}
封彭城王紘子俊為高密王|{
	初元帝以紘繼高密王據後及彭城王雄以附蘇峻誅紘還繼本宗以俊奉高密王後 考異曰宗室傳作俊今從帝紀}
冬十月成大將軍壽督征南將軍費黑等攻巴東建平拔之巴東太守楊謙監軍毋丘奥退保宜都|{
	費扶沸翻監工銜翻考異曰帝紀作陽謙今從李雄載記}


六年春正月趙劉徵復寇婁縣掠武進|{
	婁縣前漢屬會稽郡後漢晉屬吳郡吳孫權嘉禾三年改丹徒曰武進晉武帝太康三年復曰丹徒仍分丹徒曲阿立武進縣屬毗陵郡晉改毗陵曰晉陵劉昫曰唐蘇州崑山縣漢婁縣地復扶又翻下同}
郗鑒擊却之 三月壬戌朔日有食之 夏趙主勒如鄴將營新宮廷尉上黨續咸苦諫勒怒欲斬之中書令徐光曰咸言不可用亦當容之奈何一旦以直言斬列卿乎勒歎曰為人君不得自專如是乎匹夫家貲滿百匹猶欲市宅况富有四海乎此宮終當營之且敕停作以成吾直臣之氣因賜咸絹百匹稻百斛又詔公卿以下歲舉賢良方正仍令舉人得更相薦引以廣求賢之路起明堂辟雍靈臺于襄國城西|{
	史言石勒能矯其獷悍之習而修文}
秋七月成大將軍壽攻隂平武都楊難敵降之|{
	降戶江翻}
九月趙主勒復營鄴宮以洛陽為南都置行臺 冬蒸祭太廟詔歸胙於司徒導|{
	禮記冬祭曰烝史漢亦作蒸祭餘肉曰胙今謂之祭福肉}
且命無下拜|{
	晉以周之禮齊桓公者禮王導}
導辭疾不敢當初帝即位冲幼每見導必拜與導手詔則云惶恐言中書作詔則曰敬問有司議元會日帝應敬導不|{
	不讀曰否}
博士郭熙杜援議以為禮無拜臣之文謂宜除敬侍中馮懷議以為天子臨辟雍拜三老况先帝師傅謂宜盡敬侍中荀奕議以為三朝之首|{
	元旦為三朝謂歲之朝月之朝日之朝朝如字}
宜明君臣之體則不應敬若它日小會自可盡禮|{
	以君拜臣謂之盡禮可乎}
詔從之奕組之子也 慕容廆遣使與太尉陶侃牋勸以興兵北伐共清中原僚屬宋該等共議以廆立功一隅位卑任重等差無别不足以鎮華夷宜表請進廆官爵參軍韓恒駁曰|{
	廆戶罪翻使疏吏翻恒戶登翻駁北角翻}
夫立功者患信義不著不患名位不高桓文有匡復之功不先求禮命以令諸侯宜繕甲兵除羣凶功成之後九錫自至比於邀君以求寵不亦榮乎廆不悅出恒為新昌令|{
	新昌縣屬遼東郡}
於是東夷校尉封抽等疏上侃府|{
	上時掌翻}
請封廆為燕王行大將軍事侃復書曰夫功成進爵古之成制也車騎雖未能為官摧勒|{
	廆加車騎將軍故侃稱之官謂天子勒謂石勒也騎奇寄翻}
然忠義竭誠今騰牋上聽|{
	騰牋以達上聽}
可不遲速當在天臺也|{
	陶侃復書殊得體天臺尊晉室也不讀曰否}


資治通鑑卷九十四
