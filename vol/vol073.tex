










 


 
 


 

  
  
  
  
  





  
  
  
  
  
 
  

  

  
  
  



  

 
 

  
   




  

  
  


    資治通鑑卷七十三   宋 司馬光 撰

  胡三省 音註

  魏紀五【起旃蒙單閼盡彊圉大荒落凡三年】

  烈祖明皇帝中之下

  青龍三年春正月戊子以大將軍司馬懿為太尉 丁巳皇太后郭氏殂帝數問甄后死狀於太后【甄后死見六十五卷文帝之黄初年數所角翻甄之人翻】 由是太后以憂殂 漢楊儀既殺魏延【事見上卷上年】自以為有大功宜代諸葛亮秉政而亮平生密指以儀狷狹【密指蓋亮密以語諸僚佐特儀不知耳狷吉掾翻】意在蔣琬儀至成都拜中軍師無所統領從容而已【從千容翻】初儀事昭烈帝為尚書琬時為尚書郎後雖俱為丞相參軍長史儀每從行當其勞劇自謂年宦先琬才能踰之【先悉薦翻】於是怨憤形于聲色歎咤之音發於五内【咤叱稼翻噴也叱怒也五内五藏之内也】時人畏其言語不節莫敢從也惟後軍師費禕往慰省之【費父沸翻省悉景翻】儀對禕恨望前後云云【云云師古曰猶言如此如此也】又語禕曰往者丞相亡沒之際吾若舉軍以就魏氏處世寧當落度如此邪【語牛倨翻處昌呂翻度徒洛翻落度失意也】令人追悔不可復及【復扶又翻下同】禕密表其言漢主廢儀為民徙漢嘉郡【漢嘉縣故青衣也漢順帝陽嘉二年改為漢嘉屬蜀郡屬國都尉蜀郡屬國安帝延光元年所置蜀分為漢嘉郡】儀至徙所復上書誹謗辭指激切遂下郡收儀【上時掌翻下遐稼翻】儀自殺 三月庚寅葬文德皇后【文德郭后也郭后諡曰德甄后諡曰昭】 夏四月漢主以蔣琬為大將軍録尚書事費禕代琬為尚書令 帝好土功【好呼到翻】既作許昌宫【事見上卷太和六年】又治洛陽宫【諸葛亮死帝乃大興宫室晉士燮所謂釋楚為外懼者此也治直之翻】起昭陽太極殿【水經注明帝上法太極於洛陽南宫起太極殿即漢崇德殿之故處】築總章觀高十餘丈【舜有總章之訪相傳以為總章即明堂也觀闕也總章觀蓋在太極殿前觀古玩翻高居傲翻】力役不已農桑失業司空陳羣上疏曰昔禹承唐虞之盛猶卑宫室而惡衣服况今喪亂之後人民至少【喪息浪翻少詩沼翻】比漢文景之時不過漢一大郡【漢自秦項之争民死於兵者多矣雖文景與民休息戶口蕃息重以武帝窮奢極欲又減其半平帝元始之初民戶一千三百二十三萬三千六百一十二以班志考之汝南一郡戶四十六萬一千五百八十七光武興於南陽至永和元年戶五十餘萬三國虎爭人衆之損萬有一存景元四年與蜀通計民戶九十四萬三千二百四十三耳當此之時謂不過漢文景時一大郡非虚語也】加以邊境有事將士勞苦【將即亮翻】若有水旱之患國家之深憂也昔劉備自成都至白水多作傳舍【典略曰備鎮成都拔魏延督漢中於是起館舍築亭障從成都至白水關四百餘區傳株戀翻】興費人役太祖知其疲民也今中國勞力亦吳蜀之所願此安危之機也惟陛下慮之帝荅曰王業宫室亦宜並立滅賊之後但當罷守禦耳豈可復興役邪【復扶又翻下同】是固君之職蕭何之大略也【此指蕭何治未央宫事為言】羣曰昔漢祖惟與項羽争天下羽已滅宫室燒焚是以蕭何建武庫太倉皆是要急然高祖猶非其壯麗【羣因帝蕭何之言以陳善閉邪蕭何事見十一卷高帝七年】今二虜未平誠不宜與古同也夫人之所欲莫不有辭况乃天王莫之敢違前欲壞武庫謂不可不壞也後欲置之謂不可不置也【此皆指帝拒諫實事壞音怪】若必作之固非臣下辭言所屈若少留神【少詩沼翻下同】卓然回意亦非臣下之所及也漢明帝欲起德陽殿鍾離意諫即用其言後乃復作之殿成謂羣臣曰鍾離尚書在不得成此殿也夫王者豈憚一人蓋為百姓也【為于偽翻下同】今臣曾不能少凝聖聽【凝定也停也言帝不為之留聽也】不及意遠矣帝乃為之少有減省帝耽于内寵婦官秩石擬百官之數【西漢婦官十四等秩石視内外百官之數魏武建國始命王后其下五等曰夫人昭儀倢伃容華美人文帝增貴嬪淑媛脩容順成良人明帝增淑妃昭華脩儀除順成官太和中始復命夫人登其位於淑妃之上自夫人以下爵凡十二等貴嬪夫人位次皇后爵無所視淑妃位視相國爵比諸侯王淑媛位視御史大夫爵比縣公昭儀比縣侯昭華比鄉侯脩容比亭侯脩儀比關内侯倢伃比中二千石容華視真二千石美人視比二千石良人視千石】自貴人以下至掖庭灑掃凡數千人【灑所賣翻掃素報翻又並如字】選女子知書可付信者六人以為女尚書使典省外奏事處當畫可【漢東都之末宫中有女尚書處當奏事有不合上意區處其當而下之也畫可畫從其所奏省悉景翻處昌呂翻】廷尉高柔上疏曰昔漢文惜十家之資不營小臺之娛去病慮匈奴之害不遑治第之事【冶直之翻】况今所損者非惟百金之費所憂者非徒北狄之患乎可粗成見所營立以充朝宴之儀【粗坐五翻見賢遍翻朝直遥翻】訖罷作者使得就農二方平定復可徐興周禮天子后妃以下百二十人【王立后三夫人九嬪二十七世婦八十一御妻是為百二十人】嬪嬙之儀既已盛矣竊聞後庭之數或復過之【嬪毘賓翻嬙慈良翻復扶又翻下同】聖嗣不昌殆能由此臣愚以為可妙簡淑媛以備内官之數【媛美女也淑善也媛于絹翻】其餘盡遣還家且以育精養神專静為寶如此則螽斯之徵可庶而致矣【詩螽斯后妃子孫衆多也】帝報曰輒克昌言他復以聞【輒以昌言自克也楊子曰勝己之私之謂克】是時獵法嚴峻殺禁地鹿者身死財產沒官有能覺告者厚加賞賜柔復上疏曰中間以來百姓供給衆役親田者既減【親田謂躬親田畝者】加頃復有獵禁羣鹿犯暴殘食生苖處處為害所傷不訾【不訾言不可計量也】民雖障防力不能禦至如滎陽左右周數百里歲略不收方今天下生財者甚少而麋鹿之損者甚多卒有兵戎之役凶年之災【卒讀曰猝】將無以待之惟陛下寛放民間使得捕鹿遂除其禁則衆庶永濟莫不悦豫矣帝又欲平北芒令於其上作臺觀望見孟津【黃圖曰登之可以遠觀故曰觀觀古玩翻】衛尉辛毗諫曰天地之性高高下下【國語周太子晉曰天地成而聚於高歸物於下四岳佐禹高高下下封崇九山决汨九川】今而反之既非其理加以損費人功民不堪役且若九河盈溢洪水為害而丘陵皆夷將何以禦之帝乃止少府楊阜上疏曰陛下奉武皇帝開拓之大業守文皇帝克終之元緒【元始也緒絲端也言文帝克終武帝之志受禪易制此絲端所從始也】誠宜思齊往古聖賢之善治【治直吏翻】總觀季世放蕩之惡政曩使柦靈不廢高祖之法度文景之恭儉太祖雖有神武於何所施而陛下何由處斯尊哉【處昌呂翻】今吳蜀未定軍旅在外諸所繕治惟陛下務從約節【治直之翻】帝優詔荅之阜復上疏曰堯尚茅茨而萬國安其居【堯土堦三尺茅茨不翦】禹卑宫室而天下樂其業【樂音洛】及至殷周或堂崇三尺度以九筵耳【周官考工記曰殷人重屋堂脩七尋堂崇三尺周人明堂度九尺之筵東西九筵南北七筵堂崇一筵五室凡室二筵】桀作璇室象廊【史記龜策傳曰桀為瓦室紂為象廊與此稍異】紂為傾宫鹿臺【新序曰鹿臺其大三里高千仞臣瓚曰今在朝歌城中】以喪其社稷【喪息浪翻】楚靈以築章華而身受禍【楚靈王為章華之臺民不堪命從亂如歸王走而死于芋尹氏】秦始皇作阿房二世而滅【事見七卷三十五年】夫不度萬民之力以從耳目之欲【度徒洛翻】未有不亡者也陛下當以堯舜禹湯文武為法則夏桀殷紂楚靈秦皇為深誡而乃自暇自逸惟宫臺是飾必有顛覆危亡之禍矣君作元首臣為股肱存亡一體得失同之臣雖駑怯敢忘爭臣之義【駑音奴爭讀曰諍】言不切至不足以感悟陛下陛下不察臣言恐皇祖烈考之祚墜于地使臣身死有補萬一則死之日猶生之年也謹叩棺沐浴伏俟重誅奏御【叩近也御進也】帝感其忠言手筆詔荅帝嘗著帽被縹綾半袖【著陟略翻說文曰帽小兒蠻夷頭衣縹普沼翻青白色綾紋帛或謂之綺或謂之紋繒半袖半臂也晉志曰帽名猶冠也義取於蒙覆其首其本纚也古者冠無幘冠下有纚以繒為之後世施幘於冠因或裁纚為帽自乘輿宴居下至庶人無爵者皆服之被皮義翻】阜問帝曰此於禮何法服也帝默然不答自是不法服不以見阜阜又上疏欲省宫人諸不見幸者乃召御府吏問後宫人數【少府屬官有御府令典官婢員吏七十人吏從官三十人】吏守舊令對曰禁密不得宣露阜怒杖吏一百數之曰【數所具翻】國家不與九卿為密反與小吏為密乎帝愈嚴憚之散騎常侍蔣濟上疏曰昔句踐養胎以待用【國語越王句踐困於會稽既反國命壯者無取老婦老者無取壯妻女子十七不嫁丈夫二十不娶其父母有罪將免乳者以告公令醫守之生丈夫二壺酒一犬生女子二壺酒一豚生三人公與之母生二人公與之餼散悉亶翻騎奇寄翻】昭王恤病以雪仇【燕昭王於破燕之後弔死問疾欲以報齊雪先王之恥】故能以弱燕服強齊羸越滅勁吳今二敵強盛當身不除百世之責也【謂當帝之身不能滅吳蜀後世之責必歸於帝】以陛下聖明神武之略舍其緩者【舍讀曰捨】專心討賊臣以為無難矣中書侍郎東萊王基上疏曰【按此則魏已改通事郎為中書侍郎矣】臣聞古人以水喻民曰水所以載舟亦所以覆舟【家語載孔子之言】顔淵曰東野子之御馬力盡矣而求進不已殆將敗矣【荀子魯定公問於顔淵曰東野子善御乎顔淵曰善則善矣雖然其馬將失定公曰何以知之顔淵曰臣以政知之昔舜巧於使民造父巧於使馬舜不窮其民力造父不窮其馬力是舜無失民造父無失馬也今東野畢之御上車執轡御體正矣步驟馳騁朝禮畢矣歷險致遠馬力盡矣然猶求進不已是以知之也】今事役勞苦男女離曠願陛下深察東野之敝留意舟水之喻息奔駟於未盡節力役於未困昔漢有天下至孝文時唯有同姓諸侯而賈誼憂之曰置火積薪之下而寢其上因謂之安【見十四卷漢文帝六年】今寇賊未殄猛將擁兵檢之則無以應敵久之則難以遺後【謂五大在邊尾大不掉非善計以詒後人也遺于季翻】當盛明之世不務以除患若子孫不競【競強也】社稷之憂也使賈誼復起必深切於曩時矣【言不特痛哭流涕長太息而已復扶又翻下同】帝皆不聽殿中監督役擅收蘭臺令史【此殿中監以其時營造宫守使監作殿中耳非唐殿中監之官也觀後所謂校事可知矣又據晉書輿服志大駕鹵簿左殿中御史右殿中監則魏時殿中監已有定官蘭臺令史屬御史臺會要曰漢謂御史臺為蘭臺】右僕射衛臻奏案之詔曰殿舍不成吾所留心卿推之何也【推考鞫也】臻曰古制侵官之法【古者百官不相踰越左傳欒鍼曰侵官冒也】非患其勤事也【惡烏路翻】誠以所益者小所墮者大也【墮讀曰隳】臣每察校事類皆如此【魏武建國置校事使察羣下】若又縱之懼羣司將遂越職以至陵夷矣尚書涿郡孫禮固請罷役帝詔曰欽納讜言【讜音黨】促遣民作監作者復奏留一月有所成訖【成訖言欲成殿舍以訖事也監古衘翻】禮徑至作所不復重奏【重直龍翻】稱詔罷民帝奇其意而不責帝雖不能盡用羣臣直諫之言然皆優容之秋七月洛陽崇華殿災帝問侍中領太史令泰山高堂隆【太史令屬太常隆以侍中領之漢儒有高堂生魯人隆其後也姓譜齊公族有高堂氏風俗通齊卿高恭仲食采於高堂】曰此何咎也於禮寧有祈禳之義乎對曰易傳曰上不儉下不節孽火燒其室又曰君高其臺天火為災【京房易傳之辭傳直戀翻孽魚列翻】此人君務飾宫室不知百姓空竭故天應之以旱火從高殿起也詔問隆吾聞漢武之時栢梁災而大起宫殿以厭之【事見二十一卷漢武帝太初元年厭益涉翻下同】其義云何對曰夷越之巫所為非聖賢之明訓也五行志曰栢梁災其後有江充巫蠱事如志之言越巫建章無所厭也今宜罷散民役宫室之制務從約節清埽所災之處不敢於此有所立作則萐莆嘉禾必生此地【萐山輒翻又色洽翻莆音蒲說文萐莆瑞草也堯時生於庖厨扇暑而凉】若乃疲民之力竭民之財非所以致符瑞而懷遠人也 八月庚午立皇子芳為齊王詢為秦王帝無子養二王為子宫省事秘莫有知其所由來者或云芳任城王楷之子也【楷任城王彰之子任音壬】 丁巳帝還洛陽 詔復立崇華殿【復扶又翻】更名曰九龍【據高堂隆傳時郡國有九龍見因以名殿更工衡翻】通引穀水過九龍殿前【水經注穀渠東歷故金市南直千秋門枝流入石逗伏流注靈芝九龍池】為玉井綺欄蟾蜍含受神龍吐出使博士扶風馬鈞作司南車【司南車即指南車也崔豹古今注曰黄帝與蚩尤戰于涿鹿蚩尤作大霧士皆迷路乃作指南車以正四方述征記曰指南車上有木仙人持信旛車轉而人常指南】水轉百戲【傅玄曰人有上百戲而不能動帝問鈞可動否對曰可動其巧可益否對曰可益受詔作之以大木彫構使其形若輪平地施之潛以水發焉設為女樂舞象至令木人擊鼔吹簫作山嶽使木人跳絙擲劒緣絙倒立出入自在百官行署舂磨鬬雞變巧百端】陵霄闕始構有鵲巢其上帝以問高堂隆對曰詩曰惟鵲有巢惟鳩居之【詩召南鵲巢之辭也】今興宫室起陵霄闕而鵲巢之此宫未成身不得居之象也天意若曰宫室未成將有他姓制御之斯乃上天之戒也夫天道無親惟與善人太戊武丁覩災悚懼故天降之福【太戊桑榖生朝武丁飛雉雊鼎皆能戒懼轉災為福】今若罷休百役增崇德政則三王可四五帝可六豈惟商宗轉禍為福而已哉帝為之動容【為于偽翻下同】帝性嚴急其督脩宫室有稽限者【立為期限以必其成及期而不成為稽限】帝親召問言猶在口身首已分散騎常侍領秘書監王肅【漢桓帝延熹二年置秘書監秩四百石】上疏曰今宫室未就見作者三四萬人【見賢遍翻】九龍可以安聖體其内足以列六宫惟泰極已前功夫尚大【泰極謂太極殿】願陛下取常食禀之士非急要者之用選其丁壯擇留萬人使一期而更之【更工衡翻】咸知息代有日則莫不悦以即事勞而不怨矣【易曰說以使民民忘其勞】計一歲有三百六十萬夫亦不為小當一歲成者聽且三年分遣其餘使皆即農無窮之計也夫信之於民國家大寶也前車駕當幸洛陽發民為營有司命以營成而罷【此營壘之營】既成又利其功力不以時遣有司徒營目前之利【此營求之營】不顧經國之體臣愚以為自今已後儻復使民【復扶又翻】宜明其令使必如期以次有事寧使更發無或失信【謂始焉於甲處營造發民就役次焉於乙處營造不可仍用甲處就役之民寧使更發民以供乙處之役也】凡陛下臨時之所行刑皆有罪之吏宜死之人也然衆庶不知謂為倉卒故願陛下下之於吏【卒讀曰猝下之之下音戶稼翻下同】鈞其死也無使汙于宫掖【鈞與均同汙烏故翻】而為遠近所疑且人命至重難生易殺【易以䜴翻】氣絶而不續者也是以聖賢重之昔漢文帝欲殺犯蹕者廷尉張釋之曰方其時上使誅之則已今下廷尉廷尉天下之平不可傾也【事見十四卷漢文帝三年下遐稼翻】臣以為大失其義非忠臣所宜陳也廷尉者天子之吏也猶不可以失平而天子之身反可以惑謬乎【斯論誠足以矯張釋之之失言】斯重於為己而輕於為君【為于偽翻】不忠之甚者也不可不察中山恭王衮疾病令官屬曰男子不死於婦人之手

  【喪大記之言】亟以時營東堂堂成輿疾往居之又令世子曰汝幼為人君知樂不知苦必將以驕奢為失者也兄弟有不良之行【樂音洛行下孟翻】當造䣛諫之【造䣛詣䣛前也造七到翻䣛與膝同】諫之不從流涕喻之喻之不改乃白其母猶不改當以奏聞并辭國土與其守寵罹禍不若貧賤全身也此亦謂大罪惡耳其微過細故當掩覆之【覆敷救翻】冬十月己酉衮卒 十一月丁酉帝行如許昌 是歲幽州刺史王雄使勇士韓龍刺殺鮮卑軻比能自是種落離散【刺七亦翻種章勇翻】互相侵伐強者遠遁弱者請服邊陲遂安 張掖柳谷口水溢涌【魏氏春秋曰張掖刪丹縣金山玄川謚漢晉春秋曰氐池縣大柳谷口夜激波涌溢刪丹氐池二縣漢志皆屬張掖晉志無之當是併省也五代志甘州張掖縣有大柳谷又後周廢金山縣入刪丹縣蓋歷代廢置無常疆土有離合也】寶石負圖狀象靈龜立于川西有石馬七及鳳凰麒麟白虎犧牛璜玦八卦列宿孛彗之象【宿音秀孛蒲内翻彗徐芮翻乂徐醉翻又祥歲翻】又有文曰大討曹【石圖之文天意蓋昭昭矣】詔書班天下以為嘉瑞任令于綽連齎以問鉅鹿張臶【任縣前漢屬廣平國後漢屬鉅鹿郡魏復屬廣平郡師古曰任本晉邑也鄭皇頡奔晉為任大夫劉昫曰唐邢州任縣漢鉅鹿南䜌縣地晉置任縣治苑郷城連齎者連詔書及班下石圖齎以問張臶也張臶兼内外學故以問之臶徂悶翻又在甸翻祖悶翻】臶密謂綽曰夫神以知來不追既往祥兆先見而後廢興從之【見賢遍翻】今漢已久亡魏已得之何所追興祥兆乎此石當今之變異而將來之符瑞也【後人以此為晉繼魏之徵牛繼馬又以為元帝本牛氏繼司馬之徵】 帝使人以馬易珠璣翡翠玳瑁於吳【珠不圓者為璣又曰粗瑀為璣玳徒耐翻瑁蒲佩翻】吳主曰此皆孤所不用而可以得馬孤何愛焉盡以與之四年春吳人鑄大錢一當五百【杜佑曰孫權嘉平五年鑄大錢一當五百文曰大錢五百徑一寸三分重十二銖】 三月吳張昭卒年八十一昭容貌矜嚴有威風吳主以下舉邦憚之 夏四月漢主至湔登觀阪觀汶水之流【湔即漢之湔氐道屬蜀郡汶水即㟭江水也㟭江出氐道西徼外㟭山東流歷都安縣沈約曰縣蜀所立水經注曰都安縣有桃關蜀守李氷作大堰于此謂之湔塴亦曰湔堰觀阪在其上裴松之曰湔音翦晉書音義汶讀與㟭同諸葛亮既沒漢主游觀莫之敢止】旬日而還【還從宣翻又如字】 武都氐苻健請降於漢【以此觀之諸氐固先有苻姓矣不待蒲堅以背文草付之祥乃姓苻也杜佑曰氐者西戎别種漢武帝開武都郡排其種人分竄山谷或在上禄或在河隴左右魏武令夏侯淵討叛氐阿貴千萬等後因拔棄漢中遂徙武都之種於秦川是曰楊氐苻堅之先是曰苻氐楊氐苻氐同出略陽世為婚姻降戶江翻】其弟不從將四百戶來降五月乙卯樂平定侯董昭卒【諡法大慮静民曰定純行不爽曰定】 冬十月己卯帝還洛陽宫 甲申有星孛于大辰【公羊傳曰大辰者何大火也何休注曰大火與伐天之所以示民時早晚天下之所以取正故謂之大辰蔡邕曰自亢八度至尾四度謂之大火陳卓曰自氐五度至尾九度曰大火之次於辰在卯孛蒲内翻】又孛于東方高堂隆上疏曰凡帝王徙都立邑皆先定天地社稷之位【所謂圜丘方澤南北郊及社稷神位也】敬恭以奉之將營宫室則宗廟為先廏庫為次居室為後【記曲禮之言】今圜丘方澤南北郊明堂社稷神位未定宗廟之制又未如禮而崇飾居室士民失業外人咸云宫人之用與軍國之費略齊民不堪命皆有怨怒書曰天聰明自我民聰明天明畏自我民明威【書皋陶謨之言孔安國注曰言天因民而降之福民所歸者天命之天視聽人君之行用民為聰明天明可畏亦用民成其威民所叛者天討之是天明可畏之效也】言天之賞罰隨民言順民心也夫采椽卑宫唐虞大禹之所以垂皇風也【采椽即采來之木為椽不加斵削也】玉臺瓊室夏癸商辛之所以犯昊天也【張藴古曰彼昏不知瑶其臺而瓊其室文選東都賦注曰紂為瓊室以瓊瑶飾之】今宫室過盛天彗章灼【彗祥歲翻音又見上】斯乃慈父懇切之訓當崇孝子祇聳之禮不宜有忽以重天怒隆數切諫【數所角翻下同】帝頗不悦侍中盧毓進曰臣聞君明則臣直古之聖王惟恐不聞其過此乃臣等所以不及隆也帝乃解毓植之子也十二月癸巳潁隂靖侯陳羣卒【諡法恭已鮮言曰靖寛樂令終曰靖】羣

  前後數陳得失【數所角翻】每上封事輒削其草時人及其子弟莫能知也論者或譏羣居位拱默【言拱手而已默無一言】正始中詔撰羣臣上書以為名臣奏議【撰雛免翻】朝士乃見羣諫事皆歎息焉

  袁子論曰或云少府楊阜豈非忠臣哉見人主之非則勃然觸之與人言未嘗不道【道者言之也】答曰夫仁者愛人施之君謂之忠施於親謂之孝今為人臣見人主失道力詆其非而播揚其惡可謂直士未為忠臣也故司空陳羣則不然談論終日未嘗言人主之非書數十上【上時掌翻】外人不知君子謂羣於是乎長者矣

  乙未帝行如許昌 詔公卿舉才德兼備者各一人司馬懿以兖州刺史太原王昶應選【兖州統陳留東郡濟隂任城東平濟北泰山昶丑兩翻】昶為人謹厚名其兄子曰默曰沈【沈時林翻】名其子曰渾曰深為書戒之曰吾以四者為名欲使汝曹顧名思義不敢違越也夫物速成則疾亡晚就而善終朝華之草夕而零落松栢之茂隆寒不衰是以君子戒於闕黨也【論語闕黨童子將命或問之曰益者歟孔子曰吾見其居於位也見其與先生並行也非求益者也欲速成者也】夫能屈以為伸讓以為得弱以為強鮮不遂矣【鮮息淺翻】夫毁譽者愛惡之原而禍福之機也【譽音余惡烏路翻】孔子曰吾之於人誰毁誰譽【見論語】以聖人之德猶尚如此况庸庸之徒而輕毁譽哉人或毁己當退而求之於身若已有可毁之行則彼言當矣若已無可毁之行則彼言妄矣當則無怨於彼【當丁浪翻】妄則無害於身乂何反報焉諺曰救寒莫如重裘【重直龍翻】止謗莫如自脩斯言信矣【昶之所以戒子姪如此然高貴鄉公之難王沈陷於不忠平吳之役王渾與王濬爭功馬伏波萬里還書以戒兄子固無益於兄子也】

  景初元年【以改歷紀元景初】春正月壬辰山茌縣言黃龍見【山茌前漢曰茌縣後漢及魏晉曰山茌屬泰山郡師古曰茌士疑翻應劭音淄裴松之音仕狸翻見賢遍翻】高堂隆以為魏得土德故其瑞黃龍見宜改正朔易服色以神明其政變民耳目帝從其議三月下詔改元以是月為孟夏四月服色尚黃犧牲用白從地正也【是月春三月也殷為地正以建丑十二月為歲首服色尚黄以土代火之次犧牲用白從殷也】更命太和歷曰景初歷【太和歷注見目録七卷太和元年更工衡翻】 五月己巳帝還洛陽己丑大赦 六月戊申京都地震 己亥以尚書令

  陳矯為司徒令僕射衛臻為司空【晉志曰尚書僕射漢本置一人獻帝建安四年以執金吾榮郃為尚書左僕射僕射分置左右蓋自此始自晉迄于江左省置無恒置二則為左右僕射或不兩置但曰尚書僕射令闕則左為省主若左右並闕則置尚書僕射以主左事】 有司奏以武皇帝為魏太祖文皇帝為魏高祖帝為魏烈祖三祖之廟萬世不毁【沈約曰時羣公有司始奏更定七廟之制曰武皇帝肇建洪基撥亂夷險為魏太祖文皇帝繼天革命應期受禪為魏高祖上集成大命清定華夏興制禮樂為魏烈祖明帝在阼而其下先擬定廟號非禮也諡法有功安民曰烈秉德尊業曰烈】

  孫盛論曰夫諡以表行【行下孟翻】廟以存容未有當年而逆制祖宗未終而豫自尊顯魏之羣司於是乎失正矣【羣司百執事之臣也】

  秋七月丁卯東鄉貞公陳矯卒【諡法不隱無屈曰貞清白守節曰貞】 公孫淵數對國中賓客出惡言【數所角翻】帝欲討之以荆州刺史毋丘儉為幽州刺史【毋丘複姓毋音無】儉上疏曰陛下即位以來未有可書吳蜀恃險未可卒平【卒讀曰猝】聊可以此方無用之士克定遼東【鄭玄曰聊且略之辭】光禄大夫衛臻曰儉所陳皆戰國細術非王者之事也吳頻歲稱兵【稱舉也】寇亂邊境而猶按甲養士未果致討者誠以百姓疲勞故也淵生長海表相承三世【度康淵凡三世長知兩翻】外撫戎夷内脩戰射而儉欲以偏軍長驅朝至夕卷【卷讀曰捲】知其妄矣帝不聽使儉帥諸軍及鮮卑烏桓屯遼東南界【帥讀曰率】璽書徵淵淵遂發兵反逆儉於遼隧【遼隧縣二漢屬遼東郡晉志無其地蓋在遼水東㟁水經注玄菟郡高句麗縣有遼山小遼水所出西南至遼隧縣入于大遼水璽斯氏翻】會天雨十餘日遼水大漲儉與戰不利引軍還右北平淵因自立為燕王改元紹漢置百官遣使假鮮卑單于璽封拜邊民誘呼鮮卑以侵擾北方【誘音酉】 漢張后殂 九月冀兖徐豫大水【冀州統趙鉅鹿安平平原樂陵勃海河間博陵清河中山常山徐州統彭城下邳東海琅邪廣陵臨淮豫州統潁川汝南汝隂梁沛譙魯弋陽安豐】 西平郭夫人有寵於帝【夫人河右大族黄初中以本郡反叛沒入宫】毛后愛弛帝游後園曲宴極樂【曲宴禁中之宴猶言私宴也樂音洛下同】郭夫人請延皇后帝不許因禁左右使不得宣【宣布也露其事也】后知之明日謂帝曰昨日游宴北園樂乎【後園在洛城北隅】帝以左右泄之所殺十餘人庚辰賜后死然猶加諡曰悼【諡法中年早夭曰悼肆行無禮曰悼】癸丑葬愍陵遷其弟曾為散騎常侍 冬十月帝用高堂隆之議營洛陽南委粟山為圓丘【魏氏春秋曰洛陽有委粟山在隂鄉魏時營為圓丘孔穎逹曰委粟山在洛陽南二十里】詔曰昔漢氏之初承秦滅學之後採摭殘缺以備郊祀四百餘年廢無禘禮【摭之石翻禮五年一禘禘其祖之所自出以其祖配之審諦昭穆而祭于太祖也禘所以異於祫者毁廟之主陳於太祖廟與祫同未毁廟之主則各就其廟以祭此其異也春秋吉禘于莊公左傳晉人曰寡君之未禘祀杜預注曰禘祀三年之吉祭也僖八年禘于太廟杜預曰三年大祭之名二者不同禮有禘有大禘以下文觀之則此乃禮記祭法所謂郊禘之禘鄭氏注曰禘郊祖宗謂祭祀以配食也此禘謂祭昊天於圜丘也】曹氏世系出自有虞今祀皇皇帝天於圓丘以始祖虞舜配祭皇皇后地於方丘以舜妃伊氏配【舜妃堯女也堯伊祁氏】祀皇天之神於南郊以武帝配祭皇地之祗於北郊以武宣皇后配 廬江主簿呂習密使人請兵於吳欲開門為内應吳主使衛將軍全琮督前將軍朱桓等赴之既至事露吳軍還【琮徂宗翻還從宣翻又如字】 諸葛恪至丹陽移書四部屬城長吏【四部當作四郡謂吳郡會稽新都鄱陽皆與丹陽隣接山越依阻出沒故令各保其疆界也或曰四部謂東南西北四部都尉也】令各保其疆界明立部伍其從化平民悉令屯居乃内諸將羅兵幽阻【使諸將入扼幽阻之地故謂之内内讀曰納】但繕藩籬不與交鋒俟其穀稼將熟輒縱兵芟刈使無遺種【芟所衘翻種章勇翻】舊穀既盡新穀不收平民屯居略無所入於是山民饑窮漸出降首【降戶江翻首式救翻】恪乃復敕下曰【復扶又翻敕下者出教令約敕其下也】山民去惡從化皆當撫慰徙出外縣不得嫌疑有所拘執臼陽長胡伉得降民周遺【臼陽既置長必以為縣其地當在丹陽郡而今無所考】遺舊惡民【困迫暫出伉】縛送言府恪以伉違教遂斬以徇民聞伉坐執人被戮【伉胡朗翻又去浪翻】知官惟欲出之而已於是老幼相攜而出歲期人數皆如本規【歲期人數見上卷青龍二年】恪自領萬人餘分給諸將吳主嘉其功拜恪威北將軍【威北將軍亦孫氏所創置】封都鄉侯徙屯廬江皖口【皖水自霍山縣東南流三百四十里入大江謂之皖口皖戶版翻】 是歲徙長安鐘簴槖佗銅人承露盤於洛陽盤折【簴音巨佗徒河翻折而設翻】聲聞數十里【聞音問】銅人重不可致留于霸城【霸城即漢京兆霸陵縣故城也】大發銅鑄銅人二號曰翁仲列坐於司馬門外又鑄黄龍鳳皇各一龍高四丈鳳高三丈餘【高古號翻】置内殿前起土山於芳林園西北陬【水經注大夏門内東側際城有景陽山即芳林園之西北陬也裴松之曰芳林園即今華林園齊王芳即位改曰華林園陬將侯翻】使公卿羣僚皆負土樹松竹雜木善草於其上捕山禽雜獸置其中司徒軍議掾董尋上疏諫曰【漢公府無軍議掾此官魏置也掾俞絹翻】臣聞古之直士盡言於國不避死亡故周昌比高祖於桀紂劉輔譬趙后於人婢【周昌注已見前劉輔事見三十一卷漢成帝永始元年】天生忠直雖白刃沸湯往而不顧者誠為時主愛惜天下也【為于偽翻】建安以來野戰死亡或門殫戶盡雖有存者遺孤老弱若今宫室狹小當廣大之猶宜隨時不妨農務况乃作無益之物黃龍鳳皇九龍承露盤此皆聖明之所不興也其功三倍於殿舍陛下既尊羣臣顯以冠冕被以文繡【被皮義翻】載以華輿所以異於小人而使穿方舉土【方穴土為方也漢書所謂方中亦此義】面目垢黑衣冠了鳥【了鳥衣冠摧敝之貌】毁國之光以崇無益甚非謂也孔子曰君使臣以禮臣事君以忠【見論語孔子對魯定公之辭】無忠無禮國何以立臣知言出必死而臣自比於牛之一毛生既無益死亦何損【司馬遷答任安書曰假令僕伏法受誅若九牛亡一毛與螻蟻何異】秉筆流涕心與世辭臣有八子臣死之後累陛下矣【累力瑞翻】將奏沐浴以待命帝曰董尋不畏死邪主者奏收尋有詔勿問高堂隆上疏曰今世之小人好說秦漢之奢靡以蕩聖心【好呼到翻】求取亡國不度之器【不度之器謂長安鍾簴槖佗銅人承露盤也】勞役費損以傷德政非所以興禮樂之和保神明之休也帝不聽隆又上疏曰昔洪水滔天二十二載【隆之此言蓋取鯀九載績用弗成禹治兖州作十有三載乃同合以為二十二載之數載于亥翻】堯舜君臣南面而已今無若時之急而使公卿大夫並與厮徒共供事役聞之四夷非嘉聲也垂之竹帛非令名也今吳蜀二賊非徒白地小虜聚邑之寇【白地謂大幕不生草木多白沙也小虜謂烏桓鮮卑也聚邑之寇謂盜賊竊發屯據鄉邑聚落者】乃潛號稱帝欲與中國爭衡【衡所以稱輕重爭衡者言吳蜀自謂國埶與中國鈞無所輕重也】今若有人來告權禪並脩德政輕省租賦動咨耆賢事遵禮度陛下聞之豈不惕然惡其如此【惡烏路翻】以為難卒討滅【卒讀曰猝】而為國憂乎若使告者曰彼二賊並為無道崇侈無度役其士民重其賦斂【歛力贍翻】下不堪命吁嗟日甚陛下聞之豈不幸彼疲敝而取之不難乎苟如此則可易心而度事義之數亦不遠矣【度徒洛翻義禮也高堂隆之論諫可謂深切著明矣】亡國之主自謂不亡然後至於亡賢聖之君自謂亡然後至於不亡今天下彫敝民無儋石之儲【儋丁濫翻】國無終年之蓄外有強敵六軍暴邊内興土功州郡騷動若有寇警則臣懼版築之士不能投命虜庭矣又將吏奉禄稍見折減【將子亮翻奉扶用翻】方之於昔五分居一諸受休者又絶禀賜【禀筆錦翻給也】不應輸者今皆出半此為官入兼多於舊其所出與參少於昔【參三分也】而度支經用更每不足牛肉小賦前後相繼【此蓋犒饗工徒度支經用不足以給故賦牛肉以供之度徒洛翻】尺而推之凡此諸費必有所在【指言諸費皆在於營繕也】且夫禄賜穀帛人主所以惠養吏民而為之司命者也若今有廢是奪其命矣既得之而又失之此生怨之府也帝覽之謂中書監令曰觀隆此奏使朕懼哉【中書監令典奏事因觀隆奏遂以語之】尚書衛覬上疏曰今議者多好悦耳【覬音冀好呼到翻】其言政治則比陛下於堯舜【治直吏翻】其言征伐則比二虜於貍鼠臣以為不然四海之内分而為三羣士陳力各為其主【為于偽翻】是與六國分治無以為異也當今千里無烟遺民困苦陛下不善留意將遂凋敝難可復振【復扶又翻】武皇帝之時後宫食不過一肉衣不用錦繡茵蓐不緣飾【緣俞絹翻茵蓐之字從草蓋古人用草為之後世鞇字有旁從革者用皮為之也裀褥二字有旁從衣者用帛為之也古樸散而文飾盛又從而加緣飾焉觀書顧命敷席有黼純綴純畫純玄粉純之别則成周之時已然矣純之尹翻緣也】器物無丹漆【古者樸素舜造漆器而羣臣諫者不止况加丹乎】用能平定天下遺福子孫此皆陛下之所覽也當今之務宜君臣上下計校府庫量入為出猶恐不及【量音良】而工役不輟侈靡日崇帑藏日竭【帑徒朗翻藏徂浪翻】昔漢武信神仙之道謂當得雲表之露以餐玉屑故立仙掌以承高露陛下通明每所非笑漢武有求於露而猶尚見非陛下無求於露而空設之不益於好而糜費功夫誠皆聖慮所宜裁制也時有詔録奪士女【録收也】前已嫁為吏民妻者還以配士聽以生口自贖又簡選其有姿首者内之掖庭【姿謂有色者首謂鬒髪者】太子舍人沛國張茂上書諫曰陛下天之子也百姓吏民亦陛下子也今奪彼以與此亦無以異於奪兄之妻妻弟也【妻妻下七細翻】於父母之恩偏矣又詔書得以生口年紀顔色與妻相當者自代故富者則傾家盡產貧者舉假貸貰貴買生口以贖其妻縣官以配士為名而實内之掖庭其醜惡乃出與士得婦者未必喜而失妻者必有憂或窮或愁皆不得志夫君有天下而不得萬姓之懽心者鮮不危殆【鮮息淺翻】且軍師在外數十萬人一日之費非徒千金舉天下之賦以奉此役猶將不給况復有掖庭非員無録之女【非員謂出於員數之外者無録謂宫中録籍無其名者復扶又翻】椒房母后之家賞賜橫與【横戶孟翻】内外交引其費半軍【謂其費與給軍之費相半也】昔漢武帝掘地為海封土為山【掘地為海謂開昆明池封土為山謂作三神山漸臺也】賴是時天下為一莫敢與爭者耳自衰亂以來四五十載【載子亥翻下同】馬不捨鞍士不釋甲強寇在疆圖危魏室陛下不戰戰業業念崇節約而乃奢靡是務中尚方作玩弄之物【晉志少府統中左右三尚方】後園建承露之盤斯誠快耳目之觀然亦足以騁寇讐之心矣【騁丑郢翻】惜乎舍堯舜之節儉而為漢武帝之侈事臣竊為陛下不取也帝不聽【舍讀曰捨竊為于偽翻】高堂隆疾篤口占上疏曰【疾篤不能自書故口占而使人書之】曾子有言曰人之將死其言也善【見論語】臣寢疾有增無損常恐奄忽忠欵不昭臣之丹誠願陛下少垂省覽【省悉景翻】臣觀三代之有天下聖賢相承歷數百載尺土莫非其有一民莫非其臣然癸辛之徒縱心極欲皇天震怒宗國為墟紂梟白旗【武王斬紂首懸之太白之旗梟堅堯翻】桀放鳴條【商湯破桀於鳴條遂放之于南巢孔安國曰鳴條地在安邑之西】天子之尊湯武有之豈伊異人皆明王之胄也黃初之際天兆其戒異類之鳥育長燕巢口爪胸赤此魏室之大異也【晉書五行志黃初元年未央宫中有燕生鷹口爪俱赤長知兩翻】宜防鷹揚之臣於蕭牆之内【司馬氏之事隆固逆知之矣】可選諸王使君國典兵往往棊時鎮撫皇畿翼亮帝室夫皇天無親惟德是輔【書蔡仲之命之辭】民詠德政則延期過歷下有怨歎則輟録授能【録圖録也】由此觀之天下乃天下之天下非獨陛下之天下也帝手詔深慰勞之未幾而卒【勞力到翻幾居豈翻】陳夀評曰高堂隆學業脩明志存匡君因變陳戒發於懇誠忠矣哉及至必改正朔俾魏祖虞所謂意過其通者歟【謂是年黄龍見之議也意過其通謂意料之說執之甚堅反過其學之所通習者也】

  帝深疾浮華之士詔吏部尚書盧毓曰【毓余六翻】選舉莫取有名名如畫地作餅不可啖也【啖徒覽翻噍也食也又徒濫翻】毓對曰名不足以致異人而可以得常士常士畏教慕善然後有名非所當疾也愚臣既不足以識異人又主者正以循名案常為職但當有以驗其後耳古者敷奏以言明試以功【言唐虞之治也】今考績之法廢而以毁譽相進退故真偽渾雜虚實相蒙帝納其言【渾胡本翻】詔散騎常侍劉邵作考課法邵作都官考課法七十二條又作說略一篇【說略者說考課之大畧也】詔下百官議【下遐稼翻】司隸校尉崔林曰案周官考課其文備矣【周冢宰總百官歲終則令百官府各正其治受其會聽其政事而詔王廢置三歲則大計羣吏之治而誅賞之其詳見於周禮】自康王以下遂以陵夷此即考課之法存乎其人也及漢之季其失豈在乎佐吏之職不密哉方今軍旅或猥或卒【猥積也卒倉猝也讀曰猝】增減無常固難一矣且萬目不張舉其綱【以綱為譬也】衆毛不整振其領【以裘為譬也】皋陶仕虞伊尹臣殷不仁者遠【用論語子夏答樊遲之言陶音遥】若大臣能任其職式是百辟【詩烝民曰王命仲山甫式是百辟注云汝施行法度於是百君】則孰敢不肅烏在考課哉黃門侍郎杜恕曰明試以功三載考績誠帝王之盛制也然歷六代而考績之法不著關七聖而課試之文不垂【六代唐虞夏商周漢七聖堯舜禹湯文武周公關通也】臣誠以為其法可粗依其詳難備舉故也【粗坐五翻】語曰世有亂人而無亂法若使法可專任則唐虞可不須稷契之佐殷周無貴伊呂之輔矣【契息列翻】今奏考功者陳周漢之云為綴京房之本旨【漢京房有考功課吏法】可謂明為課之要矣於以崇揖讓之風興濟濟之治臣以為未盡善也【濟子禮翻治直吏翻】其欲使州郡考士必由四科【四科即漢左雄所上黄瓊所增者也見五十二卷順帝漢安二年】皆有事效然後察舉試辟公府為親民長吏【長知兩翻】轉以功次補郡守者或就增秩賜爵此最考課之急務也臣以為便當顯其身用其言使具為課州郡之法法具施行立必信之賞施必行之罰至於公卿及内職大臣亦當俱以其職考課之古之三公坐而論道【周官考工記曰坐而論道謂之三公】内職大臣納言補闕無善不紀無過不舉且天下至大萬機至衆誠非一明所能徧照故君為元首臣作股肱明其一體相須而成也是以古人稱廊廟之材非一木之支帝王之業非一士之略【師古曰此語出於慎子班固引以贊婁敬叔孫通】由是言之焉有大臣守職辦課可以致雍熙者哉【焉於䖍翻】誠使容身保位無放退之辜而盡節在公抱見疑之埶公義不脩而私議成俗雖仲尼為課猶不能盡一才又况於世俗之人乎司空掾北地傅嘏曰夫建官均職清理民物所以立本也循名責實糾勵成規所以治末也【治直之翻】本綱未舉而造制末程【綱維紘繩綱總也舉綱則衆目張矣言所繫者大也十髪為程一程為分言其細也又曰程品式也】國略不崇而考課是先【國略國經也先必薦翻】懼不足以料賢愚之分精幽明之理也【料音聊】議久之不决事竟不行臣光曰為治之要莫先於用人【治直吏翻】而知人之道聖賢所難也【書皋陶曰在知人在安民禹曰吁咸若時惟帝其難之】是故求之於毁譽則愛憎競進而善惡渾殽【譽音余渾戶本翻】考之於功狀則巧詐橫生而真偽相冒要之其本在於至公至明而已矣為人上者至公至明則羣下之能否焯然形於目中無所復逃矣【焯職略翻明也復扶又翻】苟為不公不明則考課之法適足為曲私欺罔之資也何以言之公明者心也功狀者迹也己之心不能治【治直之翻下同】而以考人之迹不亦難乎為人上者誠能不以親疎貴賤異其心喜怒好惡亂其志【好呼到翻惡烏路翻】欲知治經之士則視其記覽博洽【博廣也大也通也洽徧也】講論精通斯為善治經矣欲知治獄之士則視其曲盡情偽無所寃抑斯為善治獄矣欲知治財之士則視其倉庫盈實百姓富給斯為善治財矣欲知治兵之士則視其戰勝攻取敵人畏服斯為善治兵矣至於百官莫不皆然雖詢謀於人而决之在己雖考求於迹而察之在心研覈其實而斟酌其宜至精至微不可以口述不可以書傳也安得豫為之法而悉委有司哉【温公之論善矣然必英明之君然後能行之自漢以下循名責實莫孝宣若也宣帝之政非由師傅之諭教公輔之啟沃也公所謂不可以口述不可以書傳其萬世之名言也歟】或者親貴雖不能而任職疎賤雖賢才而見遺所喜所好者敗官而不去所怒所惡者有功而不録【喜許記翻好呼到翻敗補邁翻惡烏路翻】詢謀於人則毁譽相半而不能决考求其迹則文具實亡而不能察雖復為之善法【復扶又翻】繁其條目謹其簿書安能得其真哉或曰人君之治【治直吏翻】大者天下小者一國内外之官以千萬數考察黜陟安得不委有司而獨任其事哉曰非謂其然也凡為人上者不特人君而已太守居一郡之上刺史居一州之上九卿居屬官之上三公居百執事之上皆用此道以考察黜陟在下之人為人君者亦用此道以考察黜陟公卿太守奚煩勞之有哉【古人有言曰舉一綱衆目張又曰正其本萬事理此之謂也而所謂本者豈易言哉】或曰考績之法唐虞所為京房劉邵述而脩之耳烏可廢哉曰唐虞之官其居位也久其受任也專其立法也寛其責成也遠是故鯀之治水九載績用弗成然後治其罪【事見尚書治其罪謂殛鯀於羽山也治直之翻下同】禹之治水九州攸同四隩既宅然後賞其功【隩於六翻事亦見尚書賞其功謂錫禹以玄珪也】非若京房劉邵之法校其米鹽之課責其旦夕之效也事固有名同而實異者不可不察也考績非可行於唐虞而不可行於漢魏由京房劉邵不得其本而奔趨其末故也【趨七喻翻】

  初右僕射衛臻典選舉中護軍蔣濟遺臻書曰【蔣濟已自中護軍遷護軍將軍此復書中護軍蓋先時事也遺于季翻】漢主遇亡虜為上將【謂韓信】周武拔漁父為太師【謂呂望】布衣厮養【厮音斯養羊尚翻】可登王公何必守文試而後用臻曰不然子欲同牧野於成康喻斷蛇於文景【謂草創之規略不可用於承平之時也】好不經之舉【好呼到翻經常也】開拔奇之津【津江河濟度之要故以為喻】將使天下馳騁而起矣盧毓論人及選舉皆先性行而後言才【先悉薦翻行下孟翻】黃門郎馮翊李豐嘗以問毓毓曰才所以為善也故大才成大善小才成小善今稱之有才而不能為善是才不中器也豐服其言【中竹仲翻】

  資治通鑑卷七十三


    


 


 



 

 
  







 


  
  
 
 
 


  

 















	
	









































 
  



















 





 












  
  
  

 





