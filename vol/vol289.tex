\section{資治通鑑卷二百八十九}
宋 司馬光 撰

胡三省 音註

後漢紀四|{
	上章閹茂一年}


隱皇帝下

乾祐三年春正月丁未加鳳翔節度使趙暉兼侍中密州刺史王萬敢請益兵以攻唐|{
	王萬敢去年已殘荻水鎮今請益兵攻之}
詔以前沂州刺史郭瓊為東路行營都部署帥禁軍及齊州兵赴之|{
	因王萬敢請兵使郭瓊將以赴之道過青州因以易置劉銖帥讀曰率}
郭威請勒兵北臨契丹之境詔止之 丙寅遣使詣河中鳳翔收瘞戰死及餓殍遺骸時有僧已聚二十萬矣|{
	瘞于計翻殍被表翻已聚者二十萬史言其未聚者尚多大兵攻圍積久其禍如此}
唐主聞漢兵盡平三叛始罷李金全北面行營招討使|{
	唐命李金全見二百八十七卷元年}
唐清淮節度使劉彦貞多斂民財以賂權貴權貴爭譽之在壽州積年|{
	譽音余晉開運元年唐徙劉彦貞鎮濠州劉崇俊鎮壽州漢乾祐元年清淮節度使劉彦貞副李金全北伐未知彦貞以何年徙鎮壽州}
恐被代欲以警急自固妄奏稱漢兵將大舉南伐|{
	被皮義翻}
二月唐主以東都留守燕王弘冀為潤宣二州大都督鎮潤州寧國節度使周宗為東都留守|{
	以漢兵大舉弘冀年少恐不能調用扞禦周宗為唐祖佐命宿望也故徙鎮揚州}
朝廷欲移易藩鎮因其請赴嘉慶節上壽|{
	五代會要帝以三月九日為嘉慶節洪邁隨筆日唐穆宗即位之初年詔曰七月六日是朕載誕之辰其日百寮命婦宜於光順門進名參賀朕於門内與百寮相見明日又勑受賀儀宜停先是左丞韋綬奏行之宰臣以為古無降誕受賀之禮奏罷之然次年復行賀禮誕節之制始于明皇令天下宴集休假三日受賀之事蓋自長慶至今用之也上時掌翻}
許之 甲申郭威行北邊還|{
	去年冬十月郭威北征今還行下孟翻還從宣翻又如字}
福州人或詣建州告唐永安留後查文徽云吳越兵已弃城去請文徽為帥|{
	查鈕如翻帥所類翻}
文徽信之遣劒州刺史陳誨將水軍下閩江|{
	薛史曰李景保大三年以延平為劒州析建州之劒浦汀州之沙縣隸焉劒溪上接建溪下逹福唐亦謂之閩江下戶嫁翻將即亮翻}
文徽自以步騎繼之會大雨水漲誨一夕行七百里至城下敗福州兵|{
	敗補邁翻}
執其將馬先進等庚寅文徽至福州吳越知威武軍吳程詐遣數百人出迎|{
	吳越未命吳程為威武節度使先令知威武軍事}
誨曰閩人多詐未可信也宜立寨徐圖文徽曰疑則變生不若乘機據其城因引兵徑進誨整衆鳴鼔止於江湄|{
	湄旻悲翻水草之交曰湄詩巧言居河之麋註云本作湄水草交也}
文徽不為備程勒兵出擊之唐兵大敗文徽墜馬為福人所執士卒死者萬人誨全軍歸劒州程送文徽於錢唐吳越王弘俶獻於五廟而釋之|{
	俶昌六翻吳越用諸侯之制立五廟}
丁亥汝州奏防禦使劉審交卒吏民詣闕上書以審交有仁政乞留葬汝州得奉事其丘壟詔許之|{
	上時掌翻}
州人相與聚哭而葬之為立祠歲時享之|{
	為于偽翻}
太師馮道曰吾嘗為劉君僚佐|{
	按歐史劉審交燕人劉守光之僭號以審交為兵部尚書馮道事守光為參軍嘗為僚佐必是時也}
觀其為政無以踰人非能減其租税除其繇役也|{
	繇讀曰徭}
但推公亷慈愛之心以行之耳此亦衆人所能為但它人不為而劉君獨為之故汝人愛之如此使天下二千石皆效其所為何患得民不如劉君哉|{
	五代之諸州防禦使曾未足以當漢郡守二千石後人特以專城分守故稱之}
甲午吳越丞相昭化節度使趙匡贊為左驍衛上將軍|{
	趙匡贊自長安入朝見二百八十七卷高祖乾祐元年}
三月丙午嘉慶節鄴都留守高行周天平節度使慕容彦超泰寧節度使符彦卿昭義節度使常思安遠節度使楊信安國節度使薛懷讓成德節度使武行德彰德節度使郭謹保大留後王饒皆入朝|{
	許之赴嘉慶節上夀故皆入朝}
甲寅詔營寢廟於高祖長陵世祖原陵以時致祭有司以費多寢其事以至國亡二陵竟不霑一奠|{
	是年十一月郭威入大梁十二月將士扶立以時致祭之詔有司既停寢不行六七月之間宜乎不霑一奠也}
壬戌徙高行周為天平節度使符彦卿為平盧節度使甲子徙慕容彦超為泰寧節度使 永安節度使折從阮舉族入朝|{
	折從阮自府州入朝}
夏四月戊辰朔徙薛懷讓為匡國節度使庚午徙折從阮為武勝節度使|{
	按五代會要周廣順二年三月始改鄧州威勝軍為武勝軍避周太祖名也史以後來所改軍名而書之耳}
壬申徙楊信為保大節度使徙鎮國節度使劉詞為安國節度使永清節度使王令温為安遠節度使李守貞之亂王饒潛與之通|{
	王饒潛以鄜州與河中通}
守貞平衆謂饒必居散地|{
	冗散之官為散地散悉但翻}
及入朝厚結史弘肇遷護國節度使聞者駭之|{
	駭其不惟免罪又得大鎮}
楊邠求解樞密使帝遺中使諭止之宣徽北院使吳䖍裕在旁曰|{
	吳䖍裕時蓋在楊邠旁}
樞密重地難以久居當使後來者迭為之相公辭之是也帝聞之不悦辛巳以䖍裕為鄭州防禦使 朝廷以契丹近入寇横行河北諸藩鎮各自守無捍禦之者|{
	事見上卷上年十月}
議以郭威鎮鄴都使督諸將以備契丹史弘肇欲威仍領樞密使蘇逢吉以為故事無之|{
	言故事無帶樞密使出鎮者}
弘肇曰領樞密使則可以便宜從事諸軍畏服號令行矣帝卒從弘肇議|{
	卒子恤翻}
弘肇怨逢吉異議逢吉曰以内制外順也今反以外制内其可乎壬午制以威為鄴都留守天雄節度使樞密使如故仍詔河北兵甲錢糓但見郭威文書立皆稟應明日朝貴會飲於竇貞固之第弘肇舉大觴屬威|{
	屬之欲翻}
厲聲曰昨日廷議一何同異今日為弟飲之|{
	史弘肇呼郭威為弟為于偽翻}
逢吉楊邠亦舉觴曰是國家之事何足介意弘肇又厲聲曰安定國家在長槍大劒安用毛錐王章曰無毛錐則財賦何從可出|{
	毛錐謂筆也以東毛為筆其形如錐也王章為三司使實掌財賦故云然}
自是將相始有隙 癸未罷永安軍|{
	復以府州隸河東也}
壬辰以左監門衛將軍郭榮為貴州刺史天雄牙内都指揮使|{
	貴州時屬南漢宋白曰貴州故西甌落越之地秦雖立桂林郡仍有甌駱之名漢武帝改桂林為鬰林郡梁武帝以鬰林郡為桂州後割桂州之鬰林寧浦立定州尋改為南定州隋改南定州為尹州唐改貴州漢以郭榮遥領刺史而其職則天雄牙將也}
榮本姓柴|{
	考異曰世宗實録曰太祖皇帝之長子也母曰聖穆皇后柴氏以唐天祐十八年九月二十四日丙午生于邢臺之别墅薛史世宗紀云太祖之養子蓋聖穆皇后之姪也本姓柴氏父守禮太子少保致仕帝未童冠因侍聖穆皇后在太祖左右時太祖無子乃養為己子按今舉世皆知世宗為柴氏子謂之柴世宗而世宗實録云太祖長子誣亦甚矣}
父守禮郭威之妻兄也威未有子時養以為子|{
	郭榮始見於此}
五月己亥以府州蕃漢馬步都指揮使折德扆為本州團練使|{
	前此置永安軍於府州以寵折從阮也今從阮移鎮其子德扆守府州資序未至而府州被邊一城之地耳故降為團練使其後復以為節鎮以寵折民}
德扆從阮之子也庚子郭威辭行言於帝曰太后從先帝久多歷天下

事陛下富於春秋有事宜稟其教而行之親近忠直放遠讒邪|{
	近其靳翻遠于願翻}
善惡之間所宜明審蘇逢吉楊邠史弘肇皆先帝舊臣盡忠徇國願陛下推心任之必無敗矣|{
	郭威言及此蓋已知帝之信近習而間勲舊也}
至於疆場之事臣願竭其愚駑|{
	駑音奴}
庶不負驅策帝斂容謝之威至鄴都以河北困弊戒邊將謹守疆場嚴守備無得出侵掠契丹入寇則堅壁清野以待之|{
	兵法所謂先為不可勝以待敵之可勝也}
辛丑敕防禦團練使自非軍期無得專奏事皆先申觀察使斟酌以聞|{
	言軍期事須朝廷應副則不及聞於廉使許得專達朝廷如尋常公事須先申本管斟酌以聞}
丙午以皇弟山南西道節度使承勲為開封尹加兼中書令實未出閤|{
	年尚幼且有羸疾也山南西道時為蜀境}
平盧節度使劉銖貪虐恣横|{
	横戶孟翻}
朝廷欲徵之恐其拒命因沂密用兵於唐遣沂州刺史郭瓊將兵屯青州|{
	歐史作郭淮攻南唐還以兵屯青州}
銖不自安置酒召瓊伏兵幕下欲害之瓊知其謀悉屏左右從容如會了無懼色|{
	屛必郢翻又卑正翻從千容翻}
銖不敢發瓊因諭以禍福銖感服詔至即行庚戌銖入朝辛亥以瓊為潁州團練使 癸丑王章置酒會諸朝貴酒酣為手勢令|{
	會飲而行酒令以佐歡唐末之俗也類說曰亞其虎膺謂手掌曲其松根謂指節以蹲䲭間虎膺之下蹲鴟大指也以鉤戟差玉柱之旁鉤戟頭指玉柱中指也濳虬闊玉柱三分潛虬無名指也奇兵闊濳虬一寸奇兵小指也死其三洛謂嚲其腕也生其五峯五峯通呼五指也謂之招手令蓋亦手勢令之類也乎哉}
史弘肇不閑其事|{
	言不素習其事}
客省使閻晉卿坐次弘肇屢教之蘇逢吉戲之曰旁有姓閻人何憂罰爵|{
	壺射之事不勝者罰爵自古有之行令則末世之為耳}
弘肇妻閻氏本酒家倡也|{
	倡音昌酒家倡善為酒令}
意逢吉譏之大怒以醜語詬逢吉|{
	詬古侯翻又許侯翻}
逢吉不應弘肇欲毆之逢吉起去弘肇索劒欲追之|{
	毆烏口翻索山客翻}
楊邠泣止之曰蘇公宰相公若殺之置天子何地願孰思之|{
	孰與熟同}
弘肇即上馬去邠與之聯鑣送至其第而還|{
	上時掌翻鑣悲驕翻馬衘也還從宣翻又如字}
於是將相如水火矣帝使宣徽使王峻置酒和解之不能得逢吉欲求出鎮以避之既而中止曰吾去朝廷止煩史公一處分吾虀粉矣王章亦忽忽不樂|{
	處昌呂翻分扶問翻韲牋西翻樂音洛}
欲求外官楊史固止之閏月宫中數有怪癸巳大風發屋拔木吹鄭門扉起

十餘步而落震死者六七人水深平地尺餘|{
	數所角翻鄭門大梁城西面南來第一門也梁改為開明門晉改為金義門周改為迎秋門汴人只以舊門名呼之深式禁翻}
帝召司天監趙延乂問以禳祈之術對曰臣之業在天文時日禳祈非所習也然玉者欲弭災異莫如修德延乂歸帝遣中使問如何為修德延乂對請讀貞觀政要而灋之|{
	觀古玩翻}
六月河決鄭州|{
	歐史曰六月癸卯河決原武按原武縣屬鄭州九域志云原武縣在鄭州之北六十里}
馬希萼既敗歸|{
	僕射洲之敗也事見上卷上年八月}
乃以書誘辰溆州及梅山蠻|{
	誘音酉溆音叙宋白日潭州西有梅山洞為草寇之窟穴}
欲與共擊湖南蠻素聞長沙帑藏之富|{
	帑它朗翻藏徂浪翻}
大喜爭出兵赴之遂攻益陽|{
	益陽縣屬潭州漢古縣城在唐縣東八十里九域志益陽在潭州西北一百八十二里宋白曰以其地在益水之陽故名其城魯肅所築}
楚王希廣遣指揮使陳璠拒之戰於淹溪璠敗死|{
	璠音翻}
秋七月唐歸馬先進等於吳越以易查文徽|{
	馬先進等被擒見上二月查文徽亦以是月為吳越所禽}
馬希萼又遣羣蠻攻迪田八月戊戌破之殺其鎮將張延嗣楚王希廣遣指揮使黄處超救之處超敗死|{
	處昌呂翻}
潭人震恐復遣牙内指揮使崔洪璉將兵七千屯玉潭|{
	九域志潭州湘鄉縣有玉潭鎮在潭州西復扶又翻}
庚子蜀主立其弟仁毅為夔王仁贄為雅王仁裕為彭王仁操為嘉王己酉立子玄喆為秦王|{
	喆音哲}
玄珏為褒王 晉李太后在建州|{
	契丹遷晉主及其家于建州見上卷上年三月}
臥病無醫藥惟與晉主仰天號泣|{
	號戶刀翻}
戟手罵杜重威李守貞曰吾死不置汝|{
	以其降契丹而亡晉也事見二百八十六卷開運二年}
戊午卒周顯德中有自契丹來者云晉主及馮后尚無恙其從者亡歸及物故則過半矣|{
	恙余亮翻從才用翻過音戈}
馬希萼表請别置進奏務於京師九月辛巳詔以湖南已有進奏務不許亦賜楚王希廣詔勸以敦睦 馬希萼以朝廷意佑楚王希廣怒遣使稱藩於唐乞師攻楚唐加希萼同平章事以鄂州今年租税賜之命楚州刺史何敬洙將兵助希萼冬十月丙午希廣遣使上表告急言荆南嶺南江南連謀欲分湖南之地|{
	荆南高氏嶺南劉氏江南李氏}
乞發兵屯灃州以扼江南荆南援朗州之路|{
	江南遣兵援朗道出岳州岳州西至灃州三百餘里荆南遣兵援朗徑度江南趨澧州亦三百里自澧州東南至朗州三百五十九里}
丁未以吳越王弘俶為諸道兵馬元帥 楚王希廣以朗州與山蠻入寇諸將屢敗憂形於色劉彦瑫言於希廣曰朗州兵不滿萬馬不滿千都府精兵十萬|{
	朗桂以潭州為都府}
何憂不勝願假臣兵萬餘人戰艦百五十艘|{
	艦戶黯翻艘蘇遭翻}
徑入朗州縛取希萼以解大王之憂王悦以彦瑫為戰棹都指揮使朗州行營都統彦瑫入朗州境|{
	九域志潭州北至朗州界二百一十七里}
父老爭以牛酒犒軍曰百姓不願從亂望都府之兵久矣彦瑫厚賞之戰艦過則運竹木以斷其後|{
	斷音短}
是日馬希萼遣朗兵及蠻兵六千戰艦百艘逆戰於湄州|{
	歐史作湄州}
彦瑫乘風縱火以焚其艦頃之風囘反自焚彦瑫還走江路已斷|{
	自斷歸路則當死戰還走何為}
士卒戰及溺死者數千人 |{
	考異曰湖湘故事彦瑫敗在九月十三日今從十國紀年}
希廣聞之涕泣不知所為希廣平日罕頒賜至是大出金帛以取悦於士卒或告天策左司馬希崇流言惑衆反狀已明請殺之希廣曰吾自害其弟何以見先王於地下|{
	希崇與希萼通謀者也當斷不斷反受其亂希廣之亡宜矣}
馬軍指揮使張暉將兵自它道擊朗州至龍陽|{
	龍陽縣屬朗州隋所置也取龍陽州以名縣九域志在朗州東南八十五里宋白曰龍陽故漢索縣地吴分其地立龍陽縣}
聞彦瑫敗退屯益陽希萼又遣指揮使朱進忠等將兵三千急攻益陽張暉紿其衆曰我以麾下出賊後汝輩留城中待我相與合勢擊之既出遂自竹頭市遁歸長沙朗兵知城中無主急擊之士卒九千餘人皆死吳越王弘俶歸查文徽於唐文徽得瘖疾以工部尚

書致仕|{
	史言唐不能正查文徽敗軍之罪瘖於今翻}
十一月甲子朔日有食之 蜀太師中書令宋忠武王趙廷隱卒 楚王希廣遣其僚属孟駢說馬希萼曰公忘父兄之讐北面事唐|{
	自馬殷以來與楊徐世為仇讐說式芮翻}
何異袁譚求救於曹公邪|{
	事見六十四卷漢獻帝建安八年}
希萼將斬之駢曰古者兵交使在其閒|{
	春秋左氏傳之言使疏吏翻}
駢若愛死安肯此來駢之言非私於潭人實為公謀也|{
	為于偽翻}
乃釋之使還報曰大義絶矣非地下不相見也朱進忠請希萼自將兵取潭州辛未希萼留其子光贊守朗州悉發境内之兵趣長沙|{
	趣七喻翻考異曰湖湘故事希萼以十月二卜一日直往湖南今從十國紀年}
自稱順天王 詔侍衛步軍都指揮使寧江節度使王殷將兵屯澶州以備契丹|{
	侍衛親軍都指揮使之下又有侍衛馬軍步軍二都指揮此皆梁唐所置寧江軍夔州時屬蜀王殷遥領也}
殷瀛州人也 朝廷議發兵以安遠節度使王令温為都部署以救潭州會内難作不果|{
	内難謂殺楊邠等以召郭威之禍難乃旦翻}
帝自即位以來樞密使右僕射同平章事楊邠總機政樞密使兼侍中郭威主征伐歸德節度使侍衛親軍都指揮使兼中書令史弘肇典宿衛三司使同平章事王章掌財賦邠頗公忠退朝門無私謁雖不却四方饋遺有餘輒獻之|{
	遺唯季翻}
弘肇督察京城道不拾遺是時承契丹蕩覆之餘|{
	契丹入汴中原蕩覆契丹北歸漢承其後}
公私困竭章捃摭遺利吝於出納以實府庫屬三叛連衡|{
	捃居運翻摭之石翻屬之欲翻三叛謂李守貞王景崇趙思綰衡讀曰横}
宿兵累年而供饋不乏及事平賜予之外尚有餘積|{
	予讀曰與積子賜翻}
以是國家粗安|{
	粗坐五翻}
章聚斂刻急|{
	斂力贍翻}
舊制田税每斛更輸二升謂之雀鼠耗章始令更輸二斗謂之省耗|{
	按唐明宗天成元年四月赦文應納夏秋税子先有省耗每斗一升今後祗納正税數不量省耗如此則天成已前已有省耗每斛更輸一斗天成罷輸之後至漢興王章復令輸省耗而又倍舊數取之也}
舊錢出入皆以八十為陌章始令入者八十出者七十七謂之省陌|{
	沈括曰今之數錢百錢謂之陌者借陌字用之其實只是百字如什與伍耳唐自皇甫鏄為墊錢法至昭宗時乃定八十為陌}
有犯鹽礬酒麴之禁者錙銖涓滴罪皆死|{
	鹽禁之設久矣酒之為禁或罷或榷歷代不常自唐中世始申榷酒之禁及其末也又禁造麴至於礬禁新舊唐書食貨志皆未著言其事是必起於五代之初本草圖經曰礬石生河西山谷及隴西武都石門今白礬則晉州慈州拱州無為軍緑礬則隰州温泉縣池州銅陵縣並煎礬處處出焉初生皆石也採得碎之煎煉乃成礬凡有五種其色各異謂白礬綠礬黄礬黑礬絳礬也自岐伯至陶隱居之書皆言之}
由是百姓愁怨章尤不喜文臣嘗曰此輩授之握筭不知縱横|{
	喜許記翻縱子容翻}
何益於用俸禄皆以不堪資軍者給之吏已高其估章更增之|{
	估價也}
帝左右嬖倖浸用事|{
	嬖卑義翻又博計翻}
太后親戚亦干預朝政|{
	朝直遙翻}
邠等屢裁抑之太后有故人子求補軍職弘肇怒而斬之武德使李業太后之弟也|{
	太后昆弟七人業最幼故尤憐之}
高祖使掌内帑|{
	帑底朗翻}
帝即位尤蒙寵任會宣徽使闕業意欲之|{
	吳䖍裕出鄭州闕宣徽北院使}
帝及太后亦諷執政邠弘肇以為内使遷補有次不可以外戚超居乃止内客省使閻晉卿次當為宣徽使久而不補樞密承旨聶文進飛龍使後匡贊翰林茶酒使郭允明皆有寵於帝久不遷官共怨執政|{
	姓譜後姓望出東海開封}
文進并州人也劉銖罷青州歸久奉朝請未除官常戟手於執政|{
	是年夏五月劉銖自青州召歸戟手者戟其手而詬怨之}
帝初除三年喪聽樂賜伶人錦袍玉帶伶人詣弘肇謝弘肇怒曰士卒守邊苦戰猶未有以賜之汝曹何功而得此皆奪以還官帝欲立所幸耿夫人為后邠以為太速夫人卒帝欲以后禮葬之邠復以為不可|{
	復扶又翻}
帝年益壯厭為大臣所制邠弘肇嘗議事於帝前帝曰審圖之勿令人有言邠曰陛下但禁聲|{
	禁聲謂禁口勿言使不出聲也}
有臣等在帝積不能平左右因乘間譛之於帝|{
	間古莧翻}
云邠等專恣終當為亂帝信之嘗夜聞作坊鍛聲|{
	作坊造兵甲之所作坊使領之鍛都玩翻鍛鐵以為兵甲}
疑有急兵達旦不寐司空同平章事蘇逢吉既與弘肇有隙知李業等怨弘肇屢以言激之帝遂與業文進匡贊允明謀誅邠等議既定入白太后太后曰兹事何可輕發更宜與宰相議之業時在㫄曰先帝嘗言朝廷大事不可謀及書生懦怯誤人太后復以為言|{
	史言殺邠等非太后之意復扶又翻}
帝忿曰國家之事非閨門所知拂衣而出乙亥業等以其謀告閻晉卿晉卿恐事不成詣弘肇第欲告之弘肇以它故辭不見|{
	使閻晉卿得見史弘肇則李業等之死不待郭威之入也天方授郭威故史弘肇等先死以除其偪豈特人事哉}
丙子旦邠等入朝有甲士數十自廣政殿出殺邠弘肇章於東廡下|{
	廡罔甫翻按薛史晉天福四年二月辛卯改東京玉華殿為永福殿周顯德四年新修永福殿改為廣政殿此蓋以後來殿名書之}
文進亟召宰相朝臣班于崇元殿宣云邠等謀反已伏誅與卿等同慶又召諸軍將校至萬歲殿庭|{
	五代會要梁開平元年改汴京正衙殿為崇元殿東殿為玄德殿萬歲堂為萬歲殿晉天福二年八月改玄德殿為廣政殿將即亮翻校戶教翻}
帝親諭之且曰邠等以穉子視朕朕今始得為汝主汝輩免横憂矣皆拜謝而退|{
	穉直利翻横戶孟翻}
又召前節度使刺史等升殿諭之分遣使者帥騎收捕邠等親戚黨與傔從盡殺之|{
	帥讀曰率騎奇寄翻傔苦念翻從才用翻}
弘肇待侍衛步軍都指揮使王殷尤厚邠等死帝遣供奉官孟業齎密詔詣澶州及鄴都令鎮寧節度使李洪義殺殷又令鄴都行營馬軍都指揮使郭崇威步軍都指揮使真定曹威殺郭威及監軍宣徽使王峻洪義太后之弟也又急詔徵天平節度使高行周平盧節度使符彦卿永興節度使郭從義泰寧節度使慕容彦超匡國節度使薛懷讓鄭州防禦使吳䖍裕陳州刺史李穀入朝|{
	急徵諸帥欲其以從兵衛宫闕李穀一刺史耳而亦預徵入朝之數必其智略聞於時也}
以蘇逢吉權知樞密院事前平盧節度使劉銖權知開封府侍衛馬軍都指揮使李洪建權判侍衛司事内侍省使閻晉卿權侍衛馬軍都指揮使|{
	内侍省當作内客省}
洪建業之兄也時中外人情憂駭|{
	駭其變起於倉猝而憂禍至之無日也}
蘇逢吉雖惡弘肇|{
	以弘肇詬怒逢吉欲殺之故惡之也惡烏路翻}
而不預李業等謀聞變驚愕私謂人曰事太怱怱|{
	怱怱急遽不審諦之意}
主上儻以一言見問不至於此業等命劉銖誅郭威王峻之家銖極其慘毒嬰孺無免者|{
	嬰嬰兒鄭玄曰嬰猶鷖彌也孺乳子飲乳之子也}
命李洪建誅王殷之家洪建但使人守視仍飲食之|{
	飲於禁翻食祥吏翻}
丁丑使者至澶州李洪義畏懦慮王殷已知其事不敢發乃引孟業見殷殷囚業遣副使陳光穗以密詔示郭威威詔樞密使魏仁浦示以詔書曰柰何仁浦曰公國之大臣功名素著加之握彊兵據重鎮一旦為羣小所搆禍出非意此非辭說之所能解|{
	解佳買翻釋也說也}
時事如此不可坐而待之|{
	勸之舉兵也歐史曰威匿詔書召樞密院吏魏仁浦謀於臥内仁浦勸威反倒用留守印更為詔書詔威誅諸將校以激怒之將校皆憤然效用竊意歐史必有所本通鑑所書必本於周史周臣為其君諱復為魏仁浦緣飾耳}
威乃召郭崇威曹威及諸將告以楊邠等寃死及有密詔之狀且曰吾與諸公披荆棘從先帝取天下受託孤之任竭力以衛國家今諸公已死吾何心獨生君輩當奉行詔書取吾首以報天子庶不相累|{
	累力瑞翻}
郭崇威等皆泣曰天子幼沖此必左右羣小所為若使此輩得志國家其得安乎崇威願從公入朝自訴盪滌鼠輩以清朝廷不可為單使所殺受千載惡名翰林天文趙修己謂郭威曰公徒死何益不若順衆心擁兵而南此天啟也|{
	趙修己諫李守貞而勸郭威自信其術也}
郭威乃留其養子榮鎮鄴都命郭崇威將騎兵前驅戊寅自將大軍繼之|{
	將即亮翻}
慕容彦超方食得詔捨匕筯入朝帝悉以軍事委之|{
	九域志兖州至大梁六百里慕容彦超三日而至自以於先帝同產之親急於赴闕而不知其才智之不足以濟也}
己卯吳䖍裕入朝|{
	九域志鄭州至大梁一百四十里}
帝聞郭威舉兵南向議發兵拒之前開封尹侯益曰鄴都戍兵家屬皆在京師官軍不可輕出不若閉城以挫其鋒使其母妻登城招之可不戰而下也慕容彦超曰侯益衰老為懦夫計耳帝乃遣益及閻晉卿吳䖍裕前保大節度使張彦超將禁軍趣澶州|{
	趣七喻翻}
是日郭威已至澶州|{
	魏州南至澶州一百五十里兩日而至欲掩漢之未備}
李洪義納之王殷迎謁慟哭以所部兵從郭威涉河帝遣内養鸗脱覘郭威威獲之|{
	鸗力鍾翻又盧紅翻歐史作鸗亦音龍 考異曰隱帝實録丁丑孟業至澶州戊寅鄴兵至河上己卯吳䖍裕入朝庚辰詔侯益等赴澶州守捉鄴軍獲鸗脱又云庚辰郭諱次滑州宋延渥納軍辛巳鸗脱還宫薛史隱帝紀丁丑李洪義得密詔遣陳光穗至鄴都翌日郭威以衆南行戊寅至澶州庚辰至滑州是日詔侯益等赴澶州守捉餘與實録同周太祖實録十四日陳光穗至翌日遵路明日遇鸗脱云見召侯益等令守澶州十六日趣滑臺十七日賞諸軍令奉行前詔十八日自滑而南薛史周太祖紀十六日至澶州獲鸗脱十七日至滑州餘與實録同按丁丑十四日也若十七日始詔侯益赴澶州則十六日郭威獲鸗脱何故已見之也蓋帝遣侯益赴澶州必在十六日鸗脱行在遣益之後今從薛史周太祖紀}
以表置鸗脱衣領中使歸白帝曰臣昨得詔書延頸俟死郭崇威等不忍殺臣云此皆陛下左右貪權無厭者譖臣耳|{
	厭於塩翻}
逼臣南行詣闕請罪臣求死不獲力不能制臣數日當至闕庭陛下若以臣為有罪安敢逃刑若實有譖臣者願執付軍前以快衆心臣敢不撫諭諸軍退歸鄴都庚辰郭威趣滑州|{
	澶州西南至滑州一百餘里趣七喻翻}
辛巳義成節度使宋延渥迎降|{
	考異曰隱帝實録十一月丙子誅楊史丁丑孟業至澶州王殷錮業送郭威即日首塗戊寅至河上見王殷庚辰次滑州周太祖實録云十三日夜太祖夢入朝見至詣旦以夢示峻是日陳洪穗至鄴都是十四日丁丑也翌日為衆所迫遵路十五日戊寅也明日行次遇鸗脱欲住澶州十六日己卯也下文又云十六日趣滑臺按大梁至澶州二百七十里澶州至鄴都一百四十里至滑州百二十里不應往還如是之速漢周實録首塗與至滑州日不同蓋十六日趣滑州十七日至滑州也今從周太祖實録}
延渥洛陽人其妻晉高祖女永寧公主也郭威取滑州庫物以勞將士|{
	勞力到翻}
且諭之曰聞侯令公已督諸軍自南來|{
	侯益兼中書令故稱之為令公}
今遇之交戰則非入朝之義不戰則為其所屠吾欲全汝曹功名不若奉行前詔吾死不恨|{
	郭威以此觀衆心向背耳}
皆曰國家負公公不負國所以萬人爭奮如報私讐侯益輩何能為乎王峻徇於衆曰我得公處分俟克京城聽旬日剽掠衆皆踊躍|{
	處昌呂翻分扶問翻剽匹妙翻許士卒以剽掠之利以濟其私可以得而不可長守也}
辛巳鸗脱至大梁前此帝議欲自往澶州聞郭威已至河上而止帝甚有悔懼之色私謂竇貞固曰屬者亦太草草|{
	屬之欲翻屬者猶言頃者也草草亦言率爾欠審諦商量之意}
李業等請空府庫以賜諸軍蘇禹珪以為未可業拜禹珪於帝前曰相公且為天子勿惜府庫|{
	為於偽翻下當為冋}
乃賜禁軍人二十緡下軍半之將士在北者給其家使通家信以誘之|{
	誘音酉}
壬午郭威軍至封丘人情忷懼太后泣曰不用李濤之言宜其亡也|{
	李濤之言見上卷元年}
慕容彦超恃其驍勇言於帝曰臣視北軍猶蠛蠓耳|{
	爾雅註蠛蠓翳蒙之所生一名醯雞孫炎曰此蟲微細羣飛列子曰蠛蠓生朽壤之上因雨而生覩陽而死莊子謂之醯雞蠛莫結翻蠓莫孔翻}
當為陛下生致其魁|{
	為于偽翻}
退見聶文進問北來兵數及將校姓名頗懼曰是亦劇賊未易輕也|{
	將即亮翻校戶教翻易以䜴翻}
帝復遣左神武統軍袁義前威勝節度使劉重進等帥禁軍與侯益等會屯赤岡㠖象先之子也|{
	復扶又翻㠖宜崎翻帥讀曰率袁象先梁將也事見梁紀}
彦超以大軍屯七里店癸未南北軍遇於劉子陂|{
	劉子陂在封丘之南汴郊之北}
帝欲自出勞軍|{
	勞力到翻}
太后曰郭威吾家故舊非死亡切身何以至此但按兵守城飛詔諭之觀其志趣必有辭理則君臣之禮尚全慎勿輕出帝不從|{
	使帝從太后之言雖不能保其往猶未至於野死也}
時扈從軍甚盛|{
	從才用翻下從官同}
太后遣使戒聶文進曰大須在意對曰有臣在雖郭威百人可擒也至暮兩軍不戰帝還宫慕容彦超大言曰陛下來日宫中無事幸再出觀臣破賊臣不必與之戰但叱散使歸營耳甲申帝欲再出太后力止之不可既陳|{
	陳讀曰陣}
郭威戒其衆曰吾來誅羣小非敢敵天子也慎勿先動久之慕容彦超引輕騎直前奮擊郭崇威與前博州刺史李榮帥騎兵拒之|{
	騎奇寄翻帥讀曰率}
彦超馬倒幾獲之|{
	幾居依翻}
彦超引兵退麾下死者百餘人於是諸軍奪氣稍稍降於北軍侯益吳䖍裕張彦超袁㠖劉重進皆潛往見郭威威各遣還營又謂宋延渥曰天子方危公近親宜以牙兵往衛乘輿且附奏陛下願乘間早幸臣營|{
	宋延渥主壻故云近親牙兵謂延渥所領義成牙兵也衛乘䋲證翻間古莧翻}
延渥未至御營亂兵雲擾不敢進而還|{
	還從宣翻又如字}
比暮南軍多歸於北|{
	比必利翻及也}
慕容彦超與麾下十餘騎奔還兖州是夕帝獨與三相及從官數十人宿於七里寨餘皆逃潰|{
	三相竇貞固蘇逢吉禹珪七里寨即慕容彦超所屯七里店寨}
乙酉旦郭威望見天子旌旗在高阪上下馬免冑往從之至則帝已去矣帝策馬將還宫至玄化門|{
	玄化門大梁城北面東來第一門也木酸棗門梁開平元年改曰興和門晉天福三年改曰玄化門}
劉銖在門上問帝左右兵馬何在因射左右|{
	劉銖之射左右其意何為射而亦翻}
帝囘轡西北至趙村追兵已至帝下馬入民家為亂兵所弑 |{
	考異曰實録帝至玄化門劉銖射帝左右帝迴詣西北郭允明露刃隨後西北至趙村前鋒已及亂兵騰沸上懼下馬入於民室郭允明知事不濟乃抽刃犯蹕而崩薛史隱帝紀郭允明知事不濟乃剚刃於帝而崩允明自殺周太祖紀云允明弑漢帝于北郊劉恕曰允明帝所親信何由弑逆蓋郭威兵殺帝事成之後諱之因允明自殺歸罪耳按弑帝未必是允明但莫知為誰故止云亂兵}
蘇逢吉閻晉卿郭允明皆自殺聶文進挺身走軍士追斬之李業奔陜州|{
	九域志大梁至陜州六百五十九里李業欲依其兄耳陜失冉翻}
後匡贊奔兖州|{
	欲依慕容彦超也}
郭威聞帝遇弑號慟曰老夫之罪也|{
	號戶刀翻}
威至玄化門劉銖雨射城外|{
	雨射者射矢如雨也}
威自迎春門入歸私第|{
	迎春門汴城東面北來第一門也本名曹門梁開平元年改曰建陽門晉天福三年改曰迎春門}
遣前曹州防禦使何福進將兵守明德門諸軍大掠通夕煙火四發軍士入前義成節度使白再榮之第執再榮盡掠其財既而進曰某等昔嘗趨走麾下一旦無禮至此何面目復見公|{
	復抉又翻}
遂刎其首而去|{
	以白再榮真定之虐今罹此禍抑天道也刎武粉翻}
吏部侍郎張允家貲以萬計而性吝雖妻亦不之委常自繫衆鑰於衣下行如環珮是夕匿於佛殿藻井之上|{
	風俗通云殿堂象東井刻為荷菱荷菱水物所以厭火杜佑曰漢宫殿率號屋仰為井皆畫水藻蓮菱之屬以厭火何晏景福殿賦繚以藻井編以綷疏又王文考靈光殿賦圓淵方井反植荷渠蓋為方井而畫藻其上也陸佃埤雅曰屋上覆橑謂之藻井}
登者浸多板壞而墜軍士掠其衣遂以凍卒|{
	卒子恤翻}
初作坊使賈延徽有寵於帝與魏仁浦為隣欲併仁浦所居以自廣屢譖仁浦於帝幾至不測|{
	言幾至于死也幾居依翻}
至是有擒延徽以授仁浦者仁浦謝曰因亂而報怨吾所不為也郭威聞之待仁浦益厚右千牛衛大將軍棗彊趙鳳曰郭侍中舉兵欲誅君側之惡以安國家耳而鼠輩敢爾乃賊也豈侍中意邪執弓矢踞胡床坐於巷首掠者至輒射殺之里中皆賴以全|{
	射而亦翻}
丙戌獲劉銖李洪建囚之|{
	考異曰五代史闕文周祖自鄴起兵銖盡誅周祖之家子孫婦女十數人極其慘毒及隱帝遇害周祖以漢}


|{
	太后令收銖下獄使人責銖殺其家對曰銖為漢家戮叛族耳不知其他威怒殺之王禹偁曰周世宗朝史官修漢隱帝實録銖之忠言諱而不載銖今有子孝和推進士第按銖所至貪婪酷虐在青州謀不受代賴郭瓊諭之始入朝私怨楊史快其就戮隱帝敗歸射而不納使至野死其屠滅周祖之家出於殘忍之性耳豈忠義之士邪王禹偁所記蓋憑孝和之言耳今不取}
銖謂其妻曰我死汝且為人婢乎妻曰以公所為雅當然耳王殷郭崇威言於郭威曰不止剽掠|{
	剽匹妙翻}
今夕止有空城耳威乃命諸將分部禁止掠者不從則斬之|{
	分扶問翻}
至晡乃定竇貞固蘇禹珪自七里寨逃歸郭威使人訪求得之尋復其位貞固為相值楊史弄權|{
	楊邠史弘肇}
李業等作亂但以凝重處其間自全而已|{
	處昌呂翻下處分同}
郭威命有司遷隱帝梓宫於西宫或請如魏高貴鄉公故事葬以公禮|{
	高貴鄉公事見六十七卷魏元帝景元元年}
威不許曰倉猝之際吾不能保衛乘輿|{
	乘䋲證翻}
罪已大矣况敢貶君乎太師馮道帥百官謁見郭威|{
	帥讀曰率下同}
威見猶拜之道受拜如平時 |{
	考異曰五代史闕文周祖入京師百官謁之周祖見道猶設拜意道便行推戴道受拜如平時徐曰侍中此行不易周祖氣沮故禪代之謀稍緩按周祖舉兵既克京城所以不即為帝者蓋以漢之宗室崇在河東信在許州贇在徐州若遽代漢慮三鎮舉兵以興復為辭則中外必有響應者故陽稱輔立宗子信素庸愚不足畏贇乃崇子故迎贇而立之使兩鎮息謀俟其離徐已遠去京稍近然後併信除之則三鎮去其二矣然後自立則所與為敵者唯崇而已此其謀也豈馮道受拜之所能沮乎道之所以受拜如平時者正欲示器宇凝重耳}
徐曰侍中此行不易|{
	易以䜴翻}
丁亥郭威帥百官詣明德門起居太后且奏稱軍國事殷請早立嗣君太后誥稱郭允明弑逆|{
	太后之誥云然郭威之志也此事考異已辯之于前}
神器不可無主河東節度使崇忠武節度使信皆高祖之弟武寧節度使贇開封尹勲高祖之子其令百官議擇所宜贇崇之子也高祖愛之養視如子|{
	路振九國志劉崇之長子曰贇少慧黠高祖憐之録為己子贇午倫翻}
郭威王峻入見太后於萬歲宫|{
	按薛史唐莊宗同光二年以太后宫為長壽宫晉漢蓋以為萬歲宫也或曰因萬歲殿為名見賢遍翻}
請以勲為嗣太后曰勲久羸疾不能起|{
	羸倫為翻}
威出諭諸將諸將請見之太后令左右以臥榻舉之示諸將諸將乃信之於是郭威與峻議立贇己丑郭威帥百官表請以贇承大統太后誥所司擇日備灋駕迎贇即皇帝位郭威奏遣太師馮道及樞密直學士王度祕書監趙上交詣徐州奉迎 |{
	考異曰周太祖實録己丑太祖奏遣前太師馮道往彼諭旨太祖將奉表於徐州未知所遣樞密直學士王度請行許之宰臣百寮表祕書監趙上交齎詔同日首塗五代史闕文周祖請道詣徐州冊湘隂公為漢嗣道曰侍中由衷乎周祖設誓道曰莫教老夫為謬語人及行謂人曰平生不謬語今為謬語人矣王禹偁曰周世宗朝詔史臣修周祖實録故道之事迹所宜諱矣按道亷智自將陽愚遠禍恐不肯觸周祖未發之機其徒欲歸美而云耳又隱帝實録云初議立徐帥太后遣中使馳諭劉崇請崇入纘大位崇知立其子上章謙遜恐無此事今不取}
郭威之討三叛也|{
	事見上卷元年二年}
每見朝廷詔書處分軍事皆合機宜問使者誰為此詔使者以翰林學士范質對威曰宰相器也入城訪求得之甚喜時大雪威解所服紫袍衣之|{
	衣于既翻}
令草太后誥令迎新君儀注蒼黄之中討論撰定皆得其宜|{
	蒼黄者猝遽之狀論盧昆翻撰士免翻}
初隱帝遣供奉官押班陽曲張永德賜昭義節度使常思生辰物|{
	供奉官押班供奉官之長也生辰物謂聖節囘賜}
永德郭威之壻也會楊邠等誅密詔思殺永德思素聞郭威多奇異囚永德以觀變及威克大梁思乃釋永德而謝之庚寅郭威帥百官上言比皇帝到闕動涉浹旬|{
	比必利翻十日為浹旬徐州至大梁七百里郭威計程言之也}
請太后臨朝聽政 |{
	考異曰周太祖實録云太后自臨朝令稱制隱帝實録自是至國亡止稱誥今從之朝直遙翻}
先是馬希萼遣蠻兵圍玉潭朱進忠引兵會之崔洪璉兵敗奔還長沙|{
	馬希廣遣崔洪璉屯玉潭事始見上六月}
希萼引兵繼進攻岳州刺史王贇拒之五日不克希萼使人謂贇曰公非馬氏之臣乎不事我欲事異國乎為人臣而懷貳心豈不辱其先人贇曰贇父環為先王將六破淮南兵|{
	王贇父環馬氏之良將也將即亮翻}
今大王兄弟不相容贇常恐淮南坐收其弊一旦以遺體臣淮南誠辱先人耳大王苟能釋憾罷兵兄弟雍睦如初贇敢不盡死以事大王兄弟豈有二心乎希萼慙引兵去辛卯至湘隂焚掠而過|{
	湘隂古羅縣之地唐屬岳州宋屬潭州九域志湘隂縣在潭州東北一百五十五里宋白曰湘隂縣本羅子國秦為羅縣宋元徽二年分益陽羅三縣界處巴峽流人因立湘隂縣以地在湘江之隂故名}
至長沙軍於湘西步兵及蠻兵軍於嶽麓|{
	盛弘之荆州記長沙兩岸有麓山蓋衡山之足又名靈麓峯乃嶽山七十二峯之數自湘西古渡登岸夾徑喬松泉澗盤繞諸峯疊秀下瞰湘江道林嶽麓等寺皆在焉}
朱進忠自玉潭引兵會之馬希廣遣劉彦瑫召水軍指揮使許可瓊帥戰艦五百艘屯城北津屬於南津|{
	帥讀曰率属之欲翻}
以馬希崇為監軍|{
	馬希崇在長沙常為希萼詷希廣希萼又以利啖許可瓊希廣使可瓊為將希崇監軍所謂藉寇兵也}
又遣馬軍指揮使李彦温將騎兵屯駝口扼湘隂路|{
	劉江口有駱駝觜因謂之駝口}
步軍指揮使韓禮將二千人屯楊柳橋扼柵路|{
	朗兵柵于湘西以兵扼其路}
可瓊德勲之子也|{
	許德勲亦楚之良將}
壬辰太后始臨朝以王峻為樞密使袁㠖為宣徽南院使王殷為侍衛馬步軍都指揮使郭崇威為侍衛馬軍都指揮使曹威為侍衛步軍都指揮使陳州刺史李穀權判三司 劉銖李洪建及其黨皆梟首於市而赦其家 |{
	考異曰實録國子博士司天監洛陽王處訥素與周祖善因言劉氏祚短事處訥曰漢歷未盡但以即位後讐殺人夷人之族怨結天下所以社稷不得久長耳時周祖方以兵圍蘇逢吉劉銖之第俟旦而族之聞其言蹶然遽命釋之按周祖時方迎湘隂公立之豈得遽言劉氏祚短乎今不取}
郭威謂公卿曰劉銖屠吾家吾復屠其家怨讐反覆庸有極乎由是數家獲免王殷屢為洪建請免死|{
	王殷先與李洪建分掌侍衛馬步軍以同僚故為之請為于偽翻}
郭威不許後匡贊至兗州慕容彦超執而獻之李業至陜州|{
	陜失冉翻}
其兄保義節度使洪信不敢匿於家業懷金將奔晉陽至絳州盜殺之而取其金蜀施州刺史田行臯奔荆南高保融曰彼貳於蜀安

肯盡忠於我執之歸於蜀伏誅 鎮州邢州奏契丹主將數萬騎入寇攻内丘|{
	内丘本漢中丘縣隋避武元帝諱改為内丘唐屬邢州九域志在州北四十七里范成大北使録邢州三十五里至内丘縣}
五日不克死傷甚衆有戍兵五百叛應契丹引契丹入城屠之又陷饒陽|{
	九域志饒陽縣在深州北九十里}
太后敕郭威將大軍擊之國事權委竇貞固蘇禹珪王峻軍事委王殷十二月甲午朔郭威發大梁 丁酉以翰林學士戶部侍郎范質為樞密副使初蠻酋彭師暠降於楚|{
	見二百八十二卷晉天福五年酋慈由翻暠古老翻}
楚人惡其獷直|{
	惡烏路翻獷古猛翻}
楚王希廣獨憐之以為強弩指揮使領辰州刺史師暠常欲為希廣死|{
	為于偽翻}
及朱進忠與蠻兵合七千餘人至長沙營於江西|{
	湘江之西}
師暠登城望之言於希廣曰朗人驟勝而驕雜以蠻兵攻之易破也願假臣步卒三千自巴溪度江出嶽麓之後至水西令許可瓊以戰艦度江腹背合擊必破之前軍敗則其大軍自不敢輕進矣希廣將從之時馬希萼已遣閒使以厚利啖許可瓊|{
	間古莧翻啖吐濫翻}
許分湖南而治可瓊有貳心乃謂希廣曰師暠與梅山諸蠻皆族類安可信也可瓊世為楚將|{
	許可瓊德勲之子故自言爾}
必不負大王希萼竟何能為希廣乃止希萼尋以戰艦四百餘艘泊江西希廣命諸將皆受可瓊節度日賜可瓊銀伍百兩希廣屢造其營計事|{
	造七到翻}
可瓊常閉壘不使士卒知朗軍進退希廣歎曰真將軍也吾何憂哉|{
	臨亂之君各賢其臣斯言信矣}
可瓊或夜乘單舸詐稱巡江與希萼會水西約為内應|{
	舸苦我翻}
一旦彭師暠見可瓊瞋目叱之|{
	瞋昌眞翻}
拂衣入見希廣曰可瓊將叛國人皆知之請速除之無貽後患希廣曰可瓊許侍中之子豈有是邪|{
	楚加許德勲侍中故希廣稱之}
師暠退歎曰王仁而不斷|{
	斷丁亂翻}
敗亡可翹足俟也潭州大雪平地四尺潭朗兩軍久不得戰希廣信巫覡及僧語塑鬼于江上|{
	覡刑狄翻塑桑故翻摶埴為鬼神之形曰塑}
舉手以却朗兵又作大像于高樓手指水西怒目視之|{
	怒奴古翻}
命衆僧日夜誦經希廣自衣僧服膜拜求福|{
	衣于既翻下暉衣同膜莫乎翻}
甲辰朗州步軍指揮使武陵何敬眞等 |{
	考異曰湖湘故事作何景真今從十國紀年}
以蠻兵三千陳於楊柳橋敬真望韓禮營旌旗紛錯|{
	先是希廣命韓禮營于楊柳橋紛亂也錯雜也陳讀曰陣}
曰彼衆已懼擊之易破也朗人雷暉衣潭卒之服濳入禮寨手劒擊禮不中|{
	手式又翻中竹仲翻}
軍中驚擾敬真等乘其亂擊之禮軍大潰禮被創走至家而卒|{
	創初良翻}
于是朗兵水陸急攻長沙步軍指揮使吳宏小門使楊滌相謂曰以死報國此其時矣各引兵出戰宏出清泰門戰不利滌出長樂戰自辰至午朗兵小却|{
	長樂之下當有門字}
許可瓊劉彦瑫按兵不救滌士卒飢疲退就食彭師暠戰于城東北隅蠻兵自城東縱火城上人招許可瓊軍使救城可瓊舉全軍降希萼長沙遂陷朗兵及蠻兵大掠三日殺吏民焚廬舍自武穆王以來所營宮室皆為灰燼|{
	楚王馬殷謚武穆}
所積寶貨皆入蠻落李彦温望見城中火起自駝口引兵救之朗人已據城拒戰彦温攻清泰門不克與劉彦瑫各將千餘人奉文昭王及希廣諸子趣袁州遂奔唐|{
	楚王希範謚文昭九域志潭州東南至袁州六百三十四里趣七喻翻}
張暉降於希萼|{
	張暉先是自益陽遁歸長沙長沙既陷遂降于希萼}
左司馬希崇帥將吏詣希萼勸進|{
	馬希崇通希萼事始二百八十七卷天福十二年帥讀曰率}
吳宏戰血滿袖見希萼曰不幸為許可瓊所誤今日死不愧先王矣彭師暠投槊于地大呼請死|{
	呼火故翻}
希萼歎曰鐵石人也皆不殺乙巳希崇迎希萼入府視事閉城分捕希廣及掌書記李弘臯弟弘節都軍判官唐昭胤及鄧懿文楊滌等皆獲之希萼謂希廣曰承父兄之業豈無長幼乎希廣曰將吏見推朝廷見命耳希萼皆囚之|{
	卒如張少敵拓拔恒之言}
丙午希萼命内外巡檢侍衛指揮使劉賓禁止焚掠丁未希萼自稱天策上將軍武安武平静江寧遠等軍節度使|{
	馬氏舊有此四鎮之地是時寜遠巡屬已屬南漢}
楚王|{
	此皆父兄官爵希萼未稟命于中國而自稱之}
以希崇為節度副使判軍府事|{
	為希崇殺希萼張本}
湖南要職悉以朗人為之臠食李弘臯弘節唐昭胤楊滌斬鄧懿文於市戊申希萼謂將吏曰希廣懦夫為左右所制耳吾欲生之可乎諸將皆不對朱進忠嘗為希廣所笞對曰大王三年血戰始得長沙|{
	天福十二年希萼希廣始爭國次年交兵至是三年矣}
一國不容二主它日必悔之戊申賜希廣死希廣臨刑猶誦佛書彭師暠葬之干瀏陽門外|{
	瀏陽門潭州城東門瀏音劉}
武寧節度使贇留右都押牙鞏延美元從都教練使楊温守徐州|{
	為二人以徐州拒周張本鞏延美據下卷及歐史當作鞏廷美鞏以邑為姓周有卿士鞏簡公晉有大夫鞏朔從才用翻}
與馮道等西來|{
	自彭城而西來大梁}
在道仗衛皆如王者左右呼萬歲郭威至滑州留數日贇遣使慰勞諸將|{
	勞力到翻}
受命之際相顧不拜私相謂曰我輩屠陷京城其罪大矣若劉氏復立我輩尚有種乎|{
	種章勇翻}
己酉威聞之即引兵行趣澶州|{
	趣七喻趣}
辛亥遣蘇禹珪如宋州迎嗣君 楚王希萼以子光贊為武平留後以何敬真為朗州牙内都指揮使將兵戍之希萼召拓抜恒欲用之恒稱疾不起|{
	自希廣之立拓抜恒已杜門矣事見二百八十七卷天福十二年}
壬子郭威度河館于澶州|{
	館古玩翻澶時連翻}
癸丑旦將發將士數千人忽大譟威命閉門將士踰垣登屋而入曰天子須侍中自為之將士已與劉氏為仇不可立也或裂黄旗以被威體|{
	被皮義翻}
共扶抱之呼萬歲震地因擁威南行威乃上太后牋請奉宗廟事太后為母丙辰至韋城|{
	隋分白馬置韋城縣治韋氏國城屬滑州九域志在州東南五十里丁度曰韋城縣古豕韋國也上時掌翻}
下書撫諭大梁士民以昨離河上在道秋毫不犯勿有憂疑|{
	恐京城士民懲前者剽掠之禍奔迸四出故撫安之離力智翻}
戊午威至七里店竇貞固帥百官出迎拜謁因勸進威營於臯門村|{
	臯門村蓋在臯門之外按大梁城無臯門詩大雅綿之篇曰乃立臯門臯門有伉毛氏傳曰王之郭門曰臯門鄭氏箋曰諸侯之宮外門曰臯門朝門曰應門内有應門天子之宮加之庫雉至禮記明堂位記周賜魯公以天子之制其言曰庫門天子臯門雉門天子應門鄭注又云天子五門臯庫雉應路魯有庫雉路則諸侯三門歟詳而味之詩箋記注微有不同而五代之時汴城之外所謂臯門村蓋以郭門之外有村遂呼曰臯門村合于毛氏詩傳臯門村屬開封縣薛史云王擅葬于開封縣之臯門原以是知之}
武寧節度使贇已至宋州王峻王殷聞澶州軍變遣侍衛馬軍都指揮使郭崇威將七百騎往拒之又遣前申州刺史馬鐸將兵詣許州巡檢崇威忽至宋州陳于府門外贇大驚闔門登樓詰之|{
	詣許州巡檢備劉信也汴京至宋州二百八十五里耳贇不意其至故驚而詰之詰去吉翻}
對曰澶州軍變郭公慮陛下未察故遣崇威來宿衛無它也贇召崇威崇威不敢進馮道出與崇威語|{
	先是使馮道迎贇故道在贇所}
崇威乃登樓贇執崇威手而泣崇威以郭威意安諭之少頃崇威出時護聖指揮使張令超帥部兵為贇宿衛|{
	按薛史護聖漢侍衛馬軍也帥讀曰率下同}
徐州判官董裔說贇曰|{
	說式苪翻}
觀崇威視瞻舉措必有異謀道路皆言郭威已為帝而陛下深入不止禍其至哉請急召張令超諭以禍福使夜以兵刼崇威奪其兵明日掠睢陽金帛募士卒北走晉陽|{
	宋州睢陽郡贇父崇鎮晉陽睢音須走音奏}
彼新定京邑未暇追我此策之上也贇猶豫未決是夕崇威密誘令超令超帥衆歸之|{
	誘音酉}
贇大懼郭威遺贇書云為諸軍所迫召馮道先歸留趙上交王度奉侍道辭行贇曰寡人此來所恃者以公三十年舊相故無疑耳|{
	馮道唐明宗天成二年為相至是二十四年曰三十年舉成數也遺於季翻}
今崇威奪吾衛兵事危矣公何以為計道默然|{
	無以答贇故默馮道自謂癡頑老子良不妄也}
客將賈貞數目道欲殺之|{
	將即亮翻數所角翻}
贇曰汝輩勿草草此無預馮公事|{
	契丹主入汴責劉繼勲繼勲歸罪于道道幾死矣宋州之事使劉贇從賈貞之意道亦必死矣而契丹主謂道非多事者劉贇謂無預馮公事豈非以其在位素懷沖澹與物無競人皆敬其名德而然邪道之全身固為得矣有國者焉用彼相哉然自後唐同光以來樞密使任事丞相取充位而已責人斯無難惟受責俾如流以此而言道未肯受責也}
崇威遷贇於外館殺其腹心董裔賈貞等數人己未太后誥廢贇為湘隂公馬鐸引兵入許州劉信惶惑自殺庚申太后誥以侍中監國|{
	太后兩誥皆郭威之志也侍中稱郭威官}
百官藩鎮相繼上表勸進壬戌夜監國營有步兵將校醉揚言曏者澶州騎兵扶立今步兵亦欲扶立監國斬之 南漢主以宫人盧瓊仙黄瓊芝為女侍中朝服冠帶參決政事|{
	朝直遙翻}
宗室勲舊誅戮殆盡惟宦官林延遇等用事|{
	史言南漢終以宦官女寵亡國而南漢主所以能終其世者以僻處海隅而中國未有真主耳}


資治通鑑卷二百八十九
