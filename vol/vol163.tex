\section{資治通鑑卷一百六十三}
宋 司馬光 撰

胡三省 音註

梁紀十九|{
	上章敦牂一年}


太宗簡文皇帝上|{
	諱綱字世讚小字六通武帝第三子昭明太子母弟也謚法平易不訾曰簡學勤好問曰文}


大寶元年春正月辛亥朔大赦改元 陳霸先發始興至大庾嶺|{
	大庾嶺在今南安軍大庾縣西南二十里吳錄南野縣有大庾山自嶺嶠九嶝二里至嶺下平行十里至平亭}
蔡路養將二萬人軍於南野以拒之|{
	南野縣漢屬豫章郡晋屬南康郡劉昫曰唐韶州始興縣漢南野縣地宋白曰䖍州之南康大庾嶺南雄州之始興皆漢南野縣地輿地志今䖍州龍南縣漢南野縣地將即亮翻}
路養妻姪蘭陵蕭摩訶年十三單騎出戰無敢當者杜僧明馬被傷|{
	騎奇寄翻被皮義翻}
陳霸先救之授以所乘馬僧明上馬復戰|{
	上時掌翻復扶又翻}
衆軍因而乘之路養大敗脱身走霸先進軍南康 |{
	考異曰太清紀在二月今從陳帝紀}
湘東王繹承制授霸先明威將軍交州刺史戊辰東魏進太原公高洋位丞相都督中外諸軍錄尚書事大行臺齊郡王 庚午邵陵王綸至江夏郢州刺史南康王恪郊迎|{
	南康當作南平參考前後及梁書可見夏戶雅翻}
以州讓之綸不受乃推綸爲假黄钺都督中外諸軍事承制置百官 |{
	考異曰太清紀云三月倫逼奪恪州徙恪於郡廨今從梁書典略}
魏楊忠圍安陸柳仲禮馳歸救之|{
	去年仲禮將兵趣襄陽}
諸將恐仲禮至則安陸難下請急攻之|{
	將即亮翻下同}
忠曰攻守勢殊未可猝拔若引日勞師表裏受敵非計也南人多習水軍不閑野戰仲禮師在近路吾出其不意以奇兵襲之彼怠我奮一舉可克克仲禮則安陸不攻自拔諸城可傳檄定也乃選騎二千衘枚夜進敗仲禮於漴頭|{
	敗補邁翻杜佑曰漴音崇水所衝曰漴 考異曰太清紀作潼頭在去年十二月今從典略}
獲仲禮及其弟子禮盡俘其衆馬岫以安陸别將王叔孫以竟陵皆降於忠|{
	降戶江翻}
於是漢東之地盡入於魏國廣陵人來嶷|{
	嶷魚力翻}
說前廣陵太守祖皓曰董紹先輕而無謀|{
	輕率正翻}
人情不附襲而殺之此壯士之任也今欲糾帥義勇奉戴府君|{
	帥讀曰率下同}
若其克捷可立桓文之勲必天未悔禍猶足爲梁室忠臣皓曰此僕所願也乃相與糾合勇士得百餘人癸酉襲廣陵斬南兖州刺史董紹先據城馳檄遠近推前太子舍人蕭勔爲刺史|{
	勔彌兖翻}
仍結東魏爲援皓暅之子|{
	祖暅見一百四十七卷武帝天監十二年諸本作暅之之子者衍一之字暅古鄧翻}
勔勃之兄也乙亥景遣郭元建帥衆奄至皓嬰城固守 二月魏楊忠乘勝至石頭|{
	五代志竟陵郡長壽縣後周置石頭郡}
欲進逼江陵湘東王繹遣舍人庾恪說忠曰詧來伐叔|{
	說式芮翻來伐事見上卷上年}
而魏助之何以使天下歸心忠遂停湕北|{
	湕水之北也湕紀偃翻}
繹遣舍人王孝祀等送子方略爲質以求和魏人許之繹與忠盟曰魏以石城爲封梁以安陸爲界請同附庸并送質子貿遷有無|{
	質音致貿易也遷徙也徙有之無以相貿易貿音茂}
永敦隣睦忠乃還|{
	還從宣翻又如字}
宕昌王梁彌定|{
	宕徒浪翻}
爲其宗人獠甘所襲彌定奔魏獠甘自立羌酋傍乞鐵忽據渠株川|{
	獠魯皓翻酋慈秋翻}
與渭川民鄭五醜合諸羌以叛魏丞相泰使大將軍宇文貴凉州刺史史寧討之擒斬鐵忽五醜寧别擊獠甘破之獠甘將百騎奔生羌鞏亷玉|{
	生羌遠在塞外不羈屬於魏者騎奇寄翻}
寧復納彌定於宕昌置岷州於渠株川|{
	五代志臨洮郡臨洮縣西魏置溢樂縣并置岷州復扶又翻}
進擊鞏亷玉斬獠甘虜亷玉送長安侯景遣任約于慶等帥衆二萬攻諸藩|{
	諸藩謂梁之宗室分任藩}


|{
	維者任音壬帥讀曰率}
邵陵王綸欲救河東王譽而兵糧不足乃致書於湘東王繹曰天時地利不及人和|{
	用孟子之言}
况於手足肱支豈可相害今社稷危恥創巨痛深|{
	禮記三年問曰創巨者其疾久痛甚者其愈遲三年者稱情而立文所以爲至痛極也斬衰苴杖居倚廬食粥寢苫枕塊所以爲至痛飾也創音瘡}
唯應剖心嘗膽泣血枕戈|{
	越王句踐卧薪嘗膽求以報吴晋劉琨枕戈待旦志梟逆虜枕之任翻子執親喪泣血三年孔穎達曰說文哭哀聲也泣無聲出淚也則無聲謂之泣矣連言血者以淚出於目猶血出於體故以淚比血禮記曰子羔執親之喪泣血三年注曰無聲而血出是也}
其餘小忿或宜容貰|{
	貰貸也}
若外難未除家禍仍構料今訪古|{
	難乃旦翻料音聊}
未或不亡夫征戰之理唯求克勝至於骨肉之戰愈勝愈酷捷則非功敗則有喪|{
	喪息浪翻}
勞兵損義虧失多矣侯景之軍所以未窺江外者|{
	荆州治江陵在江北故曰江外}
良爲藩屏盤固宗鎮彊密|{
	爲于偽翻屏必郢翻}
弟若陷洞庭|{
	湘州之地襟帶洞庭故謂湘州爲洞庭}
不戢兵刃雍州疑迫何以自安必引進魏軍以求形援|{
	以綸之昏狂猶能言及於此盖勢有所必至也戢阻立翻雍於用翻}
弟若不安家國去矣必希解湘州之圍存社稷之計繹復書陳譽過惡不赦|{
	言其過大惡極法所不赦}
且曰詧引楊忠來相侵逼頗遵談笑用却秦軍|{
	魯仲連談笑而却秦軍繹引此以爲大言}
曲直有在不復自陳|{
	復扶又翻}
臨湘旦平|{
	臨湘縣自漢以來屬長沙郡時爲州郡治所隋改臨湘縣曰長沙縣}
暮便即路|{
	即路就路也承上文而言若欲攻襄陽考之下文盖謂討侯景}
綸得書投之於案慷慨流涕曰天下之事一至於斯湘州若敗吾亡無日矣|{
	綸知繹兵東下必將圖已故云然使在京口之時能思及此則何亡國喪身之有古人所以重居安慮危居寵思畏者也史爲綸北奔張本}
侯景遣侯子鑒帥舟師八千自帥徒兵一萬|{
	徒兵步兵也帥讀曰率下同}
攻廣陵三日克之執祖皓縛而射之箭徧體然後車裂以徇城中無少長皆埋之於地馳馬射而殺之|{
	射而亦翻少詩照翻長知兩翻 考異曰太清紀曰城中數百人典畧曰死者八千人今從南史}
以子鑒爲南兖州刺史鎮廣陵|{
	廣陵之人既殱矣子鑒所鎮者空城耳}
景還建康 丙戍以安陸王大春爲東揚州刺史省吳州|{
	景置吳州見上卷上年}
乙巳以尚書僕射王克爲左僕射 庚寅東魏以尚書令高隆之爲太保宣城内史楊白華進據安吳|{
	孫吳立安吳縣屬宣城郡隋省安吳入涇縣}
侯景遣于子悦帥衆攻之不克 東魏行臺辛術將兵入寇圍陽平不克 侯景納上女溧陽公主甚愛之|{
	溧音栗}
三月甲申請上禊宴於樂遊苑|{
	禊宴因袚禊而開宴禊胡計翻樂遊苑在玄武湖南}
帳飲三日上還宫景與公主共據御牀南面並坐羣臣文武列坐侍宴 庚申東魏進丞相洋爵爲齊王臨川内史始興王毅等擊莊鐵|{
	王毅始興人毅即毅字後人傳寫變立為耳}
鄱陽王範遣其將巴西侯瑱救之|{
	瑱他甸翻}
毅等敗死 鄱陽世子嗣與任約戰於三章約敗走嗣因徙鎮三章謂之安樂柵|{
	去年鄱陽王範西上使世子嗣守安樂柵今移柵三章亦以安樂名柵樂音洛}
夏四月庚辰朔湘東王繹以上甲侯韶爲長沙王|{
	上甲侯韶奔江陵見上卷上年按梁書長沙王懿子淵業嗣封卒子孝儼嗣封又卒子齊嗣封眘懿曾孫也時在建康湘東以韶歸之遂以爲長沙王非禮也}
丙午侯景請上幸西州上御素輦侍衛四百餘人景浴鐵數千翼衛左右|{
	浴鐵者言鐵甲堅滑若以水浴之也}
上聞絲竹悽然泣下命景起舞景亦請上起舞酒闌坐散|{
	坐徂卧翻}
上抱景于牀曰我念丞相景曰陛下如不念臣臣何得至此逮夜乃罷時江南連年旱蝗江揚尤甚百姓流亡相與入山谷江湖采草根木葉菱芡而食之|{
	菱芰也芡巨險翻說文曰鷄頭也方言曰南楚謂之鷄頭北燕謂之䓈青徐淮泗之間謂之芡}
所在皆盡死者蔽野富室無食皆鳥面鵠形衣羅綺懷珠玉俯伏牀帷待命聽終千里絶烟|{
	衣於既翻烟與煙同}
人迹罕見白骨成聚如丘隴焉景性殘酷於石頭立大碓|{
	碓都内翻}
有犯法者擣殺之常戒諸將曰破柵平城當浄殺之使天下知吾威名故諸將每戰勝專以焚掠爲事斬刈人如草芥以資戲笑由是百姓雖死終不附之|{
	史言侯景將敗}
又禁人偶語犯者刑及外族|{
	男子謂舅家爲外家婦人謂父母之家爲外家外族外家之族}
爲其將帥者悉稱行臺來降附者悉稱開府其親寄隆重者曰左右廂公勇力兼人者曰庫直都督|{
	南史侯景傳作庫真都督誤也}
魏封皇子儒爲燕王公爲吳王 侯景召宋子仙還京口|{
	去年宋子仙克東揚州今召還京口}
邵陵王綸在郢州以聽事爲正陽殿内外齋閤悉加題署其部下陵暴軍府郢州將佐莫不怨之諮議參軍江仲舉南平王恪之謀主也說恪圖綸|{
	說式芮翻}
恪驚曰若我殺邵陵寜靜一鎮荆益兄弟必皆内喜|{
	帝既蒙塵綸於兄弟之次當立恪若除綸則人望歸於荆益必當内以爲喜時湘東王繹在荆州武陵王紀在益州}
海内若平則以大義責我矣且巨逆未梟|{
	巨逆謂侯景梟謂梟其首梟堅堯翻}
骨肉相殘自亡之道也卿且息之仲舉不從部分諸將|{
	分扶問翻}
刻日將謀泄綸壓殺之恪狼狽往謝綸曰羣小所作非由兄也兇黨已斃兄勿深憂|{
	南平王恪偉之子也於綸為兄}
王僧辯急攻長沙辛巳克之執河東王譽斬之傳首江陵湘東王繹反其首而葬之|{
	反其首於長沙與身俱葬}
初世子方等之死|{
	見上卷上年}
臨蒸周鐵虎功最多|{
	吳立臨蒸縣以縣臨蒸水而名屬衡陽郡晋屬湘東郡酈道元曰臨蒸即故酃縣縣即湘東郡劉昫曰隋改臨蒸為衡陽縣今衡州所治衡陽縣是也衡州志吳分酃縣立臨蒸縣俯臨蒸水其氣如蒸姚思亷陳書曰周鐵虎不知何許人梁世南度事河東王譽為臨蒸令如此則通鑑逸令字}
譽委遇甚重僧辯得鐵虎命烹之呼曰|{
	呼火故翻}
侯景未滅柰何殺壯士僧辯奇其言而釋之還其麾下繹以僧辯爲左衛將軍加侍中鎮西長史繹自去歲聞高祖之喪以長沙未下故匿之壬寅始喪刻檀爲高祖像置於百福殿事之甚謹動靜必咨焉繹以為天子制於賊臣不肯從大寶之號猶稱太清四年丙午繹下令大舉討侯景移檄遠近 鄱陽王範至湓城以晋熙為晋州|{
	晉安帝分廬江郡立晋熙郡五代志同安郡懷寜縣舊置晋熙郡唐以同安郡為舒州}
遣其世子嗣為刺史江州郡縣多輒改易|{
	改易郡縣守令也}
尋陽王大心政令所行不出一郡|{
	惟尋陽一郡而已}
大心遣兵擊莊鐵嗣與鐵素善請發兵救之範遣侯瑱帥精甲五千助鐵|{
	瑱他甸翻又音鎮帥讀曰率下同}
由是二鎮互相猜忌無復討賊之志|{
	復扶又翻}
大心使徐嗣徽帥衆二千築壘稽亭以備範|{
	據齊書晋安王子懋傳子懋謀舉兵於江州宣城王遣裴叔業襲取湓城子懋先已具船於稽亭渚聞叔業得湓城乃據州自衛則稽亭渚在江州城東也帥讀曰率}
市糴不通範數萬之衆無所得食多餓死範憤恚疽於背五月乙卯卒|{
	恚於避翻 考異曰典略作己酉今從大清紀}
其衆祕不喪奉範弟安南侯恬為主有衆數千人 丙辰侯景以元思䖍為東道大行臺鎮錢唐丁巳以侯子鑒為南兖州刺史東魏齊王洋之為開府也|{
	洋爲開府見一百五十七卷武帝大同元年}
勃

海高德政爲管記|{
	管記即記室參軍之職}
由是親昵言無不盡|{
	昵尼質翻}
金紫光祿大夫丹陽徐之才北平太守廣宗宋景業|{
	漢晋以來中山有北平縣後魏孝昌中分置北平郡屬定州廣宗縣漢屬鉅鹿郡後屬安平國後魏太和二十一年分置廣宗郡時屬司州}
皆善圖䜟|{
	䜟楚譛翻}
以為太歲在午當有革因德政以白洋勸之受禪洋以告婁太妃太妃曰汝

父如龍兄如虎猶以天位不可妄據終身北汝獨何人欲行舜禹之事乎洋以告之才之才曰正爲不及父兄|{
	爲于僞翻}
故宜早升尊位耳洋鑄像卜之而成乃使開府儀同三司段韶問肆州刺史斛律金金來見洋固言不可以宋景業首陳符命請殺之洋與諸貴議於太妃前太妃曰吾兒懦直必無此心高德政樂禍教之耳|{
	樂音洛}
洋以人心不壹遣高德政如鄴察公卿之意未還洋擁兵而東至平都城|{
	九域志遼州遼山縣有平城鎮宋白曰遼州平城縣本漢湼縣地晋置武鄉縣此地屬焉隋開皇十六年於趙簡子所立平都故城置平城縣}
召諸勲貴議之莫敢對長史杜弼曰關西國之勍敵|{
	謂宇文氏也勍渠京翻}
若受魏禪恐彼挾天子自稱義兵而東向|{
	此天子謂西魏主}
王何以待之徐之才曰今與王爭天下者彼亦欲為王所爲縱其屈彊|{
	屈其勿翻彊其兩翻}
不過隨我稱帝耳弼無以應高德政至鄴諷公卿莫有應者司馬子如逆洋於遼陽|{
	遼陽縣自漢末以來屬樂平郡隋開皇十一年改曰遼山縣我宋朝爲遼州治所}
固言未可洋欲還倉丞李集曰|{
	北齊之制太倉及水次諸倉皆有令丞北齊紀作尚食丞李集}
王來爲何事而今欲還|{
	爲于僞翻}
洋僞使於東門殺之而别令賜絹十匹遂還晋陽自是居常不悦徐之才宋景業等日陳隂陽雜占云宜早受命高德政亦敦勸不已洋使術士李密卜之遇大横曰漢文之卦也|{
	大横見十三卷漢高后八年}
又使宋景業筮之遇乾之鼎|{
	乾之初九九五二爻動變而之鼎}
曰乾君也鼎五月卦也宜以仲夏受禪或曰五月不可入官犯之終於其位|{
	隂陽家之說上官忌正月五月九月}
景業曰王爲天子無復下期|{
	復扶又翻}
豈得不終於其位乎洋大悦乃晋陽高德政錄在鄴諸事條進於洋洋令左右陳山提馳驛齎事條并密書與楊愔|{
	愔於今翻}
是月山提至鄴楊愔即召太常卿邢邵議造儀注祕書監魏收草九錫禪讓勸進諸文|{
	凡禪代皆奉表三讓百僚三表勸進而後即位故令收預草諸文}
引魏宗室諸王入北宫留於東齋甲寅東魏進洋位相國總百揆備九錫洋行至前亭|{
	前亭在晋陽之東平都城之西}
所乘馬忽倒意甚惡之至平都城不復肯進|{
	惡烏路翻復扶又翻}
高德政徐之才苦請曰山提先去恐其漏泄即命司馬子如杜弼馳驛續入觀察物情子如等至鄴衆人以事勢已决無敢異言洋至鄴召夫|{
	夫民夫也}
齎築具集城南高隆之請曰用此何爲洋作色曰我自有事君何問爲欲族滅邪隆之謝而退於是作圓丘備法物丙辰司空潘樂侍中張亮黄門郎趙彦深等求入啟事東魏孝靜帝在昭陽殿見之亮曰五行遞運有始有終|{
	謂五德之運以木代水以火代木以土代火以金代土以水代金也}
齊王聖德欽明萬方歸仰願陛下遠法堯舜帝歛容曰此事推挹已久|{
	推吐雷翻挹遜也}
謹當遜避又曰若爾須作制書中書郎崔劼裴讓之曰|{
	劼丘八翻}
制已作訖使侍中楊愔進之東魏主既署曰居朕何所愔對曰北城别有館宇乃下御坐步就東廊|{
	坐徂卧翻}
詠范蔚宗後漢書贊曰獻生不辰身播國屯終我四百永作虞賓|{
	范曄字蔚宗作後漢書此其贊獻帝之辭也賢注曰辰時也播遷也言獻帝生不逢時身既播遷國又屯難也漢有天下四百年而運終虞賓謂虞以堯子丹朱爲賓書曰虞賓在位蔚紆勿翻屯陟倫翻}
所司請|{
	所司謂掌禪代事者請者請出宫居别邸也此實楊愔等使人請之}
帝曰古人念遺簪弊履朕欲與六宫别可乎高隆之曰今日天下猶陛下之天下况在六宫帝步入與妃嬪已下别舉宫皆哭趙國李嬪誦陳思王詩云王其愛玉體俱享黄髮期|{
	曹植魏武帝之子封陳王謚曰思殯毗賓翻}
直長趙道德以車一乘於東閣|{
	直長官名凡殿中諸局各有奉御有直長趙道德盖尚乘直長也亦高氏之私人東閣即東閣門長竹兩翻乘繩證翻}
帝登車道德超上抱之|{
	上時掌翻}
帝叱之曰朕自畏天順人何物奴敢逼人如此道德猶不下出雲龍門王公百僚拜辭高隆之灑泣遂入北城居司馬子如南宅|{
	司馬子如有宅在太原故謂鄴城之宅爲南宅}
遣太尉彭城王韶等奉璽綬禪位于齊|{
	璽斯氏翻綬音受東魏十六年而亡考異曰北齊書北史高德政傳云五月六日留咸陽王坦等七日司馬子如等至鄴九日文宣至城南頓案後魏書北史帝紀皆云辛亥王如鄴甲寅加九錫丙辰魏主遜位戊午王即帝位典略辛亥王還鄴以長歷推之此月己酉朔皆不與德政傳曰相應盖辛亥始自晋陽如鄴非到鄴之日也}
戊午齊王即皇帝位于南郊|{
	帝諱洋字子進勃海王高歡第二子澄之母弟也歡以勃海王贈齊王洋又進爵齊王且高氏本勃海人勃海故齊地也國遂號曰齊}
大赦改元天保自魏敬宗以來百官絶祿至是始復給之|{
	復扶又翻}
己未封東魏主為中山王待以不臣之禮追尊齊獻武王為獻武皇帝廟號太祖後改為高祖文襄王為文襄皇帝廟號世宗辛酉尊王太后婁氏為皇太后乙丑降魏朝封爵有差其宣力霸朝|{
	自高歡起兵以來諸勲貴皆宣力霸朝者也朝直遙翻}
及西南投化者不在降限|{
	謂自關西及江南來投者}
文成侯寧起兵於吳有衆萬人己巳進攻吳郡|{
	吳郡帶吳縣寧蓋起兵於吳縣界進攻吳郡城也按侯景傳寧起兵於吳西鄉去年陸緝等推寧據吳郡宋子仙擊之敗走今復起兵於西鄉}
行吳郡事侯子榮逆擊殺之寧範之弟也子榮因縱兵大掠郡境自晋氏渡江三吳最為富庶貢賦商旅皆出其地及侯景之亂掠金帛既盡乃掠人而食之或賣於北境遺民殆盡矣是時唯荆益所部尚完實太尉益州刺史武陵王紀移告征鎮使世子圓照帥兵三萬受湘東王節度|{
	帥讀曰率}
圓照軍至巴水|{
	巴郡巴縣有巴水水折三迴如巴字巴郡唐為渝州 考異曰南史云六月辛酉紀遣圓照東下按六月己卯朔無辛酉典畧在五月或者五月辛酉歟}
繹授以信州刺史令屯白帝|{
	梁置信州於白帝唐改夔州}
未許東下 六月辛巳以南郡王大連行揚州事江夏王大欵山陽王大成宜都王大封自信安閒道奔江陵|{
	閒古莧翻}
齊主封宗室高岳等十人功臣庫狄干等七人皆爲王|{
	高岳及隆之歸彦思宗長弼普子瑗顯國叡孝緒凡十人庫狄干斛律金賀拔仁韓軌可朱渾道元彭樂潘相樂凡七人}
癸未封弟浚爲永安王淹爲平陽王浟爲彭城王演爲常山王渙爲上黨王淯爲襄城王湛爲長廣王湝爲任城王|{
	浟以周翻淯音育湝戶皆翻}
湜爲高陽王濟爲博陵王凝爲新平王潤爲馮翊王洽爲漢陽王 鄱陽王範既卒侯瑱往依莊鐵鐵忌之瑱不自安丙戍詐引鐵謀事因殺之自據豫章 尋陽王大心遣徐嗣徽夜襲湓城安南侯恬裴之横等擊走之 齊主娶趙郡李希宗之女生子殷及紹德又納段韶之妹及將建中宫高隆之高德政欲結勲貴之援乃言漢婦人不可爲天下母宜更擇美配帝不從丁亥立李氏爲皇后 |{
	考異曰典畧在五月乙丑今從北齊帝紀}
以段氏爲昭儀子殷爲皇太子庚寅以庫狄干爲太宰彭樂爲太尉潘相樂爲司徒司馬子如爲司空辛卯以清河王岳爲司州牧 侯景以羊鴉仁爲五兵尚書庚子鴉仁出奔江西將赴江陵至東莞|{
	南徐州有東莞郡不在江西意者東莞其東關之誤歟}
盗疑其懷金邀殺之 |{
	考異曰太清紀在十月今從梁帝紀典畧}
魏人欲令岳陽王詧哀嗣位詧辭不受丞相泰使榮權冊命詧爲梁王始建臺置百官|{
	詧字理孫武帝之孫昭明太子之第三子}
陳霸先修崎頭古城徙居之|{
	崎渠希翻曲岸曰崎九域志南安軍治大庾縣古南野也有南康古城又有峽頭鎮}
初燕昭成帝奔高麗|{
	見一百二十三卷宋文帝元嘉十三年麗力知翻}
使其族人馮業以三百人浮海奔宋因留新會|{
	晋恭帝元熙二年分南海郡立新會郡隋唐爲新會縣屬廣州九域志新會縣在廣州之西南三百三十里}
自業至孫融世爲羅州刺史|{
	五代志高凉郡石龍縣舊置羅州我朝爲化州治所}
融子寶爲高凉太守|{
	高凉縣漢屬合浦郡獻帝建安二十二年吳分立高凉郡梁置高州}
高凉洗氏|{
	洗音銑丁度集韻姺國名或作䢾姓氏韻纂又音綿 考異曰典畧作沈氏今從隋書}
世爲蠻酋|{
	酋慈秋翻}
部落十餘萬家有女多籌略善用兵諸洞皆服其信義融聘以爲寶婦融雖累世爲方伯非其土人號令不行洗氏約束本宗使從民禮每與寶參决辭訟首領有犯雖親戚無所縱舍|{
	舍讀曰捨}
由是馮氏始得行其政高州刺史李遷仕據大臯口|{
	五代志高凉郡梁置高州}
遣使召寶|{
	使疏吏翻下同}
寶欲往洗氏止之曰刺史無故不應召太守必欲詐君共反耳寶曰何以知之洗氏曰刺史被召援臺|{
	被皮義翻}
乃稱有疾鑄兵聚衆而後召君此必欲質君以發君之兵也|{
	質音致}
願且無往以觀其變數日遷仕果反遣主帥杜平虜將兵入灨石城魚梁以逼南康|{
	帥所類翻魚梁亦地名近灨石灨古暗翻}
霸先使周文育擊之洗氏謂寶曰平虜驍將也|{
	驍堅堯翻將即亮翻下同}
今入灨石與官軍相拒勢未得還遷仕在州無能爲也君若自往必有戰鬬宜遣使卑辭厚禮告之曰身未敢出欲遣婦參彼聞之必憙而無備|{
	憙與喜同}
我將千餘人步擔雜物唱言輸賧|{
	擔都甘翻賧吐濫翻}
得至柵下破之必矣寶從之遷仕果不設備洗氏襲擊大破之遷仕走保寧都|{
	吳分漢贛縣立陽都縣晋武太康元年更名寧都五代志南康䖍化縣舊曰寧都}
文育亦擊走平虜據其城洗氏與霸先會于灨石還謂寶曰陳都督非常人也甚得衆心必能平賊君宜厚資之湘東王繹以霸先爲豫州刺史領豫章内史 辛丑裴之横攻稽亭徐嗣徽擊走之 秋七月辛亥齊立世宗妃元氏爲文襄皇后|{
	后東魏孝靜帝女}
宫曰静德又封世宗子孝琬爲河間王孝瑜爲河南王乙卯以尚書令封隆之錄尚書事尚書左僕射平陽王淹爲尚書令 辛酉梁王詧入朝於魏|{
	自此魏益厚詧矣朝直遥翻}
初東魏遣儀同武威牒雲洛等|{
	雲當作云牒云虜複姓柔然阿那瓖之求附於魏也魏遣牒云具仁往使牒云之爲姓尚矣}
迎鄱陽世子嗣使鎮皖城|{
	劉昫曰懷寧宿松望江太湖等縣皆漢晥縣地蓋此城即皖縣古城也皖戶板翻}
嗣未及行任約軍至洛等引去嗣遂失援出戰敗死約遂略地至湓城尋陽王大心遣司馬韋質出戰而敗帳下猶有戰士千餘人咸勸大心走保建州|{
	後漢汝南郡有苞信縣江左僑置於弋陽界五代志弋陽郡殷城縣舊曰苞信梁置義城郡及建州帳下勸大心走保之者便於入齊也九域志殷城縣併入固始}
大心不能用戊辰以江州降約先是大心使太子洗馬韋臧鎮建昌|{
	降戶江翻下同先悉薦翻洗亦同音}
有甲士五千聞尋陽不守欲帥衆奔江陵|{
	帥讀曰率下同}
未發爲麾下所殺臧粲之子也|{
	太清三年青塘之戰韋粲死之}
于慶略地至豫章侯瑱力屈降之慶送瑱於建康景以瑱同姓待之甚厚留其妻子及弟為質|{
	質音致}
遣瑱隨慶徇蠡南諸郡|{
	蠡南謂彭蠡湖以南也}
以瑱爲湘州刺史 |{
	考異曰太清紀在十一月今從典畧}
初巴山人黄灋有勇力|{
	五代志臨川郡崇仁縣梁置巴山郡劉昫曰吳分臨汝爲新建縣梁置巴山郡巨俱翻}
侯景之亂合徒衆保鄉里太守賀詡下江州|{
	自巴山順流赴江州爲下}
命灋監郡事|{
	監工衡翻}
灋屯新淦于慶自豫章分兵襲新淦|{
	新淦縣漢屬豫章郡五代志屬廬陵郡淦古暗翻}
灋敗之|{
	敗補邁翻}
陳霸先使周文育進軍擊慶灋引兵會之 邵陵王綸聞任約將至使司馬蔣思安將精兵五千襲之約衆潰思安不設備約收兵襲之思安敗走 湘東王繹改宜都爲宜州|{
	沈約曰劉備分南郡立宜都郡梁以宜都郡置宜州隋并宜都入夷陵郡}
以王琳爲刺史 是月以南郡王大連爲江州刺史 魏丞相泰以齊主稱帝帥諸軍討之|{
	帥讀曰率下同}
以齊王廓鎮隴右徵秦州刺史宇文導爲大將軍都督二十三州諸軍事屯咸陽鎮關中 益州沙門孫天英帥徒數千人夜攻州城武陵王紀與戰斬之 邵陵王綸大脩鎧仗將討侯景湘東王繹惡之|{
	惡綸由此兵力益彊將不利於己也惡烏路翻}
八月甲午遣左衛將軍王僧辯信州刺史鮑泉等帥舟師一萬東趣江郢|{
	趣七喻翻 考異曰典略云九月戊申朔繹遣僧辯按太清紀事在八月末今從梁簡文帝紀}
聲言拒任約且云迎邵陵王還江陵授以湘州 齊王初立勵精爲治|{
	治直吏翻}
趙道德以事屬黎陽太守清河房超|{
	屬之欲翻下請屬同守式又翻}
超不書棓殺其使|{
	魏收志孝昌中分汲郡置黎陽郡屬司州治黎陽城棓步頃翻使疏吏翻下同}
齊主善之命守宰各設棓以誅屬請之使久之都官中郎宋軌奏曰|{
	中郎當作郎中五代志都官郎中掌畿内非爲得失事北齊制也}
若受使請賕猶致大戮身爲枉灋何以加罪乃罷之司都功曹張老上書請定齊律|{
	司都功曹司州之功曹也時都鄴以鄴爲司州治所按北齊主或居晋陽不常居鄴也}
詔右僕射薛琡等取魏麟趾格更討論損益之|{
	麟趾格見一百五十八卷武帝大同七年琡昌六翻}
齊主簡練六坊之人每一人必當百人任其臨陳必死|{
	魏齊之間六軍宿衛之士分爲六坊任保任也陳讀曰陣}
然後取之謂之百保鮮卑|{
	百保言其勇可保一人當百人也高氏以鮮卑創業當時號爲健鬬故衝士皆用鮮卑猶今北人謂勇士爲霸都魯也}
又簡華人之勇力絶倫者謂之勇士以備邊要|{
	邊要邊上要害之地}
始立九等之戶|{
	戶有上中下三等每等又分上中下是為九等}
富者稅其錢貧者役其力 九月丁巳魏軍發長安|{
	發長安而東伐齊}
王僧辯軍至鸚鵡|{
	鸚鵡洲在江夏江中昔黄祖使禰衡賦鸚鵡賦於此洲因以得名洲之下即黄鵠磯}
郢州司馬劉龍虎等潜送質於僧辯|{
	質音致}
邵陵王綸聞之遣其子威正侯礩將兵擊之|{
	礩職日翻}
龍虎敗奔于僧辯綸以書責僧辯曰將軍前年殺人之姪|{
	謂殺河東王譽也}
今歲伐人之兄|{
	綸於繹兄也}
以此求榮恐天下不許僧辯送書於湘東王繹繹命進軍辛酉綸集其麾下於西園|{
	園在郢城西偏故曰西園又有東園在城東東湖上}
涕泣言曰我本無佗志在滅賊湘東常謂與之爭帝遂爾見伐今日欲守則交絶糧儲欲戰則取笑千載|{
	載子亥翻}
不容無事受縛當於下流避之麾下壯士爭請出戰綸不從與礩自倉門登舟北出|{
	據姚思亷梁書倉門郢城北門帶江阻險}
僧辯入據郢州繹以南平王恪爲尚書令開府儀同三司世子方諸爲郢州刺史王僧辯爲領軍將軍綸遇鎮東將軍裴之高於道之高之子畿掠其軍器綸與左右輕舟奔武昌澗飲寺|{
	武昌今壽昌軍是也}
僧灋馨匿綸於巖宂之下綸長史韋質司馬姜律等聞綸尚在馳往迎之說七柵流民以求糧仗|{
	時流民於北江州結七柵以相保說式芮翻}
綸出營巴水流民八九千人附之稍收散卒屯于齊昌|{
	據魏收志梁武帝置北江州治鹿城關領義陽齊昌新昌梁安齊興光城郡五代志黄州木蘭縣梁曰梁安郡又有義陽郡後齊置湘州後改曰北江州則齊昌亦當在木蘭縣界唐省木蘭入黄岡縣宋白曰吳置蘄春郡晋惠帝改西陽郡南齊北齊改西陽爲齊昌郡唐爲蘄州}
遣使請和于齊|{
	使疏吏翻}
齊以綸爲梁王 湘東王繹改封皇子大欵爲臨川王大成爲桂陽王大封爲汝南王 癸亥魏軍至潼關 庚午齊主如晋陽命太子殷居凉風堂監國|{
	據北史齊樂陵王百年傳凉風堂在鄴宫玄都苑監工衘翻}
南郡王中兵參軍張彪等|{
	南郡王大連之鎮會稽也以張彪爲中兵參軍}
起兵於若邪山|{
	若邪山在今越州東南四十里}
攻破浙東諸縣有衆數萬吳郡人陸令公等說太守南海王大臨往依之|{
	說式芮翻}
大臨曰彪若成功不資我力如其橈敗|{
	橈奴教翻杜預曰橈曲也勢屈爲橈}
以我自解|{
	言將歸罪於大臨以自解於侯景}
不可往也任約進寇西陽武昌初寜州刺史彭城徐文盛募兵

數萬人討侯景湘東王繹以爲秦州刺史|{
	五代志江都郡六合縣舊置秦郡後齊置秦州抑梁已置之歟}
使將兵東下與約遇於武昌繹以廬陵王應爲江州刺史以文盛爲長史行府州事督諸將拒之|{
	將即亮翻下同}
應續之子也|{
	廬陵王續卒於武帝太清元年}
邵陵王綸引齊兵未至移營馬柵距西陽八十里|{
	西陽即今黄州黄岡縣古之邾城也}
任約聞之遣儀同叱羅子通等將鐵騎二百襲之|{
	叱羅虜複姓騎奇寄翻}
綸不爲備策馬亡走時湘東王繹亦與齊連和故齊人觀望不助綸定州刺史田祖龍迎綸綸以祖龍爲繹所厚懼爲所執復歸齊昌|{
	復扶又翻}
行至汝南魏所署汝南城主李素綸之故吏也開城納之|{
	魏收志郢州有汝南郡治上蔡縣五代志竟陵郡舊置郢州所領有漢東縣舊曰上蔡則汝南城即漢東縣城也又按姚思亷梁書汝南治安陸重城宋白曰晋汝南郡人流寓夏口因僑立汝南郡汝南縣於潼口荆湘記云金水北岸有汝南舊城是也}
任約遂據西陽武昌 |{
	考異曰梁帝紀在十一月今從太清紀}
裴之高帥子弟部曲千餘人至夏首|{
	帥讀曰率夏戶雅翻}
湘東王繹召之以爲新興永寜二郡太守|{
	新興郡置於江陵縣界永寧郡置於襄陽南漳縣界}
又以南平王恪爲武州刺史鎮武陵|{
	武陵唐爲朗州至我朝改爲鼎州}
初邵陵王綸以衡陽王獻爲齊州刺史鎮齊昌任約擊擒之送建康殺之 |{
	考異曰梁帝紀在十一月今從太清紀}
獻暢之孫也|{
	暢武帝之孫}
乙亥進侯景位相國封二十郡爲漢王加殊禮岳陽王詧還襄陽|{
	自朝魏而還也前已書詧梁王矣今復書詧舊爵以義例言之合改正}
黎州民攻刺史張賁賁弃城走|{
	五代志義城郡梁曰黎州唐之利州是也}


州民引氐酋北益州刺史楊法琛據黎州|{
	魏以武興爲東益州氐王楊氏居之梁盖以爲北益州按下卷楊法琛治平興則梁置北益州於平興也酋慈秋翻}
命王賈二姓詣武陵王紀請法琛爲刺史紀深責之囚法琛質子崇顒崇虎|{
	質音致顒魚容翻}
冬十月丁丑朔法琛遣使附魏|{
	使疏吏翻}
己卯齊主至晋陽宫|{
	晋陽宫齊獻武王所置唐志晋陽宫在北都之西北宫城周二千五百二十步都城左汾右晋潜丘在中長四千三百二十一步廣三千一百二十二步周萬五千一百五十三步汾東曰東城}
廣武王長弼與并州刺史段韶不恊齊主將如晋陽長弼言於帝曰韶擁彊兵在彼恐不如人意|{
	言恐韶爲變}
豈可徑往投之帝不聽既至以長弼語告之曰如君忠誠人猶有讒况其餘乎長弼永樂之弟也|{
	高永樂不内高昂使之喪元長弼又讒段韶高歡父子爲失刑矣}
乙酉以特進元韶爲尚書左僕射段韶爲右僕射 乙未侯景自加宇宙大將軍都督六合諸軍事以詔文呈上上驚曰將軍乃有宇宙之號乎 立皇子大鈞爲西陽王大威爲武寧王大球爲建安王大昕爲義安王|{
	簫子顒齊志寧蠻府所領有義安郡}
大摯爲綏建王|{
	沈約志宋文帝元嘉十三年立綏建郡於漢南海郡四會縣也}
大圜爲樂梁|{
	樂梁史無所考此時諸王所封皆郡名也當在大同中所分二十餘州不知處所之數 考異曰太清紀在十一月十四日今從梁帝紀}
齊東徐州刺史行臺辛術鎮下邳十一月侯景徵租入建康術帥衆度淮斷之|{
	帥讀曰率斷音短}
燒其穀百萬石遂圍陽平景行臺郭元建引兵救之壬戍術略三千餘家還下邳 武陵王紀帥諸軍發成都 |{
	考異曰南史云十一月壬寅按是月壬子朔無壬寅}
湘東王繹遣使以書止之曰蜀人勇悍易動難安|{
	使疏吏翻易弋䜴翻}
弟可鎮之吾自當滅賊又别紙曰地擬孫劉各安境界情深魯衛書信恒通|{
	地挺孫劉欲吳蜀各爲一國也情深魯衛謂兄弟也恒戶登翻}
甲子南平王恪帥文武拜牋推湘東王繹爲相國總百揆繹不許 魏丞相泰自弘農爲橋濟河至建州丙寅齊主自將出頓東城|{
	即晉陽之東城也將即亮翻}
泰聞其軍容嚴盛歎曰高歡不死矣會久雨自秋及冬魏軍畜產多死乃自蒲阪還於是河南自洛陽河北自平陽已東皆入于齊|{
	邊民見魏師無功齊能自立心無反側疆場遂定}
丁卯徐文盛軍貝磯任約帥水軍逆戰|{
	帥讀曰率}
文盛大破之斬叱羅子通趙威方仍進軍大舉口|{
	水經注江水東過邾縣南東逕白虎磯北又東逕貝磯北又東逕黎磯北北岸烽火洲即舉州也北對舉口舉水出龜頭山逕梁定州城南又逕梁司豫州城東又南歷齊安郡西又東南歷赤亭下分爲二水南流注于江}
侯景遣宋子仙等將兵二萬助約|{
	將即亮翻}
以約守西陽久不能進自出屯晋熙 |{
	考異曰典略七月景軍次濡須使梁仲宣知留府事按典略九月景請梁妃主同宴梁帝紀十月乙未景逼太宗幸西州不容七月已在濡須今因南康王會理事見之太清紀梁書典畧晋熙皆作皖口今從南史}
南康王會理以建康空虚與太子左衛將軍柳敬禮西鄉侯勸東鄉侯勔|{
	勔彌兖翻}
謀起兵誅王偉安樂侯乂理出奔長蘆|{
	今其州六合縣有長蘆鎮及長蘆寺淳熙十二年徙寺於滁口山之東張舜民曰長蘆鎮在滁河西南}
集衆得千餘人建安侯賁中宿世子子邕知其謀|{
	中宿世子中宿侯之世子也沈約曰中宿漢舊縣屬南海郡吳度屬始興郡}
以告偉偉收會理敬禮勸勔及會理弟祁陽侯通理俱殺之|{
	沈約曰祁陽縣吳立屬零陵郡 考異曰典畧云十二月癸未建安侯賁等告會理按梁帝紀十月壬寅景害會理今從太清紀}
乂理爲左右所殺錢塘禇冕以會理故舊捶掠千計|{
	捶止蘂翻掠音亮}
終無異言會理隔壁謂之曰禇郎卿豈不爲我致此|{
	爲于僞翻}
卿雖忍死明我我心實欲殺賊冕竟不服景乃宥之勸昺之子賁正德之弟子子邕憺之孫也|{
	昺音丙憺徒敢翻又徒濫翻}
帝自即位以來景防衛甚嚴外人莫得進見|{
	見賢遍翻}
唯武林侯諮及僕射王克舍人殷不害並以文弱得出入卧内帝與之講論而已及會理死克不害懼禍稍自疎諮獨不離帝|{
	離力智翻}
朝請無絶景惡之使其仇人刁戍刺殺諮於廣莫門外|{
	朝直遥翻惡烏路翻刺七亦翻 考異曰太清紀在會理死前今從南史}
帝之即位也景與帝登重雲殿|{
	據梁紀重雲殿在華林園項安世曰梁華林園重雲殿前置銅儀重直龍翻}
禮佛爲誓云自今君臣兩無猜貳臣固不負陛下陛下亦不得負臣及會理謀泄景疑帝知之故殺諮帝自知不久指所居殿謂殷不害曰龐涓當死此下景自帥衆討楊白華于宣城白華力屈而降景以其北人|{
	楊白華大眼之子魏胡太后私幸之白華懼禍奔梁帥讀曰率華讀曰花降戶江翻}
全之以爲左民尚書誅其兄子彬以報來亮之怨|{
	楊白華殺來亮見上卷太清三年}
十二月丙子朔景封建安侯賁爲竟陵王中宿世子子邕爲隨王仍賜姓侯氏|{
	侯景賞其告會理之功也}
辛丑齊主還鄴|{
	自晋陽還也}
邵陵王綸在汝南脩城池

集士卒將圖安陸魏安州刺史馬祐以告丞相泰|{
	五代志安陸郡梁置南司州西魏改曰安州今爲德安府}
泰遣楊忠將萬人救安陸|{
	將即亮翻}
武陵王紀遣潼州刺史楊乾運南梁州刺史譙淹合

兵二萬討楊法琛|{
	元和郡縣志梓潼郡梁置潼州潼音同}
法琛兵據劒閣以拒之 侯景還建康|{
	自晉熙還也}
初魏敬宗以爾朱榮爲柱國大將軍|{
	柱國大將軍魏初官也世祖以加太尉長孫嵩}
位在丞相上榮敗此官遂廢大統三年文帝復以丞相泰爲之|{
	復扶又翻}
其後功參佐命望實俱重者亦居此官凡八人曰安定公宇文泰廣陵王欣趙郡公李弼隴西公李虎河内公獨孤信南陽公趙貴常山公于謹彭城公侯莫陳崇謂之八柱國泰始籍民之才力者爲府兵身租庸調一切蠲之|{
	唐府兵之法本諸此凡受田之丁歲輸粟謂之租隨鄉所出每丁歲輸絹綾絁布緜麻非□鄉則歲輸銀謂之調用人之力歲二十日閏加二日不役者日爲絹三尺謂之庸調徒弔翻}
以農隙講閱戰陳馬畜糧備六家供之合為百府每府一郎將主之|{
	將即亮翻}
分屬二十四軍泰任總百揆督中外諸軍欣以宗室宿望從容禁闥而已|{
	陳讀曰陣從千容翻}
餘六人各督二大將軍凡十二大將軍 |{
	平王元贊淮王元育齊王元廓章武郡公宇文導平原郡公侯莫陳順高陽郡公達奚武陽平公李遠范陽公豆盧寧化政公宇文貴博陵公賀蘭祥陳留公楊忠武威公王雄凡十二人皆使持節大將軍}
每大將軍各統開府二人開府各領一軍是後功臣位至柱國大將軍開府儀同三司儀同三司者甚衆率為散官無所統御雖有繼掌其事者聞望皆出諸公之下云|{
	散悉亶翻聞音問}
齊主命散騎侍郎宋景業造天保歷行之|{
	時齊主命景業叶圖䜟造天保歷景業奏依握誠圖及元命包言齊受禪之期當魏終之紀得乘三十五以為蔀應六百七十六以為章齊主大悦乃施用之}


資治通鑑卷一百六十三
