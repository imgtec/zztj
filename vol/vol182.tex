資治通鑑卷一百八十二
宋 司馬光 撰

胡三省 音註

隋紀六|{
	起昭陽作噩盡旃蒙大淵獻凡三年}


煬皇帝中

大業九年春正月丁丑詔徵天下兵集涿郡始募民為驍果|{
	為驍果作逆張本驍古堯翻}
修遼東古城以貯軍糧|{
	漢晉以來遼東郡皆治襄平慕容氏始鎮平郭前伐高麗圍遼東言即漢襄平城今言復修古城盖城郭有遷徙也貯下呂翻}
靈武賊帥白瑜娑|{
	帝改靈州為靈武郡帥所類翻娑桑何翻 考異曰隋書作白榆妄今從略記}
劫掠牧馬北連突厥|{
	厥九勿翻}
隴右多被其患|{
	被皮義翻}
謂之奴賊 戊戍赦天下 己亥命刑部尚書衛文昇等輔代王侑留守西京|{
	是後李淵得以尊立代王守手又翻}
二月壬午詔宇文述以兵糧不繼遂陷王師|{
	事見上卷上年}
乃軍吏失於支科非述之罪宜復其官爵 |{
	考異曰雜記在去年十二月今從隋書}
尋又加開府儀同三司 帝謂侍臣曰高麗小虜侮慢上國今抜海移山猶望克果|{
	克能也果决也麗力知翻}
况此虜乎乃復議伐高麗|{
	復扶又翻}
左光祿大夫郭榮諫曰戎狄失禮臣下之事千鈞之弩不為鼷鼠發機|{
	杜襲諫曹操嘗有是言鼷音奚小鼠也}
柰何親辱萬乘以敵小寇乎|{
	乘䋲證翻}
帝不聽 二月丙子濟隂孟海公起為盗保㨿周橋|{
	帝改曹州為濟隂郡濟子禮翻}
衆至數萬見人稱引書史輒殺之 丁丑發丁男十萬城大興|{
	大興城西京城}
戊寅帝幸遼東命民部尚書樊子蓋等|{
	開皇三年改度支尚書為戶部尚書帝又改為民部尚書併曹郎亦改之}
輔越王侗留守東都|{
	是後遂階王世充僭竊侗他紅翻}
時所在盗起齊郡王薄孟讓北海郭方預清河張金稱平原郝孝德河間格謙勃海孫宣雅各聚衆攻剽|{
	帝改青州為北海郡瀛州為河間郡滄州為勃海郡姓苑格姓允格之後東觀漢記有侍御史東平相格班稱尺證翻郝呼各翻格古百翻剽匹妙翻}
多者十餘萬少者數萬人|{
	少詩沼翻}
山東苦之天下承平曰久人不習戰郡縣吏每與賊戰望風沮敗|{
	沮在呂翻}
唯齊郡丞閺鄉張須陀|{
	隋志閺鄉縣屬河南郡本湖城開皇十六年改焉閺音旻}
得士衆心勇决善戰將郡兵擊王薄於泰山下|{
	隋志泰山在魯郡慱城縣須陀蓋越郡城而戰將即亮翻}
薄恃其驟勝不設備須陀掩擊大破之薄收餘兵北度河須陀追擊於臨邑又破之|{
	隋志臨邑縣屬齊郡}
薄北連孫宣雅郝孝德等十餘萬攻章丘|{
	隋志章丘縣舊曰高唐開皇十六年改焉亦屬齊郡}
須陀帥步騎二萬擊之賊衆大敗賊帥裴長才等衆二萬掩至城下大掠須陀未暇集兵帥五騎與戰|{
	帥讀曰率賊帥所類翻騎奇寄翻}
賊競赴之圍百餘重身中數創|{
	重直龍翻中竹仲翻創初良翻}
勇氣彌厲會城中兵至賊稍却須陀督衆擊之長才等敗走庚子郭方預等合軍攻陷北海大掠而去須陀謂官屬曰賊恃其彊謂我不能救吾今速行破之必矣乃簡精兵倍道進擊大破之斬數萬級前後獲賊輜重不可勝計|{
	重直用翻勝音升}
歷城羅士信|{
	歷城縣舊置濟南郡隋為齊郡治所}
年十四從須陀擊賊於濰水上|{
	隋志濰水在北海郡下密縣濰音維}
賊始布陳士信馳至陳前刺殺數人|{
	陳讀曰陣下同刺七亦翻}
斬一人首擲空中以稍盛之掲以畧陳賊徒愕眙莫敢近|{
	矟色角翻盛受也時征翻掲其謁翻擔掲也眙丑吏翻近其靳翻}
須陀因引兵奮擊賊衆大潰士信逐北每殺一人劓其鼻懷之|{
	劓魚器翻}
還以驗殺賊之數須陀歎賞引置左右每戰須陀先登士信為副帝遣使慰諭并畫須陀士信戰陳之狀而觀之|{
	使疏吏翻畫與晝通}
夏四月庚午車駕度遼壬申遣宇文述與上大將軍楊義臣趣平壤|{
	趣七喻翻}
左光祿大夫王仁恭出扶餘道仁恭進軍至新城|{
	新城在南蘇城之西}
高麗兵數萬拒戰仁恭帥勁騎一千擊破之高麗嬰城固守帝命諸將攻遼東聽以便宜從事|{
	麗力知翻帥讀曰率騎奇寄翻將即亮翻}
飛樓橦雲梯地道四面俱進|{
	橦宅江翻}
晝夜不息而高麗應變拒之二十餘日不拔主客死者甚衆|{
	守者為主攻者為客}
衝梯竿長十五丈驍果吳興沈光升其端|{
	隋志吳郡烏程縣舊置吳興郡史以舊郡名書長直亮翻驍堅堯翻下同}
臨城與高麗戰短兵接殺十數人高麗競擊之而墜未及地適遇竿有垂絙|{
	絙古恒翻索也}
光接而復上|{
	復扶又翻上時掌翻}
帝望見壯之即拜朝散大夫|{
	朝直遙翻散悉但翻}
恒置左右|{
	恒戶登翻}
禮部尚書楊玄感驍勇便騎射好讀書喜賓客|{
	騎奇寄翻好呼到翻下同喜許記翻}
海内知名之士多與之遊與蒲山公李密善密弼之曾孫也|{
	李密襲爵蒲山郡公蒲山郡闕李弼宇文氏佐命功臣}
少有才略志氣雄遠輕財好士為左親侍|{
	隋制左右翊衛府領親勲武三侍左親侍屬左翊衛少詩照翻}
帝見之謂宇文述曰向者左仗下黑色小兒聸視異常勿令宿衛述乃諷密使稱病自免密遂屏人事|{
	屏必郢翻}
專務讀書嘗乘黄牛讀漢書楊素遇而異之因召至家與語大悦謂其子玄感等曰李密識度如此汝等不及也由是玄感與為深交時或侮之密曰人言當指實寧可面諛若决機兩陳之間喑嗚咄嗟使敵人震懾|{
	陳讀曰陣喑於今翻咄當没翻懾之涉翻}
密不如公驅策天下賢俊各申其用公不如密豈可以階級稍崇而輕天下士大夫邪玄感笑而服之|{
	李密事始此}
素恃功驕倨朝宴之際或失臣禮|{
	朝直遙翻下司}
帝心銜而不言素亦覺之及素薨帝謂近臣曰使素不死終當族滅玄感頗知之且自以累世貴顯在朝文武多父之故吏見朝政日紊|{
	紊音問}
而帝多猜忌内不自安乃與諸弟濳謀作亂帝方事征伐玄感自言世荷國恩願為將領帝喜曰將門必有將相門必有相固不虚也|{
	荷下可翻將即亮翻相悉亮翻}
由是寵遇日隆頗預朝政帝伐高麗命玄感於黎陽督運遂與虎賁郎將王仲伯汲郡贊治趙懷義等謀|{
	按隋志帝改州為郡郡置太守罷長史司馬置贊務一人以貳之贊務即贊治也隋書成於唐臣避高宗名故改治為務治直吏翻}
故逗遛漕運不時進發|{
	逗音豆遛音留}
欲令度遼諸軍乏食帝遣使者促之|{
	令力丁翻使疏吏翻}
玄感揚言水路多盗不可前後而發玄感弟虎賁郎將玄縱鷹揚郎將萬石並從幸遼東玄感濳遣人召之二人皆亡還萬石至高陽|{
	高陽縣屬河間郡}
為監事許華所執|{
	按唐六典武庫署太倉署皆有監事蓋因隋制也監工銜翻}
斬於涿郡時右驍衛大將軍來護兒以舟師自東萊將入海趣平壤玄感遣家奴偽為使者從東方來詐稱護兒反|{
	驍堅堯翻}
六月乙巳玄感入黎陽閉城大索男夫|{
	索山客翻}
取帆布為牟甲|{
	帆施於船上以汎風時軍興織蒲不給擬布為之牟兜牟也}
署官屬皆凖開皇之舊|{
	不用帝所改官制}
移書傍郡以討護兒為名各令發兵會於倉所|{
	倉謂黎陽倉}
郡縣官有幹用者玄感皆以運糧追集之以趙懷義為衛州刺史東光尉元務本為黎州刺史河内郡主簿唐禕為懷州刺史|{
	復開皇之制以郡為州以太守為刺史隋志州郡皆置主簿東光縣屬平原郡宋白曰今定遠軍治東光縣漢舊縣也故城在今縣東二十里齊天保七年移於今縣東南三十里陶氏故城隋開皇三年又移於後魏廢勃海郡城縣置令丞尉正禕許韋翻考異曰雜記作懷州司功書佐今從隋書}
治書侍御史游元|{
	隋御史臺置治書侍御史二員帝增為正五品治直之翻}
督運在黎陽玄感謂曰獨夫肆虐陷身絶域此天亡之時也我今親帥義兵以誅無道卿意如何元正色曰尊公荷國寵靈近古無比公之弟兄青紫交映當謂竭誠盡節上荅鴻恩豈意墳土未乾|{
	帥讀曰率荷下可翻乾音干}
親圖反噬僕有死而已不敢聞命玄感怒而囚之屢脅以兵不能屈乃殺之元明根之孫也|{
	游明根元魏太和中以儒名}
玄感選運夫少壯者得五千餘人|{
	少詩照翻}
丹陽宣城篙梢三千餘人|{
	帝改蔣州為丹陽郡改宣州為宣城郡篙梢習於用舟者篙音高進船竿也}
刑三牲誓衆且諭之曰主上無道不以百姓為念天下騷擾死遼東者以萬計今與君等起兵以救兆民之弊何如衆皆踴躍稱萬歲乃勒兵部分|{
	分扶問翻}
唐禕自玄感所逃歸河内先是玄感隂遣家僮至長安|{
	先悉薦翻}
召李密及弟玄挺赴黎陽及舉兵密適至玄感大喜以為謀主謂密曰子常以濟物為己任今其時矣計將安出密曰天子出征遠在遼外去幽州猶隔千里南有巨海北有強胡中間一道理極艱危公擁兵出其不意長驅入薊據臨渝之險|{
	臨渝關隋屬平州盧龍縣即所謂盧龍之險也顔師古曰渝音喻今多讀如榆}
扼其咽喉|{
	咽於賢翻}
歸路既絶高麗聞之必躡其後不過旬月資糧皆盡其衆不降則潰可不戰而擒此上計也|{
	麗力知翻降戶江翻}
玄感曰更言其次密曰關中四塞天府之國雖有衛文昇不足為意今帥衆鼓行而西|{
	帥讀曰率下同}
經城勿攻直取長安收其豪傑撫其士民據險而守之天子雖還失其根本可徐圖也玄感曰更言其次密曰簡精鋭晝夜倍道襲取東都以號令四方但恐唐禕告之先已固守|{
	禕許韋翻}
若引兵攻之百日不克天下之兵四面而至非僕所知也玄感曰不然今百官家口並在東都若先取之足以動其心且經城不拔何以示威|{
	其後玄感攻弘農自速敗亡其識度已見於此}
公之下計乃上策也遂引兵向洛陽遣楊玄挺將驍勇千人為前鋒|{
	將即亮翻下同驍堅堯翻}
先取河内唐禕㨿城拒守玄挺無所獲禕又使人告東都越王侗與樊子蓋等勒兵為備|{
	果如李密所料侗他紅翻}
修武民相帥守臨清關|{
	修武縣屬河内郡}
玄感不得度乃於汲郡南渡河從之者如市使弟積善將兵三千自偃師南緣洛水西入|{
	隋志偃師縣舊廢開皇十六年復置屬河南郡}
玄挺自白司馬坂逾邙山南入|{
	白司馬坂在邙山北邙山在洛城北坂音反}
玄感將三千餘人隨其後相去十里許自稱大軍其兵皆執單刀柳楯|{
	楯食尹翻干也以扜弓矢}
無弓矢甲胄東都遣河南令達奚善意|{
	隋志河南縣帶河南郡}
將精兵五千人拒積善將作監河南贊治裴弘策將八千人拒玄挺|{
	治直吏翻}
善意度洛南營於漢王寺明日積善兵至不戰自潰鎧仗皆為積善所取弘策出至白司馬坂一戰敗走棄鎧仗者大半|{
	鎧可亥翻}
玄挺亦不追弘策退三四里收散兵復結陳以待之|{
	復扶又翻陳讀曰陣}
玄挺徐至坐息良久忽起擊之弘策又敗如是五戰丙辰玄挺直抵太陽門弘策將十餘騎馳入宫城自餘無一人返者皆歸於玄感玄感屯上春門|{
	隋志河南郡舊置洛州大業元年移都改曰豫州東面三門北曰上春中曰建陽無太陽門當考將即亮翻 考異曰玄感傳云屯兵上春門又云屯兵尚書省按劉仁軌河洛記東都羅郭東面北頭第一曰上春門唐改曰上東門又尚書省在宣仁門内玄感不容至此}
每誓衆曰我身為上柱國家累鉅萬金至於富貴無所求也今不顧滅族者但為天下解倒懸之急耳|{
	為于偽翻}
衆皆悦父老爭獻牛酒子弟詣軍門請自効者日以千數内史舍人韋福嗣洸之兄子也|{
	韋洸著聲績於平陳之後洸古黄翻}
從軍出拒玄感為玄感所獲玄感厚禮之使與其黨胡師耽共掌文翰玄感令福嗣為書遺樊子蓋數帝罪惡|{
	遺于季翻數所具翻}
云今欲廢昏立明願勿拒小禮自貽伊戚樊子蓋新自外藩入為京官|{
	樊子蓋自涿郡留守為東都留守}
東都舊官多慢之至於部分軍事未甚承稟|{
	分扶問翻}
裴弘策與子蓋同班|{
	贊治次留守立班故言同班}
前出討賊失利子蓋更使出戰不肯行子蓋命引出斬之以狥國子祭酒河東楊汪小有不恭|{
	楊汪傳本弘農華隂人曾祖順徙河東}
子蓋又將斬之汪頓首流血乃得免於是將吏震肅無敢仰視|{
	是將子亮翻}
令行禁止玄感盡鋭攻城子蓋隨方拒守玄感不能克然達官子弟應募從軍者聞弘策死皆不敢入城韓擒虎子世咢|{
	咢五各翻}
觀王雄子恭道虞世基子柔來護兒子淵裴藴子爽大理卿鄭善果子儼周羅㬋子仲等四十餘人皆降於玄感|{
	降戶江翻下同}
玄感悉以親重要任委之善果譯之兄子也|{
	鄭譯高祖佐命}
玄感收兵得五萬餘人分五千守慈磵道|{
	隋志河南郡壽安縣有慈磵}
五千守伊闕道|{
	隋志河南郡伊闕縣舊曰新城開皇十八年改名以伊闕山名縣也}
遣韓世咢將三千人圍滎陽|{
	隋志滎陽縣屬滎陽郡郡治管城滎陽在郡西}
顧覺將五千人取虎牢|{
	隋志滎陽郡汜水縣舊曰成臯即虎牢也開皇十八年改曰汜水大業初置虎牢都尉府}
虎牢降|{
	降戶江翻}
以覺為鄭州刺史鎮虎牢代王侑使刑部尚書衛文昇帥兵四萬救東都|{
	代王侑時守長安帥讀曰率 考異曰隋書云步騎七萬按玄感衆不過十萬而下云衆寡不敵今從雜記}
文昇至華隂掘楊素冢|{
	隋志華隂縣屬京兆郡楊素世居華隂死葬焉華戶化翻}
焚其骸骨示士卒以必死遂鼓行出崤澠直趨東都城北|{
	崤崤谷澠澠池澠彌兖翻趨七喻翻}
玄感逆拒之文昇且戰且行屯於金谷|{
	金谷即晉石崇之金谷也水經注穀水自千金堨東逕睪門橋又東左會金谷水水出太白原東南流歷金谷謂之金谷澗在河南縣界}
遼東城久不拔帝遣造布囊百餘萬口滿貯土|{
	貯丁呂翻}
欲積為魚梁大道|{
	築道若魚梁然}
闊三十步高與城齊使戰士登而攻之又作八輪樓車|{
	樓車下施八輪}
高出於城夾魚梁道欲俯射城内|{
	射而亦翻}
指期將攻城内危蹙會楊玄感反書至帝大懼引納言蘇威入帳中謂曰此兒聰明得無為患威曰人識是非審成敗乃謂之聰明|{
	夫音扶}
玄感麤疎必無所慮但恐因此寖成亂階耳帝又聞達官子弟皆在玄感所益憂之兵部侍郎斛斯政素與玄感善玄感之反政與之通謀玄縱兄弟亡歸政濳遣之帝將窮治玄縱等黨與|{
	治直之翻}
政内不自安戊辰亡奔高麗|{
	史言段文振之言驗麗力知翻}
庚午夜二更|{
	二更乙夜也甲乙丙丁戊分為五夜守卒分番守漏鳴鼓以相警謂之持更更工衡翻}
帝密召諸將使引軍還|{
	還從宣翻又音如字}
軍資器械攻具積如丘山營壘帳幕案堵不動皆棄之而去衆心忷懼|{
	忷許拱翻}
無復部分諸道分散|{
	復扶又翻部分扶問翻}
高麗即時覺之然不敢出但於城内鼓譟至來日午時方漸出外四遠覘偵|{
	覘丑亷翻又丑塹翻偵丑鄭翻}
猶疑隋軍詐之經二日乃出數千兵追躡畏隋兵之衆不敢逼常相去八九十里將至遼水知御營畢度乃敢逼後軍時後軍猶數萬人高麗隨而抄擊最後羸弱數千人為所殺略|{
	麗力知翻抄楚交翻羸倫為翻}
初帝再征高麗復問太史令庾質曰今段何如|{
	今段言自今以後一段事也復扶又翻}
對曰臣實愚迷猶執前見|{
	庾質前對見上卷八年}
陛下若親動萬乘勞費實多|{
	乘繩證翻}
帝怒曰我自行猶不克直遣人去安得有功及還謂質曰卿前不欲我行當為此耳|{
	還從宣翻又音如字為于偽翻}
玄感其有成乎質曰玄感地勢雖隆素非人望因百姓之勞冀幸成功今天下一家未易可動|{
	易以䜴翻}
帝遣虎賁郎將陳稜攻元務本於黎陽又遣左翊衛大將軍宇文述右候衛將軍屈突通乘傳發兵以討玄感|{
	屈區勿翻傳株戀翻}
來護兒至東萊聞玄感圍東都召諸將議旋軍救之諸將咸以無敕不宜擅還固執不從護兒厲聲曰洛陽被圍心腹之疾高麗逆命猶疥癬耳|{
	將即亮翻被皮義翻疥癬皮膚之疾}
公家之事知無不為專擅在吾不關諸人有沮議者軍法從事|{
	沮在呂翻}
即日迴軍令子弘整馳驛奏聞帝時還至涿郡已敕護兒救東都見弘整甚悦|{
	按隋書來護兒傳弘整護兒之二子}
賜護兒璽書曰公旋師之時是朕敕公之日君臣意合遠同符契先是右武候大將軍李子雄坐事除名今從軍自効從來護兒在東萊帝疑之|{
	按隋書子雄傳因玄感反而疑之璽斯氏翻先悉薦翻}
詔鎻子雄送行在所子雄殺使者逃奔玄感|{
	使疏吏翻}
衛文昇以步騎二萬度瀍水|{
	水經瀍水出河南穀城縣北山東過洛陽偃師而入于洛騎奇寄翻瀍音㕓}
與玄感戰玄感屢破之玄感每戰身先士卒|{
	先悉薦翻}
所向摧陷又善撫悦其下皆樂為致死|{
	樂音洛為于偽翻}
由是每戰多捷衆益盛至十萬人文昇衆寡不敵死傷大半且盡 |{
	考異曰雜記曰每戰刃纔接官軍皆坐地棄甲以白布裹頭聽賊所掠前後十三戰皆不利今從文昇傳}
乃更進屯邙山之陽與玄感决戰一日十餘合會楊玄挺中流矢死|{
	中竹仲翻}
玄感軍乃稍却秋七月癸未餘杭民劉元進起兵以應玄感元進手長尺餘|{
	手長尺餘言自指至掌腕横紋之長長直亮翻}
臂垂過膝|{
	言臂垂則其手過膝過古禾翻}
自以相表非常|{
	相息亮翻}
隂有異志會帝再發三吳兵征高麗三吳兵皆相謂曰往歲天下全盛吾輩父兄征高麗者猶大半不返|{
	此指八年事}
今已罷弊復為此行|{
	罷讀曰疲復扶又翻}
吾屬無遺類矣由是多亡命郡縣捕之急聞元進舉兵亡命者雲集旬月間衆至數萬始楊玄感至東都自謂天下響應得韋福嗣委以心膂不復專任李密|{
	復扶又翻}
福嗣每畫策皆持兩端密揣知其意|{
	揣初委翻}
謂玄感曰福嗣元非同盟實懷觀望明公初起大事而姦人在側聽其是非必為所誤請斬之玄感曰何至於此密退謂所親曰楚公好反而不欲勝|{
	玄感襲爵楚國公故稱之好呼到翻}
吾屬今為虜矣李子雄勸玄感速稱尊號玄感以問密密曰昔陳勝自欲稱王張耳諫而被外|{
	事見七卷秦二世元年被皮義翻}
魏武將求九錫荀彧止而見誅|{
	事見六十六卷漢獻帝建安十七年}
今者密欲正言還恐追縱二子阿諛順意又非密之本圖何者兵起以來雖復頻捷|{
	復扶又翻下蓋復同}
至於郡縣未有從者東都守禦尚彊天下救兵益至公當挺身力戰早定關中廼亟欲自尊何示人不廣也|{
	亟已力翻}
玄感笑而止屈突通引兵屯河陽宇文述繼之玄感問計於李子雄子雄曰通曉習兵事若一得度河則勝負難决不如分兵拒之通不能濟則樊衛失援|{
	樊衛謂樊子蓋衛文昇也}
玄感然之將拒通樊子蓋知其謀數擊其營玄感不得往|{
	斯亦伐謀之一也使援兵不合樊子蓋堅守都城兵何由解數所角翻下軍數同}
通濟河軍於破陵|{
	破陵當在河陽南岸洛城東北}
玄感分為兩軍西抗文昇東拒通子蓋復出兵大戰玄感軍屢敗與其黨謀之李子雄曰東都援軍益至我軍數敗不可久留不如直入關中開永豐倉以振貧乏|{
	新唐志華隂縣有永豐倉蓋隋所置也}
三輔可指麾而定|{
	此指言漢三輔之地}
據有府庫東面而爭天下亦霸王之業也李密曰弘化留守元弘嗣握彊兵在隴右|{
	帝改慶州為弘化郡其地屬隴右}
可聲言其反遣使迎公因此入關可以紿衆|{
	使疏吏翻紿徒亥翻}
會華隂諸楊請為鄉導|{
	華隂諸楊玄感之宗黨也華戶化翻鄉讀曰嚮}
壬辰玄感解東都圍引兵西趣潼關|{
	潼關在華隂縣趣七喻翻}
宣言我已破東都取關西矣宇文述等諸軍躡之至弘農宫|{
	隋志河南郡陜縣後魏置陜州恒農郡開皇初廢郡大業初廢州置弘農宫恒農即弘農後魏避諱改曰恒農}
父老遮說玄感曰宫城空虚又多積粟攻之易下|{
	說式芮翻易以豉翻}
玄感以為然弘農太守蔡王智積|{
	觀此則帝廢陜州改為弘農郡也智積高祖弟整之子守式又翻}
謂官屬曰玄感聞大將軍至欲西圖關中若成其計則難克也當以計縻之使不得進不出一旬可以成擒及玄感軍至城下智積登陴詈之|{
	陴符支翻城上女垣詈力智翻罵也}
玄感怒留攻之李密諫曰公今詐衆西入軍事貴速况乃追兵將至安可稽留若前不得據關|{
	謂據潼關也}
退無所守大衆一散何以自全玄感不從遂攻之燒其城門智積於内益火玄感兵不得入三日不拔乃引而西至閺鄉|{
	閺音旻}
宇文述衛文昇來護兒屈突通等軍追及於皇天原|{
	水經注玉澗水南出王溪北流逕皇天原西周固記開山東首上平慱方可里餘三面壁立高千許仭漢世祭天於其上名之為皇天原上有漢武思子臺又北逕閺鄉城西又有全鳩澗水出南山北逕皇天原東隋志閺鄉縣有玉澗全鳩澗}
玄感上槃豆|{
	即西魏將于謹所攻拔盤豆也上時掌翻}
布陳亘五十里|{
	陳讀曰陣下同亘古鄧翻}
且戰且行玄感一日三敗八月壬寅玄感陳於董杜原諸軍擊之玄感大敗獨與十餘騎奔上洛|{
	帝改商州為上洛郡玄感欲由華陽以奔上洛也騎奇寄翻下同}
追騎至玄感叱之皆反走至葭蘆戍|{
	葭音加}
獨與弟積善徒步走自度不免|{
	度徒洛翻}
謂積善曰我不能受人戮辱汝可殺我積善抽刀斫殺之因自刺不死|{
	刺七亦翻}
為追兵所執與玄感首俱送行在所磔玄感尸於東都市三日復臠而焚之|{
	磔陟格翻復扶又翻臠力兖翻}
玄感弟玄奬為義陽太守|{
	隋志義陽郡齊梁之司州後魏改曰郢州後周改曰申州大業初改義州尋改為義陽郡守式又翻}
將赴玄感為郡丞周旋玉所殺|{
	隋書楊玄感傳作周旋玉}
仁行為朝請大夫伏誅於長安|{
	楊素之門於是滅矣朝直遙翻}
玄感之圍東都也梁郡民韓相國舉兵應之|{
	帝改宋州為梁郡相息亮翻}
玄感以為河南道元帥|{
	帥所類翻}
旬月間衆十餘萬攻剽郡縣至襄城|{
	隋志襄城郡東魏置北荆州後周改曰和州開皇初改為伊州大業初改為汝州尋改為郡剽匹妙翻}
聞玄感敗衆稍散為吏所獲傳首東都帝以元弘嗣斛斯政之親也留守弘化郡|{
	隋書元弘嗣傳云屯兵安定}
遣衛尉少卿李淵馳往執之|{
	少始照翻}
因代為留守關右十三郡兵皆受徵發|{
	十三郡天水隴西金城枹罕臨洮漢陽靈武朔方平凉弘化延安雕隂上郡也}
淵御衆寛簡人多附之帝以淵相表奇異|{
	相息譛翻}
又名應圖䜟忌之|{
	䜟楚譛翻}
未幾徵詣行在所淵遇疾未謁|{
	謁見也幾居豈翻}
其甥王氏在後宫帝問曰汝舅來何遲王氏以疾對帝曰可得死否淵聞之懼因縱酒納賂以自晦|{
	李淵事始此}
癸卯吳郡朱爕晉陵管崇聚衆寇掠江左|{
	隋志吳郡陳置吳州}


|{
	平陳改曰蘇州大業初復曰吳州尋改為吳郡毗陵郡平陳置常州帝改為晉陵郡}
爕本還俗道人涉獵經史頗知兵灋形容眇小為崑山縣慱士|{
	劉昫曰崑山本漢婁縣地梁分婁縣置信義縣又分信義置崑山縣取縣界山名時屬吳郡隋制縣慱士不見于志蓋在曹佐市令之下}
與數十學生起兵民苦役者赴之如歸崇長大美姿容志氣倜儻|{
	倜他狄翻}
隱居常熟|{
	常熟吳晉南沙之地本屬吳縣晉分吳縣置海虞縣梁置常熟縣劉昫曰今崑山縣東一百三十里常熟故城是也隋治南沙城屬吳郡}
自言有王者相|{
	相息亮翻}
故羣盗相與奉之時帝在涿郡命虎牙郎將趙六兒將兵萬人屯楊子|{
	楊子地名時屬江都縣將即亮翻下同}
分為五營以備南賊|{
	南賊謂劉元進及崇爕等}
崇遣其將陸顗度江夜襲六兒破其兩營收其器械軍資而去|{
	顗魚豈翻}
衆益盛至十萬 辛酉司農卿雲陽趙元淑坐楊玄感黨伏誅|{
	隋書楊玄感傳作博陵趙元淑此言雲陽隋志博陵郡舊定州雲陽縣屬京兆郡地理相去遠甚當考}
帝使大理卿鄭善果御史大夫裴藴刑部侍郎骨儀與留守樊子蓋推玄感黨與|{
	帝置六部侍郎以貳尚書之職守式又翻推尋也考鞠也 考異曰雜記作滑儀今從隋書雜記推玄感黨在十月疑太晩今因誅趙元淑言之}
儀本天竺胡人也|{
	隋書隂壽傳言骨儀京兆長安人蓋本天竺胡人而居京兆長安也}
帝謂藴曰玄感一呼而從者十萬|{
	呼火故翻}
益知天下人不欲多多即相聚為盗耳不盡加誅無以懲後子蓋性既殘酷藴復受此旨|{
	復扶又翻}
由是峻灋治之|{
	治直之翻}
所殺三萬餘人皆籍没其家枉死者大半流徙者六千餘人玄感之圍東都也開倉賑給百姓凡受米者皆阬之於都城之南玄感所善文士會稽虞綽琅邪王胄|{
	隋志會稽郡梁置東楊州平陳改曰吳州大業初改越州尋復改為會稽郡琅邪郡舊置北徐州後周改曰沂州帝復改為琅邪郡會古外翻邪讀曰耶}
俱坐徙邊綽胄亡命捕得誅之帝善屬文|{
	屬之欲翻}
不欲人出其右薛道衡死|{
	道衡死見上卷五年}
帝曰更能作空梁落燕泥否王胄死帝誦其佳句曰庭草無人隨意綠復能作此語邪|{
	復扶又翻}
帝自負才學每驕天下之士嘗謂侍臣曰天下皆謂朕承籍緒餘而有四海設令朕與士大夫高選亦當為天子矣|{
	令力丁翻}
帝從容謂秘書郎虞世南曰我性不喜人諫|{
	從千容翻喜許記翻}
若位望通顯而諫以求名彌所不耐至於卑賤之士雖少寛假然不置之地上|{
	少詩沼翻卒子恤翻}
汝其知之世南世基之弟也 帝使裴矩安集隴右因之會寧存問曷薩那可汗部落|{
	八年帝處曷薩那部落于會寧今遣矩存問之薩桑葛翻可從刋入聲汗音寒}
遣闕度設寇掠吐谷渾以自富還而奏狀帝大賞之|{
	吐從暾入聲谷音浴還音旋又如字賞稱奬也}
九月己卯東海民彭孝才起為盗|{
	帝改海州為東海郡}
有衆數萬 甲午車駕至上谷|{
	隋志開皇元年以易縣置易州帝改為上谷郡按秦置上谷郡本治沮易王隱晉書地道志曰郡在谷之頭故因以上谷名焉隋之易縣則漢涿郡故安縣地也非古上谷}
以供費不給免太守虞荷等官|{
	守武又翻}
閏月己巳幸博陵 冬十月丁丑賊帥呂明星圍東郡|{
	東郡古白馬之地隋開皇九年置杞州十六年改滑州大業二年改為兖州尋改為東郡帥所類翻}
虎賁郎將費青奴擊破之|{
	賁音奔將即亮翻陳湘姓林曰費氏音蜚夏禹之後趙明誠曰費字有兩姓音讀不同源流亦異其一音蜚嬴姓出於伯益之後史記所載費昌費中楚費無極漢費將軍費直費長房費禕之徒是其後也其一音秘出於魯季友姓苑所載琅邪費氏則是其後也}
劉元進帥其衆將度江|{
	帥讀曰率}
會楊玄感敗朱爕管崇共迎元進推以為主㨿吳郡稱天子爕崇俱為尚書僕射署置百官毗陵東陽會稽建安豪傑多執長吏以應之|{
	帝改婺州為東陽郡大業初改泉州為閩州尋改為建安郡長知兩翻會工外翻}
帝遣左屯衛大將軍代人吐萬緒|{
	隋書吐萬緒代郡鮮卑人吐萬蓋鮮卑複姓也隋志大業初於馬邑善陽縣置代郡}
光祿大夫下邽魚俱羅|{
	隋志下邽縣屬馮翊郡風俗通魚姓宋公子魚之後}
將兵討之|{
	將即亮翻又音如字}
十一月己酉右候衛將軍馮孝慈討張金稱於清河孝慈敗死|{
	稱尺證翻帝改貝州為清河郡}
楊玄感之西也韋福嗣亡詣東都歸首|{
	嗣祥吏翻首手又翻}
是時如其比者皆不問樊子蓋收玄感文簿得其書草|{
	即福嗣所草遺子蓋之書}
封以呈帝帝命執送行在李密亡命為人所獲亦送東都 |{
	考異曰隋書密傳云密間行入關與玄感從叔詢相隨匿於馮翊詢妻之舍尋為鄰人所告遂捕獲囚於京兆獄又云及出關外防禁漸弛又云至邯鄲密等七人皆穿墻而遁唐書雖不云囚於京兆獄亦云出關按密若自關中送高陽不當與韋福嗣同行今從賈閏甫蒲山公傳及劉仁軌河洛行年記}
樊子蓋鎻送福嗣密及楊積善王仲伯等十餘人詣高陽密與王仲伯等竊謀亡去悉使出其所齎金以示使者曰吾等死日此金並留付公幸用相瘞|{
	使疏吏翻下同瘞於計翻}
其餘即皆報德使者利其金許諾防禁漸弛密請通市酒食每宴飲諠譁竟夕使者不以為意行至魏郡石梁驛飲防守者皆醉穿墻而逸|{
	帝改相州為魏郡飲防之飲於禁翻 考異曰河洛記曰左梁驛今從蒲山公傳}
密呼韋福嗣同去福嗣曰我無罪天子不過一面責我耳至高陽帝以書草示福嗣收付大理宇文述奏凶逆之徒臣下所當同疾若不為重灋無以肅將來帝曰聽公所為十二月甲申述就野外縛諸應刑者於格上以車輪括其頸使文武九品以上皆持兵斫射|{
	射而亦翻}
亂發矢如蝟毛支體糜碎猶在車輪中積善福嗣仍加車裂皆焚而揚之積善自言手殺玄感冀得免死帝曰然則梟類耳因更其姓曰梟氏|{
	梟食母說文曰不孝鳥更工衡翻梟古堯翻}
唐縣人宋子賢善幻術|{
	隋志唐縣屬博陵郡幻術者化無為有以眩惑人幻戶辦翻}
能變佛形自稱彌勒出世遠近信惑遂謀因無遮大會舉兵襲乘輿|{
	天子曰乘輿乘繩證翻}
事泄伏誅并誅黨與千餘家扶風桑門向海明亦自稱彌勒出世人有歸心者輒獲吉夢由是三輔人翕然奉之|{
	帝改岐州為扶風郡扶風漢三輔之一也}
因舉兵反衆至數萬丁亥海明自稱皇帝改元白烏詔太僕卿楊義臣擊破之 帝召衛文昇樊子蓋詣行在慰勞之賞賜極厚遣還所任|{
	賞其拒討楊玄感之功遣各還留臺勞力到翻}
劉元進攻丹陽|{
	隋書吐萬緒傳云時元進以兵攻潤州按唐志武德三年始以延陵縣地}


|{
	置潤州而潤州管下丹陽縣本曲阿亦唐改名元進所攻蓋此丹陽非隋志之丹陽郡隋之丹陽郡治石頭城隋書成於武德之後書潤州書丹陽皆以唐州縣書之也}
吐萬緒濟江擊破之元進解圍去緒進屯曲阿|{
	觀此及以隋書證之則元進所攻正潤州}
元進結栅拒緒相持百餘日緒擊之賊衆大潰死者以萬數元進挺身夜遁保其壘朱爕管崇等屯毗陵連營百餘里緒乘勝進擊復破之賊退保黄山|{
	隋志吳縣有黄山}
緒圍之元進爕僅以身免於陳斬崇及其將卒五千餘人|{
	陳讀曰陣將即亮翻}
收其子女三萬餘口進解會稽圍|{
	會古外翻}
魚俱羅與緒偕行戰無不捷然百姓從亂者如歸市賊敗而復聚|{
	復扶又翻下同}
其勢益盛元進退據建安帝令緒進討緒以士卒疲弊請息甲待來春 |{
	考異曰帝紀云緒與俱羅連年不能克按緒請待來春而王世充十年又擊孟讓然則元進敗正在今年冬春之交矣元進退據建安而得拒世充於江上者蓋復來也}
帝不悦俱羅亦以賊非歲月可平諸子在洛京|{
	洛陽為東都故謂之洛京}
濳遣家僕迎之帝怒有司希旨奏緒怯懦俱羅敗衂|{
	懦乃卧翻又奴亂翻衂女六翻}
俱羅坐斬徵緒詣行在緒憂憤道卒|{
	卒子恤翻}
帝更遣江都丞王世充|{
	王世充為江都郡丞帝又改郡贊治為丞更江衡翻}
發淮南兵數萬人討元進世充度江頻戰皆捷元進爕敗死於吳|{
	隋志吳縣吳郡治所}
其餘衆或降或散世充召先降者於通玄寺瑞像前焚香為誓約降者不殺|{
	降戶江翻}
散者始欲入海為盗聞之旬月之間歸首略盡|{
	首手又翻}
世充悉阬之於黄亭澗死者三萬餘人 |{
	考異曰略記阬其衆二十餘萬於黄亭澗澗長數里深濶數丈積屍與之平雜記世充貪而無信利在子女資財並阬歸首八千餘人於黄山之下今從隋書}
由是餘黨復相聚為盗官軍不能討以至隋亡帝以世充有將帥才益加寵任|{
	將即亮翻帥所類翻}
是歲詔為盗者籍没其家時羣盗所在皆滿郡縣官因之各專威福生殺任情矣章丘杜伏威與臨濟輔公祏為刎頸交|{
	章丘臨濟二縣隋志皆屬}


|{
	齊郡章丘漢陽丘縣宋魏之高唐也開皇十六年改為章丘宋白曰高齊天保七年移高唐縣治古黄巾城隋改章丘縣因縣東南章丘為名臨濟漢之管縣久廢開皇六年置朝陽縣十六年改曰臨濟輔姓出於晉大夫輔躒又智果别族為輔氏顔師古曰刎斷也刎頸交言託契深重雖斷頸絶頭無所顧也濟子禮翻祏音石刎武粉翻}
俱亡命為羣盗|{
	唐書杜伏威傳公祏數盗姑家牧羊以餽伏威縣迹捕急乃相與亡命為盗}
伏威年十六每出則居前入則殿後|{
	殿丁練翻}
由是其徒推以為帥下邳苗海潮亦聚衆為盗|{
	隋志下邳郡後魏置南徐州梁改為東徐州東魏改為東楚州陳改為安州後周改為泗州帝改為下邳郡風俗通曰楚大夫伯棼之後賁皇奔晉食采於苗因為苗氏}
伏威使公祏謂之曰今我與君同苦隋政各舉大義力分勢弱常恐被擒|{
	被皮義翻}
若合為一則足以敵隋矣君能為主吾當敬從自揆不堪宜來聽命不則一戰以决雌雄|{
	不讀曰否}
海潮懼即帥其衆降之|{
	帥讀曰率降戶江翻}
伏威轉掠淮南自稱將軍江都留守遣校尉宋顥討之|{
	帝置鷹揚府郎將副郎將每府置越騎校尉二人掌騎士步兵校尉二人掌步兵守式又翻校戶教翻}
伏威與戰陽為不勝引顥衆入葭葦中因從上風縱火顥衆皆燒死海陵賊帥趙破陳以伏威兵少輕之|{
	海陵漢縣屬臨淮郡梁置海陵郡開皇初廢郡為縣隋志屬江都郡帥所類翻陳讀曰陣少詩沼翻}
召與并力伏威使公祏嚴兵居外自與左右十人齎牛酒入謁於座殺破陳并其衆|{
	史言杜伏威寖強}


十年春 |{
	考異曰雜記是年正月又以許公宇文述為元帥將兵十六萬刻到鴨綠水乙支文德遣行人為請降以緩我師又求與述相見以觀我軍形勢述與之歡飲良久乃去停五日王師食盡燒甲札食之病不能興文德乃縱兵大戰敗績死者十餘萬此蓋序八年事誤在此耳}
二月辛未詔百僚議伐高麗數日無敢言者戊子詔復徵天下兵|{
	復扶又翻}
百道俱進 丁酉扶風賊帥唐弼立李弘芝為天子|{
	帥所類翻考異曰隋帝紀作李弘今從唐書薛舉傳}
有衆十萬自稱唐王 三月壬

子帝行幸涿郡士卒在道亡者相繼癸亥至臨渝宫|{
	隋志北平郡盧龍縣有臨渝宫諭音俞又音榆}
禡祭黄帝|{
	鄭玄曰禡師祭也在野曰禡應劭曰黄帝戰于阪泉以定天下故祭以求福祥杜佑曰禡師祭也為兵禱也其神蓋蚩尤或云黄帝北齊之制天子親征將屇戰所卜剛日備玄牲列軍容設於辰地為墠而禡祭大司馬奠矢有司奠毛血樂奏大濩之音禮畢徹牲柴燎按記王制天子出征禡於所征之地其禮亡矣杜佑所載者北齊之禮耳禡馬嫁翻}
斬叛軍者以釁鼓|{
	斬人以血塗鼓}
亡者亦不止 夏四月榆林太守成紀董純|{
	帝改勝州為榆林郡隋志成紀縣屬天水郡守式又翻}
與彭城賊帥張大虎戰於昌慮|{
	帝改徐州為彭城郡昌慮漢縣晉宋魏志皆有之隋已廢省其地當入彭城郡蘭陵縣界帥所類翻下同慮音廬}
大破之斬首萬餘級 甲午車駕至北平|{
	帝改平州為北平郡}
五月庚申延安賊帥劉迦論|{
	隋志延安郡後魏置東夏州西魏改為延州帝改為延安郡迦音加 考異曰唐書作安定人按安定去上郡太遠今從隋書}
自稱皇王建元大世有衆十萬與稽胡相表裏為寇詔以左驍衛大將軍屈突通為關内討捕大使發兵擊之戰於上郡|{
	隋志上郡後魏置東秦州後改為北華州西魏改為敷州大業二年改為鄜城郡後改為上郡驍堅堯翻屈居勿翻使疏吏翻鄜音膚}
斬迦論將卒萬餘級|{
	將即亮翻}
虜男女數萬口而還|{
	還從宣翻又如字}
秋七月癸丑車駕次懷遠鎮時天下已亂所徵兵多失期不至高麗亦困弊來護兒至畢奢城|{
	即卑沙城自登萊海道趨平壤先至卑沙城唐貞觀未程名振亦由此道麗力知翻}
高麗舉兵逆戰護兒擊破之將趣平壤|{
	趣七喻翻}
高麗王元懼甲子遣使乞降囚送斛斯政|{
	斛斯政去年奔高麗使疏吏翻下同降戶江翻}
帝大悦遣使持節召護兒還護兒集衆曰大軍三出未能平賊此還不可復來|{
	還從宣翻復扶又翻}
勞而無功吾竊耻之今高麗實困以此衆擊之不日可克吾欲進兵徑圍平壤取高元獻捷而歸不亦善乎答表請行不肯奉詔長史崔君肅固爭|{
	長知兩翻}
護兒不可曰賊勢破矣獨以相任自足辦之吾在閫外事當專决寧得高元還而獲譴捨此成功所不能矣君肅告衆曰若從元帥違拒詔書必當聞奏皆應獲罪諸將懼俱請還乃始奉詔|{
	帥所類翻下賊帥將帥同將即亮翻下同}
八月己巳帝自懷遠鎮班師邯鄲賊帥楊公卿帥其黨八千人|{
	隋志邯鄲縣屬武安郡帥讀曰率}
抄駕後第八隊得飛黄上廐馬四十二匹而去|{
	抄楚交翻帝置殿内省統尚食尚藥尚衣尚舍尚乘尚輦等六局尚乘局置左右六閑一曰左右飛黄閑二左右吉良閑三左右龍媒閑四左右騊駼閑五左右駃騠閑六左右天苑閑}
冬十月丁卯上至東都己丑還西京以高麗使者及斛斯政告太廟仍徵高麗王元入朝元竟不至|{
	朝直遙翻下同}
敕將帥嚴裝更圖後舉竟不果行初開皇之末國家殷盛朝野皆以高麗為意劉炫獨以為不可|{
	炫榮絹翻}
作撫夷論以刺之至是其言始驗十一月丙申殺斛斯政於金光門外|{
	金光門大興城西面三門之中門}
如楊積善之法|{
	去年殺楊積善}
仍烹其肉使百官噉之佞者或噉之至飽|{
	噉與啗同徒濫翻}
收其餘骨焚而揚之 乙巳有事于南郊上不齋于次|{
	鄭氏曰次自修正之處}
詰朝備法駕|{
	漢仍秦制大駕八十一乘法駕三十六乘隋開皇中大駕十二乘法駕减半帝更定其制大駕用三十六法駕用十二詰去吉翻}
至即行禮是日大風上獨獻上帝三公分獻五帝禮畢御馬疾驅而歸 乙卯離石胡劉苗王反|{
	隋志離石郡後齊置西汾州後周改為汾州帝改離石郡}
自稱天子衆至數萬將軍潘長文討之不克|{
	長知兩翻}
汲郡賊帥王德仁擁衆數萬保林慮山為盗|{
	隋志汲郡東魏置義州後周為衛州帝改汲郡林慮山在魏郡林慮縣帥所類翻下同慮音廬}
帝將如東都太史令庾質諫曰比歲伐遼|{
	比毗至翻}
民實勞弊陛下宜鎮撫關中使百姓盡力農桑三五年間四海稍豐實然後廵省|{
	省悉景翻}
於事為宜帝不悦質辭疾不從|{
	從才用翻}
帝怒下質獄|{
	下遐嫁翻}
竟死獄中十二月壬申帝如東都赦天下戊子入東都 東海賊帥彭孝才轉掠沂水|{
	劉昫曰沂水漢東莞縣隋改為東安縣尋改為沂水屬琅邪郡}
彭城留守董純討擒之|{
	守式又翻}
純戰雖屢捷而盗賊日滋或譖純怯懦|{
	懦乃卧翻又乃亂翻}
帝怒鎻純詣東都誅之 孟讓自長白山寇掠諸郡至盱眙衆十餘萬據都梁宫|{
	隋志盱眙縣屬江都郡有都梁山都梁宫在焉山出都梁香故名盱眙音吁怡}
阻淮為固江都丞王世充將兵拒之為五栅以塞險要|{
	將即亮翻下同塞悉則翻}
羸形示弱|{
	羸倫為翻}
讓笑曰世充文法小吏安能將兵吾今生縳取鼓行入江都耳時民皆結堡自固野無所掠賊衆漸餒乃少留兵圍五栅分人於南方抄掠世充伺其懈|{
	少詩沼翻抄楚交翻伺相吏翻懈居隘翻}
縱兵出擊大破之讓以數十騎遁去|{
	騎奇寄翻下同}
斬首萬餘級 齊郡賊帥左孝友衆十萬屯蹲狗山|{
	蹲狗者以形得名蹲徂尊翻}
郡丞張須陀列營逼之孝友窘迫出降須陀威振東夏|{
	降戶江翻窘巨隕翻夏戶雅翻}
以功遷齊郡通守|{
	帝罷州置郡郡置太守其後諸郡各加置通守一人位次太守守式又翻}
領河南道十二郡黜陟討捕大使涿郡賊帥盧明月衆十餘萬軍祝阿|{
	隋志祝阿縣屬齊郡唐改為禹城縣 考異曰唐秦叔寶傳作下邳今從隋書}
須陀將萬人邀之|{
	將即亮翻下謂將分將同}
相持十餘日糧盡將退謂將士曰賊見吾退必悉衆來追若以千人襲據其營可有大利此誠危事誰能往者衆莫對唯羅士信及歷城秦叔寶請行|{
	隋志歷城縣帶齊郡}
於是須陀委栅而遁使二人分將千兵伏葭葦中明月悉衆追之士信叔寶馳至其栅柵門閉二人超升其樓各殺數人營中大亂二人斬關以納外兵因縱火焚其三十餘栅烟焰漲天明月奔還須陀回軍奮擊大破之明月以數百騎遁去所俘斬無筭叔寶名瓊以字行

十一年春正月增祕書省官百二十員|{
	隋制秘書省監丞各一人郎四人校書郎十二人正字四人著作郎二人佐郎八人校書郎正字各二人帝增少監一人减校書郎為十人加置佐郎四人又置儒林郎十人文林郎二十人增校書郎員四十人加置楷書郎員二十人凡百一十七人}
並以學士補之帝好讀書著述自為揚州總管|{
	開皇十年帝為揚州總管好呼到翻}
置王府學士至百人常令修撰以至為帝前後近二十載|{
	近其靳翻載作亥翻}
修撰未嘗暫停自經術文章兵農地理醫卜釋道乃至蒱博鷹狗|{
	蒱音蒲摴蒱也}
皆為新書無不精洽共成三十一部萬七千餘卷初西京嘉則殿有書三十七萬卷帝命祕書監柳顧言等詮次除其複重猥雜|{
	詮此緣翻說文具也重直龍翻}
得正御本三萬七千餘卷納於東都修文殿又寫五十副本簡為三品分置西京東都宫省官府其正書皆裝翦華淨寶軸錦褾|{
	褾方小翻卷端也}
於觀文殿前為書室十四間窻戶牀褥厨幔|{
	幔莫半翻}
咸極珍麗每三間開方戶垂錦幔上有二飛仙戶外地中施機發帝幸書室有宫人執香爐|{
	香爐始於漢漢官典職曰尚書郎給女史二人著潔衣服執香爐燒薰}
前行踐機則飛仙下收幔而上|{
	踐慈演翻上時掌翻}
戶扉及厨扉皆自啓帝出則垂閉復故 帝以戶口逃亡盗賊繁多二月庚午詔民悉城居田隨近給郡縣驛亭村塢皆築城 上谷賊帥王須拔自稱漫天王|{
	帥所類翻幔謨官翻}
國號燕賊帥魏刀兒自稱歷山飛衆各十餘萬北連突厥南寇燕趙|{
	燕因肩翻}
初高祖夢洪水没都城意惡之|{
	惡烏路翻}
故遷都大興|{
	開皇三年遷新都}
申明公李穆薨|{
	穆薨見一百七十六卷陳長城公至德四年}
孫筠襲爵叔父渾忿其吝嗇使兄子善衡賊殺之而證其從父弟瞿曇|{
	從才用翻曇徒含翻}
使之償死渾謂其妻兄左衛率宇文述曰若得紹封當歲奉國賦之半述為之言於太子|{
	率所律翻為于偽翻}
奏高祖以渾為穆嗣二歲之後不復以國賦與述|{
	復扶又翻}
述大恨之帝即位渾累官至右驍衛大將軍改封郕公|{
	驍堅堯翻}
帝以其門族彊盛忌之會有方士安伽陁|{
	伽求加翻}
言李氏當為天子勸帝盡誅海内凡李姓者渾從子將作監敏小名洪兒|{
	從才用翻}
帝疑其名應䜟常面告之冀其引决敏大懼數與渾及善衡屏人私語|{
	數所角翻屏必郢翻}
述譖之於帝仍遣虎賁郎將河東裴仁基表告渾反|{
	帝改蒲州為河東郡賁音奔將即亮翻下同}
帝收渾等家遣尚書左丞元文都御史大夫裴藴雜治之|{
	治直之翻下同}
按問數日不得反狀以實奏聞帝更遣述窮治之述誘教敏妻宇文氏為表誣告渾謀因度遼與其家子弟為將領者共襲取御營立敏為天子述持入奏之帝泣曰吾宗社幾傾|{
	幾居依翻}
賴公獲全耳三月丁酉殺渾敏善衡及宗族三十二人自三從以上皆徙邊徼|{
	從才用翻徼吉弔翻}
後數月敏妻亦鴆死|{
	敏妻宇文氏周天元之女帝之姊子也}
有二孔雀自西苑飛集寶城朝堂前|{
	西苑在洛城西元年所築也後唐兵攻王世充世充使其弟世偉守寶城則寶城在洛城羅郭之内自為一城既朝堂在焉則百司廨署皆在焉自為一城附於宫城之東南也唐因隋制亦以洛陽為東京六典云東城在皇城之東皇城在東城之内百僚廨署如京城之制皇宫在皇城之北吾以此推之皇城蓋即隋之寶城在宫城東南也朝直遙翻下同}
親衛校尉高德儒等十餘人見之奏以為鸞|{
	帝置親侍鷹揚府領親勲武三侍三侍即三衛也各置越騎校尉步兵校尉 考異曰雜記云五年三月馬德儒奏孔雀為鸞今年月及姓皆從畧記并温大雅創業起居注}
時孔雀已飛去無可得驗於是百僚稱賀詔以德儒誠心冥會肇見嘉祥擢拜朝散大夫|{
	朝直遙翻散悉亶翻}
賜物百段餘人皆賜束帛仍於其地造儀鸞殿 己酉帝行幸太原夏四月幸汾陽宫避暑宫城迫隘百官士卒布散山谷間結草為營而居之 以衛尉少卿李淵為山西河東撫慰大使|{
	少始照翻使疏吏翻 考異曰創業注云帝自衛尉少卿轉右驍衛將軍奉詔為太原道安撫大使即隋大業十二年煬帝幸樓煩時也按十二年帝未嘗幸樓煩今從高祖實錄在幸汾陽宫時 余按隋志汾陽宫正屬樓煩郡自可以言幸樓煩但有十二年十一年之差耳}
承制黜陟選補郡縣文武官仍發河東兵討捕羣盗淵行至龍門擊賊帥毋端兒破之|{
	劉昫曰龍門漢艾氏縣後魏改為龍門隋志屬河東郡帥所類翻毋音無姓也}
秋八月乙丑帝廵北塞 |{
	考異曰雜記六月突厥賊入嵐城鎮抄掠遣范安貴討}


|{
	擊之王師敗績安貴死百司震懼七月帝幸鴈門先至天池值雨山谷泥深二尺從官狼狽帳幕多不至一夜並露坐雨中至曉多死宫人無食貸糒於衛士今從隋書}
初裴矩以突厥始畢可汗部衆漸盛|{
	厥九勿翻可從刋入聲汗音寒}
獻策分其勢欲以宗女嫁其弟叱吉設拜為南面可汗叱吉不敢受始畢聞而漸怨突厥之臣史蜀胡悉多謀略為始畢所寵任矩詐與為互市誘至馬邑下殺之|{
	帝改朔州為馬邑郡誘音酉}
遣使詔始畢曰史蜀胡悉叛可汗來降我已相為斬之|{
	使疏吏翻降戶江翻為于偽翻}
始畢知其狀由是不朝戊辰始畢帥騎數十萬謀襲乘輿|{
	朝直遙翻帥讀曰率騎奇寄翻下同乘繩證翻下同}
義成公主先遣使者告變壬申車駕馳入鴈門|{
	隋志鴈門郡後周置肆州開皇五年改為代州帝改鴈門郡}
齊王以後軍保崞縣|{
	崞漢縣後魏置石城縣開皇十年改曰平寇大業初改為崞縣屬鴈門郡古限翻崞音郭}
癸酉突厥圍鴈門上下惶怖|{
	怖普布翻}
撤民屋為守禦之具城中兵民十五萬口食僅可支二旬鴈門四十一城突厥克其三十九唯鴈門崞不下突厥急攻鴈門矢及御前上大懼抱趙王杲而泣目盡腫左衛大將軍宇文述勸帝簡精鋭數千騎潰圍而出納言蘇威曰城守則我有餘力輕騎乃彼之所長陛下萬乘之主豈宜輕動|{
	乘繩證翻}
民部尚書樊子蓋曰陛下乘危徼幸|{
	徼堅堯翻徼幸覬非望也}
一朝狼狽悔之何及不若據堅城以挫其鋭坐徵四方兵使入援陛下親撫循士卒諭以不復征遼|{
	復扶又翻}
厚為勲格必人人自奮何憂不濟内史侍郎蕭瑀以為突厥之俗可賀敦預知軍謀|{
	突厥可汗之妻為可賀敦可從刋入聲}
且義成公主以帝女嫁外夷必恃大國之援若使一介告之借使無益庸有何損又將士之意恐陛下既免突厥之患還事高麗|{
	麗力知翻}
若發明詔諭以赦高麗專討突厥則衆心皆安人自為戰矣瑀皇后之弟也虞世基亦勸帝重為賞格下詔停遼東之役帝從之帝親廵將士謂之曰努力擊賊|{
	將即亮翻下將士同}
苟能保全凡在行陳|{
	行戶剛翻陳讀曰陣並下同}
勿憂富貴必不使有司弄刀筆破汝勲勞乃下令守城有功者無官直除六品賜物百段有官以次增益使者慰勞相望於道|{
	使疏吏翻勞力到翻}
於是衆皆踊躍晝夜拒戰死傷甚衆甲申詔天下募兵守令競來赴難|{
	守式又翻難乃旦翻}
李淵之子世民年十六應募隸屯衛將軍雲定興說定興多齎旗鼓為疑兵|{
	說式苪翻}
曰始畢敢舉兵圍天子必謂我倉猝不能赴援故也宜晝則引旌旗數十里不絶夜則鉦鼓相應虜必謂救兵大至望風遁去不然彼衆我寡若悉軍來戰必不能支定興從之帝遣間使求救於義成公主|{
	間古莧翻使疏吏翻下同}
公主遣使告始畢云北邊有急東都及諸郡援兵亦至忻口|{
	九域志忻州秀容縣有忻口寨隋志秀容屬樓煩郡杜佑曰隋置忻州因忻口為名}
九月甲辰始畢解圍去帝使人出偵山谷皆空無胡馬|{
	偵丑鄭翻}
乃遣二千騎追躡至馬邑得突厥老弱二千餘人而還丁未車駕還至太原|{
	帝改并州為太原郡騎奇寄翻還從宣翻又如字}
蘇威言於帝曰今盗賊不息士馬疲弊願陛下亟還西京深根固本為社稷計帝初然之宇文述曰從官妻子多在東都|{
	亟紀力翻從才用翻}
宜便道向洛陽自潼關而入帝從之冬十月壬戌帝至東都 |{
	考異曰畧記九月辛未帝入東都今從隋帝記}
顧盻街衢|{
	盻莫甸翻斜視}
謂侍臣曰猶大有人在意謂曏日平楊玄感殺人尚少故也|{
	少詩沼翻}
蘇威追論勲格太重宜加斟酌樊子蓋固請以為不宜失信帝曰公欲收物情邪|{
	邪音耶}
子蓋懼不敢對帝性吝官賞初平楊玄感應授勲者多乃更置戎秩建節尉為正六品次奮武宣惠綏德懷仁秉義奉誠立信等尉遞降一階將士守鴈門者萬七千人得勲者纔千五百人皆凖平玄感勲一戰得第一勲者進一階其先無戎秩者止得立信尉三戰得第一勲者至秉義尉其在行陳而無勲者四戰進一階亦無賜會仍議伐高麗由是將士無不憤怨|{
	行戶剛翻陳讀曰陣麗力知翻將即亮翻}
初蕭瑀以外戚有才行|{
	行下孟翻}
嘗事帝於東宫累遷至内史侍郎委以機務瑀性剛鯁數言事忤旨|{
	數所角翻忤五故翻}
帝漸疎之及鴈門圍解帝謂羣臣曰突厥狂悖|{
	厥九勿翻悖蒲妹翻又蒲没翻}
勢何能為少時未散|{
	少詩沼翻}
蕭瑀遽相恐動情不可恕出為河池郡守|{
	隋志河池郡後魏置南岐州後周改曰鳳州帝改河池郡守式又翻}
即日遣之候衛將軍楊子崇從帝在汾陽宫知突厥必為寇屢請早還京師帝怒曰子崇怯懦驚動衆心|{
	懦乃卧翻又乃亂翻}
不可居爪牙之官出為離石郡守子崇高祖之族弟也 楊玄感之亂龍舟水殿皆為所焚詔江都更造凡數千艘|{
	艘蘇遭翻}
制度仍大於舊者 壬申盧明月帥衆十萬寇陳汝|{
	陳州淮揚郡汝州襄城郡帥讀曰率}
東海李子通有勇力先依長白山賊帥左才相|{
	帥所類翻}


|{
	相息亮翻}
羣盗皆殘忍而子通獨寛仁由是人多歸之未半歲有衆萬人才相忌之子通引去度淮與杜伏威合伏威選軍中壯士養為假子凡三十餘人濟隂王雄誕臨濟闞稜為之冠|{
	隋志濟隂郡後魏置西兖州後周改曰曹州帝改濟隂郡闞姓也左傳齊有大夫闞止濟子禮翻冠古玩翻}
既而李子通謀殺伏威遣兵襲之伏威被重創墜馬|{
	被皮義翻創初良翻}
雄誕負之逃葭葦中收散兵復振將軍來整擊伏威破之其將西門君儀之妻王氏勇而多力負伏威以逃雄誕帥壯士十餘人衛之|{
	將即亮翻帥讀曰率下同}
與隋兵力戰由是得免來整又擊李子通破之子通帥其餘衆奔海陵復收兵得二萬人自稱將軍 城父朱粲|{
	隋志城父縣屬譙郡父音甫}
始為縣佐史|{
	隋郡縣皆有佐史}
從軍遂亡命聚衆為盗謂之可達寒賊自稱迦樓羅王|{
	迦音加}
衆至十餘萬引兵轉掠荆沔|{
	荆州南郡沔州沔陽郡沔彌兖翻}
及山南郡縣|{
	山南者長安南山之南}
所過噍類無遺|{
	噍才笑翻}
十二月庚寅詔民部尚書樊子蓋發關中兵數萬擊絳賊敬盤陀等|{
	絳賊絳郡賊也風俗通敬姓陳敬仲之後姓苑黄帝孫敬康之後}
子蓋不分臧否|{
	否音鄙}
自汾水之北村塢盡焚之賊有降者皆阬之百姓怨憤益相聚為盗詔以李淵代之有降者淵引置左右由是賊衆多降前後數萬人餘黨散入他郡|{
	降戶江翻}


資治通鑑卷一百八十二
