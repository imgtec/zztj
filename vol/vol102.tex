<!DOCTYPE html PUBLIC "-//W3C//DTD XHTML 1.0 Transitional//EN" "http://www.w3.org/TR/xhtml1/DTD/xhtml1-transitional.dtd">
<html xmlns="http://www.w3.org/1999/xhtml">
<head>
<meta http-equiv="Content-Type" content="text/html; charset=utf-8" />
<meta http-equiv="X-UA-Compatible" content="IE=Edge,chrome=1">
<title>資治通鑒_103-資治通鑑卷一百二_103-資治通鑑卷一百二</title>
<meta name="Keywords" content="資治通鑒_103-資治通鑑卷一百二_103-資治通鑑卷一百二">
<meta name="Description" content="資治通鑒_103-資治通鑑卷一百二_103-資治通鑑卷一百二">
<meta http-equiv="Cache-Control" content="no-transform" />
<meta http-equiv="Cache-Control" content="no-siteapp" />
<link href="/img/style.css" rel="stylesheet" type="text/css" />
<script src="/img/m.js?2020"></script> 
</head>
<body>
 <div class="ClassNavi">
<a  href="/24shi/">二十四史</a> | <a href="/SiKuQuanShu/">四库全书</a> | <a href="http://www.guoxuedashi.com/gjtsjc/"><font  color="#FF0000">古今图书集成</font></a> | <a href="/renwu/">历史人物</a> | <a href="/ShuoWenJieZi/"><font  color="#FF0000">说文解字</a></font> | <a href="/chengyu/">成语词典</a> | <a  target="_blank"  href="http://www.guoxuedashi.com/jgwhj/"><font  color="#FF0000">甲骨文合集</font></a> | <a href="/yzjwjc/"><font  color="#FF0000">殷周金文集成</font></a> | <a href="/xiangxingzi/"><font color="#0000FF">象形字典</font></a> | <a href="/13jing/"><font  color="#FF0000">十三经索引</font></a> | <a href="/zixing/"><font  color="#FF0000">字体转换器</font></a> | <a href="/zidian/xz/"><font color="#0000FF">篆书识别</font></a> | <a href="/jinfanyi/">近义反义词</a> | <a href="/duilian/">对联大全</a> | <a href="/jiapu/"><font  color="#0000FF">家谱族谱查询</font></a> | <a href="http://www.guoxuemi.com/hafo/" target="_blank" ><font color="#FF0000">哈佛古籍</font></a> 
</div>

 <!-- 头部导航开始 -->
<div class="w1180 head clearfix">
  <div class="head_logo l"><a title="国学大师官网" href="http://www.guoxuedashi.com" target="_blank"></a></div>
  <div class="head_sr l">
  <div id="head1">
  
  <a href="http://www.guoxuedashi.com/zidian/bujian/" target="_blank" ><img src="http://www.guoxuedashi.com/img/top1.gif" width="88" height="60" border="0" title="部件查字,支持20万汉字"></a>


<a href="http://www.guoxuedashi.com/help/yingpan.php" target="_blank"><img src="http://www.guoxuedashi.com/img/top230.gif" width="600" height="62" border="0" ></a>


  </div>
  <div id="head3"><a href="javascript:" onClick="javascript:window.external.AddFavorite(window.location.href,document.title);">添加收藏</a>
  <br><a href="/help/setie.php">搜索引擎</a>
  <br><a href="/help/zanzhu.php">赞助本站</a></div>
  <div id="head2">
 <a href="http://www.guoxuemi.com/" target="_blank"><img src="http://www.guoxuedashi.com/img/guoxuemi.gif" width="95" height="62" border="0" style="margin-left:2px;" title="国学迷"></a>
  

  </div>
</div>
  <div class="clear"></div>
  <div class="head_nav">
  <p><a href="/">首页</a> | <a href="/ShuKu/">国学书库</a> | <a href="/guji/">影印古籍</a> | <a href="/shici/">诗词宝典</a> | <a   href="/SiKuQuanShu/gxjx.php">精选</a> <b>|</b> <a href="/zidian/">汉语字典</a> | <a href="/hydcd/">汉语词典</a> | <a href="http://www.guoxuedashi.com/zidian/bujian/"><font  color="#CC0066">部件查字</font></a> | <a href="http://www.sfds.cn/"><font  color="#CC0066">书法大师</font></a> | <a href="/jgwhj/">甲骨文</a> <b>|</b> <a href="/b/4/"><font  color="#CC0066">解密</font></a> | <a href="/renwu/">历史人物</a> | <a href="/diangu/">历史典故</a> | <a href="/xingshi/">姓氏</a> | <a href="/minzu/">民族</a> <b>|</b> <a href="/mz/"><font  color="#CC0066">世界名著</font></a> | <a href="/download/">软件下载</a>
</p>
<p><a href="/b/"><font  color="#CC0066">历史</font></a> | <a href="http://skqs.guoxuedashi.com/" target="_blank">四库全书</a> |  <a href="http://www.guoxuedashi.com/search/" target="_blank"><font  color="#CC0066">全文检索</font></a> | <a href="http://www.guoxuedashi.com/shumu/">古籍书目</a> | <a   href="/24shi/">正史</a> <b>|</b> <a href="/chengyu/">成语词典</a> | <a href="/kangxi/" title="康熙字典">康熙字典</a> | <a href="/ShuoWenJieZi/">说文解字</a> | <a href="/zixing/yanbian/">字形演变</a> | <a href="/yzjwjc/">金 文</a> <b>|</b>  <a href="/shijian/nian-hao/">年号</a> | <a href="/diming/">历史地名</a> | <a href="/shijian/">历史事件</a> | <a href="/guanzhi/">官职</a> | <a href="/lishi/">知识</a> <b>|</b> <a href="/zhongyi/">中医中药</a> | <a href="http://www.guoxuedashi.com/forum/">留言反馈</a>
</p>
  </div>
</div>
<!-- 头部导航END --> 
<!-- 内容区开始 --> 
<div class="w1180 clearfix">
  <div class="info l">
   
<div class="clearfix" style="background:#f5faff;">
<script src='http://www.guoxuedashi.com/img/headersou.js'></script>

</div>
  <div class="info_tree"><a href="http://www.guoxuedashi.com">首页</a> > <a href="/SiKuQuanShu/fanti/">四库全书</a>
 > <h1>资治通鉴</h1> <!--         下载:【右键另存为】即可 --></div>
  <div class="info_content zj clearfix">
  
<div class="info_txt clearfix" id="show">
<center style="font-size:24px;">103-資治通鑑卷一百二</center>
    資治通鑑卷一百二   宋 司馬光 撰<br />
<br />
  胡三省 音註<br />
<br />
  晉記二十四【起屠維大荒落盡上章敦牂凡二年】<br />
<br />
  海西公下<br />
<br />
  太和四年春三月大司馬温請與徐兖二州刺史郗愔江州刺史桓冲豫州刺史袁真等伐燕【慕容恪死温乃伐燕自謂相時而動可以制勝豈知為慕容垂所敗哉郗丑之翻愔挹淫翻】初愔在北府【晉都建康以京口為北府歷陽為西府姑孰為南州】温常云京口酒可飲兵可用【京口兵可用蓋山川風氣然也豈必至謝玄用之而後敵人知畏哉】深不欲愔居之而愔暗於事機乃遺温牋【遺于季翻】欲共奬王室請督所部出河上愔子超為温參軍取視寸寸毁裂乃更作愔牋【更工衡翻】自陳非將帥才不堪軍旅【將即亮翻帥所類翻】老病乞閒地自養勸温并領已所統温得牋大喜即轉愔冠軍將軍會稽内史【冠古玩翻會稽為王國改太守為内史會工外翻】温自領徐兖二州刺史夏四月庚戌温帥步騎五萬發姑孰【帥讀曰率騎奇寄翻】 甲子燕主暐立皇后可足渾氏太后從弟尚書令豫章公翼之女也【從才用翻】 大司馬温自兖州伐燕郗超曰道遠汴水又淺【兵亂之餘汴水填淤未嘗有人浚治故淺汴皮變翻】恐漕運難通温不從六月辛丑温至金鄉【金鄉縣後漢屬山陽郡晉屬高平郡隋屬濟隂郡唐屬兖州我宋屬濟州縣在州東南九十里】天旱水道絶温使冠軍將軍毛虎生鑿鉅野三百里引汶水會于清水【班固地理志汶水出泰山萊蕪縣西南入濟水經注濟水東北入鉅野其故瀆又東北右合洪水洪水上承鉅野薛訓渚謂之桓公瀆濟自是北注杜佑曰濟水因王莽末渠涸不復截河過今東平濟南淄川北海界中有水流入海謂之清河實菏澤汶水合流亦曰濟河蓋因舊名非濟水也汶音問】虎生寶之子也【毛寶預有平蘇峻之功注又見前】温引舟師自清水入河舳艫數百里【舳音逐艫音盧】郗超曰清水入河難以通運【自清水入河皆是泝流又道里回遠故言難以通運】若寇不戰運道又絶因敵為資復無所得【復扶又翻】此危道也不若盡舉見衆直趨鄴城【見賢遍翻趨七喻翻】彼畏公威名必望風逃潰北歸遼碣【碣音竭】若能出戰則事可立決若欲城鄴而守之則當此盛夏難為功力百姓布野盡為官有易水以南必交臂請命矣但恐明公以此計輕鋭勝負難必欲務持重則莫若頓兵河濟【濟子禮翻】控引漕運俟資儲充備至來夏乃進兵雖如賖遲【賖遠也】然期于成功而已捨此二策而連軍北上【上時掌翻】進不速決退必愆乏【愆差爽也乏匱竭也此言糧運】賊因此勢以日月相引漸及秋冬水更澁滯【澁色立翻】且北土早寒三軍裘褐者少【少詩沼翻】恐於時所憂非獨無食而已温又不從【郗超之謀畧豈常人所及哉宜桓温重之也重之而不從其計者直趨鄴城決勝負于一戰温所不敢頓兵河濟以待來年使燕得為備温亦不為也】温遣建威將軍檀玄攻湖陸拔之【湖陸縣前漢曰湖陵屬山陽郡章帝更名湖陸晉分屬高平郡賢曰湖陸故城在今兖州方與縣東南】獲燕寜東將軍慕容忠燕主暐以下邳王厲為征討大都督帥步騎二萬逆戰于黄墟【水經注陳留小黄縣有黄鄉杜預曰外黄縣東有黄城兵亂之後城邑丘墟故曰黄墟帥讀曰率騎奇寄翻】厲兵大敗單馬犇還高平太守徐翻舉郡來降前鋒鄧遐朱序敗燕將傅顔于林渚【水經注華水東逕棐城北即北林亭也春秋諸侯會于棐林以救鄭遇于北林按林鄉故城在新鄭北又有白鴈陂在長社東北林鄉西南敗補邁翻】暐復遣樂安王臧統諸軍拒温【復扶又翻】臧不能抗乃遣散騎常侍李鳳求救於秦【散悉亶翻騎奇寄翻】秋七月温屯武陽【此東武陽也漢屬東郡魏晉屬陽平郡唐改曰朝城縣屬魏州】燕故兖州刺史孫元帥其族黨起兵應温温至枋頭【帥讀曰率枋音方】暐及太傅評大懼謀犇和龍吳王垂曰臣請擊之若其不捷走未晚也暐乃以垂代樂安王臧為使持節南討大都督【使疏吏翻】帥征南將軍范陽王德等衆五萬以拒温垂表司徒左長史申胤黄門侍郎封孚尚書郎悉羅騰皆從軍【悉羅騰蓋夷人以部落為氏如魏書官氏志所載神元時餘部諸姓内入者叱羅氏如羅氏之類】胤鍾之子孚放之子也【申鍾見九十五卷成帝咸和九年封放見九十九卷穆帝永和七年】暐又遣散騎侍郎樂嵩請救于秦許賂以虎牢以西之地秦王堅引羣臣議于東堂皆曰昔桓温伐我至灞上【見九十九卷永和十年】燕不救我今温伐燕我何救焉且燕不稱藩于我我何為救之王猛密言於堅曰燕雖彊大慕容評非温敵也若温舉山東進屯洛邑收幽冀之兵引并豫之粟觀兵崤澠【澠彌兖翻】則陛下大事去矣今不如與燕合兵以退温温退燕亦病矣然後我承其弊而取之不亦善乎【王猛之取李儼其計亦出此】堅從之八月遣將軍苟池洛州刺史鄧羌帥步騎二萬以救燕出自洛陽軍至潁川【潁川郡治許昌】又遣散騎侍郎姜撫報使于燕【使疏吏翻】以王猛為尚書令太子太傅封孚問于申胤曰温衆彊士整乘流直進今大軍徒逡巡高岸兵不接刃未見克殄之理事將何如胤曰以温今日聲勢似能有為然在吾觀之必無成功何則晉室衰弱温專制其國晉之朝臣未必皆與之同心【朝直遙翻】故温之得志衆所不願也必將乖阻以敗其事【乖異也阻隔也敗補邁翻】又温驕而恃衆怯于應變大衆深入值可乘之會反更逍遙中流不出赴利欲望持久坐取全勝【温之為計正如此申胤料之審矣】若糧廪愆懸情見勢屈必不戰自敗此自然之數【温攻秦而不度霸水攻燕而徘徊枋頭人皆咎其不進知彼知己温蓋臨敵而方有見乎此也温之智雖不足以禁暴定功然其去衆人亦遠矣愆謂糧運失期必至懸絶也見賢遍翻】温以燕降人段思為鄉導【降戶江翻鄉讀曰嚮】悉羅騰與温戰生擒思温使故趙將李述徇趙魏騰又與虎賁中郎將染干津擊斬之【染干亦夷姓如悉羅之類】温軍奪氣初温使豫州刺史袁真攻譙梁開石門以通水運真克譙梁而不能開石門【譙梁譙郡及梁國也】水運路塞【塞悉則翻】九月燕范陽王德帥騎一萬蘭臺侍御史劉當帥騎五千屯石門豫州刺史李邽帥州兵五千斷温糧道【燕豫州刺史治許昌斷丁管翻】當佩之子也【劉佩為慕容皝將却石虎攻宇文皆有功】德使將軍慕容宙帥騎一千為前鋒與晉兵遇宙曰晉人輕剽【剽匹妙翻急也】怯于陷敵勇于乘退宜設餌以釣之乃使二百騎挑戰【挑徒了翻】分餘騎為三伏挑戰者兵未交而走晉兵追之宙帥伏以擊之晉兵死者甚衆温戰數不利糧儲復竭【數所角翻復扶又翻下同】又聞秦兵將至丙申焚舟弃輜重鎧仗【重直用翻】自陸道犇還以毛虎生督東燕等四郡諸軍事領東燕太守【沈約曰東燕郡江左分濮陽所立也余按石虎分東燕郡屬洛州則是郡蓋祖逖在豫州時所置也燕於賢翻】温自東燕出倉垣鑿井而飲【汴水濟瀆皆自北而南恐追兵毒其上流故鑿井而飲】行七百餘里燕之諸將爭欲追之吳王垂曰不可温初退惶恐必嚴設警備簡精鋭為後拒擊之未必得志不如緩之彼幸吾未至必晝夜疾趨俟其士衆力盡氣衰然後擊之無不克矣乃帥八千騎徐行躡其後温果兼道而進數日垂告諸將曰温可擊矣乃急追之及温於襄邑【襄邑縣自漢以來屬陳留郡】范陽王德先帥勁騎四千伏于襄邑東澗中與垂夾擊温大破之斬首三萬級秦苟池邀擊温于譙又破之死者復以萬計孫元遂據武陽以拒燕燕左衛將軍孟高討擒之冬十月己巳大司馬温收散卒屯於山陽【劉昫曰山陽漢射陽縣地晉置山陽郡改為山陽縣唐為楚州治所】温深恥喪敗【喪息浪翻】乃歸罪於袁真【以石門不開糧運不繼為真罪】奏免真為庶人又免冠軍將軍鄧遐官【冠古玩翻】真以温誣已不服表温罪狀朝廷不報真遂據壽春叛降燕且請救亦遣使如秦【降戶江翻使疏吏翻下同】温以毛虎生領淮南太守守歷陽【淮南太守本治壽春壽春既叛以虎生領淮南而守歷陽歷陽本淮南屬縣虎生守之外以備壽春内以衛江南】 燕秦既結好【好呼到翻】使者數往來【數所角翻】燕散騎侍郎郝晷給事黄門侍郎梁琛相繼如秦【琛丑林翻】晷與王猛有舊猛接以平生問以東方之事晷見燕政不修而秦大治【治直吏翻】陰欲自託於猛頗洩其實琛至長安秦王堅方畋於萬年【萬年秦之櫟陽漢高帝更名屬馮翊晉屬京兆】欲引見琛【見賢遍翻】琛曰秦使至燕燕之君臣朝服備禮灑掃宫庭【朝直遥翻灑所賣翻又如字掃所報翻又如字】然後敢見今秦主欲野見之使臣不敢聞命尚書郎辛勁謂琛曰賓客入境惟主人所以處之君焉得專制其禮且天子稱乘輿【處昌呂翻焉於䖍翻乘繩證翻】所至曰行在所何常居之有又春秋亦有遇禮【春秋隱四年公及宋公遇于清公羊傳曰遇者何不期也杜預曰遇者草次之期二國各簡其禮若道路相逢遇也】何為不可乎琛曰晉室不綱靈祚歸德【靈祚猶班彪王命論所謂神明之祚也】二方承運俱受明命而桓温猖狂闚我王畧【左傳侵敗王略杜預注曰略經略法度余謂此略封略也如左傳王與之武公之略之略】燕危秦孤勢不獨立是以秦主同恤時患要結好援【要一遙翻好呼到翻下同】東朝君臣引領西望愧其不競以為鄰憂【競彊也朝直遙翻下同】西使之辱敬待有加今彊寇既退交聘方始謂宜崇禮篤義以固二國之歡若忽慢使臣是卑燕也豈脩好之義乎夫天子以四海為家故行曰乘輿止曰行在今海縣分裂【騶衍曰中國有赤縣神州赤縣神州内有九州禹所敘九州是也其外有禆海環之海縣之說蓋本諸此】天光分曜安得以乘輿行在為言哉禮不期而見曰遇蓋因事權行其禮簡略豈平居容與之所為哉客使單行誠勢屈於主人然苟不以禮亦不敢從也堅乃為之設行宫【為于偽翻】百僚陪位然後延客如燕朝之儀事畢堅與之私宴【倣古私覿之禮也】問東朝名臣為誰琛曰太傅上庸王評明德茂親光輔王室車騎大將軍吳王垂雄畧冠世【冠古玩翻】折衝禦侮其餘或以文進或以武用官皆稱職【稱尺證翻】野無遺賢琛從兄奕為秦尚書郎【從才用翻】堅使典客館琛於弈舍【漢有典客之官後改為大鴻臚此特臨時使之典客耳館音貫下果館同】琛曰昔諸菖瑾為吳聘蜀與諸葛亮惟公朝相見退無私面【瑾亮兄弟也為于偽翻】余竊慕之今使之即安私室所不敢也乃不果館奕數來就邸舍與琛卧起閒問琛東國事【數所角翻閒古莧翻】琛曰今二方分據兄弟並蒙榮寵論其本心各有所在琛欲言東國之美恐非西國之所欲聞【燕在關東秦在關西二方分據故謂燕為東國秦為西國】欲言其惡又非使臣之所得論也【使疏吏翻】兄何用問為堅使太子延琛相見秦人欲使琛拜太子先諷之曰鄰國之君猶其君也鄰國之儲君亦何以異乎琛曰天子之子視元士欲其由賤以登貴也【禮記郊特牲曰天子之元子士也天下無生而貴者也】尚不敢臣其父之臣况它國之臣乎苟無純敬則禮有往來情豈忘恭但恐降屈為煩耳【言當荅拜也】乃不果拜王猛勸堅留琛堅不許 燕主暐遣大鴻臚温統拜袁真使持節都督淮南諸軍事征南大將軍揚州刺史封宣城公【臚陵如翻使疏吏翻】統未踰淮而卒 吳王垂自襄邑還鄴威名益振太傅評愈忌之垂奏所募將士忘身立効將軍孫蓋等椎鋒陷陳【立効句絶椎擣也直擣其鋒也】應蒙殊賞評皆抑而不行垂數以為言與評廷爭怨隙愈深【數所角翻爭讀如字】太后可足渾氏素惡垂【事見一百卷穆帝升平元年惡烏路翻】毁其戰功與評密謀誅之太宰恪之子楷及垂舅蘭建知之以告垂曰先發制人【兵法曰先發制人後發者人制之】但除評及樂安王臧餘無能為矣垂曰骨肉相殘而首亂于國吾有死而已不忍為也頃之二人又以告曰内意已決【内意謂可足渾后之意也】不可不早發垂曰必不可彌縫吾寜避之于外餘非所議垂内以為憂而未敢告諸子世子令請曰尊比者如有憂色【令呼其父曰尊比毗至翻】豈非以主上幼冲太傅疾賢功高望重愈見猜邪垂曰然吾竭力致命以破彊寇本欲保全家國豈知功成之後返令身無所容汝既知吾心何以為吾謀令曰主上闇弱委任太傅一旦禍發疾于駭機【機弩牙也譬之彀弩不虞而機先發使人震駭故曰駭機】今欲保族全身不失大義莫若逃之龍城遜辭謝罪以待主上之察若周公之居東庶幾感寤而得還此幸之大者也【書武王有疾周公册祝于太王王季文王請以身代武王既喪管叔及其羣弟流言曰公將不利于孺子周公東征之周公居東二年則罪人斯得乃為詩以詒王名之曰䲭鴞王亦未敢誚公天大雷電以風王啟金滕得周公代武王之說乃執書以泣迎周公而歸幾居希翻】如其不然則内撫燕代外懷羣夷守肥如之險以自保亦其次也【肥如之險即盧龍之塞也】垂曰善十一月辛亥朔垂請畋于大陸【續漢志曰鉅鹿故大鹿有大陸澤即廣阿澤】因微服出鄴將趨龍城至邯鄲【趨七喻翻邯鄲縣漢屬趙國本趙都也晉屬廣平郡東魏廢隋復置唐屬磁州邯鄲音寒丹】少子麟素不為垂所愛逃還告狀【少詩照翻】垂左右多亡叛太傅評白燕主暐遣西平公強帥精騎追之【帥讀曰率騎奇寄翻下同】及于范陽世子令斷後【斷丁管翻】強不敢逼會日暮令謂垂曰本欲保東都以自全【燕既都鄴謂龍城為東都】今事已泄謀不及設秦主方招延英傑不如往歸之垂曰今日之計舍此安之【舍讀曰捨】乃散騎迹傍南山復還鄴【傍步浪翻自范陽傍南山蓋由中山常山山谷間南還也】隱于趙之顯原陵【顯原陵趙主石虎虚葬處】俄有獵者數百騎四面而來抗之則不能敵逃之則無路不知所為會獵者鷹皆飛颺衆騎散去【颺戶章翻】垂乃殺白馬以祭天且盟從者【從才用翻】世子令言于垂曰太傅忌賢疾能搆事以來人尤忿恨【謂搆殺垂之謀也】今鄴城之中莫知尊處如嬰兒之思母夷夏同之【夏戶雅翻】若順衆心襲其無備取之如指掌耳事定之後革弊簡能大匡朝政【朝直遙翻】以輔主上安國存家功之大者也今日之便誠不可失願給騎數人足以辦之垂曰如汝之謀事成誠為大福不成悔之不及不如西犇可以萬全子馬奴潛謀逃歸殺之而行至河陽為津吏所禁斬之而濟遂自洛陽與段夫人世子令令弟寶農隆兄子楷舅蘭建郎中令高弼俱犇秦留妃可足渾氏于鄴【段夫人垂前妃之女弟可足渾妃可足渾太后之妹也詳見一百卷穆帝升平二年高弼垂之國卿】乙泉戌主吳歸追及於閺鄉【乙泉戍即魏該所保乙泉塢也在宜陽縣西南洛水之北原上閺鄉在弘農湖縣閺音旻】世子令擊之而退初秦王堅聞太宰恪卒陰有圖燕之志憚垂威名不敢發及聞垂至大喜郊迎執手曰天生賢傑必相與共成大功此自然之數也要當與卿共定天下告成岱宗然後還卿本邦世封幽州使卿去國不失為子之孝歸朕不失事君之忠不亦美乎垂謝曰羈旅之臣免罪為幸本邦之榮非所敢望堅復愛世子令及慕容楷之才【復扶又翻】皆厚禮之賞賜鉅萬每進見屬目觀之【見賢遍翻屬之欲翻】關中士民素聞垂父子名皆嚮慕之王猛言于堅曰慕容垂父子譬如龍虎非可馴之物【馴擾也從也順也豢養猛獸使之擾狎順人之意曰馴馴詳遵翻】若借以風雲將不可復制不如早除之堅曰吾方收攬英雄以清四海柰何殺之且其始來吾已推誠納之矣匹夫猶不棄言况萬乘乎乃以垂為冠軍將軍封賓徒侯【乘繩證翻冠古玩翻賓徒漢縣名屬遼西郡】楷為積弩將軍燕魏尹范陽王德素與垂善及車騎從事中郎高泰皆坐免官【垂在燕為車騎大將軍以泰為從事中郎】尚書右丞申紹言于太傅評曰今吳王出犇外口籍籍【師古曰籍籍猶紛紛也】宜徵王僚屬之賢者顯進之粗可消謗【粗坐五翻】評曰誰可者紹曰高泰其領袖也乃以泰為尚書郎泰聸之從子【高聸見九十一卷元帝太興二年從才用翻】紹胤之子也秦留梁琛月餘乃遣歸琛兼程而進【程驛程也謂行者以二驛為程若一程而行四驛是兼程也】比至鄴【比必寐翻】吳王垂已犇秦琛言于太傅評曰秦人日閲軍旅多聚糧于陜東【陜失冉翻】以琛觀之為和必不能久今吳王又往歸之秦必有窺燕之謀宜早為之備評曰秦豈肯受叛臣而敗和好哉【敗補邁翻好呼到翻下同】琛曰今二國分據中原常有相吞之志桓温之入寇彼以計相救非愛燕也若燕有釁彼豈忘其本志哉【苻堅王猛之為謀梁琛固已窺見之矣】評曰秦主何如人琛曰明而善斷【斷丁亂翻】問王猛曰名不虚得評皆不以為然琛又以告燕主暐暐亦不然之以告皇甫真真深憂之上疏言苻堅雖聘問相尋然實有窺上國之心非能慕樂德義不忘久要也【樂音洛要一遙翻朱熹曰久要舊約也】前出兵洛川【謂苟池鄧羌救燕時也】及使者繼至【使疏吏翻】國之險易虚實【易以豉翻】彼皆得之矣今吳王垂又往從之為其謀主伍員之禍不可不備【伍員去楚奔吳借吳兵以報楚入郢事見左傳員音云】洛陽太原壺關皆宜選將益兵以防未然【秦後伐燕之路果如真所料杜佑曰潞州上黨縣漢為壺關縣】暐召太傅評謀之評曰秦國小力弱恃我為援且苻堅庶幾善道【言苻堅雖未能純以善道交鄰猶庶幾焉幾居希翻】終不肯納叛臣之言絶二國之好不宜輕自驚擾以啟寇心卒不為備【卒子恤翻】秦遣黄門郎石越聘于燕太傅評示之以奢欲以誇燕之富盛高泰及太傅參軍河間劉靖言于評曰越言誕而視遠非求好也乃觀釁也宜耀兵以示之用折其謀今乃示之以奢益為其所輕矣評不從泰遂謝病歸是時太后可足渾氏侵撓國政太傅評貪昧無厭【撓奴教翻又奴巧翻厭於鹽翻貪昧者貪財昧利不顧其害也】貨賂上流【流水行也水行就下無逆而上流之理貨賂上行謂之上流言其逆于常理也上時掌翻下同】官非才舉羣下怨憤尚書左丞申紹上疏以為守宰者致治之本【治直吏翻】今之守宰率非其人或武臣出于行伍或貴戚生長綺紈既非鄉曲之選又不更朝廷之職【守式又翻行戶剛翻長知兩翻更工衡翻】加之黜陟無法貪惰者無刑罰之懼清修者無旌賞之勸是以百姓困弊寇盜充斥綱頹紀紊莫相糾攝【糾督也攝録也紊音問】又官吏猥多踰于前世公私紛然不勝煩擾【勝音升】大燕戶口數兼二寇【以晉秦為二寇】弓馬之勁四方莫及而比者戰則屢北皆由守宰賦調不平【比毗至翻調徒釣翻】侵漁無已行留俱窘莫肯致命故也後宫之女四千餘人僮侍厮役尚在其外【厮音斯】一日之費厥直萬金士民承風競為奢靡彼秦吳僭僻【謂秦僭號而吳僻在一隅也】猶能調治所部有兼并之心【治直之翻】而我上下因循日失其序我之不修彼之願也謂宜精擇守宰併官省職存恤兵家使公私兩遂節抑浮靡愛惜用度賞必當功罰必當罪如此則温猛可梟【謂桓温王猛梟堅堯翻】二方可取豈特保境安民而已哉又索頭什翼犍疲病昏悖【蕭子顯曰鮮卑被髪左衽故呼為索頭索昔各翻犍居言翻悖蒲内翻】雖乏貢御【御進也】無能為患而勞兵遠戍有損無益【燕戌雲中以備代】不若移于并土控制西河南堅壺關北重晉陽西寇來則拒守過則斷後【斷丁管翻】猶愈于戍孤城守無用之地也疏奏不省【省悉景翻】 辛丑丞相昱與大司馬温會涂中【楊正衡曰涂音除涂中今滁州全椒縣真州六合縣地】以謀後舉以温世子熙為豫州刺史假節 初燕人許割虎牢以西賂秦晉兵既退燕人悔之謂秦人曰行人失辭【謂使者許割地為失辭也】有國有家者分災救患理之常也秦王堅大怒遣輔國將軍王猛建威將軍梁成洛州刺史鄧羌帥步騎三萬伐燕十二月進攻洛陽【帥讀曰率騎奇寄翻考異曰燕少帝紀此年十二月王猛攻洛明年正月拔洛十六國秦春秋十一月王猛伐燕遺慕容紀書紀請降十二月猛受降而歸今按獻莊紀云慕容令之奔還鄴建熙元年二月也時王猛猶在洛又猛遺紀書云去年桓温起師故從燕書】大司馬温發徐兖州民築廣陵城徙鎮之時征役既<br />
<br />
  頻加之疫癘死者什四五百姓嗟怨祕書監孫盛【漢桓帝置祕書監晉武帝以祕書併中書省惠帝復置祕書監其屬有丞有郎并統著作省】作晉春秋直書時事大司馬温見之怒謂盛子曰枋頭誠為失利何至乃如尊君所言【晉人于人子之前稱其父為尊君尊公】若此史遂行自是關君門戶事【言欲滅其門也】其子遽拜謝請改之時盛年老家居性方嚴有軌度子孫雖斑白待之愈峻至是諸子乃共號泣稽顙請為百口切計【稽音啟】盛大怒不許諸子遂私改之盛先已寫别本傳之外國及孝武帝購求異書得之于遼東人與見本不同【見賢遍翻】遂兩存之【史言桓温唯以威逼改孫盛之書終不能没其實】 五年春正月己亥袁真以梁國内史沛郡朱憲及弟汝南内史斌陰通大司馬温殺之【斌音彬】 秦王猛遺燕荆州刺史武威王筑書【遺于季翻燕荆州治洛陽筑張六翻】曰國家今已塞成皋之險【塞悉則翻】杜盟津之路【盟讀曰孟】大駕虎旅百萬自軹關取鄴都金墉窮戌外無救援城下之師將軍所監【監視也猶言目所見也】豈三百弊卒所能支也筑懼以洛陽降【降戶江翻】猛陳師受之燕衛大將軍樂安王臧城新樂破秦兵于石門【石門在滎陽新樂亦當在滎陽界宋白曰衛州新鄉縣治古新樂城新樂城十六國時燕將樂安王臧所築】執秦將楊猛王猛之發長安也請慕容令參其軍事以為鄉導將行造慕容垂飲酒從容謂垂曰【鄉讀曰嚮造七到翻從干容翻】今當遠别何以贈我使我覩物思人垂脱佩刀贈之猛至洛陽賂垂所親金熙使詐為垂使者謂令曰吾父子來此以逃死也今王猛疾人如讐讒毁日深秦王雖外相厚善其心難知丈夫逃死而卒不免【卒子恤翻】將為天下笑吾聞東朝比來始更悔悟【朝直遙翻比毗至翻】主后相尤【主后謂燕主暐及可足渾后也相尤言相責過】吾今還東故遣告汝吾已行矣便可速發令疑之躊躇終日【躊直留翻躇陳如翻猶豫住足之意】又不可審覆乃將舊騎【舊騎自燕奔秦所從者騎奇寄翻下同】詐為出獵遂奔樂安王臧于石門猛表令叛狀垂懼而出走及藍田為追騎所獲秦王堅引見東堂勞之曰【勞力到翻】卿家國失和委身投朕賢子心不忘本猶懷首丘【禮記檀弓曰太公封於齊五世皆反葬于周君子曰樂樂其所自生禮不忘其本古之人有言曰狐死正丘首仁也首式又翻】亦各其志不足深咎然燕之將亡非令所能存惜其徒入虎口耳且父子兄弟罪不相及【晉臼季薦冀缺于晉文公公曰其父有罪可乎對曰舜之罪也殛鯀其舉也興禹康誥曰父不慈子不祇兄不友弟不共不相及也】卿何為過懼而狼狽如是乎【狼進則跋其胡退則疐其尾狽狼屬也生子欠一足二者相附而後能行故世謂進退不可而不能行者為狼狽】待之如舊燕人以令叛而復還其父為秦所厚疑令為反間【復扶又翻間古莧翻】徙之沙城在龍都東北六百里【沙城在沙野龍都即龍城】<br />
<br />
  臣光曰昔周得微子而革商命【殷紂暴虐日甚微子抱祭器而奔周武王乃告諸侯曰殷有重罪不可不伐遂伐紂殺之而革殷命】秦得由余而霸西戎【史記戎使由余使于秦繆公留由余而遺戎王以女樂戎王受而說之繆公乃歸由余由余數諫不聽繆公使人間要由余由余遂降秦繆公問以伐戎之形并國十二開地千里遂霸西戎】吳得伍員而克彊楚【楚殺伍奢其子員奔吳吳王闔閭用其謀而伐楚破楚入郢】漢得陳平而誅項籍【事見九卷漢高帝二年至四年】魏得許攸而破袁紹【事見六十三卷漢獻帝建安五年】彼敵國之材臣來為已用進取之良資也王猛知慕容垂之心久而難信獨不念燕尚未垂以材高功盛無罪見疑窮困歸秦未有異心遽以猜忌殺之是助燕為無道而塞來者之門也【塞悉則翻】如何其可哉故秦王堅禮之以收燕望親之以盡燕情寵之以傾燕衆信之以結燕心未為過矣猛何汲汲于殺垂乃為市井鬻賣之行【行下孟翻】有如嫉其寵而讒之者豈雅德君子所宜為哉<br />
<br />
  樂安王臧進屯滎陽王猛遣建威將軍梁成洛州刺史鄧羌擊走之留羌鎮金墉以輔國司馬桓寅為弘農太守【猛為輔國將軍以寅為司馬】代羌戌陜城而還【秦初以洛州刺史鎮陜今鄧羌既進屯金墉故以桓寅代戌陜陜失冉翻】秦王堅以王猛為司徒録尚書事封平陽郡侯猛固辭曰今燕吳未平戎車方駕而始得一城即受三事之賞【三事三公也】若克殄二寇將何以加之堅曰苟不蹔抑朕心何以顯卿謙光之美已詔有司權聽所守封爵酬庸【庸功也】其勉從朕命 二月癸酉袁真卒陳郡太守朱輔立真子瑾為建威將軍豫州刺史以保壽春遣其子乾之及司馬爨亮如鄴請命燕人以瑾為揚州刺史輔為荆州刺史【瑾渠吝翻】 三月秦王堅以吏部尚書權翼為尚書右僕射夏四月復以王猛為司徒録尚書事【復扶又翻下同】猛固辭乃止 燕秦皆遣兵助袁瑾大司馬温遣督護竺瑤等禦之燕兵先至瑤等與戰于武丘破之【武丘即丘頭文王平諸葛誕改曰武丘以旌武功杜佑曰丘頭即潁州沈丘縣】南頓太守桓石䖍克其南城【惠帝分汝南立南頓郡南城壽春南城也】石䖍温之弟子也 秦王堅復遣王猛督鎮南將軍楊安等十將步騎六萬以伐燕 慕容令自度終不得免【度徒洛翻】密謀起兵沙城中讁戍士數千人令皆厚撫之【讁陟革翻】五月庚午令殺牙門孟媯城大涉圭懼請自效【姓譜涉姓也左傳晉有大夫涉佗媯居為翻】令信之引置左右遂帥讁戌士東襲威德城【威德城即宇文涉夜于所居城也燕王皝改曰威德城】殺城郎慕容倉據城部署遣人招東西諸戍翕然皆應之鎮東將軍勃海王亮鎮龍城令將襲之其弟麟以告亮亮閉城拒守癸酉涉圭因侍直擊令【令引涉圭置左右故得因侍直而擊之】令單馬走其黨皆潰涉圭追令至薛黎澤擒而殺之詣龍城白亮亮為誅涉圭【為于偽翻】收令尸而葬之 六月乙卯秦王堅送王猛于灞上曰今委卿以關東之任當先破壺關平上黨【魏收曰秦置上黨郡治壺關城前漢治長子城董卓治壺關城慕容氏治安民城後遷壺關城】長驅取鄴所謂疾雷不及掩耳【淮南子之言】吾當親督萬衆繼卿星發【星謂戴星而發行也】舟車糧運水陸俱進卿勿以為後慮也猛曰臣杖威靈奉成筭盪平殘胡【盪徒朗翻】如風掃葉願不煩鑾輿親犯塵霧但願速敇所司部置鮮卑之所【言預為治舍以待其至】堅大悦 秋七月癸酉朔日有食之 秦王猛攻壺關楊安攻晉陽八月燕主暐命太傅上庸王評將中外精兵三十萬以拒秦 【考異曰載記云四十萬今從晉春秋】暐以秦寇為憂召散騎侍郎李鳳【散悉亶翻騎奇寄翻下同】黄門侍郎梁琛中書侍郎樂嵩問曰秦兵衆寡何如今大軍既出秦能戰乎鳳曰秦國小兵弱非王師之敵景略常才又非太傅之比不足憂也【王猛字景略】琛嵩曰勝敗在謀不在衆寡秦遠來為寇安肯不戰且吾當用謀以求勝豈可冀其不戰而已乎暐不悦王猛克壺關執上黨太守南安王越所過郡縣皆望風降附【降戶江翻】燕人大震黄門侍郎封孚問司徒長史申胤曰事將何如胤歎曰鄴必亡矣吾屬今兹將為秦虜然越得歲而吳伐之卒受其禍【左傳昭三十二年吳伐越史墨曰不及四十年越其有吳乎越得歲而吳伐之必受其凶杜預注曰此年歲在星紀星紀吳越之分也歲星所在其國有福吳先用兵故反受其殃卒子恤翻】今福德在燕【福德在燕亦謂歲星在燕分也後苻堅所謂昔吾燕亦犯歲而捷是也】秦雖得志而燕之復建不過一紀耳【為後燕復興張本復扶又翻又如字】 大司馬温自廣陵帥衆二萬討袁瑾以襄城太守劉波為淮南内史將五千人鎮石頭波隗之孫也【元帝之末劉隗避王敦之亂因北奔于後趙帥讀曰率將即亮翻下同】癸丑温敗瑾于壽春【敗補邁翻】遂圍之燕左衛將軍孟高將騎兵救瑾至淮北未渡會秦伐燕燕召高還【還從宣翻又如字】 廣漢妖賊李弘詐稱漢歸義侯勢之子聚衆萬餘人自稱聖王年號鳳凰【妖於驕翻】隴西人李高詐稱成主雄之子攻破涪城【涪音浮】逐梁州刺史楊亮九月益州刺史周楚遣子瓊討高又使瓊子梓潼太守虓討弘皆平之【虓虚交翻】 秦楊安攻晉陽晉陽兵多糧足久之未下王猛留屯騎校尉苟長戍壺關【苟長當作苟萇】引兵助安攻晉陽為地道使虎牙將軍張蚝帥壯士數百潜入城中大呼斬關納秦兵【呼紱處翻】辛巳猛安入晉陽執燕并州刺史東海王莊太傅評畏猛不敢進屯于潞川【水經注潞川在上黨潞縣北闞駰曰潞水即漳水也為冀州浸】冬十月辛亥猛留將軍武都毛當戍晉陽進兵潞川與慕容評相持壬戍猛遣將軍徐成覘燕軍形要【形者見於外要者有諸中覘見其形未足以決勝負覘見其要則勝負之機決矣覘丑亷翻又丑艷翻】期以日中及昏而返猛怒將斬之鄧羌請之曰今賊衆我寡詰朝將戰【杜預曰詰朝平旦也詰去吉翻朝如字】成大將也宜且宥之猛曰若不殺成軍法不立羌固請曰成羌之郡將也【成蓋為羌本郡太守將即亮翻下同】雖違期應斬羌願與成効戰以贖之【効戰謂効力決戰也】猛弗許羌怒還營嚴鼓勒兵將攻猛猛問其故羌曰受詔討遠賊今有近賊自相殺欲先除之猛謂羌義而有勇使語之曰將軍止吾今赦之成既免羌詣猛謝猛執其手曰吾試將軍耳將軍于郡將尚爾况國家乎吾不復憂賊矣【語牛倨翻復扶又翻】太傅評以猛懸軍深入欲以持久制之評為人貪鄙鄣固山泉鬻樵及水【山者樵之所仰泉者汲之所仰障固山泉使軍士不得樵汲而鬻薪水以牟利】積錢帛如丘陵【賈公彦曰高曰丘大阜曰陵】士卒怨憤莫有鬬志猛聞之笑曰慕容評真奴才雖億兆之衆不足畏况數十萬乎吾今茲破之必矣乃遣游擊將軍郭慶帥騎五千夜從間道出評營後燒評輜重火見鄴中【閒古莧翻重直用翻見賢遍翻潞川地形高而近鄴且火盛故鄴中望而見之】燕主暐懼遣侍中蘭伊讓評曰王高祖之子也【慕容廆廟號高祖】當以宗廟社稷為憂奈何不撫戰士而榷賣樵水專以貨殖為心乎【榷古岳翻】府庫之積朕與王共之何憂于貧若賊兵遂進家國喪亡【喪息浪翻】王持錢帛欲安所置之乃命悉以其錢帛散之軍士【酈道元曰評鬻水與軍人絹匹與水二石】且趨使戰【趨讀曰趣音趨玉翻】評大懼遣使請戰于猛【使疏吏翻】甲子猛陳於渭源而誓之【按渭水不出潞縣水經注有涅水出潞縣西覆甑山或者渭字其涅字之誤乎又按温公稽古録書王猛破評于清原杜預曰河東聞喜縣北有清原其地又與潞川相遠姑存疑以待知者杜佑通典作潞源陣讀曰陣下同】曰王景略受國厚恩任兼内外今與諸君深入賊地當竭力致死有進無退共立大功以報國家受爵明君之朝稱觴父母之室不亦美乎【受爵明君之朝謂有功而受賞於朝也稱觴父母之室謂受賞而歸舉酒為父母壽也朝直遙翻】衆皆踴躍破釡棄糧大呼競進【呼火故翻】猛望燕兵之衆謂鄧羌曰今日之事非將軍不能破勍敵【勍渠京翻】成敗之機在茲一舉將軍勉之羌曰若能以司隸見與者公勿以為憂猛曰此非吾所及也必以安定太守萬戶侯相處【秦雍州刺史治安定安定在秦中為大郡處昌呂翻】羌不悦而退俄而兵交猛召羌羌寢不應猛馳就許之羌乃大飲帳中與張蚝徐成等跨馬運矛馳赴燕陳出入數四旁若無人所殺傷數百及日中燕兵大敗俘斬五萬餘人乘勝追擊所殺及降者又十萬餘人【降戶江翻】評單騎走還鄴<br />
<br />
  崔鴻曰鄧羌請郡將以撓法徇私也【撓奴教翻又女巧翻】勒兵欲攻王猛無上也臨戰豫求司隸邀君也有此三者罪孰大焉猛能容其所短收其所長若馴猛虎馭悍馬以成大功詩曰采葑采菲無以下體【詩谷風之辭毛氏曰葑須也菲芴也下體根莖也鄭氏曰此二菜者蔓菁與葍之類也皆上下可食然而其根有美時有惡時采之者不可以根惡時并棄其葉陸璣草木疏曰葑蕪菁也菲蒠菜郭璞曰葑菘菜也江南有菘江北有蔓菁相似而異菲芴土瓜也蒠菜似蕪菁華紫赤色可食葍大葉白華根如指色白可食葍方六翻】猛之謂矣<br />
<br />
  秦兵長驅而東【自潞川而東攻鄴】丁卯圍鄴猛上疏稱臣以甲子之日大殱醜類【謂甲子之日克勝事同周武王克紂殱息廉翻】順陛下仁愛之志使六州士庶不覺易主自非守迷違命一無所害秦王堅報之曰將軍役不踰時【三月為一時】而元惡克舉勲高前古朕今親帥六軍星言電赴【詩曰星言夙駕謂早駕見星而行也電赴言其疾也帥讀曰率】將軍其休養將士以待朕至然後取之猛之未至也鄴旁剽刼公行【剽匹妙翻】及猛至遠近帖然號令嚴明軍無私犯【言軍士不敢私犯鄴民也】法簡政寛燕民各安其業更相謂曰不圖今日復見太原王【更工衡翻復扶又翻】王猛聞之歎曰慕容玄恭信奇士也可謂古之遺愛矣【慕容恪字玄恭封太原王】設太牢以祭之十一月秦王堅留李威輔太子守長安陽平公融鎮洛陽自帥精鋭十萬赴鄴七日而至安陽【晉志安陽縣屬魏郡魏收志曰天平初併蕩陰安陽屬鄴又汲郡北修武縣有安陽城】宴祖父時故老【苻洪父子先屯枋頭有故老尚存聞堅之來迎于安陽故宴之】猛潜如安陽謁堅堅曰昔周亞夫不迎漢文帝【見十五卷漢文帝後六年】今將軍臨敵而棄軍何也猛曰亞夫前却人主以求名臣竊少之【少詩沼翻】且臣奉陛下威靈擊垂亡之虜譬如釡中之魚何足慮也監國冲幼【太子守曰監國監工銜翻】鸞駕遠臨脱有不虞悔之何及陛下忘臣灞上之言邪初燕宜都王桓帥衆萬餘屯沙亭【杜預曰陽平元城縣有沙亭】為太傅評後繼聞評敗引兵屯内黄【内黄縣自漢以來屬魏郡】堅使鄧羌攻信都丁丑桓帥鮮卑五千犇龍城戊寅燕散騎侍郎餘蔚帥扶餘高句麗及上黨質子五百餘人【蔚於勿翻燕蓋遣兵戍上黨取其子弟留於鄴以為質餘蔚扶餘王子故陰率諸質子開門以納秦兵質音致句如字又音駒麗力知翻】夜開鄴北門納秦兵燕主暐與上庸王評樂安王臧定襄王淵左衛將軍孟高殿中將軍艾朗等犇龍城【姓譜艾姓晏子春秋齊有大夫艾孔風俗通有龎儉母艾氏】辛巳秦王堅入鄴宫慕容垂見燕公卿大夫及故時僚吏有愠色【慍於問翻】高弼言于垂曰大王憑祖宗積累之資負英傑高世之畧遭值迍阨【迍株倫翻】棲集外邦今雖家國傾覆安知其不為興運之始邪愚謂國之舊人宜恢江海之量有以慰結其心以立覆簣之基成九仞之功【言譬如為山自覆一簣而進成九仞之功簣求位翻土籠也八尺曰仞】奈何以一怒捐之愚竊為大王不取也【高弼先從垂奔秦故敢進言為于偽翻】垂悦從之燕主暐之出鄴也衛士猶千餘騎既出城皆散惟十餘騎從行秦王堅使游擊將軍郭慶追之時道路艱難孟高扶侍暐經護二王【二王謂樂安王臧定襄王洲也】極其勤瘁【瘁秦醉翻】又所在遇盜轉鬬而前數日行至福禄依冢解息【解息解鞍息馬也冢知隴翻】盜二十餘人猝至皆挾弓矢高持刀與戰殺傷數人高力極【力疲極也】自度必死乃直前抱一賊頓擊于地大呼曰男兒窮矣餘賊從旁射高殺之【度徒洛翻射而亦翻】艾朗見高獨戰亦還趨賊并死【趨七喻翻】暐失馬步走郭慶追及於高陽部將巨武將縛之【姓譜巨姓也】暐曰汝何小人敢縛天子武曰我受詔追賊何謂天子執以詣秦王堅堅詰其不降而走之狀【詰去吉翻降戶江翻下同】對曰狐死首丘欲歸死于先人墳墓耳【慕容氏之先皆葬昌黎】堅哀而釋之令還宫帥文武出降【晉穆帝永和八年燕主儁改元稱帝傳子暐共十九年而亡帥讀曰率】暐稱孟高艾朗之忠于堅堅命厚加斂葬【斂力贍翻】拜其子為郎中郭慶進至龍城太傅評犇高句麗高句麗執評送于秦宜都王桓殺鎮東將軍勃海王亮并其衆犇遼東遼東太守韓稠先已降秦桓至不得入攻之不克郭慶遣將軍朱嶷擊之桓棄衆單走嶷獲而殺之【嶷魚力翻】諸州牧守及六夷渠帥盡降于秦【帥所類翻】凡得郡百五十七戶二百四十六萬口九百九十九萬以燕宫人珍寶分賜將士【將即亮翻】下詔大赦曰朕以寡薄猥承休命不能懷遠以德柔服四維【四維東南維西南維東北維西北維】至使戎車屢駕有害斯民雖百姓之過然亦朕之罪也其大赦天下與之更始【更工衡翻】初梁琛之使秦也【使疏吏翻】以侍輦苟純為副【侍輦之官蓋燕所置近臣也】琛每應對不先告純純恨之歸言於燕主暐曰琛在長安與王猛甚親善疑有異謀琛又數稱秦王堅及王猛之美【數所角翻】且言秦將興師宜為之備已而秦果伐燕皆如琛言暐乃疑琛知其情及慕容評敗遂收琛繫獄秦王堅入鄴而釋之除中書著作郎【秦蓋循晉初之制併袐書于中書省也】引見【見賢遍翻】謂之曰卿昔言上庸王吳王皆將相奇材【將即亮翻相息亮翻】何為不能謀畫自使亡國對曰天命廢興豈二人所能移也堅曰卿不能見幾而作虚稱燕美忠不自防反為身禍可謂智乎對曰臣聞幾者動之微吉之先見者也【易大傳之辭幾居希翻見賢遍翻】如臣愚昧實所不及然為臣莫如忠為子莫如孝自非有一至之心者莫能呆忠孝之始終是以古之烈士臨危不改見死不避以徇君親彼知幾者心達安危身擇去就不顧家國臣就使知之尚不忍為况非所及邪【梁琛忠于所事秦王堅不能顯而庸之識者有以知秦祚之不長矣】堅聞悦綰之忠【悦綰事見上卷三年】恨不及見拜其子為郎中堅以王猛為使持節都督關東六州諸軍事車騎大將軍開府儀同三司冀州牧鎮鄴【使疏吏翻騎奇寄翻】進爵清河郡侯悉以慕容評第中之物賜之賜楊安爵博平縣侯以鄧羌為使持節征虜將軍安定太守賜爵真定郡侯郭慶為持節都督幽州諸軍事幽州刺史鎮薊【薊音計】賜爵襄城侯【賜爵者賜之侯爵非有國有土也一曰先未列爵今始賜之】其餘將士封賞各有差堅以京兆韋鐘為魏郡太守彭豹為陽平太守【燕都鄴以魏郡太守為京尹陽平輔郡也故堅皆以秦人守之】其餘州縣牧守令長皆因舊以授之【盡易州縣牧守今長既駭觀聽且人情新舊不相安故皆因舊】以燕常山太守申紹為散騎侍郎使與散騎侍郎京兆韋儒俱為繡衣使者循行關東州郡觀省風俗【行下孟翻省悉景翻】勸課農桑振恤窮困收葬死亡旌顯節行燕政有不便于民者皆變除之【並用燕秦之人為繡衣使者用燕人者以其習關東風俗用秦人者使宣堅之德意也行下孟翻】十二月秦王堅遷慕容暐及燕后妃王公百官并鮮卑四萬餘戶于長安【為後鮮卑乘亂攻秦張本】王猛表留梁琛為主簿領記室督【晉制諸公府有主簿記室督各一人今猛以琛兼之】它日猛與僚屬宴語及燕朝使者猛曰人心不同昔梁君至長安專美本朝樂君但言桓温軍盛郝君微說國弊【梁琛樂嵩郝晷也本朝國弊皆謂燕也朝直遙翻使疏吏翻】參軍馮誕曰今三子皆為國臣【此國謂秦也】敢問取臣之道何先猛曰郝君知幾為先【幾居希翻】誕曰然則明公賞丁公而誅季布也【言取臣之道與漢高帝異】猛大笑秦王堅自鄴如枋頭宴父老改枋頭曰永昌復之終世【復方目翻除賦役也復除賦役終秦王之世也】甲寅至長安封慕容暐為新興侯以燕故臣慕容評為給事中皇甫真為奉車都尉李洪為駙馬都尉皆奉朝請【三人者燕之三公也】李邽為尚書封衡為尚書郎慕容德為張掖太守【為德兄子超留張掖而入姚氏張本】燕國平叡為宣威將軍悉羅騰為三署郎【漢有五官署郎左署郎右署郎故曰三署郎舊制郎年五十以上屬五官其次分在左右署秦遂以三署郎為官稱】其餘封署各有差衡裕之子也【慕容皝之興也封裕以忠諫顯】燕故太史黄泓歎曰燕必中興其在吳王乎恨吾老不及見耳【慕容之初興黄泓歸之及儁之取中原黄泓贊其決泓知數者也】汲郡趙秋曰天道在燕【謂歲星在燕分也】不及十五年秦必復為燕有慕容桓之子鳳年十一陰有復讐之志鮮卑丁零有氣幹者皆傾身與之交結【為後慕容鳳與丁零起兵攻秦張本】權翼見而謂之曰兒方以才望自顯勿效爾父不識天命鳳厲色曰先王欲建忠而不遂此乃人臣之節君侯之言豈奬勸將來之義乎翼改容謝之言于秦王堅曰慕容鳳忼慨有才器但狼子野心恐終不為人用耳【左傳楚令尹子文曰狼子野心史言燕之臣子非久下人者】 秦省雍州【秦置雍州於安定今省雍州入司隸校尉】 是歲仇池公楊世卒子纂立始與秦絶叔父武都太守統與之爭國起兵相攻【為秦攻仇池張本】<br />
<br />
  資治通鑑卷一百二<br />
<br />
<史部,編年類,資治通鑑>  <br>
   </div> 

<script src="/search/ajaxskft.js"> </script>
 <div class="clear"></div>
<br>
<br>
 <!-- a.d-->

 <!--
<div class="info_share">
</div> 
-->
 <!--info_share--></div>   <!-- end info_content-->
  </div> <!-- end l-->

<div class="r">   <!--r-->



<div class="sidebar"  style="margin-bottom:2px;">

 
<div class="sidebar_title">工具类大全</div>
<div class="sidebar_info">
<strong><a href="http://www.guoxuedashi.com/lsditu/" target="_blank">历史地图</a></strong>  
<a href="http://www.880114.com/" target="_blank">英语宝典</a>  
<a href="http://www.guoxuedashi.com/13jing/" target="_blank">十三经检索</a> 
<br><strong><a href="http://www.guoxuedashi.com/gjtsjc/" target="_blank">古今图书集成</a></strong> 
<a href="http://www.guoxuedashi.com/duilian/" target="_blank">对联大全</a> <strong><a href="http://www.guoxuedashi.com/xiangxingzi/" target="_blank">象形文字典</a></strong> 

<br><a href="http://www.guoxuedashi.com/zixing/yanbian/">字形演变</a>  <strong><a href="http://www.guoxuemi.com/hafo/" target="_blank">哈佛燕京中文善本特藏</a></strong>
<br><strong><a href="http://www.guoxuedashi.com/csfz/" target="_blank">丛书&方志检索器</a></strong> <a href="http://www.guoxuedashi.com/yqjyy/" target="_blank">一切经音义</a>  

<br><strong><a href="http://www.guoxuedashi.com/jiapu/" target="_blank">家谱族谱查询</a></strong>  <strong><a href="http://shufa.guoxuedashi.com/sfzitie/" target="_blank">书法字帖欣赏</a></strong> 
<br>

</div>
</div>


<div class="sidebar" style="margin-bottom:0px;">

<font style="font-size:22px;line-height:32px">QQ交流群9:489193090</font>


<div class="sidebar_title">手机APP 扫描或点击</div>
<div class="sidebar_info">
<table>
<tr>
	<td width=160><a href="http://m.guoxuedashi.com/app/" target="_blank"><img src="/img/gxds-sj.png" width="140"  border="0" alt="国学大师手机版"></a></td>
	<td>
<a href="http://www.guoxuedashi.com/download/" target="_blank">app软件下载专区</a><br>
<a href="http://www.guoxuedashi.com/download/gxds.php" target="_blank">《国学大师》下载</a><br>
<a href="http://www.guoxuedashi.com/download/kxzd.php" target="_blank">《汉字宝典》下载</a><br>
<a href="http://www.guoxuedashi.com/download/scqbd.php" target="_blank">《诗词曲宝典》下载</a><br>
<a href="http://www.guoxuedashi.com/SiKuQuanShu/skqs.php" target="_blank">《四库全书》下载</a><br>
</td>
</tr>
</table>

</div>
</div>


<div class="sidebar2">
<center>


</center>
</div>

<div class="sidebar"  style="margin-bottom:2px;">
<div class="sidebar_title">网站使用教程</div>
<div class="sidebar_info">
<a href="http://www.guoxuedashi.com/help/gjsearch.php" target="_blank">如何在国学大师网下载古籍?</a><br>
<a href="http://www.guoxuedashi.com/zidian/bujian/bjjc.php" target="_blank">如何使用部件查字法快速查字?</a><br>
<a href="http://www.guoxuedashi.com/search/sjc.php" target="_blank">如何在指定的书籍中全文检索?</a><br>
<a href="http://www.guoxuedashi.com/search/skjc.php" target="_blank">如何找到一句话在《四库全书》哪一页?</a><br>
</div>
</div>


<div class="sidebar">
<div class="sidebar_title">热门书籍</div>
<div class="sidebar_info">
<a href="/so.php?sokey=%E8%B5%84%E6%B2%BB%E9%80%9A%E9%89%B4&kt=1">资治通鉴</a> <a href="/24shi/"><strong>二十四史</strong></a>&nbsp; <a href="/a2694/">野史</a>&nbsp; <a href="/SiKuQuanShu/"><strong>四库全书</strong></a>&nbsp;<a href="http://www.guoxuedashi.com/SiKuQuanShu/fanti/">繁体</a>
<br><a href="/so.php?sokey=%E7%BA%A2%E6%A5%BC%E6%A2%A6&kt=1">红楼梦</a> <a href="/a/1858x/">三国演义</a> <a href="/a/1038k/">水浒传</a> <a href="/a/1046t/">西游记</a> <a href="/a/1914o/">封神演义</a>
<br>
<a href="http://www.guoxuedashi.com/so.php?sokeygx=%E4%B8%87%E6%9C%89%E6%96%87%E5%BA%93&submit=&kt=1">万有文库</a> <a href="/a/780t/">古文观止</a> <a href="/a/1024l/">文心雕龙</a> <a href="/a/1704n/">全唐诗</a> <a href="/a/1705h/">全宋词</a>
<br><a href="http://www.guoxuedashi.com/so.php?sokeygx=%E7%99%BE%E8%A1%B2%E6%9C%AC%E4%BA%8C%E5%8D%81%E5%9B%9B%E5%8F%B2&submit=&kt=1"><strong>百衲本二十四史</strong></a>  <a href="http://www.guoxuedashi.com/so.php?sokeygx=%E5%8F%A4%E4%BB%8A%E5%9B%BE%E4%B9%A6%E9%9B%86%E6%88%90&submit=&kt=1"><strong>古今图书集成</strong></a>
<br>

<a href="http://www.guoxuedashi.com/so.php?sokeygx=%E4%B8%9B%E4%B9%A6%E9%9B%86%E6%88%90&submit=&kt=1">丛书集成</a> 
<a href="http://www.guoxuedashi.com/so.php?sokeygx=%E5%9B%9B%E9%83%A8%E4%B8%9B%E5%88%8A&submit=&kt=1"><strong>四部丛刊</strong></a>  
<a href="http://www.guoxuedashi.com/so.php?sokeygx=%E8%AF%B4%E6%96%87%E8%A7%A3%E5%AD%97&submit=&kt=1">說文解字</a> <a href="http://www.guoxuedashi.com/so.php?sokeygx=%E5%85%A8%E4%B8%8A%E5%8F%A4&submit=&kt=1">三国六朝文</a>
<br><a href="http://www.guoxuedashi.com/so.php?sokeytm=%E6%97%A5%E6%9C%AC%E5%86%85%E9%98%81%E6%96%87%E5%BA%93&submit=&kt=1"><strong>日本内阁文库</strong></a> <a href="http://www.guoxuedashi.com/so.php?sokeytm=%E5%9B%BD%E5%9B%BE%E6%96%B9%E5%BF%97%E5%90%88%E9%9B%86&ka=100&submit=">国图方志合集</a> <a href="http://www.guoxuedashi.com/so.php?sokeytm=%E5%90%84%E5%9C%B0%E6%96%B9%E5%BF%97&submit=&kt=1"><strong>各地方志</strong></a>

</div>
</div>


<div class="sidebar2">
<center>

</center>
</div>
<div class="sidebar greenbar">
<div class="sidebar_title green">四库全书</div>
<div class="sidebar_info">

《四库全书》是中国古代最大的丛书,编撰于乾隆年间,由纪昀等360多位高官、学者编撰,3800多人抄写,费时十三年编成。丛书分经、史、子、集四部,故名四库。共有3500多种书,7.9万卷,3.6万册,约8亿字,基本上囊括了古代所有图书,故称“全书”。<a href="http://www.guoxuedashi.com/SiKuQuanShu/">详细>>
</a>

</div> 
</div>

</div>  <!--end r-->

</div>
<!-- 内容区END --> 

<!-- 页脚开始 -->
<div class="shh">

</div>

<div class="w1180" style="margin-top:8px;">
<center><script src="http://www.guoxuedashi.com/img/plus.php?id=3"></script></center>
</div>
<div class="w1180 foot">
<a href="/b/thanks.php">特别致谢</a> | <a href="javascript:window.external.AddFavorite(document.location.href,document.title);">收藏本站</a> | <a href="#">欢迎投稿</a> | <a href="http://www.guoxuedashi.com/forum/">意见建议</a> | <a href="http://www.guoxuemi.com/">国学迷</a> | <a href="http://www.shuowen.net/">说文网</a><script language="javascript" type="text/javascript" src="https://js.users.51.la/17753172.js"></script><br />
  Copyright &copy; 国学大师 古典图书集成 All Rights Reserved.<br>
  
  <span style="font-size:14px">免责声明:本站非营利性站点,以方便网友为主,仅供学习研究。<br>内容由热心网友提供和网上收集,不保留版权。若侵犯了您的权益,来信即刪。scp168@qq.com</span>
  <br />
ICP证:<a href="http://www.beian.miit.gov.cn/" target="_blank">鲁ICP备19060063号</a></div>
<!-- 页脚END --> 
<script src="http://www.guoxuedashi.com/img/plus.php?id=22"></script>
<script src="http://www.guoxuedashi.com/img/tongji.js"></script>

</body>
</html>
