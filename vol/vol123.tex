










 


 
 


 

  
  
  
  
  





  
  
  
  
  
 
  

  

  
  
  



  

 
 

  
   




  

  
  


    資治通鑑卷一百二十三 宋 司馬光 撰

  胡三省 音註

  宋紀五【起柔兆困敦盡重光大荒落凡六年}


  太祖文皇帝中之上

  元嘉十三年春正月癸丑朔上有疾不朝會【朝直遥翻下同}
甲寅魏主還宫 二月戊子燕王遣使入貢於魏【使疏吏翻下同}
請送侍子魏主不許【燕王屢請送侍子而不至魏主知其詐故不許}
將舉兵討之壬辰遣使者十餘輩詣東方高麗等諸國告諭之【諭以燕王之罪使不得與通或有奔逸使不得容受之也}
司空江州刺史永脩公檀道濟【漢靈帝中平中立永脩縣屬豫章郡隋開皇九年併入建昌縣}
立功前朝威名甚重左右腹心並經百戰諸子又有才氣朝廷疑畏之【朝直遥翻}
帝久疾不愈劉湛說司徒義康以為宫車一日晏駕道濟不復可制【說輸芮翻復扶又翻下足復同}
會帝疾篤義康言于帝召道濟入朝其妻向氏謂道濟曰【姓譜祁姓之後為向國向式亮翻又如字}
高世之勲自古所忌今無事相召禍其至矣既至留之累月帝稍間【間如字}
將遣還已下渚【道濟將還江州船已下秦淮渚}
未會帝疾動義康矯詔召道濟入祖道因執之三月己未下詔稱道濟潜散金貨招誘剽猾【誘音酉剽匹妙翻}
因朕寢疾規肆禍心收付廷尉并其子給事黄門侍郎植等十一人誅之唯宥其孫孺【唯宥諸孫之在童孺者}
又殺司空參軍薛彤高進之二人皆道濟腹心有勇力時人比之關張【關羽張飛也}
道濟見收憤怒目光如炬脫幘投地曰乃壞汝萬里長城【壞音怪}
魏人聞之喜曰道濟死吳子輩不足復憚【為後魏人入寇帝思道濟張本}
庚申大赦以中軍將軍南譙王義宣為江州刺史 辛未魏平東將軍娥清安西將軍古弼將精騎一萬伐燕平州刺史拓跋嬰帥遼西諸軍會之【將即亮翻騎奇寄翻帥讀曰率下同}
 氐王楊難當自稱大秦王改元建義立妻為王后世子為太子置百官皆如天子之制然猶貢奉宋魏不絶 夏四月魏娥清古弼攻燕白狼城克之【白狼縣漢屬右北平郡燕以白狼城為重鎮置并州魏後併入建德郡廣都縣有白狼山白狼水}
高麗遣其將葛盧孟光將衆數萬随陽伊至和龍迎燕王【去年燕遣陽伊請迎于高麗}
高麗屯于臨川【臨川在和龍城東}
燕尚書令郭生因民之憚遷開城門納魏兵【考異曰後魏古弼傳作大臣古泥今從十六國春秋鈔}
魏人疑之不入生遂勒兵攻燕王王引高麗兵入自東門與生戰于闕下生中流矢死【中竹仲翻}
葛盧孟光入城命軍士脱弊褐取燕武庫精仗以給之大掠城中五月乙卯燕王帥龍城見戶東徙【帥讀曰率見賢遍翻馮氏歷二主二千八年而滅}
焚宫殿火一旬不滅令婦人被甲居中陽伊等勒精兵居外葛盧孟光帥騎殿後【被皮義翻殿丁甸翻}
方軌而進前後八十餘里古弼部將高苟子帥騎欲追之【將即亮翻帥讀曰率騎奇寄翻}
弼醉拔刀止之故燕王得逃去魏主聞之怒檻車徵弼及娥清至平城皆黜為門卒戊午魏主遣散騎常侍封撥使高麗【散悉亶翻騎奇寄翻使疏吏翻}
令送燕王 丁卯魏主如河西 六月詔寧朔將軍蕭汪之將兵討程道養軍至郪口【郪江源出今潼川府銅山縣歷遂寧府長江縣而合于涪水謂之郪口郪音妻}
帛氐奴請降【降戶江翻}
道養兵敗還入郪山 赫連定之西遷也【事見上卷八年}
楊難當遂據上邽秋七月魏主遣驃騎大將軍樂平王丕尚書令劉絜督河西高平諸軍以討之先遣平東將軍崔賾齎詔書諭難當【驃匹妙翻騎奇寄翻賾士革翻}
 魏散騎侍郎游雅來聘【姓譜鄭公子偃字子游後以為氏魏為廣平望姓}
 己未零陵王太妃禇氏卒追諡曰晉恭思皇后葬以晉禮 八月魏主畋于河西 魏主遣廣平公張黎發定州兵一萬二千通莎泉道【莎泉在靈丘魏收地形志靈丘郡有莎泉縣隋廢靈丘為縣併莎泉入焉莎素何翻}
九月庚戌魏樂平王丕等至略陽楊難當懼請奉詔攝上邽守兵還仇池諸將議以為不誅其豪帥【帥所類翻}
軍還之後必相聚為亂又大衆遠出不有所掠無以充軍實賞將士丕將從之中書侍郎高允參丕軍事諫曰如諸將之謀是傷其向化之心大軍既還為亂必速丕乃止【還從宣翻又如字}
撫慰初附秋毫不犯秦隴遂安難當以其子順為雍州刺史鎮下辨【雍於用翻辨步莧翻}
 高麗不送燕王於魏遣使奉表稱當與馮弘俱奉王化魏主以高麗違詔議擊之將隴右騎卒【麗力知翻使疏吏翻騎奇寄翻}
劉絜曰秦隴新民且當優復【新民新附之民也復方目翻}
俟其饒實然後用之樂平王丕曰和龍新定宜廣脩農桑以豐軍實然後進取則高麗一舉可滅也魏主乃止 癸丑封皇子濬為始興王駿為武陵王 冬十一月己酉魏主如棝陽【棝音固}
驅野馬於雲中置野馬苑閏月壬子還宫 初高祖克長安【事見一百十八卷晉安帝義熙十三年}
得古銅渾儀【渾戶本翻}
儀狀雖舉不綴七曜【日月五星謂之七曜}
是歲詔太史令錢樂之更鑄渾儀徑六尺八分以水轉之昏明中星與天相應【孟春之月昏參中旦尾中仲春之月昏弧中旦建星中季春之月昏七星中旦牽牛中孟夏之月昏翼中旦婺女中仲夏之月昏亢中旦危中季夏之月昏火中旦奎中孟秋之月昏建星中旦畢中仲秋之月昏牽牛中旦觜觿中季秋之月昏虚中旦柳中孟冬之月昏危中旦七星中仲冬之月昏東壁中旦軫中季冬之月昏婁中旦氐中更工衡翻}
柔然與魏絶和親犯魏邊【柔然與魏和見上卷八年}
 吐谷渾惠王慕璝卒弟慕利延立【璝古冋翻}
十四年春正月戊子魏北平宣王長孫嵩卒【長知兩翻}
 辛卯大赦 二月乙卯魏主如幽州三月丁丑魏主以南平王渾為鎮東大將軍儀同三司鎮和龍己卯還宫 帝遣散騎常侍劉熙伯如魏議納幣【散悉亶翻騎奇寄翻}
會帝女亡而止【魏請婚始上卷十年}
 夏四月趙廣張尋梁顯等各帥衆降别將王道恩斬程道養送首餘黨悉平【九年趙廣等反今乃平帥讀曰率降戶江翻將即亮翻}
丁未以輔國將軍周籍之為益州刺史 魏主以民官多貪【郡守縣令親民之官}
夏五月己丑詔吏民得舉告守令不如法者於是姦猾專求牧宰之失廹脇在位横於閭里【横戶孟翻}
而長吏咸降心待之貪縱如故【長知兩翻}
 丙申魏主如雲中 秋七月戊子魏永昌王健等討山胡白龍餘黨于西河滅之八月甲辰魏主如河西九月甲申還宫 丁酉魏主

  遣使者拜吐谷渾王慕利延為鎮西大將軍儀同三司改封西平王 冬十月癸卯魏主如雲中十一月壬申還宫 魏主復遣散騎侍郎董琬高明等多齎金帛使西域招撫九國【復扶又翻下不復為復同九國入貢見上卷十二年}
琬等至烏孫其王甚喜曰破落那者舌二國【破落那漢大宛國也去代萬四千四百五十里者舌漢康居國也去代萬五千四百五十里}
皆欲稱臣致貢于魏但無路自致耳今使君宜過撫之乃遣導譯送琬詣破落那明詣者舌旁國聞之爭遣使者随琬等入貢凡十六國自是每歲朝貢不絶【朝直遥翻}
 魏主以其妹武威公主妻河西王牧犍【妻七細翻犍居言翻}
河西王遣宋繇奉表詣平城謝且問公主所宜稱魏主使羣臣議之皆曰母以子貴妻從夫爵【春秋之義母以子貴禮記婦人無爵從夫之爵}
牧犍母宜稱河西國太后公主於其國稱王后於京師則稱公主魏主從之初牧犍娶凉武昭王之女及魏公主至李氏與其母尹氏遷居酒泉頃之李氏卒尹氏撫之不哭曰汝國破家亡今死晚矣牧犍之弟無諱鎮酒泉謂尹氏曰后諸孫在伊吾【李寶奔伊吾見一百十九卷營陽王景平元年}
后欲就之乎尹氏未測其意紿之曰吾子孫漂蕩託身異域餘生無幾當死此不復為氊裘之鬼也未幾潜奔伊吾無諱遣騎追及之尹氏謂追騎曰沮渠酒泉許吾歸北何為復追汝取吾首以往吾不復還矣追騎不敢逼引還尹氏卒于伊吾【史言尹氏義烈紿蕩亥翻幾居豈翻復扶又翻騎奇寄翻卒子恤翻}
牧犍遣將軍沮渠旁周入貢于魏魏主遣侍中古弼尚書李順賜其侍臣衣服并徵世子封壇入侍是歲牧犍遣封壇如魏亦遣使詣建康【使疏吏翻}
獻雜書及敦煌趙所撰甲寅元歷【敦徒門翻讀為斐魏書作撰士免翻}
并求雜書數十種【種章勇翻}
帝皆與之李順自河西還魏主問之曰卿往年言取凉州之策朕以東方有事未遑也今和龍已平吾欲即以此年西征可乎對曰臣疇昔所言【見上卷十年}
以今觀之私謂不謬然國家戎車屢動士馬疲勞西征之議請俟他年魏主乃止

  十五年春二月丁未以吐谷渾王慕利延為都督西秦河沙三州諸軍事鎮西大將軍西秦河二州刺史隴西王 三月癸未魏主詔罷沙門年五十以下者【以其彊壯罷使為民以從征役}
 初燕王弘至遼東高麗王璉遣使勞之曰【麗力知翻璉力展翻使疏吏翻勞力到翻}
龍城王馮君爰適野次士馬勞乎弘慙怒稱制讓之高麗處之平郭【處昌呂翻}
尋徙北豐弘素侮高麗政刑賞罰猶如其國高麗乃奪其侍人取其太子王仁為質【質音致}
弘怨高麗遣使上表求迎上遣使者王白駒等迎之并令高麗資遣高麗王不欲使弘南來遣將孫漱高仇等殺弘于北豐并其子孫十餘人【果如楊㟭之言將即亮翻下同}
諡弘曰昭成皇帝白駒等帥所領七千餘人掩討漱仇殺仇生擒漱高麗王以白駒等專殺遣使執送之上以遠國不欲違其意下白駒等獄【下遐稼翻}
已而原之夏四月納故黄門侍郎殷淳女為太子劭妃 五月

  戊寅魏大赦 丙申魏主如五原秋七月自五原北伐柔然命樂平王丕督十五將出東道永昌王健督十五將出西道魏主自出中道至浚稽山復分中道為二陳留王崇從大澤向涿邪山魏主從浚稽北向天山西登白阜【天山在漢北即唐鐵勒思結多濫葛所保之地非伊吾之折羅漫山也白阜疑即雪山復扶又翻邪讀曰耶}
不見柔然而還時漠北大旱無水草人馬多死冬十一月丁卯朔日冇食之 十二月丁巳魏主至平城 豫章雷次宗好學【好呼到翻下同}
隱居廬山【廬山在尋陽今在南康城北十五里尋陽之南正面廬山}
嘗徵為散騎侍郎不就是歲以處士徵至建康為開舘於雞籠山【雞籠山在臺城北郊散悉亶翻騎奇寄翻處昌呂翻為于偽翻}
使聚徒教授帝雅好藝文使丹楊尹廬江何尚之立玄學太子率更令何承天立史學【晉志太子率更令主宫殿門戶及賞罰事職如光禄勲衛尉更工衡翻}
司徒參軍謝元立文學并次宗儒學為四學元靈運之從祖弟也帝數幸次宗學館令次宗以巾褠侍講【從才用翻數所角翻褠古侯翻江南人士交際以為盛服蓋次于朝服毛脩之不肯以巾褠到殷景仁之門是也蜀註曰巾謂巾幘褠謂單衣}
資給甚厚又除給事中不就久之還廬山

  臣光曰易曰君子多識前言往行以畜其德【行下孟翻易大畜彖辭}
孔子曰辭達而已矣【論語所記}
然則史者儒之一端文者儒之餘事至于老莊虛無固非所以為教也夫學者所以求道天下無二道安有四學哉

  帝性仁厚恭儉勤於為政守法而不峻容物而不弛百官皆久于其職守宰以六朞為斷吏不苟免民有所係三十年間四境之内晏安無事戶口蕃息【斷丁亂翻蕃音頓}
出租供徭止於歲賦【歲賦常賦也言不額外取民}
晨出暮歸自事而已【自適已事而已}
閭閻之間講誦相聞士敦操尚鄉恥輕薄江左風俗於斯為美後之言政治者皆稱元嘉焉【治直吏翻}
十六年春正月庚寅司徒義康進位大將軍領司徒【自漢以來大將軍位三公上司徒丞相職也義康既進位猶領司徒職}
南兗州刺史江夏王義恭進位司空【夏戶雅翻}
 魏主如定州 初高祖遺詔令諸子次第居荆州臨川王義慶在荆州八年欲為之選代【為于偽翻}
其次應在南譙王義宣帝以義宣人才凡鄙置不用二月己亥以衡陽王義季為都督荆湘等八州諸軍事荆州刺史義季嘗春月出畋有老父被苫而耕【被皮義翻苫詩廉翻}
左右斥之老父曰盤于遊畋古人所戒【夏太康以遊畋失國周公以文王戒成王盤樂也}
今陽和布氣一日不耕民失其時奈何以從禽之樂而驅斥老農也【樂音洛}
義季止馬曰賢者也命賜之食辭曰大王不奪農時則境内之民皆飽大王之食老夫何敢獨受大王之賜乎義季問其名不告而退 三月魏雍州刺史葛那宼上洛【魏雍州刺史治長安此北上洛也南上洛寄治魏興魏書官氏志内入諸姓賀葛氏改為葛氏雍於用翻}
上洛太守鐔長生棄郡走【鐔徐林翻}
 辛未魏主還宫 楊保宗與兄保顯自童亭奔魏【保宗鎮童亭見上卷十二年}
庚寅魏主以保宗為都督隴西諸軍事征西大將軍開府儀同三司秦州牧武都王鎮上邽妻以公主【妻七細翻}
保顯為鎮西將軍晉夀公 河西王牧犍通于其嫂李氏兄弟三人傳嬖之【傳逓也以便辟得幸曰嬖嬖卑義翻陸德明必計翻}
李氏與牧犍之姊共毒魏公主【公主魏主妹也}
魏主遣解毒醫乘傳救之得愈【傳知戀翻}
魏主徵李氏牧犍不遣厚資給使居酒泉魏每遣使者詣西域【使疏吏翻下同}
常詔牧犍導護送出流沙使者自西域還至武威牧犍左右有告魏使者曰我君承蠕蠕可汗妄言云【可從刋入聲汗音寒}
去歲魏天子自來伐我士馬疫死大敗而還【還從宣翻又如字下同}
我擒其長弟樂平王丕我君大喜宣言於國又聞可汗遣使告西域諸國稱魏已削弱今天下唯我為彊若更有魏使勿復供奉西域諸國頗有貳心【復扶又翻下能復同兩屬曰貳西域既貢奉魏又信柔然之言是有貳心}
使還具以狀聞魏主遣尚書賀多羅使凉州觀虛實多羅還亦言牧犍雖外脩臣禮内實乖悖【悖蒲内翻又蒲没翻}
魏主欲討之以問崔浩對曰牧犍逆心已露不可不誅官軍往年北伐雖不克獲實無所損戰馬三十萬匹計在道死傷不滿八千常歲羸死亦不減萬匹【羸倫為翻}
而遠方乘虛遽謂衰耗不能復振今出其不意大軍猝至彼必駭擾不知所為擒之必矣魏主曰善吾意亦以為然於是大集公卿議於西堂【魏平城太極殿有東西堂}
弘農王奚斤等三十餘人皆曰牧犍西垂下國雖心不純臣然繼父位以來職貢不乏朝廷待以藩臣妻以公主【妻七細翻}
今其罪惡未彰宜加恕宥國家新征蠕蠕【蠕人兖翻}
士馬疲弊未可大舉且聞其土地鹵瘠【鹹地曰鹵瘠地曰瘠}
難得水草大軍既至彼必嬰城固守攻之不拔野無所掠此危道也初崔浩惡尚書李順【伐夏之役浩順有隙順以使凉為魏主所寵待浩愈惡之惡烏路翻}
順使凉州凡十二返【使疏吏翻}
魏主以為能凉武宣王數與順遊宴【沮渠蒙遜諡武宣王數所角翻}
對其羣下時為驕慢之語恐順泄之随以金寶納于順懷順亦為之隱【亦為于偽翻}
浩知之密以白魏主魏主未之信及議伐凉州順與尚書古弼皆曰自温圉水以西至姑臧【據北史温圉水當作温圉}
地皆枯石絶無水草彼人言姑臧城南天梯山上冬有積雪深至丈餘春夏消釋下流成川居民引以溉灌彼聞軍至決此渠口水必乏絶【彼人謂凉人也}
環城百里之内【環音宦}
地不生草人馬饑渴難以久留斤等之議是也魏主乃命浩與斤等相詰難衆無復他言但云彼無水草浩曰漢書地理志稱凉州之畜為天下饒若無水草畜何以蕃【詰去吉翻難乃旦翻復扶又翻下敢復同漢書地理志曰凉州土廣民稀水草宜畜牧故凉州之畜為天下饒畜許救翻又許六翻蕃音繁}
又漢人終不於無水草之地築城郭建郡縣也且雪之消釋僅能斂塵何得通渠溉灌乎此言大為欺誣矣【欺誑也誣詐也}
李順曰耳聞不如目見吾嘗目見何可共辨浩曰汝受人金錢欲為之遊說謂我目不見便可欺邪帝隱聽聞之【為于偽翻說輸芮翻隱聽者隱屏而聽也}
乃出見斤等辭色嚴厲羣臣不敢復言唯唯而已【唯于癸翻}
羣臣既出振威將軍代人伊馛言於帝【馛蒲撥翻}
曰凉州若果無水草彼何以為國衆議皆不可用宜從浩言帝善之夏五月丁丑魏主治兵於西郊【治直之翻}
六月甲辰平城使侍中宜都王穆夀輔太子晃監國決留臺事内外聽焉【監工銜翻}
又使大將軍長樂王稽敬【稽敬北史作嵇敬當從之魏書官氏志北方諸姓紇奚氏改為嵇氏}
輔國大將軍建寧王崇將二萬人屯漠南以備柔然【王將即亮翻}
命公卿為書以讓河西王牧犍數其十二罪且曰若親帥羣臣委贄遠迎【數所具翻帥讀曰率古者執贄以見拜贄首則委之於地起則取而進之此之謂委贄}
謁拜馬首上策也六軍既臨面縛輿櫬其次也若守迷窮城不時悛悟【櫬初覲翻悛丑緣翻}
身死族滅為世大戮宜思厥中自求多福 己酉改封隴西王吐谷渾慕利延為河南王 魏主自雲中濟河秋七月己巳至上郡屬國城【漢置屬國於邉郡以處降胡此屬國城漢舊城也班書地理志上郡龜兹縣屬國都尉治}
壬午留輜重部分諸軍【重直用翻分扶問翻}
使撫軍大將軍永昌王健尚書令劉絜與常山王素為前鋒兩道並進驃騎大將軍樂平王丕太宰陽平王杜超為後繼【驃匹妙翻騎奇寄翻}
以平西將軍源賀為鄉導【鄉讀曰嚮}
魏主問賀以取凉州方略對曰姑臧城旁有四部鮮卑皆臣祖父舊民【秃髪傉檀據姑臧既而為沮渠所取有四部鮮卑留居城外賀傉檀之子也}
臣願處軍前【處昌呂翻}
宣國威信示以禍福必相帥歸命【帥讀曰率}
外援既服然後取其孤城如反掌耳魏主曰善八月甲午永昌王健獲河西畜產二十餘萬河西王牧犍聞有魏師驚曰何為乃爾用左丞姚定國計不肯出迎求救于柔然遣其弟征南大將軍董來將兵萬餘人出戰於城南【來將即亮翻}
望風奔潰劉絜用卜者言以為日辰不利斂兵不追董來遂得入城魏主由是怒之【為魏誅劉絜張本}
丙申魏主至姑臧遣使諭牧犍令出降【使疏吏翻降戶江翻}
牧犍聞柔然欲入魏邉為寇冀幸魏主東還遂嬰城固守其兄子祖踰城出降魏主具知其情乃分軍圍之源賀引兵招慰諸部下三萬餘落故魏主得專攻姑臧無復外慮【復扶又翻下不復同}
魏主見姑臧城外水草豐饒由是恨李順謂崔浩曰卿之昔言今果驗矣【自是魏主决意誅李順矣}
對曰臣之言不敢不實類皆如此魏主之將伐凉州也太子晃亦以為疑至是魏主賜太子詔曰姑臧城西門外涌泉合於城北其大如河自餘溝渠流入漠中其間乃無燥地故有此敕以釋汝疑 庚子立皇子鑠為南平王【鑠書藥翻}
 九月丙戌河西王牧犍兄子萬年帥所領降魏 【考異曰宋書氐胡傳曰茂䖍兄子萬年為虜内應茂䖍見執今從後魏書帥讀曰率降戶江翻下同}
姑臧城潰牧犍帥其文武五千人面縛請降魏主釋其縛而禮之收其城内戶口二十餘萬倉庫珍寶不可勝計使張掖王秃髮保周【保周奔魏封張掖公今進為王保周源賀之兄也勝音升}
龍驤將軍穆羆【驤思將翻}
安遠將軍源賀分徇諸郡雜胡降者又數十萬初牧犍以其弟無諱為沙州刺史都督建康以西諸軍事領酒泉太守宜得為秦州刺史都督丹嶺以西諸軍事領張掖太守【丹嶺在姑臧西即刪丹嶺}
安周為樂都太守【乞伏衰滅樂都亦為沮渠所有樂音洛 考異曰宋書宜得作儀德安周作從子豐周今從後魏書}
從弟唐兒為敦煌太守【從才用翻敦徒門翻}
及姑臧破魏主遣鎮南將軍代人奚眷擊張掖鎮北將軍封沓擊樂都宜得燒倉庫西奔酒泉安周南奔吐谷渾封沓掠數千戶而還【還從宣翻又如字}
奚眷進攻酒泉無諱宜得收遺民奔晉昌遂就唐兒於敦煌魏主使弋陽公元絜守酒泉及武威張掖皆置將守之【將即亮翻}
魏主置酒姑臧謂羣臣曰崔公知略有餘吾不復以為奇【復扶又翻}
伊馛弓馬之士而所見乃與崔公同深可奇也馛善射能曳牛却行走及奔馬而性忠謹故魏主特愛之魏主之西伐也穆夀送至河上【自平城送魏主西至河}
魏主敕之曰吳提與牧犍相結素深聞朕討牧犍吳提必犯塞【柔然敕連可汗名吳提}
朕故留壯兵肥馬使卿輔佐太子收田既畢即兵詣漠南分伏要害以待虜至引使深入然後擊之無不克矣凉州路遠朕不得救卿勿違朕言夀頓首受命夀雅信中書博士公孫質以為謀主夀質皆信卜筮以為柔然必不來不為之備質軌之弟也【公孫軌見一百二十卷四年}
柔然敕連可汗聞魏主向姑臧乘虚入宼留其兄乞列歸與嵇敬建寜王崇相拒于北鎮【北鎮即魏主破降高車所置六鎮也以在平城之北故曰北鎮或曰北鎮直代都北即懷朔鎮}
自帥精騎深入【帥讀曰率騎奇寄翻}
至善無七介山平城大駭民爭走中城【走音奏}
穆壽不知所為欲塞西郭門【塞悉則翻}
請太子避保南山竇太后不聽而止【竇太后即保太后}
遣司空長孫道生征北大將軍張黎拒之於吐頹山會嵇敬建寧王崇擊破乞列歸於隂山之北擒之并其伯父他吾無鹿胡及將帥五百人【將即亮翻帥所類翻}
斬首萬餘級敕連聞之遁去追至漠南而還冬十月辛酉魏主東還留樂平王丕及征西將軍賀多羅鎮凉州徙沮渠牧犍宗族及吏民三萬戶于平城【考異曰十六國春秋鈔云十萬戶今從後魏書}
 癸亥秃髪保周帥諸部鮮卑據張掖叛魏【帥讀曰率}
 十二月乙亥太子劭加元服大赦劭美鬚眉好讀書便弓馬喜延賓客【好呼到翻喜許記翻下尤喜同}
意之所欲上必從之東宫置兵與羽林等【師古曰羽林宿衛之官言其如羽之疾如林之多也為邵以東宫兵弑逆張本}
 壬午魏主至平城以柔然入寇無大失亡故穆壽等得不誅魏主猶以妹婿待沮渠牧犍征西大將軍河西王如故牧犍母卒葬以太妃之禮武宣王置守冢三十家【為沮渠蒙遜置守冢}
凉州自張氏以來號為多士【永嘉之亂中州之人士避地河西張氏禮而用之子孫相承衣冠不墜故凉州號為多士}
沮渠牧犍尤喜文學以敦煌闞駰為姑臧太守【敦徒門翻闞苦濫翻駰音因}
張湛為兵部尚書【曹魏置五兵尚書據此則兵部之號起于河西}
劉昞索敞隂興為國師助教金城宋欽為世子洗馬【索昔各翻洗悉薦翻}
趙柔為金部郎【曹魏置二十三郎金部其一也主財帛委輸}
廣平程駿駿從弟弘為世子侍講【從才用翻}
魏主克凉州皆禮而用之以闞駰劉昞為樂平王丕從事中郎安定胡叟少有俊才【少詩照翻}
往從牧犍牧犍不甚重之叟謂程弘曰貴主居僻陋之國而淫名僭禮以小事大而心不純壹外慕仁義而實無道德其亡可翹足待也吾將擇木【左傳衛孔文子將攻太叔疾訪於仲尼仲尼曰甲兵之事未之學也退命駕而行曰鳥則擇木木豈能擇鳥}
先集于魏與子暫違非久闊也遂適魏歲餘而牧犍敗魏主以叟為先識拜虎威將軍賜爵始復男【按地名無始復漢書地理志越嶲郡有姑復縣或者始字其姑字之誤乎}
河内常爽世寓凉州不受禮命魏主以為宣威將軍河西右相宋繇從魏主至平城而卒【相息亮翻卒子恤翻}
魏主以索敞為中書博士時魏朝方尚武功【朝直遥翻}
貴遊子弟不以講學為意【鄭玄曰貴遊子弟王公之子弟遊無官司者}
敞為博士十餘年勤於誘導肅而有禮【誘音酉}
貴遊皆嚴憚之多所成立前後顯達至尚書牧守者數十人【守手又翻}
常爽置館於温水之右【水經註桑乾城西十里有温湯}
教授七百餘人爽立賞罰之科弟子事之如嚴君由是魏之儒風始振高允每稱爽訓厲有方曰文翁柔勝先生剛克【漢景帝末文翁為蜀郡守仁愛好教化選郡縣小吏開敏冇才者詣京師受業博士又修起學宫于成都市中招下縣子弟為學官弟子為除更繇由是大化至今巴蜀好儒雅文翁之教也克亦勝也言文翁以柔勝而常爽以剛勝也}
立教雖殊成人一也陳留江強寓居凉州獻經史諸子千餘卷及書法亦拜中書博士【魏延昌三年強孫式上表曰臣聞伏羲氏作而八卦形其畫軒轅氏興而靈龜彰其彩古史倉頡覽二象之爻觀鳥獸之迹别剏文字以代結繩用書契以維事迄于三代厥體頗異雖依類取制未能達倉氏矣故周禮八歲入小學保氏教以六書蓋是史頡之遺法及宣王太史史籀著大篆十五篇與古文或同或異時書謂之籀書孔子脩六經左丘明述春秋皆以古文厥意可得而言其後七國殊軌文字乖引暨秦兼天下丞相李斯乃奏蠲罷不合秦文者斯作倉頡篇車府令高作爰歷篇太史令胡母敬作博學篇皆取史籀式頗有省改所謂小者也於是秦燒經書滌除舊典官獄煩多以趣簡易始用隸書古文自此息矣隸書者始皇使下杜人程邈附于小篆所作也世人以邈徒隸即謂之隸書故秦有八體一曰大二曰小三曰符書四曰蟲書五曰摹印六曰署書七曰殳書八曰隸書漢興有尉律學復教以籀書又習八體試之課最以為尚書史書省字不正輒舉劾焉又有草書莫知誰始其形書雖無厥誼亦一時之變通也孝宣時召通倉頡讀者獨張敞從受之凉州刺史杜業沛人爰禮講學大夫秦近亦能言之孝平時徵禮等百餘人說文字於未央宫中以禮為小學元士黄門侍郎揚雄採以作訓纂篇及亡新居攝自以運應制作使大司馬甄豐校文字之部頗改定古文時有六書一曰古文孔子壁中書也二曰奇字即古文而異者三曰書云小也四曰佐書秦隸書也五曰繆所以摹印也六曰鳥蟲所以書旛信也壁中書者魯恭王壞孔子宅而得尚書春秋論語孝經也又北平侯張蒼獻春秋左氏傳書體與孔氏相類即前代之古文矣後漢扶風曹喜號曰工篆小異斯法而甚精巧自是後學皆其法也又詔侍中賈逵脩理舊文殊藝異術王教一端苟有可以加于國者靡不悉集逵即汝南許慎古學之師也後慎嗟時人之好奇歎俗儒之穿鑿故撰說文解字十五篇首一終玄各有部屬可謂類聚羣分雜而不越文質彬彬最可得而論也左中郎將蔡邕採李斯曹喜之法以為古今雜形詔于太學立石碑刋載五經題書楷法多是邕書也後開鴻都書畫奇能莫不雲集特諸方獻篆無出邕者魏初博士清河張揖著埤蒼廣雅古今字詁方之許篇古今體用或得或失陳留邯鄲淳亦與揖同博聞古藝特善蒼雅許氏字指八體六書精究厥理有名於揖以書教諸皇子又建三字石經於漢碑西其文蔚煥三體復宣較之說文篆隸大同而古字小異又有京兆韋誕河東衛覬二家並號能篆當時臺觀牋題寶器之銘悉是誕書咸傳之子孫世稱其妙晉世呂忱表上字林六卷尋其况趣附託許慎說文而按偶章句隱别古籀奇惑之字文得正隸不差篆意也忱弟静别倣故左校令李登聲類之法作韻集五卷使宫商録徵羽各為一篇而文字與兄便是魯衛音讀楚夏時有不同皇魏承百王之季紹五運之緒世易風移文字改變篆形謬錯隸體失真俗學鄙習復加虚造巧談辨士以意為疑炫惑于時難以釐改乃曰追來為歸巧言為辯小兎為□神虫為蠶如斯甚衆皆不合孔氏古書史籀大篆許氏說文石經三字也嗟夫文字者六籍之宗王教之始前人所以垂今今人所以識古臣六世祖瓊家世陳留住晉之初與從父兄皆受學于衛覬古篆之法蒼雅方言說文之誼當時並收善譽而祖遇洛陽之亂避地河西數世傳習斯業所以不墜也世祖大延中牧犍内附臣亡祖文威杖策歸國奉獻五世傳掌之書古篆八體之法時蒙褒録叙列于儒林官班文省家號世業臣藉六世之資奉遵祖考之訓切慕古人乏軌企踐儒門之轍求撰集古來文字以許慎說文為主及孔氏尚書五經音註籀篇爾雅三蒼凡將方言通俗文祖文宗埤蒼廣雅古今字詁三字石經字林韻集諸賦文字有六書之誼者以類編聯文無複重統為一部其石籀奇惑俗隸諸體咸使班于篆下各有區别訓詁假借之誼随文而解音讀楚夏之聲逐字而註其所不知則闕如也冀省百氏之觀而同文字之域詔如所請中書自曹魏置監令以來未嘗置博士蓋拓抜氏初置是官也}
魏主命崔浩監秘書事【監工銜翻}
綜理史職以中書侍郎高允散騎侍郎張偉參典著作【曹魏明帝景初初中書改置監令又置通事郎及晉改曰中書侍郎散悉亶翻騎奇寄翻}
浩啟稱殷仲逵段承根凉土美才請同修國史皆除著作郎仲逵武威人承根暉之子也【段暉事乞伏熾磐暮末父子}
浩集諸歷家考校漢元以來日月薄食五星行度【漢元漢初也}
并譏前史之失别為魏歷以示高允允曰漢元年十月五星聚東井【見九卷漢高帝元年考異}
此乃歷術之淺事今譏漢史而不覺此謬恐後人之譏今猶今之譏古也浩曰所謬云何允曰案星傳太白辰星常附日而行十月日在尾箕【孟冬之月日在尾言在尾箕者竟一月言之也傳直戀翻}
昏没于申南而東井方出于寅北二星何得背日而行【背蒲妹翻}
是史官欲神其事不復推之於理也【復如字又扶又翻}
浩曰天文欲為變者何所不可邪允曰此不可以空言爭宜更審之坐者咸怪允之言唯東宫少傅游雅曰【東宫少傅即太子少傅少詩照翻}
高君精於歷數當不虚也後歲餘浩謂允曰先所論者本不經心及更考究果如君言五星乃以前三月聚東井非十月也衆乃歎服允雖明歷初不推步及為人論說【為于偽翻}
唯游雅知之雅數以災異問允【數所角翻}
允曰隂陽災異知之甚難既已知之復恐漏泄【復扶又翻}
不如不知也天下妙理至多何以問此雅乃止魏主問允為政何先時魏多封禁良田允曰臣少賤【允自言其少也賤少詩照翻}
唯知農事若國家廣田積穀公私有備則飢饉不足憂矣帝乃命悉除田禁以賦百姓 吐谷渾王慕利延聞魏克凉州大懼帥衆西遁踰沙漠【帥讀曰率}
魏主以其兄慕璝有擒赫連定之功【事見上卷八年}
遣使撫諭之【使疏吏翻}
慕利延乃還故地 氐王楊難當將兵數萬寇魏上邽【將即亮翻下同}
秦州人多應之東平呂羅漢說鎮將拓跋意頭曰難當衆盛今不出戰示之以弱衆情離沮不可守也意頭遣羅漢將精騎千餘出衝難當陳所向披靡【說輸芮翻陳讀曰陣披普彼翻}
殺其左右騎八人難當大驚會魏主以璽書責讓難當【璽斯氏翻}
難當引還仇池南豐太妃司馬氏卒故營陽王之后也【九年帝以江夏王義恭子朗為南豐王奉營陽王祀以后為南豐太妃}
 趙廣張尋等復謀反伏誅【十四年廣尋降至建康復謀反復扶又翻下復能復那不復無復同}


  十七年春正月己酉沮渠無諱寇魏酒泉元絜輕之出城與語壬子無諱執絜以圍酒泉 二月魏假通直常侍邢頴來聘【散騎常侍秦官也曹魏末增置員外散騎常侍晉武帝秦始十年使員外二人與散騎常侍通直故謂之通直散騎常侍頴假以出使非正官也}
 三月沮渠無諱拔酒泉 夏四月戊午朔日有食之 庚辰沮渠無諱寇魏張掖禿髪保周屯刪丹【刪丹縣漢屬張掖郡後分屬西郡唐屬甘州居延海在縣界}
丙戌魏主遣撫軍大將軍永昌王健督諸將討之【將即亮翻}
司徒義康專總朝權上羸疾積年【羸倫為翻}
心勞輒屢

  至危殆義康盡心營奉藥石非口所親嘗不進或連夕不寐内外衆事皆專決施行性好吏職【好呼到翻下好於同}
糾剔文案莫不精盡上由是多委以事凡所陳奏入無不可方伯以下並令義康選用生殺大事或以録命斷之【義康録尚書故謂其命為録命斷丁亂翻}
勢傾遠近朝野輻湊每旦府門常有車數百乘【朝直遥翻乘繩證翻}
義康傾身引接未嘗懈倦復能強記耳目所經終身不忘好于稠人廣席標題所憶以示聰明士之幹練者多被意遇【懈古隘翻復扶又翻下同好呼到翻被皮義翻}
嘗謂劉湛曰王敬弘王球之屬竟何所堪坐取富貴復那可解【王敬弘恬淡有重名王球簡貴虚静皆以門望位入坐不以文按關心故義康云然解戶買翻}
然素無學術不識大體朝士有才用者皆引入已府府僚無施及忤旨者乃斥為臺官【晉宋以來謂天朝為天臺忤五故翻}
自謂兄弟至親不復存君臣形迹率心而行曾無猜防私置僮六千餘人不以言臺四方獻饋皆以上品薦義康而以次者供御上嘗冬月噉甘【甘似橘而巨其皮黄於橘其味甘於橘}
歎其形味並劣義康曰今年甘殊有佳者遣人還東府取甘大供御者三寸領軍劉湛與僕射殷景仁有隙【事見上卷十二年}
湛欲倚義康之重以傾之義康權勢已盛湛愈推崇之無復人臣之禮上浸不能平湛初入朝上恩禮甚厚湛善論治道諳前代故事叙致銓理【致極致也理文理也言叙其極致又銓次其文理也治直吏翻諳烏含翻}
聽者忘疲每入雲龍門御者即解駕左右及羽儀随意分散不夕不出以此為常及晚節驅煽義康【馬方走而疾其鞭策曰驅火方熾而鼔其氣燄曰煽}
上意雖内離而接遇不改嘗謂所親曰劉班方自西還宫與語常視日早晚慮其將去比入吾亦視日早晚苦其不去【湛小字班虎故稱之為班比毗至翻近也}
殷景仁密言於上曰相王權重非社稷計宜少加裁抑【相息亮翻少詩沼翻}
上隂然之司徒左長史劉斌湛之宗也【斌音彬}
大將軍從事中郎王履謐之孫也【王謐識武帝於微時晉宋之際位任通顯}
及主簿劉敬文祭酒魯郡孔胤秀皆以傾謟有寵于義康見上多疾皆謂宫車一日晏駕宜立長君【長知兩翻}
上嘗疾篤使義康具顧命詔義康還省流涕以告湛及景仁湛曰天下艱難詎是幼主所御義康景仁並不答【考異曰南史以為義康有此言湛景仁並不答按義康雖不識大體豈敢自為此言湛常欲推崇義康豈肯}


  【聞而不答今從宋書及宋略}
而胤秀等輒就尚書議曹索晉咸康末立康帝舊事【議曹南史作儀曹當從之曹魏置二十三郎儀曹其一也立康帝事見九十七卷索山客翻}
義康不知也及上疾瘳微聞之【瘳丑留翻}
而斌等密謀欲使大業終歸義康遂邀結朋黨伺察禁省【伺相吏翻}
有不與己同者必百方構䧟之又採拾景仁短長或虛造異同以告湛自是主相之勢分矣【相息亮翻}
義康欲以劉斌為丹楊尹【斌音彬}
言次啟上陳其家貧言未卒【卒子恤翻}
上曰以為吳郡後會稽太守羊玄保求還【會工外翻}
義康又欲以斌代之啟上曰羊玄保求還不審以誰為會稽上時未有所擬倉猝曰我已用王鴻自去年秋上不復往東府【史言帝已疎忌義康而義康貪戀權勢惑于附麗者不能引退復扶又翻下同}
五月癸巳劉湛遭母憂去職湛自知罪舋已彰無復全地【舋許覲翻}
謂所親曰今年必敗常日正賴口舌爭之故得推遷耳今既窮毒【謂母子相訣則人理窮而罹荼毒也}
無復此望禍至其能久乎 【考異曰南史云湛伏甲於室以俟上臨弔謀又泄竟弗之幸宋書無此事按湛若謀泄當即伏誅豈得尚延半歲今從宋書}
乙巳沮渠無諱復圍張掖不克退保臨松【臨松郡臨松縣當是沮渠氏所置後宇文周廢入張掖復扶又翻下同}
魏主不復加討但以詔諭之六月丁丑魏皇孫濬生大赦改元太平真君取寇謙之神書云輔佐北方太平真君故也【寇謙之神書見一百十九卷營陽王景平元年}
 太子劭詣京口拜京陵司徒義康竟陵王誕等並從【從才用翻}
南兖州刺史江夏王義恭自江都會之【夏戶雅翻}
秋七月己丑魏永昌王健擊破秃髪保周于番禾保周走遣安南將軍尉眷追之【番音盤尉紆勿翻}
 丙申魏太后竇氏殂 壬子皇后袁氏殂【太子邵弑逆之心萌於此矣}
 癸丑秃髪保周窮廹自殺八月甲申沮渠無諱使其中尉梁偉詣魏永昌王健請降【降戶江翻}
歸酒泉郡及所虜將士元絜等魏主使尉眷留鎮凉州 九月壬子葬元皇后 上以司徒彭城王義康嫌隙已著將成禍亂冬十月戊申收劉湛付廷尉下詔暴其罪惡就獄誅之并誅其子黯亮儼及其黨劉斌劉敬文孔胤秀等八人徙尚書庫部郎何默子等五人於廣州【曹魏置尚書二十三郎庫部其一也}
因大赦是日敕義康入宿留止中書省其夕分收湛等青州刺史杜驥勒兵殿内以備非常遣人宣旨告義康以湛等罪狀義康上表遜位詔以義康為江州刺史侍中大將軍如故出鎮豫章初殷景仁卧疾五年【景仁卧疾始上卷十二年}
雖不見上而密函去來日以十數朝政大小必以咨之【朝直遥翻}
影迹周密莫有窺其際者收湛之日景仁使拂拭衣冠左右皆不曉其意其夜上出華林園延賢堂召景仁景仁猶稱脚疾以小牀輿就坐誅討處分一以委之【坐徂卧翻處昌呂翻分扶問翻}
初檀道濟薦吳興沈慶之忠謹曉兵上使領隊防東掖門劉湛為領軍嘗謂之曰卿在省歲久比當相論【省謂領軍省比毗寐翻謂當為之論叙也}
慶之正色曰下官在省十年自應得轉不復以此仰累【復扶又翻下同累力瑞翻}
收湛之夕上開門召慶之慶之戎服縛袴而入上曰卿何意乃爾急裝慶之曰夜半喚隊主【江南軍制呼長帥為隊主軍主隊主者主一隊之稱軍主者主一軍之稱}
不容緩服【史言沈慶之有識畧}
上遣慶之收劉斌殺之驍騎將軍徐湛之逵之之子也【徐逵之武帝愛壻死于司馬楚之魯宗之之難驍堅堯翻騎奇寄翻}
與義康尤親厚上深銜之義康敗湛之被收罪當死其母會稽公主於兄弟為長嫡【被皮義翻會工外翻長知兩翻}
素為上所禮家事大小必咨而後行高祖微時嘗自于新洲伐荻有納布衫襖【納與衲同}
臧皇后手所作也既貴以付公主曰後世有驕奢不節可以此衣示之至是公主入宫見上號哭不復施臣妾之禮以錦囊盛納衣擲地【號戶高翻盛時征翻}
曰汝家本貧賤此是我母為汝父所作【為于偽翻下右為嘗為同}
今日得一飽餐遽欲殺我兒邪上乃赦之吏部尚書王球履之叔父也以簡淡有美名為上所重履性進利【言履務進而好利也}
深結義康及湛球屢戒之不從誅湛之夕履徒跣告球【跣先典翻}
球命左右為取履先温酒與之謂曰常日語汝云何【語牛倨翻}
履怖懼不得答【怖普布翻}
球徐曰阿父在汝亦何憂【江南人士呼叔父伯父為阿父為伯父叔父者以自呼阿烏葛翻}
上以球故履得免死廢於家【據南史帝初為宜都王以球為友簡淡見重蓋素知之也}
義康方用事人爭求親暱【暱尼質翻}
唯司徒主簿江湛早能自疎求出為武陵内史檀道濟嘗為其子求婚于湛湛固辭道濟因義康以請之湛拒之愈堅故不染于二公之難【難巧旦翻}
上聞而嘉之湛夷之子也【江夷嚮用於元嘉之初}
彭城王義康停省十餘日見上奉辭便下渚上惟對之慟哭餘無所言上遣沙門慧琳視之義康曰弟子有還理不【不讀曰否}
慧琳曰恨公不讀數百卷書初吳興太守謝述裕之弟也【謝裕見一百十五卷晉安帝義熙五年}
累佐義康數有規益早卒【義康鎮江陵述為驃騎長史南郡太守義康入相又為司徒左長史數所角翻}
義康將南【將自建康南徙豫章}
歎曰昔謝述惟勸吾退劉班惟勸吾進今班存而述死其敗也宜哉上亦曰謝述若存義康必不至此以征虜司馬蕭斌為義康諮議參軍領豫章太守事無大小皆以委之斌摹之之子也【蕭摹之見上卷十二年}
使龍驤將軍蕭承之將兵防守【驤思將翻之將即亮翻}
義康左右愛念者並聽随從【從才用翻}
資奉優厚信賜相係朝廷大事皆報示之久之上就會稽公主宴集甚懽主起再拜叩頭悲不自勝【勝音升}
上不曉其意自起扶之主曰車子歲暮必不為陛下所容今特請其命【義康小字車子}
因慟哭上亦流涕指蔣山曰必無此慮若違今誓便是負初寧陵【高祖葬初寧陵在蔣山}
即封所飲酒賜義康并書曰會稽姊飲宴憶弟所餘酒今封送故終主之身義康得無恙【恙余亮翻}


  臣光曰文帝之於義康友愛之情其始非不隆也終于失兄弟之歡虧君臣之義迹其亂階正由劉湛權利之心無有厭已【厭於鹽翻}
詩云貪人敗類【芮良夫桑柔之詩敗補邁翻}
其是之謂乎

  徵南兖州刺史江夏王義恭為司徒録尚書事戊寅以臨川王義慶為南兖州刺史殷景仁為揚州刺史僕射吏部尚書如故義恭懲彭城之敗雖為總録奉行文書而已上乃安之【彭城義康也}
上年給相府錢二千萬他物稱此【相息亮翻稱尺證翻}
而義恭性奢用常不足上又别給錢年至千萬 十一月丁亥魏主如山北 殷景仁既拜揚州羸疾遂篤【羸倫為翻}
上為之敕西州道上不得有車聲【為于偽翻揚州治所在建康臺城西故謂之西州宋白曰秣陵縣秦屬鄣郡丹楊圖云自句容以西屬鄣郡以東屬會稽郡武帝元封二年改鄣郡為丹楊郡置揚州刺史理秣陵西州橋冶城之間是其理處劉縣為揚州刺史始移理曲阿孫策號此為西州}
癸丑卒【卒子恤翻}
十二月癸亥以光禄大夫王球為僕射戊辰以始興王濬為揚州刺史時濬尚幼州事悉委後軍長史范曄主簿沈璞曄泰之子璞林子之子也【范泰為高祖所賞愛林子從高祖為將有功}
曄尋遷左衛將軍以吏部郎沈演之為右衛將軍對掌禁旅又以庾炳之為吏部郎俱參機密演之勁之曾孫也【沈勁守死于洛陽以雪父充為逆之罪}
曄有雋才而薄情淺行數犯名教為士流所鄙【行下孟翻數所角翻}
性躁競自謂才用不盡常怏怏不得志【為後范曄謀亂張本}
吏部尚書何尚之言于帝曰范曄志趨異常【趨與趣同七著翻}
請出為廣州刺史若在内舋成不得不加鈇鉞鈇鉞亟行【舋許覲翻亟區記翻}
非國家之美也帝曰始誅劉湛復遷范曄【復扶又翻下復稱同}
人將謂卿等不能容才朕信受讒言但共知其如此無能為害也 是歲魏寧南將軍王慧龍卒呂玄伯留守其墓終身不去【慧龍不殺玄伯見上卷八年}
 魏主欲以伊馛為尚書封郡公馛辭曰尚書務殷公爵至重非臣年少愚近所宜膺受【馛蒲撥翻少詩照翻}
帝問其所欲對曰中袐二省多諸文士【中袐謂中書省袐書省也}
若恩矜不已請參其次帝善之以為中護軍將軍袐書監 大秦王楊難當復稱武都王【十三年難當自稱大秦王}


  十八年春正月癸卯魏以沮渠無諱為征西大將軍凉州牧酒泉王 彭城王義康至豫章辭刺史【辭江州刺史也}
甲辰以義康都督江交廣三州諸軍事前龍驤參軍巴東扶令育詣闕上表【驤思將翻扶姓也}
稱昔袁盎諫漢文帝曰淮南王若道路遇霜露死陛下有殺弟之名文帝不用追悔無及【見十四卷文帝六年}
彭城王義康先朝之愛子【朝直遥翻}
陛下之次弟若有迷謬之愆正可數之以善惡【數所具翻}
導之以義方奈何信疑似之嫌一旦黜削遠送南垂草萊黔首皆為陛下痛之【為于偽翻下竊為同}
廬陵往事足為龜鑑【見百二十卷元年}
恐義康年窮命盡奄忽于南臣雖微賤竊為陛下羞之陛下徒知惡枝之宜伐豈知伐枝之傷樹伏願亟召義康返于京甸兄弟協和君臣輯睦則四海之望塞多言之路絶矣【塞悉則翻}
何必司徒公揚州牧然後可以置彭城王哉若臣所言於國為非請伏重誅以謝陛下表奏即收付建康獄賜死

  裴子野論曰夫在上為善若雲行雨施【施式智翻}
萬物受其賜及其惡也若天裂地震萬物所驚駭其誰弗知其誰弗見豈戮一人之身鉗一夫之口所能攘逃所能弭滅哉是皆不勝其忿怒而有增于疾疹也【勝音升疹丑刃翻}
以太祖之含弘尚掩耳于彭城之戮自斯以後誰易由言【鄭玄曰由用也易以䜴翻}
有宋累葉罕聞直諒豈骨鯁之氣俗愧前古抑時王刑政使之然乎張約隕于權臣【事見百二十卷元年}
扶育斃于哲后宋之鼎鑊吁可畏哉

  魏新興王俊荒淫不法三月庚戌降爵為公俊母先得罪死俊積怨望有逆謀事覺賜死 辛亥魏賜郁久閭乞列歸爵為朔方王沮渠萬年為張掖王【十六年魏擒乞列歸沮渠萬年亦以是年以姑臧降魏}
 夏四月沮渠唐兒叛沮渠無諱無諱留從弟天周守酒泉【從才用翻}
與弟宜得引兵擊唐兒唐兒敗死魏以無諱終為邉患庚辰遣鎮南將軍奚眷擊酒泉 秋八月辛亥魏遣散騎侍郎張偉來聘 九月戊戌魏永昌王健卒 冬十一月戊子王球卒己亥以丹楊尹孟顗為尚書僕射【顗魚豈翻}
 酒泉城中食盡萬餘口皆餓死沮渠天周殺妻以食戰士【食祥吏翻}
庚子魏奚眷拔酒泉獲天周送平城殺之沮渠無諱乏食且畏魏兵之盛乃謀西度流沙遣其弟安周西擊鄯善鄯善王欲降【鄯上扇翻降戶江翻}
會魏使者至勸令拒守安周不能克退保東城【鄯善國之東城也}
 氐王楊難當傾國入寇謀據蜀土遣其建忠將軍苻冲出東洛以禦梁州兵【五代志義城郡景谷縣舊白水縣也後周省東洛郡入焉子按白水縣漢屬廣漢晉屬梓潼時屬晉壽則東洛在晉壽界也}
梁秦二州刺史劉真道擊冲斬之真道懷敬之子也【劉懷敬見一百一十一卷晉安帝隆安三年}
難當攻拔葭萌獲晉壽太守申坦遂圍涪城【涪音浮}
巴西梓潼二郡太守劉道錫嬰城固守難當攻之十餘日不克乃還道錫道產之弟也十二月癸亥詔龍驤將軍裴方明等帥甲士三千人又發荆雍二州兵以討難當皆受劉真道節度【裴方明益州之良將也程道養趙廣之亂屢有戰功故用之為明年平仇池張本驤思將翻帥讀曰率雍於用翻}
 晉寧太守爨松子反寧州刺史徐循討平之【晉惠帝永安二年分建寧西七縣為益州郡至懷帝更名晉寧郡}
天門蠻田向求等反破漊中【沈約曰漊中縣二漢無晉太康地志有疑是吳}


  【立屬天門郡漊郎侯翻}
荆州刺史衡陽王義季遣行參軍曹孫念討破之 魏寇謙之言於魏主曰今陛下以真君御世建静輪天宫之法開古以來未之有也應登受符書以彰聖德帝從之

  資治通鑑卷一百二十三


    


 


 



 

 
  







 


  
  
 
 
 


  

 















	
	









































 
  



















 





 












  
  
  

 





