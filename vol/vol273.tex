










 


 
 


 

  
  
  
  
  





  
  
  
  
  
 
  

  

  
  
  



  

 
 

  
   




  

  
  


    資治通鑑卷二百七十三 宋 司馬光 撰

  胡三省 音註

  後唐紀二【起閼逢涒灘盡旃蒙作噩十月凡一年有奇】

  莊宗光聖神閔孝皇帝中

  同光二年春正月甲辰幽州奏契丹入寇至瓦橋【李存審奏也】以天平軍節度使李嗣源為北面行營都招討使陜州留後霍彦威副之宣徽使李紹宏為監軍將兵救幽州【陜失冉翻監古銜翻將即亮翻】 孔謙復言於郭崇韜曰首座相公萬機事繁居第且遠【復扶又翻豆盧革時為首相故稱之為首座相公】租庸簿書多留滯宜更圖之【請改用人為租庸使孔謙意欲自得之也更工衡翻】豆盧革嘗以手書便省庫錢數十萬【今俗謂借錢為便錢言借貸以便用也時租庸錢皆入省庫】謙以手書示崇韜崇韜微以諷革革懼奏請崇韜專判租庸崇韜固辭上曰然則誰可者崇韜曰孔謙雖久典金穀【自帝得魏博孔謙即為支度務使】若遽委大任恐不叶物望請復用張憲帝即命召之謙彌失望【謙自去年四月帝即位之初即望為租庸使事見上卷】 岐王聞帝入洛内不自安【聞帝自大梁入洛懼移兵西伐也】遣其子行軍司馬彰義節度使兼侍中繼曮入貢【李繼曮以鳳翔行軍司馬領涇州節】始上表稱臣帝以其前朝耆舊與太祖比肩【前朝謂唐僖昭之朝帝即位追尊考晉王克用曰武皇帝廟號太祖上時掌翻朝直遥翻下同】特加優禮每賜詔但稱岐王而不名庚戌加繼曮中書令遣還【曮魚險翻還從宣翻又如字】 敕内官不應居外應前朝内官及諸道監軍并私家先所畜者不以貴賤並遣詣闕【唐末誅宦官其有逃逸者散投外鎮及為私家所養畜旴玉翻】時在上左右者已五百人至是殆及千人皆給贍優厚委之事任以為腹心内諸司使自天祐以來以士人代之【唐昭宗天復三年誅宦官以士人為内諸司使時所存者九使而已至梁有客省使改小馬坊使為天驥使飛龍使莊宅使儀鸞使文思使五坊使如京使尚食使改御食使為司膳使洛苑使教坊使東上閤門使西上閤門使内園栽接使弓箭庫使大内皇墻使武備庫使引進使左藏庫使閑廏使宫苑使翰林使大和庫使豐德庫使乾文院使後唐雖不用梁制而復唐之舊内諸司使其官亦多】至是復用宦者寖干政事既而復置諸道監軍節度使出征或留闕下軍府之政皆監軍决之陵忽主帥怙勢爭權由是藩鎮皆憤怒【為後諸藩鎮乘變殺監軍張本】 契丹出塞召李嗣源旋師命泰寧節度使李紹欽澤州刺史董璋戍瓦橋 李繼曮見唐甲兵之盛歸語岐王【語牛倨翻】岐王益懼癸丑表請正藩臣之禮優詔不許 孔謙惡張憲之來【時自魏召張憲復為租庸使憲方正故謙惡其來惡烏路翻】言於豆盧革曰錢穀細事一健吏可辦耳魏都根本之地顧不重乎興唐尹王正言操守有餘智力不足必不得已使之居朝廷衆人輔之猶愈於專委方面也革為之言於崇韜【為于偽翻】崇韜乃奏留張憲於東京甲寅以正言為租庸使正言昏懦謙利其易制故也【易以豉翻】 李存審奏契丹去復得新州【新州陷見二百六十九卷梁均王貞明三年】 戊午敕鹽鐵度支戶部三司並隸租庸使【租庸使之權愈重矣】 上遣皇弟存渥皇子繼岌迎太后太妃於晉陽太妃曰陵廟在此若相與俱行歲時何人奉祀遂留不來【帝即位尊曾祖執宜廟號懿祖陵曰永興國昌廟號獻祖陵曰長寧克用廟號太祖陵曰建極三陵皆在代州雁門縣親廟在晉陽太妃之不來夫豈專陵廟之為其心固有所見也且其辭義甚正為太后太妃俱以憂邑成疾張本】太后至庚申上出迎於河陽辛酉從太后入洛陽 二月己巳朔上祀南郊大赦孔謙欲聚斂以求媚【斂力贍翻】凡赦文所蠲者謙復徵之【蠲圭淵翻除也復扶又翻】自是每有詔令人皆不信百姓愁怨郭崇韜初至汴洛頗受藩鎮饋遺【遺唯季翻】所親或諫之崇韜曰吾位兼將相【郭崇韜為樞密使加侍中領成德節樞密使天下事無所不關侍中三省長官又領節鎮故言位兼將相】禄賜巨萬豈藉外財但以偽梁之季賄賂成風今河南藩鎮皆梁之舊臣主上之仇讎也若拒其意能無懼乎吾特為國家藏之私室耳【為于偽翻郭崇韜受饋遺未足以安藩鎮疑懼之心乃所以成其主好貨之惡】及將祀南郊崇韜首獻勞軍錢十萬緡先是宦官勸帝分天下財賦為内外府【勞力到翻先悉薦翻】州縣上供者入外府充經費【供居用翻】方鎮貢獻者入内府充宴遊及給賜左右於是外府常虛竭無餘而内府山積及有司辦郊祀乏勞軍錢崇韜言於上曰臣已傾家所有以助大禮願陛下亦出内府之財以助有司上默然久之曰吾晉陽自有儲積【積子賜翻又如字】可令租庸輦取以相助於是取李繼韜私第金帛數十萬以益之【李繼韜父嗣昭從晉王克用起于晉陽故私第在焉繼韜以反誅其家貲没官】軍士皆不滿望始怨恨有離心矣【為後諸軍離叛張本】 河中節度使李繼麟請榷安邑解縣鹽每季輸省課【每三月一輸鹽課於省也榷古岳翻解戶買翻】己卯以繼麟充制置兩池榷鹽使 辛巳進岐王爵為秦王 【考異日茂貞改封秦王薛史無的確年月實録同光元年十一月壬寅巳稱秦王茂貞遣使賀收復自後皆稱秦王至二年辛巳制秦王李茂貞可封秦王豈有秦王封秦王之理必是至是時始自岐王封秦王也通鑑考異正本在二年正月岐王上表稱臣之下今移置於此】仍不名不拜 郭崇韜知李紹宏怏怏乃置内句使掌句三司財賦以紹宏為之冀弭其意而紹宏終不悦【李紹宏恨郭崇韜見上卷元年句音鉤】徒使州縣增移報之煩【按薛史云同光元年十一月以李紹宏兼内句凡天下錢穀簿書悉委裁遣自是州縣供帳煩費議者非之與此有歲月之差】崇韜位兼將相復領節旄以天下為己任權侔人主旦夕車馬填門性剛急遇事輒發嬖倖僥求多所摧抑【嬖卑義翻又必計翻僥堅堯翻】宦官疾之朝夕短之於上崇韜扼腕欲制之不能【腕烏貫翻】豆盧革韋說嘗問之曰汾陽王本太原人徙華隂【說讀曰悦華戶化翻】公世家雁門豈其枝派邪崇韜因曰遭亂亡失譜諜嘗聞先人言上距汾陽四世耳【譜博古翻藉録也諜徒協翻漢郊祀歌披圖按諜蘇林注曰諜譜第也汾陽王謂郭子儀也】革曰然則固從祖也【從才用翻】崇韜由是以膏梁自處多甄别流品【處昌呂翻别彼列翻】引拔浮華鄙弃勲舊有求官者崇韜曰深知公功能然門地寒素不敢相用恐為名流所嗤【嗤丑之翻笑也】由是嬖倖疾之於内勲舊怨之於外崇韜屢請以樞密使讓李紹宏上不許又請分樞密院事歸内諸司以輕其權而宦官謗之不已崇韜鬱鬱不得志與所親謀赴本鎮以避之其人曰不可蛟龍失水螻蟻足以制之先是上欲以劉夫人為皇后【先悉薦翻】而有正妃韓夫人在【歐史曰莊宗正室曰衛國夫人韓氏其次曰燕國夫人伊氏次魏國夫人劉氏耳】太后素惡劉夫人【按歐史劉氏為袁建豐所得内之太后宫教以吹笙歌舞莊宗悦之太后以賜莊宗然而惡之者以其所出微而妒悍也】崇韜亦屢諫上以是不果於是所親說崇韜曰【說式芮翻】公若請立劉夫人為皇后上必喜内有皇后之助則伶宦輩不能為患矣崇韜從之與宰相帥百官共奏劉夫人宜正位中宫癸未立魏國夫人劉氏為皇后【郭崇韜以是求自全乃所以自禍也為殺郭崇韜張本帥讀曰率下同】皇后生於寒微既貴專務蓄財其在魏州薪蘇果茹皆販鬻之【採木為薪採草為蘇果覈也茹菜也】及為后四方貢獻皆分為二一上天子一上中宫【上時掌翻】以是寶貨山積惟用寫佛經施尼師而已【施式豉翻】是時皇太后誥皇后教與制敕交行於藩鎮奉之如一【婦言與王言並行自古亂政未有如同光之甚者也】 詔蔡州刺史朱勍浚索水通漕運【水經注車關水出於嵩渚之山發于層阜之上一源兩枝分流瀉注世謂之石泉水東流為索水西注為車關水索水在成臯北勍渠京翻索山客翻】 三月己亥朔蜀主宴近臣於怡神亭酒酣君臣及宫人皆脱冠露髻喧譁自恣知制誥京兆李龜禎諫曰君臣沈湎不憂國政【沈持林翻】臣恐啟北敵之謀【北敵謂唐也】不聽 乙巳鎮州言契丹將犯塞【此據諜報而上言也】詔横海節度使李紹斌北京左廂馬軍指揮使李從珂帥騎兵分道備之天平節度使李嗣源屯邢州紹斌本姓趙名行實幽州人也【斌悲巾翻】 丙午加高季興兼尚書令進封南平王 李存審自以身為諸將之首【李存審時為蕃漢馬步軍都總管】不得預克汴之功感憤疾益甚【李存審自滄徙幽時已寢疾】屢表求入覲郭崇韜抑而不許存審疾亟表乞生覩龍顔乃許之初帝嘗與右武衛上將軍李存賢手搏存賢不盡其技【存賢本許州王賢少為軍卒善角觝晉王克用得之賜以姓名養為子技渠綺翻】帝曰汝能勝我當授藩鎮存賢乃奉詔僅仆帝而止及許存審入覲帝以存賢為盧龍行軍司馬旬日除節度使曰手搏之約吾不食言矣【以手搏而得大藩是節鎮可以戲取矣】 庚戌幽州奏契丹寇新城【新城縣屬涿州唐太和六年以故督亢地置匈奴須知新城縣北至涿州六十里】 勲臣畏伶官之讒皆不自安蕃漢内外馬步副總管李嗣源求解兵柄帝不許 自唐末喪亂【喪息浪翻】搢紳之家或以告赤鬻於族姻【赤當作勅鬻於族姻則既非矣安知後世有鬻於非其族姻者乎】遂亂昭穆【昭上招翻】至有舅叔拜甥姪者【言舅拜其甥叔拜其姪也】選人偽濫者衆郭崇韜欲革其弊請令銓司精加考覈【銓司吏部也選須絹翻覈下革翻】時南郊行事官千二百人【凡郊祀預執事者皆謂之行事官】注官者纔數十人塗毁告身者十之九選人或號哭道路【號戶刀翻】或餒死逆旅 唐室諸陵先為温韜所發【帝不能正温韜之罪見上卷上年】庚申以工部郎中李途為長安按視諸陵使 皇子繼岌代張全義判六軍諸衛事 夏四月己巳朔羣臣上尊號曰昭文睿武至德光孝皇帝【唐諸帝尊號皆有孝字蓋因漢制今此又因唐制也】 帝遣客省使李嚴使於蜀嚴盛稱帝威德有混一天下之志且言朱氏簒竊諸侯曾無勤王之舉王宗儔以其語侵蜀請斬之蜀主不從宣徽北院使宋光葆上言晉王有憑陵我國家之志宜選將練兵屯戍邊鄙積糗糧治戰艦以待之【上時掌翻糗去久翻治直之翻艦戶黯翻言治戰艦欲以防峽江】蜀主乃以光葆為梓州觀察使充武德節度留後【蜀置武德軍於梓州】 乙亥加楚王殷兼尚書令 庚辰賜前保義留後霍彦威姓名李紹真【唐既滅梁改陜州鎮國軍為保義軍】 秦忠敬王李茂貞卒遺奏以其子繼曮權知鳳翔軍府事初安義牙將楊立有寵於李繼韜【李繼韜之求世襲也改昭義軍為安】

  【義軍】繼韜誅【見上卷上年】常邑邑思亂會發安義兵三千戍涿州立謂其衆曰前此潞兵未嘗戍邊【晉與梁兵爭潞兵未嘗北戍蓋以備梁耳】今朝廷驅我輩投之絶塞蓋不欲置之潞州耳與其暴骨沙場不若據城自守【涿州在幽州之南未為絶塞也唐人謂沙漠之地為沙場豈涿州之地乎楊立以此言激怒潞兵耳】事成富貴不成為羣盜耳因聚譟攻子城東門焚掠市肆節度副使李繼珂監軍張弘祚弃城走立自稱留後遣將士表求旌節詔以天平節度使李嗣源為招討使武寧節度使李紹榮為部署【部署之官始見於通鑑本在招討使之下其後有都部署遂為專任主帥之任】帳前都指揮使張廷藴為馬步都指揮使以討之 孔謙貸民錢使以賤估償絲【估音古價也以錢貸民而以賤價徵絲償所貸錢】屢檄州縣督之翰林學士承旨權知汴州盧質上言梁趙巖為租庸使舉貸誅斂結怨於人【斂力贍翻】陛下革故鼎新為人除害【易雜卦曰革去故也鼎取新也為于偽翻】而有司未改其所為是趙巖復生也【復扶又翻】今春霜害稼繭絲甚薄但輸正税猶懼流移况益以稱貸【稱舉也貸借也】人何以堪臣惟事天子不事租庸敕旨未頒省牒頻下【省牒謂租庸使所下文書下戶嫁翻】願早降明命帝不報 漢主引兵侵閩屯於汀漳境上【閩之汀漳二州皆與漢之潮州接境】閩人擊之漢主敗走 初胡柳之役【見二百七十卷梁均王貞明四年】伶人周匝為梁所得帝每思之【帝思周匝而不思周德威此其所以亡也】入汴之日匝謁見於馬前【入汴見上卷上年見賢遍翻】帝甚喜匝涕泣言曰臣之所以得生全者皆梁教坊使陳俊内園栽接使儲德源之力也【梁内園栽接使猶唐之内園使也宋白曰栽接使貞元中已有之職官分紀五代有内園栽接使國朝止名内園使】願就陛下乞二州以報之帝許之郭崇韜諫曰陛下所與共取天下者皆英豪忠勇之士今大功始就封賞未及一人而先以伶人為刺史恐失天下心以是不行踰年伶人屢以為言帝謂崇韜曰吾已許周匝矣使吾慙見此三人【三人謂周匝陳俊儲德源也周匝李存賢之事帝自以為踐言矣可以為政乎】公言雖正當為我屈意行之【為于偽翻】五月壬寅以俊為景州刺史德源為憲州刺史【憲州本樓煩監牧唐昭宗龍紀元年晉王克用表置憲州】時親軍有從帝百戰未得刺史者莫不憤歎【宜其離叛也】 乙巳右諫議大夫薛昭文上疏以為諸道僭竊者尚多【當是時諸道奉貢者有所不論如蜀如吳如漢皆唐之諸道也】征伐之謀未可遽息又士卒久從征伐賞給未豐貧乏者多【此正時病也】宜以四方貢獻及南郊羨餘【羨戈戰翻】更加頒賚又河南諸軍皆梁之精鋭恐僭竊之國潛以厚利誘之宜加收撫又戶口流亡者宜寛徭薄賦以安集之又土木不急之役宜加裁省又請擇隙地牧馬勿使踐京畿民田皆不從戊申蜀主遣李嚴還【李嚴四月入蜀至是而還還從宣翻又如字 考異曰實録七月戊午蜀遣歐陽彬朝貢十月癸巳遣客省使李嚴充蜀川回信使八月戊辰嚴自西川回蜀書四月己巳朔唐使李嚴來聘五月戊申遣嚴歸本國十一月己未朔遣彬為唐國通好使按錦里耆舊傳是歲遣歐陽彬通聘洛京莊宗遣李嚴來修好笏記云豈謂大蜀皇帝特遣蘇張之士來追唐蜀之歡吾皇迴感於蜀皇復禮遠酬於厚禮然則嚴為回信使也或者歐陽彬之前蜀已有入洛之使乎若如實錄年月則李嚴以二年十月奉使至三年八月方歸何留之久乎十國紀年蜀史又云九月己亥唐帝遣李彦稠來使十一月辛丑遣彦稠東還又八月以後遣王宗鍔等戍利州以備東師似用宋光葆之言十一月以後以唐國通好召諸軍還似因彦稠來而罷之今並從蜀書年月】初帝因嚴入蜀令以馬市宫中珍玩而蜀法禁錦綺珍奇不得入中國其粗惡者乃聽入中國謂之入草物【粗讀曰麤自盛唐以來蜀貢賦歲至京師此法乃王衍之法也】嚴還以聞帝怒曰王衍寧免為入草之人乎嚴因言於帝曰衍童騃荒縱不親政務斥遠故老昵比小人【騃語駭翻遠丁願翻昵尼質翻比毘至翻】其用事之臣王宗弼宋光嗣等諂諛專恣黷貨無厭賢愚易位刑賞紊亂【厭於鹽翻紊音問】君臣上下專以奢淫相尚以臣觀之大兵一臨瓦解土崩可翹足而待也帝深以為然【為伐蜀張本】 帝以潞州叛故庚戌詔天下州鎮無得修城濬隍悉毁防城之具【毁防城之具慮天下將卒有憑城而拒命者耳然趙在禮攻魏而魏不能守趙在禮據魏而攻不能拔而帝由是亦死於亂兵防患之道固不在此也】 壬子新宣武節度使兼中書令蕃漢馬步總管李存審卒於幽州【李存審受宣武之命而未離幽州也】存審出於寒微常戒諸子曰爾父少提一劍去鄉里【少詩照翻存審陳州宛丘人從李罕之歸晉王】四十年間位極將相【言以節度使同平章事也】其間出萬死獲一生者非一破骨出鏃者凡百餘因授以所出鏃命藏之曰爾曹生於膏粱當知爾父起家如此也 幽州言契丹將入寇甲寅以横海節度使李紹斌充東北面行營招討使將大軍度河而北契丹屯幽州東南城門之外虜騎充斥饋運多為所掠 壬戌以李繼曮為鳳翔節度使【嗣李茂貞帥岐】乙丑以權知歸義留後曹義金為節度使時瓜沙

  與吐蕃雜居義金遣使間道入貢故命之【唐懿宗咸通八年張義潮入朝以族子惟深守歸義十三年惟深卒以義金權知留後自咸通十三年至是五十四年蓋曹義金亦已老矣間古莧翻】 李嗣源大軍前鋒至潞州日已暝【暝莫定翻夕也】泊軍方定張廷藴帥麾下壯士百餘輩踰塹坎城而上【帥讀曰率上時掌翻】守者不能禦即斬關延諸軍入比明【比必利翻及也下比起同】嗣源及李紹榮至城已下矣嗣源等不悦【以張廷藴不待其至而先取城也】丙寅嗣源奏潞州平六月丙子磔楊立及其黨於鎮國橋【磔陟格翻】潞州城池高深帝命夷之【夷平也】 丙戌以武寧節度使李紹榮為歸德節度使同平章事【梁都汴移宣武軍於宋州唐滅梁復以汴州為宣武軍以宋州為歸德軍】留宿衛寵遇甚厚帝或時與太后皇后同至其家帝有幸姬色美嘗生子矣劉后妒之會紹榮喪妻【喪息浪翻】一日侍禁中帝問紹榮汝復娶乎【復扶又翻】為汝求昏【為于偽翻下為之同】后因指幸姬曰大家憐紹榮何不以此賜之帝難言不可微許之后趣紹榮拜謝【趣讀曰促】比起顧幸姬已肩輿出宫矣帝為之託疾不食者累日【史言帝憚劉后之妒悍】 壬辰以天平節度使李嗣源為宣武節度使代李存審為蕃漢内外馬步總管【自副總管陞都總管】 秋七月壬寅蜀以禮部尚書許寂為中書侍郎同平章事 孔謙復短王正言於郭崇韜【復扶又翻】又厚賂伶官求租庸使終不獲意怏怏癸卯表求解職帝怒以為避事將置於法景進救之得免 梁所决河連年為曹濮患【梁决河見二百七十卷均王貞明四年濮博木翻】 甲辰命右監門上將軍婁繼英督汴滑兵塞之未幾復壞【塞悉則翻幾居豈翻】庚申置威塞軍於新州 契丹恃其強盛遣使就

  帝求幽州以處盧文進【處昌呂翻】時東北諸夷皆役屬契丹惟勃海未服契丹主謀入寇恐勃海犄其後【勃海時為海東盛國置五京十五府六十二州盡有高麗肅慎之地犄居蟻翻】乃先舉兵擊勃海之遼東遣其將托諾及盧文進據營平等州以擾燕地【燕於賢翻】 八月戊辰蜀主以右定遠軍使王宗鍔為招討馬步使帥二十一軍屯洋州【帥讀曰率】乙亥以長直馬軍使林思鍔為昭武節度使戍利州以備唐 租庸使王正言病風恍惚不能治事【恍許昉翻惚音忽治直之翻】景進屢以為言癸酉以副使衛尉卿孔謙為租庸使右威衛大將軍孔循為副使循即趙殷衡也梁亡復其姓名【歐史曰孔循不知其家世何人也少孤流落于汴州富人李讓闌得之養以為子梁太祖以李讓為養子循乃冒姓朱氏給事太祖帳中太祖諸兒乳母有愛之者養循為子乳母之夫姓趙又冒姓趙名殷衡梁亡事唐始改孔名循按唐天祐二年趙殷衡已權判宣徽院事見二百六十五卷】謙自是得行其志重斂急徵以充帝欲民不聊生癸未賜謙號豐財贍國功臣【記曰與其有聚斂之臣寧有盜臣而以是為功臣之號以寵孔謙唐之君臣不知其非也民困軍怨其能久乎為明宗誅謙張本】 帝復遣使者李彦稠入蜀九月己亥至成都【復扶又翻下復蹂同】 癸卯帝獵於近郊時帝屢出遊獵從騎傷民禾稼洛陽令何澤伏於叢薄【草聚生曰叢草木交錯曰薄】俟帝至遮馬諫曰陛下賦斂既急今稼穡將成復蹂踐之【蹂人九翻又如又翻踐慈演翻】使吏何以為理民何以為生臣願先賜死帝慰而遣之【諫獵一也中牟令幾不免於死洛陽令乃蒙勞遣者意必有伶官為之容也夷考何澤終身之行實非亮直之士】澤廣州人也【薛史何澤廣州人梁貞明中清海節度使劉陟薦其才以進士擢第】契丹攻勃海無功而還【還從宣翻又如字】 蜀前山南節度使兼中書令王宗儔以蜀主失德與王宗弼謀廢立宗弼猶豫未决庚戌宗儔憂憤而卒宗弼謂樞密使宋光嗣景潤澄等曰宗儔教我殺爾曹今日無患矣光嗣輩俯伏泣謝宗弼子承班聞之謂人曰吾家難乎免矣 乙卯蜀主以前鎮江軍節度使張武為峽路應援招討使【蜀置鎮江軍於夔州】 丁巳幽州言契丹入寇冬十月辛未天平節度使李存霸平盧節度使符

  習言屬州多稱直奉租庸使帖指揮公事使司殊不知有紊規程【使司謂節度使司也紊音問】租庸使奏近例皆直下【時租庸使帖下諸州調發不關節度觀察使謂之直下下戶嫁翻】敕朝廷故事制敕不下支郡【節鎮為會府巡屬諸州為支郡】牧守不專奏陳今兩道所奏乃本朝舊規租庸所陳是偽廷近事【時以梁為偽廷黜之也】自今支郡自非進奉皆須本道騰奏租庸徵催亦須牒觀察使【唐制節度使掌兵事觀察使掌民事故敕租庸徵催止牒觀察使司】雖有此敕竟不行【史言徵斂嚴急但期趣辦竟不奉敕而行】 易定言契丹入寇蜀宣徽北院使王承休請擇諸軍驍勇者萬二千

  人置駕下左右龍武步騎四十軍兵械給賜皆優異於它軍以承休為龍武軍馬步都指揮使以禆將安重霸副之舊將無不憤恥重霸雲州人以狡佞賄賂事承休故承休悦之【為安重霸背王承休而降唐張本】 吳越王鏐復修本朝職貢【錢鏐本唐臣唐亡事梁梁亡復事唐故云復修本朝職貢】壬午帝因梁官爵而命之鏐厚貢獻并賂權要求金印玉冊賜詔不名稱國王有司言故事惟天子用玉冊王公皆用竹冊【竹冊編竹為之以存古意】又非四夷無封國王者帝皆曲從鏐意 吳王如白沙觀樓船更命白沙曰迎鑾鎮【路振九國志曰楊溥巡白沙太學博士王穀上書請改白沙為迎鑾其畧曰日月所經星辰盡為黄道鑾輿所止井邑皆為赤縣】徐温自金陵來朝【白沙楊子縣地五季之末改楊子為永貞縣宋朝乾德二年以揚州永貞縣迎鑾鎮為建安軍大中祥符六年升為真州而永貞縣先是復改為楊子其地東至揚州六十里南臨大江度江而南至金陵亦六十里更工衡翻】先是温以親吏翟䖍為閤門宫城武備等使使察王起居【先悉薦翻】䖍防制王甚急【使鍾泰章殺張顥閉牙城門討朱瑾皆翟䖍也故徐温親任之翟直格翻】至是王對温名雨為水温請其故王曰翟䖍父名吾諱之熟矣因謂温曰公之忠誠我所知也然翟䖍無禮宫中及宗室所須多不獲【須者意所欲也求也】温頓首謝罪請斬之王曰斬則太過遠徙可也乃徙撫州 十一月蜀主遣其翰林學士歐陽彬來聘 【考異曰實録七月戊午蜀主遣戶部侍郎歐陽彬來使致書用敵國之禮蜀書後主紀十一月乙未命翰林學士兵部侍郎歐陽彬為唐國通好使今從之】彬衡山人也又遣李彦稠東還【李彦稠至蜀見上九月還從宣翻又如字】 癸卯帝帥親軍獵於伊闕【伊闕縣在洛陽南二百餘里有伊闕山大禹所鑿也宋朝省伊闕縣為鎮入伊陽縣帥讀曰率】命從官拜梁太祖墓【梁祖帝之仇讎前欲發墓斵棺今使從官拜之何前後之相違也從才用翻】涉歷山險連日不止或夜合圍士卒墜崖谷死及折傷者甚衆【史言帝荒於從禽而不恤士卒折而設翻】丙午還宫 蜀以唐修好罷威武城戍召關宏業等二十四軍還成都戊申又罷武定武興招討劉潛等三十七軍 丁巳賜護國節度使李繼麟鐵劵以其子令德令錫皆為節度使諸子勝衣者即拜官【勝音升】寵冠列藩【朱友謙之寵乃所以速禍也是其反覆多矣能無及乎冠工喚翻】 庚申蔚州言契丹入寇 辛酉蜀主罷天雄軍招討命王承騫等二十九軍還成都 十二月乙丑朔蜀主以右僕射張格兼中書侍郎同平章事初格之得罪【事見二百七十卷梁均王貞明四年】中書吏王魯柔乘危窘之【窘渠隕翻】及再為相用事杖殺之許寂謂人曰張公才高而識淺戮一魯柔它人誰敢自保此取禍之端也【張格則失矣許寂同在相位不知蜀有垂亡之勢但知張格有取禍之端蜀亡為相者得免禍乎】 蜀主罷金州屯戍命王承勲等七軍還成都【蜀主恃與唐和而徹邊備是馴狎虎豹而不嚴設圈檻也】己巳命宣武節度使李嗣源將宿衛兵三萬七千人赴汴州遂如幽州禦契丹【命李嗣源將兵赴鎮因而北出備邊】 庚午帝及皇后如張全義第全義大陳貢獻酒酣皇后奏稱妾幼失父母見老者輒思之請父事全義帝許之全義惶恐固辭再三彊之竟受皇后拜復貢獻謝恩【劉后利張全義之財此如倡婢屈膝於人志在求貨耳惡可以母天下乎彊其兩翻復扶又翻】明日后命翰林學士趙鳳草書謝全義鳳密奏自古無天下之母拜人臣為父者帝嘉其直然卒行之【卒子恤翻】自是后與全義日遣使往來問遺不絶【遺唯季翻】 初唐僖昭之世宦官雖盛未嘗有建節者蜀安重霸勸王承休求秦州節度使承休言於蜀主曰秦州多美婦人請為陛下采擇以獻【為于偽翻】蜀主許之庚午以承休為天雄節度使封魯國公【史言蜀政之亂有唐末之所無者】以龍武軍為承休牙兵【是年十月蜀方置龍武軍】 乙亥蜀主以前武德節度使兼中書令徐延瓊為京城内外馬步都指揮使【蜀以成都城為京城】延瓊以外戚代王宗弼居舊將之右衆皆不平【蜀主之母之妃皆徐氏也蜀主建遺命不以徐氏兄弟典兵雖王衍昏縱而蜀之臣亦無以建遺命為衍言者王宗弼亦何足任衆之所以不平徐延瓊者但以非次耳】 壬午北京言契丹寇嵐州【同光之初以鎮州為北都太原為西京尋廢北都復為鎮州以太原為北京嵐盧含翻】 辛卯蜀主改明年元曰咸康 盧龍節度使李存賢卒 是歲蜀主徙普王宗仁為衛王雅王宗輅為豳王褒王宗紀為趙王榮王宗智為韓王興王宗澤為宋王彭王宗鼎為魯王忠王宗平為薛王資王宗特為莒王宗輅宗智宗平皆罷軍役【蜀以諸王為軍使見二百七十卷梁均王貞明四年】

  三年春正月甲午朔蜀大赦 丙申敕有司改葬昭宗及少帝【以其遭朱温之弑葬故多闕也少詩照翻】竟以用度不足而止【後唐自以為承唐後終不能改葬昭宗少帝後漢自以為纂漢緒而長陵原陵終乾祐之世不沾一奠史書之以見譏】 契丹寇幽州 庚子帝發洛陽庚戌至興唐【時以魏州為興唐府】 詔平盧節度使符習治酸棗遥隄以禦決河【遥隄者遠於平地為之以捍水治直之翻】 初李嗣源北征【謂去年北禦契丹時也】過興唐東京庫有供御細鎧嗣源牒副留守張憲取五百領憲以軍興不暇奏而給之帝怒曰憲不奉詔擅以吾鎧給嗣源何意也罰憲俸一月令自往軍中取之【往嗣源軍中取細鎧】帝以義武節度使王都將入朝欲闢毬場憲曰比以行宫闕廷為毬場前年陛下即位於此其壇不可毁【比毘至翻同光元年帝築壇於魏州牙城之南告天即位】請闢毬場於宫西數日未成帝命毁即位壇憲謂郭崇韜曰此壇主上所以禮上帝始受命之地也若之何毁之崇韜從容言於帝【從千容翻】帝立命兩虞候毁之【兩虞候馬軍虞候及步軍虞候一曰左右兩虞候】憲私於崇韜曰忘天背本不祥莫大焉【背蒲妹翻張憲郭崇韜相與私議而不敢廷爭以帝之鷙悍而不可回也】 二月甲戌以横海節度使李紹斌為盧龍節度使【李紹斌至明宗時復姓趙賜名德均德均守幽州不為無功其後乘危以邀君外與契丹為市不但父子為虜幽州亦為虜有矣】丙子李嗣源奏敗契丹於涿州【敗補邁翻】 上以契丹為

  憂與郭崇韜謀以威名宿將零落殆盡李紹斌位望素輕欲徙李嗣源鎮真定為紹斌聲援崇韜深以為便時崇韜領真定上欲徙崇韜鎮汴州【欲使二人兩易節鎮】崇韜辭曰臣内典樞機外預大政富貴極矣何必更領藩方且羣臣或從陛下歲久身經百戰所得不過一州臣無汗馬之勞徒以侍從左右【侍從才用翻】時贊聖謨致位至此常不自安今因委任勲賢使臣得解旄節乃大願也且汴州關東衝要【汴州在成臯關之東南通淮泗北接滑魏衝要之地也】地富人繁臣既不至治所徒令它人攝職何異空城非所以固國基也上曰深知卿忠藎然卿為朕畫策襲取汶陽保固河津既而自此路直趨大梁成朕帝業【為于偽翻取汶陽謂取鄆州保固河津謂築壘馬家口與取大梁事並見上卷元年汶音問趨七喻翻】豈百戰之功可比乎今朕貴為天子豈可使卿曾無尺寸之地乎崇韜固辭不已上乃許之庚辰徙李嗣源為成德節度使 漢主聞帝滅梁而懼遣宫苑使何詞入貢且覘中國彊弱【覘丑亷翻又丑艶翻】甲申詞至魏【時帝在魏都】及還【還從宣翻又如字】言帝驕淫無政不足畏也漢主大悦自是不復通中國【復扶又翻無敵國外患者國恒亡漢主既知唐之不足畏奢虐亦由是滋矣】 帝性剛好勝【好呼到翻】不欲權在臣下入洛之後信伶宦之讒頗疎忌宿將李嗣源家在太原三月丁酉表衛州刺史李從珂為北京内牙馬步都指揮使以便其家帝怒曰嗣源握兵權居大鎮軍政在吾安得為其子奏請【得為于偽翻】乃黜從珂為突騎指揮使帥數百人戍石門鎮【石門鎮即唐之横水柵帥讀曰率】嗣源憂恐上章申理久之方解【上時掌翻申者重也重自理說】辛丑嗣源乞至東京朝覲不許郭崇韜以嗣源功高位重亦忌之私謂人曰總管令公非久為人下者【李嗣源為中書令蕃漢内外馬步軍都總管故以稱之】皇家子弟皆不及也密勸帝召之宿衛罷其兵權又勸帝除之帝皆不從【為李嗣源疑懼張本郭崇韜其亦自知為伶宦所忌乎】 己酉帝發興唐自德勝濟河歷楊村戚城觀昔時戰處指示羣臣以為樂【此即帝自言我於十指上得天下之故態也樂音洛】 洛陽宫殿宏邃宦者欲上增廣嬪御詐言宫中夜見鬼物上欲使符呪者攘之【符水厭祝巫覡挾術以欺世者為之攘却也】宦者曰臣昔逮事咸通乾符天子【逮及也咸通唐懿宗年號乾符僖宗年號】當是時六宫貴賤不減萬人今掖庭太半空虛故鬼物遊之耳上乃命宦者王允平伶人景進采擇民間女子遠至太原幽鎮以充後庭不啻三千人不問所從來上還自興唐【還從宣翻又如字】載以牛車纍纍盈路張憲奏諸營婦女亡逸者千餘人慮扈從諸軍挾匿以行其實皆入宫矣【諸營謂魏州諸營也史言帝之結怨于魏卒者非一事從才用翻】庚辰帝至洛陽辛酉詔復以洛陽為東都興唐府為鄴都【唐之盛時以洛陽為東都同光之初以晉陽為西京魏州為東京尋以洛陽為洛都今復唐舊以洛陽為東都則亦復以長安為西京矣晉陽之西京先已改為北都洛陽既復東京之舊又改魏州之東京為鄴都然相州乃古鄴地魏州治元城非鄴地也鄴戰國時為魏邑漢為鄴縣魏郡治焉漢末曹操為魏王居鄴前燕慕容暐都鄴置貴鄉縣屬昌樂郡水經注所謂沙丘堰有貴鄉者也隋開皇三年罷昌樂郡貴鄉縣屬魏州遂為州治所此時與興唐縣並置於鄴下興唐本元城莊宗以魏州為鄴都特以漢魏郡治鄴曹操以魏王都鄴而名之耳然相州自隋以來治安陽而鄴為屬縣魏州相州治所皆非古鄴也】 夏四月癸亥朔日有食之 初五臺僧誠惠以妖妄惑人自言能降伏天龍【降戶江翻】命風召雨帝尊信之親師后妃及皇弟皇子拜之【帥讀曰率】誠惠安坐不起羣臣莫敢不拜時大旱帝自鄴都迎誠惠至洛陽使祈雨士民朝夕瞻仰數旬不雨或謂誠惠【謂者告語之也】官以師祈雨無驗將焚之【官謂莊宗師謂誠惠】誠惠逃去慙懼而卒【史言異端率妖妄不足信】 庚寅中書侍郎同平章事趙光胤卒 太后自與太妃别【二年正月太后離晉陽】常忽忽不樂【樂音洛】雖娛玩盈前未嘗解顔太妃既别太后亦邑邑成疾太后遣中使醫藥相繼於道聞疾稍加輒不食又謂帝曰吾與太妃恩如兄弟欲自往省之【省悉景翻】帝以天暑道遠苦諫久之乃止但遣皇弟存渥等往迎侍五月丁酉北都奏太妃薨太后悲哀不食者累日帝寛譬不離左右太后自是得疾又欲自往會太妃葬帝力諫而止【離力智翻太后之悲慕以太妃有以得其心耳】 閩王審知寢疾命其子節度副使延翰權知軍府事 自春夏大旱六月壬申始雨 帝苦溽暑【溽儒欲翻溽暑濕熱也】於禁中擇高凉之所皆不稱旨【稱尺證翻】宦者因言臣見長安全盛時大明興慶宫樓觀以百數【唐都長安大明宫東内也興慶宫南内也觀工喚翻】今日宅家曾無避暑之所宫殿之盛曾不及當時公卿第舍耳帝乃命宫苑使王允平别建一樓以清暑宦者曰郭崇韜常不伸眉為孔謙論用度不足【為于偽翻】恐陛下雖欲營繕終不可得上曰吾自用内府錢無關經費【經費謂國之經常調度其費仰于租庸使者】然猶慮崇韜諫遣中使語之曰【語牛倨翻】今歲盛暑異常朕昔在河上與梁人相拒行營卑濕被甲乘馬【被皮義翻】親當矢石猶無此暑今居深宫之中而暑不可度奈何對曰陛下昔在河上勍敵未滅【勍渠京翻】深念讎恥雖有盛暑不介聖懷今外患已除海内賓服故雖珍臺閒館猶覺鬱蒸也陛下儻不忘艱難之時則暑氣自消矣【郭崇韜之言其指明居養之移人可謂婉切其如帝不聽何】帝默然宦者曰崇韜之第無異皇居宜其不知至尊之熱也帝卒命允平營樓【卒子恤翻】日役萬人所費巨萬崇韜諫曰今兩河水旱軍食不充願且息役以俟豐年帝不聽 帝將伐蜀辛卯詔天下括市戰馬 吳鎮海節度判官楚州團練使陳彦謙有疾【陳彦謙徐温所親信者也】徐知誥恐其遺言及繼嗣事遺之醫藥金帛相屬於道【遺唯季翻下同屬之欲翻】彦謙臨終密留書遺徐温請以所生子為嗣【以父子血氣所屬之親感動徐温】 太后疾甚秋七月甲午成德節度使李嗣源以邊事稍弭表求入朝省太后【省悉景翻】帝不許壬寅太后殂帝哀毁過甚五日方食 八月癸未杖殺河南令羅貫初貫為禮部員外郎性強直為郭崇韜所知用為河南令為政不避權豪伶宦請托書積几案一不報皆以示崇韜崇韜奏之由是伶宦切齒河南尹張全義亦以貫高伉惡之【伉苦浪翻惡烏路翻】遣婢訴於皇后【劉后以父事張全義故得遣婢出入宫掖】后與伶宦共毁之帝含怒未發會帝自往壽安視坤陵役者【九域志壽安縣在洛陽西南七十里五代會要曰上欲祔太后於代州太祖園陵中書門下奏議曰人君以四海為家不當分南北洛陽是帝王之宅四時朝拜理須便近不能遠幸代州漢朝諸陵皆近秦雍國家園寢布列京畿後魏文帝自代遷洛之後園陵皆在河南兼勅應勲臣之家不許北葬今魏氏諸陵尚在京畿祔葬代州理未為允於是作坤陵】道路泥濘【濘乃定翻淖也】橋多壞帝問主者為誰宦官對屬河南帝怒下貫獄獄吏榜掠【下戶嫁翻榜音彭掠音亮】體無完膚明日傳詔殺之崇韜諫曰貫坐橋道不修法不至死帝怒曰太后靈駕將發天子朝夕往來橋道不修卿言無罪是黨也崇韜曰陛下以萬乘之尊怒一縣令使天下謂陛下用法不平臣之罪也帝曰既公所愛任公裁之拂衣起入宫崇韜隨之論奏不已帝自闔殿門崇韜不得入貫竟死㬥尸府門遠近寃之【羅貫之死崇韜可以去而不能去自致夷滅哀哉】 丁亥遣吏部侍郎李德休等賜吳越國王玉冊金印紅袍御衣 九月蜀主與太后太妃遊青城山歷丈人觀上清宫【青城山在蜀州青城縣北三十三里杜光庭曰岷山連峯接岫千里不絶青城山乃第一峯也丈人觀在青城北二十里上清宫在高臺山丈人祠之側高臺山在岷山上有天池晉朝立天宫於上號上清宫】遂至彭州陽平化【彭州濛陽縣北四十里有葛仙山二十四化之第五化也】漢州三學山而還【還從宣翻又如字】乙未立皇子繼岌為魏王 丁酉帝與宰相議伐蜀威勝節度使李紹欽素諂事宣徽使李紹宏紹宏薦紹欽有蓋世奇才雖孫吳不如可以大任郭崇韜曰段凝亡國之將姦諂絶倫不可信也【改鄧州宣化軍為威勝軍段凝降賜姓名李紹欽事並見上卷元年】衆舉李嗣源崇韜曰契丹方熾總管不可離河朔【離力智翻】魏王地當儲副未立殊功請依故事以為伐蜀都統【安禄山之亂玄宗分命諸子為諸道都統此唐故事也】成其威名帝曰兒幼豈能獨往當求其副既而曰無以易卿庚子以魏王繼岌充西川四面行營都統崇韜充東北面行營都招討制置等使軍事悉以委之又以荆南節度使高季興充東南面行營都招討使鳳翔節度使李繼曮充都供軍轉運應接等使同州節度使李令德充行營副招討使陜州節度使李紹琛充蕃漢馬步軍都排陳斬斫使兼馬步軍都指揮使【李令德朱友謙之子也李紹琛康延孝也皆降唐賜姓名陳讀曰陣】西京留守張筠充西川管内安撫應接使華州節度使毛璋充左廂馬步都虞候邠州節度使董璋充右廂馬步都虞候客省使李嚴充西川管内招撫使將兵六萬伐蜀仍詔季興自取夔忠萬三州為巡屬【唐時夔忠萬三州本屬荆南節度唐末之亂王建據蜀併而有之】都統置中軍以供奉官李從襲充中軍馬步都指揮監押高品李廷安呂知柔充魏王府通謁【李從襲等皆宦官也】辛丑以工部尚書任圜翰林學士李愚並參預都統軍機 自六月甲午雨罕見日星江河百川皆溢凡七十五日乃霽 郭崇韜以北都留守孟知祥有薦引舊恩【事見二百七十卷梁均王貞明五年】將行言於上曰孟知祥信厚有謀若得西川而求帥無踰此人者【帥所類翻】又薦鄴都副留守張憲謹重有識可為相戊申大軍西行 蜀安重霸勸王承休請蜀主東遊秦州承休到官即毁府署作行宫大興力役強取民間女子教歌舞圖形遺韓昭【遺唯季翻韓昭諛佞蜀主狎而信之】使言於蜀主又獻花木圖盛稱秦州山川土風之美蜀主將如秦州羣臣諫者甚衆皆不聽王宗弼上表諫蜀主投其表於地太后涕泣不食止之亦不能得前秦州節度判官蒲禹卿上表幾二千言【上時掌翻幾居依翻】其畧曰先帝艱難創業欲傳之萬世陛下少長富貴【少詩照翻長知兩翻】荒色惑酒秦州人雜羌胡地多瘴癘萬衆困於奔馳郡縣罷於供億【瘴之亮翻罷讀曰疲】鳳翔久為仇讎必生釁隙唐國方通歡好恐懷疑貳【好呼到翻言無事舉兵東出恐因而致寇】先皇未嘗無故盤游陛下率意頻離宫闕【離力智翻】秦皇東狩鑾駕不還【見秦紀】煬帝南巡龍舟不返【見隋紀】蜀都強盛雄視隣邦邊庭無烽火之虞境内有腹心之疾百姓失業盜賊公行昔李勢屈於桓温【見九十七卷晉孝宗永和三年】劉禪降於鄧艾【見七十七卷魏元帝景元四年降戶江翻】山河險固不足憑恃韓昭謂禹卿曰吾收汝表俟主上西歸【自秦州歸成都曰西歸】當使獄吏字字問汝【蜀主歸未及以問蒲禹卿而韓昭身首已異處矣】王承休妻嚴氏美蜀主私焉故鋭意欲行 冬十月排陳斬斫使李紹琛與李嚴將驍騎三千步兵萬人為前鋒招討判官陳乂至寶雞稱疾乞留李愚厲聲曰陳乂見利則進懼難則止今大軍涉險【自寶雞入散關則涉棧閣之險】人心易揺【易以豉翻】宜斬以徇由是軍中無敢顧望者乂薊州人也【薊音計】癸亥蜀主引兵數萬發成都甲子至漢州武興節度使王承捷告唐兵西上【蜀置武興軍於鳳州唐自關東進兵攻蜀為西上上時掌翻】蜀主以為羣臣同謀沮已【沮在呂翻】猶不信大言曰吾方欲耀武遂東行在道與羣臣賦詩殊不為意 丁丑李紹琛攻蜀威武城蜀指揮使唐景思將兵出降城使周彦禋等知不能守亦降 【考異曰實録十月戊寅魏王繼岌至鳳州王承捷以鳳興文成四州降前一日康延孝李嚴至故鎮威武城唐景思等降按今故鎮在鳳州西四程延孝未下鳳州何能先至故鎮又蜀之守禦必在鳳州之東或者當時鳳州之東别有威武城亦名故鎮非今之故鎮歟】景思秦州人也得城中糧二十萬斛紹琛縱其敗兵萬餘人逸去因倍道趣鳳州【縱敗兵先去以懼蜀人而倍道踵其後以趣鳳州趣七喻翻】李嚴飛書以諭王承捷李繼曮竭鳳翔蓄積以饋軍不能充人情憂恐郭崇韜入散關指其山曰吾輩進無成功不得復還此矣當盡力一决【一决者一决戰也復扶又翻下同】今饋運將竭宜先取鳳州因其糧諸將皆言蜀地險固未可長驅宜按兵觀釁崇韜以問李愚愚曰蜀人苦其主荒淫莫為之用宜乘其人心崩離風驅霆擊彼皆破膽雖有險阻誰與守之兵勢不可緩也是日李紹琛告捷【是日崇韜入散關之日也蓋即丁丑】崇韜喜謂李愚曰公料敵如此吾復何憂乃倍道而進【復扶又翻】戊寅王承捷以鳳興文扶四州印節迎降【四州州印及武興節度使印及旌節也】得兵八千糧四十萬斛崇韜曰平蜀必矣【兵威已振有糧可因知功必成】即以都統牒命承捷攝武興節度使己卯蜀主至利州威武敗卒奔還始信唐兵之來王宗弼宋光嗣言於蜀主曰東川山南兵力尚完【東川謂梓遂諸州山南謂興元諸州】陛下但以大軍扼利州唐人安敢懸兵深入從之庚辰以隨駕清道指揮使王宗勲王宗儼兼侍中王宗昱為三招討將兵三萬逆戰從駕兵自綿漢至深渡【從才用翻深渡在利州綿谷縣北大漫天小漫天之間】千里相屬【屬之欲翻】皆怨憤曰龍武軍糧賜倍於它軍【龍武糧賜優厚事見上年】它軍安能禦敵李紹琛等過長舉【長舉漢沮縣地西魏置盤頭郡隋置長舉縣唐屬興州九域志在州西一百里】興州都指揮使程奉璉將所部兵五百來降且請先治橋棧以俟唐軍【璉力展翻治直之翻棧士限翻】由是軍行無險阻之虞辛巳興州刺史王承鑒弃城走紹琛等克興州【考異曰實録甲申魏王至故鎮康延孝收興州十國紀年辛巳承鑒出奔甲申繼岌郭崇韜至威武城今從之】郭崇韜以唐景思攝興州刺史乙酉成州刺史王承朴棄城走【九域志興州西至成州二百一十五里】李紹琛等與蜀三招討戰于三泉【三泉縣唐屬興元府九域志興州東南至三泉一百四十五里有百牢關金牛道之險】蜀兵大敗斬首五千級餘衆潰走又得糧十五萬斛於三泉由是軍食優足【優饒也】 戊子葬貞簡太后于坤陵蜀主聞王宗勲等敗自利州倍道西走斷桔柏津浮梁【桔古屑翻斷音短】命中書令判六軍諸衛事王宗弼將大軍守利州且令斬王宗勲等三招討【以三泉之敗也】李紹琛晝夜兼行趣利州【九域志三泉西至利州一百八十九里趣七喻翻】蜀武德留後宋光葆遺郭崇韜書【遺唯季翻】請唐兵不入境當舉巡屬内附苟不如約則背城决戰以報本朝【背蒲昧翻宋光葆謂蜀為本朝朝直遥翻】崇韜復書撫納之乙丑魏王繼岌至興州光葆以梓綿劍龍普五州武定節度使王承肇以洋蓬壁三州山南節度使王宗威以梁開通渠麟五州【渠州潾山縣唐武德元年置潾州八年州廢以潾山縣屬渠州當是蜀復置潾州也麟當作潾音力珍翻又唐貞觀中置麟州以處生羌歸附者屬松州都督府唐至德後淪没久矣當以渠潾之潾為是】階州刺史王承岳以階州皆降承肇宗侃之子也自餘城鎮皆望風欵附天雄節度使王承休與副使安重霸謀掩擊唐軍【欲自秦州掩撃唐軍之後】重霸曰擊之不勝則大事去矣蜀中精兵十萬天下險固唐兵雖勇安能直度劍門邪然公受國恩聞難不可不赴【難乃旦翻】願與公俱西【言自秦州西赴成都】承休素親信之以為然重霸請賂羌人買文扶州路以歸承休從之使重霸將龍武軍及所募兵萬二千人以從將行州人餞於城外承休上道【以從才用翻上時掌翻】重霸拜於馬前曰國家竭力以得秦隴【蜀得秦隴見二百六十九卷梁均王貞明元年】若從開府還朝【朝直遥翻】誰當守之開府行矣重霸請為公留守【蜀蓋加王承休開府儀同三司故稱之為于偽翻下為陳同守式又翻】承休業已上道無如之何遂與招討副使王宗汭自扶文而南其地皆不毛羌人抄之【抄楚交翻】且戰且行士卒凍餒比至茂州餘衆二千而已【此自秦州取道文扶循山至茂州也為王承休宗汭為魏王繼岌所誅張本比必利翻】重霸遂以秦隴來降 高季興常欲取三峽畏蜀峽路招討使張武威名不敢進至是乘唐兵勢使其子行軍司馬從誨權軍府事自將水軍上峽取施州張武以鐵鏁斷江路【斷音短】季興遣勇士乘舟斫之會風大起舟絓於鏁不能進退【絓音掛】矢石交下壞其戰艦【壞音怪】季興輕舟遁去【使蜀之邊帥盡如張武散關豈易入哉為後孟知祥復用張武張本】既而聞北路陷敗以夔忠萬三州遣使詣魏王降 郭崇韜遺王宗弼等書為陳利害【遺唯季翻】李紹琛未至利州宗弼棄城引兵西歸王宗勲等三招討追及宗弼於白芀【九域志簡州金水縣有白芀鎮芀都聊翻】宗弼懷中探詔書示之曰【探吐南翻】宋光嗣令我殺爾曹因相持而泣遂合謀送欵於唐

  資治通鑑卷二百七十三


    


 


 



 

 
  







 


  
  
 
 
 


  

 















	
	









































 
  



















 





 












  
  
  

 





