<!DOCTYPE html PUBLIC "-//W3C//DTD XHTML 1.0 Transitional//EN" "http://www.w3.org/TR/xhtml1/DTD/xhtml1-transitional.dtd">
<html xmlns="http://www.w3.org/1999/xhtml">
<head>
<meta http-equiv="Content-Type" content="text/html; charset=utf-8" />
<meta http-equiv="X-UA-Compatible" content="IE=Edge,chrome=1">
<title>資治通鑒_116-資治通鑑卷一百十五_116-資治通鑑卷一百十五</title>
<meta name="Keywords" content="資治通鑒_116-資治通鑑卷一百十五_116-資治通鑑卷一百十五">
<meta name="Description" content="資治通鑒_116-資治通鑑卷一百十五_116-資治通鑑卷一百十五">
<meta http-equiv="Cache-Control" content="no-transform" />
<meta http-equiv="Cache-Control" content="no-siteapp" />
<link href="/img/style.css" rel="stylesheet" type="text/css" />
<script src="/img/m.js?2020"></script> 
</head>
<body>
 <div class="ClassNavi">
<a  href="/24shi/">二十四史</a> | <a href="/SiKuQuanShu/">四库全书</a> | <a href="http://www.guoxuedashi.com/gjtsjc/"><font  color="#FF0000">古今图书集成</font></a> | <a href="/renwu/">历史人物</a> | <a href="/ShuoWenJieZi/"><font  color="#FF0000">说文解字</a></font> | <a href="/chengyu/">成语词典</a> | <a  target="_blank"  href="http://www.guoxuedashi.com/jgwhj/"><font  color="#FF0000">甲骨文合集</font></a> | <a href="/yzjwjc/"><font  color="#FF0000">殷周金文集成</font></a> | <a href="/xiangxingzi/"><font color="#0000FF">象形字典</font></a> | <a href="/13jing/"><font  color="#FF0000">十三经索引</font></a> | <a href="/zixing/"><font  color="#FF0000">字体转换器</font></a> | <a href="/zidian/xz/"><font color="#0000FF">篆书识别</font></a> | <a href="/jinfanyi/">近义反义词</a> | <a href="/duilian/">对联大全</a> | <a href="/jiapu/"><font  color="#0000FF">家谱族谱查询</font></a> | <a href="http://www.guoxuemi.com/hafo/" target="_blank" ><font color="#FF0000">哈佛古籍</font></a> 
</div>

 <!-- 头部导航开始 -->
<div class="w1180 head clearfix">
  <div class="head_logo l"><a title="国学大师官网" href="http://www.guoxuedashi.com" target="_blank"></a></div>
  <div class="head_sr l">
  <div id="head1">
  
  <a href="http://www.guoxuedashi.com/zidian/bujian/" target="_blank" ><img src="http://www.guoxuedashi.com/img/top1.gif" width="88" height="60" border="0" title="部件查字,支持20万汉字"></a>


<a href="http://www.guoxuedashi.com/help/yingpan.php" target="_blank"><img src="http://www.guoxuedashi.com/img/top230.gif" width="600" height="62" border="0" ></a>


  </div>
  <div id="head3"><a href="javascript:" onClick="javascript:window.external.AddFavorite(window.location.href,document.title);">添加收藏</a>
  <br><a href="/help/setie.php">搜索引擎</a>
  <br><a href="/help/zanzhu.php">赞助本站</a></div>
  <div id="head2">
 <a href="http://www.guoxuemi.com/" target="_blank"><img src="http://www.guoxuedashi.com/img/guoxuemi.gif" width="95" height="62" border="0" style="margin-left:2px;" title="国学迷"></a>
  

  </div>
</div>
  <div class="clear"></div>
  <div class="head_nav">
  <p><a href="/">首页</a> | <a href="/ShuKu/">国学书库</a> | <a href="/guji/">影印古籍</a> | <a href="/shici/">诗词宝典</a> | <a   href="/SiKuQuanShu/gxjx.php">精选</a> <b>|</b> <a href="/zidian/">汉语字典</a> | <a href="/hydcd/">汉语词典</a> | <a href="http://www.guoxuedashi.com/zidian/bujian/"><font  color="#CC0066">部件查字</font></a> | <a href="http://www.sfds.cn/"><font  color="#CC0066">书法大师</font></a> | <a href="/jgwhj/">甲骨文</a> <b>|</b> <a href="/b/4/"><font  color="#CC0066">解密</font></a> | <a href="/renwu/">历史人物</a> | <a href="/diangu/">历史典故</a> | <a href="/xingshi/">姓氏</a> | <a href="/minzu/">民族</a> <b>|</b> <a href="/mz/"><font  color="#CC0066">世界名著</font></a> | <a href="/download/">软件下载</a>
</p>
<p><a href="/b/"><font  color="#CC0066">历史</font></a> | <a href="http://skqs.guoxuedashi.com/" target="_blank">四库全书</a> |  <a href="http://www.guoxuedashi.com/search/" target="_blank"><font  color="#CC0066">全文检索</font></a> | <a href="http://www.guoxuedashi.com/shumu/">古籍书目</a> | <a   href="/24shi/">正史</a> <b>|</b> <a href="/chengyu/">成语词典</a> | <a href="/kangxi/" title="康熙字典">康熙字典</a> | <a href="/ShuoWenJieZi/">说文解字</a> | <a href="/zixing/yanbian/">字形演变</a> | <a href="/yzjwjc/">金 文</a> <b>|</b>  <a href="/shijian/nian-hao/">年号</a> | <a href="/diming/">历史地名</a> | <a href="/shijian/">历史事件</a> | <a href="/guanzhi/">官职</a> | <a href="/lishi/">知识</a> <b>|</b> <a href="/zhongyi/">中医中药</a> | <a href="http://www.guoxuedashi.com/forum/">留言反馈</a>
</p>
  </div>
</div>
<!-- 头部导航END --> 
<!-- 内容区开始 --> 
<div class="w1180 clearfix">
  <div class="info l">
   
<div class="clearfix" style="background:#f5faff;">
<script src='http://www.guoxuedashi.com/img/headersou.js'></script>

</div>
  <div class="info_tree"><a href="http://www.guoxuedashi.com">首页</a> > <a href="/SiKuQuanShu/fanti/">四库全书</a>
 > <h1>资治通鉴</h1> <!--         下载:【右键另存为】即可 --></div>
  <div class="info_content zj clearfix">
  
<div class="info_txt clearfix" id="show">
<center style="font-size:24px;">116-資治通鑑卷一百十五</center>
    資治通鑑卷一百十五<br />
<br />
  宋 司馬光 撰<br />
<br />
  胡三省 音注<br />
<br />
  晉紀三十七【起屠維作噩盡上章閹茂凡二年】<br />
<br />
  安皇帝庚<br />
<br />
  義熙五年春正月庚寅朔南燕主超朝會羣臣歎太樂不備【三年超獻太樂伎于秦故歎其不備朝直遥翻】議掠晉人以補伎【伎渠綺翻】領軍將軍韓曰【丁度曰竹角翻】先帝以舊京傾覆戢翼三齊【中山陷慕容德棄鄴保滑臺既而復失滑臺乃東取齊地而據之事並見前戢疾立翻】陛下不養士息民以伺魏釁恢復先業而更侵掠南鄰以廣讐敵可乎超曰我計已定不與卿言【史言慕容超愎諫致寇而亡伺相吏翻】辛卯大赦 庚戌以劉毅為衛將軍開府儀同三司毅愛才好士【好呼到翻】當世名流莫不輻凑獨揚州主簿吳郡張邵不往或問之邵曰主公命世人傑何煩多問【劉裕領揚州故稱之為主公】 秦王興遣其弟平北將軍冲征虜將軍狄伯支等帥騎四萬【帥讀曰率騎奇寄翻】擊夏王勃勃冲至嶺北謀還襲長安伯支不從而止因酖殺伯支以滅口 秦王興遣使册拜譙縱為大都督相國蜀王加九錫承制封拜悉如王者之儀二月南燕將慕容興宗斛穀提公孫歸等帥騎寇宿豫拔之【宿豫城在淮北帝置宿豫郡及宿豫縣唐代宗諱豫改為宿遷縣屬徐州宋白曰宿豫城在下邳東南百八十里蓋本宋人遷宿處也宋滅為邑漢為仇猶縣屬臨淮郡晉安帝立宿豫縣唐改宿遷縣將即亮翻】大掠而去簡男女二千五百付太樂教之歸五樓之兄也是時五樓為侍中尚書領左衛將軍專總朝政【朝直遥翻】宗親並居顯要王公内外無不憚之南燕王超論宿豫之功封斛穀提等並為郡縣公桂林王鎮諫曰此數人者勤民頓兵【頓讀曰鈍】為國結怨【為于偽翻】何功而封超怒不答尚書都令史王儼謟事五樓【漢尚書有令史十八人後增為二十一人其後員數愈增置都令史以總之】比歲屢遷官至左丞【比毗至翻禮記比年入學注每歲也漢書比年頻年也】國人為之語曰欲得侯事五樓超又遣公孫歸等寇濟南俘男女千餘人而去【此濟南郡亦是僑置於淮北濟子禮翻】自彭城以南民皆堡聚以自固詔并州刺史劉道憐鎮淮隂以備之 乞伏熾磐入見秦太原公懿於上邽【熾昌志翻】彭奚念乘虚伐之熾磐聞之怒不告懿而歸擊奚念破之遂圍枹罕乞伏乾歸從秦王興如平凉熾磐克枹罕【彭奚念據枹罕枹音膚】遣人告乾歸乾歸逃還苑川【乾歸為秦所留見上卷三年】馮翊人劉厥聚衆數千據萬年作亂【秦王興在平涼故厥乘間作亂】秦太子泓遣鎮軍將軍彭白狼帥東宫禁兵討之斬厥赦其餘黨諸將請露布表言廣其首級【帥讀曰率將即亮翻】泓不許曰主上委吾後事不能式遏寇逆當責躬請罪尚敢矜誕自為功乎【姚泓優游文義自儒者觀之似得子道然非撥亂才也】秦王興自平涼如朝那聞姚沖之謀【謂欲還襲長安也】賜沖死 三月劉裕抗表伐南燕朝議皆以為不可【朝直遥翻】惟左僕射孟昶車騎司馬謝裕參軍臧熹以為必克勸裕行裕以昶監中軍留府事【監中軍將軍留府事也昶丑兩翻監古銜翻】謝裕安之兄孫也初苻氏之敗也王猛之孫鎮惡來奔以為臨澧令【武帝太康四年立臨澧縣屬天門郡隋唐併入澧州澧陽縣澧音禮】鎮惡騎乘非長關弓甚弱【關讀曰彎】而有謀略善果斷喜論軍國大事或薦鎮惡於劉裕裕與語說之【斷丁亂翻喜許記翻說讀曰悦】因留宿明旦謂參佐曰吾聞將門有將【將即亮翻】鎮惡信然即以為中軍參軍 恒山崩【恒戶登翻】 夏四月乞伏乾歸如枹罕留世子熾磐鎮之收其衆得二萬徙都度堅山【度堅山乞伏之先司繁所居也】 雷震魏天安殿東序魏主珪惡之命左校以衝車攻東西序皆毁之初珪服寒食散【晉人多服寒食散今千金方中有數方蘇軾曰世有食鍾乳烏喙而縱酒色以求長年者蓋始於何晏晏少而富貴故服寒食散以濟其欲凡服之者疽背嘔血相踵也】久之藥發性多躁擾忿怒無常至是寖劇【躁則到翻】又災異數見【見賢遍翻】占者多言當有急變生肘腋【腋音亦】珪憂懣不安【懣音悶又音滿】或數日不食或達旦不寐追計平生成敗得失獨語不止疑羣臣左右皆不可信每百官奏事至前追計其舊惡輒殺之其餘或顔色變動或鼻息不調【氣一出一入謂之息】或步趨失節或言語差繆皆以為懷惡在心發形於外往往手擊殺之【史言魏主珪死期將至】死者皆陳天安殿前朝廷人不自保百官苟免莫相督攝盜賊公行里巷之間人為希少【為于偽翻少詩沼翻】珪亦知之曰朕故縱之使然待過災年更當清治之耳【治直之翻】是時羣臣畏罪多不敢求親近【近其靳翻】唯著作郎崔浩恭勤不懈或終日不歸浩吏部尚書宏之子也宏未嘗忤旨亦不諂諛故宏父子獨不被譴【懈居隘翻忤五故翻被皮義翻】 夏王勃勃率騎二萬攻秦【騎奇寄翻】掠取平涼雜胡七千餘戶進屯依力川【魏收地形志平涼城在漢安定鶉隂界唐為原州之地依力川又當在其東南】 己巳劉裕發建康帥舟師自淮入泗【帥讀曰率】五月至下邳留船艦輜重【艦戶黯翻重直用翻】步進至琅邪所過皆築城留兵守之【慮南燕以奇兵斷其後也】或謂裕曰燕人若塞大峴之險【水經注沐水出琅琊東莞縣西北山東南流右合峴水水北出大峴山今有大峴關魏收志齊郡盤陽縣有大峴山五代志臨胊縣有大峴山杜佑曰大峴在沂州沂水縣北塞悉則翻峴戶典翻】或堅壁清野大軍深入不唯無功將不能自歸奈何裕曰吾慮之熟矣鮮卑貪婪【婪盧含翻】不知遠計進利虜獲退惜禾苖謂我孤軍遠入不能持久不過進據臨朐【魏收志曰臨朐即漢之朐縣也屬東海郡晉曰臨胊屬東莞郡宋白曰因臨胊山而名朐音劬】退守廣固必不能守險清野敢為諸君保之【為于偽翻】南燕主超聞有晉師引羣臣會議征虜將軍公孫五樓曰吳兵輕果利在速戰不可爭鋒宜據大峴使不得入曠日延時沮其銳氣【沮在呂翻】然後徐簡精騎二千循海而南絶其糧道别敕段暉帥兖州之衆緣山東下【南然兖州治梁父緣梁父之山而東下也騎奇寄翻帥讀曰率】腹背擊之此上策也各命守宰依險自固校其資儲之外餘悉焚蕩芟除禾苖【芟所銜翻下同】使敵無所資彼僑軍無食【僑渠嬌翻】求戰不得旬月之間可以坐制此中策也縱賊入峴出城逆戰此下策也超曰今歲星居齊以天道推之不戰自克客主埶殊以人事言之彼遠來疲弊埶不能久吾據五州之地【南燕以并州牧鎮隂平幽州刺史鎮干徐州刺史鎮莒城兖州刺史鎮梁父青州刺史鎮東萊所謂五州也】擁富庶之民鐵騎萬羣麥禾布野奈何芟苖徙民先自蹙弱乎不如縱使入峴以精騎蹂之何憂不克【蹂人九翻】輔國將軍廣寜王賀賴盧苦諫不從退謂五樓曰必若此亡無日矣太尉桂林王鎮曰陛下必以騎兵利平地者宜出峴逆戰戰而不勝猶可退守不宜縱敵入峴自棄險固也超不從鎮出謂韓曰【丁度曰竹角翻】主上既不能逆戰却敵又不肯徙民清野延敵入腹坐待攻圍酷似劉璋矣【劉璋事見六十七卷漢獻帝建安十八年】今年國滅吾必死之卿中華之士復為文身矣【古者東南之民斷髮文身故鎮云然】超聞之大怒收鎮下獄【下戶稼翻】乃攝莒梁父二戍【父音甫】修城隍簡士馬以待之劉裕過大峴燕兵不出裕舉手指天喜形於色左右曰公未見敵而先喜何也裕曰兵已過險士有必死之志【謂已得過大峴之險】餘糧棲畝人無匱乏之憂【謂燕人不芟除禾苖】虜已入吾掌中矣六月己巳裕至東莞【莞音官】超先遣公孫五樓賀賴盧及左將軍段暉等將步騎五萬屯臨朐【朐音劬】聞晉兵入峴自將步騎四萬往就之使五樓帥騎進據巨蔑水【巨蔑水國語謂之具水袁宏謂之巨昧水水經謂之巨洋水水出朱虚縣太山北過其縣西又北過臨朐縣東上下沿水悉是劉裕伐廣固營壘所在】前鋒孟龍符與戰破之五樓退走裕以車四千乘為左右翼【乘繩證翻】方軌徐進與燕兵戰於臨朐南日向昃【日過中為向昃也】勝負猶未決参軍胡藩言於裕曰燕悉兵出戰臨朐城中留守必寡願以奇兵從間道取其城此韓信所以破趙也【間古莧翻韓信事見九卷漢高帝三年】裕遣藩及諮議參軍檀韶建威將軍河内向彌潜師出燕兵之後攻臨朐聲言輕兵自海道至矣向彌擐甲先登遂克之【向式亮翻擐音宦】超大驚單騎就段暉於城南【超自臨胊城中出城南就暉】裕因縱兵奮擊燕衆大敗斬段暉等大將十餘人超遁還廣固獲其玉璽輦及豹尾【服䖍曰大駕屬車八十一乘作三行尚書御史乘之最後一乘懸豹尾豹尾以前皆為省中晉志法駕屬車三十六乘最後車懸豹尾璽斯氏翻】裕乘勝逐北至廣固丙子克其大城超收衆入保小城裕築長圍守之圍高三丈穿塹三重【高古號翻重直龍翻塹七艶翻】撫納降附采拔賢俊華夷大悦于是因齊地糧儲悉停江淮漕運超遣尚書郎張綱乞師於秦赦桂林王鎮以為録尚書都督中外諸軍事引見謝之且問計焉鎮曰百姓之心係於一人今陛下親董六師奔敗而還【還從宣翻又如字】羣臣離心士民喪氣聞秦人自有内患【謂秦内有赫連之患也喪息浪翻】恐不暇分兵救人散卒還者尚有數萬宜悉出金帛以餌之更決一戰若天命助我必能破敵如其不然死亦為美比於閉門待盡不猶愈乎司徒樂浪王惠曰不然【樂浪音洛琅】晉兵乘勝氣埶百倍我以敗軍之卒當之不亦難乎秦雖與勃勃相持不足為患且與我分據中原埶如脣齒安得不來相救但不遣大臣則不能得重兵尚書令韓範為燕秦所重【事見上卷二年】宜遣乞師超從之秋七月加劉裕北青冀二州刺史【晉氏南渡立南青冀二州於淮南北青冀二州於齊地】南燕尚書畧陽垣尊及弟京兆太守苖踰城來降裕以為行參軍【垣氏子孫後遂為南國邊將著功名】尊苖皆超所委任以為腹心者也或謂裕曰張綱有巧思【思相吏翻】若得綱使為攻具廣固必可拔也會綱自長安還太山太守申宣執之送於裕裕升綱於樓車【杜預曰樓車車上望櫓】使周城呼曰【呼火故翻】劉勃勃大破秦軍無兵相救城中莫不失色江南每發兵及遣使者至廣固裕輒潜遣兵夜迎之明日張旗鳴鼓而至【董卓之入洛計亦出此】北方之民執兵負糧歸裕者日以千數圍城益急張華封愷皆為裕所獲超請割大峴以南地為藩臣裕不許秦王興遣使謂裕曰慕容氏相與鄰好【好呼到翻】今晉攻之急秦已遣鐵騎十萬屯洛陽晉軍不還當長驅而進裕呼秦使者謂曰語汝姚興【使疏吏翻下同語牛倨翻下相語同】我克燕之後息兵三年當取關洛今能自送便可速來劉穆之聞有秦使馳入見裕而秦使者已去裕以所言告穆之穆之尤之【尤怪也過也】曰常日事無大小必賜預謀此宜善詳【善謂善為之辭詳謂審諦也】云何遽爾答之此語不足以威敵適足以怒之若廣固未下羌寇奄至不審何以待之裕笑曰此是兵機非卿所解【解尸買翻曉也】故不相語耳【語牛倨翻】夫兵貴神速彼若審能赴救必畏我知寧容先遣信命逆設此言是自張大之辭也晉師不出為日久矣羌見伐齊殆將内懼自保不暇何能救人邪 乞伏乾歸復即秦王位【復扶又翻】大赦改元更始【更工衡翻】公卿以下皆復本位【乾歸降公卿將帥為僚佐偏裨見一百十二卷隆安五年】 慕容氏在魏者百餘家謀逃去魏主珪盡殺之初魏太尉穆崇與衛王儀伏甲謀弑魏主珪不果珪惜崇儀之功袐而不問及珪有疾殺大臣儀自疑而出亡追獲之八月賜儀死 封融詣劉裕降【封融奔魏見上卷二年魏殺慕容氏故融歸裕降戶江翻】 九月加劉裕太尉裕固辭 秦王興自將擊夏王勃勃至貳城【貳城貳縣城也在杏城西北平東南】遣安遠將軍姚詳等分督租運勃勃乘虚奄至興懼欲輕騎就詳等【騎奇寄翻】右僕射韋華曰若鑾輿一動衆心駭懼必不戰自潰詳營亦未必可至也興與勃勃戰秦兵大敗將軍姚榆生為勃勃所禽左將軍姚文崇等力戰勃勃乃退興還長安勃勃復攻秦敕奇堡黄石固【魏收地形志原州長城郡有黄石縣五代志西魏改黄石為長城隋開皇初廢郡為縣大業初改長城縣為百泉縣復扶又翻】我羅城皆抜之徙七千餘家於大城以其丞相右地代領幽州牧以鎮之初興遣衛將軍姚強帥步騎一萬隨韓範往就姚紹於洛陽并兵以救南燕【帥讀曰率騎奇寄翻】及為勃勃所敗【敗補邁翻】追強兵還長安韓範歎曰天滅燕矣南燕尚書張俊自長安還降於劉裕【降戶江翻】因說裕曰【說輸芮翻】燕人所恃者謂韓範必能致秦師也今得範以示之燕必降矣裕乃表範為散騎常侍【散悉亶翻騎奇寄翻】且以書招之長水校尉王蒲勸範犇秦範曰劉裕起布衣滅桓玄復晉室今興師伐燕所向崩潰此殆天授非人力也燕亡則秦為之次矣吾不可以再辱遂降於裕【漢李陵降匈奴霍光上官桀使其故人任立政招之使歸陵曰大丈夫不能再辱】裕將範循城城中人情離沮【將如字引也沮在呂翻】或勸燕主超誅範家超以範弟盡忠無貳并範家赦之冬十月段宏自魏奔于裕【宏奔魏見上卷三年】張綱為裕造攻具盡諸奇巧超怒縣其母於城上支解之【為于偽翻縣讀曰懸】 西秦王乾歸立夫人邊氏為皇后世子熾磐為太子仍命熾磐都督中外諸軍錄尚書事【熾昌志翻】以屋引破光為河州刺史鎮枹罕【枹音膚】以南安焦遺為太子太師與參軍國大謀乾歸曰焦生非特名儒乃王佐之才也謂熾磐曰汝事之當如事吾熾磐拜遺於床下遺子華至孝乾歸欲以女妻之【妻七細翻】辭曰凡娶妻者欲與之共事二親也今以王姬之貴【周姬姓也故王女謂之王姬後世因而稱之凡王者之女皆謂之王姬】下嫁蓬茅之士誠非其匹臣懼其闕於中饋【易家人之六二曰在中饋言以隂應陽居中得正盡婦人之義職乎中饋巽順而已饋食也】非所願也乾歸曰卿之所行古人之事孤女不足以強卿乃以為尚書民部郎【魏尚書郎有民曹晉初分置左民右民江左以後省右民郎有左民郎民部郎至是始見于通鑑強其兩翻】 北燕王雲自以無功德而居大位内懷危懼常畜養壯士以為腹心爪牙【畜吁玉翻】寵臣離班桃仁專典禁衛【離桃皆姓也班仁其名】賞賜以巨萬計衣食起居皆與之同而班仁志願無厭【厭於鹽翻】猶有怨憾戊辰雲臨東堂班仁懷劒執帋而入【帋與紙同通俗書也】稱有所啟班抽劒擊雲雲以几扞之仁從旁擊雲弑之【高雲以勇力發身叨居君位自謂非壯士以為翼衛不足以防其身豈知小人之難養也是以古之綴衣虎賁左右擕僕必用吉士其慮患誠深遠也雲得燕見上卷三年】馮跋升洪光門以觀變帳下督張泰李桑言於跋曰此豎埶何所至請為公斬之【為于偽翻】乃奮劒而下桑斬班于西門泰殺仁于庭中衆推跋為主跋以讓其弟范陽公素弗素弗不可跋乃即天王位於昌黎【載記馮跋字文起長樂信都人其先畢萬之後也萬之子孫有食采馮鄉者因氏焉】大赦詔曰陳氏代姜不改齊國【周師尚父始封于齊姜姓也戰國時齊太公田和陳敬仲之後也簒姜氏之後而取其國仍號曰齊】宜即國號曰燕改元太平諡雲曰惠懿皇帝跋尊母張氏為太后立妻孫氏為王后子永為太子以范陽公素弗為車騎大將軍録尚書事孫護為尚書令張興為左僕射汲郡公弘為右僕射廣川公萬泥為幽平二州牧上谷公乳陳為并青二州牧素弗少豪俠放蕩【少詩沼翻俠戶頰翻蕩徒浪翻】嘗請婚於尚書左丞韓業業拒之及為宰輔待業尤厚好申拔舊門【好呼到翻】謙恭儉約以身帥下【帥讀曰率】百僚憚之論者美其有宰相之度【温公作通鑑雖相小國者苟有片善必因舊史而表章之以言為輔之難】 魏主珪將立齊王嗣為太子魏故事凡立嗣子輒先殺其母乃賜嗣母劉貴人死珪召嗣諭之曰漢武帝殺鉤弋夫人以防母后豫政外家為亂也【事見二十二卷漢武帝後元元年】汝當繼統吾故遠迹古人【蜀本作故吾】為國家長久之計耳嗣性孝哀泣不自勝珪怒之嗣還舍日夜號泣【勝音升號戶高翻】珪知而復召之【復扶又翻】左右曰上怒甚入將不測不如且避之俟上怒解而入嗣乃逃匿於外帷帳下代人車路頭【車焜氏拓跋氏之疏屬也至後魏孝文改為車氏】京兆王洛兒二人隨之初珪如賀蘭部見獻明賀太后之妹美【珪父寔魏昭成帝什翼犍之嫡子也先昭成而薨追諡獻明皇帝賀太后從夫諡】言於賀太后請納之賀太后曰不可是過美必有不善【左傳晉叔向欲娶於申公巫臣氏其母止之曰甚美必有甚惡此語類之】且已有夫不可奪也珪密令人殺其夫而納之生清河王紹紹兇狠無賴好輕遊里巷刼剥行人以為樂【狠戶墾翻好呼到翻樂音洛】珪怒之嘗倒懸井中垂死乃出之齊王嗣屢誨責之紹由是與嗣不協戊辰珪譴責賀夫人【譴去戰翻】囚將殺之會日暮未決夫人密使告紹曰汝何以救我左右以珪殘忍人人危懼紹年十六夜與帳下及宦者宫人數人通謀踰垣入宫至天安殿左右呼曰賊至【呼火故翻】珪驚起求弓刀不獲遂弑之【年三十九明元帝永興二年上諡曰宣武皇帝廟號烈祖泰常五年改諡道武】己巳宫門至日中不開紹稱詔集百官於端門前【宫門正南門曰端門】北面立【句斷】紹從門扉間【扉門扇也】謂百官曰我有叔父亦有兄公卿欲從誰衆愕然失色莫有對者良久南平公長孫嵩曰從王【長知兩翻】衆乃知宫車晏駕而不測其故莫敢出聲唯隂平公烈大哭而去烈儀之弟也【魏之克燕儀有功焉是年八月賜死】於是朝野恟恟人懷異志【朝直遥翻恟許拱翻】肥如侯賀護舉烽於安陽城北【安陽城即漢代郡之東安陽縣城也魏收地形志永熙中置高柳郡治安陽】賀蘭部人皆赴之其餘諸部亦各屯聚紹聞人情不安大出布帛賜王公以下崔宏獨不受【史言崔宏有識】齊王嗣聞變乃自外還晝伏匿山中夜宿王洛兒家洛兒鄰人李道潛奉給嗣民間頗知之喜而相告紹聞之收道斬之紹募人求訪嗣欲殺之獵郎叔孫俊【拓跋氏起于代北俗尚獵故置獵郎以豪望子弟有材勇者為之亦漢期門郎羽林郎之類也魏書官氏志天賜元年置散騎郎獵郎諸省令史省事典籖等後魏孝文以獻帝叔父之後乙旃氏為叔孫氏】與宗室疏屬拓跋磨渾【磨渾元城侯屈之子也】自云知嗣所在紹使帳下二人與之偕往俊磨渾得出即執帳下詣嗣斬之俊建之子也王洛兒為嗣往來平城通問大臣【為于偽翻】夜告安遠將軍安同等衆聞之翕然響應爭出奉迎嗣至城西衛士執紹送之嗣殺紹及其母賀氏并誅紹帳下及宦官宫人為内應者十餘人其先犯乘輿者群臣臠食之【乘繩證翻】壬申嗣即皇帝位【嗣道武皇帝之長子也蕭子顯曰嗣字木末】大赦改元永興追尊劉貴人曰宣穆皇后公卿先罷歸第不預朝政者悉召用之【朝直遥翻】詔長孫嵩與北新侯安同山陽侯奚斤【後魏孝文以獻帝第三兄之後為達奚氏尋又改為奚氏】白馬侯崔宏元城侯拓跋屈等八人坐止車門右【臣子至宫門皆下車而入故謂之止車門】共聽朝政時人謂之八公屈磨渾之父也嗣以尚書燕鳳逮事什翼犍【什翼犍為代王以鳳為左長史犍居言翻】使與都坐大官封懿等【魏謂尚書都省為尚書都坐都坐大官蓋尚書長官也坐徂卧翻】入侍講論出議政事以王洛兒車路頭為散騎常侍叔孫俊為衛將軍【散悉亶翻騎奇寄翻】拓跋磨渾為尚書皆賜爵郡縣公嗣問舊臣為先帝所親信者為誰王洛兒言李先【先慕容永之謀主也永滅徙中山魏伐燕先歸魏道武親信之】嗣召問先卿以何才何功為先帝所知對曰臣不才無功但以忠直為先帝所知耳詔以先為安東將軍常宿於内以備顧問朱提王悦䖍之子也【拓跋䖍見二百八卷孝武太元二十一年朱提音銖時】有罪自疑懼閏十一月丁亥悅懷匕首入侍將作亂叔孫俊覺其舉止有異引手掣之索懷中得匕首【掣昌列翻索山客翻】遂殺之 十二月乙巳太白犯虚危【虚二星危三星晉天文志自須女八度至危十五度為玄枵齊之分野屬青州】南燕靈臺令張光勸南燕主超出降【降戶江翻下同】超手殺之 柔然侵魏<br />
<br />
  六年春正月甲寅朔南燕主超登天門【天門廣固内城南門也】朝羣臣於城上【朝直遥翻】乙卯超與寵姬魏夫人登城見晉兵之盛握手對泣韓諫曰【竹角翻】陛下遭堙厄之運正當努力自強以壯士民之志而更為兒女子泣邪【為于偽翻下為民同】超拭目謝之尚書令董詵勸超降超怒囚之【詵疎臻翻】魏長孫嵩將兵伐柔然 魏主嗣以郡縣豪右多為民患悉以優詔徵之民戀土不樂内徙【樂音洛】長吏逼遣之於是無賴少年逃亡相聚【長知兩翻少詩沼翻】所在寇盜羣起嗣引八公議之曰朕欲為民除蠧而守宰不能綏撫使之紛亂今犯者既衆不可盡誅吾欲大赦以安之何如元城侯屈曰民逃亡為盜不罪而赦之是為上者反求於下也不如誅其首惡赦其餘黨崔宏曰聖王之御民務在安之而已不與之較勝負也夫赦雖非正可以行權屈欲先誅後赦要為兩不能去【兩不能去言先不能去誅後又不能去赦也去羌呂翻】曷若一赦而遂定乎赦而不從誅未晩也嗣從之二月癸未朔遣將軍于栗磾將騎一萬討不從命者所向皆平【史言魏有謀臣所以靖亂磾丁奚翻將即亮翻騎奇寄翻下同】 南燕賀賴盧公孫五樓為地道出擊晉兵不能却城久閉城中男女病脚弱者大半出降者相繼【降戶江翻】超輦而登城尚書悦夀說超曰【說輸芮翻】今天助寇為虐戰士凋瘁【瘁秦醉翻】獨守窮城絶望外援天時人事亦可知矣苟歷數有終堯舜避位陛下豈可不思變通之計乎超歎曰廢興命也吾寜奮劒而死不能銜璧而生丁亥劉裕悉衆攻城或曰今日往亡不利行師【歷書二月以驚蟄後十四日為往亡日】裕曰我往彼亡何為不利四面急攻之悦夀開門納晉師超與左右數十騎踰城突圍出走追獲之裕數以不降之罪【數所具翻降戶江翻】超神色自若一無所言惟以母託劉敬宣而已【敬宣先嘗奔燕故超以母託之夫孝莫大於寧親超以母之故屈節事秦竭聲伎以奉之既又掠取晉人以足聲伎由是致寇至于母子並為俘虜乃更欲以託劉敬宣何庸淺也】裕忿廣固久不下欲盡阬之以妻女賞將士韓範諫曰晉室南遷中原鼎沸士民無援強則附之既為君臣必須為之盡力【為于偽翻】彼皆衣冠舊族先帝遺民今王師弔伐而盡阬之使安所歸乎竊恐西北之人無復來蘇之望矣【湯征諸侯東面而征西夷怨南面而征北狄怨曰奚為後我攸徂之民室家胥慶曰傒我后后來其蘇】裕改容謝之然猶斬王公以下三千人沒入家口萬餘夷其城隍送超詣建康斬之【隆安二年慕容德建國號南燕二主十三年而亡】<br />
<br />
  臣光曰晉自濟江以來威靈不競戎狄横騖虎噬中原劉裕始以王師翦平東夏【騖音務夏戶雅翻】不於此時旌禮賢俊慰撫疲民宣愷悌之風滌殘穢之政使羣士嚮風遺黎企踵而更恣行屠戮以快忿心迹其施設曾苻姚之不如宜其不能蕩壹四海成美大之業豈非雖有智勇而無仁義使之然哉<br />
<br />
  初徐道覆聞劉裕北伐勸盧循乘虛襲建康循不從道覆自至番禺【番禺音潘愚】說循曰本住嶺外【說輸芮翻交廣之地在五嶺之外】豈以理極於此傳之子孫邪正以劉裕難與為敵故也今裕頓兵堅城之下未有還期我以此思歸死士【孫泰徒黨本三吳之人孫恩所掠者又三吳人也久在海中故皆懷土思歸】掩擊何劉之徒如反掌耳【何劉謂何無忌劉毅也】不乘此機而苟求一日之安朝廷常以君為腹心之疾若裕平齊之後息甲歲餘以璽書徵君裕自將屯豫章遣諸將帥鋭師過嶺【璽音斯氏翻將音即亮翻帥讀曰率下同】雖復以將軍之神武恐必不能當也【復音扶又翻】今日之機萬不可失若先克建康傾其根蒂裕雖南還無能為也君若不同便當帥始興之衆直指尋陽【元興三年循使道覆攻陷始興因使守之】循甚不樂此舉而無以奪其計乃從之【樂音洛】初道覆使人伐船材於南康山【南康山南康縣之山也吳立安南縣於漢豫章梅嶺武帝太康元年更名南康所謂梅嶺今大庾嶺是也南康山即大庾諸山皆在今南安軍界】至始興賤賣之【自南康西至始興四百里】居人争市之船材大積而人不疑至是悉取以裝艦【艦戶黯翻】旬日而辦循自始興寇長沙道覆寇南康廬陵豫章諸守相皆委任奔走【守式又翻相息亮翻】道覆順流而下【順贑石之流而下】舟械甚盛時克燕之問未至朝廷急徵劉裕裕方議留鎮下邳經營司雍【雍於用翻】會得詔書乃以韓範為都督八郡軍事燕郡太守【青州舊督齊濟南樂安城陽東莱長廣平昌高密八郡而所謂燕郡者蓋南燕於廣固置燕都尹而今改為燕郡太守耳】封融為勃海太守檀韶為琅邪太守戊申引兵還韶祗之兄也久之劉穆之稱範融謀反皆殺之【二人燕之舊臣穆之恐其為變故殺之】 安成忠肅公何無忌自尋陽引兵拒盧循【諡法危身奉上曰忠剛德克就曰肅】長史鄧潜之諫曰國家安危在此一舉聞循兵艦大盛埶居上流宜決南塘守二城以待之【贑水出漢豫章南壄縣聶都山漢南壄晉南康之地也贑水至南昌縣歷南塘南塘在徐孺子宅西二城謂豫章尋陽也水經注曰豫章城東大湖十里二百二百六步北與城齊南緣迴折至南塘本通贑江增減與江水同漢永元中太守張躬築塘以通南路兼遏此水若決南塘則盧循之舟兵無所用可以堅守而待其敝】彼必不敢捨我遠下蓄力養鋭俟其疲老然後擊之此萬全之策也今決成敗於一戰萬一失利悔將無及參軍殷闡曰循所將之衆皆三吳舊賊百戰餘勇始興溪子拳捷善鬭未易輕也【始興溪子謂徐道覆所統始興兵也詩云無拳無勇毛傳曰拳力也將即亮翻易以䜴翻】將軍宜留屯豫章徵兵屬城兵至合戰未為晚也若以此衆輕進殆必有悔無忌不聽三月壬申與徐道覆遇於豫章賊令彊弩數百登西岸小山邀射之【射而亦翻】會西風暴急飄無忌所乘小艦向東㟁賊乘風以大艦逼之衆遂奔潰無忌厲聲曰取我蘇武節來節至執以督戰賊衆雲集無忌辭色無撓【撓奴教翻】握節而死於是中外震駭朝議欲奉乘輿北走就劉裕【朝直遥翻乘繩證翻】既而知賊未至乃止 西秦王乾歸攻秦金城郡拔之 夏王勃勃遣尚書胡金纂攻平涼秦王興救平涼擊金纂殺之勃勃又遣兄子左將軍羅提攻拔定陽【魏收地形志敷城郡有定陽縣在今鄜州鄜城縣界】阬將士四千餘人秦將曹熾曹雲王肆佛等各將數千戶内徙【將即亮翻】興處之湟山及陳倉【據載記湟山澤名處昌呂翻】勃勃寇隴右破白崖堡遂趣清水【清水縣前漢屬天水郡後漢省晉分屬畧陽郡元豐九域志清水縣在秦州東九十里有白沙鎮縣西又有白石堡趣七喩翻】畧陽太守姚夀都棄城走勃勃徙其民萬六千戶於大城興自安定追之至夀渠川不及而還 初南涼王傉檀遣左將軍枯木等伐沮渠蒙遜掠臨松千餘戶而還【張天錫分張掖置臨松郡五代志甘州張掖縣後周併臨松入焉傉奴沃翻沮子余翻還從宣翻又如字下同】蒙遜伐南涼至顯美徙數千戶而去【顯美縣前漢屬張掖郡後漢晉屬武威郡五代志後周廢顯美入姑臧縣】南涼太尉俱延復伐蒙遜大敗而歸【復扶又翻】是月傉檀自將五萬騎伐蒙遜【將即亮翻】戰于窮泉傉檀大敗單馬奔還蒙遜乘勝進圍姑臧姑臧人懲王鍾之誅皆驚潰【王鍾誅見上卷四年】夷夏萬餘戶降於蒙遜【夏戶雅翻降戶江翻下同】傉檀懼遣司隸校尉敬歸及子佗為質於蒙遜以請和【何承天姓苑敬姓黄帝孫敬康之後風俗通陳敬仲之後質音致】蒙遜許之歸至胡阬逃還佗為追兵所執蒙遜徙其衆八千餘戶而去右衛將軍折掘奇鎮據石驢山以叛【石驢山在姑臧西南長寜川西北屬晉昌郡界張寔討曹袪於晉昌自姑臧西踰石驢據長寜折而設翻掘其月翻】傉檀畏蒙遜之逼且懼嶺南為奇鎮所據乃遷於樂都【樂音洛】留大司農成公緒守姑臧傉檀纔出城魏安人侯諶等【晉書載記作焦諶王侯等諶氏壬翻】閉門作亂收合三千餘家據南城推焦朗為大都督龍驤大將軍諶自稱涼州刺史降于蒙遜【傉檀自據姑臧之後與四鄰交兵所遇輒敗不惟失姑臧亦不能保樂都矣詩曰毋田甫田惟莠驕驕我思遠人勞心忉忉正謂此也】劉裕至下邳以船載輜重【重直用翻】自帥精鋭步歸【帥讀曰率】<br />
<br />
  至山陽聞何無忌敗死慮京邑失守卷甲兼行【卷讀曰捲】與數十人至淮上【李延夀南史作江上當從之盖裕至山陽則已渡淮也】問行人以朝廷消息行人曰賊尚未至劉公若還便無所憂裕大喜將濟江風急衆咸難之裕曰若天命助國風當自息若其不然覆溺何害【溺奴狄翻】即命登舟舟移而風止過江至京口衆乃大安夏四月癸未裕至建康以江州覆沒表送章綬詔不許【綬音受】青州刺史諸葛長民兖州刺史劉藩并州刺史劉道憐各將兵入衛建康【青州兖州并州時皆僑在江淮間將即亮翻】藩豫州刺史毅之從弟也【從才用翻】毅聞盧循入寇將拒之而疾作既瘳將行劉裕遺毅書曰吾往習擊妖賊【孫泰以左道惑衆孫恩盧循皆其黨也故謂之妖賊遺于季翻妖於嬌翻】曉其變態賊新獲姦利其鋒不可輕今修船垂畢當與弟同舉克平之日上流之任皆以相委又遣劉藩往諭止之毅怒謂藩曰往以一時之功相推耳汝便謂我真不及劉裕邪投書於地帥舟師二萬發姑孰【毅時以豫州刺史鎮姑孰帥讀曰率】循之初入寇也使徐道覆向尋陽循自將攻湘中諸郡【湘中諸郡漢長沙零桂之地】荆州刺史劉道規遣軍逆戰敗於長沙循進至巴陵將向江陵徐道覆聞毅將至馳使報循曰【使疏吏翻】毅兵甚盛成敗之事係之於此宜并力摧之若此克捷江陵不足憂也循即日發巴陵與道覆合兵而下五月戊午毅與循戰于桑落洲毅兵大敗棄船以數百人步走餘衆皆為循所虜所棄輜重山積【重直用翻】初循至尋陽聞裕已還猶不信既破毅乃得審問【審者悉其實也問音問也】與其黨相視失色循欲退還尋陽攻取江陵據二州以抗朝廷【二州謂荆江也】道覆謂宜乘勝徑進固争之循猶豫累日乃從之己未大赦裕募人為兵賞之同京口赴義之科【裕起兵於京口以討桓玄赴義之人酬賞重於當時】民治石頭城【治直之翻】議者謂宜分兵守諸津要裕曰賊衆我寡若分兵屯守則測人虚實且一處失利則沮三軍之心【沮在呂翻】今聚衆石頭隨宜應赴既令彼無以測多少【少詩沼翻】又於衆力不分若徒旅轉集徐更論之耳朝廷聞劉毅敗人情恟懼時北師始還將士多創病【恟許拱翻創初良翻】建康戰士不盈數千循既克二鎮【二鎮謂江豫也】戰士十餘萬舟車百里不絶樓船高十二丈【高古號翻】敗還者争言其彊盛孟昶諸葛長民欲奉乘輿過江裕不聽【時江西江北皆無城池可倚昶長民欲奉天子過江不過東走廣陵西據歷陽耳昶丑兩翻】初何無忌劉毅之南討也昶策其必敗已而果然至是又謂裕必不能抗循衆頗信之惟龍驤將軍東海虞丘進廷折昶等以為不然【驤思將翻虞丘複姓史記楚相有虞丘子折之舌翻】中兵參軍王仲德言於裕曰明公命世作輔新建大功威震六合【魏尚書曹有中兵郎諸公府征鎮亦因而置中兵參軍新建大功謂滅燕也】妖賊乘虛入寇既聞凱還自當奔潰若先自遁逃則勢同匹夫匹夫號令何以威物此謀若立請從此辭裕甚悦昶固請不已裕曰今重鎮外傾彊寇内逼人情危駭莫有固志若一旦遷動便自土崩瓦解江北亦豈可得至設令得至不過延日月耳今兵士雖少【少詩沼翻下同】自足一戰若其克濟則臣主同休苟厄運必至我當横尸廟門遂其由來以身許國之志不能竄伏草間苟求存活也我計決矣卿勿復言昶恚其言不行【恚於避翻】且以為必敗因請死裕怒曰卿且申一戰【申重也】死復何晚【復扶又翻】昶知裕終不用其言乃抗表自陳曰臣裕北討衆並不同唯臣贊裕行計【事見上年】致使彊賊乘間社稷危逼臣之罪也【間古莧翻】謹引咎以謝天下封表畢仰藥而死乙丑盧循至淮口【秦淮入江之口也】中外戒嚴琅邪王德文都督宫城詣軍事屯中堂皇【堂無四壁曰皇】劉裕屯石頭諸將各有屯守裕子義隆始四歲裕使諮議參軍劉粹輔之鎮京口粹毅之族弟也裕見民臨水望賊怪之以問參軍張劭劭曰若節鉞未反民奔散之不暇亦何能觀望今當無復恐耳【復扶又翻】裕謂將佐曰賊若於新亭直進其鋒不可當宜且迴避勝負之事未可量也【量音良】若迴泊西岸【西岸即蔡州】此成禽耳徐道覆請於新亭至白石焚舟而上【上時掌翻】數道攻裕循欲以萬全為計謂道覆曰大軍未至孟昶便望風自裁以大勢言之自當計日潰亂今決勝負於一朝乾沒求利【乾音干】既非必克之道且殺傷士卒不如案兵待之道覆以循多疑少決乃歎曰我終為盧公所誤事必無成使我得為英雄驅馳天下不足定也【為于偽翻】裕登石頭城望循軍初見引向新亭顧左右失色既而迴泊蔡洲乃悦【蔡洲在石頭西㟁今建康府上元縣西二十五里有蔡洲】于是衆軍轉集裕恐循侵軼【軼徒結翻】用虞丘進計伐樹柵石頭淮口修治越城築查浦藥園廷尉三壘【查浦在大江南岸近秦淮口藥園蓋種芍藥之所廷尉寺舍所在因以為地名查莊加翻據晉書帝紀三壘皆在淮口】皆以兵守之劉毅經涉蠻晉【西陽上下羣蠻所居之地謂之蠻其為王民應租税征役者謂之晉】僅能自免從者飢疲死亡什七八【從才用翻】丙寅至建康待罪裕慰勉之使知中外留事【知都督中外諸軍府留事也】毅乞自貶詔降為後將軍魏長孫嵩至漠北而還【還從宣翻又如字下同】柔然追圍之於<br />
<br />
  牛川壬申魏主嗣北擊柔然柔然可汗社崘聞之遁走道死【崘盧昆翻】其子度拔尚幼部衆立社崘弟斛律號藹豆蓋可汗【可讀從刋入聲汗音寒】嗣引兵還参合陂 盧循伏兵南岸【南岸即秦淮口南岸】使老弱乘舟向白石聲言悉衆自白石步上【上時掌翻】劉裕留參軍沈林子徐赤特戍南岸斷查浦【斷丁管翻】戒令堅守勿動裕及劉毅諸葛長民北出拒之林子曰妖賊此言未必有實宜深為之防裕曰石頭城險且淮柵甚固留卿在後足以守之林子穆夫之子也【沈穆夫吳興武康人隆安三年孫恩寇會稽三吳響應穆夫在會稽恩以為餘姚令恩為劉牢之所破并殺穆夫】庚辰盧循焚查浦進至張侯橋徐赤特將擊之林子曰賊聲往白石而屢來挑戰【挑徒了翻】其情可知吾衆寡不敵不如守險以待大軍赤特不從遂出戰伏兵赤特大敗單舸奔淮北【秦淮北岸也】林子及將軍劉鍾據柵力戰朱齡石救之賊乃退循引精兵大上至丹陽郡【丹陽郡丹陽尹治所也上時掌翻】裕帥諸軍馳還石頭【帥讀曰率】斬徐赤特解甲久之乃出陳於南塘【南塘秦淮南㟁也陳讀曰陣】 六月以劉裕為太尉中書監加黄鉞裕受黄鉞餘固辭以車騎中軍司馬庾悦為江州刺史悦凖之子也【劉裕為車騎將軍以劉敬宣征蜀失利乞降號中軍將軍故車騎中軍二府共一司馬庾凖庾亮之孫也】 司馬國璠及弟叔璠叔道奔秦【璠孚袁翻】秦王興曰劉裕方誅桓玄輔晉室卿何為來對曰裕削弱王室臣宗族有自修立者裕輒除之方為國患甚於桓玄耳興以國璠為揚州刺史叔道為交州刺史 盧循寇掠諸縣無所得謂徐道覆曰師老矣不如還尋陽并力取荆州據天下三分之二徐更與建康争衡耳秋七月庚申循自蔡洲南還尋陽留其黨范崇民將五千人據南陵【南陵在宣城郡宣城縣西梁置南陵郡及南陵縣蓋漢丹陽郡石城縣之界也今為池州貴池縣地循慮兵有利鈍欲南歸番禺故使崇民守之以固彭蠡湖口宋白曰柵江口對岸即舊南陵縣地今為繁昌縣】甲子裕使輔國將軍王仲德廣川太守劉鍾河間内史蘭陵蒯恩中軍諮議參軍孟懷玉等帥衆追循【帥讀曰率】 乙丑魏主嗣還平城 西秦王乾歸討越質屈機等十餘部【越質鮮卑種也其酋曰叱黎叱黎之子曰詰歸孝武太元十六年降于乾歸二十一年叛降秦屈機即詰歸也語稍訛耳】降其衆二萬五千【降戶江翻】徙於苑川八月乾歸復都苑川【乞伏氏本都度堅山乾歸強盛始都苑川旣為秦所破而降於秦秦使鎮苑川復叛恐為秦所襲還保度堅山今部衆浸盛不畏秦復都苑川】 沮渠蒙遜伐西涼敗西涼世子歆于馬廟【古者祭馬祖後世因立廟祭之故名其地為馬廟】禽其將朱元虎而還涼公暠以銀二千斤金二千兩贖元虎蒙遜歸之遂與暠結盟而還【將子亮翻暠古老翻】 劉裕還東府【盧循退裕乃還東府】大治水軍【治直之翻】遣建威將軍會稽孫處振武將軍沈田子帥衆三千自海道襲番禺【會工外翻處昌呂翻帥讀曰率番禺音潘愚】田子林子之兄也衆皆以為海道艱遠必至為難且分撤見力【見賢遍翻下同】非目前之急裕不從敇處曰大軍十二月之交必破妖虜卿至時先傾其巢窟使彼走無所歸也譙縱遣侍中譙良等入見於秦請兵以伐晉【見賢遍翻】縱以桓謙為荆州刺史譙道福為梁州刺史帥衆二萬寇荆州秦王興遣前將軍苟林帥騎兵會之江陵自盧循東下不得建康之問【問音問也】羣盜互起荆州刺史劉道規遣司馬王鎮之帥天門太守檀道濟【吳孫休永安六年分武陵立天門郡充縣有松梁山山有石石開處數十丈其高以弩仰射不至其上名天門因以名郡輿地志澧州石門縣古天門郡帥讀曰率下同】廣武將軍彭城到彦之入援建康道濟祗之弟也鎮之至尋陽為苟林所破盧循聞之以林為南蠻校尉分兵配之使乘勝伐江陵聲言徐道覆已破建康桓謙於道召募義舊【桓氏世居荆楚舊恩所結義不相忘謂之義舊】民投之者二萬人謙屯枝江【枝江縣自漢以來屬南郡春秋之羅國也江水於縣西别出為沱而東復合於江故曰枝江我朝熙寜六年省枝江為鎮入松滋縣】林屯江津二寇交逼江陵士民多懷異心道規乃會將士告之曰桓謙今在近道聞諸長者頗有去就之計【長知兩翻】吾東來文武足以濟事【東來文武謂道規從行將佐兵士也】若欲去者本不相禁因夜開城門達曉不閉衆咸憚服莫有去者雍州刺史魯宗之帥衆數千自襄陽赴江陵【雍於用翻】或謂宗之情未可測道規單馬迎之宗之感悦道規使宗之居守【守式又翻】委以腹心自帥諸軍攻謙諸將佐皆曰今遠出討謙其勝難必苟林近在江津伺人動静【伺相吏翻】若來攻城宗之未必能固脱有蹉跌【蹉倉何翻跌徒結翻】大事去矣道規曰苟林愚懦無它奇計以吾去未遠必不敢向城吾今取謙往至便克沈疑之間已自還返【沈持林翻沈吟不決也還音旋】謙敗則林破膽豈暇得來且宗之獨守何為不支數日乃馳往攻謙水陸齊進謙等大陳舟師兼以步騎戰於枝江檀道濟先進陷陳【陷陳讀曰陣】謙等大敗謙單舸奔苟林道規追斬之還至涌口【水經注涌水自夏水南通於江謂之涌口春秋左氏傳所謂閻敖游涌而逸者也在江陵城東杜預曰涌水在南郡華容縣涌音勇】討林林走道規遣諮議參軍臨淮劉遵帥衆追之初謙至枝江江陵士民皆與謙書言城内虛實欲為内應至是檢得之道規悉焚不視衆於是大安 江州刺史庾悦以鄱陽太守虞丘進為前驅屢破盧循兵進據豫章絶循糧道九月劉遵斬苟林于巴陵桓石綏因循入寇起兵洛口【水經注漢水過魏興安陽縣又東至灙城南與洛谷水合水北出洛谷谷北通長安其水南流注漢水所謂洛口也】自號荆州刺史徵陽令王天恩自號梁州刺史【徵陽當作微陽晉地里志微陽縣屬上庸郡沈約曰魏立建始縣晉武帝改曰微陽周武王之伐紂庸蜀羌髳微盧彭濮八國從之竊意微陽縣蓋因古微國而得名而史無其據】襲據西城梁州刺史傅韶遣其子魏興太守弘之討石綏等皆斬之桓氏遂滅韶暢之孫也 西秦王乾歸攻秦略陽南安隴西諸郡皆克之徙民二萬五千戶於苑川及枹罕【枹音膚】 甲寅葬魏主珪於盛樂金陵諡曰宣武廟號烈祖【宋高祖永初元年魏改諡珪曰道武皇帝】 劉毅固求追討盧循長史王誕密言於劉裕曰毅既喪敗不宜復使立功【喪息浪翻復扶又翻下無復同】裕從之冬十月裕帥兖州刺史劉藩寧朔將軍檀韶冠軍將軍劉敬宣等南擊盧循【帥讀曰率下同冠古玩翻】以劉毅監太尉留府後事皆委焉【監音工衘翻】癸巳裕建康 徐道覆率衆三萬趣江陵奄至破冢【破冢在江岸之東趣七喻翻】時魯宗之已還襄陽追召不及人情大震或傳循已平京邑遣道覆來為刺史江漢士民感劉道規焚書之恩無復貳志【復扶又翻】道規使劉遵别為遊軍自拒道覆於豫章口前驅失利遵自外横擊大破之斬首萬餘級赴水死者殆盡道覆單舸走還湓口【湓蒲奔翻】初道規使遵為遊軍衆咸以為彊敵在前唯患衆少不應分割見力置無用之地【見賢遍翻】及破道覆卒得遊軍之力【卒子恤翻】衆心乃服 鮮卑僕渾羌句豈輸報鄧若等帥戶二萬降于西秦【鮮卑有僕渾部句豈輸報鄧若則羌種也句古侯翻】 王仲德等聞劉裕大軍且至進攻范崇民於南陵崇民戰艦夾屯西岸【艦戶黯翻】十一月劉鍾自行覘賊【覘丑亷翻又丑艷翻】天霧賊鉤得其舸鍾因帥左右攻艦戶【艦戶今舟人謂之馬門】賊遽閉戶拒之鍾乃徐還與仲德共攻崇民崇民走【崇民走則裕可徑進循失湖口之險】 癸丑益州刺史鮑陋卒謙道福陷巴東殺守將温祚時延祖【温祚本巴東太守時延祖自劉敬宣黄虎之退皆屯巴東將即亮翻】 盧循兵守廣州者不以海道為虞庚戌孫處乘海奄至會大霧四面攻之即日拔其城處撫其舊民戮循親黨勒兵謹守分遣沈田子等擊嶺表諸郡劉裕軍雷池盧循揚聲不攻雷池當乘流徑下裕知其欲戰十二月己卯進軍大雷【杜佑曰晉大雷戍舒州望江縣是今晥口之西有雷江口即其地宋書志云望江縣西岸有大雷江自尋陽柴桑沿流三百里入江即望江縣】庚辰盧循徐道覆帥衆數萬塞江而下【塞悉則翻】前後莫見舳艫之際【舳音逐艫音盧】裕悉出輕艦帥衆軍齊力擊之又分步騎屯於西岸先備火具裕以勁弩射循軍【射而亦翻】因風水之勢以蹙之循艦悉泊西岸岸上軍投火焚之烟炎漲天【烟與煙同炎讀曰燄】循兵大敗走還尋陽將趣豫章【趣七喻翻】乃悉力柵斷左里【左里以其地在章江之左故名杜佑曰左里即江州尋陽縣彭蠡湖口斷丁管翻】丙申裕軍至左里不得進裕麾兵將戰所執麾竿折幡沈于水衆並怪懼裕笑曰往年覆舟之戰【謂討桓玄與桓謙等戰時也折而設翻沈持林翻】幡竿亦折今者復然【復扶又翻】賊必破矣即攻柵而進循兵雖殊死戰弗能禁循單舸走【舸古我翻】所殺及投水死者凡萬餘人納其降附宥其逼略【降戶江翻】遣劉藩孟懷玉輕軍追之循收散卒尚有數千人徑還番禺【番禺音潘愚】道覆走保始興裕板建威將軍禇裕之行廣州刺史裕之裒之曾孫也【褚裒崇德太后之父裒蒲侯翻】裕還建康劉毅惡劉穆之每從容與裕言穆之權太重【惡烏路翻從干容翻】裕益親任之燕廣川公萬泥上谷公乳陳自以宗室有大功【慕容熙之死萬泥乳陳皆有功】謂當入為公輔燕王跋以二藩任重久而弗徵【跋以萬泥為幽平二州牧鎮肥如乳陳為并青二州牧鎮白狼】二人皆怨是歲乳陳密遣人告萬泥曰乳陳有至謀願與叔父圖之萬泥遂奔白狼與乳陳俱叛跋遣汲郡公弘與張興將步騎二萬討之弘先遣使諭以禍福萬泥欲降乳陳不可【將即亮翻騎奇寄翻使疏吏翻降戶江翻下同】興謂弘曰賊明日出戰今夜必來驚我營宜為之備弘乃密令人課草十束畜火伏兵以待之是夜乳陳果遣壯士千餘人來斫營衆火俱起伏兵邀擊俘斬無遺萬泥乳陳懼而出降弘皆斬之跋以范陽公素弗為大司馬改封遼西公弘為驃騎大將軍改封中山公【驃匹妙翻騎奇寄翻】<br />
<br />
  資治通鑑卷一百十五  <br>
   </div> 

<script src="/search/ajaxskft.js"> </script>
 <div class="clear"></div>
<br>
<br>
 <!-- a.d-->

 <!--
<div class="info_share">
</div> 
-->
 <!--info_share--></div>   <!-- end info_content-->
  </div> <!-- end l-->

<div class="r">   <!--r-->



<div class="sidebar"  style="margin-bottom:2px;">

 
<div class="sidebar_title">工具类大全</div>
<div class="sidebar_info">
<strong><a href="http://www.guoxuedashi.com/lsditu/" target="_blank">历史地图</a></strong>  
<a href="http://www.880114.com/" target="_blank">英语宝典</a>  
<a href="http://www.guoxuedashi.com/13jing/" target="_blank">十三经检索</a> 
<br><strong><a href="http://www.guoxuedashi.com/gjtsjc/" target="_blank">古今图书集成</a></strong> 
<a href="http://www.guoxuedashi.com/duilian/" target="_blank">对联大全</a> <strong><a href="http://www.guoxuedashi.com/xiangxingzi/" target="_blank">象形文字典</a></strong> 

<br><a href="http://www.guoxuedashi.com/zixing/yanbian/">字形演变</a>  <strong><a href="http://www.guoxuemi.com/hafo/" target="_blank">哈佛燕京中文善本特藏</a></strong>
<br><strong><a href="http://www.guoxuedashi.com/csfz/" target="_blank">丛书&方志检索器</a></strong> <a href="http://www.guoxuedashi.com/yqjyy/" target="_blank">一切经音义</a>  

<br><strong><a href="http://www.guoxuedashi.com/jiapu/" target="_blank">家谱族谱查询</a></strong>  <strong><a href="http://shufa.guoxuedashi.com/sfzitie/" target="_blank">书法字帖欣赏</a></strong> 
<br>

</div>
</div>


<div class="sidebar" style="margin-bottom:0px;">

<font style="font-size:22px;line-height:32px">QQ交流群9:489193090</font>


<div class="sidebar_title">手机APP 扫描或点击</div>
<div class="sidebar_info">
<table>
<tr>
	<td width=160><a href="http://m.guoxuedashi.com/app/" target="_blank"><img src="/img/gxds-sj.png" width="140"  border="0" alt="国学大师手机版"></a></td>
	<td>
<a href="http://www.guoxuedashi.com/download/" target="_blank">app软件下载专区</a><br>
<a href="http://www.guoxuedashi.com/download/gxds.php" target="_blank">《国学大师》下载</a><br>
<a href="http://www.guoxuedashi.com/download/kxzd.php" target="_blank">《汉字宝典》下载</a><br>
<a href="http://www.guoxuedashi.com/download/scqbd.php" target="_blank">《诗词曲宝典》下载</a><br>
<a href="http://www.guoxuedashi.com/SiKuQuanShu/skqs.php" target="_blank">《四库全书》下载</a><br>
</td>
</tr>
</table>

</div>
</div>


<div class="sidebar2">
<center>


</center>
</div>

<div class="sidebar"  style="margin-bottom:2px;">
<div class="sidebar_title">网站使用教程</div>
<div class="sidebar_info">
<a href="http://www.guoxuedashi.com/help/gjsearch.php" target="_blank">如何在国学大师网下载古籍?</a><br>
<a href="http://www.guoxuedashi.com/zidian/bujian/bjjc.php" target="_blank">如何使用部件查字法快速查字?</a><br>
<a href="http://www.guoxuedashi.com/search/sjc.php" target="_blank">如何在指定的书籍中全文检索?</a><br>
<a href="http://www.guoxuedashi.com/search/skjc.php" target="_blank">如何找到一句话在《四库全书》哪一页?</a><br>
</div>
</div>


<div class="sidebar">
<div class="sidebar_title">热门书籍</div>
<div class="sidebar_info">
<a href="/so.php?sokey=%E8%B5%84%E6%B2%BB%E9%80%9A%E9%89%B4&kt=1">资治通鉴</a> <a href="/24shi/"><strong>二十四史</strong></a>&nbsp; <a href="/a2694/">野史</a>&nbsp; <a href="/SiKuQuanShu/"><strong>四库全书</strong></a>&nbsp;<a href="http://www.guoxuedashi.com/SiKuQuanShu/fanti/">繁体</a>
<br><a href="/so.php?sokey=%E7%BA%A2%E6%A5%BC%E6%A2%A6&kt=1">红楼梦</a> <a href="/a/1858x/">三国演义</a> <a href="/a/1038k/">水浒传</a> <a href="/a/1046t/">西游记</a> <a href="/a/1914o/">封神演义</a>
<br>
<a href="http://www.guoxuedashi.com/so.php?sokeygx=%E4%B8%87%E6%9C%89%E6%96%87%E5%BA%93&submit=&kt=1">万有文库</a> <a href="/a/780t/">古文观止</a> <a href="/a/1024l/">文心雕龙</a> <a href="/a/1704n/">全唐诗</a> <a href="/a/1705h/">全宋词</a>
<br><a href="http://www.guoxuedashi.com/so.php?sokeygx=%E7%99%BE%E8%A1%B2%E6%9C%AC%E4%BA%8C%E5%8D%81%E5%9B%9B%E5%8F%B2&submit=&kt=1"><strong>百衲本二十四史</strong></a>  <a href="http://www.guoxuedashi.com/so.php?sokeygx=%E5%8F%A4%E4%BB%8A%E5%9B%BE%E4%B9%A6%E9%9B%86%E6%88%90&submit=&kt=1"><strong>古今图书集成</strong></a>
<br>

<a href="http://www.guoxuedashi.com/so.php?sokeygx=%E4%B8%9B%E4%B9%A6%E9%9B%86%E6%88%90&submit=&kt=1">丛书集成</a> 
<a href="http://www.guoxuedashi.com/so.php?sokeygx=%E5%9B%9B%E9%83%A8%E4%B8%9B%E5%88%8A&submit=&kt=1"><strong>四部丛刊</strong></a>  
<a href="http://www.guoxuedashi.com/so.php?sokeygx=%E8%AF%B4%E6%96%87%E8%A7%A3%E5%AD%97&submit=&kt=1">說文解字</a> <a href="http://www.guoxuedashi.com/so.php?sokeygx=%E5%85%A8%E4%B8%8A%E5%8F%A4&submit=&kt=1">三国六朝文</a>
<br><a href="http://www.guoxuedashi.com/so.php?sokeytm=%E6%97%A5%E6%9C%AC%E5%86%85%E9%98%81%E6%96%87%E5%BA%93&submit=&kt=1"><strong>日本内阁文库</strong></a> <a href="http://www.guoxuedashi.com/so.php?sokeytm=%E5%9B%BD%E5%9B%BE%E6%96%B9%E5%BF%97%E5%90%88%E9%9B%86&ka=100&submit=">国图方志合集</a> <a href="http://www.guoxuedashi.com/so.php?sokeytm=%E5%90%84%E5%9C%B0%E6%96%B9%E5%BF%97&submit=&kt=1"><strong>各地方志</strong></a>

</div>
</div>


<div class="sidebar2">
<center>

</center>
</div>
<div class="sidebar greenbar">
<div class="sidebar_title green">四库全书</div>
<div class="sidebar_info">

《四库全书》是中国古代最大的丛书,编撰于乾隆年间,由纪昀等360多位高官、学者编撰,3800多人抄写,费时十三年编成。丛书分经、史、子、集四部,故名四库。共有3500多种书,7.9万卷,3.6万册,约8亿字,基本上囊括了古代所有图书,故称“全书”。<a href="http://www.guoxuedashi.com/SiKuQuanShu/">详细>>
</a>

</div> 
</div>

</div>  <!--end r-->

</div>
<!-- 内容区END --> 

<!-- 页脚开始 -->
<div class="shh">

</div>

<div class="w1180" style="margin-top:8px;">
<center><script src="http://www.guoxuedashi.com/img/plus.php?id=3"></script></center>
</div>
<div class="w1180 foot">
<a href="/b/thanks.php">特别致谢</a> | <a href="javascript:window.external.AddFavorite(document.location.href,document.title);">收藏本站</a> | <a href="#">欢迎投稿</a> | <a href="http://www.guoxuedashi.com/forum/">意见建议</a> | <a href="http://www.guoxuemi.com/">国学迷</a> | <a href="http://www.shuowen.net/">说文网</a><script language="javascript" type="text/javascript" src="https://js.users.51.la/17753172.js"></script><br />
  Copyright &copy; 国学大师 古典图书集成 All Rights Reserved.<br>
  
  <span style="font-size:14px">免责声明:本站非营利性站点,以方便网友为主,仅供学习研究。<br>内容由热心网友提供和网上收集,不保留版权。若侵犯了您的权益,来信即刪。scp168@qq.com</span>
  <br />
ICP证:<a href="http://www.beian.miit.gov.cn/" target="_blank">鲁ICP备19060063号</a></div>
<!-- 页脚END --> 
<script src="http://www.guoxuedashi.com/img/plus.php?id=22"></script>
<script src="http://www.guoxuedashi.com/img/tongji.js"></script>

</body>
</html>
