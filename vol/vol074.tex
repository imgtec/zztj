<!DOCTYPE html PUBLIC "-//W3C//DTD XHTML 1.0 Transitional//EN" "http://www.w3.org/TR/xhtml1/DTD/xhtml1-transitional.dtd">
<html xmlns="http://www.w3.org/1999/xhtml">
<head>
<meta http-equiv="Content-Type" content="text/html; charset=utf-8" />
<meta http-equiv="X-UA-Compatible" content="IE=Edge,chrome=1">
<title>資治通鑒_75-資治通鑑卷七十四_75-資治通鑑卷七十四</title>
<meta name="Keywords" content="資治通鑒_75-資治通鑑卷七十四_75-資治通鑑卷七十四">
<meta name="Description" content="資治通鑒_75-資治通鑑卷七十四_75-資治通鑑卷七十四">
<meta http-equiv="Cache-Control" content="no-transform" />
<meta http-equiv="Cache-Control" content="no-siteapp" />
<link href="/img/style.css" rel="stylesheet" type="text/css" />
<script src="/img/m.js?2020"></script> 
</head>
<body>
 <div class="ClassNavi">
<a  href="/24shi/">二十四史</a> | <a href="/SiKuQuanShu/">四库全书</a> | <a href="http://www.guoxuedashi.com/gjtsjc/"><font  color="#FF0000">古今图书集成</font></a> | <a href="/renwu/">历史人物</a> | <a href="/ShuoWenJieZi/"><font  color="#FF0000">说文解字</a></font> | <a href="/chengyu/">成语词典</a> | <a  target="_blank"  href="http://www.guoxuedashi.com/jgwhj/"><font  color="#FF0000">甲骨文合集</font></a> | <a href="/yzjwjc/"><font  color="#FF0000">殷周金文集成</font></a> | <a href="/xiangxingzi/"><font color="#0000FF">象形字典</font></a> | <a href="/13jing/"><font  color="#FF0000">十三经索引</font></a> | <a href="/zixing/"><font  color="#FF0000">字体转换器</font></a> | <a href="/zidian/xz/"><font color="#0000FF">篆书识别</font></a> | <a href="/jinfanyi/">近义反义词</a> | <a href="/duilian/">对联大全</a> | <a href="/jiapu/"><font  color="#0000FF">家谱族谱查询</font></a> | <a href="http://www.guoxuemi.com/hafo/" target="_blank" ><font color="#FF0000">哈佛古籍</font></a> 
</div>

 <!-- 头部导航开始 -->
<div class="w1180 head clearfix">
  <div class="head_logo l"><a title="国学大师官网" href="http://www.guoxuedashi.com" target="_blank"></a></div>
  <div class="head_sr l">
  <div id="head1">
  
  <a href="http://www.guoxuedashi.com/zidian/bujian/" target="_blank" ><img src="http://www.guoxuedashi.com/img/top1.gif" width="88" height="60" border="0" title="部件查字,支持20万汉字"></a>


<a href="http://www.guoxuedashi.com/help/yingpan.php" target="_blank"><img src="http://www.guoxuedashi.com/img/top230.gif" width="600" height="62" border="0" ></a>


  </div>
  <div id="head3"><a href="javascript:" onClick="javascript:window.external.AddFavorite(window.location.href,document.title);">添加收藏</a>
  <br><a href="/help/setie.php">搜索引擎</a>
  <br><a href="/help/zanzhu.php">赞助本站</a></div>
  <div id="head2">
 <a href="http://www.guoxuemi.com/" target="_blank"><img src="http://www.guoxuedashi.com/img/guoxuemi.gif" width="95" height="62" border="0" style="margin-left:2px;" title="国学迷"></a>
  

  </div>
</div>
  <div class="clear"></div>
  <div class="head_nav">
  <p><a href="/">首页</a> | <a href="/ShuKu/">国学书库</a> | <a href="/guji/">影印古籍</a> | <a href="/shici/">诗词宝典</a> | <a   href="/SiKuQuanShu/gxjx.php">精选</a> <b>|</b> <a href="/zidian/">汉语字典</a> | <a href="/hydcd/">汉语词典</a> | <a href="http://www.guoxuedashi.com/zidian/bujian/"><font  color="#CC0066">部件查字</font></a> | <a href="http://www.sfds.cn/"><font  color="#CC0066">书法大师</font></a> | <a href="/jgwhj/">甲骨文</a> <b>|</b> <a href="/b/4/"><font  color="#CC0066">解密</font></a> | <a href="/renwu/">历史人物</a> | <a href="/diangu/">历史典故</a> | <a href="/xingshi/">姓氏</a> | <a href="/minzu/">民族</a> <b>|</b> <a href="/mz/"><font  color="#CC0066">世界名著</font></a> | <a href="/download/">软件下载</a>
</p>
<p><a href="/b/"><font  color="#CC0066">历史</font></a> | <a href="http://skqs.guoxuedashi.com/" target="_blank">四库全书</a> |  <a href="http://www.guoxuedashi.com/search/" target="_blank"><font  color="#CC0066">全文检索</font></a> | <a href="http://www.guoxuedashi.com/shumu/">古籍书目</a> | <a   href="/24shi/">正史</a> <b>|</b> <a href="/chengyu/">成语词典</a> | <a href="/kangxi/" title="康熙字典">康熙字典</a> | <a href="/ShuoWenJieZi/">说文解字</a> | <a href="/zixing/yanbian/">字形演变</a> | <a href="/yzjwjc/">金 文</a> <b>|</b>  <a href="/shijian/nian-hao/">年号</a> | <a href="/diming/">历史地名</a> | <a href="/shijian/">历史事件</a> | <a href="/guanzhi/">官职</a> | <a href="/lishi/">知识</a> <b>|</b> <a href="/zhongyi/">中医中药</a> | <a href="http://www.guoxuedashi.com/forum/">留言反馈</a>
</p>
  </div>
</div>
<!-- 头部导航END --> 
<!-- 内容区开始 --> 
<div class="w1180 clearfix">
  <div class="info l">
   
<div class="clearfix" style="background:#f5faff;">
<script src='http://www.guoxuedashi.com/img/headersou.js'></script>

</div>
  <div class="info_tree"><a href="http://www.guoxuedashi.com">首页</a> > <a href="/SiKuQuanShu/fanti/">四库全书</a>
 > <h1>资治通鉴</h1> <!--         下载:【右键另存为】即可 --></div>
  <div class="info_content zj clearfix">
  
<div class="info_txt clearfix" id="show">
<center style="font-size:24px;">75-資治通鑑卷七十四</center>
    資治通鑑卷七十四   宋 司馬光 撰<br />
<br />
  胡三省 音註<br />
<br />
  魏紀六【起著雍敦牂盡旃蒙赤奮若凡八年】<br />
<br />
  烈祖明皇帝下<br />
<br />
  景初二年春正月帝召司馬懿於長安使將兵四萬討遼東【討公孫淵也留司馬懿於長安以備蜀也諸葛亮死乃敢召之遠略將即亮翻】議臣或以為四萬兵多役費難供【議臣當時謀議之臣也】帝曰四千里征伐【續漢志遼東郡在洛陽東北三千六百里】雖云用奇亦當任力不當稍計役費也帝謂懿曰公孫淵將何計以待君對曰淵棄城豫走上計也據遼東拒大軍其次也【遼東當作遼水】坐守襄平此成禽耳【襄平縣漢遼東郡治所公孫淵所都】帝曰然則三者何出對曰唯明智能審量彼我【量音良】乃豫有所割棄此既非淵所及又謂今往孤遠【言孤軍遠征也】不能支久必先拒遼水後守襄平也帝曰還往幾日對曰往百日攻百日還百日以六十日為休息如此一年足矣公孫淵聞之復遣使稱臣求救於吳吳人欲戮其使【欲報張彌許晏之忿也事見七十二卷青龍元年復扶又翻使疏吏翻】羊衜曰【衜古道字】不可是肆匹夫之怒而捐霸王之計也不如因而厚之遣奇兵潛往以要其成【要一遥翻】若魏代不克而我軍遠赴是恩結遐夷義形萬里若兵連不解首尾離隔則我虜其傍郡驅略而歸亦足以致天之罰報雪曩事矣吳主曰善乃大勒兵謂淵使曰請俟後問當從簡書【左傳狄伐邢管敬仲言於齊侯曰詩云豈不懷歸畏此簡書簡書同惡相恤之謂也請救邢以從簡書】必與弟同休戚【淵遣使謝吳自稱燕王求為兄弟之國故權因而稱之為弟】又曰司馬懿所向無前深為弟憂之【此晉史臣為此語耳權必無此言為于偽翻】帝問於護軍將軍蔣濟曰孫權其救遼東乎濟曰彼知官備已固【魏晉之間謂國家為官】利不可得深入則非力所及淺入則勞而無獲權雖子弟在危猶將不動况異域之人兼以往者之辱乎【亦謂斬張彌許晏也】今所以外揚此聲者譎其行人【譎古穴翻詐也】疑之於我我之不克冀其折節事已耳然沓渚之間去淵尚遠若大軍相守事不速决則權之淺規或得輕兵掩襲未可測也【淺規謂規圖淺攻不敢深入吳君臣之為謀已不逃蔣濟所料矣】 帝問吏部尚書盧毓誰可為司徒者毓薦處士管寧【處昌呂翻】帝不能用更問其次對曰敦篤至行則太中大夫韓暨【行下孟翻】亮直清方則司隸校尉崔林貞固純粹則太常常林二月癸卯以韓暨為司徒 漢主立皇后張氏前后之妹也立王貴人子璿為皇太子【璿旬緣翻】瑶為安定王大司農河南孟光問太子讀書及情性好尚於祕書郎卻正【東漢以馬融為祕書郎詣東觀典校書祕書郎蓋自融始好呼到翻下同卻綺戟翻】正曰奉親䖍恭夙夜匪懈有古世子之風【懈古隘翻】接待羣僚舉動出於仁恕光曰如君所道皆家戶所有耳【謂其才行不逾中人也】吾今所問欲知其權略智謀何如也正曰世子之道在於承志竭歡【承志謂承君父之志竭歡謂左右就養承顔順色以盡親之歡】既不得妄有施為智謀藏於胸懷權略應時而發此之有無焉可豫知也【焉於䖍翻】光知正慎宜【慎宜者謹言語擇其所宜言乃言也】不為放談乃曰吾好直言無所回避今天下未定智意為先智意自然不可力彊致也【彊其兩翻】儲君讀書寧當傚吾等竭力博識以待訪問如博士探策講試以求爵位邪【按漢書音義作簡策難問例置案上在試者意投射取而答之謂之射策即探策也若録政化得失顯而問之謂之對策探吐南翻】當務其急者正深謂光言為然正儉之孫也【儉為益州刺史漢靈帝中平五年為盜賊所殺】 吳人鑄當千大錢【杜佑曰孫權赤烏元年鑄一當千大錢徑一寸四分重十六銖】 夏四月庚子南鄉恭侯韓暨卒 庚戌大赦 六月司馬懿軍至遼東公孫淵使大將軍卑衍楊祚【姓譜卑卑耳國之後或云鮮卑之後蔡邕胡太傅碑有太傅掾雁門卑登】將步騎數萬屯遼隧圍塹二十餘里 【考異曰晉宣紀云南北六七十里今從淵傳】諸將欲擊之懿曰賊所以堅壁欲老吾兵也今攻之正墮其計且賊大衆在此其巢窟空虚直指襄平破之必矣乃多張旗幟欲出其南【幟昌志翻】衍等盡銳趣之懿潛濟水出其北直趣襄平【趣七喻翻】衍等恐引兵夜走諸軍進至首山【首山在襄平西南】淵復使衍等逆戰【復扶又翻下同】懿擊大破之遂進圍襄平秋七月大霖雨遼水暴漲運船自遼口徑至城下【遼口遼水津渡之口也】雨月餘不止平地水數尺三軍恐欲移營懿令軍中敢有言徙者斬都督令史張静犯令斬之【晉職官志魏制諸公加兵者置都督令史一人】軍中乃定賊恃水樵牧自若諸將欲取之懿皆不聽司馬陳珪曰昔攻上庸八部俱進晝夜不息故能一旬之半拔堅城斬孟達【事見七十一卷太和二年】今者遠來而更安緩愚竊惑焉懿曰孟達衆少而食支一年將士四倍於達而糧不淹月【淹留也言所留之糧不支一月也】以一月圖一年安可不速以四擊一正令失半而克猶當為之是以不計死傷與糧競也【競争也懿之語珪猶有廋辭蓋其急攻孟達豈特與糧競哉懼吳蜀救兵至耳】今賊衆我寡賊饑我飽水雨乃爾【爾如此也】功力不設雖當促之亦何所為自發京師不憂賊攻但恐賊走今賊糧垂盡而圍落未合掠其牛馬抄其樵采【抄楚交翻】此故驅之走也夫兵者詭道善因事變【言善兵者能因事而變化也】賊憑衆恃雨故雖饑困未肯束手當示無能以安之取小利以驚之非計也【懿知淵可禽欲以全取之】朝廷聞師遇雨咸欲罷兵帝曰司馬懿臨危制變禽淵可計日待也雨霽懿乃合圍作土山地道楯櫓鉤衝【楯干也攻城之士以扞蔽其身櫓樓車登之以望城中鉤鉤梯也所以鉤引上城者衝衝車也以衝城】晝夜攻之矢石如雨淵窘急【窘巨隕翻】糧盡人相食死者甚多其將楊祚等降八月淵使相國王建御史大夫柳甫請解圍却兵當君臣面縳懿命斬之檄告淵曰楚鄭列國而鄭伯猶肉袒牽羊迎之【左傳楚莊王圍鄭克之入自皇門至于逵路鄭伯肉袒牽羊以逆】孤天子上公【漢太傅位上公懿時為太尉而自謂上公以太尉於三公為上也】而建等欲孤解圍退舍豈得禮邪二人老耄傳言失指已相為斬之【為于偽翻】若意有未已可更遣年少有明决者來【少詩照翻】淵復遣侍中衛演乞克日送任【送任謂送質子也復扶又翻】懿謂演曰軍事大要有五能戰當戰不能戰當守不能守當走餘二事但有降與死耳【降戶江翻】汝不肯面縛此為决就死也不須送任壬午襄平潰淵與子脩將數百騎突圍東南走大兵急擊之斬淵父子於梁水之上【班志遼東郡遼陽縣注云大梁水西南至遼陽入遼水水經注小遼水出玄莬高句麗縣遼山西南流逕襄平縣入大梁水水出北塞外西南流而入于遼水】懿既入城誅其公卿以下及兵民七千餘人築為京觀【杜預曰積尸封土於其上謂之京觀觀古玩翻】遼東帶方樂浪玄菟四郡皆平【漢帶方縣屬樂浪郡公孫氏分立郡陳夀曰建安中公孫康分屯有以南荒地為帶方郡倭韓諸國羈屬焉樂浪音洛琅莬同都翻】淵之將反也將軍綸直賈範等苦諫【綸姓直名其先以邑為姓】淵皆殺之懿乃封直等之墓顯其遺嗣釋淵叔父恭之囚【淵囚恭事見七十一卷太和二年】中國人欲還舊鄉者恣聽之遂班師【司馬懿與諸葛亮相守閉壁若無能為者及討公孫淵智計横出鄙語有云棊逢敵手難藏行其是之謂乎】初淵兄晃為恭任子在洛陽先淵未反時數陳其變【先悉薦翻數所角翻下同】欲令國家討淵及淵謀逆帝不忍市斬欲就獄殺之【晃數陳淵之必反非同逆者也帝欲殺之以絶其類刑之於市則無名故欲就獄殺之】廷尉高柔上疏曰臣竊聞晃先數自歸陳淵禍萌雖為凶族原心可恕夫仲尼亮司馬牛之憂【司馬牛宋司馬桓魋之弟也魋凶惡牛憂之曰人皆有兄弟我獨亡謂魋之積惡將死亡無日】祁奚明叔向之過【左傳晉人逐欒盈殺羊舌虎囚虎兄叔向祁奚見范宣子曰管蔡為戮周公右王若之何以虎也棄社稷宣子言諸公而免之】在昔之美義也臣以為晃信有言宜貸其死苟自無言便當市斬今進不赦其命退不彰其罪閉著囹圄使自引分【著直略翻引分即引决也】四方觀國或疑此舉也帝不聽竟遣使齎金屑飲晃及其妻子【飲於㺔翻】賜以棺衣殯殮於宅【宅晃所居者殮力贍翻】 九月吳改元赤烏【權以赤烏集於殿前改元】 吳步夫人卒初吳主為討虜將軍在吳娶吳郡徐氏太子登所生庶賤吳主令徐氏母養之徐氏妬故無寵及吳主西徙【謂自吳而西徙都武昌也】徐氏留處吳而臨淮步夫人寵冠後庭【步夫人隲之族也處昌呂翻冠古玩翻】吳主欲立為皇后而羣臣議在徐氏吳主依違者十餘年【依違不决也】會步氏卒羣臣奏追贈皇后印綬【綬音受】徐氏竟廢卒於吳吳主使中書郎呂壹典校諸官府及州郡文書壹因<br />
<br />
  此漸作威福深文巧詆排陷無辜毁短大臣纎介必聞太子登數諫【數所角翻下同】吳主不聽羣臣莫敢復言【復扶又翻】皆畏之側目壹誣白故江夏太守刁嘉謗訕國政【訕山諫翻】吳主怒收嘉繫獄驗問時同坐人皆畏怖壹【其時與嘉同坐者坐徂卧翻】並言聞之侍中北海是儀獨云無聞【是姓儀名儀本姓氏孔融嘲儀以氏字為民上無頭遂改姓是】遂見窮詰累日詔旨轉厲羣臣為之屏息【為于偽翻屛必郢翻屛息不敢舒氣也】儀曰今刀鋸已在臣頸臣何敢為嘉隱諱自取夷滅為不忠之鬼顧以聞知當有本末據實答問辭不傾移吳主遂舍之【舍讀曰捨】嘉亦得免上大將軍陸遜太常潘濬憂壹亂國每言之輒流涕壹白丞相顧雍過失吳主怒詰責雍【詰去吉翻】黃門侍郎謝厷語次問壹【厷與宏同乎萌翻】顧公事何如壹曰不能佳厷又問若此公免退誰當代之壹未答厷曰得無潘太常得之乎壹曰君語近之也【近其靳翻】厷曰潘太常常切齒於君但道無因耳【謂欲奏舉其罪而非太常之職故其道無因也】今日代顧公恐明日便擊君矣【漢制丞相御史舉奏百官有罪者】壹大懼遂解散雍事潘濬求朝詣建業【濬本留武昌朝直遥翻】欲盡辭極諫至聞太子登已數言之【至建業而知太子數言壹事】而不見從濬乃大請百寮欲因會手刃殺壹以身當之【以身當擅殺之罪】為國除患【為于偽翻下同】壹密聞知稱疾不行西陵督步隲上疏曰顧雍陸遜潘濬志在竭誠寢食不寧念欲安國利民建久長之計可謂心膂股肱社稷之臣矣宜各委任不使他官監其所司課其殿最【監古衘翻殿丁甸翻賢曰殿軍後也課居後也最凡要之先也課居先也】此三臣思慮不到則已豈敢欺負所天乎【君天也】左將軍朱據部曲應受三萬緡工王遂詐而受之壹疑據實取考問主者【主者據軍吏也】死於杖下據哀其無辜厚棺斂之【棺古玩翻斂力驗翻】壹又表據吏為據隱故厚其殯吳主數責問據據無以自明藉草待罪數日典軍吏劉助覺言王遂所取【劉助覺其事而言之】吳主大感悟曰朱據見枉况吏民乎乃窮治壹罪【治直之翻】賞助百萬丞相雍至廷尉斷獄【斷丁亂翻】壹以囚見雍和顔色問其辭狀臨出又謂壹曰君意得無欲有所道乎【道言也】壹叩頭無言時尚書郎懷叙【懷姓叙名姓譜無懷氏之後】面詈辱壹雍責叙曰官有正法何至於此有司奏壹大辟【辟毗亦翻】或以為宜加焚裂用彰元惡【殷紂用炮烙之刑項羽燒殺紀信漢武帝焚蘇文於横橋然未以為刑名也王莽作焚如之刑後世不復遵用裂謂車裂古之轘刑】吳主以訪中書令會稽闞澤【會古外翻闞姓也左傳齊有大夫闞止】澤曰盛明之世不宜復有此刑【復扶又翻下同】吳主從之壹既伏誅吳主使中書郎袁禮告謝諸大將因問時事所當損益禮還復有詔責諸葛瑾步隲朱然呂岱等曰袁禮還云與子瑜子山義封定公相見【諸葛瑾字子瑜步隲字子山朱然字義封呂岱字定公瑾渠吝翻隲職日翻】並咨以時事當有所先後【謂時事所當行何者為先何者為後也】各自以不掌民事不肯便有所陳悉推之伯言承明【推吐雷翻陸遜字伯言潘濬字承明】伯言承明見禮泣涕懇惻辭旨辛苦至乃懷執危怖有不自安之心【怖普布翻】聞之悵然深自刻怪【刻怪也】何者夫惟聖人能無過行【行下孟翻】明者能自見耳人之舉厝何能悉中【中竹仲翻】獨當已有以傷拒衆意忽不自覺故諸君有嫌難耳不爾何緣乃至於此乎與諸君從事自少至長髪有二色【二色謂班白也少詩照翻長知兩翻】以謂表裏足以明露公私分計足用相保【分扶問翻】義雖君臣恩猶骨肉榮福喜戚相與共之忠無匿情智無遺計事統是非【言行事是則君臣同其是非則同其非也】諸君豈得從容而已哉【從于容翻】同船濟水將誰與易【易如字】齊桓有善管子未嘗不歎有過未嘗不諫諫而不得終諫不止今孤自省無桓公之德【省悉景翻】而諸君諫諍未出於口仍執嫌難以此言之孤於齊桓良優未知諸君於管子何如耳【下之於上不從其令而從其意孫權自謂優於齊桓而責其臣以管子使吳誠有管子亦不敢盡言於權觀諸陸遜可見矣】 冬十一月壬午以司空衛臻為司徒司隸校尉崔林為司空 十二月漢蔣琬出屯漢中 乙丑帝不豫 辛巳立郭夫人為皇后初太祖為魏公以贊令劉放【酇縣漢屬沛郡王莽改曰贊治魏分屬譙郡或曰贊相也凡出令使之贊相因以為官名蓋魏武霸府所置也】參軍事孫資皆為祕書郎文帝即位更名祕書曰中書以放為監資為令遂掌機密【漢桓帝延熹二年置祕書監魏武為魏王置祕書令丞典尚書奏事黃初初改為中書置監令中書有監令自此始自魏及晉遂為要官荀朂所謂鳳凰池也更工衡翻】帝即位尤見寵任皆加侍中光禄大夫封本縣侯【放涿郡方城人資太原中都人】是時帝親覽萬機數興軍旅【數所角翻】腹心之任皆二人管之每有大事朝臣會議常令决其是非擇而行之中護軍蔣濟上疏曰【此疏亦是濟為中護軍時所上通鑑因叙放資事而書之於此】臣聞大臣太重者國危左右太親者身蔽古之至戒也往者大臣秉事外内扇動【蓋謂文帝時也或曰謂受遺大臣也】陛下卓然自覽萬機莫不祇肅夫大臣非不忠也然威權在下則衆心慢上勢之常也陛下既已察之於大臣願無忘之於左右左右忠正遠慮未必賢於大臣至於便辟取合或能工之【便毗連翻辟讀曰僻】今外所言輒云中書雖使恭慎不敢外交但有此名猶惑世俗况實握事要日在目前儻因疲倦之間有所割制【謂因人主疲倦之時有所剖割而制斷也】衆臣見其能推移於事即亦因時而向之一有此端私招朋援臧否毁譽必有所興【否音鄙譽音余】功負賞罰必有所易【負罪也易則賞罰不當乎功罪】直道而上者或壅曲附左右者反達因微而入緣形而出意所狎信不復猜覺此宜聖智所當早聞外以經意則形際自見【言放資日在左右狎而信之不復覺其為姦非若早聞忠言自覧萬機外以示經意國事則放資之形際必呈露而不可掩矣復扶又翻見賢遍翻】或恐朝臣畏言不合而受左右之怨莫適以聞【適丁歷翻】臣竊亮陛下潛神默思公聽並觀若事有未盡於理而物有未周於用將改曲易調【調徒釣翻以琴瑟為喻也】遠與黃唐角功【角者兩兩相當也黄唐黃帝唐堯】近昭武文之績豈牽近習而已哉然人君不可悉任天下之事必當有所付若委之一臣自非周公旦之忠管夷吾之公則有弄權敗官之敝【敗補邁翻】當今柱石之士雖少至於行稱一州【少詩沼翻行戶孟翻】智效一官忠信竭命各奉其職可並驅策不使聖明之朝有專吏之名也【專吏謂專任放資】帝不聽【自此以前皆非此年事通鑑因放資患失之心以誤帝託孤之事遂書之於此以先事】及寢疾深念後事乃以武帝子燕王宇為大將軍與領軍將軍夏侯獻武衛將軍曹爽【魏制領軍將軍主中壘五校武衛等三營武衛將軍蓋領武衛營也太祖以許禇典宿衛遷武衛中郎將武衛之號自此始後有遷武衛將軍於是武衛始有將軍之號晉泰始初罷武衛將軍官】屯騎校尉曹肇驍騎將軍秦朗等對輔政爽真之子肇休之子也帝少與燕王宇善故以後事屬之【少詩照翻屬之欲翻】劉放孫資久典機任獻肇心内不平殿中有雞棲樹二人相謂曰此亦久矣其能復幾【殿中畜雞以司晨棲於樹上因謂之雞棲樹獻肇指以喻放資一言而發司馬氏簒魏之機言之不可不謹也如是夫以此觀獻肇之輕脫又何足以託孤哉復扶又翻幾居豈翻】放資懼有後害隂圖間之【間古莧翻】燕王性恭良陳誠固辭帝引放資入卧内問曰燕王正爾為【言其性恭良為事正如此也】對曰燕王實自知不堪大任故耳帝曰誰可任者時惟曹爽獨在側放資因薦爽且言宜召司馬懿與相參帝曰爽堪其事不【不讀曰否】爽流汗不能對放躡其足耳之曰【附耳語之也】臣以死奉社稷帝從放資言欲用爽懿既而中變敕停前命放資復入見說帝【復扶又翻見賢遍翻說輸芮翻】帝又從之放曰宜為手詔帝曰我困篤不能放即上牀執帝手強作之【強其兩翻】遂齎出大言曰有詔免燕王宇等官不得停省中皆流涕而出 【考異曰放傳曰宇性恭良陳誠固辭帝引見放資入卧内問曰燕王正爾為放資對曰燕王實自知不堪大任故耳帝曰曹爽可代字否放資因贊成之又添陳宜速召太尉司馬宣王帝納其言放資既出帝意復變詔止宣王勿來尋更見放資曰我自召太尉而曹肇等反使吾止之命更為詔帝獨召爽與放資俱受詔命遂免宇獻肇朗官按陳夀當晉世作魏志若言放資本惰則於時非美故遷就而為之諱也今依習鑿齒漢晉春秋郭頒世語似得其實】甲申以曹爽為大將軍帝嫌爽才弱復拜尚書孫禮為大將軍長史以佐之【為下爽出孫禮張本復扶又翻】是時司馬懿在汲【時自遼東還師次于汲也汲縣自漢以來屬河内郡】帝令給使辟邪【辟邪給使之名猶漢丞相倉頭呼為宜禄也】齎手詔召之先是燕王為帝畫計【先悉薦翻為于偽翻】以為關中事重宜遣懿便道自軹關西還長安【關中事重謂備蜀及撫安氐羌也軹縣屬河内郡賢曰軹故城在洛州濟源縣東南五代志軹關在河内郡王屋縣杜佑曰軹關在河南府濟源縣界】事已施行懿斯須得二詔前後相違疑京師有變乃疾驅入朝【朝直遥翻】<br />
<br />
  三年春正月懿至入見【見賢遍翻】帝執其手曰吾以後事屬君【見賢遍翻屬之欲翻】君與曹爽輔少子【少詩照翻】死乃可忍吾忍死待君得相見無所復恨矣【復扶又翻】乃召齊秦二王以示懿别指齊王芳謂懿曰此是也君諦視之勿誤也【諦丁計翻審也】入教齊王令前抱懿頸懿頓首流涕旦日立齊王為皇太子帝尋殂【陳夀曰年三十六歲考異曰按魏武以建安九年七月定鄴文帝始納甄后明帝應以十年生至於是年正月整三十四年耳時改正朔以故年十二月為元年正月可強名三十五年不得三十六也】帝沈毅明敏【沈持林翻】任心而行料簡功能【料音聊】屏絶浮偽【屛必郢翻】行師動衆論决大事謀臣將相咸服帝之大略性特強識雖左右小臣官簿性行名跡所履【行戶孟翻】及其父兄子弟一經耳目終不遺忘【忘巫放翻】<br />
<br />
  孫盛論曰聞之長老魏明帝天姿秀出立髪垂地口吃少言【吃居乞翻言蹇也】而沈毅好斷【沈持林翻好呼到翻斷丁亂翻】初諸公受遺輔導帝皆以方任處之【謂使曹休鎮淮南曹真鎮閧中司馬懿屯宛也處昌呂翻】政自己出優禮大臣開容善直雖犯顔極諫無所摧戮其君人之量如此其偉也然不思建德垂風不固維城之基【詩曰宗子維城此言帝猜忌宗室以亡魏】至使大權偏據社稷無衛悲夫<br />
<br />
  太子即位年八歲大赦尊皇后曰皇太后加曹爽司馬懿侍中假節鉞都督中外諸軍録尚書事【晉職官志曰持節都督無定員前漢遣使始有持節光武建武初征伐四方始權時置督軍御史事竟罷建安中魏武為相始遣大將軍督之二十一年征孫權還遣夏侯惇督二十六軍是也文帝黃初三年始置都督諸州軍事或領刺史又上軍大將軍曹真都督中外諸軍事假黄鉞則總統内外諸軍矣録尚書事漢東都諸公之重任也今爽懿既督中外諸軍又録尚書事則文武大權盡歸之矣自此迄于六朝凡權臣壹是專制國命】諸所興作宫室之役皆以遺詔罷之【曰以者非遺詔真有此指也】爽懿各領兵三千人更宿殿内【更工衡翻】爽以懿年位素高常父事之每事諮訪不敢專行【或問使爽能守此而不變可以免魏室之禍否曰猫鼠不可以同穴使爽能率此而行之亦終為懿所啖食耳】初并州刺史東平畢軌【姓譜畢本畢公高之後】及鄧颺李勝何晏丁謐【颺余章翻又余亮翻】皆有才名而急於富貴趨時附埶明帝惡其浮華【趨七喻翻惡烏路翻】皆抑而不用曹爽素與親善及輔政驟加引擢以為心腹晏進之孫謐斐之子也【何進見漢靈帝紀丁斐事見六十六卷獻帝建安十六年】晏等咸共推戴爽以為重權不可委之於人丁謐為爽畫策【為于偽翻】使爽白天子發詔轉司馬懿為太傅外以名號尊之内欲令尚書奏事先來由已得制其輕重也爽從之【為下懿族爽等張本】二月丁丑以司馬懿為太傅以爽弟羲為中領軍訓為武衛將軍彦為散騎常侍侍講【以在少帝左右令侍講說侍講之官起乎此也】其餘諸弟皆以列侯侍從【從才用翻】出入禁闥貴寵莫盛焉爽事太傅禮貌雖存而諸所興造希復由之【復扶又翻下同】爽徙吏部尚書盧毓為僕射【毓余六翻】而以何晏代之以鄧颺丁謐為尚書畢軌為司隸校尉晏等依埶用事附會者升進違忤者罷退内外望風莫敢忤旨【忤五故翻】黃門侍郎傳嘏謂爽弟羲曰何平叔外静而内躁銛巧好利不念務本【何晏字平叔銛思亷翻利也好呼到翻】吾恐必先惑子兄弟仁人將遠而朝政廢矣【遠于願翻】晏等遂與嘏不平因微事免嘏官又出盧毓為廷尉【尚書内朝官九卿外朝官故云出】畢軌又枉奏毓免官衆論多訟之乃復以為光禄勲孫禮亮直不撓爽心不便出為揚州刺史【傳嘏盧毓孫禮所以不合於曹爽者其心未背曹氏也及其甘於司馬懿則事不可言矣三子者豈本心所欲哉勢有必至事有固然也撓奴教翻】 三月以征東將軍滿寵為太尉 夏四月吳督軍使者羊衜擊遼東守將俘人民而去【果如蔣濟所料督軍使者漢官也魏黄初二年罷督軍官而吳猶仍漢制】漢蔣琬為大司馬東曹掾犍為楊戲素性簡略琬與言論時不應答或謂琬曰公與戲言而不應其慢甚矣琬曰人心不同各如其面【左傳鄭子產謂子皮曰人心不同各如其面吾豈敢謂子面如吾面乎】面從後言古人所誡【尚書舜禹君臣之相告戒其言曰汝無面從退有後言】戲欲贊吾是邪則非其本心欲反吾言則顯吾之非是以默然是戲之快也又督農楊敏嘗毁琬曰作事憒憒【督農猶魏吳之典農也憒古悔翻悶悶也】誠不及前人或以白琬主者請推治敏【治直之翻】琬曰吾實不如前人無可推也主者乞問其憒憒之狀琬曰苟其不如則事不理事不理則憒憒矣後敏坐事繫獄衆人猶懼其必死琬心無適莫【論語孔子曰君子之於天下也無適也無莫也義之與比謝顯道曰適可也莫不可也適丁歷翻】敏得免重罪【此諸葛孔明所以屬琬也】 秋七月帝始親臨朝【朝直遥翻】 八月大赦冬十月吳太常潘濬卒吳主以鎮南將軍呂岱代濬<br />
<br />
  與陸遜共領荆州文書岱時年已八十體素精勤躬親王事與遜同心協規有善相讓南土稱之 十二月吳將廖式殺臨賀太守嚴綱等【臨賀縣漢屬蒼梧郡縣臨賀水因以為名吳分立為臨賀郡唐為賀州廖力救翻今力弔翻】自稱平南將軍攻零陵桂陽揺動交州諸郡衆數萬人呂岱自表輒行星夜兼路吳主遣使追拜交州牧及遣諸將唐咨等絡繹相繼攻討一年破之斬式及其支黨郡縣悉平【當方面者當如呂岱委人以方面者當如孫權】岱復還武昌 吳都鄉侯周胤將兵千人屯公安有罪徙廬陵諸葛瑾步隲為之請吳主曰昔胤年少初無功勞橫受精兵【為于偽翻少詩照翻横戶孟翻】爵以侯將【謂既受侯爵又將兵也將即亮】翻蓋念公瑾以及於胤也而胤恃此酗淫自恣【酗吁句翻】前後告諭曾無悛改【悛丑緣翻】孤於公瑾義猶二君【二君謂諸葛瑾步隲也】樂胤成就豈有已哉【樂音洛】迫胤罪惡未宜便還且欲苦之使自知耳以公瑾之子而二君在中間苟使能改亦何患乎瑜兄子偏將軍峻卒全琮請使峻子護領其兵吳主曰昔走曹操拓有荆州皆是公瑾【事見六十五卷漢獻帝建安十三年】常不忘之初聞峻亡仍欲用護聞護性行危險用之適為作禍【行下孟翻為于偽翻】故更止之孤念公瑾豈有已哉十二月詔復以建寅之月為正【用地正事見上卷景初元年是時仍用景】<br />
<br />
  【初歷但不以十一月為正月耳】<br />
<br />
  邵陵厲公上【諱芳字蘭卿明帝無子養以為子諡法殺戮無辜曰厲帝後以失權為司馬氏所廢以其不終加以一惡諡陳夀志三少帝紀皆書本爵此書見廢後之爵自此以後例如此惟高貴鄉公書本爵蓋見弑之後不復有他號也帝之廢也歸藩于齊魏世譜曰晉受禪封齊王為邵陵縣公泰始十年薨諡曰厲扈蒙曰暴慢無親曰厲】<br />
<br />
  正始元年春旱 越嶲蠻夷數叛漢殺太守【自諸葛亮平高定之後越嶲夷數反殺太守龔禄焦璜嶲音髓數所角翻】是後太守不敢之郡寄治安定縣去郡八百餘里【安定縣不見于志當是因越嶲移治而蹔立也】漢主以巴西張嶷為越嶲太守嶷招慰新附誅討強猾蠻夷畏服郡界悉平復還舊治【漢越嶲郡治卭都縣嶷魚力翻】 冬吳饑<br />
<br />
  二年春吳人將伐魏零陵太守殷札言於吳主曰今天棄曹氏喪誅累見【殷札一作殷禮喪誅謂魏累有大喪蓋天誅也見賢遍翻】虎爭之際而幼童涖事陛下身自御戎取亂侮亡【書仲虺之誥之辭】宜滌荆揚之地【滌洗也言舉國興師後無留者其地如洗也】舉強羸之數使強者執戟羸者轉運【羸倫為翻】西命益州軍于隴右【益州謂蜀也】授諸葛瑾朱然大衆直指襄陽陸遜朱桓别征夀春大駕入淮陽歷青徐【前漢之淮陽後漢章帝改曰陳郡此直謂淮水之陽也】襄陽夀春困於受敵長安以西務禦蜀軍許洛之衆埶必分離掎角並進民必内應將帥對向或失便宜一軍敗績則三軍離心便當秣馬脂車陵蹈城邑乘勝逐北以定華夏若不悉軍動衆循前輕舉則不足大用易於屢退【易以䜴翻】民疲威消時往力竭非上策也吳主不能用【傾國出師决勝負於一戰苻堅之所以亡也吳主非不能用殷札之計不肯用也】夏四月吳全琮略淮南决芍陂【賢曰芍陂今在夀州安豐縣東陂徑百里灌田萬頃華夷對境圖芍陂周回二百二十四里與陽泉大業並孫叔敖所作開溝引渒水為子午渠開六門灌田萬頃芍音鵲】諸葛恪攻六安朱然圍樊諸葛瑾攻柤中【襄陽記曰祖讀如租稅之租柤中在上黃界去襄陽一百五十里魏時夷正權敷兄弟三人部曲萬餘家屯此分布在中廬宜城西山沔二谷中土地平敞宜桑麻有水陸良田沔南之膏腴沃壤謂之柤中杜佑曰柤中在襄州南漳縣界楊正衡曰柤側瓜翻】征東將軍王凌揚州刺史孫禮與全琮戰於芍陂琮敗走荆州刺史胡質以輕兵救樊【魏荆州統江夏襄陽南陽新城魏興上庸】或曰賊盛不可迫質曰樊城卑兵少故當進軍為之外援不然危矣遂勒兵臨圍城中乃安 五月吳太子登卒 吳兵猶在荆州太傅懿曰柤中民夷十萬隔在水南流離無主樊城被攻歷月不解此危事也請自討之六月太傅懿督諸軍救樊吳軍聞之夜遁追至三州口【三州口謂荆豫楊三州之口魏荆州之地東至江夏豫州之地南至弋陽揚州之地西至六安三州口當在其間又按王昶傳昶督荆豫諸軍事自宛徙屯新野習水軍於三州則三州蓋地名口水口】大獲而還 閏月吳大將軍諸葛瑾卒瑾太子恪先已封侯【恪以適當為世子曰太子誤也恪以出山民功封侯事見上卷景初元年】吳主以恪弟融襲爵攝兵業【攝領也承也領父之兵承父之業也】駐公安 漢大司馬蔣琬以諸葛亮數出秦川【關中之地沃野千里秦之故國謂之秦川數所角翻】道險運糧難卒無成功【卒子恤翻】乃多作舟船欲乘漢沔東下襲魏興上庸【漢沔之水自漢中東歷魏興上庸以達于襄陽欲争天下則當出兵秦川魏興上庸非其地也】會舊疾連動未時得行漢人咸以為事有不捷還路甚難非長策也漢主遣尚書令費禕【費父沸翻】中監軍姜維等喻指【中監軍即中護軍之任也蜀置前監軍後監軍中監軍位三軍師之下】琬乃上言今魏跨帶九州根蔕滋蔓平除未易【易以䜴翻】若東西并力首尾掎角【掎居蟻翻】雖未能速得如志且當分裂蠶食先摧其支黨然吳期二三連不克果【克能也果决也言不能决然進兵也】輒與費禕等議以凉州胡塞之要進退有資且羌胡乃心思漢如渇宜以姜維為凉州刺史【凉州之地蜀惟得武都隂平二郡而已】若維征行銜制河右臣當帥軍為維鎮繼【帥讀曰率】今涪水陸四通惟急是應若東西有虞赴之不難請徙屯涪【涪縣漢屬廣漢郡蜀屬梓潼郡涪音浮】漢主從之 朝廷欲廣田畜穀於揚豫之間使尚書郎汝南鄧艾行陳項以東至夀春【陳縣漢屬陳國項縣漢屬汝南郡晉志二縣並屬梁國行下孟翻】艾以為昔太祖破黃巾因為屯田積穀許都以制四方【事見六十二卷漢獻帝建安元年】今三隅已定事在淮南每大軍出征運兵過半功費巨億陳蔡之間上下田良可省許昌左右諸稻田并水東下【汝水潁水蒗蕩渠水滆水皆經陳蔡之間而東入淮】令淮北二萬人淮南三萬人什二分休常有四萬人且田且守【五萬人分一萬番休迭戍周而復始是常有四萬人屯田】益開河渠以增溉灌通漕運計除衆費歲完五百萬斛以為軍資六七年間可積三千萬斛於淮上此則十萬之衆五年食也以此乘吳無不克矣太傅懿善之是歲始開廣漕渠每東南有事大興軍衆汎舟而下達于江淮資食有餘而無水害【史究言鄧艾興屯田之利】 管寧卒寧名行高潔人望之者邈然若不可及即之熙熙和易【行下孟翻易以䜴翻】能因事導人於善人無不化服及卒天下知與不知無不嗟歎<br />
<br />
  三年春正月漢姜維率偏軍自漢中還住涪【蜀諸軍時皆屬蔣琬姜維所領偏軍耳】 吳主立其子和為太子大赦 三月昌邑景侯滿寵卒秋七月乙酉以領軍將軍蔣濟為太尉吳主遣將軍聶友校尉陸凱將兵三萬擊儋耳珠崖【儋耳珠崖漢武帝開以為郡屬交趾州元帝以後棄之聶尼輒翻儋都甘翻】 八月吳主封子霸為魯王霸和母弟也寵愛崇特與和無殊【為後吳廢和誅霸張本】尚書僕射是儀領魯王傳上疏諫曰臣竊以為魯王天挺懿德兼資文武當今之宜宜鎮四方為國藩輔宣揚德美廣耀威靈乃國家之良規海内所瞻望且二宫宜有降殺【殺所戒翻】以正上下之序明教化之本書三四上吳主不聽【四上時掌翻】<br />
<br />
  四年春正月帝加元服 吳諸葛恪襲六安【漢六安國都六縣後漢為六安侯國屬廬江郡晉為六縣屬廬江郡】掩其人民而去 夏四月立皇后甄氏【甄之人翻】大赦后文昭皇后兄儼之孫也 五月朔日有食之既 冬十月漢蔣琬自漢中還住涪疾益甚以漢中太守王平為前監軍鎮北大將軍督漢中【監古衘翻】十一月漢主以尚書令費禕為大將軍録尚書事<br />
<br />
  吳丞相顧雍卒 吳諸葛恪遠遣諜人觀相徑要【諜逵協翻相息亮翻】欲圖夀春太傅懿將兵入舒【舒縣屬廬江郡春秋之故國也時在吳魏境上棄而不耕去皖口甚近】欲以攻恪吳主徙恪屯於柴桑【柴桑縣漢屬豫章郡吳屬武昌郡有柴桑山在今江州德化西九十里杜佑曰江州尋陽縣南楚城驛即古之柴桑縣宋白曰江州瑞昌縣蓋柴桑之舊城】 步隲朱然各上疏於吳主曰自蜀還者咸言蜀欲背盟【隲入日翻上時掌翻背蒲妹翻】與魏交通多作舟船繕治城郭【治直之翻】又蔣琬守漢中聞司馬懿南向不出兵乘虚以掎角之【掎居蟻翻】反委漢中還近成都事已彰灼無所復疑【近其靳翻復扶又翻】宜為之備吳主答曰吾待蜀不薄聘享盟誓無所負之何以致此司馬懿前來入舒旬日便退蜀在萬里何知緩急而便出兵乎昔魏欲入漢川【曹真欲入漢中事見七十一卷明帝太和四年】此間始嚴亦未舉動【謂嚴兵而未發也】會聞魏還而止【還從宣翻又如字】蜀寧可復以此有疑邪人言苦不可信朕為諸君破家保之【復扶又翻為于偽翻】 征東將軍都督楊豫諸軍事王昶【據王昶傳楊當作荆】上言地有常險守無常勢今屯宛去襄陽三百餘里有急不足相赴遂徙屯新野【新野縣屬南陽郡】 宗室曹冏【裴松之曰冏中常侍兄叔興之後少帝之族祖也】上書曰古之王者必建同姓以明親親必樹異姓以明賢賢親親之道專用則其漸也微弱賢賢之道偏任則其敝也刼奪【謂威權陵偪刼其君而奪之也】先聖知其然也故博求親疎而並用之故能保其社稷歷紀長久【紀年紀也】今魏尊尊之法雖明親親之道未備或任而不重或釋而不任臣竊惟此【惟思也】寢不安席謹撰合所聞論其成敗【撰具也述也音雛免翻】曰昔夏商周歷世數十而秦二世而亡何則三代之君與天下共其民故天下同其憂【呂延濟曰與天下共其民謂建立諸侯與之共理同有其利也故天下有難則諸侯同憂】秦王獨制其民故傾危而莫救也秦觀周之敝以為小弱見奪於是廢五等之爵立郡縣之官【呂向曰秦皇觀周所以敝者乃以埶弱而諸侯奪其國也遂廢五等之爵而立郡縣之吏五等公侯伯子男也】内無宗子以自毗輔【毗亦輔也】外無諸侯以為藩衛譬猶芟刈股肱獨任胸腹觀者為之寒心【芟所衛翻為于偽翻】而始皇晏然自以為子孫帝王萬世之業也豈不悖哉【悖蒲内翻】故漢祖奮三尺之劒驅烏合之衆五年之中遂成帝業何則伐深根者難為功摧枯朽者易為力理勢然也【用班固漢宗室諸侯王表文意易以䜴翻】漢監秦之失封殖子弟及諸呂擅權圖危劉氏而天下所以不傾動者徒以諸侯彊大磐石膠固也然高祖封建地過古制故賈誼以為欲天下之治安莫若衆建諸侯而少其力【賈誼治安策之言見十四卷文帝六年少詩沼翻治直吏翻】文帝不從至於孝景猥用鼂錯之計削黜諸侯遂有七國之患【事見十六卷漢景帝三年】蓋兆發高帝釁鍾文景【鍾聚也】由寛之過制急之不漸故也所謂末大必折尾大難掉【左傳由無宇之言折而設翻掉徒弔翻】尾同於體猶或不從况乎非體之尾其可掉哉武帝從主父之策下推恩之令【事見十八卷漢武帝元朔二年】自是之後遂以陵夷子孫微弱衣食租稅不預政事至于哀平王氏秉權假周公之事而為田常之亂宗室諸侯或乃為之符命頌莽恩德【事見三十六卷王莽初始元年】豈不哀哉由斯言之非宗子獨忠孝於惠文之間而叛逆於哀平之際也徒權輕埶弱不能有定耳賴光武皇帝挺不世之姿擒王莽於己成紹漢嗣於既絶斯豈非宗子之力也【嗣祥使翻】而曾不監秦之失策襲周之舊制至於桓靈閹䆠用事君孤立於上臣弄權於下由是天下鼎沸姦宄並争宗廟焚為灰燼宫室變為榛藪【謂董卓之亂也】太祖皇帝龍飛鳳翔掃除凶逆大魏之興于今二十有四年矣【自黃初受禪至是二十四年】觀五代之存亡而不用其長策【五代夏商周秦漢】覩前車之傾覆而不改於轍迹子弟王空虚之地君有不使之民【空虚謂有封國之名實不能有其地也君不使之民謂抗藩王之尊於國民之上不得而臣使也王于况翻】宗室竄於閭閻不聞邦國之政權均匹夫埶齊凡庶内無深根不拔之固外無盤石宗盟之助【呂延濟曰盤石大石也以其堅重不可轉易也宗盟謂同姓諸侯盟會者也】非所以安社稷為萬世之業也且今之州牧郡守【守式又翻】古之方伯諸侯皆跨有千里之土兼軍武之任或比國數人【比毗必翻又毗至翻】或兄弟並據而宗室子弟曾無一人間厠其間【人間古莧翻】與相維制非所以彊榦弱枝備萬一之虞也今之用賢或超為名都之主或為偏師之帥【帥所類翻】而宗室有文者必限小縣之宰有武者必置百人之上【張銑曰言宗室位卑也百人之上百夫長也】非所以勸進賢能褒異宗室之禮也語曰百足之蟲至死不僵【馬蚿百足僵居良翻】以其扶之者衆也此言雖小可以譬大是以聖王安不忘危存不忘亡故天下有變而無傾危之患矣冏冀以此論感悟曹爽爽不能用【以明帝之明且不能用陳思王之言况曹爽之愚闇哉】<br />
<br />
  五年春正月吳主以上大將軍陸遜為丞相其州牧都護領武昌事如故【遜先為荆州牧右都護領武昌事】 征西將軍都督雍凉諸軍事夏侯玄【雍於用翻】大將軍爽之姑子也玄辟李勝為長史勝及尚書鄧颺欲令爽立威名於天下勸使伐蜀太傅懿止之不能得三月爽西至長安發卒十餘萬人與玄自駱口入漢中【駱口駱谷口也駱谷在漢中成固縣東北北達扶風郿縣】漢中守兵不滿三萬諸將皆恐欲守城不出以待涪兵【自蔣琬屯涪蜀之重兵在焉】王平曰漢中去涪垂千里賊若得關便為深禍【垂幾及也關關城也杜佑曰關城俗名張魯城在西縣西四十里嗚呼王侯設險以守其國其後關城失守鍾會遂平行至漢中王平謂賊若得關遂為深禍斯言驗矣】今宜先遣劉護軍據興埶【水經注小成固城北百二十二里有興埶坂寰字記興埶山在洋州興道縣北四十三里今郡城所枕形如一盆外險而内有大谷為盤道上數里方及四門因名興埶東坡指掌圖以為在興元恐非也杜佑曰興埶即洋州興道縣寰字記與通典合矣宋白曰興勢山在今興道縣西北二十里劉護軍劉敏也】平為後拒若賊分向黃金【黄金谷在興道縣山有黃金峭黃金谷有黃金戍傍山依峭險折七里杜佑曰黄金戍在洋州黃金縣西北八十里張魯所築南接漢川北枕古道險固之極】平帥千人下自臨之【帥讀曰率下同】比爾間涪軍亦至【比必寐翻】此計之上也諸將皆疑惟護軍劉敏與平意同遂帥所領據興埶多張旗幟彌亘百餘里【幟昌志翻】閏月漢主遣大將軍費禕督諸軍救漢中將行光禄大夫來敏詣禕别求共圍棊于時羽檄交至人馬擐甲嚴駕已訖【擐胡慣翻貫甲也】禕與敏對戲色無厭倦敏曰向聊觀試君耳君信可人必能辦賊者也 夏四月丙辰朔日有食之 大將軍爽兵距興埶不得進關中及氐羌轉輸不能供牛馬騾驢多死民夷號泣道路【騾盧戈翻號戶高翻】涪軍及費禕兵繼至參軍楊偉為爽陳形埶【為于偽翻】宜急還不然將敗鄧颺李勝與偉争於爽前偉曰颺勝將敗國家事可斬也【後將敗補邁翻】爽不悦太傅懿與夏侯玄書曰春秋責大德重【責責望也德恩德也言責望之甚大者其恩之為甚重也】昔武皇帝再入漢中幾至大敗【事見六十七卷漢獻帝建安二十年及六十八卷建安二十四年幾居希翻】君所知也今興埶至險蜀已先據若進不獲戰退見邀絶覆軍必矣將何以任其責【任音壬】玄懼言於爽五月引軍還費禕進據三嶺以截爽【自駱谷出扶風隔以中南山其間有三嶺一曰沈嶺近芒水一曰衙嶺一曰分水嶺】爽争險苦戰僅乃得過失亡甚衆關中為之虚耗【為于偽翻】 秋八月秦王詢卒 冬十二月安陽孝侯崔林卒【諡法大慮行節曰孝五宗安之曰孝慈惠愛親曰孝】 是歲漢大司馬琬以病固讓州職於大將軍禕漢主乃以禕為益州刺史以侍中董允守尚書令為禕之副時戰國多事【戰國者謂國日有戰爭也】公務煩猥【猥雜也】禕為尚書令識悟過人每省讀文書【省悉京翻】舉目暫視已究其意旨其速數倍於人終亦不忘常以朝晡聽事其間接納賓客飲食嬉戲加之博弈每盡人之歡事亦不廢及董允代禕欲斆禕之所行旬日之中事多愆滯【愆違也】允乃歎曰人才力相遠若此非吾之所及也乃聽事終日而猶有不暇焉<br />
<br />
  六年春正月以票騎將軍趙儼為司空【票匹妙翻】 吳太子和與魯王同宫禮秩如一羣臣多以為言【是儀之諫見於是卷三年蓋諫者不特是儀也】吳主乃命分宫别僚二子由是有隙【史言和霸之隙亦兩宫僚屬交搆以成之别彼列翻】衛將軍全琮遣其子寄事魯王以書告丞相陸遜遜報曰子弟苟有才不憂不用不宜私出以要榮利【私出謂出私門也要一遥翻】若其不佳終為取禍且聞二宫勢敵必有彼此此古人之厚忌也寄果阿附魯王輕為交搆遜書與琮曰卿不師日磾而宿留阿寄【日磾事見二十一卷漢武帝後元二年宿音秀留音溜阿相傳從安入聲】終為足下家門致禍矣琮既不答遜言更以致隙魯王曲意交結當時名士偏將軍朱績以膽力稱王自至其廨【廨居隘翻公宇也】就之坐欲與結好【好呼到翻】績下地住立辭而不當績然之子也於是自侍御賓客造為二端仇黨疑貳滋延大臣舉國中分吳主聞之假以精學禁斷賓客往來【斷讀曰短】督軍使者羊衜上疏曰聞明詔省奪二宫備衛抑絶賓客使四方禮敬不復得通【復扶又翻】遠近悚然大小失望或謂二宫不遵典式就如所嫌猶且補察密加斟酌不使遠近得容異言臣懼積疑成謗久將宣流而西北二隅【蜀在西魏在北】去國不遠將謂二宫有不順之愆不審陛下何以解之吳主長女魯班適左護軍全琮少女小虎適驃騎將軍朱據【二女步夫人所生也】全公主與太子母王夫人有隙吳主欲立王夫人為后公主阻之恐太子立怨已心不自安數譖毁太子吳主寢疾遣太子禱於長沙桓王廟【孫策追諡長沙桓王杜佑曰孫權都建業立兄長沙桓王廟於朱雀橋南】太子妃叔父張休居近廟邀太子過所居全公主使人覘視【近其靳翻過工禾翻覘丑廉翻窺也】因言太子不在廟中專就妃家計議又言王夫人見上寢疾有喜色吳主由是發怒夫人以憂死太子寵益衰魯王之黨楊笁全寄吳安孫奇等共譖毁太子吳主惑焉陸遜上疏諫曰太子正統宜有盤石之固魯王藩臣當使寵秩有差彼此得所上下獲安書三四上【四上時掌翻】辭情危切【論語孔子曰邦有道危言危行程顥曰危獨也與衆異而不安之謂余按此所謂危者謂嫡庶無别則亡國之禍隨之人不敢言而遜獨言之所謂危也】又欲詣都口陳嫡庶之義吳主不悦 【考異曰吳録曰權時見楊笁辟左右而論霸之才笁深述霸有文武英姿宜為嫡嗣於是權乃許立焉既而遜有表極諫權疑笁泄之乃斬笁按笁死在太子廢後吳録所述妄也】太常顧譚遜之甥也亦上疏曰臣聞有國有家者必明嫡庶之端異尊卑之禮使高下有差等級踰邈如此則骨肉之恩全覬覦之望絶【覬音冀覦音俞】昔賈誼陳治安之計論諸侯之埶以為埶重雖親必有逆節之累【治直吏翻累力瑞翻】埶輕雖疏必有保全之祚故淮南親弟不終饗國失之於勢重也吳芮疏臣傳祚長沙得之於埶輕也昔漢文帝使慎夫人與皇后同席袁盎退夫人之位帝有怒色及盎辨上下之義陳人彘之戒帝既悦懌夫人亦悟【事見十三卷漢文帝二年】今臣所陳非有所偏誠欲以安太子而便魯王也由是魯王與譚有隙芍陂之役【二年全琮與魏戰于芍陂】譚弟承及張休皆有功全琮子端緒與之爭功【端緒琮之二子】譖承休於吳主吳主徙譚承休於交州又追賜休死太子太傅吾粲請使魯王出鎮夏口【姓譜吾本已姓夏昆吾氏之後夏戶雅翻】出楊笁等不得令在京師又數以消息語陸遜【數所角翻下同語半倨翻】魯王與楊笁共譖之吳主怒收粲下獄誅【下遐稼翻】數遣中使責問陸遜遜憤恚而卒【恚於避翻】其子抗為建武校尉代領遜衆送葬東還【自荆州東還葬吳還從宣翻又如字】吳主以楊笁所白遜二十事問抗抗事事條答吳主意乃稍解 夏六月都鄉穆侯趙儼卒【諡法中情見貌曰穆】 秋七月吳將軍馬茂謀殺吳主及大臣以應魏事泄并黨與皆伏誅【吳歷曰茂本魏淮南鍾離長叛降吳】 八月以太常高柔為司空 漢甘太后殂【甘太后後主之母據陳夀志先已卒於南郡此吳太后也吳懿之妹先主入蜀始納焉證以蜀志其殂在是年】 吳主遣校尉陳勲將屯田及作士三萬人鑿句容中道自小其至雲陽西城通會市作邸閣【沈約曰句容漢舊縣屬丹陽郡今在建康府南九十里有茅山亦謂之句曲山班固曰會稽曲阿縣本秦雲陽縣也後漢屬吳郡沈約曰曲阿本曰雲陽秦始皇改曰曲阿吳嘉禾三年復曰雲陽今相傳秦時或言其地有天子氣始皇鑿坑以敗其勢截直道使阿曲故謂之曲阿劉昫曰潤州金壇縣本曲阿縣地會市者作市以會商旅句如字】 冬十一月漢大司馬琬卒 十二月漢費禕至漢中行圍守【魏延鎭漢中實兵諸圍以禦敵所謂圍守也行下孟翻】 漢尚書令董允卒以尚書呂乂為尚書令董允秉心公亮獻可替否備盡忠益漢主甚嚴憚之䆠人黃皓便僻佞慧【便毗連翻】漢主愛之允上則正色規主下則數責於皓【數所具翻】皓畏允不敢為非終允之世皓位不過黃門丞【續漢志黃門令丞一人以䆠者為之】費禕以選曹郎汝南陳祇代允為侍中【漢六曹尚書一曹有郎六人選曹郎屬選部選須絹翻】祇矜厲有威容多技藝挾智數故禕以為賢越次而用之祇與皓相表裏皓始預政累遷至中常侍操弄威柄終以覆國自陳祇有寵而漢主追怨董允日深謂為自輕【謂允為輕已也】由祇阿意迎合而皓浸潤搆間故也【間古莧翻】<br />
<br />
  資治通鑑卷七十四  <br>
   </div> 

<script src="/search/ajaxskft.js"> </script>
 <div class="clear"></div>
<br>
<br>
 <!-- a.d-->

 <!--
<div class="info_share">
</div> 
-->
 <!--info_share--></div>   <!-- end info_content-->
  </div> <!-- end l-->

<div class="r">   <!--r-->



<div class="sidebar"  style="margin-bottom:2px;">

 
<div class="sidebar_title">工具类大全</div>
<div class="sidebar_info">
<strong><a href="http://www.guoxuedashi.com/lsditu/" target="_blank">历史地图</a></strong>  
<a href="http://www.880114.com/" target="_blank">英语宝典</a>  
<a href="http://www.guoxuedashi.com/13jing/" target="_blank">十三经检索</a> 
<br><strong><a href="http://www.guoxuedashi.com/gjtsjc/" target="_blank">古今图书集成</a></strong> 
<a href="http://www.guoxuedashi.com/duilian/" target="_blank">对联大全</a> <strong><a href="http://www.guoxuedashi.com/xiangxingzi/" target="_blank">象形文字典</a></strong> 

<br><a href="http://www.guoxuedashi.com/zixing/yanbian/">字形演变</a>  <strong><a href="http://www.guoxuemi.com/hafo/" target="_blank">哈佛燕京中文善本特藏</a></strong>
<br><strong><a href="http://www.guoxuedashi.com/csfz/" target="_blank">丛书&方志检索器</a></strong> <a href="http://www.guoxuedashi.com/yqjyy/" target="_blank">一切经音义</a>  

<br><strong><a href="http://www.guoxuedashi.com/jiapu/" target="_blank">家谱族谱查询</a></strong>  <strong><a href="http://shufa.guoxuedashi.com/sfzitie/" target="_blank">书法字帖欣赏</a></strong> 
<br>

</div>
</div>


<div class="sidebar" style="margin-bottom:0px;">

<font style="font-size:22px;line-height:32px">QQ交流群9:489193090</font>


<div class="sidebar_title">手机APP 扫描或点击</div>
<div class="sidebar_info">
<table>
<tr>
	<td width=160><a href="http://m.guoxuedashi.com/app/" target="_blank"><img src="/img/gxds-sj.png" width="140"  border="0" alt="国学大师手机版"></a></td>
	<td>
<a href="http://www.guoxuedashi.com/download/" target="_blank">app软件下载专区</a><br>
<a href="http://www.guoxuedashi.com/download/gxds.php" target="_blank">《国学大师》下载</a><br>
<a href="http://www.guoxuedashi.com/download/kxzd.php" target="_blank">《汉字宝典》下载</a><br>
<a href="http://www.guoxuedashi.com/download/scqbd.php" target="_blank">《诗词曲宝典》下载</a><br>
<a href="http://www.guoxuedashi.com/SiKuQuanShu/skqs.php" target="_blank">《四库全书》下载</a><br>
</td>
</tr>
</table>

</div>
</div>


<div class="sidebar2">
<center>


</center>
</div>

<div class="sidebar"  style="margin-bottom:2px;">
<div class="sidebar_title">网站使用教程</div>
<div class="sidebar_info">
<a href="http://www.guoxuedashi.com/help/gjsearch.php" target="_blank">如何在国学大师网下载古籍?</a><br>
<a href="http://www.guoxuedashi.com/zidian/bujian/bjjc.php" target="_blank">如何使用部件查字法快速查字?</a><br>
<a href="http://www.guoxuedashi.com/search/sjc.php" target="_blank">如何在指定的书籍中全文检索?</a><br>
<a href="http://www.guoxuedashi.com/search/skjc.php" target="_blank">如何找到一句话在《四库全书》哪一页?</a><br>
</div>
</div>


<div class="sidebar">
<div class="sidebar_title">热门书籍</div>
<div class="sidebar_info">
<a href="/so.php?sokey=%E8%B5%84%E6%B2%BB%E9%80%9A%E9%89%B4&kt=1">资治通鉴</a> <a href="/24shi/"><strong>二十四史</strong></a>&nbsp; <a href="/a2694/">野史</a>&nbsp; <a href="/SiKuQuanShu/"><strong>四库全书</strong></a>&nbsp;<a href="http://www.guoxuedashi.com/SiKuQuanShu/fanti/">繁体</a>
<br><a href="/so.php?sokey=%E7%BA%A2%E6%A5%BC%E6%A2%A6&kt=1">红楼梦</a> <a href="/a/1858x/">三国演义</a> <a href="/a/1038k/">水浒传</a> <a href="/a/1046t/">西游记</a> <a href="/a/1914o/">封神演义</a>
<br>
<a href="http://www.guoxuedashi.com/so.php?sokeygx=%E4%B8%87%E6%9C%89%E6%96%87%E5%BA%93&submit=&kt=1">万有文库</a> <a href="/a/780t/">古文观止</a> <a href="/a/1024l/">文心雕龙</a> <a href="/a/1704n/">全唐诗</a> <a href="/a/1705h/">全宋词</a>
<br><a href="http://www.guoxuedashi.com/so.php?sokeygx=%E7%99%BE%E8%A1%B2%E6%9C%AC%E4%BA%8C%E5%8D%81%E5%9B%9B%E5%8F%B2&submit=&kt=1"><strong>百衲本二十四史</strong></a>  <a href="http://www.guoxuedashi.com/so.php?sokeygx=%E5%8F%A4%E4%BB%8A%E5%9B%BE%E4%B9%A6%E9%9B%86%E6%88%90&submit=&kt=1"><strong>古今图书集成</strong></a>
<br>

<a href="http://www.guoxuedashi.com/so.php?sokeygx=%E4%B8%9B%E4%B9%A6%E9%9B%86%E6%88%90&submit=&kt=1">丛书集成</a> 
<a href="http://www.guoxuedashi.com/so.php?sokeygx=%E5%9B%9B%E9%83%A8%E4%B8%9B%E5%88%8A&submit=&kt=1"><strong>四部丛刊</strong></a>  
<a href="http://www.guoxuedashi.com/so.php?sokeygx=%E8%AF%B4%E6%96%87%E8%A7%A3%E5%AD%97&submit=&kt=1">說文解字</a> <a href="http://www.guoxuedashi.com/so.php?sokeygx=%E5%85%A8%E4%B8%8A%E5%8F%A4&submit=&kt=1">三国六朝文</a>
<br><a href="http://www.guoxuedashi.com/so.php?sokeytm=%E6%97%A5%E6%9C%AC%E5%86%85%E9%98%81%E6%96%87%E5%BA%93&submit=&kt=1"><strong>日本内阁文库</strong></a> <a href="http://www.guoxuedashi.com/so.php?sokeytm=%E5%9B%BD%E5%9B%BE%E6%96%B9%E5%BF%97%E5%90%88%E9%9B%86&ka=100&submit=">国图方志合集</a> <a href="http://www.guoxuedashi.com/so.php?sokeytm=%E5%90%84%E5%9C%B0%E6%96%B9%E5%BF%97&submit=&kt=1"><strong>各地方志</strong></a>

</div>
</div>


<div class="sidebar2">
<center>

</center>
</div>
<div class="sidebar greenbar">
<div class="sidebar_title green">四库全书</div>
<div class="sidebar_info">

《四库全书》是中国古代最大的丛书,编撰于乾隆年间,由纪昀等360多位高官、学者编撰,3800多人抄写,费时十三年编成。丛书分经、史、子、集四部,故名四库。共有3500多种书,7.9万卷,3.6万册,约8亿字,基本上囊括了古代所有图书,故称“全书”。<a href="http://www.guoxuedashi.com/SiKuQuanShu/">详细>>
</a>

</div> 
</div>

</div>  <!--end r-->

</div>
<!-- 内容区END --> 

<!-- 页脚开始 -->
<div class="shh">

</div>

<div class="w1180" style="margin-top:8px;">
<center><script src="http://www.guoxuedashi.com/img/plus.php?id=3"></script></center>
</div>
<div class="w1180 foot">
<a href="/b/thanks.php">特别致谢</a> | <a href="javascript:window.external.AddFavorite(document.location.href,document.title);">收藏本站</a> | <a href="#">欢迎投稿</a> | <a href="http://www.guoxuedashi.com/forum/">意见建议</a> | <a href="http://www.guoxuemi.com/">国学迷</a> | <a href="http://www.shuowen.net/">说文网</a><script language="javascript" type="text/javascript" src="https://js.users.51.la/17753172.js"></script><br />
  Copyright &copy; 国学大师 古典图书集成 All Rights Reserved.<br>
  
  <span style="font-size:14px">免责声明:本站非营利性站点,以方便网友为主,仅供学习研究。<br>内容由热心网友提供和网上收集,不保留版权。若侵犯了您的权益,来信即刪。scp168@qq.com</span>
  <br />
ICP证:<a href="http://www.beian.miit.gov.cn/" target="_blank">鲁ICP备19060063号</a></div>
<!-- 页脚END --> 
<script src="http://www.guoxuedashi.com/img/plus.php?id=22"></script>
<script src="http://www.guoxuedashi.com/img/tongji.js"></script>

</body>
</html>
