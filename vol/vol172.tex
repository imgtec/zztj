資治通鑑卷一百七十二
宋 司馬光 撰

胡三省 音註

陳紀六|{
	起旃蒙協洽盡柔兆涒灘凡二年}


高宗宣皇帝中之上

太建七年春正月辛未上祀南郊 癸酉周主如同州乙亥左衛將軍樊毅克潼州|{
	五代志下邳郡夏丘縣梁後齊置潼州治取慮}


|{
	城潼音童}
齊主還鄴|{
	去年二月齊主如晉陽討思好尋已還鄴八月復如晉陽今還}
辛巳上祀北郊|{
	陳制亦以間歲正月上辛用特牛一祀天地於南北二郊間歲者一歲祀南郊一歲祀北郊也}
二月丙戍朔日有食之 戊申樊毅克下邳高柵等六城|{
	地形志下邳郡有柵淵縣武定八年分宿豫置柵楚格翻}
齊主言語澁呐不喜見朝士自非寵私昵狎未嘗交語|{
	澁色立翻不滑順也呐女劣翻聲不出也趙文子其言呐呐不能出諸口喜許既翻朝直遥翻昵尼質翻}
性懦不堪人視|{
	所謂弱顔也懦乃卧翻又奴亂翻}
雖三公令録奏事|{
	令尚書令録録尚書事}
莫得仰視皆略陳大指驚走而出承世祖奢泰之餘|{
	齊主之父廟號世祖}
以為帝王當然後宫皆寶衣玉食一裙之費至直萬匹競為新巧朝衣夕弊|{
	朝衣於既翻}
盛脩宫苑窮極壯麗所好不常數毁又復|{
	好呼到翻下同數所角翻}
百工土木無時休息夜則然火照作寒則以湯為泥鑿晉陽西山為大像一夜然油萬盆光照宫中|{
	晉陽宫也}
每有災異寇盗不自貶損唯多設齋以為脩德|{
	後之有天下者可以鍳矣}
好自彈琵琶為無愁之曲近侍和之者以百數民間謂之無愁天子|{
	五代志帝倚絃而歌别採新聲為無愁曲音韻窈窕極於哀思使胡兒閹䆠輩齊唱和之曲終樂闋無不殞涕雖行幸道路或時馬上奏之樂往哀來竟以亡國和戶卧翻}
於華林園|{
	鄴都有華林園華如字}
立貧兒村帝自衣藍縷之服行乞其間以為樂|{
	衣服穿弊如懸鶉者為藍縷衣於既翻下人衣同樂盧各翻}
又寫築西鄙諸城使人衣黑衣攻之帝自帥内參拒鬬|{
	寫築者寫諸城之形而築以象之黑衣者象周之戎衣内參者諸閹宦也帥讀曰率}
寵任陸令萱穆提婆高阿那肱韓長鸞等宰制朝政宦官鄧長顒陳德信胡兒何洪珍等並參預機權|{
	朝直遥翻顒魚容翻}
各引親黨超居顯位官由財進獄以賄成競為姦謟蠧政害民舊蒼頭劉桃枝等皆開府封王其餘宦官胡兒歌舞人見鬼人官奴婢等濫得富貴者殆將萬數庶姓封王者以百數開府千餘人儀同無數領軍一時至二十人侍中中常侍數十人乃至狗馬及鷹亦有儀同郡君之號有鬬雞號開府皆食其幹禄|{
	魏齊官制凡禄各以品秩為差官一品每歲禄八百匹二百匹為一秩從一品七百匹一百七十五匹為一秩二品六百匹一百五十匹為一秩從二品五百匹一百二十五匹為一秩三品四百匹一百匹為一秩從三品三百匹七十五匹為一秩四品二百四十匹六十匹為一秩從四品二百匹五十匹為一秩五品一百六十匹四十匹為一秩從五品一百二十匹三十匹為一秩六品一百匹二十五匹為一秩從六品八十匹二十匹為一秩七品六十匹十五匹為一秩從七品四十匹十匹為一秩八品三十六匹九匹為一秩從八品三十二匹八匹為一秩九品二十八匹七匹為一秩從九品二十四匹六匹為一秩禄率一分以帛一分以粟一分以錢幹出所部之人一幹輸絹十八匹幹身放之}
諸嬖倖朝夕娛侍左右|{
	嬖卑義翻又博計翻}
一戱之費動踰巨萬既而府藏空竭|{
	藏徂浪翻}
乃賜二三郡或六七縣使之賣官取直由是為守令者率皆富商大賈|{
	守音狩賈音古}
競為貪縱民不聊生|{
	史極言齊氏政亂以啟敵國兼并之心又一年而齊亡有天下者可不以為鑒乎書名通鑑豈苟然哉}
周高祖謀伐齊命邊鎭益儲偫加戍卒|{
	偫直里翻}
齊人聞之亦增脩守禦柱國于翼諫曰疆場相侵互有勝負徒損兵儲無益大計不如解嚴繼好使彼懈而無備然後乘間出其不意|{
	好呼到翻懈古隘翻間古莧翻下間隙乘間同}
一舉可取也周主從之韋孝寛上疏陳三策|{
	上時掌翻}
其一曰臣在邉積年頗見間隙不因際會難以成功是以往歲出軍徒有勞費功績不立由失機會|{
	周主保定初再伐齊攻并州圍洛陽趣懸瓠出斬關皆無功事見一百六十九卷世祖天嘉四年五年争宜陽争汾北事見一百七十卷太建元年至三年}
何者長淮之南舊為沃土陳氏以破亡餘燼猶能一舉平之|{
	曰破亡餘燼者言陳氏承梁元帝江陵破亡之後收合餘燼再立國於江南燼徐刃翻火餘燭餘曰燼}
齊人歷年赴救喪敗而返|{
	事見上卷五年六年喪息浪翻}
内離外叛計盡力窮讎敵有舋不可失也|{
	左傳鬬伯比曰讎有釁不可失也舋與釁同宇文高氏世為讎敵}
今大軍若出軹關方軌而進|{
	五代志軹關在河内郡王屋縣周師若自軹關出險趨鄴前無阻隘可以方軌横行}
兼與陳氏共為掎角|{
	欲約陳共攻之左傳譬如捕鹿晉人角之諸戎掎之角者當其前掎者掎其後掎居綺翻}
并令廣州義旅出自三鵶|{
	魏永安中置廣州於魯陽魏分東西廣州西屬三鵶谷在魯陽界}
又募山南驍鋭沿河而下|{
	周都長安以褒漢荆襄為山南驍堅堯翻下同}
復遣北山稽胡絶其并晉之路|{
	稽胡南匈奴之餘種散在河東西河郡界阻山而居在長安北復扶又翻并卑經翻}
凡此諸軍仍令各募關河之外勁勇之士厚其爵賞使為前驅|{
	關河之外指齊境而言欲募其土人以為郷導}
岳動川移雷駭電激百道俱進並趨虜庭|{
	趨七喻翻}
必當望旗奔潰所向摧殄一戎大定實在此機|{
	武王伐紂一戎衣而天下大定}
其二曰若國家更為後圖未即大舉宜與陳人分其兵勢三鵶以北萬春以南|{
	萬春地名新唐志武德五年析龍門置萬春縣盖以舊地名名縣也三鵶以北萬春以南韋孝寛袤指周東北之境舉两端而言}
廣事屯田預為貯積|{
	貯直呂翻}
募其驍悍立為部伍|{
	悍侯旰翻又下罕翻}
彼既東南有敵戎馬相持|{
	謂齊人與陳人為敵也}
我出奇兵破其疆場|{
	場音亦}
彼若興師赴援我則堅壁清野待其去遠還復出師|{
	復扶又翻}
常以邉外之軍引其腹心之衆我無宿春之費|{
	莊子適百里者宿舂糧}
彼有奔命之勞|{
	左傳申公巫臣遺楚令尹子重司馬子反書曰吾必使汝疲於奔命以死於是導吳伐楚子重子反一歲七奔命}
一二年中必自離叛且齊氏昏暴政出多門鬻獄賣官唯利是視荒淫酒色忌害忠良闔境嗷然不勝其弊|{
	闔戶臘翻勝音升}
以此而觀覆亡可待然後乘間電掃事等摧枯|{
	間古莧翻}
其三曰昔句踐亡吳尚期十載|{
	左傳伍員曰越十年生聚十年教訓二十年之外吳其為沼乎此言十載以教訓言之也句音鉤踐慈演翻載作亥翻}
武王取紂猶煩再舉|{
	史記武王三年喪畢觀兵孟津諸侯不期而會者八百皆曰紂可伐武王曰汝未知天命乃還師三年紂淫暴日甚武王告諸侯曰殷有重罪不可不伐遂復伐紂㓕之}
今若更存遵養|{
	詩周頌於鑠王師遵養時晦毛傳云遵率養取晦昩也鄭箋云文王率殷之叛國以事紂養是晦昧之君以老其惡自是說者悉祖其義故云然}
且復相時|{
	左傳相時而動無累後人復扶又翻相息亮翻}
臣謂宜還崇鄰好申其盟約安民和衆通商惠工蓄鋭養威觀舋而動斯乃長策遠馭坐自兼并也|{
	好呼到翻通商惠工左傳語自古以來謀臣智士陳三策者其上策率非常人所能行中策亦必度其才足以行之然後能聽而用之通鑑盖謂于翼韋孝寛所見略同也}
書奏周主引開府儀同三司伊婁謙入内殿|{
	伊婁虜複姓拓拔之興於代北也獻帝以其次弟為伊婁氏}
從容謂曰朕欲用兵何者為先對曰齊氏沈溺倡優耽昏麯糵|{
	從七容翻沈持林翻倡音昌糵魚列翻}
其折衝之將斛律明月已斃於讒口|{
	事見上卷四年將即亮翻}
上下離心道路以目|{
	道路以目本周語言道路以目相視而不敢言}
此易取也|{
	易以豉翻}
帝大笑|{
	喜其所見與已同}
三月丙辰使謙與小司寇元衛聘於齊以觀舋|{
	為周㓕齊張本 考異曰謙傳作拓拔偉今從周書帝紀 余按伊婁與拓拔同所自出而各為氏則伊婁謙本傳作拓拔不為無据}
丙寅周主還長安|{
	自同州還還從宣翻又音如字}
夏四月甲午上享太廟 監豫州陳桃根得青牛獻之|{
	監工衘翻}
詔遣還民又表上織成羅文錦被各二百首|{
	上時掌翻}
詔於雲龍門外焚之 庚子齊以中書監陽休之為尚書右僕射 六月壬辰以尚書右僕射王瑒為左僕射|{
	瑒雉杏翻又音暢}
甲戍齊主如晉陽 秋七月丙戍周主如雲陽宮大將軍揚堅姿相奇偉|{
	堅為人龍顔額有五柱入頂目光外射有文在手曰王長上短下沈深嚴重相息亮翻下同}
畿伯下大夫長安來和|{
	畿伯周置屬大司徒杜佑曰周地官之屬每方畿伯中大夫也每縣小畿伯則下大夫}
嘗謂堅曰公眼如曙星無所不照當王有天下|{
	曙星向曉之星其光閃爍曙常恕翻王于况翻}
願忍誅殺|{
	盖以其姿相殺氣重也後堅之簒内夷宇文外翦尉遲迥檀讓王謙死者不可勝數人固有相乎}
周主待堅素厚齊王憲言於帝曰普六茹堅相貌非常|{
	堅父忠從周太祖屡有戰功賜姓普六茹氏}
臣每見之不覺自失恐非人下請早除之帝亦疑之以問來和和詭對曰隨公止是守節人可鎭一方若為將領陳無不破|{
	言不以實曰詭將即亮翻}
丁卯周主還長安先是周主獨與齊王憲及内史王誼謀伐齊|{
	先悉薦翻}
又遣納言盧韞|{
	周保定四年改宗伯為納言}
乘馹三日詣安州總管于翼問策|{
	馹人質翻驛傳也周置安州於安陸}
餘人皆莫之知丙子始召大將軍以上於大德殿告之丁丑下詔伐齊以柱國陳王純滎陽公司馬消難鄭公逹奚震為前三軍總管|{
	難乃旦翻}
越王盛周昌公侯莫陳崇|{
	侯莫陳崇已死於保定三年此又一侯莫陳崇也不則崇字誤}
趙王招為後三軍總管齊王憲帥衆二萬趨黎陽|{
	帥讀曰率下同趨七喻翻}
隨公楊堅廣寧公薛迥將舟師三萬自渭入河|{
	將即亮翻}
梁公侯莫陳芮帥衆二萬守太行道|{
	太行道在河陽北守之欲以斷并冀殷定之兵行戶剛翻}
申公李穆帥衆三萬守河陽道|{
	自河隂北渡河為河陽周主將攻河陽洛陽守之以斷其相往來}
常山公于翼帥衆二萬出陳汝|{
	盖令于翼自安州出陳汝自齊王憲以下皆指授諸將所出之道}
誼盟之兄孫震武之子也|{
	王盟逹奚武皆周初功臣}
周主將出河陽内史上士宇文㢸曰|{
	㢸音弼}
齊氏建國於今累世雖曰無道藩鎭之任尚有其人今之出師要須擇地河陽衝要精兵所聚盡力攻圍恐難得志如臣所見出於汾曲戍小山平|{
	汾曲汾水之曲也}
攻之易拔用武之地莫過於此|{
	易以䜴翻}
民部中大夫天水趙煚曰|{
	民部盖屬大司徒煚俱永翻}
河南洛陽四面受敵縱得之不可以守請從河北直指太原|{
	此即出蒲晉抵晉陽路其後周主再舉卒出於此}
傾其巢穴可一舉而定遂伯下大夫鮑宏曰|{
	遂伯盖髣髴周官遂師之職杜佑曰周地官之屬有左右遂伯中大夫也小遂伯則下大夫每鄉一人}
我彊齊弱我治齊亂何憂不克|{
	治直吏翻}
但先帝往日屢出洛陽|{
	先帝謂宇文泰}
彼既有備每有不捷如臣計者進兵汾潞|{
	汾潞謂汾川潞川鮑宏欲出師以攻平陽上黨也}
直掩晉陽出其不虞|{
	不虞謂不備也虞慮也度也測也三者所及則為備}
似為上策周主皆不從|{
	周主盖欲淺攻以觀舋觀其再舉所以告羣臣者可知}
宏泉之弟也|{
	鮑泉事梁元帝江陵破宏入關}
壬午周主帥衆六萬直指河隂楊素請帥其父麾下先驅周主許之|{
	楊素父敷死事見一百七十卷太建三年帥讀曰率}
八月癸卯周遣使來聘 周師入齊境禁伐樹踐稼犯者皆斬丁未周主攻河隂大城拔之齊王憲拔武濟|{
	武濟城名周武王伐紂由此濟河故以名城}
進圍洛口|{
	洛水入河之口於此置城}
拔東西二城縱火焚浮橋橋絶齊永橋大都督太安傳伏自永橋夜入中潬城周人既克南城圍中潬二旬不下|{
	河陽有三城南城北城中潬是也永橋地近三城按懷縣有永橋鎮懷縣隋唐為懷州武德縣宋白日隋大業十一年移脩武縣於永橋即今武陟縣潬徒旱翻水中沙曰潬地形志朔州有太安郡}
洛州刺史獨孤永業守金墉周主自攻之不克永業通夜辦馬槽二千周人聞之以為大軍且至而憚之九月齊右丞高阿那肱自晉陽將兵拒周師 |{
	考異曰北齊書云閏月己丑案是月癸丑朔無己丑又下有庚辰盖誤也}
至河陽會周主有疾辛酉夜引兵還|{
	還音旋又如字}
水軍焚其舟艦|{
	河水迅急泝流西歸追兵且至故焚其舟艦由陸道退還艦戶黯翻}
傳伏謂行臺乞伏貴和曰周師疲弊願得精騎二千追擊之可破也貴和不許齊王憲于翼李穆所向克捷降拔三十餘城|{
	降者迎降拔者以兵力攻拔降戶江翻下同}
皆棄而不守唯以王藥城要害令儀同三司韓正守之正尋以城降齊戊寅周主還長安庚辰齊以趙彦深為司徒斛阿列羅為司空|{
	斛阿列虜三字}


|{
	姓}
閠月車騎大將軍吳明徹將兵擊齊彭城壬辰敗齊兵數萬於呂梁|{
	敗補邁翻}
甲午周主如同州 冬十月己巳立皇子叔齊為新蔡王叔文為晉熙王 十二月辛亥朔日有食之 壬戍以王瑒為尚書左僕射|{
	瑒雉杏翻又音暢}
太子詹事吳郡陸繕為右僕射 庚午周主還長安

八年春正月癸未周主如同州辛卯如河東涑川|{
	杜預曰涑永出河東聞喜縣西南至蒲坂入河涑音速}
甲午復還同州 甲寅齊大赦乙卯齊主還鄴|{
	去年六月如晉陽今還}
二月辛酉周主命太

子廵撫西土因伐土谷渾|{
	吐從暾入聲谷音浴}
上開府儀同大將軍王軌|{
	建德四年改驃騎大將軍開府儀同三司為開府儀同大將軍仍增上開府儀同大將軍}
宮正宇文孝伯從行軍中節度皆委二人太子仰成而已|{
	仰如字又五亮翻}
齊括雜戶未嫁者悉集|{
	魏虜西凉之人沒入名為隸戶魏武入關隸戶皆在東魏後齊因之仍供厮役周平齊乃悉放諸雜戶為百姓}
有隱匿者家長坐死|{
	長知两翻}
壬申以開府儀同三司吳明徹為司空 三月壬寅周主還長安夏四月乙卯復如同州 己未上享太廟 尚書左僕射王瑒卒 |{
	考異曰陳書庚寅瑒卒案長歷是月己酉朔無庚寅陳書誤}
五月壬辰周主還長安 六月戊申朔日有食之 辛亥周主享太廟 初太子叔寶欲以左戶部尚書江總為事|{
	按五代志梁置吏部祠部度支左戶都官五兵等六尚書陳因梁制此盖左戶也部字衍}
令管記陸瑜言於吏部尚書孔奐奐謂瑜曰江有潘陸之華|{
	晉惠帝為太子潘岳陸機皆為東宫官}
而無園綺之實|{
	園公綺里季羽翼漢太子盈高帝遂不易太子}
輔弼儲宫竊有所難太子深以為恨自言於帝帝將許之奐奏曰江總文華之士今皇太子文華不少|{
	少詩沼翻}
豈藉於總如臣所見願選敦重之才以居輔導之職帝曰即如卿言誰當居此奐曰都官尚書王廓世有懿德識性敦敏可以居之太子時在側乃曰廓王泰之子不宜為太子詹事|{
	謂迴避父諱不宜居是官也}
奐曰宋朝范曄|{
	朝直遥翻}
即范泰之子亦為太子事前代不疑太子固争之帝卒以總為詹事|{
	卒子恤翻}
總斆之曾孫也|{
	江斆湛之子齊朝以風流冠冕一時斆音效}
甲寅以尚書右僕射陸繕為左僕射帝欲以孔奐代繕詔已出太子沮之而止更以晉陵太守王克為右僕射|{
	更工衡翻守音狩}
頃之總與太子為長夜之飲養良娣陳氏為女太子亟微行遊總家|{
	亟去吏翻}
上怒免總官 周利州刺史紀王康|{
	五代志義城郡古晉壽也後魏立益州世號小益州梁曰黎州西魏復曰益州又改曰利州}
驕矜無度繕脩戎器隂有異謀司録裴融諫止之康殺融丙辰賜康死 丁巳周主如雲陽 庚申齊宜陽王趙彦深卒彦深歷事累朝常參機近|{
	趙彦深事齊神武已掌機密至後主歷事六君朝直遥翻}
以温謹著稱既卒朝貴典機密者唯侍中開府儀同三司斛律孝卿一人而已其餘皆嬖倖也|{
	卒子恤翻朝直遥翻嬖卑義翻又博計翻}
孝卿羌舉之子|{
	斛律羌舉見一百五十七卷梁武帝大同三年}
比於餘人差不貪穢 秋八月乙卯周主還長安 周太子伐吐谷渾至伏俟城而還|{
	伏俟城吐谷渾國都也其地即漢西海允谷鹽池在清海西吐從暾入聲谷音浴還音旋又如字下軍還司}
宫尹鄭譯王端等|{
	周置太子宫尹盖即詹事之職}
皆有寵於太子太子在軍中多失德譯等皆預焉軍還王軌等言之於周主周主怒杖太子及譯等仍除譯等名宫臣親幸者咸被譴|{
	還從宣翻又音如字被皮義翻下同}
太子復召譯戱狎如初譯因曰殿下何時可得據天下太子悦益昵之|{
	復扶又翻又音如字昵尼質翻}
譯儼之兄孫也|{
	亂魏朝使靈太后不得良死者鄭儼也}
周主遇太子甚嚴每朝見|{
	朝直遥翻見賢遍翻}
進止與羣臣無異雖隆寒盛暑不得休息以其耆酒|{
	耆讀曰嗜}
禁酒不得至東宫有過輒加捶撻|{
	捶止橤翻}
嘗謂之曰古來太子被廢者幾人餘兒豈不堪立邪|{
	邪音耶}
乃敕東宫官屬録太子言語動作每月奏聞太子畏帝威嚴矯情脩飾由是過惡不上聞|{
	上時掌翻}
王軌嘗與小内史賀若弼言太子必不克負荷|{
	賀若虜複姓北史云北人謂忠貞為賀若魏孝文帝以其先祖有忠貞之節遂以賀若為氏若人者翻荷下可翻又如字}
弼深以為然勸軌陳之軌後因侍坐|{
	坐徂卧翻}
言於帝曰皇太子仁孝無聞恐不了陛下家事愚臣短暗不足可信陛下恒以賀若弼有文武奇才|{
	恒戶登翻}
亦常以此為憂帝以問弼對曰皇太子養德春宫|{
	太子居東宫東方主春故亦曰春宫}
未聞有過既退軌讓弼曰平生言論無所不道今者對揚|{
	對揚本於傳說召虎對答也揚稱也後人遂以面對敷奏為對揚}
何得乃爾反覆|{
	爾如此也}
弼曰此公之過也太子國之儲副豈易言事有蹉跌|{
	易以䜴翻蹉七何翻跌徒結翻}
便至㓕族本謂公密陳臧否|{
	否音鄙}
何得遂至昌言|{
	昌顯也昌言顯言也}
軌默然久之乃曰吾專心國家遂不存私計向者對衆良實非宜後軌因内宴|{
	内宴宴於宫中也}
上壽|{
	上時掌翻}
捋帝須曰|{
	捋郎括翻須與鬚同}
可愛好老公但恨後嗣弱耳先是帝問右宮伯宇文孝伯曰吾兒比來何如|{
	先悉薦翻比毗至翻}
對曰太子比懼天威更無過失罷酒帝責孝伯曰公常語我云太子無過今軌有此言公為誑矣|{
	語牛倨翻誑居况翻}
孝伯再拜曰父子之際人所難言臣知陛下不能割慈忍愛遂爾結舌|{
	孝伯此言亦不可謂之不忠切也}
帝知其意默然久之乃曰朕已委公矣公其勉之王軌驟言於帝曰皇太子非社稷主普六茹堅貌有反相|{
	不從容而言之為驟言相息亮翻}
帝不悦曰必天命有在將若之何楊堅聞之甚懼深自晦匿帝深以軌等言為然|{
	為太子得位殺軌等張本}
但漢王贊次長|{
	長知两翻}
又不才餘子皆幼故得不廢|{
	史言周武帝明於知子而不廢太子之由}
丁卯以司空吳明徹為南兖州刺史|{
	五代志江都郡梁置南兖州後齊改為東廣州陳復曰南兖}
齊主如晉陽營邯鄲宫|{
	此二事也既如晉陽又營宫於邯郸以趙故都也其地在隋唐臨洛縣邯鄲音寒丹}
九月戊戍以皇子叔彪為淮南王 周主謂羣臣曰朕去歲屬有疾疹|{
	屬之欲翻疹丑刃翻丁度曰熱病也}
遂不得克平逋寇前入齊境備見其情彼之行師殆同兒戱况其朝廷昏亂|{
	朝直遥翻}
政由羣小百姓嗷然朝不謀夕天與不取恐貽後悔前出河外直為拊背未扼其喉|{
	謂去年河隂之役漢婁敬曰今與人闘不扼其吭而拊其背未能全勝}
晉州本高歡所起之地|{
	高歡起兵晉州事始見一百五十四卷梁武帝中大通二年}
鎭攝要重|{
	攝總持也}
今往攻之彼必來援吾嚴軍以待擊之必克然後乘破竹之勢皷行而東足以窮其巢穴混同文軌|{
	記曰今天下書同文車同軌}
諸將多不願行|{
	將即亮翻}
帝曰機不可失有沮吾軍者當以軍法裁之|{
	沮在呂翻}
冬十月己酉周主自將伐齊|{
	將即亮翻}
以越王盛公亮隨公楊堅為右三軍譙王倫大將軍竇泰廣化公丘崇為左三軍|{
	廣化郡公五代志河池縣後魏曰廣化置廣化郡}
齊王憲陳王純為前軍亮導之子也丙辰齊主獵於祁連池癸亥還晉陽先是晉州行臺左丞張延雋公直勤敏儲偫有備|{
	先悉薦翻偫直里翻}
百姓安業疆場無虞諸嬖倖惡而代之|{
	場音亦嬖卑義翻又博計翻惡烏路翻}
由是公私煩擾周主至晉州軍於汾曲|{
	汾曲汾水之曲在平陽南水經汾水南過平陽縣東又南過臨汾縣東又屈從縣南西流是汾曲也}
遣齊王憲將兵二萬守雀鼠谷|{
	水經汾水南過冠爵津在介休縣西南俗謂之雀鼠谷數十里間道隘水左右悉給偏梁閣道累石就路縈帶巖側或去水一丈或高六丈上戴山阜下臨絶澗俗謂之魯般橋盖通古之津隘又在今之地險也將即亮翻}
陳王純步騎二萬守千里徑|{
	千里徑亦當在平陽北要路之一也杜佑曰汾州界北接太原當千里徑騎奇寄翻下同}
鄭公逹奚震步騎一萬守統軍川|{
	統軍川地闕}
大將軍韓明步騎五千守齊子嶺|{
	齊子嶺在邵郡東}
焉氏公尹升步騎五千守皷鐘鎮|{
	焉氏讀曰燕支燕平聲焉氏縣公也地形志凉州番和郡有燕支縣因燕支山以名縣隋併入番和縣水經注教水出垣縣北教山其水南歷皷鐘上峽又南流歷皷鐘川西南有治官世人謂之鼓鐘城山海經曰鼓鐘之山帝臺之所以觴百神即是山也垣縣後魏於此置邵郡}
凉城公辛韶步騎五千守蒲津關|{
	此凉城郡公也後魏立凉城郡於漢沃陽縣鹽澤北七里池西有舊城俗謂之凉城郡取名也按後魏自六鎮反亂此地皆棄之不能有後周特取郡名以封爵耳漢魏以後五等之封皆無實土其來久矣蒲津關在蒲坂因津濟處以立關漢書武帝元封六年立蒲津關}
趙王招步騎一萬自華谷攻齊汾州諸城|{
	水經涑水出河東聞喜縣東山黍葭谷俗謂之華谷即齊將斛律光取周汾北以進築者也}
柱國宇文盛步騎一萬守汾水關|{
	汾水關當在霍邑縣南臨汾縣北自此以上凡言守者皆以斷齊援兵之路獨守蒲津關者為後繼括地志汾州靈石縣有雀鼠谷汾水關}
遣内史王誼監諸軍攻平陽城|{
	監工衡翻}
齊行臺僕射海昌王尉相貴嬰城拒守|{
	尉紆勿翻}
甲子齊集兵晉祠|{
	地形志晉陽有晉王祠}
庚午齊主自晉陽帥諸軍趣晉州|{
	帥讀曰率下同趣七喻翻}
周主日自汾曲至城下督戰城中窘急|{
	窘巨隕翻}
庚午行臺左丞侯子欽出降於周|{
	降戶江翻}
壬申晉州刺史崔景嵩守北城夜遣使請降於周王軌帥衆應之未明周將北海段文振杖槊與數十人先登|{
	使疏吏翻帥讀曰率將即亮翻槊色角翻}
與景嵩同至尉相貴所拔佩刀劫之城上鼓譟齊兵大潰遂克晉州虜相貴及甲士八千人齊主方與馮淑妃獵於天池 |{
	考異曰馮淑妃傳云獵於三堆今從高阿那肱傳余按宋白續通典嵐州静樂縣本三堆也天池亦在縣界}
晉州告急者自旦至午驛馬三至右丞相高阿那肱曰大家正為樂|{
	樂音洛}
邊鄙小小交兵乃是常事何急奏聞至暮使更至|{
	使疏吏翻}
云平陽已陷乃奏之齊主將還淑妃請更殺一圍齊主從之|{
	按齊主獵於祁連池癸亥還晉陽甲子即集兵庚午自晉陽帥兵趣晉州壬申晉州陷時齊主方獵於天池馮淑妃請更殺一圍審如是則晉州陷之日齊主猶在天池天池今在憲州静樂縣至晉陽一百七十餘里自晉陽南至晉州又五百有餘里齊主既以庚午違晉陽而南無緣復北至天池竊謂獵祁連池與獵天池其是一事北人謂天為祁連故天池亦謂之祁連池通鑑稡集諸書成一家言自癸亥排日書至庚午晉陽是據北齊紀書高阿那肱不急奏邊報是據阿那肱傳書請更殺一圍是據馮淑妃傳合三者而書之不能不相牴牾又馮淑妃傳以為獵於三堆三堆在肆州永安郡平寇縣界亦在晉陽北}
周齊王憲攻拔洪洞永安二城|{
	二城皆在晉州北洪洞城在楊縣取城北洪洞嶺名之永安古彘縣地隋改曰霍邑}
更圖進取齊人焚橋守險軍不得進乃屯永安使永昌公椿屯雞栖原|{
	永昌郡公五代志巴東郡大昌縣後周置永昌郡雞栖原在永安北}
伐柏為菴以立營|{
	菴烏含翻漢皇甫規親入菴廬廵視三軍}
椿廣之弟也癸酉齊主分軍萬人向千里徑|{
	壬申晉州陷癸酉齊軍已向千里徑則知晉州陷不與獵天池同日明矣}
又分軍出汾水關自帥大軍上雞栖原|{
	上時掌翻}
宇文盛遣人告急齊王憲自救之齊師退盛追擊破之俄而椿告齊師稍逼憲復還救之|{
	復扶又翻}
與齊對陳至夜不戰|{
	陳讀曰陣}
會周主召憲還|{
	還從宣翻又如字}
憲引兵夜去齊人見柏菴在不之覺明日始知之齊主使高阿那肱將前軍先進仍節度諸軍|{
	將即亮翻}
甲戍周以上開府儀同大將軍安定梁士彦為晉州刺史留精兵一萬鎮之十一月己卯齊主至平陽周主以齊兵新集聲勢甚盛且欲西還以避其鋒開府儀同大將軍宇文忻諫曰以陛下之聖武乘敵人之荒縱何患不克若使齊得令主君臣協力雖湯武之勢未易平也|{
	易以䜴翻}
今主暗臣愚士無鬬志雖有百萬之衆實為陛下奉耳軍正京兆王紘曰|{
	時因行軍倣漢制置軍正之官不常置也}
齊失紀綱於兹累世|{
	世祖嗣立齊政不綱今再世矣}
天奬周室一戰而扼其喉取亂侮亡正在今日|{
	取亂侮亡商仲虺之誥}
釋之而去臣所未諭周主雖善其言竟引軍還忻貴之子也|{
	宇文貴本朔方人徙京兆仕周為大司馬非周之族也}
周主留齊王憲為後拒齊師追之憲與宇文忻各將百騎與戰斬其驍將賀蘭豹子等|{
	將即亮翻下同騎奇寄翻驍堅堯翻}
齊師乃退憲引軍度汾追及周主於玉壁齊師遂圍平陽晝夜攻之城中危急樓堞皆盡|{
	樓城上敵樓堞城短垣堞徒協翻}
所存之城尋仭而已|{
	六尺為尋七尺為仭}
或短兵相接|{
	槍槊為短兵}
或交馬出入外援不至衆皆震懼梁士彦忼慨自若|{
	忼苦朗翻}
謂將士曰死在今日吾為爾先於是勇烈齊奮呼聲動地|{
	呼火故翻}
無不一當百齊師少却|{
	少詩沼翻}
乃令妻妾軍民婦女晝夜脩城三日而就周主使齊王憲將兵六萬屯涑川遥為平陽聲援齊人作地道攻平陽城陷十餘步將士乘勢欲入齊主敕且止召馮淑妃觀之淑妃粧點不時至周人以木拒塞之|{
	塞悉則翻}
城遂不下舊俗相傳晉州城西石上有聖人跡淑妃欲往觀之齊主恐弩矢及橋乃抽攻城木造遠橋|{
	舊橋近城别造遠橋}
齊主與淑妃度橋橋壞至夜乃還癸巳周主還長安甲午復下詔以齊人圍晉州更帥諸軍擊之丙申縱齊降人使還|{
	帥讀曰率縱之使還使齊師知周師將復至而懼亦以堅晉州守者之心降戶江翻}
丁酉周主發長安|{
	還長安僅三日復出師明引歸者欲使齊師疲於攻平陽而後取之}
壬寅濟河與諸軍合十二月丁未周主至高顯|{
	高顯盖近涑川}
遣齊王帥所部先向平陽戊申周主至平陽庚戍諸軍總集凡八萬人稍進逼城置陳東西二十餘里先是齊人恐周師猝至於城南穿塹自喬山屬於汾水齊主大出兵陳於塹北|{
	陳讀曰陣先悉薦翻塹七艶翻屬之欲翻喬山當在平陽城西}
周主命齊王憲馳往觀之憲復命曰易與耳|{
	易以䜴翻}
請破之而後食|{
	左傳齊晉戰于鞌齊侯曰余姑翦㓕此而後朝食}
周主悦曰如汝言吾無憂矣周主乘常御馬從數人廵陳所至輒呼主帥姓名慰勉之|{
	帥所類翻}
將士喜於見知咸思自奮將戰有司請換馬周主曰朕獨乘良馬欲何之周主欲薄齊師礙塹而止自旦至中相持不决齊主謂高阿那肱曰戰是邪不戰是邪|{
	邪音耶}
阿那肱曰吾兵雖多堪戰不過十萬病傷及繞城樵㸑者復三分居一|{
	復扶又翻}
昔攻玉壁援軍來即退|{
	攻玉壁事見一百五十九卷梁武帝中大同元年}
今日將士豈勝神武時邪|{
	高歡諡神武皇帝}
不如勿戰却守高梁橋|{
	地形志晉州平陽縣有高梁城水經注汾水逕高梁故城西故高梁之墟也晉文公害懷公於此汾水又南過平陽縣東新唐志晉州臨汾縣東北十里有高梁堰}
安吐根曰一撮許賊馬上刺取擲著汶水中耳|{
	一撮言其少也撮倉括翻刺七亦翻著直畧翻不知兵勢而輕敵大言未有不敗者也}
齊主意未决諸内參曰彼亦天子我亦天子彼尚能遠來我何為守塹示弱齊主曰此言是也於是填塹南引周主大喜勒諸軍擊之兵纔合齊主與馮淑妃並騎觀戰東偏少却淑妃怖曰軍敗矣|{
	騎奇寄翻少詩沼翻怖普故翻惶懼也}
録尚書事城陽王穆提婆曰大家去大家去齊主即以淑妃奔高梁橋開府儀同三司奚長諫曰半進半退戰之常體今兵衆全整未有虧傷陛下捨此安之馬足一動人情駭亂不可復振願速還安慰之|{
	復扶又翻還從宣翻又音如字下同}
武衛張常山自後至|{
	武衛屬左右武衛將軍}
亦曰軍尋收訖甚完整圍城兵亦不動至尊宜囘不信臣言乞將内參往視|{
	將領也與也偕也攜也挾也}
齊主將從之穆提婆引齊主肘曰此言難信齊主遂以淑妃北走齊師大潰死者萬餘人軍資器械數百里間委棄山積|{
	齊人所棄皆為稽胡所取後周人由此討稽胡}
安德王延宗獨全軍而還|{
	延宗在亂能整未易才也惜大厦將顛非一木所支耳}
齊主至洪洞淑妃方以粉鏡自玩|{
	施粉添粧臨鏡自玩也}
後聲亂唱賊至於是復走|{
	復扶又翻}
先是齊主以淑妃為有功勲將立為左皇后遣内參詣晉陽取皇后服御褘翟等|{
	五代志梁制皇后謁廟服袿䙱大衣蓋嫁服也皁上皁下親蠶則青上縹下齊制皇后助祭朝會以褘衣祠郊禖以䄖狄小宴以闕狄親蠶以鞠衣禮見皇帝以展衣宴居以緑衣六服俱有蔽織成緄帶周制皇后翟衣六祠郊禖朝享則翬衣素質五色祭隂社朝命婦則衣青質五色祭羣小祀受獻璽則鷩衣赤色采桑則鴇衣黃色從皇帝見賓客聽女教則衣白色食命婦歸寧則衣玄色隋制皇后褘衣深青織成為之為翬翟之形素質五色十二等先悉薦翻褘許韋翻袿涓畦翻釋名婦人上服曰袿具下垂者䙱未欲翻縹匹小翻褕音遥展與襢同陟戰翻沈將輦翻緄古本翻與褕同音鷩必列翻赤雉也鴇補抱翻勑角翻雉名直質翻}
至是遇於中塗齊主為按轡|{
	為于偽翻}
命淑妃著之然後去|{
	史言齊師之敗皆由馮小憐以婦人從軍國之禍也齊主既敗而寵其所嬖以速亡著職畧翻}
辛亥周主入平陽梁士彦見周主持周主須而泣曰臣幾不見陛下周主亦為之流涕|{
	史叙後周君臣相與之情須與鬚同幾居依翻為于偽翻上主為下善為同}
周主以將士疲弊欲引還|{
	將息亮翻還從宣翻又音如字}
士彦叩馬諫曰今齊師遁散衆心皆動因其懼而攻之其勢必舉周主從之執其手曰余得晉州為平齊之基若不固守則大事不成朕無前憂唯慮後變汝善為我守之|{
	用兵而能慮後患者善師者也}
遂帥諸將追齊師|{
	帥讀曰率}
諸將固請西還周主曰縱敵患生卿等若疑朕將獨往諸將乃不敢言癸丑至汾水關齊主入晉陽憂懼不知所之甲寅齊大赦齊主問計於朝臣皆曰宜省賦息役以慰民心收遺兵背城死戰以安社稷|{
	朝直遥翻背蒲妹翻}
齊主欲留安德王延宗廣寧王孝珩守晉陽|{
	珩音行}
自向北朔州|{
	魏孝昌中改懷朔鎭為朔州本漢五原郡地尋即䧟沒而朔州寄治并州界後齊置朔州於古馬邑城於西河郡置南朔州故謂馬邑為北朔州新唐志曰朔州本治善陽建中中馬遂徙治馬邑大元以朔州置順義節度領鄯陽窟谷二縣而以馬邑縣置固州}
若晉陽不守則奔突厥|{
	厥九勿翻}
羣臣皆以為不可帝不從開府儀同三司賀拔伏恩等宿衛近臣三十餘人西奔周軍周主封賞各有差高阿那肱所部兵尚一萬守高壁|{
	高壁嶺名在雀鼠谷南括地志汾州靈石縣有高壁嶺杜佑曰在縣東南宋白曰靈石縣東南有高壁嶺雀鼠谷汾水關皆汾西險固之所}
餘衆保洛女砦|{
	砦與寨同柴夬翻下同}
周主引軍向高壁阿那肱望風退走齊王憲攻洛女砦拔之有軍士告阿那肱招引西軍齊主令侍中斛律孝卿檢校孝卿以為妄還至晉陽阿那肱腹心復告阿那肱謀反|{
	復扶又翻}
又以為妄斬之乙卯齊主詔安德王延宗廣寜王孝珩募兵延宗入見|{
	珩音行見賢遍翻}
齊主告以欲向北朔州|{
	後魏太和中置朔州於定襄故城高齊天保於馬邑西南置朔州相去三百八十里故定襄古城之朔州有北朔州之稱}
延宗泣諫不從密遣左右先送皇太后太子於北朔州丙辰周主與齊王憲會於介休|{
	介休縣屬西河郡}
齊開府儀同三司韓建業舉城降以為上柱國封郇公|{
	降戶江翻郇音荀郇古國名}
是夜齊主欲遁去諸將不從|{
	將即亮翻}
丁巳周師至晉陽齊主復大赦|{
	復扶又翻}
改元隆化以安德王延宗為相國并州刺史總山西兵|{
	并卑經翻鄴都謂并州之地為山西}
謂曰并州兄自取之兒今去矣延宗曰陛下為社稷勿動臣為陛下出死力戰必能破之|{
	為于偽翻}
穆提婆曰至尊計已成王不得輒沮|{
	沮在呂翻}
齊主乃夜斬五龍門而出欲奔突厥從官多散|{
	厥九勿翻從才用翻下同}
領軍梅勝郎叩馬諫乃囘向鄴時唯高阿那肱等十餘騎從|{
	騎奇寄翻}
廣寧王孝珩襄城王彦道繼至得數十人與俱穆提婆西奔周軍陸令萱自殺家屬皆誅沒周主以提婆為柱國宜州刺史|{
	五代志京兆郡華原縣後魏置北雍州西魏改為宜州}
下詔諭齊羣臣曰若妙盡人謀深逹天命官榮爵賞各有加隆或我之將卒逃逸彼朝|{
	將即亮翻朝直遥翻}
無問貴賤皆從蕩滌自是齊臣降者相繼|{
	降戶江翻}
初齊高祖為魏丞相|{
	齊尊高歡廟號曰高祖相息亮翻}
以唐邕典外兵曹太原白建典騎兵曹|{
	騎奇寄翻}
皆以善書計工簿帳受委任及齊受禪諸司咸歸尚書唯二曹不廢更名二省|{
	更工衡翻}
邕官至録尚書事建官至中書令常典二省世稱唐白邕兼領度支與高阿那肱有隙阿那肱譖之齊主敕侍中斛律孝卿總知騎兵度支|{
	度徒洛翻}
孝卿事多專决不復詢禀|{
	復扶又翻}
邕自以宿習舊事為孝卿所輕意甚|{
	者受抑而氣不得舒也}
及齊主還鄴邕遂留晉陽并州將帥請於安德王延宗曰王不為天子諸人實不能為王出死力|{
	將即亮翻帥所類翻為于偽翻}
延宗不得已戊午即皇帝位下詔曰武平孱弱|{
	曰武平者稱齊主年號孱士顔翻又士眼翻}
政由宫豎斬關夜遁莫知所之王公卿士猥見推逼|{
	猥遝也}
今祗承寶位大赦改元德昌以晉昌王唐邕為宰相齊昌王莫多婁敬顯沭陽右衛大將軍段暢開府儀同三司韓骨胡等為將帥敬顯貸文之子也|{
	此皆齊所封郡王也五代志西城郡石泉縣舊置晉昌郡蘄春郡蘄春縣後齊置齊昌郡東海郡沭陽縣東魏置沭陽郡莫多婁貸文戰死事見一百五十八卷梁武帝大同四年沭音術}
衆聞之不召而至者前後相屬延宗發府藏及後宫美女以賜將士|{
	屬之欲翻藏徂浪翻將即亮翻}
籍沒内參十餘家齊主聞之謂近臣曰我寧使周得并州不欲安德得之左右曰理然延宗見士卒皆親執手稱名流涕嗚咽衆爭為死|{
	為于偽翻}
童兒女子亦乘屋攘袂投甎石以禦敵|{
	乘登也}
己未周主至晉陽 |{
	考異曰周書武帝紀丁巳大軍次并州又云己未軍次并州盖丁巳前軍至己未帝乃至也}
庚申齊主入鄴周師圍晉陽四合如黑雲|{
	周戎衣及旗幟皆黑且兵多故如黑雲}
安德王延宗命莫多婁敬顯韓骨胡拒城南和阿干子段暢拒城東自帥衆拒齊王憲於城北|{
	帥讀曰率}
延宗素肥前如偃後如伏人常笑之至是奮大矟往來督戰|{
	矟所角翻}
勁捷若飛所向無前和阿干子段暢以千騎奔周軍|{
	騎奇寄翻}
周主攻東門際昏遂入之進焚佛寺延宗敬顯自門入夾擊之周師大亂爭門相填壓塞路不得進齊人從後斫刺死者二千餘人|{
	塞悉則翻刺七亦翻}
周主左右略盡自拔無路承御上士張壽牽馬首|{
	承御上士盖侍衛左右之官}
賀拔伏恩以鞭拂其後 |{
	考異曰北齊書安德王延宗傳作佛恩今從周齊帝紀}
崎嶇得出|{
	崎丘奇翻嶇音區}
齊人奮擊幾中之|{
	幾居依翻又巨希翻近也中竹仲翻}
城東道阨曲|{
	阨與阸同烏懈翻}
伏恩及降者皮子信導之僅得免|{
	降戶江翻}
時已四更|{
	夜分五更四更丁夜也更工衡翻}
延宗謂周主為亂兵所殺使於積尸中求長鬛者不得|{
	鬛良涉翻鬚也}
時齊人既捷入坊飲酒盡醉臥延宗不復能整|{
	復扶又翻}
周主出城饑甚欲遁去諸將亦多勸之還|{
	將即亮翻還從宣翻又音如字}
宇文忻勃然進曰陛下自克晉州乘勝至此今偽主奔波關東響震自古行兵未有若斯之盛昨日破城將士輕敵微有不利何足為懷丈夫當死中求生敗中取勝今破竹之勢已成柰何棄之而去齊王憲柱國王誼亦以為去必不免段暢等又盛言城内空虛周主乃駐馬鳴角收兵俄頃復振|{
	散兵復聚則摧沮之勢振迅而起復扶又翻又音如字}
辛酉旦還攻東門克之|{
	還復也}
延宗戰力屈走至城北周人擒之周主下馬執其手延宗辭曰死人手何敢廹至尊周主曰两國天子非有怨惡直為百姓來耳|{
	言為救民而來為于偽翻}
終不相害勿怖也|{
	怖蒲故翻}
使復衣帽而禮之|{
	五代志㡌自天子下及士人通冠之盖常服也然亦有白紗烏紗之異又有繒皂雜紗為之者}
唐邕等皆降於周|{
	降戶江翻}
獨莫多婁敬顯奔鄴齊主以為司徒延宗初稱尊號遣使脩啟於瀛州刺史任城王湝|{
	後魏置瀛州於河間使疏吏翻下同任音壬湝戶皆翻又音皆}
曰至尊出奔宗廟事重群公勸廹權主號令事寧終歸叔父湝曰我人臣何容受此啟執使者送鄴壬戍周主大赦削除齊制收禮文武之士鄴伊婁謙聘於齊|{
	周遣伊婁謙聘齊事見去年二月此上不應有鄴字蓋初字之誤也}
其參軍高遵以情輸於齊|{
	言周將伐齊使謙來觀舋}
齊人拘之於晉陽周主既克晉陽召謙勞之|{
	勞力到翻下親勞同}
執遵付謙任其報復謙頓首請赦之周主曰卿可聚衆唾面使其知愧謙曰以遵之罪又非唾面可責|{
	唾湯卧翻}
帝善其言而止謙待遵如初

臣光曰賞有功誅有罪此人君之任也高遵奉使異國漏泄大謀斯叛臣也|{
	使疏吏翻}
周高祖不自行戮乃以賜謙使之復怨失政刑矣孔子謂以德報怨者何以報德為謙者宜辭而不受歸諸有司以正典刑乃請而赦之以成其私名美則美矣亦非公義也

齊主命立重賞以募戰士而竟不出物廣寧王孝珩請使任城王湝將幽州道兵入土門|{
	按新唐志井徑故關一名土門關珩音行任音壬湝音皆又古皆翻將即亮翻下同}
揚聲趣并州|{
	趣七喻翻下同}
獨孤永業將洛州道兵入潼關揚聲趣長安|{
	後魏自平城遷都洛陽置司州孝武西入關東魏北都鄴以鄴為司州以洛陽為洛州}
臣請將京畿兵出滏口鼓行逆戰|{
	滏口滏水之口山海經滏水出神茵之山圖經泉源瀵涌若湯焉滏音釡}
敵聞南北有兵自然逃潰又請出宫人珍寶賞將士齊主不悦斛律孝卿請齊主親勞將士為之撰辭|{
	將即亮翻勞力到翻為于偽翻撰士免翻}
且曰宜忼慨流涕以感激人心齊主既出臨衆將令之不復記所受言|{
	令讀如軍令之令復扶又翻}
遂大笑左右亦笑將士怒曰身尚如此吾輩何急皆無戰心於是自大丞相以下太宰三師大司馬大將軍三公等官|{
	後齊制官多循後魏大丞相太宰位望最為崇重太師太傅太保是為三師擬古上公非勲德不居次有大司馬大將軍是為二大並典司武事次置太尉司徒司空是為三公三師二大三公府三門當中開黄閣設内屏其階皆正一品}
並增員而授或三或四不可勝數|{
	勝音升數所矩翻計也舊所具翻}
朔州行臺僕射高勱將兵侍衛太后太子自土門道還鄴|{
	勱音邁}
時䆠官儀同三司苟子溢猶恃寵縱暴民間雞彘縱鷹犬噬取之勱執以徇將斬之太后救之得免或謂勱曰子溢之徒言成禍福獨不慮後患邪勱攘袂曰今西寇已據并州|{
	周在齊之西故謂之西寇}
逹官率皆委叛|{
	有位任而光顯於時者為逹官委棄也委叛者言棄官而叛去}
正坐此輩濁亂朝廷|{
	朝直遥翻}
若得今日斬之明日受誅亦無所恨勱岳之子也|{
	高岳從高歡起兵有功}
甲子齊太后至鄴丙寅周主出齊宫中珍寶服玩及宫女二千人班賜將士加立功者官爵各有差|{
	將即亮翻}
周主問高延宗以取鄴之策辭曰此非亡國之臣所及強問之|{
	強其两翻}
乃曰若任城王據鄴臣不能知|{
	任音壬}
若今主自守陛下兵不血刃癸酉周師趣鄴|{
	趣七喻翻}
命齊王憲先驅以上柱國陳王純為并州總管齊主引諸貴臣入朱雀門|{
	朱雀門鄴宫城正南門也}
賜酒食問以禦周之策人人異議齊主不知所從是時人情忷懼莫有鬬心朝士出降書夜相屬|{
	忷許勇翻朝直遥翻屬之欲翻}
高勱曰今之叛者多是貴人至於卒伍猶未離心請追五品已上家屬置之三臺|{
	勱音邁考之齊制五品已上謂自尚書郎中書侍郎諫議大夫九寺少卿給事黄門侍郎通直散騎常侍尚書左右丞三公府長史諮議參軍太子三卿直閣將軍東宫正都督已上也三臺魏武帝所建齊文宣帝又增崇之時改為寺}
因脅之以戰若不捷則焚臺此曹顧惜妻子必當死戰且王師頻北賊徒輕我今背城一决|{
	背蒲妹翻}
理必破之齊主不能用望氣者言當有革易齊主引尚書令高元海等議依天統故事禪位皇太子|{
	天統禪位事見一百六十九卷世祖天嘉六年}


資治通鑑卷一百七十二
