資治通鑑卷二百七十五
宋 司馬光 撰

胡三省 音註

後唐紀四|{
	起柔兆閹茂四月盡彊圉大淵獻六月凡一年有奇}


明宗聖德和武欽孝皇帝上之下

天成元年夏四月丁亥朔嚴辦將發|{
	凡天子將出侍中奏中嚴外辦此時未必能爾沿襲舊來嚴辦之言而言之耳}
騎兵陳於宣仁門外|{
	唐昭宗天祐二年勅改東都延喜門為宣仁門又唐六典東都東城在皇城之東東曰宣仁門南曰承福門陳讀曰陣下同}
步兵陳於五鳳門外從馬直指揮使郭從謙不知睦王存乂已死|{
	存乂養郭從謙為假子及其被誅事並見上卷本年二月時諸王不出閤者皆在禁中故存乂死而從謙不知從才用翻}
欲奉之以作亂帥所部兵|{
	帥讀曰率下同}
自營中露刃大呼|{
	呼火故翻}
與黄甲兩軍攻興教門|{
	唐昭宗之遷洛也改延喜門為宣政門重明門為興教門五鳳門蓋宫城南門也唐六典曰洛陽皇城南面三門中曰應天左曰興教右曰光政}
帝方食聞變帥諸王及近衛騎兵擊之逐亂兵出門時蕃漢馬步使朱守殷將騎兵在外帝遣中使急召之欲與同擊賊守殷不至引兵憩于北邙茂林之下|{
	憩去例翻息也邙莫郎翻}
亂兵焚興教門緣城而入近臣宿將皆釋甲潛遁|{
	李紹榮必已遁矣}
獨散員都指揮使李彦卿及宿衛軍校何福進王全斌等十餘人力戰俄而帝為流矢所中|{
	李彦卿蓋即苻彦卿存審之子散悉亶翻校戶教翻中竹仲翻斌音彬}
鷹坊人善友扶帝自門樓下至絳霄殿廡下|{
	鷹坊唐時五坊之一也姓譜善姓也堯師善卷門樓興教門樓廡罔甫翻}
抽矢渴懣求水皇后不自省視遣宦者進酪|{
	懣音悶省昔景翻酪歷各翻乳漿也凡中矢刃傷血悶者得水尚可活飲酪是速死也}
須臾帝殂|{
	年四十二}
李彦卿等慟哭而去左右皆散善友斂廡下樂器覆帝尸而焚之|{
	覆敷又翻自此以上至是年正月書帝者皆指言莊宗莊宗好優而斃于郭門高好樂而焚以樂器故歐陽公引君以始此以此終之言以論其事示戒深矣}
彦卿存審之子福進全斌皆太原人也|{
	李彦卿後復姓苻與何福進王全斌皆以功名自見}
劉后囊金寶繫馬鞍與申王存渥及李紹榮引七百騎焚嘉慶殿自師子門出走通王存確雅王存紀奔南山|{
	洛陽之南入伊川皆大山}
宫人多逃散朱守殷入宫選宫人三十餘人各令自取樂器珍玩内於其家於是諸軍大掠都城是日李嗣源至甖子谷 |{
	考異曰莊宗實録云今上至鄭州聞變今從明宗實録余嘗按甖子谷在鄭州境}
聞之慟哭謂諸將曰主上素得士心正為羣小蔽惑至此今我將安歸乎戊子朱守殷遣使馳白嗣源以京城大亂諸軍焚掠不已願亟來救之乙丑嗣源入洛陽止於私第禁焚掠拾莊宗骨於灰燼之中而殯之嗣源之入鄴也前直指揮使平遥侯益脱身歸洛陽|{
	前直指揮使領上前直衛之兵劉昫曰平遥即漢平陶縣魏避國諱改陶為遥唐屬汾州宋白曰後魏以太武帝名燾改平陶為平遥}
莊宗撫之流涕至是益自縛請罪嗣源曰爾為臣盡節又何罪也使復其職嗣源謂朱守殷曰公善巡徼以待魏王|{
	徼吉弔翻言善巡徼宫闕及皇城内外坊市以待魏王繼岌繼岌莊宗嫡長子也西征而還未至示若待其至而嗣位然}
淑妃德妃在宫供給尤宜豐備|{
	韓淑妃伊德妃先在晉陽宫蓋莊宗都洛之後迎至洛宫及其遭變不從劉后出奔時在宫中也按淑妃韓氏本莊宗元妃衛國夫人也德妃伊氏次妃燕國夫人也劉后之次在三越次而正位中宫雖莊宗之過亦郭崇韜希指迎合之罪也五代會要曰同光二年十二月冊德妃淑妃以宰臣豆盧革韋說為冊使出應天門外登輅車鹵簿鼓吹前導至右永福門降車入右銀臺門至淑妃宫受冊於内文武百官立班稱賀通鑑書二年二月冊劉后蓋冊后之後至十二月冊二妃也}
吾俟山陵畢社稷有奉則歸藩為國家捍禦北方耳|{
	歸藩言欲歸真定為於偽翻}
是日豆盧革帥百官上牋勸進|{
	下之於上不從其令而從其意帥讀曰率上時掌翻}
嗣源面諭之曰吾奉詔討賊不幸部曲叛散欲入朝自訴又為紹榮所隔披猖至此吾本無它心諸君遽爾見推殊非相悉|{
	悉息七翻諳也究也詳也盡也}
願勿言也革等固請嗣源不許李紹榮欲奔河中就永王存霸從兵稍散庚寅至平陸|{
	從才用翻唐書地理志曰括地志陜州河北縣本漢大陽縣天寶元年太守李齊物開三門以利漕運得古刃有篆文曰平陸因更河北縣為平陸縣九域志縣在陜州北五里隔大河}
止餘數騎為人所執折足送洛陽|{
	折而設翻}
存霸亦帥衆千人棄鎮奔晉陽 辛卯魏王繼岌至興平聞洛陽亂復引兵而西|{
	復扶又翻}
謀保據鳳翔 向延嗣至鳳翔以莊宗之命誅李紹琛|{
	莊宗已殂故不書帝而以廟號書之也李紹琮反於蜀被擒見上卷本年三月}
初莊宗命呂鄭二内養在晉陽一監兵一監倉庫|{
	監工銜翻}
自留守張憲以下皆承應不暇及鄴都有變又命汾州刺史李彦超為北都巡檢彦超彦卿之兄也莊宗既殂推官河間張昭遠勸張憲奉表勸進憲曰吾一書生自布衣至服金紫皆出先帝之恩豈可偷生而不自愧乎昭遠泣曰此古人之事公能行之忠義不朽矣|{
	張昭遠儒者也故勉成張憲之志節其後昭遠避漢高祖名止名昭}
有李存沼者莊宗之近屬 |{
	考異曰唐愍帝實録符彦超傳云皇弟存沼薛史歐陽史彦超傳作存霸莊宗列傳薛史張憲傳但云李存沼按莊宗弟無名存沼者存霸自河中衣僧服而往非今日傳莊宗之命者也或者武皇之姪莊宗之弟别無所據不敢决定故但云近屬余按莊宗諡光聖神閔皇帝唐愍帝實録即莊宗實録也愍閔字通}
自洛陽犇晉陽矯傳莊宗之命隂與二内養謀殺憲及彦超據晉陽拒守彦超知之密告憲欲先圖之憲曰僕受先帝厚恩不忍為此徇義而不免於禍乃天也彦超謀未决壬辰夜軍士共殺二内養及存沼於牙城因大掠達旦憲聞變出犇忻州|{
	九域志太原府東北至忻州二百里此以宋氏徙府後言也}
會嗣源移書至彦超號令士卒城中始安遂權知太原軍府 百官三牋請嗣源監國 |{
	考異曰監國本太子之事非官非爵然五代唐明宗潞王周太祖皆嘗監國漢太后令曰中外事取監國處分又誥曰監國可即皇帝位是時直以監國為稱號也今從之}
嗣源乃許之甲午入居興聖宫|{
	按是時莊宗之殯在西宫興聖宫在西宫之東按薛史莊宗即位於魏州以子繼岌充北都留守興聖宫使及平定河南充東京留守興聖宫使則東京北都皆有興聖宫宋白所記見前}
始受百官班見|{
	示即真之漸見賢遍翻}
下令稱教百官稱之曰殿下莊宗後宫存者猶千餘人宣徽使選其美少者數百獻於監國|{
	少諸沼翻}
監國曰奚用此為對曰宫中職掌不可闕也監國曰宫中職掌宜諳故事|{
	諳音烏含翻}
此輩安知乃悉用老舊之人補之其少年者皆出歸其親戚無親戚者任其所適蜀中所送宫人亦準此 乙未以中門使安重誨為樞密使|{
	安重誨本成德軍中門使監國所親任者也}
鎮州别駕張延朗為副使延朗開封人也仕梁為租庸吏|{
	按歐史張延朗仕梁以租庸史為鄆州糧料使明宗克鄆州得之復以為糧料使後徙鎮宣武成德以為元從孔目官蓋由此選為鎮州别駕也}
性纎巧善事權貴以女妻重誨之子|{
	妻七細翻}
故重誨引之監國令所在訪求諸王通王存確雅王存紀匿民間或密告安重誨重誨與李紹真謀曰今殿下既監國典喪諸王宜早為之所以壹人心殿下性慈不可以聞乃密遣人就田舍殺之後月餘監國初聞之切責重誨傷惜久之劉皇后與申王存渥奔晉陽在道與存渥私通存渥至晉陽李彦超不納走至風谷|{
	風谷恐當作嵐谷唐長安三年分宜芳縣置嵐谷縣屬嵐州}
為其下所殺明日永王存霸亦至晉陽從兵逃散俱盡|{
	從才用翻}
存霸削髪僧服謁李彦超願為山僧幸垂庇護軍士爭欲殺之彦超曰六相公來當奏取進止|{
	存霸第六}
軍士不聽殺之於府門之碑下劉皇后為尼於晉陽監國使人就殺之薛王存禮及莊宗幼子繼嵩繼潼繼蟾繼嶤|{
	嶤倪么翻}
遭亂皆不知所終惟邕王存美以病風偏枯得免居於晉陽|{
	沙陁自唐末強盛蓋至於此恐赤心之支胤或有存者晉王父子相傳其血嗣殱矣且明宗晉王義兒也得國之後坐視義父之遺育為魚為肉何忍也它日詎可望麥飯灑陵乎}
徐温高季興聞莊宗遇弑益重嚴可求梁震|{
	嚴可求料唐有内變見二百七十二卷莊宗同光元年梁震料莊宗必亡見二百七十四卷三年}
梁震薦前陵州判官貴平孫光憲於季興使掌書記|{
	貴平縣漢廣都縣之東南界後魏置和仁郡仍置平井貴平可曇三縣唐廢平井可曇以貴平縣治和仁城開元十四年移治禄川屬陵州宋省貴平入廣都縣}
季興大治戰艦欲攻楚|{
	治直之翻艦戶黯翻}
光憲諫曰荆南亂離之後賴公休息士民始有生意若又與楚國交惡它國乘吾之弊良可憂也季興乃止 戊戌李紹榮至洛陽|{
	陜州械送至洛陽}
監國責之曰吾何負於爾而殺吾兒|{
	謂紹榮殺從審也見上卷本年三月}
紹榮瞋目直視曰|{
	瞋昌真翻}
先帝何負於爾遂斬之|{
	元行欽雖死監國豈不有愧於其言}
復其姓名曰元行欽|{
	李紹榮賜姓名見二百六十九卷梁均王貞明元年}
監國恐征蜀軍還為變|{
	還從宣翻又如字}
以石敬瑭為陜州留後己亥以李從珂為河中留後|{
	陜州以備其徑至洛陽河中以備其北歸晉陽陜失冉翻}
樞密使張居翰乞歸田里許之李紹真屢薦孔循之才庚子以循為樞密副使李紹宏請復姓馬|{
	李紹宏賜姓名見二百七十卷梁均王貞明五年}
監國下教數租庸使孔謙姧佞侵刻窮困軍民之罪而斬之|{
	數所角翻}
凡謙所立苛斂之法|{
	斂力贍翻}
皆罷之因廢租庸使及内勾司|{
	租庸使唐末及梁置内勾司莊宗同光二年置}
依舊為鹽鐵戶部度支三司委宰相一人專判|{
	唐制戶部度支以本司郎中侍郎判其事又置鹽鐵轉運使其後用兵以國計為重遂以宰相領其職乾符已後天下喪亂國用愈空始置租庸使用兵無常隨時調斂兵罷則止梁興置租庸使領天下錢穀廢鹽鐵戶部度支之官莊宗滅梁因而不改明宗入立誅租庸使孔謙而廢其使職以大臣一人判戶部度支鹽鐵號曰判三司至長興元年張延朗因請置三司使事下中書中書用唐故事拜延朗特進工部尚書充諸道鹽鐵轉運等使兼判戶部度支事詔以延朗充三司使班在宣徽使下三司置使則自梁始宋白曰同光二年左諫議大夫竇專奏請廢租庸使名目歸三司畧曰伏見天下諸色錢穀比屬戶部設度支金部倉部各有郎中員外將地賦山海鹽鐵分擘支計徵輸後為租賦繁多添置三司使領同資國力共致豐財安史作亂民戶流亡征租不時經費多闕惟江淮嶺表郡縣完全總三司貨財發一使徵賦在處勘覆名曰租庸收復京城尋廢其職務廣明中黄巢叛逆僖宗播遷依前又以江淮徵賦置租庸使及至還京旋亦停廢偽梁將四鎮節制徵輸置宫使名目後廢宫使改置租庸}
又罷諸道監軍使以莊宗由宦官亡國命諸道盡殺之 魏王繼岌自興平退至武功宦者李從襲曰禍福未可知退不如進請王亟東行以救内難繼岌從之還至渭水權西都留守張籛已斷浮梁|{
	難乃旦翻籛則前翻斷音短}
循水浮渡是日至渭南腹心呂知柔等皆已竄匿從襲謂繼岌曰時事已去王宜自圖繼岌徘徊流涕乃自伏於牀命僕夫李環縊殺之|{
	繼岌以李從襲呂知柔而殺郭崇韜而殺繼岌者豈它人哉李環即撾殺崇韜者也 考異曰莊宗實録征蜀初為都監後勸繼岌殺郭崇韜者李從襲也明宗實録云宦者都監李繼襲勸繼岌東遷及令自殺又云任圜監軍李廷襲欲存康延孝及至華州為李冲所殺者復云李從襲蓋從襲誤為繼襲延襲今從莊宗實録}
任圜代將其衆而東監國命石敬瑭慰撫之軍士皆無異言|{
	史言西軍歸心於新主}
先是監國命所親李冲為華州都監應接西師|{
	先昔薦翻華戶化反西師即謂魏王繼岌之師}
冲擅逼華州節度使史彦鎔入朝同州節度使李存敬過華州冲殺之并屠其家又殺西川行營都監李從襲|{
	李從襲死有餘罪監國未即肆諸市朝而李冲殺之則為失刑耳}
彦鎔泣訢於安重誨重誨遣彦鎔還鎮召冲歸朝自監國入洛内外機事皆决於李紹真紹真擅收威勝節度使李紹欽太子少保李紹冲下獄|{
	下戶嫁翻}
欲殺之安重誨謂紹真曰温段罪惡皆在梁朝今殿下新平内難冀安萬國豈專為公報仇邪|{
	難乃旦翻為於偽翻按歐史霍彦威素與温段有隙}
紹真由是稍沮|{
	沮在呂翻}
辛丑監國教李紹冲紹欽復姓名為温韜段凝|{
	温韜段凝賜姓名並見二百七十二卷莊宗同光元年}
並放歸田里 壬寅以孔循為樞密使 有司議即位禮李紹真孔循以為唐運已盡宜自建國號監國問左右何謂國號對曰先帝賜姓於唐為唐復讎|{
	賜姓於唐謂獻祖以平龐勛之功始賜姓李也為唐復讎謂莊宗滅梁也為於偽翻}
繼昭宗後故稱唐|{
	言以同光元年繼天祐二十年也}
今梁朝之人不欲殿下稱唐耳|{
	霍彦威孔循皆嘗事梁者也當時在監國左右者未必皆儒生觀其所對辭意於正閏之位致其辯甚嚴雖儒生不能易也}
監國曰吾年十三事獻祖獻祖以吾宗屬視吾猶子|{
	莊宗即位尊其祖國昌為獻祖監國亦沙陀種故云宗屬}
又事武皇垂三十年|{
	莊宗追尊父晉王克用為太祖武皇帝}
先帝垂二十年經綸攻戰未嘗不預武皇之基業則吾之基業也先帝之天下則吾之天下也安有同家而異國乎令執政更議|{
	更工行反}
吏部尚書李琪曰若改國號則先帝遂為路人梓宫安所託乎不惟殿下忘三世舊君|{
	以監國歷事獻祖太祖莊宗三世也}
吾曹為人臣者能自安乎前代以旁支入繼多矣宜用嗣子柩前即位之禮|{
	記曰在牀曰尸在棺曰柩鄭氏注曰尸陳也言形體在柩之言究也白虎通云久也柩音新舊之舊}
衆從之丙午監國自興聖宫赴西宫服斬衰於柩前即位|{
	斬衰下不緶子為父服之衰倉回翻自己丑入洛至此二十日先是未敢即位者魏王繼岌猶在故也繼岌既死乃决為之}
百官縞素既而御衮冕受冊|{
	徐無黨曰既用嗣君之禮矣遽釋衰而服冕可以見其情詐}
百官吉服稱賀戊申勅中外之臣毋得獻鷹犬奇玩之類 有司劾

奏太原尹張憲委城之罪庚戌賜憲死|{
	以張憲前朝大臣加之罪而殺之耳}
任圜將征蜀兵二萬六千人至洛陽|{
	征蜀之初出師六萬除留戍于蜀及康延孝叛死亡之外還洛者二萬六千人耳}
明宗慰撫之各令還營|{
	以通鑑書法言之明宗二字當書帝字此因前史成文偶遺而不之改耳}
甲寅大赦改元|{
	始改元天成}
量留後宫百人宦官三十人教坊百人鷹坊二十人御厨五十人|{
	量音良}
自餘任從所適諸司使務有名無實者皆廢之分遣諸軍就食近畿以省饋運除夏秋税省耗|{
	舊例夏秋二税先有省耗每斗一升今後祗納正税數不量省耗}
節度防禦等使正至端午降誕四節聽貢奉|{
	元正冬至端午并降誕節為四案五代會要唐咸通八年九月九日帝始生于代北金鳳城以其日為應聖節}
毋得斂百姓|{
	斂力贍翻}
刺史以下不得貢奉選人先遭塗毁文書者|{
	塗毁選人告身見二百七十三卷莊宗同光二年}
令三銓止除詐偽餘復舊規|{
	唐六典吏部尚書侍郎之職掌天下官吏以三銓分其選一曰尚書銓二曰中銓三曰東銓或云吏部東西銓并流外銓為三銓宋白曰太和四年七月吏部奏當司西銓侍郎廳舊以尚書之次為中銓次為東銓乾元中侍郎崔器奏改中銓為西銓以久次侍郎居左新除侍郎居右因循倒置議者非之請自今久次侍郎居西銓新除侍郎居東銓勅旨依又曰兵部尚書為中銓并東銓西銓為三銓}
五月丙辰朔以太子賓客鄭珏|{
	珏古岳翻}
工部尚書任圜並為中書侍郎同平章事圜仍判三司圜憂公如家簡拔賢俊杜絶僥倖期年之間|{
	僥堅堯翻期讀曰朞}
府庫充實軍民皆足朝綱粗立|{
	史言任圜輔相有績粗坐五翻}
圜每以天下為己任由是安重誨忌之|{
	為安重誨譖殺任圜張本}
武寧節度使李紹真忠武節度使李紹瓊貝州刺史李紹英齊州防禦使李紹䖍河陽節度使李紹奇洺州刺史李紹能各請復舊姓名為霍彦威萇從簡房知温王晏球夏魯奇米君立許之|{
	李紹真紹䖍以梁將歸降賜姓名李紹瓊紹英紹奇紹能以事莊宗有戰功賜姓名通鑑不盡載其賜姓名之由畧之也}
從簡陳州人也晏球本王氏子畜於杜氏|{
	畜吁玉翻}
故請復姓王 丁巳初令百官正衙常朝外五日一赴内殿起居|{
	時正衙常朝御文明殿朔望御之内殿中興殿也朝直遥翻}
宦官數百人竄匿山林或落髪為僧至晉陽者七十餘人詔北都指揮使李從温悉誅之從温帝之姪也帝以前相州刺史安金全有功於晉陽|{
	事見二百六十九卷梁均}


|{
	王貞明二年相息亮翻}
壬戌以金全為振武節度使同平章事丙寅趙在禮請帝幸鄴都戊辰以在禮為義成節度使辭以軍情未聽不赴鎮|{
	趙在禮實為魏兵所劫制不容其赴滑州}
李彦超入朝帝曰河東無虞爾之力也|{
	河東軍府在晉陽李存沼死張憲出走鎮定軍固李彦超之力也}
庚午以為建雄留後|{
	使之鎮晉州而未授節旌且為留後}
甲戌加王延翰同平章事|{
	王延翰承其先業據有閩地}
帝目不知書四方奏事皆令安重誨讀之重誨亦不能盡通乃奏稱臣徒以忠實之心事陛下得典樞機今事粗能曉知至於古事非臣所及願倣前朝侍講侍讀近代直崇政樞密院|{
	侍講侍讀盛唐之制也直崇政院梁制也直樞密院莊宗制也宋白曰同光二年崇政院依舊為樞密院以宰臣兼使置直院一人}
選文學之臣與之共事以備應對乃置端明殿學士|{
	春明退朝錄端明殿西京正衙殿蓋改文明曰端明五代會要唐同光二年正月改解缷殿為端明殿按端明殿是燕閒接御儒臣之地必非正衙殿當以五代會要為據端明殿學士始此宋白曰長興四年劉昫入相中謝是日大祠明宗不御中興殿而坐於端明殿昫至中興殿門中使曰舊禮宰臣謝恩須于正殿通喚今日上以大祠不坐正殿請俟來日趙延壽曰命相之制已下三日中謝無宜後時即奏聞昫雖中謝於端明殿而自端明學士拜相復謝於本殿人士榮之}
乙亥以翰林學士馮道趙鳳為之 丙子聽郭崇韜歸葬復朱友謙官爵|{
	二人以讒死見上卷本年正月}
兩家貨財田宅前籍没者皆歸之 戊寅以安重誨領山南東道節度使重誨以襄陽要地|{
	襄陽控蜀扼荆故曰要地}
不可乏帥|{
	帥所類翻下同}
無宜兼領固辭許之 詔發汴州控鶴指揮使張諫等三千人戍瓦橋六月丁酉出城復還作亂|{
	控鶴梁之侍衛親軍積驕而憚遠戍故作亂蓋當時天下皆驕兵也復扶又翻}
焚掠坊市殺權知州推官高逖逼馬步都指揮使曹州刺史李彦饒為帥彦饒曰汝欲吾為帥當用吾命禁止焚掠衆從之己亥旦彦饒伏甲於室諸將入賀彦饒曰前日唱亂者數人而已遂執張諫等四人斬之其黨張審瓊帥衆大譟於建國門|{
	帥讀曰率}
彦饒勒兵擊之盡誅其衆四百人軍州始定即日以軍州事牒節度推官韋儼權知具以狀聞|{
	符彦饒攝於汴而亂於滑豈當時將士驕悖習以成俗彦饒久而與之俱化邪}
庚子詔以樞密使孔循知汴州收為亂者三千家悉誅之彦饒彦超之弟也 蜀百官至洛陽永平節度使兼侍中馬全曰國亡至此生不如死不食而卒|{
	書馬全之官蜀官也蜀置永平軍於雅州}
以平章事王鍇等|{
	鍇口駭翻}
為諸州府刺史少尹判官司馬亦有復歸蜀者|{
	復扶又翻}
辛丑滑州都指揮使于可洪等縱火作亂攻魏博戍兵三指揮逐出之 乙巳勅朕二名但不連稱皆無所避|{
	二名不偏諱古也}
戊申加西川節度使孟知祥兼侍中 李繼曮至華州聞洛中亂復歸鳳翔帝為之誅柴重厚|{
	為于偽翻柴重厚不納李從曮見上卷本年二月}
高季興表求夔忠萬三州為屬郡詔許之|{
	莊宗之伐蜀也詔高}


|{
	季興自取夔忠萬三州為巡屬季興不能取王衍既敗三州歸唐季興乃求為巡屬雖不許可也為季興不式王命興兵致討張本 考異曰莊宗實録云王建于夔州置鎮江軍節度以夔忠萬施為屬郡雲安監有榷鹽之利建升為安州上舉軍平蜀詔季興自收元管屬郡荆南軍未進夔州連帥以州降繼岌十國紀年荆南史天成元年二月王表請夔忠萬州及雲安監隸木道莊宗許之詔命未下莊宗遇弑六月王表求三州明宗許之劉恕按莊宗實録及薛史帝紀同光三年十一月庚戌荆南高季興奏收復夔忠等州曾顔勃海行年記云得夔忠萬等州明宗實録及薛史韋說傳云討西蜀季興請攻峽内先朝許之如能得三州俾為屬郡三川既定季興無尺寸之功莊宗實録同光四年三月丙寅高季興請峽内夔忠萬等州割歸當道明宗實録天成元年六月甲寅高季興奏去冬先朝詔命攻取峽内屬郡尋有施州官吏知臣上峽率先歸投忠萬夔三州旦夕期於收復被郭崇韜專將文字約臣回歸方欲陳論便值更變此說頗近實故從之蓋三年十月夔忠萬三州降於繼岌十一月庚戌季興奏請三州為屬郡舊史誤云奏收復也行年記差繆最多不可為據或者夔州雖自降于繼岌季興表云收復三州攘為己功亦無足怪今從明宗實録}
安重誨恃恩驕横|{
	横戶孟翻}
殿直馬延誤衝前導|{
	左右班殿直天子侍官也宋熙寧以前以為西班小使臣寄禄官職官分紀曰殿直五代本曰殿前承旨晉天福五年詔除翰林承旨外殿前承旨改曰殿直按天成元年安重誨斬殿直馬延潞王清泰元年殿直承旨都知趙處願等令具襴鞹則殿直名官已在晉天福之前職官分紀誤矣後周廣順間殿直楚延祚殿直王巒亦見于史}
斬之於馬前御史大夫李琪以聞|{
	李琪憚安重誨權勢不敢劾奏但以其事聞耳}
秋七月重誨白帝下詔稱延陵突重臣戒諭中外|{
	只此一事安重誨已足以取死}
于可洪與魏博戍將互相奏云作亂帝遣使按驗得實辛酉斬可洪於都市其首謀滑州左崇牙全營族誅助亂者右崇牙兩長劍建平將校百人亦族誅|{
	校戶教翻}
壬申初令百官每五日起居轉對奏事|{
	時依盛唐之制百官轉對各奏本司公事}
契丹主攻勃海拔其夫餘城|{
	即唐高麗之夫餘城也時高麗王王建有國限混同江而守之混同江之西不能有也故夫餘城屬勃海國混同江即鴨緑水夫音扶}
更命曰東丹國|{
	更工衡翻}
命其長子托雲鎮東丹號人皇王以次子德光守西樓號元帥太子|{
	為托雲來奔張本宋白曰耶律德光本名耀屈之慕中國文字改焉}
帝遣供奉官姚坤告哀於契丹 |{
	考異曰漢高祖實録作苗紳今從莊宗列傳}
契丹主聞莊宗為亂兵所害慟哭曰我楚德爾也吾方欲救之以勃海未下不果往致吾兒及此哭不已楚德爾者猶華言朋友也又謂坤曰今天子聞洛陽有急何不救對曰地遠不能及曰何故自立坤為言帝所以即位之由契丹主曰漢兒喜飾說毋多談|{
	為于偽翻喜許計翻}
托雲侍側曰牽牛以蹊人之田而奪之牛可乎|{
	引左傳申叔之言史言契丹慕中國效中國人道書語}
坤曰中國無主唐天子不得已而立亦猶天皇王初有國豈強取之乎|{
	指言安巴堅不肯受代撃滅七部事也強如字}
契丹主曰理當然|{
	聞姚坤言不得不服}
又曰聞吾兒專好聲色遊畋|{
	好呼報翻}
宜其及此我自聞之舉家不飲酒散遣伶人解縱鷹犬若亦效吾兒所為行自亡矣|{
	契丹主智識如此固宜其能立國傳世也}
又曰吾兒與我雖世舊然屢與我戰爭於今天子則無怨足以修好若與我大河之北吾不復南侵矣坤曰此非使臣之所得專也|{
	復扶又翻下復召乃復同}
契丹主怒囚之旬餘復召之曰河北恐難得得鎮定幽州亦可也給紙筆趣令為狀|{
	趣讀曰促}
坤不可欲殺之韓延徽諫乃復囚之|{
	囚而復囚欲姚坤之為狀縱使姚坤為狀中國肯割地而與之乎此欲用抵冒度湟之故智耳}
丙子葬光聖神閔孝皇帝於雍陵|{
	雍陵在河南新安縣 考異曰實録乙亥梓宫發引是日遷幸雍陵按莊宗實録哀冊文云丙子今從之}
廟號莊宗 丁丑鎮州留後王建立奏涿州刺史劉殷肇不受代謀作亂已討擒之|{
	唐之方鎮涿州幽州節度屬郡也不屬鎮州節度而王建立得討之者明宗初得天下方鎮州郡反側者尚多王建立明宗之所親者越境討擒劉殷肇奏以為不受代朝廷亦聽之耳}
己卯置彰國軍於應州|{
	新舊唐書地理志未有應州歐史職方考始有應州故屬大同節度而不載其建置之始意晉王克用分雲州置應州也九域志化外州應州領金城混源二縣竊意金城即以明宗所生之地金鳳城置縣也今置彰國軍節度亦以帝鄉也匈奴須知應州東至幽州八百五十里又薛史周密傳神武川屬應州蓋朱邪執宜徙河東始保神武川之黄花堆沙陀由是而基霸業故以其地置應州也}
門下侍郎同平章事豆盧革韋說奏事帝前或時禮貌不盡恭|{
	說讀曰悦}
百官俸錢皆折估|{
	折之舌翻估音古價也}
而革父子獨受實錢百官自五月給而革父子自正月給由是衆論沸騰說以孫為子奏官受選人王傪賂|{
	選須絹翻傪七感翻又音倉含翻}
除近官|{
	近官近畿州縣之官}
中旨以庫部郎中蕭希甫為諫議大夫革說覆奏希甫恨之上疏言革說不忠前朝阿諛取容因誣革強奪民田縱田客殺人說奪隣家井取宿藏物|{
	宿藏物前人所窖藏而不及發取者此蓋言藏之于井}
制貶革辰州刺史說溆州刺史|{
	溆音叙}
庚辰賜希甫金帛擢為散騎常侍|{
	散昔亶翻騎奇計翻}
辛巳契丹主安巴堅卒於夫餘城|{
	卒子恤翻}
舒嚕后召諸將及酋長難制者之妻|{
	酋慈秋翻長知兩翻}
謂曰我今寡居汝不可不效我又集其夫泣問曰汝思先帝乎對曰受先帝恩豈得不思曰果思之宜往見之遂殺之|{
	為舒嚕后囚於安巴堅墓張本}
癸未再貶豆盧革費州司戶韋說夷州司戶甲申革流陵州說流合州|{
	自唐末以來流竄者率賜死革說其得至流所乎}
孟知祥隂有據蜀之志閲庫中得鎧甲二十萬置左右牙等兵十六營凡萬六千人營於牙城内外 八月乙酉朔日有食之 丁亥契丹舒嚕后使少子安端少君守東丹|{
	少詩沼翻}
與長子托雲奉契丹主之喪將其衆發夫餘城 初郭崇韜以蜀騎兵分左右驍衛等六營凡三千人步兵分左右寧遠等二十營凡二萬四千人庚寅孟知祥增置左右衝山等六營凡六千人營於羅城内外又置義寧等二十營凡萬六千人分戍管内州縣就食|{
	因分戍而使就食於所戍州縣}
又置左右牢城四營凡四千人分戍成都境内王公儼既殺楊希望|{
	事見上卷本年三月}
欲邀節鉞揚言符習

為治嚴急軍府衆情不願其還|{
	治直吏翻}
習還至齊州公儼拒之習不敢前|{
	齊州東至青州三百四十餘里中間猶隔淄州符習聞王公儼阻兵遽不敢前欲使之戡難難矣}
公儼又令將士上表請已為帥|{
	帥所類翻}
詔除登州刺史公儼不時之官託云軍情所留帝乃徙天平節度使霍彦威為平盧節度使聚兵淄州以圖攻取|{
	九域志淄州東北至青州一百二十里}
公儼懼乙未始之官丁酉彦威至青州追擒之并其族黨悉斬之支使北海韓叔嗣預焉其子熙載將奔吳密告其友汝隂進士李穀穀送至正陽|{
	九域志潁州潁上縣有正陽鎮在淮津之西淮之東津曰東正陽則吳境也}
痛飲而别熙載謂穀曰吳若用我為相當長驅以定中原穀笑曰中原若用吾為相取吳如囊中物耳|{
	其後周世宗以李穀為相用其謀以取淮南而韓熙載亦相南唐終不能有所為也相息亮翻}
庚子幽州言契丹寇邊命齊州防禦使安審通將兵禦之 九月壬戌孟知祥置左右飛棹兵六營凡六千人分戍濱江諸州習水戰以備夔峽 癸酉盧龍節度使李紹斌請復姓趙|{
	歐史曰趙德鈞幽州人也事劉守文守光為軍使莊宗伐燕得之賜姓名李紹斌}
從之仍賜名德鈞德鈞養子延壽尚帝女興平公主故德鈞尤蒙寵任延壽本蓚令劉邟之子也|{
	蓚音條邟若浪翻}
加楚王殷守尚書令 契丹舒嚕后愛中子德光欲立之|{
	中讀曰仲}
至西樓|{
	西樓契丹上都也先是契丹主使德光留守}
命與托雲俱乘馬立帳前謂諸酋長曰二子吾皆愛之莫知所立汝曹擇可立者執其轡酋長知其意爭執德光轡讙躍曰願事元帥太子后曰衆之所欲吾安敢違遂立之為天皇王突欲愠帥數百騎欲奔唐為邏者所遏|{
	讙許元翻愠于問翻朱子曰愠不是大段怒但心裏畧有不平意便是愠邏音郎佐翻}
舒嚕后不罪遣歸東丹天白王王尊舒嚕后為太后國事皆决焉太后復納其姪為天皇王后|{
	復扶又翻}
天皇王性孝謹母病不食亦不食侍於母前應對或不稱旨|{
	稱尺證翻}
母揚眉視之輒懼而趨避非復召不敢見也|{
	復扶又翻}
以韓延徽為政事令|{
	歐史契丹以韓延徽為相號政事令}
聽姚坤歸復命|{
	阿保機囚姚坤事見上}
遣其臣阿斯默古内來告哀 |{
	考異曰漢高祖實録作蒙克鼐爾今從明宗實録及會要}
壬午賜李繼曮名從曮|{
	以子行待之也}
冬十月甲申朔初賜文武官春冬衣|{
	五代會要同光三年租庸院奏新定四京及諸道副使判官以下俸料有春衣絹冬衣絹此蓋賜在京文武官以已成之衣}
昭武節度使同平章事王延翰|{
	昭武當作威武}
驕淫殘暴己丑自稱大閩國王立宫殿置百官威儀文物皆倣天子之制羣下稱之曰殿下赦境内追尊其父審知曰昭武王|{
	為王延翰不終張本}
靜難節度使毛璋驕僭不法訓卒繕兵有跋扈之志|{
	若毛璋者其跋扈亦何能為不過欲據邠州耳}
詔以潁州團練使李承約為節度副使以察之壬辰徙璋為昭義節度使|{
	莊宗改潞州昭義軍為安義軍尋復舊}
璋欲不奉詔承約與觀察判官長安邊蔚從容說諭|{
	蔚音鬱從千容翻說式芮翻下說之同}
久之乃肯受代 庚子幽州奏契丹盧龍節度使盧文進來奔|{
	盧文進入契丹見二百七十卷梁均王乾化三年}
初文進為契丹守平州帝即位遣間使說之|{
	為于偽翻間古莧翻}
以易代之後無復嫌怨|{
	莊宗怨盧文進殺其弟而奔契丹又引契丹而擾邊今莊宗殂而明宗立則無復嫌怨矣}
文進所部皆華人思歸乃殺契丹戍平州者帥其衆十餘萬車帳八千乘來奔|{
	為後盧文進又奔淮南張本帥讀曰率}
初魏王繼岌郭崇韜率蜀中富民輸犒賞錢五百萬緡聽以金銀繒帛充|{
	犒苦到翻繒慈陵翻}
晝夜督責有自殺者給軍之餘猶二百萬緡至是任圜判三司知成都富饒|{
	同光之末任圜從軍伐蜀故知其富饒}
遣鹽鐵判官太僕卿趙季良為孟知祥官告國信兼三川都制置轉運使|{
	帝即位加孟知祥侍中故使趙季良奉官告國信入蜀因制置轉運}
甲辰季良至成都蜀人欲皆不與知祥曰府庫它人所聚輸之可也州縣租税以贍鎮兵十萬决不可得|{
	觀孟知祥此語專制蜀土之心已呈露矣}
季良但發庫物不敢復言制置轉運職事矣|{
	復扶又翻}
安重誨以知祥及東川節度使董璋皆據險要擁強兵恐久而難制又知祥乃莊宗近姻|{
	孟知祥之妻莊宗之從姊妹也}
隂欲圖之客省使泗州防禦使李嚴|{
	職官分紀曰梁有客省使宋因之掌四方進奉及四夷朝貢牧伯朝覲賜酒饌饔餼給宰相近臣禁軍將校節儀諸州進奉使賜物回詔之事李嚴領泗州防禦耳泗州時屬吳}
自請為西川監軍必能制知祥己酉以嚴為西川都監文思使太原朱弘昭為東川副使|{
	文思使掌文思院宋以為西班使臣以處武臣}
李嚴母賢明謂嚴曰汝前啟滅蜀之謀|{
	事見二百七十三卷莊宗同光二年}
今日再往必以死報蜀人矣|{
	為李嚴為孟知祥所殺張本}
舊制吏部給告身先責其人輸朱膠綾軸錢|{
	宋白曰故事如封建諸王内命婦及宰相翰林學士中書舍人諸道節度觀察團練防禦日後即中書帖官告院素綾紙褾軸下所司書寫印署畢進入宣賜其文武兩班并諸道官員及奏薦將校勅下後並合是本道進奏院或本官自於所司送納朱膠綾紙價錢各請出給陸游曰江隣幾嘉祐雜志言唐告身初用紙肅宗朝有用絹貞元後始用綾余在成都見周世宗除劉仁贍侍中告乃用紙在金彦亨尚書之子處}
喪亂以來|{
	喪息浪翻}
貧者但受勅牒多不取告身|{
	受勅牒以照驗供職苟得一時之禄利告身無其錢則不及取矣}
十一月甲戌吏部侍郎劉岳上言告身有褒貶訓戒之辭|{
	此中書所行辭也}
豈可使其人初不之覩勅文班丞郎給諫|{
	丞郎謂尚書左右丞及二十四曹郎給謂給事中諫謂諫議大夫}
武班大將軍以上宜賜告身其後執政議以為朱膠綾軸厥費無多朝廷受以官禄何惜小費|{
	受當作授歐史曰故事吏部官告身皆輸朱膠綾軸錢然後給其品高則賜之貧者不能輸錢往往但得勅牒而無告身五代之亂因以為常卑者無復給告身中書但録其制辭而編為勅甲劉岳建言以謂制辭或任其才能或褒其功行或申之以訓誡而受官者既不給告身皆不知受命之所以然非王言所告詔之意請一切賜之由是百官皆賜告身自岳始也}
乃奏凡除官者更不輸錢皆賜告身當是時所除正員官之外其餘試銜帖號止以寵激軍中將校而已|{
	試銜謂試某官某階皆以入銜也帖號謂試以諸銜將軍郎將之號}
及長興以後所除浸多乃至軍中卒伍使州鎮戍胥吏皆得銀青階及憲官|{
	使疏吏翻使謂諸道節度使觀察使司御史臺官謂之憲官此亦言試銜官也}
歲賜告身以萬數矣|{
	史因賜告身又言當時除授之濫}
閩王延翰蔑棄兄弟襲位纔踰月出其弟延鈞為泉州刺史延翰多取民女以充後庭采擇不已延鈞上書極諫延翰怒由是有隙父審知養子延稟為建州刺史|{
	延稟本周氏子王審知養以為子}
延翰與書使之采擇延稟復書不遜亦有隙十二月延稟延鈞合兵襲福州延稟順流先至|{
	自建溪順流東下福州水路縈紆幾數百里而水勢湍疾輕舟朝發夕至九域志建州東南至福州五百二十里蓋言陸路也}
福州指揮使陳陶帥衆拒之兵敗陶自殺是夜延稟帥壯士百餘人趣西門|{
	帥讀曰率趣七喻翻}
梯城而入執守門者發庫取兵仗及寢門延翰驚匿别室辛卯旦延稟執之暴其罪惡且稱延翰與妻崔氏共弑先王|{
	誣以弑君父之罪}
告諭吏民斬于紫宸門外|{
	唐都長安内中有紫宸殿紫宸門閩人僭倣其名耳}
是日延鈞至城南延稟開門納之推延鈞為威武留後|{
	王延鈞審知次子也}
癸巳以盧文進為義成節度使同平章事 庚子以

皇子從榮為天雄節度使同平章事 趙季良等運蜀金帛十億至洛陽|{
	詩萬億及秭釋云萬億曰兆孔頴達曰萬億曰兆者依如算法億之數有大小二法其小數以十為等十萬曰億十億曰兆也其大數以萬為等數萬至萬為億是萬萬為億又從億數至萬億為兆故詩頌毛氏傳云數壘萬曰億數億至億曰秭兆在億秭之間是大數之法魏風刺在位貪殘胡取禾三百億兮魏國褊小不應過多故以小數言之故云十萬曰億今趙季良運金帛十億若以小億計之則百萬耳安能濟朝廷之匱乏哉若以大億計之則十萬萬也未知孰是}
時朝廷方匱乏賴此以濟 是歲吳越王鏐以中國喪亂朝命不通改元寶正其後復通中國乃諱而不稱|{
	喪息浪翻朝直遥翻復扶又翻 考異曰閻自若唐末汎聞録云同光四年京師亂朝命斷絶鏐遂僭大號改元保正明年明宗錫命至乃去號復用唐正朔紀年通譜云鏐雖外勤貢奉而隂為僭竊私改年號于其國其後子孫奉中朝正朔漸諱改元事及錢俶納土凡其境土有石刻偽號者悉使人交午鑿滅之惟杭州西湖落星山塔院中有鏐封此山為壽星寶石山偽詔刻之於石雖經鑱毁其文尚可讀後題云寶正六年歲在辛卯明宗長興二年也其元年即天成元年也好事者或傳曰保正非也余公綽閩王事迹云同光元年梁策錢鏐為尚父來年改寶正元年永隆三年吳越世宗文穆王薨林仁志王氏啟運圖云同光元年梁封浙東尚父為吳越國王尋自改元寶正長興三年吳越武肅王崩子世皇嗣永隆二年吳越世皇崩子成宗嗣公綽仁志所記年歲差繆然可見錢氏改元及廟號故兼載焉至今兩浙民間猶謂錢鏐為錢太祖今參取諸書為據}


二年春正月癸丑朔帝更名亶|{
	更工衡翻}
孟知祥聞李嚴來監其軍惡之|{
	惡烏路翻}
或請奏止之知祥曰何必然|{
	猶言何必如此也}
吾有以待之遣吏至綿劍迎候|{
	綿劍二州名}
會武信節度使李紹文卒知祥自言嘗受密詔許便宜從事|{
	孟知祥自言嘗受莊宗密詔也}
壬戌以西川節度副使内外馬步軍都指揮使李敬周為遂州留後|{
	代李紹文}
趣之上道|{
	趣讀曰促上時兩翻}
然後表聞嚴先遣使至成都知祥自以於嚴有舊恩|{
	孟知祥救李嚴之死見二百六十八卷梁均王乾化二年}
冀其懼而自回乃盛陳甲兵以示之嚴不以為意 安重誨以孔循少侍宫禁謂其諳練故事知朝士行能多聽其言|{
	孔循少給事梁太祖帳中唐末歷宣徽樞密院故安重誨意其諳練及知人少詩沼翻諳烏含翻行下孟翻}
豆盧革韋說既得罪|{
	見上年}
朝廷議置相循意不欲用河北人|{
	孔循少長河南故不欲用河北人}
先已薦鄭珏又薦太常卿崔協任圜欲用御史大夫李琪鄭珏素惡琪|{
	惡烏路翻}
故循力沮之謂重誨曰李琪非無文學但不亷耳宰相但得端重有器度者足以儀刑多士矣它日議於上前上問誰可相者重誨以協對圜曰重誨未悉朝中人物|{
	悉詳也}
為人所賣協雖名家識字甚少|{
	少詩沼翻}
臣既以不學忝相位奈何更益以協為天下笑乎上曰宰相重任卿輩更審議之吾在河東時見馮書記多才博學與物無競此可相矣|{
	馮書記謂馮道也道事晉王克用為河東掌書記}
既退孔循不揖拂衣徑去曰天下事一則任圜二則任圜圜何者|{
	孔循之衆辱任圜亦甚矣而圜不以為怒者憚安重誨也史言五季待宰相之輕}
使崔協暴死則已不死會須相之因稱疾不朝者數日上使重誨諭之方入重誨私謂圜曰今方乏人協且備員可乎圜曰明公捨李琪而相崔協是猶棄蘇合之丸|{
	後漢書西域傳曰大秦國合會諸香煎其汁以為蘇合}
取蛣蜣之轉也|{
	蛣蜣蜣蜋也陶隱居曰莊子云蛣蜣之智在於轉丸其喜入人糞中取屎丸而却推之俗名為推丸陸佃埤雅曰蜣蜋黑甲翅在甲下五六月之間經營穢場之下車走糞丸一前挽之一後推之若僕人轉車蛣去吉翻蜣丘良翻}
循與重誨共事|{
	安重誨為樞密使孔循為副使}
日短琪而譽協|{
	譽音余}
癸亥竟以端明殿學士馮道及崔協並為中書侍郎同平章事協邠之曾孫也|{
	崔邠郾之兄也}
戊辰王延稟還建州王延鈞送之將别謂延鈞曰善守先人基業勿煩老兄再下延鈞遜謝甚恭而色變|{
	為王延禀再下攻延鈞而敗死張本}
庚午初令天下長吏每旬親引慮繋囚|{
	引慮繫囚即漢書所謂録囚徒也自唐以來率曰慮囚考之先儒音義慮亦讀為録}
孟知祥禮遇李嚴甚厚一日謁知祥知祥謂曰公前奉使王衍歸而請兵伐蜀莊宗用公言遂致兩國俱亡|{
	謂莊宗空國以伐蜀蜀亡而謀臣死根本虛而莊宗亦亡}
今公復來|{
	復扶又翻}
蜀人懼矣且天下皆廢監軍|{
	罷諸道監軍見本卷上年}
公獨來監吾軍何也嚴惶怖求哀|{
	怖普故翻}
知祥曰衆怒不可遏也遂揖下斬之|{
	李嚴卒如其母之言}
又召左廂馬步都虞候丁知俊知俊大懼知祥指嚴尸謂曰昔嚴奉使汝為之副然則故人也為我瘞之|{
	為于偽翻瘞於計翻}
因誣奏嚴詐宣口勅云代臣赴闕|{
	言李嚴矯勅云代知祥使知祥赴闕}
又擅許將士優賞臣輒已誅之内八作使楊令芝以事入蜀|{
	八作使掌八作司之八作工匠}
至鹿頭關聞嚴死奔還朱弘昭在東川|{
	朱弘昭為東川副使與李嚴同時受命}
聞之亦懼謀歸洛會有軍事董璋使之入奏弘昭偽辭然後行由是得免|{
	兩川跋扈之迹著矣安重誨制之之術窮矣及乎分鎮增兵則兩川反矣}
癸酉以皇子從厚同平章事充河南尹判六軍諸衛事從榮聞之不悦|{
	既尹京邑又握兵柄地親權重從榮惡其偪故不悦為從榮忌從厚張本}
己卯加樞密使安重誨兼侍中孔循同平章事 吳馬軍都指揮使柴再用戎服入朝御史彈之再用恃功不服侍中徐知誥陽於便殿誤通起居退而自劾|{
	劾戶槩翻又戶得翻}
吳王優詔不問知誥固請奪一月俸由是中外肅然|{
	法之不行自上犯之法行于上故中外肅然}
契丹改元天顯葬其主按巴堅於木葉山|{
	契丹主以其所居為上京起樓其間號西樓又于其東千里起東樓北三百里起北樓南木葉山起南樓按木葉山契丹置錦州匈奴須知錦州東北至東京四百里木葉山西南至上京三百里則錦州與木葉山又是兩處通鑑後書晉之齊王北遷至錦州契丹令拜按巴堅墓則又似木葉山在錦州歐史諸書言契丹于南木葉山起南樓是在上京之南也須知謂木葉山西南至上京三百里是在上京東北也無亦契丹中有南木葉山又有北木葉山邪}
舒嚕太后左右有桀黠者|{
	黠下八翻}
后輒謂曰為我達語於先帝|{
	為于偽翻}
至墓所則殺之前後所殺以百數最後平州人趙思温當往思温不行后曰汝事先帝嘗親近何為不行對曰親近莫如后后行臣則繼之后曰吾非不欲從先帝於地下也顧嗣子幼弱國家無主不得往耳乃斷一腕|{
	斷音短腕烏貫翻}
令置墓中思温亦得免 帝以冀州刺史烏震三將兵運糧入幽州|{
	時契丹常以勁騎徜徉幽州四郊之外抄掠糧運故以三將兵運糧善達者為勞績}
二月戊子以震為河北道副招討領寧國節度使|{
	寧國軍宣州屬吳}
屯盧臺軍|{
	句斷盧臺軍臨御河之岸周建乾寧軍東至倉州一百里西至瀛州百七十里}
代泰寧節度使同平章事房知温歸兖州|{
	房知温本鎮兖州}
庚寅以保義節度使石敬瑭兼六軍諸衛副使|{
	石敬瑭時鎮陜州}
丙申以從馬直指揮使郭從謙為景州刺史既至遣使族誅之|{
	討其弑君之罪也}
高季興既得三州請朝廷不除刺史|{
	去年以三州與高季興}
自以子弟為之不許及夔州刺史潘炕罷官|{
	潘炕蜀王氏之舊臣炕若浪翻}
季興輒遣兵突入州城殺戍兵而據之朝廷除奉聖指揮使西方鄴為刺史|{
	五代會要應順元年改龍武神武四十指揮為捧聖左右軍捧聖即奉聖也應順乃閔帝元年而此時已有奉聖軍}
不受又遣兵襲涪州不克|{
	九域志涪州東至忠州三百五十里高季興既得夔忠萬三州又襲涪州而不克涪音浮}
魏王繼岌遣押牙韓珙等部送蜀珍貨金帛四十萬浮江而下季興殺珙等於峽口|{
	此峽口謂西陵陜口珙居勇翻}
盡掠取之|{
	此去年事蓋同光天成間也掠奪也}
朝廷詰之對曰珙等舟行下峽涉數千里欲知覆溺之故自宜按問水神|{
	此慢辭也若春秋楚人答齊桓公問昭王南征不復之辭}
帝怒壬寅制削奪季興官爵以山南東道節度使劉訓為南面招討使知荆南行府事忠武節度使夏魯奇為副招討使將步騎四萬討之東川節度使董璋充東南面招討使新夔州刺史西方鄴副之 |{
	考異曰按梓夔皆在荆南之西南而云東南面者蓋據夔梓所向言之耳}
將蜀兵下峽|{
	此峽謂自瞿唐峽直至西陵峽口所謂三峽也}
仍會湖南軍三面進攻|{
	湖南軍楚王馬殷之軍}
三月甲寅以李敬周為武信留後|{
	從孟知祥之請也}
丙辰初置監牧蕃息國馬|{
	蕃扶元翻唐置監牧以畜馬喪亂以來馬政廢矣今復置監牧以蕃息之然此時監牧必置于并代之間若河隴諸州不能復盛唐之舊是後帝問樞密使范延光馬數幾何對曰騎軍三萬五千帝曰吾居兵間四十年太祖在太原時馬數不過七千莊宗與梁戰河上馬纔萬匹今馬多矣不能一天下奈何延光曰一馬之費足以養步卒五人帝曰肥戰馬以瘠吾人其愧多矣今因置監牧事併録之}
初莊宗之克梁也以魏州牙兵之力及其亡也皇甫暉張破敗之亂亦由之|{
	以魏州牙兵克梁事始二百六十九卷梁均王貞明元年終二百七十卷莊宗同光元年皇甫暉張破敗之亂事見二百七十四卷天成元年}
趙在禮之徙滑州不之官亦實為其下所制|{
	事見上年}
在禮欲自謀脱禍隂遣腹心詣闕求移鎮帝乃為之除皇甫暉陳州刺史趙進貝州刺史|{
	為于偽翻皇甫暉趙進制趙在禮不得左右者也}
徙在禮為横海節度使以皇子從榮鎮鄴都命宣徽北院使范延光將兵送之且制置鄴都軍事乃出奉節等九指揮三千五百人使軍校龍晊部之|{
	晊之日翻}
戍盧臺軍以備契丹不給鎧仗但繫幟於長竿以别隊伍由是皆俛首而去|{
	繫音計幟昌志翻别彼列翻俛音免}
中塗聞孟知祥殺李嚴軍中籍籍已有譌言既至會朝廷不次擢烏震為副招討使譌言益甚房知温怨震驟來代己|{
	房知温自莊宗時戍邊以舉兵從帝建節烏震自刺史領節又代知温為副招討故怨其驟}
震至未交印壬申震召知温及諸道先鋒馬軍都指揮使齊州防禦使安審通博於東寨|{
	時盧臺戍軍夾河東西為兩寨}
知温誘龍晊所部兵殺震於席上其衆譟於營外|{
	譟者烏震親兵也歐史以為譟者亂兵誘音酉}
安審通脱身走奪舟濟河將騎兵按甲不動知温恐事不濟亦上馬出門甲士攬其轡曰公當為士卒主去欲何之知温紿之曰騎兵皆在河西不收取之獨有步兵何能集事遂躍馬登舟濟河與審通合謀擊亂兵亂兵遂南行騎兵徐踵其後部伍甚整亂者相顧失色列炬宵行疲於荒澤詰朝騎兵四合擊之|{
	詰去吉翻}
亂兵殆盡餘衆復趣故寨審通已焚之亂兵進退失據遂潰其匿於叢薄溝塍|{
	塍石陵翻}
得免者什無一二范延光還至淇門聞盧臺亂發滑州兵復如鄴都以備奔逸帝遣客省使李仁矩如西川傳詔安諭孟知祥及吏

民|{
	以孟知祥殺李嚴懼其不自安也知祥自此浸驕}
甲戌至成都 劉訓兵至荆南楚王殷遣都指揮使許德勲等將水軍屯岳州|{
	以應劉訓也}
高季興堅壁不戰求救於吳吳人遣水軍援之夏四月庚寅勅盧臺亂兵在營家屬並全門處斬|{
	處昌呂翻自帝即位已來汴州張諫之亂滑州于可洪之亂以至盧臺之亂凡亂兵皆夷其家然而流言不息盼盼然疾視其上者相環也此無它以亂止亂故爾}
勅至鄴都闔九指揮之門驅三千五百家凡萬餘人於石灰窰悉斬之永濟渠為之變赤|{
	唐開元二十八年魏州刺史盧暉徙永濟渠自石灰窑引流至城西至魏橋以通江淮之漕為于偽翻}
朝廷雖知房知温首亂欲安反仄癸巳加知温兼侍中先是孟知祥遣牙内指揮使文水武漳迎其妻瓊華

長公主及子仁贊於晉陽|{
	孟仁贊後改名昶}
及鳳翔|{
	行及鳳翔也}
李從曮聞知祥殺李嚴止之以聞帝聽其歸蜀丙申至成都 鹽鐵判官趙季良與孟知祥有舊知祥奏留季良為副使朝廷不得已丁酉以季良為西川節度副使|{
	趙季良由此遂為孟知祥佐命之臣}
李昊歸蜀|{
	李昊隨王衍東遷至是歸蜀}
知祥以為觀察推官 江陵卑濕復值久雨|{
	復扶又翻}
糧道不繼將士疾疫劉訓亦寢疾癸卯帝遣樞密使孔循往視之且審攻戰之宜 五月癸丑以威武留後王延鈞為本道節度使琅邪王 孔循至江陵攻之不克遣人入城說高季興|{
	說式芮翻}
季興不遜丙寅遣使賜湖南行營夏衣萬襲丁卯又遣使賜楚王殷鞍馬玉帶督饋糧於行營竟不能得|{
	湖南荆南輔車相依雖厚賜楚人以督其饋軍終不奉詔}
庚午詔劉訓等引兵還楚王殷遣中軍使史光憲入貢帝賜之駿馬十美女

二過江陵高季興執光憲而奪之且請舉鎮自附於吳徐温曰為國者當務實效而去虛名|{
	去羌呂翻}
高氏事唐久矣|{
	自唐滅梁高氏即事之}
洛陽去江陵不遠|{
	舊唐書地理志洛陽至江陵一千三百一十五里}
唐人步騎襲之甚易|{
	易以䜴翻}
我以舟師沂流救之甚難夫臣人而弗能救使之危亡能無愧乎乃受其貢物辭其稱臣聽其自附於唐|{
	史言徐温能自保其國不務遠畧}
任圜性剛直且恃與帝有舊|{
	任圜與帝同事莊宗且全征蜀之兵以歸帝}
勇於敢為權倖多疾之舊制館劵出於戶部|{
	唐舊制使臣出門方皆自戶部給劵}
安重誨請從内出|{
	請從内出則樞密院得專其事}
與圜爭於上前往復數四聲色俱厲上退朝宫人問上適與重誨論事為誰|{
	常語近方為適}
上曰宰相宫人曰妾在長安宫中|{
	此蓋唐時宫人老於事者}
未嘗見宰相樞密奏事敢如是者蓋輕大家耳上愈不悦|{
	唐明宗起於行伍而為天子常疑宰相輕已豆盧革韋說之死猶曰自取然以此而斥任圜卒亦置之死地大誤矣}
卒從重誨議|{
	卒子恤翻}
圜因求罷三司|{
	為安重誨讒殺任圜張本}
詔以樞密承旨孟鵠充三司副使權判|{
	五代置樞密院都承旨副承旨以諸衛將軍充權判者權判三司事也}
鵠魏州人也 六月庚辰太子詹事温輦請立太子 丙戌門下侍郎同平章事任圜罷守太子少保 己丑以宣徽北院使張延朗判三司 壬辰貶劉訓為檀州刺史|{
	以征荆南無功也檀州密雲郡因白檀古縣名以名州}
丙申封楚王殷為楚國王 西方鄴敗荆南水軍於峽中復取夔忠萬三州|{
	敗補邁翻}


資治通鑑卷二百七十五
