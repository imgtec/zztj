<!DOCTYPE html PUBLIC "-//W3C//DTD XHTML 1.0 Transitional//EN" "http://www.w3.org/TR/xhtml1/DTD/xhtml1-transitional.dtd">
<html xmlns="http://www.w3.org/1999/xhtml">
<head>
<meta http-equiv="Content-Type" content="text/html; charset=utf-8" />
<meta http-equiv="X-UA-Compatible" content="IE=Edge,chrome=1">
<title>資治通鑒_150-資治通鑑卷一百四十九_150-資治通鑑卷一百四十九</title>
<meta name="Keywords" content="資治通鑒_150-資治通鑑卷一百四十九_150-資治通鑑卷一百四十九">
<meta name="Description" content="資治通鑒_150-資治通鑑卷一百四十九_150-資治通鑑卷一百四十九">
<meta http-equiv="Cache-Control" content="no-transform" />
<meta http-equiv="Cache-Control" content="no-siteapp" />
<link href="/img/style.css" rel="stylesheet" type="text/css" />
<script src="/img/m.js?2020"></script> 
</head>
<body>
 <div class="ClassNavi">
<a  href="/24shi/">二十四史</a> | <a href="/SiKuQuanShu/">四库全书</a> | <a href="http://www.guoxuedashi.com/gjtsjc/"><font  color="#FF0000">古今图书集成</font></a> | <a href="/renwu/">历史人物</a> | <a href="/ShuoWenJieZi/"><font  color="#FF0000">说文解字</a></font> | <a href="/chengyu/">成语词典</a> | <a  target="_blank"  href="http://www.guoxuedashi.com/jgwhj/"><font  color="#FF0000">甲骨文合集</font></a> | <a href="/yzjwjc/"><font  color="#FF0000">殷周金文集成</font></a> | <a href="/xiangxingzi/"><font color="#0000FF">象形字典</font></a> | <a href="/13jing/"><font  color="#FF0000">十三经索引</font></a> | <a href="/zixing/"><font  color="#FF0000">字体转换器</font></a> | <a href="/zidian/xz/"><font color="#0000FF">篆书识别</font></a> | <a href="/jinfanyi/">近义反义词</a> | <a href="/duilian/">对联大全</a> | <a href="/jiapu/"><font  color="#0000FF">家谱族谱查询</font></a> | <a href="http://www.guoxuemi.com/hafo/" target="_blank" ><font color="#FF0000">哈佛古籍</font></a> 
</div>

 <!-- 头部导航开始 -->
<div class="w1180 head clearfix">
  <div class="head_logo l"><a title="国学大师官网" href="http://www.guoxuedashi.com" target="_blank"></a></div>
  <div class="head_sr l">
  <div id="head1">
  
  <a href="http://www.guoxuedashi.com/zidian/bujian/" target="_blank" ><img src="http://www.guoxuedashi.com/img/top1.gif" width="88" height="60" border="0" title="部件查字,支持20万汉字"></a>


<a href="http://www.guoxuedashi.com/help/yingpan.php" target="_blank"><img src="http://www.guoxuedashi.com/img/top230.gif" width="600" height="62" border="0" ></a>


  </div>
  <div id="head3"><a href="javascript:" onClick="javascript:window.external.AddFavorite(window.location.href,document.title);">添加收藏</a>
  <br><a href="/help/setie.php">搜索引擎</a>
  <br><a href="/help/zanzhu.php">赞助本站</a></div>
  <div id="head2">
 <a href="http://www.guoxuemi.com/" target="_blank"><img src="http://www.guoxuedashi.com/img/guoxuemi.gif" width="95" height="62" border="0" style="margin-left:2px;" title="国学迷"></a>
  

  </div>
</div>
  <div class="clear"></div>
  <div class="head_nav">
  <p><a href="/">首页</a> | <a href="/ShuKu/">国学书库</a> | <a href="/guji/">影印古籍</a> | <a href="/shici/">诗词宝典</a> | <a   href="/SiKuQuanShu/gxjx.php">精选</a> <b>|</b> <a href="/zidian/">汉语字典</a> | <a href="/hydcd/">汉语词典</a> | <a href="http://www.guoxuedashi.com/zidian/bujian/"><font  color="#CC0066">部件查字</font></a> | <a href="http://www.sfds.cn/"><font  color="#CC0066">书法大师</font></a> | <a href="/jgwhj/">甲骨文</a> <b>|</b> <a href="/b/4/"><font  color="#CC0066">解密</font></a> | <a href="/renwu/">历史人物</a> | <a href="/diangu/">历史典故</a> | <a href="/xingshi/">姓氏</a> | <a href="/minzu/">民族</a> <b>|</b> <a href="/mz/"><font  color="#CC0066">世界名著</font></a> | <a href="/download/">软件下载</a>
</p>
<p><a href="/b/"><font  color="#CC0066">历史</font></a> | <a href="http://skqs.guoxuedashi.com/" target="_blank">四库全书</a> |  <a href="http://www.guoxuedashi.com/search/" target="_blank"><font  color="#CC0066">全文检索</font></a> | <a href="http://www.guoxuedashi.com/shumu/">古籍书目</a> | <a   href="/24shi/">正史</a> <b>|</b> <a href="/chengyu/">成语词典</a> | <a href="/kangxi/" title="康熙字典">康熙字典</a> | <a href="/ShuoWenJieZi/">说文解字</a> | <a href="/zixing/yanbian/">字形演变</a> | <a href="/yzjwjc/">金 文</a> <b>|</b>  <a href="/shijian/nian-hao/">年号</a> | <a href="/diming/">历史地名</a> | <a href="/shijian/">历史事件</a> | <a href="/guanzhi/">官职</a> | <a href="/lishi/">知识</a> <b>|</b> <a href="/zhongyi/">中医中药</a> | <a href="http://www.guoxuedashi.com/forum/">留言反馈</a>
</p>
  </div>
</div>
<!-- 头部导航END --> 
<!-- 内容区开始 --> 
<div class="w1180 clearfix">
  <div class="info l">
   
<div class="clearfix" style="background:#f5faff;">
<script src='http://www.guoxuedashi.com/img/headersou.js'></script>

</div>
  <div class="info_tree"><a href="http://www.guoxuedashi.com">首页</a> > <a href="/SiKuQuanShu/fanti/">四库全书</a>
 > <h1>资治通鉴</h1> <!--         下载:【右键另存为】即可 --></div>
  <div class="info_content zj clearfix">
  
<div class="info_txt clearfix" id="show">
<center style="font-size:24px;">150-資治通鑑卷一百四十九</center>
    資治通鑑卷一百四十九 宋 司馬光 撰<br />
<br />
  胡三省 音註<br />
<br />
  梁紀五【起屠維大淵獻盡昭陽單閼凡五年】<br />
<br />
  高祖武皇帝五<br />
<br />
  天監十八年春正月甲申以尚書左僕射袁昂為尚書令右僕射王暕為左僕射【暕古限翻】太子詹事徐勉為右僕射 丁亥魏主下詔稱太后臨朝踐極歲將半紀【胡后臨朝見上卷十四年】宜稱詔以令宇内 辛卯上祀南郊 魏征西將軍張彞之子仲瑀上封事求銓削選格【瑀音禹上時掌翻銓量也選須絹翻下入選應選同】排抑武人不使豫清品於是喧謗盈路立榜大巷克期會集屠害其家彞父子晏然不以為意【方羽林虎賁立榜克期之初魏朝既不為之嚴加禁遏縱彞父子欲以為意奈之何哉】二月庚午羽林虎賁近千人【賁音奔近其靳翻】相帥至尚書省詬罵【帥讀曰率詬戶遘翻又古侯翻】求仲瑀兄左民郎中始均不獲【尚書左民郎晉武帝置】以瓦石擊省門上下懾懼莫敢禁討【懾之涉翻】遂持火掠道中薪蒿以杖石為兵器直造其第曳彞堂下捶辱極意焚其第舍始均踰垣走復還拜賊【造七到翻捶止蕊翻復扶又翻下不復誰復同】請其父命賊就毆擊生投之火中仲瑀重傷走免彞僅有餘息【言氣息奄奄僅未絶耳】再宿而死遠近震駭胡太后收掩羽林虎賁凶彊者八人斬之其餘不復窮治【治直之翻】乙亥大赦以安之因令武官得依資入選識者知魏之將亂矣時官員既少【少詩沼翻】應選者多吏部尚書李韶銓注不行大致怨嗟更以殿中尚書崔亮為吏部尚書亮奏為格制不問士之賢愚專以停解月日為斷【斷丁亂翻】沈滯者皆稱其能【沈持林翻】亮甥司空諮議劉景安與亮書曰殷周以鄉塾貢士【王制命鄉論秀士升之司徒曰選士司徒論秀士而升之學曰俊士】兩漢由州郡薦才【謂賢良文學孝廉之舉也事見漢紀】魏晉因循又置中正【事見六十九卷魏文帝黄初元年】雖未盡美應什收六七而朝廷貢才止求其文不取其理察孝廉唯論章句不及治道【治直吏翻】立中正不考才行空辯氏姓取士之途不博沙汰之理未精舅屬當銓衡宜改張易調【行下孟翻屬之欲翻董仲舒曰譬如琴瑟不調必改而更張之不調謂不和也易調之調徒釣翻音調也】如何反為停年格以限之天下士子誰復修厲名行哉【行下孟翻】亮復書曰汝所言乃有深致吾昨為此格有由而然古今不同時宜須異昔子產鑄刑書以救弊叔向譏之以正法【復扶又翻左傳昭六年鄭人鑄刑書叔向語子產書曰先王議事以制不為刑辟閑之以義糾之以政行之以禮守之以信制為禄位以勸其從嚴斷刑罰以威其淫懼其未也故誨之以忠聳之以行教之以務使之以和臨之以敬涖之以彊斷之以剛猶求聖哲之士明察之官忠信之長慈惠之師民於是可任使也而不生禍亂民知有辟則不忌其上並有争心以徵於書而儌幸以成之弗可為矣亂獄滋豐賄賂並行終子之世鄭其敗乎復書曰僑不才不能及子孫吾以救世也】何異汝以古禮難權宜哉【難乃旦翻】洛陽令代人薛琡【魏書官氏志西方叱于氏後改為薛氏琡之六翻又音俶】上書言黎元之命繫於長吏若以選曹唯取年勞不簡能否義均行鴈次若貫魚【行鴈貫魚皆以論資次先後以序而進也上時掌翻長知兩翻選須絹翻行戶剛翻】執簿呼名一吏足矣數人而用何謂銓衡書奏不報後因請見復奏乞令王公貴臣薦賢以補郡縣【見賢遍翻復扶又翻】詔公卿議之事亦寢其後甄琛等繼亮為吏部尚書【甄之人翻琛丑林翻】利其便己踵而行之魏之選舉失人自亮始也初燕燕郡太守高湖奔魏【事見一百一十一卷晉安帝隆安三年燕因肩翻】其子謐為侍御史【考異曰李百藥北齊書作諡北史作謐今從之】坐法徙懷朔鎮世居北邊遂<br />
<br />
  習鮮卑之事謐孫歡沈深有大志【沈持林翻】家貧執役在平城富人婁氏女見而奇之遂嫁焉始有馬得給鎮為函使【凡書表皆函封函使者使奉函詣京師也使疏吏翻】至洛陽見張彞之死還家傾貲以結客或問其故歡曰宿衛相帥焚大臣之第【帥讀曰率 考異曰北齊書云領軍張彞按彞未嘗為領軍故但云大臣】朝廷懼其亂而不問為政如此事可知矣財物豈可常守邪歡與懷朔省事雲中司馬子如【省事鎮吏也省悉景翻】秀容劉貴【魏太宗永興二年置秀容郡及秀容縣世祖真君七年置肆州秀容郡屬焉】中山賈顯智戶曹史咸陽孫騰外兵史懷朔侯景【史亦吏職也】獄掾善無尉景【善無縣前漢屬雁門郡後漢屬定襄郡拓跋氏置善無郡屬恒州李延夀曰秦漢置尉候官景之先有居此職者因氏焉】廣寧蔡儁【廣寧郡魏收志屬朔州隋并入朔州善陽縣】特相友善並以任俠雄於鄉里【高歡事始此】 夏四月丁巳大赦 五月戊戌魏以任城王澄為司徒京兆王繼為司空 魏累世彊盛東夷西域貢獻不絶又立互市以致南貨至是府庫盈溢胡太后嘗幸絹藏【藏徂浪翻】命王公嬪主從行者百餘人各自負絹稱力取之【稱尺證翻】少者不減百餘匹【少詩沼翻下同】尚書令儀同三司李崇章武王融負絹過重顛仆於地崇傷腰融損足太后奪其絹使空出時人笑之融太洛之子也【章武王太洛見一百三十二卷宋明帝泰始二年】侍中崔光止取兩匹太后怪其少對曰臣兩手唯堪兩匹衆皆愧之時魏宗室權倖之臣競為豪侈高陽王雍富貴冠一國宫室園圃侔於禁苑僮僕六千妓女五百出則儀衛塞道路【冠古玩翻塞悉則翻】歸則歌吹連日夜一食直錢數萬李崇富埒於雍而性儉嗇嘗謂人曰高陽一食敵我千日【埒力輟翻】河間王琛每欲與雍争富駿馬十餘匹皆以銀為槽窻戶之上玉鳳銜鈴金龍吐斾嘗會諸王宴飲酒器有水精鋒【後漢書大秦國出水精以為宫室柱及食器一本鋒作鍾】馬腦椀【本草衍義曰馬腦非石非玉自是一類有紅白黑色三種亦有紋如纒絲者生西國玉石間】赤玉巵【王逸論或問玉符曰赤如雞冠黄如蒸栗白如脂肪黑如點漆玉之符也】制作精巧皆中國所無又陳女樂名馬及諸奇寶復引諸王歷觀府庫金錢繒布不可勝計【復扶又翻勝音升】顧謂章武王融曰不恨我不見石崇恨石崇不見我【石崇事見八十一卷晉武帝太康三年】融素以富自負歸而惋歎三日【惋烏貫翻】京兆王繼聞而省之謂曰卿之貨財計不減於彼何為愧羨乃爾融曰始謂富於我者獨高陽耳不意復有河間繼曰卿似袁術在淮南不知世間復有劉備耳融乃笑而起【物盛而衰固其理也史言魏君臣驕侈乃其衰亂之漸復扶又翻下無復司復同】太后好佛營建諸寺無復窮已令諸州各建五級浮圖民力疲弊諸王貴人宦官羽林各建寺于洛陽相高以壯麗太后數設齋會施僧物動以萬計【好呼到翻數所角翻施式豉翻】賞賜左右無節所費不貲而未嘗施惠及民府庫漸虚乃減削百官禄力【禄在官所受之禄力在官所用白直也】任城王澄上表以為蕭衍常蓄窺覦之志【覦音俞】宜及國家彊盛將士旅力早圖混壹之功比年以來【比毗至翻】公私貧困宜節省浮費以周急務太后雖不能用常優禮之魏自永平以來【天監七年魏改元永平】營明堂辟雍役者多不過千人有司復借以修寺及供它役十餘年竟不能成起部郎源子恭上書以為廢經國之務資不急之費宜徹減諸役早圖就功使祖宗有嚴配之期【孝經孔子曰孝莫大于嚴父嚴父莫大于配天昔者周公郊祀后稷以配天宗祀文王于明堂以配上帝】蒼生有禮樂之富詔從之然亦不能成也 魏人陳仲儒請依京房立準以調八音有司詰仲儒京房立準今雖有其器曉之者鮮【詰去吉翻鮮息淺翻】仲儒所受何師出何典籍仲儒對言性頗愛琴又嘗讀司馬彪續漢書見京房準術成數昞然【司馬彪志曰京房六十律相生之法以上生下皆三生二以下生上皆三生四陽下生隂隂上生陽終于中呂而十二律畢矣中呂上生執始執始下生去滅上下相生終于南事而六十律畢矣夫十二律之變至于六十猶八卦之變至于六十四也宓羲作易紀陽氣之初以為律法建日冬至之宫以黄鍾為宫太簇為商姑洗為角林鍾為徵南呂為羽應鍾為變宫蕤賓為變徵此聲氣之元五音之正也故各終一日其餘以次運行當日者各自為宫而商徵以類從焉禮運曰五聲六律十二管還相為宫此之謂也以六十律分朞之日黄鍾自冬至始及冬至而復隂陽寒燠風雨之占生焉於以檢攝羣音考其高下苟非草木之聲則無所不合虞書曰律和聲此之謂也房又曰竹聲不可以度調故作準以定數準之狀如瑟長丈而十三弦隱間九尺以應黄鍾之律九寸中央一弦下有畫分寸以為六十律清濁之節房言律詳於劉歆所奏其術施行於史官侯部用之律術曰陽以圓為形其性動隂以方為節其性静動者數三静者數一以陽生隂倍之以隂生陽四之皆三而一陽生隂曰下生隂生陽曰上生上生不得過黄鍾之清濁下生不得及黄鍾之數實皆參天兩地圓蓋方覆六耦承奇之道也黄鍾律呂之首而生十一律者也其相生也皆三分而損益之是故十二律之得十七萬七千一百四十七是為黄鍾之實又以二乘而三約之是為下生林鍾之實又以四乘而三約之是為上生太簇之實推此上下以定六十律之實以九三之數萬九千六百八十三為法律為寸於準為尺不盈者十之所得為分又不盈十之所得為小分以其餘正其強弱以黄鍾十七萬七千一百四十七下生林鍾黄鍾為宫太簇商林鍾徵一日律九寸準九尺色育十七萬六千七百七十六下生謙待未知商謙待徵六日律八寸九分小分八微強準八尺九寸萬五千九百七十三執始十七萬四千七百六十二下生去滅執始為宫時息商去滅徵六日律八寸八分小分七大強準八尺八寸萬五千五百一十六丙盛十七萬二千四百一十下生安度丙盛為宫屈齊商安度徵六日律八寸七分小分六微弱準八尺七寸萬一千六百七十九分動十七萬八十九下生歸嘉分動為宫随期商歸嘉徵六日律八寸六分小分四強準八尺六寸八千一百五十二質未十六萬七千八百下生否與質未為宫形晉商否與徵六日律八寸五分小分二微強準八尺五寸四千九百四十五大呂十六萬五千八百八十八下生夷則大呂為宫夾鍾商夷則徵八日律八寸四分小分三弱準八尺四寸五千五百八分否十六萬三千六百五十四下生解形分否為宫開時商解形徵八日律八寸三分小分一強準八尺三寸二千八百五十一凌隂十六萬一千四百五十二下生去南凌隂為宫族嘉商去南徵八日律八寸二分小分一弱準八尺二寸五百一十四少出十五萬九千二百八十下生分積少出為宫争南商分積徵六日律八寸小分九強準八尺萬八千一百六十太蔟十五萬七千四百六十四下生南呂太簇為宫姑洗商南呂徵一日律八寸準八尺未知十五萬七千一百三十四下生白呂未知為宫南授商白呂徵六日律七寸九分小分八強準七尺九寸萬六千三百八十三時息十五萬五千三百四十四下生結躬時息為官變虞商結躬徵六日律七寸八分小分九少強準七尺八寸萬八千一百六十六屈齊十五萬三千二百五十三下生歸期屈齊為宫路時商歸期徵六日律七寸七分小分九弱準七尺七寸萬六千九百三十九随期十五萬一千一百九十下生未卯随期為宮形始商未卯徵六日律七寸六分小分八強準七尺六寸萬五千九百九十二形晉十四萬九千一百五十五下生夷汗形晉為宫依行商夷汗徵六日律七寸五分小分八弱準七尺五寸萬五千三百二十五夾鍾十四萬七千四百五十六下生無射夾鍾為宫中呂商無射徵六日律七寸四分小分九強凖七尺四寸萬八千四十八開時十四萬五千四百七十下生閉掩開時為宫中呂商閉掩徵八日律七寸三分小分九微弱準七尺三寸萬七千八百四十一族嘉十四萬三千五百一十三下生鄰齊族嘉為宫内負商鄰齊徵八日律七寸二分小分九微強準七尺二寸萬七千九百五十四争南十四萬一千五百八十二下生期保争南為宫物應商期保徵八日律七寸一分小分九強準七尺一寸萬八千三百二十七姑洗十三萬九千九百六十八下生應鍾姑洗為宫蕤賓商應鍾徵一日律七寸一分小分一微強準七尺一寸二千一百八十七南授十三萬九千六百七十下生分烏南授為宫南事商分烏徵六日律七寸小分九大強準七尺萬八千九百三十變虞十三萬八千八十四下生遲内變虞為官盛變商遲内徵六日律七寸小分一半強準七尺三千三十路時十三萬六千二百二十五下生未育路時為宫離宫商未育徵六日律六寸九分小分二微強準六尺九寸四千一百三十三形始十三萬四千三百九十二下生遲時形始為宮制時商遲時徵五日律六寸八分小分三弱準六尺八寸五千四百七十六依行十三萬二千五百八十二上生色育依行為宫謙待商色育徵七日律六寸七分小分三大強準六尺七寸七千五十九中呂十三萬一千七十二上生執始中呂為宫去滅商執始徵八日律六寸六分小分六弱準六尺六寸萬一千六百四十二南中十二萬九千三百八上生丙盛南中為宫安度商丙盛徵七日律六寸五分小分七微弱準六尺五寸萬三千六百八十五内負十二萬七千五百六十七上生分動内負為宫歸嘉商分動徵八日律六寸四分小分八強準六尺四寸萬五千四百五十物應十二萬五千八百五十上生質未物應為宫否與商質未徵七日律六寸三分小分九強準六尺三寸萬八千四百八十蕤賓十二萬四千四百一十六上生大呂蕤賓為宫夷則商大呂徵一日律六寸三分小分二微強準六尺三寸四千一百三十一南事十二萬四千一百五十四下生南事窮無商徵不為宮十日律六寸三分小分一弱準六尺三寸一千五百三十一盛變十二萬二千七百四十一上生分否盛變為宫解形商分否徵七日律六寸二分小分三大強準六尺二寸七千六十四離宮十二萬一千八百一十九上生凌隂離宫為宫去南商凌隂徵七日律六寸一分小分五微強凖六尺一寸萬二百二十七制時十一萬九千四百六十上生少出制時為宫分積商少出徵八日律六寸小分七弱準六尺萬三千六百二十林鍾十一萬八千九十八上生太蔟林鍾為宫南呂商太蔟徵一日律六寸準六尺謙待十一萬七千八百五十一上生未知謙待為宫白呂商未知徵五日律五寸九分小分九弱準五尺九寸萬七千二百一十三去滅十一萬六千五百八上生時息去滅為宫結躬商時息徵七日律五寸九分小分二弱準五尺九寸三千七百八十三安度十一萬四千九百四十上生屈齊安度為宫歸期商屈齊徵六日律五寸八分小分四弱準五尺八寸七千七百八十六歸嘉十一萬三千三百九十三上生随期歸嘉為宫未卯商随期徵六日律五寸七分小分六微強準五尺七寸萬一千九百九十九否與十一萬一千八百六十七上生形晉否與為宫夷汗商形晉徵五日律五寸六分小分八強準五尺六寸萬六千四百二十二夷則十一萬五百九十二上生夾鍾夷則為宫無射商夾鍾徵八日律五寸六分小分二弱準五尺六寸三千六百七十二解形十一萬九千一百三上生開時解形為宫閉掩商開時徵八日律五寸五分小分四強準五尺五寸八千四百六十五去南十萬七千六百三十五上生族嘉去南為宫鄰齊商族嘉徵八日律五寸四分小分六大強準五尺四寸萬三千四百六十八分積十萬六千一百八十八上生争南分積為宫期保商争南徵五日律五寸三分小分九半強準五尺三寸萬八千六百八十一南呂八萬四千九百七十六上生姑洗南呂為宮應鍾商姑洗徵一日律五寸三分小分三強準五尺三寸六千五百六十一白呂十萬四千七百五十六上生南授白呂為宫分烏商南授徵五日律五寸三分小分二強準五尺三寸四千三百七十一結躬十萬三千五百六十三上生變虞結躬為宫遲内商變虞徵六日律五寸二分小分六少強準五尺二寸萬二千一百一十四歸期十萬二千一百六十九上生路時歸期為宫未育商路時徵六日律五寸一分小分九微強準五尺一寸萬七千八百五十七未卯十萬七百九十四上生形始未卯為宫遲時商形始徵六日律五寸一分小分二微強準五尺一寸四千八十七夷汗九萬九千四百三十七上生依行夷汗為宫色育商依行徵七日律五寸小分五強準五尺萬二百二十無射九萬八千三百四上生中呂無射為宫執始商中呂徵八日律四寸九分小分九強準四尺九寸萬八千五百七十三閉掩九萬六千九百八十上生南中閉掩為宫丙盛商南中徵八日律四寸九分小分三弱準四尺九寸五千三百三十三鄰齊九萬五千六百七十五上生内負鄰齊為宫分動商内負徵七日律四寸八分小分六微強準四尺八寸萬一千九百六十六期保九萬四千三百八十八上生物應期保為宫質未商物應徵八日律四寸七分小分九微弱準四尺七寸萬八千七百七十九應鍾九萬三千三百一十二上生蕤賓應鍾為宫大呂商蕤賓徵一日律四寸七分小分四微強準四尺七寸八千十九分烏九萬三千一百一十七上生南事分烏窮次無徵不為宫七日律四寸七分小分三微強準四尺七寸六千五十九遲内九萬二千九十六上生盛變遲内為宫分否商盛變徵八日律四寸六分小分八弱準四尺六寸萬五千一百四十二未育九萬八百一十七上生離宫未育為宫凌隂商離宫徵八日律四寸六分小分一少強準四尺六寸二千七百五十二遲時八萬九千五百九十五上生制時遲時為宫少出商制時徵六日律四寸五分小分五強準四尺五寸萬二百一十五截管為律吹以考聲列以物氣道之本也術家以其聲微而體難知其分數不明故作準以代之準之聲明暢易達分寸又粗然弦以緩急清濁非管無以正也均其中弦令與黄鍾相得案畫以求諸律無不如數而應者矣音聲精微綜之者解晉書樂志宫中也中和之道無往而不理商強也謂金性堅強角觸也象諸陽觸物而生徵止也言物盛則止羽舒也陽氣將復萬物孳育而舒生宋白曰合宫通音謂之宫其音雄雄洪洪然開口吐聲謂之商其音鏘鏘倉倉然張牙湧唇謂之角其音喔喔礭礭然齒合唇開謂之徵其音倚倚然齒開唇聚謂之羽其音詡酗于吁然】遂竭愚思【思相吏翻】鑽研甚久頗有所得夫準者所以代律取其分數調校樂器竊尋調聲之體宫商宜濁徵羽宜清【徵陟里翻下同】若依公孫崇止以十二律聲而云還相為宫【還音旋】清濁悉足唯黄鍾管最長故以黄鍾為宫則往往相順若均之八音猶須錯采衆音配成其美若以應鍾為宫蕤賓為徵則徵濁而宫清雖有其韻不成音曲若以中呂為宫則十二律中全無所取今依京房書中呂為宫乃以去滅為商執始為徵然後方韻而崇乃以中呂為宫猶用林鍾為徵何由可諧【中呂陸德明曰中音仲乂如字】但音聲精微史傳簡略舊志準十三絃隱間九尺不言須柱以不【不讀曰否今刑統疏議多用以否二字蓋當時常用疑辭也】又一寸之内有萬九千六百八十三分微細難明仲儒私曾考驗準當施柱但前却柱中以約準分則相生之韻已自應合其中絃粗細【粗讀曰麤】須與琴宫相類施軫以調聲令與黄鍾相合中絃下依數畫六十律清濁之節其餘十二絃須施柱如筝即於中絃案盡一周之聲度著十二絃上【著直略翻】然後依相生之法以次運行取十二律之商徵商徵既定又依琴五調調聲之法以均樂器【五調之調徒釣翻調聲之調如字】然後錯采衆聲以文飾之若事有乖此聲則不和且燧人不師資而習火【古者未有火化燧人氏始鑽燧出火教民熟食】延夀不束修以變律【延夀即京房之師焦延夀也言無所師承而變十二律為六十律也孔子曰自行束脩以上吾未嘗無誨焉朱元晦注曰脩脯也十脡為束古者相見必執贄以為禮束脩其至薄者也】故云知之者欲教而無從心達者體知而無師苟有一毫所得皆關心抱豈必要經師受然後為奇哉尚書蕭寶寅奏仲儒學不師受輕欲制作不敢依許事遂寢 魏中尉東平王匡以論議數為任城王澄所奪憤恚復治其故棺【匡造棺見一百四十七卷七年數所角翻恚於避翻復扶又翻治直之翻】欲奏攻澄澄因奏匡罪狀三十餘條廷尉處以死刑【處昌呂翻】秋八月己未詔免死削除官爵以車騎將軍侯剛代領中尉【騎奇寄翻】三公郎中辛雄奏理匡【曹魏置尚書三公郎】以為歷奉三朝骨鯁之迹朝野具知【朝直遥翻】故高祖賜名曰匡先帝既已容之於前陛下亦宜寛之於後若終貶黜恐杜忠臣之口未幾復除匡平州刺史【幾居豈翻復扶又翻】雄琛之族孫也【辛琛見一百四十七卷天監六年琛丑林翻】 九月庚寅朔胡太后遊嵩高癸巳還宫太后從容謂兼中書舍人楊昱曰親姻在外不稱人心【從千容翻稱尺證翻】卿有聞慎勿諱隱昱奏揚州刺史李崇五車載貨相州刺史楊鈞造銀食器餉領軍元乂【相息亮翻】太后召乂夫妻泣而責之【泣而責之愛誨之意也】乂由是怨昱昱叔父舒妻武昌王和之妹也和即乂之從祖【從才用翻】舒卒元氏頻請别居昱父椿泣責不聽元氏恨之會瀛州民劉宣明謀反事覺逃亡义使和及元氏誣告昱藏匿宣明且云昱父定州刺史椿叔父華州刺史津並送甲仗三百具謀為不逞【華戶化翻】乂復構成之遣御仗五百人夜圍昱宅收之一無所獲太后聞其狀昱具對為元氏所怨太后解昱縛處和及元氏死刑【復扶乂翻處昌呂翻】既而乂營救之和直免官元氏竟不坐【史言靈后昵庇元乂以自為患】 冬十二月癸丑魏任城文宣王澄卒【任音壬卒子恤翻】 庚申魏大赦 是歲高句麗王雲卒世子安立【句如字又音駒麗力知翻】 魏以郎選不精大加沙汰【以水淘去沙石謂之沙汰故以喻去不肖】唯朱元旭辛雄羊深源子恭及范陽祖瑩等八人以才用見留餘皆罷遣深祉之子也【正始之初羊祉鎮梁益】<br />
<br />
  普通元年春正月乙亥朔改元大赦 丙子日有食之己卯以臨川王宏為太尉揚州刺史金紫光禄大夫<br />
<br />
  王份為尚書左僕射份奐之弟也【份彼陳翻王奐死于齊武帝永明十一年】左軍將軍豫寧威伯馮道根卒【五代志豫章郡建昌縣舊有豫寧縣宋白】<br />
<br />
  【曰漢建安中分建昌立西安縣晉太康元年改為豫寧縣】是日上春祠二廟【帝立太廟祀太祖文皇帝以上為六親廟皆同一堂共庭而别室又有小廟太祖太夫人廟也非嫡故别立廟皇帝每祭太廟訖乃詣小廟亦以一太牢如太廟禮有二廟令掌廟事】既出宫有司以聞上問中書舍人朱异曰吉凶同日今可行乎對曰昔衛獻公聞柳莊死不釋祭服而往【記檀弓曰衛太史柳莊寢疾公曰若疾革雖當祭必告公再拜稽首請於尸曰有臣柳莊也者非寡人之臣社稷之臣也聞之死請往不釋服而往遂以襚之】道根雖未為社稷之臣亦有勞王室臨之禮也上即幸其宅哭之甚慟 高句麗世子安遣使入貢二月癸丑以安為寧東將軍高句麗王【句音駒麗力知翻使疏吏翻】遣使者江法盛授安衣冠劍佩魏光州兵就海中執之送洛陽【魏皇興四年分青州置光州領東萊長廣東牟郡治掖】 魏太傅侍中清河文獻王懌美丰儀胡太后逼而幸之然素有才能輔政多所匡益好文學【好呼到翻】禮敬士人時望甚重侍中領軍將軍元乂在門下兼總禁兵恃寵驕恣志欲無極懌每裁之以法乂由是怨之衛將軍儀同三司劉騰權傾内外吏部希騰意奏用騰弟為郡人資乖越【人非其才為乖資非其次為越】懌抑而不奏騰亦怨之龍驤府長史宋維弁之子也【宋弁見用于魏孝文帝驤思將翻長之兩翻】懌薦為通直郎【通直郎即通直散騎侍郎隋後遂為寄禄官】浮薄無行【行下孟翻】乂許維以富貴使告司染都尉韓文殊父子謀作亂立懌【周官有染人漢有平準令主練染作采色後魏置司染都尉後齊太府寺屬官有司染署令丞陸德明染而艷翻劉音而險翻】懌坐禁止案驗無反狀得釋維當反坐【反坐誣告失實者以其所告之罪坐之】乂言於太后曰今誅維後有真反者人莫敢告乃黜維為昌平郡守【昌平縣漢屬上谷郡後魏置昌平郡屬燕州隋復廢郡為縣屬幽州】乂恐懌終為己害乃與劉騰密謀使主食中黄門胡定自列【主食主御食者也列陳也】云懌貨定使毒魏主若已得為帝許定以富貴帝時年十一信之秋七月丙子太后在嘉福殿未御前殿乂奉帝御顯陽殿騰閉永巷門太后不得出懌入遇乂於含章殿後乂厲聲不聽懌入懌曰汝欲反邪乂曰乂不反正欲縛反者耳命宗士及直齋執懌衣袂將入含章東省【魏置宗師宗士其屬也直齋直殿内齋閤者也屬直閣將引也送也】使人防守之騰稱詔集公卿議論懌大逆衆咸畏乂無敢異者唯僕射新泰文貞公游肇抗言以為不可【五代志新泰縣屬琅邪郡】終不下署【不下筆署名也】乂騰持公卿議入俄而得可【魏主可其奏也】夜中殺懌於是詐為太后詔自稱有疾還政於帝幽太后於北宮宣光殿宫門晝夜長閉内外斷絶騰自執管鑰帝亦不得省見【省悉景翻見賢遍翻】裁聽傳食而已太后服膳俱廢不免飢寒乃歎曰養虎得噬我之謂矣又使中常侍賈粲侍帝書密令防察動止乂遂與太師高陽王雍等同輔政帝謂乂為姨父乂與騰表裏擅權乂為外禦騰為内防常直禁省共裁刑賞政無巨細决於二人威振内外百僚重跡【重直龍翻言懼之甚不敢妄舉足而行步步踏陳迹也】朝野聞懌死莫不喪氣【喪息浪翻】胡夷為之剺面者數百人【為于偽翻剺里之翻胡夷臨喪剺面而哭哀甚】游肇憤邑而卒己卯江淮海並溢 辛卯魏主加元服大赦改元正<br />
<br />
  光 魏相州刺史中山文莊王熙英之子也【元英事魏孝文宣武數將兵有功相息亮翻】與弟給事黄門侍郎略司徒祭酒纂【自曹魏以來公府有東西閤祭酒】皆為清河王懌所厚聞懌死起兵於鄴上表欲誅元乂劉騰纂亡奔鄴後十日長史柳元章等帥城人鼓譟而入【帥讀曰率下同】殺其左右執熙纂并諸子置於高樓八月甲寅元乂遣尚書左丞盧同就斬熙於鄴街并其子弟熙好文學有風義【好呼到翻】名士多與之遊將死與故知書曰吾與弟俱蒙皇太后知遇兄據大州弟則入侍殷勤言色恩同慈母今皇太后見廢北宫太傅清河王横受屠酷【横戶孟翻】主上幼年獨在前殿君親如此無以自安故帥兵民欲建大義於天下但智力淺短旋見囚執【旋反也】上慙朝廷下愧相知本以名義干心不得不爾流腸碎首復何言哉凡百君子各敬爾儀為國為身善勗名節【復扶又翻為于偽翻下為知力為同】聞者憐之熙首至洛陽親故莫敢視前驍騎將軍刁整獨收其尸而藏之【驍堅堯翻騎奇寄翻】整雍之孫也【刁雍去晉入魏著功淮汝之間】盧同希乂意窮治熙黨與【治直之翻】鎖濟隂内史楊昱赴鄴【濟隂郡漢晉屬兖州魏屬西兖州濟子禮翻】考訊百日乃得還任乂以同為黄門侍郎元略亡抵故人河内司馬始賓始賓與略縛荻筏夜渡孟津詣屯留栗法光家【屯留縣自漢晉以來屬上黨郡姓譜栗姓栗陸氏之後漢長安富室有栗氏師古曰屯音純】轉依西河太守刁雙匿之經年時購略甚急略懼求送出境雙曰會有一死所難遇者為知己死耳願不以為慮略固求南奔雙乃使從子昌送略渡江【從才用翻】遂來奔上封略為中山王【為略後還魏張本】雙雍之族孫也乂誣刁整送略并其子弟收繫之御史王基等力為辯雪乃得免 甲子侍中車騎將軍永昌嚴侯韋叡卒【五代志零陵郡零陵縣舊分置永昌縣諡法服敵公莊曰嚴威而不猛曰嚴騎奇寄翻】時上方崇釋氏士民無不從風而靡獨叡自以位居大臣不欲與俗俯仰所行略如平日【史言韋叡於事佛之朝矯之以正幾於以道事君者】 九月戊戌魏以高陽王雍為丞相總攝内外與元乂同决庶務 初柔然佗汗可汗納伏名敦之妻候呂陵氏生伏跋可汗及阿那瓌等六子【可從刋入聲汗音寒瓌古回翻】伏跋既立忽亡其幼子祖惠求募不能得有巫地萬言祖惠今在天上我能呼之乃於大澤中施帳幄祀天神祖惠忽在帳中自云恒在天上【恒戶登翻】伏跋大喜號地萬為聖女納為可賀敦【柔然之主曰可汗其正室曰可賀敦】地萬既挾左道復有姿色伏跋敬而愛之信用其言干亂國政如是積歲祖惠浸長語其母曰我常在地萬家未嘗上天上天者地萬教我也【長知兩翻語牛倨翻上時掌翻】其母具以狀告伏跋伏跋曰地萬能前知未然勿為讒也既而地萬懼譖祖惠於伏跋而殺之侯呂陵氏遣其大臣具列等絞殺地萬伏跋怒欲誅具列等會阿至羅入寇【阿至羅虜之别種居北河之東世附於魏一曰阿至羅高車種】伏跋擊之兵敗而還【還從宣翻又如字】侯呂陵氏與大臣共殺伏跋立其弟阿那瓌為可汗阿那瓌立十日其族兄示帥衆數萬擊之【帥讀曰率】阿那瓌戰敗與其弟乙居伐輕騎奔魏【騎奇寄翻】示發殺侯呂陵氏及阿那瓌二弟【史言柔然亂】 魏清河王懌死汝南王悦了無恨元乂之意以桑落酒候之【水經注河東郡多徙民民有姓劉名墮者宿擅工釀採挹河流醖成芳酎懸食同枯枝之年排于桑落之辰故酒得其名香醑之色清白若滫漿焉别調氛氳不與他同蘭薰麝越自成馨邁方土之貢最為佳酌自王公庶友牽拂相招每云索郎索郎返語為桑落也更為籍徵之儁句中書之英談】盡其私佞乂大喜冬十月乙卯以悦為侍中太尉悦就懌子亶求懌服玩不時稱旨【既遷延不以時納所納者又不稱悦意也稱尺證翻】杖亶百下幾死【幾居依翻】 柔然可汗阿那瓌將至魏魏主使司空京兆王繼侍中崔光等相次迎之賜勞甚厚魏主引見阿那瓌於顯陽殿【勞力到翻見賢遍翻】因置宴置阿那瓌位於親王之下宴將罷阿那瓌執啟立于座後詔引至御座前阿那瓌再拜言曰臣以家難【難乃旦翻】輕來詣闕本國臣民皆已逃散陛下恩隆天地乞兵送還本國誅翦叛逆收集亡散臣當統帥遺民奉事陛下【帥讀曰率】言不能盡别有啟陳仍以啟授中書舍人常景以聞景爽之孫也【常爽見一百二十三卷宋文帝元嘉六年】十一月己亥魏立阿那瓌為朔方公蠕蠕王賜以衣服軺車【蠕人兖翻軺音遥】禄恤儀衛一如親王時魏方彊盛於洛水橋南御道東作四館道西立四里有自江南來降者處之金陵館三年之後賜宅於歸正里自北夷降者處燕然館賜宅於歸德里自東夷降者處扶桑館賜宅於慕化里自西夷降者處崦嵫館賜宅於慕義里【四館皆因四方之地為名金陵在江南燕然在漠北扶桑在東日所出崦嵫在西日所入山海經曰大荒之中暘谷上有扶桑日所出也灰野之山有樹青葉赤華名曰若木日所入也生崐崘西鳥鼠山西南曰崦嵫淮南子曰經細柳西方之地崦嵫日所入也十洲記曰扶桑在碧海中長數千丈一千餘圍兩榦同根更相依倚是以名扶桑降戶江翻處昌呂翻下同燕因肩翻崦依廉翻又依檢翻嵫音兹】及阿那瓌入朝以燕然館處之阿那瓌屢求返國朝議異同不决【朝直遥翻】阿那瓌以金百斤賂元乂遂聽北歸十二月壬子魏勑懷朔都督簡鋭騎二千護送阿那瓌達境首【境首猶言界首也騎奇寄翻】觀機招納若彼迎候宜賜繒帛車馬禮餞而返【繒慈陵翻】如不容受聽還闕庭其行裝資遣付尚書量給【量音良】辛酉魏以京兆王繼為司徒 魏遣使者劉善明來<br />
<br />
  聘始復通好【自齊明帝建元二年盧昶北歸之後魏不復遣使南聘至是復通復扶又翻】二年春正月辛巳上祀南郊 置孤獨園於建康以收養窮民【古者鰥寡孤獨廢疾者有養帝非能法古也祖釋氏須達多長者之為耳】 戊子大赦 魏南秦州氐反 魏近郡兵萬五千人【近郡近輔諸郡也】使懷朔鎮將楊鈞將之【將息亮翻下同】送柔然可汗阿那瓌返國尚書左丞張普惠上疏以為蠕蠕久為邊患今兹天降喪亂【喪息浪翻】荼毒其心蓋欲使之知有道之可樂革面稽首以奉大魏也【樂音洛稽音啓】陛下宜安民恭己以悦服其心阿那瓌束身歸命撫之可也乃更先自勞擾興師郊甸之内投諸荒裔之外救累世之勍敵資天亡之醜虜臣愚未見其可也【勍渠京翻】此乃邊將貪竊一時之功不思兵為凶器王者不得已而用之【用老子語意】况今旱暵方甚聖慈降膳【暵呼旱翻】乃以萬五千人使楊鈞為將欲定蠕蠕干時而動其可濟乎竊有顛覆之變楊鈞之肉其足食乎【用左傳楚孫叔敖斥伍參語意】宰輔專好小名【好呼到翻】不圖安危大計此微臣所以寒心者也且阿那瓌之不還負何信義臣賤不及議【漢自議郎以上皆得預朝廷大議尚書二丞於當時位不為卑而以為賤不及議蓋自曹魏以後朝廷大議止及八坐以上】文書所過【文書皆過尚書二丞之手】不敢不陳阿那瓌辭於西堂詔賜以軍器衣被雜采糧畜事事優厚【采與綵同畜許救翻】命侍中崔光等勞遣於外郭【勞力到翻】阿那瓌之南奔也其從父兄婆羅門帥衆數萬入討示發破之【從才用翻帥讀曰率】示發奔地豆干【魏書曰地豆干國在室韋西千餘里】地豆干殺之國人推婆羅門為彌偶可社句可汗【魏收曰魏言安静也】楊鈞表稱柔然已立君長【長知兩翻】恐未肯以殺兄之人郊迎其弟輕往虛返徒損國威自非廣加兵衆無以送其入北二月魏人使舊嘗奉使柔然者牒云具仁【牒云姓具仁名魏書官氏志内入諸姓有牒云氏奉使疏吏翻】往諭婆羅門使迎阿那瓌 辛丑上祀明堂 庚戌魏使假撫軍將軍邴虯討南秦叛氐【姓譜邴即丙姓】 魏元乂劉騰之幽胡太后也右衛將軍奚康生預其謀乂以康生為撫軍大將軍河南尹仍使之領左右【領仗身左右】康生子難當娶侍中左衛將軍侯剛女剛子乂之妹夫也乂以康生通姻深相委託三人率多俱宿禁中時或迭出以難當為千牛備身【御左右有千牛刀謂之防身刀千牛刀者利刀也取庖丁解數千牛而芒刃不頓為義千牛備身執千牛刀以侍左右者也】康生性麤武言氣高下乂稍憚之見于顔色【見賢遍翻】康生亦微懼不安甲午魏主朝太后于西林園文武侍坐酒酣迭舞【朝直遥翻坐徂卧翻】康生乃為力士儛【蓋為勇士進退坐作之氣勢而舞也儛與舞同】及折旋之際每顧視太后舉手蹈足瞋目頷首為執殺之勢【折之舌翻瞋七入翻】太后解其意而不敢言【解戶買翻】日暮太后欲攜帝宿宣光殿侯剛曰至尊己朝訖嬪御在南【宣光殿在洛陽北官元乂等幽胡太后於此魏主與嬪御居南宫故侯剛云然嬪毗賓翻】何必留宿康生曰至尊陛下之兒随陛下將東西更復訪誰【復扶又翻】羣臣莫敢應太后自起援帝臂【援于元翻引也】下堂而去康生大呼唱萬歲【呼火故翻】帝前入閤左右競相排閤不得閉康生奪難當千牛刀斫直後元思輔【直後官名直閤之屬也】乃得定帝既升宣光殿左右侍臣俱立西階下康生乘酒勢將出處分【處昌呂翻分扶問翻】為乂所執鎖於門下【以此知一夫之勇終受制於人也】光禄勲賈粲紿太后曰侍官懷恐不安【言其心懷恐懼也紿徒亥翻】陛下宜親安慰太后信之適下殿粲即扶帝出東序前御顯揚殿閉太后於宣光殿至晩乂不出令侍中黄門僕射尚書等十餘人就康生所訊其事處康生斬刑難當絞刑【處昌呂翻】乂與剛並在内矯詔决之康生如奏難當恕死從流難當哭辭父康生慷慨不悲曰我不反死汝何哭也時已昏闇有司驅康生赴市斬之尚食典御奚混與康生同執刀入内亦坐絞【尚食典御唐為尚食奉御進御必辨時禁先嘗之】難當以侯剛壻得留百餘日竟流安州【魏顯祖皇興二年置安州治方城領密雲廣陽安樂郡】久之乂使行臺盧同就殺之【魏太祖既得中山將北還慮中原有變乃於鄴中山置行臺後因之】以劉騰為司空八坐九卿常旦造騰宅【坐徂卧翻造七到翻】參其顔色然後赴省府亦有終日不得見者【得或作能非也】公私屬請【屬之欲翻】唯視貨多少舟車之利山澤之饒所在榷固刻剥六鎮交通互市歲入利息以巨萬萬計【巨萬萬萬也巨萬萬計者萬萬萬也少詩沼翻榷古岳翻】逼奪鄰舍以廣其居遠近苦之京兆王繼自以父子權位太盛固請以司徒讓車騎大將軍儀同三司崔光【騎奇寄翻】夏四月庚子以繼為太保侍中如故繼固辭不許壬寅以崔光為司徒侍中祭酒著作如故 魏牒云具仁至柔然婆羅門殊驕慢無遜避心責具仁禮敬具仁不屈婆羅門乃遣大臣丘升頭等將兵二千随具仁迎阿那瓌五月具仁還鎮【還懷朔鎮也將息浪翻下同】具道其狀阿那瓌懼不敢進上表請還洛陽 辛巳魏南荆州刺史桓叔興據所部來降【魏置南荆州見一百四十七卷天監十一年降戶江翻下同 考異曰梁帝紀七月叔興帥衆降蓋記奏到之日今從魏帝紀】六月丁卯義州刺史文僧明【此義州當置於齊安郡木蘭縣界蕭子顯齊志木蘭縣屬寧蠻左郡唐省木蘭縣入黄岡縣以下文裴邃復義州觀之恐義州與邊城皆置於安豐界】邊城太守田守德【沈約志宋文帝元嘉十五年以豫部蠻民立邊城左郡酈道元曰安豐縣故城今邊城郡治也此時梁境未得至安豐五代志黄岡縣界舊有邊城郡此正田守德所居之地】擁所部降魏皆蠻酋也【酋慈由翻】魏以僧明為西豫州刺史守德為義州刺史 癸卯琬琰殿火延燒後宫三千間 秋七月丁酉以大匠卿裴邃為信武將軍假節督衆軍討義州破魏義州刺史封夀於檀公峴【水經注决水出廬江雩婁縣大别山注云俗謂之檀公峴蓋大别之異名也又北過安豐縣東安豐故城今邊城郡治也此魏邊城郡峴戶典翻】遂圍其城夀請降復取義州【復扶又翻下聞復同】魏以尚書左丞張普惠為行臺將兵救之不及 【考異曰普惠傳云棄城走今從裴邃傳】以裴邃為豫州刺史鎮合肥邃欲絶夀陽隂結夀陽民李瓜花等為内應邃已勒兵為期日恐魏覺之先移揚州云魏始於馬頭置戍如聞復欲修白捺故城【馬頭置戍蓋即沈約志所謂馬頭太守治所而置之白捺當作馬頭東北或東南捺奴葛翻】若爾便相侵逼此亦須營歐陽設交境之備今板卒已集【板榦所以築城卒士卒也】唯聽信還揚州刺史長孫雉謀於僚佐皆曰此無修白捺之意宜以實報之録事參軍楊侃曰白捺小城本非形勝邃好狡數【好呼到翻】今集兵遣移恐有他意雉大寤曰録事可亟作移報之侃報移曰彼之纂兵【纂集也】想别有意何為妄構白捺它人有心予忖度之【詩巧言之辭度徒洛翻】勿謂秦無人也【左傳秦大夫繞朝之言】邃得移以為魏人已覺即散其兵瓜花等以失期遂相告發伏誅者十餘家稚觀之子【長孫觀道生之孫見一百三十三卷宋鬱林王元徽元年】侃播之子【楊播見一百四十卷齊明帝建武二年】初高車王彌俄突死【事見上卷天監十五年】其衆悉歸嚈噠後<br />
<br />
  數年嚈噠遣彌俄突弟伊匐帥餘衆還國伊匐擊柔然可汗婆羅門大破之婆羅門帥十部落詣凉州請降於魏柔然餘衆數萬相帥迎阿那瓌阿那瓌表稱本國大亂姓姓别居迭相抄掠當今北人鵠望待拯【言鵠立而望魏拯救也嚈益涉翻噠當割翻又宅軌翻帥讀曰率降戶江翻抄楚交翻】乞依前恩給臣精兵一萬送臣磧北撫定荒民【磧七迹翻】詔付中書門下博議凉州刺史袁翻以為自國家都洛以來蠕蠕高車迭相吞噬始則蠕蠕授首【謂佗汗也事見一百四十七卷天監七年】既而高車被擒【謂彌俄突也】今高車自奮於衰微之中克雪讐恥誠由種類繁多【種章勇翻】終不能相滅自二虜交鬬邊境無塵數十年矣此中國之利也今蠕蠕兩主相繼歸誠【兩主謂阿那瓌婆羅門】雖戎狄禽獸終無純固之節然存亡繼絶帝王本務若棄而不受則虧我大德若納而撫養則損我資儲或全徙内地則非直其情不願亦恐終為後患劉石是也【謂漢徙胡羯於内地至於晉世卒階劉石之亂】且蠕蠕尚存則高車猶有内顧之憂未暇窺窬上國若其全滅則高車跋扈之勢豈易可知【易以䜴翻】今蠕蠕雖亂而部落猶衆處處棊布以望舊主高車雖彊未能盡服也愚謂蠕蠕二主並宜存之居阿那瓌於東處婆羅門於西分其降民各有攸屬阿那瓌所居非所經見不敢臆度婆羅門請修西海故城以處之【處昌呂翻度徒洛翻】西海在酒泉之北去高車所居金山千餘里【此西海非王莽所置西海郡之西海但言在酒泉之北則别有西海故城也按北史蠕蠕傳西海郡即漢晉舊鄣袁翻又曰直張掖西北千二百里又按晉志漢獻帝興平二年武威太守張稚請置西海郡於居延蓋此即漢晉舊鄣也金山形如兜鍪其後突厥居金山之陽即此山】實北虜往來之衝要土地沃衍大宜耕稼宜遣一良將配以兵仗監護婆羅門【將即亮翻監工銜翻】因令屯田以省轉輸之勞其北則臨大磧野獸所聚【磧七迹翻】使蠕蠕射獵彼此相資足以自固外以輔蠕蠕之微弱内亦防高車之畔援【韓詩云畔援武強也鄭玄云跋扈也】此安邊保塞之長計也若婆羅門能收離聚散復興其國者【復扶又翻】漸令北轉徙度流沙則是我之外藩高車勍敵【勍渠京翻】西北之虞可以無慮如其姦回反覆不過為逋逃之寇於我何損哉朝議是之【朝直遥翻】九月柔然可汗俟匿伐詣懷朔鎮請兵且迎阿那瓌俟匿伐阿那瓌之兄也 冬十月録尚書事高陽王雍等奏懷朔鎮北吐若奚泉原野平沃請置阿那瓌於吐若奚泉【吐若奚泉在懷朔鎮北無結山下】婆羅門於故西海郡令各帥部落收集離散阿那瓌所居既在境外宜少優遣【少詩沼翻】婆羅門不得比之其婆羅門未降以前蠕蠕歸化者悉令州鎮部送懷朔鎮以付阿那瓌詔從之【為阿那瓌婆羅門皆叛去張本】 十一月癸丑魏侍中車騎大將軍侯剛加儀同三司【騎奇寄翻】 魏以東益南秦氐皆反庚辰以秦州刺史河間王琛為行臺以討之琛恃劉騰之勢【琛求為劉騰養子見上卷天監十七年琛丑林翻】貪暴無所畏忌大為氐所敗【敗補邁翻】中尉彈奏【彈徒丹翻】會赦除名尋復王爵 魏以安西將軍元洪超兼尚書行臺詣敦煌安置柔然婆羅門【敦徒門翻】<br />
<br />
  三年春正月庚子以尚書令袁昂為中書監吳郡太守王暕為尚書左僕射【暕古限翻】 辛亥魏主耕籍田 魏宋雲與惠生自洛陽西行四千里至赤嶺乃出魏境【赤嶺在唐鄯州鄯城縣西二百餘里】又西行再朞至乾羅國而還二月達洛陽得佛經一百七十部【魏遣宋雲求佛經事始見上卷天監十七年】 高車王伊匐遣使入貢于魏【匐蒲北翻使疏吏翻】夏四月庚辰魏以伊匐為鎮西將軍西海郡公高車王久之伊匐與柔然戰敗其弟越居殺伊匐自立 五月壬辰朔日有食之既癸巳大赦 冬十一月甲午領軍將軍始興忠武王憺卒【憺徒敢翻又徒濫翻】 乙巳魏主祀圜丘 初魏世祖以玄始歷浸疎【宋文帝元嘉二十九年魏行玄始歷】命更造新歷【更工衡翻】至是著作郎崔光表取盪寇將軍張龍祥等九家所上歷候驗得失合為一歷以壬子為元應魏之水德【壬癸水也水旺於子故以壬子為元上時掌翻】命曰正光歷丙午初行正光歷大赦 【考異曰後魏律歷志云歷成會孝明帝加元服改元正光因命曰正光歷按帝紀正光元年七月辛卯加元服三年十一月丙午行正光歷今從之】 十二月乙酉魏以車騎大將軍尚書右僕射元欽為儀同三司【騎奇寄翻】太保京兆王繼為太傅司徒崔光為太保 初太子統之未生也上養臨川王宏之子正德為子正德少麤險【少詩照翻】上即位正德意望東宫及太子統生正德還本賜爵西豐侯【沈約宋志臨川郡有西豐縣】正德怏怏不滿意常蓄異謀【怏於兩翻】是歲正德自黄門侍郎為輕車將軍頃之亡奔魏自稱廢太子避禍而來魏尚書左僕射蕭寶寅上表曰豈有伯為天子父作揚州【元年臨川王宏為揚州刺史】弃彼密親遠投他國不如殺之由是魏人待之甚薄正德乃殺一小兒稱為己子遠營葬地魏人不疑明年復自魏逃歸【復扶又翻 考異曰梁書正德傳普通六年為輕車將軍頃之奔魏七年自魏逃歸魏書蕭衍傳正光二年弟子正德來奔南史正德傳普通三年為輕車將軍頃之奔魏又自魏逃歸六年随豫章王北侵輒棄軍走北史蕭寶寅傳正光四年表論考課後乃云表論正德後乃云莫折大提反按大提反在正光五年唯南北史年月前後相近今從之】上泣而誨之復其封爵【為後正德納侯景張本】 柔然阿那瓌求粟為種【種章勇翻】魏與之萬石婆羅門帥部落叛魏亡歸嚈噠魏以平西府長史代人費穆兼尚書右丞西北道行臺將兵討之【魏收官氏志西方費連氏後改為費氏】柔然遁去穆謂諸將曰戎狄之性見敵即走乘虚復出若不使之破膽終恐疲于奔命【左傳巫臣遺子重子反書曰吾必使汝疲于奔命以死奔命者赴急之兵也復扶又翻】乃簡練精騎伏于山谷以步兵之羸者為外營【騎奇寄翻羸倫為翻】柔然果至奮擊大破之婆羅門為凉州軍所擒送洛陽<br />
<br />
  四年春正月辛卯上祀南郊大赦丙午祀明堂二月乙亥耕籍田 柔然大飢阿那瓌帥其衆入魏境表求賑給【帥讀曰率賑之忍翻】己亥魏以尚書左丞元孚為行臺尚書持節撫諭柔然孚譚之孫也【魏孝昌元年元譚為幽州都督後此三年按魏書譚太武之子蓋魏宗室多有同名者】將行表陳便宜以為蠕蠕久來彊大昔在代京常為重備今天祚大魏使彼自亂亡稽首請服【蠕人兖翻稽音啓】朝廷鳩其散亡禮送令返宜因此時善思遠策昔漢宣之世呼韓欵塞漢遣董忠韓昌領邊郡士馬送出朔方因留衛助【事見二十七卷漢宣帝甘露三年】又光武時亦使中郎將段彬置安集掾史随單于所在參察動静【事見四十四卷漢光武建武二十六年掾俞絹翻單音蟬】今宜略依舊事借其閒地聽其田牧粗置官屬示相慰撫【粗坐五翻】嚴戒邊兵因令防察使親不至矯詐疎不容反叛最策之得者也魏人不從柔然俟匿伐入朝于魏【朝直遥翻】 三月魏司空劉騰卒宦官為騰義息重服者四十餘人衰絰送葬者以百數朝貴送葬者塞路滿野【衰倉回翻塞悉則翻】 夏四月魏元孚持白虎幡勞阿那瓌于柔玄懷荒二鎮之間【懷荒鎮在柔玄鎮之東禦夷鎮之西勞力到翻】阿那瓌衆號三十萬隂有異志遂拘留孚載以輼車【應劭注漢帝曰轒輼匈奴車師古曰輼於云翻】每集其衆坐孚東廂稱為行臺甚加禮敬引兵而南所過剽掠【剽匹妙翻】至平城乃聽孚還有司奏孚辱命抵罪甲申魏遣尚書令李崇左僕射元纂帥騎十萬擊柔然【帥讀曰率騎奇寄翻】阿那瓌聞之驅良民二千公私馬牛羊數十萬北遁崇追之三千餘里不及而還【還從宣翻又如字】纂使鎧曹參軍于謹帥騎二千追柔然至郁對原前後十七戰屢破之【鎧可亥翻】謹忠之從曾孫也【于忠以保護胡太后恃功專恣從才用翻】性深沈有識量涉獵經史少時屏居田里不求仕進【沈持林翻少詩照翻屏必郢翻】或勸之仕謹曰州郡之職昔人所鄙【後漢梁竦曰州郡之職徒勞人耳】台鼎之位須待時來纂聞其名而辟之後帥輕騎出塞覘候屬鐵勒數千騎奄至【覘丑廉翻又丑豔翻屬之欲翻高車部或曰敕勒訛為鐵勒】謹以衆寡不敵退必不免乃散其衆騎使匿叢薄之間又遣人升山指麾若部分軍衆者【分扶問翻】鐵勒望見雖疑有伏兵自恃其衆進軍逼謹謹以常乘駿馬一紫一騧【騧古花翻】鐵勒所識乃使二人各乘一馬突陣而出鐵勒以為謹也争逐之謹帥餘軍擊其追騎鐵勒遂走謹因得入塞李崇長史鉅鹿魏蘭根說崇曰昔緣邊初置諸鎮地廣人稀或徵中原彊宗子弟或國之肺腑寄以爪牙中年以來有司號為府戶役同厮養【說式芮翻厮音斯養余亮翻】官婚班齒致失清流而本來族類各居榮顯顧瞻彼此理當憤怨宜改鎮立州分置郡縣凡是府戶悉免為民入仕次敘一準其舊文武兼用威恩並施此計若行國家庶無北顧之慮矣崇為之奏聞事寢不報【為後改鎮為州無及於事張本為于偽翻】 初元乂既幽胡太后常入直於魏主所居殿側曲盡佞媚帝由是寵信之乂出入禁中恒令勇士持兵以自先後【恒戶登翻先悉薦翻後胡茂翻】時出休於千秋門外施木欄楯【欄檻也楯食尹翻】使腹心防守以備竊發士民求見者遥對之而已其始執政之時矯情自飾以謙勤接物時事得失頗以關懷既得志遂自驕慢嗜酒好色貪吝寶賄與奪任情紀綱壞亂父京兆王繼尤貪縱與其妻子各受賂遺請屬有司【好呼到翻遺于季翻屬之欲翻】莫敢違者乃至郡縣小吏亦不得公選牧守令長率皆貪汚之人【守式又翻令郎定翻長知兩翻】由是百姓困窮人人思亂武衛將軍于景忠之弟也謀廢乂乂黜為懷荒鎮將【將即亮翻宋白曰後魏懷荒禦夷二鎮皆在蔚州界】及柔然入寇鎮民請糧景不肯給鎮民不勝忿遂反執景殺之未幾沃野鎮民破六韓拔陵聚衆反殺鎮將改元真王【勝音升幾居豈翻魏收曰破六韓單于之苗裔也初呼㕑泉入朝于漢為魏武所留遣其叔父右賢王去卑監本國戶魏氏方興率部南轉去卑遣弟右谷蠡王潘六奚帥軍北禦軍敗奚及五子俱沒于魏其子孫遂以潘六奚為氏後人遂誤以為破六韓又曰破洛汗 考異曰魏帝紀正光五年破六汗拔陵反詔臨淮王彧討之五月彧敗削官按令狐德棻周書賀拔勝傳衛可孤圍懷朔經年勝乃告急于彧然則拔陵反當在四年蓋帝紀因詔彧討拔陵而言之非拔陵于時始反也周書作破六韓今從之】諸鎮華夷之民往往響應拔陵引兵南侵遣别帥衛可孤圍武川鎮 【考異曰北史孤作瓌今從周書】又攻懷朔鎮尖山賀拔度拔及其三子允勝岳皆有材勇【魏收志尖山縣屬神武郡薛居正五代史周密傳神武川屬應州令狐德棻曰賀拔之先與魏氏同出隂山魏書官氏志内入諸姓有賀拔氏】懷朔鎮將楊鈞擢度拔為統軍三子為軍主以拒之 魏景明之初世宗命宦者白整為高祖及文昭高后鑿二佛龕於龍門山【龕口含翻此龍門山即伊闕山也為于偽翻下復為同】皆高百尺【高居報翻】永平中劉騰復為世宗鑿一龕【復扶又翻】至是二十四年凡用十八萬二千餘工而未成 秋七月辛亥魏詔見在朝官依令七十合解者【見賢遍翻七十而致仕合解所任】可給本官半禄以終其身 九月魏詔侍中太尉汝南王悦入居門下與丞相高陽王雍參决尚書奏事 冬十月庚午以中書監中衛將軍袁昂為尚書令即本號開府儀同三司【本號中衛將軍號】 魏平恩文宣公崔光疾篤魏主親撫視之拜其子勵為齊州刺史為之撤樂罷遊眺丁酉光卒帝臨哭之慟為減常膳【以光擁立之功也為于偽翻】光寛和樂善【樂音洛】終日怡怡未嘗忿恚【恚於避翻】于忠元乂用事以光舊德皆尊敬之事多咨决而不能救裴郭清河之死【裴郭死見上卷天監十四年清河王懌死見上元年】時人比之張禹胡廣光且死薦都官尚書賈思伯為侍講帝從思伯受春秋思伯雖貴傾身下士【下遐稼翻】或問思伯曰公何以能不驕思伯曰衰至便驕何常之有當時以為雅談 十一月癸未朔日有食之甲辰尚書左僕射王暕卒【暕古限翻】 梁初唯揚荆郢江湘梁益七州用錢交廣用金銀餘州雜以穀帛交易上乃鑄五銖錢肉好周郭皆備【華昭曰肉錢形也好孔也杜佑曰内郭為肉外郭為好孟康曰周郭周帀為郭也肉疾僦翻好呼到翻】别鑄無肉郭者謂之女錢民間私用女錢交易禁之不能止乃議盡罷銅錢十二月戊午始鑄鐵錢 魏以汝南王悦為太保<br />
<br />
  資治通鑑卷一百四十九<br />
<br />
<史部,編年類,資治通鑑>  <br>
   </div> 

<script src="/search/ajaxskft.js"> </script>
 <div class="clear"></div>
<br>
<br>
 <!-- a.d-->

 <!--
<div class="info_share">
</div> 
-->
 <!--info_share--></div>   <!-- end info_content-->
  </div> <!-- end l-->

<div class="r">   <!--r-->



<div class="sidebar"  style="margin-bottom:2px;">

 
<div class="sidebar_title">工具类大全</div>
<div class="sidebar_info">
<strong><a href="http://www.guoxuedashi.com/lsditu/" target="_blank">历史地图</a></strong>  
<a href="http://www.880114.com/" target="_blank">英语宝典</a>  
<a href="http://www.guoxuedashi.com/13jing/" target="_blank">十三经检索</a> 
<br><strong><a href="http://www.guoxuedashi.com/gjtsjc/" target="_blank">古今图书集成</a></strong> 
<a href="http://www.guoxuedashi.com/duilian/" target="_blank">对联大全</a> <strong><a href="http://www.guoxuedashi.com/xiangxingzi/" target="_blank">象形文字典</a></strong> 

<br><a href="http://www.guoxuedashi.com/zixing/yanbian/">字形演变</a>  <strong><a href="http://www.guoxuemi.com/hafo/" target="_blank">哈佛燕京中文善本特藏</a></strong>
<br><strong><a href="http://www.guoxuedashi.com/csfz/" target="_blank">丛书&方志检索器</a></strong> <a href="http://www.guoxuedashi.com/yqjyy/" target="_blank">一切经音义</a>  

<br><strong><a href="http://www.guoxuedashi.com/jiapu/" target="_blank">家谱族谱查询</a></strong>  <strong><a href="http://shufa.guoxuedashi.com/sfzitie/" target="_blank">书法字帖欣赏</a></strong> 
<br>

</div>
</div>


<div class="sidebar" style="margin-bottom:0px;">

<font style="font-size:22px;line-height:32px">QQ交流群9:489193090</font>


<div class="sidebar_title">手机APP 扫描或点击</div>
<div class="sidebar_info">
<table>
<tr>
	<td width=160><a href="http://m.guoxuedashi.com/app/" target="_blank"><img src="/img/gxds-sj.png" width="140"  border="0" alt="国学大师手机版"></a></td>
	<td>
<a href="http://www.guoxuedashi.com/download/" target="_blank">app软件下载专区</a><br>
<a href="http://www.guoxuedashi.com/download/gxds.php" target="_blank">《国学大师》下载</a><br>
<a href="http://www.guoxuedashi.com/download/kxzd.php" target="_blank">《汉字宝典》下载</a><br>
<a href="http://www.guoxuedashi.com/download/scqbd.php" target="_blank">《诗词曲宝典》下载</a><br>
<a href="http://www.guoxuedashi.com/SiKuQuanShu/skqs.php" target="_blank">《四库全书》下载</a><br>
</td>
</tr>
</table>

</div>
</div>


<div class="sidebar2">
<center>


</center>
</div>

<div class="sidebar"  style="margin-bottom:2px;">
<div class="sidebar_title">网站使用教程</div>
<div class="sidebar_info">
<a href="http://www.guoxuedashi.com/help/gjsearch.php" target="_blank">如何在国学大师网下载古籍?</a><br>
<a href="http://www.guoxuedashi.com/zidian/bujian/bjjc.php" target="_blank">如何使用部件查字法快速查字?</a><br>
<a href="http://www.guoxuedashi.com/search/sjc.php" target="_blank">如何在指定的书籍中全文检索?</a><br>
<a href="http://www.guoxuedashi.com/search/skjc.php" target="_blank">如何找到一句话在《四库全书》哪一页?</a><br>
</div>
</div>


<div class="sidebar">
<div class="sidebar_title">热门书籍</div>
<div class="sidebar_info">
<a href="/so.php?sokey=%E8%B5%84%E6%B2%BB%E9%80%9A%E9%89%B4&kt=1">资治通鉴</a> <a href="/24shi/"><strong>二十四史</strong></a>&nbsp; <a href="/a2694/">野史</a>&nbsp; <a href="/SiKuQuanShu/"><strong>四库全书</strong></a>&nbsp;<a href="http://www.guoxuedashi.com/SiKuQuanShu/fanti/">繁体</a>
<br><a href="/so.php?sokey=%E7%BA%A2%E6%A5%BC%E6%A2%A6&kt=1">红楼梦</a> <a href="/a/1858x/">三国演义</a> <a href="/a/1038k/">水浒传</a> <a href="/a/1046t/">西游记</a> <a href="/a/1914o/">封神演义</a>
<br>
<a href="http://www.guoxuedashi.com/so.php?sokeygx=%E4%B8%87%E6%9C%89%E6%96%87%E5%BA%93&submit=&kt=1">万有文库</a> <a href="/a/780t/">古文观止</a> <a href="/a/1024l/">文心雕龙</a> <a href="/a/1704n/">全唐诗</a> <a href="/a/1705h/">全宋词</a>
<br><a href="http://www.guoxuedashi.com/so.php?sokeygx=%E7%99%BE%E8%A1%B2%E6%9C%AC%E4%BA%8C%E5%8D%81%E5%9B%9B%E5%8F%B2&submit=&kt=1"><strong>百衲本二十四史</strong></a>  <a href="http://www.guoxuedashi.com/so.php?sokeygx=%E5%8F%A4%E4%BB%8A%E5%9B%BE%E4%B9%A6%E9%9B%86%E6%88%90&submit=&kt=1"><strong>古今图书集成</strong></a>
<br>

<a href="http://www.guoxuedashi.com/so.php?sokeygx=%E4%B8%9B%E4%B9%A6%E9%9B%86%E6%88%90&submit=&kt=1">丛书集成</a> 
<a href="http://www.guoxuedashi.com/so.php?sokeygx=%E5%9B%9B%E9%83%A8%E4%B8%9B%E5%88%8A&submit=&kt=1"><strong>四部丛刊</strong></a>  
<a href="http://www.guoxuedashi.com/so.php?sokeygx=%E8%AF%B4%E6%96%87%E8%A7%A3%E5%AD%97&submit=&kt=1">說文解字</a> <a href="http://www.guoxuedashi.com/so.php?sokeygx=%E5%85%A8%E4%B8%8A%E5%8F%A4&submit=&kt=1">三国六朝文</a>
<br><a href="http://www.guoxuedashi.com/so.php?sokeytm=%E6%97%A5%E6%9C%AC%E5%86%85%E9%98%81%E6%96%87%E5%BA%93&submit=&kt=1"><strong>日本内阁文库</strong></a> <a href="http://www.guoxuedashi.com/so.php?sokeytm=%E5%9B%BD%E5%9B%BE%E6%96%B9%E5%BF%97%E5%90%88%E9%9B%86&ka=100&submit=">国图方志合集</a> <a href="http://www.guoxuedashi.com/so.php?sokeytm=%E5%90%84%E5%9C%B0%E6%96%B9%E5%BF%97&submit=&kt=1"><strong>各地方志</strong></a>

</div>
</div>


<div class="sidebar2">
<center>

</center>
</div>
<div class="sidebar greenbar">
<div class="sidebar_title green">四库全书</div>
<div class="sidebar_info">

《四库全书》是中国古代最大的丛书,编撰于乾隆年间,由纪昀等360多位高官、学者编撰,3800多人抄写,费时十三年编成。丛书分经、史、子、集四部,故名四库。共有3500多种书,7.9万卷,3.6万册,约8亿字,基本上囊括了古代所有图书,故称“全书”。<a href="http://www.guoxuedashi.com/SiKuQuanShu/">详细>>
</a>

</div> 
</div>

</div>  <!--end r-->

</div>
<!-- 内容区END --> 

<!-- 页脚开始 -->
<div class="shh">

</div>

<div class="w1180" style="margin-top:8px;">
<center><script src="http://www.guoxuedashi.com/img/plus.php?id=3"></script></center>
</div>
<div class="w1180 foot">
<a href="/b/thanks.php">特别致谢</a> | <a href="javascript:window.external.AddFavorite(document.location.href,document.title);">收藏本站</a> | <a href="#">欢迎投稿</a> | <a href="http://www.guoxuedashi.com/forum/">意见建议</a> | <a href="http://www.guoxuemi.com/">国学迷</a> | <a href="http://www.shuowen.net/">说文网</a><script language="javascript" type="text/javascript" src="https://js.users.51.la/17753172.js"></script><br />
  Copyright &copy; 国学大师 古典图书集成 All Rights Reserved.<br>
  
  <span style="font-size:14px">免责声明:本站非营利性站点,以方便网友为主,仅供学习研究。<br>内容由热心网友提供和网上收集,不保留版权。若侵犯了您的权益,来信即刪。scp168@qq.com</span>
  <br />
ICP证:<a href="http://www.beian.miit.gov.cn/" target="_blank">鲁ICP备19060063号</a></div>
<!-- 页脚END --> 
<script src="http://www.guoxuedashi.com/img/plus.php?id=22"></script>
<script src="http://www.guoxuedashi.com/img/tongji.js"></script>

</body>
</html>
