<!DOCTYPE html PUBLIC "-//W3C//DTD XHTML 1.0 Transitional//EN" "http://www.w3.org/TR/xhtml1/DTD/xhtml1-transitional.dtd">
<html xmlns="http://www.w3.org/1999/xhtml">
<head>
<meta http-equiv="Content-Type" content="text/html; charset=utf-8" />
<meta http-equiv="X-UA-Compatible" content="IE=Edge,chrome=1">
<title>資治通鑒_255-資治通鑑卷二百五十四_255-資治通鑑卷二百五十四</title>
<meta name="Keywords" content="資治通鑒_255-資治通鑑卷二百五十四_255-資治通鑑卷二百五十四">
<meta name="Description" content="資治通鑒_255-資治通鑑卷二百五十四_255-資治通鑑卷二百五十四">
<meta http-equiv="Cache-Control" content="no-transform" />
<meta http-equiv="Cache-Control" content="no-siteapp" />
<link href="/img/style.css" rel="stylesheet" type="text/css" />
<script src="/img/m.js?2020"></script> 
</head>
<body>
 <div class="ClassNavi">
<a  href="/24shi/">二十四史</a> | <a href="/SiKuQuanShu/">四库全书</a> | <a href="http://www.guoxuedashi.com/gjtsjc/"><font  color="#FF0000">古今图书集成</font></a> | <a href="/renwu/">历史人物</a> | <a href="/ShuoWenJieZi/"><font  color="#FF0000">说文解字</a></font> | <a href="/chengyu/">成语词典</a> | <a  target="_blank"  href="http://www.guoxuedashi.com/jgwhj/"><font  color="#FF0000">甲骨文合集</font></a> | <a href="/yzjwjc/"><font  color="#FF0000">殷周金文集成</font></a> | <a href="/xiangxingzi/"><font color="#0000FF">象形字典</font></a> | <a href="/13jing/"><font  color="#FF0000">十三经索引</font></a> | <a href="/zixing/"><font  color="#FF0000">字体转换器</font></a> | <a href="/zidian/xz/"><font color="#0000FF">篆书识别</font></a> | <a href="/jinfanyi/">近义反义词</a> | <a href="/duilian/">对联大全</a> | <a href="/jiapu/"><font  color="#0000FF">家谱族谱查询</font></a> | <a href="http://www.guoxuemi.com/hafo/" target="_blank" ><font color="#FF0000">哈佛古籍</font></a> 
</div>

 <!-- 头部导航开始 -->
<div class="w1180 head clearfix">
  <div class="head_logo l"><a title="国学大师官网" href="http://www.guoxuedashi.com" target="_blank"></a></div>
  <div class="head_sr l">
  <div id="head1">
  
  <a href="http://www.guoxuedashi.com/zidian/bujian/" target="_blank" ><img src="http://www.guoxuedashi.com/img/top1.gif" width="88" height="60" border="0" title="部件查字,支持20万汉字"></a>


<a href="http://www.guoxuedashi.com/help/yingpan.php" target="_blank"><img src="http://www.guoxuedashi.com/img/top230.gif" width="600" height="62" border="0" ></a>


  </div>
  <div id="head3"><a href="javascript:" onClick="javascript:window.external.AddFavorite(window.location.href,document.title);">添加收藏</a>
  <br><a href="/help/setie.php">搜索引擎</a>
  <br><a href="/help/zanzhu.php">赞助本站</a></div>
  <div id="head2">
 <a href="http://www.guoxuemi.com/" target="_blank"><img src="http://www.guoxuedashi.com/img/guoxuemi.gif" width="95" height="62" border="0" style="margin-left:2px;" title="国学迷"></a>
  

  </div>
</div>
  <div class="clear"></div>
  <div class="head_nav">
  <p><a href="/">首页</a> | <a href="/ShuKu/">国学书库</a> | <a href="/guji/">影印古籍</a> | <a href="/shici/">诗词宝典</a> | <a   href="/SiKuQuanShu/gxjx.php">精选</a> <b>|</b> <a href="/zidian/">汉语字典</a> | <a href="/hydcd/">汉语词典</a> | <a href="http://www.guoxuedashi.com/zidian/bujian/"><font  color="#CC0066">部件查字</font></a> | <a href="http://www.sfds.cn/"><font  color="#CC0066">书法大师</font></a> | <a href="/jgwhj/">甲骨文</a> <b>|</b> <a href="/b/4/"><font  color="#CC0066">解密</font></a> | <a href="/renwu/">历史人物</a> | <a href="/diangu/">历史典故</a> | <a href="/xingshi/">姓氏</a> | <a href="/minzu/">民族</a> <b>|</b> <a href="/mz/"><font  color="#CC0066">世界名著</font></a> | <a href="/download/">软件下载</a>
</p>
<p><a href="/b/"><font  color="#CC0066">历史</font></a> | <a href="http://skqs.guoxuedashi.com/" target="_blank">四库全书</a> |  <a href="http://www.guoxuedashi.com/search/" target="_blank"><font  color="#CC0066">全文检索</font></a> | <a href="http://www.guoxuedashi.com/shumu/">古籍书目</a> | <a   href="/24shi/">正史</a> <b>|</b> <a href="/chengyu/">成语词典</a> | <a href="/kangxi/" title="康熙字典">康熙字典</a> | <a href="/ShuoWenJieZi/">说文解字</a> | <a href="/zixing/yanbian/">字形演变</a> | <a href="/yzjwjc/">金 文</a> <b>|</b>  <a href="/shijian/nian-hao/">年号</a> | <a href="/diming/">历史地名</a> | <a href="/shijian/">历史事件</a> | <a href="/guanzhi/">官职</a> | <a href="/lishi/">知识</a> <b>|</b> <a href="/zhongyi/">中医中药</a> | <a href="http://www.guoxuedashi.com/forum/">留言反馈</a>
</p>
  </div>
</div>
<!-- 头部导航END --> 
<!-- 内容区开始 --> 
<div class="w1180 clearfix">
  <div class="info l">
   
<div class="clearfix" style="background:#f5faff;">
<script src='http://www.guoxuedashi.com/img/headersou.js'></script>

</div>
  <div class="info_tree"><a href="http://www.guoxuedashi.com">首页</a> > <a href="/SiKuQuanShu/fanti/">四库全书</a>
 > <h1>资治通鉴</h1> <!--         下载:【右键另存为】即可 --></div>
  <div class="info_content zj clearfix">
  
<div class="info_txt clearfix" id="show">
<center style="font-size:24px;">255-資治通鑑卷二百五十四</center>
    資治通鑑卷二百五十四 宋 司馬光 撰<br />
<br />
  胡三省 音註<br />
<br />
  唐紀七十【起上章困敦十一月盡玄黓攝提格四月凡一年有奇】<br />
<br />
  僖宗惠聖恭定孝皇帝中之上<br />
<br />
  廣明元年十一月河中都虞侯王重榮作亂剽掠坊市俱空【重直龍翻剽匹妙翻】 宿州刺史劉漢宏怨朝廷賞薄【漢宏降見上卷七月賞盜而盜怨其賞薄彼果有以窺朝廷也】甲寅以漢宏為浙東觀察使【為漢宏為錢鏐所㓕張本】 詔河東節度使鄭從讜以本道兵授諸葛爽及代州刺史朱玫使南討黄巢【玫莫杯翻】乙卯以代北都統李琢為河陽節度使【代北已定李琢内徙亦以備黄巢也】 初黄巢將度淮豆盧瑑請以天平節钺授巢【黄巢初求天平節豆盧瑑欲以是中其欲】俟其到鎮討之盧曰盜賊無厭【厭於鹽翻】雖與之節不能止其剽掠【剽匹妙翻】不若急諸道兵扼泗州汴州節度使為都統賊既前不能入關必還掠淮浙偷生海渚耳從之既而淮北相繼告急稱疾不出 【考異曰驚聽録曰宰臣豆盧瑑奏緣淮南九驛使至泗州恐高駢固守城壘不遮截大寇黄巢必若過淮落寇之計又徵兵不及須且誘之請降節旄授鄆州節度使候其至止討亦不難宰臣盧言之不可奏以黄巢為國之患久矣昨與江西節制擁節而行攻劫荆南劫奪其節但徵諸道驍勇把截泗州因此不内使罷建雙旌乃使臣諸道而去尋汴州徐州兩道告急到京報黄巢過淮盧託疾不出按朝廷未嘗以江西節與巢借使與之安可復奪此驚聽録不足信也】京師大恐庚申東都奏黄巢入汝州境 辛酉以王重榮權知河中留後以河中節度使同平章事李都為太子少傅【以王重榮作亂不能制故召李都以河中授之】 汝鄭把截制置都指揮使齊克讓奏黄巢自稱天補大將軍轉牒諸軍云各宜守壘勿犯吾鋒吾將入東都即至京邑自欲問罪無預衆人【言自欲問罪於朝廷於衆人無預也】上召宰相議之豆盧瑑崔沆請關内諸鎮及兩神策軍守潼關壬戍日南至上開延英對宰相泣下【大盜將至無以禦之君相相對灑泣果何益哉】觀軍容使田令孜奏請選左右神策軍弓弩手守潼關臣自為都指揮制置把截使上曰侍衛將士不習征戰恐未足用令孜曰昔安禄山搆逆玄宗幸蜀以避之崔沆曰禄山衆纔五萬比之黄巢不足言矣豆盧瑑曰哥舒翰以十五萬衆不能守潼關【事見玄宗肅宗紀】今黄巢衆六十萬而潼關又無哥舒之兵若令孜為社稷計三川帥臣皆令孜腹心【謂陳敬瑄楊師立牛朂也帥所類翻】比於玄宗則有備矣上不懌【僖宗雖曰童昏此時此意豈不知高枕京邑之為樂越在草莽之為可憂也哉禍至而後憂之則無及矣古之明主居安而思危所以能常有其安也】謂令孜曰卿且為朕兵守潼關【為于偽翻】是日上幸左神策軍親閱將士令孜薦左軍馬軍將軍張承範右軍步軍將軍王師會左軍兵馬使趙珂【珂丘何翻】上召見三人【見賢遍翻】以承範為兵馬先鋒使兼把截潼關制置使師會為制置關塞糧料使珂為句當寨柵使【句古候翻當丁浪翻】令孜為左右神策軍内外八鎮及諸道兵馬都指揮制置招討等使飛龍使楊復恭為副使癸亥齊克讓奏黄巢已入東都境臣收軍退保潼關於關外置寨將士屢經戰鬭久乏資儲州縣殘破人煙殆絶東西南北不見王人凍餒交逼兵械刓弊【刓吾官翻鈍也】各思鄉閭恐一旦潰去乞早遣資糧及援軍上命選兩神策弩手得二千八百人令張承範等將以赴之【將即亮翻】丁卯黄巢陷東都留守劉允章帥百官迎謁巢入城勞問而已【帥讀曰率勞力到翻】閭里晏然允章迺之曾孫也【劉迺見二百三十卷德宗興元元年允章可謂忝厥祖矣】田令孜奏募坊市人數千以補兩軍辛未陜州奏東都已陷壬申以田令孜為汝洛晉絳同華都統將左右軍東討【左右神策軍陜失冉翻華戶化翻】是日賊陷虢州【九域志虢州東北至陜州八十五里】 以神策將羅元杲為河陽節度使【羅元杲亦田令孜之腹心】 以周岌為忠武節度使【周岌既殺薛能遂以忠武節授之岌逆及翻】初薛能遣牙將上蔡秦宗權調至蔡州【調徒弔翻自元和末廢彰義軍以蔡州屬忠武軍故得而調之】聞許州亂託云赴難【難乃旦翻】選募蔡兵遂逐刺史據其城及周岌為節度使即以宗權為蔡州刺史【為秦宗權以蔡州稱兵僭號張本】 乙亥張承範等將神策弩手京師【將即亮翻】神策軍士皆長安富家子賂宦官竄名軍籍厚得禀賜【禀給也禀賜猶言給賜也】但華衣怒馬【怒馬者鞭之以其怒而疾馳也】憑勢使氣未嘗更戰陳【更工衡翻陳讀曰陣】聞當出征父子聚泣多以金帛雇病坊貧人代行【唐置病坊於京城以養病人會要開元五年宋璟等奏悲田病坊從長安已來置使專知乞罷之至二十二年京城乞兒有疾病分置諸寺病坊至德二年兩京市各置普救病坊病坊之置其來久矣】往往不能操兵【操七刀翻】是日上御章信門樓臨遣之 【考異曰新傳曰帝餞令孜章信門賚遣豐優按令孜雖為招討都統賜節賚物其實不離禁闥是日所遣者承範等耳新傳云餞令孜誤也】承範進言聞黄巢擁數十萬之衆鼓行而西齊克讓以飢卒萬人依託關外復遣臣以二千餘人屯於關上又未聞為饋餉之計以此拒賊臣竊寒心願陛下趣諸道精兵早為繼援【趣讀曰促】上曰卿輩第行兵尋至矣丁丑承範等至華州會刺史裴䖍餘徙宣歙觀察使軍民皆逃入華山城中索然【華戶化翻索昔各翻】州庫唯塵埃鼠迹賴倉中猶有米千餘斛軍士裹三日糧而行 十二月庚辰朔承範等至潼關搜菁中【菁中草茂密處也史炤曰林菁】得村民百許使運石汲水為守禦之備與齊克讓軍皆絶糧士卒莫有鬭志是日黄巢前鋒軍抵關下白旗滿野不見其際克讓與戰賊小却俄而巢至舉軍大呼聲振河華【呼火故翻華戶化翻華山臨河言黄巢軍聲之盛撼振河山也】克讓力戰自午至酉始解士卒飢甚遂諠譟燒營而潰克讓走入關關左有谷平日禁人往來以榷征稅【榷訖岳翻】謂之禁阬賊至倉猝官軍忘守之【忘巫放翻】潰兵自谷而入谷中灌木壽藤茂密如織【灌木叢生之木壽藤即今之萬歲藤】一夕踐為坦塗承範盡散其輜囊以給士卒【輜囊謂輜重囊橐也輜重隨軍之物囊槖私裝也】遣使上表告急稱臣離京六日【離力智翻】甲卒未增一人餽餉未聞影響到關之日巨寇已來以二千餘人拒六十萬衆外軍飢潰蹋開禁阬【蹋與踏翻】臣之失守鼎鑊甘心朝廷謀臣愧顔何寄或聞陛下已議西巡【謂議幸蜀】苟鑾輿一動則上下土崩臣敢以猶生之軀奮冒死之語願與近密及宰臣熟議【近密謂兩中尉兩樞密】急徵兵以救關防則高祖太宗之業庶幾猶可扶持【幾居依翻】使黄巢繼安禄山之亡微臣勝哥舒翰之死辛巳賊急攻潼關承範悉力拒之自寅及申關上矢盡投石以擊之關外有天塹賊驅民千餘人入其中掘土填之【塹七艶翻掘其月翻填亭年翻】須臾即平引兵而度夜縱火焚關樓俱盡承範分兵八百人使王師會守禁阬比至【比必利翻】賊已入矣壬午旦賊夾攻潼關關上兵皆潰師會自殺承範變服帥餘衆脫走至野狐泉遇奉天援兵二千繼至承範曰汝來晩矣博野鳳翔軍還至渭橋【博野軍即穆宗長慶二年李寰帥以歸京師之兵也見二百四十二卷帥讀曰率】見所募新軍衣裘温鮮【新軍即田令孜所募坊市人以補兩軍者也】怒曰此輩何功而然我曹反凍餒遂掠之更為賊鄉導【鄉讀曰嚮】以趣長安【趣七喻翻】賊之攻潼關也朝廷以前京兆尹蕭廩為東道轉運糧料使廩稱疾請休官貶賀州司戶【賀州漢蒼梧郡之臨賀縣吳置臨賀郡唐置賀州京師東南四千一百三十里】黄巢入華州留其將喬鈐守之【鈐其亷翻】河中留後王重榮請降於賊【降戶江翻】癸未制以巢為天平節度使甲申以翰林學士承旨尚書左丞王徽為戶部侍郎翰林學士戶部侍郎裴澈為工部侍郎並同平章事以盧為太子賓客分司田令孜聞黄巢已入關恐天子責已乃歸罪於而貶之薦徽澈為相是夕攜飲藥死澈休之從子也【裴休見二百四十九卷宣宗大中六年】百官退朝聞亂兵入城布路竄匿【布路分路也朝直遥翻】令孜帥神策兵五百奉帝自金光門出【帥讀曰率下同長安城西面三門北來第一門曰開遠門第二門曰金光門第三門曰延平門】惟福穆澤壽四王及妃嬪數人從行【從才用翻下皆同】百官皆莫知之上奔馳晝夜不息從官多不能及車駕既去軍士及坊市民競入府庫盜金帛晡時黄巢前鋒將柴存入長安金吾大將軍張直方帥文武數十人迎巢於覇上巢乘金裝肩輿其徒皆被髪約以紅繒衣錦繡執兵以從甲騎如流輜重塞塗【被皮義翻衣於既翻騎奇計翻重直龍翻塞悉則翻】千里絡驛不絶民夾道聚觀尚讓歷諭之曰黄王起兵本為百姓【為于偽翻】非如李氏不愛汝曹汝曹但安居無恐巢館于田令孜第其徒為盜久不勝富【館古玩翻勝音升】見貧者往往施與之【施式豉翻】居數日各出大掠焚市肆殺人滿街巢不能禁尤憎官吏得者皆殺之 上趣駱谷【趣七喻翻】鳳翔節度使鄭畋謁上於道次 【考異曰續寶運録戊子帝至駱谷壻水驛乃下詔與牛朂楊師立陳敬瑄云今月七日已次駱谷壻水驛按此月庚辰朔戊子九日而詔云七日九誤為七也實録辛卯車駕次鳳翔鄭畋候謁於路舊畋傳云候駕於斜谷新紀辛卯次鳳翔丁酉至興元按甲申上離長安辛卯始次鳳翔太緩丁酉已至興元太速又路出駱谷則不過鳳翔及斜谷蓋車駕涉鳳翔之境而畋往見耳非鳳翔與斜谷也實録賊以數萬衆西追車駕而不言追不及又不言為誰所拒而還諸書皆無之人不取】請車駕留鳳翔上曰朕不欲密邇巨寇且幸興元徵兵以圖收復卿東扞賊鋒西撫諸蕃糾合鄰道勉建大勲畋曰道路梗澁奏報難通請得便宜從事許之戊子上至壻水【九域志洋州興道縣有壻水鎮相傳云仙人唐公昉盡室升天其壻不得偕升遂以名水誕矣】詔牛朂楊師立陳敬瑄諭以京城不守且幸興元若賊勢猶盛將幸成都宜豫為備擬庚寅黄巢殺唐宗室在長安者無遺類辛卯巢始入宫壬辰巢即皇帝位于含元殿畫皁繒為衮衣擊戰鼓數百以代金石之樂登丹鳳樓下赦書國號大齊改元金統謂廣明之號去唐下體而著黄家日月以為己符瑞【著側畧翻言唐字去丑口而著黄字為廣字合日月為明字也】唐官三品以上悉停任四品以下位如故以妻曹氏為皇后【考異曰實録巢傳立妻曲氏為皇后今從新傳】以尚讓為太尉兼中書令趙璋兼侍中崔璆楊希古並同平章事孟楷蓋洪為左右僕射知左右軍事【蓋古盍翻黄巢自以其軍分左右耳】費傳古為樞密使【費父沸翻姓也】以太常博士皮日休為翰林學士【陸游老學菴筆記曰該聞録言皮日休陷黄巢為翰林學士巢敗被誅今唐書取其事按尹師魯作大理寺丞皮子良墓誌稱曾祖日休避廣明之難徙籍會稽依錢氏官太常博士贈禮部尚書祖光業為吳越丞相父璨為元帥府判官三世皆以文雄江東據此則日休未嘗陷黄巢為其翰林學士被誅也小說謬妄無所不有師魯文章傳世且剛正有守非欺後世者】璆邠之子也【崔邠郾之兄也德宗朝為右補闕嘗論裴延齡有直聲子恐當作孫】時罷浙東觀察使在長安巢得而相之【璆之在浙東也固與巢信使往來又為之表奏朝廷】諸葛爽以代北行營兵屯櫟陽黄巢將碭山朱温屯東渭橋【碭山在漢碭縣界後魏置安陽縣治麻城隋開皇十八年改名碭山唐屬宋州九域志在單州東南九十里碭徒郎翻朱温始此】巢使温誘說之【說式芮翻】爽遂降於巢温少孤貧與兄昱存随母王氏依蕭縣劉崇家崇數笞辱之【按五代史温凶悍無賴崇患太祖慵惰不作業數笞責之獨崇母憐之時時自為櫛沐戒家人曰朱三非常人也宜善遇之數所角翻】崇母獨憐之戒家人曰朱三非常人也汝曹善遇之【朱温第三】巢以諸葛爽為河陽節度使爽赴鎮羅元杲兵拒之士卒皆弃甲迎爽元杲逃奔行在 鄭畋還鳳翔召將佐議拒賊皆曰賊勢方熾宜且從容以俟兵集【從千容翻從容舒徐不迫之貌言欲以緩圖之】乃圖收復畋曰諸君勸畋臣賊乎因悶絶仆地甃傷其面【鄭畋以將佐怠於勤王忠憤之氣一時鬱勃至於悶絶而僵仆於地故甃傷其面甃則救翻甓也】自午至明旦尚未能言會巢使者以赦書至監軍袁敬柔與將佐序立宣示代畋草表署名以謝巢監軍與巢使者宴樂奏將佐以下皆哭使者怪之幕客孫儲曰以相公風痺不能來故悲耳【痺必至翻】民間聞者無不泣畋聞之曰吾固知人心尚未厭唐賊授首無日矣乃刺指血為表遣所親間道詣行在【刺七亦翻下同間古莧翻】召將佐諭以逆順皆聽命復刺血與盟【復扶又翻】然後完城塹繕器械訓士卒密約鄰道合兵討賊鄰道皆許諾兵會於鳳翔時禁兵分鎮關中者尚數萬【禁兵分鎮關中即神策八鎮兵也】聞天子幸蜀無所歸畋使人招之皆往從畋畋分財以結其心軍勢大振 丁酉車駕至興元詔諸道各出全軍收復京師【悉所統之軍皆行謂之全軍】己亥黄巢下令百官詣趙璋第投名銜者復其官【名銜】<br />
<br />
  【顯官位姓名也】豆盧瑑崔沆及左僕射于琮右僕射劉鄴太子少師裴諗御史中丞趙濛刑部侍郎李溥京兆尹李湯扈從不及匿民間巢搜獲皆殺之廣德公主曰我唐室之女誓與于僕射俱死執賊刃不置賊并殺之盧尸戮之於市將作監鄭綦庫部郎中鄭係義不臣賊舉家自殺【唐屢更喪亂至于廣明舉家殉國猶不乏人恩義有結之素也】左金吾大將軍張直方雖臣於巢多納亡命匿公卿於複壁巢殺之初樞密使楊復恭薦處士河間張濬拜太常博士【處昌呂翻】遷度支員外郎黄巢逼潼關濬避亂商山上幸興元道中無供頓漢隂令李康以騾負糗糧數百馱獻之【漢隂漢中安陽縣地晉武帝改為安康縣唐至德二載更名漢隂縣屬金州九域志在州西北一百六十五里馱徒何翻以驢馬負物為馱唐逓馱每馱一百斤】從行軍士始得食上問康卿為縣令何能如是對曰臣不及此乃張濬員外教臣上召濬詣行在拜兵部郎中【唐諸司郎中從五品上員外郎從六品上】 義武節度使王處存聞長安失守號哭累日【號戶高翻】不俟詔命舉軍入援遣二千人間道詣興元衛車駕 黄巢遣使調河中【調徙釣翻】前後數百人吏民不勝其苦【勝音升】王重榮謂衆曰始吾屈節以紓軍府之患【屈節謂臣賊也紓商居翻緩也】今調財不已又將徵兵吾亡無日矣不如兵拒之衆皆以為然乃悉驅巢使者殺之巢遣其將朱温自同州弟黄鄴自華州合兵擊河中重榮與戰大破之獲糧仗四十餘船遣使與王處存結盟引兵營於渭北 【考異曰舊王處存傳曰時李都守河中降賊會王重榮斬偽使通使於處存乃同盟誓營於渭北時巢賊僭號天下藩鎮多受其偽命惟鄭畋守鳳翔鄭從讜守太原處存王重榮首倡義舉俄而鄭畋破賊前鋒王鐸自行在至故諸鎮翻然改圖以出勤王之師按鐸中和二年始至於時未也王重榮傳曰初重榮為河中馬步都虞侯巢賊據長安蒲帥李都不能拒稱臣於賊賊偽授重榮節度副使重榮以賊徵求無已欲拒之都曰吾兵微力寡絶之立見其患願以節鉞假公翌日都歸行在重榮知留後事乃斬賊使求援鄰藩北夢瑣言曰重榮始為牙將黄巢犯闕元戎李都奉偽畏重榮附者多因薦為副使一日忽謂都曰令公助賊陷一邦於不忠而又日加箕歛衆口紛紜倏忽變生何以遏也遽命斬其偽使都無以對因以軍印授重榮而去及都至行在朝廷又以前京兆尹竇潏間道至河中代都重榮迎之潏前為京兆尹有慘酷之名時謂之垜疊及至翌日進軍校于庭謂曰天子命重臣作鎮將遏賊衝安可輕議斥逐令北門出去且為惡者必一兩人而已爾等可言之潏不知軍衆皆重榮之親黨也衆皆不對重榮乃屏肅佩劔歷階而上謂潏曰為惡者非我而誰遂召潏之僕吏控馬及堦請依李都前例乃云速去潏不敢仰視躍馬復由北門而出新傳取之按十一月辛亥朔重榮已作亂掠坊市辛酉以重榮為留後都為太子少傅則都已去河中矣及黄巢犯闕都何嘗奉偽亦未嘗聞以潏代都今不取】 陳敬瑄聞車駕出幸遣步騎三千奉迎表請幸成都時從兵浸多興元儲偫不豐【偫丈里翻】田令孜亦勸上上從之中和元年【是年七月方改元】春正月車駕興元加牛朂同平章事陳敬瑄以扈從之人驕縱難制【從才用翻】有内園小兒先至成都【唐時給役於坊廐及内園者皆謂之小兒】遊於行宫笑曰人言西川是蠻今日觀之亦不惡敬瑄執而杖殺之 【考異曰新傳曰敬瑄殺五十人尸諸衢錦里耆舊傳曰有内園小兒三箇連手行遶行宫數内一人笑云云巡者亂打執之敬瑄咄曰今日且欲棒殺汝三五十輩必不令錯按三五十輩者敬瑄語也非實殺五十人也新傳誤】由是衆皆肅然敬瑄迎謁於鹿頭關辛未上至綿州東川節度使楊師立謁見【東川治梓州北至綿州一百六十八里見賢遍翻】壬申以工部侍郎判度支蕭遘同平章事 鄭畋約前朔方節度使唐弘夫涇原節度使程宗楚同討黄巢巢遣其將王暉齎詔召畋畋斬之遣其子凝績詣行在凝績追及上於漢州【自綿州西南至漢州一百九十里】 丁丑車駕至成都【自漢州西南至成都八十五里】館於府舍【館古玩翻就西川府舍為行宫】 上遣使趣高駢討黄巢【趣讀曰促】道路相望駢終不出兵上至蜀猶冀駢立功詔駢巡内刺史及諸將有功者自監察至常侍聽以墨勑除訖奏聞 裴澈自賊中奔詣行在時百官未集乏人草制右拾遺樂朋龜謁田令孜而拜之由是擢為翰林學士張濬先亦拜令孜令孜嘗召宰相及朝貴飲酒【朝直遥翻】濬耻於衆中拜令孜乃先謁令孜謝酒及賓客畢集令孜言曰令孜與張郎中清濁異流嘗蒙中外【中外謂與之相表裏】既慮玷辱何憚改更【更工衡翻】今日於隱處謝酒則又不可濬慙懼無所容 二月乙卯朔以太子少師王鐸守司徒兼門下侍郎同平章事 丙申加鄭畋同平章事 加淮南節度使高駢東面都統加河東節度使鄭從讜兼侍中依前行營招討使代北監軍陳景思 【考異曰實録作景斯今從薛居正五代史】帥沙陀酋長李友金及薩葛安慶吐谷渾諸部入援京師【帥讀曰率酋慈由翻長知兩翻薩桑葛翻】至絳州將濟河絳州刺史瞿稹亦沙陀也【瞿權俱翻稹止忍翻】謂景思曰賊勢方盛未可輕進不若且還代北募兵遂與景思俱還雁門 以樞密使楊復光為京西南面行營都監 黄巢以朱温為東南面行營都虞侯將兵攻鄧州三月辛亥陷之執刺史趙戒因戍鄧州以扼荆襄【九域志鄧州南至襄州一百八十里襄州南至荆州四百五十七里】 壬子加陳敬瑄同平章事甲寅敬瑄奏遣左黄頭軍使李鋌將兵擊黄巢【西川黄頭軍崔安濳所置也事始見上卷乾符六年鋌時廷翻】 辛酉以鄭畋為京城四面諸軍行營都統賜畋詔凡蕃漢將士赴難有功者【難乃旦翻】並聽以墨勑除官畋奏以涇原節度使程宗楚為副都統前朔方節度使唐弘夫為行軍司馬黄巢遣其將尚讓王播帥衆五萬寇鳳翔【王播新書作王璠】畋使弘夫㐲兵要害自以兵數千多張旗幟踈陳於高岡【陳讀曰陣】賊以畋書生輕之鼓行而前無復行伍【行伍剛翻】伏賊大敗於龍尾陂【新舊書皆作龍尾坡惟舊紀作陂鳳翔府岐山縣唐初治張堡武德七年移治龍尾城在平陽故城之東北】斬首二萬餘級伏尸數十里 有書尚書省間為詩以嘲賊者尚讓怒應在省官及門卒悉抉目倒懸之【抉於決翻】大索城中能為詩者【索山客翻】盡殺之識字者給賤役凡殺三千餘人 瞿稹李友金至代州募兵踰旬得三萬人皆北方雜胡屯於崞西【代州崞縣之西也崞音郭】獷悍暴横【獷古猛翻悍下罕翻又侯幹翻横戶孟翻】稹與友金不能制友金乃說陳景思曰【說式芮翻】金雖有衆數萬苟無威信之將以統之終無成功吾兄司徒父子勇畧過人為衆所服驃騎誠奏天子赦其罪召以為帥【李國昌以平龎勛功檢校司徒驃騎唐制武散階極品唐自高力士以來宦官多官至驃騎故以稱景思】則代北之人一麾響應狂賊不足平也景思以為然遣使詣行在言之詔如所請友金以五百騎齎詔詣達靼迎之【李克用入達靼見上卷廣明元年】李克用帥達靼諸部萬人赴之 【考異曰實録陳景斯齎詔入達靼召李克用軍屯蔚州克用因大掠雁門以北軍鎮薛居正五代史先是景思與李友金沙陀諸部五千騎南赴京師友金即武皇之族父也中和元年二月友金軍至絳州將度河刺史瞿稹謂景思曰巢賊方盛不如且還代北徐圖利害四月友金旋軍雁門瞿稹至代州半月之間募兵三萬營於崞縣之西其軍皆北邊五部之衆不閑軍法瞿稹李友金不能制友金謂景思云云景思然之促奏行在天子乃以武皇為雁門節度使仍令以本軍討賊李友金五百騎齎詔召武皇於逹靼武皇即帥逹靼諸部萬人趨雁門按景思請赦國昌父子而克用至者蓋國昌已老獨使克用來耳是歲克用但攻掠太原又陷忻代二州明年十二月始自忻代留後除雁門節度使蓋此際止赦其罪復為大同防禦使乃陷忻代自稱留後朝廷再召之始除雁門薛史誤也新表中和二年以河東忻代二州隸雁門節度更大同節度為雁門節度治代州此其證也】 羣臣追從車駕者稍集成都南北司朝者近二百人【朝直遥翻近其靳翻】諸道及四夷貢獻不絶蜀中府庫充實與京師無異賞賜不乏士卒欣悦 黄巢得王徽逼以官徽陽瘖不從月餘逃奔河中遣人間道奉絹表詣行在【間古莧翻】詔以徽為兵部尚書 前夏綏節度使諸葛爽復自河陽奉表自歸【去年黄巢入關諸葛爽降之】即以為河陽節度使 宥州刺史拓跋思恭【開元十六年以六胡州殘人置宥州乾元元年理經畧軍後移治長澤縣長澤漢朔方郡三封縣地 考異曰歐陽修五代史作拓拔思敬意謂薛史避國諱耳按舊唐書實録皆作思恭實録天復二年九月武定軍節度使李思敬以城降王建思敬本姓拓跋鄜夏節度使思恭保大節度使思孝之弟也思孝致仕以思敬為保大留後遂升節度又徙武定軍新唐書党項傳曰思恭為定難節度使卒弟思諫代為節度思孝為保大節度以老薦弟思敬為保大留後俄為節度然則思恭思敬乃是兩人思敬後附李茂貞或賜國姓故更姓李修合以為一人誤也】本党項羌也【新書党項以姓别為部落而拓跋氏最疆】糾合夷夏兵會鄜延節度使李孝昌於鄜州同盟討賊【夏戶雅翻鄜音夫】奉天鎮使齊克儉遣使詣鄭畋求自效甲子畋傳檄天下藩鎮合兵討賊時天子在蜀詔令不通天下謂朝廷不能復振及得畋檄爭兵應之賊懼不敢復窺京西【復扶又翻】 夏四月戊寅朔加王鐸兼侍中 以拓跋思恭權知夏綏節度使【為拓跋氏疆盛遂跨㨿西夏張本】 黄巢以其將王玫為邠寧節度使邠州通塞鎮將朱玫起兵討之【玫莫杯翻】讓别將李重古為節度使自將兵討巢是時唐弘夫屯渭北王重榮屯沙苑王處存屯渭橋拓跋思恭屯武功鄭畋屯盩厔弘夫乘龍尾之捷進薄長安壬午黄巢帥衆東走程宗楚先自延秋門入【長安苑城有門西出謂之延秋門】弘夫繼至處存帥銳卒五千夜入城【帥讀曰率下同】坊市民喜爭讙呼出迎官軍【讙讀曰喧】或以瓦礫擊賊【礫狼狄翻】或拾箭以供官軍宗楚等恐諸將分其功不報鳳翔鄜夏【句斷】軍士釋兵入第舍掠金帛妓妾【妓渠綺翻】處存令軍士繫白為號【詢趨翻繒頭也以約髪謂之頭】坊市少年或竊其號以掠人賊露宿覇上【宿無室廬曰露宿】詗知官車不整【詗翾正翻又火迥翻】且諸軍不相繼引兵還襲之自諸門分入大戰長安中宗楚弘夫死 【考異曰舊紀傳新傳皆云弘夫敗在二年六月驚聽録唐年補録新紀實録皆在此年四月新紀日尤詳今從之】軍士重負不能走是以甚敗死者什八九處存收餘衆還營丁亥巢復入長安怒民之助官軍縱兵屠殺流血成川謂之洗城於是諸軍皆退賊勢愈熾賊所署同州刺史王溥華州刺史喬謙商州刺史宋巖聞巢弃長安皆率衆奔鄧州朱温斬溥謙釋巖使還商州 庚寅拓拔思恭李孝昌與賊戰於王橋不利 詔以河中留後王重榮為節度使 賊衆上黄巢尊號曰承天應運啓聖睿文宣武皇帝 有雙雉集廣陵府舍占者以為野鳥來集城邑將空之兆高駢惡之【惡烏路翻】乃移檄四方云將入討黄巢悉巡内兵八萬舟二千艘旌旗甲兵甚盛五月乙未出屯東塘【東塘在今揚州城東即今灣頭至宜陵一帶塘岸也艘蘇遭翻考異曰妖亂志曰自五月十二日出東塘至九月六日歸府九十餘日禳雉雊之變也按五月十二日至九月六日乃是一百六十三日非九十餘日今從舊傳】諸將數請行期【數所角翻】駢託風濤為阻或云時日不利竟不 李克用牒河東稱奉詔將兵五萬討黄巢令具頓逓【緣道設酒食以供軍為頓置郵驛為逓】鄭從讜閉城以備之克用屯於汾東從讜犒勞【犒口到翻勞力到翻】給其資糧累日不克用自至城下大呼【呼火故翻】求與從讜相見從讜登城謝之癸亥復求軍賞給【復扶又翻】從讜以錢千緡米千斛遺之【遺唯季翻】甲子克用縱沙陀剽掠居民【剽匹妙翻】城中大駭從讜求救於振武節度使契苾璋【契欺訖翻】璋引突厥吐谷渾救之破沙陀兩寨克用追戰至晉陽城南璋引兵入城沙陀掠陽曲榆次而戰 黄巢之克長安也忠武節度使周岌降之【去年十一月授周岌忠武節十二月而黄巢克長安】岌嘗夜宴急召監軍楊復光【先是以楊復光為忠武監軍屯鄧州扼賊右衝巢既陷長安遣朱温屯鄧州復光遂至許州依周岌故召之夜宴】左右曰周公臣賊將不利於内侍不可往【唐内侍省以内侍監為之長内侍為貳故左右以稱復光】復光曰事已如此義不圖全即詣之酒酣岌言及本朝【朝直遥翻】復光泣下良久曰丈夫所感者恩義耳公自匹夫為公侯奈何捨十八葉天子而臣賊乎【自高祖之僖宗十八世】岌亦流涕曰吾不能獨拒賊故貌奉而心圖之今日召公正為此耳【為于偽翻】因瀝酒為盟【史炤曰以酒滴瀝也】是夕復光遣其養子守亮殺賊使者於驛時秦宗權據蔡州不從岌命復光將忠武兵三千詣蔡州說宗權同舉兵討巢【說式芮翻】宗權遣其將王淑將兵三千從復光擊鄧州逗留不進復光斬之併其軍分忠武八千人為八都遣牙將鹿晏弘晉暉王建韓建張造李師泰龎從等八人將之 【考異曰劉恕十國紀年上云八都而下止有王建等八人姓名諸書不可考故也王建始此】 王建舞陽人韓建長社人晏弘暉造師泰皆許州人也復光帥八都與朱温戰敗之【帥讀曰率敗補邁翻】遂克鄧州逐北至藍橋而還【藍橋在藍田關南還從宣翻又如字】 昭義節度使高潯會王重榮攻華州克之 六月戊戍以鄭畋為司空兼門下侍郎同平章事都統如故 李克用遇大雨引兵北還陷忻代二州因留居代州 【考異曰唐末見聞録六月三十日沙陀軍却囘收却忻代州太祖紀年録遇大雨六月二十三日班師雁門薛居正五代史與紀年録同按忻代先屬河東中和二年始割隸雁門今從見聞録實録】鄭從讜遣教練使論安等軍百井以備之 邠寧節度副使朱玫屯興平【興平縣在長安西八十五里余靖曰周文丘今之興平】黄巢將王播圍興平玫退屯奉天及龍尾陂 西州黄頭軍使李鋌將萬人鞏咸將五千人【鞏姓也周卿士鞏簡公晉大夫鞏朔】屯興平為二寨與黄巢戰屢捷陳敬瑄遣神機營使高仁厚將二千人益之【神機營亦崔安潜置事見上卷乾符六年】 秋七月丁巳改元赦天下【改元中和】 庚申以翰林學士承旨兵部侍郎韋昭度同平章事 論安自百井擅還鄭從讜不解鞾衫斬之滅其族【鞾與靴同 考異曰唐末見聞録六月三十日沙陀收却忻代州使司差教練使論安軍使王蟾高弁囘鶻吐蕃等軍於百井下寨守禦當月内論安等拔寨却迴到府按當月内即三十日也一日之中不容有爾許事必非也又曰至七月十四日相公排飯大將等於坐上把起論安不脱靴於毬塲内處置族滅其家又差都頭温漢臣將兵依前於百井下寨當月契苾尚書領兵馬却歸振武今從之】更遣都頭温漢臣將兵屯百井契苾璋引兵還振武 初車駕至成都蜀軍賞錢人三緡田令孜為行在都指揮處置使每四方貢金帛輒頒賜從駕諸軍無虛月不復及蜀軍【復扶又翻】蜀軍頗有怨言丙寅令孜宴土客都頭【土軍蜀軍客軍從駕諸軍唐之中世以諸軍總帥為都頭至其後也一部之軍謂之一都其部帥呼為都頭】以金杯行酒因賜之諸都頭皆拜而受西川黄頭軍使郭琪獨不受起言曰諸將月受俸料豐贍有餘常思難報豈敢無厭【俸扶用翻厭於鹽翻】顧蜀軍與諸軍同宿衛而賞賚懸殊頗有觖望【觖古穴翻怨望】恐萬一致變願軍容減諸將之賜以均蜀軍使土客如一則上下幸甚令孜默然有間曰【間如字】汝嘗有何功對曰琪生長山東【長知兩翻】征戍邉鄙嘗與党項十七戰契丹十餘戰【契欺訖翻】金創滿身【創初良翻】又嘗征吐谷渾傷脇腸出線縫復戰【復扶又翻】令孜乃自酌酒於别樽以賜琪琪知其毒不得已再拜飲之歸殺一婢吮其血以解毒吐黑汁數升【吮如兖翻吐土故翻】遂帥所部作亂【帥讀曰率】丁卯焚掠坊市令孜奉天子保東城閉門登樓命諸軍擊之琪引兵還營陳敬瑄命都押牙安金山將兵攻之琪夜突圍出奔廣都【隋改廣都縣為雙流縣唐龍朔二年復分雙流置廣都縣屬成都府九域志在府西四十五里】從兵皆潰【從才用翻】獨廳吏一人從息於江岸琪謂廳吏曰陳公知吾無罪然軍府驚擾不可以莫之安也汝事吾能始終今有以報汝汝齎吾印劒詣陳公曰郭琪走度江我以劒擊之墜水尸隨湍流下矣得其印劒以獻陳公必據汝所言牓懸印劒於市以安衆汝當獲厚賞吾家亦保無恙吾自此適廣陵歸高公【言欲奔揚州歸高駢也】後數日汝可密以語吾家也【語牛倨翻】遂解印劒授之而逸廳吏以獻敬瑄果免琪家上日夕專與宦者同處議天下事待外臣殊踈薄【處昌呂翻外臣謂外廷之臣宰相以下百執事皆是也】庚午左拾遺孟昭圖上疏以為治安之代遐邇猶應同心【治直吏翻】多難之時中外尤當一體【難乃旦翻】去冬車駕西幸不告南司遂使宰相僕射以下悉為賊所屠【謂豆盧瑑崔沆及于琮等也】獨北司平善况今朝臣至者皆冒死崎嶇遠奉君親所宜自茲同休等戚伏見前夕黄頭軍作亂陛下獨與令孜敬瑄及諸内臣閉城登樓並不召王鐸已下及收朝臣入城翌日又不對宰相又不宣慰朝臣【翌日明日也】臣備位諫官至今未知聖躬安否况踈宂乎【宂而隴翻散也】儻羣臣不顧君上罪固當誅若陛下不恤羣臣於義安在夫天下者高祖太宗之天下非北司之天下天子者四海九州之天子非北司之天子北司未必盡可信南司未必盡無用豈天子與宰相了無關涉朝臣皆若路人如此恐收復之期尚勞聖慮尸禄之士得以宴安臣躬被寵榮【被皮義翻】職在禆益雖遂事不諫而來者可追【二語皆本之論語】疏入令孜屏不奏【屏必郢翻】辛未矯詔貶昭圖嘉州司戶遣人沈於蟇頤津【蟇頤山在眉州眉山東七里山狀如蟇頤因名山臨江津今有孟拾遺祠蟇謨加翻】聞者氣塞而莫敢言【天子殺諫臣者必亡其國以閹官而專殺諫臣自古以來未之有也此不特害于而國實亦凶于而身是以唐未亡而令孜之身先亡也塞悉則翻】 鄜延節度使李孝昌權夏州節度使拓跋思恭屯東渭橋黄巢遣朱温拒之以義武節度使王處存為東南面行營招討使以邠寧節度副使朱玫為節度使 八月己丑夜星交流如織或大如杯椀至丁酉乃止 武寧節度使支詳【按新書方鎮表懿宗咸通十一年復徐州節鎮賜號感化軍自此迄於天復末嘗復武寧舊額以下文以感化留後時溥為節度使證之武寧誤也當作感化】遣牙將時溥陳璠將兵五千入關討黄巢【璠孚袁翻】二人皆詳所奬拔也溥至東都矯稱詳命召師還與璠合兵屠河隂掠鄭州而東及彭城詳迎勞犒賞甚厚【璠音番勞力到翻】溥遣所親說詳曰【說式芮翻】衆心見迫請公解印以相授詳不能制出居大彭舘溥自知留務璠謂溥曰支僕射有惠於徐人不殺必成後悔溥不許送詳歸朝【朝直遥翻】璠伏甲於七里亭【亭去彭城七里因名】并其家屬殺之詔以溥為武寧留後溥表璠為宿州刺史璠到官貪虐溥以都將張友代還殺之 楊復光奏升蔡州為奉國軍以秦宗權為防禦使壽州屠者王緒與妹夫劉行全聚衆五百盜據本州月餘復陷光州【復扶又翻下同】自稱將軍有衆萬餘人秦宗權表為光州刺史固始縣佐王潮【世率以縣丞為縣佐唐制諸縣丞簿尉之下有司功佐司倉佐司戶佐司兵佐司法佐司士佐皆縣佐也路振九國志王潮少為縣佐史或者傳寫逸史字歟】及弟審邽審知皆以材氣知名緒以潮為軍正使典資糧閲士卒信用之【王潮兄弟始此為潮廢緒張本】 高潯與黄巢將李詳戰于石橋【石橋即晉將王鎮惡破秦兵處】潯敗奔河中詳乘勝復取華州【是年五月高潯克華州】巢以詳為華州刺史 以權知夏綏節度使拓跋思恭為節度使 宗正少卿嗣曹王龜年自南詔還驃信上表欵附請悉遵詔旨【龜年使南詔見上卷廣明元年六月上時掌翻】 李孝昌拓跋思恭與尚讓朱温戰于東渭橋不利引去【史言諸鎮之勤王者皆以師老遷延引退】 初高駢與鎮海節度使周寶俱出神策軍駢以兄事寶及駢先貴有功浸輕之既而封壤相鄰數爭細故遂有隙【淮南與鎮海軍鄰壤止一江為界耳數所角翻】駢檄寶入援京師寶治舟師以俟之【治直之翻】怪其久不行訪諸幕客或曰高公幸朝廷多故有并吞江東之志聲云入援其實未必非圖我也宜為備寶未之信使人覘駢殊無北上意【覘丑亷翻上時掌翻自淮南而北向勤王為北上】會駢使人約寶面會瓜洲議軍事寶遂以言者為然辭疾不往且謂使者曰吾非李康高公復欲作家門功勲以欺朝廷邪【高崇文斬李康事見二百三十七卷憲宗之元和元年】駢怒復遣使責寶何敢輕侮大臣【復扶又翻】寶詬之曰彼此夾江為節度使汝為大臣我豈坊門卒邪【詬古侯翻又許侯翻寶自言與駢等夷非有貴賤之異也長安城中百坊坊皆有垣有門門皆有守卒】由是遂為深仇駢留東塘百餘日詔屢趣之【趣讀曰促】駢上表託以寶及浙東觀察使劉漢宏將為後患辛亥復罷兵還府其實無赴難心但欲禳雉集之異耳【難乃旦翻禳如羊翻厭除也】 高駢召石鏡鎮將董昌至廣陵欲與之俱擊黄巢昌將錢鏐說昌曰【說式芮翻】觀高公無討賊心不若以扞禦鄉里為辭而去之昌從之駢聽昌還會杭州刺史路審中將之官行至嘉興【嘉興漢由拳縣吳改名唐屬蘇州在州西南百四十里】昌自石鏡引兵入杭州審中懼而還昌自稱杭州都押牙知州事遣將吏請於周寶寶不能制表為杭州刺史 臨海賊杜雄陷台州 辛酉立皇子震為建王 昭義十將成麟殺高潯【成麟因高潯石橋敗退而殺之】引兵還據潞州天井關戍將孟方立起兵攻麟殺之 【考異曰實録澤潞牙將劉廣據潞州叛天井關戍將孟方立帥戍卒攻廣殺之自稱留後仍移軍額於邢州初高潯援京師廣帥師至陽平謀為亂不行還據潞州自稱留後用法嚴酷三軍畏之方立乘虛襲殺焉又曰貶昭義節度使高潯為端州刺史中和三年實録又曰初孟方立殺高潯自立薛居正五代史方立傳曰中和二年為澤州天井關戍將時黄巢犯關輔州郡易帥有同博奕先是沈詢高湜相繼為昭義節度怠於軍政及有歸秦劉廣之亂方立見潞帥交代之際乘其無備率戍兵徑入潞州自稱留後新紀八月昭義軍節度使高潯及黄巢戰於石橋敗績十將成麟殺潯入于潞州九月己巳昭義軍戍將孟方立殺成麟自稱留後方立傳惟以成麟為成鄰餘如新紀按乾符二年實録十月昭義軍亂逐節度使高湜貶湜象州司戶柳玭傳云貶高要尉三年十一月詔魏博韓簡云劉廣逐帥擅權云云是廣逐湜擅據潞州也薛史孟方立傳亦云沈詢高湜怠於軍政致有歸秦劉廣之亂是廣亂在前也舊紀九月高潯牙將劉廣擅還潞州是月潯天井關戍將孟方立攻廣殺之自稱留後貶潯端州刺史此蓋舊紀誤實録因之薛史方立傳曰見潞帥交代之際帥兵入潞州不言何帥交代若不逐帥何能據州事無所因殊為踈畧舊紀恐是誤以高湜事為高潯事實録此云殺廣明年又云殺潯自相違新紀傳皆云成麟殺潯方立斬麟月日事實頗詳必有所出今從之】方立汧州人也 忠武監軍楊復光屯武功 永嘉賊朱褒陷温州【宋白曰温州永嘉郡漢會稽郡之東境後漢永和四年置永寧縣晉明帝立永寧郡尋屬永嘉郡隋平陳廢郡唐武悳六年置東嘉州貞觀元年廢州以縣屬栝州上元二年分栝州之永嘉安固二縣置温州以温嶠嶺為名】鳳翔行軍司馬李昌言將本軍屯興平時鳳翔倉庫虛竭犒賞稍薄糧饋不繼昌言知府中兵少因激怒其衆冬十月引軍還襲府城鄭畋登城與士卒言其衆皆下馬羅拜曰相公誠無負我曹畋曰行軍苟能戢兵愛人為國滅賊亦可以順守矣【逐帥為逆取討賊以取旌節為順守為于偽翻】乃以留務委之即日西赴行在 天平節度使南面招討使曹全晸與賊戰死軍中立其兄子存實為留後 十一月乙巳孟楷朱温襲鄜夏二軍於富平二軍敗奔歸本道【二軍李孝昌拓跋思恭之軍也】 鄭畋至鳳州【自鳳翔西至鳳州三百九十五里】累表辭位詔以畋為太子少傅分司以李昌言為鳳翔節度行營招討使 以門下侍郎同平章事裴澈為鄂岳觀察使 加鎮海節度使周寶同平章事遂昌賊盧約陷處州【吳孫權赤烏二年分太末立平昌縣晉武帝改曰遂昌唐武德八年併入松陽景雲元年復置遂昌縣屬處州九域志在州西二百四十里按温處二州本晉永嘉一郡之地隋為栝州永嘉郡唐武德置栝州又分置東嘉州始分為二州東嘉州後為温州栝州改為處州避德宗名也】十二月江西將閔朂戍湖南還過潭州逐觀察使李裕自為留後 【考異曰實録新傳作閔頊今從程匡袤唐補紀】 以感化留後時溥為節度使 賜夏州號定難軍【難乃旦翻】 初高駢鎮荆南【乾符五年駢鎮荆南】補武陵蠻雷滿為牙將領蠻軍從駢至淮南逃歸聚衆千人襲朗州殺刺史崔翥【翥章恕翻】詔以滿為朗州留後歲中率三四引兵寇荆南入其郛焚掠而去大為荆人之患陬溪人周岳嘗與滿獵爭肉而鬭欲殺滿不果【陬溪當在武陵界陬側鳩翻】聞滿據朗州亦聚衆襲衡州逐刺史徐顥詔以岳為衡州刺史石門蠻向瓌亦集夷獠數千攻陷澧州殺刺史呂自牧自稱刺史【吳分零陽縣置天門郡隋廢為石門縣唐屬澧州九域志在州西九十二里】 王鐸以高駢為諸道都統無心討賊自以身為首相憤請行懇欵流涕至於再三上許之【懇欵懇誠也相息亮翻】<br />
<br />
  二年春正月辛亥以王鐸兼中書令充諸道行營都都統 【考異曰舊紀中和元年七月鐸為都統十二月帥師二萬至京畿屯於盩厔舊鐸傳亦在元年唐年補録元年十一月乙巳制以鐸為都統十二月乙亥鐸屯盩厔續寶運録元年八月鐸拜天下都統唐補紀中和元年四月高駢帥師駐泊東塘自五月出府九月却歸朝廷即以鐸統諸道兵馬收復長安鐸為都統諸書年月不同如此新紀二年正月辛亥王鐸為諸道行營都都統高駢罷都統據寶録四月答高駢詔罷都都統當在此年今從實録新紀舊駢傳云僖宗知駢無赴難意乃以鐸為京城四面諸道行營兵馬都統韋昭度領江淮鹽鐵轉運使駢既失兵柄又落利權攘袂大詬累上章自訴語詞不遜按駢罷都統依前為諸道鹽鐵轉運使五月方罷北夢瑣言曰王鐸初鎮荆南黄巢入寇望風而遁他日將兵潼關黄巢令人傳語云相公儒生且非我敵無汚我鋒刃自取敗亡也後到成都行朝拜諸道都統所以高駢上表目之為敗軍之將也按鐸自荆南喪師貶官未嘗將兵潼關皮光業見聞録為都統在此年二月亦誤又舊紀傳新傳鐸正為都都統新紀作都統實録初除及罷時皆為都統中間多云都都統又西門思恭為都都監按此時諸將為都統者甚多疑鐸為都都統是也】權知義成節度使俟罷兵復還政府高駢但領鹽鐵轉運使罷其都統及諸使聽王鐸自辟將佐以太子少師崔安潜為副都統辛未以周岌王重榮為都都統左右司馬諸葛爽及宣武節度使康實為左右先鋒使時溥為催遣綱運租賦防遏使【綱運自江淮來者皆由徐州巡内故以溥任此職】以右神策觀軍容使西門思恭為諸道行營都都監又以王處存李孝昌拓跋思恭為京城東北西面都統以楊復光為南面行營都監使又以中書舍人鄭昌圖為義成節度行軍司馬給事中鄭畯為判官直弘文舘王摶為推官【畯祖峻翻摶徒官翻】司勲員外郎裴贄為掌書記昌圖從讜之從祖兄弟畯畋之弟摶璵之曾孫【王璵以祠禱歷事玄肅見前紀】贄坦之子也【裴坦見二百五十一卷懿宗咸通十年】又以陜虢觀察使王重盈為東面都供軍使重盈重榮之兄也【陜失冉翻重直龍翻】 黄巢以朱温為同州刺史令温自取之二月同州刺史米誠奔河中温遂據之【為朱温以同州歸國張本】 己卯以太子少傅分司鄭畋為司空兼門下侍郎同平章事召詣行在軍務一以諮之以王鐸判戶部事 朱温寇河中王重榮擊敗之【敗補邁翻】以李昌言為京城西面都統朱玫為河南都統【朱玫時鎮邠寧安得出關東統河南諸鎮此河南蓋自龍門河東至蒲津一帶大河南岸也】 涇原節度使胡公素薨軍中請命於都統王鐸承制以大將張鈞為留後 李克用寇蔚州【蔚紆勿翻】三月振武節度使契苾璋奏與天德大同共討克用詔鄭從讜與相知應接 陳敬瑄多遣人歷縣鎮詗事【詗翾正翻又火迥翻】謂之尋事人所至多所求取有二人過資陽鎮【時蓋置鎮於資州資陽縣後魏分資中置資陽縣以其地在資水之陽也九域志資陽在資州西北一百二十里】獨無所求鎮將謝弘讓邀之不至自疑有罪夜亡入羣盜中明旦二人去弘讓實無罪也捕盜使楊遷誘弘讓出首而執以送使【首式又翻下同送使送之節度使府也使疏吏翻】云討擊擒獲以求功敬瑄不之問杖弘讓脊二十釘於西城二七日煎油潑之又以膠麻掣其瘡【釘丁定翻二七十四日也潑普活翻掣尺列翻】備極慘酷見者寃之又有卭州牙官阡能【卭渠容翻考異曰張錦里耆舊傳作千能句延慶錦里耆舊傳作忏能續寶運録作玕能實録新傳作阡能按北夢瑣言安仁土豪阡能注云姓繤無此蓋西南夷之種今從之】因公事違期避杖亡命為盜楊遷復誘之【復扶又翻】能方出首聞弘讓之寃大罵楊遷憤為盜驅掠良民不從者舉家殺之踰月衆至萬人立部伍署職級【職級謂牙前將吏自押牙孔目官而下分職各有等級】横行卭雅二州間攻陷城邑所過塗地先是蜀中少盜賊【先悉薦翻少詩沼翻】自是紛紛競起州縣不能制敬瑄遣牙將楊行遷將三千人胡洪畧莫匡時各將二千人以討之 以右神策將軍齊克儉為左右神策内外八鎮兼博野奉天節度使賜鄜坊軍號保大【鄜音夫】 夏四月甲午加陳敬瑄兼<br />
<br />
  侍中 赫連鐸李可舉與李克用戰不利 初高駢好神仙【好呼到翻】有方士呂用之坐妖黨亡命歸駢駢厚待之補以軍職【妖於遥翻】用之鄱陽茶商之子也【鄱陽漢古縣唐帶饒州古縣在今縣東界】久客廣陵熟其人情爐鼎之暇【爐鼎所以鍊金石化丹砂為金銀之類】頗言公私利病故駢愈奇之稍加信任駢舊將梁纘陳珙馮綬董瑾俞公楚姚歸禮素為駢所厚用之欲專權浸以計去之駢遂奪纘兵族珙家綬瑾公楚歸禮咸見踈用之又引其黨張守一諸葛殷共蠱惑駢守一本滄景村民【去羌呂翻張守一蓋居滄景二州間】以術干駢無所遇窮困甚用之謂曰但與吾同心勿憂不富貴遂薦於駢駢寵待埒於用之【埒龍輟翻等也】殷始自鄱陽來用之先言於駢曰玉皇以公職事繁重輟左右尊神一人佐公為理公善遇之欲其久留亦可縻以人間重職明日殷謁見詭辯風生駢以為神補鹽鐵劇職駢嚴潔甥姪輩未嘗得接坐殷病風疽【史炤曰疽千余切又子與切痒病一本從疒從旦音多但翻又音旦釋云瘡也】搔捫不替手膿血滿爪駢獨與之同席促膝傳杯器而食左右以為言駢曰神仙以此試人耳駢有畜犬【搔爬也捫摸也替廢也畜吁玉翻】聞其腥穢多來近之【近其靳翻】駢怪之殷笑曰殷嘗於玉皇前見之【道家謂天帝為玉皇大帝】别來數百年猶相識駢與鄭畋有隙用之謂駢曰宰相有遣劒客來刺公者【刂七亦翻】今夕至矣駢大懼問計安出用之曰張先生嘗學斯術可以禦之駢請於守一守一許諾乃使駢衣婦人之服【衣於既翻】濳於他室而守一代居駢寢榻中夜擲銅器於階令鏗然有聲又密以囊盛彘血【鏗丘耕翻盛時征翻彘豕也】灑於庭宇如格鬭之狀及旦笑謂駢曰幾落奴手【幾居希翻】駢泣謝曰先生於駢乃更生之惠也厚酬以金寶有蕭勝者賂用之求鹽城監【鹽城漢鹽瀆縣地久無城邑唐武德七年置鹽城縣有監亭一百二十三有監屬楚州九域志縣在州東南二百四十里】駢有難色用之曰用之非為勝也【為于偽翻】近得上仙書云有寶劒在鹽城井中須一靈官往取之以勝上仙左右之人欲使取劒耳駢乃許之勝至監數月函一銅首以獻用之見稽首曰【稽音啓】此北帝所佩得之則百里之内五兵不能犯駢乃飾以珠玉常置坐隅【坐徂臥翻】用之自謂磻溪真君謂守一乃赤松子殷乃葛將軍勝乃秦穆公之壻也【各以其姓傅會以為仙磻蒲官翻】用之又刻青石為奇字云玉皇授白雲先生高駢密令左右置道院香按駢得之驚喜用之曰玉皇以公焚修功著將補真官計鸞鶴不日當降此際用之等謫限亦滿必得陪幢節同歸上清耳【用之自言與守一殷等本皆神仙以謫降在人間限期既滿當復升天列於仙官又道家之說有太清玉清上清是為三清之境幢簿江翻】是後駢於道院庭中刻木鶴時著羽服跨之【著陟畧翻】日夕齋醮鍊金燒丹費以巨萬計用之微時依止江陽后土廟【貞觀十八年分江都置江陽縣與江都俱在揚州郭下后土廟今揚州城東南隅蕃釐觀是也然揚州古城在蜀岡之上北連雷塘今城周世宗所徙則此時后土廟在揚州城外也宋白曰宋武帝分江都縣置廣陵縣隋初改為江陽縣以處江之正北故曰江陽】舉動祈禱及得志白駢崇大其廟極江南工材之選每軍旅大事以少牢禱之【少詩照翻】用之又言神仙好樓居【用漢方士語好呼到翻】說駢作迎仙樓【說式芮翻】費十五萬緡又作延和閣高八丈【高居傲翻】用之每對駢呵叱風雨仰揖空際云有神仙過雲表駢輒隨而拜之然常厚賂駢左右使伺駢動静共為欺罔駢不之寤【伺相吏翻】左右小有異議者輒為用之陷死不旋踵但濳撫膺鳴指【鳴指即彈指也】口不敢言駢倚用之如左右手公私大小之事皆決於用之退賢進不肖淫刑濫賞駢之政事於是大壞矣用之知上下怨憤恐有竊請置巡察使駢即以用之領之募險獪者百餘人縱横閭巷間【獪古外翻縱子容翻】謂之察子民間呵妻詈子靡不知之用之欲奪人貨財掠人婦女輒誣以叛逆搒掠取服【搒音彭掠音亮】殺其人而取之所破滅者數百家道路以目將吏士民雖家居皆重足屏氣【重直龍翻屏必郢翻】用之又欲以兵威脅制諸將請選募諸軍驍勇之士二萬人號左右莫邪都【邪讀曰耶】駢即以張守一及用之為左右莫邪軍使署置將使如帥府【帥所類翻】器械精利衣装華潔每出入導從近千人【從才用翻近其靳翻】用之侍妾百餘人自奉奢靡用度不足輒留三司綱輸其家【三司綱謂戶部度支鹽鐵所綱運輸朝廷者】用之猶慮人泄其姧謀乃言於駢曰神仙不難致但恨學者不能絶俗累【累良瑞翻】故不肯降臨耳駢乃悉去姬妾【去羌呂翻】謝絶人事賓客將吏皆不得見有不得已見之者皆先令沐浴齋祓【祓敷勿翻又方廢翻祓除穢惡也】然後見拜起纔畢已復引出由是用之得專行威福無所忌憚境内不復知有駢矣【為畢師鐸討用之殺駢張本復扶又翻】 王鐸將兩川興元之軍屯靈感寺涇原屯京西易定河中屯渭北邠寧鳳翔屯興平保大定難屯渭橋【難乃旦翻】忠武屯武功官軍四集黄巢勢已蹙號令所行不出同華【黄巢將朱温時據同州李詳據華州故號令之行止此二州華戶化翻】民避亂皆入深山築柵自保農事俱廢長安城中斗米直三十緡賊賣人於官軍以為糧官軍或執山寨之民鬻之人直數百緡以肥瘠論價<br />
<br />
  資治通鑑卷二百五十四  <br>
   </div> 

<script src="/search/ajaxskft.js"> </script>
 <div class="clear"></div>
<br>
<br>
 <!-- a.d-->

 <!--
<div class="info_share">
</div> 
-->
 <!--info_share--></div>   <!-- end info_content-->
  </div> <!-- end l-->

<div class="r">   <!--r-->



<div class="sidebar"  style="margin-bottom:2px;">

 
<div class="sidebar_title">工具类大全</div>
<div class="sidebar_info">
<strong><a href="http://www.guoxuedashi.com/lsditu/" target="_blank">历史地图</a></strong>  
<a href="http://www.880114.com/" target="_blank">英语宝典</a>  
<a href="http://www.guoxuedashi.com/13jing/" target="_blank">十三经检索</a> 
<br><strong><a href="http://www.guoxuedashi.com/gjtsjc/" target="_blank">古今图书集成</a></strong> 
<a href="http://www.guoxuedashi.com/duilian/" target="_blank">对联大全</a> <strong><a href="http://www.guoxuedashi.com/xiangxingzi/" target="_blank">象形文字典</a></strong> 

<br><a href="http://www.guoxuedashi.com/zixing/yanbian/">字形演变</a>  <strong><a href="http://www.guoxuemi.com/hafo/" target="_blank">哈佛燕京中文善本特藏</a></strong>
<br><strong><a href="http://www.guoxuedashi.com/csfz/" target="_blank">丛书&方志检索器</a></strong> <a href="http://www.guoxuedashi.com/yqjyy/" target="_blank">一切经音义</a>  

<br><strong><a href="http://www.guoxuedashi.com/jiapu/" target="_blank">家谱族谱查询</a></strong>  <strong><a href="http://shufa.guoxuedashi.com/sfzitie/" target="_blank">书法字帖欣赏</a></strong> 
<br>

</div>
</div>


<div class="sidebar" style="margin-bottom:0px;">

<font style="font-size:22px;line-height:32px">QQ交流群9:489193090</font>


<div class="sidebar_title">手机APP 扫描或点击</div>
<div class="sidebar_info">
<table>
<tr>
	<td width=160><a href="http://m.guoxuedashi.com/app/" target="_blank"><img src="/img/gxds-sj.png" width="140"  border="0" alt="国学大师手机版"></a></td>
	<td>
<a href="http://www.guoxuedashi.com/download/" target="_blank">app软件下载专区</a><br>
<a href="http://www.guoxuedashi.com/download/gxds.php" target="_blank">《国学大师》下载</a><br>
<a href="http://www.guoxuedashi.com/download/kxzd.php" target="_blank">《汉字宝典》下载</a><br>
<a href="http://www.guoxuedashi.com/download/scqbd.php" target="_blank">《诗词曲宝典》下载</a><br>
<a href="http://www.guoxuedashi.com/SiKuQuanShu/skqs.php" target="_blank">《四库全书》下载</a><br>
</td>
</tr>
</table>

</div>
</div>


<div class="sidebar2">
<center>


</center>
</div>

<div class="sidebar"  style="margin-bottom:2px;">
<div class="sidebar_title">网站使用教程</div>
<div class="sidebar_info">
<a href="http://www.guoxuedashi.com/help/gjsearch.php" target="_blank">如何在国学大师网下载古籍?</a><br>
<a href="http://www.guoxuedashi.com/zidian/bujian/bjjc.php" target="_blank">如何使用部件查字法快速查字?</a><br>
<a href="http://www.guoxuedashi.com/search/sjc.php" target="_blank">如何在指定的书籍中全文检索?</a><br>
<a href="http://www.guoxuedashi.com/search/skjc.php" target="_blank">如何找到一句话在《四库全书》哪一页?</a><br>
</div>
</div>


<div class="sidebar">
<div class="sidebar_title">热门书籍</div>
<div class="sidebar_info">
<a href="/so.php?sokey=%E8%B5%84%E6%B2%BB%E9%80%9A%E9%89%B4&kt=1">资治通鉴</a> <a href="/24shi/"><strong>二十四史</strong></a>&nbsp; <a href="/a2694/">野史</a>&nbsp; <a href="/SiKuQuanShu/"><strong>四库全书</strong></a>&nbsp;<a href="http://www.guoxuedashi.com/SiKuQuanShu/fanti/">繁体</a>
<br><a href="/so.php?sokey=%E7%BA%A2%E6%A5%BC%E6%A2%A6&kt=1">红楼梦</a> <a href="/a/1858x/">三国演义</a> <a href="/a/1038k/">水浒传</a> <a href="/a/1046t/">西游记</a> <a href="/a/1914o/">封神演义</a>
<br>
<a href="http://www.guoxuedashi.com/so.php?sokeygx=%E4%B8%87%E6%9C%89%E6%96%87%E5%BA%93&submit=&kt=1">万有文库</a> <a href="/a/780t/">古文观止</a> <a href="/a/1024l/">文心雕龙</a> <a href="/a/1704n/">全唐诗</a> <a href="/a/1705h/">全宋词</a>
<br><a href="http://www.guoxuedashi.com/so.php?sokeygx=%E7%99%BE%E8%A1%B2%E6%9C%AC%E4%BA%8C%E5%8D%81%E5%9B%9B%E5%8F%B2&submit=&kt=1"><strong>百衲本二十四史</strong></a>  <a href="http://www.guoxuedashi.com/so.php?sokeygx=%E5%8F%A4%E4%BB%8A%E5%9B%BE%E4%B9%A6%E9%9B%86%E6%88%90&submit=&kt=1"><strong>古今图书集成</strong></a>
<br>

<a href="http://www.guoxuedashi.com/so.php?sokeygx=%E4%B8%9B%E4%B9%A6%E9%9B%86%E6%88%90&submit=&kt=1">丛书集成</a> 
<a href="http://www.guoxuedashi.com/so.php?sokeygx=%E5%9B%9B%E9%83%A8%E4%B8%9B%E5%88%8A&submit=&kt=1"><strong>四部丛刊</strong></a>  
<a href="http://www.guoxuedashi.com/so.php?sokeygx=%E8%AF%B4%E6%96%87%E8%A7%A3%E5%AD%97&submit=&kt=1">說文解字</a> <a href="http://www.guoxuedashi.com/so.php?sokeygx=%E5%85%A8%E4%B8%8A%E5%8F%A4&submit=&kt=1">三国六朝文</a>
<br><a href="http://www.guoxuedashi.com/so.php?sokeytm=%E6%97%A5%E6%9C%AC%E5%86%85%E9%98%81%E6%96%87%E5%BA%93&submit=&kt=1"><strong>日本内阁文库</strong></a> <a href="http://www.guoxuedashi.com/so.php?sokeytm=%E5%9B%BD%E5%9B%BE%E6%96%B9%E5%BF%97%E5%90%88%E9%9B%86&ka=100&submit=">国图方志合集</a> <a href="http://www.guoxuedashi.com/so.php?sokeytm=%E5%90%84%E5%9C%B0%E6%96%B9%E5%BF%97&submit=&kt=1"><strong>各地方志</strong></a>

</div>
</div>


<div class="sidebar2">
<center>

</center>
</div>
<div class="sidebar greenbar">
<div class="sidebar_title green">四库全书</div>
<div class="sidebar_info">

《四库全书》是中国古代最大的丛书,编撰于乾隆年间,由纪昀等360多位高官、学者编撰,3800多人抄写,费时十三年编成。丛书分经、史、子、集四部,故名四库。共有3500多种书,7.9万卷,3.6万册,约8亿字,基本上囊括了古代所有图书,故称“全书”。<a href="http://www.guoxuedashi.com/SiKuQuanShu/">详细>>
</a>

</div> 
</div>

</div>  <!--end r-->

</div>
<!-- 内容区END --> 

<!-- 页脚开始 -->
<div class="shh">

</div>

<div class="w1180" style="margin-top:8px;">
<center><script src="http://www.guoxuedashi.com/img/plus.php?id=3"></script></center>
</div>
<div class="w1180 foot">
<a href="/b/thanks.php">特别致谢</a> | <a href="javascript:window.external.AddFavorite(document.location.href,document.title);">收藏本站</a> | <a href="#">欢迎投稿</a> | <a href="http://www.guoxuedashi.com/forum/">意见建议</a> | <a href="http://www.guoxuemi.com/">国学迷</a> | <a href="http://www.shuowen.net/">说文网</a><script language="javascript" type="text/javascript" src="https://js.users.51.la/17753172.js"></script><br />
  Copyright &copy; 国学大师 古典图书集成 All Rights Reserved.<br>
  
  <span style="font-size:14px">免责声明:本站非营利性站点,以方便网友为主,仅供学习研究。<br>内容由热心网友提供和网上收集,不保留版权。若侵犯了您的权益,来信即刪。scp168@qq.com</span>
  <br />
ICP证:<a href="http://www.beian.miit.gov.cn/" target="_blank">鲁ICP备19060063号</a></div>
<!-- 页脚END --> 
<script src="http://www.guoxuedashi.com/img/plus.php?id=22"></script>
<script src="http://www.guoxuedashi.com/img/tongji.js"></script>

</body>
</html>
