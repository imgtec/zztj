






























































資治通鑑卷五十二   宋 司馬光 撰

胡三省 音註

漢紀四十四|{
	起閼逢閹茂盡旃蒙作噩凡十二年}


孝順皇帝下

陽嘉三年夏四月車師後部司馬率後王加特奴掩擊北匈奴於閶吾陸谷|{
	閶音昌}
大破之獲單于母 五月戊戌詔以春夏連旱赦天下上親自露坐德陽殿東廂請雨|{
	按范書桓帝紀德陽殿在北宫掖庭中}
以尚書周舉才學優深特加策問舉對曰臣聞隂陽閉隔則二氣否塞|{
	否皮鄙翻塞悉則翻}
陛下廢文帝光武之法而循亡秦奢侈之欲内積怨女外有曠夫自枯旱以來彌歷年歲未聞陛下改過之効徒勞至尊暴露風塵誠無益也|{
	謂露坐無益}
陛下但務其華不尋其實猶緣木希魚却行求前|{
	賢曰緣木求魚孟子之文韓詩外傳曰夫明鏡所以照形往古所以知今惡知往古之所以危亡無異却行而求達於前人也}
誠宜推信革政崇道變惑出後宫不御之女除太官重膳之費易傳曰陽感天不旋日|{
	易稽覽圖中孚傳曰陽感天不旋日諸侯不旋時大夫不過朞鄭玄注云陽者天子為善一日天立應以善為惡一日天立應以惡一說不旋時立應之重直龍翻傳直戀翻}
惟陛下留神裁察帝復召舉面問得失舉對以宜慎官人去貪汙遠佞邪|{
	復扶又翻去羌呂翻遠于願翻}
帝曰官貪汙佞邪者為誰乎對曰臣從下州超備機密|{
	舉自冀州刺史徵拜尚書}
不足以别羣臣然公卿大臣數有直言者忠貞也|{
	别彼列翻數所角翻}
阿諛苟容者佞邪也太史令張衡亦上疏言前年京師地震土裂裂者威分震者民擾也竊懼聖思厭倦制不專己恩不忍割與衆共威威不可分德不可共願陛下思惟所以稽古率舊勿使刑德八柄不由天子|{
	周禮王以八柄馭羣臣一曰爵以馭其貴二曰禄以馭其富三曰予以馭其幸四曰置以馭其行五曰生以馭其福六曰奪以馭其貧七曰廢以馭其罪八曰誅以馭其過}
然後神望允塞|{
	塞悉則翻}
災消不至矣衡又以中興之後儒者爭學圖緯|{
	緯七緯也七緯者易緯稽覽圖乾鑿度坤靈圖通卦驗是類謀辯終備也書緯璇璣鈐考靈耀刑德放帝命驗運期授也詩緯推度災記歷樞含神霧也禮緯含文嘉稽命徵斗威儀也樂緯動聲儀稽耀嘉什國徵也孝經緯援神契鉤命决也春秋緯演孔圖元命包文耀鉤運斗樞感精符合誠圖考異郵保乾圖漢含孳佑助期握誠圖濳潭巴說題辭也}
上疏言春秋元命包有公輸班與墨翟事見戰國又言别有益州益州之置在於漢世|{
	賢曰前書武帝始置益州}
又劉向父子領校祕書閲定九流亦無䜟録|{
	賢曰成哀時劉向及子歆為祕書校定經傳諸子等九流謂儒家道家隂陽家法家名家墨家縱横家雜家農家見蓻文志並無䜟說䜟楚譖翻}
則知圖䜟成於哀平之際皆虚偽之徒以要世取資|{
	要一遙翻}
欺罔較然莫之糾禁且律歷卦侯九宫風角|{
	黄帝命伶倫吹律大橈作甲子容成造歷而律歷之學傳矣京房分六十四卦更直日用事以風雨寒温為候伏羲之時龍馬負圖出於河戴九履一左三右七二四為肩六八為足而五居中伏羲觀河圖而畫八卦隂陽家謂之九宫一六八為白二黑三綠四碧五黄七赤九紫至今承用之又易乾鑿度曰太一取其數而行九宫鄭玄注云太一者北辰神名也下行八卦之宫每四乃還於中央中央者地神之所居故謂之九宫天數大分以陽出以隂入陽起於子隂起於午是以太一下九宫從坎宫始自此而從於坤宫又自此而從於震宫又自此而從於巽宫所以從半矣還息於中央之宫既又自此而從於乾宫又自此而從於兑宫又自此而從於艮宫又自此而從於離宫行則周矣上遊息於太一之星而反於紫宫行起從坎宫始終於離宫也此雖緯書之說而九宫定位則一也賢曰風角謂候四方四隅之風以占吉凶}
數有徵効|{
	數所角翻下同}
世莫肯學而競稱不占之書|{
	賢曰謂競稱䜟家也}
譬猶畫工惡圖犬馬而好作鬼魅誠以實事難形而虚偽不窮也|{
	惡烏路翻好呼到翻魅音媚韓子曰客有為齊王畫者問畫孰難對曰狗馬最難孰易曰鬼魅最易狗馬人所知也故難鬼魅無形故易也}
宜收藏圖䜟一禁絶之則朱紫無所眩典籍無瑕玷矣 秋七月鍾羌良封等復寇隴西漢陽|{
	復扶又翻}
詔拜前校尉馬賢為謁者鎮撫諸種|{
	種章勇翻}
冬十月護羌校尉馬續遣兵擊良封破之十一月壬寅司徒劉崎司空孔扶免用周舉之言也|{
	崎丘宜翻}
乙巳以大司農黄尚為司徒光禄勲河東王卓為司空 耿䝿人數為耿氏請|{
	為于偽翻}
帝乃紹封耿寶子箕為牟平侯|{
	耿寶貶死事見上卷安帝延光四年}


四年春北匈奴呼衍王侵車師後部帝令敦煌太守發兵救之不利|{
	敦徒門翻}
二月丙子初聽中官得以養子襲爵|{
	曹操階之遂移漢祚其所由來者漸矣}
初帝之復位宦官之力也|{
	事見上卷延光四年}
由是有寵參與政事|{
	與讀曰預}
御史張綱上書曰竊尋文明二帝德化尤盛中官常侍不過兩人近倖賞賜裁滿數金惜費重民故家給人足而頃者以來無功小人皆有官爵非愛民重器承天順道者也書奏不省|{
	省悉景翻}
綱皓之子也|{
	張皓見五十卷安帝延光三年}
旱 謁者馬賢擊鍾羌大破之 夏四月甲子太尉施延免 戊寅以執金吾梁商為大將軍故太尉龐參為太尉|{
	龐皮江翻}
商稱疾不起且一年帝使太常桓焉奉策就第即拜商乃詣闕受命|{
	杜祐曰後漢策拜諸王侯三公之儀百官會位定謁者引光禄勲前謁者引當拜者前伏殿下光祿前一拜舉手曰制詔其以某為某讀策書畢拜者稱臣再拜尚書郎以璽印綬付侍御史前面立受印璽綬當受策者再拜頓首三贊謁者曰某王臣某新封某公某初除謝中謁者報謹謝贊者立曰皇帝為公興重坐受策者拜謝起就位禮畢自漢以來惟衛青以有功即軍中拜大將軍未聞有就第即拜者也况以此異數加之后父乎}
商少通經傳謙恭好士|{
	少詩沼翻好呼到翻下同}
辟漢陽巨覽|{
	巨姓覽名}
上黨陳龜為掾屬|{
	掾余絹翻}
李固為從事中郎楊倫為長史李固以商柔和自守不能有所整裁乃奏記於商曰數年以來災怪屢見|{
	見賢遍反}
孔子曰智者見變思形愚者覩怪諱名|{
	范書李固傳形作刑此二語蓋亦本之緯書}
天道無親可為祗畏|{
	賢曰祗敬也言天無親疎惟善是與可敬而畏也}
誠令王綱一整道行忠立明公踵伯成之高|{
	莊子曰伯成子高唐虞時為諸侯至禹去而耕於野}
全不朽之譽豈與此外戚凡輩耽榮好位者同日而論哉商不能用 秋閏八月丁亥朔日有食之 冬十月烏桓寇雲中度遼將軍耿曅追擊不利十一月烏桓圍曅於蘭池城|{
	續漢志雲中郡沙南縣有蘭池城}
發兵數千人救之烏桓乃退 十二月丙寅京師地震

永和元年春正月己巳改元赦天下 冬十月丁亥承福殿火 十一月丙子太尉龎參罷 十二月象林蠻夷反|{
	象林縣屬日南郡晉宋以下為林邑國}
乙巳以前司空王龔為太尉龔疾宦官專權上書極言其狀諸黄門使各誣奏龔罪上命龔亟自實李固奏記於梁商曰王公以堅貞之操横為讒佞所構|{
	横戶孟翻}
衆人聞知莫不歎慄夫三公尊重無詣理訴寃之義|{
	哀帝時丞相王嘉召詣廷尉主簿曰將相不對理陳寃相踵以為故事君侯宜引决}
纎微感槩輒引分决是以舊典不有大罪不至重問|{
	賢曰大臣獄重故曰重問成帝時丞相薛宣御史大夫翟方進有罪上使五二千石雜問音義云大臣獄重故以二千石五人同問之}
王公卒有他變|{
	卒讀曰猝}
則朝廷獲害賢之名羣臣無救護之節矣語曰善人在患饑不及餐|{
	言當速救之也}
斯其時也商即言之於帝事乃得釋 是歲以執金吾梁冀為河南尹冀性嗜酒逸遊自恣居職多縱暴非灋父商所親客雒陽令呂放以告商商以讓冀冀遣人於道刺殺放|{
	刺七亦翻}
而恐商知之乃推疑放之怨仇|{
	推吐雷翻惡自冀出欲嫁之他人故託其辭疑放之怨仇為之}
請以放弟禹為雒陽令使捕之|{
	賢曰安慰放家欲以滅口余謂賢說非也冀請於商以放弟為令謂必急於捕賊而隂使禹滅其兄之宗親賓客以快己忿耳}
盡滅其宗親賓客百餘人 武陵太守上書以蠻夷率服|{
	言相率而來服}
可比漢人增其租賦議者皆以為可尚書令虞詡曰自古聖王不臣異俗先帝舊典貢賦多少所由來久矣|{
	漢興令武陵諸蠻大人歲輸布一匹小口二丈是謂之賨布}
今猥增之必有怨叛計其所得不償所費必有後悔帝不從澧中漊中蠻各爭貢布非舊約|{
	漊即侯翻}
遂殺鄉吏舉種反

二年春武陵蠻二萬人圍充城八千人寇夷道|{
	賢曰充縣屬武陵郡故城在澧州崇義縣東北充音衝夷道屬南郡}
二月廣漢屬國都尉擊破白馬羌|{
	安帝改蜀郡北部都尉為廣漢屬國都尉别領隂平甸氐剛氐三道屬益州}
帝遣武陵太守李進擊叛蠻破平之進乃簡選良吏撫循蠻夷郡境遂安 三月司空王卓薨丁丑以光祿勲郭䖍為司空 |{
	考異曰袁書作乾今從范書}
夏四月丙申京師地震 五月癸丑山陽君宋娥坐構姦誣罔收印綬歸里舍黄龍楊佗孟叔李建張賢史汎王道李元李剛等九侯坐與宋娥更相賂遺|{
	更工衡翻遺于季翻}
求高官增邑並遣就國減租四分之一 |{
	考異曰孫程傳云龍等誣罔曹騰孟賁按梁商傳誣罔騰賁者張逵等非龍等也}
象林蠻區憐等|{
	區烏侯翻今廣中猶有此姓姓譜云今長沙有此姓音豈俱翻}
攻縣寺殺長吏交趾刺史樊演發交趾九真兵萬餘人救之兵士憚遠役秋七月二郡兵反攻其府府雖擊破反者而蠻埶轉盛 冬十月甲申上行幸長安扶風田弱薦同郡灋真博通内外學|{
	東都諸儒以七緯為内學六經為外學}
隱居不仕宜就加衮職|{
	賢曰毛詩曰衮職有闕謂三公也}
帝虚心欲致之前後四徵終不屈友人郭正稱之曰灋真名可得聞身難得而見逃名而名我隨避名而名我追可謂百世之師者矣真雄之子也|{
	灋雄見四十九卷安帝永初四年}
丁卯京師地震太尉王龔以中常侍張昉等專弄國權欲奏誅之宗親有以楊震行事諫之者|{
	楊震事見五十卷安帝延光三年}
龔乃止十二月乙亥上還自長安

三年春二月乙亥京師及金城隴西地震二郡山崩夏閏四月己酉京師地震 五月吳郡丞羊珍反攻郡府太守王衡破斬之 侍御史賈昌與州郡幷力討區憐不尅為所攻圍歲餘兵穀不繼帝召公卿百官及四府掾屬|{
	大將軍府掾屬二十九人太尉府二十四人司徒府三十一人司空府二十九人}
問以方略皆議遣大將發荆揚兖豫四萬人赴之李固駮曰若荆揚無事發之可也今二州盜賊磐結不散|{
	二州謂荆揚也}
武陵南郡蠻夷未輯長沙桂陽數被徵發如復擾動|{
	數所角翻被皮義翻復扶又翻下同}
必更生患其不可一也又兖豫之人卒被徵發|{
	卒讀曰猝}
遠赴萬里無有還期詔書迫促必致叛亡其不可二也南州水土温暑加有瘴氣|{
	瘴之亮翻度嶺而南瘴氣甚重炎熱蒸鬰之所生也中之者輒死}
致死亡者十必四五其不可三也遠涉萬里士卒疲勞比至嶺南|{
	比必寐翻及也}
不復堪鬬其不可四也軍行三十里為程而去日南九千餘里三百日乃到計人禀五升|{
	賢曰古升小故曰五升也禀給也}
用米六十萬斛不計將吏驢馬之食但負甲自致費便若此其不可五也設軍所在死亡必衆既不足禦敵當復更發此為刻割心腹以補四支其不可六也九真日南相去千里發其吏民猶尚不堪何况乃苦四州之卒以赴萬里之艱哉其不可七也前中郎將尹就討益州叛羌益州諺曰虜來尚可尹來殺我後就徵還以兵付刺史張喬喬因其將吏旬月之間破殄寇虜|{
	事見四十九卷安帝元初二年至五十卷五年}
此發將無益之效|{
	將即亮翻}
州郡可任之驗也宜更選有勇略仁惠任將帥者以為刺史太守悉使共住交趾今日南兵單無穀|{
	言孤軍處叛蠻之中又乏糧也}
守既不足戰又不能可一切徙其吏民北依交趾事靜之後乃命歸本還募蠻夷使自相攻轉輸金帛以為其資有能反間致頭首者|{
	間古莧翻頭首謂諸蠻渠帥也}
許以封侯裂土之賞故并州刺史長沙祝良性多勇决又南陽張喬前在益州有破虜之功皆可任用昔太宗就加魏尚為雲中守|{
	魏尚見十四卷文帝十四年就加事未見守式又翻下同}
哀帝即拜龔舍為泰山守|{
	前書龔舍楚人初徵為諫大夫病免復徵為博士又病去頃之哀帝遣使即楚拜舍為泰山太守}
宜即拜良等便道之官四府悉從固議即拜祝良為九真太守張喬為交趾刺史喬至開示慰誘並皆降散良到九真單車入賊中設方略招以威信降者數萬人皆為良築起府寺由是嶺外復平|{
	為于偽翻復扶又翻}
秋八月已未司徒黄尚免九月己酉以光祿勲長沙劉壽為司徒 丙戌令大將軍三公舉剛毅武猛謀謨任將帥者各二人特進卿校尉各一人|{
	校戶教翻}
初尚書令左雄薦冀州刺史周舉為尚書既而雄為司隸校尉舉故冀州刺史馮直任將帥直嘗坐臧受罪|{
	臧古贓字通}
舉以此劾奏雄|{
	劾所舉非其人也劾戶槩翻又戶得翻}
雄曰詔書使我選武猛不使我選清高舉曰詔書使君選武猛不使君選貪汚也雄曰進君適所以自伐也舉曰昔趙宣子任韓厥為司馬厥以軍灋戮宣子僕宣子謂諸大夫曰可賀我矣吾選厥也任其事|{
	秦晉戰于河曲趙宣子將中軍韓厥為司馬宣子使以其乘車于行韓厥戮其僕衆曰韓厥必不沒矣其主朝升之而暮戮其車宣子謂諸大夫曰可賀我矣吾舉厥也任其事吾今乃知免於戾矣任音壬}
今君不以舉之不才誤升諸朝不敢阿君以為君羞不寤君之意與宣子殊也雄悦謝曰吾嘗事馮直之父又與直善今宣光以此奏吾是吾之過也|{
	周舉字宣光}
天下益以此賢之|{
	聞過而服天下以此益賢左雄諱過者為何如邪}
是時宦官競賣恩勢|{
	挟勢市恩以此自鬻也}
唯大長秋良賀清儉退厚|{
	春秋鄭穆公子子良後為良氏賢曰謙退而厚重也余謂退厚者不與儕輩爭進趣競浮薄也}
及詔舉武猛賀獨無所薦帝問其故對曰臣生自草茅長於宫掖|{
	長知兩翻}
既無知人之明又未嘗交加士類昔衛鞅因景監以見有識知其不終|{
	事見二卷周顯王三十一年}
今得臣舉者匪榮伊辱|{
	言不足為榮適以為辱也 考異曰宦者傳云陽嘉中詔舉武猛良賀獨無所薦按此詔蓋誤以永和為陽嘉也}
是以不敢帝由是賞之 冬十月燒當羌每離等三千餘騎寇金城校尉馬賢擊破之 十二月戊戌朔日有食之 大將軍商以小黄門南陽曹節等用事於中遣子冀不疑與為交友而宦官忌其寵反欲陷之中常侍張逵蘧政楊定等|{
	蘧姓也衛有大夫蘧伯玉}
與左右連謀共譖商及中常侍曹騰孟賁云欲徵諸王子圖議廢立請收商等案罪帝曰大將軍父子我所親騰賁我所愛必無是但汝曹共妒之耳|{
	妒與妬同}
逵等知言不用懼迫|{
	言既不用懼禍且及也}
遂出矯詔收縛騰賁於省中帝聞震怒敇宦者李歙急呼騰賁釋之收逵等下獄|{
	歙許及翻下遐稼翻}
四年春正月庚辰逵等伏誅事連弘農太守張鳳安平相楊皓皆坐死辭所連染延及在位大臣商懼多侵枉乃上疏曰春秋之義功在元帥罪止首惡|{
	春秋左氏傳晉卻克帥師敗齊師于鞌師歸范文子後入曰師有功國人喜以逆之先入必屬耳目焉是代帥受名也故不敢虞師晉師滅下陽公羊傳曰虞微國也曷為序于大國之上使虞首惡也帥所類翻}
大獄一起無辜者衆死囚久繫纎微成大|{
	賢曰言久繫之則細微之事牽引以成大也}
非所以順迎和氣平政成化也宜早訖竟以止逮捕之煩|{
	謂孟春之月當行慶施惠順天地生物之心以迎和氣不宜使獄事枝蔓賢曰逮及也辭所連及即追捕之也}
帝納之罪止坐者二月帝以商少子虎賁中郎將不疑為步兵校尉商上書辭曰不疑童孺猥處成人之位|{
	處昌呂翻}
昔晏平仲辭鄁殿以守其富|{
	左傳齊討慶封與晏子鄁殿其鄙六十弗受子尾曰富人之所欲也何故弗受對曰慶氏之邑足欲故亡吾邑不足欲也益之以鄁殿乃足欲亡無日矣不受鄁殿非惡富也恐失富也鄁蒲對翻殿多薦翻又如字}
公儀休不受魚飱以定其位|{
	公儀休為魯相客有遺相魚者相不受客曰聞君嗜魚遺君魚何故不受也相曰以嗜魚故不受也今為相能自給魚受魚而免誰復給我魚者故不受也}
臣雖不才亦願固福祿於聖世上乃以不疑為侍中奉車都尉|{
	梁商之讓通經傳之力也}
三月乙亥京師地震燒當羌每離等復反|{
	復扶又翻下同}
夏四月癸卯護羌校尉馬賢討斬之獲首虜千二百餘級 戊午赦天下 五月戊辰封故濟北惠王壽子安為濟北王|{
	去年濟北王多薨無子今以安紹封范書列傳作安國此從帝紀濟子禮翻}
秋八月太原旱五年春二月戊申京師地震 南匈奴句龍王吾斯車紐等反|{
	句古侯翻車尺遮翻}
寇西河招誘右賢王合兵圍美稷殺朔方代郡長吏夏五月度遼將軍馬續與中郎將梁並等|{
	此護匈奴中郎將也}
發邊兵及羌胡合二萬餘人掩擊破之吾斯等復更屯聚攻沒城邑天子遣使責讓單于單于本不預謀乃脱帽避帳詣並謝罪並以病徵五原太守陳龜代為中郎將龜以單于不能制下|{
	賢曰吾斯等攻沒城邑單于雖不預謀然不能制下即是不堪其任}
逼迫單于及其弟左賢王皆令自殺龜又欲徙單于近親於内郡而降者遂更狐疑龜坐下獄免|{
	降戶江翻下同龜所施行必冇未究其方畧者而遽坐免也下遐稼翻}
大將軍商上表曰匈奴寇畔自知罪極窮鳥困獸皆知救死|{
	鳥窮則攫獸困則搏傳曰困獸猶鬭}
況種類繁熾不可單盡|{
	賢曰單亦盡也種章勇翻}
今轉運日增三軍疲苦虛内給外非中國之利度遼將軍馬續素有謀謨且典邊日久深曉兵要每得續書與臣策合宜令續深溝高壘以恩信招降宣示購賞明為期約如此則醜類可服|{
	賢曰醜等也余謂醜類言凶醜之黨類也}
國家無事矣帝從之乃詔續招降畔虜商又移書續等曰中國安寧忘戰日久良騎夜合交鋒接矢决勝當時戎狄之所長而中國之所短也彊弩乘城堅營固守以待其衰中國之所長而戎狄之所短也宜務先所長而觀其變|{
	先悉薦翻}
設購開賞宣示反悔|{
	反音幡宣示招降之意以開其反悔之心}
勿貪小功以亂大謀於是右賢王部抑鞮等萬三千口皆詣續降|{
	鞮丁奚翻}
己丑晦日有食之 初每離等既平朝廷以來機為并州刺史劉秉為凉州刺史機等天性虐刻多所擾發且凍傅難種羌遂反|{
	賢曰且音子余翻種章勇翻下同}
攻金城與雜種羌胡大寇三輔殺害長吏機等並坐徵於是拜馬賢為征西將軍以騎都尉耿叔為副將左右羽林五校士及諸州郡兵十萬人屯漢陽 九月令扶風漢陽築隴道塢三百所置屯兵 辛未太尉王龔以老病罷 且凍羌寇武都燒隴關|{
	賢曰隴山之關也今名大震關在今隴州汧源縣西}
壬午以太常桓焉為太尉 匈奴句龍王吾斯等立車紐為單于東引烏桓西收羌胡等數萬人攻破京兆虎牙營殺上郡都尉及軍司馬遂寇掠并涼幽冀四州乃徙西河治離石|{
	賢曰離石即西河之屬縣也在郡南五百九里郡本都平定縣至此徙於離石}
上郡治夏陽朔方治五原十二月遣使匈奴中郎將張耽將幽州烏桓諸郡營兵擊車紐等戰於馬邑斬首三千級獲生口甚衆車紐乞降而吾斯猶率其部曲與烏桓寇鈔|{
	鈔楚交翻下同}
初上命馬賢討西羌大將軍商以為賢老不如太中大夫宋漢帝不從漢由之子也|{
	宋由為公於章和之間}
賢到軍稽留不進武都太守馬融上疏曰今雜種諸羌|{
	種章勇翻}
轉相鈔盜宜及其未并亟遣深入破其支黨|{
	并合也及其勢未合而攻其支黨}
而馬賢等處處留滯羌胡百里望塵千里聽聲今逃匿避回|{
	回胡對翻繞也曲也}
漏出其後則必侵寇三輔為民大害臣願請賢所不可用關東兵五千裁假部隊之號盡力率厲埋根行首以先吏士|{
	賢曰埋根言不退也行戶剛翻先悉薦翻}
三旬之中必克破之臣又聞吳起為將暑不張蓋寒不披裘今賢野次垂幕珍肴雜遝兒子侍妾事與古反臣懼賢等專守一城賢攻於西而羌出於東且其將士將不堪命必有高克潰叛之變也|{
	鄭高克好利而不顧其君文公使克將兵而禦狄于竟陳其師旅翺翔河上衆潰而歸}
安定人皇甫規亦見賢不恤軍事審其必敗|{
	審悉也察也}
上書言狀朝廷皆不從

六年春正月丙子征西將軍馬賢與且凍羌戰于射姑山|{
	且子余翻射音夜按續漢書天文志射姑山在北地}
賢軍敗賢及二子皆沒東西羌遂大合|{
	羌居安定北地上郡西河者謂之東羌居隴西漢陽延及金城塞外者謂之西羌}
閏月鞏唐羌寇隴西遂及三輔燒園陵殺掠吏民 二月丁巳有星孛于營室|{
	晉書天文志營室二星天子之宫也又為軍糧之府及土功事孛蒲内翻}
三月上已大將軍商大會賓客讌于雒水|{
	司馬彪曰三月上已宫人皆潔於東流上洗濯祓除為大潔也按古以三月上巳日為上已今以三月三日為上已}
酒闌繼以䪥露之歌|{
	纂文曰䪥露今之挽歌也崔豹古今注曰䪥上露何易晞露晞明朝還復落人死一去何時歸䪥下戒翻一作薤}
從事中郎周舉聞之歎曰此所謂哀樂失時|{
	樂音洛}
非其所也殃將及乎|{
	左傳曰哀樂失時殃咎必至}
武都太守趙冲追擊鞏唐羌 |{
	考異曰西羌傳作武威太守今從帝紀皇甫規傳云與護羌校尉趙冲按西羌傳冲時尚為太守規傳誤也}
斬首四百餘級降二千餘人詔冲督河西四郡兵為節度|{
	余按冲以追羌之功詔督河西四郡兵則武威太守為是武都西北接漢陽東北接扶風南接漢中無緣遠督河西四郡兵}
安定上計掾皇甫規上疏曰臣比年以來數陳便宜羌戎未動策其將反馬賢始出知其必敗誤中之言在可考校|{
	比毗至翻數所角翻中竹仲翻}
臣每惟賢等擁衆四年未有成功縣師之費且百億計|{
	賢曰縣猶停也余謂出師遠征其勢縣絶不能相及故曰縣師縣讀曰懸}
出於平民|{
	平民謂齊民也}
回入姦吏|{
	謂為姦吏所侵盜也}
故江湖之人羣為盜賊青徐荒饑襁負流散|{
	襁居兩翻}
夫羌戎潰叛不由承平皆因邊將失於綏御乘常守安則加侵暴|{
	言前後相乘以侵暴羌戎為常也}
苟競小利則致大害微勝則虚張首級軍敗則隱匿不言軍士勞怨困於猾吏進不得快戰以徼功|{
	徼一遙翻}
退不得温飽以全命餓死溝渠暴骨中原徒見王師之出不聞振旅之聲|{
	賢曰振整也旅衆也穀梁傳曰出曰治兵入曰振旅}
酋豪泣血驚懼生變|{
	酋慈秋翻}
是以安不能久叛則經年臣所以搏手扣心而增歎者也願假臣兩營二郡屯列坐食之兵五千|{
	賢曰兩營謂馬賢及趙冲等二郡安定隴西也余謂兩營者扶風雍營及京兆虎牙營也}
出其不意與趙冲共相首尾土地山谷臣所曉習兵勢巧便臣已更之|{
	更工衡翻經也歷也}
可不煩方寸之印尺帛之賜高可以滌患下可以納降|{
	降戶江翻}
若謂臣年少官輕不足用者凡諸敗將非官爵之不高年齒之不邁|{
	賢曰邁往也}
臣不勝至誠沒死自陳|{
	勝音升沒死猶言昧死也冒死也}
帝不能用 庚子司空郭䖍免丙午以太僕趙戒為司空 夏使匈奴中郎將張耽度遼將軍馬續率鮮卑到穀城擊烏桓於通天山大破之|{
	穀城盖即西河郡之穀羅縣城通天山蓋即土軍縣之石樓山以其高絶故曰通天}
鞏唐羌寇北地 |{
	考異曰西羌傳作䍐種羌今從帝紀}
北地太守賈福與趙冲擊之不利 秋八月乘氏忠侯梁商病篤|{
	乘繩證翻}
敇子冀等曰吾生無以輔益朝廷死何可耗費帑藏|{
	帑他朗翻藏徂浪翻}
衣衾飯含玉匣珠貝之屬|{
	賢曰含口實也白虎通曰大夫飯以玉含以貝士飯以珠含以貝也飯父遠翻含戶紺翻}
何益朽骨百僚勞擾紛華道路祗增塵垢耳宜皆辭之丙辰薨帝親臨喪諸子欲從其誨朝廷不聽賜以東園祕器銀鏤黄腸玉匣|{
	賢曰棺以銀鏤之以柏木黄心為椁曰黄腸孔穎達曰喪服大記君松椁大夫柏椁士雜木椁鄭注椁謂周棺者也天子柏椁以端長六尺正義曰君松椁君諸侯也諸侯用松為椁材也盧云以松黄腸為椁庾云黄腸松心也大夫柏椁以柏為椁不用黄腸下天子也}
及葬賜輕車介士|{
	賢曰輕車兵車也介士甲士也}
中宫親送帝幸宣陽亭|{
	賢曰每城門各有亭即宣陽門之亭也余按續漢志雒陽城十二門無宣陽門魏晉之間洛城始有宣陽門正南門也漢雒城正南曰平城門}
瞻望車騎壬戌以河南尹乘氏侯梁冀為大將軍冀弟侍中不疑為河南尹

臣光曰成帝不能選任賢俊委政舅家可謂闇矣猶知王立之不材棄而不用|{
	事見三十二卷元延元年}
順帝援大柄授之后族|{
	援于元翻}
梁冀頑嚚凶暴著於平昔而使之繼父之位終於悖逆|{
	嚚魚巾翻悖蒲内翻又蒲沒翻}
蕩覆漢室校於成帝闇又甚焉

初梁商病篤帝親臨幸問以遺言對曰臣從事中郎周舉清高忠正可重任也由是拜舉諫議大夫|{
	續漢志曰武帝元符五年置諫大夫世祖中興以為諫議大夫}
九月諸羌寇武威 辛亥晦日有食之 冬十月癸丑以羌寇充斥凉部震恐復徙安定居扶風北地居馮翊|{
	永建四年二郡還舊治今復徙之復扶又翻}
十一月庚子以執金吾張喬行車騎將軍事將兵萬五千人屯三輔 荆州盜賊起彌年不定以大將軍從事中郎李固為荆州刺史固到遣吏勞問境内赦寇盜前釁與之更始|{
	勞力到翻更工衡翻}
於是賊帥夏密等率其魁黨六百餘人自縛歸首|{
	帥所類翻首式救翻}
固皆原之遣還自相招集開示威灋半歲間餘類悉降|{
	降戶江翻}
州内清平奏南陽太守高賜等臧穢|{
	臧古贓字通}
賜等重賂大將軍梁冀冀為之千里移檄|{
	賢曰言移檄一日行千里救之急也為于偽翻}
而固持之愈急冀遂徙固為泰山太守時泰山盜賊屯聚歷年郡兵常千人追討不能制固到悉罷遣歸農但選留任戰者百餘人|{
	任音壬}
以恩信招誘之未滿歲賊皆弭散|{
	誘音酉弭止也散逃潰而去也}


漢安元年春正月癸巳赦天下改元 秋八月南匈奴句龍吾斯與薁鞬臺耆等復反|{
	薁音郁鞬居言翻復扶又翻}
寇掠并部 丁卯遣侍中河内杜喬周舉|{
	按范書紀傳周舉汝南人時為光禄大夫}
守光禄大夫周栩馮羨魏郡欒巴張綱|{
	張綱犍為武陽人栩况羽翻}
郭遵劉班分行州郡|{
	行下孟翻}
表賢良顯忠勤其貪汚有罪者刺史二千石驛馬上之墨綬以下便輒收舉|{
	刺史二千石大吏驛馬上奏其罪取旨黜免驛馬欲速達京闕也墨綬縣令長也令長以下便收案舉劾其罪上時掌翻}
喬等受命之部張綱獨埋其車輪於雒陽都亭|{
	漢郡國縣道皆有都亭}
曰豺狼當路安問狐狸|{
	前漢京兆督郵侯文對孫寶之辭}
遂劾奏大將軍冀河南尹不疑以外戚蒙恩居阿衡之任而專肆貪叨縱恣無極謹條其無君之心十五事斯皆臣子所切齒者也書御|{
	賢曰御進也}
京師震竦時皇后寵方盛諸梁姻戚滿朝|{
	朝直遙翻}
帝雖知綱言直不能用也杜喬至兖州表奏泰山太守李固政為天下第一上徵固為將作大匠八使所劾奏多梁冀及宦者親黨互為請救事皆寢遏|{
	劾戶槩翻又戶得翻寢者已御其奏寢而不行遏者其奏未達遏而不上}
侍御史河南种暠疾之|{
	种音冲暠古老翻}
復行案舉|{
	復扶又翻}
廷尉吳雄將作大匠李固亦上言八使所糾宜急誅罰帝乃更下八使奏章令考正其罪|{
	下遐稼翻}
梁冀恨張綱思有以中傷之|{
	中竹仲翻}
時廣陵賊張嬰寇亂揚徐間積十餘年二千石不能制冀乃以綱為廣陵太守前太守率多求兵馬綱獨單車之職既到徑詣嬰壘門嬰大驚遽走閉壘綱於門罷遣吏民獨留所親者十餘人以書喻嬰請與相見嬰見綱至誠乃出拜謁綱延置上坐|{
	坐才卧翻}
譬之曰前後二千石多肆貪暴故致公等懷憤相聚二千石信有罪矣然為之者又非義也今主上仁聖欲以恩德服叛故遣太守來思以爵祿相榮不願以刑罰相加今誠轉禍為福之時也若聞義不服天子赫然震怒荆揚兖豫大兵雲合身首横分血嗣俱絶|{
	賢曰凡祭皆用牲故曰血嗣或曰父子氣血相傳故曰血嗣}
二者利害公其深計之嬰聞泣下曰荒裔愚民不能自通朝廷不堪侵枉遂復相聚偷生|{
	復扶又翻}
若魚游釡中知其不可久且以喘息須臾間耳|{
	人以氣一出入之頃為一息喘者息之疾音尺兖翻}
今聞明府之言乃嬰等更生之辰也乃辭還營明日將所部萬餘人與妻子面縛歸降 |{
	考異曰帝紀九月張嬰寇郡縣又云是歲嬰詣綱降按張綱傳云寇亂十餘年則非今年九月始寇郡縣也袁紀置嬰降事於八月下十月上今從之}
綱單車入嬰壘大會置酒為樂|{
	樂音洛}
散遣部衆任從所之親為卜居宅相田疇|{
	賢曰相視也田並畔曰疇為于偽翻相息亮翻}
子孫欲為吏者皆引召之人情悦服南州晏然朝廷論功當封梁冀遏之在郡一歲卒|{
	卒子恤翻}
張嬰等五百餘人為之制服行喪|{
	為于偽翻}
送到犍為|{
	犍居言翻}
負土成墳詔拜其子續為郎中賜錢百萬是時二千石長吏有能政者有雒陽令任峻|{
	任音壬}
冀州刺史京兆蘇章膠東相陳留吳祐雒陽令自王渙之後皆不稱職|{
	王渙事見四十八卷和帝元興元年稱尺證翻}
峻能選用文武吏各盡其用發姦不旋踵民間不畏吏其威禁猛於渙而文理政教不如也章為冀州刺史有故人為清河太守章行部欲案其姦臧|{
	行下孟翻臧古贓字通用}
乃請太守為設酒肴陳平生之好甚歡|{
	為于偽翻好呼到翻}
太守喜曰人皆有一天我獨有二天|{
	謂章必能覆蓋其惡也}
章曰今夕蘇孺文與故人飲者私恩也|{
	蘇章字孺文}
明日冀州刺史案事者公灋也遂舉正其罪州境肅然後以摧折權豪忤旨坐免時天下日敝民多愁苦論者日夜稱章朝廷遂不能復用也|{
	復扶又翻}
祐為膠東相|{
	續漢志膠東侯國屬北海國}
政崇仁簡民不忍欺嗇夫孫性私賦民錢市衣以進其父|{
	百官志縣置嗇夫一人主知民善惡為役先後知民貧富為賦多少平其差品風俗通曰嗇省也夫賦也言省息百姓均其役賦嗇音色}
父得而怒曰有君如是何忍欺之促歸伏罪性慙懼詣閣持衣自首|{
	首式救翻}
祐屏左右問其故|{
	屛必郢翻}
性具談父言祐曰掾以親故受汚穢之名所謂觀過斯知仁矣|{
	論語載孔子之言也此言觀性之過在於取民則知其心主於奉父}
使歸謝其父還以衣遺之|{
	遺于季翻}
冬十月辛未太尉桓焉司徒劉壽免 罕羌邑落五

千餘戶詣趙冲降唯燒何種據參䜌未下|{
	種章勇翻參䜌縣屬安定郡䜌音力全翻}
甲戌罷張喬軍屯 十一月壬午以司隸校尉下邳趙峻為太尉大司農胡廣為司徒

二年夏四月庚戌護羌校尉趙冲與漢陽太守張貢擊燒當羌於參䜌破之|{
	當當作何此承范紀之誤燒當燒何羌兩種也}
六月丙寅立南匈奴守義王兜樓儲為呼蘭若尸逐就單于|{
	自永和五年吾斯車紐反陳龜逼殺單于休利南庭虚位至是始立單于 考異曰袁紀去年六月立兜樓儲為單于今從范書}
時兜樓儲在京師上親臨軒授璽綬引上殿賜車馬器服金帛甚厚|{
	璽斯氏翻綬音受引上時掌翻}
詔太常大鴻臚與諸國侍子於廣陽門外祖會饗賜作樂角抵百戲|{
	太常掌樂大鴻臚典四夷之客故詔使祖單于祖會為祖道之會也賢曰廣陽城西面南頭門角抵之戲則魚龍爵馬之屬言兩兩相當亦角而為抵對即今之鬭角古之角抵也臚陵如翻}
冬閏十月趙冲擊燒當羌於阿陽破之|{
	賢曰阿陽縣屬漢陽郡故城在今秦州隴城縣西北}
十一月使匈奴中郎將扶風馬寔遣人刺殺句龍吾斯|{
	刺七亦翻句古侯翻}
凉州自九月以來地百八十震山谷坼裂壞敗城寺|{
	壞音怪敗補邁翻}
民壓死者甚衆 尚書令黄瓊以前左雄所上孝廉之選專用儒學文吏|{
	亊見上卷陽嘉元年上時掌翻}
於取士之義猶有所遺乃奏增孝悌及能從政為四科帝從之

建康元年|{
	是年四月改元}
春護羌從事馬玄為諸羌所誘將羌衆亡出塞|{
	誘音酉將如字領也}
領護羌校尉衛琚追擊玄等斬首八百餘級|{
	琚音居}
趙冲復追叛羌到建威鸇隂河|{
	賢曰續漢書建威作武威鸇隂縣名屬安定郡又曰凉州姑臧縣東南有鸇隂縣故城縣因水以為名宋白曰會州會寧縣漢鸇隂縣地黄河西南自蘭州金城縣界流注水經云河水又東過勇士縣北東流即此處復扶又翻}
軍度竟所將降胡六百餘人叛走|{
	降戶江翻}
冲將數百人追之遇羌伏兵與戰而殁冲雖死而前後多所斬獲羌遂衰耗詔封冲子為義陽亭侯 夏四月使匈奴中郎將馬寔擊南匈奴左部破之|{
	左部即句龍吾斯之黨}
於是胡羌烏桓悉詣寔降 辛巳立皇子炳為太子|{
	炳虞貴人之子也}
改元赦天下太子居承光宫帝使侍御史种暠監太子家|{
	監古銜翻}
中常侍高梵從中單駕出迎太子|{
	暠工老翻梵房戎翻又房汎翻}
時太傅杜喬等疑不欲從而未决暠乃手劒當車曰|{
	手守又翻}
太子國之儲副天命所係今常侍來無詔信何以知非姦邪今日有死而已梵辭屈不敢對馳還奏之詔報太子乃得去喬退而歎息愧暠臨事不惑|{
	愧者愧己之不能然也}
帝亦嘉其持重稱善者良久 揚徐盜賊羣起盤互連歲秋八月九江范容周生等寇掠城邑屯據歷陽|{
	歷陽縣屬九江郡賢曰今和州縣}
為江淮巨患遣御史中丞馮緄督州兵討之|{
	緄古本翻考異曰帝紀作馮赦袁紀作馮放皆誤今據緄傳}
庚午帝崩于玉堂前殿|{
	年三十}
太子即皇帝位年二歲尊皇后曰皇太后太后臨朝丁丑以太尉趙峻為太傅大司農李固為太尉參録尚書事 九月丙午葬孝順皇帝于憲陵|{
	賢曰憲陵在雒陽西十五里}
廟曰敬宗 是日京師及太原鴈門地震 庚戌詔舉賢良方正之士策問之皇甫規對曰伏惟孝順皇帝初勤王政紀綱四方幾以獲安|{
	幾讀曰冀}
後遭姦偽威分近習|{
	賢曰近習謂佞幸親近小人也}
受賂賣爵賓客交錯天下擾擾從亂如歸官民並竭上下窮虚陛下體兼乾坤|{
	以坤母臨朝以君天下行乾之德故曰體兼乾坤}
聰哲純茂攝政之初拔用忠貞其餘維綱多所改正遠近翕然望見太平而災異不息寇賊縱横|{
	縱子容翻}
殆以姦臣權重之所致也其常侍尤無狀者|{
	賢曰無狀謂無善狀}
宜亟黜遣披埽凶黨|{
	披開也埽除也}
收入財賄以塞痛怨|{
	塞悉則翻}
以答天誡大將軍冀河南尹不疑亦宜增修謙節輔以儒術省去遊娛不急之務|{
	去羌呂翻}
割減廬第無益之飾夫君者舟也民者水也|{
	家語孔子曰君者舟也民者水也水可載舟亦以覆舟君以此思危則危可知也}
羣臣乘舟者也將軍兄弟操檝者也|{
	操千高翻檝與楫同}
若能平志畢力以度元元所謂福也如其怠弛將淪波濤可不慎乎夫德不稱祿猶鑿墉之趾以益其高豈量力審功安固之道哉|{
	稱尺證翻量音良}
凡諸宿猾酒徒戲客皆宜貶斥以懲不軌令冀等深思得賢之福失人之累|{
	累良瑞翻}
梁冀忿之以規為下第拜郎中託疾免歸州郡承冀旨幾陷死者再三遂沈廢於家積十餘年|{
	幾居希翻沈持林翻 考異曰規傳云冲質之間規對策免歸積十四年檢帝紀此後别無舉賢良事或者此時規舉賢良其至對策時已在質帝世也故云冲質之間自明年數至梁冀誅亦整十四年也}
揚州刺史尹耀九江太守鄧顯討范容等於歷陽敗歿 冬十月日南蠻夷復反|{
	復扶又翻}
攻燒縣邑交趾刺史九江夏方招誘降之|{
	夏戶雅翻}
十一月九江盜賊徐鳳馬勉攻燒城邑鳳稱無上將軍勉稱皇帝 |{
	考異曰帝紀永嘉元年三月勉稱皇帝今據滕撫傳}
築營於當塗山中|{
	賢曰當塗縣之山在今宣州余按兩漢志當塗縣屬九江郡續志曰縣有馬丘聚徐鳳反於此又有塗山禹會諸侯處也又有芍陂陂在夀州安豐縣東塗山在濠州鍾離縣西九十五里以此證之漢當塗縣地當在唐濠夀二州界晉氏南渡淮民避亂渡江晉成帝乃僑立當塗縣於于湖於唐屬宣州今當塗縣非漢舊當塗縣也}
建年號置百官 十二月九江賊黄虎等攻合肥|{
	合肥縣屬九江郡賢曰故城在今廬州北應劭曰夏水出父城東南至此與淮合故曰合肥}
是歲羣盜發憲陵

孝冲皇帝|{
	諱炳諡法幼少在位曰冲司馬彪曰冲幼早夭故諡曰冲伏侯古今注曰炳之

字曰明}


永嘉元年 |{
	考異曰袁紀作元嘉誤}
春正月戊戌帝崩于玉堂前殿|{
	年三歲}
梁太后以揚徐盜賊方盛欲須所徵諸王侯到乃發喪太尉李固曰帝雖幼少猶天下之父今日崩亡人神感動豈有人子反共掩匿乎|{
	人子當作臣子}
昔秦皇沙丘之謀及近日北鄉之事皆祕不發喪|{
	沙丘事見七卷秦始皇三十七年北鄉事見上卷安帝延光四年}
此天下大忌不可之甚者也太后從之即暮發喪徵清河王蒜及渤海孝王鴻之子纘皆至京師蒜父曰清河恭王延平延平及鴻皆樂安夷王寵之子千乘貞王伉之孫也|{
	千乘貞王伉章帝建初四年封薨子寵嗣和帝永元七年改千乘國曰樂安薨子為嗣是生質帝帝既立梁太后以樂安國土卑濕租委鮮薄改封鴻渤海王清河王慶子虎威嗣國三年而薨無子鄧太后立延平為清河王諡法安心好靜曰夷蒜蘇貫翻伉音抗}
清河王為人嚴重動止有灋度公卿皆歸心焉李固謂大將軍冀曰今當立帝宜擇長年高明有德任親政事者|{
	長知兩翻任如林翻堪也}
願將軍審詳大計察周霍之立文宣戒鄧閻之利幼弱|{
	周勃事見十三卷高后八年霍光事見二十四卷昭帝元平元年鄧氏事見四十八卷和帝元興元年四十九卷殤帝延平元年閻氏事見上卷安帝延光四年}
冀不從與太后定策禁中丙辰冀持節以王青蓋車迎纘入南宫丁巳封為建平侯其日即皇帝位年八歲蒜罷歸國 將卜山陵李固曰今處處寇賊軍興費廣新創憲陵賦發非一帝尚幼小可起陵於建陵塋内依康陵制度|{
	康陵殤帝陵亦在慎陵塋内塋音營}
太后從之己未葬孝冲皇帝於懷陵 太后委政宰輔李固所言太后多從之宦官為惡者一皆斥遣天下咸望治平|{
	治直吏翻}
而梁冀深忌疾之初順帝時所除官多不以次及固在事奏免百餘人此等既怨又希望冀旨遂共作飛章誣奏固曰太尉李固因公假私依正行邪離間近戚|{
	間古莧翻}
自隆支黨大行在殯路人掩涕|{
	掩涕者掩面而泣也}
固獨胡粉飾貌|{
	燒鉛汞成粉以傅面北史曰胡粉出龜兹國}
搔頭弄姿|{
	西京雜記曰武帝遇李夫人就取玉簪搔頭自此宫人搔頭皆用玉}
槃旋偃仰從容治步|{
	從七容翻從容舒緩也治步言修治容儀行步中規矩也治直之翻}
曾無慘怛傷悴之心|{
	悴秦醉翻}
山陵未成違矯舊政善則稱己過則歸君斥逐近臣不得侍送作威作福莫固之甚矣夫子罪莫大於累父|{
	累功瑞翻}
臣惡莫深於毁君固之過釁事合誅辟|{
	辟毗亦翻}
書奏冀以白太后使下其書|{
	下遐稼翻}
太后不聽 廣陵賊張嬰復聚衆數千人反據廣陵|{
	復扶又翻下同}
二月乙酉赦天下西羌叛亂積年費用八十餘億諸將多斷盜牢禀|{
	前書音義曰牢價直也稟給也賢曰牢稟食也古者宅廩為牢斷割也減割牢廩而盜之斷丁管翻}
私自潤入皆以珍寶貨賂左右上下放縱不恤軍事士卒不得其死者白骨相望於野左馮翊梁並以恩信招誘叛羌離湳狐奴等五萬餘戶皆詣並降|{
	誘音酉湳乃感翻降戶江翻}
隴右復平 太后以徐揚盜賊益熾博求將帥三公舉涿令北海滕撫有文武才|{
	姓譜滕侯之後以國為氏}
詔拜撫九江都尉與中郎將趙序助馮緄合州郡兵數萬人共討之|{
	緄古本翻}
又廣開賞募錢邑各有差|{
	謂立賞格錢邑以功之高下為差錢賜錢也邑封邑也}
又議遣太尉李固未及行三月撫等進擊衆賊大破之斬馬勉范容周生等千五百級徐鳳以餘衆燒東城縣|{
	東城縣屬九江郡賢曰在今濠州定遠縣南}
夏五月下邳人謝安應募率其宗親設伏擊鳳斬之封安為平鄉侯拜滕撫中郎將督揚徐二州事 丙辰詔曰孝殤皇帝即位踰年君臣禮成孝安皇帝承襲統業而前世遂令恭陵在康陵之上先後相踰失其次序今其正之 六月鮮卑寇代郡 秋廬江盜賊攻尋陽|{
	尋陽縣屬廬江郡班志注云禹貢九江在南皆東合為大江余按尋陽縣本在大江之北尋水之陽吳立蘄春郡尋陽縣屬焉蘄春縣漢屬江夏郡唐蘄州之地元豐九域志蘄州東南至江州二百四十里 江州得尋陽之名由司馬氏置尋陽太守於柴桑於是江南之尋陽著於此江北之尋陽晦矣}
又攻旴台|{
	旴台縣屬下邳國音吁怡}
滕撫遣司馬王章擊破之九月庚戌太傅趙峻薨 滕撫進擊張嬰冬十一月丙午破嬰斬獲千餘人丁未中郎將趙序坐畏懦詐增首級棄市 |{
	考異曰東觀記曰取錢縑三百七十五萬今從滕撫傳}
歷陽賊華孟自稱黑帝|{
	華戶化翻}
攻殺九江太守楊岑滕撫進擊破之斬孟等三千八百級虜獲七百餘人於是東南悉平振旅而還|{
	還從宣翻又如字}
以撫為左馮翊 永昌太守劉君世鑄黄金為文蛇以獻大將軍冀益州刺史种暠糾發逮捕馳傳上言|{
	傳株戀翻下傳逮同上時掌翻}
冀由是恨暠會巴郡人服直|{
	姓譜服周内史叔服之後漢有江夏太守服徹}
聚黨數百人自稱天王暠與太守應承討捕不克吏民多被傷害|{
	被皮義翻}
冀因此陷之傳逮暠承|{
	逮暠承傳詣京師也}
李固上疏曰臣伏聞討捕所傷本非暠承之意實由縣吏懼法畏罪迫逐深苦致此不詳|{
	詳審也言不能審知賊勢驅民赴戰以致死傷也}
比盜賊羣起|{
	比毗至翻}
處處未絶暠承以首舉大姦而相隨受罪臣恐沮傷州縣糾發之意更共飾匿|{
	賢曰言各飾偽辭隱匿真狀也}
莫復盡心太后省奏|{
	復扶又翻省悉景翻}
乃赦暠承罪免官而已金蛇輸司農|{
	大司農掌諸錢穀金帛故金蛇輸司農 考異曰种暠傳云二府畏懦不敢按之今從杜喬傳}
冀從大司農杜喬借觀之喬不肯與冀小女死令公卿會喪喬獨不往冀由是銜之|{
	為冀殺喬張本}


資治通鑑卷五十二
















































































































































