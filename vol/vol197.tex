






























































資治通鑑卷一百九十七 宋 司馬光 撰

胡三省 音註

唐紀十三|{
	起昭陽單閼四月盡旃蒙大荒落五月凡二年有奇}


太宗文武大聖大廣孝皇帝中之下

貞觀十七年夏四月庚辰朔承基上變告太子謀反|{
	觀古玩翻上時掌翻}
勑長孫無忌房玄齡蕭瑀李世勣與大理中書門下參鞫之|{
	唐制凡國之大獄三司詳决三司謂給事中中書舍人與御史參鞫也今令三省與大理參鞫重其事長知兩翻瑀音禹}
反形已具上謂侍臣將何以處承乾|{
	處昌呂翻}
羣臣莫敢對通事舍人來濟進曰陛下不失為慈父太子得盡天年則善矣上從之濟護兒之子也|{
	來護兒隋將也死於宇文化及之難}
乙酉詔廢太子承乾為庶人幽於右領軍府上欲免漢王元昌死羣臣固争乃賜自盡於家而宥其母妻子|{
	元昌母孫嬪}
侯君集李安儼趙節杜荷等皆伏誅左庶子張玄素右庶子趙弘智令狐德棻等以不能諫争皆坐免為庶人|{
	令音零棻符分翻爭讀曰諍}
餘當連坐者悉赦之詹事于志寜以數諫獨蒙勞勉|{
	數所角翻勞力到翻}
以紇干承基為祐川府折衝都尉爵平棘縣公|{
	唐志岷州有祐川府隋志岷州臨洮縣後周置祐川郡唐盖因周郡名以為府也}
侯君集被收|{
	被皮義翻}
賀蘭楚石復詣闕告其事|{
	復扶又翻}
上引君集謂曰朕不欲令刀筆吏辱公故自鞫公耳君集初不承引楚石具陳始末又以所與承乾往來啟示之君集辭窮乃服上謂侍臣曰君集有功欲乞其生可乎|{
	乞如字匄也}
羣臣以為不可上乃謂君集曰與公長訣矣因泣下|{
	泣涙也}
君集亦自投於地遂斬之於市君集臨刑謂監刑將軍曰君集蹉跌至此|{
	監古銜翻蹉七何翻跌徒結翻}
然事陛下於藩邸|{
	上在藩時引君集入幕府數從征伐}
擊取二國|{
	謂吐谷渾高昌也}
乞全一子以奉祭祀上乃原其妻及子徙嶺南籍沒其家得二美人自幼飲人乳而不食初上使李靖教君集兵法君集言於上曰李靖將反矣上問其故對曰靖獨教臣以其粗而匿其精|{
	粗讀與麤同倉乎翻}
以是知之上以問靖靖對曰此乃君集欲反耳今諸夏已定|{
	夏戶雅翻下同}
臣之所教足以制四夷而君集固求盡臣之術非反而何江夏王道宗嘗從容言於上曰|{
	從于容翻}
君集志大而智小自負微功耻在房玄齡李靖之下雖為吏部尚書未滿其志以臣觀之必將為亂上曰君集材器亦何施不可朕豈惜重位但次第未至耳豈可億度|{
	朱子曰億未見而意之也度徒洛翻}
妄生猜貳邪及君集反誅上乃謝道宗曰果如卿言李安儼父年九十餘上愍之賜奴婢以養之太子承乾既獲罪魏王泰日入侍奉上面許立為太子岑文本劉洎亦勸之長孫無忌固請立晉王治|{
	請立嫡也}
上謂侍臣曰昨青雀投我懷云臣今日始得為陛下子乃更生之日也|{
	泰小字青雀}
臣有一子臣死之日當為陛下殺之|{
	為于偽翻}
傳位晉王人誰不愛其子朕見其如此甚憐之諫議大夫褚遂良曰陛下言大失願審思勿誤也安有陛下萬歲後魏王據天下肯殺其愛子傳位晉王者乎|{
	殺子而立弟非人情也褚遂良探其心術之微而言之}
陛下日者既立承乾為太子復寵魏王|{
	復扶又翻下復同}
禮秩過於承乾以成今日之禍前事不遠足以為鑒陛下今立魏王願先措置晉王始得安全耳|{
	遂良此語亦以激帝}
上流涕曰我不能爾因起入宫魏王泰恐上立晉王治謂之曰汝與元昌善元昌今敗得無憂乎治由是憂形于色上怪屢問其故治乃以狀告上憮然|{
	憮文甫翻}
始悔立泰之言矣上面責承乾承乾曰臣為太子復何所求但為泰所圖時與朝臣謀自安之術|{
	朝直遥翻}
不逞之人遂教臣為不軌耳今若泰為太子所謂落其度内承乾既廢上御兩儀殿|{
	按唐六典兩儀殿在太極殿之後盖古之内朝也常日視朝而聼事焉}
羣臣俱出獨留長孫無忌房玄齡李世勣褚遂良謂曰我三子一弟所為如是|{
	三子謂齊王祐太子承乾魏王泰一弟謂漢王元昌}
我心誠無聊賴因自投于牀無忌等爭前扶抱上又抽佩刀欲自刺|{
	刺七亦翻}
遂良奪刀以授晉王治無忌等請上所欲上曰我欲立晉王無忌曰謹奉詔有異議者臣請斬之上謂治曰汝舅許汝矣宜拜謝治因拜之上謂無忌等曰公等已同我意未知外議何如對曰晉王仁孝天下属心久矣|{
	屬之欲翻}
乞陛下試召問百官有不同者臣負陛下萬死上乃御太極殿|{
	西内正門曰承天門正殿曰太極太極之後曰兩儀殿六典朔望御太極殿視朝盖古之中朝也}
召文武六品以上謂曰承乾悖逆|{
	悖蒲内翻又蒲沒翻}
泰亦凶險皆不可立朕欲選諸子為嗣誰可者卿輩明言之衆皆讙呼曰晉王仁孝當為嗣|{
	讙與諠同}
上悦是日泰從百餘騎至永安門|{
	六典太極宫城南面三門中曰承天東曰長樂西曰永安}
勑門司盡辟其騎|{
	辟音闢六典門下省有城門郎四人掌京城皇城宫殿諸門開闔之節置門僕八百人分番上下}
引泰入肅章門幽于北苑|{
	程大昌曰太極宫之北有内苑以其在宫北故亦曰北苑苑之北門曰啓運門又北即禁苑禁苑廣矣}
丙戍詔立晉王治為皇太子御承天門樓赦天下酺三日|{
	治直吏翻酺薄乎翻}
上謂侍臣曰我若立泰則是太子之位可經營而得自今太子失道藩王窺伺者皆兩棄之|{
	伺相吏翻}
傳諸子孫永為後法且泰立承乾與治皆不全治立則承乾與泰皆無恙矣|{
	恙余亮翻}


臣光曰唐太宗不以天下大器私其所愛以杜禍亂之原可謂能遠謀矣

丁亥以中書令楊師道為吏部尚書|{
	尚辰羊翻}
初長廣公主適趙慈景生節慈景死|{
	武德元年慈景為堯君素所殺}
更適師道|{
	更工衡翻}
師道與長孫無忌等共鞫承乾獄隂為趙節地師道由是獲譴上至公主所公主以首擊地泣謝子罪上亦拜泣曰賞不避仇讎罰不阿親戚此天下至公之道不敢違也以是負姊己丑詔以長孫無忌為太子太師房玄齡為太傅肅瑀為太保|{
	東宫三師並從一品}
李世勣為詹事瑀世勣並同中書門下三品同中書門下三品自此始|{
	歐陽脩曰謂同侍中中書令也}
又以左衛大將軍李大亮領右衛率|{
	率所律翻}
前詹事于志寧中書侍郎馬周為左庶子吏部侍郎蘇朂中書舍人高季輔為右庶子刑部侍郎張行成為少詹事|{
	少詹事正四品為詹事之貳}
諫議大夫褚遂良為賓客|{
	太子賓客正三品古無此官唐始置掌侍從規諫贊相禮儀}
李世勣嘗得暴疾方云須灰可療上自翦須為之和藥|{
	為于偽翻下同須與鬚同和戶臥翻}
世勣頓首出血泣謝上曰為社稷非為卿也何謝之有世勣嘗侍宴上從容謂曰|{
	從于容翻}
朕求羣臣可託幼孤者無以踰公公往不負李密|{
	見一百八十六卷武德元年}
豈負朕哉世勣流涕辭謝齧指出血因飲沈醉上解御服以覆之|{
	齧魚結翻沈持林翻覆敷又翻}
癸巳詔解魏王泰雍州牧相州都督左武候大將軍降爵為東萊郡王|{
	雍於用翻相息亮翻}
泰府僚屬為泰所親狎者皆遷嶺表以杜楚客兄如晦有功免死廢為庶人給事中崔仁師嘗密請立魏王泰為太子左遷鴻臚少卿|{
	臚陵如翻}
庚子定太子見三師儀迎于殿門外|{
	殿門東宫之殿門也}
先拜三師答拜每門讓于三師三師坐太子乃坐其與三師書前後稱名惶恐 五月癸酉太子上表|{
	上時掌翻}
以承乾泰衣服不過隨身飲食不能適口幽憂可愍乞勑有司優加供給上從之黄門侍郎劉洎上言以太子宜勤學問親師友今入侍宫闈動踰旬朔師保以下接對甚希伏願少抑下流之愛弘遠大之規則海内幸甚|{
	少詩沼翻}
上乃命洎與岑文本褚遂良馬周更日詣東宫|{
	更工衡翻}
與太子遊處談論|{
	處昌呂翻}
六月己卯朔日有食之 丁亥太常丞鄧素使高麗還請于懷遠鎮增戍兵以逼高麗|{
	使疏吏翻麗力知翻}
上曰遠人不服則修文德以來之|{
	論語孔子之言}
未聞一二百戍兵能威絶域者也 丁酉右僕射高士亷遜位許之其開府儀同三司勲封如故|{
	勲勲級封封邑也}
仍同門下中書三品知政事 閏月辛亥上謂侍臣曰朕自立太子遇物則誨之見其飯則曰汝知稼穡之艱難則常有斯飯矣|{
	書無逸曰惟不知稼穡之艱難乃逸}
見其乘馬則曰汝知其勞逸不竭其力則常得乘之矣|{
	顔淵曰昔造父巧于使馬造父不窮其馬力是造父無佚馬也}
見其乘舟則曰水所以載舟亦所以覆舟民猶水也君猶舟也|{
	孔子家語之言}
見其息於木下則曰木從繩則正后從諫則聖|{
	書說命之言}
丁巳詔太子知左右屯營兵馬事其大將軍以下並受處分|{
	左右十二衛屯營也處昌呂翻分扶問翻}
薛延陀真珠可汗|{
	可從刋入聲汗音寒}
使其姪突利設來納幣獻馬五萬匹牛橐駝萬頭羊十萬口庚申突利設獻饌|{
	饌皺戀翻又皺皖翻}
上御相思殿|{
	按褚遂良疏云御幸北門受其獻食則相思殿盖在玄武門内}
大饗羣臣設十部樂|{
	增樂為十部見一百九十五卷十四年}
突利設再拜上壽賜賚甚厚契苾何力上言薛延陀不可與昏|{
	上時掌翻契欺訖翻苾毘必翻}
上曰吾已許之矣豈可為天子而食言乎何力對曰臣非欲陛下遽絶之也願且遷延其事臣聞古有親迎之禮若敕夷男使親迎|{
	迎魚敬翻}
雖不至京師亦應至靈州彼必不敢來則絶之有名矣夷男性剛戾既不成昏其下復攜貳|{
	復扶又翻}
不過一二年必病死二子爭立則可以坐制之矣上從之乃徵真珠可汗使親迎仍發詔將幸靈州與之會真珠大喜欲詣靈州其臣諫曰脱為所留悔之無及真珠曰吾聞唐天子有聖德我得身往見之死無所恨且漠北必當有主我行决矣勿復多言上發使三道受其所獻雜畜薛延陀先無庫廐真珠調歛諸部|{
	復扶又翻使疏吏翻下三使同畜許救翻調徒釣翻歛力贍翻}
往返萬里道涉沙磧無水草|{
	磧七迹翻}
耗死將半失期不至議者或以為聘財未備而與為昏將使戎狄輕中國上乃下詔絶其昏停幸靈州追還三使禇遂良上疏以為薛延陀本一俟斤|{
	良上時掌翻俟渠之翻}
陛下盪平沙塞萬里蕭條|{
	謂平突厥也塞北皆沙磧故曰沙塞}
餘寇奔波須有酋長璽書鼓纛立為可汗|{
	見一百九十三卷二年酋慈由翻長知兩翻璽斯氏翻纛徒到翻}
比者復降鴻私許其姻媾|{
	比毗至翻復扶又翻見上卷十六年}
西告吐蕃|{
	吐從暾入聲}
北諭思摩中國童幼靡不知之御幸北門受其獻食羣臣四夷宴樂終日|{
	樂音洛}
咸言陛下欲安百姓不愛一女凡在含生孰不懷德今一朝生進退之意有改悔之心臣為國家惜茲聲聼|{
	為于偽翻}
所顧甚少所失殊多|{
	少詩沼翻}
嫌隙既生必構邊患彼國蓄見欺之怒此民懷負約之慙恐非所以服遠人訓戎士也陛下君臨天下十有七載|{
	載子亥翻}
以仁恩結庶類以信義撫戎夷莫不欣然負之無力|{
	此二語攷之舊書禇遂良傳亦是如此然其意義難于強解或曰力當作益言負延陀之約為無益也}
何惜不使有始有卒乎|{
	卒子恤翻}
夫龍沙以北部落無算|{
	匈奴庭謂之龍城無常處故沙幕因謂之龍沙}
中國誅之終不能盡當懷之以德使為惡者在夷不在華失信者在彼不在此則堯舜禹湯不及陛下遠矣上不聼是時羣臣多言國家既許其昏受其聘幣不可失信戎狄更生邊患上曰卿曹皆知古而不知今昔漢初匈奴彊中國弱故飾子女捐金絮以餌之得事之宜今中國彊戎狄弱以我徒兵一千可擊胡騎數萬|{
	騎奇計翻}
薛延陀所以匍匐稽顙惟我所欲不敢驕慢者以新為君長雜姓非其種族欲假中國之勢以威服之耳|{
	匍薄乎翻匐蒲北翻稽音啓種章勇翻}
彼同羅僕骨回紇等十餘部|{
	紇下沒翻}
兵各數萬并力攻之立可破滅所以不敢發者畏中國所立故也今以女妻之彼自恃大國之婿雜姓誰敢不服戎狄人而獸心一旦微不得意必反噬為害今吾絶其昏殺其禮|{
	妻七細翻下可妻同殺所界翻}
雜姓知我棄之不日將瓜剖之矣卿曹第志之|{
	瓜剖猶瓜分也志猶記之}


臣光曰孔子稱去食去兵不可去信|{
	見論語去羌呂翻}
唐太宗審知薛延陀不可妻則初勿許其昏可也既許之矣乃復恃彊棄信而絶之|{
	復扶又翻}
雖滅薛延陀猶可羞也王者發言出令可不慎哉

上曰蓋蘇文弑其君而專國政|{
	見上卷十六年}
誠不可忍以今日兵力取之不難但不欲勞百姓吾欲且使契丹靺鞨擾之何如|{
	契欺訖翻又音喫靺鞨音末曷}
長孫無忌曰蓋蘇文自知罪大畏大國之討必嚴設守備陛下姑為之隐忍|{
	為于偽翻}
彼得以自安必更驕惰愈肆其惡然後討之未晩也上曰善|{
	觀此則知帝之雄心末嘗一日不在高麗也}
戊辰詔以高麗王藏為上桂國遼東郡王高麗王遣使持節册命|{
	麗力知翻使疏吏翻}
丙子徙東萊王泰為順陽王 初太子承乾失德上密謂中書侍郎兼左庶子杜正倫曰吾兒足疾乃可耳但疎遠賢良狎昵羣小卿可察之|{
	言承乾之足不良於行猶云可也若其遠賢良近羣小則不可不諫誨之遠于願翻昵尼質翻}
果不可教示當來告我正倫屢諫不聼乃以上語告之太子抗表以聞上責正倫漏泄對曰臣以此恐之冀其遷善耳上怒出正倫為穀州刺史及承乾敗秋七月辛卯復左遷正倫為交州都督|{
	復扶又翻}
初魏徵嘗薦正倫及侯君集有宰相材請以君集為僕射且曰國家安不忘危不可無大將諸衛兵馬宜委君集專知上以君集好誇誕不用|{
	將即亮翻好呼到翻}
及正倫以罪黜君集謀反誅上始疑徵阿黨又有言徵自録前後諫辭以示起居郎褚遂良者上愈不悦乃罷叔玉尚主而踣所撰碑|{
	許昏撰碑事見上卷本年踣滿北翻仆也}
初上謂監修國史房玄齡曰|{
	歷代史官隸祕書省著作局皆著作郎掌修國史北齊詔魏收撰史又詔平原王高隆之總監之書名而已貞觀三年始移史館于禁中在門下省北宰相監修國史自是著作郎始罷史職監古銜翻}
前世史官所記皆不令人主見之何也對曰史官不虛美不隐惡若人主見之必怒故不敢獻也上曰朕之為心異於前世帝王欲自觀國史知前日之惡為後來之戒公可撰次以聞|{
	撰士免翻}
諫議大夫朱子奢上言|{
	上時掌翻下上之同}
陛下聖德在躬舉無過事史官所述義歸盡善陛下獨覽起居於事無失若以此法傳示子孫竊恐曾玄之後或非上智飾非護短史官必不免刑誅如此則莫不希風順旨全身遠害悠悠千載何所信乎所以前代不觀蓋為此也|{
	遠于願翻載子亥翻為于偽翻}
上不從玄齡乃與給事中許敬宗等刪為高袓今上實録癸巳書成上之上見書六月四日事語多微隐|{
	謂誅建成元吉事也}
謂玄齡曰周公誅管蔡以安周季友鴆叔牙以存魯|{
	周公弟也管叔兄也成王幼周公攝政管蔡流言挾武庚以叛周公誅之以安周室魯公子慶父叔牙季友皆桓公子也莊公疾問後於叔牙牙曰慶父才問季友友曰臣以死奉般遂鴆叔牙而立般}
朕之所為亦類是耳史官何諱焉即命削去浮詞|{
	去羌呂翻}
直書其事 八月庚戍以洛州都督張亮為刑部尚書參預朝政|{
	朝直遥翻}
以左衛大將軍太子右衛率李大亮為工部尚書大亮身居三職宿衛兩宫|{
	三職即謂為工部尚書及衛兩宫也率所律翻}
㳟儉忠謹每宿直必坐寐逹旦房玄齡甚重之每稱大亮有王陵周勃之節可當大位初大亮為龎玉兵曹為李密所獲同輩皆死賊帥張弼見而釋之遂與定交|{
	帥所類翻}
及大亮貴求弼欲報其德弼時為將作丞|{
	唐監丞從六品下}
自匿不言大亮遇諸途而識之持弼而泣多推家貲以遺弼|{
	推吐雷翻遺于季翻}
弼拒不受大亮言於上乞悉以其官爵授弼上為之擢弼為中郎將|{
	上為于偽翻將即亮翻}
時人皆賢大亮不負恩而多弼之不代也 九月庚辰新羅遣使言百濟攻取其國四十餘城復與高麗連兵|{
	使疏吏翻復扶又翻}
謀絶新羅入朝之路乞兵救援|{
	朝直遥翻下同}
上命司農丞相里玄奬齎璽書賜高麗|{
	相里姓玄奬名姓譜臯陶之後為理氏商末理證孫仲師遭難去王姓里至里克為晉所誅其妻携少子逃居相城因為相里氏}
曰新羅委質國家朝貢不乏|{
	質職日翻}
爾與百濟各宜戢兵|{
	戢阻立翻}
若更攻之明年發兵擊爾國矣 癸未徙承乾于黔州|{
	黔其今翻}
甲午徙順陽王泰于均州|{
	武當縣漢属南陽郡晉属順陽郡宋属始平郡梁置武當郡及興州後周改豐州隋開皇初改均州大業初廢為武當縣属浙陽郡義寧二年分浙陽之武當均陽置均州孫愐曰汮水出折縣北山入沔汮今作均隋置均州以水名州也}
上曰父子之情出於自然朕今與泰生離|{
	生離謂生而離别也楚辭曰哀莫哀兮生别離}
亦何心自處然朕為天下主但使百姓安寧私情亦可割耳又以泰所上表示近臣曰泰誠為俊才朕心念之卿曹所知但以社稷之故不得不斷之以義|{
	處昌呂翻上時掌翻斷丁亂翻}
使之居外者亦所以兩全之耳 先是諸州長官或上佐歲首親奉貢物入京師謂之朝集使|{
	朝集使自隋以來有之先悉薦翻長知兩翻朝直遥翻使疏吏翻下同}
亦謂之考使京師無邸率僦屋與商賈雜居上始命有司為之作邸|{
	僦即就翻賈音古為于偽翻}
冬十一月己卯上祀圜丘|{
	貞觀禮冬至祀昊天上帝于圜丘}
初上與隐太子巢刺王有隙|{
	刺盧逹翻}
密明公贈司空封德彝隐持兩端楊文幹之亂上皇欲廢隐太子而立上|{
	見一百九十一卷武德七年}
德彞固諫而止其事甚祕上不之知薨後乃知之壬辰治書侍御史唐臨始追劾其事請黜官奪爵|{
	治直之翻劾戶槩翻又戶得翻}
上命百官議之尚書唐儉等議德彞罪暴身後恩結生前所歷衆官不可追奪請降贈改諡詔黜其贈官改諡曰繆削所食實封|{
	諡法名與實爽曰繆蔽仁傷賢曰繆六典曰魏氏五等皆以鄉亭多假空名不食本邑隋氏始立王公侯以下制度至唐因之率多虚名其言食實封者乃得真戶舊制戶皆三丁已上一分入國開元中定以三丁為限租賦全入封家}
勑選良家女以實東宫癸巳太子遣左庶子于志寧辭之上曰吾不欲使子孫生於微賤耳今既致辭當從其意上疑太子仁弱密謂長孫無忌曰公勸我立雉奴|{
	治小字雉奴}
雉奴懦|{
	懦奴臥翻又萬亂翻}
恐不能守社稷奈何吳王恪英果類我我欲立之何如無忌固爭以為不可上曰公以恪非己之甥邪無忌曰太子仁厚真守文良主儲副至重豈可數易|{
	數所角翻}
願陛下熟思之上乃止十二月壬子上謂吳王恪曰父子雖至親及其有罪則天下之法不可私也漢已立昭帝燕王旦不服隂圖不軌霍光折簡誅之|{
	見二十三卷漢昭帝元鳳元年}
為人臣子不可不戒|{
	為後無忌殺恪張本}
庚申車駕幸驪山温湯庚午還宫|{
	驪力知翻}


十八年春正月乙未車駕幸鍾官城|{
	漢鍾官在上林苑中至唐時盖故城猶存也其地當在鄠杜二縣界}
庚子幸鄠縣|{
	鄠音戶}
壬寅幸驪山温湯相里玄奬至平壤莫離支已將兵擊新羅破其兩城|{
	將即亮翻}
高麗王使召之乃還|{
	麗力知翻還從宣翻又音如字}
玄奬論使勿攻新羅莫離支曰昔隋人入寇新羅乘釁侵我地五百里|{
	謂隋焬帝伐高麗時}
自非歸我侵地恐兵未能已玄奬曰既往之事焉可追論|{
	焉於䖍翻}
至於遼東諸城本皆中國郡縣|{
	高麗之地漢魏皆為郡縣晉氏之亂始與中國絶}
中國尚且不言高麗豈得必求故地莫離支竟不從二月乙巳朔玄奬還具言其狀上曰蓋蘇文弑其君賊其大臣殘虐其民今又違我詔命侵暴鄰國不可以不討諫議大夫褚遂良曰陛下指麾則中原清晏顧眄則四夷讋服|{
	眄眠見翻讋之涉翻}
威望大矣今乃渡海遠征小夷若指期克捷猶可也萬一蹉跌|{
	蹉七何翻跌徒結翻}
傷威損望更興忿兵則安危難測矣李世勣曰間者薛延陀入寇|{
	謂十五年擊突厥思摩也}
陛下欲發兵窮討魏徵諫而止使至今為患曏用陛下之策北鄙安矣上曰然此誠徵之失朕尋悔之而不欲言恐塞良謀故也|{
	塞悉則翻}
上欲自征高麗褚遂良上疏|{
	上時掌翻}
以為天下譬猶一身兩京心腹也州縣四支也四夷身外之物也高麗罪大誠當致討但命二三猛將將四五萬衆|{
	將即亮翻下名將同}
仗陛下威靈取之如反掌耳今太子新立年尚幼穉|{
	穉直二翻}
自餘藩屏陛下所知|{
	屛必郢翻}
一旦棄金湯之全踰遼海之險以天下之君輕行遠舉皆愚臣之所甚憂也上不聼時羣臣多諫征高麗者上曰八堯九舜不能冬種野夫童子春種而生得時故也夫天有其時人有其功|{
	夫天音扶}
蓋蘇文陵上虐下民延頸待救此正高麗可亡之時也議者紛紜但不見此耳 己酉上幸靈口|{
	新書作零口九域志京兆臨潼縣有零口鎮臨潼唐之昭應縣昭應唐初之新豐縣按宋白續通典京兆新豐縣界有零水零口蓋零水之口}
乙卯還宫 三月辛卯以左衛將軍薛萬徹守右衛大將軍上嘗謂侍臣曰於今名將惟世勣道宗萬徹三人而已世勣道宗不能大勝亦不大敗萬徹非大勝則大敗夏四月上御兩儀殿皇太子侍上謂羣臣曰太子性行外人亦聞之乎|{
	行下孟翻}
司徒無忌曰太子雖不出宫門天下無不欽仰聖德上曰吾如治年時頗不能循常度治自幼寛厚諺曰生狼猶恐如羊|{
	曹大家女誡曰生男如狼猶恐其羊生女如鼠猶恐其虎盖古語也}
冀其稍壯自不同耳無忌對曰陛下神武乃撥亂之才太子仁恕實守文之德趣尚雖異各當其分|{
	趣七喻翻分扶問翻}
此乃皇天所以祚大唐而褔蒼生者也|{
	無忌之保護大子至矣迨其後也以元舅之親為婦人所間不能保其身保其家而唐亦幾于不祀則太子不可謂之寛厚謂之闇弱可也}
辛亥上幸九成宫壬子至太平宫|{
	京兆鄠縣東南三十里有隋太平宫}
謂侍臣曰人臣順旨者多犯顔則少|{
	少詩沼翻}
今朕欲自聞其失諸公其直言無隐長孫無忌等皆曰陛下無失劉洎曰頃有上書不稱旨者陛下皆面加窮詰無不慙懼而退恐非所以廣言路|{
	洎其冀翻上時掌翻稱尺證翻詰去吉翻}
馬周曰陛下比來賞罰微以喜怒有所高下此外不見其失上皆納之上好文學而辯敏羣臣言事者上引古今以折之|{
	比毗至翻好呼到翻折之列翻}
多不能對劉洎上書諫曰帝王之與凡庶聖哲之與庸愚上下相懸擬倫斯絶是知以至愚而對至聖以極卑而對至尊徒思自強不可得也陛下降恩旨假慈顔凝旒以聼其言虛襟以納其說猶恐羣下未敢對敭|{
	敭與楊同}
况動神機縱天辯飾辭以折其理引古以排其議欲令凡庶何階應答且多記則損心多語則損氣心氣内損形神外勞初雖不覺後必為累|{
	累力瑞翻下之累同}
須為社稷自愛豈為性好自傷乎|{
	為于偽翻性好謂性之所好也好呼到翻}
至如秦政彊辯失人心於自矜魏文宏才虧衆望於虛說此材辯之累較然可知矣上飛白答之|{
	飛白書也}
曰非慮無以臨下非言無以述慮比有談論遂致煩多|{
	比毗至翻}
輕物驕人恐由茲道形神心氣非此為勞今聞讜言虛懷以改|{
	讜音黨}
己未至顯仁宫|{
	是時幸九成宫為避暑也至八月甲子始自九成宫還京師顯仁宫在河南壽安縣幸東都則為中頓幸九成宫非其所經之路岐州郿縣有隋安仁宫顯恐當作安}
上將征高麗秋七月辛卯勑將作大監閻立德等詣洪饒江三州造船四百艘以載軍糧|{
	艘蘇遭翻}
甲午下詔遣營州都督張儉等帥幽營二都督兵及契丹奚靺鞨先擊遼東以觀其勢|{
	帥讀曰率契欺訖翻又音喫}
以太常卿韋挺為饋運使|{
	使疏吏翻}
以民部侍郎崔仁師副之自河北諸州皆受挺節度聼以便宜從事又命太僕少卿蕭銳運河南諸州糧入海銳瑀之子也 八月壬子上謂司徒無忌等曰人苦不自知其過卿可為朕明言之|{
	為于偽翻}
對曰陛下武功文德臣等將順之不暇|{
	孝經君子之事上也將順其美匡救其惡}
又何過之可言上曰朕問公以已過公等乃曲相諛悦朕欲面舉公等得失以相戒而改之何如皆拜謝上曰長孫無忌善避嫌疑應物敏速决斷事理古人不過而總兵攻戰非其所長|{
	斷丁亂翻}
高士亷涉獵古今心術明逹臨難不改節當官無朋黨所乏者骨鯁規諫耳|{
	難乃旦翻}
唐儉言辭辯捷善和解人事朕三十年遂無言及于獻替|{
	帝未起兵時儉在晉陽雅與帝游}
楊師道性行純和自無愆違而情實怯懦緩急不可得力岑文本性質敦厚文章華贍而持論恒據經遠自當不負於物劉洎性最堅貞有利益然其意尚然諾私於朋友馬周見事敏速性甚貞正論量人物直道而言朕比任使多能稱意|{
	行下孟翻贍而艷翻恒戶登翻論盧昆翻量音良比毗至翻稱尺證翻}
禇遂良學問稍長性亦堅正每寫忠誠親附於朕譬如飛鳥依人人自憐之 甲子上還京師 丁卯以散騎常侍劉洎為侍中|{
	散悉亶翻騎奇寄翻}
行中書侍郎岑文本為中書令太子左庶子中書侍郎馬周守中書令文本既拜還家有憂色母問其故文本曰非勲非舊濫荷寵榮位高責重所以憂懼親賓有來賀者文本曰今受弔不受賀也文本弟文昭為校書郎喜賓客|{
	荷下可翻唐校書郎正九品上掌讐校典籍属祕書肖喜許記翻}
上聞之不悦嘗從容謂文本曰卿弟過爾交結恐為卿累|{
	從千容翻累力瑞翻}
朕欲出為外官何如文本泣曰臣弟少孤老母特所鍾愛未嘗信宿離左右|{
	少詩照翻離力智翻}
今若出外母必愁悴|{
	悴秦醉翻}
儻無此弟亦無老母矣因歔欷嗚咽|{
	歔音虚欷音希又許既翻}
上愍其意而止惟召文昭嚴戒之亦卒無過|{
	卒于恤翻}
九月以諫議大夫褚遂良為黄門侍郎參預朝政|{
	黄門侍郎即門下侍郎正四品上掌貳侍中之職凡政之弛張事之與奪皆參預焉朝直遥翻下同}
焉耆貳於西突厥西突厥大臣屈利啜為其弟娶焉耆王女|{
	啜陟劣翻為于偽翻}
由是朝貢多闕安西都護郭孝恪請討之|{
	按唐六典永徽中始置安南安西大都護又按舊書郭孝恪傳貞觀十六年行安西都護西州刺史盖滅高昌後便置安西都護而加大字則在永徽中也安西都護府時治西州西至焉耆七百一十里}
詔以孝恪為西州道行軍總管帥步騎三千出銀山道以擊之|{
	帥讀曰率騎奇寄翻}
會焉耆王弟頡鼻兄弟三人至西州孝恪以頡鼻弟栗婆準為鄉導|{
	鄉讀曰嚮}
焉耆城四面皆水恃險而不設備孝恪倍道兼行夜至城下命將士浮水而度|{
	將即亮翻}
比曉登城執其王突騎支|{
	比必寐翻舊唐書作龍突騎支騎奇寄翻下同}
獲首虜七千級留栗婆凖攝國事而還|{
	還從宣翻又如字}
孝恪去三日屈利啜引兵救焉耆不及執栗婆凖以勁騎五千追孝恪至銀山孝恪還擊破之追奔數十里辛卯上謂侍臣曰孝恪近奏稱八月十一日往擊焉耆二十日應至必以二十二日破之朕計其道里使者今日至矣|{
	使疏吏翻下同}
言未畢驛騎至西突厥處那啜使其吐屯攝焉耆遣使入貢上數之曰我發兵擊得焉耆汝何人而據之吐屯懼返其國焉耆立栗婆準從父兄薛婆阿那支為王仍附於處那啜|{
	處那啜盖亦西突厥之部落酋長數所其翻從才用翻}
乙未鴻臚奏高麗莫離支貢白金|{
	臚陵知翻}
禇遂良曰莫離支弑其君九夷所不容|{
	後漢書東方有九夷曰畎夷干夷方夷黄夷白夷赤夷玄夷風夷陽夷白虎通夷者蹲也言無禮儀或云夷者抵也言仁而好生抵地而出故天性柔順易以道禦}
今將討之而納其金此郜鼎之類也|{
	春秋桓公取郜大鼎于宋納于太廟非禮也郜古到翻}
臣謂不可受上從之上謂高麗使者曰汝曹皆事高武有官爵莫離支弑逆汝曹不能復讎今更為之遊說以欺大國罪孰大焉悉以属大理|{
	為于偽翻属之欲翻}
冬十月辛丑朔日有食之 甲寅車駕行幸洛陽以房玄齡留守京師|{
	守手又翻}
右衛大將軍工部尚書李大亮副之 郭孝恪鏁焉耆王突騎支及其妻子詣行在勑宥之丁巳上謂太子曰焉耆王不求賢輔不用忠謀自取滅亡係頸束手溧揺萬里人以此思懼則懼可知矣己巳畋于澠池之天池|{
	澠池縣漢晉属弘農郡後魏置澠池郡後周置河南郡大象中廢郡以縣属洛州唐属穀州酈道元曰熊耳山際有池池水東南流水側有一池世謂之澠池澠彌兖翻}
十一月壬申至洛陽前宜州刺史鄭元璹已致仕上以其嘗從隋焬帝伐高麗|{
	鄭元璹仕隋為右武候將軍從伐高麗璹殊玉翻}
召詣行在問之對曰遼東道遠糧運艱阻東夷善守城攻之不可猝下上曰今日非隋之比公但聼之|{
	帝所謂恃國家之大甲兵之強筭畧之足以取勝欲見威于敵者也}
張儉等值遼水漲久不得濟上以為畏懦召儉詣洛陽至具陳山川險易水草美惡|{
	懦乃臥翻又奴亂翻易以豉翻}
上悦上聞洛州刺史程名振善用兵召問方略嘉其才敏勞勉之曰|{
	洛音名勞力到翻}
卿有將相之器|{
	將即亮翻相息亮翻}
朕方將任使名振失不拜謝上試責怒以觀其所為曰山東鄙夫得一刺史以為富貴極邪敢於天子之側言語麤疎又復不拜|{
	復扶又翻}
名振謝曰疎野之臣未嘗親奉聖問適方心思所對故忘拜耳舉止自若應對愈明辯上乃歎曰房玄齡處朕左右二十餘年每見朕譴責餘人顔色無主|{
	此玄齡所以為忠謹也處昌呂翻}
名振平生未嘗見朕朕一旦責之曾無震懾辭理不失真奇士也即日拜右驍衛將軍|{
	懾之涉翻驍堅堯翻}
甲午以刑部尚書張亮為平壤道行軍大總管帥江淮嶺峽兵四萬|{
	硤中諸州夔硤歸是也帥讀曰率下同}
長安洛陽募士三千戰艦五百艘自萊州泛海趨平壤|{
	艦戶黯翻艘蘇遭翻趨七喻翻}
又以太子詹事左衛率李世勣為遼東道行軍大總管帥步騎六萬及蘭河二州降胡趣遼東|{
	率所律翻騎奇寄翻降戶江翻趣與趨同音七喻翻}
兩軍合勢竝進庚子諸軍大集於幽州遣行軍總管姜行本少府少監丘行淹先督衆工造梯衝于安蘿山時遠近勇士應募及獻攻城器械者不可勝數上皆親加損益取其便易|{
	勝音升易以豉翻}
又手詔諭天下以高麗蓋蘇文弑主虐民情何可忍今欲廵幸幽薊問罪遼碣|{
	碣其謁翻}
所過營頓無為勞費且言昔隋焬帝殘暴其下高麗王仁愛其民以思亂之軍擊安和之衆故不能成功今略言必勝之道有五一曰以大擊小二曰以順討逆三曰以治乘亂|{
	治直吏翻}
四曰以逸待勞五曰以悦當怨何憂不克布告元元勿為疑懼|{
	太宗以高麗為必可克而卒不克所謂常勝之家難與慮敵也}
於是凡頓舍供費之具减者大半十一月辛丑武陽懿公李大亮卒于長安|{
	卒子恤翻}
遺表請罷高麗之師家餘米五斛布三十匹親戚早孤為大亮所養喪之如父者十有五人|{
	喪息浪翻}
壬寅故太子承乾卒于黔州上為之廢朝|{
	卒子恤翻為于偽翻}
葬以國公禮 甲寅詔諸軍及新羅百濟奚契丹分道擊高麗 初上遣突厥俟利苾可汗北度河|{
	見上卷十五年}
薛延陀真珠可汗恐其部落翻動意甚惡之|{
	夷狄畏服大種其天性也俟利苾承袓父之餘威依中國之大援還主部落薛延陀雖據漢北突厥之種類與鐵勒諸部舊属突厥聞俟利苾之來恐翻而從之故甚惡焉惡烏路翻}
豫蓄輕騎於漠北欲擊之上遣使戒勑無得相攻|{
	騎奇寄翻使疏吏翻}
真珠可汗對曰至尊有命安敢不從然突厥翻覆難期當其未破之時歲犯中國殺人以千萬計臣以為至尊克之當翦為奴婢以賜中國之人乃反養之如子其恩德至矣而結社率竟反|{
	見一百九十五卷十三年}
此屬獸心安可以人理待也臣荷恩深厚請為至尊誅之自是數相攻|{
	荷下可翻為于偽翻數所角翻}
俟利苾之北度也有衆十萬勝兵四萬人|{
	勝音升}
俟利苾不能撫御衆不愜服戊午悉棄俟利苾南度河請處於勝夏之間上許之羣臣皆以為陛下方遠征遼左而置突厥於河南距京師不遠|{
	勝州去京師一千八百三十里夏州去京師一千一百一十里處昌呂翻夏戶雅翻}
豈得不為後慮願留鎮洛陽遣諸將東征上曰夷狄亦人耳其情與中夏不殊|{
	將即亮翻夏戶雅翻}
人主患德澤不加不必猜忌異類蓋德澤洽則四夷可使如一家猜忌多則骨肉不免為讎敵煬帝無道失人已久遼東之役人皆斷手足以避征役|{
	斷丁管翻}
玄感以運卒反於黎陽|{
	見一百八十二卷隋焬帝大業九年}
非戎狄為患也朕今征高麗皆取願行者募十得百募百得千其不得從軍者皆憤歎鬱邑豈比隋之行怨民哉|{
	行怨民語法本之晁錯}
突厥貧弱吾收而養之計其感恩入於骨髓豈肯為患且彼與薛延陀嗜欲略同彼不北走薛延陀而南歸我其情可見矣顧謂褚遂良曰爾知起居為我志之|{
	走音奏為于偽翻}
自今十五年保無突厥之患俟利苾既失衆輕騎入朝|{
	騎奇寄翻朝直遥翻}
上以為右武衛將軍

十九年春正月韋挺坐不先行視漕渠運米六百餘艘至盧思臺側|{
	據舊書盧思臺去幽州八百里此漕渠盖即曹操伐烏丸所開泉州渠也上承桑乾河行下孟翻艘蘇遭翻}
淺塞不能進|{
	塞悉則翻}
械送洛陽丁酉除名以將作少監李道裕代之崔仁師亦坐免官 滄州刺史席辯坐贓汚二月庚子詔朝集使臨觀而戮之|{
	朝直遥翻使疏吏翻}
庚戍上自將諸軍發洛陽以特進蕭瑀為洛陽宫留守|{
	將即亮翻守手又翻}
乙卯詔朕發定州後宜令皇太子監國開府儀同三司致仕尉遲敬德上言陛下親征遼東太子在定州長安洛陽心腹空虛恐有玄感之變且邊隅小夷不足以勤萬乘|{
	監工銜翻尉紆勿翻上時掌翻乘繩證翻}
願遣偏師征之指期可殄上不從以敬德為左一馬軍總管使從行丁巳詔諡殷太師比干曰忠烈所司封其墓春秋祠

以少牢給隨近五戶供灑掃|{
	少詩照翻灑所賣翻掃素報翻又並上聲}
上之發京師也命房玄齡得以便宜從事不復奏請|{
	復扶又翻}
或詣留臺稱有密玄齡問密謀所在對曰公則是也玄齡驛送行在上聞留守有表送告密人上怒使人持長刀於前而後見之問告者為誰曰房玄齡上曰果然叱令腰斬璽書讓玄齡以不能自信|{
	璽斯氏翻}
更有如是者可專决之癸亥上至鄴自為文祭魏太袓|{
	魏太袓葬鄴城西鄴縣本相州治所後周太象二年隋文帝輔政尉遲迴起兵于鄴兵敗鄴城破文帝令焚鄴城徙其居人南遷四十五里以安陽城為相州治所焬帝復于鄴故都大慈寺置鄴縣貞觀八年始築今治所小城}
曰臨危制變料敵設奇一將之智有餘萬乘之才不足|{
	將即亮翻乘繩證翻}
是月李世勣軍至幽州|{
	洛陽至幽州一千六百里}
三月丁丑車駕至定州|{
	洛陽至定州一千二百里}
丁亥上謂侍臣曰遼東本中國之地隋氏四出師而不能得|{
	隋文帝開皇十八年伐高麗焬帝大業八年九年十年三伐高麗}
朕今東征欲為中國報子弟之讎|{
	言中國之人其父兄死于高麗今伐之是為其子弟報父兄之讎為于偽翻}
高麗雪君父之恥耳|{
	言蓋蘇文弑其主而其臣子不能討恥莫大焉今討其罪是為高麗雪恥}
且方隅大定惟此未平故及朕之未老用士大夫餘力以取之朕自發洛陽惟噉肉飯|{
	噉徒濫翻又徒覽翻}
雖春蔬亦不之進懼其煩擾故也上見病卒召至御榻前存慰付州縣療之士卒莫不感悦有不預征名|{
	謂不預東征之名籍者}
自願以私裝從軍動以千計皆曰不求縣官勲賞惟願效死遼東上不許上將發太子悲泣數日上曰今留汝鎮守輔以俊賢欲使天下識汝風采夫為國之要在於進賢退不肖賞善罰惡至公無私汝當努力行此悲泣何為命開府儀同三司高士亷攝太子太傅與劉洎馬周少詹事張行成|{
	詹事秦官自漢以來掌東宫内外衆務員一人後魏置二人分左右尋復置一人至唐又置少詹事一人正四品上洎其冀翻}
右庶子高季輔同掌機務輔太子長孫無忌岑文本與吏部尚書楊師道從行壬辰車駕發定州親佩弓矢手結雨衣於鞍後命長孫無忌攝侍中楊師道攝中書令李世勣軍發柳城|{
	柳城縣營州治所}
多張形勢若出懷遠鎮者|{
	營州有懷遠守捉城}
而濳師北趣甬道出高麗不意夏四月戊戍朔世勣自通定濟遼水|{
	通定鎮在遼水西隋大業八年伐遼所置甬道隋起浮橋度遼水所築趣七喻翻甬余隴翻}
至玄菟|{
	陳壽曰漢武帝開玄菟郡治沃沮後為夷貊所侵徙郡句驪縣西北有遼山遼水所出}
高麗大駭城邑皆閉門自守壬寅遼東道副大總管江夏王道宗將兵數千至新城|{
	夏戶雅翻將即亮翻下同考異曰唐歷張儉懼敵不敢深入江夏王道宗固請將百騎覘賊帝許之因問往返幾日對曰往十日周覽十日返十日總經一月望謁陛下遂秣馬束兵經歷險阻直登遼東城南觀其地形險易安營置陳之所及還賊已引兵斷其歸路道宗擊之盡殪斬關而出如期謁見帝歎曰賁育之勇何以過此賜金五十斤絹干匹今從實録}
折衝都尉曹三良引十餘騎直壓城門城中驚擾無敢出者|{
	騎奇寄翻}
營州都督張儉將胡兵為前鋒進渡遼水趍建安城|{
	自遼東城西行三百里至建安城漢平郭縣地趍七喻翻}
破高麗兵斬首數千級 太子引高士亷同榻視事又令更為士亷設案士亷固辭丁未車駕發幽州上悉以軍中資糧器械簿書委岑

文本文本夙夜勤力躬自料配籌筆不去手|{
	籌所以計筭筆所以書}
精神耗竭言辭舉措頗異平日上見而憂之謂左右曰文本與我同行恐不與我同返是日遇暴疾而薨其夕上聞嚴鼓聲|{
	晉灼曰嚴鼓疾擊之鼓司馬法曰昏鼓四通為大鼜}
曰文本殞沒所不忍聞命撤之時右庶子許敬宗在定州與高士亷等同知機要文本薨上召敬宗以本官檢校中書侍郎壬子李世勣江夏王道宗攻高麗蓋牟城|{
	蓋牟城在遼東城東}


|{
	北唐取之以其地為蓋州大元遼陽府路有蓋州遼海軍節度領建安陽地熊岳秀岩四縣}
丁巳車駕至北平|{
	此古北平也舊志平州隋為北平郡}
癸亥李世勣等拔蓋牟城獲二萬餘口糧十餘萬石張亮帥舟師自東萊渡海襲卑沙城|{
	帥讀曰率}
其城四面懸絶惟西門可上|{
	上時掌翻}
程名振引兵夜至副總管王大度先登五月己巳拔之獲男女八千口分遣總管丘孝忠等曜兵於鴨緑水|{
	杜佑曰鴨緑水在平壤城西北四百五十里源出靺鞨長白山漢書謂之馬訾水今謂之混同江李心傳曰鴨緑水發源契丹東北長白山鴨緑水之源蓋古肅慎氏之地今女真居之}
李世勣進至遼東城下庚午車駕至遼澤泥淖二百餘里|{
	淖奴教翻}
人馬不可通將作大匠閻立德布土作橋軍不留行壬申度澤東乙亥高麗步騎四萬救遼東江夏王道宗將四千騎逆擊之|{
	騎奇寄翻下同將即亮翻下同}
軍中皆以為衆寡懸絶不若深溝高壘以俟車駕之至道宗曰賊恃衆有輕我心遠來疲頓擊之必敗且吾屬為前軍當清道以待乘輿乃更以賊遺君父乎|{
	不以賊遺君父漢耿弇之言乘䋲證翻遺于季翻}
李世勣以為然果毅都尉馬文舉曰不遇勍敵何以顯壯士策馬趨敵所向皆靡|{
	勍渠京翻趨七喻翻}
衆心稍安既合戰行軍總管張君乂退走唐兵不利道宗收散卒登高而望見高麗陳亂|{
	陳讀曰陣}
與驍騎數十衝之|{
	驍堅堯翻}
左右出入李世勣引兵助之高麗大敗斬首千餘級丁丑車駕度遼水撤橋以堅士卒之心軍於馬首山勞賜江夏王道宗超拜馬文舉中郎將斬張君乂|{
	有功必賞退懦必誅則將士知所懲勸矣勞力到翻}
上自將數百騎至遼東城下見士卒負土填塹|{
	塹七艷翻}
上分其尤重者於馬上持之從官爭負土致城下|{
	從才用翻}
李世勣攻遼東城晝夜不息旬有二日上引精兵會之圍其城數百重|{
	重直龍翻}
鼓譟聲震天地甲申南風急上遣銳卒登衝竿之末爇其西南樓|{
	爇如劣翻}
火延燒城中因麾將士登城高麗力戰不能敵遂克之所殺萬餘人得勝兵萬餘人男女四萬口|{
	勝音升}
以其城為遼州|{
	今大元遼陽府}
乙未進軍白巖城丙申右衛大將軍李思摩中弩矢上親為之吮血將士聞之莫不感動|{
	中竹仲翻為于偽翻}
烏骨城遣兵萬餘為白巖聲援|{
	自登州東北海行至烏湖島又行五百里東傍海壖過青泥浦桃花浦杏人浦石人江槖駞灣乃至烏骨江}
將軍契苾何力以勁騎八百擊之|{
	契欺訖翻苾毗必翻}
何力挺身陷陳槊中其腰|{
	陳讀曰陣中竹仲翻}
尚輦奉御薛萬備單騎往救之拔何力於萬衆之中而還|{
	還從宣翻又如字}
何力氣益憤束瘡而戰從騎奮擊|{
	從才用翻}
遂破高麗兵追奔數十里斬首千餘級會暝而罷|{
	暝莫定翻}
萬備萬徹之弟也

資治通鑑卷一百九十七
















































































































































