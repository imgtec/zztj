<!DOCTYPE html PUBLIC "-//W3C//DTD XHTML 1.0 Transitional//EN" "http://www.w3.org/TR/xhtml1/DTD/xhtml1-transitional.dtd">
<html xmlns="http://www.w3.org/1999/xhtml">
<head>
<meta http-equiv="Content-Type" content="text/html; charset=utf-8" />
<meta http-equiv="X-UA-Compatible" content="IE=Edge,chrome=1">
<title>資治通鑒_131-資治通鑑卷一百三十_131-資治通鑑卷一百三十</title>
<meta name="Keywords" content="資治通鑒_131-資治通鑑卷一百三十_131-資治通鑑卷一百三十">
<meta name="Description" content="資治通鑒_131-資治通鑑卷一百三十_131-資治通鑑卷一百三十">
<meta http-equiv="Cache-Control" content="no-transform" />
<meta http-equiv="Cache-Control" content="no-siteapp" />
<link href="/img/style.css" rel="stylesheet" type="text/css" />
<script src="/img/m.js?2020"></script> 
</head>
<body>
 <div class="ClassNavi">
<a  href="/24shi/">二十四史</a> | <a href="/SiKuQuanShu/">四库全书</a> | <a href="http://www.guoxuedashi.com/gjtsjc/"><font  color="#FF0000">古今图书集成</font></a> | <a href="/renwu/">历史人物</a> | <a href="/ShuoWenJieZi/"><font  color="#FF0000">说文解字</a></font> | <a href="/chengyu/">成语词典</a> | <a  target="_blank"  href="http://www.guoxuedashi.com/jgwhj/"><font  color="#FF0000">甲骨文合集</font></a> | <a href="/yzjwjc/"><font  color="#FF0000">殷周金文集成</font></a> | <a href="/xiangxingzi/"><font color="#0000FF">象形字典</font></a> | <a href="/13jing/"><font  color="#FF0000">十三经索引</font></a> | <a href="/zixing/"><font  color="#FF0000">字体转换器</font></a> | <a href="/zidian/xz/"><font color="#0000FF">篆书识别</font></a> | <a href="/jinfanyi/">近义反义词</a> | <a href="/duilian/">对联大全</a> | <a href="/jiapu/"><font  color="#0000FF">家谱族谱查询</font></a> | <a href="http://www.guoxuemi.com/hafo/" target="_blank" ><font color="#FF0000">哈佛古籍</font></a> 
</div>

 <!-- 头部导航开始 -->
<div class="w1180 head clearfix">
  <div class="head_logo l"><a title="国学大师官网" href="http://www.guoxuedashi.com" target="_blank"></a></div>
  <div class="head_sr l">
  <div id="head1">
  
  <a href="http://www.guoxuedashi.com/zidian/bujian/" target="_blank" ><img src="http://www.guoxuedashi.com/img/top1.gif" width="88" height="60" border="0" title="部件查字,支持20万汉字"></a>


<a href="http://www.guoxuedashi.com/help/yingpan.php" target="_blank"><img src="http://www.guoxuedashi.com/img/top230.gif" width="600" height="62" border="0" ></a>


  </div>
  <div id="head3"><a href="javascript:" onClick="javascript:window.external.AddFavorite(window.location.href,document.title);">添加收藏</a>
  <br><a href="/help/setie.php">搜索引擎</a>
  <br><a href="/help/zanzhu.php">赞助本站</a></div>
  <div id="head2">
 <a href="http://www.guoxuemi.com/" target="_blank"><img src="http://www.guoxuedashi.com/img/guoxuemi.gif" width="95" height="62" border="0" style="margin-left:2px;" title="国学迷"></a>
  

  </div>
</div>
  <div class="clear"></div>
  <div class="head_nav">
  <p><a href="/">首页</a> | <a href="/ShuKu/">国学书库</a> | <a href="/guji/">影印古籍</a> | <a href="/shici/">诗词宝典</a> | <a   href="/SiKuQuanShu/gxjx.php">精选</a> <b>|</b> <a href="/zidian/">汉语字典</a> | <a href="/hydcd/">汉语词典</a> | <a href="http://www.guoxuedashi.com/zidian/bujian/"><font  color="#CC0066">部件查字</font></a> | <a href="http://www.sfds.cn/"><font  color="#CC0066">书法大师</font></a> | <a href="/jgwhj/">甲骨文</a> <b>|</b> <a href="/b/4/"><font  color="#CC0066">解密</font></a> | <a href="/renwu/">历史人物</a> | <a href="/diangu/">历史典故</a> | <a href="/xingshi/">姓氏</a> | <a href="/minzu/">民族</a> <b>|</b> <a href="/mz/"><font  color="#CC0066">世界名著</font></a> | <a href="/download/">软件下载</a>
</p>
<p><a href="/b/"><font  color="#CC0066">历史</font></a> | <a href="http://skqs.guoxuedashi.com/" target="_blank">四库全书</a> |  <a href="http://www.guoxuedashi.com/search/" target="_blank"><font  color="#CC0066">全文检索</font></a> | <a href="http://www.guoxuedashi.com/shumu/">古籍书目</a> | <a   href="/24shi/">正史</a> <b>|</b> <a href="/chengyu/">成语词典</a> | <a href="/kangxi/" title="康熙字典">康熙字典</a> | <a href="/ShuoWenJieZi/">说文解字</a> | <a href="/zixing/yanbian/">字形演变</a> | <a href="/yzjwjc/">金 文</a> <b>|</b>  <a href="/shijian/nian-hao/">年号</a> | <a href="/diming/">历史地名</a> | <a href="/shijian/">历史事件</a> | <a href="/guanzhi/">官职</a> | <a href="/lishi/">知识</a> <b>|</b> <a href="/zhongyi/">中医中药</a> | <a href="http://www.guoxuedashi.com/forum/">留言反馈</a>
</p>
  </div>
</div>
<!-- 头部导航END --> 
<!-- 内容区开始 --> 
<div class="w1180 clearfix">
  <div class="info l">
   
<div class="clearfix" style="background:#f5faff;">
<script src='http://www.guoxuedashi.com/img/headersou.js'></script>

</div>
  <div class="info_tree"><a href="http://www.guoxuedashi.com">首页</a> > <a href="/SiKuQuanShu/fanti/">四库全书</a>
 > <h1>资治通鉴</h1> <!--         下载:【右键另存为】即可 --></div>
  <div class="info_content zj clearfix">
  
<div class="info_txt clearfix" id="show">
<center style="font-size:24px;">131-資治通鑑卷一百三十</center>
    資治通鑑卷一百三十  宋 司馬光 撰<br />
<br />
  胡三省 音註<br />
<br />
  宋紀十二【旃蒙大荒落一年】<br />
<br />
  太宗明皇帝上之上【諱彧字休景小字榮期文帝第十一子也】<br />
<br />
  泰始元年春正月乙未朔廢帝改元永光大赦【是歲八月殺江夏王義恭柳元景顔師伯改元景和既弑廢帝改元泰始一年凡三改元】 丙申魏大赦二月丁丑魏主如樓煩宫【樓煩縣漢屬雁門郡魏晉棄之荒外魏收地形志雁門郡原平縣有樓煩城賢曰樓煩故城在今代州崞縣東北余按唐書憲州古樓煩地也】 自孝建以來民間盗鑄濫錢【事始一百二十八卷孝武帝孝建二年】商貨不行庚寅更鑄二銖錢【更工衡翻】形式轉細官錢每出民間即模效之而更薄小無輪郭不磨鑢謂之耒子【鑢良倨翻錯也耒盧對翻杜佑通典耒子作來子】 三月乙巳魏主還平城 夏五月癸卯魏高祖殂【年二十六謚曰文成皇帝】初魏世祖經營四方國頗虚耗重以内難【重直用翻難乃旦翻内難謂宗愛既弑世祖又弑南安王余】朝野楚楚【楚楚酸痛之貌朝直遥翻】高宗嗣之與時消息【消衰减也息生長也】靜以鎮之懷集中外民心復安【復扶又翻】甲辰太子弘即皇帝位【弘文成帝之長子也蕭子顯曰弘字萬民】大赦尊皇后曰皇太后顯祖【魏獻文帝廟號顯祖】時年十二侍中車騎大將軍乙渾專權【騎奇寄翻】矯詔殺尚書楊保年平陽公賈愛仁南陽公張天度于禁中侍中司徒平原王陸麗治疾於代郡温泉【魏土地記曰代城北九十里有桑乾城城西渡桑乾水去桑乾城十里有温湯療疾有驗治直之翻】乙渾使司衛監穆多侯召之【魏官有司衛監典宿衛】多侯謂麗曰渾有無君之心今宫車晏駕王德望素重【魏高宗之立麗冇功焉而又忠篤故德望重於一時】姦臣所忌宜少淹留以觀之朝廷安靜然後入未晩也【少詩照翻】麗曰安有聞君父之喪慮患而不赴者乎即馳赴平城乙渾所為多不灋麗數爭之【數所角翻】戊申渾又殺麗及穆多侯多侯壽之弟也【穆壽事魏世祖封宜都王】己酉魏以渾為太尉録尚書事東安王劉尼為司徒尚書左僕射代人和其奴為司空殿中尚書順陽公郁謀誅乙渾渾殺之【主少國疑姦臣擅命屠戮忠賢魏之不亡者幸也】 壬子魏以淮南王它為鎮西大將軍儀同三司鎮凉州魏開酒禁【魏設酒禁見一百二十八卷孝武孝建三年】 壬午加柳元景南豫州刺史加顔師伯丹陽尹 秋七月癸巳魏以太尉乙渾為丞相位居諸王上事無大小皆决於渾 廢帝幼而狷暴【狷吉掾翻】及即位始猶難太后大臣及戴法興等未敢自恣太后既殂【去年太后殂】帝年漸長【長知兩翻】欲有所為灋興輒抑制之謂帝曰官所為如此欲作營陽邪【營陽王事見一百二十卷文帝元嘉元年廢帝固狂暴戴法興此言亦足以取死】帝稍不能平所幸閹人華願兒賜與無算【華戶化翻】灋興常加裁減願兒恨之帝使願兒於外察聽風謡願兒言於帝曰道路皆言宫中有二天子灋興真天子官為贗天子【贗五晏翻考異曰宋書作應天子宋畧作鴈天子按字書贗偽物也韓愈詩曰居然見真贗書或作鴈今從宋畧】且<br />
<br />
  官居深宫與人物不接灋興與太宰顔柳共為一體【義恭録尚書事柳元景為尚書令顔師伯為僕射而事皆法興專决故云然】往來門客恒有數百【恒戶登翻】内外士庶莫不畏服灋興是孝武左右久在宫闈今與它人作一家深恐此坐席非復官有【坐徂卧翻復扶又翻又如字】帝遂詔免灋興【免者免其所居官也】遣還田里仍徙遠郡八月辛酉賜灋興死解巢尚之舍人【巢尚之自孝武時為中書通事舍人】員外散騎侍郎東海奚顯度【散悉亶翻騎奇寄翻】亦有寵於世祖常典作役課督苛虐捶扑慘毒【捶止橤翻扑普卜翻擊也】人皆苦之帝嘗戲曰顯度為百姓患比當除之【比毗寐翻】左右因唱諾即宣旨殺之尚書右僕射領衛尉卿丹陽尹顔師伯居權日久【孝武大明四年徵顔師伯於歷城自侍中遷尚書僕射居權要】驕奢淫恣為衣冠所疾帝欲親朝政【朝直遥翻】庚午以師伯為尚書左僕射解卿尹【解衛尉卿及丹陽尹】以吏部尚書王彧為右僕射分其權任師伯始懼【彧於六翻】初世祖多猜忌王公大臣重足屏息【重直龍翻屏必郢翻】莫敢妄相過從【過古禾翻】世祖殂太宰義恭等皆相賀曰今日始免横死矣【横戶孟翻】甫過山陵義恭與柳元景顔師伯等聲樂酣飲不捨書夜【酣戶甘翻】帝内不能平既殺戴法興諸大臣無不震慴【慴之涉翻】各不自安於是元景師伯密謀廢帝立義恭日夜聚謀而持疑不能决元景以其謀告沈慶之慶之與義恭素不厚又師伯常專斷朝事不與慶之參懷【斷丁亂翻朝直遥翻下同孝武遺詔令慶之參决大事見上卷上年】謂令史曰【尚書令史也】沈公爪牙耳安得預政事慶之恨之乃其事癸酉帝自帥羽林兵討義恭殺之并其四子斷絶義恭支體分裂腸胃挑取眼睛以蜜漬之謂之鬼目粽【宋人以蜜漬物曰粽盧循以益智粽遺文帝即蜜漬益智也帥讀曰率斷丁管翻睛七真翻眼珠子漬疾智翻粽子宋翻】别遣使者稱詔召柳元景以兵隨之【使疏吏翻下同】左右奔告兵刃非常【非常言異於常時也】元景知禍至入辭其母整朝服乘車應召弟車騎司馬叔仁戎服帥左右壯士欲拒命元景苦禁之既出巷軍士大至元景下車受戮容色恬然并其八子六弟及諸姪獲顔師伯於道殺之并其六子又殺廷尉劉德願改元景和文武進位二等遣使誅湘州刺史江夏世子伯禽【義恭命其世子曰伯禽是以周公自處矣】自是公卿以下皆被捶曳如奴隸矣【卑賤之人有所附屬謂之隸人之下者謂之奴被皮義翻捶止橤翻】初帝在東宫多過失世祖欲廢之而立新安王子鸞侍中袁顗盛稱太子好學有日新之美世祖乃止帝由是德之既誅羣公欲引進顗任以朝政遷為吏部尚書【顗魚豈翻好呼到翻為袁顗寵衰求出張本】與尚書左丞徐爰皆以誅義恭等功賜爵縣子徐爰便辟善事人頗涉書傳【便皮連翻傳直戀翻】自元嘉初入侍左右豫參顧問既長於附會又飾以典文故為太祖所任遇大明之世委寄尤重時殿省舊人多見誅逐唯爰巧於將迎始終無迕廢帝待之益厚【迕五故翻迎也】羣臣莫及帝每出常與沈慶之及山隂公主同輦爰亦預焉【徐爰得志於大明景和之間宜也而啓寵實在於元嘉便僻之足以惑人雖明君不能免也漢宣用恭顯而遺禍於元帝事正如此】山隂公主帝姊也適駙馬都尉何戢戢偃之子也【何偃尚之之子戢疾立翻】公主尤淫恣嘗謂帝曰妾與陛下男女雖殊俱託體先帝陛下六宫萬數而妾唯駙馬一人事太不均帝乃為公主置面首左右三十人【面取其貌美首取其髪美】進爵會稽郡長公主秩同郡王【長知兩翻】吏部郎褚淵貌美公主就帝請以自侍帝許之淵侍公主十餘日備見逼廹以死自誓乃得免淵湛之之子也【褚湛之進用於元嘉孝建之間】帝令太廟别畫祖考之像【畫讀曰畫】帝入廟指高祖像曰渠大英雄生擒數天子【謂擒桓玄慕容超姚泓也大讀曰太】指太祖像曰渠亦不惡但末年不免兒斫去頭【謂為元凶劭所弑也】指世祖像曰渠大齄鼻如何不齄立召畫工令齄之【齄壯加翻鼻上麭也柳宗元詩曰嗜酒鼻成齄】 以建安王休仁為雍州刺史【雍於用翻】湘東王彧爲南豫州刺史皆留不遣 甲戍以司徒揚州刺史豫章王子尚領尚書令以始興公沈慶之為侍中太尉慶之固辭徵青冀二州刺史王玄謨為領軍將軍 魏葬文成皇帝于金陵廟號高宗九月癸巳帝如湖熟【湖熟漢湖熟侯國屬丹楊郡晉宋為縣淳化中廢為鎮屬上元縣】戊戌還建康新安王子鸞寵於世祖【事見上卷大明五年】帝疾之辛丑遣使賜子鸞死又殺其母弟南海王子師及其母妹殷貴妃墓【殷貴妃盖生子鸞子師及一女母弟母妹謂同母弟妹也】又欲掘景寧陵太師以為不利於帝乃止初金紫光禄大夫謝莊為殷貴妃誄【誄魯水翻誄丈夫者述其功德誄婦人者述其容德也】曰贊軌堯門帝以莊比貴妃於鉤弋夫人【鉤弋事見二十二卷漢武帝大始三年】欲殺之或說帝曰死者人之所同一往之苦不足為困莊生長富貴【說輸芮翻長知兩翻謝莊弘微之子謝萬之玄孫諸謝自晉以來貴盛故云然】今繫之尚方使知天下苦劇【劇甚也言莊享天下之所甚樂而未知天下之所甚苦今繫囚則使知之】然後殺之未晚也帝從之 徐州刺史義陽王昶素為世祖所惡【昶丑兩翻昶文帝子世祖之諸弟也惡烏路翻】民間每訛言昶當反是歲訛言尤甚廢帝常謂左右曰我即大位以來遂未嘗戒嚴使人邑邑【邑邑不得志也】昶使典籖蘧法生奉表詣建康求入朝【蘧姓也春秋時衛有大夫蘧伯玉蘧其於翻朝直遥翻】帝謂灋生曰義陽與太宰謀反我正欲討之今知求還甚善又屢詰問灋生【詰去吉翻】義陽謀反何故不啓灋生懼逃還彭城帝因此用兵己酉下詔討昶内外戒嚴帝自將兵渡江【將即亮翻】命沈慶之統諸軍前驅法生至彭城昶即聚兵反移檄統内諸郡【沈約宋志以大明八年為正徐州統彭城沛郡下邳蘭陵東海東莞東安琅邪淮陽陽平濟隂比濟隂鍾離馬頭等郡】皆不受命斬昶使將佐文武悉懷異心昶知事不成棄母妻擕愛妾夜與數十騎開北門奔魏【使疏吏翻將即亮翻騎奇寄翻】昶頗涉學能屬文【屬之欲翻】魏人重之使尚公主拜侍中征南將軍駙馬都尉賜爵丹陽王【為後魏挾劉昶以伐齊張本】 吏部尚書袁顗始為帝所寵任俄而失指待遇頓衰使有司糾奏其罪白衣領職顗懼詭辭求出甲寅以顗督雍梁諸軍事雍州刺史【雍於用翻】顗舅蔡興宗謂之曰襄陽星惡何可往【興宗盖以天道言之】顗曰白刃交前不救流矢【言白刃交乎前則流矢之來不暇救禍近而急故圖出外以求賖死後患非所計也】今者之行唯願生出虎口耳且天道遼遠何必皆驗是時臨海王子頊為都督荆湘等八州諸軍事荆州刺史朝廷以興宗為子頊長史南郡太守行府州事興宗辭不行顗說興宗曰【說輸芮翻】朝廷形勢人所共見在内大臣朝不保夕舅今出居陜西為八州行事【蕭子顯曰江左大鎮莫過荆揚弘農郡陜縣周世二伯主諸侯周公主陜東召公主陜西故稱荆州為陜西】顗在襄沔地勝兵彊去江陵咫尺水陸流通【襄陽至江陵水則由漢沔陸則由長林當陽沔彌兖翻】若朝廷有事可以共立桓文之功豈比受制凶狂臨不測之禍乎今得間不去【間隙也謂得可行之隙而興宗不肯去間古莧翻】後復求出豈可得邪【復扶又翻】興宗曰吾素門平進【蔡興宗蔡廓之子蔡謨之玄孫以方嚴自處官以序遷謂之平進可也謂之素門可乎盖江左以王謝為高門其餘有才望者或以姻戚擢用或以舊思興宗此言盖亦感切其甥指其在世祖之世調護昏狂階此以見寵任寵衰則求出以避禍進退皆無所據也】與主上甚踈未容有患宫省内外人不自保會應有變若内難得弭外舋未必可量【難乃旦翻舋許覲翻量音良】汝欲在外求全我欲居中免禍各行其志不亦善乎【蔡興宗可謂先見矣】顗於是狼狽上路【狼狽者倉皇而行如恐不及之意上時掌翻】猶慮見追行至尋陽喜曰今始免矣鄧琬為晉安王子勛鎮軍長史尋陽内史行江州事【子勛以鎮軍將軍為江州刺史鎮尋陽鄧琬為長史行事】顗與之欵狎過常每清閒必盡日窮夜顗與琬人地本殊【袁顗有清望又名門也鄧琬性貪鄙又寒族也故云人地本殊】見者知其有異志矣【為顗琬起兵奉子勛張本】尋復以蔡興宗為吏部尚書【復扶又翻】 戊午解嚴帝因自白下濟江至瓜步【晉宋都建康新亭白下皆江津要地新亭在西白下在東白下盖今之龍灣也按白下城合白石壘唐武德中移江寧縣於此名白下縣】 沈慶之復啓聽民私鑄錢【慶之始議見一百二十八卷孝武帝孝建二年】由是錢貨亂敗千錢長不盈三寸大小稱此謂之鵝眼錢【稱尺證翻】劣於此者謂之綖環錢【綖與線同私箭翻】貫之以縷入水不沈【沈持林翻】隨手破碎市井不復料數【料音聊料量也料數者料其多少之數也】十萬錢不盈一掬【鄭玄曰兩手曰掬】斗米一萬商貨不行 冬十月丙寅帝還建康 帝舅東陽太守王藻尚世祖女臨川長公主公主妬譖藻於帝己卯藻下獄死【王太后晉丞相導之玄孫女藻后弟也下戶稼翻】會稽太守孔靈符所至有政績以忤犯近臣近臣譖之帝遣使鞭殺靈符并誅其二子【會工外翻忤立故翻使疏吏翻】寧朔將軍何邁瑀之子也【何瑀見一百二十八卷孝武孝建二年】尚帝姑新蔡長公主【主文帝第十女也名英媚長知兩翻】帝納主於後宫謂之謝貴嬪詐言公主薨殺宫婢送邁第殯葬行喪禮庚辰拜貴嬪為夫人【嬪毗賓翻】加鸞輅龍旂出警入蹕邁素豪俠多養死士【俠戶頰翻】謀因帝出遊廢之立晉安王子勛事泄十一月壬辰帝自將兵誅邁初沈慶之既發顔柳之謀遂自昵於帝數盡言規諫【將即亮翻昵尼質翻數所角翻】帝浸不悦慶之懼杜門不接賓客嘗遣左右范羨至吏部尚書蔡興宗所興宗使羨謂慶之曰公閉門絶客以避悠悠請託者耳如興宗非有求於公者也何為見拒慶之使羨邀興宗興宗往見慶之因說之曰主上比者所行人倫道盡【言内亂也說輸芮翻比毗至翻】率德改行無可復望【行下孟翻復扶又翻下不復同】今所忌憚唯在於公百姓喁喁【喁魚容翻喁魚口向上也以喻百姓仰望如羣魚然】所瞻賴者亦在公一人而已【瞻仰視也賴倚恃也】公威名素著天下所服今舉朝遑遑【遑遑急也朝直遥翻】人懷危怖【怖普布翻】指麾之日誰不響應如猶豫不斷【斷丁亂翻】欲坐觀成敗豈惟旦夕及禍四海重責將有所歸【言慶之自昵於廢帝今忤帝意不惟行且及禍若他人舉事必謂慶之從君於昏慶之何所逃其責】僕蒙眷異常故敢盡言願公詳思其計慶之曰僕誠知今日憂危不復自保但盡忠奉國始終以之當委任天命耳【言委之於天任命所至】加老退私門兵力頓闕【頓讀曰鈍又讀如字】雖欲為之事亦無成興宗曰當今懷謀思奮者非欲邀功賞富貴止求脱朝夕之死耳殿中將帥唯聼外間消息【將即亮翻帥所類翻】若一人唱首則俯仰可定况公統戎累朝【慶之自元嘉以來統兵歷事三世朝直遥翻】舊日部曲布在宫省受恩者多【謂宗越等】沈攸之輩皆公家子弟耳【沈攸之者慶之從父兄子也】何患不從且公門徒義附並三吳勇士殿中將軍陸攸之公之鄉人【陸攸之盖亦吳興人】今入東討賊大有鎧仗在青溪未公取其器仗以配衣麾下【鎧可亥翻衣於既翻】使陸攸之帥以前驅【帥讀曰率下同】僕在尚書中自當帥百僚案前代故事更簡賢明以奉社稷天下之事立定矣又朝廷諸所施為民間傳言公悉豫之公今不决當有先公起事者公亦不免附從之禍【此興宗所謂重責將有所歸也先悉薦翻】聞車駕屢幸貴第【貴第謂時貴之宅第也】酣醉淹留又聞屏左右【屏必郢翻】獨入閤内此萬世一時不可失也慶之曰感君至言然此大事非僕所能行事至固當抱忠以没耳【事至猶言若事果至如興宗所言當抱忠以死也】青州刺史沈文秀慶之弟子也將之鎮帥部曲出屯白下亦說慶之曰【說輸芮翻】主上狂暴如此禍亂不久而一門受其寵任萬物皆謂與之同心【盈天地之間者萬物人亦物也此萬物謂人】且若人愛憎無常猜忍特甚【若人謂廢帝】不測之禍進退難免今因此衆力圖之易於反掌【易以䜴翻】機會難值不可失也再三言之至於流涕慶之終不從文秀遂行【沈慶之從君於昏狂杜門以待死伊霍之事固非常人所能行也】及帝誅何邁量慶之必當入諫【量音良度也】先閉青溪諸橋以絶之慶之聞之果往不得進而還【還從宣翻又如字】帝乃使慶之從父兄子直閤將軍攸之賜慶之藥慶之不肯飲攸之以被揜殺之【攸之隨慶之討隨王誕有功慶之抑其賞由是恨之故果於殺從才用翻】時年八十慶之子侍中文叔欲亡恐如太宰義恭被支解【被皮義翻】謂其弟中書郎文季曰我能死爾能報【中書郎即中書侍郎左傳楚平王信費無極之譛執伍奢無極曰奢之子材若在吳必憂楚國盍以免其父召之彼仁必來不然將為患王使召之曰來吾免而父棠君尚謂其弟員曰爾適吳我將歸死吾智不若我能死爾能報伍尚至楚弁奢殺之員奔吳藉兵以報楚入郢鞭平王之墓】遂飲慶之之藥而死弟秘書郎昭明亦自經死【杜佑通典曰後漢馬融為秘書郎詣東觀典校書晉掌中外三閣經書校閲脱誤亦謂之郎中武帝分秘圖書籍為甲乙丙丁四部使祕書郎中四人各掌其一宋齊尤為美職皆為甲族起家之選居職例十日便遷齊梁末多以貴遊子弟為之無其才實】文季揮刀馳馬而去追者不敢逼遂得免【為後沈文季盡誅沈攸之親屬以報仇張本】帝詐言慶之病薨贈侍中太尉謚曰忠武公葬禮甚厚領軍將軍王玄謨數流涕諫帝以刑殺過差【數所角翻】帝大怒玄謨宿將有威名【王玄謨自元嘉末為將孝建初有破臧質平義宣之功將即亮翻】道路訛言玄謨已見誅蔡興宗嘗為東陽太守玄謨典籖包灋榮家在東陽玄謨使灋榮至興宗所興宗謂灋榮曰領軍殊當憂懼灋榮曰領軍比日殆不復食夜亦不眠【比日近日也比毗寐翻復扶又翻】恒言收已在門不保俄頃【收謂帝將遣吏兵收之也恒戶登翻】興宗曰領軍憂懼當為方略那得坐待禍至因使灋榮勸玄謨舉事玄謨使灋榮謝曰此亦未易可行【易以䜴翻】期當不泄君言右衛將軍劉道隆為帝所寵任專典禁兵興宗嘗與之俱從帝夜出【從才用翻】道隆過興宗車後興宗曰劉君比日思一閒寫【閒寫者謂欲清閒寫其所懷也】道隆解其意【解戶買翻曉也】搯興宗手【搯苦洽翻以瓜搯之也】曰蔡公勿多言【廢昏立明非常之謀也蔡興宗建非常之謀既以告沈慶之又以告王玄謨又以擿劉道隆而人不敢泄其言何也昏暴之朝人不自保時日害喪予及汝皆亡盖人心之所同然也】 壬寅立皇后路氏太皇太后弟道慶之女也 帝畏忌諸父恐其在外為患皆聚之建康拘於殿内毆捶陵曳無復人理【親親悌長人之常理廢帝悖之毆烏口翻捶止橤翻復扶又翻又如字】湘東王彧建安王休仁山陽王休佑皆肥壯帝為竹籠盛而稱之【彧於六翻盛時征翻下同稱勑陵翻稱其輕重也】以彧尤肥謂之猪王謂休仁為殺王休祐為賊王以三王年長尤惡之【長知兩翻惡烏路翻】常録以自隨不離左右【録收也攝也離力智翻】東海王禕性凡劣【劣弱也鄙也禕吁韋翻】謂之驢王桂陽王休範巴陵王休若年尚少故並得從容【少詩照翻從于容翻】嘗以木槽盛飯并雜食攪之【攪古巧翻】掘地為坑實以泥水裸彧内坑中使以口就槽食之用為歡笑【裸即果翻】前後欲殺三王以十數休仁多智數每以談笑佞諛說之故得推遷【推移也遷轉也言以談笑佞諛轉移帝意也或曰推遷延緩之意說讀曰悦】少府劉曚妾孕臨月【曚謨蓬翻將產之月曰臨月 考異曰宋書帝紀作少府劉勝始安王休仁傳作廷尉劉曚宋畧及南史帝紀皆作少府劉曚休仁傳作廷尉劉曚今從其多者】帝迎入後宫俟其生男欲立為太子彧嘗忤旨帝裸之縛其手足貫之以杖使人擔付太官【忤五故翻裸郎果翻擔都甘翻荷也】曰今日屠猪休仁笑曰猪未應死帝問其故休仁曰待皇子生殺猪取其肝肺帝怒乃解曰且付廷尉一宿釋之丁未曚妾生子名曰皇子為之大赦【為于偽翻】賜為父後者爵一級帝又以太祖世祖在兄弟數皆第三【太祖高祖第三子世祖太祖第三子】江州刺史晉安王子勛亦第三【子勛世祖第三子】故惡之【惡烏路翻】因何邁之謀使左右朱景雲送藥賜子勛死景雲至湓口停不進子勛典籖謝道遇主帥潘欣之侍書褚靈嗣聞之【諸王有侍讀掌授王經有侍書掌教王書帥所類翻】馳以告長史鄧琬泣涕請計 【考異曰子勛傳云景雲遣信使告琬宋略曰帝使道遇賫敕至潯陽琬謂道遇云云今從琬傳】琬曰身南土寒士【鄧琬南昌人起於寒素】蒙先帝殊恩以愛子見託豈得惜門戶百口期當以死報効幼主昏暴社稷危殆雖曰天子事猶獨夫【猶若也似也】今便指帥文武直造京邑【帥讀曰率造七到翻】與羣公卿士廢昏立明耳戊申琬稱子勛教令所部戒嚴子勛戎服出聽事集僚佐使潘欣之口宣旨諭之四座未對録事參軍陶亮首請效死前驅衆皆奉旨乃以亮為諮議參軍領中兵摠統軍士功曹張沈為諮議參軍統作舟艦【沈持林翻艦戶黯翻】南陽太守沈懷寶岷山太守薛常寶【岷山即漢武帝所開汶山郡也宣帝地節三年併入蜀郡劉蜀復立沈懷寶薛常寶先常為郡守因各以其官稱之守手又翻】彭澤令陳紹宗等並為將帥【將即亮翻帥所類翻】初帝使荆州録送前軍長史荆州行事張悦至湓口琬稱子勛命釋其桎梏【桎職日翻梏工沃翻】迎以所乘車以為司馬悦暢之弟也【張暢見一百二十五卷六卷文帝元嘉二十六年七年】琬悦二人共掌内外衆事遣將軍俞伯奇帥五百人斷大雷【帥讀曰率斷丁管翻】禁絶商旅及公私使命遣使上諸郡民丁【遣使詣江州部内諸郡籍民丁上之以為兵使疏吏翻上時掌翻】收斂器械旬日之内得甲士五千人出頓大雷於兩岸築壘又以巴東建平二郡太守孫冲之為諮議參軍領中兵與陶亮並統前軍移檄遠近戊午帝召諸妃主列於前彊左右使辱之【彊其兩翻】南平王鑠妃江氏不從【妃即江湛之妹鑠式灼翻】帝怒殺妃三子南平王敬猷廬陵王敬先安南侯敬淵鞭江妃一百先是民間訛言湘中出天子【先悉薦翻】帝將南巡荆湘二州以厭之【厭一涉翻師古曰塞當也】明旦欲先誅湘東王彧然後初帝既殺諸公恐羣下謀已以直閤將軍宗越譚金童太一沈攸之等有勇力引為爪牙賞賜美人金帛充牣其家【江左以直閤將軍出入省閤總領宿衛牣滿也】越等久在殿省衆所畏服皆為帝盡力【為于偽翻】帝恃之益無所顧憚恣為不道中外騷然左右宿衛之士皆有異志而畏越等不敢發時三王久幽不知所為湘東王彧主衣會稽阮佃夫内監始興王道隆【江左之制天子及諸王皆有内監内監齋監也齋内自主帥以下皆得監察之會工外翻佃音田】學官令臨淮李道兒【晉制諸王國置學官令一人】與直閤將軍柳光世及帝左右琅邪淳于文祖等謀弑帝帝以立后故假諸王閹人【閹於廉翻】彧左右錢藍生亦在中密使候帝動止先是帝遊華林園竹林堂【先悉薦翻竹林堂華林園後堂也】使宫人倮相逐【倮郎果翻】一人不從命斬之夜夢在竹林堂有女子罵曰帝悖虐不道【悖蒲妹翻】明年不及熟矣帝於宫中求得一人似所夢者斬之又夢所殺者罵曰我已訴上帝矣【通鑑不語怪而獨書此事者以明人不可妄殺而天聰明為不可欺也】於是巫覡言竹林堂有鬼是日晡時帝出華林園建安王休仁山陽王休祐會稽公主並從【覡刑狄翻晡奔謨翻從才用翻】湘東王彧獨在秘書省【祕書省藏圖書之所在禁中】不被召益憂懼【被皮義翻】帝素惡主衣吳興壽寂之見輒切齒【惡烏路翻風俗通壽姓吳王壽夢之後又有大夫壽越】阮佃夫以其謀告寂之及外監典事東陽朱幼【李延壽恩倖傳論曰若徵兵動衆大興人役擾劇遠近斷於内監之心譴辱詆訶恣於典事之口抑符緩詔姦偽非一書死為生請謁成市左臂揮金右手刋字紙為銅落筆為利染】細鎧主南彭城姜產之【晉氏渡江立南彭城郡於晉陵界鎧可亥翻 考異曰產或作彦宋書宋畧南史皆作產今從之】細鎧將晉陵王敬則【吳分吳郡無錫以西為毘陵郡晉東海王越世子名毗而東海國故食毘陵懷帝永嘉五年改為晉陵郡將即亮翻】中書舍人戴明寶寂之等聞之皆響應幼豫約勒内外使錢藍生密報休仁休祐時帝欲南廵腹心宗越等並聽出外裝束唯隊主樊僧整防華林閤【防守華林閤門也】柳光世與僧整鄉人因密邀之僧整即受命【柳氏本河東人僑居襄陽樊僧整盖亦河東人也】凡同謀十餘人阮佃夫慮力少不濟更欲招合【少詩沼翻】壽寂之曰謀廣或泄不煩多人其夕帝悉屏侍衛【屏必郢翻】與羣巫及綵女數百人【綵女倣後漢采女之制】射鬼於竹林堂【射而亦翻下射之同】事畢將奏樂壽寂之抽刀前入姜產之次之淳于文祖等皆隨其後休仁聞行聲甚急謂休祐曰事作矣相隨奔景陽山【文帝元嘉二十三年起景陽山於華林園】帝見寂之至引弓射之不中綵女皆迸走帝亦走大呼寂寂者三【中竹仲翻迸北孟翻呼火故翻】寂之追而弑之【年十七】宣令宿衛曰湘東王受太皇太后令除狂主今已平定殿省惶惑未知所為休仁就祕書省見湘東王即稱臣引升西堂登御座召見諸大臣【見賢遍翻】于時事起倉猝王失履跣至西堂猶著烏帽【著陟畧翻】坐定休仁呼主衣以白帽代之【江南天子宴居著白紗㡌】令備羽儀雖未即位凡事悉稱令書施行宣太皇太后令數廢帝罪惡【數所具翻】命湘東王纂承皇極及明宗越等始入湘東王撫接甚厚廢帝母弟司徒揚州刺史豫章王子尚頑悖有兄風【悖蒲内翻又蒲没翻下同】己未湘東王以太皇太后令賜子尚及會稽公主死【會工外翻】建安王休仁等始得出居外舍【外舍外第也】釋謝莊之囚【謝莊以誄殷貴妃被囚事見上】廢帝猶横尸太醫閤口蔡興宗謂尚書右僕射王彧曰此雖凶悖要是天下之主宜使喪禮粗足若直如此四海必將乘人【言乘此以奉辭伐罪王彧湘東王妃兄也故蔡興宗與之言粗坐五翻】乃葬之秣陵縣南【葬於秣陵縣南郊壇西】初湘東王母沈婕妤早卒【婕妤音接予卒子恤翻】路太后養之王事太后甚謹太后愛王亦篤王既弑廢帝欲慰太后心下令以太后弟子休之為黄門侍郎茂之為中書侍郎論功行賞壽寂之等十四人皆封縣侯縣子十二月庚申朔以東海王禕為中書監太尉【禕於湘東王兄也禕吁韋翻】進鎮軍將軍江州刺史晉安王子勛為車騎將軍開府儀同三司【騎奇寄翻】癸亥以建安王休仁為司徒尚書令揚州刺史以山陽王休祐為荆州刺史桂陽王休範為南徐州刺史 乙丑徙安陸王子綏為江夏王 丙寅湘東王即皇帝位大赦改元【至此始改元奉始】其廢帝時昏制謬封並加刋削庚午以右衛將軍劉道隆為中護軍道隆暱於廢帝常無禮於建安太妃【此廢帝使左右辱妃主時事暱尼質翻】至是建安王休仁求解職明帝乃賜道隆死【書明帝者因前史成文】宗越譚金童太一等雖為上所撫接内不自安上亦不欲使居中從容謂之曰【從于容翻】卿等遭罹暴朝【朝直遥翻】勤勞日久應得自養之地兵馬大郡隨卿等所擇越等素已自疑聞之皆相顧失色因謀作亂以告沈攸之攸之以聞上收越等下獄死攸之復入直閤【沈攸之繼此有平尋陽之功遂總戎北討歷居方面之任下戶嫁翻復扶又翻】 辛未徙臨賀王子產為南平王晉熙王子輿為廬陵王 壬申以尚書右僕射王景文為尚書僕射景文即彧也避上名以字行 乙亥追尊沈太妃曰宣太后【即上母沈婕妤也】陵曰崇寧 初豫州刺史山陽王休祐入朝以長史南梁郡太守殷琰行府州事【晉孝武太元中僑立南梁郡於淮南安帝義熙中土斷始有淮南故地屬南豫州五代志淮南郡壽春縣舊有南梁郡朝直遥翻】及休祐徙荆州即以琰為督豫司二州諸軍事豫州刺史【為殷琰舉州以附尋陽張本】 有司奏路太后宜即前號移居外宮上不許戊寅尊路太后為崇憲皇太后居崇憲宫供奉禮儀不異舊日立妃王氏為皇后后景文之妹也 罷二銖錢禁鵝眼綖環錢【二銖鵝眼綖環並見卷首綖與線同】餘皆通用 江州佐吏得上所下令書皆喜共造鄧琬曰暴亂既除殿下又開黄閤【造七到翻時加子勛開府儀同三司故云開黄閤】實為公私大慶琬以晉安王子勛次第居三又以尋陽起事與世祖同符【世祖於兄弟次第三以尋陽起兵事見一百二十七卷文帝元嘉三十年】謂事必有成取令書投地曰殿下當開端門【天子開端門宫門正南門曰端門】黄閤是吾徒事耳衆皆駭愕琬更與陶亮等繕治器甲徵兵四方【治直之翻】袁顗既至襄陽即與諮議參軍劉胡繕修兵械簡集士卒詐稱被太皇太后令使其起兵【顗魚豈翻被皮義翻】即建牙馳檄奉表勸子勛即大位辛巳更以山陽王休祐為江州刺史【欲以代子勛更工衡翻】荆州刺史臨海王子頊即留本任【休祐不之荆州留子頊本任以安之】先是廢帝以邵陵王子元為湘州刺史【先悉薦翻】中兵參軍沈仲玉為道路行事【未至州使為道路行事沿途之事一以委之】至鵲頭聞尋陽兵起不敢進琬遣數百人刼迎之令子勛建牙於桑尾【桑尾即桑落洲尾】傳檄建康稱孤志遵前典黜幽陟明【子勛自稱曰孤黜幽陟明即廢昏立明】又謂上矯立明荗【明荗謂明德荗親謂上矯太皇太后令賜豫章王子尚死也】簒竊大寶干我昭穆【禮父為昭子為穆昭時招翻】寡我兄弟藐孤同氣猶有十三【左傳晉獻公曰以是藐諸孤藐猶小也孝武帝二十八子時存者子勛子綏子房子頊子仁子真子元子輿子孟子嗣子趨子期子 凡十三人藐妙小翻又亡角翻】聖靈何辜而當乏饗【聖靈謂世祖之靈也乏饗不祀也】郢州刺史安陸王子綏承子勛初檄欲攻廢帝【言初檄者以别今此傳建康之檄】聞廢帝已隕即解甲下標【初起兵立標以募兵罷兵故下標】既而聞江雍猶治兵【江謂鄧琬雍謂袁顗雍於用翻治直之翻】郢府行事苟卞之大懼【郢州居江雍之間懼其夾攻以問罷兵之由】即遣諮議領中兵參軍鄭景玄帥衆馳下并送軍糧【帥讀曰率】荆州行事孔道存奉刺史臨海王子頊會稽將佐奉太守尋陽王子房皆舉兵以應子勛【會工外翻將即亮翻】<br />
<br />
  資治通鑑卷一百三十<br />
<br />
<史部,編年類,資治通鑑>  <br>
   </div> 

<script src="/search/ajaxskft.js"> </script>
 <div class="clear"></div>
<br>
<br>
 <!-- a.d-->

 <!--
<div class="info_share">
</div> 
-->
 <!--info_share--></div>   <!-- end info_content-->
  </div> <!-- end l-->

<div class="r">   <!--r-->



<div class="sidebar"  style="margin-bottom:2px;">

 
<div class="sidebar_title">工具类大全</div>
<div class="sidebar_info">
<strong><a href="http://www.guoxuedashi.com/lsditu/" target="_blank">历史地图</a></strong>  
<a href="http://www.880114.com/" target="_blank">英语宝典</a>  
<a href="http://www.guoxuedashi.com/13jing/" target="_blank">十三经检索</a> 
<br><strong><a href="http://www.guoxuedashi.com/gjtsjc/" target="_blank">古今图书集成</a></strong> 
<a href="http://www.guoxuedashi.com/duilian/" target="_blank">对联大全</a> <strong><a href="http://www.guoxuedashi.com/xiangxingzi/" target="_blank">象形文字典</a></strong> 

<br><a href="http://www.guoxuedashi.com/zixing/yanbian/">字形演变</a>  <strong><a href="http://www.guoxuemi.com/hafo/" target="_blank">哈佛燕京中文善本特藏</a></strong>
<br><strong><a href="http://www.guoxuedashi.com/csfz/" target="_blank">丛书&方志检索器</a></strong> <a href="http://www.guoxuedashi.com/yqjyy/" target="_blank">一切经音义</a>  

<br><strong><a href="http://www.guoxuedashi.com/jiapu/" target="_blank">家谱族谱查询</a></strong>  <strong><a href="http://shufa.guoxuedashi.com/sfzitie/" target="_blank">书法字帖欣赏</a></strong> 
<br>

</div>
</div>


<div class="sidebar" style="margin-bottom:0px;">

<font style="font-size:22px;line-height:32px">QQ交流群9:489193090</font>


<div class="sidebar_title">手机APP 扫描或点击</div>
<div class="sidebar_info">
<table>
<tr>
	<td width=160><a href="http://m.guoxuedashi.com/app/" target="_blank"><img src="/img/gxds-sj.png" width="140"  border="0" alt="国学大师手机版"></a></td>
	<td>
<a href="http://www.guoxuedashi.com/download/" target="_blank">app软件下载专区</a><br>
<a href="http://www.guoxuedashi.com/download/gxds.php" target="_blank">《国学大师》下载</a><br>
<a href="http://www.guoxuedashi.com/download/kxzd.php" target="_blank">《汉字宝典》下载</a><br>
<a href="http://www.guoxuedashi.com/download/scqbd.php" target="_blank">《诗词曲宝典》下载</a><br>
<a href="http://www.guoxuedashi.com/SiKuQuanShu/skqs.php" target="_blank">《四库全书》下载</a><br>
</td>
</tr>
</table>

</div>
</div>


<div class="sidebar2">
<center>


</center>
</div>

<div class="sidebar"  style="margin-bottom:2px;">
<div class="sidebar_title">网站使用教程</div>
<div class="sidebar_info">
<a href="http://www.guoxuedashi.com/help/gjsearch.php" target="_blank">如何在国学大师网下载古籍?</a><br>
<a href="http://www.guoxuedashi.com/zidian/bujian/bjjc.php" target="_blank">如何使用部件查字法快速查字?</a><br>
<a href="http://www.guoxuedashi.com/search/sjc.php" target="_blank">如何在指定的书籍中全文检索?</a><br>
<a href="http://www.guoxuedashi.com/search/skjc.php" target="_blank">如何找到一句话在《四库全书》哪一页?</a><br>
</div>
</div>


<div class="sidebar">
<div class="sidebar_title">热门书籍</div>
<div class="sidebar_info">
<a href="/so.php?sokey=%E8%B5%84%E6%B2%BB%E9%80%9A%E9%89%B4&kt=1">资治通鉴</a> <a href="/24shi/"><strong>二十四史</strong></a>&nbsp; <a href="/a2694/">野史</a>&nbsp; <a href="/SiKuQuanShu/"><strong>四库全书</strong></a>&nbsp;<a href="http://www.guoxuedashi.com/SiKuQuanShu/fanti/">繁体</a>
<br><a href="/so.php?sokey=%E7%BA%A2%E6%A5%BC%E6%A2%A6&kt=1">红楼梦</a> <a href="/a/1858x/">三国演义</a> <a href="/a/1038k/">水浒传</a> <a href="/a/1046t/">西游记</a> <a href="/a/1914o/">封神演义</a>
<br>
<a href="http://www.guoxuedashi.com/so.php?sokeygx=%E4%B8%87%E6%9C%89%E6%96%87%E5%BA%93&submit=&kt=1">万有文库</a> <a href="/a/780t/">古文观止</a> <a href="/a/1024l/">文心雕龙</a> <a href="/a/1704n/">全唐诗</a> <a href="/a/1705h/">全宋词</a>
<br><a href="http://www.guoxuedashi.com/so.php?sokeygx=%E7%99%BE%E8%A1%B2%E6%9C%AC%E4%BA%8C%E5%8D%81%E5%9B%9B%E5%8F%B2&submit=&kt=1"><strong>百衲本二十四史</strong></a>  <a href="http://www.guoxuedashi.com/so.php?sokeygx=%E5%8F%A4%E4%BB%8A%E5%9B%BE%E4%B9%A6%E9%9B%86%E6%88%90&submit=&kt=1"><strong>古今图书集成</strong></a>
<br>

<a href="http://www.guoxuedashi.com/so.php?sokeygx=%E4%B8%9B%E4%B9%A6%E9%9B%86%E6%88%90&submit=&kt=1">丛书集成</a> 
<a href="http://www.guoxuedashi.com/so.php?sokeygx=%E5%9B%9B%E9%83%A8%E4%B8%9B%E5%88%8A&submit=&kt=1"><strong>四部丛刊</strong></a>  
<a href="http://www.guoxuedashi.com/so.php?sokeygx=%E8%AF%B4%E6%96%87%E8%A7%A3%E5%AD%97&submit=&kt=1">說文解字</a> <a href="http://www.guoxuedashi.com/so.php?sokeygx=%E5%85%A8%E4%B8%8A%E5%8F%A4&submit=&kt=1">三国六朝文</a>
<br><a href="http://www.guoxuedashi.com/so.php?sokeytm=%E6%97%A5%E6%9C%AC%E5%86%85%E9%98%81%E6%96%87%E5%BA%93&submit=&kt=1"><strong>日本内阁文库</strong></a> <a href="http://www.guoxuedashi.com/so.php?sokeytm=%E5%9B%BD%E5%9B%BE%E6%96%B9%E5%BF%97%E5%90%88%E9%9B%86&ka=100&submit=">国图方志合集</a> <a href="http://www.guoxuedashi.com/so.php?sokeytm=%E5%90%84%E5%9C%B0%E6%96%B9%E5%BF%97&submit=&kt=1"><strong>各地方志</strong></a>

</div>
</div>


<div class="sidebar2">
<center>

</center>
</div>
<div class="sidebar greenbar">
<div class="sidebar_title green">四库全书</div>
<div class="sidebar_info">

《四库全书》是中国古代最大的丛书,编撰于乾隆年间,由纪昀等360多位高官、学者编撰,3800多人抄写,费时十三年编成。丛书分经、史、子、集四部,故名四库。共有3500多种书,7.9万卷,3.6万册,约8亿字,基本上囊括了古代所有图书,故称“全书”。<a href="http://www.guoxuedashi.com/SiKuQuanShu/">详细>>
</a>

</div> 
</div>

</div>  <!--end r-->

</div>
<!-- 内容区END --> 

<!-- 页脚开始 -->
<div class="shh">

</div>

<div class="w1180" style="margin-top:8px;">
<center><script src="http://www.guoxuedashi.com/img/plus.php?id=3"></script></center>
</div>
<div class="w1180 foot">
<a href="/b/thanks.php">特别致谢</a> | <a href="javascript:window.external.AddFavorite(document.location.href,document.title);">收藏本站</a> | <a href="#">欢迎投稿</a> | <a href="http://www.guoxuedashi.com/forum/">意见建议</a> | <a href="http://www.guoxuemi.com/">国学迷</a> | <a href="http://www.shuowen.net/">说文网</a><script language="javascript" type="text/javascript" src="https://js.users.51.la/17753172.js"></script><br />
  Copyright &copy; 国学大师 古典图书集成 All Rights Reserved.<br>
  
  <span style="font-size:14px">免责声明:本站非营利性站点,以方便网友为主,仅供学习研究。<br>内容由热心网友提供和网上收集,不保留版权。若侵犯了您的权益,来信即刪。scp168@qq.com</span>
  <br />
ICP证:<a href="http://www.beian.miit.gov.cn/" target="_blank">鲁ICP备19060063号</a></div>
<!-- 页脚END --> 
<script src="http://www.guoxuedashi.com/img/plus.php?id=22"></script>
<script src="http://www.guoxuedashi.com/img/tongji.js"></script>

</body>
</html>
