\chapter{資治通鑑卷二百八十二}
宋 司馬光 撰

胡三省 音註

後晉紀三|{
	起屠維太淵獻盡重光赤奮若凡三年}


高祖聖文章武明德孝皇帝中

天福四年春正月辛亥以澶州防禦使太原張從恩為樞密副使|{
	澶時連翻}
朔方節度使張希崇卒羌胡寇鈔無復畏憚|{
	鈔楚交翻復扶又翻}
甲寅以義成節度使馮暉為朔方節度使党項酋長拓拔彦超最為彊大|{
	酋慈秋翻長知兩翻}
暉至彦超入賀|{
	自其部落入靈州城以賀}
暉厚遇之因為于城中治第|{
	為于偽翻下主為同治直之翻}
豐其服玩留之不遣封内遂安|{
	質彦超於城中則党項諸部不敢鈔暴于外故安}
唐羣臣江王知證等累表請唐主復姓李立唐宗廟乙丑唐主許之羣臣又請上尊號|{
	上時掌翻}
唐主曰尊號虚美且非古遂不受其後子孫皆踵其法不受尊號又不以外戚輔政宦者不得預事皆他國所不及也二月乙亥改太祖廟號曰義祖|{
	唐主初受禪尊徐温為太祖今復姓李以温為義父故改廟號為義祖}
己卯唐主為李氏考妣發哀與皇后斬衰居廬如初喪禮朝夕臨凡五十四日|{
	衰倉回翻臨力鴆翻初喪之禮自古無五十四日之制唐主亦是依傍漢晉以日易月之制居父喪母喪各二十七日故為五十四日}
江王知證饒王知諤請亦服斬衰不許|{
	知證知諤皆徐温子}
李建勲之妻廣德長公主假衰絰入哭盡禮如父母之喪|{
	建勲妻徐温女也勢利所在非血氣之親而親長知兩翻}
辛巳詔國事委齊王璟詳決惟軍旅以聞庚寅唐主更名昇|{
	更工衡翻昇皮變翻}
詔百官議二祚合享禮|{
	二祚徐李二姓之先也}
辛卯宋齊丘等議以義祖居七室之東唐主命居高祖於西室太宗次之義祖又次之皆為不祧之主羣臣言義祖諸侯不宜與高祖太宗同享請於太廟正殿後别建廟祀之帝曰|{
	通鑑旣帝晉此帝字與晉帝渾殽此亦因江南舊史失于更定}
吾自幼託身義祖|{
	事見二百六十卷唐昭宗乾寧二年}
曏非義祖有功於吳朕安能啓此中興之業羣臣乃不敢言唐主欲祖吳王恪或曰恪誅死|{
	吳王恪死于唐高宗朝為房遺愛所誣引非其罪也}
不若祖鄭王元懿唐主命有司考二王苖裔以吳王孫禕有功禕子峴為宰相|{
	玄宗廟信安王禕有邊功峴相肅宗峴戶典翻}
遂祖吳王云自峴五世至父榮其名率皆有司所撰 |{
	考異曰周世宗實録及薛史稱昪唐玄宗第六子永王璘苖裔江南録憲宗第八子建王恪之玄孫李昊蜀後主實録云唐嗣薛王知柔為嶺南節度使卒於官其子知誥流落江淮遂為徐温養子吳越備史并本潘氏湖州安吉人父為安吉砦將吳將李神福攻衣錦軍過湖州虜昇歸為僕隸徐温嘗過神福愛其謹厚求為假子以識云東海鯉魚飛上天昇始事神福後歸温故冒李氏以應䜟劉恕以為昪復姓附會李氏而吳越與唐人仇敵亦非實録昪少孤遭亂莫知其祖系昪曾祖超祖志乃與義祖之曾祖祖同名知其皆附會也}
唐主又以歷十九帝三百年疑十世太少有司曰三十年為世陛下生於文德已五十年矣|{
	文德唐僖宗末年之號言唐主之生至是年為五十年}
遂從之 盧損至福州|{
	盧損去年十一月奉冊使閩今乃至於福州}
閩主稱疾不見命弟繼恭主之遣其禮部員外郎鄭元弼奉繼恭表隨損入貢閩主不禮于損有士人林省鄒私謂損曰吾主不事其君不愛其親不恤其民不敬其神不睦其鄰不禮其賓|{
	賓謂盧損也}
其能久乎余將僧服而北逃會相見於上國耳|{
	時假號偏隅者以中原為上國以余觀之林省鄒亦非善士有樊若水之志而不得遂其志耳}
三月庚戌唐主追尊吳王恪為定宗孝静皇帝自曾祖以下皆追尊廟號及諡 己未詔歸德節度使劉知遠忠武節度使杜重威並加同平章事知遠自以有佐命功重威起于外戚無大功耻與之同制|{
	制麻制也黄忠有功關羽猶耻與之同列杜重威何如人劉知遠其肯與之同制乎英雄倔彊之氣大抵然也}
制下數日杜門四表辭不受帝怒謂趙瑩曰重威朕之妹夫知遠雖有功何得堅拒制命可落軍權|{
	劉知遠時總宿衛諸軍}
令歸私第瑩拜請曰陛下昔在晉陽兵不過五千為唐兵十餘萬所攻|{
	事見上卷上年}
危于朝露非知遠心如鐵石豈能成大業奈何以小過棄之竊恐此語外聞|{
	聞音問}
非所以彰人君之大度也帝意乃解命端明殿學士和凝詣知遠第諭旨知遠惶恐起受命|{
	君降心以撫其臣則臣亦自悔馴服勲舊彊悍之氣不容不爾}
靈州戍將王彦忠據懷遠城叛|{
	懷遠縣屬靈州趙珣聚米圖經曰唐懷遠鎮在靈州北約一百餘里宋時西夏彊盛即其地置興州其西九十餘里即賀蘭山}
上遣供奉官齊延祚往詔諭之彦忠降延祚殺之|{
	降戶江翻}
上怒曰朕踐阼以來未嘗失信于人彦忠已輸伏出迎延祚何得擅殺之除延祚名重杖配流議者猶以為延祚不應免死|{
	以其殺降失信繼此將無以懷遠人也}
辛酉冊回鶻可汗仁美為奉化可汗|{
	時回鶻比年遣使朝貢故冊命之按五代會要回鶻自唐會昌間為黠戞斯所破西奔居于甘州梁乾化元年遣使入貢至唐同光二年四月其本國權知可汗仁美遣使入貢命鄭績何延嗣持節冊仁美為英義可汗其年十一月仁美卒其弟狄銀嗣立遣都督安千等來朝貢狄銀卒阿咄欲立亦遣使來貢天成三年其權知可汗仁裕遣使入貢其年三月命使冊仁裕為順化可汗晉天福三年遣使朝貢四年三月又遣使來朝兼貢方物其月命衛尉卿邢德昭持節就冊為奉化可汗若據會要則仁美當作仁裕}
夏四月唐江王徐知證等請亦姓李|{
	欲改其本姓從國姓以自親}
不許 辛巳唐主祀南郊癸未大赦 梁太祖以來軍國大政天子多與崇政樞密使議|{
	梁與崇政使議唐與樞密使議崇政使即樞密使之職也}
宰相受成命行制敕講典故治文事而已|{
	治直之翻}
帝懲唐明宗之世安重誨專横|{
	專横事見唐明宗紀横戶孟翻}
故即位之初但命桑維翰兼樞密使及劉處讓為樞密使奏對多不稱旨|{
	劉處讓攘桑維翰樞密使見上卷上年稱尺證翻}
會處讓遭母喪甲申廢樞密院以印付中書院事皆委宰相分判以副使張從恩為宣徽使直學士倉部郎中司徒詡工部郎中顔衎並罷守本官|{
	鄭樵氏族畧曰帝王世紀舜為堯司徒支孫氏焉直學士樞密直學士也二人本官倉部工部也衎苦旱翻}
然勲臣近習不知大體習于故事每欲復之|{
	史言帝王命相當悉委以政事不當置樞密使以分其權}
帝以唐之大臣除名在兩京者皆貧悴|{
	李專美等除名見上卷元年悴秦醉翻}
復以李專美為贊善大夫丙戌以韓昭胤為兵部尚書馬胤孫為太子賓客房暠為右驍衛大將軍並致仕 閩主忌其叔父前建州刺史延武戶部尚書延望才名巫者林興與延武有怨託鬼神語云延武延望將為變閩主不復詰|{
	詰去吉翻}
使興帥壯士就第殺之|{
	帥讀曰率}
并其五子閩主用陳守元言作三清殿於禁中|{
	道家以上清玉清太清為三清}
以黃金數千斤鑄寶皇大帝天尊老君像晝夜作樂焚香禱祀求神丹政無大小皆林興傳寶皇命決之 戊申加楚王希範天策上將軍賜印聽開府置官屬|{
	梁間平四年已嘗加楚王殷天策上將軍今晉復以命其子希範}
辛亥唐徙吉王景遂為夀王立夀陽公景達為宣城王 乙卯唐鎮海節度使兼中書令梁懷王徐知諤卒 唐人遷讓皇之族於泰州號永寧宫防衛甚嚴|{
	泰州本楊州海陵縣吳乾貞中立制置院南唐昇元元年升為泰州 考異曰十國紀年唐人遷讓皇之族於泰州號永寧宫守衛甚嚴不敢與國人通昏姻久而男女自為匹偶江表志讓皇子及五歲遣中使拜官賜朝服即日而卒按唐烈祖受禪使讓皇居故宫稱臣上表慕仁厚之名若惡楊氏則滅之而已何必如此之迂也他書皆未之見不知紀年據何書今不取}
康化節度使兼中書令楊珙稱疾罷歸永寧宫|{
	康化軍亦吳於統内所置節鎮或南唐置之其地今無可考}
乙丑以平盧節度使兼中書令楊璉為康化節度使璉固辭請終喪|{
	終讓皇之喪也}
從之 唐主將立齊王璟為太子固辭乃以為諸道兵馬大元帥判六軍諸衛守太尉録尚書事昇揚二州牧|{
	南唐以昇州為西都揚州為東都故二州置牧}
閩判六軍諸衛建王繼嚴得士心閩主忌之六月罷其兵柄更名繼裕|{
	更工衡翻}
以弟繼鎔判六軍去諸衛字|{
	去羌呂翻}
林興詐覺流泉州望氣者言宫中有災乙未閩主徙居長春宫秋七月庚子朔日有食之 成德節度使安重榮出於行伍|{
	行戶剛翻}
性粗率|{
	粗與麤同}
恃勇驕暴每謂人曰今世天子兵彊馬壯則為之耳|{
	安重榮麤暴一夫耳使其彊梁亦何所至然其所以彊梁者亦習見當時之事遂起非望之心耳}
府廨有幡竿高數十尺嘗挾弓矢謂左右曰我能中竿上龍者必有天命一發中之|{
	廨古隘翻高居號翻中竹仲翻}
以是益自負帝之遣重榮代祕瓊也|{
	見上卷二年}
戒之曰瓊不受代當别除汝一鎮勿以力取恐爲患滋深重榮由是以帝爲怯謂人曰祕瓊匹夫耳天子尚畏之況我以將相之重士馬之衆乎每所奏請多踰分|{
	分扶問翻}
爲執政所可否|{
	可者則從之否者不從也}
意憤憤不快乃聚亡命市戰馬有飛揚之志帝知之義武節度使皇甫遇與重榮姻家甲辰徙遇爲昭義節度使|{
	鎮定接境恐其合而爲變徙令稍遠以離析之}
乙巳閩北宫火焚宫殿殆盡 戊申薛融等上所定編敇行之|{
	三年令薛融等詳定編敕今始上而行之上時掌翻}
丙辰敕先令天下公私鑄錢|{
	見上卷上年}
今私錢多用鈆錫小弱缺薄宜皆禁之專令官司自鑄 西京留守楊光遠疏中書侍郎同平章事桑維翰遷除不公及營邸肆于两都與民爭利帝不得已閏月壬申出維翰爲彰德節度使兼侍中 初義武節度使王處直子威避王都之難亡在契丹|{
	王都之難謂囚處直也見二百七十一卷梁均王龍德元年難乃旦翻}
至是義武缺帥|{
	皇甫遇徙潞故義武缺帥帥所類翻}
契丹主遣使來言請使威襲父土地如我朝之法|{
	我朝契丹自謂也朝直遥翻}
帝辭以中國之法必自刺史團練防禦序遷乃至節度使請遣威至此漸加進用契丹主怒復遣使來言曰|{
	復扶又翻}
爾自節度使爲天子亦有階級邪帝恐其滋蔓不已厚賂契丹且請以處直兄孫彰德節度使廷胤爲義武節度使以厭其意|{
	厭於涉翻又如字}
契丹怒稍解 初閩惠宗以太祖元從爲拱宸控鶴都|{
	閩王審知廟號太祖從才用翻下同}
及康宗立更募壯士二千爲腹心號宸衛都祿賜皆厚于二都或言二都怨望將作亂閩主欲分隸漳泉二州二都益怒閩主好爲長夜之飲強羣臣酒|{
	好呼到翻強其兩翻}
醉則令左右伺其過失|{
	伺相吏翻}
從弟繼隆醉失禮斬之屢以猜怒誅宗室叔父左僕射同平章事延羲陽爲狂愚以避禍閩主賜以道士服置武夷山中|{
	武夷山在建州崇安縣南三十里朱元晦武夷圖序曰武夷君之名著自漢世祀以乾魚不知果何神也今崇安有山名武夷相傳即神仙所宅峯巒巖壑秀抜奇偉清溪九曲流出其間兩崖絶壁人迹所不到處往往有枯查挿石罅間以度舟船棺柩之属柩中遺骸外列陶器尚且未壞頗疑前世道阻未通川壅未決時夷俗所居而漢祀者即其君長蓋亦避世之士生爲衆所臣服而傳以爲仙也武夷山中有道士觀閩主蓋置延羲於觀中}
尋復召還幽于私第|{
	復扶又翻}
閩主數侮拱宸控鶴軍使永泰朱文進光山連重遇|{
	數所角翻永泰縣屬福州晉分弋陽置西陽縣宋孝武大明初置光城縣梁於縣置光州後廢州置光城郡隋廢郡置光山縣仍置光州以縣屬焉九域志縣在州西六十里連重遇之先蓋與王潮兄弟同入閩連姓也左傳齊有連稱}
二人怨之會北宫火求賊不獲閩主命重遇將内外營兵掃除餘燼日役萬人士卒甚苦之又疑重遇知縱火之謀欲誅之内學士陳郯私吿重遇辛巳夜重遇入直帥二都兵焚長春宫|{
	帥讀曰率}
以攻閩主使人迎延羲于瓦礫中呼萬歲|{
	礫郎擊翻}
復召外營兵共攻閩主|{
	復扶又翻下同}
獨宸衛都拒戰閩主乃與李后如宸衛都|{
	李后李春鸞也如往也}
比明|{
	比必利翻}
亂兵焚宸衛都宸衛都戰敗餘衆千餘人奉閩主及李后出北關至梧桐嶺衆稍逃散延羲使兄子前汀州刺史繼業將兵追之及于村舍閩主素善射引弓殺數人俄而追兵雲集閩主知不免投弓謂繼業曰卿臣節安在繼業曰君無君德臣安有臣節新君叔父也舊君昆弟也孰親孰疎閩主不復言繼業與之俱還至陁莊飲以酒醉而縊之|{
	還從宣翻又如字飲于禁翻}
并李后及諸子王繼恭皆死宸衛餘衆奔吳越延羲自稱威武節度使閩國王更名曦|{
	更工衡翻曦王審知少子也}
改元永隆 |{
	考異曰十國紀年通文四年延羲自稱威武節度使改元永隆即晉天福四年也周世宗寶錄薛史唐餘錄南唐烈祖實録吳越備史及運歷圖紀年通譜皆同惟閩中啓運圖通文四年己亥閏七月延羲立明年庚子改元永隆五年甲辰被弑林仁志閩國人載延羲改年宜不差失然五代士人撰錄圖書多不憑舊文出于記憶及傳聞雖本國近事亦有抵牾者高遠敘事頗有本末余公綽雖在仁志之後然亦閩人故不敢獨從仁志所記又王曦既立若但稱節度使則不應改元及以其臣爲三公平章事按晉高祖實錄天福五年十一月甲申授閩國王延曦威武軍節度使閩國王是曦先已自稱閩國王紀年脱漏耳}
赦繫囚頒賚中外以宸衛弑閩主赴于鄰國諡閩主曰聖神英睿文明廣武應道大弘孝皇帝廟號康宗遣商人間道奉表稱藩于晉|{
	間古莧翻}
然其在國置百官皆如天子之制以太子太傅致仕李真爲司空兼中書侍郎同平章事連重遇之攻康宗也陳守元在宫中易服將逃兵人殺之|{
	陳守元蠱惑閩主者二世其死晩矣}
重遇執蔡守蒙數以賣官之罪而斬之|{
	數所具翻蔡守蒙賣官見上卷上年}
閩王曦既立遣使誅林興于泉州|{
	林興流泉州見上六月蜀本誅作追}
河决薄州|{
	薄州當作博州}
八月辛丑以馮道守司徒兼侍中壬寅詔中書知印止委上相|{
	舊制凡宰臣更日知印}
由是事無巨細悉委于道帝嘗訪以軍謀對曰征伐大事在聖心獨斷|{
	斷丁亂翻}
臣書生惟知謹守歷代成規而已帝以爲然道嘗稱疾求退帝使鄭王重貴詣第省之|{
	省悉景翻}
曰來日不出朕當親往道乃出視事當時寵遇羣臣無與爲比 己酉以吳越王元瓘爲天下兵馬元帥 黔南巡内溪州刺史彭士愁引蔣錦州蠻萬餘人寇辰澧州|{
	唐之盛時溪州屬黔中觀察唐末陞黔中觀察爲黔南節度後號武泰軍時屬蜀境廵内言在巡屬之内也蔣當作奬唐長安四年以沅州之夜郎渭溪二縣置舞州開元十三年以舞武聲相近更名鶴州二十年又更名業州大歷五年又更名奬州辰澧時屬楚黔渠今翻又其亷翻}
焚掠鎭戍遣使乞師于蜀蜀主以道遠不許九月辛未楚王希範命左靜江指揮使劉勍決勝指揮使廖匡齊帥衡山兵五千討之|{
	勍渠京翻廖力救翻今讀從力弔翻帥讀曰率}
癸未以唐許王從益爲郇國公奉唐祀從益尚幼李后養從益于宫中奉王淑妃如事母|{
	李后唐明宗曹皇后之女王淑妃明宗次妃也故后事之如母}
冬十月庚戌閩康宗所遣使者鄭元弼至大梁|{
	是年十二月閩遣鄭元弼隨盧損入貢至是逹大梁而康宗己於閩七月爲閩人所弑矣}
康宗遺執政書曰|{
	遺于季翻}
閩國一從興運久歷年華見北辰之帝座頻移|{
	言中國屢易主也}
致東海之風帆多阻|{
	言由比不修職貢}
又求用敵國禮致書往來帝怒其不遜壬子詔却其貢物及福建諸州綱運並令元弼及進奏官林恩部送速歸兵部|{
	員}
外郎李知損上言王昶僭慢宜執留使者籍没其貨乃下元弼恩獄|{
	下戶嫁翻}
吳越恭穆夫人馬氏卒夫人雄武節度使綽之女也|{
	路振九國志馬綽餘杭人少與錢鏐俱事董昌以女弟妻鏐鏐復為元瓘娶綽女按薛史梁貞明四年秦州節度使檢校太傅同平章事馬綽加檢校太尉秦州雄武軍也鏐傳又曰鏐恃崇盛分兩浙為數鎮其節制署而後奏則其國内節帥皆稟朝命也}
初武肅王鏐禁中外畜聲伎文穆王元瓘年三十餘無子夫人為之請于鏐|{
	為于偽翻}
鏐喜曰吾家祭祀汝實主之|{
	禮冢婦主先世之祭祀今馬夫人不妬忌而廣嗣續故鏐喜其有託}
乃聽元瓘納妾鹿氏生弘僔弘倧許氏生弘佐吳氏生弘俶衆妾生弘偡弘億弘偓弘仰弘信|{
	僔子損翻倧徂冬翻俶昌六翻偡仗减翻}
夫人撫視慈愛如一常置銀鹿于帳前坐諸兒于上而弄之 十一月戊子契丹遣其臣伊扎來使遂如吳越|{
	如往也使疏吏翻}
楚王希範始開天策府|{
	是年夏加天策上將軍至是始開府}
置護軍中尉領軍司馬等官以諸弟及將校為之又以幕僚拓跋恒李弘臯廖匡圖徐仲雅等十八人為學士|{
	倣唐太宗天策府文學舘立學士員路振九國志載李鐸潘起曹梲李莊徐牧彭繼英裴頏何仲舉孟玄暉劉昭禹鄧懿文李弘節蕭洙彭繼勲併拓拔恒等四人凡十八人恒戶登翻}
劉勍等進攻溪州彭士愁兵弃州走保山寨石崖四絶勍為梯棧上圍之|{
	棧士限翻上時掌翻}
廖匡齊戰死楚王希範遣弔其母其母不哭謂使者廖氏三百口受王温飽之賜舉族効死未足以報况一子乎願王無以為念王以其母為賢厚恤其家 十二月丙戌禁剙造佛寺|{
	剏與創同初亮翻前所無而今創為之者禁之}
閩王作新宫徙居之|{
	閩北宫燬于火曦改作新宫而徙居之}
是歲漢門下侍郎同平章事趙光裔言于漢主曰自馬后崩|{
	漢主娶于楚唐清泰元年馬后殂}
未嘗通使于楚親鄰舊好不可忘也|{
	劉馬通姻故曰親潭廣接境故曰鄰好呼到翻}
因薦諫議大夫李紓可以將命|{
	紓音舒}
漢主從之楚亦遣使報聘光裔相漢二十餘年府庫充實邊境無虞及卒漢主復以其子翰林學士承旨尚書左丞損為門下侍郎同平章事五年春正月帝引見閩使鄭元弼等|{
	見賢遍翻使疏吏翻}
元弼曰王昶蠻夷之君不知禮義陛下得其善言不足喜惡言不足怒臣將命無狀願伏鈇鑕以贖昶罪帝憐之辛未詔釋元弼等 |{
	考異曰洛中紀異云昶旣為朝命所責乃遣使越海聘于契丹即將籍没之物為贄晉祖方卑辭以奉戎主戎主降偽詔曰閩國禮物並付喬榮放其使人還本國晉主不敢拒之旣而昶又遣使于契丹求馬由滄濟淮甸路南去自兹往復不一時人無不憤惋昶以天福四年閏七月被弑十月元弼等至京下獄昶安得知而告契丹今不取}
楚劉勍等因大風以火箭焚彭士愁寨而攻之愁帥麾下逃入奬錦深山乙未遣其子師暠帥諸酋長納溪錦奬三州印請降于楚|{
	為彭師暠盡節于馬氏張本帥讀曰率}
二月庚戌北都留守同平章事安彦威入朝|{
	北都自後唐以來建于太原}
上曰吾所重者信與義昔契丹以義救我我今以信報之聞其徵求不已公能屈節奉之深稱朕意|{
	稱尺證翻}
對曰陛下以蒼生之故猶卑辭厚幣以事之臣何屈節之有上悦 劉勍引兵還長沙楚王希範徙溪州于便地|{
	便地者徙近楚境便於制令}
表彭士愁為溪州刺史以劉勍為錦州刺史自是羣蠻服於楚希範自謂伏波之後|{
	漢馬援為伏波將軍}
以銅五千斤鑄柱高丈二尺|{
	高居號翻}
入地六尺銘誓狀于上立之溪州|{
	今辰州會溪城西南一里有銅柱即馬希範所立也天策府學士李弘臯為之銘}
唐康化節度使兼中書令楊璉謁平陵還|{
	平陵盖楊璉之父讓皇陵也還從宣翻又如字}
一夕大醉卒於舟中|{
	唐主使然也路振九國志曰惕璉拜陵至竹篠口維舟大醉一夕而卒}
追封諡曰弘農靖王|{
	因楊氏其先受封之郡追封為弘農王諡日靖}
閩主曦旣立驕淫苛虐猜忌宗族多尋舊怨其弟建州刺史延政數以書諫之|{
	數所角翻}
曦怒復書罵之遣親吏業翹監建州軍|{
	史炤曰業當作鄴風俗通漢有梁令鄴鳳監古銜翻}
教練使杜漢崇監南鎮軍|{
	按福州西北與建州鄰閩主蓋置南鎮軍于福建二州界扼往來之要故是後王延政攻南鎮而福州西鄙戍兵皆潰}
二人爭捃延政隂事告于曦|{
	捃君運翻}
由是兄弟積相猜恨一日翹與延政議事不叶翹訶之曰公反邪延政怒欲斬翹翹奔南鎮延政發兵就攻之敗其戍兵|{
	訶虎何翻敗補邁翻下同}
翹漢崇奔福州西鄙戍兵皆潰 二月曦遣統軍使潘師逵吳行真將兵四萬撃延政師逵軍於建州城西行真軍于城南皆阻水置營焚城外廬舍延政求救于吳越壬戌吴越王元瓘遣寧國節度使同平章事仰仁詮|{
	宣州寧國軍時屬南唐吳越使仰仁詮遥領耳當時列國自相署置多此類仰姓也何氏姓苑有此}
内都監使薛萬忠將兵四萬救之丞相林鼎諫不聽三月戊辰師逵分兵三千遣都軍使蔡弘裔將之出戰延政遣其將林漢徹等敗之于茶山斬首千餘級|{
	茶山在建州東二十五里今亦謂之鳳凰山北苑茶焙即其地}
安彦威王建立皆請致仕不許辛未以歸德節度使侍衛馬步都指揮使同平章事劉知遠為鄴都留守徙彦威為歸德節度使加兼侍中癸酉徙建立為昭義節度使進爵韓王以建立遼州人割遼沁二州隸昭義|{
	遼沁二州自唐以來本屬河東節度沁午鴆翻}
徙建雄節度使李德珫為北都留守|{
	珫昌終翻守式乂翻}
山南東道節度使同平章事安從進恃其險固|{
	襄陽之地正得屈完所謂方城以為城漢水以為池之險故安從進恃之以傲朝廷}
隂蓄異謀擅邀取湖南貢物|{
	湖南貢物馬希範所進者也}
招納亡命增廣甲卒元隨都押牙王令謙押牙潘知麟諫皆殺之及王建立徙潞州帝使問之曰朕虚青州以待卿|{
	青州平盧軍}
卿有意則降制從進對曰若移青州置漢南|{
	襄陽在漢水之南}
臣即赴鎮帝不之責|{
	帝非姑息之主也慊然内顧其所以取中原者而思其所以守中原者畏首畏尾故諸鎮之桀驁者皆俛首而撫馴之}
丁丑王延政募敢死士千餘人夜涉水濳入潘師逵壘因風縱火城上鼓譟以應之戰棹都頭建安陳誨殺師逵|{
	建安漢治縣地吳置建安縣唐帶建州}
其衆皆潰戊寅引兵欲攻吳行真寨建人未涉水行真及將士棄營走死者萬人延政乘勝取永平順昌二城|{
	吳分建安置南平縣晉武帝改為延平縣王審知置延平鎮其于延翰改日永平鎮今南劒州治所即其地九域志南劒州管下有順昌縣在州西一百八十里宋白曰順昌縣本建安縣之校鄉地也吳永安三年置將樂縣隋併入邵武 唐置景福二年又置將水鎮改為永順場尋立為順昌縣}
自是建州之兵始盛 夏四月蜀太保兼門下侍郎同平章事趙季良請與門下侍郎同平章事母昭裔中書侍郎同平章事張業分判三司癸卯蜀主命季良判戶部昭裔判鹽鉄業判度支|{
	度徒洛翻}
庚戌以前橫海節度使馬全節為安遠節度使|{
	以代李全全也}
甲子吳越孝獻世子弘僔卒|{
	僔子損翻}
吳越仰仁詮等兵至建州王延政以福州兵已敗去奉牛酒犒之|{
	犒苦到翻}
請班師仁詮等不從營于城之西北延政懼|{
	見仰仁詮逼城而屯有圖建州之心是以懼}
復遣使乞師于閩王|{
	復扶又翻}
閩王以泉州刺史王繼業為行營都統將兵二萬救之且移書責吳越|{
	所謂歸曲以直責也}
遣輕兵絶吳越糧道會久雨吳越食盡五月延政遣兵出擊大破之俘斬以萬計癸未仁詮等夜遁 胡漢筠旣違詔命不詣闕又聞賈仁沼二子欲訴諸朝|{
	賈仁沼死見上卷二年朝直遥翻}
及除馬全節鎮安州代李金全漢筠紿金全曰進奏吏遣人倍道來言|{
	進奏吏謂安遠軍進奏院之主吏在大梁者也}
朝延俟公受代即按賈仁沼死狀以為必有異圖金全大懼漢筠因說金全拒命自歸于唐金全從之|{
	說式芮翻}
丙戌帝聞金全叛命馬全節以汴洛汝鄭單宋陳蔡曹濮申唐之兵討之|{
	如此則河之南濟之西諸鎮之兵盡發矣單音善濮音卜}
以保大節度使安審暉為之副審暉審琦之兄也李金全遣推官張緯奉表請降於唐|{
	降戶江翻}
唐主遣鄂州屯營使李承裕段處恭將兵三千逆之|{
	處昌呂翻}
唐主遣客省使尚全恭如閩和閩王曦及王延政六月延政遣牙將及女奴持誓書及香爐至福州與曦盟于宣陵|{
	古者盟誓坎用牲加載書于上歃血以質諸天地鬼神宗廟之祭焫蕭合馨香而已至于灌獻尚鬰食品用椒荀卿言芬若椒蘭漢皇后椒房取其芬馥郎官含雞舌香奏事西京雜記載長安巧工丁緩作被下香爐劉向銘博山爐漢官典職尚書郎給女史二人執香爐燒薰皆未以奉鬼神漢武内傳載西王母降爇嬰香多品疑皆後人傅會而言之宋范曄作香序備言諸香以譏評時人至其作後漢書亦不載漢人焚香事疑以香禮神之習出于魏晉已下程大昌演繁露曰梁武帝祭天始用沈香古未用也祀地用上和香注云以地于人近宜加雜馥即合諸香為之言不止一香也閩主璘之舉大號尊其父審知墓為宣陵}
然兄弟相猜恨猶如故 癸卯唐李承裕等至安州是夕李金全將麾下數百人詣唐軍妓妾資財皆為承裕所奪|{
	妓渠綺翻}
承裕入據安州甲辰馬全節自應山進軍大化鎮|{
	應山古應國漢屬隨縣界梁分隨縣置永陽縣隋改曰應山唐屬安州九域志在州北一百八十里大化鎮屬應山縣}
與承裕戰于城南大破之承裕掠安州南走全節入安州丙午安審暉追敗唐兵於黃花谷段處恭戰死丁未審暉又敗唐兵于雲夢澤中|{
	九域志安州安陸縣有雲夢鎮今安陸縣南五十里有雲夢澤宋白曰安州雲夢縣本漢安陸縣地後魏大統十六年於雲城古城置雲夢縣敗補邁翻}
虜承裕及其衆唐將張建崇據雲夢橋拒戰審暉乃還馬全節斬承裕及其衆千五百人于城下送監軍杜光業等五百七人于大梁上曰此曹何罪皆賜馬及器服而歸之初盧文進之奔吳也|{
	事見二百八十卷元年}
唐主命祖全恩將兵逆之戒無入安州城陳于城外|{
	陳讀曰陣}
俟文進出殿之以歸無得剽掠|{
	自文進至此皆言唐主相吳時事也殿丁練翻剽匹妙翻}
及李承裕逆李金全戒之如全恩承裕貪剽掠與晉兵戰而敗失亡四千人唐主惋恨累日自以戒敇之不熟也|{
	惋烏貫翻唐主生于兵間老于兵間軍之利鈍熟知之矣其惋恨者誠有罪已之心惜不能如秦穆公耳至馮延已輩乃訕笑先朝至于蹙國殄民而後已書曰否則侮厥父母曰昔之人無聞知延已之謂矣後之守國者尚鑒兹哉}
杜光業等至唐唐主以其違命而敗不受復送于淮北遺帝書曰邊校貪功乘便據壘|{
	復扶又翻遺唯季翻校戶教翻}
又曰軍法朝章彼此不可|{
	言律之以軍法則喪師者此所必誅盜邊者彼所不恕䋲之以朝章則兩國皆不可容之立于朝也}
帝復遣之歸使者將自桐墟濟淮|{
	九域志宿州蘄縣有桐墟鎮自桐墟而南至渦口則濟淮矣金人疆域圖桐墟在宿州臨渙縣}
唐主遣戰艦拒之乃還帝悉授唐諸將官以其士卒為顯義都命舊將劉康領之|{
	舊將蓋從起於晉陽者}


臣光曰違命者將也士卒從將之令者也又何罪乎受而戮其將以謝敵弔士卒而撫之斯可矣|{
	將即亮翻}
何必棄民以資敵國乎

唐主使宦者祭廬山|{
	廬山在江州潯陽縣山南即唐都昌縣山北即唐之潯陽縣都昌今南康軍軍城北十五里即廬山}
還勞之曰|{
	勞力到翻}
卿此行甚精潔宦者曰臣自奉詔蔬食至今唐主曰卿某處市魚為羮某日市肉為胾何為蔬食宦者慙服|{
	胾側吏翻臠肉為之唐主之察衛嗣君之儔也}
倉吏歲終獻羨餘萬餘石唐主曰出納有數苟非掊民刻軍安得羨餘邪|{
	羨延面翻掊蒲侯翻}
秋七月閩主曦城福州西郭以備建人又度民為僧民避重賦多為僧凡度萬一千人|{
	嗚呼使度僧而有福田利益則閩國至今存可也}
乙丑帝賜鄭元弼等帛遣歸|{
	遣歸閩也去年十月囚之今釋而遣之}
李金全之叛也安州馬步副都指揮使桑千威和指揮使王萬金成彦温不從而死馬步都指揮使龎守榮誚其愚以徇金全之意|{
	誚才笑翻}
己巳詔贈賈仁沼及桑千等官遣使誅守榮于安州李金全至金陵唐主待之甚薄|{
	李金全為姦將所惑背父母之國委身于他邦其見薄宜也}
丁巳唐主立齊王璟為太子兼大元帥錄尚書事 太子太師致仕范延光請歸河陽私第|{
	范延光仕唐先有私第在河陽}
帝許之延光重載而行西京留守楊光遠兼領河陽利其貨且慮為子孫之患|{
	當范延光以廣晉自歸之時楊光遠為元帥必有以陵暴之故懼其為子孫之患}
奏延光叛臣不家汴洛而就外藩恐其逃逸入敵國宜早除之帝不許光遠請敇延光居西京從之光遠使其子承貴以甲士圍其第逼令自殺|{
	嗚呼財之累人如此祕瓊以是而殺董温琪之家范延光復以是而殺祕瓊楊光遠又以是而殺范延光而光遠亦卒不免財之累人如此夫}
延光曰天子在上賜我鐵劵許以不死|{
	賜鐵劵見上卷三年}
爾父子何得如此己未承貴以白刃驅延光上馬至浮梁擠于河|{
	上時掌翻擠子細翻又子西翻}
光遠奏云自赴水死帝知其故憚光遠之彊不敢詰為延光輟朝贈太師|{
	為于偽翻}
唐齊王璟固辭太子|{
	位居嫡長則當為太子辭之非所以繋臣民之望也}
九月乙丑唐主許之詔中外致牋如太子禮 丁卯以翰林學士承旨戶部侍郎和凝為中書侍郎同平章事己巳鄴都留守劉知遠入朝|{
	是年二月劉知遠代安彦威鎮魏州}
辛未李崧奏諸州倉糧于計帳之外所餘頗多|{
	計帳謂歲計其數造帳以申三司者倉吏于受納之時斛面取贏俟出給之時而私其利此皆官吏相與為弊至今然也必般量而後知其所餘而般量之際為弊又多竊意李崧亦因時人旣言而奏之耳}
上曰法外税民罪同枉法倉吏特貸其死各痛懲之|{
	不知當時所謂痛懲者為何畢竟言之而不能行}
翰林學士李澣輕薄多酒失上惡之丙子罷翰林學士併其職于中書舍人|{
	惡烏路翻當是時樞密直學士旣罷僅有翰林學士尚為親近儒生李澣之酒失罷之是也因而罷翰林學士非也}
澣濤之弟也 楊光遠入朝帝欲徙之他鎮謂光遠曰圍魏之役卿左右皆有功尚未之賞|{
	圍魏見上卷二年三年}
今當各除一州以榮之因以其將校數人為刺史|{
	所以分楊光遠之黨而弱其勢}
甲申徙光遠為平盧節度使進爵東平王|{
	開運之初楊光遠遂以平盧叛}
冬十月丁酉加吳越王元瓘天下兵馬都元帥尚書令 壬寅唐大赦詔中外奏章無得言睿聖犯者以不敬論術士孫智永以四星聚斗分野有災|{
	分扶問翻}
勸唐主廵東都|{
	勸之東巡江都}
乙巳唐主命齊王璟監國光政副使太僕少卿陳覺以私憾奏泰州刺史禇仁規貪殘|{
	泰州漢時吳國之海陵倉地東晉分廣陵置海陵郡唐初置吳州更海陵縣為吳陵縣武德七年廢吳州復為海陵縣南唐升為泰州}
丙午罷仁規為扈駕都部署覺始用事|{
	為陳覺亂唐政張本}
庚戌唐主發金陵甲寅至江都 閩主曦因商人奉表自理|{
	言己未嘗稱大號稱大號者王昶之為也}
十一月甲申以曦為威武節度使兼中書令封閩王唐主欲遂居江都以水凍漕運不給乃還|{
	還從宣翻又如字}


十二月丙申至金陵 唐右僕射兼門下侍郎同平章事張延翰卒 是歲漢門下侍郎同平章事趙損卒以寧遠節度使南昌王定保為中書侍郎同平章事不踰年亦卒 初帝割鴈門之北以賂契丹|{
	見二百八十卷元年}
由是吐谷渾皆屬契丹苦其貪虐思歸中國成德節度使安重榮復誘之|{
	復扶又翻誘音酉}
于是吐谷渾帥部落千餘帳自五臺來奔|{
	歐陽脩曰吐谷渾本居青海唐至德中為吐蕃所攻部族分散其内附者唐處之河西唐末其首領有赫連鐸為大同節度使為晉王克用所破部族益微散處蔚州界中余按唐高宗之時吐谷渾為吐蕃所破棄青海而内徙至至德中青海不復有吐谷渾而吐藩東吞河隴吐谷渾復東徙居雲蔚之間自五臺來奔蓋取飛狐道奔鎮州也宋白曰吐谷渾謂之退渾蓋語急而然聖歷後吐蕃陷安樂州其衆東徙散在朔方赫連鐸以開成元年將本部三千帳來投豐州文宗命振武節度使劉沔以善地處之及沔移鎮河東遂散居川界音訛謂之退渾其後吐谷渾白姓皆赫連之部落赫連鐸為李克用所逐歸幽州李匡儔遂居蔚州界部落代建其氏不常白承福自莊宗後為都督依北山北石門為柵賜其額為寧朔府以都督為節度使}
契丹大怒遣使讓帝以招納叛人|{
	為契丹誚讓不已帝憂悒而殂張本}
六年春正月丙寅帝遣供奉官張澄將兵二千索吐谷渾在并鎮忻代四州山谷者逐之使還故土|{
	索山客翻吐谷渾旣仇視契丹雖逐之不去其後劉知遠遂殺之以為資}
王延政城建州周二十里請于閩王曦欲以建州為威武軍自為節度使曦以威武軍福州也乃以建州為鎮安軍以延政為節度使封富沙王|{
	建州本漢治縣地後分治地南部曰建安唐置建州州有古富沙驛又南劍州管内有富沙里}
延政改鎮安曰鎮武而稱之 二月壬辰作浮梁于德勝口|{
	是為澶州河橋}
彰義節度使張彦澤欲殺其子掌書記張式素為彦澤所厚諫止之彦澤怒射之左右素惡式從而讒之|{
	射而亦翻惡烏路翻}
式懼謝病去彦澤遣兵追之式至邠州静難節度使李周以聞帝以彦澤故流式商州彦澤遣行軍司馬鄭元昭詣闕求之且曰彦澤不得張式恐致不測|{
	是以反而脅丄也}
帝不得已與之癸未式至涇州彦澤命決口剖心斷其四支|{
	斷音短父子之道天性也張彦澤欲殺其子其於天性何有張式其所親者也以諫而殺之極其慘酷其于所親亦何有晉祖欲以君臣之分柔服之難矣此其所以貽負義侯之禍也}
涼州軍亂留後李文謙閉門自焚死|{
	趙珣聚米圖經涼州東至會州六百里西至甘州五百里南至鄯州三百六十里北至故突厥界三百里宋白續通典四至同而里數之遠近異}
蜀自建國以來|{
	唐清泰元年蜀建國}
節度使多領禁兵或以他職留成都委僚佐知留務專事聚斂政事不治|{
	歛力贍翻治直之翻}
民無所訴蜀主知其弊丙辰加衛聖馬步都指揮使武德節度使兼中書令趙廷隱|{
	蜀以東川為武德軍以定董璋克梓州取武有七德以為軍號}
樞密使武信節度使同平章事王處回捧聖控鶴都指揮便保寧節度使同平章事張公鐸檢校官並罷其節度使三月甲戌以翰林學士承旨李昊知武寧軍散騎常侍劉英圖知保寧軍諫議大夫崔鑾知武信軍給事中謝從志知武泰軍將作監張讚知寧江軍|{
	使之各知節度事非正帥也}
夏四月閩王曦以其子亞澄同平章事判六軍諸衛曦疑其弟汀州刺史延喜與延政通謀|{
	汀建接壤故疑之}
遣將軍許仁欽以兵三千如汀州執延喜以歸 唐主以陳覺及萬年常夢錫為宣徽副使辛巳北京留守李德珫遣牙校以吐谷渾酋長白承

福入朝|{
	旣遣張澄逐吐谷渾之在四州山谷者矣而又容其酋長入朝豈非容其大而逐其細歟晉高祖之與契丹主以術相遇者也珫昌中翻}
唐主遣通事舍人歐陽遇求假道以通契丹帝不許|{
	契丹求假道以通淮浙晉無所不可至唐求假道以通契丹則不許之隨其所輕重而應之也}
自黄巢犯長安以來|{
	唐僖宗廣明元年黃巢入長安}
天下血戰數十年然後諸國各有分土|{
	分扶問翻}
兵革稍息及唐主即位江淮比年豐稔兵食有餘羣臣爭言陛下中興今北方多難宜出兵恢復舊疆|{
	比毗至翻難乃旦翻舊彊謂盛唐時疆土也此豈易恢復邪宜唐主之不從之也}
唐主曰吾少長軍旅|{
	少詩照翻長知兩翻}
見兵之為民害深矣不忍復言|{
	復扶又翻}
使彼民安則吾民亦安矣又何求焉漢主遣使如唐謀共取楚分其地唐主不許|{
	史言唐主能保境息民}
山南東道節度使安從進謀反遣使奉表詣蜀請出師金商以為聲援|{
	自金商取道均房則至襄陽}
丁亥使者至成都蜀主與羣臣謀之皆曰金商險遠少出師則不足制敵多則漕輓不繼|{
	水運曰漕陸運曰輓輓音晩}
蜀主乃辭之又求援于荆南高從誨遺從進書|{
	遺唯季翻}
諭以禍福從進怒反誣奏從誨荆南行軍司馬王保義勸從誨具奏其狀且請發兵助朝廷討之從誨從之 成德節度使安重榮耻臣契丹見契丹使者必箕踞慢罵使過其境或濳遣人殺之契丹以讓帝帝為之遜謝|{
	使並疏吏翻為于偽翻}
六月戊午重榮執契丹使伊喇|{
	伊戶結翻剌來達翻}
遣騎掠幽州南境軍于博野|{
	博野縣屬定州宋雍熙四年以其地置寧邊軍景德元年改永定軍天聖七年改永寧軍金陞為蠡州其疆域圖云北至燕京四百九十里}
上表稱吐谷渾兩突厥渾契苾沙陁各帥部衆歸附|{
	兩突厥東突厥西突厥也帥讀曰率}
党項等亦遣使納契丹吿身職牒言為虜所陵暴|{
	党底朗翻}
又言自二月以來令各具精甲壯馬將以上秋南寇|{
	上秋謂七月}
恐天命不佑與之俱滅願自備十萬衆與晉共擊契丹又朔州節度副使趙崇已逐契丹節度使劉山|{
	朔州舊非節鎮蓋契丹所置也}
求歸命朝廷臣相繼以聞陛下屢敇臣承奉契丹勿自起舋端其如天道人心難以違拒機不可失時不再來諸節度使没于虜庭者|{
	此謂趙德鈞董温琪揚彦珣翟璋等}
皆延頸企踵以待王師|{
	企去智翻舉踵不至地也}
良可哀閔願早決計表數千言大抵斥帝父事契丹竭中國以媚無厭之虜|{
	厭於鹽翻}
又以此意為書遺朝貴|{
	遺唯季翻}
及移藩鎮云已勒兵必與契丹決戰帝以重榮方握彊兵不能制甚患之時鄴都留守侍衛馬步都指揮使劉知遠在大梁|{
	去年劉知遠自魏來朝時尚留大梁}
泰寧節度使桑維翰知重榮已蓄姦謀又慮朝廷重違其意|{
	重難也}
密上疏曰陛下免于晉陽之難而有天下|{
	難乃旦翻}
皆契丹之功也不可負之今重榮恃勇輕敵吐渾假手報仇皆非國家之利不可聽也臣竊觀契丹數年以來士馬精彊吞噬四隣戰必勝攻必取割中國之土地收中國之器械|{
	此謂降楊光遠虜趙德鈞時也}
其君智勇過人其臣上下輯睦牛羊蕃息|{
	蕃音煩}
國無天災此未可與為敵也且中國新敗|{
	謂張敬達晉安之敗趙德鈞圑柏之敗}
士氣彫沮以當契丹乘勝之威其勢相去甚遠又和親旣絶則當發兵守塞兵少則不足以待寇兵多則饋運無以繼之我出則彼歸我歸則彼至臣恐禁衛之士疲于奔命鎮定之地無復遺民|{
	幽涿瀛莫旣屬契丹鎮定滄景悉為邊銷滄景之地近海卑下又多塘濼虜騎不可得而入其入寇多依山而趨鎮定故其地為虜衝}
今天下粗安|{
	粗坐五翻}
瘡痍未復府庫虚竭蒸民困弊|{
	蒸衆也}
靜而守之猶懼不濟其可妄動乎契丹與國家恩義非輕信誓甚著彼無間隙|{
	間古莧翻}
而自啓舋端就使克之後患愈重萬一不克大事去矣議者以歲輸繒帛謂之耗蠧有所卑遜謂之屈辱殊不知兵連而不休禍結而不解財力將匱耗蠧孰甚焉用兵則武吏功臣過求姑息邊藩遠郡得以驕矜下陵上替屈辱孰大焉|{
	桑維翰權利害之輕重而言之一時之論也}
臣願陛下訓農習戰養兵息民俟國無内憂民有餘力然後觀舋而動則動必有成矣又鄴都富盛國家藩屛今主帥赴闕|{
	屛必郢翻主帥赴闕謂劉知遠來朝帥所類翻}
軍府無人臣竊思慢藏誨盜之言勇夫重閉之義|{
	慢藏誨盜易大傳之言勇夫重閉左傳申公巫臣之言藏徂浪翻重直龍翻}
乞陛下畧加廵幸以杜姦謀帝謂使者曰朕比日以來煩懣不決今見卿奏如醉醒矣|{
	比毗至翻懣音悶醒先挺翻醉寤也}
卿勿以為憂 閩王曦聞王延政以書招泉州刺史王繼業召繼業還賜死于郊外|{
	福州之郊外也城外三十里為郊蓋殺之野也}
殺其子于泉州初繼業為汀州刺史司徒兼門下侍郎同平章事楊沂豐為士曹參軍與之親善或告沂豐與繼業同謀沂豐方侍宴即收下獄|{
	下戶嫁翻}
明日斬之夷其族沂豐涉之從弟也|{
	楊涉為相於唐梁禪代之際從才用翻}
時年八十餘國人哀之自是宗族勲舊相繼被誅人不自保諫議大夫黃峻舁櫬詣朝堂極諫曦曰老物狂發矣貶章州司戶|{
	被皮義翻舁音余又羊茹翻櫬初覲翻章州當作漳州}
曦淫侈無度資用不給謀于國計使南安陳匡範|{
	南安縣隋置唐屬泉州九域志在州北四十七里}
匡範請日進萬金曦悦加匡範禮部侍郎匡範增筭商賈數倍曦宴羣臣舉酒屬匡範曰|{
	賈音古屬之欲翻}
明珠美玉求之可得如匡範人中之寶不可得也未幾商賈之筭不能足日進貸諸省務錢以足之恐事覺憂悸而卒|{
	幾居豈翻悸其季翻}
曦祭贈甚厚諸省務以匡範貸帖聞|{
	貸帖貸錢之文書也}
曦大怒斵棺斷其屍棄水中|{
	斷音短}
以連江人黃紹頗代為國計使|{
	唐武德元年分閩縣置温麻縣尋改曰連江屬福州九域志在州東北一百六十里}
紹頗請令欲仕者自非䕃補皆聽輸錢即授之以資望高下及州縣戶口多寡定其直自百緡至千緡從之 唐主自以專權取吳尤忌宰相權重|{
	此王莽隋文帝之故智也奸雄事成與不成有幸不幸耳}
以右僕射兼中書侍郎同平章事李建勲執政歲久欲罷之會建勲上疏言事意其留中旣而唐主下有司施行|{
	下音戶嫁翻}
建勲自知事挾愛憎密取所奏改之秋七月戊辰罷建勲歸私第 帝憂安重榮跋扈己巳以劉知遠為北京留守河東節度使復以遼沁隸河東|{
	去年以遼沁隸昭義軍沁午鴆翻}
以北京留守李德珫為鄴都留守知遠微時為晉陽李氏贅婿嘗牧馬犯僧田僧執而笞之知遠至晉陽首召其僧命之坐慰諭贈遺衆心大悦|{
	遺唯季翻不念惡怨故衆心大悦為劉知遠自河東成大業張本}
吳越府署火宫室府庫幾盡|{
	幾居依翻}
吳越王元瓘驚懼發狂疾唐人爭勸唐主乘弊取之唐主曰奈何利人之災遣使唁之且賙其乏|{
	唁魚戰翻弔生曰唁賙音周振贍之也}
閩主曦自稱大閩皇領威武節度使|{
	旣稱皇矣又領威武節度使古之私立名字者無此比也}
與王延政治兵相攻|{
	治直之翻}
互有勝負福建之間暴骨如莽鎮武節度判官晉江潘承祐屢請息兵脩好|{
	唐開元八年分南安縣置晉江縣後遂為泉州治所好呼到翻}
延政不從閩主使者至延政大陳甲卒以示之對使者語甚悖慢|{
	悖蒲妹翻又蒲没翻}
承祐長跪切諫延政怒顧左右曰判官之肉可食乎承祐不顧聲色愈厲閩主曦惡泉州刺史王繼嚴得衆心罷歸酖殺之|{
	惡烏路翻}
八月戊子朔以開封尹鄭王重貴為東京留守 馮道李崧屢薦天平節度使兼侍衛親軍馬步副都指揮使同平章事杜重威之能|{
	此希上指而薦之也}
以為都指揮使充隨駕御營使代劉知遠知遠由是恨二相|{
	為馮道不用於漢李崧見殺張本}
重威所至黷貨民多逃亡嘗出過市謂左右曰人言我驅盡百姓何市人之多也壬辰帝發大梁 己亥至鄴都壬寅大赦帝以詔諭安重榮曰爾身為大臣家有老母忿不思難棄君與親吾因契丹得天下爾因吾致富貴|{
	謂重榮降帝于晉陽從此得富貴}
吾不敢忘德爾乃忘之何邪今吾以天下臣之爾欲以一鎮抗之不亦難乎宜審思之無取後悔重榮得詔愈驕聞山南東道節度使安從進有異志隂遣使與之通謀|{
	安從進反而重榮亦反矣}
吳越文穆王元瓘寢疾察内都監章德安忠厚能斷大事欲屬以後事語之曰弘佐尚少當擇宗人長者立之|{
	監古銜翻斷丁亂翻屬之欲翻語牛倨翻少詩照翻長知兩翻}
德安曰弘佐雖少羣下伏其英敏願王勿以為念王曰汝善輔之吾無憂矣德安處州人也辛亥元瓘卒|{
	年五十五}
初内牙指揮使戴惲為元瓘所親任悉以軍事委之元瓘養子弘侑乳母惲妻之親也|{
	惲於粉翻}
或吿惲謀立弘侑德安祕不發喪與諸將謀伏甲士于幕下壬子惲入府執而殺之廢弘侑為庶人復姓孫幽之明州是日將吏以元瓘遺命承制以鎮海鎮東副大使弘佐為節度使時年十四|{
	歐史曰年十三}
九月庾申弘佐即王位命丞相曹仲達攝政軍中言賜與不均舉仗不受諸將不能制仲達親諭之皆釋仗而拜弘佐温恭好書禮士躬勤政務發摘姦伏人不能欺|{
	摘他狄翻}
民有獻嘉禾者弘佐問倉吏今蓄積幾何對曰十年王曰然則軍食足矣可以寛吾民乃命復其境内税三年|{
	復方目翻除免也史言弘佐雖少而敏於政}
辛酉滑州言河決 |{
	考異日薛史紀載九月辛酉滑州河決而不載庚午濮州決高祖實録載庚午濮州奏而不載辛酉滑州決五代會要及志皆云天福六年九月決滑州兖濮州界皆為水漂溺史匡翰傳亦云天福六年白馬河決按辛酉滑州河已決則下流皆涸濮州無庚午再決之理蓋滑州河決漂浸及濮州耳}
帝以安重榮殺契丹使者恐其犯塞乙亥遣安國節度使楊彦珣使于契丹彦珣至其帳契丹責以使者死狀彦珣曰譬如人家有惡子父母所不能制將如之何契丹主怒乃解 閩主曦以其子瑯邪王亞澄為威武節度使兼中書令改號長樂王|{
	樂音洛}
劉知遠遣親將郭威以詔指說吐谷渾酋長白承福|{
	時朝廷陽為逐吐谷渾而隂撫納之又懼契丹知之而怒之也不敢明降詔書故劉知遠承帝密指使郭威稱詔指以說之將即亮翻說式苪翻酋慈秋翻長知兩翻}
令去安重榮歸朝廷許以節鉞威還謂知遠曰虜惟利是嗜安鐵胡止以袍袴賂之|{
	還從宣翻安重榮小字鐵胡}
今欲其來莫若重賂乃可致耳知遠從之且使謂承福曰朝廷已割爾曹隸契丹爾曹當自安部落今乃南來助安重榮為逆重榮已為天下所棄朝夕敗亡爾曹宜早從化勿俟臨之以兵南北無歸悔無及矣|{
	言吐谷渾若助安重榮重榮敗亡之後吐谷渾南不可歸晉北不可歸契丹}
承福懼冬十一月帥其衆歸于知遠知遠處之太原東山及嵐石之間|{
	帥讀曰率處昌呂翻嵐盧含翻}
表承福領大同節度使|{
	雲州大同軍時已屬契丹}
收其精騎以隸麾下|{
	為劉知遠殺白承福張木}
始安重榮移檄諸道云與吐谷渾達靼契苾同起兵旣而承福降知遠達靼契苾亦莫之赴|{
	靼當割翻}
重榮勢大沮|{
	沮在呂翻}
閩主曦即皇帝位王延政自稱兵馬元帥閩同平章事李敏卒|{
	書閩同平章事以别他國之相}
帝之發大梁也和凝請曰車駕已行安從進若反何以備之帝曰卿意如何凝請密留空名宣敇十數通|{
	宣出於樞密院敇出於中書門下時并樞密院於中書空苦貢翻}
付留守鄭王聞變則書諸將名遣撃之帝從之十一月從進舉兵攻鄧州唐州刺史武延翰以聞|{
	九域志襄陽北至鄧州一百七十八里東北至唐州二百五十里}
鄭王遣宣徽南院使張從恩武德使焦繼勲護聖都指揮使郭金海作坊使陳思讓將大梁兵就申州刺史李建崇兵于葉縣以討之|{
	漢有葉縣中廢隋復置葉縣唐屬汝州九域志在州東南一百四十里葉式涉翻}
金海本突厥思讓幽州人也|{
	厥九勿翻}
丁丑以西京留守高行周為南面軍前都部署前同州節度使宋彦筠副之張從恩監焉又以郭金海為先鋒使陳思讓監焉|{
	監古銜翻}
彦筠滑州人也庚辰以鄴都留守李德珫權東京留守召鄭王重貴如鄴都安從進攻鄧州威勝節度使安審暉據牙城拒之|{
	鄧州牙城也}
從進不能克而退癸未從進至花山|{
	九域志唐州湖陽縣有花山銀場按花山在湖陽北}
遇張從恩兵不意其至之速合戰大敗從恩獲其子牙内都指揮使弘義從進以數十騎奔還襄州嬰城自守 唐主性節儉常躡蒲屢盥頮用鐵盎|{
	躡尼輒翻織蒲為屨江淮之人多能之頮呼内翻澡手為盥滌面為頮}
暑則寢于青葛帷左右使令惟醜宫人服飾粗畧|{
	粗讀曰麤}
死國事者皆給祿三年分遣使者按行民田以肥瘠定其稅|{
	行下孟翻}
民間稱其平允自是江淮調兵興役及他賦斂皆以税錢為率|{
	調徒鉤翻斂力贍翻}
至今用之唐主勤于聽政以夜繼晝還自江都不復宴樂頗傷躁急|{
	復扶又翻樂音洛躁則到翻}
内侍王紹顔上書以為今春以來羣臣獲罪者衆中外疑懼唐主手詔釋其所以然令紹顔告諭中外 十二月丙戌朔徙鄭王重貴為齊王充鄴都留守以李德珫為東都留守 丁亥以高行周知襄州行府事詔荆南湖南共討襄州高從誨遣都指揮使李端將水軍數千至南津|{
	漢水南津也}
楚王希範遣天策都軍使張少敵將戰艦百五十艘入漢江助行周仍各運糧以饋之少敵佶之子也|{
	張佶與楚王馬殷同起事者也少詩沼翻艦戶黯翻艘蘇遭翻佶其吉翻}
安重榮聞安從進舉兵反謀遂決大集境内飢民衆至數萬南向鄴都聲言入朝初重榮與深州人趙彦之俱為散指揮使相得歡甚|{
	散悉亶翻}
重榮鎮成德|{
	二年安重榮始帥鎮州}
彦之自關西歸之重榮待遇甚厚使彦之招募黨衆然心實忌之及舉兵止用為排陳使|{
	陳讀曰陣}
彦之恨之帝聞重榮反壬辰遣護聖等馬步三十九指揮擊之以天平節度使杜重威為招討使安國節度使馬全節副之前永清節度使王清為馬步都虞候 安從進遣其弟從貴將兵逆均州刺史蔡行遇|{
	行遇者安從進巡内刺史時蓋以兵援襄陽故遣弟逆之}
焦繼勲邀擊敗之獲從貴斷其足而歸之|{
	敗補邁翻斷音短}
戊戌杜重威與安重榮遇于宗城西南|{
	九域志宗城縣在魏州之西北一百七十里}
重榮為偃月陳官軍再擊之不動重威懼欲指揮使宛丘王重胤曰兵家忌|{
	則敵得而乘之或士卒因退而潰亂故忌之陳讀曰陣}
鎮之精兵盡在中軍請公分銳士撃其左右翼重胤為公以契丹直衝其中軍|{
	為于偽翻}
彼必狼狽重威從之鎮人陳稍却趙彦之卷旗策馬來降彦之以銀飾鎧胄及鞍勒官軍殺而分之重榮聞彦之叛大懼退匿于輜重中|{
	重直用翻}
官軍從而乘之鎮人大潰斬首萬五千級重榮收餘衆走保宗城官軍進攻夜分拔之重榮以十餘騎走還鎮州嬰城自守會天寒鎮人戰及凍死者二萬餘人契丹聞重榮反乃聽楊彦詢還|{
	是年五月楊彦詢使契丹}
庚子冀州刺史張建武等取趙州|{
	冀趙二州皆安重榮巡屬}
漢主寢疾有胡僧謂漢主名龔不利漢主自造龑字名之義取飛龍在天|{
	易曰飛龍在天利見大人}
讀若儼 庚戌制以錢弘佐為鎮海鎮東軍節度使兼中書令吳越國王

資治通鑑卷二百八十二
