資治通鑑卷一百十七

宋 司馬光 撰

胡三省 音注

晉紀三十九|{
	起旃蒙單閼盡柔兆執徐凡三年}


安皇帝壬

義熙十一年春正月丙辰魏主嗣還平城|{
	至自伐柔然也}
太尉裕收司馬休之次子文寶兄子文祖竝賜死發兵擊之詔加裕黄鉞領荆州刺史庚午大赦 丁丑以吏部尚書謝裕為尚書左僕射 辛巳太尉裕發建康以中軍將軍劉道憐監留府事|{
	監工銜翻}
劉穆之兼右僕射事無大小皆決於穆之又以高陽内史劉鍾領石頭戍事屯冶亭|{
	冶亭今謂之東冶亭在半山寺後自建康東門往蔣山至此半道因以為名王安石詩遙望鍾山岑因知冶城路陸游曰今天慶觀在冶城山之麓}
休之府司馬張裕南平太守檀範之聞之皆逃歸建康|{
	守式又翻}
裕邵之兄也|{
	張邵見一百十五卷五年}
雍州刺史魯宗之自疑不為太尉裕所容|{
	雍於用翻}
與其子竟陵太守軌起兵應休之二月休之上表罪狀裕勒兵拒之裕密書招休之府録事參軍南陽韓延之延之復書曰承親帥戎馬遠履西畿|{
	周禮王畿千里之外曰侯畿甸畿男畿采畿衛畿蠻畿夷畿鎮畿蕃畿謂之畿者責以共王稅貢為職韓延之以荆楚為西畿取此義帥讀曰率}
闔境士庶莫不惶駭辱疏知以譙王前事良增歎息司馬平西體國忠貞|{
	休之為平西將軍故稱之}
欵懷待物以公有匡復之勲家國蒙賴推德委誠每事詢仰譙王往以微事見劾猶自表遜位|{
	事見上卷上年劾戶槩翻又戶得翻}
况以大過而當嘿然邪前已表奏廢之所不盡者命耳推寄相與正當如此|{
	推寄謂推心置人腹中也}
而遽興兵甲所謂欲加之罪其無辭乎|{
	左傳晉大夫里克之言}
劉裕足下海内之人誰不見足下此心而復欲欺誑國士|{
	誑居况翻}
來示云處懷期物自有由來|{
	復扶又翻處昌呂翻下同}
今伐人之君啗人以利真可謂處懷期物自有由來者乎|{
	啗土濫翻又土覽翻}
劉藩死於閶闔之門諸葛斃於左右之手|{
	劉藩事見上卷八年諸葛事見九年}
甘言詫方伯襲之以輕兵|{
	謂襲劉毅也事見上卷八年詫丑亞翻}
遂使席上靡欵懷之士閫外無自信諸侯以是為得算良可恥也貴府將佐及朝廷賢德寄命過日|{
	將即亮翻朝直遙翻}
吾誠鄙劣嘗聞道於君子以平西之至德寧可無授命之臣乎必未能自投虎口比迹郗僧施之徒明矣|{
	郗僧施事見上卷八年郗丑之翻}
假令天長喪亂九流渾濁|{
	太史談序九流班固曰儒家者流蓋出於司徒之官助人君順隂陽明教化者也道家者流蓋出於史官歷紀成敗禍福古今之道此人君南面之術也法家者流蓋出於理官信賞必罰以輔禮制隂陽家者流蓋出於羲和之官敬順昊天歷象日月星辰名家者流蓋出於禮官古者名位不同禮亦異數孔子曰必也正名乎墨家者流蓋出於清廟之守茅屋采椽是以貴儉養三老五更是以兼愛選士大射是以尚賢宗祀嚴父是以右鬼從横家者流蓋出於行人之官雜家者流蓋出於議官農家者流蓋出於農稷之官皆六經之支與流裔有益於治道而不能無弊使其渾濁則無所取衷矣長知兩翻喪息浪翻}
當與臧洪遊於地下|{
	臧洪事見六十二卷漢獻帝興平二年}
不復多言|{
	復扶又翻}
裕視書歎息以示將佐曰事人當如此矣延之以裕父名翹字顯宗乃更其字曰顯宗|{
	更工衡翻}
名其子曰翹以示不臣劉氏 瑯邪太守劉朗帥二千餘家降魏|{
	帥讀曰率降戶江翻}
庚子河西胡劉雲等帥數萬戶降魏 太尉裕使參軍檀道濟朱超石將步騎出襄陽|{
	將即亮翻騎奇寄翻}
超石齡石之弟也江夏太守劉䖍之將兵屯三連|{
	夏戶雅翻}
立橋聚糧以待道濟等積日不至魯軌襲擊䖍之殺之裕使其壻振威將軍東海徐逵之統參軍蒯恩王允之沈淵子為前鋒出江夏口|{
	水經江水過江陵城南又東至華容縣西夏水出焉又東過公安縣北又東左合子夏口註云江水左迤北出通於夏水故曰子夏也蒯苦怪翻}
逵之等與魯軌戰於破冢兵敗逵之允之淵子皆死獨蒯恩勒兵不動|{
	蒯苦怪翻}
軌乘勝力攻之不能克乃退淵子林子之兄也裕軍於馬頭|{
	據水經注馬頭岸在大江之南北對江陵之江津戍}
聞逵之死怒甚三月壬午帥諸將濟江魯軌司馬文思將休之兵四萬臨峭岸置陳|{
	峭七笑翻陳讀曰陣}
軍士無能登者裕自被甲欲登|{
	被皮義翻}
諸將諫不從怒愈甚太尉主簿謝晦前抱持裕裕抽劒指晦曰我斬卿晦曰天下可無晦不可無公|{
	此裕所謂晦頗識機變者也}
建武將軍胡藩領遊兵在江津裕呼藩使登藩有疑色裕命左右録來欲斬之|{
	録敢也}
藩顧曰正欲擊賊不得奉教乃以刀頭穿岸劣容足指騰之而上|{
	劣少也上時掌翻}
隨之者稍多既登岸直前力戰休之兵不能當稍引却|{
	當胡藩之初登也精騎數十可以制之休之之兵不動故得以直前力戰又人心素懾服裕故藩既進而不能當也}
裕兵因而乘之休之兵大潰遂克江陵休之宗之俱北走軌留石城裕命閬中侯下邳趙倫之太尉參軍沈林子攻之遣武陵内史王鎮惡以舟師追休之等有羣盜數百夜襲冶亭京師震駭劉鍾討平之 秦廣平公弼譛姚宣於秦王興|{
	去年宣入朝力言弼罪弼銜而譛之}
宣司馬權丕至長安興責以不能輔導將誅之丕懼誣宣罪惡以求自免興怒遣使就杏城收宣下獄命弼將三萬人鎮秦州|{
	下遐稼翻將即亮翻}
尹昭曰廣平公與皇太子不平今握彊兵於外陛下一旦不諱社稷必危小不忍亂大謀|{
	論語載孔子之言}
陛下之謂也興不從 夏王勃勃攻秦杏城拔之執守將姚逵阬士卒二萬人秦王興如北地遣廣平公弼及輔國將軍歛曼嵬向新平興還長安 河西王蒙遜攻西秦廣武郡拔之西秦王熾磐遣將軍乞伏魋尼寅邀蒙遜於浩亹蒙遜擊斬之|{
	浩亹音告門}
又遣將軍折斐等帥騎一萬據勒姐嶺|{
	闞駰志金城安夷縣東有勒姐河與金城河合勒姐嶺蓋勒姐河所出之山也漢時勒姐羌居之因以為名姐子也翻又音紫}
蒙遜擊禽之 河西饑胡相聚於上黨推胡人白亞栗斯為單于|{
	單音蟬}
改元建平以司馬順宰為謀主|{
	順宰起兵見上卷二年}
寇魏河内夏四月魏主嗣命公孫表等五將討之|{
	將即亮翻下同}
青冀二州刺史劉敬宣參軍司馬道賜宗室之疏屬也聞太尉裕攻司馬休之道賜與同府辟閭道秀|{
	道賜與道秀俱為敬宣僚屬故曰同府}
左右小將王猛子謀殺敬宣據廣固以應休之乙卯敬宣召道秀屏人語|{
	屏必郢翻}
左右悉出戶猛子逡巡在後取敬宣備身刀殺敬宣文武佐吏即時討道賜等皆斬之 己卯魏主嗣北巡 西秦王熾磐子元基自長安逃歸|{
	元基蓋從熾磐入秦以朝因留長安}
熾磐以為尚書左僕射 五月丁亥魏主嗣如大甯 趙倫之沈林子破魯軌於石城司馬休之魯宗之救之不及遂與軌奔襄陽宗之參軍李應之閉門不納甲午休之宗之軌及譙王文思新蔡王道賜|{
	此又一司馬道賜也新蔡王晃以武陵王晞事廢後以道賜襲爵}
梁州刺史馬敬南陽太守魯範俱奔秦宗之素得士民心爭為之衛送出境王鎮惡等追之盡境而還|{
	不敢窮兵追之懼出境而遇伏也}
初休之等求救於秦魏秦征虜將軍姚成王及司馬國璠引兵至南陽|{
	璠孚袁翻}
魏長孫嵩至河東聞休之等敗皆引還休之至長安秦王興以為揚州刺史使侵擾襄陽侍御史唐盛言於興曰據符䜟之文司馬氏當復得河洛|{
	䜟楚譛翻}
今使休之擅兵於外猶縱魚於淵也不如以高爵厚禮留之京師興曰昔文王卒免羑里|{
	紂囚文王於羑里既而釋之復扶又翻卒子恤翻}
高祖不斃鴻門|{
	見九卷漢高祖元年}
苟天命所在誰能違之脱如符䜟之言留之適足為害遂遣之|{
	史言姚興知命}
詔加太尉裕太傅揚州牧劒履上殿入朝不趨贊拜不名|{
	朝直遙翻}
以兖青二州刺史劉道憐為都督荆湘益秦寜梁雍七州諸軍事驃騎將軍荆州刺史|{
	雍於用翻驃匹妙翻騎奇寄翻}
道憐貪鄙無才能裕以中軍長史晉陵太守謝方明為驃騎長史南郡相道憐府中衆事皆諮決於方明方明冲之子也|{
	謝冲奕之從子方明裕之從祖弟也}
益州刺史朱齡石遣使詣河西王蒙遜|{
	使疏吏翻}
諭以朝廷威德蒙遜遣舍人黄迅詣齡石且上表言伏聞車騎將軍裕欲清中原願為右翼驅除戎虜 夏王勃勃遣御史中丞烏洛孤與蒙遜結盟蒙遜遣其弟湟河太守漢平莅盟于夏|{
	夏戶雅翻}
西秦王熾磐率衆三萬襲湟河沮渠漢平拒之遣司馬隗仁夜出擊熾磐破之|{
	沮子余翻隗五罪翻熾昌志翻}
熾磐將引去漢平長史焦昶將軍段景潜召熾磐熾磐復攻之|{
	昶丑兩翻復扶又翻}
昶景因說漢平出降|{
	說輸芮翻降戶江翻下同}
仁勒壯士百餘據南門樓三日不下力屈為熾磐所禽熾磐欲斬之散騎常侍武威段暉諫曰仁臨難不畏死|{
	散悉亶翻騎奇寄翻難乃旦翻}
忠臣也宜宥之以厲事君乃囚之熾磐以左衛將軍匹達為湟河太守擊乙弗窟乾降其三千餘戶而歸以尚書右僕射出連䖍為都督嶺北諸軍事|{
	嶺北洪地嶺北也}
涼州刺史以涼州刺史謙屯為鎮軍大將軍河州牧隗仁在西秦五年段暉又為之請|{
	史書武威段暉以别南燕之段暉也又為于偽翻}
熾磐免之使還姑臧 戊午魏主嗣行如濡源遂至上谷涿鹿廣甯|{
	涿鹿縣漢屬上谷郡晉分屬廣甯郡魏土地記下洛縣東南六十里有涿鹿城西北百三十里有大甯城即漢廣甯縣也蓋在唐媯州界濡乃官翻}
秋七月癸未還平城 西秦王熾磐以秦州刺史曇達為尚書令|{
	曇徒含翻}
光禄勲王松夀為秦州刺史 辛亥晦日有食之八月甲子太尉裕還建康固辭太傅州牧其餘受命以豫章公世子義符為兖州刺史 丁未謝裕卒以劉穆之為左僕射 九月己亥大赦 魏比歲霜旱|{
	比毗至翻}
雲代之民多饑死|{
	雲代雲中代郡二郡之地}
太史令王亮蘇坦言於魏主嗣曰案䜟書魏當都鄴可得豐樂|{
	樂音洛}
嗣以問羣臣博士祭酒崔浩特進京兆周澹曰|{
	澹徒覽翻}
遷都於鄴可以救今年之饑非久長之計也山東之人以國家居廣漢之地|{
	廣漢據北史崔浩傳作廣漠當從之漠大也}
謂其民畜無涯號曰牛毛之衆今留兵守舊都|{
	謂平城也}
分家南徙不能滿諸州之地參居郡縣情見事露|{
	見賢遍翻}
恐四方皆有輕侮之心且百姓不便水土疾疫死傷者必多又舊都守兵既少|{
	少詩沼翻下同}
屈丐柔然將有窺窬之心舉國而來雲中平城必危朝廷隔恒代千里之險|{
	自恒山至代有飛狐之口倒馬之關夏屋廣昌五迴之險}
難以赴救此則聲實俱損也今居北方假令山東有變我輕騎南下布濩林薄之間|{
	騎奇寄翻濩胡故翻郭璞曰布濩猶布露也毛晃曰布濩流散也草叢生曰薄}
孰能知其多少百姓望塵懾服此國家所以威制諸夏也|{
	懾之涉翻夏戶雅翻}
來春草生湩酪將出|{
	湩覩勇翻又多貢翻乳汁也酪歷各翻乳漿也西漢太僕屬官有挏馬應劭曰主乳馬取其汁挏治之味酢可飲因以名官如淳曰主乳馬以韋革為夾兜受數斗盛馬乳挏取其上肥因名挏馬今梁州名馬酪為馬酒師古曰挏音徒孔翻}
兼以菜果得及秋熟則事濟矣嗣曰今倉廩空竭既無以待來秋若來秋又饑將若之何對曰宜簡饑貧之戶使就食山東若來秋復饑當更圖之|{
	復扶又翻}
但方今不可遷都耳嗣悦曰唯二人與朕意同乃簡國人尤貧者詣山東三州就食|{
	拓跋氏起于漠北統國三十六姓九十九道武既并中原徙其豪傑於雲代與北人雜居以其北來部落為國人山東三州定相冀也}
遣左部尚書代人周幾帥衆鎮魯口以安集之|{
	魏初四方四維置八部大人分東西南北左右前後後又置八部尚書魏書官氏志拓跋鄰以次兄為普氏後改為周氏盖魏建代都周幾遂為代人帥讀曰率}
嗣躬耕籍田且命有司勸課農桑明年大熟民遂富安 夏赫連建將兵擊秦執平涼太守姚軍都|{
	將即亮翻}
遂入新平廣平公弼與戰於龍尾堡|{
	劉昫地理志鳳翔府岐山縣唐武德七年移治龍尾城}
禽之 秦王興藥動廣平公弼稱疾不朝|{
	朝直遙翻}
聚兵於第興聞之怒收弼黨唐盛孫玄等殺之太子泓請曰臣不肖不能緝諧兄弟|{
	緝當作輯}
使至於此皆臣之罪也若臣死而國家安願賜臣死若陛下不忍殺臣乞退就藩興惻然憫之召姚讚梁喜尹昭斂曼嵬與之謀囚弼將殺之窮治黨與|{
	嵬五回翻治直之翻}
泓流涕固請乃并其黨赦之泓待弼如初無忿恨之色 魏太史奏熒惑在匏瓜中|{
	據晉書天文志匏瓜在天津之南天漢分流夾之張淵觀象賦註曰匏瓜五星在麗珠北天津九星在匏瓜北}
忽亡不知所在於法當入危亡之國先為童謠妖言然後行其禍罰|{
	法謂推占之常法妖於驕翻}
魏主嗣召名儒十餘人使與太史議熒惑所詣崔浩對曰按春秋左氏傳神降于莘以其至之日推知其物|{
	浩蓋據春秋左氏外傳也外傳曰周惠王十五年有神降於莘王問於内史過對曰其丹朱乎王曰其誰受之對曰在虢土王曰虢其幾何對曰昔堯臨民以五今其胄見神之見也不過其物若由是觀之不過五年十九年晉取虢傳直戀翻}
庚午之夕辛未之朝天有隂雲熒惑之亡當在二日庚之與午皆主於秦辛為西夷|{
	晉書天文志自東井十六度至柳八度為鶉首於辰在未秦之分野自柳九度至張十二度為鶉火於辰在午周之分野時姚秦兼有關洛之地故云皆主於秦庚辛西方也故為西夷}
今姚興據長安熒惑必入秦矣衆皆怒曰天上失星人間安知所詣浩笑而不應後八十餘日熒惑出東井留守句已久之乃去|{
	新唐書天文志曰去而復來是謂句已晉書天文志曰熒惑為亂為賊為疾為喪為飢為兵所居國受殃環繞鉤已芒角動搖變色乍前乍後乍左乍右其殃愈甚句讀曰鉤鉤已謂環繞而行如鉤又成已字也}
秦大旱昆明池竭童謠訛言|{
	徒歌謂之謠}
國人不安間一歲而秦亡衆乃服浩之精妙 冬十月壬子秦王興使散騎常侍姚敞等送其女西平公主於魏|{
	散悉亶翻騎奇寄翻下同}
魏主嗣以后禮納之鑄金人不成|{
	魏丘嗣立后皆鑄金人以卜之}
乃以為夫人而寵遇甚厚 辛酉魏主嗣如沮洳城|{
	沮將豫翻洳昌庶翻沮洳漸濕之地北方地高燥此城蓋以下濕而得名}
癸亥還平城十一月丁亥復如豺山宫|{
	復扶又翻}
庚子還 西秦王熾磐遣襄武侯曇達等將騎一萬擊南羌彌姐康薄于赤水降之|{
	水經註赤亭水出南安郡東山赤谷西流逕城北南入渭水曇徒含翻將即亮翻姐子也翻又音紫降戶江翻}
以王孟保為略陽太守鎮赤水 燕尚書令孫護之弟伯仁為昌黎尹與其弟叱支乙拔皆有才勇從燕王跋起兵有功|{
	謂殺慕容熙時也事見一百十四卷三年}
求開府不得有怨言跋皆殺之進護開府儀同三司録尚書事以慰其心護怏怏不悦跋酖殺之|{
	怏於兩翻}
遼東太守務銀提自以有功出為邊郡怨望謀外叛跋亦殺之|{
	萬泥乳陳既死孫護兄弟及務銀提又誅馮跋亦少恩矣}
林邑寇交州州將擊敗之|{
	將即亮翻敗補邁翻}


十二年春正月甲申魏主嗣如豺山宫戊子還平城加太尉裕兖州刺史都督南秦州凡都督二十二州|{
	二十二州徐南徐豫南豫兖南兖青冀幽并司郢荆江湘雍梁益寧交廣南秦也}
以世子義符為豫州刺史 秦王興使魯宗之將兵宼襄陽未至而卒|{
	卒子恤翻}
其子軌引兵入宼雍州刺史趙倫之擊敗之|{
	雍於用翻敗補邁翻}
西秦王熾磐攻秦洮陽公彭利和於漒川|{
	洮土刀翻漒其良翻}


沮渠蒙遜攻石泉以救之熾磐至沓中引還二月熾磐遣襄武侯曇達救石泉|{
	曇徒含翻}
蒙遜亦引去蒙遜遂與熾磐結和親|{
	自熾磐滅秃髪氏與蒙遜為鄰敵歲歲交兵今乃結和}
秦王興如華隂使太子泓監國|{
	華戶化翻監工銜翻}
入居西宫|{
	太子居東宫西宫秦王所居也}
興疾篤還長安黄門侍郎尹冲謀因泓出迎而殺之興至泓將出迎宫臣諫曰|{
	凡東宫官屬皆曰宫臣}
主上疾篤姦臣在側|{
	姦臣謂尹冲等}
殿下今出進不得見主上退有不測之禍泓曰臣子聞君父疾篤而端居不出何以自安對曰全身以安社稷孝之大者也泓乃止尚書姚沙彌謂尹冲曰太子不出迎宜奉乘輿幸廣平公第宿衛將士聞乘輿所在自當來集|{
	乘繩證翻將即亮翻}
太子誰與守乎且吾屬以廣平公之故已陷名逆節將何所自容今奉乘輿以舉事乃杖大順不惟救廣平之禍吾屬前罪亦盡雪矣|{
	杖仗同}
冲以興死生未可知欲隨興入宫作亂不用沙彌之言興入宫命太子泓録尚書事東平公紹及右衛將軍胡翼度典兵禁中防制内外|{
	紹興之弟也}
遣殿中上將軍歛曼嵬收弼第中甲仗内之武庫|{
	晉置殿中將軍姚秦復有殿中上將軍使統殿中諸主帥}
興疾轉篤其妹南安長公主問疾不應|{
	長知兩翻}
幼子耕兒出吿其兄南陽公愔曰上已崩矣宜速決計愔即與尹冲帥甲士攻端門|{
	愔於今翻帥讀曰率}
歛曼嵬胡翼度等勒兵閉門拒戰愔等遣壯士登門緣屋而入及于馬道泓侍疾在諮議堂太子右衛率姚和都率東宫兵入屯馬道南愔等不得進遂燒端門興力疾臨前殿賜弼死禁兵見興喜躍爭進赴賊賊衆驚擾和都以東宫兵自後擊之愔等大敗愔逃于驪山其黨建康公呂隆犇雍|{
	雍於用翻}
尹冲及弟泓來犇興引東平公紹及姚讚梁喜尹昭歛曼嵬入内寢受遺詔輔政明日興卒|{
	年五十一 考異曰晉本紀三十國晉春秋皆云義熙十一年二月姚興卒魏本紀北史本紀姚興姚泓載記皆云十二年按後魏書崔鴻傳太祖天興二年姚興改號鴻以為元年故晉本紀三十國晉春秋凡弘始後事皆在前一年由鴻之誤也}
泓祕不發喪捕南陽公愔及呂隆太將軍尹元等皆誅之乃發喪即皇帝位|{
	泓字元子興之長子也}
大赦改元永和泓命齊公恢殺安定太守呂超|{
	隆超兄弟也皆黨於弼齊公恢時鎮安定}
恢猶豫久之乃殺之泓疑恢有貳心恢由是懼隂聚兵謀作亂|{
	為後姚恢舉兵張本}
泓葬興于偶陵諡曰文桓皇帝廟號高祖初興徙李閏羌三千戶於安定興卒羌酋党容叛|{
	孫愐曰党本去聲今為上聲本出西羌姚秦有將軍党耐虎自云夏后氏之後為羌豪酋慈由翻下同党底朗翻}
泓遣撫軍將軍姚讚討降之|{
	降戶江翻}
徙其酋豪于長安餘遣還李閏北地太守毛雍據趙氏塢以叛|{
	趙氏塢孝武太元九年秦主堅撃後秦所屯之地}
東平公紹討禽之時姚宣鎮李閏參軍韋宗聞毛雍叛說宣曰主上新立威德未著國家之難未可量也|{
	難乃旦翻}
殿下不可不為深慮邢望險要宜徙據之此霸王之資也宣從之帥戶三萬八千棄李閏南保邢望|{
	帥讀曰率}
諸羌據李閏以叛東平公紹進討破之宣詣紹歸罪紹殺之 二月加太尉裕中外大都督裕戒嚴將伐秦詔加裕領司豫二州刺史以其世子義符為徐兖二州刺史琅邪王德文請啓行戎路|{
	詩曰元戎十乘以先啓行行戶剛翻}
修敬山陵詔許之 夏四月壬子魏大赦改元泰常 西秦襄武侯曇達等擊秦秦州刺史姚艾於上邽破之徙其民五千餘戶於枹罕|{
	曇徒含翻抱音膚}
五月癸巳加太尉裕領北雍州刺史|{
	晉初置雍州於長安永嘉之亂沒于劉石苻秦之亂雍州流民南出樊沔孝武始於襄陽僑立雍州今裕欲取長安故領北雍州刺史以别襄陽之雍州也雍於用翻下同}
六月丁巳魏主嗣北巡 并州胡數萬落叛秦入于平陽推匈奴曹弘為大單于|{
	弘蓋匈奴右賢王曹轂子寅之後所謂東曹者也單音蟬}
攻立義將軍姚成都於匈奴堡|{
	此匈奴種落相率保聚之地因以為名}
征東將軍姚懿自蒲坂討之執弘送長安徙其豪右萬五千落於雍州|{
	秦雍州治安定}
氐王楊盛攻秦祁山拔之進逼秦州秦後將軍姚平救之盛引兵退平與上邽守將姚嵩追之|{
	將即亮翻}
夏王勃勃帥騎四萬襲上邽|{
	帥讀曰率騎奇寄翻}
未至嵩與盛戰於竹嶺敗死|{
	水經注籍水歷當亭川又東南流與竹嶺水合水出南山竹嶺東北入籍水籍水東北入上邽縣}
勃勃攻上邽二旬克之殺秦州刺史姚軍都及將士五千餘人因毁其城進攻隂密|{
	隂密古密人之國詩所謂密人不恭敢距大邦者也自漢以來為縣屬安定郡括地志隂密故城在涇州鶉觚縣西其東接縣城即古密國}
又殺秦將姚良子及將士萬餘人|{
	將即亮翻}
以其子昌為雍州刺史鎮隂密征北將軍姚恢棄安定奔還長安安定人胡儼等帥戶五萬據城降於夏|{
	帥讀曰率降戶江翻}
勃勃使鎮東將軍羊苟兒將鮮卑五千鎮安定進攻秦鎮西將軍姚諶於雍城諶委鎮犇長安勃勃據雍進掠郿城|{
	將鮮即亮翻諶氏壬翻雍於用翻郿音媚又音眉}
秦東平公紹及征虜將軍尹昭等將步騎五萬擊之勃勃退趨安定|{
	趨七喻翻}
胡儼閉門拒之殺羊苟兒及所將鮮卑復以安定降秦|{
	復扶又翻下同}
紹進擊勃勃於馬鞍阪破之追至朝那不及而還|{
	還從宣翻又如字}
勃勃歸杏城楊盛復遣兄子倦擊秦至陳倉秦歛曼嵬擊却之夏王勃勃復遣兄子提南侵泄陽|{
	晉書載記作池陽當從之池陽縣屬扶風郡唐為京兆雲陽縣復扶又翻}
秦車騎將軍姚裕等擊却之 涼司馬索承明上書勸涼公暠伐河西王蒙遜|{
	索昔各翻暠古老翻}
暠引見謂之曰蒙遜為百姓患孤豈忘之顧勢力未能除耳卿有必禽之策當為孤陳之|{
	為于偽翻}
直唱大言使孤東討此與言石虎小豎宜肆諸市朝者何異|{
	朝直遙翻}
承明慙懼而退 秋七月魏主嗣大獵于牛川臨殷繁水而還|{
	北史曰登釜山臨殷繁水括地志曰釜山在媯州懷戎縣北三里}
戊戌至平城 八月丙午大赦 寜州獻琥珀枕於太尉裕|{
	琥珀出哀牢夷廣雅曰琥珀生地中其上及旁不生草深者八九尺大如斛削去皮尖琥珀如斗初時如桃膠凝堅乃成博物志松脂淪入地千年化為茯苓茯苓千年化為琥珀今太山有茯苓而無琥珀永昌有琥珀而無茯苓}
裕以琥珀治金創|{
	治直之翻創初良翻}
得之大喜命碎擣分賜北征將士裕以世子義符為中軍將軍監太尉留府事劉穆之為左僕射領監軍中軍二府軍司|{
	監軍謂義符監太尉留府軍也監工銜翻}
入居東府摠攝内外以太尉左司馬東海徐羨之為穆之之副左將軍朱齡石守衛殿省徐州刺史劉懷慎守衛京師揚州别駕從事史張裕任留州事|{
	任留州事任揚州留後事也}
懷慎懷敬之弟也劉穆之内摠朝政外供軍旅決斷如流事無擁滯|{
	朝直遙翻斷丁亂翻}
賓客輻湊求訴百端内外諮禀盈階滿室|{
	盈階滿室謂諮禀之文書也}
目覽辭訟手答牋書耳行聽受口並酬應不相參涉悉皆贍舉又喜賓客|{
	贍時艶翻喜許記翻}
言談賞笑彌日無倦裁有閒暇手自寫書尋覽校定性奢豪食必方丈旦輒為十人饌未嘗獨餐|{
	饌雛戀翻又雛皖翻}
嘗白裕曰穆之家本貧賤贍生多闕自叨沗以來|{
	叨卜刀翻}
雖每存約損而朝夕所須微為過豐自此外一毫不以負公中軍諮議參軍張邵言於裕曰人生危脆必當遠慮穆之若邂逅不幸|{
	脆此芮翻邂戶隘翻逅胡茂翻}
誰可代之尊業如此|{
	尊業言裕已成之功業也尊者尊稱之也}
苟有不諱處分云何|{
	處昌呂翻分扶問翻}
裕曰此自委穆之及卿耳丁巳裕發建康遣龍驤將軍王鎮惡冠軍將軍檀道濟將步軍自淮淝向許洛|{
	驤思將翻冠古玩翻將步即亮翻}
新野太守朱超石寧朔將軍胡藩趨陽城振武將軍沈田子建威將軍傅弘之趨武關|{
	趨七喻翻}
建武將軍沈林子彭城内史劉遵考將水軍出石門自汴入河|{
	汴水首受濟東南與淮通漢書地理志所謂狼湯渠是也狼音浪湯音宕昔大禹塞熒澤開此渠以通淮泗禹貢所謂導沇水東流為濟入于河溢為滎東出于陶丘北者也漢修河隄始立石門以遏水水盛則通于河水耗則輟流}
以冀州刺史王仲德督前鋒諸軍開鉅野入河|{
	水經濟水北至東燕縣與河合酈道元註曰濟水自乘氏縣兩分東北入于鉅野濟之故瀆又北右合洪水洪水上承鉅野薛訓渚自渚迄于北口一百二十里名曰洪水桓温以太和四年率衆北入掘渠通濟義熙十三年劉武帝西入長安又廣其功自洪口以上又謂桓公瀆濟自是北注也}
遵考裕之族弟也劉穆之謂王鎮惡曰公今委卿以伐秦之任卿其勉之鎮惡曰吾不克關中誓不復濟江|{
	復扶又翻}
裕既行青州刺史檀祇自廣陵輒率衆至涂中掩討亡命|{
	涂讀曰滁}
劉穆之恐祇為變議欲遣軍時檀韶為江州刺史張邵曰今韶據中流道濟為軍首|{
	謂為伐秦諸軍之首}
若有相疑之跡則大府立危|{
	大府謂太尉留府其實指建康也}
不如逆遣慰勞以觀其意必無患也|{
	勞力到翻}
穆之乃止 初魏主嗣使公孫表討白亞栗斯|{
	事見上年}
曰必先與秦洛陽戍將相聞使備河南岸然後擊之表未至胡人廢白亞栗斯更立劉虎為率善王表以胡人内自攜貳勢必敗散遂不吿秦將而擊之大為虎所敗|{
	將即亮翻敗補邁翻}
士卒死傷甚衆嗣謀於羣臣曰胡叛踰年討之不克其衆繁多為患日深今盛秋不可復發兵妨民農務|{
	謂妨農收也復扶又翻}
將若之何白馬侯崔宏曰胡衆雖多無健將御之|{
	將御即亮翻下同}
終不能成大患表等諸軍不為不足但法令不整處分失宜以致敗耳|{
	處昌呂翻分扶問翻}
得大將素有威望者將數百騎往攝表軍無不克矣|{
	攝持也}
相州刺史叔孫建前在并州為胡魏所畏服|{
	胡魏猶言胡晉也}
諸將莫及可遣也嗣從之以建為中領軍督表等討虎九月戊午大破之斬首萬餘級虎及司馬順宰皆死俘其衆十萬餘口 太尉裕至彭城加領徐州刺史以太原王玄謨為從事史|{
	裕領徐州以玄謨為徐州從事史漢制諸州刺史皆有從事史假佐其後宋文帝用玄謨以喪師至孝武之初義宣臧質之變卒賴以寧則裕之用人猶有漢高祖諸葛孔明之讖唐太宗託徐世勣喜薛仁貴未足以進此也}
初王廞之敗也|{
	事見一百九卷隆安元年廞許今翻}
沙門曇永匿其幼子華|{
	曇徒含翻}
使提衣襆自隨|{
	襆防玉翻帕也以裹衣物魏舒襆被而出韓文襆被入直皆此義也}
津邏疑之曇永呵華曰奴子何不速行棰之數十|{
	邏郎佐翻棰止蘂翻}
由是得免遇赦還吳以其父存亡不測布衣蔬食絶交遊不仕十餘年裕聞華賢欲用之乃發廞喪使華制服服闋辟為徐州主簿|{
	裕用王華亦留以遺文帝闋苦穴翻}
王鎮惡檀道濟入秦境所向皆捷秦將王苟生以漆丘降鎮惡|{
	漆丘蓋在梁郡蒙縣晋莊周為蒙漆園吏後人因以漆丘名城將即亮翻降戶江翻下同}
徐州刺史姚掌以項城降道濟諸屯守皆望風欵附惟新蔡太守董遵不下|{
	新蔡縣漢屬汝南郡蔡平侯自蔡徙此故曰新蔡魏分屬汝隂郡晉惠帝分汝隂立新蔡郡}
道濟攻拔其城執遵殺之進克許昌獲秦潁川太守姚垣及大將楊業沈林子自汴入河襄邑人董神虎聚衆千餘人來降太尉裕板為參軍林子與神虎共攻倉垣克之秦兖州刺史韋華降神虎擅還襄邑林子殺之秦東平公紹言於秦主泓曰晉兵已過許昌安定孤遠難以救衛宜遷其鎭戶内實京畿可得精兵十萬|{
	姚萇之興也以安定為根本後得關中以安定為重鎮徙民以實之謂之鎮戶}
雖晉夏交侵猶不亡國不然晉攻豫州夏攻安定將若之何|{
	夏戶雅翻}
事機已至宜在速決左僕射梁喜曰齊公恢有威名為嶺北所憚鎮人已與勃勃深仇|{
	謂鎮兵常與勃勃血戰有父兄子弟之仇}
理應守死無貳勃勃終不能越安定遠宼京畿若無安定虜馬必至於郿|{
	郿音媚又音眉}
今關中兵足以拒晉無為豫自損削也泓從之吏部郎懿横密言於泓曰恢於廣平之難有忠勲於陛下|{
	姓譜曰懿以諡為氏謂殺呂超也難乃旦翻}
自陛下龍飛紹統未有殊賞以答其意今外則置之死地内則不豫朝權|{
	朝直遙翻}
安定人自以孤危逼寇思南遷者十室而九若恢擁精兵數萬鼓行而向京師得不為社稷之累乎|{
	累力瑞翻}
宜徵還朝廷以慰其心泓曰恢若懷不逞之心徵之適所以速禍耳又不從王仲德水軍入河將逼滑臺魏兖州刺史尉建畏懦帥衆棄城北渡河|{
	帥讀曰率}
仲德入滑臺宣言曰晉本欲以布帛七萬匹假道于魏不謂魏之守將棄城遽去|{
	將即亮翻}
魏主嗣聞之遣叔孫建公孫表自河内向枋頭|{
	既破劉虎因遣建等引兵南向枋音方}
因引兵濟河斬尉建於城下投尸於河呼仲德軍人問以侵寇之狀仲德使司馬竺和之對曰劉太尉使王征虜自河入洛清掃山陵非敢為寇於魏也魏之守將自棄滑臺去王征虜借空城以息兵行當西引於晉魏之好無廢也|{
	好呼到翻下好持同}
何必揚旗鳴鼓以耀威乎嗣使建以問太尉裕裕遜辭謝之曰洛陽晉之舊都而羌據之晉欲修復山陵久矣諸桓宗族司馬休之國璠兄弟魯宗之父子皆晉之蠧也而羌收之以為晉患|{
	義熙元年桓謙等奔秦六年入寇十一年司馬休之魯宗之等奔秦秦使將兵擾襄陽六年司馬國璠等奔秦數帥衆擾邊璠孚袁翻}
今晉將伐之欲假道于魏非敢為不利也魏河内鎮將于栗磾有勇名築壘於河上以備侵軼|{
	將即亮翻磾丁奚翻軼宜結翻突也陸德明曰又音逸}
裕以書與之題曰黑矟公麾下栗磾好操黑矟以自標故裕以此目之魏因拜栗磾為黑矟將軍|{
	矟色角翻通俗文矛長丈八者謂之矟}
冬十月壬戌魏主嗣如豺山宫初燕將庫傉官斌降魏既而復叛歸燕|{
	將即亮翻傉奴沃翻斌音}


|{
	彬降戶江翻下同復扶又翻}
魏主嗣遣驍騎將軍延普渡濡水擊斌斬之|{
	水經濡水從塞外來過遼西令支縣北又東南過海陽縣西南入于海驍堅堯翻騎奇寄翻魏書官氏志神元時餘部諸姓内入者可地延氏孝文時改為延氏濡乃官翻}
遂攻燕幽州刺史庫傉官昌征北將軍庫傉官提皆斬之 秦陽城滎陽二城皆降晉兵進至成臯秦征南將軍陳留公洸鎮洛陽|{
	洸姑黄翻}
遣使求救于長安秦主泓遣越騎校尉閻生帥騎三千救之|{
	使疏吏翻帥讀曰率}
武衛將軍姚益男將步卒一萬助守洛陽又遣并州牧姚懿南屯陜津|{
	陜縣在大河之南攷之水經則陜縣故城在大河之北二城之間謂之陜津左傳秦伯伐晉自茅津濟封殽尸而還茅津即陜津也姚秦并冀二州治蒲阪陜式冉翻}
為之聲援寧朔將軍趙玄言於洸曰今晉宼益深人情駭動衆寡不敵若出戰不捷則大事去矣宜攝諸戍之兵固守金墉以待西師之救金墉不下晉必不敢越我而西是我不戰而坐收其弊也司馬姚禹隂與檀道濟通主簿閻恢楊䖍皆禹之黨也共嫉玄言於洸曰殿下以英武之畧受任方面今嬰城示弱得無為朝廷所責乎洸以為然乃遣趙玄將兵千餘南守柏谷塢|{
	水經注洛水東逕偃師縣南又東逕百谷塢北戴延之西征記曰塢在川南因高為塢高一十餘丈杜佑曰柏谷塢在緱氏縣東北}
廣武將軍石無諱東戍鞏城玄泣謂洸曰玄受三帝重恩所守正有死耳|{
	萇興泓為三帝}
但明公不用忠臣之言為姦人所誤後必悔之既而成臯虎牢皆來降|{
	降戶江翻下同}
檀道濟等長驅而進無諱至石關犇還|{
	自洛城東至偃師四十五里偃師西山有漢廣野君酈食其廟廟東有二石闕}
龍驤司馬滎陽毛德祖與玄戰于栢谷玄兵敗被十餘創據地大呼|{
	驤思將翻被皮義翻創初良翻呼火故翻}
玄司馬蹇鑒冒刃抱玄而泣玄曰吾創已重|{
	創初良翻}
君宜速去鑒曰將軍不濟鑒去安之與之皆死姚禹踰城犇道濟甲子道濟進逼洛陽丙寅洸出降道濟獲秦人四千餘人議者欲盡阬之以為京觀|{
	杜預曰積尸封土其上謂之京觀觀古亂翻}
道濟曰伐罪弔民正在今日皆釋而遣之於是夷夏感悦歸之者甚衆|{
	夏戶雅翻}
閻生姚益男未至聞洛陽已没不敢進己丑詔遣兼司空高密王恢之修謁五陵置守衛|{
	彭城王紘之子俊嗣高密王畧國恢之其孫也五陵宣帝陵在河隂曰高原景帝陵曰峻平文帝陵曰崇陽武帝陵曰峻陽惠帝陵曰太陽}
太尉裕以冠軍將軍毛脩之為河南河内二郡太守行司州事戍洛陽|{
	冠古玩翻}
西秦王熾磐使秦州刺史王松夀鎮馬頭以逼秦之上邽|{
	丁度曰嶓冢山在古上邽縣西有神馬山}
十一月甲戌魏主嗣還平城太尉裕遣左長史王弘還建康諷朝廷求九錫時劉穆之掌留任而旨從北來穆之由是愧懼發病|{
	劉穆之輔劉裕豈惟才智不及荀或而識又不及焉}
弘珣之子也|{
	王珣始見重於桓温後為孝武所親任}
十二月壬申詔以裕為相國總百揆揚州牧封十郡為宋公備九錫之禮位在諸侯王上領征西將軍司豫北徐雍四州刺史如故|{
	雍於用翻}
裕辭不受 西秦王熾磐遣使詣太尉裕|{
	使疏吏翻下同}
求擊秦以自効裕拜熾磐平西將軍河南公 秦姚懿司馬孫暢說懿使襲長安|{
	說輸芮翻}
誅東平公紹廢秦主泓而代之懿以為然乃散穀以賜河北夷夏|{
	河北縣自漢以來屬河東郡在蒲阪東時夷夏之民錯居之懿至陜津因散穀以賜焉夏戶雅翻}
欲樹私恩左常侍張敞侍郎左雅諫曰|{
	左常侍侍郎皆懿國官}
殿下以母弟居方面安危休戚與國同之今吳寇内侵四州傾没|{
	秦徐州鎮項城兖州鎮倉垣豫州鎮洛陽荆州當鎮上洛時悉為晉所取}
西虜擾邊秦涼覆敗|{
	謂赫連勃勃克上邽沮渠蒙遜入姑臧}
朝廷之危有如累卵穀者國之本也而殿下無故散之虛損國儲將若之何懿怒笞殺之泓聞之召東平公紹密與之謀紹曰懿性識鄙淺從物推移造此謀者必孫暢也但馳使徵暢遣撫軍將軍讚據陜城臣向潼關為諸軍節度若暢奉詔而至臣當遣懿帥河東見兵共禦晉師|{
	帥讀曰率見賢遍翻}
若不受詔命便當聲其罪而討之泓曰叔父之言社稷之計也乃遣姚讃及冠軍將軍司馬國璠建義將軍虵玄屯陜津|{
	冠古玩翻璠孚袁翻虵以者翻又如字}
武衛將軍姚驢屯潼關懿遂舉兵稱帝傳檄州郡欲運匈奴堡穀以給鎮人|{
	匈奴堡在平陽鎮人懿鎮蒲阪所領之衆也}
寧東將軍姚成都拒之懿卑辭誘之送佩刀為誓成都不從|{
	誘音酉}
懿遣驍騎將軍王國帥甲士數百攻成都成都擊禽之遣使讓懿曰明公以至親當重任國危不能救而更圖非望三祖之靈其肯佑明公乎|{
	姚弋仲廟號始祖萇廟號太祖興廟號高祖所謂三祖也}
成都將糾合義兵往見明公於河上耳|{
	蒲阪臨河故曰河上}
於是傳檄諸城諭以逆順徵兵調食以討懿|{
	調徒釣翻}
懿亦發諸城兵莫有應者惟臨晉數千戶應懿成都引兵濟河擊臨晉叛者破之鎮人安定郭純等起兵圍懿東平公紹入蒲阪執懿誅孫暢等 是歲魏衛將軍安城孝元王叔孫俊卒魏主嗣甚惜之謂其妻桓氏曰生同其榮能沒同其戚乎桓氏乃縊而祔焉|{
	嗣之立也叔孫俟有功事見一百一十五卷四年}
丁零翟猛雀驅掠吏民入白澗山為亂|{
	白澗山當在漢河東濩澤縣西水經註濩澤水出濩澤城西白澗嶺東逕濩澤濩澤唐澤州陽城縣即其地師古曰濩音烏虢翻}
魏内都大官河内張蒲與冀州刺史長孫道生討之道生嵩之從子也|{
	長知兩翻從才用翻}
道生欲進兵擊猛雀蒲曰吏民非樂為亂為猛雀所迫脅耳今不分别并擊之|{
	樂音洛别彼列翻}
雖欲返善其道無由必同心協力據險以拒官軍未易猝平也|{
	易以䜴翻}
不如先遣使諭之以不與猛雀同謀者皆不坐則必喜而離散矣|{
	使疏吏翻}
道生從之降者數千家使復舊業猛雀與其黨百餘人出走蒲等追斬猛雀首左部尚書周幾窮討餘黨悉誅之

資治通鑑卷一百十七
















































































































































