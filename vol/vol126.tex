<!DOCTYPE html PUBLIC "-//W3C//DTD XHTML 1.0 Transitional//EN" "http://www.w3.org/TR/xhtml1/DTD/xhtml1-transitional.dtd">
<html xmlns="http://www.w3.org/1999/xhtml">
<head>
<meta http-equiv="Content-Type" content="text/html; charset=utf-8" />
<meta http-equiv="X-UA-Compatible" content="IE=Edge,chrome=1">
<title>資治通鑒_127-資治通鑑卷一百二十六_127-資治通鑑卷一百二十六</title>
<meta name="Keywords" content="資治通鑒_127-資治通鑑卷一百二十六_127-資治通鑑卷一百二十六">
<meta name="Description" content="資治通鑒_127-資治通鑑卷一百二十六_127-資治通鑑卷一百二十六">
<meta http-equiv="Cache-Control" content="no-transform" />
<meta http-equiv="Cache-Control" content="no-siteapp" />
<link href="/img/style.css" rel="stylesheet" type="text/css" />
<script src="/img/m.js?2020"></script> 
</head>
<body>
 <div class="ClassNavi">
<a  href="/24shi/">二十四史</a> | <a href="/SiKuQuanShu/">四库全书</a> | <a href="http://www.guoxuedashi.com/gjtsjc/"><font  color="#FF0000">古今图书集成</font></a> | <a href="/renwu/">历史人物</a> | <a href="/ShuoWenJieZi/"><font  color="#FF0000">说文解字</a></font> | <a href="/chengyu/">成语词典</a> | <a  target="_blank"  href="http://www.guoxuedashi.com/jgwhj/"><font  color="#FF0000">甲骨文合集</font></a> | <a href="/yzjwjc/"><font  color="#FF0000">殷周金文集成</font></a> | <a href="/xiangxingzi/"><font color="#0000FF">象形字典</font></a> | <a href="/13jing/"><font  color="#FF0000">十三经索引</font></a> | <a href="/zixing/"><font  color="#FF0000">字体转换器</font></a> | <a href="/zidian/xz/"><font color="#0000FF">篆书识别</font></a> | <a href="/jinfanyi/">近义反义词</a> | <a href="/duilian/">对联大全</a> | <a href="/jiapu/"><font  color="#0000FF">家谱族谱查询</font></a> | <a href="http://www.guoxuemi.com/hafo/" target="_blank" ><font color="#FF0000">哈佛古籍</font></a> 
</div>

 <!-- 头部导航开始 -->
<div class="w1180 head clearfix">
  <div class="head_logo l"><a title="国学大师官网" href="http://www.guoxuedashi.com" target="_blank"></a></div>
  <div class="head_sr l">
  <div id="head1">
  
  <a href="http://www.guoxuedashi.com/zidian/bujian/" target="_blank" ><img src="http://www.guoxuedashi.com/img/top1.gif" width="88" height="60" border="0" title="部件查字,支持20万汉字"></a>


<a href="http://www.guoxuedashi.com/help/yingpan.php" target="_blank"><img src="http://www.guoxuedashi.com/img/top230.gif" width="600" height="62" border="0" ></a>


  </div>
  <div id="head3"><a href="javascript:" onClick="javascript:window.external.AddFavorite(window.location.href,document.title);">添加收藏</a>
  <br><a href="/help/setie.php">搜索引擎</a>
  <br><a href="/help/zanzhu.php">赞助本站</a></div>
  <div id="head2">
 <a href="http://www.guoxuemi.com/" target="_blank"><img src="http://www.guoxuedashi.com/img/guoxuemi.gif" width="95" height="62" border="0" style="margin-left:2px;" title="国学迷"></a>
  

  </div>
</div>
  <div class="clear"></div>
  <div class="head_nav">
  <p><a href="/">首页</a> | <a href="/ShuKu/">国学书库</a> | <a href="/guji/">影印古籍</a> | <a href="/shici/">诗词宝典</a> | <a   href="/SiKuQuanShu/gxjx.php">精选</a> <b>|</b> <a href="/zidian/">汉语字典</a> | <a href="/hydcd/">汉语词典</a> | <a href="http://www.guoxuedashi.com/zidian/bujian/"><font  color="#CC0066">部件查字</font></a> | <a href="http://www.sfds.cn/"><font  color="#CC0066">书法大师</font></a> | <a href="/jgwhj/">甲骨文</a> <b>|</b> <a href="/b/4/"><font  color="#CC0066">解密</font></a> | <a href="/renwu/">历史人物</a> | <a href="/diangu/">历史典故</a> | <a href="/xingshi/">姓氏</a> | <a href="/minzu/">民族</a> <b>|</b> <a href="/mz/"><font  color="#CC0066">世界名著</font></a> | <a href="/download/">软件下载</a>
</p>
<p><a href="/b/"><font  color="#CC0066">历史</font></a> | <a href="http://skqs.guoxuedashi.com/" target="_blank">四库全书</a> |  <a href="http://www.guoxuedashi.com/search/" target="_blank"><font  color="#CC0066">全文检索</font></a> | <a href="http://www.guoxuedashi.com/shumu/">古籍书目</a> | <a   href="/24shi/">正史</a> <b>|</b> <a href="/chengyu/">成语词典</a> | <a href="/kangxi/" title="康熙字典">康熙字典</a> | <a href="/ShuoWenJieZi/">说文解字</a> | <a href="/zixing/yanbian/">字形演变</a> | <a href="/yzjwjc/">金 文</a> <b>|</b>  <a href="/shijian/nian-hao/">年号</a> | <a href="/diming/">历史地名</a> | <a href="/shijian/">历史事件</a> | <a href="/guanzhi/">官职</a> | <a href="/lishi/">知识</a> <b>|</b> <a href="/zhongyi/">中医中药</a> | <a href="http://www.guoxuedashi.com/forum/">留言反馈</a>
</p>
  </div>
</div>
<!-- 头部导航END --> 
<!-- 内容区开始 --> 
<div class="w1180 clearfix">
  <div class="info l">
   
<div class="clearfix" style="background:#f5faff;">
<script src='http://www.guoxuedashi.com/img/headersou.js'></script>

</div>
  <div class="info_tree"><a href="http://www.guoxuedashi.com">首页</a> > <a href="/SiKuQuanShu/fanti/">四库全书</a>
 > <h1>资治通鉴</h1> <!--         下载:【右键另存为】即可 --></div>
  <div class="info_content zj clearfix">
  
<div class="info_txt clearfix" id="show">
<center style="font-size:24px;">127-資治通鑑卷一百二十六</center>
    資治通鑑卷一百二十六 宋 司馬光 撰<br />
<br />
  胡三省 音註<br />
<br />
  宋紀八【起重光單閼盡玄黓執徐凡二年】<br />
<br />
  太祖文皇帝下之上<br />
<br />
  元嘉二十八年春正月丙戌朔魏主大會羣臣於瓜步山上班爵行賞有差魏人緣江舉火太子左衛率尹弘言於上曰六夷如此必走【北兵欲退慮南兵之追截故舉火以示威尹弘習知北人軍情因言于上自晉氏失馭劉石以來始有六夷之名率所律翻】丁亥魏掠居民焚廬舍而去胡誕世之反也【見上卷二十四年】江夏王義恭等奏彭城王義康數有怨言揺動民聽故不逞之族因以生心【夏戶雅翻數所角翻不逞之族謂廢放之家不得逞志于時者也】請徙義康廣州上將徙義康先遣使語之【使疏吏翻語牛倨翻】義康曰人生會死吾豈愛生必為亂階雖遠何益請死於此恥復屢遷【復扶又翻屢力住翻又如字】竟未及往魏師至瓜步人情忷懼【忷許拱翻】上慮不逞之人復奉義康為亂太子劭及武陵王駿尚書左僕射何尚之屢啓宜早為之所【武陵王駿時在彭城蓋馳密啓言之也】上乃遣中書舍人嚴龍齎藥賜義康死義康不肯服曰佛教不許自殺【佛教謂自殺者不復得人身】願随宜處分【處昌呂翻分扶問翻】使者以被揜殺之 江夏王義恭以碻磝不可守召王玄謨還歷城魏人追擊敗之遂取碻磝【敗蒲賣翻去年蕭斌使王玄謨戍碻磝】初上聞魏將入寇命廣陵太守劉懷之逆燒城府船乘【守手又翻敵未至而先燒故曰逆乘謂車也音繩證翻】盡帥其民渡江【帥讀曰率】山陽太守蕭僧珍悉斂其民入城臺送糧仗詣盱眙及滑臺者以路不通皆留山陽【晉安帝義熙中土斷分廣陵立山陽郡境内有地名山陽因以名郡今楚州即其地盱眙音吁怡】蓄陂水令滿須魏人至決以灌之【須待也】魏人過山陽不敢留因攻盱眙魏主就臧質求酒質封溲便與之【溲踈鳩翻便毗連翻】魏主怒築長圍一夕而合運東山土石以填塹作浮橋於君山絶水陸道【今盱眙縣北七里有長圍山圖經云臧質守盱眙魏太武于都梁山築長城造浮橋絶水路即此塹七艷翻】魏主遺質書曰【遺于季翻】吾今所遣鬬兵盡非我國人【國人謂與拓拔氏同出北荒之子孫也凡九十九姓】城東北是丁零與胡南是氐羌設使丁零死正可減常山趙郡賊【丁零自翟真叛慕容皆投常山趙郡界阻山而居故云然】胡死減并州賊【自後漢納南匈奴分居并州界其地率皆雜處胡漢西河離石以西則皆稽胡據之為寇】氐羌死減關中賊【自苻姚據關中其種類蕃滋雖有國已滅而其種實繁】卿若殺之無所不利【言於魏國無所不利】質復書曰省示具悉姦懷【省示省來書所示也悉詳也盡也省悉景翻】爾自恃四足屢犯邉【恃四足謂負戎馬足也】王玄謨退於東申坦散於西【按王玄謨自滑臺敗退蕭斌使申坦據清口戴延之所謂清口在夀張縣西界安民亭南以水經注考之其地不在滑臺之西此當謂梁坦出上蔡之師至虎牢潰散耳】爾知其所以然邪爾獨不聞童謡之言乎蓋卯年未至故以二軍開飲河之路耳冥期使然非復人事【謂冥冥之中大期將至天使之然非由人事為之也復扶又翻下容復同】寡人受命相滅【古者諸侯自稱曰寡人質自以當藩方之任自稱寡人】期之白登師行未遠爾自送死豈容復令爾生全饗有桑乾哉【白登山桑乾川皆在平城左右質言本期直指白登師行至淮而逢魏兵要當勦滅不容令魏主生歸饗有桑乾之地也此嫚書也兩陣相向惡聲至必反之毋庸以此為據也乾音干】爾有幸得為亂兵所殺不幸則生相鎖縛載以一驢直送都市耳我本不圖全若天地無靈力屈於爾韲之粉之【細切薑韮謂之韲研碎米麥謂之粉韲牋西翻】屠之裂之猶未足以謝本朝【朝直遥翻】爾智識及衆力豈能勝苻堅邪今春雨已降兵方四集爾但安意攻城勿遽走糧食乏者可見語【語牛倨翻下爾語同】當出廩相貽得所送劍刃欲令我揮之爾身邪魏主大怒作鐵床於其上施鐵鑱【鑱士衫翻又士懴翻刺也錐也】曰破城得質當坐之此上質又與魏衆書曰爾語虜中諸士庶佛狸所與書相待如此【以魏主書言其兵鬭死正減國中賊也因而擕之術莫近乎此矣魏主得質此書豈不悔前所與質書乎】爾等正朔之民何為自取糜滅豈可不知轉禍為福邪【中原之民本禀漢晉正朔故謂之正朔之民】并寫臺格以與之云斬佛狸首封萬戶侯賜布絹各萬匹【臺格宋臺所立賞格也佛讀如弼】魏人以鉤車鉤城樓城内繫以彄絙【彄格侯翻絙古恒翻大索也】數百人叫呼引之車不能退既夜縋桶懸卒出截其鉤獲之【縋馳偽翻桶他董翻箍木為之】明旦又以衝車攻城城土堅密每至【句絶謂衝車至著城身也】頹落不過數升魏人乃肉薄登城分番相代墜而復升【復扶又翻】莫有退者殺傷萬計尸與城平凡攻之三旬不拔會魏軍中多疾疫或告以建康遣水軍自海入淮【水軍自建康下江自江出海轉料角則入淮】又勑彭城斷其歸路【斷丁管翻】二月丙辰朔魏主燒攻具退走盱眙人欲追之沈璞曰今兵不多雖可固守不可出戰但整舟楫示若欲北渡者【示若欲自盱眙渡淮而北以追截其後者】以速其走計不須實行也臧質以璞城主使之上露板【露板者書獲捷之狀露板上聞使天下悉知之也上時掌翻】璞固辭歸功於質上聞益嘉之【已嘉璞之功又益嘉其讓】魏師過彭城江夏王義恭震懼不敢擊【夏戶雅翻】或告虜驅南口萬餘夕應宿安王陂去城數十里今追之可悉得諸將皆請行義恭禁不許明日驛使至【使疏吏翻】上勑義恭悉力急追魏師已遠義恭乃遣鎮軍司馬檀和之向蕭城魏人先已聞之盡殺所驅者而去程天祚逃歸【天祚為魏所禽見上卷二十六年】魏人凡破南兖徐兖豫青冀六州【殘破六州之生聚耳六州城守未嘗失也】殺傷不可勝計【勝音升】丁壯者即加斬截嬰兒貫于槊上【槊色角翻】盤舞以為戲所過郡縣赤地無餘春燕歸巢於林木【室廬焚蕩燕無所歸故巢林木】魏之士馬死傷亦過半國人皆尤之上每命將出師常授以成律交戰日時亦待中詔是以將帥趑趄莫敢自决【將即亮翻帥所類翻趑取私翻趄七余翻趑趄不進也】又江南白丁輕易進退【易以䜴翻】此其所以敗也自是邑里蕭條元嘉之政衰矣【史言亟用兵之禍】癸酉詔賑恤郡縣民遭寇者蠲其稅調【賑津忍翻蠲工玄翻調徒釣翻】甲戌降太尉義恭為驃騎將軍開府儀同三司【驃匹妙翻騎奇寄翻】戊寅魏主濟河【自丙辰盱眙退師一十三日始濟河】辛巳降鎮軍將軍武陵王駿為北中郎將壬午上如瓜步是日解嚴初魏中書學生盧度世玄之子也【魏神䴥四年徵盧玄】坐崔浩事亡命匿高陽鄭羆家【崔浩事見上卷二十七年高陽縣前漢屬涿郡後漢屬河間國晉分屬高陽郡】吏囚羆子掠治之【掠音亮】羆戒其子曰君子殺身成仁【論語載孔子之言】雖死不可言其子奉父命吏以火爇其體【爇如悦翻】終不言而死及魏主臨江上遣殿上將軍黄延年使於魏【自晉以來有殿中將軍殿上將軍當是宋所置使疏吏翻】魏主問曰盧度世亡命已應至彼延年曰都下不聞有度世也魏主乃赦度世及其族逃亡籍没者【凡度世之族逃亡而籍没其家者並赦之】度世自出魏主以為中書侍郎 【考異曰宋柳元景傳元景從祖弟光世先留鄉里索虜以為折衝將軍河北太守封西陵男光世姊夫為司徒崔浩虜之相也元嘉二十七年虜主拓拔燾南寇汝潁浩密有異圖光世要河北義士與浩應接謀泄被誅河東大姓坐連謀夷族者甚衆光世南奔得免太祖以為振武將軍與魏事不同今從魏書】度世為其弟娶鄭羆妹以報德【為于偽翻】三月乙酉帝還宫己亥魏主還平城【魏主戊寅濟河行二十二日至平城】飲至告廟【左傳凡公行告于廟反行飲至舍爵策勲焉禮也又曰三年而治兵入而振旅歸而飲至以數軍實杜預注曰飲于廟以數車徒器械及所獲也】以降民五萬餘家分置近畿【近畿謂還平城千里之地降戶江翻】初魏主過彭城遣人語城中曰【語牛倨翻】食盡且去須麥熟更來及期江夏王義恭議欲芟麥翦苗移民堡聚【芟所衘翻】鎮軍録事參軍王孝孫曰【白氏六帖曰州主簿郡督郵並今録事參軍余按晉瑯邪王睿都督揚州以陳頵為録事參軍當時自别有州主簿督郵之吏亦猶存古而録事之職掌正違失蒞符印】虜不能復來【復扶又翻】既自可保如其更至此議亦不可立百姓閉在内城飢饉日久方春之月野採自資一入堡聚餓死立至民知必死何可制邪虜若必來芟麥無晚四坐默然莫之敢對【坐徂卧翻】長史張暢曰孝孫之議實有可尋【尋繹理也用也左傳將尋師焉又曰日尋干戈杜預注皆云尋用也】鎮軍府典籖董元嗣侍武陵王駿之側進曰王録事議不可奪别駕王子夏曰此論誠然暢斂板白駿曰【板手板僚佐于府公之前斂板白事崇敬也】下官欲命孝孫彈子夏【録事參軍掌糾彈故云然彈徒丹翻】駿曰王别駕有何事邪暢曰芟麥移民可謂大議一方安危事繫於此子夏親為州端【州别駕居羣僚之右故曰州端】曾無同異及聞元嗣之言則懽笑酬答阿意左右何以事君子夏元嗣皆大慙義恭之議遂寢 初魯宗之奔魏【晉安帝義熙十一年魯宗之自襄陽奔秦十三年秦亡奔魏】其子軌為魏荆州刺史襄陽公鎮長社常思南歸以昔殺劉康祖及徐湛之父【劉康祖父䖍之徐湛之父達之義熙十一年為魯軌所殺】故不敢來軌卒子爽襲父官爵爽少有武幹【少詩照翻】與弟秀皆有寵於魏主既而兄弟各有罪魏主詰責之【爽麤中使酒多過失秀以檢校鄴人謀反事因病還遲並為魏主所詰責詰去吉翻】爽秀懼誅從魏主自瓜步還至湖陸請曰奴與南有仇每兵來常恐禍及墳墓【爽祖父皆葬長社】乞共迎喪還葬平城魏主許之爽至長社殺魏戍兵數百人帥部曲及願從者千餘家奉汝南【自長社至汝南不及三百里帥讀曰率】夏四月爽遣秀詣夀陽奉書於南平王鑠以請降【鑠式灼翻降戶江翻】上聞之大喜以爽為司州刺史鎮義陽【沈約曰司州刺史漢之司隸校尉也晉江左以來淪没戎寇雖永和大元王化蹔及及大和隆安還復湮䧟武帝北平關洛河南底定置司州刺史治虎牢領河南滎陽弘農實土三郡少帝景平初司州復没元嘉末僑立治汝南是後遂治羲陽領義陽随陽安陸南汝南郡】秀為潁川太守 【考異曰宋畧云滎陽郡太守今從宋書予謂帝蓋以秀兄弟自潁川來降遂因以潁川太守授秀】餘弟姪並授官爵賞賜甚厚魏人毁其墳墓徐湛之以為廟筭遠圖特所奬納不敢苟申私怨乞屏居田里不許【屏必郢翻】 青州民司馬順則自稱晉室近屬聚衆號齊王梁鄒戌主崔勲之詣州五月乙酉順則乘虛襲梁鄒城【梁鄒縣漢屬濟南郡晉省宋置梁鄒戍為平原太守治所水經注濟水自管縣東過梁鄒縣北又東北過臨濟縣南參而考之其地在唐齊州臨濟縣界】又有沙門自稱司馬百年亦聚衆號安定王以應之 壬寅魏大赦 己巳以江夏王義恭領南兖州刺史徙鎮盱眙增督十二州諸軍事 戊申以尚書左僕射何尚之為尚書令太子詹事徐湛之為僕射護軍將軍【晉志曰自魏晉迄于江左僕射置二則分左右或不兩置但曰尚書僕射令闕則左為省主若左右並闕則置尚書僕射以主左事今湛之蓋以尚書僕射領護軍將軍也】尚之以湛之國戚【湛之帝之甥會稽公主之子】任遇隆重每事推之詔湛之與尚之並受辭訴尚之雖為令而朝事悉歸湛之【朝直遥翻】 六月壬戌魏改元正平 魏主命太子少傅游雅中書侍郎胡方回等更定律令多所增損凡三百九十一條 魏太子晃監國【監工衘翻】頗信任左右又營園田收其利高允諫曰天地無私故能覆載【覆敷又翻】王者無私故能容養今殿下國之儲貳萬方所則而營立私田畜養鷄犬【畜許六翻】乃至酤販市㕓與民爭利【㕓市中空地一曰居也說文曰㕓一畝半一家之居也孔頴達曰市㕓而不税者㕓謂公家邸舍使商人停物于中直税其所舍之處不稅其在市所賣之物市内空地曰㕓城内空地曰肆按載師云以㕓里任國中之地鄭注云㕓里邑居里矣㕓民居之區城也司農云㕓市中空地未有肆城中空地未有宅者也遂人授民田夫一㕓田百畝詩胡取禾三百㕓兮傳云一夫之居曰㕓謂一夫之田百畝也揚子云有田一㕓謂百畝之居與詩傳同夫田之㕓與市㕓之㕓其義不同各有攸當也】謗聲流布不可追掩夫天下者殿下之天下富有四海何求而無乃與販夫販婦競此尺寸之利乎昔虢之將亡神賜之土田【注見前】漢靈帝私立府藏【事見五千七卷光和元年藏徂浪翻】皆有顛覆之禍前鑒若此甚可畏也武王愛周召齊畢所以王天下【史記周紀武王即位太公望為師周公旦為輔召公畢公之徒左右王師一戎衣而天下大定王于况翻】殷紂愛飛廉惡來所以喪其國【飛廉多力惡來善走父子俱以才力事紂惡來善毀讒諸侯以此益疏喪息浪翻】今東宫儁乂不少頃來侍御左右者恐非在朝之選【少詩沼翻朝直遥翻】願殿下斥去佞邪親近忠良【去羌呂翻近其靳翻】所在田園分給貧下販賣之物以時收散【收謂收藏其物散謂散與貧民一曰以時收散者言穫斂之時民力可以償稱通負則收之停滯居物至民所欲得之時則散之】如此則休聲日至謗議可除矣不聽太子為政精察而中常侍宗愛性險暴多不法太子惡之【惡烏路翻】給事中仇尼道盛侍郎任平城【侍郎即給事黄門侍郎仇尼複姓出於徒河任音壬】有寵於太子頗用事皆與愛不協愛恐為道盛等所糾遂構告其罪魏主怒斬道盛等於都街【都街即都市】東宫官屬多坐死帝怒甚戊辰太子以憂卒 【考異曰宋索虜傳云燾至汝南瓜步晃私遣取諸營鹵獲甚衆燾歸聞知大加搜撿晃懼謀殺燾燾乃詐死使其近習召晃迎喪於道執之及國罩以鉄籠尋殺之蕭子顯齊書亦云晃謀殺佛狸見殺宋畧曰燾既南侵晃淫于内謀欲殺燾燾知之歸而詐死召晃迎喪晃至執之罩以鐵籠捶之三百曳于樷棘以殺焉又索虜傳云晃弟秦王烏奕盱與晃對掌國事晃疾之訴其貪暴燾鞭之二百遣鎮枹罕此皆江南傳聞之誤今從後魏書】壬申葬金陵諡曰景穆帝徐知太子無臯甚悔之【為後宗愛弑帝張本】 秋七月丁亥魏主如隂山青冀二州刺史蕭斌遣振武將軍劉武之等擊司馬順則司馬百年皆斬之【斌音彬】癸亥梁鄒平 蕭斌王玄謨皆坐退敗免官上問沈慶之曰斌欲斬玄謨而卿止之何也對曰諸將奔退莫不懼罪【將即亮翻】自歸而死將至逃散故止之【慶之諫斬玄謨事見上卷上年】 九月癸巳魏主還平城冬十月庚申復如隂山【復扶又翻】 上遣使至魏魏遣殿中將軍郎法祐來修好【好呼到翻】 己巳魏上黨靖王長孫道生卒 十二月丁丑魏主封景穆太子之子濬為高陽王既而以皇孫世嫡不當為藩王乃止【觀此則魏世祖立孫之意定矣】時濬生四年聰達過人魏主愛之常置左右徙秦王翰為東平王燕王譚為臨淮王楚王建為廣陽王吳王余為南安王【翰等皆魏主子以國王徙封郡王當考】帝使沈慶之徙彭城流民數千家于瓜步征北參軍程天祚徙江西流民數千家於姑孰【彭城江西流民皆避魏寇而南者】 帝以吏部郎王僧綽為侍中僧綽曇首之子也【曇首輔政于元嘉之初曇徒含翻】幼有大成之度衆皆以國器許之好學有思理【好呼到翻思相吏翻思理猶言思致也】練悉朝典【朝直遥翻下同】尚帝女東陽獻公主在吏部諳悉人物舉拔咸得其分【諳烏含翻分扶問翻言能随其分量而授任也】及為侍中年二十九沈深有局度【有局則能處事有度則能容物沈持林翻】不以才能高人帝頗以後事為念以其年少【少詩照翻】欲大相付託朝政大小皆與參焉帝之始親政事也委任王華王曇首殷景仁謝弘微劉湛次則范曄沈演之庾炳之最後江湛徐湛之何瑀之及僧綽凡十二人【何瑀之恐當作何尚之】 唐和入朝于魏魏主厚禮之【唐和鎮焉耆有撫安西域之功故厚禮之】<br />
<br />
  二十九年春正月魏所得宋民五千餘家在中山者謀叛州軍討誅之【州軍定州之軍也】冀州刺史張掖王沮渠萬年坐與叛者通謀賜死【沮子余翻】 魏世祖追悼景穆太子不已中常侍宗愛懼誅二月甲寅弑帝【年四十五諡曰太武皇帝考異曰宋書作庚申今從魏書】尚書左僕射蘭延【魏書官氏志北方諸姓烏洛蘭氏改為蘭氏】侍中和疋薛提等祕不發喪疋以皇孫濬冲幼欲立長君【疋五下翻長知兩翻】徵秦王翰置之祕室【祕室祕密之室】提以濬嫡皇孫不可廢議久不决宗愛知之自以得罪於景穆太子而素惡秦王翰【惡烏路翻】善南安王余乃密迎余自中宫便門入禁中矯稱赫連皇后令召延等【赫連皇后夏主勃勃之女也】延等以愛素賤不以為疑皆随入愛先使宦者三十人持兵伏於禁中延等入以次收縛斬之殺秦王翰於永巷而立余大赦改元承平尊皇后為皇太后以愛為大司馬大將軍太師都督中外諸軍事領中秘書封馮翊王【史言魏亂】 庚午立皇子休仁為建安王 三月辛卯魏葬太武皇帝于金陵【葬雲中金陵】廟號世祖 上聞魏世祖殂更謀北伐魯爽等復勸之【復扶又翻】上訪於羣臣太子中庶子何偃以為淮泗數州【淮泗數州謂青冀徐兖司豫也】瘡痍未復不宜輕動上不從偃尚之之子也夏五月丙申詔曰虐虜窮凶著於自昔未勞資斧已伏天誅拯溺蕩穢今其會也可符驃騎司空二府【時江夏王義恭降號驃騎將軍鎮盱眙南譙王義宣鎮江陵進位司空驃匹妙翻騎奇計翻】各部分所統【分扶問翻】東西應接歸義建績者随勞酬奬于是遣撫軍將軍蕭思話督冀州刺史張永等向碻磝魯爽魯秀程天祚將荆州甲士四萬出許洛【據魯爽傳天祚去年助戍彭城為魏所獲勸爽弟秀南歸是年遂與爽秀俱來奔故並用之將即亮翻】雍州刺史臧質帥所領趣潼關【帥讀曰率 考異曰索虜徐爰張永傳並云王玄謨亦北伐玄謨傳中不曾行蓋脫誤魏紀載六月劉義隆將檀和之寇濟州梁坦及魯安生軍于京索龎萌薛安都寇農都不言蕭思話等而宋紀亦無此數人者至七月云韓元興討之和之退梁坦安生亦走不言思話之歸宋畧有臧質遣柳元景狥蒲阪元景傳亦有之今從宋書宋畧 今按考異所謂索虜徐爰張永傳亦宋書也】永茂度之子也【張裕字茂度避武帝諱以字行】沈慶之固諫北伐上以其異議不使行青州刺史劉興祖上言以為河南阻飢【書曰黎民阻飢孔安國注曰阻難也】野無所掠脱諸城固守非旬月可拔稽留大衆轉輸方勞應機乘勢事存急速今偽帥始死【帥所類翻】兼逼暑時國内猜擾不暇遠赴愚謂宜長驅中山據其關要【自中山至代有倒馬關飛狐關】冀州以北民人尚豐兼麥已向熟因資為易【謂因敵取資於事為易易弋豉翻】嚮義之徒必應響赴若中州震動黄河以南自當消潰臣請發青冀七千兵遣將領之【將即亮翻】直入其心腹若前驅克勝張永及河南衆軍宜一時濟河使聲實兼舉並建司牧撫柔初附西拒太行北塞軍都【欲因山險置兵以苞舉相定幽冀之地行戶剛翻塞息則翻】因事指揮随宜加授【加授謂仕於魏有官者加其官未有官而能聚衆以應宋師者先授之以官】畏威欣寵人百其懷【言其懷恩百倍於常時也】若能成功清壹可待【謂河南北肅清混壹之功可待也】若不克捷不為大傷並催促裝束伏聽勑旨上意止存河南亦不從【劉興祖之言上策也上策非命世之英不可行】上又使員外散騎侍郎琅邪徐爰随軍向碻磝銜中旨授諸將方畧臨時宣示【散悉亶翻騎奇計翻】 尚書令何尚之以老請致仕退居方山【方山在建康東北有方山埭截淮立埭于山南曰方山者山形方如印】議者咸謂尚之不能固志既而詔書敦諭者數四六月戊申朔尚之復起視事【復扶又翻】御史中丞袁淑録自古隱士有迹無名者為真隱傳以嗤之【有迹無名如晨門荷蕢荷蓧野王二老漢隂丈人之類】 秋七月張永等至碻磝引兵圍之【考異曰宋畧七月壬辰永師及碻磝下又有乙酉壬辰按長歷此月丁丑朔四日庚辰六日壬午十六日壬辰疑永以庚辰壬午至碻磝非壬辰也】 壬辰徙汝隂王渾為武昌王淮陽王彧為湘東王【彧於六翻】 初潘淑妃生始興王濬 【考異曰太子劭傳云濬母卒使潘淑妃養之濬傳及宋九王傳皆云濬實潘子南史亦云淑妃養為子淑妃愛濬濬心不附今從濬本傳】元皇后性妬以淑妃有寵於上恚恨而殂【袁皇后諡曰元后殂於十七年恚於避翻】淑妃專總内政由是太子劭深惡淑妃及濬【惡烏故翻】濬懼為將來之禍乃曲意事劭劭更與之善吳興巫嚴道育【嚴道育女巫也其夫為劫坐没入奚官】自言能辟穀服食役使鬼物因東陽公主婢王鸚鵡出入主家道育謂主曰神將有符賜主主夜卧見流光若螢飛入書笥【笥相吏翻竹器也篋也圓曰簞方曰笥】開視得二青珠由是主與劭濬皆信惑之劭濬並多過失數為上所詰責使道育祈請欲令過不上聞【數所角翻詰去吉翻聞音問】道育曰我已為上天陳請【為于偽翻上時掌翻】必不泄露劭等敬事之號曰天師其後遂與道育鸚鵡及東陽主奴陳天與黄門陳慶國共為巫蠱琢玉為上形像埋於含章殿前劭補天與為隊主東陽主卒【卒子恤翻】鸚鵡應出嫁劭濬恐語泄【慮巫蠱之語泄也】濬府佐吳興沈懷遠素為濬所厚以鸚鵡嫁之為妾上聞天與領隊以讓劭曰汝所用隊主副並是奴邪劭懼以書告濬濬復書曰彼人若所為不已正可促其餘命或是大慶之漸耳【據此則弑逆之謀濬實啓之劭在都濬在京口故以書往來詳察書意則劭濬逆謀豈一朝一夕之故哉其所由來者漸矣此書乃贊决其逆謀非啓之也】劭濬相與往來書疏常謂上為彼人或曰其人謂江夏王義恭為佞人【夏戶雅翻】鸚鵡先與天與私通既適懷遠恐事泄白劭使密殺之陳慶國懼曰巫蠱事惟我與天與宣傳往來今天與死我其危哉乃具以其事白上上大驚即遣收鸚鵡封籍其家得劭濬書數百紙皆呪咀巫蠱之言【呪職救翻咀莊助翻】又得所埋玉人命有司窮治其事【治直之翻】道育亡命捕之不獲先是濬自揚州出鎮京口【十八年濬為揚州刺史出鎮京口史逸其事始先息薦翻】及廬陵王紹以疾解揚州【紹帝第五子出繼廬陵王義真後】意謂已必復得之既而上用南譙王義宣濬殊不樂乃求鎮江陵【濬求代義宣鎮江陵然義宣未及離江陵濬自京口至都則弑逆之禍矣復扶又翻下同樂音洛】上許之濬入朝【朝直遥翻】遣還京口為行留處分至京口數日而巫蠱事上惋歎彌日【處昌呂翻分扶問翻惋烏貫翻驚惋也】謂潘淑妃曰太子圖富貴更是一理虎頭復如此【濬小字虎頭】非復思慮所及汝母子豈可一日無我邪【言一日無帝則淑妃及濬將為劭所殺也】遣中使切責劭濬【使疏吏翻】劭濬惶懼無辭惟陳謝而已上雖怒甚猶未忍罪也【當斷不斷反受其亂文帝之謂也】 諸軍攻碻磝治三攻道張永等當東道濟南太守申坦等當西道揚武司馬崔訓當南道攻之累旬不拔【自帝經畧河南到彦之之出師四鎮皆斂戍北去王玄謨之出師碻磝望風而下滑臺則堅壁矣今之出師碻磝亦固守以抗張永等魏人固習知宋人之情態以為無能為也治直之翻濟子禮翻】八月辛亥夜魏人自地道潛出燒崔訓營及攻具癸丑夜又燒東圍及攻具尋復毁崔訓攻道【復扶又翻】張永夜撤圍退軍不告諸將【將即亮翻】士卒驚擾魏人乘之死傷塗地蕭思話自往增兵力攻旬餘不拔是時青徐不稔軍食乏丁卯思話命諸軍皆退屯歷城斬崔訓繫張永申坦於獄魯爽至長社魏戍主秃髠幡棄城走【秃髠恐當作秃髪魯爽父子兄弟先居長社以南兵來聲勢既盛秃髪幡恐其有内應故不能守而走】臧質頓兵近郊【謂頓兵襄陽之近郊也杜子春周禮注曰五十里為近郊百里為遠郊】不以時獨遣冠軍司馬柳元景【臧質以冠軍將軍鎮襄陽以柳元景為司馬冠古玩翻下同】帥後軍行參軍薛安都等進據洪關【水經注洛水自上洛縣東北于拒城之西北分為二水枝渠東北出為門水門水又北歷陽華之山又東北歷峽謂之鴻關水水東有城即關亭也水西有堡謂之鴻關堡帥讀曰率】梁州刺史劉秀之遣司馬馬汪與左軍中兵參軍蕭道成將兵向長安道成承之之子也【蕭道成始見於此蕭承之有復漢中之功見一百二十二卷元嘉十年將即亮翻】魏冠軍將軍封禮自浢津南渡赴弘農【水經注門水自鴻關東北流又北逕弘農縣故城東故城即故函谷關也其水側城北流而注于河河水于此有浢津之名浢音豆】九月司空高平公兒烏于屯潼關【魏書官氏志内入諸姓賀兒氏為兒氏】平南將軍黎公遼屯河内 吐谷渾王慕利延卒樹洛干之子拾寅立【樹洛干卒于晉安帝義熙十三年】始居伏羅川【居伏羅川猶未敢遠離白蘭之險也】遣使來請命亦請命于魏丁亥以拾寅為安西將軍西秦河沙三州刺史河南王魏以拾寅為鎮西大將軍沙州刺史西平王 庚寅魯爽與魏豫州刺史拓拔僕蘭戰于大索破之【杜預曰成臯東有大索城京相璠曰京縣有大索亭小索亭大小索氏兄弟居之故有大小之號括地志曰滎陽即大索城小索故城在滎陽縣北四里】進攻虎牢聞碻磝敗退與柳元景皆引兵還蕭道成馬汪等聞魏救兵將至還趣仇池【趣七喻翻】己丑詔解蕭思話徐州更領冀州刺史鎮歷城【更工衡翻】上以諸將屢出無功不可專責張永等賜思話詔曰虜既乘利方向盛冬若脱敢送死兄弟父子自共當之耳【言諸將皆不可任也】言及增憤可以示張永申坦【使示永坦以激厲之】又與江夏王義恭書曰早知諸將輩如此恨不以白刃驅之今者悔何所及【亦憤憤之辭也】義恭尋奏免思話官從之 魏南安隱王余自以違次而立【余以少子為宗愛所立非次也諡法不顯尸國曰隱】厚賜羣下欲以收衆心旬月之間府藏虛竭【藏徂浪翻】又好酣飲及聲樂畋獵不恤政事【好呼到翻】宗愛為宰相録三省【魏蓋以尚書侍中中秘書為三省亦猶今以尚書門下中書為三省也】總宿衛坐召公卿專恣日甚余患之謀奪其權愛憤怒冬十月丙午朔余夜祭東廟【魏書明元帝永興四年立太祖道武廟於白登山歲一祭具太牢無常月又於白登山西太祖舊遊之處立昭成獻明太祖廟常以九月十月之交帝親祭牲用馬牛羊白登在平城東故曰東廟】愛使小黄門賈周等就弑余而祕之【余立纔二百二十餘日】惟羽林郎中代人劉尼知之【羽林郎自漢以來有之漢羽林郎秩比三百石郎中可以槩推矣魏以劉尼為羽林郎中與殿中尚書俱典兵宿衛則其位任蓋重於漢朝也】尼勸愛立皇孫濬愛驚曰君大癡人皇孫若立豈忘正平時事乎【景穆太子之死魏正平元年也正平元年即上年】尼曰若爾今當立誰愛曰待還宫當擇諸王賢者立之尼恐愛為變密以狀告殿中尚書源賀賀時與尼俱典兵宿衛乃與南部尚書陸麗謀曰宗愛既立南安還復殺之【復扶又翻】今又不立皇孫將不利於社稷遂與麗定謀共立皇孫麗俟之子也【史言陸侯父子皆有智畧忠于後魏】戊申賀與尚書長孫渴侯嚴兵守衛宫禁使尼麗迎皇孫於苑中【魏都平城有鹿苑】麗抱皇孫於馬上入平城賀渴侯開門納之尼馳還東廟大呼曰宗愛弑南安王大逆不道【呼火故翻劉尼僅以弑南安王為宗愛罪不能正其弑世祖之罪也】皇孫已登大位有詔宿衛之士皆還宫衆咸呼萬歲遂執宗愛賈周等勒兵而入奉皇孫即皇帝位【帝諱濬太武皇帝之嫡孫景穆太子之長子也蕭子顯曰濬字烏雷直勤】登永安殿【北史魏太武帝始光二年改東宫為萬夀宫起永安安樂二殿】大赦改元興安 【考異曰宋索虜傳燾以烏弈旰有武畧用以為太子會燾死使嬖人宗愛立可博真為後宗愛博真恐為弈旰所危矯殺之而自立號年承平博真懦弱不為國人所附晃子濬字烏雷直勤素為燾所愛燕王謂國人曰博真非正不宜立直勤嫡孫應立耳乃殺博真及宗愛而立濬為主號年正平與後魏書不同又云在二十八年皆宋書之誤也】殺愛周皆具五刑夷三族 西陽五水羣蠻反【水經注蘄水出江夏蘄春縣北山水首受希水枝津西南流歷蘄山出蠻中故以此水為五水蠻五水謂巴水蘄水希水赤亭水西歸水蠻左憑阻山川世為抄暴宋沈慶之於西陽上下誅討即五水蠻也】自淮汝至於江沔咸被其患【南史曰蠻所在深阻種落熾盛北接淮汝南極江漢地方數千里沔迷遠翻被皮義翻下同】詔太尉中兵參軍沈慶之督江豫荆雍四州兵討之【為沈慶之以討蠻之兵輔武陵王駿起義張本雍於用翻】 魏以驃騎大將軍拓拔壽樂為太宰都督中外諸軍録尚書事【壽樂拓拔悉鹿之後驃匹妙翻騎奇計翻樂音洛】長孫渴侯為尚書令加儀同三司【賞定策之功也】十一月壽樂渴侯坐爭權並賜死 癸未魏廣陽簡王建臨淮宣王譚皆卒 甲申魏主母閭氏卒【按北史魏主母姓郁久閭氏河東王毗之妹也】魏南安王余之立也以古弼為司徒張黎為太尉及高宗立弼黎議不合旨黜為外都大官坐有怨言且家人告其為巫蠱皆被誅【古弼張黎魏世祖之所親任者也宗愛弑逆不能聲其罪而誅之南安之立首居公位雖不為巫蠱罪固不容於死矣被皮義反】 壬寅廬陵昭王紹卒魏追尊景穆太子為景穆皇帝皇妣閭氏為恭皇后<br />
<br />
  尊乳母常氏為保太后 隴西屠各王景文叛魏【屠直於翻】署置王侯魏統萬鎮將南陽王惠壽外都大官于洛拔督四州之衆討平之【四州謂秦雍河涼】徙其黨三千餘家於趙魏【此言戰國時趙魏大界】 十二月戊申魏葬恭皇后于金陵魏世祖晩年佛禁稍弛【魏禁佛見一百二十四卷二十三年】民間往往有私習者及高宗即位羣臣多請復之乙卯詔州郡縣衆居之所各聽建佛圖一區民欲為沙門者聽出家【捨俗為僧謂之出家】大州五十人小州四十人於是曏所毁佛圖率皆修復【佛圖即浮屠或白佛圖即佛寺】魏主親為沙門師賢等五人下髪【為于偽翻下髪剃髪也亦謂之祝髪】以師賢為道人統【道人統猶宋之都僧録北人謂之僧總攝魏書沙門師賢本罽賓國王種人少入道東遊凉州凉平赴代罷佛法時師賢假為毉術還俗而守道不改於修復日即反沙門為道人統和平初師賢卒曇曜代之更名沙門統】 丁巳魏以樂陵王周忸為太尉【忸女九翻】南部尚書陸麗為司徒鎮西將軍杜元寶為司空麗以迎立之功受心膂之寄朝臣無出其右者【朝直遥翻下同】賜爵平原王麗辭曰陛下國之正統【世嫡皇孫故曰正紈】當承基緒效順奉迎臣子常職不敢慆天之功【慆義與叨同貪也】以干大賞再三不受魏主不許麗曰臣父奉事先朝忠勤著效【陸俟事世祖威行北鎮功著關中】今年逼桑榆【桑榆晚景也】願以臣爵授之帝曰朕為天下主豈不能使卿父子為二王邪戊午進其父建業公俟爵為東平王 【考異曰魏紀曰戊申按上有丁巳下有癸亥不當中有戊申蓋戊午字誤耳】又命麗妻為妃復其子孫【復方目翻】麗力辭不受帝益嘉之以東安公劉尼為尚書僕射西平公源賀為征北將軍並進爵為王帝班賜羣臣謂源賀曰卿任意取之賀辭曰南北未賓府庫不可虛也【謂魏南有宋北有柔然不可一日弛備府庫所以供軍國之用不可虛於賞賜】固與之乃取戎馬一匹【示欲宣力於邉垂】高宗之立也高允預其謀陸麗等皆受重賞而不及允允終身不言【高允不言功其後位遇隆厚天豈嗇其報也】甲子周忸坐事賜死時魏法深峻源賀奏謀反之家男子十三以下本不預謀者宜免死没官從之 江夏王義恭還朝【自盱眙還也夏戶雅翻朝直遥翻】辛未以義恭為大將軍南徐州刺史【欲以代始興王濬也】録尚書如故 初魏入中原【晉孝武帝太元二十一年魏伐燕至安帝隆安二年克中山始得中原】用景初歷【景初歷楊偉所造曹魏明帝景初元年行之】世祖克沮渠氏【見二百二十三卷十六年沮子余翻】得趙玄始歷【徧考字書無字以偏傍從匪從文離而合之於上下讀如斐字】時人以為密是歲始行之<br />
<br />
  資治通鑑卷一百二十六  <br>
   </div> 

<script src="/search/ajaxskft.js"> </script>
 <div class="clear"></div>
<br>
<br>
 <!-- a.d-->

 <!--
<div class="info_share">
</div> 
-->
 <!--info_share--></div>   <!-- end info_content-->
  </div> <!-- end l-->

<div class="r">   <!--r-->



<div class="sidebar"  style="margin-bottom:2px;">

 
<div class="sidebar_title">工具类大全</div>
<div class="sidebar_info">
<strong><a href="http://www.guoxuedashi.com/lsditu/" target="_blank">历史地图</a></strong>  
<a href="http://www.880114.com/" target="_blank">英语宝典</a>  
<a href="http://www.guoxuedashi.com/13jing/" target="_blank">十三经检索</a> 
<br><strong><a href="http://www.guoxuedashi.com/gjtsjc/" target="_blank">古今图书集成</a></strong> 
<a href="http://www.guoxuedashi.com/duilian/" target="_blank">对联大全</a> <strong><a href="http://www.guoxuedashi.com/xiangxingzi/" target="_blank">象形文字典</a></strong> 

<br><a href="http://www.guoxuedashi.com/zixing/yanbian/">字形演变</a>  <strong><a href="http://www.guoxuemi.com/hafo/" target="_blank">哈佛燕京中文善本特藏</a></strong>
<br><strong><a href="http://www.guoxuedashi.com/csfz/" target="_blank">丛书&方志检索器</a></strong> <a href="http://www.guoxuedashi.com/yqjyy/" target="_blank">一切经音义</a>  

<br><strong><a href="http://www.guoxuedashi.com/jiapu/" target="_blank">家谱族谱查询</a></strong>  <strong><a href="http://shufa.guoxuedashi.com/sfzitie/" target="_blank">书法字帖欣赏</a></strong> 
<br>

</div>
</div>


<div class="sidebar" style="margin-bottom:0px;">

<font style="font-size:22px;line-height:32px">QQ交流群9:489193090</font>


<div class="sidebar_title">手机APP 扫描或点击</div>
<div class="sidebar_info">
<table>
<tr>
	<td width=160><a href="http://m.guoxuedashi.com/app/" target="_blank"><img src="/img/gxds-sj.png" width="140"  border="0" alt="国学大师手机版"></a></td>
	<td>
<a href="http://www.guoxuedashi.com/download/" target="_blank">app软件下载专区</a><br>
<a href="http://www.guoxuedashi.com/download/gxds.php" target="_blank">《国学大师》下载</a><br>
<a href="http://www.guoxuedashi.com/download/kxzd.php" target="_blank">《汉字宝典》下载</a><br>
<a href="http://www.guoxuedashi.com/download/scqbd.php" target="_blank">《诗词曲宝典》下载</a><br>
<a href="http://www.guoxuedashi.com/SiKuQuanShu/skqs.php" target="_blank">《四库全书》下载</a><br>
</td>
</tr>
</table>

</div>
</div>


<div class="sidebar2">
<center>


</center>
</div>

<div class="sidebar"  style="margin-bottom:2px;">
<div class="sidebar_title">网站使用教程</div>
<div class="sidebar_info">
<a href="http://www.guoxuedashi.com/help/gjsearch.php" target="_blank">如何在国学大师网下载古籍?</a><br>
<a href="http://www.guoxuedashi.com/zidian/bujian/bjjc.php" target="_blank">如何使用部件查字法快速查字?</a><br>
<a href="http://www.guoxuedashi.com/search/sjc.php" target="_blank">如何在指定的书籍中全文检索?</a><br>
<a href="http://www.guoxuedashi.com/search/skjc.php" target="_blank">如何找到一句话在《四库全书》哪一页?</a><br>
</div>
</div>


<div class="sidebar">
<div class="sidebar_title">热门书籍</div>
<div class="sidebar_info">
<a href="/so.php?sokey=%E8%B5%84%E6%B2%BB%E9%80%9A%E9%89%B4&kt=1">资治通鉴</a> <a href="/24shi/"><strong>二十四史</strong></a>&nbsp; <a href="/a2694/">野史</a>&nbsp; <a href="/SiKuQuanShu/"><strong>四库全书</strong></a>&nbsp;<a href="http://www.guoxuedashi.com/SiKuQuanShu/fanti/">繁体</a>
<br><a href="/so.php?sokey=%E7%BA%A2%E6%A5%BC%E6%A2%A6&kt=1">红楼梦</a> <a href="/a/1858x/">三国演义</a> <a href="/a/1038k/">水浒传</a> <a href="/a/1046t/">西游记</a> <a href="/a/1914o/">封神演义</a>
<br>
<a href="http://www.guoxuedashi.com/so.php?sokeygx=%E4%B8%87%E6%9C%89%E6%96%87%E5%BA%93&submit=&kt=1">万有文库</a> <a href="/a/780t/">古文观止</a> <a href="/a/1024l/">文心雕龙</a> <a href="/a/1704n/">全唐诗</a> <a href="/a/1705h/">全宋词</a>
<br><a href="http://www.guoxuedashi.com/so.php?sokeygx=%E7%99%BE%E8%A1%B2%E6%9C%AC%E4%BA%8C%E5%8D%81%E5%9B%9B%E5%8F%B2&submit=&kt=1"><strong>百衲本二十四史</strong></a>  <a href="http://www.guoxuedashi.com/so.php?sokeygx=%E5%8F%A4%E4%BB%8A%E5%9B%BE%E4%B9%A6%E9%9B%86%E6%88%90&submit=&kt=1"><strong>古今图书集成</strong></a>
<br>

<a href="http://www.guoxuedashi.com/so.php?sokeygx=%E4%B8%9B%E4%B9%A6%E9%9B%86%E6%88%90&submit=&kt=1">丛书集成</a> 
<a href="http://www.guoxuedashi.com/so.php?sokeygx=%E5%9B%9B%E9%83%A8%E4%B8%9B%E5%88%8A&submit=&kt=1"><strong>四部丛刊</strong></a>  
<a href="http://www.guoxuedashi.com/so.php?sokeygx=%E8%AF%B4%E6%96%87%E8%A7%A3%E5%AD%97&submit=&kt=1">說文解字</a> <a href="http://www.guoxuedashi.com/so.php?sokeygx=%E5%85%A8%E4%B8%8A%E5%8F%A4&submit=&kt=1">三国六朝文</a>
<br><a href="http://www.guoxuedashi.com/so.php?sokeytm=%E6%97%A5%E6%9C%AC%E5%86%85%E9%98%81%E6%96%87%E5%BA%93&submit=&kt=1"><strong>日本内阁文库</strong></a> <a href="http://www.guoxuedashi.com/so.php?sokeytm=%E5%9B%BD%E5%9B%BE%E6%96%B9%E5%BF%97%E5%90%88%E9%9B%86&ka=100&submit=">国图方志合集</a> <a href="http://www.guoxuedashi.com/so.php?sokeytm=%E5%90%84%E5%9C%B0%E6%96%B9%E5%BF%97&submit=&kt=1"><strong>各地方志</strong></a>

</div>
</div>


<div class="sidebar2">
<center>

</center>
</div>
<div class="sidebar greenbar">
<div class="sidebar_title green">四库全书</div>
<div class="sidebar_info">

《四库全书》是中国古代最大的丛书,编撰于乾隆年间,由纪昀等360多位高官、学者编撰,3800多人抄写,费时十三年编成。丛书分经、史、子、集四部,故名四库。共有3500多种书,7.9万卷,3.6万册,约8亿字,基本上囊括了古代所有图书,故称“全书”。<a href="http://www.guoxuedashi.com/SiKuQuanShu/">详细>>
</a>

</div> 
</div>

</div>  <!--end r-->

</div>
<!-- 内容区END --> 

<!-- 页脚开始 -->
<div class="shh">

</div>

<div class="w1180" style="margin-top:8px;">
<center><script src="http://www.guoxuedashi.com/img/plus.php?id=3"></script></center>
</div>
<div class="w1180 foot">
<a href="/b/thanks.php">特别致谢</a> | <a href="javascript:window.external.AddFavorite(document.location.href,document.title);">收藏本站</a> | <a href="#">欢迎投稿</a> | <a href="http://www.guoxuedashi.com/forum/">意见建议</a> | <a href="http://www.guoxuemi.com/">国学迷</a> | <a href="http://www.shuowen.net/">说文网</a><script language="javascript" type="text/javascript" src="https://js.users.51.la/17753172.js"></script><br />
  Copyright &copy; 国学大师 古典图书集成 All Rights Reserved.<br>
  
  <span style="font-size:14px">免责声明:本站非营利性站点,以方便网友为主,仅供学习研究。<br>内容由热心网友提供和网上收集,不保留版权。若侵犯了您的权益,来信即刪。scp168@qq.com</span>
  <br />
ICP证:<a href="http://www.beian.miit.gov.cn/" target="_blank">鲁ICP备19060063号</a></div>
<!-- 页脚END --> 
<script src="http://www.guoxuedashi.com/img/plus.php?id=22"></script>
<script src="http://www.guoxuedashi.com/img/tongji.js"></script>

</body>
</html>
