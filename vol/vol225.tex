\chapter{資治通鑑卷二百二十五}
宋 司馬光 撰

胡三省 音註

唐紀四十一|{
	起閼逢攝提格盡屠維恊洽七月凡五年有奇始甲寅終己未七月凡五年零七月}


代宗睿文孝武皇帝中之下

大歷九年春正月壬寅田神功薨於京師|{
	田神功入朝見上卷上平}
澧朗鎮遏使楊猷自澧州沿江而下擅出境至鄂州

詔聽入朝猷遂泝漢江而上|{
	澧音禮使疏吏翻朝直遥翻泝蘇故翻下遐稼翻上時掌翻}
復州郢州皆閉城自守|{
	復郢二州皆賓漢此時盖為山南東道廵屬守式又翻}
山南東道節度使梁崇義兵備之 二月辛未徐州軍亂刺史梁乘逾城走|{
	徐州治彭城}
諫議大夫吳損使吐蕃留之累年竟病死虜中|{
	吐從暾入聲損出使見上卷六年}
庚辰汴宋兵防秋者千五百人盜庫財潰歸|{
	汴皮變翻 考異曰唐歷作十日己酉按長歷是月庚午朔十日乃己卯也今從實録}
田神功薨故也己丑以神功弟神玉知汴宋留後 癸巳郭子儀入朝上言朔方國之北門中間戰士耗散什纔有一今吐蕃兼河隴之地雜羌渾之衆|{
	朝直遥翻史炤曰羌党項之屬渾吐谷渾也}
勢強十倍願更於諸道各精卒成四五萬人則制勝之道必矣 三月戊申以皇女永樂公主許妻魏博節度使田承嗣之子華|{
	樂音洛妻七細翻嗣祥吏翻}
上意欲固結其心而承嗣益驕慢|{
	史言狼子野心不可以恩結}
以澧朗鎮遏使楊猷為洮州刺史隴右節度兵馬使|{
	澧音禮使疏吏翻洮土刀翻洮州時已陷吐蕃楊猷特領刺史耳}
夏四月甲申郭子儀辭還邠州|{
	子儀自邠入朝今還還從宣翻又音如字 考異曰唐歷作癸未今從實錄}
復為上言邊事|{
	復扶又翻為于偽翻下為已仍為同}
至涕泗交流 壬辰赦天下 五月丙午楊猷自澧州入朝|{
	是年正月已書楊猷離澧州沿江泝漢今方至京師朝直遥翻}
涇原節度使馬璘入朝諷將士為己表求平章事|{
	璘離珍翻將即亮翻}
丙寅以璘為左僕射|{
	璘離珍翻}
六月盧龍節度使朱泚遣弟滔奉表請入朝且請自將步騎五千防秋上許之|{
	泚且禮翻又音此將即亮翻又音如字騎奇寄翻}
仍為先築大第於京師以待之 癸未興善寺胡僧不空卒贈開府儀同三司司空賜爵肅國公謚曰大辯正廣智不空三藏和尚|{
	釋典云佛在多羅柰最初為五人說契經修多羅藏佛在羅閲祗最初為須那提說毗尼藏佛在毗舍離獼㺅池最初為跂耆說阿毗曇藏五百羅漢夜集阿毗曇相續解說經此為三藏學又三藏學經律論也卒子恤翻藏徂浪翻}
京師旱京兆尹黎幹作土龍祈雨自與巫覡更舞|{
	覡刑狄翻更工衡翻}
彌月不雨又禱於文宣王上聞之命撤土龍減膳節用秋七月戊午雨朱泚入朝至蔚州有疾|{
	蔚紆勿翻此自幽州西出山後取太原路入朝宋白曰蔚州西南至代州四百六十里}
諸將請還俟閒而行|{
	待病閒而行也將即亮翻閒讀如字}
泚曰死則輿尸而前諸將不敢復言|{
	復扶又翻}
九月庚子至京師士民觀者如堵|{
	安史亂後河北諸帥阻兵不朝朱泚之來長安士民以為美事}
辛丑宴泚及將士於延英殿|{
	盧文紀曰上元以來置延英殿或宰相欲有奏對或天子欲有咨度皆非時召見程大昌曰高宗初剏蓬萊宫諸門殿亭皆已立名至上元二年延英殿當御座生玉芝則是初有大明宫即有延英殿顧召對宰臣則始於代宗耳代宗以苖晉卿年老蹇甚聽入閭不趨為御延英此優禮也案六典宣政殿前西上閤門之西即為延英門延英門之左即延英殿故陽城欲救陸䞇約拾遺王仲舒守延英殿上疏伏閤不去也}
犒賞之盛近時未有 壬寅回紇擅出鴻臚寺白晝殺人|{
	紇下没翻臚陵如翻}
有司擒之上釋不問 甲辰命郭子儀李抱玉馬璘朱泚分統諸道防秋之兵|{
	璘離珍翻泚且禮翻又音此}
冬十月壬申信王瑝薨乙亥梁王璿薨|{
	二王皆玄宗子信以州為國名瑝戶肓翻又音皇璿似宣翻}
魏博節度使田承嗣誘昭義將吏使作亂|{
	使疏吏翻嗣祥吏翻誘羊久翻}
十年春正月丁酉昭義兵馬使裴志清逐留後薛㟧帥其衆歸承嗣|{
	帥讀曰率}
承嗣聲言救援引兵襲相州取之|{
	相息亮翻下同}
㟧奔洺州|{
	洺音名}
上表請入朝許之|{
	上時掌翻朝直遥翻}
辛丑郭子儀入朝壬寅壽王瑁薨|{
	瑁亦玄宗子音莫報翻}
乙巳朱泚表請留闕

下以弟滔知幽州盧龍留後許之 昭義禆將薛擇為相州刺史薛雄為衛州刺史薛堅為洺州刺史皆薛嵩之族也|{
	相衛洺本皆昭義廵屬}
戊申上命内侍魏知古如魏州諭田承嗣使各守封疆承嗣不奉詔癸丑遣大將盧子期取洺州楊光朝攻衛州 乙卯西川節度使崔寧奏破吐蕃數萬於西山|{
	吐從暾入聲}
斬首萬級捕虜數千人 丙辰詔諸道兵有逃亡者非承制勑無得輒召募 二月乙丑田承嗣誘衛州刺史薛雄雄不從使盜殺之屠其家盡據相衛四州之地|{
	相衛已䧟於承嗣磁洺雖未下而承嗣已據其地相衛二州自此屬魏博}
自置長吏掠其精兵良馬悉歸魏州逼魏知古與共廵磁相二州|{
	磁疾之翻}
使其將士割耳剺面|{
	剺力之翻}
請承嗣為帥|{
	帥所類翻下同}
辛未立皇子述為睦王逾為郴王連為恩王遘為鄜王|{
	郴丑林翻鄜芳無翻}
迅為隨王造為忻王暹為韶王運為嘉王遇為端王遹為循王|{
	遹余律翻}
通為恭王逹為原王逸為雅王|{
	諸皇子皆以州為王國名}
丙子以華州刺史李承昭知昭義留後|{
	華戶化翻}
河陽三城使常休明|{
	河陽縣本屬懷州顯慶二年分屬河南府城臨大河長橋架水古稱設險此城後魏之北中城也東西魏兵爭又築中潬及南城謂之河陽三城乾元中史思明再䧟東京李光弼以重兵守河陽及雍王平賊令魚朝恩守河陽乃以河南之河清濟源温四縣租税入河陽三城使河南尹但禮領其縣額尋又以汜水軍賦屬之使疏吏翻}
苛刻少恩其軍士防秋者歸休明出城勞之|{
	少詩沼翻勞力到翻}
防秋兵與城内兵合謀攻之休明奔東都軍士奉兵馬使王惟恭為帥大掠數日乃定上命監軍冉庭蘭慰撫之|{
	監古銜翻}
三月甲午陜州軍亂逐兵馬使趙令珍|{
	陜失冉翻 考異曰唐歷三月二十八日辛卯陜州軍亂實録唐統紀云甲午朔今從之}
觀察使李國清不能禁卑辭徧拜將士乃得脱去|{
	陜失冉翻將即亮翻}
軍士大掠庫物會淮西節度使李忠臣入朝過陜|{
	朝直遥翻過古禾翻}
上命忠臣按之將士畏忠臣兵威不敢動忠臣設棘圍令軍士匿名投庫物|{
	設棘四圍周之令投所掠庫物於其中}
一日獲萬緡盡以給其從兵為賞|{
	緡眉巾翻從才用翻}
乙巳薛㟧常休明皆詣闕請罪上釋不問 初成德節度使李寶臣淄青節度使李正巳皆為田承嗣所輕|{
	使疏吏翻淄莊持翻嗣祥吏翻}
寶臣弟寶正娶承嗣女在魏州與承嗣子維擊毬馬驚誤觸維死承嗣怒囚寶正以告寶臣寶臣謝教勑不謹封杖授承嗣使撻之承嗣遂杖殺寶正由是兩鎮交惡及承嗣拒命寶臣正巳皆上表請討之|{
	上時掌翻}
上亦欲因其隙討承嗣夏四月乙未勑貶承嗣為永州刺史|{
	永州漢零陵郡隋廢郡為永州}
仍命河東成德幽州淄青淮西永平汴宋河陽澤潞諸道兵前臨魏博若承嗣尚或稽違即令進討罪止承嗣及其姪悦自餘將士弟姪苟能自拔一切不問|{
	令力丁翻將即亮翻}
時朱滔方恭順與寶臣及河東節度使薛兼訓攻其北正已與淮西節度使李忠臣等攻其南五月乙未承嗣將霍榮國以磁州降|{
	磁州陽郡本漢廣平縣地隋廢郡於陽縣置磁州治陽本漢成安縣地將即亮翻降戶江翻下同磁牆之翻}
丁未李正已攻德州拔之|{
	德州自此屬平盧軍}
李忠臣統永平河陽懷澤步騎四萬進攻衛州|{
	統地綜翻俗讀從上聲}
六月辛未田承嗣遣其將裴志清等攻冀州志清以其衆降李寶臣甲戌承嗣自將圍冀州寶臣使高陽軍使張孝忠將精騎四千禦之寶臣大軍繼至承嗣燒輜重而遁|{
	重直用翻冀州治信都縣將即亮翻又音如字騎奇寄翻高陽軍當置于瀛州高陽縣兵志横海北平高陽等軍皆屬平盧道盖安史之亂以兵授張孝忠統制而屬於李寶臣因授高陽軍使耳}
孝忠本奚也|{
	張孝忠奚人世為乙失活部酋長}
田承嗣以諸道兵四合部將多叛而懼秋八月遣使奉表請束身歸朝|{
	嗣祥吏翻將即亮翻使疎吏翻朝直遥翻}
辛巳郭子儀還邠州|{
	入朝而還所鎮還從宣翻又音如字邠卑旻翻考異曰汾陽家傳作丁丑今從實録}
子儀嘗奏除州縣官一人不報僚

佐相謂曰以令公勲德奏一屬吏而不從何宰相之不知體子儀聞之謂僚佐曰自兵興以來方鎮武臣多跋扈凡有所求朝廷常委曲從之此無它乃疑之也今子儀所奏事人主以其不可行而置之是不以武臣相待而親厚之也諸君可賀矣又何怪焉聞者皆服|{
	史言郭子儀忠純}
己丑田承嗣遣其將盧子期寇磁州|{
	將即亮翻磁牆之翻}
九月戊申回紇白晝刺市人腸出|{
	紇下没翻刺七亦翻}
有司執之繫萬年獄其酋長赤心馳入縣獄|{
	酋才由翻長知兩翻}
斫傷獄吏劫囚而去上亦不問 壬子吐蕃寇臨涇|{
	吐從暾入聲臨涇漢縣屬安定郡隋大業初改曰湫谷縣尋復曰臨涇縣唐屬涇州}
癸丑寇隴州及普潤大掠人畜而去百姓往往遣家屬出城竄匿丙辰鳳翔節度使李抱玉奏破吐蕃於義寧|{
	隴州華亭縣大歷八年置義寧軍}
李寶臣正已會于棗強|{
	棗強縣前漢屬清河郡後漢省晉復置屬廣川郡魏隋以來屬冀州}
進圍貝州田承嗣出兵救之兩軍各饗士卒成德賞厚平盧賞薄既罷平盧士卒有怨言正已恐其為變引兵退寶臣亦退李忠臣聞之釋衛州南度河屯陽武|{
	陽武縣屬鄭州本原武城武德四年置陽武縣北至衛州五十許里}
寶臣與朱滔攻滄州承嗣從父弟庭玠守之|{
	從才用翻}
寶臣不能克 吐蕃寇涇州涇原節度使馬璘破之於百里城|{
	吐從暾入聲使疏吏翻璘離珍翻 考異曰汾陽家傳九月吐蕃略潘原西而還八日至小石門白草川十八日下朝那川二十三日至里城營支磨原入華亭十月公遣渾瑊李懷光與幽州義寧汴宋軍會于故平凉縣三日詰朝大破之今從實録}
戊午命盧龍節度使朱泚出鎮奉天行營|{
	泚且禮翻又音此}
冬十月辛酉朔日有食之 盧子期攻磁州|{
	磁牆之翻 考異曰舊李寶臣傳作攻邢州今從實録}
城幾陷|{
	幾居依翻}
李寶臣與昭義留後李承昭共救之大破子期於清水|{
	按新書田承嗣傳清水作臨水晉置臨水縣於口之右屬廣平郡後魏及隋屬魏郡唐初省永泰元年薛嵩表于臨水故城置昭義縣屬磁州}
擒子期送京師斬之河南諸將又大破田悦于陳留|{
	陳留漢縣後魏廢隋開皇六年復置時屬汴州九域志在州東五十二里將即亮翻}
田承嗣懼初李正已遣使至魏州承嗣囚之至是禮而遣之遣使盡籍境内戶口甲兵穀帛之數以與之曰承嗣今年八十有六|{
	嗣祥吏翻考異曰按承嗣卒時年七十五此云八十六盖欺正巳}
溘死無日|{
	溘口答翻奄也莊子曰溘然而死謂奄然也}
諸子不肖悦亦孱弱|{
	孱士山翻}
凡今日所有為公守耳|{
	為于偽翻}
豈足以辱公之師旅乎立使者於庭南向拜而授書又圖正巳之像焚香事之正巳悦遂按兵不進於是河南諸道兵皆不敢進承嗣既無南顧之虞得專意北方上嘉李寶臣之功遣中使馬承倩齎詔勞之|{
	勞力到翻}
將還寶臣詣其館遺之百縑|{
	遺于季翻}
承倩詬詈擲出道中寶臣慙其左右|{
	為承倩詈辱顧見左右而自慙也詬許候翻又古候翻}
兵馬使王武俊說寶臣曰|{
	使疎吏翻說式芮翻下客說同}
今公在軍中新立功豎子尚爾况寇平之後以一幅詔書召歸闕下一匹夫耳不如釋承嗣以為己資寶臣遂有玩寇之志承嗣知范陽寶臣鄉里心常欲之|{
	寶臣本范陽内屬奚范陽將張瑣高畜為假子因冒其姓歸唐又賜姓李}
因刻石作䜟云二帝同功勢萬全將田為侣入幽燕密令瘞寶臣境内|{
	䜟楚譖翻瘞於計翻}
使望氣者言彼有玉氣寶臣掘而得之又令客說之曰公與朱滔共取滄州得之則地歸國非公所有公能捨承嗣之罪請以滄州歸公仍願從公取范陽以自効公以精騎前驅承嗣以步卒繼之蔑不克矣|{
	令力丁翻說式芮翻騎奇寄翻}
寶臣喜謂事合符䜟遂與承嗣通謀密圖范陽承嗣亦陳兵境上寶臣謂滔使者曰聞朱公儀貌如神願得畫像觀之滔與之寶臣置于射堂與諸將共觀之曰眞神人也滔軍於瓦橋|{
	將即亮翻瓦橋古易京之地在莫州北三十里唐置瓦橋關宋白曰涿州歸義縣瓦子濟橋在涿州南易州東當九河之末舊置瓦橋關後周置雄州}
寶臣選精騎二千通夜馳三百里襲之戒曰取貌如射堂者時兩軍方睦滔不虞有變狼狽出戰而敗會衣它服得免|{
	衣於既翻}
寶臣欲乘勝取范陽滔使雄武軍使昌平劉怦守留府|{
	怦普耕翻}
寶臣知有備不敢進承嗣聞幽恒兵交即引軍南還|{
	嗣祥吏翻恒戶登翻還從宣翻又音如字}
使謂寶臣曰河内有警不暇從公石上䜟文吾戲為之耳寶臣慙怒而退 |{
	考異曰舊王武俊傳曰代宗嘉其功使中貴人馬承倩齎詔宣勞承倩將歸止傳舍寶臣親遺百縑承倩詬罵擲出道中王武俊勸玩養承嗣以為己資寶臣曰今與承嗣有釁矣可推腹心哉武俊曰勢同患均轉寇讎為父子欬唾間耳若傳虛言無益也今中貴人劉清譚在驛斬首送承嗣承嗣立質妻孥矣寶臣曰吾不能如此武俊曰朱滔為國屯兵滄州請擒送承嗣以取信許之按承嗣方求解于寶臣何必擒滔以取信且承倩尚在傳舍武俊何不勸斬承倩而斬清譚乎寶臣自以承嗣誘之共取幽州故襲朱滔非因承倩之辱也今從唐紀}
寶臣旣與朱滔有隙以張孝忠為易州刺史使將精騎七千以備之|{
	史言田承嗣凶狡過于諸帥宋白曰易州東至幽州二百一十四里將即亮翻又音如字騎奇寄翻}
丙寅貴妃獨孤氏薨丁卯追謚貞懿皇后|{
	諡神至翻}
十一月丁酉田承嗣將吴希光以瀛州降嶺南節度使路嗣恭擢流人孟瑶敬冕為將 |{
	考異曰鄴侯家傳作敬俛今從舊傳}
討哥舒晃瑶以大軍當其衝冕自間道輕入|{
	間古莧翻輕墟正翻}
丁未克廣州斬哥舒晃 |{
	考異曰舊嗣恭傳云嗣恭平廣州商舶之徒多因晃事誅之嗣恭前後没其家財寶數百萬貫盡入私室不以貢獻代宗心甚銜之故嗣恭雖有平方面功止轉檢校兵部尚書無所酬勞建中實録曰自兵興以來諸軍殺將帥而要君者多矣皆因授其任以苟安之其王師征討不失有罪始斯役也既而有謗其收南海府庫閲上不實不得用久之按代宗以嗣恭附元載遺載瑠璃盤惡之故不用耳事見鄴侯家傳或當時亦有人迎合以匿貨謗嗣恭不可知也今不取李肇國史補云路嗣恭初平五嶺元載奏嗣恭多取南人金寶是欲為亂陛下不信試召必不入朝三伏中追詔至嗣恭不慮請待秋凉以修覲禮江西判官柳渾入雨泣曰公有功方暑而追是為執政所中今少遷延必族滅也嗣恭懼曰為之柰何渾曰健步追還表緘公今日過江宿石頭驛乃可從之代宗謂元載曰嗣恭不俟駕行矣載無以對按嗣恭素附元載載誅賴李泌營救得免事見鄴侯家傳載豈有譖嗣恭云欲為亂之理盖載已被誅而召嗣恭適在三伏渾冇此疑時人因以為渾美事耳今不取 余按去年命路嗣恭為嶺南節度使討哥舒晃嗣恭既誅晃而平廣州則當在廣州柳渾若以江西判官從嗣恭亦當在廣州今諫嗣恭請奉詔就道乃言過江宿石頭驛石頭驛在豫章江之西岸嗣恭自江西觀察赴召可言宿石頭驛自嶺南節度赴召安得宿石頭驛哉亦可以明李肇之誤}
及其黨萬餘人嗣恭之討晃也容管經略使王雄遣將將兵助之西原賊帥覃問乘虛襲容州|{
	姓譜梁有東寧州刺史覃元先則蠻中有覃姓尚矣}
翃伏兵擊擒之十二月回紇千騎寇夏州 |{
	考異曰此事出汾陽家傳實録新舊紀皆無之按}


|{
	實録明年二月加朔方戍兵以備回紇則是回紇嘗入寇也}
州將梁榮宗破之於烏水|{
	烏水在夏州朔方縣貞觀七年開延化渠引烏水入庫狄澤將即亮翻}
郭子儀遣兵三千救夏州回紇遁去|{
	夏戶雅翻紇下没翻}
元載王縉奏魏州鹽貴請禁鹽入其境以困之|{
	載祖亥翻又音如字縉音晉}
上不許曰承嗣負朕百姓何罪 田承嗣請入朝李正已屢為之上表乞許其自新|{
	嗣祥吏翻朝直遥翻為于偽翻上時掌翻}


十一年春正月壬辰遣諫議大夫杜亞使魏州宣慰|{
	使疎吏翻}
辛亥西川節度使崔寧奏破吐蕃四節度及突厥吐谷渾氐羌羣蠻衆二十餘萬斬首萬餘級|{
	吐從暾入聲厥九勿翻}
二月庚辰田承嗣復遣使上表請入朝|{
	復扶又翻}
上乃下詔赦承嗣罪復其官爵聽與家屬入朝其所部拒朝命者一切不問 辛巳增朔方五城戍兵以備回紇|{
	朔方先統三受降城并振武豐安定遠為六城時三受降城屬振武軍使朔方統豐安定遠新昌豐寧保寧謂之塞下五城紇下没翻}
三月戊子河陽軍亂逐監軍冉庭蘭出城大掠三日庭蘭成備而入誅亂者數十人乃定|{
	監古銜翻}
五月汴宋留後田神玉卒都虞候李靈曜殺兵馬使濮州刺史孟鑒北結田承嗣為援|{
	汴皮變翻卒子恤翻使踈吏翻濮博木翻嗣祥吏翻}
癸巳以永平節度使李勉兼汴宋等八州留後|{
	汴宋曹濮兖鄆徐泗八州濮博木翻}
乙未以靈曜為濮州刺史靈曜不受詔六月戊午以靈曜為汴宋留後遣使宣慰 秋七月田承嗣遣兵寇滑州敗李勉|{
	滑州永平節度治所敗補邁翻}
吐蕃寇石門入長澤川|{
	長澤川後魏置闡熙郡隋廢郡為長澤縣屬夏州吐蕃寇原州遂北入夏州界也宋白曰長澤縣漢朔方郡三封縣之地三封故城赫連勃勃據之築為統萬城又按原州北有長澤監吐從暾入聲}
八月丙寅加盧龍節度使朱泚同平章事 |{
	考異曰實録閏八}


|{
	月己亥遣朱泚如奉天行營按去年已云泚出鎮奉天行營至此又云明年九月又云盖泚每年往奉天防秋至春還京師但實録不載其入朝耳}
李靈曜既為留後益驕慢悉以其黨為管内八州刺史縣令欲效河北諸鎮甲申詔淮西節度使李忠臣永平節度使李勉河陽三城使馬燧討之淮南節度使陳少游淄青節度使李正已皆進兵擊靈曜汴宋兵馬使攝節度副使李僧惠|{
	少始照翻 考異曰汾陽家傳作李思惠今從舊傳}
靈曜之謀主也宋州牙門將劉昌遣僧神表濳說僧惠僧惠召問計昌為之泣陳逆順|{
	將即亮翻說式芮翻為于偽翻}
僧惠乃與汴宋牙將高憑石隱金遣神表奉表詣京師請討靈曜九月壬戌以僧惠為宋州刺史憑為曹州刺史隱金為鄆州刺史乙丑李忠臣馬燧軍于鄭州|{
	汴皮變翻鄆音運}
靈曜引兵逆戰兩軍不意其至退軍滎澤|{
	隋分滎陽縣置滎澤縣唐屬鄭州九域志在州西北四十五里}
淮西軍士潰去者什五六鄭州士民皆驚走入東都忠臣將歸淮西燧固執不可曰以順討逆何憂不克柰何自弃功名堅壁不動忠臣聞之稍收散卒數日皆集軍勢復振|{
	史言李忠臣因馬燧以成功復扶又翻}
戊辰李正已奏克鄆濮二州壬申李僧惠敗靈曜兵于雍丘|{
	濮博木翻敗補邁翻下敗永同}
冬十月李忠臣馬燧進擊靈曜忠臣行汴南燧行汴北屢破靈曜兵壬寅與陳少游前軍合與靈曜大戰於汴州城西靈曜敗入城固守癸卯忠臣等圍之田承嗣遣田悦將兵救靈曜敗永平淄青兵於匡城|{
	少始照翻匡城漢長垣縣隋開皇十六年更名時屬滑州}
乘勝進軍汴州營於城北數里丙午忠臣遣禆將李重倩將輕騎數百夜入其營縱横貫穿|{
	縱子容翻穿尺絹翻}
斬數十人而還|{
	還從宣翻又如字}
營中大駭忠臣燧因以大軍乘之鼓譟而入悦衆不戰而潰悦脱身北走將士死者相枕藉不可勝數|{
	譟則竈翻枕職任翻藉慈夜翻勝音升}
靈曜聞之開門夜遁汴州平重倩本奚也丁未靈曜至韋城|{
	隋分白馬縣置韋城縣於古韋國之墟故曰韋城時屬滑州九域志韋城縣在滑州東南五十里}
永平將杜如江擒之燧知忠臣暴戾以己功讓之不入汴城|{
	將即亮翻汴皮變翻}
引軍西屯板橋忠臣入城果專其功宋州刺史李僧惠與之爭功忠臣因會擊殺之又欲殺劉昌昌遁逃得免甲寅李勉械送李靈曜至京師斬之 十二月丁亥李正已李寶臣並加同平章事 涇原節度使馬璘疾亟以行軍司馬段秀實知節度事付以後事秀實嚴兵以備非常丙申璘薨|{
	使疎吏翻璘離珍翻 考異曰實録庚寅璘薨段公别傳曰十二月景申馬公薨十二年正月八日奉制除涇州刺史知節度事實録又云丁酉以段秀實知河東留後按時馬璘新薨秀實涇原留後備禦吐蕃豈可撤之使攝河東盖奏報未至有斯命尋聞璘薨遂除涇原耳亟紀力翻}
軍中奔哭者數千人喧咽門屏秀實悉不聽入命押牙馬頔治喪事於内|{
	頔徒歷翻治直之翻}
李漢惠接賓客於外妻妾子孫位於堂宗族位於庭將佐位於前牙士卒哭於營伍百姓各守其家有離立偶語於衢路輒執而囚之|{
	記曲禮曰離立者不出中間注云離兩也}
非護喪從行者無得遠送致祭拜哭皆有儀節送喪近遠皆有定處違者以軍法從事都虞候史廷幹兵馬使崔珍十將張景華謀因喪作亂秀實知之奏廷幹入宿衛徙珍屯靈臺補景華外職不戮一人軍府晏然|{
	自高仙芝喪師於大食段秀實始見於史其後責李嗣業不赴難水之潰保河清以濟歸師在邠州誅郭晞暴横之卒與馬璘議論不阿及治喪曲防周慮以安軍府最後笏擊朱泚以身徇國其事業風節卓然表出於唐諸將中}
璘家富有無算治第京師甲於勲貴|{
	以功勲致貴顯故曰勲貴治直之翻}
中堂費二十萬緡它室所減無幾|{
	幾居豈翻}
其子孫無行|{
	行下孟翻}
家貲尋盡|{
	史言殖貨無厭者適以為不肖子孫之資}
戊戌昭義節度使李承昭表稱疾篤以澤潞行軍司馬李抱眞兼知磁邪兩州留後|{
	使疎吏翻磁牆之翻}
庚戌加淮西節度使李忠臣同平章事仍領汴州刺史治汴州|{
	李忠臣破李靈曜得汴州即以與之汴皮變翻}


十二年春三月乙卯兵部尚書同平章事鳳翔懷澤潞秦隴節度使李抱玉薨|{
	尚辰羊翻}
弟抱真仍領懷澤潞留後癸亥以河東行軍司馬鮑防為河東節度使防襄州

人也 田承嗣竟不入朝又助李靈曜上復命討之承嗣乃復上表謝罪|{
	嗣祥吏翻朝直遥翻復扶又翻上時掌翻}
上亦無如之何庚午悉復承嗣官爵仍令不必入朝|{
	令力丁翻}
中書侍郎同平章事元載專横|{
	載祖亥翻又音如字横戶孟翻}
黄門侍郎同平章事王縉附之二人俱貪載妻王氏|{
	載妻王忠嗣之女縉音晉}
及子伯和仲武縉弟妹及尼出入者爭納賄賂|{
	尼女夷翻}
又以政事委羣吏士之求進者不結其子弟及主書卓英倩等無由自逹上含容累年載縉不悛|{
	悛丑緣翻}
上欲誅之恐左右漏泄無可與言者獨與左金吾大將軍吴湊謀之湊上之舅也|{
	吴湊章敬皇后弟也}
會有告載縉夜醮圖為不軌者庚辰上御延英殿命湊收載縉於政事堂|{
	政事堂在東省屬門下自中宗後徙堂於中書省則堂在右省也案裴炎傳故事宰相於門下省議事謂之政事堂故長孫無忌為司空房玄齡為僕射魏徵為太子太師皆知門下省事至中宗時裴炎為中書令執政事筆故徒政事堂於中書省載祖亥翻又音如字縉音晉}
又收仲武及卓英倩等繫獄命吏部尚書劉晏與御史大夫李涵等同鞫之|{
	尚辰羊翻}
問端皆出禁中|{
	問端猶今言問頭也}
仍遣中使詰以隂事載縉皆伏罪|{
	使疎吏翻詰去吉翻}
是日先杖殺左衛將軍知内侍省事董秀於禁中乃賜載自盡於萬年縣載請主者願得快死主者曰相公須受少汚辱|{
	相息亮翻少詩沼翻}
勿恠乃脱穢韈塞其口而殺之|{
	韈勿伐翻足衣塞悉則翻}
王縉初亦賜自盡劉晏謂李涵等曰故事重刑覆奏况大臣乎且法有首從|{
	首謂罪首從謂從坐從才用翻}
宜更禀進止涵等從之上乃貶縉栝州刺史|{
	宋白曰栝州晉為永嘉郡隋改處州尋為栝州因栝蒼山為名劉昫曰京師東南四千二百七十八里}
載妻王氏忠嗣之女也及子伯和仲武季能皆伏誅有司籍載家財胡椒至八百石|{
	嗣祥吏翻本草曰胡椒生西戎形如鼠李子調食用之味甚辛辣徐表南中記曰生南海諸國}
它物稱是|{
	稱尺證翻}
夏四月壬午以太常卿楊綰為中書侍郎禮部侍郎常衮為門下侍郎並同平章事綰性清儉簡素制下之日朝野相賀郭子儀方宴客聞之減坐中聲樂五分之四|{
	朝直遥翻坐徂臥翻}
京兆尹黎幹騶從甚盛|{
	騶仄尤翻從才用翻}
即日省之止存十騎中丞崔寛第舍宏侈亟毁撤之癸未貶吏部侍郎楊炎諫議大夫韓洄包佶|{
	佶巨乙翻}
起居

舍人韓會等皆載黨也炎鳳翔人載常引有文學才望者一人親厚之異日欲以代己故炎及于貶|{
	載祖亥翻又音如字}
洄滉之弟會南陽人也上初欲盡誅炎等吴湊諫救百端始貶官 丁酉吐蕃寇黎雅州|{
	武后大足元年以雅州之漢源飛越嶲州之陽山置黎州吐從暾入聲}
西川節度使崔寧擊破之 元載以仕進者多樂京師惡其逼己|{
	樂音洛惡烏路翻}
乃制俸禄厚外官而薄京官京官不能自給常從外官乞貸楊綰常衮奏京官俸太薄|{
	貸他代翻又土得翻俸扶用翻}
己酉詔加京官俸歲約十五萬六千餘緡|{
	唐會要開元二十四年勑百官料錢宜合為一色都以月俸為名各據本官隨月給付一品三十千二品二十四千三品十七千四品十一千五品九千二百六品五千三百七品四千五十八品二千四百七十五九品一千九百一十七至大歷十二年加京官俸三師三公侍中中書令每月各一百二十貫文中書門下侍郎月各一百貫文東宫三太左右僕射各八十貫文東宫三少各七十貫文尚書御史大夫太常卿各六十貫文常寺宗正卿太子詹事國子祭酒各五十貫文左右丞及諸司侍郎給舍中丞賓客殿中祕書監司農等卿將作等監各四十五貫文太子二庶子太常少卿各四十貫文諫議諸司少卿少監各三十五貫文國子司業内侍東宫三卿各三十貫文郎中侍御史司天監少詹事諸王傅國子博士諭德中允中舍殿中袐書太常宗正卿各二十五貫文殿中侍御史著作郎大理正都水使者總監内常侍内給事各二十貫文員外郎通事舍人起居王府長史各十八貫文監察御史臺主簿補闕王府司馬司天少監太子典内太常博士主簿宗正主簿門下録事中書主書各十五貫文拾遺司議太子文學祕書著作佐郎國子太學四門廣文博士大理司直詹事府丞及諸寺監丞謁者監中書門下主事各十二貫文洗馬贊善諸寺監主簿詹事府司直各十貫文評事八貫文諸校正各六貫文諸奉御九成總監諸王諮議及諸陵令各六貫二百文城門符寶國子助教六局郎王府掾屬太常侍醫文學録事参軍主簿記室諸衛及六軍長史兩市令諸副總監武庫署令太公廟令各五貫三百文太子通事舍人東宫三寺丞太學廣文助教内坊丞諸直長内寺伯千牛衛及諸率府長史諸陵丞諸陵署諸王府判司司竹温泉監尚書都事都水及諸總監丞司天臺丞太子侍醫諸司上局署令王府國令苑四面副監公主邑司令各四貫一百一十六文國子四門助教律醫學博士協律郎内謁者諸衛六軍左右衛率府等衛佐諸王府參軍大農都省兵吏禮房考功主事春坊録事司竹副監諸司中局署令都水主簿諸司上局署及監廟司丞司天臺靈臺郎保章挈壺正太常針醫及醫監尚醫局司醫各二貫四百七十文太祝奉禮省中諸行主事門下典儀御史臺殿中祕書内侍省春坊詹事府主事諸寺監諸衛六軍諸司録事諸司中局署丞及大理獄丞諸司監作監事諸率府録事殿中省醫佐食醫奉輦司庫司廩奉乘鴻臚寺掌客司儀太僕主乘内坊典直司天臺司辰司歷監内侍省宫教博士東宫三寺主簿太常太樂鼓吹丞醫正按摩咒禁卜筮博士及針醫卜助教國子書筭博士及助教諸王府國丞尉諸總監主簿各一貫九百一十七文武官左右金吾大將軍各四十五貫文六軍大將軍左右金吾將軍各四十貫文諸衛大將軍六軍將軍各三十貫文諸衛將軍各二十五貫文諸衛及六軍中郎將諸率府率副率各一十貫五百六十七文諸衛及六軍郎將諸王府典軍副各九貫二百文諸衛及六軍司陛千牛及左右備身各五貫三百文諸衛及六軍中候太子千牛各四貫一百六十六文諸衛及六軍司戈太子備身各二貫四百七十五文諸衛及六軍執戟及長上各一貫九百一十七文京兆及諸府尹各八十貫文少尹兩縣令各五十貫文奉先昭應醴泉等縣令司録各四十五貫文畿令各四十貫文判司兩縣丞各三十五貫文兩縣簿尉奉先等縣丞各三十貫文奉先縣簿尉諸畿令各二十五貫文畿簿尉各二十貫文參軍文學博士録事各一十貫文應給百司正員文武官月料錢外檢校官同中書門下平章事每月一百二十貫文内侍省監每月四十五貫文每年約加一十五萬六千貫文}
五月辛亥詔自都團練使外悉罷諸州團練守捉使又令諸使非軍事要急無得擅召刺史及停其職務差人權攝又定諸州兵皆有常數其召募給家糧春冬衣者謂之官健差點土人春夏歸農秋冬追集給身糧醬菜者謂之團結自兵興以來州縣官俸給不一重以元載王縉隨情徇私|{
	俸扶用翻重直用翻載祖亥翻又音如字縉音晉}
刺史月給或至千緡或數十緡至是始定節度使以下至主簿尉俸禄|{
	緡眉巾翻使疏吏翻自是年定俸之後至于會昌則又倍之節度使三十萬都防禦使副使監軍十五萬觀察使十萬諸府尹大都督府長史都團練使副使上州刺史八萬節度副使中下州刺史知軍事七萬上州别駕五萬五千長史司馬五萬觀察團練判官掌書記五萬諸大都督府司録參軍事鴘赤縣令四萬五千節度推官支使防禦判官上州録事參軍畿縣上縣令四萬諸大都督府判官赤縣丞三萬五千觀察防禦團練推官廵官鴘赤縣丞兩赤縣主簿尉上州功曹參軍以下上縣丞三萬畿縣丞鴘赤縣簿尉二萬五千畿縣上縣主簿尉二萬由會昌以前其間世有增減不可計也按類篇鴘翻阮切鷹二歲色新地理志唐京兆有赤縣次赤縣諸負郭亦皆為次赤縣鴘赤字義不可曉盖次赤也}
掊多益寡上下有叙法制粗立|{
	粗坐五翻}
庚午上遣中使發元載祖父墓斵棺弃尸毁其家廟焚其木主戊寅卓英倩等皆杖死英倩之用事也弟英璘横於鄉里|{
	璘離珍翻横戶孟翻}
及英倩下獄|{
	下戶嫁翻}
英璘遂據險作亂上禁兵討之乙巳金州刺史孫道平擊擒之 上方倚楊綰使釐革弊政會綰有疾秋七月己巳薨上痛悼之甚謂羣臣曰天不欲朕致太平何奪朕楊綰之速 八月癸未賜東川節度使鮮于叔明姓李氏|{
	使疎吏翻}
元載王縉之為相也上日賜以内厨御饌可食十人|{
	饌雛戀翻又雛晥翻食祥吏翻}
遂為故事癸卯常衮與朱泚上言餐錢已多|{
	餐錢盖所謂食料錢也泚且禮翻又音此上時掌翻}
乞停賜饌許之衮又欲辭堂封同列不可而止|{
	唐制堂封歲三千六百縑興元後纔千一百德宗尋復舊}
時人譏衮以為朝廷厚禄所以養賢不能當辭位不當辭禄|{
	朝直遥翻}


臣光曰君子恥食浮於人衮之辭禄廉恥存焉與夫固位貪禄者不猶愈乎|{
	夫音扶}
詩云彼君子兮不素餐兮|{
	詩伐檀之辭}
如衮者亦未可以深譏也

楊綰常衮薦湖州刺史顔眞卿|{
	大歷元年眞卿以攻元載貶峽州别駕尋改為吉州司馬遷撫湖二州刺史}
上即日召還|{
	還從宣翻又音如字}
甲辰以為刑部尚書|{
	尚辰羊翻}
綰衮又薦淮南判官汲人關播擢為都官員外郎|{
	唐都官郎掌配役隸簿録俘囚以給衣糧藥療以理訴競雪寃凡公私良賤必周知之屬刑部}
九月辛酉以四鎮北庭行營兼涇原鄭潁節度使段秀實為節度使 |{
	考異曰段公别傳曰自受鉞三五年間西鄰無烽燧之警又曰戎帥論乞力陀慕公清德不敢侵陵我疆舊傳亦曰三四年間吐蕃不敢犯塞按是月吐蕃寇原州十二月朱泚拒吐蕃自涇州還明年九月吐蕃逼涇州云三四年間不敢犯塞盖史家溢美之辭耳}
秀實軍令簡約有威惠奉身清儉室無姬妾非公會未嘗飲酒聽樂 吐蕃八萬衆軍于原州北長澤監|{
	盖長澤川唐舊置馬監於此吐從暾入聲下同}
己巳破方渠|{
	方渠漢縣屬北地郡後省中宗神龍三年分馬嶺置方渠縣屬慶州宋白續通典靈州方渠鎮宋初置通遠軍秦長城在城北一里}
入拔谷郭子儀使禆將李懷光救之吐蕃退庚午吐蕃寇坊州 冬十月乙酉西川節度使崔寧奏大破吐蕃於望漢城|{
	吐蕃築城於西山以望蜀因名望漢城使疎吏翻}
先是秋霖河中府池鹽多敗|{
	河中府管下安邑解縣皆有鹽池先悉薦翻}
戶部侍郎判度支韓滉恐鹽戶減税丁亥奏雨雖多不害鹽仍有瑞鹽生|{
	度徒洛翻滉呼廣翻宋白曰大歷初韓滉進漫生鹽以為靈瑞後又奏乳鹽生}
上疑其不然遣諫議大夫義興蔣鎮往視之|{
	義興漢陽羨縣地晉置義興郡及縣隋廢郡存縣以屬常州}
吐蕃寇鹽夏州又寇長武|{
	邠州宜禄縣有長武城時郭子儀遣李懷光築長武城據原首臨涇水俯瞰通道吐蕃自是不敢輕犯宋白曰長武鎮在鳳翔府麟遊縣界西至涇州四十里夏戶雅翻}
郭子儀遣將拒却之|{
	將即亮翻}
以永平軍押牙匡城劉洽為宋州刺史 |{
	考異曰舊劉玄佐傳云李靈曜據汴州洽將兵乘其無備徑入宋州按劉昌以宋州牙門將說李僧惠歸順則是僧惠先已為靈曜守宋州朝廷因授宋州刺史耳若僧惠未降則洽不能得宋州已降則不敢取宋州盖僧惠已為李忠臣所殺洽因引兵據宋州耳舊傳欲以為洽功故云然其實非也永平軍治滑州}
仍以宋泗二州隸永平軍 京兆尹黎幹奏秋霖損稼韓滉奏幹不實上命御史按視丁未還奏所損凡三萬餘頃渭南令劉澡|{
	渭南縣唐初屬華州時屬雍州宋白曰郭緣生述征記云渭南縣夷狄所置謂苻姚也}
阿附度支|{
	謂阿附韓滉度徒洛翻}
稱縣境苗獨不損御史趙計奏與澡同上曰霖雨溥博豈得渭南獨無更命御史朱敖視之損三千餘頃上歎息久之曰縣令字人之官不損猶應言損廼不仁如是乎貶澡南浦尉|{
	後魏分朐䏰縣置漁泉縣後周改曰萬川隋改曰南浦唐帶萬州}
計澧州司戶而不問滉|{
	澧音禮滉呼廣翻}
十一月壬子山南西道節度使張獻恭奏破吐蕃萬餘衆于岷州|{
	使疎吏翻吐從暾入聲}
丙辰蔣鎮還奏言瑞鹽實如韓滉所言仍上表賀|{
	還從宣翻又音如字上時掌翻}
請宣付史臣錫以嘉名上從之賜號寶應靈應池|{
	賜號者安邑池也}
時人醜之 十二月丙戌朱泚自涇州還京師|{
	朱泚自幽州入朝遂留京師因遣防秋而還還從宣翻又音如字}
丁亥崔寧奏破吐蕃十餘萬衆斬首八千餘級|{
	吐從暾入聲}
庚子以朱泚兼隴右節度使知河西澤潞行營 平盧節度使李正已先有淄青齊海登萊沂密德棣十州之地及李靈曜之亂諸道合兵攻之所得之地各為己有正已又得曹濮徐兖鄆五州因自青州徙治鄆州使其子前淄州刺史納守青州|{
	濮博木翻鄆音運考異曰實録此年二月丙戌以納為青州刺史充淄青留後至此又云為青州刺史舊正已傳云正已自青州徙居鄆州使子納及腹心之將分理其地納傳云正已擊田承嗣署奏留後尋遷青州刺史今從之}
正已用刑嚴峻所在不敢偶語然法令齊一賦均而輕擁兵十萬雄據東方鄰藩皆畏之是時田承嗣據魏博相衛洺貝澶七州|{
	澶州漢東郡頓丘縣地隋開皇十六年分頓丘置澶淵縣唐改曰澶水避高祖諱也武德四年分黎州之澶水魏州之頓丘觀州置澶州貞觀元年州廢大歷七年田承嗣表以魏州之頓丘臨黄復置澶州嗣祥吏翻相息亮翻洺音名澶時連翻}
李寶臣據恒易趙定深冀滄七州各擁衆五萬梁崇義據襄鄧均房復郢六州有衆二萬相與相根據蟠結雖奉事朝廷而不用其法令|{
	恒戶登翻朝直遥翻}
官爵甲兵租賦刑殺皆自專之上寛仁一聽其所為朝廷或完一城增一兵輒有怨言以為猜貳常為之罷役|{
	為于偽翻}
而自於境内築壘繕兵無虛日以是雖在中國名藩臣而實如蠻貊異域焉

十三年春正月辛酉勑毁白渠支流碾磑以漑田|{
	碾尼展翻磑五對翻磨也公輸班作磑後人又激水為之不煩人力引水激輪使自旋轉謂之水磨史炤曰碾磨也磑䃺也}
昇平公主有二磑入見於上請存之|{
	見賢遍翻}
上曰吾欲以利蒼生汝識吾意當為衆先公主即日毁之 戊辰回紇寇太原河東押牙泗水李自良曰|{
	紇下没翻唐節度使置都押牙牙前重職也即今制置使司帳前都提舉之職泗水縣屬兖州漢之卞縣也隋時分西界為汶陽縣於卞縣古城置泗水縣}
回紇精鋭遠來求鬬難與爭鋒不如築二壘於歸路以兵戍之虜至堅壁勿與戰彼師老自歸乃出軍乘之二壘抗其前大軍蹙其後無不捷矣留後鮑防不從遣大將焦伯瑜等逆戰癸酉遇虜於陽曲|{
	將即亮翻宋白曰陽曲漢舊縣後漢末移於太原北四十五里後魏南移於陽曲故城隋改曰陽直又移於汾陽故城改曰汾陽縣因漢汾陽縣名也煬帝又改陽直移理木井城今縣是也}
大敗而還死者萬餘人回紇縱兵大掠二月代州都督張光晟擊破之於羊武谷|{
	還從宣翻又音如字紇下没翻晟成正翻九域志代州崞縣有陽武寨}
乃引去上亦不問回紇入寇之故待之如初 己亥吐蕃遣其將馬重英帥衆四萬寇靈州|{
	吐從暾入聲帥讀曰率}
奪填漢御史尚書三渠水口以弊屯田|{
	史炤曰三渠謂填漢渠御史渠尚書渠也填讀曰鎮}
三月甲戌回紇使還過河中朔方軍士掠其輜重|{
	朔方軍士之留屯河中者使疎吏翻過古禾翻又古臥翻輜莊持翻重直用翻}
因大掠坊市 夏四月甲辰吐蕃寇靈州朔方留後常謙光擊破之 六月戊戌隴右節度使朱泚獻猫鼠同乳不相害者以為瑞|{
	泚且禮翻又音此乳如住翻}
常衮帥百官稱賀|{
	帥讀曰率}
中書舍人崔祐甫獨不賀曰物反常為妖猫捕鼠乃其職也今同乳妖也|{
	妖於驕翻}
何乃賀為宜戒法吏之不察姧邊吏之不禦寇者以承天意上嘉之祐甫沔之子也|{
	崔沔開元名臣沔彌兖翻}
秋七月以祐甫知吏部選事祐甫數以公事與常衮爭由是惡之|{
	為衮奏貶祐甫張本選須絹翻數所角翻惡烏路翻}
戊午郭子儀奏以回紇猶在塞上邊人恐懼請遣邠州刺史渾瑊將兵鎮振武軍|{
	邠卑旻翻瑊古咸翻將即亮翻又音如字振武軍在單于都護府城内秦漢之雲中郡城也宋白曰振武軍即漢定襄郡之盛樂縣也在隂山之陽黄河之北}
從之回紇始去 辛未吐蕃將馬重英二萬衆寇鹽慶二州|{
	重直龍翻}
郭子儀遣朔方都虞候李懷光擊却之 八月乙亥成德節度使李寶臣請復姓張許之|{
	寶臣賜姓見二百二十二卷寶應元年使疏吏翻}
吐蕃二萬衆寇銀麟州略党項雜畜|{
	吐從暾入聲銀州漢西河郡圁隂縣地麟州漢新秦中之地党底朗翻畜許救翻}
郭子儀遣李懷光擊破之 上悼念貞懿皇后不已|{
	貞懿皇后獨孤妃也十年薨}
殯于内殿累年不忍葬丁酉始葬于莊陵|{
	莊陵在京兆三原縣西北五里}
九月庚午吐蕃萬騎下青石嶺逼涇州|{
	青石嶺在涇州保定縣西宋白曰臨涇城直涇州西北九十里其界有青石嶺騎奇寄翻}
詔郭子儀朱泚與段秀實共却之 冬十二月丙戌以吏部尚書轉運鹽鐵等使劉晏為左僕射知三銓及使職如故|{
	尚辰羊翻射寅謝翻歐陽修曰凡選有文武文選吏部主之武選兵部主之皆為三銓尚書侍郎分主之尚書掌其一侍郎分其二}
郭子儀入朝命判官京兆杜黄裳主留務李懷光隂謀代子儀矯為詔書欲誅大將温儒雅等黄裳察其詐以詰懷光|{
	朝直遥翻將即亮翻詰去吉翻}
懷光流汗服罪於是諸將之難制者黄裳矯子儀之命皆出之於外軍府乃安 以給事中杜亞為江西觀察使上召江西判官李泌入見|{
	泌毗必翻李泌出佐江西見上卷五年見賢遍翻}
語以元載事曰與卿别八年乃能誅此賊賴太子其隂謀|{
	此歸功於太子耳語牛倨翻}
不然幾不見卿對曰臣昔日固嘗言之陛下知羣臣有不善則去之|{
	幾居依翻去羌呂翻}
含容太過故至於此上曰事亦應十全不可輕上因言朕面屬卿於路嗣恭|{
	屬之欲翻}
而嗣恭取載意奏卿為䖍州别駕嗣恭初平嶺南獻琉璃盤徑九寸朕以為至寶及破載家得嗣恭所遺載琉璃盤徑尺|{
	遺于季翻程大昌曰漢西域傳罽賓國有琥珀流離師古注曰魏略云大秦國出赤白黑黄青緑縹紺紅紫十種流離此盖自然之物采澤光潤踰于衆玉今俗所用皆消冶石汁加以衆藥灌而為之虛脆不耐實非眞物案流離今書附玉旁為琉璃字師古之記是矣亦未得其詳也穆天子傳天子東征有采石之山凡好石之器於是出升山取采石鑄以成器注云采石文采之石也則鑄石為器古有之矣顔氏謂為自然之物恐不詳也北史大月氏傳魏太武時月氏人商販京師自云能鑄石為五色琉璃於是采礦於山中即京師鑄之既成光澤乃美於西方來者自是琉璃遂賤由此言推之則雖西域琉璃亦用石鑄無自然生成者兼外國奇產中國未始無之獨不聞有所謂眞琉璃也然中國所鑄有與西域異者鑄之中國色甚光鮮而質則輕脆沃以熱酒隨手破裂其來自海舶者制差鈍樸而色亦微暗其可異者雖百沸湯注之與磁銀無異了不復動是名蕃琉璃也蕃琉璃之異於中國其别盖如此而未嘗聞以石琢之也余謂路嗣恭所獻者盖師古所謂大秦琉璃自然之物否則代宗何以謂之至寶哉程大昌考之不詳耳}
俟其至當與卿議之泌曰嗣恭為人小心善事人畏權勢精勤吏事而不知大體昔為縣令有能名|{
	路嗣恭始名劍客為蕭關令連徙神烏姑臧二縣考績為天下最玄宗以為可嗣漢魯恭因賜名}
陛下未暇知之而為載所用故為之盡力陛下誠知而用之彼亦為陛下盡力矣|{
	為之亦為于偽翻}
䖍州别駕臣自欲之非其罪也且嗣恭新立大功|{
	即謂平嶺南之功}
陛下豈得以一琉璃盤罪之邪上意乃解以嗣恭為兵部尚書|{
	邪音耶尚辰羊翻}
郭子儀以朔方節度副使張曇性剛率謂其以武人輕已銜之|{
	曇徒含翻銜戶緘翻}
孔目官吴曜為子儀所任|{
	諸鎮州皆有孔目官以綜理衆事吏職也言一孔一目皆所綜理也}
因而搆之子儀怒誣奏曇扇動軍衆誅之掌書記高郢力爭之子儀不聽奏貶郢猗氏丞|{
	猗氏縣屬河中府宋白曰本郇國地猗頓於此起富故曰猗氏}
既而僚佐多以病求去子儀悔之悉薦之於朝曰吴曜誤我遂逐之|{
	史言郭子儀過而能改朝直遥翻}
常衮言於上曰陛下久欲用李泌|{
	泌毘必翻}
昔漢宣帝欲用人為公卿必先試理人請且以為刺史使周知人間利病俟報政而用之|{
	太公治齊五月而報政伯禽治魯三年而報政常衮用此語也}


十四年春正月壬戍以李泌為澧州刺史|{
	澧音禮}
二月癸未魏博節度使田承嗣薨|{
	使疏吏翻嗣祥吏翻}
有子十一人以其姪中軍兵馬使悦為才使知軍事而諸子佐之甲申以悦為魏博留後|{
	為田緒殺悦張本}
淮西節度使李忠臣貪殘好色將吏妻女美者多逼淫之悉以軍政委妺壻節度副使張惠光惠光挾勢暴横軍州苦之忠臣復以惠光子為牙將暴横甚於其父|{
	好呼到翻横戶孟翻復扶又翻將即亮翻}
左廂都虞候李希烈忠臣之族子也為衆所服希烈因衆心怨怒三月丁未與大將丁暠等|{
	暠古老翻}
殺惠光父子而逐忠臣忠臣單騎奔京師|{
	騎奇寄翻}
上以其有功|{
	吐蕃寇京師忠臣先諸鎮赴援又有平李靈曜之功}
使以檢校司空同平章事留京師以希烈為蔡州刺史淮西留後|{
	為李希烈以淮蔡畔援張本}
以永平節度使李勉兼汴州刺史增領汴潁二州徙鎮汴州|{
	永平軍本治滑州汴皮變翻}
辛酉以容管經略使王翃為河中少尹知府事河東副元帥留後部將凌正暴横翃抑之|{
	使疏吏翻翃戶萌翻少始照翻帥所類翻將即亮翻横戶孟翻}
正與其徒乘夜作亂翃知之故縮漏水數刻以差其期賊驚潰走擒正誅之軍府乃安 成德節度使張寶臣既請復姓|{
	去年寶臣請復姓張}
又不自安更請賜姓夏四月癸未復賜姓李|{
	復扶又翻}
五月癸卯上始有疾辛酉制皇太子監國|{
	監古銜翻}
是夕上崩于紫宸之内殿|{
	上年五十二紫宸殿在東内宣政殿之北蓬萊殿之南}
遺詔以郭子儀攝冢宰癸亥德宗即位在諒隂中動遵禮法嘗召韓王迥食|{
	迥德宗弟也}
食馬齒羹不設鹽酪|{
	馬齒莧也酪音洛乳酪也}
常衮性剛急為政苛細不合衆心時羣臣朝夕臨|{
	臨力鴆翻哭也}
衮哭委頓從吏或扶之|{
	從才用翻}
中書舍人崔祐甫指以示衆曰臣哭君前有扶禮乎衮聞益恨之會議羣臣喪服衮以為禮臣為君斬衰三年|{
	為于偽翻衰倉回翻}
漢文權制猶三十六日|{
	事見十五卷前漢文帝後七年}
高宗以來皆遵漢制及玄宗肅宗之喪始服二十七日今遺詔云天下吏人三日釋服古者卿大夫從君而服皇帝二十七日而除在朝羣臣亦當如之祐甫以為遺詔無朝臣庶人之别|{
	朝直遥翻下同别彼列翻}
朝野内外莫非天下凡百執事孰非吏人皆應釋服相與力爭聲色陵厲衮不能堪乃奏祐甫率情變禮請貶潮州刺史上以為太重閏月壬申貶祐甫為河南少尹初肅宗之世天下務殷宰相常有數人更直决事|{
	更工衡翻}
或休沐各歸私第詔直事者代署其名而奏之自是踵為故事時郭子儀朱泚雖以軍功為宰相皆不預朝政衮獨居政事堂|{
	唐初政事堂在門下省裴炎自侍中遷中書令乃徙政事堂於中書省三省長官議事於此}
代二人署名奏祐甫祐甫既貶二人表言其非罪上問卿向言可貶今云非罪何也二人對初不知上初即位以衮為欺罔大駭甲辰百官衰絰序立于月華門|{
	程大昌日按六典宣政殿前有兩廡兩廡各自有門其東曰日華日華之東則門下省也西廊有門曰月華月華之西則中書省也衰倉回翻}
有制貶衮為潮州刺史|{
	潮州去京師五千許里}
以祐甫為門下侍郎同平章事聞者震悚祐甫至昭應而還|{
	昭應縣本新豐縣垂拱二年改曰應山神龍元年復故名玄宗更名昭應隋新豐治古新豐城北天寶昭應縣治昭應宫北還從宣翻又音如字}
既而羣臣喪服竟用衮議上時居諒隂庶政皆委於祐甫所言無不允初至德以後天下用兵諸將競論功賞故官爵不能無濫及永泰以來天下稍平而元載王縉秉政四方以賄求官者相屬於門|{
	將即亮翻論魯昆翻載祖亥翻又音如字縉音晉屬之欲翻}
大者出於載縉小者出於卓英倩等皆如所欲而去及常衮為相思革其弊杜絶僥幸四方奏請一切不與而無所甄别賢愚同滯|{
	相息亮翻僥堅堯翻甄稽延翻察也别彼列翻}
崔祐甫代之欲收時望推薦引拔常無虛日作相未二百日除官八百人 |{
	考異曰舊紀云祐甫作相未逾年凡除吏幾八百員多稱允當今從建中實録}
前後相矯終不得其適上嘗謂祐甫曰人或謗卿所用多涉親故何也對曰臣為陛下選擇百官|{
	為于偽翻}
不敢不詳慎苟平生未之識何以諳其才行而用之|{
	諳烏含翻行下孟翻下同}
上以為然

臣光曰臣聞用人者無親疎新故之殊惟賢不肖之為察其人未必賢也以親故而取之固非公也苟賢矣以親故而捨之亦非公也夫天下之賢固非一人所能盡也若必待素識熟其才行而用之所遺亦多矣|{
	夫音扶行下孟翻}
古之為相者則不然舉之以衆取之以公衆曰賢矣己雖不知其詳姑用之待其無功然後退之有功則進之所舉得其人則賞之非其人則罰之進退賞罰皆衆人所共然也己不置豪髪之私於其間苟推是心以行之又何遺賢曠官之足病哉

詔罷省四方貢獻之不急者又罷梨園使及樂工三百餘人|{
	使疏吏翻梨園事始二百一十一卷玄宗開元二年程大昌曰梨園在光化門北光化門者禁苑南面西頭第一門}
所留者悉隸太常 郭子儀以司徒中書令領河中尹靈州大都督單于鎮北大都護關内河東副元帥朔方節度關内支度鹽池六城水運大使押蕃部及營田及河陽道觀察等使|{
	河中靈夏皆有鹽池朔方塞下有六城單音蟬帥所類翻使疏吏翻}
權任既重功名復大|{
	復扶又翻}
性寛大政令頗不肅代宗欲分其權而難之久不决甲申詔尊子儀為尚父|{
	太公望為周師尚父說者謂可尚可父天子師也}
加太尉兼中書令增實封滿二千戶月給一千五百人糧二百馬食子弟諸壻遷官者十餘人所領副元帥諸使悉罷之以其禆將河東朔方都虞候李懷光為河中尹邠寧慶晉絳慈隰節度使以朔方留後兼靈州長史常謙光為靈州大都督西受降城定遠天德鹽夏豐等軍州節度使振武軍使渾瑊為單于大都護東中二受降城振武鎮北綏銀麟勝等軍州節度使分領其任|{
	將即亮翻邠卑旻翻長知兩翻降戶江翻夏戶雅翻渾戶昆翻又戶本翻瑊古咸翻}
丙戌詔曰澤州刺史李鷃|{
	鷃烏諫翻}
上慶雲圖|{
	上時掌翻下得上上言同}
朕以時和年豐為嘉祥以進賢顯忠為良瑞如卿雲靈芝珍禽奇獸怪草異木何益於人|{
	卿雲即慶雲也}
布告天下自今有此無得上獻内莊宅使上言諸州有官租萬四千餘斛上令分給所在充軍儲先是諸國屢獻馴象凡四十有二|{
	令力丁翻先悉薦翻}
上曰象費豢養而違物性將安用之命縱於荆山之陽|{
	此禹貢所謂導汧及岐至于荆山者也唐屬京兆富平縣界}
及豹貀鬬雞獵犬之類悉縱之|{
	貀似豹無前足音女滑翻史炤曰貀似貍蒼黑無前足善捕鼠}
又出宫女數百人於是中外皆悦淄青軍士至投兵相顧曰明主出矣吾屬猶反乎|{
	淄莊持翻}
戊子以淮西留後李希烈為節度使|{
	使疏吏翻}
辛卯以河陽鎮遏使馬燧為河東節度使河東承百井之敗|{
	謂去年鮑防之敗也按東都事略張齊賢傳柏井在并州城北四十里宋朝徙并州城於陽曲縣}
騎士單弱燧悉召牧馬厮役得數千人教之數月皆為精騎|{
	騎奇寄翻}
造甲必為長短三等稱其所衣|{
	稱尺證翻衣於既翻}
以便進趨又造戰車行則載兵甲止則為營陳或塞險以遏奔衝|{
	陳讀曰陣塞悉則翻}
器械無不精利居一年得選兵三萬辟兖州人張建封為判官署李自良代州刺史委任之 兵部侍郎黎幹狡險諛佞與宦官特進劉忠翼相親善忠翼本名清譚恃寵貪縱二人皆為衆所惡|{
	惡烏路翻}
時人或言幹忠翼嘗勸代宗立獨孤貴妃為皇后妃子韓王迥為太子上即位幹密乘轝詣忠翼謀事事覺丙申幹忠翼並除名長流至藍田賜死 以戶部侍郎判度支韓滉為太常卿以吏部尚書劉晏判度支|{
	度徒洛翻滉呼廣翻尚辰羊翻}
先是晏滉分掌天下財賦|{
	先悉薦翻大歷六年韓滉判度支分掌財賦當在此時}
晏掌江南山南江淮嶺南滉掌關内河東劒南至是晏始兼之上素聞滉掊克過甚|{
	掊蒲侯翻}
故罷其利權尋出為晉州刺史|{
	晉州治臨汾縣古平陽也京師東北七百二十五里}
至德初第五琦始榷鹽以佐軍用|{
	事見二百一十九卷至德元載琦音奇榷古岳翻}
及劉晏代之法益精密初歲入錢六十萬緡末年所入逾十倍而人不厭苦大歷末計一歲所入總一千二百萬緡|{
	緡眉巾翻}
而鹽利居其大半以鹽爲漕傭自江淮至渭橋|{
	此東渭橋也}
率萬斛傭七千緡自淮以北列置廵院擇能吏主之不煩州縣而集事 六月己亥朔赦天下 西川節度使崔寧永平節度使李勉並同平章事|{
	使疏吏翻}
詔天下寃滯州府不為理|{
	為于偽翻}
聽詣三司使|{
	所謂三司使即御史中丞中書省舍人門下省給事中也三人者各以一司官來朝堂受詞故謂之三司非五代時理財之三司使也}
以中丞舍人給事中各一人日於朝堂受詞推决尚未盡者聽撾登聞鼓|{
	唐時登聞鼓在西朝堂之前撾側瓜翻}
自今無得復奏置寺觀|{
	復扶又翻}
及請度僧尼於是撾登聞鼓者甚衆右金吾將軍裴諝上疏|{
	諝私呂翻}
以為訟者所爭皆細故若天子一一親之則安用吏理乎上乃悉歸之有司 制應山陵制度務從優厚當竭帑藏以供其費|{
	帑他朗翻藏徂浪翻}
刑部員外郎令狐峘上疏諫|{
	令力丁翻峘胡登翻上時掌翻}
其略曰臣伏讀遺詔務從儉約若制度優厚豈顧命之意邪|{
	邪音耶}
上答詔略曰非唯中朕之病|{
	中竹仲翻}
抑亦成朕之美敢不聞義而徙峘德棻之玄孫也|{
	令狐德棻事太宗棻扶分翻}
庚子立皇子誦為宣王謨為舒王諶為通王諒為䖍王詳為肅王乙巳立皇弟廼為益王傀為蜀王|{
	諶氏壬翻傀口猥翻又公回翻皆以州名為王國名}
丙午舉先天故事六品以上清望官雖非供奉侍衛之官日令二人更直待制以備顧問|{
	新志曰初太宗即位命京官五品以上更宿中書門下兩省以備訪問永徽巾命弘文館學士一人日待制於武德殿西門文明元年詔京官五品以上清官日一人待制于章善明福門先天末又命朝集使六品以上隨仗待制永泰時勲臣罷節制無職事皆待制于集賢門凡十三人崔祐甫為相建議文官一品以下更直待制其後著令正衙待制官日二人宋白曰時祐甫奏准元敕文官一品以下更直待制其待制官待奏事官盡然後趨出便於内廊賜待進止至酉時然後放令力丁翻更工衡翻下同}
庚戍以朱泚為鳳翔尹|{
	泚且禮翻又音此}
代宗優寵宦官奉使四方者不禁其求取嘗遣中使賜妃族還問所得頗少|{
	使疎吏翻還從宣翻又音如字少詩沼翻}
代宗不悦以為輕我命妃懼遽以私物償之由是中使公求賂遺無所忌憚|{
	遺于季翻}
宰相嘗貯錢於閣中每賜一物宣一旨無徒還者|{
	相息亮翻貯丁呂翻}
出使所歷州縣移文取貨與賦税同皆重載而歸上素知其弊遣中使邵光超賜李希烈旌節希烈贈之僕馬及縑七百匹黄茗二百斤|{
	茗莫迥翻茶之晩取者}
上聞之怒杖光超六十而流之於是中使之未歸者皆濳弃所得於山谷雖與之莫敢受 甲子以神策都知兵馬使右領軍大將軍王駕鶴為東都園苑使|{
	東都園苑使唐初苑總監之職也}
以司農卿白琇珪代之|{
	琇息救翻}
更名志貞駕鶴典禁兵十餘年權行中外上恐其生變崔祐甫呂駕鶴與語留連久之琇珪己視事矣李正已畏上威名表獻錢三十萬緡上欲受之恐見

欺却之則無辭崔祐甫請遣使慰勞淄青將士|{
	使疏吏翻勞力到翻將即亮翻}
因以正已所獻錢賜之使將士人人戴上恩又諸道聞之知朝廷不重貨財上悦從之正已大慙服天下以為太平之治庶幾可望焉|{
	治直吏翻幾居衣翻}
秋七月戊辰朔日有食之 禮儀使吏部尚書顔眞卿上言上元中政在宫壼始增祖宗之諡|{
	尚辰羊翻上時掌翻壼若本翻宫中道也按咸亨五年八月十五日改元上元是日追尊高祖太宗政在宫壼謂武后專政也}
玄宗末姦臣竊命累聖之諡有加至十一字者|{
	按天寶十三載加祖宗謚號并廟號皆為九字而羣臣上玄宗尊號凡十四字未知顔眞卿所謂加至十一字何帝也謚神至翻}
按周之文武言文不稱武言武不稱文豈盛德所不優乎蓋羣臣稱其至者故也故諡多不為褒少不為貶|{
	少始紹翻}
今累聖諡號太廣有踰古制請自中宗以上皆從初謚|{
	初謚高祖太武皇帝太宗文皇帝高宗天皇大帝中宗孝和皇帝}
睿宗曰聖眞皇帝玄宗曰孝明皇帝肅宗曰宣皇帝以省文尚質正名敦本上命百官集議儒學之士皆從眞卿議獨兵部侍郎袁傪|{
	傪昌含翻又七感翻}
官以兵進奏言陵廟玉册木主皆已刋勒不可輕改事遂寢不知陵中玉册所刻乃初謚也|{
	按唐陵中玉册自睿宗聖眞皇帝以上所刻皆初謚然玄宗謚册曰至道大聖大明孝皇帝肅宗謚册曰文明武德大聖大宣孝皇帝袁傪所謂木主玉册皆已刋勒有見乎此耳}
初代宗之世事多留滯四夷使者及四方奏計或連歲不遣乃於右銀臺門|{
	使疏吏翻右銀臺門在東内宫城西面又北則九仙門}
置客省以處之|{
	處昌呂翻}
及上書言事失職未叙亦寘其中動經十歲常有數百人并部曲畜產動以千計度支廩給其費甚廣|{
	度徒洛翻}
上悉命疏理拘者出之事竟者遣之當叙者任之歲省穀萬九千二百斛 壬申毁元載馬璘劉忠翼之第|{
	載祖亥翻又音如字璘離珍翻}
初天寶中貴戚第舍雖極奢麗而垣屋高下猶存制度然李靖家廟已為楊氏馬廐矣及安史亂後法度墮弛|{
	墮讀曰隳}
大臣將帥競治第舍各窮其力而後止時人謂之木妖|{
	將即亮翻帥所類翻治直之翻妖於驕翻}
上素疾之故毁其尤者仍命馬氏獻其園隸宫司|{
	宫司掌宫禁園籞者也}
謂之奉成園|{
	雍録奉成園在安邑坊自丹鳳門南出東街第六坊為安邑坊}
癸丑減常貢宫中服用錦千匹服玩數千事 庚辰詔回紇諸胡在京師者各服其服無得效華人先是回紇留京師者常千人|{
	紇下没翻先悉薦翻}
商胡偽服而雜居者又倍之縣官日給饔餼|{
	熟曰饔生曰餼餼許既翻}
殖貲產開第舍市肆美利皆歸之日縱貪横|{
	横戶孟翻}
吏不敢問或衣華服|{
	衣於既翻華服中華之服也}
誘取妻妾故禁之|{
	誘羊久翻}
辛卯罷天下榷酒收利|{
	唐初無酒禁乾元元年京師酒貴肅宗以廩食方絀乃禁京師酤酒期以麥熟如初二年饑復禁酤廣德二年定天下酤戶以月收税榷古岳翻}
上之在東宫也國子博士河中張涉為侍讀|{
	太宗時晉王府有侍讀及為太子亦置焉}
即位之夕召涉入禁中事無大小皆咨之明日置於翰林為學士|{
	翰林故事曰翰林院者在銀臺門内以藝能伎術召見者之所處也玄宗初置翰林待詔掌四方表疏批答應和文章又以詔勑文誥悉由中書多壅滯始選朝官有才藝學識者入居翰林供奉别旨然亦未定名制詔書敕猶或分在集賢開元二十六年始以翰林供奉改稱學士别建學士院於翰林院之南俾專内命其後又置東翰林院於金鑾殿之西隨上所在凡學士無定員下自校書郎上及諸曹尚書皆為之入院一歲則遷知制誥未知制誥者不作文書久次者一人為承旨}
親重無比|{
	為張涉以賦賄得罪張本}
乙未以涉為右散騎常侍仍為學士|{
	散悉亶翻騎奇寄翻}


資治通鑑卷二百二十五
