\section{資治通鑑卷六十七}
宋 司馬光 撰

胡三省 音註

漢紀五十九|{
	起閼逢敦牂盡柔兆涒灘凡三年}


孝獻皇帝壬

建安十九年春馬超從張魯求兵北取凉州魯遣超還圍祁山姜敘告急於夏侯淵諸將議欲須魏公操節度淵曰公在鄴反覆四千里比報敘等必敗非救急也|{
	比必寐翻}
遂行使張郃督步騎五千為前軍|{
	郃古合翻又曷閤翻}
超敗走韓遂在顯親|{
	顯親縣屬漢陽郡班志無之盖光武所置以封竇友賢曰顯親故城在今秦州成紀縣東}
淵欲襲取之遂走淵追至畧陽城去遂三十餘里諸將欲攻之或言當攻興國氐|{
	魏畧曰建安中興國氐王阿貴百頃氐王千萬各有部落萬餘從馬超為亂超破之後阿貴為夏侯淵所攻滅千萬南入蜀}
淵以為遂兵精興國城固攻不可卒抜不如擊長離諸羌|{
	水經註瓦亭水南逕隴西成紀縣東歷長離川謂之長離水燒當等羌居之卒讀曰猝}
長離諸羌多在遂軍必歸救其家若捨羌獨守則孤|{
	謂遂若捨羌而不救獨擁兵自守則其勢孤}
救長離則官兵得與野戰必可虜也淵乃留督將守輜重|{
	重直用翻}
自將輕兵到長離攻燒羌屯遂果救長離諸將見遂兵衆欲結營作塹乃與戰|{
	塹七艷翻}
淵曰我轉鬭千里今復作營塹則士衆罷敝不可復用|{
	復扶又翻罷讀曰疲}
賊雖衆易與耳|{
	易以䜴翻}
乃鼔之大破遂軍進圍興國氐王千萬犇馬超餘衆悉降轉擊高平屠各皆破之|{
	屠直如翻}
三月詔魏公操位在諸侯王上改授金璽赤紱遠游冠|{
	漢制諸侯王金印赤紱遠遊冠董巴曰遠游冠制如通天高九寸正豎頂少邪乃直下為鐵卷梁有展筩横之於前無山述}
夏四月旱五月雨水 初魏公操遣廬江太守朱光屯皖|{
	皖戶板翻}
大開稻田呂蒙言於孫權曰皖田肥美若一收孰彼衆必增|{
	收孰謂稻成熟而收之也有糧則可以增衆孰古熟字通}
宜早除之閏月權親攻皖城諸將欲作土山添攻具呂蒙曰治攻具及土山必歷日乃成|{
	治直之翻}
城備既修外救必至不可圖也且吾乘雨水以入若留經日水必向盡還道艱難蒙竊危之今觀此城不能甚固以三軍鋭氣四面並攻不移時可拔及水以歸全勝之道也權從之蒙薦甘寧為升城督寧手持練身緣城為士卒先蒙以精鋭繼之手執枹鼓|{
	枹音膚}
士卒皆騰踊侵晨進攻食時破之獲朱光及男女數萬口既而張遼至夾石|{
	夾石在今安慶府桐城縣北四十七里今名西峽山}
聞城已拔乃退權拜呂蒙為廬江太守|{
	守式又翻}
還屯尋陽 諸葛亮留關羽守荆州與張飛趙雲將兵泝流克巴東|{
	譙周巴記曰初平六年趙韙分巴郡安漢以下為永寧郡建安六年劉璋以永寧為巴東郡唐夔州開州之地也}
至江州破巴郡太守嚴顔生獲之飛呵顔曰|{
	呵虎何翻}
大軍既至何以不降而敢拒戰顔曰卿等無狀侵奪我州我州但有斷頭將軍無降將軍也|{
	降戶江翻下同我州謂益州也}
飛怒令左右牽去砍頭顔容止不變曰砍頭便砍頭何為怒邪飛壯而釋之引為賓客分遣趙雲從外水定江陽犍為|{
	江陽縣本屬犍為郡劉璋分立江陽郡唐為瀘州犍為郡唐為資簡嘉眉之地今渝州亦漢巴郡地也對二水口右則涪内水左則蜀外水自渝上合州至綿州者謂之内水自渝上戎瀘至蜀者謂之外水犍居言翻}
飛定巴西德陽|{
	譙周巴記建安六年劉璋分巴郡墊江以上為巴西德陽縣屬廣漢郡唐遂州地}
劉備圍雒城且一年龎統為流矢所中卒灋正牋與劉璋為陳形勢彊弱|{
	中竹仲翻卒子恤翻為于偽翻}
且曰左將軍從舉兵以來舊心依依實無薄意|{
	盖時人以璋倚備為用備反襲璋議備之薄也}
愚以為可圖變化以保尊門|{
	尊門謂璋家門}
璋不答雒城潰備進圍成都諸葛亮張飛趙雲引兵來會馬超知張魯不足與計事又魯將楊昂等數害其能超内懷於邑|{
	數所角翻師古曰於邑短氣貌讀並如字又於音烏邑音烏合翻}
備使建寧督郵李恢往說之|{
	蜀志後主建興三年改益州郡為建寧郡恢此時盖為益州郡督郵史因後改郡名而書之耳說輸芮翻下同}
超遂從武都逃入氐中密書請降於備備使人止超而潛以兵資之超到令引軍屯城北城中震怖|{
	怖普布翻}
備圍城數十日使從事中郎涿郡簡雍入說劉璋|{
	簡姓也魯有大夫簡叔蜀志曰簡雍姓耿後音訛為簡}
時城中尚有精兵三萬人糓帛支一年吏民咸欲死戰璋言父子在州二十餘年|{
	靈帝中平五年劉焉牧益州至是二十七年}
無恩德以加百姓百姓攻戰三年肌膏草野者以璋故也|{
	膏古報翻}
何心能安遂開城與簡雍同輿出降|{
	降戶江翻}
羣下莫不流涕備遷璋於公安盡歸其財物佩振威將軍印綬|{
	曹公先加璋振威將軍故仍佩其印綬}
備入成都置酒大饗士卒取蜀城中金銀分賜將士還其糓帛|{
	凡城中公私所有金銀悉取以分賜將士至於穀帛則各還所主也}
備領益州牧以軍師中郎將諸葛亮為軍師將軍益州太守|{
	此益州太守非漢武帝所開置之益州郡也武帝所置之益州郡劉蜀為南中地宅盖劉璋置益州太守與蜀郡太守並治成都郭下}
南郡董和為掌軍中郎將並署左將軍府事|{
	署府事者總録軍府事也}
偏將軍馬超為平西將軍|{
	晉百官志四平立于喪亂謂平柬平西平南平北四將軍也}
軍議校尉灋正為蜀郡太守揚武將軍禆將軍南陽黄忠為討虜將軍從事中郎糜竺為安漢將軍|{
	漢大將軍府有從事中郎職參謀議}
簡雍為昭德將軍北海孫乾為秉忠將軍|{
	安漢昭德秉忠皆備所置將軍號也}
廣漢長黄權為偏將軍|{
	長知兩翻}
汝南許靖為左將軍長史龎羲為司馬|{
	龎皮江翻}
李嚴為犍為太守|{
	犍居言翻}
費觀為巴郡太守|{
	費父沸翻}
山陽伊籍為從事中郎零陵劉巴為西曹掾|{
	掾俞絹翻}
廣漢彭羕為益州治中從事|{
	羕餘亮翻}
初董和在郡清儉公直為民夷所愛信蜀中推為循吏故備舉而用之備之自新野犇江南也|{
	事見六十五卷十三年}
荆楚羣士從之如雲而劉巴獨北詣魏公操操辟為掾遣招納長沙零陵桂陽會備畧有三郡巴事不成欲由交州道還京師時諸葛亮在臨蒸|{
	沈約曰吳立衡陽郡臨蒸縣屬焉盖吳所置也水經註蒸水出衡陽重安縣西邵陵縣界耶薑山東北流過臨蒸縣北東注于湘謂之蒸口}
以書招之巴不從備深以為恨巴遂自交阯入蜀依劉璋及璋迎備巴諫曰備雄人也入必為害既入巴復諫曰|{
	復扶又翻}
若使備討張魯是放虎於山林也璋不聽巴閉門稱疾備攻成都令軍中曰有害巴者誅及三族及得巴甚喜是時益州郡縣皆望風景附獨黄權閉城堅守須璋稽服乃降|{
	稽音啟言稽顙服從也降戶江翻下同}
於是董和黄權李嚴等本璋之所授用也|{
	璋以和為益州太守權為府主簿嚴為護軍}
吴懿費觀等璋之婚親也|{
	璋兄瑁娶吳懿妹璋母費氏}
彭羕璋之所擯棄也|{
	羕仕益州不過書佐人毁之於璋髠鉗為徒隸}
劉巴宿昔之所忌恨也備皆處之顯任|{
	處昌呂翻}
盡其器能有志之士無不競勸益州之民是以大和初劉璋以許靖為蜀郡太守成都將潰靖謀踰城降備備以此薄靖不用也灋正曰天下有獲虚譽而無其實者許靖是也|{
	許靖與弟劭並有高名汝南月旦評二人者為之也}
然今主公始創大業|{
	主公之稱始於東都改明公稱主公尊事之為主也}
天下之人不可戶說|{
	不可戶戶而說之也說如字}
宜加敬重以慰遠近之望備乃禮而用之成都之圍也備與士衆約若事定府庫百物孤無預焉及拔成都士衆皆捨干戈赴諸藏|{
	藏徂浪翻}
競取寶物軍用不足備甚憂之劉巴曰此易耳|{
	易以䜴翻}
但當鑄直百錢|{
	直百錢一錢直百也杜佑曰蜀鑄直百錢文曰直百亦冇勒為五銖者大小稱兩如一焉並徑七分重四銖}
平諸物價令吏為官市備從之數月之間府庫充實時議者欲以成都名田宅分賜諸將趙雲曰霍去病以匈奴未滅無用家為|{
	事見十九卷武帝元狩四年}
今國賊非但匈奴未可求安也須天下都定各反桑梓|{
	都定猶言皆定也桑梓謂其故郷祖父之所樹者詩云維桑與梓必恭敬止}
歸耕本土乃其宜耳益州人民初罹兵革田宅皆可歸還令安居復業然後可役調|{
	調徒弔翻}
得其歡心不宜奪之以私所愛也備從之備之襲劉璋也留中郎將南郡霍峻守葭萌城張魯遣楊昂誘峻求共守城峻曰小人頭可得城不可得昂乃退後璋將扶禁向存等帥萬餘人由閬水上|{
	扶姓禁名帥讀曰率閬水即西漢水禹貢所謂嶓冢導漾東流為漢者也水經註漾水出隴西氐道縣嶓冢山謂之西漢水東南至廣漢白水縣西又東南至葭萌縣又東南過巴郡閬中縣與閬水會水出閬陽縣而東逕其縣南又東注漢水昔劉璋攻霍峻於葭萌也自此水上又東南入漢州江津縣東南入于江余據此水今謂之嘉陵江}
攻圍峻且一年峻城中兵纔數百人伺其怠隙選精鋭出擊大破之斬存|{
	伺相吏翻}
備既定蜀乃分廣漢為梓潼郡|{
	唐梓州之地宋白曰綿州巴西縣本漢涪縣屬廣漢郡華陽國志漢元初二年廣漢自繩鄉移治涪後治雒劉備立梓潼郡以縣屬焉隋改為巴西縣唐梓州治郪天寶方改為梓潼郡}
以峻為梓潼太守灋正外統都畿|{
	備都成都以蜀郡為都畿}
内為謀主一飱之德睚眦之怨無不報復|{
	飱千安翻睚五懈翻眦士懈翻}
擅殺毁傷己者數人或謂諸葛亮曰灋正太縱横|{
	横戶孟翻}
將軍宜啟主公抑其威福亮曰主公之在公安也北畏曹操之彊東憚孫權之逼近則懼孫夫人生變於肘腋|{
	事見上卷十四年}
灋孝直為之輔翼令翻然翱翔不可復制|{
	謂迎備入益州也復扶又翻}
如何禁止孝直使不得少行其意邪|{
	法正字孝直少詩沼翻}
諸葛亮佐備治蜀頗尚嚴峻人多怨歎者|{
	治直之翻}
灋正謂亮曰昔高祖入關約灋三章秦民知德|{
	事見九卷高帝元年}
今君假借威力跨據一州初有其國未垂惠撫且客主之義宜相降下|{
	下遐稼翻}
願緩刑弛禁以慰其望|{
	以亮等初至為客益州人士則主也}
亮曰君知其一未知其二秦以無道政苛民怨匹夫大呼|{
	呼火故翻}
天下土崩高祖因之可以弘濟|{
	弘大也}
劉璋暗弱自焉以來|{
	焉璋父也}
有累世之恩文灋覊縻互相承奉德政不舉威刑不肅蜀土人士專權自恣君臣之道漸以陵替寵之以位位極則賤順之以恩恩極則慢所以致敝實由於此吾今威之以灋灋行則知恩限之以爵爵加則知榮榮恩並濟上下有節為治之要於斯而著矣|{
	孔子曰政寬則濟之以猛孔明其知之治直吏翻}
劉備以零陵蔣琬為廣都長|{
	長知兩翻}
備嘗因游觀奄至廣都見琬衆事不治時又沈醉|{
	沈持林翻沈醉言為酒所沉滯也}
備大怒將加罪戮諸葛亮請曰蔣琬社稷之器非百里之才也其為政以安民為本不以修飾為先願主公重加察之|{
	重直用翻言再三加察也}
備雅敬亮乃不加罪倉卒但免官而已 秋七月魏公操擊孫權留少子臨菑侯植守鄴|{
	少詩照翻 考異曰植傳云太祖戒之曰吾昔為頓丘令年二十三思此時所行無悔于今今汝年亦二十三矣又云植太和六年薨年三十一按植今年年二十三則死時當年四十一矣本傳誤也}
操為諸子高選官屬|{
	為于偽翻}
以邢顒為植家丞顒防閑以禮無所屈橈|{
	顒魚容翻防隄也閑闌也防以制水閑以制獸皆禁止之義也橈奴教翻}
由是不合庶子劉楨美文辭植親愛之|{
	漢制列侯置家丞庶子各一人主侍侯使理家事楨音貞}
楨以書諫植曰君侯採庶子之春華忘家丞之秋實為上招謗其罪不小愚實懼焉 魏尚書令荀攸卒攸深密有智防|{
	智以料事防以保身}
自從魏公操攻討常謀謨帷幄時人及子弟莫知其所言操嘗稱荀文若之進善不進不休荀公達之去惡不去不止|{
	去惡之去羌呂翻荀彧字文若荀攸字公達}
又稱二荀令之論人久而益信吾沒世不忘|{
	彧為漢尚書令攸為魏尚書令}
初枹罕宋建因涼州亂自號河首平漢王|{
	枹罕縣前漢屬全城郡後漢屬隴西郡枹音膚賜支河首在金城河關之西建自以居河上流故以為號}
改元置百官三十餘年冬十月魏公操使夏侯淵自興國討建圍枹罕拔之斬建淵别遣張郃等渡河入小湟中|{
	湟水源出西海鹽池之西北東至金城允吾縣入河夾湟兩岸之地通謂之湟中又有湟中城在西平張掖之間小月氐之地也故謂之小湟中}
河西諸羌皆降|{
	降戶江翻}
隴右平 帝自都許以來守位而已左右侍衛莫非曹氏之人者議郎趙彦常為帝陳言時策魏公操惡而殺之|{
	為于偽翻惡烏路翻}
操後以事入見殿中帝不任其懼|{
	見賢遍翻下同任音壬勝也不任猶言不勝也}
因曰君若能相輔則厚不爾幸垂恩相捨操失色俛仰求出舊儀三公領兵朝見令虎賁執刃挾之|{
	以其領兵懼其為變故防之也朝直遙翻下同見賢遍翻}
操出顧左右汗流浹背|{
	浹即協翻}
自後不復朝請|{
	復扶又翻}
董承女為責人操誅承求貴人殺之帝以貴人有|{
	如林翻孕也}
累為請不能得|{
	為于偽翻}
伏皇后由是懷懼乃與父完書言曹操殘逼之狀|{
	賊人者謂之殘逼言其逼上也}
令密圖之完不敢發至是事乃泄|{
	董承誅事見六十三卷五年}
操大怒十一月使御史大夫郗慮持節策收皇后璽綬|{
	郗丑之翻璽斯氏翻綬音受}
以尚書令華歆為副勒兵入宫收后后閉戶藏壁中歆壞戶發壁就牽后出|{
	華子魚有名稱于時與邴原管寧號三人為三龍歆為龍頭原為龍腹寧為龍尾歆所為乃爾邴原亦為操爵所縻高尚其事獨管寧耳當時頭尾之論盖以名位言也嗚呼壞音怪}
時帝在外殿引慮於坐|{
	坐徂卧翻}
后被髪徒跣行泣過訣曰不能復相活邪|{
	被皮義翻復扶又翻}
帝曰我亦不知命在何時顧謂慮曰郗公|{
	漢御史大夫三公也故以呼之}
天下寧有是邪遂將后下暴室以幽死|{
	下遐稼翻}
所生二皇子皆酖殺之兄弟及宗族死者百餘人 十二月魏公操至孟津 操以尚書郎高柔為理曹掾|{
	理曹漢公府無之盖操所置掾俞絹翻}
舊灋軍征士亡考竟其妻子|{
	考覈而窮竟之也}
而亡者猶不息操欲更重其刑并及父母兄弟柔啟曰士卒亡軍誠在可疾|{
	疾惡也書曰爾母忿疾于頑}
然竊聞其中時有悔者愚謂乃宜貸其妻子一可使誘其還心|{
	誘音酉}
正如前科固已絶其意望而猥復重之|{
	復扶又翻下同}
柔恐自今在軍之士見一人亡逃誅將及已亦且相隨而走不可復得殺也此重刑非所以止亡乃所以益走耳操曰善即止不殺

二十年春正月甲子立貴人曹氏為皇后魏公操之女也 三月魏公操自將擊張魯將自武都入氐|{
	武都本白馬氐所居之地武帝開以為郡}
氐人塞道|{
	塞悉則翻}
遣張郃朱靈等攻破之|{
	郃古合翻又曷閤翻}
夏四月操自陳倉出散關至河池|{
	陳倉縣屬右扶風唐岐州寶雞縣是大散關在其西南河池縣屬武都郡余據大散關在今鳳州梁泉縣}
氐王竇茂衆萬餘人恃險不服五月攻屠之西平金城諸將麯演蔣石等共斬送韓遂首|{
	漢末分金城為西平郡}
初劉備在荆州周瑜甘寧等數勸孫權取蜀|{
	數所角翻}
權遣使謂備曰|{
	使疏吏翻}
劉璋不武不能自守若使曹操得蜀則荆州危矣今欲先攻取璋次取張魯一統南方雖有十操無所憂也備報曰益州民富地險劉璋雖弱足以自守今暴師於蜀漢轉運於萬里欲使戰克攻取舉不失利此孫吳所難也|{
	孫吳謂孫武吳起也}
議者見曹操失利於赤壁謂其力屈無復遠念|{
	復扶又翻}
今操三分天下已有其二將欲飲馬於滄海觀兵於吳會|{
	吳會謂吳地為一都會會讀如字一說吳會謂吳會稽二郡之地會音工外翻}
何肯守此坐須老乎而同盟無故自相攻伐借樞於操|{
	樞者門戶所由以運動也言操欲揺動吳蜀而未得其樞若自相攻伐是借之以可動之樞也}
使敵乘其隙非長計也且備與璋託為宗室冀憑威靈以匡漢朝|{
	朝直遙翻}
今璋得罪於左右備獨悚懼非所敢聞願加寛貸權不聽遣孫瑜率水軍住夏口備不聽軍過謂瑜曰汝欲取蜀吾當被髪入山不失信於天下也|{
	言宗室被攻而不能救無面目以立於天下也被皮義翻}
使關羽屯江陵張飛屯秭歸|{
	秭歸縣屬南郡唐之歸州}
諸葛亮據南郡|{
	南郡本治江陵吳得荆州置南郡於江南晉平吳以江陵為南郡以江南之南郡為南平郡亮所據盖江南之南郡也}
備自住孱陵|{
	孱應劭音踐師古士連翻}
權不得已召瑜還及備西攻劉璋權曰猾虜乃敢挾詐如此備留關羽守江陵魯肅與羽鄰界羽數生疑貳肅常以歡好撫之|{
	數所角翻好呼到翻}
及備巳得益州權令中司馬諸葛瑾從備求荆州諸郡|{
	時權署置諸將有别部司馬則中司馬者蓋中軍司馬也瑾自長史轉中司馬位任蓋不輕矣瑾渠吝翻}
備不許曰吾方圖凉州凉州定乃盡以荆州相與耳權曰此假而不反乃欲以虚辭引歲也|{
	謂延引歲時也孟子曰久假而不歸焉知其非有也}
遂置長沙零陵桂陽三郡長吏|{
	長知兩翻}
關羽盡逐之權大怒遣呂蒙督兵二萬以取三郡蒙移書長沙桂陽皆望風歸服惟零陵太守郝普城守不降|{
	降戶江翻下同}
劉備聞之自蜀親至公安遣關羽爭三郡孫權進住陸口為諸軍節度使魯肅將萬人屯益陽以拒羽|{
	益陽縣屬長沙郡應劭曰在益水之陽輿地志今潭州安化縣本漢益陽縣杜佑曰潭州益陽縣漢故城在今縣東宋白曰益陽故城在今益陽縣東八十里其城魯肅所築}
飛書召呂蒙使捨零陵急還助肅蒙得書祕之夜召諸將授以方畧晨當攻零陵顧謂郝普故人南陽鄧玄之曰郝子太聞世間有忠義事亦欲為之而不知時也|{
	郝普字子太郝呼各翻}
今左將軍在漢中為夏侯淵所圍關羽在南郡至尊身自臨之彼方首尾倒縣|{
	縣讀曰懸}
救死不給豈有餘力復營此哉今吾計力度慮而以攻此|{
	復扶又翻度徒洛翻下同}
曾不移日而城必破城破之後身死何益於事而令百歲老母戴白受誅豈不痛哉度此家不得外問|{
	此家謂郝普也}
謂援可恃故至於此耳君可見之為陳禍福|{
	為于偽翻}
玄之見普具宣蒙意普懼而出降蒙迎執其手與俱下船語畢出書示之因拊手大笑普見書知備在公安而羽在益陽慚恨入地蒙留孫河委以後事 |{
	考異曰按孫河巳死或他人同姓名耳}
即日引軍赴益陽魯肅欲與關羽會語諸將疑恐有變議不可往肅曰今日之事宜相開譬劉備負國是非未决羽亦何敢重欲干命乃邀羽相見各駐兵馬百步上但諸將軍單刀俱會肅因責數羽以不返三郡|{
	數所具翻}
羽曰烏林之役左將軍身在行間戮力破敵|{
	即謂赤壁之戰也行戶剛翻}
豈得徒勞無一塊土而足下來欲收地邪|{
	塊苦潰翻}
肅曰不然始與豫州覲於長阪|{
	事並見六十五卷十三年}
豫州之衆不當一校|{
	校戶教翻}
計窮慮極志埶摧弱圖欲遠竄|{
	謂欲投吳巨也}
望不及此主上矜愍豫州之身無有處所不愛土地士民之力使有所庇䕃以濟其患而豫州私獨飾情愆德墮好|{
	私獨謂私其一已之所獨也墮讀曰隳好呼到翻下同}
今已藉手於西州矣|{
	謂得益州有以藉手也}
又欲翦并荆州之土斯蓋凡夫所不忍行而况整領人物之主乎羽無以答會聞魏公操將攻漢中 |{
	考異曰備傳云曹公定漢中孫權傳云入漢中按操以七月入漢中備未應即聞之而八月權已攻合肥盖聞曹公兵始欲向漢中即引兵還耳}
劉備懼失益州使使求和於權權令諸葛瑾報命更尋盟好遂分荆州以湘水為界長沙江夏桂陽以東屬權南郡零陵武陵以西屬備|{
	班志湘水出零陵陽海山至酃入江過郡二行二千五百三十里吳蜀分荆州長沙桂陽零陵武陵以湘水為界耳南郡江夏各自依其郡界夏戶雅翻}
諸葛瑾每奉使至蜀|{
	使疏吏翻}
與其弟亮但公會相見退無私面 秋七月魏公操至陽平|{
	水經註濜水發武都氐中南逕張魯城東城因崤嶺周迴五里東臨峻谷杳然百尋西北二面連峯接崖莫究其極從南為盤道登陟二里有餘庾仲雍謂山為白馬塞東對白馬城一名陽平關濜水南流入沔謂之濜口或曰陽平關即今興元百牢關是也杜佑曰陽平關在漢中褒城縣西北}
張魯欲舉漢中降|{
	降戶江翻下同}
其弟衛不肯率衆數萬人拒關堅守横山築城十餘里初操承凉州從事及武都降人之辭說張魯易攻|{
	易以䜴翻}
陽平城下南北山相遠|{
	遠于願翻}
不可守也信以為然及往臨履不如所聞乃歎曰他人商度|{
	度徒洛翻}
少如人意|{
	少詩沼翻}
攻陽平山上諸屯山峻難登既不時拔士卒傷夷者多軍食且盡操意沮便欲拔軍截山而還|{
	沮在呂翻截山者防其追尾也還從宣翻又如字下同}
遣大將軍夏侯惇將軍許褚呼山上兵還會前軍夜迷惑誤入張衛别營營中大驚退散侍中辛毗主簿劉曄等在兵後語惇褚|{
	語牛倨翻}
言官兵已據得賊要屯賊已散走猶不信之惇前自見乃還白操進兵攻衛衛等夜遁 |{
	考異曰武帝紀曰公至陽平張魯使弟衛等據關攻之不拔乃引還賊守備解散公乃密遣解檦等乘險夜襲大破之劉曄傳曰太祖欲還令曄督後諸軍曄策魯可克馳白太祖不如致攻遂進兵魯乃奔走郭頒世語魯遣五官掾降弟衛拒王師不得進魯走巴中軍糧盡太祖將還西曹掾郭諶曰魯已降留使既未反衛雖不同偏攜可攻縣軍深入以進必克退必不免太祖疑之夜有野麋數千突壞衛營軍大驚高祚等誤與衛衆遇衛以為大軍見掩遂降魏名臣奏載楊暨表曰武皇帝征張魯以十萬之衆身親臨履張衛之守蓋不足言地險守易雖有精兵虎將勢不能施對兵三日欲抽軍還天祚大魏魯守自壞因以定之又載董昭表其承凉州以下皆昭表所述必得實今從之}
張魯聞陽平已陷欲降閻圃曰今以迫往功必輕不如依杜濩赴朴胡|{
	杜濩賨邑侯也朴胡巴七姓夷王也余據板楯蠻渠帥有羅朴督鄂度夕龔七姓不輸租賦此所謂七姓夷王也其餘戶歲入賨錢口四十故有賨侯孫盛曰朴音浮濩音戶}
與相拒然後委質|{
	質如字}
功必多乃犇南山入巴中|{
	今興元府古漢中之地也興元之南有大行路通於巴州其路險峻三日而達于山頂其絶高處謂之孤雲兩角去天一握孤雲兩角二山名也今巴州漢巴郡宕渠縣之北界也三巴之地此居其中謂之中巴巴之北境冇米倉山下視興元實孔道也}
左右欲悉燒寶貨倉庫魯曰本欲歸命國家而意未得達今之走避銳鋒非有惡意寶貨倉庫國家之有遂封藏而去操入南鄭|{
	南鄭縣漢中郡治所}
甚嘉之又以魯本有善意遣人慰喻之丞相主簿司馬懿言於操曰劉備以詐力虜劉璋蜀人未附而遠爭江陵此機不可失也今克漢中益州震動進兵臨之勢必瓦解聖人不能違時亦不可失時也操曰人苦無足既得隴復望蜀邪|{
	光武詔岑彭等曰人苦不知足既得隴復望蜀}
劉曄曰劉備人傑也有度而遲得蜀日淺蜀人未恃也今破漢中蜀人震恐其勢自傾以公之神明因其傾而壓之無不克也若少緩之諸葛亮明於治國而為相關羽張飛勇冠三軍而為將|{
	少詩沼翻治直之翻相息亮翻冠古玩翻將即亮翻下同}
蜀民既定據險守要則不可犯矣今不取必為後憂操不從居七日蜀降者說蜀中一日數十驚守將雖斬之而不能安也 |{
	考異曰劉曄傳云備雖斬之按備傳云備下公安聞曹公定漢中乃還如此則備時猶在公安也}
操問曄曰今尚可擊不|{
	不讀曰否}
曰今已小定未可擊也|{
	七日之間何以遽謂之小定曄蓋窺覘備之守蜀有不可犯者故為此言以對操焉耳}
乃還以夏侯淵為都護將軍|{
	都護將軍以盡護諸將而立號光武始以命賈復}
督張郃徐晃等守漢中以丞相長史杜襲為駙馬都尉留督漢中事襲綏懷開導百姓自樂出徙洛鄴者八萬餘口|{
	樂音洛}
八月孫權率衆十萬圍合肥時張遼李典樂進將七千餘人屯合肥魏公操之征張魯也為教與合肥護軍薛悌署函邊曰賊至乃發及權至發教教曰若孫權至者張李將軍出戰樂將軍守護軍勿得與戰|{
	操以遼典勇銳使之戰樂進持重使之守薛悌文吏也使勿得與戰}
諸將以衆寡不敵疑之張遼曰公遠征在外比救至彼破我必矣|{
	比必寐翻}
是以教指及其未合逆擊之折其盛勢以安衆心然後可守也進等莫對遼怒曰成敗之機在此一戰諸君若疑遼將獨决之|{
	欲獨出戰也}
李典素與遼不睦慨然曰此國家大事顧君計何如耳吾可以私憾而忘公義乎請從君而出於是遼夜募敢從之士得八百人椎牛犒饗|{
	犒苦到翻}
明旦遼被甲持戟先登陷陳殺數十人斬二大將大呼自名|{
	陳讀曰陣呼火故翻}
衝壘入至權麾下權大驚不知所為走登高冢以長戟自守遼叱權下戟權不敢動望見遼所將衆少乃聚圍遼數重|{
	少詩沼翻重直龍翻}
遼急擊圍開將麾下數十人得出餘衆號呼曰將軍棄我乎|{
	號戶高翻}
遼復前突圍拔出餘衆|{
	復扶又翻下同}
權人馬皆披靡無敢當者|{
	披普靡翻}
自旦戰至日中吳人奪氣乃還修守備衆心遂安權守合肥十餘日城不可拔徹軍還兵皆就路權與諸將在逍遙津北|{
	水經注合肥東有逍遙津水上舊有梁}
張遼覘望知之|{
	覘丑亷翻又丑艷翻}
即將步騎奄至甘寧與呂蒙等力戰扞敵凌統率親近扶權出圍復還與遼戰左右盡死身亦被創度權已免乃還|{
	被皮義翻創初良翻度徒洛翻}
權乘駿馬上津橋|{
	上時掌翻}
橋南已徹丈餘無版親近監谷利在馬後|{
	親近監官也谷姓也利名也江表傳曰谷利者本左右給使也以謹直為親近監}
使權持鞍緩控|{
	控即馬鞚}
利於後著鞭以助馬勢|{
	著陟畧翻}
遂得超度賀齊率三千人在津南迎權權由是得免權入大船宴飲賀齊下席涕泣曰至尊人主常當持重今日之事幾致禍敗羣下震怖|{
	幾居希翻怖普布翻}
若無天地願以此為終身之誡權自前收其淚曰大慙|{
	權慙謝賀齊也}
謹已刻心非但書紳也|{
	論語子張問於孔子以孔子之言書諸紳故以答賀齊}
九月巴賨夷帥朴胡杜濩任約各舉其衆來附|{
	賨臧宗翻帥所類翻}
於是分巴郡以胡為巴東太守濩為巴西太守約為巴郡太守皆封列侯|{
	後三人皆為劉備所破}
冬十月始置名號侯以賞軍功|{
	魏書曰置名號爵十八級關中侯爵十七}


|{
	級皆金印紫綬又置關内外侯十六級銅印龜鈕墨綬皆不食租裴松之曰今之虛封蓋自此始}
十一月張魯將家屬出降|{
	降戶江翻下同}
魏公操逆拜魯鎮南將軍待以客禮封閬中侯|{
	賢曰閬中縣屬巴郡今隆州余據隆州後避唐玄宗諱改為閬州杜佑曰閬中今閬州城閬音浪}
邑萬戶封魯五子及閻圃等皆為列侯

習鑿齒論曰閻圃諫魯勿王|{
	事見六十四卷建安六年}
而曹公追封之將來之人孰不思順塞其本源而末流自止|{
	塞悉則翻}
其此之謂歟若乃不明於此而重焦爛之功|{
	此引前書徐福焦頭爛額事見二十五卷漢宣帝地節四年}
豐爵厚賞止於死戰之士則民利於有亂俗競於殺伐阻兵杖力|{
	杖除兩翻}
干戈不戢矣曹公之此封可謂知賞罰之本矣

程銀侯選龎悳皆隨魯降|{
	程銀侯選關中部帥也龎惪馬超將也渭南冀城之敗皆奔張魯惪古德字}
魏公操復銀選官爵拜惪立義將軍 張魯之走巴中也黄權言於劉備曰若失漢中則三巴不振此為割蜀之股臂也|{
	三巴巴東巴西巴郡也}
備乃以權為護軍率諸將迎魯魯已降權遂擊朴胡杜濩任約破之魏公操使張郃督諸軍狥三巴欲徙其民於漢中進軍宕渠|{
	宕渠縣本屬巴郡時屬巴西郡賢曰宕渠故城在今渠州流江縣東北杜佑曰俗號車騎城是也宋白曰宕渠城漢車騎將軍馮緄增修俗名車騎城師古曰宕音徒浪翻}
劉備使巴西太守張飛與郃相拒五十餘日飛襲擊郃大破之郃走還南鄭備亦還成都操徙出故韓遂馬超等兵五千餘人使平難將軍殷署等督領|{
	平難將軍曹氏所置難乃旦翻}
以扶風太守趙儼為關中護軍操使儼發千二百兵助漢中守禦殷署督送之行者不樂|{
	樂音洛}
儼護送至斜谷口|{
	斜余遮翻谷音浴又古祿翻}
還未至營署軍叛亂儼自隨步騎百五十人皆叛者親黨也聞之各驚被甲持兵|{
	被皮義翻}
不復自安|{
	復扶又翻}
儼徐喻以成敗慰勵懇切皆慷慨曰死生當隨護軍不敢有二前到諸營各召料簡諸姦結叛者|{
	料音聊量度也理也}
八百餘人散在原野儼下令惟取其造謀魁率治之|{
	率讀曰帥所類類治直之翻}
餘一不問郡縣所收送皆放遣乃即相率還降儼密白宜遣將詣大營|{
	大營謂操營也將讀如字送也}
請舊兵鎮守關中魏公操遣將軍劉柱將二千人往當須到乃發遣俄而事露諸營大駭不可安諭|{
	不可以言語諭之使安帖也}
儼遂宣言當差留新兵之溫厚者千人鎮守關中|{
	差初皆翻擇也}
其餘悉遣東|{
	遣之東赴操營}
便見主者内諸營兵名籍立差别之|{
	主者主兵籍者也差初皆翻擇也又初加翻言以等差别異之也别彼列翻分也異也}
留者意定與儼同心其當去者亦不敢動儼一日盡遣上道|{
	上時掌翻}
因使所留千人分布羅落之|{
	分布于行者之間羅列而遮落之也}
東兵尋至|{
	東兵劉柱所將之兵也}
乃復脅諭|{
	復扶又翻}
并徙千人令相及共東凡所全致二萬餘口

二十一年春二月魏公操還鄴 夏五月進魏公操爵為王初中尉崔琰薦鉅鹿楊訓於操|{
	中尉秦官漢因之至武帝改為執金吾今操復置中尉實則漢執金吾之職也}
操禮辟之及操進爵訓發表稱頌功德或笑訓希世浮偽謂琰為失所舉琰從訓取表草視之與訓書曰省表事佳耳|{
	省悉景翻}
時乎時乎會當有變時琰本意譏論者好譴呵而不尋情理也|{
	好呼到翻}
時有與琰宿不平者白琰傲世怨謗意旨不遜|{
	以會當有變為意旨不遜}
操怒收琰付獄髠為徒隸前白琰者復白之云琰為徒對賓客虬須直視|{
	虬須卷鬚也直視者目不他矚也復扶又翻下同}
若有所瞋|{
	瞋昌真翻怒目也}
遂賜琰死尚書僕射毛玠傷琰無辜心不悦人復白玠怨謗操收玠付獄侍中桓階和洽皆為之陳理|{
	為于偽翻}
操不聽階求案實其事王曰言事者白玠不但謗吾也乃復為崔琰觖望|{
	觖有二音音窺瑞翻者望也言有所覬望也音古穴翻者怨望也此當從入聲}
此捐君臣恩義妄為死友怨歎|{
	死友言其背公而相為死也為于偽翻}
殆不可忍也洽曰如言事者言玠罪過深重非天地所覆載|{
	覆敷又翻}
臣非敢曲理玠以枉大倫也|{
	孟子曰内則父子外則君臣人之大倫也}
以玠歷年荷寵|{
	荷下可翻}
剛直忠公為衆所憚不宜有此然人情難保要宜考覈兩驗其實今聖恩不忍致之于理更使曲直之分不明|{
	分扶問翻}
操曰所以不考欲兩全玠及言事者耳洽對曰玠信有謗主之言當肆之市朝|{
	論語子服景伯曰吾力猶能肆諸市朝應劭曰大夫以上尸諸朝士以下尸諸市朝直遙翻}
若玠無此言言事者加誣大臣以誤主聽不加檢覈臣竊不安操卒不窮治|{
	卒子恤翻治直之翻下同}
玠遂免黜終於家是時西曹掾沛國丁儀用事玠之獲罪儀有力焉羣下畏之側目尚書僕射何夔及東曹屬東莞徐奕|{
	東莞縣屬琅琊國春秋之鄆邑也晉置東莞郡唐密州莒縣即其地也莞姑丸翻}
獨不事儀儀譖奕出為魏郡太守|{
	操既居鄴建安十七年割河内之蕩隂朝歌林慮東郡之衛國頓丘東武陽發于鉅鹿之癭陶曲陽南和廣平之廣平任趙國之襄國邯鄲易陽以益魏郡十八年分置東西都尉此以自相府掾屬補郡為出}
賴桓階左右之得免|{
	左右讀曰佐佑}
尚書傅選謂何夔曰儀已害毛玠子宜少下之夔曰為不義適足害其身焉能害人|{
	少詩沼翻下遐稼翻焉于䖍翻}
且懷姦佞之心立於明朝其得久乎|{
	為丁儀被誅張本朝直遙翻}
崔琰從弟林|{
	從才用翻}
嘗與陳羣共論冀州人士稱琰為首羣以智不存身貶之林曰大丈夫為有邂逅耳|{
	邂下懈翻逅戶茂翻}
即如卿諸人良足貴乎 五月己亥朔日有食之 代郡烏桓三大人皆稱單于|{
	代郡烏桓單于其一曰普盧其二曰無臣氏其三則未之聞也}
恃力驕恣太守不能治魏王操以丞相倉曹屬裴潛為太守|{
	漢公府有倉曹有掾有屬主倉穀事}
欲授以精兵潛曰單于自知放横日久|{
	横戶孟翻}
今多將兵往必懼而拒境少將則不見憚宜以計謀圖之遂單車之郡單于驚喜潛撫以恩威單于讋服|{
	讋質涉翻}
初南匈奴久居塞内|{
	南匈奴自光武建武二十六年即入居塞内}
與編戶大同而不輸貢賦議者恐其戶口滋蔓浸難禁制宜豫為之防秋七月南單于呼㕑泉入朝于魏|{
	朝直遙翻}
魏王操因留之於鄴使右賢王去卑監其國|{
	監古銜翻下同}
單于歲給綿絹錢穀如列侯子孫傳襲其號分其衆為五部各立其貴人為帥|{
	分為左右前後中五部分居并州諸郡而監國者居平陽帥所類翻}
選漢人為司馬以監督之八月魏以大理鍾繇為相國 冬十月魏王操治兵

擊孫權十一月至譙

資治通鑑卷六十七
