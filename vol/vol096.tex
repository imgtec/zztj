\chapter{資治通鑑卷九十六}
宋 司馬光 撰

胡三省 音註

晉紀十八|{
	起著雍閹茂盡重光赤奮若凡四年}


顯宗成皇帝中之下

咸康四年春正月燕王皝遣都尉趙槃如趙聽師期|{
	皝呼廣反}
趙王虎將擊段遼募驍勇者三萬人|{
	驍堅堯翻}
悉拜龍騰中郎|{
	據載記咸康二年虎改直盪為龍騰冠以絳幘}
會遼遣段屈雲襲趙幽州幽州刺史李孟退保易京虎乃以桃豹為横海將軍|{
	横海將軍蓋石氏創置}
王華為渡遼將軍帥舟師十萬出漂榆津|{
	水經曰清河東北過漂榆邑入于海注云漂榆故城俗謂之角飛城趙記云石勒使王述煮鹽于角飛魏土地記曰勃海郡高城縣東北一百里北盡漂榆東臨巨海民咸煮鹽為業帥讀曰率}
支雄為龍驤大將軍姚弋仲為冠軍將軍|{
	驤思將翻}
帥步騎七萬為前鋒以伐遼|{
	冠古玩翻帥讀曰率騎奇寄翻}
三月趙槃還至棘城燕王皝引兵攻掠令支以北諸城|{
	令音鈴師古郎定翻支音祁}
段遼將追之慕容翰曰今趙兵在南當并力禦之而更與燕鬭燕王自將而來|{
	將即亮翻下悉將同}
其士卒精鋭若萬一失利將何以禦南敵乎段蘭怒曰吾前為卿所誤|{
	事見上卷咸和八年}
以成今日之患吾不復墮卿計中矣乃悉將見衆追之|{
	復扶又翻下同見賢遍翻}
皝設伏以待之大破蘭兵斬首數千級掠五千戶及畜產萬計以歸趙王虎進屯金臺|{
	按水經注金臺在涿郡故安縣有金臺陂臺在陂北十餘步即燕昭王築以事郭隗之臺}
支雄長驅入薊|{
	薊音計}
段遼所署漁陽上谷代郡守相皆降取四十餘城北平相陽裕帥其民數千家登燕山以自固|{
	五代志北平無終縣有燕山守手又翻相息亮翻燕於賢翻}
諸將恐其為後患欲攻之虎曰裕儒生矜惜名節恥於迎降耳|{
	降戶江翻下同}
無能為也遂過之至徐無|{
	徐無縣屬北平郡其地在唐薊州玉田縣界}
段遼以其弟蘭既敗不敢復戰帥妻子宗族豪大千餘家|{
	豪大猶言豪帥也是時東北夷率謂主帥為大部帥曰部大城主曰城大是也}
棄令支奔密雲山|{
	水經注密雲戍在禦夷鎮東南九十里鮑邱水逕其西唐檀州治密雲縣西南去范陽二百里又據晉紀云遼奔于平岡蓋密雲山在漢平岡縣界宋白曰檀州密雲縣本漢虒奚縣西南至幽州百九十里西至媯州二百五十里東北至長城障塞百一十里東南至薊州百九十里}
將行執慕容翰手泣曰不用卿言自取敗亡我固甘心令卿失所深以為愧翰北奔宇文氏遼左右長史劉羣盧諶崔悦等封府庫請降|{
	羣諶悦奔令支見九十卷元帝大興元年}
虎遣將軍郭太麻秋帥輕騎二萬追遼至密雲山獲其母妻斬首三千級遼單騎走險|{
	赴險以自保走音奏}
遣其子乞特真奉表及獻名馬於趙虎受之虎入令支宫|{
	段氏都令支以其所居為宫}
論功封賞各有差徙段國民二萬餘戶於司雍兖豫四州|{
	雍於用翻}
士大夫之有才行皆擢敘之|{
	行下孟翻}
陽裕詣軍門降虎讓之曰卿昔為奴虜走今為士人來豈識知天命將逃匿無地耶對曰臣昔事王公不能匡濟|{
	王公謂王浚也裕奔令支見八十九卷愍帝建興二年}
逃于段氏復不能全今陛下天網高張籠絡四海幽冀豪傑莫不風從如臣比肩無所獨愧生死之命惟陛下制之虎悅即拜北平太守 夏四月癸丑以慕容皝為征北大將軍幽州牧領平州刺史 成主期驕虐日甚多所誅殺而籍沒其資財婦女由是大臣多不自安漢王壽素貴重有威名期及建寧王越等皆忌之壽懼不免每當入朝常詐為邊書辭以警急|{
	壽時鎮涪城朝直遙翻}
初巴西處士龔壯父叔皆為李特所殺|{
	父及叔父也處昌呂翻}
壯欲報仇積年不除喪壽數以禮辟之|{
	數所角翻下同}
壯不應而往見壽壽密問壯以自安之策壯曰巴蜀之民本皆晉臣節下若能發兵西取成都稱藩於晉誰不爭為節下奮臂前驅者|{
	魏晉以來持節假節出當方面者人皆稱之為節下為于偽翻}
如此則福流子孫名垂不朽豈徒脫今日之禍而已壽然之陰與長史畧陽羅恒巴西解思明謀攻成都期頗聞之數遣許涪至壽所伺其動靜|{
	涪音浮伺相吏翻}
又鴆殺壽養弟安北將軍攸壽乃詐為妹夫任調書云期當取壽|{
	詐言期欲取壽以怒其衆任音壬下同}
其衆信之遂帥步騎萬餘人自涪襲成都|{
	帥讀曰率涪音浮}
許賞以城中財物以其將李奕為前鋒|{
	將即亮翻}
期不意其至初不設備壽世子勢為翊軍校尉開門納之遂克成都屯兵宫門期遣侍中勞壽|{
	勞力到翻}
壽奏建寧王越景騫田褒姚華許涪及征西將軍李遐將軍李西等懷奸亂政皆收殺之縱兵大掠數日乃定壽矯以太后任氏令廢期為卭都縣公幽之别宫|{
	卭都縣屬越嶲郡卭渠容翻}
追諡戾太子曰哀皇帝|{
	咸和九年期越弑其主班諡曰戾太子}
羅恒解思明李奕等勸壽稱鎮西將軍益州牧成都王稱藩于晉|{
	解戶買翻}
送卭都公於建康任調及司馬蔡興侍中李豔等勸壽自稱帝壽命筮之|{
	龜為卜蓍為筮}
占者曰可數年天子調喜曰一日尚足況數年乎思明曰數年天子孰與百世諸侯壽曰朝聞道夕死可矣|{
	引論語孔子之言}
遂即皇帝位|{
	壽字武考驤之子也}
改國號曰漢大赦改元漢興以安車束帛徵龔壯為太師壯誓不仕壽所贈遺一無所受|{
	遺于季翻}
壽改立宗廟追尊父驤曰獻皇帝|{
	驤思將翻}
母昝氏曰皇太后|{
	昝子感翻姓也}
立妃閻氏為皇后世子勢為皇太子更以舊廟為大成廟|{
	舊廟祀李特李雄者也雄建國號曰成壽改曰漢故以特雄廟為大成廟}
凡諸制度多所更易|{
	更工衡翻}
以董皎為相國羅恒為尚書令解思明為廣漢太守任調為鎮北將軍梁州刺史李奕為西夷校尉從子權為寧州刺史|{
	從才用翻}
公卿州郡悉用其僚佐代之成氏舊臣近親及六郡士人皆見疎斥|{
	六郡士人與李特兄弟同入蜀者}
卭都公期歎曰天下主乃為小縣公不如死五月縊而卒|{
	載記期死於三年年二十五縊於賜翻又於計翻}
壽諡曰幽公葬以王禮 趙王虎以燕王皝不會趙兵攻段遼而自專其利|{
	以皝掠段氏人民畜產不待趙師至而北歸也}
欲伐之太史令趙攬諫曰歲星守燕分師必無功|{
	天文志歲星贏縮以其舍命國其所居久其國有德厚五穀豐昌不可伐也分扶問翻}
虎怒鞭之皝聞之嚴兵設備罷六卿納言常伯冗騎常侍官|{
	去年皝置六卿等官冗而隴翻}
趙戎卒數十萬燕人震恐皝謂内史高詡曰將若之何|{
	内史燕國内史也}
對曰趙兵雖彊然不足憂但堅守以拒之無能為也虎遣使四出招誘民夷|{
	誘音酉}
燕成周内史崔燾居就令游泓武原令常霸東夷校尉封抽護軍宋晃等皆應之凡得三十六城泓邃之兄子也冀陽流寓之士共殺太守宋燭以降於趙燭晃之從兄也營邱内史鮮于屈亦遣使降趙武寧令廣平孫興曉諭吏民共收屈數其罪而殺之閉城拒守|{
	成周冀陽營邱郡皆慕容廆所置見八十九卷愍帝建興二年居就縣漢晉屬遼東郡武原蓋亦慕容氏所置縣也武寧縣亦慕容氏所置帶營邱郡游邃見八十八卷愍帝建興元年}
朝鮮令昌黎孫泳帥衆拒趙|{
	帥讀曰率}
大姓王請等密謀應趙泳收斬之同謀數百人惶怖請罪|{
	怖普布翻}
泳皆釋之與同拒守樂浪太守鞠彭以境内皆叛選鄉里壯士二百餘人共還棘城|{
	樂浪非漢古郡地也慕容廆所置見八十八卷愍帝建興元年以五代志攷之樂浪冀陽營邱郡朝鮮武寧等縣當盡在隋遼西郡柳城縣界鞠彭率鄉人歸燕見九十一卷元帝太興二年樂浪音洛琅}
戊子趙兵進逼棘趙燕王皝欲出亡帳下將慕輿根諫曰|{
	將即亮翻}
趙彊我弱大王一舉足則趙之氣勢遂成使趙人收畧國民|{
	國民謂燕國之民也}
兵彊穀足不可復敵|{
	復扶又翻}
竊意趙人正欲大王如此耳奈何入其計中乎今固守堅城其勢百倍縱其急攻猶足枝持觀形察變閒出求利|{
	謂伺間出擊趙以求利也閒古莧翻}
如事之不濟不失於走奈何望風委去為必亡之理乎皝乃止然猶懼形於色玄菟太守河間劉佩曰今彊寇在外衆心恟懼|{
	菟同都翻守式又翻恟許拱翻}
事之安危繫於一人大王此際無所推委|{
	推吐雷翻言難推此責以委人也}
當自彊以厲將士不宜示弱事急矣臣請出擊之縱無大捷足以安衆乃將敢死數百騎出衝趙兵所向披靡|{
	披普彼翻開也分也散也靡偃也}
斬獲而還|{
	還從宣翻又如字}
於是士氣自倍皝問計於封奕對曰石虎凶虐已甚民神共疾禍敗之至其何日之有|{
	杜預曰言今至}
今空國遠來攻守勢異戎馬雖彊無能為患頓兵積日釁隙自生但堅守以俟之耳皝意乃安或說皝降皝曰孤方取天下何謂降也|{
	說輸芮翻降戶江翻}
趙兵四面蟻附緣城|{
	言肉薄附城而上若羣蟻然}
慕輿根等晝夜力戰凡十餘日趙兵不能克壬辰引退皝遣其子恪帥二千騎追擊之|{
	帥讀曰率}
趙兵大敗斬獲三萬餘級趙諸軍皆弃甲逃潰惟游擊將軍石閔一軍獨全閔父瞻内黄人|{
	内黄縣屬魏郡以陳留有外黄故加内}
本姓冉趙主勒破陳午獲之命虎養以為子閔驍勇善戰多策畧虎愛之比於諸孫|{
	冉閔始此石勒養石虎以自滅其種石虎養冉閔併其種類而夷之蓋天道也驍堅堯翻}
虎還鄴以劉羣為中書令盧諶為中書侍郎|{
	諶是壬翻}
蒲洪以功拜使持節都督六夷諸軍事冠軍大將軍封西平郡公|{
	使疏吏翻冠古玩翻}
石閔言於虎曰蒲洪雄雋得將士死力諸子皆有非常之才且握彊兵五禺屯據近畿|{
	近畿謂洪屯枋頭距鄴為近}
宜密除之以安社稷虎曰吾方倚其父子以取吳蜀奈何殺之待之愈厚|{
	石虎之不能殺蒲洪猶苻堅之不能殺慕容垂姚萇也}
燕王皝分兵討諸叛城皆下之拓境至凡城|{
	水經注自盧龍東越青陘至凡城二百許里自凡城東北出趣平剛故城可百八十里向黄龍城則五百里}
崔燾常霸奔鄴封抽宋晃游泓奔高句麗皝賞鞠彭慕輿根等而治諸叛者誅滅甚衆|{
	治直之翻}
功曹劉翔為之申理多所全活|{
	為于偽翻}
趙之攻棘城也燕右司馬李洪之弟普以為棘城必敗勸洪出避禍洪曰天道幽遠人事難知且當委任勿輕動取悔普固請不已洪曰卿意見明審者當自行之吾受慕容氏大恩義無去就當效死於此耳與普流涕而訣|{
	訣别也}
普遂降趙|{
	降戶江翻}
從趙軍南歸死於喪亂|{
	喪息浪翻}
洪由是以忠篤著名趙王虎遣渡遼將軍曹伏將青州之衆戍海島|{
	據載記虎遣伏渡海戍蹋頓城無水而還因戍于海島}
運穀三百萬斛以給之又以船三百艘運穀三十萬斛詣高句麗|{
	句如字又音鉤麗力知翻}
使典農中郎將王典帥衆萬餘屯田海濱又令青州造船千艘以謀擊燕|{
	石虎忿棘城之敗再謀擊燕而卒不能也艘蘇遭翻}
趙太子宣帥步騎二萬擊朔方鮮卑斛摩頭破之斬首四萬餘級|{
	帥讀曰率騎奇寄翻}
冀州八郡大蝗趙司隸請坐守宰|{
	趙都鄴以冀州為司部}
趙王虎曰此朕失政所致而欲委咎守宰豈罪已之意邪司隸不進讜言佐朕不逮而欲妄陷無辜可白衣領職|{
	黜其品秩同於臣庶而仍領司隸之職讜音黨}
虎使襄城公涉歸上庸公日歸帥衆戍長安|{
	二歸亦石氏之族}
二歸告鎮西將軍石廣私樹恩澤潛謀不軌虎追廣至鄴殺之 乙未以司徒導為太傅都督中外諸軍事郗鍳為太尉|{
	郗丑之翻}
庾亮為司空六月以導為丞相罷司徒官以并丞相府|{
	東漢司徒即丞相之職也沈約曰丞奉也相助也時以王導為丞相罷司徒府并丞相府導薨罷丞相復為司徒府宋世祖初以南郡王義宣為丞相而司徒府如故}
導性寬厚委任諸將趙胤賈寧等多不奉法大臣患之庾亮與郗鑒牋曰主上自八九歲以及成人入則在宫人之手出則唯武官小人讀書無從受音句顧問未嘗遇君子秦政欲愚其黔首|{
	秦始皇名政命民曰黔首焚詩書以愚黔首}
天下猶知不可況欲愚其主哉人主春秋既盛宜復子明辟不稽首歸政|{
	稽音啓}
甫居師傅之尊|{
	甫方也始也}
多養無賴之士公與下官並荷託付之重|{
	言受遺先帝付以幼孤而託之也荷下可翻}
大姧不掃何以見先帝於地下乎欲共起兵廢導鑒不聽南蠻校尉陶稱|{
	南蠻校尉武帝初置於襄陽後治江陵}
侃之子也以亮謀語導|{
	語牛倨翻}
或勸導密為之備導曰吾與元規休戚是同悠悠之談宜絶智者之口|{
	言智者之口不宜亦傳道悠悠之談}
即如君言元規若來吾便角巾還第復何懼哉|{
	復扶又翻}
又與稱書以為庾公帝之元舅宜善事之|{
	此導之識量所以為弘遠也}
征西參軍孫盛密諫亮曰王公常有世外之懷|{
	言導心常欲謝事優游於人世之外}
豈肯為凡人事邪此必佞邪之徒欲閒内外耳亮乃止|{
	庾亮之謀微郗鑒拒之於外孫盛諫止於内必再亂天下矣閒古莧翻}
盛楚之孫也|{
	孫楚晉初名士}
是時亮雖居外鎮而遥執朝廷之權既據上流擁彊兵趨勢者多歸之|{
	趨七喻翻}
導内不能平常遇西風塵起|{
	常當作嘗}
舉扇自蔽徐曰元規塵汚人|{
	汚烏故翻史言導不平之心不能自禁於言語之間者惟此而已}
導以江夏李充為丞相掾|{
	夏戶雅翻掾俞絹翻}
充以時俗崇尚浮虚乃著學箴以為老子云絶仁弃義民復孝慈豈仁義之道絶然後孝慈乃生哉蓋患乎懷仁義者寡而利仁義者衆將寄責於聖人而遺累乎陳迹也|{
	累力瑞翻}
凡人見形者衆及道者鮮|{
	鮮息淺翻}
逐迹逾篤離本逾遠|{
	離力智翻}
故作學箴以祛其蔽|{
	祛邱于翻攘却也}
曰名之攸彰道之攸廢乃損所隆乃崇所替非仁無以長物|{
	長丁丈翻今知兩翻}
非義無以齊恥仁義固不可遠去其害仁義者而已|{
	遠于願翻去羌呂翻}
漢李奕從兄廣漢太守乾告大臣謀廢立|{
	從才用翻}
秋七月漢主壽使其子廣與大臣盟于前殿徙乾為漢嘉太守以李閎為荆州刺史鎮巴郡閎恭之子也|{
	恭李攀之弟見八十四卷惠帝永寧元年}
八月蜀中久雨百姓飢疫壽命羣臣極言得失龔壯上封事稱陛下起兵之初上指星辰昭告天地歃血盟衆舉國稱藩|{
	謂將稱藩于晉也歃色洽翻}
天應人悦大功克集而論者未諭權宜稱制|{
	謂壽即皇帝位也}
今淫雨百日飢疫並臻天其或者將以監示陛下故也|{
	監古陷翻}
愚謂宜遵前盟推奉建康彼必不愛高爵重位以報大功雖降階一等|{
	王降皇帝一等}
而子孫無窮永保福祚不亦休哉論者或言二州附晉則榮|{
	二州謂梁益也}
六郡人事之不便昔公孫述在蜀羈客用事|{
	荆邯王元田戎延岑皆羈客也}
劉備在蜀楚士多貴|{
	龎統黄忠董和劉巴馬良兄弟呂乂廖立李嚴楊儀魏延蔣琬費褘董允等皆楚士也}
及吳鄧西伐|{
	吳鄧吳漢鄧艾也}
舉國屠滅寧分客主論者不達安固之基苟惜名位以為劉氏守令方仕州郡曾不知彼乃國亡主易豈同今日義舉主榮臣顯哉|{
	舉國奉晉為義舉晉加以寵秩則主榮臣顯}
論者又謂臣當為法正|{
	法正啓劉備以取成都壯亦教夀取李期故論者以比之}
臣蒙陛下大恩恣臣所安至於榮祿無問漢晉臣皆不處復何為效法正乎壽省書内慙|{
	處昌呂翻復扶又翻省悉景翻}
祕而不宣 九月漢僕射任顔謀反誅顔任太后之弟也漢主壽因盡誅成主雄諸子|{
	任后雄之正室也夀以任顔之反必以立諸甥為主故盡誅雄諸子以絶人望任音壬}
冬十月光祿勲顔含以老遜位|{
	引年致仕也}
論者以王導帝之師傅名位隆重百僚宜為降禮|{
	降禮謂拜之為于偽翻下同}
太常馮懷以問含含曰王公雖貴重理無偏敬|{
	臣子惟拜君父施之於導則為偏敬偏不正也}
降禮之言或是諸君事宜鄙人老矣不識時務既而告人曰吾聞伐國不問仁人|{
	董仲舒曰昔者魯君問柳下惠吾欲伐齊何如柳下惠曰不可歸而有憂色曰吾聞伐國不問仁人此言何為至於我哉}
向馮祖思問佞於我|{
	馮懷字祖思}
我豈有邪德乎郭璞嘗遇含欲為之筮|{
	因舍請老併及辭郭璞事以見其有識有守}
含曰年在天位在人修已而天不與者命也守道而人不知者性也自有性命無勞著龜|{
	蓍升脂翻}
致仕二十餘年年九十三而卒 代王翳槐之弟什翼犍質於趙|{
	為質見九十四卷咸和四年犍居言翻質音至}
翳槐疾病命諸大人立之翳槐卒諸大人梁蓋等以新有大故|{
	有大喪謂之大故滕文公曰今也不幸至於大故}
什翼犍在遠來未可必比其至恐有變亂謀更立君|{
	比必寐翻更古衡翻}
而翳槐次弟屈剛猛多詐不如屈弟孤仁厚乃相與殺屈而立孤孤不可自詣鄴迎什翼犍請身留為質|{
	質音至}
趙王虎義而俱遣之十一月什翼犍即代王位於繁畤北|{
	繁畤縣屬雁門郡畤音止}
改元曰建國分國之半以與孤初代王猗盧既卒國多内難部落離散|{
	事見八十九卷愍帝建興四年難乃旦翻}
拓跋氏寖衰及什翼犍立雄勇有智畧能修祖業國人附之始置百官分掌衆務以代人燕鳳為長史許謙為郎中令始制反逆殺人姧盜之法號令明白政事清簡無繫訊連逮之煩百姓安之於是東自濊貊|{
	濊音穢貊莫白翻}
西及破落那|{
	新唐書西域傳曰寜遠者本拔汗那或曰潑汗元魏時謂之破落那去長安八千里居西鞬城在真珠河之北}
南距陰山北盡沙漠率皆歸服有衆數十萬人|{
	史言代復彊}
十二月段遼自密雲山遣使求迎於趙|{
	使疏吏翻下同}
既而中悔復遣使求迎於燕|{
	復扶又翻}
趙王虎遣征東將軍麻秋帥衆三萬迎之|{
	帥讀曰率下同}
敕秋曰受降如受敵不可輕也|{
	降戶江翻}
以尚書左丞陽裕遼之故臣使為秋司馬燕王皝自帥諸軍迎遼|{
	帥讀曰率}
遼密與燕謀覆趙軍皝遣慕容恪伏精兵七千於密雲山大敗麻秋於三藏口|{
	水經注安州東有武列水其水三川泒合西源曰西藏水西南流而東藏水注之水出東溪西南流出谷與中藏水合水導中溪南流出谷南注東藏水東藏水又南右入西藏水故目其川曰三藏川魏收地形志曰皇興二年置安州統密雲等郡隋廢郡為密雲縣唐為檀州治所敗補邁翻}
死者什六七秋步走得免陽裕為燕所執趙將軍范陽鮮于亮失馬步緣山不能進因止端坐燕兵環之|{
	環音宦}
叱令起亮曰身是貴人義不為小人所屈汝曹能殺亟殺不能則去亮儀觀豐偉|{
	觀古玩翻}
聲氣雄厲燕兵憚之不敢殺以白皝皝以馬迎之與語大悦用為左常侍|{
	晉制諸王國大國置左右常侍}
以崔毖之女妻之|{
	妻七細翻}
皝盡得段遼之衆待遼以上賓之禮以陽裕為郎中令趙王虎聞麻秋敗怒削其官爵

五年春正月辛丑大赦 三月乙丑廣州刺史鄧岳將兵擊漢寧州漢建寧太守孟彦執其刺史霍彪以降|{
	咸和八年成取寧州今復之成以霍彪刺寧州見上卷咸和九年降戶江翻}
征西將軍庾亮欲開復中原表桓宣為都督沔北前鋒諸軍事司州刺史鎮襄陽|{
	沔彌兖翻}
又表其弟臨川太守懌為監梁雍二州諸軍事梁州刺史鎮魏興|{
	自李矩以司州刺史退屯卒于魯陽司州已寄治荆州界今始以司州治襄陽周訪領梁州治襄陽今司州既治襄陽故梁州治魏興監工銜翻下同}
西陽太守翼為南蠻校尉領南郡太守鎮江陵皆假節又請解豫州以授征虜將軍毛寶詔以寶監揚州之江西諸軍事豫州刺史與西陽太守樊峻帥精兵萬人戍邾城|{
	邾城在江北漢江夏郡邾縣之故城也楚宣王滅邾徙其君於此因以為名今黄州城是也杜佑曰黄州東南百二十里臨江與武昌相對有邾城此言唐黄州治所也西陽縣漢屬江夏郡魏分屬弋陽郡晉惠帝分弋陽為西陽國江左廢國為郡帥讀曰率下同}
以建威將軍陶稱為南中郎將江夏相入沔中稱將二百人下見亮|{
	亮鎮武昌稱自上流下見之相息亮翻將即亮翻}
亮素惡稱輕狡數稱前後罪惡收而斬之|{
	亮素怨陶侃而稱自間亮於王導蓋以私忿殺之素惡烏路翻數所具翻}
後以魏興險遠命庾懌徙屯半洲|{
	半洲在江州界康帝時褚裒為江州刺史鎮半洲}
更以武昌太守陳嚻為梁州刺史趣漢中|{
	趣七喻翻}
遣參軍李松攻漢巴郡江陽夏四月執漢荆州刺史李閎巴郡太守黄植送建康|{
	漢置荆州於巴郡}
漢主壽以李奕為鎮東將軍代閎守巴郡庾亮上疏言蜀甚弱而胡尚彊欲帥大衆十萬移鎮石城遣諸軍羅布江沔為伐趙之規帝下其議|{
	下遐稼翻}
丞相導請許之太尉鑒議以為資用未備不可大舉太常蔡謨議以為時有否泰|{
	否皮鄙翻}
道有屈伸苟不計彊弱而輕動則亡不終日何功之有為今之計莫若養威以俟時時之可否繫胡之彊弱胡之彊弱繫石虎之能否自石勒舉事虎常為爪牙百戰百勝遂定中原所據之地同於魏世勒死之後虎挾嗣君誅將相|{
	謂殺石堪程遐徐光諸將相也}
内難既平|{
	難乃旦翻}
剪削外寇一舉而拔金墉再戰而禽石生誅石聰如拾遺取郭權如振槁|{
	咸和八年虎殺石聰又拔金墉進殺石生九年取郭權事並見上卷}
四境之内不失尺土以是觀之虎為能乎將不能也論者以胡前攻襄陽不能拔|{
	事見上卷咸康元年}
謂之無能為夫百戰百勝之彊而以不拔一城為劣譬如射者百發百中而一失可以謂之拙乎|{
	中竹仲翻}
且石遇偏師也桓平北邊將也|{
	桓宣為平北將軍將即亮翻下同}
所爭者疆場之士|{
	士讀曰事}
利則進否則退非所急也今征西以重鎮名賢自將大軍欲席卷河南虎必自帥一國之衆來决勝負|{
	卷讀曰捲帥讀曰率}
豈得以襄陽為比哉今征西欲與之戰何如石生若欲城守何如金墉欲阻沔水何如大江欲拒石虎何如蘇峻凡此數者宜詳校之石生猛將關中精兵征西之戰殆不能勝也又當是時洛陽關中皆舉兵擊虎今此三鎮反為其用|{
	洛陽關中而曰三鎮併郭權據上邽為三也}
方之於前倍半之勢也石生不能敵其半而征西欲當其倍愚所疑也蘇峻之彊不及石虎沔水之險不及大江大江不能禦蘇峻而欲以沔水禦石虎又所疑也昔祖士雅在譙佃於城北界|{
	佃亭年翻}
胡來攻豫置軍屯以禦其外穀將熟胡果至丁夫戰於外老弱穫於内|{
	穫戶郭翻}
多持炬火急則燒穀而走如此數年竟不得其利當是時胡唯據河北方之於今四分之一耳|{
	言祖逖與石勒對境時勒僅有河北之地比之今來石虎據有之地止四分之一也}
士雅不能捍其一而征西欲以禦其四又所疑也然此但論征西既至之後耳|{
	謂既至中原之後也}
尚未論道路之慮也自沔以西水急岸高魚貫泝流首尾百里|{
	言水狹而急舟不得駢為一列而近也}
若胡無宋襄之義|{
	左傳宋襄公及楚人戰于泓宋人既成列楚人未既濟司馬子魚請擊之公曰不可既濟而未成列又以告公曰未可既陳而後擊之宋師敗績國人皆咎公公曰古之為軍也不以阻隘也寡人雖亡國之餘不鼓不成列}
及我未陣而擊之將若之何今王土與胡水陸異勢便習不同|{
	南便於用舟北便於用馬}
胡若送死則敵之有餘若弃江遠進以我所短擊彼所長懼非廟勝之算|{
	蔡謨之議量彼量已深切著明後郗鑒薦之自代蓋有見乎此也}
朝議多與謨同|{
	朝直遥翻}
乃詔亮不聽移鎮 燕前軍師慕容評廣威將軍慕容軍|{
	沈約志廣威將軍曹魏置}
折衝將軍慕輿根蕩寇將軍慕輿埿襲趙遼西俘獲千餘家而去趙鎮遠將軍石成|{
	鎮遠將軍蓋石氏所置}
積弩將軍呼延晃建威將軍張支等追之評等與戰斬晃支首 段遼謀反於燕燕人殺遼及其黨與數十人送遼首於趙 五月代王什翼犍會諸大人於參合陂|{
	參合縣前漢屬代郡後漢晉省東魏天平二年置梁城郡參合縣屬焉水經注參合陘在縣西北俗謂之倉鶴陘犍居言翻}
議都灅源川其母王氏曰吾自先世以來以遷徙為業|{
	謂逐水草為行國草盡水竭則徙而之他也灅力水翻又作㶟}
今國家多難若城郭而居一旦寇來無所避之乃止|{
	是後鍮勿崙之諫禿髪利鹿孤其說不過如此難乃旦翻}
代人謂他國之民來附者皆為烏桓什翼犍分之為二部各置大人以監之弟孤監其北子寔君監其南|{
	監工銜翻}
什翼犍求昏於燕燕王皝以其妹妻之|{
	妻七細翻}
秋七月趙王虎以太子宣為大單于建天子旌旗|{
	單音蟬}
庚申始興文獻公王導薨喪葬之禮視漢博陸侯及安平獻王故事|{
	霍光事見二十五卷漢宣帝地節二年安平王孚事見七十九卷武帝泰始八年}
參用天子之禮導簡素寡欲善因事就功雖無日用之益而歲計有餘輔相三世|{
	莊子曰日計之不足歲計之有餘向秀注云日計之不足無旦夕小利也歲計之有餘順時而大穰也三世元明成相息亮翻}
倉無儲穀衣不重帛|{
	重直龍翻}
初導與庾亮共薦丹陽尹何充於帝請以為已副且曰臣死之日願引充内侍則社稷無虞矣由是加吏部尚書及導薨徵庾亮為丞相揚州刺史錄尚書事亮固辭辛酉以充為護軍將軍亮弟會稽内史氷為中書監揚州刺史參錄尚書事|{
	會工外翻}
氷既當重任經綸時務不捨晝夜賓禮朝賢升擢後進由是朝野翕然稱之以為賢相|{
	朝直遥翻相息亮翻下同}
初王導輔政每從寬恕氷頗任威刑丹陽尹殷融諫之氷曰前相之賢猶不堪其弘|{
	堪任也言過於寛弘而不任也}
況如吾者哉范汪謂氷曰頃天文錯度|{
	七曜失行為錯度}
足下宜盡消禦之道氷曰玄象豈吾所測正當勤盡人事耳又隱實戶口料出無名萬餘人以充軍實|{
	隱度也料音聊}
氷好為糾察近於繁細後益矯違|{
	謂矯前之繁細而流於寬縱愈違於正道也好呼到翻}
復存寬縱|{
	復扶又翻下同}
疎密自由律令無用矣 八月壬午復改丞相為司徒|{
	去年省司徒并丞相府復扶又翻}
南昌文成公郗鑒疾篤以府事付長史劉遐|{
	此又一劉遐也}
上疏乞骸骨且曰臣所統錯雜率多北人或逼遷徙|{
	謂中原之人有戀土不肯南度者以兵威逼遷之也}
或是新附百姓懷土皆有歸本之心臣宣國恩示以好惡處與田宅漸得少安|{
	好呼到翻惡烏路翻處昌呂翻少詩沼翻}
聞臣疾篤衆情駭動若當北渡必啓寇心|{
	蓋時議欲徙京口之鎮渡江而北故鑒云然}
太常臣謨平簡貞正素望所歸謂可以為都督徐州刺史詔以蔡謨為太尉軍司加侍中辛酉鑒薨即以謨為征北將軍都督徐兖青三州諸軍事徐州刺史假節時左衛將軍陳光請伐趙詔遣光攻夀陽|{
	夀陽即夀春晉避簡文鄭太后諱改曰夀陽自祖約之敗為趙所據}
謨上疏曰壽陽城小而固自壽陽至琅邪城壁相望|{
	此琅邪謂古琅邪郡趙既取譙郡彭城下邳又得夀春故自夀春至琅邪城壁相望南琅邪在江乘之蒲洲上渡江而西歷歷陽合肥至夀春皆晉境趙未能置城壁也}
一城見攻衆城必救又王師在路五十餘日前驅未至聲息久聞|{
	聞音問}
賊之郵驛一日千里河北之騎足以相赴|{
	郵音尤騎奇寄翻下同}
夫以白起韓信項藉之勇猶發梁焚舟背水而陣|{
	戰國策白起曰楚王恃其國大城池不修又無守備故起得以引兵深入多倍城邑發梁焚舟以專民當是之時秦中士卒以軍中為家將為父母不約而親不謀而信一心同功死不旋踵楚人自戰其地咸顧其家各有散心莫有鬬志是以能有功也項羽焚舟即湛船以救鉅鹿事也見八十卷秦二世三年韓信背水事見九卷漢高帝三年背蒲昧翻}
今欲停船水渚引兵造城|{
	造七到翻}
前對堅敵顧臨歸路此兵法之所誡若進攻未拔胡騎卒至|{
	卒讀曰猝}
懼桓子不知所為而舟中之指可掬也|{
	左傳晉中行桓子帥師與楚戰于邲楚人車馳卒奔乘晉師桓子不知所為鼓於軍中曰先濟者有賞中軍與下軍爭舟舟中之指可掬也}
今光所將皆殿中精兵|{
	將即亮翻}
宜令所向有征無戰而頓之堅城之下以國之爪士|{
	詩曰祈父予王之爪士毛萇注曰士事也今謨直謂殿中兵為爪牙之士}
擊寇之下邑得之則利薄而不足損敵失之則害重而足以益寇懼非策之長者也乃止 初陶侃在武昌議者以江北有邾城宜分兵戍之侃每不答而言者不已侃乃渡水獵引將佐語之曰我所以設險而禦寇者正以長江耳邾城隔在江北内無所倚外接羣夷|{
	接西陽諸蠻也將即亮翻語牛倨翻}
夷中利深晉人貪利夷不堪命必引虜入寇此乃致禍之由非以禦寇也且吳時戍此城用三萬兵|{
	吳都武昌故屯重兵於邾城}
今縱有兵守亦無益於江南若羯虜有可乘之會此又非所資也|{
	羯居謁翻}
及庾亮鎮武昌卒使毛寶樊峻戍邾城趙王虎惡之|{
	卒子恤翻惡烏路翻}
以夔安為大都督帥石鑒石閔李農張貉李菟等五將軍|{
	帥讀曰率貉音鶴菟同都翻}
兵五萬人寇荆揚北鄙二萬騎攻邾城毛寶求救於庾亮亮以城固不時遣兵九月石閔敗晉兵於沔陰|{
	水南為陰即沔南也敗補邁翻下同}
殺將軍蔡懷夔安李農陷沔南|{
	晉人蓋置戍於沔南以備津要}
朱保敗晉兵於白石殺鄭豹等五將軍|{
	水經注柵水導源菓湖東逕南譙僑郡城南又東左會清溪水又東左會白石山水水發源白石山西}
張貉陷邾城死者六千人毛寶樊峻突圍出走赴江溺死|{
	溺奴狄翻}
夔安進據胡亭|{
	續漢志汝南汝陰縣西北有胡城春秋胡子之國也}
寇江夏義陽將軍黄冲義陽太守鄭進皆降於趙|{
	夏戶雅翻降戶江翻}
安進圍石城|{
	賢曰石城故城在今復州沔陽縣東南水經注沔水逕石城西城因山為固晉惠帝元康九年分江夏西部置竟陵郡治此}
竟陵太守李陽拒戰破之斬首五千餘級安乃退遂掠漢東擁七千餘戶遷于幽冀是時庾亮猶上疏欲遷鎮石城聞邾城陷乃止上表陳謝自貶三等行安西將軍|{
	晉方伯帶將軍有征鎮安平亮本征西將軍乞自貶三等行安西將軍上時掌翻}
有詔復位以輔國將軍庾懌為豫州刺史監宣城廬江歷陽安豐四郡諸軍事假節鎮蕪湖|{
	監工銜翻下同}
趙王虎患貴戚豪恣乃擢殿中御史李拒為御史中丞|{
	曹魏之制蘭臺遣二御史居殿中伺察非法此殿中御史之始也}
特加親任中外肅然虎曰朕聞良臣如猛虎高步曠野而豺狼避路信哉虎以撫軍將軍李農為使持節監遼西北平諸軍事征東將軍營州牧鎮令支|{
	趙置營州統遼西北平二郡使疏吏翻令音鈴又郎定翻支音祁}
農帥衆三萬與征北大將軍張舉攻燕凡城|{
	帥讀曰率}
燕王皝以榼盧城大悦綰為禦難將軍|{
	榼苦盍翻水經注曰渝水南流東屈與一水會世名之曰㯼倫水姓譜悦姓傅說之後難乃旦翻}
授兵一千使守凡城及趙兵至將吏皆恐欲弃城走綰曰受命禦寇死生以之且憑城堅守一可敵百敢有妄言惑衆者斬衆然後定綰身先士卒|{
	先息薦翻}
親冒矢石舉等攻之經旬不能克乃退虎以遼西迫近燕境數遭攻襲|{
	近其靳翻數所角翻}
乃悉徙其民於冀州之南 漢主壽疾病羅恒解思明復議奉晉壽不從李演復上書言之|{
	恒戶登翻解戶買翻復扶又翻}
壽怒殺演壽常慕漢武魏明之為人恥聞父兄時事上書者不得言先世政教自以為勝之也|{
	書無逸曰相小人厥父母勤勞稼穡厥子乃不知稼穡之艱難乃逸乃諺既誕否則侮厥父母曰昔之人無聞知其李夀之謂乎}
舍人杜襲作詩十篇託言應璩以諷諫|{
	應璩魏人有文名璩求於翻}
壽報曰省詩知意|{
	省悉景翻視也}
若今人所作乃前哲之話言|{
	話言善言也}
若古人所作則死鬼之常辭耳燕王皝自以稱王未受晉命冬遣長史劉翔參軍鞠

運來獻捷論功且言權假之意|{
	獻捷獻趙捷也權假謂自稱王也皝呼廣翻}
并請刻期大舉共平中原皝擊高句麗兵及新城|{
	新城高句麗之西鄙西南傍山東北接南蘇木底等城句如字又音駒麗力知翻}
高句麗王釗乞盟乃還又使其子恪霸擊宇文别部霸年十三勇冠三軍|{
	冠古玩翻}
張駿立辟雍明堂以行禮十一月以世子重華行凉州事|{
	重直龍翻}
十二月丁丑趙太保桃豹卒 丙戌以驃騎將軍琅邪王岳為侍中司徒|{
	驃匹妙翻}
漢李奕寇巴東守將勞楊敗死|{
	將即亮翻勞姓楊名}


六年春正月庚子朔都亭文康侯庾亮薨以護軍將軍錄尚書何充為中書令|{
	錄尚書即錄尚書事}
庚戌以南郡太守庾翼為都督江荆司雍梁益六州諸軍事安西將軍荆州刺史假節代亮鎮武昌|{
	雍於用翻}
時人疑翼年少不能繼其兄|{
	少詩照翻}
翼悉心為治戎政嚴明數年之閒公私充實人皆稱其才|{
	治直吏翻}
辛亥以左光禄大夫陸玩為侍中司空 宇文逸豆歸忌慕容翰才名翰乃佯狂酣飲或卧自便利|{
	便毘連翻溲也利下泄也}
或被髪歌呼拜跪乞食|{
	被皮義翻}
宇文舉國賤之不復省錄|{
	省察也視也錄采也收也記也音悉景翻}
以故得行來自遂山川形便皆默記之|{
	行來猶言往來也}
燕王皝以翰初非叛亂以猜嫌出奔|{
	事見上卷咸和八年}
雖在他國常潛為燕計|{
	如牛尾谷之戰是也}
乃遣商人王車通市於宇文部以窺翰翰見車無言撫膺頷之而已|{
	撫擊也膺胸也}
皝曰翰欲來也復使車迎之|{
	復扶又翻下同}
翰彎弓三石餘矢尤長大皝為之造可手弓矢|{
	可手便手也言惟翰手可用耳為于偽翻}
使車埋於道旁而密告之二月翰竊逸豆歸名馬攜其二子過取弓矢逃歸逸豆歸使驍騎百餘追之|{
	驍堅堯翻騎奇寄翻}
翰曰吾久客思歸既得上馬無復還理吾曏日陽愚以誑汝|{
	上時掌翻誑居況翻}
吾之故藝猶在無為相逼自取死也追騎輕之直突而前翰曰吾居汝國久悢悢|{
	李陵贈蘇武詩悢悢不能辭呂向注曰悢悢相戀之情}
不欲殺汝汝去我百步立汝刀吾射之一發中者汝可還不中者可來前追騎解刀立之一發正中其環|{
	孔穎達曰禮進劎者左首首劒拊鐶也少儀曰澤劒首鄭云澤弄也推尋劒刀利不容可弄正是劒鐶也又云刀郤刃授頴鄭云穎鐶也鐶與環同射而亦翻中竹仲翻}
追騎散走皝聞翰至大喜恩遇甚厚 庚辰有星孛于太微|{
	晉書天文志曰太微天子庭也五帝坐也十二諸侯府也孛蒲内翻}
三月丁卯大赦 漢人攻拔丹川守將孟彦劉齊李秋皆死|{
	五年孟彦以建寧降丹川當在建寧界}
代王什翼犍始都雲中之盛樂宫|{
	水經注白渠水出雲中塞外西北逕成樂固北魏土地記曰雲中城東八十里有成樂城今雲中郡治一名石盧城白渠水又西逕魏雲中宫南魏土地記曰雲中宫在雲中故城東四十里魏之盛樂即漢成樂縣也魏書曰猗盧城盛樂以為北都杜佑曰雲川雲中郡治雲中縣後魏道武自雲中徙都平城即此今馬邑郡北平城即今郡隋為雲内郡恒安鎮縣界有白登山白登臺高柳城參合陂後魏盛樂縣亦在今郡界單于臺在今縣西北百餘里}
趙王虎遺漢主壽書|{
	遺于季翻}
欲與之連兵入寇約中分江南壽大喜遣散騎常侍王嘏中常侍王廣使于趙|{
	散悉亶翻騎奇寄翻使疏吏翻}
龔壯諫不聽夀大修舟艦繕兵聚糧|{
	艦戶黯翻}
秋九月以尚書令馬當為六軍都督徵集士卒七萬餘人為舟師大閲於成都鼓譟盈江|{
	秦時蜀守李氷穿二江成都中皆可行舟}
壽登城觀之有吞噬江南之志解思明諫曰我國小兵弱吳會險遠圖之未易|{
	易以䜴翻}
壽乃命羣臣大議利害龔壯曰陛下與胡通孰若與晉通胡豺狼也既滅晉不得不北面事之若與之爭天下則彊弱不敵危亡之勢也虞虢之事已然之戒|{
	左傳晉獻公假道于虞以伐虢既滅虢遂滅虞}
願陛下熟慮之羣臣皆以壯言為然壽乃止士卒咸稱萬歲|{
	士無樂戰之心驅之而赴死地未有不敗者使李夀不用龔壯之言固不待李勢而蜀亡也}
龔壯以為人之行莫大於忠孝既報父叔之仇|{
	謂假手於夀以夷李特之子孫也行下孟翻}
又欲使壽事晉壽不從乃詐稱耳聾手不制物|{
	手不制物若病風緩然也}
辭歸以文籍自娛終身不復至成都|{
	復扶又翻}
趙尚書令夔安卒 趙王虎命司冀青徐幽并雍七州之民五丁取三四丁取二|{
	雍於用翻}
合鄴城舊兵滿五十萬具船萬艘|{
	艘蘇遭翻}
自河通海運穀千一百萬斛于樂安城|{
	水經注濡水東南過遼西海陽縣又逕牧城南分為二水北水謂之小濡水東逕樂安亭北東南入海濡水東南流逕樂安亭南東與新河故瀆合魏太祖征蹋頓所導也濡乃官翻}
徙遼西北平漁陽萬餘戶於兖豫雍洛四州之地|{
	石虎置司州於鄴以晉之司州為洛州雍於用翻}
自幽州以東至白狼|{
	白狼縣漢屬北平郡晉省水經注白狼水出白狼縣東南北逕白狼山又東北逕昌黎縣故城西又北逕黄龍城東又東北出東流為二水右水即渝水地理志曰渝水自塞南入海一水東北出塞為白狼水又東南流至房縣注于遼}
大興屯田悉括取民馬有敢私匿者腰斬凡得四萬餘匹大閲於宛陽|{
	水經注漳水自西門豹祠北逕趙閲馬臺西臺高五丈列觀其上石虎講武於其下列觀以望之}
欲以擊燕燕王皝謂諸將曰石虎自以樂安城防守重複|{
	重直龍翻}
薊城南北必不設備今若詭路出其不意可盡破也冬十月皝帥諸軍入自蠮螉塞|{
	帥讀曰率自龍城取西道入蠮螉塞蠮一結翻螉烏公翻}
襲趙戍將當道者皆禽之直抵薊城|{
	將即亮翻薊音計}
趙幽州刺史石光擁兵數萬閉城不敢出燕兵進破武遂津|{
	武遂縣前漢屬河間國後漢晉屬安平國時屬武邑郡易水過其南曰武遂津}
入高陽所至焚燒積聚|{
	積子賜翻聚才喻翻}
略三萬餘家而去|{
	考異曰燕書云畧燕范陽二郡男女數千口而還今從後趙燕載記}
石光坐懦弱徵還|{
	懦乃亂翻}
趙王虎以秦公韜為太尉與太子宣迭日省可尚書奏事|{
	迭日更日也省悉景翻}
專决賞刑不復啓白|{
	復扶又翻}
司徒申鍾諫曰賞刑者人君之大柄不可以假人所以防微杜漸消逆亂於未然也太子職在視膳不當豫政庶人邃以豫政致敗|{
	事見上卷咸康三年}
覆車未遠也且二政分權鮮不階禍|{
	左傳辛伯諗周桓公曰並后匹嫡兩政耦國亂之本也兩政即二政此指宣韜迭日决事鮮息淺翻}
愛之不以道適所以害之也虎不聽|{
	為宣殺韜張本}
中謁者令申扁|{
	中謁者令宦官也楊正衡曰扁芳連翻}
以慧悟辯給有寵於虎宣亦昵之|{
	昵尼質翻}
使典機密虎既不省事而宣韜皆好酣飲畋獵|{
	好呼到翻下同}
由是除拜生殺皆决於扁自九卿以下率皆望塵而拜太子詹事孫珍病目求方於侍中崔約約戲之曰溺中則愈|{
	戲言溺目中則病愈溺乃弔翻}
珍曰目何可溺約曰卿目睕睕正耐溺中|{
	楊正衡曰睕睕目深也音一丸翻耐乃代翻}
珍恨之以白宣宣於兄弟中最胡狀目深聞之怒誅約父子於是公卿以下畏珍側目燕公斌督邊州|{
	斌與張賀度共事蓋督北邊州也斌音彬}
亦好畋獵常懸管而入|{
	管者城門之管鑰也欲便於出故常懸管}
征北將軍張賀度每裁諫之斌怒辱賀度虎聞之使主書禮儀持節監之|{
	自東漢以來尚書諸曹各有主書蓋吏職也至齊梁之間其權任甚重禮姓也儀名也春秋時衛有大夫禮至監古街翻}
斌殺儀又欲殺賀度賀度嚴衛馳白之虎遣尚書張離帥騎追斌|{
	帥讀曰率}
鞭之三百免官歸第誅其親信十餘人|{
	史言虎無令子}
張駿遣别駕馬詵入貢於趙表辭蹇傲虎怒欲斬詵侍中石璞諫曰今國家所當先除者遺晉也河西僻陋不足為意今斬馬詵必征張駿則兵力分而為二建康復延數年之命矣|{
	復扶又翻}
乃止璞苞之曾孫也|{
	石苞事晉文帝武帝功參佐命}
初漢將李閎為晉所獲|{
	事見上卷}
逃奔于趙漢主壽致書於趙王虎以請之署曰趙王石君虎不悦付外議之中書監王波曰今李閎以死自誓曰苟得歸骨於蜀當糾帥宗族混同王化|{
	帥讀曰率}
若其信也則不煩一旅|{
	古者行軍五百人為一旅}
坐定梁益若有前却|{
	一前一却猶今人言心懷進退也}
不過失一亡命之人於趙何損李壽既僭大號今以制詔與之彼必酬返|{
	酬答也返還也}
不若復為書與之會挹婁國獻楛矢石砮於趙|{
	挹婁古肅慎氏之國也楛矢長尺有咫其國東北有山出石其利入鐵取以為砮杜佑曰挹婁國在不咸山北夫餘東北千餘里濱大海南與北沃沮接不知其北所極廣袤數千里人衆雖少而多勇力處山險善射弓長四尺力如砮矢用楛長尺八寸青石為鏃所謂石砮也其取石也必先祈神楛侯古翻木名似蓍砮音奴矢鏃也}
波因請以遺漢|{
	遺于季翻}
曰使其知我能服遠方也虎從之遣李閎歸厚為之禮閎至成都壽下詔曰羯使來庭貢其楛矢|{
	羯居謁翻使疏吏翻}
虎聞之怒黜王波以白衣領職

七年春正月燕王皝使唐國内史陽裕等|{
	慕容廆置唐國郡}
築城於柳城之北龍山之西立宗廟宫闕命曰龍城|{
	由是改柳城為龍城縣}
二月甲子朔日有食之 劉翔至建康帝引見|{
	見賢遍翻}
問慕容鎮軍平安對曰臣受遣之日朝服拜章|{
	言朝服南面拜發章表於庭朝直遥翻下同}
翔為燕王皝求大將軍燕王章璽|{
	璽斯氏翻}
朝議以為故事大將軍不處邊|{
	處昌呂翻}
自漢魏以來不封異姓為王所求不可許翔曰自劉石構亂長江以北翦為戎藪|{
	藪蘇口翻周禮注曰澤無水曰藪爾雅曰翦齊也}
未聞中華公卿之胄有一人能攘臂揮戈摧破凶逆者也獨慕容鎮軍父子竭力心存本朝以寡擊衆屢殄彊敵使石虎畏懼悉徙邊陲之民散居三魏|{
	謂徙遼西之民也魏郡陽平廣平為三魏}
蹙國千里以薊城為北境功烈如此而惜海北之地不以為封邑何哉昔漢高祖不愛王爵於韓彭故能成其帝業項羽刓印不忍授卒用危亡|{
	事見漢高祖紀卒子恤翻}
吾之至心非苟欲尊其所事竊惜聖朝疎忠義之國使四海無所勸慕耳尚書諸葛恢翔之姊夫也獨主異議以為夷狄相攻中國之利惟器與名不可輕許乃謂翔曰借使慕容鎮軍能除石虎乃是復得一石虎也|{
	復扶又翻}
朝廷何賴焉翔曰嫠婦猶知恤宗周之隕|{
	左傳鄭子太叔見范獻子曰嫠不恤緯而知宗周之隕王室之不寜晉之恥也嫠陵之翻}
今晉室阽危|{
	阽余廉翻}
君位侔元凱曾無憂國之心邪嚮使靡鬲之功不立則少康何以祀夏桓文之戰不捷則周人皆為左袵矣|{
	左傳夏之方衰也后羿因夏民以代夏政其臣寒浞殺羿而滅夏后相后緡逃歸有仍生少康焉靡奔有鬲氏自有鬲收衆以滅浞而立少康祀夏配天不失舊物齊桓公北伐山戎南伐楚晉文公勝楚於城濮皆率諸侯以尊周室孔子曰微管仲吾其被髪左袵矣}
慕容鎮軍枕戈待旦志殄凶逆而君更唱邪惑之言忌閒忠臣|{
	閒古莧翻}
四海所以未壹良由君輩耳翔留建康歲餘衆議終不决翔乃說中常侍彧弘曰|{
	或通作郁郁姓也姓譜有魯相郁貢說輸芮翻}
石虎苞八州之地帶甲百萬志吞江漢自索頭宇文暨諸小國無不臣服|{
	索昔各翻}
惟慕容鎮軍翼戴天子精貫白日而更不獲殊禮之命竊恐天下移心解體無復南向者矣|{
	復扶又翻}
公孫淵無尺寸之益於吳吳主封為燕王加以九錫|{
	事見七十二卷魏明帝青龍元年}
今慕容鎮軍屢摧賊鋒威震秦隴虎比遣重使|{
	比毘寐翻}
甘言厚幣欲授以曜威大將軍遼西王|{
	劉翔詭為是言耳然當時將軍必有曜威之號}
慕容鎮軍惡其非正却而不受|{
	惡烏路翻}
今朝廷乃矜惜虛名阻抑忠順豈社稷之長計乎|{
	沮在呂翻}
後雖悔之恐無及巳弘為之入言於帝|{
	為于偽翻下同}
帝意亦欲許之會皝上表稱庾氏兄弟擅權召亂|{
	以庾亮召蘇峻祖約之變使據上流庾亮死弟翼握兵于外弟氷專政于内也上時掌翻}
宜加斥退以安社稷又與庾氷書責其當國秉權不能為國雪恥氷甚懼以其絶遠非所能制乃與何充奏從其請乙卯以慕容皝為使持節大將軍都督河北諸軍事幽州牧大單于燕王|{
	單音蟬}
備物典策皆從殊禮|{
	師古曰既有備物而加之策書也杜預云典策春秋之制也余謂車輅旂章弓矢斧鉞皆可以言備物周成王分魯公以大路大旂封父之繁弱夏后氏之璜備物典策典者典法也策者策書也}
又以其世子㒞為假節安北將軍東夷校尉左賢王賜軍資器械以千萬計又封諸功臣百餘人以劉翔為代郡太守封臨泉鄉侯加員外散騎常侍|{
	晉志曰員外散騎常侍魏末置}
翔固辭不受翔疾江南士大夫以驕奢酣縱相尚嘗因朝貴宴集|{
	酣戶甘翻朝直遥翻}
謂何充等曰四海板蕩奄踰三紀|{
	板蕩刺周厲王之詩也板板反也言厲王為政反先王與天之道天下之民盡病也蕩蕩法度廢壞之貌言天下蕩蕩無綱紀文章也惠帝永興元年劉淵肇亂至是三十六年矣}
宗社為墟黎民塗炭斯乃廟堂焦慮之時忠臣畢命之秋也而諸君宴安江沱|{
	沱徒河翻江水别為沱}
肆情縱欲以奢靡為榮以傲誕為賢謇諤之言不聞征伐之功不立將何以尊主濟民乎充等甚慙詔遣兼大鴻臚郭悕持節詣棘城冊命燕王|{
	臚陵如翻悕香衣翻}
與翔等偕北公卿餞於江上翔謂諸公曰昔少康資一旅以滅有窮|{
	左傳少康邑於綸有田一成有衆一旅能布其德而兆其謀以收夏衆遂滅有窮少詩照翻}
句踐憑會稽以報彊吳|{
	越王句踐棲於會稽卧薪嘗膽卒以滅吳句音鉤}
蔓草猶宜早除|{
	左傳鄭祭仲曰無使滋蔓蔓難圖也蔓草猶不可除蔓音萬}
況寇讐乎今石虎李壽志相吞噬王師縱未能澄清北方且當從事巴蜀一旦石虎先人舉事|{
	先悉薦翻}
併壽而有之據形便之地以臨東南雖有智者不能善其後矣中護軍謝廣曰是吾心也 三月戊戌皇后杜氏崩夏四月丁卯葬恭皇后于興平陵 詔實王公以下至庶人皆正土斷白籍|{
	時王公庶人多自北來僑寓江左今皆以土著為斷著之白籍也白籍者戶口版籍也宋齊以下有黄籍斷丁亂翻}
秋七月郭悕劉翔等至燕燕王皝以翔為東夷護軍領大將軍長史以唐國内史陽裕為左司馬典書令李洪為右司馬|{
	晉制王國置典書典祠典衛學官令各一人典書令天朝吏部尚書之職中朝制典書令在常侍侍郎上及渡江則侍郎次常侍而典書令居三軍下}
中丞鄭林為軍諮祭酒八月辛酉東海哀王冲薨|{
	冲後東海王越事見八十七卷懷帝永嘉五年}


九月代王什翼犍築盛樂城於故城南八里|{
	犍居言翻樂音洛}
代王妃慕容氏卒 冬十月匈奴劉虎寇代西部代

王什翼犍遣軍逆擊大破之虎卒子務桓立遣使求和於代什翼犍以女妻之|{
	妻七細翻}
務桓又朝貢於趙|{
	朝直遥翻}
趙以務桓為平北將軍左賢王 趙横海將軍王華帥舟師自海道襲燕安平破之|{
	此遼東郡之西安平也四年華以青州之衆戍海島故得襲破之帥讀曰率}
燕王皝以慕容恪為渡遼將軍鎮平郭自慕容翰慕容仁之後諸將無能繼者及恪至平郭撫舊懷新屢破高句麗兵高句麗畏之不敢入境 十一月興平康伯陸玩薨 漢主壽以其太子勢領大將軍錄尚書事初成主雄以儉約寬惠得蜀人心及李閎王嘏還自鄴|{
	王嘏去年聘趙與李閎俱歸}
盛稱鄴中繁庶宫殿壯麗且言趙王虎以刑殺御下故能控制境内壽慕之徙旁郡民三代以上者以實成都大修宫室治器玩人有小過輒殺以立威|{
	治直之翻}
左僕射蔡興右僕射李嶷皆坐直諫死|{
	嶷魚力翻}
民疲於賦役吁嗟滿道思亂者衆矣|{
	史言漢將亡}


資治通鑑卷九十六
