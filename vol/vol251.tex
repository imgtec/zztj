<!DOCTYPE html PUBLIC "-//W3C//DTD XHTML 1.0 Transitional//EN" "http://www.w3.org/TR/xhtml1/DTD/xhtml1-transitional.dtd">
<html xmlns="http://www.w3.org/1999/xhtml">
<head>
<meta http-equiv="Content-Type" content="text/html; charset=utf-8" />
<meta http-equiv="X-UA-Compatible" content="IE=Edge,chrome=1">
<title>資治通鑒_252-資治通鑑卷二百五十一_252-資治通鑑卷二百五十一</title>
<meta name="Keywords" content="資治通鑒_252-資治通鑑卷二百五十一_252-資治通鑑卷二百五十一">
<meta name="Description" content="資治通鑒_252-資治通鑑卷二百五十一_252-資治通鑑卷二百五十一">
<meta http-equiv="Cache-Control" content="no-transform" />
<meta http-equiv="Cache-Control" content="no-siteapp" />
<link href="/img/style.css" rel="stylesheet" type="text/css" />
<script src="/img/m.js?2020"></script> 
</head>
<body>
 <div class="ClassNavi">
<a  href="/24shi/">二十四史</a> | <a href="/SiKuQuanShu/">四库全书</a> | <a href="http://www.guoxuedashi.com/gjtsjc/"><font  color="#FF0000">古今图书集成</font></a> | <a href="/renwu/">历史人物</a> | <a href="/ShuoWenJieZi/"><font  color="#FF0000">说文解字</a></font> | <a href="/chengyu/">成语词典</a> | <a  target="_blank"  href="http://www.guoxuedashi.com/jgwhj/"><font  color="#FF0000">甲骨文合集</font></a> | <a href="/yzjwjc/"><font  color="#FF0000">殷周金文集成</font></a> | <a href="/xiangxingzi/"><font color="#0000FF">象形字典</font></a> | <a href="/13jing/"><font  color="#FF0000">十三经索引</font></a> | <a href="/zixing/"><font  color="#FF0000">字体转换器</font></a> | <a href="/zidian/xz/"><font color="#0000FF">篆书识别</font></a> | <a href="/jinfanyi/">近义反义词</a> | <a href="/duilian/">对联大全</a> | <a href="/jiapu/"><font  color="#0000FF">家谱族谱查询</font></a> | <a href="http://www.guoxuemi.com/hafo/" target="_blank" ><font color="#FF0000">哈佛古籍</font></a> 
</div>

 <!-- 头部导航开始 -->
<div class="w1180 head clearfix">
  <div class="head_logo l"><a title="国学大师官网" href="http://www.guoxuedashi.com" target="_blank"></a></div>
  <div class="head_sr l">
  <div id="head1">
  
  <a href="http://www.guoxuedashi.com/zidian/bujian/" target="_blank" ><img src="http://www.guoxuedashi.com/img/top1.gif" width="88" height="60" border="0" title="部件查字,支持20万汉字"></a>


<a href="http://www.guoxuedashi.com/help/yingpan.php" target="_blank"><img src="http://www.guoxuedashi.com/img/top230.gif" width="600" height="62" border="0" ></a>


  </div>
  <div id="head3"><a href="javascript:" onClick="javascript:window.external.AddFavorite(window.location.href,document.title);">添加收藏</a>
  <br><a href="/help/setie.php">搜索引擎</a>
  <br><a href="/help/zanzhu.php">赞助本站</a></div>
  <div id="head2">
 <a href="http://www.guoxuemi.com/" target="_blank"><img src="http://www.guoxuedashi.com/img/guoxuemi.gif" width="95" height="62" border="0" style="margin-left:2px;" title="国学迷"></a>
  

  </div>
</div>
  <div class="clear"></div>
  <div class="head_nav">
  <p><a href="/">首页</a> | <a href="/ShuKu/">国学书库</a> | <a href="/guji/">影印古籍</a> | <a href="/shici/">诗词宝典</a> | <a   href="/SiKuQuanShu/gxjx.php">精选</a> <b>|</b> <a href="/zidian/">汉语字典</a> | <a href="/hydcd/">汉语词典</a> | <a href="http://www.guoxuedashi.com/zidian/bujian/"><font  color="#CC0066">部件查字</font></a> | <a href="http://www.sfds.cn/"><font  color="#CC0066">书法大师</font></a> | <a href="/jgwhj/">甲骨文</a> <b>|</b> <a href="/b/4/"><font  color="#CC0066">解密</font></a> | <a href="/renwu/">历史人物</a> | <a href="/diangu/">历史典故</a> | <a href="/xingshi/">姓氏</a> | <a href="/minzu/">民族</a> <b>|</b> <a href="/mz/"><font  color="#CC0066">世界名著</font></a> | <a href="/download/">软件下载</a>
</p>
<p><a href="/b/"><font  color="#CC0066">历史</font></a> | <a href="http://skqs.guoxuedashi.com/" target="_blank">四库全书</a> |  <a href="http://www.guoxuedashi.com/search/" target="_blank"><font  color="#CC0066">全文检索</font></a> | <a href="http://www.guoxuedashi.com/shumu/">古籍书目</a> | <a   href="/24shi/">正史</a> <b>|</b> <a href="/chengyu/">成语词典</a> | <a href="/kangxi/" title="康熙字典">康熙字典</a> | <a href="/ShuoWenJieZi/">说文解字</a> | <a href="/zixing/yanbian/">字形演变</a> | <a href="/yzjwjc/">金 文</a> <b>|</b>  <a href="/shijian/nian-hao/">年号</a> | <a href="/diming/">历史地名</a> | <a href="/shijian/">历史事件</a> | <a href="/guanzhi/">官职</a> | <a href="/lishi/">知识</a> <b>|</b> <a href="/zhongyi/">中医中药</a> | <a href="http://www.guoxuedashi.com/forum/">留言反馈</a>
</p>
  </div>
</div>
<!-- 头部导航END --> 
<!-- 内容区开始 --> 
<div class="w1180 clearfix">
  <div class="info l">
   
<div class="clearfix" style="background:#f5faff;">
<script src='http://www.guoxuedashi.com/img/headersou.js'></script>

</div>
  <div class="info_tree"><a href="http://www.guoxuedashi.com">首页</a> > <a href="/SiKuQuanShu/fanti/">四库全书</a>
 > <h1>资治通鉴</h1> <!--         下载:【右键另存为】即可 --></div>
  <div class="info_content zj clearfix">
  
<div class="info_txt clearfix" id="show">
<center style="font-size:24px;">252-資治通鑑卷二百五十一</center>
    資治通鑑卷二百五十一  宋 司馬光 撰<br />
<br />
  胡三省 音註<br />
<br />
  唐紀六十七【起著雍困敦盡屠維赤奮若凡二年】<br />
<br />
  懿宗昭聖恭惠孝皇帝中<br />
<br />
  咸通九年夏六月鳳翔少尹李師望上言嶲州控扼南詔為其要衝成都道遠難以節制請建定邊軍屯重兵於嶲州以邛州為理所【理所猶言治所也上時掌翻嶲音髓卭渠容翻】朝廷以為信然以師望為嶲州刺史充定邊軍節度眉蜀邛雅嘉黎等州觀察統押諸蠻并統領諸道行營制置等使師望利於專制方面故建此策其實邛距成都纔百六十里嶲距邛千里其欺罔如此【為李師望以定邊軍致寇張本】 初南詔陷安南【見上卷四年】敕徐泗募兵二千赴援分八百人别戍桂州初約三年一代徐泗觀察使崔彦曾慎由之從子也【崔慎由始見上卷宣宗大中十二年從才用翻下從孫同】性嚴刻朝廷以徐兵驕命鎮之都押牙尹戡教練使杜璋【大中六年五月敕天下軍府有兵馬處宜選會兵法能弓馬等人充教練使每年合教習時常令教習】兵馬使徐行儉用事軍中怨之戍桂州者已六年屢求代還戡言於彦曾以軍帑空虚【帑它朗翻】兵所費頗多請更留戍卒一年彦曾從之戍卒聞之怒都虞候許佶【佶其吉翻】軍校趙可立姚周張行實皆故徐州羣盜州縣不能討招出之補牙職會桂管觀察使李叢移湖南新使未至【校戶教翻使疏吏翻】秋七月佶等作亂殺都將王仲甫【將即亮翻】推糧料判官龎勛為主【唐制凡行軍置隨軍糧料使兵少者置糧料判官勛許云翻】刼庫兵北還所過剽掠【刼桂州庫兵北歸徐州還音旋又如字剽匹妙翻】州縣莫能禦朝廷聞之八月遣高品張敬思赦其辠【新書百官志内侍省有高品一千六百九十六人】部送歸徐州戍卒乃止剽掠 以前静海節度使高駢為右金吾大將軍駢請以從孫潯代鎮交趾從之【潯徐林翻考異曰補國史曰高公姪孫潯將先鋒軍每遇陳敵身當矢石及高公内舉交代朝廷命潯節制交趾實録但云高潯以下勒姓名於碑隂不云潯為節度使新傳曰駢之戰其從孫潯常為先鋒冒矢石以勸士駢徙天平薦潯自代詔拜交州節度使按駢為金吾半歲始除天平今從補國史】 九月戊戍以山南東道節度使盧耽為西川節度使以有定邊軍之故不領統押諸蠻安撫等使【既分西川置定邊軍則諸蠻皆在定邊軍巡内】 龎勛等至湖南【湖南觀察治潭州】監軍以計誘之使悉輸其甲兵【誘音酉】山南東道節度使崔鉉嚴兵守要害徐卒不敢入境泛舟沿江東下許佶等相與謀曰吾輩罪大於銀刀【銀刀見上卷三年】朝廷所以赦之者慮緣道攻刧或潰散為患耳若至徐州必葅醢矣乃各以私財造甲兵旗幟【幟昌志翻】過浙西入淮南淮南節度使令狐綯遣使慰勞給芻米【勞力到翻芻以飼馬米以給軍】都押牙李湘言於綯曰徐卒擅歸埶必為亂雖無敕令誅討藩鎮大臣當臨事制宜高郵岸峻而水深狹請將奇兵伏於其側焚荻舟以塞其前【塞悉則翻】以勁兵蹙其後可盡擒也不然縱之使得度淮至徐州與怨憤之衆合為患必大綯素懦怯且以無敕書乃曰彼在淮南不為暴聽其自過餘非吾事也勛招集銀刀等都竄匿及諸亡命匿於舟中衆至千人丁巳至泗州【泗州晉宋宿豫之地後魏置南徐州又置宿豫郡又改東徐州又改東楚州周大象三年改泗州開元二十四年移州治臨淮縣臨淮本漢徐城縣地當泗水口南北衝要之所】刺史杜慆饗之於毬場【慆他刀翻】優人致辭【致辭者今諸藩府有大宴則樂部頭當筵致辭稱頌賓主之美所謂致語者是也】徐卒以為玩已擒優人欲斬之坐者驚散慆素為之備徐卒不敢為亂而止慆悰之弟也【杜悰歷事穆文武宣屢入相位咸通初又為相】先是朝廷屢敕崔彦曾慰撫戍卒擅歸者勿使憂疑【先悉薦翻】彦曾遣使以敕意諭之道路相望勛亦申狀相繼辭禮甚恭戊午行及徐城【徐城縣屬泗州宋朝省徐城為鎮入臨淮縣在泗州北百餘里自此而西北則入徐州界然其道里迂遠故龎勛等西入宿州至符離則距徐州纔一百四十里耳】勛與許佶等乃言於衆曰吾輩擅歸思見妻子耳今聞已有密敕下本軍至則支分滅族矣【下戶嫁翻支分謂被支解而支體異處也即冎刑】丈夫與其自投網羅為天下笑曷若相與戮力同心赴蹈湯火豈徒脱禍兼富貴可求况城中將士皆吾輩父兄子弟吾輩一唱於外彼必響應於内矣然後遵王侍中故事【王侍中謂王智興也事見二百四十二卷穆宗長慶二年】五十萬賞錢可翹足待也衆皆呼躍稱善將士趙武等十二人獨憂懼欲逃去悉斬之遣使致其首於彦曾且為申狀稱勛等遠戍六年實懷鄉里而武等因衆心不安輒萌奸計將士誠知詿誤【詿古賣翻】敢避誅夷今既蒙恩全宥輒共誅首惡以補愆尤冬十月甲子使者至彭城彦曾執而訊之具得其情乃囚之丁卯勛復於逓中申狀【復扶又翻逓中謂入郵筒逓送使府】稱將士自負罪戾各懷憂疑今已及符離尚未釋甲【符離漢古縣時屬宿州九域志宿州北至徐州百二十里宋白曰爾雅莞符離北地尤多此草故名】蓋以軍將尹戡杜璋徐行儉等狡詐多疑必生釁隙乞且停此三人職任以安衆心仍乞戍還將士别置二營共為一將【將並即亮翻】時戍卒拒彭城止四驛【唐制三十里一驛四驛百二十里】闔城忷懼彦曾召諸將謀之皆泣曰比以銀刀兇悍【比毗至翻悍侯盰翻又下罕翻】使一軍皆蒙惡名殱夷流竄不無枉濫今寃痛之聲未已而桂州戍卒復爾猖狂【復扶又翻下同】若縱使入城必為逆亂如此則闔境塗地矣不若乘其遠來疲弊兵擊之我逸彼勞往無不捷彦曾猶豫未決團練判官温廷皓復言於彦曾曰安危之兆已在目前得失之機決於今日今擊之有三難而捨之有五害詔釋其罪而擅誅之一難也帥其父兄討其子弟二難也【帥讀曰率】枝黨鉤連刑戮必多三難也然當道戍卒擅歸不誅則諸道戍邊者皆效之無以制禦一害也將者一軍之首而輒敢害之【謂戍卒殺都將王仲甫也】則凡為將者何以號令士卒二害也所過剽掠【剽匹妙翻】自為甲兵招納亡命此而不討何以懲惡三害也軍中將士皆其親属銀刀餘黨潜匿山澤一旦内外俱何以支梧四害也【如淳曰枝梧猶枝扞也薛□曰小柱為枝邪柱為梧今屋梧邪柱是也】逼脇軍府誅所忌三將又欲自為一營【三將謂尹戡杜璋徐行儉及乞别營事並見上】從之則銀刀之患復起違之則託此為作亂之端五害也惟明公去其三難【去羌呂翻】絶其五害早定大計以副衆望時城中有兵四千三百彦曾乃命都虞侯元密等將兵三千人討勛數勛之罪以令士衆【數所具翻】且曰非惟塗炭平人實亦汚染將士【汚烏故翻染如艶翻又如險翻】儻國家兵誅討則玉石俱焚矣【書曰火炎崑岡玉石俱焚天吏逸德烈于猛火】又曰凡彼親屬無用憂疑罪止一身必無連坐仍命宿州出兵符離泗州出兵於虹以邀之【虹漢古縣宋魏廢省古城在夏丘縣界武德置虹縣於古虹城貞觀八年移治夏丘故城時屬宿州九域志在州東一百八十里顔師古曰虹音貢今音絳】且奏其狀彦曾戒元密無傷敕使【時張敬思尚在勛等軍中】戊辰元密彭城軍容甚盛諸將至任山北數里【任山在彭城西南三十里】頓兵不進共思所以奪敕使之計欲俟賊入舘乃縱兵擊之遣人變服負薪以詗賊【詗翾正翻又火迥翻】日暮賊至任山舘中空無人又無供給疑之見負薪者執而榜之【榜音彭】果得其情乃為偶人列於山下而潜遁比夜官軍始覺之【比必利翻及也下比官軍比追及皆同音】恐賊潜伏山谷及間道來襲【間古莧翻】復引兵退宿於城南明旦乃進追之時賊已至符離宿州戍卒五百人出戰於濉水上【濉水在虹縣靈壁東】望風奔潰賊遂抵宿州時宿州闕刺史觀察副使焦璐攝州事城中無復餘兵庚午賊攻陷之璐走免【璐音路 考異曰舊紀九月甲午勛陷宿州今從鄭樵彭門紀亂及新紀】賊悉聚城中貨財令百姓來取之一日之中四遠雲集然後選募為兵有不願者立斬之自旦至暮得數千人於是勒兵乘城龎勛自稱兵馬留後宿官軍始至賊守備已嚴不可復攻先是焦璐聞符離敗【先悉薦翻】決汴水以斷北路【斷音短】賊至水尚淺可涉比官軍至已深矣壬申元密引兵渡水將圍城會大風賊以火箭射城外茅屋【射而亦翻】延及官軍營士卒進則冒矢石退則限水火賊急擊之死者近三百人【近其靳翻】元密等以為賊必固守但為攻取之計賊夜使婦人持更【夜有五更使人各直一更撃鼓以警衆謂之持更顔之推曰一更二更三更四更皆以五為節西都賦云衛以嚴更之署所以爾者假令正月建寅斗柄夕則指寅晝則指午自寅至午凡歷五辰冬夏之月雖復長短然辰間遼闊盈不至六縮不至四進退常在五者之間更歷也經也故曰五更更工衡翻】掠城中大船三百艘備載資糧順流而下欲入江湖為盜【宿州居汴河之會漕運及商旅所經故城中有大船沿汴而下入淮則可以入江湖矣艘蘇遭翻】以千縑贈張敬思遣騎送至汴之東境【此謂汴州東境也】縱使西歸【謂西歸長安】明旦官軍知賊已去狼狽追之士卒皆未食比追及已飢乏賊檥舟隄下而陳於隄外【陳讀曰陣下同】伏千人於舟中【檥魚豈翻】官軍將至陳者皆走入陂中密以為畏已縱兵追之賊自舟中出夾攻之自午及申官軍大敗密引兵走陷於荷涫【涫古凡翻】賊追及之密等諸將及監陳敕使皆死士卒死者殆千人其餘皆降於賊無一人還徐者賊問降卒以彭城人情計謀知其無備始有攻彭城之志乙亥龎勛引兵北度濉水踰山趣彭城【趣七踰翻】其夕崔彦曾始知元密敗移牒鄰道求救明日塞門【塞悉則翻】選城中丁壯為守備内外震恐無復固志或勸彦曾奔兖州【九域志徐州北至兖州三百六十里】彦曾怒曰吾為元帥城陷而死職也立斬言者丁丑賊至城下衆六七千人鼓譟動地民居在城外者賊皆慰撫無所侵擾由是人爭歸之不移時克羅城彦曾退保子城【羅城外大城也子城内小城也】民助賊攻之推草車塞門而焚之【推吐雷翻塞悉則翻】城陷 【考異曰舊紀九月乙未龎勛陷徐州殺節度使崔彦曾判官焦璐等賊令别將梁丕守宿州又遣劉行及丁景琮吳迴攻圍泗州今從彭門紀亂及新紀舊彦曾傳曰九年九月十四日賊逼徐州十五日後每旦大霧十八日彦曾並誅逆卒家口十七日昏霧尤甚賊四面斬關而入實録自勛知徐州出兵退至符離已後皆置於十一月今從彭門紀亂】賊囚彦曾於大彭舘執尹戡杜璋徐行儉刳而剉之【刳其腹而寸剉之】盡滅其族勛坐聽事【徐州觀察廳事也聽讀曰廳】盛陳兵衛文武將吏伏謁莫敢仰視即日城中願附從者萬餘人戊寅勛召温庭皓使草表求節钺庭皓曰此事甚大非頃刻可成請還家徐草之勛許之明旦勛使趣之【趣讀曰促】庭皓來見勛曰昨日所以不即拒者欲一見妻子耳今已與妻子别謹來就死勛熟視笑曰書生敢爾不畏死邪龎勛能取徐州何患無人草表遂釋之有周重者每以才略自負勛迎為上客重為勛草表【重為于偽翻】稱臣之一軍乃漢室興王之地【漢高帝起於沛唐沛縣屬徐州故稱之以自夸大】頃因節度使刻削軍府刑賞失中遂致迫逐【言士卒所以迫逐主帥者皆其所自致】陛下奪其節制翦滅一軍【見上卷三年】或死或流寃横無數【横戶孟翻】今聞本道復欲誅夷將士不勝痛憤推臣權兵馬留後彈壓十萬之師撫有四州之地【勝音升四州謂徐宿濠泗】臣聞見利乘時帝王之資也臣見利不失遇時不疑伏乞聖慈復賜旌節不然揮戈曳戟詣闕非遲庚辰遣押牙張琯奉表詣京師勛以許佶為都虞侯趙可立為都遊奕使黨與各補牙職分將諸軍又遣舊將劉行及將千五百人屯濠州李圓將二千人屯泗州梁丕將千人屯宿州自餘要害縣鎮悉繕完戍守徐人謂旌節之至不過旬月願効力獻策者遠近輻湊乃至光蔡淮浙兖鄆沂密羣盜皆倍道歸之闐溢郛郭【闐停年翻郛芳無翻】旬日間米斗直錢二百【人來從亂者多故米踊貴】勛詐為崔彦曾請剪滅徐州表其略曰一軍暴卒盡可翦除五縣愚民各宜配【五縣彭城蕭豐沛滕也】又作詔書依其所讀傳布境内徐人信之皆歸怨朝廷曰微桂州將士回戈吾徒悉為魚肉矣劉行及引兵至渦口【渦口至濠州僅隔淮水耳渦音戈】道路附從者增倍濠州兵纔數百刺史盧望回素不設備不知所為乃開門具牛酒迎之行及入城囚望回自行刺史事 【考異曰舊紀實録新紀濠州陷在十一月按濠本徐之屬郡勛始得徐州則遣行及取之望回猶未及為備豈得至十一月今從彭門紀亂】泗州刺史杜慆聞勛作亂完守備以待之且求救於江淮李圓遣精卒百人先入泗州封府庫慆遣人迎勞【勞力到翻】誘之入城悉誅之【誘音酉】明日圓至即引兵圍城城上矢石雨下賊死者數百乃歛兵屯城西勛以泗州當江淮之衝益兵助圓攻之衆至萬餘終不能克【史於此略言其終下文始詳言其事】 初朝廷聞龎勛自任山還趣宿州【趣七喻翻】遣高品康道偉齎敕書撫慰之十一月道偉至彭城勛出郊迎自任山至子城三十里大陳甲兵號令金皷響震山谷城中丁壯悉驅使乘城宴道偉於毬場使人詐為羣盜降者數千人諸寨告捷者數十輩復作求節鉞表【復扶又翻】附道偉以聞 初辛雲京之孫讜【辛雲京見二百二十二卷肅宗寶應二年讜多曩翻】寓居廣陵喜任俠【喜許記翻如淳曰相與信為任同是非為俠所謂權行州里力折公侯者也或曰俠之為言挾也以權力俠輔人也】年五十不仕與杜慆有舊聞龎勛作亂詣泗州勸慆挈家避之慆曰安平享其禄位危難弃其城池【難乃旦翻】吾不為也且人各有家誰不愛之我獨求生何以安衆誓與將士共死此城耳讜曰公能如是僕與公同死乃還廣陵與其家訣壬辰復如泗州【復扶又翻】時民避亂扶老攜幼塞塗而來【塞悉則翻】見讜皆止之曰人皆南走子獨北行取死何為讜不應至泗州賊已至城下讜急棹小舟得入慆即署團練判官城中危懼都押牙李雅有勇略為慆設守備帥衆皷譟四出擊賊賊退屯徐城衆心稍安龎勛募人為兵人利於剽掠爭赴之【帥讀曰率剽匹妙翻】至父遣其子妻勉其夫皆斷鉏首而鋭之【據陸德明春秋左氏傳釋文斷音丁管翻讀如短齊景公使王黑以靈姑鉟率請斷三尺而用之楚令尹圍為王旌以田芋尹無宇斷之是也】執以應募鄰道聞勛據徐州各遣兵據要害而官軍尚少賊衆日滋官軍數不利【少詩沼翻數所角翻】賊遂破魚臺近十縣【近其靳翻】宋州東有磨山民逃匿其上勛遣其將張玄稔圍之會旱山泉竭數萬口皆渴死或說勛曰【說式芮翻下同】留後止欲求節钺當恭順盡禮以事天子外戢士卒内撫百姓庶幾可得勛雖不能用然國忌猶行香【唐自中世以後每國忌日令天下州悉於寺觀設齋焚香開成初禮部侍郎李蠡以其事無經據奏罷之尋而復舊畢仲荀幕府清閒録曰國忌行香起於後魏唐會要曰天寶七年敕華同等州僧尼道士國忌日各就龍興寺行道散齋至貞元五年處州奏當州不在行香之數乞同衢婺等州行香有旨依注又見前】饗士卒必先西向拜謝【凡方鎮大饗將士必朝服帥將佐西向望闕謝恩言皆出於君賜也】癸卯勛聞敕使入境以為必賜旌節衆皆賀明日敕使至但責崔彦曾及監軍張道謹貶其官勛大失望遂囚敕使不聽歸詔以右金吾大將軍康承訓為義成節度使徐州行營都招討使神武大將軍王晏權為徐州北面行營招討使羽林將軍戴可師為徐州南面行營招討使 【考異曰舊紀十年正月以神武大將軍王晏權為武寧節度使晏權智興之從子也以右神策大將軍康承訓充徐泗行營都招討使凡十八將分董諸道之兵七萬三千一十五人正月一日進軍攻徐州又曰承訓大軍攻宿州賊將梁丕出戰屢敗乃授承訓義成節度使實録九年十二月以右金吾大將軍康承訓為義成節度使充徐泗行營兵馬都招討使承訓不赴鎮以節度副使陳魴句當留後以王晏權為徐泗濠宿等州觀察使充徐州北面行營招討等使羽林將軍戴可師為徐州南面行營招討等使彭門紀亂新紀承訓等除招討使皆在十一月唐年補録十一月庚申以太原節度使康承訓為都統討徐州按庚申乃十二月一日承訓舊官亦非大原節度使補録誤也今從彭門紀亂新紀】大諸道兵以隸三帥【帥所類翻】承訓奏乞沙陀三部落使朱邪赤心【沙陀薩葛安慶分為三部】及吐谷渾達靼契苾酋長各帥其衆以自隨【靼當葛翻帥讀曰率】詔許之龎勛以李圓攻泗州久不克遣其將吳迥代之丙午復進攻泗州晝夜不息時敕使郭厚本 【考異曰舊紀實録作郗厚本今從彭門紀亂及舊傳】將淮南兵千五百人救泗州至洪澤【九域志楚州淮隂縣有洪澤鎮】畏賊彊不敢進辛讜請往求救杜慆許之丁未夜乘小舟潜度淮至洪澤說厚本厚本不聽比明復還己酉賊攻城益急欲焚水門城中幾不能禦【說式芮翻比必利翻幾居依翻】讜請復往求救慆曰前往徒還今往何益讜曰此行得兵則生返不得則死之慆與之泣别讜復乘小舟負戶突圍出見厚本為陳利害【為于偽翻下皆為同】厚本將從之淮南都將袁公弁曰賊埶如此自保恐不足何暇救人讜拔劒瞋目謂公弁曰【瞋昌真翻】賊百道攻城陷在朝夕公受詔救援而逗留不進豈惟上負國恩若泗州不守則淮南遂為寇場公詎能獨存邪我當殺公而後止耳起欲擊之厚本起抱止之公弁僅免讜乃回望泗州慟哭終日士卒皆為之流涕【為于偽翻】厚本乃許分五百人與之仍問將士將士皆願行讜舉身叩頭以謝將士遂帥之抵淮南岸【帥讀曰率】望賊方攻城有軍吏言曰賊埶已似入城還去則便【憚賊不敢進兵言還軍而去則於事為便也】讜逐之攬得其髻【攬撮持也】舉劒擊之士卒共救之曰予五百人判官不可殺也讜曰臨陳妄言惑衆【陳讀曰陣】必不可捨衆請不能得乃共奪之讜素多力衆不能奪讜曰將士但登舟我則捨此人衆競登舟乃捨之士卒有回顧者則斫之驅至淮北勒兵擊賊慆於城上布兵與之相應賊遂敗走皷譟逐之至晡而還【還從宣翻又如字】龎勛遣其將劉佶將精兵數千助吳迴攻泗州劉行及自濠州遣其將王弘立引兵會之戊午鎮海節度使杜審權【鎮海軍治潤州】遣都頭翟行約將四千人救泗州【翟直格翻】己未行約引兵至泗州賊逆擊於淮南圍之城中兵少不能救行約及士卒盡死先是令狐綯遣李湘將兵數千救泗州【先悉薦翻】與郭厚本袁公弁合兵屯都梁城【都梁城在泗州盱眙縣北都梁山項安世曰都梁縣有小山山上水極清淺其山中悉產蘭草緑葉紫莖俗謂蘭為都梁因以名縣】與泗州隔淮相望賊既破翟行約乘勝圍之十二月甲子李湘等引兵出戰大敗賊遂陷都梁城執湘及郭厚本送徐州 【考異曰舊紀十月賊攻泗州勢急令狐綯慮失淮口乃令大將李湘赴援舉軍皆沒湘與都監郭厚本俱為賊所執送徐州令狐綯傳曰賊聞湘來援遣人致書于綯辭情遜順言朝廷累有詔赦宥但抗拒者三兩人耳旦夕圖去之即束身請命願相公保任之綯即奏聞請賜勛節钺仍誠李湘但戍淮口賊已招降不得立異繇是湘軍解甲安寢去警撤備日與賊相對歡笑交言一日賊軍乘閒步騎徑入湘壘淮卒五千人皆被生縶送徐州為賊蒸而食之湘與監軍郭厚本為龎勛斷手足以徇於康承訓軍時浙西杜審權軍千人與李湘約會兵大將翟行約勇敢知名浙軍未至而湘軍敗賊乃分兵立淮南旗幟為交鬬之狀行約軍望見急趣之千人並為賊所縳送徐州綯既喪師朝廷以馬舉代綯為淮南節度使辛讜傳曰湘率五千來援賊詐降敗于淮口湘與郭厚本皆為賊所執彭門紀亂曰勛以泗州堅守遣劉佶共謀攻取時淮南宣潤三道兵戍都梁山舊城與泗州隔淮而已賊衆乃夜潜師屯淮及明而逼城濠州賊帥劉行及亦遣王弘立侵掠淮南於是合衆急攻官軍遂弃城出戰十一月三十日賊乃大敗官軍殺害二千人生降七八百人并虜其將李湘等咸送於徐州賊遂據有淮口斷絶驛路又曰賊既破戴可師令狐綯愳乃遣使誘諭約為奏請節旄續皇王寶運録曰十一月二十九日浙西節度使杜審權差都頭翟行約將兵二千來救三十日行約領兵方欲入泗州又破賊奔突行約古山尋被圍合城中兵士無可出救賊又開圍行約不知是計便走欲去而被着山下伏兵須臾被殺匹馬不餘賊遂圍淮口鎮有淮南都押衙李湘鎮將袁公弁領馬步三千人被圍從十一月三十日至十二月五日李湘束甲出軍被襲逐殺盡却入鎮者使豎降旗鎮内兵士老小一萬餘人被刼驅送濠州郭厚本此時遇害今從續寶運録】據淮口【泗水入淮之口】漕驛路絶【謂東南漕驛入上都之路絶】康承訓軍於新興【九域志宋州寧陵縣有新興鎮】賊將姚周屯柳子【九域志宿州臨渙縣有柳子鎮今在宿州北九十里范成大北使録曰自臨渙縣北行四十五里至柳子鎮張舜民郴行録曰柳子鎮在永城縣南九域志永城属亳州在州東北一百一十五里】出兵拒之時諸道兵集者纔萬人承訓以衆寡不敵退屯宋州龎勛以為官軍不足畏乃分遣其將丁從實等各將數千人南寇舒廬北侵沂海破沭陽下蔡烏江巢縣【沭陽漢廩丘縣後魏改曰沭陽唐属海州九域志在州西南一百八十里下蔡漢古縣唐属穎州烏江漢東城縣之烏江亭也隋置烏江縣唐屬和州九域志在州東北三十五里巢漢居巢縣隋為襄安縣武德七年改襄安為巢縣属廬州沐食聿翻】攻陷滁州殺刺史高錫望又寇和州【滁州南至和州百五十里】刺史崔雍遣人以牛酒犒之引賊登樓共飲命軍士皆釋甲指所愛二人為子弟乞全之其餘惟賊所處【處昌呂翻】賊遂大掠城中殺士卒八百餘人 【考異曰彭門紀亂光蔡山中草賊數攻破滁州殺刺史高錫望歸附龎勛舊紀十一月吳迥既執李湘乃令小將張行簡吳約攻滁州執刺史高錫望手刃之屠其城而去行簡又進攻和州刺史崔雍登城樓謂吳約云云遂剽城中居民殺判官張涿以涿浚城濠故也勛又令劉贄攻濠州陷之囚刺史盧望回於迴車舘望囘鬱憤而死實録閏月賊陷和州濠州明年二月又云勛遣張行簡攻滁州入城害刺史高錫望新紀十二月賊陷滁和今陷濠州從彭門紀亂陷滁和置執李湘下】泗州援兵既絶糧且盡人食薄粥閏月己亥辛讜言於杜慆請出求救於淮浙夜帥敢死士十人執長柯斧【柯斧柄也帥讀曰率】乘小舟潜往斫賊水寨而出明旦賊乃覺之以五舟遮其前以五千人夾岸追之賊舟重行遲讜舟輕行疾力鬭三十餘里乃得免癸卯至揚州見令狐綯甲辰至潤州見杜審權【揚州南至潤州五十餘里】時泗州久無聲問或傳已陷讜既至審權乃遣押牙趙翼將甲士二千人與淮南共輸米五千斛鹽五百斛以救泗州戴可師將兵三萬渡淮轉戰而前賊盡弃淮南之守可師欲先奪淮口後救泗州壬申圍都梁城城中賊少【少詩沼翻】拜於城上曰方與都頭議出降可師為之退五里【為于偽翻】賊夜遁明旦惟空城可師恃勝不設備是日大霧賊將王弘立引兵數萬疾徑奄至【疾徑猶言捷徑也不由正路直徑而行取其便疾】縱擊官軍官軍不及成列遂大敗將士觸兵及溺淮死得免者纔數百人亡器械資糧車馬以萬計賊傳可師及監軍將校首於彭城 【考異曰續寶運録曰正月十八日戴可師陷失賊遂凶狂彭門紀亂曰可師引兵三萬欲先奪淮口遂救泗州十二月十三日遲明圍賊於都梁山下賊已就降而可師自恃兵強不為備賊將王弘立者將兵數萬人捷徑赴救奔突而前官軍潰亂遂為所敗可師并監使將校已下咸沒於陣於是龎勛自謂前無彊敵矣舊紀十一月可師與賊轉戰賊黨屢敗盡弃淮南之守十年正月以可師充曹州行營招討使時賊將劉行及吳迴攻圍泗州可師乘勝救之屯於石梁驛賊退去可師追擊生擒行及賊保都梁城登城拜曰見與都頭謀歸降可師既知其窘乃退軍五里其城西面有水三面大軍賊乃夜中涉水而遁明早開城門惟病嫗數人而已王師入壘未整翌日詰旦重霧賊軍大至可師方大醉單馬奔出為虹縣人郭真所殺一軍盡沒賊將吳迴進軍復圍泗州又曰龎勛奏當道先戍嶺南兵士三千人春冬衣今欲差人送赴邕管鄂岳觀察使劉允章上書言龎勛聚徒十萬今若遣人逹嶺表如戍卒與勛合勢則禍難非細尋詔龎勛止絶兼令江淮諸道紀綱捕之實録可師敗繫於閏月下亦云十二月十三日新紀十二月壬申亦用紀亂之日也按紀亂上有臘月又云十二月十三日其下無閏月疑謂閏月十三日也然據續寶運録閏月十一日辛讜離泗州十四日至揚州乞兵粮若於時可師在都梁則讜必不舍可師而詣揚潤也若讜出在可師敗後則令狐綯方自救不暇何暇救泗州若可師敗在正月則新紀十二月已除馬舉南面招討使要之必在辛讜適揚潤之後故置於此】龎勛自謂無敵於天下作露布散示諸寨及鄉村於是淮南士民震恐往往避地江左令狐綯畏其侵軼【軼徒結翻】遣使詣勛說諭【說式芮翻】許為奏請節钺【為于偽翻】勛乃息兵俟命由是淮南稍得收散卒修守備時汴路既絶江淮往來者皆出壽州【自壽州泝淮即入穎汴路】賊既破戴可師乘勝圍壽州掠諸道貢獻及商人貨其路復絶【復扶又翻下同】勛益自驕日事遊宴周重諫曰自古驕滿奢逸得而復失成而復敗多矣况未得未成而為之者乎諸道兵大集於宋州徐州始懼應募者益少而諸寨求益兵者相繼勛乃使其黨散入鄉村驅人為兵又見兵已及數萬人【見賢遍翻】資糧匱竭乃歛富室及商旅財什取其七八坐匿財夷宗者數百家又與勛同舉兵於桂州者尤驕暴奪人資財掠人婦女勛不能制由是境内之民皆厭苦之不聊生矣王晏權兵數退衂【數所角翻】朝廷命泰寧節度使曹翔代晏權為徐州北面招討使【兖海號泰寧軍 考異正文曰曹翔馬舉為徐州南北招討使注曰彭門紀亂作馬士舉今從新紀紀亂曰王晏權數為賊所攻雖不敗傷亦時退縮朝廷復除隴州牧曹翔領兖海節度使充北面都統招討等使又魏博元帥何公遣行軍薛尤將兵三萬人掎角破賊曹翔軍於沛魏博軍於豐蕭其衆都六七萬人又言賊寇海州夀州皆敗又言辛讜救泗州雖繫正月之下蓋追叙以前之事實録二月以馬舉為淮南節度使充南面招討使初康承訓率諸將正月一日進軍攻徐州不克賊圍壽州王晏權數為賊所攻退縮不敢出戰乃以曹翔為兖海等州節度使充北面招討使魏博遣薛尤將兵三千掎角討賊賊衆攻海州戍兵擊之大敗康承訓率衆屯於柳子之西皆承此而誤也新紀翔舉除南北招討在十二月而無閏今因翔與魏博同討徐州而見之置於歲末余據考異及明年馬舉解泗州圍事則通鑑正文曹翔為徐州北面招討使之下當有以馬舉為淮南節度使充南面招討使十四字傳寫逸之也】前天雄節度使何全皥【按何全皥為魏博節度使魏博本號天雄軍未嘗徙他鎮疑史衍前字或曰是時秦州號天雄軍罷魏博軍號故加前字】遣其將薛尤將兵萬三千人討龎勛 【考異曰彭門紀亂曰尤將三萬人并曹翔軍都六七萬人實録魏博奏請出兵三千人助討徐泗舊紀魏博何弘敬奏當道點檢兵馬一萬三千赴行營姓名雖誤今取其人數】翔軍於滕沛尤軍於豐蕭【四縣皆屬徐州滕春秋滕子之國隋置滕縣宋白曰以縣西南四十里有滕城也豐漢古縣九域志滕在州北一百九十五里沛在西北一百四十里豐在西北一百四十里蕭在西五十里蕭縣亦以古蕭國為名】 是歲江淮旱蝗十年春正月康承訓將諸道軍七萬餘人屯柳子之西自新興至鹿塘三十里壁壘相屬【屬之欲翻】徐兵分戍四境城中不及數千人龎勛始懼民多穴地匿其中勛遣人搜掘為兵日不過得三二十人勛將孟敬文守豐縣狡悍而兵多謀貳于勛自為符䜟勛聞之會魏博攻豐勛遣腹心將將三千助敬文守豐敬文與之約共擊魏博軍且譽其勇【三千之下當有人字將將並即亮翻譽音余】使為前鋒新軍既與魏博戰【新軍謂龎勛新附之軍】敬文引兵退走新軍盡沒勛乃遣使紿之曰【紿徒亥翻】王弘立已克淮南留後欲自往鎮之悉召諸將欲選一人可守徐州者敬文喜即馳詣彭城未至城數里勛伏兵擒之辛酉殺之 丁卯同昌公主適右拾遺韋保衡以保衡為起居郎駙馬都尉【同昌隋郡名唐為疊州常芬縣】公主郭淑妃之女上特愛之傾宫中珍玩以為資送賜第於廣化里窗戶皆飾以雜寶井欄藥臼槽匱亦以金銀為之編金縷以為箕筐賜錢五百萬緡他物稱是【稱尺證翻】 徐賊寇海州【徐賊者龎勛所遣兵也九域志徐州東至海州四百八十里】時諸道兵戍海州者已數千人斷賊所過橋柱而弗殊【殊絶也斷橋柱而不使絶待賊過踐踏而自陷斷音短下鎻斷斧斷同】仍伏兵要害以待之賊過橋崩蒼黄散亂伏兵發盡殪之【殪壹計翻】其攻壽州者復為南道軍所破斬獲數千人【南道軍淮浙之兵也復扶又翻】辛讜以浙西之軍至楚州敕使張存誠以舟助之徐賊水陸布兵鎻斷淮流浙西軍憚其彊不敢進讜曰我請為前鋒勝則繼之敗則汝走猶不可讜乃募選軍中敢死士數十人牒補職名先以米舟三艘鹽舟一艘乘風逆流直進賊夾攻之矢著舟板如急雨【夾攻者兩岸賊兵也艘蘇遭翻著直畧翻】及鎻讜帥衆死戰斧斷其鎻乃得過城上人喧呼動地【帥讀曰率呼火故翻】杜慆及將佐皆泣迎之乙酉城上望見舟師張帆自東來識其旗浙西軍也去城十餘里賊列火船拒之帆止不進慆令讜帥死士出迎之乘戰艦衝賊陳而過【陳讀曰陣】見張存誠帥米舟九艘曰將士在道前却存誠屢欲自殺【憚敵而不敢進故為之一前一却】僅得至此今又不進讜揚言賊不多甚易與耳【所以作衆氣而使之進易以䜴翻】帥衆揚旗鼓譟而前【帥讀曰率下同】賊見其埶猛鋭避之遂得入城 二月端州司馬楊收長流驩州尋賜死其僚屬黨友坐長流嶺表者十餘人初尚書右丞裴坦子娶收女資送甚盛器用飾以犀玉坦見之怒曰破我家矣立命壞之【壞音怪】已而收竟以賄敗 康承訓使朱邪赤心將沙陀三千騎為前鋒陷陳却敵【陳讀曰陣】十鎮之兵伏其驍勇【十鎮謂義成魏博鄜延義武鳳翔横海泰寧宣武忠武天平也】承訓嘗引麾下千人渡渙水【宿州臨渙縣以臨渙水得名南北對境圖渙水出亳州南流入淮正直五河口】賊伏兵圍之赤心帥五百騎奮楇衝圍拔出承訓賊埶披靡【楇步爪翻披普彼翻】因合擊敗之【敗補邁翻】承訓數與賊戰【數所角翻】賊軍屢敗王弘立自矜淮口之捷【謂破戴可師也】請獨將所部三萬人破承訓龎勛許之己亥弘立引兵度濉水夜襲鹿塘寨黎明圍之弘立與諸將臨望自謂功在漏刻沙陀左右突圍出入如飛賊紛擾移避沙陀縱騎蹂之【蹂人九翻】寨中諸軍爭出奮擊賊大敗官軍蹙之於濉水溺死者不可勝紀【勝音升】自鹿塘至襄城【此襄城非汝州之襄城蓋徐宿間别自有襄城也】伏尸五十里斬首二萬餘級弘立單騎走免所驅掠平民皆散走山谷不復還營【復扶又翻】委弃資粮器械山積時有敕諸軍破賊得農民皆釋之自是賊每與官軍遇其驅掠之民先自潰龎勛許佶以弘立驕惰致敗欲斬之周重為之說勛曰【為于偽翻下為敵同說式芮翻】弘立再勝未賞【再勝謂取濠州破戴可師】一敗而誅之弃功録過為敵報讐諸將咸愳矣不若赦之責其後効勛乃釋之弘立收散卒纔數百人請取泗州以補過勛益其兵而遣之 三月辛未以起居郎韋保衡為左諫議大夫充翰林學士 徙郢王侃為威王【侃皇子也】 康承訓既破王弘立進逼柳子與姚周一月之間數十戰丁亥周引兵渡水【謂渡渙水也】官軍急擊之周退走官軍逐之遂圍柳子會大風四面縱火賊弃寨走沙陀以精騎邀之屠殺殆盡自柳子至芳城【芳城新書作芳亭】死者相枕【枕職任翻】斬其將劉豐周將麾下數十人奔宿州宿州守將梁丕素與之有隙開城聽入執而斬之龎勛聞之大懼與許佶議自將出戰【將即亮翻】周重泣言於勛曰柳子地要兵精姚周勇敢有謀今一旦覆沒危如累卵不若遂建大號悉兵四出決力死戰又勸殺崔彦曾以絶人望術士曹君長亦言徐州山川不容兩帥【帥所類翻】今觀察使尚在故留後未興賊黨皆以為然夏四月壬辰勛殺彦曾及監軍張道謹宣慰使仇大夫僚佐焦璐温庭皓并其親屬賓客僕妾皆死斷淮南監軍郭厚本都押衙李湘手足【斷丁管翻】以示康承訓軍勛乃集衆揚言曰勛始望國恩【大言以播告曰揚言望國恩謂望旌節也】庶全臣節今日之事前志已乖自此勛與諸君真反者也當掃境内之兵戮力同心轉敗為功耳衆皆稱善於是命城中男子悉集毬場仍分遣諸將比屋大索【比毘必翻索山客翻】敢匿一男子者族其家選丁壯得三萬人更造旗幟【更工衡翻幟昌志翻】給以精兵許佶等共推勛為天冊將軍大會明王勛辭王爵先是辛讜復自泗州引驍勇四百人迎粮於揚潤【先悉薦翻復扶又翻】賊夾岸攻之轉戰百里乃得出至廣陵止于公館不敢歸家舟載鹽米二萬石錢萬三千緡乙未還至斗山【斗山在今盱眙縣亦曰陡山臨淮流斗山之東則古盱眙】賊將王弘芝帥衆萬餘拒之於盱眙密布戰艦百五十艘以塞淮流【帥讀曰率塞悉則翻】又縱火船逆之讜命以長义托過自卯戰及未衆寡不敵官軍不利賊縛木於戰艦【艦戶黯翻】旁出四五尺為戰棚【棚浦庚翻】讜命勇士乘小舟入其下矢刃所不能及以槍揭火牛焚之【掲其謁翻火牛縳草為之熱以燒敵今沿邉州郡防城庫積草謂之火牛草】戰艦既然【然謂火然也】賊皆潰走官軍乃得過入城 【考異曰續寶運録曰二月七日辛讜㨂點驍勇領空船十二隻般糧二十日却到楚州四月六日離楚八日至斗山下是日二更後入泗州按正月二十七日讜迎米船九隻入泗州二月六日未應食盡復出又二十日却到楚州不應住四十五日然後離彼又上有二月十日授讜御史不應下云二月七日讜出般糧疑是三月字也】 龎勛以父舉直為大司馬與許佶等留守徐州或曰將軍方耀兵威不可以父子之親失上下之節乃令舉直趨拜於庭勛據案而受之時魏博屢圍豐縣龎勛欲先擊之丙申引兵徐州 戊戍以前淮南節度使同平章事令狐綯為太保分司【以綯在淮南喪師命馬舉代之】 龎勛夜至豐縣潜入城魏博軍皆不之知魏博分為五寨其近城者屯數千人【近其靳翻】勛縱兵圍之諸寨救之勛伏兵要路殺官軍二千人餘皆返走賊攻寨不克至夜解圍去官軍畏其衆且聞勛自來諸寨皆宵潰曹翔方圍滕縣聞魏博敗引兵退保兖州【曹翔泰寧帥本治兖州故退保之】賊悉毁其城栅運其資糧傳檄徐州盛自誇大謂官軍為國賊云 馬舉將精兵三萬救泗州乙巳分軍三道度淮至中流大譟聲聞數里【聞音問】賊大驚不測衆寡斂兵屯城西寨舉就圍之縱火焚栅賊衆大敗斬首數千級王弘立死吳迥退保徐城泗州之圍始解泗州被圍凡七月【泗州自去年九月末受圍】守城者不得寐面目皆生瘡 龎勛留豐縣數日欲引兵西擊康承訓或曰天時向暑蠶麥方急不若且休兵聚食然後圖之或曰將軍出師數日摧七萬之衆【謂破魏博之兵也】西軍震恐【西軍謂康承訓之軍也時屯柳子其地在豐縣之西】乘此聲埶彼破走必矣時不可失龎舉直以書勸勛乘勝進軍勛意遂決丁未豐縣庚戍至蕭約襄城留武小睢諸寨兵合五六萬人以二十九日遲明攻柳子【遲直利翻待也】淮南敗卒在賊中者【李湘袁公弁之兵也】逃詣康承訓告以其期承訓得先為之備秣馬整衆設伏以待之丙辰襄城等兵先至柳子遇伏敗走龎勛既自失期遽引兵自三十里外赴之比至【比必利翻】諸寨已敗勛所將皆市井白徒覩官軍勢盛皆不戰而潰承訓命諸將急追之以騎兵邀其前步卒蹙其後賊狼狽不知所之自相蹈藉僵尸數十里【藉慈夜翻僵居良翻】死者數萬人勛解甲服布襦而遁【襦汝朱翻短衣也】收散卒纔及三千人歸彭城 【考異曰實録勛敗於柳子在五月蓋約奏到書之其他皆如此雖有月日不可用今從彭門紀亂】使其將張實分諸寨兵屯第城驛【第城驛在宿州西】勛初起下邳土豪鄭鎰【鎰音逸】聚衆三千自備資糧器械以應之勛以為將謂之義軍五月沂州遣軍圍下邳【下邳縣屬徐州九域志在州東一百八十里】勛命鎰救之鎰帥所部來降【帥讀曰率】 六月陜民作亂逐觀察使崔蕘【陜失冉翻蕘如招翻】蕘以器韵自矜不親政事民訴旱蕘指庭樹曰此尚有葉何旱之有杖之民怒故逐之蕘逃於民舍渴求飲民以溺飲之【溺奴弔翻飲之於鴆翻】坐貶昭州司馬 以中書侍郎同平章事徐商同平章事充荆南節度使癸卯以翰林學士承旨戶部侍郎劉瞻同平章事 【考異曰玉泉子聞見録曰徐公商判鹺以瞻為從事商拜相命官曾不及瞻瞻出於覊旅以楊玄翼樞密權重可倚以圖事而密啗閽者謁焉瞻有儀表加之詞辯俊利玄翼一見悦之每玄翼歸第瞻輒候之由是日加親熟遂許以内廷之拜既有日矣瞻即復謁徐公曰相公過聽以某辱在門舘幸遇相公登庸四海之人孰不受相公之惠某故相公從事窮飢日加且環歲矣相公曾不以下位處之某雖不佞亦相公之恩不終也今已别有計矣請從此辭即下拜焉商初聞瞻言徒唯唯而已迨聞别有計不覺愕然方欲遜謝瞻已疾趨出矣明日内牓子出以瞻為翰林學士舊瞻傳劉瑑作相以宗人遇之薦為翰林學士按瞻素有清節必不至如玉泉子所云恐出於愛憎之說聞見録又云玄翼為鳳翔監軍瞻即出為太原亞尹鄭從讜為節度使殊不禮焉洎復入翰林而作相也常謂人曰吾在北門為鄭尚書冷將息不復病熱矣從讜南海之命瞻所致也按舊傳瞻自戶部侍郎承旨出為太原尹河東節度使瞻為學士若非罪謫恐不為少尹又舊紀咸通十二年十二月鄭從讜自宣武節度使為廣州在瞻驩州後故知玉泉子所記皆虛今不取】瞻桂州人也 馬舉自泗州引兵攻濠州拔招義鍾離定遠【招義漢睢陵縣地宋置濟隂郡隋廢郡為化明縣武德七年改為招義鍾離漢古縣定遠漢曲陽縣地梁改為定遠唐皆属濠州九域志招義在州東一百二十四里定遠在州南八十里】劉行及設寨於城外以拒守舉先遣輕騎挑戰【挑徒了翻】賊見其衆少爭出寨西擊之舉引大軍數萬自它道擊其東南遂焚其寨賊入固守舉塹其三面而圍之北面臨淮賊猶得與徐州通龎勛遣吳迴助行及守濠州屯兵北津以相應【北津淮水之北岸也凡臨水濟渡之處謂之津】舉遣别將度淮擊之斬獲數千平其寨 曹翔之退屯兖州也留滄州卒四千人戍魯橋【滄州卒横海之兵也九域志濟州任城縣有魯橋鎮】卒擅還翔曰以龎勛作亂故討之今滄卒不從約束是自亂也勒兵迎之圍於兖州城外擇違命者二千人悉誅之朝廷聞魏博軍敗以將軍宋威為徐州西北面招討使將兵三萬屯於豐蕭之間翔復引兵會之【復扶又翻】秋七月康承訓克臨渙殺獲萬人遂拔襄城留武小睢等寨曹翔拔滕縣進擊豐沛賊諸寨戍兵多相帥逃匿保據山林【帥讀曰率】賊抄掠者過之【抄楚交翻】輒為所殺而五八村尤甚有陳全裕者為之帥【帥所類翻】凡叛勛者皆歸之衆至數千人戰守之具皆備環地數千里【環音宦】賊莫敢近【近其靳翻】康承訓遣人招之遂舉衆來降賊黨益離蘄縣士豪李衮殺賊守將舉城降於承訓【靳漢古縣唐属宿州九域志在州南三十六里】沛縣守將李直詣彭城計事裨將朱玫舉城降于曹翔【玫莫柸翻】直自彭城還玫逆擊走之翔發兵戍沛玫邠州人也勛遣其將孫章許佶各將數千人攻陳全裕朱玫皆不克而還【還從宣翻】康承訓乘勝長驅拔第城進抵宿州之西築城而守之龎勛憂懣不知所為但禱神飯僧而已【懣莫困翻心煩也飯扶晩翻】 初龎勛怒梁丕專殺姚周黜之使徐州舊將張玄稔代之治州事【治直之翻】以其黨張儒張實等將城中兵數萬拒官軍儒等列寨數重于城外環水自固【重直龍翻環音宦】康承訓圍之張實夜遣人潜出以書白勛曰今國兵盡在城下【國兵謂官軍也】西方必虚將軍宜引兵出其不意掠宋亳之郊彼必解圍而西將軍設伏要害迎擊其前實等出城中兵蹙其後破之必矣時曹翔使朱玫擊豐破之乘勝攻徐城下邳皆拔之斬獲萬計勛方憂懼欲走得實書即從其策使龎舉直許佶守徐州引兵而西八月壬子康承訓焚外寨張儒等入保羅城【外寨宿州城外之寨羅城宿州羅城也】官軍攻之死者數千人不能克 【考異曰舊紀實録皆云八月康承訓攻柳子寨垂克而賊將王弘立救至王師大敗承訓退保宋州龎勛乘勝自帥徐州勁卒倂攻泗州留其都將許佶守徐州詔馬舉援泗州按弘立救柳子為承訓所敗兼於時弘立已死於泗州勛亦未嘗親攻泗州舊紀實録誤也】承訓患之遣辯士於城下招諭之張玄稔嘗戍邊有功雖脅從於賊心嘗憂憤【心嘗當作常】時將所部兵守子城夜召所親數十人謀歸國因稍令布諭恊同者衆乃遣腹心張臯夜出以狀白承訓約期殺賊將舉城降至日請立青旌為應使衆心無疑【木行色青木主生使立青旌以示不殺】承訓大喜從之九月丁巳張儒等飲酒於柳溪亭玄稔使部將董厚等勒兵於亭西玄稔先躍馬而前大呼曰龎勛已梟首於僕射寨中【呼火故翻僕射謂承訓也】此輩何得尚存士卒競進遂斬張儒等數十人城中大擾玄稔諭以歸國之計及暮而定戊午開門出降玄稔見承訓肉袒行涕泣謝罪承訓慰勞【勞力到翻下賞勞同】即宣敕拜御史中丞賜遺甚厚【遺唯季翻】玄稔復進言【復扶又翻】今舉城歸國四遠未知請詐為城陷引衆趨符離及徐州【趣七喻翻下同】賊黨不疑可盡擒也承訓許之宿州舊兵三萬承訓益以數百騎皆賞勞而遣之玄稔復入城暮平安火如常日己未向晨玄稔積薪數千束縱火焚之如城陷軍潰之狀直趨符離符離納之既入斬其守將號令城中皆聽命收其兵復得萬人【復扶又翻】北趨徐州龎舉直許佶聞之嬰城拒守辛酉玄稔至彭城引兵圍之按兵未攻先諭城上人曰朝廷唯誅逆黨不傷良人汝曹奈何為賊城守【為于偽翻】若尚狐疑須臾之間同為魚肉矣於是守城者稍稍弃甲投兵而下崔彦曾故吏路審中開門納官軍龎舉直許佶帥其黨保子城【帥讀曰率】日昃賊黨自北門出玄稔遣兵追之斬舉直佶首餘黨多赴水死悉捕戍桂州者親族斬之死者數千人徐州遂平龎勛將兵二萬自石山西出所過焚掠無遺庚申承訓始知引步騎八萬西擊之使朱邪赤心將數千騎為前鋒 【考異曰彭門紀亂云沙陀都頭朱邪赤衷按獻祖紀年録當作赤心紀亂誤也】勛襲宋州陷其南城刺史鄭處沖守其北城【處昌呂翻】賊知有備捨去度汴南掠亳州【九域志宋州南至亳州一百二十里】沙陀追及之勛引兵循渙水而東將歸彭城為沙陀所逼不暇飲食至蘄【蘄秦漢古縣宋置譙郡齊為北譙郡時為縣屬宿州九域志在州南三十六里】將濟水李衮橋勒兵拒之賊惶惑不知所之至縣西官軍大集縱擊殺賊近萬人【近其靳翻】餘皆溺死降者纔及千人勛亦死而人莫之識數日乃獲其尸 【考異曰彭門紀亂曰初龎勛之求節也必希歲内得之於是閭里小兒競歌之曰得節不得節不過十二月即龎勛九年十月十七日作亂十年九月十九日就戮通其閏月計之正一歲而滅按六日承訓知勛掠亳宋即追之至蘄縣得之恐未至十九日疑是九日也新紀九月癸酉龎勛伏誅用彭門紀亂也】賊宿遷等諸寨皆殺其守將而降【宿遷晉宿預縣也唐避代宗諱改曰宿遷屬徐州在下邳東南一百八十里】宋威亦取蕭縣吳迥獨守濠州不下冬十月以張玄稔為右驍衛大將軍御史大夫馬舉攻濠州自夏及冬不克城中粮盡殺人而食之官軍深塹重圍以守之【重直龍翻】辛丑夜吳迥突圍走舉勒兵追之殺獲殆盡迥死於招義以康承訓為河東節度使同平章事以杜慆為義成節度使上嘉朱邪赤心之功置大同軍於雲州以赤心為節度使【會昌中已置大同軍團練使於雲州尋為防禦今陞為節鎮】召見留為左金吾上將軍賜姓名李國昌【其後李國昌父子卒以雲州起兵蓋尋遣之還鎮也薛史曰赤心賜姓名系鄭王房見賢遍翻】賞賚甚厚以辛讜為亳州刺史讜在泗州犯圍出迎兵糧往返凡十二及除亳州上表言臣之功非杜惂不能成也賜和州刺史崔雍自盡【以其開門延賊也 考異曰舊紀八月和州防禦行官石侔等訟雍罪其月賜自盡實録訟在八月賜自盡在十月今從之】家屬流康州兄弟五人皆遠貶 上荒宴不親庶政委任路巖巖奢靡頗通賂遺【遺唯季翻】左右用事至德令陳蟠叟因上書召對【肅宗至德元載分鄱陽秋浦置至德縣屬饒州】言請破邊咸一家可贍軍二年上問咸為誰對曰路巖親吏上怒流蟠叟於愛州自是無敢言者 初南詔遣使者楊酋慶來謝釋董成之囚【釋董成見上卷七年】定邊節度使李師望欲激怒南詔以求功遂殺酋慶西州大將恨師望分裂巡屬【謂分四川巡屬邛嶲等州别立定邊軍也事見上九年六月】隂遣人致意南詔使入寇師望貪殘聚私貨以百萬計戍卒怨怒欲生食之師望以計免朝廷徵還以太府少卿竇滂代之滂貪殘又甚於師望故蠻寇未至而定邊固已困矣是月南詔驃信酋龍傾國入寇引數萬衆擊董舂烏部破之【董舂烏部西州附寨蠻也】十一月蠻進寇嶲州定邊都頭安再榮守清溪關蠻攻之再榮退屯大渡河北與之隔水相射【射而亦翻】九日八夜蠻密分軍開道逾雪坡奄至沐源川【雪坡雪嶺之坡也沐源川在嘉州羅目縣界麟德二年開生撩置羅目縣及沐州後廢沐州以羅目屬嘉州宋朝又廢羅目為鎮屬峨眉縣又今嘉州犍為縣有沐川鎮】滂遣兖海將黄卓帥五百人拒之舉軍覆沒【帥讀曰率】十二月丁酉蠻衣兖海之衣詐為敗卒至江岸呼船【蠻衣於既翻此江青衣江也】已濟衆乃覺之遂陷犍為縱兵焚掠陵榮二州之境【犍為漢郡名後周置武陽縣隋開皇初改名犍為因山為名也唐屬嘉州九域志在州東南一百二十里犍居言翻】後數日蠻軍大集於陵雲寺與嘉州對岸【嘉州漢犍為郡南安縣地梁武帝開通外徼立青州取青衣以為名西魏改青州為眉州取峨眉山以為名後周復曰青州又改曰嘉州取漢嘉郡以為名隋又改曰眉州唐復曰嘉州别置眉州於漢武陽縣地陵雲寺在嘉州南山開元中僧海通於瀆江沫水濛水三江之會悍流怒浪之濱鑿山為彌勒大像高踰三百六十尺建七層閣以覆之】刺史楊忞【忞莫中翻】與定邊監軍張允瓊勒兵拒之蠻潜遣奇兵自東津濟夾擊官軍殺忠武都將顔慶師餘衆皆潰忞允瓊脱身走壬子陷嘉州慶師慶復之弟也竇滂自將兵拒蠻於大渡河驃信詐遣清平官數人詣滂結和滂與語未畢蠻乘船栰爭渡忠武徐宿兩軍結陳抗之【徐宿舊武寧軍以其軍數亂逆罷節鎮陳讀曰陣】滂懼自經於帳中徐州將苖全緒解之曰都統何至於是全緒與安再榮及忠武將勒兵出戰滂遂單騎宵遁三將謀曰今衆寡不敵明旦復戰吾屬盡矣【復扶又翻】不若乘夜攻之使之驚亂然後解去于是夜入蠻軍弓弩亂蠻大驚三將乃全軍引去蠻進陷黎雅民竄匿山谷敗軍所在焚掠滂奔導江【導江本劉蜀所置都安縣後周改為汶山唐改曰導江屬彭州九域志在州西九十里】邛州軍資儲偫皆散於亂兵之手【偫丈里翻】蠻至城已空通行無礙矣 【考異曰張雲咸通解圍録曰十年十月南蠻衆擊董舂烏部落傾其巢窟舂烏以其衆保北栅俄而蠻掩至沐源川遂逼嘉州南自清黎關寇黎雅張錦里耆舊傳曰十一年庚寅節度使盧僕射耽冬雲南蠻數萬寇邊突破清溪關犯大渡河遂進陷沈黎突卭崍直雅卭按解圍録新舊紀皆在十年冬而獨以為十一年冬誤也新傳曰十年乃入寇以兵綴清溪關密引衆伐木開道徑雪坡盛夏卒凍死者二千人出沐源闚嘉州按蠻以十一月至沐源川非盛夏新傳誤也實録又曰驃信以十月三日離善闡每人止將米炒一斗隨身乃詔高駢乘其國内無兵備進攻善闡以解衝突按駢時為鄆州節度使不在安南恐實録誤也】詔左神武將軍顔慶復將兵赴援<br />
<br />
  資治通鑑卷二百五十一<br />
<br />
<史部,編年類,資治通鑑>  <br>
   </div> 

<script src="/search/ajaxskft.js"> </script>
 <div class="clear"></div>
<br>
<br>
 <!-- a.d-->

 <!--
<div class="info_share">
</div> 
-->
 <!--info_share--></div>   <!-- end info_content-->
  </div> <!-- end l-->

<div class="r">   <!--r-->



<div class="sidebar"  style="margin-bottom:2px;">

 
<div class="sidebar_title">工具类大全</div>
<div class="sidebar_info">
<strong><a href="http://www.guoxuedashi.com/lsditu/" target="_blank">历史地图</a></strong>  
<a href="http://www.880114.com/" target="_blank">英语宝典</a>  
<a href="http://www.guoxuedashi.com/13jing/" target="_blank">十三经检索</a> 
<br><strong><a href="http://www.guoxuedashi.com/gjtsjc/" target="_blank">古今图书集成</a></strong> 
<a href="http://www.guoxuedashi.com/duilian/" target="_blank">对联大全</a> <strong><a href="http://www.guoxuedashi.com/xiangxingzi/" target="_blank">象形文字典</a></strong> 

<br><a href="http://www.guoxuedashi.com/zixing/yanbian/">字形演变</a>  <strong><a href="http://www.guoxuemi.com/hafo/" target="_blank">哈佛燕京中文善本特藏</a></strong>
<br><strong><a href="http://www.guoxuedashi.com/csfz/" target="_blank">丛书&方志检索器</a></strong> <a href="http://www.guoxuedashi.com/yqjyy/" target="_blank">一切经音义</a>  

<br><strong><a href="http://www.guoxuedashi.com/jiapu/" target="_blank">家谱族谱查询</a></strong>  <strong><a href="http://shufa.guoxuedashi.com/sfzitie/" target="_blank">书法字帖欣赏</a></strong> 
<br>

</div>
</div>


<div class="sidebar" style="margin-bottom:0px;">

<font style="font-size:22px;line-height:32px">QQ交流群9:489193090</font>


<div class="sidebar_title">手机APP 扫描或点击</div>
<div class="sidebar_info">
<table>
<tr>
	<td width=160><a href="http://m.guoxuedashi.com/app/" target="_blank"><img src="/img/gxds-sj.png" width="140"  border="0" alt="国学大师手机版"></a></td>
	<td>
<a href="http://www.guoxuedashi.com/download/" target="_blank">app软件下载专区</a><br>
<a href="http://www.guoxuedashi.com/download/gxds.php" target="_blank">《国学大师》下载</a><br>
<a href="http://www.guoxuedashi.com/download/kxzd.php" target="_blank">《汉字宝典》下载</a><br>
<a href="http://www.guoxuedashi.com/download/scqbd.php" target="_blank">《诗词曲宝典》下载</a><br>
<a href="http://www.guoxuedashi.com/SiKuQuanShu/skqs.php" target="_blank">《四库全书》下载</a><br>
</td>
</tr>
</table>

</div>
</div>


<div class="sidebar2">
<center>


</center>
</div>

<div class="sidebar"  style="margin-bottom:2px;">
<div class="sidebar_title">网站使用教程</div>
<div class="sidebar_info">
<a href="http://www.guoxuedashi.com/help/gjsearch.php" target="_blank">如何在国学大师网下载古籍?</a><br>
<a href="http://www.guoxuedashi.com/zidian/bujian/bjjc.php" target="_blank">如何使用部件查字法快速查字?</a><br>
<a href="http://www.guoxuedashi.com/search/sjc.php" target="_blank">如何在指定的书籍中全文检索?</a><br>
<a href="http://www.guoxuedashi.com/search/skjc.php" target="_blank">如何找到一句话在《四库全书》哪一页?</a><br>
</div>
</div>


<div class="sidebar">
<div class="sidebar_title">热门书籍</div>
<div class="sidebar_info">
<a href="/so.php?sokey=%E8%B5%84%E6%B2%BB%E9%80%9A%E9%89%B4&kt=1">资治通鉴</a> <a href="/24shi/"><strong>二十四史</strong></a>&nbsp; <a href="/a2694/">野史</a>&nbsp; <a href="/SiKuQuanShu/"><strong>四库全书</strong></a>&nbsp;<a href="http://www.guoxuedashi.com/SiKuQuanShu/fanti/">繁体</a>
<br><a href="/so.php?sokey=%E7%BA%A2%E6%A5%BC%E6%A2%A6&kt=1">红楼梦</a> <a href="/a/1858x/">三国演义</a> <a href="/a/1038k/">水浒传</a> <a href="/a/1046t/">西游记</a> <a href="/a/1914o/">封神演义</a>
<br>
<a href="http://www.guoxuedashi.com/so.php?sokeygx=%E4%B8%87%E6%9C%89%E6%96%87%E5%BA%93&submit=&kt=1">万有文库</a> <a href="/a/780t/">古文观止</a> <a href="/a/1024l/">文心雕龙</a> <a href="/a/1704n/">全唐诗</a> <a href="/a/1705h/">全宋词</a>
<br><a href="http://www.guoxuedashi.com/so.php?sokeygx=%E7%99%BE%E8%A1%B2%E6%9C%AC%E4%BA%8C%E5%8D%81%E5%9B%9B%E5%8F%B2&submit=&kt=1"><strong>百衲本二十四史</strong></a>  <a href="http://www.guoxuedashi.com/so.php?sokeygx=%E5%8F%A4%E4%BB%8A%E5%9B%BE%E4%B9%A6%E9%9B%86%E6%88%90&submit=&kt=1"><strong>古今图书集成</strong></a>
<br>

<a href="http://www.guoxuedashi.com/so.php?sokeygx=%E4%B8%9B%E4%B9%A6%E9%9B%86%E6%88%90&submit=&kt=1">丛书集成</a> 
<a href="http://www.guoxuedashi.com/so.php?sokeygx=%E5%9B%9B%E9%83%A8%E4%B8%9B%E5%88%8A&submit=&kt=1"><strong>四部丛刊</strong></a>  
<a href="http://www.guoxuedashi.com/so.php?sokeygx=%E8%AF%B4%E6%96%87%E8%A7%A3%E5%AD%97&submit=&kt=1">說文解字</a> <a href="http://www.guoxuedashi.com/so.php?sokeygx=%E5%85%A8%E4%B8%8A%E5%8F%A4&submit=&kt=1">三国六朝文</a>
<br><a href="http://www.guoxuedashi.com/so.php?sokeytm=%E6%97%A5%E6%9C%AC%E5%86%85%E9%98%81%E6%96%87%E5%BA%93&submit=&kt=1"><strong>日本内阁文库</strong></a> <a href="http://www.guoxuedashi.com/so.php?sokeytm=%E5%9B%BD%E5%9B%BE%E6%96%B9%E5%BF%97%E5%90%88%E9%9B%86&ka=100&submit=">国图方志合集</a> <a href="http://www.guoxuedashi.com/so.php?sokeytm=%E5%90%84%E5%9C%B0%E6%96%B9%E5%BF%97&submit=&kt=1"><strong>各地方志</strong></a>

</div>
</div>


<div class="sidebar2">
<center>

</center>
</div>
<div class="sidebar greenbar">
<div class="sidebar_title green">四库全书</div>
<div class="sidebar_info">

《四库全书》是中国古代最大的丛书,编撰于乾隆年间,由纪昀等360多位高官、学者编撰,3800多人抄写,费时十三年编成。丛书分经、史、子、集四部,故名四库。共有3500多种书,7.9万卷,3.6万册,约8亿字,基本上囊括了古代所有图书,故称“全书”。<a href="http://www.guoxuedashi.com/SiKuQuanShu/">详细>>
</a>

</div> 
</div>

</div>  <!--end r-->

</div>
<!-- 内容区END --> 

<!-- 页脚开始 -->
<div class="shh">

</div>

<div class="w1180" style="margin-top:8px;">
<center><script src="http://www.guoxuedashi.com/img/plus.php?id=3"></script></center>
</div>
<div class="w1180 foot">
<a href="/b/thanks.php">特别致谢</a> | <a href="javascript:window.external.AddFavorite(document.location.href,document.title);">收藏本站</a> | <a href="#">欢迎投稿</a> | <a href="http://www.guoxuedashi.com/forum/">意见建议</a> | <a href="http://www.guoxuemi.com/">国学迷</a> | <a href="http://www.shuowen.net/">说文网</a><script language="javascript" type="text/javascript" src="https://js.users.51.la/17753172.js"></script><br />
  Copyright &copy; 国学大师 古典图书集成 All Rights Reserved.<br>
  
  <span style="font-size:14px">免责声明:本站非营利性站点,以方便网友为主,仅供学习研究。<br>内容由热心网友提供和网上收集,不保留版权。若侵犯了您的权益,来信即刪。scp168@qq.com</span>
  <br />
ICP证:<a href="http://www.beian.miit.gov.cn/" target="_blank">鲁ICP备19060063号</a></div>
<!-- 页脚END --> 
<script src="http://www.guoxuedashi.com/img/plus.php?id=22"></script>
<script src="http://www.guoxuedashi.com/img/tongji.js"></script>

</body>
</html>
