






























































資治通鑑卷一百四十五 宋 司馬光 撰

胡三省 音註

梁紀一|{
	起玄黓敦牂盡閼逢涒灘凡三年 齊宣德太后詔蕭衍自建安郡公進爵梁公衍志也尋進爵為王尋受齊禪國因號曰梁}


高祖武皇帝|{
	諱衍字叔達小字練兒南蘭陵中都里人姓蕭氏與齊同出淮陰令整三世至順之順之於齊高帝為族弟帝順之之子也按通鑑武皇帝紀凡十八卷以一二為次此卷武皇帝之下合有一字}


天監元年|{
	自是年三月以前猶是齊和帝中興二年}
春正月齊和帝遣兼侍中席闡文等慰勞建康|{
	勞力到翻}
大司馬衍下令凡東昏時浮費自非可以習禮樂之容繕甲兵之備者餘皆禁絶 戊戌迎宣德太后入宫臨朝稱制衍解承制|{
	衍承制見上卷上年蹔解之以覘人心朝直遥翻}
己亥以寧朔將軍蕭昺監南兖州諸軍事昺衍之從父弟也|{
	昺兵永翻昺與帝同祖治書侍御史道賜監工衘翻從才用翻}
壬寅進大司馬衍都督中外諸軍事劒履上殿贊拜不名|{
	上時掌翻}
己酉以大司馬長史王亮為中書監尚書令 初大司馬與黃門侍郎范雲南清河太守沈約司徒右長史任昉同在竟陵王西邸|{
	事見一百三十六卷齊武帝永明二年守式又翻任音壬昉分兩翻}
意好敦密|{
	敦厚也好呼到翻}
至是引雲為大司馬諮議參軍領録事|{
	衍録尚書其録府事使雲領之}
約為驃騎司馬|{
	為衍驃騎大將軍府司馬驃匹妙翻騎奇寄翻}
昉為記室參軍與參謀議前吳興太守謝朏國子祭酒何胤先皆棄官家居|{
	齊明帝建武初朏胤皆棄官去朏敷尾翻先悉薦翻}
衍奏徵為軍諮祭酒朏胤皆不至大司馬内有受禪之志沈約微扣其端大司馬不應他日又進曰今與古異不可以淳風期物|{
	淳風謂淳古之風也}
士大夫攀龍附鳳皆望有尺寸之功今兒童牧豎皆知齊祚已終明公當承其運天文䜟記又復炳然|{
	䜟楚譖翻}
天心不可違人情不可失苟歷數所在雖欲謙光亦不可得已|{
	易曰謙尊而光}
大司馬曰吾方思之約曰公初建牙樊沔此時應思|{
	沔彌兖翻}
今王業已成何所復思若不早定大業脫有一人立異即損威德且人非金石時事難保豈可以建安之封遺之子孫|{
	復扶又翻下無復豈復同遺唯季翻}
若天子還都公卿在位則君臣分定|{
	分扶問翻}
無復異心君明於上臣忠於下豈復有人方更同公作賊大司馬然之約出大司馬召范雲告之雲對略同約旨大司馬曰智者乃爾暗同卿明早將休文更來|{
	將攜也挾也領也休文沈約字也}
雲出語約約曰卿必待我雲許諾而約先期入|{
	語牛倨翻先悉薦翻}
大司馬命草具其事約乃出懷中詔書并諸選置大司馬初無所改俄而雲自外來至殿門不得入徘徊夀光閣外但云咄咄|{
	江南禁中有夀光省咄當没翻毛晃曰咄咄咨嗟語也}
約出問曰何以見處約舉手向左|{
	謂處之以尚書左僕射也處昌呂翻}
雲笑曰不乖所望有頃大司馬召雲入嘆約才智縱橫|{
	縱子容翻}
且曰我起兵於今三年矣|{
	東昏侯永元二年十一月衍起兵至是首尾三年}
功臣諸將實有其勞|{
	將即亮翻}
然成帝業者卿二人也甲寅詔進大司馬位相國總百揆揚州牧封十郡為梁公|{
	時以豫州之梁郡歷陽南徐州之義興揚州之淮南宣城吳興會稽新安東陽凡十郡為梁公國相息亮翻}
備九錫之禮置梁百司去録尚書之號|{
	去羌呂翻}
驃騎大將軍如故二月辛酉梁公始受命齊湘東王寶晊|{
	晊之日翻}
安陸昭王緬之子也|{
	緬齊明帝之弟緬彌兖翻}
頗好文學|{
	好呼到翻}
東昏侯死寶晊望物情歸已坐待法駕既而王珍國等送首梁公梁公以寶晊為太常寶晊心不自安壬戌梁公稱寶晊謀反并其弟江陵公寶覽汝南公寶宏皆殺之 丙寅詔梁國選諸要職悉依天朝之制|{
	朝直遥翻}
於是以沈約為吏部尚書兼右僕射范雲為侍中梁公納東昏余妃頗妨政事范雲以為言梁公未之從雲與侍中領軍將軍王茂同入見|{
	自沈約至王茂皆梁國官也見賢遍翻}
雲曰昔沛公入關婦女無所幸此范增所以畏其志大也|{
	事見九卷漢高帝元年}
今明公始定建康海内相望風聲奈何襲亂亡之迹以女德為累乎|{
	左傳富辰曰女德無極杜預注云婦女之志近之則不知止足累力瑞翻}
王茂起拜曰范雲言是也公必以天下為念無宜留此梁公默然雲即請以余氏賚王茂|{
	賚洛代翻}
梁公賢其意而許之明日賜雲茂錢各百萬丙戌詔梁公增封十郡進爵為王|{
	時以豫州之南譙廬江江州之尋陽郢州之武昌西陽南徐州之南琅邪南東海晉陵揚州之臨海永嘉十郡益梁國}
癸巳受命赦國内及府州殊死以下|{
	自進爵為王已上凡詔皆以宣德太后稱制行之}
辛丑殺齊邵陵王寶攸晉熙王寶嵩桂陽王寶貞|{
	南史齊紀作寶攸本傳作寶脩三王皆明帝之子}
梁王將殺齊諸王防守猶未急鄱陽王寶寅家閹人顔文智與左右麻拱等密謀穿牆夜出寶寅具小船於江㟁著烏布襦|{
	著則畧翻襦汝朱翻短衣也}
腰繋千餘錢濳赴江側躡屩徒步足無完膚|{
	屩居勺翻草履也}
防守者至明追之寶寅詐為釣者隨流上下十餘里追者不疑待散乃渡西㟁投民華文榮家|{
	待散待追者散也華戶化翻}
文榮與其族人天龍惠連棄家將寶寅遁匿山澗賃驢乘之晝伏夜行抵夀陽之東城魏戍主杜元倫馳告揚州刺史任城王澄以車馬侍衛迎之|{
	任音壬}
寶寅時年十六徒步憔悴|{
	悴奏醉翻}
見者以為掠賣生口澄待以客禮寶寅請喪君斬衰之服澄遣人曉示情禮以喪兄齊衰之服給之|{
	喪息浪翻衰倉回翻齊音咨}
澄帥官僚赴吊寶寅居處有禮一同極哀之節|{
	禮居君父之喪極哀帥讀曰率處昌呂翻}
夀陽多其義故皆受慰喭|{
	撫而安之曰慰弔生曰唁唁與喭同魚戰翻}
唯不見夏侯一族|{
	夏侯之族本譙郡譙人居於夀陽夏戶雅翻}
以夏侯詳從梁王故也澄深器重之|{
	為蕭寶寅貴顯於魏而不終張本}
齊和帝東歸|{
	將東歸建康也}
以蕭憺為都督荆湘等六州諸軍事荆州刺史|{
	憺徒敢翻又徒濫翻}
荆州軍旅之後公私空乏憺厲精為治|{
	治直吏翻}
廣屯田省力役存問兵死之家供其乏困自以少年居重任|{
	少詩照翻}
謂佐吏曰政之不臧士君子所宜共惜吾今開懷卿其無隱於是人人得盡意民有訟者皆立前待符教决於俄頃曹無留事荆人大悦 齊和帝至姑孰丙辰下詔禪位於梁 丁巳廬陵王寶源卒|{
	非疾也寶源者齊明帝第五子}
魯陽蠻魯北鷰等起兵攻魏潁州|{
	魏置潁州於汝陰又潁川郡舊置潁州}
夏四月辛酉宣德太后令曰西詔至|{
	齊和帝雖已至姑孰其地猶在建康之西故曰西詔}
帝憲章前代|{
	憲章前代者以前代為法度也}
敬禪神器於梁明可臨軒|{
	明謂明旦也}
遣使恭授璽紱未亡人歸於别宫|{
	古者君薨其夫人在者自稱未亡人使疏吏翻璽斯氏翻紱音弗}
壬戍策遣兼太保尚書令亮等奉皇帝璽紱詣梁宫|{
	亮王亮也}
丙寅梁王即皇帝位於南郊大赦改元|{
	始改元天監}
是日追贈兄懿為丞相封長沙王諡曰宣武葬禮依晉安平獻王故事|{
	懿為東昏侯所殺葬不成禮今依晉葬安平王孚禮葬之}
丁卯奉和帝為巴陵王宫於姑孰優崇之禮皆倣齊初|{
	倣齊奉汝陰王之禮}
奉宣德太后為齊文帝妃王皇后為巴陵王妃齊世王侯封爵悉從降省|{
	降者王降公公降侯省者除其封國省所梗翻}
唯宋汝陰王不在除例|{
	備三恪也}
追尊皇考為文皇帝廟號太祖皇妣為獻皇后|{
	考異曰南史云五月追尊今從梁書}
追諡妃郗氏曰德皇后|{
	東昏侯永元元年郗氏卒於襄陽郗丑之翻}
封文武功臣車騎將軍夏侯詳等十五人為公侯|{
	騎奇寄翻}
立皇弟中護軍宏為臨川王南徐州刺史秀為安成王雍州刺史偉為建安王|{
	雍於用翻}
左衛將軍恢為鄱陽王荆州刺史憺為始興王以宏為揚州刺史 丁卯以中書監王亮為尚書令相國左長史王瑩為中書監吏部尚書沈約為尚書僕射長兼侍中范雲為散騎常侍吏部尚書 詔凡後宫樂府西解暴室諸婦女一皆放遣|{
	解讀曰廨一皆放遣一切盡放遣之也}
戊辰巴陵王卒時上欲以南海郡為巴陵國徙王居之沈約曰古今殊事魏武所云不可慕虚名而受實禍|{
	沈約夢齊和帝劒斷其舌天之報應固不爽也}
上頷之乃遣所親鄭伯禽詣姑孰以生金進王王曰我死不須金醇酒足矣乃飲沈醉|{
	沈持林翻}
伯禽就摺殺之|{
	時年十五摺洛合翻}
王之鎮荆州也琅邪顔見遠為録事參軍及即位為治書侍御史兼中丞|{
	治直之翻}
既禪位見遠不食數日而卒|{
	史言齊臣以死殉和帝者僅一顔見遠}
上聞之曰我自應天從人|{
	曰從人者避皇考順之諱也}
何預天下士大夫事而顔見遠乃至於此|{
	此言不可以訓}
庚午詔有司依周漢故事議贖刑條格|{
	舜典曰金作贖刑注曰誤入而刑出金以贖罪周穆王訓夏贖刑亦以五刑之辟疑者罰贖至漢文帝令民入粟以贖罪武帝令死罪入贖錢五十萬減死一等盖自虞及周疑誤者贖漢則凡犯罪者皆可得而入贖}
凡在官身犯鞭杖之罪悉入贖停罰其臺省令史士卒欲贖者聽之 以謝沐縣公寶義為巴陵王奉齊祀|{
	上之受禪也寶義以晉安王降封謝沐縣公晉志謝沐縣屬臨賀郡沐食聿翻}
寶義幼有廢疾不能言故獨得全齊南康侯子恪及弟祁陽侯子範常因事入見|{
	子恪子範齊豫章王嶷子也祁陽縣吳立宋屬零陵郡見賢遍翻}
上從容謂曰|{
	從千容翻}
天下公器非可力取苟無期運雖項籍之力終亦敗亡宋孝武性猜忌兄弟粗有令名者皆鴆之|{
	謂南平王鑠也粗坐五翻}
朝臣以疑似枉死者相繼|{
	謂顔峻王僧達周朗沈懷文等朝直遥翻}
然或疑而不能去|{
	去羌呂翻下同}
或不疑而卒為患如卿祖以材畧見疑而無如之何|{
	此正指疑而不能去者謂齊高帝也卒子恤翻}
湘東以庸愚不疑而子孫皆死其手|{
	此正指不疑而卒為患者謂明帝盡殺孝武帝子孫也}
我於時已生彼豈知我應有今日固知有天命者非人所害我初平建康人皆勸我除去卿輩以壹物心我於時依而行之誰謂不可正以江左以來代謝之際必相屠滅感傷和氣所以國祚不長又齊梁雖云革命事異前世我與卿兄弟雖復絶服|{
	五服之親至於袒免則無服矣去羌呂翻復扶又翻下可復無復同}
宗屬未遠齊業之初亦共甘苦|{
	齊宋禪代之際帝父順之參預佐命}
情同一家豈可遽如行路之人卿兄弟果有天命非我所殺若無天命何忽行此適足示無度量耳且建武塗炭卿門|{
	謂齊明帝建武中誅高武子孫}
我起義兵非唯自雪門恥亦為卿兄弟報仇|{
	為于偽翻}
卿若能在建武永元之世|{
	永元齊東昏侯年號}
撥亂反正|{
	謂齊明帝父子為亂高武子孫為正}
我豈得不釋戈推奉邪我自取天下於明帝家非取之於卿家也昔劉子輿自稱成帝子光武言假使成帝更生天下亦不可復得况子輿乎|{
	事見三十九卷漢更始元年}
曹志魏武帝之孫為晉忠臣|{
	事見八十一卷晉武帝太康四年}
况卿今日猶是宗室我方坦然相期卿無復懷自外之意|{
	無當作毋}
小待當自知我寸心子恪兄弟凡十六人皆仕梁子恪子範子質子顯子雲子暉並以才能知名歷官清顯各以夀終|{
	史言帝所誅夷者齊明帝之後高帝之後固無恙也}
詔徵謝朏為左光禄大夫開府儀同三司|{
	朏敷尾翻}
何胤為右光禄大夫何點為侍中胤點終不就 癸酉詔公車府謗木肺石傍各置一函|{
	周禮大司寇以肺石達窮民注云肺石赤石也肺芳廢翻}
若肉食莫言欲有橫議投謗木函|{
	杜佑曰肉食在位者布衣處士而議朝政謂之橫議横戶孟翻}
若以功勞才器寃沈莫達投肺石函|{
	沈持林翻}
上身服浣濯之衣常膳唯以菜蔬每簡長吏務選廉平皆召見於前勗以政道|{
	見賢遍翻勗許玉翻勉也}
擢尚書殿中郎到溉為建安内史左戶侍郎劉鬷為晉安太守|{
	杜佑曰宋齊度支尚書統度支左戶右戶金部庫部六曹鬷子公翻沈約曰建安本閩越秦立為閩中郡漢武帝滅閩越徙其民於江淮間虛其地後有遁逃山谷間者頗出立為冶縣屬會稽司馬彪云章安是故冶然則臨海亦冶地也從分冶地為會稽東南二部都尉東部臨海是也南部建安是也吳孫休永安三年分南部立為建安郡晉武帝太康三年分建安立晉安郡詳考沈志建安郡則今南劒邵武建寧之地晉安郡則今福州之地沈志洪氏隸釋辯之甚詳注已見前}
二人皆以廉潔著稱溉彦之曾孫也|{
	到彦之宋文帝將}
又著令小縣令有能遷大縣大縣有能遷二千石以山陰令丘仲孚為長沙内史武康令東海何遠為宣城太守由是廉能莫不知勸 魯陽蠻圍魏湖陽|{
	湖陽縣漢屬南陽郡晉省元魏後於此置西淮安郡及南襄州隋為湖陽縣唐并湖陽入棗陽縣}
撫軍將軍李崇將兵擊破之|{
	將即亮翻}
斬魯北鷰徙萬餘戶於幽并諸州及六鎮尋叛南走所在追討比及河殺之皆盡|{
	比必利翻}
閏月丁巳魏頓丘匡公穆亮卒|{
	諡法真心大度曰匡}
齊東昏侯嬖臣孫文明等雖經赦令猶不自安五月乙亥夜帥其徒數百人因運荻炬束仗入南北掖門作亂|{
	荻炬者束荻為火炬用也因運此遂束兵仗於荻中以入嬖卑義翻又博計翻帥讀曰率}
燒神虎門總章觀入衛尉府殺衛尉洮陽愍侯張弘策|{
	觀古玩翻洮陽縣屬零陵郡洮音兆}
前軍司馬呂僧珍直殿内以宿衛兵拒之不能却上戎服御前殿曰賊夜來是其衆少曉則走矣|{
	少詩沼翻}
命擊五鼓領軍將軍王茂驍騎將軍張惠紹聞難引兵赴救盗乃散走討捕悉誅之|{
	擊五鼓晉檀祇破司馬國璠之故智也驍堅堯翻騎奇寄翻難乃旦翻}
江州刺史陳伯之目不識書得文牒辭訟唯作大諾而已有事典籖傳口語與奪决於主者|{
	伯之手不能書典籖傳其口之所言}
豫章人鄧繕永興人戴永忠|{
	漢會稽諸暨縣吳更名永興}
有舊恩於伯之伯之以繕為别駕永忠為記室參軍河南禇緭居建康|{
	緭于貴翻 考異曰蕭寶寅傳作褚胃今從梁書}
素薄行仕宦不得志頻造尚書范雲雲不禮之|{
	行下孟翻造七到翻范雲時為吏部尚書}
緭怒私謂所親曰建武以後草澤下族悉化成䝿人吾何罪而見棄今天下草創饑饉不已喪亂未可知|{
	喪息浪翻}
陳伯之擁彊兵在江州非主上舊臣有自疑之意且熒惑守南斗詎非為我出邪|{
	晉天文志將有天子之事占於南斗南斗六星天廟也主兵為于偽翻}
今者一行事若無成入魏不失作河南郡守|{
	守式又翻}
遂投伯之大見親狎伯之又以鄉人朱龍符為長流參軍|{
	陳伯之濟陰人職官分紀長流參軍主禁防晉從公府有長流參軍小府無長流參軍置禁防參軍顔氏家訓或問何故名治獄參軍為長流答曰帝王世紀云帝少昊崩其神降於長流之山此事本出山海經於祀主秋按周禮秋官司寇主刑罰長流之職漢魏捕賊掾耳晉宋以來始為參軍上屬司寇故取秋帝所居為嘉名焉}
並乘伯之愚闇恣為姦利上聞之使陳虎牙私戒伯之又遣人代鄧繕為别駕伯之並不受命表云龍符驍勇|{
	驍堅堯翻}
鄧繕有績效臺所遣别駕請以為治中繕於是日夜說伯之云臺家府藏空竭復無器仗三倉無米東境饑流|{
	三倉太倉石頭倉及常平倉又按五代史志梁司農卿主農功倉廪統太倉等令又管左右中部三倉丞東境三吳會稽之地說式芮翻復扶又翻下若復同}
此萬世一時也機不可失緭永忠共贊成之伯之謂繕今啟卿若復不得即與卿共反上敕伯之以部内一郡處繕|{
	處昌呂翻}
於是伯之集府州僚佐謂曰奉齊建安王教帥江北義勇十萬已次六合|{
	齊建安王蕭寶寅也時奔魏據宋史六合山在烏江縣界五代志江都郡六合縣宋齊之秦郡尉氏縣也帥讀曰率下同}
見使以江州見力運糧速下|{
	見力之見賢遍翻}
我荷明帝厚恩誓死以報即命纂嚴使緭詐為蕭寶寅書以示僚佐於聽事前為壇㰱血共盟緭說伯之曰|{
	荷下可翻聽讀曰廳㰱色甲翻說式芮翻}
今舉大事宜引衆望長史程元冲不與人同心臨川内史王觀僧䖍之孫人身不惡可召為長史以代元冲|{
	觀古玩翻}
伯之從之仍以緭為尋陽太守永忠為輔義將軍龍符為豫州刺史觀不應命豫章太守鄭伯倫起郡兵拒守程元冲既失職於家合帥數百人|{
	合衆而帥之以攻伯之}
乘伯之無備突入至聽事前|{
	聽讀與聽同}
伯之自出格鬭元冲不勝逃入廬山|{
	廬山在江州南}
伯之密遣信報虎牙兄弟皆逃奔盱眙|{
	盱眙音吁怡}
戊子詔以領軍將軍王茂為征南將軍江州刺史帥衆討之 魏揚州小峴戍主党灋宗|{
	党底朗翻姓也杜佑通典德浪翻峴戶典翻下同}
襲大峴戍破之虜龍驤將軍邾菩薩|{
	驤思將翻菩薄乎翻薩桑葛翻}
陳伯之聞王茂來謂褚緭等曰王觀既不受命鄭伯倫又不肯從便應空手受困今先平豫章開通南路多發丁力益運資糧然後席卷北向以撲饑疲之衆不憂不濟|{
	卷讀曰捲北向謂北下攻建康也撲普木翻}
六月留鄉人唐盖人守城|{
	守尋陽城}
引兵趣豫章攻伯倫不能下|{
	趣七喻翻}
王茂軍至伯之表裏受敵遂敗走間道渡江與虎牙等及褚緭俱奔魏|{
	間古莧翻}
上遣左右陳建孫送劉季連子弟三人入蜀使諭旨慰勞|{
	勞力到翻}
季連受命飭還装益州刺史鄧元起始得之官初季連為南郡太守不禮於元起|{
	鄧元起南郡當陽人}
都録朱道琛有罪|{
	都録盖郡之首吏總録諸吏者也琛丑林翻}
季連欲殺之逃匿得免至是道琛為元起典籖說元起曰|{
	說式芮翻下或說同}
益州亂離已久公私虛耗劉益州臨歸豈辦遠遣迎候道琛請先使檢校|{
	使疏吏翻}
緣路奉迎不然萬里資糧未易可得元起許之道琛既至言語不恭又歷造府州人士見器物輒奪之有不獲者語曰|{
	易以豉翻造七到翻語牛倨翻}
會當屬人何須苦惜於是軍府大懼謂元起必誅季連禍及黨與競言之於季連季連亦以為然且懼昔之不禮於元起乃召兵算之有精甲十萬歎曰據天險之地握此彊兵進可以匡社稷退不失作劉備捨此安之遂召佐史矯稱齊宣德太后令聚兵復反收朱道琛殺之|{
	史言劉季連阻兵釁起於朱道琛}
召巴西太守朱士畧及涪令李膺並不受命|{
	涪音浮}
是月元起至巴西士畧開門納之先是蜀民多逃亡聞元起至爭出投附皆稱起義兵應朝廷軍士新故三萬餘人|{
	新謂蜀民新附者故謂元起從行者先式薦翻}
元起在道久糧食乏絶或說之曰蜀土政慢民多詐疾若檢巴西一郡籍注因而罰之所獲必厚|{
	謂民多詐疾注之於籍以避征役說輸芮翻}
元起然之李膺諫曰使君前有嚴敵後無繼援山民始附於我觀德|{
	言山民觀望我德則附否則攜貳使疏吏翻}
若糾以刻薄民必不堪衆心一離雖悔無及何必起疾可以濟師|{
	起疾謂糾之以刻薄民所不堪則是興長病端一曰起疾謂起詐疾者杜預曰濟益也}
膺請出圖之不患資糧不足也元起曰善一以委卿膺退帥富民上軍資米|{
	帥讀曰率上時掌翻}
得三萬斛 秋八月丁未命尚書刪定郎濟陽蔡灋度損益王植之集註舊律|{
	王植之集定張杜律見一百三十七卷齊武帝永明九年濟子禮翻}
為梁律仍命與尚書令王亮侍中王瑩尚書僕射沈約吏部尚書范雲等九人同議定 上素善鍾律欲釐正雅樂乃自制四器名之為通|{
	五代史志通受聲廣九寸宣聲長九尺臨岳高一寸二分每通皆施三絃一曰玄英通二曰青陽通三曰朱明通四曰白藏通}
每通施三絃黃鍾絃用二百七十絲長九尺應鍾絃用一百四十二絲長四尺七寸四分差彊中間十律以是為差|{
	黃鍾律長九寸引而伸之為九尺應鍾律長四寸二十七分寸之二十引而伸之為四尺七寸四分差彊中間十律以是為差者即上生下生三分益一三分去一之數也長直亮翻下同}
因以通聲轉推月氣悉無差違而還得相中又制十二笛黃鍾笛長三尺八寸應鍾笛長二尺三寸中間十律以是為差以寫通聲飲古鍾玉律並皆不差|{
	樂有飲聲飲者隨其聲而酌其清濁高下也鄭譯因琵琶七調以其所捻琵琶絃柱相飲為七均合成十二以應十二律是也}
於是被以八音|{
	八音金石絲竹匏土革木被皮義翻}
施以七聲|{
	七聲宮商角徵羽及變宫變徵}
莫不和韻先是宫懸止有四鎛鍾雜以編鍾編磬衡鍾凡十六虡|{
	古者天子宫懸周禮注云宫懸四面四面象宫室有牆故謂之宫懸凡鍾十六枚同在於虡謂之編鍾特懸者謂之鎛鍾爾雅曰大鍾謂之鎛編磬亦十六枚而同虡先悉薦翻鏄補各翻虡其呂翻}
上始命設十二鏄鍾各有編鍾編磬凡三十六虡而去衡鍾四隅植建鼔|{
	建鼔大鼔也少昊氏作之為建鼔之節去羌呂翻}
魏高祖之喪前太傅平陽公丕自晉陽來赴|{
	此太和二十三年事}
遂留洛陽丕年八十餘歷事六世|{
	丕拓跋翳槐之曾孫從世祖臨江歷景穆文成獻文孝文及今主凡六世}
位極公輔而還為庶人|{
	丕得罪見一百四十一卷齊明帝建武四年}
魏主以其宗室耆舊矜而禮之乙卯以丕為三老 魏揚州刺史任城王澄表請攻鍾離魏主使羽林監敦煌范紹詣夀陽共量進止澄曰當用兵十萬往來百日乞朝廷速辦糧仗紹曰今秋已向末方欲調發|{
	任音壬敦徒門翻量音良調徒弔翻}
兵仗可集糧何由致有兵無糧何以克敵澄沈思良久曰實如卿言乃止|{
	沈持林翻}
九月丁巳魏主如鄴冬十月庚子還至懷與宗室近侍射遠帝射三百五十餘步羣臣刻銘以美之甲辰還洛陽 十一月己未立小廟以祭太祖之母|{
	太祖之母帝祖母也}
每祭太廟畢以一太牢祭之甲子立皇子統為太子 魏洛陽宫室始成|{
	齊武帝永明十一年魏始營洛陽至是宫室乃成}
十二月將軍張之侵魏淮南取木陵戍魏任城王澄遣輔國將軍成興擊之之敗走魏復取木陵|{
	水經注木陵山在黃水西南有木陵關黃水東逕晉西陽城南又東逕南光城南又東逕弋陽郡東又東北入於淮謂之黃口唐志木陵關在光州光山縣南黃州麻城縣東北復扶又翻}
劉季連遣其將李奉伯等拒鄧元起|{
	將即亮翻}
元起與戰互有勝負久之奉伯等敗還成都元起進屯西平|{
	晉安帝以秦雍流民立懷寧郡宋文帝元嘉十六年寄治成都其屬縣有西平盖亦寄治成都城外遂為實土}
季連驅略居民閉城固守元起進屯蔣橋去成都二十里留輜重於郫奉伯等間道襲郫陷之|{
	重直用翻郫音疲間古莧翻}
軍備盡没元起捨郫徑圍州城城局參軍江希之謀以城降不克而死|{
	宋有十八曹參軍城局其一也降戶江翻}
魏陳留公主寡居僕射高肇秦州刺史張彛皆欲尚之公主許彛而不許肇肇怒譖彛於魏主坐沈廢累年|{
	沈持林翻}
是歲江東大旱米斗五千民多餓死

二年春正月乙卯以尚書僕射沈約為左僕射吏部尚書范雲為右僕射尚書令王亮為左光禄大夫丙辰亮坐正旦詐疾不登殿削爵廢為庶人 乙亥魏主耕籍田 魏梁州氐楊會叛行梁州事楊椿等討之|{
	齊東昏永元二年書魏梁州刺史楊椿招降氐王楊集始今乃為行梁州事當考}
成都城中食盡升米三千人相食劉季連食粥累月計無所出上遣主書趙景悦宣詔受季連降|{
	降戶江翻下同}
季連肉袒請罪鄧元起遷季連於城外俄而造焉待之以禮|{
	造士到翻}
季連謝曰早知如此豈有前日之事|{
	盖言前日所以阻兵拒命實為朱道琛搆間也}
郫城亦降元起誅李奉伯等送季連詣建康初元起在道懼事不集無以為賞士之至者皆許以辟命於是受别駕治中檄者將二千人季連至建康入東掖門數步一稽顙以至上前|{
	稽音啟}
上笑曰卿欲慕劉備而曾不及公孫述|{
	謂公孫述不肯降漢也}
豈無卧龍之臣邪|{
	卧龍謂諸葛孔明}
赦為庶人三月己巳魏皇后蠶於北郊 庚辰魏揚州刺史任城王澄遣長風戍主奇道顯入寇|{
	姓譜奇姓伯奇之後}
取陰山白藁二戍|{
	據水經注陰山關在弋陽縣西南唐志黃州麻城縣東北有陰山關}
蕭寶寅伏於魏闕之下|{
	此魏朝之闕門也闕即古之象魏}
請兵伐梁雖暴風大雨終不蹔移會陳伯之降魏亦請兵自効魏主乃引八坐門下入定議|{
	八坐謂令僕及諸曹尚書門下謂侍中散騎常侍等官蹔與暫同降戶江翻坐徂卧翻}
夏四月癸未朔以寶寅為都督東揚等三州諸軍事鎮東將軍揚州刺史丹陽公齊王禮賜甚厚配兵一萬令屯東城|{
	此盖漢晉之東城縣地以其地在夀陽之東故置東揚州}
以伯之為都督淮南諸軍事平南將軍江州刺史屯陽石|{
	即羊石城也在廬江西北霍丘東南}
俟秋冬大舉寶寅明當拜命|{
	明謂明旦也}
其夜慟哭至晨魏人又聽寶寅募四方壯勇得數千人以顔文智華文榮等六人皆為將軍軍主|{
	顔文智將寶寅投華文榮文榮等將寶寅投北華戶化翻}
寶寅志性雅重過朞猶絶酒肉|{
	禮為兄弟服朞喪}
慘形悴色蔬食麤衣未嘗嬉笑|{
	悴秦醉翻}
癸卯蔡灋度上梁律二十卷|{
	上時掌翻}
令三十卷科四十卷詔班行之 五月丁巳霄城文侯范雲卒|{
	霄城縣侯也五代志沔陽郡竟陵縣舊曰霄城卒子恤翻下同}
雲盡心事上知無不為臨繁處劇|{
	處昌呂翻}
精力過人及卒衆謂沈約宜當樞管|{
	樞管謂管樞機也今人猶謂樞密院為樞管以此觀之沈約之位雖在范雲之右而親任不及雲遠矣}
上以約輕易|{
	易以豉翻}
不如尚書左丞徐勉乃以勉及右衛將軍周捨同參國政捨雅量不及勉而清簡過之兩人俱稱賢相常留省内罕得休下|{
	休下謂休假下直也}
勉或時還宅羣犬驚吠每有表奏輒焚其藁捨豫機密二十餘年未嘗離左右|{
	離力智翻}
國史詔誥儀體灋律軍旅謀謨皆掌之與人言謔終日不絶|{
	謔迄却翻戲言也}
而竟不漏泄機事衆尤服之|{
	史究言二人終身大槩}
壬申斷諸郡縣獻奉二宫惟諸州及會稽許貢任土若非地產亦不得貢|{
	斷音短下頻斷同二宫上宫及東宫也會稽東土大郡也故使之同於諸州會工外翻}
甲戌魏楊椿等大破叛氐斬首數千級|{
	是年春氐楊會叛}
六月壬午朔魏立皇弟悦為汝南王 魏揚州刺史任城王澄表稱蕭衍頻斷東關|{
	斷音短}
欲令漅湖汎溢以灌淮南諸戍吳楚便水且灌且掠淮南之地將非國有夀陽去江五百餘里衆庶惶惶並懼水害脫乘民之願攻敵之虚豫勒諸州纂集士馬首秋大集應機經略雖混壹不能必果江西自是無虞矣丙戌魏發冀定瀛相并濟六州二萬人馬一千五百匹|{
	相息亮翻濟子禮翻}
令仲秋之中畢會淮南并夀陽先兵三萬|{
	先兵先屯夀陽之兵}
委澄經略蕭寶寅陳伯之皆受澄節度 謝朏輕舟出詣闕|{
	朏敷尾翻}
詔以為侍中司徒尚書令朏辭脚疾不堪拜謁角巾自輿詣雲龍門謝詔見於華林園|{
	見賢遍翻}
乘小車就席明旦上幸朏宅|{
	謝朏仕宋及齊有宅在建康}
宴語盡懽朏固陳本志不許因請自還東迎母許之臨上復臨幸|{
	復扶又翻}
賦詩餞别王人送迎相望於道|{
	凡將上命者皆謂之王人}
及還詔起府於舊宅禮遇優異朏素憚煩不省職事衆頗失望|{
	謝朏之於樊英又不及遠甚省悉景翻}
甲午以中書監王瑩為尚書右僕射 秋七月乙卯魏平陽平公丕卒 魏既罷鹽池之禁|{
	魏主踐阼之初中尉甄琛表弛鹽禁彭城王勰與邢巒以為不可魏主詔從琛請通鑑目録已提其要此事今載於一百四十三卷齊東昏永元二年而通鑑正文逸其事錯簡置於百四十六卷天監五年}
而其利皆為富彊所專庚午復收鹽池利入公|{
	復扶又翻}
辛未魏以彭城王勰為太師|{
	勰音恊}
勰固辭魏主賜詔敦諭又為家人書祈請懇至|{
	為家人書用家人叔姪之禮也}
勰不得已受命 八月庚子魏以鎮南將軍元英都督征義陽諸軍事司州刺史蔡道恭聞魏軍將至遣驍騎將軍楊由帥城外居民三千餘家保賢首山為三柵|{
	驍堅堯翻騎奇寄翻帥讀曰率}
冬十月元英勒諸軍圍賢首柵柵民任馬駒斬由降魏|{
	任音壬降戶江翻}
任城王澄命統軍党灋宗傅豎眼太原王神念等分兵寇東關大峴淮陵九山|{
	淮陵恐當作雎陵齊置徐州於鍾離又僑置濟陰郡雎陵縣於郡界五代志鍾離郡化明縣舊曰睢陵置濟陰郡化明唐濠州之招義縣也或曰宋志南徐州領淮陵郡雎陵淮陵皆屬漢徐郡是時既置徐州於鍾離則亦置淮陵於鍾離界未可知也魏收志陳留鍾離二郡有朝歌縣縣有九山城黃溪水按水經注黃水出黃武山東北流逕南光城弋陽等郡今按今招信軍盱眙縣西南一十五里有三城又西十五里至淮陵城臨池河池河過淮陵城西而北入於淮謂之池河口九山店在淮北南直淮陵九山店之東則陷堈湖南則馬城淮流至此謂之九山灣其東則鳳皇洲在淮水中約長十里今土人亦呼九山灣為獅子渡北兵渡淮之津要也峴戶典翻}
高祖珍將三千騎為遊軍澄以大軍繼其後豎眼靈越之子也|{
	傳靈越從薛安都起兵攻張永以應義嘉兵潰而死豎而主翻}
魏人拔關要潁川大峴三城|{
	魏收志霍州有北潁川郡領潁川等三縣水經注梁立霍州治灊縣天柱山}
白塔牽城清溪皆潰徐州刺史司馬明素將兵三千救九山|{
	將即亮翻下同}
徐州長史潘伯鄰救淮陵寧朔將軍王燮保焦城党灋宗等進拔焦城破淮陵十一月壬午擒明素斬伯鄰先是南梁太守馮道根戍阜陵|{
	馮道根傳以南梁太守領阜陵戍先悉薦翻}
初到修城隍遠斥堠如敵將至衆頗笑之道根曰怯防勇戰此之謂也|{
	其周防若怯而臨戰則勇}
城未畢党灋宗等衆二萬奄至城下衆皆失色道根命大開門緩服登城選精銳二百人出與魏兵戰破之魏人見其意思閒暇|{
	思相吏翻}
戰又不利遂引去道根將百騎擊高祖珍破之|{
	騎奇寄翻}
魏諸軍糧運絶引退以道根為豫州刺史|{
	此時梁豫州治晉熙道根盖猶戍阜陵特帶刺史耳}
武興安王楊集始卒己未魏立其世子紹先為武興王紹先幼國事决於二叔父集起集義乙亥尚書左僕射沈約以母憂去職魏既遷洛陽北邊荒遠因以饑饉百姓困弊魏主加尚書左僕射源懷侍中行臺|{
	魏道武置行臺之官於鄴中山今置於北邊杜佑曰魏末司馬師討諸葛誕散騎常侍裴秀尚書僕射陳泰黃門侍郎鍾會等以行臺從北齊行臺兼統民事自辛術始隋謂之行臺省}
使持節巡行北邊六鎮恒燕朔三州|{
	六鎮列置於三州塞下使疏吏翻下同行下孟翻恒戶登翻燕因肩翻}
賑給貧乏考論殿最|{
	既使之賑恤貧民又使之按察官吏殿丁練翻}
事之得失皆先决後聞懷通濟有無饑民賴之沃野鎮將于祚|{
	沃野漢朔方郡之屬縣也魏平赫連與統萬同置鎮不在六鎮之數將即亮翻下同}
皇后之世父|{
	世父伯父承世嫡者}
與壞通婚時于勁方用事勢傾朝野|{
	朝直遥翻}
祚頗有受納懷將入鎮祚郊迎道左懷不與語即劾奏免官|{
	劾戶槩翻又戶得翻下同}
懷朔鎮將元尼須與懷舊交貪穢狼籍|{
	蘇鶚演義曰狼籍者物雜亂之貌狼所卧籍之草皆穢亂}
置酒請懷謂懷曰命之長短繫卿之口豈可不相寛貸懷曰今日源懷與故人飲酒之坐|{
	坐徂卧翻}
非鞫獄之所也明日公庭始為使者檢鎮將罪狀之處耳尼須揮淚無以對竟按劾抵罪懷又奏邊鎮事少而置官猥多|{
	少詩沼翻}
沃野一鎮自將以下八百餘人|{
	將謂鎮將也將即亮翻}
請一切五分損二魏主從之乙酉將軍吳子陽與魏元英戰於白沙子陽敗績|{
	白沙在齊安郡界魏收志有沙州治白沙關城注云梁置唐志黃州黃陂縣有白沙關}
魏東荆州蠻樊素安作亂乙酉以左衛將軍李崇為鎮南將軍都督征蠻諸軍事將步騎討之|{
	將即亮翻騎奇寄翻}
馮翊吉翂父為原鄉令|{
	翂撫文翻漢靈帝中平二年分故鄣立原鄉縣屬吳興郡}
為姦吏所誣逮詣廷尉罪當死翂年十五撾登聞鼔乞代父命|{
	撾則瓜翻}
上以其幼疑人教之使廷尉卿蔡灋度嚴加誘脅取其款實|{
	誘者開之以生路脅者威之以纆索杻械示將拷訊之款誠也誘音酉}
灋度盛陳拷訊之具詰翂曰爾求代父敕已相許審能死不|{
	所謂脅之也拷音考詰去吉翻不讀曰否}
且爾童騃若為人所教亦聽悔異|{
	騃五駭翻所謂誘之也悔異猶律文所謂飜異}
翂曰囚雖愚幼豈不知死之可憚顧不忍見父極刑故求代之此非細故奈何受人教邪明詔聽代不異登仙豈有回貳|{
	反前說為回異前說為貳}
灋度乃更和顔誘之曰主上知尊侯無罪行當得釋觀君足為佳童今若轉辭幸可父子同濟翂曰父掛深劾必正刑書囚瞑目引領唯聽大戮|{
	劾戶槩翻又戶得翻瞑莫定翻}
無言復對時翂備加杻械灋度愍之命更著小者|{
	復扶又翻杻女九翻更工衡翻著陟畧翻}
翂不聽曰死罪之囚唯宜益械豈可減乎竟不脱灋度具以聞上乃宥其父罪丹陽尹王志求其在廷尉事并問鄉里欲於歲首舉充純孝|{
	魏晉以來舉士皆由州鄉故問其鄉里}
翂曰異哉王尹何量翂之薄乎父辱子死道固當然若翂當此舉乃是因父取名何辱如之固拒而止|{
	量音良翂之拒王志是也梁武帝知翂之孝節而不能叙用以厲流俗非也}
魏主納高肇兄偃之女為貴嬪|{
	嬪毗賓翻}
魏散騎常侍趙脩寒賤暴䝿恃寵驕恣陵轢王公為

衆所疾|{
	散悉亶翻騎奇寄翻轢郎擊翻}
魏主為脩治第舍擬於諸王|{
	為于偽翻治直之翻}
鄰居獻地者或超補大郡脩請告歸葬其父凡財役所須並從官給脩在道淫縱|{
	脩自洛歸趙郡在道淫縱}
左右乘其出外頗發其罪惡及還舊寵小衰高肇密構成其罪侍中領御史中尉甄琛黄門郎李憑廷尉卿陽平王顯素皆諂附於脩至是懼相連及|{
	懼以黨附連坐及禍甄之人翻琛丑林翻}
爭助肇攻之帝命尚書元紹檢訊下詔暴其姦惡免死鞭一百徙敦煌為兵|{
	敦徒門翻}
而脩愚疎初不之知方在領軍于勁第樗蒱羽林數人稱詔呼之送詣領軍府甄琛王顯監罸先具問事有力者五人迭鞭之|{
	監工銜翻問事行杖者也}
欲令必死脩素肥壯堪忍楚毒密加鞭至三百不死即召驛馬促之上道出城不自勝|{
	上時掌翻勝音升}
舉縛置鞍中|{
	舉擎也脩困極不能自勝乘騎兩人對舉而置之馬上縛著鞍中}
急驅之行八十里乃死帝聞之責元紹不重聞|{
	重直用翻聞奏也}
紹曰脩之佞幸為國深蠧臣不因釁除之|{
	釁隙也釁許覲翻}
恐陛下受萬世之謗帝以其言正不罪也紹出廣平王懷拜之曰翁之直過於汲黯紹曰但恨戮之稍晩以為愧耳紹素之孫也|{
	常山玉素見一百二十二卷宋文帝元嘉十一年廣平王懷孝文之子以族屬長幼之次呼紹為翁}
明日甄琛李憑以脩黨皆坐免官左右與脩連坐死黜者二十餘人散騎常侍高聰與脩素親狎而又以宗人諂事高肇故獨得免

三年春正月庚戌征虜將軍趙祖悦與魏江州刺史陳伯之戰於東關祖悦敗績 癸丑以尚書右僕射王瑩為左僕射太子詹事柳惔為右僕射|{
	惔徒甘翻}
丙辰魏東荆州刺史楊大眼擊叛蠻樊季安等大破之季安素安之弟也 丙寅魏大赦改元正始 蕭寶寅行及汝陰東城已為梁所取乃屯夀陽棲賢寺二月戊子將軍姜慶真乘魏任城王澄在外|{
	去年魏遣澄入寇宿師於外}
襲夀陽據其外郭長史韋纘倉猝失圖任城太妃孟氏勒兵登陴先守要便|{
	敵所必攻我所必守曰要便者形勝可據便於制敵之處陴頻彌翻}
激厲文武安慰新舊|{
	新者夀陽兵民舊者北來將士或曰新者新附舊者舊民}
將士咸有奮志太妃親巡城守|{
	守手又翻}
不避矢石蕭寶寅引兵至與州軍合擊之自四皷戰至下晡|{
	日未入之前為下晡}
慶真敗走韋纘坐免官任城王澄攻鍾離上遣冠軍將軍張惠紹等將兵五千送糧詣鍾離|{
	冠古玩翻將即亮翻}
澄遣平遠將軍劉思祖等邀之丁酉戰於邵陽|{
	即邵陽州也}
大敗梁兵俘惠紹等十將殺虜士卒殆盡思祖芳之從子也|{
	劉芳以儒學親重於太和之間敗補邁翻將即亮翻從才用翻}
尚書論思祖功應封千戶侯侍中領右衛將軍元暉求二婢於思祖不得事遂寢|{
	史言魏賞罰失當}
暉素之孫也上遣平西將軍曹景宗後軍王僧炳等帥步騎三萬救義陽|{
	後軍者後軍將軍也帥讀曰率}
僧炳將二萬人據鑿峴|{
	鑿峴在關南今信陽軍南三十五里有曹店即景宗屯鑿峴口所築峴戶典翻}
景宗將萬人為後繼元英遣冠軍將軍元逞等據樊城以拒之三月壬申大破僧炳於樊城俘斬四千餘人|{
	僧炳敗於樊城未得至鑿峴也否則此非襄陽之樊城自别是一處}
魏詔任城王澄以四月淮水將漲舟行無礙南軍得時勿昧利以取後悔會大雨淮水暴漲澄引兵還夀陽魏軍還既狼狽失亡四千餘人中書侍郎齊郡賈思伯為澄軍司居後為殿|{
	殿丁練翻}
澄以其儒者謂之必死及至大喜曰仁者必有勇|{
	論語孔子之言}
於軍司見之矣思伯託以失道不伐其功有司奏奪澄開府仍降三階上以所獲魏將士請易張惠紹於魏魏人歸之 |{
	考異曰惠紹傳無被獲及復還事今從魏書}
魏太傅領司徒録尚書北海王詳驕奢好聲色貪冒無厭|{
	好呼到翻冒莫北翻厭於鹽翻}
廣營第舍奪人居室嬖昵左右所在請託中外嗟怨|{
	嬖卑義翻又博計翻昵尼質翻}
魏主以其尊親恩禮無替軍國大事皆與參决所奏請無不開允魏主之初親政也以兵召諸叔|{
	事見上卷齊和帝中興元年}
詳與咸陽彭城王共車而入防衛嚴固高太妃大懼乘車隨而哭之既得免謂詳曰自今不願富貴但使母子相保與汝掃市為生耳及詳再執政|{
	齊和帝中興元年正月魏主親政十一月詳為司徒}
太妃不復念前事|{
	復扶又翻}
專助詳為貪虐冠軍將軍茹皓以巧思有寵於帝|{
	茹音如思相吏翻}
常在左右傳可門下奏事弄權納賄朝野憚之詳亦附焉皓娶尚書令高肇從妹皓妻之姊為詳從父安定王燮之妃詳烝於燮妃由是與皓益相昵狎|{
	朝直遥翻下同從才用翻昵尼質翻}
直閣將軍劉胄本詳所引薦殿中將軍常季賢以善養馬陳掃靜掌櫛|{
	櫛側瑟翻梳也}
皆得幸於帝與皓相表裏賣權勢高肇本出高麗時望輕之|{
	麗力知翻}
帝既黜六輔|{
	魏高祖殂使六人受遺輔幼主事見一百四十二卷齊東昏侯永元元年}
誅咸陽王禧|{
	事見上卷齊和帝中興元年}
專委事於肇肇以在朝親族至少|{
	少詩沼翻}
乃邀結朋援附之者旬月超擢不附者陷以大辠尤忌諸王以詳位居其上欲去之獨執朝政|{
	去羌呂翻}
乃譛之於帝云詳與皓胄季賢掃靜謀為逆亂夏四月帝夜召中尉崔亮入禁中使彈奏詳貪淫奢縱及皓等四人怙權貪橫收皓等擊南臺|{
	横戶孟翻南臺御史臺也}
遣虎賁百人圍守詳第|{
	賁音奔}
又慮詳驚懼逃逸遣左右郭翼開金墉門馳出諭旨示以中尉彈狀詳曰審如中尉所糾何憂也正恐更有大罪橫至耳|{
	橫戶孟翻}
人與我物我實受之詰朝有司奏處皓等罪皆賜死|{
	詰去吉翻處昌呂翻}
帝引高陽王雍等五王入議詳罪詳單車防衛送華林園母妻隨入給小奴弱婢數人圍守甚嚴内外不通五月丁未朔下詔宥詳死免為庶人頃之徙詳於太府寺圍禁彌急母妻皆還南第五日一來視之初詳娶宋王劉昶女待之疎薄|{
	昶丑兩翻}
詳既被禁高太妃乃知安定高妃事大怒曰汝妻妾甚多如此安用彼高麗婢陷罪至此|{
	麗力知翻}
杖之百餘被創膿潰旬餘乃能立|{
	被皮義翻創初良翻}
又杖劉妃數十曰婦人皆妬何獨不妬劉妃笑而受罰卒無所言|{
	卒子恤翻下同}
詳家奴數人陰結黨輩欲劫出詳密書姓名託侍婢通於詳詳始得執省|{
	省悉景翻省猶視也}
而門防主司遥見突入就詳手中攬得奏之詳慟哭數聲暴卒詔有司以禮殯葬|{
	門防主司主門衛之兵以防守詳者}
先是典事史元顯獻雞雛四翼四足|{
	典事猶今尚書六部主事吏職也江南制局監有典事先悉薦翻}
詔以問侍中崔光光上表曰漢元帝初元中丞相府史家雌鷄伏子漸化為雄|{
	師古曰初尚伏子後乃稍稍化為雄也伏音房富翻}
冠距鳴將|{
	師古曰距雞附足骨鬭時所用刺之將謂帥領其羣也}
永光中有獻雄雞生角劉向以為雞者小畜主司時起居人|{
	畜許又翻師古曰至時而鳴以為人起居之節}
小臣執事為政之象也|{
	事見西漢書五行志}
竟寧元年石顯伏辜此其效也靈帝光和元年南宫寺雌雞欲化為雄但頭冠未變詔以問議郎蔡邕對曰頭為元首人君之象也今雞一身已變未至於頭而上知之是將有其事而不遂成之象也若應之不精政無所改頭冠或成為患滋大|{
	事見後漢書蔡邕傳}
是後黄巾破壞四方天下遂大亂今之雞狀雖與漢不同而其應頗相類誠可畏也臣以向邕言推之翼足衆多亦羣下相扇助之象雛而未大足羽差小亦其勢尚微易制御也|{
	易以豉翻}
臣聞災異之見皆所以示吉凶|{
	見賢遍翻}
明君覩之而懼乃能致福闇主覩之而慢所以致禍或者今亦有自賤而貴關預政事如前世石顯之比者邪願陛下進賢黜佞則妖弭慶集矣|{
	妖於遥翻}
後數日皓等伏誅帝愈重光|{
	魏主以茹皓等伏誅為光言之驗高肇獨非自賤而貴關預政事者邪}
高肇說帝使宿衛隊主帥羽林虎賁守諸王第殆同幽禁彭城王勰切諫不聽勰志尚高邁不樂榮勢避事家居而出無山水之適處無知己之遊獨對妻子常鬱鬱不樂|{
	說式芮翻帥所類翻樂音洛處昌呂翻}
魏人圍義陽城中兵不滿五千人食纔支半歲魏軍攻之晝夜不息刺史蔡道恭隨方抗禦皆應手摧却相持百餘日前後斬獲不可勝計|{
	魏自去年十月圍義陽蔡道恭卒於今年五月自此以上謂道恭疾未甚之前勝音升}
魏軍憚之將退會道恭疾篤乃呼從弟驍騎將軍靈恩|{
	從才用翻}
兄子尚書郎僧勰及諸將佐謂曰吾受國厚恩不能攘滅寇賊今所苦轉篤勢不支久汝等當以死固節無令吾没有遺恨衆皆流涕道恭卒靈恩攝行州事代之城守|{
	守式又翻}
六月癸未大赦 魏大旱散騎常侍兼尚書邢巒奏稱昔者明王重粟帛輕金玉何則粟帛養民而安國金玉無用而敗德故也|{
	散悉亶翻騎奇寄翻敗補邁翻}
先帝深鑒奢泰務崇節儉至以紙絹為帳扆|{
	扆於豈翻禮疏曰扆屏風}
銅鐵為轡勒府藏之金裁給而已|{
	藏徂浪翻}
不復買積以費國資|{
	復扶又翻}
逮景明之初承升平之業四境清宴遠邇來同於是貢篚相繼|{
	貢篚二語本之禹貢謂貴細之物盛之以篚筐而入貢也}
商估交入諸所獻納倍多於常金玉恒有餘國用恒不足|{
	估音古恒戶登翻}
苟非為之分限|{
	分扶問翻}
但恐歲計不充自今請非要須者一切不受魏主納之 秋七月癸丑角城戍主柴慶宗以城降魏|{
	降戶江翻}
魏徐州刺史元鑒遣淮陽太守吳秦生將千餘人赴之淮陰援軍斷其路|{
	守式又翻將即亮翻淮陰梁重鎮也以角城叛遣軍援其不從叛者斷音短}
秦生屢戰破之遂取角城 甲子立皇子綜為豫章王 魏李崇破東荆叛蠻生擒樊素安進討西荆諸蠻悉降之|{
	西荆正指荆州也魏太和中徙荆州治穰城領南陽順陽新野東恒農漢廣襄城北清恒農等郡其地正在東荆州之西}
魏人聞蔡道恭卒攻義陽益急短兵日接曹景宗頓鑿峴不進但耀兵遊獵而已上復遣寧朔將軍馬仙琕救義陽|{
	復扶又翻下箭復琕復乃復同琕部田翻}
仙琕轉戰而前兵勢甚銳元英結壘於上雅山|{
	上雅山當作士雅山據水經注義陽之東有大木山即晉祖逖將家避難所居也逖字士雅後人因以之名山杜佑曰唐州桐柏縣有大木山晉祖逖為豫州刺史藏家屬於此山}
分命諸將伏於四山示之以弱仙琕乘勝直抵長圍掩英營英偽北以誘之|{
	誘音酉}
至平地縱兵擊之統軍傳永擐甲執槊單騎先入|{
	將即亮翻下同擐音宦槊色角翻}
唯軍主蔡三虎副之突陳橫過梁兵射永洞其左股|{
	陳讀曰陣射而亦翻}
永拔箭復入仙琕大敗一子戰死仙琕退走英謂永曰公傷矣且還營永曰昔漢高捫足不欲人知|{
	事見十卷漢高祖四年}
下官雖微國家一將奈何使賊有傷將之名|{
	將即亮翻}
遂與諸軍追之盡夜而返時年七十餘矣軍中莫不壯之仙琕復帥萬餘人進擊英|{
	帥讀曰率}
英又破之殺將軍陳秀之仙琕知義陽危急盡銳决戰一日三交皆大敗而返|{
	馬仙琕力戰使曹景宗以大軍繼之魏必敗退義陽全矣}
蔡靈恩勢窮八月乙酉降於魏|{
	降戶江翻}
三關戍將聞之辛酉亦棄城走|{
	乙酉距辛酉三十六日太遠或者其辛卯歟}
英使司馬陸希道為露板嫌其不精命傅永改之永不增文彩直為之陳列軍事處置形要而已|{
	為于偽翻處昌呂翻}
英深賞之曰觀此經算雖有金城湯池不能守矣|{
	史言英伐其功故深賞傅永能為之陳列}
初南安惠王以預穆泰之謀追奪爵邑|{
	穆泰事見一百四十卷齊明帝建武三年}
及英克義陽乃復立英為中山王御史中丞任昉奏彈曹景宗上以其功臣寢而不治|{
	任音壬昉甫兩翻彈徒丹翻治直之翻}
衛尉鄭紹叔忠於事上外所聞知纎豪無隱每為上言事|{
	為于偽翻}
善則推功於上不善則引咎歸己上以是親之詔於南義陽置司州移鎮關南以紹叔為刺史|{
	南義陽治鹿城關隋為黃州木蘭縣唐并木蘭入黃岡縣}
紹叔立城隍繕器械廣田積穀招集流散百姓安之魏置郢州於義陽以司馬悦為刺史|{
	魏收地形志郢州領安陽城陽汝南郡}
上遣馬仙琕築竹敦麻陽二城於三關南|{
	麻陽即今黃州麻城縣也 考異曰司馬悦傳作豫州刺史馬仙琕按仙琕於時未曾為豫州也}
司馬悦遣兵攻竹敦拔之 九月壬子以吐谷渾王伏連籌為西秦河二州刺史河南王|{
	吐從暾入聲谷音浴}
柔然侵魏之沃野及懷朔鎮|{
	漢沃野縣屬朔方郡後魏為鎮魏收志太和元年置偏城郡沃野縣屬焉此時鎮猶未廢也注已見前}
詔車騎大將軍源懷出行北邉|{
	騎奇寄翻行下孟翻}
指授方畧隨須徵發皆以便宜從事|{
	隨須者隨軍行之所須以為用者也}
懷至雲中柔然遁去懷以為用夏制夷莫如城郭還至恒代按視諸鎮左右要害之地可以築城置戍之處欲東西為九城及儲糧積仗之宜犬牙相救之勢凡五十八條表上之曰今定鼎成周去北遥遠代表諸國頗或外叛|{
	代表謂魏代都之塞外也諸國謂高車諸部夏戶雅翻恒戶登翻上時掌翻}
仍遭旱饑戎馬甲兵十分闕八謂宜準舊鎮東西相望令形勢相接築城置戍分兵要害勸農積粟警急之日隨便翦討彼遊騎之寇|{
	騎奇寄翻}
終不敢攻城亦不敢越城南出如此北方無憂矣魏主從之 魏太和之十六年高祖詔中書監高閭與給事中公孫崇考定雅樂|{
	見一百三十七卷齊武帝永明十一年}
久之未就會高祖殂高閭卒景明中崇為太樂令上所調金石及書|{
	卒子恤翻上時掌翻}
至是世宗始命八坐已下議之 冬十一月戊午魏詔營繕國學|{
	據目録是年置四門小學袁翻曰太和二十年敕立四門博士於四門置學按自周以上學惟以二或尚東或尚西或貴在國或貴在郊爰暨周室學盖有六師氏居内大學在國四小在郊大戴保傳篇云帝入東學尚親而貴仁帝入南學尚齒而貴信帝入西學尚賢而貴德帝入北學尚貴而尊爵帝入太學承師而問道周之五學於此彰彰}
時魏平寧日久學業大盛燕齊趙魏之間教授者不可勝數|{
	燕因肩翻勝音升}
弟子著録多者千餘人少者猶數百|{
	少詩沼翻}
州舉茂異郡貢孝廉每年逾衆 甲子除以金贖罪之科|{
	聽贖事見上元年}
十二月丙子魏詔殿中郎陳郡袁翻等議立律令彭城王勰等監之|{
	勰音協監工銜翻}
己亥魏主幸伊闕|{
	自南北分治人主出行所至通鑑皆曰如自此以後率書幸未曉義例所由變盖一時失於刊正也}
上雅好儒術|{
	好呼到翻}
以東晉宋齊雖開置國學不及十年輒廢之其存亦文具而已無講授之實|{
	晉元帝建武元年戴邈請建太學王敦蘇峻之難學校廢矣成帝咸康三年復立而儒術終不振穆帝永和八年殷浩以軍興罷太學生宋文帝元嘉十五年徵雷次宗開館教授而儒玄文史四學並立齊高帝建元四年置國子學生二百人隆昌建武之間已倚席而不講矣}


資治通鑑卷一百四十五
















































































































































