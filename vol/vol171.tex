\chapter{資治通鑑卷一百七十}
宋 司馬光 撰

胡三省 音註

陳紀五|{
	起玄黓執徐盡逢敦牂凡三年}


高宗宣皇帝上之下

太建四年春正月丙午以尚書僕射徐陵為左僕射|{
	尚書置二僕射分為左右若省一僕射則止稱僕射}
中書監王勱為右僕射|{
	勱莫敗翻}
己巳齊主祀南郊|{
	五代志後齊制圓丘方澤並三年一祭謂之禘祀其南北郊則歲一祀皆以正月上辛今書己巳以致齊之日為始也南郊為壇於國南廣輪三十六尺高九尺四面各一陛為三壝内壝去壇二十五步中壝外壝相去如内壝四面各通一門又為大營於外壝之外廣輪二百七十步營壍廣一丈深八尺四面各一門又為燎壇於中壝之外丙地廣輪二十七尺高一尺八寸四面各一陛祀所感帝靈威仰於壇以高祖神武皇帝配禮用四圭有邸幣各如方色其上帝及配帝各用騂特牲一}
庚午上享太廟 辛未齊主贈琅邪王儼為楚恭哀帝以慰太后心|{
	儼死見上卷上年}
又以儼妃李氏為楚帝后 二月癸酉周遣大將軍昌城公深聘於突厥|{
	厥九勿翻}
司賓李除小賓部賀遂禮聘於齊|{
	後周倣成周之制以建官司賓盖周官大行人之職小賓部其小行人之職歟杜佑曰後周秋官之屬有小賓部下大夫上士}
深護之子也 己卯齊以衛菩薩為太尉|{
	菩薄胡翻薩桑割翻 考異曰北齊書北史並同不知菩薩何人亦不言何官}
辛巳以并省吏部尚書高元海為尚書左僕射|{
	自元魏置諸道行臺各置令僕尚書等官齊神武破爾朱兆得晉陽建大丞相府而居之文宣受禪遂置尚書省}
乙酉封皇子叔卿為建安王 庚寅齊以尚書左僕射唐邕為尚書令侍中祖珽為左僕射|{
	珽它鼎翻}
初胡太后既幽於北宫|{
	事見上卷上年}
珽欲以陸令萱為太后為令萱言魏保太后故事|{
	保太后事見百二十卷宋帝元嘉二年為令之為于偽翻}
且謂人曰陸雖婦人然實䧺傑自女媧以來未之有也|{
	司馬貞曰女媧亦風姓有神聖之德代宓羲立號曰女希氏盖宓羲之後已經數世金木輪環周而復始也孫愐曰女媧古女后也媧古華翻}
令萱亦謂珽為國師國寶由是得僕射 三月癸卯朔日有食之 初周太祖為魏相立左右十二軍摠屬相府太祖殂皆受晉公護處分|{
	立十二軍事見百六十二卷梁簡文帝大寶元年敬帝太平元年周太祖殂十二軍受護處分始是年也世祖天嘉元年護歸政於周世宗實武成元年護猶摠軍旅次年護弑世宗立高祖改元保定政悉歸護事具百六十六卷至百六十八卷相息亮翻處昌呂翻分扶問翻}
凡所徵非護書不行護第屯兵侍衛盛於宫闕諸子僚屬皆貪殘恣橫|{
	橫戶孟翻}
士民患之周主深自晦匿無所關預人不測其淺深護問稍伯大夫庾季才曰比日天道何如|{
	後周稍伯蓋周官稍人之職周官稍人主為縣師令都鄙丘甸之政距王城三百里曰稍杜佑曰後周地官之屬有每方稍伯中大夫又每遂有小稍伯稍大夫皆下大夫又有小稍伯稍正上士中士庾季才明於天文故護問之稍所教翻比毗至翻}
季才對曰荷恩深厚敢不盡言頃上台有變|{
	荷下可反隋志曰三台六星兩兩而居起文昌列招揺三公之位也西近文昌二星謂之上台}
公宜歸政天子請老私門此則享期頤之壽|{
	曲禮百年曰期頤鄭玄曰期猶要也頤養也}
受旦奭之美|{
	周公旦召公奭}
子孫常為藩屏|{
	屏必郢翻}
不然非復所知|{
	復扶又翻}
護沈吟久之曰吾本志如此但辭未獲免耳|{
	沈持林翻沈吟者深味其言微於聲而不能自决之貌}
公既為王官可依朝例|{
	朝直遥翻下同}
無煩别參寡人也自是踈之衛公直帝之母弟也深昵於護及沌口之敗|{
	沌口敗事見上卷臨海王光大二年昵尼質翻沌柱兖翻}
坐免官由是怨護勸帝誅之冀得其位帝乃密與直及右宫伯中大夫宇文神舉内史下大夫太原王軌右侍上士宇文孝伯謀之|{
	周官宫伯掌王宫宿衛次舍之職事内史掌詔王爵祿廢置殺生予奪之法命諸侯孤卿大夫則策命之凡四方之事書内史讀之後周盖髣髴其意以置官至隋諱忠字以中書為内史其位任尤重左右侍亦倣周官侍御以置官而剙其名五代志周置左右宫伯掌侍衛之禁各更直於内小宫伯貮之臨朝則在前侍之首行則夾路車左右中侍掌御寢之禁左右侍陪中侍之後左右前侍掌御寢南門之左右左右後侍掌寢北門之左右杜佑曰周制宫伯中大夫属天官内史属春官有中大夫下大夫}
神舉顯和之子孝伯安化公深之子也|{
	安化公爵以别護子深安化郡唐之慶州}
帝每於禁中見護常行家人禮|{
	以兄弟齒}
太后賜護坐帝立侍於㫄丙辰護自同州還長安帝御文安殿見之因引護入含仁殿謁太后且謂之曰太后春秋高頗好飲酒|{
	好呼到反}
雖屢諫未蒙垂納兄今入朝願更啓請因出懷中酒誥授之|{
	周成王作酒誥戒毋彛酒毋敢崇飲}
曰以此諫太后護既入如帝所戒讀酒誥未畢帝以玉珽自後擊之|{
	記天子搢珽鄭玄曰珽亦笏也珽之言珽然無所屈也或謂之大圭長三尺杼上終葵首終葵首者於杼上又廣其首如椎頭隋志今制珽長尺二寸方而不折以球玉為之珽他頂翻}
護踣於地|{
	踣蒲北翻}
帝令宦者何泉以御刀斫之泉惶懼斫不能傷衛公直匿於戶内躍出斬之時神舉等皆在外更無知者|{
	史言周主勇决}
帝召宫伯長孫覽等以護已誅|{
	長知兩翻}
令收護子柱國譚公會大將軍莒公至|{
	譚莒古國名}
崇業公静正平公乾嘉|{
	崇業正平皆郡公按隋書帝紀随州有崇業郡而志不載五代志絳郡正平縣舊置正平郡}
及其弟乾基乾光乾蔚乾祖乾威|{
	蔚紆勿翻}
并柱國北地侯龍恩龍恩弟大將軍萬壽|{
	侯姓也}
大將軍劉勇中外府司錄尹公正袁傑|{
	護都督中外故置中外府其属有長史司馬司錄}
膳部下大夫李安等於殿中殺之覽稚之孫也|{
	長孫稚著功名於正光永安之間}
初護既殺趙貴等|{
	護殺貴等事見百六十七卷高祖永定元年}
侯龍恩為護所親其從弟開府儀同三司植謂龍恩曰|{
	從才用翻}
主上春秋既富安危繋於數公若多所誅戮以自立威權豈唯社稷有累卵之危恐吾宗亦緣此而敗兄安得知而不言龍恩不能從植又承間言於護曰|{
	間古莧翻}
公以骨肉之親當社稷之寄願推誠王室擬迹伊周則率土幸甚護曰我誓以身報國卿豈謂吾有他志邪|{
	邪音耶}
又聞其先與龍恩言隂忌之植以憂卒|{
	卒子恤翻}
及護敗龍恩兄弟皆死高祖以植為忠特免其子孫大司馬兼小冢宰雍州牧齊公憲素為護所親任|{
	雍於用翻}
賞罸之際皆得參預權埶頗盛護欲有所陳多令憲聞奏其問或有可不|{
	不讀曰否}
憲慮主相嫌隙|{
	相息亮翻}
每曲而暢之帝亦察其心及護死召憲入憲免冠拜謝帝慰勉之使詣護第收兵符及諸文籍衛公直素忌憲固請誅之帝不許護世子訓為蒲州刺史是夜帝遣柱國越公盛乘傳徵訓至同州賜死|{
	自蒲州西南至同州一百三十里同州西南至長安二百二十五里傳張戀翻}
昌城公深使突厥未還|{
	使疏吏翻}
遣開府儀同三司宇文德齎璽書就殺之護長史代郡叱羅協|{
	叱羅虜複姓魏收官氏志拓拔内入諸姓有叱羅氏協時在同州璽斯氏翻}
司錄弘農馮遷及所親任者皆除名丁巳大赦改元|{
	改元建德}
以宇文孝伯為車騎大將軍|{
	騎奇寄翻}
與王軌並加開府儀同三司初孝伯與帝同日生太祖愛之養於第中幼與帝同學及即位欲引致左右託言欲與孝伯講習舊經故護弗之疑也以為右侍上士出入卧内預聞機務孝伯為人沈正忠諒|{
	沈持林翻}
朝政得失外間細事無不使帝聞之|{
	朝直遥翻}
帝閲護書記有假託符命妄造異謀者皆坐誅唯得庾季才書两紙盛言緯候災祥|{
	緯謂七緯日月五星之行失行則為災候謂月令七十二候失節則為災緯于貴反}
宜返政歸權帝賜季才粟三百石帛二百段遷太中大夫癸亥以尉遲迴為太師|{
	尉紆勿翻}
柱國竇熾為太傅李穆為太保齊公憲為大冢宰衛公直為大司徒陸通為大司馬柱國辛威為大司寇趙公招為大司空|{
	後周之制三公九命六官七命}
時帝始親覽朝政頗事威刑雖骨肉無所寛借齊公憲雖遷冢宰實奪之權又謂憲侍讀裴文舉曰昔魏末不綱|{
	後周諸王有侍讀之官不綱言人君不能操持大綱致衆目紊亂}
太祖輔政及周室受命晉公復執大權積習生常愚者謂法應如是豈有年三十天子而可為人所制乎詩云夙夜匪懈以事一人|{
	詩大雅烝民之詩懈古隘翻}
一人謂天子耳卿雖陪侍齊公不得遽同為臣欲死於所事宜輔以正道勸以義方輯穆我君臣協和我兄弟勿令自致嫌疑文舉咸以白憲憲指心撫几曰吾之夙心公寧不知但當盡忠竭節耳知復何言|{
	復扶又翻}
衛公直性浮詭貪狠|{
	狠戶懇翻}
意望大冢宰既不得殊怏怏更請為大司馬欲據兵權帝揣知其意曰汝兄弟長幼有序豈可返居下列由是用為大司徒|{
	為後三年衛公直作亂張本怏於兩翻揣初委翻長知兩翻}
夏四月周遣工部成公建小禮部辛彦之聘於齊|{
	杜佑通典周制工部中大夫屬冬官五命禮部屬春官中大夫五命小禮部上士也三命}
庚寅周追尊略陽公為孝閔皇帝|{
	廢略陽公事見百六十七卷高祖永定元年}
癸巳周立皇子魯公贇為太子|{
	贇於倫翻}
大赦 五月癸卯王勱卒|{
	勱音邁卒子恤翻}
齊尚書右僕射祖珽埶傾朝野左丞相咸陽王斛律光惡之|{
	珽他鼎翻朝直遥翻惡烏路翻}
遥見輒罵曰多事乞索小人|{
	乞索求取也小人求取無厭致國家多事索山客翻}
欲行何計又嘗謂諸將曰兵馬處分趙令恒與吾輩參論|{
	趙令謂趙彦深為尚書令以其官稱之也處昌呂翻分扶問翻將即亮翻恒戶登翻}
盲人掌機密以來|{
	祖珽病盲故詆之盲事始上卷臨海王光大元年}
全不與吾輩語正恐誤國家事耳光嘗在朝堂垂簾坐|{
	朝直遥反下同}
珽不知乘馬過其前光怒曰小人乃敢爾|{
	爾猶言如此也}
後珽在内省|{
	齊盖以門下省為内省}
言聲高慢光適過聞之又怒珽覺之私賂光從奴問之|{
	從才用翻}
奴曰自公用事相王每夜抱嘆曰盲人入國必破矣|{
	相息亮反}
穆提婆求娶光庶女不許齊主賜提婆晉陽田光言於朝曰此田神武帝以來常種禾飼馬數千匹以擬寇敵|{
	飼祥吏翻}
今賜提婆毋乃闕軍務也由是祖穆皆怨之斛律后無寵珽因而間之|{
	間古莧翻}
光弟羨為都督幽州刺史行臺尚書令亦善治兵|{
	治直之翻}
士馬精彊鄣候嚴整突厥畏之謂之南可汗|{
	可從刋入聲汗音寒}
光長子武都為開府儀同三司梁兖二州刺史|{
	長知兩翻}
光雖貴極人臣性節儉不好聲色|{
	好呼到反}
罕接賓客杜絶饋餉不貪權勢每朝廷會議常獨後言言輒合理或有表疏令人執筆口占之務從省實|{
	語省而事實}
行兵倣其父金之法營舍未定終不入幕或竟日不坐身不脱介胄常為士卒先士卒有罪唯大杖撾背|{
	撾側瓜翻}
未嘗妄殺衆皆争為之死|{
	為于偽翻}
自結髪從軍未嘗敗北深為鄰敵所憚周勲州刺史韋孝寛|{
	高歡宇文秦兵争秦使韋孝寛守玉璧歡盡力攻之不克而歸遂死宇文氏於此立勲州以旌其功其地在隋絳郡稷山縣}
密為謡言曰百升飛上天明月照長安|{
	上時掌反}
又曰高山不推自崩|{
	推吐雷翻}
槲木不扶自舉令諜人傳之於鄴|{
	諜逹叶翻}
鄴中小兒歌之於路珽因續之曰盲老公背受大斧饒舌老母不得語|{
	珽屯鼎翻今人猶謂多口為饒舌}
使其妻兄鄭道盖奏之帝以問珽珽與陸令萱皆曰實聞有之珽因解之曰百升者斛也盲老公謂臣也與國同憂饒舌老母似謂女侍中陸氏也且斛律累世大將明月聲震關西豐樂威行突厥|{
	斛律光字明月羨字豐樂將即亮翻樂音洛}
女為皇后男尚公主謡言甚可畏也帝以問韓長鸞長鸞以為不可事遂寢珽又見帝請間|{
	間古莧翻又音如字}
唯何洪珍在側帝曰前得公啓即欲施行長鸞以為無此理珽未對洪珍進曰若本無意則可既有此意而不决行萬一泄露如何帝曰洪珍言是也然猶未决會丞相府佐封士讓密啓云光前西討還敕令散兵光引兵逼帝城將行不軌事不果而止|{
	還從宣翻又音如字令力丁翻事見上卷三年封士讓密啓亦珽等使之也}
家藏弩甲僮奴千數每遣使往豐樂武都所|{
	使疏吏翻下同}
隂謀往來若不早圖恐事不可測帝遂信之謂何洪珍曰人心亦大靈我前疑其欲反果然帝性怯恐即有變令洪珍馳召祖珽告之欲召光恐其不從命珽請遣使賜以駿馬語云|{
	使疏吏翻語牛倨翻}
明日將遊東山王可乘此同行光必入謝因而執之帝如其言六月戊辰光入至凉風堂劉桃枝自後撲之不仆|{
	撲弼角翻}
顧曰桃枝常為如此事|{
	齊自文宣以來每殺諸王大臣劉桃枝率攘臂為之故光云然}
我不負國家桃枝與三力士以弓弦罥其頸拉而殺之|{
	罥古縣翻拉盧合翻}
血流於地剗之迹終不㓕|{
	剗初限翻削也}
於是下詔稱其欲反并殺其子開府儀同三司世雄儀同三司恒伽|{
	齊制開府儀同三司從一品儀同三司第二品恒戶登翻伽求加翻}
祖珽使二千石郎邢祖信簿錄光家|{
	齊制二千石郎掌畿外得失等事}
珽於都省問所得物|{
	都省即尚書都省五代志後齊制錄令僕射摠理六尚書事謂之都省}
祖信曰得弓十五宴射箭百刀七陽矟二|{
	明非私藏兵器矟色角翻}
珽厲聲曰更得何物曰得棗杖二十束擬奴僕與人闘者不問曲直即杖之一百|{
	棗木堅而密理可以為杖}
珽大慙乃下聲曰朝廷已加重刑郎中何宜為雪|{
	為于偽翻}
及出人尤其抗直祖信慨然曰賢宰相尚死我何惜餘生齊主遣使就州斬斛律武都又遣中領軍賀拔伏恩乘驛捕斛律羨仍以洛州行臺僕射中山獨孤永業代羨與大將軍鮮于桃枝發定州騎卒續進|{
	騎奇寄翻}
伏恩等至幽州門者白使人衷甲|{
	使疏吏翻下同}
馬有汗宜閉城門羨曰敕使豈可疑拒|{
	敕使謂使者奉敕而來至唐時率以稱宦者}
出見之伏恩執而殺之初羨常以盛滿為懼表解所職不許臨刑歎曰富貴如此女為皇后公主滿家常使三百兵何得不敗及其五子伏護世逹世遷世辨世酋皆死|{
	酋慈由翻}
周主聞光死為之大赦|{
	幸其死也為于偽翻}
祖珽與侍中高元海共執齊政元海妻陸令萱之甥也元海數以令萱密語告珽|{
	數所角翻}
珽求為領軍齊主許之元海密言於帝曰孝徵漢人|{
	祖珽字孝徵}
兩目又盲豈可為領軍因言珽與廣寧王孝珩交結|{
	孝珩文襄之子齊主所忌也珩戶庚翻}
由是中止珽求見自辨|{
	見賢遍翻}
且言臣與元海素嫌必元海譛臣帝弱顔不能諱以實告之|{
	見人輒自羞面顔有忸怩者為弱顔今人猶有是言}
珽因言元海與司農卿尹子華等結為朋黨又以元海所泄密語告令萱令萱怒出元海為鄭州刺史|{
	地形志天平初置潁州治長社城武定七年改鄭州治潁隂城周㓕齊改鄭州曰許州於榮陽置鄭州}
子華等皆被黜|{
	被皮義翻}
珽自是專主機衡|{
	尚書職掌機密任居銓衡}
摠知騎兵外兵事|{
	後齊制尚書郎有中兵外兵各分左右左外兵掌河南及潼関已東諸州右外兵掌河北及潼関已西諸州丁帳及發召徵兵等事騎奇寄翻}
内外親戚皆得顯位帝常令中要人扶侍出入|{
	中要人宦官之清要者}
直至永巷每同御榻論决政事委任之重羣臣莫比 秋七月遣使如周|{
	使疏吏翻下同}
八月庚午齊廢皇后斛律氏為庶人|{
	斛律光既死則其女廢矣}
以任城王湝為右丞相|{
	湝戶皆翻又音皆}
馮翊王潤為太尉蘭陵王長恭為大司馬廣寧王孝珩為大將軍安德王延宗為大司徒 齊使領軍封輔相聘於周|{
	相息亮反}
辛未周使司城中大夫杜杲來聘|{
	宋以武公名改司空為司城侯國之卿也後周倣成周之遺制必不以諸侯之卿名官盖髣髴周官掌固之職}
上謂之曰若欲合從圖齊|{
	從子容翻}
宜以樊鄧見與對曰合從圖齊豈弊邑之利必須城鎮宜待得之於齊先索漢南|{
	索山客翻}
使臣不敢聞命|{
	使疏吏翻}
初齊胡太后自愧失德|{
	胡太后失德久矣事則見上卷上年}
欲求悅於齊主乃飾其兄長仁之女置宫中令帝見之帝果悦納為昭儀|{
	元魏以來昭儀次於皇后位視大司馬}
及斛律后廢陸令萱欲立穆夫人太后欲立胡昭儀力不能遂乃卑辭厚禮以求令萱結為姊妹令萱亦以胡昭儀寵幸方隆不得已與祖珽白帝立之戊子立皇后胡氏 己丑齊以北平王仁堅為尚書令特進許季良為左僕射彭城王寶德為右僕射 癸巳齊主如晉陽 九月庚子朔日有食之 辛亥大赦 冬十月庚午周詔江陵所虜充官口者悉免為民|{
	梁元帝承聖三年江陵破士民皆為魏所虜入関}
辛未周遣小匠師楊勰等來聘|{
	匠師大司空之屬也杜佑通典周小匠師下大夫屬冬官四命又有上士三命勰音恊}
周綏德公陸通卒|{
	綏德縣公也西魏於綏州上縣置綏德縣卒子恤翻}
乙酉上享太廟|{
	五代志陳立七廟一歲五祠春夏秋冬臘也每祭共以一大牢始祖以三牲首餘惟骨體而已}
齊陸令萱欲立穆昭儀為皇后私謂齊主曰豈有男為皇太子而身為婢妾者胡后有寵於帝不可離間|{
	間古莧翻}
令萱乃使人行厭蠱之術|{
	厭一琰翻}
旬朔之間胡后精神恍惚|{
	恍呼廣翻惚音忽}
言笑無恒帝漸畏而惡之|{
	恒戶登翻惡烏路翻}
令萱一旦忽以皇后服御衣被昭儀|{
	衣於既翻被皮義翻}
又别造寶帳爰及枕席器玩莫非珍奇坐昭儀於帳中謂帝曰有一聖女出將大家看之及見昭儀令萱乃曰如此人不作皇后遣何物人作帝納其言甲午立穆氏為右皇后以胡氏為左皇后十一月庚戌周主行如羌橋|{
	羌橋在長安東以符姚諸羌而得名}
集長

安以東諸軍都督以上頒賜有差乙卯還宫以趙公招為大司馬壬申周主如斜谷|{
	斜音邪又似嗟翻谷音欲又古禄翻}
集長安已西都督已上頒賜有差丙戍還宫 庚寅周主遊道會苑以上善殿壯麗焚之 十二月辛巳周主祀南郊|{
	五代志周憲章姬周祭祀之式多依儀禮南郊為方壇於國南五里其崇一丈二尺其廣四丈其壝方百二十步内壝半之祭以正月上辛以始祖獻侯莫那配所感帝靈威仰於其上也}
齊胡后之立非陸令萱意令萱一旦於太后前作色而言曰何物親姪作如此語太后問其故令萱曰不可道因問之乃曰語大家云|{
	語大牛倨翻}
太后行多非法不可以訓|{
	行下孟翻}
太后大怒呼后出立剃其髪送還家|{
	剃他計翻}
辛丑廢胡后為庶人然齊主猶思之每致物以通意自是令萱與其子侍中穆提婆勢傾内外賣官鬻獄聚歛無厭|{
	斂力膽翻厭一鹽翻}
每一賜與動傾府藏|{
	藏徂浪翻}
令萱則自太后以下皆受其指麾提婆則唐邕之徒皆重足屏氣|{
	重直龍翻屏必郢翻}
殺生予奪唯意所欲 乙巳周以柱國田弘為大司空 乙卯周主享太廟|{
	五代志後周思復古之道右宗廟而左社稷置太祖之廟并高祖已下二昭二穆凡五親盡則遷其有德者謂之祧廟亦不毁}
是歲突厥木杆可汗卒復捨其子大邏便而立其弟是為佗鉢可汗佗鉢以攝國為爾伏可汗統其東面又以其弟褥但可汗之子為步離可汗居西面|{
	厥九勿翻梁世祖承聖二年突厥士門可汗卒捨其子攝圖立其弟俟斗稱為木杆可汗褥但既佗鉢之弟盖小可汗也扞古按翻可從刋入聲汗音寒卒子恤翻復扶又翻佗徒何翻}
周人與之和親歲給繒絮錦綵十萬段|{
	繒慈陵翻}
突厥在長安者衣錦食肉常以千數|{
	衣於既翻}
齊人亦畏其為寇争厚賂之佗鉢益驕謂其下曰但使我在南两兒常孝何憂於貧|{
	在南两兒謂尔伏步離二人所部分西北皆南近中國}
阿史那后無寵於周主 |{
	考異曰周書曰后有姿貌善容止周帝甚敬焉案房玄齡唐高祖實録云武帝納突厥女陋而無寵太穆皇后勸帝強撫慰之今從之}
神武公竇毅尚襄陽公主|{
	神武郡公拓拔魏置神武郡於尖山}
生女尚幼密言於帝曰今齊陳鼎峙|{
	齊陳及周三國鼎峙}
突厥方彊願舅抑情慰撫以生民為念帝深納之|{
	此女後適李淵是為唐高祖竇皇后}


五年春正月癸酉以吏部尚書沈君理為右僕射 戊寅齊以并省尚書令高阿那肱録尚書事摠知外兵及内省機密與侍中城陽王穆提婆領軍大將軍昌黎王韓長鸞共處衡軸|{
	處昌呂翻車軛曰衡持輪者曰軸車非二者不行故以為喩}
號曰三貴蠧國害民日月滋甚|{
	滋益也}
長鸞弟萬歲子寶行寶信並開府儀同三司萬歲仍兼侍中寶行寶信皆尚公主每羣臣旦參|{
	參朝參也}
帝常先引長鸞顧訪出後方引奏事官若不視事内省有急奏事皆附長鸞奏聞軍國要密無不經手尤疾士人朝夕宴私唯事譛訴常帶刀走馬未嘗安行瞋目張拳|{
	瞋昌真翻}
有噉人之勢朝士咨事莫敢仰視動致呵叱每罵云漢狗大不可耐唯須殺之|{
	呵虎何翻叱昌栗翻耐奴代翻}
庚辰齊遣崔象來聘 辛巳上幸南郊甲午享大廟二月辛丑祀明堂 乙巳齊立右皇后穆氏為皇后穆后母名輕霄本穆氏之婢也面有黥字|{
	北史輕霄本穆子倫婢轉入宋欽道家欽道之婦妬輕霄黥其面為宋字姦私而生此女莫知其姓}
后既以陸令萱為母穆提婆為外家號令萱曰太姬太姬者齊皇后母號也視一品班在長公主上|{
	長知两翻}
由是不復問輕霄|{
	復扶又翻}
輕霄自療面欲求見后太姬使禁掌之竟不得見齊主頗好文學|{
	好呼到翻}
丙午祖珽奏置文林館多引文學之士以充之謂之待詔以中書侍郎博陵李德林黃門侍郎琅邪顔之推同判館事又命共撰修文殿御覽|{
	齊大統中毁東宫起修文等殿撰士免翻}
甲寅周太子贇廵省西土|{
	省悉景翻}
乙卯齊以北平王堅録尚書事|{
	按上年秋八月齊以北平王仁堅録尚書事此恐逸仁字}
丁巳齊主如晉陽 壬戍周遣司會侯莫陳凱等聘於齊|{
	周官司會屬冢宰鄭玄曰會大計也司會主天下之大計計官之長若今尚書後周司會屬天官中大夫也會工外翻}
庚辰齊主還鄴 三月己卯周太子於岐州獲二白鹿以獻|{
	因廵省而獲白鹿}
周主詔曰在德不在瑞 帝謀伐齊公卿各有異同唯鎮前將軍吳明徹决策請行|{
	梁武帝置八鎮將軍東西南北止施在外左右前後止施在内}
帝謂公卿曰朕意已决卿可共舉元帥|{
	帥所類翻}
衆議以中權將軍淳于量位重共署推之|{
	梁置四中將軍與八鎮將軍同擬官品第二然四中班四征之上八鎮班四征之下故以量位為重}
尚書左僕射徐陵獨曰吳明徹家在淮左|{
	吳明徹秦郡人}
悉彼風俗將略人才當今亦無過者|{
	將即亮翻下同}
都官尚書河東裴忌曰臣同徐僕射陵應聲曰非但明徹良將裴忌即良副也壬午分命衆軍以明徹都督征討諸軍事忌監軍事|{
	將即亮翻監工銜翻}
統衆十萬伐齊明徹出秦郡都督黄法出歷陽|{
	巨俱翻}
夏四月己亥周主享太廟 癸卯前巴州刺史魯廣逹與齊師戰于大峴破之|{
	梁置巴州於巴陵此大峴在合肥之南歷陽之北峴戶典翻}
戊申齊以蘭陵王長恭為太保南陽王綽為大司馬安德王延宗為大尉武興王普為司徒開府儀同三司宜陽王趙彦深為司空 齊人於秦郡置秦州州前江浦通涂水|{
	今真州閘即其地涂讀曰滁}
齊人以大木為柵於水中辛亥吳明徹遣豫章内史程文季將驍勇拔其柵克之|{
	將即亮翻又音如字領也驍堅堯翻}
文季靈洗之子也|{
	梁陳之間程靈洗以勇鳴}
齊人議禦陳師開府儀同三司王紘曰官軍比屢失利人情騷動|{
	攷之史比年以來齊師未嘗失利盖争宜陽汾北周齊更勝迭負周師雖退齊師亦疲也比毗至翻}
若復出頓江淮恐北狄西寇乘弊而來|{
	復扶又翻北狄謂突厥西寇謂周}
莫若薄賦省徭息民養士使朝廷輯睦遐邇歸心天下皆當肅清豈直陳氏而已不從遣軍救歷陽|{
	考異曰陳書帝紀云齊遣兵十萬救歷陽黄法傳云步騎五萬援歷陽蕭摩訶傳尉破胡等帥衆十萬來援案源文宗之語恐無此數今不取}
庚申黄法擊破之又遣開府儀同三司尉破胡長孫洪略救秦州|{
	尉紆勿翻}
趙彦深私問計於秘書監源文宗曰吳賊侏張遂至于此|{
	侏舊音張流翻盖因書譸張為幻爾雅譸作侜遂有此音按類篇侏音張流切其義華也書所謂譸張其義誕也以文理求之皆於此不近姑闕之以待知者}
弟往為秦涇刺史|{
	齊置秦州於秦郡涇州於石梁}
悉江淮間情事今何術以禦之文宗曰朝廷精兵必不肯多付諸將|{
	將音即亮翻}
數千已下適足為吳人之餌尉破胡人品王之所知|{
	彦深封宜陽王故稱之}
敗績之事匪朝伊夕國家待遇淮南失之同於蒿箭|{
	唐高祖遣李密徇山東廷臣多諫帝曰如以蒿箭射蒿中耳言不足惜也乃知此語之來久矣}
如文宗計者不過專委王琳招募淮南三四萬人風俗相通能得死力兼令舊將將兵屯於淮北|{
	將音即亮翻}
且琳之於頊必不肯北面事之明矣竊謂此計之上者若不推赤心於琳更遣餘人掣肘|{
	宓子賤為單父宰言於魯君請與二吏俱至邑使二吏書而掣其肘書不工輒怒之吏不能堪歸以告魯君魯君曰是慮我掣其肘耳宓賤是以能為單父後之言掣肘者本此掣音昌列翻}
復成速禍彌不可為|{
	是役卒如文宗之言復音扶又翻}
彥深歎曰弟此策誠足制勝千里但口舌争之十日已不見從時事至此安可盡言因相顧流涕文宗名彪以字行子恭之子也|{
	源子㳟以幹用稱於熙平永安之間}
文宗子師為左外兵郎中攝祠部嘗白高阿那肱龍見當雩阿那肱驚曰何處龍見其色如何師曰龍星初見禮當雩祭非眞龍也阿那肱怒曰漢兒多事強知星宿遂不祭師出竊歎曰禮既廢矣齊能久乎|{
	見賢遍翻強其兩翻春秋左氏傳曰龍見而雩杜預注云龍見建巳之月蒼龍之體昏見東方萬物始盛待雨而大故祭天遠為百穀祈甘雨鄭玄曰雩吁嗟求雨之祭孔穎逹曰天之四方皆有七宿各成一形東方成龍形西方成虎形皆南首而北尾南方成鳥形北方成龜形皆西首而東尾五代志後齊以孟夏龍見而雩祭太微五精帝於夏郊之東為圓壇於其上祈穀實以顯祖文宣帝配通鑑言國之將亡其禮先亡諸源本出於鮮卑未嘗以為諱鮮卑遂自謂貴種率謂華人為漢兒率侮詬之諸源世仕魏朝貴顯習知典禮遂有雩祭之請冀以取重乃以取詬通鑑詳書之又一嘅也因源文宗不敢盡言併及其子竊歎之事}
齊師選長大有膂力者為前隊又有蒼頭犀角大力其鋒甚鋭又有西域胡善射弦無虛發衆軍尤憚之辛酉戰于呂梁|{
	呂梁在彭城疑此即石梁}
將戰吳明徹謂巴山太守蕭摩訶曰若殪此胡|{
	守音式又翻殪音一計翻}
則彼軍奪氣君才不减關羽矣摩訶曰願示其狀當為公取之|{
	為音于偽翻}
明徹乃召降人有識胡者使指示之自酌酒以飲摩訶|{
	降音戶江翻飲音於禁翻}
摩訶飲畢馳馬衝齊軍胡挺身出陳前十餘步|{
	陳讀曰陣}
彀弓未摩訶遥擲銑鋧|{
	銑音蘇典翻鋧音他典翻類篇曰銑鋧小鑿也}
正中其額應手而仆|{
	中音竹仲翻}
齊軍大力十餘人出戰摩訶又斬之於是齊軍大敗尉破胡走 |{
	考異曰北齊書破胡敗在五月今從齊書}
長孫洪略戰死破胡之出師也齊人使侍中王琳與之俱琳謂破胡曰吳兵甚鋭宜以長策制之慎勿輕鬬破胡不從而敗琳單騎僅免|{
	騎奇寄翻}
還至彭城齊人即使之赴壽陽召募以拒陳師復以盧潜為楊州道行臺尚書|{
	復扶又翻盧潜在壽陽與王琳不協事見一百六十八卷世祖天嘉二年}
甲子南譙太守徐槾克石梁城|{
	槾謨于翻五代志石梁在江都郡永福縣齊置涇州於此}
五月己巳瓦梁城降|{
	以五代志攷之瓦梁城當在江都郡六合縣界}
癸酉陽平郡降|{
	以地形志攷之梁置淮州治淮隂城其屬有陽平郡治陽平城其地當在淮隂城西}
甲戍徐槾克廬江城|{
	按地形志梁置廬江郡治潜縣潜縣今属無為軍界徐槾之師盖漸西向}
歷陽窘蹙乞降黃法緩之則又拒守法怒帥卒急攻|{
	帥讀曰率}
丙子克之盡殺戍卒進軍合肥合肥望旗請降法禁侵掠撫勞戍卒|{
	勞力到翻}
與之盟而縱之 丁丑周以柱國侯莫陳瓊為大宗伯榮陽公司馬消難為大司寇|{
	難乃旦翻}
江陵摠管陸騰為大司空瓊崇之弟也|{
	侯莫陳崇八柱國之一也}
己卯齊北高唐郡降|{
	五代志同安郡宿松縣梁置高塘郡降戶江翻}
辛巳詔南豫州刺史黃法|{
	巨俱翻}
徙鎮歷陽|{
	晉氏南渡豫部殱覆刺史祖約自譙城退屯壽春咸和間庾亮治蕪湖咸康間毛寶治邾城永和初趙胤鎮牛渚二年謝尚鎮蕪湖四年進壽春九年鎮歷陽十一年進馬頭寧康初桓冲戍姑孰宋永初中分淮東為南豫州治歷陽淮西為豫州泰始中淮西䧟沒以揚州之淮南宣城為南豫州治宣城蕭齊治姑孰梁武佳兵治無定所侯景之亂江淮之地皆歸高齊陳治宣城今復歷陽命徙鎮焉}
乙酉南齊昌太守黃詠克齊昌外城|{
	五代志蘄春郡蘄春縣舊曰蘄陽梁改蘄水後齊改曰齊昌置齊昌郡守式又翻}
丙戍廬陵内史任忠軍於東關克其東西二城|{
	東關東西二城吳諸葛恪所築也任音壬}
進克蘄城|{
	五代志廬江郡襄安縣梁曰蘄蘄居衣翻又音其}
戊子又克譙郡城 |{
	地形志合州之南譙郡城也亦在蘄縣界}
秦州城降|{
	自四月辛亥拔秦州水柵至是三十八日州城始降}
癸巳瓜步胡墅二城降|{
	二城皆在六合縣界臨江降戶江翻}
帝以秦郡吳明徹之鄉里詔具太牢令拜祠上冢|{
	上時掌翻}
文武羽儀甚盛鄉人榮之 齊自和士開用事以來政體隳紊|{
	和士開用事始一百六十九卷世祖天嘉四年隳音揮紊扶問翻}
及祖珽執政頗收舉才望内外稱美珽復欲增損政務沙汰人物|{
	汰音太}
官號服章並依故事又欲黜諸閹豎及羣小輩為致治之方|{
	豎臣庾翻童僕未冠者也又音樹治直之翻}
陸令萱穆提婆議頗同異珽乃諷御史中丞麗伯律令劾主書王子冲納賂|{
	麗姓伯律名姓苑有麗姓後齊制中書省有舍人主書各十人劾戶槩翻又戶得翻}
知其事連提婆欲使罪相及望因此并坐及令萱猶恐齊主溺於近習|{
	溺奴狄翻}
欲引后黨為援乃請以胡后兄君瑜為侍中中領軍又徵君瑜兄梁州刺史君璧|{
	元魏置梁州於大梁城}
欲以為御史中丞令萱聞而懷怒百方排毁出君瑜為金紫光禄大夫解中領軍|{
	出者自内省出就朝列金紫光禄大夫本晉之左右光禄大夫假金章紫綬後遂於左右光禄大夫之下又置光禄大夫而光禄大夫假銀印青綬者為銀青光禄大夫後齊制金紫光禄大夫從二品中領軍第三品君瑜既解中領軍有品秩而無職事}
君璧還鎮梁州胡后之廢頗亦由此釋王子冲不問珽日以益踈諸宦者更共譖之帝以問陸令萱令萱憫默不對|{
	憫默憂而不敢言之貌}
三問乃下牀拜曰老婢應死|{
	言其罪應死也}
老婢始聞和士開言孝徵多才博學意謂善人故舉之比來觀之|{
	比毗至翻}
大是姧臣人寔難知老婢應死帝令韓長鸞檢案|{
	檢察也搜也校也舉也案考驗也亦舉也}
長鸞素惡珽|{
	惡烏路翻}
得其詐出勑受賜等十餘事帝以嘗與之重誓故不殺解珽侍中僕射出為北徐州刺史|{
	五代志琅邪郡舊置北徐州}
珽求見帝長鸞不許遣人推出柏閤|{
	見賢遍翻推吐雷翻}
珽坐不肯行長鸞令牽曳而出癸巳齊以領軍穆提婆為尚書左僕射侍中中書監段孝言為右僕射孝言韶之弟也|{
	段韶歷事高氏五世著忠孝戰功為多}
初祖珽執政引孝言為助除吏部尚書孝言凡所進擢非賄則舊求仕者或於廣會行跪伏公自陳請孝言顔色揚揚以為己任隨事酬許將作丞崔成忽於衆中抗言曰尚書天下尚書豈獨段家尚書也孝言無辭以應唯厲色遣下而已既而與韓長鸞共搆祖珽逐而代之 齊蘭陵武王長恭貌美而勇以邙山之捷威名大盛 |{
	考異曰北齊書長恭與周戰於邙山後主謂曰入陳太深失利悔無所及對曰家事親切不覺遂然帝嫌其稱家事遂忌之案邙山之戰在河清三年後主時年九歲尚未即位何得有此問且稱家事亦何足致忌今不取}
武士歌之為蘭陵王入陳曲|{
	杜佑曰北齊蘭陵王長㳟才武而貌美嘗著假面以對敵嘗擊周師金墉城下勇冠三軍齊人壯之作此舞以效其指麾擊刺之容謂之蘭陵王入陳曲}
齊主忌之及代段韶督諸軍攻定陽頗務聚歛|{
	邙山之捷見一百六十九卷世祖天嘉五年攻定陽見上卷大建三年陳讀曰陣歛力贍翻}
其所親尉相願|{
	尉紆勿翻}
問之曰王受朝寄何得如此|{
	朝直遥翻}
長恭未應相願曰豈非以邙山之捷欲自穢乎長恭曰然相願曰朝廷若忌王即當用此為罪無乃避禍而更速之乎長㳟涕泣前膝問計|{
	俯身而問前於席故曰前}
相願曰王前既有功今復告捷|{
	復扶又翻}
威聲太重宜屬疾在家|{
	屬之欲翻}
勿預時事長恭然其言未能退及江淮用兵恐復為將|{
	復扶又翻將息亮翻}
歎曰我去年面腫今何不發自是有疾不療齊主遣使酖殺之|{
	療力弔翻使疏吏翻}
六月郢州刺史李綜克灄口城|{
	水經注江水逕魯山南左得湖口水又東合灄口水水上承沔水於安陸縣而東逕灄陽縣北東南注于江灄書涉翻}
乙巳任忠克合州外城|{
	按地形志及五代志皆云合州治合肥合肥前已降黄法今任忠乂克合州外城何也當攷}
庚戍淮陽沭陽郡皆棄城走|{
	五代志梁置淮陽郡於下邳郡之淮陽縣又置潼陽郡於東海郡之沭陽縣東魏改曰沭陽郡沐食聿翻}
壬子周皇孫衍生 齊主遊南苑從官賜死者六十人|{
	史言齊主淫刑以逞從才用翻}
以高阿那肱為司徒 癸丑程文季攻齊涇州拔之|{
	按齊涇州治石梁是年四月徐槾已克石梁城}
乙卯宣毅司馬湛陀克新蔡城|{
	梁置鎮兵翊師宣惠宣毅四將軍代舊四中郎將盖皆有長史司馬湛姓陀名後漢有大司農湛重五代志廬江郡渒水縣弋陽郡定城縣殷城縣皆有梁所置新蔡郡又固始縣有後齊所置新蔡郡未知孰是湛徒减翻陀徒何翻}
丙辰齊使開府儀同三司王紘聘于周 癸亥黃法克合州|{
	以此觀之則前請降者合肥戍卒也巨俱翻}
吳明徹進攻仁州|{
	地形志梁置仁州治赤坎城盖在山陽縣界}
甲子克之 治明堂|{
	五代志陳制明堂殿屋十二間中央六間安六座四方帝各依其方黄帝居坤維治直之翻}
秋七月戊辰齊遣尚書左丞陸騫將兵二萬救齊昌出自巴蘄|{
	出自巴水蘄水之間也將即亮翻蘄音機又音祈}
遇西陽太守汝南周炅|{
	西陽郡在黄岡縣界炅枯迥翻又古惠翻}
炅留羸弱設疑兵以當之身帥精鋭由間道邀其後大破之|{
	羸倫為翻帥讀曰率間古莧翻}
己巳征北大將軍吳明徹軍至峽口克其北㟁城南岸守者棄城走|{
	峽口峽石口也夾岸築两城以扼淮流吳明徹以功進律自從二品升第二品}
周炅克巴州|{
	後齊置巴州於黄岡}
淮北絳城及穀陽士民並殺其戍主以城降|{
	絳城蓋虹縣城音同而字異耳五代志彭城郡穀陽縣後齊置穀陽郡}
齊巴陵王王琳與揚州刺史王貴顯保壽陽外郭吳明徹以琳初入衆心未固丙戍乘夜攻之城潰齊兵據相國城及金城|{
	二城皆在壽陽城中相國城劉裕伐長安所築故名金城夀陽中城也自晉以來率謂中城為金城}
八月乙未山陽城降|{
	五代志江都郡山陽縣舊置山陽郡}
壬寅盱眙城降|{
	盱眙縣亦屬江都舊置盱眙郡盱眙音吁怡}
壬子戎昭將軍徐敬辯克海安城|{
	陳制戎昭將軍品第八秩六百石海安城在海陵縣東今為海安鎮}
青州東海城降|{
	東海郡梁置南北二青州今朐山縣}
戊午平固侯敬泰等克晉州|{
	平固縣沈約志屬南康郡吳立曰平陽晉武帝太康元年更名五代志同安郡梁置晉州因晉熙以為名}
九月甲子陽平城降|{
	五代志江都郡安宜縣梁置陽平郡}
壬申高陽太守沈善慶克馬頭城|{
	五代志鍾離郡塗山縣古當塗也後齊置馬頭郡}
甲戍齊安城降|{
	五代志永安郡黄岡縣後齊置齊安郡}
丙子左衛將軍樊毅克廣陵楚子城|{
	此廣陵非江都之廣陵按魏太和中蠻帥田益宗納土於魏魏為立東豫州治廣陵城五代志汝南郡新息縣魏置東豫州則此廣陵乃新息之廣陵也又梁武帝置楚州於汝南郡之城陽縣治楚城即楚子城也水經淮水先過城陽縣而後過新息縣則知廣陵城與楚子城相近}
壬午周太子贇納妃楊氏妃大將軍隨公堅之女也|{
	為楊堅由后父而簒周張本}
太子好昵近小人|{
	好呼到翻昵尼質翻近其靳翻}
左宫正宇文孝伯言於周主曰皇太子四海所屬|{
	屬之欲翻}
而德聲未聞臣忝宮官實當其責且春秋尚少|{
	少詩照翻}
志業未成請妙選正人為其師友調護聖質猶望日就月將|{
	就從也將行也從事於學將以行之也鄭玄曰日就月將言當習之以積}
如或不然悔無及矣帝歛容曰卿世載鯁直|{
	鯁古杏翻毛晃曰鯁魚骨又骨不下咽世謂謇諤者為骨鯁謂直言難受如骨之咈咽也}
竭誠所事觀卿此言有家風矣孝伯拜謝曰非言之難受之難也帝曰正人豈復過卿|{
	復扶又翻}
於是以尉遲運為右宫正運迥之弟子也|{
	尉紆勿翻}
帝嘗問萬年縣丞南陽樂運曰卿言太子何如人對曰中人帝顧謂齊公憲曰百官佞我皆稱太子聰明睿智唯運所言忠直耳因問運中人之狀對曰如齊桓公是也管仲相之則覇豎貂輔之則亂|{
	相息亮翻}
可與為善可與為惡帝曰我知之矣乃妙選宫官以輔之仍擢運為京兆丞|{
	由赤縣丞擢京郡丞}
太子聞之意甚不悦 癸未沈君理卒|{
	卒子恤翻}
壬辰晦前鄱陽内史魯天念克黄城|{
	地形志譙州下蔡郡有黄城縣按東魏置譙州於渦陽則黄城亦其屬縣也盖下蔡在淮北而黄城在壽陽西水經注柴水東逕黄城西故式陽縣也城内二城西即黄城也柴水東北入于淮謂之淮口}
甲午郭默城降|{
	晉氏不競劉石強盛郭默轉徙而南築城以自保故有其名}
己亥以特進領國子祭酒周弘正為尚書右僕射 齊國子祭酒張雕以經授齊主為侍讀帝甚重之雕與寵胡何洪珍相結穆提婆韓長鸞等惡之|{
	惡烏路翻}
洪珍薦雕為侍中加開府儀同三司奏度支事|{
	度徒洛翻後齊六尚書度支其一也統度支倉部左戶右戶金部庫部六曹凡度支事雕以奏聞}
大為帝所委信常呼博士雕自以出於微賤致位大臣欲立効以報恩論議抑揚無所囘避省宫掖不急之費禁約左右驕縱之臣數譏切寵要獻替帷幄|{
	數所角翻}
帝亦深倚仗之雕遂以澄清為己任意氣甚高貴倖皆側目尚書左丞封孝琰隆之之弟子|{
	封隆之高氏起兵佐命之臣}
與侍中崔季舒皆為祖珽所厚孝琰嘗謂珽曰公是衣冠宰相異於餘人近習聞之大以為恨會齊主將如晉陽季舒與張雕議以為壽陽被圍|{
	被皮義翻}
大軍出拒之信使往還|{
	使疏吏翻}
須禀節度且道路小人或相驚恐以為大駕向并州畏避南寇若不啓諫恐人情駭動遂與從駕文官連名進諫|{
	從才用翻}
時貴臣趙彦深唐邕段孝言等意有異同季舒與争未决長鸞遽言於帝曰諸漢官連名總署聲云諫幸并州其實未必不反宜加誅勠辛丑齊主悉召已署名者集含章殿斬季舒雕孝琰及散騎常侍劉逖黄門侍郎裴澤郭遵於殿庭家屬皆徙北邉婦女配奚官幼男下蠶室|{
	下戶嫁翻}
沒入貲產癸卯遂如晉陽 吳明徹攻壽陽堰肥水以灌城|{
	按水經注肥水過壽陽城而入淮然引流入城交絡城中吳明徹堰之以灌城其勢順易}
城中多病腫泄|{
	泄私列翻}
死者什六七齊行臺右僕射琅邪皮景和等救壽陽|{
	琅邪之下疑脱王字}
以尉破胡新敗怯懦不敢前屯於淮口|{
	淮口盖即頴口景和之師自頴上出至淮而屯因謂之淮口}
敕使屢促之|{
	使疏吏翻下同}
然始度淮|{
	然如此也如此始度淮也}
衆數十萬去壽陽三十里頓軍不進諸將皆懼|{
	將即亮翻下同}
曰堅城未拔大援在近將若之何明徹曰兵貴神速而彼結營不進自挫其鋒吾知其不敢戰明矣乙巳躬擐甲胄|{
	擐音宦}
四面疾攻一鼓拔之生擒王琳王貴顯盧潜及扶風王可朱渾道裕尚書左丞李騊駼送建康|{
	可朱渾虜三字姓騊徒刀翻駼同都翻}
景和北遁盡收其駝馬輜重|{
	重直用翻}
琳體貌閑雅喜怒不形於色彊記明敏軍府佐吏千數皆能識其姓名|{
	識職吏翻}
刑罰不濫輕財愛士得將卒心雖失地流寓在鄴齊人皆重其忠義及被擒故麾下將卒多在明徹軍中見者皆歔欷不能仰視争為之請命|{
	將即亮翻被皮義翻歔音虛欷音希又許既翻為于偽翻}
及致資給明徹恐其為變遣使追斬之於壽陽東二十里|{
	使疏吏翻}
哭者聲如雷有一叟以酒脯來祭哭盡哀收其血而去田夫野老知與不知聞者莫不流涕齊穆提婆韓長鸞聞壽陽陷握槊不輟曰本是彼物從其取去|{
	南北兵争壽陽本屬江南故云然史言齊之君臣以樂慆憂}
齊主聞之頗以為憂提婆等曰假使國家盡失黄河以南猶可作一龜兹國|{
	龜兹音丘慈唐人又讀為屈佳}
更可憐人生如寄唯當行樂|{
	樂音洛}
何用愁為左右嬖臣因共贊和之|{
	嬖卑義翻又匹計翻和戶卧翻}
帝即大喜酣飲鼓舞|{
	酣戶甘翻}
仍使於黎陽臨河築城戍|{
	懼陳兵之來真欲畫河自保}
丁未齊遣兵萬人至潁口|{
	潁水入淮之口}
樊毅擊走之辛亥遣兵援蒼陵又破之|{
	地形志揚州淮南郡壽春縣故楚有蒼陵城水經注淮水東流與穎口會東南逕蒼陵北又東北流逕壽春縣故城西}
齊主以皮景和全軍而還賞之除尚書令|{
	還從宣翻又音如字}
丙辰詔以壽陽復為豫州|{
	復宋齊之舊也}
以黄城為司州以明徹為都督豫合等六州諸軍事車騎大將軍豫州刺史遣謁者蕭淳風就壽陽冊命|{
	騎奇寄翻陳依梁制謁者僕射秩千石}
於城南設壇士卒二十萬陳旗鼓戈甲明徹登壇拜受成禮而退將卒榮之上置酒舉杯屬徐陵曰賞卿知人|{
	將即亮翻屬之欲翻}
陵避席曰定策聖衷非臣力也以黄法為征西大將軍合州刺史戊午湛陀克齊昌城十一月甲戍淮隂城降|{
	五代志江都郡山陽縣有淮隂郡巨俱翻降戶江翻}
庚辰威虜將軍劉桃枝克朐山城|{
	陳制威虜將軍品第八秩六百石此劉桃枝自是陳將非齊之劉桃枝五代志東海郡有朐山縣朐音劬}
辛巳樊毅克濟隂城|{
	五代志鍾離郡化明縣故曰睢陵置濟隂郡濟子禮翻}
己丑魯廣逹攻濟南徐州克之|{
	濟當作齊書齊南徐以别京口之南徐以五代史攷之齊之南徐州本置於下邳郡宿豫縣詳又注於考異之下}
以廣逹為北徐州刺史鎭其地 |{
	考異曰陳書帝紀及廣逹傳皆云北徐州案北齊書祖珽傳珽保全北徐州城不陷蓋南人謂京口為南徐州故謂此為北徐州其實乃北齊之南徐州也 按此所謂齊南徐州乃舊琅邪郡宋泰始初琅邪沒魏莊帝永安二年置北徐州於琅邪時齊以祖珽為北徐州鎮琅邪魏收地形志太和中立北徐州於宿豫蕭衍置北徐州於鍾離南北兵争疆場之間一彼一此各立州郡當隨其所立州名書之恐不可以齊之北徐州為齊南徐州也}
齊北州民多起兵以應陳逼其州城|{
	按五代志齊置北徐州於琅邪珽他鼎翻}
祖珽命不閉城門禁人不得出衢路|{
	爾雅四逹謂之衢}
城中寂然反者不測其故疑人走城空不設備珽忽令鼓譟震天反者皆驚走既而復結陳向城|{
	復扶又翻陳讀曰陣}
珽令錄事參軍王君植將兵拒之自乘馬臨陳左右射|{
	射食亦翻}
反者先聞其盲謂其必不能出忽見之大驚穆提婆欲令城䧟不遣援兵珽且戰且守十餘日反者竟散走詔懸王琳首於建康市故吏梁驃騎倉曹參軍朱瑒|{
	梁制將軍府有功曹倉曹中兵外兵騎兵長流城局等參軍驃匹妙翻騎奇寄翻瑒雉杏翻又音暢}
致書徐陵求其首曰竊以典午將滅徐廣為晉家遺老|{
	典司也午屬馬故謂司馬為典午徐廣事見一百十九卷宋高祖永初元年}
當塗已謝馬孚稱魏主忠臣|{
	當塗高者魏也司馬孚事見七十九卷晉世祖泰始元年}
梁故建寧公琳當離亂之辰摠方伯之任天厭梁德尚思匡繼徒藴包胥之志|{
	左傳吳破楚入郢申包胥赴秦請救以秦師破吳而復楚}
終遘萇弘之眚|{
	周靈王即位諸侯不朝萇弘乃明鬼神事設射諸侯之不來者欲依怪物以致諸侯諸侯不從而周室愈微後二世至敬王晉人殺萇弘}
至使身沒九泉頭行千里伏惟聖恩博厚明詔爰發赦王經之哭|{
	按魏元帝景元元年司馬昭弑高貴鄉公併收王經其吏向䧺抱經哭於東市昭赦䧺}
許田横之葬|{
	漢高帝葬田横事見十一卷五年}
不使壽春城下唯傳報葛之人|{
	魏高貴卿公甘露三年司馬昭破壽春諸葛誕麾下不降而死事見七十七卷}
滄洲島上獨有悲田之客|{
	即田横事}
陵為之啓上|{
	為于偽翻}
十二月壬辰朔并熊曇朗等首皆還其親屬|{
	熊曇朗誅見一百六十八卷世祖天嘉元年}
瑒瘞琳於八公山側義故會葬者數千人瑒間道奔齊|{
	義故故舊以義結者瘞於計翻間古莧翻}
别議迎葬尋有壽陽人茅智勝等五人密送其柩於鄴|{
	柩音舊}
齊贈琳開府儀同三司録尚書事謚曰忠武王給輼輬車以葬之|{
	自秦漢以來天子葬用輼輬車輼音温輬音凉}
癸巳周主集群臣及沙門道士帝自升高坐辯三教先後|{
	坐徂卧翻}
以儒為先道為次釋為後 乙未譙城降|{
	五代志譙郡山桑縣梁置渦州東魏改曰譙州降戶江翻}
乙巳立皇子叔明為宜都王叔獻為河東王 壬午任忠克霍州|{
	五代志廬江郡霍山縣梁置霍州任音壬}
詔徵安州刺史周炅入朝|{
	按五代志西魏置安州於安陸梁陳無安州隋書周法尚傳周炅為定州刺史或者安字其定字之誤歟朝直遥翻}
初梁定州刺史田龍升以城降|{
	定州梁置治蒙龍城水經注舉水出龜頭山西北流逕蒙龍城南梁定州治五代志永安都麻城縣陳置定州}
詔仍舊任及炅入朝龍升以江北六州七鎭叛入於齊齊遣歷陽王景安將兵應之詔以炅為江北道大都督總衆軍以討龍升斬之景安退走盡復江北之地 是歲突厥求昏於齊六年春正月壬戍朔周齊公憲等七人進爵為王 己巳周主享太廟乙亥耕籍田|{
	籍秦昔翻}
壬子上享太廟甲申廣陵金城降|{
	去年九月樊毅克廣陵楚子城其金城至是始降降戶江翻}
二月壬午朔日有食之 乙未齊主還鄴|{
	去年十月如晉陽至是始還}
丁酉周紀國公賢等六人進爵為王 辛亥上耕藉

田|{
	梁初依宋齊以正月用事天監十二年武帝以為啓蟄而耕書云以殷仲春藉田理在建卯於是改用二月陳因而不改藉而亦翻}
齊朔州行臺南安王思好本高氏養子驍勇得邉鎮人心|{
	驍堅堯翻}
齊主使嬖臣斫骨光弁至州|{
	斫骨虜複姓嬖卑義翻又慱計翻}
光弁不禮於思好思好怒遂反云欲入除君側之惡進軍至陽曲|{
	五代志太原郡汾陽縣舊曰陽曲}
自號大丞相武衛將軍趙海在晉陽倉猝不暇奏矯詔發兵拒之帝聞變使尚書令唐邕等馳之晉陽|{
	之往也}
辛丑帝勒兵繼進未至思好軍敗投水死其麾下二千人劉桃枝圍之且殺且招終不降以至於盡先是有人告思好謀反|{
	降戶江翻先悉薦翻}
韓長鸞女適思好子奏言是人誣告貴臣不殺無以息後乃斬之思好既誅告者弟伏闕下求贈官長鸞不為通|{
	通鑑言齊嬖倖壅蔽之禍為于偽翻}
丁未齊王還鄴甲寅以以唐邕為録尚書事 乙卯周主如雲陽宫 丙辰周大赦 庚申周叱奴太后有疾|{
	叱奴虜複姓}
三月辛酉周主還長安癸酉太后殂帝居倚廬|{
	陸德明曰廬倚東牆而為之故曰倚廬 考異曰隋書張衡傳云武帝居憂與左右出獵衡露髪輿櫬切諫案帝居喪有禮疑衡自叙之妄}
朝夕進一溢米|{
	鄭玄曰二十两為溢於粟米之法一溢為米一升二十四分升之一溢音逸}
羣臣表請累旬乃止命太子摠釐庶政|{
	釐治也}
衛王直譛齊王憲於帝曰憲飲酒食肉無異平日帝曰吾與齊王異生|{
	異生謂異母也}
俱非正嫡特以吾故同袒括髪|{
	袒者肉袒括者結也杜預曰以麻約髪}
汝當愧之何論得失汝親太后之子特承慈愛但當自勉無論他人 夏四月乙卯齊遣侍中薛孤康買弔於周|{
	薛孤虜複姓}
且會葬初齊世祖為胡后造珠裙袴|{
	為于偽翻下同}
所費不可勝計|{
	勝音升}
為火所焚至是齊主復為穆后營之|{
	復扶又翻}
使商胡齎錦綵三萬與弔使偕往市珠|{
	使疏吏翻}
周人不與齊主竟自造之及穆后愛衰其侍婢馮小憐大幸拜為淑妃與齊主坐則同席出則並馬誓同生死五月庚申周葬文宣皇后於永固陵周主跣行至陵

所辛酉詔曰三年之喪逹於天子但軍國務重須自聼朝衰麻之節苫廬之禮率遵前典|{
	喪服小記斬衰括髪以麻寢苫居廬朝直遥翻衰七回翻苫息亷翻}
以申罔極|{
	詩父兮生我母兮鞠我欲報之德吴天罔極}
百僚宜依遺令既葬而除|{
	除服也}
公卿固請依權制帝不許卒申三年之制|{
	卒子恤翻}
五服之内亦令依禮|{
	五服者斬衰三年服齊衰期年服大功九月服小功五月服緦麻三月服}
庚午齊大赦 齊人恐陳師渡淮使皮景和屯西兖州以備之|{
	西兖州治定陶}
丙子周禁佛道二教經像悉毁|{
	經謂二教之書像謂佛像天尊}
罷沙門道士並令還俗并禁諸淫祀非祀典所載者盡除之 六月壬辰周弘正卒|{
	卒子恤翻}
壬子周更鑄五行大布錢一當十與布泉並行|{
	行布泉見一百六十八卷世祖天嘉元年更工衡翻}
戊午周立通道觀以壹聖賢之教|{
	觀古玩翻}
秋七月庚申周主如雲陽以右宫正尉遲運兼司武|{
	左傳宋平公曰司武而梏於朝杜預注司武司馬周建六官已有大司馬司武盖其屬也尉紆勿翻}
與薛公長孫覽輔太子守長安|{
	長知两翻}
初帝取衛王直第為東宫|{
	建德元年立太子始建東宫}
使直自擇所居直歷觀府署無如意者末取廢陟屺寺欲居之|{
	陟屺寺取望母為名直意欲以同母感動周主}
齊王憲謂直曰弟子孫多此無乃小|{
	屺墟里翻補辨翻}
直曰一身尚不自容何論子孫直嘗從帝校獵而亂行帝對衆撻之直積怨憤因帝在外遂作亂乙酉帥其黨襲肅章門|{
	肅章宮門名唐長安太極宮太極殿後两儀殿前中為朱明門東則䖍化門西則肅章門盖周遺制帥讀曰率}
長孫覽懼奔詣帝所尉遲運偶在門中直兵奄至手自闔門直黨與運争門斫傷運指僅而得閉直不得入縱火焚門運恐火盡直黨得進取宮中材木及牀榻以益火膏油灌之火轉熾久之直不得進乃退運帥留守兵因其退而擊之|{
	帥讀曰率下同}
直大敗帥百餘騎奔荆州|{
	欲就梁騎奇寄翻}
戊子帝還長安八月辛卯擒直廢為庶人囚於别宫尋殺之以尉遲運為大將軍賜賚甚厚丙申周主復如雲陽|{
	復扶又翻}
癸丑齊主如晉陽甲辰齊以高勱為尚書右僕射|{
	勱音邁}
九月庚申周主如同州 冬十月丙申周遣御正弘農楊尚希禮部盧愷來聘|{
	武成元年增御正四人位上大夫保定四年又改禮部為司宗大司禮為禮部杜佑曰周制禮部中大夫屬春官}
愷柔之子也|{
	盧柔仕魏為中書監}
甲寅周主如蒲州丙辰如同州十一月甲戍還長安 十二月戊戍以吏部尚書王瑒為右僕射度支尚書孔奐為吏部尚書瑒冲之子也|{
	瑒雉杏翻又音暢度徒洛翻}
時新復淮泗攻戰降附功賞紛紜|{
	攻戰叙其功降附叙其賞降戶江翻}
奐識鑒精敏不受請託事無凝滯人皆悦服湘州刺史始興王叔陵屢諷有司求為三公奐曰衮章之職本以德舉|{
	三公一命衮命服身之章也}
未必皇枝因以白帝帝曰始興那忽望公且朕兒為公須在鄱陽王後|{
	言世祖之子當先為公}
奐曰臣之所見亦如聖旨 齊定州刺史南陽王綽喜為殘虐|{
	此定州治中山喜許既翻}
嘗出行見婦人抱兒奪以飼狗婦人號哭|{
	飼祥吏翻號戶高翻}
綽怒以兒血塗婦人縱狗使食之常云我學文宣伯之為人齊主聞之鎻詣行在至而宥之問在州何事最樂|{
	樂音洛下同}
對曰多聚蠍於器置狙其中觀之極樂帝即命夜索蠍一斗比曉得三二升置浴斛使人裸臥斛中號叫宛轉帝與綽臨觀喜噱不已|{
	蠍許竭翻螫人蟲渡淮以北即有之通俗文長尾曰蠆短尾曰蠍索山客翻比必寐翻及也浴斛浴器也裸郎果翻赤體也號戶刀翻噱其虐翻嗢噱笑不止}
因讓綽曰如此樂事何不馳驛奏聞|{
	樂音洛}
由是有寵拜大將軍朝夕同戱韓長鸞疾之是歲出為齊州刺史將使人誣告其反奏云此犯國法不可赦帝不忍明誅使寵胡何猥薩與之手搤而殺之|{
	薩桑割翻搤於革翻}


資治通鑑卷一百七十一
