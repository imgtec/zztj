<!DOCTYPE html PUBLIC "-//W3C//DTD XHTML 1.0 Transitional//EN" "http://www.w3.org/TR/xhtml1/DTD/xhtml1-transitional.dtd">
<html xmlns="http://www.w3.org/1999/xhtml">
<head>
<meta http-equiv="Content-Type" content="text/html; charset=utf-8" />
<meta http-equiv="X-UA-Compatible" content="IE=Edge,chrome=1">
<title>資治通鑒_110-資治通鑑卷一百九_110-資治通鑑卷一百九</title>
<meta name="Keywords" content="資治通鑒_110-資治通鑑卷一百九_110-資治通鑑卷一百九">
<meta name="Description" content="資治通鑒_110-資治通鑑卷一百九_110-資治通鑑卷一百九">
<meta http-equiv="Cache-Control" content="no-transform" />
<meta http-equiv="Cache-Control" content="no-siteapp" />
<link href="/img/style.css" rel="stylesheet" type="text/css" />
<script src="/img/m.js?2020"></script> 
</head>
<body>
 <div class="ClassNavi">
<a  href="/24shi/">二十四史</a> | <a href="/SiKuQuanShu/">四库全书</a> | <a href="http://www.guoxuedashi.com/gjtsjc/"><font  color="#FF0000">古今图书集成</font></a> | <a href="/renwu/">历史人物</a> | <a href="/ShuoWenJieZi/"><font  color="#FF0000">说文解字</a></font> | <a href="/chengyu/">成语词典</a> | <a  target="_blank"  href="http://www.guoxuedashi.com/jgwhj/"><font  color="#FF0000">甲骨文合集</font></a> | <a href="/yzjwjc/"><font  color="#FF0000">殷周金文集成</font></a> | <a href="/xiangxingzi/"><font color="#0000FF">象形字典</font></a> | <a href="/13jing/"><font  color="#FF0000">十三经索引</font></a> | <a href="/zixing/"><font  color="#FF0000">字体转换器</font></a> | <a href="/zidian/xz/"><font color="#0000FF">篆书识别</font></a> | <a href="/jinfanyi/">近义反义词</a> | <a href="/duilian/">对联大全</a> | <a href="/jiapu/"><font  color="#0000FF">家谱族谱查询</font></a> | <a href="http://www.guoxuemi.com/hafo/" target="_blank" ><font color="#FF0000">哈佛古籍</font></a> 
</div>

 <!-- 头部导航开始 -->
<div class="w1180 head clearfix">
  <div class="head_logo l"><a title="国学大师官网" href="http://www.guoxuedashi.com" target="_blank"></a></div>
  <div class="head_sr l">
  <div id="head1">
  
  <a href="http://www.guoxuedashi.com/zidian/bujian/" target="_blank" ><img src="http://www.guoxuedashi.com/img/top1.gif" width="88" height="60" border="0" title="部件查字,支持20万汉字"></a>


<a href="http://www.guoxuedashi.com/help/yingpan.php" target="_blank"><img src="http://www.guoxuedashi.com/img/top230.gif" width="600" height="62" border="0" ></a>


  </div>
  <div id="head3"><a href="javascript:" onClick="javascript:window.external.AddFavorite(window.location.href,document.title);">添加收藏</a>
  <br><a href="/help/setie.php">搜索引擎</a>
  <br><a href="/help/zanzhu.php">赞助本站</a></div>
  <div id="head2">
 <a href="http://www.guoxuemi.com/" target="_blank"><img src="http://www.guoxuedashi.com/img/guoxuemi.gif" width="95" height="62" border="0" style="margin-left:2px;" title="国学迷"></a>
  

  </div>
</div>
  <div class="clear"></div>
  <div class="head_nav">
  <p><a href="/">首页</a> | <a href="/ShuKu/">国学书库</a> | <a href="/guji/">影印古籍</a> | <a href="/shici/">诗词宝典</a> | <a   href="/SiKuQuanShu/gxjx.php">精选</a> <b>|</b> <a href="/zidian/">汉语字典</a> | <a href="/hydcd/">汉语词典</a> | <a href="http://www.guoxuedashi.com/zidian/bujian/"><font  color="#CC0066">部件查字</font></a> | <a href="http://www.sfds.cn/"><font  color="#CC0066">书法大师</font></a> | <a href="/jgwhj/">甲骨文</a> <b>|</b> <a href="/b/4/"><font  color="#CC0066">解密</font></a> | <a href="/renwu/">历史人物</a> | <a href="/diangu/">历史典故</a> | <a href="/xingshi/">姓氏</a> | <a href="/minzu/">民族</a> <b>|</b> <a href="/mz/"><font  color="#CC0066">世界名著</font></a> | <a href="/download/">软件下载</a>
</p>
<p><a href="/b/"><font  color="#CC0066">历史</font></a> | <a href="http://skqs.guoxuedashi.com/" target="_blank">四库全书</a> |  <a href="http://www.guoxuedashi.com/search/" target="_blank"><font  color="#CC0066">全文检索</font></a> | <a href="http://www.guoxuedashi.com/shumu/">古籍书目</a> | <a   href="/24shi/">正史</a> <b>|</b> <a href="/chengyu/">成语词典</a> | <a href="/kangxi/" title="康熙字典">康熙字典</a> | <a href="/ShuoWenJieZi/">说文解字</a> | <a href="/zixing/yanbian/">字形演变</a> | <a href="/yzjwjc/">金 文</a> <b>|</b>  <a href="/shijian/nian-hao/">年号</a> | <a href="/diming/">历史地名</a> | <a href="/shijian/">历史事件</a> | <a href="/guanzhi/">官职</a> | <a href="/lishi/">知识</a> <b>|</b> <a href="/zhongyi/">中医中药</a> | <a href="http://www.guoxuedashi.com/forum/">留言反馈</a>
</p>
  </div>
</div>
<!-- 头部导航END --> 
<!-- 内容区开始 --> 
<div class="w1180 clearfix">
  <div class="info l">
   
<div class="clearfix" style="background:#f5faff;">
<script src='http://www.guoxuedashi.com/img/headersou.js'></script>

</div>
  <div class="info_tree"><a href="http://www.guoxuedashi.com">首页</a> > <a href="/SiKuQuanShu/fanti/">四库全书</a>
 > <h1>资治通鉴</h1> <!--         下载:【右键另存为】即可 --></div>
  <div class="info_content zj clearfix">
  
<div class="info_txt clearfix" id="show">
<center style="font-size:24px;">110-資治通鑑卷一百九</center>
    資治通鑑卷一百九   宋 司馬光 撰<br />
<br />
  胡三省 音註<br />
<br />
  晉紀三十一【強圉作噩一年】<br />
<br />
  安皇帝甲【諱德宗字德宇孝武帝長子也諡法好和不爭曰安又曰生而少斷曰安帝即位後桓玄纂奪劉裕反正南征北伐事多而中原亦多事通鑑所書凡十卷故以十干書卷數】<br />
<br />
  隆安元年春正月己亥朔帝加元服改元以左僕射王珣為尚書令領軍將軍王國寶為左僕射領選【領選者領吏部選選須戀翻】仍加後將軍丹楊尹會稽王道子悉以東宮兵配國寶使領之【會工外翻】 燕范陽王德求救於秦秦兵不出鄴中恟懼【恟許洪翻】賀賴盧自以魏王珪之舅不受東平公儀節度由是與儀有隙儀司馬丁建隂與德通從而構間之【間古莧翻】射書入城中言其狀【射而亦翻】甲辰風霾晝晦【霾謨皆翻風雨土曰霾】賴盧營失火建言於儀曰賴盧燒營為變矣儀以為然引兵退賴盧聞之亦退建帥其衆詣德降【帥讀曰率降戶江翻】且言儀師老可擊德遣桂陽王鎮南安王青帥騎七千追擊魏軍大破之【師克在和將帥不和敗之本也】燕主寶使左衛將軍慕輿騰攻博陵殺魏所置守宰王建等攻信都六十餘日不下士卒多死庚申魏王珪自攻信都壬戌夜燕宜都王鳳踰城犇中山【鳳知珪至膽破而走】癸亥信都降魏 凉王光以西秦王乾歸數反覆【謂乾歸既稱藩於光而悔之也數所角翻】舉兵伐之乾歸羣下請東奔成紀以避之【成紀縣自漢以來屬天水郡治小坑川唐併顯親縣入成紀縣移成紀縣治顯親川】乾歸曰軍之勝敗在於巧拙不在衆寡光兵雖衆而無法其弟延勇而無謀不足憚也且其精兵盡在延所延敗光自走矣光軍于長最遣太原公纂等帥步騎三萬攻金城乾歸帥衆二萬救之未至纂等拔金城光又遣其將梁㳟等以甲卒萬餘出陽武下峽【陽武下峽在高平西河水所經也將即亮翻】與秦州刺史沒奕干攻其東天水公延以枹罕之衆攻臨洮武始河關皆克之【臨洮縣漢屬隴西郡惠帝分屬狄道郡武始郡故狄道縣地河關縣前漢屬金城郡後漢屬隴西郡晉屬狄道郡枹音膚洮土刀翻】乾歸使人紿延云【紿待亥翻】乾歸衆潰犇成紀延欲引輕騎追之司馬耿稚諫曰乾歸勇略過人安肯望風自潰前破王廣楊定皆羸師以誘之【破楊定見上卷孝武太元十九年太元十一年王廣為鮮卑匹蘭所執送於後秦此時乾歸未統國事也乾歸破廣當在乞伏國仁之時稚直利翻羸倫為翻】今告者視高色動殆必有姦宜整陳而前使步騎相屬【陳讀曰陣屬之欲翻】俟諸軍畢集然後擊之無不克矣延不從進與乾歸遇延戰死稚與將軍姜顯收散卒還屯枹罕光亦引兵還姑臧 秃髪烏孤自稱大都督大將軍大單于西平王【單音蟬】大赦改元太初治兵廣武攻凉金城克之凉主光遣將軍竇苟伐之戰于街亭凉兵大敗 燕主寶聞魏王珪攻信都出屯深澤【深澤縣前漢屬涿郡後漢屬安平國晉屬博陵郡宋白曰深澤縣以界内水澤深廣為名】遣趙王麟攻楊城【郡國志中山蒲隂縣有楊城】殺守兵三百寶悉出珍寶及宫人募郡國羣盜以擊魏二月己巳朔珪還屯楊城沒根兄子醜提為并州監軍聞其叔父降燕懼誅帥所部兵還國作亂【監工銜翻降戶江翻帥讀曰率下同】珪欲北還遣其國相涉延求和於燕且請以其弟為質【相息亮翻質音致】寶聞魏有内難不許【兵法曰知彼知己百戰不殆慕容寶徒欲乘拓跋珪之有内釁而困之而不知己之才略不足辦也難乃旦翻】使冗從僕射蘭真責珪負恩【冗而隴翻從才用翻】悉發其衆步卒十二萬騎三萬七千屯於曲陽之栢肆【此趙國之下曲陽縣也有栢肆塢隋開皇十六年置栢肆縣後廢入常山槀城縣魏書帝紀作鉅鹿之栢肆塢按地形志鉅鹿郡治曲陽】營於滹沲水北以邀之【滹音呼沲徒河翻】丁丑魏軍至營於水南寶濳師夜濟募勇敢萬餘人襲魏營寶陳於營北以為之援【陳讀曰陣下同】募兵因風縱火急擊魏軍魏軍大亂珪驚起棄營跣走燕將軍乞特真帥百餘人至其帳下得珪衣鞾【鞾許戈翻】既而募兵無故自驚互相斫射【射而亦翻】珪於營外望見之乃擊鼓收衆左右及中軍將士稍稍來集多布火炬於營外縱騎衝之募兵大敗【敵出其不意故走見敵之不整乃還戰善用兵者固觀變而動也】還赴寶陳寶引兵復渡水北戊寅魏整衆而至與燕相持燕軍奪氣寶引還中山魏兵隨而擊之燕兵屢敗寶懼棄大軍帥騎二萬犇還時大風雪凍死者相枕【枕職任翻】寶恐為魏軍所及命士卒皆棄袍仗兵器數十萬寸刃不返燕之朝臣將卒降魏及為魏所係虜者甚衆【朝直遥翻將即亮翻降戶江翻】先是張衮嘗為魏王珪言燕秘書監崔逞之材【據張衮傳衮未嘗與逞相識也聞其才而稱之先悉薦翻】珪得之甚喜以逞為尚書使錄三十六曹【漢光武分尚書為六曹置郎三十四人并左右丞為三十六人至魏尚書郎有殿中吏部駕部金部虞曹比部南主客祠部度支庫部農部水部儀曹三公倉部民曹二千石中兵外兵都兵别兵考功定課凡二十三郎明帝青龍二年置都官騎兵合二十五郎晉武帝罷農部定課置直事殿中祠部儀曹吏部三公比部金部倉部度支都官二千石左民右民虞曹屯田起部水部左右主客駕部車部庫部左右中兵左右外兵别兵都兵騎兵左右士北主客南主客凡三十四曹後又置運曹凡三十五曹置郎二十三人更相統攝今魏又增為三十六曹】任以政事魏軍士有自栢肆亡歸者言大軍敗散不知王處道過晉陽晉陽守將封真因起兵攻并州刺史曲陽侯素延素延擊斬之南安公順守雲中聞之欲自攝國事幢將代人莫題曰此大事不可輕爾宜審待後問不然為禍不細順乃止順什翼鞬之孫也【幢直江翻將即亮翻鞬居言翻】賀蘭部帥附力眷紇隣部帥匿物尼紇奚部帥叱奴根皆舉兵反【紇戶骨翻帥所類翻】順討之不克珪遣安遠將軍庾岳帥萬騎還討三部皆平之國人乃安珪欲撫慰新附深悔參合之誅【事見上卷孝武帝太元二十年珪以燕人懲參合之禍苦戰不下故深悔之】素延坐討反者殺戮過多免官以奚牧為并州刺史牧與東秦主興書稱頓首與之均禮【時乞伏氏建國隴西號秦故史書姚秦為東秦以别之】興怒以告珪珪為之殺牧【為于偽翻】己卯夜燕尚書郎慕輿皓謀弑燕主寶立趙王麟不克斬關出犇魏麟由是不自安【為麟犇西山張本】 三月燕以儀同三司武鄉張崇為司空【石勒分上黨置武鄉郡及武鄉縣唐遼州榆社縣即其地】 初燕清河王會聞魏軍東下表求赴難【難乃旦翻】燕主寶許之會初無去意【初無去龍城之意也】使征南將軍庫傉官偉建威將軍餘崇將兵五千為前鋒崇嵩之子也【餘嵩見上卷孝武帝太元二十一年傉奴沃翻】偉等頓盧龍近百日【遼東新昌縣有盧龍山唐為平州盧龍縣慕容令所謂守肥如之險即其地也此遼東新昌後人置於漢遼西郡界非漢舊郡縣地也近其靳翻】無食噉馬牛且盡會不發寶怒累詔切責會不得已以治行簡練為名復留月餘【治直之翻復扶又翻】時道路不通偉欲使輕軍前行通道偵魏彊弱且張聲勢【偵丑鄭翻】諸將皆畏避不欲行餘崇奮曰今巨寇滔天京都危逼【京都謂中山】匹夫猶思致命以救君父諸君荷國寵任而更惜生乎【荷下可翻】若社稷傾覆臣節不立死有餘辱諸君安居於此崇請當之偉喜簡給步騎五百人崇進至漁陽遇魏千餘騎崇謂其衆曰彼衆我寡不擊則不得免乃鼓譟直進崇手殺十餘人魏騎潰去崇亦引還斬首獲生具言敵中濶狹衆心稍振會乃上道徐進【上時掌翻】是月始達薊城【薊音計】魏圍中山既久城中將士皆思出戰征北大將軍隆言於寶曰涉珪雖屢獲小利然頓兵經年【涉歲為經年去年十一月魏攻中山】凶勢沮屈【沮在呂翻】士馬死傷大半人心思歸諸部離解【謂賀蘭紇鄰紇奚三部】正是可破之時也加之舉城思奮若因我之鋭乘彼之衰往無不克如其持重不決將卒氣喪【將即亮翻喪息浪翻】日益困逼事久變生後雖欲用之不可得也寶然之而衛大將軍麟每沮其議【麟有異志故沮隆議】隆成列而罷者前後數四寶使人請於魏主珪欲還其弟觚【觚留燕事見一百七卷孝武太元十六年】割常山以西皆與魏以求和【常山以西并州之地也】珪許之既而寶悔之己酉珪如盧奴【魏書地形志中山郡治盧奴酈道元曰盧奴城内西北隅有水淵而不流南北一百步東西百餘步水色正黑曰盧不流曰奴故城以此得名】辛亥復圍中山【杜佑曰後燕都中山今博陵郡唐昌縣唐昌本漢苦陘縣章帝改漢昌曹魏改魏昌隋改隋昌唐武德中改唐昌復扶又翻】燕將士數千人俱自請於寶曰今坐守窮城終於困弊臣等願得一出樂戰【士皆赴死願戰為樂戰也樂音洛】而陛下每抑之此為坐自摧敗也且受圍歷時無他奇變徒望積久寇賊自退今内外之勢彊弱懸絶彼必不自退明矣宜從衆一决寶許之隆退而勒兵召諸參佐謂之曰皇威不振寇賊内侮臣子同恥義不顧生今幸而破賊吉還固善若其不幸亦使吾志節獲展卿等有北見吾母者為吾道此情也【隆初鎮龍城與母俱北及垂召隆伐魏其母留龍城為于偽翻】乃被甲上馬詣門俟命麟復固止寶【被皮義翻復扶又翻】衆大忿恨隆涕泣而還【還從宣翻又如字下同】是夜麟以兵刼左衛將軍北地王精使帥禁兵弑寶【帥讀曰率】精以義拒之麟怒殺精出犇西山依丁零餘衆【中山西北二百里有狼山自狼山而西南連常山山谷深險漢末黑山張燕五代孫方簡兄弟皆依阻其地丁零餘衆翟真之黨也為燕所敗退聚西山西山曲陽之西山也】於是城中人情震駭寶不知麟所之【之往也】以清河王會軍在近恐麟奪會軍先據龍城乃召隆及驃騎大將軍農【驃匹妙翻騎奇寄翻】謀去中山走保龍城隆曰先帝櫛風沐雨以成中興之業崩未朞年而天下大壞豈得不謂之孤負邪今外寇方盛而内難復起【難乃旦翻復扶又翻下復朝同】骨肉乖離百姓疑懼誠不可以拒敵北遷舊都亦事之宜然龍川地狹民貧【龍川即謂和龍之地】若以中國之意取足其中復朝夕望有大功此必不可若節用愛民務農訓兵數年之中公私充實而趙魏之間厭苦寇暴民思燕德庶幾返斾克復故業【幾居希翻】如其未能則憑險自固猶足以優游養鋭耳寶曰卿言盡理朕一從卿意耳【隆策固善其如運命何兵家因敗為成隆之智不足以及此也使寶始終一從隆之說猶可以免蘭汗之禍】遼東高撫善卜筮素為隆所信厚私謂隆曰殿下北行終不能達太妃亦不可得見若使主上獨往殿下潛留於此必有大功隆曰國有大難【難乃旦翻】主上蒙塵且老母在北吾得北首而死猶無所恨卿是何言也【首式救翻】乃遍召僚佐問其去留唯司馬魯恭參軍成岌願從【從才用翻】餘皆欲留隆竝聽之農部將谷會歸說農曰【說輸芮翻】城中之人皆涉珪參合所殺者父兄子弟泣血踊躍欲與魏戰而為衛軍所抑【慕容麟為衛大將軍故稱之為衛軍】今聞主上當北遷皆曰得慕容氏一人奉而立之以與魏戰死無所恨大王幸留此以副衆望擊退魏軍撫寧畿甸奉迎大駕亦不失為忠臣也農欲殺歸而惜其材力謂之曰必如此以望生不如就死【農隆皆號為有智略而所見類如此天之廢燕智者失其智矣】壬子夜寶與太子策遼西王農高陽王隆長樂王盛等萬餘騎出赴會軍河間王熙勃海王朗博陵王鑒皆幼不能出城隆還入迎之自為鞁乘【鞁平義翻說文曰車駕具】俱得免燕將王沈等降魏【沈持林翻】樂浪王惠中書侍郎韓範員外郎段宏太史令劉起等帥工伎三百犇鄴【樂浪音洛琅帥讀曰率伎渠綺翻】中山城中無主百姓惶惑東門不閉魏王珪欲夜入城冠軍將軍王建志在虜掠乃言恐士卒盗府庫物請俟明旦珪乃止燕開封公詳從寶不成城中立以為主閉門拒守珪盡衆攻之連日不拔使人登巢車【杜預曰巢車車上為櫓陸德明曰兵車高如巢以望敵也杜佑曰以八輪車上樹高竿竿上安轆轤以繩挽板屋上竿首以窺城中板屋方四尺高五尺有十二孔四面别布車可進退圜城而行於營中遠視如鳥之巢亦謂之巢車】臨城諭之曰慕容寶已棄汝走汝曹百姓空自取死欲誰為乎【為于偽翻】皆曰羣小無知恐復如參合之衆【復扶又翻下復出同】故苟延旬月之命耳珪顧王建而唾其面【唾土賀翻王建既鼔成參合之誅又沮止珪乘夜入中山失計者再故唾其面】使中領將軍長孫肥左將軍李栗將三千騎追寶至范陽不及破其新城戍而還【前漢志中山國有北新城縣郡國志涿郡有北新城縣晉省水經註新城縣在武遂縣南燕督亢之地也】 甲寅尊皇太后李氏為太皇太后戊午立皇后王氏 燕主寶出中山與趙王麟遇于城【攷之字書無字有阱字疾郢翻】麟不意寶至驚駭帥其衆奔蒲隂【蒲隂縣屬中山郡前漢之曲逆縣也後章帝醜其名改曰蒲隂帥讀曰率下同】 復出屯望都【復扶又翻】土人頗供給之慕容詳遣兵掩擊麟獲其妻子麟脱走入山中甲寅寶至薊殿中親近散亡略盡惟高陽王隆所領數百騎為宿衛清河王會帥騎卒二萬迎于薊南寶怪會容止怏怏有恨色【恨不得為嗣也事見上卷孝武帝太元二十一年怏於兩翻】密告隆及遼西王農農隆俱曰會年少【少詩照翻】專任方面習驕所致豈有他也臣等當以禮責之寶雖從之然猶詔解會兵以屬隆隆固辭乃減會兵分給農隆又遣西河公庫傉官驥帥兵三千助守中山【傉奴沃翻】丙辰寶盡徙薊中府庫北趣龍城【趣七喻翻】魏石河頭引兵追之戊午及寶於夏謙澤【石河頭時屯漁陽夏謙澤在薊北二百餘里】寶不欲戰清河王會曰臣撫教士卒惟敵是求今大駕蒙塵人思效命而虜敢自送衆心忿憤兵法曰歸師勿遏又曰置之死地而後生【孫武子之言】今我皆得之何患不克若其捨去賊必乘人或生餘變寶乃從之會整陳與魏兵戰【陳讀曰陣】農隆等將南來騎衝之魏兵大敗追犇百餘里斬首數千級隆又獨追數十里而還謂故吏留臺治書陽璆曰【留臺治書為留臺治書侍御史也燕建留臺於龍城見一百七卷孝武太元十四年隆時録留臺故璆為故吏璆渠幽翻】中山城中積兵數萬不得展吾意今日之捷令人遺恨因慷慨流涕會既敗魏兵【敗蒲邁翻】矜狠滋甚隆屢訓責之會益忿恚【狠戶墾翻恚於避翻】會以農隆皆嘗鎮龍城【孝武太元十年農鎮龍城十四年隆代農】屬尊位重名望素出已右恐至龍城權政不復在已【復扶又翻】又知終無為嗣之望【以寶違垂命立策為太子也】乃謀作亂幽平之兵皆懷會恩不樂屬二王【樂音洛】請於寶曰清河王勇略高世臣等與之誓同生死願陛下與皇太子諸王留薊宫臣等從王南解京師之圍還迎太駕寶左右皆惡會【惡烏路翻】言於寶曰清河王不得為太子神色甚不平且其才武過人善收人心陛下若從衆請臣恐解圍之後必有衛輒之事【衛靈公世子蒯聵出奔靈公立其子輒靈公卒輒立蒯聵復入輒拒而不納】寶乃謂衆曰道通年少【會字道通少詩照翻】才不及二王【二王謂農隆】豈可當專征之任且朕方自統六師仗會以為羽翼何可離左右也【離力智翻】衆不悦而退左右勸寶殺會侍御史仇尼歸聞之告會曰大王所恃者父父已異圖所仗者兵兵已去手欲於何所自容乎不如誅二王廢太子大王自處東宮【處昌呂翻】兼將相之任【將即亮翻相息亮翻】以匡復社稷此上策也會猶豫未許寶謂農隆曰觀道通志趣必反無疑宜早除之農隆曰今寇敵内侮中土紛紜社稷之危有如累卵會鎮撫舊都遠赴國難【難乃旦翻】其威名之重足以震動四鄰逆狀未彰而遽殺之豈徒傷父子之恩亦恐大損威望寶曰會逆志已成卿等慈恕不忍早殺恐一旦為變必先害諸父然後及吾至時勿悔自負也會聞之益懼夏四月癸酉寶宿廣都黄榆谷【魏收地形志廣都縣屬建德郡在漢北平白狼縣界隋省入遼西柳城縣】會遣其黨仇尼歸吳提染干帥壯士二十餘人【帥讀曰率下同】分道襲農隆殺隆於帳下農被重創【被皮義翻創初良翻】執仇尼歸逃入山中會以仇尼歸被執事終顯發乃夜詣寶曰農隆謀逆臣已除之寶欲討會陽為好言以安之曰吾固疑二王久矣除之甚善甲戌旦會立仗嚴備乃引道會欲棄隆喪餘崇涕泣固請乃聽載隨軍農出自歸寶呵之曰何以自負邪【寶昜責農而以前言相擿發】命執之行十餘里寶顧召羣臣食且議農罪會就坐【坐徂臥翻】寶目衛軍將軍慕輿騰使斬會傷其首不能殺會走赴其軍勒兵攻寶寶帥數百騎馳二百里晡時至龍城會遣騎追至石城不及【石城縣漢屬北平郡後魏屬建德郡隋併入柳城縣】乙亥會遣仇尼歸攻龍城寶夜遣兵襲擊破之會遣使請誅左右佞臣并求為太子【使疏吏翻】寶不許會盡收乘輿器服以後宮分給將帥【乘繩證翻將即亮翻帥所類翻】署置百官自稱皇太子錄尚書事引兵向龍城以討慕輿騰為名丙子頓兵城下寶臨西門會乘馬遥與寶語寶責讓之會命軍士向寶大譟以耀威城中將士皆憤怒向暮出戰大破之會兵死傷大半走還營侍御郎高雲夜帥敢死士百餘人襲會軍【寶之為太子雲以武藝給事侍東宫拜侍御郎】會衆皆潰會將十餘騎犇中山開封公詳殺之寶殺會母及其三子丁丑寶大赦凡與會同謀者皆除罪復舊職論功行賞拜將軍封侯者數百人遼西王農骨破見腦寶手自裹創【創初良翻】僅而獲濟以農為左僕射尋拜司空領尚書令餘崇出自歸寶嘉其忠拜中堅將軍使典宿衛贈高陽王隆司徒諡曰康寶以高雲為建威將軍封夕陽公養以為子雲高句麗之支屬也【高句麗自云高陽氏之後裔故以高為氏句如字又音駒麗力知翻】燕王皝破高句麗徙於青山【破高句麗見九十七卷成帝咸康八年青山遼西徒河縣之青山也】由是世為燕臣雲沈厚寡言【沈持林翻】時人莫知惟中衛將軍長樂馮跋【魏收曰漢高帝置信都郡景帝二年為廣川國明帝更名樂成國安帝改為安平國晉改為長樂郡考之晉志有安平而無長樂不知何時更名也樂音洛】奇其志度與之為友【高雲馮跋事始見於此為後得燕張本】跋父和事西燕王永為將軍永敗徙和龍 僕射王國寶建威將軍王緒依附會稽王道子【會工外翻】納賄窮奢不知紀極惡王恭殷仲堪【惡烏路翻】勸道子裁損其兵權中外恟恟不安【恟許拱翻】恭等各繕甲勒兵表請北伐道子疑之詔以盛夏妨農悉使解嚴恭遣使與仲堪謀討國寶等【遣使疏吏翻】桓玄以仕不得志欲假仲堪兵勢以作亂【玄仕不得志事見孝武太元十七年】乃說仲堪曰國寶與君諸人素已為對【事始一百七卷孝武太元十五年對敵也說輸芮翻下同】唯患相斃之不速耳今既執大權與王緒相表裏其所迴易無不如志孝伯居元舅之地必未敢害之【王恭字孝伯孝武王皇后之兄弟也】君為先帝所拔超居方任人情皆以君為雖有思致非方伯才【亦見太元十五年思相吏翻】彼若發詔徵君為中書令用殷覬為荆州【南蠻校尉資次可為荆州故云覬音冀】君何以處之【處昌呂翻】仲堪曰憂之久矣計將安出玄曰孝伯疾惡深至君宜潛與之約興晉陽之甲以除君側之惡【春秋公羊傳曰趙鞅興晉陽之甲以除君側之惡】東西齊舉【江陵在西京口在東故曰東西齊舉也】玄雖不肖願帥荆楚豪傑荷戈先驅【帥讀曰率荷下可翻】此桓文之勳也仲堪心然之乃外結雍州刺史郗恢【雍於用翻郗丑之翻】内與從兄南蠻校尉覬南郡相陳留江績謀之【從才用翻南蠻府南郡相與荆州刺史府同治江陵】覬曰人臣當各守職分朝廷是非豈藩屏之所制也【分扶問翻屏必郢翻】晉陽之事不敢預聞仲堪固邀之覬怒曰吾進不敢同退不敢異績亦極言其不可覬恐績及禍於坐和解之【坐徂臥翻】績曰大丈夫何至以死相脅邪江仲元行年六十但未獲死所耳【江績字仲元】仲堪憚其堅正以楊佺期代之朝廷聞之徵績為御史中丞覬遂稱散發辭位【晉人多服寒食散其藥毒發或致死今千金方中有數方散悉亶翻】仲堪往省之【省悉井翻】謂覬曰兄病殊為可憂覬曰我病不過身死汝病乃當滅門宜深自愛勿以我為念郗恢亦不肯從仲堪疑未決會王恭使至【使疏吏翻】仲堪許之恭大喜甲戌恭上表罪狀國寶舉兵討之初孝武帝委任王珣及帝暴崩不及受顧命珣一旦失勢循默而已【循默者循常而無一言也】丁丑王恭表至内外戒嚴道子問珣曰二藩作逆卿知之乎珣曰朝政得失珣弗之預【朝直遥翻】王殷作難【難乃旦翻】何由可知王國寶惶懼不知所為遣數百人戍竹里【竹里今建康府竹篠鎮是其地在行宮城東北三十許里】夜遇風雨各散歸王緒說國寶矯相王之命召王珣車殺之以除時望因挾君相發兵以討二藩國寶許之【說輸芮翻相息亮翻車尺遮翻】珣至國寶不敢害更問計於珣珣曰王殷與卿素無深怨所競不過勢利之間耳國寶曰將曹爽我乎【謂珣如蔣濟說曹爽釋權而司馬懿終族之也事見七十五卷魏邵陵厲公嘉平元年】珣曰是何言歟卿寧有爽之罪王孝伯豈宣帝之儔邪又問計於曰昔桓公圍壽陽彌時乃克【見一百二卷海西公太和五年及一百三卷簡文帝咸安元年】今朝廷遣軍恭必城守若京口未拔而上流奄至君將何以待之國寶尤懼遂上疏解職詣闕待罪既而悔之詐稱詔復其本官道子闇懦欲求姑息【姑且也息止也姑息猶言且止】乃委罪國寶遣驃騎諮議參軍譙王尚之收國寶付廷尉【洪景伯曰諮議參軍晉江左初置因軍諮祭酒也其位在諸參軍之右驃匹妙翻騎奇寄翻】尚之恬之子也甲申賜國寶死斬緒於市遣使詣恭深謝愆失恭乃罷兵還京口國寶兄侍中愷驃騎司馬愉竝請解職道子以愷愉與國寶異母又素不恊皆釋不問戊子大赦殷仲堪雖許王恭猶豫不敢下聞國寶等死乃始抗表舉兵遣楊佺期屯巴陵【沈約曰巴陵縣晉武太康元年置屬長沙酈道元曰湘水北至巴丘山入江山在右岸有巴陵故城本吳之巴丘邸閤晉立巴陵縣後置建昌郡】道子以書止之仲堪乃還會稽世子元顯年十六有儁才為侍中說道子以王殷終必為患請潛為之備【說輸芮翻】道子乃拜元顯征虜將軍以其衛府及徐州文武悉配之【為元顯討王殷張本】 魏王珪以軍食不給命東平公儀去鄴徙屯鉅鹿積租楊城慕容詳出步卒六千人伺間襲魏諸屯【伺相吏翻間古莧翻】珪擊破之斬首五千生擒七百人皆縱之【縱之所以携中山城中之人心】 初張掖盧水胡沮渠羅仇匈奴沮渠王之後也【盧水胡分居安定張掖史各以其所居郡係之沮子余翻北史曰沮渠世居張掖臨松盧水】世為部帥【帥所類翻】凉王光以羅仇為尚書從光伐西秦及呂延敗死羅仇弟三河太守麴粥謂羅仇曰【呂光得凉州自號三河王此郡蓋光置也賢曰三河謂金城河賜支河湟河此郡當置於漢張掖金城郡界】主上荒耄信讒今軍敗將死【將即亮翻】正其猜忌智勇之時也吾兄弟必不見容與其死而無名不若勒兵向西平出苕藋【河西張氏置西平郡唐為鄯州之地苕藋地名在漢張掖郡番禾縣界藋徒弔翻番如淳音盤】奮臂一呼【呼火故翻】凉州不足定也羅仇曰誠如汝言然吾家世以忠孝著於西土寧使人負我我不忍負人也光果聽讒以敗軍之罪殺羅仇及麴粥羅仇弟子蒙遜雄傑有策略涉獵書史以羅仇麴粥之喪歸葬諸部多其族姻會葬者凡萬餘人蒙遜哭謂衆曰呂王昏荒無道多殺不辜吾之上世虎視河西【蒙遜之先世為匈奴左且渠河西匈奴左地也世居盧水為酋豪其高曾皆雄健有勇名】今欲與諸部雪二父之恥復上世之業何如衆咸稱萬歲遂結盟起兵攻凉臨松郡拔之【臨松郡張天錫置後周廢入張掖郡張掖縣】屯據金山【五代史志張掖刪丹縣有金山沮渠蒙遜事始此】司徒左長史王廞導之孫也【廞許金翻】以母喪居吳王恭<br />
<br />
  之討王國寶也版廞行吳國内史【以白版授官非朝命也】使起兵於東方【三吳皆在建康之東】廞使前吳國内史虞嘯父等入吳興義興召募兵衆【父音甫】赴者萬計未幾國寶死【幾居豈翻】恭罷兵符廞去職反喪服廞以起兵之際誅異已者頗多勢不得止遂大怒不承恭命使其子泰將兵伐恭牋於會稽王道子稱恭罪惡【將即亮翻會工外翻】道子以其牋送恭五月恭遣司馬劉牢之帥五千人擊泰斬之【帥讀曰率】又與廞戰於曲阿衆潰廞單騎走不知所在【騎奇寄翻】收虞嘯父下廷尉以其祖潭有功【虞潭有討蘇峻之功下遐嫁翻】免為庶人 燕庫傉官驥入中山與開封公詳相攻【慕容寶遣驥助守中山因與詳相攻傉奴沃翻】詳殺驥盡滅庫傉官氏又殺中山尹苻謨夷其族中山城無定主民恐魏兵乘之男女結盟人自為戰【使慕容農慕容隆留中山而用之未可知也】甲辰魏王珪罷中山之圍就穀河間督諸郡義租甲寅以東平公儀為驃騎大將軍都督中外諸軍事兖豫雍荆徐揚六州牧左丞相封衛王【驃匹妙翻騎奇寄翻雍於用翻】慕容詳自謂能却魏兵威德已振乃即皇帝位改元建始置百官以新平公可足渾潭為車騎大將軍尚書令【潭當作譚】殺拓跋觚以固衆心【觚先使燕為燕所留珪之弟也】鄴中官屬勸范陽王德稱尊號會有自龍城來者知燕主寶猶存乃止 凉王光遣太原公纂將兵擊沮渠蒙遜於怱谷破之【怱谷當在刪丹縣界】蒙遜逃入山中蒙遜從兄男成為凉將軍【從才用翻】聞蒙遜起兵亦合衆數千屯樂涫【樂涫縣漢屬酒泉郡後周廢入福禄縣涫姑歡翻又古玩翻】酒泉太守壘澄討男成兵敗澄死【壘姓澄名】男成進攻建康遣使說建康太守段業曰【說輸芮翻】呂氏政衰權臣擅命刑殺無常人無容處一州之地【處昌呂翻】叛者相望瓦解之形昭然在目百姓嗷然無所依附府君奈何以蓋世之才欲立忠於垂亡之國男成等既唱大義欲屈府君撫臨鄙州使塗炭之餘蒙來蘇之惠【書曰徯我后后來其蘇】何如業不從相持二旬外救不至郡人高逵史惠等勸業從男成之請業素與凉侍中房晷僕射王詳不平懼不自安乃許之男成等推業為大都督龍驤大將軍凉州牧建康公【驤思將翻】改元神璽【璽斯氏翻】以男成為輔國將軍委以軍國之任蒙遜帥衆歸業【帥讀曰率】業以蒙遜為鎮西將軍光命太原公纂將兵討業不克【將即亮翻】六月西秦王乾歸徵北河州刺史彭奚念為鎮衛將<br />
<br />
  軍以鎮西將軍屋弘破光為河州牧【屋弘當作屋引魏書官氏志内入諸姓有屋引氏後改為房氏張駿分興晉金城武始南安永晉大夏武成漢中為河州北河州乞伏氏所置也治枹罕鎮衛將軍劉聰所置】定州刺史翟瑥為興晉太守鎮枹罕【張茂分武興金城西平安故為定州興晉郡亦張氏置枹罕縣漢屬金城郡後漢屬隴西郡後又分屬西平郡張駿分屬晉興郡後又分置興晉郡瑥音温枹音膚】 秋七月慕容詳殺可足渾潭詳嗜酒奢淫不恤士民刑殺無度所誅王公以下五百餘人羣下離心城中饑窘詳不聽民出采稆【稆音呂禾不布種而自生曰稆】死者相枕【枕職任翻】舉城皆謀迎趙王麟詳遣輔國將軍張驤帥五千餘人督租於常山【驤思將翻帥讀曰率】麟自丁零入驤軍潛襲中山城門不閉執詳斬之麟遂稱尊號聽人四出采稆人既飽求與魏戰麟不從稍復窮餒【復扶又翻】魏王珪軍魯口遣長孫肥帥騎七千襲中山入其郛麟追至泒水【泒水在中山新市縣輿地志云盧奴城北臨滱水面泒河泒攻乎翻】為魏所敗而還【敗蒲邁翻還從宣翻又如字下同】八月丙寅朔魏主珪徙軍常山之九門【常山郡有九門縣】軍中大疫人畜多死將士皆思歸珪問疫於諸將對曰在者纔什四五珪曰此固天命將若之何四海之民皆可為國在吾所以御之耳何患無民羣臣乃不敢言遣撫軍大將軍略陽公遵襲中山入其郛而還 燕以遼西王農為都督中外諸軍事大司馬錄尚書事 凉散騎常侍太常西平郭黁善天文數術【散悉亶翻騎奇寄翻黁奴昆翻】國人信重之會熒惑守東井黁謂僕射王詳曰凉之分野將有大兵【分扶問翻】主上老病太子闇弱太原公凶悍【悍下罕翻又侯旰翻】一旦不諱禍亂必起吾二人久居内要彼常切齒將為誅首矣田胡王乞基部落最彊【田胡胡之一種也】二苑之人多其舊衆吾欲與公舉大事推乞基為主二苑之衆盡我有也【凉州治姑臧有東西苑城】得城之後徐更議之詳從之黁夜以二苑之衆燒洪範門使詳為内應事泄詳被誅【被皮義翻】黁遂據東苑以叛民間皆言聖人舉兵事無不成從之者甚衆凉王光召太原公纂使討黁纂將還諸將皆曰段業必躡軍後宜潛師夜發纂曰業無雄才憑城自守若潛師夜出適足張其氣勢耳【張知亮翻】乃遣使告業曰郭黁作亂吾今還都【都謂姑臧使疏吏翻】卿能决者可早出戰於是引還業不敢出纂司馬楊統謂其從兄桓曰【從才用翻】郭黁舉事必不虚發吾欲殺纂推兄為主西襲呂弘據張掖號令諸郡此千載一時也桓怒曰吾為呂氏臣安享其祿危不能救豈可復增其難乎【復扶又翻難乃旦翻】呂氏若亡吾為弘演矣【春秋衛懿公與狄人戰于滎澤為狄人所殺弘演納肝以殉之桓女配纂其見親異於他臣故云然】統至番禾遂叛歸黁【番禾縣漢屬張掖郡晉屬武威郡唐天寶中改為天寶縣番音盤】弘纂之弟也纂與西安太守石元良共擊黁大破之乃得入姑臧黁得光孫八人於東苑及敗而恚【恚於避翻】悉投於鋒上枝分節解飲其血以盟衆衆皆掩目凉人張捷宋生等招集戎夏三千人反於休屠城【夏戶雅翻休屠縣漢屬武威郡因休屠王城以為名也晉省縣水經註姑臧城西有馬城東城即休屠縣故城也屠直於翻】與黁共推凉後將軍楊軌為盟主軌略陽氐也將軍程肇諫曰卿棄龍頭而從虵尾非計也軌不從自稱大將軍凉州牧西平公纂擊破黁將王斐于城西黁兵勢漸衰遣使請救于秃髪烏孤【使疏吏翻】九月烏孤使其弟驃騎將軍利鹿孤帥騎五千赴之【帥讀曰率騎奇寄翻】秦太后虵氏卒【虵以者翻虜姓也又食遮翻又音他】秦主興哀毁過禮<br />
<br />
  不親庶政羣臣請依漢魏故事既葬即吉尚書郎李嵩上疏曰孝治天下先王之高事也【治直之翻】宜遵聖性以光道訓既葬之後素服臨朝【朝直遥翻】尹緯駁曰嵩矯常越禮【尹緯習於聞見反謂李嵩為矯常越禮嗚呼自短喪之制行人之不知禮也久矣駁北角翻】請付有司論罪興曰嵩忠臣孝子有何罪乎其一從嵩議 鮮卑薛勃叛秦【薛勃據貳城為魏所攻而降于秦】秦主興自將討之【將即亮翻】勃敗犇沒弈干沒弈干執送之 秦泫氏男姚買得謀弑秦主興不克而死【泫師古曰工玄翻楊正衡胡大翻】 秦主興入寇湖城弘農太守陶仲山華山太守董邁皆降之遂至陜城進寇上洛拔之【置陜湖二戍見一百六卷孝武大元十一年華山郡晉分弘農之華隂京兆之鄭馮翊之夏陽郃陽置上洛縣前漢屬弘農後漢屬京兆晉武帝泰始二年分京兆南部置上洛郡華戶化翻陜失冉翻】遣姚崇寇洛陽河南太守夏侯宗之固守金墉崇攻之不克乃徙流民二萬餘戶而還武都氐屠飛啖鐵等據方山以叛秦【據晉書載記時飛鐵殺隴東太守姚迴屯據方山則方山當在隴東郡界祝穆曰方山在武都郡東西四十里】興遣姚紹等討之斬飛鐵興勤於政事延納善言京兆杜瑾等皆以論事得顯拔天水姜龕等以儒學見尊禮【瑾渠吝翻龕口含翻】給事黄門侍郎古成詵等以文章參機密【古成姓也詵疎臻翻】詵剛介雅正以風教為己任京兆韋高慕阮籍之為人居母喪彈琴飲酒詵聞之而泣持劍求高欲殺之高懼而逃匿 中山饑甚慕容麟帥二萬餘人出據新市【新市縣自漢以來屬中山劉昫曰新市古鮮虞子國唐為定州新樂縣杜佑曰唐鎮州治真定縣漢新市縣故城在東北帥讀曰率】甲子晦魏王珪進軍攻之太史令鼂崇曰不吉昔紂以甲子亡謂之疾日【左傳辰在子卯謂之疾日杜預註云疾惡也紂以甲子喪桀以乙卯亡故以為忌日鼂直遥翻】兵家忌之珪曰紂以甲子亡周武不以甲子興乎崇無以對冬十月丙寅麟退阻?水【?音觚】甲戌珪與麟戰於義臺大破之【據李延壽北史義臺塢名魏收地形志新市縣有義臺城】斬首九千餘級麟與數十騎馳取妻子入西山遂奔鄴甲申魏克中山燕公卿尚書將吏士卒降者二萬餘人【將即亮翻降戶江翻】張驤李沈先嘗降魏復亡去【復扶又翻】珪入城皆赦之得燕璽綬圖書府庫珍寶以萬數【璽斯氏翻綬音受】班賞羣臣將士有差追諡弟觚為秦愍王發慕容詳冡斬其尸收殺觚者高霸程同皆夷五族【五族謂五服内親也】以大刃剉之丁亥遣三萬騎就衛王儀將攻鄴 秦長水校尉姚珍奔西秦西秦王乾歸以女妻之【妻七細翻】 河南鮮卑吐秣等十二部大人皆附於秃髪烏孤【此金城河南也】 燕人有自中山至龍城者言拓跋涉珪衰弱司徒德完守鄴城會德表至勸燕主寶南還寶於是大簡士馬將復取中原遣鴻臚魯邃冊拜德為丞相冀州牧【臚陵如翻】南夏公侯牧守皆聽承制封拜【夏戶雅翻】十一月癸丑燕大赦十二月調兵悉集戒嚴在頓【調徒弔翻頓者次舍之所】遣將軍啓崙南視形勢【崙盧昆翻】乙亥慕容麟至鄴復稱趙王說范陽王德曰魏既克中山將乘勝攻鄴鄴中雖有蓄積然城大難固且人心恇懼【說輸芮翻恇去王翻】不可守也不如南趣滑臺【趣七喻翻】阻河以待魏伺釁而動河北庶可復也時魯陽王和鎮滑臺和垂之弟子也亦遣使迎德德許之【使疏吏翻】<br />
<br />
  資治通鑑卷一百九  <br>
   </div> 

<script src="/search/ajaxskft.js"> </script>
 <div class="clear"></div>
<br>
<br>
 <!-- a.d-->

 <!--
<div class="info_share">
</div> 
-->
 <!--info_share--></div>   <!-- end info_content-->
  </div> <!-- end l-->

<div class="r">   <!--r-->



<div class="sidebar"  style="margin-bottom:2px;">

 
<div class="sidebar_title">工具类大全</div>
<div class="sidebar_info">
<strong><a href="http://www.guoxuedashi.com/lsditu/" target="_blank">历史地图</a></strong>  
<a href="http://www.880114.com/" target="_blank">英语宝典</a>  
<a href="http://www.guoxuedashi.com/13jing/" target="_blank">十三经检索</a> 
<br><strong><a href="http://www.guoxuedashi.com/gjtsjc/" target="_blank">古今图书集成</a></strong> 
<a href="http://www.guoxuedashi.com/duilian/" target="_blank">对联大全</a> <strong><a href="http://www.guoxuedashi.com/xiangxingzi/" target="_blank">象形文字典</a></strong> 

<br><a href="http://www.guoxuedashi.com/zixing/yanbian/">字形演变</a>  <strong><a href="http://www.guoxuemi.com/hafo/" target="_blank">哈佛燕京中文善本特藏</a></strong>
<br><strong><a href="http://www.guoxuedashi.com/csfz/" target="_blank">丛书&方志检索器</a></strong> <a href="http://www.guoxuedashi.com/yqjyy/" target="_blank">一切经音义</a>  

<br><strong><a href="http://www.guoxuedashi.com/jiapu/" target="_blank">家谱族谱查询</a></strong>  <strong><a href="http://shufa.guoxuedashi.com/sfzitie/" target="_blank">书法字帖欣赏</a></strong> 
<br>

</div>
</div>


<div class="sidebar" style="margin-bottom:0px;">

<font style="font-size:22px;line-height:32px">QQ交流群9:489193090</font>


<div class="sidebar_title">手机APP 扫描或点击</div>
<div class="sidebar_info">
<table>
<tr>
	<td width=160><a href="http://m.guoxuedashi.com/app/" target="_blank"><img src="/img/gxds-sj.png" width="140"  border="0" alt="国学大师手机版"></a></td>
	<td>
<a href="http://www.guoxuedashi.com/download/" target="_blank">app软件下载专区</a><br>
<a href="http://www.guoxuedashi.com/download/gxds.php" target="_blank">《国学大师》下载</a><br>
<a href="http://www.guoxuedashi.com/download/kxzd.php" target="_blank">《汉字宝典》下载</a><br>
<a href="http://www.guoxuedashi.com/download/scqbd.php" target="_blank">《诗词曲宝典》下载</a><br>
<a href="http://www.guoxuedashi.com/SiKuQuanShu/skqs.php" target="_blank">《四库全书》下载</a><br>
</td>
</tr>
</table>

</div>
</div>


<div class="sidebar2">
<center>


</center>
</div>

<div class="sidebar"  style="margin-bottom:2px;">
<div class="sidebar_title">网站使用教程</div>
<div class="sidebar_info">
<a href="http://www.guoxuedashi.com/help/gjsearch.php" target="_blank">如何在国学大师网下载古籍?</a><br>
<a href="http://www.guoxuedashi.com/zidian/bujian/bjjc.php" target="_blank">如何使用部件查字法快速查字?</a><br>
<a href="http://www.guoxuedashi.com/search/sjc.php" target="_blank">如何在指定的书籍中全文检索?</a><br>
<a href="http://www.guoxuedashi.com/search/skjc.php" target="_blank">如何找到一句话在《四库全书》哪一页?</a><br>
</div>
</div>


<div class="sidebar">
<div class="sidebar_title">热门书籍</div>
<div class="sidebar_info">
<a href="/so.php?sokey=%E8%B5%84%E6%B2%BB%E9%80%9A%E9%89%B4&kt=1">资治通鉴</a> <a href="/24shi/"><strong>二十四史</strong></a>&nbsp; <a href="/a2694/">野史</a>&nbsp; <a href="/SiKuQuanShu/"><strong>四库全书</strong></a>&nbsp;<a href="http://www.guoxuedashi.com/SiKuQuanShu/fanti/">繁体</a>
<br><a href="/so.php?sokey=%E7%BA%A2%E6%A5%BC%E6%A2%A6&kt=1">红楼梦</a> <a href="/a/1858x/">三国演义</a> <a href="/a/1038k/">水浒传</a> <a href="/a/1046t/">西游记</a> <a href="/a/1914o/">封神演义</a>
<br>
<a href="http://www.guoxuedashi.com/so.php?sokeygx=%E4%B8%87%E6%9C%89%E6%96%87%E5%BA%93&submit=&kt=1">万有文库</a> <a href="/a/780t/">古文观止</a> <a href="/a/1024l/">文心雕龙</a> <a href="/a/1704n/">全唐诗</a> <a href="/a/1705h/">全宋词</a>
<br><a href="http://www.guoxuedashi.com/so.php?sokeygx=%E7%99%BE%E8%A1%B2%E6%9C%AC%E4%BA%8C%E5%8D%81%E5%9B%9B%E5%8F%B2&submit=&kt=1"><strong>百衲本二十四史</strong></a>  <a href="http://www.guoxuedashi.com/so.php?sokeygx=%E5%8F%A4%E4%BB%8A%E5%9B%BE%E4%B9%A6%E9%9B%86%E6%88%90&submit=&kt=1"><strong>古今图书集成</strong></a>
<br>

<a href="http://www.guoxuedashi.com/so.php?sokeygx=%E4%B8%9B%E4%B9%A6%E9%9B%86%E6%88%90&submit=&kt=1">丛书集成</a> 
<a href="http://www.guoxuedashi.com/so.php?sokeygx=%E5%9B%9B%E9%83%A8%E4%B8%9B%E5%88%8A&submit=&kt=1"><strong>四部丛刊</strong></a>  
<a href="http://www.guoxuedashi.com/so.php?sokeygx=%E8%AF%B4%E6%96%87%E8%A7%A3%E5%AD%97&submit=&kt=1">說文解字</a> <a href="http://www.guoxuedashi.com/so.php?sokeygx=%E5%85%A8%E4%B8%8A%E5%8F%A4&submit=&kt=1">三国六朝文</a>
<br><a href="http://www.guoxuedashi.com/so.php?sokeytm=%E6%97%A5%E6%9C%AC%E5%86%85%E9%98%81%E6%96%87%E5%BA%93&submit=&kt=1"><strong>日本内阁文库</strong></a> <a href="http://www.guoxuedashi.com/so.php?sokeytm=%E5%9B%BD%E5%9B%BE%E6%96%B9%E5%BF%97%E5%90%88%E9%9B%86&ka=100&submit=">国图方志合集</a> <a href="http://www.guoxuedashi.com/so.php?sokeytm=%E5%90%84%E5%9C%B0%E6%96%B9%E5%BF%97&submit=&kt=1"><strong>各地方志</strong></a>

</div>
</div>


<div class="sidebar2">
<center>

</center>
</div>
<div class="sidebar greenbar">
<div class="sidebar_title green">四库全书</div>
<div class="sidebar_info">

《四库全书》是中国古代最大的丛书,编撰于乾隆年间,由纪昀等360多位高官、学者编撰,3800多人抄写,费时十三年编成。丛书分经、史、子、集四部,故名四库。共有3500多种书,7.9万卷,3.6万册,约8亿字,基本上囊括了古代所有图书,故称“全书”。<a href="http://www.guoxuedashi.com/SiKuQuanShu/">详细>>
</a>

</div> 
</div>

</div>  <!--end r-->

</div>
<!-- 内容区END --> 

<!-- 页脚开始 -->
<div class="shh">

</div>

<div class="w1180" style="margin-top:8px;">
<center><script src="http://www.guoxuedashi.com/img/plus.php?id=3"></script></center>
</div>
<div class="w1180 foot">
<a href="/b/thanks.php">特别致谢</a> | <a href="javascript:window.external.AddFavorite(document.location.href,document.title);">收藏本站</a> | <a href="#">欢迎投稿</a> | <a href="http://www.guoxuedashi.com/forum/">意见建议</a> | <a href="http://www.guoxuemi.com/">国学迷</a> | <a href="http://www.shuowen.net/">说文网</a><script language="javascript" type="text/javascript" src="https://js.users.51.la/17753172.js"></script><br />
  Copyright &copy; 国学大师 古典图书集成 All Rights Reserved.<br>
  
  <span style="font-size:14px">免责声明:本站非营利性站点,以方便网友为主,仅供学习研究。<br>内容由热心网友提供和网上收集,不保留版权。若侵犯了您的权益,来信即刪。scp168@qq.com</span>
  <br />
ICP证:<a href="http://www.beian.miit.gov.cn/" target="_blank">鲁ICP备19060063号</a></div>
<!-- 页脚END --> 
<script src="http://www.guoxuedashi.com/img/plus.php?id=22"></script>
<script src="http://www.guoxuedashi.com/img/tongji.js"></script>

</body>
</html>
