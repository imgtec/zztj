






























































資治通鑑卷二百九十四 宋 司馬光 撰

胡三省 音註

後周紀五|{
	起著雍敦牂盡屠維栛洽凡二年}


世宗睿文孝武皇帝下

顯德五年春正月乙酉廢匡國軍|{
	唐末以同州置匡國軍}
唐改元中興 丁亥右龍武將軍王漢璋奏克海州 己丑以侍衛馬軍都指揮使韓令坤權揚州軍府事 上欲引戰艦自淮入江阻北神堰不得度|{
	北神鎮在楚州城北五里吳王夫差溝通江淮後人於此立堰者以淮水低溝水高防其洩也舟行度堰入淮今號為平水堰艦戶黯翻下同}
欲鑿楚州西北鸛水以通其道|{
	今楚州城西老鸛河是也}
遣使行視還言地形不便計功甚多|{
	行下孟翻還從宣翻又如字}
上自往視之授以規畫發楚州民夫浚之旬日而成用功甚省巨艦數百艘皆達于江|{
	艘蘇遭翻}
唐人大驚以為神 壬辰抜靜海軍始通吳越之路|{
	先是唐於海陵之東境置靜海都鎮制置院西至海陵二百七十五里宋白曰靜海軍本揚州狼山鎮地南唐於狼山北立靜海制置院周得之建靜海軍尋升為通州}
先是帝遣左諫議大夫長安尹日就等使吳越語之曰卿今去雖汎海比還淮南已平當陸歸耳|{
	自靜海軍東南至江口於狼山之西度江登陸抵福山鎮則蘇州常熟縣界吳越之境也先悉薦翻語牛倨翻比必利翻}
已而果然 甲辰蜀右補闕章九齡見蜀主言政事不治由奸佞在朝|{
	朝直遥翻}
蜀主問奸佞為誰指李昊王昭遠以對蜀主怒以九齡為毁斥大臣貶維州録事參軍|{
	臨亂之君各賢其臣卒之亡蜀者昊昭遠也}
周兵攻楚州踰四旬唐楚州防禦使張彦卿固守不下乙巳帝自督諸將攻之宿於城下丁未克之彦卿與都監鄭昭業猶帥衆拒戰|{
	帥讀曰率}
矢刃皆盡彦卿舉繩牀以鬭而死所部千餘人至死無一人降者|{
	唐失淮南死於城郭封疆者猶有人焉}
高保融遣指揮使魏璘|{
	璘離珍翻}
將戰船百艘東下會伐唐至于鄂州 庚戌蜀置永寧軍於果州以通州隸之 唐以天長為雄州以建武軍使易文贇為刺史二月甲寅文贇舉城降|{
	贇於倫翻}
戊午帝發楚州丁卯至揚州命韓令坤發丁夫萬餘築故城之東南隅為小城以治之|{
	今揚州大城是也揚州古城西據蜀岡北包雷陂治直之翻}
乙亥黄州刺史司超奏與控鶴右廂都指揮使王審琦攻唐舒州擒其刺史施仁望 丙子建雄節度使真定楊廷璋奏敗北漢兵於隰州城下|{
	敗補邁翻}
時隰州刺史孫議暴卒廷璋謂都監閑廄使李謙溥曰今大駕南征澤州無守將|{
	澤州當作隰州}
河東必生心若奏請待報則孤城危矣即牒謙溥權隰州事謙溥至則修守備未幾北漢兵果至|{
	幾居豈翻}
諸將請速救之廷璋曰隰州城堅將良未易克也|{
	將即亮翻易以䜴翻}
北漢攻城久不下廷璋度其疲困無備|{
	度徒洛翻}
濳與謙溥約各募死士百餘夜襲其營|{
	九域志晉州西北至隰州二百五十里楊廷璋蓋濳軍而至與隰州約表裏相應也}
北漢兵驚潰斬首千餘級北漢兵遂解去 三月壬午朔帝如泰州 丁亥唐大赦改元交泰 唐太弟景遂前後凡十表辭位且言今國危不能扶請出就藩鎮燕王弘冀嫡長有軍功|{
	弘冀唐主之嫡長子軍功謂用柴克弘敗吳越兵以解常州之圍也事見上卷三年長知兩翻}
宜為嗣謹奏上太弟寶冊|{
	上時掌翻}
齊王景達亦以敗軍辭元帥唐主乃立景遂為晉王加天策上將軍江南西道兵馬元帥洪州大都督太尉尚書令以景達為浙西道元帥潤州大都督景達以浙西方用兵固辭|{
	吳越之兵雖於常州敗退然猶遥應中國}
改撫州大都督立弘冀為皇太子參决庶政弘冀為人猜忌嚴刻景遂左右有未出東宫者立斥逐之|{
	為弘冀毒殺景遂張本}
其弟安定公從嘉畏之不敢預事專以經籍自娯|{
	從嘉是為後主煜}
辛卯上如迎鑾鎮|{
	迎鑾鎮本宋之白沙也吳主楊溥至白沙閲舟師徐温自金陵來見因以白沙為迎鑾鎮白沙之地本屬江都唐分江都置永貞縣吴為迎鑾鎮宋為真州}
屢至江口遣水軍擊唐兵破之上聞唐戰艦數百艘洎東㳍州將趣海口扼蘇杭路|{
	東㳍州在泰州東南大江中元是海嶼沙島之地宋白曰東㳍洲在通州東南通州海門縣界㳍音布州當作洲}
遣殿前都虞候慕容延釗將步騎右神武統軍宋延渥將水軍循江而下甲午延釗奏大破唐兵於東㳍州上遣李重進將兵趣廬州|{
	唐末楊行密自廬州起既建國遂為重鎮周師度淮舒靳黄先皆欵附獨廬未下蓋宿兵多周師不敢輕犯也趣七喻翻}
唐主聞上在江上恐遂南度又恥降號稱藩乃遣兵部侍郎陳覺奉表 |{
	考異曰十國紀年遣樞密使陳覺奉表實録載其表云今遣左諫議大夫兵部侍郎臣陳覺躬聽敕命蓋當時所假之官耳今從之}
請傳位於太子弘冀使聽命於中國時淮南惟廬舒蘄黄未下|{
	蘄渠希翻}
丙申覺至迎鑾見周兵之盛白上請遣人度江取表獻四州之地畫江為境以求息兵辭指甚哀上曰朕本興師止取江北爾主能舉國内附朕復何求|{
	復扶又翻}
覺拜謝而退丁酉覺請遣其屬閤門承旨劉承遇如金陵上賜唐主書稱皇帝恭問江南國主慰納之戊戌吳越奏遣上直指揮使處州刺史邵可遷秀州刺史路彦銖以戰艦四百艘士卒萬七千人屯通州南岸|{
	周既克靜海軍置通州通州南岸蘇州常熟縣福山鎮之地即東晉之南沙也}
唐主復遣劉承遇奉表稱唐國主請獻江北四州歲輸貢物十萬於是江北悉平得州十四縣六十|{
	光夀廬舒蘄黄滁和濠泗楚揚泰通十四州}
庚子上賜唐主書諭以緣江諸軍及兩浙湖南荆南兵並當罷歸其廬蘄黄三道亦令歛兵近外|{
	謂周所遣進攻廬蘄黄之軍也近外謂近郊之外}
俟彼將士及家屬就道可遣人召將校以城邑付之|{
	將即亮翻校戶教翻}
江中舟艦有須往來者並令就北岸引之|{
	凡唐舟艦在北岸者皆許令引就南岸}
辛丑陳覺辭行又賜唐主書諭以不必傳位於子壬寅上自迎鑾復如揚州癸卯詔吳越荆南軍各歸本道賜錢弘俶犒軍帛三萬匹高保融一萬匹|{
	吳越軍臨南沙荆南軍至鄂州各犒之使罷歸犒苦到翻}
甲辰置保信軍於廬州以右龍武統軍趙匡贊為節度使丙午唐主遣馮延己獻銀絹錢茶穀共百萬以犒軍|{
	銀兩絹匹錢貫茶斤穀石各以萬計其數共為百萬}
己酉命宋延渥將水軍三千泝江廵警庚戌敇故淮南節度使楊行密故昇府節度使徐温等墓並量給守戶|{
	削其僭謚存其故鎮昇府即金陵金陵唐之昇州故曰昇府}
其江南羣臣墓在江北者亦委長吏以時檢校|{
	長知兩翻}
辛亥唐主遣其臨汝公徐遼代已來上夀|{
	言奉酒上夀非聖節也帝生於九月二十四日上時掌翻}
是月浚汴口導河流達于淮於是江淮舟檝始通|{
	此即唐時運路也自江淮割據運漕不通水路湮塞今復浚之}
夏四月乙卯帝自揚州北還|{
	還從宣翻又如字}
新作太廟

成庚申神主入廟|{
	太祖廣順三年作太廟於大梁至是始成五代會要太祖廣順元年七月追尊高祖璟為睿和皇帝廟號信祖曾祖諶為明憲皇帝廟號僖祖祖藴為翼順皇帝廟號義祖考簡為章肅皇帝廟號慶祖}
辛酉夜錢塘城南火延及内城官府廬舍幾盡|{
	幾居依翻}
壬戌旦火將及鎮國倉吳越王弘俶久疾自強出救火火止謂左右曰吾疾因災而愈衆心稍安 帝之南征也契丹乘虚入寇壬申帝至大梁命張永德將兵備禦北邊 五月辛巳朔日有食之詔賞勞南征士卒及淮南新附之民|{
	勞力到翻}
辛卯以太祖皇帝領忠武節度使徙安審琦為平盧節度使成德節度使郭崇攻契丹束城抜之|{
	束城漢勃海郡之束州縣隋改曰束城唐屬瀛州宋熙寧六年省束城為鎮屬河間}
以報其入寇也 唐主避周諱更名景|{
	避周信祖諱也更工衡翻}
下令去帝號稱國主凡天子儀制皆有降損去年號用周正朔|{
	去羌呂翻 考異曰世宗實録薛史顯德二年乙卯十一月伐淮南唐之保大十三年也三年正月四年二月十月三幸淮南五年戊午三月江北平唐之交泰元年也而江南録誤以保大十五年事合十四年十五年丁巳改交泰五月去帝號明年乃顯德五年又明年即建隆元年中間實少顯德六年江南録最為差誤其記李昇復姓亦先一年它事倣此不可考按故世宗取淮南年月專以實録及薛史為據}
仍告于太廟左僕射同平章事馮延己罷為太子太傅門下侍郎同平章事嚴續罷為少傅樞密使兵部侍郎陳覺罷守本官初馮延己以取中原之策說唐主由是有寵延己嘗笑烈祖戢兵為齷齪|{
	說式芮翻戢則立翻齷於角翻齪敇角翻}
曰安陸所喪纔數千兵為之輟食咨嗟者旬日|{
	謂晉高祖天福五年李承裕安州之敗也事見二百八十二卷喪息浪翻為于偽翻}
此田舍翁識量耳安足與成大事豈如今上暴師數萬於外而擊毬宴樂無異平日真英主也延己與其黨談論常以天下為己任更相唱和|{
	樂音洛更工衡翻和戶臥翻}
翰林學士常夢錫屢言延己等浮誕不可信|{
	誕徒旱翻}
唐主不聽夢錫曰奸言似忠陛下不悟國必亡矣及臣服於周延己之黨相與言有謂周為大朝者夢錫大笑曰諸公常欲致君堯舜何意今日自為小朝邪衆默然|{
	朝直遥翻}
自唐主内附帝止因其使者賜書未嘗遣使至其國己酉始命太僕卿馮延魯衛尉少卿鍾謨使于唐|{
	二人者本皆唐臣}
賜以御衣玉帶等及犒軍帛十萬并今年欽天歷|{
	犒苦到翻是年正月始行王朴所上欽天歷}
劉承遇之還自金陵也|{
	見上三月還從宣翻又如字}
唐主使陳覺白帝以江南無鹵田|{
	海濱鹹鹵可以煮鹽鹵田今謂之鹻地鹵郎古翻鹻古斬翻}
願得海陵監南屬以贍軍帝曰海陵在江北難以交居|{
	言難使周之官吏與唐之官吏雜居也}
當别有處分|{
	處昌呂翻分扶問翻}
至是詔歲支鹽三十萬斛以給江南所俘獲江南士卒稍稍歸之六月壬子昭義節度使李筠奏擊北漢石會關抜其

六寨乙卯晉州奏都監李謙溥擊北漢破孝義|{
	孝義漢中陽縣地後魏曰永安唐貞觀元年改曰孝義屬汾州在州東南宋熙寧五年省孝義為鎮屬介休縣宋白曰孝義縣本漢慈氏縣地曹魏移中陽縣於今理永嘉後省入隰城後魏又分隰城於今靈石縣東三十里置永安縣貞觀元年以縣名與涪州縣同改為孝義因縣人鄭興有行義為名}
高保融遣使勸蜀主稱藩于周蜀主報以前歲遣胡立致書于周而不答|{
	見上卷上年}
秋七月丙戌初行大周刑統 帝欲均田租丁亥以元稹均田圖徧賜諸道|{
	時詔曰近覽元稹長慶集見在同州時所上均田表較當時之利病曲盡其情俾一境之生靈咸受其賜傳於方冊可得披尋因令製素成圖直書其事稹止忍翻}
閏月唐清源節度使兼中書令留從効|{
	唐置清源軍於泉州}
遣牙將蔡仲贇衣商人服以絹表置革帶中間道來稱藩|{
	贇於倫翻衣於既翻間古莧翻}
唐江西元帥晉王景遂之赴洪州也|{
	見上三月}
以時方用兵啓求大臣以自副唐主以樞密副使工部侍郎李徵古為鎮南節度副使徵古傲狠專恣景遂雖寛厚久而不能堪常欲斬徵古自拘於有司左右諫而止景遂忽忽不樂|{
	樂音洛}
太子弘冀在東官多不法唐主怒嘗以毬杖擊之曰吾當復召景遂|{
	復扶又翻}
昭慶宫使袁從範從景遂為洪州都押牙或譖從範之子於景遂景遂欲殺之從範由是怨望弘冀聞之密遣從範毒之八月庚辰景遂擊毬渇甚從範進漿景遂飲之而卒|{
	卒子恤翻}
未殯體已潰唐主不之知贈皇太弟謚曰文成辛巳南漢中宗殂|{
	年三十九}
長子繼興即帝位更名鋹|{
	長知}


|{
	兩翻更工衡翻鋹丑兩翻}
改元大寶鋹年十六國事皆决於宦官玉清宫使龔澄樞|{
	歐史曰劉氏作離宫以遊獵有南宫大明昌華甘泉玩華秀華玉清太微諸宫皆置宫使領之}
及女侍中盧瓊仙等臺省官備位而已 甲申唐始置進奏院于大梁|{
	臣屬故也}
壬辰命西上閤門使靈夀曹彬使于吳越賜吳越王弘俶騎軍鋼甲二百|{
	鋼古郎翻堅鐵也}
步軍甲五千及他兵器彬事畢亟返不受饋遺|{
	遺唯季翻下以遺同}
吳越人以輕舟追與之至于數四彬曰吾終不受是竊名也盡籍其數歸而獻之帝曰曏之奉使乞匄無厭|{
	厭於鹽翻}
使四方輕朝命|{
	朝直遥翻}
卿能如是甚善然彼以遺卿卿自取之彬始拜受悉以散於親識家無留者辛丑馮延魯鍾謨來自唐唐主手表謝恩|{
	手表者手書之}
其畧曰天地之恩厚矣父母之恩深矣子不謝父人何報天惟有赤心可酬大造又乞比藩方賜詔書又稱有情事令鍾謨上奏乞令早還|{
	還從宣翻又如字}
唐主復令謨白帝欲傳位太子|{
	復扶又翻下復遣同}
九月丁巳以延魯為刑部侍郎謨為給事中唐主復遣吏部尚書知樞密院殷崇義來賀天清節|{
	帝生於九月二十四日以為天清節}
帝謀伐蜀冬十月己卯以戶部侍郎高防為西南面水陸制置使右贊善大夫李玉為判官 甲午帝歸馮延魯及左監門衛上將軍許文稹右千牛衛上將軍邊鎬衛尉卿周廷構于唐|{
	馮延魯被擒見二百九十二卷三年許文稹邊鎬被擒見上卷上年周廷搆降亦見是年}
唐主以文稹等皆敗軍之俘棄不復用|{
	復扶又翻}
高保融再遺蜀主書|{
	先遺書見上六月}
勸稱臣於周蜀主集將相議之李昊曰從之則君父之辱違之則周師必至諸將能拒周乎諸將皆曰以陛下聖明江山險固豈可望風屈服秣馬厲兵正為今日|{
	為于偽翻}
臣等請以死衛社稷丁酉蜀主命昊草書極言拒絶之 詔左散騎常侍須城艾潁等三十四人分行諸州均定田租|{
	須城縣帶鄆州即唐之須昌縣後唐避獻祖廟諱改曰須城艾姓也晏子春秋齊有大夫艾孔風俗通龎儉母艾氏行下孟翻}
庚子詔諸州併鄉村率以百戶為團團置耆長三人|{
	耆老也每團以老者三人為之長長知兩翻}
帝留心農事刻木為耕夫蠶婦置之殿庭 命武勝節度使宋延渥以水軍廵江 高保融奏聞王師將伐蜀請以水軍趣三峽|{
	趣七喻翻}
詔褒之 十一月庚戌敇竇儼編集大周通禮大周正樂|{
	去年竇儼請定禮樂疏見上卷}
辛亥南漢葬文武光明孝皇帝于昭陵廟號中宗 乙丑唐主復遣禮部侍郎鍾謨入見|{
	復扶又翻見賢遍翻}
李玉至長安或言蜀歸安鎮在長安南三百餘里可襲取也|{
	歸安鎮當在蜀金州界}
玉信之牒永興節度使王彦超索兵二百彦超以為歸安道阻隘難取|{
	索山各翻隘烏懈翻}
玉曰吾自奉密旨彦超不得已與之玉將以往|{
	將即亮翻}
十二月蜀歸安鎮遏使李承勲據險邀之斬玉其衆皆沒 乙酉蜀主以右衛聖步軍都指揮使趙崇韜為北面招討使丙戌以奉鑾肅衛都指揮使武信節度使兼中書令孟貽業為昭武文州都招討使|{
	昭武軍利州自利州以至文州委以控扼江油劒閣之險}
左衛聖馬軍都指揮使趙思進為東面招討使山南西道節度使韓保貞為北面都招討使將兵六萬分屯要害以備周 丙戌詔凡諸色課戶及俸戶並勒歸州縣|{
	唐初諸司置公廨本錢以貿易取息計員多少為月料其後罷諸司公廨本錢以天下上戶七千人為胥士而收其課計官多少而給之此所謂課戶也唐又薄歛一歲税以高戶主之月收息給俸此所謂俸戶也}
其幕職州縣官自今並支俸錢及米麥 初唐太傅兼中書令楚公宋齊丘多樹朋黨欲以專固朝權|{
	朝直遥翻}
躁進之士爭附之推奬以為國之元老樞密使陳覺副使李徵古恃齊丘之勢尤驕慢及許文稹等敗於紫金山覺與齊丘景達自濠州遁歸|{
	事見上卷上年}
國人忷懼|{
	忷許拱翻}
唐主嘗歎曰吾國家一朝至此因泣下徵古曰陛下當治兵以扞敵|{
	治直之翻}
涕泣何為豈飲酒過量邪將乳母不至邪唐主色變而徵古舉止自若會司天奏天文有變人主宜避位禳災唐主乃曰禍難方殷|{
	禳如羊翻難乃旦翻}
吾欲釋去萬機棲心冲寂誰可以託國者徵古曰宋公造國手也陛下如厭萬幾何不舉國授之覺曰陛下深居禁中國事皆委宋公先行後聞臣等時入侍談釋老而已唐主心愠|{
	愠於運翻}
即命中書舍人豫章陳喬草詔行之|{
	洪州豫章郡}
喬惶恐請見曰陛下一署此詔臣不復得見矣|{
	見賢遍翻復扶又翻}
因極言其不可唐主笑曰爾亦知其非邪乃止由是因晉王出鎮以徵古為之副|{
	事見上}
覺自周還|{
	還從宣翻又如字}
亦罷近職鍾謨素與李德明善以德明之死怨齊丘|{
	李德明死見上卷三年}
及奉使歸唐言於唐主曰齊丘乘國之危遽謀簒竊陳覺李徵古為之羽翼理不可容陳覺之自周還|{
	見上三月}
矯以帝命謂唐主曰聞江南連歲拒命皆宰相嚴續之謀當為我斬之|{
	為于偽翻}
唐主知覺素與續有隙固未之信鍾謨請覆之於周|{
	審覆其言之虚實於周也}
唐主乃因謨復命上言久拒王師皆臣愚迷非續之罪帝聞之大驚曰審如此則續乃忠臣|{
	言嚴續果能為其主設謀以拒周乃忠臣也}
朕為天下主豈教人殺忠臣乎謨還以白唐主|{
	還從宣翻}
唐主欲誅齊丘等復遣謨入禀於帝|{
	復扶又翻}
帝以異國之臣無所可否己亥唐主命知樞密院殷崇義草詔暴齊丘覺徵古罪惡聽齊丘歸九華山舊隱官爵悉如故|{
	宋齊丘隱九華山見二百七十七卷唐明宗長興二年吳睿皇之太和三年也}
覺責授國子博士宣州安置徵古削奪官爵賜自盡黨與皆不問遣使告于周 丙午蜀以峽路廵檢制置使高彦儔為招討使 平盧節度使太師中書令陳王安審琦僕夫安友進與其嬖妾通|{
	嬖卑義翻又必計翻}
妾恐事泄與友進謀殺審琦友進不可妾曰不然我當反告汝友進懼而從之

六年春正月癸丑審琦醉熟寢妾取審琦所枕劒|{
	枕職任翻}
授友進而殺之仍盡殺侍婢在帳下者以滅口後數日其子守忠始知之執友進等冎之|{
	冎古瓦翻}
初有司將立正仗宿設樂縣於殿庭|{
	前一夕設之謂之宿設縣讀曰懸下同}
帝觀之見鐘磬有設而不擊者問樂工皆不能對乃命竇儼討論古今考正雅樂王朴素曉音律帝以樂事詢之朴上疏以為禮以檢形樂以治心|{
	治直之翻}
形順於外心和於内然而天下不治者未之有也是以禮樂修於上萬國化於下聖人之教不肅而成其政不嚴而治|{
	孝經所載孔子之言治直之翻}
用此道也夫樂生於人心而聲成於物物聲既成復能感人之心|{
	復扶又翻}
昔黄帝吹九寸之管得黄鍾正聲半之為清聲倍之為緩聲三分損益之以生十二律|{
	三分其一而損益之上生下生而十二律備矣}
十二律旋相為宮以生七調為一均凡十二均八十四調而大備遭秦滅學歷代治樂者罕能用之|{
	朴之言曰自秦而下旋宫聲廢逮東漢雖有太子丞鮑鄴興之亦人亡而音息漢至隋垂十代凡數百年所有者黄鍾之宮一調而已十二律中唯用七聲其餘五律謂之啞鍾蓋不用故也}
唐太宗之世祖孝孫張文收考正大樂備八十四調|{
	見一百九十二卷貞觀元年調徒釣翻下同}
安史之亂器與工什亡八九至于黄巢蕩盡無遺時有太常博士殷盈孫按考工記鑄鎛鍾十二編鍾二百四十|{
	大鍾謂之鎛小鍾十六枚同在一虡謂之編鍾鎛補各翻}
處士蕭承訓校定石磬今之在縣者是也雖有鐘磬之狀殊無相應之和其鎛鍾不問音律但循環而擊編鍾編磬徒懸而已絲竹匏土僅有七聲名為黄鍾之宫其存者九曲考之三曲協律六曲參涉諸調蓋樂之廢缺無甚於今陛下武功既著垂意禮樂以臣嘗學律呂宣示古今樂録命臣討論臣謹如古法以秬黍定尺長九寸徑三分為黄鍾之管|{
	匏蒲爻翻論盧昆翻秬音巨黑黍也長直亮翻}
與今黄鍾之聲相應因而推之得十二律以為衆管互吹用聲不便乃作律凖十有三弦其長九尺|{
	律凖蓋梁武帝之遺法而梁武帝又本之京房}
皆應黄鍾之聲以次設柱為十一律及黄鍾清聲旋用七律以為一均為均之主者宫也徵商羽角變宫變徵次焉|{
	徵陟里翻}
發其均主之聲歸于本音之律迭應不亂乃成其調凡八十四調|{
	朴之言曰奉詔遂依周法以秬黍校定尺度長九寸虚徑三分為黄鍾之管與見在黄鍾之聲相應以上下相生之法推之得十二律管以為衆管互吹用聲不便乃作律凖十三弦宣聲長九尺張絃各如黄鍾之聲以第八絃六尺設柱為林鍾第三絃八尺設柱為太簇第十絃五尺三寸四分設柱為南呂第五絃七尺一寸三分設柱為姑洗第十三絃四尺七寸五分設柱為應鍾第七絃六尺三寸三分設柱為蕤賓第三絃八尺四寸四分設柱為大呂第九絃五尺六寸三分設柱為夷則第四絃七尺五寸一分設柱為夾鍾第十一絃五尺一分設柱為無射第六絃六尺六寸八分設柱為中呂第十三絃四尺五寸設柱為黄鍾清聲十二聲中旋用七聲為均為均之主者唯宫徵商羽角變宫變徵次焉發其均主之聲歸乎本音之律七聲迭應而不亂乃成其調均有七調聲有十二均合八十四調歌奏之曲出焉旋宫之聲久絶一日而補出臣獨見通鑑撮其要今備載之}
此法久絶出臣獨見乞集百官校其得失詔從之百官皆以為然乃行之|{
	時兵部尚書張昭等議曰昔帝鴻氏之作樂也候八節之風聲測四時之正氣氣之清濁不可以筆授聲之善否不可以口傳故鳬氏鑄鍾伶倫截竹為律呂相生之筭宫商正和之音乃播之於管絃宣之於鐘石然後覆載之情訢合隂陽之氣和同八風從律而不奸五色成文而不亂空桑孤竹之韻足以禮神雲門大夏之容無虧觀德然月律有旋宫之法備於太師之職經秦滅學雅道陵夷漢初制氏所調唯存鼔舞旋宫十二均更用之法世莫得聞漢元帝時京房善易别音探求古義以周宫均法每月更用五音乃至凖調旋相為宫成六十調又以日法析為三百六十傳於樂府而編懸復舊律呂無差遭漢中微雅音淪缺京房律凖屢有言者事終不成錢樂空記其名沈重但條其說六十律法寂寥不嗣梁武帝素精音律自造四通十二笛以叙八音又引古五正二變之音旋相為宫得八十四調與律凖所調音同數異侯景之亂其音又絶隋朝初定雅樂羣黨沮議歷載不成而沛公鄭譯因龜茲琵琶七音以飲月律五正二變七調克諧旋相為宫復為八十四調工人萬寶常又減其絲數稍令古淡隋高祖不重雅樂令羣臣集議博士何妥駁奏其鄭萬所奏八十四調並廢隋代郊廟所奏惟黄鍾一均與五郊迎氣雜用蕤賓但七調而已其餘五鍾懸而不作三朝宴樂用縵樂九部迄於革命未能改更唐太宗爰命舊工祖孝孫張文收整比鄭譯寶常所均七音八十四調方得絲管並施鍾石俱奏七始之音復振四廂之韻皆調自安史亂離咸秦蕩覆崇牙樹羽之器掃地無餘戞擊搏拊之工窮年不嗣郊廟所奏何異南箕波蕩不還知音殆絶臣等竊以音之所起出自人心夔曠不能長存人事不能常泰人亡則音息世亂則樂崩若不得知禮樂之情安明制作之本臣等據樞密使王朴條奏採京房之凖法練梁武之通音考鄭譯寶常之七均校孝孫文收之九變積黍累以審其度聽聲詩以測其情依權衡嘉量之前文得備數和聲之大旨施於鍾虡足洽簫韶臣等今月十九日於太常寺集命太樂令賈峻奏王朴新法黄鍾調七均音律和諧不相凌越其餘十一管諸調望依新法教習以備禮寺施用其五郊天地宗廟社稷三廟大禮合用十二管諸調並載唐史開元禮近代常行廣順中太常卿邊蔚奉勅定前件祠祭朝會舞名樂曲歌詞寺司合有簿籍伏恐所定與新法曲調聲韻不叶請下太常寺檢詳校試若或乖舛請本寺依新法聲調别撰樂章舞曲令歌者誦習從之}
唐宋齊丘至九華山唐主命鎻其第穴牆給飲食齊丘歎曰吾昔獻謀幽讓皇帝族於泰州|{
	事見二百八十一卷晉天福二年}
宜其及此乃縊而死 |{
	考異曰江表志齊丘至青陽絶食數日家人亦菜色中使云令公捐館方始供食家人以絮塞口而卒今從江南録紀年}
諡曰醜繆|{
	繆靡切翻}
初翰林學士常夢錫知宣政院參預機政深疾齊丘之黨數言於唐主曰|{
	數所角翻}
不去此屬|{
	去羌呂翻}
國必危亡與馮延己魏岑之徒日有爭論久之罷宣政院夢錫鬱鬱不得志不復預事縱酒成疾而卒|{
	通鑑二百八十五卷晉齊王開運三年已書常夢錫縱酒事去年又書夢錫笑馮延己之黨事蓋縱酒已非一日久乃成疾而卒}
及齊丘死唐主曰常夢錫平生欲殺齊丘恨不使見之贈夢錫左僕射 二月丙子朔命王朴如河隂按行河隄|{
	行下孟翻}
立斗門於汴口壬午命侍衛都指揮使韓通宣徽南院使吳廷祚|{
	廷祚當作延祚}
發徐宿宋單等州丁夫數萬|{
	單音善}
浚汴水甲申命馬軍都指揮使韓令坤自大梁城東導汴水入于蔡水|{
	魏收地形志曰汴水在大梁城東分為蔡渠九域志曰浚儀縣之琵琶溝即蔡河也朝會要曰惠民河與蔡河一水即閔河也建隆元年始命陳承昭督丁夫導閔河自新鄭與蔡水合貫京師南歷陳潁達壽春以通淮右舟楫相繼商賈畢至都下利之於是以西南為閔河東南為蔡河至開寶六年始改閔河為惠民河}
以通陳潁之漕命步軍都指揮使袁彦浚五文渠東過曹濟梁山泊以通青鄆之漕發畿内及滑亳丁夫數千以供其役 丁亥開封府奏田税舊一十萬二千餘頃今按行得羨苗四萬二千餘頃敇減三萬八千頃諸州行苗使還所奏羨苗減之倣此|{
	行下孟翻羨弋戰翻使疏吏翻還從宣翻又如字}
淮南饑|{
	大兵之後必有凶年}
上命以米貸之或曰民貧恐不能償上曰民吾子也安有子倒懸而父不為之解哉|{
	為于偽翻}
安在責其必償也 庚申樞密使王朴卒上臨其喪以玉鉞卓地慟哭數四不能自止朴性剛而鋭敏智略過人上以是惜之 甲子詔以北鄙未復將幸滄州|{
	九域志大梁至滄州一千二百里}
命義武節度使孫行友扞西山路|{
	扞定州西山路以防北漢救契丹也}
以宣徽南院使吳廷祚|{
	廷當作延}
權東京留守判開封府事三司使張美權大内都部署丁卯命侍衛親軍都虞候韓通等將水陸軍先發甲戌上發大梁夏四月庚寅韓通奏自滄州治水道入契丹境柵於乾寧軍南|{
	時置乾寧軍於滄州永安縣九域志在滄州西一百里宋白曰乾寧軍本古盧臺軍地治直之翻}
補壞防開游口三十六遂通瀛莫|{
	游口者於水不至處開之以備漲溢而洩游水也瀛莫相去一百一十里}
辛卯上至滄州即日帥步騎數萬發滄州直趨契丹之境|{
	帥讀曰率趨七喻翻自滄州西行九十八里即契丹瀛州界正北行五百七十五里直抵幽州}
河北州縣非車駕所過|{
	過音戈}
民間皆不之知壬辰上至乾寧軍契丹寧州刺史王洪舉城降|{
	契丹蓋置寧州於乾寧軍}
乙未大治水軍|{
	治直之翻}
分命諸將水陸俱下以韓通為陸路都部署太祖皇帝為水路都部署丁酉上御龍舟沿流而北舳艫相連數十里己亥至獨流口|{
	九域志獨流口在乾寧軍北一百二十里金人疆域圖涿州管下固安縣有獨流村}
泝流而西辛丑至益津關|{
	益津關在莫州文安縣九域志在乾寧軍西北一百六十里宋白曰益津關本幽州會昌縣唐天寶中改永清縣}
契丹守將終廷輝以城降自是以西水路漸隘不能勝巨艦|{
	隘烏懈翻勝音升}
乃捨之壬寅上登陸而西宿於野次侍衛之士不及一旅從官皆恐懼|{
	五百人為一旅從才用翻}
胡騎連羣出其左右不敢逼癸卯太祖皇帝先至瓦橋關|{
	瓦橋關在涿州歸義縣九域志在益津關東八十里宋白曰瓦子濟橋在涿州南易州東當九河之末}
契丹守將姚内斌舉城降|{
	斌音彬}
上入瓦橋關内斌平州人也甲辰契丹莫州刺史劉楚信舉城降五月乙巳朔侍衛親軍都指揮使天平節度使李重進等始引兵繼至契丹瀛州刺史高彦暉舉城降彦暉薊州人也|{
	薊音計}
於是關南悉平|{
	關南謂瓦橋關以南}
丙午宴諸將於行宫議取幽州諸將以為陛下離京四十二日|{
	離力智翻甲戍至丙午四十三日除宴日不數}
兵不血刃取燕南之地此不世之功也今虜騎皆聚幽州之北未宜深入上不悦是日趣先鋒都指揮使劉重進先發據固安|{
	固安漢縣名唐屬涿州今治所乃漢方城縣地匈奴須知固安縣西北至燕京一百二十里宋白曰隋開皇九年自今易州淶水縣移固安縣於漢方城縣地取漢故安縣為名其漢故安縣故城自在易州易縣東南七百步趣讀曰促}
上自至安陽水命作橋會日暮還宿瓦橋是日上不豫而止契丹主遣使者日馳七百里詣晉陽命北漢主發兵撓周邊|{
	撓奴巧翻又火高翻}
聞上南歸乃罷兵戊申孫行友奏抜易州擒契丹刺史李在欽獻之斬於軍市|{
	軍中有市聽軍人各以土物自相貿易}
己酉以瓦橋關為雄州|{
	九域志雄州治歸義容城二縣蓋皆置於郭下金人疆域圖雄州西北至燕京三百二十里}
割容城歸義二縣隸之|{
	宋白曰容城漢縣唐武德中改為酋縣天寶中改容城縣歸義縣本涿州屬邑今移於瓦橋而涿州之歸義自治漢易縣故城屬契丹界歸義縣宋朝避太宗濳藩舊名改為歸信縣}
以益津關為覇州|{
	金人疆域圖覇州至燕京三百五十五里}
割文安大城二縣隸之|{
	九域志大城縣在益津關東南一百五里五代之時所置也宋白曰文安漢舊縣晉置章武國在古文安城隋大業征遼途經河口當三河合流處置豐利縣唐貞觀二年以豐利文安二縣相逼移文安縣就豐利城周世宗置覇州治焉大城本漢東平舒縣晉於此置章武郡北齊廢郡為平舒縣五季改大城縣}
發濱棣丁夫數千城覇州命韓通董其役|{
	帝置賓州領勃海招安二縣九域志在滄州東南三百七卜五里濱棣二州瀕海無軍行供億之擾故發其丁夫築城按薛史濱州本贍國軍周顯德三年升為州割棣州之勃海蒲臺兩縣屬焉棣州樂安郡秦齊郡地宋為樂陵郡隋開皇十年於郡置厭次縣十七年又於陽信縣置棣州貞觀十七年自陽信移理厭次}
庚戌命李重進將兵出土門擊北漢辛亥以侍衛馬步都指揮使韓令坤為覇州都部署義成節度留後陳思讓為雄州都部署各將部兵以戍之壬子上自雄州南還|{
	九域志雄州至大梁一千二百里還從宣翻又如字}
己巳李重進奏敗北漢兵於百井|{
	敗補邁翻}
斬首二千餘級甲戌帝至大梁 六月乙亥朔昭義節度使李筠奏擊北漢抜遼州獲其刺史張丕 丙子鄭州奏河決原武|{
	原武縣屬鄭州九域志在州北六十里}
命宣徽南院使吳延祚發近縣二萬餘夫塞之|{
	塞悉則翻}
唐清源節度使留從効遣使入貢請置進奏院於京師直隸中朝|{
	中朝謂中國留從効以唐國勢削弱不欲復臣事之}
詔報以江南近服方務綏懷卿久奉金陵|{
	晉開運二年留從効以泉州附唐}
未可改圖若置邸上都與彼抗衡|{
	與唐比肩事周是抗衡也}
受而有之罪在於朕卿遠修職貢足表忠勤勉事舊君且宜如故如此則於卿篤始終之義於朕盡柔遠之宜惟乃通方諒達予意|{
	乃猶汝也諒想也}
唐主遣其子紀公從善與鍾謨俱入貢上問謨曰江南亦治兵修守備乎|{
	治直之翻}
對曰既臣事大國不敢復爾|{
	復扶又翻爾猶言如此也}
上曰不然曏時則為仇敵今日則為一家吾與汝國大義已定保無他虞然人生難期至于後世則事不可知歸語汝主|{
	語牛倨翻}
可及吾時完城郭繕甲兵據守要害為子孫計謨歸以告唐主唐主乃城金陵凡諸州城之不完者葺之戍兵少者益之

臣光曰或問臣五代帝王唐莊宗周世宗皆稱英武二主孰賢臣應之曰夫天子所以統治萬國|{
	治直之翻}
討其不服撫其微弱行其號令壹其法度敦明信義以兼愛兆民者也莊宗既滅梁海内震動湖南馬氏遣子希範入貢|{
	見二百七十二卷唐莊宗同光元年}
莊宗曰比聞馬氏之業終為高郁所奪|{
	比毗至翻}
今有兒如此郁豈能得之哉郁馬氏之良佐也希範兄希聲聞莊宗言卒矯其父命而殺之|{
	見二百七十六卷唐明宗天成四年卒子恤翻}
此乃市道商賈之所為|{
	賈音古}
豈帝王之體哉蓋莊宗善戰者也故能以弱晉勝彊梁既得之曾不數年外内離叛置身無所|{
	事並見梁均王及唐莊宗紀}
誠由知用兵之術不知為天下之道故也世宗以信令御羣臣以正義責諸國王環以不降受賞|{
	見二百九十二卷顯德二年}
劉仁贍以堅守蒙褒|{
	見上卷四年}
嚴續以盡忠獲存|{
	見上正月}
蜀兵以反覆就誅|{
	見上卷三年}
馮道以失節被棄|{
	見二百九十一卷二年被皮義翻}
張美以私恩見疎|{
	見二百九十二卷二年}
江南未服則親犯矢石期於必克既服則愛之如子推誠盡言為之遠慮|{
	為于偽翻}
其宏規大度豈得與莊宗同日語哉書曰無偏無黨王道蕩蕩|{
	洪範之言}
又曰大邦畏其力小邦懷其德|{
	武成之言}
世宗近之矣|{
	近其靳翻}


辛巳建雄節度使楊廷璋奏擊北漢降堡寨一十三癸未立皇后符氏宣懿皇后之女弟也|{
	宣懿符后殂見上卷三年}
立皇子宗訓為梁王領左衛上將軍宗讓為燕公領左驍衛上將軍|{
	宗讓後更名熙讓以恭帝嗣位避宗字也燕於賢翻}
上欲相樞密使魏仁浦議者以仁浦不由科第不可為相|{
	魏仁浦以樞密院吏歷仕至樞密使}
上曰自古用文武才略者為輔佐豈盡由科第邪己丑加王溥門下侍郎與范質皆參知樞密院事以仁浦為中書侍郎同平章事樞密使如故仁浦雖處權要而能謙謹上性嚴急近職有忤旨者仁浦多引罪歸已以救之所全活什七八|{
	處昌呂翻忤五故翻}
故雖起刀筆吏致位宰相時人不以為沗又以宣徽南院使吳延祚為左驍衛上將軍充樞密使加歸德節度使侍衛將軍都虞候韓通鎮寧節度使兼殿前都點檢張永德並同平章事仍以通充侍衛親軍副都指揮使以太祖皇帝兼殿前都點檢上嘗問大臣可為相者於兵部尚書張昭昭薦李濤上愕然曰濤輕薄無大臣體朕問相而卿首薦之何也對曰陛下所責者細行也|{
	行下孟翻}
臣所舉者大節也昔晉高祖之世張彦澤虐殺不辜濤累疏請誅之以為不殺必為國患|{
	見二百八十三卷晉高祖天福三年}
漢隱帝之世濤亦上疏請解先帝兵權|{
	見二百八十八卷漢隱帝乾祐元年}
夫國家安危未形而能見之此真宰相器也臣是以薦之上曰卿言甚善且至公然如濤者終不可置之中書濤喜詼諧不修邊幅與弟澣俱以文學著名雖甚友愛而多謔浪無長幼體上以是薄之|{
	喜許記翻謔迄却翻浪力葬翻韓氏詩傳云起也}
上以翰林學士單父王著幕府舊僚屢欲相之|{
	單父縣帶單州單音善父音甫}
以其嗜酒無檢而罷癸巳大漸召范質等入受顧命上曰王著藩邸故人朕若不起當相之質等出相謂曰著終日遊醉鄉豈堪為相慎勿泄此言是日上殂|{
	年三十九}
上在藩多務韜晦及即位破高平之寇|{
	見二百九十二卷元年}
人始服其英武其御軍號令嚴明人莫敢犯攻城對敵矢石落其左右人皆失色而上略不動容應機决策出人意表又勤於為治百司簿籍過目無所忘|{
	治直吏翻忘巫放翻}
發姦擿伏聰察如神閒暇則召儒者讀前史商榷大義性不好絲竹珍玩之物|{
	擿他狄翻榷古岳翻好呼到翻}
常言太祖養成王峻王殷之惡致君臣之分不終|{
	貶王峻誅王殷見二百九十一卷太祖廣順三年分扶問翻}
故羣臣有過則面質責之服則赦之有功則厚賞之文武參用各盡其能人無不畏其明而懷其惠故能破敵廣地所向無前然用法太嚴羣臣職事小有不舉往往寘之極刑雖素有才幹聲名無所開宥尋亦悔之末年寖寛登遐之日遠邇哀慕焉甲午宣遺詔命梁王宗訓即皇帝位生七年矣|{
	帝世宗第四子也當此之時主少國疑宿衛將士多歸心於太祖皇帝明年正月遂因出師翼戴而天下為宋改元建隆}
秋七月壬戌以侍衛親軍都指揮使李重進領淮南節度使副都指揮使韓通領天平節度使太祖皇帝領歸德節度使以山南東道節度使同平章事向拱為西京留守庚申加拱兼侍中拱即向訓也避恭帝名改焉|{
	帝後禪于宋奉為鄭王后崩謚曰恭帝}
丙寅大赦 唐主以金陵去周境纔隔一水|{
	時周境南至于江金陵北至江二十二里耳}
洪州險固居上游|{
	洪州據南江之要會其地居金陵上游}
集羣臣議徙都之羣臣多不欲徙惟樞密副使給事中唐鎬勸之乃命經營豫章為都城之制唐自淮上用兵及割江北|{
	顯德二年冬十二月周師度淮五年春三月唐割江北}
臣事於周歲時貢獻府藏空竭錢益少物價騰貴|{
	藏徂浪翻少詩沼翻騰踊也}
禮部侍郎鍾謨請鑄大錢一當五十中書舍人韓熙載請鑄鐵錢唐主始皆不從謨陳請不已乃從之是月始鑄當十大錢文曰永通泉貨又鑄當二錢文曰唐國通寶與開元錢並行|{
	開元錢唐武德初所鑄}
八月戊子蜀主以李昊領武信節度使右補闕李起上言故事宰相無領方鎮者蜀主曰昊家多穴費|{
	穴而隴翻}
以厚禄優之耳起卭州人性婞直李昊嘗語之曰|{
	卭渠容翻婞戶隕翻語牛倨翻}
以子之才苟能慎默當為翰林學士起曰俟無舌乃不言耳 庚寅立皇弟宗讓為曹王更名熙讓熙謹為紀王熙誨為蘄王|{
	更宗為熙避帝名也歐史曰本朝乾德二年十月熙謹卒熙讓熙誨不知所終蓋諱之也更工衡翻}
九月丙午唐太子弘冀卒有司引浙西之功|{
	謂遣柴克宏敗吳越兵於常州也}
謚曰武宣句容尉全椒張洎上言|{
	句容縣屬昇州九域志在州東九十里全椒漢縣名梁置北譙郡尋改曰臨滁郡隋改曰滁水縣大業初復曰全椒唐屬滁州九域志在州南五十里句如字洎其冀翻}
太子之德主於孝敬今謚以武功非所以防微而慎德也乃更謚曰文獻擢洎為上元尉|{
	唐都金陵以上元為赤縣句容為畿縣自畿縣尉升赤縣尉為擢}
唐禮部侍郎知尚書省事鍾謨數奉使入周|{
	數所角翻下數於同}
傳世宗命於唐主世宗及唐主皆厚待之恃此驕横於其國|{
	横下孟翻}
三省之事皆預焉文獻太子總朝政|{
	朝直遥翻}
謨求兼東宫官不得乃薦其所善閻式為司議郎掌百司關啓李德明之死也|{
	見上卷三年}
唐鎬預其謀謨聞鎬受賕嘗面詰之鎬甚懼謨與天威都虞候張巒善數於私第屛人語至夜分|{
	詰其吉翻屛必郢翻又卑正翻}
鎬譖諸唐主曰謨與巒氣類不同而過相親狎謨屢使上國巒北人恐其有異謀又言永通大錢民多盗鑄犯法者衆及文獻太子卒唐主欲立其母弟鄭王從嘉謨嘗與紀公從善同奉使于周相厚善言於唐主曰從嘉德輕志懦又酷信釋氏非人主才從善果敢凝重宜為嗣唐主由是怒|{
	居人父子之間而欲廢長立少宜鍾謨之死也}
尋徙從嘉為吳王尚書令知政事居東宫冬十月謨請令張巒以所部兵廵徼都城|{
	正與唐鎬所譖合遂速罪徼吉弔翻}
唐主乃下詔暴謨侵官之罪貶國子司業流饒州貶張巒為宣州副使未幾皆殺之|{
	幾居豈翻}
廢永通錢 十一月壬寅朔葬睿武孝文皇帝于慶陵|{
	陵在鄭州管城縣}
廟號世宗 南漢主以中書舍人鍾允章藩府舊僚擢為尚書右丞參政事甚委任之允章請誅亂法者數人以正綱紀南漢主不能從宦官聞而惡之|{
	惡烏路翻}
南漢主將祀圓丘前三日允章帥禮官登壇四顧指揮設神位|{
	帥讀曰率}
内侍監許彦真望之曰此謀反也即帶劔登壇允章叱之彦真馳入宫告允章欲於郊祀日作亂南漢主曰朕待允章厚豈有此邪玉清宫使龔澄樞内侍監李托等共證之以彦真言為然乃收允章繫含章樓下命宦者與禮部尚書薛用丕雜治之|{
	治直之翻}
用丕素與允章善告以必不免允章執用丕手泣曰老夫今日猶机上肉耳分為仇人所烹|{
	分扶問翻}
但恨邕昌幼不知吾寃及其長也公為我語之|{
	鍾允章被讒抱不測之罪正恐累及妻子乃為是言是自禍之也長知兩翻為于偽翻語牛倨翻}
彦真聞之罵曰反賊欲使其子報仇邪復曰南漢主曰|{
	復扶又翻}
允章與二子共登壇濳有所禱俱斬之自是宦官益横|{
	横戶孟翻}
李托封州人也辛亥南漢主祀圓丘大赦未幾以龔澄樞為左龍虎觀軍容使内太師軍國之事皆取决焉凡羣臣有才能及進士狀頭|{
	進士第一人謂之狀頭}
或僧道可與談者皆先下蠶室|{
	下戶稼翻}
然後得進亦有自宫以求進者亦有免死而宫者由是宦者近二萬人|{
	近其靳翻}
貴顯用事之人大抵皆宦者也謂士人為門外人不得預事卒以此亡國|{
	至宋開寶四年而南漢亡卒子恤翻}
唐更命洪州曰南昌府建南都|{
	更工衡翻}
以武清節度使何敬洙為南都留守|{
	武清軍衡州屬湖南何敬洙遥領耳}
以兵部尚書陳繼善為南昌尹|{
	將徙都豫章也}
周人之攻秦鳳也蜀中忷懼|{
	忷許拱翻}
都官郎中徐及甫自負才略仕不得志隂結黨與謀奉前蜀高祖之孫少府少監王令儀為主以作亂|{
	前蜀主王建廟號高祖}
會周兵退而止至是其黨有告者收捕之及甫自殺十二月甲午賜令儀死 端明殿學士兵部侍郎竇儀使於唐天雨雪唐主欲受詔於廡下|{
	雨王遇翻廡文甫翻}
儀曰使者奉詔而來不敢失舊禮若雪霑服請俟他日唐主乃拜詔於庭 契丹主遣其舅使於唐泰州團練使荆罕儒募客使殺之唐人夜宴契丹使者於清風驛酒酣起更衣|{
	荆姓也燕有刺客荆軻楚國本曰荆此楚之前受氏更工衡翻}
久不返視之失其首矣自是契丹與唐絶罕儒冀州人也

資治通鑑卷二百九十四















































































































































