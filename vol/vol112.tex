<!DOCTYPE html PUBLIC "-//W3C//DTD XHTML 1.0 Transitional//EN" "http://www.w3.org/TR/xhtml1/DTD/xhtml1-transitional.dtd">
<html xmlns="http://www.w3.org/1999/xhtml">
<head>
<meta http-equiv="Content-Type" content="text/html; charset=utf-8" />
<meta http-equiv="X-UA-Compatible" content="IE=Edge,chrome=1">
<title>資治通鑒_113-資治通鑑卷一百十二_113-資治通鑑卷一百十二</title>
<meta name="Keywords" content="資治通鑒_113-資治通鑑卷一百十二_113-資治通鑑卷一百十二">
<meta name="Description" content="資治通鑒_113-資治通鑑卷一百十二_113-資治通鑑卷一百十二">
<meta http-equiv="Cache-Control" content="no-transform" />
<meta http-equiv="Cache-Control" content="no-siteapp" />
<link href="/img/style.css" rel="stylesheet" type="text/css" />
<script src="/img/m.js?2020"></script> 
</head>
<body>
 <div class="ClassNavi">
<a  href="/24shi/">二十四史</a> | <a href="/SiKuQuanShu/">四库全书</a> | <a href="http://www.guoxuedashi.com/gjtsjc/"><font  color="#FF0000">古今图书集成</font></a> | <a href="/renwu/">历史人物</a> | <a href="/ShuoWenJieZi/"><font  color="#FF0000">说文解字</a></font> | <a href="/chengyu/">成语词典</a> | <a  target="_blank"  href="http://www.guoxuedashi.com/jgwhj/"><font  color="#FF0000">甲骨文合集</font></a> | <a href="/yzjwjc/"><font  color="#FF0000">殷周金文集成</font></a> | <a href="/xiangxingzi/"><font color="#0000FF">象形字典</font></a> | <a href="/13jing/"><font  color="#FF0000">十三经索引</font></a> | <a href="/zixing/"><font  color="#FF0000">字体转换器</font></a> | <a href="/zidian/xz/"><font color="#0000FF">篆书识别</font></a> | <a href="/jinfanyi/">近义反义词</a> | <a href="/duilian/">对联大全</a> | <a href="/jiapu/"><font  color="#0000FF">家谱族谱查询</font></a> | <a href="http://www.guoxuemi.com/hafo/" target="_blank" ><font color="#FF0000">哈佛古籍</font></a> 
</div>

 <!-- 头部导航开始 -->
<div class="w1180 head clearfix">
  <div class="head_logo l"><a title="国学大师官网" href="http://www.guoxuedashi.com" target="_blank"></a></div>
  <div class="head_sr l">
  <div id="head1">
  
  <a href="http://www.guoxuedashi.com/zidian/bujian/" target="_blank" ><img src="http://www.guoxuedashi.com/img/top1.gif" width="88" height="60" border="0" title="部件查字,支持20万汉字"></a>


<a href="http://www.guoxuedashi.com/help/yingpan.php" target="_blank"><img src="http://www.guoxuedashi.com/img/top230.gif" width="600" height="62" border="0" ></a>


  </div>
  <div id="head3"><a href="javascript:" onClick="javascript:window.external.AddFavorite(window.location.href,document.title);">添加收藏</a>
  <br><a href="/help/setie.php">搜索引擎</a>
  <br><a href="/help/zanzhu.php">赞助本站</a></div>
  <div id="head2">
 <a href="http://www.guoxuemi.com/" target="_blank"><img src="http://www.guoxuedashi.com/img/guoxuemi.gif" width="95" height="62" border="0" style="margin-left:2px;" title="国学迷"></a>
  

  </div>
</div>
  <div class="clear"></div>
  <div class="head_nav">
  <p><a href="/">首页</a> | <a href="/ShuKu/">国学书库</a> | <a href="/guji/">影印古籍</a> | <a href="/shici/">诗词宝典</a> | <a   href="/SiKuQuanShu/gxjx.php">精选</a> <b>|</b> <a href="/zidian/">汉语字典</a> | <a href="/hydcd/">汉语词典</a> | <a href="http://www.guoxuedashi.com/zidian/bujian/"><font  color="#CC0066">部件查字</font></a> | <a href="http://www.sfds.cn/"><font  color="#CC0066">书法大师</font></a> | <a href="/jgwhj/">甲骨文</a> <b>|</b> <a href="/b/4/"><font  color="#CC0066">解密</font></a> | <a href="/renwu/">历史人物</a> | <a href="/diangu/">历史典故</a> | <a href="/xingshi/">姓氏</a> | <a href="/minzu/">民族</a> <b>|</b> <a href="/mz/"><font  color="#CC0066">世界名著</font></a> | <a href="/download/">软件下载</a>
</p>
<p><a href="/b/"><font  color="#CC0066">历史</font></a> | <a href="http://skqs.guoxuedashi.com/" target="_blank">四库全书</a> |  <a href="http://www.guoxuedashi.com/search/" target="_blank"><font  color="#CC0066">全文检索</font></a> | <a href="http://www.guoxuedashi.com/shumu/">古籍书目</a> | <a   href="/24shi/">正史</a> <b>|</b> <a href="/chengyu/">成语词典</a> | <a href="/kangxi/" title="康熙字典">康熙字典</a> | <a href="/ShuoWenJieZi/">说文解字</a> | <a href="/zixing/yanbian/">字形演变</a> | <a href="/yzjwjc/">金 文</a> <b>|</b>  <a href="/shijian/nian-hao/">年号</a> | <a href="/diming/">历史地名</a> | <a href="/shijian/">历史事件</a> | <a href="/guanzhi/">官职</a> | <a href="/lishi/">知识</a> <b>|</b> <a href="/zhongyi/">中医中药</a> | <a href="http://www.guoxuedashi.com/forum/">留言反馈</a>
</p>
  </div>
</div>
<!-- 头部导航END --> 
<!-- 内容区开始 --> 
<div class="w1180 clearfix">
  <div class="info l">
   
<div class="clearfix" style="background:#f5faff;">
<script src='http://www.guoxuedashi.com/img/headersou.js'></script>

</div>
  <div class="info_tree"><a href="http://www.guoxuedashi.com">首页</a> > <a href="/SiKuQuanShu/fanti/">四库全书</a>
 > <h1>资治通鉴</h1> <!--         下载:【右键另存为】即可 --></div>
  <div class="info_content zj clearfix">
  
<div class="info_txt clearfix" id="show">
<center style="font-size:24px;">113-資治通鑑卷一百十二</center>
    資治通鑑卷一百十二<br />
<br />
  宋 司馬光 撰<br />
<br />
  胡三省 音注<br />
<br />
  晉紀三十四【起重光赤奮若盡玄黓攝提格凡二年】<br />
<br />
  安皇帝下<br />
<br />
  隆安五年春正月武威王利鹿孤欲稱帝羣臣皆勸之安國將軍鍮勿崙曰【安國將軍漢獻帝以授張楊鍮託侯翻崙盧昆翻】吾國自上世以來被髮左袵【被皮義翻】無冠帶之飾逐水草遷徙無城郭室廬故能雄視沙漠抗衡中夏【夏戶雅翻】今舉大號誠順民心然建都立邑難以避患儲蓄倉庫啟敵人心不如處晉民於城郭勸課農桑以供資儲帥國人以習戰射鄰國弱則乘之強則避之此久長之良策也【自漢以來善為夷狄謀者莫過此策矣處昌呂翻帥讀曰率】且虚名無實徒足為世之質的將安用之【質受斧的受矢按詩發彼有的毛傳云的質也正義曰毛氏於射侯之事正鵠不明惟猗嗟傳云二尺曰正亦不言正之所施周禮鄭衆馬融注皆云十尺曰侯四尺曰鵠二尺曰正四寸曰質則以為侯皆一丈鵠及正質於一侯之中為此等級則以質為四寸也王肅引爾雅云射張皮謂之侯侯中謂之鵠鵠中謂之正正方二尺正中謂之方六寸則質也舊云方四寸今云方六寸爾雅說明宜從之肅意惟改質為六寸餘同鄭馬賈逵周禮注云四尺曰正正五重鵠居其内而方二尺以為正正大於鵠鵠在正内雖内外不同亦共在一侯鄭於周禮上下檢之以為大射之侯其中制皮為鵠賓射之侯其中采畫為正正大如鵠皆居侯中三分之一其燕射則射獸侯侯中畫為獸形即鄉射記所謂熊侯白質之類射義云孔子曰循聲而發發而不失正鵠者其惟賢者乎詩云發彼有的以祈爾爵旣言正鵠即引此的則詩人之意以的為正鵠之謂也司裘注說皮侯之狀云以虎熊豹麋之皮飾其側又方制之以為質謂之鵠是鄭意以侯中所射之處為質也此毛傳唯言的質也】利鹿孤曰安國之言是也乃更稱河西王【更工衡翻王武威則一郡而已王河西則欲兼漢四郡之地此利鹿孤之志也】以廣武公傉檀為都督中外諸軍事凉州牧録尚書事【傉奴沃翻】 二月丙子孫恩出浹口【浹即叶翻】攻句章不能拔劉牢之撃之恩復走入海【復扶又翻】 秦王興使乞伏乾歸還鎮苑川盡以其故部衆配之【為乞伏氏復強張本】 凉王纂嗜酒好獵【好呼到翻】太常楊頴諫曰陛下應天受命當以道守之今疆宇日蹙崎嶇二嶺之間【姑臧南有洪池嶺西有丹嶺一作刪丹嶺】陛下不兢兢夕愓以恢弘先業而沈湎遊畋【沈持林翻】不以國家為事臣竊危之纂遜辭謝之然猶不悛番禾太守呂超擅擊鮮卑思盤【番禾縣漢屬張掖郡後漢晉省番音盤此郡盖呂氏置劉昫曰唐凉州天寶縣漢番禾縣地悛七緣翻番音盤】思盤遣其弟乞珍訴於纂纂命超及思盤皆入朝【朝直遙翻】超懼至姑臧深自結於殿中監杜尚纂見超責之曰卿恃兄弟桓桓【孔安國曰桓桓武貌】乃敢欺吾【今人謂相陵為相欺】要當斬卿天下乃定超頓首謝纂本以恐愒超【愒許葛翻】實無意殺之因引超思盤及羣臣同宴於内殿超兄中領軍隆數勸纂酒【數所角翻】纂醉乘步輓車【步輓車不用牛馬若羊等令人步而輓之魏書禮志步輓車天子小駕亦為副乘】將超等游禁中【將如字】至琨華堂東閤車不得過纂親將竇川駱騰倚劒於壁推車過閤【將即亮翻推吐雷翻】超取劒擊纂纂下車禽超超刺纂洞胷【刺七亦翻】川騰與超格戰超殺之纂后楊氏命禁兵討超杜尚止之【超之結尚也蓋有密約】皆捨仗不戰將軍魏益多入取纂首楊氏曰人已死如土石無所復知何忍復殘其形骸乎【復扶又翻】益多罵之遂取纂首以徇曰纂違先帝之命殺太子而自立【事見上卷三年】荒淫暴虐番禾太守起順人心而除之以安宗廟凡我士庶同茲休慶纂叔父巴西公佗【佗徒河翻】弟隴西公緯皆在北城【緯于貴翻】或說緯曰超為逆亂公以介弟之親【杜預曰介大也說輸芮翻下同】仗大義而討之姜紀焦辨在南城楊桓田誠在東苑皆吾黨也何患不濟緯嚴兵欲與佗共撃超佗妻梁氏止之曰緯超俱兄弟之子何為舍超助緯自為禍首乎【舍讀曰捨】佗乃謂緯曰超舉事已成據武庫擁精兵圖之甚難且吾老矣無能為也超弟邈有寵於緯說緯曰纂賊殺兄弟【謂殺紹又殺弘也說輸芮翻】隆超順人心而討之正欲尊立明公耳方今明公先帝之長子當主社稷人無異望夫復何疑【長知兩翻復扶又翻】緯信之乃與隆超結盟單馬入城超執而殺之讓位於隆隆有難色超曰今如乘龍上天豈可中下隆遂即天王位【隆字永基光弟寶之子也】大赦改元神鼎【超先於番禾得小鼎以為神瑞故以紀元】尊母衛氏為太后妻楊氏為后以超為都督中外諸軍事輔國大將軍録尚書事封安定公諡纂曰靈帝纂后楊氏將出宫超恐其挾珍寶命索之【索山客翻】楊氏曰爾兄弟不義手刃相屠我旦夕死人安用寶為超又問玉璽所在【璽斯氏翻】楊氏曰已毁之矣后有美色超將納之謂其父右僕射桓曰后若自殺禍及卿宗桓以告楊氏楊氏曰大人賣女與氐以圖富貴一之謂甚其可再乎【引左傳之言】遂自殺諡曰穆后桓犇河西王利鹿孤利鹿孤以為左司馬 三月孫恩北趣海鹽【海鹽縣本武原鄉秦以為海鹽縣漢屬會稽郡後漢晉屬吳郡今在秀州東南八十里趣七喻翻】劉裕隨而拒之築城於海鹽故治恩日來攻城裕屢擊破之斬其將姚盛城中兵少不敵【將即亮翻少詩沼翻】裕夜偃旗匿衆明晨開門使羸疾數人登城【羸倫為翻】賊遙問劉裕所在曰夜已走矣賊信之爭入城裕奮擊大破之恩知城不可拔乃進向滬瀆裕復棄城追之【滬音戶復扶又翻】海鹽令鮑陋遣子嗣之帥吳兵一千請為前驅【帥讀曰率】裕曰賊兵甚精吳人不習戰若前驅失利必敗我軍【敗補邁翻】可在後為聲埶嗣之不從裕乃多伏旗鼓前驅既交諸伏皆出裕舉旗鳴鼓賊以為四面有軍乃退嗣之追之戰沒裕且戰且退所領死傷且盡至向戰處令左右脱取死人衣以示閒暇【閒讀曰閑】賊疑之不敢逼裕大呼更戰【呼火故翻】賊懼而退裕乃引歸 河西王利鹿孤伐凉與凉王隆戰大破之徙二千餘戶而歸 夏四月辛卯魏人罷鄴行臺【魏置鄴行臺見一百一十卷隆安二年】以所統六郡置相州以庾岳為刺史【魏相州統魏郡陽平廣平汲郡頓丘清河六郡杜佑曰後魏置相州於鄴取河亶甲居相以名州】乞伏乾歸至苑川以邊芮為長史王松壽為司馬公<br />
<br />
  卿將帥皆降為僚佐偏禆【將即亮翻帥所類翻】 北凉王業憚沮渠蒙遜勇略欲遠之【沮子余翻遠于願翻】蒙遜亦深自晦匿業以門下侍郎馬權代蒙遜為張掖太守【守式又翻】權素豪雋為業所親重常輕侮蒙遜蒙遜譛之於業曰天下不足慮惟當憂馬權耳業遂殺權【以余觀之索嗣馬權皆庸夫耳恃倚世資而使氣無能為也】蒙遜謂沮渠男成曰段公無鑒斷之才【鑒明也斷决也斷丁亂翻】非撥亂之主曏所憚者惟索嗣馬權今皆已死【索嗣死見上卷四年】蒙遜欲除之以奉兄何如男成曰業本孤客為吾家所立恃吾兄弟猶魚之有水夫人親信我而圖之不祥蒙遜乃求為西安太守業喜其出外許之蒙遜與男成約同祭蘭門山而陰使司馬許咸告業曰男成欲以取假日為亂【假居訝翻休假也】若求祭蘭門山臣言驗矣至期果然業收男成賜死男成曰蒙遜先與臣謀反臣以兄弟之故隱而不言今以臣在恐部衆不從故約臣祭山而反誣臣其意欲王之殺臣也乞詐言臣死暴臣罪惡蒙遜必反臣然後奉王命而討之無不克矣業不聽殺之蒙遜泣告衆曰男成忠於段王而段王無故枉殺之諸君能為報仇乎【為于偽翻】且始者共立段王欲以安衆耳今州土紛亂非段王所能濟也男成素得衆心衆皆憤泣爭奮比至氐池【氐池縣漢屬張掖郡晉省其地屬唐甘州張掖縣界比必寐翻及也氐丁尼翻又音低】衆逾一萬鎮軍將軍臧莫孩率所部降之【孩河開翻降戶江翻下同】羌胡多起兵應蒙遜者蒙遜進逼侯塢業先疑右將軍田昂囚之至是召昂謝而赦之使與武衛將軍梁中庸共討蒙遜别將王豐孫言於業曰【將即亮翻】西平諸田世有反者昂貌恭而心險不可信也業曰吾疑之久矣但非昂無可以討蒙遜者昂至侯塢率騎五百降於蒙遜業軍遂潰中庸亦詣蒙遜降【危疑反側之時用言為難而用人為尤難當此之際非有明略雄斷不能濟也】五月蒙遜至張掖田昂兄子承愛斬關内之業左右皆散蒙遜至業謂蒙遜曰孤孑然一已為君家所推願匄餘命【匄古泰翻乞也】使得東還與妻子相見蒙遜斬之【北凉段業四年而亡】業儒素長者【長知兩翻】無他權略威禁不行羣下擅命尤信卜筮巫覡【覡它狄翻】故至於敗沮渠男成之弟富占將軍俱傫帥戶五百降于河西王利鹿孤傫石子之子也【傫倫追翻俱石子見一百六卷孝武太元十年帥讀曰率下同】 孫恩陷滬瀆殺吳國内史袁崧死者四千人【崧當作山松】 凉王隆多殺豪望以立威名内外囂然人不自保魏安人焦朗【魏安縣在武威昌松縣界盖曹魏所置也而晉志不見後魏置魏安郡】遣使說秦隴西公碩德曰呂氏自武皇棄世【呂光偽諡懿武皇帝說輸芮翻】兄弟相攻政綱不立競為威虐百姓饑饉死者過半今乘其簒奪之際取之易於返掌【易以豉翻返當作反】不可失也碩德言於秦王興帥步騎六萬伐凉乞伏乾歸帥騎七千從之 六月甲戌孫恩浮海奄至丹徒【丹徒縣古朱方也後日谷陽秦改曰丹徒漢屬會稽郡後漢屬吳郡晉屬晉陵郡地理志曰秦時望氣者云其地有天子氣始皇使赭衣三千人鑿城敗其勢改曰丹徒】戰士十餘萬樓船千餘艘【艘蘇遭翻】建康震駭乙亥内外戒嚴百官入居省内冠軍將軍高素等守石頭【冠古玩翻】輔國將軍劉襲柵斷淮口【秦淮入江之口也斷丁管翻】丹陽尹司馬恢之戍南岸冠軍將軍桓謙等備白石左衛將軍王嘏等屯中堂徵豫州刺史譙王尚之入衛京師劉牢之自山陰引兵邀擊恩未至而恩已過乃使劉裕自海鹽入援裕兵不滿千人倍道兼行與恩俱至丹徒裕衆既少【少詩沼翻】加以涉遠疲勞而丹徒守軍莫有鬭志恩帥衆鼓譟登蒜山【蒜山今在鎮江府城西三里山上多蒜故名蒜蘇貫翻】居民皆荷擔而立【荷下可翻擔都濫翻】裕帥所領奔擊大破之【帥讀曰率下同】投崖赴水者甚衆恩狼狽僅得還船然恩猶恃其衆尋復整兵徑向京師【復扶又翻下同】後將軍元顯帥兵拒戰頻不利會稽王道子無他謀略唯日禱蔣侯廟【蔣侯廟在蔣山在今建康府上元縣東北十八里漢末秣陵尉蔣子文討賊戰死山下吳孫權為立廟江東朝野禱之率有靈應】恩來漸近百姓忷懼【忷許拱翻】譙王尚之帥精鋭馳至徑屯積弩堂恩樓船高大泝風不得疾行數日乃至白石恩本以諸軍分散欲掩不備既而知尚之在建康復聞劉牢之已還至新洲【新洲在京口西大江中意即今之珠金沙是也復扶又翻】不敢進而去浮海北走郁洲【水經註曰束海朐縣東北海中有大洲謂之郁洲山海經所謂郁山在海中者也】恩别將攻陷廣陵殺三千人寧朔將軍高雅之擊恩於郁洲為恩所執【寧朔將軍盖晉置】桓玄厲兵訓卒常伺朝廷之隙【伺相吏翻】聞孫恩逼京師建牙聚衆上疏請討之元顯大懼會恩退元顯以詔書止之玄乃戒嚴 梁中庸等共推沮渠蒙遜為大都督大將軍凉州牧張掖公赦其境内改元永安蒙遜署從兄伏奴為張掖太守和平侯弟挐為建忠將軍都谷侯【從才用翻挐女余翻】田昂為西郡太守臧莫孩為輔國將軍房晷梁中庸為左右長史張騭謝正禮為左右司馬【騭之目翻】擢任賢才文武咸悦 河西王利鹿孤命羣臣極言得失西曹從事史暠曰【暠古老翻】陛下命將出征【將即亮翻下同】往無不捷然不以綏寧為先唯以徙民為務民安土重遷故多離叛此所以斬將拔城而地不加廣也利鹿孤善之 秋七月魏兖州刺史長孫肥【魏未得兖州也使肥以兖州刺史南略地耳】將步騎二萬南徇許昌東至彭城將軍劉該降之【將步即亮翻騎奇寄翻降戶江翻下同】 秦隴西公碩德自金城濟河直趣廣武河西王利鹿孤攝廣武守軍以避之【趣七喻翻攝收也】秦軍至姑臧凉王隆遣輔國大將軍超龍驤將軍邈等逆戰【驤思將翻】碩德大破之生禽邈俘斬萬計隆嬰城固守巴西公佗帥東苑之衆二萬五千降於秦【帥讀曰率】西凉公暠河西王利鹿孤沮渠蒙遜各遣使奉表入貢於秦【暠古老翻使疏吏翻沮子余翻】初凉將姜紀降於河西王利鹿孤廣武公傉檀與論兵略甚愛重之【傉奴沃翻】坐則連席出則同車每談論以夜繼晝利鹿孤謂傉檀曰姜紀信有美才然視候非常必不久留於此不如殺之紀若入秦必為人患傉檀曰臣以布衣之交待紀紀必不相負也八月紀將數十騎奔秦軍【秃髪兄弟皆推傉檀之明略余究觀傉檀始末未敢許也又究觀姜紀自凉入秦始末則紀盖反覆詭譎之士而傉檀愛重之則傉檀盖以才辨為諸兄所重而智略不能濟此其所以亡國也】說碩德曰呂隆孤城無援明公以大軍臨之其勢必請降然彼徒文降而已未肯遂服也請給紀步騎三千與王松怱因焦朗華純之衆【王松怱秦將也焦朗華純皆凉人說輸芮翻華戶化翻】伺其釁隙隆不足取也不然今秃髪在南兵彊國富若兼姑臧而據之威埶益盛沮渠蒙遜李暠不能抗也必將歸之如此則為國家之大敵矣碩德乃表紀為武威太守配兵二千屯據晏然【班固地理志武威休屠縣王莽改曰晏然後復曰休屠永寧中張軌於姑臧西北置武興郡晏然縣屬焉】秦王興聞楊桓之賢而徵之利鹿孤不敢留【史言諸凉畏秦之強】 詔以劉裕為下邳太守討孫恩於郁洲累戰大破之恩由是衰弱復緣海南走【復扶又翻】裕亦隨而邀擊之 燕王盛懲其父寶以懦弱失國務峻威刑又自矜聰察多所猜忌羣臣有纎介之嫌皆先事誅之【先悉薦翻】由是宗親勲舊人不自保丁亥左將軍慕容國與殿上將軍秦輿段讃謀帥禁兵襲盛【殿上將軍盖慕容所置緣晉之殿中將軍而名官也帥讀曰率】事發死者五百餘人壬辰夜前將軍段璣與秦輿之子興段讃之子泰潜於禁中鼓譟大呼【呼火故翻】盛聞變帥左右出戰【帥讀曰率】賊衆逃潰璣被創【創初良翻】匿廂屋間俄有一賊從闇中擊盛盛被傷輦升前殿申約禁衛事定而卒【年二十九慕容盛臨變而整此其雄略亦有過人者然以猜忌好殺致斃則天下之人固非一人可舞其智略而盡殺也】中壘將軍慕容拔冗從僕射郭仲白太后丁氏以為國家多難宜立長君【冗而隴翻從才用翻難乃旦翻長知兩翻】時衆望在盛弟司徒尚書令平原公元而河間公熙素得幸於丁氏丁氏乃廢太子定密迎熙入宫明旦羣臣入朝【朝直遥翻】始知有變因上表勸進於熙熙以讓元元不敢當癸巳熙即天王位【熙字道文垂之少子也】捕獲段璣等皆夷三族甲午大赦丙申平原公元以嫌賜死閏月辛酉葬盛於興平陵諡曰昭武皇帝廟號中宗丁氏送葬未還中領軍慕容提步軍校尉張佛等謀立故太子定事覺伏誅定亦賜死【燕立定為太子見上卷四年】丙寅大赦改元光始 秦隴西公碩德圍姑臧累月東方之人在城中者多謀外叛魏益多復誘扇之【復扶又翻下復生同】欲殺凉王隆及安定公超事發坐死者三百餘家碩德撫納夷夏分置守宰【夏戶雅翻守式又翻】節食聚粟為持久之計凉之羣臣請與秦連和隆不許安定公超曰今資儲内竭上下嗷嗷雖使張陳復生亦無以為策【張良陳平智謀之士故稱之】陛下當思權變屈伸何愛尺書單使為卑辭以退敵【使疏吏翻】敵去之後修德政以息民若卜世未窮何憂舊業之不復【周成王定鼎于郟鄏卜世三十卜年七百】若天命去矣亦可以保全宗族不然坐守困窮終將何如隆乃從之九月遣使請降於秦【降戶江翻下同 考異曰姚興載記姚平伐魏與姚碩德伐呂隆同時魏書天興五年五月姚平來侵晉元興元年秦弘始四年也晉帝紀晉春秋皆云隆安五年降秦十六國西秦春秋云太初十四年五月乾歸隨姚碩德伐凉南凉春秋云建和二年七月姚碩德伐呂隆孤攝廣武守軍以避之皆隆安五年也按秦小國既與魏相持豈暇更興兵伐凉盖載記之誤也今以晉帝紀晉春秋十六國西秦南凉春秋為據】碩德表隆為鎮西大將軍凉州刺史建康公隆遣子弟及文武舊臣慕容筑楊頴等五十餘家入質于長安【慕容筑燕宗室也苻堅滅燕其宗室悉補邊郡故筑留河西筑張六翻質音致下為質同】碩德軍令嚴整秋毫不犯祭先賢禮名士西土悦之沮渠蒙遜所部酒泉凉寧二郡叛降於西凉【酒泉郡治福禄縣魏收地形志凉寧郡領園池貢澤二縣】又聞呂隆降秦大懼遣其弟建忠將軍挐牧府長史張潜【蒙遜自稱凉州牧置牧府長史挐女居翻】見碩德於姑臧請帥其衆東遷【帥讀曰率】碩德喜拜潜張掖太守挐建康太守潜勸蒙遜東遷挐私謂蒙遜曰姑臧未拔呂氏猶存碩德糧盡將還不能久也何為自棄土宇受制於人乎臧莫孩亦以為然【孩何開翻】蒙遜遣子奚念為質於河西王利鹿孤【蒙遜既不東遷故為質於利鹿孤以求援】利鹿孤不受曰奚念年少可遣挐也【少詩沼翻】冬十月蒙遜復遣使上疏於利鹿孤曰臣前遣奚念具披誠欵而聖旨未昭復徵弟挐【復扶又翻】臣竊以為苟有誠信則子不為輕若其不信則弟不為重今寇難未夷【難乃旦翻】不獲奉詔願陛下亮之利鹿孤怒遣張松侯俱延興城侯文支將騎一萬襲蒙遜至萬歲臨松【晉書地理志酒泉郡有延壽縣當是後改為萬歲張天錫置臨松郡五代志曰臨松縣有臨松山後周省入張掖縣宋白曰隋煬帝併萬歲入刪丹縣屬張掖郡將即亮翻騎奇寄翻】執蒙遜從弟鄯善苟子【從才用翻下同鄯時戰翻康曰鄯善複姓其先西域人以國為姓苟子其名余據紀文以鄯善苟子為蒙遜從弟則鄯善非姓也明矣】虜其民六千餘戶蒙遜從叔孔遮入朝于利鹿孤【朝直遙翻】許以挐為質利鹿孤乃歸其所掠召俱延等還文支利鹿孤之弟也 南燕主備德宴羣臣於延賢堂【備德本名德既據齊地增上一字名備德】酒酣謂羣臣曰朕可方自古何等主青州刺史鞠仲曰陛下中興聖主少康光武之儔【少詩照翻】備德顧左右賜仲帛千疋仲以所賜多辭之備德曰卿知調朕朕不知調卿邪【調徒了翻又如字調戲也】卿所對非實故朕亦以虛言賞卿耳韓範進曰天子無戲言今日之論君臣俱失備德大悦賜範絹五十匹備德母及兄納皆在長安備德遣平原人杜弘往訪之弘曰臣至長安若不奉太后動止當西如張掖【德仕秦為張掖太守其兄納因家于張掖故弘欲往張掖訪之】以死為効臣父雄年踰六十乞本縣之祿以申烏鳥之情【慈烏反哺故云然李密陳情表曰烏烏私情願乞終養】中書令張華曰杜弘未行而求禄要君之罪大矣【要音邀】備德曰弘為君迎母為父求禄【為于偽翻】忠孝備矣何罪之有以雄為平原令弘至張掖為盗所殺 十一月劉裕追孫恩至滬瀆海鹽又破之【滬音扈】俘斬以萬數恩遂自浹口遠竄入海【浹音接】十二月辛亥魏主珪遣常山王遵定陵公和跋帥衆五萬襲沒弈干於高平【高平漢屬安定魏收志屬涇州新平郡又原州有高平郡酈道元云高平川西南去安定三百四十里帥讀曰率】 乙卯魏虎威將軍宿沓干伐燕攻令支【令音鈴又郎定翻支音祁又音祗】乙丑燕中領軍宇文拔救之壬午宿沓干拔令支而戍之 呂超攻姜紀不克遂攻焦朗【姜紀時據晏然焦朗據魏安】朗遣其弟子嵩為質於河西王利鹿孤以請迎利鹿孤遣車騎將軍傉檀赴之比至超已退【質音致比必寐翻】朗閉門拒之傉檀怒將攻之鎮北將軍俱延諫曰安土重遷人之常情朗孤城無食今年不降【降戶江翻】後年自服何必多殺士卒以攻之若其不捷彼必去從他國棄州境士民以資鄰敵非計也不如以善言諭之傉檀乃與朗連和遂曜兵姑臧壁於胡阬【胡阬在姑臧西】傉檀知呂超必來斫營畜火以待之超夜遣中壘將軍王集帥精兵二千斫傉檀營【帥讀曰率下同】傉檀徐嚴不起集入壘中内外皆舉火光照如晝縱兵擊之斬集及甲首三百餘級呂隆懼偽與傉檀通好【好呼到翻】請於苑内結盟傉檀遣俱延入盟俱延疑其有伏毁苑牆而入超伏兵擊之俱延失馬步走淩江將軍郭祖力戰拒之俱延乃得免傉檀怒攻其昌松太守孟禕於顯美【昌松顯美漢晉皆為縣屬武威郡呂光改昌松為東張掖郡尋復為昌松郡五代志後周廢顯美入姑臧禕許韋翻】隆遣廣武將軍荀安國寧遠將軍石可帥騎五百救之安國等憚傉檀之彊遁還 桓玄表其兄偉為江州刺史鎮夏口【夏戶雅翻】司馬刁暢為輔國將軍督八郡軍事鎮襄陽遣其將皇甫敷馮該戍湓口移沮漳蠻二千戶於江南【左傳曰江漢沮漳楚之望也水經沮水出漢中房陵東南過臨沮界又束過枝江縣南入于江漳水出臨沮縣東荆山南至枝江縣北入于沮二水上下皆蠻所居也沮于余翻漳諸良翻】立武寧郡更招集流民立綏安郡【綏安郡治長寧縣隋省長寧入長林】詔徵廣州刺史刁逵豫章太守郭昶之【昶丑兩翻】玄皆留不遣玄自謂有晉國三分之二數使人上已符瑞【數所角翻上時掌翻】欲以惑衆又致牋於會稽王道子曰賊造近郊以風不得進以雨不致火食盡故去耳非力屈也【會工外翻造七到翻謂孫恩也】昔國寶死後王恭不乘此威入統朝政【朝直遙翻下在朝同】足見其心非侮於明公也而謂之不忠今之貴要腹心有時流清望者誰乎豈可云無佳勝直是不能信之耳【江東人士其名位通顯於時者率謂之佳勝名勝】爾來一朝一夕遂成今日之禍【爾來猶言如此以來也】在朝君子皆畏禍不言玄沗任在遠是以披寫事實元顯見之大懼張法順謂元顯曰桓玄承藉世資素有豪氣既并殷楊專有荆楚【殷楊謂殷仲堪楊佺期也并殷楊事見上卷隆安三年】第下之所控引止三吳耳【第府第也第下猶言門下閤下之類】孫恩為亂東土塗地公私困竭玄必乘此縱其姦兇竊用憂之元顯曰為之奈何法順曰玄始得荆州人情未附方務綏撫未暇佗圖若乘此際使劉牢之為前鋒而第下以大軍繼進玄可取也元顯以為然會武昌太守庾楷以玄與朝廷構怨恐事不成禍及於已【庾楷歸桓玄見一百十卷隆安二年】密使人自結於元顯云玄大失人情衆不為用若朝廷遣軍已當為内應元顯大喜遣張法順至京口謀於劉牢之牢之以為難法順還謂元顯曰觀牢之言色必貳於我不如召入殺之不爾敗人大事【敗必邁翻】元顯不從於是大治水軍徵兵裝艦以謀討玄【治直之翻艦戶黯翻】<br />
<br />
  元興元年春正月庚午朔【是年三月元顯敗復隆安年號桓玄尋改曰大亨玄簒又改曰永始元興之元改於是年正月通鑑自是年迄義熙初元皆不改元興之元不與桓玄之簒撥亂世返之正也】下詔罪狀桓玄以尚書令元顯為驃騎大將軍征討大都督都督十八州諸軍事加黄鉞【時晉之境内有揚徐南徐兖南兖豫南豫青冀司荆江雍梁益寧交廣十八州而已元顯盡督之使其果能誅桓玄亦必凌上而簒奪之驃匹妙翻騎奇寄翻】又以鎮北將軍劉牢之為前鋒都督前將軍譙王尚之為後部因大赦改元内外戒嚴加會稽王道子太傅【會工外翻】元顯欲盡誅諸桓中護軍桓脩驃騎長史王誕之甥也誕有寵於元顯因陳脩等與玄志趣不同元顯乃止誕導之曾孫也張灋順言於元顯曰桓謙兄弟每為上流耳目宜斬之以杜姦謀且事之濟不繫在前軍【不讀曰否】而牢之反覆萬一有變則禍敗立至可令牢之殺謙兄弟以示無貳心若不受命當逆為之所【逆為之所及禍患未來而先為之圖欲殺牢之也】元顯曰今非牢之無以敵玄且始事而誅大將【將即亮翻】人情不安再三不可【法順與元顯言之再三終以為不可也】又以桓氏世為荆土所附桓冲特有遺惠而謙冲之子也乃自驃騎司馬除都督荆益寧梁四州諸軍事荆州刺史欲以結西人之心 丁丑燕慕容拔攻魏令支戍克之宿沓干走執魏遼西太守那頡【那諾何翻姓也魏書官氏志内入諸姓有那氏頡胡結翻】燕以拔為幽州刺史鎮令支以中堅將軍遼西陽豪為本郡太守丁亥以章武公淵為尚書令博陵公䖍為尚書左僕射尚書王騰為右僕射 戊子魏材官將軍和突攻黜弗素古延等諸部破之初魏主珪遣北部大人賀狄干獻馬千匹求昏於秦秦王興聞珪已立慕容后【立慕容后事見上卷隆安四年】止狄干而絶其昏沒弈干黜弗素古延皆秦之屬國也而魏攻之由是秦魏有隙庚寅珪大閲士馬命并州諸郡積穀於平陽之乾壁以備秦【魏收地形志平陽禽昌縣漢晉之北屈也有乾城隋并禽昌入襄陵又據姚興載記乾壁即乾城】柔然社崙方睦於秦遣將救黜弗素古延【崙盧昆翻將即亮翻】辛卯和突逆擊大破之社崙帥其部落遠遁漠北奪高車之地而居之【帥讀曰率】斛律部帥倍侯利擊社崙大為所敗【帥所類翻敗補邁翻】倍侯利奔魏社崙於是西北擊匈奴遺種日拔也雞大破之【種章勇翻】遂吞併諸部士馬繁盛雄於北方其地西至焉耆東接朝鮮【朝音潮鮮音仙】南臨大漠旁側小國皆羈屬焉自號豆代可汗【魏收書作丘豆代魏言駕馭開張也汗音寒社佑曰可汗之號起於柔然社崙猶言皇帝也而拓拔氏之先通鑑皆書可汗又在社崙之前】始立約束以千人為軍軍有將百人為幢幢有帥【軍將幢帥皆魏制社崙盖效而立之將即亮翻幢宅江翻帥所類翻】攻戰先登者賜以虜獲畏懦者以石擊其首而殺之【柔然為魏患自此始】 秃髪傉檀克顯美執孟禕而責之以其不早降【秃髪傉檀自去年攻顯美至是乃克】禕曰禕受呂氏厚恩分符守土若明公大軍甫至望旗歸附恐獲罪於執事矣傉檀釋而禮之徙二千餘戶而歸以褘為左司馬褘辭曰呂氏將亡聖朝必取河右【朝直遙翻】人無愚智皆知之但褘為人守城不能全復忝顯任於心竊所未安【為于偽翻復扶又翻】若蒙明公之惠使得就戮姑臧死且不朽傉檀義而歸之 東土遭孫恩之亂因以饑饉漕運不繼桓玄禁斷江路【斷讀曰短】公私匱乏以粰橡給士卒【粰房尤翻博雅曰粰粰饊也又曰鬻也又稃穀皮也音同橡似兩翻說文曰栩實也】玄謂朝廷方多憂虞必未暇討已可以蓄力觀釁及大軍將發從兄太傅長史石生密以書報之【從才用翻】玄大驚欲完聚保江陵長史卞範之曰明公英威振於遠近元顯口尚乳臭劉牢之大失物情若兵臨近畿示以禍福土崩之勢可翹足而待何有延敵入境自取窮蹙者乎玄從之留桓偉守江陵抗表傳檄罪狀元顯舉兵東下檄至元顯大懼二月丙午帝餞元顯於西池元顯下船而不發【元顯内畏桓玄故下船而不發】癸丑魏常山王遵等至高平【去年十二月魏遣遵等襲高平至是而至】沒<br />
<br />
  弈干棄其部衆帥數千騎與劉勃勃奔秦州【秦州治上邽帥讀曰率騎奇寄翻】魏軍追至瓦亭不及而還盡獲其府庫蓄積馬四萬餘匹雜畜九萬餘口【畜許救翻】徙其民於代都【魏以平城為代都】餘種分迸【種章勇翻迸北孟翻】平陽太守貳塵復侵秦河東【姓譜春秋貳軫二國後皆為氏貳塵之貳則虜姓也復扶又翻】長安大震關中諸城晝閉秦人簡兵訓卒以謀伐魏 秦王興立子泓為太子大赦泓孝友寛和喜文學【喜許記翻】善談詠而懦弱多病興欲以為嗣而狐疑不决久乃立之【為姚泓以懦弱亡秦張本】 姑臧大饑米斗直錢五千人相食餓死者十餘萬口城門晝閉樵采路絶民請出城為胡虜奴婢者日有數百呂隆惡其沮動衆心【惡烏路翻沮在呂翻】盡阬之積尸盈路沮渠蒙遜引兵攻姑臧【沮子余翻】隆遣使求救於河西王利鹿孤利鹿孤遣廣武公傉檀帥騎一萬救之【使疏吏翻傉奴沃翻帥讀曰率騎奇寄翻】未至隆擊破蒙遜軍蒙遜請與隆盟留穀萬餘斛遺之而還【遺于季翻】傉檀至昌松聞蒙遜已退乃徙凉澤段冢民五百餘戶而還【凉澤即禹貢之豬野澤也在武威縣東亦曰休屠澤還從宣翻】中散騎常侍張融言於利鹿孤曰【散悉亶翻騎奇寄翻】焦朗兄弟據魏安潛通姚氏數為反覆【數所角翻】今不取後必為朝廷憂利鹿孤遣傉檀討之朗面縛出降【焦朗以魏安招秦軍事見去年五月降戶江翻下同】傉檀送於西平徙其民於樂都【樂音洛】 桓玄發江陵慮事不捷常為西還之計及過尋陽不見官軍意甚喜將士之氣亦振【史言桓玄畏怯劉牢之等不能仗順取之將即亮翻下同】庾楷謀泄玄囚之丁巳詔遣齊王柔之以騶虞幡宣告荆江二州使罷兵【騶則尤翻】玄前鋒殺之柔之宗之子也【孝武太元十年以柔之襲封齊王紹攸冏之祀宗封南頓王】丁卯玄至姑孰使其將馮該等攻歷陽【豫州刺史治歷陽】襄城太守司馬休之嬰城固守玄軍斷洞浦【洞浦即洞口魏曹休破呂範處斷丁管翻】焚豫州舟艦【艦戶黯翻】豫州刺史譙王尚之帥步卒九千陣於浦上【帥讀曰率】遣武都太守楊秋屯横江秋降于玄軍尚之衆潰逃于涂中玄捕獲之【涂與滁同】司馬休之出戰而敗棄城走劉牢之素惡驃騎大將軍元顯【惡烏路翻】恐桓玄既滅元顯益驕恣又恐已功名愈盛不為元顯所容且自恃材武擁彊兵欲假玄以除執政復伺玄之隙而自取之【復扶又翻伺相吏翻】故不肯討玄元顯日夜昏酣以牢之為前鋒牢之驟詣門不得見及帝出餞元顯遇之公坐而已【坐徂卧翻】牢之軍溧州【溧音栗溧水出溧陽縣在建康東南元顯遣牢之西上擊桓玄非其路也晉書劉牢之傳作洌洲又桓冲發建康謝安送至溧洲宋武陵王討元凶邵四月戊午至南州辛酉次溧洲丙寅次江寧今舟行自采石東下未至三山江中有洌山即洌洲也洌溧聲相近故又為溧洲張舜民曰過三山十餘里至溧洲自溧洲過白土磯入慈湖夾舜民郴行録言沂流之先後水程也】參軍劉裕請擊玄牢之不許玄使牢之族舅何穆說牢之曰自古戴震主之威挾不賞之功而能自全者誰邪越之文種秦之白起漢之韓信皆事明主為之盡力【說輸芮翻為于偽翻】功成之日猶不免誅夷况為凶愚者之用乎君如今日戰勝則傾宗戰敗則覆族欲以此安歸乎不若翻然改圖則可以長保富貴矣古人射鉤斬袪猶不害為輔佐【齊桓公與子糾爭國管仲射桓公中帶鉤子糾死桓公釋管仲之囚而以為相晉獻公使寺人披伐公子重耳於蒲城重耳踰垣而走披斬其袪重耳反國披屢納忠射而亦翻】况玄與君無宿昔之怨乎時譙王尚之已敗人情愈恐牢之頗納穆言與玄交通東海中尉東海何無忌牢之之甥也與劉裕極諫不聽其子驃騎從事中郎敬宣諫曰今國家衰危天下之重在大人與玄玄藉父叔之資【玄父溫叔冲】據有全楚割晉國三分之二一朝縱之使陵朝廷玄威望旣成恐難圖也董卓之變將在今矣【董卓事見五十九卷漢靈帝中平六年獻帝初平元年】牢之怒曰吾豈不知今日取玄如反覆手耳但平玄之後令我奈驃騎何【元顯為驃騎將軍故稱之】三月乙巳朔牢之遣敬宣詣玄請降【降戶江翻】玄陰欲誅牢之乃與敬宣宴飲陳名書畫共觀之以安悦其意敬宣不之覺玄佐吏莫不相視而笑玄板敬宣為諮議參軍【未有朝命以板授之也】元顯將聞玄已至新亭棄船退屯國子學辛未陳於宣陽門外軍中相驚言玄已至南桁【陳讀曰陣南桁即朱雀桁在臺城南桁戶剛翻】元顯引兵欲還宫玄遣人拔刀隨後大呼曰放仗軍人皆崩潰【呼火故翻】元顯乘馬走入東府唯張灋順一騎隨之【騎奇寄翻】元顯問計於道子道子但對之涕泣玄遣太傅從事中郎毛泰收元顯送新亭縛於舫前而數之【數所具翻舫甫曠翻】元顯曰為王誕張灋順所誤耳壬申復隆安年號帝遣侍中勞玄於安樂渚【勞力到翻樂音洛】玄入京師稱詔解嚴以玄總百揆都督中外諸軍事丞相録尚書事揚州牧領徐荆江三州刺史假黄鉞【是時晉土全有荆江楊三州徐州率多僑郡而京口則重鎮也玄悉領之全有晉國矣且將奪劉牢之之兵故領徐州以制之】玄以桓偉為荆州刺史桓謙為尚書左僕射桓脩為徐兖二州刺史桓石生為江州刺史卞範之為丹陽尹【卞範之為玄長史謀主也丹陽京邑故使為尹】初玄之舉兵侍中王謐奉詔詣玄玄親禮之及玄輔政以謐為中書令謐導之孫也新安太守殷仲文覬之弟也【殷覬見一百九卷隆安元年覬音冀】玄姊為仲文妻仲文聞玄克京師棄郡投玄玄以為諮議參軍劉邁往見玄玄曰汝不畏死而敢來邪邁曰射鉤斬袪并邁為三玄悦以為參軍【邁折玄事見一百八卷孝武太元十七年】癸酉有司奏會稽王道子酣縱不孝當棄市【會工外翻】詔徙安成郡【吳孫皓寶鼎二年分豫章長沙廬陵立安成郡唐吉州安福縣及袁州諸縣皆其地也劉昫曰安福縣吳安成郡治】斬元顯及東海王彦璋【彦璋元顯之子隆安初使繼東海王後】譙王尚之庾楷張灋順毛泰等於建康市桓脩為王誕固請得流嶺南【為于偽翻】玄以劉牢之為會稽内史牢之曰始爾便奪我兵禍其至矣劉敬宣請歸諭牢之使受命玄遣之敬宣勸牢之襲玄牢之猶豫不决移屯班瀆【班瀆在新洲西南】私告劉裕曰今當北就高雅之於廣陵舉兵以匡社稷卿能從我去乎裕曰將軍以勁卒數萬望風降服彼新得志威震天下朝野人情皆已去矣【降戶江翻朝直遙翻】廣陵豈可得至邪裕當反服還京口耳【反服謂反初服也離騷曰退將脩吾初服此言釋戎服而服常服】何無忌謂裕曰我將何之裕曰吾觀鎮北必不免【牢之以討孫恩功進號鎮北將軍】卿可隨我還京口桓玄若守臣節當與卿事之不然當與卿圖之【此時劉裕已有誅玄之心】於是牢之大集僚佐議據江北以討玄參軍劉襲曰事之不可者莫大於反將軍往年反王兖州【王兖州謂王㳟】近日反司馬郎君【司馬郎君謂元顯】今復反桓公【復扶又翻下復推同】一人三反何以自立語畢趨出佐吏多散走牢之懼使敬宣之京口迎家失期不至牢之以為事已泄為玄所殺乃率部曲北走【帥讀曰率】至新洲縊而死敬宣至不暇哭即渡江奔廣陵將吏共殯殮牢之【將即亮翻殮力贍翻】以其喪歸丹徒玄令斲棺斬首暴尸於市【暴步卜翻又如字】 大赦改元大亨 桓玄讓丞相荆江徐三州【玄既以其兄弟領荆江徐三州且已得建康握朝權不復以丞相為重故悉讓之】改授太尉都督中外諸軍事揚州牧領豫州刺史總百揆以琅邪王德文為太宰 司馬休之劉敬宣高雅之俱奔洛陽各以子弟為質於秦以求救【質音致】秦王興與之符信使於關中募兵得數千人復還屯彭城間 孫恩寇臨海臨海太守辛景擊破之恩所虜三吳男女死亡殆盡恩恐為官軍所獲乃赴海死其黨及妓妾從死者以百數【妓渠綺翻從才用翻】謂之水仙餘衆數千人復推恩妹夫盧循為王【復扶又翻】循諶之曾孫也【盧諶盧志之子初從劉琨琨死仕於段氏段遼敗仕趙諶氏壬翻】神采清秀雅有材藝少時【少詩照翻】沙門惠遠嘗謂之曰君雖體涉風素而志存不軌如何太尉玄欲撫安東土乃以循為永嘉太守【明帝太寧元年分臨海立永嘉郡今之溫州】循雖受命而寇暴不已【盧循事始此】 甲戌燕大赦 河西王秃髪利鹿孤寢疾遺令以國事授弟傉檀初秃髪思復鞬愛重傉檀【傉奴沃翻鞬居言翻】謂諸子曰傉檀器識非汝曹所及也故諸兄不以傳子而傳於弟【吳壽夢以少子季札為賢故其諸子兄弟相傳欲以次傳國於季札而季札終於不受禿髪烏孤利鹿孤致國於傉檀猶吳志也豈知國亡於傉檀之手哉】利鹿孤在位垂拱而已【垂拱謂垂衣拱手無所為也】軍國大事皆委於傉檀利鹿孤卒傉檀襲位更稱凉王【自此史稱秃髪氏為南凉】改元弘昌遷于樂都【樂音洛】諡利鹿孤曰康王 夏四月太尉玄出屯姑孰辭録尚書事詔許之而大政皆就諮焉小事則决於尚書令桓謙及卞範之自隆安以來中外之人厭於禍亂及玄初至黜姦佞擢雋賢京師欣然冀得少安【少詩沼翻】旣而玄奢豪縱逸政令無常朋黨互起陵侮朝廷裁損乘輿供奉之具帝幾不免饑寒【乘繩證翻幾居希翻】由是衆心失望三吳大饑戶口減半會稽減什三四【會工外翻】臨海永嘉殆盡富室皆衣羅紈懷金玉閉門相守餓死【此固上之人失政所致而人消物盡亦天地之大數也周餘黎民靡有孑遺以此觀之容有是事衣於旣翻】乞伏熾磐自西平逃歸苑川【乞伏乾歸送熾磐於西平見上卷隆安四年】南凉王傉檀歸其妻子乞伏乾歸使熾磐入朝于秦【朝直遙翻】秦主興以熾磐為興晉太守 五月盧循自臨海入東陽太尉玄遣撫軍中兵參軍劉裕將兵擊之【將即亮翻下同】循敗走永嘉【走音奏】 高句麗攻宿軍【宿軍城在龍城東北句如字又音駒麗力知翻】燕平州刺史慕容歸棄城走【北燕平州刺史治宿軍】 秦主興大發諸軍遣義陽公平尚書右僕射狄伯支等將步騎四萬伐魏興自將大軍繼之以尚書令姚晃輔太子泓守長安沒弈干權鎮上邽廣陵公欽權鎮洛陽平攻魏乾壁六十餘日拔之秋七月魏主珪遣毗陵王順及豫州刺史長孫肥將六萬騎為前鋒自將大軍繼發以擊之 八月太尉玄諷朝廷以玄平元顯功封豫章公平殷楊功封桂陽公并本封南郡如故玄以豫章封其子昇桂陽封其兄子俊 魏主珪至永安【永安本漢彘縣屬河東郡順帝改曰永安晉屬平陽郡隋唐晉州之霍邑縣本永安縣也酈道元曰永安故霍伯之都也縣有霍太山】秦義陽公平遣驍將帥精騎二百覘魏軍【驍堅堯翻將即亮翻騎奇寄翻覘丑亷翻又丑艶翻】長孫肥逆擊盡擒之平退走珪追之乙巳及於柴壁平嬰城固守魏軍圍之秦王興將兵四萬七千救之將據天渡運糧以餽平【柴壁在汾東天渡盖汾津之名在汾水西岸】魏博士李先曰兵法高者為敵所棲深者為敵所囚今秦皆犯之宜及興未至遣騎兵先據天渡柴壁可不戰而取也珪命增築重圍【重直龍翻下同】内以防平之出外以拒興之入廣武將軍安同曰汾東有蒙坑東西三百餘里蹊徑不通興來必從汾西直臨柴壁如此虜聲勢相接重圍雖固不能制也不如為浮梁渡汾西築圍以拒之虜至無所施其智力矣珪從之興至蒲阪【阪音反】憚魏之彊久乃進兵甲子珪帥步騎三萬逆擊興於蒙坑之南【帥讀曰率下同】斬首千餘級興退走四十餘里平亦不敢出珪乃分兵四據險要使秦兵不得近柴壁【近其靳翻】興屯汾西憑壑為壘東栢材從汾上流縱之欲以毁浮梁魏人皆鈎取以為薪蒸【麄曰薪細曰蒸】冬十月平糧竭矢盡夜悉衆突西南圍求出興列兵汾西舉烽鼓譟為應興欲平力戰突免平望興攻圍引接但叫呼相和【和戶卧翻】莫敢逼圍平不得出計窮乃帥麾下赴水死諸將多從平赴水珪使善游者鉤捕之無得免者執狄伯支及越騎校尉唐小方等四十餘人餘衆二萬餘人皆斂手就禽興坐視其窮力不能救舉軍慟哭聲震山谷數遣使求和於魏【數所角翻】珪不許乘勝進攻蒲阪秦晉公緒固守不戰會柔然謀伐魏珪聞之戊申引兵還【還從宣翻又如字】或告太史令鼂崇及弟黃門侍郎懿潜召秦兵珪至晉陽賜崇懿死【鼂直遙翻】秦徙河西豪右萬餘戶于長安 太尉玄殺吳興太<br />
<br />
  守高素將軍竺謙之及謙之從兄朗之劉襲并襲弟季武皆劉牢之北府舊將也【從才用翻將即亮翻】襲兄冀州刺史軌邀司馬休之劉敬宣高雅之等共據山陽【沈約曰山陽本射陽縣境地名義熙土斷始分廣陵郡立山陽郡及山陽縣唐楚州即其地】欲起兵攻玄不克而走將軍袁䖍之劉壽高長慶郭恭等皆往從之將奔魏至陳留南分為二輩軌休之敬宣奔南燕䖍之壽長慶恭奔秦魏主珪初聞休之等當來大喜後怪其不至令兖州求訪【隆安五年魏以長孫肥為兖州刺史南徇地然未能有兖州也】獲其從者問其故皆曰魏朝威聲遠被【從才用翻被皮義翻朝直遙翻】是以休之等咸欲歸附旣而聞崔逞被殺【崔逞死見上卷三年】故犇二國珪深悔之自是士人有過頗見優容 南凉王傉檀攻呂隆於姑臧 燕王熙納故中山尹苻謨二女【燕王實即位之初苻謨為中山尹】長曰娀娥為貴人【長知兩翻娀音戎】幼曰訓英為貴嬪貴嬪尤有寵丁太后怨恚【恚於避翻】與兄子尚書信謀廢熙立章武公淵事覺熙逼丁太后令自殺葬以后禮諡曰獻幽皇后【丁太后素與熙通事見上年】十一月戊辰殺淵及信辛未熙畋于北原【龍城北原也】石城令高和【石城縣漢屬北平郡燕屬建德郡高和本為石城令時以大喪會于龍城耳】與尚方兵於後作亂殺司隸校尉張顯人掠宫殿取庫兵脅營署閉門乘城熙馳還城上人皆投仗開門盡誅反者唯和走免甲戌大赦 魏以庾岳為司空 十二月辛亥魏主珪還雲中柔然可汗社崙聞珪伐秦自參合陂侵魏至豺山【豺山在善無縣北魏天興六年築宫於此崙盧昆翻】及善無北澤魏常山王遵以萬騎追之不及而還太尉玄使御史杜林防衛會稽文孝王道子至安成<br />
<br />
  【會工外翻】林承玄旨酖道子殺之 沮渠蒙遜所署西郡太守梁中庸叛奔西凉蒙遜聞之笑曰吾待中庸恩如骨肉而中庸不我信但自負耳孤豈在此一人邪乃盡歸其孥西凉公暠問中庸曰我何如索嗣中庸曰未可量也暠曰嗣才度若敵我者我何能於千里之外以長繩絞其頸邪【索嗣死見上卷四年孥音奴暠古老翻索晋各翻量音良】中庸曰智有短長命有成敗殿下之與索嗣得失之理臣實未之能詳若以身死為負計行為勝則公孫瓚豈賢於劉虞邪【公孫瓚劉虞事見六十卷漢獻帝初平四年】暠默然 袁䖍之等至長安秦王興問曰桓玄才略何如其父卒能成功乎【卒子恤翻】䖍之曰玄乘晉室衰亂盜據宰衡猜忌安忍刑賞不公以臣觀之不如其父遠矣玄今已執大柄其埶必將簒逆正可為它人驅除耳興善之以䖍之為廣州刺史【秦以廣州授袁䖍之示以名位寵綏之耳】 是歲秦王興立昭儀張氏為皇后封子懿弼洸宣諶愔璞質逵裕國兒皆為公【洸古黃翻諶氏壬翻愔於今翻】遣使拜秃髮傉檀為車騎將軍廣武公沮渠蒙遜為鎮西將軍沙州刺史西海侯李暠為安西將軍高昌侯秦鎮遠將軍趙曜帥衆二萬西屯金城建節將軍王松怱帥騎助呂隆守姑臧松怱至魏安傉檀弟文真擊而虜之傉檀大怒送松怱還長安深自陳謝【史言河湟諸國皆畏姚秦之彊】<br />
<br />
  資治通鑑卷一百十二<br />
<br />
<史部,編年類,資治通鑑>  <br>
   </div> 

<script src="/search/ajaxskft.js"> </script>
 <div class="clear"></div>
<br>
<br>
 <!-- a.d-->

 <!--
<div class="info_share">
</div> 
-->
 <!--info_share--></div>   <!-- end info_content-->
  </div> <!-- end l-->

<div class="r">   <!--r-->



<div class="sidebar"  style="margin-bottom:2px;">

 
<div class="sidebar_title">工具类大全</div>
<div class="sidebar_info">
<strong><a href="http://www.guoxuedashi.com/lsditu/" target="_blank">历史地图</a></strong>  
<a href="http://www.880114.com/" target="_blank">英语宝典</a>  
<a href="http://www.guoxuedashi.com/13jing/" target="_blank">十三经检索</a> 
<br><strong><a href="http://www.guoxuedashi.com/gjtsjc/" target="_blank">古今图书集成</a></strong> 
<a href="http://www.guoxuedashi.com/duilian/" target="_blank">对联大全</a> <strong><a href="http://www.guoxuedashi.com/xiangxingzi/" target="_blank">象形文字典</a></strong> 

<br><a href="http://www.guoxuedashi.com/zixing/yanbian/">字形演变</a>  <strong><a href="http://www.guoxuemi.com/hafo/" target="_blank">哈佛燕京中文善本特藏</a></strong>
<br><strong><a href="http://www.guoxuedashi.com/csfz/" target="_blank">丛书&方志检索器</a></strong> <a href="http://www.guoxuedashi.com/yqjyy/" target="_blank">一切经音义</a>  

<br><strong><a href="http://www.guoxuedashi.com/jiapu/" target="_blank">家谱族谱查询</a></strong>  <strong><a href="http://shufa.guoxuedashi.com/sfzitie/" target="_blank">书法字帖欣赏</a></strong> 
<br>

</div>
</div>


<div class="sidebar" style="margin-bottom:0px;">

<font style="font-size:22px;line-height:32px">QQ交流群9:489193090</font>


<div class="sidebar_title">手机APP 扫描或点击</div>
<div class="sidebar_info">
<table>
<tr>
	<td width=160><a href="http://m.guoxuedashi.com/app/" target="_blank"><img src="/img/gxds-sj.png" width="140"  border="0" alt="国学大师手机版"></a></td>
	<td>
<a href="http://www.guoxuedashi.com/download/" target="_blank">app软件下载专区</a><br>
<a href="http://www.guoxuedashi.com/download/gxds.php" target="_blank">《国学大师》下载</a><br>
<a href="http://www.guoxuedashi.com/download/kxzd.php" target="_blank">《汉字宝典》下载</a><br>
<a href="http://www.guoxuedashi.com/download/scqbd.php" target="_blank">《诗词曲宝典》下载</a><br>
<a href="http://www.guoxuedashi.com/SiKuQuanShu/skqs.php" target="_blank">《四库全书》下载</a><br>
</td>
</tr>
</table>

</div>
</div>


<div class="sidebar2">
<center>


</center>
</div>

<div class="sidebar"  style="margin-bottom:2px;">
<div class="sidebar_title">网站使用教程</div>
<div class="sidebar_info">
<a href="http://www.guoxuedashi.com/help/gjsearch.php" target="_blank">如何在国学大师网下载古籍?</a><br>
<a href="http://www.guoxuedashi.com/zidian/bujian/bjjc.php" target="_blank">如何使用部件查字法快速查字?</a><br>
<a href="http://www.guoxuedashi.com/search/sjc.php" target="_blank">如何在指定的书籍中全文检索?</a><br>
<a href="http://www.guoxuedashi.com/search/skjc.php" target="_blank">如何找到一句话在《四库全书》哪一页?</a><br>
</div>
</div>


<div class="sidebar">
<div class="sidebar_title">热门书籍</div>
<div class="sidebar_info">
<a href="/so.php?sokey=%E8%B5%84%E6%B2%BB%E9%80%9A%E9%89%B4&kt=1">资治通鉴</a> <a href="/24shi/"><strong>二十四史</strong></a>&nbsp; <a href="/a2694/">野史</a>&nbsp; <a href="/SiKuQuanShu/"><strong>四库全书</strong></a>&nbsp;<a href="http://www.guoxuedashi.com/SiKuQuanShu/fanti/">繁体</a>
<br><a href="/so.php?sokey=%E7%BA%A2%E6%A5%BC%E6%A2%A6&kt=1">红楼梦</a> <a href="/a/1858x/">三国演义</a> <a href="/a/1038k/">水浒传</a> <a href="/a/1046t/">西游记</a> <a href="/a/1914o/">封神演义</a>
<br>
<a href="http://www.guoxuedashi.com/so.php?sokeygx=%E4%B8%87%E6%9C%89%E6%96%87%E5%BA%93&submit=&kt=1">万有文库</a> <a href="/a/780t/">古文观止</a> <a href="/a/1024l/">文心雕龙</a> <a href="/a/1704n/">全唐诗</a> <a href="/a/1705h/">全宋词</a>
<br><a href="http://www.guoxuedashi.com/so.php?sokeygx=%E7%99%BE%E8%A1%B2%E6%9C%AC%E4%BA%8C%E5%8D%81%E5%9B%9B%E5%8F%B2&submit=&kt=1"><strong>百衲本二十四史</strong></a>  <a href="http://www.guoxuedashi.com/so.php?sokeygx=%E5%8F%A4%E4%BB%8A%E5%9B%BE%E4%B9%A6%E9%9B%86%E6%88%90&submit=&kt=1"><strong>古今图书集成</strong></a>
<br>

<a href="http://www.guoxuedashi.com/so.php?sokeygx=%E4%B8%9B%E4%B9%A6%E9%9B%86%E6%88%90&submit=&kt=1">丛书集成</a> 
<a href="http://www.guoxuedashi.com/so.php?sokeygx=%E5%9B%9B%E9%83%A8%E4%B8%9B%E5%88%8A&submit=&kt=1"><strong>四部丛刊</strong></a>  
<a href="http://www.guoxuedashi.com/so.php?sokeygx=%E8%AF%B4%E6%96%87%E8%A7%A3%E5%AD%97&submit=&kt=1">說文解字</a> <a href="http://www.guoxuedashi.com/so.php?sokeygx=%E5%85%A8%E4%B8%8A%E5%8F%A4&submit=&kt=1">三国六朝文</a>
<br><a href="http://www.guoxuedashi.com/so.php?sokeytm=%E6%97%A5%E6%9C%AC%E5%86%85%E9%98%81%E6%96%87%E5%BA%93&submit=&kt=1"><strong>日本内阁文库</strong></a> <a href="http://www.guoxuedashi.com/so.php?sokeytm=%E5%9B%BD%E5%9B%BE%E6%96%B9%E5%BF%97%E5%90%88%E9%9B%86&ka=100&submit=">国图方志合集</a> <a href="http://www.guoxuedashi.com/so.php?sokeytm=%E5%90%84%E5%9C%B0%E6%96%B9%E5%BF%97&submit=&kt=1"><strong>各地方志</strong></a>

</div>
</div>


<div class="sidebar2">
<center>

</center>
</div>
<div class="sidebar greenbar">
<div class="sidebar_title green">四库全书</div>
<div class="sidebar_info">

《四库全书》是中国古代最大的丛书,编撰于乾隆年间,由纪昀等360多位高官、学者编撰,3800多人抄写,费时十三年编成。丛书分经、史、子、集四部,故名四库。共有3500多种书,7.9万卷,3.6万册,约8亿字,基本上囊括了古代所有图书,故称“全书”。<a href="http://www.guoxuedashi.com/SiKuQuanShu/">详细>>
</a>

</div> 
</div>

</div>  <!--end r-->

</div>
<!-- 内容区END --> 

<!-- 页脚开始 -->
<div class="shh">

</div>

<div class="w1180" style="margin-top:8px;">
<center><script src="http://www.guoxuedashi.com/img/plus.php?id=3"></script></center>
</div>
<div class="w1180 foot">
<a href="/b/thanks.php">特别致谢</a> | <a href="javascript:window.external.AddFavorite(document.location.href,document.title);">收藏本站</a> | <a href="#">欢迎投稿</a> | <a href="http://www.guoxuedashi.com/forum/">意见建议</a> | <a href="http://www.guoxuemi.com/">国学迷</a> | <a href="http://www.shuowen.net/">说文网</a><script language="javascript" type="text/javascript" src="https://js.users.51.la/17753172.js"></script><br />
  Copyright &copy; 国学大师 古典图书集成 All Rights Reserved.<br>
  
  <span style="font-size:14px">免责声明:本站非营利性站点,以方便网友为主,仅供学习研究。<br>内容由热心网友提供和网上收集,不保留版权。若侵犯了您的权益,来信即刪。scp168@qq.com</span>
  <br />
ICP证:<a href="http://www.beian.miit.gov.cn/" target="_blank">鲁ICP备19060063号</a></div>
<!-- 页脚END --> 
<script src="http://www.guoxuedashi.com/img/plus.php?id=22"></script>
<script src="http://www.guoxuedashi.com/img/tongji.js"></script>

</body>
</html>
