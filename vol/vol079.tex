<!DOCTYPE html PUBLIC "-//W3C//DTD XHTML 1.0 Transitional//EN" "http://www.w3.org/TR/xhtml1/DTD/xhtml1-transitional.dtd">
<html xmlns="http://www.w3.org/1999/xhtml">
<head>
<meta http-equiv="Content-Type" content="text/html; charset=utf-8" />
<meta http-equiv="X-UA-Compatible" content="IE=Edge,chrome=1">
<title>資治通鑒_80-資治通鑑卷七十九_80-資治通鑑卷七十九</title>
<meta name="Keywords" content="資治通鑒_80-資治通鑑卷七十九_80-資治通鑑卷七十九">
<meta name="Description" content="資治通鑒_80-資治通鑑卷七十九_80-資治通鑑卷七十九">
<meta http-equiv="Cache-Control" content="no-transform" />
<meta http-equiv="Cache-Control" content="no-siteapp" />
<link href="/img/style.css" rel="stylesheet" type="text/css" />
<script src="/img/m.js?2020"></script> 
</head>
<body>
 <div class="ClassNavi">
<a  href="/24shi/">二十四史</a> | <a href="/SiKuQuanShu/">四库全书</a> | <a href="http://www.guoxuedashi.com/gjtsjc/"><font  color="#FF0000">古今图书集成</font></a> | <a href="/renwu/">历史人物</a> | <a href="/ShuoWenJieZi/"><font  color="#FF0000">说文解字</a></font> | <a href="/chengyu/">成语词典</a> | <a  target="_blank"  href="http://www.guoxuedashi.com/jgwhj/"><font  color="#FF0000">甲骨文合集</font></a> | <a href="/yzjwjc/"><font  color="#FF0000">殷周金文集成</font></a> | <a href="/xiangxingzi/"><font color="#0000FF">象形字典</font></a> | <a href="/13jing/"><font  color="#FF0000">十三经索引</font></a> | <a href="/zixing/"><font  color="#FF0000">字体转换器</font></a> | <a href="/zidian/xz/"><font color="#0000FF">篆书识别</font></a> | <a href="/jinfanyi/">近义反义词</a> | <a href="/duilian/">对联大全</a> | <a href="/jiapu/"><font  color="#0000FF">家谱族谱查询</font></a> | <a href="http://www.guoxuemi.com/hafo/" target="_blank" ><font color="#FF0000">哈佛古籍</font></a> 
</div>

 <!-- 头部导航开始 -->
<div class="w1180 head clearfix">
  <div class="head_logo l"><a title="国学大师官网" href="http://www.guoxuedashi.com" target="_blank"></a></div>
  <div class="head_sr l">
  <div id="head1">
  
  <a href="http://www.guoxuedashi.com/zidian/bujian/" target="_blank" ><img src="http://www.guoxuedashi.com/img/top1.gif" width="88" height="60" border="0" title="部件查字,支持20万汉字"></a>


<a href="http://www.guoxuedashi.com/help/yingpan.php" target="_blank"><img src="http://www.guoxuedashi.com/img/top230.gif" width="600" height="62" border="0" ></a>


  </div>
  <div id="head3"><a href="javascript:" onClick="javascript:window.external.AddFavorite(window.location.href,document.title);">添加收藏</a>
  <br><a href="/help/setie.php">搜索引擎</a>
  <br><a href="/help/zanzhu.php">赞助本站</a></div>
  <div id="head2">
 <a href="http://www.guoxuemi.com/" target="_blank"><img src="http://www.guoxuedashi.com/img/guoxuemi.gif" width="95" height="62" border="0" style="margin-left:2px;" title="国学迷"></a>
  

  </div>
</div>
  <div class="clear"></div>
  <div class="head_nav">
  <p><a href="/">首页</a> | <a href="/ShuKu/">国学书库</a> | <a href="/guji/">影印古籍</a> | <a href="/shici/">诗词宝典</a> | <a   href="/SiKuQuanShu/gxjx.php">精选</a> <b>|</b> <a href="/zidian/">汉语字典</a> | <a href="/hydcd/">汉语词典</a> | <a href="http://www.guoxuedashi.com/zidian/bujian/"><font  color="#CC0066">部件查字</font></a> | <a href="http://www.sfds.cn/"><font  color="#CC0066">书法大师</font></a> | <a href="/jgwhj/">甲骨文</a> <b>|</b> <a href="/b/4/"><font  color="#CC0066">解密</font></a> | <a href="/renwu/">历史人物</a> | <a href="/diangu/">历史典故</a> | <a href="/xingshi/">姓氏</a> | <a href="/minzu/">民族</a> <b>|</b> <a href="/mz/"><font  color="#CC0066">世界名著</font></a> | <a href="/download/">软件下载</a>
</p>
<p><a href="/b/"><font  color="#CC0066">历史</font></a> | <a href="http://skqs.guoxuedashi.com/" target="_blank">四库全书</a> |  <a href="http://www.guoxuedashi.com/search/" target="_blank"><font  color="#CC0066">全文检索</font></a> | <a href="http://www.guoxuedashi.com/shumu/">古籍书目</a> | <a   href="/24shi/">正史</a> <b>|</b> <a href="/chengyu/">成语词典</a> | <a href="/kangxi/" title="康熙字典">康熙字典</a> | <a href="/ShuoWenJieZi/">说文解字</a> | <a href="/zixing/yanbian/">字形演变</a> | <a href="/yzjwjc/">金 文</a> <b>|</b>  <a href="/shijian/nian-hao/">年号</a> | <a href="/diming/">历史地名</a> | <a href="/shijian/">历史事件</a> | <a href="/guanzhi/">官职</a> | <a href="/lishi/">知识</a> <b>|</b> <a href="/zhongyi/">中医中药</a> | <a href="http://www.guoxuedashi.com/forum/">留言反馈</a>
</p>
  </div>
</div>
<!-- 头部导航END --> 
<!-- 内容区开始 --> 
<div class="w1180 clearfix">
  <div class="info l">
   
<div class="clearfix" style="background:#f5faff;">
<script src='http://www.guoxuedashi.com/img/headersou.js'></script>

</div>
  <div class="info_tree"><a href="http://www.guoxuedashi.com">首页</a> > <a href="/SiKuQuanShu/fanti/">四库全书</a>
 > <h1>资治通鉴</h1> <!--         下载:【右键另存为】即可 --></div>
  <div class="info_content zj clearfix">
  
<div class="info_txt clearfix" id="show">
<center style="font-size:24px;">80-資治通鑑卷七十九</center>
    資治通鑑卷七十九   宋 司馬光 撰<br />
<br />
  胡三省 音註<br />
<br />
  晉紀一【起旃蒙作噩盡玄黓執徐凡八年 司馬氏河内温縣人宣王懿得魏政傳景王師至文王昭始封晉公以温縣本晉地故以為國號】<br />
<br />
  世祖武皇帝上之上【諱炎字安世姓司馬氏宣王懿之孫文王昭之長子文王廟號太祖故帝廟號世祖謚法克定禍亂曰武】<br />
<br />
  泰始元年【是年十二月方受禪改元此猶是魏咸熙二年】春三月吳主使光禄大夫紀陟五官中郎將洪璆【璆渠尤翻】與徐紹孫彧偕來報聘【紹彧聘吳見上卷上年】紹行至濡須有言紹譽中國之美者【譽音余】吳主怒追還殺之 夏四月吳改元甘露【時因蒋陵言甘露降改元】 五月魏帝加文王殊禮【謂旌旗車馬樂舞冕服皆如帝者之儀】進王妃曰后世子曰太子 癸未大赦 秋七月吳主逼殺景皇后遷景帝四子於吳尋又殺其長者二人【吳主貶景后封四弟事見上卷上年長知兩翻】 八月辛卯文王卒太子嗣為相國晉王 九月乙未大赦 戊子以魏司徒何曾為晉丞相癸亥以票騎將軍司馬望為司徒【票匹妙翻騎奇寄翻】 乙亥葬文王于崇陽陵 【考異曰晉書文紀作癸酉今從魏志陳留王紀】 冬吳西陵督步闡【西陵即夷陵吳主權黄武元年改夷陵曰西陵宜都郡治焉】表請吳主徙都武昌吳主從之使御史大夫丁固右將軍諸葛靚守建業【靚正翻】闡騭之子也【吳主權時騭為西陵督騭之日翻】 十二月壬戌魏帝禪位于晉【魏元帝時年二十困敦上章魏文帝始受漢禪傳五世歷四十六年而亡】甲子出舍于金墉城【金墉城在洛陽城西北角】太傅司馬孚拜辭執帝手流涕歔欷不自勝【歔音虚欷音希又許既翻勝音升】曰臣死之日固大魏之純臣也丙寅王即皇帝位大赦改元【至是方改元泰始】丁卯奉魏帝為陳留王即宫于鄴【即就也】優崇之禮皆倣魏初故事【見六十九卷魏文帝黄初元年】魏氏諸王皆降為侯追尊宣王為宣皇帝景王為景皇帝文王為文皇帝尊王太后曰皇太后封皇叔祖孚為安平王叔父幹為平原王亮為扶風王伷為東莞王駿為汝隂王肜為梁王倫為琅邪王弟攸為齊王鑑為樂安王機為燕王又封羣從司徒望等十七人皆為王【望孚之子也帝封諸王以郡為國邑二萬戶為大國置上中下三軍兵五千人萬戶為次國置上軍下軍兵三千人五千戶為小國置一軍兵五百人王不之國官於京師伷音胄從才用翻莞音官肜余中翻燕於賢翻】以石苞為大司馬鄭冲為太傅王祥為太保何曾為太尉賈充為車騎將軍王沈為驃騎將軍【騎奇寄翻沈持林翻驃匹妙翻】其餘文武增位進爵有差乙亥以安平王孚為太宰都督中外諸軍事【晉志曰太宰太傅太保周之三公官也晉初以景帝諱故又採周官官名置太宰以代太師之任秩增三師與太傅太保皆為上公大司馬古官也漢制以冠大將軍驃騎將軍之上以代太尉之職故恒與太尉迭置不並列及魏有太尉而大司馬大將軍各自為官位在三司上晉因其制以太宰太傅太保司徒司空為文官公左右光禄大夫光祿大夫開府者位從公冠進賢三梁黑介幘大司馬大將軍太尉為武官公驃騎車騎衛將軍伏波撫軍都護鎮軍中軍四征四鎮龍驤典軍上軍輔國等大將軍開府者位從公皆著武冠平上黑幘】未幾【幾居豈翻】又以車騎將軍陳騫為大將軍與司徒義陽王望司空荀顗凡八公同時並置帝懲魏氏孤立之敝故大封宗室授以職任又詔諸王皆得自選國中長吏【長知兩翻】衛將軍齊王攸獨不敢皆令上請【上時掌翻】 詔除魏宗室禁錮罷部曲將及長吏納質任【魏防禁宗室甚峻又錮不得仕進今除之又諸將征戍及長吏仕州郡者皆留質任於京師今亦罷之將即亮翻質音致】 帝承魏氏刻薄奢侈之後矯以仁儉太常丞許奇允之子也【晉太常光禄勲衛尉太僕廷尉大鴻臚宗正大司農少府將作大匠太后三卿大長秋皆為列卿各置丞功曹主簿五官等員】帝將有事於太廟朝議以奇父受誅【奇父允誅事見七十六卷高貴鄉公正元元年朝直遥翻】不宜接近左右【近其靳翻】請出為外官帝乃追述允之宿望稱奇之才擢為祠部郎【魏尚書曹有祠部郎晉因之】有司言御牛青紖斷【紖直忍翻索也牛系也禮迎牲君執紖周禮封人祭祀飾其牛牲置其絼注曰絼著牛鼻繩所以牽牛者今人謂之雉疏曰自漢以前皆謂之絼案禮記少儀牛則執紖紖則絼之别名今亦謂之為紖陸德明曰絼與紖同又以忍翻又周禮釋音羊晉翻】詔以青麻代之 初置諫官以散騎常侍傳玄皇甫陶為之【秦漢以來有諫大夫鄭昌所謂官以諫為名者也東漢有諫議大夫魏不復置晉以散騎常侍拾遺補闕即諫官職也】玄幹之子也【傅幹漢傅爕之子】玄以魏末士風頹敝上疏曰臣聞先王之御天下敎化隆於上清議行於下近者魏武好法術而天下貴刑名【好呼到翻】魏文慕通逹而天下賤守節其後綱維不攝【攝整也】放誕盈朝【謂何晏阮籍輩也朝直遥翻】遂使天下無復清議陛下龍興受禪弘堯舜之化惟未舉清遠有禮之臣以敦風節未退虚鄙之士以懲不恪臣是以猶敢有言上嘉納其言使玄草詔進之然亦不能革也 初漢征西將軍司馬鈞【鈞事見五十卷漢安帝元初二年】生豫章太守量量生潁川太守雋雋生京兆尹防防生宣帝【序司馬氏之世為下立廟張本】<br />
<br />
  二年春正月丁亥即用魏廟祭征西府君以下并景帝凡七室【沈約志曰晉初祭征西將軍豫章府君潁川府君京兆府君與宣皇帝景皇帝文皇帝為三昭三穆是時宣皇未升太祖虚位所以祠六世與景帝為七廟其禮則據王肅說也】 尊景帝夫人羊氏曰景皇后居弘訓宫 丙午立皇后弘農楊氏后魏通事郎文宗之女也【魏黄初初中書既置監令又置通事郎】 羣臣奏五帝即天帝也王氣時異故名號有五自今明堂南郊宜除五帝座從之帝王肅外孫也故郊祀之禮有司多從肅議【周禮曰祀昊天上帝則服大裘而冕祀五帝亦如之鄭玄以為昊天上帝者天皇大帝北辰耀魄寶也五帝者五行精氣之神也曰青帝靈威仰曰赤帝赤熛怒曰黄帝含樞紐曰白帝白招矩曰黑帝汁光紀由是有六天之說六天者指其尊極清虚之體其實是一論其五時生育之功其别有五故為六天據其在上之體謂之天天為體稱故說天云天顛也因其生育之功謂之帝帝為德稱故毛詩傳云審諦如帝王肅駁之以為五帝非天唯用家語之文謂太皥炎帝黃帝少皥顗頊五帝為五人帝晉羣臣祖肅之說以為五帝即天帝王氣時異故殊其號雖五其實一神明堂南郊宜除五帝之座五郊改五精之號同稱昊天上帝從之王于况翻】 二月除漢宗室禁錮【魏既代漢禁錮諸劉今除之】 三月吳遣大鴻臚張儼五官中郎將丁忠來弔祭【以文王之喪也臚陵如翻】 吳散騎常侍王蕃體氣高亮不能承顔順指吳主不悦散騎常侍萬彧中書丞陳聲從而譖之【散悉亶翻騎奇寄翻】丁忠使還【使疏吏翻還從宣翻又如字】吳主大會羣臣蕃沈醉頓伏【沈持林翻下王沈同】吳主疑其詐轝蕃出外【轝羊如翻】頃之召還蕃好治威儀【好呼到翻治直之翻】行止自若吳主大怒呵左右於殿下斬之出登來山【水經注武昌城南有來山即樊山也吳孫皓登之使親近擲王蕃首而虎爭之】使親近擲蕃首作虎跳狼爭咋齧之【跳它弔翻咋側革翻啖也齧魚結翻噬也】首皆碎壞丁忠說吳主曰北方無守戰之備弋陽可襲而取【弋陽縣漢屬汝南郡魏文帝分立弋陽郡說輸芮翻】吳主以問羣臣鎮西大將軍陸凱曰北方新并巴蜀遣使求和非求援於我也欲蓄力以俟時耳敵埶方彊而欲徼幸求勝未見其利也【徼工堯翻】吳主雖不出兵然遂與晉絶凱遜之族子也夏五月壬子博陵元公王沈卒【沈持林翻】 六月丙午晦<br />
<br />
  日有食之 文帝之喪臣民皆從權制三日除服既葬帝亦除之然猶素冠疏食【食祥吏翻】哀毁如居喪者秋八月帝將謁崇陽陵羣臣奏言秋暑未平恐帝悲感摧傷帝曰朕得奉瞻山陵體氣自佳耳又詔曰漢文不使天下盡哀亦帝王至謙之志【漢文帝遺詔見十五卷後七年真德秀曰文帝此詔乃短喪之始也然本文蓋為吏民設耳景帝嗣君也可緣此而短其喪乎】當見山陵何心無服其議以衰絰從行【衰七回翻】羣臣自依舊制尚書令裴秀奏曰陛下既除而復服義無所依若君服而臣不服亦未之敢安也詔曰患情不能跂及耳衣服何在【言患哀慕之情不至耳不在乎衣服也跂去智翻舉踵也】諸君勤勤之至豈苟相違遂止中軍將軍羊祜謂傅玄曰三年之喪雖貴遂服禮也【三年之喪自天子逹于庶人言雖以天子之貴亦得以遂其孝思為三年之服】今主上至孝雖奪其服實行喪禮若因此復先王之法不亦善乎玄曰以日易月已數百年【以日易月漢儒之謬說也注見十五卷漢文帝後七年】一旦復古難行也祜曰不能使天下如禮且使主上遂服不猶愈乎玄曰主上不除而天下除之此為但有父子無復君臣也乃止戊辰羣臣奏請易服復膳詔曰每感念幽冥而不得終苴絰之禮【左傳齊晏桓子卒晏嬰麤縗苴絰帶杜預注云苴麻之有子者取其麤也苴七余翻】以為沈痛【沈持林翻深也】况當食稻衣錦乎【衣於既翻】適足激切其心非所以相解也朕本諸生家傳禮來久何至一旦便易此情於所天相從已多可試省孔子答宰我之言【論語宰我問三年之喪朞已久矣君子三年不為禮禮必壞三年不為樂樂必崩舊穀既没新穀既升朞可已矣孔子曰食夫稻衣夫錦於女安乎曰安孔子曰女安則為之宰我出孔子曰予之不仁也子生三年然後免於父母之懷夫三年之喪天下之通喪也儀禮曰父者子之天省悉景翻】無事紛紜也遂以疏素終三年<br />
<br />
  臣光曰三年之喪自天子逹於庶人此先王禮經百世不易者也漢文師心不學變古壞禮【壞音怪】絶父子之恩虧君臣之義後世帝王不能篤於哀戚之情而羣臣謟諛莫肯釐正【釐力之翻理也】至于晉武獨以天性矯而行之可謂不世之賢君而裴傅之徒固陋庸臣習常玩故而不能將順其美惜哉【孝經曰君子之事上也將順其美匡救其惡注云將奉也】<br />
<br />
  吳改元寶鼎【以所在得大鼎改元】 吳主以陸凱為左丞相萬彧為右丞相吳主惡人視已羣臣侍見莫敢舉目【惡音烏路翻見音賢遍翻】陸凱曰君臣無不相識之道若猝有不虞不知所赴吳主乃聽凱自視而它人如故【唯凱得視之它人仍舊不得視也】吳主居武昌揚州之民泝流供給甚苦之【吳武昌屬荆州而丹陽宣城毗陵吳吳興會稽東陽新都臨海建安豫章臨川鄱陽廬陵皆屬揚州故苦於西上泝流以供給】又奢侈無度公私窮匱凱上疏曰今四邉無事當務養民豐財而更窮奢極欲無災而民命盡無為而國財空臣竊憂之昔漢室既衰三家鼎立今曹劉失道皆為晉有此目前之明驗也臣愚但為陛下惜國家耳武昌土地危險塉确【塉秦昔翻土薄也确音克角翻山多大石也】非王者之都且童謠云寜飲建業水不食武昌魚寧還建業死不止武昌居【此苦於泝流供給而為是謠也】以此觀之足明人心與天意矣今國無一年之蓄【禮記王制國無六年之蓄曰急無三年之蓄曰國非其國也况無一年之蓄乎】民有離散之怨國有露根之漸【以木為喻也本之所以能生殖者以有根本也根漸露則其本將撥】而官吏務為苛急莫之或恤大帝時後宫列女及諸織絡數不滿百景帝以來乃有千數此耗財之甚也又左右之臣率非其人羣黨相扶害忠隱賢此皆蠧政病民者也臣願陛下省息百役罷去苛擾料出宫女【去羌呂翻料音聊】清選百官則天悦民附國家永安矣吳主雖不悦以其宿望特優容之 【考異曰陳夀曰予連從荆揚來者得凱所諫皓二十事博問吳人多云不聞凱有此表又按其文殊甚切直恐非皓之所能容忍也或以為凱藏之篋笥未敢宣行病困皓遣董朝省問欲言因以付之虛實難明故不著于篇然愛其指擿皓事足為後戒故鈔列于凱傳左今不取】 九月詔自今雖詔有所欲及已奏得可而於事不便者皆不可隱情【既不可希指迎合又不可以遂事而不諫也】 戊戍有司奏大晉受禪於魏宜一用前代正朔服色如虞遵唐故事從之【家語季康子問於孔子曰唐虞二帝其所尚何色孔子曰堯以火德王色尚黄舜以土德王色尚青董仲舒策引孔子曰無為而治者其舜乎改正朔易服色以順天命而已其餘盡循堯道何更為哉如二說則舜之承堯固改正朔易服色矣然考之古文尚書堯命羲和歷象日月星辰敬授人時舜正月上日受終于文祖協時用正日而已不言改正朔也易大傳曰黄帝堯舜埀衣裳而天下治書益稷帝曰予欲觀古人之象以五采彰施于五色作服而已不言易服色也漢興六歷有黄帝歷顓頊歷夏歷殷歷周歷魯歷無堯舜歷豈堯舜時用顓頊歷邪孔頴逹以為古之真歷至戰國及秦而亡漢初所存六歷後人託而為之此固無從考正也】 冬十月丙午朔日有食之 【考異曰宋書志無此食今從晉書】 永安山賊施但【吳録曰永安今武康縣也沈約曰吳分烏程餘杭立永安縣晉武帝太康元年更名武康屬吳興郡宋白曰永安縣本漢烏程縣之餘不鄉】因民勞怨聚衆數千人刼吳主庶弟永安侯謙作亂北至建業衆萬餘人未至三十里住擇吉日入城遣使以謙命召丁固諸葛靚固靚斬其使發兵逆戰於牛屯【據吳歷牛屯去建業城二十一里靚正翻】但兵皆無甲胄即時敗散謙獨坐車中生獲之固不敢殺以狀白吳主吳主并其母及弟俊皆殺之初望氣者云荆州有王氣當破揚州【王于况翻】故吳主徙都武昌及但反自以為得計遣數百人鼓譟入建業殺但妻子云天子使荆州兵來破揚州賊 十一月初并圓丘方丘之祀於南北郊【鄭氏注禮記為高必因丘陵謂冬至祭天於圓丘之上為下必因川澤謂夏至祭地於方澤之中而四郊之祭又在圓丘方澤之外魏景初元年始營洛陽南委粟山為圓丘以冬至祭皇皇帝天於圓丘夏至祭皇皇后地於方丘而天郊所祭曰皇天之神地郊所祭曰皇地之祗今以二至之祀合於二郊是後圓丘方澤不别立】罷山陽國督軍除其禁制【魏奉漢獻帝為山陽公國於河内山陽縣之濁鹿城】<br />
<br />
  【置督軍以防衛之至晉時帝孫康嗣立人心去漢久矣故罷其衛兵除其禁制】 十二月吳主還都建業 【考異曰吳志陸凱傳或曰寶鼎元年十二月凱與丁奉丁固謀因皓謁廟欲廢皓立孫休子時左將軍留平領兵先驅故密語平平拒而不許誓以不泄是以不果按凱盡忠執義必不為此事况皓殘酷猜忌留平庸人若聞凱謀必不能不泄殆虛語耳今不取】使后父衛將軍錄尚書事滕牧留鎮武昌朝士以牧尊戚頗推令諫爭【爭讀曰諍】滕后之寵由是漸衰更遣牧居蒼梧雖爵位不奪其實遷也在道以憂死何太后常保佑滕后太史又言中宮不可易吳主信巫覡【在女曰巫在男曰覡覡刑狄翻】故得不廢常供養升平宮【皓尊其母何太后宫曰升平宮供居用翻養羊尚翻】不復進見【見賢遍翻】諸姬佩皇后璽紱者甚衆滕后受朝賀表疏而已【璽斯氏翻紱音弗朝直遥翻】吳主使黄門徧行州郡料取將吏家女【行戶孟翻料音聊】其二千石大臣子女歲歲言名年十五六一簡閲簡閲不中乃得出嫁【中竹仲翻】後宫以千數而採擇無已<br />
<br />
  三年春正月丁卯立子衷為皇太子【為惠帝亡晉張本】詔以近世每立太子必有赦【漢高帝為漢王立太子赦有罪文景武立太子賜民爵至宣帝立太子始大赦天下元帝立太子復賜民爵光武立太子彊赦天下其後立太子陽及明章立太子皆不赦魏文明率病篤然後立太子尋而踐阼有赦故革之】今世運將平當示之以好惡【好呼到翻惡烏路翻】使百姓絶多幸之望曲惠小人朕無取焉遂不赦 司隸校尉上黨李憙【憙許記翻又讀曰憙】劾故立進令劉友前尚書山濤中山王睦尚書僕射武陔各占官稻田【劾戶槩翻又戶得翻陔柯開翻占之贍翻】請免濤睦等官陔已亡請貶其謚詔曰友侵剥百姓以繆惑朝士其考竟以懲邪佞濤等不貳其過皆勿有所問憙亢志在公當官而行【憙與喜同又音憙亢與抗同口浪翻】可謂邦之司直矣【詩鄭國風羔裘之辭】光武有云貴戚且歛手以避二鮑【事見四十二卷建武十一年】其申敕羣僚各慎所司寛宥之恩不可數遇也【數所角翻】睦宣帝之弟子也<br />
<br />
  臣光曰政之大本在於刑賞刑賞不明政何以成晉武帝赦山濤而褒李憙其於刑賞兩失之使憙所言為是則濤不可赦所言為非則憙不足褒褒之使言言而不用怨結於下威玩於上將安用之且四臣同罪劉友伏誅而濤等不問避貴施賤可謂政乎創業之初而政本不立將以垂統後世不亦難乎<br />
<br />
  帝以李憙為太子太傅徵犍為李密為太子洗馬【犍居言翻洗馬自漢以來有之晉職官志太子洗馬職為謁者祕書掌圖書釋奠講經則掌其事出則直者前驅導威儀洗漢書作先如淳曰先前驅也國語越王句踐親為夫差先馬先一作洗音悉薦翻】密以祖母老固辭許之【密所以辭者以旁無兼侍祖母與孫相依為命故也】密與人交每公議其得失而切責之常言吾獨立於世顧影無儔然而不懼者以無彼此於人故也 吳大赦以右丞相萬彧鎮巴丘 夏六月吳主作昭明宫【晉太康地記曰昭明宮方五百丈吳歷曰昭明宮在太初宮之東】二千石以下皆自入山督伐木大開苑囿起土山樓觀窮極伎巧【觀古玩翻伎渠綺翻】功役之費以億萬計陸凱諫不聽中書丞華覈上疏曰【華戶化翻覈戶革翻上時掌翻】漢文之世九州晏然賈誼獨以為如抱火厝於積薪之下而寢其上【事見十四卷漢文帝六年】今大敵據九州之地有大半之衆欲與國家為相吞之計非徒漢之淮南濟北而已也【濟子禮翻】比於賈誼之世孰為緩急今倉庫空匱編戶失業而北方積穀養民專心東向【自洛進師而造江濱自蜀下兵而臨荆楚皆東向也】又交趾淪没嶺表動揺【事見上卷魏元帝咸熙元年】胷背有嫌首尾多難乃國朝之厄會也若舍此急務盡力功作卒有風塵不虞之變【難乃旦翻舍讀曰捨卒讀曰猝】當委版築而應烽燧驅怨民而赴白刃此乃大敵所因以為資者也時吳俗奢侈覈又上疏曰今事多而役繁民貧而俗奢百工作無用之器婦人為綺靡之飾轉相倣效耻獨無有兵民之家猶復逐俗【言下至兵民之家亦隨俗好而事奢侈也復扶又翻】内無甔石之儲【應劭曰齊人名小甕曰甔受二斛晉灼曰石斗石也師古曰甔音都濫翻】而出有綾綺之服上無尊卑等級之差下有耗財費力之損求其富給庸可得乎吳主皆不聽 秋七月王祥以睢陵公罷【睢音雖】 九月甲申詔增吏俸【俸扶用翻】 以何曾為太保義陽王望為太尉荀顗為司徒【顗魚豈翻】 禁星氣䜟緯之學【星為星者氣望氣者東漢以來有䜟緯之學】 吳主以孟仁守丞相奉法駕東迎其父文帝神於明陵【明陵在吳興烏程縣沈約曰孫皓改葬其父於烏程西山曰明陵】中使相繼奉問起居巫覡言見文帝被服顔色如平生【覡刑狄翻被皮義翻】吳主悲喜迎拜於東門之外【建業城東門也】既入廟比七日三祭設諸倡伎晝夜娛樂【比毗寐翻倡音昌樂音洛】 是歲遣鮮卑拓跋沙漠汗歸其國【沙漠汗入質見七十七卷魏元帝景元二年汗音寒】<br />
<br />
  四年春正月丙戌賈充等上所刋修律令【充等所刋修就漢律九章增十一篇合二十篇六百二十條其不入律者悉以為令施行凡律令合二千九百二十六條上時掌翻】帝親自臨講使尚書郎裴楷執讀 【考異曰刑法志云泰始三年事畢表上今從武紀裴楷傳云文帝時詔楷於御前執讀今從刑法志】楷秀之從弟也【從才用翻】侍中盧珽【珽它鼎翻】中書侍郎范陽張華請抄新律死罪條目【抄楚交翻謄寫也】懸之亭傳以示民從之【傳株戀翻】又詔河南尹杜預為黜陟之課預奏古者黜陟擬議於心不泥於法【泥乃計翻】末世不能紀遠而專求密微疑心而信耳目疑耳目而信簡書簡書愈繁官方愈偽【方術也言為官之方術也】魏氏考課即京房之遺意【劉劭考課法其畧見七十三卷魏明帝景初元年】其文可謂至密然失於苛細以違本體故歷代不能通也豈若申唐堯之舊制取大捨小去密就簡俾之易從也【易以豉翻下難易】夫曲盡物理神而明之存乎其人去人而任法則以文傷理莫若委任逹官各考所統【逹官顯官也居一官之長其事得專逹於上】歲第其人言其優劣如此六載【載子亥翻年也】主者總集採案其言六優者超擢六劣者廢免【六優謂六載俱優六劣謂六載俱劣】優多劣少者平叙劣多優少者左遷其間所對不鈞品有難易主者固當凖量輕重微加降殺【量音良殺所戒翻】不足曲以法盡也其有優劣徇情不叶公論者當委監司隨而彈之【監古銜翻監司御史司隸及諸州刺史也彈唐干翻劾也抨也】若令上下公相容過此為清議大頹雖有考課之法亦無益也事竟不行 丁亥帝耕籍田於洛水之北 戊子大赦 二月吳主以左御史大夫丁固為司徒右御史大夫孟仁為司空【吳錄曰孟仁本名宗避皓字易焉】 三月戊子皇太后王氏殂帝居喪之制一遵古禮 夏四月戊戍睢陵元公王祥卒門無雜弔之賓其族孫戎歎曰太保當正始之世不在能言之流及閒與之言理致清遠豈非以德掩其言乎【正始所謂能言者何平叔數人也魏轉而為晉何益於世哉王祥所以可尚者孝於後母與不拜晉王耳君子猶謂其任人柱石而傾人棟梁也理致清遠言乎德乎清談之禍迄乎永嘉流及江左猶未巳也】 己亥葬文明皇后有司又奏既虞除衰服【葬日虞遇柔日再虞而三虞用剛日三虞必反而行之鄭氏曰虞安神之祭也骨肉歸于土䰟氣則無所不之孝子為其彷徨故三祭以安之】詔曰受終身之愛而無數年之報情所不忍也有司固請詔曰患在不能篤孝勿以毁傷為憂前代禮典質文不同何必限以近制使逹喪闕然乎【逹喪猶通喪也】羣臣請不已乃許之然猶素冠疏食以終三年如文帝之喪 秋七月衆星西流如雨而隕 己卯帝謁崇陽陵 九月青徐兖豫四州大水【青州統齊國濟南樂安城陽東萊徐州統彭城下邳東海琅邪廣陵臨淮兖州統陳留濮陽濟隂高平任城東平濟北泰山豫州統潁川汝南襄城汝隂梁國沛譙魯弋陽安豐晉志曰青州取土居少陽其色青為名徐州取舒緩之義兖端也信也又云取兖水以名州豫者舒也言禀中和之氣性理安舒也】大司馬石苞久在淮南威惠甚著【魏高貴鄉公甘露三年平諸葛誕苞代鎮淮南至是凡十一年】淮北監軍王琛惡之【監古䘖翻惡烏路翻】密表苞與吳人交通會吳人將入宼苞築壘遏水以自固帝疑之羊祜深為帝言苞必不然【為于偽翻】帝不信乃下詔以苞不料賊埶築壘遏水勞擾百姓策免其官 【考異曰晉書武紀及苞傳皆無苞免官年月蕭方等三十國春秋杜延業晉春秋置在此今從之苞傳又云敕琅邪王伷自下邳會夀春按武紀伷明年二月乃鎮下邳恐傳誤蕭方等梁元帝子也】遣義陽王望帥大軍以徵之【帥讀曰率】苞辟河内孫鑠為掾【掾俞絹翻】鑠先與汝隂王駿善駿時鎮許昌鑠過見之駿知臺已遣軍襲苞私告之曰無與於禍【與讀曰預】鑠既出馳詣夀春勸苞放兵步出都亭待罪【夀春都亭也】苞從之帝聞之意解苞詣闕以樂陵公還第 吳主出東關冬十月使其將施績入江夏萬彧寇襄陽【夏戶雅翻彧於六翻 考異曰晉帝紀作郁今從吳志】詔義陽王望統中軍步騎二萬屯龍陂【龍陂即摩陂更名見七十二卷魏明帝青龍元年】為二方聲援會荆州刺史胡烈拒績破之望引兵還 吳交州刺史劉俊大都督修則【姓譜元冥之佐有修氏漢有屯騎校尉修炳】將軍顧容前後三攻交趾交趾太守楊稷皆拒破之鬰林九真皆附於稷稷遣將軍毛炅董元攻合浦戰于古城【古城蓋合浦郡古城也炅古迥翻又古惠翻】大破吳兵殺劉俊修則餘兵散還合浦稷表炅為鬰林太守元為九真太守 十一月吳丁奉諸葛靚出芍陂攻合肥【靚疾正翻芍音鵲】安東將軍汝隂王駿拒却之 以義陽王望為大司馬荀顗為太尉【顗魚豈翻】石苞為司徒<br />
<br />
  五年春正月吳王立子瑾為皇太子 二月分雍凉梁州置秦州【晉志曰雍州以其四山之地故以雍名焉亦謂西北之位陽所不及隂陽氣雍閼也統京兆馮翊扶風安定北地新平始平凉州以其地處西方當寒凉也統金城西平武威張掖西郡燉煌酒泉西海梁州以西方金剛之氣彊梁也統漢中梓潼廣漢新都涪陵巴西巴東秦州統隴西南安天水略陽武都隂平等郡】以胡烈為刺史先是鄧艾納鮮卑降者數萬【先悉薦翻降戶江翻下同】置於雍凉之間與民雜居朝廷恐其久而為患以烈素著名於西方故使鎮撫之【此河西鮮卑也】 青徐兖三州大水 帝有滅吳之志壬寅以尚書左僕射羊祜都督荆州諸軍事鎮襄陽征東大將軍衛瓘都督青州諸軍事鎮臨菑鎮東大將軍東莞王伷都督徐州諸軍事鎮下邳祜綏懷遠近甚得江漢之心與吳人開布大信降者欲去皆聽之【降戶江翻】減戍邏之卒【邏郎佐翻】以墾田八百餘頃其始至也軍無百日之糧及其季年乃有十年之積祜在軍常輕裘緩帶身不披甲【被皮義翻】鈴閤之下侍衛不過十數人【鈴下卒及閤下威儀也鈴下者有使令則掣鈴以呼之因以為名閣下威儀掌出入贊導及納謁受事】 濟隂太守巴西文立【濟子禮翻守式又翻】上言故蜀之名臣子孫流徙中國者宜量才叙用【量音良】以慰巴蜀之心以傾吳人之望帝從之 【考異曰立傳載此表在遷太子中庶子後按泰始七年立舉郤詵時猶為濟隂太守於今未為庶子也若諸葛京署吏不因立表則京先已署吏立不當更云宜量才叙用也】乙未詔曰諸葛亮在蜀盡其心力其子瞻臨難而死義【事見七十八卷魏元帝景元四年難乃旦翻】其孫京宜隨才署吏又詔曰蜀將傅僉父子死於其主【傅肜死見六十九卷魏文帝黄初三年傅僉死與諸葛瞻同年】天下之善一也豈由彼此以為異哉僉息著募没入奚官【息子也著與募二子之名也少府有奚官令凡男女没入者屬焉魏以來鄴都又有奚官督】宜免為庶人 帝以文立為散騎常侍漢故尚書犍為程瓊雅有德業【犍居言翻】與立深交帝聞其名以問立對曰臣至知其人但年垂八十禀性謙退無復當時之望【言其意望不求聞逹於當時也】故不以上聞耳瓊聞之曰廣休可謂不黨矣【文立字廣休論語曰君子不黨】此吾所以善夫人也 秋九月有星孛于紫宫【孛蒲内翻】 冬十月吳大赦改元建衡 封皇子景度為城陽王 初汝南何定嘗為吳大帝給使及吳主即位自表先帝舊人求還内侍吳主以為樓下都尉典知酤糴事遂專為威福吳主信任之委以衆事左丞相陸凱面責定曰卿見前後事主不忠傾亂國政寧有得以夀終者邪何以專為姦邪塵穢天聽宜自改厲不然方見卿有不測之禍定大恨之凱竭心公家忠懇内發表疏皆指事不飾【皆指實事不為文飾也】及疾病吳主遣中書令董朝問所欲言凱陳何定不可信用宜授以外任奚熙小吏建起浦里塘亦不可聽【吳主休之時嚴密嘗建此議熙盖祖其說】姚信樓玄賀卲張悌郭逴【逴敕角翻又勅畧翻】薛瑩滕修及族弟喜抗或清白忠勤或資才卓茂皆社稷之良輔願陛下重留神思【思相吏翻】訪以時務使各盡其忠拾遺萬一邵齊之孫【賀齊為吳主權將】瑩綜之子玄沛人修南陽人也凱尋卒吳主素銜其切直【有所恨怒蓄而不發者為銜】且日聞何定之譖久之竟徙凱家於建安 吳主遣監軍虞汜【汜音祀】威南將軍薛珝【珝况羽翻】蒼梧太守丹陽陶璜從荆州道監軍李朂督軍徐存從建安海道【從荆州道踰嶺而入交廣也從建安海道汎海而南也沈約曰建安本閩越秦立為閩中郡漢虛其地後立為冶縣屬會稽郡後分冶地為會稽東南二部都尉東部臨海是也南部建安是也吳主休永安三年分南部立為建安郡宋白曰孫策於建安十二年分東候官之地立建安縣即以年號為名】皆會於合浦以擊交趾 十二月有司奏東宫施敬二傅其儀不同【晉制太子太傅中二千石少傅二千石太子先拜諸傅然後答之時末置詹事宮事大小皆由二傳】帝曰夫崇敬師傅所以尊道重教也何言臣不臣乎【臣不臣盖有司所奏之言】其令太子申拜禮<br />
<br />
  六年春正月吳丁奉入渦口【水經渦水首受河南陽武縣蒗蕩渠東南至下邳淮陵縣入淮謂之渦口渦音戈 考異曰吳志丁奉傳建衡元年攻晉穀陽晉帝紀不載奉傳不言入渦口疑是一事】揚州刺史牽弘擊走之 吳萬彧自巴丘還建業夏四月吳左大司馬施績卒以鎮軍大將軍陸抗都督信陵西陵夷道樂鄉公安諸軍事治樂鄉【水經注樂鄉城在南平郡之孱陵縣江水逕其北江水又東逕公安縣北宋白曰樂鄉者春秋鄀國之地其城陸抗所築在松滋縣界晉地理志信陵縣屬建平郡沈約曰疑是吳立水經注曰江水自夔城而東逕信陵縣南又東過夷陵縣南夷陵即西陵也樂鄉城在今江陵府松滋縣東樂鄉城北江中有砂磧對岸淺可渡江津要害之地也】抗以吳主政事多闕上疏曰臣聞德均則衆者勝寡力侔則安者制危此六國所以并於秦西楚所以屈於漢也今敵之所據非特關右之地鴻溝以西而國家外無連衡之援内非西楚之彊庶政陵遲黎民未乂議者所恃徒以長江峻山限帶封域此乃守國之末事非智者之所先也臣每念及此中夜撫枕臨餐忘食夫事君之義犯而勿欺謹陳時宜十七條以聞【抗傳云十七條失本不載】吳主不納李朂以建安道不利殺導將馮斐引軍還【將即亮翻】初何定嘗為子求㛰於勗勗不許乃白勗枉殺馮斐擅徹軍還誅勗及徐存并其家屬仍焚勗尸定又使諸將各上御犬【上時掌翻】一犬至直縑數十匹纓紲直錢一萬【紲私列翻係也】以捕兎供㕑吳人皆歸罪於定而吳主以為忠勤賜爵列侯陸抗上疏曰小人不明理道所見既淺雖使竭情盡節猶不足任况其姦心素篤而憎愛移易哉吳主不從六月戊午胡烈討鮮卑秃髪樹機能於萬斛堆【樹機能祖】<br />
<br />
  【夀闐之在孕也其母相掖氏因寢而產於被中鮮卑謂被為秃髪因而氏焉至南凉秃髪烏孤則樹機能之五世孫也萬斛堆在温圍水東北安定郡高平縣界】兵敗被殺都督雍凉州諸軍事扶風王亮遣將軍劉旂救之旂觀望不進亮坐貶為平西將軍旂當斬亮上言節度之咎由亮而出乞丐其死【丐居太翻貸其死命也】詔曰若罪不在旂當有所在乃免亮官遣尚書樂陵石鑑行安西將軍都督秦州諸軍事【樂陵縣漢屬平原郡後分屬樂陵國】討樹機能樹機能兵盛鑑使秦州刺史杜預出兵擊之預以虜乘勝馬肥而官軍縣乏【縣讀曰懸】宜并力大運芻糧須春進討鑑奏預稽乏軍興檻車徵詣廷尉以贖論【時預以尚主在八議以侯贖論】既而鑑討樹機能卒不能克【卒子恤翻】 秋七月乙巳城陽王景度卒 丁未以汝隂王駿為鎮西大將軍都督雍凉等州諸軍事鎮關中 冬十一月立皇子東為汝南王 吳主從弟前將軍秀為夏口督吳主惡之民間皆言秀當見圖【秀吳主權弟匡之孫從才用翻惡烏路翻】會吳主遣何定將兵五千人獵夏口秀驚夜將妻子親兵數百人來奔十二月拜秀票騎將軍開府儀同三司封會稽公【厚其封賞以攜吳人票匹妙翻會工外翻】 是歲吳大赦初魏人居南匈奴五部於并州諸郡與中國民雜居【南匈奴自東漢以來分居并州諸郡魏但分其衆為五部耳事見六十七卷漢獻帝建安二十一年時左部所統可萬餘落居太原故兹氏縣右部可六千餘落居祁縣南部可三千餘落居浦子縣北部可四千餘落居新興縣中部可六千餘落居大陵縣】自謂其先漢氏外孫因改姓劉氏【初漢高帝以女妻單于故自謂漢氏外孫冒姓劉氏】<br />
<br />
  七年春正月匈奴右賢王劉猛叛出塞 豫州刺史石鑑坐擊吳軍虛張首級詔曰鑑備大臣吾所取信而乃下同為詐義得爾乎【爾猶言如此也】今遣歸田里終身不得復用【復扶又翻】 吳人刁玄詐增讖文曰黄旗紫蓋見於東南終有天下者荆揚之君【姓譜刁姓齊大夫豎刁之後予按豎刁安得有後漢書貨殖傳有刁閒江表傳曰玄使蜀得司馬徽論運命歷數事因詐增其文以誑吳人見賢遍翻】吳主信之是月晦大舉兵出華里【華里在建業西】載太后皇后及後宫數千人從牛渚西上【水經注牛渚在姑孰烏江兩縣界中今太平州當塗縣北三十里有牛渚山山下有牛渚磯與和州横江渡相對杜佑曰牛渚圻即今當塗縣采石】東觀令華覈等固諫不聽【東觀令典校圖書及記述觀古玩翻華戶化翻覈戶革翻】行遇大雪道途䧟壞兵士被甲持仗【被皮義翻】百人共引一車寒凍殆死皆曰若遇敵便當倒戈【紂發兵與周武王會戰于牧野前徒倒戈攻其後以北】吳主聞之乃還【還從宣翻又如字】帝遣義陽王望統中軍二萬騎三千屯夀春以備之聞吳師退乃罷 三月丙戍鉅鹿元公裴秀卒 夏四月吳交州刺史陶璜襲九真太守董元殺之楊稷以其將王素代之 【考異曰璜傳云出其不意徑至交趾按元乃九真太守非交趾也華陽國志云元病亡楊稷更以王素代之按武帝紀四月九真太守董元為吳將虞汜所攻軍敗死之則元非病亡蓋稷雖以素代元未至郡而元死也】 北地胡宼金城凉州刺史牽弘討之衆胡皆内叛與樹機能共圍弘於青山【續漢志青山在北地郡參䜌縣界賢曰青山在今慶州有青山水】弘軍敗而死 【考異曰崔鴻十六國春秋秃髪烏孤傳云其先樹機能本河西鮮卑泰始中殺秦州刺史胡烈斬凉州刺史牽弘晉帝紀叛虜殺胡烈北地胡殺牽弘皆不言鮮卑蓋言羣虜内叛則鮮卑亦在其中矣或北地胡即樹機能也】初大司馬陳騫言於帝曰胡烈牽弘皆勇而無謀彊於自用非綏邉之材也將為國耻時弘為揚州刺史多不承順騫命【時騫以大司馬都督揚州諸軍鎮夀春】帝以為騫與弘不協而毁之于是徵弘既至尋復以為凉州刺史騫竊歎息以為必敗二人果失羌戎之和兵敗身没征討連年僅而能定帝乃悔之 五月立皇子憲為城陽王 辛丑義陽成王望卒 侍中尚書令車騎將軍賈充自文帝時寵任用事帝之為太子充頗有力【事見七十七卷七十八卷魏紀】故益有寵於帝充為人巧謟與太尉行太子太傅荀顗【晉志曰帝以儲副體尊命諸公居二傅職以本位尊故或行或領顗魚豈翻】侍中中書監荀勗越騎校尉安平馮統【安平縣前漢屬涿郡後漢屬安平國晉屬博陵郡紞都感翻】相為黨友朝野惡之【惡烏路翻】帝問侍中裴楷以方今得失對曰陛下受命四海承風所以未比德於堯舜者但以賈充之徒尚在朝耳【朝直遥翻】宜引天下賢人與弘政道不宜示人以私侍中樂安任愷河南尹潁川庾純皆與充不協充欲解其近職【近職謂侍中任音壬】乃薦愷忠貞宜在東宫帝以愷為太子少傅而侍中如故【晉志曰侍中任愷帝所親敬使領少傅盖一時之制也觀此則充欲以計踈愷】會樹機能寇亂秦雍【雍於用翻】帝以為憂愷曰宜得威望重臣有智畧者以鎮撫之帝曰誰可者愷因薦充純亦稱之秋七月癸酉以充為都督秦凉二州諸軍事侍中車騎將軍如故 【考異曰三十國春秋晉春秋充出並在八年二月按武帝紀充出在此月盖二春秋以太子納妃在八年二月致此誤也】充患之 吳大都督薛珝【珝况羽翻】與陶璜等兵十萬共攻交趾城中糧盡援絶為吳所䧟虜揚稷毛炅等璜愛炅勇健欲活之炅謀殺璜璜乃殺之修則之子允生剖其腹割其肝曰復能作賊不【不讀曰否】炅猶罵曰恨不殺汝孫皓汝父何死狗也【允父則為炅所殺見上四年 考異曰漢晉春秋曰初霍弋遣楊稷毛炅等戊交趾與之誓曰若賊圍城未百日而降者家屬誅若過百日救兵不至而城没者吾受其罪稷等守未百日糧盡乞降於璜不許而給糧使守諸將並諫璜曰霍弋已死不能救稷等必矣可須其日滿然後受降使彼得無罪而我取有義内訓吾民外懷鄰國不亦可乎稷等期訖糧盡救兵不至乃納之華陽國志則云稷等城破被囚稷歐血死炅罵賊死二者相戾不可得合而晉陶璜傳兼載之按孫皓猜暴恐璜不敢以糧資敵今從華陽國志】王素欲逃歸南中吳人獲之九真日南皆降於吳【降戶江翻】吳大赦以陶璜為交州牧璜討降夷獠【獠魯皓翻】州境皆平 八月丙申城陽王憲卒 分益州南中四郡置寧州【寧州以建寧郡名州統建寧興古雲南永昌四郡】 九月吳司空孟仁卒 冬十月丁丑朔日有食之【考異曰宋書五行志有五月庚寅食無十月丁丑食晉書紀及天文志有十月丁丑食無五月庚寅食今從】<br />
<br />
  【晉書】 十一月劉猛宼并州并州刺史劉欽擊破之【晉志并州不以衛水為號又不以恒為稱而云并者以其在兩谷之間也統太原上黨西河樂平雁門新興按晉志所云以周禮并州鎮曰恒山春秋元命包曰營室流為并州分為衛國也】 賈充將之鎮公卿餞於夕陽亭【賢曰夕陽亭在河南城西】充私問計於荀勗勗曰公為宰相乃為一夫所制不亦鄙乎然是行也辭之實難獨有結婚太子可不辭而自留矣充曰然則孰可寄懷勗曰勗請言之因謂馮紞曰賈公遠出吾等失埶太子婚尚未定何不勸帝納賈公之女乎紞亦然之初帝將納衛瓘女為太子妃充妻郭槐賂楊后左右使后說帝求納其女帝曰衛公女有五可賈公女有五不可衛氏種賢而多子美而長白【五可種賢一也多子二也美三也長四也白五也五不可可以類推說輸芮翻種章勇翻下同】賈氏種妬而少子醜而短黑后固以為請荀顗荀勗馮紞皆稱充女絶美且有才德帝遂從之留充復居舊任【為賈氏亂晉張本】 十二月以光禄大夫鄭袤為司空袤固辭不受【袤音茂】 是歲安樂思公劉禪卒【樂音洛考異曰晉春秋云禪謚惠公今從王隱蜀記】 吳以武昌都督廣陵范慎為太尉右將軍司馬丁奉卒【據丁奉傳以救夀春之功拜左將軍誅孫綝拜大將軍加左右都護共迎吳主皓遷右大司馬左軍師當書右大司馬左軍師】 吳改明年元曰鳳皇【以西苑言鳳皇集改元】<br />
<br />
  八年春正月監軍何楨討劉猛屢破之濳以利誘其左部帥李恪【左部五部之一也帥所類翻】恪殺猛以降【降戶江翻】 二月辛卯皇太子納賈妃妃年十五長於太子二歲【長知兩翻】妬忌多權詐太子嬖而畏之【嬖卑義翻又博計翻】 壬辰安平獻王孚卒年九十三孚性忠慎宣帝執政孚常自退損後逢廢立之際未嘗預謀景文二帝以孚屬尊亦不敢逼【孚於廢立之際柔而能正事見七十六卷正元元年七十七卷景元元年】及帝即位恩禮尤重元會詔孚乘輿上殿帝於阼階迎拜【阼階東階主階也上時掌翻下同】既坐親奉觴上夀如家人禮帝每拜孚跪而止之孚雖見尊寵不以為榮常有憂色臨終遺令曰有魏貞士河内司馬孚字叔逹不伊不周不夷不惠立身行道終始若一當衣以時服歛以素棺【衣於既翻歛力瞻翻】詔賜東園温明祕器【服䖍曰東園温明形如方漆桶開一面漆畫之以鏡置其中以懸尸上大歛并蓋之師古曰東園署名也屬少府其署主作此器祕器梓棺以凶器故祕之】諸所施行皆依漢東平獻王故事【見四十八卷漢章帝建初八年】其家遵孚遺旨所給器物一不施用 帝與右將軍皇甫陶論事【泰始五年罷鎮軍將軍復置左右將軍姓譜左傳宋有皇父充石公族也漢初有皇父鸞自魯徙居茂陵改父為甫】陶與帝爭言散騎常侍鄭徽表請罪之帝曰忠讜之言【讜多曩翻善言也】唯患不聞徽越職妄奏豈朕之意遂免徽官 夏汶山白馬胡侵掠諸種【漢武帝誅冉駹開汶山郡宣帝地節三年合於蜀郡蜀漢劉氏又立汶山郡白馬胡即白馬夷也汶讀與㟭同種章勇翻】益州刺史皇甫晏欲討之【益州統蜀犍為汶山漢嘉江陽朱提越嶲䍧柯晉志曰益之為言阨言所在之地險阨也亦曰疆壤益大故以名焉】典學從事蜀郡何旅等【典學從事典學挍及部諸郡文學掾漢諸州刺史有孝經師主監試經月令師主時節祭祀魏晉合其職為典學從事】諫曰胡夷相殘固其常性未為大患今盛夏出軍水潦將降必有疾疫宜須秋冬圖之晏不聽胡康木子燒香言軍出必敗【康木子燒香胡人之名】晏以為沮衆斬之軍至觀阪【水經注觀阪在都安縣晉書地理志都安縣屬汶山郡沈約曰都安縣蜀立宋白曰永康軍導江縣蜀都安縣地沮在呂翻】牙門張弘等以汶山道險且畏胡衆因夜作亂殺晏軍中驚擾兵曹從事犍為楊倉勒兵力戰而死【自漢以來諸州有軍事則置兵曹從事犍居言翻】弘遂誣晏云率已共反故殺之傳首京師晏主簿蜀郡何攀【州主簿錄閤下事省文書郡主簿所職略同】方居母喪聞之詣洛證晏不反弘等縱兵抄掠【抄楚交翻】廣漢主簿李毅言於太守弘農王濬曰皇甫侯起自諸生何求而反且廣漢與成都密邇而統於梁州者朝廷欲以制益州之衿領【漢廣漢郡治雒泰始二年分新都郡治雒而廣漢郡治廣漢縣與成都相近衿衣系領衣要禬著項頷處也】正防今日之變也今益州有亂乃此郡之憂也張弘小豎衆所不與宜即時赴討不可失也濬欲先上請【上時掌翻下先上同】毅曰殺主之賊為惡尤大當不拘常制何請之有濬乃發兵討弘詔以濬為益州刺史濬擊弘斬之夷三族 【考異曰華陽國志弘殺晏在十年五月武帝紀在今年六月按王濬請伐吳表云臣作船七年日有朽敗濬再為益州刺史方受詔作船咸寧五年下詔伐吳借使濬以其年上表則再為益州亦在泰始九年之前矣今從晉紀為定】封濬關内侯初濬為羊祜參軍【晉制諸位從公為持節都督參軍六人】祜深知之祜兄子暨白濬為人志大奢侈不可專任宜有以裁之祜曰濬有大才將以濟其所欲必可用也更轉為車騎從事中郎【祜為車騎將軍其屬有從事中郎秩比千石】濬在益州明立威信蠻夷多歸附之俄遷大司農時帝與羊祜隂謀伐吳祜以為伐吳宜藉上流之埶密表留濬復為益州刺史使治水軍【治直之翻】㝷加龍驤將軍監益梁諸軍事【龍驤將軍之號始此驤思將翻監工銜翻晉制方面之任資重者為都督諸軍事資望輕者為監軍事 考異曰羊祜傳曰表留濬監益州諸軍事加龍驤將軍按濬傳祜密表留濬重拜益州刺史又曰尋以謠言拜龍驤將軍監梁益諸軍事然則作刺史與監軍自是二事也華陽國志又云咸寧四年濬遷大司農五年拜龍驤監梁益二州按是時羊祜已卒尤不可據】詔濬罷屯田軍大作舟艦【艦戶黯翻】别駕何攀以為屯田兵不過五六百人作船不能猝辦後者未成前者已腐宜召諸郡兵合萬餘人造之歲終可成濬欲先上須報【上時掌翻】攀曰朝廷猝聞召萬兵必不聽不如輒召【輒專也】設當見却功夫已成埶不得止濬從之令攀典造舟艦器仗於是作大艦長百二十步【長直亮翻】受二千餘人以木為城起樓櫓開四出門其上皆得馳馬往來 【考異曰華陽國志云咸寧二年三月濬受詔作船按濬表云作船七年則國志不可據也】時作船木柹蔽江而下【柹芳廢翻說文曰削木札樸也字本作柹詳見辨誤】吳建平太守吳郡吾彦【建平郡漢南郡之巫縣吳主權分置宜都郡吳主休永安三年分宜都立建平郡領信陵興山秭歸沙渠四縣杜佑曰建平今巴東郡吳置建平郡於柹歸姓譜吾本已姓夏昆吾氏之後】取流柹以白吳主曰晉必有攻吳之計宜增建平兵以塞其衝要【塞悉則翻】吳主不從彥乃為鐵鎻斷江路【斷丁管翻為後王濬燒斷銕鎻張本】王濬雖受中制募兵而無虎符廣漢太守敦煌張斆收濬從事列上【敦徒門翻斆胡教翻上時掌翻】帝召斆還責曰何不密啟而便收從事斆曰蜀漢絶遠劉備嘗用之矣輒收臣猶以為輕帝善之 壬辰大赦 秋七月以賈充為司空侍中尚書令領兵如故【充自文帝時統城外諸軍】充與侍中任愷皆為帝所寵任充欲專名埶而忌愷於是朝士各有所附【朝直遥翻】朋黨紛然帝知之召充愷宴于式乾殿而謂之曰朝廷宜壹大臣當和充愷等各拜謝既而充愷以帝已知而不責愈無所憚外相崇重内怨益深充乃薦愷為吏部尚書愷侍覲轉希【既不為侍中則侍覲希矣】充因與荀勗馮紞承間共譖之【間古莧翻】愷由是得罪廢於家 八月吳主徵昭武將軍西陵督步闡闡世在西陵【自吳主權用步隲督西陵隲卒子協繼之關協弟也】猝被徵自以失職且懼有讒九月據城來降遣兄子璣璿詣洛陽為任【璣璿皆協子降戶江翻璿如緣翻】詔以闡為都督西陵諸軍事衛將軍開府儀同三司侍中領交州牧封宜都公 冬十月辛未朔日有食之 敦煌太守尹璩卒【敦徒門翻璩求於翻】凉州刺史楊欣表敦煌令梁澄領太守功曹宋質輒廢澄表議郎令狐豐為太守 【考異曰晉春秋璩作據今從武紀武紀云令狐豐廢澄自領郡事今從晉春秋】楊欣遣兵擊之為質所敗【敗補邁翻】吳陸抗聞步闡叛亟遣將軍左弈吾彥等討之帝<br />
<br />
  遣荆州刺史楊肇迎闡於西陵車騎將軍羊祜帥步軍出江陵巴東監軍徐胤帥水軍擊建平以救闡【帥讀曰率】陸抗敕西陵諸軍築嚴圍自赤谿至于故市【水經注江水出西陵峽東南流逕故城洲洲北附岸洲頭曰郭洲長二里廣一里上有步闡故城方圓稱洲周迴畧滿故城洲上城周里闡父騭所築也又東逕陸抗故城今峽州遠安縣在江北有孤山有陸抗故城有丹山時有赤氣意赤溪當出於丹山故市即步騭故城所居成市而闡别築城故曰故市】内以圍闡外以禦晉兵晝夜催切【切迫也】如敵已至衆甚苦之諸將諫曰今宜及三軍之鋭急攻闡比晉救至必可拔也【比必寐翻】何事於圍以敝士民之力抗曰此城處埶既固【處昌呂翻】糧穀又足且凡備禦之具皆抗所宿規【抗先嘗督西陵】今反攻之不可猝拔北兵至而無備表裏受難【難乃旦翻】何以禦之諸將皆欲攻闡抗欲服衆心聽令一攻果無利圍備始合而羊祜兵五萬至江陵諸將咸以抗不宜上【自樂鄉而西赴西陵為上上時掌翻】抗曰江陵城固兵足無可憂者假令敵得江陵必不能守所損者小若晉據西陵則南山羣夷皆當擾動其患不可量也乃自帥衆赴西陵【南山謂江南諸山羣夷所依阻量音良帥讀曰率】初抗以江陵之北道路平易【易以豉翻】敕江陵督張咸作大堰遏水漸漬平土以絶宼叛【堰於扇翻今江陵有三海八櫃引諸湖及沮漳之水注之瀰漫數百里即作堰之故智也漸將亷翻】羊祜欲因所遏水以船運糧揚聲將破堰以通步軍抗聞之使咸亟破之諸將皆惑屢諫不聽祜至當陽聞堰敗乃改船以車運糧大費功力十一月楊肇至西陵陸抗令公安督孫遵循南岸拒羊祜【防托南岸使祜軍不得渡而已】水軍督留慮拒徐胤【恐胤順流東下故以水軍拒之】抗自大將軍憑圍對肇【憑長圍以對之則彼為客我為主】將軍朱喬營都督俞贊亡請肇【姓譜俞古善醫俞附之後】抗曰贊軍中舊吏知吾虛實吾常慮夷兵素不簡練若敵攻圍必先此處【先悉薦翻】即夜易夷兵皆以精兵守之明日肇果攻故夷兵處抗命擊之矢石雨下肇衆死者相屬【屬之欲翻】十二月肇計屈夜遁抗欲追之而慮步闡畜力伺間【間古莧翻】兵不足分于是但鳴鼔戒衆若將追者肇衆懼悉解甲挺走【許拱翻恐懼聲挺待鼎翻拔也挺走拔身而走也】抗使輕兵躡之肇兵大敗【躡尼輒翻】祜等皆引軍還抗遂拔西陵誅闡及同謀將吏數十人皆夷三族自餘所請赦者數萬口【元非同謀而脅從者請而赦之】東還樂鄉貌無矜色謙冲如常吳主加抗都護【吳官有左右都護今加都護盡護諸將也】羊祜坐貶平南將軍【征鎮安平四平最下車騎位次驃騎自此而下六等至四征祜自車騎貶平南凡降十四號】楊肇免為庶人吳主既克西陵自謂得天助志益張大使術士尚廣筮取天下【姓譜尚姓師尚父之後後漢有高士尚子平】對曰吉庚子歲青蓋當入洛陽【其後吳亡皓入洛歲在庚子】吳主喜不修德政專為兼并之計 賈充與朝士宴飲【朝直遥翻 考異曰三十國春秋在十一月晉春秋在十月己巳恐皆非實故附于冬末】河南尹庾純醉與充爭言充曰父老不歸供養【供居用翻養羊尚翻】卿為無天地純曰高貴鄉公何在【斥其弑君也】充慙怒上表解職純亦上表自劾詔免純官仍下五府正其臧否【當時除賈充之外居公位者有五故下五府下遐稼翻否音鄙】石苞以為純榮官忘親當除名齊王攸等以為純於禮律未有違詔從攸議復以純為國子祭酒【帝初立國子學定置國子祭酒博士各一人助教十五人以敎生徒】 吳主之游華里也【事見上七年】右丞相萬彧與右大司馬丁奉左將軍留平密謀曰若至華里不歸社稷事重不得不自還吳主頗聞之以彧等舊臣隱忍不發是歲吳主因會以毒酒飲彧傳酒人私减之又飲留平【飲於鴆翻】平覺之服它藥以解得不死彧自殺 【考異曰吳志孫皓傳云彧被譴憂死今從江表傳】平憂懣月餘亦死【懣音悶又音滿】徙彧子弟於廬陵初彧請選忠清之士以補近職吳主以大司農樓玄為宫下鎮主殿中事【吳舊事禁中主者自用親近人皓以彧言用玄主殿中事】玄正身帥衆【帥讀曰率】奉法而行應對切直吳主浸不悦中書令領太子太傅賀邵上疏諫曰自頃年以來朝列紛錯【朝直遥翻】真偽相貿【貿音茂】忠良排墜信臣被害【被皮義翻】是以正士摧方【摧方言刓稜角而為圓也】而庸臣苟媚先意承指各希時趣【先悉薦翻】人執反理之評士吐詭道之論【詭違也異也】遂使清流變濁忠臣結舌陛下處九天之上隱百里之室【管子曰堂上遠於百里處昌呂翻】言出風靡令行景從親洽寵媚之臣日聞順意之辭將謂此輩實賢而天下已平也臣聞興國之君樂聞其過荒亂之主樂聞其譽【樂音洛譽音余或音如字】聞其過者過日消而福臻聞其譽者譽日損而禍至陛下嚴刑法以禁直辭黜善士以逆諫口杯酒造次死生不保【造七到翻】仕者以退為幸居者以出為福誠非所以保光洪緒熙隆道化也何定本僕隸小人身無行能【行下孟翻】而陛下愛其佞媚假以威福夫小人求入必進奸利定間者妄興事役發江邉戍兵以驅麋鹿老弱饑凍大小怨歎傳曰國之興也視民如赤子其亡也以民為草芥【左傳曰陳逢滑曰國之興也視民如傷其亡也以民為土芥】今法禁轉苛賦調益繁【調徒釣翻】中官近臣所在興事而長吏畏罪苦民求辦是以人力不堪家戶離散呼嗟之聲感傷和氣今國無一年之儲家無經月之蓄田而後宫之中坐食者萬冇餘人又北敵注目伺國盛衰【伺相吏翻】長江之限不可久恃苟我不能守一葦可杭也【詩云誰謂河廣一葦杭之毛氏曰抗渡也鄭玄曰言一葦加之則可以渡也】願陛下豐基彊本割情從道則成康之治興聖祖之祚隆矣【治直吏翻聖祖謂孫權】吳主深恨之於是左右共誣樓玄賀邵相逢駐共耳語大笑【駐駐車也】謗訕政事俱被詰責【訕山諫翻詰去吉翻】送玄付廣州邵原復職既而復徙玄於交趾竟殺之久之何定姦穢發聞亦伏誅【聞音問】 羊祜歸自江陵務修德信以懷吳人每交兵刻日方戰不為掩襲之計將帥有欲進譎計者輒飲以醇酒使不得言【譎古宂翻飲於鴆翻】祜出軍行吳境刈穀為糧皆計所侵送絹償之每會衆江沔遊獵常止晉地若禽獸先為吳人所傷而為晉兵所得者皆送還之于是吳邉人皆悦服【成伐吳之計者祜也凡其所為皆豢吳也正以陸抗對境無間可乘故為是耳若曰務修德信則吾不知也】祜與陸抗對境使命常通抗遺祜酒祜飲之不疑【使疏吏翻遺于季翻】抗疾求藥於祜祜以成藥與之抗即服之人多諫抗抗曰豈有酖人羊叔子哉【羊祜字叔子】抗告其邊戍曰彼專為德我專為暴是不戰而自服也各保分界而已【分扶問翻】無求細利吳主聞二境交和以詰抗抗曰一邑一鄉不可以無信義况大國乎臣不如此正是彰其德於祜無傷也吳主用諸將之謀數侵盜晉邉【數所角翻】陸抗上疏曰昔有夏多罪而殷湯用師紂作淫虐而周武授鉞【湯數夏之罪曰有夏多罪天命殛之武王數紂之罪曰淫酗肆虐穢德彰聞戎商必克上時掌翻】苟無其時雖復大聖亦宜養威自保不可輕動也今不務力農富國審官任能明黜陟任刑賞訓諸司以德【諸司謂百執事之人有司存者】撫百姓以仁而聽諸將徇名窮兵黷武動費萬計士卒彫瘁【瘁秦醉翻】宼不為衰而我已大病矣今爭帝王之資而昧十百之利此人臣之姦便非國家之良策也昔齊魯三戰魯人再克而亡不旋踵何則大小之埶異也【祖張儀說齊湣王之言而畧變其文】况今師所克獲不補所喪乎吳主不從【喪息浪翻】羊祜不附結中朝權貴【朝直遥翻】荀勗馮紞之徒皆惡之從甥王衍嘗詣祜陳事【紞都感翻惡烏路翻從才用翻下同】辭甚清辯祜不然之衍拂衣去祜顧謂賓客曰王夷甫方當以盛名處大位然敗俗傷化必此人也【史言羊祜知人之鑑為懷帝時王衍誤國亡身張本夷甫衍字也敗補邁翻處昌呂翻】及攻江陵祜以軍法將斬王戎衍戎之從弟也故二人皆憾之言論多毁祜時人為之語曰二王當國羊公無德<br />
<br />
  資治通鑑卷七十九  <br>
   </div> 

<script src="/search/ajaxskft.js"> </script>
 <div class="clear"></div>
<br>
<br>
 <!-- a.d-->

 <!--
<div class="info_share">
</div> 
-->
 <!--info_share--></div>   <!-- end info_content-->
  </div> <!-- end l-->

<div class="r">   <!--r-->



<div class="sidebar"  style="margin-bottom:2px;">

 
<div class="sidebar_title">工具类大全</div>
<div class="sidebar_info">
<strong><a href="http://www.guoxuedashi.com/lsditu/" target="_blank">历史地图</a></strong>  
<a href="http://www.880114.com/" target="_blank">英语宝典</a>  
<a href="http://www.guoxuedashi.com/13jing/" target="_blank">十三经检索</a> 
<br><strong><a href="http://www.guoxuedashi.com/gjtsjc/" target="_blank">古今图书集成</a></strong> 
<a href="http://www.guoxuedashi.com/duilian/" target="_blank">对联大全</a> <strong><a href="http://www.guoxuedashi.com/xiangxingzi/" target="_blank">象形文字典</a></strong> 

<br><a href="http://www.guoxuedashi.com/zixing/yanbian/">字形演变</a>  <strong><a href="http://www.guoxuemi.com/hafo/" target="_blank">哈佛燕京中文善本特藏</a></strong>
<br><strong><a href="http://www.guoxuedashi.com/csfz/" target="_blank">丛书&方志检索器</a></strong> <a href="http://www.guoxuedashi.com/yqjyy/" target="_blank">一切经音义</a>  

<br><strong><a href="http://www.guoxuedashi.com/jiapu/" target="_blank">家谱族谱查询</a></strong>  <strong><a href="http://shufa.guoxuedashi.com/sfzitie/" target="_blank">书法字帖欣赏</a></strong> 
<br>

</div>
</div>


<div class="sidebar" style="margin-bottom:0px;">

<font style="font-size:22px;line-height:32px">QQ交流群9:489193090</font>


<div class="sidebar_title">手机APP 扫描或点击</div>
<div class="sidebar_info">
<table>
<tr>
	<td width=160><a href="http://m.guoxuedashi.com/app/" target="_blank"><img src="/img/gxds-sj.png" width="140"  border="0" alt="国学大师手机版"></a></td>
	<td>
<a href="http://www.guoxuedashi.com/download/" target="_blank">app软件下载专区</a><br>
<a href="http://www.guoxuedashi.com/download/gxds.php" target="_blank">《国学大师》下载</a><br>
<a href="http://www.guoxuedashi.com/download/kxzd.php" target="_blank">《汉字宝典》下载</a><br>
<a href="http://www.guoxuedashi.com/download/scqbd.php" target="_blank">《诗词曲宝典》下载</a><br>
<a href="http://www.guoxuedashi.com/SiKuQuanShu/skqs.php" target="_blank">《四库全书》下载</a><br>
</td>
</tr>
</table>

</div>
</div>


<div class="sidebar2">
<center>


</center>
</div>

<div class="sidebar"  style="margin-bottom:2px;">
<div class="sidebar_title">网站使用教程</div>
<div class="sidebar_info">
<a href="http://www.guoxuedashi.com/help/gjsearch.php" target="_blank">如何在国学大师网下载古籍?</a><br>
<a href="http://www.guoxuedashi.com/zidian/bujian/bjjc.php" target="_blank">如何使用部件查字法快速查字?</a><br>
<a href="http://www.guoxuedashi.com/search/sjc.php" target="_blank">如何在指定的书籍中全文检索?</a><br>
<a href="http://www.guoxuedashi.com/search/skjc.php" target="_blank">如何找到一句话在《四库全书》哪一页?</a><br>
</div>
</div>


<div class="sidebar">
<div class="sidebar_title">热门书籍</div>
<div class="sidebar_info">
<a href="/so.php?sokey=%E8%B5%84%E6%B2%BB%E9%80%9A%E9%89%B4&kt=1">资治通鉴</a> <a href="/24shi/"><strong>二十四史</strong></a>&nbsp; <a href="/a2694/">野史</a>&nbsp; <a href="/SiKuQuanShu/"><strong>四库全书</strong></a>&nbsp;<a href="http://www.guoxuedashi.com/SiKuQuanShu/fanti/">繁体</a>
<br><a href="/so.php?sokey=%E7%BA%A2%E6%A5%BC%E6%A2%A6&kt=1">红楼梦</a> <a href="/a/1858x/">三国演义</a> <a href="/a/1038k/">水浒传</a> <a href="/a/1046t/">西游记</a> <a href="/a/1914o/">封神演义</a>
<br>
<a href="http://www.guoxuedashi.com/so.php?sokeygx=%E4%B8%87%E6%9C%89%E6%96%87%E5%BA%93&submit=&kt=1">万有文库</a> <a href="/a/780t/">古文观止</a> <a href="/a/1024l/">文心雕龙</a> <a href="/a/1704n/">全唐诗</a> <a href="/a/1705h/">全宋词</a>
<br><a href="http://www.guoxuedashi.com/so.php?sokeygx=%E7%99%BE%E8%A1%B2%E6%9C%AC%E4%BA%8C%E5%8D%81%E5%9B%9B%E5%8F%B2&submit=&kt=1"><strong>百衲本二十四史</strong></a>  <a href="http://www.guoxuedashi.com/so.php?sokeygx=%E5%8F%A4%E4%BB%8A%E5%9B%BE%E4%B9%A6%E9%9B%86%E6%88%90&submit=&kt=1"><strong>古今图书集成</strong></a>
<br>

<a href="http://www.guoxuedashi.com/so.php?sokeygx=%E4%B8%9B%E4%B9%A6%E9%9B%86%E6%88%90&submit=&kt=1">丛书集成</a> 
<a href="http://www.guoxuedashi.com/so.php?sokeygx=%E5%9B%9B%E9%83%A8%E4%B8%9B%E5%88%8A&submit=&kt=1"><strong>四部丛刊</strong></a>  
<a href="http://www.guoxuedashi.com/so.php?sokeygx=%E8%AF%B4%E6%96%87%E8%A7%A3%E5%AD%97&submit=&kt=1">說文解字</a> <a href="http://www.guoxuedashi.com/so.php?sokeygx=%E5%85%A8%E4%B8%8A%E5%8F%A4&submit=&kt=1">三国六朝文</a>
<br><a href="http://www.guoxuedashi.com/so.php?sokeytm=%E6%97%A5%E6%9C%AC%E5%86%85%E9%98%81%E6%96%87%E5%BA%93&submit=&kt=1"><strong>日本内阁文库</strong></a> <a href="http://www.guoxuedashi.com/so.php?sokeytm=%E5%9B%BD%E5%9B%BE%E6%96%B9%E5%BF%97%E5%90%88%E9%9B%86&ka=100&submit=">国图方志合集</a> <a href="http://www.guoxuedashi.com/so.php?sokeytm=%E5%90%84%E5%9C%B0%E6%96%B9%E5%BF%97&submit=&kt=1"><strong>各地方志</strong></a>

</div>
</div>


<div class="sidebar2">
<center>

</center>
</div>
<div class="sidebar greenbar">
<div class="sidebar_title green">四库全书</div>
<div class="sidebar_info">

《四库全书》是中国古代最大的丛书,编撰于乾隆年间,由纪昀等360多位高官、学者编撰,3800多人抄写,费时十三年编成。丛书分经、史、子、集四部,故名四库。共有3500多种书,7.9万卷,3.6万册,约8亿字,基本上囊括了古代所有图书,故称“全书”。<a href="http://www.guoxuedashi.com/SiKuQuanShu/">详细>>
</a>

</div> 
</div>

</div>  <!--end r-->

</div>
<!-- 内容区END --> 

<!-- 页脚开始 -->
<div class="shh">

</div>

<div class="w1180" style="margin-top:8px;">
<center><script src="http://www.guoxuedashi.com/img/plus.php?id=3"></script></center>
</div>
<div class="w1180 foot">
<a href="/b/thanks.php">特别致谢</a> | <a href="javascript:window.external.AddFavorite(document.location.href,document.title);">收藏本站</a> | <a href="#">欢迎投稿</a> | <a href="http://www.guoxuedashi.com/forum/">意见建议</a> | <a href="http://www.guoxuemi.com/">国学迷</a> | <a href="http://www.shuowen.net/">说文网</a><script language="javascript" type="text/javascript" src="https://js.users.51.la/17753172.js"></script><br />
  Copyright &copy; 国学大师 古典图书集成 All Rights Reserved.<br>
  
  <span style="font-size:14px">免责声明:本站非营利性站点,以方便网友为主,仅供学习研究。<br>内容由热心网友提供和网上收集,不保留版权。若侵犯了您的权益,来信即刪。scp168@qq.com</span>
  <br />
ICP证:<a href="http://www.beian.miit.gov.cn/" target="_blank">鲁ICP备19060063号</a></div>
<!-- 页脚END --> 
<script src="http://www.guoxuedashi.com/img/plus.php?id=22"></script>
<script src="http://www.guoxuedashi.com/img/tongji.js"></script>

</body>
</html>
