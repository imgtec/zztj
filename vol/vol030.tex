<!DOCTYPE html PUBLIC "-//W3C//DTD XHTML 1.0 Transitional//EN" "http://www.w3.org/TR/xhtml1/DTD/xhtml1-transitional.dtd">
<html xmlns="http://www.w3.org/1999/xhtml">
<head>
<meta http-equiv="Content-Type" content="text/html; charset=utf-8" />
<meta http-equiv="X-UA-Compatible" content="IE=Edge,chrome=1">
<title>資治通鑒_31-資治通鑑卷三十_31-資治通鑑卷三十</title>
<meta name="Keywords" content="資治通鑒_31-資治通鑑卷三十_31-資治通鑑卷三十">
<meta name="Description" content="資治通鑒_31-資治通鑑卷三十_31-資治通鑑卷三十">
<meta http-equiv="Cache-Control" content="no-transform" />
<meta http-equiv="Cache-Control" content="no-siteapp" />
<link href="/img/style.css" rel="stylesheet" type="text/css" />
<script src="/img/m.js?2020"></script> 
</head>
<body>
 <div class="ClassNavi">
<a  href="/24shi/">二十四史</a> | <a href="/SiKuQuanShu/">四库全书</a> | <a href="http://www.guoxuedashi.com/gjtsjc/"><font  color="#FF0000">古今图书集成</font></a> | <a href="/renwu/">历史人物</a> | <a href="/ShuoWenJieZi/"><font  color="#FF0000">说文解字</a></font> | <a href="/chengyu/">成语词典</a> | <a  target="_blank"  href="http://www.guoxuedashi.com/jgwhj/"><font  color="#FF0000">甲骨文合集</font></a> | <a href="/yzjwjc/"><font  color="#FF0000">殷周金文集成</font></a> | <a href="/xiangxingzi/"><font color="#0000FF">象形字典</font></a> | <a href="/13jing/"><font  color="#FF0000">十三经索引</font></a> | <a href="/zixing/"><font  color="#FF0000">字体转换器</font></a> | <a href="/zidian/xz/"><font color="#0000FF">篆书识别</font></a> | <a href="/jinfanyi/">近义反义词</a> | <a href="/duilian/">对联大全</a> | <a href="/jiapu/"><font  color="#0000FF">家谱族谱查询</font></a> | <a href="http://www.guoxuemi.com/hafo/" target="_blank" ><font color="#FF0000">哈佛古籍</font></a> 
</div>

 <!-- 头部导航开始 -->
<div class="w1180 head clearfix">
  <div class="head_logo l"><a title="国学大师官网" href="http://www.guoxuedashi.com" target="_blank"></a></div>
  <div class="head_sr l">
  <div id="head1">
  
  <a href="http://www.guoxuedashi.com/zidian/bujian/" target="_blank" ><img src="http://www.guoxuedashi.com/img/top1.gif" width="88" height="60" border="0" title="部件查字,支持20万汉字"></a>


<a href="http://www.guoxuedashi.com/help/yingpan.php" target="_blank"><img src="http://www.guoxuedashi.com/img/top230.gif" width="600" height="62" border="0" ></a>


  </div>
  <div id="head3"><a href="javascript:" onClick="javascript:window.external.AddFavorite(window.location.href,document.title);">添加收藏</a>
  <br><a href="/help/setie.php">搜索引擎</a>
  <br><a href="/help/zanzhu.php">赞助本站</a></div>
  <div id="head2">
 <a href="http://www.guoxuemi.com/" target="_blank"><img src="http://www.guoxuedashi.com/img/guoxuemi.gif" width="95" height="62" border="0" style="margin-left:2px;" title="国学迷"></a>
  

  </div>
</div>
  <div class="clear"></div>
  <div class="head_nav">
  <p><a href="/">首页</a> | <a href="/ShuKu/">国学书库</a> | <a href="/guji/">影印古籍</a> | <a href="/shici/">诗词宝典</a> | <a   href="/SiKuQuanShu/gxjx.php">精选</a> <b>|</b> <a href="/zidian/">汉语字典</a> | <a href="/hydcd/">汉语词典</a> | <a href="http://www.guoxuedashi.com/zidian/bujian/"><font  color="#CC0066">部件查字</font></a> | <a href="http://www.sfds.cn/"><font  color="#CC0066">书法大师</font></a> | <a href="/jgwhj/">甲骨文</a> <b>|</b> <a href="/b/4/"><font  color="#CC0066">解密</font></a> | <a href="/renwu/">历史人物</a> | <a href="/diangu/">历史典故</a> | <a href="/xingshi/">姓氏</a> | <a href="/minzu/">民族</a> <b>|</b> <a href="/mz/"><font  color="#CC0066">世界名著</font></a> | <a href="/download/">软件下载</a>
</p>
<p><a href="/b/"><font  color="#CC0066">历史</font></a> | <a href="http://skqs.guoxuedashi.com/" target="_blank">四库全书</a> |  <a href="http://www.guoxuedashi.com/search/" target="_blank"><font  color="#CC0066">全文检索</font></a> | <a href="http://www.guoxuedashi.com/shumu/">古籍书目</a> | <a   href="/24shi/">正史</a> <b>|</b> <a href="/chengyu/">成语词典</a> | <a href="/kangxi/" title="康熙字典">康熙字典</a> | <a href="/ShuoWenJieZi/">说文解字</a> | <a href="/zixing/yanbian/">字形演变</a> | <a href="/yzjwjc/">金 文</a> <b>|</b>  <a href="/shijian/nian-hao/">年号</a> | <a href="/diming/">历史地名</a> | <a href="/shijian/">历史事件</a> | <a href="/guanzhi/">官职</a> | <a href="/lishi/">知识</a> <b>|</b> <a href="/zhongyi/">中医中药</a> | <a href="http://www.guoxuedashi.com/forum/">留言反馈</a>
</p>
  </div>
</div>
<!-- 头部导航END --> 
<!-- 内容区开始 --> 
<div class="w1180 clearfix">
  <div class="info l">
   
<div class="clearfix" style="background:#f5faff;">
<script src='http://www.guoxuedashi.com/img/headersou.js'></script>

</div>
  <div class="info_tree"><a href="http://www.guoxuedashi.com">首页</a> > <a href="/SiKuQuanShu/fanti/">四库全书</a>
 > <h1>资治通鉴</h1> <!--         下载:【右键另存为】即可 --></div>
  <div class="info_content zj clearfix">
  
<div class="info_txt clearfix" id="show">
<center style="font-size:24px;">31-資治通鑑卷三十</center>
    資治通鑑卷三十    宋 司馬光 撰<br />
<br />
  胡三省 音註<br />
<br />
  漢紀二十二【起屠維赤奮若盡著雍閹茂凡十年】<br />
<br />
  孝成皇帝上之上【荀悦曰諱驁字太孫驁之字曰俊應劭曰諡法安民立政曰成】<br />
<br />
  建始元年春正月乙丑悼考廟災【宣帝尊史皇孫曰悼廟】 石顯遷長信中太僕【百官表長信中太僕掌皇太后輿馬不常置】秩中二千石顯既失倚離權【顯嬖於元帝帝崩為失倚自中書令樞機之官遷太后宮官為離權離力智翻】於是丞相御史條奏顯舊惡及其黨牢梁陳順皆免官顯與妻子徙歸故郡憂懣不食道死【顯故濟南人師古曰懣音悶】諸所交結以顯為官者皆廢罷少府五鹿充宗左遷玄菟太守【菟音塗守式又翻】御史中丞伊嘉為鴈門都尉【姓譜伊姓出於伊尹】司隸校尉涿郡王尊劾奏丞相衡御史大夫譚【劾戶槩翻又戶得翻】知顯等顓權擅執大作威福為海内患害不以時白奏行罰而阿諛曲從附下罔上懷邪迷國無大臣輔政之義皆不道在赦令前【去年七月大赦】赦後衡譚舉奏顯不自陳不忠之罪而反揚著先帝任用傾覆之徒妄言百官畏之甚於主上卑君尊臣非所宜稱失大臣體於是衡慙懼免冠謝罪上丞相侯印綬【衡封樂安侯】天子以新即位重傷大臣乃左遷尊為高陵令然羣下多是尊者衡嘿嘿不自安每有水旱連乞骸骨讓位上輒以詔書慰撫不許 立故河間王元弟上郡庫令良為河間王【元廢事見上卷元帝建昭元年如淳曰漢北邊郡庫官兵器之所藏故置令】 有星孛于營室【晉書天文志營室二星天子之宮也一曰玄宮一曰清廟又為軍糧之府及士功事孛蒲内翻】 赦天下壬子封舅諸吏光禄大夫關内侯王崇為安成侯【恩澤侯表安成侯食邑於汝南】賜舅譚商立根逢時爵關内侯夏四月黃霧四塞【元命包曰隂陽亂為霧爾雅曰地氣發天不應曰霧釋名曰霧冒也氣蒙冒地之物也師古曰塞滿也言四方皆滿塞悉則翻】詔博問公卿大夫無有所諱諫大夫楊興博士駟勝等皆以為隂盛侵陽之氣也高祖之約非功臣不侯今太后諸弟皆以無功為侯外戚未曾有也故天為見異【師古曰見顯示也為于偽翻見賢遍翻】於是大將軍鳳懼上書乞骸骨辭職上優詔不許 御史中丞東海薛宣上疏曰陛下至德仁厚而嘉氣尚凝【師古曰凝謂不通也】隂陽不和殆吏多苛政部刺史或不循守條職【師古曰刺史所察本有六條今則踰越故事信意舉劾妄為苛刻也漢官典職儀云刺史班宣周行郡國省察治狀黜陟能否斷治寃獄以六條問事非條所問即不省一條強宗豪右田宅踰制以彊陵弱以衆暴寡二條二千石不奉詔書遵承典制倍公向私旁詔牟利侵漁百姓聚歛為姦三條二千石不恤疑獄風厲殺人怒則任刑喜則淫賞煩擾刻暴剥截黎元為百姓所疾山崩石裂妖祥訛言四條二千石選署不平苟阿所愛蔽賢寵頑五條二千石子弟恃怙榮埶請托所監六條二千石違公下比阿附豪彊通行貨賂割損正令】舉錯各以其意多與郡縣事【師古曰錯置也音千故翻與讀曰豫豫干也】至開私門聽讒佞以求吏民過譴呵及細微責義不量力【師古曰言求備於人量音良】郡縣相迫促亦内相刻流及衆庶是故鄉黨闕於嘉賓之歡九族忘其親親之恩飲食周急之厚彌衰送往勞來之禮不行【師古曰勞郎到翻來郎代翻余謂來讀如字亦通】夫人道不通則隂陽否隔【否皮鄙翻】和氣不通未必不由此也詩云民之失德乾餱以愆【小雅伐木之詩也毛氏曰餱食也鄭氏曰失德謂見謗訕也民尚以乾餱之食獲愆過於人况上之人乎乾音干餱音侯】鄙語曰苛政不親煩苦傷恩方刺史奏事時宜明申敕【師古曰申束也謂約束也】使昭然知本朝之要務上嘉納之八月有兩月相承晨見東方【服䖍曰相承在上下也應劭曰案京房易傳云】<br />
<br />
  【君弱如婦為隂所乘則兩月出見賢遍翻】 冬十二月作長安南北郊罷甘泉汾陰祠【匡衡奏祭天於南郊就陽之義也瘞地於北郊即陰之象也甘泉郊見皇天反北之太陰汾陰祠后土反東之少陽甘泉河東之祠宜徙就正陽太隂之處於長安定南北郊上從之】及紫壇偽飾女樂鸞路騂駒龍馬石壇之屬【衡又言甘泉泰畤紫壇有文章采鏤黼黻之飾及玉女樂石壇仙人祠瘞鸞路騂駒寓龍非古於是悉罷之師古曰漢舊儀云祭天用六綵綺席六重用玉几玉飾器凡七十女樂即禮樂志所云使童男童女俱歌也】二年春正月罷雍五畤及陳寶祠【秦作畤於雍以祠上帝有白青黄赤帝之祠至漢高帝立北畤祠黑帝而五畤具有司進祠上不親往至文帝時始幸雍郊見五畤陳寶者秦文公獲若石於陳倉北阪上祠之其神來常以夜光輝若流星從東方來集於祠城若雄雉其聲殷殷云野雞皆鳴以應之祠以一牢名曰陳寶衡以為不應禮皆奏罷之雍於用翻畤音止】皆從匡衡之請也辛巳上始郊祀長安南郊赦奉郊縣【應劭曰天郊在長安城南地郊在長安城北長陵界中二縣有奉郊之勤故並赦之余按帝紀二縣長安及長陵也】及中都官耐罪徒【師古曰中都官京師諸官府應劭曰輕罪不至於髠完其耏鬢故曰耏古耏字從彡髮膚之意也杜林以為法度之字皆從寸後改如字耐音若能如淳曰耐猶任也任其事也師古曰依應氏之說耏當音而如氏之說則音乃代翻其義亦兩通耏謂頬旁毛也彡毛髪貌也音所廉翻又先亷翻而功臣表宣曲侯通耏為鬼薪則應氏之說斯為長矣】減天下賦錢筭四十【孟康曰本筭百二十今減四十為八十】 閏月以渭城延陵亭部為初陵 三月辛丑上始祠后土于北郊 丙午立皇后許氏后車騎將軍嘉之女也元帝傷母恭哀后居位日淺而遭霍氏之辜【事見二十四卷宣帝本始三年】故選嘉女以配太子 上自為太子時以好色聞【好呼倒翻聞音問】及即位皇太后詔采良家女以備後宮大將軍武庫令杜欽【此大將軍之軍中武庫令也欽傳軍下更有軍字】說王鳳曰禮一娶九女所以廣嗣重祖也【張晏曰陽數一三五七九九數之極也臣瓚曰天子一娶九女夏殷之制也欽故舉古之約以刺今之奢也說輸芮翻】娣姪雖缺不復補所以養夀塞争也【師古曰媵女之内兄弟之女則謂之姪己之女弟則謂之娣塞絶也復扶又翻塞悉則翻】故后妃有貞淑之行【行下孟翻下同】則胤嗣有賢聖之君制度有威儀之節則人君有夀考之福廢而不由則女德不厭【師古曰由用也從也女德不厭言好色之甚也】女德不厭則夀命不究於高年【師古曰究竟也】男子五十好色未衰婦人四十容貌改前以改前之容侍於未衰之年而不以禮為制則其原不可救而後來異態後來異態則正后自疑而支庶有間適之心【師古曰間代也音居莧翻適讀曰嫡下亦同】是以晉獻被納讒之謗申生蒙無罪之辜【晉獻公嬖驪姬驪姬欲立其子讒世子申生獻公信之申生雉經而死被皮義翻】今聖主富於春秋未有適嗣方鄉術入學【鄉讀曰嚮】未親后妃之議將軍輔政宜因始初之隆建九女之制詳擇有行義之家【行下孟翻】求淑女之質毋必有聲色技能為萬世大法【師古曰惟求淑質無論美色及音聲技能如此則可為萬世法技渠綺翻】夫少戒之在色【師古曰論語孔子曰君子有三戒少之時血氣未定戒之在色言好色無節則致損敗故戒之也少詩照翻】小卞之作可為寒心【詩小雅也張晏曰小卞刺幽王廢申后而立褒姒黜太子宜臼而立伯服也臣瓚曰小卞之詩太子之傅作也哀太子之放逐愍周室之大壞也卞音盤】唯將軍常以為憂鳳白之太后太后以為故事無有鳳不能自立法度循故事而己鳳素重欽故置之莫府國家政謀常與欽慮之【師古曰慮計也】數稱達名士裨正闕失【數所角翻】當世善政多出於欽者 夏大旱 匈奴呼韓邪單于嬖左伊秩訾兄女二人長女顓渠閼氏【長知兩翻下同閼於乾翻氏音支下同】生二子長曰且莫車【師古曰且音子余翻下且麋胥同車昌遮翻】次曰囊知牙斯少女為大閼氏【少詩照翻下同】生四子長曰雕陶莫臯次曰且麋胥皆長於且莫車少子咸樂二人皆小於囊知牙斯又它閼氏子十餘人顓渠閼氏貴且莫車愛呼韓邪病且死欲立且莫車顓渠閼氏曰匈奴亂十餘年不絶如髮賴蒙漢力故得復安【復扶又翻下同】今平定未久人民創艾戰鬬【師古曰創音初亮翻艾讀曰乂】且莫車年少【少詩照翻下同】百姓未附恐復危國我與大閼氏一家共子【師古曰一家言親姊妹也共子兩人所生恩慈無别也】不如立雕陶莫臯大閼氏曰且莫車雖少大臣共持國事今舍貴立賤【師古曰舍謂棄置也舍讀曰捨】後世必亂單于卒從顓渠閼氏計立雕陶莫臯【卒子恤翻】約令傳國與弟呼韓邪死雕陶莫臯立為復株累若鞮單于【師古曰復音服累力追翻賢曰匈奴謂孝為若鞮自呼韓邪降後與漢親密見漢帝諡常為孝慕之至其子復株累單于以下皆稱若鞮】復株累若鞮單于以且麋胥為左賢王且莫車為左谷蠡王【谷音鹿蠡盧奚翻】囊知牙斯為右賢王復株累單于復妻王昭君生二女長女云為須卜居次小女為當于居次【文頴曰須卜氏匈奴貴族也當于亦匈奴大族也師古曰須卜當于皆其夫家氏族】<br />
<br />
  三年春三月赦天下徒 秋關内大雨四十餘日京師民相驚言大水至百姓犇走相蹂躪【師古曰蹂踐也躪轢也蹂音人九翻躪音藺】老弱呼號【號戶高翻呼火故翻】長安中大亂天子親御前殿召公卿議大將軍鳳以為太后與上及後宮可御船令吏民上長安城以避水【上時掌翻下同】羣臣皆從鳳議左將軍王商獨曰自古無道之國水猶不冒城郭【師古曰冒蒙覆也】今政治和平世無兵革上下相安何因當有大水一日暴至此必訛言也【師古曰訛偽也治直吏翻】不宜令上城重驚百姓【師古曰重音直用翻】上乃止有頃長安中稍定問之果訛言上於是美壮商之固守數稱其議【數所角翻】而鳳大慙自恨失言【為王鳳排斥王商張本】 上欲專委任王鳳八月策免車騎將軍許嘉以特進侯就朝位【漢制列侯奉朝請在長安者位次三公賜位特進者在凡列侯之上位亦次三公朝直遥翻】 張譚坐選舉不實免冬十月光禄大夫尹忠為御史大夫 十二月戊申朔日有食之其夜地震未央宮殿中詔舉賢良方正能直言極諫之士杜欽及太常丞谷永上對【續漢志太常丞比千石掌凡行禮及祭祀小事總署曹事漢舊儀曰丞舉廟中非法者】皆以為後宮女寵太盛嫉妬專上將害繼嗣之咎【此盖指許后及班倢伃也】 越巂山崩 丁丑匡衡坐多取封邑四百頃監臨盜所主守直十金以上免為庶人【衡本封臨淮郡僮縣之樂安鄉鄉本田提封三千一百頃南以閩陌為界後誤封平陵陌為界多四百頃師古曰十金以上當時律定罪之次若今律條言一尺以上一匹以上】<br />
<br />
  四年春正月癸卯隕石于亳四隕于肥累二【漢書五行志亳作槀孟康曰槀肥累皆縣名故屬真定師古曰槀音工老翻累音力追翻】 罷中書宦官初置尚書員五人【臣瓚曰漢初中人有中謁者令孝武加中謁者令為中書謁者令置僕射宣帝時任中書官弘恭為令石顯為僕射元帝即位數年恭死顯代為中書令專權用事至帝乃罷其官師古曰漢舊儀云尚書四人為四曹常侍尚書主丞相御史事二千石尚書主刺史二千石事戶曹尚書主庶人上書事主客尚書主外國事帝置五人有三公曹主斷獄事】 三月甲申以左將軍樂昌侯王商為丞相 夏上悉召前所舉直言之士詣白虎殿對策【師古曰此殿在未央宮】是時上委政王鳳議者多歸咎焉谷永知鳳方見柄用【師古曰柄用言任用之授以權也】隂欲自託乃曰方今四夷賓服皆為臣妾北無葷粥冒頓之患【葷許云翻師古曰粥音弋六翻太史公曰唐虞以上有葷粥孟子曰太王事獯粥冒頓為患見高惠呂后紀】南無趙佗呂嘉之難【趙佗見高惠呂后孝文紀呂嘉見孝武紀難乃旦翻】三垂晏然靡有兵革之警諸侯大者乃食數縣漢吏制其權柄不得有為無吴楚燕梁之埶【吴楚梁見孝景紀燕見孝昭紀】百官盤互親疏相錯【師古曰盤互盤結而交互也錯間雜也】骨肉大臣有申伯之忠【師古曰申伯周申后之父余據詩崧高云不顯申伯王之元舅是則申伯乃宣王之舅永正以之况王鳳也】洞洞屬屬【師古曰洞洞敬肅也屬屬專謹也洞音動屬音之欲翻】小心畏忌無重合安陽博陸之亂【師古曰重合侯莽通安陽侯上官桀博陸侯霍禹也余按莽通即馬通事見二十二卷武帝後元元年安陽侯事見二十三卷昭帝元鳳元年霍禹事見二十五卷宣帝地節四年】三者無毛髮之辜竊恐陛下舍昭昭之白過忽天地之明戒聽晻昧之瞽說【師古曰舍謂留也晻與暗同又音一感翻瞽說言不中道若無目之人也余謂舍置也讀曰舍】歸咎乎無辜倚異乎政事【師古曰倚依也】重失天心【師古曰重音直用翻】不可之大者也【師古曰此則為大不可也】陛下誠深察愚臣之言抗湛溺之意解偏駁之愛【師古曰抗舉也湛讀曰沈駁不周普也湛持林翻】奮乾剛之威平天覆之施使列妾得人人更進【覆敷又翻施式智翻更工衡翻】益納宜子婦人毋擇好醜毋避嘗字【如淳曰王鳳上小妻弟以納後宮以嘗字乳王章言之坐死今永及此為鳳洗前過也仲馮曰按王章言事坐誅在陽朔初而永此對乃是建始四年則非為鳳而言也然觀永前後之文實若為鳳但班固於此對後乃云永為上第擢為光禄大夫則同是建始四年中事余謂此時鳳盖已納張美人於後宮故永為之言若王章指言鳳過則在陽朔初也】毋論年齒推法言之陛下得繼嗣於微賤之間乃反為福得繼嗣而已母非有賤也【師古曰苟得子耳勿論其母之貴賤】後宮女史使令有直意者【鄭玄曰女史女奴曉書者使令給役後宮無爵秩者也師古曰直當也令音力成翻】廣求於微賤之間以遇天所開右【師古曰右讀曰佑佑助也】慰釋皇太后之憂愠【師古曰釋散也愠於問翻】解謝上帝之譴怒則繼嗣蕃滋災異訖息【師古曰蕃多也訖止也蕃音扶元翻】杜欽亦倣此意上皆以其書示後宮擢永為光禄大夫 夏四月雨雪【雨于具翻】秋桃李實 大雨水十餘日河决東郡金隄【師古曰金隄者】<br />
<br />
  【河隄之名今在滑州界】先是清河都尉馮逡奏言郡承河下流【據溝洫志逡言清河郡承河下流與兖州東郡分水為界先悉薦翻逡七倫翻】土壤輕脆易傷【易以豉翻】頃所以濶無大害者以屯氏河通兩川分流也【師古曰濶稀也】今屯氏河塞靈鳴犢口又益不利【屯氏河塞見上卷元帝永光五年塞悉則翻】獨一川兼受數河之任雖高增隄防終不能泄如有霖雨旬日不霽必盈溢【師古曰雨止曰霽音子詣翻又音才詣翻】九河故迹今既滅難明【夏禹疏九河孔安國曰河水分九道在兖州界爾雅曰徒駭一太史二馬頰三覆釡四胡蘇五簡六潔七鉤盤八鬲津九】屯氏河新絶未久其處易浚【師古曰浚謂治導之令其深也浚音峻】又其口所居高於以分殺水力道里便宜【殺音所介翻減也】可復浚以助大河【復扶又翻】泄暴水備非常不豫修治【治直之翻】北决病四五郡南决病十餘郡然後憂之晚矣事下丞相御史白遣博士許商行視【下遐稼翻師古曰白白於天子也行音下更翻】以為方用度不足【師古曰言國家少財役也】可且勿浚後三歲河果决于館陶及東郡金隄泛濫兖豫及平原千乘濟南【乘繩證翻濟子禮翻】凡灌四郡三十二縣水居地十五萬餘頃深者三丈壞敗官亭室廬且四萬所【壞音怪敗補邁翻】冬十一月御史大夫尹忠以對方略疏闊上切責其不憂職自殺遣大司農非調【師古曰大司農名非調也姓譜非姓也秦非子之後】調均錢穀河决所灌之郡【師古曰令其調發均平錢穀遭水之郡使存給也調音徒釣翻】謁者二人發河南以東船五百㮴【師古曰一船為一㮴音先勞翻其字從木】徙民避水居丘陵九萬七千餘口 壬戌以少府張忠為御史大夫 南山羣盗傰宗等數百人為吏民害【蘇林曰傰音朋晉灼曰音倍師古曰晉音是也】詔發兵千人逐捕歲餘不能禽或說大將軍鳳【說輸納翻】以賊數百人在轂下【師古曰在天子輦轂之下明其逼近也】討不能得難以示四夷獨選賢京兆乃可於是鳳薦故高陵令王尊徵為諫大夫守京輔都尉【武帝元鼎四年更置三輔都尉京兆曰京輔都尉馮翊曰左輔都尉扶風曰右輔都尉】行京兆尹事旬月間盜賊清後拜為京兆尹 上即位之初丞相匡衡復奏射聲校尉陳湯【武帝置北軍八校尉射聲其一也秩二千石掌待詔射聲士服䖍曰工射者也冥冥中聞聲則中之因以名也復扶又翻】以吏二千石奉使【湯為西域副校尉秩比二千石】顓命蠻夷中不正身以先下【先悉薦翻】而盗所收康居財物戒官屬曰絶域事不覆校【言外域之事漢朝務存寛大必不考覆也】雖在赦前【言其事在竟寧元年七月赦前也】不宜處位【處昌呂翻】湯坐免後湯上言康居王侍子非王子按驗實王子也湯下獄當死【下遐稼翻】太中大夫谷永【按是年夏谷永方擢為光禄大夫河平二年議受伊邪莫演降永猶為光禄大夫此書太中大夫谷永據陳湯傳也】上疏訟湯曰臣聞楚有子玉得臣文公為之仄席而坐【師古曰子玉楚大夫也得臣其名也春秋僖十八年子玉帥師與晉文公戰於城濮楚師敗績晉師三月館穀而文公猶有憂色曰得臣猶在憂未歇也及楚殺子玉公喜而後可知也禮記口有憂者仄席而坐盖自貶也為于偽翻仄古側翻】趙有亷頗馬服彊秦不敢窺兵井陘【師古曰亷頗趙將也馬服君趙奢亦趙將也井陘之口趙之西界山險道也陘音刑】近漢有郅都魏尚匈奴不敢南鄉沙幕【景帝以郅都為鴈門太守匈奴素聞都節舉邊為引兵去竟都死不敢近鴈門魏尚事見十五卷文帝十四年鄉讀曰嚮】由是言之戰克之將國之爪牙不可不重也盖君子聞鼓鼙之聲則思將帥之臣【師古曰禮之樂記曰鼔鼙之聲讙讙以立動動以進衆君子聽鼔鼙之聲則思將帥之臣鼙駢迷翻將即亮翻帥所類翻】竊見關内侯陳湯前斬郅支威震百蠻武暢西海漢元以來征伐方外之將未嘗有也【漢元謂漢初也】今湯坐言事非是幽囚久繫歷時不决執憲之吏欲致之大辟【辟毗亦翻】昔白起為秦將南拔郢都北阬趙括以纎介之過賜死杜郵秦民憐之莫不隕涕【事見周赧王紀】今湯親秉鉞席卷喋血萬里之外【師古曰如席之卷言其疾也服䖍曰喋音蹀屣履之蹀如淳曰殺人流血滂沱為喋血師古曰喋音大類翻本字當作蹀蹀謂履涉之耳】薦功祖廟告類上帝【張晏曰謂以所征之國事類告天也】介胄之士靡不慕義以言事為罪無赫赫之惡周書曰記人之功忘人之過宜為君者也【師古曰尚書之外間書也】夫犬馬有勞於人尚加帷盖之報【師古曰禮記稱孔子云敝帷弗棄為埋馬也敝盖弗棄為埋狗也】况國之功臣者哉竊恐陛下忽於鼙鼔之聲不察周書之意而忘帷盖之施【施式智翻】庸臣遇湯卒從吏議【師古曰以待庸臣者待湯也卒猶終也卒子恤翻】使百姓介然有秦民之恨【師古曰介然猶耿耿】非所以厲死難之臣也【難乃旦翻】書奏天子出湯奪爵為士伍會西域都護段會宗為烏孫兵所圍驛騎上書願發城郭燉煌兵以自救【城郭謂西域城郭諸國也燉音屯徒門翻】丞相商大將軍鳳及百寮議數日不决鳳言陳湯多籌策習外國事可問上召湯見宣室湯擊郅支時中寒【見賢遍翻中竹仲翻】病兩臂不屈申湯入見有詔毋拜示以會宗奏湯對曰臣以為此必無可憂也上曰何以言之湯曰夫胡兵五而當漢兵一何者兵刃朴鈍弓弩不利今聞頗得漢巧然猶三而當一又兵法曰客倍而主人半然後敵【此言憑城而守者主人之半可以敵客之倍】今圍會宗者人衆不足以勝會宗唯陛下勿憂且兵輕行五十里重行三十里今會宗欲發城郭燉煌歷時乃至所謂報讐之兵非救急之用也上曰奈何其解可必乎度何時解【度徒洛翻】湯知烏孫瓦合不能久攻【師古曰謂如碎瓦之雜居不齊同】故事不過數日【師古曰故事謂以舊事測之】因對曰已解矣屈指計其日曰不出五日當有吉語聞【師古曰吉善也善謂兵解之事】居四日軍書到言已解大將軍鳳奏以為從事中郎莫府事壹决於湯【續漢志大將軍府有從事中郎二人秩六百石職參謀議】<br />
<br />
  河平元年【以河决隄塞輒平改元】春杜欽薦犍為王延世於王鳳使塞决河【犍居言翻塞悉則翻】鳳以延世為河隄使者延世以竹落長四丈大九圍盛以小石【長直亮翻盛時征翻】兩船夾載而下之三十六日河隄成三月詔以延世為光禄大夫秩中二千石賜爵關内侯黄金百斤 夏四月己亥晦日有食之詔公卿百僚陳過失無有所諱大赦天下光禄大夫劉向對曰四月交於五月月同孝惠日同孝昭【孝惠七年五月丁卯先晦一日日食今四月己亥晦而日食故曰四月交於五月月同孝惠孝昭元年七月己亥晦日食故曰日同孝昭二帝尋皆晏駕而無嗣】其占恐害繼嗣是時許皇后專寵後宮希得進見【見賢遍翻】中外皆憂上無繼嗣故杜欽谷永及向所對皆及之上於是減省椒房掖庭用度【師古曰椒房殿皇后所居以椒和泥塗壁取其温且芬也】服御輿駕所發諸官署及所造作遺賜外家羣臣妾【師古曰外家謂后之家族言在外也劉向曰婦人内夫家而外父母家遺于季翻】皆如竟寧以前故事皇后上書自陳以為時世異制長短相補不出漢制而已纎微之間未必可同若竟寧前與黄龍前豈相放哉【晉灼曰竟寧元帝時也黄龍宣帝時也言二帝奢儉不同豈相放哉師古曰放依也音甫往翻】家吏不曉【師古曰家吏皇后之官屬】今壹受詔如此且使妾揺手不得設妾欲作某屏風張於某所曰故事無有或不能得則必繩妾以詔書矣【師古曰言或有所求吏不肯備因云詔書不許也繩約也】此誠不可行唯陛下省察【省悉井翻】故事以特牛祠大父母戴侯敬侯皆得蒙恩以太牢祠【一牲曰特三牲備為一牢平恩戴侯許廣漢后父嘉紹其封於后為祖樂成敬侯許延夀后父嘉所自出也嘉繼大宗延夀於后為叔祖】今當率如故事【謂將復以特牛祠也】唯陛下哀之今吏甫受詔讀記【師古曰甫始也】直豫言使后知之非可復若私府有所取也【師古曰若謂如未奉詔之前也復扶又翻下同】其萌牙所以約制妾者恐失人理【師古曰萌牙言其初始發意若草木之方生也】唯陛下深察焉上於是采谷永劉向所言災異咎驗皆在後宮之意以報之且曰吏拘於法亦安足過【過罪過也言何足以為罪也】盖矯枉者過直古今同之【師古曰矯正也枉曲也言意在正曲遂過於直】且財幣之省特牛之祠其於皇后所以扶助德美為華寵也咎根不除災變相襲【師古曰襲重累也】祖宗且不血食何戴侯也傳不云乎以約失之者鮮【師古曰論語載孔子之言鮮少也謂能行儉約而有過失之事如此者少也鮮音先踐翻】審皇后欲從其奢與【師古曰與讀曰歟】朕亦當法孝武皇帝也如此則甘泉建章可復興矣孝文皇帝朕之師也皇太后皇后成法也假使太后在彼時不如職今見親厚又惡可以踰乎【師古曰言假令太后昔時不得其志不依常理而皇后今被親厚何可踰於太后制度乎婦不可踰姑也烏音惡】皇后其刻心秉德謙約為右【師古曰以謙約為先】垂則列妾使有法焉【師古曰言垂法於後宮使皆遵行也】 給事中平陵平當上言太上皇漢之始祖廢其寢廟園非是【事見上卷元帝竟寧元年】上亦以無繼嗣遂納當言秋九月復太上皇寢廟園詔曰今大辟之刑千有餘條律令煩多百有餘萬言<br />
<br />
  奇請它比日以益滋【師古曰奇請謂常文之外主者别有所請以定罪也它比謂引它類以比附之稍增律條也奇音居宜翻】自明習者不知所由欲以曉喻衆庶不亦難乎於以羅元元之民夭絶無辜豈不哀哉【由從也羅謂設禁綱而民無所逃罪也夭絶亡辜謂無罪而陷於刑辟死於非命至於短折也夭於紹翻】其議減死刑及可蠲除約省者令較然易知條奏【易以豉翻】時有司不能廣宣上意徒鉤摭微細毛舉數事以塞詔而已【師古曰毛舉言舉毫毛之事輕小之甚塞猶當也塞悉則翻】 匈奴單于遣右臯林王伊邪莫演等奉獻朝正月【師古曰演音衍朝直遥翻考異曰匈奴傳河平元年單于遣莫演朝正月下云明年單于上書願朝河平四年正月遂入朝據此則是莫演以元年至漢朝二年正月也而荀紀繫于元年正月之下恐誤漢紀又以莫演為黄渾今從漢書】<br />
<br />
  二年春伊邪莫演罷歸【朝罷遣歸也】自言欲降【降戶江翻下同】即不受我我自殺終不敢還歸使者以聞下公卿議【下遐稼翻】議者或言宜如故事受其降光禄大夫谷永議郎杜欽以為漢興匈奴數為邊害【數所角翻】故設金爵之賞以待降者今單于屈體稱臣列為北藩遣使朝賀無有二心漢家接之宜異于往時今既享單于聘貢之質【師古曰享當也質誠也】而更受其逋逃之臣是貪一夫之得而失一國之心擁有罪之臣而絶慕義之君也假令單于初立【師古曰假令猶言或當也】欲委身中國未知利害私使伊邪莫演詐降以卜吉凶受之虧德沮善令單于自疏【疏與疎同】不親邊吏或者設為反間【間居莧翻】欲因以生隙受之適合其策使得歸曲而責直【師古曰歸曲於漢而以直義來責也】此誠邊境安危之原師旅動静之首不可不詳也不如勿受以昭日月之信抑詐諼之謀懷附親之心便【師古曰諼詐辭也音許遠翻又許元翻】對奏天子從之遣中郎將王舜往問降狀伊邪莫演曰我病狂妄言耳遣去歸到官位如故不肯令見漢使【史言永欽能得匈奴之情】 夏四月楚國雨雹【雨于具翻】大如釜 徙山陽王康為定陶王六月上悉封諸舅王譚為平阿侯商為成都侯立為<br />
<br />
  紅陽侯根為曲陽侯逢時為高平侯【恩澤侯表平阿侯食邑於沛成都侯食邑於山陽紅陽侯食邑於南陽曲陽侯食邑於九江高平侯食邑於臨淮】五人同日封故世謂之五侯太后母李氏更嫁為河内苟賓妻【據太后傳母李以妬去更嫁更工衡翻】生子參太后欲以田蚡為比而封之【李奇曰田蚡與孝景王后同母異父得封故也蚡扶粉翻師古曰比例也音頻寐翻】上曰封田氏非正也以參為侍中水衡都尉 御史大夫張忠奏京兆尹王尊暴虐倨慢尊坐免官吏民多稱惜之湖三老公乘興等【師古曰湖縣名也今虢州湖城縣取其名地理志湖縣屬京兆公乘以爵為姓乘繩證翻】上書訟尊治京兆撥劇整亂誅暴禁邪皆前所希有名將所不及【此將謂郡將也治直之翻將即亮翻】雖拜為真【尊自行尹事為真】未有殊絶褒賞加於尊身今御史大夫奏尊傷害隂陽為國家憂無承用詔書意靖言庸違象恭滔天【師古曰引虞書堯典之辭也靖治也庸用也違僻也滔漫也謂其言假托於治實用違僻貌象恭敬過惡漫天也漫音莫干翻一曰慆慢也】原其所以出御史丞楊輔【御史大夫有兩丞秩千石一曰中丞】素與尊有私怨外依公事建畫為此議傅致奏文【師古曰建立謀畫為此議也傅讀曰附謂益其事而引致於罪狀據尊傳輔故為尊書佐嘗醉過尊大奴利家利家捽搏其頰兄子閎拔刀欲剄之以故深怨欲傷害尊】浸潤加誣【師古曰浸潤猶漸染也】臣等竊痛傷尊修身潔己砥節首公【師古曰砥厲也首向也砥音指首音式救翻】刺譏不憚將相誅惡不避豪彊誅不制之賊【賊謂傰宗等】解國家之憂功著職修威信不廢誠國家爪牙之吏折衝之臣今一旦無辜制於仇人之手傷於詆欺之文上不得以功除罪下不得蒙棘木之聽【張晏曰周禮三槐九棘公卿於下聽訟王制大司寇聽獄於棘木之下棘者欲其赤心而留意于三刺也】獨掩怨讐之偏奏被共工之大惡【仲馮曰共工之大惡謂上劾奏云靖言庸違象恭滔天是也被皮義翻共音恭】無所陳寃愬罪尊以京師廢亂羣盜並興選賢徵用起家為卿賊亂既除豪猾伏辜即以佞巧廢黜一尊之身三期之間【師古曰期年也期音基】乍賢乍佞豈不甚哉孔子曰愛之欲其生惡之欲其死是惑也【論語所載答樊遲之言惡路烏翻】浸潤之譖不行焉可謂明矣【答子張之言】願下公卿大夫博士議郎定尊素行【下戶嫁翻行下孟翻】夫人臣而傷害隂陽死誅之罪也靖言庸違放殛之刑也【師古曰殛誅也音居力翻】審如御史章尊乃當伏觀闕之誅【張晏曰孔子誅少正卯於兩觀之間觀古玩翻】放於無人之域不得苟免【師古曰非止坐免官而已也】及任舉尊者當獲選舉之辜不可但已【任保也漢法選舉而其人不稱者與同罪】即不如章飾文深詆以愬無罪亦宜有誅以懲讒賊之口絶詐欺之路【師古曰懲創也】唯明主參詳使白黑分别【别彼列翻】書奏天子復以尊為徐州刺史【徐州部琅邪東海臨淮等郡及楚廣陵等國復扶又翻下同】 夜郎王興鉤町王禹漏卧侯俞更舉兵相攻【孟康曰漏卧夷邑名後為縣地理志夜郎鉤町漏卧三縣皆屬䍧柯郡鉤町音劬挺師古曰俞音踰】䍧柯太守請發兵誅興等【䍧柯音臧哥】議者以為道遠不可擊乃遣太中大夫蜀郡張匡持節和解興等不從命刻木象漢吏立道旁射之【射而亦翻】杜欽說大將軍王鳳曰蠻夷王侯輕易漢使不憚國威【說輸芮翻易以豉翻】恐議者選耎復守和解【師古曰選耎怯不前之意也選音息兖翻耎音人兖翻】太守察動静有變乃以聞如此則復曠一時【師古曰曠空也一時三月也言空廢一時不早發兵也】王侯得收獵其衆申固其謀黨助衆多各不勝忿【勝音升下同】必相殄滅自知罪成狂犯守尉【師古曰言起狂悖之心而殺守尉也】遠臧温暑毒草之地【臧古藏字通】雖有孫吴將賁育士【師古曰孫孫武吴吴起賁孟賁育夏育也將即亮翻賁音奔】若入水火往必焦没智勇亡所施【亡讀曰無】屯田守之費不可勝量【量音良】宜因其罪惡未成未疑漢家加誅隂敕旁郡守尉練士馬【師古曰練簡也守式又翻下同】大司農豫調穀積要害處【調徒釣翻】選任職太守往以秋凉時入誅其王侯尤不軌者即以為不毛之地無用之民聖王不以勞中國【師古曰即猶若也不毛言不生草木】宜罷郡放棄其民絶其王侯勿復通【復扶又翻】如以先帝所立累世之功不可墮壞【師古曰墮音火規翻毀也】亦宜因其萌牙早斷絶之及已成形然後戰師則萬姓被害【斷丁管翻下同被皮義翻】於是鳳薦金城司馬臨卭陳立為䍧柯太守【漢列郡守尉之下有長史司馬地理志臨卭縣屬蜀郡卭音渠容翻】立至䍧柯諭告夜郎王興興不從命立請誅之未報乃從吏數十人出行縣【行下孟翻】至興國且同亭【師古曰且音子餘翻按地理志夜郎縣王莽改曰同亭盖因亭以名縣也】召興興將數千人往至亭從邑君數十人入見立【按西南夷傳夷人椎結耕田有邑聚各有君長】立數責因斷頭【數所具翻】邑君曰將軍誅無狀為民除害【為于偽翻】願出曉士衆以興頭示之皆釋兵降【師古曰釋解也降戶江翻】鉤町王禹漏卧侯俞震恐入粟千斛牛羊勞吏士【勞力到翻】立還歸郡興妻父翁指與子邪務收餘兵廹脅旁二十二邑反至冬立奏募諸夷與都尉長史分將攻翁指等【將即亮翻】翁指據阸為壘立使奇兵絶其饟道【饟與餉同音息亮翻】縱反間以誘其衆【間居莧翻誘音酉】都尉萬年曰兵久不决費不可共【師古曰共讀曰供】引兵獨進敗走趨立營【師古曰趨讀曰趣趣向也音七喻翻】立怒叱戲下令格之【戲讀曰麾】都尉復還戰立救之時天大旱立攻絶其水道蠻夷共斬翁指持首出降西夷遂平【考異曰西夷傳但云河平中而胡旦漢春秋云在此年十一月未知何据也】<br />
<br />
  三年春正月楚王囂來朝【囂宣帝子於帝為叔父】二月乙亥詔以囂素行純茂特加顯異封其子勲為廣戚侯【廣戚侯國屬沛郡行下孟翻】 丙戌犍為地震山崩壅江水水逆流【犍居言翻】 秋八月乙卯晦日有食之 上以中秘書頗散亡【師古曰言中以别外藝文志曰武帝建藏書之策劉歆曰外則有太常太史博士之藏内則有延閣廣内祕室之府】使謁者陳農求遺書於天下詔光禄大夫劉向校經傳諸子詩賦步兵校尉任宏校兵書【百官表步兵校尉掌上林苑門屯兵武帝所置八校尉之一也任音壬校尉之校戶教翻餘並居效翻】太史令尹咸校數術【百官表太史令屬太常師古曰數術占卜之書】侍醫李柱國校方技【侍醫屬太醫令在天子左右者也師古曰方技醫藥之書也技音渠綺翻】每一書已向輒條其篇目撮其指意録而奏之【已終也竟也師古曰撮摠取也音千括翻】 劉向以王氏權位太盛而上方嚮詩書古文向乃因尚書洪範集合上古以來歷春秋六國至秦漢符瑞災異之記推迹行事連傅禍福【傅讀曰附】著其占驗比類相從各有條目凡十一篇號曰洪範五行傳論奏之天子心知向忠精故為鳳兄弟起此論也【為于偽翻】然終不能奪王氏權 河復决平原流入濟南千乘所壞敗者半建始時【復扶又翻下同濟子禮翻乘繩證翻壞音怪敗補邁翻】復遣王延世與丞相史楊焉及將作大匠許商【百官表曰將作少府秦官掌治宮室景帝中六年更名將作大匠】諫大夫乘馬延年同作治【孟康曰乘馬姓也師古曰乘音食證翻】六月乃成復賜延世黄金百斤治河卒非受平賈者為著外繇六月【蘇林曰平賈以錢取人作卒顧其時庸之平賈也如淳曰律說平賈一月得錢二千又律說戍邊一歲當罷若有急當留守六月今以卒治河之故復留六月孟康曰外繇戍邊也治水不復戍邊也師古曰如孟二說皆非也以卒治河有勞雖執役日近皆得比繇戍六月也著謂著簿籍也治直之翻賈讀曰價著音竹助翻繇讀曰徭】<br />
<br />
  四年春正月匈奴單于來朝 赦天下徒 三月癸丑朔日有食之 琅琊太守楊肜與王鳳連昏【如淳曰連昏者昏家之姻親也肜音以中翻】其郡有災害丞相王商按問之鳳以為請商不聽竟奏免肜奏果寢不下【下遐嫁翻下同】鳳以是怨商隂求其短使頻陽耿定上書【頻陽縣屬左馮翊】言商與父傅婢通及女弟淫亂奴殺其私夫【師古曰私夫女弟之私與姦通者】疑商教使天子以為暗昩之過不足以傷大臣鳳固争下其事司隸【句斷下遐嫁翻】太中大夫蜀郡張匡素佞巧復上書極言詆毁商【復扶又翻】有司奏請召商詣詔獄上素重商知匡言多險制曰勿治【治直之翻】鳳固争之夏四月壬寅詔收商丞相印綬商免相三日發病歐血薨諡曰戾侯而商子弟親屬為駙馬都尉侍中中常侍諸曹大夫郎吏者皆出補吏莫得留給事宿衛者有司奏請除國邑有詔長子安嗣爵為樂昌侯【宣帝之母黨微矣】 上之為太子也受論語於蓮勺張禹【蓮勺音輦酌】及即位賜爵關内侯拜為諸吏光禄大夫秩中二千石給事中領尚書事禹與王鳳並領尚書内不自安數病上書乞骸骨【數所角翻下同】欲退避鳳上不許撫待愈厚六月丙戌以禹為丞相封安昌侯【恩澤侯表安昌侯食邑於汝南】 庚戌楚孝王囂薨 初武帝通西域罽賓自以絶遠漢兵不能至獨不服【罽賓國治循鮮城去長安萬二千二百里不屬都護罽音計】數剽殺漢使【數所角翻師古曰剽刼也音頻妙翻】久之漢使者文忠與容屈王子隂末赴合謀攻殺其王【王曰烏頭勞即數殺漢使者也】立隂末赴為罽賓王後軍候趙德使罽賓與隂末赴相失隂末赴鎖琅當德【師古曰相失相失意也琅當長鎖也若今之禁繫人鎖矣琅音郎】殺副已下七十餘人遣使者上書謝孝元帝以其絶域不録放其使者於縣度【縣度在烏秅國西縣度者石山也谿谷不通以繩索相引而度縣古懸字通師古曰懸繩而度也烏秅鄭氏音鷃拏師古曰烏音一加翻秅音直加翻急言之聲如鷃拏耳非正音也】絶而不通及帝即位復遣使謝罪【復扶又翻下同】漢欲遣使者報送其使杜欽說王鳳曰前罽賓王隂末赴本漢所立後卒畔逆【說輸芮翻卒子恤翻】夫德莫大於有國子民罪莫大於執殺使者所以不報恩不懼誅者自知絶遠兵不至也有求則卑辭無欲則驕慢終不可懷服凡中國所以為通厚蠻夷㥦快其求者為壤比而為寇【師古曰比近也為其土壤接近能為寇也㥦音苦頰翻為壤之為于偽翻比毗寐翻】今縣度之阸非罽賓所能越也其鄉慕不足以安西域【鄉讀曰嚮】雖不附不能危城郭前親逆節惡暴西域【師古曰暴謂章露也】故絶而不通今悔過來而無親屬貴人奉獻者皆行賈賤人【賈音古下同】欲通貨市買以獻為名故煩使者送至縣度恐失實見欺凡遣使送客者欲為防護寇害也【為于偽翻】起皮山南更不屬漢之國四五【皮山國去長安萬五千里師古曰言經歷不屬漢者凡四五國更音工衡翻】斥候士百餘人五分夜擊刁斗自守【師古曰夜有五更分而持之】尚時為所侵盜驢畜負糧須諸國稟食得以自贍【師古曰稟給也贍足也食讀曰飤下同】國或貧小不能食或桀黠不肯給【黠下八翻】擁彊漢之節餒山谷之閒乞匄無所得【師古曰匄亦乞也音工大翻】離一二旬則人畜棄捐曠野而不反又歷大頭痛小頭痛之山赤土身熱之阪令人身熱無色頭痛嘔吐驢畜盡然【嘔一口翻吐土故翻畜許救翻下同】又有三池盤石阪道陿者尺六七寸長者徑三十里臨峥嶸不測之深【陿與狭同師古曰峥嶸深險之貌峥音仕耕翻嶸音宏余謂峥嶸山峻貌】行者騎步相持【騎奇寄翻】繩索相引【索昔各翻】二千餘里乃到縣度畜墜未半阬谷盡靡碎【師古曰靡散也音縻】人墮勢不得相收視險阻危害不可勝言【勝音升】聖王分九州制五服【師古曰九州冀兖豫青徐荆揚梁雍也五服甸侯綏要荒余謂此言禹迹也周職方九州有幽并無徐梁又分為九服】務盛内不求外今遣使者承至尊之命送蠻夷之賈勞吏士之衆涉危難之路【賈音古難乃旦翻】罷敝所恃以事無用【師古曰罷讀曰疲所恃謂中國之人無用謂遠方蠻夷之國】非久長計也使者業已受節可至皮山而還【師古曰言已立計遣之不能即止可至皮山也】於是鳳白從欽言罽賓實利賞賜賈市【賈音古】其使數年而壹至云<br />
<br />
  陽朔元年【應劭曰時隂盛陽微故改元曰陽朔欲陽氣之蘇息也師古曰應說非也朔始也以山陽火生石中言陽氣之始】春二月丁未晦日有食之 三月赦天下徒 冬京兆尹泰山王章下獄死【下遐稼翻】時大將軍鳳用事上謙讓無所顓左右嘗薦光禄大夫劉向少子歆通逹有異材上召見歆誦讀詩賦甚悦之欲以為中常侍【百官表中常侍加官得出入禁中盖此時以士人為之東都始純用宦者】召取衣冠臨當拜左右皆曰未曉大將軍【師古曰曉猶白余謂曉開諭也】上曰此小事何須關大將軍左右叩頭争之上於是語鳳鳳以為不可乃止【劉向忠於漢室子歆附從王莽得無由此邪爵賞之柄不自上出則貪爵禄苟富貴之人視其柄所在而趨之矣語牛倨翻】王氏子弟皆卿大夫侍中諸曹分據埶官滿朝廷杜欽見鳳專政泰重戒之曰願將軍由周公之謙懼損穰侯之威放武安之欲毋使范睢之徒得間其說【周公相成王管蔡流言周公狼跋而東其懼可知矣吐握以下士其謙可知矣穰侯范睢事見周紀武安侯田蚡事見武帝紀間居莧翻】鳳不聽時上無繼嗣體常不平【師古曰言多疾疢】定陶共王來朝【共讀曰恭下同】太后與上承先帝意遇共王甚厚賞賜十倍於它王不以往事為纎介【師古曰往事謂先帝時欲以代太子也言無纎介之嫌】留之京師不遣歸國上謂共王我未有子人命不諱【師古曰人命無常不可諱】一朝有它【師古曰它謂晏駕也】且不復相見【復扶又翻】爾長留侍我矣其後天子疾益有瘳共王因留國邸【定陶邸也】旦夕侍上上甚親重之大將軍鳳心不便共王在京師會日食鳳因言日食隂盛之象定陶王雖親於禮當奉藩在國今留侍京師詭正非常【師古曰詭違也】故天見戒【見賢遍翻】宜遣王之國上不得已於鳳而許之共王辭去上與相對涕泣而決【決與訣同别也】王章素剛直敢言雖為鳳所舉【章以選為京兆鳳所舉也】非鳳專權不親附鳳乃奏封事言日食之咎皆鳳專權蔽主之過上召見章延問以事【見賢遍翻】章對曰天道聰明佑善而災惡以瑞應而符效今陛下以未有繼嗣引近定陶王【近其靳翻】所以承宗廟重社稷上順天心下安百姓此正議善事當有祥瑞何故致災異災異之發為大臣專政者也【為于偽翻】今聞大將軍猥歸日食之咎於定陶王【師古曰猥猶曲也】建遣之國【師古曰建立其議也】苟欲使天子孤立於上顓擅朝事以便其私非忠臣也【朝直遥翻】且日食隂侵陽臣顓君之咎今政事大小皆自鳳出天子曾不壹舉手鳳不内省責【省悉井翻】反歸咎善人推遠定陶王【推吐雷翻遠于願翻】且鳳誣罔不忠非一事也前丞相樂昌侯商本以先帝外屬【商宣帝舅王武之子】内行篤【行下孟翻】有威重位歷將相國家柱石臣也其人守正不肯屈節隨鳳委曲卒用閨門之事為鳳所罷【卒子恤翻】身以憂死衆庶愍之又鳳知其小婦弟張美人己嘗適人【師古曰小婦妾也弟謂女弟即妹也今俗猶謂妾為小妻】於禮不宜配御至尊託以為宜子内之後宮苟以私其妻弟聞張美人未嘗任身就館也【師古曰是則不為宜子明鳳所言非實婦人將生子及月辰出就他館任讀曰姙】且羌胡尚殺首子以盪腸正世【師古曰盪洗滌也言婦初來所生之子或他姓】况於天子而近已出之女也【已出謂己出嫁也近其靳翻】此三者皆大事陛下所自見足以知其餘及它所不見者【師古曰以所見者譬之則不見者可知】鳳不可令久典事宜退使就第選忠賢以代之自鳳之白罷商後遣定陶王也上不能平及聞章言天子感寤納之謂章曰微京兆尹直言【師古曰微無也】吾不聞社稷計且唯賢知賢君試為朕求可以自輔者【為于偽翻下同】於是章奏封事薦信都王舅琅邪太守馮野王忠信質直智謀有餘上自為太子時數聞野王名【數所角翻】方倚以代鳳章每召見上輒辟左右【見賢遍翻師古曰辟讀曰闢】時太后從弟子侍中音【元后傳曰太后從弟長樂衛尉弘子侍中音師古曰弘者太后之叔父也音則從父弟余據後云音以從舅用事則顔注良是】獨側聽具知章言以語鳳【語牛倨翻】鳳聞之甚憂懼杜欽令鳳出就第上疏乞骸骨其辭指甚哀太后聞之為垂涕不御食【為于偽翻御進也】上少而親倚鳳【少詩照翻】弗忍廢乃優詔報鳳彊起之【彊其兩翻】於是鳳起視事上使尚書劾奏章知野王前以王舅出補吏而私薦之【成帝立有司奏野王王舅不宜備九卿出為上郡太守】欲令在朝阿附諸侯【朝直遥翻】又知張美人體御至尊而妄稱引羌胡殺子盪腸非所宜言下章吏【下遐稼翻】廷尉致其大逆罪【致文致也】以為比上夷狄欲絶繼嗣之端背畔天子【背蒲妹翻】私為定陶王章竟死獄中妻子徙合浦【合音蛤】自是公卿見鳳側目而視馮野王懼不自安遂病滿三月賜告與妻子歸杜陵就醫藥大將軍鳳風御史中丞劾奏野王【師古曰風讀曰諷劾戶槩翻又戶得翻】賜告養病而私自便【師古曰便安也音頻面翻】持虎符出界歸家奉詔不敬杜欽奏記於鳳曰二千石病賜告得歸有故事不得去郡亡著令【如淳曰律施行無不得去郡之文也亡讀曰無】傳曰賞疑從予所以廣恩勸功也【師古曰疑當賞不當賞則與之疑厚薄則從厚予讀曰與】罰疑從去所以慎刑闕難知也【師古曰疑當罰不當罰則赦之疑輕重則從輕去謂赦之也】今釋令與故事而假不敬之法【師古曰釋廢去也假謂假託法律以致其罪】甚違闕疑從去之意即以二千石守千里之地任兵馬之重【任音壬】不宜去郡將以制刑為後法者則野王之罪在未制令前也刑賞大信不可不慎鳳不聽竟免野王官時衆庶多寃王章譏朝廷者欽欲救其過復說鳳曰京兆尹章所坐事密自京師不曉况於遠方恐天下不知章實有罪而以為坐言事如是塞争引之原【復扶又翻下同說輸芮翻塞悉則翻争讀曰諍師古曰争引謂引事類以諫争之也一曰下有諫争之言上引而納之也】損寛明之德欽愚以為宜因章事舉直言極諫並見郎從官【從才用翻】展盡其意加於往前以明示四方使天下咸知主上聖明不以言罪下也若此則流言消釋疑惑著明鳳白行其策焉【杜欽之罪浮於谷永以其與王鳳計議為之文過也】 是歲陳留太守薛宣為左馮翊宣為郡所至有聲迹宣子惠為彭城令宣嘗過其縣心知惠不能不問以吏事或問宣何不教戒惠以吏職宣笑曰吏道以法令為師可問而知及能與不能自有資材何可學也衆人傳稱以宣言為然【時人雖以宣言為然實未必然也】<br />
<br />
  二年春三月大赦天下 御史大夫張忠卒 夏四月丁卯以侍中太僕王音為御史大夫於是王氏愈盛郡國守相刺史皆出其門下【師古曰言為其家寮屬者皆得大官】五侯羣弟争為奢侈【按元后傳王鳳兄弟八人鳳崇以與元后同母先侯譚商立根逢時同日侯世謂之五侯曼乃五侯之兄早死不侯五侯無羣弟疑羣字當作兄】賂遺珍寶四面而至皆通敏人事好士養賢傾財施予以相高尚【遺于季翻好呼到翻施式䜴翻予讀曰與】賓客滿門競為之聲譽劉向謂陳湯曰今災異如此而外家日盛其漸必危劉氏吾幸得以同姓末屬累世蒙漢厚恩身為宗室遺老歷事三主【三主宣元成】上以我先帝舊臣每進見常加優禮【見賢遍翻】吾而不言孰當言者遂上封事極諫曰臣聞人君莫不欲安然而常危莫不欲存然而常亡失御臣之術也夫大臣操權柄持國政未有不為害者也【操千高翻】故書曰臣之有作威作福害于而家凶于而國【師古曰周書洪範也而汝也言唯君得作威作福臣下為之則致凶害也】孔子曰禄去公室政逮大夫危亡之兆也【臣瓚曰政不由君下及大夫也師古曰論語孔子曰禄之去公室五世矣政逮於大夫四世矣故夫三桓之子孫微矣】今王氏一姓乘朱輪華轂者二十三人青紫貂蟬充盈幄内魚鱗左右【漢制列侯紫綬二千石青綬侍中中常侍皆銀璫左貂金附蟬師古曰言在帝之左右相次若魚鱗也】大將軍秉事用權五侯驕奢僭盛並作威福擊斷自恣行汚而寄治身私而託公【師古曰内為汚私之行而外則寄託治公之道也斷丁亂翻行下孟翻】依東宮之尊【師古曰東宮太后所居也余按漢制太后率居長樂宮在未央宮東故曰東宮】假甥舅之親以為威重尚書九卿州牧郡守皆出其門筦執樞機朋黨比周【比毗至翻】稱譽者登進忤恨者誅傷游談者助之說執政者為之言【譽音余忤五故翻為于偽翻】排擯宗室孤弱公族其有智能者尤非毁而不進遠絶宗室之任不令得給事朝省【遠于願翻朝直遥翻】恐其與己分權數稱燕王蓋主以疑上心【燕盖事見昭帝紀師古曰以示宗室親近而反逆也數所角翻】避諱呂霍而弗肯稱【呂氏事見呂后紀霍氏事見宣帝紀師古曰呂霍二家皆坐專擅誅滅故為王氏諱而不言也】内有管蔡之萌外假周公之論兄弟據重宗族磐互歷上古至秦漢外戚僭貴未有如王氏者也物盛必有非常之變先見為其人微象【言伏於微而著於象也見賢遍翻】孝昭帝時冠石立於泰山仆柳起於上林【事見二十三卷昭帝元鳳三年】而孝宣帝即位今王氏先祖墳墓在濟南者【王氏本濟南東平陵人武帝時繡衣御史王賀既免官乃徙居魏郡元城】其梓柱生枝葉扶疏上出屋根□地中【康曰□則洽切余按字書測洽之臿從千從臼與今臿字不同漢書作根垂地中意□即垂字也】雖立石起柳無以過此之明也事埶不兩大王氏與劉氏亦且不並立如下有泰山之安則上有累卵之危陛下為人子孫守持宗廟而令國祚移於外親降為皁隸【師古曰皁隸卑賤之人也春秋左氏傳曰大夫臣士士臣皁皁臣輿輿臣隸也】縱不為身【為于偽翻】奈宗廟何婦人内夫家而外父母家此亦非皇太后之福也【如淳曰内猶親也而皇太后反外夫家也】孝宣皇帝不與舅平昌侯權所以安全之也【平昌侯王無故宣帝舅也】夫明者起福於無形銷患於未然宜發明詔吐德音援近宗室親而納信【師古曰援引也謂升引而附近之也援音爰近其靳翻】黜遠外戚毋授以政【師古曰遠謂疏而離之也音于萬翻】皆罷令就弟【弟與第同漢書率作弟孟康曰第宅也有甲乙次第也亦作弟】以則效先帝之所行厚安外戚全其宗族誠東宮之意外家之福也王氏永存保其爵禄劉氏長安不失社稷所以褒睦外内之姓子子孫孫無疆之計也如不行此策田氏復見於今六卿必起於漢【田氏篡齊六卿分晉言漢亦將有此禍也復扶又翻】為後嗣憂昭昭甚明唯陛下深留聖思書奏天子召見向歎息悲傷其意謂曰君且休矣吾將思之然終不能用其言 秋關東大水 八月甲申定陶共王康薨 是歲徙信都王興為中山王<br />
<br />
  資治通鑑卷三十  <br>
   </div> 

<script src="/search/ajaxskft.js"> </script>
 <div class="clear"></div>
<br>
<br>
 <!-- a.d-->

 <!--
<div class="info_share">
</div> 
-->
 <!--info_share--></div>   <!-- end info_content-->
  </div> <!-- end l-->

<div class="r">   <!--r-->



<div class="sidebar"  style="margin-bottom:2px;">

 
<div class="sidebar_title">工具类大全</div>
<div class="sidebar_info">
<strong><a href="http://www.guoxuedashi.com/lsditu/" target="_blank">历史地图</a></strong>  
<a href="http://www.880114.com/" target="_blank">英语宝典</a>  
<a href="http://www.guoxuedashi.com/13jing/" target="_blank">十三经检索</a> 
<br><strong><a href="http://www.guoxuedashi.com/gjtsjc/" target="_blank">古今图书集成</a></strong> 
<a href="http://www.guoxuedashi.com/duilian/" target="_blank">对联大全</a> <strong><a href="http://www.guoxuedashi.com/xiangxingzi/" target="_blank">象形文字典</a></strong> 

<br><a href="http://www.guoxuedashi.com/zixing/yanbian/">字形演变</a>  <strong><a href="http://www.guoxuemi.com/hafo/" target="_blank">哈佛燕京中文善本特藏</a></strong>
<br><strong><a href="http://www.guoxuedashi.com/csfz/" target="_blank">丛书&方志检索器</a></strong> <a href="http://www.guoxuedashi.com/yqjyy/" target="_blank">一切经音义</a>  

<br><strong><a href="http://www.guoxuedashi.com/jiapu/" target="_blank">家谱族谱查询</a></strong>  <strong><a href="http://shufa.guoxuedashi.com/sfzitie/" target="_blank">书法字帖欣赏</a></strong> 
<br>

</div>
</div>


<div class="sidebar" style="margin-bottom:0px;">

<font style="font-size:22px;line-height:32px">QQ交流群9:489193090</font>


<div class="sidebar_title">手机APP 扫描或点击</div>
<div class="sidebar_info">
<table>
<tr>
	<td width=160><a href="http://m.guoxuedashi.com/app/" target="_blank"><img src="/img/gxds-sj.png" width="140"  border="0" alt="国学大师手机版"></a></td>
	<td>
<a href="http://www.guoxuedashi.com/download/" target="_blank">app软件下载专区</a><br>
<a href="http://www.guoxuedashi.com/download/gxds.php" target="_blank">《国学大师》下载</a><br>
<a href="http://www.guoxuedashi.com/download/kxzd.php" target="_blank">《汉字宝典》下载</a><br>
<a href="http://www.guoxuedashi.com/download/scqbd.php" target="_blank">《诗词曲宝典》下载</a><br>
<a href="http://www.guoxuedashi.com/SiKuQuanShu/skqs.php" target="_blank">《四库全书》下载</a><br>
</td>
</tr>
</table>

</div>
</div>


<div class="sidebar2">
<center>


</center>
</div>

<div class="sidebar"  style="margin-bottom:2px;">
<div class="sidebar_title">网站使用教程</div>
<div class="sidebar_info">
<a href="http://www.guoxuedashi.com/help/gjsearch.php" target="_blank">如何在国学大师网下载古籍?</a><br>
<a href="http://www.guoxuedashi.com/zidian/bujian/bjjc.php" target="_blank">如何使用部件查字法快速查字?</a><br>
<a href="http://www.guoxuedashi.com/search/sjc.php" target="_blank">如何在指定的书籍中全文检索?</a><br>
<a href="http://www.guoxuedashi.com/search/skjc.php" target="_blank">如何找到一句话在《四库全书》哪一页?</a><br>
</div>
</div>


<div class="sidebar">
<div class="sidebar_title">热门书籍</div>
<div class="sidebar_info">
<a href="/so.php?sokey=%E8%B5%84%E6%B2%BB%E9%80%9A%E9%89%B4&kt=1">资治通鉴</a> <a href="/24shi/"><strong>二十四史</strong></a>&nbsp; <a href="/a2694/">野史</a>&nbsp; <a href="/SiKuQuanShu/"><strong>四库全书</strong></a>&nbsp;<a href="http://www.guoxuedashi.com/SiKuQuanShu/fanti/">繁体</a>
<br><a href="/so.php?sokey=%E7%BA%A2%E6%A5%BC%E6%A2%A6&kt=1">红楼梦</a> <a href="/a/1858x/">三国演义</a> <a href="/a/1038k/">水浒传</a> <a href="/a/1046t/">西游记</a> <a href="/a/1914o/">封神演义</a>
<br>
<a href="http://www.guoxuedashi.com/so.php?sokeygx=%E4%B8%87%E6%9C%89%E6%96%87%E5%BA%93&submit=&kt=1">万有文库</a> <a href="/a/780t/">古文观止</a> <a href="/a/1024l/">文心雕龙</a> <a href="/a/1704n/">全唐诗</a> <a href="/a/1705h/">全宋词</a>
<br><a href="http://www.guoxuedashi.com/so.php?sokeygx=%E7%99%BE%E8%A1%B2%E6%9C%AC%E4%BA%8C%E5%8D%81%E5%9B%9B%E5%8F%B2&submit=&kt=1"><strong>百衲本二十四史</strong></a>  <a href="http://www.guoxuedashi.com/so.php?sokeygx=%E5%8F%A4%E4%BB%8A%E5%9B%BE%E4%B9%A6%E9%9B%86%E6%88%90&submit=&kt=1"><strong>古今图书集成</strong></a>
<br>

<a href="http://www.guoxuedashi.com/so.php?sokeygx=%E4%B8%9B%E4%B9%A6%E9%9B%86%E6%88%90&submit=&kt=1">丛书集成</a> 
<a href="http://www.guoxuedashi.com/so.php?sokeygx=%E5%9B%9B%E9%83%A8%E4%B8%9B%E5%88%8A&submit=&kt=1"><strong>四部丛刊</strong></a>  
<a href="http://www.guoxuedashi.com/so.php?sokeygx=%E8%AF%B4%E6%96%87%E8%A7%A3%E5%AD%97&submit=&kt=1">說文解字</a> <a href="http://www.guoxuedashi.com/so.php?sokeygx=%E5%85%A8%E4%B8%8A%E5%8F%A4&submit=&kt=1">三国六朝文</a>
<br><a href="http://www.guoxuedashi.com/so.php?sokeytm=%E6%97%A5%E6%9C%AC%E5%86%85%E9%98%81%E6%96%87%E5%BA%93&submit=&kt=1"><strong>日本内阁文库</strong></a> <a href="http://www.guoxuedashi.com/so.php?sokeytm=%E5%9B%BD%E5%9B%BE%E6%96%B9%E5%BF%97%E5%90%88%E9%9B%86&ka=100&submit=">国图方志合集</a> <a href="http://www.guoxuedashi.com/so.php?sokeytm=%E5%90%84%E5%9C%B0%E6%96%B9%E5%BF%97&submit=&kt=1"><strong>各地方志</strong></a>

</div>
</div>


<div class="sidebar2">
<center>

</center>
</div>
<div class="sidebar greenbar">
<div class="sidebar_title green">四库全书</div>
<div class="sidebar_info">

《四库全书》是中国古代最大的丛书,编撰于乾隆年间,由纪昀等360多位高官、学者编撰,3800多人抄写,费时十三年编成。丛书分经、史、子、集四部,故名四库。共有3500多种书,7.9万卷,3.6万册,约8亿字,基本上囊括了古代所有图书,故称“全书”。<a href="http://www.guoxuedashi.com/SiKuQuanShu/">详细>>
</a>

</div> 
</div>

</div>  <!--end r-->

</div>
<!-- 内容区END --> 

<!-- 页脚开始 -->
<div class="shh">

</div>

<div class="w1180" style="margin-top:8px;">
<center><script src="http://www.guoxuedashi.com/img/plus.php?id=3"></script></center>
</div>
<div class="w1180 foot">
<a href="/b/thanks.php">特别致谢</a> | <a href="javascript:window.external.AddFavorite(document.location.href,document.title);">收藏本站</a> | <a href="#">欢迎投稿</a> | <a href="http://www.guoxuedashi.com/forum/">意见建议</a> | <a href="http://www.guoxuemi.com/">国学迷</a> | <a href="http://www.shuowen.net/">说文网</a><script language="javascript" type="text/javascript" src="https://js.users.51.la/17753172.js"></script><br />
  Copyright &copy; 国学大师 古典图书集成 All Rights Reserved.<br>
  
  <span style="font-size:14px">免责声明:本站非营利性站点,以方便网友为主,仅供学习研究。<br>内容由热心网友提供和网上收集,不保留版权。若侵犯了您的权益,来信即刪。scp168@qq.com</span>
  <br />
ICP证:<a href="http://www.beian.miit.gov.cn/" target="_blank">鲁ICP备19060063号</a></div>
<!-- 页脚END --> 
<script src="http://www.guoxuedashi.com/img/plus.php?id=22"></script>
<script src="http://www.guoxuedashi.com/img/tongji.js"></script>

</body>
</html>
