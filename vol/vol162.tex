<!DOCTYPE html PUBLIC "-//W3C//DTD XHTML 1.0 Transitional//EN" "http://www.w3.org/TR/xhtml1/DTD/xhtml1-transitional.dtd">
<html xmlns="http://www.w3.org/1999/xhtml">
<head>
<meta http-equiv="Content-Type" content="text/html; charset=utf-8" />
<meta http-equiv="X-UA-Compatible" content="IE=Edge,chrome=1">
<title>資治通鑒_163-資治通鑑卷一百六十二_163-資治通鑑卷一百六十二</title>
<meta name="Keywords" content="資治通鑒_163-資治通鑑卷一百六十二_163-資治通鑑卷一百六十二">
<meta name="Description" content="資治通鑒_163-資治通鑑卷一百六十二_163-資治通鑑卷一百六十二">
<meta http-equiv="Cache-Control" content="no-transform" />
<meta http-equiv="Cache-Control" content="no-siteapp" />
<link href="/img/style.css" rel="stylesheet" type="text/css" />
<script src="/img/m.js?2020"></script> 
</head>
<body>
 <div class="ClassNavi">
<a  href="/24shi/">二十四史</a> | <a href="/SiKuQuanShu/">四库全书</a> | <a href="http://www.guoxuedashi.com/gjtsjc/"><font  color="#FF0000">古今图书集成</font></a> | <a href="/renwu/">历史人物</a> | <a href="/ShuoWenJieZi/"><font  color="#FF0000">说文解字</a></font> | <a href="/chengyu/">成语词典</a> | <a  target="_blank"  href="http://www.guoxuedashi.com/jgwhj/"><font  color="#FF0000">甲骨文合集</font></a> | <a href="/yzjwjc/"><font  color="#FF0000">殷周金文集成</font></a> | <a href="/xiangxingzi/"><font color="#0000FF">象形字典</font></a> | <a href="/13jing/"><font  color="#FF0000">十三经索引</font></a> | <a href="/zixing/"><font  color="#FF0000">字体转换器</font></a> | <a href="/zidian/xz/"><font color="#0000FF">篆书识别</font></a> | <a href="/jinfanyi/">近义反义词</a> | <a href="/duilian/">对联大全</a> | <a href="/jiapu/"><font  color="#0000FF">家谱族谱查询</font></a> | <a href="http://www.guoxuemi.com/hafo/" target="_blank" ><font color="#FF0000">哈佛古籍</font></a> 
</div>

 <!-- 头部导航开始 -->
<div class="w1180 head clearfix">
  <div class="head_logo l"><a title="国学大师官网" href="http://www.guoxuedashi.com" target="_blank"></a></div>
  <div class="head_sr l">
  <div id="head1">
  
  <a href="http://www.guoxuedashi.com/zidian/bujian/" target="_blank" ><img src="http://www.guoxuedashi.com/img/top1.gif" width="88" height="60" border="0" title="部件查字,支持20万汉字"></a>


<a href="http://www.guoxuedashi.com/help/yingpan.php" target="_blank"><img src="http://www.guoxuedashi.com/img/top230.gif" width="600" height="62" border="0" ></a>


  </div>
  <div id="head3"><a href="javascript:" onClick="javascript:window.external.AddFavorite(window.location.href,document.title);">添加收藏</a>
  <br><a href="/help/setie.php">搜索引擎</a>
  <br><a href="/help/zanzhu.php">赞助本站</a></div>
  <div id="head2">
 <a href="http://www.guoxuemi.com/" target="_blank"><img src="http://www.guoxuedashi.com/img/guoxuemi.gif" width="95" height="62" border="0" style="margin-left:2px;" title="国学迷"></a>
  

  </div>
</div>
  <div class="clear"></div>
  <div class="head_nav">
  <p><a href="/">首页</a> | <a href="/ShuKu/">国学书库</a> | <a href="/guji/">影印古籍</a> | <a href="/shici/">诗词宝典</a> | <a   href="/SiKuQuanShu/gxjx.php">精选</a> <b>|</b> <a href="/zidian/">汉语字典</a> | <a href="/hydcd/">汉语词典</a> | <a href="http://www.guoxuedashi.com/zidian/bujian/"><font  color="#CC0066">部件查字</font></a> | <a href="http://www.sfds.cn/"><font  color="#CC0066">书法大师</font></a> | <a href="/jgwhj/">甲骨文</a> <b>|</b> <a href="/b/4/"><font  color="#CC0066">解密</font></a> | <a href="/renwu/">历史人物</a> | <a href="/diangu/">历史典故</a> | <a href="/xingshi/">姓氏</a> | <a href="/minzu/">民族</a> <b>|</b> <a href="/mz/"><font  color="#CC0066">世界名著</font></a> | <a href="/download/">软件下载</a>
</p>
<p><a href="/b/"><font  color="#CC0066">历史</font></a> | <a href="http://skqs.guoxuedashi.com/" target="_blank">四库全书</a> |  <a href="http://www.guoxuedashi.com/search/" target="_blank"><font  color="#CC0066">全文检索</font></a> | <a href="http://www.guoxuedashi.com/shumu/">古籍书目</a> | <a   href="/24shi/">正史</a> <b>|</b> <a href="/chengyu/">成语词典</a> | <a href="/kangxi/" title="康熙字典">康熙字典</a> | <a href="/ShuoWenJieZi/">说文解字</a> | <a href="/zixing/yanbian/">字形演变</a> | <a href="/yzjwjc/">金 文</a> <b>|</b>  <a href="/shijian/nian-hao/">年号</a> | <a href="/diming/">历史地名</a> | <a href="/shijian/">历史事件</a> | <a href="/guanzhi/">官职</a> | <a href="/lishi/">知识</a> <b>|</b> <a href="/zhongyi/">中医中药</a> | <a href="http://www.guoxuedashi.com/forum/">留言反馈</a>
</p>
  </div>
</div>
<!-- 头部导航END --> 
<!-- 内容区开始 --> 
<div class="w1180 clearfix">
  <div class="info l">
   
<div class="clearfix" style="background:#f5faff;">
<script src='http://www.guoxuedashi.com/img/headersou.js'></script>

</div>
  <div class="info_tree"><a href="http://www.guoxuedashi.com">首页</a> > <a href="/SiKuQuanShu/fanti/">四库全书</a>
 > <h1>资治通鉴</h1> <!--         下载:【右键另存为】即可 --></div>
  <div class="info_content zj clearfix">
  
<div class="info_txt clearfix" id="show">
<center style="font-size:24px;">163-資治通鑑卷一百六十二</center>
    資治通鑑卷一百六十二 宋 司馬光 撰<br />
<br />
  胡三省 音註<br />
<br />
  梁紀十八【屠維大荒落凡一年】<br />
<br />
  高祖武皇帝十八<br />
<br />
  太清三年春正月丁巳朔柳仲禮自新亭徙營大桁會大霧韋粲軍迷失道比及青塘【比必利翻】夜已過半立柵未合侯景望見之亟帥鋭卒攻粲【過工禾翻帥讀曰率】粲使軍主鄭逸逆擊之命劉叔胤以舟師截其後【截其渡淮之路】叔胤畏懦不敢進逸遂敗景乘勝入粲營左右牽粲避賊粲不動叱子弟力戰遂與子尼及三弟助警搆從弟昂皆戰死【從才用翻】親戚死者數百人【史言韋粲忠勇】仲禮方食投箸被甲與其麾下百騎馳往救之【箸竹助翻被皮義翻騎奇寄翻】與景戰於青塘大破之斬首數百級沈淮水死者千餘人【沈持林翻】仲禮矟將及景【矟色角翻】而賊將支伯仁自後斫仲禮中肩馬陷于淖【支伯仁當作支化仁將即亮翻下同中竹仲翻淖奴教翻泥也】賊聚矟刺之騎將郭山石救之得免仲禮被重瘡會稽人惠臶吮瘡斷血【刺七亦翻被皮義翻臶徂悶翻吮徂兖翻】故得不死自是景不敢復濟南㟁【復扶又翻下不復綸復同】仲禮亦氣索【索蘇各翻】不復言戰矣邵陵王綸復收散卒【邵陵王綸敗走見上卷上年】與東揚州刺史臨城公大連新淦公大成等自東道並至【淦古暗翻】庚申列營于桁南亦推柳仲禮為大都督大連大臨之弟也朝野以侯景之禍共尤朱异【朝直遥翻】异慙憤發疾庚申卒 【考異曰梁帝紀作乙丑今從太清紀典略】故事尚書官不以為贈上痛惜异特贈尚書右僕射甲子湘東世子方等及王僧辯軍至 【考異曰梁帝紀作戊辰今從太清紀】 戊辰封山侯正表以北徐州降東魏東魏徐州刺史高歸彦遣兵赴之歸彦歡之族弟也 己巳太子遷居永福省【永福省在禁中自宋以來太子居之取其福國于有永也】高州刺史李遷仕【五代志高凉郡梁置高州】天門太守樊文皎將援兵萬餘人至城下臺城與援軍信命久絶有羊車兒獻策作紙䲭【紙䲭即紙鳶也今俗謂之紙鷂䲭丑之翻】繫以長繩寫勑於内放以從風冀達衆軍題云得䲭送援軍賞銀百兩太子自出太極殿前乘西北風縱之賊怪之以為厭勝射而下之【厭於恊翻射而亦翻】援軍募人能入城送啟者鄱陽世子嗣左右李朗請先受鞭詐為得罪叛投賊因得入城城中方知援兵四集舉城鼓譟上以朗為直閤將軍賜金遣之朗緣鍾山之後宵行晝伏積日乃達癸未鄱陽世子嗣永安侯確莊鐵羊鴉仁柳敬禮李遷仕樊文皎將兵度淮攻東府前柵焚之侯景退衆軍營於青溪之東遷仕文皎帥鋭卒五千獨進【帥讀曰率】深入所向摧靡至菰首橋東【橋在青溪上菰音孤菰首今人謂之茭白】景將宋子仙伏兵擊之【將即亮翻下同】文皎戰死遷仕遁還敬禮仲禮之弟也仲禮神情傲狠陵蔑諸將邵陵王綸每日執鞭至門亦移時弗見【凡部將見諸帥執鞭以為禮狠戶墾翻】由是與綸及臨城公大連深相仇怨大連又與永安侯確有隙【永安本漢彘縣順帝陽嘉元年更名永安魏晉屬平陽郡江左僑立南河東郡併僑立永安縣屬荆州注又見前】諸軍互相猜阻莫有戰心援軍初至建康士民扶老擕幼以候之纔過淮即縱兵剽掠【㔄匹妙翻】由是士民失望賊中有謀應官軍者聞之亦止【史言臺城覆陷之由】王顯貴以夀陽降東魏【侯景命王顯貴守夀陽見上卷上年】 臨賀王記室吳郡顧野王起兵討侯景二月己丑引兵來至初臺城之閉也公卿以食為念男女貴賤並出負米得四十萬斛收諸府藏錢帛五十萬億並聚德陽堂【藏徂浪翻】而不備薪芻魚鹽至是壞尚書省為薪撒薦剉以飼馬薦盡又食以飯【薦以藁秸為之所以藉寢壞音怪飼食並詳吏翻】軍士無膎【膎戶皆翻脯也又肉食肴】或煮鎧熏鼠捕雀而食之【鎧可亥翻】御甘露厨有乾苔味酸鹹分給戰士【釋氏謂營膳之所曰甘露厨乾音干苔生于海其形如髪春二三月間海人採取之成片納土窖中出而曬之令乾南人多食之】軍人屠馬於殿省間雜以人肉食者必病侯景衆亦飢抄掠無所獲【抄楚交翻】東城有米可支一年【東城即東府城】援軍斷其路【斷音短】又聞荆州兵將至景甚患之王偉曰今臺城不可卒拔援兵日盛吾軍乏食若偽求和以緩其勢東城之米足支一年因求和之際運米入石頭援軍必不得動然後休士息馬繕脩器械伺其懈怠擊之一舉可取也【伺相吏翻】景從之遣其將任約于子悦至城下拜表求和乞復先鎮【將即亮翻任音壬先鎮謂夀陽時已降齊矣】太子以城中窮困白上請許之上怒曰和不如死太子固請曰侯景圍逼已久援軍相仗不戰【仗除兩翻】宜且許其和更為後圖上遲回久之乃曰汝自圖之勿令取笑千載遂報許之【太子綱疑范桃棒之來降而信侯景之請和何其昧也載子亥翻】景乞割江右四州之地【江右四州南豫西豫合州光州】并求宣城王大器出送然後濟江中領軍傅岐固争曰豈有賊舉兵圍宫闕而更與之和乎此特欲却援軍耳戎狄獸心必不可信且宣城嫡嗣之重國命所繫豈可為質【梁之智士唯傅岐一人而已質音致下同】上乃以大器之弟石城公大欵為侍中出質於景又勑諸軍不得復進【復扶又翻】下詔曰善兵不戰止戈為武可以景為大丞相都督江西四州諸軍事豫州牧河南王如故己亥設壇於西華門外遣僕射王克上甲侯韶吏部郎蕭瑳【韶帝室也封上甲侯宋白曰江州德安縣本蒲塘塲晉建興初始以為郡領尋陽上甲柴桑九江等縣義熙中以尋陽入柴桑上甲入彭凙瑳七何翻又七可翻】與于子悦任約王偉登壇共盟太子詹事柳津出西華門景出柵門遥相對更殺牲歃血為盟【更工衡翻歃色甲翻】既盟而景長圍不解專脩鎧仗【鎧可亥翻】託云無船不得即發又云恐南軍見躡【援軍時皆屯秦淮南岸故謂之南軍】遣石城公還臺求宣城王出送邀求稍廣了無去志太子知其詐言猶羈縻不絶韶懿之孫也庚子前南兖州刺史南康王會理前青冀二州刺史湘潭侯退西昌侯世子彧衆合三萬至于馬卬洲【馬卬洲盖即今王家沙老鸛觜一帶 考異曰梁帝紀作丁未今從太清紀典略典略云至于琅邪今從太清紀梁帝紀 按晉置琅邪郡於江乘蒲洲上即前所謂今王家沙也】景慮其自白下而上【上時掌翻】啓云請北軍聚還南岸【以地望言之馬卬洲在臺城之北故云北軍南岸即謂秦淮南岸】不爾妨臣濟江太子即勒會理自白下城移軍江潭苑 【考異曰梁帝紀作蘭亭苑今從太清紀典略】退恢之子也辛丑以邵陵王綸為司空鄱陽王範為征北將軍柳仲禮為侍中尚書右僕射景以于子悦任約傅士悊皆為儀同三司【悊與哲同】夏侯譒為豫州刺史【諸補過翻】董紹先為東徐州刺史徐思玉為北徐州刺史王偉為散騎常侍【散悉覽翻騎奇寄翻】上以偉為侍中乙卯景又啟曰適有西岸信至【大江西岸即歷陽】高澄已得夀陽鍾離臣今無所投足求借廣陵并譙州俟得夀陽即奉還朝廷又云援軍既在南岸須於京口度江太子並荅許之癸卯大赦庚戍景又啟曰永安侯確直閤趙威方頻隔柵見詬云天子自與汝盟我終當破汝乞召侯及威方入即當引路【言引兵就路還北詬古候翻又許候翻】上遣吏部尚書張綰召確辛亥以確為廣州刺史威方為盱眙太守【盱眙音吁怡守手又翻】確累啟固辭不入上不許確先遣威方入城因欲南奔【確盖欲南奔荆江二鎮】邵陵王綸泣謂確曰圍城既久聖人憂危臣子之情切于湯火故欲且盟而遣之更申後計成命已决何得拒違時臺使周石珍東宫主書左法生在綸所【使疏吏翻下同】確謂之曰侯景雖云欲去而不解長圍意可見也今召僕入城何益于事石珍曰勑旨如此郎那得辭確意尚堅綸大怒謂趙伯超曰譙州為我斬之【為于偽翻】持其首去伯超揮刃眄確【眄眠見翻目斜視也】曰伯超識君侯刀不識也確乃流涕入城【景凡所請上父子無不從求以却其攻乃所以速其攻也】上常蔬食及圍城日久上厨蔬茹皆絶乃食雞子綸因使者蹔通上雞子數百枚【上雞時掌翻】上時自料簡【料音聊】歔欷哽咽【歔音虚欷許既翻又音希哽古杏翻】湘東王繹軍於郢州之武城【水經注武口水上通安陸之延頭南至武城入大江吳舊屯所在荆州界盡此盖今之沙武口即其地】湘州刺史河東王譽軍於青草湖【水經注湘水自汨羅口西北逕壘石山西北對青草湖祝穆曰青草湖一名巴丘湖北洞庭南瀟湘東納汨羅之水自昔與洞庭並稱按一湖之内南名青草北名洞庭中有沙洲間之所謂重湖也】信州刺史桂陽王慥軍於西峽口【五代志巴東郡梁置信州唐之夔州也水經注江水自巴東魚復縣東逕廣溪峡斯乃三峡首也峡中有瞿塘黄龕二灘慥七到翻】託云俟四方援兵淹留不進中記室參軍蕭賁骨鯁士也以繹不早下心非之嘗與繹雙六食子未下賁曰殿下都無下意【雙六亦傳之一名續事始云陳思王製雙六局置骰子二唐末有葉子之戲遂加至六戰國策曰博之所以貴梟者便則食不便則止可以食子而未下者擬議其便否也賁因其未下借雙六以諷其不下救君父】繹深銜之及得上勑繹欲旋師賁曰景以人臣舉兵向闕今若放兵未及渡江童子能斬之矣必不為也大王以十萬之衆未見賊而退柰何繹不悦未幾因事殺之【幾居起翻】慥懿之孫也 東魏河内民四千餘家以魏北徐州刺史司馬裔其鄉里也相帥歸之【卧讀曰率下同】丞相泰欲封裔裔固辭曰士大夫遠歸皇化裔豈能帥之賣義士以求榮非所願也【據周書裔司馬楚之之後司馬氏本河内温人魏孝武西遷裔始歸鄉里於温城起義附西魏與東魏交戰頻有克獲授河内郡守尋加持節平東將軍北徐州刺史帥讀曰率】 侯景運東府米入石頭既畢王偉聞荆州軍退【謂湘東王繹旋師也】援軍雖多不相統壹乃說景曰王以人臣舉兵圍守宫闕逼辱妃主殘穢宗廟擢王之髮不足數罪【用史記須賈之言擢拔也說式芮翻數所角翻】今日持此欲安所容身乎背盟而捷自古多矣【背蒲妺翻】願且觀其變臨賀王正德亦謂景曰大功垂就豈可弃去景遂上啟陳帝十失且曰臣方事暌違所以冒陳讜直【讜音黨】陛下崇飾虚誕惡聞實録【上時掌翻惡烏路翻】以祅怪為嘉禎【祅于驕翻禎音真祥也】以天譴為無咎敷演六藝排擯前儒王莽之法也以鐵為貨輕重無常公孫之制也【漢公孫述據蜀用鐵錢】爛羊鐫印朝章鄙雜更始趙倫之化也【漢更始濫授官爵長安為之語曰爛羊胃騎都尉爛羊頭關内侯晉趙王倫篡位貂蟬盈坐時人為之語曰貂不足狗尾續朝直遥翻更工衡翻】豫章以所天為血讎【見一百五十卷普通六年】邵陵以父存而冠布【事見同上冠古玩翻又如字】石虎之風也【石虎父子事見晉成帝紀】修建浮圖百度糜費使四民飢餒苲融姚興之代也【苲融事佛事見漢獻帝紀姚興事佛事見晉安帝紀苲在各翻】又言建康宫室崇侈陛下唯與主書參斷萬機政以賄成諸閹豪盛衆僧殷實皇太子珠玉是好酒色是耽【斷丁亂翻好呼到翻】吐言止於輕薄賦詠不出桑中【桑中見詩衛國風淫放之詩也】邵陵所在殘破湘東羣下貪縱南康定襄之屬皆如沐猴而冠耳【南康王會理帝子續之子時鎮廣陵定襄侯祗南平王偉之子時鎮淮隂沐猴而冠用漢書語】親為孫姪位則藩屏【屏必郢翻】臣至百日誰肯勤王此而靈長未之有也昔鬻拳兵諫王卒改善【左傳鬻拳彊諫楚子楚子弗從臨之以兵懼而從之卒子恤翻】今日之舉復奚罪乎【復扶又翻】伏願陛下小懲大戒【引易大傳之言指斥甚矣】放讒納忠使臣無再舉之憂陛下無嬰城之辱則萬姓幸甚上覽啟且慙且怒【言皆指實而無如之何有慙怒而已】三月丙辰朔立壇於太極殿前告天地以景違盟舉烽鼓譟初閉城之日男女十餘萬擐甲者二萬餘人 【考異曰南史作三萬今從典略】被圍既久人多身腫氣急【氣急上氣喘急也被皮義翻】死者什八九乘城者不滿四千人率皆羸喘横尸滿路不可瘞埋【羸倫為翻喘昌兖翻瘞於計翻】爛汁滿溝而衆心猶望外援柳仲禮唯聚妓妾置酒作樂【妓渠綺翻】諸將日往請戰仲禮不許安南侯駿說邵陵王綸曰【說式芮翻 考異曰典略云綸已下咸說柳仲禮如此今從太清紀】城危如此而都督不救若萬一不虞殿下何顔自立于世今宜分軍為三道出賊不意攻之可以得志綸不從柳津登城謂仲禮曰汝君父在難【難乃旦翻】不能竭力百世之後謂汝為何仲禮亦不以為意上問策於津對曰陛下有邵陵臣有仲禮不忠不孝賊何由平 【考略曰典略云柳仲禮族兄暉謂仲禮曰天下事勢如此何不自取富貴仲禮曰兄今若為取之暉曰正當堅營不戰使賊平臺城囚天子徐而縱兵既破之後復挟天子令諸侯也仲禮納之按景既克城則人情皆去援軍自散仲禮安能帥以破景仲禮閉壁不出自為重傷而懼耳非用暉計也今從太清紀及南史太清紀又云景嘗登朱雀門樓與之語又遺以金自是以後閉壁不戰典略云景遺以金鐶亦又近誣今不取】戊午南康王會理與羊鴉仁趙伯超等進營於東府城北約夜度軍既而鴉仁等暁猶未至景衆覺之營未立景使宋子仙擊之趙伯超望風退走【寒山之敗玄武湖側之敗及此時之敗皆趙伯超為之也】會理等兵大敗戰及溺死者五千人【溺奴狄翻】景積其首於闕下以示城中景又使于子悦求和上使御史中丞沈浚至景所景實無去志謂浚曰今天時方熱軍未可動乞且留京師立効浚發憤責之景不對横刀叱之【示將殺浚也】浚曰負恩忘義違弃詛盟【詛莊助翻】固天地所不容沈浚五十之年常恐不得死所何為以死相懼邪因徑去不顧景以其忠直捨之於是景決石闕前水【石闕前水景決玄武湖以灌城者也】百道攻城晝夜不息邵陵世子堅屯太陽門【太陽門臺城六門之一也】終日蒱飲【蒱音蒲蒱飲樗蒱且飲酒也】不恤吏士其書佐董勛熊曇朗恨之【考之南史此熊曇朗非後來為盗于豫章之熊曇朗也南史侯景傳作白曇朗曇徒含翻】丁卯夜向曉勛曇朗於城西北樓引景衆登城永安侯確力戰不能却乃排闥入啟上云城已陷上安臥不動曰猶可一戰乎確曰不可上嘆曰自我得之自我失之亦復何恨因謂確曰汝速去語汝父勿以二宫為念因使慰勞在外諸軍【復扶又翻語牛倨翻勞力到翻】俄而景遣王偉入文德殿奉謁上命褰簾開戶引偉入偉拜呈景啟稱為奸佞所蔽領衆入朝【朝直遥翻】驚動聖躬今詣闕待罪上問景何在可召來景入見於太極東堂以甲士五百人自衛景稽顙殿下典儀引就三公榻【典儀典朝儀者也至唐猶有典儀之職掌殿上贊唱之節及設殿庭服位之次見賢遍翻稽音啓】上神色不變問曰卿在軍中日久無乃為勞景不敢仰視汗流被面【被皮義翻】又曰卿何州人而敢至此妻子猶在北邪景皆不能對任約從旁代對曰臣景妻子皆為高氏所屠唯以一身歸陛下【自此以上上問景景猶慴伏】上又問初度江有幾人景曰千人圍臺城幾人曰十萬今有幾人曰率土之内莫非己有【自此以上景之辭氣悖矣】上俛首不言【上辭窮勢屈故俛首不言嗚呼】景復至永福省見太子太子亦無懼容侍衛皆驚散唯中庶子徐摛【太子中庶子職如侍中摛丑知翻】通事舍人陳郡殷不害側侍【東宫通事舍人職如中書通事舍人】摛謂景曰侯王當以禮見【見賢遍翻】何得如此景乃拜【荀子曰善敗者不亡帝父子於此亦亡而不失其守者悲夫】太子與言又不能對景退謂其廂公王僧貴曰【景之親貴隆重者號曰左右廂公】吾嘗跨鞌對陳【陳讀曰陣】矢刃交下而意氣安緩了無怖心今見蕭公使人自慴【怖普布翻慴之涉翻】豈非天威難犯吾不可以再見之於是悉撒兩宫侍衛【兩宫謂上臺及東宫】縱兵掠乘輿服御宫人皆盡收朝士王侯送永福省【乘繩證翻朝直遥翻】使王偉守武德殿于子悦屯太極東堂矯詔大赦自加大都督中外諸軍録尚書事 【考異曰梁帝紀無赦加景官在庚午今從太清紀】建康士民逃難四出【難乃旦翻】太子洗馬蕭允【洗悉薦翻】至京口端居不行曰死生有命如何可逃禍之所來皆生於利苟不求利禍從何生己巳景遣石城公大欵以詔命解外援軍【五代志宣城郡秋浦縣舊曰石城 考異曰典略在庚午梁帝紀在辛未今從太清紀】柳仲禮召諸將議之【將即亮翻】邵陵王綸曰今日之命委之將軍仲禮熟視不對裴之高王僧辯曰將軍擁衆百萬致宫闕淪沒正當悉力决戰何所多言仲禮竟無一言諸軍乃随方各散【言諸軍各随所來之方散去也】南兖州刺史臨城公大連【按姚思亷梁書大連封臨城縣公自東揚州入援臺城既陷復還會稽參考通鑑前後所書皆然此誤以東揚州為南兖州當書南兖州刺史南康王會理東揚州刺史臨城公大連盖傳寫逸南康王會理東揚州刺史十字】湘東世子方等鄱陽世子嗣北兖州刺史湘潭侯退【亦當書北兖州刺史定襄侯祇前青冀二州刺史湘潭侯退五代志衡山郡有湘潭縣】吴郡太守袁君正晉陵太守陸經等各還本鎮君正昂之子也【帝平建康袁昂以節義見褒位至台司】邵陵王綸奔會稽【會工外翻】仲禮及弟敬禮羊鴉仁王僧辯趙伯超並開營降【降戶江翻下同】軍士莫不歎憤仲禮等入城先拜景而後見上【見賢遍翻下見父同】上不與言仲禮見父津津慟哭曰汝非我子何勞相見湘東王繹使全威將軍會稽王琳送米二十萬石以饋軍至姑孰聞臺城陷沈米於江而還【沈待林翻】景命燒臺内積尸病篤未絶者【未絶謂猶有餘息者】亦聚而焚之庚午詔征鎮牧守可復本任景留柳敬禮羊鴉仁而遣柳仲禮歸司州王僧辯歸竟陵【王僧辯得歸竟陵為湘東王繹用之以平侯景張本】初臨賀王正德與景約平城之日不得全二宫及城開正德帥衆揮刀欲入【帥讀曰率】景先使其徒守門故正德不果入景更以正德為侍中大司馬百官皆復舊職正德入見上【更工衡翻見賢遍翻】拜且泣上曰啜其泣矣何嗟及矣【詩中谷有蓷之辭啜張劣翻啜者泣多而不止也讀如輟】秦郡陽平盱眙三郡皆降景【沈約曰晉武帝分扶風為秦國中原亂其民南流寄居堂邑堂邑本為縣西漢屬臨淮郡後漢屬廣陵國晉惠帝永興元年以臨淮淮陵立堂邑郡安帝改堂邑為秦郡五代志曰江都郡六合縣舊曰蔚氏置秦郡又有安宜縣梁置陽平郡】景改陽平為北滄州改秦郡為西兖州東徐州刺史湛海珍北青州刺史王奉伯【五代志東海郡懷仁】<br />
<br />
  【縣梁置南北二青州下邳郡梁置東徐州 考異曰北青州典略作南冀州今從太清紀】並以地降東魏青州刺史明少遐山陽太守蕭鄰棄城走【五代志海州懷仁縣舊置南北二青州江都郡山陽縣舊置山陽郡 考異曰梁紀在四月今從太清紀】東魏據其地 侯景以儀同三司蕭邕為南徐州刺史代西昌侯淵藻鎮京口又遣其將徐相攻晉陵【將即亮翻】陸經以郡降之 初上以河東王譽為湘州刺史徙湘州刺史張纘為雍州刺史代岳陽王詧纘恃其才望輕譽少年迎候有闕譽至檢括州府付度事【付度者前刺史以州府之若事若物付度後刺史雍於用翻少詩沼翻】留纘不遣聞侯景作亂頗陵蹙纘纘恐為所害輕舟夜遁將之雍部復慮詧拒之【復扶又翻】纘與湘東王繹有舊欲因之以殺譽兄弟乃如江陵及臺城陷諸王各還州鎮譽自湖口歸湘州【洞庭青草共為一湖湖口在巴陵】桂陽王慥以荆州督府【湘東王繹以荆州刺史都督荆雍等九州慥譽詧皆其屬也】留軍江陵欲待繹至拜謁乃還信州纘遺繹書曰河東戴檣上水欲襲江陵【檣船上桅竿也所以掛帆帆汎風則船行自洞庭至江陵泝江而上故曰上水遺于季翻上時掌翻】岳陽在雍共謀不逞江陵游軍主朱榮【游軍主領游軍之將也】亦遣使告繹云桂陽留此欲應譽詧【使疏吏翻】繹懼鑿船沈米斬纜【沈持林翻纜盧闞翻維舟索也】自蠻中步道馳歸江陵囚慥殺之【繹與譽詧自此隙矣】侯景以前臨江太守董紹先為江北行臺【五代志歷陽郡烏江縣梁置臨江郡董紹先降侯景見上卷上年】使齎上手勑召南兖州刺史南康王會理壬午紹先至廣陵衆不滿二百皆積日飢疲會理士馬甚盛僚佐說會理曰【說式芮翻】景已陷京邑欲先除諸藩然後簒位若四方拒絶立當潰敗奈何委全州之地以資寇手不如殺紹先發兵固守與魏連和以待其變會理素懦即以城授之紹先既入衆莫敢動會理弟通理請先還建康謂其姊曰事既如此豈可闔家受斃前途亦思自効但未知天命如何耳紹先悉收廣陵文武部曲鎧仗金帛【鎧可亥翻】遣會理單馬還建康【為會理兄弟謀誅王偉不克而死張本】 湘潭侯退與北兖州刺史定襄侯祗出奔東魏侯景以蕭弄璋為北兖州刺史州民發兵拒之景遣直閣將軍羊海將兵助之海以其衆降東魏【將即亮翻下同降戶江翻】東魏遂據淮隂祗偉之子也癸未侯景遣于子悦等將羸兵數百東略吳郡【羸倫為翻】新城戍主戴僧逷有精甲五千【沈約曰浙江西南名曰相溪吴立為新城縣屬吳郡今杭州新城縣即其地逷他歷翻】說太守袁君正曰【說式芮翻】賊今乏食臺中所得不支一旬若閉關拒守立可餓死土豪陸映公恐不能勝而資產被掠皆勸君正迎之【被皮義翻】君正素怯載米及牛酒郊迎子悦執君正掠奪財物子女東人皆立堡拒之景又以任約為南道行臺鎮姑孰【任音壬】夏四月湘東世子方等至江陵湘東王繹始知臺城不守命於江陵四旁七里樹木為柵掘塹三重而守之【塹七艶翻重直龍翻】 東魏高岳等攻魏潁川不克大將軍澄益兵助之道路相繼踰年猶不下【去年四月高岳等攻潁川】山鹿忠武公劉豐生建策堰洧水以灌之【五代志朔方郡長澤縣後魏置闡熙郡及山鹿縣水經洧水出河南密縣西南馬領山東南過長社縣北堰於扇翻】城多崩頹岳悉衆分休迭進【言分兵為十數部甲休則乙進乙休則丙進丙休則丁進至於癸休則甲復進矣攻者得番休而應者不勝其勞也】王思政身當矢石與士卒同勞苦城中泉湧懸釡而炊太師泰遣大將軍趙貴督東南諸州兵救之自長社以北皆為陂澤兵至穰【穰即穰城】不得前東魏使善射者乘大艦臨城射之【艦戶黯翻射之而亦翻下射殺同】城垂陷燕郡景惠公慕容紹宗與劉豐生臨堰視之【燕因肩翻】見東北塵起同入艦坐避之俄而暴風至遠近晦冥纜斷飄船徑向城【纜盧瞰翻】城上人以長鉤牽船弓弩亂發紹宗赴水溺死【溺奴狄翻】豐生游上向土山【浮水而行曰游上時掌翻】城上人射殺之 甲辰東魏進大將軍勃海王澄位相國封齊王加殊禮【時令澄贊拜不名入朝不趨劒履上殿】丁未澄入朝於鄴固辭不許澄召將佐密議之皆勸澄宜膺朝命【朝直遥翻】獨散騎常侍陳元康以為未可澄由是嫌之崔暹乃薦陸元規為大行臺郎以分元康之權 湘東王繹之入援也令所督諸州皆發兵雍州刺史岳陽王詧遣府司馬劉方貴將兵出漢口【雍於用翻將即亮翻】繹召詧使自行詧不從方貴潛與繹相知謀襲襄陽未發會詧以他事召方貴方貴以為謀泄遂據樊城拒命詧遣軍攻之繹厚資遣張纘使赴鎮纘至大堤【沈約志華山郡治大隄五代志襄陽郡漢南縣宋置華山郡唐併漢南入宜城縣九域志宜城在襄州南九十里曾鞏曰宋武帝築宜城之大隄為城今縣治是也】詧已拔樊城斬方貴纘至襄陽詧推遷未去但以城西白馬寺處之【處昌呂翻】詧猶摠軍府之政聞臺城陷遂不受代助防杜岸紿纘曰觀岳陽勢不容使君不如且往西山以避禍【西山謂萬山以西中廬縣諸山也紿待亥翻】岸既襄陽豪族兄弟九人皆以驍勇著名【杜氏兄弟嵩岑嶷岌巚岸崱嵸幼安凡九人驍堅堯翻】纘乃與岸結盟著婦人衣【著陟略翻】乘青布輿逃入西山詧使岸將兵追擒之纘乞為沙門更名法纘詧許之【張纘間搆譽詧兄弟於湘東凶于而身害于而國更工衡翻】 荆州長史王冲等上牋於湘東王繹請以太尉都督中外諸軍事承制主盟【主盟者主諸藩之盟】繹不許丙辰又請以司空主盟亦不許 上雖外為侯景所制而内甚不平景欲以宋子仙為司空上曰調和隂陽焉用此物【三公燮理隂陽言宋子仙非其人也】景又請以其黨二人為便殿主帥【梁禁中諸殿皆有主帥杜佑曰凡言便殿者皆非正大之處又曰便殿寢側之别殿帥所類翻】上不許景不能強【強其兩翻】心甚憚之太子入泣諫上曰誰令汝來若社稷有靈猶當克復如其不然何事流涕景使其軍士入直省中或驅驢馬帶弓刀出入宫庭上怪而問之直閤將軍周石珍對曰侯丞相甲士上大怒叱石珍曰是侯景何謂丞相左右皆懼是後上所求多不遂志飲膳亦為所裁節憂憤成疾太子以幼子大圜屬湘東王繹【屬之欲翻託也】并剪爪髪以寄之五月丙辰上卧浄居殿口苦索蜜不得【索山客翻】再曰荷荷【荷下可翻】遂殂年八十六景祕不發喪遷殯於昭陽殿【侯景時居昭陽殿】迎太子於永福省使如常入朝【朝直遥翻】王偉陳慶皆侍太子太子嗚咽流涕不敢泄聲殿外文武皆莫之知 東魏高岳既失慕容紹宗等志氣沮喪不敢復逼長社城【杜佑曰許州長葛縣故長社城王思政所守也沮在呂翻喪息浪翻復扶又翻】陳元康言於大將軍澄曰王自輔政以來未有殊功雖破侯景本非外賊【太清元年高澄輔政次年破侯景】今潁川垂陷願王自以為功澄從之戊寅自將步騎十萬攻長社【將即亮翻騎奇寄翻】親臨作堰堰三決澄怒推負土者及囊并塞之【推吐雷翻塞悉則翻】 辛巳發高祖喪【帝殂二十六日而後發喪】升梓宫於太極殿是日太子即皇帝位大赦侯景出屯朝堂分兵守衛【景自昭陽殿出屯朝堂朝堂盖在太極殿左右朝直遥翻】 壬午詔北人在南為奴婢者皆免之所免萬計景或更加超擢冀收其力高祖之末建康士民服食器用争尚豪華糧無半年之儲常資四方委輸【毛晃曰漢有三輔委輸官掌委輸者也凡以物送之曰輸則音平聲指所送之物曰輸則音去聲委輸之委亦音去聲】自景作亂道路斷絶數月之間人至相食猶不免餓死存者百無一二【金陵記曰梁都之時戶二十八萬西石頭城東至倪塘南至石子岡北過蒋山南北各四十里侯景之亂至于陳時中外人物不迨宋齊之半】貴戚豪族皆自出採稆【稆音呂禾不因播種而生曰稆】填委溝壑不可勝紀【勝音升】癸未景遣儀同三司來亮入宛陵【宛陵縣漢屬丹陽郡晉分為宣城郡治所五代志宣城郡治宣城縣舊曰宛陵】宣城太守楊白華誘而斬之甲申景遣其將李賢明攻之不克【華讀曰花誘音酉將即亮翻下同 考異曰典略在四月今從太清紀】景又遣中軍侯子鑒入吴郡【中軍中軍都督也】以廂公蘇單于為吴郡太守【單音蟬】遣儀同宋子仙等將兵東屯錢塘新城戍主戴僧逷拒之御史中丞沈浚避難東歸【難乃旦翻】至吳興太守張嵊與之合謀舉兵討景嵊稷之子也【嵊石證翻張稷弑齊東昏侯後死于鬰洲】東揚州刺史臨城公大連亦據州不受景命【五代志會稽郡梁置東揚州】景號令所行唯吳郡以西南陵以北而已 魏詔太和中代人改姓者皆復其舊【改姓見一百四十卷齊明帝建武三年】 六月丙戌以南康王會理為侍中司空 【考異曰梁紀作戊戌今從太清紀】 丁亥立宣城王大器為皇太子 【考異曰太清紀云七日今從梁帝紀及典略】 初侯景將使太常卿南陽劉之遴授臨賀王正德璽綬之遴剃髪僧服而逃之【遴良刃翻璽斯氏翻綬音受剃他計翻】之遴博學能文嘗為湘東王繹長史將歸江陵繹素嫉其才己丑之遴至夏口繹密送藥殺之【夏戶雅翻】而自為誌銘厚其贈賻【賻音附】 壬辰封皇子大心為尋陽王大欵為江陵王大臨為南海王大連為南郡王大春為安陸王大成為山陽王大封為宜都王 【考異曰太清紀典略並與立太子同日今從梁帝紀】 長社城中無鹽人病攣腫【孿呂緣翻】死者什八九大風從西北起吹水入城城壞東魏大將軍澄令城中曰有能生致王大將軍者封侯若大將軍身有損傷親近左右皆斬【太清元年西魏授王思政大將軍故以稱之】王思政帥衆據土山【東魏築土山以攻潁川思政奪而據之帥讀曰率】告之曰吾力屈計窮唯當以死謝國因仰天大哭西向再拜欲自刎【刎扶粉翻】都督駱訓曰公常語訓等【語牛倨翻】汝齎我頭出降【降戶江翻】非但得富貴亦完一城人今高相既有此令【相息亮翻】公獨不哀士卒之死乎衆共執之不得引决澄遣通直散騎趙彦深就土山遺以白羽扇【遺于季翻】執手申意牽之以下澄不令拜延而禮之思政初入潁川將士八千人【將即亮翻下同】及城陷纔三千人卒無叛者【卒子恤翻】澄悉散配其將卒於遠方改潁州為鄭州【按魏收志潁州本治長社既改鄭州徙治潁隂城領許昌潁川陽翟郡】禮遇思政甚重西閣祭酒盧曰【後齊之制三師二大三公各置東西閣祭酒二人大司馬大將軍也】思政不能死節何足可重澄謂左右曰我有盧乃是更得一王思政度世之曾孫也【盧度世事魏大武帝】初思政屯襄城欲以長社為行臺治所遣使者魏仲啟陳於太師泰【使疏吏翻】并致書於浙州刺史崔猷【魏收志浙州領修陽朱陽南上洛浙陽固郡五代志浙陽郡西魏置浙州唐志鄧州内鄉縣本浙陽郡治】猷復書曰襄城控帶京洛實當今之要地如有動静易相應接潁川既鄰寇境又無山川之固賊若潛來徑至城下莫若頓兵襄城為行臺之所潁川置州遣良將鎮守則表裏膠固人心易安【將即亮翻易以豉翻】縱有不虞豈能為患仲見泰具以啓聞【具以思政所請崔猷所報二者皆啓聞也】泰令依猷策思政固請且約賊水攻期年【期讀曰朞】陸攻三年之内朝廷不煩赴救泰乃許之及長社不守泰深悔之猷孝芬之子也【崔孝芬為高歡所殺子猷入關事見一百五十六卷中大通六年】侯景之南叛也【事見一百六十卷元年】丞相泰恐東魏復取景所部地【復扶又翻】使諸將分守諸城及潁川陷泰以諸城道路阻絶皆令拔軍還【史言宇文泰不求廣地之名而審計利害之實】 上甲侯韶自建康出奔江陵稱受高祖密詔徵兵 【考異曰梁帝紀在五月今從太清紀】以湘東王繹為侍中假黄钺大都督中外諸軍事司徒承制自餘藩鎮並加位號 宋子仙圍戴僧逷不克丙午吳盜陸緝等 【考異曰典略作戊子陸黯今從太清紀南史】起兵襲吴郡殺蘇單于推前淮南太守文成侯寧為主 臨賀王正德怨侯景賣已密書召鄱陽王範使以兵入景遮得其書癸丑縊殺正德【縊於賜翻又於計翻 考異曰典略五月正德死今從太清紀南史】景以儀同三司郭元建為尚書僕射北道行臺摠江北諸軍事鎮新秦【舊置秦郡於六合新秦即秦郡也簡文帝之廢也元建自秦郡馳還諫景此可證也】封元羅等諸元十餘人皆為王 【考異曰太清紀在八月二十八日今從典略】景愛永安侯確之勇常寘左右邵陵王綸遣人呼之確曰景輕佻【佻土彫翻】一夫力耳我欲手刃之正恨未得其便卿還啟家王勿以確為念景與確遊鍾山引弓射鳥因欲射景【射而亦翻】弦斷不發景覺而殺之 【考異曰太清紀確死在九月今從典略】 湘東王繹娶徐孝嗣孫女為妃生世子方等妃醜而妬又多失行【行下孟翻下穢行同】繹二三年一至其室妃聞繹當至以繹目眇為半面糚以待之繹怒而出故方等亦無寵及自建康還江陵繹見其御軍和整始歎其能入告徐妃妃不對垂泣而退繹怒疏其穢行牓于大閤方等見之益懼湘州刺史河東王譽驍勇得士心繹將討侯景遣使督其糧衆【驍堅堯翻使疏吏翻下同】譽曰各自軍府何忽隸人使者三返譽不與方等請討之繹乃以少子安南侯方矩為湘州刺史方等將精卒二萬送之【少詩照翻將即亮翻考異曰太清紀云初上遣諮議參軍周弘直往湘州報河東王譽云侯景既須撲滅今欲遣荆州兵力使汝】<br />
<br />
  【東往但使諸蕭有一人能匡國難吾無所惜譽對弘直攘云身始至鎮百度俱闕征伐之任便未能行又遣舍人虞預至譽所曰周弘直還知汝必不能自出師吾今便長驅席卷還望三湘兵糧以相資給譽又拒絶意色殊憤上又遣録事參軍劉㲄往雍宣旨於岳陽王詧曰吾舟艦足乘唯糧仗闕少湘州有米已就譽求雍部精兵必能分遣行留之計爾自擇之詧答曰兵馬蕃扞所須非敢減徹襄陽形勝之地豈可暫虚㲄出謂雍州别駕甄玄成曰觀殿下辭色曾無匡復之意卿是股肱所寄可相毗贊邪答曰樊沔衝要王業所基人情驍勇山川險固君其雅識寧俟多言㲄曰本論東討共征獯逆義異西伯非敢聞命於是湘雍二藩成亂謀矣是月上遣世子方等往湘州具陳軍國之計誡方等曰吾近累遣使往湘並未相唇齒今故令汝至彼必望申吾意若能相随不可留王冲權知州事譽遂不受命潜圖搆逆此皆蕭韶為元帝隱惡飾辭耳今從梁書南史】方等將行謂所親曰是行也吾必死之死得其所吾復奚恨【方等不死于救臺城之時而死于伐湘州之日可謂得其死乎復扶又翻】 侯景以趙威方為豫章太守江州刺史尋陽王大心遣軍拒之擒威方繫州獄威方逃還建康 湘東世子方等軍至麻溪【據水經注麻溪水口在臨湘縣北瀏口戍南】河東王譽將七千人撃之方等軍敗溺死【溺奴狄翻】安南侯方矩收餘衆還江陵湘東王繹無戚容繹寵姬王氏生子方諸王氏卒繹疑徐妃為之【疑其毒殺之】逼令自殺妃赴井死葬以庶人禮不聽諸子制服【史言湘東猜薄】 西江督護陳霸先欲起兵討侯景景使人誘廣州刺史元景仲許奉以為主【元景仲法僧之子普通六年父子歸梁誘音酉】景仲由是附景隂圖霸先霸先知之與成州刺史王懷明等集兵南海【五代志蒼梧郡梁置成州南海郡即廣州治所】馳檄以討景仲曰元景仲與賊合從【從子容翻】朝廷遣曲陽侯勃為刺史軍已頓朝亭【酈道元曰廣州城東北三十里有朝臺昔尉佗因罡作臺北而朝漢圓基千步直峭百丈頂上三畝複道環迴逶迤曲折朔望升拜名曰朝臺朝直遥翻】景仲所部聞之皆弃景仲而散秋七月甲寅景仲縊於閤下【縊於賜翻】霸先迎定州刺史蕭勃鎮廣州前高州刺史蘭裕欽之弟也【五代志高凉郡梁置高州蘭欽見一百五十七卷大同元年】與其諸弟扇誘始興等十郡攻監衡州事歐陽頠【誘音酉監工銜翻下同頠魚毁翻】勃使霸先救之悉擒裕等 【考異曰太清紀檎裕在八月今從陳書】勃因以霸先監始興郡事 湘東王繹遣竟陵太守王僧辯信州刺史東海鮑泉撃湘州分給兵糧刻日就道僧辯以竟陵部下未盡至欲俟衆集然後行與泉入白繹求申期【申重也重為期日】繹疑僧辯觀望案劒厲聲曰卿憚行拒命欲同賊邪今日唯有死耳因斫僧辯中其左髀【中竹仲翻】悶絶久之方蘇即送獄泉震怖不敢言【怖普布翻】僧辯母徒行流涕入謝自陳無訓繹意解賜以良藥故得不死丁卯鮑泉獨將兵伐湘州【將即亮翻 考異曰太清紀作八日或者八日受命丁卯乃行也】 陸緝等競為暴掠吳人不附宋子仙自錢塘旋軍擊之【時宋子仙攻戴僧逷屯錢塘】壬戌緝弃城奔海鹽【吴記曰海鹽本名武原鄉秦以為縣屬吴郡今屬嘉興府在府東南八十里】子仙復據吴郡【復扶又翻】戊辰侯景置吳州于吳郡以安陸王大春為刺史 庚午以南康王會理兼尚書令 【考異曰太清紀在八月二十六日今從典略】 鄱陽王範聞建康不守戒嚴欲入僚佐或說之曰今魏人已據夀陽大王移足則虜騎必窺合肥前賊未平後城失守【說式芮翻守式又翻】將若之何不如待四方兵集使良將將精卒赴之【將即亮翻下同】進不失勤王退可固本根範乃止會東魏大將軍澄遣西兖州刺史李伯穆逼合肥又使魏收為書諭範範方謀討侯景藉東魏為援乃帥戰士二萬出東關以合州輸伯穆【帥讀曰率】并遣諮議劉靈議送二子勤廣為質於東魏以乞師【諮議者諮議參軍質音致】範屯濡須以待上游之軍遣世子嗣將千餘人守安樂柵【安樂柵者範所立柵以安樂名之然臺城覆陷父兄蒙塵此子弟沬血枕戈之時以安樂名柵非名也樂音洛】上游諸軍皆不下範糧乏采苽稗菱藕以自給【苽與菰同音孤雕菰米本草又謂之茭白歲久中心生白臺謂之菰米其臺中有黑者謂之茭欎至後結實乃雕胡黑米也稗蒲賣翻稗者似稻其實尖圓而細】勤廣至鄴東魏人竟不為出師【為于偽翻】範進退無計【進則孤羸之軍不足以制侯景退則合肥已為東魏人所據】乃泝流西上【上時掌翻】軍於樅陽【樅陽縣漢屬廬江郡晉書五代志同安郡同安縣舊曰樅陽并置樅陽郡師古曰樅七容翻】景出屯姑孰範將裴之悌以衆降之【降戶江翻】之悌之高之弟也 東魏大將軍澄詣鄴辭爵位殊禮且請立太子澄謂濟隂王暉業曰比讀何書暉業曰數尋伊霍之傳不讀曹馬之書【謂伊霍輔少主曹馬篡國也濟子禮翻比毗至翻數所角翻傳直戀翻】 八月甲申朔侯景遣其中軍都督侯子鑒等擊吴興 己亥鮑泉軍於石椁寺河東王譽逆戰而敗辛丑又敗于橘洲【晏公類要曰橘洲在長沙西南四十里湘江中四洲橘洲其一也水經注沅水東逕龍陽縣之汎洲洲長二十里吴丹陽太守李衡植柑於其上臨死勑其子曰吾州里有木奴千頭不責衣食歲絹千匹今洲上猶有遺枿余按汎洲乃柑洲非橘洲水經注又云湘水北過臨湘縣西又北過南津城西西對橘洲此則是也類要亦指此張舜民郴行録橘洲東對潭洲城】戰及溺死者萬餘人【溺奴狄翻】譽退保長沙泉引軍圍之辛卯東魏立皇子長仁為太子勃海文襄王澄以其弟太原公洋次長【次長言於兄弟行澄居長而洋次之長竹丈翻】意常忌之洋深自晦匿言不出口常自貶退與澄言無不順從澄輕之常曰此人亦得富貴相書亦何可解【相息亮翻古之唐舉許負皆相視人之骨法狀貌以知吉凶貴賤有相書傳於世解戶買翻】洋為其夫人趙郡李氏營服玩小佳【為于偽翻下為爾同】澄輒奪取之夫人或恚未與【恚於避翻】洋笑曰此物猶應可求兄須何容吝惜【須者意所欲亦求也】澄或愧不取洋即受之亦無飾讓每退朝還第【朝直遥翻】輒閉閤静坐雖對妻子能竟日不言或時袒跣奔躍夫人問其故洋曰為爾漫戲【漫戲言漫爾作戲】其實盖欲習勞也澄獲徐州刺史蘭欽子京以為膳奴【蘭欽仕梁為徐州刺史考異曰陳元康傳作蘭固成今從北齊帝紀】欽請贖之不許京屢自訴澄杖之曰更訴當殺汝京與其黨六人謀作亂澄在鄴居北城東柏堂嬖瑯邪公主【琅邪公主事始見一百五十九卷大同十一年】欲其往來無間【間古莧翻】侍衛者常遣出外辛卯澄與散騎常侍陳元康吏部尚書侍中楊愔黄門侍郎崔季舒屏左右謀受魏禪【愔於今翻屏必郢翻】署擬百官蘭京進食澄却之謂諸人曰昨夜夢此奴斫我當急殺之京聞之寘刀盤下冒言進食澄怒曰我未索食【索山客翻求也】何為遽來京揮刀曰來殺汝澄自投傷足入于牀下賊去牀弑之【去羌呂翻】愔狼狽走遺一靴【靴詩戈翻】季舒匿于厠中元康以身蔽澄與賊争刀被傷腸出【被皮義翻下同】庫直王紘冒刃禦賊紇奚舍樂鬬死時變起倉猝内外震駭太原公洋在城東雙堂聞之顔色不變指揮部分【分扶問翻】入討羣賊斬而臠之【臠力兖翻割切其肉也】徐出曰奴反大將軍被傷無大苦也内外莫不驚異【洋素自晦匿今遇變而不為之變故皆驚而異之】洋祕不發喪陳元康手書辭母口占使功曹參軍祖珽作書陳便宜【珽待鼎翻】至夜而卒洋殯之第中詐云出使【使疏吏翻】虚除元康中書令【并秘陳元康死問亦所以鎮安人情】以王紘為領左右都督紘基之子也【王基見一百五十六卷中大通六年】勲貴以重兵皆在并州勸洋早如晉陽洋從之夜召大將軍督護太原唐邕使部分將士鎮遏四方【分扶問翻】邕支配須臾而畢【支分也配隸也支配猶今人言品配】洋由是重之癸巳洋諷東魏主以立太子大赦【託建儲大赦以安蘭京之黨心懷反側者】澄死問漸露東魏主竊謂左右曰大將軍今死似是天意威權當復歸帝室矣【復扶又翻下可復同】洋留太尉高岳太保高隆之開府儀同三司司馬子如侍中楊愔守鄴餘勲貴皆自随甲午入謁東魏主於昭陽殿從甲士八千人登階者二百餘人皆攘扣刃若對嚴敵令主者傳奏曰臣有家事須詣晉陽【主者主朝儀者也】再拜而出東魏主失色目送之曰此人又似不相容朕不知死在何日晉陽舊臣宿將素輕洋及至大會文武神彩英暢言辭敏洽衆皆大驚【晉陽文武之驚洋猶郼城内外也】澄政令有不便者洋皆改之高隆之司馬子如等惡度支尚書崔暹【惡烏路翻】奏暹及崔季舒過惡鞭二百徙邊【二人素為澄所親任故隆之等惡之】 侯景以宋子仙為司徒郭子建為尚書左僕射與領軍任約等四十人並開府儀同三司仍詔自今開府儀同不須更加將軍【梁制雖三公亦加將軍號今開府儀同三司亦不加】是後開府儀同至多不可復記矣 鄱陽王範自樅陽遣信告江州刺史尋陽王大心【樅七容翻】大心遣信邀之範引兵詣江州大心以湓城處之【一棲不兩䧺為範與大心互相猜忌張本處昌呂翻】 吳興兵力寡弱張嵊書生不閑軍旅【閑習也】或勸嵊效袁君正以郡迎侯子鑒嵊歎曰袁氏世濟忠貞【袁氏自淑至顗粲及昂皆以忠貞著節】不意君正一旦隳之吾豈不知吳郡既沒吳興勢難久全但以身許國有死無貳耳九月癸丑朔子鑒軍至吳興嵊戰敗還府整服安坐子鑒執送建康侯景嘉其守節欲活之嵊曰吾忝任專城朝廷傾危不能匡復今日速死為幸景猶欲全其一子嵊曰吾一門己在鬼録【魏文帝書曰觀其姓名已在鬼録録籍也】不就爾虜求恩景怒盡殺之【張嵊閣門死義以雪其父弑君之醜血祀絶矣】并殺沈浚【沈浚責侯景之時視死如歸其後與張嵊起兵豈望生邪】 河東王譽告急於岳陽王詧詧留諮議參軍濟陽蔡大寶守襄陽帥衆二萬騎二千伐江陵以救湘州【濟子禮翻帥讀曰率騎奇寄翻】湘東王繹大懼遣左右就獄中問計於王僧辯僧辯具陳方略繹乃赦之以為城中都督乙卯詧至江陵作十三營以攻之會大雨平地水深四尺【深式浸翻】詧軍氣沮【沮在呂翻】繹與新興太守杜崱有舊【漢獻帝建安十二年省雲中定襄五原朔方四郡郡立一縣合為新興郡属并州晉江左僑立於荆州界領定襄廣牧等縣五代志南郡安興縣舊置廣牧定襄縣唐省安興縣入江陵則新興固荆州所統矣何待繹以舊好密邀崱哉盖崱雖帶新興太守實從軍在襄陽也崱士力翻】密邀之乙丑崱與兄岌岸弟幼安兄子龕各帥所部降于繹【岌魚及翻龕苦含翻降戶江翻】岸請以五百騎襲襄陽晝夜兼行去襄陽三十里城中覺之蔡大寶奉詧母龔保林【保林宫中女官自漢以來有之顔師古曰保安也言其可安衆如林也齊高帝建元三年太子宫置三内職良姊比開國侯保林比五等侯才人比駙馬都尉】登城拒戰詧聞之夜遁弃糧食金帛鎧仗於湕水不可勝紀【鎧可亥翻丁度曰湕紀偃切水名出南郡今荆門軍北百里有建水盖即此水也勝音升】張纘病足詧載以随軍及敗走守者恐為追兵所及殺之弃尸而去 【考異曰太清紀云詧使制文檄纘曰吾蒙朝廷不世之榮又荷湘東王國士之眷今日雖死義無操筆及軍敗將殺之纘曰若使南師必振北賊將亡吾雖死無所恨遂殺之弃尸于江陵北湖又云諸將並欲追上以如子之情情所未忍曰彼不應來而來明其為逆我應逐不逐見我之弘此盖亦蕭韶之虚美今從南史】詧至襄陽岸奔廣平依其兄南陽太守巚【晉渡江南僑立廣平郡於襄陽宋以漢南陽郡之朝陽縣為實土水經注古朝陽縣在新野縣西巚與巘同魚蹇翻】 湘東王繹以鮑泉圍長沙久不克怒之以平南將軍王僧辯代為都督數泉十罪【數所具翻】命舍人羅重懽與僧辯偕行【重直龍翻】泉聞僧辯來愕然曰得王竟陵來助我賊不足平拂席待之僧辯入背泉而坐【背蒲妹翻】曰鮑郎卿有罪令旨使我鎻卿【時繹下書于所部稱令故曰令旨】卿弗以故意見期使重懽宣令鎻之牀側泉為啓自申【申明也理也】且謝淹緩之罪繹怒解遂釋之 冬十月癸未朔東魏以開府儀同三司潘相樂為司空 初歷陽太守莊鐵帥衆歸尋陽王大心【鐵歸大心見一百六十一卷太清二年帥讀曰率】大心以為豫章内史鐵至郡即叛推觀寜侯永為主永範之弟也丁酉鐵引兵襲尋陽大心遣其將徐嗣徽逆擊破之鐵走至建昌【建昌縣漢和帝永元十六年分海昏立屬豫章郡】光遠將軍韋構邀擊之鐵失其母弟妻子單騎還南昌【南昌漢舊縣豫章郡治所】大心遣構將兵追討之 宋子仙自吳郡趣錢塘劉神茂自吳興趣富陽前武州刺史富陽孫國恩以城降之【趣七喻翻降戶江翻下同】 十一月乙卯葬武皇帝於脩陵 【考異曰太清紀云十四日梓宫達于修陵今從梁書】廟號高祖 百濟遣使入貢【使疏吏翻】見城闕荒圮【圮部鄙翻毀也】異於曏來【毛晃曰昔來謂之曏來也】哭於端門【端門臺城正南之中門】侯景怒録送莊嚴寺【録拘也收也莊嚴寺近建康南郊壇】不聽出 壬戌宋子仙急攻錢塘戴僧逷降之 岳陽王詧使將軍薛暉攻廣平拔之獲杜岸送襄陽詧拔其舌鞭其面支解而烹之又發其祖父墓焚其骸而揚之以其頭為漆椀詧既與湘東王繹為敵恐不能自存遣使求援於魏請為附庸【鄭康成曰附庸以國事附於大國也】丞相泰令東閤祭酒榮權使於襄陽繹使司州刺史柳仲禮鎮竟陵以圖詧詧懼遣其妃王氏及世子嶚為質于魏【嶚力么翻質音致】丞相泰欲經略江漢以開府儀同三司楊忠都督三荆等十五州諸軍事鎮穰城仲禮至安陸安陸太守柳勰以城降之【勰音協】仲禮留長史馬岫與其弟子禮守之帥衆一萬趣襄陽泰遣楊忠及行臺僕射長孫儉將兵擊仲禮以救詧【為西魏因詧而并繹張本】 宋子仙乘勝度浙江至會稽【會工外翻】邵陵王綸聞錢塘已敗出奔鄱陽 【考異曰南史云東土皆附綸臨城公大連懼將害己乃圖之綸覺之乃去今從典略】鄱陽内史開建侯蕃以兵拒之【五代志開建縣屬熙平郡隋以熙平郡為連州】範進擊蕃破之【範當作綸】魏楊忠將至義陽太守馬伯符以下溠城降之【五代志漢東郡唐城縣後魏曰㵐西置義陽郡西魏改㵐西為下溠杜佑曰下溠戌在漢東郡棗陽縣東南百餘里九域志唐城在随州西北八十五里左傳楚人除道梁溠營軍臨随即此溠水溠音側駕翻字林壮加翻】忠以伯符為鄉導【鄉讀曰嚮】伯符岫之子也 南郡王大連為東揚州刺史時會稽豐沃勝兵數萬【會工外翻勝音升】糧仗山積東土人懲侯景殘虐咸樂為用【樂音洛】而大連朝夕酣飲不恤軍事司馬東陽留異凶狡殘暴為衆所患大連悉以軍事委之十二月庚寅宋子仙攻會稽大連弃城走異奔還鄉里尋以其衆降於子仙大連欲奔鄱陽異為子仙鄉導追及大連於信安【漢獻帝初平二年分太末立新安縣晉武帝太康元年更名信安五代志東陽郡信安縣有江山即今江山縣也宋白曰信安縣漢太末縣地漢末為新安晉為信安唐為衢州治所唐又分信安之南川為江山縣 考異曰典略云十二月庚子朔擒大連按是月壬午朔今從太清紀】執送建康帝聞之引帷自蔽掩袂而泣【不忍見其子之俘執也】於是三吳盡沒於景公侯在會稽者俱南度嶺景以留異為東陽太守【為後留異據東陽張本】收其妻子為質【質音致】 乙酉東魏以并州刺史彭樂為司徒 邵陵王綸進至九江尋陽王大心以江州讓之綸不受引兵西上【上時掌翻】始興太守陳霸先結郡中豪傑欲討侯景郡人侯安都張偲等各帥衆千餘人歸之【偲新兹翻又倉才翻】霸先遣主帥杜僧明將二千人頓于嶺上【大庾嶺也帥所類翻將即亮翻】廣州刺史蕭勃遣人止之曰侯景驍雄天下無敵【驍堅堯翻】前者援軍十萬士馬精彊猶不能克【謂柳仲禮等】君以區區之衆將何所之如聞嶺北王侯又皆鼎沸親尋干戈【謂荆雍湘三州】以君疎外詎可暗投【漢鄒陽曰明月之珠夜光之璧以暗投人於道路人無不按劎而相眄者無因而至前也】未若且留始興遥張聲勢保太山之安也霸先曰僕荷國恩往聞侯景渡江即欲赴援遭值元蘭梗我中道【荷下可翻元蘭謂元景仲及蘭裕也】今京都覆沒君辱臣死誰敢愛命君侯體則皇枝【蕭勃武帝從弟吳平侯昺之子故云然】任重方岳遣僕一軍猶賢乎己【猶賢乎已用孔子語已止也此言猶勝乎止而不遣軍也】乃更止之乎乃遣使間道詣江陵受湘東王繹節度【使疏吏翻間古莧翻】時南康土豪蔡路養起兵據郡勃乃以腹心譚世遠為曲江令【曲江縣漢屬桂陽郡吴屬始興郡唐為韶州】與路養相結同遏霸先【為陳霸先破蔡路養張本】 魏楊忠拔随郡執太守桓和【五代志漢東郡随縣舊置随郡】東魏使金門公潘樂等將兵五萬襲司州刺史夏侯強降之於是東魏盡有淮南之地【太清二年東魏使辛術略江淮之地至是方盡有淮南之地】<br />
<br />
  資治通鑑卷一百六十二  <br>
   </div> 

<script src="/search/ajaxskft.js"> </script>
 <div class="clear"></div>
<br>
<br>
 <!-- a.d-->

 <!--
<div class="info_share">
</div> 
-->
 <!--info_share--></div>   <!-- end info_content-->
  </div> <!-- end l-->

<div class="r">   <!--r-->



<div class="sidebar"  style="margin-bottom:2px;">

 
<div class="sidebar_title">工具类大全</div>
<div class="sidebar_info">
<strong><a href="http://www.guoxuedashi.com/lsditu/" target="_blank">历史地图</a></strong>  
<a href="http://www.880114.com/" target="_blank">英语宝典</a>  
<a href="http://www.guoxuedashi.com/13jing/" target="_blank">十三经检索</a> 
<br><strong><a href="http://www.guoxuedashi.com/gjtsjc/" target="_blank">古今图书集成</a></strong> 
<a href="http://www.guoxuedashi.com/duilian/" target="_blank">对联大全</a> <strong><a href="http://www.guoxuedashi.com/xiangxingzi/" target="_blank">象形文字典</a></strong> 

<br><a href="http://www.guoxuedashi.com/zixing/yanbian/">字形演变</a>  <strong><a href="http://www.guoxuemi.com/hafo/" target="_blank">哈佛燕京中文善本特藏</a></strong>
<br><strong><a href="http://www.guoxuedashi.com/csfz/" target="_blank">丛书&方志检索器</a></strong> <a href="http://www.guoxuedashi.com/yqjyy/" target="_blank">一切经音义</a>  

<br><strong><a href="http://www.guoxuedashi.com/jiapu/" target="_blank">家谱族谱查询</a></strong>  <strong><a href="http://shufa.guoxuedashi.com/sfzitie/" target="_blank">书法字帖欣赏</a></strong> 
<br>

</div>
</div>


<div class="sidebar" style="margin-bottom:0px;">

<font style="font-size:22px;line-height:32px">QQ交流群9:489193090</font>


<div class="sidebar_title">手机APP 扫描或点击</div>
<div class="sidebar_info">
<table>
<tr>
	<td width=160><a href="http://m.guoxuedashi.com/app/" target="_blank"><img src="/img/gxds-sj.png" width="140"  border="0" alt="国学大师手机版"></a></td>
	<td>
<a href="http://www.guoxuedashi.com/download/" target="_blank">app软件下载专区</a><br>
<a href="http://www.guoxuedashi.com/download/gxds.php" target="_blank">《国学大师》下载</a><br>
<a href="http://www.guoxuedashi.com/download/kxzd.php" target="_blank">《汉字宝典》下载</a><br>
<a href="http://www.guoxuedashi.com/download/scqbd.php" target="_blank">《诗词曲宝典》下载</a><br>
<a href="http://www.guoxuedashi.com/SiKuQuanShu/skqs.php" target="_blank">《四库全书》下载</a><br>
</td>
</tr>
</table>

</div>
</div>


<div class="sidebar2">
<center>


</center>
</div>

<div class="sidebar"  style="margin-bottom:2px;">
<div class="sidebar_title">网站使用教程</div>
<div class="sidebar_info">
<a href="http://www.guoxuedashi.com/help/gjsearch.php" target="_blank">如何在国学大师网下载古籍?</a><br>
<a href="http://www.guoxuedashi.com/zidian/bujian/bjjc.php" target="_blank">如何使用部件查字法快速查字?</a><br>
<a href="http://www.guoxuedashi.com/search/sjc.php" target="_blank">如何在指定的书籍中全文检索?</a><br>
<a href="http://www.guoxuedashi.com/search/skjc.php" target="_blank">如何找到一句话在《四库全书》哪一页?</a><br>
</div>
</div>


<div class="sidebar">
<div class="sidebar_title">热门书籍</div>
<div class="sidebar_info">
<a href="/so.php?sokey=%E8%B5%84%E6%B2%BB%E9%80%9A%E9%89%B4&kt=1">资治通鉴</a> <a href="/24shi/"><strong>二十四史</strong></a>&nbsp; <a href="/a2694/">野史</a>&nbsp; <a href="/SiKuQuanShu/"><strong>四库全书</strong></a>&nbsp;<a href="http://www.guoxuedashi.com/SiKuQuanShu/fanti/">繁体</a>
<br><a href="/so.php?sokey=%E7%BA%A2%E6%A5%BC%E6%A2%A6&kt=1">红楼梦</a> <a href="/a/1858x/">三国演义</a> <a href="/a/1038k/">水浒传</a> <a href="/a/1046t/">西游记</a> <a href="/a/1914o/">封神演义</a>
<br>
<a href="http://www.guoxuedashi.com/so.php?sokeygx=%E4%B8%87%E6%9C%89%E6%96%87%E5%BA%93&submit=&kt=1">万有文库</a> <a href="/a/780t/">古文观止</a> <a href="/a/1024l/">文心雕龙</a> <a href="/a/1704n/">全唐诗</a> <a href="/a/1705h/">全宋词</a>
<br><a href="http://www.guoxuedashi.com/so.php?sokeygx=%E7%99%BE%E8%A1%B2%E6%9C%AC%E4%BA%8C%E5%8D%81%E5%9B%9B%E5%8F%B2&submit=&kt=1"><strong>百衲本二十四史</strong></a>  <a href="http://www.guoxuedashi.com/so.php?sokeygx=%E5%8F%A4%E4%BB%8A%E5%9B%BE%E4%B9%A6%E9%9B%86%E6%88%90&submit=&kt=1"><strong>古今图书集成</strong></a>
<br>

<a href="http://www.guoxuedashi.com/so.php?sokeygx=%E4%B8%9B%E4%B9%A6%E9%9B%86%E6%88%90&submit=&kt=1">丛书集成</a> 
<a href="http://www.guoxuedashi.com/so.php?sokeygx=%E5%9B%9B%E9%83%A8%E4%B8%9B%E5%88%8A&submit=&kt=1"><strong>四部丛刊</strong></a>  
<a href="http://www.guoxuedashi.com/so.php?sokeygx=%E8%AF%B4%E6%96%87%E8%A7%A3%E5%AD%97&submit=&kt=1">說文解字</a> <a href="http://www.guoxuedashi.com/so.php?sokeygx=%E5%85%A8%E4%B8%8A%E5%8F%A4&submit=&kt=1">三国六朝文</a>
<br><a href="http://www.guoxuedashi.com/so.php?sokeytm=%E6%97%A5%E6%9C%AC%E5%86%85%E9%98%81%E6%96%87%E5%BA%93&submit=&kt=1"><strong>日本内阁文库</strong></a> <a href="http://www.guoxuedashi.com/so.php?sokeytm=%E5%9B%BD%E5%9B%BE%E6%96%B9%E5%BF%97%E5%90%88%E9%9B%86&ka=100&submit=">国图方志合集</a> <a href="http://www.guoxuedashi.com/so.php?sokeytm=%E5%90%84%E5%9C%B0%E6%96%B9%E5%BF%97&submit=&kt=1"><strong>各地方志</strong></a>

</div>
</div>


<div class="sidebar2">
<center>

</center>
</div>
<div class="sidebar greenbar">
<div class="sidebar_title green">四库全书</div>
<div class="sidebar_info">

《四库全书》是中国古代最大的丛书,编撰于乾隆年间,由纪昀等360多位高官、学者编撰,3800多人抄写,费时十三年编成。丛书分经、史、子、集四部,故名四库。共有3500多种书,7.9万卷,3.6万册,约8亿字,基本上囊括了古代所有图书,故称“全书”。<a href="http://www.guoxuedashi.com/SiKuQuanShu/">详细>>
</a>

</div> 
</div>

</div>  <!--end r-->

</div>
<!-- 内容区END --> 

<!-- 页脚开始 -->
<div class="shh">

</div>

<div class="w1180" style="margin-top:8px;">
<center><script src="http://www.guoxuedashi.com/img/plus.php?id=3"></script></center>
</div>
<div class="w1180 foot">
<a href="/b/thanks.php">特别致谢</a> | <a href="javascript:window.external.AddFavorite(document.location.href,document.title);">收藏本站</a> | <a href="#">欢迎投稿</a> | <a href="http://www.guoxuedashi.com/forum/">意见建议</a> | <a href="http://www.guoxuemi.com/">国学迷</a> | <a href="http://www.shuowen.net/">说文网</a><script language="javascript" type="text/javascript" src="https://js.users.51.la/17753172.js"></script><br />
  Copyright &copy; 国学大师 古典图书集成 All Rights Reserved.<br>
  
  <span style="font-size:14px">免责声明:本站非营利性站点,以方便网友为主,仅供学习研究。<br>内容由热心网友提供和网上收集,不保留版权。若侵犯了您的权益,来信即刪。scp168@qq.com</span>
  <br />
ICP证:<a href="http://www.beian.miit.gov.cn/" target="_blank">鲁ICP备19060063号</a></div>
<!-- 页脚END --> 
<script src="http://www.guoxuedashi.com/img/plus.php?id=22"></script>
<script src="http://www.guoxuedashi.com/img/tongji.js"></script>

</body>
</html>
