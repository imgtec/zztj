資治通鑑卷二十一
宋 司馬光 撰

胡三省 音註

漢紀十三|{
	起玄黓涒灘盡玄黓敦牂凡十一年}


世宗孝武皇帝下之上

元封二年冬十月上行幸雍祠五畤還祝祠泰一以拜德星|{
	師古曰拜而祠之加祝辭}
春正月公孫卿言見神人東莱山若云欲見天子天子於是幸緱氏城|{
	緱工侯翻}
拜卿為中大夫遂至東莱宿留之數日無所見|{
	宿留音秀溜}
見大人迹云復遣方士求神怪採芝藥以千數|{
	復扶又翻}
時歲旱天子既出無名乃禱萬里沙|{
	應劭曰萬里沙神祠也在東莱曲城孟康曰沙徑三百餘里杜佑通典萬里沙在莱州掖縣界}
夏四月還過祠泰山 初河决瓠子|{
	河始决見十八卷元光二年}
後二十餘歲不復塞|{
	復扶又翻塞悉則翻下同}
梁楚之地尤被其害|{
	被皮義翻}
是歲上使汲仁郭昌二卿發卒數萬人塞瓠子河决天子自泰山還自臨决河沈白馬玉璧于河|{
	沈持林翻}
令羣臣從官自將軍以下皆負薪卒填决河|{
	從才用翻卒子恤翻}
築宮其上名曰宣防宮導河北行二渠復禹舊迹|{
	溝洫志禹導河自積石歷龍門南到華隂東下底柱及孟津洛汭至于大伾於是禹以為河所從來者高水湍悍難以行平地數為敗乃釃二渠以引其河北載之高地過洚水至於大陸播為九河同為迎河入渤海孟康曰二渠其一出貝丘西南南折者也其一則漯川也河自王莽時遂空惟用漯耳釃山支翻漯吐合翻}
而梁楚之地復寧無水災 上還長安 初令越巫祠上帝百鬼而用雞卜|{
	越俗用雞卜李奇曰持雞奇卜如鼠卜史記正義曰雞卜法用雞一狗一生祝願訖即殺雞狗煮熟又祭獨取雞兩眼骨上自有孔裂似人物形則吉不足則凶今嶺南猶行此法范成大桂海虞衡志雞卜南人占法以雄雞雛執其兩足焚香禱所占撲雞殺之拔兩股骨淨洗線束之以竹筳挿束處使兩骨相背於筳端執竹再祝左骨為儂儂我也右骨為人人所占事也視兩骨之側所有細竅以細竹筳長寸餘偏挿之斜直偏正各隨竅之自然以定吉凶法有十八變大扺直而正或近骨者多吉曲而斜或遠骨者多凶亦有用雞卵卜者握卵以卜書墨於殻記其四維煮熟横截視當墨處辨殻中白之厚薄以定儂人吉凶}
公孫卿言僊人好樓居|{
	好呼到翻}
於是上令長安作蜚廉桂觀甘泉作益夀延夀觀|{
	應劭曰蜚廉神禽名能致風氣晉灼曰身似鹿頭如爵有角而蛇尾文如豹文桂觀漢志作桂舘師古曰蜚廉桂舘益壽延壽四舘名觀古玩翻}
使卿持節設具而候神人又作通天莖臺|{
	通天臺在甘泉宮漢舊儀曰臺高五十丈去長安二百里望見長安城}
置祠具其下更置甘泉前殿益廣諸宮室 初全燕之世嘗畧屬真番朝鮮|{
	徐廣曰遼東有番汙縣應劭曰玄菟本真番國番普安翻張晏曰朝鮮有濕水冽水汕水三水合為洌水疑樂浪朝鮮取名於此括地志高麗都平壤城本樂浪郡王險城又古云朝鮮索應曰案朝音潮直驕翻鮮音仙以有汕水故也汕一音訕}
為置吏築障塞|{
	為于偽翻下同}
秦滅燕屬遼東外徼|{
	徼古弔翻}
漢興為其遠難守復修遼東故塞至浿水為界|{
	班志浿水出遼東塞外西南至樂浪縣西入海水經浿水出樂浪鏤方縣東南過臨浿縣東入海酈道元註曰滿自浿水而至朝鮮若浿水東流無渡浿之理余訪蕃使言城在浿水之陽其水西流逕樂浪郡朝鮮縣故志曰浿水西至增地縣入海經誤浿普盖翻又滂沛翻普大翻杜佑曰浿滂拜翻}
屬燕燕王盧綰反入匈奴|{
	見十二卷高祖十三年}
燕人衛滿亡命聚黨千餘人椎髻蠻夷服而東走出塞渡浿水居秦故空地上下障稍役屬真番朝鮮蠻夷及燕亡命者王之|{
	王于况翻}
都王險|{
	韋昭曰王險故邑名應劭曰遼東有險瀆縣即滿所都因水險故曰險瀆臣瓚曰王險在樂浪郡浿水之東師古曰瓚說是貢曰即平壤城}
會孝惠高后時天下初定遼東太守即約滿為外臣保塞外蠻夷無使盜邊諸蠻夷君欲入見天子勿得禁止|{
	見賢遍翻下同}
以故滿得以兵威財物侵降其旁小邑真番臨屯皆來服屬|{
	臨屯帝後開為郡註見下三年降戶江翻}
方數千里傳子至孫右渠所誘漢亡人滋多又未嘗入見|{
	誘音酉見賢遍翻下同}
辰國欲上書見天子又雍閼不通|{
	師古曰辰國即辰韓之國雍讀曰壅閼一曷翻}
是歲漢使涉何誘諭|{
	涉姓也左傳晉有大夫涉佗}
右渠終不肯奉詔何去至界上臨浿水使御刺殺送何者朝鮮裨王長|{
	刺七亦翻}
即渡馳入塞遂歸報天子曰殺朝鮮將上為其名美|{
	將即亮翻為于偽翻下同}
即不詰拜何為遼東東部都尉|{
	遼東東部都尉治武次縣}
朝鮮怨何發兵襲攻殺何 六月甘泉房中產芝九莖|{
	時芝產於甘泉齋房九莖連葉論衡芝生於土土氣和則芝艸生瑞命記王者慈仁則芝草生}
上為之赦天下 上以旱為憂公孫卿曰黄帝時封則天旱乾封三年上乃下詔曰天旱意乾封乎|{
	乾音干}
秋作明堂於汶上|{
	班志泰山郡莱蕪縣禹貢汶水出西南入濟桑欽所言又曰琅邪郡朱虛縣東泰山汶水所出東至安丘入濰有五帝祠師古曰前言汶水出莱蕪入濟此又言出朱虚入濰將桑欽所言有異或者有二汶水乎予據班志明堂在泰山奉高縣西南四里又禹貢浮于汶達于濟此明堂當在濟之汶上琅邪之汶入于濰而濰入于海其地僻遠非立明堂處汶音問}
上募天下死罪為兵遣樓船將軍楊僕從齊浮渤海|{
	僕從齊浮渤海蓋自青莱以北幽平以南皆濱于海其海通謂之渤海非指渤海郡而言也}
左將軍荀彘出遼東以討朝鮮 初上使王然于以越破及誅南夷兵威喻滇王入朝滇王者其衆數萬人其旁東北有勞深靡莫皆同姓相仗未肯聽|{
	仗直亮翻}
勞深靡莫數侵犯使者吏卒|{
	數所角翻}
於是上遣將軍郭昌中郎將衛廣發巴蜀兵擊滅勞深靡莫以兵臨滇滇王舉國降請置吏入朝於是以為益州郡|{
	續漢志益州郡去雒陽五千六百里魏晉為南中寧州之地唐為昆州姚州之地後沒于南詔師古曰唐南寧州昆州裒州也降戶江翻朝直遥翻}
賜滇王王印復長其民|{
	復扶又翻又如字長丁丈翻}
是時漢滅兩越平西南夷置初郡十七|{
	臣瓚曰元鼎六年定南越地以為南海欝林蒼梧合浦九真日南交趾珠厓儋耳郡定西南夷以為武都牂牁越巂沈黎汶山郡及地理志西南夷傳所置犍為零陵益州郡凡十七}
且以其故俗治毋賦税南陽漢中以往郡各以地比給初郡吏卒奉食幣物傳車馬被具|{
	師古曰地比謂依其次第自近及遠比頻寐翻奉扶用翻傳張戀翻被皮義翻}
而初郡時時小反殺吏漢發南方吏卒往誅之間歲萬餘人費皆仰給大農大農以均輸調鹽鐵助賦故能贍之然兵所過縣為以訾給毋乏而已|{
	訾讀曰資}
不敢言擅賦法矣|{
	帝初擊胡大司農賦税專以奉戰士故有擅賦之法}
是歲以御史中丞南陽杜周為廷尉|{
	姓譜杜本陶唐氏劉累之後在周為唐杜氏有杜伯}
周外寛内深次骨|{
	李奇曰其用法深刻至骨}
其治大放張湯|{
	言大抵依放張湯也放甫往翻}
時詔獄益多二千石繫者新故相因不減百餘人廷尉一歲至千餘章|{
	章者諸獄告劾之書上之廷尉者也}
章大者連逮證案數百小者數十人遠者數千近者數百里會獄|{
	師古曰往赴對也}
廷尉及中都官詔獄逮至六七萬人|{
	師古曰中都官凡京師諸官府也獄辭所及進考問者六七萬人也}
吏所增加十萬餘人|{
	師古曰吏又於此外以文致之更增也}


三年冬十二月雷雨雹大如馬頭|{
	雨于具翻}
上遣將軍趙破奴擊車師破奴與輕騎七百餘先至虜樓蘭王遂破車師因舉兵威以困烏孫大宛之屬|{
	宛於元翻}
春正月甲申封破奴為浞野侯王恢佐破奴擊樓蘭封恢為浩侯|{
	從票侯趙破奴元鼎五年坐酎金失侯今以功復封浞野侯浞野侯浩侯功臣表不書所食邑浞士角翻}
於是酒泉列亭障至玉門矣 初作角抵戲魚龍曼延之屬|{
	文頴曰名此樂為角扺兩兩相當角力角技藝射御蓋襍技樂也師古曰魚龍者為舍利之獸先戲於庭極畢乃入殿前化成比目魚跳躍漱水作霧障曰畢化成黄龍八丈散戲於庭炫耀日光西京賦云海鱗變而成龍即謂此也曼延即西京賦所謂巨獸百尋是為曼延者也延弋戰翻}
漢兵入朝鮮境朝鮮王右渠發兵距險樓船將軍將齊兵七千人先至王險右渠城守窺知樓船軍少|{
	守式又翻少詩沼翻}
即出城擊樓船樓船軍敗散遁山中十餘日稍求退散卒復聚左將軍擊朝鮮浿水西軍未能破天子為兩將未有利|{
	為于偽翻}
乃使衛山因兵威往諭右渠右渠見使者頓首謝願降恐兩將詐殺臣今見信節請復降|{
	復扶又翻降戶江翻下同}
遣太子入謝獻馬五千匹及饋軍糧人衆萬餘持兵方渡浿水使者及左將軍疑其為變謂太子已服降宜令人毋持兵太子亦疑使者左將軍詐殺之遂不渡浿水復引歸山還報天子天子誅山左將軍破浿水上軍乃前至城下圍其西北樓船亦往會居城南右渠遂堅守城數月未能下左將軍所將燕代卒多勁悍樓船將齊卒已嘗敗亡困辱卒皆恐將心慙|{
	將即亮翻悍下罕翻又侯旰翻}
其圍右渠嘗持和節左將軍急擊之朝鮮大臣乃隂間使人私約降樓船|{
	隂暗密也間空隙也言暗密遣使投空隙而出與樓船約降間古莧翻}
往來言尚未肯决左將軍數與樓船期戰|{
	數所角翻下同}
樓船欲就其約不會左將軍亦使人求間隙降下朝鮮朝鮮不肯心附樓船以故兩將不相能左將軍心意樓船前有失軍罪|{
	意疑也億度也料也}
今與朝鮮私善而又不降疑其有反計未敢發天子以兩將圍城乖異兵久不决使濟南太守公孫遂往正之|{
	濟子禮翻考異曰史記作征之盖字誤今從漢書}
有便宜得以從事遂至左將軍曰朝鮮當下久矣不下者樓船數期不會具以素所意告曰今如此不取恐為大害遂亦以為然乃以節召樓船將軍入左將軍營計事即命左將軍麾下執樓船將軍并其軍以報天子天子誅遂 |{
	考異曰漢書作許遂按左將軍亦以爭功相嫉乖計棄市則武帝必以遂執樓船為非漢書作許盖字誤今從史記}
左將軍已并兩軍即急擊朝鮮朝鮮相路人相韓隂 |{
	考異曰漢書隂作陶今從史記}
尼谿相參將軍王唊|{
	應劭曰凡五人戎狄不知官紀故皆稱相師古曰相路人一也相韓陶二也尼谿相參三也將軍王唊四也應氏乃云五人失之矣不當尋下文乎余據韓陶今作韓隂盖從史記相息量翻唊音頰}
相與謀曰始欲降樓船樓船今執獨左將軍并將|{
	將即亮翻}
戰益急恐不能與戰王又不肯降隂唊路人皆亡降漢路人道死夏尼谿參使人殺朝鮮王右渠來降王險城未下故右渠之大臣成已又反復攻吏|{
	復扶又翻}
左將軍使右渠子長降相路人之子最|{
	師古曰右渠之子名長路人先已降漢而死于道故謂之降相最者其子名}
告諭其民誅成已以故遂定朝鮮為樂浪臨屯玄菟真番四郡|{
	樂浪郡治朝鮮縣蓋以右渠所都為治所也臣瓚曰茂陵書臨屯郡治東暆縣去長安六千一百三十八里領十五縣玄菟郡本高句驪也既平朝鮮并開為郡治沃沮城後為夷貊所侵徙郡句驪西北真番郡治霅縣去長安七千六百四十里領十五縣余據後廢臨屯真番二郡班志東暆縣屬樂浪霅縣無所攷樂音洛浪音狼}
封參為澅清侯|{
	功臣表澅清侯食邑於齊澅音獲又戶卦翻}
隂為荻苴侯|{
	班書功臣表作荻苴侯食邑於勃海此從史記作荻音狄苴子餘翻}
唊為平州侯|{
	功臣表平州侯食邑於泰山梁父縣}
長為幾侯|{
	功臣表作幾侯張洛食邑於河東}
最以父死頗有功為湼陽侯|{
	湼陽縣屬南陽郡湼乃結翻}
左將軍徵至坐爭功相嫉乖計棄市樓船將軍亦坐兵至列口|{
	班志列口縣屬樂浪郡郭璞曰山海經列水在遼東余謂其地當冽水入海之口}
當待左將軍擅先縱失亡多當誅贖為庶人

班固曰玄菟樂浪本箕子所封|{
	武王封箕子於朝鮮}
昔箕子居朝鮮教其民以禮義田蠶織作為民設禁八條|{
	為于偽翻}
相殺以當時償殺相傷以穀償相盜者男沒入為其家奴女為婢欲自贖者人五十萬雖免為民俗猶羞之嫁娶無所售是以其民終不相盜無門戶之閉婦人貞信不淫辟|{
	辟讀曰僻}
其田野飲食以籩豆都邑頗放效吏往往以杯器食|{
	於用往翻}
郡初取吏於遼東吏見民無閉臧|{
	臧讀曰藏}
及賈人往者|{
	賈音古}
夜則為盜俗稍益薄今於犯禁寖多至六十餘條可貴哉仁賢之化也然東夷天性柔順異於三方之外故孔子悼道不行設浮桴于海欲居九夷|{
	並見論語桴編竹木為之大者曰筏小者曰桴桴芳無翻}
有以也夫

秋七月膠西于王端薨|{
	端景帝子三年受封諡法能優其德曰于 考異曰荀紀端皆作瑞今從漢書}
武都氐反分徙酒泉|{
	杜佑曰氐者西戎别種}


四年冬十月上行幸雍祠五畤|{
	雍於用翻畤音止}
通回中道遂北出蕭關|{
	師古曰冋中在安定高平有險阻蕭關在其北此盖自回中通道以出蕭關}
歷獨鹿鳴澤|{
	服䖍曰獨鹿山名鳴澤澤名皆在涿郡遒縣北界水經註澤渚方十五里}
自代而還幸河東春三月祠后土赦汾隂夏陽中都死罪以下夏大旱 匈奴自衛霍度幕以來|{
	度幕見十九卷元狩四年}
希復

為寇|{
	復扶又翻下同}
遠徙北方休養士馬習射獵數使使於漢|{
	數色角翻使使下疏吏翻}
好辭甘言求請和親漢使北地人王烏等窺匈奴烏從其俗去節入穹廬|{
	去羌呂翻師古曰穹廬氈帳也索隱曰盖以氈為廬崇穹然而宋均曰穹獸名亦異說也}
單于愛之佯許甘言為遣其太子入漢為質|{
	質音致}
漢使楊信於匈奴信不肯從其俗單于曰故約漢嘗遣翁主給繒絮食物有品以和親|{
	師古曰品謂等差也}
而匈奴亦不擾邊今乃欲反古|{
	師古曰反違也}
令吾太子為質無幾矣|{
	師古曰言遣太子為質則匈奴國中所餘者無幾皆當盡也余謂匈奴盖自謂本與漢為鄰敵之國今乃令以太子為質是其國勢削弱所餘無幾也幾居豈翻}
信既歸漢又使王烏往而單于復讇以甘言|{
	師古曰讇古諂字}
欲多得漢財物紿謂王烏曰吾欲入漢見天子面相約為兄弟王烏歸報漢漢為單于築邸于長安|{
	漢為于偽翻}
匈奴曰非得漢貴人使吾不與誠語|{
	師古曰誠實也}
匈奴使其貴人至漢病漢予藥欲愈之|{
	予讀曰與}
不幸而死漢使路充國佩二千石印綬往使因送其喪厚葬直數千金曰此漢貴人也單于以為漢殺吾貴使者乃留路充國不歸諸所言者單于特空紿王烏|{
	師古曰特但也}
殊無意入漢及遣太子於是匈奴數使奇兵侵犯漢邊|{
	數所角翻}
乃拜郭昌為拔胡將軍及浞野侯屯朔方以東備胡

五年冬上南巡狩至于盛唐|{
	文頴曰案地理志不得疑當在廬江左右縣名韋昭曰在南郡余據唐地理志夀州有盛唐縣盖以古地名名縣宋白曰夀州六安縣楚之潛也在漢為盛唐縣西十五里有盛唐山}
望祀虞舜于九疑|{
	地理志九疑山在陵零營道縣南亦名蒼梧山九峯相似望而疑之故曰九疑相傳舜死于蒼梧因葬焉故望祀之}
登灊天柱山|{
	班志灊縣屬廬江郡天柱山在南帝以為南嶽灊音潜唐之舒州}
自尋陽浮江|{
	班志尋陽縣屬廬江郡禹貢九江在南皆東合為大江沈約曰尋陽因水名縣水南注江余據漢尋陽縣在大江北自晉立尋陽郡於江南之柴桑而江北尋陽之名遂晦杜佑曰漢舊尋陽縣在江北今蘄春郡界晉温嶠移於江南}
親射蛟江中獲之|{
	師古曰蛟龍屬也郭璞說其狀云似蛇而有脚細頸有白嬰大者數圍卵生子如二石大甕能吞人射而亦翻}
舳艫千里薄樅陽而出|{
	李斐曰舳船後持柁處艫船前刺櫂處言其船多前後相銜千里不絶也舳音逐艫音盧班志樅陽縣屬廬江郡宋白曰舒州桐城縣漢為樅陽縣梁置樅陽郡師古曰樅千容翻}
遂北至琅邪|{
	琅邪郡秦置唐為沂州其餘地入海莱密州界}
並海|{
	並步浪翻}
所過禮祠其名山大川春三月還至太山增封甲子始祀上帝於明堂配以高祖因朝諸侯王列侯受郡國計|{
	師古曰計若今之諸州計簿也朝直遥翻}
夏四月赦天下所幸縣毋出今年租賦還幸甘泉郊泰畤 長平烈侯衛青薨|{
	考異曰漢武故事大將軍四子皆不才皇后每因太子涕泣請上削其封上曰吾自知之不令皇后憂也少子竟坐奢淫誅上遣謝后通削諸子封爵各留千戶焉按青四子無坐奢淫誅者此說妄也}
起冢象廬山|{
	廬山盖即盧山揚雄所謂填盧山之壑者也師古曰盧山匈奴中山名衛青冢在茂陵東次霍去病冢之西相併者是也}
上既攘郤胡越開地斥境乃置交阯朔方之州及冀幽幷兖徐青揚荆豫益凉等州凡十三部皆置刺史焉|{
	續漢志秦有監郡御史監諸郡漢興省之但遣丞相史分刺諸州無常官孝武初置刺史十三人秩六百石古今注曰常以春分行部郡國各遣一吏迎界上漢舊儀曰詔書舊典刺史班宣周行郡國省察治政黜陟能否斷理寃獄以六條問事非條所問即不省一條強宗豪右田宅踰制以強陵弱以衆暴寡二條二千石不奉詔書遵承典制倍公向私㫄詔牟利侵漁百姓聚歛為姦三條二千石不恤疑獄風厲殺人怒則任刑喜則任賞煩擾苛暴剝戮黎元為百姓所疾山崩石裂妖祥訛言四條二千石選置不平苟阿所愛蔽賢寵頑五條二千石子弟怙恃榮勢請託所監六條二千石違公下比阿附豪強通行貨賂割損政令續漢志又曰諸州常以八月巡行郡國録囚徒考殿最初歲盡詣京都奏事中興但因計吏與古今注異據晉志帝改禹貢雍州曰凉州梁州曰益州又置徐州復禹貢舊號北置朔方南置交阯與荆揚兖豫青冀幽幷為十三州春秋元命包及晉書地理志昴畢散為冀州其地有險有易帝王所都亂則冀安弱則冀強荒則冀豐箕星散為幽州言北方太隂故以幽冥為號營室流為幷州不以衛水為號又不以恒山為稱而云并者盖以其在兩谷之間也五星流為兖州端也信也又云盖取沇水以名焉天氐流為徐州盖取舒緩之義或云因徐丘以立名虚危流為青州周禮曰正東曰青州盖取地居少陽其色青故名牽牛流為揚州以為江南之氣躁勁厥性輕揚亦云州界多水水波輕揚也軫星散為荆州強也言其氣躁強亦曰警也言南蠻數為寇逆其人有道後服無道先強常警備也又云取名於荆山鉤鈴星别為豫州豫者舒也言稟中和之氣性理安舒也參代流為益州益之言阨言其所在之地險阨亦曰疆壤益大故以名焉凉州以地處西方常寒凉也}
上以名臣文武欲盡乃下詔曰蓋有非常之功必待非常之人故馬或奔踶而致千里|{
	師古曰奔走也踶蹈也奔踶者乘之則奔立則踶人踶徒計翻}
士或有負俗之累而立功名|{
	晉灼曰負俗謂被世譏論也累力瑞翻}
夫泛駕之馬|{
	師古曰泛覆也與覂同言馬有逸氣者多能覆車泛方勇翻}
跅㢮之士|{
	如淳曰士行有卓異不入俗㢮而見斥逐者師古曰跅者跅落無檢局也㢮者放廢不遵法度也跅音跖㢮式爾翻}
亦在御之而已其令州郡察吏民有茂才異等|{
	應劭曰舊言秀才避光武諱稱茂才異等者超等軼羣不與几同也師古曰茂美也}
可為將相及使絶國者|{
	使疏吏翻}


六年冬上行幸回中 春作首山宮|{
	應劭曰首山在上郡於其下立宮廟也文穎曰在河東蒲反界師古註漢書曰尋此下詔文及依地理志文說是}
三月行幸河東祠后土赦汾隂殊死以下 漢既通西南夷開五郡|{
	五郡犍為越巂沈黎汶山益州}
欲地接以前通大夏歲遣使十餘輩出此初郡皆閉昆明|{
	杜佑曰昆明在越巂西南諸爨所居}
為所殺奪幣物於是天子赦京師亡命令從軍遣拔胡將軍郭昌將以擊之斬首數十萬後復遣使竟不得通|{
	將即亮翻復扶又翻}
秋大旱蝗 烏孫使者見漢廣大歸報其國|{
	元鼎二年烏孫遣使隨張騫入謝天子}
其國乃益重漢匈奴聞烏孫與漢通怒欲擊之又其㫄大宛月氏之屬皆事漢烏孫於是恐使使願得尚漢公主為昆弟天子與羣臣議許之烏孫以千匹馬往聘漢女漢以江都王建女細君為公主往妻烏孫|{
	江都王建易王非之子妻七細翻下同}
贈送甚盛烏孫王昆莫以為右夫人匈奴亦遣女妻昆莫以為左夫人公主自治宮室居|{
	治直之翻}
歲時一再與昆莫會置酒飲食昆莫年老言語不通公主悲愁思歸天子聞而憐之間歲遣使者以帷帳錦繡給遺焉|{
	師古曰間歲者謂每隔一歲而往也間古莧翻遺于季翻}
昆莫曰我老欲使其孫岑娶尚公主|{
	史記作岑娶漢書作岑陬師古曰岑士林翻陬子侯翻余據漢書岑陬者其官名也本名軍須靡}
公主不聽上書言狀天子報曰從其國俗欲與烏孫共滅胡岑娶遂妻公主昆莫死岑娶代立為昆彌|{
	烏孫建國之王曰昆莫班史云昆莫王號也名獵驕靡後書昆彌云顔注云昆莫本是王號而其人名獵驕靡故書云昆彌昆取昆莫彌取驕靡彌靡音有輕重耳盖本一也後遂以昆彌為王號滅綿結翻}
是時漢使西踰葱嶺抵安息安息發使以大鳥卵及黎軒善胘人獻于漢|{
	應劭曰大鳥卵如一二石甕師古曰如汲水甕無一二石也郭義恭廣志曰大爵頸及身膺蹄都似槖駝舉頭高七八尺張翅丈餘食大麥其卵如甕即今之駞鳥也黎軒亦曰黎靬東漢為大秦國唐為拂菻國在安息烏弋之西隔大海眩與幻同即今吞刀吐火植瓜種樹屠人截鳥之術皆是魚豢魏畧曰大秦國俗善幻口中出火自縛自解跳十二丸巧妙非常靬音軒又鉅連翻}
及諸小國驩濳大益車師扜罙蘇䪥之屬|{
	據史記驩濳大益在大宛西扜罙國治杆罙城去長安九千二百八十里西通于寘三百九十里後漢曰寧罙蘇䪥康居小王國治蘇䪥城去陽關凡八千二十五里扜音烏罙與冞同䪥下戒翻}
皆隨漢使獻見天子|{
	見賢遍翻}
天子大悦西國使更來更去|{
	師古曰逓互來去前後不絶更工衡翻}
天子每巡狩海上悉從外國客大都多人則過之散財帛以賞賜厚具以饒給之以覽示漢富厚焉|{
	師古曰言示之令其觀覽}
大角抵出奇戲諸怪物多聚觀者|{
	師古曰聚都邑人令觀看以誇示之觀工喚翻下同}
行賞賜酒池肉林令外國客徧觀名倉庫府藏之積見漢之廣大傾駭之|{
	師古曰見顯示也藏徂浪翻}
大宛左右多蒲萄可以為酒多苜蓿|{
	苜蓿草名苜音目蓿音宿}
天馬嗜之漢使采其實以來天子種之於離宮别觀㫄極望然西域以近匈奴常畏匈奴使待之過於漢使焉|{
	近其靳翻}
是歲匈奴烏維單于死子烏師廬立年少號兒單于自此之後單于益西北徙左方兵直雲中右方兵直酒泉敦煌郡|{
	匈奴左方兵本直上谷以東右方兵直上郡以西單于庭直代雲中今徙去而西故改左右方亦徙}


太初元年|{
	應劭曰初用夏正以正月為歲首故改元為太初}
冬十月上行幸泰山十一月甲子朔旦冬至祠上帝于明堂東至海上考入海及方士求神者莫驗然益遣冀遇之 乙酉栢梁臺災|{
	天火曰災人火曰火}
十二月甲午朔上親禪高里|{
	伏儼曰高里山名在泰山下師古曰此高字自作高下之高而死人之里謂之蒿里字即為蓬蒿之蒿或呼為下里者也或者既見泰山神靈之府高里山又在其旁即誤以高里為蒿里混同一事陸士衡尚不免况餘人乎今流俗漢書本有作蒿字者妄增耳}
祠后土臨渤海將以望祀蓬莱之屬冀至殊廷焉|{
	師古曰蓬莱僊人之庭也}
春上還以栢梁災故朝諸侯受計於甘泉|{
	師古曰受郡國所上計簿也朝直遥翻}
甘泉作諸侯邸越人勇之曰越俗有火災復起屋必以大用勝服之於是作建章宮|{
	師古曰建章宮在未央宮西俗所呼貞女樓即建章之闕余據戾大子傳建章宮在長安城西周回二十里上林之地也括地志建章宮在雍州長安縣西二十里長安故城西}
度為千門萬戶|{
	度大各翻}
其東則鳳闕|{
	三輔黄圖曰鳳闕高二十五丈關中記曰一名别風闕以言别四方之風西京賦閶闔之内别風嶕嶢是也三輔舊事曰北有圜闕高二十丈上有銅鳳凰故曰鳳闕也}
高二十餘丈|{
	高居豪翻}
其西則唐中數十里虎圈|{
	西都賦前唐中而後太液索隱曰如淳云中唐冇甓鄭玄註唐堂庭也爾雅以廟中路謂之唐西京賦前開唐中彌望廣是也毛氏詩傳曰唐堂塗也正義曰唐是門内之路釋宮云廟中路謂之唐堂塗謂之陳班史作商中師古曰商金也於序在秋如淳謂商中為商庭盖以西方之庭也數十里言廣於菟亦西方之獸故於此置圈圈求遠翻}
其北治大池漸臺高二十餘丈命曰太液池|{
	治直之翻漸臺在太液池中師古曰為水所漸漬故曰漸臺漸子廉翻臣瓚曰太液池言承隂陽津液以作池也師古曰太液池者言其津潤所及廣也}
中有蓬莱方丈瀛洲壺梁象海中神山龜魚之屬|{
	三輔故事池北面有石魚長三丈高五尺南岸有石龞三枚長六尺}
其南有玉堂璧門大鳥之屬|{
	漢武故事玉堂基與未央前殿等去地十二丈黄圖曰璧門薄以璧玉因曰璧門大鳥立條支所產大鳥之象}
立神明臺井幹樓度五十丈|{
	漢宮閣疏神明臺高五十丈上有九室置九天道士百人然則神明井幹俱高五十丈也井幹樓積木而高為樓若井幹之形也井幹者井上木欄也其形或四角或八角西京賦井幹疊而百層即此樓也}
輦道相屬焉|{
	屬之欲翻}
大中大夫公孫卿壺遂|{
	姓譜晉大夫受邑壺口其後以為氏}
太史令司馬遷等言歷紀壞廢|{
	箕子叙大法九章而五紀明歷法故自古以來創業改制咸正歷紀}
宜改正朔上詔兒寛與博士賜等共議以為宜用夏正|{
	漢初用秦正以建亥之月為歲首夏正以建寅之月為歲首}
夏五月詔卿遂遷等共造漢太初歷以正月為歲首色尚黄數用五|{
	時議者謂漢以土德旺土色黄而數五故上黄而用五張晏曰用五謂印文也若丞相曰丞相之印章諸卿及守相印文不足五字者以之字足之}
定官名協音律定宗廟百官之儀以為典常垂之後世云 匈奴兒單于好殺伐國人不安又有天災畜多死|{
	好呼到翻畜許救翻}
左大都尉使人間告漢曰|{
	間古莧翻}
我欲殺單于降漢|{
	降戶江翻}
漢遠即兵來迎我我即發上乃遣因杅將軍公孫敖築塞外受降城以應之|{
	服䖍曰因杅匈奴地名因所征以為將軍之名杆與俱翻受降城在居延北}
秋八月上行幸安定|{
	元鼎二年置安定郡屬凉州唐為原會涇州地}
漢使入西域者言宛有善馬在貳師城|{
	張晏曰貳師大宛城名宛於元翻}
匿不肯與漢使天子使壯士車令等持千金及金馬以請之|{
	姓譜以為車姓本於田千秋據此則已自有車姓}
宛王與其羣臣謀曰漢去我遠而鹽水中數敗|{
	服䖍曰鹽水水名道從水中行師古曰沙磧之中不生草木水又鹹苦即今敦煌西北惡磧者也數冇敗言每自死亡也孔文祥曰鹽澤也言水廣遠或致風波而數敗也裴矩西域記曰鹽水在西州高昌縣東東南去瓜州一千三百里並砂磧之地道路不可凖惟以人畜駭骨及駞馬糞為標驗由此數有死亡}
出其北有胡寇出其南乏水草又且往往而絶邑|{
	師古曰言近道之處無城郭之居也}
乏食者多漢使數百人為輩來而常乏食死者過半是安能致大軍乎無奈我何貳師馬宛寶馬也遂不肯予漢使|{
	予讀曰與}
漢使怒妄言椎金馬而去|{
	謂妄發言以詬詈之且椎破金馬而去也}
宛貴人怒曰漢使至輕我遣漢使去令其東邊郁成王遮攻殺漢使取其財物於是天子大怒諸嘗使宛姚定漢等言宛兵弱|{
	姚舜姓也左傳有鄭大夫姚句耳}
誠以漢兵不過三千人彊弩射之|{
	射而亦翻}
可盡虜矣天子嘗使浞野侯以七百騎虜樓蘭王以定漢等言為然而欲侯寵姬李氏|{
	師古曰欲封其兄弟}
乃拜李夫人兄廣利為貳師將軍發屬國六千騎及郡國惡少年數萬人以往伐宛|{
	師古曰惡少年謂無行義者}
期至貳師城取善馬故號貳師將軍趙始成為軍正故浩侯王恢使導軍而李哆為校尉制軍事|{
	哆昌也翻索隱音尺奢翻}


臣光曰武帝欲侯寵姬李氏而使廣利將兵伐宛其意以為非有功不侯不欲負高帝之約也夫軍旅大事國之安危民之死生繫焉苟為不擇賢愚而授之欲徼幸咫尺之功藉以為名而私其所愛不若無功而侯之為愈也然則武帝有見於封國無見於置將|{
	高祖曰置將不善一敗塗地將即亮翻}
謂之能守先帝之約臣曰過矣

中尉王温舒坐為姦利罪當族自殺時兩弟及兩婚家|{
	婦家曰婚}
亦各自坐佗罪而族光禄勲徐自為曰|{
	帝改郎中令為光禄勲應劭曰光明也禄爵也勲功也如淳曰胡公曰勲之言閽也閽者古主門官也光禄主宮門師古曰應說是也}
悲夫古有三族而王温舒罪至同時而五族乎|{
	師古曰温舒與兄弟同三族而兩妻各一故曰五族也}
關東蝗大起飛西至燉煌|{
	燉煌郡屬凉州唐瓜州沙州地燉音屯}


二年春正月戊申牧丘恬侯石慶薨|{
	沈約曰恬亦謚法所不載}
閏月丁丑以太僕公孫賀為丞相封葛繹侯|{
	賀始以功封南奅侯元鼎五年坐酎金免今以為相封葛繹侯功臣表不書所食邑}
時朝廷多事督責大臣自公孫弘後丞相比坐事死|{
	元狩五年丞相李蔡有罪自殺元鼎二年丞相莊青翟自殺五年丞相趙周下獄死師古曰比頻也比毘寐翻}
石慶雖以謹得終然數被譴|{
	數所角翻被皮義翻}
賀引拜為丞相不受印綬頓首涕泣不肯起上乃起去賀不得已拜出曰我從是殆矣|{
	師古曰殆危也}
三月上行幸河東|{
	河東郡屬司隸三河之一也唐蒲晉解隰州地}
祠后土 夏五月籍吏民馬補車騎馬 秋蝗 貳師將軍之西也既過鹽水當道小國各城守不肯給食攻之不能下下者得食不下者數日則去比至郁成|{
	比必寐翻及也}
士至者不過數千皆飢罷|{
	罷讀曰疲}
攻郁成郁成大破之所殺傷甚衆貳師將軍與李哆趙始成等計至郁成尚不能舉况至其王都乎引兵而還至燉煌|{
	燉音屯}
士不過什一二|{
	師古曰什人之中一二人得還}
使使上書言道遠乏食且士卒不患戰而患飢人少不足以拔宛願且罷兵益發而復往|{
	復扶又翻}
天子聞之大怒使使遮玉門曰軍有敢入者輒斬之貳師恐因留燉煌 上猶以受降城去匈奴遠遣浚稽將軍趙破奴將二萬餘騎出朔方西北二千餘里期至浚稽山而還|{
	應劭曰浚稽山在武威塞北匈奴常以為障蔽浚音峻稽音雞余據班史匈奴中有東西浚稽東浚稽山在龍勒水上}
浞野侯既至期左大都尉欲發而覺單于誅之發左方兵擊浞野侯浞野侯行捕首虜得數千人還未至受降城四百里匈奴兵八萬騎圍之浞野侯夜自出求水匈奴間捕生得浞野侯|{
	間古莧翻}
因急擊其軍軍吏畏亡將而誅莫相勸歸者軍遂沒於匈奴兒單于大喜因遣奇兵攻受降城不能下乃寇入邊而去 冬十二月兒寛卒|{
	兒五兮翻}


三年春正月膠東太守延廣為御史大夫|{
	膠東郡屬青州唐入青莱州界延廣史逸其姓守式又翻}
上東巡海上考神仙之屬皆無驗令祠官禮東泰山|{
	東泰山在琅邪郡朱虛縣界}
夏四月還修封泰山禪石閭|{
	應劭曰石閭在泰山下阯南方方士以為仙人之閭}
匈奴兒單于死子年少匈奴立其季父右賢王呴犂湖為單于|{
	呴漢書作句師古曰音鉤史記作呴音同又音吁}
上遣光禄勲徐自為出五原塞數百里|{
	史記正義曰即五原郡榆林塞也在勝州榆林縣西北四十里}
遠者千餘里築城障列亭西北至廬朐|{
	晉灼曰地理志從五原稒陽縣北出石門障即得所築城師古曰廬朐山名杜佑曰廬朐在麟州銀城縣北猶謂之光禄塞銀城漢圁隂縣地}
而使游擊將軍韓說長平侯衛伉屯其旁使彊弩都尉路博德築居延澤上|{
	班志居延澤在張掖居延縣東北古文以為流沙括地志居延海在甘州張掖縣東北六十四里甘州在長安西北二千四百六十里說讀曰悦伉音抗}
秋匈奴大入定襄雲中|{
	定襄雲中二郡屬并州}
殺畧數千人敗數二千石而去行破壞光禄所築城列亭障|{
	敗補邁翻壞音怪}
又使右賢王入酒泉張掖|{
	酒泉張掖二郡屬凉州}
畧數千人會軍正任文擊救盡復失所得而去|{
	師古曰擊救者擊匈奴以救漢人任音壬}
是歲睢陽侯張昌坐為太常乏祠國除|{
	班書功臣表及公卿表皆作睢陵侯高祖功臣張敖封宣平侯傳國至曾孫壬失侯元光三年封其弟廣為睢陵侯紹國昌廣之子也睢陵縣屬陵淮郡師古曰乏祠祠事冇闕也睢音雖}
初高祖封功臣為列侯百四十有三人時兵革之餘大城名都民人散亡戶口可得而數裁什二三|{
	師古曰裁與纔同十分之内纔有二三也}
大侯不過萬家小者五六百戶其封爵之誓曰使黄河如帶泰山若厲國以永存爰及苗裔|{
	應劭曰封爵之誓國家欲使功臣傳祚無窮也帶衣帶也厲砥石也河當何時如衣帶山當何時如厲石言如帶厲國猶永存以及後世之子孫也}
申以丹書之信重以白馬之盟及高后時盡差第列侯位次藏諸宗廟副在有司|{
	師古曰副貳也其列侯功籍已藏於宗廟副貳之本又在有司}
建文景四五世間流民既歸戶口亦息列侯大者至三四萬戶小國自倍|{
	師古曰謂舊五百戶今者至千戶也曹參初封萬六百戶至後嗣宗免時有戶二萬三千是為戶口蕃息故也他皆類此}
富厚如之|{
	師古曰言其資財亦益富厚如戶口之多}
子孫驕逸多扺法禁隕身失國至是見侯裁四人|{
	鄼侯蕭夀成繆侯酈世宗汾陽侯靳石封并睢陵侯張昌為四人耳見賢遍翻}
罔亦少密焉|{
	少詩沼翻}
漢既亡浞野之兵公卿議者皆願罷宛軍專力攻胡天子業出兵誅宛宛小國而不能下則大夏之屬漸輕漢而宛善馬絶不來烏孫輪臺易苦漢使|{
	晉灼曰易輕也師古曰輪臺亦國名余按輪臺在車師西千餘里又西即大宛易以豉翻}
為外國笑乃案言伐宛尤不便者鄧光等|{
	師古曰案其罪而行罰}
赦囚徒發惡少年及邊騎歲餘而出燉煌者六萬人|{
	師古曰興發部署歲餘乃得行}
負私從者不與|{
	師古曰負私糧食及私從者不在六萬人數中也從才用翻與讀曰預}
牛十萬馬三萬匹驢槖駝以萬數齎糧兵弩甚設|{
	師古曰施張甚具也}
天下騷動轉相奉伐宛五十餘校尉宛城中無井汲城外流水於是遣水工徙其城下水空以穴其城|{
	師古曰空孔也徙其城下水者令從它道流不廹其城也空以穴其城者圍而攻之令作孔使空穴也下云决其水原移之又云圍其城攻之皆再叙其事也一曰既徙其水不令於城下流而因其舊引水入城之孔攻而穴之余謂此書遣水工將以徙水穴城也下書决水原攻城正行其初計耳非再叙其事也}
益發戍甲卒十八萬酒泉張掖北置居延休屠屯兵以衛酒泉|{
	班志居延縣屬張掖郡休屠縣屬武威郡屠音儲}
而發天下吏有罪者亡命者及贅婿賈人故有市籍父母大父母有市籍者凡七科適為兵|{
	贅之芮翻賈音古適讀曰讁}
及載糒給貳師|{
	師古曰糒乾飯也音備}
轉車人徒相連屬|{
	屬之欲翻}
而拜習馬者二人為執驅馬校尉|{
	師古曰習猶便也一人為執馬校尉一人為驅馬校尉}
備破宛擇取其善馬云於是貳師後復行|{
	復扶又翻}
兵多所至小國莫不迎出食給軍至輪臺輪臺不下攻數日屠之自此而西平行至宛城|{
	師古曰平行言無寇難}
兵到者三萬宛兵迎擊漢兵漢兵射敗之|{
	射而亦翻敗補邁翻}
宛兵走入保其城貳師欲攻郁成城恐留行而令宛益生詐|{
	師古曰留行謂留止軍廢其行}
乃先至宛决其水原移之則宛固已憂困圍其城攻之四十餘日宛貴人謀曰王母寡匿善馬殺漢使|{
	師古曰母寡宛王名}
今殺王而出善馬漢兵宜解即不解乃力戰而死未晩也宛貴人皆以為然共殺王其外城壞虜宛貴人勇將煎靡|{
	師古曰宛之貴人為將而勇者名煎靡煎子延翻}
宛大恐走入城中持王母寡頭遣人使貳師約曰漢無攻我我盡出善馬恣所取而給漢軍食即不聽我盡殺善馬康居之救又且至至我居内康居居外與漢軍戰孰計之何從|{
	師古曰令貳師孰計之而欲攻戰乎欲不攻而取馬乎孰與熟同古字通用}
是時康居侯視漢兵尚盛不敢進貳師聞宛城中新得漢人知穿井而其内食尚多計以為來誅首惡者母寡母寡頭已至如此不許則堅守而康居候漢兵罷來救宛|{
	罷讀曰疲}
破漢兵必矣乃許宛之約宛乃出其馬令漢自擇之而多出食食漢軍|{
	師古曰下食讀曰飤}
漢軍取其善馬數十匹中馬以下牝牡三千餘匹而立宛貴人之故時遇漢善者名昧蔡為宛王|{
	服䖍曰蔡音楚言蔡師古曰昧音本末之末蔡音千曷翻}
與盟而罷兵初貳師起燉煌西|{
	起發也謂發燉煌而西也}
分為數軍從南北道校尉王申生將千餘人别至郁成郁成王擊滅之數人脱亡走貳師|{
	走音奏下同}
貳師令搜粟都尉上官桀往攻郁成|{
	帝置搜栗都尉屬大司農姓譜楚莊王少子為上官大夫其後以為氏秦滅楚徙隴西之上邽}
郁成王亡走康居桀追至康居康居聞漢已破宛出郁成王與桀桀令四騎士縛守詣貳師上邽騎士趙弟恐失郁成王|{
	班志上邽縣屬隴西郡故邽戎邑也邽音圭}
拔劍擊斬其首追及貳師

四年春貳師將軍來至京師貳師所過小國聞宛破皆使其子弟從入貢獻見天子|{
	見賢遍翻}
因為質焉|{
	質音至}
軍還入馬千餘匹|{
	漢書李廣利傳云軍還入玉門者萬餘人馬千餘匹文為詳明}
後行|{
	既還敦煌而再出師故曰後行}
軍非乏食戰死不甚多而將吏貪不愛卒侵牟之|{
	師古曰言如牟賊之食苖也}
以此物故者衆天子為萬里而伐不録其過乃下詔封李廣利為海西侯|{
	班志海西縣屬東海郡}
封趙弟為新畤侯|{
	功臣表新畤侯食邑於齊地畤音止}
以上官桀為少府軍官吏為九卿者三人諸侯相郡守二千石百餘人千石以下千餘人奮行者官過其望|{
	孟康曰迅速也自樂而行者}
以謫過行皆黜其勞|{
	師古曰言以罪謫而行者免其所犯不叙其勞}
士卒賜直四萬錢|{
	師古曰或以它財物充之故云直}
匈奴聞貳師征大宛欲遮之貳師兵盛不敢當即遣騎因樓蘭候漢使後過者欲絶勿通時漢軍正任文將兵屯玉門關捕得生口知狀以聞上詔文便道引兵捕樓蘭王將詣闕簿責王對曰小國在大國間不兩屬無以自安願徙國入居漢地上直其言|{
	師古曰以其言為直}
遣歸國亦因使候司匈奴|{
	司讀曰伺}
匈奴自是不甚親信樓蘭自大宛破後西域震懼漢使入西域者益得職|{
	師古曰賞其勤勞皆得拜職也余謂顔說非也此言漢使入西域諸國不敢輕辱為得其職耳得職者不失其職也}
於是自燉煌西至鹽澤往往起亭而輪臺渠犂|{
	渠犂在輔臺東東南與且末接南與精紀接}
皆有田卒數百人置使者校尉領護|{
	師古曰統領保護屯田之事也}
以給使外國者|{
	師古曰收其五穀以供之使疏吏翻}
後歲餘宛貴人以為昧蔡善諛|{
	以其遇漢善而得王也}
使我國遇屠乃相與殺昧蔡立母寡昆弟蟬封為宛王而遣其子入侍於漢漢因使使賂賜以鎮撫之蟬封與漢約歲獻天馬二匹 秋起明光宮|{
	三輔黄圖明光宮在長樂宮後南與長樂宮相聯屬北通桂宮}
冬上行幸回中匈奴呴犂湖單于死匈奴立其弟左

大都尉且鞮侯為單于|{
	師古曰且子餘翻鞮丁奚翻}
天子欲因伐宛之威遂困胡乃下詔曰高皇帝遺朕平城之憂|{
	平城事見十一卷高祖七年遺于季翻又如字}
高后時單于書絶悖逆|{
	事見十二卷惠帝三年悖蒲内翻}
昔齊襄公復九世之讐春秋大之|{
	公羊傳莊四年春齊襄公滅紀復仇也襄公之九世祖為紀侯所譛而烹殺于周故襄公滅紀也九世猶可以復仇乎曰雖百世可也}
且鞮侯單于初立恐漢襲之乃曰我兒子安敢望漢天子漢天子我丈人行也|{
	師古曰丈人尊老之稱行戶浪翻}
因盡歸漢使之不降者路充國等|{
	充國被留見上元封四年}
使使來獻

天漢元年|{
	應劭曰時頻年苦旱故改元為天漢以祈甘雨師古曰大雅有雲漢之詩周大夫仍叔所作以美宣王遇旱災修德勤政而能致雨故依以為年號也}
春正月上行幸甘泉郊泰畤三月行幸河東祠后土 上嘉匈奴單于之義遣中郎將蘇武送匈奴使留在漢者因厚賂單于答其善意武與副中郎將張勝及假吏常惠等俱|{
	師古曰假吏猶言兼吏也時權為使之吏若今之差人充使典矣姓譜常姓黄帝相常先之後}
既至匈奴置幣遺單于|{
	遺于季翻}
單于益驕非漢所望也|{
	漢望其回心鄉善今乃益驕故曰非漢所望}
會緱王與長水虞常等|{
	緱王者匈奴渾邪王姊子與渾邪王俱降漢後隨浞野侯沒匈奴中漢有長水校尉掌長水胡騎師古曰長水胡名其註戾太子傳則又曰今鄠縣東有長水余據水經註長水出杜縣白鹿原北入霸水胡騎盖屯於此非胡名也戻傳註是虞常盖亦先沒於匈奴緱工侯翻}
及衛律所將降者隂相與謀劫單于母閼氏歸漢|{
	降戶江翻閼氏音煙支}
衛律者父故長水胡人律善協律都尉李延年延年薦言律使於匈奴使還聞延年家收遂亡降匈奴 |{
	考異曰延年傳云誅延年兄弟宗族按是後李廣利尚為將帥盖止誅延年及弟季妻子耳}
單于愛之與謀國事立為丁靈王|{
	魏畧曰丁靈在康居北去匈奴廷接習水七千里匈奴盖以丁靈王封衛律耳}
虞常在漢時素與副張勝相知私候勝曰聞漢天子甚怨衛律常能為漢伏弩射殺之|{
	為于偽翻射而亦翻}
吾母弟在漢|{
	言其母與其弟也}
幸蒙其賞賜張勝許之以貨物與常後月餘單于出獵獨閼氏子弟在虞常等七十餘人欲發其一人夜亡告之單于子弟發兵與戰緱王等皆死虞常生得|{
	師古曰被執獲也}
單于使衛律治其事|{
	治直之翻}
張勝聞之恐前語發以狀語武|{
	狀語牛倨翻}
武曰事如此此必及我見犯乃死重負國欲自殺勝惠共止之虞常果引張勝|{
	見犯言被匈奴侵犯然後乃死是為更負漢國故欲先自殺而勝惠止之引謂辭及之也重直用翻}
單于怒召諸貴人議欲殺漢使者左伊秩訾曰即謀單于何以復加|{
	臣瓚曰左伊秩訾胡官之號余據匈奴傳呼韓邪之敗其右伊秩訾王使之降漢則此乃胡王之號師古曰言謀衛律而殺之其罰太重也復扶又翻}
宜皆降之|{
	降戶江翻下同}
單于使衛律召武受辭|{
	師古曰致單于之命而取其對也}
武謂惠等|{
	謂猶語也武語惠等也}
屈節辱命雖生何面目以歸漢引佩刀自刺|{
	刺七亦翻}
衛律驚自抱持武馳召醫鑿地為坎置煴火|{
	師古曰煴謂聚火無燄者也煴于云翻}
覆武其上|{
	師古曰覆身於坎上也覆音方目翻}
蹈其背以出血武氣絶半日復息|{
	師古曰息謂出氣也}
惠等哭輿歸營單于壯其節朝夕遣人候問武而收繫張勝武益愈單于使使曉武欲降之|{
	師古曰喻說令武降也}
會論虞常欲因此時降武劒斬虞常已律曰漢使張勝謀殺單于近臣|{
	師古曰近臣衛律自謂也}
當死單于募降者赦罪舉劒欲擊之勝請降律謂武曰副有罪當相坐武曰本無謀又非親屬何謂相坐復舉劒擬之|{
	復扶又翻下同}
武不動律曰蘇君律前負漢歸匈奴幸蒙大恩賜號稱王擁衆數萬馬畜彌山|{
	師古曰彌滿也畜許又翻}
富貴如此蘇君今日降明日復然空以身膏艸野|{
	膏古號翻}
誰復知之武不應律曰君因我降與君為兄弟今不聽吾計後雖欲復見我尚可得乎武罵律曰汝為人臣子不顧恩義畔主背親|{
	背蒲妹翻}
為降虜於蠻夷何以汝為見|{
	師古曰言何用見汝為也}
且單于信汝使决人死生不平心持正反欲鬬兩主觀禍敗南越殺漢使者屠為九郡宛王殺漢使者頭縣北闕朝鮮殺漢使者即時誅滅|{
	南越事見上卷元鼎五年六年宛事見上太初三年朝鮮事見上元封二年縣讀曰懸}
獨匈奴未耳若知我不降明|{
	師古曰若汝也言汝知我不降明矣}
欲令兩國相攻匈奴之禍從我始矣律知武終不可脅白單于單于愈益欲降之乃幽武置大窖中|{
	師古曰舊米粟之窖而空者也窖工孝翻}
絶不飲食天雨雪武卧齧雪與旃毛幷咽之|{
	師古曰飲於禁翻食讀曰飤雨于具翻齧魚結翻咽音宴吞也}
數日不死匈奴以為神乃徙武北海上無人處使牧羝曰羝乳乃得歸|{
	師古曰羝牡羊也羝不當產乳故設此言示絶其事若燕太子丹烏白頭馬生角之比也羝音丁奚翻乳音人喻翻}
别其官屬常惠等各置他所 天雨白氂|{
	師古曰氂毛之彊曲者音力之翻}
夏大旱 五月赦天下 發讁戍屯五原|{
	五原郡屬并州括地志勝州連谷縣本秦九原郡漢武帝更名五原}
浞野侯趙破奴自匈奴亡歸|{
	太初元年破奴為匈奴所獲}
是歲濟南太守王卿為御史大夫|{
	濟南郡屬青州唐淄青齊州地濟子禮翻守式又翻 考異曰七月閉城門大搜臣瓚註武帝紀曰漢帝記六月禁踰侈十月大搜則搜索踰侈者不必閉城門大搜蓋搜姦人耳}


二年春上行幸東海|{
	東海郡屬徐州唐為海州地}
幸回中 夏五月遣貳師將軍廣利以三萬騎出酒泉擊右賢王於天山|{
	晉灼曰天山在西域近蒲類國去長安八千里師古曰即祈連山也匈奴謂天曰祈連西河舊事白山冬夏有雪匈奴謂之天山括地志天山一名白山今名折羅漫山在伊吾縣北百二十里伊州在長安西北四千四百一十六里余據唐志祁山在甘州張掖縣與天山似是兩處騎音渠吏翻}
得胡首虜萬餘級而還|{
	還從宣翻又如字}
匈奴大圍貳師將軍漢軍乏食數日死傷者多假司馬隴西趙充國|{
	隴西郡屬凉州唐河渭岷州地}
與壯士百餘人潰圍陷陳|{
	陳讀曰陣}
貳師引兵隨之遂得解漢兵物故什六七充國身被二十餘創貳師奏狀詔徵充國詣行在所帝親見視其創|{
	被皮義翻創初良翻}
嗟嘆之拜為中郎漢復使因杅將軍敖出西河|{
	西河郡屬并州}
彊弩都尉路博德會涿涂山|{
	徐廣曰涂音邪索隱曰涿音卓邪以奢翻漢書作涿邪山在高闕塞北千餘里}
無所得初李廣有孫陵為侍中善騎射愛人下士帝以為有廣之風拜騎都尉|{
	續漢志騎都尉比二千石杜佑曰奉車都尉駙馬都尉騎都尉並漢武帝置東晉省奉車騎都尉惟留駙馬尚主者為之}
使將丹陽楚人五千人|{
	丹陽秦鄣郡地元封二年更名丹陽屬揚州唐宣歙池昇睦州之地}
教射酒泉張掖以備胡及貳師擊匈奴上詔陵欲使為貳師將輜重|{
	將即亮翻下同重直用翻}
陵叩頭自請曰臣所將屯邊者皆荆楚勇士奇材劍客也力扼虎|{
	扼捉持也}
射命中|{
	師古曰所指名處即中之也中竹仲翻}
願得自當一隊到蘭于山南以分單于兵毋令專鄉貳師軍|{
	鄉讀曰嚮}
上曰將惡相屬邪|{
	將如字惡烏路翻}
吾發軍多無騎予汝陵對無所事騎|{
	師古曰猶言不須騎也予讀曰與女讀曰汝}
臣願以少擊衆步兵五千人涉單于庭上壯而許之因詔路博德將兵半道迎陵軍博德亦羞為陵後距奏言方秋匈奴馬肥未可與戰願留陵至春俱出上怒疑陵悔不欲出而教博德上書乃詔博德引兵擊匈奴於西河詔陵以九月發出遮虜障|{
	遮虜障在張掖居延縣路博德所築括地志漢居延故城在甘州張掖縣之東北一千五百三十里有遮虜障}
至東浚稽山南龍勒水上|{
	班志敦煌龍勒縣有龍勒水出南羌中東北入澤溉民田蓋其下流北至浚稽山下}
徘徊觀虜即無所見還抵受降城休士|{
	太初元年公孫敖築受降城}
陵於是將其步卒五千人出居延北行三十日至浚稽山止營舉圖所過山川地形使麾下騎陳步樂還以聞步樂召見道陵將率得士死力|{
	將率猶言將領也將即亮翻率如字}
上甚悦拜步樂為郎陵至浚稽山與單于相值騎可三萬圍陵軍軍居兩山間以大車為營陵引士出營外為陳|{
	陳讀曰陣}
前行持戟盾後行持弓弩|{
	行戶剛翻盾食尹翻}
虜見漢軍少直前就營陵摶戰攻之|{
	如淳曰手對戰也}
千弩俱發應弦而倒虜還走上山|{
	上時掌翻}
漢軍追擊殺數千人單于大驚召左右地兵八萬餘騎攻陵陵且戰且引南行數日抵山谷中|{
	師古曰抵當也至也}
連戰士卒中矢傷三創者載輦兩創者將車一創者持兵戰復斬首三千餘級|{
	中作仲翻創初良翻將如字詩無將大車鄭氏曰將猶扶進也復扶又翻}
引兵東南循故龍城道行四五日抵大澤葭葦中|{
	師古曰葭即蘆也釋名曰初生為葭長大為蘆成則名為葦}
虜從上風縱火陵亦令軍中縱火以自救|{
	師古曰預燒自陳前葭葦則虜火不得而延及也}
南行至山下單于在南山上使其子將騎擊陵陵軍步鬬樹木間復殺數千人因發連弩射單于|{
	服䖍曰三十弩共一弦也張晏曰三十絭共一臂也貢父曰皆無此理蓋如今之合蟬或併兩弩共一弦之類余據魏氏春秋諸葛亮損益連弩以鉄為矢矢長八寸一弩十矢俱發今之划車弩梯弩蓋亦損益連弩而為之雖不能三十臂共一弦亦十數臂共一弦射而亦翻}
單于下走是日捕得虜言單于曰此漢精兵擊之不能下日夜引吾南近塞得無有伏兵乎|{
	近其靳翻}
諸當戶君長皆言|{
	師古曰當戶匈奴官名余據匈奴之官有左右當戶骨都侯几二十四長}
單于自將數萬騎擊漢數千人不能滅後無以復使邊臣|{
	復扶又翻下同}
令漢益輕匈奴復力戰山谷間尚四五十里得平地不能破乃還是時陵軍益急匈奴騎多戰一日數十合復傷殺虜二千餘人虜不利欲去會陵軍候管敢為校尉所辱|{
	續漢志凡領軍皆有部曲部有校尉部下有曲曲有軍候一人}
亡降匈奴具言陵軍無後救射矢且盡獨將軍麾下及校尉成安侯韓延年|{
	韓延年以父千秋死南越封事見上卷元鼎五年}
各八百人為前行以黄與白為幟當使精騎射之|{
	行戶剛翻射而亦翻}
即破矣單于得敢大喜使騎並攻漢軍疾呼曰李陵韓延年趣降|{
	呼火故翻趣讀曰促}
遂遮道急攻陵陵居谷中虜在山上四面射矢如雨下漢軍南行未至鞮汗山一日五十萬矢皆盡|{
	師古曰鞮音丁奚翻漢書作百五十萬矢皆盡}
即棄車去士尚三千餘人徒斬車輻而持之|{
	師古曰徒但也輻音福}
軍吏持尺刀入陿谷單于遮其後乘隅下壘石|{
	服䖍曰壘石山名也師古曰此說非也言放石以投人因山隅曲而下也壘盧對翻}
士卒多死不得行昏後陵便衣獨步出營|{
	蘇林曰褰衣卷䄂而行也師古曰便衣謂著短衣小䄂也}
止左右毋隨我丈夫一取單于耳|{
	師古曰言一身獨取也}
良久陵還太息曰兵敗死矣於是盡斬旌旗及珍寶埋地中陵嘆曰復得數十矢足以脱矣今無兵復戰|{
	師古曰兵即謂矢及矛戟之屬也}
天明坐受縳矣各鳥獸散猶有得脱歸報天子者令軍士人持二升糒一片氷|{
	師古曰時冬寒有氷持之以備渴也糒音備}
期至遮虜障者相待|{
	與軍士期有先至遮虜障者留駐以待後至也}
夜半時擊鼓起士鼓不鳴陵與韓延年俱上馬壯士從者十餘人虜騎數千追之韓延年戰死陵曰無面目報陛下遂降軍人分散脱至塞者四百餘人陵敗處去塞百餘里|{
	史記正義曰遮虜障北百八十里直居延西北長老相傳云是李陵戰處}
邊塞以聞上欲陵死戰後聞陵降上怒甚責問陳步樂步樂自殺羣臣皆罪陵上以問太史令司馬遷遷盛言陵事親孝與士信常奮不顧身以狥國家之急|{
	師古曰狥營也一曰從也}
其素所畜積也|{
	言其胷中素所畜積者如上所言也}
有國士之風今舉事一不幸全軀保妻子之臣隨而媒糵其短|{
	服䖍曰媒音欺謂詆欺也孟康曰媒酒教糵麯也謂釀成其罪也師古曰孟說是也齊人名麴餅曰媒賈公彦曰齊人名麴餅為媒者麴麩和合得成酒醴名之為媒}
誠可痛也且陵提步卒不滿五千深蹂戎馬之地|{
	師古曰蹂人九翻踐也}
抑數萬之師虜救死扶傷不暇悉舉引弓之民共攻圍之轉鬬千里矢盡道窮士張空弮|{
	文頴曰弮弓弩弮也師古曰音去權翻又音眷}
冒白刃北首争死敵|{
	師古曰冒犯也北首北嚮也冒音莫北翻首音式救翻}
得人之死力雖古名將不過也身雖陷敗然其所摧敗亦足暴於天下|{
	師古曰所摧敗敗匈奴之兵也暴者猶章也摧敗補賣翻}
彼之不死宜欲得當以報漢也|{
	師古曰言欲立功以當罪也}
上以遷為誣罔欲沮貳師為陵游說下遷腐刑|{
	沮在呂翻為于偽翻說式芮翻下遐嫁翻如淳曰腐宮刑也丈夫割勢不復能生子如腐木不生實腐音附}
久之上悔陵無救曰陵當發出塞乃詔彊弩都尉令迎軍坐預詔之得令老將生姦詐|{
	孟康曰坐預詔博德迎陵博德老將出塞不至令陵見没也余謂此說非也帝意既悔追思前事以為當陵發出塞之時方可詔博德繼其後以迎陵軍乃於陵未行之時預詔之使博德羞為陵後距得生姦詐上奏而遂令傳德别出西河使陵軍無救也}
乃遣使勞賜陵餘軍得脱者|{
	勞力到翻}
上以法制御下好尊用酷吏|{
	好呼報翻}
而郡國二千石為治者大抵多酷暴|{
	治直之翻}
吏民益輕犯法東方盜賊滋起大羣至數千人攻城邑取庫兵|{
	漢郡國各有庫兵}
釋死罪縛辱郡太守都尉殺二千石小羣以百數掠鹵鄉里者不可勝數|{
	勝音升}
道路不通上始使御史中丞丞相長史督之弗能禁|{
	督察也禁居禽翻}
乃使光禄大夫范昆及故九卿張德等衣繡衣持節虎符發兵以興擊|{
	師古曰以軍興之法而討擊也衣繡於計翻}
斬首大郡或至萬餘級及以法誅通行飲食當連坐者諸郡甚者數千人數歲乃頗得其渠率|{
	師古曰渠大也率所類翻}
散卒失亡復聚黨阻山川者往往而羣居無可奈何於是作沈命法|{
	應劭曰沈没也敢蔽匿盗賊者没其命也孟康曰沈藏匿也命亡逃也師古曰應說是沈持林翻}
曰羣盜起不發覺發覺而捕弗滿品者|{
	師古曰品率也以人數為率也}
二千石以下至小吏主者皆死其後小吏畏誅雖有盜不敢發恐不能得坐課累府府亦使其不言|{
	師古曰縣有盗賊府亦併坐故使縣不言之也累力瑞翻}
故盜賊寖多上下相為匿以文辭避灋焉是時暴勝之為直指使者所誅殺二千石以下尤多威震州郡|{
	暴周卿士暴公之後}
至渤海|{
	渤海郡屬幽州高祖置師古曰在渤海之濱因以為名唐為滄景州之地}
聞郡人雋不疑賢|{
	師古曰雋咅徂兖翻又辭兖翻姓譜有雋姓}
請與相見不疑容貌尊嚴衣冠甚偉勝之躧履起迎|{
	文頴曰躧音纚師古曰履不著跟曰躧躧謂納履未正曳之而行躧音山爾翻}
登堂坐定不疑據地曰竊伏海瀕聞暴公子舊矣|{
	師古曰瀕厓也公子勝之字也舊久也}
今乃承顔接辭凡為吏太剛則折|{
	折而設翻}
太柔則廢威行施之以恩然後樹功揚名|{
	樹立也}
永終天禄勝之深納其戒及還表薦不疑上召拜不疑為青州刺史濟南王賀亦為繡衣御史|{
	濟子禮翻}
逐捕魏郡羣盜|{
	魏郡高帝置屬冀州唐為相魏澶衛州地}
多所縱捨以奉使不稱免|{
	師古曰不稱謂不副所委稱尺正翻}
歎曰吾聞活千人子孫有封吾所活者萬餘人後世其興乎|{
	為王氏子孫以外戚簒漢張本}
是歲以匈奴降者介和王成娩為開陵侯|{
	降戶江翻師古曰娩音晩又音免班志開陵侯國屬臨淮郡}
將樓蘭國兵擊車師匈奴遣右賢王將數萬騎救之漢兵不利引去

資治通鑑卷二十一
