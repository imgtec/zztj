<!DOCTYPE html PUBLIC "-//W3C//DTD XHTML 1.0 Transitional//EN" "http://www.w3.org/TR/xhtml1/DTD/xhtml1-transitional.dtd">
<html xmlns="http://www.w3.org/1999/xhtml">
<head>
<meta http-equiv="Content-Type" content="text/html; charset=utf-8" />
<meta http-equiv="X-UA-Compatible" content="IE=Edge,chrome=1">
<title>資治通鑒_229-資治通鑑卷二百二十八_229-資治通鑑卷二百二十八</title>
<meta name="Keywords" content="資治通鑒_229-資治通鑑卷二百二十八_229-資治通鑑卷二百二十八">
<meta name="Description" content="資治通鑒_229-資治通鑑卷二百二十八_229-資治通鑑卷二百二十八">
<meta http-equiv="Cache-Control" content="no-transform" />
<meta http-equiv="Cache-Control" content="no-siteapp" />
<link href="/img/style.css" rel="stylesheet" type="text/css" />
<script src="/img/m.js?2020"></script> 
</head>
<body>
 <div class="ClassNavi">
<a  href="/24shi/">二十四史</a> | <a href="/SiKuQuanShu/">四库全书</a> | <a href="http://www.guoxuedashi.com/gjtsjc/"><font  color="#FF0000">古今图书集成</font></a> | <a href="/renwu/">历史人物</a> | <a href="/ShuoWenJieZi/"><font  color="#FF0000">说文解字</a></font> | <a href="/chengyu/">成语词典</a> | <a  target="_blank"  href="http://www.guoxuedashi.com/jgwhj/"><font  color="#FF0000">甲骨文合集</font></a> | <a href="/yzjwjc/"><font  color="#FF0000">殷周金文集成</font></a> | <a href="/xiangxingzi/"><font color="#0000FF">象形字典</font></a> | <a href="/13jing/"><font  color="#FF0000">十三经索引</font></a> | <a href="/zixing/"><font  color="#FF0000">字体转换器</font></a> | <a href="/zidian/xz/"><font color="#0000FF">篆书识别</font></a> | <a href="/jinfanyi/">近义反义词</a> | <a href="/duilian/">对联大全</a> | <a href="/jiapu/"><font  color="#0000FF">家谱族谱查询</font></a> | <a href="http://www.guoxuemi.com/hafo/" target="_blank" ><font color="#FF0000">哈佛古籍</font></a> 
</div>

 <!-- 头部导航开始 -->
<div class="w1180 head clearfix">
  <div class="head_logo l"><a title="国学大师官网" href="http://www.guoxuedashi.com" target="_blank"></a></div>
  <div class="head_sr l">
  <div id="head1">
  
  <a href="http://www.guoxuedashi.com/zidian/bujian/" target="_blank" ><img src="http://www.guoxuedashi.com/img/top1.gif" width="88" height="60" border="0" title="部件查字,支持20万汉字"></a>


<a href="http://www.guoxuedashi.com/help/yingpan.php" target="_blank"><img src="http://www.guoxuedashi.com/img/top230.gif" width="600" height="62" border="0" ></a>


  </div>
  <div id="head3"><a href="javascript:" onClick="javascript:window.external.AddFavorite(window.location.href,document.title);">添加收藏</a>
  <br><a href="/help/setie.php">搜索引擎</a>
  <br><a href="/help/zanzhu.php">赞助本站</a></div>
  <div id="head2">
 <a href="http://www.guoxuemi.com/" target="_blank"><img src="http://www.guoxuedashi.com/img/guoxuemi.gif" width="95" height="62" border="0" style="margin-left:2px;" title="国学迷"></a>
  

  </div>
</div>
  <div class="clear"></div>
  <div class="head_nav">
  <p><a href="/">首页</a> | <a href="/ShuKu/">国学书库</a> | <a href="/guji/">影印古籍</a> | <a href="/shici/">诗词宝典</a> | <a   href="/SiKuQuanShu/gxjx.php">精选</a> <b>|</b> <a href="/zidian/">汉语字典</a> | <a href="/hydcd/">汉语词典</a> | <a href="http://www.guoxuedashi.com/zidian/bujian/"><font  color="#CC0066">部件查字</font></a> | <a href="http://www.sfds.cn/"><font  color="#CC0066">书法大师</font></a> | <a href="/jgwhj/">甲骨文</a> <b>|</b> <a href="/b/4/"><font  color="#CC0066">解密</font></a> | <a href="/renwu/">历史人物</a> | <a href="/diangu/">历史典故</a> | <a href="/xingshi/">姓氏</a> | <a href="/minzu/">民族</a> <b>|</b> <a href="/mz/"><font  color="#CC0066">世界名著</font></a> | <a href="/download/">软件下载</a>
</p>
<p><a href="/b/"><font  color="#CC0066">历史</font></a> | <a href="http://skqs.guoxuedashi.com/" target="_blank">四库全书</a> |  <a href="http://www.guoxuedashi.com/search/" target="_blank"><font  color="#CC0066">全文检索</font></a> | <a href="http://www.guoxuedashi.com/shumu/">古籍书目</a> | <a   href="/24shi/">正史</a> <b>|</b> <a href="/chengyu/">成语词典</a> | <a href="/kangxi/" title="康熙字典">康熙字典</a> | <a href="/ShuoWenJieZi/">说文解字</a> | <a href="/zixing/yanbian/">字形演变</a> | <a href="/yzjwjc/">金 文</a> <b>|</b>  <a href="/shijian/nian-hao/">年号</a> | <a href="/diming/">历史地名</a> | <a href="/shijian/">历史事件</a> | <a href="/guanzhi/">官职</a> | <a href="/lishi/">知识</a> <b>|</b> <a href="/zhongyi/">中医中药</a> | <a href="http://www.guoxuedashi.com/forum/">留言反馈</a>
</p>
  </div>
</div>
<!-- 头部导航END --> 
<!-- 内容区开始 --> 
<div class="w1180 clearfix">
  <div class="info l">
   
<div class="clearfix" style="background:#f5faff;">
<script src='http://www.guoxuedashi.com/img/headersou.js'></script>

</div>
  <div class="info_tree"><a href="http://www.guoxuedashi.com">首页</a> > <a href="/SiKuQuanShu/fanti/">四库全书</a>
 > <h1>资治通鉴</h1> <!--         下载:【右键另存为】即可 --></div>
  <div class="info_content zj clearfix">
  
<div class="info_txt clearfix" id="show">
<center style="font-size:24px;">229-資治通鑑卷二百二十八</center>
    資治通鑑卷二百二十八 宋 司馬光 撰<br />
<br />
  胡三省 音註<br />
<br />
  唐紀四十四【起昭陽大淵獻正月盡十月不滿一年是年癸亥諸藩連兵拒命而德宗玩兵召禍日尋干戈最為多事是卷所紀纔十月耳】<br />
<br />
  德宗神武聖文皇帝三<br />
<br />
  建中四年春正月丁亥隴右節度使張鎰與吐蕃尚結贊盟于清水【使疎吏翻鎰弋質翻吐從暾入聲清水漢古縣唐屬秦州九域志在州東九十里】庚寅李希烈遣其將李克誠襲陷汝州執别駕李元<br />
<br />
  平【將即亮翻汝州治梁縣漢承休侯封邑也】元平本湖南判官薄有才藝性疎傲敢大言好論兵【好呼到翻】關播奇之薦於上以為將相之器以汝州距許州最近【九域志汝州東南至許州二百七十里史言關播所用非才相息亮翻】擢元平為汝州别駕知州事元平至汝州即募工徒治城【治直之翻】希烈隂使壯士應募執役入數百人元平不之覺希烈遣克誠將數百騎突至城下【將即亮翻又音如字騎奇寄翻】應募者應之於内縳元平馳去元平為人眇小無須【須古字取象以彡類耏毛也後人從而加髟為鬚字此俗書耳】見希烈恐懼便液汚地【便毗連翻便液謂屎溺也液音亦汚烏故翻】希烈罵之曰盲宰相以汝當我何相輕也以判官周晃為汝州刺史又遣别將董待名等四出抄掠取尉氏【尉氏縣屬汴州九域志在州南九十里抄楚交翻】圍鄭州官軍數為所敗邏騎西至彭婆【數所角翻敗補邁翻邏郎佐翻騎奇寄翻邏騎廵邏遊奕之騎九域志河南府河南縣有彭婆鎮金人疆域圖洛陽縣有彭婆鎮】東都士民震駭竄匿山谷留守鄭叔則入保西苑【東都西苑在東都城西鄭叔則盖備有急易於西奔也守式又翻】上問計於盧對曰希烈年少驍將恃功驕慢將佐莫敢諫止誠得儒雅重臣奉宣聖澤為陳逆順禍福【少詩照翻驍堅堯翻將即亮翻為于偽翻】希烈必革心悔過可不勞軍旅而服顔眞卿三朝舊臣【眞卿歷事玄肅代三朝朝直遥翻】忠直剛决名重海内人所信服眞其人也上以為然甲午命眞卿詣許州宣慰希烈詔下舉朝失色【下遐嫁翻】眞卿乘驛至東都鄭叔則曰往必不免宜少留須後命【少詩沼翻須待也】眞卿曰君命也將焉避之【焉於䖍翻】遂行李勉表言失一元老為國家羞請留之又使人邀眞卿不及眞卿與其子書但敕以奉家廟撫諸孤而已至許州欲宣詔旨希烈使其養子千餘人環繞慢罵【李希烈養壯士為子謂之養子環胡慣翻】拔刃擬之為將剸啗之勢【剸旨兖翻細割也】眞卿足不移色不變希烈遽以身蔽之麾衆令退館眞卿而禮之【令力丁翻館古玩翻】希烈欲遣眞卿還【還從宣翻又音如字】會李元平在座眞卿責之元平慙而起以密啟白希烈希烈意遂變留眞卿不遣朱滔王武俊田悦李納各遣使詣希烈上表稱臣勸進使者拜舞於希烈前說希烈曰【使疏吏翻上時掌翻說式芮翻】朝廷誅滅功臣失信天下都統英武自天功烈盖世已為朝廷所猜忌將有韓白之禍【朝直遥翻統他綜翻俗從上聲韓白之禍謂韓信斬於鍾室白起死於杜郵也】願亟稱尊號使四海臣民知有所歸希烈召顔眞卿示之曰今四王遣使見推不謀而同【以朱滔稱冀王王武俊稱趙王田悦稱魏王李納稱齊王故希烈謂之四王使疏吏翻】太師觀此事勢豈吾獨為朝廷所忌無所自容邪【邪音耶】眞卿曰此乃四凶何謂四王相公不自保功業為唐忠臣乃與亂臣賊子相從求與之同覆滅邪希烈不悦扶眞卿出它日又與四使同宴四使曰久聞太師重望今都統將稱大號而太師適至是天以宰相賜都統也【顔眞卿為太子太師故皆以其官稱之相息亮翻】眞卿叱之曰何謂宰相汝知有罵安禄山而死者顔杲卿乎【顔杲卿事見二百一十七卷肅宗至德元載叱尺栗翻】乃吾兄也吾年八十知守節而死耳豈受汝輩誘脅乎【史炤曰以利動之曰誘以威廹之曰脅誘音酉】四使不敢復言【復扶又翻】希烈乃使甲士十人守眞卿於館舍掘坎于庭云欲阬之眞卿怡然見希烈曰【怡然安和之貌】死生已定何必多端亟以一劒相與豈不快公心事邪希烈乃謝之 【考異曰顔氏行狀以為公至許州希烈前後許為公表奏請汴州者數十上知而寢之舊眞卿傳以為希烈逼為章表令雪已願罷兵馬累遣眞卿兄子峴與從吏凡數輩繼來京師上皆不報希烈大宴逆黨倡優斥黷朝政眞卿拂衣起後張伯儀敗績令以首級夸示眞卿號慟周曾謀奉眞卿遂送眞卿于龍興寺按滔等推尊希烈在去年眞卿使許在今年正月盖滔等始勸希烈稱帝希烈但稱都元帥建興王故今年滔等再遣樊播等勸進稱為都統也眞卿剛烈守之以死希烈豈能逼之使為章表雪已行狀云詐為表奏是也】 戊戌以左龍武大將軍哥舒曜為東都汝州節度使將鳳翔邠寧涇原奉天好畤行營兵萬餘人討希烈【鳳翔邠寧涇原三節鎮之兵奉天好畤神策屯兵也】又詔諸道共討之曜行至郟城【郟音夾郟城縣屬汝州東魏之龍山縣也隋開皇初改曰汝南十八年改曰輔城大業初改曰郟城九域志郟城縣在汝州東南九十里宋白曰春秋楚令尹子瑕城郟即此】遇希烈前鋒將陳利貞擊破之希烈勢小沮【沮在呂翻】曜翰之子也【天寶末安禄山反哥舒翰敗没于潼關】希烈使其將封有麟據鄧州南路遂絶貢獻商旅皆不通 壬寅詔治上津山路置郵驛【上津縣屬商州治直之翻】 二月戊申朔命鴻臚卿崔漢衡送區頰贊還吐蕃【區頰贊入見事始見上卷上年臚陵如翻】 丙戌以河陽三城懷衛州為河陽軍 丁卯哥舒曜克汝州擒周晃 三月戊寅江西節度使曹王臯敗李希烈將韓霜露於黄梅斬之辛卯拔黄州時希烈兵栅蔡山【敗補邁翻九域志黄梅縣屬蘄州距州一百二十里蔡山在黄梅界即江左新蔡郡治所魯悉逹保聚之地宋白曰宋分江夏郡置南新蔡郡隋開皇十八年改為黄梅縣以界内黄梅山名之祝穆曰蔡山出大龜春秋左氏傳所謂大蔡盖以山得名也】險不可攻臯聲言西取蘄州【蘄音祈蘄州後漢為蘄春侯國吳置蘄春郡北齊置齊昌郡及羅州後周改蘄州州北有蘄水南入于江地名解云蘄春以水隈多蘄菜因名】引舟師泝江而上希烈之將引兵循江隨戰去蔡山三百餘里臯乃復放舟順流而下【上時掌翻復扶又翻下遐嫁翻】急攻蔡山拔之希烈兵還救之不及而敗臯遂進拔蘄州表伊慎為蘄州刺史王鍔為江州刺史 淮寧都虞候周曾鎮遏兵馬使王玢押牙姚憺韋清密輸欵於李勉【玢府巾翻憺徒濫翻又徒敢翻】李希烈遣曾與十將康秀琳將兵三萬攻哥舒曜至襄城【襄城縣漢屬潁川郡晉屬襄城郡後周置汝州唐貞觀元年廢州以襄城縣屬許州貞觀八年以伊州為汝州襄城仍屬許州天寶七載復屬汝州九域志襄城縣在汝州東南一百有五十里】曾等密謀還軍襲希烈奉顔眞卿為節度使使玢憺清為内應希烈知之遣别將李克誠將騾軍三千人【淮西地少馬乘騾以戰號騾子軍尤為驍鋭將即亮翻誠將音同又音如字騾落戈翻】襲曾等殺之并殺玢憺及其黨甲午詔贈曾等官始韋清與曾等約事泄不相引故獨得免清恐終及禍說希烈請詣朱滔乞師【說式芮翻】希烈遣之行至襄邑逃奔劉洽【襄邑縣屬宋州劉洽時以宣武節度鎮宋州】希烈聞周曾等有變閉壁數日其黨寇尉氏鄭州者聞之亦遁歸希烈乃上表歸咎於周曾等引兵還蔡州【上時掌翻還從宣翻蔡州治汝陽縣淮寧本鎮也希烈時自許州退還】外示悔過從順實待朱滔等之援也置顔眞卿於龍興寺【寺盖在蔡州】丁酉荆南節度使張伯儀與淮寧兵戰於安州【安州漢安陸地】官軍大敗伯儀僅以身免亡其所持節希烈使人以其節及俘馘示顔眞卿眞卿號慟投地絶而復蘇自是不復與人言【俘方無翻馘古獲翻號戶高翻復扶又翻又音如字】 夏四月上以神策軍使白志貞為京城召募使募禁兵以討李希烈志貞請諸嘗為節度觀察都團練使者不問存没並勒其子弟帥奴馬自備資裝從軍【帥讀口率】授以五品官貧者甚苦之人心始搖【史言德宗窮兵亂將作矣】 上命宰相尚書與吐蕃區頰贊盟於豐邑里區頰贊以清水之盟疆場未定不果盟【是年春張鎧與吐蕃盟于清水宋白曰張鎰與吐蕃盟文曰今國家所守界涇州西至彈箏峽西口隴州西至清水縣鳳州西至同谷縣暨劒南西山大渡河東為漢界蕃國守備在蘭渭原會西至臨洮又東至成州抵劒南西界磨些諸蠻大渡水西南為蕃界相息亮翻尚辰羊翻吐從暾入聲場音亦】己未命崔漢衡入吐蕃决於贊普【是年二月命崔漢衡送區頰贊盖欲與之盟而遣之久而盟未定又命漢衡入吐蕃决于贊普此時中國疲於兵彼固有以窺唐矣盟無益也】 庚申加永平宣武河陽都統李勉淮西招討使東都汝州節度使哥舒曜為之副以荆南節度使張伯儀為淮西應援招討使山南東道節度使賈耽江西節度使曹王臯為之副上督哥舒曜進兵曜至潁橋【九域志襄城縣有潁橋鎮】遇大雨還保襄城李希烈遣其將李光輝攻襄城曜擊却之 五月乙酉潁王璬薨【璬玄宗子音古了翻】 乙未以宣武節度使劉洽兼淄青招討使 李晟謀取涿莫二州以絶幽魏往來之路與張孝忠之子升雲圍朱滔所署易州刺史鄭景濟于清苑【水經注徐水出北平東逕清苑城東至高陽入于河劉昫曰清苑縣漢之樂鄉縣屬信都國隋為清苑縣屬瀛州唐景雲元年屬冀州至宋以清苑縣為保州治所宋白曰漢高祖訪樂毅之後得樂叔封于樂鄉高齊省仍自今易州滿城縣界移永寧縣理此城隋改為清苑縣因滿城縣界清苑河為名】累月不下滔以其司武尚書馬寔為留守【司武尚書猶天朝兵部尚書】將步騎萬餘守魏營自將步騎萬五千救清苑李晟軍大敗退保易州滔還軍瀛州張升雲奔滿城【劉昫曰滿城縣漢北平縣地後魏置永樂縣天寶元年改為滿城屬易州】會晟病甚引軍還保定州 【考異曰燕南記曰晟與張升雲等圍鄭景濟于清苑自二月至四月滔自統馬步萬五千人救清苑四月二日發館陶砦五月内到晟出戰不利城中又出攻晟晟敗去滔乘勝逐晟等大破之晟奔易州染病不復更出實録曰庚子李晟自清苑退保易州舊晟傳曰自正月至于五月會晟病甚不知人者數焉軍吏合謀乃以馬輿還定州今從之實録所云庚子盖奏到日也】王武俊以滔既破李晟留屯瀛州未還魏橋遣其給事中宋端趣之【趣讀曰促】端見滔言頗不遜滔怒使謂武俊曰滔以熱疾蹔未南還大王二兄遽有云云滔以救魏博之故叛君弃兄如脱屣耳【履不躡跟曰屣脱之易耳】二兄必相疑惟二兄所為端還報武俊自辨於馬寔寔以狀白滔言趙王知宋端無禮於大王深加責讓實無它志武俊亦遣承令官鄭和隨寔使者見滔謝之【時武俊等改要藉官為承令官】滔乃悦相待如初然武俊以是益恨滔矣六月李抱眞使參謀賈林詣武俊壁詐降【節度參謀關預軍中機密】武俊見之林曰林來奉詔非降也武俊色動問其故林曰天子知大夫宿著誠効【謂誅李惟岳也】及登壇之日【謂稱王時也】撫膺顧左右曰我本徇忠義天子不察諸將亦嘗共表大夫之志天子語使者曰朕前事誠誤悔之無及朋友失意尚可謝况朕為四海之主乎【賈林先言武俊心事後述天子詔旨鋪陳悔過之意可謂善說矣語牛倨翻】武俊曰僕胡人也為將尚知愛百姓况天子豈專以殺人為事乎今山東連兵暴骨如莽【杜預曰草之生於廣野莽莽然故曰草莽莽莫朗翻暴步卜翻又薄報翻】就使克捷與誰守之僕不憚歸國但已與諸鎮結盟胡人性直不欲使曲在己天子誠能下詔赦諸鎮之罪僕當首唱從化諸鎮有不從者請奉辭伐之如此則上不負天子下不負同列不過五旬河朔定矣使林還報抱眞隂相約結【為武俊與抱眞破走朱滔張本】 庚戌初行税間架除陌錢法時河東澤潞河陽朔方四軍屯魏縣神策永平宣武淮南浙西荆南江泗沔鄂湖南黔中劒南嶺南諸軍環淮寧之境【江謂江南西道泗當作西黔音琴環音宦】舊制諸道軍出境皆仰給度支【仰牛向翻】上優恤士卒每出境加給酒肉本道糧仍給其家一人兼三人之給故將士利之各出軍纔逾境而止【書有之威克厥愛允濟愛克厥威允罔功德宗盖未知此者】月費錢百三十餘萬緍常賦不能供判度支趙贊乃奏行二法【二法即所謂税間架及除陌錢也】所謂税間架者每屋兩架為間上屋税錢二千中税千下税五百吏執筆握算入人室廬計其數【史炤曰算所以籌算也其法用竹徑一分長六寸二百七十一枚而成六觚為一握】或有宅屋多而無它資者出錢動數百緍敢匿一間杖六十賞告者錢五十緍所謂除陌錢者公私給與及賣買每緍官留五十錢給它物及相貿易者約錢為率【貿音茂】敢隱錢百杖六十罰錢二千賞告者錢十緍其賞錢皆出坐事之家於是愁怨之聲盈於遠近 丁卯徙郴王逾為丹王鄜王遘為簡王【二王皆上弟也】 庚午答蕃判官監察御史于頔【答蕃判官因當時出使署署置以為官名頔徒歷翻】與吐蕃使者論刺没藏至自青海【刺盧逹翻】言疆場已定請遣區頰贊歸國秋七月甲申以禮部尚書李揆為入蕃會盟使【入蕃命官猶答蕃也】壬辰詔諸將相與區頰贊盟於城西李揆有才望盧惡之【惡烏路翻】故使之入吐蕃揆言於上曰臣不憚遠行恐死於道路不能逹詔命上為之惻然【為于偽翻】謂杞曰揆無乃太老曰使遠夷非諳練朝廷故事者不可【使疏吏翻諳烏含翻】且揆行則自今年少於揆者【少詩照翻】不敢辭遠使矣【使疏吏翻】 八月丁未李希烈將兵三萬圍哥舒曜於襄城詔李勉及神策將劉德信將兵救之乙卯希烈將曹季昌以隨州降尋復為其將康叔夜所殺【復扶又翻】 初上在東宫聞監察御史嘉興陸贄名【嘉興漢由拳縣地吳大帝黄龍三年以其地嘉禾生改為禾興縣後避太子和名改為嘉興縣隋廢縣唐初復置屬蘇州】即位召為翰林學士【韋執誼翰林志曰自太宗時名儒學士時召革制然猶未有名號乾封以後始號北門學士玄宗初置翰林待詔掌四方表疏批答應和文章繼以詔勑文告悉由中書多壅滯始選朝官有詞藝學識者入居翰林供奉别旨然亦未定名制詔書勑猶或分在集賢開元二十六年翰林供奉始改稱學士别建學士院於翰林院之南俾專内命其後又置東翰林院於金鑾殿之西隨上所在】數問以得失時兩河用兵久不决【兩河謂河南河北】賦役日滋贄以兵窮民困恐别生内變乃上奏其略曰克敵之要在乎將得其人馭將之方在乎操得其柄【將即亮翻下同操千高翻】將非其人者兵雖衆不足恃操失其柄者將雖材不為用又曰將不能使兵國不能馭將非止費財翫寇之弊亦有不戢自焚之災【左氏傳曰兵猶火也不戰將自焚】又曰今兩河淮西為叛亂之帥者獨四五凶人而已【四五凶人為河北則朱滔王武俊田悦河南則李納淮西則李希烈也帥所類翻】尚恐其中或遭詿誤【詿古賣翻又胡卦翻】内蓄危疑蒼黄失圖勢不得止况其餘衆盖並脅從【史炤曰書云脅從罔治孔潁逹疏云謂被脅從而距王命者余謂脅從者為威力所廹脅不得已而從于逆非同心為逆者也】苟知全生豈願為惡又曰無紓目前之虞或興意外之變人者邦之本也財者人之心也其心傷則其本傷其本傷則枝幹顛瘁矣【瘁秦醉翻】又曰人揺不寧事變難測是以兵貴拙速不貴巧遲若不靖於本而務救於末則救之所為乃禍之所起也又論關中形勢以為王者蓄威以昭德偏廢則危居重以馭輕倒持則悖王畿者四方之本也太宗列置府兵分隸禁衛大凡諸府八百餘所而在關中者殆五百焉舉天下不敵關中則居重馭輕之意明矣承平漸久武備浸微雖府衛具存而卒乘罕習【卒臧没翻乘繩證翻】故禄山竊倒持之柄乘外重之資一舉滔天兩京不守【事見玄宗天寶十四載肅宗至德元載】尚賴西邊有兵諸牧有馬每州有糧故肅宗得以中興【中竹仲翻】乾元之後繼有外虞悉師東討邊備既弛禁戎亦空吐蕃乘虛深入為寇故先皇帝莫與為禦避之東遊【事見二百二十三卷代宗廣德元年】是皆失居重馭輕之權忘深根固柢之慮【柢都禮翻又都計翻】内寇則殽函失險外侵則汧渭為戎【汧口肩翻】于斯之時雖有四方之師寧救一朝之患陛下追想及此豈不為之寒心哉【為于偽翻】今朔方太原之衆遠在山東【謂李懷光以朔方軍馬燧以太原軍討田悦兵不解也】神策六軍之兵繼出關外【左右羽林左右龍武左右神策為六軍又曰左右羽林龍武神武為六軍神策軍最盛在六軍之右時李晟哥舒曜劉德信等皆以禁兵出關討賊】儻有賊臣啗寇虜覷邊【啗徒濫翻又徒覽翻覷七慮翻伺視也】伺隙乘虛微犯亭障此愚臣所竊憂也【伺相吏翻】未審陛下其何以禦之側聞伐叛之初議者多易其事【以其事為易也易弋豉翻】僉謂有征無戰役不踰時計兵未甚多度費未甚廣【度徒洛翻】於事為無擾於人為不勞曾不料兵連禍拏變故難測日引月長漸乖始圖【曾戶增翻拏女加翻相牽引也圖謀也】往歲為天下所患咸謂除之則可致升平者李正已李寶臣梁崇義田悦是也往歲為國家所信咸謂任之則可除禍亂者朱滔李希烈是也既而正巳死李納繼之寶臣死惟岳繼之崇義平希烈叛惟岳戮朱滔攜【攜離也貳也】然則往歲之所患者四去其三矣【去羌呂翻】而患竟不衰往歲之所信今則自叛矣而餘又難保是知立國之安危在勢任事之濟否在人勢苟安則異類同心也勢苟危則舟中敵國也陛下豈可不追鑒往事惟新令圖修偏廢之柄以靖人復倒持之權以固國【漢人曰秦倒持太阿授楚其柄】而乃孜孜汲汲極思勞神【思相吏翻】徇無已之求望難必之效乎今關輔之間徵已甚宫苑之内備衛不全【北軍皆屯苑中時悉在行營】萬一將帥之中又如朱滔希烈或負固邊壘誘致豺狼【將即亮翻帥所類翻誘羊久翻】或竊發郊畿驚犯城闕此亦愚臣所竊為憂者也未審陛下復何以備之【姚令言朱泚之變卒如陸贄所料復扶又翻又音如字】陛下儻過聽愚計所遣神策六軍李晟等及節將子弟悉可追還【晟成正翻節將子弟白志貞所奏遣東征者還從宣翻又音如字】明敕涇隴邠寧但令嚴備封守【邠卑旻翻令力丁翻】仍云更不徵使知各保安居又降德音罷京城及畿縣間架等雜税則冀已輸者弭怨見處者獲寧【見賢遍翻處昌呂翻】人心不揺邦本自固上不能用 壬戌以汴西運使崔縱兼魏州四節度都糧料使【汴東西運使事始見上卷上年河東節度使馬燧澤潞節度使李抱眞河陽節度使李芁朔方節度使李懷光四軍時並在魏州行營宋白曰建中用兵諸道行營出境者皆仰給度支謂之食出界糧又於諸軍各以臺省官一人司其供億謂之糧料使余按代宗廣德初郭子儀自商州進收京師請第五琦為糧料使】縱渙之子也【崔渙者玄暐之孫玄宗幸蜀以為相】九月丙戌神策將劉德信宣武將唐漢臣與淮寧<br />
<br />
  將李克誠戰敗于滬澗【將即亮翻滬侯古翻 考異曰徐岱奉天記曰大將唐漢臣劉德信高秉哲合統兵一萬屯於汝州三帥各領本軍城小卒衆教令不一軍進至薛店更無它路又不設支軍賊謀知之乘霧而進三帥望敵大潰戈楯資實山積馬萬餘蹄皆没馬汝州遂陷攝刺史李元平為寇所獲賊邏兵北至彭婆今從實録】時李勉遣漢臣將兵萬人救襄城上遣德信帥諸將家應募者三千人助之【將即亮翻又音如字帥讀曰率下同】勉奏李希烈精兵皆在襄城許州空虛若襲許州則襄城圍自解【去年希烈徙鎮許州盖欲乘虚擣其巢穴則希烈必釋襄城之圍以自救】遣二將趣許州【趣七喻翻】未至數十里上遣中使責其違詔二將狼狽而返無復斥候克誠伏兵邀之殺傷大半漢臣奔大梁德信奔汝州希烈遊兵剽掠至伊闕【剽匹妙翻伊闕禹所鑿春秋為戎蠻子之國漢為新城縣隋為伊闕縣唐屬河南府】勉復遣其將李堅帥四千人助守東都【復扶又翻又音如字 考異曰新傳作李堅華今從實録】希烈以兵絶其後堅軍不得還【還從宣翻又音如字】汴軍由是不振襄城益危【汴皮變翻汴軍宣武兵也此時則李勉帥永平軍方鎮表大歷十四年永平軍增領汴潁二州徙治汴州故使史有汴軍之稱】 上以諸軍討淮寧者不相統壹庚子以舒王謨為荆襄等道行營都元帥更名誼【更工衡翻】以戶部尚書蕭復為長史右庶子孔巢父為左司馬諫議大夫樊澤為右司馬自餘將佐皆選中外之望【將即亮翻】未行會涇師作亂而止復嵩之孫也【蕭嵩開元中為相】巢父孔子三十七世孫也上涇原諸道兵救襄城冬十月丙午涇原節度使<br />
<br />
  姚令言將兵五千至京師【考異曰舊傳云令言率本鎮兵五萬赴援按奉天記曰哥舒曜表請加師上使涇州節度使姚令言赴援令言本領三千請加至五千今從之】軍士冒雨寒甚多擕子弟而來冀得厚賜遺其家【遺唯季翻】既至一無所賜丁未至滻水詔京兆尹王翃犒師惟糲食菜餤衆怒蹴而覆之【糲盧逹翻餤戈亷翻又徒甘翻蹴子六翻】因揚言曰吾輩將死於敵而食且不飽安能以微命拒白刃邪聞瓊林大盈二庫【玄宗時王鉷為戶口邑役使徵剥財貨每歲進錢百億寶貨稱是入百寶大盈庫以供人主宴私賞賜之用則玄宗時已有大盈庫陸贄諫帝曰瓊林大盈自古悉無其制傳諸耆舊之說皆云創自開元聚歛之臣貪權飾巧求媚乃言郡國貢獻所合區分賦税當委於有司以給經用貢獻宜歸於天子以奉私求玄宗悦之新置是二庫蕩心侈欲萌禍於兹迨乎失邦終以餌寇則庫始於玄宗明矣宋白曰大盈庫内庫也以中人主之欲至德中第五琦始悉以租賦進入大盈庫天子以出納為便故不復出】金帛盈溢不如相與取之乃擐甲張旗鼓譟還趣京師【趣七喻翻】令言入辭尚在禁中聞之馳至長樂阪遇之【長樂阪在滻水西本滻阪也隋文帝惡其名取其北對長樂改曰長樂阪亦曰長樂坡樂音洛】軍士射令言令言抱馬鬛突入亂兵呼曰【射而亦翻呼火故翻】諸君失計東征立功何患不富貴乃為族滅之計乎軍士不聽以兵擁令言而西【自長樂阪西入京城】上遽命賜帛人二匹衆益怒射中使【射而亦翻】又命中使宣慰賊已至通化門外【通化門京城東面北來第一門程大昌曰通化門北去丹鳳門止兩坊】中使出門賊殺之又命出金帛二十車賜之賊已入城喧聲浩浩不復可遏【復扶又翻】百姓狼狽駭走賊大呼告之曰汝曹勿恐不奪汝商貨僦質矣不税汝間架陌錢矣【呼火故翻僦即就翻】上遣普王誼翰林學士姜公輔出慰諭之賊已陳於丹鳳門外【陳讀曰陣】小民聚觀者以萬計初神策軍使白志貞掌召募禁兵東征死亡者志貞皆隱不以聞但受市井富兒賂而補之名在軍籍受給賜而身居市廛為販鬻司農卿段秀實上言禁兵不精其數全少卒有患難將何待之不聽【少詩沼翻卒讀曰猝難乃旦翻】至是上召禁兵以禦賊竟無一人至者賊已斬關而入上乃與王貴妃韋淑妃太子諸王唐安公主自苑北門出王貴妃以傳國寶繫衣中以從【從才用翻下同】後宫諸王公主不及從者什七八初魚朝恩既誅宦官不復典兵【事見二百二十四卷代宗大歷五年】有竇文塲霍仙鳴者嘗事上於東宫至是帥宦官左右僅百人以從【帥讀曰率】使普王誼前驅太子執兵以殿【殿丁練翻】司農卿郭曙以部曲數十人獵苑中【禁苑在京城之北東至灞水西連故長安城南連京城北枕渭水】聞蹕謁道左遂以其衆從曙曖之弟也【曖曙皆郭子儀之子】右龍武軍使令狐建方教射於軍中聞之帥麾下四百人從【帥讀曰率下相帥同】乃使建居後為殿姜公輔叩馬言曰朱泚嘗為涇帥【見二百二十六卷元年帥所類翻】坐弟滔之故廢處京師【事見上卷上年處昌呂翻】心嘗怏怏臣謂陛下既不能推心待之則不如殺之毋貽後患今亂兵若奉以為主則難制矣請召使從行上倉猝不暇用其言曰無及矣遂行夜至咸陽飯數而過【飯扶晩翻】時事出非意羣臣皆不知乘輿所之【之往也乘繩證翻】盧關播踰中書垣而出白志貞王翃及御史大夫于頎中丞劉從一戶部侍郎趙贊翰林學士陸贄吴通微等追及上於咸陽頎頔之從父兄弟【頎渠希翻之從才用翻】從一齊賢之從孫也【劉齊賢祥道之子以方正為高宗所重】賊入宫登含元殿大呼曰天子已出宜人自求富遂讙譟爭入府庫運金帛極力而止【讙許元翻】小民因之亦入宫盗庫物通夕不已其不能入者剽奪於路諸坊居民各相帥自守姚令言與亂兵謀曰今衆無主不能持久朱太尉閒居私第請相與奉之衆許諾乃遣數百騎迎泚於晉昌里第【按長安圖自京城啟夏門北入東街第二坊曰進昌坊 考異曰舊泚傳作招國里今從實録】夜半泚按轡列炬傳呼入宫居含元殿設警嚴【設鼓角以警嚴一曰設卒以警備嚴衛】自稱權知六軍戊申旦泚徙居白華殿【考李晟收復京城次第白華殿盖近光泰門内大明宫東北隅程大昌曰晟收長安亦自白華門入諸家不載何地以晟兵所届言之當在大明東苑之東】出榜於外稱涇原將士久處邊陲【處昌呂翻】不閑朝禮【閑習也朝直遥翻下同】輒入宫闕致驚乘輿西出廵幸【乘繩證翻】太尉以權臨六軍應神策軍士及文武百官凡有禄食者悉詣行在不能往者即詣本司若出三日檢勘彼此無名者皆斬於是百官出見泚或勸迎乘輿泚不悦百官稍稍遁去源休以使回紇還賞薄怨朝廷【賞薄事見上卷上年使疏吏翻】入見泚屏人密語移時【屏必郢翻又卑正翻】為泚陳成敗引符命勸之僭逆【為于偽翻】泚喜然猶未决宿衛諸軍舉白幡降者列於闕前甚衆【降戶江翻】泚夜於苑門出兵旦自通化門入駱驛不絶張弓露刃欲以威衆上思桑道茂之言【道茂言見二百二十六卷元年】自咸陽幸奉天縣僚聞車駕猝至欲逃匿山谷主薄蘇弁止之弁良嗣之兄孫也【蘇良嗣武后初為相】文武之臣稍稍繼至己酉左金吾大將軍渾瑊至奉天瑊素有威望衆心恃之稍安【瑊古咸翻】庚戌源休勸朱泚禁十城門【唐都長安京城東面通化春明延興三門南面啓夏明德安化三門西延秋金光開遠三門北光化一門凡十門】毋得出朝士朝士往往易服為傭僕濳出休又為泚說誘文武之士使之附泚【又為于偽翻說輸芮翻】檢校司空同平章事李忠臣久失兵柄太僕卿張光晟自負其才皆鬱鬱不得志【李忠臣失兵柄見二百二十五卷代宗大歷十四年張光晟事見二百二十六卷元年校古效翻晟成正翻】泚悉起而用之工部侍郎蔣鎮出亡墜馬傷足為泚所得【泚且禮翻又音此】先是休以才能光晟以節義鎮以清素都官員外郎彭偃以文學太常卿敬釭以勇略【先悉薦翻釭古紅翻又古雙翻】皆為時人所重至是皆為泚用鳳翔涇原將張廷芝段誠諫將數千人救襄城【原將即亮翻諫將音同又音如字】未出潼關聞朱泚據長安殺其大將隴右兵馬使戴蘭潰歸於泚【泚先帥鳳翔涇原故二鎮之兵聞亂皆歸之潼音同使疏吏翻】泚於是自謂衆心所歸謀反遂定以源休為京兆尹判度支【度徒洛翻】李忠臣為皇城使【唐六典皇城在京城之中東西五里一百一十五步南北三里一百四十步南面三門中曰朱雀左曰安上右曰含光東面二門北曰延喜南曰景風西面二門北曰安福南曰順義其中右社稷左宗廟百僚廨署列乎其間唐自開元以前以城門郎掌皇城諸門開闔之節中世以後置皇城使】百司供億六軍宿衛咸擬乘輿【乘繩證翻】辛亥以渾瑊為京畿渭北節度使行在都虞候白志貞為都知兵馬使令狐建為中軍鼓角使以神策都虞候侯仲莊為左衛將軍兼奉天防城使【渾戶昆翻又戶本翻瑊古咸翻使疏吏翻令力丁翻】朱泚以司農卿段秀實久失兵柄【段秀實失兵柄見二百二十六卷元年】意其必怏怏【怏於兩翻】遣數十騎召之秀實閉門拒之騎士踰垣入劫之以兵秀實自度不免【騎奇寄翻垣于元翻度徒洛翻】乃謂子弟曰國家有患吾於何避之當以死徇社稷汝曹宜人自求生乃往見泚泚喜曰段公來吾事濟矣延坐問計秀實說之曰公本以忠義著聞天下【謂泚能釋鎮入朝及與弟滔絶也說式芮翻】今涇軍以犒賜不豐遽有披猖使乘輿播越夫犒賜不豐有司之過也【犒口到翻乘繩證翻夫音扶】天子安得知之公宜以此開諭將士示以禍福奉迎乘輿復歸宫闕此莫大之功也【將即亮翻復扶又翻又音如字】泚默然不悦【泚且禮翻又音此】然以秀實與己皆為朝廷所廢遂推心委之左驍衛將軍劉海賓涇原都虞候何明禮孔目官岐靈岳皆秀實素所厚也【朝直遥翻驍堅堯翻段秀實鎮涇原時厚遇此三人唐藩鎮吏職使院有孔目官軍府事無細大皆經其手言一孔一目無不綜理也史炤曰岐姓也黄帝時有岐伯 考異曰舊傳云判官岐靈岳今從段公别傳】秀實密與之謀誅泚迎乘輿上初至奉天詔徵近道兵入援有上言朱泚為亂兵所立且來攻城宜早修守備【上時掌翻】盧切齒言曰朱泚忠貞羣臣莫及奈何言其從亂傷大臣心臣請以百口保其不反上亦以為然又聞羣臣勸泚奉迎乃詔諸道援兵至者皆營於三十里外姜公輔諫曰今宿衛單寡防慮不可不深若泚竭忠奉迎何憚於兵多如其不然有備無患上乃悉召援兵入城盧及白志貞言於上曰臣觀朱泚心迹必不至為逆願擇大臣入京城宣慰以察之上以問從臣皆畏憚莫敢行【從才用翻】金吾將軍吴溆獨請行上悦溆退而告人曰食其禄而違其難何以為臣吾幸託肺附【溆敬章皇后弟也溆音徐呂翻難乃旦翻下同】非不知往必死但舉朝無蹈難之臣使聖情慊慊耳【慊慊嫌恨不足之意朝直遥翻慊苦簟翻】遂奉詔詣泚泚反謀已决雖陽為受命館溆於客省【館古玩翻】尋殺之溆湊之兄也泚遣涇原兵馬使韓旻將鋭兵三千聲言迎大駕實襲奉天【使疏吏翻將即亮翻又音如字】時奉天守備單弱段秀實謂岐靈岳曰事急矣使靈岳詐為姚令言符令旻且還當與大軍俱竊令言印未至秀實倒用司農印印符募善走者追之旻至駱驛【駱驛地名史炤曰駱谷關之驛也余按韓旻若至駱谷關之驛則已過奉天而西南矣炤說非也但未知駱驛在何地】得符而還【還從宣翻又音如字】秀實謂同謀曰旻來吾屬無類矣我當直泚殺之不克則死終不能為之臣也乃令劉海賓何明禮隂結軍中之士欲使應之於外【令力丁翻】旻兵至泚令言大驚岐靈岳獨承其罪而死不以及秀實等是日泚召李忠臣源休姚令言及秀實等議稱帝事秀實勃然起奪休象笏【武德初因隋舊制五品以上執象笏三品以下前挫後直五品已上後屈自時厥後一例上圓下方曾不分别】前唾泚面大罵曰狂賊吾恨不斬汝萬段豈從汝反邪【唾吐卧翻邪音耶】因以笏擊泚泚舉手扞之纔中其額濺血灑地泚與秀實相搏恟恟【中竹仲翻恟許救翻恟恟喧擾之狀】左右猝愕不知所為海賓不敢進乘亂而逸忠臣前助泚泚得匍匐脱走秀實知事不成謂泚黨曰我不同汝反何不殺我衆爭前殺之泚一手承血一手止其衆曰義士也勿殺秀實既死泚哭之甚哀以三品禮葬之【唐制司農卿從三品】海賓縗服而逃【劉海賓不能助段秀實與之同死逃將焉往縗倉回翻】後二日捕得殺之 【考異曰段公别傳云五日夜泚使涇原將李忠臣高昂等統鋭兵五千以襲奉天六日賊泚又令兵馬使韓旻領馬步二千以繼之奉天記曰秀實與海賓密謀誅泚佯入請間計事而海賓置首於靴欲以相應為閽者見覺秀實遽奪源休笏挺而擊之舊泚傳曰秀實與劉海賓謀朱泚且虞叛卒之震驚法駕乃潛為賊符追所發兵至六日兵及駱驛而回因與海賓同入見泚為陳逆順之理而海賓於靴中取首為其所覺遂不得前秀實知不可以義動遽奪源休象笏挺而擊泚秀實傳曰與海賓約事急為繼而令明禮應於外及秀實擊泚而海賓等不至按李忠臣等若已將五千人襲奉天則秀實雖追還旻兵無益矣又海賓若於靴中取首為賊所覺則登時死矣焉能復逃若為閽者所覺亦應時被擒事跡諠著賊為之備秀實亦不得發矣此數者皆恐難信今但取段公行狀幸奉天録及舊傳可信者存之】亦不引何明禮明禮從泚攻奉天復謀殺泚亦死【史終言之復扶又翻】上聞秀實死恨委用不至涕泗久之 壬子以少府監李昌巙為京畿渭南節度使【巙奴刀翻】 鳳翔節度使同平章事張鎰性儒緩好修飾邊幅【好呼到翻】不習軍事聞上在奉天欲迎大駕具服用貨財獻于行在後營將李楚琳為人剽悍【將即亮翻剽匹妙翻】軍中畏之嘗事朱泚為泚所厚行軍司馬齊映與同幕齊抗言於鎰曰不去楚琳必為亂首【去羌呂翻】鎰命楚琳出戍隴州【九域志鳳翔府西至隴州一百五十里】楚琳託事不時發鎰方以迎駕為憂謂楚琳已去矣楚琳夜與其黨作亂鎰縋城而走【縋馳偽翻】賊追及殺之判官王沼等皆死映自水竇出抗為傭保負荷而逃皆免【荷下可翻又讀如字 考異曰舊映傳云鎰不從映言乃示寛大召楚琳語之曰欲令公使於外楚琳恐是夜作亂殺鎰以應泚今從鎰傳】始上以奉天迫隘欲幸鳳翔戶部尚書蕭復聞之遽請見【見賢遍翻】曰陛下大誤鳳翔將卒皆朱泚故部曲其中必有與之同惡者臣尚憂張鎰不能久豈得以鑾輿蹈不測之淵乎上曰吾行計已决試為卿留一日【為于偽翻】明日聞鳳翔亂乃止齊映齊抗皆詣奉天以映為御史中丞抗為侍御史楚琳自為節度使降于朱泚隴州刺史郝通奔于楚琳【郝呼各翻】 商州團練兵殺其刺史謝良輔 朱泚自白華殿入宣政殿【東内含元殿之北為宣正殿】自稱大秦皇帝改元應天癸丑泚以姚令言為侍中關内元帥李忠臣為司空兼侍中源休為中書侍郎同平章事判度支蔣鎮為吏部侍郎樊系為禮部侍郎彭偃為中書舍人自餘張光晟等各拜官有差立弟滔為皇太弟姚令言與源休共掌朝政【朝直遥翻下同】凡泚之謀畫遷除軍旅資糧皆禀於休休勸泚誅翦宗室在京城者以絶人望殺郡王王子王孫凡七十七人尋又以蔣鎮為門下侍郎李子平為諫議大夫並同平章事鎮憂懼每懷刀欲自殺又欲亡竄然性怯竟不果源休勸泚誅朝士之竄匿者以脅其餘鎮力救之賴以全者甚衆樊系為泚譔册文既成仰藥而死【樊系距朱泚之命不為譔册不過死耳譔册而死於義何居】大理卿膠水蔣沇詣行在為賊所得沇絶食稱病濳竄得免【沇以轉翻】 哥舒曜食盡弃襄城奔洛陽李希烈陷襄城 右龍武將軍李觀將衛兵千餘人從上於奉天上委之召募數日得五千餘人列之通衢旗鼓嚴整城人為之增氣【為于偽翻】姚令言之東出也【涇州在西故以救襄城為東出】以兵馬使京兆馮河清為涇原留後判官河中姚况知涇州事河清况聞上幸奉天集將士大哭激以忠義甲兵器械百餘車通夕輸行在【通夕而行自晩至旦也】城中方苦無甲兵得之士氣大振詔以河清為四鎮北庭行營涇原節度使况為行軍司馬 上至奉天數日右僕射同平章事崔寧始至上喜甚撫勞有加【崔寧鎮西川有威名危難之中見其至可以鎮安人心故喜甚而撫勞加於他人勞力到翻】寧退謂所親曰主上聰明英武從善如流但為盧所惑以至於此因然出涕【音刪又數板翻】聞之與王翃謀陷之翃言於上曰臣與寧俱出京城寧數下馬便液【數所角翻便液溺也便毗連翻液羊益翻】久之不至有顧望意會朱泚下詔以左丞柳渾同平章事寧為中書令渾襄陽人也時亡在山谷翃使盩厔尉康湛詐為寧遺朱泚書獻之【遺唯季翻】因譖寧與朱泚結盟約為内應故獨後至乙卯上遣中使引寧就幕下云宣密旨二力士自後縊殺之中外皆稱其寃上聞之乃赦其家朱泚遣使遺朱滔書【遺唯季翻】稱三秦之地指日克平大河之北委卿除殄當與卿會于洛陽滔得書宣示軍府移牒諸道以自誇大 上遣中使告難於魏縣行營【魏縣行營馬燧諸軍之討田悦者難乃旦翻】諸將相與慟哭李懷光帥衆赴長安【為李懷光救奉天破朱泚張本帥讀曰率】馬燧李芁各引兵歸鎮【馬燧歸太原李芁歸河陽】李抱眞退屯臨洺 丁巳以戶部尚書蕭復為吏部尚書吏部郎中劉從一為刑部侍郎翰林學士姜公輔為諫議大夫並同平章事 朱泚自將逼奉天軍勢甚盛以姚令言為元帥【泚且禮翻又音此將即亮翻帥所類翻 考異曰奉天記十月十日賊泚自統衆攻奉天以姚令言為都統今從實録舊泚傳】張光晟副之以李忠臣為京兆尹皇城留守仇敬忠為同華等州節度拓東王以扞關東之師李日月為西道先鋒經略使【晟成正翻守式又翻拓逹各翻扞戶旰翻使疎吏翻華戶化翻】邠寧留後韓遊瓌慶州刺史論惟明監軍翟文秀受詔將兵三千拒泚於便橋與泚遇於醴泉遊瓌欲還趣奉天【邠卑旻翻瓌古回翻監古銜翻翟萇伯翻將音同上又音如字還從宣翻又音如字趣七喻翻下同】文秀曰我向奉天賊亦隨至是引賊以迫天子也不若留壁於此賊必不敢越我向奉天若不顧而過則與奉天夾攻之遊懷曰賊彊我弱若賊分軍以綴我直趣奉天奉天兵亦弱何夾攻之有我今急趣奉天所以衛天子也且吾士卒饑寒而賊多財彼以利誘吾卒吾不能禁也【翟文秀欲留拒賊詔旨也夾攻之說兵家常論也挾詔旨而依兵家常論以制將帥未有不折而從之者也微韓遊瓌持之奉天殆矣誘羊久翻】遂引兵入奉天泚亦隨至官軍出戰不利泚兵爭門欲入渾瑊與遊瓌血戰竟日門内有草車數乘【渾戶昆翻又戶本翻瑊古咸翻乘繩證翻】瑊使虞候高固帥甲士以長刀斫賊皆一當百【帥讀曰率】曳車塞門縱火焚之【塞悉則翻】衆軍乘火擊賊賊乃退會夜泚營於城東三里擊柝張火布滿原野使西明寺僧法堅造攻具毁佛寺以為梯衝【西明寺在長安城中延康坊本隋楊素宅也梯雲梯衝衝車代宗飯僧以護國今朱泚乃用僧造攻具以攻奉天柝逹各翻】韓遊瓌曰寺材皆乾薪【乾音干】但具火以待之固侃之玄孫也【高侃事太宗高宗為將有功】泚自是日來攻城瑊遊瓌等晝夜力戰幽州兵救襄城者聞泚反突入潼關歸泚於奉天【幽州兵即代宗時朱泚入朝詣京西防秋兵也】普潤戍卒亦歸之【普潤戍卒神策兵也】有衆數萬上與陸贄語及亂故深自克責贄曰致今日之患皆羣臣之罪也上曰此亦天命非由人事䞇退上疏以為陛下志壹區宇四征不庭【杜預曰不庭謂不朝也下之事上皆成禮于庭中一說庭直也不庭不直也】兇渠稽誅逆將繼亂【兇渠謂田悦李納也逆將謂朱滔李希烈等也渠大也將即亮翻】兵連禍結行及三年【建中二年兵端始啟至是及三年】徵師日滋賦歛日重【歛力贍翻】内自京邑外洎邊陲【洎其既翻】行者有鋒刃之憂居者有誅求之困是以叛亂繼起怨黷並興非常之虞億兆同慮唯陛下穆然凝邃獨不得聞至使兇卒鼓行白晝犯闕豈不以乘我間隙因人擕離哉【間古莧翻】陛下有股肱之臣有耳目之任有諫諍之列有備衛之司見危不能竭其誠臨難不能效其死【難乃旦翻】臣所謂致今日之患羣臣之罪者豈徒言歟聖旨又以國家興衰皆有天命臣聞天所視聽皆因於人【書曰天視自我民視天聽自我民聽】故祖伊責紂之辭曰我生不有命在天武王數紂之罪曰乃曰吾有命罔懲其侮【並見尚書數所類翻】此又捨人事而推天命必不可之理也易曰視履考祥【履卦上九爻辭王弼曰禍福之祥生於所履處履之極履道成矣故可以視履而考祥】又曰吉凶者失得之象【易大傳之辭】此乃天命由人其義明矣然則聖哲之意六經會通皆謂禍福由人不言盛衰有命盖人事理而天命降亂者未之有也人事亂而天命降康者亦未之有也【鄭玄曰降康者下平安之福】自頃征討頗頻刑網稍密物力耗竭人心驚疑如居風濤洶洶靡定上自朝列【朝直遥翻】下逹蒸黎日夕族黨聚謀咸憂必有變故旋屬涇原叛卒果如衆庶所虞【屬之欲翻虞度也】京師之人動逾億計固非悉知算術皆曉占書則明致寇之由未必盡關天命臣聞理或生亂亂或資理【理治也唐人避高宗諱皆以治為理】有以無難而失守有以多難而興邦【難乃旦翻】今生亂失守之事則既往而不可復追矣【復扶又翻】其資理興邦之業在陛下克勵而謹脩之何憂乎亂人何畏於厄運勤勵不息足致升平豈止盪滌妖氛旋復宫闕而已 田悦說王武俊使與馬寔共擊李抱眞於臨洺【魏縣行營既散李抱眞退屯臨洺說式芮翻下林說因說復說同】抱眞復遣賈林說武俊曰臨洺兵精而有備未易輕也【復扶又翻易以豉翻】今戰勝得地則利歸魏博不勝則恒冀大傷易定滄趙皆大夫之故地也【時張孝忠據易定滄康日知據趙州】不如先取之武俊乃辭悦與馬寔北歸壬戌悦送武俊於館陶【九域志館陶在元城北四十五里】執手泣别下至將士贈遺甚厚先是武俊召回紇兵使絶李懷光等糧道【遺于季翻先悉薦翻】懷光等已西去而回紇逹干將回紇千人雜虜二千人適至幽州北境朱滔因說之【將即亮翻說式芮翻下同】欲與俱詣河南取東都應接朱泚許以河南子女賂之滔娶回紇女為側室回紇謂之朱郎且利其俘掠許之賈林復說武俊曰【復扶又翻】自古國家有患未必不因之更興况主上九葉天子【自高祖太宗高宗中宗睿宗玄宗肅宗代宗至帝凡九世】聰明英武天下誰肯捨之共事朱泚乎滔自為盟主以來輕蔑同列河朔古無冀國冀乃大夫之封域也【滔稱冀王盖奄禹跡冀州之域以自大而王武俊廵屬有冀州故林以是間之】今滔稱冀王又西倚其兄【泚者滔兄】北引回紇其志欲盡吞河朔而王之大夫雖欲為之臣不可得矣【田悦之間王武俊朱滔與賈林之說王武俊者同一利害耳人惟趨利而避害故說行非有它巧也】且大夫雄勇善戰非滔之比又本以忠義手誅叛臣【謂殺李惟岳也】當時宰相處置失宜【處昌呂翻】為滔所誑誘故蹉跌至此【蹉倉何翻跌徒結翻】不若與昭義併力取滔其勢必獲滔既亡則泚自破矣此不世之功轉禍為福之道也今諸道輻湊攻泚不日當平天下已定大夫乃悔過而歸國則已晩矣時武俊已與滔有隙因攘袂作色曰二百年天子吾不能臣豈能臣此田舍兒乎遂與抱眞及馬燧相結約為兄弟然猶外事滔禮甚謹與田悦各遣使見滔於河間【瀛州治河間縣】賀朱泚稱尊號且請馬寔之兵共攻康日知於趙州 汝鄭應援使劉德信將子弟軍在汝州【是年四月募諸嘗為節度觀察都團練使子弟帥奴馬從軍使劉德信將之以救襄城】聞難引兵入援【難乃旦翻】與泚衆戰於見子陵破之【新書本紀作思子陵水經注閺鄉縣西皇天原上有漢武帝思子臺又漢薄太后陵在覇陵之南近文帝陵故薄太后曰南望吾子北望吾夫故俗呼為見子陵也】以東渭橋有轉輸積粟癸亥進屯東渭橋【程大昌曰東渭橋在萬年縣北五十里灞水合渭之地】 朱泚夜攻奉天東西南三面甲子渾瑊力戰却之左龍武大將軍李希倩戰死乙丑泚復攻城【復扶又翻】將軍高重捷與泚將李日月戰於梁山之隅破之【梁山在奉天城北五里乾陵在焉重直龍翻】乘勝逐北身先士卒【先悉薦翻】賊伏兵擒之其麾下十餘人奮不顧死追奪之賊不能拒乃斬其首弃其身而去麾下收之入城上親撫而哭之盡哀結蒲為首而葬之贈司空朱泚見其首亦哭之曰忠臣也束蒲為身而葬之李日月泚之驍將也戰死於奉天城下泚歸其尸於長安厚葬之其母竟不哭罵曰奚奴國家何負於汝而反死已晩矣及泚敗賊黨皆族誅獨日月之母不坐己巳加渾瑊京畿渭南北金商節度使 壬申王武俊與馬寔至趙州城下 初朱泚鎮鳳翔遣其將牛雲光將幽州兵五百人戍隴州【宋白曰後魏分涇岐之地置東秦州大統十七年改為隴州因隴山為名】以隴右營田判官韋臯領隴右留後 【考異曰奉天記作鳳翔節度判官今從實録】及郝通奔鳳翔【李楚琳作亂郝通隴州奔歸之】牛雲光詐疾欲俟臯至伏兵執之以應泚事泄帥其衆奔泚至汧陽【汧陽縣屬隴州九域志在州東六十里帥讀曰率汧口肩翻】遇泚遣中使蘇玉齎詔書加臯中丞玉說雲光曰【說式芮翻】韋臯書生也君不如與我俱之隴州臯幸而受命乃吾人也不受命君以兵誅之如取孤㹠耳【㹠與豚同豕子也】雲光從之臯從城上問雲光曰曏者不告而行今而復來何也【復扶又翻下同】雲光曰曏者未知公心今公有新命【謂朱泚加臯中丞之命也】故復來願託腹心臯乃先納蘇玉受其詔書謂雲光曰大使苟無異心請悉納甲兵使城下無疑衆乃可入雲光以臯書生易之【易以豉翻輕易】乃悉以甲兵輸之而入明日臯宴玉雲光及其卒於郡舍伏甲誅之築壇盟將士曰李楚琳賊虐本使【本使謂張鎰也李楚琳鎰之部曲將而殺鎰從逆故云然】既不事上安能恤下【隴州鳳翔廵屬也言李楚琳見虐殺其帥安能恤隴州將士乎】宜相與討之遣兄平弇詣奉天【請命于行在所】復遣使求援於吐蕃【恐朱泚遣兵攻之引吐蕃以為援】<br />
<br />
  資治通鑑卷二百二十八<br />
<br />
<史部,編年類,資治通鑑>  <br>
   </div> 

<script src="/search/ajaxskft.js"> </script>
 <div class="clear"></div>
<br>
<br>
 <!-- a.d-->

 <!--
<div class="info_share">
</div> 
-->
 <!--info_share--></div>   <!-- end info_content-->
  </div> <!-- end l-->

<div class="r">   <!--r-->



<div class="sidebar"  style="margin-bottom:2px;">

 
<div class="sidebar_title">工具类大全</div>
<div class="sidebar_info">
<strong><a href="http://www.guoxuedashi.com/lsditu/" target="_blank">历史地图</a></strong>  
<a href="http://www.880114.com/" target="_blank">英语宝典</a>  
<a href="http://www.guoxuedashi.com/13jing/" target="_blank">十三经检索</a> 
<br><strong><a href="http://www.guoxuedashi.com/gjtsjc/" target="_blank">古今图书集成</a></strong> 
<a href="http://www.guoxuedashi.com/duilian/" target="_blank">对联大全</a> <strong><a href="http://www.guoxuedashi.com/xiangxingzi/" target="_blank">象形文字典</a></strong> 

<br><a href="http://www.guoxuedashi.com/zixing/yanbian/">字形演变</a>  <strong><a href="http://www.guoxuemi.com/hafo/" target="_blank">哈佛燕京中文善本特藏</a></strong>
<br><strong><a href="http://www.guoxuedashi.com/csfz/" target="_blank">丛书&方志检索器</a></strong> <a href="http://www.guoxuedashi.com/yqjyy/" target="_blank">一切经音义</a>  

<br><strong><a href="http://www.guoxuedashi.com/jiapu/" target="_blank">家谱族谱查询</a></strong>  <strong><a href="http://shufa.guoxuedashi.com/sfzitie/" target="_blank">书法字帖欣赏</a></strong> 
<br>

</div>
</div>


<div class="sidebar" style="margin-bottom:0px;">

<font style="font-size:22px;line-height:32px">QQ交流群9:489193090</font>


<div class="sidebar_title">手机APP 扫描或点击</div>
<div class="sidebar_info">
<table>
<tr>
	<td width=160><a href="http://m.guoxuedashi.com/app/" target="_blank"><img src="/img/gxds-sj.png" width="140"  border="0" alt="国学大师手机版"></a></td>
	<td>
<a href="http://www.guoxuedashi.com/download/" target="_blank">app软件下载专区</a><br>
<a href="http://www.guoxuedashi.com/download/gxds.php" target="_blank">《国学大师》下载</a><br>
<a href="http://www.guoxuedashi.com/download/kxzd.php" target="_blank">《汉字宝典》下载</a><br>
<a href="http://www.guoxuedashi.com/download/scqbd.php" target="_blank">《诗词曲宝典》下载</a><br>
<a href="http://www.guoxuedashi.com/SiKuQuanShu/skqs.php" target="_blank">《四库全书》下载</a><br>
</td>
</tr>
</table>

</div>
</div>


<div class="sidebar2">
<center>


</center>
</div>

<div class="sidebar"  style="margin-bottom:2px;">
<div class="sidebar_title">网站使用教程</div>
<div class="sidebar_info">
<a href="http://www.guoxuedashi.com/help/gjsearch.php" target="_blank">如何在国学大师网下载古籍?</a><br>
<a href="http://www.guoxuedashi.com/zidian/bujian/bjjc.php" target="_blank">如何使用部件查字法快速查字?</a><br>
<a href="http://www.guoxuedashi.com/search/sjc.php" target="_blank">如何在指定的书籍中全文检索?</a><br>
<a href="http://www.guoxuedashi.com/search/skjc.php" target="_blank">如何找到一句话在《四库全书》哪一页?</a><br>
</div>
</div>


<div class="sidebar">
<div class="sidebar_title">热门书籍</div>
<div class="sidebar_info">
<a href="/so.php?sokey=%E8%B5%84%E6%B2%BB%E9%80%9A%E9%89%B4&kt=1">资治通鉴</a> <a href="/24shi/"><strong>二十四史</strong></a>&nbsp; <a href="/a2694/">野史</a>&nbsp; <a href="/SiKuQuanShu/"><strong>四库全书</strong></a>&nbsp;<a href="http://www.guoxuedashi.com/SiKuQuanShu/fanti/">繁体</a>
<br><a href="/so.php?sokey=%E7%BA%A2%E6%A5%BC%E6%A2%A6&kt=1">红楼梦</a> <a href="/a/1858x/">三国演义</a> <a href="/a/1038k/">水浒传</a> <a href="/a/1046t/">西游记</a> <a href="/a/1914o/">封神演义</a>
<br>
<a href="http://www.guoxuedashi.com/so.php?sokeygx=%E4%B8%87%E6%9C%89%E6%96%87%E5%BA%93&submit=&kt=1">万有文库</a> <a href="/a/780t/">古文观止</a> <a href="/a/1024l/">文心雕龙</a> <a href="/a/1704n/">全唐诗</a> <a href="/a/1705h/">全宋词</a>
<br><a href="http://www.guoxuedashi.com/so.php?sokeygx=%E7%99%BE%E8%A1%B2%E6%9C%AC%E4%BA%8C%E5%8D%81%E5%9B%9B%E5%8F%B2&submit=&kt=1"><strong>百衲本二十四史</strong></a>  <a href="http://www.guoxuedashi.com/so.php?sokeygx=%E5%8F%A4%E4%BB%8A%E5%9B%BE%E4%B9%A6%E9%9B%86%E6%88%90&submit=&kt=1"><strong>古今图书集成</strong></a>
<br>

<a href="http://www.guoxuedashi.com/so.php?sokeygx=%E4%B8%9B%E4%B9%A6%E9%9B%86%E6%88%90&submit=&kt=1">丛书集成</a> 
<a href="http://www.guoxuedashi.com/so.php?sokeygx=%E5%9B%9B%E9%83%A8%E4%B8%9B%E5%88%8A&submit=&kt=1"><strong>四部丛刊</strong></a>  
<a href="http://www.guoxuedashi.com/so.php?sokeygx=%E8%AF%B4%E6%96%87%E8%A7%A3%E5%AD%97&submit=&kt=1">說文解字</a> <a href="http://www.guoxuedashi.com/so.php?sokeygx=%E5%85%A8%E4%B8%8A%E5%8F%A4&submit=&kt=1">三国六朝文</a>
<br><a href="http://www.guoxuedashi.com/so.php?sokeytm=%E6%97%A5%E6%9C%AC%E5%86%85%E9%98%81%E6%96%87%E5%BA%93&submit=&kt=1"><strong>日本内阁文库</strong></a> <a href="http://www.guoxuedashi.com/so.php?sokeytm=%E5%9B%BD%E5%9B%BE%E6%96%B9%E5%BF%97%E5%90%88%E9%9B%86&ka=100&submit=">国图方志合集</a> <a href="http://www.guoxuedashi.com/so.php?sokeytm=%E5%90%84%E5%9C%B0%E6%96%B9%E5%BF%97&submit=&kt=1"><strong>各地方志</strong></a>

</div>
</div>


<div class="sidebar2">
<center>

</center>
</div>
<div class="sidebar greenbar">
<div class="sidebar_title green">四库全书</div>
<div class="sidebar_info">

《四库全书》是中国古代最大的丛书,编撰于乾隆年间,由纪昀等360多位高官、学者编撰,3800多人抄写,费时十三年编成。丛书分经、史、子、集四部,故名四库。共有3500多种书,7.9万卷,3.6万册,约8亿字,基本上囊括了古代所有图书,故称“全书”。<a href="http://www.guoxuedashi.com/SiKuQuanShu/">详细>>
</a>

</div> 
</div>

</div>  <!--end r-->

</div>
<!-- 内容区END --> 

<!-- 页脚开始 -->
<div class="shh">

</div>

<div class="w1180" style="margin-top:8px;">
<center><script src="http://www.guoxuedashi.com/img/plus.php?id=3"></script></center>
</div>
<div class="w1180 foot">
<a href="/b/thanks.php">特别致谢</a> | <a href="javascript:window.external.AddFavorite(document.location.href,document.title);">收藏本站</a> | <a href="#">欢迎投稿</a> | <a href="http://www.guoxuedashi.com/forum/">意见建议</a> | <a href="http://www.guoxuemi.com/">国学迷</a> | <a href="http://www.shuowen.net/">说文网</a><script language="javascript" type="text/javascript" src="https://js.users.51.la/17753172.js"></script><br />
  Copyright &copy; 国学大师 古典图书集成 All Rights Reserved.<br>
  
  <span style="font-size:14px">免责声明:本站非营利性站点,以方便网友为主,仅供学习研究。<br>内容由热心网友提供和网上收集,不保留版权。若侵犯了您的权益,来信即刪。scp168@qq.com</span>
  <br />
ICP证:<a href="http://www.beian.miit.gov.cn/" target="_blank">鲁ICP备19060063号</a></div>
<!-- 页脚END --> 
<script src="http://www.guoxuedashi.com/img/plus.php?id=22"></script>
<script src="http://www.guoxuedashi.com/img/tongji.js"></script>

</body>
</html>
