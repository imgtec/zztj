<!DOCTYPE html PUBLIC "-//W3C//DTD XHTML 1.0 Transitional//EN" "http://www.w3.org/TR/xhtml1/DTD/xhtml1-transitional.dtd">
<html xmlns="http://www.w3.org/1999/xhtml">
<head>
<meta http-equiv="Content-Type" content="text/html; charset=utf-8" />
<meta http-equiv="X-UA-Compatible" content="IE=Edge,chrome=1">
<title>資治通鑒_205-資治通鑑卷二百四_205-資治通鑑卷二百四</title>
<meta name="Keywords" content="資治通鑒_205-資治通鑑卷二百四_205-資治通鑑卷二百四">
<meta name="Description" content="資治通鑒_205-資治通鑑卷二百四_205-資治通鑑卷二百四">
<meta http-equiv="Cache-Control" content="no-transform" />
<meta http-equiv="Cache-Control" content="no-siteapp" />
<link href="/img/style.css" rel="stylesheet" type="text/css" />
<script src="/img/m.js?2020"></script> 
</head>
<body>
 <div class="ClassNavi">
<a  href="/24shi/">二十四史</a> | <a href="/SiKuQuanShu/">四库全书</a> | <a href="http://www.guoxuedashi.com/gjtsjc/"><font  color="#FF0000">古今图书集成</font></a> | <a href="/renwu/">历史人物</a> | <a href="/ShuoWenJieZi/"><font  color="#FF0000">说文解字</a></font> | <a href="/chengyu/">成语词典</a> | <a  target="_blank"  href="http://www.guoxuedashi.com/jgwhj/"><font  color="#FF0000">甲骨文合集</font></a> | <a href="/yzjwjc/"><font  color="#FF0000">殷周金文集成</font></a> | <a href="/xiangxingzi/"><font color="#0000FF">象形字典</font></a> | <a href="/13jing/"><font  color="#FF0000">十三经索引</font></a> | <a href="/zixing/"><font  color="#FF0000">字体转换器</font></a> | <a href="/zidian/xz/"><font color="#0000FF">篆书识别</font></a> | <a href="/jinfanyi/">近义反义词</a> | <a href="/duilian/">对联大全</a> | <a href="/jiapu/"><font  color="#0000FF">家谱族谱查询</font></a> | <a href="http://www.guoxuemi.com/hafo/" target="_blank" ><font color="#FF0000">哈佛古籍</font></a> 
</div>

 <!-- 头部导航开始 -->
<div class="w1180 head clearfix">
  <div class="head_logo l"><a title="国学大师官网" href="http://www.guoxuedashi.com" target="_blank"></a></div>
  <div class="head_sr l">
  <div id="head1">
  
  <a href="http://www.guoxuedashi.com/zidian/bujian/" target="_blank" ><img src="http://www.guoxuedashi.com/img/top1.gif" width="88" height="60" border="0" title="部件查字,支持20万汉字"></a>


<a href="http://www.guoxuedashi.com/help/yingpan.php" target="_blank"><img src="http://www.guoxuedashi.com/img/top230.gif" width="600" height="62" border="0" ></a>


  </div>
  <div id="head3"><a href="javascript:" onClick="javascript:window.external.AddFavorite(window.location.href,document.title);">添加收藏</a>
  <br><a href="/help/setie.php">搜索引擎</a>
  <br><a href="/help/zanzhu.php">赞助本站</a></div>
  <div id="head2">
 <a href="http://www.guoxuemi.com/" target="_blank"><img src="http://www.guoxuedashi.com/img/guoxuemi.gif" width="95" height="62" border="0" style="margin-left:2px;" title="国学迷"></a>
  

  </div>
</div>
  <div class="clear"></div>
  <div class="head_nav">
  <p><a href="/">首页</a> | <a href="/ShuKu/">国学书库</a> | <a href="/guji/">影印古籍</a> | <a href="/shici/">诗词宝典</a> | <a   href="/SiKuQuanShu/gxjx.php">精选</a> <b>|</b> <a href="/zidian/">汉语字典</a> | <a href="/hydcd/">汉语词典</a> | <a href="http://www.guoxuedashi.com/zidian/bujian/"><font  color="#CC0066">部件查字</font></a> | <a href="http://www.sfds.cn/"><font  color="#CC0066">书法大师</font></a> | <a href="/jgwhj/">甲骨文</a> <b>|</b> <a href="/b/4/"><font  color="#CC0066">解密</font></a> | <a href="/renwu/">历史人物</a> | <a href="/diangu/">历史典故</a> | <a href="/xingshi/">姓氏</a> | <a href="/minzu/">民族</a> <b>|</b> <a href="/mz/"><font  color="#CC0066">世界名著</font></a> | <a href="/download/">软件下载</a>
</p>
<p><a href="/b/"><font  color="#CC0066">历史</font></a> | <a href="http://skqs.guoxuedashi.com/" target="_blank">四库全书</a> |  <a href="http://www.guoxuedashi.com/search/" target="_blank"><font  color="#CC0066">全文检索</font></a> | <a href="http://www.guoxuedashi.com/shumu/">古籍书目</a> | <a   href="/24shi/">正史</a> <b>|</b> <a href="/chengyu/">成语词典</a> | <a href="/kangxi/" title="康熙字典">康熙字典</a> | <a href="/ShuoWenJieZi/">说文解字</a> | <a href="/zixing/yanbian/">字形演变</a> | <a href="/yzjwjc/">金 文</a> <b>|</b>  <a href="/shijian/nian-hao/">年号</a> | <a href="/diming/">历史地名</a> | <a href="/shijian/">历史事件</a> | <a href="/guanzhi/">官职</a> | <a href="/lishi/">知识</a> <b>|</b> <a href="/zhongyi/">中医中药</a> | <a href="http://www.guoxuedashi.com/forum/">留言反馈</a>
</p>
  </div>
</div>
<!-- 头部导航END --> 
<!-- 内容区开始 --> 
<div class="w1180 clearfix">
  <div class="info l">
   
<div class="clearfix" style="background:#f5faff;">
<script src='http://www.guoxuedashi.com/img/headersou.js'></script>

</div>
  <div class="info_tree"><a href="http://www.guoxuedashi.com">首页</a> > <a href="/SiKuQuanShu/fanti/">四库全书</a>
 > <h1>资治通鉴</h1> <!--         下载:【右键另存为】即可 --></div>
  <div class="info_content zj clearfix">
  
<div class="info_txt clearfix" id="show">
<center style="font-size:24px;">205-資治通鑑卷二百四</center>
    資治通鑑卷二百四   宋 司馬光 撰<br />
<br />
  胡三省 音註<br />
<br />
  唐紀二十【起彊圉大淵獻盡重光單闕凡五年】<br />
<br />
  則天順聖皇后上之下<br />
<br />
  垂拱三年春閏正月丁卯封皇子成美為恒王【睿宗時為帝故成美等皆稱皇子恒戶登翻 考異曰唐歷舊本紀新傳皆作成義今從實録】隆基為楚王隆範為衛王隆業為趙王 二月丙辰突厥骨篤禄等寇昌平【昌平後漢縣屬廣陽國隋屬涿郡唐屬幽州】命左鷹揚大將軍黑齒常之帥諸軍討之【帥讀曰率】 三月乙丑納言韋思謙以太中大夫致仕 夏四月命蘇良嗣留守西京【守式又翻考異曰實録新舊本紀統紀皆無良嗣出守西京年月今據唐歷】時尚方監裴匪躬檢校京苑【光宅改少府監為尚方監京苑西京之苑】將鬻苑中蔬果以收其利良嗣曰昔公儀休相魯猶能拔葵去織婦【董仲舒曰公儀休相魯之其家見織帛怒而出其妻食於舍而茹葵愠而拔其葵曰吾已食禄又奪園夫紅女利乎相息亮翻去羌呂翻】未聞萬乘之主鬻蔬果也【乘繩證翻】乃止 壬戍裴居道為納言五月丙寅夏官侍郎京兆張光輔為鳳閣侍郎同平章事鳳閣侍郎同鳳閣鸞臺三品劉禕之竊謂鳳閣舍人<br />
<br />
  永年賈大隱曰【永年漢本曲梁縣魏為廣平郡治所隋廢郡為廣平縣後改為雞澤仁夀元年改曰永元避太子廣諱也唐帶洺州】太后既廢昏立明安用臨朝稱制【朝直遥翻】不如返正以安天下之心大隱密奏之太后不悦謂左右曰禕之我所引【劉禕之自北門學士至為相故云然】乃復叛我或誣禕之受歸誠州都督孫萬榮金【貞觀二十一年以契丹别部置歸誠州屬松漠都督府復扶又反】又與許敬宗妾有私太后命肅州刺史王本立推之本立宣勑示之禕之曰不經鳳閣鸞臺何名為勑太后大怒以為拒捍制使【使疏吏翻】庚午賜死于家禕之初下獄睿宗為之上疏申理【下遐嫁翻為音于偽翻上時掌翻】親友皆賀之禕之曰此乃所以速吾死也臨刑沐浴神色自若自草謝表立成數紙麟臺郎郭翰【光宅改祕書郎為麟臺郎】太子文學周思鈞【太子宫司經局有太子文學三人正六品掌侍奉文章】稱歎其文太后聞之左遷翰巫州司法思鈞播州司倉【貞觀八年以辰州龍標縣置巫州九年以隋牂柯郡牂柯縣置播州舊志巫州京師南四千一百九十七里東都二千九百里播州京師南四千四百五十里東都四千九百六十里】 秋七月壬辰魏玄同檢校納言嶺南俚戶舊輸半課交趾都護劉延祐使之全輸俚<br />
<br />
  戶不從延祐誅其魁首其黨李思慎等作亂攻破安南府城【高宗調露元年改交州都督府為安南都護府俚音里】殺延祐桂州司馬曹玄静將兵討思慎等斬之【將即亮翻 考異曰舊書馮元常傳云元常自眉州刺史轉廣州都督屬安南首領李嗣仙殺都督劉延祐剽陷州縣勑元常誅之帥士卒濟南海先馳檄示以威恩喻以禍福嗣仙徒黨多相帥歸降因縱兵誅其魁首安慰居人而旋今從實録】 突厥骨篤祿元珍寇朔州遣燕然道大摠管黑齒常之擊之【燕因肩翻】以左鷹揚大將軍李多祚為之副大破突厥於黄花堆【意即黄瓜堆按朔州有黄花堆在神武川】追奔四十餘里突厥皆散走磧北【走音奏磧七迹翻】多祚世為靺鞨酋長【靺鞨音末曷酋慈由翻長知兩翻】以軍功得入宿衛黑齒常之每得賞賜皆分將士有善馬為軍士所損官屬請笞之常之曰奈何以私馬笞官兵乎卒不問【卒子恤翻】 九月己卯虢州人楊初成詐稱郎將【將即亮翻下同】矯制於都市募人迎廬陵王於房州事覺伏誅 冬十月庚子右監門衛中郎將㸑寶璧與突厥骨篤禄元珍戰全軍皆没寶璧輕騎遁歸【監古銜翻騎奇寄翻】寶璧見黑齒常之有功表請窮追餘寇詔與常之計議遥為聲援寶璧欲專其功不待常之引精卒萬三千人先行出塞二千餘里掩擊其部落既至又先遣人告之使得嚴備與戰遂敗太后誅寶璧改骨篤祿曰不卒祿【卒子恤翻】 命魏玄同留守西京【守手又翻】 武承嗣又使人誣李孝逸自云名中有兎兎月中物當有天分【謂有分為天子分扶問翻】太后以孝逸有功十一月戊寅减死除名流儋州而卒【儋州舊儋耳郡武德五年置儋州舊志儋州至京師七千四百四十二里儋徒甘翻卒子恤翻 考異曰新紀天授元年殺梁公李孝逸孝逸初封梁郡公以平徐敬業功改封吳國公垂拱三年减死除名配流儋州當削爵矣新傳云流儋州薨紀傳自相違唐歷云四月十一日誅益州長史李孝逸亦舊任也統紀誅李孝逸并其黨崔元昉裴安期唐歷并其黨崔知賢董元昉裴安期等今從實録及舊傳】 太后欲遣韋待價將兵擊吐蕃 【考異曰實録十二月丙辰命待價為安息道大摠管督二十六摠管以討吐蕃不言師出勝敗如何至永昌元年五月又云命待價擊吐蕃七月敗於寅識迦河按本傳不云兩曾將兵今刪此事】鳳閣侍郎韋方質奏請如舊制遣御史監軍【監古銜翻】太后曰古者明君遣將閫外之事悉以委之比聞御史監軍【比毗至翻】軍中事無大小皆須承禀以下制上非令典也且何以責其有功遂罷之 是歲天下大飢山東關内尤甚四年春正月甲子於神都立高祖太宗高宗三廟四時享祀如西廟之儀【西廟西京宗廟也】又立崇先廟以享武氏祖考太后命有司議崇先廟室數司禮博士周悰請為七室【光宅改太常曰司禮史言周悰之請希旨迎合】又减唐太廟為五室春官侍郎賈大隱奏禮天子七廟諸侯五廟百王不易之義今周悰别引浮議廣述異聞直崇臨朝權儀【朝直遥翻】不依國家常度皇太后親承顧託光顯大猷其崇先廟室應如諸侯之數國家宗廟不應輒有變移太后乃止 太宗高宗之世屢欲立明堂諸儒議其制度不决而止及太后稱制獨與北門學士議其制不問諸儒諸儒以為明堂當在國陽丙巳之地三里之外七里之内太后以為去宫太遠二月庚午毁乾元殿於其地作明堂以僧懷義為之使【使疏吏翻】凡役數萬人 夏四月戊戌殺太子通事舍人郝象賢【唐制太子通事舍人正七品下掌導引宫臣辭見及勞問之事】象賢處俊之孫也初太后有憾於處俊【上元二年諫高宗】會奴誣告象賢反太后命周興鞫之致象賢族罪象賢家人詣朝堂訟寃於監察御史樂安任玄殖【樂安郡棣州朝直遥翻監古銜翻】玄殖奏象賢無反狀玄殖坐免官象賢臨刑極口罵太后發揚宫中隱慝奪市人柴以擊刑者金吾兵共格殺之太后命支解其尸發其父祖墳毁棺焚尸自是終太后之世法官每刑人先以木丸塞其口【塞悉則翻】 武承嗣使鑿白石為文曰聖母臨人永昌帝業末紫石雜藥物填之庚午使雍州人唐同泰奉表獻之【隋京兆郡武德元年改曰雍州雍於用翻】稱獲之於洛水太后喜命其石曰寶圖擢同泰為遊擊將軍五月戊辰詔當親拜洛受寶圖有事南郊告謝昊天禮畢御明堂朝羣臣【朝直遥翻】命諸州都督刺史及宗室外戚以拜洛前十日集神都乙亥太后加尊號為聖母神皇 六月丁亥朔日有食之 壬寅作神皇三璽【璽斯氏翻】 東陽大長公主削封邑并二子徙巫州【公主太宗之女長知兩翻】公主適高履行太后以高氏長孫無忌之舅族故惡之【惡烏路翻】 江南道廵撫大使冬官侍郎狄仁傑以吳楚多淫祠奏焚其一千七百餘所獨留夏禹吳太伯季札伍員四祠【員音云】 秋七月丁巳赦天下更命寶圖為天授聖圖洛水為永昌洛水【更工衡翻】封其神為顯聖侯加特進禁漁釣祭祀比四瀆【唐制嶽瀆為中祀】名圖所出曰聖圖泉泉側置永昌縣又改嵩山為神嶽封其神為天中王拜太師使持節神嶽大都督禁芻牧【使疏吏翻】又以先於汜水得瑞石改汜水為廣武【汜水漢之成臯縣屬河南郡後魏為成臯郡置東中府隋開皇十八年改成臯為汜水屬鄭州縣界有廣武楚漢對壘處后改縣名以協其姓汜音祀】太后潜謀革命稍除宗室絳州刺史韓王元嘉青州刺史霍王元軌邢州刺史魯王靈夔豫州刺史越王貞【豫州漢汝南郡地後魏置豫州唐因之然唐之豫州非能盡得漢汝南郡之地】及元嘉子通州刺史黄公譔【譔士免翻又音銓】元軌子金州刺史江都王緒虢王鳳子申州刺史東莞公融靈夔子范陽王藹貞子博州刺史琅邪王冲在宗室中皆以才行有美名【行下孟翻】太后尤忌之元嘉等内不自安密有匡復之志 【考異曰舊傳垂拱三年七月誤也今從實録】譔謬為書與貞云内人病浸重當速療之若至今冬恐成痼疾太后召宗室朝明堂【朝直遥翻】諸王因逓相驚曰神皇欲於大饗之際使人告密盡收宗室誅之無遺譔詐為皇帝璽書與冲云朕遭幽縶諸王宜各發兵救我冲又詐為皇帝璽書云神皇欲移李氏社稷以授武氏【璽斯氏翻】八月壬寅冲召長史蕭悳琮等令募兵 【考異曰實録作丙午蓋據奏到之日也舊傳本紀作壬寅按冲以戊申死而實録又云冲起兵七日而敗然則壬寅是也今從之】分告韓霍魯越及貝州刺史紀王慎令各起兵共趣神都【趣七喻翻】太后聞之以左金吾將軍丘神勣為清平道行軍大摠管以討之【博州有清平縣漢貝丘縣也隋更名】冲募兵得五千餘人欲度河取濟州先擊武水【武水漢東郡陽平縣地隋改為清邑又分清邑置武水縣唐屬博州濟子禮翻】武水令郭務悌詣魏州求救莘令馬玄素【莘亦漢陽平縣地後齊改曰樂平隋開皇六年復曰陽平八年改曰清邑十六年置莘州大業初州廢為莘縣唐屬魏州】將兵千七百人中道邀冲恐力不敵入武水閉門拒守冲推草車塞其南門【推吐雷翻塞悉則翻】因風縱火焚之欲乘火突入火作而風回冲軍不得進由是氣沮堂邑董玄寂【堂邑漢縣後魏廢隋分清陽縣復置屬博州】為冲將兵擊武水【為于偽翻】謂人曰琅邪王與國家交戰此乃反也冲聞之斬玄寂以徇衆懼而散入草澤不可禁止惟家僮左右數十人在冲還走博州【走音奏】戊申至城門為守門者所殺 【考異曰丘神勣傳云為勲官吳希智白丁孟青棒所殺今從實録及冲傳】凡起兵七日而敗丘神勣至博州官吏素服出迎神勣盡殺之凡破千餘家越王貞聞冲起亦舉兵於豫州遣兵陷上蔡【上蔡縣漢屬汝南郡後魏曰臨汝隋開皇初改曰武冿大業初曰上蔡唐屬豫州九域志在州北五十五里】九月丙辰命左豹韜大將軍麴崇裕為中軍大摠管岑長倩為後軍大摠管將兵十萬以討之【將即亮翻下同】又命張光輔為諸軍節度削冲屬籍更姓虺氏【更工衡翻下更其同】貞聞冲敗欲自鏁詣闕謝罪會所署新蔡令傅延慶【新蔡縣自漢以來屬汝南郡唐屬豫州九域志在州東一百六十里】募得勇士二千餘人貞乃宣言於衆曰琅邪已破魏相數州【相息亮翻】有兵二十萬朝夕至矣發屬縣兵共得五千分為五營使汝南縣丞裴守悳等將之【汝南縣舊曰上蔡隋大業初改曰汝陽帶豫州】署九品以上官五百餘人所署官皆受迫脅莫有鬪志惟守德與之同謀貞以其女妻之【妻七細翻】署大將軍委以腹心貞使道士及僧誦經以求事成左右及戰士皆帶辟兵符麴崇裕等軍至豫州城東四十里貞遣少子規及裴守德拒戰兵潰而歸【少詩照翻】貞大懼閉閤自守崇裕等至城下左右謂貞曰王豈可坐待戮辱貞規守德及其妻皆自殺 【考異曰實録庚戌貞舉兵九月丙寅豫州平又云舉兵二十日而敗庚戍至丙寅纔十七日蓋皆據奏到之日耳】與冲皆梟首東都闕下初范陽王藹遣使語貞及冲曰【梟監堯翻使疏吏翻語牛倨翻下我語同】若四方諸王一時並起事無不濟諸王往來相約結未定而冲先發惟貞狼狽應之諸王皆不敢發故敗貞之將起兵也遣使告夀州刺史趙瓌瓌妻常樂長公主【常樂公主高祖女使疏吏翻下同樂音洛長知兩翻】謂使者曰為我語越王【為于偽翻下似為同語牛倨翻】昔隋文帝將簒周室尉遲迴周之甥也猶能舉兵匡救社稷【事見一百七十四卷陳宣帝大建十四年尉紆勿翻】功雖不成威震海内足為忠烈况汝諸王先帝之子豈得不以社稷為心今李氏危若朝露汝諸王不捨生取義尚猶豫不發欲何須耶【須待也】禍且至矣大丈夫當為忠義鬼無為徒死也及貞敗太后欲悉誅韓魯等諸王命監察御史藍田蘇珦按其密狀珦訊問皆無明驗或告珦與韓魯通謀太后召珦詰之珦抗論不回【監古銜翻下同珦式亮翻詰去吉翻】太后曰卿大雅之士朕當别有任使此獄不必卿也乃命珦於河西監軍更使周興等按之【更工衡翻下同】於是收韓王元嘉魯王靈夔黄公譔常樂公主於東都迫脅皆自殺【考異曰舊傳靈夔流振州自縊死今從實録】更其姓曰虺親黨皆誅以文昌左丞狄仁傑為豫州刺史【光宅改尚書左丞為文昌左丞】時治越王貞黨與【治直之翻】當坐者六七百家藉没者五千口司刑趣使行刑【司刑寺即大理寺趣讀曰促】仁傑密奏彼皆詿誤【詿戶卦翻】臣欲顯奏似為逆人申理知而不言恐乖陛下仁恤之旨太后特原之皆流豐州道過寧州寧州父老迎勞之曰我狄使君活汝耶【勞力到翻仁傑刺寧州見上卷垂拱二年】相擕哭於悳政碑下設齋三日而後行時張光輔尚在豫州將士恃功多所求取仁傑不之應光輔怒曰州將輕元帥邪【將即亮翻帥所類翻】仁傑曰亂河南者一越王貞耳【河南當作汝南】今一貞死萬貞生光輔詰其語仁傑曰明公總兵三十萬所誅者止於越王貞城中聞官軍至踰城出降者四面成蹊【降戶江翻】明公縱將士暴掠殺已降以為功流血丹野非萬貞而何恨不得尚方斬馬劒加於明公之頸雖死如歸耳光輔不能詰歸奏仁傑不遜左遷復州刺史【自緊州左遷上州且自近州遷遠州也舊志豫州去京師一千五百四十里至東都六百七十里復州京師東南一千八百里至東都一千五百一十八里】 丁卯左肅政大夫騫味道夏官侍郎王本立並同平章事 太后之召宗室朝明堂也東莞公融密遣使問成均助教高子貢【朝直遥翻莞音官使疏吏翻下同】子貢曰來必死融乃稱疾不赴越王貞起兵遣使約融融倉猝不能應為官屬所逼執使者以聞擢拜右贊善大夫【唐東宫左右贊善大夫正五品上掌傳令諷過失贊禮儀以經教授諸郡王】未幾為支黨所引【幾居豈翻】冬十月己亥戮於市籍没其家高子貢亦坐誅濟州刺史薛顗顗弟緒緒弟駙馬都尉紹皆與琅邪王冲通謀【濟子禮翻顗魚豈翻】顗聞冲起兵作兵器募人冲敗殺録事參軍高纂以㓕口【唐武德初改州主簿為録事參軍掌正違失涖符印】十一月辛酉顗緒伏誅紹以太平公主故杖一百餓死於獄【紹以主婿免誅死】十二月乙酉司徒青州刺史霍王元軌坐與越王連謀廢徙黔州【舊志黔州京師南三千一百九十三里至東都三千二百七十七里黔音琴】載以檻車行至陳倉而死江都王緒殿中監郕公裴承先皆戮於市承先寂之孫也【裴寂武德開國功臣】 命裴居道留守西京【守式又翻】 左肅政大夫同平章事騫味道素不禮於殿中待御史周矩屢言其不能了事會有羅告味道者勑矩按之矩謂味道曰公常責矩不了事今日為公了之【為于偽翻】乙亥味道及其子辭玉皆伏誅 【考異曰御史臺記味道陷周興獄今從實録】 己酉太后拜洛受圖【受唐同泰所獻偽石也】皇帝皇太子皆從【從才用翻】内外文武百官蠻夷各依方叙立珍禽奇獸雜寶列於壇前文物鹵簿之盛唐興以來未之有也辛亥明堂成高二百九十四尺方三百尺凡三層下層法四時各隨方色中層法十二辰上為圓蓋九龍捧之上層法二十四氣亦為圓蓋上施鐵鳳高一丈【高居滶翻】飾以黄金中有巨木十圍上下通貫栭櫨㮰藉以為本【栭音而梁上柱說文曰屋枅上標櫨音盧柱上柎曰樽櫨廣韻枅也又曰柱也抽庚翻斜柱也㮰婢脂翻屋梠也】下施鐵渠為辟雍之象【以鐵為渠以通水】號曰萬象神宫宴賜羣臣赦天下縱民入觀改河南為合宫縣又於明堂北起天堂五級以貯大像【懷義所作夾紵大像也貯丁呂翻】至三級則俯視明堂矣 【考異曰舊薛懷義傳云明堂大屋凡三層計高三百尺又於明堂北起天堂廣袤亞明堂今從小說及通典】僧懷義以功拜左威衛大將軍梁國公【考異曰實録云懷義監造明堂以功擢授左武衛大將軍固辭不拜時有右玉鈐衛將軍王慈徵長上果毅元肅然請與懷義為兒既而隂有異圖欲奉之為主懷義密奏其狀由是慈徵等坐斬進拜懷義輔國大將軍封盧國公賜物三千段又表辭不受今從舊傳】侍御史王求禮上書曰古之明堂茅茨不翦采椽不斵今者飾以珠玉塗以丹青鐵鷟入雲【上時掌翻鷟士角翻鸑鷟者鳳也】金龍隱霧昔殷辛瓊臺夏癸瑶室無以加也【殷辛紂也夏癸桀也】太后不報 太后欲發梁鳳巴蜑【蜑徒旱翻】自雅州開山通道出擊生羌因襲吐蕃【貞觀五年太宗置西雅州以處生羌八年去西字吐從入聲】正字陳子昂上書【上時掌翻】以為雅州邊羌自國初以來未嘗為盗今一旦無罪戮之其怨必甚且懼誅㓕必蜂起為盗西山盗起【西山在城都西松茂二州都督府所統諸州皆西山羌也】則蜀之邊邑不得不連兵備守兵久不解臣愚以為西蜀之禍自此結矣臣聞吐蕃愛蜀富饒欲盗之久矣徒以山川阻絶障隘不通勢不能動今國家乃亂邊羌開隘道使其收奔亡之種為鄉導以攻邊是借寇兵為賊除道舉全蜀以遺之也【種章勇翻鄉讀曰嚮為賊于偽翻遺于季翻】蜀者國家之寶庫可以兼濟中國今執事者乃圖僥幸之利以事西羌【僥工堯翻】得其地不足以稼穡財不足以富國徒為糜費無益聖德况其成敗未可知哉夫蜀之所恃者險也【夫音扶】人之所以安者無役也今國家乃開其險役其人險開則便寇人役則傷財臣恐未見羌戎已有姧盗在其中矣且蜀人尫劣【尫烏黄翻弱也】不習兵戰山川阻曠去中夏遠【夏戶雅翻】今無故生西羌吐蕃之患臣見其不及百年蜀為戎矣國家近廢安北拔單于弃龜兹放踈勒【廢安北拔單子以突厥畔援也弃龜兹放踈勒以吐蕃侵逼也單音蟬龜兹音丘慈又音屈佳】天下翕然謂之盛德者蓋以陛下務在養人不在廣地也今山東飢關隴弊而徇貪夫之議謀動甲兵興大役自古國亡家敗未嘗不由黷兵願陛下熟計之既而役不果興<br />
<br />
  永昌元年春正月乙卯朔大饗萬象神宫太后服兖冕搢大圭執鎮圭為初獻【周禮注大圭長三尺杼上終葵首天子服之鎮圭尺有二寸天子守之鎮圭飾四鎮山象其高圭中約以組防其墜齊人謂槌為終葵圭首六寸為槌以下殺之】皇帝為亞獻太子為終獻先詣昊天上帝座次高祖太宗高宗次魏國先王【魏國先王武士彠也】次五方帝座太后御則天門赦天下改元丁巳太后御明堂受朝賀【朝直遥翻】戊午布政于明堂頒九條以訓百官己未御明堂饗羣臣 二月丁酉尊魏忠孝王曰周忠孝太皇妣曰忠孝太后文水陵曰章悳陵咸陽陵曰明義陵【武氏之先葬文水士彠及其妻葬咸陽】置崇先府官戊戌尊魯公曰太原靖王北平王曰趙肅恭王金城王曰魏義康王太原王曰周安成王 三月甲子張光輔守納言 壬申太后問正字陳子昂當今為政之要子昂退上疏【上時掌翻】以為宜緩刑崇德息兵革省賦役撫慰宗室各使自安辭婉意切其論甚美凡三千言 癸酉以天官尚書武承嗣為納言張光輔守内史夏四月甲辰殺辰州别駕汝南王煒連州别駕鄱陽<br />
<br />
  公諲等宗室十二人徙其家於嶲州煒惲之子諲元慶之子也【蔣王惲太宗子道王元慶高祖子也煒于鬼翻諲音因惲於粉翻】己酉殺天官侍郎藍田鄧玄挺玄挺女為諲妻又與煒善諲謀迎中宗於廬陵以問玄挺煒又嘗謂玄挺曰欲為急計何如玄挺皆不應故坐知反不告同誅 五月丙辰命文昌右相韋待價為安息道行軍大摠管擊吐蕃 浪穹州蠻酋傍時昔等二十五部先附吐蕃至是來降【酋慈由翻降戶江翻】以傍時昔為浪穹州刺史令統其衆【南詔六部號為六詔浪穹詔其一也】 己巳以僧懷義為新平軍大摠管 【考異曰舊傳為清平道大摠管今從實録 余按新平豳州軍出州而北伐也】北討突厥行至紫河【隋志定襄郡大利縣有隂山有紫河即太宗遣思摩建牙之地杜佑曰勝州榆林縣有余河紫河自馬邑郡善陽縣界流入】不見虜於單于臺刻石紀功而還【還從宣翻又如字】 諸王之起兵也貝州刺史紀王慎獨不預謀亦坐繫獄秋七月丁巳檻車徙巴州更姓虺氏行及蒲州而卒【紀王慎徙巴州蓋令取道相衛自河北路西上不得至東都歷絳至蒲而卒更工衡翻卒子恤翻】八男徐州刺史東平王續等相繼被誅【被皮義翻 考異曰舊傳云慎長子和州刺史東平王續最知名早卒今從實録】家徙嶺南女東光縣主楚媛幼以孝謹稱適司議郎裴仲將相敬如賓姑有疾親嘗藥膳接遇娣姒皆得歡心【杜預曰兄弟之妻相謂曰姒蓋妯娌相呼以身年長少為名年長為姒少為娣不以夫之長幼也俗以兄之妻為姒弟為娣非也爾雅曰長婦謂稚婦為娣婦娣婦謂長婦為姒婦媛于眷翻】時宗室諸女皆以驕奢相尚誚楚媛獨儉素曰所貴於富貴者得適志也今獨守勤苦將以何求楚媛曰幼而好禮今而行之非適志歟觀自古女子皆以恭儉為美縱侈為惡辱親是懼何所求乎富貴儻來之物何足驕人衆皆慙服及慎凶問至楚媛號慟嘔血數升【好呼到翻號戶高翻】免喪不御膏沐者垂二十年 韋待價軍至寅識迦河【據舊書待價傳寅識迦河當在弓月西南】與吐蕃戰大敗待價既無將領之才【將即亮翻】狼狽失據士卒凍餒死亡甚衆乃引軍還太后大怒丙子待價除名流繡州【繡州漢阿林縣地至隋猶屬欝林郡唐武德四年分置林州六年改曰繡州去長安六千九十里至東都五千五百里】斬副大摠管安西大都護閻温古安西副都護唐休璟收其餘衆撫安西土【璟俱永翻】太后以休璟為西州都督 戊寅以王本立同鳳閣鸞臺三品 徐敬業之敗也【事見上卷光宅元年】弟敬真流繡州逃歸將奔突厥過洛陽【厥九勿翻】洛州司馬弓嗣業【孫愐曰弓姓也】洛陽令張嗣明資遣之至定州為吏所獲嗣業縊死【縊於計翻】嗣明敬真多引海内知識云有異圖冀以免死於是朝野之士為所連引坐死者甚衆【朝直遥翻】嗣明誣内史張光輔云征豫州日私論圖䜟天文隂懷兩端【謂征越王貞時】八月甲申光輔與敬真嗣明等同誅籍没其家乙未秋官尚書太原張楚金陜州刺史郭正一鳳閣侍郎元萬頃洛陽令魏元忠並免死流嶺南【陜失冉翻】楚金等皆為敬真所引云與敬業通謀臨刑太后使鳳閣舍人王隱客馳騎傳聲赦之聲逹於市當刑者皆喜躍讙呼【騎奇寄翻讙讀如諠】宛轉不已元忠獨安坐自如或使之起元忠曰虛實未知隱客至又使起元忠曰俟宣勑已既宣勑乃徐起舞蹈再拜竟無憂喜之色是日隂雲四塞既釋楚金等天氣晴霽【塞悉則翻 考異曰唐歷七月二十四日張楚金絞死八月二十一日郭正一絞死年代紀七月甲戌楚金絞死新書紀八月辛丑殺郭正一今據實錄楚金等皆流配未死舊書楚金正一萬頃傳皆云流嶺南御史臺紀云元忠將刑至于市神色自若則天以楊楚功免死流放復叙授御史中丞復陷來俊臣獄復至市將刑神色如初其傍諸王子戮者三十餘尸重疊委積元忠顧視曰大丈夫少選居此積矣曾不介懷會鳳閣舍人王隱客馳騎傳呼勑罷刑復放嶺南又云前後坐弃市流放者四舊傳云前後三被流今從舊傳】九月壬子以僧懷義為新平道行軍大摠管將兵二<br />
<br />
  十萬討突厥骨篤禄【將即亮翻】 初高宗之世周興以河陽令召見【河陽縣自漢以來屬河内郡唐屬懷州又屬孟州見賢遍翻】上欲加擢用或奏以為非清流罷之【周興發身於尚書都事流外官也】興不知數於朝堂俟命【數所角翻朝直遥翻】諸相皆無言【相息亮翻】地官尚書檢校納言魏玄同時同平章事【光宅改戶部為地官】謂之曰周明府可去矣【唐人呼縣今為明府】興以為玄同沮已銜之玄同素與裴炎善時人以其終始不渝謂之耐久朋周興奏誣玄同言太后老矣不若奉嗣君為耐久【為于偽翻下為長同】太后怒閏月甲午賜死于家監刑御史房濟謂玄同曰丈人何不告密冀得召見可以自直【見賢遍翻】玄同歎曰人殺鬼殺亦復何殊【復扶又翻】豈能作告密人邪乃就死又殺夏官侍郎崔詧於隱處【光宅改兵部為夏官】自餘内外大臣坐死及流貶者甚衆彭州長史劉易從【彭州漢繁縣之地宋置晉夀郡故城在州北三里梁置東益州後魏置天水郡仍改繁縣為九隴縣仍置濛州隋省唐武德初復置尋省併益州垂拱二年復分置彭州易以䜴翻】亦為徐敬真所引戊申就州誅之易從為人仁孝忠謹將刑於市吏民憐其無辜遠近奔赴競解衣投地曰為長史求冥福【為于偽翻】有司平凖直十餘萬周興等誣右武衛大將軍燕公黑齒常之謀反徵下獄【燕因肩翻下遐嫁翻】冬十月戊午常之縊死【縊於計翻】己未殺宗室鄂州刺史嗣鄭王璥等六人【鄂州春秋夏汭之地江夏記云一名夏口一名魯口吳始築郡城晉末始立郢州隋平陳改為鄂州因鄂渚為名璥居影翻璥鄭王元懿之子 考異曰唐歷云撫州别駕舊傳璥作敬今從新本紀】庚申嗣滕王脩琦等六人免死流嶺南 【考異曰統紀云元嬰男脩瑶等五人免死配流今從舊傳】 丁卯春官尚書范履氷鳳閣侍郎邢文偉並同平章事 己卯詔太穆神皇后文德聖皇后宜配皇地祗忠孝太后從配【太后尊其母為忠孝太后從才用翻】 右衛胄曹參軍陳子昂【唐諸衛府皆有胄曹參軍掌戎仗器械及公廨興造决罰之事】上疏以為周頌成康漢稱文景皆以能措刑故也【上時掌翻】今陛下之政雖盡善矣然太平之朝上下樂化不宜有亂臣賊子日犯天誅比者大獄增多逆徒滋廣【朝直遥翻樂音洛比毗至翻】愚臣頑昧初謂皆實乃去月十五日陛下特察繫囚李珍等無罪百僚慶悦皆賀聖明臣乃知亦有無罪之人挂於踈網者陛下務在寛典獄官務在急刑以傷陛下之仁以誣太平之政臣竊恨之又九月二十一日勑免楚金等死初有風雨變為景雲臣聞隂慘者刑也陽舒者德也聖人法天天亦助聖天意如此陛下豈可不承順之哉今又隂雨臣恐過在獄官凡繫獄之囚多在極法道路之議或是或非陛下何不悉召見之自詰其罪【詰去吉翻】罪有實者顯示明刑濫者嚴懲獄吏使天下咸服人知政刑豈非至德克明哉<br />
<br />
  天授元年【是年九月方改元天授】十一月庚辰朔日南至太后享萬象神宫赦天下始用周正改永昌元年十一月為載初元年正月以十二月為臘月夏正月為一月以周漢之後為二王後舜禹成湯之後為三恪【古者建國有賓有恪二王之後賓也待以客禮師古曰恪敬也待之加敬亦如賓也鄭玄以二王三恪通為五代後人多祖其說唐本以後周及隋後為二王後今改之】周隋之嗣同列國【此周謂後周】 鳳閣侍郎河東宗秦客【河東蒲州】改造天地等十二字以獻【十二字照為曌天為西地為埊日為乚月為田星為○君為□臣為人為載為□年為□正為击又有證為□為聖二字】丁亥行之太后自名曌改詔曰制【避后名也】秦客太后從父姊之子也【從才用翻】 乙未司刑少卿周興奏除唐親屬籍臘月辛未以僧懷義為右衛大將軍賜爵鄂國公 春一月戊子武承嗣遷文昌左相岑長倩遷文昌右相同鳳閣鸞臺三品鳳閣侍郎武攸寧為納言邢文偉守内史左肅政大夫同鳳閣鸞臺三品王本立罷為地官尚書攸寧士彠之兄孫也【彠一虢翻】時武承嗣三思用事宰相皆下之【下遐嫁翻】地官尚書同鳳閣鸞臺三品韋方質有疾承嗣三思往問之方質據牀不為禮或諫之方質曰死生有命大丈夫安能曲事近戚以求苟免乎尋為周興等所搆甲午流儋州籍没其家【儋都甘翻】 二月辛酉太后策貢士於洛城殿【六典洛城南門之西有麗景夾城自此潜通上陽宫洛城南門之内有洛城殿】貢士殿試自此始 丁卯地官尚書王本立薨 【考異曰新紀丁卯殺王本立御史臺記木立為周興所誅今從實録】 三月丁亥特進同鳳閣鸞臺三品蘇良嗣薨 夏四月丁巳春官尚書同平章事范履氷坐嘗舉犯逆者下獄死 【考異曰新紀五月戊子殺范履氷今從實録唐歷】 醴泉人侯思止【醴泉漢池陽谷口之地後魏置寧夷縣隋開皇十八年改曰醴泉屬雍州】始以賣餅為業後事游擊將軍高元禮為僕素詭譎無賴恒州刺史裴貞杖一判司【唐謂州曹諸司參軍為判司韓愈詩所謂判司卑官不堪說未免箠楚塵埃間是也恒戶登翻】判司使思止告貞與舒王元名謀反秋七月辛巳元名坐廢徙和州【舊志和州京師東南二千六百八十三里至東都一千八百一十一里】壬午殺其子豫章王亶貞亦族㓕擢思止為遊擊將軍時吿密者往往得五品思止求為御史太后曰卿不識字豈堪御史對曰獬豸何嘗識字但能觸邪耳【異物志東北荒中有獸名獬豸一角性忠直見人鬭則觸不直者聞人論則咋不直者獬胡賣翻豸宅賣翻】太后悦即以為朝散大夫侍御史【朝直遥翻散悉亶翻】它日太后以先所籍没宅賜之思止不受曰臣惡反逆之人【惡烏路翻】不願居其宅太后益賞之衡水人王弘義素無行【行下孟翻】嘗從隣舍乞瓜不與乃告縣官瓜田中有白兎縣官使人搜捕蹂踐瓜田立盡【蹂人九翻踐息淺翻】又遊趙貝見閭里耆老作邑齋遂告以謀反殺二百餘人擢授游擊將軍俄遷殿中侍御史或告勝州都督王安仁謀反勑弘義按之安仁不服弘義即於枷上刎其首又捕其子適至亦刎其首函之以歸道過汾州司馬毛公與之對食須臾叱毛公下階【下遐嫁翻】斬之槍揭其首入洛見者無不震栗【揭其謁翻】時置制獄於麗景門内【唐六典曰洛城南門之西有麗景夾城自此潜通於上陽宫又曰洛陽皇城西面二門南曰麗景北曰宣耀】入是獄者非死不出弘義戲呼曰例竟門【竟盡也言入此門者例盡其命也劉昫曰言入此門者例皆竟也】朝士人人自危相見莫敢交言道路以目或因入朝密遭掩捕每朝輒與家人訣曰未知復相見否【朝直遥翻復扶又翻】時法官競為深酷唯司刑丞徐有功杜景儉【司刑丞即大理丞 考異曰實録及新紀表傳皆作景佺蓋實録以草書致誤新書因承之耳今從舊統紀】獨存平恕被告者皆曰遇來侯必死遇徐杜必生有功文遠之孫也【徐文遠見一百八十五卷高祖武德元年】名弘敏以字行初為蒲州司法【唐制法曹司法參軍掌鞫獄麗法督盗賊知贓賄没入】以寛為治【治直吏翻】不施敲朴吏相約有犯徐司法杖者衆共斥之迨官滿不杖一人職事亦修累遷司刑丞酷吏所誣搆者有功皆為直之【為于偽翻下右為同】前後所活數十百家嘗廷争獄事太后厲色詰之【結去吉翻】左右為戰栗有功神色不撓【撓奴教翻】争之彌切太后雖好殺【好呼到翻】知有功正直甚敬憚之景儉武邑人也【武邑漢縣屬信都郡後漢晉屬安平郡後魏屬武邑郡隋唐屬冀州】司刑丞滎陽李日知亦尚平恕少卿胡元禮欲殺一囚日知以為不可往復數四元禮怒曰元禮不離刑曹此囚終無生理日知曰日知不離刑曹此囚終無死法竟以兩狀列上【離力知翻上時掌翻下同】日知果直 東魏國寺僧法明等撰大雲經四卷表上之【撰士免翻】言太后乃彌勒佛下生當代唐為閻浮提主【釋氏以人世為閻浮提】制頒於天下 武丞嗣使周興羅告隋州刺史澤王上金【隋州春秋隨子之國漢為隨縣屬南陽郡後魏置隋州舊志隋州京師東南一千三百八十八里至東都一千八里】舒州刺史許王素節謀反徵詣行在【舊志舒州京師東南二千六百二十六里至東都一千八百九十三里】素節發舒州聞遭喪哭者歎曰病死何可得乃更哭邪丁亥至龍門【龍門山在洛州河南縣界】縊殺之【縊於計翻】上金自殺悉誅其諸子及支黨 太后欲以太平公主妻其伯父士讓之孫攸暨【垂拱四年誅薛紹太平公主寡居妻七細翻下同】攸暨時為右衛中郎將【將即亮翻】太后潜使人殺其妻而妻之公主方額廣頤多權略太后以為類已寵愛特厚常與密議天下事舊制食邑諸王不過千戶公主不過三百五十戶太平食邑獨累加至三千戶【此食實戶也若唐制以品為差則累於是劉昫曰唐制公主食封三百戶長公主加五百戶有至六百戶高宗以太平公主武后所生逾於舊制垂拱中大平公主至一千二百戶聖歷初至三千戶景雲初增至五千戶】 八月甲寅殺太子少保納言裴居道癸亥殺尚書左丞張行廉辛未殺南安王潁等宗室十二人又鞭殺故太子賢二子唐之宗室於是殆盡矣其幼弱存者亦流嶺南又誅其親黨數百家 【考異曰實録作數千家今從舊本紀】惟千金長公主以巧媚得全自請為太后女仍改姓武氏太后愛之更號延安大長公主【長知兩翻下同更工衡翻】九月丙子侍御史汲人傅遊藝【汲縣漢屬河内郡晉以來帶汲郡東魏置】<br />
<br />
  【義州隋廢為汲縣貞觀初移衛州治焉】帥關中百姓九百餘人詣闕上表【帥讀曰率上時掌翻下同】請改國號曰周賜皇帝姓武氏太后不許擢遊藝為給事中於是百官及帝室宗戚遠近百姓四夷酋長沙門道士合六萬餘人【酋慈由翻】俱上表如遊藝所請皇帝亦上表自請賜姓武氏戊寅羣臣上言有鳳皇自明堂飛入上陽宫還集左臺梧桐之上【左臺左肅政臺也】久之飛東南去及赤雀數萬集朝堂【朝直遥翻】庚辰太后可皇帝及羣臣之請壬午御則天樓【則天門樓也】赦天下以唐為周改元【改元天授】乙酉上尊號曰聖神皇帝以皇帝為皇嗣賜姓武氏以皇太子為皇孫丙戌立武氏七廟于神都追尊周文王曰始祖文皇帝妣姒氏曰文定皇后【姒太姒也姓譜姒受姓自鯀】平王少子武曰睿神康皇帝妣姜氏曰康睿皇后【后遠祖姬周誣神甚矣文王其肯饗非鬼之祭乎】太原靖王曰嚴祖成皇帝妣曰成莊皇后趙肅恭王曰肅祖章敬皇帝魏義康王曰烈祖昭安皇帝周安成王曰顯祖文穆皇帝忠孝太皇曰太祖孝明高皇帝妣皆如考謚稱皇后立武承嗣為魏王三思為梁王攸寧為建昌王士彠兄孫攸歸重規載德攸暨懿宗嗣宗攸宜攸望攸緒攸止皆為郡王諸姑姊皆為長公主【彠一虢翻長知兩翻】又以司賓卿溧陽史務滋為納言【光宅改鴻臚為司賓溧陽縣漢屬丹陽郡江左因之隋平陳廢丹陽郡以溧陽縣屬宣州溧音栗】鳳閣侍郎宗秦客檢校内史給事中傅遊藝為鸞臺侍郎平章事遊藝與岑長倩右玉鈐衛大將軍張䖍朂左金吾大將軍丘神勣侍御史來子珣等並賜姓武秦客潜勸太后革命故首為内史遊藝朞年之中歷衣青綠朱紫【一年之間自九品歷至三品衣於既翻】時人謂之四時仕宦勑改州為郡或謂太后曰陛下始革命而廢州不祥【以州周同音也】太后遽追止之命史務滋等十人廵撫諸道太后立兄孫延基等六人為郡王 冬十月甲子檢校内史宗秦客坐贓貶遵化尉【遵化縣屬欽州隋開皇二十年置】弟楚客亦以姧贓流嶺外 丁卯殺流人韋方質 【考異曰舊傳云配流儋州尋卒今從統紀新本紀】 辛未内史邢文偉坐附會宗秦客貶珍州刺史【珍州漢夜郎郡地貞觀十六年開山洞以舊播州城置珍州及夜郎縣以縣界有隆珍山因名舊志珍州至京師四千一百里東都二千七百里】頃之有制使至州【以奉制出使故謂之制使猶言詔使也使疏吏翻】文偉以為誅己遽自縊死【縊於計翻】 壬申勑兩京諸州各置大雲寺一區藏大雲經使僧升高座講解其撰疏僧雲宣等九人皆賜爵縣公【撰士免翻疏所去翻】仍賜紫袈裟銀龜袋【西域胡僧衣毛衣謂之袈裟流入中國以繒帛為之常僧皆衣緇惟賜紫者乃得衣紫袈音加裟音沙唐制給品官以隨身魚符以明貴賤應徵召高宗給五品以上以隨身銀魚袋以防召命之詐出内必合之三品以上金飾袋垂拱中都督刺史始賜魚天授二年改佩魚皆佩龜其後三品已上龜袋飾以金四品以銀五品以銅中宗初罷龜袋復給以魚】 制天下武氏咸蠲課役西突厥十姓自垂拱以來為東突厥所侵掠【東突厥謂骨篤祿等】散亡略盡濛池都護繼往絶可汗斛瑟羅收其餘衆六七萬人入居内地拜右衛大將軍改號竭忠事主可汗【可從刋入聲汗音寒】 道州刺史李行褒兄弟為酷吏所陷當族秋官郎中徐有功固爭不能得秋官侍郎周興奏有功出反囚當斬 【考異曰新舊傳有功爭行褒皆在爭裴行本下按行本得罪在長夀元年一月時周興已貶死矣行褒坐謀復李氏必在革命後今置此年之末】太后雖不許亦免有功官然太后雅重有功久之復起為侍御史【復扶又翻下掌復同】有功伏地流涕固辭曰臣聞鹿走山林而命懸庖厨勢使之然也陛下以臣為法官臣不敢枉陛下法必死是官矣太后固授之遠近聞者相賀 是歲以右衛大將軍泉獻誠為左衛大將軍太后出金寶命選南北牙善射者五人賭之獻誠第一以讓右玉鈐衛大將軍薛咄摩咄摩復讓獻誠【咄當没翻】獻誠乃奏言陛下令選善射者今多非漢官竊恐四夷輕漢【泉獻誠高麗泉勇生之子薛咄摩薛延陁之種故云然】請停此射太后善而從之<br />
<br />
  二年正月癸酉朔太后始受尊號於萬象神宫【漢哀帝自稱陳聖劉太平皇帝尊號蓋昉於此太后以女主而受尊號尤為非古是後玄宗自先天三年至天寶十三載五十年間六受徽號人主遂視為故常矣】旗幟尚赤【幟昌志翻】甲戌改置社稷於神都辛巳納武氏神主子太廟唐太廟之在長安者更命曰享悳廟【更工衡翻 考異曰案實録此年三月己卯改唐太廟為享德廟據此已祔武氏七廟主不當至三月方改唐廟新本紀元年十月辛未改唐大廟為享德廟以武氏七廟為太廟今從唐統紀】四時唯享高祖已下餘四室皆閉不享【四室宣帝元帝光帝景帝也】又改長安崇先廟為崇尊廟【垂拱四年立崇先廟】乙酉日南至大享明堂祀昊天上帝百神從祀【從才用翻】武氏祖宗配饗唐三帝亦同配 御史中丞知大夫事李嗣真以酷吏縱横【横戶孟翻】上疏以為今告事紛紜虛多實少【上時掌翻少詩沼翻】恐有凶慝隂謀離間陛下君臣【間古莧翻】古者獄成公卿參聽王必三宥然後行刑【記王制成獄辭史以獄成告于正正聽之正以獄成告于大司寇大司寇聽之于棘木之下大司寇以獄之成告于王王命三公參聽之三公以獄之成告于王王三宥然後制刑】比日獄官單車奉使推鞫既定法家依斷不令重推【比毗至翻斷丁亂翻重直龍翻】或臨時專决不復聞奏如此則權由臣下非審慎之法儻有寃濫何由可知况以九品之官專命推覆操殺生之柄竊人主之威案覆既不在秋官省審復不由門下【復扶又翻操七刀翻省悉景翻】國之利器輕以假人恐為社稷之禍太后不聽 饒陽尉姚貞亮等數百人表請上尊號曰上聖大神皇帝不許 侍御史來子珣誣尚衣奉御劉行感兄弟謀反皆坐誅 春一月地官尚書武思文及朝集使二千八百人表請封中嶽【中嶽嵩山朝直遥翻】己亥廢唐興寧永康隱陵署官【元帝陵曰興寧景帝陵曰永康興寧陵在咸陽水康陵在三原北十八里唐諸陵有署令一人從五品上府二人史四人主衣四人主輦四人主藥四人典事三人掌固二人又有陵令一人掌山陵率陵戶守衛之丞為之貳】唯量置守戶【量音良】左金吾大將軍丘神勣以罪誅 納言史務滋與來俊臣同鞫劉行感獄俊臣奏務滋與行感親密意欲寢其反狀太后命俊臣并推之務滋恐懼自殺 或告文昌右丞周興與丘神勣通謀太后命來俊臣鞫之俊臣與興方推事對食謂興曰囚多不承當為何法興曰此甚易耳【易以䜴翻】取大甕以炭四周炙之令囚入中何事不承俊臣乃索大甕火圍如興法【索山客翻】因起謂興曰有内狀推兄請兄入此甕興惶恐叩頭伏罪法當死太后原之二月流興嶺南在道為仇家所殺興與索元禮來俊臣競為暴刻【索昔各翻】興元禮所殺各數千人俊臣所破千餘家元禮殘酷尤甚太后亦殺之以慰人望 徙左衛大將軍千乘王武攸暨為定王【乘繩證翻】 立故太子賢之子光順為義豐王 【考異曰舊傳為安樂王今從唐歷統紀】 甲子太后命始祖墓曰德陵睿祖墓曰喬陵嚴祖墓曰節陵肅祖墓曰簡陵烈祖墓曰靖陵顯祖墓曰永陵改章德陵為昊陵顯義陵為順陵 追復李君羨官爵【君羨誅見一百九十九卷太宗貞觀二十二年】 夏四月壬寅朔日有食之 癸卯制以釋教開革命之階【謂大雲經也】升於道教之上 命建安王攸宜留守長安【守式又翻】 丙辰鑄大鍾置北闕 五月以岑長倩為武威道行軍大摠管擊吐蕃中道召還軍竟不出六月以左肅政大夫格輔元為地官尚書【姓譜格姓允格之後束觀漢紀有侍御史東平相格班】與鸞臺侍郎樂思晦鳳閣侍郎任知古並同平章事思晦彦暐之子也【樂彦暐見二百卷高宗顯慶元年】 秋七月徙關内戶數十萬以實洛陽 八月戊申納言武攸寧罷為左羽林大將軍夏官尚書歐陽通為司禮卿【光宅改太常為司禮】兼判納言事 庚申殺玉鈐衛大將軍張䖍朂來俊臣鞫䖍朂獄䖍朂自訟於徐有功俊臣怒命衛士以刀亂斫殺之梟首于市【鈐其廉翻梟堅堯翻】 義豐王光順嗣雍王守禮永安王守義長信縣主等皆賜姓武氏【唐制嗣王郡王從一品光順兄弟皆章懷太子賢之子嗣祥吏翻雍於用翻】與睿宗諸子皆幽閉宫中不出門庭者十餘年守禮守義光順之弟也或告地官尚書武思文初與徐敬業通謀甲子流思文於嶺南復姓徐氏【思文改姓見上卷光宅元年】 九月乙亥殺岐州刺史雲弘嗣來俊臣鞫之不問一欵【獄辭之出於囚口者為欵欵誠也言所吐者皆誠實也】先斷其首乃偽立案奏之【斷尊管翻案考也據也獄辭之成者曰案言可考據也凡官文書可考據者皆曰案】其殺張䖍朂亦然勑旨皆依海内鉗口【鉗其廉翻】 鸞臺侍郎同平章事傅遊藝夢登湛露殿以語所親【語牛倨翻】所親告之壬辰下獄自殺【下遐嫁翻】 癸巳以左羽林衛大將軍建昌王武攸寧為納言洛州司馬狄仁傑為地官侍郎與冬官侍郎裴行本並同平章事太后謂仁傑曰卿在汝南甚有善政【謂垂拱四年刺豫州時也】卿欲知譛卿者名乎仁傑謝曰陛下以臣為過臣請改之知臣無過臣之幸也不願知譛者名太后深歎美之先是鳳閣舍人脩武張嘉福【先悉薦翻修武漢山陽縣地修武古地名也魏隋以名縣唐屬懷州杜佑日懷州修武縣本殷甯邑韓詩外傳曰武王伐紂勒兵於甯故曰修武漢山陽縣故城在縣西北】使洛陽人王慶之等數百人上表【上時掌翻考異曰御史臺記作千餘人今從舊傳】請立武承嗣為皇太子文昌右相同鸞臺鳳閣三品岑長倩以皇嗣在東宫不宜有此議【嗣祥吏翻相息亮翻】奏請切責上書者告示令散太后又問地官尚書同平章事格輔元輔元固稱不可由是大忤諸武意【令力丁翻忤五故翻】故斥長倩令西征吐蕃未至徵還下制獄【下遐嫁翻】承嗣又譛輔元來俊臣又脅長倩子靈原令引司禮卿兼判納言事歐陽通等數十人皆云同反通為俊臣所訊五毒備至終無異詞俊臣乃詐為通欵冬十月己酉長倩輔元通等皆坐誅王慶之見太后【見賢遍翻下同】太后曰皇嗣我子奈何廢之慶之對曰神不歆非類民不祀非族【左傳晉大夫狐突之言】今誰有天下而以李氏為嗣乎太后諭遣之慶之伏地以死泣請不去太后乃以印紙遺之曰【遺唯李翻】欲見我以此示門者自是慶之屢求見【見賢遍翻】太后頗怒之命鳳閣侍郎李昭德賜慶之杖昭德引出光政門外【洛陽宫城南面三門中曰應天左曰興教右曰光政】以示朝士曰此賊欲廢我皇嗣立武承嗣命撲之【朝直遥翻撲弼角翻】耳目皆血出然後杖殺之 【考異曰舊傳云延載初鳳閣舍人張嘉福令洛陽人王慶之率輕薄惡少數百人詣闕上表請立武承嗣為皇太子則天不許唐歷昭德永昌元年自御史中丞貶振州凌水尉實錄長夀元年始為相舊傳杖殺慶之在為相後按御史臺記昭德自中丞轉鳳閣侍郎蓋暫貶凌水尉尋召還為鳳閣侍郎也杖殺慶之據御史臺記乃是為鳳閣侍郎時非為相後也舊傳或以載初為延載慶之上表或在載初年實錄因岑長倩格輔元之死說及耳今參取實錄御史臺記及舊傳之語】其黨乃散昭德因言於太后曰天皇陛下之夫皇嗣陛下之子陛下身有天下當傳之子孫為萬代業豈得以姪為嗣乎自古未聞姪為天子而為姑立廟者也【而為于偽翻】且陛下受天皇顧託若以天下與承嗣則天皇不血食矣太后亦以為然昭德乾祐之子也【李乾祐即貞觀初敕裴仁軌者也】 壬辰殺鸞臺侍郎同平章事樂思晦右衛將軍李安静安静綱之孫也【李綱以剛直著節隋唐之間安静可謂無忝厥祖矣】太后將革命王公百官皆上表勸進安静獨正色拒之及下制獄來俊臣詰其反狀【上時掌翻下遐嫁翻詰去吉翻】安静曰以我唐家老臣須殺即殺若問謀反實無可對俊臣竟殺之 太學生王循之上表乞假還鄉【假古訝翻休假也唐國子學生三百人太學生五百人】太后許之狄仁傑曰臣聞君人者唯殺生之柄不假人自餘皆歸之有司故左右丞徒以下不句【句音鉤】左右相流以上乃判為其漸貴故也【相息亮翻為于偽翻下為之普為同】彼學生求假丞簿事耳【唐國子監丞六品下掌判監事主簿從七品下】若天子為之發勑則天下之事幾勑可盡乎必欲不違其願請普為立制而已太后善之<br />
<br />
  資治通鑑卷二百四<br />
<br />
<史部,編年類,資治通鑑>  <br>
   </div> 

<script src="/search/ajaxskft.js"> </script>
 <div class="clear"></div>
<br>
<br>
 <!-- a.d-->

 <!--
<div class="info_share">
</div> 
-->
 <!--info_share--></div>   <!-- end info_content-->
  </div> <!-- end l-->

<div class="r">   <!--r-->



<div class="sidebar"  style="margin-bottom:2px;">

 
<div class="sidebar_title">工具类大全</div>
<div class="sidebar_info">
<strong><a href="http://www.guoxuedashi.com/lsditu/" target="_blank">历史地图</a></strong>  
<a href="http://www.880114.com/" target="_blank">英语宝典</a>  
<a href="http://www.guoxuedashi.com/13jing/" target="_blank">十三经检索</a> 
<br><strong><a href="http://www.guoxuedashi.com/gjtsjc/" target="_blank">古今图书集成</a></strong> 
<a href="http://www.guoxuedashi.com/duilian/" target="_blank">对联大全</a> <strong><a href="http://www.guoxuedashi.com/xiangxingzi/" target="_blank">象形文字典</a></strong> 

<br><a href="http://www.guoxuedashi.com/zixing/yanbian/">字形演变</a>  <strong><a href="http://www.guoxuemi.com/hafo/" target="_blank">哈佛燕京中文善本特藏</a></strong>
<br><strong><a href="http://www.guoxuedashi.com/csfz/" target="_blank">丛书&方志检索器</a></strong> <a href="http://www.guoxuedashi.com/yqjyy/" target="_blank">一切经音义</a>  

<br><strong><a href="http://www.guoxuedashi.com/jiapu/" target="_blank">家谱族谱查询</a></strong>  <strong><a href="http://shufa.guoxuedashi.com/sfzitie/" target="_blank">书法字帖欣赏</a></strong> 
<br>

</div>
</div>


<div class="sidebar" style="margin-bottom:0px;">

<font style="font-size:22px;line-height:32px">QQ交流群9:489193090</font>


<div class="sidebar_title">手机APP 扫描或点击</div>
<div class="sidebar_info">
<table>
<tr>
	<td width=160><a href="http://m.guoxuedashi.com/app/" target="_blank"><img src="/img/gxds-sj.png" width="140"  border="0" alt="国学大师手机版"></a></td>
	<td>
<a href="http://www.guoxuedashi.com/download/" target="_blank">app软件下载专区</a><br>
<a href="http://www.guoxuedashi.com/download/gxds.php" target="_blank">《国学大师》下载</a><br>
<a href="http://www.guoxuedashi.com/download/kxzd.php" target="_blank">《汉字宝典》下载</a><br>
<a href="http://www.guoxuedashi.com/download/scqbd.php" target="_blank">《诗词曲宝典》下载</a><br>
<a href="http://www.guoxuedashi.com/SiKuQuanShu/skqs.php" target="_blank">《四库全书》下载</a><br>
</td>
</tr>
</table>

</div>
</div>


<div class="sidebar2">
<center>


</center>
</div>

<div class="sidebar"  style="margin-bottom:2px;">
<div class="sidebar_title">网站使用教程</div>
<div class="sidebar_info">
<a href="http://www.guoxuedashi.com/help/gjsearch.php" target="_blank">如何在国学大师网下载古籍?</a><br>
<a href="http://www.guoxuedashi.com/zidian/bujian/bjjc.php" target="_blank">如何使用部件查字法快速查字?</a><br>
<a href="http://www.guoxuedashi.com/search/sjc.php" target="_blank">如何在指定的书籍中全文检索?</a><br>
<a href="http://www.guoxuedashi.com/search/skjc.php" target="_blank">如何找到一句话在《四库全书》哪一页?</a><br>
</div>
</div>


<div class="sidebar">
<div class="sidebar_title">热门书籍</div>
<div class="sidebar_info">
<a href="/so.php?sokey=%E8%B5%84%E6%B2%BB%E9%80%9A%E9%89%B4&kt=1">资治通鉴</a> <a href="/24shi/"><strong>二十四史</strong></a>&nbsp; <a href="/a2694/">野史</a>&nbsp; <a href="/SiKuQuanShu/"><strong>四库全书</strong></a>&nbsp;<a href="http://www.guoxuedashi.com/SiKuQuanShu/fanti/">繁体</a>
<br><a href="/so.php?sokey=%E7%BA%A2%E6%A5%BC%E6%A2%A6&kt=1">红楼梦</a> <a href="/a/1858x/">三国演义</a> <a href="/a/1038k/">水浒传</a> <a href="/a/1046t/">西游记</a> <a href="/a/1914o/">封神演义</a>
<br>
<a href="http://www.guoxuedashi.com/so.php?sokeygx=%E4%B8%87%E6%9C%89%E6%96%87%E5%BA%93&submit=&kt=1">万有文库</a> <a href="/a/780t/">古文观止</a> <a href="/a/1024l/">文心雕龙</a> <a href="/a/1704n/">全唐诗</a> <a href="/a/1705h/">全宋词</a>
<br><a href="http://www.guoxuedashi.com/so.php?sokeygx=%E7%99%BE%E8%A1%B2%E6%9C%AC%E4%BA%8C%E5%8D%81%E5%9B%9B%E5%8F%B2&submit=&kt=1"><strong>百衲本二十四史</strong></a>  <a href="http://www.guoxuedashi.com/so.php?sokeygx=%E5%8F%A4%E4%BB%8A%E5%9B%BE%E4%B9%A6%E9%9B%86%E6%88%90&submit=&kt=1"><strong>古今图书集成</strong></a>
<br>

<a href="http://www.guoxuedashi.com/so.php?sokeygx=%E4%B8%9B%E4%B9%A6%E9%9B%86%E6%88%90&submit=&kt=1">丛书集成</a> 
<a href="http://www.guoxuedashi.com/so.php?sokeygx=%E5%9B%9B%E9%83%A8%E4%B8%9B%E5%88%8A&submit=&kt=1"><strong>四部丛刊</strong></a>  
<a href="http://www.guoxuedashi.com/so.php?sokeygx=%E8%AF%B4%E6%96%87%E8%A7%A3%E5%AD%97&submit=&kt=1">說文解字</a> <a href="http://www.guoxuedashi.com/so.php?sokeygx=%E5%85%A8%E4%B8%8A%E5%8F%A4&submit=&kt=1">三国六朝文</a>
<br><a href="http://www.guoxuedashi.com/so.php?sokeytm=%E6%97%A5%E6%9C%AC%E5%86%85%E9%98%81%E6%96%87%E5%BA%93&submit=&kt=1"><strong>日本内阁文库</strong></a> <a href="http://www.guoxuedashi.com/so.php?sokeytm=%E5%9B%BD%E5%9B%BE%E6%96%B9%E5%BF%97%E5%90%88%E9%9B%86&ka=100&submit=">国图方志合集</a> <a href="http://www.guoxuedashi.com/so.php?sokeytm=%E5%90%84%E5%9C%B0%E6%96%B9%E5%BF%97&submit=&kt=1"><strong>各地方志</strong></a>

</div>
</div>


<div class="sidebar2">
<center>

</center>
</div>
<div class="sidebar greenbar">
<div class="sidebar_title green">四库全书</div>
<div class="sidebar_info">

《四库全书》是中国古代最大的丛书,编撰于乾隆年间,由纪昀等360多位高官、学者编撰,3800多人抄写,费时十三年编成。丛书分经、史、子、集四部,故名四库。共有3500多种书,7.9万卷,3.6万册,约8亿字,基本上囊括了古代所有图书,故称“全书”。<a href="http://www.guoxuedashi.com/SiKuQuanShu/">详细>>
</a>

</div> 
</div>

</div>  <!--end r-->

</div>
<!-- 内容区END --> 

<!-- 页脚开始 -->
<div class="shh">

</div>

<div class="w1180" style="margin-top:8px;">
<center><script src="http://www.guoxuedashi.com/img/plus.php?id=3"></script></center>
</div>
<div class="w1180 foot">
<a href="/b/thanks.php">特别致谢</a> | <a href="javascript:window.external.AddFavorite(document.location.href,document.title);">收藏本站</a> | <a href="#">欢迎投稿</a> | <a href="http://www.guoxuedashi.com/forum/">意见建议</a> | <a href="http://www.guoxuemi.com/">国学迷</a> | <a href="http://www.shuowen.net/">说文网</a><script language="javascript" type="text/javascript" src="https://js.users.51.la/17753172.js"></script><br />
  Copyright &copy; 国学大师 古典图书集成 All Rights Reserved.<br>
  
  <span style="font-size:14px">免责声明:本站非营利性站点,以方便网友为主,仅供学习研究。<br>内容由热心网友提供和网上收集,不保留版权。若侵犯了您的权益,来信即刪。scp168@qq.com</span>
  <br />
ICP证:<a href="http://www.beian.miit.gov.cn/" target="_blank">鲁ICP备19060063号</a></div>
<!-- 页脚END --> 
<script src="http://www.guoxuedashi.com/img/plus.php?id=22"></script>
<script src="http://www.guoxuedashi.com/img/tongji.js"></script>

</body>
</html>
