<!DOCTYPE html PUBLIC "-//W3C//DTD XHTML 1.0 Transitional//EN" "http://www.w3.org/TR/xhtml1/DTD/xhtml1-transitional.dtd">
<html xmlns="http://www.w3.org/1999/xhtml">
<head>
<meta http-equiv="Content-Type" content="text/html; charset=utf-8" />
<meta http-equiv="X-UA-Compatible" content="IE=Edge,chrome=1">
<title>資治通鑒_256-資治通鑑卷二百五十五_256-資治通鑑卷二百五十五</title>
<meta name="Keywords" content="資治通鑒_256-資治通鑑卷二百五十五_256-資治通鑑卷二百五十五">
<meta name="Description" content="資治通鑒_256-資治通鑑卷二百五十五_256-資治通鑑卷二百五十五">
<meta http-equiv="Cache-Control" content="no-transform" />
<meta http-equiv="Cache-Control" content="no-siteapp" />
<link href="/img/style.css" rel="stylesheet" type="text/css" />
<script src="/img/m.js?2020"></script> 
</head>
<body>
 <div class="ClassNavi">
<a  href="/24shi/">二十四史</a> | <a href="/SiKuQuanShu/">四库全书</a> | <a href="http://www.guoxuedashi.com/gjtsjc/"><font  color="#FF0000">古今图书集成</font></a> | <a href="/renwu/">历史人物</a> | <a href="/ShuoWenJieZi/"><font  color="#FF0000">说文解字</a></font> | <a href="/chengyu/">成语词典</a> | <a  target="_blank"  href="http://www.guoxuedashi.com/jgwhj/"><font  color="#FF0000">甲骨文合集</font></a> | <a href="/yzjwjc/"><font  color="#FF0000">殷周金文集成</font></a> | <a href="/xiangxingzi/"><font color="#0000FF">象形字典</font></a> | <a href="/13jing/"><font  color="#FF0000">十三经索引</font></a> | <a href="/zixing/"><font  color="#FF0000">字体转换器</font></a> | <a href="/zidian/xz/"><font color="#0000FF">篆书识别</font></a> | <a href="/jinfanyi/">近义反义词</a> | <a href="/duilian/">对联大全</a> | <a href="/jiapu/"><font  color="#0000FF">家谱族谱查询</font></a> | <a href="http://www.guoxuemi.com/hafo/" target="_blank" ><font color="#FF0000">哈佛古籍</font></a> 
</div>

 <!-- 头部导航开始 -->
<div class="w1180 head clearfix">
  <div class="head_logo l"><a title="国学大师官网" href="http://www.guoxuedashi.com" target="_blank"></a></div>
  <div class="head_sr l">
  <div id="head1">
  
  <a href="http://www.guoxuedashi.com/zidian/bujian/" target="_blank" ><img src="http://www.guoxuedashi.com/img/top1.gif" width="88" height="60" border="0" title="部件查字,支持20万汉字"></a>


<a href="http://www.guoxuedashi.com/help/yingpan.php" target="_blank"><img src="http://www.guoxuedashi.com/img/top230.gif" width="600" height="62" border="0" ></a>


  </div>
  <div id="head3"><a href="javascript:" onClick="javascript:window.external.AddFavorite(window.location.href,document.title);">添加收藏</a>
  <br><a href="/help/setie.php">搜索引擎</a>
  <br><a href="/help/zanzhu.php">赞助本站</a></div>
  <div id="head2">
 <a href="http://www.guoxuemi.com/" target="_blank"><img src="http://www.guoxuedashi.com/img/guoxuemi.gif" width="95" height="62" border="0" style="margin-left:2px;" title="国学迷"></a>
  

  </div>
</div>
  <div class="clear"></div>
  <div class="head_nav">
  <p><a href="/">首页</a> | <a href="/ShuKu/">国学书库</a> | <a href="/guji/">影印古籍</a> | <a href="/shici/">诗词宝典</a> | <a   href="/SiKuQuanShu/gxjx.php">精选</a> <b>|</b> <a href="/zidian/">汉语字典</a> | <a href="/hydcd/">汉语词典</a> | <a href="http://www.guoxuedashi.com/zidian/bujian/"><font  color="#CC0066">部件查字</font></a> | <a href="http://www.sfds.cn/"><font  color="#CC0066">书法大师</font></a> | <a href="/jgwhj/">甲骨文</a> <b>|</b> <a href="/b/4/"><font  color="#CC0066">解密</font></a> | <a href="/renwu/">历史人物</a> | <a href="/diangu/">历史典故</a> | <a href="/xingshi/">姓氏</a> | <a href="/minzu/">民族</a> <b>|</b> <a href="/mz/"><font  color="#CC0066">世界名著</font></a> | <a href="/download/">软件下载</a>
</p>
<p><a href="/b/"><font  color="#CC0066">历史</font></a> | <a href="http://skqs.guoxuedashi.com/" target="_blank">四库全书</a> |  <a href="http://www.guoxuedashi.com/search/" target="_blank"><font  color="#CC0066">全文检索</font></a> | <a href="http://www.guoxuedashi.com/shumu/">古籍书目</a> | <a   href="/24shi/">正史</a> <b>|</b> <a href="/chengyu/">成语词典</a> | <a href="/kangxi/" title="康熙字典">康熙字典</a> | <a href="/ShuoWenJieZi/">说文解字</a> | <a href="/zixing/yanbian/">字形演变</a> | <a href="/yzjwjc/">金 文</a> <b>|</b>  <a href="/shijian/nian-hao/">年号</a> | <a href="/diming/">历史地名</a> | <a href="/shijian/">历史事件</a> | <a href="/guanzhi/">官职</a> | <a href="/lishi/">知识</a> <b>|</b> <a href="/zhongyi/">中医中药</a> | <a href="http://www.guoxuedashi.com/forum/">留言反馈</a>
</p>
  </div>
</div>
<!-- 头部导航END --> 
<!-- 内容区开始 --> 
<div class="w1180 clearfix">
  <div class="info l">
   
<div class="clearfix" style="background:#f5faff;">
<script src='http://www.guoxuedashi.com/img/headersou.js'></script>

</div>
  <div class="info_tree"><a href="http://www.guoxuedashi.com">首页</a> > <a href="/SiKuQuanShu/fanti/">四库全书</a>
 > <h1>资治通鉴</h1> <!--         下载:【右键另存为】即可 --></div>
  <div class="info_content zj clearfix">
  
<div class="info_txt clearfix" id="show">
<center style="font-size:24px;">256-資治通鑑卷二百五十五</center>
    資治通鑑卷二百五十五 宋 司馬光 撰<br />
<br />
  胡三省 音註<br />
<br />
  唐紀七十一【起玄黓攝提格五月盡閼逢執徐五月凡二年有奇】<br />
<br />
  僖宗惠聖恭定孝皇帝中之下<br />
<br />
  中和二年五月以湖南觀察使閔勗權充鎮南節度使【咸通六年置鎮南軍於洪州閔勗時據潭州而以洪州節授之欲使之與鍾傳相斃也】勗屢求於湖南建節朝廷恐諸道觀察使效之不許先是王仙芝寇掠江西【先悉薦翻】高安人鍾傳聚蠻獠依山為堡【高安本漢豫章建城縣唐武德五年改名高安屬洪州九域志在州南一百二十里】衆至萬人仙芝陷撫州而不能守傳入據之詔即以為刺史至是又逐江西觀察使高茂卿據洪州【撫州西北至洪州二百四十里宋白曰撫州臨川郡漢南昌縣地吳置臨川郡隋平陳罷郡為州時總管楊武通奉使安撫即以撫為名】朝廷以勗本江西牙將【事見上卷上年】故復置鎮南軍使勗領之【鎮南軍中廢今復置】若傳不受代令勗因而討之勗知朝廷意欲鬭兩盜使相斃辭不行 加淮南節度使高駢兼侍中罷其鹽鐵轉運使駢既失兵柄又解利權攘袂大詬【是年春罷都統已失兵柄今解鹽鐵轉運又失利權詬古翻又許翻】遣其幕僚顧雲草表自訴言辭不遜其畧曰是陛下不用微臣固非微臣有負陛下又曰姧臣未悟陛下猶迷不思宗廟之焚燒不痛園陵之開毁又曰王鐸僨軍之將【謂乾符六年江陵之敗也僨方問翻】崔安濳在蜀貪黷【崔安濳擊賊屢捷無以指擿故言其在蜀貪黷懿宗咸通六年安濳鎮蜀】豈二儒士能戢疆兵又曰今之所用上至帥臣下及禆將以臣所料悉可坐擒【帥所類翻將即亮翻】又曰無使百代有抱恨之臣千古留刮席之耻【刮席漢淮陽王事見漢紀】臣但恐寇生東土劉氏復興【言山東寇盜縱横將有如劉季者復興於其間】即軹道之灾豈獨往日【又以秦子嬰之事指斥乘輿】又曰今賢才在野憸人滿朝【憸思亷翻朝直遥翻】致陛下為亡國之君此子等計將安出【顧雲蓋序次高駢大詬之言以為表】上命鄭畋草詔切責之其畧曰綰利則牢盆在手【謂專江淮鹽利也牢盆二語見漢武帝紀】主兵則都統當權直至京北京西神策諸鎮悉在指揮之下可知董制之權而又貴作司徒榮為太尉【按新書高駢傳駢帥西川已進檢校司徒兩京陷後天子猶冀駢立功進檢校太尉】以為不用如何為用乎又曰朕緣久付卿兵柄不能翦蕩元凶自天長漏網過淮【事見二百五十三卷廣明元年】不出一兵襲逐奄殘京國首尾三年廣陵之師未離封部【離力智翻】忠臣積望勇士興譏所以擢用元臣誅夷巨寇又曰從來倚仗之意一旦控告無門凝睇東南【睇大計翻目小視也南楚曰睇】惟增悽惻又曰謝玄破苻堅於淝水【見晉孝武帝紀】裴度平元濟於淮西【見憲宗紀】未必儒臣不如武將又曰宗廟焚燒園陵開毁龜玉毁櫝誰之過歟【用論語孔子之言寶龜寶玉皆櫝藏之在櫝而毁典守者不得辭其過也】又曰姧臣未悟之言何人肯認陛下猶迷之語朕不敢當又曰卿尚不能縳黄巢於天長安能坐擒諸將又曰卿云劉氏復興不知誰為魁首比朕於劉玄子嬰何太誣罔又曰况天步未傾皇綱尚整三靈不昧百度俱存君臣之禮儀上下之名分所宜遵守未可墮陵【分扶問翻墮讀曰隳】朕雖沖人安得輕侮【惡聲至必反之較計是非明己之直此委巷小人相詬者之為耳古者文告之辭漢魏以下數責其罪何至如此通鑑書之以為後世戒】駢臣節既虧自是貢賦遂絶 以天平留後曹存實為節度使【元年曹全晸與賊戰死遂順軍中之請命其兄子為帥】 黄巢攻興平興平諸軍退屯奉天【時鳳翔邠寧軍屯興平】 加河陽節度使諸葛爽同平章事 六月以涇原留後張鈞為節度使【是年二月王鐸承制以張釣為涇原留後事見上卷】 荆南節度使段彦謨與監軍朱敬玫相惡敬玫别選壯士三千人號忠勇軍自將之【玫莫杯翻將即亮翻】彦謨謀殺敬玫己亥敬玫先帥衆攻彦謨殺之【段彦謩據荆南事始二百五十三卷廣明元年帥讀曰率】以少尹李燧為留後 蜀人羅渾擎句胡僧羅夫子各聚衆數千人以應阡能【句古侯翻今蜀人從去聲阡能反見上卷是年三月 考異曰張耆舊傳曰三年六月補楊行遷為軍前四面都指揮使阡能亦散於諸處下寨官軍頻不利八月羅渾擎反十月句胡僧反又曰九月阡能渾擎胡僧與官軍大戰於乾谿官軍不利十二月羅夫子反衆二三千句延慶耆舊傳曰二年五月羅渾擎反六月句胡僧反有四千餘人官軍與阡能戰於乾溪官軍大敗是月羅夫子反聚衆三千人實録六月句胡僧反有衆二千餘官軍與能戰乾溪大敗按張傳上云十月胡僧反下云九月胡僧與官軍戰自相違又阡能敗差一年今從實録並附之六月】楊行遷等與之戰數不利【數所角翻】求益兵府中兵盡陳敬瑄悉搜倉庫門庭之卒以給之是月大戰於乾谿【據下文則此時諸盜至雙流與官軍對壘乾谿當在雙流界乾音干】官軍大敗行遷等恐無功獲罪多執村民為俘送府日數十百人敬瑄不問悉斬之其中亦有老弱及婦女觀者或問之皆曰我方治田績麻【治直之翻】官軍忽入村係虜以來竟不知何罪 秋七月己巳以鍾傳為江西觀察使從高駢之請也傳既去撫州南城人危全諷復據之【南城漢古縣唐屬撫州九域志在撫州西二百二十里】又遣其弟仔倡據信州【仔津之翻史炤祖似切倡齒羊翻又音唱】 尚讓攻宜君寨【後魏太平真君七年置宜君縣於宜君川後置宜君郡隋廢郡為宜君縣唐併宜君縣入京兆華原縣是時勤王之師蓋於宜君故縣立寨也】會大雪盈尺賊凍死者什二三 蜀人韓求聚衆數千人應阡能 【考異曰張耆舊傳三年六月韓求反其卭州界内賊首阡能邐迤漸侵入蜀州界今從句延慶傳及實録】 鎮海節度使周寶奏高駢承制以賊帥孫端為宣歙觀察使【帥所類翻歙書涉翻】詔寶與宣歙觀察使裴䖍餘兵拒之 南詔上書請早降公主【嗣曹王龜年之使南詔也上以宗室女為安化長公主許婚】詔報以方議禮儀 【考異曰張耆舊傳中和元年九月三月雲南驃信差布爕楊奇肱等齎國信來通和迎公主太師借副使儀注郊迎布爕始相見揖副使云請不拜太師聞極怒朝廷告以俟更議車服制數定續有旨命竟空還今從雲南事狀及實録】以保大留後東方逵為節度使充京城東面行營招討使【按李孝昌以鄜師勤王去年為黄巢所攻奔歸本道東方逵蓋代李孝昌者也】 閏月加魏博節度使韓簡兼侍中 八月以兵部侍郎判度支鄭紹業同平章事兼荆南節度使 浙東觀察使劉漢宏遣弟漢宥及馬步都虞侯辛約將兵二萬營于西陵謀兼并浙西杭州刺史董昌遣都知兵馬使錢鏐拒之壬子鏐乘霧夜濟江襲其營大破之所殺殆盡漢宥辛約皆走【自此杭越交兵而劉漢宏為錢鏐禽矣鏐力求翻】 魏博節度使韓簡亦有兼并之志自將兵三萬攻河陽敗諸葛爽於修武【敗補邁翻】爽弃城走簡留兵戍之因掠邢洺而還【還從宣翻又如字】李國昌自達靼帥其族遷于代州【李克用既據代州故其父帥其族自達靼還帥讀曰率】 黄巢所署同州防禦使朱温屢請益兵以扞河中知右軍事孟楷抑之不報温見巢兵勢日蹙知其將亡親將胡真謝瞳勸温歸國九月丙戍温殺其監軍嚴實舉州降王重榮温以舅事重榮【温母王氏以與重榮同姓故以舅事重榮監古銜翻降戶江翻重直龍翻】王鐸承制以温為同華節度使使瞳奉表詣行在【朱温因王重榮以歸唐而重榮之後夷於朱温之手唐祚亦夷於温矣華戶化翻下同】瞳福州人也李詳以重榮待温厚亦欲歸之為監軍所告黄巢殺之【詳據華州見上卷上年】以其弟思鄴為華州刺史桂州軍亂逐節度使張從訓以前容管經畧使崔焯<br />
<br />
  為嶺南西道節度使【焯職畧翻】 平盧大將王敬武逐節度使安師儒自為留後 初朝廷以龎勛降將湯羣為嵐州刺史【宋白曰嵐州漢汾陽縣地漢末其地無郡邑曹公遂立新興郡於此後魏末於此置嵐州因界内岢嵐山為名降戶江翻將即亮翻嵐盧含翻】羣潜通沙陀朝廷疑之徙羣懷州刺史鄭從讜遣使齎告身授之冬十月庚子朔羣殺使者據城叛附于沙陀壬寅從讜遣馬步都虞張彦球將兵討之 賊帥韓秀昇屈行從起兵斷峽江路【屈居勿翻斷音短斷峽江之路則荆蜀之信使不通王命將不得行于東南】癸丑陳敬瑄遣押牙莊夢蝶將二千人討之 【考異曰張耆舊傳三年九月峽路賊韓秀昇十月峽路賊屈行從反陳太師差押牙莊二夢將兵二千人十月二十日往峽路句延慶耆舊傳于中和二年七月韓求反下又云峽路韓秀昇屈行從反川主選點兵士三千人差押牙莊夢蝶押領十月癸丑峽路收討韓秀昇蓋因十月討之而言耳實録取句傳而誤于七月下云韓秀昇屈行從為亂敬瑄遣大將莊夢蝶以兵二千討之新傳曰洺州刺史韓秀昇等亂峽中今從句傳】又遣押牙胡弘畧將千人繼之韓簡復引兵擊鄆州【復扶又翻】節度使曹存實逆戰敗死<br />
<br />
  天平都將下邑朱瑄收餘衆嬰城拒守【下邑漢古縣唐屬宋州九域志在州東一百二十里將即亮翻】簡攻之不下詔以瑄權知天平留後【考異曰實録曹存實繼其叔父全晸為天平節度使未周歲而遇害舊傳瑄為青州王敬武牙卒中和初黄巢據長安詔徵天下兵王敬武遣牙將曹全晸率兵三千赴難關西瑄已為軍會青州警急敬武召全晸還路由鄆州時鄆將薛崇為草賊王仙芝所殺崔君裕權知州事全晸知其兵寡襲殺君裕據有鄆州自稱留後以瑄有功署為州刺史留將牙軍光啓初魏博韓簡欲兼并曹鄆以兵濟河收鄆全晸出兵逆戰為魏軍所敗全晸死之瑄收合殘卒保州城韓簡攻圍半年不能拔會魏軍亂退去朝廷嘉之授以節鉞新傳與之同薛居正五代史瑄傳中和二年張濬徵兵於青州敬武遣將曹全晸率軍赴之以瑄隸焉賊敗出關全晸以本軍還鎮會鄆帥薛崇卒部將崔君預據城叛全晸攻之殺君預因為留後瑄以功授濮州刺史鄆州馬步軍都將光啓初魏博韓允中攻鄆全晸為其所害瑄據城自固三軍推為留後允中敗朝廷以瑄為天平節度使按王仙芝死已久曹全晸久為節度去歲死王敬武今歲始得青州新舊傳薛史皆誤今從實録又新傳瑄作宣歐陽修五代史記注云今流俗以宣弟瑾於名加玉者非也今從舊傳薛史實録】 以朱温為右金吾大將軍河中行營招討副使賜名全忠 李克用雖累表請降而據忻代州數侵掠并汾爭樓煩監【數所角翻樓煩監本屬隴右節度以嵐州刺史兼領之至德後屬内飛龍使貞元十五年始别置監牧使】義武節度使王處存與克用世為昏姻【按新書王處存傳世籍神策軍家京兆萬年縣勝業里為天下高貲李國昌父子必利其富而與為昏姻也】詔處存諭克用若誠心欵附宜且歸朔州俟朝命若暴横如故【朝直遥翻横戶孟翻】當與河東大同軍共討之【是時鄭從讜帥河東赫連鐸帥大同】 以平盧大將王敬武為留後【王敬武既逐安師儒朝廷遂命為留後】時諸道兵皆會關中討黄巢獨平盧不至王鐸遣都統判官諫議大夫張濬往說之【說輸芮翻】敬武已受黄巢官爵不出迎濬見敬武責之曰公為天子藩臣侮慢詔使不能事上何以使下敬武愕然謝之既宣詔將士皆不應濬徐諭之曰人生當先曉逆順次知利害黄巢前日販鹽虜耳【事見三百五十二卷乾符二年】公等捨累葉天子而臣之果何利哉今天下勤王之師皆集京畿而淄青獨不至一旦賊平天子返正公等何面目見天下之人乎不亟往分功名取富貴後悔無及矣將士皆改容引咎顧謂敬武曰諫議之言是也敬武即兵從濬而西 劉漢宏又遣登高鎮將王鎮將兵七萬屯西陵【路振九國志作屯漁浦按今漁浦在西陵上游相去頗遠】錢鏐復濟江襲擊大破之斬獲萬計【復扶又翻】得漢宏補諸將官偽勑二百餘通鎮奔諸暨【宋白曰諸暨秦舊縣縣界有暨浦諸山因以為名在越州西南一百四十一里】 黄巢兵勢尚彊王重榮患之謂行營都監楊復光曰臣賊則負國討賊則力不足奈何復光曰雁門李僕射【時李克用據代州代州雁門郡也諸家多以為克用時為雁門節度使】驍勇有彊兵其家尊與吾先人嘗共事相善【楊復光養父玄价嘗監鹽州軍沙沱之歸國也先由鹽州後玄价為中尉執宜父子蓋與之善】彼亦有徇國之志所以不至者以與河東結隙耳誠以朝旨諭鄭公而召之必來【鄭公謂從讜也結隙見上卷上年朝直遥翻】來則賊不足平矣東面宜慰使王徽亦以為然時王鐸在河中乃以墨勑召李克用諭鄭從讜【王鐸為都都統便宜從事凡徵調除授皆得用墨勅】十一月克用將沙沱萬七千自嵐石路趣河中【趣七喻翻嵐州南至石州一百八十里】不敢入太原境獨與數百騎過晉陽城下與從讜别從讜以名馬器幣贈之 李詳舊卒共逐黄思鄴 【考異曰實録李詳下牙隊兵斬偽刺史黄思鄴推華陰鎮使王遇為首降河中王鐸承制除遇為刺史按黄鄴與黄巢俱死於乕狼谷實録誤也今從新黄巢傳】推華隂鎮使王遇為主以華州降于王重榮王鐸承制以遇為刺史 阡能黨愈熾侵淫入蜀州境【侵淫以癰疽侵食寖淫為喻】陳敬瑄以楊行遷等久無功以押牙高仁厚為都招討指揮使將兵五百人往代之未前一日有鬻麵者自旦至午出入營中數四邏者疑之【邏郎佐翻】執而訊之果阡能之諜也【諜達協翻】仁厚命釋縛温言問之對曰某村民阡能囚其父母妻子於獄云汝詗事歸【詗古迥翻又翾正翻】得實則免汝家不然盡死某非願爾也仁厚曰誠知汝如是我何忍殺汝今縱汝歸救汝父母妻子但語阡能云高尚書來日【語牛倨翻下潜語同時濫授官爵仁厚未立功已檢校尚書矣】所將止五百人無多兵也然我活汝一家汝當為我潜語寨中人云僕射愍汝曹皆良人為賊所制情非得已【為我于偽翻語牛倨翻僕射謂陳敬瑄】尚書欲拯救湔洗汝曹【湔則前翻湔滌也亦洗也言百姓為賊所汙染湔洗與推新】尚書來汝曹各投兵迎降【降戶江翻】尚書當使人書汝背為歸順字遣汝復舊業所欲誅者阡能羅渾擎句胡僧羅夫子韓求五人耳必不使横及百姓也【横戶孟翻】諜曰此皆百姓心上事尚書盡知而赦之其誰不舞躍聽命一口傳百百傳千川騰海沸不可遏也比尚書之至【比必利翻下比至同】百姓必盡奔赴如嬰兒之見慈母阡能孤居立成擒矣明日仁厚引兵至雙流把截使白文現出迎仁厚周視塹柵怒曰阡能役夫其衆皆耕民耳竭一府之兵歲餘不能擒今觀塹柵重複牢密如此【重直龍翻複方目翻】宜其可以安眠飽食養寇邀功也命引出斬之監軍力救久之乃得免命悉平塹柵纔留五百兵守之餘兵悉以自随又召諸寨兵相繼皆集阡能聞仁厚將至遣羅渾擎立五寨於雙流之西伏兵千人於野橋箐以邀官軍【蜀人謂篁竹之間為箐李心傳曰箐林箐也音咨盈翻又薛能工律詩有邊城作一聯云管排蠻戶遠出箐鳥巢孤自注云蜀人謂税戶為排戶謂林為叢箐史炤曰箐倉甸切蓋從去聲亦通】仁厚詗知【詗火迥翻又休正翻】引兵圍之下令勿殺遣人釋戎服入賊中告諭如昨日所以語諜者賊大喜呼譟爭弃其甲兵請降拜如摧山仁厚悉撫諭書其背【書其背為歸順字】使歸語寨中未降者寨中餘衆爭出降渾擎狼狽弃寨走其衆執以詣仁厚仁厚曰此愚夫不足與語縛以送府悉命焚五寨及其甲兵惟留旗幟所降凡四千人明旦仁厚謂降者曰始欲即遣汝歸而前塗諸寨百姓未知吾心或有憂疑藉汝曹為我前行過穿口新津寨下示以背字告諭之【穿口即新津新穿口也為于偽翻】比至延貢可歸矣【九域志卭州安仁縣有延貢寨安仁秦臨卭縣地武德二年置安仁縣九域志縣在卭州東北三十八里】乃取渾擎旗倒繫之【繫古詣翻取其旗而倒繫之示已得其渠帥也】每五十人為隊揚旗疾呼曰【呼火故翻】羅渾擎已生擒送使府大軍行至汝曹居寨中者速如我出降立得為良人無事矣至穿口句胡僧置十一寨寨中人爭出降胡僧大驚拔劍遏之衆投瓦石撃之共擒以獻仁厚其衆五千餘人皆降又明旦焚寨使降者執旗先驅一如雙流至新津韓求置十三寨皆迎降求自投深塹其衆鉤出之已死斬首以獻將士欲焚寨仁厚止之曰降人猶未食使先運出資糧然後焚之新降者競炊爨與先降來告者共食之語笑歌吹【歌嘔唱也吹吹笙笛之類也】終夜不絶明日仁厚縱雙流穿口降者先歸使新津降者執旗先驅且曰入卭州境亦可散歸矣羅夫子置九寨於延貢其衆前夕望新津火光已不眠矣及新津人至羅夫子脱身棄寨奔阡能其衆皆降明日羅夫子至阡能寨與之謀悉衆決戰計未定日向暮延貢降者至阡能羅夫子走馬巡寨欲出兵衆皆不應仁厚引兵連夜逼之明旦諸寨知大軍已近呼譟爭出執阡能阡能窘急赴井為衆所擒不死又執羅夫子羅夫子自剄【剄古鼎翻】衆挈羅夫子首縛阡能驅之前迎官軍見仁厚擁馬首大呼泣拜曰百姓負寃日久無所控訴自諜者還【即仁厚所縱鬻麵者也】百姓引領度頃刻如朞年今遇尚書如出九泉暏白日已死而復生矣讙呼不可止【讙與諠同】賊寨在他所者分遣諸將往降之仁厚出軍凡六日五賊皆平【按九域志雙流縣在成都南四十里自此而南至新穿口又南至新津又南至延貢又南至阡能寨度其道里相去各不過四五十里高仁厚知蜀民之心非樂於從亂而脅於五賊之威因其心而誘導之故脅從者皆望風降服師不留行而五賊平矣 考異曰張耆舊傳中和三年冬阡能轉盛官軍戰即不利陳敬瑄乃遣仁厚討之十一月五日仁厚進六日擒羅渾擎七日擒句胡僧得韓求首級九日擒阡能得羅夫子首級十一月二十二日回戈自城北門入三日大設五日議功高公自檢校兵部尚書檢校左僕射授眉州刺史張書語雖俚淺或有牴牾然叙事甚詳苟無此書則仁厚功業悉沈沒矣句延慶傳中和三年仁厚梟五賊之首凱旋歸府冬十二月戊寅皇帝御太玄樓高仁厚與將校等於清遠橋朝見至後三日大設高仁厚除授眉州刺史延慶不知据何書知阡能敗在二年冬然要之仁厚擒韓秀昇在三年十月前則擒阡能必更在前矣十二月己亥朔無戊寅日必誤也實録二年十月草賊阡能於蜀州敗官軍陳敬瑄遣高仁厚討之實録見句傳敘討阡能事承十月癸丑峽路收討韓秀昇下因附之十月亦悞也實録又曰十二月仁厚以阡能首來獻帝御太玄樓宣慰回戈將士以仁厚為檢校工部尚書眉州防禦使亦因句傳而去其日又此年十月戊辰昇眉漢彭綿等州並為防禦使故改刺史為防禦耳今高仁厚擒阡能既不知決在何年月故因實録附於此】每下縣鎮輒補鎮遏使使安集戶口於是陳敬瑄梟韓求羅夫子首於市釘阡能羅渾擎於城西七日而冎之【釘丁定翻冎古瓦翻】阡能孔目官張榮本安仁進士屢舉不中第歸於阡能為之謀主為草書檄【為草于偽翻】阡能敗以詩啓求哀於仁厚仁厚送府釘于馬市自餘不戮一人十二月以仁厚為眉州防禦使陳敬瑄牓卭州凡阡能等親黨皆不問未幾【幾居豈翻】卭州刺史申捕獲阡能叔父行全家三十五人繫獄請準法【準法謂反逆親屬當從坐誅】敬瑄以問孔目官唐溪對曰公已有牓令勿問而刺史復捕之【復扶又翻】此必有故今若殺之豈惟使明公失大信竊恐阡能之黨紛紛復起矣敬瑄從之遣押牙牛暈往集衆於州門破械而釋之因詢其所以然果行全有良田刺史欲買之不與故恨之敬瑄召刺史將按其罪刺史以憂死他日行全聞其家由溪以免密餉溪蝕箔金百兩【博聞録有蝕箔金法金及分數者打成大箔片以黄礬一兩雞屎礬一兩膽礬半兩碙砂一分信土一兩赤土一兩襄研以鹽膽水調金片上炙乾更搽更炙如此三度已來用牛糞灰一重重鬲下大火煅一日取出温湯洗淨其存者金也其蝕出者銀也】溪怒曰此乃太師仁明【陳敬瑄檢校太師故稱之】何預吾事汝乃懷禍相餉乎還其金斥逐使去【史言唐溪有古君子之風】 河東節度使鄭從讜奏克嵐州執湯羣斬之【湯羣以城附沙沱】 以忻代等州留後李克用為雁門節度使 初朝廷以鄭紹業為荆南節度使時段彦謨方據荆南紹業憚之踰半歲乃至鎮上幸蜀召紹業還以彦謨為節度使彦謨為朱敬玫所殺【是年三月朱敬玫殺段彦謨】復以紹業為節度使紹業畏敬玫逗遛不進軍中久無帥至是敬玫署押牙陳儒知府事儒江陵人也 加奉天節度使齊克儉河中節度使王重榮並同平章事李克用將兵四萬至河中 【考異曰實録在明年正月今從新太祖紀年録薛】<br />
<br />
  【居正五代史】遣從父弟克修先將兵五百濟河嘗賊【嘗弑也】初克用弟克讓為南山寺僧所殺其僕渾進通歸于黄巢自高潯之敗【潯敗見上卷上年】諸軍皆畏賊莫敢進及克用軍至賊憚之曰鵶軍至矣當避其鋒克用軍皆衣黑【衣於既翻】故謂之鵶軍巢乃捕南山寺僧十餘人遣使齎詔書及重賂因渾進通詣克用以求和克用殺僧哭克讓受其賂以分諸將焚其詔書歸其使者 【考異曰太祖紀年録初克讓於潼關戰敗避賊南山隱於佛寺夜為山僧所害紀綱渾進通冒刃獲免歸黄巢賊素憚太祖聞其至也將託情修好捕害克讓之僧十餘人殺之巢令其將米重威齎重賂偽詔因渾進通見太祖乃召諸將領其賂燔其偽詔以徇薛史克讓傳曰乾符巾以功授金吾將軍留宿衛初懿祖歸朝憲宗賜宅於親仁坊武皇之起雲中殺段文楚也天子詔巡使王處存夜圍親仁坊捕克讓詰旦兵合克讓與十餘騎彎躍馬突圍而出官軍數千人追之比至渭橋死者數百克讓自夏陽掠船而濟歸於雁門按克讓於時猶在雲州此克讓恐當作克用云雁門誤也後唐懿祖紀年録曰其兄克恭克儉皆伏誅按是時國昌猶自請討克用朝廷必未誅其子蓋國昌振武不受代後克恭克儉始被誅薛史又曰明年武皇昭雪克讓復入宿衛黄巢犯闕僖宗幸蜀克讓時守潼關為賊所敗按國昌以乾符五年不受代朝廷兵討之六年克用未嘗昭雪克讓何從得入宿衛廣明元年國昌父子兵敗逃入達靼其年冬黄巢陷長安克讓何嘗守潼關戰敗而死於佛寺或者為朝廷所圍捕時逃入南山佛寺為僧所殺則不可知也今事既難明故但云為寺僧所殺而已】引兵自夏陽度河【武德三年分郃陽置河西縣乾元三年更河西曰夏陽屬河中府後屬同州夏戶雅翻】軍于同州孟方立既殺成麟【見上卷元年】引兵歸邢州潞人請監軍吳全勗知留後是歲王鐸墨制以方立知邢州事方立不受囚全勗與鐸書【不受鐸命而與鐸書期必濟其私欲】願得儒臣鎮潞州鐸以鄭昌圖知昭義軍事既而朝廷以右僕射租庸使王徽同平章事充昭義節度使徽以車駕播遷中原方擾方立專據山東邢洺磁三州度朝廷力不能制【邢洺磁於潞州為山東度徒洛翻】辭不行請且委昌圖詔以徽為大明宮留守京畿安撫制置修奉園陵使【大明宮即東内也時黄巢猶據京師大明宮為賊所竊處園陵之開毁者亦多以此職命授徽以俟收復】昌圖至潞州不三月而去方立遂遷昭義軍於邢州自稱留後表其將李殷鋭為潞州刺史【為潞州叛孟方立張本 考異曰實録中和四年正月以義成行軍司馬鄭昌圖為中書舍人三月邢州軍亂殺其帥成麟以中書舍人鄭昌圖權為昭義留後按成麟前已為孟方立所殺况不在邢州邢州乃方立所治也又於時潞州已為李克修所據昌圖安得更往彼為留後又其年五月以右僕射王徽同平章事充昭義節度使徽上表懇述非便乃復以本官充大明宮留守舊王徽傳初潞州軍亂殺成麟以兵部侍郎鄭昌圖權知昭義軍事時孟方立割據山東二州别為一鎮上黨支郡惟澤州耳而軍中之人多附方立昌圖不能制宰相奏請以重臣鎮之乃授徽檢校尚書左僕射同平章事澤潞邢洺磁觀察等使時鑾輅未還關東聚盜而河東李克用與孟方立爭澤潞以朝廷兵力必不能加上表訴之曰鄭昌圖主留累月將結深根孟方立專據三州轉成積舋招其外則潞人胥怨撫其内則邢將益疑禍方熾於既焚計奈何於己失須觀勝負乃決安危伏乞聖慈博求廷議擇其可付理在從長天子乃以昌圖鎮之以徽為諸道租庸供軍等使新孟方立傳曰方立攻成麟斬之擅裂邢洺磁為鎮治邢為府號昭義軍潞人請監軍使吳全勗知兵馬留後王王鐸領諸道行營都統以潞未定墨制假方立知邢州事方立不受囚全勗以書請鐸願得儒臣守潞鐸使參謀中書舍人鄭昌圖知昭義留後欲遂為帥僖宗自用舊相王徽領節度時天子在西河關中雲擾方立擅地而李克用窺潞州徽度朝廷未能制乃固讓昌圖昌圖治不三月輒去方立更表李殷鋭為刺史乃徙治龍岡會克用為河東節度使昭義監軍祁審誨乞師求復昭義軍克用殺殷鋭遂并潞州表克修為留後按王鐸以三年正月罷都統則昌圖知昭義留後必在二年也昌圖在潞不三月引去今徽以潞讓昌圖則徽除昭義必不在四年五月實録年月皆誤也方立若已自稱昭義留後遷軍額於邢州則不止割據三州若欲别為一鎮則應别立軍名必不與潞州並稱昭義若但以潞為支郡當自除刺史不以書與王鐸更求儒臣就使求之鐸亦當以昌圖為潞州刺史不云知昭義軍事又不得以潞州為攴郡也蓋方立既殺成麟以邢州鄉里欲徙鎮之故身往邢州而潞人不從故請全勗為留後方立以衆情未洽未敢自立故囚全勗外示恭順託以中人不可為帥而請于王鐸乞除儒臣其意以儒臣易制欲外奉為帥而自專軍府之政漸謀代之也既而昌圖至潞欲行帥職而山東三州已為方立所制不受帥命獨澤州在南尚可號令耳故王徽表云昌圖主留累月已深結根言在澤潞已久人心稍附己所不如也又云方立專據三州轉成積舋謂昌圖欲行帥權而方立不率將職互相窺覦故積舋也又云招其外則潞人胥怨撫其内則邢將益疑謂今邢潞已成舋隙已至彼欲加惠于邢則潞人怨其寵賊加惠於潞則邢將疑其圖已也又云須觀勝負乃決安危謂昌圖能勝方立則昭義乃安也昌圖在潞終不自安故以軍府授方立而去方立然後自稱留後徙軍額于邢州以潞為支郡表殷鋭為刺史故新傳徒治龍岡在殷鋭為刺史下此其證也於是潞人怨而召沙陀當徽除節制之時克用猶未敢爭澤潞也吳全勗疑是方立初入潞府時監軍故王鐸使知留後方立既囚之疑其遂斥去祈審誨恐是鄭昌圖時監軍大祖紀年録云方立虜審誨自稱留後薛居正五代史方立傳云方立以邢為府以審誨知潞州事互說不同且既虜審誨必不以知潞州方立表殷鋭為刺史而審誨猶依舊必是後來監軍方立以其未嘗異已故不疑之若嘗被囚虜必不復留此之不實昭然可知疑唐末昭義數逐帥劉廣成麟作亂被殺人皆知之記事者不詳考正或以先者為後後者為先差互不同故諸書多牴牾不合耳又薛史安崇阮傳云安文佑初為潞州牙門將光啓中軍校劉廣逐節度使高潯據其城僖宗詔文佑平之既殺劉廣召赴行在授卭州刺史其後孟方立據邢洛攻上黨朝廷以文祐本潞人也授昭義節度使令討方立自蜀至澤州與方立戰敗殁於陣按諸書皆無文祐為節度使事況光啓中澤潞已為李克修所據文祐來當與克修戰不得與方立戰也其事恐虚今不取】和州刺史秦彦使其子將兵數千襲宣州逐觀察使竇潏而代之【潏食聿翻又音聿又音決秦彦降高駢見二百五十三卷乾符六年其得和州亦駢用之也為彦以宣州兵入廣陵張本】<br />
<br />
  三年春正月李克用將李存貞敗黄揆于沙苑【敗補邁翻】己巳克用進屯沙苑揆巢之弟也王鐸承制以克用為東北面行營都統以楊復光為東面都統監軍使陳景思為北面都統監軍使乙亥制以中書令充諸道行營都統王繹為義成節度使令赴鎮田令孜欲歸重北司稱鐸討黄巢久無功卒用楊復光策召沙陀而破之故罷鐸兵柄以悦復光【罷王鐸兵柄在正月李克用破黄巢在四月蓋田令孜以黄巢之勢已蹙而楊復光之功必成先以是悦之耳卒子恤翻】又以副都統崔安潜為東都留守以都都監西門思恭為右神策中尉充諸道租庸兼催促諸道進軍等使令孜自以建議幸蜀收傳國寶列聖真容散家財犒軍為己功令宰相藩鎮共請加賞上以令孜為十軍兼十二衛觀軍容使【令孜從幸蜀募神策新軍為五十四都離為十軍號神策十軍左右衛左右驍衛左右武衛左右威衛左右領軍衛左右金吾衛謂之南牙十二衛】成德節度使常山忠穆王王景崇薨軍中立其子節<br />
<br />
  度副使鎔知留後事時鎔生十年矣 以天平留後朱瑄為節度使 二月壬子李克用進軍乾阬【乾阬在沙苑西南乾音干】與河中易定忠武軍合尚讓等將十五萬衆屯于梁田陂【舊書作良天坡在城店西三十里】明日大戰自午至晡賊衆大敗俘斬數萬伏尸三十里巢將王璠黄揆襲華州據之王遇亡去【去年王璠據華州歸國璠孚袁翻】 初光州刺史李罕之為秦宗權所攻弃州奔項城【李罕之與秦彦俱降高駢蓋駢使守光州】帥餘衆歸諸葛爽【帥讀曰率】爽以為懷州刺史韓簡攻鄆州半年不能下爽復襲取河陽【去年八月韓簡破諸葛爽取河陽十月移兵攻鄆州】朱瑄請和簡乃捨之引兵擊河陽爽遣罕之逆戰于武陟魏軍大敗而還大將澶州刺史樂行達先歸據魏州軍中共立行達為留後簡為部下所殺【懿宗咸通十一年韓君雄得魏博二世十四年而滅考異曰舊傳簡攻河陽行及新郡為諸葛爽所敗單騎奔迴憂憤疽背而卒時中和元年十一月也新傳】<br />
<br />
  【亦同今從實録按新郡當作新鄉】己未以行達為魏博留後 甲子李克用進圍華州黄思鄴黄揆嬰城固守克用分騎屯渭北 以王鎔為成德留後 以鄭紹業為太子賓客分司以陳儒為荆南留後 峽路招討指揮使莊夢蝶為韓秀昇屈行從所敗退保忠州【去年遣莊夢蝶討韓秀昇等敗補邁翻】應援使胡弘略戰亦不利江淮貢賦皆為賊所阻百官無俸【時車駕在蜀江淮租賦泝峽江而上今為韓秀昇等所阻】雲安淯井路不通民間乏鹽【雲安縣漢朐䏰地後周改曰雲安縣唐屬夔州有鹽官九域志在州西一百三十三里鹽監又在縣西三十里淯井在瀘州西南二百六十三里史炤曰淯井漢犍為郡之漢陽縣地唐置長寧州淯音育按漢陽當作江陽】陳敬瑄奏以眉州防禦使高仁厚為西川行軍司馬將三千兵討之 【考異曰張耆舊傳曰中和四年甲辰春三月峽路招討指揮使莊夢蝶尚書為韓秀昇所敗退至忠州川主太師召眉州刺史高仁厚使討秀昇等許以成功除梓帥即日聞奏拜行軍司馬將步卒千人三月五日進句延慶耆舊傳中和三年二月莊夢蝶為賊所敗川主喚仁厚奏授峽路招討都指揮使將兵三千人三月辛丑進實録三年二月夢蝶為賊所敗陳敬瑄奏以仁厚代夢蝶將兵三千進討詔拜行軍司馬是月丁卯朔無辛丑辛丑乃四月五月延慶誤也實録三年二月敬瑄奏仁厚代夢蝶蓋亦用句傳年月今從之】 加鳳翔節度使李昌言同平章事黄巢兵數敗食復盡【數所角翻復扶又翻】隂為遁計兵三萬搤藍田道【搤藍田道所以通自武關南走之路搤於革翻】三月壬申遣尚讓將兵救華州李克用王重榮兵逆戰於零口破之克用進軍渭橋騎軍在渭北克用每夜令其將薛志勤康君立潜入長安燔積聚斬虜而還【零口在京兆昭應縣積子智翻聚從遇翻又慈庾翻還從宣翻】賊中大驚 以淮南押牙合肥楊行愍為廬州刺史【考異曰十國紀年云楊行密六合人今從薛居正五代史徐鉉吳録】行愍本廬州牙將勇敢屢有戰功都將忌之白刺史郎幼復連使出戍於外行愍過辭【過古禾翻過都將而辭行也】都將以甘言悦之問其所須【須者意之所欲】行愍曰正須汝頭耳遂起斬之并將諸營自稱八營都知兵馬使幼復不能制薦於高駢請以自代駢以行愍為淮南押牙知廬州事朝廷因而命之行愍聞州人王勗賢召欲用之固辭問其子弟曰子潜好學慎密可任以事弟子稔有氣節可為將行愍召潜置門下以稔及定遠人季章為騎將【楊行愍後改名行密事始此定遠漢曲陽縣地梁改為定遠縣唐屬濠州九域志在州南八十里騎奇寄翻將即亮翻】初呂用之因左驍雄軍使俞公楚得見高駢用之横甚【横戶孟翻】或以咎公楚公楚數戒用之少自歛毋相累【數所角翻少詩沼翻歛力嚴翻累力瑞翻】用之銜之右驍雄軍使姚歸禮氣直敢言尤疾用之所為時面數其罪【數所其翻】常欲手刃之癸未夜用之與其黨會倡家歸禮潜遣人爇其室【倡音昌爇如悦翻燒也】殺貌類者數人用之易服得免明旦窮治其事【治直之翻】獲縱火者皆驍雄之卒用之於是日夜譛二將於駢未幾駢使二將將驍雄卒三千襲賊於慎縣【慎縣漢九江浚道縣地古城在今縣南隋置慎縣唐屬廬州九域志在州東北六十里幾居豈翻】用之密以語楊行愍云公楚歸里欲襲廬州行愍兵掩之二將不為備舉軍盡殪【語牛倨翻殪壹計翻】以二將謀亂告駢駢不知用之謀厚賞行愍【為楊行愍以廬州起張本】 己丑以河中行營招討副使朱全忠為宣武節度使俟克復長安令赴鎮 癸巳李克用等拔華州黄揆弃城走 劉漢宏分兵屯黄嶺巖下貞女三鎮【三鎮皆當在婺越間】錢鏐將八都兵自富春擊之【自富春度江擊三鎮富春即富陽縣】破黄嶺擒巖下鎮將史弁貞女鎮將楊元宗漢宏以精兵屯諸暨鏐又擊破之漢宏走 莊夢蝶與韓秀昇屈行從戰又敗其敗兵紛紜還走所在慰諭不可遏遇高仁厚於路叱之即止仁厚斬都虞侯一人更令修娖部伍【娖側角翻娖整隊伍也】乃召耆老詢以山川蹊徑及賊寨所據喜曰賊精兵盡在舟中使老弱守寨資糧皆在寨中此所謂重戰輕防其敗必矣乃揚兵江上為欲涉之狀賊晝夜禦備遣兵挑戰【挑徒了翻】仁厚不與交兵潜勇士千人執兵負藁夜由間道攻其寨且焚之【間右莧翻】賊望見分兵往救之不及資糧蕩盡衆心已搖仁厚復募善游者鑿其其舟相繼皆沈【復扶又翻沈持林翻】賊往來惶惑不能相救仁厚遣兵於要路邀擊且招之賊衆皆降秀昇行從見衆潰揮劍亂斫欲止之衆愈怒共執二人詣仁厚仁厚詰之曰何故反秀昇曰自大中皇帝晏駕【太中皇帝謂宣宗】天下無復公道紐解綱絶今日反者豈惟秀昇成是敗非机上之肉惟所烹醢耳仁厚愀然命善食而械之【愀七小翻食詳吏翻善食善以酒食食之也】夏四月庚子獻于行在斬之 【考異曰張耆舊傳中和四年高僕射將步卒千人三月五日進莊尚書三月二十日齊進四月十四口峽路申四月一日大破峽賊句延慶耆舊傳三年四日庚午擒韓秀昇捷書到府按是月丁酉朔無庚午實録中和三年四月庚子仁厚擒韓秀昇獻於行在初仁厚至峽與賊戰其衆大敗賊中小校縛秀昇出降據鄭畋集有覆黔南觀察使陳侁奏涪州韓秀昇謀亂已收管在州候勅旨狀云秀昇刼害黔府俘掠帥臣占據涪陵扼截江路遽懷僭妄求作察廉陳侁爰命毛玭部領甲士直趍巢穴便破城池迫逐渠魁勦除逆黨而諸家之說皆云仁厚所獲新傳衆怒執秀昇以降仁厚檻車送行在斬于市張耆舊傳中和四年三月阡能反八月羅渾擎反十月句胡僧反十二月羅夫子反三年北路奏黄巢正月十日敗走收復長安正月阡能遣羅渾擎於新穿埧下二十七寨把斷水陸官路六月韓求反其卭州賊首阡能邐迤漸侵入蜀州界九月峽路賊韓秀昇反十月峽路賊屈行從反川主陳太師差押衙莊二夢將兵二千十月二十日往峽路討韓秀昇屈行從等十一月五日高仁厚進討阡能九日收卭州境内諸寨十日州縣豁平二十二日回戈朝見三日大設五日議功授眉州刺史四年三月莊夢蝶退至忠州川主差高仁厚將兵三月五日進莊尚書三月二十日齊進四月十四日申四月一日大破峽賊擒秀昇等十五日東川楊師立反句延慶耆舊傳止於鈔改張傳為之别無外事但移渾擎反於中和二年五月胡僧羅夫子反於六月韓求反於其年七月莊夢蝶討韓秀昇屈行從以其年十月癸丑進高仁厚破阡能等五賊回朝見在其年十二月戊寅三年二月莊夢蝶為賊所敗川主遣高仁厚將兵三月辛丑進四月庚午擒韓秀昇捷書到府是月楊師立反四年北路奏黄巢正月十日敗走收復長安不知延慶改移年月别有所據邪將率意為之也至於三年楊師立反四年收復長安其為乖謬尤甚於實録阡能韓秀昇等事率依句傳而誤以韓秀昇反置七月高仁厚討阡能置十月削戊寅辛丑兩日改庚午為庚子此其異于句傳也新紀三年十一月壬申西川行軍司馬高仁厚及阡能戰於邛州敗之續寶運録中和三年涪州韓秀昇反冬阡能反高仁厚討平之按賈緯唐年補録及實録所載鐵劵文維中和三年歲次癸卯十月甲午朔十六日己酉皇帝賜功臣陳敬瑄鐵劵其文有戮阡能如翦草除秀昇若焚巢然則秀昇之敗必在此月前也張傳破秀昇在四年四月其四年十月十日亦載賜川主太師鐵券乃云維中和三年歲次癸卯十月甲子朔五日戊辰文與補録實録同其昏耄如此句傳取張事而改其年實録用句年而改其日其阡能韓秀昇等起滅不知的在何時今從實録】 李克用與忠武將龎從河中將白志遷等引兵先進與黄巢軍戰於渭南一日三戰皆捷義成義武等諸軍繼之賊衆大奔甲辰克用等自光泰門入京師黄巢力戰不勝焚宮室遁去 【考異曰舊紀四月庚子沙陀等軍趍長安賊拒之於渭橋大敗而還李克用乘勝追之己卯黄巢收殘衆由藍田關而遁庚辰收京城楊復光告捷按是月丁酉朔無己卯庚辰敬翔梁太祖編遺録四月乙巳巢焚宮闈省寺居第略盡擁殘黨越藍田而逃明日上與諸軍收復長安實録甲辰李克用與忠武將龐從河中將白志遷横野將滿存朝邑將康思貞三敗賊於渭橋大破之義成義武等軍繼進乙巳巢賊燔長安宮室收餘衆自光泰門東走由藍田關以遁諸軍進取京師新紀三月壬申李克用及黄巢戰于零口敗之四月甲辰又敗之于渭橋丙午復京師舊傳曰四月八日克用合忠武騎將龐從遇賊於渭南決戰三捷大敗賊軍十日夜賊巢散走詰旦充用由光泰門入收京師巢賊出藍田七盤路東走關東新傳曰克用遣部將楊守宗率河中將白志遷忠武將龐從等最先進擊賊渭橋三戰三北於是諸節度兵皆奮無敢後入自光泰門賊崩潰逐北至望春入昇陽殿闥巢夜奔衆猶十五萬聲趍徐州出藍田入商山程匡柔唐補紀曰楊復光帥十道行營節度使王重榮李克用等兵士二萬餘人自光泰門入襲逐至昇陽殿下殺賊盈萬黄巢軍敗陣上奔逃取藍田關出後唐太祖紀年録乙巳巢敗焚宮室東走太祖進收京師唐年補録八日克用等戰渭南三敗賊軍九日巢走按楊復光露布云今月八日楊守宗等随克用自光泰門先入京師又云賊尚為堅陣來抗官軍自卯至申羣凶大潰即時奔遁南入商山然則官軍以八日入城賊戰不勝而走此最可據今從之渭南之戰必在八日以前諸書皆誤也】賊死及降者甚衆【降戶江翻】官軍暴掠無異於賊長安室屋及民所存無幾【幾居豈翻】巢自藍田入商山【黄巢先遣兵搤藍田道故得由此路遁去】多遺珍寶於路官軍爭取之不急追賊遂逸去楊復光遣使告捷 【考異曰張耆舊傳中和三年北路奏黄巢正月十日敗走收復長安城訖三月北路行營收城將士並回戈句延慶耆舊傳曰四年北路奏黄巢正月十日敗走收復長安三月北路行營破黄巢將士並回延慶悉移四年事於三年三年事於四年而不移其月日其為差謬又甚於今但云告捷更不著月日】百官入賀詔留忠武等軍二萬人委大明宮留守王徽及京畿制置使田從異部分守衛長安【分扶問翻】五月加朱玫克用東方逵同平章事升陜州為節度以王重盈為節度使又建延州為保塞軍以保大行軍司馬延州刺史李孝恭為節度使【賞破黄巢復京城之功也】克用時年二十八於諸將最少【少詩照翻】而破黄巢復長安功第一兵勢最疆諸將皆畏之克用一目微眇【眇彌沼翻一目小也】時人謂之獨眼龍詔以崔璆家貴身顯為黄巢相首尾三載不逃不隱於所在斬之【載子亥翻】 黄巢使其驍將孟楷將萬人為前鋒擊蔡州節度使秦宗權逆戰而敗賊進攻其城宗權遂稱臣於巢與之連兵初巢在長安陳州刺史宛丘趙犨謂將佐曰【宛丘後魏項縣也隋改曰宛丘唐屬陳州管下項城縣乃東魏僑置秣陵縣地隋改曰項城犨昌牛翻】巢不死長安必東走陳其衝也且巢素與忠武為仇【巢自初起與宋威張自勉等累戰皆忠武兵也】不可不為之備乃完城塹繕甲兵積芻粟六十里之内民有資糧者悉徙之入城多募勇士使其弟昶珝子麓林分將之【將即亮翻】孟楷既下蔡州移兵擊陳軍于項城犨先示之弱伺其無備襲擊之殺獲殆盡生擒楷斬之巢聞楷死驚恐悉衆屯溵水【珝況珝翻項城在陳州東南溵水在西南】六月與秦宗權合兵圍陳州掘塹五重【重直龍翻】百道攻之陳人大恐犨諭之曰忠武素著義勇陳州號為勁兵况吾家久食陳禄誓與此州存亡男子當求生於死中且狥國而死不愈於臣賊而生乎有異議者斬數引鋭兵開門出撃賊破之【數所角翻】巢益怒營於州北立宮室百司為持久之計時民間無積聚賊掠人為糧生投於碓磑【碓都内翻磑五對翻】併骨食之號給糧之處曰舂磨寨【舂磨寨即設碓磑處碓以舂磑以磨磨莫卧翻】縱兵四掠自河南許汝唐鄧孟鄭汴曹濮徐兖等數十州咸被其毒【此河南謂洛州河南府被皮義翻】 初上蔡人劉謙為嶺南小校節度使韋宙奇其器【咸通中韋宙帥嶺南】以兄女妻之【妻七細翻 考異曰新傳宙弟岫亦有名宙在嶺南以從女妻小校劉謙或諫止之宙曰吾子孫或當依之薛居正五代史韋宙出鎮南海謙時為牙校宙以猶女妻之北夢瑣言曰承相韋公宙出鎮南海有小將劉謙者職級甚卑氣宇殊異乃以從女妻之其内以非我族類慮招物議風諸幕僚諫止之丞相曰此人非常流也他日吾子孫或可依之謙以軍功拜封州刺史韋夫人生子曰隱曰巖十國紀年曰劉謙望字德光亦名知謙後止名謙唐咸通中為廣州牙將韋宙以兄女妻之新傳云岫知謙恐誤今從瑣言紀年】謙擊羣盜屢有功辛丑以謙為封州刺史【劉謙始此】 加東川節度使楊師立同平章事 宣武節度使朱全忠帥所部數百人赴鎮【帥讀曰率】秋七月丁卯至汴州時汴宋薦飢公私窮竭内外驕軍難制外為大敵所攻無日不戰衆心危懼而全忠勇氣益振詔以黄巢未平加全忠東北面都招討使【為朱全忠以宣武兵併吞諸鎮卒移唐祚張本】 南詔遣布燮楊奇肱來迎公主詔陳敬瑄與書辭以鑾輿巡幸儀物未備俟還京邑然後出降奇肱不從直前至成都 李克用自長安引兵還鴈門尋有詔以克用為河東節度使召鄭從讜詣行在克用乃自東道過榆次詣鴈門省其父【省悉景翻】克用尋牓河東安慰軍民曰勿為舊念各安家業【以河東之人前此數與克用戰恐其不自安故牓諭之考異曰舊紀五月李克用充河東節度使七月詔鄭從讜赴行在新紀五月從讜為司空同平章事賈緯唐年補録五月制李諱可同平章事充河東節度使 注云按薛史晉天福六年二月賈緯撰唐年補録上之又曰賈緯真定獲鹿人以唐諸帝實録自武宗以下缺而不紀乃採掇近代傳聞之事及諸家小說第其年月編為唐年補録凡六十五卷歷事唐晉漢周故不敢稱克用名舊從讜傳三年克用授河東節度代從讜五月十五從讜離太原道途多寇行次絳州留駐數月冬詔使追赴行在復輔政唐末見聞録曰五月勑除李尚書雁門節度使六月二十五日鴈門節度使李僕射般次於府東路過六月内有除目到相公除替赴闕雁門節度李相公除河東節度使十五日相公取西明門進當月内新使李相公有牓示安撫在城軍人百姓曰無懷舊念各仰安家又曰晉王諱克用中和三年五月一日自鴈門節度使拜平章事充河東節度使按克用除河東及從讜復輔政諸書日月不同舊紀五月除克用七月從讜赴行在不言入相新紀五月已為相尤誤舊從讜傳五月十五日離太原又與紀相違唐年補録五月制止褒賞克用朱玫東方逵三人制詞鄙俚疑其非實唐末見聞録初云六月除河東後復云五月一日據實録後唐太祖紀年録薛居正五代史皆在七月今從之從讜此年九月為東都留守光啟二年二月方再入相】左驍衛上將軍楊復光卒于河中復光慷慨喜忠義【喜許既翻】善撫士卒軍中慟哭累日八都將鹿宴弘等各以其衆散去田令孜素畏忌之聞其卒甚喜因擯斥其兄樞密使復恭為飛龍使令孜專權人莫與之抗惟復恭數與之爭得失故令孜惡之復恭因稱疾歸藍田【數所角翻惡烏路翻】 以成德留後王鎔魏博留後樂行達天平留後朱瑄為本道節度使 司徒門下侍郎同平章事鄭畋雖當播越猶謹法度田令孜為判官吳圓求郎官【吳圓田令孜之屬官為于偽翻】畋不許陳敬瑄欲立於宰相之上畋以故事使相品秩雖高皆居真相之下固爭之【唐末凡節度使帶平章事及檢校三省長官三公三師者皆謂之使相】二人乃令鳳翔節度使李昌言上言軍情猜忌不可令畋扈從過北【元年昌言逐畋以攘鳳翔故二人嗾之上言以罷畋相自是之後朝廷進退宰相率受制於藩鎮矣從才用翻】畋亦累表辭位乃罷為太子太保又以其子兵部侍郎凝績為彭州刺史使之就養【宋白曰唐垂拱三年以益州九隴縣置彭州取古天彭關為名養羊尚翻】以兵部尚書判度支裴澈為中書侍郎同平章事 八月甲辰李克用至晉陽【李克用自此以晉陽為爭天下根本】詔以前振武節度使李國昌為代北節度使鎮代州升湖南為欽化軍以觀察使閔勗為節度使 九月<br />
<br />
  加陳敬瑄兼中書令進爵潁川郡王 感化節度使時溥營於溵水【遏黄巢之兵且為陳州聲援也】加溥東面兵馬都統 以荆南留後陳儒為節度使 昭義節度使孟方立以潞州地險人勁屢簒主帥欲漸弱之乃遷治所於邢州【事見上年帥所類翻】大將家及富室皆徙山東潞人不悦監軍祁審誨因人心不安使武鄉鎮使安居受潜以蠟丸乞師於李克用請復軍府於潞州【武鄉與河東巡屬遼州鄰境故使其鎮將乞師是後方鎮率分置鎮將於諸縣縣令不得舉其職矣宋白曰武鄉縣本漢泥縣地晉始置武鄉郡縣屬焉】冬十月克用遣其將賀公雅等赴之為方立所敗【敗補邁翻】又遣李克修擊之辛亥取潞州 【考異曰實録克用表李克修為節度使於是分昭義軍五州為二鎮薛居正五代史孟方立傳曰潞人隂乞師於武皇中和三年十月武皇遣李克修將兵赴之方立拒戰大敗之由是連收澤潞二郡乃以克修為節度使按薛史張全義傳諸葛爽表全義為澤州刺火爽卒李罕之據澤州蓋克修止得潞州澤為河陽所取也】殺其刺史李殷鋭是後克用每歲出兵爭山東三州之人半為俘馘野無稼穡矣【昭義邢洺磁三州在山東】以宗女為安化長公主【慶州安化郡】妻南詔【妻七細翻】 劉漢宏<br />
<br />
  將十餘萬衆出西陵將擊董昌戊午錢鏐濟江迎戰大破之漢宏易服持鱠刀而遁【使敵人見之以為庖丁不疑為漢宏也】己未漢宏收餘衆四萬又戰鏐又破之斬其弟漢容及將辛約 十一月甲子朔秦宗權圍許州 忠武大將鹿晏弘帥所部自河中南掠襄鄧金洋所過屠滅聲云西赴行在【宋白曰金州漢漢中郡之西城縣也魏文帝置西城郡後改魏興郡梁置北梁州尋改為南梁州西魏置東梁州因其地出金改為金州洋州漢成固縣地後漢封班超於此晉為南鄉縣尋改西鄉西魏置洋州帥讀曰率洋音祥】十二月至興元逐節度使牛勗勗奔龍州西山【龍州西山松茂二州界時已没於蠻中】晏弘據興元自稱留後 武寧節度使時溥【武寧當作感化】因食中毒【中仔仲翻】疑判官李凝古而殺之凝古父損為右散騎常侍在成都溥奏凝古與父同謀田令孜受溥賂令御史臺鞫之侍御史王華為損論寃令孜矯詔移損下神策獄【為于偽翻下戶嫁翻】華拒而不遣蕭遘奏李凝古行毒事出曖昧已為溥所殺父損相别數年聲問不通安得誣以同謀溥恃功亂法陵蔑朝廷欲殺天子侍臣若徇其欲行及臣輩朝廷何以自立由是損得免死歸田里時令孜專權羣臣莫敢迕視【迕五故翻】惟遘屢與爭辯朝廷倚之 升浙東為義勝軍以劉漢宏為節度使 趙犨遣人間道求救於鄰道【聞古莧翻】於是周岌時溥朱全忠皆引兵救之全忠與黄巢之黨戰於鹿邑敗之斬首二千餘級遂引兵入亳州而據之【鹿邑後魏陳留武平縣也隋開皇十八年更名鹿邑唐屬亳州九域志在州西一百三十里敗補邁翻】<br />
<br />
  四年春正月以鹿晏弘為興元留後 賜魏博節度使樂行達名彦禎 東川節度使楊師立以陳敬瑄兄弟權寵之盛【田令孜陳敬瑄兄弟也】心不能平敬瑄之遣高仁厚討韓秀昇也【見上二月】語之曰成功而還【語牛倨翻還從宣翻】當奏天子以東川相賞師立聞之怒曰彼此列藩而遽以我疆土許人是無天地也田令孜恐其為亂因其不發兵防遏徵師立為右僕射 黄巢兵尚彊周岌時溥朱全忠不能支共求救於河東節度使李克用二月克用將蕃漢兵五萬出天井關河陽節度使諸葛爽辭以河橋不完【謂河陽橋也】屯兵萬善以拒之克用乃還兵自陜河中度河而東 【考異曰唐末見聞録晉王三月十三日發大軍討黄巢太祖紀年録正月太祖帥師五萬自澤潞將下天井關河陽屯萬善乃改轅蒲陜度河薛居正五代史但云四年春按四月已與巢戰三月十三日晉陽似太晚又克用表云昨三月内頻得陳許徐汴書牒今從舊紀又克用自訴上表云遂從陜服徑達許田是於蒲陜兩道度兵也】 楊師立得詔書怒不受代殺官告使及監軍使【官告使奉右僕射告身以徵師立者也監軍使東川監軍】舉兵以討陳敬瑄為名大將有諫者輒殺之進屯涪城【涪城漢涪縣地東晉置始平郡後魏改為涪城及潼縣隋改潼為涪城唐初屬綿州後屬梓州九域志在州西北五十五里涪音浮】遣其將郝蠲襲綿州不克丙午以陳敬瑄為西川東川山南西道都指揮招討安撫處置等使【處昌呂翻】三月甲子楊師立移檄行在百官及諸道將吏士庶數陳敬瑄十罪 【考異曰張耆舊傳中和四年四月十五日東川楊師立反下載 師立檄文則云三月三日自相違今從實録數所具翻】自言集本道將士八州壇丁共十五萬人【按新書路巖傳巖帥西川置定邉軍於卭州扼大度治故關取壇丁子弟敎擊刺史補屯籍則壇丁者蜀中邉郡民兵也又按路振九國志石處温事孟知祥補萬州管内諸壇點檢指揮使見蜀中諸郡皆得有壇丁】長驅問罪詔削師立官爵以眉州防禦使高仁厚為東川留後將兵五千討之以西川押牙楊茂言為行軍副使 朱全忠撃黄巢瓦子寨拔之【黄巢撤民居以為塞屋謂之瓦子寨】巢將陜人李唐賓楚丘王䖍裕降于全忠【陜失冉翻降戶江翻】 婺州人王鎮執刺史黄碣降于錢鏐【碣其謁翻】劉漢宏遣其將婁賚殺鎮而代之浦陽鎮將蒋瓌召鏐兵共攻婺州【水經註浦陽江源出烏傷縣東逕諸暨縣與洩洩合唐婺州漢烏傷之地也天寶十三載分婺州之義烏蘭溪及杭州之富陽置浦陽縣】擒賚而還碣閩人也 高駢從子左驍衛大將軍澞【從才用翻澞麌俱翻】疏呂用之罪狀二十餘幅密以呈駢且泣曰用之内則假神仙之說蠱惑尊聽外則盜節制之權殘賊百姓將佐懼死莫之敢言歲月浸深羽翼將成苟不除之恐高氏奕代勲庸一朝掃地矣因嗚咽不自勝【勝音升】駢曰汝醉邪命扶出明日以澞狀示用之用之曰四十郎嘗以空乏見告【澞第四十】未獲遵命故有此憾因出澞手書數幅呈之駢甚慙遂禁澞出入後月餘以澞知舒州事羣盜陳儒攻舒州澞求救於廬州楊行愍力不能救謀於其將李神福神福請不用寸刃而逐之乃多齎旗幟間道入舒州【九域志廬州南至舒州四百二十里間古莧翻】頃之引舒州兵建廬州旗幟而出指畫地形若布大陳狀賊懼宵遁【賊畏廬州兵故宵遁兵有先聲而後實此其近之陳讀曰陣】神福洺州人也【路振九國志曰李神福洺州人隸上黨軍籍高駢兼諸道行營都統補福從州將戍淮海因投楊行密】久之羣盜吳迥李本復攻舒州【復扶又翻】澞不能守弃城走駢使人就殺之楊行愍遣其將合肥陶雅清流張訓等將兵擊吳迥李本擒斬之【合肥漢古縣唐帶廬州清流漢全椒縣地隋置清流縣唐帶滁州】以雅攝舒州刺史秦宗權遣其弟將兵寇廬州據舒城【開元二十三年分合肥廬江置舒城縣屬廬州九域志在州西南一百一十里】楊行愍遣其將合肥田頵擊走之【頵於倫翻】 前杭州刺史路審中客居黄州【路審中為董昌所拒見上卷元年】聞鄂州刺史崔紹卒募兵三千人入據之武昌牙將杜洪亦逐岳州刺史而代之 黄巢圍陳州幾三百日【幾居依翻】趙犨兄弟與之大小數百戰雖兵食將盡而衆心益固李克用會許汴徐兖之軍于陳州時尚讓屯太康【太康漢陽夏縣隋改曰大康以縣東有太康城也唐屬陳州】夏四月癸巳諸軍進拔太康黄思鄴屯西華【西華漢縣唐屬陳州九域志在州西八十里】諸軍復攻之【復扶又翻下同】思鄴走黄巢聞之懼退軍故陽里【故陽里在陳州城北】陳州圍始解朱全忠聞黄巢將至引軍還大梁五月癸亥大雨平地三尺黄巢營為水所漂且聞李克用將至遂引兵東北趣汴州【趣七喻翻】屠尉氏尚讓以驍騎五千進逼大梁至于繁臺【繁臺本師曠吹臺梁孝王增築水經注吹臺在浚儀城南牧澤之右牧澤者今之蒲關澤即此澤也】宣武將豐人朱珍南華龎師古撃却之【豐漢縣唐屬徐州九域志在徐州西北一百四十里】全忠復告急於李克用丙寅克用與忠武都監使田從異許州戊辰追及黄巢於中牟北王滿渡【按舊書帝紀王滿渡乃汴河所經津濟之地】乘其半濟奮撃大破之殺萬餘人賊遂潰尚讓帥其衆降時溥【帥讀曰率下同】别將臨晉李讜曲周霍存甄城葛從周寃句張歸霸及弟歸厚帥其衆降朱全忠【臨晉古地名隋分猗氏置桑泉縣天寶十三載改為臨晉屬河中府九域志在府北六十五里曲周漢古縣中廢隋分洺水復置唐屬洺州宋廢為鎮屬雞澤縣甄城當作鄄城亦漢古縣唐帶濮州史言朱全忠後吞諸鎮多用所降黄巢將鄄吉掾翻 考異曰崇文院有梁功臣列傳不著撰人名氏云張歸厚祖興父處讓歸厚中和末與伯季自寃句相率來投薛居正五代史張歸霸祖進言父實歸厚傳無父祖但云與兄歸霸皆來降据梁功臣傳父祖與歸霸不同當是從弟】巢踰汴而北己巳克用追擊之於封丘又破之庚午夜復大雨賊驚懼東走克用追之過胙城匡城【胙城漢南燕縣隋改曰胙城唐屬滑州九域志在州南九十里宋白曰胙城縣本古之胙國又為古之燕國漢為南燕縣隋文帝因覧奏狀見南燕縣名因曰今天下一統何南燕之有遂改為胙城】巢收餘衆近千人東奔兖州【近其靳翻】辛未克用追至寃句騎能屬者纔數百人【屬之欲翻】晝夜行二百餘里人馬疲乏糧盡乃還汴州欲裹糧復追之獲巢幼子及乘輿器服符印【乘繩證翻】得所掠男女萬人悉縱遣之 癸酉高仁厚屯德陽楊師立遣其將鄭君雄張士安據鹿頭關以拒之 甲戌李克用至汴州營於城外朱全忠固請入城館於上源驛【晉天福五年改東京上源驛為都亭驛】全忠就置酒聲樂饌具皆精豐禮貌甚恭克用乘酒使氣語頗侵之【饌雛戀翻又雛晥翻李克用蓋言全忠從黄巢為寇觸其實也】全忠不平薄暮罷酒從者皆霑醉【霑醉言飲酒大醉胷襟霑濕不能自持也從才用翻】宣武將楊彦洪密與全忠謀連車樹柵以塞衢路【塞悉則翻】兵圍驛而攻之呼聲動地克用醉不之聞親兵薛志勤史敬思等十餘人格鬭侍者郭景銖滅燭扶克用匿牀下以水沃其面徐告以難【呼火故翻難乃旦翻】克用始張目援弓而起志勤射汴人死者數十【援于元翻射而亦翻】須臾煙火四合會大雨震電天地晦冥志勤扶克用帥左右數人【帥讀曰率】踰垣突圍乘電光而行汴人扼橋力戰得度史敬思為後拒戰死克用登尉氏門【尉氏門汴城南門也梁開平元年改為高明門晉天福三年改為薰風門】縋城得出監軍陳景思等三百餘人皆為汴人所殺楊彦洪謂全忠曰胡人急則乘馬見乘馬則射之是夕彦洪乘馬適在全忠前全忠射之殪【射而亦翻殪臺計翻 考異曰梁太祖編遺録甲戌并帥自曹南旋師上出封丘門迎勞之克用堅請入州内上初止之乃於門外陳設次舍將安泊之克用不諾因縱蕃騎突入馳至上源驛既不可遏上乃與之並轡送至驛亭是日晚備宴宴罷復張樂繼燭而飲克用酒酣使氣廣須樂妓頗恣無厭之欲人以醜言陵侮於上時蕃將皆被甲冑以衛克用上既甚不懼遽起圖之遂令都將楊彦洪潜率甲士入驛戮之時夜將半克用沈醉忽大雷雨暴至克用不覺侍人乃滅燭推於牀下藏之蕃戍與我師鬬戰移時方敗楊彦洪中流矢而斃是時隂黑克用遇一卒背負登尉氏門因得懸縋而出乘牛行數里以投其衆餘親衛數百人皆勦之其後克用至太原以是事表訴于唐帝蒲帥亦繼馳書請上與克用和解上終不釋憾此乃敬翔飾非今不取實録甲戌李克用次汴州駐軍近郊朱全忠請館于上源驛乃以腹心三百餘自衛全忠以克用兵從簡少大軍在遠謀害之是夜置酒宴罷以兵圍驛縱火焚之薛居正五代史梁太祖紀曰五月甲戌帝與晉軍振旅歸汴館克用於上源驛既而備犒宴之禮克用乘醉任氣帝不平之是夜命甲士圍而攻之後唐武皇紀曰班師過汴汴帥迎勞於封禪寺請武皇休於府第乃館於上源驛是夜張樂陳宴席武皇酒酣戲諸侍妓與汴帥握手叙破賊以為樂汴帥素忌武皇乃與其將楊彦洪密謀竊攻傳舍按全忠是時兵力尚微天下所與為敵者非特患克用一人而借使殺之不能併其軍奪其地也蓋克用恃功語或輕慢全忠出於一時之忿耳今從薛史梁紀】克用妻劉氏多智略左右先脱歸者以汴人為變告劉氏神色不動立斬之隂召大將約束謀保軍以還比明克用至【還從宣翻比必利翻及也】欲勒兵攻全忠劉氏曰公比為國討賊救東諸侯之急【比毗至翻近也為于偽翻東諸侯用左傳語謂東方諸鎮】今汴人不道乃謀害公自當訴之朝廷若擅舉兵相攻則天下孰能辯其曲直且彼得以有辭矣克用從之引兵去但移書責全忠全忠復書曰前夕之變僕不之知朝廷自遣使者與楊彦洪為謀彦洪既伏其辜惟公亮察克用養子嗣源年十七從克用自上源出矢石之間獨無所傷嗣源本胡人名邈信烈無姓【李嗣源始此佶極吉翻】克用擇軍中驍勇者多養為子名回鶻張政之子曰存信【按薛居正五代史存信本名張汚落】振武孫重進曰存進許州王賢曰存賢安敬思曰存孝皆冒姓李氏【此所謂義兒也歐陽修曰唐自沙陀起代北其所與俱皆一時雄傑虓武之士往往養為兒號義兒軍】丙子克用至許州故寨求糧於周岌岌辭以糧乏乃自陜濟河還晉陽 鄭君雄張士安堅壁不出高仁厚曰攻之則彼利我傷圍之則彼困我逸遂列十二寨圍之丁丑夜二鼓【夜二鼔夜二更也持更者每一更則鼔一聲二更則鼓二聲故謂二更為二鼓亦謂之乙夜】君雄等出勁兵掩撃城北副使寨楊茂言不能禦帥衆弃寨走【帥讀曰率下同】其旁數寨見副使走亦走東川人併兵南攻中軍仁厚聞之大開寨門設炬火照之自帥士卒為兩翼伏道左右賊至見門開不敢入還去仁厚伏撃之東川兵大奔追至城下蹙之壕中斬獲甚衆而還仁厚念諸弃寨走者明旦所當誅殺甚多乃密召孔目官張韶諭之曰爾速遣步探子將數十人分道追走者【步探子遣之間步以刺探敵人因名之探它紺翻】自以爾意諭之曰僕射幸不出寨皆不知【仁厚以平阡能等之功進檢校僕射】汝曹速歸來旦牙參勿憂也【凡行營諸將每旦赴大將營牙參】韶素名長者【長知兩翻】衆信之至四鼓皆還寨惟楊茂言走至張把乃追及之【九域志梓州郪縣有張把鎮把當作杷】仁厚聞諸寨漏鼓如故喜曰悉歸矣詰旦諸將牙集【詰去吉翻】以為仁厚誠不知也坐良久仁厚謂茂言曰昨夜聞副使身先士卒走至張把有諸【先悉薦翻】對曰昨夜聞賊攻中軍左右言僕射已去遂策馬參隨既而審其虛復還寨中仁厚曰仁厚與副使俱受命天子將兵討賊若仁厚先走副使當叱下馬行軍法代總軍事然後奏聞今副使既先走又為欺罔理當何如茂言拱手曰當死仁厚曰然命左右扶下斬之諸將股栗仁厚乃召昨夜所俘虜數十人釋縛縱歸【縱俘使歸言其事】君雄等聞之懼曰彼軍法嚴整如是自今兵不可復出矣【左傳晉人伐鄭蒐焉而還中行桓子之謀也曰示之以整使謀而來鄭於是懼其後卒請成於晉用兵嚴整敵人懼之蓋自古然矣復扶又翻】 庚辰時溥遣其將李師悦將兵萬人追黄巢 癸未高仁厚陳於鹿頭關城下【陳讀曰陣下同】鄭君雄等悉衆出戰仁厚設伏於陳後陽敗走君雄等追之伏君雄等大敗是夕遁歸梓州陳敬瑄兵三千以益仁厚軍進圍梓州<br />
<br />
  資治通鑑卷二百五十五<br />
<br />
<史部,編年類,資治通鑑>  <br>
   </div> 

<script src="/search/ajaxskft.js"> </script>
 <div class="clear"></div>
<br>
<br>
 <!-- a.d-->

 <!--
<div class="info_share">
</div> 
-->
 <!--info_share--></div>   <!-- end info_content-->
  </div> <!-- end l-->

<div class="r">   <!--r-->



<div class="sidebar"  style="margin-bottom:2px;">

 
<div class="sidebar_title">工具类大全</div>
<div class="sidebar_info">
<strong><a href="http://www.guoxuedashi.com/lsditu/" target="_blank">历史地图</a></strong>  
<a href="http://www.880114.com/" target="_blank">英语宝典</a>  
<a href="http://www.guoxuedashi.com/13jing/" target="_blank">十三经检索</a> 
<br><strong><a href="http://www.guoxuedashi.com/gjtsjc/" target="_blank">古今图书集成</a></strong> 
<a href="http://www.guoxuedashi.com/duilian/" target="_blank">对联大全</a> <strong><a href="http://www.guoxuedashi.com/xiangxingzi/" target="_blank">象形文字典</a></strong> 

<br><a href="http://www.guoxuedashi.com/zixing/yanbian/">字形演变</a>  <strong><a href="http://www.guoxuemi.com/hafo/" target="_blank">哈佛燕京中文善本特藏</a></strong>
<br><strong><a href="http://www.guoxuedashi.com/csfz/" target="_blank">丛书&方志检索器</a></strong> <a href="http://www.guoxuedashi.com/yqjyy/" target="_blank">一切经音义</a>  

<br><strong><a href="http://www.guoxuedashi.com/jiapu/" target="_blank">家谱族谱查询</a></strong>  <strong><a href="http://shufa.guoxuedashi.com/sfzitie/" target="_blank">书法字帖欣赏</a></strong> 
<br>

</div>
</div>


<div class="sidebar" style="margin-bottom:0px;">

<font style="font-size:22px;line-height:32px">QQ交流群9:489193090</font>


<div class="sidebar_title">手机APP 扫描或点击</div>
<div class="sidebar_info">
<table>
<tr>
	<td width=160><a href="http://m.guoxuedashi.com/app/" target="_blank"><img src="/img/gxds-sj.png" width="140"  border="0" alt="国学大师手机版"></a></td>
	<td>
<a href="http://www.guoxuedashi.com/download/" target="_blank">app软件下载专区</a><br>
<a href="http://www.guoxuedashi.com/download/gxds.php" target="_blank">《国学大师》下载</a><br>
<a href="http://www.guoxuedashi.com/download/kxzd.php" target="_blank">《汉字宝典》下载</a><br>
<a href="http://www.guoxuedashi.com/download/scqbd.php" target="_blank">《诗词曲宝典》下载</a><br>
<a href="http://www.guoxuedashi.com/SiKuQuanShu/skqs.php" target="_blank">《四库全书》下载</a><br>
</td>
</tr>
</table>

</div>
</div>


<div class="sidebar2">
<center>


</center>
</div>

<div class="sidebar"  style="margin-bottom:2px;">
<div class="sidebar_title">网站使用教程</div>
<div class="sidebar_info">
<a href="http://www.guoxuedashi.com/help/gjsearch.php" target="_blank">如何在国学大师网下载古籍?</a><br>
<a href="http://www.guoxuedashi.com/zidian/bujian/bjjc.php" target="_blank">如何使用部件查字法快速查字?</a><br>
<a href="http://www.guoxuedashi.com/search/sjc.php" target="_blank">如何在指定的书籍中全文检索?</a><br>
<a href="http://www.guoxuedashi.com/search/skjc.php" target="_blank">如何找到一句话在《四库全书》哪一页?</a><br>
</div>
</div>


<div class="sidebar">
<div class="sidebar_title">热门书籍</div>
<div class="sidebar_info">
<a href="/so.php?sokey=%E8%B5%84%E6%B2%BB%E9%80%9A%E9%89%B4&kt=1">资治通鉴</a> <a href="/24shi/"><strong>二十四史</strong></a>&nbsp; <a href="/a2694/">野史</a>&nbsp; <a href="/SiKuQuanShu/"><strong>四库全书</strong></a>&nbsp;<a href="http://www.guoxuedashi.com/SiKuQuanShu/fanti/">繁体</a>
<br><a href="/so.php?sokey=%E7%BA%A2%E6%A5%BC%E6%A2%A6&kt=1">红楼梦</a> <a href="/a/1858x/">三国演义</a> <a href="/a/1038k/">水浒传</a> <a href="/a/1046t/">西游记</a> <a href="/a/1914o/">封神演义</a>
<br>
<a href="http://www.guoxuedashi.com/so.php?sokeygx=%E4%B8%87%E6%9C%89%E6%96%87%E5%BA%93&submit=&kt=1">万有文库</a> <a href="/a/780t/">古文观止</a> <a href="/a/1024l/">文心雕龙</a> <a href="/a/1704n/">全唐诗</a> <a href="/a/1705h/">全宋词</a>
<br><a href="http://www.guoxuedashi.com/so.php?sokeygx=%E7%99%BE%E8%A1%B2%E6%9C%AC%E4%BA%8C%E5%8D%81%E5%9B%9B%E5%8F%B2&submit=&kt=1"><strong>百衲本二十四史</strong></a>  <a href="http://www.guoxuedashi.com/so.php?sokeygx=%E5%8F%A4%E4%BB%8A%E5%9B%BE%E4%B9%A6%E9%9B%86%E6%88%90&submit=&kt=1"><strong>古今图书集成</strong></a>
<br>

<a href="http://www.guoxuedashi.com/so.php?sokeygx=%E4%B8%9B%E4%B9%A6%E9%9B%86%E6%88%90&submit=&kt=1">丛书集成</a> 
<a href="http://www.guoxuedashi.com/so.php?sokeygx=%E5%9B%9B%E9%83%A8%E4%B8%9B%E5%88%8A&submit=&kt=1"><strong>四部丛刊</strong></a>  
<a href="http://www.guoxuedashi.com/so.php?sokeygx=%E8%AF%B4%E6%96%87%E8%A7%A3%E5%AD%97&submit=&kt=1">說文解字</a> <a href="http://www.guoxuedashi.com/so.php?sokeygx=%E5%85%A8%E4%B8%8A%E5%8F%A4&submit=&kt=1">三国六朝文</a>
<br><a href="http://www.guoxuedashi.com/so.php?sokeytm=%E6%97%A5%E6%9C%AC%E5%86%85%E9%98%81%E6%96%87%E5%BA%93&submit=&kt=1"><strong>日本内阁文库</strong></a> <a href="http://www.guoxuedashi.com/so.php?sokeytm=%E5%9B%BD%E5%9B%BE%E6%96%B9%E5%BF%97%E5%90%88%E9%9B%86&ka=100&submit=">国图方志合集</a> <a href="http://www.guoxuedashi.com/so.php?sokeytm=%E5%90%84%E5%9C%B0%E6%96%B9%E5%BF%97&submit=&kt=1"><strong>各地方志</strong></a>

</div>
</div>


<div class="sidebar2">
<center>

</center>
</div>
<div class="sidebar greenbar">
<div class="sidebar_title green">四库全书</div>
<div class="sidebar_info">

《四库全书》是中国古代最大的丛书,编撰于乾隆年间,由纪昀等360多位高官、学者编撰,3800多人抄写,费时十三年编成。丛书分经、史、子、集四部,故名四库。共有3500多种书,7.9万卷,3.6万册,约8亿字,基本上囊括了古代所有图书,故称“全书”。<a href="http://www.guoxuedashi.com/SiKuQuanShu/">详细>>
</a>

</div> 
</div>

</div>  <!--end r-->

</div>
<!-- 内容区END --> 

<!-- 页脚开始 -->
<div class="shh">

</div>

<div class="w1180" style="margin-top:8px;">
<center><script src="http://www.guoxuedashi.com/img/plus.php?id=3"></script></center>
</div>
<div class="w1180 foot">
<a href="/b/thanks.php">特别致谢</a> | <a href="javascript:window.external.AddFavorite(document.location.href,document.title);">收藏本站</a> | <a href="#">欢迎投稿</a> | <a href="http://www.guoxuedashi.com/forum/">意见建议</a> | <a href="http://www.guoxuemi.com/">国学迷</a> | <a href="http://www.shuowen.net/">说文网</a><script language="javascript" type="text/javascript" src="https://js.users.51.la/17753172.js"></script><br />
  Copyright &copy; 国学大师 古典图书集成 All Rights Reserved.<br>
  
  <span style="font-size:14px">免责声明:本站非营利性站点,以方便网友为主,仅供学习研究。<br>内容由热心网友提供和网上收集,不保留版权。若侵犯了您的权益,来信即刪。scp168@qq.com</span>
  <br />
ICP证:<a href="http://www.beian.miit.gov.cn/" target="_blank">鲁ICP备19060063号</a></div>
<!-- 页脚END --> 
<script src="http://www.guoxuedashi.com/img/plus.php?id=22"></script>
<script src="http://www.guoxuedashi.com/img/tongji.js"></script>

</body>
</html>
