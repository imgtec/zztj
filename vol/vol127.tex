資治通鑑卷一百二十七 宋 司馬光 撰

胡三省 音註

宋紀九|{
	昭陽大荒落一年}


太祖文皇帝下之下

元嘉三十年春正月戊寅以南譙王義宣為司徒揚州刺史|{
	用義宣刺揚州至是始出命}
蕭道成等帥氐羌攻魏武都魏高平鎮將苟莫于將突騎二千救之|{
	帥讀曰率將即亮翻騎奇寄翻}
道成等引還南鄭|{
	南鄭宋梁南秦二州刺史治所兵志所謂知難而退蕭道成有焉}
壬午以征北將軍始興王濬為荆州刺史帝怒未解故濬久留京口既除荆州乃聽入朝|{
	朝直遙翻}
戊子詔江州刺史武陵王駿統諸軍討西陽蠻軍于五洲|{
	水經注江水東逕江夏縣故城南縣故弦國也城在山之陽南對五洲江中有五洲相接故以為名其地當在今黄州江州之間孟康曰音汰師古曰又音徒系翻}
嚴道育之亡命也|{
	道育亡命事始上卷上年}
上分遣使者搜捕甚急|{
	使疏吏翻}
道育變服為尼匿於東宫又随始興王濬至京口或出上民張旿家|{
	旿疑古翻}
濬入朝復載還東宫|{
	復扶又翻}
欲與俱往江陵丁巳上臨軒濬入受拜|{
	受拜荆州刺史之命}
是日有告道育在張旿家者上遣掩捕得其二婢云道育随征北還都|{
	濬為征北將軍故稱之}
上謂濬與太子劭已斥遣道育而聞其猶與往來惆悵惋駭|{
	惆丑鳩翻惋烏貫翻}
乃命京口送二婢須至檢覆乃治劭濬之罪|{
	言待二婢至檢覆覆審其事乃罪二子也治直之翻}
潘淑妃抱濬泣曰汝前祝詛事|{
	事見上卷上年祝讀與呪同職敕翻}
猶冀能刻意思愆何意更藏嚴道育上怒甚我叩頭乞恩不能解今何用生為可送藥來當先自取盡|{
	謂欲先自殺也}
不忍見汝禍敗也濬奮衣起曰天下事尋自當判願小寛慮必不上累|{
	累力瑞翻判决也欲决意為商臣之事也濬辭氣凶悖如此潘妃承帝寵又如此而不以濬言白上何也婦人之仁知愛子而欲掩覆之不知其變愈激也}
己未魏京兆王杜元寶坐謀反誅建寧王崇及其子濟南王麗皆為元寶所引賜死|{
	史言魏難未巳濟子禮翻}
帝欲廢太子劭賜始興王濬死先與侍中王僧綽謀之使僧綽尋漢魏以來廢太子諸王典故|{
	典經常之籍也故舊事也}
送尚書僕射徐湛之及吏部尚書江湛|{
	送與故與二人也}
武陵王駿素無寵故屢出外藩不得留建康|{
	駿自彭城還復出刺江州}
南平王鑠建平王宏皆為帝所愛鑠妃江湛之妹随王誕妃徐湛之之女也湛勸帝立鑠湛之意欲立誕|{
	史言江徐各私其私以亂國殺身}
僧綽曰建立之事仰由聖懷臣謂唯宜速斷不可稽緩當斷不斷反受其亂|{
	按漢書齊相召平所引道家之言斷丁亂翻下同}
願以義割恩略小不忍|{
	論語孔子曰小不忍則亂大謀}
不爾便應坦懷如初|{
	謂坦懷待之如父子天性之初也}
無煩疑論事機雖密易致宣廣不可使難生慮表取笑千載|{
	易以豉翻難乃旦翻載祖亥翻言禍難生於思慮之外將取笑於後世也}
帝曰卿可謂能斷大事然此事至重不可不慇懃三思且彭城始亡|{
	彭城王義康死見上卷二十八年三息暫翻又音如字}
人將謂我無復慈愛之道|{
	復扶又翻}
僧綽曰臣恐千載之後言陛下惟能裁弟不能裁兒帝默然江湛同侍坐|{
	坐徂卧翻}
出閣謂僧綽曰卿向言將不太傷切直僧綽曰弟亦恨君不直|{
	僧綽年少於湛故自稱為弟}
鑠自壽陽入朝既至失旨帝欲立宏嫌其非次|{
	建平王宏之齒末也於兄弟長幼之序為非次}
是以議久不决每夜與湛之屏人語或連日累夕常使湛之自秉燭繞壁檢行慮有竊聽者|{
	屏必郢翻帝自以為謀莫密於此矣}
帝以其謀告潘淑妃淑妃以告濬|{
	左氏傳有言謀及婦人宜其死也宋文帝處此事其識畧又在吳孫亮之下}
濬馳報劭劭乃密與腹心隊主陳叔兒齋帥張超之等謀為逆|{
	齋帥主齋内仗衛又掌湯沐燈燭汛掃鋪設帥所類翻}
初帝以宗室彊盛慮有内難|{
	慮諸弟為難也難乃旦翻}
特加東宫兵使與羽林相若|{
	事見一百二十三卷十六年}
至有實甲萬人 |{
	考異曰宋元凶劭傳云二十八年彗星入太微掃帝座二十九年十一月霖雨連雪太陽罕曜三十年正月風霰且雷上憂有竊輒加劭兵衆東宫實甲萬人按二十九年劭濬巫蠱事已豈有因十二月及明年正月災異而更加劭兵今從宋畧}
劭性黠而剛猛|{
	黠下八翻桀也慧也}
帝深倚之及將作亂每夜饗將士或親自行酒王僧綽密以啟聞|{
	王僧綽又啟聞此事劭之逆狀彰灼無可疑者而帝猶豫不斷殆天奪之鑒也將即亮翻}
會嚴道育婢將至癸亥夜 |{
	考異曰劭傳云二十一日夜按長歷是月甲辰朔宋畧云癸亥夜乃二十日也今從之}
劭詐為帝詔云魯秀謀反汝可平明守闕帥衆入|{
	帥讀曰率}
因使張超之等集素所畜養兵士二千餘人皆被甲|{
	畜許六翻被皮義翻}
召内外幢隊主副豫加部勒云有所討|{
	幢傳江翻}
夜呼前中庶子右軍長史蕭斌|{
	蕭斌前嘗為太子中庶子而此時則為右軍長史也斌音彬}
左衛率袁淑中舍人殷仲素左積弩將軍王正見|{
	晉武帝泰始四年罷振威揚威護軍置左右積弩將軍宋齊之制東宫亦置左右積弩將軍}
並入宫劭流涕謂曰主上信讒將見罪廢内省無過不能受枉|{
	省所景翻}
明旦當行大事|{
	左傳楚潘崇謂商臣曰能行大事乎對曰能遂以宫甲圍其父成王而弑之}
望相與戮力因起徧拜之衆驚愕莫敢對淑斌皆曰自古無此願加善思|{
	善思猶今人言好思量也}
劭怒變色斌懼與衆俱曰當竭身奉令淑叱之曰卿便謂殿下真有是邪殿下幼嘗患風或是疾動耳|{
	言病風喪心或致有是言}
劭愈怒因眄淑曰事當克不|{
	眄眠見翻目偏合而斜視也不讀曰否}
淑曰居不疑之地何患不克但恐既克之後不為天地所容大禍亦旋至耳|{
	旋還反也疾也}
假有此謀猶將可息左右引淑出曰此何事而云可罷乎淑還省|{
	還左衛率省也}
繞床行至四更乃寢|{
	更工衡翻}
甲子宫門未開劭以朱衣加戎服上乘畫輪車|{
	朱衣太子入朝之服晉志曰畫輪車駕牛以綵漆畫輪轂故名曰畫輪車上起四夾杖左右開四望緑油幢朱絲絡其上形制事事如輦其下猶如犢車耳太子法駕亦謂之鸞路非法駕則乘畫輪車兩箱裏飾以金錦黄金塗五采}
與蕭斌共載衛從如常入朝之儀|{
	從才用翻朝直遥翻}
呼袁淑甚急淑眠不起劭停車奉化門|{
	奉化門東宫西門}
催之相續淑徐起至車後劭使登車又辭不上|{
	上時掌翻}
劭命左右殺之守門開|{
	停留以候門開曰守}
從萬春門入|{
	萬春門臺城東門}
舊制東宫隊從不得入城|{
	言不得入臺城也}
劭以偽詔示門衛曰受勑有所收討令後隊速來張超之等數十人馳入雲龍門及齋閣拔刀徑上合殿|{
	李延夀曰晉世諸帝多處内房朝宴所臨東西二堂而已孝武末年清暑方構永初受命無所改作所居惟稱西殿不製嘉名文帝因之亦有合殿之稱}
帝是夜與徐湛之屛人語至旦燭猶未滅門階戶席直衛兵尚寢未起帝見超之入舉几捍之五指皆落遂弑之|{
	年四十七}
湛之驚起趣北戶未及開兵人殺之|{
	趣七喻翻}
劭進至合殿中閤聞帝已殂出坐東堂蕭斌執刀侍直呼中書舍人顧嘏嘏震懼不時出既至問曰欲共見廢何不早啟嘏未及答即於前斬之江湛直上省|{
	侍中省有上省下省上省在禁中湛時為侍中入直上省}
聞諠譟聲歎曰不用王僧綽言以至於此乃匿傍小屋中劭遣兵就殺之宿衛舊將羅訓徐罕皆望風屈附|{
	南史卜天與傳作徐牢將即亮翻}
左細仗主廣威將軍吳興卜天與|{
	宋宿衛之官有細鎧主細鎧將細仗主等}
不暇被甲|{
	被皮義翻}
執刀持弓疾呼左右出戰徐罕曰殿下入汝欲何為天與罵曰殿下常來云何於今乃作此語只汝是賊手射劭於東堂幾中之|{
	射而亦翻幾居希翻中竹仲翻}
劭黨擊之斷臂而死隊將張泓之朱道欽陳滿與天與俱戰死|{
	斷丁管翻將即亮翻}
左衛將軍尹弘惶怖通啟求受處分|{
	怖普布翻處昌呂翻分扶問翻}
劭使人從東閤入|{
	東閤東閤門也}
殺潘淑妃及太祖親信左右數十人|{
	劭尊帝廟號中宗孝武帝即位改廟號曰太祖}
急召始興王濬使帥衆屯中堂濬時在西州|{
	濬自京口入朝蹔居西州帥讀曰率}
府舍人朱法瑜|{
	府舍人者濬府之舍人也自晉以來諸王府舍人十人}
奔告濬曰臺内喧譟宫門皆閉道上傳太子反未測禍變所至濬陽驚曰今當奈何法瑜勸入據石頭濬未得劭信不知事之濟不|{
	濟不讀曰否}
騷擾不知所為將軍王慶曰今宫内有變未知主上安危凡在臣子當投袂赴難|{
	難乃旦翻}
憑城自守非臣節也濬不聽乃從南門出徑向石頭文武從者千餘人時南平王鑠戍石頭兵士亦千餘人|{
	從才用翻史言濬鑠之衆足以討除逆亂}
俄而劭遣張超之馳馬召濬濬屏人問狀|{
	屏必逞翻}
即戎服乘馬而去朱法瑜固止濬濬不從出中門王慶又諫曰太子反逆天下怨憤明公但當堅閉城門坐食積粟|{
	石頭倉城有積粟}
不過三日凶黨自離公情事如此今豈宜去濬曰皇太子令敢有復言者斬|{
	復扶又翻}
既入見劭劭曰潘淑妃遂為亂兵所害濬曰此是下情由來所願|{
	梟食母破獍食父若濬者兼梟獍之心以為心}
劭詐以太祖詔召大將軍義恭尚書令何尚之入拘於内|{
	内謂臺内}
并召百官至者纔數十人劭遽即位下詔曰徐湛之江湛弑逆無狀吾勒兵入殿已無所及號惋崩衂|{
	號戶刀翻惋烏貫翻衂女六翻}
肝心破裂今罪人斯得元凶克殄可大赦改元太初即位畢亟稱疾還永福省|{
	永福省太子所居也在禁中}
不敢臨喪以白刃自守夜則列燈以防左右以蕭斌為尚書僕射領軍將軍以何尚之為司空前右衛率檀和之戍石頭征虜將軍營道侯義綦鎮京口義綦義慶之弟也|{
	義慶長沙王道憐第二子嗣臨川王道規國}
乙丑悉收先給諸處兵還武庫殺江徐親黨尚書左丞荀赤松右丞臧凝之等凝之燾之孫也以殷仲素為黄門侍郎王正見為左軍將軍張超之陳叔兒皆拜官賞賜有差輔國將軍魯秀在建康劭謂秀曰徐湛之常欲相危|{
	事見上卷二十八年}
我已為卿除之矣|{
	為于偽翻}
使秀與屯騎校尉龎秀之對掌軍隊|{
	騎奇寄翻校戶教翻軍隊軍主隊主所統之兵}
劭不知王僧綽之謀以僧綽為吏部尚書|{
	王僧綽于此時不受劭官繼之以死則人臣之節盡矣}
司徒左長史何偃為侍中武陵王駿屯五洲沈慶之自巴水來咨受軍略|{
	水經巴水出廬江雩婁縣之巴山南歷蠻中又南流注于江今謂之巴河在蘄州界源出板石山去年帝使沈慶之討蠻是年使武陵王駿統討蠻諸軍故慶之來詣駿咨受軍畧軍畧謂用兵之策畧也}
三月乙亥典籖董元嗣|{
	武陵王鎮彭城董元嗣已為府典籖}
自建康至五洲具言太子殺逆|{
	殺讀曰弑}
駿使元嗣以告僚佐|{
	宣劭弑逆之罪將舉兵也}
沈慶之密謂腹心曰蕭斌婦人|{
	言其怯弱無能為也}
其餘將帥皆易與耳|{
	易以豉翻}
東宫同惡不過三十人|{
	謂張超之陳叔兒等}
此外屈逼|{
	謂魯秀龎秀之等}
必不為用今輔順討逆|{
	順謂武陵王逆謂劭也}
不憂不濟也|{
	沈慶之以此言作諸人義勇之氣}
壬午魏主尊保太后為皇太后|{
	尊保太后見上卷上年以乳母為母非禮也}
追贈祖考官爵兄弟皆如外戚|{
	史言魏主寵秩私昵之遏}
太子劭分浙東五郡為會州|{
	以會稽名州也會古外翻}
省揚州立司隸校尉|{
	浙東五郡本屬揚州分為會州又改揚州為司隸校尉以統京畿欲倣魏晉都洛舊制}
以其妃父殷冲為司隸校尉冲融之曾孫也|{
	殷融見九十四卷晉成帝咸和三年}
以大將軍義恭為太保荆州刺史南譙王義宣為太尉始興王濬為驃騎將軍|{
	驃匹妙翻騎奇計翻}
雍州刺史臧質為丹陽尹|{
	雍於用翻}
會稽太守随王誕為會州刺史|{
	欲就會稽用誕統浙東五郡}
劭料檢文帝巾箱|{
	料音聊巾箱所以藏要密文書便於尋閲}
及江湛家書疏得王僧綽所啟饗士并前代故事|{
	即所上廢太子諸王典故疏所去翻}
甲申收僧綽殺之僧綽弟僧䖍為司徒左西屬|{
	左西屬左西曹屬也舊制司徒府有東西曹曹有掾有屬宋於西曹又分左右}
所親咸勸之逃僧䖍泣曰吾兄奉國以忠貞撫我以慈愛今日之事苦不見及耳若得同歸九泉猶羽化也|{
	羽化猶言登仙神仙家所謂飛昇也}
劭因誣北第諸王侯云與僧綽謀反|{
	諸王侯列第於臺城北故曰北第此皆穆武子孫也}
殺長沙悼王瑾瑾弟臨川哀王曄|{
	臨川王義慶本長沙王道憐之子嗣臨川王道規今曄又以長沙王瑾弟嗣義慶瑾渠吝翻}
桂陽孝侯覬新渝懷侯玠|{
	覬音冀新渝當作新喻 考異曰劭傳作球今從長沙王道憐傳}
皆劭所惡也|{
	惡烏路翻}
瑾義欣之子|{
	義欣長沙王道憐之子}
曄義慶之子覬玠義慶之弟子也劭密與沈慶之手書令殺武陵王駿慶之求見王王懼辭以疾慶之突入以劭書示王王泣求入内與母訣|{
	武陵王母路淑媛}
慶之曰下官受先帝厚恩今日之事惟力是視殿下何見疑之深王起再拜曰家國安危皆在將軍慶之即命内外勒兵府主簿顔竣曰|{
	竣七倫翻}
今四方未知義師之舉劭據有天府|{
	天府謂建康}
若首尾不相應|{
	首謂武陵已倡義于九江尾謂諸方征鎮}
此危道也宜待諸鎮恊謀然後舉事慶之厲聲曰今舉大事而黄頭小兒皆得參預|{
	男女始生為黄頭小兒言其如嬰兒未有知識也}
何得不敗宜斬以狥王令竣拜謝慶之慶之曰君但當知筆札事耳於是專委慶之處分旬日之間内外整辦人以為神兵|{
	宋帝紀曰三月乙未建牙于軍門是時多不悉舊儀有一翁班白自稱少從武帝征伐頗悉其事因使指麾事畢忽失所在余謂沈慶之甚練軍事西征北伐久在兵間安有不悉舊儀之理或者舉義之時託武帝神靈以昭神人之助順啟諸方赴義之心也通鑑不語怪故不書處昌呂翻分扶問翻}
竣延之之子也|{
	顔延之與謝靈運俱以文義著稱靈運死延之獨擅名於時時在建康}
庚寅武陵王戒嚴誓衆以沈慶之領府司馬襄陽太守柳元景随郡太守宗慤為諮議參軍領中兵江夏内史朱修之行平東將軍記室參軍顔竣為諮議參軍領録事兼總内外|{
	柳元景宗慤以諮議參軍領中兵參軍以前驅之任命二人也顔竣本記室參軍陞諮議領録事參軍以總録軍府之責任命竣也記室參軍掌牋記夏戶雅翻}
諮議參軍劉延孫為長史尋陽太守行留府事延孫道產之子也|{
	劉道產鎮襄陽有政績見一百二十四卷十九年}
南譙王義宣及臧質皆不受劭命與司州刺史魯爽同舉兵以應駿質爽俱詣江陵見義宣|{
	司雍皆受督于義宣故俱詣之}
且遣使勸進於王|{
	使疏吏翻}
辛卯臧質子敦等在建康者聞質舉兵皆逃亡|{
	考異曰宋畧庚申武陵王戒嚴辛亥臧敷逃按長歷是月甲戌朔無庚申辛亥又宋畧上有甲申下有癸巳}


|{
	此必庚寅辛卯字誤也宋書敷作敦今從之}
劭欲相慰悦下詔曰臧質國戚勲臣|{
	臧質高祖敬皇后之姪故曰國戚有邉功故曰勲臣}
方翼贊京輦|{
	謂用為丹楊尹也}
而子弟波迸良可怪歎|{
	迸北諍翻}
可遣宣譬令還咸復本位劭尋録得敦|{
	毛晃曰録收拾也}
使大將軍義恭行訓杖三十|{
	以外戚子弟行杖以訓勅之故曰訓杖}
厚給賜之 癸巳劭葬太祖于長寧陵|{
	據齊書豫章王嶷傳長寧陵隧道出嶷第前路則陵近臺城矣}
諡曰景皇帝廟號中宗|{
	史不用劭所上諡號而用孝武帝所改諡號正劭弑逆之罪絶之也}
乙未武陵王發西陽丁酉至尋陽庚子王命顔竣移檄四方 |{
	考異曰宋略移檄亦在庚申日按謝莊傳曰奉三月二十七日檄然則檄在庚子日也}
使共討劭州郡承檄翕然響應南譙王義宣遣臧質引兵詣尋陽與駿同下留魯爽於江陵邵以兖冀二州刺史蕭思話為徐兖二州刺史起張永為青州刺史思話自歷城引部曲還平城起兵以應尋陽|{
	濟南郡東平陵縣有平陸城余謂平城當作彭城還從宣翻又如字}
建武將軍垣護之在歷城亦帥所領赴之|{
	帥讀曰率下同}
南譙王義宣板張永為冀州刺史永遣司馬崔勲之等將兵赴義宣|{
	將即亮翻}
義宣慮蕭思話與永不釋前憾|{
	思話繫張永於獄事見上卷上年}
自為書與思話使長史張暢為書與永|{
	張暢永之羣從也故義宣使之為書}
勸使相與坦懷随王誕將受劭命|{
	受會州刺史之命}
參軍事沈正說司馬顧琛曰|{
	說輸芮翻}
國家此禍開闢未聞今以江東驍鋭之衆|{
	此江東謂浙江之東也驍堅堯翻}
唱大義於天下其誰不響應豈可使殿下北面兇逆受其偽寵乎琛曰江東忘戰日久雖逆順不同然彊弱亦異|{
	琛意謂雖以順討逆然建康彊而江東弱其勢異也}
當須四方有義舉者|{
	須待也}
然後應之不為晚也正曰天下未嘗有無父無君之國寧可自安讐恥而責義於餘方乎今正以弑逆寃酷義不共戴天|{
	禮記曰父母之讐不共戴天}
舉兵之日豈求必全邪馮衍有言大漢之貴臣將不如荆齊之賤士乎|{
	此蓋馮衍責田邑之言荆齊之賤士謂申包胥赴秦求救以存荆王孫賈殺淖齒以存齊也}
况殿下義兼臣子事實國家者哉琛乃與正共入說誕誕從之|{
	說輸芮翻}
正田子之兄子也|{
	沈田子從武帝入關有功後以殺王鎮惡受誅}
劭自謂素習武事語朝士曰卿等但助我理文書勿措意戎旅若有寇難|{
	語牛倨翻朝直遥翻難乃旦翻}
吾自當之但恐賊虜不敢動耳及聞四方兵起始憂懼戒嚴悉召下番將吏|{
	宿衛分上下番更休迭代今悉召下番將吏以自備更不分番}
遷淮南居民於北岸|{
	秦淮南岸當新亭石頭來路北岸即臺城遷淮南居民於北岸欲阻淮以自固}
盡聚諸王及大臣於城内|{
	防其出奔也}
移江夏王義恭處尚書下舍分義恭諸子處侍中下省|{
	處昌呂翻據南史侍中下省在神虎門外}
夏四月癸卯朔柳元景統寧朔將軍薛安都等十二軍湓口|{
	湓音盆}
司空中兵參軍徐遺寶以荆州之衆繼之|{
	南譙王義宣既進位司空以徐遺寶為中兵參軍}
丁未武陵王尋陽沈慶之總中軍以從|{
	從才用翻}
劭立妃殷氏為皇后庚戍武陵王檄書至建康劭以示太常顔延之曰彼誰筆也延之曰竣之筆也劭曰言辭何至於是延之曰竣尚不顧老臣安能顧陛下劭怒稍解悉拘武陵王子於侍中下省南譙王義宣子於太倉空舍劭欲盡殺三鎮士民家口|{
	三鎮謂雍荆江}
江夏王義恭何尚之皆曰凡舉大事者不顧家且多是驅逼今忽誅其室累正足堅彼意耳|{
	累切瑞翻}
劭以為然乃下書一無所問劭疑朝廷舊臣皆不為己用乃厚撫魯秀及右軍參軍王羅漢悉以軍事委之|{
	二人皆驍勇善戰故厚撫之委以軍事冀得其力}
以蕭斌為謀主殷冲掌文符蕭斌勸劭勒水軍自上决戰不爾則保據梁山|{
	上時掌翻今太平州當塗縣西南三十里有天門山亦曰蛾眉山兩山夾大江對峙東曰博望山西曰梁山}
江夏王義恭以南軍倉猝船舫陋小不利水戰|{
	江水東流至武昌以下漸漸向北流蓋南紀諸山所迫坡陁之勢漸使之然也至于江寧江流愈北建康當下流都會望尋陽武昌皆直南望歷陽夀陽皆直西故建康謂歷陽皖城以西皆曰江西而江西亦謂建康為江東建康謂采石為南州京口為北府皆地勢然也江夏王義恭在建康以義師為南軍即此義舫甫妄翻}
乃進策曰賊駿小年未習軍旅遠來疲弊宜以逸待之今遠出梁山則京都空弱東軍乘虛或能為患|{
	東軍謂會稽随王誕之兵也}
若分力兩赴則兵散勢離不如養鋭待期坐而觀舋割棄南岸柵斷石頭此先朝舊法|{
	舋許靳翻斷丁管翻朝直遥翻先朝舊法謂晉明帝拒王含及武帝拒盧循時用兵之法}
不憂賊不破也劭善之斌厲色曰南中郎二十年少能建如此大事豈復可量|{
	時武陵王駿為南中郎將江州刺史故稱之武陵王時年二十四少詩照翻復扶又翻量音良}
三方同惡勢據上流|{
	三方謂荆雍江}
沈慶之甚練軍事柳元景宗慤屢嘗立功|{
	沈慶之常與蕭斌同在碻磝柳元景討蠻出關陜皆有功宗慤有平林邑之功又有討蠻之功故斌皆憚之}
形勢如此實非小敵唯宜及人情未離尚可决力一戰端坐臺城何由得久今主相咸無戰意豈非天也|{
	弑逆事起蕭斌以宫僚之舊逼於兇威遂為同惡其心慙負天地無所自容唯欲幸一戰之勝相與苟活今劭不背逆戰斌知必敗故歸之天相息亮翻}
劭不聽或勸劭保石頭城劭曰昔人所以固石頭城者俟諸侯勤王耳我若守此誰當見救唯應力戰决之不然不克日日自出行軍慰勞將士|{
	行下孟翻勞力到翻}
親督都水治船艦|{
	都水漢官處處有之前漢屬水衡都尉後漢屬少府其後分屬郡國晉屬大司農治直之翻}
壬子焚淮南岸室屋淮内船舫悉驅民家度水北|{
	秦淮水之北也}
立子偉之為皇太子以始興王濬妃父禇湛之為丹陽尹湛之裕之之兄子也|{
	禇裕之見一百十卷晉安帝義熙六年}
濬為侍中中書監司徒録尚書六條事加南平王鑠開府儀同三司以南兖州刺史建平王宏為江州刺史|{
	欲以代武陵王}
太尉司馬龎秀之自石頭先衆南奔人情由是大震|{
	劭委龎秀之以掌軍隊秀之先奔南軍故人情大震先息薦翻}
以營道侯義綦為湘州刺史檀和之為雍州刺史|{
	欲以代臧質雍於用翻}
癸丑武陵王軍于鵲頭|{
	鵲頭在宣城郡界左傳楚以諸侯伐吳吳敗之于鵲岸唐志宣州南陵縣有鵲頭鎮兵蓋其地在鵲州之頭}
宣城太守王僧達得武陵王檄未知所從客說之曰方今舋逆滔天|{
	說輸芮翻舋許覲翻}
古今未有為君計莫若承義師之檄移告傍郡苟在有心誰不響應|{
	謂凡有人心者皆若響之應聲}
此上策也如其不能可躬帥向義之徒|{
	帥讀曰率}
詳擇水陸之便致身南歸亦其次也僧達乃自候道南奔|{
	候道伺候邉上警急之道也今沿路列置烽臺者即候道}
逢武陵王於鵲頭王即以為長史僧達弘之子也|{
	王弘歷事武文位任隆重}
王初尋陽沈慶之謂人曰王僧達必來赴義人問其故慶之曰吾見其在先帝前議論開張意向明决以此言之其至必也|{
	王氏江南冠族僧達又名公之子也沈慶之於建義之初欲致之以為民望耳}
柳元景以舟艦不堅憚於水戰乃倍道兼行丙辰至江寧步上|{
	江寧縣臨江渚晉咸和之俊以江外無事於南浦置江寧縣宋白曰江寧縣本秣陵之地晉置江寧縣在今縣南七十里故城存焉隋開皇十年移于治城按宋白所謂今縣乃天祐十四年楊氏所置縣也艦戶黯翻上時掌翻}
使薛安都帥鐵騎曜兵於淮上|{
	秦淮之上也}
移書朝士為陳順逆|{
	朝直遥翻為于偽翻觀柳元景用兵方畧固有必勝之理矣}
劭加吳興太守汝南周嶠冠軍將軍随王誕檄亦至嶠素恇怯囬惑不知所從|{
	冠古玩翻恇去王翻}
府司馬丘珍孫殺之舉郡應誕戊午武陵王至南洲降者相屬|{
	南洲屬姑孰降戶江翻下同屬之欲翻}
己未軍于溧州|{
	溧音栗}
王自尋陽有疾不能見將佐唯顔竣出入卧内|{
	在室在舟凡寢卧之所皆謂之卧内將即亮翻}
擁王於膝親視起居疾屢危篤不任咨禀竣皆專决|{
	言病甚不能决事凡内外咨禀竣皆專决任音壬}
軍政之外間以文教書檄應接遐邇|{
	間古莧翻}
昏曉臨哭若出一人|{
	臨力鴆翻}
如是累旬自舟中甲士亦不知王之危疾也|{
	按是月丁未王尋陽己未至溧洲十三日耳丙寅至江寧方二十日今曰累旬當是以至江寧為限耳}
癸亥柳元景潛至新亭依山為壘 |{
	考異曰宋畧云壬戌元景次新林依山為壘按本紀癸亥元景至新亭元景傳元景至新亭經日劭乃水陸出軍今從之}
新降者皆勸元景速進元景曰不然理順難恃同惡相濟輕進無防實啟寇心|{
	兵法所謂先為不可勝以待敵之可勝柳元景以之}
元景營未立劭龍驤將軍詹叔兒覘知之|{
	驤思將翻}
勸劭出戰劭不許甲子劭使蕭斌統步軍禇湛之統水軍與魯秀王羅漢劉簡之精兵合萬人|{
	史言唯魯秀王羅漢劉簡之所部之兵精耳}
攻新亭壘劭自登朱雀門督戰元景宿令軍中曰鼓繁氣易衰叫數力易竭|{
	宿令者先未戰之日而令之也易以豉翻數所角翻}
但衘枚疾戰一聽吾鼓聲劭將士懷劭重賞皆殊死戰元景水陸受敵意氣彌彊麾下勇士悉遣出鬬左右唯留數人宣傳|{
	宣傳號令也}
劭兵勢垂克魯秀擊退鼓劭衆遽止|{
	師之耳目在于旗鼔鼓疾所以進衆鼓徐所以退衆魯秀誤鳴退鼔天使之也}
元景乃開壘鼓譟以乘之劭衆大潰墜淮死者甚多劭更帥餘衆自來攻壘|{
	帥讀曰率}
元景復大破之所殺傷過於前戰士卒爭赴死馬澗澗為之溢|{
	死者塞澗故澗水溢復扶又翻為于偽翻}
劭手斬退者不能禁劉簡之死蕭斌被創|{
	被皮義翻創初良翻}
劭僅以身免走還宫魯秀禇湛之檀和之皆南奔丙寅武陵王至江寧丁卯江夏王義恭單騎南奔|{
	夏戶雅翻騎奇計翻}
劭殺義恭十二子劭濬憂迫無計以輦迎蔣侯神像置宫中稽顙乞恩拜為大司馬封鍾山王|{
	蔣侯蔣子文也廟食鍾山吳孫氏以其祖諱鍾改曰蔣山稽音啟}
拜蘇侯神為驃騎將軍|{
	據齊書崔祖思傳蘇侯神即蘇峻驃匹妙翻騎奇計翻}
以濬為南徐州刺史與南平王鑠並録尚書事戊辰武陵王軍于新亭大將軍義恭上表勸進散騎侍郎徐爰在殿中誑劭云自追義恭遂歸武陵王|{
	因出追義恭遂得歸順散悉亶翻誑居况翻}
時王軍府草創不曉朝章爰素所諳練|{
	諳烏舍翻}
乃以爰兼太常丞撰即位儀注己巳王即皇帝位大赦文武賜爵一等從軍者二等|{
	謂從軍自尋陽至新亭進爵二等以優之}
改諡大行皇帝曰文廟號太祖以大將軍義恭為太尉録尚書六條事南徐州刺史是日劭亦臨軒拜太子偉之大赦唯劉駿義恭義宣誕不在原例|{
	此劭所下赦文所該也}
庚子以南譙王義宣為中書監丞相録尚書六條事揚州刺史随王誕為衛將軍開府儀同三司荆州刺史臧質為車騎將軍開府儀同三司江州刺史|{
	騎奇計翻下同}
沈慶之為領軍將軍蕭思話為尚書左僕射壬申以王僧達為右僕射柳元景為侍中左衛將軍宗慤為右衛將軍張暢為吏部尚書劉延孫顔竣並為侍中五月癸酉朔臧質以雍州兵二萬至新亭|{
	雍於用翻}
豫州刺史劉遵考遣其將夏侯獻之帥步騎五千軍于瓜步|{
	將即亮翻帥讀曰率}
先是世祖遣寧朔將軍顧彬之將兵東入受随王誕節度|{
	孝武帝廟號世祖時初即位而遽以廟號書之蓋因舊史耳先悉薦翻}
誕遣參軍劉季之將兵與彬之俱向建康誕自頓西陵為之後繼|{
	西陵今紹興府蕭山縣西興鎮是也其地西臨浙江吳越王錢鏐以陵非吉語改曰西興將音即亮翻}
劭遣殿中將軍燕欽等拒之相遇於曲阿奔牛塘|{
	今常州武進縣有奔牛鎮及奔牛堰故老相傳云古有金牛奔此因以名之}
欽等大敗劭於是緣淮樹柵以自守又决破崗方山埭以絶東軍|{
	破崗在晉陵郡延陵縣西北亦冇埭埭音代}
時男丁既盡召婦女供役甲戌魯秀等募勇士攻大航克之|{
	大航即朱雀航航戶剛翻 考異曰元凶傳云其月三日按宋畧甲戌乃二日也}
王羅漢聞官軍已度即放仗降緣渚幢隊以次奔散|{
	渚謂秦淮渚也時劭兵緣渚備守以禦義師即秦淮北岸也幢隊幢隊主副所領兵也降戶江翻}
器仗鼓蓋充塞路衢|{
	塞悉則翻}
是夜劭閉守六門|{
	臺城六門大司馬門東華門西華門萬春門太陽門承明門也}
於門内鑿塹立柵城中沸亂|{
	塹七艷翻}
丹楊尹尹弘等文武將吏爭踰城出降|{
	降戶江翻下同}
劭燒輦及衮冕服于宫庭蕭斌宣令所統使皆解甲自石頭戴白幡來降詔斬斌于軍門濬勸劭載寶貨逃入海劭以人情離散不果行乙亥輔國將軍朱修之克東府丙子諸軍克臺城各由諸門入會于殿庭獲王正見斬之張超之走至合殿御床之所為軍士所殺刳腸割心諸將臠其肉生噉之|{
	噉徒覽翻又徒濫翻}
建平等七王號哭俱出|{
	七王建平王宏及東海王禕義陽王昶武昌王渾湘東王或建安王休仁餘一人當是休祐但未封劭蓋拘七王於宫中故號哭俱出號戶高翻}
劭穿西垣入武庫井中隊副高禽執之劭曰天子何在禽曰近在新亭至殿前臧質見之慟哭劭曰天地所不覆載丈人何為見哭|{
	覆敷又翻臧質武敬皇后之姪故劭呼為丈人}
又謂質曰劭可啟得遠徙不|{
	不讀曰否}
質曰主上近在航南|{
	航南謂大航之南}
自當有處分|{
	處昌呂翻分扶問翻}
縛劭於馬上防送軍門時不見傳國璽|{
	璽斯氏翻}
以問劭劭曰在嚴道育處就取得之斬劭及四子於牙下濬帥左右數十人挾南平王鑠南走|{
	帥讀曰率}
遇江夏王義恭於越城濬下馬曰南中郎今何所作義恭曰上已君臨萬國又曰虎頭來得無晚乎義恭曰殊當恨晚又曰故當不死邪義恭曰可詣行闕請罪|{
	天子出行幸所居之所謂之行宫豹尾之内同之禁中旌門之外謂之行闕}
又曰未審能賜一職自效不|{
	不讀曰否史言劭濬狂愚望生}
義恭又曰此未可量|{
	量音良}
勒與俱歸於道斬之及其三子劭濬父子首並梟於大航|{
	梟堅堯翻}
暴尸於市劭妃殷氏及劭濬諸女妾媵皆賜死於獄|{
	媵以證翻}
汙瀦劭所居齋|{
	古者臣弑君子弑父殺無赦壞其室汙其宫而瀦焉鄭玄曰瀦都也南方人謂都為瀦釋停水曰瀦}
殷氏且死謂獄丞江恪曰汝家骨肉相殘何以枉殺無罪人恪曰受拜皇后非罪而何殷氏曰此權時耳當以鸚鵡為后禇湛之之南奔也濬即與禇妃離絶故免於誅|{
	史言禇妃得免死之由}
嚴道育王鸚鵡並都街鞭殺焚尸揚灰於江殷冲尹弘王羅漢及淮南太守沈璞皆伏誅|{
	璞累為濬參佐守于湖不迎義師故誅}
庚辰解嚴辛巳帝如東府百官請罪詔釋之甲申尊帝母路淑媛為皇太后|{
	淑媛魏文帝所制晉武帝承漢魏之制淑妃淑媛淑儀修華修儀修容婕妤容華充華為九嬪位視九卿媛于眷翻}
太后丹楊人也乙酉立妃王氏為皇后后父偃導之玄孫也|{
	王導東晉元臣子孫為江左衣冠甲族}
戊子以柳元景為雍州刺史|{
	雍於用翻}
辛卯追贈袁淑為太尉諡忠憲公徐湛之為司空諡忠烈公江湛為開府儀同三司諡忠簡公王僧綽為金紫光禄大夫諡簡侯|{
	旌其死難也}
壬辰以太尉義恭為揚南徐二州刺史進位太傅領大司馬初劭以尚書令何尚之為司空領尚書令子征北長史偃為侍中父子並居權要及劭敗尚之左右皆散自洗黄閤|{
	舊制三公聽事置黄閤五代志曰三公府三門當中開黄閤設内屏}
殷冲等既誅人為之寒心|{
	為于偽翻}
帝以尚之偃素有令譽且居劭朝用智將迎時有全脱|{
	所謂全脱者活三鎮士民家口朝直遥翻}
故特免之復以尚之為尚書令偃為大司馬長史位遇無改甲午帝謁初寧長寧陵追贈卜天與益州刺史諡壯侯|{
	旌死節也}
與袁淑等四家長給稟禄|{
	卜天與袁淑徐湛之江湛四家稟筆錦翻賜穀也供給也又力錦翻廩食也}
張泓之等各贈郡守|{
	旌其戰死也}
戊戌以南平王鑠為司空建平王宏為尚書左僕射蕭思話為中書令丹陽尹六月丙午帝還宫|{
	還自謁陵也}
初帝之討西陽蠻也|{
	屯五州時}
臧質使柳元景將兵會之及質起兵欲奉南譙王義宣為主潛使元景帥所領西還|{
	帥讀曰率還從宣翻}
元景即以質書呈帝語其信曰|{
	語牛倨翻信使也}
臧冠軍當是未知殿下義舉耳|{
	臧質以冠軍將軍鎮襄陽冠古玩翻}
方應伐逆不容西還質以此恨之及元景為雍州|{
	雍於用翻}
質慮其為荆江後患建議元景當為爪牙不宜遠出帝重違其言戊申以元景為護軍將軍領石頭戍事 己酉以司州刺史魯爽為南豫州刺史庚戌以衛軍司馬徐遺寶為兖州刺史|{
	為魯爽徐遺寶與臧質同反張本}
庚申詔有司論功行賞封顔竣等為公侯|{
	竣七倫翻}
辛未徙南譙王義宣為南郡王随王誕為竟陵王立義宣次子宜陽侯愷為南譙王 閏月壬申以領軍將軍沈慶之為南兖州刺史鎮盱眙|{
	盱眙音吁怡}
癸酉以柳元景為領軍將軍 乙亥魏太皇太后赫連氏殂 丞相義宣固辭内任及子愷王爵甲午更以義宣為荆湘二州刺史|{
	沈約曰晉懷帝分荆州立湘州成帝咸和三年省安帝義熙八年復立十二年又省宋武帝永初三年又立文帝元嘉八年省十七年又立二十九年又省孝武帝孝建元年又立今按是年四月元凶劭以營道侯義綦為湘州刺史蓋以義宣以荆州舉義欲分其軍府耳帝既即位遂以義宣為荆湘二州刺史湘州之立寔在是年也更工衡翻}
愷為宜陽縣王將佐以下並加賞秩|{
	將即亮翻}
以竟陵王誕為揚州刺史秋七月辛酉朔日有食之甲寅詔求直言辛酉詔省

細作并尚方彫文塗飾貴戚競利悉皆禁絶|{
	宋有細作署今大明四年改為左右御府令}
中軍録事參軍周朗上疏以為毒之在體必割其緩處歷下泗間不足戍守|{
	歷下謂歷城泗間謂彭城湖陸}
議者必以為胡衰不足避|{
	當時議者蓋以魏連有内難遂謂之衰}
而不知我之病甚於胡矣|{
	兵甲饋餫之費虚内以給外則吾國之病甚於胡運之衰}
今空守孤城徒費財役使虜但輕騎三千更互出入|{
	騎奇計翻更工衡翻}
春來犯麥秋至侵禾水陸漕輸居然復絶|{
	虜騎至則江南之人不敢至彭泗水陸漕輸絶矣復扶又翻}
於賊不勞而邉已困不至二年卒散民盡可蹻足而待也|{
	蹻巨驕翻}
今人知不以羊追狼蟹捕鼠而令重車弱卒與肥馬悍胡相逐其不能濟固宜矣|{
	言不濟事也悍下罕翻又侯旰翻}
又三年之喪天下之達喪漢氏節其臣則可矣薄其子則亂也|{
	短喪自漢景帝始詳見十五卷漢文帝後七年}
凡法有變於古而刻於情則莫能順焉至乎敗於禮而安於身必遽而奉之今陛下以大孝始基宜反斯謬|{
	言帝既能討元凶劭之罪當行三年之喪以反短喪之謬}
又舉天下以奉一君何患不給一體炫金不及百兩|{
	炫胡練翻炫金今之銷金是也}
一歲美衣不過數襲而必收寶連櫝集服累笥目豈常視身未時親是櫝帶寶笥著衣也|{
	著陟畧翻}
何糜蠧之劇惑鄙之甚邪且細作始并以為儉節而市造華怪即傳於民如此則遷也非罷也|{
	此等語切中當時之病凡欲言時政若此可也否則迎合以徼利禄耳}
凡厥庶民制度日侈見車馬不辨貴賤視冠服不知尊卑尚方今造一物小民明已䁹睨|{
	明謂來旦也䁹與睥同匹詣翻}
宫中朝製一衣庶家晩已裁學侈麗之源實先宫閫|{
	嗚呼我宋之將亡其習俗亦如此吾是以悲二宋之一轍也嗚呼先悉薦翻}
又設官者宜官稱事立人稱官置|{
	稱尺證翻}
王侯識未堪務不應強仕|{
	此強仕謂強之使仕也強其兩翻}
且帝子未官人誰謂賤但宜詳置賓友茂擇正人亦何必列長史參軍别駕從事然後為貴哉|{
	此言亦深切宋藩王出鎮之弊}
又俗好以毁沈人不察其所以致毁|{
	好呼到翻沈持林翻沈言没人之實也}
以譽進人不察其所以致譽|{
	譽音余下同}
毁徒皆鄙則宜擢其毁者譽黨悉庸則宜退其譽者如此則毁譽不妄善惡分矣|{
	論語子貢問孔子曰鄉人皆好之何如子曰未可也鄉人皆惡之何如子曰未可也不如鄉人之善者好之其不善者惡之周朗之言正得此意蓋晉宋以來諸州中正品定人物高下其手毀譽之失實也久矣}
凡無世不有言事無時不有下令然升平不至昏危相繼何哉設令之本非實故也|{
	朗指帝求言非實}
書奏忤旨自解去職朗嶠之弟也|{
	周嶠為丘珍孫所殺事見上忤五故翻}
侍中謝莊上言詔云貴戚競利悉皆禁絶此實允愜民聽若有犯違則應依制裁糾若廢法申恩便為明詔既下而聲實乖爽也|{
	爽差也}
臣愚謂大臣在禄位者尤不宜與民爭利不審可得在此詔不|{
	不讀曰否}
莊弘微之子也|{
	謝弘微進用於元嘉之初}
上多變易太祖之制郡縣以三周為滿宋之善政於是乎衰|{
	元嘉之制守宰以六朞為斷然自時厥後率以三周為滿而又有數更數易不及三周者}
乙丑魏濮陽王閭若文征西大將軍永昌王仁皆坐謀叛仁賜死於長安若文伏誅 南平穆王鑠素負才能意常輕上又為太子劭所任出降最晩|{
	鑠為始興王濬所挾而走遇江夏王義恭乃降非本心也降戶江翻}
上潛使人毒之己巳鑠卒贈司徒以商臣之諡諡之|{
	楚世子商臣弑君父而自立辛後諡曰穆}
南海太守蕭簡據廣州反簡斌之弟也|{
	蕭斌以逆黨誅其弟懼連坐而反}
詔新南海太守南昌鄧琬 |{
	考異曰蕭簡傳作劉玩今從本紀}
始興太守沈法系討之法系慶之之從弟也|{
	從才用翻}
簡誑其衆曰臺軍是賊劭所遣衆信之為之固守|{
	誑居况翻為于偽翻}
琬先至止為一攻道法系至曰宜四面並攻若守一道何時可拔琬不從法系曰更相申五十日|{
	申容也又緩為之期曰申}
日盡又不克乃從之八道俱攻一日即破之九月丁卯斬簡廣州平法系封府庫付琬而還|{
	史言沈氏兄弟皆能宣力于一時還從宣翻又如字}
冬十一月丙午以左軍將軍魯秀為司州刺史|{
	為魯秀從臧質等稱兵張本}
辛酉魏主如信都中山 十二月癸未以將置東宫省太子率更令等官中庶子等各减舊員之半|{
	懲元凶劭之禍也晉制東宫中庶子四人中舍人四人庶子四人舍人十六人洗馬八人更工衡翻}
甲午魏主還平城

資治通鑑卷一百二十七
















































































































































