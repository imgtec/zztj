<!DOCTYPE html PUBLIC "-//W3C//DTD XHTML 1.0 Transitional//EN" "http://www.w3.org/TR/xhtml1/DTD/xhtml1-transitional.dtd">
<html xmlns="http://www.w3.org/1999/xhtml">
<head>
<meta http-equiv="Content-Type" content="text/html; charset=utf-8" />
<meta http-equiv="X-UA-Compatible" content="IE=Edge,chrome=1">
<title>資治通鑒_98-資治通鑑卷九十七_98-資治通鑑卷九十七</title>
<meta name="Keywords" content="資治通鑒_98-資治通鑑卷九十七_98-資治通鑑卷九十七">
<meta name="Description" content="資治通鑒_98-資治通鑑卷九十七_98-資治通鑑卷九十七">
<meta http-equiv="Cache-Control" content="no-transform" />
<meta http-equiv="Cache-Control" content="no-siteapp" />
<link href="/img/style.css" rel="stylesheet" type="text/css" />
<script src="/img/m.js?2020"></script> 
</head>
<body>
 <div class="ClassNavi">
<a  href="/24shi/">二十四史</a> | <a href="/SiKuQuanShu/">四库全书</a> | <a href="http://www.guoxuedashi.com/gjtsjc/"><font  color="#FF0000">古今图书集成</font></a> | <a href="/renwu/">历史人物</a> | <a href="/ShuoWenJieZi/"><font  color="#FF0000">说文解字</a></font> | <a href="/chengyu/">成语词典</a> | <a  target="_blank"  href="http://www.guoxuedashi.com/jgwhj/"><font  color="#FF0000">甲骨文合集</font></a> | <a href="/yzjwjc/"><font  color="#FF0000">殷周金文集成</font></a> | <a href="/xiangxingzi/"><font color="#0000FF">象形字典</font></a> | <a href="/13jing/"><font  color="#FF0000">十三经索引</font></a> | <a href="/zixing/"><font  color="#FF0000">字体转换器</font></a> | <a href="/zidian/xz/"><font color="#0000FF">篆书识别</font></a> | <a href="/jinfanyi/">近义反义词</a> | <a href="/duilian/">对联大全</a> | <a href="/jiapu/"><font  color="#0000FF">家谱族谱查询</font></a> | <a href="http://www.guoxuemi.com/hafo/" target="_blank" ><font color="#FF0000">哈佛古籍</font></a> 
</div>

 <!-- 头部导航开始 -->
<div class="w1180 head clearfix">
  <div class="head_logo l"><a title="国学大师官网" href="http://www.guoxuedashi.com" target="_blank"></a></div>
  <div class="head_sr l">
  <div id="head1">
  
  <a href="http://www.guoxuedashi.com/zidian/bujian/" target="_blank" ><img src="http://www.guoxuedashi.com/img/top1.gif" width="88" height="60" border="0" title="部件查字,支持20万汉字"></a>


<a href="http://www.guoxuedashi.com/help/yingpan.php" target="_blank"><img src="http://www.guoxuedashi.com/img/top230.gif" width="600" height="62" border="0" ></a>


  </div>
  <div id="head3"><a href="javascript:" onClick="javascript:window.external.AddFavorite(window.location.href,document.title);">添加收藏</a>
  <br><a href="/help/setie.php">搜索引擎</a>
  <br><a href="/help/zanzhu.php">赞助本站</a></div>
  <div id="head2">
 <a href="http://www.guoxuemi.com/" target="_blank"><img src="http://www.guoxuedashi.com/img/guoxuemi.gif" width="95" height="62" border="0" style="margin-left:2px;" title="国学迷"></a>
  

  </div>
</div>
  <div class="clear"></div>
  <div class="head_nav">
  <p><a href="/">首页</a> | <a href="/ShuKu/">国学书库</a> | <a href="/guji/">影印古籍</a> | <a href="/shici/">诗词宝典</a> | <a   href="/SiKuQuanShu/gxjx.php">精选</a> <b>|</b> <a href="/zidian/">汉语字典</a> | <a href="/hydcd/">汉语词典</a> | <a href="http://www.guoxuedashi.com/zidian/bujian/"><font  color="#CC0066">部件查字</font></a> | <a href="http://www.sfds.cn/"><font  color="#CC0066">书法大师</font></a> | <a href="/jgwhj/">甲骨文</a> <b>|</b> <a href="/b/4/"><font  color="#CC0066">解密</font></a> | <a href="/renwu/">历史人物</a> | <a href="/diangu/">历史典故</a> | <a href="/xingshi/">姓氏</a> | <a href="/minzu/">民族</a> <b>|</b> <a href="/mz/"><font  color="#CC0066">世界名著</font></a> | <a href="/download/">软件下载</a>
</p>
<p><a href="/b/"><font  color="#CC0066">历史</font></a> | <a href="http://skqs.guoxuedashi.com/" target="_blank">四库全书</a> |  <a href="http://www.guoxuedashi.com/search/" target="_blank"><font  color="#CC0066">全文检索</font></a> | <a href="http://www.guoxuedashi.com/shumu/">古籍书目</a> | <a   href="/24shi/">正史</a> <b>|</b> <a href="/chengyu/">成语词典</a> | <a href="/kangxi/" title="康熙字典">康熙字典</a> | <a href="/ShuoWenJieZi/">说文解字</a> | <a href="/zixing/yanbian/">字形演变</a> | <a href="/yzjwjc/">金 文</a> <b>|</b>  <a href="/shijian/nian-hao/">年号</a> | <a href="/diming/">历史地名</a> | <a href="/shijian/">历史事件</a> | <a href="/guanzhi/">官职</a> | <a href="/lishi/">知识</a> <b>|</b> <a href="/zhongyi/">中医中药</a> | <a href="http://www.guoxuedashi.com/forum/">留言反馈</a>
</p>
  </div>
</div>
<!-- 头部导航END --> 
<!-- 内容区开始 --> 
<div class="w1180 clearfix">
  <div class="info l">
   
<div class="clearfix" style="background:#f5faff;">
<script src='http://www.guoxuedashi.com/img/headersou.js'></script>

</div>
  <div class="info_tree"><a href="http://www.guoxuedashi.com">首页</a> > <a href="/SiKuQuanShu/fanti/">四库全书</a>
 > <h1>资治通鉴</h1> <!--         下载:【右键另存为】即可 --></div>
  <div class="info_content zj clearfix">
  
<div class="info_txt clearfix" id="show">
<center style="font-size:24px;">98-資治通鑑卷九十七</center>
    資治通鑑卷九十七   宋 司馬光 撰<br />
<br />
  胡三省 音注<br />
<br />
  晉紀十九【起玄黓攝提格盡彊圉協洽凡六年】<br />
<br />
  顯宗成皇帝下<br />
<br />
  咸康八年春正月己未朔日有食之 【考異曰天文志作乙未今從帝紀及長歷】 乙丑大赦 豫州刺史庾懌以酒餉江州刺史王允之允之覺其毒飲犬【飲於禁翻】犬斃密奏之帝曰大舅已亂天下【謂庾亮也】小舅復欲爾邪【復扶又翻】二月懌飲鴆而卒【卒子恤翻】 三月初以武悼后配食武帝廟【楊皇后惠帝永康元年幽廢而死今乃得配食武帝】 庾翼在武昌數有妖怪【數所角翻妖於驕翻】欲移鎮樂鄉征虜長史王述與庾氷牋曰樂鄉去武昌千有餘里數萬之衆一旦移徙興立城壁公私勞擾又江州當泝流數千里供給軍府力役增倍且武昌實江東鎮戍之中非但扞禦上流而已緩急赴告駿奔不難【書武成曰駿奔走駿音峻注云駿大也言皆奔走也】若移樂鄉遠在西陲一朝江渚有虞不相接救方岳重將【將即亮翻】固當居要害之地為内外形埶使闚?之心不知所向昔秦忌亡胡之讖卒為劉項之資【秦盧生奏錄圖書曰亡秦者胡也于是始皇使蒙恬北伐胡不知立子胡亥以兆亂卒子恤翻】周惡檿弧之謡而成褒姒之亂【國語曰宣王之時有童謡曰檿弧箕服實亡周國宣王聞之有夫婦鬻是器者使執而戮之府之小妾生子而非王子也懼而弃之此人也收以奔褒褒人有獄而以女入于幽王王嬖是女而生伯服是為褒姒欲廢太子宜臼而立伯服卒以成申侯西戎之亂惡烏路翻檿於琰翻】是以逹人君子直道而行禳避之道皆所不取正當擇人事之勝理思社稷之長計耳朝議亦以為然【朝直遥翻】翼乃止 夏五月乙卯帝不豫【豫順也不豫言有疾而氣體不能順適也】六月庚寅疾篤或詐為尚書符敕宮門無得内宰相衆皆失色庾氷曰此必詐也推問果然【推考也究也】帝二子丕奕皆在襁褓【襁居兩翻褓音保】庾氷自以兄弟秉權日久恐易世之後親屬愈疎為它人所閒【閒古莧翻】每說帝以國有彊敵【強敵謂漢趙也說輸芮翻】宜立長君【長知兩翻】請以母弟琅邪王岳為嗣帝許之中書令何充曰父子相傳先王舊典易之者鮮不致亂【鮮息淺翻】故武王不授聖弟【聖弟謂周公】非不愛也今琅邪踐阼將如孺子何氷不聽下詔以岳為嗣并以奕繼琅邪哀王【元帝以子裒奉琅邪恭王後薨諡曰孝子哀王安國立未踰年薨元帝復以皇子煥嗣封其日薨復以皇子昱為琅邪王咸和之初昱徙封會稽以岳為琅邪王今岳入繼大宗故以奕繼哀王後】壬辰氷充及武陵王晞會稽王昱尚書令諸葛恢並受顧命【會工外翻】癸巳帝崩【年二十二】帝幼冲嗣位不親庶政及長頗有勤儉之德【長知兩翻】 甲午琅邪王即皇帝位大赦己亥封成帝子丕為琅邪王奕為東海王 康帝亮隂不言委政於庾氷何充秋七月丙辰葬成帝于興平陵帝徒行送喪至閶闔門乃升素輿至陵所既葬帝臨軒庾氷何充侍坐【坐徂臥翻】帝曰朕嗣鴻業二君之力也充曰陛下龍飛臣氷之力也若如臣議不覩升平之世帝有慙色己未以充為驃騎將軍【驃匹妙翻】都督徐州揚州之晉陵諸軍事領徐州刺史鎮京口【晉永嘉大亂徐州淮北流民相率過淮亦有過江居晉陵郡界者咸和四年司徒郗鑒又徙流民之在淮南者於晉陵諸縣其徙過江南及留在江北者並立僑郡以司牧之徐州實郡在江北者實有廣陵堂邑鍾離三郡而揚州之境以晉陵郡屬徐州所謂都督徐州揚州之晉陵諸軍事者此也晉陵郡吳之毘陵郡也吳分吳郡無錫以西為毘陵郡晉東海王越世子名毘而東海國故食毘陵永嘉五年改為晉陵】避諸庾也 冬十月燕王皝遷都龍城【慕容廆先居徒河之青山後徙棘城今自棘城徙都龍城杜佑曰營州柳城郡古孤竹國也春秋為山戎肥子二國地漢徒河之青山在郡城東百九十里棘城即顓頊之虛在郡城東南百七十里慕容皝以柳城之北龍山之南福德之地遂遷都龍城號新宮為和龍宮柳城縣有白狼山白狼水又有僕扶犂縣故城在東南其龍山即慕容皝祭龍所也有饒樂水漢徒河縣城】赦其境内建威將軍翰言於皝曰宇文彊盛日久屢為國患今逸豆歸簒竊得國【逸豆歸逐乙得歸見九十五卷咸和八年】羣情不附加之性識庸闇將帥非才【將即亮翻帥所類翻】國無防衛軍無部伍臣久在其國悉其地形雖遠附彊羯【彊羯謂趙也羯居謁翻】聲埶不接無益救援今若擊之百舉百克然高句麗去國密邇常有闚?之志【句如字又音駒麗力知翻闚缺規翻門中視也?從門旁竇中視也音俞韻釋闚?私視也】彼知宇文既亡禍將及已必乘虚深入掩吾不備若少留兵則不足以守多留兵則不足以行此心腹之患也宜先除之觀其埶力一舉可克宇文自守之虜必不能遠來争利既取高句麗還取宇文如返手耳【返當作反下同】二國既平利盡東海國富兵彊無返顧之憂然後中原可圖也皝曰善將擊高句麗高句麗有二道其北道平闊南道險狹【北道從北置而進南道從南陜入木底城】衆欲從北道翰曰虜以常情料之必謂大軍從北道當重北而輕南王宜帥銳兵從南道擊之出其不意丸都不足取也【高句麗王居丸都帥讀曰率下同】别遣偏師從北道縱有蹉跌【蹉倉何翻跌徒結翻蹉跌失足而踣也】其腹心已潰四支無能為也皝從之十一月皝自將勁兵四萬出南道【將即亮翻下同】以慕容翰慕容霸為前鋒别遣長史王㝢等將兵萬五千出北道以伐高句麗高句麗王釗果遣弟武帥精兵五萬拒北道自帥羸兵以備南道【羸倫為翻】慕容翰等先至與釗合戰皝以大衆繼之左常侍鮮于亮曰臣以俘虜蒙王國士之恩【事見上卷咸康四年】不可以不報今日臣死日也獨與數騎先犯高句麗陳所嚮摧陷高句麗陳動【騎奇寄翻下同陳讀曰陣】大衆因而乘之高句麗兵大敗左長史韓壽斬高句麗將阿佛和度加【高句麗置官有相加大加小加】諸軍乘勝追之遂入丸都釗單騎走輕車將軍慕輿埿追獲其母周氏及妻而還會王㝢等戰於北道皆敗没由是皝不復窮追【復扶又翻下同】遣使招釗釗不出皝將還韓壽曰高句麗之地不可戍守今其主亡民散潜伏山谷大軍既去必復鳩聚【鳩亦聚也】收其餘燼【火餘曰燼猶能復然】猶足為患請載其父尸囚其生母而歸俟其束身自歸然後返之撫以恩信策之上也皝從之發釗父乙弗利墓載其尸收其府庫累世之寶虜男女五萬餘口燒其宮室毁丸都城而還【還從宣翻又如字】 十二月壬子立妃禇氏為皇后徵豫章太守褚裒為侍中尚書裒自以后父不願居中任事【裒薄侯翻】苦求外出乃除建威將軍江州刺史鎮半洲趙王虎作臺觀四十餘所於鄴【觀古玩翻】又營洛陽長安<br />
<br />
  二宮作者四十餘萬人又欲自鄴起閣道至襄國敕河南四州治南伐之備【河南四州洛豫徐兖也治直之翻】并朔秦雍嚴西討之資【晉地理志曰石勒平朔方置朔州西討欲攻河西也雍於用翻】青冀幽州為東征之計【東征欲伐燕也】皆三五發卒【三丁發二五丁發三也】諸州軍造甲者五十餘萬人船夫十七萬人為水所没虎狼所食者三分居一加之公侯牧宰競營私利百姓失業愁困貝丘人李弘【貝丘縣自漢以來屬清河郡北齊併入清河縣】因衆心之怨自言姓名應䜟連結黨與署置百寮事發誅之連坐者數千家虎畋獵無度晨出夜歸又多微行躬察作役侍中京兆韋謏諫曰【謏蘇了翻】陛下忽天下之重輕行斤斧之間猝有狂夫之變雖有智勇將安所施又興役無時廢民耘獲【穫戶郭翻】吁嗟盈路殆非仁聖之所忍為也虎賜謏穀帛而興繕滋繁游察自若秦公韜有寵於虎太子宣惡之【惡烏路翻】右僕射張離領五兵尚書【曹魏置五兵尚書沈約志五兵尚書領中兵外兵騎兵别兵都兵故謂之五兵】欲求媚於宣說之曰【說輸芮翻】今諸侯吏兵過限宜漸裁省以壯本根宣使離為奏秦燕義陽樂平四公【秦公韜燕公斌義陽公鑒樂平公苞】聽置吏一百九十七人帳下兵二百人自是以下三分置一餘兵五萬悉配東宮【配隸也】於是諸公咸怨嫌釁益深矣青州上言濟南平陵城北石虎一夕移於城東南【漢濟南郡有東平陵縣晉省後復置為平陵縣唐為齊州全節縣濟子禮翻】有狼狐千餘迹隨之迹皆成蹊虎喜曰石虎者朕也自西北徙而東南者天意欲使朕平蕩江南也其敕諸州兵明年悉集朕當親董六師以奉天命羣臣皆賀上皇德頌者一百七人【上時掌翻】制征士五人出車一乘牛二頭米十五斛絹十匹調不辦者斬【乘繩證翻調徒釣翻】民至鬻子以供軍須【行軍所須以為用故曰軍須】猶不能給自經於道樹者相望【人之自經必於溝瀆隱蔽之地死亡計迫自經于道旁之樹蓋甚不獲已也相望言其多也按目錄書是年代王還雲中】<br />
<br />
  康皇帝【諱岳字世同成帝母弟也咸和元年封吳王二年徙封琅邪王諡法温柔好樂曰康】<br />
<br />
  建元元年春二月高句麗王釗遣其弟稱臣入朝於燕【朝直遥翻】貢珍異以千數燕王皝乃還其父尸猶留其母為質【質音致】 宇文逸豆歸遣其相莫淺渾將兵擊燕諸將争欲擊之【相息亮翻將即亮翻】燕王皝不許莫淺渾以為皝畏之酣飲縱獵不復設備【酣戶甘翻復扶又翻】皝使慕容翰出擊之莫淺渾大敗僅以身免盡俘其衆 庾翼為人忼慨【忼慨同音口黨翻】喜功名【喜許記翻】琅邪内史桓温彛之子也【桓彛死于蘇峻之難】尚南康公主【公主明帝女】豪爽有風槩【言其有風力氣槩】翼與之友善相期以寧濟海内翼嘗薦温於成帝曰桓温有英雄之才願陛下勿以常人遇之常壻畜之【畜呼玉翻又許竹翻】宜委以方召之任【方叔邵虎周宣王用之以中興】必有弘濟艱難之勲時杜乂殷浩並才名冠世【冠古玩翻】翼獨弗之重也曰此輩宜束之高閣俟天下太平然後徐議其任耳浩累辭徵辟屛居墓所【屏必郢翻】幾將十年【幾居希翻】時人擬之管葛【管仲諸葛孔明也】江夏相謝尚長山令王濛【漢獻帝初平二年分烏傷立長山縣屬會稽郡吳分屬東陽郡隋改長山為金華縣今屬婺州】常伺其出處【伺相吏翻處昌呂翻下同】以卜江左興亡嘗相與省之【省悉井翻】知浩有確然之志【確然者守志堅固不移也】既返相謂曰深源不起當如蒼生何【殷浩字深源】尚鯤之子也翼請浩為司馬詔除侍中安西軍司【軍司即軍司馬】浩不應翼遺浩書曰王夷甫立名非真雖云談道實長華競【遺于季翻長知兩翻】明德君子遇會處際【言遇風雲之會處功名之際也】寧可然乎浩猶不起殷羨為長沙相【相息亮翻】在郡貪殘庾氷與翼書屬之【屬之欲翻】翼報曰殷君驕豪亦似由有佳兒【佳兒謂浩也】弟故小令物情容之【翼氷弟也】大較江東之政以嫗喣豪彊常為民蠧【嫗於具翻喣許具翻鄭玄曰體曰嫗氣曰喣】時有行法輒施之寒劣【寒者衰冷無氣燄也劣者卑弱在人下也】如往年偷石頭倉米一百萬斛皆是豪將輩而直殺倉督監以塞責【倉督監筦倉之官將即亮翻塞悉則翻】山遐為餘姚長為官出豪彊所藏二千戶【餘姚縣屬會稽郡長知兩翻為于偽翻】而衆共驅之令遐不得安席雖皆前宰之惛謬【前宰指王導惛音昏庾翼察舉小才耳當江東草創之時非王導之弘致遠識不能濟也謂之惛謬談何容易】江東事去實此之由兄弟不幸横陷此中【横戶孟翻】自不能拔足於風塵之外當共明目而治之【治直之翻】荆州所統二十餘郡【太康地志荆州統郡二十有二惠帝至元帝又立隨新野竟陵新興南河等郡】唯長沙最惡惡而不黜與殺督監復何異邪【復扶又翻】遐簡之子也【永嘉中山簡鎮襄陽】翼以滅胡取蜀為己任遣使東約燕王皝西約張駿刻期大舉朝議多以為難【使疏吏翻朝直遥翻】唯庾氷意與之同而桓温譙王無忌皆贊成之無忌承之子也【譙王承死於王敦之難承當作氶音拯】秋七月趙汝南太守戴開帥數千人詣翼降【帥讀曰率降戶江翻】丁巳下詔議經畧中原翼欲悉所部之衆北伐表桓宣為都督司雍梁三州荆州之四郡諸軍事梁州刺史【荆州四郡南陽新野襄陽南鄉也雍於用翻】前趣丹水【丹水縣前漢屬弘農郡後漢屬南陽郡晉屬順陽郡賢曰丹水故城在今鄧州内鄉縣西南臨丹水趣七喻翻】桓温為前鋒小督假節帥衆入臨淮 【考異曰帝紀温入臨淮下云庾翼為征討大都督遷鎮襄陽按翼傳翼先表移鎮安陸至夏口上表云九月十九日發武昌二十四日達夏口始請徙鎮襄陽始詔加都督征討諸軍事故知不在此月】並發所統六州奴及車牛驢馬百姓嗟怨【六州江荆司雍梁益也】 代王什翼犍復求婚於燕【犍居言翻復扶又翻】燕王皝使納馬千匹為禮什翼犍不與又倨慢無子壻禮八月皝遣世子儁帥前軍師評等擊代【帥讀曰率下同 考異曰後魏序紀八月慕容元真遣使請薦女無用兵事今從燕書】 什翼犍帥衆避去燕人無所見而還【還從宣翻又如字】 漢主壽卒【年四十四】諡曰昭文廟號中宗太子勢即位大赦【勢字子仁壽之長子也】 趙太子宣擊鮮卑斛穀提大破之斬首三萬級 宇文逸豆歸執段遼弟蘭送於趙【翟遼之敗其弟蘭奔宇文部逸豆歸今執以送趙】并獻駿馬萬匹趙王虎命蘭帥所從鮮卑五千人屯令支【令音鈴又郎定翻支音祁】 庾翼欲移鎮襄陽恐朝廷不許乃奏云移鎮安陸【安陸縣自漢以來屬江夏郡唐為安州治所】帝及朝士皆遣使譬止翼翼遂違詔北行至夏口復上表請鎮襄陽【朝直遥翻使疏吏翻夏戶雅翻復扶又翻上時掌翻】翼時有衆四萬詔加翼都督征討諸軍事先是車騎將軍揚州刺史庾氷屢求出外【先悉薦翻】辛巳以氷都督荆江寧益梁交廣七州豫州之四郡諸軍事【豫州四郡宣城歷陽廬江安豐也】領江州刺史假節鎮武昌以為翼繼援徵徐州刺史何充為都督揚豫徐州之琅邪諸軍事【永嘉之亂琅邪國人隨元帝過江者千餘戶太興三年立懷德縣丹陽雖有琅邪相而無其地是年桓温為内史鎮江乘之蒲洲金城上求割丹陽之江乘縣境立郡所謂徐州之琅邪此也】領揚州刺史錄尚書事輔政以琅邪内史桓温為都督青徐兖三州諸軍事徐州刺史褚裒為衛將軍領中書令 冬十一月己巳大赦<br />
<br />
  二年春正月趙王虎享羣臣於太武殿有白雁百餘集馬道之南【馬道者築道可以馳馬往來】虎命射之皆不獲【射而亦翻】時諸州兵集者百餘萬太史令趙攬密言於虎曰白雁集庭宮室將空之象不宜南行虎信之乃臨宣武觀大閲而罷【石虎倣洛都之制築宣武觀於鄴觀古玩翻】 漢主勢改元太和尊母閻氏為皇太后立妻李氏為皇后 燕王皝與左司馬高詡謀伐宇文逸豆歸詡曰宇文彊盛今不取必為國患伐之必克然不利於將【將即亮翻】出而告人曰吾往必不返然忠臣不避也於是皝自將伐逸豆歸【將即亮翻下同】以慕容翰為前鋒將軍劉佩副之分命慕容軍慕容恪慕容霸及折衝將軍慕輿根將兵三道並進高詡將發不見其妻使人語以家事而行【語牛倨翻】逸豆歸遣南羅大涉夜干將精兵逆戰【南羅城名大城大也慕容既克宇文改南羅城為威德城考異曰慕容皝載記作涉弈干今從燕書】皝遣人馳謂慕容翰曰涉夜干勇冠三軍【冠古玩翻】宜小避之翰曰逸豆歸掃其國内精兵以屬涉夜干【屬之欲翻】涉夜干素有勇名一國所賴也今我克之其國不攻自潰矣且吾孰知涉夜干之為人【孰與熟同】雖有虚名實易與耳不宜避之以挫吾兵氣遂進戰翰自出衝陳【易以䜴翻陳讀曰陣】涉夜干出應之慕容霸從傍邀擊遂斬涉夜干宇文士卒見涉夜干死不戰而潰燕軍乘勝逐之遂克其都城【宇文國都遼西紫蒙川】逸豆歸走死漠北宇文氏由是散亡皝悉收其畜產資貨徙其部衆五千餘落於昌黎闢地千餘里更命涉夜干所居城曰威德城使弟彪戍之而還高詡劉佩皆中流矢卒【還音旋中竹仲翻卒子恤翻】詡善天文皝嘗謂曰卿有佳書而不見與何以為忠藎詡曰臣聞人君執要人臣執職執要者逸執職者勞是以后稷播種堯不預焉占候天文晨夜甚苦非至尊之所宜親殿下將焉用之【焉於䖍翻】皝默然初逸豆歸事趙甚謹貢獻屬路【屬之欲翻】及燕人伐逸豆歸趙王虎使右將軍白勝并州刺史王霸自甘松出救之【甘松在濡源之東突門嶺之西】比至【比必寐翻】宇文氏已亡因攻威德城不克而還慕容彪追擊破之慕容翰之與宇文氏戰也為流矢所中臥病積時不出後漸差【差楚懈翻疾瘳也】於其家試騁馬或告翰稱病而私習騎乘疑欲為變燕王皝雖藉翰勇畧然中心終忌之乃賜翰死翰曰吾負罪出奔既而復還【翰出奔見九十五卷成帝咸和八年還見上卷咸康六年復扶又翻】今日死已晚矣然羯賊跨據中原吾不自量【量音良】欲為國家蕩壹區夏【為于偽翻夏戶雅翻】此志不遂没有遺恨命矣夫飲藥而卒 【考異曰三十國春秋云永和二年九月殺翰燕書翰傳翰嘗臨陳為流矢所中病臥歲時不出入後漸差試馬按自討宇文後翰未嘗預攻戰自建元二年正月至永和二年九月已踰年矣三十國春秋恐誤今從載記翰傳】 代王什翼犍遣其大人長孫秩迎婦於燕【拓跋鄰之統國也以次兄為拔拔氏厥後孝文帝用夏變夷改為長孫氏史以華言書其後所改姓】 夏四月凉州將張瓘敗趙將王擢于三交城【三交城在朔方之西宋白曰三交土堠在綏州東北七十五里將即亮翻敗補邁翻】 初趙領軍王朗言於趙王虎曰盛冬雪寒而皇太子使人伐宮材引於漳水役者數萬吁嗟滿道陛下宜因出游罷之虎從之太子宣怒會熒惑守房【天文志房四星為明堂天子布政之宮也亦四輔也下第一星上將也次次將也次次相也上星上相也熒惑守房心王者惡之熒惑天子理也故曰雖有明天子必謹視熒惑所在】宣使太史令趙攬言於虎曰房為天王今熒惑守之其殃不細宜以貴臣王姓者當之虎曰誰可者攬曰無貴於王領軍虎意惜朗使攬更言其次攬無以對因曰其次唯中書監王波耳虎乃下詔追罪波前議楛矢事【見上卷成帝咸康六年】腰斬之及其四子投尸漳水既而愍其無罪追贈司空封其孫為侯 趙平北將軍尹農攻燕凡城不克而還 漢太史令韓皓上言熒惑守心乃宗廟不修之譴【以七曜所經周天三百六十五度四分度之一考之房六度太心三度太五星入之久而不去謂之守時趙太史以為熒惑守房漢太史以為熒惑守心是則躔度之難知也】漢主勢命羣臣議之相國董皎侍中王嘏以為景武創業獻文承基至親不遠無宜疎絶乃更命祀成始祖太宗皆謂之漢【李特諡景武皇帝廟號始祖雄諡武皇帝廟號太宗驤諡獻皇帝壽諡文皇帝特驤兄弟也雄壽從兄弟也故曰至親不遠李壽改立宗廟見上卷成帝咸康四年】 征西將軍庾翼使梁州刺史桓宣擊趙將李羆於丹水為羆所敗【敗補邁翻】翼貶宣為建威將軍宣慙憤成疾秋八月庚辰卒翼以長子方之為義城太守【沈約曰義成郡晉孝武立治襄陽五代志曰襄陽郡穀城縣舊曰義城置義城郡又按晉書桓宣傳陶侃使宣鎮襄陽以其淮南部曲立義成郡則此郡立於咸和中明矣城當作成】代領宣衆又以司馬應誕為襄陽太守參軍司馬勲為梁州刺史戍西城【西城縣時屬魏興郡】 中書令褚裒固辭樞要閏月丁巳以裒為左將軍都督兖州徐州之琅邪諸軍事兖州刺史鎮金城【金城在江乘之蒲洲琅邪僑郡亦以為治所】 帝疾篤庾氷庾翼欲立會稽王昱為嗣【會工外翻】中書監何充建議立皇子聃【聃他含翻】帝從之九月丙申立聃為皇太子戊戍帝崩于式乾殿【年二十三建康宮殿皆用洛都舊名】己亥何充以遺旨奉太子即位大赦由是氷翼深恨充尊皇后褚氏為皇太后時穆帝方二歲太后臨朝稱制【朝直遥翻下同】何充加中書監錄尚書事充自陳既録尚書不宜復監中書【中書監之監古陷翻監中書之監古銜翻復扶又翻下同】許之復加侍中充以左將軍褚裒太后之父宜綜朝政上疏薦裒參錄尚書乃以裒為侍中衛將軍錄尚書事持節督刺史如故【裒蒲侯翻】裒以近戚懼獲譏嫌上疏固請居藩改授都督徐兖青三州揚州之二郡諸軍事衛將軍徐兖二州刺史鎮京口【揚州之二郡晉陵義興也】尚書奏裒見太后在公庭則如臣禮私覿則嚴父【朱熹曰私覿以私禮見也嚴尊也】從之 冬十月乙丑葬康帝于崇平陵 江州刺史庾氷有疾太后徵氷輔政氷辭十一月庚辰卒庾翼以家國情事【言以兄弟之情則當赴氷之喪以國事則當治兵以國收復】留子方之為建武將軍戍襄陽方之年少【少詩照翻】以參軍毛穆之為建武司馬以輔之穆之寶之子也【毛寶豫有平蘇峻之功邾城之陷寶死焉】翼還鎮夏口【夏戶雅翻】詔翼復督江州又領豫州刺史翼辭豫州復欲移鎮樂鄉詔不許翼仍繕修軍器大佃積穀以圖後舉【佃亭年翻】 趙王虎作河橋於靈昌津采石為中濟【滑臺故鄭之廩延也城下有延津又西為靈昌津石勒攻劉曜途出於此以河氷泮為神靈之助號是處為靈昌津大河深廣必下石為中濟兩岸繫巨絙以維船然後可以立橋如河陽橋蒲津橋之中潬是也采石採取石也濟如宇】石下輒隨流【河流漂急故石下輒隨流而去】用功五千餘萬而橋不成虎怒斬匠而罷<br />
<br />
  孝宗穆皇帝上之上【諱聃宇彭子康帝子也諡法中情見貌曰穆】<br />
<br />
  永和元年春正月甲戌朔皇太后設白紗帷於太極殿抱帝臨軒 趙義陽公鑒鎮關中役煩賦重文武有長髮者輒拔為冠纓【纓冠系也】餘以給宮人長史取髪白趙王虎虎徵鑒還鄴以樂平公苞代鎮長安發雍洛秦并州十六萬人【石虎分司州之河南弘農滎陽兖州之陳留東燕置洛州雍於用翻】治長安未央宮【治直之翻】虎好獵晚歲體重不能跨馬乃造獵車千乘【好呼到翻乘繩證翻】刻期校獵自靈昌津南至滎陽東極陽都為獵場【陽都縣前漢屬城陽國後漢晉屬琅邪國賢曰陽都故城在今沂州沂水縣南又曰在承縣南】使御史監察【監工銜翻】其中禽獸有犯者辠至大辟【辟毘亦翻】民有美女佳牛馬御史求之不得皆誣以犯獸論死者百餘人發諸州二十六萬人修洛陽宮發百姓牛二萬頭配朔州牧官【趙置牧官于朔方】增置女官二十四等東宮十二等公侯七十餘國皆九等大發民女三萬餘人料為三等以配之太子諸公私令采發者又將萬人郡縣務求美色多強奪人妻殺其夫及夫自殺者三千餘人至鄴虎臨軒簡第以使者為能封侯者十二人荆楚揚徐之民流叛畧盡【荆楚以國言揚徐以州言趙之壤地南陽汝南則故荆楚之地也壽陽則揚州之地也彭城下邳東海琅邪東筦則徐州之地也一曰荆楚揚徐之民先被掠及流入北界者今流叛畧盡】守令坐不能綏懷下獄誅者五十餘人【下遐嫁翻】金紫光禄大夫逯明【金紫光禄大夫即光禄大夫加金章紫綬者自此遂以為官稱逯盧谷翻】因侍切諫【因侍見而切諫也】虎大怒使龍騰拉殺之【盧募驍勇拜為龍騰中郎拉落合翻】 燕王皝以牛假貧民使佃苑中【佃亭年翻】稅其什之八自有牛者稅其七記室參軍封裕上書諫以為古者什一而稅天下之中正也降及魏晉仁政衰薄假官田官牛者不過稅其什六自有牛者中分之猶不取其七八也自永嘉以來海内蕩析武宣王綏之以德【慕容廆諡武宣王】華夷之民萬里輻湊襁負而歸之者若赤子之歸父母是以戶口十倍於舊無田者什有三四及殿下繼統南摧彊趙東兼高句麗北取宇文【民歸慕容廆事見八十八卷愍帝建興元年皝破趙事見上卷成帝咸康四年破高麗見上卷咸康八年取宇文見上康帝建元二年】拓地三千里增民十萬戶是宜悉罷苑囿以賦新民無牛者官賜之牛不當更收重稅也且以殿下之民用殿下之牛牛非殿下之有將何在哉如此則戎旗南指之日民誰不簞食壺漿以迎王師【用孟子語食祥吏翻】石虎誰與處矣【處昌呂翻下同】川瀆溝渠有廢塞者【塞悉則翻下同】皆應通利旱則灌溉潦則疏泄一夫不耕或受之飢况游食數萬何以得家給人足乎今官司猥多虚費廩禄苟才不周用皆宜澄汰【以用水為諭澄之使清而汏去其沙泥也】工商末利宜立常員學生三年無成徒塞英儁之路皆當歸之於農【塞悉則翻】殿下聖德寛明博察芻蕘【文王詢于芻蕘刈中曰芻采薪曰蕘蕘如招翻】參軍王憲大夫劉明並以言事忤旨主者處以大辟【主者謂其時主斷憲明之獄者忤五故翻處昌呂翻辟毘亦翻】殿下雖恕其死猶免官禁錮夫求諫諍而罪直言是猶適越而北行必不獲其所志矣右長史宋該等阿媚苟容輕劾諫士【劾戶槩翻又戶得翻】已無骨鯁【骨鯁以喻剛彊正直者毛晃曰鯁魚骨又骨不下咽為鯁以其謇諤難受如魚骨之咈咽也】嫉人有之掩蔽耳目不忠之甚者也皝乃下令稱覽封記室之諫孤實懼焉國以民為本民以穀為命可悉罷苑囿以給民之無田者實貧者官與之牛力有餘願得官牛者並依魏晉舊法溝瀆各有益者令以時修治【治直之翻】今戎事方興勲伐既多【王功曰勲積功曰伐】官未可減俟中原平壹徐更議之工商學生皆當裁擇夫人臣關言于人主至難也【關白也王褒聖主得賢臣頌曰進退得關其忠】雖有狂妄當擇其善者而從之王憲劉明雖罪應廢黜亦由孤之無大量也可悉復本官仍居諫司封生蹇蹇深得王臣之體【易曰王臣蹇蹇匪躬之故】其賜錢五萬宣示内外有欲陳孤過者不拘貴賤勿有所諱皝雅好文學【好呼到翻】常親臨庠序講授考校學徒至千餘人頗有妄濫者故封裕及之 詔徵衛將軍褚裒欲以為揚州刺史錄尚書事吏部尚書劉遐長史王胡之說裒曰【說輸芮翻】會稽王令德雅望國之周公也足下宜以大政授之裒乃固辭歸藩壬戍以會稽王昱為撫軍大將軍錄尚書六條事【劉聰以其子粲為丞相領大將軍錄尚書事劉延年錄尚書六條事錄六條事在錄尚書事之下是必魏晉之間先有是官聰承而置之也注又見前會工外翻】昱清虛寡欲尤善玄言常以劉惔王濛及潁川韓伯為談客【惔徒甘翻】又辟郗超為撫軍掾謝萬為從事中郎超鑒之孫也【郗鑒南渡初名臣掾以絹翻】少卓犖不羈【少詩照翻犖呂角翻卓犖不羈卓高也犖有力也言其氣韻甚高且有才力譬之馬駒逸羣不可得而羈縶也】父愔簡默冲退而嗇於財積錢至數千萬【愔於今翻】嘗開庫任超所取超散施親故一日都盡【史言郗超才具足以用世晉朝不能用惜其為桓温用也施式䜴翻】萬安之弟也清曠秀邁亦有時名燕有黑龍白龍見于龍山【龍山在龍城之東見賢徧翻】交首遊戲<br />
<br />
  解角而去燕王皝親祀以太牢赦其境内命所居新宮曰和龍 都亭肅侯庾翼疽發于背【諡法剛克為伐曰肅執心决斷曰肅疽千余翻】表子爰之行輔國將軍荆州刺史委以後任司馬義陽朱燾為南蠻校尉以千人守巴陵秋七月庚子卒翼部將于瓚等作亂【于姓也左傳宋有于犨瓚藏旱翻】殺冠軍將軍曹據【冠古玩翻】朱燾與安西長史江虨【虨逋聞翻】建武司馬毛穆之【庾翼以子方之為建武將軍守襄陽以穆之為司馬穆之即虎生也穆之字憲祖小字虎生名犯王靖后諱故改行宇後又以桓温母諱憲乃更稱小字按晉書后妃傳哀靖王皇后諱穆之】將軍袁真共誅之虨統之子也 八月豫州刺史路永叛奔趙趙王虎使永屯壽春【路永蘇峻降將也】 庾翼既卒朝議皆以諸庾世在西藩人情所安宜依翼所請以庾爰之代其任何充曰荆楚國之西門戶口百萬北帶彊胡西鄰勁蜀地勢險阻周旋萬里得人則中原可定失人則社稷可憂陸抗所謂存則吳存亡則吳亡者也【陸抗垂没之疏見八十卷武帝泰始十年】豈可以白面少年當之哉【少詩照翻】桓温英畧過人有文武器幹西夏之任無出温者【夏戶雅翻】議者又曰庾爰之肯避温乎如令阻兵恥懼不淺【言不能制爰之將為國恥又有可懼者蓋以王敦蘇峻待爰之也】充曰温足以制之諸君勿憂丹陽尹劉惔每奇温才然知其有不臣之志謂會稽王昱曰温不可使居形勝之地其位號常宜抑之勸昱自鎮上流以已為軍司昱不聽又請自行亦不聽【劉惔談客耳其言桓温無不中蓋深知温之才者設使昱鎮上流惔為司馬未足以敵燕秦揚子曰非苟知之亦允蹈之非知之難行之為難也惔徒甘翻】庚辰以徐州刺史桓温為安西將軍持節都督荆司雍益梁寧六州諸軍事領護南蠻校尉荆州刺史【為桓温專制晉朝張本雍於用翻】爰之果不敢争又以劉惔監沔中諸軍事領義成太守【監工銜翻】代庾方之徙方之爰之于豫章桓温嘗乘雪欲獵先過劉惔惔見其裝束甚嚴謂之曰老賊欲持此何為温笑曰我不為此卿安得坐談乎【温以此語答惔盡之矣温亦知惔之悉其才故發是言】 漢主勢之弟大將軍廣以勢無子求為太弟勢不許馬當解思明諫曰陛下兄弟不多若復有所廢【復扶又翻】將益孤危固請許之勢疑其與廣有謀收當思明斬之夷其三族【儲君不可求使馬當解思明為國計固當從容言之使其主自悟安可固以為請也相從而就死宜矣解戶買翻】遣太保李奕襲廣於涪城貶廣為臨卭侯廣自殺思明被收歎曰國之不亡以我數人在也【涪音浮卭渠容翻被皮義翻】今其殆矣言笑自若而死思明有智畧敢諫諍馬當素得人心及其死士民無不哀之冬十月燕王皝使慕容恪攻高句麗拔南蘇【南蘇城在南陜之東唐平高麗置南蘇州】置戍而還【還從宣翻又如字】 十二月張駿伐焉耆降之【降戶江翻】是歲駿分武威等十一郡為凉州【駿分武威武興西平張掖酒泉建康西郡湟河晉興須武安故合十一郡為凉州】以世子重華為刺史分興晉等八郡為河州【駿分興晉金城武始南安永晉大夏武成漢中八郡為河州】以寧戎校尉張瓘為刺史分燉煌等三郡及西域都護三營為沙州【晉志惟載敦煌晉昌二郡西域都護張茂以校尉玉門大護軍三郡三營為沙州而一郡不見于史蓋缺文也燉徒門翻】以西胡校尉楊宣為刺史駿自稱大都督大將軍假凉王督攝三州始置祭酒郎中大夫舍人謁者等官官號皆倣天朝【朝直遥翻下同】而微變其名車服旌旗擬於王者 趙王虎以冠軍將軍姚弋仲為持節十郡六夷大都督冠軍大將軍【冠古玩翻】弋仲清儉鯁直不治威儀言無畏避【虎之簒弋仲正色責之可以見其言無畏避矣治直之翻下同】虎甚重之朝之大議每與參决公卿皆憚而下之【朝直遥翻下遐嫁翻】武城左尉虎寵姬之弟也【東武城縣屬清河郡唐屬貝州弋仲營于廣川清河之灄頭】嘗入弋仲營侵擾其部衆弋仲執而數之曰爾為禁尉【禁尉者言尉職所以禁止姦非也數所具翻】迫脅小民我為大臣目所親見不可縱也命左右斬之尉叩頭流血左右固諫乃止燕王皝以為古者諸侯即位各稱元年于是始不用晉年號自稱十二年【燕自是不復禀命于晉矣】 趙王虎使征東將軍鄧恒將兵數萬屯樂安治攻具為取燕之計【恒戶登翻】燕王皝以慕容霸為平狄將軍【平狄將軍始于漢光武以命龎萌】戍徒河恒畏之不敢犯<br />
<br />
  二年春正月丙寅大赦 己卯都鄉文穆公何充卒充有器局臨朝正色以社稷為己任所選用皆以功效不私親舊 初夫餘居于鹿山【夫餘在玄莬北千餘里鹿山蓋直其地杜佑曰夫餘國有印丈曰濊王之印國有故城名濊城蓋本濊貊之地其國在長城之北去玄莬千里南與高麗東與挹婁西與鮮卑接】為百濟所侵【東夷有三韓國一曰馬韓二曰辰韓三曰弁韓馬韓有五十四國百濟其一也後漸強大兼諸小國其國本與句麗俱在遼東之東千餘里隋書曰百濟出自東明其後有仇台者始立其國漸以強盛初以百家濟海因號百濟杜佑曰百濟南接新羅北距高麗千餘里西限大海處小海之南】部落衰散西徙近燕而不設備【近其靳翻】燕王皝遣世子儁帥慕容軍慕容恪慕輿根三將軍萬七千騎襲夫餘【帥讀曰率】儁居中指授軍事皆以任恪遂拔夫餘虜其王玄及部落五萬餘口而還皝以玄為鎮軍將軍妻以女【妻子細翻】 二月癸丑以左光禄大夫蔡謨領司徒與會稽王昱同輔政 褚裒薦前光禄大夫顧和前司徒左長史殷浩三月丙子以和為尚書令浩為建武將軍揚州刺史和有母喪固辭不起謂所親曰古人有釋衰絰從王事者【衰倉回翻】以其才足幹時故也如和者正足以虧孝道傷風俗耳識者美之浩亦固辭會稽王昱與浩書曰屬當厄運【屬之欲翻】危弊理極足下沈識淹長【沈持林翻】足以經濟若復深存挹退【復扶又翻】苟遂本懷吾恐天下之事於此去矣足下去就即時之廢興則家國不異【言國興則家亦與之俱興國廢則家亦與之俱廢也】足下宜深思之浩乃就職 夏四月己酉朔日有食之 五月丙戌西平忠成公張駿薨官屬上世子重華為使持節大都督太尉護羌校尉凉州牧西平公假凉王【上時掌翻】赦其境内尊嫡母嚴氏為太王太后母馬氏為王太后 趙中黄門嚴生惡尚書朱軌【惡烏路翻】會久雨生譖軌不修道路又謗訕朝政【朝直遥翻下同】趙王虎囚之蒲洪諫曰陛下既有襄國鄴宮又修長安洛陽宮殿將以何用作獵車千乘環數千里以養禽獸【環音宦】奪人妻女十餘萬口以實後宮【事並見上年】聖帝明王之所為固若是乎今又以道路不修欲殺尚書陛下德政不修天降淫雨七旬乃霽霽方二日雖有鬼兵百萬亦未能去道路之塗潦【去羌呂翻】而况人乎政刑如此其如四海何其如後代何【言天下後世必將貶議其失也】願止作徒罷苑囿出宮女赦朱軌以副衆望虎雖不悅亦不之罪為之罷長安洛陽作役【為于偽翻】而竟誅朱軌又立私論朝政之法聽吏告其君奴告其主公卿以下朝覲以目相顧不敢復相過從談語【復扶又翻過古未翻經過也石虎之法雖周厲王之監謗秦始皇之禁耦語不如是之甚也】趙將軍王擢擊張重華襲武街執護軍曹權胡宣【張駿】<br />
<br />
  【置五屯護軍武街其一也在隴西水經注曰狄道縣西南有武街城晉志惠帝分隴西立狄道郡又立武街縣屬焉】徙七千餘戶于雍州【雍於用翻】凉州刺史麻秋【趙使麻秋攻凉州故授以刺史】將軍孫伏都攻金城太守張冲請降凉州震恐重華悉發境内兵使征南將軍裴恒將之以禦趙恒壁於廣武【張寔分金城之令居枝陽二縣又立永登縣合三縣立廣武郡水經注廣武城在枝陽縣西五代志武威郡允吾縣後魏置曰廣武劉昫曰唐蘭州廣武縣漢枝陽縣地恒戶登翻】久而不戰凉州司馬張耽言於重華曰國之存亡在兵兵之勝敗在將今議者舉將多推宿舊【將息亮翻】夫韓信之舉非舊德也【舉韓信事見九卷漢高帝元年】蓋明主之舉舉無常人才之所堪則授以大事今彊寇在境諸將不進人情危懼主簿謝艾兼資文武可用以禦趙重華召艾問以方畧艾願請兵七千人必破趙以報重華拜艾中堅將軍給步騎五千使擊秋艾引兵出振武夜有二梟鳴于牙中艾曰六博得梟者勝【爾雅翼博之采有梟博兼行惡道故以梟為采】今梟鳴牙中克敵之兆也進與趙戰大破之斬首五千級重華封艾為福禄伯【福禄縣自漢以來屬酒泉郡宋白曰肅州福禄縣周隋為樂涫縣武德改為福禄取漢舊名也】麻秋之克金城也縣令敦煌車濟不降伏劒而死【縣令謂金城縣令也敦徒門翻】秋又攻大夏【大夏縣漢屬隴西郡張軌分屬晉興郡後又分置大夏郡水經注大夏縣故城在枹罕縣西南北臨洮水劉昫曰河州大夏縣漢古縣也取縣西大夏水以名之】護軍梁式執太守宋晏以城應秋秋遣晏以書誘致宛戍都尉敦煌宋矩【誘音酉敦徒門翻】矩曰為人臣功既不成唯有死節耳先殺妻子而後自刎【刎扶粉翻】秋曰皆義士也收而葬之 冬漢太保李奕自晉壽舉兵反蜀人多從之衆至數萬漢主勢登城拒戰【時奕兵進逼成都】奕單騎突門門者射而殺之【騎奇寄翻射而亦翻】其衆皆潰勢大赦境内改元嘉寧勢驕淫不恤國事多居禁中罕接公卿疎忌舊臣信任左右讒諂並進刑罰苛濫由是中外離心蜀土先無□【□魯皓翻西南夷名北史曰□蓋南蠻之别種卭笮川洞之間散居山谷種類甚多畧無氏族之别又無名字所生男女唯以長幼次第呼之其丈夫稱阿謩阿段婦人阿夷阿等之類皆語之次第稱謂也】至是始從山出自巴西至犍為梓潼【犍居言翻】布滿山谷十餘萬落不可禁制大為民患加以饑饉四境之内遂至蕭條 安西將軍桓温將伐漢將佐皆以為不可江夏相袁喬勸之曰夫經畧大事固非常情所及智者了於胷中不必待衆言皆合也今為天下之患者胡蜀二寇而已蜀雖險固比胡為弱將欲除之宜先其易者【易以䜴翻】李勢無道臣民不附且恃其險遠不修戰備宜以精卒萬人輕齎疾趨比其覺之【比必寐翻】我已出其險要【已出其險要謂已踰險而出平地也】可一戰擒也蜀地富饒戶口繁庶諸葛武侯用之抗衡中夏【諸葛亮謚忠武侯夏戶雅翻】若得而有之國家之大利也論者恐大軍既西胡必闚覦此似是而非胡聞我萬里遠征以為内有重備必不敢動縱有侵軼【軼直結翻又音逸杜預曰軼突也】緣江諸軍足以拒守必無憂也温從之喬瓌之子也【袁瓌見九十五卷成帝咸康三年瓌工回翻】十一月辛未温帥益州刺史周撫南郡太守譙王無忌伐漢【帥讀曰率下同】拜表即行委安西長史范汪以留事加撫都督梁州之四郡諸軍事【梁州四郡涪陵巴東巴西巴郡也】使袁喬帥二千人為前鋒朝廷以蜀道險遠温衆少而深入【少詩沼翻】皆以為憂惟劉惔以為必克或問其故惔曰以博知之温善博者也不必得則不為但恐克蜀之後温終專制朝廷耳<br />
<br />
  三年春二月桓温軍至青衣【青衣縣漢屬蜀郡後漢順帝陽嘉二年更名漢嘉蜀立為漢嘉郡劉昫曰眉州青神縣臨青衣江西魏置青衣縣青衣水出盧山徼外東北流至武陽而合于江杜佑曰嘉州故夜郎國漢武開置犍為郡治龍游縣漢之青衣道也在大江青衣二水之會】漢主勢大發兵遣叔父右衛將軍福從兄鎮南將軍權前將軍昝堅等將之自山陽趣合水【山陽之地蓋在岷江之北峨眉山之陽水經注江水東南過犍為武陽縣青衣水沬水從西南來合注之所謂合水當是此地從才用翻昝子感翻將即亮翻趣七喻翻】諸將欲設伏於江南以待晉兵昝堅不從引兵自江北鴛鴦碕渡向犍為【碕渠宜翻曲岸也犍為唐嘉州犍為縣即其地在州東南】三月温至彭模【彭模即漢犍為郡武陽縣之彭亡聚也岑彭死處水經注江水自武陽東至彭亡聚謂之平模水亦曰外水平模去成都二百里在今眉州彭山縣】議者欲分為兩軍異道俱進以分漢兵之埶袁喬曰今懸軍深入萬里之外勝則大功可立不勝則噍類無遺【噍才肖翻】當合執齊力以取一戰之捷若分兩軍則衆心不一萬一偏敗【偏敗謂兩道並進或一軍為蜀所敗】大事去矣不如全軍而進棄去釡甑齎三日糧以示無還心勝可必也温從之留參軍孫盛周楚將羸兵守輜重【將即亮翻下同重直用翻】温自將步卒直指成都楚撫之子也李福進攻彭模孫盛等奮擊走之温進遇李權三戰三捷漢兵散走歸成都鎮軍將軍李位都迎詣温降【降戶江翻下同】昝堅至犍為【犍居言翻】乃知與温異道還自沙頭津濟比至【比必寐翻】温已軍於成都之十里陌堅衆自潰勢悉衆出戰于成都之笮橋【水經注萬里橋西上曰夷橋亦曰笮橋笮疾各翻】温前鋒不利參軍龔護戰死矢及温馬首衆懼欲退而鼓吏誤鳴進鼓【雷鼔以進衆曰進鼔】袁喬拔劒督士卒力戰遂大破之温乘勝長驅至成都縱火燒其城門漢人惶懼無復鬬志【復扶又翻】勢夜開東門走至葭萌【葭音家】使散騎常侍王幼送降文於温【散悉亶翻騎奇寄翻】自稱畧陽李勢叩頭死罪【李氏其先自巴西遷畧陽】尋輿櫬面縳詣軍門温解縳焚櫬送勢及宗室十餘人於建康【櫬初覲翻】引漢司空譙獻之等以為參佐舉賢旌善蜀人悅之 日南太守夏侯覽貪縱侵刻胡商又科調船材【夏戶雅翻調徒弔翻】云欲有所討由是諸國恚憤【恚於避翻】林邑王文攻陷日南將士死者五六千【將即亮翻】殺覽以尸祭天檄交州刺史朱蕃請以郡北横山為界【今邕州南界有横山其山横截江河我朝置横山寨及買馬場按劉昫舊唐志漢武帝開百越於交趾郡南三千里置日南郡治于朱吾林邑即漢日南郡之象林縣在郡南界四百里後漢時中原喪亂象林縣人區連殺縣令自稱林邑王遂為林邑國邕州渡海乃至交趾交趾三千里乃至日南此横山自在日南郡北界非今邕州之横山】文既去蕃使督護劉雄戍日南 漢故尚書僕射王誓鎮東將軍鄧定平南將軍王潤將軍隗文等皆舉兵反【隗五猥翻】衆各萬餘桓温自擊定使袁喬擊文皆破之温命益州刺史周撫鎮彭模斬王誓王潤温留成都三十日振旅還江陵【傳曰入而振旅杜預注振整也旅衆也】李勢至建康封歸義侯夏四月丁巳鄧定隗文等入據成都征虜將軍楊謙棄涪城退保德陽【涪音浮】趙凉州刺史麻秋攻枹罕晉昌太守郎坦以城大難守【惠帝分敦煌酒泉為晉昌郡枹罕縣前漢屬金城郡後漢屬隴西郡張軌分屬晉興郡水經注晉昌川在湟中浩亹縣西南劉朐曰晉昌郡漢寘安縣地唐為晉昌縣瓜州治所枹音膚】欲棄外城武成太守張悛曰【武成郡亦張氏置悛丑緣翻又七倫翻】棄外城則動衆心大事去矣寧戎校尉張璩從悛言固守大城【寧戎校尉亦張氏所置郎坦張悛蓋以各郡太守從張璩守枹罕璩求於翻】秋帥衆八萬圍壍數重【帥讀曰率下同重直龍翻壍七艶翻】雲梯地突百道皆進【地突者為地道突出於城中】城中禦之秋衆死傷數萬趙王虎復遣其將劉渾等帥步騎二萬會之【復扶又翻將即亮翻下除將軍外並同】郎坦恨言不用教軍士李嘉潜引趙兵千餘人登城璩督諸將力戰殺二百餘人趙兵乃退璩燒其攻具秋退保大夏【夏戶雅翻】虎以中書監石寧為征西將軍帥并司州兵二萬餘人為秋等後繼張重華將宋秦等帥戶二萬降于趙【降戶江翻】重華以謝艾為使持節軍師將軍【使疏吏翻】帥步騎三萬進軍臨河艾乘軺車戴白幍【軺音遥幍古洽翻】鳴鼓而行秋望見怒曰艾年少書生冠服如此輕我也命黑矟龍驤三千人馳擊之【矟音朔驤思將翻】艾左右大擾或勸艾宜乘馬艾不從下車踞胡床【胡床蓋今交椅之類孔穎達曰今之交牀制本自虜來隋以䜟有胡改名交牀】指麾處分【處昌呂翻分扶問翻】趙人以為有伏兵懼不敢進别將張瑁自閒道引兵截趙軍後【瑁莫報翻閒古莧翻】趙軍退艾乘勢進擊大破之斬其將杜勲汲魚獲首虜萬三千級秋單馬奔大夏五月秋與石寧復帥衆十二萬進屯河南【復扶又翻】劉寧王擢畧地晉興廣武武街【張軌分西平界置晉興郡】至于曲柳【曲柳地名在洪池嶺北】張重華使將軍牛旋拒之退守枹罕姑臧大震重華欲親出拒之謝艾固諫索遐曰【索昔各翻】君者一國之鎮不可輕動乃以艾為使持節都督征討諸軍事行衛將軍遐為軍正將軍【古有軍正黄帝法曰正無屬將軍將軍有罪以聞蓋軍中執法者也張氏遂以為將軍之號】帥步騎二萬拒之别將楊康敗劉寧于沙阜【敗補邁翻】寧退屯金城 六月辛酉大赦 秋七月林邑復䧟日南【復扶又翻下同】殺督護劉雄 隗文鄧定等立故國師范長生之子賁為帝而奉之【李雄以范長生為國師】以妖異惑衆蜀人多歸之【妖於驕翻】 趙王虎復遣征西將軍孫伏都將軍劉渾帥步騎二萬會麻秋軍長驅濟河擊張重華遂城長最【長最地名在金城河北 考異曰晉春秋作上最今從重華傳】謝艾建牙誓衆有風吹旌旗東南指索遐曰風為號令今旌旗指敵天所贊也【風雲氣候雜占曰風不旁勃旌旗暈暈隨風而揚舉或向敵終日軍行有功勝候也】艾軍于神鳥王擢與艾前鋒戰敗走還河南八月戊午艾進擊秋大破之秋遁歸金城虎聞之歎曰吾以偏師定九州今以九州之力困於枹罕彼有人焉未可圖也艾還討叛虜斯骨真等萬餘落皆破平之 趙王虎據十州之地【幽并冀司豫兖青徐雍秦十州】聚斂金帛【歛力贍翻】及外國所獻珍異府庫財物不可勝紀【勝音升】猶自以為不足悉發前代陵墓取其金寶沙門吳進言於虎曰【袁宏漢記曰沙門漢言息也蓋息欲以歸於無為也】胡運將衰晉當復興宜苦役晉人以厭其氣【厭一兼翻厭勝也】虎使尚書張羣發近郡男女十六萬人車十萬乘運土築華林苑及長牆于鄴北廣袤數十里【乘繩證翻華如字袤音茂】申鍾石璞趙攬等上疏陳天文錯亂百姓彫弊虎大怒曰使苑牆朝成吾夕没無恨矣促張羣使然燭夜作暴風大雨死者數萬人郡國前後送蒼麟十六白鹿七虎命司虞張曷柱調之以駕芝蓋【晉職官志太僕之屬有典虞都尉趙之司虞即是官也張曷柱人姓名芝蓋者蓋為瑞芝之形】大朝會列於殿庭【朝直遥翻】九月命太子宣出祈福于山川因行遊獵宣乘大輅羽葆華蓋建天子旌旗十有六軍戎卒十八萬出自金明門【水經注鄴城有七門南曰鳳陽門中曰中陽門次曰廣陽門東曰建春門北曰廣德門次曰廏門西曰西明門蓋即金明門也】虎從其後宮升陵霄觀望之【觀古玩翻】笑曰我家父子如此自非天崩地陷當復何愁但抱子弄孫日為樂耳【復扶又翻樂音洛】宣所舍輒列人為長圍四面各百里驅禽獸至暮皆集其所使文武皆跪立重行圍守【重直龍翻行戶剛翻】炬火如晝命勁騎百餘馳射其中宣與姬妾乘輦臨觀獸盡而止或獸有迸逸【迸比孟翻】當圍守者有爵則奪馬步驅一日無爵則鞭之一百士卒飢凍死者萬有餘人所過三州十五郡資儲皆無孑遺【以下韜所出徵之宣所過三州蓋司兖豫也】虎復命韜繼出自并州至于秦雍亦如之【復扶又翻雍於用翻】宣怒其與已鈞敵愈嫉之宦者趙生得幸于宣無寵於韜微勸宣除之於是始有殺韜之謀矣 趙麻秋又襲張重華將張瑁敗之【據載記瑁時屯河陝敗補邁翻】斬首三千餘級枹罕護軍李逵帥衆七千降于趙自河以南氐羌皆附於趙【趙強凉弱凉戰雖數勝人心隨強弱而為向背况復有張瑁之敗乎帥讀曰率降戶江翻】冬十月乙丑遣侍御史俞歸至凉州授張重華侍中<br />
<br />
  大都督督隴右關中諸軍事大將軍凉州刺史西平公歸至姑臧重華欲稱凉王未肯受詔使所親沈猛私謂歸曰主公奕世為晉忠臣【奕世累葉也】今曾不如鮮卑何也朝廷封慕容皝為燕王而主公纔為大將軍何以褒勸忠賢乎明臺宜移河右共勸州主為凉王【歸為侍御史以將命故謂之明臺移謂移文】人臣出使苟利社稷專之可也【使疏吏翻】歸曰吾子失言【古者列國之大夫率相謂曰吾子儀禮注曰子者男子之美稱言吾子相親之辭】昔三代之王也爵之貴者莫若上公【三代封建列爵五等曰公侯伯子男上公九命作伯】及周之衰吳楚始僭號稱王而諸侯不之非蓋以蠻夷畜之也借使齊魯稱王諸侯豈不四面攻之乎漢高祖封韓彭為王尋皆誅滅蓋權時之宜非厚之也聖上以貴公忠賢故爵以上公任以方伯寵榮極矣豈鮮卑夷狄所可比哉且吾聞之功有大小賞有重輕今貴公始繼世而為王【言重華始繼父位未有功于晉而求為王也】若帥河右之衆【帥讀曰率】東平胡羯修復陵廟迎天子返洛陽將何以加之乎重華乃止 武都氐王楊初遣使來稱藩詔以初為使持節征南將軍雍州刺史仇池公【使疏吏翻雍於用翻】 十二月振威護軍蕭敬文殺征虜將軍楊謙攻涪城陷之自稱益州牧遂取巴西通于漢中【振威護軍晉官也蕭敬文以晉新并蜀又有范賁之亂故亦乘之而反涪音浮】<br />
<br />
  資治通鑑卷九十七<br />
<br />
<史部,編年類,資治通鑑>  <br>
   </div> 

<script src="/search/ajaxskft.js"> </script>
 <div class="clear"></div>
<br>
<br>
 <!-- a.d-->

 <!--
<div class="info_share">
</div> 
-->
 <!--info_share--></div>   <!-- end info_content-->
  </div> <!-- end l-->

<div class="r">   <!--r-->



<div class="sidebar"  style="margin-bottom:2px;">

 
<div class="sidebar_title">工具类大全</div>
<div class="sidebar_info">
<strong><a href="http://www.guoxuedashi.com/lsditu/" target="_blank">历史地图</a></strong>  
<a href="http://www.880114.com/" target="_blank">英语宝典</a>  
<a href="http://www.guoxuedashi.com/13jing/" target="_blank">十三经检索</a> 
<br><strong><a href="http://www.guoxuedashi.com/gjtsjc/" target="_blank">古今图书集成</a></strong> 
<a href="http://www.guoxuedashi.com/duilian/" target="_blank">对联大全</a> <strong><a href="http://www.guoxuedashi.com/xiangxingzi/" target="_blank">象形文字典</a></strong> 

<br><a href="http://www.guoxuedashi.com/zixing/yanbian/">字形演变</a>  <strong><a href="http://www.guoxuemi.com/hafo/" target="_blank">哈佛燕京中文善本特藏</a></strong>
<br><strong><a href="http://www.guoxuedashi.com/csfz/" target="_blank">丛书&方志检索器</a></strong> <a href="http://www.guoxuedashi.com/yqjyy/" target="_blank">一切经音义</a>  

<br><strong><a href="http://www.guoxuedashi.com/jiapu/" target="_blank">家谱族谱查询</a></strong>  <strong><a href="http://shufa.guoxuedashi.com/sfzitie/" target="_blank">书法字帖欣赏</a></strong> 
<br>

</div>
</div>


<div class="sidebar" style="margin-bottom:0px;">

<font style="font-size:22px;line-height:32px">QQ交流群9:489193090</font>


<div class="sidebar_title">手机APP 扫描或点击</div>
<div class="sidebar_info">
<table>
<tr>
	<td width=160><a href="http://m.guoxuedashi.com/app/" target="_blank"><img src="/img/gxds-sj.png" width="140"  border="0" alt="国学大师手机版"></a></td>
	<td>
<a href="http://www.guoxuedashi.com/download/" target="_blank">app软件下载专区</a><br>
<a href="http://www.guoxuedashi.com/download/gxds.php" target="_blank">《国学大师》下载</a><br>
<a href="http://www.guoxuedashi.com/download/kxzd.php" target="_blank">《汉字宝典》下载</a><br>
<a href="http://www.guoxuedashi.com/download/scqbd.php" target="_blank">《诗词曲宝典》下载</a><br>
<a href="http://www.guoxuedashi.com/SiKuQuanShu/skqs.php" target="_blank">《四库全书》下载</a><br>
</td>
</tr>
</table>

</div>
</div>


<div class="sidebar2">
<center>


</center>
</div>

<div class="sidebar"  style="margin-bottom:2px;">
<div class="sidebar_title">网站使用教程</div>
<div class="sidebar_info">
<a href="http://www.guoxuedashi.com/help/gjsearch.php" target="_blank">如何在国学大师网下载古籍?</a><br>
<a href="http://www.guoxuedashi.com/zidian/bujian/bjjc.php" target="_blank">如何使用部件查字法快速查字?</a><br>
<a href="http://www.guoxuedashi.com/search/sjc.php" target="_blank">如何在指定的书籍中全文检索?</a><br>
<a href="http://www.guoxuedashi.com/search/skjc.php" target="_blank">如何找到一句话在《四库全书》哪一页?</a><br>
</div>
</div>


<div class="sidebar">
<div class="sidebar_title">热门书籍</div>
<div class="sidebar_info">
<a href="/so.php?sokey=%E8%B5%84%E6%B2%BB%E9%80%9A%E9%89%B4&kt=1">资治通鉴</a> <a href="/24shi/"><strong>二十四史</strong></a>&nbsp; <a href="/a2694/">野史</a>&nbsp; <a href="/SiKuQuanShu/"><strong>四库全书</strong></a>&nbsp;<a href="http://www.guoxuedashi.com/SiKuQuanShu/fanti/">繁体</a>
<br><a href="/so.php?sokey=%E7%BA%A2%E6%A5%BC%E6%A2%A6&kt=1">红楼梦</a> <a href="/a/1858x/">三国演义</a> <a href="/a/1038k/">水浒传</a> <a href="/a/1046t/">西游记</a> <a href="/a/1914o/">封神演义</a>
<br>
<a href="http://www.guoxuedashi.com/so.php?sokeygx=%E4%B8%87%E6%9C%89%E6%96%87%E5%BA%93&submit=&kt=1">万有文库</a> <a href="/a/780t/">古文观止</a> <a href="/a/1024l/">文心雕龙</a> <a href="/a/1704n/">全唐诗</a> <a href="/a/1705h/">全宋词</a>
<br><a href="http://www.guoxuedashi.com/so.php?sokeygx=%E7%99%BE%E8%A1%B2%E6%9C%AC%E4%BA%8C%E5%8D%81%E5%9B%9B%E5%8F%B2&submit=&kt=1"><strong>百衲本二十四史</strong></a>  <a href="http://www.guoxuedashi.com/so.php?sokeygx=%E5%8F%A4%E4%BB%8A%E5%9B%BE%E4%B9%A6%E9%9B%86%E6%88%90&submit=&kt=1"><strong>古今图书集成</strong></a>
<br>

<a href="http://www.guoxuedashi.com/so.php?sokeygx=%E4%B8%9B%E4%B9%A6%E9%9B%86%E6%88%90&submit=&kt=1">丛书集成</a> 
<a href="http://www.guoxuedashi.com/so.php?sokeygx=%E5%9B%9B%E9%83%A8%E4%B8%9B%E5%88%8A&submit=&kt=1"><strong>四部丛刊</strong></a>  
<a href="http://www.guoxuedashi.com/so.php?sokeygx=%E8%AF%B4%E6%96%87%E8%A7%A3%E5%AD%97&submit=&kt=1">說文解字</a> <a href="http://www.guoxuedashi.com/so.php?sokeygx=%E5%85%A8%E4%B8%8A%E5%8F%A4&submit=&kt=1">三国六朝文</a>
<br><a href="http://www.guoxuedashi.com/so.php?sokeytm=%E6%97%A5%E6%9C%AC%E5%86%85%E9%98%81%E6%96%87%E5%BA%93&submit=&kt=1"><strong>日本内阁文库</strong></a> <a href="http://www.guoxuedashi.com/so.php?sokeytm=%E5%9B%BD%E5%9B%BE%E6%96%B9%E5%BF%97%E5%90%88%E9%9B%86&ka=100&submit=">国图方志合集</a> <a href="http://www.guoxuedashi.com/so.php?sokeytm=%E5%90%84%E5%9C%B0%E6%96%B9%E5%BF%97&submit=&kt=1"><strong>各地方志</strong></a>

</div>
</div>


<div class="sidebar2">
<center>

</center>
</div>
<div class="sidebar greenbar">
<div class="sidebar_title green">四库全书</div>
<div class="sidebar_info">

《四库全书》是中国古代最大的丛书,编撰于乾隆年间,由纪昀等360多位高官、学者编撰,3800多人抄写,费时十三年编成。丛书分经、史、子、集四部,故名四库。共有3500多种书,7.9万卷,3.6万册,约8亿字,基本上囊括了古代所有图书,故称“全书”。<a href="http://www.guoxuedashi.com/SiKuQuanShu/">详细>>
</a>

</div> 
</div>

</div>  <!--end r-->

</div>
<!-- 内容区END --> 

<!-- 页脚开始 -->
<div class="shh">

</div>

<div class="w1180" style="margin-top:8px;">
<center><script src="http://www.guoxuedashi.com/img/plus.php?id=3"></script></center>
</div>
<div class="w1180 foot">
<a href="/b/thanks.php">特别致谢</a> | <a href="javascript:window.external.AddFavorite(document.location.href,document.title);">收藏本站</a> | <a href="#">欢迎投稿</a> | <a href="http://www.guoxuedashi.com/forum/">意见建议</a> | <a href="http://www.guoxuemi.com/">国学迷</a> | <a href="http://www.shuowen.net/">说文网</a><script language="javascript" type="text/javascript" src="https://js.users.51.la/17753172.js"></script><br />
  Copyright &copy; 国学大师 古典图书集成 All Rights Reserved.<br>
  
  <span style="font-size:14px">免责声明:本站非营利性站点,以方便网友为主,仅供学习研究。<br>内容由热心网友提供和网上收集,不保留版权。若侵犯了您的权益,来信即刪。scp168@qq.com</span>
  <br />
ICP证:<a href="http://www.beian.miit.gov.cn/" target="_blank">鲁ICP备19060063号</a></div>
<!-- 页脚END --> 
<script src="http://www.guoxuedashi.com/img/plus.php?id=22"></script>
<script src="http://www.guoxuedashi.com/img/tongji.js"></script>

</body>
</html>
