<!DOCTYPE html PUBLIC "-//W3C//DTD XHTML 1.0 Transitional//EN" "http://www.w3.org/TR/xhtml1/DTD/xhtml1-transitional.dtd">
<html xmlns="http://www.w3.org/1999/xhtml">
<head>
<meta http-equiv="Content-Type" content="text/html; charset=utf-8" />
<meta http-equiv="X-UA-Compatible" content="IE=Edge,chrome=1">
<title>資治通鑒_245-資治通鑑卷二百四十四_245-資治通鑑卷二百四十四</title>
<meta name="Keywords" content="資治通鑒_245-資治通鑑卷二百四十四_245-資治通鑑卷二百四十四">
<meta name="Description" content="資治通鑒_245-資治通鑑卷二百四十四_245-資治通鑑卷二百四十四">
<meta http-equiv="Cache-Control" content="no-transform" />
<meta http-equiv="Cache-Control" content="no-siteapp" />
<link href="/img/style.css" rel="stylesheet" type="text/css" />
<script src="/img/m.js?2020"></script> 
</head>
<body>
 <div class="ClassNavi">
<a  href="/24shi/">二十四史</a> | <a href="/SiKuQuanShu/">四库全书</a> | <a href="http://www.guoxuedashi.com/gjtsjc/"><font  color="#FF0000">古今图书集成</font></a> | <a href="/renwu/">历史人物</a> | <a href="/ShuoWenJieZi/"><font  color="#FF0000">说文解字</a></font> | <a href="/chengyu/">成语词典</a> | <a  target="_blank"  href="http://www.guoxuedashi.com/jgwhj/"><font  color="#FF0000">甲骨文合集</font></a> | <a href="/yzjwjc/"><font  color="#FF0000">殷周金文集成</font></a> | <a href="/xiangxingzi/"><font color="#0000FF">象形字典</font></a> | <a href="/13jing/"><font  color="#FF0000">十三经索引</font></a> | <a href="/zixing/"><font  color="#FF0000">字体转换器</font></a> | <a href="/zidian/xz/"><font color="#0000FF">篆书识别</font></a> | <a href="/jinfanyi/">近义反义词</a> | <a href="/duilian/">对联大全</a> | <a href="/jiapu/"><font  color="#0000FF">家谱族谱查询</font></a> | <a href="http://www.guoxuemi.com/hafo/" target="_blank" ><font color="#FF0000">哈佛古籍</font></a> 
</div>

 <!-- 头部导航开始 -->
<div class="w1180 head clearfix">
  <div class="head_logo l"><a title="国学大师官网" href="http://www.guoxuedashi.com" target="_blank"></a></div>
  <div class="head_sr l">
  <div id="head1">
  
  <a href="http://www.guoxuedashi.com/zidian/bujian/" target="_blank" ><img src="http://www.guoxuedashi.com/img/top1.gif" width="88" height="60" border="0" title="部件查字,支持20万汉字"></a>


<a href="http://www.guoxuedashi.com/help/yingpan.php" target="_blank"><img src="http://www.guoxuedashi.com/img/top230.gif" width="600" height="62" border="0" ></a>


  </div>
  <div id="head3"><a href="javascript:" onClick="javascript:window.external.AddFavorite(window.location.href,document.title);">添加收藏</a>
  <br><a href="/help/setie.php">搜索引擎</a>
  <br><a href="/help/zanzhu.php">赞助本站</a></div>
  <div id="head2">
 <a href="http://www.guoxuemi.com/" target="_blank"><img src="http://www.guoxuedashi.com/img/guoxuemi.gif" width="95" height="62" border="0" style="margin-left:2px;" title="国学迷"></a>
  

  </div>
</div>
  <div class="clear"></div>
  <div class="head_nav">
  <p><a href="/">首页</a> | <a href="/ShuKu/">国学书库</a> | <a href="/guji/">影印古籍</a> | <a href="/shici/">诗词宝典</a> | <a   href="/SiKuQuanShu/gxjx.php">精选</a> <b>|</b> <a href="/zidian/">汉语字典</a> | <a href="/hydcd/">汉语词典</a> | <a href="http://www.guoxuedashi.com/zidian/bujian/"><font  color="#CC0066">部件查字</font></a> | <a href="http://www.sfds.cn/"><font  color="#CC0066">书法大师</font></a> | <a href="/jgwhj/">甲骨文</a> <b>|</b> <a href="/b/4/"><font  color="#CC0066">解密</font></a> | <a href="/renwu/">历史人物</a> | <a href="/diangu/">历史典故</a> | <a href="/xingshi/">姓氏</a> | <a href="/minzu/">民族</a> <b>|</b> <a href="/mz/"><font  color="#CC0066">世界名著</font></a> | <a href="/download/">软件下载</a>
</p>
<p><a href="/b/"><font  color="#CC0066">历史</font></a> | <a href="http://skqs.guoxuedashi.com/" target="_blank">四库全书</a> |  <a href="http://www.guoxuedashi.com/search/" target="_blank"><font  color="#CC0066">全文检索</font></a> | <a href="http://www.guoxuedashi.com/shumu/">古籍书目</a> | <a   href="/24shi/">正史</a> <b>|</b> <a href="/chengyu/">成语词典</a> | <a href="/kangxi/" title="康熙字典">康熙字典</a> | <a href="/ShuoWenJieZi/">说文解字</a> | <a href="/zixing/yanbian/">字形演变</a> | <a href="/yzjwjc/">金 文</a> <b>|</b>  <a href="/shijian/nian-hao/">年号</a> | <a href="/diming/">历史地名</a> | <a href="/shijian/">历史事件</a> | <a href="/guanzhi/">官职</a> | <a href="/lishi/">知识</a> <b>|</b> <a href="/zhongyi/">中医中药</a> | <a href="http://www.guoxuedashi.com/forum/">留言反馈</a>
</p>
  </div>
</div>
<!-- 头部导航END --> 
<!-- 内容区开始 --> 
<div class="w1180 clearfix">
  <div class="info l">
   
<div class="clearfix" style="background:#f5faff;">
<script src='http://www.guoxuedashi.com/img/headersou.js'></script>

</div>
  <div class="info_tree"><a href="http://www.guoxuedashi.com">首页</a> > <a href="/SiKuQuanShu/fanti/">四库全书</a>
 > <h1>资治通鉴</h1> <!--         下载:【右键另存为】即可 --></div>
  <div class="info_content zj clearfix">
  
<div class="info_txt clearfix" id="show">
<center style="font-size:24px;">245-資治通鑑卷二百四十四</center>
    資治通鑑卷二百四十四 宋 司馬光 撰<br />
<br />
  胡三省 音注<br />
<br />
  唐紀六十【起屠維作噩盡昭陽赤奮若凡五年】<br />
<br />
  文宗元聖昭獻孝皇帝上之下<br />
<br />
  太和三年春正月亓志紹與成德合兵掠貝州【貝州在魏州北二百一十里】 義成行營兵三千人先屯齊州使之禹城【之往也一作屯禹城漢祝阿縣地天寶元年改為禹城以縣西有禹息古城也屬齊州九域志在州西北一百三十里】中道潰叛横海節度使李祐討誅之 李聽史唐合兵撃亓志紹破之志紹將其衆五千奔鎭州 李載義奏攻滄州長蘆拔之【滄州治青池縣九域志長蘆鎭屬青池】 甲辰昭義奏亓志紹餘衆萬五千人詣本道降寘之洺州 二月横海節度使李祐帥諸道行營兵擊李同捷破之【帥讀曰率】進攻德州【九域志德州東北至滄州二百三十里】 武寜捉生兵馬使石雄勇敢愛士卒王智興殘虐軍中欲逐智興而立雄智興知之因雄立功奏請除刺史丙辰以雄為壁州刺史【宋白曰壁州本漢宕渠縣地後魏大統中於今州里置諾水縣唐武德八年立壁州以縣西一里壁山為名京師西南一千八百二十二里】 史憲誠聞滄景將平而懼其子唐勸之入朝丙寅憲誠使唐奉表請入朝且請以所管聽命石雄既去武寜王智興悉殺軍中與雄善者百餘人<br />
<br />
  夏四月戊午智興奏雄揺動軍情請誅之上知雄無罪免死長流白州【為武宗復用石雄張本武德三年析合浦縣地置傳白縣四年置南州六年改白州至京師六千七百一十五里州縣皆因傳白江為名】 戊辰李載義奏攻滄州破其羅城【羅城外城也】李祐拔德州城中將卒三千餘人奔鎭州李同捷與祐書請降【降戶江翻】祐并奏其書諫議大夫柏耆受詔宣慰行營好張大聲勢以威制諸將諸將已惡之矣【好呼到翻惡烏路翻】及李同捷請降於祐祐遣大將萬洪代守滄州耆疑同捷之詐自將數百騎馳入滄州以事誅洪取同捷及其家屬詣京師乙亥至將陵【將陵漢安德縣地隋分安德於將陵故城置縣唐屬德州】或言王庭湊欲以奇兵簒同捷【簒初患翻奪也】乃斬同捷傳首滄景悉平五月庚寅加李載義同平章事 【考異曰實録作庚寅誤】諸道兵攻李同捷三年僅能下之【上初元即討同捷至是三年】而柏耆徑入城取為已功諸將疾之争上表論列辛卯貶耆為循州司戶【循州古龍川地隋置循州 考異曰實録四月李祐收德州同捷請降於祐祐疑其詐柏耆請以騎兵三百入滄州祐從之耆徑入滄收同捷與其家屬赴京師又詔曰假勢張皇乘險縱恣指揮彈壓奏報蔑聞擅入滄州專殺大將補署逆校潛送兇渠舊傳曰滄德平諸將害耆邀功争上表論列上不獲已貶循州司戶新傳曰同捷請降祐使萬洪代守滄州同捷未出也耆以三百騎馳入滄以事誅洪與同捷朝京師既行諜言王庭湊欲以奇兵刼同捷耆遂斬其首以獻諸將疾其功比奏攢詆文宗不獲已貶耆循州司戶參軍蓋耆張皇邀功則有之然諸將疾之而論奏文宗不得已而貶黜亦其實也至於賜死則因馬國亮奏其受同捷奴婢綾絹故也】李祐尋薨 壬寅攝魏博副使史唐奏改名孝章 六月丙辰詔鎭州四面行營各歸本道休息但務保境勿相往來惟庭湊効順為逹章表【為于偽翻】餘皆勿受 辛酉以史憲誠為兼侍中河中節度使以李聽兼魏博節度使【李聽本帥義成使兼魏博】分相衛澶三州以史孝章為節度使【澶時連翻】 初李祐聞柏耆殺萬洪大驚疾遂劇上曰祐若死是耆殺之也癸酉賜耆自盡 河東節度使李程奏得王庭湊書請納景州【考異曰按景州本隸横海蓋因李同捷之亂庭湊據有之同捷既平庭湊懼而復進之也】又奏亓志紹自縊【縊於賜翻又一計翻】 上遣中使賜史憲誠旌節癸酉至魏州時李聽自貝州還軍館陶遷延未進【館陶在魏州北四十五里】憲誠竭府庫以治行【治直之翻】甲戍軍亂殺憲誠奉牙内都知兵馬使靈武何進滔知留後李聽進至魏州進滔拒之不得入秋七月進滔出兵擊李聽聽不為備大敗潰走 【考異曰新進滔傳曰進縚下令曰公等既廹我當聽吾令衆唯唯孰殺前使及監軍者疏出之凡斬九十餘人釋脇從者素服臨哭將吏皆入弔詔拜留後按進滔結王庭湊以拒李聽又襲擊聽大破之安能如是新傳蓋據柳公權德政碑云公謂將士曰既廹以為長當謹而聽承命都將摠事者諭之曰害前使與監軍兇黨籍其姓名仍集之於庭無使漏網卒獲九十三人白黑既分善惡無誤會衆顯戮共棄咸悦公於是素服而哭將吏序弔此恐涉溢美之辭耳今從舊傳】晝夜兼行趣淺口【九域志魏州館陶縣有淺口鎭趣七喻翻】失亡過半輜重兵械盡棄之【重直用翻】昭義兵救之聽僅而得免歸於滑臺【李聽本鎭滑州】河北久用兵饋運不給朝廷厭苦之八月壬子以進滔為魏博節度使復以相衛澶三州歸之 滄州承喪亂之餘【喪息浪翻】骸骨蔽地城空野曠戶口存者什無三四癸丑以衛尉卿殷侑為齊德滄景節度使【是年始以齊州隸横海】侑至鎭與士卒同甘苦招撫百姓勸之耕桑流散者稍稍復業先是本軍三萬人皆仰給度支【先悉薦翻仰牛向翻】侑至一年租稅自能贍其半二年請悉罷度支給賜三年之後戶口滋殖倉廪充盈【史言方鎭得其人則可轉荒殘為富實】 王庭湊因隣道微露請服之意壬申赦庭湊及將士復其官爵 徵浙西觀察使李德裕為兵部侍郎裴度薦以為相會吏部侍郎李宗閔有宦官之助甲戍以宗閔同平章事 上性儉素九月辛巳命中尉以下毋得衣紗縠綾羅【衣於既翻】聽朝之暇惟以書史自娛聲樂遊畋未嘗留意駙馬韋處仁嘗著夾羅巾【處昌呂翻著陟畧翻劉昫曰武德已來始有巾子文官名流上平頭小様者則天時朝貢臣内賜高頭巾子呼為武家諸王様中宗景龍四年三月因内宴賜宰臣以下内様巾子開元已來文官士伍多以紫皂官絁為頭巾平頭巾子相倣為雅製玄宗開元十九年十月賜供奉及諸司長官羅頭巾及官様巾迄於今服之】上謂曰朕慕卿門地清素故有選尚【處仁尚穆宗女新豐公主】如此巾服聽其他貴戚為之卿不須爾 壬辰以李德裕為義成節度使李宗閔惡其逼已【惡烏露翻】故出之 冬十月丙辰以李聽為太子少師 路隋言於上曰宰相任重不宜兼金穀瑣碎之務如楊國忠元載皇甫鎛皆奸臣所為不足法也上以為然於是裴度辭度支上許之 十一月甲午上祀圓丘赦天下四方毋得獻奇巧之物其纎麗布帛皆禁之焚其機杼 丙申西川節度使杜元頴奏南詔入寇元頴以舊相文雅自高【杜元頴長慶初相穆宗】不曉軍事專務蓄積减削士卒衣糧西南戍邊之卒衣食不足皆入蠻境鈔盗以自給【鈔楚交翻】蠻人反以衣食資之由是蜀中虚實動静蠻皆知之南詔自嵯顚謀大舉入寇【嵯顚弑君立君遂專南詔之政嵯才何翻】邊州屢以告元頴不之信嵯顚兵至邊城一無備禦蠻以蜀卒為鄉導【鄉讀曰嚮】襲陷嶲戎二州甲辰元頴遣兵與戰於卭州南蜀兵大敗蠻遂陷卭州【卭渠容翻】 武寜節度使王智興入朝 詔東川興元荆南兵以救西川十二月丁未朔又鄂岳襄鄧陳許等兵繼之 以王智興為忠武節度使【智興自徐徙陳】 己酉以東川節度使郭釗為西川節度使兼權東川節度事嵯顚自卭州引兵徑抵成都【九域志自卭州東至成都二百六十里】庚戌陷其外郭杜元穎帥衆保牙城以拒之【帥讀曰率考異曰實錄寇及子城元穎方覺知按實録十一月丙申元頴奏南詔入寇乙巳奏圍清溪關十二月丙辰奏官軍失利蠻陷卭州至此乃云寇及子城元頴方覺知似尤之太過今不取】欲遁者數四壬子貶元頴為邵州刺史 己未以右領軍大將軍董重質為神策諸道西川行營節度使又太原鳳翔兵赴西川南詔寇東川入梓州西川【東川節度治梓州】釗兵寡弱不能戰【釗上更當有郭字蜀本正如此】以書責嵯顚嵯顚復書曰杜元頴侵擾我故興兵報之耳與釗修好而退【好呼到翻】蠻留成都西郭十日其始慰撫蜀人市肆安堵將行乃大掠子女百工數萬人及珍貨而去蜀人恐懼往往赴江流尸塞江而下【塞悉則翻】嵯顚自為軍殿【殿丁練翻】及大度水嵯顚謂蜀人曰此南吾境也聽汝哭别鄉國衆皆慟哭赴水死者以千計自是南詔工巧埓於蜀中【埓龍輒翻等也】嵯顚遣使上表稱蠻比修職貢【比毗至翻】豈敢犯邊正以杜元頴不恤軍士怨苦元頴競為鄉導祈我此行以誅虐帥【帥所類翻】誅之不遂無以慰蜀士之心願陛下誅之丁卯再貶元頴循州司馬詔董重質及諸道兵皆引還郭釗至成都與南詔立約不相侵擾詔中使以國信賜嵯顚<br />
<br />
  四年春正月辛巳武昌節度使牛僧孺入朝 戊子立子永為魯王 李宗閔引薦牛僧孺辛卯以僧孺為兵部尚書同平章事於是二人相與排擯李德裕之黨【排擠也擯斥也】稍稍逐之 南詔之寇成都也詔山南西道兵救之【事見上卷】興元兵少【山南西道節度治興元府】節度使李絳募兵千人赴之未至蠻退而還【還從宣翻又如字】興元兵有常額詔新募兵悉罷之二月乙卯絳悉召新軍諭以詔旨而遣之仍賜以廪麥皆怏怏而退【怏於兩翻】往辭監軍監軍楊叔元素惡絳不奉己以賜物薄激之衆怒大譟掠庫兵趨使牙【惡烏露翻趨七喻翻節度使所居為使宅治事之所為使牙使疏吏翻】絳方與僚佐宴不為備走登北城或勸縋而出【縋馳偽翻】絳曰吾為元帥豈可逃去麾推官趙存約令去存約曰存約受明公知何可苟免牙將王景延與賊力戰死絳存約及觀察判官薛齊皆為亂兵所害賊遂屠絳家 【考異曰新傳曰楊叔元素疾絳遣人迎說軍士曰將收募直而還為民士皆怒乃譟而入劫庫兵絳方宴不設備遂握節登陴或言縋城可以免絳不從遂遇害實錄絳召諸卒以詔旨諭而遣之廪麥以賞衆皆怏怏而退出壘門衆有請辭監軍者而監軍楊叔元貪財怙寵素怨絳之不奉已與絳為隙久矣至是因以賞薄激之散卒遂作亂今從之】戊午叔元奏絳收新軍募直以致亂庚申以尚書右丞温造為山南西道節度使是時三省官上疏共論李絳之寃諫議大夫孔敏行具呈叔元激怒亂兵上始悟 三月己亥朔以刑部尚書柳公綽為河東節度使先是囘鶻入貢及互市【先悉薦翻】所過恐其為變常嚴兵迎送防衛之公綽至鎭囘鶻遣梅錄李暢以馬萬匹互市公綽但遣牙將單騎迎勞於境【勞力到翻】至則大闢牙門受其禮謁暢感泣戒其下在路不敢馳獵無所侵擾陘北沙陀素驍勇【沙陀保神武川在陘嶺之北陘音刑】為九姓六州胡所畏伏公綽奏以其酋長朱邪執宜為隂山都督代北行營招撫使使居雲朔塞下捍禦北邊執宜與諸酋長入謁公綽與之宴執宜神彩嚴整進退有禮公綽謂僚佐曰執宜外嚴而内寛言徐而理當【酋慈由翻長知兩翻邪讀曰耶當耶浪翻】福祿人也執宜母妻入見公綽使夫人與之飲酒饋遺之【見賢遍翻遺唯季翻】執宜感恩為之盡力【為于偽翻】塞下舊有廢府十一【舊書作廢栅當從之蓋考之唐志雲朔塞下無十一府也】執宜修之使其部落三千人分守之自是雜虜不敢犯塞【雜虜謂退渾囘鶻韃靼奚室韋之屬】 温造行至褒城【褒城漢褒中縣唐屬興元府九域志褒城在府西北四十五里】遇興元都將衛志忠征蠻歸造密與之謀誅亂者以其兵八百人為牙隊五百人為前軍入府分守諸門己卯造視事饗將士於牙門造曰吾欲問新軍去留之意宜悉使來前既勞問【勞力到翻】命坐行酒志忠密以牙兵圍之既合唱殺【圍既合唱聲曰殺衆應聲而進殺之】新軍八百餘人皆死楊叔元起擁造靴求生造命囚之其手殺絳者斬之百段餘皆斬首投尸漢水以百首祭李絳三十首祭死事者具事以聞己丑流楊叔元於康州【康州漢端溪縣地武德四年置南康州貞觀十二年去南字至京師五千七百五十里】癸卯加淮南節度使段文昌同平章事為荆南節度使奚寇幽州夏四月丁未盧龍節度使李載義擊破之<br />
<br />
  辛酉擒其王茹羯以獻【羯居列翻】 裴度以高年多疾懇辭機政六月丁未以度為司徒平章軍國重事【平章軍國重事者平章大事不復煩以細務與同平章事之官不同 考異曰寶歷二年度入相時猶守司空自後未嘗遷官至此實録直言司徒裴度按制辭云遷秩上公式是殊寵又云宜其首贊機衡弘敷教典蓋此時方遷司徒實錄先云司徒裴度誤也】俟疾損三五日一入中書【疾損猶言疾减也】上患宦者彊盛憲宗敬宗弑逆之黨猶有在左右者中尉王守澄尤專横【横尸孟翻】招權納賄上不能制嘗密與翰林學士宋申錫言之申錫請漸除其偪【欲以漸去其威權偪上者偪音逼】上以申錫沈厚忠謹可倚以事擢為尚書右丞七月癸未以申錫同平章事【疑則勿任任則勿疑文宗為負宋申錫矣為申錫貶逐張本】 初裴度征淮西【謂元和討吳元濟時也】奏李宗閔為觀察判官由是漸獲進用至是怨度薦李德裕因其謝病九月壬午以度兼侍中充山南東道節度使 西川節度使郭釗以疾求代冬十月戊申以義成節度使李德裕為西川節度使蜀自南詔入寇一方殘弊郭釗多病未暇完補德裕至鎭作籌邉樓圖蜀地形南入南詔西逹吐蕃日召老於軍旅習邉事者雖走卒蠻夷無所間【蜀自清溪關則南入南詔踰西山則西逹吐蕃間古莧翻】訪以山川城邑道路險易廣狹遠近【易以豉翻】未踰月皆若身嘗涉歷上命德裕修塞清溪關以斷南詔入寇之路【塞悉則翻下同斷音短】或無土則以石壘之德裕上言通蠻細路至多不可塞惟重兵鎭守可保無虞但黎雅以來得萬人成都得二萬人精加訓練則蠻不敢動矣邉兵又不宜多須力可臨制崔旴之殺郭英乂【見二百二十四卷代宗永泰元年】張朏之逐張延賞【見二百二十九卷德宗建中四年】皆鎭兵也時北兵皆歸本道惟河中陳許三千人在成都有詔來年三月亦歸蜀人忷懼【忷許拱翻】德裕奏乞鄭滑五百人陳許千人以鎭蜀且言蜀兵脆弱新為蠻寇所困皆破膽不堪征戍【戰勝之威士氣百倍敗兵之卒没世不復正謂此也脆此芮翻】若北兵盡歸則與杜元頴時無異蜀不可保恐議者云蜀經蠻寇以來已自增兵曏者蠻寇已逼元頴始募市人為兵得三千餘人徒有其數實不可用郭釗募北兵僅得百餘人臣復召募得二百餘人【復扶又翻】此外皆元頴舊兵也恐議者又聞一夫當關之說【一夫當關萬夫莫前前人所以言蜀之險也】以為清溪可塞臣訪之蜀中老將清溪之旁大路有三自餘小徑無數皆東蠻臨時為之開通【勿鄧豐琶兩林皆東蠻也為于偽翻】若言可塞則是欺罔朝廷要須大度水北更築一城迤邐接黎州【九域志黎州南至大度河一百里宋白曰黎州古沉黎地迤以爾翻邐力紙翻】以大兵守之方可况聞南詔以所掠蜀人二千及金帛賂遺吐蕃【遺唯季翻】若使二虜知蜀虚實連兵入寇誠可深憂其朝臣建言者蓋由禍不在身望人責一狀留入堂案【堂謂政事堂案文案也】他日敗事不可令臣獨當國憲【憲法也敗補邁翻】朝廷皆從其請德裕乃練士卒葺堡鄣積糧儲以備邉蜀人粗安 是歲勃海宣王仁秀卒子新德早死孫彛震立改元咸和<br />
<br />
  五年春正月丁巳賜滄齊節度名義昌軍【張孝忠以程日華為滄州刺史朱淊之亂滄定隔絶日華以滄州自通於朝廷貞元三年以日華為横海軍節度領滄景二州元和十三年王承宗獻德棣二州而横海軍領滄景德棣四州長慶元年省景州明年復領景州太和元年增領齊州明年以棣州隸淄青平盧節度又明年罷横海節度更置齊德節度尋平李同捷得滄州更號滄齊德節度是年賜號義昌軍】 庚申盧龍監軍奏李載義與勑使宴於毬場後院副兵馬使楊志誠與其徒呼噪作亂載義與子正元奔易州志誠又殺莫州刺史張慶初【宋白曰幽州南至莫州二百八十里】上召宰相謀之牛僧孺曰范陽自安史以來非國所有劉總蹔獻其地【事見二百四十一卷穆宗長慶元年】朝廷費錢八十萬緡而無絲毫所獲今日志誠得之猶前日載義得之也【敬宗寶歷二年李載義得范陽事見上卷】因而撫之使捍北狄不必計其逆順上從之載義自易州赴京師上以載義有平滄景之功【平滄景見上三年】且事朝廷恭順二月壬辰以載義為太保同平章事如故以楊志誠為盧龍留後<br />
<br />
  臣光曰昔者聖人順天理察人情知齊民之莫能相治也【治直之翻下同】故置師長以正之知羣臣之莫能相使也故建諸侯以制之知列國之莫能相服也故立天子以統之【自師長而上至天子則所謂師長者近民之官也長知丈翻】天子之於萬國能褒善而黜惡抑彊而扶弱撫服而懲違禁暴而誅亂然後號施令而四海之内莫不率從也【率循也從順也一曰相率而從上之令也】詩曰勉勉我王綱紀四方【詩大雅棫樸之辭】載義藩屏大臣【屏必郢翻】有功於國無罪而志誠逐之此天子所宜治也若一無所問因以其土田爵位授之則是將帥之廢置殺生皆出於士卒之手天子雖在上何為哉國家之有方鎭豈專利其財賦而已乎如僧孺之言姑息偷安之術耳豈宰相佐天子御天下之道哉<br />
<br />
  新羅王彦昇卒子景徽立 上與宋申錫謀誅宦官申錫引吏部侍郎王璠為京兆尹以密旨諭之璠泄其謀【璠孚袁翻 考異曰按舊璠傳去年七月為京兆十二月遷左丞故申錫得罪時京兆尹乃崔琯也】鄭注王守澄知之隂為之備上弟漳王湊賢有人望注令神策都虞候豆盧著誣告申錫謀立漳王戊戌守澄奏之上以為信然甚怒【漳王固上之所忌因其所忌而讒間之此宋申錫之所以不免於罪也】守澄欲即遣二百騎屠申錫家飛龍使馬存亮固争曰如此則京城自亂矣宜召他相與議其事【以馬存亮定張韶之難及争宋申錫之事觀之則温公之取存亮固不特一事也飛龍使掌飛龍廐】守澄乃止是日旬休【一月三句遇旬則下直而休沐謂之句休今謂之旬假是也】遣中使悉召宰相至中書東門中使曰所召無宋公名申錫知獲罪望延英以笏扣頭而退【按閣本大明宫圖中書省與延英殿其間僅隔殿中外院殿中内院耳】宰相至延英上示以守澄所奏相顧愕眙【貽丑吏翻】上命守澄捕豆盧著所告十六宅宫市品官晏敬則及申錫親事王師文等於禁中鞫之【親事常在左右者今宰執侍從猶有親事官】師文亡命三月庚子申錫罷為右庶子自宰相大臣無敢顯言其寃者獨京兆尹崔琯大理卿王正雅連上疏請出内獄付外廷覈實【鞫於禁中故曰内獄】由是獄稍緩正雅翃之子也【王翃見德宗紀】晏敬則等自誣服稱申錫遣王師文逹意於王結異日之知獄成壬寅上悉召師保以下及臺省府寺大臣面詢之午際【午際方交午漏初刻非正午時也】左常侍崔玄亮給事中李固言諫議大夫王質補闕盧鈞舒元褒蔣係裴休韋温等復請對於延英【復扶又翻下同】乞以獄事付外覆按上曰吾已與大臣議之矣屢遣之出不退玄亮叩頭流涕曰殺一匹夫猶不可不重愼况宰相乎上意稍解曰當更與宰相議之乃復召宰相入牛僧孺曰人臣不過宰相今申錫已為宰相假使如所謀復欲何求申錫殆不至此鄭注恐覆案詐覺乃勸守澄請止行貶黜癸卯貶漳王湊為巢縣公宋申錫為開州司馬存亮即日請致仕【存亮之上更有一馬字姓名較明白按馬存亮自以知宋申錫之寃而不能救惡王守澄之横而不能退即日乞身致仕雖宦者而有古人之風】玄亮磁州人質通五世孫【王通見一百七十九卷隋文帝仁夀三年號文中子】係乂之子【蔣乂見二百三十五卷德宗貞元十三年】元褒江州人也晏敬則等坐死及流竄者數十百人申錫竟卒於貶所 夏四月己丑以李載義為山南西道節度使楊志誠為幽州節度使 五月辛丑上以太廟兩室破漏踰年不葺罰將作監支度判官宗正卿俸【將作監掌土木工匠度支掌支調宗正卿掌太廟齋郎宗廟不修故皆罰俸俸扶用翻】亟命中使帥工徒輟禁中營繕之材以葺之【帥讀曰率】左補闕韋温諫以為國家置百官各有所司苟為墮曠【墮讀曰隳】宜黜其人更擇能者代之【更工衡翻】今曠官者止於罰俸而憂軫所切即委内臣是以宗廟為陛下所私而百官皆為虚設也上善其言即追止中使命有司葺之 丙辰西川節度使李德裕奏遣使詣南詔索所掠百姓【索山客翻前年寇蜀所掠者也】得四千人而還 【考異曰德裕西南備邉録曰南詔以所虜男女五千二百六十四人歸於我舊傳曰又遣人入南詔求其所俘工匠得僧道工巧四千餘人復歸成都按實錄云約四千人今從之】 秋八月戊寅以陜虢觀察使崔郾為鄂岳觀察使【陜式冉翻郾於幰翻】鄂岳地囊山帶江處百越巴蜀荆漢之會【處昌呂翻】土多羣盗剽行舟【剽匹妙翻】無老幼必盡殺乃已郾至訓卒治兵作蒙衝追討【蒙衝戰船也治直之翻】歲中悉誅之郾在陜以寛仁為治【治直吏翻下同】或經月不笞一人及至鄂嚴峻刑罰或問其故郾曰陜土瘠民貧吾撫之不暇尚恐其驚鄂地險民雜夷俗慓狡為奸【慓匹妙翻】非用威刑不能致治【治直吏翻】政貴知變蓋謂此也 西川節度使李德裕奏蜀兵羸疾老弱者從來終身不簡臣命立五尺五寸之度簡去四千四百餘人【簡選也去羌呂翻】復簡募少壯者千人以慰其心【復扶又翻少詩照翻】所募北兵已得千五百人與土兵參居【參倉含翻間厠也】轉相訓習日益精練又蜀工所作兵器徒務華飾不堪用臣今取工於别道以治之無不堅利【治直之翻】九月吐蕃維州副使悉怛謀請降【怛當割翻】盡帥其衆奔成都德裕遣行維州刺史虞藏儉將兵入據其城庚申具奏其狀且言欲遣生羌三千燒十三橋擣西戎腹心可洗久耻是韋臯没身恨不能致者也【德宗之時韋臯屢出兵攻維州不能取】事下尚書省集百官議【下戶嫁翻】皆請如德裕策牛僧孺曰吐蕃之境四面各萬里失一維州未能損其勢比來修好約罷戍兵【比毗至翻好呼到翻 考異曰舊僧孺傳載僧孺語曰今吐蕃論董勃纔還劉元鼎未至按穆宗實錄長慶二年八月大理卿劉元鼎使吐蕃囘文宗實錄太和六年三月吐蕃遣論董勃藏入見不言元鼎再奉使杜牧僧孺墓誌亦無董勃等名蓋舊傳誤也】中國禦戎守信為上彼若來責曰何事失信養馬蔚茹川【原州蕭關縣有蔚茹水水西即白草軍蔚紆勿翻】上平凉阪【上時掌翻阪音反】萬騎綴囘中怒氣直辭不三日至咸陽橋此時西南數千里外得百維州何所用之徒棄誠信有害無利此匹夫所不為况天子乎上以為然詔德裕以其城歸吐蕃執悉怛謀及所與偕來者悉歸之吐蕃盡誅之於境上極其慘酷德裕由是怨僧孺益深【為武宗朝李德裕追論維州事張本】 冬十月戊寅李德裕奏南詔寇嶲州陷三縣【嶲音髓】<br />
<br />
  六年春正月壬子詔以水旱降繫囚羣臣上尊號曰太和文武至德皇帝右補闕韋温上疏以為今水旱為災恐非崇飾徽稱之時【稱尺證翻】上善之辭不受 三月辛丑以武寜節度使王智興兼侍中充忠武節度使以邠寜節度使李聽為武寜節度使 囘鶻昭禮可汗為其下所殺從子胡特勒立【從才用翻 考異曰舊傳云七年三月囘鶻李義節等將駞馬到且報可汗二月二十七日薨已冊親弟薩特勒廢朝三日今從新傳】 李聽之前鎭武寜也有蒼頭為牙將【考新傳書李聽前此未嘗鎭武寜切意此蒼頭蓋從聽兄愿素鎭武寜遂得為牙將也】至是聽先遣親吏至徐州慰勞將士【勞力到翻】蒼頭不欲聽復來說軍士【復扶又翻說式芮翻】殺其親吏臠食之聽懼以疾固辭辛酉以前忠武節度使高瑀為武寜節度使夏五月甲辰李德裕奏修卭崍關及移嶲州理臺登<br />
<br />
  城【卭崍關或作卭峽關誤也卭崍關在雅州榮經縣所謂卭崍九折坂王尊叱馭處也祝穆曰卭崍關在嶲州北九十里嶲州先治越嶲縣宋白曰越嶲漢卭都地臺登漢旄牛地李心傳曰卭崍關近榮經去黎州六十里】 秋七月原王逵薨【逵代宗子】 冬十月甲子立魯王永為太子初上以晉王普敬宗長子性謹愿欲以為嗣會薨【晉王普太和二年薨見上卷】上痛惜之故久不議建儲至是始行之 十一月乙卯以荆南節度使段文昌為西川節度使西川監軍王踐言入知樞密數為上言【數所角翻為于偽翻】縛送悉怛謀以快虜心絶後來降者非計也上亦悔之尤中書侍郎同平章事牛僧孺失策附李德裕者因言僧孺與德裕有隙害其功上益踈之【尤者以為愆過也踈者情不相親也】僧孺内不自安會上御延英謂宰相曰天下何時當太平卿等亦有意於此乎【責其尸位素餐無佐理興化之心】僧孺對曰太平無象今四夷不至交侵百姓不至流散雖非至理【至理猶言至治也】亦謂小康【康安也】陛下若别求太平非臣等所及退謂同列曰主上責望如此吾曹豈得久居此地乎因累表罷十二月乙丑以僧孺同平章事充淮南節度使臣光曰君明臣忠上令下從俊良在位邪佞黜遠禮修樂舉刑清政平奸宄消伏【宄音軌】兵革偃戢諸侯順附四夷懷服家給人足此太平之象也于斯之時閽<br />
<br />
  寺專權脇君於内弗能遠也【遠于願翻】藩鎭阻兵陵慢于外弗能制也士卒殺逐主帥拒命自立弗能詰也【詰起吉翻】軍旅歲興賦歛日急【歛力瞻翻】骨血縱横於原野【縱子容翻】杼軸空竭於里閭而僧孺謂之太平不亦誣乎當文宗求治之時僧孺任居承弼進則偷安取容以竊位退則欺君誣世以盗名罪孰大焉【按書冏命旦夕承弼厥辟本不專指宰相温公取翊輔之義遂以為宰相之任又公以進退之道責牛僧孺亦有見於後之竊位盗名如僧孺者治直吏翻】<br />
<br />
  珍王誠薨【新書誠作諴諴德宗子也】 乙亥昭義節度使劉從諫入朝 丁未以前西川節度使李德裕為兵部尚書初李宗閔與德裕有隙【事見二百四十一卷穆宗長慶元年】及德裕還自西川上注意甚厚朝夕且為相宗閔百方沮之不能京兆尹杜悰【沮在呂翻悰徂宗翻】宗閔黨也嘗詣宗閔見其有憂色曰得非以大戎乎【兵部掌戎政尚書其長也故悰隱語謂之大戎】宗閔曰然何以相救悰曰悰有一策可平宿憾恐公不能用宗閔曰何如悰曰德裕有文學而不由科第常用此為慊慊【慊苦簟翻慊慊不快之意】若使之知舉必喜矣【知舉知貢舉也】宗閔默然有間曰【間如字】更思其次悰曰不則用為御史大夫【不讀曰否】宗閔曰此則可矣悰再三與約乃詣德裕德裕迎揖曰公何為訪此寂寥悰曰靖安相公令悰達意【李宗閔蓋居靖安坊因以稱之如後劉崇望居光德坊呼為光德劉公之類】即以大夫之命告之德裕驚喜泣下曰此大門官【唐制大朝會御史大夫帥其屬正百官之班序遲明列於兩觀故以為大門官】小子何足以當之寄謝重沓【重直龍翻】宗閔復與給事中楊虞卿謀之【復扶又翻】事遂中止【牛僧孺患失之心重李德裕進取之心銳所謂楚則失矣齊亦未為得也】虞卿汝士之從弟也【楊汝士見二百四十一卷穆宗長慶元年】<br />
<br />
  七年春正月甲午加昭義節度使劉從諫同平章事遣歸鎭初從諫以忠義自任入朝欲請他鎭既至見朝廷事柄不一又士大夫多請託心輕朝廷 【考異曰補國史曰文宗朝劉從諫朝覲渥澤甚厚自謂河朔近無比倫頗矜臣節文武百辟盡湊其門從諫廣行金帛賂諸權要求登台席人情多可相國李公固言獨無一言從諫欲市其歡玉不可染欲諛其意氷不可穿門館不敢導其誠懇遇休假謁於私第投誠瀝懇至於再三相公正色謂曰僕射先君以東平之功鎭潞二十餘年及即世之後僕射擅領戎務生邀朝命朝廷以先君勲績不絶賞延任居蕃閫位劇南宫豈是恩澤降於等倫欲以何事效忠報國僕射若請邉陲一鎭大展籌謀拓境復疆乃為勲業朝廷豈不以衮職之重命賞封功區區躁求一何容易某比謂僕射英雄忠義首冠蕃臣今求佩相印擁節旄榮歸舊藩亦河朔尋常倔彊之臣所措履也忠節安在深為解體從諫矍然噤口無辭再拜趨出然從諫厚賂倖臣旬日間果以本官加平章事遽辭歸鎭宰相餞於郵亭李相公謂曰相公少年昌盛勉報國恩幸望保家勿殃後嗣從諫以笏扣頭灑淚而辭及至本鎭謂從事將校曰昨者入覲闕庭遍觀朝德唯李公峻直貞明凜然可懼眞社稷之重臣也按固言此年未為相其說妄也今從實錄】故歸而益驕【為劉從諫倔彊張本】 徐州承王智興之後士卒驕悖節度使高瑀不能制【悖蒲妹翻又蒲没翻 考異曰杜牧上崔相公書曰高僕射寛厚聞名不能治軍事舉動汗流拜于堂下此蓋文士筆快耳未必然也】上以為憂甲寅以嶺南節度使崔珙為武寜節度使珙至鎭寛猛適宜徐人安之珙琯之弟也【崔琯見上五年珙居竦翻】 二月癸亥加盧龍節度使檢校工部尚書楊志誠檢校吏部尚書進奏官徐廸【徐廸盧龍進奏官也宋白曰大歷十二年正月勑諸道先置上都留後便宜並改充諸道都知進奏官】詣宰相言軍中不識朝廷之制唯知尚書改僕射為遷不知工部改吏部為美勑使往恐不得出【晉宋以來以吏部尚書為大尚書諸部尚書莫敢比焉唐諸藩進奏官豈不知之徐廸敢詣宰相出是言者直以下陵上替無所忌憚耳勑使不得出言必將拘留之也】辭氣甚慢宰相不以為意 丙戌以兵部尚書李德裕同平章事德裕入謝上與之論朋黨事對曰方今朝士三分之一為朋黨時給事中楊虞卿與從兄中書舍人汝士弟戶部郎中漢公中書舍人張元夫給事中蕭澣等善交結依附權要上干執政下撓有司為士人求官及科第無不如志上聞而惡之【撓奴高翻又奴巧翻為于偽翻惡烏路翻】故與德裕言首及之德裕因得以排其所不悦者【昔人有評牛李事者謂德裕以燕伐燕有味乎其言也】初左散騎常侍張仲方嘗駮李吉甫諡【李吉甫薨有司諡曰敬憲度支郎中張仲方駮其太優憲宗以是貶仲方賜諡曰忠懿宋白曰唐制諸執事官三品已上散官二品已上身亡者佐吏録行狀申考功責歷任勘校下太常寺擬諡訖復申考功都堂集省官議定然後奏聞若藴德丘園聲實明著雖無官爵亦奏賜諡先生諡神至翻】及德裕為相仲方稱疾不出三月壬辰以仲方為賓客分司 楊志誠怒不得僕射留官告使魏寶義并春衣使焦奉鸞送奚契丹使尹士恭【唐中世已後凡藩鎭加官率遣中使奉命謂之官告使焦奉鸞以賜春衣尹士恭以送兩蕃使者同時至幽州故皆為所留】甲午遣牙將王文頴來謝恩并讓官丙申復以告身并批答賜之【自唐以來凡讓官者皆有批答不允復扶又翻下同】文頴不受而去 和王綺薨【綺順宗子】 庚戌以楊虞卿為常州刺史張元夫為汝州刺史【唐以隋毗陵郡置常州京師東南二千八百四十三里隋置伊州於襄城郡後改汝州京師東九百八十二里】他日上復言及朋黨李宗閔曰臣素知之故虞卿輩臣皆不與美官李德裕曰給舍非美官而何【給舍謂給事中中書舍人】宗閔失色丁巳以蕭澣為鄭州刺史【鄭州至京師千一百五里】 夏四月丙戌冊囘鶻新可汗為愛登里囉汨没密施合句祿毗伽彰信可汗 六月乙巳以山南西道節度使李載義為河東節度使先是囘鶻每入貢所過暴掠【先悉薦翻】州縣不敢詰但嚴兵防衛而已載義至鎭囘鶻使者李暢入貢載義謂之曰可汗遣將軍入貢以固舅甥之好【唐公主出嫁囘鶻與為舅甥之國好呼到翻】非遣將軍陵踐上國也將軍不戢部曲使為侵盗【踐慈演翻戢疾立翻】載義亦得殺之勿謂中國之法可忽也於是悉罷防衛兵但使二卒守其門暢畏服不敢犯令 壬申以工部尚書鄭覃為御史大夫初李宗閔惡覃在禁中數言事【惡烏露翻數所角翻】奏罷其侍講【覃自工部侍郎進尚書皆兼翰林侍講學士】上從容謂宰相曰【從于容翻】殷侑經術頗似鄭覃宗閔對曰覃侑經術誠可尚然議論不足聽李德裕曰覃侑議論他人不欲聞惟陛下欲聞之後旬日宣出除覃御史大夫【不由宰相進擬出宣命而除之】宗閔謂樞密使崔潭峻曰事一切宣出安用中書潭峻曰八年天子【上即位至是八年矣】聽其自行事亦可矣宗閔愀然而止【愀七小翻】乙亥以中書侍郎同平章事李宗閔同平章事充山<br />
<br />
  南西道節度使 秋七月壬寅以右僕射王涯同平章事兼度支鹽鐵轉運使 宣武節度使楊元卿有疾朝廷議除代李德裕請徙劉從諫於宣武因拔出上黨不使與山東連結上以為未可癸丑以左僕射李程為宣武節度使 上患近世文士不通經術李德裕請依楊綰議進士試論議不試詩賦【楊綰議見二百二十二卷代宗廣德元年】德裕又言昔玄宗以臨淄王定内難【事見二百九卷睿宗景雲元年難乃旦翻】自是疑忌宗室不令出閤天下議皆以為幽閉骨肉虧傷人倫曏使天寶之末建中之初宗室散處方州【處昌呂翻】雖未能安定王室尚可各全其生所以悉為安祿山朱泚所魚肉者由聚於一宫故也【事並見前紀】陛下誠因冊太子制書聽宗室年高屬踈者出閤且除諸州上佐使擕其男女出外婚嫁此則百年弊法一旦因陛下去之【去羌呂翻】海内孰不欣悦上曰兹事朕久知其不可方今諸王豈無賢才無所施耳八月庚寅冊命太子因下制諸王自今以次出閤授緊望州刺史上佐【開元中定天下州府自京都及諸都督護府外以近畿同華岐蒲為四輔鄭陜汴懷衛絳為六雄宋亳滑許汝晉洺虢魏相為十望又有十緊其後入緊望者浸多凡商寜青汾貝趙襄常宣皆望州也蔡徐鄆楚鄂彭蜀為緊州不及十數又以汝虢鄭汴魏洋蘇為雄蓋升雄望者既多所以緊不及十】十六宅縣主以時出適【出閤而適人使有配偶】進士停試詩賦諸王出閤竟以議所除官不决而罷壬寅加幽州節度使楊志誠檢校右僕射 【考異曰舊傳曰朝廷納裴度言務以含垢下詔諭之因再遣使加尚書右僕射按此時度為襄陽節度使舊傳恐誤今從實錄】仍别遣使慰諭之杜牧憤河朔三鎭之桀驁【驁五到翻】而朝廷議者專事姑息乃作書名曰罪言大畧以為國家自天寶盗起河北百餘城不得尺寸人望之若囘鶻吐蕃無敢窺者齊梁蔡被其風流因亦為寇【齊李正已梁李靈曜蔡李希烈吳氏被皮義翻】未嘗五年間不戰焦焦然七十餘年矣今上策莫如先自治中策莫如取魏最下策為浪戰不計地勢不審攻守是也又傷府兵廢壞作原十六衛以為國家始踵隋制開十六衛自今觀之設官言無謂者其十六衛乎本原事迹其實天下之大命也【唐承隋制開十六衛改左右翊衛曰左右衛府左右驍騎衛曰左右驍衛府左右屯衛曰左右威衛府左右禦衛曰左右領軍衛府左右備身曰領左右府唯左右武衛府左右監門府左右候衛府仍隋不改顯慶五年改左右府曰左右千牛府龍朔二年左右衛府驍衛府武衛府皆省府字左右威衛曰左右武威左右領軍衛曰左右戎衛左右候衛曰左右金吾衛左右監門府曰左右監門衛左右千牛府曰左右奉宸衛後復曰左右千牛衛咸亨元年復改左右戎衛曰領軍衛武后光宅元年改左右驍衛曰左右武威衛左右武衛曰左右鷹揚衛左右威衛曰左右豹韜衛左右領軍衛曰左右玉鈐衛唐初十六衛置大將軍各一人正三品將軍各二人從三品貞元二年十六衛各置上將軍一人從二品雖設官而無兵可掌故當時以為無謂】貞觀中内以十六衛蓄養武臣外開折衝果毅府五百七十四【諸府每府折衝都尉一人上府正四品上中府從四品下下府正五品下左右果毅都尉各一人上府從五品下中府正六品上下府五六品下】以儲兵伍有事則戎臣提兵居外無事則放兵居内其居内也富貴恩澤以奉其身所部之兵散舍諸府【散者分散之散舍者居舍之舍】上府不越千二百人三時耕稼一時治武籍藏將府【治直之翻將即亮翻】伍散田畝力解勢破人人自愛雖有蚩尤為帥亦不可使為亂耳【帥所類翻下同】及其居外也緣部之兵被檄乃來【被皮義翻】斧鉞在前爵賞在後飃暴交捽豈暇異畧【飃即飄字捽昨没翻】雖有蚩尤為帥亦無能為叛也【此所謂實天下之大命也】自貞觀至于開元百三十年間戎臣兵伍未始逆簒此大聖人所以能柄統輕重制鄣表裏聖算神術也至于開元末愚儒奏章曰天下文勝矣請罷府兵武夫奏章曰天下力彊矣請搏四夷於是府兵内剷【字書無剷字今以類求之音楚浪翻】邉兵外作戎臣兵伍湍奔矢往内無一人矣尾大中乾【乾音干】成燕偏重【謂成安祿山偏重之勢也燕於賢翻】而天下掀然根萌燼然七聖旰食【七聖謂肅代德順憲穆敬】求欲除之且不能也由此觀之戎臣兵伍豈可一日使出落鈐鍵哉然為國者不能無兵居外則叛居内則簒使外不叛内不簒古今以還法術最長其置府立衛乎近代以來於其將也弊復為甚【將即亮翻復扶又翻下同】率皆市兒輩多齎金玉負倚幽隂【謂負倚宦官行貨賂以進取也】折劵交貨所能致也絶不識父兄禮義之教復無慷慨感槩之氣百城千里一朝得之其彊傑愎勃者則撓削法制【愎弼力翻撓奴教翻】不使縛已斬族忠良不使違已力一勢便罔不為寇其隂泥巧狡者【泥恐當作昵】亦能家算口歛委於邪倖由卿市公去郡得都【郡謂列郡都謂五都】四履所治指為别館【左傳管仲曰賜我先君履東至于海西至于河南至于穆陵北至于無棣杜預注云履所踐履之界後人言賜履者本此此四履謂四境所至】或一夫不幸而夀則戞割生人畧帀天下【帀作答翻周也】是以天下兵亂不息齊人乾耗【乾音干】靡不由是矣嗚呼文皇帝十六衛之旨其誰原而復之乎【太宗文皇帝】又作戰論以為河北視天下猶珠璣也【言河北不資天下所產以為富】天下視河北猶四支也河北氣俗渾厚果於戰耕加以土息健馬【息生也】便於馳敵是以出則勝處則饒【處昌呂翻】不窺天下之產自可封殖亦猶大農之家不待珠璣然後以為富也國家無河北則精甲銳卒利刀良弓健馬無有也是一支兵去矣河東盟津滑臺大梁彭城東平盡宿厚兵以塞虜衝不可他使是二支兵去矣【河東太原之全軍盟津河陽軍滑臺義成軍大梁宣武軍彭城武寜軍東平天平軍盟讀曰孟塞音悉則翻】六鎭之師厥數三億低首仰給【仰牛向翻】横拱不為【横拱者言横其兩肱拱立而事其帥他無所為也】則沿淮已北循河之南東盡海西叩洛赤地盡取才能應費是三支財去矣【才能之才即纔字漢書作財後人從省便又去貝作才】咸陽西北戎夷大屯【謂自咸陽西北列大屯以防戎夷也】盡剷吳越荆楚之饒以啖兵戍【啖徒濫翻】是四支財去矣天下四支盡解頭腹兀然其能以是久為安乎今者誠能治其五敗則一戰可定四支可生夫天下無事之時殿寄大臣偷安奉私【殿寄大臣謂受殿之寄者牧蓋謂當時節度使也詩采菽殿天子之邦毛氏注云殿鎭也音丁練翻】戰士離落兵甲鈍弊是不蒐練之過其敗一也百人荷戈【荷下可翻】仰食縣官則挾千夫之名大將小禆操其餘贏【小禆謂禆將操七刀翻】以虜壯為幸以師老為娛是執兵者常少糜食常多此不責實料食之過其敗二也戰小勝則張皇其功奔走獻狀以邀上賞或一日再賜或一月累封凱還未歌書品已崇【戰勝則奏凱歌而還書品謂書其官品也還音旋】爵命極矣田宮廣矣【田宮猶言田宅也】金繒溢矣【繒慈陵翻】子孫官矣焉肯搜奇出死勤於我矣此厚賞之過其敗三也【焉於䖍翻】多喪兵士顚飜大都則跳身而來刺邦而去【跳身而來謂逃至京師也刺邦而去謂貶為刺史也喪息浪翻】囘視刀鋸氣色甚安一歲未更【更工衡翻】旋已立於壇墀之上矣【立壇墀之上謂復登大將之壇也】此輕罰之過其敗四也大將兵柄不得專恩臣勑使迭來揮之【恩臣亦指宦官之怙恩者】堂然將陳殷然將鼓一則曰必為偃月一則曰必為魚麗【陳讀曰陣麗力知翻偃月魚麗皆陣名偃月陳中軍偃居其中張兩角向前左傳為魚麗之陳先偏後伍伍承彌縫】三軍萬夫環旋翔羊愰駭之間【翔羊猶云徜徉徘徊也愰呼廣翻】虜騎乘之遂取吾之鼓旗此不專任責成之過其敗五也今者誠欲調持干戈洒掃垢汙而乃踵前非是不可為也又作守論以為今之議皆曰夫倔彊之徒吾以良將勁兵為銜策【倔渠勿翻彊其兩翻銜策所以馭馬】高位美爵充飽其腸安而不撓外而不拘【撓奴巧翻又火高翻】亦猶豢擾虎狼而不拂其心【豢養也擾馴也順也拂讀曰咈】則忿氣不萌此大歷貞元所以守邦也亦何必疾戰焚煎吾民然後以為快也愚曰大歷貞元之間適以此為禍也當是之時有城數十千百卒夫則朝廷别待之貸以法度於是闊視大言自樹一家破制削法角為尊奢天子養威而不問有司守恬而不呵王侯通爵越錄受之【凡賞功者錄其功而加之封爵無功而詔越授之以爵是謂越錄受讀曰授】覲聘不來几杖扶之【言不朝者賜之几杖以安其心】逆息虜胤皇子嬪之【息子也胤繼嗣也河北蕃將之子率多尚主】裝緣采飾無不備之【緣以絹翻】是以地益廣兵益彊僭擬益甚侈心益倡於是土田名器分劃殆盡【劃呼麥翻又音畫】而賊夫貪心未及畔岸遂有淫名越號或帝或王盟詛自立【詛莊助翻】恬淡不畏走兵四畧以飽其志者也是以趙魏燕齊卓起大唱梁蔡吳蜀躡而和之【謂朱淊王武俊田悦李納相立為王李希烈李錡劉闢繼亂也和戶卧翻】其餘混澒軒囂【澒戶孔翻】欲相效者往往而是運遭孝武【謂憲宗】宵旰不忘【宵宵衣也謂未明求衣也旰旰食也謂日旰而食也】前英後傑夕思朝議故能大者誅鋤小者惠來不然周秦之郊幾為犯獵哉【周秦之郊謂河南關内也】大抵生人油然多欲欲而不得則怒怒則争亂隨之是以教笞於家刑罰於國征伐於天下此所以裁其欲而塞其争也大歷貞元之間盡反此道提區區之有而塞無涯之争【區區之有謂朝廷爵命塞悉則翻】是以首尾指支幾不能相運掉也【幾居於翻掉徒弔翻】今者不知非此而反用以為經【經常也】愚見為盗者非止於河北而已嗚呼大歷貞元守邦之術永戒之哉又注孫子為之序以為兵者刑也【大刑用甲兵】刑者政事也為夫子之徒實仲由冉有之事也不知自何代何人分為二道曰文武離而俱行因使縉紳之士不敢言兵或耻言之苟有言者世以為粗暴異人人不比數嗚呼亡失根本斯最為甚禮曰四郊多壘此卿大夫之辱也【記曲禮之言】歷觀自古樹立其國㓕亡其國未始不由兵也主兵者必聖賢材能多聞博識之士乃能有功議於廊廟之上兵形已成然後付之於將【將即亮翻】漢祖言指蹤者人也【指蹤謂指示獸蹤此與漢書因文取義小不同】獲兎者犬也此其是也彼為相者曰兵非吾事吾不當知君子曰勿居其位可也【魏温公取杜牧此語則其平時講明相業可以見矣】 前邠寜行軍司馬鄭注依倚王守澄權勢燻灼上深惡之【惡烏露翻】九月丙寅侍御史李款閤内奏彈注内通勑使外連朝士兩地往來【兩地謂往來南牙北司間也使疏吏翻朝直遥翻】卜射財賄晝伏夜動干竊化權人不敢言道路以目請付法司旬日之間章數十上【上時掌翻】守澄匿注於右軍【王守澄時為右軍中尉故得以匿注】左軍中尉韋元素樞密使楊承和王踐言皆惡注【惡烏露翻】左軍將李弘楚說元素曰【說式芮翻】鄭注奸猾無雙卵不除【苦角翻鳥子未出者】使成羽翼必為國患今因御史所劾匿軍中弘楚請以中尉意詐為有疾召使治之來則中尉延與坐弘楚侍側伺中尉舉目擒出杖殺之【伺相吏翻】中尉因見上叩頭請罪具言其奸楊王必助中尉進言【楊王謂楊承和王踐言也】况中尉有翼戴之功【元和末穆宗立韋元素亦以預有定策之功矣】豈以除奸而獲罪乎元素以為然召之注至蠖屈鼠伏【蠖烏郭翻易大傳曰尺蠖之屈以求伸也爾雅注蠖行若今以指步尺屈而後伸】佞辭泉湧元素不覺執手款曲諦聽忘倦【諦都計翻審也】弘楚詗伺再三元素不顧以金帛厚遺注而遣之【詗火迥翻又翾正翻遺唯季翻】弘楚怒曰中尉失今日之斷必不免他日之禍矣【斷丁亂翻為元素為注所去張本】因解軍職去頃之疽發背卒王涯之為相注有力焉【是必因注以結王守澄也】且畏王守澄遂寢李款之奏守澄言注於上而釋之尋奏為侍御史充右神策判官 【考異曰開成紀事曰五年金吾將軍孟文亮出鎭邠郊以與注姻懿之故奏為軍司馬路經奉天防遏使御史大夫王從亮薄其為人不為之禮注毁從亮於守澄竟為守澄誣構决杖投荒未幾文亮没罷職還城守澄濳置為軍晝時澤潞劉從諫本欲誅注忌其權勢因辟為節度副使纔至潞州涉旬之間會上乖愈太和七年十一月驛徵之赴闕偶遭其時聖體獲愈上悦之自此恩寵漸隆凡臺省府縣軍戎莫不從風七年九月十三日侍御史李欵彈注内通勑使外連朝臣兩地往來卜射財貨晝伏夜動干竊化權人不敢言道路以目城社轉回恐為禍胎罪不容誅理合顯戮其鄭注請付有司時王涯重處台司注之所致又慮守澄黨援遂寢不行注潛遁軍司矣李德裕文武兩朝獻替記曰八年春暮上對宰相歎天下無名醫便及鄭注精於服食或欲置於翰林技術院或欲令為左神策判官注自稱衣冠皆不願此職守澄遂託從諫奏為行軍司馬及赴職宗閔又自山南令判官楊儉至澤潞與從諫要約令却薦入今從實錄】朝野駭歎 甲寅以前忠武節度使王智興為河中節度使 羣臣以上即位八年未受尊號冬十二月甲午上尊號曰太和文武仁聖皇帝會有五坊中使薛季稜自同華還【同華同州華州華戶化翻還音旋】言閭閻彫弊上歎曰關中小稔百姓尚爾况江淮比年大水其人如何【比毗至翻】吾無術以救之敢崇虚名乎因以通天帶賞季稜【通天犀帶也】羣臣凡四上表竟不受 庚子上始得風疾不能言於是王守澄薦昭義行軍司馬鄭注善醫上徵注至京師飲其藥頗有驗遂有寵【甘露之禍胎成矣】<br />
<br />
  資治通鑑卷二百四十四  <br>
   </div> 

<script src="/search/ajaxskft.js"> </script>
 <div class="clear"></div>
<br>
<br>
 <!-- a.d-->

 <!--
<div class="info_share">
</div> 
-->
 <!--info_share--></div>   <!-- end info_content-->
  </div> <!-- end l-->

<div class="r">   <!--r-->



<div class="sidebar"  style="margin-bottom:2px;">

 
<div class="sidebar_title">工具类大全</div>
<div class="sidebar_info">
<strong><a href="http://www.guoxuedashi.com/lsditu/" target="_blank">历史地图</a></strong>  
<a href="http://www.880114.com/" target="_blank">英语宝典</a>  
<a href="http://www.guoxuedashi.com/13jing/" target="_blank">十三经检索</a> 
<br><strong><a href="http://www.guoxuedashi.com/gjtsjc/" target="_blank">古今图书集成</a></strong> 
<a href="http://www.guoxuedashi.com/duilian/" target="_blank">对联大全</a> <strong><a href="http://www.guoxuedashi.com/xiangxingzi/" target="_blank">象形文字典</a></strong> 

<br><a href="http://www.guoxuedashi.com/zixing/yanbian/">字形演变</a>  <strong><a href="http://www.guoxuemi.com/hafo/" target="_blank">哈佛燕京中文善本特藏</a></strong>
<br><strong><a href="http://www.guoxuedashi.com/csfz/" target="_blank">丛书&方志检索器</a></strong> <a href="http://www.guoxuedashi.com/yqjyy/" target="_blank">一切经音义</a>  

<br><strong><a href="http://www.guoxuedashi.com/jiapu/" target="_blank">家谱族谱查询</a></strong>  <strong><a href="http://shufa.guoxuedashi.com/sfzitie/" target="_blank">书法字帖欣赏</a></strong> 
<br>

</div>
</div>


<div class="sidebar" style="margin-bottom:0px;">

<font style="font-size:22px;line-height:32px">QQ交流群9:489193090</font>


<div class="sidebar_title">手机APP 扫描或点击</div>
<div class="sidebar_info">
<table>
<tr>
	<td width=160><a href="http://m.guoxuedashi.com/app/" target="_blank"><img src="/img/gxds-sj.png" width="140"  border="0" alt="国学大师手机版"></a></td>
	<td>
<a href="http://www.guoxuedashi.com/download/" target="_blank">app软件下载专区</a><br>
<a href="http://www.guoxuedashi.com/download/gxds.php" target="_blank">《国学大师》下载</a><br>
<a href="http://www.guoxuedashi.com/download/kxzd.php" target="_blank">《汉字宝典》下载</a><br>
<a href="http://www.guoxuedashi.com/download/scqbd.php" target="_blank">《诗词曲宝典》下载</a><br>
<a href="http://www.guoxuedashi.com/SiKuQuanShu/skqs.php" target="_blank">《四库全书》下载</a><br>
</td>
</tr>
</table>

</div>
</div>


<div class="sidebar2">
<center>


</center>
</div>

<div class="sidebar"  style="margin-bottom:2px;">
<div class="sidebar_title">网站使用教程</div>
<div class="sidebar_info">
<a href="http://www.guoxuedashi.com/help/gjsearch.php" target="_blank">如何在国学大师网下载古籍?</a><br>
<a href="http://www.guoxuedashi.com/zidian/bujian/bjjc.php" target="_blank">如何使用部件查字法快速查字?</a><br>
<a href="http://www.guoxuedashi.com/search/sjc.php" target="_blank">如何在指定的书籍中全文检索?</a><br>
<a href="http://www.guoxuedashi.com/search/skjc.php" target="_blank">如何找到一句话在《四库全书》哪一页?</a><br>
</div>
</div>


<div class="sidebar">
<div class="sidebar_title">热门书籍</div>
<div class="sidebar_info">
<a href="/so.php?sokey=%E8%B5%84%E6%B2%BB%E9%80%9A%E9%89%B4&kt=1">资治通鉴</a> <a href="/24shi/"><strong>二十四史</strong></a>&nbsp; <a href="/a2694/">野史</a>&nbsp; <a href="/SiKuQuanShu/"><strong>四库全书</strong></a>&nbsp;<a href="http://www.guoxuedashi.com/SiKuQuanShu/fanti/">繁体</a>
<br><a href="/so.php?sokey=%E7%BA%A2%E6%A5%BC%E6%A2%A6&kt=1">红楼梦</a> <a href="/a/1858x/">三国演义</a> <a href="/a/1038k/">水浒传</a> <a href="/a/1046t/">西游记</a> <a href="/a/1914o/">封神演义</a>
<br>
<a href="http://www.guoxuedashi.com/so.php?sokeygx=%E4%B8%87%E6%9C%89%E6%96%87%E5%BA%93&submit=&kt=1">万有文库</a> <a href="/a/780t/">古文观止</a> <a href="/a/1024l/">文心雕龙</a> <a href="/a/1704n/">全唐诗</a> <a href="/a/1705h/">全宋词</a>
<br><a href="http://www.guoxuedashi.com/so.php?sokeygx=%E7%99%BE%E8%A1%B2%E6%9C%AC%E4%BA%8C%E5%8D%81%E5%9B%9B%E5%8F%B2&submit=&kt=1"><strong>百衲本二十四史</strong></a>  <a href="http://www.guoxuedashi.com/so.php?sokeygx=%E5%8F%A4%E4%BB%8A%E5%9B%BE%E4%B9%A6%E9%9B%86%E6%88%90&submit=&kt=1"><strong>古今图书集成</strong></a>
<br>

<a href="http://www.guoxuedashi.com/so.php?sokeygx=%E4%B8%9B%E4%B9%A6%E9%9B%86%E6%88%90&submit=&kt=1">丛书集成</a> 
<a href="http://www.guoxuedashi.com/so.php?sokeygx=%E5%9B%9B%E9%83%A8%E4%B8%9B%E5%88%8A&submit=&kt=1"><strong>四部丛刊</strong></a>  
<a href="http://www.guoxuedashi.com/so.php?sokeygx=%E8%AF%B4%E6%96%87%E8%A7%A3%E5%AD%97&submit=&kt=1">說文解字</a> <a href="http://www.guoxuedashi.com/so.php?sokeygx=%E5%85%A8%E4%B8%8A%E5%8F%A4&submit=&kt=1">三国六朝文</a>
<br><a href="http://www.guoxuedashi.com/so.php?sokeytm=%E6%97%A5%E6%9C%AC%E5%86%85%E9%98%81%E6%96%87%E5%BA%93&submit=&kt=1"><strong>日本内阁文库</strong></a> <a href="http://www.guoxuedashi.com/so.php?sokeytm=%E5%9B%BD%E5%9B%BE%E6%96%B9%E5%BF%97%E5%90%88%E9%9B%86&ka=100&submit=">国图方志合集</a> <a href="http://www.guoxuedashi.com/so.php?sokeytm=%E5%90%84%E5%9C%B0%E6%96%B9%E5%BF%97&submit=&kt=1"><strong>各地方志</strong></a>

</div>
</div>


<div class="sidebar2">
<center>

</center>
</div>
<div class="sidebar greenbar">
<div class="sidebar_title green">四库全书</div>
<div class="sidebar_info">

《四库全书》是中国古代最大的丛书,编撰于乾隆年间,由纪昀等360多位高官、学者编撰,3800多人抄写,费时十三年编成。丛书分经、史、子、集四部,故名四库。共有3500多种书,7.9万卷,3.6万册,约8亿字,基本上囊括了古代所有图书,故称“全书”。<a href="http://www.guoxuedashi.com/SiKuQuanShu/">详细>>
</a>

</div> 
</div>

</div>  <!--end r-->

</div>
<!-- 内容区END --> 

<!-- 页脚开始 -->
<div class="shh">

</div>

<div class="w1180" style="margin-top:8px;">
<center><script src="http://www.guoxuedashi.com/img/plus.php?id=3"></script></center>
</div>
<div class="w1180 foot">
<a href="/b/thanks.php">特别致谢</a> | <a href="javascript:window.external.AddFavorite(document.location.href,document.title);">收藏本站</a> | <a href="#">欢迎投稿</a> | <a href="http://www.guoxuedashi.com/forum/">意见建议</a> | <a href="http://www.guoxuemi.com/">国学迷</a> | <a href="http://www.shuowen.net/">说文网</a><script language="javascript" type="text/javascript" src="https://js.users.51.la/17753172.js"></script><br />
  Copyright &copy; 国学大师 古典图书集成 All Rights Reserved.<br>
  
  <span style="font-size:14px">免责声明:本站非营利性站点,以方便网友为主,仅供学习研究。<br>内容由热心网友提供和网上收集,不保留版权。若侵犯了您的权益,来信即刪。scp168@qq.com</span>
  <br />
ICP证:<a href="http://www.beian.miit.gov.cn/" target="_blank">鲁ICP备19060063号</a></div>
<!-- 页脚END --> 
<script src="http://www.guoxuedashi.com/img/plus.php?id=22"></script>
<script src="http://www.guoxuedashi.com/img/tongji.js"></script>

</body>
</html>
