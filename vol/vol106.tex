<!DOCTYPE html PUBLIC "-//W3C//DTD XHTML 1.0 Transitional//EN" "http://www.w3.org/TR/xhtml1/DTD/xhtml1-transitional.dtd">
<html xmlns="http://www.w3.org/1999/xhtml">
<head>
<meta http-equiv="Content-Type" content="text/html; charset=utf-8" />
<meta http-equiv="X-UA-Compatible" content="IE=Edge,chrome=1">
<title>資治通鑒_107-資治通鑑卷一百六_107-資治通鑑卷一百六</title>
<meta name="Keywords" content="資治通鑒_107-資治通鑑卷一百六_107-資治通鑑卷一百六">
<meta name="Description" content="資治通鑒_107-資治通鑑卷一百六_107-資治通鑑卷一百六">
<meta http-equiv="Cache-Control" content="no-transform" />
<meta http-equiv="Cache-Control" content="no-siteapp" />
<link href="/img/style.css" rel="stylesheet" type="text/css" />
<script src="/img/m.js?2020"></script> 
</head>
<body>
 <div class="ClassNavi">
<a  href="/24shi/">二十四史</a> | <a href="/SiKuQuanShu/">四库全书</a> | <a href="http://www.guoxuedashi.com/gjtsjc/"><font  color="#FF0000">古今图书集成</font></a> | <a href="/renwu/">历史人物</a> | <a href="/ShuoWenJieZi/"><font  color="#FF0000">说文解字</a></font> | <a href="/chengyu/">成语词典</a> | <a  target="_blank"  href="http://www.guoxuedashi.com/jgwhj/"><font  color="#FF0000">甲骨文合集</font></a> | <a href="/yzjwjc/"><font  color="#FF0000">殷周金文集成</font></a> | <a href="/xiangxingzi/"><font color="#0000FF">象形字典</font></a> | <a href="/13jing/"><font  color="#FF0000">十三经索引</font></a> | <a href="/zixing/"><font  color="#FF0000">字体转换器</font></a> | <a href="/zidian/xz/"><font color="#0000FF">篆书识别</font></a> | <a href="/jinfanyi/">近义反义词</a> | <a href="/duilian/">对联大全</a> | <a href="/jiapu/"><font  color="#0000FF">家谱族谱查询</font></a> | <a href="http://www.guoxuemi.com/hafo/" target="_blank" ><font color="#FF0000">哈佛古籍</font></a> 
</div>

 <!-- 头部导航开始 -->
<div class="w1180 head clearfix">
  <div class="head_logo l"><a title="国学大师官网" href="http://www.guoxuedashi.com" target="_blank"></a></div>
  <div class="head_sr l">
  <div id="head1">
  
  <a href="http://www.guoxuedashi.com/zidian/bujian/" target="_blank" ><img src="http://www.guoxuedashi.com/img/top1.gif" width="88" height="60" border="0" title="部件查字,支持20万汉字"></a>


<a href="http://www.guoxuedashi.com/help/yingpan.php" target="_blank"><img src="http://www.guoxuedashi.com/img/top230.gif" width="600" height="62" border="0" ></a>


  </div>
  <div id="head3"><a href="javascript:" onClick="javascript:window.external.AddFavorite(window.location.href,document.title);">添加收藏</a>
  <br><a href="/help/setie.php">搜索引擎</a>
  <br><a href="/help/zanzhu.php">赞助本站</a></div>
  <div id="head2">
 <a href="http://www.guoxuemi.com/" target="_blank"><img src="http://www.guoxuedashi.com/img/guoxuemi.gif" width="95" height="62" border="0" style="margin-left:2px;" title="国学迷"></a>
  

  </div>
</div>
  <div class="clear"></div>
  <div class="head_nav">
  <p><a href="/">首页</a> | <a href="/ShuKu/">国学书库</a> | <a href="/guji/">影印古籍</a> | <a href="/shici/">诗词宝典</a> | <a   href="/SiKuQuanShu/gxjx.php">精选</a> <b>|</b> <a href="/zidian/">汉语字典</a> | <a href="/hydcd/">汉语词典</a> | <a href="http://www.guoxuedashi.com/zidian/bujian/"><font  color="#CC0066">部件查字</font></a> | <a href="http://www.sfds.cn/"><font  color="#CC0066">书法大师</font></a> | <a href="/jgwhj/">甲骨文</a> <b>|</b> <a href="/b/4/"><font  color="#CC0066">解密</font></a> | <a href="/renwu/">历史人物</a> | <a href="/diangu/">历史典故</a> | <a href="/xingshi/">姓氏</a> | <a href="/minzu/">民族</a> <b>|</b> <a href="/mz/"><font  color="#CC0066">世界名著</font></a> | <a href="/download/">软件下载</a>
</p>
<p><a href="/b/"><font  color="#CC0066">历史</font></a> | <a href="http://skqs.guoxuedashi.com/" target="_blank">四库全书</a> |  <a href="http://www.guoxuedashi.com/search/" target="_blank"><font  color="#CC0066">全文检索</font></a> | <a href="http://www.guoxuedashi.com/shumu/">古籍书目</a> | <a   href="/24shi/">正史</a> <b>|</b> <a href="/chengyu/">成语词典</a> | <a href="/kangxi/" title="康熙字典">康熙字典</a> | <a href="/ShuoWenJieZi/">说文解字</a> | <a href="/zixing/yanbian/">字形演变</a> | <a href="/yzjwjc/">金 文</a> <b>|</b>  <a href="/shijian/nian-hao/">年号</a> | <a href="/diming/">历史地名</a> | <a href="/shijian/">历史事件</a> | <a href="/guanzhi/">官职</a> | <a href="/lishi/">知识</a> <b>|</b> <a href="/zhongyi/">中医中药</a> | <a href="http://www.guoxuedashi.com/forum/">留言反馈</a>
</p>
  </div>
</div>
<!-- 头部导航END --> 
<!-- 内容区开始 --> 
<div class="w1180 clearfix">
  <div class="info l">
   
<div class="clearfix" style="background:#f5faff;">
<script src='http://www.guoxuedashi.com/img/headersou.js'></script>

</div>
  <div class="info_tree"><a href="http://www.guoxuedashi.com">首页</a> > <a href="/SiKuQuanShu/fanti/">四库全书</a>
 > <h1>资治通鉴</h1> <!--         下载:【右键另存为】即可 --></div>
  <div class="info_content zj clearfix">
  
<div class="info_txt clearfix" id="show">
<center style="font-size:24px;">107-資治通鑑卷一百六</center>
    資治通鑑卷一百六   宋 司馬光 撰<br />
<br />
  胡三省 音註<br />
<br />
  晉紀二十八【起旃蒙作噩盡柔兆閹茂凡二年】<br />
<br />
  烈宗孝武皇帝中之上<br />
<br />
  太元十年春正月秦王堅朝饗羣臣【朝直遥翻】時長安饑人相食諸將歸吐肉以飼妻子【窮匱如此外無救援烏得不敗乎飼祥吏翻】慕容冲即皇帝位於阿房【是為西燕】改元更始冲有自得之志賞罸任情慕容盛年十三謂慕容柔曰夫十人之長【長知兩翻】亦須才過九人然後得安今中山王才不逮人功未有成而驕汰已甚殆難濟乎【冲在前燕時封中山王汰侈也溢也史言慕容盛幼而有識畧所以能自奮而有國盛柔歸冲見上卷上年】 後秦王萇留諸將攻新平自引兵撃安定擒秦安西將軍勃海公珍嶺北諸城悉降之【降戶江翻】 甲寅秦王堅與西燕主冲戰于仇班渠大破之【慕容垂復興于山東而冲稱號于關中故書西燕以别之】乙卯戰于雀桑又破之甲子戰于白渠【白渠即漢時白公所鑿者也】秦兵大敗西燕兵圍秦王堅殿中將軍鄧邁力戰却之堅乃得免壬申冲遣尚書令高蓋夜襲長安入其南城左將軍竇衝前禁將軍李辯等擊破之斬首八百級分其屍而食之乙亥高蓋引兵攻渭北諸壘太子宏與戰於成貳壁【成貳蓋人姓名關中大亂立壁自保因為地名】大破之斬首三萬 燕帶方王佐與寧朔將軍平規共攻薊【薊音計】王永兵屢敗二月永使宋敞燒和龍及薊城宫室帥衆三萬奔壺關【帥讀曰率】佐等入薊慕容農引兵會慕容麟於中山與共攻翟真麟農先帥數千騎至承營【帥讀曰率下同】觀察形勢翟真望見陳兵而出諸將欲退農曰丁零非不勁勇而翟真懦弱今簡精鋭望真所在而衝之真走衆必散矣乃邀門而蹙之可盡殺也使驍騎將軍慕容國帥百餘騎衝之【驍堅堯翻騎奇寄翻】真走其衆争門自相蹈藉【藉慈夜翻】死者太半遂拔承營外郭 癸未秦王堅與西燕主冲戰於城西【長安城西也】大破之追奔至阿城【阿城即阿房宫城冲之巢穴也】諸將請乘勝入城堅恐為冲所掩引兵還【萬乘之主固不可乘危徼幸然秦喪敗若此乘諸將之勝氣以圖萬一之功可也引兵而還何歟還從宣翻】 乙酉秦益州刺史王廣以蜀人江陽太守李丕為益州刺史守成都【沈約曰江陽郡劉璋分犍為立守式又翻】己丑廣帥所部奔還隴西蜀人隨之者三萬餘人 劉牢之至枋頭楊膺姜讓謀泄【膺讓謀見上卷上年】長樂公丕收殺之【枋音方樂音洛】牢之聞之盤不進 秦平原悼公暉數為西燕主冲所敗【數所角翻敗補遇翻】秦王堅讓之曰汝吾之才子也擁大衆與白虜小兒戰而屢敗何用生為三月暉憤恚自殺【堅責怒暉欲其死戰耳豈意其自殺哉恚于避翻】前禁將軍李辯都水使者隴西彭和正恐長安不守召集西州人【晉書職官志都水長屬大司農沈約志都水使者掌舟航及運部李辯李儼之子亦隴西人也】屯於韮園堅召之不至【為後堅襲韮園張本韮舉有翻】 西燕主冲攻秦高陽愍公方于驪山殺之【苻方戍驪山見上卷上年七月】執秦尚書韋鍾以其子謙為馮翊太守使招集三輔之民馮翊壘主邵安民等責謙曰君雍州望族【七相五公雍州之望族鍾蓋韋賢後也雍于用翻】今乃從賊與之為不忠不義何面目以行於世乎謙以告鍾鍾自殺謙來奔秦左將軍苟池右將軍俱石子與西燕主冲戰於驪山兵敗西燕將軍慕容永斬苟池俱石子奔鄴永廆弟運之孫石子難之弟也【俱難見一百四卷太元三年廆戶罪翻】秦王堅遣領軍將軍楊定擊冲大破之虜鮮卑萬餘人而還悉阬之定佛奴之孫也【北史曰定佛奴之子佛奴宋奴之子也】 滎陽人鄭燮以郡來降【降戶江翻】 燕王垂攻鄴久不下將北詣冀州乃命撫軍大將軍麟屯信都樂浪王温屯中山召驃騎大將軍農還鄴【樂浪音洛琅驃匹妙翻騎奇寄翻】於是遠近聞之以燕為不振頗懷去就農至高邑【高邑本鄗縣漢光武即位于此改曰高邑屬常山晉志屬趙國】遣從事中郎眭邃近出違期不還【師古曰眭息隨翻】長史張攀言於農曰邃目下參佐【目下參佐言其近在眼前也】敢欺罔不還請回軍討之農不應敕備假板以邃為高陽太守參佐家在趙北者悉假署遣歸【趙北趙國以北也假署者權時以假板署置其官未以白燕王垂也】凡舉補太守三人長史二十餘人退謂攀曰君所見殊誤當今豈可自相魚肉俟吾北還邃等自當迎于道左君但觀之【為後邃等迎農張本】樂浪王温在中山兵力甚弱丁零四布分據諸城温謂諸將曰以吾之衆攻則不足守則有餘驃騎撫軍首尾連兵會須滅賊但應聚糧厲兵以俟時耳於是撫舊招新勸課農桑民歸附者相繼郡縣壁壘争送軍糧倉庫充溢翟真夜襲中山温擊破之自是不敢復至【復扶又翻下復還同】温乃遣兵一萬運糧以餉垂且營中山宫室【欲迎垂都中山也】劉牢之攻燕黎陽太守劉撫于孫就柵【孫就人姓名盖立柵于黎陽界劉憮因屯焉】燕王垂留慕容農守鄴圍自引兵救之秦長樂公丕聞之出兵乘虚夜襲燕營農擊敗之【敗補邁翻】劉牢之與垂戰不勝退屯黎陽垂復還鄴 呂光以龜兹饒樂【龜兹音丘慈樂音洛】欲留居之天竺沙門鳩摩羅什謂光曰此凶亡之地不足留也【據載記鳩摩羅姓也什其名】將軍但東歸中道自有福地可居【鳩摩羅什知數知呂光必得涼州之地而據之】光乃大饗將士議進止衆皆欲還乃以駝二萬餘頭載外國珍寶奇玩驅駿馬萬餘匹而還【還音旋又如字】 夏四月劉牢之進兵至鄴燕王垂逆戰而敗遂撤圍退屯新城乙卯自新城北遁牢之不告秦長樂公丕即引兵追之丕聞之發兵繼進庚申牢之追及垂於董唐淵垂曰秦晉瓦合相待為彊【瓦合言其勢不膠固觸而動之一瓦墜碎則衆瓦俱解矣待當作恃今觀待字義亦自通】一勝則俱豪一失則俱潰非同心也今兩軍相繼勢既未合宜急擊之牢之軍疾趨二百里至五橋澤【五橋澤在臨漳縣北兵法百里而趨利者蹶上將况二百里乎】争燕輜重【重直用翻】垂邀撃大破之斬首數千級牢之單馬走會秦救至得免燕冠軍將軍宜都王鳳【冠古玩翻】每戰奮不顧身前後大小二百五十七戰未嘗無功垂戒之曰今大業甫濟汝當先自愛使為車騎將軍德之副以抑其鋭【以德持重也】鄴中饑甚長樂公丕帥衆就晉穀於枋頭【帥讀曰率】劉牢之入鄴城收集亡散兵復少振【復扶又翻少詩沼翻】坐軍敗徵還燕秦相持經年【去年正月垂攻鄴】幽冀大饑人相食邑落蕭條燕之軍士多餓死燕王垂禁民養蠶以桑椹為軍粮垂將北趣中山【趣七喻翻】以驃騎大將軍農為前驅前所假授吏眭邃等皆來迎候上下如初李攀乃服農之智略【前作張攀此作李攀未知孰是】 會稽王道子好專權【好呼到翻】復為姦謟者所搆扇【復扶又翻下同】與太保安有隙安欲避之會秦王堅來求救安乃請自將救之【將即亮翻】壬戌出鎮廣陵之步丘築壘曰新城而居之【今揚州邵伯鎮即其地也在揚州城北六十里晉史云安于步丘築埭後人謂之召伯埭】 蜀郡太守任權攻拔成都斬秦益州刺史李丕復取益州【秦取益州見一百三卷寧康元年任音壬】 新平糧竭矢盡【後秦自去年攻新平】外救不至後秦王萇使人謂苟輔曰吾方以義取天下豈讐忠臣邪卿但帥城中之人還長安吾止欲得此城耳輔以為然帥民五千口出城萇圍而阬之男女無遺【萇仲良翻帥讀曰率】獨馮傑子終得脱奔長安【馮傑勸輔拒後秦事見上卷上年】秦王堅追贈輔等官爵皆諡曰節愍侯以終為新平太守 翟真自承營徙屯行唐【即漢之南行唐縣也屬常山郡燕王垂趣中山真為所逼故徙屯】真司馬鮮于乞殺真及諸翟自立為趙王營人共殺乞立真從弟成為主其衆多降於燕【降戶江翻】 五月西燕主冲攻長安秦王堅身自督戰飛矢滿體流血淋漓冲縱兵暴掠關中士民流散道路斷絶千里無煙有堡壁三十餘推平遠將軍趙敖為主【安遠平遠二將軍號盖皆當時所置】相與結盟冒難遣兵粮助堅多為西燕所殺堅謂之曰聞來者率不善逹此誠忠臣之義然今寇難殷繁【難乃旦翻殷衆也盛也繁多也】非一人之力所能濟也徒相隨入虎口何益汝曹宜為國自愛【為于偽翻】畜糧厲兵以俟天時庶幾善不終否有時而泰也【幾居希翻否皮鄙翻】三輔之民為冲所畧者遣人密告堅請遣兵攻冲欲縱火為内應堅曰甚哀諸卿忠誠然吾猛士如虎豹利兵如霜雪困於烏合之虜豈非天乎恐徒使諸卿坐致夷滅吾不忍也其人固請不已乃遣七百騎赴之冲營縱火者反為風火所燒其得免者什一二堅祭而哭之【史言關中之人乃心為堅而力不能濟盖天棄秦也】衛將軍楊定與冲戰於城西為冲所擒定秦之驍將也【驍堅堯翻將即亮翻】堅大懼以䜟書云帝出五將久長得【據載記此䜟書謂之古符傳賈錄秦王堅始也禁人學䜟及喪敗之極乃欲用䜟書奔五將山以求免其顛倒錯繆甚矣盖死期將至也䜟楚譛翻】乃留太子宏守長安謂之曰天其或者欲導予出外汝善守城勿與賊争利吾當出隴收兵運糧以給汝遂帥騎數百與張夫人及中山公詵二女寶錦出奔五將山【新唐書地理志京兆醴泉縣有武將山水經注扶風杜陽縣有五將山又按唐杜佑鳳翔府岐山縣有五將山】宣告州郡期以孟冬救長安堅過襲韮園李辯奔燕【奔西燕也】彭和正慙自殺 閏月以廣州刺史羅友為益州刺史鎮成都 庚戌燕王垂至常山圍翟成於行唐命帶方王佐鎮龍城六月高句麗寇遼東【句如字又音駒麗力知翻】佐遣司馬郝景將兵救之為高句麗所敗【敗補邁翻】高句麗遂陷遼東玄菟【自此燕不能勝高句麗菟同都翻】 秦太子宏不能守長安將數千騎與母妻宗室西奔下辨【辨皮莧翻】百官逃散司隸校尉權翼等數百人奔後秦【權翼本姚襄僚屬苻氏既敗故奔後秦】西燕主冲入據長安縱兵大掠死者不可勝計【勝音升】 秋七月旱饑井皆竭 後秦王萇自故縣如新平【漢安定郡有安定縣後漢晉省故曰故縣】 秦王堅至五將山後秦王萇遣驍騎將軍吳忠帥騎圍之秦兵皆散走獨侍御十數人在側堅神色自若坐而待之召宰人進食俄而忠至執之送詣新平幽於别室太子宏至下辨南秦州刺史楊璧拒之【苻堅破仇池置南秦州楊璧氐之種類仕秦尚主任居方面以宏奔敗拒而不納孟子曰寡助之至親戚叛之信哉斯言】璧妻堅之女順陽公主也棄其夫從宏宏奔武都投氐豪強熙【強其兩翻】假道來奔詔處之江州【為苻宏附桓玄而誅張本處昌呂翻】長樂公丕帥衆三萬自枋頭將歸鄴城龍驤將軍檀玄擊之戰於谷口【檀玄晉將也谷口在枋頭西】玄兵敗丕復入鄴城【復扶又翻】 燕建節將軍餘巖叛自武邑北趣幽州【趣七喻翻】燕王垂馳使敕幽州將平規曰【使疏吏翻】固守勿戰俟吾破丁零自討之規出戰為巖所敗【敗補邁翻】巖入薊掠千餘戶而去遂據令支【令音鈴又郎定翻支音祁】癸酉翟成長史鮮于得斬成出降垂屠行唐盡阬成衆【去年丁零叛燕至是而滅降戶江翻】 太保安有疾求還【安屯廣陵步丘】詔許之八月安至建康 甲午大赦 丁酉建昌文靖公謝安薨【諡法柔德安衆曰靖寛樂令終曰靖】詔加殊禮如大司馬温故事【晉以此崇寵謝安安之雅志豈願與桓温同哉】庚子以司徒琅邪王道子領揚州刺史録尚書都督中外諸軍事【道子得權晉自此亂】以尚書令謝石為衛將軍後秦王萇使求傳國璽於秦王堅【璽斯氏翻】曰萇次應歷<br />
<br />
  數可以為惠堅瞋目叱之曰小羌敢逼天子五胡次序無汝羌名【胡羯鮮卑氐羌五胡之次序也無汝羌名謂䜟文耳姚萇自謂次應歷數堅故亦以䜟文為言瞋七人翻】璽已送晉不可得也萇復遣右司馬尹緯說堅求為禪代【復扶又翻說輸芮翻】堅曰禪代聖賢之事姚萇叛賊何得為之堅與緯語問緯在朕朝何官【朝直遙翻】緯曰尚書令史【後漢以來尚書有令史十八人秩二百石】堅歎曰卿王景略之儔宰相才也而朕不知卿宜其亡也堅自以平生遇萇有恩尤忿之數罵萇求死【數所角翻】謂張夫人曰豈可令羌奴辱吾兒乃先殺寶錦【堅自知身死之後女必當歸姚萇也】辛丑萇遣人縊堅於新平佛寺【年四十八】張夫人中山公詵皆自殺後秦將士皆為之哀慟【為于偽翻】萇欲隱其名諡堅曰壯烈天王<br />
<br />
  臣光曰論者皆以為秦王堅之亡由不殺慕容垂姚萇故也臣獨以為不然許劭謂魏武帝治世之能臣亂世之姦雄【事見五十八卷漢靈帝中平元年治直吏翻】使堅治國無失其道【治直之翻】則垂萇皆秦之能臣也烏能為亂哉堅之所以亡由驟勝而驕故也魏文侯問李克吳之所以亡對曰數戰數勝文侯曰數戰數勝國之福也何故亡對曰數戰則民疲數勝則主驕【數所角翻】以驕主御疲民未有不亡者也秦王堅似之矣<br />
<br />
  長樂公丕在鄴將西赴長安【樂音洛】幽州刺史王永在壺關【是年春王永自幽州奔壺關】遣使招丕丕乃帥鄴中男女六萬餘口西如潞川【使疏吏翻帥讀曰率】驃騎將軍張蚝并州刺史王騰迎之入晉陽【驃匹妙翻騎奇寄翻蚝七吏翻】丕始知長安不守堅已死乃發喪即皇帝位【丕字永叙堅之庶長子也】追諡堅曰宣昭皇帝廟號世祖大赦改元大安 燕王垂以魯王和為南中郎將鎮鄴遣慕容農出蠮螉塞【蠮于結翻螉于公翻】歷凡城趣龍城【趣七喻翻】會兵討餘巖慕容麟慕容隆自信都徇勃海清河麟擊勃海太守封懿執之因屯歷口【水經注清河自廣川東北流逕歷縣故城南前漢信都國之屬縣也應劭曰廣川縣西北三十里有歷城亭故縣也今亭在縣東津濟之所謂之歷口渡】懿放之子也【封放見九十九卷穆帝永和七年】 鮮卑劉頭眷擊破賀蘭部於善無【善無縣前漢屬雁門郡後漢屬定襄郡漢末日棄之荒外】又破柔然于意親山【意親山蓋即意辛山親辛語相近按魏書帝紀道武登國五年四月幸意辛山與質驎討賀蘭紇突鄰紇奚諸部大破之六月還幸牛川則意辛山在牛川之北】頭眷子羅辰言於頭眷曰比來行兵所向無敵【比毗至翻】然心腹之疾願早圖之頭眷曰誰也羅辰曰從兄顯忍人也【從才用翻】必將為亂頭眷不聽顯庫仁之子也頃之顯果殺頭眷自立又將殺拓跋珪【珪依劉庫仁見一百四卷太元元年】顯弟亢埿妻珪之姑也【埿與泥同】以告珪母賀氏顯謀主梁六眷代王什翼犍之甥也【犍居言翻下同】亦使其部人穆崇奚牧密告珪【魏書官氏志拓拔氏之先兼并它國各有本部部中别族為内姓丘穆陵氏後改為穆氏又拓拔鄰以弟為逹奚氏後改為奚氏】且以其愛妻駿馬付崇曰事泄當以此自明賀氏夜飲顯酒令醉【飲于鴆翻】使珪隂與舊臣長孫犍元他羅結輕騎亡去【元他羅結二人也他唐何翻】向晨賀氏故驚廐中羣馬使顯起視之賀氏哭曰吾子適在此今皆不見汝等誰殺之邪顯以故不急追珪遂奔賀蘭部依其舅賀訥【賀訥本賀蘭訥後魏孝文帝改北方舊姓以賀蘭氏為賀氏史因簡便而書之如上文穆崇奚牧之類皆是也】訥驚喜曰復國之後當念老臣珪笑曰誠如舅言不敢忘也顯疑梁六眷泄其謀將囚之穆崇宣言曰六眷不顧恩義助顯為逆我掠得其妻馬足以解忿顯乃捨之賀氏從弟外朝大人賀悦舉所部以奉珪【賀悦盖什翼犍時為外朝大人魏書官氏志曰登國二年因舊制置南北大人對治二部又置外朝大人無常員主受詔命外使出禁中國有大喪大禮皆與參知隨所典焉從才用翻】顯怒將殺賀氏賀氏奔亢埿家匿神車中三日【北人無室屋逐水草置神于車中而嚴事之因謂之神車】亢埿舉家為之請乃得免【為于偽翻】故南部大人長孫嵩帥所部七百餘家叛顯奔五原【嵩依劉氏亦見一百四卷太元元年五原秦郡魏晉棄之荒外帥讀曰率】時拓跋寔君之子渥亦聚衆自立嵩欲從之烏渥謂嵩曰逆父之子不足從也【寔君弑什翼犍見太元元年】不如歸珪嵩從之【史言長孫嵩由此遂為拓跋珪佐命功臣福流子孫】久之劉顯所部有亂故中部大人庾和辰奉賀氏奔珪【凡言故者皆什翼犍舊所署置也魏書官氏志拓拔詰汾時餘部諸姓内入者自有庾氏非中國之庾氏也】賀訥弟染干以珪得衆心忌之使其黨侯引七突殺珪代人尉古真知之以告珪侯引七突不敢發【侯引七突官氏志無此氏志云諸姓年世稍久互以改易興衰存滅間有之今舉其可知者則其不可知而不舉者亦有之矣西方尉遲氏後改為尉氏尉讀如鬱】染干疑古真泄其謀執而訊之【訊鞠問也】以兩車輪夾其頭傷一目不伏乃免之染干遂舉兵圍珪賀氏出謂染干曰汝等欲於何置我而殺吾子乎染干慙而去【史言賀訥兄弟不能舉部以奉拓跋珪為珪攻賀蘭部張本夫以珪備嘗險阻艱難以成大業而卒斃于賀蘭氏豈天道邪】 九月秦主丕以張蚝為侍中司空【蚝七吏翻】王永為侍中都督中外諸軍事車騎大將軍尚書令王騰為中軍大將軍司隸校尉苻冲為尚書左僕射封西平王又以左長史楊輔為右僕射右長史王亮為護軍將軍立妃楊氏為皇后子寧為皇太子夀為長樂王【樂音洛】鏘為平原王懿為勃海王昶為濟北王 呂光自龜兹還至宜禾【班志敦煌郡廣至縣昆侖障宜禾都尉治晉分為宜禾縣屬晉昌郡劉昫曰瓜州常樂縣漢廣至縣魏分廣至置宜禾縣李暠於此置凉興郡隨廢置常樂鎮武德五年改鎮為縣龜兹音丘慈】秦凉州刺史梁熙謀閉境拒之高昌太守楊翰言於熙曰【李延夀曰高昌者車師前王之故地昔漢武帝遣兵西討師旅頓弊其中尤困者住焉地勢高敞人庶昌盛因名高昌其地有漢時高昌壘晉為高昌郡後因為國名】呂光新破西域兵疆氣鋭聞中原喪亂【喪息浪翻】必有異圖河西地方萬里帶甲十萬足以自保若光出流沙其勢難敵高梧谷口險阻之要宜先守之而奪其水【高悟谷口當在高昌西界】彼既窮渇可以坐制如以為遠伊吾關亦可拒也【伊吾縣晉置屬晉昌郡有伊吾關】度此二阨雖有子房之策無所施矣【言地險既失雖有張良之計無所用也】熙弗聽美水令犍為張統謂熙曰今關中大亂京師存亡不可知【長安已陷而凉州不知道梗故也犍居言翻】呂光之來其志難測將軍何以抗之熙曰憂之未知所出統曰光智畧過人今擁思歸之士乘戰勝之氣其鋒未易當也【易以䜴翻】將軍世受大恩忠誠夙著立勲王室宜在今日行唐公洛上之從弟勇冠一時【從才用翻冠古玩翻】為將軍計莫若奉為盟主以收衆望推忠義以帥羣豪【帥讀曰率下同】則光雖至不敢有異心也資其精鋭東兼毛興【毛興時刺河州】連王統楊璧【王統時刺秦州楊璧時刺南秦州】合四州之衆掃兇逆寧帝室此桓文之舉也熙又弗聽殺洛於西海【洛徙西海見一百四卷太元五年梁熙既欲拒呂光又殺苻洛不過欲保據凉州非有扶顛持危之志也】光聞楊翰之謀懼不敢進杜進曰梁熙文雅有餘機鑒不足終不能用翰之謀不足憂也宜及其上下離心速進以取之光從之進至高昌楊翰以郡迎降【熙不能用楊翰之謀翰遂降於光降戶江翻下同】至玉門熙移檄責光擅命還師以子胤為鷹揚將軍與振威將軍南安姚皓别駕衛翰帥衆五萬拒光於酒泉【帥讀曰率】燉煌太守姚静晉昌太守李純以郡降光【敦徒門翻】光報檄凉州責熙無赴難之志【難乃旦翻】而遏歸國之衆遣彭晃杜進姜飛為前鋒與胤戰于安彌【安彌縣自漢以來屬酒泉郡】大破擒之於是四山胡夷皆附於光武威太守彭濟執熙以降光殺之光入姑臧自領凉州刺史表杜進為武威太守自餘將佐各受職位凉州郡縣皆降於光獨酒泉太守宋皓西郡太守宋泮【晉志曰漢分張掖之日勒刪丹等縣置西郡其地當嶺要】城守不下光攻而執之讓泮曰吾受詔平西域而梁熙絶我歸路此朝廷之罪人卿何為附之泮曰將軍受詔平西域不受詔亂凉州梁公何罪而將軍殺之泮但苦力不足不能報君父之讐耳豈肯如逆氐彭濟之所為乎主滅臣死固其常也光殺泮及皓主簿尉祐姦佞傾險【尉姓也讀如字】與彭濟俱執梁熙光寵信之祐譛殺名士姚皓等十餘人凉州人由是不悦【昔齊人伐燕勝之孟子曰取之而燕民悦則取之取之而燕民不悦則勿取其後燕卒報齊呂光始得凉士而無以收凉人之心宜其有國不永也】光以祐為金城太守祐至允吾【允吾漢縣屬金城郡晉省據水經注允吾在廣武西北其地在當時蓋屬廣武郡界劉昫曰唐鄯州龍支縣本漢允吾縣後漢改曰龍耆後魏改曰金城又改曰龍支積石山在今縣南允音鉛吾音牙】襲據其城以叛姜飛擊破之祐奔據興城【以載記參考水經興城當在允吾之西白土之東】 乞伏國仁自稱大都督大將軍單于領秦河二州牧【單音蟬】改元建義以乙旃童埿為左相屋引出支為右相獨孤匹蹄為左輔武羣勇士為右輔【乙旃屋引獨孤武羣皆夷人復姓乞伏與拓拔同出於鮮卑故其部人亦有乙旃獨孤二姓埿與泥同】弟乾歸為上將軍分其地置武城等十二郡【載記曰置武城武陽安固武始漢陽天水畧陽漒川甘松匡朋白馬苑川十二郡】築勇士城而都之【水經注苑川水出勇士縣之子城南山北逕牧師苑故漢牧苑之地也有東西二苑城其城相去七里西城即乞伏所都】 秦尚書令魏昌公纂自關中奔晉陽秦主丕拜纂太尉封東海王 冬十月西燕主冲遣尚書令高蓋帥衆五萬伐後秦戰于新平南蓋大敗降於後秦【帥讀曰率降戶江翻】初蓋以楊定為子及蓋敗定亡奔隴右復收集其舊衆【定為西燕禽財六月耳復扶又翻】 苻定苻紹苻謨苻亮聞秦主丕即位皆自河北遣使謝罪【定紹謨亮降燕見上卷上年】中山太守王兖本新平氐也固守博陵為秦拒燕【為于偽翻】十一月丕以兖為平州刺史定為冀州牧紹為冀州都督謨為幽州牧亮為幽平二州都督並進爵郡公左將軍竇衝據兹川【兹川即霸川也霸水古曰滋水秦穆公之霸也更滋水曰霸水以顯霸功今曰兹川因古名也】有衆數萬與秦州刺史王統河州刺史毛興益州刺史王廣南秦州刺史楊璧衛將軍楊定皆自隴右遣使邀丕共擊後秦丕以定為雍州牧【雍于用翻】衝為梁州牧加統鎮西大將軍興車騎大將軍璧征南大將軍並開府儀同三司加廣安西將軍皆進位州牧楊定尋徙治歷城置儲蓄於百頃【百頃自楊茂搜以來保為巢穴】自稱龍驤將軍仇池公遣使來稱藩【驤思將翻使疏吏翻】詔因其所號假之其後又取天水略陽之地自稱秦州刺史隴西王 繹幕人蔡匡據城以叛燕【繹幕縣自漢以來屬清河郡至隋廢入德州安樂縣】燕慕容麟慕容隆共攻之泰山太守任泰潜師救匡【任音壬】至匡壘南八里燕人乃覺之諸將以匡未下而外敵奄至甚患之隆曰匡恃外救故不時下今計泰之兵不過數千人及其未合擊之泰敗匡自降矣乃釋匡擊泰大敗之斬首千餘級匡遂降【降戶江翻下同】燕王垂殺之且屠其壘 慕容農至龍城【自蠮螉塞歷凡城至龍城】休士馬十餘日諸將皆曰殿下之來取道甚速今至此久留不進何也農曰吾來速者恐餘巖過山鈔盜侵擾良民耳【此山謂白狼山鈔楚交翻】巖才不踰人誑誘饑兒【誑居况翻誘音酉】烏集為羣非有綱紀吾已扼其喉久將離散無能為也今此田善熟未收而行徒自耗損當俟收畢往則梟之【梟堅堯翻】亦不出旬日耳頃之農將步騎三萬至令支巖衆震駭稍稍踰城歸農巖計窮出降農斬之進擊高句麗復遼東玄莬二郡【郝景之敗高句麗陷遼東玄菟菟同都翻】還至龍城上疏請繕修陵廟【燕自慕容皝以前皆葬遼西故陵廟在焉】燕王垂以農為使持節都督幽平二州北狄諸軍事幽州牧鎮龍城【使疏吏翻】徙平州刺史帶方王佐鎮平郭農於是創立法制事從寛簡清刑獄省賦役勸課農桑居民富贍四方流民前後至者數萬口先是幽冀流民多入高句麗【先悉薦翻】農以驃騎司馬范陽龎淵為遼東太守招撫之慕容麟攻王兖于博陵城中糧竭矢盡功曹張猗踰<br />
<br />
  城出聚衆以應麟兖臨城數之曰【數所角翻】卿是秦民吾是卿君卿起兵應賊自號義兵何名實之相違也古人求忠臣必於孝子之門【後漢韋彪之言】卿母在城棄而不顧吾何有焉今人取卿一切之功則可矣寧能忘卿不忠不孝之事乎不意中州禮義之邦乃有如卿者也十二月麟拔博陵執兖及苻鑑殺之昌黎太守宋敞帥烏桓索頭之衆救兖不及而還【帥讀曰率索昔各翻還從宣翻又如字】秦主丕以敞為平州刺史【敞時從王永在壺關】 燕王垂北如中山謂諸將曰樂浪王招流離實倉廩外給軍糧内營宫室雖蕭何之功何以加之【樂浪王温之功詳見上漢高祖與項羽相拒蕭何鎮撫關中為之根本】丙申垂始定都中山【杜佑曰後燕都中山今博陵郡唐昌縣】 秦苻定據信都以拒燕燕王垂以從弟北地王精為冀州刺史將兵攻之【從才用翻將即亮翻】 拓跋珪從曾祖紇羅與其弟建及諸部大人共請賀訥推珪為主【從才用翻】<br />
<br />
  十一年春正月戊申拓跋珪大會于牛川【自武周塞西行至牛川牛川以北皆大漠也據魏紀窟咄之來寇也珪乞師于燕自弩山至牛川屯於延水南出代谷以會燕師又據水經注于延水出長川城南則長川即牛川也班志于延水出代郡且如塞外則牛川亦當在且如塞外也又明元帝大獮于牛川登釡山括地志釡山在媯州懷戎縣北三里且如之且音子如翻】即代王位【珪什翼犍之嫡孫世子寔之子拓跋氏自此興矣】改元登國以長孫嵩為南部大人叔孫普洛為北部大人分治其衆【魏書曰魏初自北荒南遷置四部大人坐王庭决辭訟以言語約束刻契記事無囹圄考訊之法諸犯罪者皆臨時决遣治直之翻下同】以上谷張衮為左長史許謙為右司馬廣甯王建代人和跋叔孫建庾岳為外朝大人【魏書官氏志拓拔鄰命叔父之胤曰乙旃氏後改叔孫氏朝直遙翻】奚牧為治民長【長知兩翻】皆掌宿衛及參軍國謀議長孫道生賀毗等侍從左右【從才用翻】出納教命王建娶代王什翼犍之女岳和辰之弟道生嵩之從子也【庾和辰奉珪母賀氏以奔珪長孫嵩帥部衆歸珪事並見上從才用翻】 燕王垂即皇帝位【垂字道明燕主皝之第五子】 後秦王萇如安定 南安祕宜【祕姓也前漢書功臣表有戴侯祕彭時祕氏為南安豪族】帥羌胡五萬餘人攻乞伏國仁國仁將兵五千逆擊大破之【帥讀曰率將即亮翻】奔還南安 鮮于乞之殺翟真也【翟真見殺見上年四月】翟遼奔黎陽黎陽太守滕恬之甚愛信之恬之喜畋獵【喜許記翻】不愛士卒遼潜施姦惠以收衆心恬之南攻鹿鳴城【黎陽在河北鹿鳴城在河南酈道元曰案竹書紀年梁惠成王十三年取鄭鹿即是城也今城内有臺尚謂之鹿鳴臺又謂之鹿鳴城】遼于後閉門拒之恬之東奔鄄城【鄄吉掾翻】遼追執之遂據黎陽豫州刺史朱序遣將軍秦膺童斌與淮泗諸郡共討之【斌音彬】秦益州牧王廣自隴右引兵攻河州牧毛興於枹罕興遣建節將軍衛平帥其宗人一千七百夜襲廣大破之【帥讀曰率】二月秦州牧王統遣兵助廣攻興興嬰城自守【去年王廣自成都依統】 燕大赦改元建興置公卿尚書百官繕宗廟社稷 西燕主冲樂在長安【樂音洛】且畏燕主垂之彊不敢東歸課農築室為久安之計鮮卑咸怨之【鮮卑思東歸而冲安于長安故怨傳曰以欲從人則可以人從欲鮮濟其是之謂歟】左將軍韓延因衆心不悦攻冲殺之立冲將段隨為燕王改元昌平 初張天錫之南奔也【見上卷太元八年】秦長水校尉王穆匿其世子大豫與俱奔河西依秃髪思復鞬【思復鞬烏孤之父也鞬居言翻】思復鞬送魏安【五代志武威郡昌松縣後魏置昌松郡後周廢郡以揖次縣入焉又有後魏魏安郡後亦廢載記言焦松等迎大豫于揖次則魏安盖後魏所置郡晉書成於唐唐史臣以後魏郡名書之耳孟康曰揖音子如翻次音恣】魏安人焦松齊肅張濟等聚兵數千人迎大豫為主攻呂光昌松郡拔之【昌松即漢倉松縣地本屬武威郡盖河西張氏分置郡也呂光後以郭黁言改昌松為東張掖郡】執太守王世強光使輔國將軍杜進擊之進兵敗大豫進逼姑臧王穆諫曰光糧豐城固甲兵精鋭逼之非利不如席卷嶺西【卷讀曰捲】礪兵積粟然後東向與之争不及朞年光可取也大豫不從自號撫軍將軍凉州牧改元鳳凰以王穆為長史傳檄郡縣使穆說諭嶺西諸郡【自西郡至張掖酒泉建康晉昌其地皆嶺西也說愉芮翻】建康太守李隰祁連都尉嚴純皆起兵應之【建康郡張駿置屬凉州新唐書地理志甘州張掖縣西北百九十里有祁連山北有建康軍盖張氏置郡地也晉書地理志永興中張祚置漢陽縣以守牧地張玄靚改為祁連郡】有衆三萬保據楊塢【楊塢在姑臧城西】 代王珪徙居定襄之盛樂【按盛樂前漢書作成樂屬定襄後漢書作盛樂屬雲中疑定襄之盛樂即雲中之盛樂也然魏書帝紀什翼犍立三年移都于雲中之盛樂明年築盛樂城於故城南八里則已非後漢之盛樂城疑定襄之盛樂乃前漢之成樂城也蓋建武之初匈奴侵寇塞下之民悉内徙其後南單于内附北單于勢屈復歸内徙之民於塞下郡縣城郭掃地更為必有非其故處者攷宋白續通典唐振武軍漢定襄郡之盛樂也在隂山之陽黄河之北後魏所都盛樂是也在唐朔州北三百餘里後魏孝文遷洛之後于定襄故城置朔州領盛樂廣牧二郡唐初平突厥置雲中都督府於盛樂貞觀八年移雲州雲中郡及定襄縣于今雲州而雲中都督府後又改單于都護府又改安北都護府由是雲中定襄地名混亂不可考而盛樂則一也】務農息民國人悦之【史言拓跋珪所以興】 三月大赦 泰山太守張願以郡叛降翟遼【降戶江翻】初謝玄欲使朱序屯梁國玄自屯彭城以北固河上西援洛陽朝議以征役既久欲令玄置戍而還會翟遼張願繼叛北方騷動玄謝罪乞解職詔慰諭令還淮隂 燕主垂追尊母蘭氏為文昭皇后欲遷文明段后以蘭氏配享太祖【段氏燕王皝之元妃蘭氏皝之側室也皝廟號太祖諡文明皇帝】詔百官議之皆以為當然【順垂指也】博士劉詳董謐以為堯母為帝嚳妃位第三不以貴陵姜嫄【帝王紀曰帝嚳有四妃元妃有邰氏女曰姜嫄生后稷次妃有娀氏女曰簡狄生卨次妃陳豐氏女曰慶都生放勛次妃娵訾氏女曰常義生摯嚳苦沃翻】明聖之道以至公為先文昭后宜立别廟垂怒逼之詳謐曰上所欲為無問於臣臣案經奉禮不敢有貳垂乃不復問諸儒卒遷段后以蘭后代之【復扶又翻卒子戍翻】又以景昭可足渾后傾覆社稷追廢之尊烈祖昭儀段氏為景德皇后配享烈祖【可足渾氏燕主雋之元妃也傾覆事見一百二卷海西公太和四年段氏雋之側室也雋廟號烈祖諡景昭皇帝】崔鴻曰齊桓公命諸侯無以妾為妻【孟子曰齊桓公葵丘之會初命曰誅不孝無易樹子無以妾為妻】夫之於妻猶不可以妾代之况子而易其母乎春秋所稱母以子貴者君母既沒得以妾母為小君也【春秋公羊傳曰桓幼而貴隱長而卑隱長又賢諸大夫扳隱而立之隱之立為桓立也隱長又賢何以不宜立立適以長不以賢立子以貴不以長桓何以貴母貴也母貴則子何以貴子以母貴母以子貴左氏傳曰惠公元妃孟子孟子卒繼室以聲子生隱公宋武公生仲子為魯夫人生桓公而惠公薨是以隱公立而奉之】至於享祀宗廟則成風終不得配莊公也【魯莊公夫人姜氏成風者莊公之妾僖公之母也姜氏通于共仲弑閔公而欲立共仲不克遂孫于邾齊桓公殺之僖公既立請其喪以夫人之禮葬之】君父之所為臣子必習而效之猶形聲之于影響也寶之逼殺其母【事見後一百八卷太元二十一年】由垂為之漸也堯舜之讓猶為之噲之禍【事見二卷周愼靚王五年】况違禮而縱私者乎昔文姜得罪於桓公春秋不之廢【魯桓公之夫人曰文姜通于齊襄公桓公謫之夫人以告襄公遂殺桓公至莊公二十一年春秋書夫人姜氏薨二十二年書葬我小君文姜是不之廢也】可足渾氏雖有罪於前朝【朝直遥翻】然小君之禮成矣垂以私憾廢之【私憾謂譛殺垂妃段氏又譛垂而逐之奔秦也】又立兄妾之無子者皆非禮也<br />
<br />
  劉顯自善無南走馬邑【畏代之偪且懼其修怨也】其族人奴真帥所部請降於代【帥讀曰率降戶江翻】奴真有兄鞬先居賀蘭部【鞬居言翻】奴真言於代王珪請召鞬而以所部讓之珪許之鞬既領部遣弟去斤遺賀訥金馬【遺于季翻】賀染干謂去斤曰我待汝兄弟厚汝今領部宜來從我去斤許之奴真怒曰我祖父以來世為代忠臣故我以部讓汝等欲為義也今汝等無狀乃謀叛國義於何在遂殺鞬及去斤染干聞之引兵攻奴真奴真奔代珪遣使責染干染干乃止【珪與賀蘭從此隙矣使疏吏翻】 西燕僕射慕容恒尚書慕容永襲段隨殺之立宜都王子顗為燕王【顗盖燕宜都王栢之子顗魚豈翻】改元建明帥鮮卑男女四十餘萬口去長安而東【海西公太和五年秦遷鮮卑于長安至是財十七年耳而種類蕃育乃如此唐太宗阿史那結社率之變亦幸其早發耳帥讀曰率】恒弟護軍將軍韜誘顗殺之於臨晉【誘音酉】恒怒捨韜去永與武衛將軍刁雲帥衆攻韜韜敗奔恒營恒立西燕主冲之子瑶為帝改元建平諡冲曰威皇帝衆皆去瑶奔永永執瑶殺之立慕容泓子忠為帝改元建武忠以永為太尉守尚書令封河東公永持灋寛平鮮卑安之至聞喜聞燕主垂已稱尊號不敢進築燕熙城而居之【為燕主垂滅西燕張本】 鮮卑既東長安空虚前滎陽高陵趙穀等招杏城盧水胡郝奴帥戶四千入於長安渭北皆應之【帥讀曰率下同】以穀為丞相扶風王驎有衆數千保據馬嵬【馬嵬山在咸陽縣西去長安城百餘里杜佑曰京兆金城縣有馬嵬故城孫景安征塗記云馬嵬所築不知何代人金城周曰犬丘秦曰廢丘漢曰槐里驎離珍翻嵬吾回翻】奴遣弟多攻之夏四月後秦王萇自安定伐之驎奔漢中萇執多而進奴懼請降【降戶江翻】拜鎮北將軍六谷大都督【長安南山有六谷】 癸巳以尚書僕射陸納為左僕射譙王恬為右僕射【晉志曰左右僕射或不兩置但曰尚書僕射】納玩之子也【陸玩見九十四卷成帝咸和四年】 毛興襲擊王廣敗之【敗補邁翻】廣奔秦州隴西鮮卑匹蘭執廣送于後秦興復欲攻王統于上邽枹罕諸氐皆厭苦兵事【諸氐盖秦王堅使毛興領之以鎮枹罕者也復扶又翻枹音膚】乃共殺興推衛平為河州刺史【以衛平宗強故推之】遣使請命於秦【使疏吏翻下同】 燕主垂封其子農為遼西王麟為趙王隆為高陽王 代王珪初改稱魏王【拓跋氏自此國號魏】 張大豫自楊塢進屯姑臧城西王穆及禿髪思復鞬子奚于帥衆三萬屯於城南【鞬居言翻】呂光出擊大破之斬奚于等二萬餘級 秦大赦以衛平為撫軍將軍河州刺史呂光為車騎大將軍凉州牧【騎奇寄翻】使者皆没于後秦不能逹【時秦主丕在晉陽後秦隔其道故不能逹二鎮】燕主垂以范陽王德為尚書令太原王楷為左僕射樂浪王温為司隸校尉【温守中山有營宫室建都邑之功因用為司隸樂浪音洛琅】後秦王萇即皇帝位于長安【萇字景茂姚弋仲之第二十四子也】大赦改元建初國號大秦追尊其父弋仲為景元皇帝立妻地氏為皇后【類篇虵以者翻虜姓也姓譜姚萇后虵氏南安人也虵食遮翻又音它】子興為皇太子置百官萇與羣臣宴酒酣言曰諸卿皆與朕北面秦朝【朝直遥翻】今忽為君臣得無耻乎趙遷曰天不恥以陛下為子臣等何恥為臣萇大笑 魏王珪東如陵石【陵石地名在盛樂東】護佛侯部帥侯辰乙佛部帥代題皆叛走【帥所類翻】諸將請追之珪曰侯辰等累世服役有罪且當忍之方今國家草創人情未壹愚者固宜前却【一前一却言叛服不常也】不足追也【史言珪能識時知變以安反側】 六月庚寅以前輔國將軍楊亮為雍州刺史鎮衛山陵【帝置雍州于襄陽今遣亮帶雍州鎮洛雍于用翻】荆州刺史桓石民遣將軍晏謙擊弘農下之【姓譜曰左傳齊有晏氏代為大夫】初置湖陜二戍【湖陜二縣皆屬弘農陜式冉翻】 西燕刁雲等殺西燕主忠推慕容永為使持節大都督中外諸軍事大將軍大單于雍秦梁凉四州牧録尚書事河東王稱藩於燕【稱藩于燕主垂也使疏吏翻雍于用翻】 燕主垂遣太原王楷趙王麟陳留王紹章武王宙攻秦苻定苻紹苻謨苻亮等【去年苻定等背燕為秦】楷先以書與之為陳禍福定等皆降【降戶江翻】垂封定等為侯曰以酬秦主之德 秦主丕以都督中外諸軍事司徒録尚書事王永為左丞相太尉東海王纂為大司馬司空張蚝為太尉【蚝七吏翻】尚書令咸陽徐義為司空司隸校尉王騰為驃騎大將軍儀同三司【驃匹妙翻騎奇寄翻】永傳檄四方公侯牧守壘主民豪共討姚萇慕容垂令各帥所統以孟冬上旬會大駕於臨晉【帥讀曰率】於是天水姜延馮翊寇明河東王昭新平張晏京兆杜敏扶風馬朗建忠將軍高平牧官都尉扶風王敏等【建忠將軍盖亦苻氏創置高平縣漢屬安定郡晉省前趙劉曜以為朔州治所秦置牧官都尉於其地】咸承檄起兵各有衆數萬遣使詣秦【使疏吏翻】丕皆就拜將軍郡守封列侯冠軍將軍鄧景擁衆五千據彭池與竇衝為首尾以擊後秦【彭池恐當作彪池彪池在長安西詩所謂滮池北流者也竇衝據兹川在長安東南可以相為首尾又據新唐書地理志寧州有彭池但去兹川遠耳】丕以景為京兆尹景羌之子也【鄧羌秦之名將】 後秦王萇徙安定五千餘戶于長安秋七月秦平凉太守金熙安定都尉沒奕干與後秦<br />
<br />
  左將軍姚方成戰于孫丘谷【孫丘谷當在安定】方成兵敗後秦主萇以其弟征虜將軍緒為司隸校尉鎮長安自將至安定【將即亮翻】擊熙等大破之金熙本東胡之種【秦謂鮮卑之種居遼碣者為東胡種章勇翻】没奕干鮮卑多蘭部帥也【帥所類翻】枹罕諸氐【枹音膚】以衛平衰老難與成功議廢之而憚其宗彊累日不决氐啖青謂諸將曰【啖徒覽翻姓也】大事宜時定不然變生諸君但請衛公為會觀我所為會七夕大宴青抽劒而前曰今天下大亂吾曹休戚同之非賢主不可以濟大事衛公老宜返初服以避賢路【衛平本毛興部將】狄道長苻登【長知兩翻】雖王室疎屬志略雄明請共立之以赴大駕【欲承王永檄赴秦主丕也】諸君有不同者即下異議乃奮劒攘袂將斬異已者衆皆從之莫敢仰視於是推登為使持節都督隴右諸軍事撫軍大將軍雍河二州牧略陽公帥衆五萬東下隴攻南安拔之馳使請命於秦登秦主丕之族子也【苻登事始此使疏吏翻雍于用翻帥讀曰率下同】 祕宜與莫侯悌眷帥其衆三萬餘戶降于乞伏國仁國仁拜宜東秦州刺史悌眷梁州刺史【莫侯夷人複姓降戶江翻下同】 己酉魏王珪還盛樂【自陵石還也】代題復以部落來降【復扶又翻】十餘日又奔劉顯珪使其孫倍斤代領其衆劉顯弟肺泥帥衆降魏 八月燕主垂留太子寶守中山以趙王麟為尚書右僕射録留臺庚午自帥范陽王德等南略地使高陽王隆東徇平原丁零鮮于乞保曲陽西山【曲陽縣屬常山郡】聞垂南伐出營望都【望都縣屬中山郡水經注堯封于唐堯山在東北堯母慶都山在南登堯山見都山故望都縣以為名也】剽掠居民【剽匹妙翻】趙王麟自出討之諸將皆曰殿下虚鎮遠征萬一無功而返虧損威重不如遣諸將討之麟曰乞聞大駕在外無所畏忌必不設備一舉可取不足憂也乃聲言至魯口夜回趣乞【趣七喻翻】比明至其營掩擊擒之【比必寐翻及也】 翟遼寇譙朱序擊走之 秦主丕以苻登為征西大將軍開府儀同三司南安王持節州牧都督皆因其所稱而授之又以徐義為右丞相留王騰守晉陽右僕射楊輔戍壺關帥衆四萬進屯平陽【帥讀曰率】初後秦主萇之弟碩德統所部羌居隴上聞萇起兵自稱征西將軍聚衆於冀城以應之以兄孫詳為安遠將軍據隴城從孫訓為安西將軍據南安之赤亭【諸姚本赤亭羌也】與秦秦州刺史王統相持萇自安定引兵會碩德攻統天水屠各畧陽羌胡應之者二萬餘戶秦略陽太守王皮降之【屠直於翻降戶江翻】 初秦滅代遷代王什翼犍少子窟咄於長安【事見一百四卷太元元年窟苦骨翻咄當没翻】從慕容永東徙永以窟咄為新興太守劉顯遣其弟亢埿迎窟咄以兵隨之逼魏南境諸部騷動魏王珪左右于桓等與部人謀執珪以應窟咄幢將代人莫題等亦潜與窟咄交通【魏收官氏志道武登國元年置幢將六人主三郎衛士直宿禁中者侍中已下中散已上皆統之拓拔詰汾時餘部諸姓内入者有莫那婁氏後為莫氏幢直江翻】桓舅穆崇告之珪誅桓等五人莫題等七姓悉原不問【為後珪殺莫題張本】珪懼内難【難乃旦翻】北踰隂山復依賀蘭部【復扶又翻】遣外朝大人遼東安同求救於燕【姓譜安息王子入侍遂為漢人風俗通漢有安成為太守廬山記有安息國王子安高朝直遙翻】燕主垂遣趙王麟救之 九月王統以秦州降於後秦【降戶江翻】後秦主萇以姚碩德為使持節都督隴右諸軍事秦州刺史鎮上邽【使疏吏翻】 呂光得秦王堅凶問舉軍縞素諡曰文昭皇帝冬十月大赦改元大安【晉書載記作太安】 西燕慕容永遣使詣秦主丕求假道東歸【使疏吏翻】丕弗許與永戰于襄陵【襄陵縣漢屬河東郡晉屬平陽郡】秦兵大敗左丞相王永衛大將軍俱石子皆死初東海王纂自長安來【去年纂奔丕】麾下壯士三千餘人丕忌之既敗懼為纂所殺帥騎數千南奔東垣【東垣縣漢已改為真定此東垣在河南新安縣界宋白曰河南府王屋縣漢為河東郡垣縣地又絳州垣縣本河東郡縣名其地即周召分陜之所又曰河南府漢為河南郡治洛陽後漢都洛陽河南屬司隸校尉魏陳留王奐合河南等五郡置司州劉聰為荆州石虎為洛州苻堅為豫州宋武入洛更置東垣西垣二縣仍于虎牢置司州帥讀曰率騎奇寄翻】謀襲洛陽揚威將軍馮該自陜邀擊之【陜式冉翻】殺丕執其太子寧長樂王夀送建康【樂音洛】詔赦不誅以付苻宏【苻宏去年來奔處之江州】纂與其弟尚書永平侯師奴帥秦衆數萬走據杏城其餘王公百官皆沒於永永遂進據長子即皇帝位改元中興將以秦后楊氏為上夫人楊氏引劍刺永【刺七亦翻】為永所殺 甲申海西公奕薨于吳【年四十五】 燕寺人吳深據清河反【寺音侍又如字】燕主垂攻之不克 後秦主萇還安定【自上邽還安定也】秦南安王登既克南安夷夏歸之者三萬餘戶【夏戶雅翻】遂進攻姚碩德于秦州後秦主萇自往救之登與萇戰于胡奴阜【胡奴阜在上邽西】大破之斬首二萬餘級將軍啖青射萇中之【射而亦翻中竹仲翻】萇創重【創初良翻】走保上邽姚碩德代之統衆 燕趙王麟軍未至魏拓跋窟咄稍前逼魏王珪賀染干侵魏北部以應之魏衆驚擾北部大人叔孫普洛亡奔劉衛辰麟聞之遽遣安同等歸魏人知燕軍在近衆心少安【少詩沼翻】窟咄進屯高柳【高柳縣漢屬代郡晉省酈道元曰高柳在代中其山重巒疊巘霞舉雲高其山隱隱而東出遼塞注又見前】珪引兵與麟會擊之窟咄大敗奔劉衛辰衛辰殺之珪悉收其衆以代人庫狄干為北部大人【魏書官氏志次南諸部有庫狄氏後為狄氏】麟引兵還中山劉衛辰居朔方士馬甚盛後秦主萇以衛辰為大將軍大單于河西王幽州牧【單音蟬】西燕主永以衛辰為大將軍朔州牧十一月秦尚書寇遺奉勃海王懿濟北王昶自杏城奔南安【濟子禮翻】南安王登發喪行服諡秦主丕曰哀平皇帝登議立懿為主衆曰勃海王雖先帝之子然年在幼冲未堪多難【難乃旦翻】今三虜窺覦【三虜謂姚萇慕容垂慕容永也】宜立長君【長知兩翻】非大王不可登乃為壇於隴東即皇帝位【登字文高堅之族孫也】大赦改元太初置百官 慕容柔慕容盛及盛弟會皆在長子【太元九年柔等自長安得脱奔慕容冲冲死隨永東遷】盛謂柔會曰主上已中興幽冀【主上謂燕主垂】東西未壹【東謂燕主垂西謂燕主永】吾屬居嫌疑之地為智為愚皆將不免不若以時東歸無為坐待魚肉也遂相與亡歸燕後歲餘西燕主永悉誅燕主雋及燕主垂之子孫男女無遺【史言慕容盛之智】 張大豫自西郡入臨洮掠民五千餘戶保據俱城【俱城在臨洮界】 十二月呂光自稱使持節侍中中外大都督督隴右河西諸軍事大將軍凉州牧酒泉公【使疏吏翻】 秦主登立世祖神主于軍中【秦主堅廟號世祖】載以輜軿【車四面有屏蔽者曰輜軿軿蒲眠翻】建黄旗青蓋以虎賁三百人衛之【賁音奔】凡所欲為必啟主而後行引兵五萬東擊後秦將士皆刻鉾鎧為死休字【鉾頭牟鎧甲也鉾音牟言欲復讐必死乃休也楊正衡曰字林鉾古矛字】每戰以劍矟為方圓大陣【矟色角翻與槊同】知有厚薄從中分配故人自為戰所向無前初長安之將敗也【謂苻堅為慕容冲所困之時】中壘將軍徐嵩屯騎校尉胡空各聚衆五千結壘自固既而受後秦官爵後秦主萇以王禮葬秦主堅于二壘之間及登至嵩空以衆降之登拜嵩雍州刺史空京兆尹【降戶江翻雍于用翻】改葬堅以天子之禮 乙酉燕主垂攻吳深壘拔之深單馬走【吳深時據清河以叛燕】垂進屯聊城之逢關陂【聊城縣漢屬東郡晉屬平原郡】初燕太子洗馬温詳來奔以為濟北太守屯東阿【東阿縣漢屬東郡晉屬濟北郡洗悉薦翻濟子禮翻】燕主垂遣范陽王德高陽王隆攻之詳遣從弟攀守河南岸子楷守碻磝以拒之 燕主垂以魏王珪為西單于封上谷王珪不受【珪不受燕封其志不在小單音蟬】<br />
<br />
  資治通鑑卷一百六  <br>
   </div> 

<script src="/search/ajaxskft.js"> </script>
 <div class="clear"></div>
<br>
<br>
 <!-- a.d-->

 <!--
<div class="info_share">
</div> 
-->
 <!--info_share--></div>   <!-- end info_content-->
  </div> <!-- end l-->

<div class="r">   <!--r-->



<div class="sidebar"  style="margin-bottom:2px;">

 
<div class="sidebar_title">工具类大全</div>
<div class="sidebar_info">
<strong><a href="http://www.guoxuedashi.com/lsditu/" target="_blank">历史地图</a></strong>  
<a href="http://www.880114.com/" target="_blank">英语宝典</a>  
<a href="http://www.guoxuedashi.com/13jing/" target="_blank">十三经检索</a> 
<br><strong><a href="http://www.guoxuedashi.com/gjtsjc/" target="_blank">古今图书集成</a></strong> 
<a href="http://www.guoxuedashi.com/duilian/" target="_blank">对联大全</a> <strong><a href="http://www.guoxuedashi.com/xiangxingzi/" target="_blank">象形文字典</a></strong> 

<br><a href="http://www.guoxuedashi.com/zixing/yanbian/">字形演变</a>  <strong><a href="http://www.guoxuemi.com/hafo/" target="_blank">哈佛燕京中文善本特藏</a></strong>
<br><strong><a href="http://www.guoxuedashi.com/csfz/" target="_blank">丛书&方志检索器</a></strong> <a href="http://www.guoxuedashi.com/yqjyy/" target="_blank">一切经音义</a>  

<br><strong><a href="http://www.guoxuedashi.com/jiapu/" target="_blank">家谱族谱查询</a></strong>  <strong><a href="http://shufa.guoxuedashi.com/sfzitie/" target="_blank">书法字帖欣赏</a></strong> 
<br>

</div>
</div>


<div class="sidebar" style="margin-bottom:0px;">

<font style="font-size:22px;line-height:32px">QQ交流群9:489193090</font>


<div class="sidebar_title">手机APP 扫描或点击</div>
<div class="sidebar_info">
<table>
<tr>
	<td width=160><a href="http://m.guoxuedashi.com/app/" target="_blank"><img src="/img/gxds-sj.png" width="140"  border="0" alt="国学大师手机版"></a></td>
	<td>
<a href="http://www.guoxuedashi.com/download/" target="_blank">app软件下载专区</a><br>
<a href="http://www.guoxuedashi.com/download/gxds.php" target="_blank">《国学大师》下载</a><br>
<a href="http://www.guoxuedashi.com/download/kxzd.php" target="_blank">《汉字宝典》下载</a><br>
<a href="http://www.guoxuedashi.com/download/scqbd.php" target="_blank">《诗词曲宝典》下载</a><br>
<a href="http://www.guoxuedashi.com/SiKuQuanShu/skqs.php" target="_blank">《四库全书》下载</a><br>
</td>
</tr>
</table>

</div>
</div>


<div class="sidebar2">
<center>


</center>
</div>

<div class="sidebar"  style="margin-bottom:2px;">
<div class="sidebar_title">网站使用教程</div>
<div class="sidebar_info">
<a href="http://www.guoxuedashi.com/help/gjsearch.php" target="_blank">如何在国学大师网下载古籍?</a><br>
<a href="http://www.guoxuedashi.com/zidian/bujian/bjjc.php" target="_blank">如何使用部件查字法快速查字?</a><br>
<a href="http://www.guoxuedashi.com/search/sjc.php" target="_blank">如何在指定的书籍中全文检索?</a><br>
<a href="http://www.guoxuedashi.com/search/skjc.php" target="_blank">如何找到一句话在《四库全书》哪一页?</a><br>
</div>
</div>


<div class="sidebar">
<div class="sidebar_title">热门书籍</div>
<div class="sidebar_info">
<a href="/so.php?sokey=%E8%B5%84%E6%B2%BB%E9%80%9A%E9%89%B4&kt=1">资治通鉴</a> <a href="/24shi/"><strong>二十四史</strong></a>&nbsp; <a href="/a2694/">野史</a>&nbsp; <a href="/SiKuQuanShu/"><strong>四库全书</strong></a>&nbsp;<a href="http://www.guoxuedashi.com/SiKuQuanShu/fanti/">繁体</a>
<br><a href="/so.php?sokey=%E7%BA%A2%E6%A5%BC%E6%A2%A6&kt=1">红楼梦</a> <a href="/a/1858x/">三国演义</a> <a href="/a/1038k/">水浒传</a> <a href="/a/1046t/">西游记</a> <a href="/a/1914o/">封神演义</a>
<br>
<a href="http://www.guoxuedashi.com/so.php?sokeygx=%E4%B8%87%E6%9C%89%E6%96%87%E5%BA%93&submit=&kt=1">万有文库</a> <a href="/a/780t/">古文观止</a> <a href="/a/1024l/">文心雕龙</a> <a href="/a/1704n/">全唐诗</a> <a href="/a/1705h/">全宋词</a>
<br><a href="http://www.guoxuedashi.com/so.php?sokeygx=%E7%99%BE%E8%A1%B2%E6%9C%AC%E4%BA%8C%E5%8D%81%E5%9B%9B%E5%8F%B2&submit=&kt=1"><strong>百衲本二十四史</strong></a>  <a href="http://www.guoxuedashi.com/so.php?sokeygx=%E5%8F%A4%E4%BB%8A%E5%9B%BE%E4%B9%A6%E9%9B%86%E6%88%90&submit=&kt=1"><strong>古今图书集成</strong></a>
<br>

<a href="http://www.guoxuedashi.com/so.php?sokeygx=%E4%B8%9B%E4%B9%A6%E9%9B%86%E6%88%90&submit=&kt=1">丛书集成</a> 
<a href="http://www.guoxuedashi.com/so.php?sokeygx=%E5%9B%9B%E9%83%A8%E4%B8%9B%E5%88%8A&submit=&kt=1"><strong>四部丛刊</strong></a>  
<a href="http://www.guoxuedashi.com/so.php?sokeygx=%E8%AF%B4%E6%96%87%E8%A7%A3%E5%AD%97&submit=&kt=1">說文解字</a> <a href="http://www.guoxuedashi.com/so.php?sokeygx=%E5%85%A8%E4%B8%8A%E5%8F%A4&submit=&kt=1">三国六朝文</a>
<br><a href="http://www.guoxuedashi.com/so.php?sokeytm=%E6%97%A5%E6%9C%AC%E5%86%85%E9%98%81%E6%96%87%E5%BA%93&submit=&kt=1"><strong>日本内阁文库</strong></a> <a href="http://www.guoxuedashi.com/so.php?sokeytm=%E5%9B%BD%E5%9B%BE%E6%96%B9%E5%BF%97%E5%90%88%E9%9B%86&ka=100&submit=">国图方志合集</a> <a href="http://www.guoxuedashi.com/so.php?sokeytm=%E5%90%84%E5%9C%B0%E6%96%B9%E5%BF%97&submit=&kt=1"><strong>各地方志</strong></a>

</div>
</div>


<div class="sidebar2">
<center>

</center>
</div>
<div class="sidebar greenbar">
<div class="sidebar_title green">四库全书</div>
<div class="sidebar_info">

《四库全书》是中国古代最大的丛书,编撰于乾隆年间,由纪昀等360多位高官、学者编撰,3800多人抄写,费时十三年编成。丛书分经、史、子、集四部,故名四库。共有3500多种书,7.9万卷,3.6万册,约8亿字,基本上囊括了古代所有图书,故称“全书”。<a href="http://www.guoxuedashi.com/SiKuQuanShu/">详细>>
</a>

</div> 
</div>

</div>  <!--end r-->

</div>
<!-- 内容区END --> 

<!-- 页脚开始 -->
<div class="shh">

</div>

<div class="w1180" style="margin-top:8px;">
<center><script src="http://www.guoxuedashi.com/img/plus.php?id=3"></script></center>
</div>
<div class="w1180 foot">
<a href="/b/thanks.php">特别致谢</a> | <a href="javascript:window.external.AddFavorite(document.location.href,document.title);">收藏本站</a> | <a href="#">欢迎投稿</a> | <a href="http://www.guoxuedashi.com/forum/">意见建议</a> | <a href="http://www.guoxuemi.com/">国学迷</a> | <a href="http://www.shuowen.net/">说文网</a><script language="javascript" type="text/javascript" src="https://js.users.51.la/17753172.js"></script><br />
  Copyright &copy; 国学大师 古典图书集成 All Rights Reserved.<br>
  
  <span style="font-size:14px">免责声明:本站非营利性站点,以方便网友为主,仅供学习研究。<br>内容由热心网友提供和网上收集,不保留版权。若侵犯了您的权益,来信即刪。scp168@qq.com</span>
  <br />
ICP证:<a href="http://www.beian.miit.gov.cn/" target="_blank">鲁ICP备19060063号</a></div>
<!-- 页脚END --> 
<script src="http://www.guoxuedashi.com/img/plus.php?id=22"></script>
<script src="http://www.guoxuedashi.com/img/tongji.js"></script>

</body>
</html>
