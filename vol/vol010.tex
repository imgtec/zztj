










 


 
 


 

  
  
  
  
  





  
  
  
  
  
 
  

  

  
  
  



  

 
 

  
   




  

  
  


    資治通鑑卷十      宋司馬光 撰

  胡三省 音註

  漢紀二【起彊圉作噩盡著雍閹茂凡二年}


  太祖高皇帝上之下

  三年冬十月韓信張耳以兵數萬東擊趙趙王及成安君陳餘聞之聚兵井陘口【陘音刑杜佑曰井陘口在鎮州鹿泉縣今謂之土門按宋白續通典鎮州石邑縣有井陘山甚險固又鹿泉縣本漢石邑縣地隋開皇十六年置至德初改名獲鹿又井陘縣穆天子傳天子獵于鉶山即此地注云燕趙謂山脊為陘陘山在縣東南十八里四方高中央下如井故曰井陘}
號二十萬廣武君李左車說成安君曰韓信張耳乘勝而去國遠鬬【謂乘取代之勝勢也說輸芮翻}
其鋒不可當臣聞千里餽糧士有飢色樵蘇後㸑【樵取薪也蘇取草也}
師不宿飽今井陘之道車不得方軌【方軌謂車併行}
騎不得成列行數百里其勢糧食必在其後【鄭康成曰行道曰糧謂糒也止居曰食謂米也}
願足下假臣奇兵三萬人從間路絶其輜重【師古曰間路微路也間古莧翻師古曰輜衣車也重謂載重物車也故行者之資總曰輜重釋名云輜厠也所載衣服雜厠其中重直用翻}
足下深溝高壘勿與戰彼前不得鬬不得還野無所掠不至十日而兩將之頭可致於麾下否則必為二子所禽矣成安君嘗自稱義兵不用詐謀奇計曰韓信兵少而疲如此避而不擊則諸侯謂吾怯而輕來伐我矣韓信使人間視知其不用廣武君策則大喜乃敢引兵遂下未至井陘口三十里止舍【止軍而舍息也舍如字}
夜半傳發選輕騎二千人【傳發傳令軍中使發兵}
人持一赤幟【漢旗幟皆赤幟昌志翻}
從間道萆山而望趙軍【如淳曰萆音蔽依山以自覆蔽也杜佑曰萆山音蔽今名抱犢山在鎮州石邑縣井陘山亦在石邑意間道萆山即此地師古曰蔽隱于山使敵不見}
誡曰趙見我走必空壁逐我若疾入趙壁【若汝也疾速也}
拔趙幟立漢赤幟令其禆將傳餐曰【服䖍曰立駐傳餐食也如淳曰小飯曰餐言破趙乃當共飽食也餐千安翻}
今日破趙會食諸將皆莫信佯應曰諾信曰趙已先據便地為壁且彼未見吾大將旗鼔未肯擊前行【行戶剛翻}
恐吾至阻險而還也【信盖謂趙聚兵塞井陘之口欲俟信出險而後擊之若見前鋒便縱兵接戰則信必將阻險而還師也還音旋又如字}
乃使萬人先行出背水陳【史記正義曰綿蔓水自并州北流入井陘縣界即信背水陳處背蒲妹翻陳讀曰陣}
趙軍望見而大笑平旦信建大將旗皷鼔行出井陘口趙開壁撃之大戰良久於是信與張耳佯棄鼔旗走水上軍【走音奏}
水上軍開入之復疾戰趙果空壁爭漢旗皷逐信耳信耳已入水上軍軍皆殊死戰【師古曰殊絶也言决意必死}
不可敗【敗補邁翻}
信所出奇兵二千騎共候趙空壁逐利則馳入趙壁皆拔趙旗立漢赤幟二千趙軍已不能得信等欲還歸壁壁皆漢赤幟見而大驚以為漢皆已得趙王將矣【將即亮翻}
兵遂亂遁走趙將雖斬之不能禁也于是漢兵夾擊大破趙軍斬成安君泜水上【水經註泜水即井陘山水世謂之鹿泉水東北流屈經陳餘壘又東註綿蔓水師古曰泜音祗又丁計翻又丁禮翻}
禽趙王歇諸將効首虜畢賀因問信曰兵法右倍山陵前左水澤今者將軍令臣等反背水陳曰破趙會食【倍與背同蒲妹翻}
臣等不服然竟以勝此何術也信曰此在兵法顧諸君不察耳兵法不曰陷之死地而後生【孫子九地疾戰則存不戰則亡為死地曹操註曰前有高山後有大水進不得退有礙者}
置之亡地而後存且信非得素拊循士大夫也此所謂驅市人而戰之【師古曰言如忽入市㕓驅其人以赴戰非素所習練者也}
其勢非置之死地使人人自為戰今予之生地皆走寧尚可得而用之乎【予讀曰與下同}
諸將皆服曰善非臣所及也信募生得廣武君者予千金有縛致麾下者信解其縛東鄉坐師事之【予讀曰與鄉讀曰嚮}
問曰僕欲北伐燕東伐齊何若而有功【何若猶言何如也}
廣武君辭謝曰臣敗亡之虜何足以權大事乎【權所以稱物見其輕重也左車蓋謂兵者國之大事如己者敗亡之餘不足以審處其輕重}
信曰僕聞之百里奚居虞而虞亡在秦而秦覇【百里奚虞之大夫虞公不能用以亡秦穆公信而用之遂覇西戎}
非愚于虞而智于秦也用與不用聽與不聽也誠令成安君聽足下計若信者亦已為禽矣以不用足下故信得侍耳【言得侍左右以求教}
今僕委心歸計願足下勿辭廣武君曰今將軍涉西河虜魏王禽夏說東下井陘不終朝而破趙二十萬衆誅成安君名聞海内威震天下農夫莫不輟耕釋耒褕衣甘食【褕音瑜靡也此言當時之人畏信之威聲不能自得其生業皆輟耕釋耒褕靡其衣甘毳其食以苟生于旦夕不復為久遠計}
傾耳以待命者此將軍之所長也然而衆勞卒罷【罷讀曰疲}
其實難用今將軍欲舉倦敝之兵頓之燕堅城之下欲戰不得攻之不拔情見勢屈【兵詭道也乘勢以為用者也見顯露也屈盡也吾之情見則敵知所備勢屈則敵得乘吾之敝矣見賢遍翻屈其勿翻}
曠日持久糧食單竭【單與殫同盡也}
燕既不服齊必距境以自彊燕齊相持而不下則劉項之權未有所分也此將軍所短也善用兵者不以短擊長而以長擊短韓信曰然則何由【由從也言當從何計也}
廣武君對曰方今為將軍計莫如按甲休兵鎮撫趙民百里之内牛酒日至以饗士大夫北首燕路【首式救翻頭之所向曰首}
而後遣辨士奉咫尺之書【師古曰八寸曰咫咫尺者言其簡牘或長咫或長尺喩輕率也}
暴其所長于燕【暴顯也示也露也}
燕必不敢不聽從燕已從而東臨齊雖有智者亦不知為齊計矣如是則天下事皆可圖也兵固有先聲而後實者此之謂也韓信曰善從其策發使使燕燕從風而靡遣使報漢且請以張耳王趙漢王許之楚數使奇兵渡河擊趙【數所角翻}
張耳韓信往來救趙因行定趙城邑發兵詣漢 甲戍晦【月盡為晦}
日有食之 十一月癸卯晦日有食之隨何至九江九江太宰主之【此太宰非周官之太宰漢奉常屬官有太宰師古曰具食之官信使入國必使人為之主時布使太宰主何也}
三日不得見隨何說太宰曰王之不見何必以楚為彊漢為弱也此臣之所以為使【說輸芮翻下同使疏吏翻}
使何得見言之而是大王所欲聞也言之而非使何等二十人伏斧質九江市足以明王倍漢而與楚也【倍與背同蒲妹翻}
太宰乃言之王王見之隨何曰漢王使臣敬進書大王御者竊怪大王與楚何親也九江王曰寡人北鄉而臣事之隨何曰大王與項王俱列為諸侯北鄉而臣事之者【鄉讀曰嚮下同}
必以楚為彊可以託國也項王伐齊身負版築為士卒先【李奇曰版墻版也築杵也}
大王宜悉九江之衆身自將之為楚前鋒【將即亮翻}
今乃發四千人以助楚夫北面而臣事人者固若是乎漢王入彭城項王未出齊也大王宜悉九江之兵渡淮日夜會戰彭城下大王乃撫萬人之衆無一人渡淮者垂拱而觀其孰勝【垂拱者垂衣拱手也}
夫託國於人者固若是乎大王提空名以鄉楚而欲厚自託臣竊為大王不取也然而大王不背楚者以漢為弱也夫楚兵雖彊天下負之以不義之名以其背盟約而殺義帝也【背蒲妹翻}
漢王收諸侯還守成臯滎陽下蜀漢之粟深溝壁壘分卒守徼乘塞【徼循也几邉謂之邉徼蓋使人循徼機禁姦非因以名之索隱曰徼謂邉境亭障以徼繞邉陲常守之也徼吉弔翻乘登也登塞垣而守之}
楚人深入敵國八九百里【言楚自彭城至滎陽成臯中間有梁地間之彭越時反梁地是楚之敵國也故云深入敵國八九百里}
老弱轉糧千里之外漢堅守而不動楚進則不得攻則不能解故曰楚兵不足恃也使楚勝漢則諸侯自危懼而相救夫楚之彊適足以致天下之兵耳故楚不如漢其勢易見也今大王不與萬全之漢而自託於危亡之楚臣竊為大王惑之【易以䜴翻為于偽翻}
臣非以九江之兵足以亡楚也大王發兵而倍楚項王必留留數月漢之取天下可以萬全臣請與大王提劒而歸漢漢王必裂地而封大王又况九江必大王有也九江王曰請奉命隂許畔楚與漢未敢洩也楚使者在九江舍傳舍【傳舍客舍也前客舍之而去後客復來舍之傳相受也故謂之傳舍傳直戀翻}
方急責布發兵隨何直入坐楚使者上曰九江王已歸漢楚何以得發兵布愕然楚使者起何因說布曰事已構【師古曰構結也言背楚之事已結成}
可遂殺楚使者無使歸而疾走漢并力布曰如使者教於是殺楚使者因起兵而攻楚楚使項聲龍且攻九江【且子余翻龍姓且名}
數月龍且破九江軍布欲引兵走漢恐楚兵殺之乃間行與何俱歸漢十二月九江王至漢漢王方踞床洗足召布入見【見賢遍翻}
布大怒悔來欲自殺及出就舍帳御飲食從官皆如漢王居布又大喜過望【師古曰高帝以布先久為王恐其意自尊大故峻其禮令布折服已而美其帷帳厚其飲食多其從官以悦其心此權道也帳若今之帳設也御服御也從才用翻}
於是乃使人入九江楚已使項伯收九江兵盡殺布妻子布使者頗得故人幸臣將衆數千人歸漢漢益九江王兵與俱屯成臯楚數侵奪漢甬道【數所角翻}
漢軍乏食漢王與酈食其謀橈楚權【食其音異基橈女教翻弱也字從木}
食其曰㫺湯伐桀封其後於杞武王伐紂封其後於宋今秦失德棄義侵伐諸侯滅其社稷使無立錐之地陛下誠能復立六國之後此其君臣百姓必皆戴陛下之德莫不嚮風慕義願為臣妾德義已行陛下南鄉稱覇楚必歛袵而朝【袵衣襟也鄉讀曰嚮朝直遥翻}
漢王曰善趣刻印先生因行佩之矣【言將使食其行使六國授之以印而使佩之趣讀曰促下同}
食其未行張良從外來謁漢王方食曰子房前【子房張良字}
客有為我計橈楚權者具以酈生語告良曰何如良曰誰為陛下畫此計者陛下事去矣漢王曰何哉對曰臣請借前箸為大王籌之【時漢王方食故良言願借食前之箸就用指畫鄭玄曰今人或謂箸為挾提}
㫺湯武封桀紂之後者度能制其死生之命也【度徒洛翻}
今陛下能制項籍之死命乎其不可一也武王入殷表商容之閭釋箕子之囚封比干之墓【商容殷賢人里門曰閭表顯異也紂囚箕子殺比干武王克殷釋箕子囚封比干墓韓詩外傳曰商容執羽籥馮於馬徒欲以化紂而不能遂去伏於太行山武王欲以為三公辭而不受鄭玄曰商家樂官知禮容所以禮署稱容臺}
今陛下能乎其不可二也發巨橋之粟散鹿臺之錢【服䖍曰巨橋倉名許慎曰鉅鹿之大橋有漕粟杜佑曰鉅橋倉在今廣平郡曲周縣臣瓚曰鹿臺今在朝歌城中劉向曰其大三里高千尺}
以賜貧窮今陛下能乎其不可三也殷事已畢偃革為軒【蘇林曰革者兵車也軒者朱軒皮軒也謂廢兵車而用乘車也說文曰軒曲周屏車如淳曰革者革車也軒者赤黻乘軒也偃武備而治禮樂也}
倒載干戈示天下不復用兵今陛下能乎其不可四也【復扶又翻}
休馬華山之陽示以無為今陛下能乎其不可五也【華戶化翻}
放牛桃林之隂【晉灼曰桃林在弘農閺鄉南谷中山海經曰夸父之山北有林焉名曰桃林廣圍三百里十三州記弘農有桃丘聚即桃林也師古曰桃林山谷在閺鄉縣東南西南去湖城縣三十五里}
以示不復輸積今陛下能乎其不可六也天下游士離其親戚弃墳墓去故舊從陛下游者徒欲日夜望咫尺之地今復立六國之後天下游士各歸事其主從其親戚反其故舊墳墓陛下誰與取天下乎其不可七也且夫楚唯無彊六國立者復橈而從之【服䖍曰惟當使楚無彊彊則六國弱而從之晉灼曰當今惟楚大無有彊之者若復立六國六國皆橈而從之陛下安得而臣之乎}
陛下焉得而臣之其不可八也誠用客之謀陛下事去矣漢王輟食吐哺罵曰【哺音步食在口中者}
豎儒幾敗而公事【而汝也公尊稱也高祖嫚罵人率曰而公乃公盖自尊辭幾居依翻}
令趣銷印

  荀悦論曰夫立策决勝之術其要有三一曰形二曰勢三曰情形者言其大體得失之數也勢者言其臨時之宜進之機也情者言其心志可否之實也故策同事等而功殊者三術不同也初張耳陳餘說陳涉以復六國自為樹黨【事見七卷秦二世元年}
酈生亦說漢王所以說者同而得失異者陳涉之起天下皆欲亡秦而楚漢之分未有所定今天下未必欲亡項也故立六國於陳涉所謂多已之黨而益秦之敵也且陳涉未能專天下之地也所謂取非其有以與於人行虚惠而獲實福也立六國於漢王所謂割已之有而以資敵設虛名而受實禍也此同事而異形者也及宋義待秦趙之斃【事見八卷秦二世三年}
與昔卞莊刺虎同說者也【卞莊子刺虎管豎子止之曰兩虎方食牛牛甘必爭鬬則大者傷小者亡從傷而刺一舉必有兩獲莊子然之果獲二虎}
施之戰國之時鄰國相攻無臨時之急則可也戰國之立其日久矣一戰勝敗未必以存亡也其勢非能急於亡敵國也進乘利自保故累力待時乘敵之斃其勢然也今楚趙所起其與秦勢不並立安危之機呼吸成變進則定功則受禍此同事而異勢者也伐趙之役韓信軍於泜水之上而趙不能敗【事見上卷三年}
彭城之難漢王戰于睢水之上士卒皆赴入睢水而楚兵大勝【事見上卷二年難乃旦翻}
何則趙兵出國迎戰見可而進知難而懷内顧之心無出死之計韓信軍孤在水上士卒必死無有二心此信之所以勝也漢王深入敵國置酒高會士卒逸豫戰心不固楚以彊大之威而喪其國都【喪息浪翻}
士卒皆有憤激之氣救敗赴亡之急以决一旦之命此漢之所以敗也且韓信選精兵以守而趙以内顧之士攻之項羽選精兵以攻而漢以怠惰之卒應之此同事而異情者也故曰權不可豫設變不可先圖與時遷移應物變化設策之機也

  漢王謂陳平曰天下紛紛何時定乎陳平曰項王骨鯁之臣亞父鍾離昩龍且周殷之屬【鍾離古鍾離子之後以國為姓龍姓出于龍伯氏又曰出于舜納言之龍師古曰昧莫曷翻其字從本末之末且子余翻}
不過數人耳大王誠能捐數萬斤金行反間間其君臣以疑其心【間古莧翻}
項王為人意忌信讒必内相誅漢因舉兵而攻之破楚必矣漢王曰善乃出黄金四萬斤與平恣所為不問其出入平多以金縱反間於楚軍宣言諸將鍾離昧等為項王將功多矣然而終不得裂地而王欲與漢為一以滅項氏而分王其地項羽果意不信鍾離昧等夏四月楚圍漢王於滎陽急漢王請和割滎陽以西者為漢亞父勸羽急攻滎陽漢王患之項羽使使至漢陳平使為大牢具【大讀曰太古者諸侯遣使交聘其牢禮各如其命數以三牲具為一牢秦滅古法軍興之時不能備古之牢禮故以太牢具為盛禮孔頴逹曰按周禮膳夫王日一舉鼎十有二物謂太牢也是周公制禮天子日食太牢則諸侯日食少牢大夫日食特牲士日食特豚至後世衰亂玉藻云天子日食少牢朔月太牢諸侯日食特牲朔月少牢則知大夫日食特豚朔月特牲士日食無文朔月特豚故内則云見子具朔食註云天子太牢諸侯少牢大夫特豕士特豚諸侯祭以太牢得殺牛諸侯之大夫祭以少牢得殺羊天子大夫祭亦得殺牛其諸侯及大夫饗食賓得用牛也故大行人掌客諸侯待賓皆用牛也公食大夫禮大夫食賓禮亦用牛也}
舉進見楚使即佯驚曰吾以為亞父使乃項王使復持去更以惡草具進楚使【服䖍曰去肴肉更以惡草之具惡粗惡草草率也}
楚使歸具以報項王項王果大疑亞父亞父欲急攻下滎陽城項王不信不肯聽亞父聞項王疑之乃怒曰天下事大定矣君王自為之願賜骸骨歸未至彭城疽發背而死【疽千余翻癰瘡也}
五月將軍紀信言於漢王曰事急矣臣請誑楚【誑居况翻欺也}
王可以間出【間古莧翻}
於是陳平夜出女子東門二千餘人楚因四面擊之紀信乃乘王車黄屋左纛【李斐曰天子車以黄繒為盖裏纛羽幢也在乘輿車衡左方上柱之蔡邕曰以犛牛尾為之大如斗或在騑頭或在衡應劭曰雉尾為之在左驂當鑣上師古曰應說非爾雅翼犛西南夷長髦牛也似牛而四節腹下及肘皆有赤毛長尺餘而尾尤佳其大如斗天子之車左纛以此牛尾為之繫之左騑馬軛上蓋馬在中曰服在外曰騑騑即驂也安最外馬頭上以亂馬目不令相見也纛徒倒翻又音毒}
曰食盡漢王降楚皆呼萬歲之城東觀以故漢王得與數十騎出西門遁去令韓王信與周苛魏豹樅公守滎陽【樅千容翻}
羽見紀信問漢王安在曰已出去矣羽燒殺信周苛樅公相謂曰反國之王難與守城因殺魏豹漢王出滎陽至成臯入關收兵欲復東轅生說漢王曰【轅姓也姓譜陳大夫轅濤塗之後以其所本考之亦與爰袁二姓通}
漢與楚相距滎陽數歲漢常困願君王出武關項王必引兵南走王深壁勿戰令滎陽成臯間且得休息使韓信等得安輯河北趙地連燕齊【師古曰輯與集同謂和合也詩序曰勞來還定安集之春秋左氏傳曰羣臣輯睦他皆類此}
君王乃復走滎陽如此則楚所備者多力分漢得休息復與之戰破之必矣漢王從其計出軍宛葉間【班志二縣屬南陽郡史記正義曰宛鄧州縣葉汝州縣宛於元翻葉式涉翻}
與黥布行收兵羽聞漢王在宛果引兵南漢王堅壁不與戰漢王之敗彭城解而西也彭越皆亡其所下城獨將其兵北居河上常往來為漢游兵擊楚絶其後糧是月彭越渡睢與項聲薛公戰下邳破殺薛公【睢音雖}
羽乃使終公守成臯【終姓也姓譜曰陸終之後}
而自東擊彭越漢王引兵北擊破終公復軍成臯六月羽已破走彭越聞漢復軍成臯乃引兵西拔滎陽城生得周苛羽謂苛為我將以公為上將軍封三萬戶周苛罵曰若不趨降漢今為虜矣若非漢王敵也羽烹周苛并殺樅公而虜韓王信遂圍成臯漢王逃【漢書逃作跳如淳音逃史記項羽紀作逃索隱曰徒彫翻晉灼曰跳獨出意如淳曰逃謂走也余謂左氏傳例民逃其上曰潰在上曰逃太史公蓋用此例温公仍之逃當如字}
獨與公共車出成臯玉門【張晏曰玉門成臯北門}
北渡河宿小修武傳舍【晉灼曰在大修武城東}
晨自稱漢使馳入趙壁張耳韓信未起即其卧内奪其印符以麾召諸將易置之信耳起乃知漢王來大驚漢王既奪兩人軍即令張耳循行備守趙地【行下孟翻}
拜韓信為相國收趙兵未發者擊齊諸將稍稍得出成臯從漢王楚遂拔成臯欲西漢使兵距之鞏【班志鞏縣屬河南郡即東周君所居汝洛地圖云鞏固也鞏縣在洛水之間言四面有山可以鞏固}
令其不得西 秋七月有星孛于大角【隋天文志孛彗之屬也偏指曰彗芒氣四出曰孛孛者孛孛然非常惡氣之所生也内不有大亂必有大兵天下合謀暗蔽不明有所傷害晏子曰君若不改孛星將出彗何懼乎由是言之災甚於彗孛蒲内翻又蒲没翻班志房南衆星曰騎官左角理右角將大角者天王帝坐廷}
 臨江王敖薨子尉嗣 漢王得韓信軍復大振八月引兵臨河南鄉軍小修武欲復與楚戰【鄉讀曰嚮復扶又翻}
郎中鄭忠說止漢王【漢制議郎中郎秩比六百石侍郎比四百石郎中比三百石皆屬郎中令說式芮翻}
使高壘深塹勿與戰【塹七艶翻}
漢王聽其計使將軍劉賈盧綰將卒二萬人【綰烏板翻}
騎數百度白馬津入楚地佐彭越燒楚積聚以破其業【師古曰積聚所蓄軍糧芻藁之屬也積子賜翻聚才喻翻}
無以給項王軍食而已楚兵擊劉賈賈輒堅壁不肯與戰而與彭越相保 彭越攻徇梁地下睢陽外黄等十七城【睢陽秦縣屬碭郡漢屬梁國故微子所封國也唐為宋州宋城縣杜佑曰漢外黄故城在陳留郡雍丘縣東春秋齊桓公會諸侯於葵丘即此}
九月項王謂大司馬曹咎曰謹守成臯即漢王欲挑戰【挑徒了翻}
慎勿與戰勿令得東而已我十五日必定梁地復從將軍羽引兵東行擊陳留外黄睢陽等城皆下之 漢王欲捐成臯以東屯鞏洛以距楚酈生曰臣聞知天之天者王事可成王者以民為天而民以食為天【大戴禮曰食穀者知慧而巧古史考曰古者茹毛飲血燧人氏鑚火而人始裹肉而燔之曰炮神農時人方食穀加米於燒石之上而食之及黄帝時始有釜甑火食之道成矣}
夫敖倉天下轉輸久矣臣聞其下乃有藏粟甚多楚人拔滎陽不堅守敖倉乃引而東令適卒分守成臯【適讀曰讁卒謂卒之有罪讁者所謂讁戊也}
此乃天所以資漢也方今楚易取而漢反却【易以䜴翻}
自奪其便臣竊以為過矣且兩雄不俱立楚漢久相持不决海内揺蕩農夫釋耒【耒手耕曲木也}
工女下機天下之心未有所定也願足下急復進兵收取滎陽據敖倉之粟塞成臯之險杜大行之道距蜚狐之口【如淳曰上黨壺關也瓚曰飛狐口在代郡師古曰瓚說是壺關無飛狐之名地道記恒山在上曲陽縣西北百四十里北行四百五十里得恒山岋號飛狐口北則代郡也水經註代郡南四十里有蜚狐關史記正義曰按蔚州飛狐縣北百五十里有秦漢故代郡城西南有山俗號蜚狐口塞悉則翻行戶剛翻}
守白馬之津以示諸侯形制之勢【謂因地形而據之以制敵}
則天下知所歸矣王從之乃復謀取敖倉食其又說王曰方今燕趙已定唯齊未下諸田宗彊負海岱阻河濟【齊地東至海南至太山故曰負海岱西阻清濟北阻濁河故曰阻河濟濟子禮翻}
南近於楚【近其靳翻}
人多變詐足下雖遣數萬師未可以歲月破也臣請得奉明詔說齊王使為漢而稱東藩【考異曰史記漢書皆以食其勸取敖倉及請說齊合為一事獨劉向新序分為二臣謂分為二者是}
上曰善乃使酈生說齊王曰王知天下之所歸乎王曰不知也天下何所歸酈生曰歸漢曰先生何以言之曰漢王先入咸陽項王負約王之漢中項王遷殺義帝漢王聞之起蜀漢之兵擊三秦出關而責義帝之處收天下之兵立諸侯之後降城即以侯其將得賂即以分其士與天下同其利豪英賢才皆樂為之用【樂音洛}
項王有倍約之名殺義帝之負【毛晃曰背恩亡德曰負倍與背同蒲妹翻}
於人之功無所記於人之辠無所忘戰勝而不得其賞拔城而不得其封非項氏莫得用事天下畔之賢才怨之而莫為之用故天下之事歸於漢王可坐而策也夫漢王發蜀漢定三秦涉西河破北魏【河自砥柱以上龍門以下為西河索隱曰北魏謂魏王豹豹國於河北故也亦謂之西魏以大梁於安邑為東也}
出井陘誅成安君此非人之力也天之福也今已據敖倉之粟塞成臯之險守白馬之津杜大行之阪距蜚狐之口天下後服者先亡矣【酈生之說形格勢禁之說也蓋據敖倉塞成臯則項羽不能西守白馬杜太行距蜚狐則河北燕趙之地盡為漢有齊楚將安歸乎白馬津在唐滑州太行阪在唐澤州界杜佑曰蔚州飛狐縣漢廣昌縣地飛狐口在縣北即漢之飛狐道通媯川郡懷戎縣}
王疾先下漢王齊國可得而保也不然危亡可立而待也先是齊聞韓信且東兵使華無傷田解將重兵屯歷下以距漢【先悉薦翻華戶化翻姓也姓譜宋華父督始立華氏張楫曰濟南歷山之下余據酈食其傳曰軍於歷城則歷下即濟南郡歷城縣}
及納酈生之言遣使與漢平乃罷歷下守戰備與酈生日縱酒為樂【樂音洛}
韓信引兵東未度平原聞酈食其已說下齊欲止辨士蒯徹說信曰將軍受詔擊齊而漢獨發間使下齊【間古莧翻使疏吏翻}
寧有詔止將軍乎何以得毋行也且酈生一士伏軾掉三寸之舌【軾車前横木人所憑者掉徒釣翻揺也}
下齊七十餘城將軍以數萬衆歲餘乃下趙五十餘城為將數歲反不如一豎儒之功乎於是信然之遂渡河

  四年冬十月信襲破齊歷下軍遂至臨淄齊王以酈生為賣已乃烹之引兵東走高密【高密縣在膠西宣帝本始元年為高密國宋白曰高密春秋時晏平仲所食邑}
使使之楚請救田横走博陽【此據史記也班書作横走博博陽近清河博關此正韓信自趙進兵之路臨淄既破君相皆出走其後韓信既虜田廣於濰水灌嬰又敗田横於嬴下嬴縣亦屬太山郡括地志故嬴城在兖州博城縣東北百里唐之博城漢太山之博縣此博陽即博城之陽}
守相田光走城陽【相息亮翻}
將軍田既軍於膠東【括地志即墨故城在萊州膠水縣南六十里古齊地漢為膠東國以其地在膠水之東也}
 楚大司馬咎守成臯漢數挑戰【數所角翻挑徒了翻}
楚軍不出使人辱之數日咎怒渡兵汜水【張晏曰汜水在濟隂界如淳曰汜音祀左傳曰鄙在鄭地汜臣瓚曰高祖攻曹咎於成臯咎渡汜水而戰今成臯城東汜水是也師古曰瓚說得之此水不在濟隂也鄙在鄭地汜釋者云在襄城則亦非此汜水舊讀音凡今彼鄉人呼之音祀索隱曰此水今見名汜水音似臣瓚說是張晏曰在濟隂亦未全失按古濟水當此截河而南又東流溢為滎澤水南曰隂此亦在濟之隂非彼濟隂郡耳括地志汜水源出洛州汜水縣東南三十二里方山山海經浮戲之山汜水出焉}
士卒半渡漢擊之大破楚軍盡得楚國金玉貨賂咎及司馬欣皆自剄汜水上漢王引兵渡河復取成臯軍廣武【孟康曰於滎陽築兩城相對為廣武在敖倉西三皇山上括地志東廣武西廣武在鄭州滎陽縣西二十里戴延之西征記曰三皇山上有二城東曰東廣武西曰西廣武各在一山頭相去百步汴水從廣澗中東南流今涸無水城各有三面在敖倉西郭緣生述征記曰一澗横絶上過名曰廣武相對皆立城塹遂號東西廣武}
就敖倉食項羽下梁地十餘城聞成臯破乃引兵還漢軍方圍鍾離昧於滎陽東聞羽至盡走險阻羽亦軍廣武與漢相守數月楚軍食少項王患之乃為俎置太公其上告漢王曰今不急下吾烹太公漢王曰吾與羽俱北面受命懷王約為兄弟吾翁即若翁必欲烹而翁幸分我一桮羮【如淳曰俎高几之上也李奇曰軍中巢櫓謂之俎師古曰俎者所以薦肉示欲烹之故置俎上如說是俎在呂翻方言周晉秦隴謂父為翁若汝也而亦汝也古者以桮盛羮今之側盃有兩耳者也}
項王怒欲殺之項伯曰天下事未可知且為天下者不顧家雖殺之無益祇益禍耳項王從之項王謂漢王曰天下匈匈數歲者【師古曰匈匈喧擾之意公休許容翻}
徒以吾兩人耳願與漢王挑戰决雌雄毋徒苦天下之民父子為也漢王笑謝曰吾寧鬬智不能鬬力項王三令壯士出挑戰漢有善騎射者樓煩輒射殺之【應劭曰樓煩胡人也李奇曰後為縣屬雁門此縣人善騎射謂士為樓煩取其稱耳未必樓煩人也師古曰李奇說是射而亦翻}
項王大怒乃自被甲持戟挑戰樓煩欲射之項王瞋目叱之【瞋昌眞翻}
樓煩目不敢視手不敢發遂走還入壁不敢復出漢王使人間問之【間問微問也間工莧翻}
乃項王也漢王大驚於是項王乃即漢王【即就也從也}
相與臨廣武間而語羽欲與漢王獨身挑戰漢王數羽曰【數所具翻}
羽負約王我於蜀漢罪一矯殺卿子冠軍罪二救趙不還報而擅刼諸侯兵入關罪三燒秦宫室掘始皇帝冢收私其財罪四【收私者收取其財以為私有}
殺秦降王子嬰罪五詐阬秦子弟新安二十萬罪六王諸將善地而徙逐故王罪七出逐義帝彭城自都之奪韓王地并王梁楚多自與罪八使人隂殺義帝江南罪九為政不平主約不信天下所不容大逆無道罪十也吾以義兵從諸侯誅殘賊使刑餘罪人擊公何苦乃與公挑戰羽大怒伏弩射中漢王漢王傷胷乃捫足曰虜中吾指【捫音門摸也師古曰傷胷而捫足者以安衆也中竹仲翻}
漢王病創卧【創初良翻}
張良彊請漢王起行勞軍以安士卒【彊其兩翻勞力到翻}
毋令楚乘勝漢王出行軍【行下孟翻}
疾甚因馳入成臯 韓信已定臨淄遂東追齊王項王使龍且將兵號二十萬以救齊與齊王合軍高密客或說龍且曰漢兵遠鬭窮戰其鋒不可當齊楚自居其地兵易敗散【孫子九地諸侯自戰其地為散地曹操曰士卒戀士道近易散者也易以䜴翻下同}
不如深壁令齊王使其信臣招所亡城【信臣常所親信之臣也}
亡城聞王在楚來救必反漢漢兵二千里客居齊地齊城皆反之其勢無所得食可無戰而降也龍且曰吾平生知韓信為人易與耳寄食於漂母無資身之策受辱於袴下無兼人之勇【事見上卷元年}
不足畏也且夫救齊不戰而降之吾何功今戰而勝之齊之半可得也十一月齊楚與漢夾濰水而陳【徐廣曰濰水出東莞而東北流至北海都昌縣入海索隱曰濰水出琅邪箕縣東北至都昌入海水經註濰水逕高密縣故城西韓信與龍且夾水而陳即此處濰音維陳讀曰陣}
韓信夜令人為萬餘囊滿盛沙壅水上流【盛時征翻}
引軍半渡擊龍且佯不勝還走龍且果喜曰固知信怯也遂追信信使人决壅囊水大至龍且軍大半不得渡即急擊殺龍且水東軍散走齊王廣亡去信遂追北至城陽虜齊王廣【史記正義曰城陽雷澤縣是也在濮州東南九十一里予據班志濟隂郡城陽縣雷澤在西北此梁地也自濰水追北至城陽此乃漢城陽國之地正義此誤與上卷二年田横起城陽同}
漢將灌嬰追得齊守相田光進至博陽田横聞齊王死自立為齊王還擊嬰嬰敗横軍於嬴下田横亡走梁歸彭越嬰進擊齊將田吸於千乘【千乘縣屬北海郡高祖分置千乘郡括地志千乘故城在淄州高苑縣北二十五里乘䋲證翻}
曹參擊田既於膠東皆殺之盡定齊地立張耳為趙王 漢王疾愈西入關至櫟陽梟故塞王欣頭櫟陽市【師古曰縣首於木上曰梟索隱曰欣自剄於氾水上今梟之櫟陽者以其故都故梟以示之也}
留四日復如軍軍廣武 韓信使人言漢王曰齊偽詐多變反覆之國也南邉楚【師古曰邊近也}
請為假王以鎮之漢王發書大怒罵曰吾困於此旦暮望若來佐我乃欲自立為王張良陳平躡漢王足因附耳語曰漢方不利寧能禁信之自王乎不如因而立之善遇使自為守不然變生漢王亦悟因復罵曰大丈夫定諸侯即為真王耳何以假為春二月遣張良操印立韓信為齊王徵其兵擊楚【操七刀翻}
 項王聞龍且死大懼使盱眙人武涉【盱眙音吁怡}
往說齊王信曰天下共苦秦久矣相與勠力擊秦秦已破計功割地分土而王之以休士卒今漢王復興兵而東侵人之分【分扶問翻}
奪人之地已破三秦引兵出關收諸侯之兵以東擊楚其意非盡吞天下者不休其不知厭足如是甚也【厭于鹽翻}
且漢王不可必身居項王掌握中數矣【數所角翻史記正義色庾翻}
項王憐而活之然得脱輒倍約【倍蒲妹翻下同}
復擊項王其不可親信如此今足下雖自以漢王為厚交為之盡力用兵必終為所禽矣足下所以得須臾至今者以項王尚存也當今二王之事權在足下足下右投則漢王勝左投則項王勝項王今日亡則次取足下足下與項王有故何不反漢與楚連和參分天下王之【參分即三分}
今釋此時而自必於漢以擊楚且為智者固若此乎韓信謝曰臣事項王官不過郎中位不過執戟【郎中執戟宿衛信先仕楚為郎中故云然}
言不聽畫不用故倍楚而歸漢【倍蒲妹翻下同}
漢王授我上將軍印予我數萬衆【予讀曰與}
解衣衣我推食食我【衣衣下於既翻推吐雷翻食食下祥吏翻}
言聽計用故吾得以至於此夫人深親信我我倍之不祥雖死不易幸為信謝項王武涉已去蒯徹知天下權在信乃以相人之術說信曰僕相君之面不過封侯又危不安相君之背貴乃不可言【以微言動信言背漢則大貴也相息亮翻}
韓信曰何謂也蒯徹曰天下初發難也【難乃旦翻}
憂在亡秦而已【師古曰志在滅秦所憂者唯此}
今楚漢分爭使天下之人肝膽塗地父子暴骸骨於中野不可勝數【暴步木翻又如字凡暴露之暴皆同勝音升}
楚人起彭城轉鬬逐北乘利席卷威震天下然兵困于京索之間迫西山而不能進者三年於此矣漢王將十萬之衆距鞏雒阻山河之險一日數戰無尺寸之功折北不救【折挫也北奔也不救者不能自救也折而設翻}
此所謂智勇俱困者也百姓罷極怨望無所歸倚【罷讀曰疲}
以臣料之其勢非天下之賢聖固不能息天下之禍當今兩主之命縣於足下【縣讀曰懸}
足下為漢則漢勝與楚則楚勝誠能聽臣之計莫若兩利而俱存之參分天下鼎足而居其勢莫敢先動夫以足下之賢聖有甲兵之衆據彊齊從趙燕出空虛之地而制其後因民之欲西鄉為百姓請命【師古曰齊國在東故曰西鄉止楚漢之戰闘士卒不死亡故曰請命鄉讀曰嚮下同}
則天下風走而響應矣孰敢不聽割大弱彊以立諸侯諸侯已立天下服聽而歸德於齊案齊之故有膠泗之地【膠泗二水名}
深拱揖讓則天下之君王相率而朝於齊矣【師古曰深拱高拱也朝直遥翻}
蓋聞天與弗取反受其咎時至不行反受其殃願足下熟慮之韓信曰漢王遇我甚厚吾豈可鄉利而倍義乎蒯生曰始常山王成安君為布衣時相與為刎頸之交後爭張黶陳澤之事常山王殺成安君泜水之南頭足異處此二人相與天下至驩也然而卒相禽者何也【卒子恤翻}
患生於多欲而人心難測也今足下欲行忠信以交於漢王必不能固於二君之相與也而事多大於張黶陳澤者故臣以為足下必漢王之不危已亦誤矣大夫種存亡越覇句踐立功成名而身死亡野獸盡而獵狗烹夫以交友言之則不如張耳之與成安君者也以忠信言之則不過大夫種之於句踐也【種章勇翻句音鈎}
此二者足以觀矣願足下深慮之且臣聞勇略震主者身危功蓋天下者不賞今足下戴震主之威挾不賞之功歸楚楚人不信歸漢漢人震恐足下欲持是安歸乎韓信謝曰先生且休矣吾將念之後數日蒯徹復說曰夫聽者事之候也【師古曰謂能聽善謀也復扶又翻}
計者事之機也聽過計失而能久安者鮮矣【鮮息善反}
故知者决之斷也【斷丁亂反}
疑者事之害也審毫釐之小計【毫長毛也十毫為釐}
遺天下之大數智誠知之决弗敢行者百事之禍也夫功者難成而易敗時者難得而易失也時乎時乎不再來韓信猶豫不忍倍漢又自以為功多漢終不奪我齊遂謝蒯徹【謝去辭之使去也}
因去佯狂為巫 秋七月立黥布為淮南王 八月北貉燕人來致梟騎助漢【應劭曰北貉國也梟健也張晏曰梟勇也若六博之梟也師古曰貉在東北方三韓之屬皆貉類也蓋貉人及燕皆來助漢孔頴逹曰經傳說貊多是東夷故職方掌九夷九貊鄭志荅趙商云九貊即九夷也又周官貊隸注云征東北夷所獲貉讀與貊同}
 漢王下令軍士不幸死者吏為衣衾棺歛轉送其家【棺工喚翻歛力贍翻與作衣衾而歛尸於棺也轉送傳送也}
四方歸心焉 是歲以中尉周昌為御史大夫【班表中尉秦官掌徼循京師武帝太初元年更名執金吾}
昌苛從弟也【從才用翻}
 項羽自知少助食盡韓信又進兵擊楚羽患之漢遣侯公說羽請太公【太公呂后為楚所得見上卷三年}
羽乃與漢約中分天下割洪溝以西為漢以東為楚【文頴曰於滎陽下引河東南為洪溝以通宋鄭陳蔡曹衛與濟汝淮泗會于楚即今官渡水也應劭曰滎陽東南二十里蓋引河東南入淮泗也張華曰大梁城在浚儀縣北縣西北渠水東經此城南又北屈分為二渠其一渠東南流始皇鑿之引河水以灌大梁謂之洪溝其一渠東經陽武縣南為官渡水杜佑曰鄭州滎陽縣西有鴻溝楚漢分境之所}
九月楚歸太公呂后引兵解而東歸漢王欲西歸張良陳平說曰漢有天下太半【韋昭曰凡數三分有二為太半有一分為少半}
而諸侯皆附楚兵疲食盡此天亡之時也今釋弗擊此所謂養虎自遺患也【史記正義遺唯季翻予謂音如字亦通遺留也}
漢王從之

  資治通鑑卷十  
    


 


 



 

 
  







 


  
  
 
 
 


  

 















	
	









































 
  



















 





 












  
  
  

 





