資治通鑑卷二百八十一
宋 司馬光 撰

胡三省 音註

後晉紀二|{
	起彊圉作噩盡著雍閹茂凡二年}


高祖聖文章武明德孝皇帝上之下

天福二年春正月乙卯日有食之 |{
	考異曰實録正月甲寅朔乙卯日食十國紀年蜀乙卯朔日食蓋晉人避三朝日食而改歷耳}
詔以前北面招討指揮使安重榮|{
	此以在晉陽圍城中所授安重榮軍職言也故曰前重直龍翻}
為成德節度使|{
	時祕瓊自為成德留後以安重榮代之}
以祕瓊為齊州防禦使|{
	祕姓也漢魏之間有南安祕宜}
遣引進使王景崇諭瓊以利害重榮與契丹將趙思温偕如鎮州瓊不敢拒命|{
	畏契丹也}
丙辰重榮奏已視事|{
	為安重榮以成德反張本}
景崇邢州人也 契丹以幽州為南京|{
	歐史曰以幽州為燕京參考趙思温為留守事則南京為是}
李崧呂琦逃匿於伊闕民間帝以始鎮河東崧有力焉德之|{
	李崧議以帝鎮河東事見二百七十八卷唐明宗長興三年}
亦不責琦|{
	李崧呂琦建和契丹以制河東之議見上卷上年三月}
乙丑以琦為祕書監丙寅以崧為兵部侍郎判戶部初天雄節度使兼中書令范延光微時有術士張生語之云|{
	語牛倨翻}
必為將相延光旣貴信重之延光嘗夢蛇自臍入腹以問張生張生曰蛇者龍也帝王之兆延光由是有非望之志唐潞王素與延光善及趙德鈞敗延光自遼州引兵還魏州|{
	趙德鈞敗見上卷上年閏十一月范延光屯遼州見上年十月其還魏州亦必在閏十一月}
奉表請降|{
	降戶江翻}
内不自安以書濳結祕瓊欲與之為亂瓊受其書不報延光恨之瓊將之齊過魏境延光欲滅口且利其貨遣兵邀之于夏津殺之|{
	為范延光以魏反復以貨為楊光遠所殺張本夏津古鄃縣唐天寶元年更名夏津屬貝州宋以夏津屬北京在京東北二百五十里夏戶雅翻}
丁卯延光奏稱夏津捕盜兵誤殺瓊帝不問戊寅以李崧為中書侍郎同平章事充樞密使桑維翰兼樞密使時晉新得天下藩鎮多未服從或雖服從反仄不安兵火之餘府庫殫竭民間困窮而契丹徵求無厭|{
	厭於鹽翻}
維翰勸帝推誠弃怨以撫藩鎮卑辭厚禮以奉契丹訓卒繕兵以修武備務農桑以實倉廪通商賈以豐貨財數年之間中國稍安|{
	史言桑維翰有益于石晉草創之初者如此賈音古}
吳太子璉納齊王知誥女為妃|{
	璉力展翻}
知誥始建太廟社稷改金陵為江寧府|{
	先是吳以昇州為金陵府今復更名}
牙城曰宫城廳堂曰殿以左右司馬宋齊丘徐玠為左右丞相馬步判官周宗内樞判官黟人周廷玉為内樞使|{
	黟漢縣唐屬歙州九域志在州西一百五十三里黟顔師古音伊劉昫音䃜}
自餘百官皆如吳朝之制|{
	朝直遥翻}
置騎兵八軍步兵九軍 二月吳王以盧文進為宣武節度使|{
	宣武軍汴州時屬晉吳以盧文進遥領耳}
兼侍中 戊子吳主使宜陽王璪如西都|{
	吳以金陵為西都見上卷上年璪子皓翻}
冊命齊王王受冊赦境内冊王妃曰王后 吳越王元瓘之弟順化節度使同平章事元珦獲罪於元瓘廢為庶人|{
	錢元晌得罪之始見二百七十八卷唐明宗長興四年珦音向}
契丹主自上黨過雲州大同節度使沙彦珦出迎契丹主留之不使還鎮節度判官吳巒在城中謂其衆曰吾屬禮義之俗安可臣於夷狄乎衆推巒領州事閉城不受契丹之命契丹攻之不克應州馬軍都指揮使金城郭崇威亦恥臣契丹挺身南歸|{
	漢之金城唐蘭州五泉縣是也唐之金城為枝陽縣地凉置廣武郡隋廢郡為廣武縣唐乾元二年更曰金城屬蘭州按此非蘭州之金城乃應州之金城縣也唐明宗生于代北之金鳳城及即位以其地置金城縣仍置應州治焉郭崇威蓋以土人為本鎮將軍又匈奴須知云應州東至幽州八百五十里金城縣東北至朔州八百里如須知所云應州與金城縣似為兩處南北風馬牛不相及未能審其是又當從涉其地者問之挺拔也}
契丹主過新州命威塞節度使翟璋斂犒軍錢十萬緡初契丹主安巴堅彊盛室韋奚霫皆役屬焉|{
	翟直格翻又徒歷翻姓也犒苦到翻霫似入翻}
奚王渠珠克苦契丹虐帥其衆西徙媯州|{
	帥讀曰率}
依劉仁恭父子號西奚|{
	東奚居琵琶州西奚徙媯州依北山而居}
渠珠克卒子繅琳立唐莊宗滅劉守光賜繅琳

姓李名紹威娶契丹哲伯埒之姊哲伯埒獲罪於契丹奔紹威紹威納之契丹怒攻之不克紹威卒子伊喇立立|{
	伊戶結翻}
及契丹主德光自上黨北還|{
	還從宣翻又如字}
伊喇迎降|{
	降戶江翻}
時哲伯埒亦卒契丹主曰汝誠無罪繅琳哲伯埒負我皆命發其骨磑而颺之|{
	磑五對翻䃺也今人謂之磨颺余章翻}
諸奚畏契丹之虐多逃叛契丹主勞翟璋曰當為汝除代令汝南歸|{
	勞力到翻為千偽翻}
己亥璋表乞徵詣闕旣而契丹遣璋將兵討叛奚攻雲州有功留不遣璋璋鬱鬱而卒張礪自契丹逃歸為追騎所獲契丹主責之曰何故捨我去對曰臣華人飲食衣服皆不與此同生不如死願早就戮契丹主顧通事高彦英曰吾常戒汝善遇此人|{
	契丹置通事以主中國人以知華俗通華言者為之宋白曰契丹主腹心能華言者目曰通事謂其洞逹事務}
何故使之失所而亡去若失之安可復得邪|{
	復扶又翻}
笞彦英而謝礪礪事契丹主甚忠直遇事輒言無所隱避契丹主甚重之|{
	史言契丹主知重儒者}
初吳越王鏐少子元㺷|{
	少詩照翻㺷思聿翻 考異曰晉高祖實録十國紀年作元球今從吳越備史九國志}
數有軍功|{
	數所角翻}
鏐賜之兵仗及吳越王元瓘立元㺷為土客馬步軍都指揮使兼中書令恃恩驕横|{
	横戶孟翻}
增置兵仗至數千國人多附之元瓘忌之使人諷元㺷請輸兵仗出判温州元㺷不從銅官廟吏告元㺷遣親信禱神求主吳越江山又為蠟丸從水竇出入與兄元珦謀議|{
	蠟丸者蠟彈書也作書以蠟丸其外}
三月戊午元瓘遣使者召元㺷宴宫中旣至左右稱元㺷有刃墜于懷䄂即格殺之并殺元珦|{
	元珦被幽見二百七十八卷唐明宗長興四年}
元瓘欲按諸將吏與元珦元㺷交通者其子仁俊諫曰昔光武克王郎曹公破袁紹皆焚其書疏以安反側|{
	光武事見三十九卷漢更始二年曹公事見六十四卷獻帝建安五年}
今宜効之元瓘從之 或得唐潞王膂及髀骨獻之庚申詔以王禮葬于徽陵南|{
	唐閔帝之葬從徽陵封纔數尺見者悲之潞王葬于徽陵南見者莫之悲也豈非人心之公是非邪}
帝遣使詣蜀告即位且叙姻好|{
	蜀主孟知祥與帝皆後唐之主婿蜀主娶晉王克用姪女帝娶明宗姪女與蜀後主兄弟行也故叙姻好好呼到翻}
蜀主復書用敵國禮 范延光聚卒繕兵悉召巡内刺史集魏州|{
	天雄廵軍内有貝博衛澶相五州刺史}
將作亂會帝謀徙都大梁桑維翰曰大梁北控燕趙南通江淮水陸都會資用富饒今延光反形已露大梁距魏不過十驛|{
	唐制三十里一驛十驛三百里}
彼若有變大軍尋至所謂疾雷不及掩耳也丙寅下詔託以洛陽漕運有闕東巡汴州 吳徐知誥立子景通為王太子固辭不受追尊考忠武王温曰太祖武王妣明德太妃李氏曰王太后壬申更名誥|{
	更工衡翻徐知誥去名上知字單名誥示不與徐氏兄弟同也}
庚辰帝發洛陽留前朔方節度使張從賓為東都巡檢使 漢主以疾愈大赦 交州將皎公羨殺安南節度使楊廷藝而代之|{
	長興二年楊廷藝得交州見唐明宗紀皎姓也}
夏四月丙戌帝至汴州丁亥大赦 吳越王元瓘復建國如同光故事|{
	元瓘之初立罷建國事見二百七十八卷唐明宗長興三年}
丙申赦境内立其子弘僔為世子|{
	僔子損翻}
以曹仲達沈崧皮光業為丞相鎮海節度使官林鼎掌教令 丁酉加宣武節度使楊光遠兼侍中 閩主作紫微宫飾以水晶土木之盛倍于寶皇宫|{
	唐明宗長興二年閩主璘作寶皇宫}
又遣使散詣諸州伺人隱慝|{
	慝吐得翻}
五月吳徐誥用宋齊丘策欲結契丹以取中國遣使以美玉珍玩泛海修好契丹|{
	好呼到翻}
主亦遣使報之 丙辰敕權署汴州牙城曰大寧宫|{
	時御史臺奏汴州在梁有京都之號及唐莊宗廢為宣武軍至明宗行幸時掌事者修葺衙城遂掛梁時宫殿門牌額當時識者或竊非之一昨車駕省方暫居梁苑衙城内齊閣牌額一如明宗行幸之時無都號而有殿名恐非典據竊尋秦漢以來鑾輿所至多立宫名隋于揚州立江都宫太原立汾陽宫岐州立仁夀宫唐于太原立晉陽宫同州立長春宫岐州立九成宫宫中殿閣皆題署牌額以凖皇居請依故事于汴州衙城門權掛一宫門牌額則其餘齊閣並可取便為名敕行闕宜以大寧宫為名}
壬申進范延光爵臨清郡王以安其意 追尊四代考妣為帝后|{
	按五代會要高祖璟謚靖祖孝安皇帝妣秦氏謚元皇后曾祖郴謚肅祖孝簡皇帝妣安氏謚恭皇后祖昱謚睿祖孝平皇帝妣米氏謚獻皇后考紹雍謚獻祖孝元皇帝妣何氏謚懿皇后若以前史謂皇考名臬捩雞推之則四世之名意皆有司所撰者也}
己卯詔太社所藏唐室罪人首聽親舊收葬初武衛上將軍婁繼英嘗事梁均王為内諸司使至是請其首而葬之|{
	唐藏梁均王首于太社見二百七十三卷莊宗同光元年史為婁繼英請而不克葬張本}
六月吳諸道副都統徐景遷卒范延光素以軍府之政委元隨左都押牙孫鋭鋭恃恩專橫|{
	橫戶孟翻}
符奏有不如意者對延光手裂之會延光病經旬鋭密召澶州刺史馮暉與之合謀逼延光反|{
	澶時連翻}
延光亦思張生之言|{
	張生之言見上正月}
遂從之甲午六宅使張言奉使魏州還言延光反狀義成節度使符彦饒奏延光遣兵度河焚草市|{
	時天下兵爭凡民居在城外牽居草屋以成市里以其價亷功省猝遇兵火不至甚傷財以害其生也此草市在滑州城外}
詔侍衛馬軍都指揮使昭信節度使白奉進將千五百騎屯白馬津以備之|{
	白馬津在滑州白馬縣}
奉進雲州人也丁酉以東都巡檢使張從賓為魏府西南面都部署戊戌遣侍衛都軍使楊光遠|{
	侍衛都軍使即侍衛諸軍都指揮使}
將步騎一萬屯滑州己亥遣護聖都指揮使杜重威將兵屯衛州|{
	五代會要曰天福六年改成德兩軍為護聖左右軍據此則此時已有護聖軍矣}
重威朔州人也尚帝妹樂平長公主|{
	長知兩翻}
范延光以馮暉為都部署孫鋭為兵馬都監將步騎二萬循河西抵黎陽口|{
	黎陽在魏州西南故循河西上而後至}
辛丑楊光遠奏引兵踰胡梁渡|{
	此即史思明所濟胡梁渡也在滑州北岸澶州界薛史天福六年詔以胡梁渡月城為大通軍浮橋為大通橋}
以翰林學士禮部侍郎和凝為端明殿學士凝署其門不通賓客前耀州團練推官襄邑張誼|{
	襄邑縣屬宋州九域志在大梁東南一百七十里}
致書于凝以為切近之職為天子耳目宜知四方利病奈何拒絶賓客雖安身為便如負國何凝奇之薦于桑維翰未幾除左拾遺|{
	幾居豈翻}
誼上言北狄有援立之功宜外敦信好|{
	好呼到翻}
内謹邊備不可自逸以啓戎心帝深然之 契丹攻雲州半歲不能下吳巒遣使間道奉表求救帝為之致書契丹主請之|{
	間古莧翻陘北諸州皆歸契丹故間道南來為于偽翻}
契丹主乃命翟璋解圍去帝召巒歸以為武寧節度副使 丁未以侍衛使楊光遠為魏府四面都虞署|{
	侍衛使即侍衛都軍使史從省文也}
張從賓為副部署兼諸軍都虞侯昭義節度使高行周將本軍屯相州為魏府西南都部署|{
	相州在魏州之西使高行周自潞州將兵屯相州以臨范延光}
軍士郭威舊隸劉知遠當從楊光遠北征|{
	自大梁而征魏州為北征薛史周紀郭威初事李繼韜繼韜誅配從馬直晉祖領副侍衛召置麾下因而得事漢祖}
白知遠乞留人問其故威曰楊公有姦詐之才無英雄之氣得我何用能用我者其劉公乎|{
	為郭威為劉知遠佐命張本}
詔張從賓發河南兵數千人擊延光|{
	河南兵河南府兵也張從賓時為洛陽巡檢使故使發之}
延光使人誘從賓|{
	誘音酉}
從賓遂與之同反殺皇子河陽節度使重信|{
	重直龍翻下重又同}
使上將軍張繼祚知河陽留後繼祚全義之子也|{
	張全義自唐末尹河南歷唐梁}
從賓又引兵從洛陽殺皇子權東都留守重乂以東都副留守都巡檢使張延播知河南府事從軍|{
	張從賓雖以張延播知河南府事不使之在府治事而使之從軍}
取内庫錢帛以賞部兵留守判官李遐不與兵衆殺之從賓引兵扼汜水關|{
	汜水關以縣名關即虎牢關也詳見辯誤}
將逼汴州詔奉國都指揮使侯益帥禁兵五千會杜重威討張從賓|{
	帥讀曰率}
又詔宣徽使劉處讓自黎陽分兵討之時羽檄縱横從官在大梁者無不恟懼|{
	羽檄縱横言軍書紛委也從官家屬皆留東都而從駕在汴根本已拔故恟懼也縱子容翻從才用翻恟許拱翻}
獨桑維翰從容指畫軍事|{
	從干容翻}
神色自若接對賓客不改常度衆心差安|{
	史言桑維翰能以整暇鎮物}
方士言於閩主云有白龍夜見螺峯|{
	見賢遍翻螺盧戈翻}
閩主作白龍寺時百役繁興用度不足閩主謂吏部侍郎判三司侯官蔡守蒙曰|{
	後漢置東侯官縣隋廢入閩縣唐復置侯官縣屬福州九域志治州郭下}
聞有司除官皆受賂有諸對曰浮議無足信也閩主曰朕知之久矣今以委卿擇賢而受不肖及罔冒者勿拒|{
	罔冒謂欺罔偽冒而求官也以事理之所無而欺上謂之罔假它人之所有以飾偽謂之冒}
第令納賂籍而獻之守蒙素亷以為不可閩主怒守蒙懼而從之自是除官但以貨多少為差|{
	為蔡守蒙以賣官受誅張本}
閩主又以空名堂牒使醫工陳究賣官於外|{
	堂牒即今人所謂省劄空名者未書所授人名旣賣之得錢而後書填空苦貢翻}
專務聚斂無有盈厭|{
	斂力贍翻厭於鹽翻又如字}
又詔民有隱年者杖背隱口者死逃亡者族果菜雞豚皆重征之 秋七月張從賓攻汜水殺巡檢使宋廷浩帝戎服嚴輕騎將奔晉陽以避之桑維翰叩頭苦諫曰賊鋒雖盛勢不能久請少待之不可輕動帝乃止|{
	史言桑維翰有膽畧晉朝倚以為社稷之固少詩沼翻}
范延光遣使以蠟丸招誘失職者右武衛上將軍婁繼英右衛大將軍尹暉在大梁温韜之子延濬延沼延衮居許州皆應之|{
	尹暉舉軍降潞王以得節鎮今居環衛則為散官温韜自唐明宗時受誅其諸子廢弃而婁繼英子婦温延沼女也繼英亦居冗散故皆應延光}
延光令延濬兄弟取許州聚徒已及千人繼英暉事泄皆出走壬子敕以延光姦謀誣汙忠良自今獲延光諜人賞獲者殺諜人|{
	汙烏故翻諜達協翻}
禁蠟書勿以聞|{
	不欲知所招誘主名所以安反側也}
暉將奔吳為人所殺繼英奔許州依温氏忠武節度使萇從簡盛為之備延濬等不得發欲殺繼英以自明延沼止之遂同奔張從賓繼英知其謀勸從賓執三温皆斬之 白奉進在滑州|{
	是年六月遣白奉進屯白馬白馬滑州治所也}
軍士有夜掠者捕之獲五人其三隸奉進其二隸符彦饒奉進皆斬之彦饒以其不先白已甚怒明日奉進從數騎詣彦饒謝彦饒曰軍中各有部分|{
	分扶問翻}
奈何取滑州軍士并斬之殊無客主之義乎|{
	符彦饒自以鎮滑州為主白奉進屯滑州為客}
奉進曰軍士犯法何有彼我僕已引咎謝公而公怒不解豈非欲與延光同反邪拂衣而起彦饒不留帳下甲士大譟擒奉進殺之從騎走出大呼於外諸軍爭擐甲操兵諠譟不可禁止|{
	從才用翻呼火故翻擐音宦操七刀翻}
奉國左廂都指揮使馬萬惶惑不知所為帥步兵欲從亂遇右廂都指揮使盧順密帥部兵出營|{
	帥讀曰率}
厲聲謂萬曰符公擅殺白公必與魏城通謀此去行官纔二百里|{
	魏城謂魏州城也時范延光據魏州反九域志滑州南至大梁三百里時帝在大梁}
吾輩及軍士家屬皆在大梁奈何不思報國乃欲助亂自求族滅乎今日當共擒符公送天子立大功軍士從命者賞違命者誅勿復疑也|{
	復扶又翻}
萬所部兵尚有呼躍者|{
	呼火故翻}
順密殺數人衆莫敢動萬不得已從之與奉國都虞侯方太等共攻牙城執彦饒令太部送大梁甲寅敕斬彦饒於班荆館|{
	左傳楚伍舉與聲子相善伍舉出奔聲子遇于鄭郊班荆相與食而言杜預注曰班布也布荆坐地共議以班荆名館取諸此也此館必在汴州郊外}
其兄弟皆不問|{
	按符存審諸子皆有材氣而彦卿又為一時名將彦饒不能馭下倉猝成亂兄弟初不通謀罪不相及古法也}
楊光遠自白臯引兵趣滑州|{
	趣七喻翻}
士卒聞滑州亂欲推光遠為主光遠曰天子豈汝輩販弄之物晉陽之降出於窮迫|{
	謂在晉安寨殺張敬達而降也事見上卷上年降戶江翻}
今若改圖真反賊也其下乃不敢言時魏孟滑三鎮繼叛|{
	魏范延光孟張從賓滑符彦饒}
人情大震帝問計於劉知遠對曰帝者之興自有天命陛下昔在晉陽糧不支五日俄成大業今天下已定内有勁兵北結彊虜|{
	彊虜謂契丹}
鼠輩何能為乎願陛下撫將相以恩臣請戢士卒以威|{
	戢則立翻}
恩威兼著京邑自安本根深固則枝葉不傷矣知遠乃嚴設科禁|{
	科條也}
宿衛諸軍無敢犯者有軍士盜紙錢一幞|{
	幞逢玉翻釋云帊也}
主者擒之|{
	主者紙錢之主也}
左右請釋之知遠曰吾誅其情不計其直竟殺之|{
	唐法治盜計贓定罪劉知遠嚴刑以威衆欲鎮服其心以折亂萌非可常行于平世也}
由是衆皆畏服乙卯以楊光遠為魏府行營招討使兼知行府事以昭義節度使高行周為河南尹東京留守|{
	京當作都}
以杜重威為昭義節度使充侍衛馬軍都指揮使以侯益為河陽節度使|{
	侯益與杜重威同討張從賓就命鎮河陽}
帝以滑州奏事皆馬萬為首擢萬為義成節度使|{
	就以滑帥賞馬萬晉漢之間有白再榮因亂而帥成德馬萬之類也}
丙辰以盧順密為果州團練使|{
	果州時屬蜀命盧順密遥領團練使}
方太為趙州刺史旣而知皆順密之功也更以順密為昭義留後|{
	更工衡翻時杜重威領昭義節以討張從賓故以盧順密為留後}
馮暉孫銳引兵至六明鎮|{
	六明鎮在胡梁渡北}
光遠引之度河半度而擊之暉鋭衆大敗多溺死斬首三千級暉鋭走還魏杜重威侯益引兵至汜水遇張從賓衆萬餘人與戰俘斬殆盡遂克汜水從賓走乘馬渡河溺死|{
	溺奴狄翻}
獲其黨張延播繼祚婁繼英送大梁斬之滅其族|{
	符彦饒張從賓等皆死馮暉孫鋭又敗范延光之勢孤且衂矣}
史館修撰李濤上言張全義有再造洛邑之功|{
	事見二百五十七卷唐僖宗光啓三年}
乞免其族乃止誅繼祚妻子濤回之族曾孫也|{
	李回唐武宗會昌中為相}
詔東都留守司百官悉赴行在|{
	張從賓既平然後洛都留司百官得赴行在自是遂定都大梁}
楊光遠奏知博州張暉舉城降|{
	博州范延光巡屬也}
安州威和指揮使王暉|{
	五代會要唐有威和拱宸内直軍晉天福六年改為興順左右軍}
聞范延光作亂殺安遠節度使周瓌|{
	瓌古回翻}
自領軍府欲俟延光勝則附之敗則度江奔吳帝遣右領軍上將軍李金全將千騎如安州巡檢許赦王暉為唐州刺史 范延光知事不濟歸罪于孫鋭而族之|{
	孫鋭勸范延光反見上六年}
遣使奉表待罪戊寅楊光遠以聞帝不許 吳同平章事王令謀如金陵勸徐誥受禪誥讓不受 山南東道節度使安從進恐王暉奔吳遣行軍司馬張朏將兵會復州兵於要路邀之|{
	朏敷尾翻邀其自復州而奔吳鄂州之路也}
暉大掠安州將奔吳部將胡進殺之八月癸巳以狀聞李金全至安州將士之預于亂者數百人金全說諭悉遣詣闕|{
	說武芮翻}
旣而聞指揮使武彦和等數十人挾賄甚多伏兵于野執而斬之彦和且死呼曰|{
	呼火故翻}
王暉首惡天子猶赦之我輩脅從何罪乎帝雖知金全之情掩而不問 吳歷陽公濛知吳將亡甲子殺守衛軍使王宏宏子勒兵攻濛濛射殺之|{
	濛被囚見二百七十九卷唐潞王清泰元年射而亦翻}
以德勝節度使周本吳之勲舊引二騎詣廬州欲依之|{
	九域志和州西至廬州五百二十里}
本聞濛至將見之其子弘祚固諫本怒曰我家郎君來何為不使我見弘祚合扉不聽本出|{
	門闢則兩扉開門闔則兩扉合}
使人執濛于外送江都徐誥遣使稱詔殺濛于采石|{
	迎而殺之不使得至江都}
追廢為悖逆庶人絶屬籍|{
	絶楊氏屬籍悖蒲内翻又蒲没翻}
侍衛軍使郭悰殺濛妻子於和州誥歸罪于悰貶池州|{
	悰徂宗翻}
乙巳赦張從賓符彦饒王暉之黨未伏誅者皆不問梁唐以來士民奉使及俘掠在契丹者悉遣使贖還其家 吳司徒門下侍郎同平章事内樞使忠武節度使王令謀|{
	忠武軍許州時屬晉吳以王令謀遥領節鎮耳}
老病無齒或勸之致仕令謀曰齊王大事未畢吾何敢自安疾亟|{
	亟紀力翻}
力勸徐誥受禪是月吳主下詔禪位于齊李德誠復詣金陵帥百官勸進宋齊丘不署表|{
	宋齊丘以受禪之議不自已發而為周宗等所先遂堅持異議欲以為名復扶又翻帥讀曰率}
九月癸丑令謀卒|{
	王令謀所見誠不可與王琨同日語也}
甲寅以李金全為安遠節度使|{
	為李金全外叛張本}
婁繼英未及葬梁均王而誅死|{
	婁繼英求葬梁均王見上五月}
詔梁故臣右衛上將軍安崇阮與王故妃郭氏葬之 丙寅吳主命江夏王璘奉璽綬于齊|{
	楊行密據有江淮傳渥隆演至溥而亡璘離珍翻璽斯氏翻綬音受}
冬十月甲申齊王誥即皇帝位于金陵大赦改元昇元國號唐|{
	徐誥自以本李氏之子旣舉大號欲纂唐緒故改國號為唐為復李姓張本}
追尊太祖武王曰武皇帝|{
	猶不敢忘徐温而追尊之其後立李氏宗廟遂以徐温為義祖}
乙酉遣右丞相玠|{
	玠徐玠也}
奉冊詣吳主稱受禪老臣誥謹拜稽首上皇帝尊號曰高尚思玄弘古讓皇宫室乘輿服御皆如故宗廟正朔徽章服色悉從吳制|{
	乘䋲證翻}
丁亥立徐知證為江王徐知諤為饒王|{
	知證知諤皆徐温之子於誥為弟}
以吳太子璉領平盧節度使兼中書令封弘農公唐主宴羣臣于天泉閣|{
	天泉閣盖因晉宋時之天泉池故地起閣因以為名}
李德誠曰陛下應天順人惟宋齊丘不樂|{
	樂音洛}
因出齊丘止德誠勸進書 |{
	考異曰十國紀年云遣宗信書令宗信諷止德誠勸進而不云宗信何人今但云止德誠勸進書}
唐主執書不視曰子嵩三十年舊交必不相負齊丘頓首謝|{
	子嵩宋齊丘字也通鑑梁太祖乾化三年書齊丘謁知誥署昇州推官至是年二十六年今日三十年舊交盖乾化二年署推官而謁知誥又在乾化二年之前也}
己丑唐主表讓皇改東都宫殿名皆取于仙經|{
	唐都金陵以江都為東都}
讓皇常服羽衣習辟糓術辛卯吳宗室建安王珙等十二人皆降爵為公而加官增邑|{
	降王為公所以示易姓加官增邑所以慰其心珙居勇翻}
丙申以吳同平章事張延翰及門下侍郎張居詠中書侍郎李建勲並同平章事讓皇以唐主上表致書辭之唐主謝表而不改丁酉加宋齊丘大司徒齊丘雖為左丞相不預政事心愠懟|{
	愠於運翻懟直類翻}
聞制詞云布衣之交抗聲曰臣為布衣時陛下為刺史|{
	唐主為昇州刺史見二百六十六卷梁太祖乾化二年}
今日為天子可以不用老臣矣還家請罪唐主手詔謝之亦不改命久之齊丘不知所出乃更上書請遷讓皇于他州及斥遠吳太子璉絶其昏唐主不從|{
	遠于願翻宋齊丘之心迹至是畢露吾觀唐主之心豈特踈之而已蓋惡而欲遠之不能也}
乙巳立王后宋氏為皇后戊申以諸道都統判元帥府事景通為諸道副元帥判六軍諸衛事太尉尚書令吳王 閩主命其弟威武節度使繼恭上表告嗣位于晉且請置邸于都下|{
	閩與中國絶見二百七十七卷唐明宗長興三年}
十一月乙卯唐吳王景通更名璟|{
	更工衡翻璟俱永翻}
唐主賜楊璉妃號永興公主妃聞人呼公主則流涕而辭|{
	漢之孝平后周之天元后與夫吳楊璉之妃蓋異世而同轍也宋白曰永興縣本漢鄂縣地陳置永興縣唐屬卾州}
戊午唐主立其子景遂為吉王景達為夀陽公以景遂為侍中東都留守江都尹帥留司百官赴東都|{
	南唐倣盛唐兩都之制建東西都置留臺百司於江都帥讀曰率}
戊辰詔加吳越王元瓘天下兵馬副元帥進封吳越國王 |{
	考異曰實録天福二年十一月加元瓘副元帥國王程遜等為加恩使四年十月丙午以程遜沒于海廢朝贈官程遜傳云天福三年秋使吳越使回溺死元灌傳云天福三年封吳越國王蓋二年冬制下遜等以三年至杭州不知溺死在何年而晉朝以四年十月始聞之也吳越備史天福二年四月敕遣程遜等授王副元帥國王甲午王即位用建國之儀如同光故事是歲程遜還京溺于海按元瓘初立稱鏐遺命止用藩鎮禮明年明宗封吳王應順初閔帝封吳越王故以天福二年即王位而備史以為授元帥國王然後即位誤矣}
安遠節度使李金全以親吏胡漢筠為中門使軍府事一以委之漢筠貪猾殘忍聚斂無厭|{
	歛力贍翻厭於鹽翻}
帝聞之以亷吏賈仁沼代之|{
	考異曰薛史仁沼作仁紹今從實録}
且召漢筠欲授以他職庶保全功臣漢筠大懼始勸金全以異謀乙亥金全表漢筠病未任行|{
	任音壬}
金全故人龐令圖屢諫曰仁沼忠義之士以代漢筠所益多矣漢筠夜遣壯士踰垣滅令圖之族又毒仁沼舌爛而卒漢筠與推官張緯相結以諂惑金全金全愛之彌篤|{
	李金全叛奔南唐之計自是定矣}
十二月戊申蜀大赦改明年元曰明德 詔加馬希範江南諸道都統制置武平静江等軍事 是歲契丹改元會同國號大遼公卿庶官皆倣中國參用中國人以趙延壽為樞密使尋兼政事令|{
	為遼人用趙延夀以圖晉張本}
三年春正月己酉日有食之 唐德勝節度使兼中書令西平恭烈王周本以不能存吳愧恨而卒|{
	周本雖不能存吳然其過李德誠遠矣}
丙寅唐以侍中吉王景遂參判尚書都省 蜀主以武信節度使同平章事張業為左僕射兼中書侍郎同平章事樞密使武泰節度使王處回兼武信節度使同平章事|{
	黔邉於諸蠻遂蜀之内地也以此為進律}
二月庚辰左散騎常侍張允上駮赦論|{
	駮北角翻}
以為帝王遇天災多肆赦謂之修德借有二人坐獄遇赦則曲者幸免直者銜寃寃氣升聞|{
	聞音問}
乃所以致災非所以弭災也詔褒之帝樂聞讜言|{
	樂音洛讜音黨}
詔百官各上封事命吏部尚書梁文矩等十人置詳定院以考之無取者留中可者行之數月應詔者無十人乙未復降御札趣之|{
	復扶又翻趣讀曰促}
三月丁丑敕禁民作銅器初唐世天下鑄錢有三十六冶|{
	此謂後唐之世也若盛唐之世天下銅冶九十有餘所}
喪亂以來|{
	喪息浪翻}
皆廢絶錢日益耗民多銷錢為銅器故禁之 中書舍人李詳上疏以為十年以來赦令屢降諸道職掌皆許推恩而藩方薦論動踰數百乃至藏典書吏優伶奴僕|{
	藏典主帑藏之吏藏徂浪翻}
初命則至銀青階被服皆紫袍象笏|{
	被皮義翻}
名器僭濫貴賤不分請自今諸道主兵將校之外節度州聽奏朱記大將以上十人|{
	節度州者節度使所治之州朱記大將者不給銅印給木朱記以為印信}
他州止聽奏都押牙都虞候孔目官自餘但委本道量遷職名而已|{
	量音良}
從之 夏四月甲申唐宋齊丘自陳丞相不應不豫政事唐主荅以省署未備 吳讓皇固辭舊宫|{
	以旣讓位于唐不敢居江都宫}
屢請徙居李德誠等亦亟以為言|{
	亟去吏翻}
五月戊午唐主改潤州牙城為丹楊宫以李建勲為迎奉讓皇使楊光遠自恃擁重兵|{
	時范延光未平晉之重兵皆在楊光遠之手}
頗干預朝政屢有抗奏帝常屈意從之|{
	為楊光遠請易置執政張本}
庚申以其子承祚為左威衛將軍尚帝女長安公主次子承信亦拜美官寵冠當時|{
	冠古玩翻為楊光遠叛亂張本}
壬戌唐主以左宣威副統軍王輿為鎮海留後客省使公孫圭為監軍使親吏馬思讓為丹楊宫使徙讓皇居丹楊宫|{
	選用王輿等以防衛故吳王}
宋齊丘復自陳為左右所間|{
	復扶又翻間古莧翻}
唐主大怒齊丘歸第白衣待罪或曰齊丘舊臣不宜以小過弃之唐主曰齊丘有才不識大體乃命吳主璟持手詔召之六月壬午或獻毒酒方于唐主唐主曰犯吾法者自有常刑安用此為|{
	史言唐主斯言得君人之體}
羣臣爭請改府寺州縣名有吳及陽者|{
	以吳者楊氏國號而陽字與楊字同音也}
留守判官楊嗣請更姓羊|{
	留守判官東都留守判官也}
徐玠曰陛下自應天順人事非逆取|{
	逆取本之漢陸賈逆取順守之言}
而諂邪之人專事改更咸非急務不可從也唐主然之 河南留守高行周奏修洛陽宫丙戌左諫議大夫薛融諫曰今宫室雖經焚毁猶侈于帝堯之茅茨|{
	唐堯土階三尺茅茨不翦}
所費雖寡猶多於漢文之露臺|{
	露臺事見漢文帝紀}
况魏城未下|{
	謂范延光尚據魏州楊光遠攻之未下也}
公私困窘|{
	窘渠隕翻}
誠非陛下修宫館之日俟海内平寧營之未晚上納其言仍賜詔褒之 己丑金部郎中張鑄奏竊見鄉村浮戶|{
	浮戶謂未有土著定籍者言其蓬轉萍流不常厥居若浮泛于水上然}
非不勤稼穡非不樂安居|{
	樂音洛}
但以種木未盈十年墾田未及三頃似成生業已為縣司收供徭役責之重賦威以嚴刑故不免捐功捨業更思他適乞自今民墾田及五頃以上三年外乃聽縣司徭役從之 秋七月中書奏朝代雖殊|{
	朝直遥翻}
條制無異請委官取明宗及清泰時敕詳定可久行者編次之己酉詔左諫議大夫薛融等詳定 辛酉敕作受命寶以受天明命惟德允昌為文|{
	以受命寶為潞王所焚故也時中書門下奏準勅製皇帝受命寶今按貞觀十六年太宗文皇帝刻之玄璽白玉為螭頭其文曰皇帝受命有德者昌勅以受天明命惟德允昌為文按唐六典受命寶天子修封禪禮神祗則用之}
帝上尊號於契丹主及太后戊寅以馮道為太后冊禮使 |{
	考異曰周世宗實録馮道傳云虜遣使加徽號于晉祖晉亦獻徽號於虜始命兵部尚書王權御其命權辭以老病晉祖謂遒曰此行非卿不可道無難色按晉高祖實録天福三年八月戊寅道為契丹太后冊禮使十月戊寅北朝命使上帝徽號戊子王權以不受北使停任周世宗實録誤也}
左僕射劉煦為契丹主冊禮使|{
	煦本作昫}
備鹵簿儀仗車輅詣契丹行禮契丹主大悦帝事契丹甚謹奉表稱臣謂契丹主為父皇帝每契丹使至帝于别殿拜受詔敕歲輸金帛三十萬之外|{
	三十萬乃講和元約歲輸之數}
吉凶慶弔歲時贈遺玩好珍異相繼于道|{
	遺唯季翻好呼到翻}
乃至應天太后元帥太子偉王南北二王韓延徽趙延壽等諸大臣皆有賂小不如意輒來責讓帝常卑辭謝之|{
	應天太后即契丹主母舒嚕氏應天之號盖帝所上也}
晉使者至契丹契丹驕倨多不遜語使者還以聞|{
	還從宣翻}
朝野咸以為恥而帝事之曾無倦意以是終帝之世與契丹無隙然所輸金帛不過數縣租賦往往託以民困不能滿數其後契丹主屢止帝上表稱臣但令為書稱兒皇帝如家人禮初契丹旣得幽州命曰南京|{
	天福元年契丹始得幽州}
以唐降將趙思温為留守思温子延照在晉帝以為祁州刺史|{
	唐昭宗景福三年義武節度使王處存奏以定州無極深澤二縣置祁州}
思温密令延照言虜情終變請以幽州内附帝不許|{
	趙延照後遂入契丹為契丹用}
契丹遣使詣唐宋齊丘勸唐主厚賄之俟至淮北濳遣人殺之欲以間晉|{
	間古莧翻宋齊丘之意以謂殺契丹使于晉境契丹主必謂晉人殺之而誚讓晉此所以間之也}
壬午楊光遠奏前澶州刺史馮暉自廣晉城中出戰因來降|{
	馮暉自澶州入廣晉與范延光同反見上年六月}
言范延光食盡窮困己丑以暉為義成節度使楊光遠攻廣晉歲餘不下|{
	厚賞馮暉欲以攜范延光之黨楊光遠自去年六月攻范延光七月破馮暉等始進兵攻廣晉今歲餘矣而猶不下唐莊宗即位改魏州為興唐府帝革命改為廣晉}
帝以師老民疲遣内職朱憲入城諭范延光|{
	内職蓋宦者也}
許移大藩曰若降而殺汝白日在上吾無以享國延光謂節度副使李式曰主上重信云不死則不死矣乃撤守備然猶遷延未決宣徽南院使劉處讓復入諭之|{
	處昌呂翻下同復扶又翻}
延光意乃決九月乙巳朔楊光遠送延光二子守圖守英詣大梁己酉延光遣牙將奉表待罪壬子詔書至廣晉延光帥其衆素服於牙門使者宣詔釋之|{
	帥讀曰率}
朱憲汴州人也 契丹遣使如洛陽取趙延壽妻唐燕國長公主以歸|{
	趙延壽妻唐明宗女也延壽入契丹其妻留洛今延壽在北用事故來取之長知兩翻}
壬戌唐太府卿趙可封請唐主復姓李立唐宗廟 庚午楊光遠表乞入朝命劉處讓權知天雄軍府事|{
	楊光遠之討范延光也制令兼知天雄軍行府事延光旣降而光遠請入朝時劉處讓奉詔入魏諭降延光因使之權知軍府}
己巳制以范延光為天平節度使仍賜鐵劵|{
	賜鐵劵者恕其死而明之以信誓}
應廣晉城中將吏軍民今日以前罪皆釋不問|{
	今日謂制書到魏州之日也}
其張從賓符彦饒餘黨及自官軍逃叛入城者亦釋之延光腹心將佐李式孫漢威薛霸皆除防禦團練使刺史牙兵皆升為侍衛親軍初河陽行軍司馬李彦珣邢州人也父母在鄉里未嘗供饋後與張從賓同反從賓敗奔廣晉|{
	去年六月張從賓反踰月而敗}
范延光以為步軍都監使登城拒守楊光遠訪獲其母置城下以招之彦珣引弓射殺其母|{
	射而亦翻}
延光旣降帝以彦珣為坊州刺史近臣言彦珣殺母殺母惡逆不可赦|{
	律有十惡殺父母者惡逆恩赦之所不原}
帝曰赦令已行不可改也乃遣之官

臣光曰治國家者固不可無信|{
	治直之翻}
然彦珣之惡三靈所不容|{
	三靈謂天神地祗人鬼}
晉高祖赦其叛君之愆治其殺母之罪何損於信哉

辛未以楊光遠為天雄節度使 冬十月戊寅契丹遣使奉寶冊加帝尊號曰英武明義皇帝 帝以大梁舟車所會便于漕運丙辰建東京于汴州|{
	自此歷漢周至宋皆都于汴梁建東都于汴州以汴州為開封府開平三年割滑州之酸棗長垣鄭州之中牟陽武宋州之襄邑曹州之戴邑許州之扶溝鄢陵陳州之太康九縣並隸開封府以同光二年詔以陽武匡城扶溝考城四縣屬汴州餘還故屬匡城即長垣天成元年扶溝復隸許州至是詔汴州宜升東京仍升開封浚儀兩縣為赤縣其餘為畿縣應舊置開封府所管屬縣並依舊割屬收管亦升為畿縣}
復以汴州為開封府以東都為西京以西都為晉昌軍節度|{
	唐以長安為西都以洛陽為東都梁始都汴以汴州為東京洛陽為西京而以長安為節鎮後唐滅梁復唐兩京之舊而以汴州為節鎮晉今復于汴州建東京開封府以洛陽之東都為西京以長安之西都為晉昌軍}
帝遣兵部尚書王權使契丹謝尊號權自以累世將相耻之|{
	王權唐左僕射起之曾孫父蕘官至右司郎中起之先世播相唐文宗薛史王起官至左僕射山南西道節度使冊贈太尉}
謂人曰吾老矣安能向穹廬屈膝乃辭以老疾帝怒戊子權坐停官 初郭崇韜旣死|{
	郭崇韜死事見二百七十四卷唐明宗天成元年}
宰相罕有兼樞密使者帝即位桑維翰李崧兼之宣徽使劉處讓及䆠官皆不悦楊光遠圍廣晉處讓數以軍事銜命往來光遠奏請多踰分|{
	數音所角翻分音扶問翻}
帝常依違維翰獨以法裁折之|{
	依違者謂若依若違無可否一定之說折之舌翻}
光遠對處讓有不平語處讓曰是皆執政之意光遠由是怨執政范延光降光遠密表論執政過失|{
	光遠旣平范延光挾功邀上以斥執政}
帝知其故而不得已加維翰兵部尚書崧工部尚書皆罷其樞密使 |{
	考異曰竇貞固少帝實録及薛史劉處讓傳云楊光遠入朝遂于高祖前而言執政之失乃罷維翰等樞密使以處讓為之楊光遠傳云范延光降光遠面奏維翰擅權高祖以光遠方有功于國乃出維翰領安陽光遠為西京留守今按晉高祖實録天福三年十月壬辰維翰崧罷樞密使庚子光遠始入朝對于便殿十一月戊申光遠為西京留守天福四年閏七月壬申維翰出為相州節度使蓋處讓光遠傳之誤晉少帝實録及薛史桑維翰傳叙光遠鎮洛陽後疏維翰出相州是也}
以處讓為樞密使 太常奏今建東京而宗廟社稷皆在西京請遷置大梁敕旨且仍舊 戊戌大赦 楊延藝故將吳權自愛州舉兵攻皎公羨于交州|{
	延藝當作廷藝皎公羨殺楊廷藝見本卷之上年劉昫曰愛州東至小黃江口四百六十里入交州界}
羨遣使以賂求救于漢|{
	以下文考之羨上當有公字}
漢主欲乘其亂而取之以其子萬王弘操為靜海節度使徙封父王|{
	言將以交州為弘操封略}
將兵救公羨漢主自將屯于海門為之聲援漢主問策於崇文使蕭益益曰今霖雨積旬海道險遠吳權桀黠未可輕也大軍當持重多用鄉導然後可進|{
	點音下八翻鄉讀曰嚮}
不聽命弘操帥戰艦自白藤江趣交州|{
	白藤江當在峯州界自此進至花步抵峯州帥讀曰率趣七喻翻}
權已殺公羨據交州引兵逆戰先于海口多植大杙鋭其首冒之以鐵|{
	杙音與職翻橜也}
遣輕舟乘潮挑戰而偽遁|{
	挑音徙了翻}
須臾潮落漢艦皆礙鐵杙不得返|{
	艦戶黯翻}
漢兵大敗士卒覆溺者大半弘操死漢主慟哭收餘衆而還|{
	還從宣翻又如字}
先是著作佐郎侯融勸漢主弭兵息民至是以兵不振追咎融剖棺暴其屍益倣之孫也|{
	先悉薦翻蕭倣相唐懿宗}
楚順賢夫人彭氏卒彭夫人貌陋而治家有法|{
	治直之翻}


楚王希範憚之旣卒希範始縱聲色為長夜之飲内外無别|{
	别彼列翻}
有商人妻美希範殺其夫而奪之妻誓不辱自經死|{
	史以婦人能守節書其事而失其姓氏而馬希範之淫暴不可揜矣}
河決鄆州|{
	鄆音運}
十一月范延光自鄆州入朝|{
	范延光降自魏徙鄆今自鄆州入朝}
丙午以閩主昶為閩國王|{
	書閩主者表其已竊大號書以為國王者晉命也}


以左散騎常侍盧損為冊禮使賜昶赭袍|{
	赭袍天子所服賜之是許之竊號也}
戊申以威武節度使王繼恭為臨海郡王閩主聞之遣進奏官林恩白執政以旣襲帝號辭冊命及使者閩諫議大夫黄諷以閩主淫暴與妻子辭訣入諫閩主欲杖之諷曰臣若迷國不忠死亦無怨直諫被杖臣不受也閩王怒黜為民 帝患天雄節度使楊光遠跋扈難制桑維翰請分天雄之衆加光遠太尉西京留守兼河陽節度使光遠由是怨望密以賂自訴于契丹養部曲千餘人常蓄異志|{
	楊光遠雖蓄異志而帝與契丹無間則無從而發也至出帝與契丹構隙則引契丹為援而速禍矣}
辛亥建鄴都于廣晉府|{
	唐莊宗之初即位也建東京於魏州以魏州為興唐府後改為鄴都明宗天成四年廢晉受命以魏州為廣晉府今復建鄴都}
置彰德軍於相州以澶衛隸之|{
	彰德軍梁貞明間嘗置之矣張彦之變尋廢今復置之}
置永清軍于貝州以博冀隸之|{
	分天雄之貝博成德之冀州為永清軍}
澶州舊治頓丘帝慮契丹為後世之患遣前淄州刺史汲人劉繼勲徙澶州跨德勝津并頓丘徙焉|{
	澶州本治頓丘縣今併州縣皆徙治德勝按九域志之澶州距魏州一百三十里德勝之澶州晉人議者以為距魏州一百五十里有二十里之差蓋自澶州北城抵魏州止一百三十里若自南城渡河并浮梁計程則一百五十里也}
以河南尹高行周為廣晉尹鄴都留守貝州防禦使王廷胤為彰德節度使右神武統軍王周為永清節度使|{
	始升貝州為永清軍}
廷胤處存之孫|{
	唐末王處存鎮易定}
周鄴都人也 范延光屢請致仕甲寅詔以太子太師致仕居于大梁每預宴會與羣臣無異延光之反也相州刺史掖人王景拒境不從|{
	范延光帥天雄相州其廵屬也掖漢縣唐帶萊州相息亮翻}
戊午以景為耀州團練使 癸亥敇聽公私自鑄銅錢無得雜以鈆鐵每十錢重一兩以天福元寶為文仍令鹽鐵頒下模範|{
	鹽鐵者鹽鐵使司也下戶嫁翻}
惟禁私作銅器|{
	五代會要時令三京鄴都諸道州府無問公私應有銅者並令鑄錢仍以天福元寶為文左環讀之委鹽鐵鑄樣頒下諸道每一錢重二銖四參十錢重一兩或慮諸色人接便將鈆鐵鑄造雜亂銅錢仍令所屬依舊禁斷尚慮逐處銅數不多宜令諸道應有久廢銅冶許百姓取便開鍊永遠為主官司不取課利其有生熟銅仍許所在中賣入官或任自鑄錢行用不得接便别鑄銅器}
立左金吾衛上將軍重貴為鄭王充開封尹|{
	通鑑封子姪為王多書封亦或書立蓋因當時史官成文書之無義例也}
癸亥敕先許公私鑄錢慮銅難得聽輕重從便但勿令缺漏|{
	許民私鑄已非可久之法况又聽其輕重從便則民必鑄輕安有鑄重者乎惟患鑄之不輕薄耳輕薄之甚必至缺漏此錢安可久行邪}
辛丑吳讓皇卒|{
	年三十八 考異曰薛史唐餘録皆云溥禪位踰年以幽卒歐陽史但云卒九國志云溥能委運授終不罹簒殺之禍深于機者也十國紀年曰辛丑唐人弑讓皇事不可明今但云卒}
唐主廢朝二十七日|{
	唐主于舊君之卒依傍漢朝臣為君服以日易月之制為廢朝日數自古以來易姓之君惟唐主于舊君若加厚者原心定罪是猶紾兄之臂而奪之食旣奪其食矣乃引其臂而按摩之曰吾愛吾兄也朝直遥翻}
追諡曰睿皇帝是歲唐主徙吳王璟為齊王鳳翔節度使李從曮厚文士而薄武人愛農民而嚴

士卒由是將士怨之會發兵戍西邉旣出郊作亂突門入城剽掠于市|{
	突城門而入剽掠也剽囚妙翻}
從曮發帳下兵撃之亂兵敗東走欲自訴于朝廷至華州鎮國節度使張彦澤邀擊盡誅之

資治通鑑卷二百八十一
