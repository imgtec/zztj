周纪二(起昭陽赤奮若(癸丑),盡上章困敦(庚子),凡四十八年。)

  显王〔十一家谥法:行见中外曰显;受禄于天曰显;百辟惟刑曰显。周公盖未有此谥,而周之末世谥显王曰显,意谓后世传写周公諡法者遗之。〕

  元年(癸丑,公元前三六八年)

  ①齐伐魏,取观津。〔康曰:齐伐魏,魏惠王请献观以和,即观津。余按班志信都国有观津县,与齐相去甚远,且赵地也。又东郡有畔观县。水经:大河故渎东迳五鹿之野,又东迳卫国故城南,古斟观也。此其魏之观津欤!徐广曰:观,今卫县。史记正义曰:魏州观城县,古观国。国语云:观国,夏太康第五弟之所封也。观,工唤翻。〕

  ②赵侵齐,取长城。〔刘昭志:济北卢县有长城。史记苏代说燕王曰:"齐有长城巨防,"即此。〕

  三年(乙卯,公元前三六六年)

  ①魏、韩会于宅阳。〔水经注曰:荥泽之际有沙城,世谓水城,非也。魏冉走芒卯,入北宅,即此宅阳城。括地志曰:宅阳故城,在郑州荥阳县东十七里。〕

  ②秦败魏师、韩师于洛阳。〔洛阳在洛水之北,周公迁殷民于此,谓之成周。班志,属河南郡。败,补迈翻。〕

  四年(丙辰,公元前三六五年)

  ①魏伐宋。

  五年(丁巳,公元前三六四年)

  ①秦献公败三晋之师于石门,〔水经注:冯翊云阳县有石门山。括地志:在雍州三原县西北三十二里。又曰:尧门山,俗名石门,上有路,其状若门。故老云:尧凿山为门,因名之。武德中于此山南置石门县,贞观中改云阳县。〕斩首六万。王赐以黼黻之服。〔黼者,刺绣为斧形;黻者刺绣为两"己"相背。孔颖达曰:白与黑谓之黼,黑与青谓之黻。黼,音甫。黻,音弗。〕

  七年(己未,前三六二年)

  ①魏败韩师、赵师于浍。〔浍,古外翻。括地志:浍水在绛州翼城县东南二十五里,水侧有皮牢城。〕

  ②秦、魏战于少梁,〔班志:冯翊夏阳县,故少梁。师古曰:本梁国,为秦所灭,至惠文王十一年,更名夏阳。康曰:魏有大梁,故此称"少"以别之。少,诗沼翻。夏,户雅翻。更,工衡翻。〕魏师败绩;获魏公孙痤。〔左传:师大崩曰败绩。痤,才何翻〕

  ③卫声公薨,子成侯速立。

  ④燕桓公薨,子文公立。〔燕,因肩翻。考异曰:史记苏秦传谓之"燕文侯"。按春秋时北燕简公已称公,文公之子易王寻称王,岂文公独称侯乎!今从世家。〕

  ⑤秦献公薨,子孝公立。〔索隐曰:孝公,名渠梁。〕孝公生二十一年矣。是时河、山以东强国六,〔河自龙门上口,南抵华阴而东流,秦国在河之西。山自鸟鼠同穴连延为长安南山,至于泰华,秦国在山之西。韩、魏、赵、齐、楚、燕六国皆在河、山以东。华,户化翻。燕,因肩翻。〕淮、泗之间小国十余,〔南阳郡平氏县东南有桐柏、大复山,淮水所出,东南至淮陵入海。泗水出鲁国卞县西南,至方与入沛。宋、鲁、邹、滕、薛、郳等国,国于其间。齐威王所谓"泗上十二诸侯"。〕楚、魏与秦接界。魏筑长城,自郑滨洛以北有上郡;〔郑县,周宣王母弟郑桓公封邑,班志属京兆。洛,水名,非伊、洛之洛也。水经注:渭水东过华阴县北,洛水入焉。洛水,古漆、沮之水也。又有长涧水,南出泰华之山侧长城东而北流注于渭。史记所谓"魏筑长城,自郑滨洛"者也。宋白曰:今华州东南魏长城是也。上郡,汉属并州,隋、唐之绥州、延州,秦、汉之上郡地也。滨,音宾。〕楚自汉中,南有巴、黔中:〔汉中郡,汉属益州,自晋以后为梁州。巴,即春秋巴子之国,汉为巴郡,属益州,唐为巴、渝、渠、果诸州之地。黔中,汉为牂柯郡之地,唐为黔中节度。黔,渠今翻。〕皆以夷翟遇秦,〔翟,与狄同。〕摈斥之,不得与中国之会盟。〔摈,必刃翻。与,读曰预。〕于是孝公发愤,布德修政,欲以强秦。〔愤,房粉翻,懑也,怒也。朱元晦曰:愤者,心求通而未得之意。〕

  八年(庚申,公元前三六一年)

  ①孝公下令国中曰:"昔我穆公,自岐、雍之间修德行武,东平晋乱,以河为界,西霸戎翟,广地千里,天子致伯,诸侯毕贺,〔令,力正翻,号令也,命令也。令者,出于上而行于下者也。岐山,周太王所邑。班志,岐山在扶风美阳县西。雍县属扶风。秦穆公娶晋献公之女。献公卒,晋国乱,穆公纳惠公。惠公立而背河外之赂,又闭秦籴。穆公伐晋,执惠公,既而归之;始征晋河东,置官司。惠公卒,子怀公立。穆公纳文公而晋乱平。又能用由余及孟明,以霸西戎。天子致伯者,周礼九命作伯;古有九州,一为王畿,八州八伯,各主其方之诸侯;致伯者,以方伯之任致之穆公也。雍,于用翻。伯,如字。背,蒲妹翻。〕为后世开业甚光美。会往者厉、躁、简公、出子之不宁,国家内忧,未遑外事,三晋攻夺我先君河西地,丑莫大焉。〔为,于伪翻。史记:秦厉共公卒,子躁公立。躁公卒,立其弟怀公。四年,庶长鼂围怀公,公自杀,乃立灵公。灵公卒,子献公不得立,立灵公之季父,是为简公。公卒而惠公立。惠公卒,子出子立。二年,庶长改杀出子,迎立献公于河西。河西地,即魏所有西河之外。史记正义曰:自华州北至同州,并魏河西之地。躁,则到翻。共,读曰恭。鼂,古朝字。长,知两翻。华,户化翻。〕献公即位,镇抚边境,徙治栎阳,〔史记:秦献公二年,始治栎阳。徐广注曰:即汉万年县。余按汉志,栎阳、万年为两县,皆属冯翊,后汉始省并。宋白曰:栎阳,秦旧县。汉高祖既葬太上皇于万年陵,仍分栎阳置万年县以为陵邑,理栎阳城中,故栎阳城亦名万年城。后汉省栎阳县入万年县。后魏大统中,分万年置鄣丘、宣武,又分置广阳县。周明帝省万年入高陵、广阳二县,更于长安城中别置万年县。唐武德元年,又改广阳为栎阳,元和十五年,并移隶奉先县以奉景陵。栎,音药。〕且欲东伐,复穆公之故地,修穆公之政令。寡人思念先君之意,常痛于心。宾客群臣有能出奇计强秦者,吾且尊官,与之分土。"〔谓裂地以封之,使各有分土。分,扶问翻。〕于是卫公孙鞅闻是令下,乃西入秦。

  公孙鞅者,卫之庶孙也。好刑名之学。〔师古曰:刘向别录云;申子学好刑名。刑名者,循名以责实,其尊君卑臣,崇上抑下,合于六经。说者曰:刑,刑家;名,名家;即太史公所论六家之二也。此说非。刘原父曰:刑名,即并学两家术耳。公孙非姓氏,以其先出于卫,父为卫侯则称为公子,祖为卫侯则称为公孙。鞅,于两翻。〕事魏相公叔痤,痤知其贤,未及进。会病,魏惠王往问之曰:"公叔病如有〔章,十二行本二字互乙;乙十一行 本同;孔本同。〕不可讳,〔相,息亮翻。痤,才戈翻。不可讳,谓死也。俗语有之:"人不讳死"。〕将柰社稷何?"公叔曰:"痤之中庶子卫鞅,〔自战国以来,大夫之家有中庶子,有舍人。〕年虽少,有奇才,〔少,诗照翻。〕愿君举国而听之!"王嘿然。公叔曰:"君即不听用鞅,必杀之,无令出境!" 王许诺而去。〔令,力丁翻。〕公叔召鞅谢曰:"吾先君而后臣,〔先、后,皆去声。〕故先为君谋,后以告子。〔此先、后,皆如字。为,于伪翻。〕子必速行矣!"鞅曰:"君不能用子之言任臣,又安能用子之言杀臣乎!"卒不去。〔卒,子恤翻。〕王出,谓左右曰:"公叔病甚,悲乎,欲令寡人以国听卫鞅也!既又劝寡人杀之,岂不悖哉!"〔悖,蒲内翻。〕卫鞅既至秦,因嬖臣景监以求见孝公,〔嬖,博计翻,又卑义翻。史记正义:监,甲暂翻。康曰:景,姓,楚之族。监,古衔切,非。〕说以富国强兵之术;公大悦,与议国事。〔说,式芮翻。〕

  十年(壬戌,公元前三五九年)

  ①卫鞅欲变法,秦人不悦。卫鞅言于秦孝公曰:"夫民不可与虑始,而可与乐成。〔夫,音扶。乐,音洛。〕论至德者不和于俗,成大功者不谋于众。是以圣人苟可以强国,不法其故。"〔索隐曰:言救弊为政之术,所为苟可以强国,则不必须要法于故事也。〕甘龙曰:"不然,〔索隐曰:甘,姓;龙,名。甘姓出春秋时甘昭公子带之后。姓谱又曰:甘姓,商甘盘之后。〕缘法而治者,吏习而民安之。"〔治,直吏翻。〕卫鞅曰:"常人安于故俗,学者溺于所闻。〔溺,奴历翻。〕以此两者,居官守法可也,非所与论于法之外也。智者作法,愚者制焉;贤者更礼,不肖者拘焉"〔更,工衡翻。〕公曰:"善。"以卫鞅为左庶长。〔刘卲爵制曰:春秋传有庶长鲍,商君为政,备其法品为十八级,合关内侯、列侯,凡二十等。其制因古义。古者天子寄军政于六卿,居则以田,警则以战,所谓"入使治之,出使长之,素信者与众相得"也。故启伐有扈,乃召六卿,大夫之在军为将者也。及周之六卿,亦以居军。在国也,则以比长、闾胥、族师、党、州长、卿大夫为称;其在军也,则以司马、将军、卒、伍为号,所以异在国之名也。秦依古制,其在军赐爵为等级,其帅人皆更卒也。有功赐爵,则在军吏之例。自一等以上至不更,四等,皆士也。大夫以上至五大夫,五等,比大夫也。九等,依九命之义也。左庶长至大庶长,比九卿也。关内侯者,依古圻内子男之义也。秦都山西,以关内为王畿,故曰关内侯也。列侯者,依古列国诸侯之义也。然则卿、大夫、士下之品,皆仿古比朝之制而异其名,亦所以殊军国也。古者以车战,兵车一乘,步卒七十二人,分翼左右;车,大夫在左,御者处中,勇士为右,凡七十五人。一爵曰公士者,步卒之有爵为公士者也。自三爵曰上造,造,成也,古者成士升于司徒曰造士,虽依此名,皆步卒也。二爵曰簪褭,御驷马者。要褭者,古之名马也;驾驷马,其形似簪,故云簪褭也。四爵曰不更,不更者,为车右,不复与凡更卒同也。五爵曰大夫,大夫在车左者也。六爵为官大夫,七爵为公大夫,八爵为公乘,九爵为五大夫,皆军吏也。吏民爵不得过公乘者,得贳与子若同产。然则公乘者,军吏之爵最高者也;虽非临战,得公乘车,故曰公乘也。十爵为左庶长,十一爵为右庶长,十二爵为左更,十三爵为中更,十四爵为右更,十五爵为少上造,十六爵为大上造,十七爵为驷车庶长,十八爵为大庶长,十九爵为关内侯,二十爵为列侯。自左庶长至大庶长,皆卿大夫,皆军将也;所将皆庶人、更卒也,故以"庶"、"更"为名。大庶长,即大将军也;左、右庶长,即左、右偏裨将军也。长,知丈翻。〕卒定变法之令。令民为什伍而相收司、连坐,〔索隐曰:收司,谓相纠发也。一家有罪,则九家连举发;若不纠举,则九家连坐。师古曰:五人为伍,二伍为什。康曰:司,犹管也。为什伍之法,使之相司相管。秦有见知连坐法。余谓连坐者,一家有罪,什伍皆相连坐罪也;见知乃汉法。卒,子恤翻。〕告奸者与斩敌首同赏,〔索隐曰:谓告奸一人则得爵一级,故云与斩敌首同赏。〕不告奸者与降敌同罚。〔索隐曰:律:降敌者诛其身,没其家。今匿奸者,言当与之同罚。降,户江翻。〕有军功者,各以率受上爵;〔率,音律。〕为私斗者,各以轻重被刑大小。僇力本业,耕织致粟帛多者,复其身;〔僇,力竹翻,古戮字;说文:并力也。字林音辽。复,方目翻。汉法,除其赋、税、役,皆谓之复。〕事末利及怠而贫者,举以为收孥。〔索隐曰:末利,谓工、商。纠举而收录其妻子,没为奴婢。秦法,一人有罪,收其室家。至汉文帝元年,始除收孥相坐法。孥,音奴。〕宗室非有军功论,〔论,议也,有战功之可论也。论,卢困翻,康卢昆切。〕不得为属籍。〔属籍,宗属之籍也。孔颖达曰:汉之同宗有属籍,则周家系之以姓是也。周礼小史之官,掌定帝系、世本,知世代昭穆。属,殊玉翻。〕明尊卑爵秩等级,各以差次〔白虎通曰:爵者,尽也,所以尽人才也。毛晃曰:大夫以上预燕飨,然后赐爵秩,以章有德。秩,职也,官也,积也,次也,常也,序也。〕名田宅、臣妾、衣服。有功者显荣,无功者虽富无所芬华。

  令既具未布,恐民之不信,乃立三丈之木于国都巿南门,募民有能徙置北门者予十金。〔予,读曰与。〕民怪之,莫敢徙。复曰:"能徙者予五十金。"〔复,扶又翻。〕有一人徙之,辄予五十金。〔李云:金方寸重一斤,为一金。程大昌演繁露曰:二十两为一金,亦为一镒。〕乃下令。

  令行期年,秦民之国都〔之,往也,如也。〕言新令之不便者以千数。于是太子犯法。卫鞅曰: "法之不行,自上犯之。太子,君嗣也,〔嗣,祥吏翻。〕不可施刑,刑其傅公子虔,黥其师公孙贾。〔墨涅其面曰黥。黥,音渠京翻。为后秦杀商君鞅张本。〕明日,秦人皆趋令。〔索隐曰:趋者,向也,附也,音七喻翻。〕行之十年,秦国道不拾遗,山无盗贼,民勇于公战,怯于私斗,乡邑大治。〔自是年至三十一年商鞅死,盖鞅之行其法而致效在十年之间,又十年而致祸。治,直吏翻。〕秦民初言令不便者,有来言令便。卫鞅曰:"此皆乱法之民也!"尽迁之于边。其后民莫敢议令。

  臣光曰:夫信者,人君之大宝也。〔夫,音扶。〕国保于民,民保于信,非信无以使民,非民无以守国。是故古之王者 不欺四海,〔孔颖达曰:自今本昔曰古。〕霸者不欺四邻,善为国者不欺其民,善为家者不欺其亲。不善者反之,欺其邻国,欺其百姓,甚者欺其兄弟,欺其父子。上不信下,下不信上,上下离心,以至于败。所利不能药其所伤,所获不能补其所亡,岂不哀哉!昔齐桓公不背曹沫之盟,晋文公不贪伐原之利,魏文侯不弃虞人之期,〔姓谱:曹本自颛顼之玄孙陆终之子六安,是为曹姓。周武王封曹狭于邾,故邾,曹姓也。又云:曹,叔振铎之后,武王母弟也,后以为氏。史记:齐桓公伐 鲁,鲁庄公请平,桓公许之,与盟于柯。将盟,曹沫以匕首劫桓公于坛上,请反鲁之侵地。桓公许之,曹沫去匕首而就臣位。桓公后悔,欲杀曹沫,管仲不可,遂反所侵地于鲁。诸侯闻之,皆信齐而欲附焉。左传:晋文公围原,命三日之粮。原不降,命去之。谍出,曰:"原将降矣。"军吏曰:"请待之。"公曰:"得原失信,所亡滋多。"退一舍而原降。魏文侯事见上卷威烈王二十三年。背,蒲妹翻。索隐曰:沫,音亡葛翻。左传、谷梁并作"曹刿"。然则沫宜音刿,沫、刿声相近而字异耳。〕秦孝公不废徙木之赏。此四君者道非粹白,而商君尤称刻薄,又处战攻之世,天下趋于诈力,犹且不敢忘信以畜其民,〔处,昌吕翻。趋,七喻翻。畜,许六翻,养也。〕况为四海治平之政者哉!〔治,直吏翻。〕

  ②韩懿侯薨,子昭侯立。〔谥法:昭德有劳曰昭;圣闻周达曰昭。〕

  十一年(癸亥,公元前三五八年)

  ①秦败韩师于西山。〔自宜阳熊耳东连嵩高,南至鲁阳,皆韩之西山。败,补迈翻。〕

  十二年(甲子,公元前三五七年)

  ①魏、韩【章:十二行本"韩"作"赵";乙十一行本同;孔本同;张校同。】会于鄗。〔班志,鄗县属中山郡。此时为赵地,后汉改曰高邑,唐为赵州柏乡县、赞皇县地。鄗,呼各翻。〕

  十三年(乙丑,公元前三五六年)

  ①赵、燕会于阿。〔燕,因肩翻。〕

  ②赵、齐、宋会于平陆。

  十四年(丙寅,公元前三五五年)

  ①齐威王、魏惠王会田于郊。惠王曰:"齐亦有宝乎?"威王曰:"无有。"惠王曰:"寡人国虽小,尚有径寸之珠,照车前后各十二乘者十枚。〔乘,绳证翻。〕岂以齐大国而无宝乎?"威王曰:"寡人之所以为宝者与王异。吾臣有檀子者,〔姓谱云:齐公族有食采于瑕丘檀城,因以为氏。〕使守南城,〔城在齐之南境,故曰南城。〕则楚人不敢为寇,泗上十二诸侯皆来朝。〔朝,直遥翻。〕吾臣有盼子者,使守高唐,则赵人不敢东渔于河。〔盼,匹苋翻,又披班翻。按丁度集韵,盼,与盻同。盼子,齐之同姓,即田盼也。班志,高唐县属平原郡。杜预曰:祝阿西北有高唐城。宋白曰:齐州章丘县,古高唐,春秋、战国之时为齐邑,故城在废禹城县西四十里。唐之禹城,汉祝阿也。〕吾吏有黔夫者,使守徐州,〔姓谱:齐有黔敖、则黔亦姓也,音其淹翻。司马彪曰:鲁国薛县,六国时曰徐州。徐,音舒。丁度集韵"徐"作"俆",音同。〕则燕人祭北门,赵人祭西门,〔燕在齐之北,赵在齐之西。贾逵曰:燕、赵畏齐,故祭以求福。燕,因肩翻。〕徙而从者七千余家。吾臣有种首者,使备盗贼,则道不拾遗。〔种,章勇翻。〕此四臣者,将照千里,岂特十二乘哉!"惠王有惭色。

  ②秦孝公、魏惠王会于杜平。〔班志,京兆有杜陵县,故周之杜伯国也。史记灌婴传:婴以昌平侯食邑于杜平乡。正义曰:杜平在唐之同州澄城县界。魏世家作"社平"。〕

  ③鲁共公薨,子康公毛立。〔共,读曰恭。〕

  十五年(丁卯,公元前三五四年)

  ①秦败魏师于元里,〔史记正义曰:元里亦在同州澄城县界。败,补迈翻。〕斩首七千级,〔秦法战而斩敌人一首者,赐爵一级,因谓之级。〕取少梁。〔少,诗照翻。〕

  ②魏惠王伐赵,围邯郸。楚王使景舍救赵。〔邯,音寒。郸,音丹。昭、屈、景,皆楚之同姓,楚强族也,屈,九勿翻。〕

  十六年(戊辰,公元前三五三年)

  ①齐威王使田忌救赵。

  初,孙膑与庞涓俱学兵法,〔姓谱:周文王子康叔封于卫,至武公子惠孙曾耳为卫上卿,因氏焉,后有孙武、孙膑,俱 善兵。赵明诚金石录有汉安平相孙根碑云:先出自有殷之裔子,武王定周,封比干墓,胤裔分析,定曰孙焉。姓谱又曰:庞姓,毕公高之后,支庶封于庞,因氏焉。 膑,频忍翻,刖刑也,去膝盖骨。郑玄曰:周改膑作刖,刖,断足也。书传云:决关梁、逾城郭而略盗者,其刑膑。孙膑盖以刖足故呼为膑。说文:膑,膝端也;类篇:毗宾切。庞,薄江翻。涓,古玄翻。〕庞涓仕魏为将军,〔将军之官,自周以来有之。〕自以能不及孙膑,乃召之;至,则以法断其两足而黥之,〔断,丁管翻。〕欲使终身废弃。齐使者至魏,孙膑以刑徒阴见,说齐使者,〔齐使,疏吏翻,说,式芮翻。〕齐使者窃载与之齐。〔之,往也。〕田忌善而客待之,进于威王。威王问兵法,遂以为师。于是威王谋救赵,以孙膑为将;辞以刑余之人不可,乃以田忌为将而孙子为师,居辎车中,坐为计谋。〔将,即亮翻。字林曰:軿车,有衣蔽、无后辕者谓之辎。释名曰:有邸曰辎,无邸曰軿。傅子曰:周曰辎车,即辇也。康曰:軿车也,军行所以载辎重。辎,楚持翻。軿,蒲眠翻。重,直用翻。〕

  田忌欲引兵之赵。孙子曰:"夫解杂乱纷纠者不控拳,〔索隐曰:谓事之杂乱纷纠也。解杂乱纷纠者,当善以手解之,不可控拳而击之。余谓杂乱纷纠者,谓人斗者耳,非事也。康曰:拳,与絭同。絭者,攘臂绳也。余谓当从索隐说,康说非。夫,音扶。〕救斗者不搏撠,〔索隐曰:搏撠,音博戟,谓救斗者当善撝解之,毋以手相搏撠,则其怒益炽矣。按撠,谓以手持撠以刺人也。余谓索隐之说善矣,但以撠为持撠以刺人则非也。撠,如汉书"撠太后掖"之撠,师古曰:撠,谓拘持之也。毛晃曰:索持曰搏,拘持曰撠。〕批亢捣虚,形格势禁,则自为解耳。〔索隐曰:批,白结翻。亢,苦浪翻。按批者,相排批也,音白灭翻。亢,言敌人相亢拒也。捣者,击也,冲也。虚,空也。谓前人相亢,必须批之,彼兵若虚则冲捣之,若批其相亢,击捣彼虚,则是其形相格,其势自禁止,则彼自为解也。康曰:亢,极也,高也。捣,筑也。乘其高亢而批之,乘其虚而捣之,则其势自解。批亢捣虚,所谓形格势禁也。余谓索隐之说为长。盖斗者方相亢拒,则排批之使解;虚者,两敌距斗力所不及之处,捣之则虽欲斗,其势不能不解,此易见也。格,各额翻,格正也,又击也,斗也。吴都赋:"巢巢(果改禺)笑而被格",本音如字,协韵音阁。巢(果改禺),与狒同,音父沸翻。〕今梁、赵相攻,轻兵锐卒必竭于外,老弱疲于内;子不若引兵疾走魏都,据其街路,冲其方虚,〔康曰:虚,音墟。余谓虚,如字,冲其方虚,即上所谓"捣虚"也。索隐之说,义亦如此。走,则凑翻。〕彼必释赵以自救:是我一举解赵之围而收弊于魏也。"田忌从之。十月,邯郸降魏。〔邯,音寒。郸,音丹。降,户江翻。〕魏师还,与齐战于桂陵,魏师大败。〔还,从宣翻,又音如字。水经注:濮渠与酸水会,水东迳滑台城南,又东南迳瓦亭南,又东南会于濮。濮渠之侧有漆城。桂城亦曰桂陵,即田忌败魏师处。史记正义曰:桂陵在曹州乘氏县东南二十一里。濮,博木翻。〕

  ②韩伐东周,取陵观、廪丘。〔周室衰微,战国之时仅有七邑,汉时之河南、洛阳、谷成、平阴、偃师、巩、缑氏是也。晋志曰:周考王封周桓公孙惠公于巩,号东周,故战国有东、西周,芒山、首山其界也。陵观、廪丘皆当时邑聚之名,史无所考。廪丘,史记作"邢丘"。观,古玩翻。〕

  ③楚昭奚恤为相。江乙言于楚王曰:"人有爱其狗者,狗尝溺井,〔昭、屈、景,楚之强族,所谓"三闾"者也。太史公曰:嬴姓分封为江氏。相,息亮翻。溺,奴吊翻。〕其邻人见,欲入言之,狗当门而噬之。今昭奚恤常恶臣之见,亦犹是也。〔噬,时制翻。见,谓见楚王也。恶,乌路翻。〕且人有好扬人之善者,王曰:『此君子也』近之;好扬人之恶者,王曰:『此小人也,』远之。〔好,呼到翻。近者,附近之近,去声。远,于愿翻,推而远之。推,吐雷翻。〕然则且有子弑其父、臣弑其主者,而王终己不知也。〔己,音纪。终己,犹言终身也。〕何者?以王好闻人之美而恶闻人之恶也。"王曰:"善,寡人愿两闻之。"〔江乙欲毁昭奚恤,故先设是言。〕

  十七年(己巳,公元前三五二年)

  ①秦大良造伐魏。〔索隐曰:大良造,即大上造。余谓大良造,大上造之良者也。按史记秦纪:孝公十年,卫鞅为大良造,将兵围魏安邑,降之。又按六国年表,秦孝公之十年,显王之十七年,所谓大良造伐魏,即卫鞅将兵也。是时魏都安邑,其兵犹强,庞涓、太子申、公子昂未败,安邑不应遽降于秦。至显王二十九年,昂军既败,魏献河西之地于秦,始去安邑徙都大梁。史记六国表不书徙大梁而世家书之,魏世家于是年不书安邑降秦而秦纪孝公十年书之。通鉴从魏世家,于显王二十九年书魏去安邑,徙大梁,而是年不书魏安邑降秦,盖亦疑而除去之。但大良造之下当有"卫鞅"二字,意谓传写通鉴者逸之。〕【章:十二行本正有"卫鞅"二字;乙十一行本同;孔本同;退斋校同。】

  ②诸侯围魏襄陵。〔史记正义曰:襄陵故城,在兖州邹县。余按魏境时不至于邹。班志,河东有襄陵县。师古曰:晋襄公之陵,因以名县。括地志:襄陵在晋州临汾县东南三十五里。宋白曰:后魏为禽昌县;隋大业二年改为襄陵县,以赵襄子、晋襄公俱陵于是邑也〕。

  十八年(庚午,公元前三五一年)

  ①秦卫鞅围魏固阳,降之。〔魏有上郡,北至固阳,汉五原郡稒阳县是也。括地志:固阳在银州银城县界。按魏筑长城,自郑滨洛,北抵银州,至胜州固阳县为塞也。固阳有连山,东至黄河,西南至夏、会等州。降,户江翻。夏,户雅翻。〕

  ②魏人归赵邯郸。〔邯,音寒。郸,音丹。〕与赵盟漳水上。〔记曲礼曰:涖牲曰盟。盟者,杀牲歃血,誓于神也。天下太平之时,诸侯不得擅相与盟,惟天子巡狩至方岳之下,会毕,乃与诸侯相盟,同好恶,奖王室,以昭事神、训民、事君,凡国有疑则盟,诅其不信者。至于五霸,有事而会,不协而盟。盟之为法,先凿地为方坎,杀牲于坎上,割牲左耳,盛以珠盘;又取血,盛以玉敦;用血为盟书,成,乃歃血而读书。左传云:"坎用牲加书,"是也。班志:浊漳水出上党长子县鹿谷山,东至邺,入清漳。水经曰:出长子县发鸠山,东至武安县与清漳会,谓之交漳口。又东过邺县列人,又东北过巨鹿信都,谓之衡漳;又东北过平舒县南而东入海。漳,诸良翻。〕

  ③韩昭侯以申不害为相。〔諡法:昭德有劳曰昭;圣闻周达曰昭。姓谱:四岳之后封于申。周有申伯,郑有大夫申侯,齐有申鲜虞。相,息亮翻。〕

  申不害者,郑之贱臣也,学黄、老、刑名,以干昭侯。〔黄、老,黄帝老子之书。〕昭侯用为相,内修政教,外应诸侯,十五年,终申子之身,国治兵强。〔治,直吏翻。〕

  申子尝请仕其从兄,〔从,才用翻;群从之从同。〕昭侯不许,申子有怨色。昭侯曰:"所为学于子者,欲以治国也。〔为,于伪翻。治,直之翻。〕今将听子之谒而废子之术乎!已其行子之术而子之请乎?子尝教寡人修功劳,视次第;今有所私求,我将奚听乎?"申子乃辟舍请罪曰:"君真其废人也!"〔辟,读曰避。〕

  昭侯有弊袴,命藏之。〔袴,故翻,胫衣也。〕侍者曰:"君亦不仁者矣,不赐左右而藏之!"昭侯曰:"吾闻明主爱一嚬一咲,嚬有为嚬,咲有为咲。今袴岂特嚬咲哉!吾必待有功者。"〔言袴虽弊,其直犹重,固不止于嚬咲也。然人主之嚬咲,所关甚大,昭侯姑以此为言耳。为,于伪翻。嚬,与颦同,愁蹙之貌。咲,古笑字。〕

  十九年(辛未,公元前三五零年)

  ①秦商鞅筑冀阙宫庭于咸阳,〔索隐曰;冀阙,即魏阙也。尔雅:观谓之阙。郭璞曰:宫门双阙也。释名;阙在门两旁,中间阙然为道也。三辅黄图曰:人臣至此,必思其所阙少。尔雅,宫谓之室。郭璞曰:宫,谓围绕之也。说文曰:庭,朝中也。苍颉篇曰:庭,直也。风俗通曰:庭,正也。言县庭。郡庭、朝庭,皆取平均正直也。三辅黄图曰:山南为阳,水北为阳。山水皆在阳,故曰咸阳。汉高帝更名新城,武帝更名渭城,属右扶风。括地志:咸阳故城,在雍州咸阳县东十五里,在长安城北四十五里。宋白曰:咸阳县本周王季所都,秦又都之。三秦记:秦都在九嵕山南,渭水北,山水俱阳,故名咸阳。二十九年,秦始封卫鞅于商,号商君;史以后所封书之。〕徙都之。令民父子、兄弟同室内息者为禁。〔息,止也。秦俗,父子、兄弟同室居止,商君始更制,禁同室内息者。尧教民以人伦,教之有序有别。秦用西戎之俗,至于男女无别,长幼无序。商君今为之禁,古道也,乌可例言之!白虎通曰:父,矩也,以法度教子也。子,孳也,孳孳无已也。兄,况也,况父法也。,弟,悌也,心顺、行笃也。〕井诸小乡聚,集为一县,县置令、丞,凡三十一县。废井田,开阡陌。〔周礼,六乡,乡万二千五百家。又百家之内曰乡,五鄙为县,县二千五百家,此六遂之县也。四甸为县,此州里之县也。周制:天子地方千里,分为百县,县有四郡。左传赵鞅所谓"上大夫受县,下大夫受郡"者也。秦并天下,置三十六郡,以监天下之县,自是始统于郡矣。释名曰:县,悬也,悬于郡也。汉书音义所谓"大曰乡,小曰聚",亦秦制也。广雅曰:聚,聚居也,音慈谕翻。县令、丞之官始此。令,音力正翻。令,命也,告也,律也,法也,长也;使为一县之长,以行诰命法律也。丞,翊也,副贰也。成周之制,田方里为井,井九百亩,八家各耕百亩;其中百亩,八十亩为公田,二十亩为庐舍。史记正义曰:南北曰阡,东西曰陌。刘伯庄曰:开田界道,使不相干。长,知两翻。〕平斗、桶、权、衡、丈、尺。〔桶,索隐音统,非也;当作"甬",音勇,斛也。沈括曰:予受诏考钟律及铸浑仪,求秦、汉以来度、量、斗、升,计六斗当今之一斗七升九合,秤三斤当今十三两,一斤当今四两三分两之一,一两当今六铢半。为升中方,古尺二寸五分十分分之三,今尺一寸八分百分分之四十五强。〕

  ②秦、魏遇于彤。〔彤,周彤伯所封之国,国于王畿之内。史记六国年表:商君反,死彤地。则其地当在汉京兆郑县界。彤,徒冬翻。〕

  ③赵成侯薨,公子绁与太子争立;绁败,奔韩。〔绁,私列翻。赵成侯,敬侯之子,名种。太子,肃侯语也。〕

  二十一年(癸酉,公元前三四八年)

  ①秦商鞅更为赋税法,行之。〔井田既废,则周什一之法不复用,盖计亩而为赋税之法。更,工衡翻。〕

  二十二年(甲戌,公元前三四七年)

  ①赵公子范袭邯郸,不胜而死。〔邯,音寒。郸,音丹。〕

  二十三年(乙亥,公元前三四六年)

  ①齐杀其大夫牟。

  ②鲁康公薨,子景公偃立。

  ③卫更贬号曰侯,服属三晋。〔周成王封康叔为卫侯,其后世进爵为公;今寖以弱小,贬号曰侯。贬,悲检翻。〕

  二十五年(丁丑,公元前三四四年)

  ①诸侯会于京师。〔时天下宗周,以洛阳为京师。京,大也;师,众也;京师,众大之名也。〕

  二十六年(戊寅,公元前三四三年)

  ①王致伯于秦,〔伯,如字。伯者,周二伯、九伯之任。〕诸侯皆贺秦。秦孝公使公子少官帅师会诸侯于逢泽以朝王。〔左传:逢泽有介麋焉,宋地也。杜预注曰:地理志言逢泽在荥阳开封县东北;远,疑非。括地志曰:逢泽在汴州浚仪县东南二十四里。帅,音率。〕

  二十八年(庚辰,公元前三四一年)

  ①魏庞涓伐韩。韩请救于齐。齐威王召大臣而谋曰:"蚤救孰与晚救?"成侯曰:"不如勿救。"〔邹忌为齐相,封成侯。〕田忌曰:"弗救则韩且折而入于魏,〔折,而设翻。〕不如蚤救之。"孙膑曰:"夫韩、魏之兵未弊而救之。〔膑,频忍翻,又毗宾翻。夫,音扶。〕是吾代韩受魏之兵,顾反听命于韩也。且魏有破国之志,韩见亡,必东面而愬于齐矣。〔见亡,言见有亡国之势也。愬,告愬也。〕吾因深结韩之亲而晚承魏之弊,则可受重利而得尊名也。"王曰:"善。"乃阴许韩使而遣之。〔阴,暗也。使,疏吏翻。〕韩因恃齐。五战不胜,而东委国于齐。

  齐因起兵,使田忌、田婴、田盼将之,〔盼,与盻同,音匹苋翻。将,即亮翻;下同。又音如字,领也。〕孙子为师,以救韩,直走魏都。〔走,音奏。〕庞涓闻之,去韩而归。〔庞,薄江翻。涓,工玄翻。〕魏人大发兵,以太子申为将,以御齐师。孙子谓田忌曰:"彼三晋之兵素悍勇而轻齐,〔将,即亮翻。悍,下罕翻,又音汗。〕齐号为怯,善战者因其势而利导之。兵法:『百里而趣利者蹶上将,五十里而趣利者军半至。』"〔此孙武子兵法也。趣,七喻翻。魏武帝曰:蹶,其月翻。蹶,犹挫也。刘氏曰:蹶,犹毙也。半至,谓军趣利前后不相属,半至半不至也。属,陟玉翻。〕乃使齐军入魏地为十万灶,明日为五万灶,又明日为二万灶。庞涓行三日,大喜曰:"我固知齐军怯,入吾地三日,士卒亡者过半矣!"〔过,工禾翻。〕乃弃其步军,〔句断。庞,薄江翻。涓,圭渊翻。〕与其轻锐倍日并行逐之。〔并行,兼程而行也。倍日,一日行两日之程,亦兼程也。〕孙子度其行,暮当至马陵,〔司马彪志:魏郡元城县。注云:左传成七年,会马陵;杜预注,在县东南,庞涓死处。虞喜志林:马陵在濮州鄄城东北六十里,涧谷深,可以置伏。度,徒洛翻。鄄,吉掾翻。〕马陵道陿而旁多阻隘,可伏兵,〔陿,与狭同。隘,乌懈翻。〕乃斫大树,白而书之曰:"庞涓死此树下!"于是令齐师善射者万弩夹道而伏,期日暮见火举而俱发。庞涓果夜到斫木下,见白书,以火烛之,读未毕,万弩俱发,魏师大乱相失。庞涓自知智穷兵败,乃自刭,曰:"遂成竖子之名!"〔庞,薄江翻。涓,工玄翻。刭,古顶翻,断首也;康古定切,非。竖,殊遇翻。说文:竖使布短衣。〕齐因乘胜大破魏师,虏太子申。

  ②成侯邹忌恶田忌,〔邹,以国为氏。恶,乌路翻。〕使人操十金,卜于市,〔操,七刀翻。〕曰。:"我,田忌之人也。我为将三战三胜,欲行事,可乎?"卜者出,因使人执之。田忌不能自明,率其徒攻临淄,〔临淄,齐国都也;城临淄水,因以为名。班志,临淄属齐国。臣瓒曰:临淄,即营丘,太公营之。淄,庄持翻。〕求成侯;不克,出奔楚。〔为下齐复田忌张本。〕

  二十九年(辛巳,公元前三四零年)

  ①卫鞅言于秦孝公曰:"秦之与魏,譬若人有腹心之疾,非魏并秦,秦即并魏。何者?魏居岭厄之西,〔索隐曰:盖安邑以东,山岭险厄之地,今蒲州中条以东,连汾、晋之险嶝,皆其地也。厄,于革翻。〕都安邑,与秦界河,〔秦、魏以河为界也。〕而独擅山东之利,〔擅,市战翻。〕利则西侵秦,病则东收地。今以君之贤圣,国赖以盛;而魏往年大破于齐,诸侯畔之,可因此时伐魏。魏不支秦,必东徙,然后秦据河、山之固,东乡以制诸侯,〔乡,读曰向。〕此帝王之业也。"公从之,使卫鞅将兵伐魏。魏使公子昂将而御之。

  军既相距,卫鞅遗公子昂书曰:"吾始与公子驩;今俱为两国将,〔将,即亮翻。遗,于季翻。〕不忍相攻,可与公子面相见盟,乐饮而罢兵,以安秦、魏之民。"〔乐音洛。〕公子昂以为然,乃相与会;盟已,饮,〔盟已而饮也。〕而卫鞅伏甲士,袭虏公子昂,因攻魏师,大破之。

  魏惠王恐,使使献河西之地于秦以和。〔使使,下疏吏翻。〕因去安邑,徙都大梁。〔班志:陈留郡浚仪县,故大梁。杜佑曰:汴州城西古城,战国时魏惠王所筑。〕乃叹曰:"吾恨不用公叔之言!"〔公叔言见上八年。〕

  秦封卫鞅商于十五邑。〔班志:弘农郡商县,商君邑。裴駰曰:商于之地在今顺阳郡南乡、丹水二县,有商城在于中,所谓之商于。史记正义曰:丹水及商皆属弘农,今言顺阳,是魏、晋始分置顺阳郡,商及丹水皆属之也。水经注:丹水泾南乡、丹水二县之间,历于中之北,所谓商于者也。杜佑曰:今邓州内乡县东七里有于村,盖秦所谓商州。商洛县,古商邑,卨所封也;汉为商县。于,如字。〕号曰商君。

  ②齐、赵伐魏。

  ③楚宣王薨,子威王商立。

  三十一年(癸未,公元前三三八年)

  ①秦孝公薨,子惠文王立。公子虔之徒告商君欲反,发吏捕之。商君亡之魏;〔之,如也,往也。〕魏人不受,复内之秦。〔内,读曰纳。怨其挟诈以破魏师,故不受。〕商君乃与其徒之商于,发兵北击郑,〔之,往也,如也。郑,京兆之郑县也,周宣王弟郑桓公采邑,唐属华州。宋白续通典曰:郑县古城在华州郡城北。〕秦人攻商君,杀之,车裂以徇,尽灭其家。〔车裂,古之轘刑。轘,户串翻。〕

  初,商君相秦,用法严酷,尝临渭论囚,渭水尽赤。〔相,息亮翻。水经:渭水出陇西首阳县鸟鼠山,东流至秦都咸阳南。商君盖临此以论囚。决罪曰论。论,卢困翻。〕为相十年,人多怨之。〔按显王十七年,秦以商鞅为大良造;十九年,商鞅徙秦都咸阳,废井田,开阡陌,平权量。二十一年,更赋税法,为相当在是年,至今年十年矣。〕赵良见商君,商君问曰:"子观我治秦〔治,直之翻。〕孰与五羖大夫贤?"〔百里奚自卖以五羖羊之皮,为人养牛;秦穆公举以为相,秦人谓之五羖大夫。羖,牡羊也。羖,音古。〕赵良曰:"千人之诺诺,不如一士之谔谔。〔引赵简子之言。诺,应声也。谔,謇直也。〕仆请终日正言而无诛,可乎?"商君曰: "诺。"赵良曰:"五羖大夫,荆之鄙人也,穆公举之牛口之下:〔孟子:百里奚,虞人也,以食牛干秦缪公。今曰荆之鄙人,按史记:晋灭虞,执百里傒,为秦缪夫人媵。百里傒亡秦走宛,楚鄙人执之;缪公以五羖羊皮赎之,以为上大夫。傒,读与奚同。缪,读与穆同。媵,以证翻。宛,于元翻。〕而加之百姓之上,秦国莫敢望焉。相秦六七年而东伐郑,〔谓左传僖三十年与晋围郑也。相,息亮翻。〕三置晋君,一救荆祸。〔三置晋君,谓立惠公、怀公、文公也。索隐曰:十二诸侯年表,穆公二十八年,会晋伐楚朝周;此云救荆,未详。余按左传,晋既败楚于城濮,又败秦于肴,穆公使鬭克归楚求成,所谓救荆祸,盖指此也。秦讳楚,故其国记率谓楚为"荆"。太史公取秦记为史记,通鉴又因史记而成书,故亦以楚为荆。〕其为相也,劳不坐乘,〔古者车立乘,惟安车则坐乘耳。〕暑不张盖。〔周礼:轮人为盖。盖,所以覆冒车上也。〕行于国中,不从车乘,〔乘,绳证翻。〕不操干戈。〔操,七刀翻。〕五羖大夫死,秦国男女流弟,童子不歌谣,舂者不相杵。〔记:邻有丧,舂不相。注云:相杵者,以音声相劝。相,息亮翻。〕今君之见也,因嬖人景监以为主;〔事见上八年。嬖,卑义翻,又博计翻。监,甲暂翻。〕其从政也,淩轹公族,残伤百姓。〔轹,郎击翻。车践曰轹。〕公子虔杜门不出已八年矣。君又杀祝懽而黥公孙贾。〔祝,姓也。古有巫,史、祝之官,其子孙因以为姓。或曰:武王封黄帝之后于祝,其子孙因氏焉。黥,其京翻。〕诗曰:『得人者兴,失人者崩。』〔逸诗也。〕此数者,非所以得人也。君之出也,后车载甲,多力而骈胁者为骖乘,〔杜预曰:骈胁,合干也。骈,步田翻。乘,绳证翻。骖,读曰参。〕持矛而操闟戟者旁车而趋。〔薛综曰:闟之为言函也,取四戟函车边。此盖令力士旁车而趋,有急则操翕戟以御之也。后汉志有闟戟车。晋志:闟戟车,长戟邪偃在后。唐韵:戟名曰闟,音所及翻。史记正义曰:顾野王云:矛,鋋也。方言云:矛,吴、楚、江、淮之间谓之鋋。释名曰:戟,格也,旁有枝格。旁车之旁,音步浪翻。〕此一物不具,君固不出。书曰:『恃德者昌,恃力者亡。』〔逸书也。〕此数者,非恃德也。君之危若朝露,〔朝露易曦,言不久也。〕而尚贪商于之富,宠秦国之政,〔言以专秦国之政为宠也。〕畜百姓之怨。〔畜,读曰蓄。〕秦王一旦捐宾客而不立朝,〔朝,直遥翻。〕秦国之所以收君者岂其微哉!"〔微,少也。赵良言岂少,盖谓太子与其师傅将挟怨而杀之也。〕商君弗从。居五月而难作。〔难,乃旦翻。史言商君尚刑愎谏之祸速。〕

  三十二年(甲申,公元前三三七年)

  ①韩申不害卒。〔卒,子恤翻。〕

  三十三年(乙酉,公元前三三六年)

  ①宋太丘社亡。〔班志,沛郡有太丘县。又志曰:宋太丘社亡,周鼎沦没于泗水中。尔雅:右陵太丘。释云:谓丘之西有大阜者为太丘。宋太丘社亡,盖依丘作社,于时亡去,咎证也。〕

  ②邹人孟轲见魏惠王,〔邹,春秋之邾国也。班志,邹县属鲁国。宋白曰:淄州邹平县,汉旧县。〕王曰:"叟,〔叟者,尊老之称。称,尺证翻。〕不远千里而来,亦有以利吾国乎?"孟子曰:"君何必曰利,仁义而已矣!〔不远千里,言不以千里为远也。〕君曰何以利吾国,大夫曰何以利吾家,士庶人曰何以利吾身,上下交征利而国危矣。未有仁而遗其亲者也,未有义而后其君者也。"〔后,户豆翻。〕王曰:"善。"〔通鉴于此段前后书王,因孟子之文也。中间叙孟子答魏王之言,独改"王"曰"君",不与魏之称王也。〕

  初,孟子师子思,尝问牧民之道何先。子思曰:"先利之。"孟子曰:"君子所以教民者,亦仁义而已矣,何必利!"子思曰:"仁义固所以利之也,上不仁则下不得其所。上不义则下乐为诈也,〔乐,音洛。〕此为不利大矣。故易曰:『利者,义之和也。』〔易干卦文言。〕又曰:『利用安身,以崇德也。』〔易大传之辞。〕此皆利之大者也。"

  臣光曰:子思、孟子之言,一也。夫唯仁者为知仁义之为【章:十二行本无"为"字;乙十一行本同。】利,不仁者不知也。〔夫,音扶。〕故孟子对梁王直以仁义而不及利者,所与言之人异故也。

  三十四年(丙戌,公元前三三五年)

  ①秦伐韩,拔宜阳。

  三十五年(丁亥,公元前三三四年)

  ①齐王、魏王会于徐州以相王。〔史记正义曰:竹书纪年云:梁惠王三十年,下邳迁于薛,改曰徐州。续汉志曰:鲁国薛县,六国时曰徐州。与竹书合。徐,音舒。相王者,相立为王也。〕

  ②韩昭侯作高门,屈宜臼曰:"君必不出此门。〔许慎曰:屈宜臼,楚大夫,时在韩。屈,九勿翻。〕何也?不时。吾所谓时者,非时日也。夫人固有利、不利时。〔夫,音扶。〕往者君尝利矣,不作高门。前年秦拔宜阳,今年旱,君不以此时恤民之急而顾益奢,此所谓时诎举赢者也。〔诎,区勿翻。徐广曰:时衰耗而作奢侈,言国家多难而势诎,此时宜恤民之急,而举事反若有赢余者,失其所以为国之道矣。"时诎举赢",盖古语也。赢,怡成翻。〕故曰不时。"

  ③越王无彊伐齐。〔越王句践之后。自句践至无彊,凡六世。句,音钩。践,音慈浅翻。〕齐王使人说之以伐齐不如伐楚之利。〔说,式芮翻。〕越王遂伐楚。楚人大败之,〔败,补迈翻。〕乘胜尽取吴故地,东至于浙江。越以此散,诸公族争立,或为王,或为君,滨于海上,〔吴之故地,汉会稽、九江、丹杨、豫章、庐江、广陵、临淮等郡是也。越初都会稽,其境北至于御儿,不能全有汉会稽一郡地;及其灭吴,始并有吴地。今楚取吴地至于浙江;则御儿亦入于楚矣。浙江有三源:发于太末者谓之榖水,今之衢港是也;发于乌伤者,水经谓之吴宁溪,今之婺港是也;发于黝县者,班志谓之渐江水,今之徽港是也,三水合为浙江,东至钱唐入海。浙,折也,言水屈折于群山之间也。释名曰:江,共也,小水流入其中,所公共也。国于海上者,汉之瓯越、闽越、骆越其后也。浙,之列翻。滨,音宾。会,古外翻。太末之太,孟康音闼。港,古项翻。婺,亡遇翻。黝,音伊。闽,眉巾翻。骆,音洛。〕朝服于楚。〔朝,直遥翻。〕

  三十六年(戊子,公元前三三三年)

  ①楚王伐齐,围徐州。〔徐,音舒。〕

  ②韩高门成。昭侯薨,〔卒如屈宜臼言。卒,子恤翻。〕子宣惠王立。〔宣惠,复諡也。〕

  ③初,洛阳人苏秦说秦王以兼天下之术,〔说,式芮翻。姓谱:苏,己姓,颛顼裔孙吴回生陆终,陆终生昆吾,封于苏,至周,苏公。〕秦王不用其言。苏秦乃去,说燕文公曰:"燕之所以不犯寇被甲兵者,以赵之为蔽其南也。〔燕,因肩翻。被,皮义翻。〕且秦之攻燕也,战于千里之外;赵之攻燕也,战于百里之。〔内燕南与赵接境;战于百里之内,言其近也。秦欲攻燕,自蒲、潼下兵,则为赵所隔,故必迳上郡之西,出云中、九原然后至燕,故云战于千里之外。〕夫不忧百里之患而重千里之外,计无过于此者。〔夫,音扶。计无过于此者,言燕计之过,无甚于此。〕愿大王与赵从亲,〔从,子容翻。〕天下为一,则燕国必无患矣。"〔此苏秦为燕至计,先定于胸中者。〕

  文公从之,资苏秦车马,以说赵肃侯曰:"当今之时,山东之建国莫强于赵,〔说,式芮翻。建国,犹言立国也。〕秦之所害亦莫如赵。然而秦不敢举兵伐赵者,畏韩、魏之议其后也。秦之攻韩、魏也,无有名山大川之限,稍蚕食之,傅国都而止。〔傅,读曰附,傅着之傅。〕韩、魏不能支秦,必入臣于秦;秦无韩、魏之规则祸中于赵矣。〔中,竹仲翻。〕臣以天下地【章:十二行本"地"作"之";乙十一行本同;孔本同;张校同。】图案之,诸侯之地五倍于秦,料度诸侯之卒十倍于秦。〔料,音聊,又如字。度,徒洛翻。〕六国为一,并力西乡而攻秦,〔乡,读曰向。〕秦必破矣。夫衡人者皆欲割诸侯之地以与秦,〔衡,读曰横。衡人,说客之连横者。〕秦成则其身富荣,国被秦患而不与其忧,〔与,读曰预。〕是以衡人日夜务以秦权恐愒诸侯,以求割地。〔索隐曰:恐,起拱翻。愒,许曷翻,又呼曷翻,谓相恐胁也。邹氏愒音憩,义疏。〕故愿大王熟计之也!窃为大王计,莫如一韩、魏、齐、楚、燕、赵为从亲以畔秦,〔从,子容翻。畔,反也,及秦之所为也。秦之所为者衡也。〕令天下之将相会于洹水之上,〔令,卢经翻,使也。将,即亮翻。相,息亮翻。徐广曰:洹水出汲郡林虑县。水经:洹水出上党泫氏县东北,出山迳邺县南,又东过内黄县北,入于白沟。洹,音桓,又于元翻。虑,音庐。〕通质结盟,〔质,音致。〕约曰:『秦攻一国,五国各出锐师,或桡秦,〔服虔曰:桡,弱也,音奴教翻,又音乃卯翻。〕或救之。有不如约者,五国共伐之!』诸侯从亲以摈秦,秦甲必不敢出于函谷以害山东矣。"〔从,子容翻。摈,必刃翻。班志,弘农郡弘农县有秦函谷关;汉武帝从杨仆之请,移关于新安县。文颖曰:秦关在弘农县衡岭,后移在河南谷成县。师古曰:今桃林县有洪溜涧水,即古所谓函谷,其水北流入河,夹河之岸尚有旧关余迹焉。谷成,即新安。杜佑曰:汉函谷关在汉新安县东北一里,其秦关在今灵宝县。〕肃侯大说,〔索隐曰:肃侯,名语。諡法:刚德克就曰肃;执心决断曰肃。说,与悦同。〕厚待苏秦。尊宠赐赉之,以约于诸侯。

  会秦使犀首伐魏,大败其师四万余人,禽将龙贾,取雕阴,〔犀首,魏官名。公孙衍为此官,因号犀首,犹虎牙将军之称。龙姓出于龙伯氏,或云出于御龙氏。班志,上郡有雕--。括地志:雕阴故城,在鄜州洛交县北二十里。败,补迈翻。称,尺证翻。〕且欲东兵。〔言引兵东下也。〕苏秦恐秦兵至赵而败从约,〔从,子容翻。〕念莫可使用于秦者,乃激怒张仪,入之于秦。

  张仪者,魏人,与苏秦俱事鬼谷先生,〔盖居于鬼谷,因以称之。隋志,冯翊郡韩城县有鬼谷。风俗通义曰:鬼谷先生,六国时纵横家。索隐曰:扶风池阳、颖川阳城并有鬼谷,盖是其人所居,因以为号。又乐壹注鬼谷子书云:苏秦欲神祕其道,故假名鬼谷。〕学纵横之术,〔纵,与从同,音子容翻。〕苏秦自以为不及也。仪游诸侯无所遇,困于楚,苏秦故召而辱之。仪恐,〔"恐",史记作"怒"、〕念诸侯独秦能苦赵,遂入秦。苏秦阴遣其舍人赍金币资仪,〔文颖曰:舍人,主廏内小吏官名也。师古曰:舍人,亲近左右之通称也;后遂为司属官号。赍,则兮翻。〕仪得见秦王。秦王说之,〔说,读曰悦。〕以为客卿。〔秦有客卿之官,以待自诸侯来者,其位为卿而以客礼 待之也。〕舍人辞去,曰:"苏君忧秦伐赵败从约,〔败,补迈翻。从,子容翻。〕以为非君莫能得秦柄;故激怒君,使臣阴奉给君资,尽苏君之计谋也。"张仪曰:"嗟乎,此吾在术中而不悟,吾不及苏君明矣。为吾谢苏君,苏君之时,仪何敢言!"〔为吾之为,于伪翻。为后苏秦死,仪方出说六国张本。说,式芮翻。〕

  于是苏秦说韩宣惠王曰:"韩地方九百余里,带甲数十万,天下之强弓、劲弩、利剑皆从韩出。韩卒超足而射,百发不暇止。以韩卒之勇,被坚甲,跖劲弩,〔被,皮义翻。跖,之石翻,踏也。史记正义曰:欲放弩者皆坐,举足踏弩材,手引凑机,然后发之。〕带利剑,一人当百,不足言也。大王事秦,秦必求宜阳、成皋;〔韩之宜阳,西接境于秦,当函谷出兵之路。成皋,春秋郑之制邑,亦曰虎牢,战国时为郑之屏蔽,皆韩之地。班志,宜阳属弘农郡,成皋属河南郡。〕今兹效之,明年复求割地。〔复,扶又翻。〕与则无地以给之;不与则弃前功,受后祸。且大王之地有尽而秦求无已,以有尽之地逆无已之求,此所谓市怨结祸者也。〔市,买也。凡以物买卖贸易曰市。〕不战而地已削矣。鄙谚曰:『宁为鸡口,无为牛后。』〔谚,鱼变翻,俗言也。史记正义曰:鸡口虽小,犹进食;牛后虽大,乃出粪。尔雅翼曰:苏秦说韩王,"宁为鸡尸,无为牛从。"尸,主也;一群之主,所以将众也。从,从物者也,谓牛子也;随群而往,制不在我者也。言宁为鸡中之主,不为牛子之从后也。此本诸延笃注战国策。〕夫以大王之贤,挟强韩之兵,而有牛后之名,臣窃为大王羞之!" 〔夫,音扶。挟,户颊翻。为,于伪翻。〕韩王从其言。

  苏秦说魏王曰:"大王之地方千里,地名虽小,然而田舍庐庑之数,〔庑,文甫翻。数,七欲翻,密也。〕曾无所刍牧。〔曾,才登翻。刍,刈草也。牧,放牧也。言魏民居蕃庶,无刈刍放牧之地也。〕人民之众,车马之多,日夜行不绝,輷輷殷殷,若有三军之众。〔輷,呼宏翻。殷,音隐。〕臣窃量大王之国不下楚。〔量,吕张翻,量度也。〕今窃闻大王之卒,武士二十万,苍头二十万,奋击二十万,厮徒十万;〔武士,武卒也。详见后第六卷秦昭襄王五十二年。苍头,谓着青帽;项羽传有"异军苍头特起"。奋击,简军中之勇士敢奋力而击敌者异之。苏林曰:取薪之卒曰厮,音斯。〕车六百乘,骑五千匹;〔古者用车战,战国始用骑兵,车骑异用而并用矣。乘,绳证翻。骑,奇寄翻。〕乃听于群臣之说,而欲臣事秦!【章:乙十一行本"秦"下有"愿大王熟察之"六字;孔本同;张校同;退斋校同;百衲本缺叶。】故敝邑赵王使臣效愚计,奉明约,在大王之诏诏之。"魏王听之。

  苏秦说齐王曰:"齐四塞之国,地方二千余里,带甲数十万,粟如丘山。三军之良,五家之兵,〔三军,谓三晋之军。高诱曰:五家,即五国。〕进如锋矢,战如雷霆,解如风雨,即有军役,未尝倍泰山、绝清河、涉渤海者也。〔倍,与背同,音蒲妹翻,乡倍之倍也。班志:泰山在泰山郡博县东北。水经:淇水自馆陶清渊东北过广宗县东,为清河,汉因置清河郡;清河又东过修县,与大河张甲故渎合,又东过东光、南皮等县,齐之北界也。又齐东、北皆阻海,汉渤海郡亦其境也。师古曰:郡在渤海之滨,因以为名。直度曰绝;由膝以上曰涉。〕临淄之中七万户,臣窃度之,不下户三男子,不待发于远县,而临淄之卒固已二十一万矣。临淄甚富而实,其民无不斗鸡、走狗、六博、闒鞠。〔说文曰:六博,局戏也。六箸十二棋,乌胄所作。楚辞:箟蔽象棋有六博。鲍宏博经曰:琨蔽,玉箸也。各投六箸,行六棋,故曰六博。用十二棋,六棋白,六棋黑。所掷头谓之琼,琼有五采,刻为一画者谓之塞,刻为两画者谓之白,刻为三画者谓之黑,一边不刻者,五塞之间,谓之五塞。 "闒鞠",史记作"蹋鞠",以皮为之,实之以毛,蹴蹋而戏。刘向曰:蹴鞠起于战国之时,所以练武士,因嬉戏而讲习之;或言黄帝所作。闒,徒腊翻。〕临淄之涂,车毂击,人肩摩,连衽成帷,挥汗成雨。夫韩、魏之所以重畏秦者,为;与秦接境壤也。〔夫,音扶。为,于伪翻。〕兵出而相当,不十日而战,胜存亡之机决矣。〔"而战"句断。"胜"下当有"负"字。以此观之,文意明通。窃谓通鉴承史记元文之误。〕韩、魏战而胜秦,则兵半折,〔折,常列翻,摧折也。〕四境不守;战而不胜,则国已危亡随其后;是故韩、魏之所以重与秦战而轻为之臣也。今秦之攻齐则不然,倍韩、魏之地,过卫阳晋之道,经乎亢父之险,车不得方轨,骑不得比行,〔水经注:瓠子河出东郡濮阳县北河,南至济阴句阳县为新沟,又东过廪丘县与濮水俱东。瓠河又迳阳晋城南,苏秦所谓"卫阳晋之道"也。史记正义曰:阳晋故城在曹州乘氏县西北三十七里。班志,亢父县属东平国。又括地志:亢父故城,在兖州任城县南五十一里。亢,音抗,又音刚。,父音甫。说文:轨,车辙也。颜师古曰:车并行为方轨。骑,奇寄翻。比,毗义翻,次也。行,户刚翻,列也;凡行列之行皆同音。车并读曰并。〕百秦人守险,千人不敢过也。秦虽欲深入则狼顾,恐韩、魏之议其后也,〔尔雅翼:狼猛而敏给,能自顾其后;盖狼行而屡顾,恐人掎其后故也。掎,居绮翻。〕是故恫疑、虚喝、骄矜而不敢进,〔恫,他红翻,恐惧貌。高诱曰:虚喝,喘息惧貌。刘氏曰:秦自疑惧,虚作恐猲之辞以胁韩、魏也。史记正义曰:言秦虽至亢父,犹恐惧狼顾,虚作喝骂,骄溢矜夸而不敢进伐齐。喝,呼葛翻;亦作猲,音同。〕则秦之不能害齐亦明矣,夫不深料秦之无柰齐何,而欲西面而事之,是群臣之计过也。〔夫,音扶。〕今无臣事秦之名而有强国之实,臣是故愿大王少留意计之!"齐王许之。〔少,始绍翻。〕

  乃西南说楚威王曰:"楚,天下之强国也,地方六千余里,带甲百万,车千乘,骑万匹,〔楚在齐之西南,故苏秦自齐而西南诣楚。说,式芮翻。乘,绳证翻。骑,奇寄翻。〕粟支十年,此霸王之资也。秦之所害莫如楚,楚强则秦弱,秦强则楚弱,其势不两立。故为大王计,莫如从亲以孤秦。〔从,子容翻。〕臣请令山东之国〔令,卢经翻,使也。〕奉四时之献,以承大王之明诏;委社稷,奉宗庙,练士厉兵,在大王之所用之。故从亲则诸侯割地以事楚,衡合则楚割地以事秦,〔从,子容翻。衡,读曰横。〕此两策者相去远矣,大王何居焉?"楚王亦许之。

  于是苏秦为从约长,〔长,知丈翻。〕并相六国,〔相,息亮翻。〕北报赵,车骑辎重拟于王者。〔康曰:辎重,载物车也。行者之车,总曰辎重。韵书曰:辎,庄持翻,库车也。重,直用翻。考异曰:史记苏秦传:"秦兵不敢闚函谷关十五年。"又云;"其后秦使犀首欺齐、魏,与共伐赵,苏秦去赵而从约皆解。"齐、魏伐赵,败从约,止在明年耳。其自相违戾如此!秦本纪:"惠文王七年,公子昂与魏战,虏其将龙贾,"后二年事耳;乌在其不闚函谷十五年乎!此出于游谈之士夸大苏秦而云尔。今不取。〕

  ④齐威王薨,子宣王辟强立;知成侯卖田忌,〔事见上二十八年。〕乃召而复之。

  ⑤燕文公薨,子易王立。〔諡法:好更改旧曰易。燕,因肩翻。易,音如字。更,工衡翻。〕

  ⑥卫成侯薨,子平侯立〔諡法:治而无眚曰平; 执事有制曰平;布纲治纪曰平;又曰:惠无内德曰平。〕

  三十七年(己丑,公元前三三二年)

  ①秦惠王使犀首欺齐、魏,与共伐赵,以败从约,〔败,补迈翻。从,子容翻。〕赵肃侯让苏秦,苏秦恐,请使燕,必报齐。苏秦去赵而从约皆解。赵人决河水以灌齐、魏之师,齐魏之师乃去。

  ②魏以阴晋为和于秦,实华阴。〔班志,华阴,故阴晋,秦惠文王五年更名宁秦,汉高帝改曰华阴县,属京兆,以其地在华山之阴也。宋白曰:华阴分秦、晋之境:边晋之西,则曰阴晋;边秦之东,则曰宁秦。华,户化翻。〕

  ③齐王伐燕,取十城;已而复归之。

  三十九年(辛卯,公元前三三零年)

  ①秦伐魏,围焦、曲沃。〔班志,弘农郡陜县有焦城,左传所谓"晋与秦焦、瑕"者也。括地志:焦在陜城东百步。曲沃在陜西南三十二里,因曲沃水为名。鄜道元曰:案春秋文公十三年,晋侯使詹嘉处瑕,守桃林之塞以备秦,时以曲沃之官守之,故曲沃之名遂为积古之传。宋白曰:焦,古焦国。括地志:焦城在陜城东北百步,因焦水为名;周同姓所封。〕魏入少梁、河西地于秦。〔少,诗照翻。二十九年,魏已使使献河西于秦以和,今乃入其地。〕

  四十年(壬辰,公元前三二九年)

  ①秦伐魏,渡河,取汾阴、皮氏,〔班志,汾阴县属河东郡。皮氏县,故耿国,晋献公以封赵夙者也,亦属河东郡。括地志,汾阴故城,在蒲州汾阴县北九里,皮氏故城,在绛州龙门县西百八十步。〕拔焦。

  ②楚威王薨,子怀王槐立。〔諡法:慈仁短折曰怀;又怀,思也。槐,乎乖翻,又乎瑰翻。折,而设翻。〕

  ③宋公剔成之弟偃袭攻剔成;剔成奔齐,偃自立为君。〔剔,他历翻。〕

  四十一年(癸巳,公元前三二八年)

  ①秦公子华、张仪帅师围魏蒲阳,取之。〔史记正义曰:蒲阳在隰州隰川县,蒲邑故城是也。帅,读曰率。〕张仪言于秦王,请以蒲阳复与魏,而使公子繇质于魏。〔质,音致。〕仪因说魏王曰:"秦之遇魏甚厚,魏不可以无礼于秦。"魏因尽入上郡十五县以谢焉。〔说,式芮翻。括地志曰:上郡故城在绥州上县东南五十里,魏、秦之上郡地也。史记正义曰:按鄜、坊、丹、延等州,北至固阳,尽上郡地。魏筑长城界秦,自华郑县滨洛至庆州洛源县自于山,即东北至胜州固阳,东至河西上郡之地,尽入于秦。秦之与魏者小,魏之谢秦者大,史言张仪为秦计者甚巧。〕张仪归而相秦。〔相,息亮翻。〕

  四十二年(甲午,公元前三二七年)

  ①秦县义渠,以其君为臣。〔义渠,西戎国名,秦取之以为县。班志,义渠道属北地郡。括地志:宁、庆、原三州,秦之北地郡也。〕

  ②秦归焦、曲沃于魏。〔既取而复归之。秦之于魏,若玩弄婴儿于掌股之上耳。〕

  四十三年(乙未,公元前三二六年)

  ①赵肃侯薨:〔索隐曰:肃侯,名语。〕子武灵王立;置博闻师三人,左、右司过三人,先问先君贵臣肥义,加其秩。〔索隐曰:武灵王,名雍。姓谱:肥姓,肥子之后,以国为姓。〕

  四十四年(丙申,公元前三二五年)

  ①夏,四月,戊午,秦初称王。

  ②卫平侯薨,子嗣君立。〔嗣,祥吏翻。〕卫有胥靡亡之魏,〔汉书音义曰:胥,相也。靡,随也。古者相随坐轻刑之名,谓罪不至于扑刑者,令衣褐带索,相随以执役。朱元晦曰:胥靡者,连锁役作也。胥,新于翻。靡,母被翻。〕因为魏王之后治病。〔为,于伪翻。治,直之翻。〕嗣君闻之,【章:十二行本"之"下有"使人"二字;乙十一行本同;孔本同;退斋校同。】请以五十金买之。五反,魏不与,乃以左氏易之。左右谏曰:"夫以一都买一胥靡,可乎?"〔夫,音扶;下同。〕嗣君曰:"非子所知也!夫治无小,乱无大。〔治,直吏翻。〕法不立,诛不必,虽有十左氏,无益也。法立,诛必,失十左氏,无害也。"魏王闻之曰: "人主之欲,不听之不祥。"因载而往,徒献之。〔此学申、韩者为之说耳。〕

  四十五年(丁酉,公元前三二四年)

  ①秦张仪帅师伐魏,取陜。〔班志,陜县属弘农郡,故虢国。北虢在大阳,东虢在荥阳,西虢在雍州。周、召分陜而治,即此陜也。帅,读曰率。陜,失冉翻。召,读曰邵。〕

  ②苏秦通于燕文公之夫人,易王知之。苏秦恐,乃说易王曰:"臣居燕不能使燕重,而在齐则燕重。"易王许之。〔燕,因肩翻。易,音如字。说,式芮翻。諡法:好更改曰易。注:变故改常。〕乃伪得罪于燕而奔齐,齐宣王以为客卿。苏秦说齐王高宫室,大苑囿,以明得意,欲以敝齐而为燕。〔说,式芮翻。为后齐大夫杀苏秦张本。〕

  四十六年(戊戌,公元前三二三年)

  ①秦张仪及齐、楚之相会啮桑。〔相,悉亮翻。服虔曰:啮桑,翟地。徐广曰:在梁与彭城之间。裴駰曰:晋地。索隐曰:卫地。余按汉武帝瓠子歌曰:"啮桑浮兮淮、泗满,"及塞决河而梁、楚之地复宁,无水灾。后汉王梁击佼强、苏茂于楚、沛间,拔大梁、啮桑,则徐说为近之。啮,五结翻。〕

  ②韩、燕皆称王。赵武灵王独不肯,曰:"无其实,敢处其名乎?"令国人谓己曰君。〔赵武灵王之不肯称王,非守君臣之分,居之以谦也,将求其所大欲而力未能称心也。处,昌吕翻。令,力丁翻,使也;又力正翻,命令也。分,扶问翻。称,尺证翻。〕

  四十七年(己亥,公元前三二二年)

  ①秦张仪自啮桑还而免相,相魏。〔啮,鱼结翻。还,从宣翻,又音如字。免相,免秦相而相魏。相,息亮翻。〕欲令魏先事秦而诸侯效之;魏王不听。秦王伐魏,取曲沃、平周,〔令,力丁翻。此曲沃在河东,晋桓叔所封之邑;汉武帝改名闻喜。史记正义曰:绛州桐乡县,晋曲沃邑。十三州志:古平周邑在汾州介休县西四十里。〕复阴厚张仪益甚。

  四十八年(庚子,公元前三二一年)

  ①王崩,子慎靓王定立。〔靓,疾正翻。〕

  ②燕易王薨,子哙立。〔燕,因肩翻。易,音如字。哙,古夬翻。〕

  ③齐王封田婴于薛,〔班志,薛县属鲁国,夏奚仲之国;后迁于邳,仲虺居之。括地志:故薛城在今徐州滕县界。史记正义曰:薛故城在今徐州滕县南四十四里。〕号曰靖郭君。〔杜佑曰:战国之际,秦、项之间,权设班宠有加赐邑封君者,盖假其位号,或空受其爵,如靖郭、武安之类是也。至汉尤多,盖在封爵之外别加美号。史记列传云:婴諡为靖郭君。索隐曰:靖郭,或封邑号,故汉驷钧封靖郭侯。〕靖郭君言于齐王曰:"五官之计。不可不日听而数览也。" 〔记曾子问:诸侯出,命国家五官而后行。注云:五官,五大夫典事者。命者,敕之以其职。正义云:案太宰职云:建其牧,立其监,设其参,傅其伍,是诸侯有三卿、五大夫。经云五官,故云五大夫。以属官大夫,其数众多,直云五者,据典国事言之。不云命卿者,或从君出行,或虽在国留守,总主群吏,如三公然,不专主一事,且尊之。既命五大夫,则卿亦命之可知,故不显言命卿也。余谓此所谓五官,盖亦言典事五大夫也。数,所角翻。〕王从之;已而厌之,悉以委靖郭君。靖郭君由是得专齐之权。

  靖郭君欲城薛,客谓靖郭君曰:"君不闻海大鱼乎?网不能止,钩不能牵,荡而失水,则蝼蚁制焉。今夫齐,亦君之水也。〔夫,音扶。〕君长有齐,奚以薛为!苟为失齐,虽隆薛之城到于天,庸足恃乎!"乃不果城。〔隆,高也,崇也。庸,常也。〕

  靖郭君有子四十【章:十二行本"十"下有"余"字;乙十一行本同;孔本同。】人,其贱妾之子曰文。文通傥饶智略,〔通,达也。傥,倜傥卓异也。饶智略,言智略有余也。〕说靖郭君以散财养士。靖郭君使文主家待宾客,宾客争誉其美,〔说,式芮翻。誉,音余。〕皆请靖郭君以文为嗣。靖郭君卒,〔嗣,祥吏翻。卒,子恤翻。〕文嗣为薛公,号曰孟尝君。〔史记列传曰:諡曰孟尝君。索隐曰:号曰孟尝君;曰諡,非也。孟,字;尝,邑名。尝邑在薛之旁。〕孟尝君招致诸侯游士及有罪亡人,皆舍业厚遇之,〔舍业,为之筑舍,立居业也。〕存救其亲戚,食客常数千人,各自以为孟尝君亲己,由是孟尝君之名重天下。

  臣光曰:君子之养士,以为民也。易曰:"圣人养贤,以及万民。"〔颐卦彖辞也。为,于伪翻。〕夫贤者,其德足以敦化正俗,其才足以顿纲振纪,〔顿,谓整顿。夫,音扶。〕其明足以烛微虑远,其强足以结仁固义;大则利天下,小则利一国。是以君子丰禄以富之,隆爵以尊之;养一人而及万人者,养贤之道也。今孟尝君之养士也,不恤智愚,不择臧否,〔否,补美翻。〕盗其君之禄,以立私党,张虚誉,上以侮其君,下以蠹其民,是奸人之雄也,乌足尚哉!书曰:"受为天下逋逃主、萃渊薮。"此之谓也。

  ④孟尝君聘于楚,楚王遗之象床。登徒直送之,〔象床,以象齿为之。登徒,姓也;直,其名。遗,于季翻。〕不欲行,谓孟尝君门人公孙戌曰:"象床之直千金,苟伤之毫发,则卖妻子不足偿也。足下能使仆无行者,有先人之宝剑,愿献之。"公孙戌许诺,〔姓谱:公孙氏出于黄帝。释名曰:剑,检也,所以防检非常也。戌,音恤。偿,辰羊翻,报也。诺,奴各翻。以言许人曰诺。〕入见孟尝君曰:"小国所以皆致相印于君者,以君能振达贫穷,存亡继绝,故莫不悦君之义,慕君之廉也。今始至楚而受象床,则未至之国将何以待君哉!"孟尝君曰:"善。"遂不受。公孙戌趋去,未至中闺,〔闺,涓畦翻。宫中小门曰闺,上圆下方如圭,故谓之闺。〕孟尝君召而反之,曰:"子何足之高,志之扬也?"公孙戌以实对。孟尝君乃书门版曰:"有能扬文之名,止文之过,私得宝于外者,疾入谏!"

  臣光曰:"孟尝君可谓能用谏矣。苟其言之善也,虽怀诈谖之心,犹将用之,〔谖,许元翻。〕况尽忠无私以事其上乎!诗云:"采葑采菲,无以下体。"〔诗邶谷风之辞。毛氏传曰:葑,须也。菲,芴也。郑氏笺曰:此二菜,蔓菁与葍之类也,皆上下可食,然其根有美时,有恶时,采之者不可以根恶并弃其叶。下体,谓根茎也。陆玑草木疏曰:葑,芜菁也。郭璞曰:今菘菜。陆德明曰:江南有菘,江北有蔓菁,相似而异。尔雅曰:菲,芴;又曰:菲,息菜。郭璞曰:菲,芴,士瓜;息菜,似芜菁,华紫赤色,可食。葍,大叶,白华,根如指,色白,,可食。菲,敷尾翻。邶,蒲昧翻。芴,扶拂翻。蔓,谟官翻。葍,方六翻。〕孟尝君有焉。

  ⑤韩宣惠王欲两用公仲、公叔为政,问于缪留。〔缪,莫留翻,姓也;今靡幼翻,又音穆。〕对曰:"不可,晋用六卿而国分;齐简公用陈成子及阚止而见杀;魏用犀首、张仪而西河之外亡。〔晋六卿,智氏、范氏、中行氏、赵氏、韩氏、魏氏也。自晋文、襄以来,迭秉国政,后皆强大,卒分晋国。齐简公使止阚为政,陈成子惮之;已而陈常杀阚止,弑简公。阚,以邑为氏。苏代曰:魏相犀首,必右韩而左魏;相张仪,必右秦而左魏。盖二相外各倚与国以为重而内争权,所以魏日削也。阚,户监翻。行,户刚翻。恒,户登翻。卒,子恤翻。相,息亮翻。〕今君两用之,其多力者内树党,其寡力者藉外权。群臣有内树党以骄主,有外为交以削地,君之国危矣。"