\chapter{資治通鑑卷二}


宋 司馬光 撰

胡三省 音註

周紀二|{
	起昭陽赤奮若,
	盡上章困敦,
	凡四十八年。
	起癸丑,
	終庚子。
	}


顯王|{
	十一家諡法
	行見中外曰顯,
	受禄于天曰顯,
	百辟惟刑曰顯。
	周公蓋未有此諡,
	而周之末世諡顯王曰顯,
	意謂後世傳寫周公諡法者遺之。
	}


	元年

1 齊伐魏,取觀津。|{
	康曰齊伐魏c
	魏惠王請獻觀以和c即觀津c
	余按班志信都國有觀津縣c
	與齊相去甚遠c
	且趙地也又東郡有畔觀縣c
	水經c大河故瀆東逕五鹿之野c
	又東逕衛國故城南古卙觀也此其魏之觀津歟
	徐廣曰s觀今衛縣c
	史記正義曰s魏州觀城縣古觀國e
	國語云s觀國夏太康第五弟之所封也
	觀工喚翻}

2 趙侵齊取長城|{
	劉昭志濟北盧縣有長城史記蘇代說燕王曰齊有長城鉅防即此}


	三年

1 魏、韓會于宅陽。|{
	水經注曰s滎澤之際有沙城c世謂水城非也c
	魏冉走芒卯入北宅c即此宅陽城c
	括地志曰s宅陽故城在鄭州滎陽縣東十七里c
	}

2 秦敗魏師、韓師于洛陽。|{
	洛陽c在洛水之北c
	周公遷殷民於此c謂之成周c
	班志s屬河南郡c
	敗補邁反x
	}


	四年

1 魏伐宋。


	五年

1 秦獻公敗三晉之師于石門,|{
	水經注 馮翊 雲陽縣 有 石門山O
	括地志 在雍州三原縣西北三十二里O
	又曰堯門山俗名石門 上有路 其狀若門O
	故老云堯鑿山為門 因名之武德中於此山O
	南置石門縣貞觀中改雲陽縣O
	}
斬首六萬。
王賜以黼黻之服。|{
	黼者c刺繡c為斧形c
	黻者c刺繡c為兩已相背c
	孔頴達曰s白與黑謂之黼c黑與青謂之黻c
	黼音甫c
	黻音弗c
	}


	七年

1 魏敗韓師,趙師于澮。|{
	澮古外翻x
	括地志c澮水在 絳州翼城縣 東南二十五里c水側有皮牢城c
	}

2 秦、魏戰于少梁,|{
	班志 馮翊夏陽縣 故少梁o
	師古曰 本梁國為秦所滅 至惠文王十一年 更名夏陽o
	康曰 魏有大梁 故此稱少 以别之少o
	詩 沼翻o
	夏 戶雅翻o
	更 工衡翻o
	}
魏師敗績;
獲魏公孫痤。|{
	左傳 師大崩曰敗績
	痤才何翻}

3 衛聲公薨,子成侯速立。

4 燕桓公薨,子文公立。|{
	燕因肩翻c
	考異曰s史記c蘇秦傳c謂之燕文侯c
	按春秋c時北燕簡公已稱公c
	文公之子易王尋稱王c
	豈文公獨稱侯乎c
	今從世家c}

5 秦獻公薨,子孝公立。|{
	索隱曰:s孝公名渠梁
	}
孝公生二十一年矣。
是時河、山以東強國六,|{
	河自龍門上c口南c抵c華陰c而東流c
	秦國在河之西c
	山自鳥鼠同穴
	連延為長安南山至于泰華c
	秦國在山之西c
	韓魏趙齊楚燕六國c皆在河山以東c
	華戶化翻x
	燕因肩翻x}
淮泗之間小國十餘,|{
	南陽郡 平氏縣 東南有桐柏大復山 淮水所出東南 至淮陵入海c泗水出魯國卞縣西南至方與入沛宋魯鄒滕薛郳等國國於其間齊威王所謂泗上十二諸侯}
楚、魏與秦接界。
魏築長城自鄭濱洛以北有上郡;|{
	鄭縣周宣王母弟鄭桓公封邑班志屬京兆洛水名非伊洛之洛也水經注渭水東過華隂縣北洛水入焉洛水古漆沮之水也又有長澗水南出泰華之山側長城東而北流注于渭史記所謂魏築長城自鄭濱洛者也宋白曰今華州東南魏長城是也上郡漢屬并州隋唐之綏州延州秦漢之上郡地也濱音賓}
楚自漢中,南有巴、黔中:|{
	漢中郡漢屬益州自晉以後為梁州巴即春秋巴子之國漢為巴郡屬益州唐為巴渝渠果諸州之地黔中漢為牂牁郡之地唐為黔中節度黔渠今翻}
皆以夷翟遇秦,|{
	翟與狄同}
擯斥之,不得與中國之會盟。|{
	擯必刃翻與讀曰預}
於是孝公發憤,布德修政,欲以強秦。|{
	憤房粉翻懣也怒也朱元晦曰憤者心求通而未得之意}


	八年

孝公下令國中曰s
	昔我穆公c
	自岐雍之間c
	修德行武c
	東平晉亂c
	以河為界c
	西霸戎翟廣地千里c
	天子致伯諸侯畢賀|{
	令力正翻號令也命令也令者出於上而行於下者也岐山周太王所邑班志岐山在扶風美陽縣西雍縣屬扶風秦穆公娶晉獻公之女獻公卒晉國亂穆公納惠公惠公立而背河外之賂又閉秦糴穆公伐晉執惠公既而歸之始征晉河東置官司惠公卒子懷公立穆公納文公而晉亂平又能用由余及孟明以霸西戎天子致伯者周禮九命作伯古有九州一為王畿八州八伯各主其方之諸侯致伯者以方伯之任致之穆公也雍於用翻伯如字背蒲妹翻}
為後世開業甚光美會往者厲躁簡公出子之不寜國家内憂未遑外事三晉攻奪我先君河西地醜莫大焉|{
	為于偽翻史記秦厲共公卒子躁公立躁公卒立其弟懷公四年庶長鼂闈懷公公自殺乃立靈公靈公卒子獻公不得立立靈公之季父是為簡公公卒而惠公立惠公卒子出子立二年庶長改殺出子迎立獻公于河西河西地即魏所有西河之外史記正義曰自華州北至同州並魏河西之地躁則到翻共讀曰恭鼂古朝字長知兩翻華戶化翻}
獻公即位c鎭撫邊境~徙治櫟陽c|{
	史記秦獻公二年始治櫟陽徐廣註曰即漢萬年縣余按漢志櫟陽萬年為兩縣皆屬馮翊後漢始省併宋白曰櫟陽秦舊縣漢高祖既葬太上皇於萬年陵仍分櫟陽置萬年縣以為陵邑理櫟陽城中故櫟陽城亦名萬年城後漢省櫟陽縣入萬年縣後魏大統中分萬年置鄣丘宣武又分置廣陽縣周明帝省萬年入高陵廣陽二縣更於武安城中别置萬年縣唐武德元年又改廣陽為櫟陽元和十五年並移隸奉先縣以奉景陵櫟音藥}
且欲東伐c
復穆公之故地c
修穆公之政c
令寡人思念先君之意c
常痛於心E
賓客~羣臣有能出奇計~強秦者e
吾且尊官~與之分土E|{
	謂~裂地~以封之c使各有分土e
	分扶問翻}
於是衛公孫鞅聞是令下乃西入秦公孫鞅者衛之庶孫也好刑名之學|{
	師古曰劉向别錄云中子學好刑名刑名者循名以責實其尊君卑臣崇上抑下合於六經說者曰刑刑家名名家即太史公所論六家之二也此說非劉原父曰刑名即并學兩家術耳公孫非姓氏以其先出於衛父為衛侯則稱為公子祖為衛侯則稱為公孫鞅於兩翻}
事魏相公孫痤痤知其賢未及進會病魏惠王往問之曰公叔病如有不可諱|{
	相息亮翻痤才戈翻不可諱謂死也俗語有之人不諱死}
將奈社稷何公叔曰痤之中庶子\textcolor{blue}{衛鞅}|{
	自戰國以來c大夫之家~有中庶子c有舍人e}
年雖少c有奇才e|{
	少詩照翻}
願君舉國而聽之e
王嘿然e
公叔曰s君即不聽用c
	鞅必殺之c
	無令出境e
王許諾~而去e|{
	令力丁翻}
公叔召鞅謝曰s吾先君c而後臣e|{
	先後皆去聲}
故先為君謀c後以告子e|{
	此先後皆如字為于偽翻}
子必速行矣e
鞅曰s君不能用子之言~任臣c
又安能用子之言~殺臣乎e
卒不去e|{
	卒子恤翻}
王出~謂左右曰s
公叔病甚c悲乎e
欲令寡人~以國聽衛鞅也c
旣又勸寡人~殺之e
豈不悖哉E|{
	悖蒲内翻x}
衛鞅~旣至秦e
因嬖臣景監~以求見孝公c|{
	嬖博計翻x又 卑義翻x
	史記正義s
	監~甲暫翻c
	康曰景姓楚之族c
	監古銜切c
	非c}
說以~富國強兵之術e
公大悦e
與議國事e|{
	說式芮翻x
	}


	十年

衛鞅欲變法c秦人不悦c
衛鞅言於秦孝公曰\textcolor{red}{c夫民不可與慮始c而可與樂成c}|{
	夫音扶樂音洛}
論至德者c不和於俗c成大功者c不謀於衆c是以聖人c苟可以強國不灋其故|{
	索隱曰:言救弊為政之術所為苟可以強國則不必須要法於故事也}
甘龍c曰不然|{
	索隱曰:甘姓龍名甘姓出春秋時甘昭公子帶之後姓譜又曰甘姓商甘盤之後}
因民而教者c不勞而成功c緣法而治者c吏習而民安之|{
	治直吏翻}
衛鞅曰s常人安於故俗c學者溺於所聞c|{
	溺奴歷翻}
以此兩者居c官守法可也c
非所與論於灋之外也c
智者作法c
愚者制焉c
賢者更禮c
不肖者拘焉c|{
	更工衡翻}
公曰s善c
以衛鞅為左庶長c|{
	劉邵爵制曰s春秋傳 有庶長鮑商君為政備其法品為十八級合關内侯列侯凡二十等其制因古義古者天子寄軍政於六卿居則以田警則以戰所謂入使治之出使長之素信者與衆相得也故啓伐有扈乃召六卿大夫之在軍為將者也及周之六卿亦以居軍在國也則以比長閭胥族師黨州長鄉大夫為稱其在軍也則以司馬將軍卒伍為號所以異在國之名也秦依古制其在軍賜爵為等級其師人皆更卒也有功賜爵則在軍吏之例自一等以上至不更四等皆士也大夫以上至五大夫五等比大夫也九等依九命之義也曰左庶長至大庶長比九卿也關内侯者依古圻内子男之義也秦都山西以關内為王畿故曰關内侯也列侯者依古列國諸侯之義也然則卿大夫士下之品皆倣古比朝之制而異其名亦所以殊軍國也古者以車戰兵車一乘步卒七十二人分翼左右車大夫在左御者處中勇士為右凡七十五人一爵曰公士者步卒之有爵為公士者也二爵曰上造造成也古者成士升於司徒曰造士雖依此名皆步卒也三爵曰簪褭御駟馬者要褭者古之名馬也駕駟馬其形似簪故云簪褭也四爵曰不更不更者為車右不復與凡更卒同也五爵曰大夫大夫在車左者也六爵為官大夫七爵為公大夫八爵為公乘九爵為五大夫皆軍吏也吏民爵不得過公乘者得貰與子若同產然則公乘者軍吏之爵最高者也雖非臨戰得公乘車故曰公乘也十爵為左庶長十一爵為右庶長十二爵為左更十三爵為中更十四爵為右更十五爵為少上造十六爵為大上造十七爵為駟車庶長十八爵為大庶長十九爵為關内侯二十爵為列侯自左庶長至大庶長皆卿大夫皆軍將也所將皆庶人更卒也故以庶更為名大庶長即大將軍也左右庶長即左右偏裨將軍也長知丈翻}
卒定變法之令c
令民為什伍c而相收司連坐c|{
	索隱曰:收司謂相糾發也一家有罪則九家連舉發若不糾舉則九家連坐師古曰五人為伍二伍為什康曰司猶管也為什伍之法使之相司相管秦有見知連坐法余謂連坐者一家有罪什伍皆相連坐罪也見知乃漢法卒子恤翻}
告姦者 與斬敵首同賞c|{
	索隱曰:謂告姦一人則得爵一級故曰與斬敵首同賞}
不告姦者 與降敵同罰c|{
	索隱曰:s律降敵者c誅c
	其身沒其家c
	今匿姦者 言常與之同罰c
	降c戶江翻}
有軍功者c各以率受上爵c|{
	率音律}
為私鬭者c各以輕重c被刑大小僇力本業耕織致粟帛多者復其身|{
	僇 力竹翻c古戮字c說文并力也c
	字林音遼c
	復 方目翻c
	漢法除其賦税役c皆謂之復c}
事末利及c怠而貧者c舉以為收孥|{
	索隱曰:s末利謂工商糾舉而收録其妻子沒為奴婢c
	秦法c一人有罪c收其室家c
	至漢文帝 元年c始除收孥相坐法c
	孥音奴c
	}
宗室非有軍功論|{
	論c議也c
	有戰功之可論也c
	論盧困翻c
	康盧昆切c
	}
不得為屬籍c|{
	屬籍c宗屬之籍也c
	孔頴達曰s漢之同宗有屬籍c則周家繫之以姓是也c
	周禮c小史之官掌定帝繫世本知世代昭穆c
	屬殊玉翻}
明尊卑爵秩等級各以差次|{
	白虎通曰s爵者c盡也c所以盡人才也c
	毛晃曰s大夫以上預燕饗然後賜爵秩以章有德秩職也官也積也次也常也序也}
名田宅臣妾衣服有功者顯榮無功者雖富無所芬華令既具未布恐民之不信乃立三丈之木於國都市南門募民有能徙置北門者予十金|{
	予讀曰與}
民怪之莫敢徙復曰能徙者予五十金|{
	復扶又翻}
有一人徙之輒予五十金|{
	李云金方寸重一斤為一金程大昌演繁露曰二十兩為一金亦為一鎰}
乃下令令行朞年秦民之國都|{
	之往也如也}
言新令之不便者以千數於是太子犯法衛鞅曰法之不行自上犯之太子君嗣也|{
	嗣祥吏翻}
不可施刑刑其傅公子䖍黥其師公孫賈|{
	墨湼其面曰黥黥音渠京翻為後秦殺商君鞅張本}
明日秦人皆趨令|{
	索隱曰:趨者向也附也音七喻翻}
行之十年秦國道不拾遺山無盜賊民勇於公戰怯於私鬭鄉邑大治|{
	自是年至三十一年商鞅死蓋鞅之行其法而致效在十年之間又十年而致禍治直吏翻}
秦民初言令不便者有來言令便衛鞅曰此皆亂法之民也盡遷之於邊其後民莫敢議令


臣光曰S夫信者c人君之大寶也c|{
	夫音扶}
國保於民c
民保於信c
非信 無以使民c
非民 無以守國c
是故 古之王者不欺四海c|{
	孔穎達曰自今本昔曰古}
霸者不欺四鄰c
善為國者 不欺其民c
善為家者 不欺其親c
不善者 反之欺其鄰c
國欺其百姓c
甚者欺其兄弟c
欺其父子c
上不信下c
下不信上c
上下離心c
以至於敗c
所利不能藥其所傷c
所獲不能補其所亡c
豈不哀哉E
昔 齊桓公 不背曹沫之盟c
晉文公 不貪伐原之利c
魏文侯 不棄虞人之期c|{
	姓譜曹本自顓頊之玄孫陸終之子六安是為曹姓周武王封曹狹于邾故邾曹姓也又云曹叔振鐸之後武王母弟也後以為氏史記齊桓公伐魯魯莊公請平桓公許之與盟於柯將盟曹沫以匕首劫桓公於壇上請反魯之侵地桓公許之曹沬去匕首而就臣位桓公後悔欲殺曹沬管仲不可遂反所侵地於魯諸侯聞之皆信齊而欲附焉左傳晉文公圍原命三日之糧原不降命去之諜出曰原將降矣軍吏曰請待之公曰得原失信所亡滋多退一舍而原降魏文侯事見上卷威烈王二十三年背蒲妺反索隱曰:沬音亡葛翻左傳穀梁並作曹劌然則沬宜音劌沬劌聲相近而字異耳}
秦孝公 不廢徙木之賞c
此四君者 道非粹白c
而 商君尤稱刻薄c
又處戰攻之世c
天下趨於詐力c
猶且不敢忘信c
以畜其民c|{
	處昌呂翻趨七喻翻畜許六翻養也}
况為 四海治平 之政者哉E|{
	治 直吏翻x
	}


韓懿侯薨c子昭侯立e|{
	諡法s
	昭德有勞曰昭c
	聖聞周達曰昭c
	}


	十一年

秦敗韓師于西山|{
	自宜陽熊耳東連嵩高南至魯陽皆韓之西山敗補邁翻}

	十二年

魏韓會于鄗|{
	班志鄗縣屬中山郡此時為趙地後漢改為高邑唐為趙州柏鄉縣贊皇縣地鄗呼各翻}


十三年燕趙會于阿|{
	燕因肩翻}
趙齊宋會于平陸


十四年

齊威王 魏惠王 會田于郊e
惠王曰s齊亦有寶乎Q
威王曰s無有e
惠王曰S寡人 國雖小c尚有徑寸之珠  照 車前後各十二乘者十枚e|{
	乘 繩證翻x}
豈以齊大國而無寶乎Q
威王曰s
寡人之所以為寶者c
與王異e
吾臣有檀子者c|{
	姓譜云齊公族有食采於瑕丘檀城因以為氏}
使守南城c|{
	城在齊之南境故曰南城}
則楚人不敢為寇e
泗上十二諸侯 皆來朝e|{
	朝直遥翻}
吾臣有盼子者c
使守高唐c
則趙人不敢東漁于河e|{
	盼匹莧翻又披班翻按丁度集韻盼與盻同盼子齊之同姓即田盼也班志高唐縣屬平原郡杜預曰祝阿西北有高唐城宋白曰齊州章丘縣古高唐春秋戰國之時為齊邑故城在廢禹城縣西四十里唐之禹城漢祝阿也}
吾吏 有黔夫者c
使守徐州c|{
	姓譜齊有黔敖則黔亦姓也音其淹翻司馬彪曰魯國薛縣六國時曰徐州徐音舒丁度集韻徐作俆音同}
則燕人 祭北門 趙人祭西門e|{
	燕在齊之北趙在齊之西賈逵曰燕趙畏齊故祭以求福燕因肩翻}
徙而從者七千餘家e
吾臣有種首者c
使備盗賊c
則道不拾遺e|{
	種章勇翻}
此 四臣者 將照千里c
豈特 十二乘哉E
惠王有慙色

秦孝公a魏惠王會于杜平e|{
	班志京兆有杜陵縣故周之杜伯國也史記灌嬰傳嬰以昌平侯食邑于杜平鄉正義曰杜平在唐之同州澄城縣界魏世家作杜平}

魯共公薨c子康公毛立e|{
	共讀曰恭}


十五年

秦敗魏師于元里|{
	史記正義曰元里亦在同州澄城縣界敗補邁翻}
斬首七千級|{
	秦法戰而斬敵人一首者賜爵一級因謂之級}
取少梁|{
	少詩照翻}
魏惠王伐趙圍邯鄲楚王使景舍救趙|{
	邯音寒鄲音丹昭屈景皆楚之同姓楚強族也屈九勿翻}


十六年

齊威王 使田忌救趙E
初r孫臏與龎涓俱學兵法e|{
	姓譜周文王子康叔封於衛至武公子惠孫曾耳為衛上卿因氏焉後有孫武孫臏俱善兵趙明誠金石錄有漢安平相孫根碑云先出自有殷之裔子武王定周封比干墓胤裔分析定曰孫焉姓譜又曰龎姓畢公高之後支庶封于龎因氏焉臏頻忍翻刖刑也去膝蓋骨鄭玄曰周改臏作刖刖斷足也書傳云决關梁踰城郭而畧盗者其刑臏孫臏蓋以刖足故呼為臏說文臏膝耑也類篇毘賓切龎薄江翻涓古玄翻}
龎涓仕魏 為將軍e|{
	將軍之官自周以來有之}
自以能不及孫臏c
乃召之至 則
以法斷其兩足c而黥之EE|{
	斷丁管翻}
欲使終身廢棄c
齊使者至魏c
孫臏以刑徒 陰見c
說齊使者e|{
	齊使疏吏翻說式芮翻}
齊使者竊載與之齊e|{
	之往也}
田忌善而 客待之c
進於威王c
威王問兵法c遂以為師e
於是 威王謀救趙c
以孫臏為將c
辭以 刑餘之人 不可c
乃以田忌為將c
而孫子為師c
居輜車中 坐為計謀e|{
	將即亮翻字林曰軿車有衣蔽無後轅者謂之輜釋名曰有邸曰輜無邸曰輧傅子曰周曰輜車即輦也康曰輧車也軍行所以載輜重輜楚持翻輧蒲眠翻重直用翻}
田忌欲引兵之趙c
孫子曰 夫解雜亂 紛糾者 不控拳e|{
	索隱曰:謂事之雜亂紛糾也解雜亂紛糾者當善以手解之不可控拳而擊之余謂雜亂紛糾者謂人鬭者耳非事也康曰拳與絭同絭者攘臂繩也余謂當從索隱說康說非夫音扶}
救鬭者 不搏撠e|{
	索隱曰:搏撠音博戟謂救鬬者當善撝解之毋以手相搏撠則其怒益熾矣按撠謂以手持撠以刺人也余謂索隱之說善矣但以撠為持撠以刺人則非也撠如漢書撠太后掖之撠師古曰撠謂拘持之也毛晃曰索隱曰:搏拘持曰撠}
批亢擣虛形格勢禁c
則自為解耳|{
	索隱曰:批白結翻亢苦浪翻按批者相排批也音白滅翻亢言敵人相亢拒也擣者擊也衝也虚空也謂前人相亢必須批之彼兵若虚則衝擣之若批其相亢擊擣彼虛則是其形相格其勢自禁止則彼自為解也康曰亢極也高也擣築也乘其高亢而批之乘其虛而擣之則其勢自解批亢擣虛所謂形格勢禁也余謂索隱之說為長蓋鬭者方相亢拒則排批之使解虛者兩敵拒鬬力所不及之處擣之則雖欲鬭其勢不能不解此易見也格各額翻格正也又擊也鬭也吳都賦笑而被格本音如字協韻音閣與彿同音父沸翻}
今梁趙相攻c
輕兵鋭卒必竭於外c
老弱疲於内c
子不若引兵疾走魏都c
據其街路 衝其方虛c|{
	康曰虚音墟余謂虛如字衝其方虚即上所謂擣虚也索隱之說義亦如此走則凑翻}
彼必釋趙以自救c
是我一舉解趙之圍c
而收弊 於魏也e
田忌從之e
十月c邯鄲降魏e|{
	邯音寒戰音丹降戶江翻}
魏師還與齊c
戰于桂陵c
魏師大敗e|{
	還從宣翻又音如字水經注濮渠與酸水會水東逕滑臺城南又東南逕瓦亭南又東南會于濮濮渠之側有漆城桂城亦曰桂陵即田忌敗魏師處史記正義曰桂陵在曹州乘氏縣東南二十一里濮博木翻}

韓伐東周c
取陵觀a廪丘e|{
	周室衰微c
	戰國之時c
	僅有七邑c
	漢時~之河南c 洛陽c 穀成c 平陰c 偃師c 鞏緱氏c是也e
	晉志曰s
	周考王封周桓公a孫惠公a於鞏c號東周e
	故戰國 有東西周 芒山首山其界也e
	陵觀 廪丘 皆當時邑聚之名c
	史無所考c
	廪丘 史記作 邢丘c
	觀古玩翻x
	}
楚昭奚恤為相c
江乙 言於 楚王 曰 
人有愛其狗者c
狗嘗溺井e|{
	昭屈景楚之強族所謂三閭者也太史公曰嬴姓分封為江氏相息亮翻溺奴弔翻}
其鄰人 見欲入言之
狗當門 而噬之
今昭奚恤常惡臣之見 亦猶是也|{
	噬時制翻見謂見楚王也惡烏路翻}
且人有好揚人之善者王曰此君子也近之好揚人之惡者王曰此小人也遠之|{
	好呼到翻近者附近之近去聲遠于願翻推而遠之推吐雷翻}
然則且有子弑其父臣弑其主者而王終已不知也|{
	已音紀終已猶言終身也}
何者以王好聞人之美而惡聞人之惡也王曰善寡人願兩聞之|{
	江乙欲毁昭奚恤故先設是言}


十七年
秦大良造伐魏|{
	索隱曰:s大良造即大上造c
	余謂 大良造 大上造 之良者也c
	按史記秦紀孝公十年c
	衛鞅為大良造將兵c
	圍魏安邑降之c
	又按六國年表c
	秦孝公之十年c顯王之十七年c
	所謂大良造伐魏c
	即衛鞅將兵也c
	是時c魏都安邑 其兵猶強c龎涓 太子申 公子卭未敗
	安邑不應遽降於秦c
	至顯王二十九年c
	卭軍既敗魏c
	獻河西之地於秦c
	始去安邑c
	徙都大梁c
	史記x六國表不書徙大梁c
	而世家書 之 魏世家 於是年不書安邑降秦c
	而秦紀孝公十年c書之通鑑從魏世家於顯王c
	二十九年書魏去安邑c徙大梁c
	而是年不書魏安邑降秦c
	蓋亦疑而除去之c
	但大良造之下c當有衛鞅二字c意謂傳寫通鑑者c逸之e
	}
諸侯圍魏襄陵|{
	史記正義曰襄陵故城在兖州鄒縣余按魏境時不至於鄒班志河東有襄陵縣師古曰晉襄公之陵因以名縣括地志襄陵在晉州臨汾縣東南三十五里宋白曰後魏為禽昌縣隋大業二年改為襄陵縣以趙襄子晉襄公俱陵於是邑也}


十八年秦衛鞅圍魏固陽降之|{
	魏有上郡北至固陽漢五原郡稒陽縣是也括地志固陽在銀州銀城縣界按魏築長城自鄭濱洛北抵銀州至勝州固陽縣為塞也固陽有連山東至黄河西南至夏會等州降戶江翻夏戶雅翻}
魏人歸趙邯鄲|{
	邯音寒鄲音丹}
與趙盟漳水上|{
	記曲禮曰蒞牲曰盟盟者殺牲歃血誓於神也天下太平之時諸侯不得擅相與盟惟天子巡狩至方岳之下會畢乃與諸侯相盟同好惡奨王室以昭事神訓民事君凡國有疑則盟詛其不信者至於五霸有事而會不協而盟盟之為法先鑿地為方坎殺牲於坎上割牲左耳盛以珠盤又取血盛以王敦用血為盟書成乃歃血而讀書左傳云坎用牲加書是也班志濁漳水出上黨長子縣鹿谷山東至鄴入清漳水經曰出長子縣發鳩山東至武安縣與清漳會謂之交漳口又東過鄴縣列人又東北過鉅鹿信都謂之衡漳又東北過平舒縣南而東入海漳諸良翻}
韓昭侯以申不害為相|{
	諡法昭德有勞曰昭聖聞周達曰昭姓譜四岳之後封于申周有申伯鄭有大夫申侯齊有申鮮虞相息亮翻}
申不害者鄭之賤臣也學黄老刑名以干昭侯|{
	黄老黄帝老子之書}
昭侯用為相内修政教外應諸侯十五年終申子之身國治兵強|{
	治直吏翻}
申子嘗請仕其從兄|{
	從才用翻羣從之從同}
昭侯不許申子有怨色昭侯曰所為學於子者欲以治國也|{
	為于偽翻治直之翻}
今將聽子之謁而廢子之術乎已其行子之術而廢子之請乎子嘗教寡人修功勞視次第今有所私求我將奚聽乎申子乃辟舍請罪曰君眞其人也|{
	辟讀曰避}
昭侯有弊袴命藏之|{
	袴苦故翻脛衣也}
侍者曰君亦不仁者矣不賜左右而藏之昭侯曰吾聞明主愛一嚬一咲嚬有為嚬咲有為咲今袴豈特嚬咲哉吾必待有功者|{
	言袴雖弊其直猶重固不止於嚬咲也然人主之嚬咲所關甚大昭侯姑以此為言耳為于偽翻嚬與顰同愁蹙之貌咲古笑字}
十九年秦商鞅築冀闕宫庭於咸陽|{
	索隱曰:冀闕即魏闕也爾雅觀謂之闕郭璞曰宫門雙闕也釋名闕在門兩旁中間闕然為道也三輔黄圖曰人臣至此必思其所闕少爾雅宫謂之室郭璞曰宫謂圍繞之也說文曰庭朝中也蒼頡篇曰庭直也風俗通曰庭正也言縣庭郡庭朝庭皆取平均正直也三輔黄圖曰山南為陽水北為陽山水皆在陽故曰咸陽漢高帝更名新城武帝更名渭城屬右扶風括地志咸陽故城在雍州咸陽縣東十五里在長安城北四十五里宋白曰咸陽縣本周王季所都秦又都之三秦記秦都在九嵕山南渭水北山水俱陽故名咸陽二十九年秦始封衛鞅于商號商君史以後所封書之}
徙都之令民父子兄弟同室内息者為禁|{
	息止也秦俗父子兄弟同室居止商君始更制禁同室内息者堯教民以人倫教之有序有别秦用西戎之俗至於男女無别長幼無序商君今為之禁古道也烏可例言之白虎通曰父矩也以法度教子也子孳也孳孳無已也兄况也况父法也弟悌也心順行篤也}
并諸小鄉聚集為一縣縣置令丞凡三十一縣廢井田開阡陌|{
	周禮六鄉鄉萬二千五百家又百家之内曰鄉五鄙為縣縣二千五百家此六遂之縣也四甸為縣此州里之縣也周制天子地方千里分為百縣縣有四郡左傳趙鞅所謂上大夫受縣下大夫受郡者也秦并天下置三十六郡以監天下之縣自是始統于郡矣釋名曰縣懸也懸于郡也漢書音義所謂大曰鄉小曰聚亦秦制也廣雅曰聚聚居也音慈諭翻縣令丞之官始此令音力正翻令命也告也律也法也長也使為一縣之長以行誥命法律也丞翊也副貳也成周之制田方里為井井九百畝八家各耕百畝其中百畝八十畝為公田二十畝為廬舍史記正義曰南北曰阡東西曰陌劉伯莊曰開田界道使不相干長知兩翻}
平斗桶權衡丈尺|{
	桶索隱音統非也當作甬音勇斛也沈括曰予受詔考鍾律及鑄渾儀求秦漢以來度量斗升計六斗當今之一斗七升九合秤三斤當今十三兩一斤當今四兩三分兩之一一兩當今六銖半為升中方古尺二寸五分十分分之三今尺一寸八分百分分之四十五強}
秦魏遇于彤|{
	彤周彤伯所封之國國于王畿之内史記六國年表商君反死彤地則其地當}


|{
	在漢京兆鄭縣界彤徒冬翻}
趙成侯薨公子緤與太子爭立緤敗奔韓|{
	緤私列翻趙成侯敬侯之子名種太子肅侯語也}


二十一年

秦商鞅更為賦税法行之|{
	井田既廢則周什一之法不復用蓋計畝而為賦税之法更工衡翻}


二十二年

趙公子范襲邯鄲不勝而死|{
	邯音寒鄲音丹}


二十三年

齊殺其大夫牟 魯康公薨子景公偃立衛更貶號曰侯服屬三晉|{
	周成王封康叔為衛侯其後世進爵為公今寖以弱小貶號曰侯貶悲檢翻}


二十五年

\textcolor{red}{諸侯會于京師}|{
	時天下宗周以洛陽為京師京大也師衆也京師衆大之名也}


二十六年

王致伯于秦|{
	伯如字c
	伯者周 二伯 九伯 之任c
	}
諸侯皆賀秦e
秦孝公使 公子少官帥師會諸侯 于逢澤  以朝王e|{
	左傳逢澤有介麋焉宋地也杜預註曰地理志言逢澤在滎陽開封縣東北遠疑非括地志曰逢澤在汴州浚儀縣東南二十四里帥音率}


二十八年

魏龎涓伐韓韓請救于齊齊威王召大臣而謀曰蚤救孰與晚救成侯曰不如勿救|{
	鄒忌為齊相封成侯}
田忌曰弗救則韓且折而入於魏|{
	折而設翻}
不如蚤救之孫臏曰夫韓魏之兵未弊而救之|{
	臏頻忍翻又毘賓翻夫音扶}
是吾代韓受魏之兵顧反聽命于韓也且魏有破國之志韓見亡必東面而愬於齊矣|{
	見亡言見有亡國之勢也愬告愬也}
吾因深結韓之親而晚承魏之弊則可受重利而得尊名也王曰善乃隂許韓使而遣之|{
	陰闇也使疏吏翻}
韓因恃齊五戰不勝而東委國于齊齊因起兵使田忌田嬰田盼將之|{
	盼與盻同音匹莧翻將即亮翻下同又音如字領也}
孫子為師以救韓直走魏都|{
	走音奏}
龎涓聞之去韓而歸|{
	龎薄江翻涓工玄翻}
魏人大發兵以太子申為將以禦齊師孫子謂田忌曰彼三晉之兵素悍勇而輕齊|{
	將即亮翻悍下罕翻又音汗}
齊號為怯善戰者因其勢而利導之兵法百里而趨利者蹷上將五十里而趨利者軍半至|{
	此孫武子兵法也趨七喻翻魏武帝曰蹷其月翻蹷猶挫也劉氏曰蹷猶斃也半至謂軍趨利前後不相屬半至半不至也屬陟玉翻}
乃使齊軍入魏地為十萬竈明日為五萬竈又明日為二萬竈龎涓行三日大喜曰我固知齊軍怯入吾地三日士卒亡者過半矣|{
	過工禾翻}
乃棄其步軍|{
	句斷龎薄江翻涓圭淵翻}
與其輕鋭倍日并行逐之|{
	并行兼程而行也倍日一日行兩日之程亦兼程也}
孫子度其行暮當至馬陵|{
	司馬彪志魏郡元城縣註云左傳成七年會馬陵杜預志在縣東南龎涓死處虞喜志林馬陵在濮州鄄城東北六十里澗谷深可以置伏度徒洛翻鄄吉椽翻}
馬陵道陿而旁多阻隘可伏兵|{
	陿與狹同隘烏懈翻}
乃斫大樹白而書之曰龎涓死此樹下於是令齊師善射者萬弩夾道而伏期曰暮見火舉而俱發龎涓果夜到斫木下見白書以火燭之讀未畢萬弩俱發魏師大亂相失龎涓自知智窮兵敗乃自剄曰遂成豎子之名|{
	龎薄江翻涓工玄翻剄古頂翻斷首也康古定切非豎殊遇翻說文豎使布短衣}
齊因乘勝大破魏師虜太子申成侯鄒忌惡田忌|{
	鄒以國為氏惡烏路翻}
使人操十金卜於市|{
	操七刀翻}
曰我田忌之人也我為將三戰三勝欲行大事可乎卜者出因使人執之田忌不能自明率其徒攻臨淄|{
	臨淄齊國都也城臨淄水因以為名班志臨淄屬齊國臣瓚曰臨淄即營丘太公營之淄莊持翻}
求成侯不克出奔楚|{
	為下齊復田忌張本}


二十九年

衛鞅言於秦孝公曰s秦之與魏譬若人有腹心之疾非魏幷秦秦即并魏何者魏居嶺阨之西|{
	索隱曰:蓋安邑以東山巔險阨之地今蒲州中條以東連汾晉之險嶝皆其地也阨於革翻}
都安邑與秦界河|{
	秦魏以河為界也}
而獨擅山東之利|{
	擅市戰翻}
利則西侵秦病則東收地今以君之賢聖國賴以盛而魏往年大破於齊諸侯畔之可因此時伐魏魏不支秦必東徙然後秦據河山之固東鄉以制諸侯|{
	鄉讀曰嚮}
此帝王之業也公從之使衛鞅將兵伐魏魏使公子卭將而禦之軍既相距衛鞅遺公子卭書曰吾始與公子驩今俱為兩國將|{
	將即亮翻遺于季翻}
不忍相攻可與公子面相見盟樂飲而罷兵以安秦魏之民|{
	樂音洛}
公子卭以為然乃相與會盟已飲|{
	盟已而飲也}
而衛鞅伏甲士襲虜公子卭因攻魏師大破之魏惠王恐使使獻河西之地於秦以和|{
	使使下疏吏翻}
因去安邑徙都大梁|{
	班志陳留郡浚儀縣故大梁杜佑曰汴州城西古城戰國時魏惠王所築}
乃嘆曰吾恨不用公叔之言|{
	公叔言見上八年}
秦封衛鞅商於十五邑|{
	班志弘農郡商縣商君邑裴駰曰商於之地在今順陽郡南鄉丹水二縣有商城在於中故謂之商於史記正義曰丹水及商皆屬弘農今言順陽是魏晉始分置順陽郡商及丹水皆屬之也水經註丹水逕南鄉丹水二縣之間歷於中之北所謂商於者也杜佑曰今鄧州内鄉縣東七里有於村蓋秦所謂商州商洛縣古商邑卨所封也漢為商縣於如字}
號曰商君 齊趙伐魏 楚宣王薨子威王商立


三十一年

秦孝公薨子c惠文王立e

公子䖍之徒 告  商君欲反c發吏捕之e
\textcolor{red}{商君亡之魏}e|{
	之如也往也}
\textcolor{red}{魏人不受}e
復内之秦e|{
	内讀曰納怨其挾詐以破魏師故不受}
商君 乃與其徒之商於發兵北擊鄭e|{
	之往也如也鄭京兆之鄭縣也周宣王弟鄭桓公采邑唐屬華州宋白續通典曰鄭縣古城在華州郡城北}
秦人攻商君c殺之c車裂以狥c盡滅其家e|{
	車裂c古之轘刑c
	轘戶串翻c}

初R商君相秦c用法嚴酷c嘗臨渭論囚c渭水盡赤e|{
	相息亮翻水經渭水出隴西首陽縣鳥鼠山東流至秦都咸陽南商君蓋臨此以論囚
	\fbox{决罪曰論}
	論盧困翻}
為相十年c人多怨之e|{
	按顯王十七年秦以商鞅為大良造十九年商鞅徙秦都咸陽廢井田開阡陌平權量二十一年更賦税法為相當在是年至今年十年矣}
\textcolor{blue}{趙良}見商君e
商君問曰子觀我治秦c|{
	治直之翻}
孰與五羖大夫賢c|{
	百里奚自賣以五羖羊之皮為人養牛秦穆公舉以為相秦人謂之五羖大夫羖壮羊也羖音古}
趙良曰千人之諾諾c
不如一士之諤諤c|{
	引趙簡子之言諾應聲也諤謇直也}
僕請終日正言而無誅可乎c
商君曰諾c
趙良曰c
五羖大夫c荆之鄙人也c
穆公舉之牛口之下c|{
	孟子百里奚虞人也以食牛干秦繆公今曰荆之鄙人按史記晉滅虞執百里傒為秦繆夫人媵百里傒亡秦走宛楚鄙人執之繆公以五羖羊皮贖之以為上大夫傒讀與奚同繆讀與穆同媵以證翻宛於元翻}
而加之百姓之上c
秦國莫敢望焉c
相秦六七年 而東伐鄭c|{
	謂左傳僖三十年與晉圍鄭也相息亮翻}
三置晉君c一救荆禍c|{
	三置晉君c謂立惠公 懷公 文公也c
	索隱曰:s十二諸侯年表穆公二十八年會晉伐楚朝周此云救荆未詳余按左傳晉既敗楚于城濮又敗秦於殽穆公使鬭克歸楚求成所謂救荆禍蓋指此也秦諱楚故其國記率謂楚為荆太史公取秦記為史記通鑑又因史記而成書故亦以楚為荆}
其為相也c勞不坐乘c|{
	古者車立乘惟安車則坐乘耳}
暑不張蓋c|{
	周禮輪人為蓋蓋所以覆冒車上也}
行於國中c
不從車乘c|{
	乘繩證翻}
不操干戈c|{
	操五刀翻}
五羖大夫死c
秦國男女流涕c童子不歌謡c舂者不相杵c|{
	記鄰有喪舂不相註云相杵者以音聲相勸相息亮翻}
今君之見也c
因嬖人景監以為主c|{
	事見上八年嬖卑義翻又博計翻監甲暫翻}
其從政也c
凌轢公族c殘傷百姓c|{
	轢郎擊翻車踐曰轢}
公子䖍~杜門不出已八年矣c
君又殺祝懽c而黥公孫賈c|{
	祝姓也古有巫史祝之官其子孫因以為姓或曰武王封黄帝之後于祝其子孫因氏焉黥其京翻}
詩曰c得人者興c失人者崩e|{
	逸詩也}
此數者c非所以得人也c
君之出也c
後車載甲c
多力而駢脅者 為驂乘c|{
	杜預曰s駢脅c合幹也c
	駢 步田翻x
	乘 繩證翻x
	驂 讀曰參x}
持矛而操闟a戟者c
旁車而趨c|{
	薛綜曰闟之為言函也取四戟函車邊此蓋令力士旁車而趨有急則操翕戟以禦之也後漢志有闟戟車晉志闟戟車長戟邪偃在後唐韻戟名曰闟音所及翻史記正義曰顧野王云矛鋋也方言云矛吳楚江淮之間謂之鋋釋名曰戟格也車旁有枝格旁車之旁音步浪翻}
此一物不具c君固不出e
書曰恃德者昌恃力者亡|{
	逸書也}
此數者c非恃德也c
君之 危若朝露|{
	朝露易晞言不久也}
而尚
貪商於之富c寵秦國之政c|{
	言以專秦國之政為寵也}
畜百姓之怨c|{
	畜 讀曰蓄x}
秦王一旦捐賓客 而不立朝c|{
	朝直遥翻}
秦國之所以收君者c豈其微哉c|{
	微少也趙良言豈少蓋謂太子與其師傅將挾怨而殺之也}
商君弗從c居五月而難作c|{
	難乃旦翻史言商君尚刑愎諫之禍速}


三十二年

韓申不害卒c|{
	卒子恤翻}


三十三年

宋太丘社亡|{
	班志沛郡有太丘縣又志曰宋太丘社亡周鼎淪沒於泗水中爾雅右陵太丘釋云謂丘之西有大阜者為太丘也宋太丘社亡蓋依丘作社於時亡去咎證也}
鄒人孟軻見魏惠王|{
	鄒春秋之邾國也班志鄒縣屬魯國宋白曰淄州鄒平縣漢舊縣}
王曰叟|{
	叟者尊老之稱稱尺證翻}
不遠千里而來亦有以利吾國乎孟子曰君何必曰利仁義而已矣|{
	不遠千里言不以千里為遠也}
君曰何以利吾國大夫曰何以利吾家士庶人曰何以利吾身上下交征利而國危矣未有仁而遺其親者也未有義而後其君者也|{
	後戶豆翻}
王曰善|{
	通鑑于此段前後書王因孟子之文也中間叙孟子答魏王之言獨改王曰君不與魏之稱王也}
初孟子師子思嘗問牧民之道何先子思曰先利之孟子曰君子所以教民者亦仁義而已矣何必利子思曰仁義固所以利之也上不仁則下不得其所上不義則下樂為詐也|{
	樂音洛}
此為不利大矣故易曰利者義之和也|{
	易乾卦文言}
又曰利用安身以崇德也|{
	易大傳之辭}
此皆利之大者也


臣光曰子思孟子之言一也夫唯仁者為知仁義之為利不仁者不知也|{
	夫音扶}
故孟子對梁王直以仁義而不及利者所與言之人異故也

三十四年

秦伐韓拔宜陽

三十五年

齊王魏王會于徐州以相王|{
	史記正義曰竹書紀年云梁惠王三十年下邳遷于薛改曰徐州續漢志曰魯國薛縣六國時曰徐州與竹書合徐音舒相王者相立為王也}
韓昭侯作高門屈宜臼曰君必不出此門|{
	許愼曰屈宜臼楚大}


|{
	大時在韓屈九切翻}
何也不時吾所謂時者非時日也夫人固有利不利時|{
	夫音扶}
往者君嘗利矣不作高門前年秦抜宜陽今年旱君不以此時恤民之急而顧益奢此所謂時詘舉贏者也|{
	詘區勿翻徐廣曰時衰耗而作奢侈言國家多難而勢詘此時宜恤民之急而舉事反若有贏餘者失其所以為國之道矣時詘舉贏蓋古語也贏怡成翻}
故曰不時 越王無強伐齊|{
	越王句踐之後自句踐至無強凡六世句音鉤踐音慈淺翻}
齊王使人說之以伐齊不如伐楚之利|{
	說式芮翻}
越王遂伐楚楚人大敗之|{
	敗補邁翻}
乘勝盡取吳故地東至于浙江越以此散諸公族爭立或為王或為君濱于海上|{
	吳之故地漢會稽九江丹陽豫章廬江廣陵臨淮等郡是也越初都會稽其境北至于禦兒不能全有漢會稽一郡地及其滅吳始幷有吳地今楚取吳地至于浙江則禦兒亦入于楚矣浙江有三源發于太末者謂之穀水今之衢港是也發于烏傷者水經謂之吳寧溪今之婺港是也發下黝縣者班志謂之漸江水今之徽港是也三水合為浙江東至錢唐入海浙折也言水屈折于羣山之間也釋名曰江共也小水流入其中所公共也國于海上者漢之甌越閩越駱越其後也浙之列翻濱音賓會古外翻太末之太孟康音闥港古項翻婺亡遇翻黝音伊閩眉中翻駱音洛}
朝服于楚|{
	朝直遥翻}

	三十六年
	
楚王伐齊圍徐州|{
	徐音舒}
韓高門成昭侯薨|{
	卒如屈宜臼之言卒子恤翻}
子宣惠王立|{
	宣惠複諡也}
初洛陽人蘇秦說秦王以兼天下之術|{
	說式芮翻姓譜蘇己姓顓頊裔孫吳囘生陸終陸終生昆吾封於蘇至周蘇公}
秦王不用其言蘇秦乃去說燕文公曰燕之所以不犯寇被甲兵者以趙之為蔽其南也|{
	燕因肩翻被皮義翻}
且秦之攻燕也戰於千里之外趙之攻燕也戰於百里之内|{
	燕南與趙接境戰於百里之内言其近也秦欲攻燕自蒲潼下兵則為趙所隔故必逕上郡之西出雲中九原然後至燕故云戰於千里之外}
夫不憂百里之患而重千里之外計無過於此者|{
	夫音扶計無過於此者言燕計之過無甚於此}
願大王與趙從親|{
	從子容翻}
天下為一則燕國必無患矣|{
	此蘇秦為燕至計先定於胷中者}
文公從之資蘇秦車馬以說趙肅侯曰當今之時山東之建國莫強於趙|{
	說式芮翻建國猶言立國也}
秦之所害亦莫如趙然而秦不敢舉兵伐趙者畏韓魏之議其後也秦之攻韓魏也無有名山大川之限稍蠶食之傅國都而止|{
	傅讀曰附傅著之傅}
韓魏不能支秦必入臣於秦秦無韓魏之規則禍中於趙矣|{
	中竹仲翻}
臣以天下地圖案之諸侯之地五倍於秦料度諸侯之卒十倍於秦|{
	料音聊又如字度徒洛翻}
六國為一并力西鄉而攻秦|{
	鄉讀曰嚮}
秦必破矣夫衡人者皆欲割諸侯之地以與秦|{
	衡讀曰横衡人說客之連横者}
秦成則其身富榮國被秦患而不與其憂|{
	與讀曰預}
是以衡人日夜務以秦權恐喝諸侯以求割地|{
	索隱曰:恐起拱翻喝許曷翻又呼曷翻謂相恐脅也鄒氏喝音憇義疎}
故願大王熟計之也竊為大王計莫如一韓魏齊楚燕趙為從親以畔秦|{
	從子容翻畔反也反秦之所為也秦之所為者衡也}
令天下之將相會於洹水之上|{
	令盧經翻使也將即亮翻相息亮翻徐廣曰洹水出汲郡林慮縣水經洹水出上黨泫氏縣東北出山逕鄴縣南又東過内黄縣北入于白溝洹音桓又于元翻慮音廬}
通質結盟|{
	質音致}
約曰秦攻一國五國各出鋭師或撓秦|{
	服䖍曰撓弱也音奴教翻又音乃卯翻}
或救之有不如約者五國共伐之諸侯從親以擯秦秦甲必不敢出於函谷以害山東矣|{
	從子容翻擯必刃翻班志弘農郡弘農縣有秦函谷關漢武帝從楊僕之請移關於新安縣文頴曰秦關在弘農縣衡嶺後移在河南穀成縣師古曰今桃林縣有洪溜澗水即古所謂函谷其水北流入河夾河之岸尚有舊關餘跡焉穀成即新安杜佑曰漢函谷關在漢新安縣東北一里其秦關在今靈寶縣}
肅侯大說|{
	索隱曰:肅侯名語諡法剛德克就曰肅執心决斷曰肅說與悦同}
厚待蘇秦尊寵賜賚之以約於諸侯會秦使犀首伐魏大敗其師四萬餘人禽將龍賈取雕陰|{
	犀首魏官名公孫衍為此官因號犀首猶虎牙將軍之稱龍姓出於龍伯氏或云出於御龍氏班志上郡有雕隂道括地志雕陰故城在鄜州洛交縣北二十里敗補邁翻稱尺證翻}
且欲東兵|{
	言引兵東下也}
蘇秦恐秦兵至趙而敗從約|{
	從子容翻}
念莫可使用於秦者乃激怒張儀入之於秦張儀者魏人與蘇秦俱事鬼谷先生|{
	蓋居於鬼谷因以稱之隋志馮翊郡韓城縣有鬼谷風俗通義曰鬼谷先生六國時縱横家索隱曰:扶風池陽潁川陽城並有鬼谷蓋是其人所居因以為號又樂壹注鬼谷子書云蘇秦欲神祕其道故假名鬼谷}
學縱横之術|{
	縱與從同音子容翻}
蘇秦自以為不及也儀游諸侯無所遇困於楚蘇秦故召而辱之儀恐|{
	恐史記作怒}
念諸侯獨秦能苦趙遂入秦蘇秦隂遣其舍人齎金幣資儀|{
	文頴曰舍人主廐内小吏官名也師古曰舍人親近左右之通稱也後遂為司屬官號齎則兮翻}
儀得見秦王秦王悦之|{
	說讀曰悦}
以為客卿|{
	秦有客卿之官以待自諸侯來者其位為卿而以客禮待之也}
舍人辭去曰蘇君憂秦伐趙敗從約|{
	敗補邁翻從子容翻}
以為非君莫能得秦柄故激怒君使臣陰奉給君資盡蘇君之計謀也張儀曰嗟乎此吾在術中而不悟吾不及蘇君明矣為吾謝蘇君蘇君之時儀何敢言|{
	為吾之為于偽翻為後蘇秦死儀方出說六國張本說式芮翻}
於是蘇秦說韓宣惠王曰韓地方九百餘里帶甲數十萬天下之強弓勁弩利劒皆從韓出韓卒超足而射百發不暇止以韓卒之勇被堅甲蹠勁弩|{
	被皮義翻蹠之石翻踏也史記正義曰欲放弩者皆坐舉足踏弩材手引凑機然後發之}
帶利劒一人當百不足言也大王事秦秦必求宜陽成臯|{
	韓之宜陽西接境於秦當函谷出兵之路成臯春秋鄭之制邑亦曰虎牢戰國時為鄭之屛蔽皆韓之地班志宜陽屬弘農郡成臯屬河南郡}
今兹効之明年復求割地|{
	復扶乂翻}
與則無地以給之不與則棄前功受後禍且大王之地有盡而秦求無己以有盡之地逆無己之求此所謂市怨結禍者也|{
	市買也凡以物買賣貿易曰市}
不戰而地已削矣鄙諺曰寜為雞口無為牛後|{
	諺魚變翻俗言也史記正義曰雞口雖小猶進食牛後雖大乃出糞爾雅翼曰蘇秦說韓王寧為雞尸無為牛從尸主也一羣之主所以將衆也從從物者也謂牛子也随羣而往制不在我者也言寜為雞中之主不為牛子之從後也此本諸延篤注戰國策}
夫以大王之賢挾強韓之兵而有牛後之名臣竊為大王羞之|{
	夫音扶挾戶頰翻為于偽翻}
韓王從其言蘇秦說魏王曰大王之地方千里地名雖小然而田舍廬廡之數|{
	廡文甫翻數七欲翻密也}
曾無所芻牧|{
	曾才登翻芻刈草也牧放牧也言魏民居蕃庶無刈芻牧放之地也}
人民之衆車馬之多日夜行不絶輷輷殷殷若有三軍之衆|{
	輷呼宏翻殷音隱}
臣竊量大王之國不下楚|{
	量呂張翻量度也}
今竊聞大王之卒武士三十萬蒼頭二十萬奮擊二十萬厮徒十萬|{
	武士武卒也詳見後第六卷秦昭襄王五十二年蒼頭謂著青帽項羽傳有異軍蒼頭特起奮擊簡軍中之勇士敢奮力而擊敵者異之蘇林曰取薪之卒曰厮音斯}
車六百乘騎五千匹|{
	古者用車戰戰國始用騎兵車騎異用而並用矣乘繩證翻騎奇寄翻}
乃聽於羣臣之說而欲臣事秦故敝邑趙王使臣効愚計奉明約在大王之詔詔之魏王聽之蘇秦說齊王曰齊四塞之國地方二千餘里帶甲數十萬粟如丘山三軍之良五家之兵|{
	三軍謂三晉之軍高誘曰五家即五國}
進如鋒矢戰如雷霆解如風雨即有軍役未嘗倍泰山絶清河涉渤海者也|{
	倍與背同音蒲妹翻鄉倍之倍也班志泰山在泰山郡博縣東北水經淇水自館陶清淵東北過廣宗縣東為清河漢因置清河郡清河又東過脩縣與大河張甲故瀆合又東過東光南皮等縣齊之北界也又齊東北皆阻海漢渤海郡亦其境也師古曰郡在渤海之濱因以為名直度曰絶由膝以上曰涉}
臨淄之中七萬戶臣竊度之不下戶三男子不待發於遠縣而臨淄之卒固已二十一萬矣臨淄甚富而實其民無不鬭雞走狗六博闒鞠|{
	說文曰六博局戲也六箸十二棊烏胄所作楚辭箟蔽象棊有六博鮑宏博經曰琨蔽玉箸也各投六箸行六棊故曰六博用十二棊六棊白六棊黑所擲頭謂之瓊瓊有五采刻為一畫者謂之塞刻為兩畫者謂之白刻為三畫者謂之黑一邊不刻者五塞之間謂之五塞闒鞠史記作蹋鞠以皮為之實之以毛蹵蹋而戲劉向曰蹵鞠起於戰國之時所以練武士因嬉戲而講習之或言黄帝所作闒徒臘翻}
臨淄之塗車轂擊人肩摩連袵成帷揮汗成雨夫韓魏之所以重畏秦者為與秦接境壤也|{
	夫音扶為于偽翻}
兵出而相當不十日而戰勝存亡之機决矣|{
	而戰句斷勝下當有負字以此觀之文意明通竊謂通鑑承史記元文之誤}
韓魏戰而勝秦則兵半折|{
	折常列翻摧折也}
四境不守戰而不勝則國已危亡隨其後是故韓魏之所以重與秦戰而輕為之臣也今秦之攻齊則不然倍韓魏之地過衛陽晉之道經乎亢父之險車不得方軌騎不得比行|{
	水經注瓠子河出東郡濮陽縣北河南至濟隂句陽縣為新溝又東過廪丘縣與濮水俱東瓠河又逕陽晉城南蘇秦所謂衛陽晉之道也史記正義曰陽晉故城在曹州乘氏縣西北三十七里班志亢父縣屬東平國又括地志亢父故城在兖州任城縣南五十一里亢音抗又音剛父音甫說文軌車轍也顔師古曰車併行為方軌騎奇寄翻比毘義翻次也行戶剛翻列也凡行列之行皆同音車併讀曰並}
百人守險千人不敢過也秦雖欲深入則狼顧恐韓魏之議其後也|{
	爾雅翼狼猛而敏給能自顧其後蓋狼行而屢顧恐人掎其後故也掎居綺翻}
是故恫疑虛喝驕矜而不敢進|{
	恫他紅翻恐懼貌高誘曰虛喝喘息懼貌劉氏曰秦自疑懼虛作恐猲之辭以脅韓魏也史記正義曰言秦雖至亢父猶恐懼狼顧虛作喝罵驕溢矜誇而不敢進伐齊喝呼葛翻亦作猲音同}
則秦之不能害齊亦明矣夫不深料秦之無奈齊何而欲西面而事之是羣臣之計過也|{
	夫音扶}
今無臣事秦之名而有強國之實臣是故願大王少留意計之齊王許之|{
	少始紹翻}
乃西南說楚威王曰楚天下之強國也地方六千餘里帶甲百萬車千乘騎萬匹|{
	楚在齊之西南故蘇秦自齊而西南詣楚說式芮翻乘䋲證翻騎奇寄翻}
粟支十年此覇王之資也秦之所害莫如楚楚強則秦弱秦強則楚弱其勢不兩立故為大王計莫如從親以孤秦|{
	從子容翻}
臣請令山東之國|{
	令廬經翻使也}
奉四時之獻以承大王之明詔委社稷奉宗廟練士厲兵在大王之所用之故從親則諸侯割地以事楚衡合則楚割地以事秦|{
	從子容翻衡讀曰横}
此兩策者相去遠矣大王何居焉楚王亦許之於是蘇秦為從約長|{
	長知丈翻}
并相六國|{
	相息亮翻}
北報趙車騎輜重擬於王者|{
	康曰輜重載物車也行者之車總曰輜重韻書曰輜莊持翻庫車也重直用翻考異曰史記蘇秦傳秦兵不敢闚函谷關十五年又云秦使犀首欺齊魏與共伐趙蘇秦去趙而從約皆解齊魏伐趙敗從約止在明年耳秦本紀惠文王十年公子卭與魏戰虜其將龍賈後二年事耳烏在其不闚函谷十五年乎此出於游談之士誇大蘇秦而云爾今不取}
齊威王薨子宣王辟強立知成侯賣田忌|{
	事見上二十八年}
乃召而復之燕文公薨子易王立|{
	諡法好更改舊曰易燕因肩翻易音如字更工衡翻}
衛成侯薨c子平侯立e|{
	諡法治而無眚曰平執事有制曰平布綱治紀曰平又曰惠無内德曰平}


	三十七年

秦惠王使犀首欺齊a魏與共伐趙c以敗從約e|{
	敗補邁翻從子容翻}
趙肅侯讓蘇秦蘇秦恐請使燕必報齊蘇秦去趙而從約皆解趙人决河水以灌齊魏之師齊魏之師乃去 魏以陰晉為和於秦寔華陰|{
	班志華隂故隂晉秦惠文王五年更名寜秦漢高帝改曰華陰縣屬京兆以其地在華山之陰也宋白曰華隂分秦晉之境邊晉之西則曰隂晉邊秦之東則曰寧秦華戶化翻}
齊王伐燕取十城已而復歸之三十九年秦伐魏圍焦曲沃|{
	班志弘農郡陜縣有焦城左傳所謂晉與秦焦瑕者也括地志焦在陜城東百步曲沃在陜西南三十二里因曲沃水為名酈道元曰按春秋文公十三年晉侯使詹嘉處瑕守桃林之塞以備秦時以曲沃之官守之故曲沃之名遂為積古之傳宋白曰焦古焦國括地志焦城在焦城東北百步因焦水為名周同姓所封}
魏人少梁河西地於秦|{
	少詩照翻二十九年魏已使使獻河西於秦以和今乃入其地}


四十年

秦伐魏度河取汾隂皮氏|{
	班志汾隂縣屬河東郡皮氏縣故耿國晉獻公以封趙夙者也亦屬河東郡括地志汾隂故城在蒲州汾隂縣北九里皮氏故城在絳州龍門縣西百八十步}
抜焦 楚威王薨子懷王槐立|{
	諡法慈仁短折曰懷又懷思也槐乎乖翻又乎瑰翻折而設翻}
宋公剔成之弟偃襲攻剔成剔成奔齊偃自立為君|{
	剔他歷翻}


四十一年

秦公子華張儀帥師圍魏蒲陽取之|{
	史記正義曰蒲陽在隰州隰川縣蒲邑故城是也帥讀曰率}
張儀言於秦王請以蒲陽復與魏而使公子繇質於魏|{
	質音致}
儀因說魏王曰秦之遇魏甚厚魏不可以無禮於秦魏因盡入上郡十五縣以謝焉|{
	說式芮翻括地志曰上郡故城在綏州上縣東南五十里魏秦之上郡地也史記正義曰按鄜坊丹延等州北至固陽盡上郡地魏築長城界秦自華州鄭縣濱洛至慶州洛源縣白於山即東北至勝州固陽東至河西上郡之地盡入於秦秦之與魏者小魏之謝秦者大史言張儀為秦計者甚巧}
張儀歸而相秦|{
	相息亮翻}


四十二年

秦縣義渠以其君為臣|{
	義渠西戎國名秦取之以為縣班志義渠道屬北地郡括地志寧慶原三州秦之北地郡也}
秦歸焦曲沃於魏|{
	既取而復歸之秦之於魏若玩弄嬰兒於掌股之上耳}


四十三年

趙肅侯薨|{
	索隱曰:肅侯名語}
子武靈王立置博聞師三人左右司過三人先問先君貴臣肥義加其秩|{
	索隱曰:武靈王名雍姓譜肥姓肥子之後以國為姓}


四十四年

夏四月戊午秦初稱王 衛平侯薨子嗣君立|{
	嗣祥吏翻}
衛有胥靡亡之魏|{
	漢書音義曰胥相也靡隨也古者相隨坐輕刑之名謂罪不至於朴刑者令衣褐帶索相隨以執役朱元晦曰胥靡者連鎻役作也胥新於翻靡母被翻}
因為魏王之后治病|{
	為于偽翻治直之翻}
嗣君聞之請以五十金買之五反魏不與乃以左氏易之左右諫曰夫以一都買一胥靡可乎|{
	夫音扶下同}
嗣君曰非子所知也夫治無小亂無大|{
	治直吏翻}
灋不立誅不必雖有十左氏無益也灋立誅必失十左氏無害也魏王聞之曰人主之欲不聼之不祥因載而往徒獻之|{
	此學申韓者為之說耳}


四十五年

秦張儀帥師伐魏取陜|{
	班志陜縣屬弘農郡故虢國北虢在大陽東虢在滎陽西虢在雍州周召分陜而治即此陜也帥讀曰率陜失冉翻召讀曰邵}
蘇秦通於燕文公之夫人易王知之蘇秦恐乃說易王曰臣居燕不能使燕重而在齊則燕重易王許之|{
	燕因肩翻易音如字說式芮翻諡法好更改舊曰易註變故改常}
乃偽得罪於燕而奔齊齊宣王以為客卿蘇秦說齊王高宮室大苑囿以明得意欲以敝齊而為燕|{
	說式芮翻為後齊大夫殺蘇秦張本}


四十六年

秦張儀及齊楚之相會齧桑|{
	相息亮翻服䖍曰齧桑翟地徐廣曰在梁與彭城之間裴駰曰晉地索隱曰:衛地余按漢武帝瓠子歌曰齧桑浮兮淮泗滿及塞决河而梁楚之地復寧無水災後漢王梁擊佼強蘇茂於楚沛間拔大梁齧桑則徐說為近之齧五結翻}
韓燕皆稱王趙武靈王獨不肯曰無其實敢處其名乎令國人謂已曰君|{
	趙武靈王之不肯稱王非守君臣之分居之以謙也將求其所大欲而力未能稱心也處昌呂翻令力丁翻使也又力正翻命令也分扶問翻稱尺證翻}


四十七年

秦張儀自齧桑還而免相相魏|{
	齧魚結翻還從宣翻又音如字免相免秦相而相魏相息亮翻}
欲令魏先事秦而諸侯効之魏王不聼秦王伐魏取曲沃平周|{
	令力丁翻此曲沃在河東晉桓叔所封之邑漢武帝改名聞喜史記正義曰絳州桐鄉縣晉曲沃邑十三州志古平周邑在汾州介休縣西四十里}
復陰厚張儀益甚


四十八年

王崩子愼靚王定立|{
	靚疾正翻}
燕易王薨子噲立|{
	燕因肩翻易音如字噲古夬翻}
齊王封田嬰於薛|{
	班志薛縣屬魯國夏奚仲之國後遷于邳仲虺居之括地志故薛城在今徐州滕縣界史記正義曰薛故城在今徐州滕縣南四十四里}
號曰靖郭君|{
	杜佑曰戰國之際秦項之間權設班寵有加賜邑封君者盖假其位號或空受其爵如靖郭武安之類是也至漢尤多蓋在封爵之外别加美號史記列傳云嬰諡為靖郭君索隱曰:靖郭或封邑號故漢駟鈞封靖郭君}
靖郭君言於齊王曰五官之計不可不日聼而數覧也|{
	記曾子問諸侯出命國家五官而後行註云五官五大夫典事者命者勑之以其職正義云按太宰職云建其牧立其監設其參傳其伍是諸侯有三卿五大夫經云五官故云五大夫以屬官大夫其數衆多直云五者據典國事言之不云命卿者或從君出行或雖在國留守揔主羣吏如三公然不專主一事且尊之既命五大夫則卿亦命之可知故不顯言命卿也余謂此所謂五官蓋亦言典事五大夫也數所角翻}
王從之已而厭之悉以委靖郭君靖郭君由是得專齊之權靖郭君欲城薛客謂靖郭君曰君不聞海大魚乎網不能止鉤不能牽蕩而失水則螻蟻制焉今夫齊亦君之水也|{
	夫音扶}
君長有齊奚以薛為苟為失齊雖隆薛之城到于天庸足恃乎乃不果城|{
	隆高也崇也庸長也}
靖郭君有子四十人其賤妾之子曰文文通儻饒智畧|{
	通達也儻倜儻卓異也饒智畧言智畧有餘也}
說靖郭君以散財養士靖郭君使文主家待賓客賓客爭譽其美|{
	說式芮翻譽音余}
皆請靖郭君以文為嗣靖郭君卒|{
	嗣祥吏翻卒子恤翻}
文嗣為薛公號曰孟嘗君|{
	史記列傳曰諡曰孟嘗君索隱曰:號曰孟嘗君曰諡非也孟字嘗邑名嘗邑在薛之旁}
孟嘗君招致諸侯遊士及有罪亡人皆舍業厚遇之|{
	舍業為之築舍立居業也}
存救其親戚食客常數千人各自以為孟嘗君親已由是孟嘗君之名重天下


臣光曰君子之養士以為民也易曰聖人養賢以及萬民|{
	頤卦彖辭也為于偽翻}
夫賢者其德足以敦化正俗其才足以頓綱振紀|{
	頓謂整頓夫音扶}
其明足以燭微慮遠其強足以結仁固義大則利天下小則利一國是以君子豐禄以富之隆爵以尊之養一人而及萬人者養賢之道也今孟嘗君之養士也不恤智愚不擇臧否|{
	否補美翻}
盜其君之禄以立私黨張虚譽上以侮其君下以蠧其民是姦人之雄也烏足尚哉書曰受為天下逋逃主萃淵藪此之謂也

孟嘗君聘於楚楚王遺之象牀登徒直送之|{
	象牀以象齒為之登徒姓也直其名遺于季翻}
不欲行謂孟嘗君門人公孫戌曰象牀之直千金苟傷之毫髮則賣妻子不足償也足下能使僕無行者有先人之寶劒願獻之公孫戌許諾|{
	姓譜公孫氏出於黄帝釋名曰劒檢也所以防檢非常也戊音恤償辰羊翻報也諾奴各翻以言許人曰諾}
入見孟嘗君曰小國所以皆致相印於君者以君能振達貧窮存亡繼絶故莫不悦君之義慕君之廉也今始至楚而受象牀則未至之國將何以待君哉孟嘗君曰善遂不受公孫戌趨去未至中閨|{
	閨涓畦翻宮中小門曰閨上圓下方如圭故謂之閨}
孟嘗君召而反之曰子何足之高志之揚也公孫戌以實對孟嘗君乃書門版曰有能揚文之名止文之過私得寶於外者疾入諫


臣光曰
孟嘗君可謂能用諫矣c
苟 其言之善也c
雖懷詐諼之心c猶將用之e|{
	諼許元翻}
况盡忠c無私以事其上乎e
詩云c采葑采菲無以下體c|{
	詩邶谷風之辭毛氏傳曰葑須也菲芴也鄭氏箋曰此二菜蔓菁與蒚之類也皆上下可食然其根有美時冇惡時采之者不可以根惡并棄其葉下體謂根莖也陸璣草木疏曰葑蕪菁也郭璞曰今菘菜陸德明曰江南有菘江北有蔓菁相似而異爾雅曰菲芴又曰菲息菜郭璞曰菲芴土瓜息菜似菁蕪華紫赤色可食蒚大葉白華根如指色白可食菲敷尾翻邶蒲昧翻芴挨拂翻蔓謨官翻蒚方六翻}
孟嘗君有焉e


韓宣惠王欲兩用公仲公叔為政
問於繆留|{
	繆莫留翻姓也又靡幼翻又音穆}
對曰s不可e
晉用六卿 而國分e
齊簡公 用陳成子及闞止c而見殺e
魏用犀首a張儀而西河之外亡e|{
	晉六卿智氏范氏中行氏趙氏韓氏魏氏也自晉文襄以來迭秉國政後皆強大卒分晉國齊簡公使闞止為政陳成子憚之已而陳常殺闞止弑簡公闞以邑為氏蘇代曰魏相犀首必右韓而左魏相張儀必右秦而左魏盖二相外各倚與國以為重而内爭權所以魏日削也闞戶監翻行戶剛翻恒戶登翻卒子恤翻相息亮翻}
今君兩用之e
其多力者 内樹黨c
其寡力者 藉外權羣臣e
有内樹黨以驕主c
有外為交以削地e
君之國危矣e

資治通鑑卷二


\textcolor{red}{\LaTeX{}~~Du Huanpeng <dhu@hodcarrier.org>}
