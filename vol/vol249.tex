\section{資治通鑑卷二百四十九}
宋 司馬光 撰

胡三省 音註

唐紀六十五|{
	起上章敦牂盡屠維單閼几十年}


宣宗元聖至明成武獻文睿智章仁神聰懿道大孝皇帝下

大中四年春正月庚辰朔赦天下 二月以秦州隸鳳翔|{
	秦州本屬隴右節度是時新復以屬鳳翔}
夏四月庚戌以中書侍郎同平章事馬植為天平節度使上之立也左軍中尉馬元贄有力焉|{
	武宗之大漸也馬元贄為左神策護軍中尉立上為皇太叔}
由是恩遇冠諸宦者|{
	冠古玩翻}
植與之敘宗姓上賜元贄寶帶元贄以遺植|{
	遺唯季翻}
植服之以朝|{
	朝直遥翻}
上見而識之植變色不敢隱明日罷相收植親吏董牟下御史臺鞫之盡得植與元贄交通之狀|{
	下戶嫁翻}
再貶常州刺史|{
	常州古延陵季子之邑後為毗陵晉為晉陵唐為常州京師東南二千八百四十三里}
六月戊申兵部侍郎同平章事魏扶薨以戶部尚書判度支崔龜從同平章事 秋八月以白敏中判延資庫|{
	去年改備邊庫為延資庫}
盧龍節度使周綝薨軍中表請以押牙兼馬步都知兵馬使張允伸為留後九月丁酉從之 |{
	考異曰四年七月周綝薨張允伸為留後注曰舊紀亦無朝廷命綝為節度使年月至此但云幽州節度使周綝卒軍人立張允伸為留後實錄九月幽州大將表請押衙張允伸知留後事舊允伸傳曰大中四年戎帥周綝寢疾表允伸為留後朝廷可其奏今從之 今按通鑑書八月周綝薨考異以為七月}
党項為邊患諸道兵討之連年無功戍饋不已右補闕孔温裕上疏切諫上怒貶柳州司馬温裕戣之兄子也|{
	孔戣見二百四十卷憲宗元和十二年}
吐蕃論恐熱遣僧莽羅藺真將兵於雞項關南造橋以擊尚婢婢軍於白吐嶺|{
	水經注左南津西六十里有白土城城西北有白土川水其地在唐河州鳳林縣西以此推之雞項關亦在河州界}
婢婢遣其將尚鐸羅榻藏將兵據臨蕃軍以拒之不利復遣磨離羆子燭盧鞏力|{
	復扶又翻}
將兵據牛峽以拒之|{
	力之翻}
鞏力請按兵拒險勿與戰以奇兵絶其糧道使進不得戰退不得還不過旬月其衆必潰羆子不從鞏力曰吾寧為不用之人不為敗軍之將稱疾歸鄯州羆子逆戰敗死婢婢糧乏留拓跋懷光守鄯州帥部落三千餘人就水草于甘州西|{
	帥讀曰率宋白曰甘州西南至肅州福祿縣界赤柳澗三百三十里肅州南至吐蕃界四百里}
恐熱聞婢婢棄鄯州自將輕騎五千追之至瓜州|{
	宋白曰瓜州東南至肅州界三百四十里}
聞懷光守鄯州遂大掠河西鄯廓等八州|{
	宋白曰廓州北至鄯州百八十里東南至河州鳳林縣二百八十里}
殺其丁壯劓刖其羸老|{
	劓魚氣翻刖魚決翻羸倫為翻}
及婦人以槊貫嬰兒為戲焚其室廬五千里間赤地殆盡冬十月辛未以翰林學士承旨兵部侍郎令狐綯同

平章事 |{
	考異曰舊紀在十一月今從實錄新紀}
十一月壬寅以翰林學士劉瑑為京西招討党項行營宣慰使|{
	瑑持兖翻}
以盧龍留後張允伸為節度使 十二月以鳳翔節度使李業河東節度使李拭並兼招討党項使 吏部侍郎孔温業白執政求外官白敏中謂同列曰我輩須自檢點孔吏部不肯居朝廷矣温業戣之弟子也|{
	孔温業之操行不見于史時人蓋以其家世而敬之}


五年春正月壬戌天德軍奏攝沙州刺史張義潮遣使來降|{
	降戶江翻下同沙州東南至長安三千八百五十九里}
義潮沙州人也時吐蕃大亂義潮隂結豪傑謀自拔歸唐一旦帥衆被甲譟于州門|{
	帥讀曰率被皮義翻}
唐人皆應之吐蕃守將驚走義潮遂攝州事奉表來降 |{
	考異曰補國史作議潮今從實錄新舊傳紀}
以義潮為沙州防禦使 以兵部侍郎裴休為鹽鐵轉運使休肅之子也|{
	裴肅見二百三十五卷德宗貞元十二年}
自太和以來歲運江淮米不過四十萬斛吏卒侵盜沈沒舟達渭倉者什不三四大墮劉晏之法|{
	沈持林翻墮讀曰隳劉晏法見二百二十六卷德宗建中元年}
休窮究其弊立漕法十條歲運米至渭倉者百二十萬斛 上頗知党項之反由邊帥利其羊馬數欺奪之或妄誅殺党項不勝憤怨故反|{
	帥所類翻數所角翻勝音升}
乃以右諫議大夫李福為夏綏節度使自是繼選儒臣以代邊帥之貪暴者行日復面加戒勵|{
	復扶又翻}
党項由是遂安福石之弟也上以南山平夏党項久未平|{
	党項居慶州者號東山部居夏州者號平夏部其竄居南山者為南山党項趙珣聚米圖經党項部落在銀夏以北居川澤者謂之平夏党項在安鹽以南居山谷者謂之南山党項 考異曰唐年補錄曰松州南有雪山故曰南山平夏川名也 余按唐年補錄乃末學膚受者之為耳今不欲復言地理姑以通鑑義例言之考異者考羣書之同異而審其是訓釋其義付之後學南山之說既無異同之可考今而引之疑非考異本指也}
頗厭用兵崔鉉建議宜遣大臣鎮撫三月以白敏中為司空同平章事充招討党項行營都統制置等使|{
	職源曰制置使始此}
南北兩路供軍使兼邠寧節度使敏中請用裴度故事擇廷臣為將佐許之|{
	裴度故事見二百四十卷憲宗元和十二年}
夏四月以左諫議大夫孫景商為左庶子充邠寧行軍司馬知制誥蔣伸為右庶子充節度副使伸係之弟也|{
	蔣係見二百四十四卷文宗太和五年}
初上令白敏中為萬夀公主選佳壻|{
	為于偽翻}
敏中薦鄭顥時顥已昏盧氏行至鄭州堂帖追還|{
	萬夀公主適鄭顥見上卷上年鄭州去京師一百一百五里}
顥甚銜之由是數毁敏中于上敏中將赴鎮言于上曰鄭顥不樂尚主|{
	數所角翻樂音洛}
怨臣入骨髓臣在政府無如臣何今臣出外顥必中傷臣死無日矣|{
	中竹仲翻}
上曰朕知之久矣卿何言之晩邪命左右于禁中取小檉函以授敏中曰此皆鄭郎譖卿之書也朕若信之豈任卿以至今日敏中歸置檉函于佛前焚香事之|{
	檉丑貞翻說文曰河柳也}
敏中軍于寧州壬子定遠城使史元破党項九千餘帳于三交谷|{
	三交谷在夏州界}
敏中奏党項平辛未詔平夏党項已就安帖|{
	平夏地名在夏州界宋朝李繼遷之叛也徙綏州吏民之半置平夏以為巢穴蓋銀夏之要地也}
南山党項聞出山者廹于饑寒猶行鈔掠|{
	鈔楚交翻}
平夏不容窮無所歸宜委李福存諭于銀夏境内授以閒田如能革心向化則撫如赤子從前為惡一切不問或有抑屈聽于本鎮投牒自訴若再犯疆場或復入山林|{
	復扶又翻下同}
不受教令則誅討無赦將吏有功者甄奬|{
	甄稽延翻}
死傷者優恤靈夏邠鄜四道百姓給復三年鄰道量免租稅|{
	鄜音膚復方目翻量音良}
曏由邊將貪鄙致其怨叛自今當更擇廉良撫之若復致侵叛當先罪邊將後討寇虜 吐蕃論恐熱殘虐所部多叛拓跋懷光使人說誘之|{
	說式芮翻誘以久翻導引也}
其衆或散居部落或降於懷光恐熱勢孤乃揚言于衆曰吾今入朝于唐借兵五十萬來誅不服者然後以渭州為國城請唐冊我為贊普誰敢不從五月恐熱入朝上遣左丞李景讓就禮賓院問所欲恐熱氣色驕倨言語荒誕|{
	誕徒旱翻誇大也}
求為河渭節度使上不許召對三殿如常日胡客勞賜遣還|{
	勞力到翻還從宣翻}
恐熱怏怏而去復歸落門川聚其舊衆|{
	恐熱本吐蕃落門討擊使}
欲為邊患會久雨乏食衆稍散纔有三百餘人奔于廓州 六月立皇子潤為鄂王 進士孫樵上言百姓男耕女織不自温飽而羣僧安坐華屋美衣精饌|{
	饌雛晥翻又雛戀翻}
率以十戶不能養一僧武宗憤其然|{
	憤其然猶言憤其如此也}
髮十七萬僧|{
	言使僧長髮復為齊民也}
是天下一百七十萬戶始得蘇息也陛下即位以來修復廢寺天下斧斤之聲至今不絶度僧幾復其舊矣|{
	幾居依翻}
陛下縱不能如武宗除積弊奈何興之于已廢乎日者陛下欲修國東門諫官上言遽為罷役|{
	為于偽翻}
今所復之寺豈若東門之急乎所役之功豈若東門之勞乎願早降明詔僧未復者勿復寺未修者勿修庶幾百姓猶得以息肩也秋七月中書門下奏陛下崇奉釋氏羣下莫不奔走恐財力有所不逮因之生事擾人望委所在長吏量加撙節|{
	撙慈損翻}
所度僧亦委選擇有行業者|{
	行下孟翻}
若容凶麤之人則更非敬道也鄉村佛舍請罷兵日修|{
	時用兵以復河湟}
從之 八月白敏中奏南山党項亦請降時用兵歲久國用頗乏詔并赦南山党項使之安業 冬十月乙卯中書門下奏今邊事已息而州府諸寺尚未畢功望且令成之其大縣遠于州府者聽置一寺其鄉村毋得更置佛舍從之 戊辰以戶部侍郎魏謩同平章事仍判戶部時上春秋已高未立太子羣臣莫敢言謩入謝因言今海内無事惟未建儲副使正人輔導臣竊以為憂且泣時人重之|{
	重之者以其能言人所不敢言也}
蓬果羣盜依阻雞山寇掠三川|{
	雞山在蓬果二州之界而羣盜依阻以寇掠三川則其結根也廣矣三川謂東西川及山南西道也}
以果州刺史王贄弘充三川行營都知兵馬使以討之 制以党項既平罷白敏中都統但以司空平章事充邠寧節度使 張義潮兵畧定其旁瓜伊西甘肅蘭鄯河岷廓十州遣其兄義澤奉十一州圖籍入見|{
	十州并沙州為十一州見賢遍翻宋白曰瓜州西至沙州二百八十里西北至伊州九百里西州東至伊州七百五十里甘州西至肅州四百二十里肅州南至瓜州五百二十六里蘭州西至鄯州四百九十里鄯州西至廓州二百八十里河州東北至蘭州三百里岷州北至蘭州狄道縣五百三十四里西北至河州大夏縣三百六十三里}
於是河湟之地盡入于唐十一月置歸義軍於沙州以義潮為節度使 |{
	考異曰唐年補錄舊紀義潮降在五年八月獻祖紀年録及新紀在十月按實録五年二月壬戌天德軍奏沙州刺史張義潮安景旻及部落使閻英達等差使上表請以沙州降十月義潮遣弟義澤以本道瓜沙伊肅等十一州地圖戶籍來獻河隴陷沒百餘年至是悉復故地十一月建沙州為歸義軍以張義潮為節度使河沙等十一州觀察營田處置等使新紀五年十月沙州人張義潮以瓜沙伊肅鄯甘河西蘭岷廓十一州歸于有司新傳三州七關降之明年沙州首領張義潮奉十一州地圖以獻擢義潮沙州防禦使俄號歸義軍遂為節度使參考諸書蓋二月義潮使者始以得沙州來告除防禦使十月又遣義澤以十一州圖籍來上除節度使也今從實録新傳云三州降之明年誤也}
十一州觀察使又以義潮判官曹義金為歸義軍長史|{
	按新書百官志節度使有行軍司馬節度副使判官支使等其兼都督都護則有長史}
以中書侍郎同平章事崔龜從同平章事充宣武節

度使 右羽林統軍張直方坐出獵累日不還宿衛貶左驍衛將軍

六年春二月王贄弘討雞山賊平之是時山南西道節度使封敖奏巴南妖賊言辭悖慢上怒甚|{
	妖於驕翻悖蒲妹翻又蒲沒翻}
崔鉉曰此皆陛下赤子廹于飢寒盜弄陛下兵於谿谷間不足辱大軍但遣一使者可平矣乃遣京兆少尹劉潼詣果州招諭之潼上言請不兵攻討且曰今以日月之明燭愚迷之衆使之稽顙歸命其勢甚易|{
	稽音啓易以豉翻}
所慮者武臣耻不戰之功議者責欲速之效耳潼至山中盜彎弓待之潼屏左右直前|{
	屏必郢翻又卑正翻}
曰我面受詔赦汝罪使汝復為平人聞汝木弓射二百步今我去汝十步汝真欲反者可射我|{
	射而亦翻}
賊皆投弓列拜請降|{
	降戶江翻}
潼歸館而王贄弘與中使似先義逸引兵已至山下竟擊滅之 三月勑先賜右衛大將軍鄭光鄠縣及雲陽莊並免稅役中書門下奏以為稅役之法天下皆同陛下屢德音欲使中外畫一|{
	漢書蕭何為法講若畫一師古注曰畫一言整齊也}
今獨免鄭光似稍乖前意事雖至細繫體則多勑曰朕以鄭光元舅之尊貴欲優異令免征稅初不細思况親戚之間人所難議卿等苟非愛我豈進嘉言庶事能盡如斯天下何憂不理有始有卒|{
	卒子恤翻}
當共守之並依所奏 夏四月甲辰以邠寧節度使白敏中為西川節度使 湖南奏團練副使馮少端討衡州賊帥鄧裴平之|{
	帥所類翻下同}
党項復擾邊上欲擇可為邠寧帥者而難其人從容與翰林學士中書舍人須昌畢諴論邊事諴援古據今具陳方畧|{
	從千容翻諴戶嵓翻援于元翻}
上悦曰吾方擇帥不意頗牧近在禁廷卿其為朕行乎諴欣然奉命上欲重其資履|{
	為于偽翻資以序進履所歷之官也}
六月壬申先以諴為刑部侍郎癸酉乃除邠寧節度使 |{
	考異曰舊傳懿宗召問邊事今從實錄}
雍王渼薨追諡靖懷太子|{
	渼音美}
河東節度使李業縱吏民侵掠雜虜又妄殺降者由是北邊擾動閏月庚子以太子少師盧鈞為河東節度使業内有所恃人莫敢言魏謩獨請貶黜上不許但徙義成節度使盧鈞奏度支郎中韋宙為副使宙徧詣塞下悉召酋長諭以禍福|{
	酋慈由翻長知丈翻}
禁唐民毋得入虜境侵掠犯者必死雜虜由是遂安掌書記李璋杖一牙職明日牙將百餘人訴於鈞鈞杖其為首者謫戍外鎮餘皆罰之曰邊鎮百餘人無故横訴|{
	横戶孟翻}
不可不抑璋絳之子也|{
	李絳相憲宗以直諒聞帥梁為亂卒所殺}
八月甲子以禮部尚書裴休同平章事 獠寇昌資二州|{
	獠魯皓翻資州漢資中縣地宋齊為資陽戍西魏置資州至京師三千五百六十里}
冬十月邠寧節度使畢諴奏招諭党項皆降 驍衛將軍張直方坐以小過屢殺奴婢貶恩州司戶 十一月立憲宗子惴為棣王|{
	惴之睡翻}
十二月中書門下奏度僧不精則戒法墮壞|{
	墮讀曰隳}
造寺無節則損費過多請自今諸州凖元敕許置寺外有勝地靈迹許修復繁會之縣許置一院|{
	繁會謂人物浩繁舟車所會之地}
嚴禁私度僧尼若官度僧尼有闕則擇人補之仍申祠部給牒|{
	此今所謂度牒者}
其欲遠遊尋師者須有本州公驗從之|{
	公驗者自本州給公文所至以為照驗}
七年春正月戊申上祀圓丘赦天下 夏四月丙寅敕自今法司處罪|{
	處昌呂翻}
用常行杖杖脊一折法杖十|{
	法杖謂常行臋杖也脊資昔翻折之截翻}
杖臋一折笞五|{
	臋徒渾翻}
使吏用法有常凖冬十二月左補闕趙璘請罷來年元會止御宣政上

以問宰相對曰元會大禮不可罷况天下無事上曰近華州奏有賊光火刼下邽|{
	明火行刼言盜無所憚華戶化翻}
關中少雪|{
	少詩沼翻}
皆朕之憂何謂無事雖宣政亦不可御也 上事鄭太后甚謹不居别宫朝夕奉養|{
	養余亮翻}
舅鄭光歷平盧河中節度使上與之論為政光應對鄙淺上不悦留為右羽林統軍使奉朝請太后數言其貧|{
	數所角翻}
上輒厚賜金帛終不復任以民官|{
	復扶又翻民官謂治民之官}
度支奏自河湟平每歲天下所納錢九百二十五萬餘緡内五百五十萬餘緡租稅八十二萬餘緡榷酤二百七十八萬餘緡鹽利|{
	榷古岳翻酤工護翻 考異曰續皇王寶運録具載是歲度支支收之數舛錯不可曉今特存其可曉者 温公拳拳于史之闕文蓋其所重者制國用也}


八年春正月丙戌朔日有食之罷元會 上自即位以來治弑憲宗之黨宦官外戚乃至東宫官屬誅竄甚衆|{
	宣宗絶郭后景陵之合葬誅元和東宫之官屬則以為穆宗母子誠預陳弘志之謀者然文宗於穆宗父子也文宗憤元和逆黨欲盡誅之而不克以成甘露之禍使父果為商臣則子必為潘崇諱矣}
慮人情不安丙申詔長慶之初亂臣賊子頃搜擿餘黨流竄已盡|{
	温公于郭后之崩王皥之貶既詳書之矣復書此詔然王皥之議卒伸于咸通之初通鑑又書之懿宗以子繼父而天理所在者公議所在不可得而違也不可得而揜也讀通鑑者宜以是觀之}
其餘族從疎遠者一切不問|{
	從一從再從三從兄之親族袒免以外之親也從才用翻}
二月中書門下奏拾遺補闕缺員請更增補上曰諫官要在舉職不必人多如張道符牛叢趙璘輩數人使朕日聞所不聞足矣叢僧孺之子也|{
	李德裕排牛僧孺上惡德裕故親僧孺之子}
久之叢自司勲員外郎出為睦州刺史|{
	睦州吳置新都郡隋置睦州取俗阜人和内外輯睦為義京師東南三千六百五十九里}
入謝上賜之紫叢既謝前言曰|{
	謝恩之後前進而言}
臣所服緋刺史所借也上遽曰且賜緋上重惜服章有司常具緋紫衣數襲從行以備賞賜或半歲不用其一故當時以緋紫為榮上重翰林學士至于遷官必校歲月以為不可以官爵私近臣也 秋九月丙戌以右散騎常侍高少逸為陜虢觀察使有敕使過硤石|{
	硤石隋之崤縣貞觀十四年移治硤石塢改名硤石屬陜州陜失冉翻}
怒餅黑鞭驛吏見血少逸封其餅以進敕使還|{
	還音旋又如字}
上責之曰深山中如此食豈易得|{
	易以䜴翻}
讁配恭陵 立皇子洽為懷王汭為昭王汶為康王|{
	汶音問 考異曰唐年補録五年正月甲戌朔封三王今從實録新紀}
上獵于苑北|{
	此又出苑城而北獵}
遇樵夫問其縣曰涇陽人也令為誰曰李行言為政何如曰性執有強盜數人匿軍家索之|{
	軍家謂北司諸軍也唐人謂諸道節度支觀察為使家諸州為州家諸縣為縣家索山客翻}
竟不與盡殺之上歸帖其名于寢殿之柱冬十月行言除海州刺史入謝上賜之金紫問曰卿知所以衣紫乎|{
	衣於既翻下至衣紫衣黄並同}
對曰不知上命取殿柱之帖示之 上以甘露之變|{
	見二百四十五卷文宗太和八年}
惟李訓鄭注當死自餘王涯賈餗等無罪詔皆雪其寃上召翰林學士韋澳託以論詩屏左右與之語曰近日外間謂内侍權勢何如對曰陛下威斷非前朝之比|{
	澳音與屏必郢翻卑正翻斷丁亂翻朝直遥翻}
上閉目揺首曰全未全未尚畏之在|{
	句斷}
卿謂策將安出對曰若與外廷議之恐有太和之變不若就其中擇有才識者與之謀上曰此乃末策自衣黄衣緑至衣緋皆感恩纔衣紫則相與為一矣|{
	唐自上元以後三品已上服紫四品服深緋五品服淺緋六品服深緑七品服淺緑八品服緑九品深青流外官及庶人服黄太宗定制内侍省不置三品官内侍是長官階四品其職但在閤門守禦黄衣廩食而已至玄宗宦官至三品將軍門施棨戟得衣紫矣衣於既翻}
上又嘗與令狐綯謀盡誅宦官綯恐濫及無辜密奏曰但有罪勿捨有闕勿補自然漸耗至于盡矣宦者竊見其奏由是益與朝士相惡南北司如水火矣

九年春正月甲申成德軍奏節度使王元逵薨軍中立其子節度副使紹鼎癸卯以紹鼎為成德留後 二月以醴泉令李君奭為懷州刺史初上校獵渭上有父老以十數聚于佛祠上問之對曰醴泉百姓也縣令李君奭有異政考滿當罷詣府乞留故此祈佛冀諧所願耳及懷州刺史闕上手筆除君奭宰相莫之測君奭入謝上以此奬厲|{
	以所得於父老之言奬厲}
衆始知之 三月詔邠寧節度使畢諴還邠州先是以河湟初附党項未平移邠寧軍于寧州|{
	事見上卷三年先悉薦翻}
至是南山平夏皆安威鹽武三州軍食足|{
	五年以原州之蕭關置武州}
故令還理所|{
	理所猶言治所邠寧軍本理邠州北至寧州一百二十五里}
夏閏四月詔以州縣差役不均自今每縣據人貧富及役輕重作差科簿送刺史檢署訖鏁於令廳|{
	鏁蘇果翻令廳縣令廳事也}
每有役事委令據簿定差|{
	今之差役簿始此}
五月丙寅以王紹鼎為成德節度使 上聰察彊記宫中厮役給灑掃者|{
	厮音斯灑所買翻又所賣翻掃蘇老翻又素報翻}
皆能識其姓名|{
	識職吏翻}
才性所任|{
	任音壬}
呼召使令無差誤者天下奏獄吏卒姓名一覽皆記之度支奏漬汚帛|{
	汚烏故翻}
誤書漬為清樞密承旨孫隱中謂上不之見輒足成之|{
	唐末樞密承旨以院吏充五代以諸衛將軍充宋朝以士人充遂為清選}
及中書覆入|{
	内出度支奏付中書中書宣署申覆還而奏之謂之覆入}
上怒推按擅改章奏者罰讁之上密令翰林學士韋澳纂次諸州境土風物及諸利害為一書自寫而上之|{
	而上時掌翻下上疏同}
雖子弟不知也號曰處分語它日鄧州刺史薛弘宗入謝|{
	鄧州京師東南九百三十里}
出謂澳曰上處分本州事驚人澳詢之皆處分語中事也|{
	處昌呂翻分扶問翻}
澳在翰林上或遣中使宣旨草詔事有不可者澳輒曰兹事須降御札方敢施行淹留至旦上疏論之上多從之 秋七月浙東軍亂逐觀察使李訥訥遜之弟子也|{
	李遜見二百三十九卷憲宗元和十年}
性卞急|{
	杜預曰卞躁疾也音皮彦翻}
遇將士不以禮故亂作 淮南饑民多流亡節度使杜悰荒于遊宴政事不治上聞之甲午以門下侍郎同平章事崔鉉同平章事充淮南節度使丁酉以悰為太子太傅分司 九月乙亥貶李訥為朗州刺史監軍王宗景杖四十配恭陵仍詔自今戎臣失律并坐監軍以禮部侍郎沈詢為浙東觀察使詢傳師之子也|{
	傳師者沈既濟之子}
冬十一月以吏部侍郎柳仲郢為兵部侍郎充鹽鐵

轉運使有閭閻醫工劉集|{
	醫工無職于尚藥局不待詔于翰林院但以醫術自售于閭閻之間故謂之閭閻醫工}
因緣交通禁中上敕鹽鐵補場官|{
	凡銅鐵鹽場皆有官主之}
仲郢上言醫工術精宜補醫官若委務銅鹽何以課其殿最|{
	殿丁練翻}
且場官賤品非特敕所宜親臣未敢奉詔上遽批劉集宜賜絹百匹遣之它日見仲郢勞之曰|{
	批匹迷翻勞力到翻}
卿論劉集事甚佳上嘗苦不能食召醫工梁新診脈|{
	診止忍翻切捄以候驗受病之原}
治之數日良已|{
	治直之翻}
新因自陳求官上不許但敕鹽鐵使月給錢三十緡而已右威衛大將軍康季榮前為涇原節度使擅用官錢二百萬緡事覺季榮請以家財償之上以季榮有開河湟功|{
	季榮有功見上卷三年}
許之給事中封還敕書|{
	唐制凡詔敕有不便者給事中塗竄而奏還之謂之塗歸}
諫官亦上言十二月庚辰貶季榮夔州長史|{
	夔州京師南二千二百四十三里}
江西觀察使鄭祇德以其子顥尚主通顯固求散地|{
	散悉但翻}
甲午以祇德為賓客分司|{
	太子賓客分司東都}


十年春正月丁巳以御史大夫鄭朗為工部尚書同平章事 上命裴休極言時事休請早建太子上曰若建太子則朕遂為閒人|{
	孰謂唐宣宗明察吾不信也}
休不敢復言|{
	復扶又翻}
二月丙戌休以疾辭位不許 三月辛亥詔以回鶻有功于國世為昏姻|{
	有功于國謂討安史世為婚姻謂世尚公主}
稱臣奉貢此邊無警會昌中虜廷喪亂|{
	喪息浪翻}
可汗奔亡屬姦臣當軸|{
	屬之欲翻姦臣謂李德裕此大中君臣愛憎之論也}
遽加殄滅近有降者云已厖歷今為可汗尚寓安西|{
	已厖歷即厖勒以華言譯夷言語轉耳厖勒立見上卷二年}
俟其歸復牙帳當加冊命 上以京兆久不理夏五月丁卯以翰林學士工部侍郎韋澳為京兆尹 |{
	考異曰貞陵遺事東觀記皆曰帝以崔罕崔郢并敗官面除澳京兆尹按大中制集澳代罕郢代澳云罕郢并敗官誤也今從實録新紀舊紀新傳}
澳為人公直既視事豪貴歛手鄭光莊吏恣横|{
	莊吏掌主家田租者也横戶孟翻}
積年租稅不入澳執而械之上於延英問澳澳具奏其狀上曰卿何以處之|{
	處昌呂翻}
澳曰欲寘于法上曰鄭光甚愛之何如對曰陛下自内庭用臣為京兆|{
	翰林學士院在内庭}
欲以清畿甸之積弊若鄭光莊吏積年為蠧得寛重辟|{
	辟毗亦翻}
是陛下之法獨行于貧戶臣未敢奉詔上曰誠如此|{
	言韋澳所奏誠合于理}
但鄭光殢我不置|{
	此實言牽于母黨之愛殢它計翻}
卿與痛杖貸其死可乎對曰臣不敢不奉詔願聽臣且繋之俟徵足乃釋之上曰灼然可|{
	言韋澳之言灼然可行也}
朕為鄭光故撓卿法|{
	為于偽翻撓奴巧翻又奴教翻}
殊以為愧澳歸府|{
	府謂京兆府}
即杖之督租數百斛足乃以吏歸光|{
	考異曰東觀奏記曰太后為上言之上于延英問澳澳具奏本末上曰今日納租足放否澳曰尚在限内明日則不得矣上入奏太后曰韋澳不可犯也與送錢納却頃刻而租入今從柳玭續貞陵遺事}
六月戊寅以中書侍郎同平章事裴休同平章事充宣武節度使 司農卿韋厪欲求夏州節度使|{
	厪渠遴翻夏戶雅翻}
有術士知之詣厪門曰吾善醮星辰求官無不如意厪信之夜設醮具于庭術士曰請公自書官階一通既得之仰天大呼曰|{
	呼火故翻}
韋厪有異志令我祭天厪舉家拜泣曰願山人賜百口之命家之貨財珍玩盡與之邏者怪術士服鮮衣|{
	邏郎佐翻}
執以為盜術士急乃曰韋厪令我祭天我欲告之彼以家財求我耳事上聞|{
	上時掌翻}
秋九月上召厪面詰之具知其寃謂宰相曰韋厪城南甲族|{
	京城之南韋杜二族居之謂之韋曲杜曲語云城南韋杜去天尺五}
為姦人所誣勿使獄吏辱之立以術士付京兆杖死貶厪永州司馬 |{
	考異曰東觀奏記實録貶司農卿韋厪為永州司馬厪夜令術士為厭勝之術御史臺劾奏故也范攄雲谿友議曰太僕卿韋厪欲求夏州節度使云云貶潘州司馬今官名從東觀奏記及實録事取雲谿友議}
戶部侍郎判戶部駙馬都尉鄭顥|{
	唐自中世以後天下財賦皆屬戶部度支鹽鐵率以它官分判戶部侍郎判戶部乃得知戶部一司錢貨穀帛出入之事駙馬都尉尚主者為之}
營求作相甚切其父祇德與書曰聞汝已判戶部是吾必死之年又聞欲求宰相是吾必死之日也 |{
	考異曰劉崇遠金華子雜編顥既判戶部馳逐台司甚切時家君猶鎮山東聞之遺書謂顥云云按實録九年十二月顥父祇德以賓客分司金華子云鎮山東誤也}
顥懼累表辭劇務|{
	戶部之務繁劇}
冬十月乙酉以顥為祕書監上遣使詣安西鎮撫回鶻使者至靈武會回鶻可汗遣使入貢十一月辛亥冊拜為嗢祿登里羅日沒密施合俱録毗伽懷建可汗以衛尉少卿王端章充使 吏部尚書李景讓上言穆宗乃陛下兄敬宗文宗武宗乃兄之子陛下拜兄尚可拜姪可乎是使陛下不得親事七廟也宜遷四主出太廟|{
	四主謂穆敬文武四宗神主}
還代宗以下入廟詔百官議其事不決而止時人以是薄景讓|{
	薄其逢君之惡也}
敕于靈感會善二寺置戒壇僧尼應填闕者委長老僧選擇給公憑赴兩壇受戒兩京各選大德十人主其事|{
	僧之能持戒行者謂之大德宋白曰唐制諸寺有綱維有大德大德主教授}
有不堪者罷之堪者給牒遣歸本州不見戒壇公牒毋得私容仍先選舊僧尼舊僧尼無堪者乃選外人 壬辰以戶部侍郎判戶部崔慎由為工部尚書同平章事上每命相左右無知者前此一日令樞密宣旨於學士院以兵部侍郎判度支蕭鄴同平章事樞密使王歸長馬公儒覆奏鄴所判度支應罷否上以為歸長等佑之|{
	佑助也}
即手書慎由名及新命付學士院仍云落判戶部事鄴明之八世孫也|{
	明梁貞陽侯蕭淵明也唐諱淵故止曰明}
内園使李敬寔|{
	内園使亦内諸司之一五代時有内園栽接使}
遇鄭朗不避馬朗奏之上責敬寔對曰供奉官例不避上曰汝衘敕命横絶可也|{
	横度曰絶}
豈得私出而不避宰相乎命剥色配南牙|{
	禠其本色使配役南牙也}
十一年春正月丙午以御史中丞兼尚書右丞夏侯孜為戶部侍郎判戶部事先是判戶部有缺|{
	先悉薦翻}
京兆尹韋澳奏事上欲以澳補之辭曰臣比年心力衰耗難以處繁劇|{
	比毗至翻處昌呂翻}
屢就陛下乞小鎮聖恩未許上不悦及歸其甥柳玭尤之|{
	玭蒲蠲翻又蒲賓翻}
澳曰主上不與宰輔僉議私欲用我人必謂我以他岐得之|{
	岐路也}
何以自明且爾知時事浸不佳乎由吾曹貪名位所致耳丙辰以澳為河陽節度使 |{
	考異曰舊傳云十二年誤也今從實録}
玭仲郢之子也|{
	柳仲郢見上卷武宗會昌五年}
上欲幸華清宫諫官論之甚切上為之止|{
	為于偽翻}
上樂聞規諫|{
	樂音洛}
凡諫官論事門下封駮苟合于理多屈意從之得大臣章疏必焚香盥手而讀之|{
	史炤曰盥手澡手也}
二月辛巳以門下侍郎同平章事魏謩同平章事充西川節度使謩為相議事於上前它相或委曲規諷謩獨正言無所避上每歎曰謩綽有祖風|{
	謂有魏徵之風}
我心重之然竟以剛直為令狐綯所忌而出之 嶺南溪洞蠻屢為侵盜夏四月壬申以右千牛大將軍宋涯為安南邕管宣慰使五月乙巳以涯為安南經畧使容州軍亂逐經畧使王球六月癸巳以涯為容管經畧使甲午立皇子灌為衛王澭為廣王|{
	澭紆容翻又紆用翻}
秋七

月庚子以兵部侍郎判度支蕭鄴同平章事仍判度支教坊祝漢貞滑稽敏給|{
	史記索隱曰滑謂亂也稽同也以言辯捷之人言非若是}


|{
	說是若非能亂同異也崔浩云滑音骨稽流酒器也轉注吐酒終日不已言出口成章辭不窮竭若滑稽之吐酒故揚雄酒賦云鴟夷滑稽腹大如壺盡日盛酒人復借沽是也又姚察曰滑稽俳諧也滑讀如字稽音計以言諧語滑利其智計捷出故云滑稽也}
上或指物使之口占摹詠有如宿構由是寵冠諸優|{
	冠古玩翻}
一日在上前抵掌詼諧頗及外事上正色謂曰我畜養爾曹止供戲笑耳|{
	畜吁玉翻}
豈得輒預朝政邪自是疎之會其子坐贓杖死流漢貞于天德軍 |{
	考異曰實錄大中十一年七月貶嗣韓王乾裕于嶺外初伶人祝漢貞寵冠諸優復出入宫邸乾裕以金帛結之求刺史雖已納賂而未敢言至是為御史臺劾奏故貶杖漢貞流天德軍今從貞陵遺事}
樂工羅程善琵琶自武宗朝已得幸上素曉音律尤有寵程恃恩暴横以睚眦殺人|{
	横戶孟翻睚五懈翻眦士懈翻睚眦恨視也又瞋目貌顔師古曰睚舉眼也眦即眥字謂目匡睚眦言舉目相斥也}
繫京兆獄諸樂工欲為之請因上幸後苑奏樂|{
	為于偽翻}
乃設虛坐|{
	坐徂臥翻}
置琵琶而羅拜于庭且泣上問其故對曰羅程負陛下萬死然臣等惜其天下絶藝不復得奉宴遊矣|{
	復扶又翻}
上曰汝曹所惜者羅程藝朕所惜者高祖太宗法竟杖殺之 八月成德節度使王紹鼎薨紹鼎沈湎無度|{
	沈持林翻湎面善翻飲酒齊色曰湎韓詩云飲酒閉門不出客曰湎}
好登樓彈射人以為樂|{
	好呼到翻彈徒案翻又徒丹翻射而亦翻樂音洛}
衆欲逐之會病薨軍中立其弟節度副使紹懿戊寅以紹懿為成德留後 九月辛酉以太子太師盧鈞同平章事充山南西道節度使 冬十月己巳以秦成防禦使李承勛為涇原節度使承勛光弼之孫也|{
	勛與勲同}
先是吐蕃酋長尚延心以河渭二州部落來降拜武衛將軍|{
	先悉薦翻酋慈由翻長知丈翻}
承勛利其羊馬之富誘之入鳳林關|{
	河州鳳林縣北有鳳林關鳳林漢之白石縣地天寶元年以關名縣誘音酉}
居秦州之西承勛與諸將謀執延心誣云謀叛盡掠其財徙其衆于荒遠延心知之因承勛軍宴坐中謂承勛曰|{
	宴于軍中曰軍宴坐徂臥翻}
河渭二州土曠人稀因以飢疫唐人多内徙三川|{
	三川平凉川蔚茹川落門川也}
吐蕃皆遠遁於疊宕之西|{
	宕徒浪翻}
二千里間寂無人煙延心欲入見天子|{
	見賢遍翻}
請盡帥部衆分徙内地為唐百姓|{
	帥讀曰率}
使西邊永無揚塵之警其功亦不愧于張義潮矣|{
	張義潮以沙瓜等州歸唐}
承勛欲自有其功猶豫未許延心復曰延心既入朝部落内徙但惜秦州無所復恃耳|{
	復扶又翻下同}
承勛與諸將相顧默然明日諸將言于承勛曰明公首開營田置使府擁萬兵仰給度支|{
	使府秦成防禦使府仰牛向翻}
將士無戰守之勞有耕市之利|{
	耕謂營田之利市謂互市之利}
若從延心之謀則西陲無事朝廷必罷使府省戍兵還以秦州隸鳳翔吾屬無所復望矣承勛以為然即奏延心為河渭都遊弈使使統其衆居之|{
	史言唐之邊鎮自將帥至於偏禆詳於身謀畧於國事故夷人窺見其肺肝亦得行其自全之謀 考異曰此事出補國史按張義潮以十一州降河渭已在其間今延心復以河渭降者義潮所帥者漢民延心所帥者蕃族也又補國史不云延心以何年月降新傳但云張義潮降其後河渭州虜將尚延心以國破亡亦獻款秦州刺史高駢誘降延心及渾末部萬帳遂收二州拜延心武衛將軍駢收鳳林關以延心為河渭等州都遊奕使按舊傳高駢懿宗時始為秦州刺史新傳誤也今從補國史因承勛移鎮涇原并延心事置于此}
中書侍郎同平章事鄭朗以疾辭位壬申以朗為太子太師 上晩節頗好神仙遣中使迎道士軒轅集于羅浮山|{
	流軒轅集見上卷會昌六年羅浮山在循州博羅縣西北三十里漢志曰浮山自會稽浮來博于羅山故曰博羅山亦曰羅浮山}
王端章冊立回鶻可汗道為黑車子所塞不至而還|{
	王端章去年十一月使回鶻還從宣翻又如字}
辛卯貶端章賀州司馬 十一月壬寅以成德軍留後王紹懿為節度使 十二月蕭鄴罷判度支

十二年春正月以康王傅分司王式為安南都護經畧使|{
	康王汶上子也 考異曰舊紀式為安南在二月今從實録}
式有才畧至交趾樹芀木為柵可支數十年|{
	史炤曰芀都聊切又音調余按廣韻芀都聊切又音調者葦華也其字從草從刀又類篇有從艸從力者香菜也歷得切嘗見一書從艸從力者讀與棘同棘羊矢棗也此木可以支久}
深塹其外泄城中水塹外植竹寇不能冒|{
	范成大桂海虞衡志竻竹刺竹也芒刺森然廣東新州素無城桂林人黄齊守郡始以此竹植之羔豚不能徑號竹城至今以為利傳聞交趾外城亦是此竹正王式所植者也竻盧得翻}
選教士卒甚銳頃之南蠻大至|{
	南蠻謂南詔蠻也}
去交趾半日程|{
	唐制凡陸行之程馬日七十里步及驢日五十里車三十里水行之程舟之重者泝河日三十里江四十里餘水四十五里空舟泝河四十里江五十里餘水六十里沿流之舟則輕重同制河日一百五十里江一百里餘水七十里}
式意思安閒|{
	思相吏翻}
遣譯諭之中其要害|{
	中竹仲翻要害謂諭之以守禦之事于我為要于彼為害者}
蠻一夕引去遣人謝曰我自執叛獠耳非為寇也安南都校羅行恭久專府政|{
	都校猶言都將也獠魯皓翻校戶校翻}
麾下精兵二千都護中軍纔羸兵數百式至杖其背黜於邊徼|{
	羸倫為翻徼吉弔翻}
初戶部侍郎判度支劉瑑|{
	瑑柱兖翻}
為翰林學士上器重之時為河東節度使手詔徵入朝瑑奏河東外人始知之戊午以瑑同平章事 |{
	考異曰東觀奏記曰十一年上手詔追之既至拜戶部侍郎判度支十二月十七日次對上以御按歷日甘瑑令于下旬擇一吉日瑑不諭上旨上曰但擇一拜官日即得瑑跪奏二十五日甚佳上笑曰此日命卿為相秘世無知者高湜為鳳翔從事湜即瑑舊僚也二十四日辭瑑于宣平里私第湜曰竊度旬時必副具瞻之望瑑笑曰來日具瞻何旬時也湜不敢詰旦果爰立矣始以此事洩于湜實錄瑑傳曰明年正月十七日次對帝以歷日付瑑令擇吉日瑑跪奏二十五日今從之}
瑑仁軌之五世孫也|{
	劉仁軌事高宗武后出入將相}
瑑與崔慎由議政于上前慎由曰惟當甄别品流|{
	甄稽延翻别彼列翻}
上酬萬一瑑曰昔王夷甫祖尚浮華|{
	晉王衍字夷甫}
妄分流品致中原丘墟今盛明之朝當循名責實使百官各稱其職|{
	朝直遥翻稱尺證翻}
而遽以品流為先臣未知致理之日慎由無以對軒轅集至長安上召入禁中問曰長生可學乎對曰

王者屏欲而崇德|{
	屏必郢翻}
則自然受大遐福何處更求長生留數月堅求還山乃遣之|{
	軒轅集之求還懲會昌末年之事也}
二月甲子朔罷公卿朝拜光陵及忌日行香悉移宫人於諸陵|{
	以陳弘志弑逆之罪歸穆宗也唐初皇帝有謁陵之禮天子不躬謁則以太常卿行陵高宗顯慶五年詔歲春秋季一巡宜以三公行陵太常少卿貳之太常給鹵簿仍著于令始貞觀禮歲以春秋仲月巡陵至武后時乃以四季月生日忌日遣使詣陵起居貞元四年國子祭酒包佶言歲二月八月公卿朝拜諸陵陵臺所由導至陵下禮畧無以盡恭于是太常約舊禮草定其儀公卿衆官以次奉行朝拜而還忌日行香即詣陵起居之禮也又有忌日詣僧寺行香之禮宋白曰唐制國忌行香初只行于京城寺觀貞元五年八月敕天下諸上州並宜國忌日准式行香之禮凡諸帝升遐宫人無子者悉遣詣山陵供奉朝夕具盥櫛治衾枕事死如事生朝直遥翻}
戊辰以中書侍郎同平章事崔慎由為東川節度使 |{
	考異曰唐闕史曰丞相太保崔公一日備顧問于便殿上欲御樓肆赦太保奏云云後旬日罷知政事舊傳初慎由與肅鄴同在翰林情不相洽及慎由作相罷鄴學士俄而鄴自度支平章事恩顧甚隆鄴引瑑同知政事遂出慎由東川東觀奏記劉瑑既入相與慎由議政于上前慎由曰唯當甄别品流瑑云云慎由不能對因此恩澤浸衰尋罷相為東川節度使削平章事今從唐闕史}
上欲御樓肆赦|{
	唐初天子居西内肆赦率御承天門樓自高宗以後天子居東内肆赦率御丹鳳門樓}
令狐綯曰御樓所費甚廣事須有名且赦不可數|{
	唐制凡御樓肆赦六軍十二衛皆有恩賚故云所費甚廣劉温叟曰故事非肆大眚不御樓軍庶皆有恩給數所角翻}
上不悦曰遣朕于何得名慎由曰陛下未建儲宫四海屬望|{
	屬之欲翻}
若舉此禮雖郊祀亦可况于御樓時上餌方士藥已覺躁渇而外人未知疑忌方深聞之俛首不復言|{
	史言宣宗不早定國本使王宗實得以立長而竊定策之功復扶又翻}
旬日慎由罷相 勃海王彞震卒癸未立其弟䖍晃為勃海王 夏四月以右街使駙馬都尉劉異為邠寧節度使|{
	左右街使與左右金吾將軍掌分察六街徼巡}
異尚安平公主上妹也庚子嶺南都將王令寰作亂囚節度使楊蘇州人也 戊申以兵部侍郎鹽鐵轉運使夏侯孜同平章事 五月丙寅工部尚書同平章事劉瑑薨瑑病篤猶手疏論事上甚惜之 以右金吾大將軍李燧為嶺南節度使已命中使賜之節給事中蕭倣封還制書上方奏樂不暇别召中使使優人追之節及燧門而返 |{
	考異曰此出東觀奏記而燧不知以何時除嶺南按實錄大中九年韋曙除嶺南節度使今年正月薨楊代之三月蕭倣言柳珪四月燧自司農卿為右金吾大將軍五月聞嶺南亂蓋於此除燧嶺南而倣封還以燧為非定亂之才故也今置於此}
倣俛之從父弟也|{
	蕭俛穆宗長慶初為相}
辛巳以涇原節度使李承勛為嶺南節度使鄰道兵討亂者平之是日湖南軍亂都將石載順等逐觀察使韓悰殺都押牙王桂直悰待將士不以禮故及于難|{
	難乃旦翻}
六月丙申江西軍亂都將毛鶴逐觀察使鄭憲 初安南都護李涿 |{
	考異曰實録或作琢或作涿樊綽蠻書亦作涿實錄及新書皆有李琢傳聽之子也大中三年自洺州刺史除義昌節度使九年九月自金吾將軍除平盧節度使不云曾為安南都護按都護位卑琢既為義昌節度使不應為都護疑作都護者别一李涿非聽子也}
為政貪暴強市蠻中馬牛一頭止與鹽一斗又殺蠻酋杜存誠羣蠻怨怒導南詔侵盜邊境 |{
	考異曰舊紀琢侵刻獠民羣獠引林邑蠻攻安南府按蠻書寇安南者南詔非林邑也}
峯州有林西原|{
	峯州在安南西北林西原當又在峯州西}
舊有防冬兵六千|{
	南方炎瘴至冬瘴輕蠻乘此時為寇故置防冬兵}
其旁七綰洞蠻其酋長曰李由獨常助中國戍守輸租賦|{
	酋慈由翻長知丈翻}
知峯州者言于涿請罷戍兵專委由獨防遏于是由獨勢孤不能自立南詔拓東節度使以書誘之以甥妻其子補柘東押牙|{
	妻七細翻交趾在南詔東南詔于東境置拓東節度言將開拓東境也又新志自戎州開邊縣七十里至曲州又一千九百七十五里至柘東城柘從木又曰柘東城有諸葛亮石刻文曰碑即仆蠻為漢奴夷畏誓常以石榰梧}
由獨遂帥其衆臣于南詔|{
	帥讀曰率}
自是安南始有蠻患是月蠻寇安南 |{
	考異曰實録無涿除安南年月蠻書云大中八年安南都護擅罷林西原防冬戍卒洞主李由獨等七綰首領被蠻誘引復為親情日往月來漸遭侵軼又曰桃花蠻本屬由獨管轄亦為界上戍卒自大中八年被峯州知州官申文狀與李涿請罷防冬將健六千人不要味真登等州界上防遏其由獨兄弟力不禁被蠻柘東節度使與書信將外甥嫁與由獨小男補拓東押衙自此後七綰洞悉為蠻收管舊紀咸通四年十一月劉蛻等言令狐綯受李涿賄除安南生蠻寇實錄咸通二年六月詔如聞李涿在安南日殺害杜存誠貪殘頗甚致令溪洞懷怨據此則本因李涿貪暴無謀以致蠻寇明矣然則大中八年至十一年舊紀實錄不言蠻為邊患蓋但時於邊境小有鈔盜未敢犯州縣至此寇安南而舊紀實録始載之又不知此寇安南即鄭言平剡錄所謂至錦田步時非也}
秋七月丙寅宣州都將康全泰作亂逐觀察使鄭薰薰奔楊州丁卯右補闕内供奉張潛上疏以為藩府代移之際皆奏倉庫蓄積之數以羨餘多為課績|{
	羨戈線翻}
朝廷亦因而甄奬|{
	甄稽延翻}
竊惟藩府財賦所出有常苟非賦歛過差及停廢將士減削衣糧則羨餘何從而致比來南方諸鎮數有不寧皆此故也一朝有變所蓄之財悉遭剽掠|{
	比毗至翻數所角翻剽匹妙翻}
又兵致討費用百倍然則朝廷竟有何利乞自今藩府長吏不增賦歛不減糧賜獨節遊宴省浮費能致羨餘者然後賞之上嘉納之 容管奏都虞侯來正謀叛經畧使宋涯捕斬之初忠武軍精兵皆以黄冒首號黄頭軍李承勛以百人定嶺南宋涯使麾下效其服裝亦定容州安南有惡民屢為亂聞之驚曰黄頭軍度海求襲我矣|{
	求當作來}
相與夜圍交趾城鼓譟願送都護北歸我須此城禦黄頭軍王式方食或勸出避之式曰吾足一動則城潰矣徐食畢擐甲|{
	擐音宦}
率左右登城建大將旗坐而責之亂者反走明日悉捕誅之有杜守澄者自齊梁以來擁衆據溪洞不可制|{
	言杜守澄之先自齊梁以來不可制也}
式離間其親黨守澄走死|{
	間古莧翻參考本末則杜守澄杜存誠之子也存誠後為安南都護李鄠所殺前又云李涿所殺未知孰是}
安南飢亂相繼六年無上供|{
	上供者錢帛之輸京師以供上用者也}
軍中無犒賞式始修貢賦饗將士占城真臘皆復通使|{
	占城在大海中西直三佛齊南與崖州對岸真臘一名吉蔑本扶南屬國去長安二萬八百里東距車渠西屬驃南濱海北與道明接東北抵驩州}
淮南節度使崔鉉奏已出兵討宣州賊八月甲午以鉉兼宣歙觀察使己亥以宋州刺史温璋為宣州團練使璋造之子也|{
	温造見二百四十四卷文宗太和四年}
河南北淮南大水徐泗水深五丈|{
	深式禁翻}
漂沒數萬家 冬十月建州刺史于延陵入辭上曰建州去京師幾何對曰八千里|{
	舊志建州在長安東南四千九百三十五里}
上曰卿到彼為政善惡朕皆知之勿謂其遠此階前則萬里也|{
	遠于願翻}
卿知之乎延陵悸懾失緒|{
	悸其季翻懾之涉翻絲端曰緒言延陵悸懾應對錯亂失其端緒}
上撫而遣之到官竟以不職貶復州司馬|{
	復州京師東南一千八百里}
令狐綯擬李遠杭州刺史|{
	吳分餘杭立臨水縣晉改臨水為臨安陳為錢塘郡隋置杭州自臨安移居錢塘尋移州于柳浦西依山築城京師東南三千五百五十六里}
上曰吾聞遠詩云長日惟消一局棊安能理人綯曰詩人託此為高興耳|{
	興許應翻}
未必實然上曰且令往試觀之上詔刺史毋得外徙必令至京師面察其能否然後除之令狐綯嘗徙其故人為鄰州刺史便道之官上見其謝上表|{
	謝上時掌翻今諸州守臣有謝到任表}
以問綯對曰以其道近省送迎耳上曰朕以刺史多非其人為百姓害故欲一一見之訪問其所施設知其優劣以行黜陟而詔命既行直廢格不用|{
	格音閣}
宰相可畏有權|{
	如令狐綯之欺蔽罷其相而罪之可也若任之為相而畏其有權則宰相取充位而已}
時方寒綯汗透重裘|{
	重直龍翻}
上臨朝接對羣臣如賓客|{
	朝直遥翻}
雖左右近習未嘗見其有惰容每宰相奏事旁無一人立者威嚴不可仰視奏事畢忽怡然曰可以閒語矣因問閭閻細事或談宫中遊宴無所不至一刻許|{
	漏上一刻許也}
復整容曰卿輩善為之朕常恐卿輩負朕後日不復得相見|{
	復扶又翻}
乃起入宫令狐綯謂人曰吾十年秉政|{
	大中四年令狐綯為相至懿宗即位方罷}
最承恩遇然每延英奏事未嘗不汗霑衣也初山南東道節度使徐商以封疆險闊素多盜賊選

精兵數百人别置營訓練號捕盜將及湖南逐帥|{
	事見上五月將即亮翻帥所類翻}
詔商討之商遣捕盜將二百人討平之崔鉉奏克宣州斬康全泰及其黨四百餘人 上以光祿卿韋宙父丹有惠政于江西|{
	事見三年}
以宙為江西觀察使鄰道兵以討毛鶴 崔鉉以宣州已平辭宣歙觀察使十一月戊寅以温璋為宣歙觀察使 兵部侍郎判戶部蔣伸從容言于上曰近日官頗易得人思徼幸|{
	從千容翻易以䜴翻徼堅堯翻}
上警曰如此則亂矣對曰亂則未亂但徼幸者多亂亦非難上稱歎再三伸起上三留之曰異日不復得獨對卿矣|{
	次對官獨坐宰相皆同入對復扶又翻}
伸不諭|{
	不諭者不解上旨}
十二月甲寅以伸同平章事 韋宙奏克洪州斬毛鶴及其黨五百餘人宙過襄州徐商遣都將韓季友帥捕盜將從行宙至江州季友請夜帥其衆自陸道間行比明至洪州|{
	帥讀曰率間古莧翻比必利翻及也江州西南至洪州一百九十五里}
州人不知即日討平之宙奏留捕盜將二百人于江西以季友為都虞候

十三年春正月戊午朔赦天下 三月割河東雲蔚朔三州隸大同軍|{
	時置大同軍節度冶雲州宋白曰朔州東至蔚州四百六十里東北至故雲州二百六十里今雲州治雲中本古平城也}
夏四月辛卯以校書郎于琮為左拾遺内供奉初上欲以琮尚永福公主既而中寢宰相請其故上曰朕近與此女子會食對朕輒折匕筯性情如是豈可為士大夫妻乃更命悰尚廣德公主|{
	折而設翻更工衡翻}
二公主皆上女琮敖之子也 武寧節度使康季榮不卹士卒士卒譟而逐之上以左金吾大將軍田牟嘗鎮徐州有能名|{
	新書曰牟三為武寧帥皆有能名按武宗會昌四年田牟方為天德軍使則其初除武寧必在會昌之間而史不記其歲月}
復以為武寧節度使一方遂安貶季榮於嶺南六月癸巳封憲宗子惕為彭王|{
	惕它歷翻}
初上長子鄆王温無寵居十六宅餘子皆居禁中夔

王滋第三子也上愛之欲以為嗣為其非次|{
	為其于偽翻}
故久不建東宫上餌醫官李玄伯道士虞紫芝山人王樂藥疽于背八月疽甚宰相及朝士皆不得見|{
	見賢遍翻}
上密以夔王屬樞密使王歸長馬公儒宣徽南院使王居方使立之|{
	屬之欲翻}
三人及右軍中尉王茂玄皆上平日所厚也獨左軍中尉王宗實素不同心三人相與謀出宗實為淮南監軍宗實已受敕于宣化門外將自銀臺門出左軍副使亓元實 |{
	考異曰或作邢元實今從東觀奏記懿宗實錄亓苦堅翻海陵本作亓渠之切姓也}
謂宗實曰聖人不豫踰月中尉止隔門起居|{
	中尉謂王宗實}
今日除改未可辨也何不見聖人而出宗實感寤復入諸門已踵故事增人守捉矣亓元實翼導宗實直至寢殿上已崩|{
	年五十}
東首環泣矣|{
	首式又翻環音宦}
宗實叱歸長等責以矯詔皆捧足乞命乃遣宣徽北院使齊元簡迎鄆王壬辰下詔立鄆王為皇太子權句當軍國政事仍更名漼|{
	句古侯翻當丁浪翻鄆音運更工衡翻漼七罪翻}
收歸長公儒居方皆殺之癸巳宣遺制以令狐綯攝冢宰宣宗性明察沈斷|{
	沈持林翻斷丁管翻 考異曰續貞陵遺事曰越守嘗進女樂有絶色者上初悦之數月錫賚盈積一旦晨興忽不樂曰玄宗只一楊妃天下至今未平我豈敢忘乃召美人曰應留汝不得左右或奏可以放還上曰放還我必思之可命賜酒一杯此太不近人情恐譽之太過今不取}
用法無私從諫如流重惜官賞恭謹節儉惠愛民物故大中之政訖于唐亡人思詠之謂之小太宗|{
	衛嗣君之聰察不足以延衛唐宣宗之聰察不足以延唐}
丙申懿宗即位癸卯尊皇太后為太皇太后以王宗實為驃騎上將軍李玄伯虞紫芝王樂皆伏誅 |{
	考異曰東觀奏記畢諴在翰林上恩顧特異許用為相深為丞相令狐綯緩其入相之謀諴思有以結綯在北門求得絶色非人世所有盛飾珠翠專使獻綯綯一見之心動謂其子曰畢太原于吾無分今以是餌吾將傾吾家族也諴又瀝血輸啓事于綯綯終不納乃命邸貨之東頭醫官李玄伯上所狎昵者以錢七十萬致于家乃舍正堂坐之玄伯夫妻執賤役以事焉踰月盡得其歡心矣乃進于上上一見惑之寵冠六宫玄伯燒伏火丹砂連進以市恩澤致上瘡疾皆玄伯之罪也懿宗即位玄伯與山人王岳道士虞紫芝俱棄市今從實録}
九月追尊上母晁昭容為元昭皇太后 加魏博節度使何弘敬兼中書令 |{
	考異曰東觀奏記大中十三年魏博何弘敬就加中書令據實録二月弘敬加太傅此月乃加中書令在懿宗即位後東觀奏記誤也}
幽州節度使張允伸同平章事 冬十月辛卯赦天下 十一月戊午以門下侍郎同平章事蕭鄴同平章事充荆南節度使 十二月甲申以翰林學士承旨兵部侍郎杜審權同平章事審權元頴之弟子也|{
	杜元穎穆宗長慶初為相後以帥西川致寇敗}
浙東賊帥裘甫攻陷象山|{
	孫愐曰裘本仇氏避仇改作裘或曰衛大夫柳莊邑于裘氏神龍元年分寧海及鄮置象山縣屬台州廣德二年度屬明州帥所類翻考異曰實録作仇甫按平剡録作裘甫今從之}
官軍屢敗明州城門晝閉進逼剡縣|{
	剡漢古縣唐屬越州九域志在州東南一百八十里}
有衆百人浙東騷動觀察使鄭祇德遣討擊副使劉勍|{
	勍渠京翻}
副將范居植將兵三百合台州軍共討之 司空門下侍郎同平章事令狐綯執政歲久忌勝己者中外側目其子滈頗招權受賄|{
	滈胡老翻}
宣宗既崩言事者競攻其短丁酉以綯同平章事充河中節度使以前荆南節度使同平章事白敏中守司徒兼門下侍郎同平章事 初韋臯在西川開青溪道以通羣蠻|{
	青溪道即清溪關路}
使由蜀入貢又選羣蠻子弟聚之成都教以書數欲以慰悦羈縻之業成則去復以它子弟繼之|{
	復扶又翻}
如是五十年羣蠻子弟學于成都者殆以千數軍府頗厭于稟給又蠻使入貢利于賜與所從傔人浸多|{
	傔苦念翻}
杜悰為西川節度使奏請節減其數詔從之南詔豐祐怒其賀冬使者留表付嶲州而還又索習學子弟移牒不遜|{
	還從宣翻又如字索山客翻}
自是入貢不時頗擾邊境會宣宗崩遣中使告哀時南詔豐祐適卒子酋龍立|{
	酋慈秋翻}
怒曰我國亦有喪朝廷不弔祭又詔書乃賜故王遂置使者于外館禮遇甚薄使者還具以狀聞上以酋龍不遣使來告喪又名近玄宗諱|{
	龍字近玄宗諱}
遂不行冊禮酋龍乃自稱皇帝國號大禮|{
	至今雲南國號大理}
改元建極遣兵陷播州|{
	為南詔攻蜀攻交趾張本考異曰舊紀實錄今年皆無陷播州事惟新紀有之實錄咸通六年三月盧潘奏云大中十三年南蠻陷播}


|{
	州補國史曰雲南自大中初朝貢使及西川質子人數漸多節度使奏請釐革減省有詔許之錄詔報雲南雲南回牒不遜新南詔傳曰朝貢歲至從者多杜悰自西川入朝表無多内蠻傔豐祐怒即慢言索質子蓋謂蠻子弟學成都者也按杜悰以咸通二年七月入朝而豐祐大中十三年已死則建議減蠻傔者必非悰入朝後事新傳誤也}


資治通鑑卷二百四十九
