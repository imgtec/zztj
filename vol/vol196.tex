\chapter{資治通鑑卷一百九十六}
宋 司馬光 撰

胡三省 音註

唐紀十二|{
	起重光赤奮若盡昭陽單閼三月凡二年有奇}


太宗文武大聖大廣孝皇帝中之中

貞觀十五年春正月甲戍以吐蕃禄東贊為右衛大將軍|{
	觀古玩翻吐從暾入聲}
上嘉禄東贊善應對以瑯邪公主外孫段氏妻之|{
	妻七細翻}
辭曰臣國中自有婦父母所聘不可棄也且贊普未得謁公主陪臣何敢先娶上益賢之然欲撫以厚恩竟不從其志|{
	史言夷狄之人猶能以禮自處而中國乃不能以禮處之}
丁丑命禮部尚書江夏王道宗持節送文成公主于吐蕃|{
	尚辰羊翻夏戶雅翻吐從暾人聲}
贊普大喜見道宗盡子壻禮慕中國衣服儀衛之美為公主别築城郭宫室而處之自服紈綺以見公主其國人皆以赭塗面公主惡之|{
	為于偽翻處昌呂翻惡烏路翻}
贊普下令禁之亦漸革其猜暴之性遣子弟入國學受詩書 乙亥突厥候利苾可汗始帥部落濟河|{
	前年受詔今始濟河厥九勿翻苾毗必翻可從刋入聲汗音寒帥讀曰率}
建牙於故定襄城|{
	杜佑曰故定襄城在朔州馬邑郡北三百許里}
有戶三萬勝兵四萬|{
	勝音升}
馬九萬匹仍奏言臣非分蒙恩為部落之長|{
	分扶問翻長知兩翻}
願子子孫孫為國家一犬守吠北門若薛延陀侵逼請從家屬入長城詔許之 上將幸洛陽命皇太子監國|{
	監古銜翻}
留右僕射高士亷輔之|{
	射寅謝翻}
辛巳行及温湯|{
	新豐有驪山温湯華州有温湯府}
衛士崔卿刁文懿憚於行役冀上驚而止乃夜射行宫|{
	射而亦翻}
矢及寢庭者五皆以大逆論|{
	十惡二曰謀大逆注云為謀毁宗廟山陵及宫闕刑統議曰此條之人于紀犯順違道悖德逆莫大焉故曰大逆以大逆論者未是犯大逆正條以其干紀犯順以大逆論罪}
三月戊辰幸襄城宫地既煩熱復多毒蛇|{
	復扶又翻}
庚午罷襄城宫分賜百姓免閻立德官|{
	營襄城宫見上卷上年}
夏四月辛卯朔詔以來年二月有事于泰山上以近世隂陽雜書訛偽尤多命太常博士呂才|{
	漢叔孫通為博士属太常隋唐最為清選太常博士從七品上掌五禮之儀式本先王之法制適變隨時而損益焉}
與諸術士刋定可行者凡四十七卷己酉書成上之|{
	上時掌翻}
才皆為之敘質以經史其序宅經以為近世巫覡|{
	覡他狄翻}
妄分五姓如張王為商武庾為羽似取諧韻至於以柳為宫以趙為角又復不類或同出一姓分屬宫商或複姓數字莫辨徵羽此則事不稽古義理乖僻者也|{
	近世相傳以字學分五音只在唇舌齒調之舌居中者為宫口開張者為商舌縮却者為角舌拄齒者為徵唇撮聚者為羽隂陽家以五姓分屬五音說正如此徵陟里翻}
敘禄命以為禄命之書多言或中|{
	中竹仲翻}
人乃信之然長平阬卒未聞共犯三刑|{
	長平之戰死者四十五萬人三刑寅刑己巳刑申申刑寅丑刑戍戌刑未未刑丑子刑卯卯刑子又辰辰午午酉酉亥亥謂之自刑}
南陽貴士何必俱當六合|{
	子與丑合寅與亥合卯與戌合辰與酉合巳與申合午與未合漢光武中興南陽人士多貴}
今亦有同年同禄而貴賤懸殊共命共胎而壽夭更異|{
	夭於紹翻}
按魯莊公法應貧賤又尫弱短陋|{
	尫烏黄翻}
惟得長壽秦始皇法無官爵縱得禄少奴婢為人無始有終|{
	少詩紹翻}
漢武帝後魏孝文帝皆法無官爵宋武帝禄與命並當空亡|{
	甲巳申酉乙庚午未丙辛辰巳丁壬寅卯戊癸子丑戊亥謂之截路空亡甲子旬戍亥甲戍旬申酉甲申旬午未甲午旬辰巳甲辰旬寅卯甲寅旬子丑謂之旬中空亡}
惟宜長子雖有次子法當早夭|{
	長知兩翻夭於紹翻下壽夭同}
此皆禄命不驗之著明者也其叙葬以為孝經云卜其宅兆而安厝之蓋以窀穸既終|{
	杜預曰窀厚也穸夜也厚夜猶言長夜窀株倫類}
永安體魄而朝市遷變泉石交侵不可前知故謀之龜筮近歲或選年月或相墓田|{
	朝直遥翻相息亮翻}
以為一事失所禍及死生按禮天子諸侯大夫葬皆有月數是古人不擇年月也|{
	古者天子七月而葬同軌畢至諸侯五月同盟至大夫三月同位至}
春秋九月丁巳葬定公雨不克葬戊午日下昃乃克葬是不擇日也鄭葬簡公司墓之室當路毁之則朝而窆|{
	窆必驗翻}
不毁則日中而窆子產不毁是不擇時也古之葬者皆於國都之北兆域有常處是不擇地也今葬書以為子孫富貴貧賤壽夭皆因卜葬所致夫子文為令尹而三已柳下惠為士師而三黜計其丘隴未嘗改移而野俗無識妖巫妄言遂於擗踊之際擇葬地以希官爵|{
	夫音扶妖於驕翻擗頻亦翻}
荼毒之秋選葬時以規財利或云辰日不可哭泣遂莞爾而對弔客或云同属忌於臨壙遂吉服不送其親傷教敗禮莫斯為甚術士皆惡其言|{
	敗補邁翻惡烏路翻}
而識者皆以為確論 丁巳果毅都尉席君買帥精騎百二十襲擊吐谷渾丞相宣王破之斬其兄弟三人|{
	帥讀曰率騎奇寄翻吐從暾入聲谷音浴相息亮翻 考異曰舊傳云鄯州刺史杜鳳舉與威信王合兵擊丞相王破之殺其兄弟三人今從實録}
初丞相宣王專國政隂謀襲弘化公主|{
	帝以宗室女為弘化公主下嫁吐谷渾}
劫其王諾曷鉢奔吐蕃諾曷鉢聞之輕騎奔鄯善城|{
	隋焬帝破吐谷渾置四郡鄯善郡治鄯善城即古之樓蘭城騎奇寄翻鄯時戰翻}
其臣威信王以兵迎之故君買為之討誅宣王|{
	為于偽翻}
國人猶驚擾遣戶部尚書唐儉等慰撫之|{
	尚辰羊翻}
五月壬申并州父老詣闕請上封泰山畢還幸晉陽上許之|{
	并卑名翻}
丙子百濟來告其王扶餘璋之喪遣使冊命其嗣子

義慈|{
	使疏吏翻嗣祥吏翻}
己酉有星孛于太微太史令薛頤上言未可東封|{
	孛蒲内翻上時掌翻}
辛亥起居郎褚遂良亦言之丙辰詔罷封襌 太子詹事于志寧遭母喪尋起復就職|{
	按會要武德年制文官遭父母喪聼去職起復者起之於苫塊之中而復其官職也亦謂之奪情}
太子治宫室妨農功又好鄭衛之樂|{
	治直之翻好呼到翻}
志寧諫不聼又寵昵宦官常在左右|{
	昵尼質翻}
志寧上書以為自易牙以來宦官覆亡國家者非一今殿下親寵此属使陵易衣冠不可長也|{
	上時掌翻下同易以䜴翻長竹兩翻}
太子役使司馭等半歲不許分番|{
	太僕寺典廐署有執馭一百人舊番上二宫六典太子僕寺有廐牧署有翼馭十五人駕士三十人}
又私引突厥逹哥友入宫|{
	新書作逹哥支}
志寧上書切諫太子大怒遣刺客張思政紇干承基殺之|{
	紇下沒翻}
二人入其第見志寧寢處苫塊|{
	孔穎逹曰寢苫枕塊謂孝子居於廬中寢臥於苫頭枕於塊處昌呂翻}
意不忍殺而止 西突厥沙鉢羅葉護可汗數遣使入貢|{
	數所角翻下勿數同使疏吏翻下同}
秋七月甲戍命左領軍將軍張大師持節即其所號立為可汗賜以鼓纛|{
	纛徒到翻}
上又命使者多齎金帛歷諸國市良馬魏徵諫曰可汗位未定而先市馬彼必以為陛下志在市馬以立可汗為名耳使可汗得立荷德必淺|{
	荷下可翻}
若不得立為怨實深諸國聞之亦輕中國市或不得得亦非美苟能使彼安寧則諸國之馬不求自至矣上欣然止之乙毗咄陸可汗與沙鉢羅葉護互相攻乙毗咄陸浸彊大西域諸國多附之未幾乙毗咄陸使石國吐屯擊沙鉢羅葉護擒之以歸殺之|{
	吐屯突厥官名使分主諸國沙鉢羅葉護立見上卷十三年幾居豈翻}
丙子上指殿屋謂侍臣曰治天下如建此屋|{
	治直之翻}
營構既成勿數改移更易一榱|{
	榱所追翻屋橑秦名為屋椽周謂之榱魯謂之桷}
正一瓦踐履動揺必有所損若慕奇功變法度不恒其德勞擾實多|{
	恒戶登翻}
上遣職方郎中陳大德使高麗|{
	職方掌天下地圖及城隍鎮戍烽候之數辯其邦國之遠近及四夷之歸化凡五方之區域都邑之廢置疆場之爭訟舉而正之使疏吏翻麗力知翻}
八月己亥自高麗還大德初入其境欲知山川風俗所至城邑以綾綺遺其守者曰吾雅好山水|{
	遺于季翻好呼到翻}
此有勝處吾欲觀之守者喜導之遊歷無所不至往往見中國人自云家在某郡隋末從軍沒于高麗高麗妻以游女|{
	妻七細翻}
與高麗錯居殆將半矣因問親戚存沒大德紿之曰皆無恙|{
	紿蕩亥翻恙余亮翻}
咸涕泣相告數日後隋人望之而哭者徧於郊野大德言於上曰其國聞高昌亡大懼館候之勤加于常數上曰高麗本四郡地耳|{
	漢武帝置臨屯真番樂浪玄菟四郡高麗有其地}
吾發卒數萬攻遼東彼必傾國救之别遣舟師出東萊自海道趨平壤|{
	趨七喻翻}
水陸合勢取之不難但山東州縣彫瘵未復吾不欲勞之耳|{
	觀帝此言已冇取高麗之心瘵則界翻}
乙巳上謂侍臣曰朕有二喜一懼比年豐稔|{
	比毗至翻}
長安斗粟直三四錢一喜也北虜久服邊鄙無虞二喜也治安則驕侈易生|{
	治直吏翻易以豉翻}
驕侈則危亡立至此一懼也 冬十月辛卯上校獵伊闕壬辰幸嵩陽|{
	伊闕縣舊曰新城隋開皇十八年更名有伊闕嵩陽縣舊曰潁陽隋開皇六年改曰武林十八年改曰輪氏大業元年改曰嵩陽有嵩高山並属洛州}
辛丑還宫 并州大都督長史李世勣在州十六年令行禁止民夷懷服上曰隋煬帝勞百姓築長城以備突厥卒無所益|{
	卒子恤翻}
朕唯置李世勣於晉陽而邊塵不驚其為長城豈不壯哉十一月庚申以世勣為兵部尚書 壬申車駕西歸長安薛延陀真珠可汗聞上將東封謂其下曰天子封泰

山士馬皆從|{
	從才用翻}
邊境必虛我以此時取思摩如拉朽耳|{
	拉盧合翻}
乃命其子大度設發同羅僕骨迴紇靺鞨霫等兵|{
	紇下沒翻靺鞨音末曷霤而立翻}
合二十萬度漠南屯白道川據善陽嶺以擊突厥|{
	善陽嶺在朔州善陽縣北}
俟利苾可汗不能禦帥部落入長城保朔州遣使告急|{
	苾毗必翻帥讀曰率下同使疏吏翻下同}
癸酉上命營州都督張儉帥所部騎兵及奚霫契丹壓其東境以兵部尚書李世勣為朔州道行軍總管將兵六萬騎千二百屯羽方|{
	騎奇寄翻下同羽方新書作朔州}
右衛大將軍李大亮為靈州道行軍總管將兵四萬騎五千屯靈武|{
	靈武縣属靈州靈武郡將兵即亮翻}
右屯衛大將軍張士貴將兵一萬七千為慶州道行軍總管出雲中凉州都督李襲譽為凉州道行軍總管出其西諸將辭行|{
	將即亮翻}
上戒之曰薛延陀負其疆盛踰漠而南行數千里馬已疲瘦凡用兵之道見利速進不利速退薛延陀不能掩思摩不備急擊之思摩入長城又不速退吾已勑思摩燒薙秋草|{
	薙他計翻耘除也}
彼糧糗日盡野無所獲頃偵者來云其馬齧林木枝皮略盡卿等當與思摩共為掎角|{
	糗去久翻偵丑鄭翻倚居蟻翻}
不須速戰俟其將退一時奮擊破之必矣 十二月戊子車駕至京師 己亥薛延陀遣使入見請與突厥和親甲辰李世勣敗薛延陀於諾真水|{
	出雲中古城西北行四百許里至諾真水見賢遍翻敗補邁翻}
初薛延陀擊西突厥沙鉢羅及阿史那社爾皆以步戰取勝及將入寇乃大教步戰使五人為伍一人執馬四人前戰戰勝則授以馬追奔於是大度設將三萬騎逼長城欲擊突厥而思摩已走知不可得遣人登城罵之會李世勣引唐兵至塵埃漲天大度設懼將其衆自赤柯濼北走|{
	將即亮翻騎奇寄翻下同濼匹各翻自淮以北率以積水處為濼}
世勣選麾下及突厥精騎六千自直道邀之踰白道川追及於青山大度設走累日至諾真水勒兵還戰陳三十里|{
	陳讀曰陣}
突厥先與之戰不勝還走大度設乘勝追之遇唐兵薛延陀萬矢俱發唐馬多死世勣命士卒皆下馬執長矟直前衝之|{
	矟色角翻}
薛延陀衆潰副總管薛萬徹以數千騎收其執馬者薛延陀失馬不知所為唐兵縱擊斬首三千餘級捕虜五萬餘人大度設脱身走萬徹追之不及其衆至漠北值大雪人畜凍死者什八九李世勣還軍定襄突厥思結部居五臺者叛走|{
	五臺本漢太原慮虒縣久廢後魏改曰驢夷大業初改曰五臺有五臺山属代州}
州兵追之會世勣軍還夾擊悉誅之丙子薛延陀使者辭還上謂之曰吾約汝與突厥以大漠為界有相侵者我則討之汝自恃其強踰漠攻突厥李世勣所將纔數千騎耳汝已狼狽如此歸語可汗|{
	將即亮翻語牛倨翻厥九勿翻}
凡舉措利害可善擇其宜 上問魏徵比來朝臣何殊不論事|{
	朝直遥翻比毗至翻}
對曰陛下虛心采納必有言者凡臣徇國者寡愛身者多彼畏罪故不言耳上曰然人臣關說忤旨動及刑誅與夫蹈湯火冒白刃者亦何異哉|{
	忤五故翻冒莫北翻}
是以禹拜昌言|{
	見書三謨}
良為此也|{
	為千偽翻}
房玄齡高士亷遇少府少監竇德素於路|{
	秦置少府掌山澤之税漢掌内府珍貨梁始為卿隋改為監唐從三品少監從四品掌供百工伎巧之事凡天子之服御百官之儀制展采備物皆率其属以供之}
問北門近何營繕德素奏之上怒讓玄齡等曰君但知南牙政事北門小營繕何預君事|{
	唐正牙在南故曰南牙玄武門在北曰北門劉馮事始兵書曰牙旗者將軍之精凡始建牙必以制日制日者其辰之在五行以上剋下之日也又尚書曰門旗二口八幅色紅大將牙門之旗出引將軍前列又黄帝出軍訣曰牙旗者將軍之精金鼔者將軍之氣周禮司常職云軍旅會同置旌門夫以旌為門即旗門也後世軍中遂置牙門將乂有牙兵典總此兵以押衙為名至於官府早晩軍吏兩謁亦名為衙呼謂既熟雖天子正殿受朝謁亦名正衙}
玄齡等拜謝魏徵進曰臣不知陛下何以責玄齡等而玄齡等亦何所謝玄齡等為陛下股肱耳目於中外事豈有不應知者使所營為是當助陛下成之為非當請陛下罷之問於有司理則宜然不知何罪而責亦何罪而謝也上甚愧之 上嘗臨朝謂侍臣曰朕為人主常兼將相之事給事中張行成退而上書以為禹不矜伐而天下莫與之爭|{
	書舜謂禹曰汝惟不矜天下莫與汝爭能汝惟不伐天下莫與汝爭功朝直遥翻下同將即亮翻相息亮翻而上時掌翻}
陛下撥亂反正羣臣誠不足望清光然不必臨朝言之以萬乘之尊乃與羣臣校功爭能臣竊為陛下不取|{
	乘繩證翻為于偽翻}
上甚善之十六年春正月乙丑魏王泰上括地志|{
	上時掌翻下同}
泰好學司馬蘇朂說泰以古之賢王皆招士著書故泰奏請修之|{
	時泰奏引蕭德言顔胤蔣亞卿許偃等就府修撰好呼到翻說輸芮翻}
於是大開館舍廣延時俊人物輻湊門庭如市泰月給踰于太子諫議大夫褚遂良上疏以為聖人制禮尊嫡卑庶世子服物不會與王者共之|{
	周禮王及世子惟膳不會其他服物世子猶皆會會古外翻}
庶子雖愛不得踰嫡所以塞嫌疑之漸除禍亂之源也若當親者疎當尊者卑則佞巧之姦乘機而動矣昔漢竇太后寵梁孝王卒以憂死|{
	見漢景帝紀塞悉則翻卒于恤翻}
宣帝寵淮陽憲王亦幾至於敗|{
	見宣帝元帝紀幾居希翻下同}
今魏王新出閤宜示以禮則訓以謙儉乃為良器此所謂聖人之教不肅而成者也|{
	孝經載孔子之言}
上從之上又令泰徙居武德殿魏徵上書以為陛下愛魏王常欲使之安全宜每抑其驕奢不處嫌疑之地|{
	處昌呂翻}
今移居此殿乃在東宫之西海陵昔嘗居之|{
	元吉追封海陵刺王}
時人不以為可雖時異事異然亦恐魏王之心不敢安息也上曰幾致此誤遽遣泰歸第辛未徙死罪者實西州其犯流徒則充戍各以罪輕

重為年限 勑天下括浮遊無籍者限來年末附畢|{
	附者附籍也}
以兼中書侍郎岑文本為中書侍郎專知機密|{
	中書侍郎二員時獨用文本故專典機密}
夏四月壬子上謂諫議大夫褚遂良曰卿猶知起居注|{
	唐六典曰漢獻帝及西晉以後諸帝皆有起居注皆史官所録隋置起居舍人始為職員列為侍臣專掌其事每季為卷送付史官其以他官兼者則謂之知起居注}
所書可得觀乎對曰史官書人君言動備記善惡庶幾人君不敢為非未聞自取而觀之也|{
	幾居希翻}
上曰朕有不善卿亦記之邪|{
	邪音耶}
對曰臣職當載筆|{
	記曲禮曰史載筆}
不敢不記黄門侍郎劉洎曰借使遂良不記天下亦皆記之|{
	洎其冀翻}
上曰誠然 六月庚寅詔息隐王可追復皇太子海陵刺王元吉追封巢王|{
	息王海陵王皆帝踐阼追封刺來逹翻}
謚並依舊|{
	謚神至翻}
甲辰詔自今皇太子出用庫物所司勿為限制於是太子發取無度左庶子張玄素上書以為周武帝平定山東隋文帝混一江南勤儉愛民皆為令主有子不肖卒亡宗祀|{
	謂天元及焬帝也卒子恤翻}
聖上以殿下親則父子事兼家國所應用物不為節限恩旨未踰六旬用物已過七萬驕奢之極孰云過此况宫臣正士未嘗在側羣邪淫巧昵近深宫在外瞻仰已有此失居中隐密寧可勝計|{
	昵尼質翻近其靳翻勝音升}
苦藥利病苦言利行|{
	因張良之言而品節之}
伏惟居安思危日慎一日太子惡其書|{
	惡烏路翻}
令戶奴伺玄素早朝|{
	戶奴官奴掌守門戶伺相吏翻朝直遥翻}
密以大馬箠擊之幾斃|{
	箠止蕊翻幾居希翻又音祁}
秋七月戊子以長孫無忌為司徒房玄齡為司空 庚申制自今有自傷殘者據法加罪仍從賦役隋末賦役重數人往往自折支體謂之褔手褔足|{
	數所角翻折而設翻}
至是遺風猶存故禁之 特進魏徵有疾上手詔問之且言不見數日朕過多矣今欲自往恐益為勞若有聞見可封狀進來徵上言|{
	上時掌翻下上表同}
比者弟子陵師奴婢忽主下多輕上皆有為而然漸不可長又言陛下臨朝常以至公為言退而行之未免私僻或畏人知横加威怒|{
	比毗至翻為于偽翻長知兩翻朝直遥翻横戶孟翻}
欲蓋彌彰竟有何益徵宅無堂上命輟小殿之材以搆之|{
	程大昌曰魏徵宅在丹鳳門直出南面永興坊内}
五日而成仍賜以素屏風素褥几杖等以遂其所尚徵上表謝上手詔稱處卿至此蓋為黎元與國家豈為一人|{
	處昌呂翻為于偽翻}
何事過謝 八月丁酉上曰當今國家何事最急諫議大夫褚遂良曰今四方無虞唯太子諸王宜有定分最急|{
	分扶問翻}
上曰此言是也時太子承乾失德魏王泰有寵羣臣日有疑議上聞而惡之|{
	惡烏路翻}
謂侍臣曰方今羣臣忠直無踰魏徵我遣傳太子用絶天下之疑九月丁巳以魏徵為太子太師徵疾少愈詣朝堂表辭|{
	少詩沼翻朝直遥翻}
上手詔諭以周幽晉獻廢嫡立庶危國亡家|{
	周幽王廢太子而立褒姒之子為犬戎所殺周室遂微晉獻公廢世子立驪姬之子晉國大亂}
漢高袓幾廢太子賴四皓然後安|{
	見漢高紀及考異幾居希翻}
我今賴公即其義也知公疾病|{
	說文病疾加也}
可臥護之徵乃受詔 癸亥薛延陀真珠可汗遣其叔父沙鉢羅泥熟俟斤來請婚|{
	俟渠之翻}
獻馬三千貂皮三萬八千馬腦鏡一 癸酉以凉州都督郭孝恪行安西都護西州刺史高昌舊民與鎮兵及謫徙者雜居西州|{
	鎮兵謂鎮守之兵謫徙謂死罪流徒謫徙者}
孝恪推誠撫御咸得其歡心 西突厥乙毗咄陸可汗既殺沙鉢羅葉護并其衆又擊吐火羅滅之|{
	杜佑曰吐火羅一名土壑宜後魏時吐呼羅都䓤嶺西五百里在烏滸河南即媯水也}
自恃彊大遂驕倨拘留唐使者侵暴西域遣兵寇伊州郭孝恪將輕騎二千自烏骨邀擊敗之|{
	將即亮翻騎奇寄翻敗補邁翻}
乙毗咄陸又遣處月處密二部圍天山|{
	西州西南有南平安昌兩城又百二十里至天山軍}
孝恪擊走之乘勝進拔處月俟斤所居城追奔至遏索山降處密之衆而歸|{
	俟渠之翻索昔各翻降戶江翻}
初高昌既平歲發兵千餘人戍守其地褚遂良上疏以為聖王為治先華夏而後夷狄|{
	上時掌翻治直吏翻先悉薦翻夏戶雅翻後戶遘翻}
陛下興兵取高昌數郡蕭然累年不復|{
	不復不能復承平之舊也}
歲調千餘人屯戍|{
	調徒弔翻}
遠去鄉里破產辦裝又謫徙罪人皆無賴子弟適足騷擾邊鄙豈能有益行陳|{
	行戶剛翻陳讀曰陣}
所遣多復逃亡徒煩追捕|{
	復扶又翻}
加以道塗所經沙磧千里冬風如割夏風如焚行人往來遇之多死設使張掖酒泉有烽燧之警|{
	磧七迹翻掖音亦}
陛下豈得高昌一夫斗粟之用終當發隴右諸州兵食以赴之耳然則河西者中國之心腹高昌者他人之手足奈何糜弊本根以事無用之土乎且陛下得突厥吐谷渾皆不有其地為之立君長以撫之高昌獨不得與為比乎叛而執之服而封之刑莫威焉德莫厚焉願更擇高昌子弟可立者使君其國子子孫孫負荷大恩永為唐室藩輔内安外寧不亦善乎|{
	為于偽翻長知兩翻荷下可翻 考異曰貞觀政要載遂良疏云數郡蕭然五年不復下言十六年西突厥遣兵寇西州按實録此年唯有西突厥寇伊州不云寇西州蓋以伊州隸西州属部故云爾自十四年滅高昌距此適三年耳何得云五年不復或者三字誤為五字耳舊傳置此疏於十八年蓋亦因此而誤十八年無西突厥寇西州事故附於此}
上弗聼及西突厥入寇上悔之曰魏徵褚遂良勸我復立高昌|{
	復扶又翻又如字}
吾不用其言今方自咎耳乙毗咄陸西擊康居道過米國破之|{
	米國一曰彌末一曰弭秣賀治末息德城北百里距康居國}
虜獲甚多不分與其下其將泥熟啜輒奪取之|{
	將即亮翻下同啜陟劣翻下同}
乙毗咄陸怒斬泥熟啜以狥衆皆憤怨泥熟啜部將胡禄屋襲擊之乙毗咄陸衆散走保白水胡城於是弩失畢諸部及乙毗咄陸所部屋利啜等遣使詣闕請廢乙毗咄陸更立可汗|{
	使疏吏翻下同更工衡翻}
上遣使齎璽書立莫賀咄之子|{
	莫賀咄見一百九十三卷之二年璽斯氏翻}
為乙毗射匱可汗乙毗射匱既立悉禮遣乙毗咄陸所留唐使者帥所部擊乙毗咄陸於白水胡城|{
	帥讀曰率}
乙毗咄陸出兵擊之乙毗射匱大敗乙毗咄陸遣使招其故部落故部落皆曰使我千人戰死一人獨存亦不汝從乙毗咄陸自知不為衆所附乃西奔吐火羅 |{
	考異曰舊突厥傳云都護郭孝恪敗咄陸十五年屋利啜等請立可汗按上已云十五年册授沙鉢羅葉護可汗下不應更云十五年疑六字誤為五字耳二十年實録叙咄陸兵散居白水胡城事亦云是歲貞觀十五年也按十六年實録九月癸酉以凉州都督郭孝恪為安西都督則咄陸寇伊州應在其後豈得十五年已敗散乎突厥傳誤盖亦由此今因孝恪為都護并言之耳乙毗咄陸立事見上卷十二年}
冬十月丙申殿中監郢縱公宇文士及卒|{
	賀琛諡法敗亂百度曰縱怠德敗禮曰縱卒子恤翻}
上嘗止樹下愛之士及從而譽之不已|{
	譽音余}
上正色曰魏徵常勸我遠佞人|{
	遠于願翻}
我不知佞人為誰意疑是汝今果不謬士及叩頭謝 上謂侍臣曰薛延陀屈強漠北|{
	屈其勿翻強其兩翻}
今御之止有二策苟非發兵殄滅之則與之婚姻以撫之耳二者何從房玄齡對曰中國新定兵凶戰危臣以為和親便上曰然朕為民父母苟可利之何愛一女先是左領軍將軍契苾何力母姑臧夫人及弟賀蘭州都督沙門皆在凉州|{
	先悉薦翻鐵勒諸部初降以契苾部置榆溪州後又分置賀蘭州何力來降見一百九十四卷六年契欺訖翻苾毗必翻}
上遣何力歸覲且撫其部落時薛延陀方彊契苾部落皆欲歸之何力大驚曰主上厚恩如是奈何遽為叛逆其徒曰夫人都督先已詣彼若之何不往何力曰沙門孝於親我忠於君必不汝從其徒執之詣薛延陀置真珠牙帳前何力箕倨拔佩刀東向大呼曰|{
	呼火故翻}
豈有唐烈士而受屈虜庭天地日月願知我心因割左耳以誓真珠欲殺之其妻諫而止上聞契苾叛曰必非何力之意左右曰戎狄氣類相親何力入薛延陀如魚趍水耳|{
	趍七喻翻}
上曰不然何力心如鐵石必不叛我會有使者自薛延陀來具言其狀上為之下泣|{
	使疏吏翻下同為于偽翻下泣下涙也}
謂左右曰何力果如何即命兵部侍郎崔敦禮持節諭薛延陀以新興公主妻之|{
	妻七細翻}
以求何力|{
	新興公主皇女也}
何力由是得還拜右驍衛大將軍|{
	驍堅堯翻}
十一月丙辰上校獵於武功 丁巳營州都督張儉奏高麗東部大人泉蓋蘇文弑其王武|{
	泉姓也新書曰蓋蘇文者或號蓋金姓泉氏自云生水中以惑衆麗力知翻}
蓋蘇文凶暴多不法其王及大臣議誅之蓋蘇文密知之悉集部兵若校閲者并盛陳酒饌于城南|{
	饌雛戀翻又雛皖翻}
召諸大臣共臨視勒兵盡殺之死者百餘人因馳入宫手弑其王斷為數段棄溝中|{
	斷丁管翻}
立王弟子藏為王自為莫離支其官如中國吏部兼兵部尚書也於是號令遠近專制國事蓋蘇文狀貌雄偉意氣豪逸身佩五刀左右莫敢仰視每上下馬常令貴人武將伏地而履之|{
	將即亮翻}
出行必整隊伍前導者長呼則人皆奔迸不避阬谷路絶行者國人甚苦之|{
	呼火故翻迸比孟翻為征高麗張本}
壬戍上校獵于岐陽|{
	貞觀七年分岐州岐山雍州上宜置岐陽縣属岐州}
因幸慶善宫召武功故老宴賜極歡而罷庚午還京師 壬申上曰朕為兆民之主皆欲使之富貴若教以禮義使之少敬長婦敬夫則皆貴矣輕徭薄歛使之各治生業則皆富矣若家給人足朕雖不聼管絃樂在其中矣|{
	少詩照翻長知兩翻治直之翻歛力贍翻樂音洛}
亳州刺史裴行莊奏請伐高麗|{
	亳旁各翻麗力知翻}
上曰高麗王武職貢不絶為賊臣所弑朕哀之甚深固不忘也但因喪乘亂而取之雖得之不貴且山東彫弊吾未忍言用兵也 高袓之入關也隋武勇郎將馮翊党仁弘將兵二千餘人歸高袓於蒲反從平京城|{
	此皆隋㳟帝義寧元年事將即亮翻党抵朗翻}
尋除陜州總管大軍東討仁弘轉餉不絶|{
	謂討王世充時也陜失冉翻}
歷南寧戎廣州都督|{
	梁以犍為郡置戎州隋廢州為郡唐復改郡為州}
仁弘有材略所至著聲迹上甚器之然性貪罷廣州為人所訟贓百餘萬罪當死上謂侍臣曰吾昨見大理五奏誅仁弘|{
	五年制令死罪囚三日五覆奏}
哀其白首就戮方晡食遂命撤案然為之求生理|{
	為于偽翻}
終不可得今欲曲法就公等乞之十二月壬午朔上復召五品已上集太極殿前|{
	復扶又翻}
謂曰法者人君所受於天不可以私而失信今朕私党仁弘而欲赦之是亂其法上負於天欲席藁于南郊日一進蔬食以謝罪於天三日房玄齡等皆曰生殺之柄人主所得專也何至自貶責如此上不許羣臣頓首固請於庭自旦至日昃上乃降手詔自稱朕有三罪知人不明一也以私亂法二也善善未賞惡惡未誅三也|{
	惡惡上烏路翻下如字}
以公等固諫且依來請於是黜仁弘為庶人徙欽州癸卯上幸驪山温湯甲辰獵于驪山|{
	驪力知翻}
上登山見

圍有斷處顧謂左右曰吾見其不整而不刑則墮軍法|{
	墮讀曰隳}
刑之則是吾登高臨下以求人之過也乃託以道險引轡入谷以避之乙巳還宫 刑部以反逆緣坐律兄弟沒官為輕請改從死敕八座議之議者皆以為秦漢魏晉之法反者皆夷三族今宜如刑部請為是給事中崔仁師駁曰古者父子兄弟罪不相及奈何以亡秦酷法變隆周中典|{
	周禮秋官刑平國用中典父子兄弟罪不相及周法也駁北角翻}
且誅其父子足累其心|{
	累力瑞翻}
此而不顧何愛兄弟上從之上問侍臣曰自古或君亂而臣治或君治而臣亂二

者孰愈魏徵對曰君治則善惡賞罰當|{
	治直吏翻下同當丁浪翻}
臣安得而亂之苟為不治縱暴愎諫|{
	愎弼力翻}
雖有良臣將安所施上曰齊文宣得楊遵彦非君亂而臣治乎對曰彼纔能救亡耳|{
	事見一百六十六卷梁敬帝大平元年}
烏足為治哉

十七年春正月丙寅上謂羣臣曰聞外間士人以太子有足疾|{
	承乾病足不良行}
魏王頴悟多從遊幸遽生異議徼幸之徒|{
	徼堅壵翻}
已有附會者太子雖病足不廢步履且禮嫡子死立嫡孫|{
	記公儀仲子之喪舍其孫而立其子檀弓曰我未之前聞也問子服伯子曰仲子舍其孫而立其子何也曰昔文王舍伯邑考而立武王微子舍其孫□而立衍也夫仲子亦猶行古之道也子游問諸孔子曰否立孫}
太子男已五歲朕終不以孽代宗啟窺窬之源也|{
	孽魚列翻孽支庶也宗嫡子也}
鄭文貞公魏徵寢疾上遣使者問訊賜以藥餌相望於道又遣中郎將李安儼宿其第動静以聞|{
	使疏吏翻將即亮翻}
上復與太子同至其第指衡山公主欲以妻其子叔玉|{
	復扶又翻妻七細翻}
戊辰徵薨命百官九品以上皆赴喪給羽葆鼓吹陪葬昭陵|{
	吹昌瑞翻}
其妻裴氏曰徵平生儉素今葬以一品羽儀非亡者之志悉辭不受以布車載柩而葬|{
	柩音舊}
上登苑西樓|{
	長安禁苑之西樓也}
望哭盡哀上自製碑文并為書石|{
	為于偽翻}
上思徵不已謂侍臣曰人以銅為鏡可以正衣冠以古為鏡可以見興替以人為鏡可以知得失魏徵沒朕亡一鏡矣 鄠尉游文芝告代州都督劉蘭成謀反|{
	鄠音戶}
戊申蘭成坐腰斬右武候將軍丘行恭探蘭成心肝食之上聞而讓之曰蘭成謀反國有常刑何至如此若以為忠孝則太子諸王先食之矣豈至卿邪行恭慙而拜謝 二月壬午上問諫議大夫褚遂良曰舜造漆器諫者十餘人|{
	說苑堯釋天下舜受之作為飲器斬木而裁之猶漆黑之諸侯侈國之不服者十有三}
此何足諫對曰奢侈者危亡之本漆器不已將以金玉為之忠臣愛君必防其漸若禍亂已成無所復諫矣|{
	復扶又翻}
上曰然朕有過卿亦當諫其漸朕見前世帝王拒諫者多云業已為之或云業已許之終不為改|{
	不為于偽翻}
如此欲無危亡得乎時皇子為都督刺史者多幼稺遂良上疏以為漢宣帝云與我共治天下者其惟良二千石乎|{
	見二十四卷漢宣帝地節二年稺與稚同直利翻上時掌翻治直之翻}
今皇子幼稚未知從政不若且留京師教以經術俟其長而遣之|{
	長知兩翻}
上以為然 壬辰以太子詹事張亮為洛州都督侯君集自以有功而下吏|{
	見上卷十四年下遐嫁翻}
怨望有異志亮出為洛州君集激之曰何人相排亮曰非公而誰君集曰我平一國來逢嗔如屋大|{
	嗔昌真翻}
安能仰排因攘袂曰鬱鬱殊不聊生公能反乎與公反亮密以聞上曰卿與君集皆功臣語時旁無他人若下吏君集必不服如此事未可知卿且勿言待君集如故 鄜州都督尉遲敬德表乞骸骨|{
	鄜音膚尉紆勿翻}
乙巳以敬德為開府儀同三司五日一參|{
	參猶朝也}
丁未上曰人主惟有一心而攻之者甚衆或以勇力或以辯口或以諂諛或以姦詐或以嗜欲輻湊攻之各求自售以取寵禄人主少懈而受其一|{
	少詩沼翻懈古隘翻}
則危亡隨之此其所以難也 戊申上命圖畫功臣趙公長孫無忌趙郡元王孝恭|{
	諡法茂績丕德曰元主善行德曰元}
萊成公杜如晦|{
	如晦始封蔡國公既薨徙封萊國公}
鄭文貞公魏徵梁公房玄齡申公高士亷鄂公尉遲敬德衛公李靖宋公蕭瑀褒忠壮公段志玄夔公劉弘基蔣忠公屈突通鄖節公殷開山|{
	諡法好亷自克曰節鄖音云下同}
譙襄公柴紹|{
	柴紹當作許紹}
邳襄公長孫順德鄖公張亮陳公侯君集郯襄公張公謹盧公程知節永興文懿公虞世南渝襄公劉政會莒公唐儉英公李世勣胡壯公秦叔寶等於凌煙閣|{
	書爵不書謚者其人存書爵書謚者其人已死南部新書曰凌煙閣在西内三清殿側畫功臣皆北面閣中有中隔内面北寫功高侯王隔外面次第功臣程大昌曰閣中凡設三隔内一層畫功高宰輔外一層寫功高侯王又外一層次第功臣此三隔者雖分内外其所畫功臣象貌皆面北恐是在三清殿側以北面為恭邪余謂北面者臣禮也非以在三清殿側之故}
齊州都督齊王祐性輕躁其舅尚乘直長弘智說之曰|{
	尚乘局属殿中監有奉御有直長掌内外閑廐之馬辨其麄良而率其習馭者也乘繩證翻長知兩翻說輪芮翻}
王兄弟既多陛下千秋萬歲後宜得壯士以自衛祐以為然弘智因薦妻兄燕弘信|{
	燕因肩翻}
祐悦之厚賜金玉使隂募死士上選剛直之士以輔諸王為長史司馬諸王有過以聞祐昵近羣小好畋獵|{
	昵尼質翻近其靳翻好呼到翻}
長史權萬紀驟諫不聽壯士昝君謩梁猛彪得幸於祐萬紀皆劾逐之|{
	昝子感翻劾戶槩翻又戶得翻}
祐濳召還寵之逾厚上數以書切責祐萬紀恐并獲罪謂祐曰王審能自新萬紀請入朝言之乃條祐過失迫令表首|{
	數所角翻朝直遥翻下同首式又翻}
祐懼而從之萬紀至京師言祐必能悛改|{
	悛丑緣翻}
上甚喜勉萬紀而數祐前過以勑書戒之|{
	數所具翻}
祐聞之大怒曰長史賣我勸我而自以為功|{
	言萬紀勸祐令自首而自以為匡輔之功是為所賣也}
必殺之上以校尉京兆韋文振謹直用為祐府典軍|{
	唐諸府各有校尉每一校尉領旅帥二人王國親事府帳内府各有典軍二人正五品上副典軍二人從五品上掌率校尉以下守衛陪從之事校戶教翻}
文振數諫祐亦惡之|{
	數所角翻惡烏路翻}
萬紀性褊專以刻急拘持祐城門外不聼出悉解縱鷹犬斥君謩猛彪不得見祐會萬紀宅中有塊夜落|{
	塊苦對翻土塊}
萬紀以為君謩猛彪謀殺已悉收繫發驛以聞并劾與祐同為非者數十人|{
	劾戶槩翻又戶得翻}
上遣刑部尚書劉德威往按之事頗有驗詔祐與萬紀俱入朝祐既積忿遂與燕弘信兄弘亮等謀殺萬紀萬紀奉詔先行祐遣弘亮等二十餘騎追射殺之|{
	騎奇寄翻射而亦翻}
祐黨共逼韋文振欲與同謀文振不從馳走數里追及殺之寮屬股慄稽首伏地莫敢仰視|{
	稽音啓}
祐因私署上柱國開府等官開庫物行賞驅民入城繕甲兵樓堞置拓東王拓西王等官吏民棄妻子夜縋出亡者相繼祐不能禁|{
	乘夜縋城而出恐為逆黨汚染也堞逹恊翻縋馳偽翻}
三月丙辰詔兵部尚書李世勣等發懷洛汴宋潞滑濟鄆海九州兵討之|{
	濟子禮翻鄆音運}
上賜祐手勑曰吾常戒汝勿近小人正為此耳|{
	近其靳翻為于偽翻}
祐召燕弘亮等五人宿于臥内餘黨分統士衆廵城自守祐每夜與弘亮等對妃宴飲以為得志戲笑之際語及官軍弘亮等曰王不須憂弘亮等右手持酒巵左手為王揮刀拂之|{
	為于偽翻}
祐喜以為信然傳檄諸縣皆莫肯從事李世勣兵未至而青淄等數州兵已集其境|{
	淄州淄川郡武德元年分齊州之淄川置為郡}
齊府兵曹杜行敏等|{
	唐六典王府有兵曹參軍專掌武官簿書考課儀衛假使等事}
隂謀執祐祐左右及吏民非同謀者無不響應庚申夜四面鼓譟聲聞數十里|{
	聞音問}
祐黨有居外者衆皆攢刃殺之祐問何聲|{
	攢徂九翻}
左右紿云英公統飛騎已登城矣|{
	李世勣封英國公飛騎北門屯兵也紿蕩亥翻騎奇寄翻下同}
行敏分兵鑿垣而入祐與弘亮等被甲執兵入室閉扉拒戰|{
	垣于元翻被皮義翻}
行敏等千餘人圍之自旦至日中不克行敏謂祐曰王昔為帝子今乃國賊不速降立為煨燼矣|{
	煨烏回翻}
因命積薪欲焚之祐自牖間謂行敏曰即啟扉獨慮燕弘亮兄弟死耳行敏曰必相全祐等乃出或抉弘亮目投睛於地|{
	抉於决翻睛音精}
餘皆撾折其股而殺之執祐出牙前示吏民還鏁之於東廂齊州悉平乙丑勑李世勣等罷兵祐至京師賜死于内侍省|{
	星經有宦者四星在天市垣帝座之西周官有巷伯寺人之職皆内官也前漢宫官多用士人後漢始用宦者為宫官晉置大長秋卿為後宫官以宦者為之隋為内侍省焬帝改為長秋監武德初復為内侍省}
同黨誅者四十四人餘皆不問祐之初反也齊州人羅石頭面數其罪援槍前欲刺之|{
	數所具翻援于元翻刺七亦翻}
為燕弘亮所殺祐引騎擊高村村人高君狀遥責祐曰主上提三尺劒取天下億兆蒙德仰之如天王忽驅城中數百人欲為逆亂以犯君父無異一手揺泰山何不自量之甚也|{
	量音良}
祐縱擊虜之慙不能殺勑贈石頭亳州刺史以君狀為榆社令|{
	隋義寧元年分上黨之鄉縣置榆社縣属并州武德元年属韓州三年置榆州六年廢州以榆社属遼州亳旁各翻}
以杜行敏為巴州刺史封南陽郡公其同謀執祐者官賞有差上檢祐家文疏得記室郟城孫處約諫書|{
	郟城即漢潁川郡之郟縣也後魏置郟城縣及龍山縣隋開皇初改龍山曰汝南十八年改汝南曰輔城大業初改輔城曰郟城倂後魏之郟城地属焉師古曰郟音夾處昌呂翻}
嗟賞之累遷中書舍人庚午贈權萬紀齊州都督賜爵武都郡公謚曰敬韋文振左武衛將軍賜爵襄陽縣公 初太子承乾喜聲色及畋獵|{
	喜許記翻}
所為奢靡畏上知之對宫臣常論忠孝或至於涕泣退歸宫中則與羣小相䙝狎宫臣有欲諫者太子先揣知其意|{
	䙝息列翻揣初委翻}
輒迎拜歛容危坐引咎自責言辭辯給宫臣拜答不暇宫省袐密外人莫知故時論初皆稱賢太子作八尺銅鑪六隔大鼎募亡奴盗民間馬牛|{
	亡奴謂官奴之亡命在逃者}
親臨烹煮與所幸厮役共食之又好效突厥語及其服飾|{
	厮音斯今人讀若瑟好呼到翻}
選左右貌類突厥者五人為一落辮髪羊裘而牧羊作五狼頭纛及幡旗設穹廬太子自處其中|{
	纛徒到翻處昌呂翻}
歛羊而烹之抽佩刀割肉相啗|{
	啗徒濫翻又徒覽翻}
又嘗謂左右曰我試作可汗死汝曹效其喪儀因僵臥于地衆悉號哭|{
	僵居良翻號戶高翻}
跨馬環走臨其身剺面良久太子歘起|{
	環音宦剺里之翻欻許勿翻}
曰一朝有天下當帥數萬騎獵於金城西|{
	金城恐當作金河帥讀曰率騎奇寄翻}
然後解髪為突厥委身思摩若當一設不居人後矣|{
	自謂得為思摩典兵當一設之任必當表表自見史言承乾之狂愚}
左庶子于志寧右庶子孔頴逹數諫太子|{
	數所角翻下素數上數同}
上嘉之賜二人金帛以風勵太子|{
	風音諷又如字}
仍遷志寧為詹事志寧與左庶子張玄素數上書切諫太子隂使人殺之不果|{
	上時掌翻}
漢王元昌所為多不法|{
	元昌上弟也}
上數譴責之由是怨望太子與之親善朝夕同遊戲分左右為二隊太子與元昌各統其一被氊甲操竹矟|{
	被皮義翻操七高翻稍色角翻}
布陳大呼交戰擊刺流血以為娛樂|{
	陳讀曰陣呼火故翻樂音洛下不樂同}
有不用命者披樹撾之|{
	披其手足引之就樹而撾之撾陟瓜翻}
至有死者且曰使我今日作天子明日於苑中置萬人營與漢王分將|{
	將即亮翻}
觀其戰鬬豈不樂哉又曰我為天子極情縱欲有諫者輒殺之不過殺數百人衆自定矣魏王泰多藝能有寵於上見太子有足疾濳有奪嫡之志折節下士以求聲譽|{
	折而設翻下遐嫁翻}
上命黄門侍郎韋挺攝泰府事後命工部尚書杜楚客代之二人俱為泰要結朝士|{
	為于偽翻要一遥翻朝直遥翻下同}
楚客或懷金以賂權貴因說以魏王聰明宜為上嗣文武之臣各有附託濳為朋黨太子畏其逼遣人詐為泰府典籖上封事其中皆言泰罪惡勑捕之不獲|{
	籖上時掌翻}
太子私幸太常樂童稱心|{
	樂童童子能執樂隸籍太常者稱心其名也舊書承乾傳云有太常樂人年十餘歲美姿容善歌舞承乾時加寵幸號曰稱心}
與同臥起道士秦英韋靈符挾左道得幸太子上聞之大怒悉收稱心等殺之連坐死者數人誚讓太子甚至|{
	誚才笑翻}
太子意泰告之怨怒愈甚思念稱心不已於宫中構室立其像朝夕奠祭徘徊流涕又於苑中作冢私贈官樹碑上意浸不懌太子亦知之稱疾不朝謁者動涉數月隂養刺客紇干承基等及壯士百餘人謀殺魏王泰|{
	紇下沒翻}
吏部尚書侯君集之壻賀蘭楚石為東宫千牛|{
	東宫左右内率府有千牛十六人掌執千牛刀侍奉左右}
太子知君集怨望數令楚石引君集入東宫問以自安之術|{
	數所角翻}
君集以太子暗劣欲乘釁圖之因勸之反舉手謂太子曰此好手當為殿下用之|{
	為于偽翻}
又曰魏王為上所愛恐殿下有庶人勇之禍|{
	以隋事動太子}
若有勑召宜密為之備太子大然之太子厚賂君集及左屯衛中郎將頓丘李安儼|{
	頓邱縣漢属東郡晉置頓邱郡後齊省隋開皇十六年復置属魏州武德初属澶州貞觀初廢澶州以頓邱縣還属於魏州將即亮翻}
使詗上意動静相語安儼先事隐太子隐太子敗安儼為之力戰|{
	詗火迴翻又休正翻語牛倨翻為于偽翻}
上以為忠故親任之使典宿衛安儼深自託於太子漢王元昌亦勸太子反且曰比見上側有美人|{
	比毗至翻}
善彈琵琶事成願以垂賜太子許之洋州刺史開化公趙節慈景之子也|{
	趙慈景高袓使之攻河東為堯君素所殺}
母曰長廣公主|{
	長廣公主高袓之女}
駙馬都尉杜荷如晦之子也尚城陽公主|{
	上女也}
皆為太子所親暱|{
	暱尼質翻}
預其反謀凡同謀者皆割臂以帛拭血燒灰和酒飲之誓同生死濳謀引兵入西宫|{
	西宫謂大内以在東宫西故稱之}
杜荷謂太子曰天文有變當速發以應之殿下但稱暴疾危篤主上必親臨視因茲可以得志太子聞齊王祐反於齊州謂紇干承基等曰我宫西牆去大内止可二十步耳與卿為大事豈比齊王乎會治祐反事連承基承基坐繫大理獄當死|{
	為紇干承基告變張本治直之翻}


資治通鑑卷一百九十六
