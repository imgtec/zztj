










 


 
 


 

  
  
  
  
  





  
  
  
  
  
 
  

  

  
  
  



  

 
 

  
   




  

  
  


    資治通鑑卷二百七十  宋 司馬光 撰

  胡三省 音註

  後梁紀五【起彊圉赤奮若七月盡屠維單閼九月凡二年有奇}


  均王中

  貞明三年秋七月庚戌蜀主以桑弘志為西北面第一招討王宗宏為東北面第二招討己未以兼中書令王宗侃為東北面都招討武信節度使劉知俊為西北面都招討【以伐岐也}
 晉王以李嗣源閻寶兵少未足以敵契丹辛未更命李存審將兵益之 蜀飛龍使唐文扆居中用事【扆隱豈翻}
張格附之與司徒判樞密院事毛文錫爭權文錫將以女適左僕射兼中書侍郎同平章事庾傳素之子會親族于樞密院用樂不先表聞蜀主聞樂聲怪之文扆從而譖之八月庚寅貶文錫茂州司馬其子司封員外郎詢流維州籍没其家貶文錫弟翰林學士文晏為榮經尉【榮經漢嚴道縣地唐武德四年置榮經縣屬雅州九域志在州南一百一十里}
傳素罷為工部尚書以翰林學士承旨庾凝績權判内樞密院事凝績傳素之再從弟也【同曾祖之弟為再從弟從才用翻}
清海建武節度使劉巖即皇帝位于番禺【漢書音義番音潘禺音愚}
國號大越大赦改元乾亨以梁使趙光裔為兵部尚書節度副使楊洞濳為兵部侍郎節度判官李殷衡為禮部侍郎並同平章事建三廟追尊祖安仁曰太祖文皇帝父謙曰代祖聖武皇帝兄隱曰烈宗襄皇帝以廣州為興王府 契丹圍幽州且二百日【是年三月契丹圍幽州事始見上卷}
城中危困李嗣源閻寶李存審步騎七萬會于易州【閻寶班在李存審之下而先書寶者嗣源與寶先進屯淶水而存審繼之也匈奴須知淶水西至易州四十里易州東北至幽州二百二十里}
存審曰虜衆吾寡虜多騎吾多步若平原相遇虜以萬騎蹂吾陳吾無遺類矣【蹂人九翻又徐又翻陳讀曰陣}
嗣源曰虜無輜重【重直用翻}
吾行必載糧食自隨若平原相遇虜抄吾糧【抄楚交翻}
吾不戰自潰矣不若自山中濳行趣幽州【趣七喻翻}
與城中合勢若中道遇虜則據險拒之甲午自易州北行庚子踰大房嶺【水經注聖水出上谷郡西南谷東南流逕大防嶺又曰良鄉縣西北有大防山防水出其南按易州即漢上谷郡地范成大北使錄自良鄉六十五里至幽州城外此又驛路也}
循澗而東嗣源與養子從珂將三千騎為前鋒距幽州六十里與契丹遇契丹驚却晉兵翼而隨之【張左右翼而踵其後}
契丹行山上晉兵行澗下每至谷口契丹輒邀之嗣源父子力戰乃得進至山口契丹以萬餘騎遮其前將士失色嗣源以百餘騎先進免胄揚鞭胡語謂契丹曰汝無故犯我疆場晉王命我將百萬衆直抵西樓滅汝種族【此史家以華言譯胡語而筆之於史也胡嶠入遼記曰自幽州西北入居庸關行幾一月乃至上京所謂西樓也西樓有邑屋市肆歐陽四夷附錄曰契丹好鬼而貴日每朔月旦東向而拜日其大會聚視國事皆以東向為尊西樓門屋皆東向薛史曰西樓距幽州三千里場音亦種章勇翻}
因躍馬奮檛三入其陳斬契丹酋長一人【檛側瓜翻陳讀曰陣下同酋慈秋翻長知兩翻}
後軍齊進契丹兵却晉兵始得出李存審命步兵伐木為鹿角人持一枝止則成寨契丹騎環寨而過寨中發萬弩射之流矢蔽日契丹人馬死傷塞路【環音患射而亦翻塞悉則翻}
將至幽州契丹列陳待之存審命步兵陳於其後【陳于契丹陳後將夾擊之也一曰以騎兵前進令步兵陳於其後}
戒勿動先令羸兵曳柴然草而進煙塵蔽天契丹莫測其多少因鼔譟合戰存審乃趣後陳起乘之【羸倫為翻趣讀曰促}
契丹大敗席卷其衆自北山去【取古北口路而去卷讀曰捲}
委棄車帳鎧仗羊馬滿野晉兵追之俘斬萬計辛丑嗣源等入幽州周德威見之握手流涕【為虜所困得救而解喜極涕流}
契丹以盧文進為幽州留後其後又以為盧龍節度使文進常居平州帥奚騎歲入北邊殺掠吏民【帥讀曰率下同}
晉人自瓦橋運糧輸薊城【九域志瓦橋北至涿州一百二十里涿州北至薊城一百二十里薊音計}
雖以兵援之不免抄掠契丹每入寇則文進帥漢卒為鄉導【鄉讀曰嚮}
盧龍巡屬諸州為之殘弊【盧龍諸州自唐中世以來自為一域外而捍禦兩蕃内而連兵河朔其力常有餘及并于晉則歲遣糧援繼之而不足此其故何也保有一隅者其心力專廣土衆民其心力有所不及也詩云無田甫田維莠驕驕信矣為于偽翻下為承誓為為吾請為同}
 劉鄩自滑州入朝朝議以河朔失守責之【河朔失守事見上卷朝直遥翻}
九月落鄩平章事左遷亳州團練使【當其時不能治也待其入朝而後責之失政刑矣}
 冬十月己亥加吳越王鏐天下兵馬元帥 晉王還晉陽【自魏州還晉陽}
王連歲出征凡軍府政事一委監軍使張承業承業勸課農桑畜積金穀收市兵馬徵租行法不寛貴戚由是軍城肅清【軍城謂晉陽軍城也}
饋餉不乏王或時須錢蒱博及給賜伶人而承業靳之【靳居焮翻吝惜也}
錢不可得王乃置酒錢庫令其子繼岌為承業舞承業以寶帶及幣馬贈之王指錢積呼繼岌小名謂承業曰和哥乏錢七哥宜以錢一積與之帶馬未為厚也【張承業第七晉王以兄事承業呼之為七哥}
承業曰郎君纒頭皆出承業俸祿【唐人凡為人舞人則以錢綵寶貨謝之謂之纒頭俸扶用翻}
此錢大王所以養戰士也承業不敢以公物為私禮王不悦憑酒以語侵之承業怒曰僕老敕使耳非為子孫計惜此庫錢所以佐王成霸業也不然王自取用之何問僕為不過財盡民散無所成耳【晉王他日卒如張承業之言}
王怒顧李紹榮索劒承業起挽王衣【索山客翻挽武遠翻引也}
泣曰僕受先王顧託之命誓為國家誅汴賊【朱氏居汴李氏名其為賊}
若以惜庫物死于王手僕下見先王無愧矣【先王謂晉王克用}
今日就王請死閻寶從旁解承業手令退承業奮拳敺寶踣地罵曰【敺烏口翻踣蒲北翻}
閻寶朱温之黨受晉大恩【言閻寶背梁降晉晉不殺而寵貴之}
曾不盡忠為報顧欲以諂媚自容邪曹太夫人聞之遽令召王【史書曹太夫人者以見嫡母劉夫人不可得而令其子}
王惶恐叩頭謝承業曰吾以酒失忤七哥【忤五故翻}
必且得罪于太夫人七哥為吾痛飲以分其過王連飲四巵承業竟不肯飲王入宫太夫人使人謝承業曰小兒忤特進【張承業于時官特進意亦晉王承制授之也}
適已笞之矣明日太夫人與王俱至承業第謝之【史言晉王之在魏皆張承業足饋餉以輔之亦内有曹夫人故承業得行其志}
未幾【幾居豈翻}
承制授承業開府儀同三司左衛上將軍燕國公承業固辭不受但稱唐官以至終身掌書記盧質嗜酒輕傲嘗呼王諸弟為豚犬王衘之承業恐其及禍乘間言曰盧質數無禮【間古莧翻數所角翻}
請為大王殺之王曰吾方招納賢才以就功業七哥何言之過也承業起立賀曰王能如此何憂不得天下質由是獲免【史言張承業不惟能足兵且能保護士君子}
晉王元妃衛國韓夫人次燕國伊夫人次魏國劉夫人劉夫人最有寵【書晉宫之次者以見其宫中貫魚失序}
其父成安人【成安漢斥丘縣北齊置成安縣唐屬相州時屬魏州九域志成安在魏州西一百里}
以醫卜為業夫人幼時晉將袁建豐掠得之入于王宫性狡悍淫妬【悍下罕翻又侯旰翻}
從王在魏父聞其貴詣魏宫上謁【上時掌翻}
王召袁建豐示之建豐曰始得夫人時有黄鬚丈人護之此是也王以語夫人【語牛倨翻}
夫人方與諸夫人爭寵以門地相高恥其家寒微大怒曰妾去鄉時略可記憶妾父不幸死亂兵妾守尸哭之而去今何物田舍翁敢至此命笞劉叟于宫門【父且笞之而何有于君異日李存渥之事無足怪也}
 越主巖遣客省使劉瑭使于吳告即位【是年八月劉巖稱帝}
且勸吳王稱帝 閏月戊申蜀主以判内樞密院庾凝績為吏部尚書内樞密使 十一月丙子朔日南至蜀主祀圓丘晉王聞河冰合曰用兵數歲限一水不得度【貞明元年晉得魏博兵始窺河上若以破夾寨為用兵之始則已十年矣}
今冰自合天贊我也亟如魏州 蜀主以劉知俊為都招討使【見是年七月}
諸將皆舊功臣多不用其命且疾之故無成功【伐岐無功也}
唐文扆數毁之【數所角翻}
蜀主亦忌其才嘗謂所親曰吾老矣知俊非爾輩所能馭也十二月辛亥收知俊稱其謀叛斬于炭市【劉知俊懼不見容于梁而奔岐懼不見容于岐而奔蜀卒亦不為蜀所容挾虎狼之性而附人人必虞其搏噬其能容之乎}
 癸丑蜀大赦改明年元曰光天 壬戌以張宗奭為天下兵馬副元帥 帝論平慶州功【賀瓌平慶州見上卷上年}
丁卯以左龍虎統軍賀瓌為宣義節度使同平章事尋以為北面行營招討使【為賀瓌不能拒晉張本}
 戊辰晉王畋于朝城【朝城本漢東武陽縣後周曰武陽唐改曰朝城九域志朝城縣在魏州東南八十里又三十里至河}
是日大寒晉王視河冰已堅引步騎稍度梁甲士三千戍楊劉城緣河數十里列柵相望晉王急攻皆陷之進攻楊劉城使步兵斬其鹿角負葭葦塞塹【陸佃埤雅曰葦即今之蘆一名葭葭葦之未秀者也萑即今之荻一名蒹蒹萑之未秀者也至秋堅成謂之萑葦萑小而葦大字說曰蘆謂之葭其小曰萑荻謂之兼其小曰葦荻強而葭弱荻高而葭下塞悉則翻}
四面進攻即日拔之獲其守將安彦之先是租庸使戶部尚書趙巖言于帝曰陛下踐阼以來尚未南郊議者以為無異藩侯【先悉薦翻}
為四方所輕請幸西都行郊禮遂謁宣陵【宣陵在河南伊闕縣故請帝因郊而謁陵}
敬翔諫曰自劉鄩失利以來【劉鄩敗見上卷二年}
公私困竭人心惴恐【惴之睡翻}
今展禮圓丘必行賞賚是慕虛名而受實弊也且勍敵近在河上【勍敵謂晉也勍渠京翻}
乘輿豈宜輕動【乘繩證翻}
俟北方旣平報本未晚【晉書曰郊祀者帝王之重事所以報本反始也}
帝不聽己巳如洛陽閲車服飾宫闕郊祀有日聞楊劉失守道路訛言晉軍已入大梁扼汜水矣【扼汜水謂扼虎牢之險也}
從官皆憂其家相顧涕泣【從才用翻}
帝惶駭失圖遂罷郊祀奔歸大梁 甲戌以河南尹張宗奭為西都留守 是歲閩王審知為其子牙内都指揮使延鈞娶越主巖之女【為于偽翻}


  四年春正月乙亥朔蜀大赦復國號曰蜀【蜀改國號見上卷二年}
帝至大梁【自洛陽還至大梁}
晉兵侵掠至鄆濮而還【晉拔楊劉楊劉}


  【屬鄆州界又西則濮州界鄆音運濮博木翻}
敬翔上疏曰國家連年喪師【上時掌翻喪息浪翻}
疆土日蹙陛下居深宫之中所與計事者皆左右近習豈能量敵國之勝負乎【量音良}
先帝之時奄有河北【開平之間幽滄鎮定魏皆附于梁故云然}
親御豪傑之將猶不得志【謂夾寨柏鄉蓨縣之師皆不得志于晉}
今敵至鄆州陛下不能留意臣聞李亞子繼位以來于今十年【開平元年晉王存朂嗣位于今十一年}
攻城野戰無不親當矢石近者攻楊劉身負束薪為士卒先一鼓拔之陛下儒雅守文晏安自若使賀瓌輩敵之而望攘逐寇讐非臣所知也陛下宜詢訪黎老【黎衆也}
别求異策不然憂未艾也臣雖駑怯【駑音奴}
受國重恩陛下必若乏才乞于邊垂自効疏奏趙張之徒言翔怨望帝遂不用吳以右都押牙王祺為虔州行營都指揮使將洪撫

  袁吉之兵擊譚全播嚴可求以厚利募贑石水工故吳兵奄至虔州城下虔人始知之【虔州水行至吉州有贑石之險吳先募水工習于水道故舟行無礙注詳見辯誤贑音紺}
 蜀太子衍好酒色樂遊戲【好呼到翻樂五教翻}
蜀主嘗自夾城過聞太子與諸王鬭雞擊毬喧呼之聲【蜀蓋倣長安之制附夾城為諸王宅}
歎曰吾百戰以立基業此輩其能守之乎由是惡張格而徐賢妃為之内主竟不能去也【張格贊立宗衍見二百六十八卷乾化二年惡烏路翻去羌呂翻}
信王宗傑有才略屢陳時政蜀主賢之有廢立意二月癸亥宗傑暴卒蜀主深疑之 河陽節度使北面行營排陳使謝彦章將兵數萬攻楊劉城甲子晉王自魏州輕騎詣河上彦章築壘自固決河水瀰浸數里以限晉兵晉兵不得進【謝彥章梁之騎將也懼晉兵之衝突決河水以限之幽并之突騎非南兵之所能敵自古然也瀰音彌}
彦章許州人也安彦之散卒多聚于兖鄆山谷為羣盜以觀二國成敗晉王招募之多降于晉【降戶江翻}
 己亥蜀主以東面招討使王宗侃為東西兩路諸軍都統【此伐岐東西兩路之兵也東路出寶雞西路出秦隴}
 三月吳越王鏐初立元帥府置官屬【前年梁加錢鏐諸道兵馬元帥去年又加天下兵馬元帥}
 夏四月癸卯朔蜀主立子宗平為忠王宗特為資王 岐王復遣使求好于蜀【岐與蜀絶見二百六十七卷乾化元年復扶又翻}
 己酉以吏部侍郎蕭頃為中書侍郎同平章事 保大節度使高萬金卒癸亥以忠義節度使高萬興兼保大節度使并鎮鄜延【太祖改保塞軍為忠義軍高萬興萬金之兄也兄弟並鎮今倂為一}
 司空兼門下侍郎同平章事趙光逢告老己巳以司徒致仕 蜀主自永平末【梁乾化元年蜀改元永平梁貞明二年蜀改元通正}
得疾昏瞀【瞀莫候翻}
至是增劇以北面行營招討使兼中書令王宗弼沈靜有謀五月召還以為馬步都指揮使乙亥召大臣入寢殿告之曰太子仁弱朕不能違諸公之請踰次而立之【即謂張格令諸公署表時事}
若其不堪大業可寘諸别宫幸勿殺之但王氏子弟諸公擇而輔之徐妃兄弟止可優其祿位慎勿使之掌兵預政以全其宗族内飛龍使唐文扆久典禁兵參預機密欲去諸大臣【去羌呂翻}
遣人守宫門王宗弼等三十餘人日至朝堂不得入見【見賢遍翻}
文扆屢以蜀主之命慰撫之伺蜀主殂即作難【伺相吏翻難乃旦翻}
遣其黨内皇城使潘在迎偵察外事【偵丑鄭翻伺也}
在迎以其謀告宗弼等宗弼等排闥入言文扆之罪以天冊府掌書記崔延昌權判六軍事【蜀置天策府見上卷乾化四年將罪唐文扆先奪其判六軍事}
召太子入侍疾丙子貶唐文扆為眉州刺史翰林學士承旨王保晦坐附會文扆削官爵流瀘州在迎炕之子也【潘炕亦蜀主所親任者也入筦樞密出居方鎮炕苦浪翻}
丙申蜀主詔中外財賦中書除授諸司刑獄案牘專委庾凝績都城及行營軍旅之事委宣徽南院使宋光嗣丁酉削唐文扆官爵流雅州辛丑以宋光嗣為内樞密使與兼中書令王宗弼宗瑤宗綰宗夔並受遺詔輔政初蜀主雖因唐制置樞密使專用士人【唐制樞密使本用宦者}
及唐文扆得罪蜀主以諸將多許州故人【蜀主本許州舞陽人其諸將亦多許人}
恐其不為幼主用故以光嗣代之自是宦者始用事【為蜀以宦者亡張本}
六月壬寅蜀主殂 【考異曰北夢瑣言云余聞宗弼親吏王處琪言建疑信王暴卒唐文扆與徐妃張格隂謀使尚食進雞燒餅因寘毒建疾困大臣魏弘夫等請誅文扆建曰太子好酒色若不克負荷幸無殺之徐氏兄弟勿與兵權言訖長吁而逝劉恕按舊史貶文扆後二十七日蜀主始殂疑曹處琪之妄孫光憲從而記之}
癸卯太子即皇帝位【名衍字化源建幼子也}
尊徐賢妃為太后【衍母也}
徐淑妃為太妃以宋光嗣判六軍諸衛事乙卯殺唐文扆王保晦命西面招討副使王全昱殺天雄節度使唐文裔于秦州【貞明二年蜀主遣唐文扆伐岐遂鎮秦州}
免左保勝軍使領右街使唐道崇官 吳内外馬步都軍使昌化節度使同平章事徐知訓驕倨淫㬥威武節度使知撫州李德誠【歐史職方考曰五代之際外屬之州揚州曰淮南宣州曰寧國鄂州曰武昌洪州曰鎮南復州曰武威杭州曰鎮海越州曰鎮東江陵府曰荆南益州梓州曰劒南東西川遂州曰武信興元府曰山南西道洋州曰武定黔州曰黔南潭州曰武安桂州曰靜江容州曰寧遠邕州曰建武廣州曰清海皆唐故號更五代無所易而今因之者也其餘僭偽改置之名不可悉考而不足道其因著于今者略著于譜按歐公之時去五代未遠十國僭偽自相署置其當時節鎮之名已無所考况欲考之于二三百年之後乎今台州有魯洵作杜雄墓碑云唐僖宗光啓三年陞台州為德化軍洵乃雄吏時為德化軍判官者也又嘉定中黄巖縣永寧江有泅于水者拾一銅印其文曰台州德化軍行營朱記宋太祖乾德元年錢昱以德化軍節度使本路安撫使兼知台州台州小郡猶置節度其它州郡從可知矣吳之昌化威武蓋亦置之境内屬城但不可得而考其地耳}
有家妓數十知訓求之【妓渠綺翻}
德誠遣使謝曰家之所有皆長年【長知兩翻謂年已長也}
或有子不足以侍貴人當更為公求少而美者【為于偽翻少詩照翻}
知訓怒謂使者曰會當殺德誠并其妻取之知訓狎侮吳王無復君臣之禮嘗與王為優自為參軍使王為蒼鶻總角弊衣執帽以從【優人為優以一人幞頭衣綠謂之參軍以一人髽角弊衣如僮奴之狀謂之蒼鶻從才用翻}
又嘗泛舟濁河王先起知訓以彈彈之【上彈徒旦翻下彈徒丹翻}
又嘗賞花于襌智寺【宋白曰襌智寺在揚州城東寺前有橋跨舊官河}
知訓使酒悖慢王懼而泣【悖蒲没翻又蒲妹翻}
四座股栗左右扶王登舟知訓乘輕舟逐之不及以鐵檛殺王親吏【檛側瓜翻}
將佐無敢言者父温皆不之知知訓及弟知詢皆不禮于徐知誥【以知誥養子也}
獨季弟知諫以兄禮事之【為徐知諫附于知誥以奪知詢金陵張本}
知訓嘗召兄弟飲知誥不至知訓怒曰乞子不欲酒欲劒乎又嘗與知誥飲伏甲欲殺之知諫躡知誥足【躡尼輒翻}
知誥陽起如厠遁去知訓以劒授左右刁彦能使追殺之彦能馳騎及于中塗舉劒示知誥而還以不及告【還從宣翻又如字還告知訓以追之不及也余謂楊渥徐知訓之于知誥皆知所惡者也}
平盧節度使同平章事諸道副都統朱瑾遣家妓通候問于知訓【妓渠綺翻}
知訓強欲私之瑾已不平知訓惡瑾位加已上【惡烏路翻}
置靜淮軍于泗州出瑾為靜淮節度使瑾益恨之然外事知訓愈謹瑾有所愛馬冬貯于幄夏貯于幬【貯丁呂翻幬徒到翻今之葛罩紗罩是也又直由翻唐韻曰單帳也冬貯于幄欲其煖也夏貯于幬旣欲其涼且隔蚊蝱以養人者養畜可謂愛之過矣}
寵妓有絶色知訓過别瑾【過音戈過瑾而言别}
瑾置酒自捧觴出寵妓使歌以所愛馬為壽知訓大喜瑾因延之中堂伏壯士于戶内出妻陶氏拜之【路振九國志瑾妻陶氏雅之女也}
知訓答拜瑾以笏自後擊之踣地【踣蒲北翻}
呼壯士出斬之瑾先繫二悍馬於廡下將圖知訓密令人解縱之馬相蹄齧【廡罔甫翻蹄大計翻齧魚結翻}
聲甚厲以是外人莫聞瑾提知訓首出知訓從者數百人皆散走瑾馳入府以首示吳王曰僕已為大王除害【從才用翻為于偽翻下吾為同}
王懼以衣障面走入内曰舅自為之我不敢知【吳王行密先娶朱氏與瑾同姓因呼之為舅}
瑾曰婢子不足與成大事以知訓首擊柱挺劒將出【挺待鼎翻拔也}
子城使翟虔等已闔府門勒兵討之乃自後踰城墜而折足【翟虔徐温親將也使之防衛吳王翟直格翻折而設翻}
顧追者曰吾為萬人除害以一身任患遂自剄【任音壬剄古頂翻}
徐知誥在潤州聞難【揚潤夾江相去五十餘里難乃旦翻}
用宋齊丘策即日引兵濟江 【考異曰吳錄九國志徐鉉江南錄知訓死知誥過江皆無日江南錄曰先主聞亂即日以州兵渡江至廣陵會瑾自殺因撫定其衆十國紀年吳史六月乙卯瑾殺知訓踰城自殺戊午知誥入揚州代知訓執政己未誅瑾黨與廣本戊午知誥親吏馬仁裕聞知訓死自蒜山渡白知誥知誥即日帥兵入揚州撫定吏民按揚潤相去至近知誥豈得四日然後聞之今從江南錄 按徐知誥勉就潤州以俟變本宋齊丘之策也事見上卷三年}
瑾已死因撫定軍府時徐温諸子皆弱温乃以知誥代知訓執吳政沈朱瑾尸于雷塘而滅其族【沈持林翻}
瑾之殺知訓也泰寧節度使米志誠從十餘騎問瑾所向聞其已死乃歸宣諭使李儼貧而困寓居海陵【李儼宣諭淮南見二百六十三卷唐昭宗天復二年}
温疑其與瑾通謀皆殺之嚴可求恐志誠不受命詐稱袁州大破楚兵將吏皆入賀伏壯士于戟門擒志誠斬之并其諸子 壬戌晉王自魏州勞軍于楊劉【勞力到翻}
自泛舟測河水其深沒槍王謂諸將曰梁軍非有戰意但欲阻水以老我師當涉水攻之甲子王引親軍先涉諸軍隨之褰甲横槍結陳而進是日水落深纔及膝匡國節度使北面行營排陳使謝彦章帥衆臨岸拒之【前書河陽節度使謝彦章此書匡國節度使蓋自河陽徙匡國也陳讀曰陣帥讀曰率}
晉兵不得進乃稍引却梁兵從之及中流鼓譟復進【復扶又翻}
彦章不能支稍退登岸晉兵因而乘之梁兵大敗死傷不可勝紀【臨岸與涉水者戰則據高者得其利俱戰于水中則勇者勝此謝彦章之所以敗也勝音升}
河水為之赤彦章僅以身免是日晉人遂陷濱河四寨 蜀唐文扆旣死太傅門下侍郎同平章事張格内不自安【張格附唐文扆見上三年}
或勸格稱疾俟命禮部尚書楊玢自恐失勢謂格曰【玢方貧翻}
公有援立大功【謂草表使諸公請立宗衍}
不足憂也庚午貶格為茂州刺史玢為榮經尉吏部侍郎許寂戶部侍郎潘嶠皆坐格黨貶官格尋再貶維州司戶庾凝績奏徙格於合水鎮【九域志卭州蒲江縣有合水鎮}
令茂州刺史顧承郾伺格隂事王宗侃妻以格同姓欲全之謂承郾母曰戒汝子勿為人報仇【郾于建翻為于偽翻}
他日將歸罪于汝承郾從之凝績怒因公事抵承郾罪秋七月壬申朔蜀主以兼中書令王宗弼為鉅鹿王宗瑤為臨淄王宗綰為臨洮王【洮土刀翻}
宗播為臨潁王宗裔宗夔及兼侍中宗黯皆為琅邪郡王【自典午度江以來江左以琅邪之王為衣冠甲族故三人皆封琅邪}
甲戌以王宗侃為樂安王丙子以兵部尚書庾傳素為太子少保兼中書侍郎同平章事蜀主不親政事内外遷除皆出于王宗弼宗弼納賄多私上下多怨宋光嗣通敏善希合【希主迎合也}
蜀主寵任之蜀由是遂衰【有政事則國強無政事則國衰衰者亡之漸也可不戒哉}
 吳徐温入朝于廣陵【自昇州入朝}
疑諸將皆預朱瑾之謀欲大行誅戮徐知誥嚴可求具陳徐知訓過惡所以致禍之由温怒稍解乃命網瑾骨于雷塘而葬之【徐温審知罪在其子故葬朱瑾}
責知訓將佐不能匡救皆抵罪獨刁彦能屢有諫書温賞之戊戌以知誥為淮南節度行軍副使内外馬步都軍副使通判府事 【考異曰按十國紀年六月乙卯知訓被殺至此四十四日吳之政事必有所出蓋知誥至廣陵即代知訓執吳政至此方除官耳}
兼江州團練使以徐知諫權潤州團練事【代知誥也}
温還鎮金陵總吳朝大綱【朝直遙翻}
自餘庶政皆決于知誥知誥悉反知訓所為事吳王盡恭接士大夫以謙御衆以寛約身以儉以吳王之命悉蠲天祐十三年以前逋税【梁旣簒唐淮南仍稱天祐至是歲為天祐十五年徐知誥蠲天祐十三年以前逋税是年以後其逋者徵之}
餘俟豐年乃輸之【謂天祐十四年逋租也}
求賢才納規諫除姧猾杜請託于是士民翕然歸心雖宿將悍夫無不悦服【史言徐知訓之驕倨淫暴適為徐知誥之資悍下罕翻又侯旰翻}
先是吳有丁口錢又計畝輸錢錢重物輕民甚苦之【先悉薦翻}
齊丘說知誥以為錢非耕桑所得今使民輸錢是教民弃本逐末也請蠲丁口錢【程大昌演繁露曰今之丁錢即漢世筭錢也以其計口輸錢故亦名口賦也漢四年初為筭賦如淳曰漢儀注民年十五以上至五十六出賦錢人百二十為一算治庫兵車馬至文帝時人多丁衆則遂取高帝本額歲減三之二則一口一年輸錢止于四十也賈捐之曰文帝偃武行文民賦四十丁男三年而一事如淳曰常賦歲百二十歲一事文帝時天下民多故出賦四十凡三歲而一事此之謂賦即高帝時百二十至此而減為四十者也此之謂事即古法一歲一丁供役無過三日者是也民年十五以上雖未成丁亦輸口錢所謂民賦四十者也及已成丁則每歲當供三日之役者至此減為三年而才受一年之役也唐制成丁而就役不役則計日收其庸末世所謂丁口錢本此說式芮翻}
自餘稅悉輸穀帛紬絹匹直千錢者當稅三千【以直千錢之物當稅額之三千}
或曰如此縣官歲失錢億萬計齊丘曰安有民富而國家貧者邪知誥從之由是江淮間曠土盡闢【曠土空曠不耕之土}
桑柘滿野國以富強知誥欲進用齊丘而徐温惡之【宋齊丘為徐知誥謀奪徐氏之政使温知之豈特惡之而已蓋齊丘之為人輕佻褊躁温以此惡之耳惡烏路翻}
以為殿直軍判官【殿直使之入直吳殿軍判官行軍判官也}
知誥每夜引齊丘于水亭屏語常至夜分【屏語屏左右而與齊丘密語也水亭則四旁空闊無耳屬于垣之虞夜分夜半也屏必郢翻}
或居高堂悉去屏障獨置大爐相向坐不言以鐵筯畫灰為字隨以匙滅去之【去屏障所以防左右隱蔽其身而竊窺者去羌呂翻}
故其所謀人莫得而知也 虔州險固吳軍攻之久不下【是年二月吳攻虔州}
軍中大疫王祺病吳以鎮南節度使劉信為虔州行營招討使未幾祺卒【幾居豈翻}
譚全播求救于吳越閩楚吳越王鏐以統軍使傳球為西南面行營應援使將兵二萬攻信州【統軍使吳越所置官}
楚將張可求將萬人屯古亭閩兵屯鄠都以救之【鄠都漢古縣唐屬虔州九域志在州南一百七十里}
信州兵纔數百逆戰不利吳越兵圍其城刺史周本啓關張虛幕于門内召僚佐登城樓作樂宴飲飛矢雨集安坐不動吳越疑有伏兵中夜解圍去吳以前舒州刺史陳璋為東南面應援招討使將兵侵蘇湖【侵蘇湖以牽制吳越救虔州之兵力}
錢傳球自信州南屯汀州【按九域志汀州北至虔州四百八十里移兵屯汀州示將救虔也}
晉王遣間使持帛書會兵于吳吳人辭以虔州之難【間古莧翻難乃旦翻}
 晉王謀大舉入寇周德威將幽州步騎三萬李存審將滄景步騎萬人李嗣源將邢洺步騎萬人王處直遣將將易定步騎萬人及麟勝雲蔚新武等州諸部落奚契丹室韋吐谷渾皆以兵會之八月并河東魏博之兵大閲于魏州【兵莫難於用衆是舉也晉兵先敗周德威父子死焉晉王特危而後濟耳蔚音鬱}
 蜀諸王皆領軍使彭王宗鼎謂其昆弟曰親王典兵禍亂之本今主少臣彊讒間將興【少詩照翻間古莧翻}
繕甲訓士非吾輩所宜為也因固辭軍使蜀主許之但營書舍植松竹自娛而已【史言王宗鼎為保身之謀而無維城之助}
 泰寧節度使張萬進輕險好亂【好呼到翻}
時嬖倖用事多求賂于萬進【嬖卑義翻又博計翻}
萬進聞晉兵將出己酉遣使附于晉且求援以亳州團練使劉鄩為兖州安撫制置使將兵討之 【考異曰莊宗實錄天祐十五年八月己酉張萬進歸欵薛史末帝紀貞明五年三月癸未削奪張守進官爵命劉鄩為制置使十月下兖州族守進萬進傳云貞明四年七月叛五年冬拔其城劉鄩傳五年萬進反冬拔其城莊宗實錄萬進傳云劉鄩攻圍歷年屠其城莊宗列傳云天祐十五年八月萬進歸于我均王無實錄紀傳多不同難以為據今以莊宗實錄列傳為定}
 甲子蜀順德皇后殂【周氏蜀主建正室也}
 乙丑蜀主以内給事王廷紹歐陽晃李周輅朱光葆宋承蕰田魯儔等為將軍及軍使【朱光葆當作宋光葆蕰音緼}
皆干預政事驕縱貪暴大為蜀患周庠切諫不聽【周庠與蜀主建同起于兵間歷事多矣}
晃患所居之隘夜因風縱火焚西鄰軍營數百間明旦召匠廣其居蜀主亦不之問光葆光嗣之從弟也【從才用翻}
 晉王自魏州如楊劉引兵略鄆濮而還循河而上軍于麻家渡【還從宣翻上時掌翻麻家渡蓋在濮州界}
賀瓌謝彦章將梁兵屯濮州北行臺村相持不戰【凡言相持不戰度其力未足以相勝而各伺其勢之有可乘者也}
晉王好自引輕騎迫敵營挑戰危窘者數四【好呼到翻挑徒了翻窘巨隕翻}
賴李紹榮力戰翼衛之得免趙王鎔及王處直皆遣使致書曰元元之命繫于王本朝中興繫于王【本朝謂唐也朝直遙翻}
奈何自輕如此王笑謂使者曰定天下者非百戰何由得之安可深居帷房以自肥乎【晉王此語謂王鎔也然王鎔志守祖父業自豢養而已晉王則至于滅梁以雪讎耻者也及梁旣滅莊宗之志滿矣馳騁田獵意以為不居帷房以自肥不知以帷房自禍也}
一旦王將出營都營使李存審扣馬泣諫曰大王當為天下自重彼先登陷陳將士之職也【都營使都總行營之事一時署置之官名也為于偽翻下王為之同陳讀曰陣}
存審輩宜為之非大王之事也王為之攬轡而還它日伺存審不在策馬急出顧謂左右曰老子妨人戲【以戰為戱何晉王之輕也至聞嗣源入大梁又何其衰也歟伺相吏翻}
王以數百騎抵梁營謝彦章伏精甲五千于隄下王引十餘騎度隄伏兵發圍王數十重【重直龍翻}
王力戰于中後騎繼至者攻之于外僅得出會李存審救至梁兵乃退王始以存審之言為忠【史言晉王勇而輕屢經危殆其得免者幸也然再危而再免者皆李存審援兵之力謂老子妨人戱可乎}
 吳劉信遣其將張宣等夜將兵三千襲楚將張可求于古亭破之又遣梁詮等擊吳越及閩兵二國聞楚兵敗俱引歸【虔州之勢孤矣詮且緣翻}
 梅山蠻寇邵州【梅山蠻居邵州界宋熙寧五年開置新化縣在邵州東北二百五十里}
楚將樊須擊走之九月壬午蜀内樞密使宋光嗣以判六軍讓兼中書令王宗弼蜀主許之 吳劉信書夜急攻虔州斬首數千級不能克使人說譚全播取質納賂而還【說式芮翻質音致還從宣翻又如字}
徐温大怒杖信使者信子英彦典親兵温授英彦兵三千曰汝父居上游之地將十倍之衆【劉信本鎮洪州南江自洪州至湖口馬當而會于大江廣陵當江之下流是信所居者上游之地也時淮南攻虔之兵十倍于虔人}
不能下一城是反也汝可以此兵往與父同反又使昇州牙内指揮使朱景瑜與之俱曰全播守卒皆農夫飢窘踰年妻子在外重圍旣解【重直龍翻}
相賀而去聞大兵再往必皆逃遁全播所守者空城耳往必克之【史言徐温旣能御將又能料敵}
冬十一月壬申蜀葬神武聖文孝德明惠皇帝于永

  陵廟號高祖 越主巖祀南郊大赦改國號曰漢 劉信聞徐温之言大懼引兵還擊虔州先鋒始至虔兵皆潰【果如徐溫所料}
譚全播奔雩都追執之【唐僖宗光啓元年譚全播推盧光稠據虔州中更二姓及全播自為之而亡}
吳以全播為右威衛將軍領百勝節度使先是吳越王鏐常自虔州入貢至是道絶【吳越自虔州道入貢詳見上卷二年今虔州入于吳故道絶先悉薦翻}
始自海道出登萊抵大梁【此即閩越入貢大梁水程也但吳越必就許浦或定海就舟水程比閩為近耳}
 初吳徐温自以權重而位卑說吳王曰今大王與諸將皆為節度使雖有都統之名不足相臨制【唐授吳王行密諸道行營都統其子渥隆演嗣位皆宣諭使李儼承制授之}
請建吳國稱帝而治王不許嚴可求屢勸温以次子知詢代徐知誥知吳政知誥與駱知祥謀出可求為楚州刺史可求旣受命至金陵見温說之曰【說式芮翻}
吾奉唐正朔常以興復為辭今朱李方爭朱氏日衰李氏日熾一旦李氏有天下吾能北面為之臣乎不若先建吳國以繫民望温大悦復留可求【復扶又翻}
參總庶政使草具禮儀【草貝建國儀注}
知誥知可求不可去【去羌呂翻}
乃以女妻其子續【妻千細翻其後嚴續遂相南唐}
 晉王欲趣大梁【趣七喻翻下同}
而梁軍扼其前堅壁不戰百餘日十二月庚子朔晉王進兵距梁軍十里而舍【自麻家渡進兵逼行臺村}
初北面行營招討使賀瓌善將步兵排陳使謝彦章善將騎兵瓌惡其與已齊名【史言賀瓌忌能以誤國事惡烏路翻}
一日瓌與彦章治兵于野【治直之翻}
瓌指一高地曰此可以立柵至是晉軍適置柵於其上瓌疑彦章與晉通謀瓌屢欲戰謂彦章曰主上悉以國兵授吾二人社稷是賴今彊寇壓吾門而逗留不戰可乎彦章曰彊寇憑陵利在速戰今深溝高壘據其津要彼安敢深入若輕與之戰萬一蹉跌則大事去矣【謝彦章欲持久以老晉師賀瓌欲決勝負于一戰以此觀之其智識固有間矣蹉七何翻跌徒結翻}
瓌益疑之密譛之于帝與行營馬步都虞曹州刺史朱珪謀因享士伏甲殺彦章及濮州刺史孟審澄别將侯温裕以謀叛聞【誣謝彦章等以謀叛聞奏于上}
審澄温裕亦騎將之良者也【梁之騎將皆死獨王彦章在耳}
丁未以朱珪為匡國留後癸丑又以為盧節度使兼行營馬步副指揮使以賞之【賀瓌為之請也}
晉王聞彦章死喜曰彼將帥自相魚肉亡無日矣【將即亮翻帥所類翻}
賀瓌殘虐失士卒心我若引兵直指其國都【國都謂大梁}
彼安得堅壁不動幸而一與之戰蔑不勝矣王欲自將萬騎直趣大梁周德威曰梁人雖屠上將【謂殺謝彦章也}
其軍尚全輕行儌利未見其福【儌一遙翻}
不從戊午下令軍中老弱悉歸魏州起師趨汴【趨七喻翻}
庚申毁營而進衆號十萬辛酉蜀改明年元曰乾德 賀瓌聞晉王已西【自行臺村趨大梁為自東徂西}
亦棄營而踵之晉王發魏博白丁三萬從軍以供營柵之役所至營柵立成壬戌至胡柳陂【胡柳陂在濮州西臨濮縣界}
癸亥旦者言梁兵自後至矣周德威曰賊倍道而來未有所舍我營柵已固守備有餘旣深入敵境動須萬全不可輕發此去大梁至近梁兵各念其家内懷憤激不以方略制之恐難得志王宜按兵勿戰德威請以騎兵擾之使彼不得休息至暮營壘未立樵爨未具乘其疲乏可一舉滅也【此周德威所以破王景仁者也若晉王能用之賀瓌必不能支梁事去矣豈必待李嗣源取東平哉}
王曰前在河上恨不見賊今賊至不擊尚復何待【復扶又翻}
公何怯也顧李存審曰敕輜重先發吾為爾殿後破賊而去【重直用翻為于偽翻殿丁練翻}
即以親軍先出德威不得已引幽州兵從之【晉王旣先出周德威若不以兵從之則為顧望不進此誠有不得已者矣史言其心}
謂其子曰吾無死所矣賀瓌結陳而至横亘數十里王帥銀槍都陷其陳【陳讀曰陣下同帥讀曰率}
衝盪擊斬往返十餘里行營左廂馬軍都指揮使鄭州防禦使王彦章軍先敗西走趣濮陽【梁之騎兵先敗走趣七喻翻下同}
晉輜重在陳西望見梁旗幟驚潰【晉輜重見梁騎兵西嚮謂其來犯故驚而潰}
入幽州陳幽州兵亦擾亂自相蹈藉【藉慈夜翻}
周德威不能制父子皆戰死【陳旣擾亂周德威雖勇一夫敵耳}
魏博節度副使王緘與輜重俱行亦死晉兵無復部伍梁兵四集勢甚盛晉王據高丘收散兵至日中軍復振【據高丘則散兵望旗聞鼓而集故其軍復振復振者言其師徒已橈敗復振迅而起也}
陂中有土山賀瓌引兵據之晉王謂將士曰今日得此山者勝吾與汝曹奪之即引騎兵先登李從珂與銀槍大將王建及以步卒繼之梁兵紛紛而下遂奪其山【用兵之勢據高以臨下者勝晉兵旣奪土山賀瓌失地利矣珂丘何翻}
日向晡【晡奔謨翻}
賀瓌陳于山西晉兵望之有懼色諸將以為諸軍未盡集不若斂兵還營詰朝復戰【詰去吉翻復扶又翻}
天平節度使東南面招討使閻寶曰王彦章騎兵已入濮陽【言王彦章所領騎兵已敗而西去}
山下惟步卒【山下謂土山之下此即指言賀瓌陳于山西之兵}
向晚皆有歸志我乘高趣下擊之破之必矣今王深入敵境偏師不利【謂周德威之兵喪敗}
若復引退必為所乘【復扶又翻下同}
諸軍未集者聞梁再克必不戰自潰凡決勝料敵惟觀情勢情勢已得斷在不疑【斷丁亂翻}
王之成敗在此一戰若不決力取勝縱收餘衆北歸河朔非王有也【言晉大舉而敗退梁兵乘勝度河則河朔必望風而歸梁}
昭義節度使李嗣昭曰賊無營壘日晚思歸但以精騎擾之使不得夕食俟其引退追擊可破也我若斂兵還營彼歸整衆復來勝負未可知也王建及擐甲横槊而進【擐音宦}
曰賊大將已遁【大將指王彦章}
王之騎軍一無所失今擊此疲乏之兵如拉朽耳【拉盧合翻}
王但登山觀臣為王破賊王愕然曰非公等言吾幾誤計【為于偽翻幾居依翻}
嗣昭建及以騎兵大呼陷陳【呼火故翻}
諸軍繼之梁兵大敗元城令吳瓊貴鄉令胡裝各帥白丁萬人于山下曳柴揚塵鼓譟以助其勢梁兵自相騰藉弃甲山積死亡者幾三萬人【帥讀曰率幾居依翻}
裝證之曾孫也【胡證在唐歷事憲穆位通顯家富于財證音正}
是日兩軍所喪士卒各三之二皆不能振【此所謂俱傷而兩敗也喪息浪翻下喪吾同}
晉王還營聞周德威父子死哭之慟曰喪吾良將是吾罪也以其子幽州中軍兵馬使光輔為嵐州刺史【晉王悔不用周德威之言致其戰死故罪已而擢其子嵐盧含翻}
李嗣源與李從珂相失見晉軍橈敗【橈奴教翻勢屈為橈}
不知王所之或曰王以北度河矣【以當作已}
嗣源遂乘冰北度將之相州【欲自相州歸邢州相息亮翻}
是日從珂從王奪山【謂奪土山也}
晚戰皆有功甲子晉王進攻濮陽拔之【九域志濮陽縣在濮州西九十里按唐志濮陽屬濮州九域志為澶州治所唐澶州治頓丘縣宋熙寧六年省頓丘入清豐縣清豐縣在澶州北六十里縣有舊州鎮即澶州所治頓丘城也蓋五代以前濮陽在河南而九域志之濮陽晉天福四年移就澶州南郭者也}
李嗣源知晉軍之捷復來見王于濮陽王不悦曰公以吾為死邪度河安之嗣源頓首謝罪王以從珂有功但賜大鍾酒以罰之自是待嗣源稍薄 初契丹主之弟名博囉鄂博者號北大王謀作亂於其國事覺契丹主數之曰汝與吾如手足【數所具翻兄弟之親如手如足}
而汝興此心我若殺汝則與汝何異乃囚之朞年而釋之博囉鄂博帥其衆奔晉【帥讀曰率}
晉王厚遇之養為假子任為刺史【官之為刺史而不釐務}
胡柳之戰以其妻子來奔晉軍至德勝渡【德勝渡在濮州北河津之要也}
王彦章敗卒有走至大梁者曰晉人戰勝將至矣頃之晉兵有先至大梁問次舍者【此亦晉之散兵也}
京城大恐帝驅市人登城又欲奔洛陽遇夜而止敗卒至者不滿千人傷夷逃散各歸鄉里月餘僅能成軍五年春正月辛巳蜀主祀南郊大赦 晉李存審于德勝南北築兩城而守之【唐澶州治頓丘縣自築德勝南北城及晉天福三年遂移澶州及頓丘縣于德勝以防河津懼契丹南牧也宋景德澶淵之役猶在德勝熙寧以來澶州治濮陽又非石晉所移之地}
晉王以存審代周德威為内外蕃漢馬步總管晉王還魏州遣李嗣昭權知幽州軍府事 漢主巖立越國夫人馬氏為皇后殷之女也【巖逆婦于楚見上卷元年}
 三月丙戌蜀北路行營都招討武德節度使王宗播等自散關擊岐度渭水【此寶雞渭河也}
破岐將孟鐵山會大雨而還【還從宣翻又如字}
分兵戍興元鳳州及威武城【威武城在鳳州此蜀所築也}
戊子天雄節度使同平章事王宗昱攻隴州不克 蜀主奢縱無度日與太后太妃遊宴于貴臣之家及遊近郡名山飲酒賦詩所費不可勝紀【勝音升}
仗内教坊使嚴旭強取士民女子内宫中或得厚賂而免之以是累遷至蓬州刺史太后太妃各出教令賣刺史令錄等官【令縣令錄錄事參軍}
每一官闕數人爭納賂賂多者得之【史言蜀朝政濁亂}
 晉王自領盧龍節度使【周德威死難其代且北邊大鎮士馬彊鋭故自領之}
以中門使李紹宏提舉軍府事代李嗣昭【以宦者代功臣失之矣}
紹宏宦者也本姓馬晉王賜姓名使與知嵐州事孟知祥俱為中門使知祥又薦教練使鴈門郭崇韜能治劇【治直之翻}
王以為中門副使崇韜倜儻有智略【倜他狄翻}
臨事敢決王寵待日隆【郭崇韜由此佐晉王滅梁}
先是中門使吳珪張虔厚相繼獲罪【吳珪薛史作吳珙先悉薦翻}
及紹宏出幽州知祥懼禍稱疾辭位王乃以知祥為河東馬步都虞自是崇韜專典機密【為郭崇韜德孟知祥薦之帥蜀張木}
 詔吳越王鏐大舉討淮南鏐以節度副大使傳瓘為諸軍都指揮使帥戰艦五百艘自東洲擊吳【自常州東洲出海復泝江而入以擊吳帥讀曰率下同艦戶黯翻艘蘇遭翻}
吳遣舒州刺史彭彦章及裨將陳汾拒之 吳徐温帥將吏藩鎮請吳王稱帝吳王不許夏四月戊戌朔即吳國王位大赦改元武義建宗廟社稷置百官宫殿文物皆用天子禮以金繼土【唐土行也吳欲繼唐故言以金德王}
臘用丑改諡武忠王曰孝武王廟號太祖【楊行密初諡武忠王}
威王曰景王【楊渥初諡威王}
尊母為太妃以徐温為大丞相都督中外諸軍事諸道都統鎮海寧國節度使守太尉兼中書令東海郡王以徐知誥為左僕射參政事兼知内外諸軍事仍領江州團練使以揚府左司馬王令謀為内樞使【吳都廣陵故謂揚州為揚府}
營田副使嚴可求為門下侍郎鹽鐵判官駱知祥為中書侍郎前中書舍人盧擇為吏部尚書兼太常卿【前中書舍人蓋唐官也}
掌書記殷文圭為翰林學士館驛巡官游恭為知制誥前駕部員外郎楊迢為給事中擇醴泉人迢敬之之孫也【敬之楊憑弟子也}
 錢傳瓘與彭彦章遇傳瓘命每船皆載灰豆及沙乙巳戰于狼山江【今通州靜海縣南五里有狼山山外即大江絶江南渡舟行八十里抵蘇州界自江順流出大海}
吳船乘風而進傳瓘引舟避之旣過自後隨之【自後隨之則風為傳瓘用陳侯瑱破王琳亦如此}
吳回船與戰傳瓘使順風揚灰吳人不能開目及船舷相接【舷胡田翻船邊也}
傳瓘使散沙于己船而散豆于吳船豆為戰血所漬吳人踐之皆僵仆【漬疾智翻踐慈演翻僵居良翻}
傳瓘因縱火焚吳船吳兵大敗彦章戰甚力兵盡繼之以木身被數十創【被皮義翻創初良翻}
陳汾按兵不救彦章知不免遂自殺傳瓘俘吳禆將七十人斬首千餘級吳人誅汾籍沒家貲以其半賜彦章家稟其妻子終身【稟筆錦翻給也}
 賀瓌攻德勝南城百道俱進以竹笮聨艨艟十餘艘蒙以牛革設睥睨戰格如城狀【笮才各翻竹索也艨艟即蒙衝戰艦也城上短垣謂之睥睨睥匹計翻睨五計翻}
横于河流以斷晉之救兵使不得度【斷音短}
晉王自引兵馳往救之陳于北岸不能進【陳讀曰陣}
遣善游者馬破龍入南城見守將氏延賞延賞言矢石將盡陷在頃刻晉王積金帛于軍門募能破艨艟者衆莫知為計親將李建及曰【李建及即王建及時為銀槍大將銀槍晉王帳前親兵也故曰親將建及少事李罕之為養子後復姓王故史或書李建及或書王建及}
賀瓌悉衆而來冀此一舉若我軍不度則彼為得計今日之事建及請以死決之乃選効節敢死士得三百人被鎧操斧【被皮義翻操七刀翻}
帥之乘舟而進【帥讀曰率}
將至艨艟流矢雨集建及使操斧者入艨艟間斧其竹笮又以木甖載薪沃油然火於上流縱之【木甖蓋即用韓信舊法漢書注所載者為之操七刀翻甖於耕翻}
隨以巨艦實甲士鼓譟攻之艨艟旣斷隨流而下梁兵焚溺者殆半晉兵乃得度瓌解圍走晉兵逐之至濮州而還【德勝至濮州九十里還從宣翻又如字}
瓌退屯行臺村 蜀主命天策府諸將無得擅離屯戍【離力智翻}
五月丁卯朔左散旗軍使王承愕承勲承會違命蜀主皆原之【散悉但翻原者赦其罪也}
自是禁令不行 楚人攻荆南高季昌求救于吳吳命鎮南節度使劉信等帥洪吉撫信步兵自瀏陽趣潭州【帥讀曰率下同九域志瀏陽西南至潭州一百六十里瀏力求翻又音柳趣七喻翻}
武昌節度使李簡等帥水軍攻復州【自鄂州以水軍攻復州由大江入漢口泝漢而上}
信等至潭州東境楚兵釋荆南引歸簡等入復州執其知州鮑唐 六月吳人敗吳越兵于沙山【敗補邁翻}
 秋七月吳越王鏐遣錢傳瓘將兵三萬攻吳常州徐温帥諸將拒之右雄武統軍陳璋以水軍下海門出其後【海門在今通州東海門縣界大江至此入海遵海東南則太湖入海之口舟行由此入太湖可以達常州之東洲}
壬申戰于無錫會温病熱不能治軍【治直之翻}
吳越攻中軍飛矢雨集鎮海節度判官陳彦謙遷中軍旗鼓于左取貌類温者擐甲胄號令軍事温得少息俄頃疾稍間【間如字}
出拒之時久旱草枯吳人乘風縱火吳越兵亂遂大敗殺其將何逢吳建斬首萬級傳瓘遁去追至山南復敗之【復扶又翻敗補邁翻下同}
陳璋敗吳越于香彎温募生獲叛將陳紹者賞錢百萬指揮使崔彦章獲之紹勇而多謀温復使之典兵【霍兵之役陳紹之功居多温不討其外叛之罪而念其功故復使之典兵}
初衣錦之役【見二百六十八卷乾化三年}
吳馬軍指揮曹筠叛奔吳越【指揮之下當有使字}
徐温赦其妻子厚遇之遣間使告之曰【間古莧翻}
使汝不得志而去吾之過也汝無以妻子為念及是役筠復奔吳温自數昔日不用筠言者三【數所具翻}
而不問筠去來之罪歸其田宅復其軍職筠内愧而卒【史言徐温能御將}
知誥請帥步卒二千易吳越旗幟鎧仗躡敗卒而東襲取蘇州【躡尼輒翻}
温曰爾策固善然吾且求息兵未暇如汝言也諸將皆以為吳越所恃者舟楫今大旱水道涸此天亡之時也宜盡步騎之勢一舉滅之温歎曰天下離亂久矣民困已甚錢公亦未易可輕若連兵不解方為諸君之憂今戰勝以懼之戢兵以懷之使兩地之民各安其業君臣高枕豈不樂哉【易以豉翻戢則立翻枕職任翻樂音洛史言徐温能保勝安民}
多殺何為遂引還【還從宣翻又如字}
吳越王鏐見何逢馬悲不自勝故將士心附之【勝音升史言錢鏐亦能結士心以保其國錢楊之勢所以莫能相尚也}
寵姬鄭氏父犯法當死左右為之請【為于偽翻}
鏐曰豈可以一婦人亂我法出其女而斬之鏐自少在軍中【少詩照翻}
夜未嘗寐倦極則就圓木小枕或枕大鈴寐熟輒欹而寤名曰警枕【或枕職任翻記少儀茵席枕几熲鄭氏註曰熲警枕也孔穎達疏云以經枕外别言穎穎是頴發之義故為警枕余謂錢鏐枕圓木小枕或枕大鈴令欹而寤名曰警枕彼豈知有禮記注疏哉英雄之心雖寤寐之間不忘自警其闇與古合有如此者}
置粉盤于卧内有所記則書盤中比老不倦【比必利翻及也}
或寢方酣外有白事者令侍女振紙即寤時彈銅丸于樓牆之外以警直更者【直更者即持更之卒也更工衡翻}
嘗微行夜叩北城門吏不肯啓關曰雖大王來亦不可啓乃自他門入明日召北門吏厚賜之【史言錢鏐之公勤皆所以保其國}
 丙戌吳王立其弟濛為廬江郡公溥為丹陽郡公潯為新安郡公澈為鄱陽郡公子繼明為廬陵郡公 晉王歸晉陽以巡官馮道為掌書記中門使郭崇韜以諸將陪食者衆請省其數【晉王與諸將同甘苦凡食召諸將侍食必有不當預而預者故郭崇韜請省之省所景翻減也}
王怒曰孤為効死者設食【為于偽翻}
亦不得專可令軍中别擇河北帥孤自歸太原【帥所類翻}
即召馮道令草詞以示衆道執筆逡巡不為【逡七倫翻}
曰大王方平河南定天下崇韜所請未至大過【大讀曰太}
大王不從可矣何必以此驚動遠近使敵國聞之謂大王君臣不和非所以隆威望也會崇韜入謝王乃止 初唐滅高麗【唐高宗時滅高麗麗力智翻又力兮翻}
天祐初高麗石窟寺眇僧躬乂聚衆據開州稱王【眇僧僧之眇目者此開州高麗所置在平壤之東今高麗以為國都謂之開城府亦曰蜀莫郡其地左溪右山眇彌沼翻 考異曰薛史唐餘錄歐陽史皆云唐末其國自立王前王姓高氏後王王建此據十國紀年}
號大封國至是遣佐良尉金立奇入貢于吳 八月乙未朔宣義節度使賀瓌卒以開封尹王瓚為北面行營招討使【代賀瓌也瓚藏旱翻}
瓚將兵五萬自黎陽度河掩擊澶魏至頓丘遇晉兵而旋【初欲掩其不備遇晉兵而退旋與還同}
瓚為治嚴令行禁止【治直吏翻}
據晉人上游十八里楊村【據德勝上游也}
夾河築壘運洛陽竹木造浮橋自滑州饋運相繼晉蕃漢馬步副總管振武節度使李存進亦造浮梁于德勝或曰浮梁須竹笮鐵牛石囷【竹笮所以維浮梁鐵牛石囷所以繫竹笮囷區倫翻}
我皆無之何以能成存進不聽以葦笮維巨艦繫于土山巨木踰月而成人服其智 吳徐温遣使以吳王書歸無錫之俘于吳越吳越王鏐亦遣使請和于吳【無錫之戰吳越兵敗走徐温不窮追講和之計固已定于胷中矣}
自是吳國休兵息民三十餘州民樂業者二十餘年【史言息兵之利是時吳有揚楚泗滁和光黄舒蘄廬夀濠海潤常昇宣歙池饒信江鄂江洪撫袁吉虔等州}
吳王及徐温屢遺吳越王鏐書勸鏐自王其國【遺唯季翻王于况翻}
鏐不從 九月丙寅詔削劉巖官爵命吳越王鏐討之【以劉巖稱大號而職貢不入也}
鏐雖受命竟不行【受命者不逆梁之意不行者不肯自弊其力以伐與國此割據者之常計也}
 吳廬江公濛有材氣常歎曰我國家而為他人所有可乎徐温聞而惡之【為濛見殺張本惡烏路翻}


  資治通鑑卷二百七十  
    


 


 



 

 
  







 


  
  
 
 
 


  

 















	
	









































 
  



















 





 












  
  
  

 





