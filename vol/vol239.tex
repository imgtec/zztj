資治通鑑卷二百三十九 宋 司馬光 撰

胡三省 音注

唐紀五十五|{
	起玄黓執徐十月盡柔兆涒灘凡四年有奇}


憲宗昭文章武大聖至神孝皇帝中之上

元和七年冬十月乙未魏博監軍以狀聞|{
	以魏兵廢懷諫立田興之狀聞}
上亟召宰相謂李絳曰卿揣魏博若符契|{
	揣初委翻}
李吉甫請遣中使宣慰以觀其變李絳曰不可今田興奉其土地兵衆坐待詔命不乘此際推心撫納結以大恩必待敕使至彼持將士表來為請節鉞然後與之|{
	此大歷貞元之弊也為于偽翻下亦為正為度為當為同}
則是恩出於下非出於上將士為重朝廷為輕其感戴之心亦非今日之比也機會一失悔之無及吉甫素與樞密使梁守謙相結守謙亦為之言於上曰故事皆遣中使宣勞|{
	勞力到翻}
今此鎮獨無恐更不諭|{
	言恐其更不諭上意也}
上竟遣中使張忠順如魏博宣慰欲俟其還而議之癸卯李絳復上言|{
	復扶又翻}
朝廷恩威得失在此一舉時機可惜奈何弃之利害甚明願聖心勿疑計忠順之行甫應過陜|{
	甫始也陜失冉翻}
乞明旦即降白麻除興節度使猶可及也上且欲除留後絳曰興㳟順如此|{
	言興守朝廷法令申版籍請官吏異乎河北諸鎮之為也}
自非恩出不次則無以使之感激殊常上從之甲辰以興為魏博節度使忠順未還制命已至魏州興感恩流涕士衆無不鼓舞 庚戌更名皇子寛曰惲察曰悰寰曰忻寮曰悟審曰恪|{
	更工衡翻惲於粉翻}
李絳又言魏博五十餘年不霑皇化|{
	魏博自田承嗣以來倔彊拒命至是四十九年}
一旦舉六州之地來歸|{
	六州魏博貝衛澶相}
刳河朔之腹心傾叛亂之巢穴不有重賞過其所望則無以慰士卒之心使四鄰勸慕請發内庫錢百五十萬緡以賜之左右宦官以為所與太多後有此比將何以給之上以語絳|{
	語牛據翻}
絳曰田興不貪專地之利不顧四鄰之患歸命聖朝陛下奈何愛小費而遺大計不以收一道人心錢用盡更來機事一失不可復追|{
	復扶又翻}
借使國家發十五萬兵以取六州期年而克之|{
	期讀曰朞}
其費豈止百五十萬緡而已乎上悦曰朕所以惡衣菲食蓄聚貨財正為欲平定四方|{
	為于偽翻下同}
不然徒貯之府庫何為|{
	貯丁呂翻}
十一月辛酉遣知制誥裴度至魏博宣慰以錢百五十萬緡賞軍士六州百姓給復一年|{
	復方目翻復除其賦役也}
軍士受賜歡聲如雷成德兖鄆使者數輩見之相顧失色歎曰倔強者果何益乎|{
	兖鄆即淄青平盧軍也鄆音運倔其勿翻強其兩翻}
度為興陳君臣上下之義興聽之終夕不倦待度禮極厚請度徧至所部州縣宣布朝命|{
	朝直遥翻}
奏乞除節度副使於朝廷詔以戶部郎中河東胡証為之|{
	証之盛翻}
興又奏所部缺官九十員請有司注擬行朝廷法令輸賦税田承嗣以來室屋僭侈者皆避不居鄆蔡恒遣遊客閒說百方興終不聽|{
	鄆李師道蔡吳少陽恒王承宗也恒戶登翻間古莧翻說輸芮翻}
李師道使人謂宣武節度使韓弘曰我世與田氏約相保援今興非田氏族又首變兩河事|{
	言田興悉心奉朝廷變兩河藩鎮故事}
亦公之所惡也|{
	惡烏路翻}
我將與成德合軍討之弘曰我不知利害知奉詔行事耳若兵北度河我則以兵東取曹州|{
	曹州李師道廵屬也}
師道懼不敢動田興既葬田季安送田懷諫于京師辛巳以懷諫為右監門衛將軍 李絳奏振武天德左右良田可萬頃請擇能吏開置營田可以省費足食上從之絳命度支使盧坦經度用度|{
	度支經度皆徒洛翻}
四年之間開田四千八百頃收穀四千餘萬斛|{
	千當作十}
歲省度支錢二十餘萬緡邊防賴之 上嘗於延英謂宰相曰卿輩當為朕惜官|{
	為于偽翻}
勿用之私親故李吉甫權德輿皆謝不敢李絳曰崔祐甫有言非親非故不諳其才諳者尚不與官不諳者何敢復與但問其才器與官相稱否耳|{
	諳烏含翻復扶又翻稱尺證翻}
若避親故之嫌使聖朝虧多士之美此乃偷安之臣非至公之道也苟所用非其人則朝廷自有典刑誰敢逃之上曰誠如卿言 是歲吐蕃寇涇州及西門之外|{
	先寇涇州界進及涇州西門之外}
驅掠人畜而去上患之李絳上言京西京北皆有神策鎮兵|{
	京西鳳翔秦隴原涇渭也京北邠寧丹延鄜坊慶靈鹽夏綏銀宥也鎮兵注已見前}
始置之欲以備禦吐蕃使與節度使掎角相應也今則鮮衣美食坐耗縣官每有寇至節度使邀與俱進則云申取中尉處分|{
	唐神策鎮兵分屯于外皆屬左右神策中尉處昌呂翻分扶問翻}
比其得報虜去遠矣|{
	比必利翻及也}
縱有果鋭之將聞命奔赴節度使無刑戮以制之相視如平交左右前却莫肯用命何所益乎請據所在之地士馬及衣糧器械皆割隸當道節度使使號令齊壹如臂之使指則軍威大振虜不敢入寇矣上曰朕不知舊事如此當亟行之既而神策軍驕恣日久不樂隸節度使|{
	樂音洛}
竟為宦者所沮而止

八年春正月癸亥以博州刺史田融為相州刺史|{
	融興之兄也}
融興幼孤融長養而教之|{
	兄弟皆幼失父母而兄年差長故長養其弟而教之長知丈翻}
興嘗於軍中角射|{
	角競也角射者以中為勝}
一軍莫及融退而抶之|{
	抶丑栗翻打也}
曰爾不自晦禍將及矣故興能自全於猜暴之時|{
	猜暴之時謂田季安時也}
勃海定王元瑜卒弟言義權知國務庚午以言義為勃海王 李吉甫李絳數爭論於上前禮部尚書同平章事權德輿居中無所可否上鄙之|{
	數所角翻鄙陋也}
辛未德輿罷守本官 辛卯賜魏博節度使田興名弘正 司空同平章事于頔久留長安鬱鬱不得志|{
	二年頔入朝見二百三十二卷}
有梁正言者自言與樞密使梁守謙同宗能為人屬請|{
	為于偽翻下同屬之欲翻}
頔使其子太常丞敏重賂正言求出鎮久之正言詐漸露敏索其賂不得|{
	索山客翻}
誘其奴支解之弃溷中|{
	誘音酉溷戶困翻厠也}
事覺頔帥其子殿中少監季友等素服詣建福門請罪門者不内|{
	帥讀曰率唐大明宫端門曰丹鳳門其西曰建福門内即納字也}
退負南牆而立遣人上表閤門以無印引不受|{
	唐制凡四方章表皆閤門受而進之頔方請罪既無職印又無内引所以不受}
日暮方歸明日復至|{
	復扶又翻}
丁酉頔左授恩王傅仍絶朝謁|{
	朝直遥翻}
敏流雷州|{
	舊志雷州至京師六千五百一十二里}
季友等皆貶官僮奴死者數人敏至秦嶺而死|{
	自藍田關南出度秦嶺}
事連僧鑒虛鑒虛自貞元以來以財交權倖受方鎮賂遺|{
	遺音唯季翻}
厚自奉養吏不敢詰至是權倖爭為之言上欲釋之中丞薛存誠不可上遣中使詣臺宣旨曰朕欲面詰此僧非釋之也存誠對曰陛下必欲面釋此僧請先殺臣然後取之不然臣期不奉詔上嘉而從之三月丙辰杖殺鑒虛沒其所有之財 |{
	考異曰實録在二月按長歷二月乙酉朔三月甲寅朔丙辰三月三日甲子武元衡入知政事十一日也實録脱不書月日}
甲子徵前西川節度使同平章事武元衡入知政事|{
	元和元年武元衡出鎮西川至是召還}
夏六月大水上以為隂盈之象辛丑出宫人二百車 秋七月振武節度使李光進請修受降城兼理河防|{
	理治也}
時受降城為河所毁|{
	河毁受降城見上卷十年}
李吉甫請徙其徒於天德故城|{
	天德故城在東受降城西二百里大同川乾元後徙天德軍於永齊柵宋白續通典作永清柵其城則隋大同城之舊墟}
李絳及戶部侍郎盧坦以為受降城張仁愿所築|{
	事見二百九卷中宗景龍元年}
當磧口據虜要衝美水草守邊之利地今避河患退二三里可矣奈何捨萬代永安之策徇一時省費之便乎况天德故城僻處确瘠|{
	處昌呂翻确克角翻磽确也瘠土薄也}
去河絶遠烽候警急不相應接虜忽唐突勢無由知是無故而蹙國二百里也及城使周懷義奏利害與絳坦同上卒用吉甫策|{
	卒子恤翻}
以受降城騎士隸天德軍李絳言於上曰邊軍徒有其數而無其實虛費衣糧將帥但緣私役使|{
	緣私者並緣公役之名而私使之}
聚貨財以結權倖而已未嘗訓練以備不虞此不可不於無事之時豫留聖意也時受降城兵籍舊四百人及天德軍交兵止有五十人 |{
	考異曰實録李光進請東受降城兼理河防又云以中受降城及所管騎士一千一百四十人隸于天德軍舊傳盧坦與李絳叶議以為西城張仁愿所築不可廢三者不同莫知孰是今但云受降城所闕疑也又李司空論事云中城舊屬振武有鎮兵四百人其時割屬天德交割惟有五十人人數如此不同或者一千一百四十人是三城都數耳}
器械止有一弓自餘稱是|{
	稱尺證翻}
故絳言及之上驚曰邊兵乃如是其虛邪卿曹當加按閲會絳罷相而止 乙巳廢天威軍|{
	元和初并左右神威為一軍號天威軍神威軍本殿前射生軍也}
以其衆隸神策軍 丁未辰溆賊帥張伯靖請降|{
	辰溆賊反事始上卷六年}
辛亥以伯靖為歸州司馬委荆南軍前驅使|{
	委屬也付也}
初吐蕃欲作烏蘭橋|{
	新志會州烏蘭縣有烏蘭關在縣西南吐蕃於河上作橋}
先貯材於河側|{
	貯丁呂翻}
朔方常潛遣人投之於河終不能成虜知朔方靈鹽節度使王佖貪|{
	佖支筆翻又頻筆翻}
先厚賂之然後併力成橋仍築月城守之自是朔方禦寇不暇 冬十月回鶻發兵度磧南自柳谷西擊吐蕃|{
	新志西州交河縣北二百一十里經柳谷渡}
壬寅振武天德軍奏回鶻數千騎至鸊鵜泉|{
	鸊鵜泉在西受降城北三百里鸊扶歷翻鵜徒奚翻}
邊軍戒嚴 振武節度使李進賢不恤士卒判官嚴澈綬之子也|{
	於時嚴綬尚在綬音受}
以刻覈得幸於進賢進賢使牙將楊遵憲將五百騎趣東受降城以備回鶻所給資裝多虛估|{
	資裝不給本色虚估其價給以他物趣士喻翻}
至鳴沙遵憲屋處|{
	處昌呂翻}
而士卒暴露衆發怒夜聚薪環其屋而焚之|{
	環音宦}
卷甲而還|{
	還從宣翻又如字}
庚寅夜焚門攻進賢進賢踰城走軍士屠其家并殺嚴澈進賢奔靜邊軍|{
	靜邊軍在雲州西一百八十里}
羣臣累表請立德妃郭氏為皇后上以妃門宗彊盛|{
	妃郭曖之女子儀之孫女也}
恐正位之後後宫莫得進託以歲時禁忌竟不許 丁酉振武監軍駱朝寛奏亂兵已定請給將士衣上怒以夏綏節度使張煦為振武節度使|{
	煦吁句翻}
將夏州兵二千赴鎮仍命河東節度使王鍔以兵二千納之聽以便宜從事駱朝寛歸罪於其將蘇若方而殺之發鄭滑魏博卒鑿黎陽古河十四里以紓滑州水患

|{
	大河故瀆逕黎陽山之東後南徙為滑州患故復鑿古河}
上問宰相人言外間朋黨大盛何也李絳對曰自古人君所甚惡者莫若人臣為朋黨故小人譖君子必曰朋黨何則朋黨言之則可惡|{
	惡烏路翻}
尋之則無跡故也東漢之末凡天下賢人君子宦官皆謂之黨人而禁錮之遂以亡國|{
	見漢桓靈二帝紀}
此皆羣小欲害善人之言願陛下深察之夫君子固與君子合豈可必使之與小人合然後謂之非黨邪

九年春正月甲戌王鍔遣兵五千會張煦於善羊柵|{
	善羊當作善陽唐朔州治善陽縣西北至單于府百二十里柵蓋立於縣界}
乙亥煦入單于都護府|{
	振武節度使治單于都護府}
誅亂者蘇國珍等二百五十三人二月丁丑貶李進賢為通州刺史甲午駱朝寛坐縱亂者杖之八十奪色配役定陵|{
	奪色者奪其品色也}
李絳屢以足疾辭位癸卯罷為禮部尚書初上欲相絳先出吐突承璀為淮南監軍|{
	見上卷六年相息亮翻下同}
至是上召還承璀先罷絳相甲辰承璀至京師復以為弓箭庫使|{
	觀李絳立朝本末亦庶乎有大臣之節矣}
左神策中尉|{
	承璀以喪師罷中尉為弓箭庫使今遂兼為之此憲宗之巧蓋持兩端以觀朝議也李絳既罷誰敢復以為言乎}
李吉甫奏國家舊置六胡州於靈鹽之境|{
	調露元年於靈夏南境以降突厥置魯州麗州含州塞州依州契州以唐人為刺史謂之六胡州鹽州與靈夏接境}
開元中廢之更置宥州以領降戶天寶中宥州寄理於經畧軍|{
	長安四年併六胡州為長二州開元二十六年以廢州置懷恩縣帶宥州縣管内有榆多勒城天寶中王忠嗣奏置經畧軍在宥州故城東北三百里宋白曰宥州應接天德南援夏州治長澤縣本漢三封縣地}
寶應以來因循遂廢今請復之以備回鶻撫党項上從之|{
	党㡳朗翻}
夏五月庚申復置宥州理經畧軍取鄜城神策屯兵九千以實之|{
	大歷六年置肅戎軍於鄜州之鄜城}
先是回鶻屢請昏|{
	先悉薦翻}
朝廷以公主出降其費甚廣故未之許禮部尚書李絳上言以為回鶻凶彊不可無備淮西窮蹙事要經營今江淮大縣歲所入賦有二十萬緡者足以備降主之費陛下何愛一縣之賦不以羈縻勁虜回鶻若得許昏必喜而無猜然後可以修城塹蓄甲兵邊備既完得專意淮西功必萬全今既未降公主而虛弱西城|{
	西城謂西受降城}
磧路無備更修天德以疑虜心|{
	謂徙受降城於天德也}
萬一北邊有驚則淮西遺醜復延歲月之命矣|{
	復扶又翻}
儻虜騎南牧國家非步兵三萬騎五千則不足以抗禦借使一歲而勝之其費豈特降主之比哉上不聽 乙丑桂王綸薨|{
	綸上弟也}
六月壬寅以河中節度使張弘靖為刑部尚書同平章事弘靖延賞之子也|{
	張延賞相德宗於貞元之間}
翰林學士獨狐郁權德輿之壻也上歎郁之才美曰德輿得壻郁我反不及邪先是尚主皆取貴戚及勲臣之家|{
	先悉薦翻}
上始命宰相選公卿大夫子弟文雅可居清貫者|{
	史炤曰貫事也清貫猶言清職也}
諸家多不願惟杜佑孫司議郎悰不辭|{
	悰藏宗翻}
秋七月戊辰以悰為殿中少監駙馬都尉尚岐陽公主公主上長女郭妃所生也八月癸巳成昏公主有賢行|{
	行下孟翻}
杜氏大族尊行不趐數十人|{
	尊行之行下浪翻不趐與不啻同}
公主卑委怡順一同家人禮度二十年間人未嘗以絲髪間指為貴驕始至則與悰謀曰上所賜奴婢卒不肯窮屈|{
	卒子恤翻終也}
奏請納之悉自市寒賤可制指者|{
	制指謂可制御而指使者也}
自是閨門落然不聞人聲 閏月丙辰彰義節度使吳少陽薨 |{
	考異曰實録少陽卒在閏月己丑下壬辰上而并元濟焚舞陽言之統紀舊紀少陽卒皆在九月按舊傳曰少陽卒凡四十日不為輟朝唐紀張弘靖請為少陽廢朝贈官而實錄辛丑贈少陽右僕射然則己丑至辛丑才十二日耳豈容四十日不輟朝乎今從新紀}
少陽在蔡州隂聚亡命牧養馬騾時抄掠壽州茶山以實其軍|{
	壽州有茶山抄楚交翻}
其子攝蔡州刺史元濟匿喪以病聞自領軍務上自平蜀|{
	元和初平蜀}
即欲取淮西淮南節度使李吉甫上言少陽軍中上下攜離請徙理壽州以經營之|{
	淮南節度使治揚州欲徙治夀州以經畧淮西}
會朝廷方討王承宗|{
	事見上卷四年五年}
未暇也及吉甫入相田弘正以魏博歸附|{
	事見七年}
吉甫以為汝州扞蔽東都河陽宿兵本以制魏博今弘正歸順則河陽為内鎮不應屯重兵以示猜阻辛酉以河陽節度使烏重胤為汝州刺史充河陽懷汝節度使徙理汝州己巳弘正檢校右僕射賜其軍錢二十萬緡弘正曰吾未若移河陽軍之為喜也|{
	喜者喜朝廷之不猜防魏博}
九月庚辰以洺州刺史李光顔為陳州刺史充忠武都知兵馬使|{
	九域志陳州西南至蔡州一百九十里}
以泗州刺史令狐通為壽州防禦使通彰之子也|{
	肅宗時令狐彰背史思明歸順}
丙戌以山南東道節度使袁滋為荆南節度使以荆南節度使嚴綬為山南東道節度使吳少陽判官蘇兆楊元卿大將侯惟清皆勸少陽入朝元濟惡之|{
	惡烏路翻}
殺兆囚惟清元卿先奏事在長安具以淮西虛實及取元濟之策告李吉甫請討之時元濟猶匿喪元卿勸吉甫凡蔡使入奏者所在止之少陽死近四十日不為輟朝但易環蔡諸鎮將帥|{
	近其靳翻為于偽翻下同朝直遥翻環音宦}
益兵為備元濟殺元卿妻及四男以圬射堋|{
	圬哀乎翻墁也堋補鄧翻射埻也}
淮西宿將董重質吳少誠之壻也元濟以為謀主 戊戌加河東節度使王鍔同平章事 李吉甫言於上曰淮西非如河北四無黨援國家常宿數十萬兵以備之勞費不可支也失今不取後難圖矣上將討之張弘靖請先為少陽輟朝贈官遣使弔贈待其有不順之迹然後加兵上從之遣工部員外郎李君何弔祭|{
	唐工部郎掌城池土木之工役程式}
元濟不迎敕使發兵四出屠舞陽|{
	舞陽漢縣唐屬許州九域志在州西南一百八十里}
焚葉|{
	葉式涉翻}
掠魯山襄城關東震駭君何不得入而還|{
	還從宣翻又如字}
冬十月丙午中書侍郎同平章事趙公李吉甫薨 壬戌以忠武節度副使李光顔為節度使甲子以嚴綬為申光蔡招撫使督諸道兵招討吳元濟|{
	綬音受}
乙丑命内常侍知省事崔潭峻監其軍 |{
	考異曰實録作談峻今從舊傳}
戊辰以尚書左丞呂元膺為東都留守 党項寇振武十二月戊辰以尚書右丞韋貫之同平章事

十年春正月乙酉加韓弘守司徒弘鎮宣武十餘年不入朝頗以兵力自負朝廷亦不以忠純待之王鍔加平章事弘恥班在其下與武元衡書頗露不平之意朝廷方倚其形勢以制吳元濟故遷官使居鍔上以寵慰之吳元濟縱兵侵掠及於東畿|{
	東都畿也}
己亥制削元濟官

爵命宣武等十六道進軍討之嚴綬擊淮西兵小勝不設備淮西兵夜還襲之二月甲辰綬敗于磁丘|{
	磁丘當作慈丘縣屬唐州隋分北陽縣置取縣界慈丘山為名在州東北}
却五十餘里馳入唐州而守之|{
	九域志唐州東至蔡州三百五十里}
壽州團練使令狐通為淮西兵所敗|{
	敗補邁翻}
走保州城境上諸柵盡為淮西所屠癸丑以左金吾大將軍李文通代之貶通昭州司戶詔鄂岳觀察使柳公綽以兵五千授安州刺史李聽使討吳元濟公綽曰朝廷以吾書生不知兵邪即奏請自行許之公綽至安州李聽屬櫜鞬迎之|{
	屬之欲翻櫜姑勞翻鞬居言翻}
公綽以鄂岳都知兵馬使先鋒行營兵馬都虞侯二牒授之選卒六千以屬聽戒其部校曰|{
	校戶教翻}
行營之事一决都將|{
	總諸部之軍者謂之都將}
聽感恩畏威如出麾下公綽號令整肅區處軍事|{
	處昌呂翻}
諸將無不服士卒在行營者其家疾病死喪厚給之妻淫泆者沈之於江|{
	泆弋質翻沈持林翻}
士卒皆喜曰中丞為我治家|{
	為于偽翻治且之翻}
我何得不前死故每戰皆捷公綽所乘馬踶殺圉人|{
	踶特計翻圉人掌養馬者}
公綽命殺馬以祭之或曰圉人自不備耳此良馬可惜公綽曰材良性駑何足惜也竟殺之|{
	駑音奴}
河東將劉輔殺豐州刺史燕重旰王鍔誅之及其黨|{
	燕於賢翻旰古案翻}
王叔文之黨坐謫官者凡十年不量移|{
	永貞元年貶王叔文之黨事見二百三十六卷量音良}
執政有憐其才欲漸進之者悉召至京師諫官爭言其不可上與武元衡亦惡之|{
	惡烏路翻}
三月乙酉皆以為遠州刺史官雖進而地益遠永州司馬柳宗元為柳州刺史朗州司馬劉禹錫為播州刺史|{
	永州古零陵郡隋置永州以永水為名京師南三千二百七十四里柳州漢潭中縣地隋置馬平縣唐初置昆州貞觀改柳州至京師水陸相乘五千四百七十里朗州古武陵郡梁置武州隋為朗州京師東南二千一百五十九里播州即漢夜郎且蘭二國西南隅之地漢置䍧柯郡唐置播州京師南四千四百五十里}
宗元曰播非人所居而夢得親在堂|{
	劉禹鍚字夢得}
萬無母子俱往理欲請於朝願以柳易播會中丞裴度亦為禹錫言曰|{
	為于偽翻}
禹錫誠有罪然母老與其子為死别良可傷上曰為人子尤當自謹勿貽親憂此則禹錫重可責也|{
	重直用翻}
度曰陛下方侍太后恐禹錫在所宜矜上良久乃曰朕所言以責為人子者耳然不欲傷其親心退謂左右曰裴度愛我終切明日禹錫改連州刺史|{
	連州漢桂陽陽山地唐置連州以郡南有黄連嶺為名京師南三千六百六十五里 考異曰舊禹錫傳元和十年自武陵召還宰相復欲置之郎署時禹錫作遊玄都觀詠看花君子詩語涉譏刺執政不悦復出為播州刺史禹錫集載其詩曰玄都觀裏桃千樹盡是劉郎去後栽按當時叔文之黨一切除遠州刺史不止禹錫一人豈緣此詩蓋以此得播州惡處耳實録曰中丞裴度奏其母老必與此子為死别臣恐傷陛下孝理之風憲宗曰為子尤須謹慎恐貽親之憂禹錫更合重於它人卿豈可以此論之度無以對良久帝改容而言曰朕所言是責人子之事然終不欲傷其所親之心明日改授禹錫連州趙元拱唐諫諍集裴度曰陛下方侍太后以孝理天下至如禹錫誠合哀矜憲宗乃從之明日制授禹錫連州既而語左右裴度終愛我切趙璘因話錄曰憲宗初徵柳宗元劉禹錫至京城俄而柳為柳州刺史劉為播州刺史柳以劉須侍親播州最為惡處請以柳州換上不許宰相對曰禹錫有老親上曰但要與郡豈繫母在裴晉公進曰陛下方侍太后不合發此言上有愧色劉遂改為連州按柳宗元墓誌將拜疏而未上耳非已上而不許也禹錫除播州時裴度未為相今從實録及諫諍集}
宗元善為文嘗作梓人傳|{
	傳直戀翻}
以為梓人不執斧斤刀鋸之技專以尋引規矩繩墨度羣木之材|{
	技渠綺翻引羊晉翻度徒洛翻}
視棟宇之制相高深圓方短長之宜指麾衆工各趨其事不勝任者退之|{
	相息亮翻趨七喻翻勝音升}
大夏既成|{
	夏與厦同胡雅翻}
則獨名其功受禄三倍亦猶相天下者立綱紀整法度擇天下之士使稱其職|{
	稱尺證翻}
居天下之人使安其業能者進之不能者退之萬國既理而談者獨稱伊傅周召|{
	召讀曰邵}
其百執事之勤勞不得紀焉或者不知體要衒能矜名|{
	衒熒絹翻}
親小勞侵衆官听听於府庭|{
	听魚隱翻又魚巾翻}
而遺其大者遠者是不知相道者也又作種樹郭槖駞傳曰槖駞之所種無不生且茂者或問之對曰槖駞非能使木壽且孳也|{
	孳津之翻生也}
凡木之性其根欲舒其土欲故既植之勿動勿慮去不復顧其蒔也若子|{
	蒔音侍更種也}
其置也若弃則其天全而性得矣它植者則不然根拳而土易愛之太恩憂之太勤旦視而暮撫已去而復顧甚者爪其膚以驗其生枯|{
	瓜側絞翻}
揺其本以觀其疎密而木之性日以離矣雖曰愛之其實害之雖曰憂之其實讎之故不我若也為政亦然吾居鄉見長人者|{
	長知兩翻}
好煩其令|{
	好呼到翻}
若甚憐焉而卒以禍之|{
	卒子恤翻}
旦暮吏來聚民而令之促其耕穫督其蠶織吾小人輟饔飱以勞吏之不暇|{
	饔於容翻飱蘇昆翻饔飱熟食也勞力到翻}
又何以蕃吾生而安吾性邪|{
	蕃音煩}
凡病且怠職此故也|{
	杜預曰職主也}
此其文之有理者也|{
	梓人傳以諭相種樹傳以諭守令故温公取之以其有資於治道也}
庚子李光顔奏破淮西兵於臨潁 田弘正遣其子布將兵三千助嚴綬討吳元濟 甲辰李光顔又奏破淮西兵於南頓|{
	南頓漢縣屬汝南郡唐屬陳州}
吳元濟遣使求救於恒鄆王承宗李師道數上表請赦元濟上不從|{
	恒戶登翻鄆音運數所角翻}
是時發諸道兵討元濟而不及淄青師道使大將將二千人趣壽春|{
	趣七喻翻}
聲言助官軍討元濟實欲為元濟之援也師道素養刺客姧人數十人厚資給之其人說師道曰|{
	說輸芮翻}
用兵所急莫先糧儲今河隂院積江淮租賦請潛往焚之募東都惡少年數百刼都市焚宫闕則朝廷未暇討蔡先自救腹心此亦救蔡一奇也師道從之自是所在盗賊竊發辛亥暮盗數十人攻河隂轉運院殺傷十餘人燒錢帛三十餘萬緡匹穀二萬餘斛於是人情恇懼|{
	恇去王翻怯也}
羣臣多請罷兵上不許 諸軍討淮西久未有功五月上遣中丞裴度詣行營宣慰察用兵形勢度還|{
	還音旋又如字}
言淮西必可取之狀且曰觀諸將惟李光顔勇而知義必能立功上悦|{
	言有必克之勢故悦}
考功郎中知制誥韓愈上言以為淮西三小州|{
	三小州申光蔡}
殘弊困劇之餘而當天下之全力其破敗可立而待|{
	此以大小強弱之勢言也}
然所未可知者在陛下斷與不斷耳|{
	斷丁亂翻此以大歷貞元以來積習言也}
因條陳用兵利害以為今諸道發兵各二三千人勢力單弱羈旅異鄉與賊不相諳委望風懾懼|{
	諳烏含翻懾之涉翻}
將帥以其客兵待之既薄使之又苦或分割隊伍兵將相失心孤意怯難以有功|{
	將即亮翻下同}
又其本軍各須資遣道路遼遠勞費倍多聞陳許安唐汝壽等州與賊連接處村落百姓悉有兵器習於戰鬭識賊深淺比來未有處分|{
	比毗至翻近也處昌呂翻分扶問翻}
猶願自備衣糧保護鄉里若令召募立可成軍賊平之後易使歸農|{
	易以豉翻}
乞悉罷諸道軍募土人以代之又言蔡州士卒皆國家百姓若勢力窮不能為惡者不須過有殺戮 丙申李光顔奏敗淮西兵於時曲|{
	時曲在陳州溵水縣西南敗補邁翻}
淮西兵晨壓其壘而陳|{
	陳讀曰陣下同}
光顔不得出乃自毁其柵之左右出騎以擊之光顔自將數騎衝其陳出入數四賊皆識之矢集其身如蝟毛其子攬轡止之|{
	攬以手擥取也}
光顔舉刃叱去於是人爭致死淮西兵大潰殺數千人上以裴度為知人 上自李吉甫薨悉以用兵事委武元衡李師道所養客說李師道曰天子所以鋭意誅蔡者元衡贊之也請密往刺之|{
	說輸芮翻下同刺七亦翻}
元衡死則它相不敢主其謀爭勸天子罷兵矣師道以為然即資給遣之王承宗遣牙將尹少卿奏事為吳元濟遊說|{
	為吳于偽翻}
少卿至中書辭旨不遜元衡叱出之承宗又上書詆毁元衡六月癸卯天未明元衡入朝出所居靖安坊東門有賊自暗中突出射之從者皆散走|{
	射而亦翻從才用翻}
賊執元衡馬行十餘步而殺之取其顱骨而去|{
	顱龍都翻首骨也}
又入通化坊擊裴度傷其首墜溝中度氈㡌厚得不死傔人王義自後抱賊大呼賊斷義臂而去|{
	傔苦念翻傔從也呼火故翻斷音短}
京城大駭於是詔宰相出入加金吾騎士張弦露刃以衛之所過坊門呵索甚嚴|{
	呵叱也索搜也索山客翻下大索同}
朝士未曉不敢出門上或御殿久之班猶未齊賊遺紙於金吾及府縣|{
	遺弃也左右金吾掌邏捕姦非府縣京兆府及兩赤縣}
曰毋急捕我我先殺汝故捕賊者不敢甚急兵部侍郎許孟容見上言自古未有宰相横尸路隅而盗不獲者此朝廷之辱也因涕泣又詣中書揮涕言請奏起裴中丞為相大索賊黨窮其姦源戊申詔中外所在搜捕獲賊者賞錢萬緡官五品敢庇匿者舉族誅之於是京城大索公卿家有複壁重橑者皆索之|{
	複壁夾壁也重橑大屋覆小屋上下施椽其間皆可容物橑魯皓翻椽也史照憐蕭切}
成德軍進奏院有恒州卒張晏等數人行止無狀|{
	無狀者無善狀也恒戶登翻}
衆多疑之庚戌神策將軍王士則等告王承宗遣晏等殺元衡吏捕得晏等八人命京兆尹裴武監察御史陳中師鞫之癸亥詔以王承宗前後三表出示百僚議其罪裴度病瘡卧二旬詔以衛兵宿其第中使問訊不絶或請罷度官以安恒鄆之心上怒曰若罷度官是姧謀得成朝廷無復綱紀吾用度一人足破二賊|{
	史言憲宗明斷故能成功}
甲子上召度入對乙丑以度為中書侍郎同平章事度上言淮西腹心之疾不得不除且朝廷業已討之兩河藩鎮跋扈者將視此為高下不可中止上以為然悉以用兵事委度討賊甚急初德宗多猜忌朝士有相過從者|{
	過古禾翻}
金吾皆伺察以聞|{
	伺相吏翻}
宰相不敢私第見客度奏今寇盗未平宰相宜招延四方賢才與參謀議始請於私第見客許之陳中師按張晏等具服殺武元衡張弘靖疑其不實屢言於上上不聽戊辰斬晏等五人殺其黨十四人李師道客竟潛匿亡去 |{
	考異曰舊張弘靖傳曰初盗殺元衡京師索賊未得時王承宗邸中有鎮卒張晏輩數人行止無狀人多意之詔録附御史臺御史陳中師按之皆附致其罪如京中所說弘靖疑其不實驟於上前言之憲宗不聽及田弘正入鄆按簿書亦有殺元衡者但事曖昧互有所說卒未得其實按舊呂元膺傳獲李師道將訾嘉珍門察皆稱害武元衡者然則元衡之死必師道所為也但以元衡叱尹少卿及承宗上表詆元衡故時人皆指承宗耳今從薛圖存河南記}
秋七月庚午朔靈武節度使李光進薨光進與弟光顔友善光顔先娶其母委以家事母卒光進後娶光顔使其妻奉管籥籍財物歸于其姒|{
	毛晃曰杜預云兄弟之妻相謂曰姒蓋妯娌相呼以身年長少為名長曰姒少曰娣不以夫之長幼也今俗呼兄之妻曰姒弟之妻曰娣姒音詳里翻}
光進反之曰新婦逮事先姑先姑命主家事不可易也因相持而泣 甲戌詔數王承宗罪惡|{
	數所具翻}
絶其朝貢曰冀其翻然改過束身自歸攻討之期更俟後命 八月己亥朔日有食之 李師道置留後院於東都本道人雜沓往來吏不敢詰|{
	本道人謂兖鄆淄青人也}
時淮西兵犯東畿防禦兵悉屯伊闕師道潛内兵於院中至數十百人謀焚宫闕縱兵殺掠已烹牛饗士明日將發其小卒詣留守呂元膺告變元膺亟追伊闕兵圍之賊衆突出防禦兵踵其後不敢廹|{
	呂元膺以東都防禦使為留守其所統兵曰防禦兵}
賊出長夏門望山而遁|{
	唐六典東都城南面三門中曰定鼎左曰長夏右曰厚載東面三門中曰建春南曰永道北曰上東北面二門東曰安喜西曰徽安西連禁苑苑西四門南迎秋次遊義次籠煙北靈溪考異曰河南記曰賊帥訾嘉珍果於東都留後院潛召募二百餘人兼造置兵仗部署已定會門子健兒有小過被笞責之遂使兄弟一人告河南府當時飭兩縣驅丁壯悉持弓矢刀棒圍興道坊院數重賊黨廹蹙遞相蹂四面矢下如雨俄然殄滅因縱火焚其院宇悉為煨燼今從實録}
是時都城震駭留守兵寡弱元膺坐皇城門|{
	唐六典東都皇城在都城西北隅南面三門中曰端門左曰左掖門右曰右掖門東面一門曰賓耀西面二門南曰麗景北曰宣耀元膺坐于左掖門下}
指使部分|{
	分扶問翻}
意氣自若都人賴以安東都西南接鄧虢|{
	九域志河南府西南抵虢州界三百二十五里稍南抵鄧州界六百里}
皆高山深林民不耕種專以射獵為生人皆趫勇|{
	趫丘妖翻捷也}
謂之山棚元膺設重購以捕賊數日有山棚鬻鹿賊遇而奪之山棚走召其儕類|{
	儕士皆翻}
且引官軍共圍之谷中盡獲之按驗得其魁乃中岳寺僧圓淨故嘗為史思明將勇悍過人為師道謀多買田於伊闕陸渾之間以舍山棚而衣食之|{
	為師于偽翻舍始夜翻}
有訾嘉珍門察者|{
	訾即移翻姓也門亦姓也}
潛部分以屬圓淨圓淨以師道錢千萬陽為治佛光寺|{
	分扶問翻治直之翻下同}
結黨定謀約令嘉珍等竊發城中圓淨舉火於山中集二縣山棚入城助之|{
	二縣陸渾伊闕也}
圓淨時年八十餘捕者既得之奮鎚擊其脛不能折|{
	鎚直追翻脛戶定翻脚脛釋名曰脛莖也直而長似物莖折而設翻}
圓淨罵曰鼠子折人脛且不能敢稱健兒乃自置其脛教使折之臨刑歎曰誤我事不得使洛城流血黨與死者凡數千人留守防禦將二人|{
	留守兵之將及防禦兵之將也}
及驛卒八人皆受其職名|{
	職名李師道私所署衙前管軍職名給帖者也}
為之耳目元膺鞫訾嘉珍門察始知殺武元衡者乃師道也元膺密以聞以檻車送二人詣京師上業已討王承宗不復窮治|{
	復扶又翻}
元膺上言近日藩鎮跋扈不臣有可容貸者至於師道謀屠都城燒宫闕悖逆尤甚不可不誅|{
	悖蒲内翻又蒲沒翻}
上以為然而方討吳元濟絶王承宗故未暇治師道也|{
	史說得憲宗心事出}
乙丑李光顔敗於時曲 初上以嚴綬在河東所遣禆將多立功|{
	謂李光顔等也}
故使鎮襄陽|{
	襄陽山南東道節度治所}
且督諸軍討吳元濟綬無它材能到軍之日傾府庫賚士卒累年之積一朝而盡又厚賂宦官以結聲援擁八州之衆萬餘人屯境上|{
	八州襄鄧唐隨均房郢復}
閉壁經年無尺寸功裴度屢言其軍無政九月癸酉以韓弘為淮西諸軍都統弘樂於自擅|{
	樂音洛}
欲倚賊自重不願淮西速平 |{
	考異曰舊傳曰弘鎮汴州當兩河賊之衝要朝廷慮其異志欲以兵柄授之而令李光顔烏重胤實當旗鼓乃授弘淮西諸軍都統弘雖居統帥常不欲諸軍立功隂為逗撓之計每聞獻捷輒數日不怡其危國邀功如是按弘承宣武積亂之後鎮定一方居強寇之間威望甚著若有異志與諸鎮連衡跋扈如反掌耳然觀其始末未嘗失臣節朝廷若疑其有異志而更用為都統光顔重胤更受其節制非所以防之也且數日不怡有何狀可尋恐毁之過其實今從其可信者}
李光顔在諸將中戰最力弘欲結其歡心舉大梁城索得一美婦人|{
	宣武節度治大梁索山客翻}
教之歌舞絲竹飾以珠玉金翠直數百萬錢遣使遺之|{
	遺唯季翻}
使者先致書光顔大饗將士使者進妓容色絶世一座盡驚光顔謂使者曰相公愍光顔羈旅賜以美妓荷德誠深|{
	妓渠綺翻荷下可翻}
然戰士數萬皆弃家遠來冒犯白刃光顔何忍獨以聲色自娯悦乎因流涕座者皆泣|{
	座恐當作坐為文從字順}
即於席上厚以繒帛贈使者并妓返之曰為光顔多謝相公|{
	繒慈陵翻為于偽翻}
光顔以身許國誓不與逆賊同戴日月死無貳矣 冬十月庚子始分山南東道為兩節度以戶部侍郎李遜為襄復郢均房節度使以右羽林大將軍高霞寓為唐隨鄧節度使朝議以唐與蔡接故使霞寓專事攻戰而遜調五州之賦以餉之|{
	調徒弔翻}
辛丑刑部侍郎權德輿奏自開元二十五年修格式律令事類後|{
	唐六典叙文法之名格三十四篇式三十三篇律十二篇令二十七篇會要曰開元二十五年刪緝成律十二卷律疏三十卷式二十卷開元新格十卷又撰格式律令事類四十卷以類相從便於省覽}
至今長行敕近刪定為三十卷請施行從之|{
	會要開元十九年裴光庭等奏令有司刪撰格後長行敕六卷今又刪定二十五年以後長行勅為三十卷}
上雖絶王承宗朝貢未有詔討之魏博節度使田弘正屯兵於其境承宗屢敗之|{
	敗補邁翻下同}
弘正忿表請擊之上不許表十上|{
	上時掌翻}
乃聽至貝州丙午弘正軍于貝州 庚戍東都奏盗焚栢崖倉|{
	宋白曰河清縣有栢崖城杜佑曰栢崖城侯景所築在河清縣西}
十一月壽州刺史李文通奏敗淮西兵壬申韓弘請命衆軍合攻淮西從之李光顔烏重胤敗淮西兵於小溵水拔其城乙亥以嚴綬為太子少保|{
	以討淮西無功也}
盗焚襄州佛寺軍儲盡徙京城積草於四郊以備火丁丑李文通敗淮西兵於固始|{
	固始前漢汝南郡之寖縣春秋之寖丘後漢更名固始唐屬光州九域志在州東北一百四十五里}
戊寅盗焚獻陵寢宫永巷 詔發振武兵二千會義武軍以討王承宗 己丑吐蕃欵隴州塞請互市許之 初吳少陽聞信州人吳武陵名邀以為賓友武陵不答及元濟反武陵以書諭之曰足下勿謂部曲不我欺人情與足下一也足下反天子人亦欲反足下易地而論則其情可知矣 丁酉武寧節度使李愿奏敗李師道之衆時師道數遣兵攻徐州|{
	數所角翻}
敗蕭沛數縣|{
	敗補邁翻蕭沛皆漢縣唐屬徐州九域志蕭在州西五十里沛在州西北一百四十里}
愿悉以步騎委都押牙温人王智興擊破之十二月甲辰智興又破師道之衆斬首二千餘級逐北至平隂而還|{
	平隂古肥子國漢肥城縣之地隋開皇十四年置榆山縣大業初改曰平隂取界内平隂古城為名時屬鄆州九域志在州東北一百二十里還從宣翻又如字}
愿晟之子也 東都防禦使呂元膺請募山棚以衛宫城從之 乙丑河東節度使王鍔薨王承宗縱兵四掠幽滄定三鎮皆苦之爭上表請討

承宗上欲許之中書侍郎同平章事張弘靖以為兩役並興|{
	兩役謂既討淮西又討恒冀也}
恐國力所不支請併力平淮西乃征恒冀|{
	恒戶登翻}
上不為之止|{
	為于偽翻}
弘靖乃求罷 十一年春正月己巳幽州節度使劉緫奏敗成德兵拔武強斬首千餘級|{
	敗補邁翻}
庚辰翰林學士中書舍人錢徽駕部郎中知制誥蕭俛各解職守本官時羣臣請罷兵者衆上患之故黜徽俛以警其餘徽吳人也 癸未制削王承宗官爵命河東幽州義武横海魏博昭義六道進討韋貫之屢請先取吳元濟後討承宗曰陛下不見建中之事乎始於討魏及齊而蔡燕趙皆應卒致朱泚之亂|{
	事見二百二十六卷止二百二十八卷卒子恤翻}
由德宗不能忍數年之憤邑欲太平之功速成故也上不聽|{
	佳兵者不祥之器張弘靖韋貫之之言蓋未可厚非}
甲申盗斷建陵門戟四十七枝|{
	斷音短}
二月西川奏吐蕃贊普卒新贊普可黎可足立 乙巳以中書舍人李逢吉為門下侍郎同平章事逢吉玄道之曾孫也|{
	李玄道事太宗為文學館學士}
乙卯昭義節度使郗士美奏破成德兵|{
	郗丑之翻}
斬首千餘級 南詔勸龍晟淫虐不道上下怨疾弄棟節度王嵯巔弑之立其弟勸利勸利德嵯巔賜姓蒙氏謂之大容容蠻言兄也|{
	南詔置弄棟節度於唐姚州之地程大昌曰南詔有六節度曰弄棟永昌銀生劍川拓東麗水南詔王姓蒙氏嵯昨何翻巔音顛}
己未劉緫破成德兵斬首千餘級 荆南節度使袁滋父祖墓在朗山|{
	袁滋陳袁憲之後陳亡憲入中國後居蔡州朗山縣宋白曰朗山漢安昌縣漢末改朗山以界内朗山為名劉昫曰朗山漢安昌縣隋改朗山杜佑曰朗山漢朗陵縣宋避聖祖諱改朗山為確山}
請入朝欲勸上罷兵行至鄧州聞蕭俛錢徽貶官及見上更以必克勸之|{
	更工衡翻}
僅得還鎮 辛酉魏博奏敗成德兵拔其固城乙丑又奏拔其鵶城|{
	固城鵶城當在冀州南宫縣界}
三月庚午太后崩|{
	太后王氏上之母也}
辛未敕以國哀諸司公事權取中書門下處分|{
	只令宰相參决百司公事處昌呂翻分扶問翻}
不置攝冢宰|{
	唐中世以來天子崩置攝冢宰倣古者百官緫已聽于冢宰之制然非能盡行古道也}
壽州團練使李文通奏敗淮西兵於固始拔山|{
	五高翻又五到翻}
己卯唐鄧節度使高霞寓奏敗淮西兵於朗山斬首千餘級焚二柵 幽州節度使劉總圍樂壽 夏四月庚子李光顔烏重胤奏敗淮西兵於陵雲柵|{
	陵雲柵在溵水西南郾城東北蔡人立柵於此以陵雲為名}
斬首三千級 辛亥司農卿皇甫鏄以兼中承權判度支鏄始以聚歛得幸|{
	鏄補各翻歛力贍翻}
乙卯劉總奏破成德兵於深州斬首二千五百級乙丑義武節度使渾鎬奏破成德兵於九門殺千餘人鎬瑊之子也|{
	渾瑊事肅代德有大功}
宥州軍亂逐刺史駱怡夏州節度使田進討平之 五月壬申李光顔烏重胤奏敗淮西兵於陵雲柵斬首二千餘級 六月甲辰高霞寓大敗於鐵城僅以身免|{
	據舊書霞寓自蕭陂進至文城柵遇伏而敗意鐵城即文城柵以其堅不可破故謂之鐵城耳宋白曰鐵城在新興柵東北新興柵在吳房縣西南文城東北}
時諸將討淮西者勝則虛張殺獲敗則匿之至是大敗不可掩始上聞|{
	上時掌翻}
中外駭愕宰相入見|{
	見賢遍翻}
將勸上罷兵上曰勝負兵家之常今但當論用兵方畧察將帥之不勝任者易之|{
	不勝音升}
兵食不足者助之耳豈得以一將失利遽議罷兵邪於是獨用裴度之言它人言罷兵者亦稍息矣己酉霞寓退保唐州 上責高霞寓之敗霞寓稱李遜應接不至|{
	李遜主餉霞寓軍因得以罪歸之}
秋七月貶霞寓為歸州刺史|{
	歸州古之秭歸吳立建平郡唐置歸州京師南二千二百六十八里}
遜亦左遷恩王傅|{
	恩王連代宗之子}
以河南尹鄭權為山南東道節度使以荆南節度使袁滋為彰義節度申光蔡唐隨鄧觀察使以唐州為理所壬午宣武軍奏破郾城之衆二萬殺二千餘人捕虜千餘人 田弘正奏破成德兵於南宫殺二千餘人中書侍郎同平章事韋貫之性高簡好甄别流品|{
	好呼到翻甄稽延翻察也别彼列翻}
又數請罷用兵|{
	數所角翻}
左補闕張宿毁之於上云其朋黨八月壬寅貫之罷為吏部侍郎 諸軍討王承宗者互相觀望獨昭義節度使郗士美引精兵壓其境己未士美奏大破承宗之衆於栢鄉殺千餘人降者亦如之為三壘以環栢鄉|{
	栢鄉漢縣屬鉅鹿郡故城在今縣西南十七里今治在彭水之陽隋所置也屬趙州宋白曰趙州栢鄉縣春秋時晉鄗邑地漢置鄗縣光武改曰高邑北齊天保六年移高邑縣於漢房子縣東界今高邑縣是也隋開皇十六年於漢縣故城南十八里置栢鄉縣遥取漢栢鄉之名宋省栢鄉為鎮屬高邑環音宦}
庚申葬莊憲皇后于豐陵|{
	從順宗也}
九月乙亥右拾遺獨孤朗坐請罷兵貶興元府倉曹朗及之子也|{
	獨孤及事代宗為文長於議論}
饒州大水漂失四千七百戶 丙子以韋貫之為湖南觀察使猶坐前事也|{
	前事謂請罷用兵也}
辛巳以吏部侍郎韋顗考功員外郎韋處厚等皆為遠州刺史張宿讒之以為貫之之黨也覬見素之孫|{
	韋見素天寶末為相}
處厚夐之九世孫也|{
	韋夐後周韋孝寛之兄夐翾正翻}
乙酉李光顔烏重胤奏拔吳元濟陵雲柵丁亥光顔又奏拔石越二柵壽州奏敗殷城之衆拔六柵|{
	殷城漢期思縣屬汝南郡宋置苞信縣隋改曰殷城唐屬光州按九域志固始縣有殷城鎮}
冬十一月壬戍朔容管奏黄洞蠻為寇乙丑邕管奏擊黄洞蠻却之復賓蠻等州|{
	賓蠻當作賓巒武德四年以故秦桂林郡地置淳州永貞元年更名巒州}
丙寅加幽州節度使劉總同平章事 李師道聞拔陵雲柵而懼詐請輸欵上以力未能討加師道檢校司空 王鍔家二奴告鍔子稷改父遺表匿所獻家財|{
	去年王鍔薨}
上命鞫於内仗|{
	新書儀衛志凡朝會之仗三衛番上分為五仗號衙内五衛一曰供奉仗以左右衛為之二曰親仗以親衛為之三曰勲仗以勲衛為之四曰翊仗以翊衛為之五曰散手仗以親勲翊衛為之皆帶刀捉仗列坐東西廊下每月以四十六人立内廊閤外號曰内仗以左右金吾將軍當上中郎將一人押之}
遣中使詣東都檢括鍔家財裴度諫曰王鍔既沒其所獻之財已為不少今又因奴告檢括其家臣恐諸將帥聞之各以身後為憂上遽止使者己巳以二奴付京兆杖殺之 庚子以給事中柳公綽為京兆尹公綽初赴府|{
	赴京兆府初治事也}
有神策小將躍馬横衝前導公綽駐馬杖殺之明日入對延英上色甚怒詰其專殺之狀對曰陛下不以臣無似|{
	無似猶言不肖也}
使待罪京兆京兆為輦轂師表今視事之初而小將敢爾唐突此乃輕陛下詔命非獨慢臣也臣知杖無禮之人不知其為神策將軍也上曰何不奏對曰臣職當杖之不當奏上曰誰當奏者對曰本軍當奏若死於街衢金吾街使當奏|{
	金吾左右街使各一人掌分察六街徼廵凡城内坊角有武候鋪衛士彍騎分守大城門百人大鋪三十人小城門二十人小鋪五人日暮鼓八百聲而門閉乙夜街使以騎卒廵行叫呼武官暗探五更二點鼓自内發諸街鼓承振坊市門皆啟鼓三千撾而止}
在坊内左右廵使當奏|{
	程大昌雍録曰長安四郭之内縱横皆十坊大率當為百坊亦有一面不啻十坊者故六典曰一百一十坊也坊皆有垣有門隨晝夜鼓聲以行啟閉廵使掌左右街百坊之内謹啟閉徼廵者也宋白曰廣德二年九月命御史中丞兼戶部侍郎王延昌充左廵使御史中丞源休充右廵使辛亥源休充都左右廵使元和八年薛存誠奏得兩廵御史狀以承平舊例兩街本屬臺司其所由每月衙集動靜申報如所報差繆舉勘悉在臺中又按唐監察御史十員裏行五員掌内外糺察分為左右廵糺察違失以承天朱雀街為界每月一代將晦即廵刑部大理東西徒坊金吾及縣獄}
上無以罪之退謂左右曰汝曹須作意此人|{
	言須為此人作意務自謹敕}
朕亦畏之 |{
	考異曰柳氏叙訓曰公穆宗朝為大京兆有禁軍校冒騶卒唱駐馬斃之明日延英對上云云朝退上顧左右曰爾輩大須作意如此神采我亦怕他因話録曰憲宗正色詰公專殺之狀公曰京兆尹在取則之地臣初受陛下奬擢軍中偏禆躍馬衝過此乃輕陛下法不獨輕臣臣杖無禮之人不打神策將軍按公綽憲宗穆宗朝俱嘗為京兆尹此事恐非穆宗所能為叙訓之誤也今從因話録}
討淮西諸軍近九萬|{
	近其靳翻}
上怒諸將久無功辛巳命知樞密梁守謙宣慰因留監其軍授以空名告身五百通及金帛以勸死事庚寅先加李光顔等檢校官而詔書切責示以無功必罰 辛卯李文通奏敗淮西兵於固始斬首千餘級 十二月壬寅程執恭奏敗成德兵於長河|{
	長河漢信都廣川縣地隋於廣川縣東八十里置長河縣元和四年移就白橋於永濟河西岸置縣十年又置於河東小胡城屬德州}
斬首千餘級 義武節度使渾鎬與王承宗戰屢勝遂引全師壓其境距恒州三十里而軍|{
	恒戶登翻}
承宗懼潛遣兵入鎬境焚掠城邑人心始内顧而揺會中使督其戰鎬引兵進薄恒州|{
	薄伯各翻}
與承宗戰大敗奔還定州|{
	九域志恒州至定州一百三十五里}
丙午詔以易州刺史陳楚為義武節度使軍中聞之掠鎬及家人衣至於倮露|{
	倮郎果翻}
陳楚馳入定州|{
	易州南至定州百四十里}
鎮遏亂者歛軍中衣以歸鎬|{
	歛軍中所掠鎬家之衣也}
以兵衛送還朝|{
	朝直遥翻}
楚定州人張茂昭之甥也|{
	史言河朔之人習於叛亂知奉其帥之親黨而已}
丁未以翰林學士王涯為中書侍郎同平章事 袁滋至唐州去斥候|{
	去羌呂翻}
止其兵不使犯吳元濟境|{
	袁滋所謂開門揖盗者也}
元濟圍其新興柵|{
	新興柵當在唐州東北界新立之以備蔡人}
滋卑辭以請之元濟由是不復以滋為意|{
	復扶又翻}
朝廷知之甲寅以太子詹事李愬為唐隨鄧節度使愬聽之兄也|{
	愬聽皆李晟之子}
初置淮潁水運使揚子院米自淮隂泝淮入潁至項城入溵|{
	據舊史時運米泝淮至壽州四十里入潁口又泝流至潁州沈丘界五百里至於項城又泝流五百里入溵河又三百里輸于郾城得米五十萬石茭五百萬束省汴運之費七萬六千緡項城漢項縣屬汝南郡唐屬陳州九域志在州東南七十里據水經註溵水汝水之别流潁水至古南頓縣與溵水合唐之溵水縣漢汝陽縣地也}
輸于郾城以饋討淮西諸軍省汴運之費七萬餘緡|{
	郾音偃}
己未容管奏黄洞蠻屠巖州|{
	容管統容辨白牢欽巖禺湯瀼古等州}


資治通鑑卷二百三十九














































































































































