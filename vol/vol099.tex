\chapter{資治通鑑卷九十九}
宋 司馬光 撰

胡三省 音註

晉紀二十一|{
	起重光大淵獻盡閼逢攝提格凡四年}


孝宗穆皇帝中之上

永和七年春正月丁酉日有食之 苻健左長史賈玄碩等請依劉備稱漢中王故事|{
	事見六十八卷漢獻帝建安二十四年}
表健爲都督關中諸軍事大將軍大單于秦王|{
	玄碩欲表言之於晉朝單音蟬}
健怒曰吾豈堪爲秦王邪且晉使未返|{
	使疏吏翻}
我之官爵非汝曹所知也旣而密使梁安諷玄碩等上尊號|{
	上時掌翻}
健辭讓再三然後許之丙辰健即天王大單于位|{
	苻健字建業洪第三子}
國號大秦大赦改元皇始追尊殳洪爲武惠皇帝廟號太祖立妻強氏爲天王后|{
	強其兩翻氏姓也}
子萇爲太子靚爲平原公|{
	萇仲良翻靚疾正翻}
生爲淮南公覿爲長樂公|{
	樂音洛}
方爲高陽公碩爲北平公騰爲淮陽公柳爲晉公桐爲汝南公廋爲魏公|{
	廋所鳩翻}
武爲燕公幼爲趙公以苻雄爲都督中外諸軍事丞相領車騎大將軍雍州牧東海公|{
	雍於用翻}
苻菁爲衛大將軍平昌公宿衛二宫|{
	二宫健所居及子萇所居也}
雷弱兒爲太尉毛貴爲司空略陽姜伯周爲尚書令梁楞爲左僕射|{
	楞盧登翻}
王墮爲右僕射魚遵爲太子太師強平爲太傅段純爲太保呂婆樓爲散騎常侍|{
	散悉亶翻騎奇寄翻}
伯周健之舅平王后之弟婆樓本略陽氐酋也|{
	酋慈由翻}
段龕請以青州内附二月戊寅以龕爲鎭北將軍封齊公|{
	段龕據廣固始上卷上年龕苦含翻}
魏主閔攻圍襄國百餘日|{
	去年十一月閔攻襄國}
趙主祇危急乃去皇帝之號稱趙王|{
	去羌呂翻}
遣太尉張舉乞師於燕許送傳國璽|{
	璽斯氏翻}
中軍將軍張春乞師於姚弋仲弋仲遣其子襄帥騎二萬八千救趙|{
	帥讀曰率}
誡之曰冉閔弃仁背義屠滅石氏|{
	事見上卷五年六年背蒲妹翻}
我受人厚遇|{
	謂石虎遇之厚也}
當爲復讐老病不能自行汝才十倍於閔若不梟擒以來不必復見我也|{
	爲于僞翻梟堅堯翻復扶又翻}
弋仲亦遣使告於燕|{
	使疏吏翻下同}
燕主雋遣禦難將軍悦綰|{
	禦難將軍蓋慕容氏創置難乃旦翻}
將兵三萬往會之冉閔聞雋欲救趙遣大司馬從事中郎廣甯常煒使於燕雋使封裕詰之曰冉閔石氏養息|{
	息子也詰去吉翻}
負恩作逆何敢輒稱大號煒曰湯放桀武王伐紂以興商周之業曹孟德養於宦官莫知所出卒立魏氏之基|{
	曹操事見六十八卷漢靈帝中平元年操字孟德卒子恤翻}
苟非天命安能成功推此而言何必致問裕曰人言冉閔初立鑄金爲已像以卜成敗而像不成信乎煒曰不聞裕曰南來者皆云如是何故隱之煒曰姦僞之人欲矯天命以惑人者乃假符瑞託蓍龜以自重|{
	蓍升脂翻}
魏主握符璽據中州受命何疑而更反眞爲僞取决於金像乎裕曰傳國璽果安在煒曰在鄴裕曰張舉言在襄國煒曰殺胡之日在鄴者殆無孑遺|{
	孑吉列翻孤也單也言無孤單得遺者}
時有迸漏者皆潛伏溝瀆中耳|{
	爾雅水注谷曰溝水注澮曰瀆迸北諍翻}
彼安知璽之所在乎彼求救者爲妄誕之辭無所不可况一璽乎雋猶以張舉之言爲信乃積柴其㫄使裕以其私誘之曰君更熟思無爲徒取灰㓕|{
	誘音酉}
煒正色曰石氏貪暴親帥大兵攻燕國都雖不克而返|{
	事見九十六卷成帝咸康四年帥讀曰率}
然志在必取故運資糧聚器械於東北者非以相資乃欲相㓕也|{
	事見九十六卷咸康四年六年}
魏主誅翦石氏雖不爲燕臣子之心聞仇讐之㓕義當如何而更爲彼責我不亦異乎|{
	異猶言可怪也爲于僞翻}
吾聞死者骨肉下于土|{
	下戶嫁翻}
精䰟升于天蒙君之惠速益薪縱火使僕得上訴於帝足矣左右請殺之雋曰彼不憚殺身以狥其主忠臣也且冉閔有罪使臣何預焉|{
	使疏吏翻}
使出就館使其鄉人趙瞻往勞之|{
	勞力到翻}
且曰君何不以實言王怒欲處君於遼碣之表|{
	遼海及碣石爲遼碣杜佑曰盧龍漢肥如縣有碣石山碣然而立在海㫄秦築長城所起自碣石在今高麗舊界非此碣石也趙瞻所謂遼碣蓋即杜佑所言者也處昌呂翻}
奈何煒曰吾結髪以來尚不欺布衣况人主乎曲意苟合性所不能直情盡言雖沈東海不敢避也|{
	沈持林翻}
遂臥向壁不復與瞻言|{
	復扶又翻}
瞻具以白雋雋乃囚煒於龍城 趙并州刺史張平遣使降秦|{
	使疏吏翻降戶江翻}
秦王以平爲大將軍冀州牧 燕王雋還薊|{
	自龍城還薊薊音計}
三月姚襄及趙汝陰王琨各引兵救襄國|{
	琨自信都進兵救襄國}
冉閔遣車騎將軍胡睦拒襄於長蘆|{
	水經注漳水過堂陽縣西分爲二水其右水東北注出石門謂之長蘆水長蘆水西逕堂陽縣故城南又東逕九門陂又東逕扶都縣五代志隋置長蘆縣屬河間郡劉昫曰長蘆漢參戶縣地}
將軍孫威拒琨於黄丘|{
	魏收地形志鉅鹿郡□縣有黄丘□苦么翻}
皆敗還士卒略盡閔欲自出撃之衛將軍王泰諫曰今襄國未下外救雲集若我出戰必覆背受敵|{
	覆當作腹}
此危道也不若固壘以挫其銳徐觀其釁而撃之|{
	釁隙也}
且陛下親臨行陳|{
	行戶剛翻陳讀曰陣}
如失萬全則大事去矣閔將止道士法饒進曰陛下圍襄國經年無尺寸之功今賊至又避不撃將何以使將士乎|{
	將即亮翻}
且太白入昴當殺胡王|{
	晉天文志昴七星為旄頭胡星也}
百戰百克不可失也閔攘袂大言曰吾戰决矣敢沮衆者斬|{
	攘如羊翻沮在呂翻}
乃悉衆出與襄琨戰悦綰適以燕兵至去魏兵數里疏布騎卒曳柴揚塵|{
	疏讀與踈同騎奇寄翻下同}
魏人望之恟懼|{
	自棘城之敗趙人固畏燕兵見其至而勢盛故恟懼恟許拱翻}
襄琨綰三面撃之趙王祗自後衝之魏兵大敗|{
	果如王泰之言腹背受敵而敗}
閔與十餘騎走還鄴降胡栗特康等執大單于胤及左僕射劉琦以降趙趙王祗殺之|{
	果如韋謏之言史言冉閔不能用羣策以取敗降戶江翻單音蟬}
胡睦及司空石璞尚書令徐機中書監盧諶等并將士死者凡十餘萬人|{
	劉隗盧諶不能爲晉死而卒死於兵人誰不死貴得其死所耳諶是壬翻}
閔潛還人無知者鄴中震恐訛言閔已沒射聲校尉張艾請閔親郊以安衆心|{
	親郊親出郊祀也}
閔從之訛言乃息閔支解灋饒父子|{
	解剝其支體而殺之}
贈韋謏大司徒|{
	謏死見上卷上年謏蘇鳥翻}
姚襄還灄頭|{
	灄書涉翻}
姚弋仲怒其不擒閔杖之一百初閔之爲趙相也|{
	相息亮翻}
悉散倉庫以樹私恩與羌胡相攻無月不戰趙所徙青雍幽荆四州之民|{
	石虎破曹嶷徙青州之民破劉胤石生再徙雍州之民破段匹殫及爲燕所敗徙幽州之民石勒南掠江漢徙荆州之民雍於用翻}
及氐羌胡蠻數百萬口以趙法禁不行各還本土道路交錯互相殺掠其能逹者什有二三中原大亂因以飢疫人相食無復耕者|{
	復扶又翻}
趙王祗使其將劉顯帥衆七萬攻鄴|{
	帥讀曰率}
軍于明光宫|{
	此明光宫石氏所建也}
去鄴二十三里魏主閔恐召王泰欲與之謀泰恚前言之不從辭以瘡甚|{
	戰敗被傷故因以瘡甚辭恚於避翻}
閔親臨問之泰固稱疾篤閔怒還宫謂左右曰巴奴乃公豈假汝爲命邪|{
	王泰蓋巴蠻也乃公冉閔自謂也自漢高祖已有是語}
要將先滅羣胡却斬王泰乃悉衆出戰大破顯軍追奔至陽平|{
	陽平縣漢屬東郡魏晉分屬陽平郡而陽平郡治在魏郡東北宋白曰魏州莘縣漢為陽平縣後趙移陽平理館陶縣}
斬首三萬餘級顯懼密使請降|{
	使疏吏翻降戶江翻}
求殺祗以自效閔乃引歸有告王泰欲叛入秦者閔殺之夷其三族 秦王健分遣使者問民疾苦搜羅雋異寛重歛之稅弛離宫之禁|{
	趙修長安宮殿亦有離宮之禁歛力贍翻}
罷無用之器去侈靡之服|{
	去羌呂翻}
凡趙之苛政不便於民者皆除之|{
	史言苻健所以能據有關中}
杜洪張琚遣使召梁州刺史司馬勲夏四月勲帥步騎三萬赴之|{
	帥讀曰率騎奇寄翻}
秦王健禦之於五丈原勲屢戰皆敗退歸南鄭健以中書令賈玄碩始者不上尊號銜之|{
	上時掌翻}
使人告玄碩與司馬勲通并其諸子皆殺之 渤海人逢約|{
	逢皮江翻}
因趙亂擁衆數千家附於魏魏以約爲渤海太守故太守劉凖隗之兄子也土豪封放奕之從弟也|{
	從才用翻}
别聚衆自守閔以凖爲幽州刺史與約中分渤海燕王雋使封奕討約使昌黎太守高開討準放開瞻之子也|{
	高瞻見九十一卷元帝太興二年}
奕引兵直抵約壘遣人謂約曰相與鄉里隔絶日久|{
	封奕本渤海人懷帝永嘉五年託於慕容廆見八十七卷}
會遇甚難時事利害人皆有心非所論也願單出一相見以寫佇結之情|{
	人立而待之曰佇企望之情欝積而不散曰結}
約素信重奕即出見奕於門外各屏騎卒|{
	屏必郢翻騎奇寄翻}
單馬交語奕與論叙平生畢因說之曰與君累世同鄉情相愛重誠欲君享祚無窮今既獲展奉|{
	展省視也奉承也事也說輸芮翻}
不可不盡所懷冉閔乘石氏之亂奄有成資是宜天下服其彊矣而禍亂方始固知天命不可力爭也燕王奕世載德|{
	奕世載德班彪王命論之言師古曰載乘也言相因不絶}
奉義討亂所征無敵今已都薊南臨趙魏遠近之民襁負歸之民厭荼毒|{
	孔安國曰荼毒苦也襁居兩翻}
咸思有道冉閔之亡匪朝伊夕成敗之形昭然易見|{
	易以豉翻}
且燕王肇開王業虛心賢雋君能翻然改圖則功參絳灌慶流苖裔孰與爲亡國將守孤城以待必至之禍哉|{
	將即亮翻}
約聞之悵然不言奕給使張安有勇力|{
	給使在左右給使令者也}
奕豫戒之俟約氣下安突前持其馬鞚|{
	鞚空貢翻}
因挾之而馳至營奕與坐謂曰君計不能自决故相為决之|{
	爲于僞翻}
非欲取君以邀功乃欲全君以安民也高開至渤海凖放迎降|{
	降戶江翻}
雋以放爲渤海太守凖爲左司馬約參軍事以約誘於人而遇獲|{
	誘音酉}
更其名曰釣|{
	更工衡翻}
劉顯弑趙王祗及其丞相樂安王炳太宰趙庶等十餘人傳首于鄴驃騎將軍石寜奔栢人|{
	栢人縣自漢以來屬趙國劉昫曰唐邢州堯山縣古之柏人城驃匹妙翻騎奇寄翻}
魏主閔焚祗首于通衢拜顯上大將軍大單于冀州牧|{
	單音蟬}
五月趙兖州刺史劉啓自鄄城來奔|{
	鄄城縣漢屬濟隂郡晉屬濮陽國唐爲濮州治所鄄吉縣翻}
秋七月劉顯復引兵攻鄴|{
	復扶又翻}
魏主閔撃敗之|{
	敗蒲邁翻}
顯還稱帝於襄國 八月魏徐州刺史周成兖州刺史魏統荆州刺史樂弘豫州牧張遇以廪丘許昌等諸城來降|{
	時周成據廪丘張遇據許昌降戶江翻}
平南將軍高崇征虜將軍呂護執洛州刺史鄭系以其地來降|{
	時崇護以三河之地來降}
燕王雋遣慕容恪攻中山慕容評攻王午于魯口魏中山太守上谷侯龕閉城拒守|{
	龕苦含翻}
恪南狥常山軍于九門|{
	九門縣自漢以來屬常山郡}
魏趙郡太守遼西李邽舉郡降恪厚撫之將邽還圍中山侯龕乃降恪入中山遷其將帥土豪數十家詣薊|{
	將即亮翻帥所類翻薊音計}
餘皆安堵軍令嚴明秋毫不犯慕容評至南安王午遣其將鄭生拒戰評撃斬之悦綰還自襄國雋乃知張舉之妄而殺之常煒有四男二女在中山雋釋煒之囚使諸子就見之煒上疏謝恩雋手令答曰卿本不爲生計孤以州里相存耳|{
	雋居昌黎煒居廣甯二郡皆屬幽州}
今大亂之中諸子盡至豈非天所念邪天且念卿况於孤乎賜妾一人穀三百斛使居凡城以北平太守孫興爲中山太守興善於綏撫中山遂安 庫傉官偉帥部衆自上黨降燕|{
	傉奴沃翻庫傉官漁陽烏桓大人庫傉之餘種按溫公與劉道原書以為庫當作厙詳見前例厙音舍}
姚弋仲遣使來請降|{
	趙亡弋仲乃降晉史言其盡忠於石氏使疏吏翻下同}
冬十月以弋仲爲使持節六夷大都督督江北諸軍事|{
	江北恐當作河北}
車騎大將軍開府儀同三司大單于高陵郡公|{
	騎奇寄翻單音蟬}
又以其子襄爲持節平北將軍都督并州諸軍事并州刺史平鄉縣公 逢釣亡歸渤海招集舊衆以叛燕樂陵太守賈堅 |{
	考異曰燕書賈堅傳烈祖問堅年對以受新命始及三載烈祖悦其言拜樂陵太守按堅以去年九月獲於燕至明年始三年若未為樂陵太守豈能安集諸縣告諭逢釣故知堅先已為樂陵太守非因問年而授}
使人告諭鄉人示以成敗釣部衆稍散遂來奔 吐谷渾葉延卒子碎奚立|{
	晉書作辟奚按一百三卷簡文帝咸安元年鍾惡地殺三弟事亦當作辟奚}
初桓温聞石氏亂上疏請出師經略中原|{
	温蓋上疏於五年出屯安陸之時}
事久不報温知朝廷杖殷浩以抗已甚忿之然素知浩之爲人亦不之憚也以國無他釁遂得相持彌年覊縻而已八州士衆資調殆不爲國家用|{
	永和元年温都督荆司雍益梁寜六州五年遣滕峻帥交廣之兵伐林邑蓋是時已加督交廣二州矣資財也調賦也調徒釣翻}
屢求北伐詔書不聼十二月辛未温拜表輒行帥衆四五萬順流而下軍於武昌|{
	帥讀曰率}
朝廷大懼殷浩欲去位以避温又欲以騶虞幡駐温軍吏部尚書王彪之言於會稽王昱曰此屬皆自為計非能保社稷為殿下計也|{
	會工外翻}
若殷浩去職人情離駭天子獨坐當此之際必有任其責者非殿下而誰乎又謂浩曰彼若抗表問罪卿為之首事任如此|{
	謂浩當朝政也}
猜釁已成|{
	謂浩與温有隙也}
欲作匹夫豈有全地邪且當靜以待之令相王與手書|{
	相息亮翻}
示以欵誠為陳成敗彼必旋師若不從則遣中詔又不從乃當以正義相裁|{
	謂正温舉兵向闕之罪}
奈何無故忩忩先自猖獗乎|{
	忩倉紅翻}
浩曰决大事正自難頃日來欲使人悶聞卿此謀意始得了|{
	了决也}
彪之彬之子也|{
	王敦之亂彬能守正彪之可謂克紹矣}
撫軍司馬高崧|{
	昱為撫軍大將軍以崧為司馬}
言於昱曰王宜致書諭以禍福自當返斾如其不爾便六軍整駕逆順於兹判矣|{
	言温若不還則當整六師奉順討逆也}
乃於坐爲昱草書曰宼難宜平時會宜接|{
	謂是時中原豪傑來降有恢復之會所宜應接之也難乃旦翻坐狙臥翻為于僞翻下同}
此實為國遠圖經略大筭能弘斯會非足下而誰但以比興師動衆要當以資實為本|{
	比毗寐翻}
運轉之艱古人所難不可易之於始而不熟慮|{
	易以䜴翻}
頃所以深用爲疑惟在此耳然異常之舉衆之所駭遊聲噂?|{
	噂祖本翻?徒合翻噂?聚語也}
想足下亦少聞之|{
	少詩沼翻}
苟患失之無所不至|{
	論語孔子之言}
或能望風振擾一時崩散如此則望實並喪|{
	喪息浪翻}
社稷之事去矣皆由吾闇弱德信不著不能鎭靜羣庶保固維城|{
	詩曰宗子維城}
所以内愧於心外慙良友吾與足下雖職有内外安社稷保國家其致一也|{
	致極也言事理詣極之地則一也}
天下安危繋之明德當先思寜國而後圖其外使王基克隆大義弘著所望於足下區區誠懷豈可復顧嫌而不盡哉|{
	復扶又翻}
温即上疏惶恐致謝回軍還鎭 朝廷將行郊祀會稽王昱問於王彪之曰郊祀應有赦否彪之曰自中興以來郊祀往往有赦愚意常謂非宜凶愚之人以為郊必有赦將生心於儌幸矣|{
	儌堅堯翻}
昱從之 燕王雋如龍城 丁零翟鼠帥所部降燕封為歸義王|{
	丁零居中山其後翟斌等皆其種類也帥讀曰率}


八年春正月辛卯日有食之 秦丞相雄等請秦王健正尊號依漢晉之舊不必效石氏之初|{
	謂石虎兄弟皆先稱天王後即皇帝位}
健從之即皇帝位大赦諸公皆進爵為王且言單于所以統一百蠻非天子所宜領|{
	此亦雄等之言也單音蟬}
以授太子萇 司馬勲既還漢中杜洪張琚屯宜秋|{
	水經注鄭渠自中山西瓠口東流逕宜秋城北又東逕中山南又東逕池陽縣故城北}
洪自以右族輕琚琚遂殺洪自立爲秦王改元建昌 劉顯攻常山魏主閔留大將軍蔣幹使輔太子智守鄴自將八千騎救之顯大司馬清河王寧以棗彊降魏|{
	棗彊縣前漢屬清河郡後漢晉省尋復置屬信都郡}
閔撃顯敗之|{
	敗補邁翻}
追奔至襄國顯大將軍曹伏駒開門納閔閔殺顯及其公卿已下百餘人 |{
	考異曰閔殺顯晉帝紀在正月十六國春秋鈔在二月燕王在三月己酉未知孰是今從帝紀}
焚襄國宫室遷其民於鄴趙汝隂王琨以其妻妾來奔斬於建康市石氏遂絶|{
	自古無不亡之國宗族誅夷固亦有之未有至於絶姓者石氏窮凶極暴而子孫無遺種足以見天道之不爽矣}
尚書左丞孔嚴言於殷浩曰比來衆情良可寒心|{
	比毗至翻}
不知使君當何以鎭之愚謂宜明受任之方韓彭專征伐蕭曹守管籥|{
	事見漢高帝紀曹參當高帝時從韓信用兵其後相齊未嘗守管籥嚴以蕭曹相繼爲相而言之}
内外之任各有攸司深思廉藺屈身之義|{
	事見四卷周赧王三十六年}
平勃交歡之謀|{
	事見十三卷漢高后七年}
令穆然無間|{
	穆然和而靜之貌間古莧翻}
然後可以保大定功也|{
	嚴欲浩與桓温兩釋猜嫌降心相從以圖國事也保大定功左傳楚莊王所謂武有七德此其二也}
觀近日降附之徒皆人面獸心貪而無親恐難以義感也|{
	段龕張遇姚襄之徒孔嚴固見其肺肝矣降戶江翻}
浩不從嚴愉之從子也|{
	從才用翻}
浩上疏請北出許洛詔許之以安西將軍謝尚北中郎將荀羨爲督統|{
	晉志曰四中郎將並後漢置歷魏及晉並有其職江左彌重時謝尚鎭壽春荀羨鎭京口浩欲兩道俱進故使二人並為督統各統其方之兵}
進屯壽春謝尚不能撫尉張遇|{
	尉與慰同}
遇怒據許昌叛使其將上官恩據洛陽樂弘攻督護戴施於倉垣浩軍不能進三月命荀羨鎭淮隂尋加監青州諸軍事|{
	監工銜翻}
又領兖州刺史鎭下邳 乙巳燕王雋還薊稍徙軍中文武兵民家屬於薊|{
	自北徙其家屬而南又恐其懷居而無樂遷之心故稍徙之}
姚弋仲有子四十二人及病謂諸子曰石氏待吾厚吾本欲爲之盡力|{
	爲于僞翻}
今石氏已滅中原無主我死汝亟自歸於晉當固執臣節無爲不義也弋仲卒子襄祕不發喪帥戶六萬南攻陽平元城發干破之屯于碻磝津|{
	帥讀曰率下同襄自灄頭而南也元城縣漢屬魏郡晉屬陽平郡發干縣漢屬東郡晉屬陽平郡劉昫曰唐魏州莘縣漢陽平縣地碻磝城即漢東郡荏平縣故城其西南即河津謂之碻磝津後魏置濟州於碻磝城杜佑曰碻磝即今濟陽郡城碻口交翻磝音敖楊正衡曰碻五勞翻磝口勞翻毛晃曰碻丘交翻磝牛交翻或曰碻音確磝音爻}
以太原王亮為長史天水尹赤為司馬太原薛瓚略陽權翼為參軍|{
	姓譜權本顓頊之後楚武王使闘緡尹權因以為氏韓愈權德輿墓碑曰殷武丁之子降封於權權江漢間國也周衰入楚為權氏瓚藏旱翻}
襄與秦兵戰敗亡三萬餘戶南至滎陽始發喪又與秦將高昌李歷戰于麻田|{
	高昌李歷本趙將也時附於秦故稱秦將滎洛之間地名有豆田麻田各因人所種藝而名之}
馬中流矢而斃|{
	中竹仲翻}
弟萇以馬授襄|{
	萇仲良翻}
襄曰汝何以自免萇曰但令兄濟豎子必不敢害萇會救至俱免尹赤奔秦秦以赤爲并州刺史鎭蒲阪襄遂帥衆歸晉送其五弟爲質|{
	質音致}
詔襄屯譙城襄單騎度淮見謝尚于壽春尚聞其名命去仗衛幅巾待之|{
	騎奇寄翻去丘呂翻}
歡若平生襄博學善談論江東人士皆重之魏主閔既克襄國因遊食常山中山諸郡趙立義將軍段勤聚胡羯萬餘人保據繹幕|{
	繹幕縣自漢以來屬清河郡}
自稱趙帝夏四月甲子燕王雋遣慕容恪等撃魏慕容霸等擊勤魏主閔將與燕戰大將軍董閏車騎將軍張温諫曰|{
	騎奇寄翻下同}
鮮卑乘勝鋒銳且彼衆我寡宜且避之俟其驕惰然後益兵以撃之閔怒曰吾欲以此衆平幽州斬慕容雋今遇恪而避之人謂我何司徒劉茂特進郎闓相謂曰吾君此行必不還矣吾等何為坐待戮辱皆自殺|{
	闓苦亥翻又音開}
閔軍于安喜|{
	安喜縣前漢曰安險屬中山郡後漢章帝更名唐復爲安險縣屬定州而定州所治之安喜縣漢盧奴縣也}
慕容恪引兵從之閔趣常山|{
	趣七喻翻下同}
恪追之及于魏昌之廉臺|{
	魏昌縣屬中山郡本苦陘漢章帝改爲漢昌魏文帝改爲魏昌唐爲定州唐昌縣魏收地形志中山毋極縣有廉臺蓋晉省無極縣廉臺遂在魏昌界}
閔與燕兵交戰燕兵皆不勝閔素有勇名所將兵精銳|{
	將即亮翻}
燕人憚之慕容恪廵陳|{
	陳讀曰陣}
謂將士曰冉閔勇而無謀一夫敵耳其士卒飢疲甲兵雖精其實難用不足破也閔以所將多步卒|{
	將即亮翻}
而燕皆騎兵引兵將趣林中恪參軍高開曰吾騎兵利平地若閔得入林不可復制|{
	復扶又翻}
宜亟遣輕騎邀之既合而陽走誘致平地然後可撃也|{
	誘音酉}
恪從之魏兵還就平地恪分軍為三部謂諸將曰閔性輕銳又自以衆少必致死於我我厚集中軍之陳以待之|{
	少詩沼翻陳讀曰陣下同}
俟其合戰卿等從㫄撃之無不克矣乃擇鮮卑善射者五千人以鐵鎻連其馬為方陳而前閔所乘駿馬曰朱龍日行千里閔左操兩刃矛右執鉤戟以撃燕兵斬首三百餘級望見大幢知其為中軍直衝之燕兩軍從㫄夹撃大破之|{
	恪以鐵鎻連馬則閔兵雖致死而陳不可破兩軍從㫄夹撃則閔兵三面受敵不敗何待操千刀翻幢直江翻}
圍閔數重|{
	重直龍翻}
閔潰圍東走二十餘里朱龍忽斃爲燕兵所執燕人殺魏僕射劉羣執董閔張温及閔皆送於薊|{
	董閔當作董閏冉閔自立事始上卷六年至是而㓕}
閔子操奔魯口高開被創而卒|{
	創初良翻}
慕容恪進屯常山雋命恪鎭中山己卯冉閔至薊雋大赦立閔而責之曰汝奴僕下才何得妄稱帝閔曰天下大亂爾曹夷狄禽獸之類猶稱帝况我中土英雄何得不稱帝邪雋怒鞭之三百送於龍城慕容覇軍至繹幕段勤與弟思聰舉城降|{
	降戶江翻下同}
甲申雋遣慕容評及中尉侯龕帥精騎萬人攻鄴|{
	龕苦含翻}
癸巳至鄴魏蔣幹及太子智閉城拒守城外皆降於燕劉寜及弟崇帥胡騎三千奔晉陽|{
	劉寜劉顯將也以棗強降閔帥讀曰率}
秦以張遇為征東大將軍豫州牧 五月秦主健攻張琚於宜秋斬之 鄴中大飢人相食故趙時宫人被食略盡|{
	被皮義翻}
蔣幹使侍中繆嵩詹事劉猗奉表請降且求救於謝尚|{
	繆靡幼翻}
庚寅燕王雋遣廣威將軍慕容軍殿中將軍慕輿根右司馬皇甫眞等帥步騎二萬助慕容評攻鄴 辛卯燕人斬冉閔於龍城會大旱蝗燕王雋謂閔爲祟|{
	祟雖遂翻神禍曰祟}
遣使祀之諡曰悼武天王 初謝尚使戴施據枋頭|{
	遣施之時指令據枋頭}
施聞蒋幹求救乃自倉垣徙屯棘津|{
	棘津即石濟南津有棘津亭}
止幹使者求傳國璽|{
	璽斯氏翻}
劉猗使繆嵩還鄴白幹幹疑尚不能救沈吟未决|{
	沈持林翻}
六月施帥壮士百餘人入鄴助守三臺紿之曰今燕寇在外道路不通璽未敢送也卿且出以付我我當馳白天子天子聞璽在吾所信卿至誠必多發兵糧以相救餉幹以爲然出璽付之施宣言使督護何融迎糧隂令懷璽送于枋頭|{
	江南之未得璽也中原謂之白板天子傳國璽至此歸晉藺相如全璧歸趙趙王擢之自繆賢舍人為上大夫戴施能復致累代傳國之寶未聞晉朝以顯賞甄之也何居}
甲子蔣幹帥銳卒五千及晉兵出戰慕容評大破之斬首四千級幹脫走入城 甲申秦主健還長安|{
	自宜秋還長安也}
謝尚姚襄共攻張遇于許昌秦主健遣丞相東海王雄衛大將軍平昌王菁略地關東帥步騎二萬救之丁亥戰于潁水之誡橋|{
	據晉紀誡橋在許昌}
尚等大敗死者萬五千人尚奔還淮南襄弃輜重送尚于芍陂|{
	重直用翻}
尚悉以後事付襄|{
	謝尚既敗姚襄知晉之不足恃固有去晉之心矧殷浩又從而速之乎}
殷浩聞尚敗退屯夀春秋七月秦丞相雄徙張遇及陳潁許洛之民五萬餘戶於關中|{
	張遇據有許潁豈肯歛手受羈制於人乎苻雄乘勝以兵威徙之自此遇之死命制於苻氏矣}
以右衛將軍楊羣為豫州刺史鎭許昌謝尚降號建威將軍 趙故西中郎將王擢遣使請降拜擢秦州刺史|{
	王擢自石虎時當秦隴之任降戶江翻下同}
丁酉以武陵王晞為太宰 丙辰燕王雋如中山王午聞魏敗時鄧恒已死|{
	恒戶登翻}
午自稱安國王八月

戊辰燕王雋遣慕容恪封奕陽騖攻之午閉城自守送冉操詣燕軍燕人掠其禾稼而還|{
	慕容恪善用兵知魯口之未可取徒久攻以斃士卒故掠其禾稼全師而退金城湯池非粟不守孤城之外春取其麥而秋取其禾彼將焉仰哉還從宣翻又如字}
庚午魏長水校尉馬願等開鄴城納燕兵戴施蒋幹懸縋而下|{
	縋直僞翻}
犇于倉垣慕容評送魏后董氏太子智太尉申鍾司空條枚等|{
	條姓也周亞夫封條侯其後以為氏}
及乘輿服御于薊|{
	乘繩證翻}
尚書令王簡左僕射張乾右僕射郎肅皆自殺燕王雋詐云董氏得傳國璽獻之賜號奉璽君賜冉智爵海賓侯以申鍾爲大將軍右長史命慕容評鎭鄴 桓温使司馬勲助周撫討蕭敬文於涪城斬之|{
	蕭敬文據涪城始九十七卷永和三年涪音浮}
謝尚自枋頭迎傳國璽至建康百僚畢賀 秦以雷弱兒爲大司馬毛貴爲太尉張遇爲司空 殷浩之北伐也中軍將軍王羲之以書止之不聼既而無功復謀再舉|{
	復扶又翻下所復故復同}
羲之遺浩書曰今以區區江左天下寒心固已久矣|{
	寒心者恐不能自保遺于季翻}
力爭武功非所當作|{
	作為也}
自頃處内外之任者|{
	處昌呂翻下而處同}
未有深謀遠慮而疲竭根本各從所志竟無一功可論遂令天下將有土崩之勢任其事者豈得辭四海之責哉|{
	言殷浩不得辭其責也}
今軍破於外資竭於内保淮之志非所復及莫若還保長江督將各復舊鎭|{
	將即亮翻}
自長江以外覊縻而已引咎責躬更爲善治|{
	治直吏翻}
省其賦役與民更始庶可以救倒懸之急也|{
	保江之說此王導佐元帝之規摹世之議者譏其忘讐忍耻置中原於度外若以量時度力保固本根言之此策未爲非也至於引咎責躬省民賦役所謂善敗不亡諸葛孔明街亭喪師之後正亦如是而已}
使君起於布衣任天下之重當董統之任而敗喪至此|{
	喪息浪翻}
恐闔朝羣賢未有與人分其謗者|{
	朝直遥翻}
若猶以前事爲未工故復求之分外|{
	分扶問翻}
宇宙雖廣自容何所此愚智所不解也|{
	解胡買翻曉也其後殷浩廢黜卒如羲之之言}
又與會稽王昱牋曰爲人臣誰不願尊其主比隆前世况遇難得之運哉顧力有所不及豈可不權輕重而處之也|{
	處昌呂翻}
今雖有可喜之會内求諸已而所憂乃重於所喜功未可期遺黎殱盡勞役無時徵求日重以區區吳越經緯天下十分之九不亡何待而不度德量力|{
	度徒洛翻量音良}
不弊不已此封内所痛心歎悼而莫敢吐誠者也往者不可諫來者猶可追|{
	論語載楚狂接輿之言}
願殿下更垂王思先為不可勝之基須根立勢舉謀之未晩|{
	兵法曰先為不可勝以待敵之可勝}
若不行|{
	此處文意短蹙恐有脫文}
恐麋鹿之游將不止林藪而已|{
	吳伍子胥曰臣恐麋鹿游於姑蘇此祖其意而微其言澤無水曰藪}
願殿下蹔廢虛遠之懷|{
	羲之此言蓋譏昱好談清虚玄遠也蹔與暫同}
以救倒懸之急可謂以亡為存轉禍為福也不從九月浩屯泗口遣河南太守戴施據石門滎陽太守劉遯據倉垣浩以軍興罷遣太學生徒學校由此遂廢|{
	元帝建武元年始立太學今復以軍興廢校戶敎翻}
冬十月謝尚遣冠軍將軍王俠攻許昌克之|{
	冠古玩翻侠戶頰翻}
秦豫州刺史楊羣退屯弘農徵尚為給事中戍石頭 丁卯燕王雋還薊 故趙將擁兵據州郡者各遣使降燕|{
	將即亮翻使疏吏翻降戶江翻}
燕王雋以王擢為益州刺史夔逸為秦州刺史張平為并州刺史李歷為兖州刺史高昌爲安西將軍劉寜為車騎將軍 慕容恪屯安平|{
	安平縣前漢屬涿郡後漢屬安平國晉屬博陵郡唐屬深州}
積糧治攻具將討王午|{
	治直之翻}
丙戌中山蘇林起兵於無極|{
	無極縣漢屬中山國晉省無本作毋唐武后萬歲通天二年始改毋字爲無此當作毋}
自稱天子恪自魯口還討林閏月戊子燕王雋遣廣威將軍慕輿根助恪攻林斬之王午為其將秦興所殺呂護殺興復自稱安國王|{
	復扶又翻}
燕羣僚共上尊號於燕王雋雋許之|{
	上時掌翻}
十一月丁卯始置百官以國相封奕為太尉左長史陽騖為尚書令右司馬皇甫眞為尚書左僕射典書令張悕為右僕射|{
	悕香衣翻}
其餘文武拜授有差戊辰雋即皇帝位大赦自謂獲傳國璽改元元璽追尊武宣王為高祖武宣皇帝文明王為太祖文明皇帝|{
	廆諡武宣王皝諡文明王}
時晉使適至燕|{
	使疏吏翻}
雋謂曰汝還白汝天子我承人乏為中國所推已為帝矣|{
	謂中國無主已爲士民所推遂承人乏而即尊位也}
改司州為中州建留臺於龍都|{
	趙置司州於鄴燕初都龍城時遷于薊故建留臺於龍城謂之龍都}
以玄菟太守乙逸爲尚書專委留務|{
	菟同都翻}
秦丞相雄攻王擢于隴西擢奔凉州雄還屯隴東|{
	隴東漢汧縣地}
張重華以擢為征虜將軍秦州刺史特寵待之|{
	重華寵待王擢以圖秦隴豈知擢非苻雄之敵也}


九年春正月乙卯朔大赦 二月庚子燕主雋立其妃可足渾氏為皇后|{
	為可足渾后亂燕張本可足渾北方三字姓}
世子曄為皇太子皆自龍城遷于薊宫 張重華遣將軍張弘宋修會王擢帥步騎萬五千伐秦|{
	帥讀曰率騎奇寄翻}
秦丞相雄衛將軍菁拒之大敗凉兵於龍黎|{
	新唐書地理志隴州吳山縣有龍盤府龍盤城吳山後魏之南由縣地疑龍黎即龍盤也敗補邁翻下所敗同}
斬首萬二千級虜張弘宋修王擢弃秦州犇姑臧秦主健以領軍將軍苻願為秦州刺史鎭上邽 三月交州刺史阮敷討林邑破五十餘壘 趙故衛尉常山李犢聚衆數千人叛燕 西域胡劉康詐稱劉曜子聚衆於平陽自稱晉王夏四月秦左衛將軍苻飛討擒之 以安西將軍謝尚爲尚書僕射 五月張重華復使王擢帥衆二萬伐上邽|{
	復扶又翻}
秦州郡縣多應之苻願戰敗奔長安重華因上疏請伐秦詔進重華凉州牧 燕主雋遣衛將軍恪討李犢犢降|{
	降戶江翻下同}
遂東擊呂護於魯口 六月秦苻飛攻氐王楊初於仇池為初所敗|{
	楊初據險以拒秦秦兵雖彊故爲初所敗敗補邁翻}
丞相雄平昌王菁帥步騎四萬屯于隴東秦主健納張遇繼母韓氏爲昭儀數於衆中謂遇曰|{
	數所角翻}
卿吾假子也遇耻之因雄等精兵在外隂結關中豪傑欲㓕苻氏以其地來降秋七月遇與黄門劉晃謀夜襲健晃約開門以待之會健使晃出外晃固辭不得已而行遇不知引兵至門門不開事覺伏誅於是孔持起池陽劉珍夏侯顯起鄠喬秉起雍胡陽赤起司竹呼延毒起灞城|{
	池陽縣漢屬馮翊晉屬扶風唐爲雲陽縣屬京兆鄠縣漢屬扶風晉屬始平郡唐屬京兆雍縣漢屬扶風唐改爲天興縣爲鳳翔府治所霸陵縣漢屬京兆晉改曰霸城喬秉載記作喬景避唐諱也孔持作孔特鄠音戶雍於用翻}
衆數萬人各遣使來請兵|{
	使疏吏翻下同}
秦以左僕射魚遵為司空 九月秦丞相雄帥衆二萬還長安|{
	帥讀曰率下同}
遣平昌王菁略定上洛置荆州于豐陽川|{
	上洛縣漢西都屬弘農郡東漢屬京兆武帝泰始二年分置上洛郡豐陽川在郡界續漢志南陽郡析縣有豐陽城後魏太安二年置豐陽縣左傳所謂司馬起豐析即其地劉昫曰唐商州豐陽縣漢商縣地晉分商縣置豐陽縣因豐陽川爲名}
以步兵校尉金城郭敬為刺史雄與清河王灋苻飛分討孔持等 姚襄屯歷陽以燕秦方彊未有北伐之志乃夹淮廣興屯田訓厲將士|{
	將即亮翻}
殷浩在夀春惡其彊盛|{
	惡烏路翻下同}
囚襄諸弟屢遣刺客刺之|{
	刺之七亦翻}
刺客皆以情告襄安北將軍魏統卒|{
	魏統來降見上七年}
弟憬代領部曲浩潜遣憬帥衆五千襲之襄斬憬并其衆|{
	憬古迥翻}
浩愈惡之使龍驤將軍劉啓守譙|{
	驤思將翻}
遷襄于梁國蠡臺|{
	司馬彪郡國志睢陽縣有盧門亭城内有高臺甚秀廣巍然介立超焉獨上謂之蠡臺杜預曰盧門宋城南門也續述征記曰迴道似蠡故謂之蠡臺蠡如字若如述征記之說音盧戈翻}
表授梁國内史魏憬子弟數往來夀春|{
	數所角翻}
襄益疑懼遣參軍權翼使於浩浩曰身與姚平北共為王臣休戚同之平北每舉動自專甚失輔車之理豈所望也|{
	左傳曰輔車相依杜預注曰輔頰輔車牙車車尺奢翻}
翼曰平北英姿絶世擁兵數萬遠歸晉室者以朝廷有道宰輔明哲故也今將軍輕信讒慝之言與平北有隙愚謂猜嫌之端在此不在彼也浩曰平北姿性豪邁生殺自由又縱小人掠奪吾馬王臣之體固若是乎翼曰平北歸命聖朝|{
	朝直遥翻}
豈肯妄殺無辜姦宄之人亦王灋所不容也殺之何害浩曰然則掠馬何也翼曰將軍謂平北雄武難制終將討之故取馬欲以自衛耳浩笑曰何至是也|{
	權翼之言得浩之情故笑史言浩不能綏御新附}
初浩隂遣人誘梁安雷弱兒使殺秦主健許以關右之任弱兒僞許之且請兵應接浩聞張遇作亂健兄子輔國將軍黄眉自洛陽西奔以為安等事已成冬十月浩自壽春帥衆七萬北伐|{
	帥讀曰率}
欲進據洛陽修復園陵吏部尚書王彪之上會稽王昱牋|{
	上時掌翻}
以爲弱兒等容有詐偽浩未應輕進不從|{
	藉使梁雷果受浩間而殺健浩亦未能越關陕以取長安其欲乘苻黄眉之去而據洛陽不過欲以修復園陵爲功耳昱遂以爲眞可立功而不聼王彪之之言宜桓温得因以廢浩而制昱也}
浩以姚襄為前驅襄引兵北行度浩將至|{
	度徒洛翻}
詐令部衆夜遁隂伏甲以邀之浩聞而追襄至山桑|{
	山桑縣前漢屬沛郡後漢屬汝南郡晉屬譙郡按山桑六朝兵爭為渦陽之地唐為亳州蒙城縣地}
襄縱兵撃之浩大敗弃輜重走保譙城|{
	重直龍翻}
襄俘斬萬餘悉收其資仗使兄益守山桑襄復如淮南|{
	復扶又翻}
會稽王昱謂王彪之曰君言無不中|{
	中竹仲翻}
張陳無以過也|{
	張陳謂張良陳平}
西平敬烈公張重華有疾子曜靈纔十歲立為世子赦其境内重華庶兄長寜侯祚有勇力吏幹|{
	水經注金城西平西北四十里有長寧亭晉室西平郡有長寧縣}
而傾巧善事内外與重華嬖臣趙長尉緝等結異姓兄弟|{
	尉姓也左傳有鄭大夫尉止嬖卑義翻又博計翻}
都尉常據請出之重華曰吾方以祚為周公使輔幼子君是何言也|{
	託孤之難尚矣况張重華乎}
謝艾以枹罕之功|{
	事見九十七卷永和三年枹音膚}
有寵於重華左右疾之譛艾出為酒泉太守艾上疏言權倖用事公室將危乞聼臣入侍且言長寜侯祚及趙長等將爲亂宜盡逐之十一月己未重華疾甚手令徵艾爲衛將軍監中外諸軍事輔政|{
	監工銜翻}
祚長等匿而不宣丁卯重華卒世子曜靈立稱大司馬凉州刺史西平公趙長等矯重華遺令以長寜侯祚為都督中外諸軍事撫軍大將軍輔政|{
	史言張氏之亂}
殷浩使部將劉啓王彬之攻姚益于山桑姚襄自淮南撃之啓彬之皆敗死|{
	啓劉輿之孫也}
襄進據芍陂 趙末樂陵朱秃平原杜能清河丁嬈|{
	嬈乃了翻又如紹翻}
陽平孫元各擁兵分據城邑至是皆請降於燕|{
	降戶江翻}
燕主雋以秃爲青州刺史能爲平原太守嬈為立節將軍元爲兖州刺史各留撫其營 秦丞相雄克池陽斬孔持十二月清河王灋苻飛克鄠斬劉珍夏侯顯 姚襄濟淮屯盱眙|{
	盱眙音吁怡}
招掠流民衆至七萬分置守宰勸課農桑|{
	姚襄所為僅如此而晉人已為之震懼蓋姧雄所竊笑也}
遣使詣建康罪狀殷浩并自陳謝|{
	使疏吏翻}
詔以謝尚都督江西淮南諸軍事豫州刺史鎭歷陽|{
	以尚得襄之歡心既以招撫之又以備之}
凉右長史趙長等建議以為時難未夷宜立長君|{
	難乃旦翻長知兩翻}
曜靈冲幼請立長寜侯祚張祚先得幸於重華之母馬氏馬氏許之乃廢張曜靈為凉寜侯立祚為大都督大將軍凉州牧凉公祚既得志恣為淫虐殺重華妃裴氏及謝艾|{
	淫者烝其君母虐者殺裴妃謝艾即此二端他所淫虐又其餘毒也}
燕衛將軍恪撫軍將軍軍左將軍彪等屢薦給事黄門侍郎霸有命世之才宜總大任是歲燕主雋以霸為使持節安東將軍冀州刺史鎭常山|{
	冀州刺史鎭信都今置北冀州於常山}


十年春正月張祚自稱凉王改建興四十二年為和平元年|{
	河西張氏乃心晉室奉建興年號至四十餘年張祚凶淫改元僭擬祖父之所不相也}
立妻辛氏為王后子太和為太子封弟天錫為長寜侯子庭堅為建康侯|{
	建康郡蓋張氏所置張茂分屬凉州}
曜靈弟玄靚為凉武侯|{
	靚疾正翻}
置百官郊祀天地用天子禮樂尚書馬岌切諫坐免官|{
	岌魚及翻}
郎中丁琪復諫曰我自武公以來|{
	張軌諡武公復扶又翻}
世守臣節抱忠履謙五十餘年|{
	履謙謂未嘗建國自王也惠帝永寧元年張軌鎭凉土至是五十四年}
故能以一州之衆抗舉世之虜師徒歲起民不告疲殿下勲德未高於先公而亟謀革命臣未見其可也彼士民所以用命四遠所以歸嚮者以吾能奉晉室故也今而自尊則中外離心安能以一隅之地拒天下之彊敵乎祚大怒斬之於闕下|{
	自古戮諫臣未有不亡者也}
故魏降將周成反|{
	周成降見上七年降戶江翻將即亮翻}
自宛襲洛陽

|{
	宛於元翻}
辛酉河南太守戴施奔鮪渚|{
	水經注河水過河南鞏縣北有山臨河謂之崟原丘其下有宂謂之鞏宂言潜通浦北逹于河直宂有渚謂之鮪渚鮪于軌翻}
秦丞相雄克司竹胡陽赤奔霸城依呼延毒 中軍將軍揚州刺史殷浩連年北伐師徒屢敗糧械都盡征西將軍桓温因朝野之怨上疏數浩之罪請廢之|{
	數所具翻}
朝廷不得已免浩為庶人徙東陽之信安|{
	東陽郡本會稽西郡都尉吳孫皓寶鼎元年立郡信安縣漢獻帝初平三年分太末立新安縣武帝太康元年更名信安東陽郡唐為婺州信安縣唐為衢州治所}
自此内外大權一歸於温矣|{
	史言晉氏失權由用殷浩違其才}
浩少與温齊名|{
	少詩照翻}
而心競不相下温常輕之浩既廢黜雖愁怨不形辭色常書空作咄咄怪事字|{
	咄當沒翻咄咄嗟咨語也}
久之温謂掾郗超曰|{
	掾于絹翻郗丑之翻}
浩有德有言嚮為令僕足以儀刑百揆朝廷用違其才耳將以浩爲尚書令以書告之浩欣然許焉將答書慮有謬誤開閉者十數竟逹空函温大怒由是遂絶卒於徙所以前會稽内史王述為揚州刺史|{
	卒子恤翻會工外翻}
二月乙丑桓温統步騎四萬江陵水軍自襄陽入均口至南郷|{
	水經注順陽郡筑陽縣有涉都城沔水逕東北均水於縣入沔謂之均口筑陽與南郷縣漢皆屬南陽郡漢建安中分南陽右壤立南郷郡二縣屬焉武帝更名順陽郡成帝咸康四年復曰南郷郡}
步兵自淅川趣武關|{
	析縣前漢屬弘農郡後漢屬南陽郡春秋之白羽也武關在其西文穎曰武關去析縣百七十里賢曰析即今鄧州内郷縣後魏置淅川縣有淅水後周併入内郷縣}
命司馬勲出子午道以伐秦|{
	命勲從梁州出師}
燕衛將軍恪圍魯口三月拔之呂護奔野王遣弟奉表謝罪於燕燕以護爲河内太守 姚襄遣使降燕|{
	使疏吏翻降戶江翻}
燕王雋以慕容評為鎭南將軍都督秦雍益梁江揚荆徐兖豫十州諸軍事權鎭洛水|{
	雍於用翻}
以慕容彊為前鋒都督督荆徐二州緣淮諸軍事進據河南|{
	此河南謂大河之南}
桓温别將攻上洛獲秦荆州刺史郭敬進撃青泥破之|{
	青泥城在藍田縣南}
司馬勲掠秦西鄙凉秦州刺史王擢攻陳倉以應温|{
	趙亡王擢歸張氏故以凉繋之}
秦主健遣太子萇|{
	萇仲良翻}
丞相雄淮南王生平昌王菁北平王碩帥衆五萬軍于嶢柳以拒温|{
	土地記曰藍田縣南有嶢關地名嶢柳道通荆州晉地道記曰關當上洛縣西北帥讀曰率下同嶢五聊翻}
夏四月己亥温與秦兵戰于藍田秦淮南王生單騎突陳|{
	騎奇寄翻下同陳讀曰陣}
出入以十數殺傷晉將士甚衆温督衆力戰秦兵大敗將軍桓冲又敗秦丞相雄于白鹿原|{
	水經注霸川之西有白鹿原三秦記曰麗山西有白鹿原魏收地形志京兆藍田縣有白鹿原又敗補邁翻}
冲温之弟也温轉戰而前壬寅進至灞上秦太子萇等退屯城南秦主健與老弱六千固守長安小城悉發精兵三萬遣大司馬雷弱兒等與萇合兵以拒温二輔郡縣皆來降|{
	降戶江翻}
温撫諭居民使安堵復業民爭持牛酒迎勞|{
	勞力到翻}
男女夹路觀之耆老有垂泣者曰不圖今日復覩官軍|{
	復扶又翻}
秦丞相雄帥騎七千襲司馬勲於子午谷破之勲退屯女媧堡 戊申燕主雋封撫軍將軍軍為襄陽王左將軍彭為武昌王以衛將軍恪為大司馬侍中大都督錄尚書事封太原王鎭南將軍評為司徒驃騎將軍封上庸王|{
	驃匹妙翻}
封安東將軍霸為吳王左賢王友為范陽王散騎常侍厲為下邳王散騎常侍宜為廬江王|{
	散悉亶翻}
寜北將軍度爲樂浪王|{
	樂浪音洛琅}
又封弟桓為宜都王逮為臨賀王徽為河間王龍為歷陽王納為北海王秀為蘭陵王嶽為安豐王德為梁公默為始安公僂為南康公|{
	僂隴主翻}
子咸為樂安王|{
	參考後卷咸當作臧}
亮為勃海王温為帶方王涉為漁陽王暐為中山王以尚書令陽騖為司空仍守尚書令命冀州刺史吳王霸徙治信都|{
	去年霸治常山}
初燕王皝奇霸之才故名之曰霸將以為世子羣臣諫而止然寵遇猶踰於世子由是雋惡之|{
	惡烏路翻下同}
以其嘗墜馬折齒更名曰|{
	更工衡翻下同傾雪翻}
尋以其應䜟文更名曰垂遷侍中錄留臺事徙鎭龍城垂大得東北之和雋愈惡之復召還|{
	雋雖忌垂卒之復燕祚者垂也天之所置其可廢乎}
五月江西流民郭敞等執陳留内史劉仕降于姚襄|{
	晉南渡後陳留郡寄治譙郡長垣縣界按載記劉仕時在堂邑}
建康震駭以吏部尚書周閔為中軍將軍屯中堂豫州刺史謝尚自歷陽還衛京師固江備守 王擢拔陳倉殺秦扶風内史毛難 北海王猛少好學|{
	少詩照翻好呼到翻}
倜儻有大志不屑細務人皆輕之猛悠然自得隱居華隂|{
	王猛傳猛北海劇人家于魏郡徐統召而不應遂隱于華隂山華隂縣前漢屬京兆後漢晉屬弘農郡倜他狄翻華戶化翻}
聞桓温入關披褐詣之捫蝨而談當世之務|{
	褐毛布蝨色櫛翻}
㫄若無人温異之問曰吾奉天子之命將銳兵十萬為百姓除殘賊|{
	將即亮翻為于偽翻}
而三秦豪傑未有至者何也猛曰公不遠數千里深入敵境今長安咫尺而不度灞水百姓未知公心所以不至温嘿然無以應徐曰江東無卿比也|{
	猛蓋指出温之心事以為温之伐秦但欲以功名鎭服江東非眞有心於伐罪弔民恢復境土不然何以不渡灞水徑攻長安此温所以無以應也然余觀桓温用兵伐秦至覇上伐燕至枋頭皆乘勝進兵逼其國都乃持重觀望卒以取敗蓋温姧雄也乘勝進兵逼其國都冀其望風畏威有内潰之變也逼其國都而敵無内變故持重以待之情見勢屈敵因而乘之故至於敗蘇子由所謂以智遇智則其智不足恃者此也}
乃署猛軍謀祭酒温與秦丞相雄等戰于白鹿原温兵不利死者萬餘人初温指秦麥以為糧既而秦人悉芟麥清野以待之温軍乏食六月丁丑徙關中三千餘戶而歸以王猛為高官督護|{
	職為督護而加之以高官也魏晉之間凡居節鎭者其部將有督護其後又置高官督護王敦鎭武昌有高官督護繆坦}
欲與俱還猛辭不就|{
	猛不肯從温温豈不欲殺之邪蓋温軍已敗怱怱退師不暇殺之也}
呼延毒帥衆一萬從温還|{
	帥讀曰率}
秦太子萇等隨温撃之比至潼關|{
	比必寐翻}
温軍屢敗失亡以萬數温之屯灞上也順陽太守薛珍勸温徑進逼長安|{
	成帝改順陽曰南郷郡既而復舊}
温弗從珍以偏師獨濟頗有所獲及温退乃還顯言於衆自矜其勇而咎温之持重温殺之 秦丞相雄撃司馬勲王擢於陳倉|{
	勲自女媧堡會擢攻陳倉}
勲奔漢中擢奔略陽 秦以光祿大夫趙俱為洛陽刺史鎭宜陽 秦東海敬武王雄攻喬秉于雍|{
	雍於用翻}
丙申卒秦主健哭之嘔血曰天不欲吾平四海邪何奪吾元才之速也|{
	苻雄字元才}
贈魏王葬禮依晉安平獻王故事雄以佐命元勲權侔人主而謙恭汎愛遵奉灋度故健重之常曰元才吾之周公也子堅襲爵|{
	堅襲爵東海王}
堅性至孝幼有志度博學多能交結英豪呂婆樓彊汪及略陽梁平老皆與之善|{
	苻堅事始此}
燕樂陵太守慕容鉤翰之子也|{
	慕容翰有破高句麗㓕宇文之功}
與青州刺史朱禿共治厭次|{
	厭於琰翻}
鉤自恃宗室每陵侮禿禿不勝忿|{
	勝音升}
秋七月襲鉤殺之南奔段龕|{
	為後燕主誅禿張本}
秦太子萇攻喬秉于雍八月斬之關中悉平秦主健賞拒桓温之功以雷弱兒為丞相毛貴為太傅魚遵為太尉淮南王生為中軍大將軍平昌王菁為司空健勤於政事數延公卿|{
	數所角翻}
咨講治道|{
	治直吏翻}
承趙人苛虐奢侈之後易以寛簡節儉崇禮儒士由是秦人悦之 燕大調兵衆|{
	調徒釣翻發也}
因發詔之日號曰丙戌舉 九月桓温還自伐秦帝遣侍中黄門勞温于襄陽|{
	侍中黄門侍郎自魏以來為要近之職勞力到翻}
或告燕黄門侍郎宋斌等謀奉冉智為主而反皆伏誅斌燭之子也|{
	宋燭見九十六卷成帝咸康四年斌音彬}
秦太子萇之拒桓温也為流矢所中|{
	中竹仲翻}
冬十月卒諡曰獻哀 燕主雋如龍城桓温之入關也王擢遣使告凉王祚|{
	使疏吏翻}
言温善用兵其志難測祚懼且畏擢之叛已遣人刺之|{
	刺七亦翻}
事泄祚益懼大發兵聲言東伐實欲西保敦煌|{
	敦徒門翻}
會温還而止既而遣秦州刺史牛霸等帥兵三千擊擢破之十一月擢帥衆降秦|{
	帥讀曰率降戶江翻}
秦以擢為尚書以上將軍啖鐵為秦州刺史|{
	啖氐姓也毛晃曰音徒覽翻}
秦王健叔父武都王安自晉還|{
	健遣安來請朝命見上卷六年}
為姚襄所虜以為洛州刺史十二月安亡歸秦健以安為大司馬驃騎大將軍并州刺史鎭蒲阪|{
	驃匹妙翻}
是歲秦大飢米一升直布一匹

資治通鑑卷九十九
