資治通鑑卷二百六十四
宋 司馬光 撰

胡三省 音註

唐紀八十|{
	起昭陽大淵獻二月盡閤逢困敦閏月凡一年有奇}


昭宗聖穆景文孝皇帝下之上

天復三年二月壬申朔詔比在鳳翔府所除官一切停|{
	比毗至翻近也停所除官者以皆出李茂貞韓全誨之意也}
時宦官盡死惟河東監軍張承業幽州監軍張居翰清海監軍程匡柔西川監軍魚全禋及致仕嚴遵美為李克用劉仁恭楊行密王建所匿得全斬它囚以應詔|{
	禋伊真翻嚴遵美時隱蜀之青城山据通鑑所書程匡柔蓋楊行密匿之}
甲戌門下侍郎同平章事陸扆責授沂王傅分司|{
	沂王禮皇子也禮一作禋}
車駕還京師賜諸道詔書獨鳳翔無之扆曰茂貞罪雖大然朝廷未與之絶今獨無詔書示人不廣 |{
	考異曰舊傳帝還京後赦諸道皆降詔書獨鳳翔無詔扆奏云云按是時未赦恐止是降詔書或赦前扆議如此故胤怒耳}
崔胤怒奏貶之宫人宋柔等十一人皆韓全誨所獻|{
	獻宋柔等見上卷元年}
及僧道士與宦官親厚者二十餘人並送京兆杖殺 上謂韓偓曰崔胤雖盡忠然比卿頗用機數對曰凡為天下者萬國皆屬之耳目|{
	屬之欲翻}
安可以機數欺之莫若推誠直致雖日計之不足而歲計之有餘也|{
	用莊子語}
丙子工部侍郎同平章事蘇檢吏部侍郎盧光啓並賜自盡|{
	蘇檢盧光啓皆鳳翔所命相崔胤惡其黨附韓全誨李茂貞故殺之 考異曰實録檢光啟並賜自盡一說檢長流環州唐太祖紀年録初從幸鳳翔命盧光啓韋貽範為相又命蘇檢平章事及車駕還宫胤積前事怒之不一月皆貶謫之左遷陸扆沂王傅王漙太子賓客蘇檢自盡續寶運録二月五日應是岐王駕前宰臣盧光啓等一百餘人並賜自盡新紀朱全忠殺蘇檢盧光啓舊胤傳昭宗初幸鳳翔命盧光啟韋貽範蘇檢等作相及還京胤皆貶斥之新光啟傳云檢長流環州光啓賜死與寶運録注同檢流環州不見本出何書}
戊寅賜朱全忠號回天再造竭忠守正功臣賜其僚佐敬翔等號迎鑾恊贊功臣諸將朱友寧等號迎鑾果毅功臣都頭以下號四鎮靜難功臣|{
	難乃旦翻}
上議褒崇全忠欲以皇子為諸道兵馬元帥以全忠副之崔胤請以輝王祚為之上曰濮王長|{
	帥所類翻濮博木翻長知兩翻}
胤承全忠密旨利祚沖幼固請之己卯以祚為諸道兵馬元帥 |{
	考異曰金鑾記上曰朕以濮王處長云云新傳帝十七子德王裕棣王祤䖍王禊沂王禋遂王禕景王祕輝王祚祁王祺雅王瓊王祥端王禎豐王祁和王福登王禧嘉王祐潁王禔蔡王祜何皇后生裕及祚餘皆失其母之氏位舊傳云昭宗十子無端王禎以下七人按新舊傳昭宗諸子皆無濮王孫光憲續通歷濮王名紃昭宗之子母曰太后王氏哀帝被殺朱全忠冊紃為天子改元天夀明年禪位於梁此乃光憲傳聞謬誤也昭宗亦無王皇后金鑾記所云濮王蓋德王改封耳}
庚辰加全忠守太尉充副元帥進爵梁王以胤為司徒兼侍中胤恃全忠之勢專權自恣天子動靜皆禀之|{
	稟筆錦翻}
朝臣從上幸鳳翔者凡貶逐三十餘人|{
	黨附宦官者可罪扈從天子者何罪邪朝直遙翻}
刑賞繫其愛憎|{
	愛者賞之憎者刑之}
中外畏之重足一迹|{
	重直龍翻史言崔胤怙權不知死期將至}
以敬翔守太府卿朱友寧領寧遠節度使|{
	寧遠軍容州時為龎巨昭所據五季以來有名號節度使此類是也}
全忠表苻道昭同平章事充天雄節度使遣兵援送之秦州|{
	之往也}
不得至而還|{
	岐兵塞道故不得至還從宣翻又如字}
初翰林學士承旨韓偓之登進士第也御史大夫趙崇知貢舉上返自鳳翔欲用偓為相偓薦崇及兵部侍郎王贊自代上欲從之崔胤惡其分已權|{
	惡烏路翻}
使朱全忠入爭之全忠見上曰|{
	見賢遍翻}
趙崇輕薄之魁王贊無才用韓偓何得妄薦為相上見全忠怒甚不得已癸未貶偓濮州司馬上密與偓泣别偓曰是人非復前來之比|{
	謂朱全忠也}
臣得遠貶及死乃幸耳不忍見簒弑之辱|{
	嗚呼韓偓何見之晚也然昭宗聞偓此言亦何以為懷哉惟有縱酒而已}
己丑上令朱全忠與李茂貞書取平原公主茂貞不敢違遽歸之|{
	平原公主嫁茂貞子宋侃見上卷上年}
壬辰以朱友裕為鎮國節度使 |{
	考異曰實録壬辰以興德府復為華州賜名感化軍以友裕為節度使按編遺録天祐三年閏十二月乙丑敕鎮國之號興德之名並宜停薛居正五代史地理志華州梁為感化軍梁功臣傳天復三年友裕權知鎮國軍留後今從實録}
乙未全忠奏留步騎萬人於故兩軍|{
	時神策兩軍已散而營署尚存}
以朱友倫為左軍宿衛都指揮使又以汴將張廷範為宫苑使王殷為皇城使蔣玄暉充街使於是全忠之黨布列徧於禁衛及京輔|{
	唐北門禁衛之兵皆屯於宫苑百司庶府及南衙諸衛皆分居皇城之内百官私第及坊市居人皆分居朱雀街之左右街今全忠悉以腹心為使則京輔之權一歸之矣去虺得虎昭宗之謂也}
戊戌全忠辭歸鎮|{
	辭歸大梁}
留宴壽春殿又餞之於延喜樓上臨軒泣别令於樓前上馬|{
	示寵異之也前上時掌翻}
上又賜全忠詩全忠亦和進|{
	和胡臥翻}
又進楊柳枝辭五首|{
	楊柳枝辭即今之令曲也今之曲如清平調水調歌柘枝菩薩蠻八聲甘州皆唐季之餘聲又唐人多賦楊柳枝皆是七言四絶相傳以為出於開元棃園樂章故張祜有折楊柳詞云莫折宫前楊柳枝玄宗曾向笛中吹}
百官班辭於長樂驛崔胤獨送至霸橋|{
	以唐制驛程考之霸橋驛當在長樂驛東三十里}
自置餞席夜二鼓胤始還入城上復召對|{
	復扶又翻}
問以全忠安否置酒奏樂至四鼓乃罷|{
	史言帝徵召不時宴飲無節}
以清海節度使裴樞為門下侍郎同平章事|{
	裴樞以朱全忠之薦而相以忤朱全忠之意而死白馬之禍皆自取之也}
李克用使者還晉陽言崔胤之横|{
	横戶孟翻}
克用曰胤為人臣外倚賊勢内脅其君既執朝政又握兵權權重則怨多勢侔則釁生破家亡國在眼中矣|{
	史言李克用有識朝直遥翻}
朱全忠將行奏克用於臣本無大嫌乞厚加寵澤遣大臣撫慰俾知臣意進奏吏以白克用|{
	河東進奏吏也}
克用笑曰賊欲有事淄青畏吾犄其後耳|{
	有事淄青謂攻王師範史言朱全忠狡譎李克用已逆知其情犄居蟻翻}
三月戊午朱全忠至大梁王師範弟師魯圍齊州|{
	朱全忠并兖鄆遂兼有齊州九域志兖州北至齊州三百六十里}
朱友寧引兵擊走之師範遣兵益劉鄩軍友寧擊取之由是兖州援絶葛從周引兵圍之|{
	劉鄩取兖州見上卷本年正月}
友寧進攻青州戊辰全忠引四鎮及魏博兵十萬繼之 淮南將李神福圍鄂州|{
	是年正月楊行密遣李神福攻杜洪事始上卷}
望城中積荻謂監軍尹建峯曰今夕為公焚之|{
	為于偽翻}
建峯未之信時杜洪求救于朱全忠神福遣部將秦臯乘輕舟至灄口|{
	灄口在武口之上對岸即夏浦灄書涉翻}
舉火炬於樹杪|{
	杪弭沼翻}
洪以為救兵至果焚荻以應之 夏四月己卯以朱全忠判元帥府事|{
	輝王沖幼以朱全忠判元帥府事則天下兵權盡歸之矣}
知温州事丁章為木工李彦所殺|{
	丁章得温州見上卷二年未有朝命為刺史止稱知州事}
其將張惠據温州 王師範求救於淮南乙未楊行密遣其將王茂章以步騎七千救之又遣别將將兵數萬攻宿州全忠遣其將康懷英救宿州淮南兵遁去|{
	康懷英當作懷貞是時未改名也}
楊行密遣使詣馬殷言朱全忠跋扈請殷絶之約為兄弟湖南大將許德勲曰全忠雖無道然挾天子以令諸侯明公素奉王室不可輕絶也|{
	言絶全忠則道路梗塞併絶朝廷貢奉}
殷從之|{
	馬殷附汴之心自此堅矣}
杜洪求救於朱全忠全忠遣其將韓勍將萬人屯灄口|{
	勍渠京翻}
遣使語荆南節度使成汭武安節度使馬殷武貞節度使雷彦威|{
	語牛倨翻曰語者無朝廷詔敕以意諭之}
令出兵救洪汭畏全忠之彊且欲侵江淮之地以自廣發舟師十萬沿江東下汭作巨艦三年而成|{
	艦戶黯翻}
制度如府署謂之和舟載|{
	署廨舍也言其舟長闊和荆州皆載其上舟當作州}
其餘謂之齊山截海劈浪之類甚衆|{
	齊山言其高也截海言其長也劈浪言其輕疾也劈匹歷翻}
掌書記李珽諫曰|{
	珽它鼎翻}
今每艦載甲士千人稻米倍之緩急不可動也吳兵剽輕|{
	剽匹妙翻輕若定翻}
難與角逐武陵長沙皆吾讐也|{
	武陵謂雷彦威長沙謂馬殷}
豈得不為反顧之慮乎不若遣驍將屯巴陵|{
	九域志巴陵東北至鄂州三百五十里}
大軍與之對岸堅壁勿戰不過一月吳兵食盡自遁鄂圍解矣|{
	楊行密時封吳王故謂其兵為吳兵}
汭不聽珽憕之五世孫也|{
	李憕天寶之末死於安禄山之難珽後歸中原仕於梁}
王建出兵攻秦隴乘李茂貞之弱也遣判官韋莊入貢亦修好於朱全忠|{
	好呼到翻}
全忠遣押牙王殷報聘建與之宴殷言蜀甲兵誠多但乏馬耳建作色曰當道江山險阻騎兵無所施然馬亦不乏押牙少留當共閲之乃集諸州馬大閲於星宿山官馬八千私馬四千部隊甚整殷歎服|{
	王建以多馬衒王殷殷遽歎服非善覘者也宿音秀}
建本騎將|{
	王建從楊復光起許州及扈從昭宗皆為騎將}
故得蜀之後於文黎維茂州市胡馬十年之間遂及茲數|{
	史言蜀中互市可以得西蕃之馬然王建取興元而得騎五千則東西川之馬亦必多此一萬二千之數蓋集成都近州耳}
五月丁未李克用雲州都將王敬暉殺刺史劉再立叛降劉仁恭克用遣李嗣昭李存審將兵討之|{
	李存審即符存審降戶江翻}
仁恭遣將以兵五萬救敬暉嗣昭退保樂安|{
	畏燕兵之彊也}
敬暉舉衆弃城而去|{
	乘嗣昭之退弃城而走}
先是振武將契苾讓|{
	先悉薦翻契欺訖翻}
逐戍將石善友據城叛嗣昭等進攻之讓自燔死復取振武城殺吐谷渾叛者二千餘人|{
	吐谷渾自赫連鐸與克用作敵鐸雖敗死其部落終未肯心服故屢叛}
克用怒嗣昭存審失王敬暉皆杖之削其官|{
	爾朱榮以失万俟道落而杖爾朱天光事亦如此}
成汭行未至鄂州馬殷遣大將許德勲將舟師萬餘人雷彦威遣其將歐陽思將舟師三千餘人會於荆江口|{
	大江自蜀東流入荆州界謂之荆江荆江口即洞庭之水與大江之水會處}
乘虚襲江陵庚戌陷之盡掠其人及貨財而去將士亡其家皆無鬭志|{
	此言成汭之將士也}
李神福聞其將至自乘輕舟前覘之|{
	覘丑亷翻又丑艷翻}
謂諸將曰彼戰艦雖多而不相屬易制也|{
	屬之欲翻易以豉翻}
當急擊之壬子神福遣其將秦裴楊戎將衆數千逆擊汭於君山|{
	君山在洞庭湖中方六十里亦名洞庭之山巴陵志曰湘君所遊故曰君山將即亮翻}
大破之因風縱火焚其艦士卒皆潰汭赴水死|{
	僖宗文德元年成汭襲據荆南至是敗亡 考異曰新紀彦威之弟彦恭陷江陵今從編遺録舊紀及薛居正五代史十國紀年皆云汭未至鄂渚江陵已陷將士亡其家皆無鬭志按新紀十國紀年皆云壬子汭敗死壬子此月十二日也而編遺録云二十二日陷江陵今不取北夢瑣言云天祐中汭死尤誤也}
獲其戰艦二百艘|{
	艘蘇遭翻}
韓勍聞之亦引兵去許德勲還過岳州刺史鄧進忠開門具牛酒犒軍德勲諭以禍福進忠遂舉族遷于長沙|{
	僖宗光啟二年鄧進思取岳州傳弟進忠至是而亡 考異曰馬氏行年記天復三年自荆南振旅還遂入岳州降刺史鄧進忠九國志楚世家天祐二年七月岳州刺史鄧進忠帥其衆來降許德勲傳云天祐二年領兵畧地荆南還經岳州刺史鄧進忠以城歸附新紀全用九國志年月湖湘故事言開平中收荆南回進忠以城降又載何致雍天策寺碑銘云乃克桂林乃襲荆渚彼岳之陽旋師而取天祐二年十月朱全忠謀討襄州趙匡凝九月克襄州始命楊師厚攻荆南然則七月許德勲何繇畧地荆南蓋九國志之誤天復三年成汭敗死德勲及雷彦威襲江陵還取岳州與何致雍碑意畧同故以行年紀為据}
馬殷以德勲為岳州刺史以進忠為衡州刺史雷彦威狡獪殘忍有父風|{
	獪古外翻雷彦威父滿}
常泛舟焚掠鄰境荆鄂之間殆至無人 李茂貞畏朱全忠自以官為尚書令在全忠上|{
	朱全忠守中書令茂貞為尚書令官在其上}
累表乞解去詔復以茂貞為中書令 崔胤奏左右龍武羽林神策等軍|{
	此崔胤所判六軍也}
名存實亡侍衛單寡請每軍募步兵四將每將二百五十人騎兵一將百人合六千六百人|{
	六軍各軍步兵千人騎兵百人合六千六百人}
選其壯健者分番侍衛從之令六軍諸衛副使京兆尹鄭元規立格召募於市|{
	朱全忠自此疑崔胤而有圖之之心}
朱全忠表潁州刺史朱友恭為武寧節度使 朱友寧攻博昌|{
	博昌漢縣唐屬青州十三州志云昌水其勢平故曰博昌後唐避廟諱改曰博興九域志博興在青州西北一百二十里管下有博昌鎮}
月餘不拔朱全忠怒遣客將劉捍往督之|{
	今閫府州軍皆有客將主贊導賓客蓋古之舍人中涓漢之鈴下威儀之職唐末藩鎮置客將往往升轉至大官位望不輕}
捍至友寧驅民丁十餘萬負木石牽牛驢詣城南築土山既成并人畜木石排而築之寃號聲聞數十里俄而城陷盡屠之|{
	爭城而戰殺人盈城朱友寧之隕身喪元未足以謝寃䰟也號戶刀翻聞音問 考異曰唐太祖紀年録師範之舉兵也朱温令朱友寧討之三月己酉朱温至汴州大舉魏博四鎮之衆十萬擊師範朱友寧楊師厚攻博興旬餘不下攻城之衆死者大半俄而朱温至大怒斬其主將復起土山翌日而拔城中無少長皆屠之仍毁其垣四月進陷臨淄傅青州别將攻北海渡膠水寇登萊等郡實録據此而置於四月梁太祖實録四月丙子上至鄆領事辛卯從子友寧帥師破青州之博昌臨淄二邑殺戮五千餘衆暨北海焉編遺録五月辛亥却離歷下宿豐齊驛甲寅上到汶陽乙卯奏王師範逆狀己未上又往歷下壬戌上以兵士攻取博昌寨下少樹木時當炎毒却勒親從騎兵皆歸齊州因又前行夜將半客將劉捍謀曰捍請馳赴軍前傳諭上意敦將士今戮力速攻必可尅也今請上却歸歷下上悦而從之便令捍馳騎東往上乃西歸汶陽丙寅捷音至攻拔博昌盡戮其黨矣據此則破博昌在五月今從朱友寧傳}
進拔臨淄|{
	臨淄漢古縣久廢隋復置於古齊國城唐屬青州九域志在州西北四十里}
扺青州城下遣别將攻登萊淮南將王茂章會王師範弟萊州刺史師誨攻密州拔之斬其刺史劉康乂|{
	九域志萊州南至密州三百里東北至登州二百四十里劉康乂朱全忠所用也}
以淮海都遊奕使張訓為刺史|{
	楊行密據有淮南西盡淮源東暨于海邊而延袤數千里故置都遊奕使以謹防遏也}
六月乙亥汴兵拔登州師範帥登萊兵拒朱友寧於石樓為兩柵|{
	據舊書石樓近臨淄}
丙子夜友寧擊登州柵柵中告急師範趣茂章出戰|{
	趣讀曰促}
茂章按兵不動友寧破登州柵進攻萊州柵比明茂章度其兵力已疲|{
	比必利翻及也度徒洛翻下同}
乃與師範合兵出戰大破之友寧旁自峻阜馳騎赴敵馬仆青州將張土梟斬之|{
	梟堅堯翻}
傳首淮南兩鎮兵逐北至米河|{
	王師範以平盧之兵王茂章以淮南之兵是兩鎮兵也}
俘斬萬計魏博之兵殆盡全忠聞友寧死自將兵二十萬晝夜兼行赴之秋七月壬子至臨胊|{
	臨朐漢縣唐屬青州九域志曰在州東南四十里}
命諸將攻青州王師範出戰汴兵大破之王茂章閉壘示怯伺汴兵稍懈|{
	伺相吏翻懈古隘翻}
毁柵而出驅馳疾戰戰酣退坐召諸將飲酒已而復戰全忠登高望見之問降者|{
	降戶江翻}
知為茂章歎曰使吾得此人為將天下不足平也|{
	朱全忠見王茂章臨敵整暇故欲得之然茂章後歸梁攻淮南攻鎮并皆折北而不振人固未易知也}
至晡汴兵乃退茂章度衆寡不敵|{
	度徒洛翻}
是夕引軍還全忠遣曹州刺史楊師厚追之及於輔唐|{
	輔唐漢安丘縣乾元二年移治古昌安城因改曰輔唐屬密州九域志在州西北一百二十里薛史地理志曰密州輔唐縣梁開平二年改為安丘唐同光元年復舊名晉天福七年改為膠西避廟諱也宋復曰安丘}
茂章命先鋒指揮使李䖍裕將五百騎為殿|{
	殿丁練翻下同}
䖍裕殊死戰師厚擒而殺之|{
	李䖍裕以死全王茂章之軍其勇難能也楊師厚自此受知於朱全忠矣}
師厚潁州人也張訓聞茂章去謂諸將曰汴人將至何以禦之諸將請焚城大掠而歸訓曰不可封府庫植旗幟於城上遣羸弱居前|{
	植直吏翻幟昌至翻羸倫為翻}
自以精兵殿其後而去全忠遣左踏白指揮使王檀攻密州|{
	凡軍行前軍之前有踏白隊所以踏伏候望敵之遠近衆寡}
既至望旗幟數日乃敢入城|{
	疑其有伏故遲遲不敢進}
見府庫城邑皆完遂不復追|{
	復扶又翻}
訓全軍而還|{
	史言楊行密所以能保有江淮一時諸將皆能盡其智力}
全忠以檀為密州刺史 丁卯以山南西道留後王宗賀為節度使|{
	王建之請也}
睦州刺史陳詢叛錢鏐舉兵攻蘭溪|{
	咸亨五年分婺州之金華西界置蘭溪縣因溪水為名九域志在州西北五十五里}
鏐遣指揮使方永珍擊之武安都指揮使杜建徽與詢連姻鏐疑之建徽不言會詢親吏來奔得建徽與詢書皆勸戒之辭鏐乃悦建徽從兄建思譛建徽私蓄兵仗謀作亂鏐使人索之|{
	從才用翻索山客翻}
建徽方食使者直入卧内|{
	使疏吏翻}
建徽不顧鏐以是益親重之 八月戊辰朔朱全忠留齊州刺史楊師厚攻青州身歸大梁|{
	朱全忠以朱友寧之死興忿兵以攻青州豈不欲一鼔而屠之乃置之而歸汴者知青州城堅而王師範兵力尚彊未易以旦夕取故使楊師厚圍守之}
庚辰加西川節度使西平王王建守司徒進爵蜀王|{
	自郡王進國王}
前渝州刺史王宗本|{
	王宗本前此刺渝州亦王建命之也罷官歸成都故稱前}
言於王建請出兵取荆南建從之以宗本為開道都指揮使將兵下峽|{
	峽三峽也}
初寧國節度使田頵破馮弘鐸|{
	事見上卷二年}
詣廣陵謝楊行密因求池歙為巡屬|{
	唐置宣歙池觀察使二州本宣州巡屬故田頵因有功而求之}
行密不許|{
	與之則田頵愈彊故不許}
行密左右下及獄吏皆求賂於頵|{
	以其破馮弘鐸多得寶貨也}
頵怒曰吏知吾將下獄邪|{
	下戶嫁翻}
及還指廣陵南門曰吾不可復入此矣|{
	復扶又翻下復出同}
頵兵彊財富好攻取|{
	好呼到翻}
行密既定淮南欲保境息民每抑止之頵不從及解釋錢鏐|{
	事見上卷二年}
頵尤恨之隂有叛志李神福言於行密曰頵必反宜早圖之行密曰頵有大功|{
	田頵從楊行密起盧州破趙鍠孫儒馮弘鐸皆有大功}
反狀未露今殺之諸將人人自危矣頵有良將曰康儒與頵謀議多不合行密知之擢儒為廬州刺史|{
	擢儒所以間頵也}
頵以儒為貳於已族之儒曰吾死田公亡無日矣頵遂與潤州團練使安仁義同舉兵 |{
	考異曰十國紀年朱全忠聞田頵等叛矯制削奪王官爵命頵及杜洪鍾傳錢鏐充四面招討使布制書於境上王知其詐妄按新舊紀實録梁太祖紀皆無削奪行密官爵命杜洪等為招討使事今不取}
仁義悉焚東塘戰艦|{
	東塘即楊州東塘淮南之戰艦聚焉對岸即潤州界故仁義得焚之艦戶黯翻}
頵遣二使詐為商人詣壽州約奉國節度使朱延壽|{
	朝廷命朱延壽領奉國節度使見上卷二年使疏吏翻}
行密將尚公迺遇之曰非商人也殺一人得其書以告行密|{
	尚公迺歸行密見上卷二年}
行密召李神福於鄂州神福恐杜洪邀之宣言奉命攻荆南勒兵具舟楫及暮遂沿江東下始告將士以討田頵己丑安仁義襲常州|{
	九域志潤州東南至常州一百七十一里}
常州刺史李遇逆戰極口罵仁義仁義曰彼敢辱我必有備乃引去壬辰行密以王茂章為潤州行營招討使擊仁義不克使徐温將兵會之温易其衣服旗幟皆如茂章兵仁義不知益兵復出戰|{
	復扶又翻}
温奮擊破之|{
	李存審救河中擒梁騎兵亦用此術}
行密夫人朱延壽之姊也行密狎侮延壽延壽怨怒隂與田頵通謀|{
	書旅獒曰德盛不狎侮狎侮君子罔以盡其心狎侮小人罔以盡其力楊行密狎侮朱延壽幾至於亡國喪家蓋危而後濟耳可不戒哉}
頵遣前進士杜荀鶴至壽州與延壽相結又遣至大梁告朱全忠全忠大喜遣兵屯宿州以應之|{
	朱全忠喜楊行密有隙之可乘而不能舉大兵以掎其後者内有淄青未服而西又有鳳翔北又有太原恐其乘間動搖朝廷也}
荀鶴池州人也楊師厚屯臨朐聲言將之密州留輜重於臨朐|{
	九域志臨}


|{
	胊縣在青州東南四十里又二百六十里至密州朐音劬重直用翻}
九月癸卯王師範出兵攻臨朐師厚伏兵奮擊大破之殺萬餘人獲師範弟師克明日萊州兵五千救青州師厚邀擊之殺獲殆盡遂徙寨抵其城下 |{
	考異曰梁太祖實録九月癸卯楊師厚勵衆決鬭青人大敗北走殺戮一萬人擒師範弟師克翌日東萊郡遣州兵洎土團五千人將援青壘我師邀截翦撲無一二存焉即時徙寨逼其闉闍唐實録畧與此同編遺録冬十月丁卯楊師厚繼告捷於臨朐北及青州四面累殺賊黨擒斬頗衆至十一月萊州刺史王師克領六千人欲徑入青丘助其守禦師厚伏兵邀之殺戮將盡下又有丁亥上誕辰聞朱友倫死誕辰乃十月二十一日友倫死亦十月中事也下又别有十一月疑上文十一月是十一日字或七日字又曰一日師範請降疑脱二十字二十一日即戊午也今從梁實録}
朱延壽謀頗泄|{
	朱延壽與田頵通謀久而頗露}
楊行密詐為目疾對延壽使者多錯亂所見或觸柱仆地|{
	見甲以為乙見犬以為猫是錯亂所見也柱至易見者而行觸之皆詐為失明以愚人}
謂夫人曰吾不幸失明諸子皆幼軍府事當悉以授三舅夫人屢以書報延壽|{
	夫人即延壽姊也延夀第三}
行密又自遣召之隂令徐温為之備延壽至廣陵行密迎及寢門執而殺之|{
	按尚公迺執田頵二使田頵繼遣杜荀鶴至壽州朱延壽亦必知前二使之見執矣楊行密召之了不自疑至于送死豈其智有所不及邪抑天奪之鑒也}
部兵驚擾徐温諭之皆聽命|{
	徐温從楊行密起廬州與劉威陶雅之徒號三十六英雄是必有以服朱延壽部兵之心矣故諭之皆聽命}
遂斬延壽兄弟黜朱夫人初延壽赴召其妻王氏謂曰君此行吉凶未可知願日發一使以安我一日使不至王氏曰事可知矣部分僮僕|{
	使疏吏翻下同分扶問翻}
授兵闔門捕騎至乃集家人聚寶貨發百燎焚府舍曰妾誓不以皎然之軀為讐人所辱赴火而死|{
	史言朱延壽妻有智識而能守節}
延壽用法嚴好以寡擊衆|{
	好呼到翻}
嘗遣二百人與汴兵戰有一人應留者請行延壽以違命立斬之 田頵襲昇州得李神福妻子善遇之|{
	天復二年田頵克昇州楊行密以李神福為昇州刺史時行密遣神福攻鄂故頵乘虛襲之九域志宣州北至昇州三百六十里}
神福自鄂州東下頵遣使謂之曰公見機與公分地而王不然妻子無遺神福曰吾以卒伍事吳王|{
	楊行密封吳王故稱之}
今為上將義不以妻子易其志頵有老母不顧而反三綱且不知|{
	或疑行密留田頵之母於廣陵詳考本末田頵母殷自從頵在宣州李神福蓋言頵有母在不當輕為舉措稱兵而敗則禍必及母也三綱者謂君為臣綱父為子綱夫為妻綱}
烏足與言乎斬使者而進士卒皆感勵頵遣其將王壇汪建將水軍逆戰|{
	光化二年田頵將康儒取婺州王壇歸之}
丁未神福至吉陽磯與壇建遇壇建執其子承鼎示之神福命左右射之|{
	射而亦翻}
神福謂諸將曰彼衆我寡當以奇取勝及暮合戰神福佯敗引舟泝流而上|{
	逆流曰泝泝蘇故翻上時掌翻}
壇建追之神福復還順流擊之壇建樓船大列火炬神福令軍中曰望火炬輒擊之|{
	望壇建所在而擊之船列火炬不能以自照見而敵人望之洞見表裏聚而攻之安有不敗者乎}
壇建軍皆滅火旗幟交雜神福因風縱火焚其艦壇建大敗|{
	李神福之陽敗也必逆風而戰故引舟順風泝流而上其縱火焚壇建之艦也必因風轉乘風水之勢以破之居然可知也}
士卒焚溺死者甚衆戊申又戰于皖口|{
	舒州懷寧縣有皖口鎮當皖水入江之口皖何板翻}
壇建僅以身免獲徐綰行密以檻車載之遺錢鏐鏐剖其心以祭高渭|{
	徐綰殺高渭事見上卷二年遺唯季翻}
頵聞壇建敗自將水軍逆戰神福曰賊弃城而來此天亡也臨江堅壁不戰遣使告行密請發步兵斷其歸路|{
	斷音短}
行密遣漣水制置使臺濛將兵應之王茂章攻潤州久未下行密命茂章引兵會濛擊頵|{
	安仁義雖善戰而兵弱自守虜耳田頵兵勢方挫故命合兵擊之}
辛亥汴將劉重霸拔棣州執刺史邵播殺之|{
	全忠滅朱瑄已得棣州邵播又以州叛附王師範重直龍翻}
甲寅朱全忠如洛陽遇疾復還大梁 |{
	考異曰梁實録云壬戌唐實録云十月丁卯朔今從編遺録}
戊午王師範遣副使李嗣業及弟師悦請降於楊師厚曰師範非敢背德|{
	降戶江翻下同背蒲妹翻}
韓全誨李茂貞以朱書御札使之舉兵師範不敢違仍請以其弟師魯為質|{
	質音致}
時朱全忠聞李茂貞楊崇本將起兵逼京畿|{
	邠岐連兵其事詳見後岐本亦京畿李茂貞據之遂為彊藩今所謂京畿特京兆府之京縣畿縣耳}
恐其復劫天子西去|{
	復扶又翻}
欲迎車駕都洛陽乃受師範降 |{
	考異曰舊紀及薛居正五代史劉鄩傳皆云十一月師範降編遺録曰十一月敗萊州刺史王師克一日師範差人捧款檄至軍前請舉城歸降按梁太祖實録薛史梁紀唐實録皆云九月戊午今從之}
選諸將使守登萊淄棣等州即以師範權淄青留後|{
	史言朱全忠本欲殺王師範而力有所未及為後屠師範一家張本}
師範仍言先遣行軍司馬劉鄩將兵五千據兖州|{
	事始見上卷本年}
非其自專願釋其罪亦遣使語鄩|{
	語牛倨翻}
田頵聞臺濛將至自將步騎逆戰留其將郭行悰以精兵二萬及王壇汪建水軍屯蕪湖|{
	悰徂宗翻蕪湖漢古縣晉氏南渡以上黨襄垣遺民僑立郡縣於蕪湖江左遂為襄垣縣隋廢襄垣入當塗至唐蕪湖之地入當塗太平二縣界唐末始復置蕪湖縣屬宣州今屬太平州九域志在太平州西南六十五里}
以拒李神福覘者言濛營寨小纔容二千人頵易之|{
	覘昌占翻又丑艶翻徧補典翻易以豉翻}
不召外兵濛入頵境番陳而進|{
	番陳者分兵為數部更番列陳整兵而後進以備倉猝薄戰陳讀曰陣}
軍中笑其怯濛曰頵宿將多謀不可不備|{
	將即亮翻}
冬十月戊辰與頵遇於廣德|{
	九域志廣德西至宣州一百八十里宋白曰廣德縣秦鄣郡地漢為故鄣縣}
濛先以楊行密書徧賜頵將皆下馬拜受濛因其挫伏|{
	挫伏者言其將士之氣摧挫而厭伏也}
縱兵擊之頵兵遂敗又戰于黄池兵交濛偽走頵追之遇伏大敗奔還宣州城守濛引兵圍之頵亟召蕪湖兵還不得入郭行悰王壇汪建及當塗廣德諸戍皆帥其衆降|{
	帥讀曰率}
行密以臺濛已破田頵命王茂章復引兵攻潤州|{
	知臺濛兵力足以制田頵故命王茂章復攻安仁義復扶又翻}
初夔州刺史侯矩從成汭救卾州汭死矩奔還|{
	成汭死見上四月}
會王宗本兵至矩以州降之宗本遂定夔忠萬施四州|{
	夔忠萬荆南巡屬施黔中巡屬}
王建復以矩為夔州刺史更其姓名曰王宗矩|{
	更工衡翻}
宗矩易州人也蜀之議者以瞿唐蜀之險要|{
	瞿唐峽在夔州東一里舊名西陵峽乃三峽之門兩崖對峙中貫一江望之如門}
乃弃歸峽屯軍夔州|{
	荆南自此止領荆歸峽三州}
建以宗本為武泰留後武泰軍舊治黔州宗本以其地多瘴癘請徙治涪州建許之|{
	史言王建全據峽江之險九域志自黔州西北至涪州一百八十二里黔其今翻又其炎翻瘴之亮翻涪音浮}
葛從周急攻兖州劉鄩使從周母乘板輿登城謂從周曰劉將軍事我不異於汝新婦輩皆安居人各為其主汝可察之從周歔欷而退攻城為之緩|{
	新婦謂葛從周妻也為于偽翻歔音虛欷音希又許既翻劉鄩用兵十步九計自得兖州先定此策以伐葛從周之心}
鄩悉簡婦人及民之老疾不足當敵者出之獨與少壯者|{
	少詩照翻}
同辛苦分衣食堅守以扞敵號令整肅兵不為暴民皆安堵久之外援既絶節度副使王彦温踰城出降城上卒多從之不可遏鄩遣人從容語彦温曰|{
	從千容翻語牛倨翻}
軍士非素遣者勿多與之俱又遣人徇於城上曰軍士非素遣從副使而敢擅往者族之士卒皆惶惑不敢出敵人果疑彦温斬之城下由是衆心益固及王師範力屈|{
	謂屢為汴兵所敗也}
從周以禍福諭之鄩曰受王公命守此城一旦見王公失勢不俟其命而降非所以事上也及師範使者至|{
	王師範所遣語鄩使降者也}
丁丑始出降 |{
	考異曰梁實録四年正月辛丑劉鄩自兖州來降舊紀十一月鄩以兖州降實録十一月鄩降薛居正五代史梁紀十一月丁酉鄩降鄩傳曰天復三年十一月師範告降且言先差鄩領兵入兖州請釋其罪亦以告鄩鄩即出城聽命新紀十一月丁丑劉鄩以兖州叛附于朱全忠按青兖相距不遠師範之降亦以告鄩豈有自戊午至丁酉四十日師範使者始至兖州邪十月丁丑日差近今從新紀}
從周為具齎裝送鄩詣大梁鄩曰降將未受梁王寛釋之命安敢乘馬衣裘乎|{
	為于偽翻衣于既翻}
乃素服乘驢至大梁|{
	素服囚服也渠帥俘虜載以驢}
全忠賜之冠帶辭請囚服入見不許全忠慰勞飲之酒辭以量小|{
	勞力到翻飲于禁翻量音亮飲酒之多少各有量}
全忠曰取兖州量何大邪以為元從都押牙|{
	從才用翻}
是時四鎮將吏皆功臣舊人|{
	朱全忠迎車駕于鳳翔諸將皆賜迎鑾果毅功臣舊人與全忠出入于行間最久者}
鄩一旦以降將居其上諸將具軍禮拜於廷鄩坐受自如全忠益奇之|{
	劉鄩自降將擢為四鎮牙前右職而居之若固有之自知其才之足以當之也全忠以此益奇之}
未幾表為保大留後|{
	幾居豈翻保大軍鄜州以捍李茂貞}
葛從周久病全忠以康懷英為泰寧節度使代之|{
	懷英當作懷貞}
宿衛都指揮使朱友倫與客擊毬於左軍墜馬而卒 |{
	考異曰編遺録丁亥趙廷隱自長安馳來告今月十四日朱友倫墜馬而卒十四日則庚辰也後唐紀年録薛居正五代史及昭宗實録皆云辛巳今從之}
全忠悲怒疑崔胤故為之|{
	有為為之謂之故}
凡與同戲者十餘人盡殺之遣其兄子友諒代典宿衛 山南東道節度使趙匡凝遣兵襲荆南朗人弃城走|{
	朗人雷彦威之兵成汭既死荆南無帥朗人遂守之}
匡凝表其弟匡明為荆南留後時天子微弱諸道貢賦多不上供惟匡明兄弟委輸不絶|{
	唐二税有上供以輸京師供居用翻輸舂遇翻}
楊行密求兵於錢鏐鏐遣方永珍屯潤州從弟鎰屯宣州|{
	屯潤州以助攻安仁義屯宣州以助攻田頵從才用翻鎰弋質翻}
又遣指揮使楊習攻睦州|{
	陳詢時據睦州背錢鏐而睦于田頵}
鳳翔邠州屢出兵近京畿|{
	鳳翔李茂貞邠李繼徽近其靳翻}
朱全忠疑其復有刼遷之謀|{
	復扶又翻}
十一月發騎兵屯河中 十一月乙亥田頵帥死士數百出戰|{
	帥讀曰率}
臺濛陽退以示弱頵兵踰濠而鬬濛急擊之頵不勝還走城|{
	走音奏}
橋陷墜馬斬之其衆猶戰以頵首示之乃潰濛遂克宣州|{
	景福元年田頵鎮宣州至是而亡}
初行密與頵同閭里少相善約為兄弟|{
	少詩照翻}
及頵首至廣陵行密視之泣下赦其母殷氏行密與諸子皆以子孫禮事之|{
	行密以通家諸子禮事殷氏其子以諸孫禮事之史言行密雖以法裁部曲而有恩於故舊}
行密以李神福為寧國節度使|{
	欲以代田頵}
神福以杜洪未平固讓不拜宣州長史駱知祥善治金穀|{
	治直之翻}
觀察牙推沈文昌為文精敏嘗為頵草檄罵行密|{
	嘗為于偽翻}
行密以知祥為淮南支計官|{
	支計官蓋唐世節度支度判官之屬唐末藩鎮變其名稱耳}
以文昌為節度牙推|{
	唐制節度觀察牙推在巡官之下幕府右職也}
文昌湖州人也初頵每戰不勝輒欲殺錢傳瓘其母及宣州都虞候郭師從常保護之師從合肥人頵之婦弟也頵敗傳瓘歸杭州|{
	錢傳瓘質于田頵見上卷上年}
錢鏐以師從為鎮東都虞候 辛巳以禮部尚書獨孤損為兵部侍郎同平章事損及之從曾孫也|{
	獨孤及見二百二十三卷代宗永泰元年從才用翻}
中書侍郎兼戶部尚書同平章事裴贄罷為左僕射 左僕射致仕張濬居長水王師範之舉兵濬豫其謀|{
	事見上卷上年}
朱全忠將謀簒奪恐濬扇動藩鎮諷張全義使圖之丙申全義遣牙將楊麟將兵詐為刼盜圍其墅而殺之|{
	張濬之死夷考本末過于白馬朝士遠矣墅承與翻}
永寧縣吏葉彦素為濬所厚知麟將至密告濬子格曰相公禍不可免郎君宜自為謀濬謂格曰汝留則俱死去則遺種|{
	種章勇翻}
格哭拜而去葉彦帥義士三十人送之渡漢而還|{
	帥讀曰率還從宣翻又如字}
格遂自荆南入蜀|{
	張格入蜀而亡王氏者格也}
盧龍節度使劉仁恭習知契丹情偽常選將練兵乘秋深入踰摘星嶺擊之契丹畏之每霜降仁恭輒遣人焚塞下野草契丹馬多飢死常以良馬賂仁恭買牧地|{
	北荒寒早至秋草先枯死近塞差暖霜降草猶未盡衰故契丹南並塞放牧焚其野草則馬無所食而飢死契欺紇翻}
契丹王阿保機遣其妻兄阿鉢將萬騎寇渝關|{
	契丹阿保機始此宋白曰平州東北至榆關守捉一百九十里渝漢書音義音喻今讀如榆}
仁恭遣其子守光戍平州守光偽與之和設幄犒饗於城外|{
	犒苦到翻}
酒酣伏兵執之以入虜衆大哭契丹以重賂請於仁恭然後歸之 |{
	考異曰薛居正五代史及莊宗列傳皆云光啓中守光禽舍利王子其王欽德以重賂贖之按是時仁恭猶未得幽州也今從薛史蕭翰傳及王皥唐餘録}
初崔胤假朱全忠兵力以誅宦官|{
	事始二百六十二卷天復元年終上卷三年}
全忠既破李茂貞併吞關中威震天下遂有簒奪之志胤懼與全忠外雖親厚私心漸異乃謂全忠曰長安密邇茂貞不可不為守禦之備六軍十二衛但有空名請召募以實之使公無西顧之憂全忠知其意曲從之隂使麾下壯士應募以察其變胤不之知與鄭元規等繕治兵仗日夜不息及朱友倫死|{
	募兵見上五月朱友倫死見上十月治直之翻}
全忠益疑胤且欲遷天子都洛恐胤立異|{
	恐其立異論以沮遷洛之計}


天祐元年|{
	是年四月方改元}
春正月全忠密表司徒兼侍中判六軍十二衛事充鹽鐵轉運使判度支崔胤專權亂國離間君臣|{
	間古莧翻}
并其黨刑部尚書兼京兆尹六軍諸衛副使鄭元規威遠軍使陳班等皆請誅之乙巳詔責授胤太子少傅分司貶元規循州司戶班湊州司戶|{
	時無湊州湊當作溱}
丙午下詔罪狀胤等以裴樞判左三軍事充鹽鐵轉運使獨孤損判右三軍事兼判度支胤所募兵並縱遣之以兵部尚書崔遠為中書侍郎翰林學士左拾遺柳璨為右諫議大夫並同平章事 |{
	考異曰舊傳崔胤得罪前一日召璨入内殿草制敕胤死之日既夕璨自内出前驅傳呼相公來人未見制敕莫測所以新傳曰崔胤死昭宗密許璨相外無知者日暮自禁中出傳呼宰相人大驚按胤未死璨已除平章事新舊傳云胤死後誤也}
璨公綽之從孫也|{
	自元和以來柳氏以清正文雅世濟其美至柳璨而隤其家聲所謂九世卿族一舉而滅之柳玭之家訓為空言矣從才用翻}
戊申朱全忠密令宿衛都指揮使朱友諒以兵圍崔胤第殺胤及鄭元規陳班并胤所親厚者數人|{
	崔胤有誤國之罪無負國之心 考異曰舊傳全忠攻鳳翔胤寓居華州為全忠畫圖王之策又曰天子還宫全忠東歸胤以事權在已慮全忠急於簒代乃與鄭元規謀召致兵甲以扞茂貞為辭全忠知其意從之令汴州軍人入關應募者數百人及友倫死全忠怒遣其子宿衛軍使友諒誅胤而應募者突然而出唐太祖紀年録曰及事權既失知朱温懷簒奪之志慮一朝禍發與國俱亡因圖自安之計與朱温外貌相厚私心漸異與元規密為計畫倍招兵數繕治鎧甲朝夕不止朱温察之乃隂令部下驍果數千紿為散卒於京師應募胤每日教閲弓弩梁卒偽示怯懦或倒弓背矢有若不能胤莫之識俄而朱友倫打毬墜死温愈不悦又聞胤欲挾天子出幸荆襄温乃抗言胤將交亂天下傾覆朝廷宜急誅之無令事發天子將罷胤知政事貶太子賓客鄭元規循州司戶事未行温子友諒引兵攻胤詰旦擒之又攻鄭元規於京府擒之崔鄭俱獻首岐下實録胤重世宰相而志滅唐祚按崔胤隂狡險躁其罪固多然本召全忠欲假其兵力以除宦官耳宦官既誅全忠兵勢益彊遂有簒奪之心胤復欲以譎詐并圖全忠故全忠覺而殺之若云唐室因胤而亡則可矣舊傳云胤為全忠畫圖王之策實録云胤志滅唐祚恐未必然也胤任唐已為上相滅唐立梁於已何益假令胤實有此志則惟患全忠簒代之不速何故復謀拒之此所謂天下之惡皆歸焉者也紀年録序朱崔之情近得其實今從之然紀年録云傳首岐下誤也又全忠之去長安也留步騎萬人何患無兵何必更令汴卒應募若在訓練之際突出擒胤猶須此卒胤既貶官家居一夫可制安用此計邪蓋全忠以胤募兵既多或能圖已故使汴卒應募察其動靜以壞其謀非籍此兵以誅胤也人始不知及誅胤之際皆突出人方知是汴卒耳}
初上在華州|{
	乾寧三年四年上在華州事見二百六十卷二百六十一卷}
朱全忠

屢表請上遷都洛陽|{
	發此機者則崔胤之罪也}
上雖不許全忠常令東都留守佑國軍節度使張全義繕修宫室全忠之克邠州也質靜難軍節度使楊崇本妻子於河中|{
	事見二百六十二卷天復元年質音致難乃旦翻}
崇本妻美全忠私焉既而歸之崇本怒使謂李茂貞曰唐室將滅父何忍坐視之乎|{
	李茂貞養崇本為子更姓名曰李繼徽故呼之為父}
遂相與連兵侵逼京畿復姓名為李繼徽|{
	楊崇本復本姓名見二百六十二卷天復元年}
己酉全忠引兵屯河中丁巳上御延喜樓朱全忠遣牙將寇彦卿奉表稱邠岐兵逼畿甸請上遷都洛陽及下樓裴樞已得全忠移書促百官東行|{
	裴樞為首相且朱全忠所薦也故使之促百官以此觀之謂非朋附全忠可乎}
戊午驅徙士民號哭滿路|{
	號戶刀翻}
罵曰賊臣崔胤召朱温來傾覆社稷使我曹流離至此|{
	歸罪于天復元年胤召朱全忠誅宦官其禍遂至此胤不得不任其責也}
老幼繦屬月餘不絶|{
	繦舉兩翻錢貫也屬之欲翻言老幼相隨而東若繦之貫錢相屬不絶也}
壬戌車駕發長安全忠以其將張廷範為御營使|{
	時以天子東遷扈衛兵士為御營置使以提舉一行事務御營使之官始此}
毁長安宫室百司及民間廬舍取其材浮渭沿河而下長安自此遂丘墟矣全忠發河南北諸鎮丁匠數萬|{
	時河南北諸鎮皆附於朱全忠發丁匠必不及鎮定幽滄四鎮}
令張全義治東都宫室|{
	治直之翻}
江浙湖嶺諸鎮附全忠者皆輸貨財以助之|{
	江則鄂岳杜洪洪州鍾傳浙則錢鏐湖則潭州馬殷澧州雷彦威嶺則廣州劉隱皆附全忠者也}
甲子車駕至華州民夾道呼萬歲上泣謂曰勿呼萬歲朕不復為汝主矣館於興德宫|{
	復扶又翻館古玩翻光化元年上將自華州還長安以華州為興德府以所居府署為興德宫}
謂侍臣曰鄙語云紇干山頭凍殺雀何不飛去生處樂|{
	樂音洛}
朕今漂泊不知竟落何所因泣下霑襟左右莫能仰視二月乙亥車駕至陜|{
	陜失冉翻 考異曰梁實録丁巳詔以今月二十二日先遣士庶出京朕將翌日命駕壬戌襄宗發自秦雍甲子暨華州二月丁卯上至河中乙亥天子駐蹕陜郡翌日上來覲于行在編遺録正月丁酉上聞闕下人心不遑遂往河中以審都邑動静己酉離梁園行至汜水聞崔胤死是時皆言崔胤以下潜諫帝不令東遷雒陽又密與岐鳳交通及斯禍也洎上至蒲津帝謀東幸決取二十一日屬車離長安是日丁巳王鑾東指癸亥到甘棠二月乙亥上離河中丁丑到陜郊戊寅朝上欲躬往洛下催促百工壬辰朝辭明日東邁唐太祖紀年録丁巳下詔與梁實録同又曰壬戌昭宗發長安遷幸洛陽丁卯車駕次華州乙亥駐蹕陜州丙子朱温自汴州迎覲見已先發自此人使相望于路請駕早行幸洛陽舊紀正月己酉全忠帥師屯河中遣牙將寇彦卿奉表請車駕遷都洛陽丁巳車駕發京師癸亥次陜州全忠迎謁于路二月丙寅朔乙亥全忠辭赴洛陽親督工作薛居正五代史梁紀正月辛酉帝發自大梁西赴河中京師聞之為之震懼唐年補録丁巳帝御延喜樓全忠迎扈表至及還宫至暮全忠已移書宰臣裴樞促百官東行是日下詔與梁實録同尋以張廷範為御營使便毁拆宫室沿河而下仍起豪民從行貧者亦繼焉車駕以其月二十三日己未至華州二月丙寅車駕駐陜郊又曰三月三日戊辰車駕離華下其差舛如此實録丁巳全忠遣牙將寇彦卿奉表言慮邠岐兵士侵迫請車駕遷都洛陽乃下詔與梁實録同二月丙寅朔丁卯次華州時朱全忠屯河中乙亥駐陜州丙子全忠來朝又賜王建絹詔云正月二十日朕登樓二十二日東軍兵士擁脇朕東去新紀正月戊午全忠遷唐都于洛陽二月戊寅次陜州朱全忠來朝按梁實録唐紀年録唐年補録唐實録所載詔書皆云二十二日遣士庶出京朕翌日命駕而諸書月日各不同莫有與此詔相應者編遺汴人所録比唐紀年宜得其實而正月二十一日丁巳全忠請遷都表始至長安車駕當日豈能便發長安去陜猶八程而癸亥已到甘棠首尾七日太似怱遽實録全用紀年録正月二十六日始離長安二月二日至華州駐留數日故同以十日至陜差似相近今從之}
以東都宫室未成駐留於陜丙子全忠自河中來朝上延全忠入寢室見何后后泣曰自今大家夫婦委身全忠矣 甲申立皇子禎為端王祈為豐王福為和王禧為登王祐為嘉王 上遣間使以御札告難于王建|{
	間古莧翻使疏吏翻下同難乃旦翻}
建以卭州刺史王宗祐為北路行營指揮使|{
	卭渠容翻}
將兵會鳳翔兵迎車駕至興平遇汴兵不得進而還建始自用墨制除官云俟車駕還長安表聞|{
	楊行密以便宜除官猶曰以李儼將命為據至王建則自為之矣}
三月丁未以朱全忠兼判左右神策及六軍諸衛事

|{
	崔胤既誅朱全忠遂專摠禁衛其實布分私人於天子左右而駕言判其事耳}
癸丑全忠置酒私第|{
	朱全忠奔走兵間得陜州何暇建私第其實以到陜州所即安之地即為私第耳}
邀上臨幸|{
	天王狩於河陽晉文公以諸侯見也仲尼曰以臣召君不可以訓安有置酒私第邀人主臨之者乎}
乙卯全忠辭上先赴洛陽督修宫室上與之宴羣臣既罷上獨留全忠及忠武節度使韓建飲皇后出自捧玉巵以飲全忠|{
	以飲於禁翻}
晉國夫人可證附上耳語建躡全忠足|{
	躡尼輒翻}
全忠以為圖已不飲陽醉而出全忠奏以長安為佑國軍|{
	光啟三年置佑國軍節度於洛陽今遷都洛陽故徙佑國軍於長安 考異曰按河南府先已為佑國軍今京兆府乃與同名者蓋車駕既在河南則無用軍額故移其名於京兆耳天祐二年鄭賨猶為西京留守判官然則雖立軍額京名尚在耳}
以韓建為佑國節度使以鄭州刺史劉知俊為匡國節度使丁巳上復遣間使以絹詔告急於王建楊行密李克用等令糾帥藩鎮以圖匡復|{
	上復扶又翻帥讀曰率 考異曰續寶運録天復四年三月二十二日丑時襄宗在陜府行營密遣絹詔告晉楚蜀末云三月二十三日四月二十七日齎到西川頒示管内州縣實録此月絹詔在四月據十國紀年楊行密三月王建四月得詔與寶運録畧相應今移置此月}
曰朕至洛陽則為所幽閉詔敕皆出其手朕意不復得通矣|{
	昭宗絹詔當時居方面者未必動心而讀其書者往往掩卷}
楊行密遣錢傳璙及其婦并顧全武歸錢塘|{
	錢傳璙為質於楊行密見上卷天復二年}
以淮南行軍司馬李神福為鄂岳招討使復將兵擊杜洪|{
	田頵已平故復遣李神福擊杜洪}
朱全忠遣使詣行密請捨鄂岳復修舊好行密報曰俟天子還長安然後罷兵修好|{
	楊行密之心在廣土朱全忠之心在簒唐全忠力不能救杜洪故有是言行密之報假天討以折其辭其所志不在此也好呼到翻}
夏四月辛巳朱全忠奏洛陽宫室己成請車駕早發表章相繼上屢遣宫人諭以皇后新產未任進路|{
	任音壬堪也}
請俟十月東行全忠疑上徘徊俟變|{
	疑上徘徊以待諸道勤王之師}
怒甚謂牙將寇彦卿曰汝速至陜即日促官家發來|{
	以臣迎君此何等語華督有無君之心而後動於惡君子於其攻孔氏之時始知之若朱全忠之心徵于色發于聲為有君乎為無君乎又按西漢羣臣謂天子為縣官東漢以來謂為國家唐時宫中率呼天子為宅家又羣小呼之為官家或曰其義蓋取五帝官天下三王家天下}
閏月丁酉車駕發陜壬寅全忠逆於新安|{
	九域志新安縣在洛陽西七十里}
上之在陜也司天監奏星氣有變期在今秋不利東行|{
	此椒殿弑逆之徵也天之垂象示戒豈不昭昭也哉}
故上欲以十月幸洛至是全忠令醫官許昭遠告醫官使閻祐之司天監王墀内都知韋周晉國夫人可證等謀害元帥悉收殺之|{
	唐末置醫官使以主醫官内都知盛唐知内侍省之職事也至宋沿唐之制有内侍省左右班都都知左右班都知副都知閻佑之王墀之死以言星氣也韋周可證之死以附耳語也元帥朱全忠}
癸卯上憩於穀水|{
	穀水在洛城西憩音去例翻}
自崔胤之死六軍散亡俱盡所餘擊毬供奉内園小兒共二百餘人從上而東 |{
	考異曰後唐紀年録云五百人實録據之今從舊紀薛史}
全忠猶忌之為設食於幄盡縊殺之|{
	為于偽翻縊一既翻又於賜翻}
豫選二百餘人大小相類者衣其衣服|{
	衣其于既翻}
代之侍衛上初不覺累日乃寤自是上之左右職掌使令|{
	令音零}
皆全忠之人矣甲辰車駕發穀水入宫御正殿受朝賀|{
	時以貞觀殿為正殿崇勲殿為入閣朝直遥翻}
乙巳御光政門|{
	時遷洛之後易宫門名改長樂門為光政門}
赦天下改元|{
	改元天祐}
更命陜州曰興唐府|{
	更工衡翻}
詔討李茂貞楊崇本戊申敕内諸司惟留宣徽等九使|{
	時惟留宣徽兩院小馬坊豐德庫御㕑客省閤門飛龍莊宅九使}
外餘皆停廢仍不以内夫人充使 |{
	考異曰編遺録曰戊申鑾輿初到洛都經費甚廣况國用未豐庶事草創因刪畧閒冗司局今後除留宣徽等九使外餘並停廢仍不差内中夫人充使蓋初誅宦官後内諸司使皆以内夫人領之至此使用外人也而實録改充使為宣事誤也 按宦官既誅以内夫人宣傳詔命及充内諸司使夫既宣傳詔命則實録云宣事亦未為誤但天祐三年方罷宫人宣傳詔命故以實録為誤}
以蔣玄暉為宣徽南院使兼樞密使王殷為宣徽北院使兼皇城使張廷範為金吾將軍充街使以韋震為河南尹兼六軍諸衛副使又徵武寧留後朱友恭為左龍武統軍保大節度使氏叔琮為右龍武統軍典宿衛皆全忠之腹心也癸丑以張全義為天平節度使乙卯以全忠為護國宣武宣義忠武四鎮節度使|{
	朱全忠先為宣武天平宣義護國四鎮節度使以張全義有積年葺理洛陽之功今洛陽建都不為節鎮故以天平授全義而已兼忠武為四鎮}
鎮海鎮東節度使越王錢鏐求封吳越王朝廷不許朱全忠為之言於執政乃更封吳王|{
	天復元年錢鏐封越王為于偽翻更工衡翻}
更命魏博曰天雄軍|{
	代宗以魏博為天雄軍以寵田承嗣至德宗時田悦逆命後復歸順命為魏博節度使今復舊天雄軍號}
癸亥進天雄節度使長沙郡王羅紹威爵鄴王

資治通鑑卷二百六十四
