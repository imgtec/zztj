






























































資治通鑑卷二百八十四 宋 司馬光 撰

胡三省 音註

後晉紀五|{
	起閼逢執徐二月盡旃蒙大荒落七月凡一年有奇}


齊王中

開運元年二月甲辰朔命前保義節度使石贇守麻家口前威勝節度使何重建守楊劉鎮護聖都指揮使白再榮守馬家口西京留守安彦威守河陽|{
	贇於倫翻按是時凡緣河津要皆以兵守之亦由燕冀瀛莫既入于北遼人南寇了無關山塘濼之阻其兵可以徑造河上故不得不緣河為備也}
未幾周儒引契丹滿達勒自馬家口濟河營於東岸攻鄆州北津以應楊光遠滿達勒契丹主從弟也|{
	幾居豈翻從才用翻鄆音運}
乙巳遣侍衛馬軍都揮使義成節度使李守貞神武統軍皇甫遇陳州防禦使梁漢璋懷州刺史薛懷讓將兵萬人緣河水陸俱進守貞河陽漢璋應州懷讓太原人也丙午契丹圍高行周符彦卿及先鋒指揮使石公霸於戚城|{
	春秋時戚屬衛地河上邑也東坡指掌圖以為衛之戚今在博州界按是時晉與契丹相拒于澶衛之間此戚城當在澶州之北魏州之南疑不在博州之界也}
先是景延廣令諸將分地而守無得相救行周等告急延廣徐白帝帝自將救之契丹解去三將泣訴救兵之緩幾不免|{
	幾居衣翻}
戊申李守貞等至馬家口契丹遣步卒萬人築壘散騎兵於其外餘兵數萬屯河西船數千艘度兵未已|{
	艘蘇遭翻}
晉兵薄之契丹騎兵退走晉兵進攻其壘拔之契丹大敗乘馬赴河溺死者數千人俘斬亦數千人河西之兵慟哭而去由是不敢復東|{
	楊光遠之援絶矣復扶又翻}
辛亥定難節度使李彞殷奏將兵四萬自麟州濟河侵契丹之境|{
	定難軍夏州九域志麟州西北至夏州一百二十里自麟州東北至府州又自府州東北行入契丹境難乃旦翻}
壬子以彛殷為契丹西南面招討使初契丹主得貝州博州皆撫慰其人或拜官賜服章及敗於戚城及馬家口忿恚|{
	恚於避翻}
所得民皆殺之得軍士燔灸之由是晉人憤怒戮力争奮楊光遠將青州兵欲西會契丹戊午詔石贇分兵屯鄆州以備之|{
	石贇時屯麻家口}
詔劉知遠將部兵自土門出恒州擊契丹又詔會杜威馬全節於邢州知遠引兵屯樂平不進|{
	樂平離太原三百餘里耳}
帝居喪期年即於宫中奏細聲女樂|{
	細聲女樂欲其不聞于外也}
及出師常令左右奏三絃琵琶和以羌笛|{
	和戶卧翻}
擊鼓歌舞曰此非樂也庚申百官表請聽樂詔不許|{
	居喪而納叔母尚何責乎聽樂}
壬戌楊光遠圍棣州刺史李瓊出兵擊敗之|{
	楊光遠自青州歷淄州而圍棣州敗補賣翻}
光遠燒營走還青州|{
	還從宣翻又如字}
癸亥以前威勝節度使何重建為東面馬步都部署將兵屯鄆州階州義軍指揮使王君懷帥所部千餘人叛降蜀請為鄉道以取階成|{
	鄉讀曰向階成二州名}
甲子蜀人攻階州 契丹偽弃元城去伏精騎於古頓丘城|{
	頓丘漢古縣爾雅丘一成曰頓丘後移治所於陰安城唐頓丘縣又移治於陰安城之南天福三年徙澶州跨德勝津併頓丘縣徙焉頓丘凡三徙矣古城盖陰安城也}
以俟晉軍與恒定之兵合而擊之|{
	時詔杜威馬全節以兵來會契丹欲俟其合而邀擊之}
鄴都留守張從恩屢奏虜已遁去大軍欲進追之會霖雨而止契丹設伏旬日人馬飢疲趙延夀曰晉軍悉在河上畏我鋒鋭必不敢前不如即其城下|{
	即就也}
四合攻之奪其浮梁|{
	謂澶州德勝渡之河梁也}
則天下定矣契丹主從之三月癸酉朔自將兵十餘萬陳於澶州城北|{
	宋白曰契丹時駐兵澶州鐵丘陳讀曰陣下同}
東西横掩城之兩隅登城望之不見其際高行周前軍在戚城之南與契丹戰自午至晡互有勝負契丹主以精兵當中軍而來帝亦出陳以待之契丹主望見晉軍之盛謂左右曰楊光遠言晉兵半已餒死|{
	楊光遠誘契丹入寇見上卷上年}
今何其多也以精騎左右略陳晉軍不動萬弩齊發飛矢蔽地契丹稍却又攻晋陳之東偏不克苦戰至暮兩軍死者不可勝數|{
	勝音升}
昏後契丹引去營於三十里之外|{
	不敢逼城而營懼晉軍政劫也}
乙亥契丹主帳中小校竊其馬亡來云契丹已傳木書收軍北去|{
	校戶教翻木書者書之于木以為信契}
景延廣疑其詐閉壁不敢追漢主命中書令都元帥越王弘昌謁烈宗陵於海曲

|{
	劉龔舉大號追尊其兄隐為烈宗}
至昌華宫使盜殺之 契丹主自澶州北分為兩軍一出滄德一出深冀而歸所過焚掠方廣千里|{
	廣古曠翻}
民物殆盡留趙延照為貝州留後麻荅䧟德州擒刺史尹居璠|{
	璠音煩}
閩拱宸都指揮使朱文進閤門使連重遇既弑康宗|{
	見二百八十二卷天福四年}
常懼國人之討相與結昏以自固閩主曦果於誅殺嘗遊西園因醉殺控鶴指揮使魏從朗從朗朱連之黨也又嘗酒酣誦白居易詩云惟有人心相對間咫尺之情不能料因舉酒屬二人|{
	易以豉翻屬之欲翻}
二人起流涕再拜曰臣子事君父安有它志曦不應二人大懼李后妬尚賢妃之寵欲弑曦而立其子亞澄|{
	尚賢妃有寵見上卷天福八年閩王之永隆四年也亞澄時封閩王}
使人告二人曰主上殊不平於二公奈何會后父李真有疾乙酉曦如真第問疾文進重遇使拱宸馬步使錢逹弑曦於馬上召百官集朝堂告之曰太祖昭武皇帝光啟閩國|{
	朝直遥翻閩主王璘舉大號追尊其父審知曰太祖昭武皇帝}
今子孫淫虐荒墜厥緒天厭王氏宜更擇有德者立之|{
	更工行翻}
衆莫敢言重遇乃推文進升殿被衮冕|{
	被皮義翻}
帥羣臣北面再拜稱臣|{
	帥讀曰率}
文進自稱閩主悉收王氏宗族延喜以下少長五十餘人皆殺之|{
	延喜閩主之弟也少詩照翻長知兩翻}
葬閩主曦謚曰睿文廣武明聖元德隆道大孝皇帝廟號景宗以重遇搃六軍禮部尚書判三司鄭元弼抗辭不屈黜歸田里將奔建州|{
	欲奔王延政也}
文進殺之文進下令出宫人罷營造以反曦之政殷主延政遣統軍使吳成義將兵討文進不克文進加樞密使鮑思潤同平章事以羽林統軍使黄紹頗為泉州刺史左軍使程文緯為漳州刺史汀州刺史同安許文稹舉郡降之|{
	九域志泉州同安縣在州西一百三十五里盖王氏所置也宋白曰開元十九年析泉州南安縣界四鄉置大同場閩王升為同安縣稹章忍翻}
丁亥詔太原恒安兵各還本鎮|{
	契丹已退故也}
辛卯馬全節攻契丹泰州拔之|{
	五代會要後唐天成三年升奉化軍為泰州以清苑縣為理所至晉開運二年九月移治滿城縣至周廣順二年二月廢州其滿城縣割隸易州時馬全節自定州攻泰州}
勑天下籍鄉兵每七戶共出兵械資一卒 秦州兵救階州出黄階嶺敗蜀兵於西平|{
	敗補賣翻}
漢以戶部侍郎陳偓同平章事 夏四月丁未緣河巡檢使梁進以鄉社兵復取德州|{
	鄉社兵民兵也時契丹寇掠緣河之民自備兵械各随其鄉團結為社以自保衛契丹陷德州而北歸梁進乘其去而復取之}
己酉命歸德節度使高行周保義節度使王周留鎮澶州庚戌帝發澶州甲寅至大梁侍衛馬步都指揮使天平節度使同平章事景延廣既為上下所惡|{
	上謂將相大臣下謂軍民惡烏路翻}
帝亦憚其不遜難制桑維翰引其不救戚城之罪|{
	引牽也牽發其罪猶人收捲衣物於懷袖間從而牽出之然}
辛酉加延廣兼侍中出為西京留守|{
	晉徙都汴以河南府為西京}
以歸德節度使兼侍中高行周為侍衛馬步都指揮使延廣鬱鬱不得志|{
	橛豎小人得權則驕溢使氣失權則鬱鬱不得志乃其常也}
見契丹彊盛始憂國破身危遂日夜縱酒|{
	自知無復全地苟取朝夕之樂}
朝廷因契丹入寇國用愈竭復遣使者三十六人分道括率民財各封劔以授之|{
	示使專斷斬此以威脅取民財也復扶又翻}
使者多從吏卒攜鎖械刀杖入民家小大驚懼求死無地州縣吏復因緣為姦|{
	復扶又翻下同}
河南府出緡錢二十萬|{
	此括率合出之數也}
景延廣率三十七萬|{
	景延廣增率十七萬欲以入已}
留守判官盧億言於延廣曰公位兼將相富貴極矣今國家不幸府庫空竭不得已取於民公何忍復因而求利為子孫之累乎|{
	累力瑞翻}
延廣慙而止|{
	史言景延廣差愈于杜重威}
先是詔以楊光遠判命兖州脩守備|{
	青兖鄰鎮故命之為備先昔薦翻}
泰寧節度使安審信以治樓堞為名|{
	堞逹協翻}
率民財以實私藏|{
	藏徂浪翻下同}
大理卿張仁愿為括率使至兖州賦緡錢十萬值審信不在|{
	不在者適不在鎮}
拘其守藏吏指取錢一囷已滿其數|{
	史言晉之藩鎮利國有難浚民以肥家}
戊寅命侍衛馬步軍都虞候泰寧節度使李守貞將步騎二萬討楊光遠於青州|{
	李守貞盖代安審信帥泰寧也}
又遣神武統軍洛陽潘環及張彦澤等將兵屯澶州以備契丹契丹遣兵救青州齊州防禦使堂陽薛可言邀撃敗之|{
	堂陽縣屬冀州宋皇祐四年省縣為鎮入南宫縣九域志曰地在堂水之陽敗補賣翻}
丙戌詔諸州所籍鄉兵號武定軍凡得七萬餘人時兵荒之餘復有此擾民不聊生|{
	異時契丹入汴武定軍曷嘗能北向發一矢乎復扶又翻}
丁亥鄴都留守張從恩上言趙延照雖據貝州麾下兵皆久客思歸宜速進軍攻之詔以從恩為貝州行營都部署督諸將擊之辛卯從恩奏趙延照縱火大掠弃城而遁屯於瀛莫阻水自固|{
	瀛莫之間多水濼故趙延照阻以為固瀛莫相去一百一十里}
朱文進遣使如唐唐主囚其使將伐之|{
	唐主欲討朱文進弑君之罪}
會天暑疾疫而止 六月辛酉官軍拔淄州斬其刺史劉翰|{
	淄州楊光遠之巡屬也}
太尉侍中馮道雖為首相|{
	馮道自唐潞王之時已正拜三公晉高祖入洛用以為相位任在執政之右}
依違兩可無所操決|{
	此馮道保身固位之術一生所受用者也操七刀翻}
或謂帝曰馮道承平之良相今艱難之際譬如使禪僧飛鷹耳|{
	言禪以静寂為宗僧以慈悲不殺為教為禪僧者第能機辯無窮而不能應物使之飛鷹搏撃非其任也}
癸卯以道為匡國節度使兼侍中|{
	出馮道鎮同州將别命相也}
乙巳漢主幽齊王弘弼于私第 或謂帝曰陛下欲禦北狄安天下非桑維翰不可|{
	請罷馮道請用桑維翰盖出一人之口前史謂維翰倩人以言于帝通鑑皆曰或者疑其辭}
丙午復置樞密院|{
	罷樞密院見二百八十二卷高祖天福四年}
以維翰為中書令兼樞密使事無大小悉以委之數月之間朝廷差治|{
	治直吏翻}
滑州河決浸汴曹單濮鄆五州之境環梁山合于汶|{
	梁山在鄆州夀張縣汶水自東北來與濟水會于梁山東北今決河之水瀰漫環梁山而合于汶單音善濮音卜環音宦汶音問}
詔大發數道丁夫塞之|{
	塞昔則翻}
既塞帝欲刻碑紀其事中書舍人楊昭儉諫曰陛下刻石紀功不若降哀痛之詔染翰頌美不若頒罪已之文帝善其言而止 初高祖割北邉之地以賂契丹|{
	事見二百八十卷高祖天福元年}
由是府州刺史折從遠亦北屬|{
	府州領府谷一縣後唐以麟州東北河濱之地置宋白曰府州本河西蕃界府谷鎮上人拆嗣倫世為鎮將後唐莊宗天祐七年升鎮為府谷縣八年升建府州以扼蕃界以嗣倫男從遠為刺史折姓從遠名姓氏略折常列翻}
契丹欲盡徙河西之民以實遼東州人大恐從遠因保險拒之及帝與契丹絶遣使諭從遠使攻契丹從遠引兵深入拔十餘寨戊午以從遠為府州團練使從遠雲州人也|{
	歐史曰折從遠雲中人盖指古雲中郡大界言之}
甲子復置翰林學士|{
	廢翰林學士見二百八十二卷天福五年}
戊辰以右散騎常侍李慎儀為兵部侍郎翰林學士承旨都官郎中劉温叟金部郎中知制誥武強徐台符|{
	武強縣屬深州九域志在州東北六十里}
禮部郎中李澣主客員外郎宗城范質皆為學士温叟岳之子也|{
	劉岳見二百五十卷唐明宗天成元年}
秋七月辛未朔大赦改元|{
	改元天運}
己丑以太子太傅劉昫為司空兼門下侍郎同平章事 八月辛丑朔以河東節度使劉知遠為北面行營都統順國節度使杜威為都招討使督十三節度以備契丹桑維翰兩秉朝政出楊光遠景延廣於外|{
	楊光遠景延廣先皆嘗摠宿衛兵天福初桑維翰秉政出楊光遠是時再秉政出景延廣朝直遥翻}
至是一制指揮節度使十五人無敢違者|{
	劉知遠杜威并十三節度為十五人按薛史載十三節度鄆州張從恩充馬步都監西京留守景延廣充都排陣使徐州趙在禮充都虞候晉州安叔千充左廂排陣使前兖帥安審信充右廂河中安審琦充馬步都指揮使河陽符彦卿充馬軍左廂滑州皇甫遇充右廂右神武統軍張彦澤充馬軍排陣使滄州王延胤充步軍左廂都指揮使陜州宋彦筠充右廂前金帥田武充步軍左廂排陣使右廂武統軍潘環充右廂}
時人服其膽略朔方節度使馮暉上章自陳未老可用而制書見遺維翰詔禁直學士|{
	詔禁直學士者以詔旨詔之也禁直學士學士之入直禁中者也}
使為答詔曰非制書忽忘|{
	忘巫放翻}
實以朔方重地非卿無以彈壓比欲移卿内地|{
	比毗至翻}
受代亦須奇才|{
	受當作授}
暉得詔甚喜時軍國多事百司及使者咨請輻輳維翰随事裁決初若不經思慮人疑其疎略退而熟議之亦終不能易也然為相頗任愛憎一飯之恩睚眦之怨必報人以此少之|{
	史稱桑維翰之長併及其短所以明是非示勸警睚五懈翻眦士懈翻少詩照翻}
契丹之入寇也帝再命劉知遠會兵山東|{
	太原以河北之地為山東帝初詔劉知遠自土門出恒州尋又詔會兵邢州並見上}
皆後期不至帝疑之謂所親曰太原殊不助朕必有異圖果有分何不速為之|{
	言若有分為天子何不速為之怒之之辭也分扶問翻}
至是雖為都統而實無臨制之權密謀大計皆不得預知遠亦自知見疎但慎事自守而已郭威見知遠有憂色謂知遠曰河東山川險固|{
	河東治晉陽東阻太行常山西限龍門西河南有霍太山雀鼠谷之隘北有㕍門五臺諸山之險故云然}
風俗尚武土多戰馬|{
	此所謂恃險與馬也}
静則勤稼穡動則習軍旅此霸王之資也|{
	王于况翻}
何憂乎 朱文進自稱威武留後權知閩國事遣使奉表稱藩于晉癸丑以文進為威武節度使知閩國事 癸亥置鎮寧軍於澶州以濮州隸焉|{
	割天平巡屬之濮州以隸鎮寧軍}
初吳濠州刺史劉金卒子仁規代之仁規卒子崇俊代之唐烈祖置定遠軍於濠州|{
	唐置定遠軍于濠州通鑑書于天福八年三月元宗即位之後見上卷}
以崇俊為節度使會清淮節度使姚景卒|{
	唐置清淮軍于夀州}
崇俊厚賂權要求兼領夀州唐主陽為不知其意徙崇俊為清淮節度使以楚州刺史劉彦貞為濠州觀察使馳往代之崇俊悔之彦貞信之子也|{
	劉信事吳楊氏四世有戰功}
九月庚午朔日有食之 丙子契丹寇遂城樂夀|{
	遂城縣屬易州宋太平興國六年置威虜軍景德元年改廣信軍在易州東南八十里當五迴嶺及狼山之要金置遂州樂夀縣屬深州宋分屬瀛州九域志在瀛州之南八十里}
深州刺史康彦進擊却之 冬十月丙午漢主毒殺鎮王弘澤于邕州 殷主延政遣其將陳敬佺以兵三千屯尤溪及古田|{
	唐永泰二年分侯官尤溪置古田縣屬福州九域志在州西北一百八十里尤溪縣在南劔州南一百九十五里宋白曰按尤溪縣理今當延平東南二百四十里在福州西北八百三十五里其地與漳州龍巖縣劔州沙縣及福州侯官縣三處交界山洞幽深灘溪險峻内有千里諸境逃人多投此洞開元二十八年經畧使唐修忠招諭其人因以名縣此源先號尤溪因名古田縣亦開元二十九年開山洞置}
盧進以兵二千屯長溪|{
	唐武德六年置長溪縣屬福州九域志在州東北三百四十五里宋白曰長溪縣本漢閩縣地唐置温麻縣以縣界温麻溪為名天寶九年改為長溪縣}
泉州散員指揮使桃林留從效|{
	九域志泉州有桃林溪盖留從效所居之地散昔亶翻}
謂同列王忠順董思安張漢思曰朱文進屠滅王氏遣腹心分據諸州吾屬世受王氏恩而交臂事賊一旦富沙王克福州|{
	殷王延政本封富沙王}
吾屬死有餘愧衆以為然十一月從效等各引軍中所善壮士夜飲於從效之家從效紿之曰富沙王已平福州密旨令吾屬討黄紹頗|{
	朱文進時以黄紹頗為泉州刺史}
吾觀諸君狀貌皆非久處貧賤者從吾言富貴可圖不然禍且至矣衆皆踊躍操白梃踰垣而入執紹頗斬之|{
	處昌呂翻操七刀翻梃大鼎翻}
從效持州印詣王繼勲第請主軍府從效自稱平賊統軍使函紹頗首遣副兵馬使臨淮陳洪進齎詣建州|{
	唐長安四年分徐城南界兩鄉于沙熟淮口置臨淮縣開元二十三年移治泗州郭下陳洪進盖本臨淮人而從軍泉州}
洪進至尤溪福州戍兵數千遮道洪進紿之曰義師已誅朱福州|{
	朱文進據福州故以稱之}
吾倍道逆嗣君于建州|{
	嗣君謂殷王延政當嗣有閩國}
爾輩尚守此何為乎以紹頗首示之衆遂潰大將數人從洪進詣建州延政以繼勲為侍中泉州刺史從效忠順思安洪進皆為都指揮使漳州將程謨聞之|{
	按九域志泉州西南至漳州三百六十里鄰郡也}
亡殺刺史程文緯|{
	亡當作立筆誤也否則亦字}
立王繼成權州事繼勲繼成皆延政之從子也|{
	從才用翻}
朱文進之滅王氏|{
	事見上三月}
二人以疎遠獲全汀州刺史許文稹奉表請降於殷|{
	稹上忍翻}
十二月癸丑加朱文進同平章事封閩國王|{
	癸丑大梁出命之日也命未逹而文進誅矣}
李守貞圍青州經時|{
	是年五月李守貞圍青州}
城中食盡餓死者大半契丹援兵不至楊光遠遥稽首于契丹|{
	稽音啟}
曰皇帝皇帝誤光遠矣其子承勲承祚承信勸光遠降冀全其族光遠不許曰吾昔在代北嘗以紙錢祭天池而沈|{
	楊光遠本沙陀部人居代北天池即汾陽縣之天池時屬嵐州静樂縣界沈持林翻}
人皆言當為天子姑待之丁巳承勲斬勸光遠反者節度判官丘濤等送其首于守貞縱火大譟劫其父出居私第上表待罪開城納官軍朱文進聞黄紹頗死大懼以重賞募兵二萬遣統軍

使林守諒内客省使李廷鍔將之攻泉州鉦鼔相聞五百里|{
	福州至泉州不及四百里史家張大以言其聲勢耳將即亮翻}
殷主延政遣大將軍杜進將兵二萬救泉州留從效開門與福州兵戰大破之斬守諒執廷鍔延政遣統軍使吳成義帥戰艦千艘攻福州|{
	艦戶黯翻艘疎刀翻}
朱文進遣子弟為質於吳越以求救|{
	質音致}
初唐翰林待詔臧循|{
	盛唐之時有翰林待詔以處伎藝之人}
與樞密副使查文徽同鄉里循常為賈人習福建山川為文徽畫取建州之策|{
	賈音古為文于偽翻}
文徽表請用兵擊王延政國人多以為不可唐主以文徽為江西安撫使循行境上覘其可否|{
	行下孟翻覘丑亷翻又丑艷翻}
文徽至信州奏言攻之必克唐主以洪州營屯都虞候邉鎬為行營招討諸軍都虞候將兵從文徽伐殷文徽自建陽進屯盖竹|{
	唐武德四年分建安縣置建陽縣屬建州建陽在建州西一百三十里建陽縣之南二十五里有地名盖竹}
聞漳泉汀三州皆降于殷殷將張漢卿自鏞州將兵八千將至文徽懼退保建陽臧循屯邵武|{
	邵武亦本漢治縣之地吳于此立昭武鎮隋平吳更昭武鎮曰邵武縣隋廢而復置唐屬建州九域志在州西南二百七十里宋白曰邵武縣本東侯官縣之北鄉也孫策置南平縣吳景帝三年置昭武縣晉太康三年改為邵武}
邵武民導殷兵襲破循軍執循送建州斬之 朝廷以楊光遠罪大而諸子歸命難於顯誅命李守貞以便宜從事閏月癸酉守貞入青州遣人拉殺光遠於别第以病死聞|{
	拉盧合翻}
丙戌起復楊承勲除汝州防禦使|{
	昔楚令尹子南以罪誅其子弃疾以不忍弃父事讐而死李懷光之反河中既破唐德宗欲活其子璀而不可得彼二子者以父子之親居君臣之變審義安命以死殉親夫豈不樂生義不可也若楊承勲兄弟出于蕃洛梟獍其心囚父歸命以希苟活晉朝以不殺降為說于理且未安又從而録用之宜異時契丹得假大義以洩其憤也}
殷吳成義聞有唐兵詐使人告福州吏民曰唐助我討賊臣大兵今至矣福人益懼乙未朱文進遣同平章事李光準等奉國寶于殷丁酉福州南廊承旨林仁翰|{
	南廊承旨閩所置官盖亦侍衛武臣之職也}
謂其徒曰吾曹世事王氏今受制賊臣富沙王至何面見之帥其徒三十人被甲趣連重遇第|{
	帥讀曰率被皮義翻趣七喻翻}
重遇方嚴兵自衛三十人者望之稍稍遁去仁翰執槊直前刺重遇殺之|{
	刺七亦翻}
斬其首以示衆曰富沙王且至汝輩族矣今重遇已死何不亟取文進以贖罪衆踊躍從之遂斬文進迎吳成義入城函二首送建州 契丹復大舉入寇|{
	復扶又翻}
盧龍節度使趙延夀引兵先進|{
	契丹復以趙延夀為軍鋒}
契丹前鋒至邢州順國節度使杜威遣使間道告急|{
	契丹前鋒已至邢州恒州信使路絶故間道而來間古莧翻}
帝欲自將拒之會有疾|{
	將即亮翻}
命天平節度使張從恩鄴都留守馬全節護國節度使安審琦會諸道兵屯邢州武寧節度使趙在禮屯鄴都|{
	馬全節自鄴都進兵邢州令趙在禮自徐州進屯鄴都為後鎮}
契丹主以大兵繼至建牙於元氏|{
	元氏縣屬恒州九域志在州南九十八里}
朝廷憚契丹之盛詔從恩等引兵稍却于是諸軍忷懼無復部伍|{
	忷許拱翻復扶又翻下同}
委弃器甲所過焚掠比至相州不復能整|{
	比毗至翻}


二年春正月詔趙在禮還屯澶州馬全節還鄴都|{
	還從宣翻}
又遣右神武統軍張彦澤屯黎陽西京留守景延廣自滑州引兵守胡梁渡庚子張從恩奏契丹逼邢州詔滑州鄴都復進軍拒之義成節度使皇甫遇將兵趣邢州|{
	皇甫遇奉詔自滑州進兵趣七喻翻}
契丹寇邢洺磁三州殺掠殆盡入鄴都境|{
	九域志鄴都之境西距磁州五十五里西北距洺州五十里磁墻之翻}
壬子張從恩馬全節安審琦悉以行營兵數萬陳於相州安陽水之南|{
	陳讀曰陣相息亮翻}
皇甫遇與濮州刺史慕容彦超將數千騎前覘契丹|{
	覘丑亷翻又丑艶翻}
至鄴縣|{
	鄴漢古縣唐屬相州在州東北劉昫曰鄴魏相州治所隋文輔政尉遲迴舉兵既討平之乃焚鄴城徙其居人南遷四十五里以安陽城為相州治所隋煬帝于鄴故都大慈寺置鄴縣唐貞觀八年始築今治所小城余按此皆言鄴縣也若五代唐晉之所謂鄴都則今魏州大名府是也非鄴縣也夷考此時契丹與晉兵相距本末前所謂入鄴都境當作入相州境一說虜騎散漫大勢兵馬向相州遊騎亦有入鄴都境者}
將度漳水遇契丹數萬遇等且戰且却至榆林店契丹大至二將謀曰吾属今走死無遺矣乃止布陳|{
	陳讀曰陣下同}
自午至未力戰百餘合相殺傷甚衆遇馬斃因步戰其僕杜知敏以所乘馬授之遇乘馬復戰|{
	復扶又翻}
久之稍解顧知敏已為契丹所擒遇曰知敏義士不可弃也與彦超躍馬入契丹陳取知敏而還|{
	還從宣翻下同}
俄而契丹繼出新兵來戰二將曰吾属勢不可走以死報國耳日且暮安陽諸將怪覘兵不還安審琦曰皇甫太師寂無音問必為虜所困語未卒|{
	卒子恤翻}
有一騎白遇等為虜數萬所圍審琦即引騎兵出將救之張從恩曰此言未足信必若虜衆猥至|{
	猥雜也雜然而至言其數多不可勝計也}
盡吾軍恐未足以當之公往何益審琦曰成敗天也萬一不濟當共受之借使虜不南來坐失皇甫太師|{
	按皇甫遇未必加官至太師也而安審琦以太師稱之盖五季之亂官賞無章當時相稱謂不復論其品秩就人臣極品而稱之}
吾属何顔以見天子遂踰水而進契丹望見塵起即解去|{
	知援兵來故解而去}
遇等乃得還與諸將俱歸相州軍中皆服二將之勇彦超本吐谷渾也與劉知遠同母|{
	吐谷渾慕容涉歸之庶長子故其種姓慕容氏}
契丹亦引軍退其衆自相驚曰晉軍悉至矣時契丹主在邯郸聞之即時北遁不再宿至鼓城|{
	邯鄲縣属磁州在州東北七十里鼓城縣属恒州宋端拱二年以鼓城隸祁州在州西南一百里自邯郸至鼔城約三百餘里}
是夕張從恩等議曰契丹傾國而來吾兵不多城中糧不支一旬萬一姦人往告吾虛實虜悉衆圍我死無日矣不若引軍就黎陽倉南倚大河以拒之可以萬全議未決從恩引兵先發諸軍繼之擾亂失亡復如發邢州之時|{
	復扶又翻}
從恩留步兵五百守安陽橋夜四鼓知相州事符彦倫謂將佐曰此夕紛紜人無固志五百弊卒安能守橋即召入乘城為備至曙望之契丹數萬騎已陳于安陽水北|{
	契丹主雖先北遁而趙延夀與特哩衮軍猶南向而不去陳讀曰陣下同}
彦倫命城上揚旌鼓譟約束|{
	約束者申嚴號令也}
契丹不測日加辰趙延壽與契丹特哩衮衆踰水環相州而南|{
	帥讀曰率環音宦}
詔右神武統軍張彦澤將兵趣相州延壽等至湯陰聞之|{
	湯陰本漢蕩陰後并入安陽唐武德四年分安陽置湯源縣貞觀元年改為湯陰属相州九域志在州南四十里}
甲寅引還|{
	還從宣翻又如字}
馬全節等擁大軍在黎陽不敢追延夀悉陳甲騎於相州城下若將攻城狀符彦倫曰此虜將走耳出甲卒五百陳於城北以待之契丹果引去以天平節度使張從恩權東京留守庚申振武節度使折從遠擊契丹圍勝州遂攻朔州|{
	時折從遠守府州命領鎮武節度使勝州本係天福初所割十六州之數契丹乘勢併取之也匈奴須知朔州東至燕京一千里宋白曰勝州正東至黄河四十里去朔州四百二十里}
帝疾小愈河北相繼告急帝曰此非安寢之時乃部分諸將為行計|{
	分扶問翻}
更命武定軍曰天威軍|{
	去年夏籍諸州鄉兵為武定軍更工行翻}
北面副招討使馬全節等奏據降者言虜衆不多宜乘其散歸種落|{
	種章勇翻}
大舉徑襲幽州帝以為然徵兵諸道壬戌下詔親征乙丑帝發大梁 閩之故臣共迎殷王延政請歸福州改國號曰閩延政以方有唐兵未暇徙都以從子門下侍郎同平章事繼昌都督南都内外諸軍事鎮福州|{
	殷主居建州故以福州為南都}
以飛捷指揮使黄仁諷為鎮遏使將兵衛之林仁翰至福州|{
	林仁翰既誅朱連故自福州至建州見王延政福州當作建州}
閩主賞之甚薄仁翰未嘗自言其功發南都侍衛及兩軍甲士萬五千人詣建州以拒唐|{
	福州侍衛之外有左右軍置軍使以領之或曰兩軍謂拱宸控鶴兩都也}
二月壬辰朔帝至滑州命安審琦屯鄴都甲戌帝發滑州乙亥至澶州己卯馬全節等諸軍以次北上|{
	上時兩翻}
劉知遠聞之曰中國疲弊自守恐不足乃横挑強胡|{
	挑徒了翻}
勝之猶有後患况不勝乎契丹自恒州還|{
	還從宣翻又如字}
以羸兵驅牛羊|{
	羸倫為翻}
過祁州城下|{
	以誘城下也}
刺史下邳沈斌出兵擊之|{
	斌悲巾翻}
契丹以精騎奪其城門州兵不得還|{
	還從宣翻}
趙延夀知城中無餘兵引契丹急攻之斌在上|{
	在字之下當逸城字}
延夀語之曰沈使君吾之故人擇禍莫若輕|{
	語牛倨翻擇禍莫若輕引文子之言}
何不早降斌曰侍中父子失計陷身虜庭|{
	言趙延夀與其父德鈞不能救張敬逹邀契丹求帝中國玩寇致禍並為俘虜也趙延夀聞斌言尚欲復求帝乎陷身事見二百八十卷高祖天福元年趙延夀在唐時加侍中沈斌稱其舊官}
忍帥犬羊以殘父母之邦|{
	帥讀曰率}
不自愧耻更有驕色何哉沈斌弓折矢盡寧為國家死耳|{
	折而設翻為于偽翻}
終不効公所為明日城陷斌自殺 丙戌詔北面行營都招討使杜威以本道兵會馬全節等進軍 端明殿學士戶部侍郎馮玉宣徽北院使權侍衛馬步都虞候太原李彦韜皆挾恩用事惡中書令桑維翰數毁之|{
	惡烏路翻數所角翻}
帝欲罷維翰政事李崧劉昫固諫而止維翰知之請以玉為樞密副使玉殊不平丙申中旨以玉為戶部尚書樞密使以分維翰之權|{
	馮玉以后兄進故旨由中出詩云婦有長舌維厲之階信矣}
彦韜少事閻寶|{
	少詩照翻}
為僕夫後隸高祖帳下高祖自太原南下留彦韜侍帝為腹心|{
	高祖留守太原見二百八十卷天福元年}
由是有寵性纎巧與嬖幸相結以蔽帝耳目帝委信之至于升黜將相亦得預議常謂人曰吾不知朝廷設文官何所用且欲澄汰徐當盡去之|{
	去音羌呂翻嗚呼此等氣習自唐劉蕡已為文宗言之李彦韜史弘肇當右武之世張其氣而奮其舌以其人品夫何足責然非有國者之福也雖然吾黨亦有過焉盍亦反其本矣}
唐查文徽表求益兵唐主以天威都虞候何敬洙為建州行營招討馬步都指揮使將軍祖全恩為應援使姚鳳為都監|{
	監工銜翻}
將兵數千會攻建州自崇安進屯赤嶺|{
	九域志建州有崇安縣在州北二百五十里亦王氏所置也宋白曰崇安場本建陽縣東北三里南唐保大九年割為場盖宋方置縣也}
閩主延政遣僕射楊思恭統軍使陳望將兵萬人拒之列柵水南旬餘不戰唐人不敢逼思恭以延政之命督望戰望曰江淮兵精其將習武事國之安危繫此一舉不可不萬全而後動思恭怒曰唐兵深侵陛下寢不交睫|{
	睫即涉翻}
委之將軍今唐兵不出數千將軍擁衆萬餘不乘其未定而擊之有如唐兵懼而自退將軍何面目以見陛下乎|{
	楊思恭急于破敵以為功不知一跌而危國也}
望不得已引兵涉水與唐戰全恩等以大兵當其前使奇兵出其後大破之望死思恭僅以身免|{
	亡閩者楊思恭也然其所以亡閩者不在于此戰而在於得楊剥皮之名}
延政大愳嬰城自守召董思安王忠順使將泉州兵五千詣建州分守要害初高祖置德清軍於故澶州城|{
	澶州本治頓丘天福三年徙澶州于德勝并頓丘徙焉九域志澶州清豐縣有舊州鎮即置德清軍之地置德清軍將以接澶魏聲援然城池未固也}
及契丹入寇澶州鄴都之間城戍俱陷議者以為澶州鄴都相去百五十里宜於中塗築城以應接南北從之三月戊戌更築德清軍城合德清南樂之民以實之|{
	樂音洛}
初光州人李仁逹仕閩為元從指揮使|{
	王潮兄弟本光州人乘唐}


|{
	末之亂割據閩中其後兵多光州人今福州人多能自言其上世出于浮光者從才用翻}
十五年不遷職閩主曦之世叛奔建州閩主延政以為將|{
	是時王延政國號殷}
及朱文進弑曦|{
	事見去年三月}
復叛奔福州陳取建州之策文進惡其反覆黜居福清|{
	九域志福州有福清縣在州東南一百七十七里王氏所置也宋白曰福清木閩縣地唐聖歷元年析閩縣東南之地置萬安縣天寶元年改為福唐縣朱梁改永昌縣晉天福初改南臺縣尋改為福清縣}
浦城人陳繼珣|{
	新唐書地理志浦城縣本名吳興唐武德初改為唐興天寶元年更名浦城属建州九域志浦城縣在建州東北三百三十里宋白曰城臨柘蒲故曰浦城}
亦叛閩主延政奔福州為曦畫策取建州|{
	為于偽翻}
曦以為著作郎及延政得福州二人皆不自安王繼昌闇弱嗜酒不恤將士將士多怨仁達潜入福州說黄仁諷曰|{
	說式芮翻}
今唐兵乘勝建州孤危富沙王不能保建州安能保福州昔王潮兄弟光山布衣耳取福建如反掌|{
	事見唐紀}
况吾輩乘此機會自圖富貴何患不如彼乎仁諷然之是夕仁達等引甲士突入府舍殺繼昌及吳成義 |{
	考異曰閩中實録閩王列傳九國志皆云四月殺繼昌今從十國紀年}
仁達欲自立恐衆心未服以雪峯寺僧卓巖明素為衆所重|{
	雪峯在福州侯官縣西百餘里}
乃言此僧目重瞳子手垂過膝|{
	重直龍翻瞳音同過音戈}
真天子也相與迎之己亥立為帝 |{
	考異曰閩録啟運圖啟國實録江南録作巖明閩中實録閩王列傳九國志薛史唐餘録王審知傳吳越備史作儼明按啟運圖巖明本名偃為僧名體明即位改巖明今從之江南錄云繼昌為禆將王延諷所殺旬日故内臣李義殺諷立巖明為主今從十國紀年}
解去衲衣被以衮冕|{
	去羌呂翻衲奴荅翻被皮義翻}
帥將吏北面拜之|{
	帥讀曰率}
然猶稱天福十年遣使奉表稱藩于晉延政聞之族黄仁諷家命統軍使張漢真將水軍五千會漳泉兵討巖明乙巳杜威等諸軍會于定州以供奉官蕭處鈞權知祁州事庚戌諸軍攻契丹泰州刺史晉廷謙舉州降|{
	晉姓也以國為氏}
甲寅取滿城|{
	按五代會要是年九月徙泰州治滿城是時泰州猶治清苑宋白曰滿城本漢北平縣後魏置永樂縣天寶元年改滿城縣}
獲契丹酋長默喇|{
	酋慈秋翻長知兩翻喇來逹翻}
及其兵二千人乙卯取遂城趙延夀部曲有降者言契丹主還至虎北口|{
	太原汾水之北亦有地名虎北口時契丹兵自祁易北去非其路也此乃幽檀以北之古北口宋人使遼行程記云自檀州北行八十里又八十里至虎北口館則檀州之古北口亦名虎北口也}
聞晉取泰州復擁衆南向|{
	復扶又翻}
約八萬餘騎計來夕當至宜速為備杜威等懼丙辰退保泰州戊午契丹至泰州己未晉軍南行契丹踵之晉軍至陽城|{
	續漢志中山蒲陰縣有陽城水經注博水出中山望都縣東逕陽城縣散為澤渚世謂之陽城澱陽城在蒲陰縣東南三十里}
庚申契丹大至晉軍與戰逐北十餘里契丹踰白溝而去|{
	此南白溝也水經注所謂淇水北出為白溝者也北白溝在涿州新城縣南六十里宋人北使行程記曰雄州之北界河之南有白溝驛又范成大北使録曰自安肅軍出北門十五里至白溝河又一百五里至涿州此言北白溝也}
壬戌晉軍結陳而南|{
	陳讀曰陣}
胡騎四合如山諸軍力戰拒之是日纔行十餘里人馬饑乏癸亥晉軍至白團衛村 |{
	考異曰漢高祖實録作白檀今從晉少帝實録}
埋鹿角為行寨契丹圍之數重奇兵出寨後斷糧道|{
	重直龍翻斷音短}
是夕東北風大起破屋折樹|{
	折而設翻}
營中掘井方及水輒崩士卒取其泥帛絞而飲之人馬俱渴至曙風尤甚契丹主坐大奚車中|{
	沈括曰奚人業伐山陸種斵車契丹之車皆資于奚按輜車之制如中國後廣前殺而無般材儉易敗不能任重而利于行山長轂廣輪輪之牙其厚不能四寸而軫之材不能五寸其乘車駕之以駞上施㡛帷富者加氈幰文繡之飾蜀本奚車之上無大字}
令其衆曰晉軍止此耳當盡擒之然後南取大梁命鐵鷂四面下馬拔鹿角而入奮短兵以擊晉軍|{
	契丹謂精騎為鐵鷂謂其身被甲而馳突輕疾如鷂之搏鳥雀也鷂弋召召翻}
又順風縱火揚塵以助其勢軍士皆憤怒大呼|{
	呼火故翻}
曰都招討使何不用兵令士卒徒死諸將請出戰杜威曰俟風稍緩徐觀可否馬步都監李守貞曰彼衆我寡風沙之内莫測多少惟力鬬者勝此風乃助我也若俟風止吾属無類矣即呼曰諸軍齊擊賊又謂威曰令公善守禦|{
	杜威時帶中書令故稱之}
守貞以中軍決死矣馬軍左廂都排陳使張彦澤召諸將問計皆曰虜得風勢宜俟風回與戰彦澤亦以為然諸將退馬軍右廂副排陳使太原藥元福獨留謂彦澤曰今軍中饑渴已甚若俟風回吾属已為虜矣敵謂我不能逆風以戰宜出其不意急擊之此兵之詭道也|{
	矢不逆風此古法也若用短兵薄戰則逆風而勝者多矣}
馬步左右廂都排陳使符彦卿曰與其束手就擒曷若以身殉國乃與彦澤元福及左廂都排陳使皇甫遇引精騎出西門擊之|{
	行寨之西門也風從東北來出西門接戰亦順風勢也}
諸將繼至契丹却數百步彦卿等謂守貞曰且曳隊往來乎|{
	曳讀為拽音羊列翻}
直前奮擊以勝為度乎守貞曰事勢如此安可迴鞚|{
	鞚苦貢翻馬勒也}
宜長驅取勝耳彦卿等躍馬而去風勢益甚昏晦如夜彦卿等擁萬餘騎横擊契丹呼聲動天地|{
	呼火故翻}
契丹大敗而走勢如崩山李守貞亦令步兵盡拔鹿角出鬬步騎俱進逐北二十餘里鐵鷂既下馬蒼皇不能復上|{
	復扶又翻下同上時掌翻}
皆委弃馬及鎧仗蔽地契丹散卒至陽城東南水上稍復布列杜威曰賊已破膽不宜更令成列遣精騎擊之皆度水去契丹主乘奚車走十餘里追兵急獲一槖駞乘之而走諸將請急追之杜威揚言曰逢賊幸不死更索衣囊邪|{
	言逢賊被劫而幸不死而更從賊求衣囊則必將怒而殺而殺之索山客翻}
李守貞曰兩日人馬渴甚今得水飲之皆足重難以追寇不若全軍而還|{
	還從宣翻又如字}
乃退保定州契丹主至幽州散兵稍集以軍失利杖其酋長各數百唯趙延夀得免乙丑諸軍自定州引歸詔以泰州隸定州|{
	隸定州義武軍}
夏四月辛巳帝發澶州甲申還大梁|{
	是年正月下詔親征二月至澶州今諸軍以勝歸故復還大梁}
己丑復以鄴都為天雄軍|{
	唐莊宗同光元年以魏州為東京興唐府罷天雄節鎮三年罷東京以為鄴都晉興因之改興唐府為廣晉府今復為天雄軍}
閩張漢真至福州攻其東關黄仁諷聞家夷滅開門力戰大破閩兵執漢真入城斬之卓巖明無它方略但於殿上噀水散豆作諸法事而已|{
	噀蘇困翻含水而噴之為噀作諸佛事以為厭勝}
又遣使迎其父於莆田|{
	唐武德初分南安縣置莆田縣属泉州宋太平興國四年分置興化軍在泉州東北一百六十里}
尊為太上皇李仁達既立巖明自判六軍諸衛事使黄仁諷屯西門陳繼珣屯北門仁諷從容謂繼珣曰|{
	從行容翻}
人之所以為人者以有忠信仁義也吾頃嘗有功於富沙中間叛之非忠也人以從子託我而與人殺之非信也|{
	王繼昌閩主延政從子也從才用翻}
屬者與建兵戰|{
	属之欲翻屬者猶言頃者也}
所殺皆鄉曲故人非仁也弃妻子使人魚肉之非義也此身十沈九浮|{
	沈持林翻}
死有餘愧因拊膺慟哭繼珣曰大丈夫徇功名何顧妻子宜置此事勿以取禍仁達聞之使人告仁諷繼珣謀反皆殺之由是兵權盡歸仁達 五月丙申朔大赦 順國節度使杜威久鎮恒州|{
	高祖天福七年杜威始鎮恒州見二百八十三卷恒戶登翻}
性貪殘自恃貴戚|{
	杜威尚高祖妹宋國長公主}
多不法每以備邉為名斂吏民錢帛以充私藏|{
	藏徂浪翻}
富室有珍貨或名姝駿馬皆虐取之|{
	姝逡須翻}
或誣以罪殺之籍沒其家又畏懦過甚每契丹數十騎入境威已閉門登陴或數騎驅所掠華人千百過城下威但瞋目延頸望之無意邀取|{
	陴頻眉翻瞋昌真翻}
由是虜無所忌憚屬城多為所屠威竟不出一卒救之千里之間暴骨如莽|{
	暴骨如莽左傳語如莽者如草之生于廣野莽莽然暴步卜翻}
村落殆盡威見所部殘弊為衆所怨又畏契丹之強累表請入朝帝不許威不俟報遽委鎮入朝朝廷聞之驚駭桑維翰言於帝曰威固違朝命擅離邉鎮|{
	離力智翻}
居常憑恃勲舊邀求姑息及疆場多事|{
	場音亦}
曾無守禦之意宜因此時廢之庶無後患帝不悦維翰曰陛下不忍廢之宜授以近京小鎮勿復委以雄藩|{
	復扶又翻}
帝曰威朕之密親必無異志|{
	言其無它志}
但宋國長公主切欲相見耳|{
	長知兩翻}
公勿以為疑維翰自是不敢復言國事以足疾辭位|{
	杜威不可去而桑維翰求去晉殆矣復扶又翻}
丙辰威至大梁 丁巳李仁達大閲戰士請卓巖明臨視仁達陰教軍士突前登階刺殺巖明|{
	刺七亦翻}
仁達陽驚狼狽而走軍士共執仁達使居巖明之坐|{
	坐徂卧翻}
仁達乃自稱威武留後用保大年號|{
	是年南唐保大三年}
奉表稱藩于唐亦遣使入貢于晉并殺巖明之父唐以仁達為威武節度使同平章事賜名弘義編之屬籍|{
	以其同姓也編之屬籍而賜名弘義齒於諸子之列}
弘義又遣使修好於吳越|{
	為李仁達背唐而附吳越張本好呼到翻}
己未杜威獻部曲步騎合四千人并鎧仗庚申又獻粟十萬斛芻二十萬束云皆在本道|{
	言皆在恒州也使誠有之皆虐取於民倉皇離鎮不可運而實私家故獻之耳}
帝以其所獻騎兵隸扈聖步兵隸護國威復請以為衙隊而稟賜皆仰縣官|{
	杜威之愚弄朝廷如此而帝不能察其姦所以成恒州中渡之變復扶又翻稟筆錦翻給也仰牛向翻}
威又令公主白帝求天雄節鉞帝許之 唐兵圍建州屢破泉州兵|{
	泉州兵董思安王忠順所將以救建州者也}
許文稹敗唐兵于汀州|{
	稹止忍翻敗補賣翻}
執其將時厚卿 六月癸酉以杜威為天雄節度使 契丹連歲入寇|{
	契丹入寇自去年正月陷貝州始}
中國疲於奔命|{
	左傳申公巫臣遺子重子反書曰吾必使爾疲于奔命而死奔命者邉境有急奔而赴救}
邉民塗地契丹人畜亦多死國人厭苦之舒嚕太后謂契丹主曰使漢人為胡主可乎曰不可太后曰然則汝何故欲為漢主曰石氏負恩不可容太后曰汝今雖得漢地不能居也|{
	後卒如舒嚕后之言}
萬一蹉跌|{
	蹉七何翻跌徒結翻}
悔何所及又謂其羣下曰漢兒何得一向眠|{
	人寢不安席則輾轉反側而不成寐一向眠則其眠安矣}
自古但聞漢和蕃未聞蕃和漢漢兒果能回意我亦何惜與和桑維翰屢勸帝復請和於契丹以紓國患|{
	復扶又翻紓音舒緩也}
帝假開封軍將張暉供奉官|{
	開封軍將開封府之軍將也}
使奉表稱臣詣契丹卑辭謝過契丹主曰使景延廣桑維翰自來仍割鎮定兩道隸我則可和朝廷以契丹語忿謂其無和意乃止及契丹主入大梁謂李崧等曰曏使晉使再來則南北不戰矣|{
	史言契丹通國上下本自厭兵}
秋七月閩人或訛言赴援兵謀叛|{
	是年正月閩主發福州兵赴建州以拒唐}
閩主延政收其鎧仗遣還伏兵於隘|{
	還從宣翻隘烏戒翻險狹之道也}
盡殺之死者八千餘人脯其肉以歸為食唐邉鎬拔鐔州|{
	鐔州東至建州一百八十里}
查文徽之黨魏岑馮延己延魯以師出有功皆踴躍贊成之徵求供億府庫為之耗竭|{
	為于偽翻}
洪饒撫信之民尤苦之延政遣使奉表稱臣於吳越請為附庸以求救 楚王希範疑静江節度使兼侍中知朗州希杲得人心遣人伺之希杲懼稱疾求歸不許遣醫往視疾因毒殺之|{
	希範忌希杲事始二百八十卷高祖天福元年}


資治通鑑卷二百八十四














































































































































