<!DOCTYPE html PUBLIC "-//W3C//DTD XHTML 1.0 Transitional//EN" "http://www.w3.org/TR/xhtml1/DTD/xhtml1-transitional.dtd">
<html xmlns="http://www.w3.org/1999/xhtml">
<head>
<meta http-equiv="Content-Type" content="text/html; charset=utf-8" />
<meta http-equiv="X-UA-Compatible" content="IE=Edge,chrome=1">
<title>資治通鑒_208-資治通鑑卷二百七_208-資治通鑑卷二百七</title>
<meta name="Keywords" content="資治通鑒_208-資治通鑑卷二百七_208-資治通鑑卷二百七">
<meta name="Description" content="資治通鑒_208-資治通鑑卷二百七_208-資治通鑑卷二百七">
<meta http-equiv="Cache-Control" content="no-transform" />
<meta http-equiv="Cache-Control" content="no-siteapp" />
<link href="/img/style.css" rel="stylesheet" type="text/css" />
<script src="/img/m.js?2020"></script> 
</head>
<body>
 <div class="ClassNavi">
<a  href="/24shi/">二十四史</a> | <a href="/SiKuQuanShu/">四库全书</a> | <a href="http://www.guoxuedashi.com/gjtsjc/"><font  color="#FF0000">古今图书集成</font></a> | <a href="/renwu/">历史人物</a> | <a href="/ShuoWenJieZi/"><font  color="#FF0000">说文解字</a></font> | <a href="/chengyu/">成语词典</a> | <a  target="_blank"  href="http://www.guoxuedashi.com/jgwhj/"><font  color="#FF0000">甲骨文合集</font></a> | <a href="/yzjwjc/"><font  color="#FF0000">殷周金文集成</font></a> | <a href="/xiangxingzi/"><font color="#0000FF">象形字典</font></a> | <a href="/13jing/"><font  color="#FF0000">十三经索引</font></a> | <a href="/zixing/"><font  color="#FF0000">字体转换器</font></a> | <a href="/zidian/xz/"><font color="#0000FF">篆书识别</font></a> | <a href="/jinfanyi/">近义反义词</a> | <a href="/duilian/">对联大全</a> | <a href="/jiapu/"><font  color="#0000FF">家谱族谱查询</font></a> | <a href="http://www.guoxuemi.com/hafo/" target="_blank" ><font color="#FF0000">哈佛古籍</font></a> 
</div>

 <!-- 头部导航开始 -->
<div class="w1180 head clearfix">
  <div class="head_logo l"><a title="国学大师官网" href="http://www.guoxuedashi.com" target="_blank"></a></div>
  <div class="head_sr l">
  <div id="head1">
  
  <a href="http://www.guoxuedashi.com/zidian/bujian/" target="_blank" ><img src="http://www.guoxuedashi.com/img/top1.gif" width="88" height="60" border="0" title="部件查字,支持20万汉字"></a>


<a href="http://www.guoxuedashi.com/help/yingpan.php" target="_blank"><img src="http://www.guoxuedashi.com/img/top230.gif" width="600" height="62" border="0" ></a>


  </div>
  <div id="head3"><a href="javascript:" onClick="javascript:window.external.AddFavorite(window.location.href,document.title);">添加收藏</a>
  <br><a href="/help/setie.php">搜索引擎</a>
  <br><a href="/help/zanzhu.php">赞助本站</a></div>
  <div id="head2">
 <a href="http://www.guoxuemi.com/" target="_blank"><img src="http://www.guoxuedashi.com/img/guoxuemi.gif" width="95" height="62" border="0" style="margin-left:2px;" title="国学迷"></a>
  

  </div>
</div>
  <div class="clear"></div>
  <div class="head_nav">
  <p><a href="/">首页</a> | <a href="/ShuKu/">国学书库</a> | <a href="/guji/">影印古籍</a> | <a href="/shici/">诗词宝典</a> | <a   href="/SiKuQuanShu/gxjx.php">精选</a> <b>|</b> <a href="/zidian/">汉语字典</a> | <a href="/hydcd/">汉语词典</a> | <a href="http://www.guoxuedashi.com/zidian/bujian/"><font  color="#CC0066">部件查字</font></a> | <a href="http://www.sfds.cn/"><font  color="#CC0066">书法大师</font></a> | <a href="/jgwhj/">甲骨文</a> <b>|</b> <a href="/b/4/"><font  color="#CC0066">解密</font></a> | <a href="/renwu/">历史人物</a> | <a href="/diangu/">历史典故</a> | <a href="/xingshi/">姓氏</a> | <a href="/minzu/">民族</a> <b>|</b> <a href="/mz/"><font  color="#CC0066">世界名著</font></a> | <a href="/download/">软件下载</a>
</p>
<p><a href="/b/"><font  color="#CC0066">历史</font></a> | <a href="http://skqs.guoxuedashi.com/" target="_blank">四库全书</a> |  <a href="http://www.guoxuedashi.com/search/" target="_blank"><font  color="#CC0066">全文检索</font></a> | <a href="http://www.guoxuedashi.com/shumu/">古籍书目</a> | <a   href="/24shi/">正史</a> <b>|</b> <a href="/chengyu/">成语词典</a> | <a href="/kangxi/" title="康熙字典">康熙字典</a> | <a href="/ShuoWenJieZi/">说文解字</a> | <a href="/zixing/yanbian/">字形演变</a> | <a href="/yzjwjc/">金 文</a> <b>|</b>  <a href="/shijian/nian-hao/">年号</a> | <a href="/diming/">历史地名</a> | <a href="/shijian/">历史事件</a> | <a href="/guanzhi/">官职</a> | <a href="/lishi/">知识</a> <b>|</b> <a href="/zhongyi/">中医中药</a> | <a href="http://www.guoxuedashi.com/forum/">留言反馈</a>
</p>
  </div>
</div>
<!-- 头部导航END --> 
<!-- 内容区开始 --> 
<div class="w1180 clearfix">
  <div class="info l">
   
<div class="clearfix" style="background:#f5faff;">
<script src='http://www.guoxuedashi.com/img/headersou.js'></script>

</div>
  <div class="info_tree"><a href="http://www.guoxuedashi.com">首页</a> > <a href="/SiKuQuanShu/fanti/">四库全书</a>
 > <h1>资治通鉴</h1> <!--         下载:【右键另存为】即可 --></div>
  <div class="info_content zj clearfix">
  
<div class="info_txt clearfix" id="show">
<center style="font-size:24px;">208-資治通鑑卷二百七</center>
    資治通鑑卷二百七   宋 司馬光 撰<br />
<br />
  胡三省 音注<br />
<br />
  唐紀二十三【起上章困敦七月盡㫋蒙大荒落正月凡四年有奇】<br />
<br />
  則天順聖皇后下<br />
<br />
  久視元年秋七月獻俘於含樞殿【季楷固獻契丹之俘也含樞殿盖在三陽宫】太后以楷固為左玉鈐衛大將軍燕國公【鈐其廉翻燕因肩翻】賜姓武氏召公卿合宴【召公卿謂將帥合宴也】舉觴屬仁傑曰【屬之欲翻】公之功也將賞之對曰此乃陛下威靈將帥盡力【將帥上即亮翻下所類翻】臣何功之有固辭不受 閏月戊寅車駕還宫【自三陽宫還洛陽宫】 己丑以天官侍郎張錫為鳳閣侍郎同平章事鸞臺侍郎同平章事李嶠罷為成均祭酒錫嶠之舅也故罷嶠政事 丁酉吐蕃將麴莽布支寇凉州圍昌松【吐從瞰入聲將即亮翻昌松縣即漢武威郡蒼松縣呂光改為昌松】隴右諸軍大使唐休璟與戰於洪源谷【使疏吏翻璟居永翻】麴莽布支兵甲鮮華休璟謂諸將曰諸論既死【諸論死見上卷聖歷二年】麴莽布支新為將不習軍事望之雖如精鋭實易與耳請為諸君破之乃被甲先陷陳【易以䜴翻為于偽翻被皮義翻陳讀曰陣】六戰皆捷吐蕃大奔斬首二千五百級獲二禆將而還【還音旋又如音】司府少卿楊元亨【光宅元年改大府寺為司府寺】尚食奉御楊元禧皆弘武之子也【楊弘武見二百一卷高宗乾封二年】元禧嘗忤張易之【忤五故翻】易之言於太后元禧楊素之族素父子隋之逆臣子孫不應供奉太后從之壬寅制楊素及其兄弟子孫皆不得任京官左遷元亨睦州刺史元禧貝州刺史【馬何羅為逆於漢武之時而馬援貴顕於東都再造之日沈充失身於王敦而沈勁盡節於司馬惡惡止其身追罪異代之臣而併弃其子孫此盖出於一時之愛憎姑以是說而藉口耳睦州京師東南三千六百五十九里至東都二千八百二十一里貝州京師東北一千七百八十二里至東都九百九十三里】庚戌以魏元忠為隴右諸軍大使撃吐蕃 庚申太后欲造大像使天下僧尼日出一錢以助其功【尼女夷翻】狄仁傑上疏諫其略曰今之伽藍【上時掌翻疏所去翻伽藍佛寺也梵語云僧伽藍摩僧伽藍摩猶中華言衆園也伽求加翻】制過宫闕功不使鬼止在役人物不天來終須地出不損百姓將何以求又曰游僧皆託佛法詿誤生人【詿戶卦翻】里陌動有經坊闤闠亦立精舍【崔豹古今注闤市垣闠市門闤戶關翻闠戶對翻】化誘所急切於官徵【誘音酉】法事所須嚴於制敕又曰梁武簡文捨施無限【施式䜴翻】及三淮沸浪五嶺騰烟【用太宗詔中語】列刹盈衢無救危亡之禍【刹初鎋翻】緇衣蔽路豈有勤王之師又曰雖歛僧錢百未支一尊容既廣不可露居覆以百層【覆敷又翻】尚憂未遍自餘廊宇不得全無如來設敎以慈悲為主【釋氏謂佛為如來】豈欲勞人以存虚飾又曰比來水旱不節【比毗至翻】當今邊境未寧若費官財又盡人力一隅有難將何以救之【難乃旦翻】太后曰公敎朕為善何得相違遂罷其役 阿悉吉薄露叛【阿悉吉即西突厥弩失畢五俟斤之阿悉結也薄露其名】遣左金吾將軍田揚名殿中侍御史封思業討之軍至碎葉薄露夜於城傍剽掠而去思業將騎追之反為所敗【剽匹妙翻將即亮翻騎奇寄翻敗補邁翻】揚名引西突厥斛瑟羅之衆攻其城旬餘不克九月薄露詐降思業誘而斬之【降戶江翻誘音酉】遂俘其衆 太后信重内史梁文惠公狄仁傑羣臣莫及常謂之國老而不名仁傑好面引廷爭【好呼到翻爭讀曰諍】太后每屈意從之嘗從太后遊幸遇風吹仁傑巾墜而馬驚不能止太后命太子追執其鞚而繫之【鞚苦貢翻】仁傑屢以老疾乞骸骨太后不許入見常止其拜【見賢遍翻】曰每見公拜朕亦身痛仍免其宿直戒其同僚曰自非軍國大事勿以煩公辛丑薨太后泣曰朝堂空矣自是朝廷有大事衆或不能决太后輒嘆曰天奪吾國老何太早邪太后嘗問仁傑朕欲得一佳士用之誰可者仁傑曰未審陛下欲何所用之太后曰欲用為將相【將即亮翻相悉亮翻】仁傑對曰文學緼藉【緼於問翻藉慈夜翻】則蘇味道李嶠固其選矣必欲取卓犖奇才【犖呂角翻】則有荆州長史張柬之其人雖老宰相才也太后擢柬之為洛州司馬【自大州長史進神州司馬故曰擢】數日又問仁傑對曰前薦柬之尚未用也太后曰已遷矣對曰臣所薦者可為宰相非司馬也乃遷秋官侍郎久之卒用為相【卒子恤翻】仁傑又嘗薦夏官侍郎姚元崇監察御史曲阿桓彦範太州刺史敬暉等數十人【監古衘翻武德三年以并州之太谷祈縣置太州六年州廢當是此時復置也 考異曰梁公傳云張柬之桓彦範敬暉崔玄暐袁恕己皆公所薦公嘗退食之後謂五公曰所恨衰老身先朝露不得見五公盛事冀各保愛願盡本心五公心知目擊懸悟公意公寢疾五公候問偶對終日竟無一言少頃流涕及枕但相視而已五公退出遽不測其由袁恕己曰豈不氣力轉羸須問家事乎張柬之曰未聞大賢廢國謀家者也斯須命張柬之袁恕己桓彦範三公入餘二公立於門外曰向者無言盖以二公之故此二公能斷而不能密若先與議之事必外泄一泄之後則國異而家亡也至是時或不與共之事亦不就梁王三思掌權可先取而後行也不然則必反生大禍狄公沒後經歲餘五公潜會於幽聞之處叙公當時之言重結盟約徹饌之後相顧欲言未至其時恐負前諾欲言又止前後數四桓彦範乃叙其言言猶未畢聞戶牖之外聲若雷霆須臾風雨咫尺莫辨所坐牀褥悉擲於階下五公戰懼不知所據乃相謂曰此是狄公忠烈之至假此靈變以警衆心不欲吾輩先論此事未至其時不可復言也斯須天清日明不異於初易之等既誅袁謂張公曰昔有遺言使先收三思豈可舍諸張公曰但大事畢功此是机上之物豈有逃乎後梁王交通於内五公果為所譛俱遭流竄所期興廢年月遺約軌模少無異也按東之等五人偶同時在位恊力立功仁傑豈能預知其事舉此五人專欲使之輔立太子耶且易之等若有可誅之便太子有可立之勢仁傑身為宰相豈待五年之後須柬之等然後發邪此盖作傳者因五人建興復之功附會其事云皆仁傑所舉受教於仁傑耳其言譎怪無稽今所不取舊傳惟著舉柬之彦範暉三人姓名今從之】率為名臣或謂仁傑曰天下桃李悉在公門矣【程大昌演繁露趙簡子謂陽虎曰惟賢者為能報恩不肖者不能矣夫植桃李者夏得休息秋得其食植蒺藜者夏不得休息秋得其刺焉今子之所得者蒺藜也今世通以所薦士為桃李者說皆此本】仁傑曰薦賢為國非為私也【為于偽翻下為之同】初仁傑為魏州刺史【見二百五卷萬歲通天元年】有惠政百姓為之立生祠後其子景暉為魏州司功參軍貪暴為人患人遂毁其像焉【史言狄仁傑盡忠所以勸天下之為人臣言其以景暉貪暴而毁祠所以戒天下之為人子】 冬十月辛亥以魏元忠為蕭關道大總管以備突厥【蕭關在原州平高縣界貞觀六年以突厥降戶置緣州治平高之他樓城高宗置他樓縣神龍元年省更置蕭關縣厥九勿翻】甲寅制復以正月為十一月一月為正月【以十一月為正月事見二百四卷天授元年以一月為正月用夏正建寅也復扶又翻】赦天下 丁巳納言韋巨源罷以文昌右丞韋安石為鸞臺侍郎同平章事【納言侍中文昌左丞尚書右丞鸞臺門下】安石津之孫也【韋津死隋事見一百八十五卷高祖武德元年】時武三思張易之兄弟用事安石數面折之【數所角翻折之舌翻】嘗侍宴禁中易之引蜀商宋覇子等數人在座同博安石跪奏曰商賈賤類不應得預此會顧左右逐出之座中皆失色太后以其言直勞勉之【賈音古勞力到翻】同列皆歡服 【考異曰舊傳曰時鳳閣侍郎陸元方在座退而告人曰此乃真宰相非吾屬所及也按新紀元方已罷相今不取】 丁卯太后幸新安壬申還宫【還從宣翻又音如字】十二月甲寅突厥掠隴右諸監馬萬餘匹而去【厥九勿翻】時屠禁尚未解【禁屠見二百五卷長夀元年】鳳閣舍人全節崔融上言【鳳閣中書全節縣屬齊州漢晉之東平陵縣地後魏曰平陵屬濟南郡貞觀十七年齊王祐反平陵人不從更名全節上時掌翻】以為割烹犧牲弋獵禽獸聖人著之典禮不可廢闕又江南食魚河西食肉一日不可無富者未革貧者難堪况貧賤之人仰屠為生日戮一人終不能絶但資恐喝【喝呼葛翻】徒長姦欺【長知兩翻】為政者苟順月令合禮經自然物遂其生人得其性矣戊午復開屠禁【復扶又翻又音如字】祠祭用牲牢如故<br />
<br />
  長安元年【是年十月始改元長安】春正月丁丑以成州言佛迹見【見賢遍翻】改元大足【自此以後是大足元年考異曰朝野僉載云司刑寺囚三百餘人秋分後無計可作乃於圓獄外羅墻角邊作聖人迹五尺至夜半三百人一時大叫内使推問云昨夜有一聖人見身長三丈面作金色云汝等並枉寃不須怕懼天子萬年即有恩赦放汝把火照之見有偽跡即大赦天下改為大足元年識者相謂曰武家理天下足也按改元在春不在秋又無赦今不取】 二月己酉以鸞臺侍郎柏人李懷遠同平章事【鸞臺門下柏人縣自漢以來屬鉅鹿郡鉅鹿唐邢州天寶改曰堯山縣】 三月鳳閣侍郎同平章事張錫坐知選漏泄禁中語贓滿數萬當斬臨刑釋之流循州【舊志循州至東都四千八百里選須絹翻】時蘇味道亦坐事與錫俱下司刑獄【下遐嫁翻】錫乘馬意氣自若舍於三品院【先是制獄既繁司刑寺别置三品院以處三品以上官之下獄者】帷屏食飲無異平居味道步至繫所席地而臥蔬食而已太后聞之赦味道復其位 是月大雪蘇味道以為瑞帥百官入賀【帥讀曰率】殿中侍御史王求禮止之曰三月雪為瑞雪臘月雷為瑞雷乎味道不從既入求禮獨不賀進言曰今陽和布氣草木發榮而寒雪為災豈得誣以為瑞賀者皆諂諛之士也太后為之罷朝【為于偽翻下同 考異曰統紀在延載元年僉載在久視二年統紀云左拾遺僉載云侍御史御史臺記云殿中侍御史統紀云味道無以對舊傳云求禮止之味道不從今年從僉載官從臺記事則參取諸書】時又有獻三足牛者宰相復賀【復扶又翻】求禮颺言曰【孔安國曰大言而疾曰颺颺于章翻】凡物反常皆為妖【妖於喬翻】此鼎足非其人【三公鼎足承君】政敎不行之象也太后為之愀然【愀七小翻】 夏五月乙亥太后幸三陽宫 以魏元忠為靈武道行軍大總管以備突厥 天官侍郎鹽官顧琮同平章事【鹽官縣漢屬吴郡吴屬嘉興置海昌都尉梁陳屬錢塘郡隋屬餘杭郡唐屬杭州】 六月庚申以夏官尚書李迥秀同平章事迥秀性至孝其母本微賤妻崔氏常叱媵婢毋聞之不悦迥秀即時出之【迥戶頃翻媵以證翻】或曰賢室雖不避嫌疑然過非七出【律妻犯七出者弃之一無子二淫佚三不事舅姑四口舌五竊盗六妬忌七惡疾】何遽如是迥秀曰娶妻本以養親今乃違忤顔色【養余亮翻忤五故翻】安敢留也竟出之 秋七月甲戌太后還宫 甲申李懷遠罷為秋官尚書 八月突厥默啜寇邊命安北大都護相王為天兵道元師【相悉亮翻帥所類翻】統諸軍撃之未行而虜 丙寅武邑人蘇安恒上疏曰陛下欽先聖之顧托受嗣子之推讓【先聖謂大帝嗣子謂皇嗣相王恒戶登翻上時掌翻疏所去翻下同推吐雷翻】敬天順人二十年矣豈不聞帝舜搴裳周公復辟舜之於禹事祗族親旦與成王不離叔父【史記舜黄帝之八代孫禹黄帝之玄孫故云族親周公武王之弟成王之叔父旦其名也離力智翻】族親何如子之愛叔父何如母之恩今太子孝敬是崇春秋既壯若使統臨宸極何異陛下之身陛下年德既尊寶位將倦機務煩重浩蕩心神何不禪位東宫自怡聖體自昔理天下者不見二姓而俱王也當今梁定河内建昌諸王【武三思封梁王攸暨封定王懿宗封河内王攸寧封建昌王】承陛下之䕃覆【覆敷又翻】並得封王臣謂千秋萬歲之後於事非便臣請黜為公侯任以閒簡臣又聞陛下有二十餘孫今無尺寸之封此非長久之計也臣請分土而王之擇立師傅教其孝敬之道以夾輔周室屏藩皇家斯為美矣【屏卑郢翻】疏奏太后召見【見賢遍翻】賜食慰諭而遣之 太后春秋高政事多委張易之兄弟邵王重潤與其妹永泰郡主主婿魏王武延基竊議其事【重直龍翻】易之訴於太后九月壬申太后皆逼令自殺 【考異曰重潤傳云重潤為人所構與其妹永泰郡主壻武延基等竊議張易之兄弟何得恣入宫中則天令杖殺今從實録】延基承嗣之子也【承嗣太后之姪】丙申以相王知左右羽林衛大將軍事冬十月壬寅太后西入關辛酉至京師赦天下改元【改元長安】 十一月戊寅改含元宫為大明宫【長安東内本曰大明宫高宗龍朔三年曰蓬萊宫咸亨元年曰含元宫今復舊名】 天官侍郎安平崔玄暐【安平縣漢屬涿郡後漢屬安平國後魏屬博陵郡唐屬定州】性介直未嘗請謁執政惡之【惡烏路翻】改文昌左丞月餘太后謂玄暐曰自卿改官以來聞令史設齋自慶【唐吏部四司令史八十人】此欲盛為姦貪耳今還卿舊任乃復拜天官侍郎【復扶又翻又如字】仍賜綵七十段【唐制凡品十段其率絹三疋布三端綿四屯若雜綵十段則絲布二疋紬二疋綾二疋縵四疋】 以主客郎中郭元振為凉州都督隴右將軍大使【唐主客郎掌二王後及諸蕃朝聘之事屬禮部使疏吏翻】先是凉州南北境不過四百餘里【先悉薦翻】突厥吐蕃頻歲奄至城下百姓苦之元振始於南境硤口置和戎城北境磧中置白亭軍【杜佑曰白亭守捉在凉州城西北五百里磧七迹翻】控其衝要拓州境千五百里自是寇不復至城下【復扶又翻】元振又令甘州刺史李漢通開置屯田盡水陸之利舊凉州粟麥斛至數千及漢通收率之後【收率者收民而率其耕】一縑糴數十斛積軍粮支數十年元振善於撫御在凉州五年夷夏畏慕令行禁止牛羊被野【夏戶雅翻被皮義翻】路不拾遺二年春正月乙酉初設武舉【武舉之制有長垜馬射步射平射筒射馬槍翹】<br />
<br />
  【關負重身材之選唐六典曰武舉以七等閲其人一曰射長垜試射長垜三十發不出第三院為第入中院為上入次院為次上入外院為次二曰騎射發而並中為上或中或不中為次上總不中為次三曰馬槍三板四板為上二板為次上一板反不中為次四曰步射射草人中者為次上雖中而不法雖法而不中者為次五曰材貌以身長六尺已上者為次上已下為次六曰言語冇神采堪統領者為次上無者為次七曰舉重謂翹□率以五次上為第皆試其高第以名聞】突厥寇鹽夏二州三月庚寅突厥破石嶺【忻州定襄縣有石嶺關杜佑曰定襄縣本漢陽曲縣有石嶺關甚險固漢定襄郡在今馬邑郡地】寇并州以雍州長史薛季昶攝右臺大夫充山東防禦軍大使滄瀛幽易恒定等州諸軍皆受季昶節度【使疏吏翻恒戶登翻】夏四月以幽州刺史張仁愿專知幽平媯檀防禦【媯居為翻】仍與季昶相知以拒突厥 五月壬申蘇安恒復上疏曰【復扶又翻】臣聞天下者神堯文武之天下也【高祖神堯皇帝太宗文武皇帝】陛下雖居正統實因唐氏舊基當今太子追回【謂召廬陵王自房陵回復為太子】年德俱盛陛下貪其寶位而忘母子深恩將何聖顔以見唐家宗廟將何誥命以謁大帝墳陵【高宗稱天皇大帝】陛下何故日夜積憂不知鐘鳴漏盡【魏田豫告老曰譬猶鐘鳴漏盡而夜行不休此辠人也】臣愚以為天意人事還歸李家陛下雖安天位殊不知物極則反器滿則傾臣何惜一朝之命而不安萬乘之國哉【言不顧其死而上疏欲以安國也乘䋲證翻】太后亦不之辠 乙未以相王為并州牧充安北道行軍元帥【帥所翻翻】以魏元忠為之副 六月壬戌召神都留守韋巨源詣京師以副留守李嶠代之【守手又翻】 秋七月甲午突厥寇代州 司僕卿張昌宗【光宅元年改太僕寺為司僕寺】兄弟貴盛埶傾朝野【朝直遥翻】八月戊午太子相王太平公主上表請封昌宗為王制不許壬戌又請乃賜爵鄴國公敕自今有告言揚州及豫博餘黨【揚州事見二百三卷光宅元年豫博】<br />
<br />
  【事見二百四卷垂拱四年】一無所問内外官司無得為理【為于偽翻】 九月乙丑朔日有食之不盡如鉤神都見其既 壬申突厥寇忻州 己卯吐蕃遣其臣論彌薩來求和【薩桑葛翻】庚辰以太子賓客武三思為大谷道大總管洛州長史敬暉為副辛巳又以相王旦為并州道元帥三思與武攸宜魏元忠為之副姚元崇為長史司禮少卿鄭杲為司馬【欲以撃突厥】然竟不行 癸未宴論彌薩於麟德殿【麟德殿在大明宫右銀臺門内殿西重廊之後即翰林院是殿有三面亦曰三殿】時凉州都督唐休璟入朝亦預宴【璟居永翻朝直遥翻】彌薩屢窺之太后問其故對曰洪源之戰此將軍猛厲無敵故欲識之太后擢休璟為右武威金吾二衛大將軍【龍朔改左右威衛曰左右武威衛】休璟練習邊事自碣石以西踰四鎮綿亘萬里山川要害皆能記之【碣石在遼西四鎮在西域此言唐之西北二邊其山川要害休璟皆能記之也碣其謁翻亘古鄧翻】冬十月甲辰天官侍郎同平章事顧琮薨 戊申吐<br />
<br />
  蕃贊普將萬餘人寇茂州【將即亮翻】都督陳大慈與之四戰皆破之斬首千餘級 十一月辛未監察御史魏靖上疏以為陛下既知來俊臣之姦處以極法【監古衘翻上時掌翻疏所去翻處昌呂翻】乞詳覆俊臣等所推大獄伸其枉濫太后乃命監察御史蘇頲按覆俊臣等舊獄由是雪寃者甚衆【考異曰松窗雜録中宗嘗召宰相蘇瓌李嶠子進見二丞相子皆童年迎撫於赭前賜與甚厚因語二兒曰爾宜意所通書可為奏吾者言之頲應曰木從䋲則正后從諫則聖嶠子亡其名亦進曰斮朝涉之脛剖賢人之心上曰蘇瓌有子李嶠無兒按頲此年已為御史瓌為相時頲為中書舍人父子同掌樞密非童年也今不取】頲夔之曾孫也【頲它鼎翻蘇夔威之子隋開皇初議樂】 戊子太后祀南郊赦天下 十二月甲午以魏元忠為安東道安撫大使【使疏吏翻下同】羽林衛大將軍李多祚檢校幽州都督右羽林衛將軍薛訥左武衛將軍駱務整為之副 戊申置北庭都護府於庭州【太宗平高昌於西州之北置庭州即漢車師後王之地】侍御史張循憲為河東采訪使有疑事不能决病之問侍吏曰此有佳客可與議事者乎吏言前平鄉尉猗氏張嘉貞有異才【魏收志廣平郡任縣有平鄉城隋置平鄉縣治古鉅鹿城屬邢州猗氏縣古郇國自漢以來屬河東郡】循憲召見詢以事嘉貞為條析理分【隨條而析之隨理而分之為于偽翻】莫不洗然【洗與洒同蘇蟹翻洗然悚然也】循憲因請為奏皆意所未及循憲還見太后【見賢遍翻】太后善其奏循憲具言嘉貞所為且請以己之官授之太后曰朕寧無一官自進賢邪因召嘉貞入見内殿【見賢遍翻】與語大悦即拜監察御史擢循憲司勲郎中【唐司勲郎掌邦國宫人之勲級屬吏部監古衘翻】賞其得人也<br />
<br />
  三年春三月壬戌朔日有食之 夏四月吐蕃遣使獻馬千匹金二千兩以求昏【使疏吏翻下同】 閏月丁丑命韋安石留守神都 己卯改文昌臺為中臺【光宅元年改尚書省為文昌臺】以中臺左丞李嶠知納言事 新羅王金理洪卒【卒子恤翻】遣使立其弟崇基為王 六月辛酉突厥默啜遣其臣莫賀干來請以女妻皇太子之子【妻七細翻】 寧州大水溺殺二千餘人【溺奴狄翻】 秋七月癸卯以正諫大夫朱敬則同平章事 【考異曰新紀云壬寅唐歷云十四日癸卯今從之】 戊申以相王旦為雍州牧【相悉亮翻雍於用翻 考異曰唐歷十八日丁未今從實録】 庚戌以夏官尚書檢校凉州都督唐休璟同鳳閣鸞臺三品時突騎施酋長烏質勒與西突厥諸部相攻【騎奇寄翻酋慈由翻長知兩翻考異曰武平一景龍文舘記作烏折勒今從新舊書】安西道絶太后命休璟與<br />
<br />
  諸宰相議其事頃之奏上【上時掌翻】太后即依其議施行後十餘日安西諸州請兵應接程期一如休璟所畫太后謂休璟曰恨用卿晩謂諸宰相曰休璟練習邊事卿曹十不當一時西突厥可汗斛瑟羅用刑殘酷諸部不服烏質勒本隸斛瑟羅號莫賀逹干能撫其衆諸部歸之斛瑟羅不能制烏質勒置都督二十員各將兵七千人屯碎葉西北【將即亮翻】後攻陷碎葉徙其牙帳居之斛瑟羅部衆離散因入朝不敢復還【天授元年書斛瑟羅入居内地神功元年書來俊臣誣陷斛瑟羅則其入朝必不在是年此因書烏質勒事叙其得國之由遂及斛瑟羅失國事耳朝直遥翻】烏質勒悉併其地 九月庚寅朔日有食之既 初左臺大夫同鳳閣鸞臺三品魏元忠為洛州長史洛陽令張昌儀恃諸兄之埶每牙直上長史聽事【凡牙參者立於庭下上時掌翻聽讀曰廳】元忠到官叱下之【下遐嫁翻】張易之奴暴亂都市元忠杖殺之及為相太后召易之弟岐州刺史昌期欲以為雍州長史對仗問宰相曰誰堪雍州者元忠對曰今之朝臣無以易薛季昶【雍於用翻朝直遥翻】太后曰季昶久任京府朕欲别除一官昌期何如諸相皆曰陛下得人矣元忠獨曰昌期不堪太后問其故元忠曰昌期少年不閑吏事【少詩照翻】曏在岐州戶口逃亡且盡雍州帝京事任繁劇不若季昶彊幹習事太后默然而止元忠又嘗面奏臣自先帝以來蒙被恩渥今承乏宰相【元忠自言朝廷乏人已得承乏備位宰相被皮義翻】不能盡忠死節使小人在側臣之罪也【小人在側斥張易之兄弟】太后不悦由是諸張深怨之司禮丞高戩太平公主之所愛也【司禮丞即太常丞戩即淺翻】會太后不豫張昌宗恐太后一日晏駕為元忠所誅乃譛元忠與戩私議云太后老矣不若挾太子為久長【言為久長之計】太后怒下元忠戩獄【下遐嫁翻】將使與昌宗廷辨之昌宗密引鳳閣舍人張說賂以美官使證元忠說許之【說讀曰悦】明日太后召太子相王及諸宰相使元忠與昌宗參對往復不决昌宗曰張說聞元忠言請召問之太后召說說將入鳳閣舍人南和宋璟【南和縣漢屬廣平國宋白曰水經云北有和成縣故此縣云南後周置南和郡隋廢郡為縣唐屬邢州璟居永翻】謂說曰名義至重鬼神難欺不可黨邪陷正以求苟免若獲罪流竄其榮多矣若事有不測璟當叩閤力爭【言叩閤門而力爭也程大昌曰凡内殿便殿皆可謂之閤】與子同死努力為之萬代聸仰在此舉也殿中侍御史濟源張廷珪曰朝聞道夕死可矣【論語載孔子之言濟子禮翻】左史劉知幾曰無汚青史為子孫累【幾居希翻汚烏故翻累力瑞翻】及入太后問之說未對元忠懼謂說曰張說欲與昌宗共羅織魏元忠邪說叱之曰元忠為宰相何乃效委巷小人之言昌宗從旁廹趣說使速言【趣謂曰促】說曰陛下視之在陛下前猶逼臣如是况在外乎臣今對廣朝不敢不以實對【朝直遥翻】臣實不聞元忠有是言但昌宗逼臣使誣證之耳易之昌宗遽呼曰【呼火故翻】張說與魏元忠同反太后問其狀對曰說嘗謂元忠為伊周伊尹放太甲周公攝王位非欲反而何說曰易之兄弟小人徒聞伊周之語安知伊周之道日者元忠初衣紫【衣於既翻太宗貞觀四年詔三品以上服紫】臣以郎官往賀元忠語客曰無功受寵不勝慙愳【語牛倨翻勝音升】臣實言曰明公居伊周之任何愧三品彼伊尹周公皆為臣至忠古今慕仰陛下用宰相不使學伊周當使學誰邪且臣豈不知今日附昌宗立取台衡【三台為泰階北斗杓三星為玉衡宰輔得人則玉衡正而泰階平故謂宰輔為台衡】附元忠立教族滅但臣畏元忠寃魂不敢誣之耳太后曰張說反覆小人宜并繫治之【治直之翻】它日更引問說對如前太后怒命宰相與河内王武懿宗共鞫之說所執如初朱敬則抗疏理之曰元忠素稱忠正張說所坐無名若令抵罪失天下望蘇安恒亦上疏【恒戶登翻上時掌翻疏所去翻】以為陛下革命之初人以為納諫之主暮年以來人以為受佞之主自元忠下獄里巷恟恟【下遐嫁翻恟許勇翻】皆以為陛下委信姦宄斥逐賢良忠臣烈士皆撫髀於私室而箝口於公朝畏迕易之等意【箝其廉翻朝直遥翻迕五故翻】徒取死而無益方今賦役煩重百姓凋弊重以讒慝專恣刑賞失中【重以直用翻】竊恐人心不安别生它變爭鋒於朱雀門内問鼎於大明殿前【朱雀門謂宫城南門大明殿即含元殿】陛下將何以謝之何以禦之易之等見其疏大怒欲殺之賴朱敬則及鳳閣舍人桓彦範著作郎陸澤魏知古保救得免【先天元年方復置深州又分饒陽鹿城於古城置陸澤縣史因魏知古貴顕於開元之時遂以後來土斷書之苦么翻 考異曰舊傳云易之欲遣刺客殺之若遣刺客必不遣人知敬則等安能保護盖欲白太后殺之耳】丁酉貶魏元忠為高要尉【高要縣漢屬蒼梧郡宋齊屬南海郡陳置高要郡隋帶端州】戩說皆流嶺表元忠辭日言於太后曰臣老矣今向嶺南十死一生陛下它日必有思臣之時太后問其故時易之昌宗皆侍側元忠指之曰此二小兒終為亂階易之等下殿叩膺自擲稱寃太后曰元忠去矣殿中侍御史景城王睃【景城縣漢屬勃海郡後魏并入城平縣隋開皇十八年改曰景城屬滄州睃私潤翻又音俊】復奏申理元忠【復扶又翻下子復同】宋璟謂之曰魏公幸已得全今子復冒威怒得無狼狽乎睃曰魏公以忠獲罪睃為義所激顛沛無恨璟嘆曰璟不能申魏公之枉深負朝廷矣太子僕崔貞慎等八人餞元忠於郊外【唐制太子僕從四品下掌太子車輿乘騎儀仗之政令】易之詐為告密人柴明狀稱貞慎等與元忠謀反太后使監察御史丹徒馬懷素鞫之【丹徒春秋時吴之朱方也漢為丹徒縣屬會稽郡吴為京口戍晉以下為南徐州隋為延陵縣屬江都郡唐為丹徒縣帶潤州監古衘翻】謂懷素曰兹事皆實略問速以聞頃之中使督趣者數四【使疏吏翻趣讀曰促】曰反狀昭然何稽留如此懷素請柴明對質太后曰我自不知柴明處但據狀鞫之安用告者懷素據實以聞太后怒曰卿欲縱反者邪對曰臣不敢縱反者元忠以宰相謫官貞慎等以親故追送若誣以為反臣實不敢昔欒布奏事彭越頭下漢祖不以為辠【欒布事見十二卷漢高祖十一年】况元忠之刑未如彭越而陛下欲誅其送者乎且陛下操生殺之柄【操干高翻】欲加之罪取决聖衷可矣若命臣推鞫臣不敢不以實聞太后曰汝欲全不罪邪對曰臣智識愚淺實不見其罪太后意解貞慎等由是獲免太后嘗命朝貴宴集【朝直遥翻】易之兄弟皆位在宋璟上易之素憚璟欲悦其意虚位揖之曰公方今第一人何乃下坐璟曰才劣位卑張卿以為第一何也天官侍郎鄭杲謂璟曰中丞奈何卿五郎 【考異曰新舊傳皆作鄭善果按善果乃是高祖時人新舊傳皆誤當從御史臺記】璟曰以官言之正當為卿足下非張卿家奴何郎之有【門生家奴呼其主為郎今俗猶謂之郎主】舉坐悚惕【坐徂臥翻】時自武三思以下皆謹事易之兄弟璟獨不為之禮諸張積怒常欲中傷之【中竹仲翻】太后知之故得免 丁未以左武衛大將軍武攸宜充西京留守【守式又翻】冬十月丙寅車駕發西京乙酉至神都 十一月突厥遣使謝許昏【使疏吏翻】丙寅宴於宿羽臺【宿羽臺在東都宿羽宫中高宗調露元年所起】太子預焉宫尹崔神慶上疏【上時掌翻疏所去翻】以為今五品以上所以佩龜者為别敕徵召恐有詐妄内出龜合然後應命况太子國本古來徵召皆用玉契【唐制百官冇隨身魚符以明貴賤應召命左二右一左者進内右者隨身皇太子以玉契召勘合乃赴親王以金庶官以銅皆題其位姓名盛以魚袋天授二年改佩魚為龜張鷟朝野僉載曰唐以鯉魚為符遂為魚符至偽周武姓也玄武龜也因改魚符為龜符為于偽翻】此誠重慎之極也昨緣突厥使見太子應預朝參【使疏吏翻見賢遍翻朝直遥翻下同】直有文符下宫曾不降敕處分【下遐嫁翻處昌呂翻分扶問翻】臣愚謂太子非朔望朝參應别召者望降墨敕及玉契太后甚然之 始安獠歐陽倩【始安郡桂州范成大桂海虞衡志曰獠依山林而居無酋長版籍蠻之荒忽無常者也以射生食動為活蟲豸能蠕動者皆取食獠魯皓翻】擁衆數萬攻陷州縣朝廷思得良吏以鎮之朱敬則稱司封郎中裴懷古有文武才【唐司封郎掌國之封爵屬吏部】制以懷古為桂州都督仍充招慰討撃使【使疏吏翻】懷古纔及嶺上飛書示以禍福倩等迎降【降戶江翻】且言為吏所侵逼故舉兵自救耳懷古輕騎赴之【騎奇寄翻】左右曰夷獠無信不可忽也懷古曰吾仗忠信可通神明而况人乎遂詣其營賊衆大喜悉歸所掠貨財諸洞酋長素持兩端者皆來欵附【酋慈由翻長知兩翻】嶺外悉定是歲分命使者以六條察州縣【使疏吏翻】 吐蕃南境諸部皆叛贊普器弩悉弄自將撃之卒於軍中【將即亮翻卒子恤翻】諸子爭立久之國人立其子弃隸蹜贊為贊普生七年矣【史言諸論既死吐蕃國勢稍衰】<br />
<br />
  四年春正月丙申冊拜右武衛將軍阿史那懷道為西突厥十姓可汗懷道斛瑟羅之子也【厥九勿翻可從刋入聲汗音寒】丁未毁三陽宫以其材作興泰宫於萬安山【萬安山在洛州夀安縣西南四十里】二宫皆武三思建議為之請太后每歲臨幸功費甚廣百姓苦之左拾遺盧藏用上疏以為左右近臣多以順意為忠朝廷具僚皆以犯忤為戒【上時掌翻疏所去翻朝直遥翻忤五故翻】致陛下不知百姓失業傷陛下之仁陛下誠能以勞人為辭發制罷之則天下皆知陛下苦已而愛人也不從藏用承慶之弟孫也【盧承慶見二百卷顕慶二年】 壬子以天官侍郎韋嗣立為鳳閣侍郎同平章事【天官吏部嗣祥吏翻】夏官侍郎同鳳閣鸞臺三品李迥秀頗受賄賂監察御史馬懷素劾奏之【夏官兵部鳳閣鸞臺中書門下迥戶頃翻監古衘翻劾戶槩翻又戶得翻】二月癸亥迥秀貶廬州刺史【隋改梁周之合州為廬州唐因之舊志廬州京師東南二千三百八十七里至東都一千五百六十九里】 壬申正諫大夫同平章事朱敬則以老疾致仕敬則為相【相悉亮翻】以用人為先自餘細務不之視 太后嘗與宰相議及刺史縣令三月己丑李嶠唐休璟等奏竊見朝廷物議遠近人情莫不重内官輕外職每除授牧伯皆再三披訴比來所遣外任多是貶累之人【比毗至翻累力瑞翻罪累也】風俗不澄實由於此望於臺閣寺監妙簡賢良分典大州共康庶績臣等請輟近侍率先具僚太后命書名探之【探吐南翻】得韋嗣立及御史大夫楊再思等二十人癸巳制各以本官檢校刺史嗣立為汴州刺史【舊志汴州京師東一千三百五十里東都四百一里】其後政績可稱者唯常州刺史薛謙光徐州刺史司馬鍠而已【鍠戶萌翻又音皇常州京師東南二千八百四十三里至東都一千九百八十三里徐州京師東二千六百四里東都東一千二百五十七里】 丁丑徙平恩王重福為譙王【重直龍翻】 以夏官侍郎宗楚客同平章事 鳳閣侍郎同鳳閣鸞臺三品蘇味道謁歸葬其父制州縣供葬事【味道趙州欒城縣人】味道因之侵毁鄉人墓田役使過度監察御史蕭至忠劾奏之左遷坊州刺史【唐之先元皇帝周天和中為敷州刺史於中部縣置馬坊高祖武德二年因分鄜州之中部鄜城置坊州】至忠引之玄孫也【蕭引見一百七十卷陳宣帝太建二年監古衘翻劾戶槩翻又戶得翻】 夏四月壬戌同鳳閣鸞臺三品韋安石知納言李嶠知内史事 太后幸興泰宫 太后復税天下僧尼作大像於白司馬阪【復扶又翻洛城北邛山有白司馬阪】令春官尚書武攸寧檢校糜費巨億李嶠上疏以為天下編戶貧弱者衆造像錢見有一十七萬餘緡若將散施【見賢遍翻下見在同散如字施式豉翻】人與一千濟得一十七萬餘戶拯饑寒之弊省勞役之勤順諸佛慈悲之心霑聖君亭育之意人神胥悦功德無窮方作過後因緣豈如見在果報監察御史張廷珪上疏諫曰臣以時政論之則宜先邊境蓄府庫養人力以釋教論之則宜救苦厄滅諸相【先悉薦翻相息亮翻】崇無為伏願陛下察臣之愚行佛之意務以理為上不以人廢言太后為之罷役【為于偽翻】仍召見廷珪【見賢遍翻】深賞慰之 鳳閣侍郎同鳳閣鸞臺三品姚元崇以母老固請歸侍六月辛酉以元崇行相王府長史秩位並同三品 乙丑以天官侍郎崔玄暐同平章事召鳳閣侍郎同平章事檢校汴州刺史韋嗣立赴興<br />
<br />
  泰宫 丁丑以李嶠同鳳閣鸞臺二品嶠自請解内史壬午以相王府長史姚元崇兼知夏官尚書同鳳閣<br />
<br />
  鸞臺三品 秋七月丙戌以神都副留守楊再思為内史【守手又翻】再思為相專以諂媚取容司禮少卿張同休易之之兄也嘗召公卿宴集酒酣戲再思曰楊内史面似高麗再思欣然即翦紙帖巾反披紫袍為高麗舞【唐十部樂有高麗伎舞者四人楊思盖倣之為此舞】舉坐大笑【坐徂卧翻】時人或譽張昌宗之美【譽音余】曰六郎面似蓮花再思獨曰不然昌宗問其故再思曰乃蓮花似六郎耳 甲午太后還宫 乙未司禮少卿張同休汴州刺史張昌期尚方少監張昌儀皆坐贓下獄【下遐嫁翻】命左右臺共鞫之丙申敕張易之張昌宗作威作福亦命同鞫辛丑司刑正賈敬言奏張昌宗強市人田【光宅改大理正為司刑正從五品掌參議刑辟詳正科條之事】應徵銅二十斤制可乙巳御史大夫李承嘉中丞桓彦範奏張同休兄弟贓共四千餘緡張昌宗法應免官昌宗奏臣有功於國所犯不至免官太后問諸宰相昌宗有功乎楊再思曰昌宗合神丹【合音閣】聖躬服之有驗此莫大之功太后悦赦昌宗罪復其官左補闕戴令言作兩脚狐賦以譏再思【言再思妖媚如狐特兩脚耳】再思出令言為長社令丙午夏官侍郎同平章事宗楚客有罪左遷原州都督充靈武道行軍大總管 癸丑張同休貶岐山丞【後魏分扶風雍縣置平秦郡西魏改為岐山郡隋廢郡為縣屬岐州】張昌儀貶博望丞鸞臺侍郎知納言事同鳳閣鸞臺三品韋安石舉奏張易之等罪敕付安石及右庶子同鳳閣鸞臺三品唐休璟鞫之未竟而事變八月甲寅以安石兼檢校揚州刺史【考異曰唐歷云五日戊午今從實録】庚申以休璟兼幽營都督安東都護休璟將行密言於太子曰二張恃寵不臣必將為亂殿下宜備之 相王府長史兼知夏官尚書事同鳳閣鸞臺三品姚元崇上言臣事相王不宜典兵馬【夏官即兵部也故云然相息亮翻】臣不敢愛死恐不益於王辛酉改春官尚書餘如故元崇字元之時突厥叱列元崇反太后命元崇以字行突厥默啜既和親戊寅始遣淮陽王武延秀還【武延秀被拘見上卷聖歷元年】九月壬子以姚元之充靈武道行軍大總管辛酉以元之為靈武道安撫大使【使疏吏翻】元之將行太后令舉外司堪為宰相者【外司謂外朝諸司官】對曰張柬之沈厚有謀能斷大事【沈持林翻斷丁亂翻】且其人已老惟陛下急用之冬十月甲戌以秋官侍郎張柬之同平章事時年且八十矣乙亥以韋嗣立檢校魏州刺史餘如故 壬午以懷<br />
<br />
  州長史河南房融同平章事 太后命宰相各舉堪為員外郎者韋嗣立薦廣武令岑羲曰但恨其伯父長倩為累【長倩死見二百四卷天授二年累力瑞翻下同】太后曰苟或有才此何所累遂拜天官員外郎【唐六典曰周官太宰之屬官有上士盖今員外郎之任也宋百官階次有員外郎美遷為尚書郎隋文帝開皇六年尚書二十四曹各置員外郎一人品從第六謂曹郎本員之外復置郎也焬帝大業三年又廢二十四司員外郎每司減一員置承務郎一人同開皇員外郎之職曰選部承務郎唐尚書諸曹各置員外郎吏部置二人天官即吏部】由是諸緣坐者始得進用十一月丁亥以天官侍郎韋承慶為鳳閣侍郎同平章<br />
<br />
  事 癸卯成均祭酒同鳳閣鸞臺三品李嶠罷為地官尚書 十二月甲寅敕大足已來新置官並停 丙辰鳳閣侍郎同平章事韋嗣立罷為成均祭酒檢校魏州刺史如故以兄承慶入相故也【相息亮翻】 太后寢疾居長生院【長生院即長生殿明年五王誅二張進至太后所寢長生殿同此處也盖唐寑殿皆謂之長生殿此武后寑疾之長生殿洛陽宫寑殿也肅宗大漸越王係授甲長生殿長安大明宫之寑殿也白居易長恨歌所謂七月七日長生殿夜半無人私語時華清宫之長生殿也】宰相不得見者累月惟張易之昌宗侍側疾少間【間如字】崔玄暐奏言皇太子相王仁明孝友足恃湯藥【相息亮翻】宫禁事重伏願不令異姓出入太后曰德卿厚意易之昌宗見太后疾篤恐禍及已引用黨援隂為之備屢有人為飛書及牓其書於通衢云易之兄弟謀反太后皆不問辛未許州人楊元嗣告昌宗嘗召術士李弘泰占相弘泰言昌宗有天子相【相息亮翻】勸於定州造佛寺則天下歸心 【考異曰實録云長安四年秋元嗣告之太后令鳳閣侍郎韋承慶推鞫按十一月丁亥承慶始為鳳閣侍郎今從唐歷】太后命韋承慶及司刑卿崔神慶御史中丞宋璟鞫之神慶神基之弟也承慶神慶奏言昌宗欵稱弘泰之語尋已奏聞凖法首原【法自首者原其罪承慶神慶欲凖此條以脱昌宗之罪首式又翻】弘泰妖言請收行法【妖於喬翻下同】璟與大理丞封全禎奏昌宗寵榮如是復召術士占相【復扶又翻】志欲何求弘泰稱筮得純乾天子之卦昌宗儻以弘泰為妖妄何不執送有司雖云奏聞終是包藏禍心法當處斬破家【處昌呂翻】請收付獄窮理其罪太后久之不應璟又曰儻不即收繫恐其揺動衆心太后曰卿且停推【停其事且莫推究】俟更檢詳文狀璟退左拾遺江都李邕進曰【江都縣帶楊州】向觀宋璟所奏志安社稷非為身謀願陛下可其奏太后不聽尋敕璟揚州推按又敕璟按幽州都督屈突仲翔贓汚【屈九勿翻】又敕璟副李嶠安撫隴蜀璟皆不肯行奏曰故事州縣官有罪品高則侍御史卑則監察御史按之中丞非軍國大事不當出使【監古衘翻使疏吏翻】今隴蜀無變不識陛下遣臣出外何也臣皆不敢奉制 【考異曰御史臺記云易之昌宗冀璟使後當列狀誅璟按易之等若果可以列狀誅璟則何必待其出使然後為之此盖璟方奏請收禁昌宗故太后欲遣璟出以散其事耳璟必欲收禁故辭不肯行大后自省理屈故不迫遣耳不然璟若無事不行太后豈不能以拒違制命罪之邪又云時璟家禮會易之等伺其夕以刺之有密告暻者乘庫車於它所而免按若實有其迹璟安得不自陳於太后若無其迹則人妄言耳今不取】司刑少卿桓彦範上疏以為昌宗無功荷寵【荷下可翻】而包藏禍心自招其咎此乃皇天降怒陛下不忍加誅則違天不祥且昌宗既云奏訖則不當更與弘泰往還使之求福禳災是則初無悔心所以奏者擬事發則云先已奏陳不發則俟時為逆此乃奸臣詭計若云可捨誰為可刑况事已再發陛下皆釋不問使昌宗益自負得計天下亦以為天命不死此乃陛下養成其亂也苟逆臣不誅社稷亡矣請付鸞臺鳳閣三司考竟其罪【三司謂尚書刑部大理寺御史臺也唐制大獄則召大三司考竟又詔中書門下同鞫之】疏奏不報崔玄暐亦屢以為言太后令法司議其罪玄暐弟司刑少卿昪處以大辟【昪皮變翻處昌呂翻辟毗亦翻】宋璟復奏收昌宗下獄【復抉又翻下遐嫁翻】太后曰昌宗已自奏聞對曰昌宗為飛書所逼窮而自陳埶非得已且謀反大逆無容首免【首式又翻】若昌宗不伏大刑安用國法太后温言解之璟聲色逾厲曰昌宗分外承恩【分扶問翻】臣知言出禍從然義激於心雖死不恨楊再思恐其忤旨遽宣敕令出【忤五故翻】璟曰聖主在此不煩宰相擅宣敕命太后乃可其奏遣昌宗詣臺璟庭立而按之事未畢太后遣中使召昌宗特敕赦之【使疏吏翻】璟嘆曰不先擊小子腦裂負此恨矣太后乃使昌宗詣璟謝璟拒不見傳【考異曰御史臺記唐歷舊並云收按易之等按璟止鞫昌宗占相事耳無緣及易之今所不取舊張易之傳云宋璟請按易之則天陽許尋敕宋璟使幽州按都督屈突仲翔令司禮卿崔神慶希旨雪昌宗兄弟唐歷云桓彦範上疏不報璟登時出使按璟傳云特敕原易之仍令詣璟謝則是昌宗敕免時璟在都不出使也實録云令韋承慶崔神慶與璟推鞫當時璟執正其罪而神慶寛之耳非璟出使後神慶始鞫之也舊宋璟易之傳自相違今從御史臺記】左臺中丞桓彦範右臺中丞東光袁恕己共薦詹事司直陽嶠為御史【光宅分御史左右臺各置大夫中丞侍御史東光縣漢屬勃海郡唐屬滄州詹事司直正九品上掌彈劾宫僚糾舉職事】楊再思曰嶠不樂搏擊之任如何彦範曰為官擇人豈必待其所欲所不欲者尤須與之所以長難進之風抑躁求之路【樂音洛為于偽翻長知兩翻躁則到翻】乃擢為右臺侍御史嶠休之之玄孫也【陽休之仕高齊貴顕】先是李嶠崔玄暐奏往屬革命之時人多逆節遂致刻薄之吏恣行酷法其周興等所劾破家者並請雪免司刑少卿桓彦範又奏陳之表疏前後十上【先悉薦翻屬之欲翻劾戶槩翻又戶得翻上時掌翻】太后乃從之<br />
<br />
  中宗大和大聖大昭孝皇帝上<br />
<br />
  【諱顕高宗第七子也中更名哲已而復舊名景雲元年諡孝和皇帝廟號中宗天寶八年追尊大和大聖皇帝十三載追尊大和大聖大昭孝皇帝】<br />
<br />
  神龍元年春正月壬午朔赦天下改元 【考異曰新紀長安五年正月壬午大赦甲子太子監國改元按則天實録神龍元年正月壬午朔大赦改元舊紀唐歷統紀會要皆同紀年通譜亦以神龍為武后年號中宗因之新紀誤也】自文明以來得罪者非揚豫博三州及諸反逆魁首咸赦除之 太后疾甚麟臺監張易之春官侍郎張昌宗居中用事張東之崔玄暐與中臺右丞敬暉【光宅元年改尚書左右丞為文昌左右丞長安三年又改為中臺左右丞】司刑少卿桓彦範相王府司馬袁怒己謀誅之柬之謂右羽林衛大將軍李多祚曰將軍今日富貴誰所致也多祚泣曰大帝也柬之曰今大帝之子為二豎所危將軍不思報大帝之德乎多祚曰苟利國家惟相公處分【處昌呂翻分扶問翻】不敢顧身及妻子因指天地以自誓遂與定謀初柬之與荆府長史閺鄉楊元琰相代【荆州都督府長史故曰荆府閺鄉在漢弘農湖縣界隋分置縣屬洛州唐屬虢州二人相代當在久視元年閺音旻】同泛江至中流語及太后革命事元琰慨然有匡復之志及柬之為相引元琰為右羽林將軍謂曰君頗記江中之言乎今日非輕授也柬之又用彦範暉及右散騎侍郎李湛【魏晋置散騎常侍侍郎與侍中黄門共平尚書奏事其後用人或雜江左不重此官或省或置隋初省侍郎置常侍從三品掌陪從朝直煬帝又省之武德初以為加官貞觀初置常侍二人屬門下省為職事官顕慶二年又置二員屬中書省始冇左右之號並金蟬珥貂左常侍與侍中左貂右常侍與中書令右貂謂之八貂唐末嘗置散騎侍郎也據舊書湛時為右散騎常侍當從之散悉亶翻騎奇寄翻】皆為左右羽林將軍委以禁兵易之等疑懼乃更以其黨武攸宜為右羽林大將軍易之等乃安俄而姚元之自靈武至柬之彦範相謂曰事濟矣遂以其謀告之彦範以事白其母母曰忠孝不兩全先國後家可也【先悉薦翻後戶遘翻】時太子於北門起居【洛陽宫北門亦曰玄武門不從端門入而從北門入問起居取便近也】彦範暉謁見【見賢遍翻】密陳其策太子許之癸卯柬之玄暐彦範與左威衛將軍薛思行等帥左右羽林兵五百餘人至玄武門【帥讀曰率】遣多祚湛及内直郎駙馬都尉安陽王同皎【唐東宫内直局有内直郎二人從六品下掌符璽扇繖几案衣服之事安陽漢侯國屬魏郡其故城在湯隂曹魏時廢安陽併入鄴後周移鄴置縣於安陽故城仍為鄴縣隋又改為安陽縣為魏州治所漢魏郡城在縣西北七里】詣東宫迎太子太子疑不出同皎曰先帝以神器付殿下横遭幽廢【横戶孟翻】人神同憤二十三年矣【按光宅元年廢太子廬陵王至是三十二年】今天誘其衷【誘音酉】北門南牙同心協力以誅凶豎復李氏社稷【南牙謂宰相北門謂羽林諸將】願陛下蹔至玄武門以副衆望【蹔與暫同】太子曰凶豎誠當夷滅然上體不安得無驚怛【怛當割翻】諸公更為後圖李湛曰諸將相不顧家族以狥社稷殿下奈何欲納之鼎鑊乎請殿下自出止之【考異曰舊李湛傳曰湛與右羽林大將軍李多祚等詣東宫迎皇太子拒而不時出湛進啟曰逆竪反道亂】<br />
<br />
  【常將圖不軌宗社危敗實在須臾湛等諸將與南衙執事克期誅翦伏願陛下暫至玄武門以副衆望太子曰凶竪悖亂誠合誅夷然聖躬不豫慮有驚動公等且止以俟後圖湛曰諸將弃家族共宰相同心匡輔社稷陛下奈何欲陷之鼎鑊陛下速出自止遏太子乃上馬就路按劉子玄中宗實録唐歷統紀皆以此為王同皎之言而舊傳以為李湛進說今從實録唐歷等参取舊傳】太子乃出同皎扶抱太子上馬從至玄武門斬關而入【上時掌翻從才用翻】太后在迎仙宫東之等斬易之昌宗於廡下【廡音武】進至太后所寢長生殿環繞侍衛【環音宦】太后驚起問曰亂者誰邪對曰張易之昌宗謀反臣等奉太子令誅之恐有漏洩故不敢以聞稱兵宫禁罪當萬死太后見太子曰乃汝邪小子既誅可還東宫彦範進曰太子安得更歸昔天皇以愛子託陛下今年齒已長【長知兩翻】久居東宫天意人心久思李氏羣臣不忘太宗天皇之德故奉太子誅賊臣願陛下傳位太子以順天人之望李湛義府之子也【李義府朋附武后高宗以取相位】太后見之謂曰汝亦為誅易之將軍邪我於汝父子不薄乃有今日湛慙不能對又謂崔玄暐曰它人皆因人以進惟卿朕所自擢亦在此邪對曰此乃所以報陛下之大德於是收張昌期同休昌儀皆斬之與易之昌宗梟首天津南【梟堅堯翻】是日袁恕己從相王統南牙兵以備非常【相息亮翻】收韋承慶房融及司禮卿崔神慶繫獄皆易之之黨也初昌儀新作第甚美逾於王主【王主謂儲王及諸公主也】或夜書其門曰一日絲能作幾日絡【言其且誅滅能作樂得幾日也】滅去復書之【去羌呂翻復扶又翻】如是六七昌儀取筆注其下曰一日亦足乃止甲辰制太子監國【監古銜翻】赦天下以袁恕己為鳳閣侍郎同平章事分遣十使齎璽書宣慰諸州【十道各遣一使使疏吏翻璽斯氏翻】乙巳太后傳位於太子丙午中宗即位赦天下惟張易之黨不原其為周興等所枉者咸令清雪子女配沒者皆免之相王加號安國相王拜太尉同鳳閣鸞臺三品太平公主加號鎮國太平公主皇族先配沒者子孫皆復屬籍仍量叙官爵【量音良】 丁未太后徙居上陽宫李湛留宿衛戊申帝帥百官詣上陽宫上太后尊號曰則天大聖皇帝【帥讀曰率上時掌翻】庚戌以張柬之為夏官尚書同鳳閣鸞臺三品崔玄暐為内史袁恕己同鳳閣鸞臺三品敬暉桓彦範皆為納言並賜爵郡公李多祚賜爵遼陽郡王王同皎為右千牛將軍琅邪郡公李湛為右羽林大將軍趙國公自餘官賞有差 【考異曰中宗實録初冬官侍郎朱敬則以張易之等權寵日盛恐冇異圖時敬暉為右羽林將軍敬則謂之曰公若假皇太子之令舉北軍誅易之兄弟兩飛騎之力耳暉等竟用其策及易之昌宗伏誅暉遂矜功自恃故賞不及於敬則俄出為鄭州刺史按敬則長安四年以老罷知政事累轉冬官侍郎而則天實録誅易之時有庫部員外郎朱敬則恐誤】張柬之等之討張易之也殿中監田歸道將千騎宿玄武門【貞觀初太宗選善射者百人為二番於北門長上曰百騎武后改曰千騎將即亮翻騎奇寄翻】敬暉遣使就索千騎【使疏吏翻索山客翻】歸道先不預謀拒而不與事寧暉欲誅之歸道以理自陳乃免歸私第帝嘉其忠壯召拜太僕少卿<br />
<br />
  資治通鑑卷二百七  <br>
   </div> 

<script src="/search/ajaxskft.js"> </script>
 <div class="clear"></div>
<br>
<br>
 <!-- a.d-->

 <!--
<div class="info_share">
</div> 
-->
 <!--info_share--></div>   <!-- end info_content-->
  </div> <!-- end l-->

<div class="r">   <!--r-->



<div class="sidebar"  style="margin-bottom:2px;">

 
<div class="sidebar_title">工具类大全</div>
<div class="sidebar_info">
<strong><a href="http://www.guoxuedashi.com/lsditu/" target="_blank">历史地图</a></strong>  
<a href="http://www.880114.com/" target="_blank">英语宝典</a>  
<a href="http://www.guoxuedashi.com/13jing/" target="_blank">十三经检索</a> 
<br><strong><a href="http://www.guoxuedashi.com/gjtsjc/" target="_blank">古今图书集成</a></strong> 
<a href="http://www.guoxuedashi.com/duilian/" target="_blank">对联大全</a> <strong><a href="http://www.guoxuedashi.com/xiangxingzi/" target="_blank">象形文字典</a></strong> 

<br><a href="http://www.guoxuedashi.com/zixing/yanbian/">字形演变</a>  <strong><a href="http://www.guoxuemi.com/hafo/" target="_blank">哈佛燕京中文善本特藏</a></strong>
<br><strong><a href="http://www.guoxuedashi.com/csfz/" target="_blank">丛书&方志检索器</a></strong> <a href="http://www.guoxuedashi.com/yqjyy/" target="_blank">一切经音义</a>  

<br><strong><a href="http://www.guoxuedashi.com/jiapu/" target="_blank">家谱族谱查询</a></strong>  <strong><a href="http://shufa.guoxuedashi.com/sfzitie/" target="_blank">书法字帖欣赏</a></strong> 
<br>

</div>
</div>


<div class="sidebar" style="margin-bottom:0px;">

<font style="font-size:22px;line-height:32px">QQ交流群9:489193090</font>


<div class="sidebar_title">手机APP 扫描或点击</div>
<div class="sidebar_info">
<table>
<tr>
	<td width=160><a href="http://m.guoxuedashi.com/app/" target="_blank"><img src="/img/gxds-sj.png" width="140"  border="0" alt="国学大师手机版"></a></td>
	<td>
<a href="http://www.guoxuedashi.com/download/" target="_blank">app软件下载专区</a><br>
<a href="http://www.guoxuedashi.com/download/gxds.php" target="_blank">《国学大师》下载</a><br>
<a href="http://www.guoxuedashi.com/download/kxzd.php" target="_blank">《汉字宝典》下载</a><br>
<a href="http://www.guoxuedashi.com/download/scqbd.php" target="_blank">《诗词曲宝典》下载</a><br>
<a href="http://www.guoxuedashi.com/SiKuQuanShu/skqs.php" target="_blank">《四库全书》下载</a><br>
</td>
</tr>
</table>

</div>
</div>


<div class="sidebar2">
<center>


</center>
</div>

<div class="sidebar"  style="margin-bottom:2px;">
<div class="sidebar_title">网站使用教程</div>
<div class="sidebar_info">
<a href="http://www.guoxuedashi.com/help/gjsearch.php" target="_blank">如何在国学大师网下载古籍?</a><br>
<a href="http://www.guoxuedashi.com/zidian/bujian/bjjc.php" target="_blank">如何使用部件查字法快速查字?</a><br>
<a href="http://www.guoxuedashi.com/search/sjc.php" target="_blank">如何在指定的书籍中全文检索?</a><br>
<a href="http://www.guoxuedashi.com/search/skjc.php" target="_blank">如何找到一句话在《四库全书》哪一页?</a><br>
</div>
</div>


<div class="sidebar">
<div class="sidebar_title">热门书籍</div>
<div class="sidebar_info">
<a href="/so.php?sokey=%E8%B5%84%E6%B2%BB%E9%80%9A%E9%89%B4&kt=1">资治通鉴</a> <a href="/24shi/"><strong>二十四史</strong></a>&nbsp; <a href="/a2694/">野史</a>&nbsp; <a href="/SiKuQuanShu/"><strong>四库全书</strong></a>&nbsp;<a href="http://www.guoxuedashi.com/SiKuQuanShu/fanti/">繁体</a>
<br><a href="/so.php?sokey=%E7%BA%A2%E6%A5%BC%E6%A2%A6&kt=1">红楼梦</a> <a href="/a/1858x/">三国演义</a> <a href="/a/1038k/">水浒传</a> <a href="/a/1046t/">西游记</a> <a href="/a/1914o/">封神演义</a>
<br>
<a href="http://www.guoxuedashi.com/so.php?sokeygx=%E4%B8%87%E6%9C%89%E6%96%87%E5%BA%93&submit=&kt=1">万有文库</a> <a href="/a/780t/">古文观止</a> <a href="/a/1024l/">文心雕龙</a> <a href="/a/1704n/">全唐诗</a> <a href="/a/1705h/">全宋词</a>
<br><a href="http://www.guoxuedashi.com/so.php?sokeygx=%E7%99%BE%E8%A1%B2%E6%9C%AC%E4%BA%8C%E5%8D%81%E5%9B%9B%E5%8F%B2&submit=&kt=1"><strong>百衲本二十四史</strong></a>  <a href="http://www.guoxuedashi.com/so.php?sokeygx=%E5%8F%A4%E4%BB%8A%E5%9B%BE%E4%B9%A6%E9%9B%86%E6%88%90&submit=&kt=1"><strong>古今图书集成</strong></a>
<br>

<a href="http://www.guoxuedashi.com/so.php?sokeygx=%E4%B8%9B%E4%B9%A6%E9%9B%86%E6%88%90&submit=&kt=1">丛书集成</a> 
<a href="http://www.guoxuedashi.com/so.php?sokeygx=%E5%9B%9B%E9%83%A8%E4%B8%9B%E5%88%8A&submit=&kt=1"><strong>四部丛刊</strong></a>  
<a href="http://www.guoxuedashi.com/so.php?sokeygx=%E8%AF%B4%E6%96%87%E8%A7%A3%E5%AD%97&submit=&kt=1">說文解字</a> <a href="http://www.guoxuedashi.com/so.php?sokeygx=%E5%85%A8%E4%B8%8A%E5%8F%A4&submit=&kt=1">三国六朝文</a>
<br><a href="http://www.guoxuedashi.com/so.php?sokeytm=%E6%97%A5%E6%9C%AC%E5%86%85%E9%98%81%E6%96%87%E5%BA%93&submit=&kt=1"><strong>日本内阁文库</strong></a> <a href="http://www.guoxuedashi.com/so.php?sokeytm=%E5%9B%BD%E5%9B%BE%E6%96%B9%E5%BF%97%E5%90%88%E9%9B%86&ka=100&submit=">国图方志合集</a> <a href="http://www.guoxuedashi.com/so.php?sokeytm=%E5%90%84%E5%9C%B0%E6%96%B9%E5%BF%97&submit=&kt=1"><strong>各地方志</strong></a>

</div>
</div>


<div class="sidebar2">
<center>

</center>
</div>
<div class="sidebar greenbar">
<div class="sidebar_title green">四库全书</div>
<div class="sidebar_info">

《四库全书》是中国古代最大的丛书,编撰于乾隆年间,由纪昀等360多位高官、学者编撰,3800多人抄写,费时十三年编成。丛书分经、史、子、集四部,故名四库。共有3500多种书,7.9万卷,3.6万册,约8亿字,基本上囊括了古代所有图书,故称“全书”。<a href="http://www.guoxuedashi.com/SiKuQuanShu/">详细>>
</a>

</div> 
</div>

</div>  <!--end r-->

</div>
<!-- 内容区END --> 

<!-- 页脚开始 -->
<div class="shh">

</div>

<div class="w1180" style="margin-top:8px;">
<center><script src="http://www.guoxuedashi.com/img/plus.php?id=3"></script></center>
</div>
<div class="w1180 foot">
<a href="/b/thanks.php">特别致谢</a> | <a href="javascript:window.external.AddFavorite(document.location.href,document.title);">收藏本站</a> | <a href="#">欢迎投稿</a> | <a href="http://www.guoxuedashi.com/forum/">意见建议</a> | <a href="http://www.guoxuemi.com/">国学迷</a> | <a href="http://www.shuowen.net/">说文网</a><script language="javascript" type="text/javascript" src="https://js.users.51.la/17753172.js"></script><br />
  Copyright &copy; 国学大师 古典图书集成 All Rights Reserved.<br>
  
  <span style="font-size:14px">免责声明:本站非营利性站点,以方便网友为主,仅供学习研究。<br>内容由热心网友提供和网上收集,不保留版权。若侵犯了您的权益,来信即刪。scp168@qq.com</span>
  <br />
ICP证:<a href="http://www.beian.miit.gov.cn/" target="_blank">鲁ICP备19060063号</a></div>
<!-- 页脚END --> 
<script src="http://www.guoxuedashi.com/img/plus.php?id=22"></script>
<script src="http://www.guoxuedashi.com/img/tongji.js"></script>

</body>
</html>
