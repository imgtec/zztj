<!DOCTYPE html PUBLIC "-//W3C//DTD XHTML 1.0 Transitional//EN" "http://www.w3.org/TR/xhtml1/DTD/xhtml1-transitional.dtd">
<html xmlns="http://www.w3.org/1999/xhtml">
<head>
<meta http-equiv="Content-Type" content="text/html; charset=utf-8" />
<meta http-equiv="X-UA-Compatible" content="IE=Edge,chrome=1">
<title>資治通鑒_117-資治通鑑卷一百十六_117-資治通鑑卷一百十六</title>
<meta name="Keywords" content="資治通鑒_117-資治通鑑卷一百十六_117-資治通鑑卷一百十六">
<meta name="Description" content="資治通鑒_117-資治通鑑卷一百十六_117-資治通鑑卷一百十六">
<meta http-equiv="Cache-Control" content="no-transform" />
<meta http-equiv="Cache-Control" content="no-siteapp" />
<link href="/img/style.css" rel="stylesheet" type="text/css" />
<script src="/img/m.js?2020"></script> 
</head>
<body>
 <div class="ClassNavi">
<a  href="/24shi/">二十四史</a> | <a href="/SiKuQuanShu/">四库全书</a> | <a href="http://www.guoxuedashi.com/gjtsjc/"><font  color="#FF0000">古今图书集成</font></a> | <a href="/renwu/">历史人物</a> | <a href="/ShuoWenJieZi/"><font  color="#FF0000">说文解字</a></font> | <a href="/chengyu/">成语词典</a> | <a  target="_blank"  href="http://www.guoxuedashi.com/jgwhj/"><font  color="#FF0000">甲骨文合集</font></a> | <a href="/yzjwjc/"><font  color="#FF0000">殷周金文集成</font></a> | <a href="/xiangxingzi/"><font color="#0000FF">象形字典</font></a> | <a href="/13jing/"><font  color="#FF0000">十三经索引</font></a> | <a href="/zixing/"><font  color="#FF0000">字体转换器</font></a> | <a href="/zidian/xz/"><font color="#0000FF">篆书识别</font></a> | <a href="/jinfanyi/">近义反义词</a> | <a href="/duilian/">对联大全</a> | <a href="/jiapu/"><font  color="#0000FF">家谱族谱查询</font></a> | <a href="http://www.guoxuemi.com/hafo/" target="_blank" ><font color="#FF0000">哈佛古籍</font></a> 
</div>

 <!-- 头部导航开始 -->
<div class="w1180 head clearfix">
  <div class="head_logo l"><a title="国学大师官网" href="http://www.guoxuedashi.com" target="_blank"></a></div>
  <div class="head_sr l">
  <div id="head1">
  
  <a href="http://www.guoxuedashi.com/zidian/bujian/" target="_blank" ><img src="http://www.guoxuedashi.com/img/top1.gif" width="88" height="60" border="0" title="部件查字,支持20万汉字"></a>


<a href="http://www.guoxuedashi.com/help/yingpan.php" target="_blank"><img src="http://www.guoxuedashi.com/img/top230.gif" width="600" height="62" border="0" ></a>


  </div>
  <div id="head3"><a href="javascript:" onClick="javascript:window.external.AddFavorite(window.location.href,document.title);">添加收藏</a>
  <br><a href="/help/setie.php">搜索引擎</a>
  <br><a href="/help/zanzhu.php">赞助本站</a></div>
  <div id="head2">
 <a href="http://www.guoxuemi.com/" target="_blank"><img src="http://www.guoxuedashi.com/img/guoxuemi.gif" width="95" height="62" border="0" style="margin-left:2px;" title="国学迷"></a>
  

  </div>
</div>
  <div class="clear"></div>
  <div class="head_nav">
  <p><a href="/">首页</a> | <a href="/ShuKu/">国学书库</a> | <a href="/guji/">影印古籍</a> | <a href="/shici/">诗词宝典</a> | <a   href="/SiKuQuanShu/gxjx.php">精选</a> <b>|</b> <a href="/zidian/">汉语字典</a> | <a href="/hydcd/">汉语词典</a> | <a href="http://www.guoxuedashi.com/zidian/bujian/"><font  color="#CC0066">部件查字</font></a> | <a href="http://www.sfds.cn/"><font  color="#CC0066">书法大师</font></a> | <a href="/jgwhj/">甲骨文</a> <b>|</b> <a href="/b/4/"><font  color="#CC0066">解密</font></a> | <a href="/renwu/">历史人物</a> | <a href="/diangu/">历史典故</a> | <a href="/xingshi/">姓氏</a> | <a href="/minzu/">民族</a> <b>|</b> <a href="/mz/"><font  color="#CC0066">世界名著</font></a> | <a href="/download/">软件下载</a>
</p>
<p><a href="/b/"><font  color="#CC0066">历史</font></a> | <a href="http://skqs.guoxuedashi.com/" target="_blank">四库全书</a> |  <a href="http://www.guoxuedashi.com/search/" target="_blank"><font  color="#CC0066">全文检索</font></a> | <a href="http://www.guoxuedashi.com/shumu/">古籍书目</a> | <a   href="/24shi/">正史</a> <b>|</b> <a href="/chengyu/">成语词典</a> | <a href="/kangxi/" title="康熙字典">康熙字典</a> | <a href="/ShuoWenJieZi/">说文解字</a> | <a href="/zixing/yanbian/">字形演变</a> | <a href="/yzjwjc/">金 文</a> <b>|</b>  <a href="/shijian/nian-hao/">年号</a> | <a href="/diming/">历史地名</a> | <a href="/shijian/">历史事件</a> | <a href="/guanzhi/">官职</a> | <a href="/lishi/">知识</a> <b>|</b> <a href="/zhongyi/">中医中药</a> | <a href="http://www.guoxuedashi.com/forum/">留言反馈</a>
</p>
  </div>
</div>
<!-- 头部导航END --> 
<!-- 内容区开始 --> 
<div class="w1180 clearfix">
  <div class="info l">
   
<div class="clearfix" style="background:#f5faff;">
<script src='http://www.guoxuedashi.com/img/headersou.js'></script>

</div>
  <div class="info_tree"><a href="http://www.guoxuedashi.com">首页</a> > <a href="/SiKuQuanShu/fanti/">四库全书</a>
 > <h1>资治通鉴</h1> <!--         下载:【右键另存为】即可 --></div>
  <div class="info_content zj clearfix">
  
<div class="info_txt clearfix" id="show">
<center style="font-size:24px;">117-資治通鑑卷一百十六</center>
    資治通鑑卷一百十六<br />
<br />
  宋 司馬光 撰<br />
<br />
  胡三省 音注<br />
<br />
  晉紀三十八【起重光大淵獻盡閼逢攝提格凡四年】<br />
<br />
  安皇帝辛<br />
<br />
  義熙七年春正月己未劉裕還建康 秦廣平公弼有寵於秦王興為雍州刺史鎮安定【姚秦分嶺北五郡置雍州刺史鎮安定雍於用翻】姜紀諂附於弼勸弼結興左右以求入朝興徵弼為尚書令侍中大將軍弼遂傾身結納朝士【朝直遙翻】收采名勢以傾東宫國人惡之【惡烏路翻】會興以西北多叛亂欲命重將鎮撫之【將即亮翻下待將同】隴東太守郭播請使弼出鎮【魏收地形志有隴東郡領涇陽祖厲撫夷三縣不載立郡之始蓋苻姚所置也西魏置隴東於汧源唐之隴州是也】興不從以太常索稜為太尉領隴西内史使招撫西秦【為索稜降西秦張本索昔各翻】西秦王乾歸遣使送所掠守宰謝罪請降【謂去年克南安畧陽隴西諸郡所得守宰也使疏吏翻降戶江翻】興遣鴻臚拜乾歸都督隴西嶺北雜胡諸軍事征西大將軍河州牧單于河南王太子熾磐為鎮西將軍左賢王平昌公【臚陵如翻單音蟬熾昌志翻】興命羣臣搜舉賢才右僕射梁喜曰臣累受詔而未得其人可謂世之乏才興曰自古帝王之興未嘗取相於昔人【相息亮翻】待將於將來隨時任才皆能致治【將即亮翻治直吏翻】卿自識拔不明豈得遠誣四海乎羣臣咸悦【姚興之折梁喜誠是矣羣臣體興之意而明揚仄陋者誰乎此所謂好虚名而無實用者也】 秦姚詳屯杏城為夏王勃勃所逼【夏戶雅翻】南奔大蘇勃勃遣平東將軍鹿弈干追斬之盡俘其衆勃勃南攻安定破尚書楊佛嵩于青石北原降其衆四萬五千【降戶江翻】進攻東鄉下之徙三千餘戶于貳城秦鎮北參軍王買德奔夏夏王勃勃問以滅秦之策買德曰秦德雖衰藩鎮猶固願且蓄力以待之勃勃以買德為軍師中郎將【買德遂為夏之謀臣】秦王興遣衛大將軍常山公顯迎姚詳弗及遂屯杏城 劉藩帥孟懷玉等諸將追盧循至嶺表【帥讀曰率】二月壬午懷玉克始興斬徐道覆 河南王乾歸徙鮮卑僕渾部三千餘戶于度堅城【僕渾降乾歸見上卷上年度堅城即乞伏先所都度堅山城也】以子敕勃為秦興太守以鎮之【乞伏乾歸本建國號日秦故置秦興郡于度堅山】 焦朗猶據姑臧【朗據姑臧見上卷上年】沮渠蒙遜攻抜其城【沮子余翻】執朗而宥之以其弟拏為秦州刺史鎮姑臧【孥女居翻】遂伐南凉圍樂都【樂音洛】三旬不克南凉王傉檀以子安周為質乃還【質音致】 吐谷渾樹洛于伐南凉敗南凉太子虎臺【敗補邁翻】 南凉王傉檀欲復伐沮渠蒙遜邯川護軍孟愷諫曰【復扶又翻水經河水自西平郡東流逕澆河郡故城北又東逕石城南又東逕邯川城南劉昫日廓州化隆縣東古邯川地杜佑日後漢和帝時侯霸置東西邯屯田五部邯水名也分流左右在寜塞郡據唐志寜塞本澆河郡唐玄宗天寶中更名今之廓州】蒙遜新并姑臧凶勢方盛不可攻也傉檀不從五道俱進至番禾苕藋【番音盤藋徒弔翻】掠五千餘戶而還【還從宣翻又如字下同】將軍屈右曰今既獲利宜倍道旋師早度險阨蒙遜善用兵若輕軍猝至大敵外逼徙戶内叛此危道也衛尉伊力延曰彼步我騎【騎奇寄翻】勢不相及今倍道而歸則示弱且捐棄資財非計也俄而昏霾風雨蒙遜兵大至傉檀敗走蒙遜進圍樂都傉檀嬰城固守以子染干為質以請和【質音致】蒙遜乃還 三月劉裕始受太尉中書監【加太尉見上卷五年加中書監見六年】以劉穆之為太尉司馬陳郡殷景仁為行參軍【行參軍未得與參軍事班也事已見前】裕問穆之曰孟昶參佐誰堪入我府者穆之舉前建威中兵參軍謝晦晦安兄據之曾孫也【孟昶為建威將軍辟晦為中兵參軍】裕即命為參軍裕嘗訊囚其旦刑獄參軍有疾以晦代之于車中一覽訊牒催促便下【下遐稼翻】相府多事【相息亮翻】獄繫殷積晦隨問酬辨曾無違謬裕由是奇之即日署刑獄賊曹【刑獄蓋分民曹賊曹賊曹掌盜賊事宋志諸府參軍有長流賊曹刑獄賊曹城局賊曹刑獄無民曹謝晦為參軍未掌曹職今乃升署】晦美風姿善言笑博贍多通【贍時艶翻】裕深加賞愛 盧循行收兵至番禺遂圍之孫處拒守二十餘日【番禺音潘愚處昌呂翻】沈田子言於劉蕃曰番禺城雖險固本賊之巢穴今循圍之或有内變且孫季高衆力寡弱【孫處字季高】不能持久若使賊還據廣州凶勢復振矣【復扶又翻】夏四月田子引兵救番禺擊循破之所殺萬餘人循走田子與處共追之又破循於蒼梧鬱林寜浦【蒼梧鬱林漢古郡寜浦郡吳分合浦郡立蒼梧唐之鬱州鬱林唐之鬱林州寜浦唐之横州】會處病不能進循奔交州初九真太守李遜作亂【九真漢古郡唐之愛州】交州刺史交趾杜瑗討斬之瑗卒【瑗于眷翻卒子息翻】朝廷以其子慧度為交州刺史詔書未至循襲破合浦【合浦漢古郡唐之廉州】徑向交州慧度帥州府文武拒循於石碕破之【碕渠羈翻岸曲曰碕帥讀曰率】循餘衆猶三千人李遜餘黨李脱等結集俚獠五千餘人以應循【俚音里獠魯皓翻】庚子循晨至龍編南津【交趾郡龍編縣州郡皆治焉水經註漢建安二十三年立州之始蛟龍磐編於水南北二津故改龍淵曰龍編余據二漢志皆作龍編無亦師古章懷避唐諱因亦改淵為編乎】慧度悉散家財以賞軍士與循合戰擲雉尾炬焚其艦【雉尾炬束草之一頭施鐵鏃草尾則散開如雉尾然爇火以投敵艦戶黯翻】以步兵夾岸射之【射而亦翻】循衆艦俱然兵衆大潰循知不免先鴆妻子召妓妾問曰【妓渠綺翻】誰能從我死者多云雀鼠貪生就死實難或云官尚當死某豈願生乃悉殺諸辭死者因自投于水慧度取其尸斬之并其父子及李脱等函七首送建康 初劉毅在京口貧困與知識射於東堂庾悦為司徒右長史後至奪其射堂衆人皆避之毅獨不去悦厨饌甚盛不以及毅毅從悦求子鵝炙【饌雛戀翻又雛睆翻炙之夜翻子鵝為炙尤肥美】悦怒不與毅由是銜之至是毅求兼督江州詔許之因奏稱江州内地以治民為職【治直之翻】不當置軍府彫耗民力宜罷軍府移鎮豫章而尋陽接蠻可即州府千兵以助郡戍於是解悦都督將軍官以刺史鎮豫章毅以親將趙恢領千兵守尋陽悦府文武三千悉入毅府符攝嚴峻【符攝符下江州追攝之也】悦忿懼至豫章疽發背卒【疽千余翻卒子恤翻】河南王乾歸徙羌句豈等部衆五千餘戶於疊蘭城<br />
<br />
  【句豈降乾歸見上卷上年疊蘭城在大夏西南嵻㟍東北】以兄子阿柴為興國太守以鎮之【漢末興國氐王阿貴據興國城在略陽郡界乞伏因其地名置郡】五月復以子木弈干為武威太守鎮嵻㟍城【嵻㟍城四年乞伏熾磐所築復扶又翻】丁卯魏主嗣謁金陵山陽侯奚斤居守【守式又翻】昌黎王慕容伯兒謀反己巳奚斤并其黨收斬之 秋七月燕王跋以太子永領大單于置西輔【太子領大單于始於劉漢時置左右輔而已跋增置前輔後輔單音蟬】柔然可汗斛律遣使獻馬三千匹於跋【可從刋入聲汗音寒使疏吏翻】求娶跋女樂浪公主【樂浪音洛琅】跋命羣臣議之遼西公素弗曰前世皆以宗女妻六夷宜許以妃嬪之女【嬪毗賓翻】樂浪公主不宜下降非類跋曰朕方崇信殊俗奈何欺之乃以樂浪公主妻之【妻七細翻】跋勤於政事勸課農桑省徭役薄賦斂【斂力贍翻】每遣守宰必親引見【見賢遍翻】問為政之要以觀其能燕人悦之 河南王乾歸遣平昌公熾磐及中軍將軍審䖍伐南涼審䖍乾歸之子也八月熾磐兵濟河【此濟金城河也熾昌志翻】南涼王傉檀遣太子虎臺逆戰於嶺南【傉奴沃翻】南涼兵敗虜牛馬十餘萬而還【還從宣翻又如字下同】 沮渠蒙遜帥輕騎襲西涼【帥讀曰率騎奇寄翻】西涼公暠曰兵有不戰而敗敵者【暠古老翻敗補邁翻】挫其鋭也蒙遜新與吾盟【事見上卷上年】而遽來襲我我閉門不與戰待其鋭氣竭而擊之蔑不克矣頃之蒙遜粮盡而歸暠遣世子歆帥騎七千邀擊之蒙遜大敗獲其將沮渠百年 河南王乾歸攻秦畧陽太守姚龍于柏陽堡克之冬十一月進攻南平太守王憬於水洛城【水經注水洛亭在隴山之西漢略陽縣界鄭戩曰水洛城西占隴坁通秦州往來路隴之二水環城西流繞帶渭河川平上沃廣數百里元豐九域志德順軍西南一百里有水洛城仁宗朝鄭戩使劉滬所築也憬居永翻】又克之徙民三千餘戶於譚郊【譚郊在冶城西北】遣乞伏審䖍帥衆二萬城譚郊【帥讀曰率】十二月西羌彭利髮襲據枹罕【枹音膚】自稱大將軍河州牧乾歸討之不克 是歲并州刺史劉道憐為北徐州刺史移鎮彭城<br />
<br />
  八年春正月河南王乾歸復討彭利髮【復扶又翻】至奴葵谷利髮棄衆南走乾歸遣振威將軍乞伏公府追至清水斬之收羌戶一萬三千以乞伏審䖍為河州刺史鎮枹罕而還 二月丙子以吳興太守孔靖為尚書右僕射河南王乾歸徙都譚郊命平昌公熾磐鎮苑川乾歸<br />
<br />
  擊吐谷渾阿若干於赤水降之【五代志隋大業五年平吐谷渾置河源郡於古赤水城蓋近積石山魏收地形志臨洮郡有赤水縣水經注赤水城亦曰臨洮東城降音戶江翻】 夏四月劉道規以疾求歸許之道規在荆州累年【元年道規刺荆州】秋毫無犯及歸府庫帷幕儼然若舊隨身甲士二人遷席於舟中道規刑之於市以後將軍豫州刺史劉毅為衛將軍都督荆寧秦雍四州諸軍事荆州刺史【雍於用翻】毅謂左衛將軍劉敬宣曰吾沗西任欲屈卿為長史南蠻【為南蠻校尉府長史也】豈有見輔意乎敬宣懼以告太尉裕裕笑曰但令老兄平安必無過慮毅性剛愎【愎弼力翻】自謂建義之功與裕相埒【埒龍輟翻等也】深自矜伐雖權事推裕而心不服【觀去年荅弟藩之言可知已】及居方岳常怏怏不得志【怏於兩翻】裕每柔而順之毅驕縱滋甚嘗云恨不遇劉項與之爭中原及敗於桑落【事見上卷六年】知物情已去彌復憤激【復扶又翻】裕素不學而毅頗涉文雅故朝士有清望者多歸之【朝直遙翻】與尚書僕射謝混丹陽尹郗僧施深相憑結僧施超之從子也【郗丑之翻郗超黨於桓温僧施黨於劉毅超僅免而僧施及禍矣從才用翻】毅既據上流隂有圖裕之志求兼督交廣二州裕許之毅又奏以郗僧施為南蠻校尉後軍司馬毛修之為南郡太守裕亦許之以劉穆之代僧施為丹陽尹毅表求至京口辭墓裕往會之於倪塘寧遠將軍胡藩言於裕曰公謂劉衛軍終能為公下乎【毅為衛將軍故稱之】裕默然久之曰卿謂何如藩曰連百萬之衆攻必取戰必克毅以此服公至於涉獵傳記【師古日涉若涉水獵若獵獸言歷覽之不專精也傳直戀翻】一談一詠自許以為雄豪以是搢紳白面之士輻輳歸之恐終不為公下不如因會取之裕曰吾與毅俱有克復之功其過未彰不可自相圖也【裕雖以是言荅藩隂有以處毅者矣】 乞伏熾磐攻南凉三河太守吳隂于白土克之以乞伏出累代之【水經河水過邯川城南又東逕臨津城北白土城南闞駰十三州志曰左南津西六十里有白土城在大河之北為緣河濟渡之地累力追翻魏收曰白土縣漢屬上郡晉屬金城郡後魏屬新平郡余謂後魏新平之白土乃漢上郡之白土晉金城之白土乃左南西之白土各是一處五代志邠州新平縣舊曰白土此漢上郡及後魏之白土也南凉之白土當在唐鄯州界】 六月乞伏公府弑河南王乾歸【公府國仁之子也以不得立故行弑逆】并殺其諸子十餘人走保大夏【夏戶雅翻下同】平昌公熾磐遣其弟廣武將軍智達揚武將軍木奕干帥騎三千討之以其弟曇達為鎮京將軍鎮譚郊【乞伏都譚郊自謂為京師故置鎮京將軍以鎮之帥讀曰率騎奇寄翻下同曇徒含翻】驍騎將軍婁機鎮苑川【驍堅堯翻】熾磐帥文武及民二萬餘戶遷于枹罕秦人多勸秦王興乘亂取熾磐興曰伐人喪非禮也夏王勃勃欲攻熾磐軍師中郎將王買德諫曰熾磐吾之與國今遭喪亂【喪息郎翻】吾不能恤又恃衆力而伐之匹夫猶且恥為况萬乘乎【乘繩證翻】勃勃乃止 閏月庚子南郡烈武公劉道規卒 秋七月己巳朔魏主嗣東巡置四廂大將十二小將以山陽侯斤元城侯屈行左右丞相【奚斤封山陽侯拓跋屈封元城侯】庚寅嗣至濡源【濡乃官翻】巡西北諸部落 乞伏智達等擊破乞伏公府於大夏公府奔疊蘭城就其弟阿柴智達等攻拔之斬阿柴父子五人公府奔嵻㟍南山【嵻音康㟍音郎】追獲之并其四子轘之於譚郊【轘音宦】八月乞伏熾磐自稱大將軍河南王【熾磐乾歸長子】大赦改元永康葬乾歸於枹罕【枹音膚】諡曰武元廟號高祖 皇后王氏崩 庚戌魏主嗣還平城【出巡而還也】 九月河南王熾磐以尚書令武始翟勍為相國【勍渠京翻】侍中太子詹事趙景為御史大夫罷尚書令僕尚書六卿侍中等官癸酉葬僖皇后于休平陵 劉毅至江陵多變易守<br />
<br />
  宰輒割豫州文武江州兵力萬餘人以自隨會毅疾篤郗僧施等恐毅死其黨危乃勸毅請從弟兖州刺史藩以自副【從才用翻下同】太尉裕偽許之藩自廣陵入朝【朝直遙翻】己卯裕以詔書罪狀毅云與藩及謝混共謀不軌收藩及混賜死初混與劉毅欵昵【款尼賢翻】混從兄澹常以為憂【澹徒覽翻】漸與之疎謂弟璞及從子瞻曰益夀此性終當破家【益夀混小字也】澹安之孫也庚辰詔大赦以前會稽内史司馬休之為都督荆雍梁秦寜益六州諸軍事荆州刺史【雍於用翻】北徐州刺史劉道憐為兖青二州刺史鎮京口【北徐州刺史治彭城使道憐鎮京口以為建康北藩之重】使豫州刺史諸葛長民監太尉留府事【監工銜翻】裕疑長民難獨任乃加劉穆之建武將軍置佐吏配給資力以防之【是時裕已有殺長民之心矣】壬午裕帥諸軍發建康參軍王鎮惡請給百舸為前驅【帥讀曰率舸古我翻】丙申至姑孰以鎮惡為振武將軍與龍驤將軍蒯恩將百舸前發【驤思將翻蒯苦怪翻將即亮翻】裕戒之曰若賊可擊擊之不可者燒其船艦【艦戶黯翻】留屯水際以待我於是鎮惡晝夜兼行揚聲言劉兖州上【上時掌翻下步上藩上同】冬十月己未鎮惡至豫章口去江陵城二十里捨船步上蒯恩軍居前鎮惡次之舸留一二人對舸岸上立六七旗旗下置鼓語所留人【語牛倨翻下同】計我將至城便鼓嚴令若後有大軍狀【鼓嚴擂鼓也】又分遣人燒江津船艦鎮惡徑前襲城語前軍士【告語在前軍士也】有問者但云劉兖州至津戍及民間皆晏然不疑未至城五六里逢毅要將朱顯之【要將者所親之將掌兵要者】欲出江津問劉兖州何在軍士曰在後顯之至軍後不見藩而見軍人擔彭排戰具【彭排即今之旁排所以扞鋒矢孫愐曰樐彭排釋名日彭排軍器也彭旁也在旁排敵禦攻也】望江津船艦已被燒鼓嚴之聲甚盛知非藩上【上時掌翻】便躍馬馳去吿毅行令閉諸城門鎮惡亦馳進門未及下關軍人因得入城衛軍長史謝純入參承毅【僚佐省府公謂之參承】出聞兵至左右欲引車歸純叱之曰我人吏也【言為人之吏】逃將安之馳還入府純安兄據之孫也鎮惡與城内兵鬭且攻其金城【凡城内牙城晉宋時謂之金城】自食時至中晡【日加申為晡中晡正申時也申末為下晡】城内人敗散鎮惡穴其金城而入遣人以詔及赦文并裕手書示毅毅皆燒不視與司馬毛修之等督士卒力戰城内人猶未信裕自來軍士從毅自東來者與臺軍多中表親戚且鬬且語知裕自來人情離駭逮夜聽事前兵皆散斬毅勇將趙蔡【將即亮翻】毅左右兵猶閉東西閤拒戰鎮惡慮闇中自相傷犯乃引軍出圍金城開其南面毅慮南有伏兵夜半帥左右三百許人【帥讀曰率】開北門突出毛修之謂謝純曰君但隨僕去純不從為人所殺毅夜投牛牧佛寺【牛牧寺在江陵城北二十里】初桓蔚之敗也【事見一百一十四卷元年蔚紆勿翻】走投牛牧寺僧昌昌保藏之毅殺昌至是寺僧拒之曰昔亡師容桓蔚為劉衛軍所殺今實不敢容異人毅嘆曰為法自弊一至於此【史記商君得罪於秦亡至關下投舍客舍人不知其是商君也曰商君之法舍人無驗者坐之商君歎曰嗟乎為法自弊一至此哉】遂縊而死明日居人以告乃斬首於市并子姪皆伏誅毅兄模奔襄陽魯宗之斬送之初毅季父鎮之閒居京口不應辟召常謂毅及藩曰汝輩才器足以得志但恐不久耳我不就汝求財位亦不同汝受罪累【累力瑞翻】每見毅藩導從到門輒詬之【從才用翻詬許候翻又古候翻罵也】毅甚敬畏未至宅數百步悉屏儀衛【屏必郢翻】與白衣數人俱進及毅死太尉裕奏徵鎮之為散騎常侍光禄大夫【散悉亶翻騎奇寄翻】固辭不至 仇池公楊盛叛秦【義熙元年盛降秦今復叛】侵擾祁山秦王興遣建威將軍趙琨為前鋒立節將軍姚伯夀繼之前將軍姚恢出鷲峽【鷲音就】秦州刺史姚嵩出羊頭峽右衛將軍胡翼度出汧城以討盛興自雍赴之【汧苦堅翻雍於用翻】與諸將會于隴口【隴道之口也將即亮翻】天水太守王松忽言於嵩曰先帝神略無方徐洛生以英武佐命再入仇池無功而還【姚萇再攻仇池當攷】非楊氏智勇能全也直地埶險固耳今以趙琨之衆使君之威準之先朝實未見成功使君具悉形便何不表聞嵩不從盛帥衆與琨相持伯夀畏懦不進琨衆寡不敵為盛所敗【帥讀曰率下同敗補邁翻】興斬伯夀而還興以楊佛嵩為雍州刺史帥嶺北見兵以擊夏【秦雍州統嶺北五郡治安定見賢遍翻】行數日興謂羣臣曰佛嵩每見敵勇不自制吾常節其兵不過五千人今所將既多遇敵必敗行已遠追之無及將若之何佛嵩與夏王勃勃戰果敗為勃勃所執絶亢而死【亢與吭同居郎翻】 秦立昭儀齊氏為后 沮渠蒙遜遷於姑臧十一月己卯太尉裕至江陵殺郗僧施初毛修之雖<br />
<br />
  為劉毅僚佐素自結於裕故裕特宥之賜王鎮惡爵漢夀子裕問毅府諮議參軍申永曰今日何施而可永曰除其宿釁倍其惠澤貫叙門次【魏晉以來率以門地高下為用人之次第貫叙者以次叙之若穿錢貫然也】顯擢才能如此而已裕納之下書寛租省調【調徒弔翻】節役原刑禮辟名士荆人悦之 諸葛長民驕縱貪侈所為多不法為百姓患常懼太尉裕按之及劉毅被誅長民謂所親曰昔年醢彭越今年殺韓信【漢薛公之言被皮義翻】禍其至矣乃屏人問劉穆之曰【屏必郢翻下同】悠悠之言皆云太尉與我不平何以至此穆之曰公泝流遠征以老母稚子委節下若一豪不盡【稚直利翻豪古毫字通】豈容如此邪長民意乃小安長民弟輔國大將軍黎民說長民曰【說輸芮翻】劉氏之亡亦諸葛氏之懼也宜因裕未還而圖之長民猶豫未發既而歎曰貧賤常思富貴富貴必履危機今日欲為丹徒布衣豈可得邪【長民琅邪陽都人僑居丹徒】因遺冀州刺史劉敬宣書曰盤龍狠戾專恣自取夷滅【遺于季翻劉毅小字盤龍】異端將盡世路方夷富貴之事相與共之敬宣報曰下官自義熙以來沗三州七郡【敬宣自北還拜晉陵太守遷江州鎮尋陽兼領郡事徵拜宣城内史領襄城太守遷鎮蠻護軍安豐太守梁國内史又遷青州刺史尋改冀州】常懼福過災生思避盈居損富貴之旨非所敢當且使以書呈裕裕曰阿夀故為不負我也【敬宣字萬夀故裕稱之曰阿夀】劉穆之憂長民為變屏人問太尉行參軍東海何承天曰公今行濟否承天曰荆州不憂不時判【判決也】别有一慮耳公昔年自左里還入石頭甚脱爾【謂破盧循還時也脱爾謂輕脱而還不為嚴備也】今還宜加重慎穆之曰非君不聞此言裕在江陵輔國將軍王誕白裕求先下裕曰諸葛長民似有自疑心卿詎宜便去誕曰長民知我蒙公垂盻今輕身單下必當以為無虞乃可以少安其意耳裕笑曰卿勇過賁育矣【盼匹莧翻少詩沼翻賁音奔】乃聽先還 沮渠蒙遜即河西王位【沮渠蒙遜臨松盧水胡人也其先世為匈奴左沮渠遂以官為氏】大赦改元玄始置官僚如涼王光為三河王故事【呂光稱三河王見一百七卷孝武太元十四年】 太尉裕謀伐蜀擇元帥而難其人【帥所類翻】以西陽太守朱齡石既有武幹又練吏職欲用之衆皆以為齡石資名尚輕難當重任裕不從十二月以齡石為益州刺史帥寜朔將軍臧熹河間太守蒯恩下邳太守劉鍾等伐蜀【帥讀曰率】分大軍之半二萬人以配之熹裕之妻弟位居齡石之右亦隸焉裕與齡石密謀進取曰劉敬宣往年出黄虎無功而退【事見一百十四卷四年】賊謂我今應從外水往而料我當出其不意猶從内水來也【庾仲雍曰巴郡江州縣對二水口右則涪内水左則蜀外水】如此必以重兵守涪城以備内道【涪音浮】若向黄虎正墮其計今以大衆自外水取成都疑兵出内水此制敵之奇也而慮此聲先馳賊審虚實别有函書封付齡石署函邉曰至白帝乃開諸軍雖進未知處分所由【處昌呂翻分扶問翻】毛修之固請行裕恐修之至蜀必多所誅殺土人與毛氏有嫌亦當以死自固不許【以毛璩之家為蜀人所滅故也】 分荆州十郡置湘州【成帝咸和三年省湘州入荆州今復置】加太尉裕太傅揚州牧 丁巳魏主嗣北巡至長城<br />
<br />
  而還【秦所築長城也】<br />
<br />
  九年春二月庚戌魏主嗣如高柳川甲寅還宫 太尉裕自江陵東還【還從宣翻又如字下同】駱驛遣輜重兼行而下【重直用翻】前刻至日每淹留不進諸葛長民與公卿頻日奉候於新亭輒差其期乙丑晦裕輕舟徑進潜入東府【劉穆之何承天所慮者裕已了了於胷中矣】三月丙寅朔旦長民聞之驚趨至門裕伏壯士丁旿於幔中【旿阮古翻】引長民却人閒語凡平生所不盡者皆及之長民甚悦丁旿自幔後出於座拉殺之【幔莫半翻拉盧合翻】輿尸付廷尉收其弟黎民黎民素驍勇【驍堅堯翻】格鬬而死并殺其季弟大司馬參軍幼民從弟寧朔將軍秀之【從才用翻】 庚午秦王興遣使至魏修好【使疏吏翻好呼到翻】 太尉裕上表曰大司馬温以民無定本傷治為深庚戌土斷以一其業【庚戌制見一百一卷哀帝興寧二年】于時財阜國豐實由於此自茲迄今漸用頹弛請申前制于是依界土斷唯徐兖青三州居晉陵者不在斷例【徐青兖三州都督率治晉陵故難以土斷斷丁亂翻】諸流寓郡縣多所併省戊寅加裕豫州刺史裕固讓太傅州牧【辭去年冬所加也】 林邑范胡達寇九真杜慧度擊斬之 河南王熾磐遣鎮東將軍曇達平東將軍王松夀將兵東擊休官權小郎呂破胡於白石川【闕】<br />
<br />
  【含翻將即亮翻】大破之虜其男女萬餘口進據白石城顯親<br />
<br />
  休官權小成呂奴迦等二萬餘戶據白阬不服【迦居牙翻】曇達攻斬之隴右休官悉降秦太尉索稜以隴西降熾磐【七年秦令索稜守隴西以招撫乞伏索昔各翻降戶江翻】熾磐以稜為太傅 夏王勃勃大赦改元鳳翔以叱干阿利領將作大匠發嶺北夷夏十萬人築都城於朔方水北黑水之南【水經注奢延水又謂之朔方水源出奢延縣西南赤沙阜東北流逕奢延縣故城南赫連於是水之南築統萬城奢延水又東流黑水入焉水出奢延縣黑澗東南歷沙陵注奢延水統萬城唐為夏州定難節度使治所夏戶雅翻】勃勃曰朕方統一天下君臨萬邦宜名新城曰統萬阿利性巧而殘忍蒸土築城錐入一寸即殺作者而并築之勃勃以為忠委任之凡造兵器成呈之工人必有死者射甲不入則斬弓人【射而亦翻】入則斬甲匠又鑄銅為一大鼓飛廉翁仲銅駝龍虎之屬飾以黄金列於宫殿之前凡殺工匠數千由是器物皆精利勃勃自謂其祖從母姓為劉非禮也【載託曰漢高祖以宗女妻單于冒頓約為兄弟故其子孫冒姓劉氏】古人氏族無常乃改姓赫連氏言帝王係天為子其徽赫與天連也其非正統者皆以鐵伐為氏【勃勃父衛辰本鐵佛氏故改其非正統者為鐵伐氏】言其剛鋭如鐵皆堪伐人也 夏四月乙卯魏主嗣西巡命鄭兵將軍奚斤【鄭兵北史作都兵】鴻飛將軍尉古眞都將閭大肥等擊越勤部於跋那山大肥柔然人也【鴻飛將軍拓跋氏所創置將即亮翻柔然姓郁久閭氏今曰閭從省便也跋那山盖在廣甯郡之塞外】 河南王熾磐遣安北將軍烏地延冠軍將軍翟紹擊吐谷渾别統句旁于泣勤川大破之【冠古玩翻别統猶别帥也别統部落者也句古侯翻】 河西王蒙遜立子政德為世子加鎮衛大將軍録尚書事 南涼王傉檀伐河西王蒙遜蒙遜敗之於若厚塢又敗之於若涼【敗補邁翻】因進圍樂都【樂音洛下長樂同】二旬不克南涼湟河太守文支以郡降于蒙遜【降戶江翻】蒙遜以文支為廣武太守蒙遜復伐南涼傉檀以太尉俱延為質乃還【復扶又翻質音致】蒙遜西如苕藋【藋徒弔翻】遣冠軍將軍伏恩將騎一萬襲卑和烏啼二部大破之【漢有卑和羌居鮮水海】俘二千餘落而還蒙遜寢于新臺閹人王懷祖擊蒙遜傷足其妻孟氏禽斬之蒙遜母車氏卒【車尺遮翻】 五月乙亥魏主嗣如雲中舊宫【唐單于都護府領金河一縣秦漢之雲中也新書云金河本後魏道武所都】丙子大赦西河胡張外等聚衆為盜乙卯嗣遣會稽公長樂劉絜等屯西河招討之【按乙亥至丙子幾四十日五月無乙卯明矣恐是己卯會江外翻】 六月嗣如五原 朱齡石等至白帝發函書曰衆軍悉從外水取成都臧熹從中水取廣漢【水經注曰洛水出洛縣章山南逕洛縣故城南廣漢郡治也又南逕新都縣與綿水合又與湔水合亦謂之郫江又逕犍為牛鞞水又東逕資中縣謂之緜水緜水至江陽縣方山下入江謂之緜水口曰中水】老弱乘高艦十餘從内水向黄虎【艦戶黯翻】於是諸軍倍道兼行譙縱果命譙道福將重兵鎮涪城【將即亮翻涪音浮下同】以備内水齡石至平模去成都二百里縱遣秦州刺史侯暉尚書僕射譙詵帥衆萬餘屯平模【詵萃臻翻】夾岸築城以拒之齡石謂劉鍾曰今天時盛熱而賊嚴兵固險攻之未必可拔祇增疲困且欲養鋭息兵以伺其隙何如鍾曰不然前揚聲言大衆向内水譙道福不敢捨涪城今重軍猝至出其不意侯暉之徒已破膽矣賊阻兵守險者是其懼不敢戰也因其兇愳【兇許勇翻】盡銳攻之其埶必克克平模之後自可鼓行而進成都必不能守矣若緩兵相守彼將知人虛實涪軍忽來并力拒我人情既安良將又集【良將謂譙道福將郎亮翻】此求戰不獲軍食無資二萬餘人悉為蜀子虜矣齡石從之諸將以水北城地險兵多欲先攻其南城齡石曰今屠南城不足以破北若盡鋭以拔北城則南城不麾自散矣秋七月齡石帥諸軍急攻北城克之斬侯暉譙詵引兵迴趣南城【帥讀曰率趣七喻翻】南城自潰齡石捨船步進譙縱大將譙撫之屯牛脾【牛脾當作牛鞞孟康日鞞音髀師古曰音必爾翻牛鞞縣自漢以來屬犍為郡何承天曰晉穆帝度屬蜀郡今簡州西岸有古牛鞞戍城】譙小苟塞打鼻【打鼻山在今眉州彭山縣南十餘里山形孤起東臨江水俗云昔周鼎淪於此或見其鼻故名塞悉則翻】臧熹擊撫之斬之小苟聞之亦潰于是縱諸營屯望風相次奔潰戊辰縱棄成都出走尚書令馬耽封府庫以待晉師壬申齡石入成都誅縱同祖之親餘皆按堵使復其業縱出成都先辭墓其女曰走必不免祇取辱焉等死死於先人之墓可也縱不從譙道福聞平模不守自涪引兵入赴縱往投之道福見縱怒曰大丈夫有如此功業而棄之將安歸乎人誰不死何怯之甚也因投縱以劔中其馬鞍【中竹仲翻】縱乃去自縊死【縊於賜翻又於計翻】巴西人王志斬其首以送齡石道福謂其衆曰蜀之存亡實係於我不在譙王今我在猶足一戰衆皆許諾道福盡散金帛以賜衆衆受之而走道福逃於獠中【獠魯皓翻】巴民杜瑾執送之斬於軍門【義熙元年譙縱據蜀九年而滅瑾渠吝翻】齡石徙馬耽於越巂【巂音髓】耽謂其徒曰朱侯不送我京師欲滅口也【謂齡石多取庫物殺耽以滅口】吾必不免乃盥洗而臥引繩而死須臾齡石使至戮其尸【使疏吏翻】詔以齡石進監梁秦州六郡諸軍事【監古銜翻】賜爵豐城縣侯 魏奚斤等破越勤於跋郍山西徙二萬餘家於大甯 河西胡曹龍等擁部衆二萬人來入蒲子張外降之推龍為大單于【降戶江翻單音蟬】 丙戌魏主嗣如定襄大洛城【二漢志定襄郡有駱縣】河南王熾磐擊吐谷渾支旁于長柳川虜旁及其民<br />
<br />
  五千餘戶而還 八月癸卯魏主嗣還平城 曹龍請降于魏執送張外斬之丁丑魏主嗣如豺山宫癸未還九月再命太尉裕為太傅揚州牧固辭 河南王熾<br />
<br />
  磐擊吐谷渾别統掘逵於渴渾川大破之虜男女二萬三千冬十月掘逵帥其餘衆降於熾磐【掘其月翻帥讀曰率】 吐京胡與離石胡出以眷叛魏【水經注曰吐京即漢西河土軍縣夷夏俗音訛也後魏真君九年置吐京郡隋為隰州石樓縣地】魏主嗣命元城侯屈督會稽公劉絜永安侯魏勤以討之丁巳出以眷引夏兵邀擊絜禽之以獻于夏勤戰死【會工外翻夏戶雅翻】嗣以屈亡二將【將即亮翻】欲誅之既而赦之使攝并州刺史屈到州縱酒廢事嗣積其前後罪惡檻車徵還斬之【魏主嗣之入立也屈子磨渾有功焉屈恃之而驕積其惡而誅之非所以保功臣之門也】 十一月魏主嗣遣使請昏於秦【使疏吏翻】秦王興許之 是歲以敦煌索邈為梁州刺史【敦徒門翻索昔各翻】苻宣乃還仇池【苻宣入漢中見一百十四卷元年】初邈寓居漢川與别駕姜顯有隙凡十五年而邈鎮漢川顯乃肉袒迎候邈無愠色【愠於問翻】待之彌厚退而謂人曰我昔寓此失志多年若讐姜顯懼者不少【少詩沼翻】但服之自佳何必逞志於是闔境聞之皆悦【鞠羨之安東萊亦若是而已世人修怨以致禍者由不知此道也】<br />
<br />
  十年春正月辛酉魏大赦改元神瑞 辛巳魏主嗣如繁畤【畤音止】二月戊戌還平城 夏王勃勃侵魏河東蒲子 庚戌魏主嗣如豺山宫 魏并州刺史婁伏連襲殺夏所置吐京護軍及其守兵【魏書官氏志内入諸姓有匹婁氏後改為婁氏去年夏破拓跋屈因置守兵於吐京】 司馬休之在江陵頗得江漢民心子譙王文思在建康【文思休之之長子也譙王尚之死於桓玄之難帝反正以文思嗣國】性凶暴好通輕俠大尉裕惡之【好呼到翻惡烏路翻】三月有司奏文思擅捶殺國吏【捶止蘂翻】詔誅其黨而宥文思休之上疏謝罪請解所任不許裕執文思送休之令自訓厲意欲休之殺之休之但表廢文思并與裕書陳謝裕由是不悦【為後裕伐休之張本】以江州刺史孟懷玉兼督豫州六郡以備之【豫州六郡宣城襄城淮南廬江安豐歷陽也】 夏五月辛酉魏主嗣還平城 秦後將軍歛成討叛羌為羌所敗【敗補邁翻】懼罪出奔夏 秦王興有疾妖賊李弘與氐仇常反於貳城【妖於驕翻】興輿疾往討之斬常執弘而還【還從宣翻又如字】 秦左將軍姚文宗有寵於太子泓廣平公弼惡之【惡烏路翻】誣文宗有怨言秦王興怒賜文宗死于是羣臣畏弼側目弼言於興無不從者以所親天水尹冲為給事黄門侍即唐盛為治書侍御史【治直之翻下同】興左右掌機要者皆其黨也右僕射梁喜侍中任謙京兆尹尹昭承間言於興曰父子之際人所難言然君臣之義不薄於父子【任音壬間古莧翻父子君臣皆人之大倫故云然】故臣等不得默然廣平公弼潜有奪嫡之志陛下寵之太過假其威權傾險無賴之徒輻湊附之道路皆言陛下將有廢立之計信有之乎興曰豈有此邪喜等曰苟無之則陛下愛弼實所以禍之願去其左右【去羌呂翻】損其威權如此非特安弼乃所以安宗廟社稷興不應大司農竇温司徒左長史王弼皆密疏勸興立弼為太子興雖不從亦不責也興疾篤弼潛聚衆數千人謀作亂姚裕遣使以弼逆狀吿諸兄在藩鎮者【使疏吏翻】于是姚懿治兵於蒲阪鎮東將軍豫州牧洸治兵於洛陽【懿洸皆興子也冶直之翻洸姑黄翻】平西將軍諶治兵於雍【諶氏壬翻雍於用翻】皆欲赴長安討弼會興疾瘳【瘳且留翻】見羣臣征虜將軍劉羌泣以告興梁喜尹昭請誅弼且曰苟陛下不忍殺弼亦當奪其權任興不得已免弼尚書令使以將軍公還第【弼為大將軍封廣平公】懿等各罷兵懿洸諶與姚宣皆入朝使裕入白興求見【朝直遙翻見賢遍翻下同】興曰汝等正欲論弼事耳吾已知之裕曰弼苟有可論陛下所宜垂聽若懿等言非是便當寘之刑辟【辟毗亦翻】奈何逆拒之於是引見懿等於諮議堂宣流涕極言興曰吾自處之【處昌呂翻下同】非汝曹所憂撫軍東曹屬姜虬上疏曰廣平公弼釁成逆著道路皆知之昔文王之化刑于寡妻【詩思齊曰刑于寡妻至于兄弟】今聖朝之亂起自愛子雖欲含忍掩蔽而逆黨扇惑不已弼之亂心何由可革宜斥散凶徒以絶禍端興以虬表示梁喜曰天下人皆以吾兒為口實【孔安國曰口實謂常不去口】將何以處之喜曰信如虬言陛下宜早裁決興默然【史言姚興不聽臣子之言養成泓弼爭國之禍】 唾契汗乙弗等部皆叛南涼【契欺訖翻汗何干翻北史曰乙弗國有契翰一部風俗亦同杜佑曰乙弗敵後魏聞焉在吐谷渾北衆有萬餘落風俗與吐谷渾同然不識五穀唯食魚與蘇子蘇子狀若中國枸杞子或赤或黑西有契翰一部風俗亦同】南涼王傉檀欲討之邯川護軍孟愷諫曰【邯戶甘翻】今連年饑饉南逼熾磐北逼蒙遜百姓不安遠征雖克必有後患不如與熾磐結盟通糴慰撫雜部足食繕兵俟時而動傉檀不從謂太子虎臺曰蒙遜近去不能猝來旦夕所慮唯在熾磐然熾磐兵少易禦汝謹守樂都【少詩沼翻易以豉翻樂音洛下同】吾不過一月必還矣乃帥騎七千襲乙弗【帥讀曰率騎奇寄翻】大破之獲馬牛羊四十餘萬河南王熾磐聞之欲襲樂都羣臣咸以為不可太府主簿焦襲曰傉檀不顧近患而貪遠利【近患謂蒙遜熾磐遠利謂乙弗】我今伐之絶其西路【樂都之西路此傉檀自乙弗還樂都路也】使不得還救則虎臺獨守窮城可坐禽也此天亡之時必不可失熾磐從之帥步騎二萬襲樂都虎臺憑城拒守熾磐四面攻之南涼撫軍從事中郎尉肅言於虎臺曰外城廣大難守殿下不若聚國人守内城【國人謂鮮卑秃髪之種落】肅等帥晉人拒戰於外雖有不捷猶足自存虎臺曰熾磐小賊旦夕當走卿何過慮之深虎臺疑晉人有異心【夷人謂華人為晉人】悉召豪望有謀勇者閉之於内孟愷泣曰熾磐乘虛内侮國家危於累卵愷等進欲報恩退顧妻子人思效死而殿下乃疑之如是邪虎臺曰吾豈不知君之忠篤懼餘人脱生慮表以君等安之耳一夕城潰熾磐入樂都遣平遠將軍捷䖍帥騎五千追傉檀以鎮南將軍謙屯為都督河右諸軍事涼州刺史鎮樂都【捷䖍謙屯皆乞伏種】秃髪赴單為西平太守鎮西平以趙恢為廣武太守鎮廣武曜武將軍王基為晉興太守鎮浩亹【浩亹音誥門】徙虎臺及其文武百姓萬餘戶於枹罕【枹音膚】赴單烏孤之子也 河間人褚匡言於燕王跋曰陛下龍飛遼碣舊邦族黨傾首朝陽【言日生於東猶馮跋興於遼碣也其族黨在長樂者傾首而東望之碣其謁翻】以日為歲請往迎之跋曰道路數千里復隔異國如何可致【復扶又翻】匡曰章武臨海【跋之先長樂信都人而章武郡則晉分漢勃海郡所置也自信都至章武可以浮海至遼西】舟楫可通出於遼西臨渝不為難也【臨渝縣漢屬遼西郡師古曰渝音喻水經曰碣石在縣南】跋許之以匡為游擊將軍中書侍郎厚資遣之匡與跋從兄買從弟睹【從才用翻】自長樂帥五千餘戶歸于和龍【漢高帝置信都郡景帝二年為廣川國明帝更名樂成安帝改曰安平晉改曰長樂樂音洛帥讀曰率】契丹庫莫奚皆降于燕【契欺訖翻又音喫降戶江翻】跋署其大人為歸善王跋弟丕避亂在高句麗【句如字又音駒麗力知翻】跋召之以為左僕射封常山公 柔然可汗斛律將嫁女於燕【可從刋入聲汗音寒】斛律兄子步鹿真謂斛律曰幼女遠嫁憂思請以大臣樹黎等女為媵【媵以證翻】斛律不許步鹿真出謂樹黎等曰斛律欲以汝女為媵遠適它國樹黎恐與步鹿真謀使勇士夜伏於斛律穹廬之後伺其出而執之與女皆送於燕【伺相吏翻】立步鹿真為可汗而相之【相息亮翻】初社崘之徙高車也【事見一百十二卷元興元年崘盧昆翻】高車人叱洛侯為之鄉導以併諸部【鄉讀曰嚮】社崘德之以為大人步鹿真與社崘之子社拔共至叱洛侯家淫其少妻妻告步鹿真曰叱洛侯欲奉大檀為主大檀者社崘季父僕渾之子也領别部鎮西境素得衆心步鹿真歸而發兵圍叱洛侯叱洛侯自殺遂引兵襲大檀大檀逆擊破之執步鹿真及社拔殺之自立為可汗號牟汗紇升蓋可汗【魏收曰魏言制勝也】斛律至和龍燕王跋賜斛律爵上谷侯館之遼東待以客禮納其女為昭儀斛律上書請還其國跋曰今棄國萬里又無内應若以重兵相送則饋運難繼兵少則不足成功【少詩沼翻】如何可還斛律固請曰不煩重兵願給三百騎送至敇勒國人必欣然來迎跋乃遣單于前輔萬陵帥騎三百送之【騎奇寄翻單音蟬帥讀曰率下同】陵憚遠役至黑山【黑山在唐振武之北塞外即殺胡山也】殺斛律而還大檀亦遣使獻馬三千匹羊萬口于燕【使疏吏翻】 六月泰山太守劉研等帥流民七千餘家河西胡酋劉遮等帥部落萬餘家皆降於魏【酋慤由翻降戶江翻】 戊申魏主嗣如豺山宫丁亥還平城 樂都之潰也南涼安西將軍樊尼自西平奔告南涼王傉檀傉檀謂其衆曰今妻子皆為熾磐所虜退無所歸卿等能與吾藉乙弗之資取契汗以贖妻子乎【契欺訖翻汗音寒】乃引兵西衆多逃還傉檀遣鎮北將軍段苟追之苟亦不還於是將士皆散唯樊尼與中軍將軍紇勃後軍將軍洛肱散騎侍郎隂利鹿不去【散悉亶翻騎奇寄翻】傉檀曰蒙遜熾磐昔皆委質於吾【蒙遜稱臣於利鹿孤見一百一十二卷隆安五年熾磐父子歸利鹿孤見一百一十一卷四年質之日翻】今而歸之不亦鄙乎四海之廣無所容身何其痛也與其聚而同死不若分而或全樊尼吾長兄之子【樊尼蓋烏孤之子也長知兩翻】宗部所寄吾衆在北者戶垂一萬蒙遜方招懷士民存亡繼絶汝其從之紇勃洛肱亦與尼俱行【紇戶骨翻】吾年老矣所適不容寧見妻子而死遂歸于熾磐唯隂利鹿隨之傉檀謂利鹿曰吾親屬皆散卿何獨留利鹿曰臣老母在家非不思歸然委質為臣忠孝之道難以兩全臣不才不能為陛下泣血求救於鄰國【為于偽翻】敢離左右乎【離力智翻】傉檀歎曰知人固未易【易以䜴翻】大臣親戚皆棄我去今日忠義終始不虧者唯卿一人而已傉檀諸城皆降於熾磐【降戶江翻】獨尉賢政屯浩亹【浩亹音告門】固守不下熾磐遣人謂之曰樂都已潰卿妻子皆在吾所獨守一城將何為也賢政曰受涼王厚恩為國藩屏【屏必郢翻】雖知樂都已陷妻子為禽先歸獲賞後順受誅然不知主上存亡【主上謂傉檀也】未敢歸命妻子小事豈足動心若貪一時之利忘委付之重者大王亦安用之熾磐乃遣虎臺以手書諭之賢政曰汝為儲副不能盡節面縳於人棄父忘君墮萬世之業【墮讀曰隳】賢政義士豈效汝乎聞傉檀至左南乃降【闞駰十三州志曰左南城在金城白土縣東六十里晉志張氏置晉興郡左南縣屬焉是縣蓋亦張氏所置也】熾磐聞傉檀至遣使郊迎待以上賓之禮【使疏吏翻】秋七月熾磐以傉檀為驃騎大將軍賜爵左南公【驃匹妙翻騎奇寄翻】南涼文武依才銓叙歲餘熾磐使人鴆傉檀左右請解之傉檀曰吾病豈宜療邪遂死諡曰景王【載記曰秃髪烏孤至傉檀三世十九年而滅】虎臺亦為熾磐所殺傉檀子保周賀俱延子覆龍利鹿孤孫副周烏孤孫承鉢皆奔河西王蒙遜久之又奔魏魏以保周為張掖王覆龍為酒泉公賀西平公副周永平公承鉢昌松公魏主嗣愛賀之才謂曰卿之先與朕同源賜姓源氏【為源氏昌大於魏張本】 八月戊子魏主嗣遣馬邑侯陋孫使於秦辛丑遣謁者于什門使於燕悦力延使於柔然【使疏吏翻】于什門至和龍不肯入見曰大魏皇帝有詔須馮王出受然後敢入燕王跋使人牽逼令入什門見跋不拜跋使人按其項什門曰馮王拜受詔吾自以賓主致敬何苦見逼邪跋怒留什門不遣什門數衆辱之左右請殺之跋曰彼各為其主耳【數所角翻為于偽翻】乃幽執什門欲降之什門終不降【降戶江翻】久之衣冠弊壞略盡蟣蝨流溢跋遺之衣冠【遺于季翻】什門皆不受 魏主嗣以博士王諒為平南參軍使以平南將軍相州刺史尉太眞書與太尉裕相聞太眞古眞之弟也 九月丁巳朔日有食之 冬十月河南王熾磐復稱秦王置百官【熾磐嗣位自稱河南王今并南涼復稱秦王】 燕主跋與夏連和夏王勃勃遣御史中丞烏洛孤如燕涖盟【春秋之時列國釋仇通好兩君不及相見而盟必使其臣涖盟左傳陳五父如鄭涖盟是也杜預曰涖臨也】 十一月壬午魏主嗣遣使者巡行諸州【行下孟翻】校閱守宰資財【守式又翻】非家所齎悉簿為贓 西秦王熾磐立妃禿髮氏為后【妃傉檀之女也】 十二月丙戌朔柔然可汗大檀侵魏丙申魏主嗣北擊之大檀走遣奚斤等追之遇大雪士卒凍死及墮指者什二三 河内人司馬順宰自稱晉王魏人討之不克 燕遼西公素弗卒燕王跋比葬七臨之【古者大臣卒君三臨其喪比必寐翻及也】 是歲司馬國璠兄弟聚衆數百潜渡淮夜入廣陵城【祇傳曰自北徐州界渡淮璠孚袁翻】青州刺史檀祇領廣陵相國璠兵直上聽事【上時掌翻聽讀作廳】祇驚出將禦之被射傷而入【射而亦翻】謂左右曰賊乘闇得入欲掩我不備但擊五鼓彼懼曉必走矣左右如其言國璠兵果走【國璠之擾淮至是十年矣】 魏博士祭酒崔浩為魏主嗣講易及洪範【為于偽翻】嗣因問浩天文術數浩占決多驗由是有寵凡軍國密謀皆預之 夏王勃勃立夫人梁氏為王后子璝為太子【璝古回翻】封子延為陽平公昌為太原公倫為酒泉公定為平原公滿為河南公安為中山公<br />
<br />
  資治通鑑卷一百十六  <br>
   </div> 

<script src="/search/ajaxskft.js"> </script>
 <div class="clear"></div>
<br>
<br>
 <!-- a.d-->

 <!--
<div class="info_share">
</div> 
-->
 <!--info_share--></div>   <!-- end info_content-->
  </div> <!-- end l-->

<div class="r">   <!--r-->



<div class="sidebar"  style="margin-bottom:2px;">

 
<div class="sidebar_title">工具类大全</div>
<div class="sidebar_info">
<strong><a href="http://www.guoxuedashi.com/lsditu/" target="_blank">历史地图</a></strong>  
<a href="http://www.880114.com/" target="_blank">英语宝典</a>  
<a href="http://www.guoxuedashi.com/13jing/" target="_blank">十三经检索</a> 
<br><strong><a href="http://www.guoxuedashi.com/gjtsjc/" target="_blank">古今图书集成</a></strong> 
<a href="http://www.guoxuedashi.com/duilian/" target="_blank">对联大全</a> <strong><a href="http://www.guoxuedashi.com/xiangxingzi/" target="_blank">象形文字典</a></strong> 

<br><a href="http://www.guoxuedashi.com/zixing/yanbian/">字形演变</a>  <strong><a href="http://www.guoxuemi.com/hafo/" target="_blank">哈佛燕京中文善本特藏</a></strong>
<br><strong><a href="http://www.guoxuedashi.com/csfz/" target="_blank">丛书&方志检索器</a></strong> <a href="http://www.guoxuedashi.com/yqjyy/" target="_blank">一切经音义</a>  

<br><strong><a href="http://www.guoxuedashi.com/jiapu/" target="_blank">家谱族谱查询</a></strong>  <strong><a href="http://shufa.guoxuedashi.com/sfzitie/" target="_blank">书法字帖欣赏</a></strong> 
<br>

</div>
</div>


<div class="sidebar" style="margin-bottom:0px;">

<font style="font-size:22px;line-height:32px">QQ交流群9:489193090</font>


<div class="sidebar_title">手机APP 扫描或点击</div>
<div class="sidebar_info">
<table>
<tr>
	<td width=160><a href="http://m.guoxuedashi.com/app/" target="_blank"><img src="/img/gxds-sj.png" width="140"  border="0" alt="国学大师手机版"></a></td>
	<td>
<a href="http://www.guoxuedashi.com/download/" target="_blank">app软件下载专区</a><br>
<a href="http://www.guoxuedashi.com/download/gxds.php" target="_blank">《国学大师》下载</a><br>
<a href="http://www.guoxuedashi.com/download/kxzd.php" target="_blank">《汉字宝典》下载</a><br>
<a href="http://www.guoxuedashi.com/download/scqbd.php" target="_blank">《诗词曲宝典》下载</a><br>
<a href="http://www.guoxuedashi.com/SiKuQuanShu/skqs.php" target="_blank">《四库全书》下载</a><br>
</td>
</tr>
</table>

</div>
</div>


<div class="sidebar2">
<center>


</center>
</div>

<div class="sidebar"  style="margin-bottom:2px;">
<div class="sidebar_title">网站使用教程</div>
<div class="sidebar_info">
<a href="http://www.guoxuedashi.com/help/gjsearch.php" target="_blank">如何在国学大师网下载古籍?</a><br>
<a href="http://www.guoxuedashi.com/zidian/bujian/bjjc.php" target="_blank">如何使用部件查字法快速查字?</a><br>
<a href="http://www.guoxuedashi.com/search/sjc.php" target="_blank">如何在指定的书籍中全文检索?</a><br>
<a href="http://www.guoxuedashi.com/search/skjc.php" target="_blank">如何找到一句话在《四库全书》哪一页?</a><br>
</div>
</div>


<div class="sidebar">
<div class="sidebar_title">热门书籍</div>
<div class="sidebar_info">
<a href="/so.php?sokey=%E8%B5%84%E6%B2%BB%E9%80%9A%E9%89%B4&kt=1">资治通鉴</a> <a href="/24shi/"><strong>二十四史</strong></a>&nbsp; <a href="/a2694/">野史</a>&nbsp; <a href="/SiKuQuanShu/"><strong>四库全书</strong></a>&nbsp;<a href="http://www.guoxuedashi.com/SiKuQuanShu/fanti/">繁体</a>
<br><a href="/so.php?sokey=%E7%BA%A2%E6%A5%BC%E6%A2%A6&kt=1">红楼梦</a> <a href="/a/1858x/">三国演义</a> <a href="/a/1038k/">水浒传</a> <a href="/a/1046t/">西游记</a> <a href="/a/1914o/">封神演义</a>
<br>
<a href="http://www.guoxuedashi.com/so.php?sokeygx=%E4%B8%87%E6%9C%89%E6%96%87%E5%BA%93&submit=&kt=1">万有文库</a> <a href="/a/780t/">古文观止</a> <a href="/a/1024l/">文心雕龙</a> <a href="/a/1704n/">全唐诗</a> <a href="/a/1705h/">全宋词</a>
<br><a href="http://www.guoxuedashi.com/so.php?sokeygx=%E7%99%BE%E8%A1%B2%E6%9C%AC%E4%BA%8C%E5%8D%81%E5%9B%9B%E5%8F%B2&submit=&kt=1"><strong>百衲本二十四史</strong></a>  <a href="http://www.guoxuedashi.com/so.php?sokeygx=%E5%8F%A4%E4%BB%8A%E5%9B%BE%E4%B9%A6%E9%9B%86%E6%88%90&submit=&kt=1"><strong>古今图书集成</strong></a>
<br>

<a href="http://www.guoxuedashi.com/so.php?sokeygx=%E4%B8%9B%E4%B9%A6%E9%9B%86%E6%88%90&submit=&kt=1">丛书集成</a> 
<a href="http://www.guoxuedashi.com/so.php?sokeygx=%E5%9B%9B%E9%83%A8%E4%B8%9B%E5%88%8A&submit=&kt=1"><strong>四部丛刊</strong></a>  
<a href="http://www.guoxuedashi.com/so.php?sokeygx=%E8%AF%B4%E6%96%87%E8%A7%A3%E5%AD%97&submit=&kt=1">說文解字</a> <a href="http://www.guoxuedashi.com/so.php?sokeygx=%E5%85%A8%E4%B8%8A%E5%8F%A4&submit=&kt=1">三国六朝文</a>
<br><a href="http://www.guoxuedashi.com/so.php?sokeytm=%E6%97%A5%E6%9C%AC%E5%86%85%E9%98%81%E6%96%87%E5%BA%93&submit=&kt=1"><strong>日本内阁文库</strong></a> <a href="http://www.guoxuedashi.com/so.php?sokeytm=%E5%9B%BD%E5%9B%BE%E6%96%B9%E5%BF%97%E5%90%88%E9%9B%86&ka=100&submit=">国图方志合集</a> <a href="http://www.guoxuedashi.com/so.php?sokeytm=%E5%90%84%E5%9C%B0%E6%96%B9%E5%BF%97&submit=&kt=1"><strong>各地方志</strong></a>

</div>
</div>


<div class="sidebar2">
<center>

</center>
</div>
<div class="sidebar greenbar">
<div class="sidebar_title green">四库全书</div>
<div class="sidebar_info">

《四库全书》是中国古代最大的丛书,编撰于乾隆年间,由纪昀等360多位高官、学者编撰,3800多人抄写,费时十三年编成。丛书分经、史、子、集四部,故名四库。共有3500多种书,7.9万卷,3.6万册,约8亿字,基本上囊括了古代所有图书,故称“全书”。<a href="http://www.guoxuedashi.com/SiKuQuanShu/">详细>>
</a>

</div> 
</div>

</div>  <!--end r-->

</div>
<!-- 内容区END --> 

<!-- 页脚开始 -->
<div class="shh">

</div>

<div class="w1180" style="margin-top:8px;">
<center><script src="http://www.guoxuedashi.com/img/plus.php?id=3"></script></center>
</div>
<div class="w1180 foot">
<a href="/b/thanks.php">特别致谢</a> | <a href="javascript:window.external.AddFavorite(document.location.href,document.title);">收藏本站</a> | <a href="#">欢迎投稿</a> | <a href="http://www.guoxuedashi.com/forum/">意见建议</a> | <a href="http://www.guoxuemi.com/">国学迷</a> | <a href="http://www.shuowen.net/">说文网</a><script language="javascript" type="text/javascript" src="https://js.users.51.la/17753172.js"></script><br />
  Copyright &copy; 国学大师 古典图书集成 All Rights Reserved.<br>
  
  <span style="font-size:14px">免责声明:本站非营利性站点,以方便网友为主,仅供学习研究。<br>内容由热心网友提供和网上收集,不保留版权。若侵犯了您的权益,来信即刪。scp168@qq.com</span>
  <br />
ICP证:<a href="http://www.beian.miit.gov.cn/" target="_blank">鲁ICP备19060063号</a></div>
<!-- 页脚END --> 
<script src="http://www.guoxuedashi.com/img/plus.php?id=22"></script>
<script src="http://www.guoxuedashi.com/img/tongji.js"></script>

</body>
</html>
