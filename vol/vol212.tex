<!DOCTYPE html PUBLIC "-//W3C//DTD XHTML 1.0 Transitional//EN" "http://www.w3.org/TR/xhtml1/DTD/xhtml1-transitional.dtd">
<html xmlns="http://www.w3.org/1999/xhtml">
<head>
<meta http-equiv="Content-Type" content="text/html; charset=utf-8" />
<meta http-equiv="X-UA-Compatible" content="IE=Edge,chrome=1">
<title>資治通鑒_213-資治通鑑卷二百十二_213-資治通鑑卷二百十二</title>
<meta name="Keywords" content="資治通鑒_213-資治通鑑卷二百十二_213-資治通鑑卷二百十二">
<meta name="Description" content="資治通鑒_213-資治通鑑卷二百十二_213-資治通鑑卷二百十二">
<meta http-equiv="Cache-Control" content="no-transform" />
<meta http-equiv="Cache-Control" content="no-siteapp" />
<link href="/img/style.css" rel="stylesheet" type="text/css" />
<script src="/img/m.js?2020"></script> 
</head>
<body>
 <div class="ClassNavi">
<a  href="/24shi/">二十四史</a> | <a href="/SiKuQuanShu/">四库全书</a> | <a href="http://www.guoxuedashi.com/gjtsjc/"><font  color="#FF0000">古今图书集成</font></a> | <a href="/renwu/">历史人物</a> | <a href="/ShuoWenJieZi/"><font  color="#FF0000">说文解字</a></font> | <a href="/chengyu/">成语词典</a> | <a  target="_blank"  href="http://www.guoxuedashi.com/jgwhj/"><font  color="#FF0000">甲骨文合集</font></a> | <a href="/yzjwjc/"><font  color="#FF0000">殷周金文集成</font></a> | <a href="/xiangxingzi/"><font color="#0000FF">象形字典</font></a> | <a href="/13jing/"><font  color="#FF0000">十三经索引</font></a> | <a href="/zixing/"><font  color="#FF0000">字体转换器</font></a> | <a href="/zidian/xz/"><font color="#0000FF">篆书识别</font></a> | <a href="/jinfanyi/">近义反义词</a> | <a href="/duilian/">对联大全</a> | <a href="/jiapu/"><font  color="#0000FF">家谱族谱查询</font></a> | <a href="http://www.guoxuemi.com/hafo/" target="_blank" ><font color="#FF0000">哈佛古籍</font></a> 
</div>

 <!-- 头部导航开始 -->
<div class="w1180 head clearfix">
  <div class="head_logo l"><a title="国学大师官网" href="http://www.guoxuedashi.com" target="_blank"></a></div>
  <div class="head_sr l">
  <div id="head1">
  
  <a href="http://www.guoxuedashi.com/zidian/bujian/" target="_blank" ><img src="http://www.guoxuedashi.com/img/top1.gif" width="88" height="60" border="0" title="部件查字,支持20万汉字"></a>


<a href="http://www.guoxuedashi.com/help/yingpan.php" target="_blank"><img src="http://www.guoxuedashi.com/img/top230.gif" width="600" height="62" border="0" ></a>


  </div>
  <div id="head3"><a href="javascript:" onClick="javascript:window.external.AddFavorite(window.location.href,document.title);">添加收藏</a>
  <br><a href="/help/setie.php">搜索引擎</a>
  <br><a href="/help/zanzhu.php">赞助本站</a></div>
  <div id="head2">
 <a href="http://www.guoxuemi.com/" target="_blank"><img src="http://www.guoxuedashi.com/img/guoxuemi.gif" width="95" height="62" border="0" style="margin-left:2px;" title="国学迷"></a>
  

  </div>
</div>
  <div class="clear"></div>
  <div class="head_nav">
  <p><a href="/">首页</a> | <a href="/ShuKu/">国学书库</a> | <a href="/guji/">影印古籍</a> | <a href="/shici/">诗词宝典</a> | <a   href="/SiKuQuanShu/gxjx.php">精选</a> <b>|</b> <a href="/zidian/">汉语字典</a> | <a href="/hydcd/">汉语词典</a> | <a href="http://www.guoxuedashi.com/zidian/bujian/"><font  color="#CC0066">部件查字</font></a> | <a href="http://www.sfds.cn/"><font  color="#CC0066">书法大师</font></a> | <a href="/jgwhj/">甲骨文</a> <b>|</b> <a href="/b/4/"><font  color="#CC0066">解密</font></a> | <a href="/renwu/">历史人物</a> | <a href="/diangu/">历史典故</a> | <a href="/xingshi/">姓氏</a> | <a href="/minzu/">民族</a> <b>|</b> <a href="/mz/"><font  color="#CC0066">世界名著</font></a> | <a href="/download/">软件下载</a>
</p>
<p><a href="/b/"><font  color="#CC0066">历史</font></a> | <a href="http://skqs.guoxuedashi.com/" target="_blank">四库全书</a> |  <a href="http://www.guoxuedashi.com/search/" target="_blank"><font  color="#CC0066">全文检索</font></a> | <a href="http://www.guoxuedashi.com/shumu/">古籍书目</a> | <a   href="/24shi/">正史</a> <b>|</b> <a href="/chengyu/">成语词典</a> | <a href="/kangxi/" title="康熙字典">康熙字典</a> | <a href="/ShuoWenJieZi/">说文解字</a> | <a href="/zixing/yanbian/">字形演变</a> | <a href="/yzjwjc/">金 文</a> <b>|</b>  <a href="/shijian/nian-hao/">年号</a> | <a href="/diming/">历史地名</a> | <a href="/shijian/">历史事件</a> | <a href="/guanzhi/">官职</a> | <a href="/lishi/">知识</a> <b>|</b> <a href="/zhongyi/">中医中药</a> | <a href="http://www.guoxuedashi.com/forum/">留言反馈</a>
</p>
  </div>
</div>
<!-- 头部导航END --> 
<!-- 内容区开始 --> 
<div class="w1180 clearfix">
  <div class="info l">
   
<div class="clearfix" style="background:#f5faff;">
<script src='http://www.guoxuedashi.com/img/headersou.js'></script>

</div>
  <div class="info_tree"><a href="http://www.guoxuedashi.com">首页</a> > <a href="/SiKuQuanShu/fanti/">四库全书</a>
 > <h1>资治通鉴</h1> <!--         下载:【右键另存为】即可 --></div>
  <div class="info_content zj clearfix">
  
<div class="info_txt clearfix" id="show">
<center style="font-size:24px;">213-資治通鑑卷二百十二</center>
    資治通鑑卷二百十二  宋 司馬光 撰<br />
<br />
  胡三省 音注<br />
<br />
  唐紀二十八【起著雍敦牂盡㫋蒙赤奮若凡八年】<br />
<br />
  玄宗至道大聖大明孝皇帝上之下<br />
<br />
  開元六年春正月辛丑突厥毗伽可汗來請和許之【厥九勿翻伽求迦翻可從刋入聲汗音寒】 廣州吏民為宋璟立遺愛碑【去年宋璟自廣州入相為于偽翻璟居永翻】璟上言臣在州無它異迹今以臣光寵成彼諂諛欲革此風望自臣始請敕下禁止【上時掌翻下遐稼翻】上從之於是它州皆不敢立 辛酉敕禁惡錢【武德四年鑄開元通寶錢其後盗鑄漸起顯慶五年以惡錢多官為市之以一善錢售五惡錢民間藏惡錢以待禁弛乾封以後私錢犯法日蕃有以舟筏鑄于江中者詔所在納惡錢而姦亦不息武后時錢非穿宂及鐵錫銅液皆得用之熟銅排斗沙澀之錢皆售自是盗鑄蜂起吏莫能捕先天之際兩京錢益濫或鎔錫模錢須臾百十故禁之】重二銖四分以上乃得行歛人間惡錢鎔之更鑄如式錢【更工衡翻】於是京城紛然賣買殆絶宋璟蘇頲請出太府錢二萬緡置南北市以平價買百姓不售之物可充官用者及聽兩京百官豫假俸錢庶使良錢流布人間從之【頲他鼎翻俸芳用翻】 二月戊子移蔚州横野軍於山北【杜佑曰横野軍在蔚州東北百四十里去太原九百里此蓋指言開元所移軍之地蔚紆勿翻】屯兵三萬為九姓之援以拔曳固都督頡質略同羅都督毗伽末啜霫都督比言囘紇都督夷健頡利發僕固都督曳勒歌等各出騎兵為前後左右軍討擊大使【頡戶結翻啜陟劣翻霫而立翻騎奇寄翻使疏吏翻下同】皆受天兵軍節度【天兵軍在并州城中考異曰實録壬辰制大舉擊突厥五都督及拔悉密金山道總管處木昆執米啜堅昆都督骨篤禄毗伽契丹都督李失活奚都督李大酺及默啜之子右賢王默特勒逾輸等夷夏之師凡三十萬並取朔方道大總管王晙節度而於後俱不見出師勝敗按此年正月突厥請和帝有答詔而二月伐之恐無此事舊紀及王晙突厥傳皆無此月出兵事新突厥傳云默棘連遣使請和帝以不情答而不許俄下詔伐之以王晙統之期以八年並集稽落水上行兵□密不應前二年半先下詔蓋取實録附會舊傳耳】有所討捕量宜追集【量音良】無事各歸部落營生仍常加存撫 三月乙巳徵嵩山處士盧鴻入見【見賢遍翻】拜諫議大夫鴻固辭 【考異曰舊傳作盧鴻一本紀新傳皆作鴻按中岳真人劉君碑云盧鴻撰今從之】 天兵軍使張嘉貞入朝【朝直遥翻】有告其在軍奢僭及贓賄者按驗無狀上欲反坐吿者【反坐者以誣告人所得罪坐之】嘉貞奏曰今若罪之恐塞言路【塞悉則翻】使天下之事無由上逹願特赦之其人遂得减死上由是以嘉貞為忠有大用之意【為張嘉貞入相張本】 有薦山人范知璿文學者并獻其所為文【璿從宣翻】宋璟判之曰觀其良宰論頗涉佞諛【良宰論蓋稱美當時宰相】山人當極言讜議【讜音黨】豈宜偷合苟容文章若高自宜從選舉求試不可别奏夏四月戊子河南參軍鄭銑朱陽丞郭仙舟投匭獻<br />
<br />
  詩【河南參軍河南府參軍也唐制諸府州諸曹參軍之外又有參軍事掌出使贊導新志注曰武德初改行書佐曰行參軍尋又改曰參軍事朱陽漢弘農縣南界地後魏分置朱陽郡屬析州後周廢郡為縣隋屬弘農郡唐龍朔初屬商州萬歲通天二年度屬洛州匭居洧翻】敕曰觀其文理乃崇道法至於時用不切事情宜各從所好【好呼到翻】並罷官度為道士 五月辛亥以突騎施都督蘇禄為左羽林大將軍順國公充金方道經略大使【騎奇寄翻】 契丹王李失活卒癸巳以其弟娑固代之【契欺訖翻又音喫娑素何翻】 秋八月頒鄉飲酒禮于州縣令每歲十二月行之【唐鄉飲酒之禮刺史為主人先召致仕鄉有德者謀之賢者為賓其次為介其次為衆賓與之行禮縣則令為主人鄉之老人年六十以上有德望者一人為賓次一人為介又其次為三賓又其次為衆賓年六十者三豆七十者四豆八十者五豆九十者及主人皆六豆主賓介三賓衆賓既升即席工持瑟升自階就位鼔鹿鳴卒歌笙入立於堂下北面奏南陔乃間歌歌南有嘉魚笙崇丘乃合樂周南關雎召南鵲巢司正升自西階贊禮揚觶而戒之以忠孝之本主賓介以下皆再拜奠酬既畢乃行無筭爵無筭樂】 唐初州縣官俸皆令富戶掌錢出息以給之息至倍稱多破產者【唐初在京諸司官及天下官置公廨本錢以典史主之收贏十之七富戶幸免徭役貧者破產甚衆稱音尺證翻】祕書少監崔沔上言請計州縣官所得俸于百姓常賦之外微有所加以給之【時沔請計戶均出每丁加升尺以給之】從之 冬十一月辛卯車駕至西京 戊辰吐蕃奉表請和乞舅甥親署誓文【吐蕃以尚文成公主與唐為舅甥之國吐從暾入聲】又令彼此宰相皆著名于其上宋璟奏括州員外司馬李邕儀州司馬鄭勉並有才略文詞【先天元年避帝名改箕州為儀州】但性多異端好是非改變若全引進則咎悔必至若長棄捐則才用可惜請除渝硤二州刺史又奏大理卿元行冲素稱才行初用之時實允僉議當事之後頗非稱職請復以為左散騎常侍以李朝隱代之【好呼到翻行下孟翻稱尺證翻復扶又翻散悉亶翻騎奇寄翻朝直遥翻】陸象先閑於政體寛不容非請以為河南尹從之<br />
<br />
  七年春二月俱密王那羅延【俱密國治山中在吐火羅東北南臨黑河其王突厥延陁種】康王烏勒伽安王篤薩波提【杜佑曰康國在米國西南三百餘里漢康居國】皆上表言為大食所侵掠乞兵救援 敕太府及府縣出粟十萬石糶之【府謂京兆府縣謂京縣及畿縣也糶他弔翻】以歛人間惡錢送少府銷毁 三月乙卯以左武衛大將軍檢校内外閑廏使苑内營田使王毛仲行太僕卿【初唐以尚乘局掌内外閑廐之馬十二閑既置内外閑廄使專掌御馬因以尚乘局隸閑廐使苑内諸監本隸司農寺今亦隸苑内營田使】毛仲嚴察有幹力萬騎功臣閑廏官吏皆憚之【騎奇寄翻】苑内所收常豐溢上以為能故有寵雖有外第常居閑廏側内宅上或時不見則悄然若有所失宦官楊思勗高力士皆畏避之 渤海王大祚榮卒 【考異曰實録六月丁卯祚榮卒遣左監門率吳思謙攝鴻盧卿充使弔祭按此月丙辰已云祚榮卒盖六月方遣思謙弔祭耳】丙辰命其子武藝襲位 夏四月壬午開府儀同三司祁公王仁皎薨其子駙馬都尉守一請用竇孝諶例築墳高五丈二尺【竇孝諶上外祖也諶氏壬翻】上許之宋璟蘇頲固爭以為凖令一品墳高一丈九尺【高居號翻】其陪陵者高出三丈而已竇太尉墳議者頗譏其高大當時無人極言其失豈可今日復踵而為之【復扶又翻下蕃復同】昔太宗嫁女資送過于長公主魏徵進諫太宗既用其言文德皇后亦賞之【事見一百九十四卷太宗貞觀六年長知兩翻】豈若韋庶人崇其父墳號曰酆陵【事見二百八卷中宗景龍元年】以自速其禍乎夫以后父之尊欲高大其墳何足為難而臣等再三進言者蓋欲成中宫之美耳况今日所為當傳無窮永以為法可不慎乎上悦曰朕每欲正身率下况于妻子何敢私之然此乃人所難言卿能固守典禮以成朕美埀法將來誠所望也賜璟頲帛四百匹 五月己丑朔日有食之上素服以俟變徹樂减膳命中書門下察繫囚賑饑乏勸農功【賑津忍翻】辛卯宋璟等奏曰陛下勤恤人隱此誠蒼生之福然臣聞日食修德月食修刑親君子遠小人絶女謁除讒慝所謂修德也君子耻言浮于行【論語孔子曰君子恥其言而過其行遠于願翻慝吐得翻行下孟翻】苟推至誠而行之不必數下制書也【數所角翻下遐稼翻】 六月戊辰吐蕃復遣使請上親署誓文上不許曰昔歲誓約已定苟信不由衷亟誓何益【用左傳語意吐從暾入聲復扶又翻亟去吏翻】 秋閏七月右補闕盧履氷上言禮父在為母服周年則天皇后改服齊衰三年【事見二百二卷高宗上元元年為于偽翻齊音咨衰倉囘翻上時掌翻】請復其舊上下其議【下遐稼翻】左散騎常侍禇無量以履氷議為是諸人爭論連年不决八月辛卯敕自今五服並依喪服傳文【散悉亶翻騎奇寄翻傳直戀翻】然士大夫議論猶不息行之各從其意無量歎曰聖人豈不知母恩之厚乎厭降之禮【厭於叶翻】所以明尊卑異戎狄也俗情膚淺不知聖人之心一紊其制誰能正之【紊音問】 九月甲寅徙宋王憲為寧王【四年成器更名憲】上嘗從複道中見衛士食畢弃餘食于竇中怒欲杖殺之左右莫敢言憲從容諫曰陛下從複道中窺人過失而殺之臣恐人人不自安且陛下惡弃食於地者為食可以養人也【憲從千容翻惡烏路翻為于偽翻】今以餘食殺人無乃失其本乎上大悟蹶然起曰微兄幾至濫刑【幾居希翻】遽釋衛士是日上宴飲極歡自解紅玉帶并所乘馬以賜憲 冬十月辛卯上幸驪山温湯癸卯還宫【驪力知翻還從宣翻又音如字】 壬子冊拜突騎施蘇禄為忠順可汗【可從刊入聲汗音寒】 十一月壬申上以岐山令王仁琛【岐山縣隋置屬岐州琛丑林翻】藩邸故吏墨敕令與五品官宋璟奏故舊恩私則有大例除官資歷非無公道仁琛曏緣舊恩已獲優改今若再蒙超奬遂于諸人不類又是后族【王仁琛蓋仁皎羣從】須杜輿言【輿衆也】乞下吏部檢勘苟無負犯於格應留請依資稍優注擬從之選人宋元超于吏部自言侍中璟之叔父冀得優假璟聞之牒吏部云元超璟之三從叔【三從同高祖下遐嫁翻選須絹翻從才用翻】常在洛城不多參見【見賢遍翻】既不敢緣尊輒隱又不願以私害公向者無言自依大例既有聲聽事須矯枉請放【元超冀得饒假今乃不得留注所謂矯枉過正也】寧王憲奏選人薛嗣先請授微官事下中書門下【選須絹翻事下遐嫁翻】璟奏嗣先兩選齋郎雖非灼然應留以懿親之故固應微假官資在景龍中常有墨敕處分謂之斜封【處昌呂翻分扶問翻】自大明臨御兹事杜絶行一賞命一官必是緣功與才皆歷中書門下至公之道唯聖能行嗣先幸預姻戚不為屈法許臣等商量望付吏部知不出正敕從之【量音良】先是朝集使往往齎貨入京師【先悉薦翻】及春將還多遷官宋璟奏一切勒還以革其弊是歲置劒南節度使領益彭等二十五州<br />
<br />
  八年春正月丙辰左散騎常侍禇無量卒【按通鑑例惟公輔書薨偏王者公輔書卒今書禇無量卒以整比羣書未竟改命元行冲故書以始事 考異曰舊本紀正月甲子朔皇太子加元服壬申右散騎常侍禇無量卒按長歷正月甲寅朔甲子十一日也唐歷亦云壬申無量卒今從實録】辛酉命右散騎常侍元行冲整比羣書【比毗至翻】 侍中宋璟疾負罪而妄訴不已者悉付御史臺治之【治直之翻】謂中丞李謹度曰服不更訴者出之尚訴未已者且繫由是人多怨者會天旱有魃【魃蒲撥翻旱神也神異經曰南方有人長二三尺袒身目在項上走行如風其名曰魃所見之國大旱赤地千里一名旱母遇者得之投溷中即死】優人作魃狀戲於上前問魃何為出對曰奉相公處分【相息亮翻處昌呂翻分扶問翻】又問何故魃曰負寃者三百餘人相公悉以繫獄抑之故魃不得不出上心以為然時璟與中書侍郎同平章事蘇頲建議嚴禁惡錢江淮間惡錢尤甚璟以監察御史蕭隱之充使括惡錢【監工衘翻使疏吏翻】隱之嚴急煩擾怨嗟盈路上于是貶隱之官辛巳罷璟為開府儀同三司頲為禮部尚書 【考異曰唐歷云二十八日辛卯舊紀云己卯按是月無辛卯今從實録】以京兆尹源乾曜為黄門侍郎并州長史張嘉貞為中書侍郎並同平章事於是弛錢禁惡錢復行矣【復扶又翻】二月戊戍皇子敏卒追立為懷王【此懷王以州為國號】諡曰哀壬子敕以役莫重於軍府一為衛士六十乃免宜促其歲限使百姓更迭為之【更工衡翻】 夏四月丙午遣使賜烏長王骨咄王俱位王冊命三國皆在大食之西【烏長即烏萇又曰烏荼骨咄在鑊沙之東或曰阿咄羅治思助建城俱位或曰商瀰治阿賖䫻師多城在大雪山勃律河北地寒冬窟室咄當沒翻】大食欲誘之叛唐【誘音酉】三國不從故褒之五月辛酉復置十道按察使【罷按察見上卷五年】 丁卯以源乾曜為侍中張嘉貞為中書令乾曜上言形要之家多任京官使俊乂之士沈廢于外【沈持林翻】臣三子皆在京請出其二人上從之因下制稱乾曜之公命文武官効之於是出者百餘人張嘉貞吏事彊敏而剛躁自用【躁則到翻】中書舍人苗延嗣呂太一考功員外郎員嘉靜殿中侍御史崔訓皆嘉貞所引進常與之議政事四人頗招權時人語曰令公四俊苗呂崔員【員音運】 六月瀍穀漲溢漂溺幾二千人【溺奴狄翻幾居希翻 考異曰實録云漂居人四百餘家舊紀云漂没九百餘戶溺死八百餘人掌閑溺死者千一百餘人今從舊紀人數按舊紀掌閑之下有番兵二字】 突厥降戶僕固都督勺磨及跌部落散居受降城側【降戶江翻勺職略翻奚結翻跌徒結翻】朔方大使王晙言其隂引突厥謀陷軍城密奏請誅之誘勺磨等宴於受降城伏兵悉殺之河曲降戶殆盡拔曳固同羅諸部在大同横野軍之側者聞之皆忷懼【大同軍即大武軍武后大足元年更名杜佑曰在代州北三百里去并州八百餘里晙子峻翻誘音酉忷許拱翻】秋并州長史天兵節度大使張說引二十騎持節即其部落慰撫之【說讀曰悦騎奇寄翻】因宿其帳下副使李憲以虜情難信馳書止之說復書曰吾肉非黄羊必不畏食【北人謂麞為黄羊】血非野馬必不畏刺【非人家及廐牧所畜而自孳生於野者謂之野馬】士見危致命【論語載子張之言】此吾效死之秋也拔曳固同羅由是遂安 冬十月辛巳上行幸長春宫壬午畋于下邽 上禁約諸王不使與羣臣交結光禄少卿駙馬都尉裴虛已與岐王範遊宴仍私挾讖緯【讖楚譛翻緯于貴翻】戊子流虛已於新州離其公主【睿宗女霍國公主下嫁虚已舊志新州至京師五千五十二里】萬年尉劉庭琦太祝張諤【唐太常寺有太祝六人正九品上】數與範飲酒賦詩貶庭琦雅州司戶諤山茌丞【山茌縣漢晉屬泰山郡宋屬東太原郡隋廢入濟州長清縣武德元年分置山茌縣屬齊州數所角翻】然待範如故謂左右曰吾兄弟自無間但趨競之徒彊相託附耳【間古莧翻彊其兩翻】吾終不以此責兄弟也上嘗不豫薛王業妃弟内直郎韋賓【唐六典東宫有内直局内直郎二人掌符璽繖扇几案衣服之事職擬尚輦奉御】與殿中監皇甫恂私議休咎事覺賓杖死恂貶錦州刺史【武后垂拱二年以辰州麻陽縣地及開山洞置錦州】業與妃惶懼待罪上降階執業手曰吾若有心猜兄弟者天地實殛之即與之宴飲仍慰諭妃令復位 十一月乙卯上還京師辛未突厥寇甘凉等州【凉州西至甘州五百里 考異曰唐歷突厥寇凉州在九】<br />
<br />
  【月舊突厥傳云八年冬御史大夫王晙為朔方大總管奏請西征抜悉密東發奚契丹兩蕃期以明年秋初引朔方兵數道俱入掩突厥牙帳于稽落河上按王晙此月為幽州都督今從實録舊紀】敗河西節度使楊敬述【敗補邁翻】掠契苾部落而去【貞觀中契苾來降處其部落于凉州契欺訖翻苾毗必翻】先是朔方大總管王晙奏請西發拔悉密【拔悉密酋長姓阿史那氏盖亦突厥之種也居北庭先悉薦翻】東發奚契丹期以今秋掩毗伽牙帳于稽落水上【稽落水盖導源稽落山】毗伽聞之大懼暾欲谷曰不足畏也拔悉密在北庭與奚契丹相去絶遠勢不相及朔方兵計亦不能來此若必能來俟其垂至徙牙帳北行三日唐兵食盡自去矣且拔悉密輕而好利【輕牽正翻好呼到翻】得王晙之約必喜而先至晙與張嘉貞不相悦奏請多不相應必不敢出兵【史言在廷在邊之謀不叶為夷狄所窺】晙兵不出拔悉密獨至擊而取之勢甚易耳【易以䜴翻】既而拔悉密果發兵逼突厥牙帳而朔方及奚契丹兵不至拔悉密懼引退毗伽欲擊之暾欲谷曰此屬去家千里將死戰未可擊也不如以兵躡之去北庭二百里暾欲谷分兵間道先圍北庭【間古莧翻】因縱兵擊拔悉密大破之拔悉密衆潰走趨北庭不得入【趨逡喻翻】盡為突厥所虜暾欲谷引兵還出赤亭掠凉州羊馬楊敬述遣禆將盧公利判官元澄將兵邀擊之【將即亮翻】暾欲谷謂其衆曰吾乘勝而來敬述出兵破之必矣公利等至刪丹【刪丹縣漢屬張掖郡後漢晉屬西郡後魏曰山丹隋復曰刪丹屬甘州】與暾欲谷遇唐兵大敗公利澄脱身走毗伽由是大振盡有默啜之衆 契丹牙官可突干驍勇得衆心李娑固猜畏欲去之【驍堅堯翻娑素何翻去羌呂翻】是歲可突干舉兵擊娑固娑固敗奔營州營州都督許欽澹遣安東都護薛泰帥驍勇五百與奚王李大酺奉娑固以討之戰敗娑固李大酺皆為可突干所殺【帥讀曰率酺音蒲】生擒薛泰營州震恐許欽澹移軍入渝關可突干立娑固從父弟鬱干為主遣使請罪上赦可突干之罪以鬱干為松漠都督以李大酺之弟魯蘇為饒樂都督【使疏吏翻下同樂音洛】<br />
<br />
  九年春正月制削楊敬述官爵【以刪丹之敗也】以白衣檢校凉州都督仍充諸使【諸使謂節度支度營田等使也】 丙辰改蒲州為河中府置中都官僚一凖京兆河南 丙寅上幸驪山温湯乙亥還宫 監察御史宇文融上言天下戶口逃移巧偽甚衆請加檢括融㢸之玄孫也【宇文㢸見一百七十二卷陳宣帝太建七年監古銜翻上時掌翻㢸古弼字】源乾曜素愛其才贊成之二月乙酉敕有司議招集流移按詰巧偽之法以聞【詰去吉翻】 丙戌突厥毗伽復使來求和上賜書諭以曩昔國家與突厥和親華夷安逸甲兵休息國家買突厥羊馬突厥受國家繒帛彼此豐給自數十年來不復如舊正由默啜無信口和心叛數出盗兵寇抄邊鄙人怨神怒隕身喪元【復扶又翻下今復同數所角翻喪息浪翻元首也斬默啜事見上卷四年】吉凶之驗皆可汗所見今復蹈前迹掩襲甘凉隨遣使人更來求好國家如天之覆如海之容【好呼到翻覆敷又翻】但取來情不追往咎可汗果有誠心則共保遐福不然無煩使者徒爾往來若其侵邉亦有以待可汗其審圖之 丁亥制州縣逃亡戶口聽百日自首或于所在附籍或牒歸故鄉各從所欲過期不首【首式又翻】即加檢括謫徙邉州公私敢容庇者抵罪以宇文融充使括逃移戶口及籍外田所獲巧偽甚衆【使疏吏翻】遷兵部員外郎兼侍御史融奏置勸農判官十人【通典及新書並云二十九人通典且列其姓名】並攝御史分行天下其新附客戶免六年賦調【調徒弔翻】使者競為刻急州縣承風勞擾百姓苦之陽翟尉皇甫憬上疏言其狀【陽翟縣漢屬穎川郡晉屬河南郡後魏置陽翟郡隋廢郡為縣屬襄城郡唐初屬嵩州貞觀元年屬許州龍朔三年度屬洛州為畿縣憬居永翻】上方任融貶憬盈川尉州縣希旨務于獲多虚張其數或以實戶為客凡得戶八十餘萬田亦稱是【稱尺證翻】 蘭池州胡康待賓誘諸降戶同反夏四月攻陷六胡州【誘音酉降戶江翻高宗調露元年于靈夏南境以降突厥置魯州麗州含州塞州依州契州以唐人為刺史謂之六胡州長安二年併爲匡長二州神龍三年置蘭池都督府分六州為縣宋白曰六胡州在夏州德靜縣北 考異曰實録四月庚寅康待賓反命王晙討平之斬于都市五月丁巳既誅康待賓下詔云云壬寅叛胡康待賓偽稱葉護安慕容以叛七月癸酉王晙擒康待賓至京師腰斬之前後重複交錯相違今從舊紀】有衆七萬進逼夏州【夏戶雅翻】命朔方大總管王晙隴右節度使郭知運共討之 戊戌敕京官五品以上外官刺史四府上佐【四府謂京兆府河南府河中府太原府】各舉縣令一人視其政善惡為舉者賞罰 以太僕卿王毛仲為朔方道防禦討擊大使與王晙及天兵軍節度大使張說相知討康待賓 六月己卯罷中都復為蒲州【復扶又翻】蒲州刺史陸象先政尚寛簡吏民有罪多曉諭遣之州録事言于象先曰明公不施箠撻何以示威【唐上州置録事三人正九品上中下州各一人下州從九品下箠止可翻】象先曰人情不遠此屬豈不解吾言邪【解戶買翻曉也】必欲箠撻以示威當從汝始録事慙而退象先嘗謂人曰天下本無事但庸人擾之耳苟清其源何憂不治【治直吏翻】 秋七月巳酉王晙大破康待賓生擒之殺叛胡萬五千人辛酉集四夷酋長腰斬康待賓于西市【酋慈由翻長知兩翻】先是叛胡濳與党項通謀攻銀城連谷據其倉庾【後周置銀城縣後改曰銀城防貞觀四年以銀城屬銀州八年屬勝州又以隋連谷戍置連谷縣亦屬勝州杜佑曰銀城連谷皆漢圁隂縣地漢光禄塞在今縣北先悉薦翻党底朗翻】張說將步騎萬人出合河關掩擊大破之【嵐州合河縣北有合河關宋白曰合河縣城下有蔚汾水西與黄河合故曰合河趙珣聚米圖經合河關在府州南二百里將即亮翻騎奇寄翻】追至駱駝堰【堰於扇翻】党項乃更與胡戰胡衆潰西走入鐵建山說安集党項使復其居業討擊使阿史那獻以党項翻覆請并誅之說曰王者之師當伐叛柔服豈可殺已降邪因奏置麟州以鎮撫党項餘衆【分勝州銀城連谷置麟州又置新秦縣為麟州治所杜佑曰麟州漠新秦中地】 九月乙巳朔日有食之 康待賓之反也詔郭知運與王晙相知討之晙上言朔方兵自有餘力請敕知運還本軍【上時掌翻】未報知運已至由是與晙不協晙所招降者知運復縱兵擊之虜以晙為賣已由是復叛【降戶江翻復扶又翻】上以晙不能遂定羣胡丙午貶晙為梓州刺史【梓州漢郪廣漢氐道之地西魏梁末置新州隋改梓州王晙貶官未必離任也如婁師德以素羅汗山之敗貶亦此類】 丁未梁文獻公姚崇薨遺令佛以清淨慈悲為本而愚者寫經造像冀以求福昔周齊分據天下周則毁經像而修甲兵齊則崇塔廟而弛刑政一朝合戰齊㓕周興近者諸武諸韋造寺度人不可勝紀【勝音升】無救族誅汝曹勿效兒女子終身不寤追薦冥福道士見僧獲利效其所為尤不可延之於家當永為後法 癸亥以張說為兵部尚書同中書門下三品【考異曰朝野僉載曰說為并州刺史諂事王毛仲毛仲廵邉說於天兵軍大設酒肴恩敕忽降授兵部尚書同中書門下三品謝訖便抱毛仲起舞吮其靴鼻今不取】 冬十月河西隴右節度大使郭知運卒知運與同縣右衛副率王君㚟【率所律翻㚟丑略翻郭知運瓜州晉昌人王君㚟瓜州常樂人】皆以驍勇善騎射著名西陲為虜所憚【驍堅堯翻騎奇寄翻】時人謂之王郭㚟遂自知運麾下代為河西隴右節度使判凉州都督 十一月丙辰國子祭酒元行冲上羣書四録【甲部經録乙部史録丙部子録丁部集録 考異曰集賢注記在九年春今從唐歷統紀舊紀】凡書四萬八千一百六十九卷 庚午赦天下 十二月乙酉上幸驪山温湯壬辰還宫 是歲諸王為都督刺史者悉召還京師【開元二年有司請依故事出諸王刺外州】 新作蒲津橋鎔鐵為牛以繫絙【時鑄八牛牛下有山皆鐵也夾岸以維浮梁蒲津東岸即河東縣西岸即河西縣絙居登翻大索也】 安州别駕劉子玄卒子玄即知幾也避上嫌名以字行【劉子玄卒重史臣也例猶禇無量上名隆基知幾犯嫌名幾居希翻】著作郎吳兢撰則天實録言宋璟激張說使證魏元忠事【事今見二百七卷武后長安三年撰士免翻】說修史見之知兢所為謬曰劉五殊不相借【知幾第五唐人多以第行相呼】兢起對曰此乃兢所為史草具在不可使明公枉怨死者同僚皆失色其後說隂祈兢改數字兢終不許曰若徇公請則此史不為直筆何以取信于後 太史上言麟德歷浸踈【是歷行于高宗麟德二年上時掌翻】日食屢不効上命僧一行更造新歷【此所謂大衍歷也歐陽修曰自太初至麟德歷有二十三家與天雖近而未密也至一行密矣其倚數立法固無以易也後世雖有改作者皆依倣而已行下孟翻更工衡翻】率府兵曹梁令瓚造黄道遊儀以測候七政【唐東宫十率府各有兵曹參軍從九品上掌判句大朝會及皇太子出則從鹵簿而涖其儀一行更造新歷欲知黄道進退而太史無黄道儀令瓚以木為遊儀一行是之請更鑄銅鐵使黄道運行以追列舍之變因二分之中以立黄道交于奎軫之間二至升降各十四度黄道内施白道月環用究隂陽朓朒動合天運七政日月五星也】置朔方節度使領單于都護府夏鹽等六州定遠豐安二軍三受降城【單音蟬夏戶雅翻降戶江翻】<br />
<br />
  十年春正月丁巳上行幸東都以刑部尚書王志愔為西京留守【愔於今翻】 癸亥命有司收公廨錢以稅錢充百官俸【武德元年制京司及州縣官各給公廨田課其管種以供公私之費又有公廨園公廨地皆收其税以給百官廨古隘翻俸方用翻】 乙丑收職田【唐文武官有職分田一品十二頃二品十頃三品九頃四品七頃五品六頃六品四頃七品三頃五十畝八品二頃五十畝九品二頃皆給百里内之地諸州都督都護親王府官二品十二頃三品十頃四品八頃五品七頃六品五頃七品四頃八品三頃九品二頃五十畝鎮戍關津嶽瀆官五品五頃六品三頃五十畝七品三頃八品二頃九品一頃五十畝貞觀十一年以職田侵漁百姓詔給逃還貧戶視職田多少每畝給粟二斗謂之地租尋以水旱復罷之】畝率給倉粟二斗 二月戊寅上至東都 夏四月巳亥以張說兼知朔方軍節度使 五月伊汝水溢漂溺數千家【溺奴狄翻漢志伊水出弘農郡盧氏縣東北入洛汝水出弘農入淮史言伊汝溢而漂數千家既二水分流相去日益遠何至能漂流數千家此必于發源之地水溢而并流也被灾之家當在虢洛三州界】 閏月壬申張說如朔方廵邊 巳丑以餘姚縣主女慕容氏為燕郡公主妻契丹王鬱干【燕因肩翻妻七細翻】 六月丁巳博州河决命按察使蕭嵩等治之嵩梁明帝之孫也【後梁主巋諡明帝治直之翻】 巳巳制增太廟為九室遷中宗主還太廟【中宗遷别廟見上卷五年】 秋八月癸卯武彊令裴景仙【武彊漢河間之武隧縣也晉更名武彊唐屬冀州】坐贓五千匹事覺亡命上怒命集衆斬之大理卿李朝隱奏景仙贓皆乞取罪不至死【朝直遥翻】又其曾祖寂有建義大功載初中以非罪破家【據裴寂傳寂孫承先武后時為酷吏所殺】惟景仙獨存今為承嫡宜宥其死投之荒遠其辭略曰十代宥賢功實宜録【左傳晉祁奚請叔向曰社稷之固也猶將十世宥之】一門絶祀情或可哀制令杖殺朝隱又奏曰生殺之柄人主得專輕重有條臣下當守今若乞取得罪便處斬刑【處昌呂翻】後有枉法當科欲加何辟【辟毗亦翻】所以為國惜法期守律文【為于偽翻】非敢以法隨人曲矜仙命又曰若寂勲都棄仙罪特加則叔向之賢何足稱者若敖之鬼不其餒而【左傳楚令尹子文之言】上乃許之杖景仙一百流嶺南惡處【考異曰實録云初上令集衆殺之李朝隱執奏又下制云集衆决殺朝隱又奏乃流嶺南蓋本欲斬之也】安南賊帥梅叔焉等攻圍州縣遣驃騎將軍兼内侍<br />
<br />
  楊思勗討之【帥所類翻驃匹妙翻騎奇寄翻 考異曰舊紀云八月丙戍按八月庚子朔無丙戍思勗傳云首領梅玄成自稱黑帝與林邑真臘國通謀陷安南府今從本紀 周尹上神宗書作梅叔鸞】思勗募羣蠻子弟得兵十餘萬襲擊大破之斬叔焉積尸為京觀而還【觀古玩翻還從宣翻又如字】 初上之誅韋氏也王皇后頗預密謀及即位數年色衰愛弛武惠妃有寵隂懷傾奪之志后心不平時對上有不遜語上愈不悦密與祕書監姜皎謀以后無子廢之皎泄其言嗣滕王嶠【按新書滕王元嬰薨長子修琦嗣為長樂王垂拱中死詔獄神龍初以少子脩信子涉嗣無嗣滕王嶠也新書姜皎傳言皎泄禁中語為嗣濮王嶠所劾又按新書太宗子魏王泰得罪後封濮王薨子欣嗣武后時為酷吏所陷貶神龍初子嶠嗣王則嗣滕王嶠當作嗣濮王嶠明矣】后之妹夫也奏之上怒張嘉貞希旨構成其罪云皎妄談休咎甲戍杖皎六十流欽州弟吏部侍郎晦貶春州司馬【舊志春州京師東南六千四百四十八里】親黨坐流死者數人皎卒于道巳亥敕宗室外戚駙馬非至親毋得往還其卜相占候之人皆不得出入百官之家【相息亮翻】 巳卯夜左領軍兵曹權楚璧與其黨李齊損等作亂立楚璧兄子梁山為光帝詐稱襄王之子【景雲二年重茂改封襄王】擁左屯營兵數百人入宫城求留守王志愔不獲比曉【比必利翻】屯營兵自潰斬楚璧等傳首東都志愔驚怖而薨楚璧懷恩之姪【權懷恩為吏以嚴能稱怖普布翻】齊損㢠秀之子也【李㢠秀始見一百六卷武后神功元年㢠戶頃翻】壬午遣河南尹王怡如京師按問宣慰 癸未吐蕃圍小勃律王沒謹忙【小勃律在大勃律西北三百里去京師九千里而贏東少南三千里距吐蕃贊普牙】謹忙求救于北庭節度使張嵩曰勃律唐之西門勃律亡則西域皆為吐蕃矣嵩乃遣疏勒副使張思禮將蕃漢步騎四千救之【據新書張嵩即張孝嵩使疏吏翻將即亮翻騎奇寄翻】晝夜倍道與謹忙合擊吐蕃大破之斬獲數萬自是累歲吐蕃不敢犯邊 王怡治權楚璧獄連逮甚衆久之不决【治直之翻】上乃以開府儀同三司宋璟為西京留守璟至止誅同謀數人餘皆奏原之 康待賓餘黨康願子反自稱可汗張說發兵追討擒之其黨悉平徙河曲六州殘胡五萬餘口於許汝唐鄧仙豫等州【貞觀八年改伊州襄城郡為汝州唐州漢南陽郡東界比陽湖陽平氏之地後魏於比陽置東荆州後改為昌州又改為淮州隋改為顯州武德五年以郡有唐城山改為唐州開元三年以汝州之葉襄城唐州之方城豫州之西平許州之舞陽置仙州】空河南朔方千里之地先是緣邊戍兵常六十餘萬【先悉薦翻】說以時無彊寇奏罷二十餘萬使還農上以為疑說曰臣久在疆場具知其情將帥苟以自衛及役使營私而已【場音亦將即亮翻帥所類翻】若禦敵制勝不必多擁冗卒以妨農務陛下若以為疑臣請以闔門百口保之上乃從之初諸衛府兵自成丁從軍六十而免其家又不免雜徭浸以貧弱逃亡略盡百姓苦之張說建議請召募壯士充宿衛不問色役優為之制逋逃者必爭出應募上從之旬日得精兵十三萬分隸諸衛更番上下【更工衡翻上時掌翻】兵農之分從此始矣【史言唐養兵之弊始于張說】 冬十月癸丑復以乾元殿為明堂【以東都明堂復為乾元殿見上卷五年復扶又翻】 甲寅上幸夀安興泰宫【夀安古新安九曲之地後魏置甘棠縣隋仁夀四年改為夀安縣屬洛州】獵于上宜川庚申還宫上欲耀兵北邊丁卯以秦州都督張守潔等為諸衛<br />
<br />
  將軍 十一月乙未初令宰相共食實封三百戶【唐會要曰舊制凡有功之臣賜實封者皆以課戶充準戶數州縣與國官邑官執帳供其租調各凖配租調遠近州縣官司收其脚直然後附國邑官司其丁凖此入國邑者收其庸】 前廣州都督裴伷先下獄【下遐嫁翻伷與胄同】上與宰相議其罪張嘉貞請杖之張說曰臣聞刑不上大夫【記曲禮之言上時掌翻】為其近于君且所以養亷耻也【為于偽翻近其靳翻】故士可殺不可辱【記儒行之言】臣曏廵北邊聞杖姜皎于朝堂【朝直遥翻】皎官登三品亦有微功有罪應死則死應流則流奈何輕加笞辱以皂隸待之【笞丑之翻皁昨早翻】姜皎事往不可復追伷先據狀當流豈可復蹈前失【復扶又翻】上深然之嘉貞不悦退謂說曰何論事之深也說曰宰相時來則為之若國之大臣皆可笞辱但恐行及吾輩吾此言非為伷先乃為天下士君子也【為于偽翻下為農同】嘉貞無以應 十二月庚子以十姓可汗阿史那懷道女為交河公主【武后長安四年冊懷道為十姓可汗】嫁突騎施可汗蘇禄 上將幸晉陽因還長安張說言于上曰汾隂脽上有漢家后土祠【立后土祠見二十卷漢武帝元鼎四年脽音誰】其禮久廢陛下宜因廵幸修之為農祈穀上從之 上女永穆公主將下嫁【永穆公主下嫁王繇】敕資送如太平公主故事僧一行諫曰武后惟太平一女故資送特厚卒以驕敗【太平公主始嫁薛紹而敗于開元之初卒子恤翻】奈何為法上遽止之<br />
<br />
  十一年春正月巳巳車駕自東都北廵庚辰至潞州給復五年【上嘗為潞州别駕故也復方目翻】辛卯至并州置北都以并州為太原府刺史為尹二月戊申還至晉州 張說與張嘉貞不平會嘉貞弟金吾將軍嘉祐贓發說勸嘉貞素服待罪于外巳酉左遷嘉貞幽州刺史 壬子祭后土于汾隂乙卯貶平遥令王同慶為贑尉【平遥即漢太原郡平陶縣後魏避國諱改平陶為平遥周隋屬介州唐屬汾州贑音紺】坐廣為儲偫煩擾百姓也【偫直里翻】 癸亥以張說兼中書令 巳巳罷天兵大武等軍以大同軍為太原以北節度使領太原遼石嵐汾代忻朔蔚雲十州【武德三年分并州之樂平遼山平城石艾置遼州八年曰箕州先天元年避上名改曰儀州至中和三年方復曰遼州此以後來一定州名書之嵐盧含翻蔚紆勿翻】 三月庚午車駕至京師 夏四月甲子以吏部尚書王晙為兵部尚書同中書門下三品 五月巳丑以王晙兼朔方軍節度大使廵河西隴右河東河北諸軍 上置麗正書院聚文學之士【漢魏以來有祕書之職梁於文德殿内藏聚羣書北齊有文林館學士後周有麟趾殿學士皆掌著述隋寫羣書正副二本藏于宫中其餘以實秘書外閣煬帝于東都觀文殿東西廂貯書自漢延嘉至隋皆秘書掌圖籍而禁中之書時或有焉太宗在藩置學士十八人其後弘文崇文二館皆有學士開元五年乾元殿寫四部書置乾元院使有刋正官四人知書官八人分掌四庫書六年更號麗正修書院置使及檢校官改修書官為麗正殿學士八年加文學直又加修撰校理判正校勘官十一年置麗正院修書學士十三年改麗正修書院為集賢殿書院五品以上為學士六品以下為直學士宰相一人為學士知院事常侍一人為副知院事又置判院一人押院中使一人又置集賢院侍講學士侍讀直學士其後又增修撰官校理官侍制官留院官知檢討官文學直之類】祕書監徐堅太常博士會稽賀知章監察御史鼔城趙冬曦等【會稽縣帶越州鼔城縣漢臨平下曲陽兩縣之地隋分槀城于下曲陽故城東五里置昔陽縣尋改為鼔城唐屬定州會古外翻監右衘翻】或修書或侍講以張說為修書使以總之有司供給優厚中書舍人洛陽陸堅以為此屬無益于國徒為糜費欲悉奏罷之 【考異曰舊傳作徐堅今從集賢注記】張說曰自古帝王于國家無事之時莫不崇宫室廣聲色今天子獨延禮文儒發揮典籍所益者大所損者微陸子之言何不逹也上聞之重說而薄堅 秋八月癸卯敕前令檢括逃人慮成煩擾天下大同宜各從所樂【樂音洛】令所在州縣安集遂其生業 戊申追尊宣皇帝廟號獻祖光皇帝廟號懿祖【宣皇帝諱熙凉武昭王暠之曾孫凉王歆之孫弘農太守重耳之子也光皇帝諱天賜宣皇帝長子也】祔于太廟九室 先是吐谷渾畏吐蕃之彊附之者數年【先悉薦翻吐從暾入聲】九月壬申帥衆詣沙州降【帥讀曰率降戶江翻】河西節度使張敬忠撫納之 冬十月丁酉上幸驪山作温泉宫【雍録曰驪山温湯在臨潼縣南一百五十步直驪山之西北十道志曰泉有三所其一處即皇堂石井後周宇文護所造隋文帝又修屋宇并植松柏千餘株貞觀十八年詔閻立本營建宫殿御賜名湯泉宫是年更名温泉宫而改作之】甲寅還宫 十一月禮儀使張說等奏以高祖配昊天上帝罷三祖並配之禮【此因郊祀置禮儀使也武德初定令圓丘以景帝配明堂以元帝配貞觀奉高祖配圓丘永徽二年又奉太宗配明堂垂拱初用元萬頃議奉高宗配圓丘自是郊祀之禮三祖並配三祖謂高祖太宗高宗也使疏吏翻】戊寅上祀南郊赦天下 【考異曰實録癸酉日長至戊寅祀南郊唐歷戊寅冬至祀南郊按長歷去年閏五月來年閏十二月唐歷近是】 戊子命尚書左丞蕭嵩與京兆蒲同岐華州長官選府兵及白丁一十二萬謂之長從宿衛【華戶化翻長知兩翻】一年兩番州縣毋得雜役使 十二月甲午上幸鳳泉湯戊申還宫 庚申兵部尚書同中書門下三品王晙坐黨引疎族貶蘄州刺史【舊志蘄州至京師二千五百六十里蘄渠希翻考異曰舊傳云上親郊祀追晙赴京以會大禮晙以時屬氷壯恐虜騎乘隙入寇表辭不赴手敕慰勉仍賜】<br />
<br />
  【衣一副會許州刺史王喬家奴告喬與晙濳謀搆逆敕侍中源乾曜中書令張說鞫其狀晙既無反狀乃以違詔追不到罪之今從實録】 是歲張說奏改政事堂曰中書門下列五房于其後分掌庶政【舊制宰相常于門下省議事謂之政事堂永淳元年中書令裴炎以中書執政事筆遂移政事堂于中書省至是說改政事堂為中書門下其政事印改為中書門下之印五房一曰吏房二曰樞機房三曰兵房四曰戶房五曰刑禮房】 初監察御史濮陽杜暹因按事至突騎施【監古衘翻濮博木翻暹息亷翻】突騎施饋之金暹固辭左右曰君寄身異域不宜逆其情乃受之埋于幕下出境移牒令取之虜大驚度磧追之不及及安西都護闕或薦暹往使安西人服其清慎【磧七迹翻使疏吏翻】時暹自給事中居母憂<br />
<br />
  十二年春三月甲子起暹為安西副大都護磧西節度等使 神龍初追復澤王上金官爵【上金死見二百四卷武后天授元年】求得庶子義珣於嶺南紹其故封許王素節之子瓘利其爵邑與弟璆謀【璆音求】使人告義珣非上金子妄冒襲封復流嶺南以璆繼上金後為嗣澤王至是玉真公主表義珣實上金子為瓘兄弟所擯夏四月庚子復立義珣為嗣澤王削璆爵貶瓘鄂州别駕【舊志鄂州京師東南二千九百四十八里】壬寅敕宗室旁繼為嗣王者並令歸宗【復扶又翻考異曰舊紀在癸卯今從實録】 壬子命太史監南宫說等【唐太史局屬秘書省景龍二年改大史局為太史監令名不改不隸祕書開元二年又改令為監說讀曰悦】於河南北平地測日晷及極星夏至日中立八尺之表同時候之陽城晷長一尺四寸八分弱【陽城縣前漢屬潁川郡後漢屬河南郡後魏置陽城郡隋置嵩州貞觀三年廢嵩州以縣屬洛州武后登封元年改曰告成中宗神龍元年復故晷居洧翻長直亮翻】夜視北極出地高三十四度十分度之四【高古號翻】浚儀岳臺晷長一尺五寸微強【項安世曰按日行黄道每歲有差地中亦當隨之故測日景以求地中周在洛邑漢在潁川陽城唐在汴州浚儀也長直亮翻 考異曰新志云浚儀岳臺晷尺五寸三分今從僧一行大衍歷議及舊志】極高三十四度八分南至朗州晷長七寸七分極高二十九度半北至蔚州晷長二尺二寸九分極高四十度南北相距三千六百八十八里九十步晷差一尺五寸二分極差十度半又南至交州晷出表南三寸三分八月海中南望老人星下衆星粲然皆古所未名大率去南極二十度以上星皆見【見賢遍翻温公作通鑑不特紀治亂之迹而已至于禮樂歷數天文地理尤致其詳讀通鑑者如飲河之鼠各充其量而已】 五月丁亥停諸道按察使【八年復置諸道按察使使疏吏翻】六月壬辰制聽逃戶自首【首式又翻】闢所在閒田隨宜收税毋得差科征役租庸一皆蠲免仍以兵部員外郎兼侍御史宇文融為勸農使廵行州縣【行下孟翻】與吏民議定賦役 上以山東旱命臺閣名臣以補刺史壬午以黄門侍郎王丘中書侍郎長安崔沔【沔彌兖翻】禮部侍郎知制誥韓休等五人出為刺史丘同皎之從父兄子【王同皎預誅二張死於武三思之手從才用翻】休大敏之孫也【按舊書韓休傳休伯父大敏則天初以雪反者賜死休父曰大智】初張說引崔沔為中書侍郎故事承宣制皆出宰相侍郎署位而已【承宣制者承宣及承制也】沔曰設官分職上下相維各申所見事乃無失侍郎令之貳也豈得拱默而已由是遇事多所異同說不悦故因是出之 秋七月突厥可汗遣其臣哥解頡利發來求昏 溪州蠻覃行璋反【溪州漢沅陵零陵二縣地梁分置大鄉縣舊置辰州武后天授二年分置溪州覃音徒含翻姓也姓譜梁有東寧州刺史覃元亮】以監門衛大將軍楊思勗為黔中道招討使將兵擊之【監古衘翻黔音琴使疏吏翻將即亮翻】癸亥思勗生擒行璋斬首三萬級而歸加思勗輔國大將軍俸禄防閤皆依品級【貞觀初百官得上考者給禄一季未幾得上下考者給禄一年出使者禀其家新至官者計日給糧中書舍人高季輔言外官卑品貧匱宜給禄養親自後以地租春秋給京官歲凡五十萬一千五百餘斛外官降京官一等一品以五十石為一等二品三品以三十石為一等四品五品以二十石為一等六品七品以五石為一等八品九品以二石五斗為一等典粟則以鹽為禄職事官又有防閤庶僕一品防閤九十六人二品七十二人三品四十八人四品三十二人五品二十四人六品庶僕十五人七品四人八品三人九品二人外官以州府縣上中下為差又按唐六典輔國大將軍勲階正二品唐制宦官不得登三品今思勗階二品矣宋白曰唐制凡京師文武職官皆有防閤州縣官僚皆有白直】赦行璋以為洵水府别駕【唐制商州有洵水府又按唐制諸府無别駕各有别將一人上府正七品下中府從七品上下府從七品下别駕當為别將洵須倫翻】 姜皎既得罪王皇后愈憂畏不安然待下有恩故無隨而譖之者上猶豫不决者累歲后兄太子少保守一以后無子使僧明悟為后祭南北斗【為于偽翻】剖霹木書天地字及上名合而佩之【霹木者霹所震之木今為張道陵之術者用霹靂木為印云有雷氣可以鎮服鬼物】祝曰佩此有子當如則天皇后事覺己卯廢為庶人移别室安置貶守一潭州别駕中路賜死戶部尚書張嘉貞坐與守一交通貶台州刺史【舊志台州京師東南四千一百七十七里】 八月丙申突厥哥解頡利發還其國以其使者輕禮數不備未許昏【使疏吏翻下同】 己亥以宇文融為御史中丞融乘驛周流天下事無大小諸州先牒上勸農使【上時掌翻】後申中書【句斷】省司亦待融指撝然後處决【省司謂尚書都省左右司主者也處昌呂翻】時上將大攘四夷急于用度州縣畏融多張虛數凡得客戶八十餘萬田亦稱是【融獻策括籍外羨田逃戶自占者給復五年每丁税錢千五百州縣希旨以正田為羨編戶為客稱尺證翻】歲終增緡錢數百萬【緡眉巾翻】悉進入宫由是有寵議者多言煩擾不利百姓上亦令集百寮於尚書省議之公卿已下畏融恩勢不敢立異惟戶部侍郎楊瑒獨抗議以為括客免税不利居人徵籍外田税使百姓困弊所得不補所失未幾瑒出為華州刺史【瑒雉杏翻又音暢幾居豈翻】 壬寅以開府儀同三司宋璟為西京留守 冬十月丁酉謝䫻王特勒遣使入奏【謝䫻國居吐火羅西南或曰漕矩吒或曰漕矩顯慶時曰訶逹羅支武后改曰謝䫻東距罽賓四百里南天竺西波斯䫻于筆翻】稱去年五月金城公主遣使詣箇失密國【箇失密或曰迦濕彌邏北距勃律五百里】云欲走歸汝箇失密王從臣國王借兵共拒吐蕃王遣臣入取進止上以為然賜帛遣之 廢后王氏卒後宫思慕后不已上亦悔之 十一月庚午上幸東都戊寅至東都 辛巳司徒申王撝薨贈諡惠莊太子 羣臣屢上表請封禪【上時掌翻】閏月丁卯制以明年十一月十日有事于泰山時張說首建封禪之議而源乾曜不欲為之由是與說不平 是歲契丹王李鬱干卒弟吐干襲位【卒子恤翻吐從暾入聲】十三年春二月庚申以御史中丞宇文融兼戶部侍郎制以所得客戶税錢均充所在常平倉本又委使司與州縣議作勸農社【使司勸農使司也使疏吏翻】使貧富相恤耕耘以時 乙亥更命長從宿衛之士曰彍騎【彍虚郭翻引滿曰彍】分隸十二衛總十二萬人為六番 上自選諸司長官有聲望者大理卿源光裕尚書左丞楊承令兵部侍郎寇泚等十一人為刺史命宰相諸王及諸司長官臺郎御史【諸司長官省寺監之長也臺郎謂尚書郎先是改尚書為中臺臺郎及御史則三省官必皆集也長知兩翻泚且禮翻又音此】餞于洛濱供張甚盛【供居用翻張知亮翻】賜以御膳太常具樂【御膳尚食奉御所掌天子日供之常膳太常具樂者使太常為之具樂耳若盡具太常之樂則雅樂鼔吹文武二舞及十部樂恐非宴餞之所得備也】内坊歌妓【内坊内敎坊也即開元二年選置宜春院之妓女妓渠綺翻】上自書十韻詩賜之光裕乾曜之從孫也【從才用翻】 三月甲午太子嗣謙更名鴻徙郯王嗣直為慶王更名潭陜王嗣昇為忠王更名浚鄫王嗣真為棣王更名洽【讀通鑑至此可以知前此嗣直之誤為嗣真矣更工衡翻下同】鄂王嗣初更名溳鄄王嗣玄為榮王更名滉又立子琚為光王濰為儀王澐為潁王【溳圭淵翻鄄吉椽翻澐音云】澤為永王清為夀王洄為延王沭為盛王【沭食聿翻】溢為濟王【濟子禮翻】 丙申御史大夫程行湛奏周朝酷吏【朝直遥翻】來俊臣等二十三人情狀尤重子孫請皆禁錮傅遊藝等四人差輕子孫不聼近任從之 汾州刺史楊承令不欲外補【舊志汾州去京師一千二百六里】意怏怏自言吾出守有由【怏於兩翻守手又翻】上聞之怒壬寅貶睦州别駕 張說草封禪儀獻之夏四月丙辰上與中書門下及禮官學士宴于集仙殿上曰仙者憑虚之論朕所不取賢者濟理之具朕今與卿曹合宴宜更名曰集賢殿【唐六典洛陽宫南面三門中曰應天左曰興敎右曰光政光政之内曰廣運其北曰明福明福之西曰崇賢門其内曰集賢殿】其書院官五品以上為學士六品以下為直學士以張說知院事右散騎常侍徐堅副之上欲以說為大學士說固辭而止 說以大駕東廵恐突厥乘閒入寇【間古莧翻】議加兵守邊召兵部郎中裴光庭謀之光庭曰封禪者告成功也今將升中于天【記曰因名山升中于天注云謂封泰山也】而戎狄是懼非所以昭盛德也說曰然則若之何光庭曰四夷之中突厥為大比屢求和親而朝廷羈縻未决許也【比毗至翻】今遣一使徵其大臣從封泰山【使疏吏翻】彼必欣然承命突厥來則戎狄君長無不皆來【長知兩翻】可以偃旗卧皷高枕有餘矣【枕職任翻】說曰善說所不及即奏行之光庭行儉之子也【裴行儉事高宗典選有識鑒為將著功名】上遣中書直省袁振【以他官直中書省謂之直省今之直省吏職也】攝鴻臚卿諭旨於突厥【臚陵如翻】小殺與闕特勒暾欲谷環坐帳中置酒謂振曰吐蕃狗種【西戎古曰犬戎故謂吐蕃為狗種】奚契丹本突厥奴也【夷言奴猶華言臣也】皆得尚主突厥前後求昏獨不許何也且吾亦知入蕃公主皆非天子女今豈問真偽但屢請不獲愧見諸蕃耳振許為之奏請小殺乃使其大臣阿史德頡利發入貢因扈從東廵【為于偽翻從才用翻】 五月庚寅妖賊劉定高帥衆夜犯通洛門【妖于喬翻帥讀曰率】悉捕斬之 秋八月張說議封禪儀請以睿宗配皇地祇從之 九月丙戌上謂宰臣曰春秋不書祥瑞惟記有年敕自今州縣毋得更奏祥瑞 冬十月癸丑作水運渾天成上具列宿注水激輪令其自轉晝夜一周別置二輪絡在天外綴以日月逆天而行淹速合度【每天西旋一周日東行一度月行十三度十九分度之七二十九轉而日月會三百六十五轉而日周天孔頴達曰天之晝夜以日出入為分人之晝夜以昏明為限日未出以前二刻半為明日入後二刻半為昏損夜五刻以禆于晝則晝多于夜復校五刻古今歷術與太史所候皆云夏至之晝六十五刻夜三十五刻冬至之晝四十五刻夜五十五刻春分秋分之晝五十五刻夜四十五刻此其不易之法也然今太史細侯之法則較常法半刻也從春分至夏至晝漸長增九刻半夏至至于秋分所减亦如之從秋分至于冬至晝漸短减十刻半從冬至至于春分其增亦如之又于每氣之間增减刻數有多有少不可通以為率漢初未能審知率九日增减一刻和帝時霍融始請改之】置木匱為地平令儀半在地下又立二木人每刻擊鼔每辰擊鐘機械皆藏匱中 辛酉車駕發東都百官貴戚四夷酋長從行【酋慈由翻長知兩翻】每置頓數十里中人畜被野【畜許救翻被皮義翻】有司輦載供具之物數百里不絶【司馬法曰夏后氏謂輦曰余車殷曰胡奴車周曰輜輦輦一斧一鑿一梩一鋤周輦加二版二築又曰夏后氏二十人而輦殷十八人而輦周十五人而輦賈公彦曰輦以其束載輜重余按司馬法及賈公彦所云皆言行軍之輦此所謂輦載兼凡器物而言】十一月丙戌至泰山下御馬登山 【考異曰實録唐歷統紀皆云備法駕登泰山開天傳信記云上將封泰山益州進白騾上親乘之不知登降之倦纔下山無疾而殪諡曰白騾將軍按泰山非法駕可登白騾近怪今從舊志】留從官於谷口【從才用翻】獨與宰相及祠官俱登儀衛環列於山下百餘里上問禮部侍郎賀知章曰前代玉牒之文何故祕之對曰或密求神仙故不欲人見上曰吾為蒼生祈福耳【為于偽翻】乃出玉牒宣示羣臣庚寅上祀昊天上帝於山上羣臣祀五帝百神於山下之壇其餘倣乾封故事【事見一百一卷高宗乾封元年】辛卯祭皇地祇于社首壬辰上御帳殿受朝覲【野次連幄以為殿因謂之帳殿朝直遥翻】赦天下封泰山神為天齊王禮秩加三公一等【古制四岳眡三公】張說多引兩省吏【兩省中書省門下省也】及以所親攝官登山禮畢推恩往往加階超入五品而不及百官中書舍人張九齡諫不聽又扈從士卒但加勲而無賜物【從才用翻勲勲級也】由是中外怨之 初隋末國馬皆為盗賊及戎狄所掠唐初纔得牝牡三千匹於赤岸澤徙之隴右命太僕張萬歲掌之 【考異曰統紀云萬歲三世典羣牧恩信行隴右故隴右人謂馬歲爲齒爲張氏諱也按公羊傳晉獻公謂荀息曰吾馬之齒則已長矣然則謂馬歲為齒有自來矣】萬歲善于其職自貞觀至麟德馬蕃息及七十萬匹【蕃音煩】分為八坊四十八監各置使以領之【唐制凡馬五千匹為上監三千匹以上為中監一千匹以上為下監麟德中置八使分總監坊秦蘭原渭四州及河曲之地凡監四十八南使有監十五西使有監十六北使有監七鹽州使有監八嵐州使有監二自京師西屬隴右有七馬坊置隴右三使領之歐陽脩曰置八坊豳岐涇寜間地廣千里一曰保樂二曰甘露三曰南普閏四曰北普閏五曰岐陽六曰太平七曰宜禄八曰安定八坊之田千二百三十頃募民耕之以給芻秣八坊之馬為四十八監而馬多地狹不能容又析八監列置河西豐曠之野】是時天下以一縑易一馬垂拱以後馬濳耗太半上初即位牧馬有二十四萬匹以太僕卿王毛仲為内外閑廐使少卿張景順副之至是有馬四十三萬匹牛羊稱是【使疏吏翻下同稱尺證翻】上之東封以牧馬數萬匹從【從才用翻下宴從同】色别為羣望之如雲錦上嘉毛仲之功癸巳加毛仲開府儀同三司甲午車駕發泰山庚申幸孔子宅致祭上還至宋州宴從官于樓上刺史寇泚預焉酒酣上謂張說曰曏者屢遣使臣分廵諸道察吏善惡今因封禪歷諸州乃知使臣負我多矣懷州刺史王丘餼牽之外一無它獻魏州刺史崔沔供張無錦繡示我以儉濟州刺史裴耀卿表數百言莫非規諫【懷魏二州在河北濟州治鉅野上行幸泰山往還皆得迎候車駕餼許氣翻供居用翻張知亮翻濟子禮翻】且曰人或重擾則不足以告成朕常寘之坐隅【重直用翻坐徂卧翻】且以戒左右如三人者不勞人以市恩真良吏矣顧謂寇泚曰比亦屢有以酒饌不豐訴于朕者【比毗至翻饌雛戀翻又雛皖翻】知卿不借譽於左右也自舉酒賜之宰臣帥羣臣起賀樓上皆稱萬歲【譽音余帥讀曰率】由是以丘為尚書左丞沔為散騎侍郎耀卿為定州刺史耀卿叔業之七世孫也【尚辰羊翻沔彌兖翻散悉亶翻騎奇寄翻蕭齊東昏侯之時裴叔業叛齊入魏】十二月乙巳還東都【還從】<br />
<br />
  【宣翻又如字】 突厥頡利發辭歸上厚賜而遣之竟不許昏【厥九勿翻頡奚結翻】 王毛仲有寵于上百官附之者輻湊毛仲嫁女上問何須【須求也索也】毛仲頓首對曰臣萬事已備但未得客上曰張說源乾曜輩豈不可呼邪【說讀為悦邪音耶】對曰此則得之上曰知汝所不能致者一人耳必宋璟也對曰然上笑曰朕明日為汝召客【為于偽翻】明日上謂宰相朕奴毛仲有昏事卿等宜與諸逹官悉詣其第既而日中衆客未敢舉筯待璟久之方至先執酒西向拜謝【謝為君命而來非為毛仲來也】飲不盡巵遽稱腹痛而歸璟之剛直老而彌篤 先是契丹王李吐干與可突干復相猜忌攜公主來奔不敢復還更封遼陽王留宿衛【吐從暾入聲可從刋入聲先悉薦翻復扶又翻更工衡翻】可突干立李盡忠之弟邵固為主車駕東廵邵固詣行在因從至泰山拜左羽林大將軍靜折軍經略大使【四年契丹來降置静折軍于松漠府以其酋長為經略大使言中國之兵不動而契丹自降以靜而折遐衝也使疏吏翻】 上疑吏部選試不公時選期已迫御史中丞宇文融密奏請分吏部為十銓甲戌以禮部尚書蘇頲等十人掌吏部選【選須絹翻下同頲他鼎翻】試判將畢遽召入禁中决定吏部尚書侍郎皆不得預左庶子吴兢上表【上時掌翻】以為陛下曲受讒言不信有司非居上臨人推誠感物之道昔陳平丙吉漢之宰相尚不對錢穀之數不問鬬死之人【陳平事見十三卷漢文帝元年丙吉事見二十六卷漢宣帝神爵三年】况大唐萬乘之君豈得下行銓選之事乎凡選人書判並請委之有司停此十銓上雖不即從明年復故 是歲東都斗米十五錢青齊五錢粟三錢 于闐王尉遲眺隂結突厥及諸胡謀叛【尉紆勿翻】安西副大都護杜暹發兵捕斬之更為立王【為于偽翻】<br />
<br />
  資治通鑑卷二百十二<br />
<br />
<史部,編年類,資治通鑑>  <br>
   </div> 

<script src="/search/ajaxskft.js"> </script>
 <div class="clear"></div>
<br>
<br>
 <!-- a.d-->

 <!--
<div class="info_share">
</div> 
-->
 <!--info_share--></div>   <!-- end info_content-->
  </div> <!-- end l-->

<div class="r">   <!--r-->



<div class="sidebar"  style="margin-bottom:2px;">

 
<div class="sidebar_title">工具类大全</div>
<div class="sidebar_info">
<strong><a href="http://www.guoxuedashi.com/lsditu/" target="_blank">历史地图</a></strong>  
<a href="http://www.880114.com/" target="_blank">英语宝典</a>  
<a href="http://www.guoxuedashi.com/13jing/" target="_blank">十三经检索</a> 
<br><strong><a href="http://www.guoxuedashi.com/gjtsjc/" target="_blank">古今图书集成</a></strong> 
<a href="http://www.guoxuedashi.com/duilian/" target="_blank">对联大全</a> <strong><a href="http://www.guoxuedashi.com/xiangxingzi/" target="_blank">象形文字典</a></strong> 

<br><a href="http://www.guoxuedashi.com/zixing/yanbian/">字形演变</a>  <strong><a href="http://www.guoxuemi.com/hafo/" target="_blank">哈佛燕京中文善本特藏</a></strong>
<br><strong><a href="http://www.guoxuedashi.com/csfz/" target="_blank">丛书&方志检索器</a></strong> <a href="http://www.guoxuedashi.com/yqjyy/" target="_blank">一切经音义</a>  

<br><strong><a href="http://www.guoxuedashi.com/jiapu/" target="_blank">家谱族谱查询</a></strong>  <strong><a href="http://shufa.guoxuedashi.com/sfzitie/" target="_blank">书法字帖欣赏</a></strong> 
<br>

</div>
</div>


<div class="sidebar" style="margin-bottom:0px;">

<font style="font-size:22px;line-height:32px">QQ交流群9:489193090</font>


<div class="sidebar_title">手机APP 扫描或点击</div>
<div class="sidebar_info">
<table>
<tr>
	<td width=160><a href="http://m.guoxuedashi.com/app/" target="_blank"><img src="/img/gxds-sj.png" width="140"  border="0" alt="国学大师手机版"></a></td>
	<td>
<a href="http://www.guoxuedashi.com/download/" target="_blank">app软件下载专区</a><br>
<a href="http://www.guoxuedashi.com/download/gxds.php" target="_blank">《国学大师》下载</a><br>
<a href="http://www.guoxuedashi.com/download/kxzd.php" target="_blank">《汉字宝典》下载</a><br>
<a href="http://www.guoxuedashi.com/download/scqbd.php" target="_blank">《诗词曲宝典》下载</a><br>
<a href="http://www.guoxuedashi.com/SiKuQuanShu/skqs.php" target="_blank">《四库全书》下载</a><br>
</td>
</tr>
</table>

</div>
</div>


<div class="sidebar2">
<center>


</center>
</div>

<div class="sidebar"  style="margin-bottom:2px;">
<div class="sidebar_title">网站使用教程</div>
<div class="sidebar_info">
<a href="http://www.guoxuedashi.com/help/gjsearch.php" target="_blank">如何在国学大师网下载古籍?</a><br>
<a href="http://www.guoxuedashi.com/zidian/bujian/bjjc.php" target="_blank">如何使用部件查字法快速查字?</a><br>
<a href="http://www.guoxuedashi.com/search/sjc.php" target="_blank">如何在指定的书籍中全文检索?</a><br>
<a href="http://www.guoxuedashi.com/search/skjc.php" target="_blank">如何找到一句话在《四库全书》哪一页?</a><br>
</div>
</div>


<div class="sidebar">
<div class="sidebar_title">热门书籍</div>
<div class="sidebar_info">
<a href="/so.php?sokey=%E8%B5%84%E6%B2%BB%E9%80%9A%E9%89%B4&kt=1">资治通鉴</a> <a href="/24shi/"><strong>二十四史</strong></a>&nbsp; <a href="/a2694/">野史</a>&nbsp; <a href="/SiKuQuanShu/"><strong>四库全书</strong></a>&nbsp;<a href="http://www.guoxuedashi.com/SiKuQuanShu/fanti/">繁体</a>
<br><a href="/so.php?sokey=%E7%BA%A2%E6%A5%BC%E6%A2%A6&kt=1">红楼梦</a> <a href="/a/1858x/">三国演义</a> <a href="/a/1038k/">水浒传</a> <a href="/a/1046t/">西游记</a> <a href="/a/1914o/">封神演义</a>
<br>
<a href="http://www.guoxuedashi.com/so.php?sokeygx=%E4%B8%87%E6%9C%89%E6%96%87%E5%BA%93&submit=&kt=1">万有文库</a> <a href="/a/780t/">古文观止</a> <a href="/a/1024l/">文心雕龙</a> <a href="/a/1704n/">全唐诗</a> <a href="/a/1705h/">全宋词</a>
<br><a href="http://www.guoxuedashi.com/so.php?sokeygx=%E7%99%BE%E8%A1%B2%E6%9C%AC%E4%BA%8C%E5%8D%81%E5%9B%9B%E5%8F%B2&submit=&kt=1"><strong>百衲本二十四史</strong></a>  <a href="http://www.guoxuedashi.com/so.php?sokeygx=%E5%8F%A4%E4%BB%8A%E5%9B%BE%E4%B9%A6%E9%9B%86%E6%88%90&submit=&kt=1"><strong>古今图书集成</strong></a>
<br>

<a href="http://www.guoxuedashi.com/so.php?sokeygx=%E4%B8%9B%E4%B9%A6%E9%9B%86%E6%88%90&submit=&kt=1">丛书集成</a> 
<a href="http://www.guoxuedashi.com/so.php?sokeygx=%E5%9B%9B%E9%83%A8%E4%B8%9B%E5%88%8A&submit=&kt=1"><strong>四部丛刊</strong></a>  
<a href="http://www.guoxuedashi.com/so.php?sokeygx=%E8%AF%B4%E6%96%87%E8%A7%A3%E5%AD%97&submit=&kt=1">說文解字</a> <a href="http://www.guoxuedashi.com/so.php?sokeygx=%E5%85%A8%E4%B8%8A%E5%8F%A4&submit=&kt=1">三国六朝文</a>
<br><a href="http://www.guoxuedashi.com/so.php?sokeytm=%E6%97%A5%E6%9C%AC%E5%86%85%E9%98%81%E6%96%87%E5%BA%93&submit=&kt=1"><strong>日本内阁文库</strong></a> <a href="http://www.guoxuedashi.com/so.php?sokeytm=%E5%9B%BD%E5%9B%BE%E6%96%B9%E5%BF%97%E5%90%88%E9%9B%86&ka=100&submit=">国图方志合集</a> <a href="http://www.guoxuedashi.com/so.php?sokeytm=%E5%90%84%E5%9C%B0%E6%96%B9%E5%BF%97&submit=&kt=1"><strong>各地方志</strong></a>

</div>
</div>


<div class="sidebar2">
<center>

</center>
</div>
<div class="sidebar greenbar">
<div class="sidebar_title green">四库全书</div>
<div class="sidebar_info">

《四库全书》是中国古代最大的丛书,编撰于乾隆年间,由纪昀等360多位高官、学者编撰,3800多人抄写,费时十三年编成。丛书分经、史、子、集四部,故名四库。共有3500多种书,7.9万卷,3.6万册,约8亿字,基本上囊括了古代所有图书,故称“全书”。<a href="http://www.guoxuedashi.com/SiKuQuanShu/">详细>>
</a>

</div> 
</div>

</div>  <!--end r-->

</div>
<!-- 内容区END --> 

<!-- 页脚开始 -->
<div class="shh">

</div>

<div class="w1180" style="margin-top:8px;">
<center><script src="http://www.guoxuedashi.com/img/plus.php?id=3"></script></center>
</div>
<div class="w1180 foot">
<a href="/b/thanks.php">特别致谢</a> | <a href="javascript:window.external.AddFavorite(document.location.href,document.title);">收藏本站</a> | <a href="#">欢迎投稿</a> | <a href="http://www.guoxuedashi.com/forum/">意见建议</a> | <a href="http://www.guoxuemi.com/">国学迷</a> | <a href="http://www.shuowen.net/">说文网</a><script language="javascript" type="text/javascript" src="https://js.users.51.la/17753172.js"></script><br />
  Copyright &copy; 国学大师 古典图书集成 All Rights Reserved.<br>
  
  <span style="font-size:14px">免责声明:本站非营利性站点,以方便网友为主,仅供学习研究。<br>内容由热心网友提供和网上收集,不保留版权。若侵犯了您的权益,来信即刪。scp168@qq.com</span>
  <br />
ICP证:<a href="http://www.beian.miit.gov.cn/" target="_blank">鲁ICP备19060063号</a></div>
<!-- 页脚END --> 
<script src="http://www.guoxuedashi.com/img/plus.php?id=22"></script>
<script src="http://www.guoxuedashi.com/img/tongji.js"></script>

</body>
</html>
