<!DOCTYPE html PUBLIC "-//W3C//DTD XHTML 1.0 Transitional//EN" "http://www.w3.org/TR/xhtml1/DTD/xhtml1-transitional.dtd">
<html xmlns="http://www.w3.org/1999/xhtml">
<head>
<meta http-equiv="Content-Type" content="text/html; charset=utf-8" />
<meta http-equiv="X-UA-Compatible" content="IE=Edge,chrome=1">
<title>資治通鑒_55-資治通鑑卷五十四_55-資治通鑑卷五十四</title>
<meta name="Keywords" content="資治通鑒_55-資治通鑑卷五十四_55-資治通鑑卷五十四">
<meta name="Description" content="資治通鑒_55-資治通鑑卷五十四_55-資治通鑑卷五十四">
<meta http-equiv="Cache-Control" content="no-transform" />
<meta http-equiv="Cache-Control" content="no-siteapp" />
<link href="/img/style.css" rel="stylesheet" type="text/css" />
<script src="/img/m.js?2020"></script> 
</head>
<body>
 <div class="ClassNavi">
<a  href="/24shi/">二十四史</a> | <a href="/SiKuQuanShu/">四库全书</a> | <a href="http://www.guoxuedashi.com/gjtsjc/"><font  color="#FF0000">古今图书集成</font></a> | <a href="/renwu/">历史人物</a> | <a href="/ShuoWenJieZi/"><font  color="#FF0000">说文解字</a></font> | <a href="/chengyu/">成语词典</a> | <a  target="_blank"  href="http://www.guoxuedashi.com/jgwhj/"><font  color="#FF0000">甲骨文合集</font></a> | <a href="/yzjwjc/"><font  color="#FF0000">殷周金文集成</font></a> | <a href="/xiangxingzi/"><font color="#0000FF">象形字典</font></a> | <a href="/13jing/"><font  color="#FF0000">十三经索引</font></a> | <a href="/zixing/"><font  color="#FF0000">字体转换器</font></a> | <a href="/zidian/xz/"><font color="#0000FF">篆书识别</font></a> | <a href="/jinfanyi/">近义反义词</a> | <a href="/duilian/">对联大全</a> | <a href="/jiapu/"><font  color="#0000FF">家谱族谱查询</font></a> | <a href="http://www.guoxuemi.com/hafo/" target="_blank" ><font color="#FF0000">哈佛古籍</font></a> 
</div>

 <!-- 头部导航开始 -->
<div class="w1180 head clearfix">
  <div class="head_logo l"><a title="国学大师官网" href="http://www.guoxuedashi.com" target="_blank"></a></div>
  <div class="head_sr l">
  <div id="head1">
  
  <a href="http://www.guoxuedashi.com/zidian/bujian/" target="_blank" ><img src="http://www.guoxuedashi.com/img/top1.gif" width="88" height="60" border="0" title="部件查字,支持20万汉字"></a>


<a href="http://www.guoxuedashi.com/help/yingpan.php" target="_blank"><img src="http://www.guoxuedashi.com/img/top230.gif" width="600" height="62" border="0" ></a>


  </div>
  <div id="head3"><a href="javascript:" onClick="javascript:window.external.AddFavorite(window.location.href,document.title);">添加收藏</a>
  <br><a href="/help/setie.php">搜索引擎</a>
  <br><a href="/help/zanzhu.php">赞助本站</a></div>
  <div id="head2">
 <a href="http://www.guoxuemi.com/" target="_blank"><img src="http://www.guoxuedashi.com/img/guoxuemi.gif" width="95" height="62" border="0" style="margin-left:2px;" title="国学迷"></a>
  

  </div>
</div>
  <div class="clear"></div>
  <div class="head_nav">
  <p><a href="/">首页</a> | <a href="/ShuKu/">国学书库</a> | <a href="/guji/">影印古籍</a> | <a href="/shici/">诗词宝典</a> | <a   href="/SiKuQuanShu/gxjx.php">精选</a> <b>|</b> <a href="/zidian/">汉语字典</a> | <a href="/hydcd/">汉语词典</a> | <a href="http://www.guoxuedashi.com/zidian/bujian/"><font  color="#CC0066">部件查字</font></a> | <a href="http://www.sfds.cn/"><font  color="#CC0066">书法大师</font></a> | <a href="/jgwhj/">甲骨文</a> <b>|</b> <a href="/b/4/"><font  color="#CC0066">解密</font></a> | <a href="/renwu/">历史人物</a> | <a href="/diangu/">历史典故</a> | <a href="/xingshi/">姓氏</a> | <a href="/minzu/">民族</a> <b>|</b> <a href="/mz/"><font  color="#CC0066">世界名著</font></a> | <a href="/download/">软件下载</a>
</p>
<p><a href="/b/"><font  color="#CC0066">历史</font></a> | <a href="http://skqs.guoxuedashi.com/" target="_blank">四库全书</a> |  <a href="http://www.guoxuedashi.com/search/" target="_blank"><font  color="#CC0066">全文检索</font></a> | <a href="http://www.guoxuedashi.com/shumu/">古籍书目</a> | <a   href="/24shi/">正史</a> <b>|</b> <a href="/chengyu/">成语词典</a> | <a href="/kangxi/" title="康熙字典">康熙字典</a> | <a href="/ShuoWenJieZi/">说文解字</a> | <a href="/zixing/yanbian/">字形演变</a> | <a href="/yzjwjc/">金 文</a> <b>|</b>  <a href="/shijian/nian-hao/">年号</a> | <a href="/diming/">历史地名</a> | <a href="/shijian/">历史事件</a> | <a href="/guanzhi/">官职</a> | <a href="/lishi/">知识</a> <b>|</b> <a href="/zhongyi/">中医中药</a> | <a href="http://www.guoxuedashi.com/forum/">留言反馈</a>
</p>
  </div>
</div>
<!-- 头部导航END --> 
<!-- 内容区开始 --> 
<div class="w1180 clearfix">
  <div class="info l">
   
<div class="clearfix" style="background:#f5faff;">
<script src='http://www.guoxuedashi.com/img/headersou.js'></script>

</div>
  <div class="info_tree"><a href="http://www.guoxuedashi.com">首页</a> > <a href="/SiKuQuanShu/fanti/">四库全书</a>
 > <h1>资治通鉴</h1> <!--         下载:【右键另存为】即可 --></div>
  <div class="info_content zj clearfix">
  
<div class="info_txt clearfix" id="show">
<center style="font-size:24px;">55-資治通鑑卷五十四</center>
    資治通鑑卷五十四   宋 司馬光 撰<br />
<br />
  胡三省 音註<br />
<br />
  漢紀四十六【起彊圉作噩盡昭陽單閼凡七年】<br />
<br />
  孝桓皇帝上之下<br />
<br />
  永壽三年春正月己未赦天下 居風令貪暴無度【居風縣屬九真郡交州記曰山有風門常有風】縣人朱達等與蠻夷同反攻殺令聚衆至四五千人夏四月進攻九真九真太守兒式戰死【守式又翻兒五兮翻】詔九真都尉魏朗討破之 閏月庚辰晦日有食之 京師蝗 或上言民之貧困以貨輕錢薄宜改鑄大錢事下四府【下遐稼翻四府三公府及大將軍府】羣僚及太學能言之士議之太學生劉陶上議曰當今之憂不在於貨在乎民飢竊見比年已來【比毗至翻】良苗盡於蝗螟之口杼軸空於公私之求民所患者豈謂錢貨之厚薄銖兩之輕重哉就使當今沙礫化為南金瓦石變為和玉【賢曰詩曰大賂南金和玉卞和之玉礫郎狄翻】使百姓渇無所飲飢無所食雖皇羲之純德【天地初立有天皇氏澹泊無所施為而民自化伏羲氏始畫八卦造書契以代結繩之政去洪荒之世未遠故其風朴畧】唐虞之文明猶不能以保蕭牆之内也【鄭氏曰蕭肅也牆謂屛也君臣相見之禮至屛而加肅敬焉是以謂之蕭牆】蓋民可百年無貨不可一朝有飢故食為至急也議者不達農殖之本多言鑄冶之便蓋萬人鑄之一人奪之猶不能給况今一人鑄之則萬人奪之乎雖以隂陽為炭萬物為銅【賈誼鵩賦之言】役不食之民使不飢之士猶不能足無猒之求也【猒於鹽翻下同】夫欲民殷財阜【楊子曰君人者務在殷民阜財】要在止役禁奪則百姓不勞而足陛下愍海内之憂戚欲鑄錢齊貨以救其弊猶養魚沸鼎之中棲鳥烈火之上水木本魚鳥之所生也用之不時必至焦爛願陛下寛鍥薄之禁【賢曰鍥刻也音口結翻】後冶鑄之議聽民庶之謠吟問路叟之所憂【通下情也賢曰列子曰昔堯理天下五十年不知天下理亂堯乃微服遊於康衢兒童謠曰立我蒸民莫匪爾極不識不知順帝之則說苑曰孔子行遊中路聞哭者其音甚悲孔子避車而問之曰夫子非有喪也何哭之悲虞丘子對曰吾有三失吾少好學周徧天下還後吾親喪是一失也事君驕奢不遂是二失也厚交友而後絶是三失也】瞰三光之文耀視山河之分流【瞰苦鑒翻視也賢曰三光日月星也分謂山流謂河言日月有讁食之變星辰有錯行之異故視其文耀也山崩川竭皆亡之徵不可不察】天下之心國家大事粲然皆見無有遺惑者矣伏念當今地廣而不得耕民衆而無所食羣小競進秉國之位鷹揚天下鳥鈔求飽【鈔楚交翻】吞肌及骨並噬無猒誠恐卒有役夫窮匠起於板築之間【卒讀曰猝賢曰役夫謂如陳涉起蘄也窮匠謂如驪山之徒也余謂陳涉黥布皆可以言役夫窮匠則山陽鐵官徒蘇令等是也】投斤攘臂登高遠呼【呼火故翻】使愁怨之民響應雲合雖方尺之錢何有能救其危也【言雖錢大方尺亦不能救天下之亂也】遂不改錢 冬十一月司徒尹頌薨 【考異曰袁紀在六月今從范書】 長沙蠻反寇益陽【益陽縣屬長沙郡賢曰縣在益水之陽今潭州縣故城在縣東】 以司空韓縯為司徒【縯以善翻】以太常北海孫朗為司空<br />
<br />
  延熹元年夏五月甲辰晦日有食之太史令陳授因小黄門徐璜陳日食之變咎在大將軍冀冀聞之諷雒陽收考授【諷雒陽令收考之也】死於獄帝由是怒冀 【考異曰袁紀曰冀以私憾專殺議郎邴尊上益怒今從范書】 京師蝗 六月戊寅赦天下改元大雩【公羊傳曰大雩旱祭也何休注曰君親之南郊以六事謝過自責曰政不善歟民失職歟宫室榮歟婦謁盛歟苞苴行歟讒夫昌歟使童男女各八人舞而呼雩故謂之雩鄭玄曰雩吁嗟求雨之祭也服䖍曰雩遠也遠為百穀祈膏雨也陸佃曰雩雨不雨未定也】 秋七月甲子太尉黄瓊免以太常胡廣為太尉 冬十月帝校獵廣成【廣成苑在河南新城縣】遂幸上林苑【此上林苑在雒陽西】 十二月南匈奴諸部並叛與烏桓鮮卑寇緣邊九郡帝以京兆尹陳龜為度遼將軍 【考異曰按匈奴傳每除度遼將軍輒書之此陳龜及前李膺後种暠皆不記一時既不當有兩官今約其事分著前後】龜臨行上疏曰臣聞三辰不軌【言三辰之行不順軌也】擢士為相蠻夷不恭拔卒為將臣無文武之材而忝鷹揚之任【詩曰維師尚父時維鷹揚爾雅翼鷹好揚隼好翔故以比尚父之武】雖殁軀體無所云補今西州邊鄙土地塉埆【塉秦昔翻賢曰埆音覺又音確土薄也】民數更寇虜【數所角翻更工衡翻下租更同】室家殘破雖含生氣實同枯朽往歲并州水雨災螟互生稼穡荒耗租更空闕【賢曰更謂卒更錢也】陛下以百姓為子焉可不垂撫循之恩哉【焉於䖍翻】古公西伯天下歸仁【古公亶父避狄去邠居岐從之者如歸市帝王世紀曰西伯至仁百姓襁負而至】豈復輿金輦寶以為民惠乎【復扶又翻】陛下繼中興之統承光武之業臨朝聽政而未留聖意且牧守不良或出中官【謂牧守出於中官之所引用也】懼逆上旨取過目前【過度也】呼嗟之聲招致災害胡虜凶悍【悍下罕翻又侯旰翻】因衰緣隙而令倉庫單於豺狼之口【單與殫同盡也】功業無銖兩之効【十絫為銖二十四銖為兩】皆由將帥不忠聚姦所致前凉州刺史祝良初除到州多所糾罰太守令長貶黜將半【長知兩翻】政未踰時功効卓然實應賞異以勸功能改任牧守去斥姦殘【去羌呂翻】又宜更選匈奴烏桓護羌中郎將校尉【護匈奴中郎將護烏桓護羌校尉更工衡翻校戶教翻】簡練文武授之灋令除并凉二州今年租更【租賦也更役也更工衡翻下同】寛赦罪隸掃除更始則善吏知奉公之祐惡者覺營私之禍胡馬可不窺長城塞下無候望之患矣帝乃更選幽并刺史自營郡太守都尉以下多所革易【京兆虎牙營扶風雍營皆都尉領之諸郡各有太守都尉】下詔為陳將軍除并凉一年租賦以賜吏民【為于偽翻】龜到職州郡重足震栗【言重足而立也重音直龍翻】省息經用歲以億計詔拜安定屬國都尉張奐為北中郎將【按奐傳即護匈奴中郎將】以討匈奴烏桓等匈奴烏桓燒度遼將軍門【賢曰時度遼將軍屯五原】引屯赤阬煙火相望兵衆大恐各欲亡去奐安坐帷中與弟子講誦自若軍士稍安乃潛誘烏桓隂與和通【誘音酉】遂使斬匈奴屠各渠帥【屠各匈奴别種也屠直於翻帥所類翻】襲破其衆諸胡悉降奐以南單于車兒不能統理國事乃拘之奏立左谷蠡王為單于【谷蠡音鹿黎】詔曰春秋大居正車兒一心向化何罪而黜其遣還庭【言春秋之義大居正賢曰春秋法五始之要故經曰元年春正月言王者即位之年宜大開恩宥其居正車兒即是桓帝即位之建和元年立自立以來一心向化宜厚宥之 考異曰袁紀元康元年四月中郎將張奐以車兒不能治國事上言更立左鹿蠡王都紺為單于詔不許范書匈奴傳在延熹元年今從之】大將軍冀與陳龜素有隙譖其沮毁國威挑取功譽<br />
<br />
  【沮在呂翻賢曰挑猶取也獨取其名如挑戰之義音徒了翻】不為胡虜所畏坐徵還以种暠為度遼將軍【种音冲暠工老翻】龜遂乞骸骨歸田里復徵為尚書【復扶又翻】冀暴虐日甚龜上疏言其罪狀請誅之帝不省【省息景翻】龜自知必為冀所害不食七日而死【東都之臣以死攻外戚者鄭弘陳龜二人而已】种暠到營所先宣恩信誘降諸胡其有不服然後加討羌虜先時有生見獲質於郡縣者【質音致】悉遣還之誠心懷撫信賞分明由是羌胡皆來順服暠乃去烽燧除候望【去羌呂翻】邊方晏然無警入為大司農二年春二月鮮卑寇鴈門 蜀郡夷寇蠶陵【賢曰蠶陵縣屬蜀郡故城在今翼州翼水縣西有蠶陵山因以名焉宋白曰翼州衛山縣本漢蠶陵縣地故城在縣西有蠶陵山】三月復斷刺史二千石行三年喪【永興二年聽行三年喪斷丁管翻】夏京師大水 六月鮮卑寇遼東 梁皇后恃姊兄<br />
<br />
  䕃埶【姊順烈皇后兄大將軍冀也䕃庇也今人謂憑藉世資得官者為䕃官蓋取木為喻言能䕃庇其本根也】恣極奢靡兼倍前世專寵妬忌六宫莫得進見及太后崩恩寵頓衰后既無嗣每宫人孕育鮮得全者【鮮息淺翻】帝雖迫畏梁冀不敢譴怒然進御轉希【按周禮注鄭衆云六宫後五前一王之妃百二十人后一人夫人三人嬪九人世婦二十七人女御八十一人鄭玄曰六宫謂后也婦人稱寢曰宫宫隱蔽之言后象王立六宫而居之亦正寢一燕寢五夫人以下分居后之六宫每宫九嬪一人世婦三人女御九人其餘九嬪三人世婦九人女御二十七人從后唯所燕息焉從后者五日而沐浴其次又上十五日而徧云夫人如三公從容論婦禮此禮所謂以時御叙于王所者也鄭玄又曰凡羣妃御見之法月與后妃其象也卑者宜先尊者宜後女御八十一人當九夕世婦二十七人當三夕九嬪九人當一夕三夫人當一夕后當一夕十五日而徧自望後反之案二鄭所云漢之宫中貫魚無序專房之讌蔽固後宫寧復有此制乎】后益憂恚【恚於避翻】秋七月丙午皇后梁氏崩乙丑葬懿獻皇后於懿陵【賢曰諡法温和聖善曰懿聰明叡知曰獻】梁冀一門前後七侯三皇后【冀祖雍封乘氏侯冀封襄邑侯及嗣乘氏侯又封其子襄邑侯弟不疑潁陽侯蒙西平侯不疑子馬潁隂侯子挑城父侯是七封侯也恭懷順烈懿獻三皇后】六貴人二大將軍夫人女食邑稱君者七人尚公主者三人其餘卿將尹校五十七人【卿九卿也將中郎將也尹河南京兆尹也校諸校尉也校戶教翻】冀專擅威柄凶恣日積宫衛近侍並樹所親【賢曰樹置也】禁省起居纎微必知其四方調發【調徒弔翻】歲時貢獻皆先輸上第於冀【賢曰上第第一也】乘輿乃其次焉【乘繩證翻】吏民齎貨求官請罪者道路相望【請罪謂請求以脱罪也】百官遷召皆先到冀門牋檄謝恩【字書牋表也識也書也左雄傳文吏課牋奏自後世言之奏者達之天子牋者用之中宫東宫將相犬臣檄者徵召傳令用之非所以謝恩也竊意自蔡倫造紙之後用紙書者曰牋用木書者曰檄故言牋檄謝恩也】然後敢詣尚書下邳吳樹為宛令【宛於元翻】之官辭冀冀賓客布在縣界以情託樹樹曰小人姦蠧比屋可誅明將軍處上將之位宜崇賢善以補朝闕【比部必翻又毗寐翻連次也補朝闕謂補朝政之闕也處昌呂翻朝直遙翻】自侍坐以來【坐徂卧翻】未聞稱一長者而多託非人誠非敢聞冀默然不悦樹到縣遂誅殺冀客為人害者數十人樹後為荆州刺史辭冀冀鴆之出死車上遼東太守侯猛初拜不謁冀冀託以他事腰斬之郎中汝南袁著年十九詣闕上書曰夫四時之運功成則退【蔡澤之言】高爵厚寵鮮不致災【鮮息淺翻】今大將軍位極功成可為至戒宜遵縣車之禮【縣讀曰懸】高枕頤神傳曰木實繁者披枝害心【范睢曰木殖繁者披其枝披其枝者傷其心】若不抑損盛權將無以全其身矣冀聞而密遣掩捕著乃變易姓名託病偽死結蒲為人市棺殯送冀知其詐求得笞殺之太原郝絜胡武好危言高論【好呼到翻】與著友善絜武嘗連名奏記三府薦海内高士而不詣冀冀追怒之敇中都官移檄禽捕【司隸校尉領中都官徒千二百人冀蓋敇都官從事使移檄禽捕也】遂誅武家死者六十餘人絜初逃亡知不得免因輿櫬奏書冀門書入仰藥而死家乃得全安帝嫡母耿貴人薨冀從貴人從子林慮侯承求貴人珍玩不能得冀怒并族其家十餘人【人從才用翻】涿郡崔琦以文章為冀所善琦作外戚箴白鵠賦以風【外戚箴曰赫赫外戚華寵煌煌昔在帝舜德隆英皇周興三母有莘崇湯宣王晏起姜后脱簪齊桓好樂衛姬不音皆輔主以禮扶君以仁達才進善以義濟身爰暨末葉漸已穨虧貫魚不序九御差池晉國之難祻起於驪惟家之索牝雞之晨專權擅愛顯已蔽人陵長間舊圮剥至親並后匹嫡淫女斃陳匪賢是上番為司徒荷爵負乘采食名都詩人是刺德用不憮暴辛惑婦拒諫自孤蝮蛇其心縱毒不辜諸父是殺孕子是刳天怒地忿人謀鬼圖甲子昧爽身首分離初為天子後為人螭非但耽色母后尤然不相率以禮而競奬以權先笑後號卒以辱殘家國泯絶宗廟燒燔妹喜喪夏褒姒斃周妲己亡殷趙靈沙丘戚姬人豕呂宗以敗陳后作巫卒死于外霍欲鴆子身乃罹廢故曰無謂我貴天將爾摧無恃常好色有歇微無怙常幸愛有陵遲無曰我能天人爾違患生不德福有慎機日不常中月盈有虧履道者固仗埶者危微臣司戚敢告在斯箴言外戚之禍深切故具載之憮音呼風讀曰諷】冀怒琦曰昔管仲相齊樂聞譏諫之言【樂音洛】蕭何佐漢乃設書過之吏今將軍屢世台輔任齊伊周而德政未聞黎元塗炭不能結納貞良以救禍敗反欲鉗塞士口【塞悉則翻】杜蔽主聽將使玄黄改色鹿馬易形乎【玄黄者天地之色也使之改色言將使天地顛倒也鹿馬易形指趙高秦二世之事琦之論可謂深切矣】冀無以對因遣琦歸琦懼而亡匿冀捕得殺之冀秉政幾二十年【順帝永和六年冀為大將軍至是歲凡十九年幾居希翻】威行内外天子拱手不得有所親與【與讀曰豫】帝既不平之及陳授死帝愈怒和熹皇后從兄子郎中鄧香妻宣生女猛【從才用翻】香卒宣更適梁紀紀孫壽之舅也壽以猛色美引入掖庭為貴人冀欲認猛為其女易猛姓為梁冀恐猛姊壻議郎邴尊沮敗宣意【賢曰沮壞也恐尊壞敗宣意不從其改梁姓也敗補邁翻】遣客刺殺之【刺七亦翻】又欲殺宣宣家與中常侍袁赦相比【賢曰相鄰比也比音毗至翻又音毗】冀客登赦屋欲入宣家赦覺之鳴鼓會衆以告宣宣馳入白帝帝大怒因如厠獨呼小黄門史唐衡【小黄門史小黄門之掌書者也】問左右與外舍不相得者誰乎【左右謂宦官也賢曰外舍謂皇后家也】衡對中常侍單超【單音善】小黄門史左悺與梁不疑有隙【悺工喚翻又音綰】中常侍徐璜黄門令具瑗【具姓也左傳有具丙瑗于眷翻 考異曰宦者傳作中常侍具瑗今從梁冀傳】常私忿疾外舍放横【横戶孟翻】口不敢道於是帝呼超悺入室謂曰梁將軍兄弟專朝【朝直遙翻】迫脅内外公卿以下從其風旨今欲誅之於常侍意如何超等對曰誠國姦賊當誅日久臣等弱劣未知聖意如何耳帝曰審然者常侍密圖之對曰圖之不難但恐陛下腹中狐疑帝曰姦臣脇國當伏其罪何疑乎於是召璜瑗五人共定其議帝齧超臂出血為盟【齧倪結翻噬也】超等曰陛下今計已决勿復更言【復扶又翻】恐為人所疑冀心疑超等八月丁丑使中黄門張惲入省宿以防其變【使惲入禁中直宿以防超等而無上旨徑使惲入自恃威行宫省故敢然惲於粉翻】具瑗敇吏收惲以輒從外入欲圖不軌【言欲謀逆不由軌道也】帝御前殿召諸尚書入發其事使尚書令尹勲持節勒丞郎以下皆操兵守省閤【丞郎尚書左右丞及尚書郎也操七刀翻】斂諸符節送省中使具瑗將左右廐騶【賢曰騶騎士也余按續漢志太僕舊有六廐中興省約但置一厩曰未央廐主乘與及廐中諸馬後又置左駿廐令别主乘輿御馬未央廐卒騶二十人右駿廐從可知也】虎賁羽林都候劒戟士【續漢志左右都候各一人秩六百石主劒戟士徼循宫中及天子有所收考屬衛尉】合千餘人與司隸校尉張彪共圍冀第使光禄勲袁盱持節收冀大將軍印綬【盱音吁】徙封比景都鄉侯冀及妻壽即日皆自殺不疑蒙先卒悉收梁氏孫氏中外宗親送詔獄無少長皆棄市【少詩照翻長知兩翻】他所連及公卿列校刺史二千石死者數十人【校戶教翻】太尉胡廣司徒韓縯司空孫朗皆坐阿附梁冀不衛宫止長壽亭減死一等免為庶人故吏賓客免黜者三百餘人朝廷為空【為于偽翻】是時事猝從中發使者交馳公卿失其度官府市里鼎沸數日乃定百姓莫不稱慶收冀財貨縣官斥賣合三十餘萬萬以充王府用減天下税租之半散其苑囿以業窮民 壬午立梁貴人為皇后追廢懿陵為貴人冢帝惡梁氏【惡烏路翻】改皇后姓為薄氏【以文帝薄太后家謹良也】久之知為鄧香女乃復姓鄧氏 詔賞誅梁冀之功封單超徐璜具瑗左悺唐衡皆為縣侯超食二萬戶璜等各萬餘戶世謂之五侯【單超新豐侯徐璜武原侯具瑗東武陽侯左悺上蔡侯唐衡汝陽侯也】仍以悺衡為中常侍又封尚書令尹勲等七人皆為亭侯【賢曰尹勲宜陽都鄉霍諝鄴都亭張敬山陽曲鄉歐陽參修武仁亭李瑋宜陽金門虞放寃句呂都亭周永下邳高遷鄉】 以大司農黄瓊為太尉光祿大夫中山祝恬為司徒大鴻臚梁國盛允為司空【臚陵如翻按西羌傳有北海太守盛苞其先姓奭避元帝諱改姓盛按戰國時秦有盛橋則先自有盛】是時新誅梁冀天下想望異政黄瓊首居公位乃舉奏州郡素行暴汙至死徙者十餘人【行下孟翻】海内翕然稱之瓊辟汝南范滂滂少厲清節為州里所服【滂普郎翻少詩照翻】嘗為清詔使【風俗通曰汝南周勃辟太尉清詔使范史第五種以司徒清詔使冀州賢注云蓋三公府有清詔員以承詔使也使疏吏翻】案察冀州【滂傳曰時冀州飢荒盜賊羣起以滂為清詔使案察之】滂登車攬轡慨然有澄清天下之志守令臧汙者皆望風解印綬去其所舉奏莫不厭塞衆議【塞悉則翻】會詔三府掾屬舉謠言【漢官儀曰三公聽採長吏臧否民所疾苦還條奏之是為舉謠言也頃者舉謠言掾屬令史都會殿上主者大言州郡行狀云何善者同聲稱之不善者默爾銜枚】滂奏刺史二千石權豪之黨二十餘人尚書責滂所劾猥多【劾戶槩翻又戶得翻】疑有私故滂對曰臣之所舉自非叨穢姦暴深為民害豈以汙簡札哉【汙烏故翻】間以會日迫促【會日謂三府椽屬會于朝堂之日也】故先舉所急其未審者方更參實【參考以究其實也】臣聞農夫去草【去羌呂翻】嘉穀必茂忠臣除姦王道以清若臣言有貳甘受顯戮尚書不能詰【詰去吉翻】 尚書令陳蕃上疏薦五處士【處昌呂翻】豫章徐穉彭城姜肱【姓譜本自炎帝居於姜水因以為氏】汝南袁閎京兆韋著潁川李曇【曇徒含翻 考異曰范書徐穉傳云延熹二年尚書令陳蕃僕射胡廣等上書薦穉袁紀五年尚書令陳蕃薦五處士按二年胡廣已為太尉五年蕃已為光祿勲今置在是年從范書去廣名從袁紀】帝悉以安車玄纁備禮徵之皆不至穉家貧常自耕稼非其力不食恭儉義讓所居服其德屢辟公府不起陳蕃為豫章太守以禮請署功曹穉不之免【不辭免也】既謁而退蕃性方峻不接賓客唯穉來特設一榻去則縣之【榻坐榻也亦謂之牀縣讀曰懸】後舉有道【有道舉見五十卷安帝建光元年】家拜太原太守【賢曰就家而拜之也】皆不就穉雖不應諸公之辟然聞其死喪輒負笈赴弔【笈極曄翻】常於家豫灸雞一隻以一兩綿絮漬酒中暴乾【暴步木翻日曬也乾音干】以裹雞徑到所赴冢隧外以水漬綿使有酒氣斗米飯白茅為藉以雞置前醊酒畢【醊株衛翻酹酒也】留謁則去【謁猶刺也】不見喪主肱與二弟仲海季江俱以孝友著聞【聞音問】常同被而寢不應徵聘肱嘗與弟季江俱詣郡夜於道為盜所刼欲殺之肱曰弟年幼父母所憐又未聘娶願殺身濟弟季江曰兄年德在前家之珍寶國之英俊乞自受戮以代兄命盜遂兩釋焉但掠奪衣資而已既至郡中見肱無衣服怪問其故肱託以他辭終不言盜盜聞而感悔就精廬求見徵君【賢曰精廬即精舍也以其嘗蒙徵聘故稱為徵君】叩頭謝罪還所略物肱不受勞以酒食而遣之【勞力到翻】帝既徵肱不至乃下彭城【下遐稼翻】使畫工圖其形狀肱卧於幽闇以被韜面【賢曰韜藏也】言患眩疾不欲出風工竟不得見之閎安之玄孫也【袁安歷事明章和以忠篤稱】苦身修節不應辟召著隱居講授不修世務曇繼母苦烈曇奉之逾謹得四時珍玩未嘗不先拜而後進鄉里以為灋帝又徵安陽魏桓【安陽縣屬汝南郡】其鄉人勸之行桓曰夫干祿求進所以行其志也今後宫千數其可損乎廐馬萬匹其可減乎左右權豪其可去乎【去羌呂翻】皆對曰不可桓乃慨然歎曰使桓生行死歸於諸子何有哉【賢曰若迕時強諫死而後歸於諸勸行者復何益也】遂隱身不出 帝既誅梁冀故舊恩私多受封爵追贈皇后父鄧香為車騎將軍封安陽侯更封后母宣為昆陽君兄子康秉皆為列侯宗族皆列校郎將【列校謂北軍五校尉郎將即三署中郎將校戶教翻】賞賜以巨萬計中常侍侯覽上縑五千匹【上時掌翻下同】帝賜爵關内侯又託以與議誅冀【與讀曰豫】進封高鄉侯又封小黄門劉普趙忠等八人為鄉侯自是權埶專歸宦官矣五侯尤貪縱傾動内外時災異數見【數所角翻見賢遍翻】白馬令甘陵李雲露布上書移副三府【白馬縣屬東郡賢曰露布謂不封之也并以副本上三公府也】曰梁冀雖恃權專擅虐流天下今以罪行誅猶召家臣搤殺之耳【家臣謂猶古之家相也搤乙革翻】而猥封謀臣萬戶以上【謂單超等五侯也】高祖聞之得無見非【謂高祖之約非有功不侯】西北列將得無解體【賢曰列將謂皇甫規段熲等】孔子曰帝者諦也【春秋運斗樞曰五帝修名立功修德成化統調隂陽招類使神故稱帝帝之為言諦也鄭玄注云審諦於物色也】今官位錯亂小人諂進財貨公行政化日損尺一拜用【賢曰尺一之板謂詔策也見漢官儀又曰尺一謂板長尺一以寫詔書也】不經御省【御進也省悉井翻猶今言省審也】是帝欲不諦乎帝得奏震怒下有司逮雲【下遐稼翻下同】詔尚書都護劒戟送黄門北寺獄【都總也護監也詔尚書總監左右都候劒戟士防送雲詔獄也或曰都護當作都候賢曰前書音義曰北寺獄即若盧獄】使中常侍管霸與御史廷尉雜考之時弘農五官掾杜衆傷雲以忠諫獲罪【續漢志郡有五官掾署功曹及諸曹事】上書願與雲同日死帝愈怒遂并下廷尉大鴻臚陳蕃上疏曰李雲所言雖不識禁忌干上逆旨其意歸於忠國而已昔高祖忍周昌不諱之諫【謂周昌比高祖於桀紂也】成帝赦朱雲腰領之誅【事見三十二卷成帝元延元年】今日殺雲臣恐剖心之譏復議於世矣【謂暴如商受剖賢人之心也復扶又翻下同】太常楊秉雒陽市長沐茂【漢官儀曰雒陽市長秩四百石屬大司農沐音木集韻曰姓也風俗通漢有東平太守沐寵】郎中上官資並上疏請雲帝恚甚【恚於避翻】有司奏以為大不敬【蓋三公及尚書奏也】詔切責蕃秉免歸田里茂資貶秩二等時帝在濯龍池【濯龍池在濯龍園中近北宫】管霸奏雲等事霸跪言曰李雲草澤愚儒杜衆郡中小吏出於狂戇不足加罪【戇陟降翻】帝謂霸曰帝欲不諦是何等語而常侍欲原之邪顧使小黄門可其奏雲衆皆死獄中【霸跪奏若為雲等言而獄辭則致之死也】於是嬖寵益横太尉瓊自度力不能制【横戶孟翻度徒洛翻】乃稱疾不起上疏曰陛下即位以來未有勝政【言政事未有以勝於前朝也】諸梁秉權豎宦充朝【朝直遙翻】李固杜喬既以忠言横見殘滅而李雲杜衆復以直道繼踵受誅【横戶孟翻復扶又翻】海内傷懼益以怨結朝野之人以忠為諱尚書周永素事梁冀假其威勢見冀將衰乃陽毁示忠【陽毁梁氏以示忠於帝室】遂因姦計亦取封侯【周永與尹勲同封侯注見上】又黄門挾邪羣輩相黨自冀興盛腹背相親朝夕圖謀共搆姦軌臨冀當誅無可設巧復託其惡以要爵賞【要一遙翻】陛下不加清徵【范書黄瓊傳徵作澂澂與澄同譬之水也若清澂則塵翳在上滓濁在下不可得而混矣】審别真偽【别彼列翻】復與忠臣並時顯封粉墨雜糅【糅汝救翻】所謂抵金玉於砂礫【賢曰抵投也音紙】碎珪璧於泥塗四方聞之莫不憤歎臣世荷國恩【瓊父香為尚書令甚為和帝所親重荷下可翻】身輕位重敢以垂絶之日陳不諱之言書奏不納 冬十月壬申上行幸長安 中常侍單超疾病壬寅以超為車騎將軍【孫程之死追贈車騎將軍今及超之生存授之】十二月己巳上還自長安 燒當燒何當煎勒姐等八種羌寇隴西金城塞【姐音紫又音且也翻種章勇翻】護羌校尉段熲擊破之追至羅亭【賢曰東觀記曰追到積石山即與羅亭相近在今鄯州】斬其酋豪以下二千級獲生口萬餘人【酋慈由翻】 詔復以陳蕃為光祿勲楊秉為河南尹單超兄子匡為濟隂太守【濟子禮翻】負埶貪放兖州刺史第五種使從事衛羽案之【百官志十二州刺史皆有從事史員職畧與司隸同無都官從事其功曹從事為治中從事其部郡國從事每郡國各一人主督促文書察舉非法皆州自辟除通為百石】得臧五六千萬種即奏匡并以劾超匡窘迫賂客任方刺羽【劾戶槩翻又戶得翻】羽覺其姦捕方囚繫雒陽匡慮楊秉窮竟其事密令方等突獄亡走尚書召秉詰責秉對曰方等無狀釁由單匡乞檻車徵匡考覈其事則姦慝蹤緒必可立得秉竟坐論作左校【校戶教翻】時泰山賊叔孫無忌寇暴徐兖州郡不能討單超以是陷第五種坐徙朔方 【考異曰楊秉傳作超弟宦者傳作弟子今從第五種傳范書李雲死在延熹三年春袁紀在二年秋按楊秉傳三年坐救雲免歸田里其年冬復徵拜河南尹坐單匡使客任方刺衛羽繫獄亡走論作左校第五種傳匡遣客刺羽超積忿以事陷種若如范書則雲死時單超已卒何得更能陷種又雲書所論者立鄧后與封五侯事皆在二年袁紀似近之種傳又云衛羽為種說叔孫無忌無忌率其黨與三千餘人降按帝紀延熹三年十一月無忌攻殺都尉侯章又臧旻訟種書稱種所坐盜賊公負筋力未就然則種必不能降無忌此說妄也】超外孫董援為朔方太守稸怒以待之【稸與蓄同】種故吏孫斌知種必死【斌與彬同】結客追種及於太原刼之以歸亡命數年會赦得免種倫之曾孫也【第五倫歷事光明】是時封賞踰制内寵猥盛陳蕃上疏曰夫諸侯上象四七【賢曰上象四七謂二十八宿各主諸侯之分野】藩屏上國【屏必郢翻】高祖之約非功臣不侯而聞追録河南尹鄧萬世父遵之微功【帝以鄧后故録遵破羌之功紹封萬世為南鄉侯】更爵尚書令黄雋先人之絶封近習以非義授邑左右以無功傳賞至乃一門之内侯者數人故緯象失度隂陽謬序【緯于貴翻】臣知封事已行【封事謂封爵之事也】言之無及誠欲陛下從是而止又采女數千【皇后紀曰光武中興六宫稱號唯皇后貴人貴人金印紫綬奉不過粟數十斛又置美人宫人采女三等並無爵歲時賞賜充給今采女數千女寵盛矣】食肉衣綺脂油粉黛不可貲計【賢曰貲量也衣於既翻】鄙諺言盜不過五女門以女貧家也今後宫之女豈不貧國乎帝頗采其言為出宫女五百餘人【為于偽翻】但賜雋爵關内侯而封萬世南鄉侯帝從容問侍中陳留爰延朕何如主也【從千容翻】對曰陛下為漢中主【中主為中材之主言可以上可以下顧輔佐者何如耳】帝曰何以言之對曰尚書令陳蕃任事則治中常侍黄門與政則亂【與讀曰豫】是以陛下可與為善可與為非【前書曰齊桓公管仲相之則霸豎刁輔之則亂可與為善可與為惡是謂中人】帝曰昔朱雲廷折欄檻【折而設翻】今侍中面稱朕違敬聞闕矣拜五官中郎將累遷大鴻臚【臚陵如翻】會客星經帝坐【帝坐一星在太微宫中坐徂卧翻】帝密以問延延上封事曰陛下以河南尹鄧萬世有龍潛之舊封為通侯恩重公卿惠豐宗室加頃引見與之對博【博塞之戲也】上下媟黷有虧尊嚴【媟私列翻】臣聞之帝左右者所以咨政德也善人同處則日聞嘉訓【處昌呂翻】惡人從游則日生邪情惟陛下遠讒諛之人【遠于願翻】納謇謇之士則災變可除帝不能用延稱病免歸<br />
<br />
  三年春正月丙申赦天下詔求李固後嗣初固既策罷【事見上卷質帝本初元年】知不免禍乃遣三子基兹燮皆歸鄉里時燮年十三姊文姬為同郡趙伯英妻見二兄歸具知事本【事本謂事之所由生也】默然獨悲曰李氏滅矣自太公已來【賢曰太公謂祖父郃也】積德累仁何以遇此密與二兄謀豫藏匿燮【先事而圖之曰豫】託言還京師人咸信之有頃難作【難乃旦翻】州郡收基兹皆死獄中文姬乃告父門生王成曰君執義先公有古人之節今委君以六尺之孤【賢曰六尺謂年十五以下】李氏存滅其在君矣成乃將燮乘江東下入徐州界變姓名為酒家傭而成賣卜於市各為異人隂相往來積十餘年梁冀既誅燮乃以本末告酒家酒家具車重厚遣之【重直用翻下重至同】燮皆不受遂還鄉里追行喪服姊弟相見悲感傍人姊戒燮曰吾家血食將絶弟幸而得濟豈非天邪宜杜絶衆人勿妄往來慎無一言加於梁氏加梁氏則連主上禍重至矣唯引咎而已【婦人之識丈夫有所不及焉】燮謹從其誨後王成卒燮以禮葬之每四節為設上賓之位而祠焉【四節之祠謂四時之祭也為于偽翻】 丙午新豐侯單超卒賜東園祕器棺中玉具【玉具即玉匣也】及葬發五營騎士將作大匠起冢塋其後四侯轉横【横戶孟翻】天下為之語曰左回天具獨坐【回天言權力能回天也賢曰獨坐言驕貴無偶也】徐卧虎唐雨墯【卧虎言無人敢攖之也雨之所墯無不沾濕言其流毒徧於天下也考異曰太子賢註范書雨墯作兩墯云隨意所為不定也諸本兩或作雨按雨墯者謂其性急暴如雨之墯無有常處也】皆競起第宅以華侈相尚其僕從皆乘牛車而從列騎【僕從才用翻】兄弟姻戚宰州臨郡辜較百姓與盜無異【較與榷同音角】虐徧天下民不堪命故多為盜賊焉中常侍侯覽小黄門段珪皆有田業近濟北界【近其靳翻濟子禮翻】僕從賓客劫掠行旅濟北相滕延一切收捕殺數十人陳尸路衢覽珪以事訴帝延坐徵詣廷尉免左悺兄勝為河東太守皮氏長京兆趙岐恥之【皮氏縣屬河東郡賢曰故城在今絳州龍門縣西長知兩翻】即日棄官西歸唐衡兄玹為京兆尹【玹音玄】素與岐有隙收岐家屬宗親陷以重灋盡殺之岐逃難四方【難乃旦翻】靡所不歷自匿姓名賣餅北海市中安丘孫嵩見而異之【安丘縣屬北海郡】載與俱歸藏於複壁中及諸唐死遇赦乃敢出【今孟子古註岐所註也其發題辭亦敘逃難之事】 閏月西羌餘衆復與燒何大豪寇張掖【復扶又翻】晨薄校尉段熲軍熲下馬大戰至日中刀折矢盡【折而設翻】虜亦引退熲追之且鬬且行晝夜相攻割肉食雪四十餘日遂至積石山【郡國志積石山在隴西郡河關縣西南賢曰積石山在今鄯州龍支縣南禹貢云導河積石即此是也】出塞二千餘里斬燒何大帥降其餘衆而還【帥師類翻降戶江翻下同】 夏五月甲戌漢中山崩 六月辛丑司徒祝恬薨 秋七月以司空盛允為司徒太常虞放為司空 長沙蠻反屯益陽零陵蠻寇長沙 九真餘賊屯據日南衆轉強盛詔復拜桂陽太守夏方為交趾刺史【復扶又翻夏戶雅翻】方威惠素著冬十一月日南賊二萬餘人相率詣方降 勒姐零吾種羌圍允街【姐音紫又且也翻零音憐種章勇翻允音鈆】段熲擊破之 泰山賊叔孫無忌攻殺都尉侯章遣中郎將宗資討破之詔徵皇甫規拜泰山太守規到官廣設方畧寇虜悉平<br />
<br />
  四年春正月辛酉南宫嘉德殿火戊子丙署火【百官志丙署長七人秩四百石黄綬宦者為之主中宫别處】大疫 二月壬辰武庫火 司徒盛允免以大司農种暠為司徒 【考異曰袁紀在去年按祝恬薨後有盛允允免暠為司徒相去半年袁紀誤也今從范書】 三月太尉黄瓊免 夏四月以太常沛國劉矩為太尉初矩為雍丘令【雍丘屬陳留郡故杞國也】以禮讓化民有訟者常引之於前提耳訓告以為忿恚可忍【恚於避翻】縣官不可入使歸更思訟者感之輒各罷去 甲寅封河間孝王子參戶亭侯博為任城王奉孝王後【賢曰杜預註左傳曰今丹水縣北有三戶亭故城在今鄧州内鄉縣西南元嘉元年任城王崇薨無子國絶今以博紹封河間孝王開也任城孝王尚也】 五月辛酉有星孛于心【晉書天文志心三星中星曰明堂天子位前星為太子後星為庶子孛蒲内翻】 丁卯原陵長壽門火【原陵光武陵】 己卯京師雨雹【雨于具翻】 六月京兆扶風及涼州地震 庚子岱山及博尤來山並穨裂【岱山在博縣西北賢曰徂來山一名尤來山博今博城縣余按二山並在博縣界而先書岱山以尤來山繫之博者岱宗人皆知之而尤來山則容有不知其在博縣界者故書法如此】 己酉赦天下 司空虞放免以前太尉黄瓊為司空 犍為屬國夷寇鈔百姓【永初元年以犍為南部都尉為犍為屬國都尉領朱提漢陽二縣犍居言翻】益州刺史山昱擊破之【姓譜山古烈山氏之後一曰周有山師掌山林後以官為氏】 零吾羌與先零諸種反寇三輔【種章勇翻】 秋七月京師雩【公羊傳曰雩旱祭也】減公卿已下奉貣王侯半租【孔穎達曰已與以字本同洪氏隸釋曰濟隂太守孟郁修堯廟碑其文有曰非所以表神聖曰以一太牢春秋秩祠曰是以好道之徒自遠方集其字皆作以曰已章聖德曰敦我已德厲我已仁字皆作已已以義同而字搆異體足以知自漢至唐已以二字通用矣奉扶用翻貣吐得翻假借也】占賣關内侯【占之贍翻】虎賁羽林緹騎營士五大夫錢各有差【緹他弟翻又音啼】 九月司空黄瓊免以大鴻臚東萊劉寵為司空寵嘗為會稽太守【會工外翻守式又翻】簡除煩苛禁察非法郡中大治【治直吏翻】徵為將作大匠山隂縣有五六老叟自若邪山谷間出【賢曰若邪在今越州會稽縣東南邪讀曰耶】人齎百錢以送寵曰山谷鄙生未嘗識郡朝【朝直遙翻郡聽事曰郡朝公府聽事曰府朝】他守時吏發求民間至夜不絶或狗吠竟夕民不得安自明府下車以來狗不夜吠民不見吏年老遭值聖明今聞當見棄去故自扶奉送寵曰吾政何能及公言邪勤苦父老為人選一大錢受之【今越州城西四十五里錢清鎮即父老送寵處為于偽翻】 冬先零沈氐羌與諸種羌寇并涼二州【種章勇翻下同】校尉段熲將湟中義從討之【湟中有義從胡即小月氐胡也從才用翻】涼州刺史郭閎貪共其功稽固熲軍【賢曰稽固猶停留也】使不得進義從役久戀鄉舊皆悉叛歸郭閎歸罪於熲熲坐徵下獄輸作左校【下遐稼翻】以濟南相胡閎代為校尉胡閎無威略羌遂陸梁覆沒營塢【賢曰說文曰塢小障也一曰庳城也音烏古翻】轉相招結唐突諸郡寇患轉盛泰山太守皇甫規上疏曰今猾賊就滅泰山略平復聞羣羌並皆反逆【復扶又翻】臣生長邠岐年五十有九【長知兩翻邠悲中翻】昔為郡吏再更叛羌豫籌其事有誤中之言【謂知馬賢必敗也事見五十二卷順帝永和五年更工衡翻中竹仲翻】臣素有痼疾恐犬馬齒窮不報大恩願乞宂官【宂而隴翻】備單車一介之使勞來三輔【使疏吏翻勞力到翻來力代翻】宣國威澤以所習地形兵埶佐助諸軍臣窮居孤危之中坐觀郡將已數十年自鳥鼠至于東岱其病一也【賢曰郡將郡守也鳥鼠山名在今渭州西即先零羌寇鈔處也東岱為泰山叔孫無忌反處也皆由郡守不加綏撫致使反叛其病一也爾雅翼鳥鼠同穴之中渭水出焉其鳥為鵌其鼠為鼵鼵如人家鼠而短尾䳜似鵽而小黄黑色入地三四尺鼠在内鳥在外在隴西首陽縣沙州記云寒嶺去太陽川三十里有鳥鼠同穴之山將即亮翻】力求猛敵不如清平勤明孫吳未若奉法【賢曰言若求猛敵不如撫以清平之政明習兵書不如郡守奉法使之無反也】前變未遠臣誠戚之【賢曰戚憂也前變謂羌反】是以越職盡其區區詔以規為中郎將持節監關西兵討零吾等【監古銜翻】十一月規擊羌破之斬首八百級先零諸種羌慕規威信相勸降者十餘萬【降戶江翻下同】<br />
<br />
  五年春正月壬午南宫丙署火 三月沈氐羌寇張掖酒泉皇甫規發先零諸種羌共討隴右【零音憐】而道路隔絶軍中大疫死者十三四規親入庵廬【庵草屋廬寄舍也毛晃曰結草木曰菴在野曰廬】巡視將士三軍感悦東羌遂遣使乞降涼州復通【復扶乂翻下同】先是安定太守孫雋受取狼藉屬國都尉李翕督軍御史張禀多殺降羌【李翕蓋安定屬國都尉然志無安定屬國以御史督軍故曰督軍御史先悉薦翻爾雅翼狼貪猛之獸聚物而不整故稱狼籍】涼州刺史郭閎漢陽太守趙熹並老弱不任職【任音壬】而皆倚恃權貴不遵法度規到悉條奏其罪或免或誅羌人聞之翕然反善沈氐大豪滇昌飢恬等十餘萬口復詣規降【滇音顛復扶又翻】 夏四月長沙賊起寇桂陽蒼梧 乙丑恭陵東闕火【恭陵安帝陵】戊辰虎賁掖門火【賁音奔】五月康陵園寢火【康陵殤帝陵】 長沙零陵賊入桂陽蒼梧南海交趾刺史及蒼梧太守望風逃犇遣御史中丞盛修督州郡募兵討之不能克 乙亥京師地震 甲申中藏府丞祿署火【百官志中藏府掌中幣帛金銀諸貨物】秋七月己未南宫承善闥火 鳥吾羌寇漢陽隴西金城諸郡兵討破之 艾縣賊攻長沙郡縣【艾縣屬豫章郡賢曰故城在今洪州建昌縣按今洪州分寧本漢艾縣又按宋白續通典分寧縣本武寧縣武寧縣本漢西安縣西安縣後漢建安中分海昏縣立而建昌縣乃永元中分海昏立在建安之前當是時艾縣故在宋元嘉二年廢海昏移建昌居焉艾故城在建昌界賢注是也】殺益陽令衆至萬餘人謁者馬睦督荆州刺史劉度撃之軍敗睦度犇走零陵蠻亦反冬十月武陵蠻反寇江陵南郡太守李肅犇走主簿胡爽扣馬首諫曰蠻夷見郡無儆備故敢乘間而進【間古莧翻】明府為國大臣連城千里舉旗鳴鼓應聲十萬奈何委符守之重而為逋逃之人乎肅拔刃向爽曰掾促去【掾俞絹翻】太守今急何暇此計爽抱馬固諫肅遂殺爽而走帝聞之徵肅棄市度睦減死一等復爽門閭【復方目翻除其賦役也】拜家一人為郎尚書朱穆舉右校令山陽度尚為荆州刺史【右校令掌右工徒秩六百石屬將作大匠趙明誠金石録有荆州刺史度尚碑云其先出自顓頊與楚同姓熊缺之後又曰統國法度按元和姓纂古掌度之官因以命氏不言其與楚同姓也】辛丑以太常馮緄為車騎將軍將兵十餘萬討武陵蠻【緄古本翻 考異曰帝紀三年十二月武陵蠻寇江陵車騎將軍馮緄討皆降散荆州刺史度尚討長沙蠻平之此事當在今年三月重出誤也】先是所遣將帥宦官多陷以折耗軍資往往抵罪【先悉薦翻折而設翻】緄願請中常侍一人監軍財費尚書朱穆奏緄以財自嫌失大臣之節有詔勿劾【監古衘翻劾戶槩翻又戶得翻】緄請前武陵太守應奉與俱拜從事中郎【將軍出征從事中郎職參謀議】十一月緄軍至長沙賊聞之悉詣營乞降進撃武陵蠻夷斬首四千餘級受降十餘萬人荆州平定【降戶江翻】詔書賜錢一億固讓不受振旅還京師推功於應奉薦以為司隸校尉而上書乞骸骨朝廷不許 滇那羌寇武威張掖酒泉【滇音顛】 太尉劉矩免以太常楊秉為太尉 皇甫規持節為將還督鄉里既無他私惠而多所舉奏又惡絶宦官不與交通【惡烏路翻】於是中外並怨遂共誣規貨賂羣羌令其文降【賢曰謂以文簿虚降非真心也降戶江翻】帝璽書誚讓相屬【屬之欲翻】規上書自訟曰四年之秋戎醜蠢戾【賢曰蠢動也戾乖也】舊都懼駭【舊都謂長安】朝廷西顧臣振國威靈羌戎稽首【稽音啟】所省之費一億以上以為忠臣之義不敢告勞【詩小雅曰密勿從事不敢告勞無罪無辜讒口囂囂】故恥以片言自及微効然比方先事【賢曰先事謂前輩敗將也】庶免罪悔前踐州界先奏孫雋李翕張禀旋師南征又上郭閎趙熹陳其過惡埶據大辟【上時掌翻辟毗亦翻】凡此五臣支黨半國其餘墨綬下至小吏所連及者復有百餘吏託報將之怨【郡守謂之郡將復扶又翻將即亮翻】子思復父之恥載贄馳車懷糧步走交構豪門競流謗讟云臣私報諸羌讎以錢貨【讎是周翻償也】若臣以私財則家無擔石【擔都濫翻】如物出於官則文簿易考【易以豉翻】就臣愚惑信如言者前世尚遺匈奴以宫姬【謂元帝以王昭君賜呼韓邪單于也遺于季翻】鎮烏孫以公主【謂武帝以江都王建女細君妻烏孫王昆莫也】今臣但費千萬以懷叛羌則良臣之才畧兵家之所貴將有何罪負義違理乎自永初以來將出不少【將出即亮翻少詩沼翻】覆軍有五【謂鄧隲敗於冀西任尚敗於平襄司馬鈞敗於丁奚城馬賢敗於射姑山趙冲敗於鸇隂河】動資巨億有旋車完封【賢曰言覆軍之將旋師之日多載珍寶封印完全便入權門余謂此言以朝廷供軍之金幣不發封識而輸之權門也】寫之權門而名成功立厚加爵封今臣還督本土糾舉諸郡絶交離親戮辱舊故衆謗隂害固其宜也帝乃徵規還拜議郎論功當封而中常侍徐璜左悺欲從求貨數遣賓客就問功狀【數所角翻】規終不答璜等忿怒陷以前事【前事即誣毁之事也】下之於吏官屬欲賦斂請謝【下遐稼翻斂力贍翻】規誓而不聽遂以餘寇不絶坐繫廷尉論輸左校【校戶教翻】諸公及太學生張鳳等三百餘人詣闕訟之會赦歸家<br />
<br />
  六年春二月戊午司徒种暠薨 三月戊戌赦天下以衛尉潁川許栩為司徒 夏四月辛亥康陵東署火五月鮮卑寇遼東屬國 秋七月甲申平陵園寢火【平陵昭帝陵】 桂陽賊李研等寇郡界武陵蠻復反太守陳奉討平之宦官素惡馮緄【復扶又翻惡烏路翻】八月緄坐軍還盜賊復發免 冬十月丙辰上校獵廣成遂幸函谷關上林苑光祿勲陳蕃上疏諫曰安平之時遊畋宜有節况今有三空之戹哉田野空朝廷空倉庫空加之兵戎未戢四方離散是陛下焦心毁顏坐以待旦之時也【毁顏謂面有憂色臨于臣民之上無以為顏也】豈宜揚旗曜武騁心輿馬之觀乎又前秋多雨民始種麥今失其勸種之時而令給驅禽除路之役非賢聖恤民之意也書奏不納 十一月司空劉寵免十二月以衛尉周景為司空景榮之孫也時宦官方熾景與太尉楊秉上言内外吏職多非其人舊典中臣子弟不得居位秉埶而今枝葉賓客【枝葉謂中臣族親也】布列職署【署官舍也】或年少庸人典據守宰上下忿患四方愁毒可遵用舊章退貪殘塞災謗【塞悉則翻】請下司隸校尉中二千石城門五營校尉北軍中候各實覈所部【司隸校尉部三輔三河弘農中二千石列卿也各率其屬城門校尉部十二城門司馬門候五營校尉屯騎越騎步兵長水射聲也各有司馬員吏北軍中候掌監五營下遐稼翻】應當斥罷自以狀言三府廉察有遺漏續上【言各官實覈所部以當斥罷者言之公府更察其遺漏者續上狀使無有佚罰者上時掌翻】帝從之於是秉條奏牧守青州刺史羊亮等五十餘人或死或免天下莫不肅然 詔徵皇甫規為度遼將軍初張奐坐梁冀故吏免官禁錮凡諸交舊莫敢為言唯規薦舉前後七上【為于偽翻上時掌翻】由是拜武威太守及規為度遼到營數月上書薦奐才畧兼優宜正元帥【元帥謂度遼將軍也】以從衆望若猶謂愚臣宜充舉事者願乞宂官以為奐副朝廷從之以奐代規為度遼將軍以規為使匈奴中郎將【使疏吏翻】 西州吏民守闕為前護羌校尉段熲訟寃者甚衆會滇那等諸種羌益熾涼州幾亡【滇音顛種章勇翻幾居希翻】乃復以熲為護羌校尉 尚書朱穆疾宦官恣横【横戶孟翻】上疏曰按漢故事中常侍參選士人建武以後乃悉用宦者自延平以來浸益貴盛假貂璫之飾處常伯之任【賢曰璫以金為之當冠前附以金蟬也漢官儀曰中常侍秦官也漢興或用士人銀璫左貂光武以後專任宦者右貂金璫常伯侍中處昌呂翻】天朝政事一更其手【朝直遙翻更工衡翻】權傾海内寵貴無極子弟親戚並荷榮任【荷下可翻】放濫驕溢莫能禁禦窮破天下空竭小民愚臣以為可悉罷省遵復往初更選海内清淳之士明達國體者以補其處即兆庶黎萌蒙被聖化矣【被皮義翻】帝不納後穆因進見【見賢遍翻】復口陳曰臣聞漢家舊典置侍中中常侍各一人省尚書事【復扶又翻賢曰省覽也省悉井翻】黄門侍郎一人傳發書奏【賢曰傳通也】皆用姓族【賢曰引用士人有族望者】自和熹太后以女主稱制不接公卿乃以閹人為常侍小黄門通命兩宫自此以來權傾人主窮困天下宜皆罷遣博選耆儒宿德與參政事帝怒不應穆伏不肯起左右傳出【賢曰傳聲令出】良久乃趨而去自此中官數因事稱詔詆毁之【數所角翻】穆素剛不得意居無幾憤懣發疽卒【幾居豈翻】<br />
<br />
  資治通鑑卷五十四<br />
<br />
<史部,編年類,資治通鑑>  <br>
   </div> 

<script src="/search/ajaxskft.js"> </script>
 <div class="clear"></div>
<br>
<br>
 <!-- a.d-->

 <!--
<div class="info_share">
</div> 
-->
 <!--info_share--></div>   <!-- end info_content-->
  </div> <!-- end l-->

<div class="r">   <!--r-->



<div class="sidebar"  style="margin-bottom:2px;">

 
<div class="sidebar_title">工具类大全</div>
<div class="sidebar_info">
<strong><a href="http://www.guoxuedashi.com/lsditu/" target="_blank">历史地图</a></strong>  
<a href="http://www.880114.com/" target="_blank">英语宝典</a>  
<a href="http://www.guoxuedashi.com/13jing/" target="_blank">十三经检索</a> 
<br><strong><a href="http://www.guoxuedashi.com/gjtsjc/" target="_blank">古今图书集成</a></strong> 
<a href="http://www.guoxuedashi.com/duilian/" target="_blank">对联大全</a> <strong><a href="http://www.guoxuedashi.com/xiangxingzi/" target="_blank">象形文字典</a></strong> 

<br><a href="http://www.guoxuedashi.com/zixing/yanbian/">字形演变</a>  <strong><a href="http://www.guoxuemi.com/hafo/" target="_blank">哈佛燕京中文善本特藏</a></strong>
<br><strong><a href="http://www.guoxuedashi.com/csfz/" target="_blank">丛书&方志检索器</a></strong> <a href="http://www.guoxuedashi.com/yqjyy/" target="_blank">一切经音义</a>  

<br><strong><a href="http://www.guoxuedashi.com/jiapu/" target="_blank">家谱族谱查询</a></strong>  <strong><a href="http://shufa.guoxuedashi.com/sfzitie/" target="_blank">书法字帖欣赏</a></strong> 
<br>

</div>
</div>


<div class="sidebar" style="margin-bottom:0px;">

<font style="font-size:22px;line-height:32px">QQ交流群9:489193090</font>


<div class="sidebar_title">手机APP 扫描或点击</div>
<div class="sidebar_info">
<table>
<tr>
	<td width=160><a href="http://m.guoxuedashi.com/app/" target="_blank"><img src="/img/gxds-sj.png" width="140"  border="0" alt="国学大师手机版"></a></td>
	<td>
<a href="http://www.guoxuedashi.com/download/" target="_blank">app软件下载专区</a><br>
<a href="http://www.guoxuedashi.com/download/gxds.php" target="_blank">《国学大师》下载</a><br>
<a href="http://www.guoxuedashi.com/download/kxzd.php" target="_blank">《汉字宝典》下载</a><br>
<a href="http://www.guoxuedashi.com/download/scqbd.php" target="_blank">《诗词曲宝典》下载</a><br>
<a href="http://www.guoxuedashi.com/SiKuQuanShu/skqs.php" target="_blank">《四库全书》下载</a><br>
</td>
</tr>
</table>

</div>
</div>


<div class="sidebar2">
<center>


</center>
</div>

<div class="sidebar"  style="margin-bottom:2px;">
<div class="sidebar_title">网站使用教程</div>
<div class="sidebar_info">
<a href="http://www.guoxuedashi.com/help/gjsearch.php" target="_blank">如何在国学大师网下载古籍?</a><br>
<a href="http://www.guoxuedashi.com/zidian/bujian/bjjc.php" target="_blank">如何使用部件查字法快速查字?</a><br>
<a href="http://www.guoxuedashi.com/search/sjc.php" target="_blank">如何在指定的书籍中全文检索?</a><br>
<a href="http://www.guoxuedashi.com/search/skjc.php" target="_blank">如何找到一句话在《四库全书》哪一页?</a><br>
</div>
</div>


<div class="sidebar">
<div class="sidebar_title">热门书籍</div>
<div class="sidebar_info">
<a href="/so.php?sokey=%E8%B5%84%E6%B2%BB%E9%80%9A%E9%89%B4&kt=1">资治通鉴</a> <a href="/24shi/"><strong>二十四史</strong></a>&nbsp; <a href="/a2694/">野史</a>&nbsp; <a href="/SiKuQuanShu/"><strong>四库全书</strong></a>&nbsp;<a href="http://www.guoxuedashi.com/SiKuQuanShu/fanti/">繁体</a>
<br><a href="/so.php?sokey=%E7%BA%A2%E6%A5%BC%E6%A2%A6&kt=1">红楼梦</a> <a href="/a/1858x/">三国演义</a> <a href="/a/1038k/">水浒传</a> <a href="/a/1046t/">西游记</a> <a href="/a/1914o/">封神演义</a>
<br>
<a href="http://www.guoxuedashi.com/so.php?sokeygx=%E4%B8%87%E6%9C%89%E6%96%87%E5%BA%93&submit=&kt=1">万有文库</a> <a href="/a/780t/">古文观止</a> <a href="/a/1024l/">文心雕龙</a> <a href="/a/1704n/">全唐诗</a> <a href="/a/1705h/">全宋词</a>
<br><a href="http://www.guoxuedashi.com/so.php?sokeygx=%E7%99%BE%E8%A1%B2%E6%9C%AC%E4%BA%8C%E5%8D%81%E5%9B%9B%E5%8F%B2&submit=&kt=1"><strong>百衲本二十四史</strong></a>  <a href="http://www.guoxuedashi.com/so.php?sokeygx=%E5%8F%A4%E4%BB%8A%E5%9B%BE%E4%B9%A6%E9%9B%86%E6%88%90&submit=&kt=1"><strong>古今图书集成</strong></a>
<br>

<a href="http://www.guoxuedashi.com/so.php?sokeygx=%E4%B8%9B%E4%B9%A6%E9%9B%86%E6%88%90&submit=&kt=1">丛书集成</a> 
<a href="http://www.guoxuedashi.com/so.php?sokeygx=%E5%9B%9B%E9%83%A8%E4%B8%9B%E5%88%8A&submit=&kt=1"><strong>四部丛刊</strong></a>  
<a href="http://www.guoxuedashi.com/so.php?sokeygx=%E8%AF%B4%E6%96%87%E8%A7%A3%E5%AD%97&submit=&kt=1">說文解字</a> <a href="http://www.guoxuedashi.com/so.php?sokeygx=%E5%85%A8%E4%B8%8A%E5%8F%A4&submit=&kt=1">三国六朝文</a>
<br><a href="http://www.guoxuedashi.com/so.php?sokeytm=%E6%97%A5%E6%9C%AC%E5%86%85%E9%98%81%E6%96%87%E5%BA%93&submit=&kt=1"><strong>日本内阁文库</strong></a> <a href="http://www.guoxuedashi.com/so.php?sokeytm=%E5%9B%BD%E5%9B%BE%E6%96%B9%E5%BF%97%E5%90%88%E9%9B%86&ka=100&submit=">国图方志合集</a> <a href="http://www.guoxuedashi.com/so.php?sokeytm=%E5%90%84%E5%9C%B0%E6%96%B9%E5%BF%97&submit=&kt=1"><strong>各地方志</strong></a>

</div>
</div>


<div class="sidebar2">
<center>

</center>
</div>
<div class="sidebar greenbar">
<div class="sidebar_title green">四库全书</div>
<div class="sidebar_info">

《四库全书》是中国古代最大的丛书,编撰于乾隆年间,由纪昀等360多位高官、学者编撰,3800多人抄写,费时十三年编成。丛书分经、史、子、集四部,故名四库。共有3500多种书,7.9万卷,3.6万册,约8亿字,基本上囊括了古代所有图书,故称“全书”。<a href="http://www.guoxuedashi.com/SiKuQuanShu/">详细>>
</a>

</div> 
</div>

</div>  <!--end r-->

</div>
<!-- 内容区END --> 

<!-- 页脚开始 -->
<div class="shh">

</div>

<div class="w1180" style="margin-top:8px;">
<center><script src="http://www.guoxuedashi.com/img/plus.php?id=3"></script></center>
</div>
<div class="w1180 foot">
<a href="/b/thanks.php">特别致谢</a> | <a href="javascript:window.external.AddFavorite(document.location.href,document.title);">收藏本站</a> | <a href="#">欢迎投稿</a> | <a href="http://www.guoxuedashi.com/forum/">意见建议</a> | <a href="http://www.guoxuemi.com/">国学迷</a> | <a href="http://www.shuowen.net/">说文网</a><script language="javascript" type="text/javascript" src="https://js.users.51.la/17753172.js"></script><br />
  Copyright &copy; 国学大师 古典图书集成 All Rights Reserved.<br>
  
  <span style="font-size:14px">免责声明:本站非营利性站点,以方便网友为主,仅供学习研究。<br>内容由热心网友提供和网上收集,不保留版权。若侵犯了您的权益,来信即刪。scp168@qq.com</span>
  <br />
ICP证:<a href="http://www.beian.miit.gov.cn/" target="_blank">鲁ICP备19060063号</a></div>
<!-- 页脚END --> 
<script src="http://www.guoxuedashi.com/img/plus.php?id=22"></script>
<script src="http://www.guoxuedashi.com/img/tongji.js"></script>

</body>
</html>
