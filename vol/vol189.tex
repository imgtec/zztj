






























































資治通鑑卷一百八十九

宋 司馬光 撰

胡三省 音注

唐紀五|{
	起重光大荒落三月盡十二月不滿一年}


高祖神堯大聖光孝皇帝中之中

武德四年三月庚申以靺鞨渠帥突地稽為燕州揔管|{
	靺鞨有七種粟末靺鞨居最南本附高麗隋煬帝初其渠帥突地稽率其部來降居之柳城新志曰隋於營州之境汝羅故城置遼西郡以處靺鞨降人武德元年曰燕州突地稽隋書作度地稽帥所類翻靺音末鞨音曷燕因肩翻}
太子建成獲稽胡千餘人釋其酋帥數十人|{
	酋才由翻帥所類翻}
授以官爵使還招其餘黨劉仚成亦降|{
	仚許延翻降戶江翻下同}
建成詐稱增置州縣築城邑命降胡年二十以上皆集以兵圍而殺之死者六千餘人 |{
	考異曰實録前言四千餘戶後云六千餘計盖前言戶後言口也}
仚成覺變亡奔梁師都 行軍總管劉世讓攻竇建德黄州拔之|{
	黄州闕}
洺州嚴備世讓不得進會突厥將入寇上召世讓還竇建德所署普樂令平恩程名振來降上遥除名振永寧令|{
	新志平恩縣屬洺州又所領雞澤縣有普樂縣竇建德平後廢入鷄澤永寧縣屬洺州本熊耳義寧二年更名時屬能州按舊書除名振永年令此承新書之誤永年漢廣平縣也隋仁壽元年改曰永年帶洺州舊志曰永年本漢曲梁縣地杜佑曰洺州春秋赤狄之地洺彌并翻厥九勿翻還從宣翻又音如字樂音洛}
使將兵徇河北名振夜襲鄴|{
	舊志鄴縣屬相州後魏於鄴置相州周末尉遲迥既平乃焚鄴以安陽為相州理所煬帝復於鄴故都大慈寺置鄴縣將即亮翻}
俘其男女千餘人去鄴八十里閲婦人乳有湩者九十餘人悉縱遣之鄴人感其仁為之飯僧|{
	湩竹用翻乳汁為于偽翻飯扶晩翻}
突厥頡利可汗承父兄之資|{
	頡利者啓民之子始畢處羅之弟厥九勿翻可從刋入聲汗音寒}
士馬雄盛有憑陵中國之志妻隋義成公主公主從弟善經|{
	從才用翻}
避亂在突厥與王世充使者王文素共說頡利曰|{
	使疏吏翻下同說式芮翻}
昔啓民為兄弟所逼脱身奔隋賴文皇帝之力有此土宇|{
	事見隋文帝紀}
子孫享之今唐天子非文皇帝子孫可汗宜奉楊政道以伐之|{
	楊政道時居定襄}
以報文皇帝之德頡利然之上以中國未寧待突厥甚厚而頡利求請無厭|{
	厭於鹽翻}
言辭驕慢甲戌突厥寇汾隂|{
	汾隂縣木屬蒲州時為泰州治所}
唐兵圍洛陽掘塹築壘而守之|{
	塹七艶翻}
城中乏食絹一匹直粟三升布十匹直鹽一升服飾珍玩賤如土芥民食草根木葉皆盡相與澄取浮泥投米屑作餅食之皆病身腫脚弱死者相枕倚於道|{
	枕職任翻}
皇泰主之遷民入宫城也|{
	見一百八十三卷隋義寧元年四月}
凡三萬家至是無三千家雖貴為公卿糠覈不充|{
	孟康曰覈麥糠中不破者也晉灼曰覈音紇京師人謂麄屑為紇頭}
尚書郎以下親自負戴|{
	負以肩背戴以首}
往往餒死竇建德使其將范願守曹州|{
	將即亮翻下同}
悉發孟海公徐圓朗之衆西救洛陽至滑州王世充行臺僕射韓洪開門納之己卯軍于酸棗|{
	酸棗縣隋屬鄭州此時屬東梁州}
壬午突厥寇石州|{
	石州隋之離石郡}
刺史王集擊却之 竇建德䧟管州殺刺史郭士安又䧟滎陽陽翟等縣|{
	榮陽縣屬鄭州陽翟縣隋屬汝州時屬嵩州}
水陸並進汎舟運糧沂河西上|{
	上時掌翻}
王世充之弟徐州行臺世辯|{
	徐州隋之彭城郡}
遣其將郭士衡將兵數千會之|{
	將即亮翻}
合十餘萬號三十萬軍於成臯之東原築宫板渚|{
	成臯即虎牢東原即東廣武水經河水過成臯而東合汜水乂東逕板城北注云有津謂之板城渚口}
遣使與王世充相聞先是建德遺秦王世民書|{
	使疏吏翻先悉薦翻遺于偽翻}
請退軍潼關返鄭侵地復修前好|{
	好呼到翻}
世民集將佐議之|{
	將即亮翻}
皆請避其鋒郭孝恪曰世充窮蹙垂將面縳建德遠來助之此天意欲兩亡之也宜據武牢之險以拒之|{
	唐諱虎改虎牢為武牢}
伺間而動破之必矣|{
	間古莧翻}
記室薛收曰世充保據東都府庫充實所將之兵皆江淮精鋭即日之患但乏糧食耳以是之故為我所持求戰不得守則難久建德親帥大衆遠來赴援|{
	帥讀曰率}
亦當極其精鋭若縱之至此兩寇合從|{
	從子容翻}
轉河北之粟以饋洛陽則戰爭方始偃兵無日混一之期殊未有涯也今宜分兵守洛陽深溝高壘世充出兵慎勿與戰大王親帥驍鋭先據成臯|{
	帥讀曰率驍堅堯翻}
厲兵訓士以待其至以逸待勞决可克也建德既破世充自下不過二旬兩主就縳矣世民善之收道衡之子也|{
	薛道衡為隋煬帝所殺隋之伐陳道衡知其必克收之識時審勢盖有父風}
蕭瑀屈突通封德彞皆曰吾兵疲老世充憑守堅城未易猝抜|{
	瑀音禹屈居勿翻易以䜴翻下同}
建德席勝而來鋒鋭氣盛吾腹背受敵非完策也不若退保新安以承其弊世民曰世充兵摧食盡上下離心不煩力攻可以坐克建德新破海公將驕卒惰吾據武牢扼其咽㗋|{
	將即亮翻下同咽音煙}
彼若冒險爭鋒吾取之甚易|{
	易弋䜴翻}
若狐疑不戰旬月之間世充自潰城破兵彊氣勢自倍一舉兩克在此行矣若不速進賊入武牢諸城新附必不能守兩賊併力其勢必彊何弊之承吾計决矣通等又請解圍據險以觀其變世民不許中分麾下使通等副齊王元吉圍守東都世民將驍勇三千五百人東趣武牢|{
	驍堅堯翻趣七喻翻又逡須翻}
時正晝出兵歷北邙抵河陽趨鞏而去|{
	鞏在東都之東一百一十里時世民大軍據都城西北以臨世充而圍之故出兵向武牢歷北邙抵河陽而趨□趨與趣同音七喻翻}
王世充登城望見莫之測也竟不敢出癸未世民入武牢甲申將驍騎五百出武牢東二十餘里覘建德之營|{
	覘丑廉翻又丑艶翻}
緣道分留從騎|{
	從才用翻下同}
使李世勣程知節秦叔寶分將之伏於道旁纔餘四騎與之偕進世民謂尉遟敬德曰|{
	騎奇寄翻尉紆勿翻}
吾執弓矢公執槊相隨|{
	槊色角翻}
雖百萬衆若我何又曰賊見我而還上策也|{
	還從宣翻}
去建德營三里所建德遊兵遇之以為斥候也世民大呼曰我秦王也引弓射之|{
	呼火故翻射而亦翻下同}
斃其一將|{
	將即亮翻}
建德軍中大驚出五六千騎逐之從者咸失色|{
	從才用翻}
世民曰汝弟前行吾自與敬德為殿|{
	弟大計翻但也漢書多用此弟字可考也殿□練翻}
於是按轡徐行追騎將至則引弓射之輒斃一人追者懼而止止而復來|{
	復扶又翻下同}
如是再三每來必有斃者世民前後射殺數人敬德殺十許人追者不敢復逼世民逡巡稍却以誘之|{
	逡七荀翻誘音酉下同}
入於伏内世勣等奮擊大破之斬首三百餘級獲其驍將殷秋石瓚以歸|{
	瓚藏旱翻}
乃為書報建德論以趙魏之地久為我有為足下所侵奪但以淮安見禮公主得歸故相與坦懷釋怨|{
	武德二年竇建德盡取趙魏虜淮安王神通及同安公主待淮安以客禮㳄年八月遣公主歸}
世充頃與足下修好已嘗反覆 |{
	武德二年王竇結好世充簒建德絶之尋有疆場之爭好呼到翻}
今亡在朝夕更飾辭相誘足下乃以三軍之衆仰哺他人千金之資坐供外費|{
	兵法曰興師十萬日費千金}
良非上策今前茅相遇彼遽崩摧|{
	左傳随武子曰前茅慮無杜預注云軍行前有斥候蹹伏茅明也備慮有無也或曰以茅為旌識}
郊勞未通能無懷愧|{
	古者諸侯相見有郊勞之禮言建德來救世充阻於唐兵使命不得通也勞力到翻}
故抑止鋒鋭冀聞擇善|{
	欲使之擇善而從}
若不獲命恐雖悔難追 立秦王世民之子泰為衛王 夏四月己丑豐州揔管張長遜入朝時言事者多云長遜久居豐州|{
	張長遜隋末守豐州唐興來降至是入朝豐州至長安二千六百六里朝直遥翻下同}
為突厥所厚非國家之利|{
	厥九勿翻}
長遜聞之請入朝上許之會太子建成北伐稽胡長遜帥所部會之因入朝拜右武侯將軍|{
	帥讀曰率下同}
益州行臺左僕射竇軌帥巴蜀兵來會秦王撃王世充以長遜檢校益州行臺右僕射 己亥突厥頡利可汗寇鴈門李大恩擊走之|{
	可從刋入聲汗音寒}
壬寅王世充騎將楊公卿單雄信引兵出戰|{
	騎奇寄翻將即亮翻單音善}
齊王元吉擊之不利行軍揔管盧君諤戰死 太子還長安 王世充平州刺史周仲隱以城來降|{
	洛州河隂縣古平隂也王世充當於此置平州降戶江翻}
戊申突厥寇并州初處羅可汗與劉武周相表裏寇井州上遣太常卿鄭元璹往諭以禍福處羅不從未幾處羅遇疾卒|{
	處昌呂翻璹殊玉翻幾居豈翻卒子恤翻 考異曰舊書鄭元璹傳作叱羅可汗今從實録}
國人疑元璹毒之留不遣上又遣漢陽公瓌賂頡利可汗以金帛頡利欲令瓌拜瓌不從亦留之|{
	瓌古回翻}
又留左驍衛大將軍長孫順德|{
	驍堅堯翻長知兩翻}
上怒亦留其使者瓌孝恭之弟也|{
	孝恭時鎮夔州}
甲寅封皇子元方為周王元禮為鄭王元嘉為宋王元則為荆王元茂為越王 竇建德迫於武牢不得進留屯累月 |{
	考異曰舊書停留七十餘日新書六十餘日案二月戊午沈悦始以武牢降唐至五月己未建德敗纔六十二日若沈悦今日降唐明日建德即至亦不能自固又吳兢太宗勲史三月己卯建德率兵十二萬次于酸棗去敗纔四十一日故但云留屯累月}
戰數不利|{
	數所角翻}
將士思歸丁巳秦王世民遣王君廓將輕騎千餘抄其糧運|{
	抄楚交翻}
又破之獲其大將軍張青特凌敬言於建德曰大王悉兵濟河攻取懷州河陽使重將守之更鳴鼓建旗踰太行入上黨|{
	行戶剛翻}
徇汾晉趣蒲津|{
	趣七喻翻}
如此有三利一則蹈無人之境取勝可以萬全二則拓地收衆形勢益彊三則關中震駭鄭圍自解為今之策無以易此|{
	凌敬之策善矣當是時洛城危急秦王定計而堅守之盖計日而收功吾恐建德未得至蒲州洛城已破矣}
建德將從之而王世充遣使告急相繼於道王琬長孫安世朝夕涕泣請救洛陽|{
	使疏吏翻長知兩翻}
又隂以金玉啗建德諸將以撓其謀|{
	啗徒濫翻將即亮翻撓奴巧翻又奴教翻}
諸將皆曰凌敬書生安知戰事其言豈可用也建德乃謝敬曰今衆心甚鋭天贊我也因之决戰必將大捷不得從公言敬固爭之建德怒令扶出|{
	令力丁翻}
其妻曹氏謂建德曰祭酒之言不可違也|{
	凌敬盖為建德國子祭酒}
今大王自滏口乘唐國之虚連營漸進以取山北|{
	建德都洺州時在山南并代汾晋皆山北也滏音釡}
又因突厥西抄關中唐必還師自救|{
	厥九勿翻抄楚交翻還從宣翻又音如字}
鄭圍何憂不解若頓兵於此老師費財欲求成功在於何日建德曰此非女子所知吾來救鄭鄭今倒懸亡在朝夕吾乃捨之而去是畏敵而棄信也不可諜者告曰建德伺唐軍芻盡牧馬於河北將襲武牢|{
	諜逹協翻伺相吏翻}
五月戊午秦王世民北濟河南臨廣武|{
	此西廣武也}
察敵形勢因留馬千餘匹牧於河渚以誘之|{
	誘音酉}
夕還武牢己未建德果悉衆而至|{
	此所謂善戰者因其勢而利導之也}
自板渚出牛口置陳北距大河西薄汜水南屬鵲山|{
	水經注汜水南出浮戲山亦謂之方山北逕虎牢城東又北流注于河陳讀曰陣下同汜音祀屬之欲翻}
亘二十里鼓行而進諸將皆懼|{
	將即亮翻懼其衆也}
世民將數騎升高丘而望之|{
	將音如字領也騎奇寄翻}
謂諸將曰賊起山東未嘗見大敵今度險而|{
	囂虛驕翻喧也}
是無紀律逼城而陳有輕我心我按甲不出彼勇氣自衰陳久卒飢勢將自退|{
	所謂以計稽之也}
追而擊之無不克者與公等約甫過日中必破之矣|{
	甫始也纔也}
建德意輕唐軍遣三百騎涉汜水距唐營一里所止遣使與世民相聞曰請選銳士數百與之劇|{
	汜音祀使疏吏翻劇戲也今俗謂戲為則劇}
世民遣王君廓將長槊二百以應之|{
	槊色角翻}
相與交戰乍進乍退兩無勝負各引還|{
	還從宣翻又音如字}
王琬乘隋煬帝驄馬|{
	煬余尚翻馬青白曰驄}
鎧仗甚鮮迥出陳前以誇衆|{
	鎧可亥翻迥戶頂翻陳讀曰陣}
世民曰彼所乘真良馬也尉遟敬德請往取之世民止曰豈可以一馬喪猛士|{
	尉紆勿翻喪息浪翻}
敬德不從與高甑生梁建方三騎直入其陳擒琬引其馬馳歸衆無敢當者|{
	騎奇寄翻}
世民使召河北馬待其至乃出戰建德列陳自辰至午士卒飢倦皆坐列|{
	杜預曰士皆坐列言無闘志}
又爭飲水逡巡欲退|{
	逡七倫翻}
世民命宇文士及將三百騎經建德陳西馳而南上|{
	所以嘗敵也將即亮翻又音如字騎奇寄翻下同上時掌翻}
戒之曰賊若不動爾宜引歸動則引兵東出士及至陳前陳果動世民曰可擊矣時河渚馬亦至乃命出戰世民帥輕騎先進|{
	帥讀曰率下同}
大軍繼之東涉汜水直薄其陳|{
	薄迫也}
建德羣臣方朝謁唐騎猝來朝臣趨就建德建德召騎兵使拒唐兵騎兵阻朝臣不得過建德揮朝臣令却|{
	朝直遥翻}
進退之間唐兵已至建德窘廹退依東陂竇抗引兵擊之戰小不利世民帥騎赴之所向皆靡淮陽王道玄挻身䧟陳直出其後復突陳而歸|{
	窘渠隕翻復扶又翻又音如字}
再入再出飛矢集其身蝟毛|{
	蝟于貴翻蟲似豪猪而小爾雅曰彚毛刺是也}
勇氣不衰射人皆應弦而仆世民給以副馬使從己於是諸軍大戰塵埃漲天世民帥史大柰程知節秦叔寶宇文歆等卷斾而入|{
	歆許今翻卷讀曰捲}
出其陳後|{
	陳讀曰陣}
張唐旗幟|{
	幟昌志翻}
建德將士顧見之大潰|{
	將即亮翻}
追奔三十里斬首三千餘級建德中槊|{
	中竹仲翻槊色角翻}
竄匿於牛口渚車騎將軍白士讓楊武威逐之建德墜馬士讓援槊欲刺之|{
	騎奇寄翻援于元翻刺七亦翻}
建德曰勿殺我我夏王也能富貴汝|{
	言得我以獻則富貴也夏戶雅翻}
武威下擒之|{
	下馬擒之也}
載以從馬|{
	從才用翻}
來見世民世民讓之曰我自討王世充何預汝事而來越境犯我兵鋒建德曰今不自來恐煩遠取建德將士皆潰去所俘獲五萬人世民即日散遣之使還郷里封德彞入賀世民笑曰不用公言得有今日智者千慮不免一失乎|{
	用李左車之言}
德彞甚慙建德妻曹氏與左僕射齊善行將數百騎遁歸洺州|{
	洺彌并翻}
甲子世充偃師鞏縣皆降乙丑以太子左庶子鄭善果為山東道撫慰大使|{
	降戶江翻下同使疏吏翻}
世充將王德仁棄故洛陽城而遁|{
	此漢魏故都之城也}
亞將趙季卿以城降秦王世民囚竇建德王琬長孫安世郭士衡等至洛陽城下以示世充世充與建德語而泣仍遣安世等入城言敗狀世充召諸將議突圍南走襄陽|{
	欲走襄陽就王弘烈王泰走音奏}
諸將皆曰吾所恃者夏王夏王今已為擒雖得出終必無成 |{
	考異曰舊書世充傳云諸將皆不荅今從河洛記}
丙寅世充素服帥其太子羣臣二千餘人詣軍門降|{
	帥讀曰率降戶江翻}
世民禮接之世充俯伏流汗世民曰卿常以童子見處|{
	處昌呂翻}
今見童子何恭之甚邪|{
	邪音耶}
世充頓首謝罪於是部分諸軍|{
	分扶問翻}
先入洛陽分守市肆禁止侵掠無敢犯者丁卯世民入宫城命記室房玄齡先入中書門下省收隋圖籍制詔已為世充所毁無所獲命蕭瑀竇軌等封府庫收其金帛頒賜將士|{
	瑀音禹將即亮翻}
收世充之黨罪尤大者段達王隆崔洪丹薛德音楊汪孟孝義單雄信楊公卿郭什柱郭士衡董叡張童兒王德仁朱粲郭善才等十餘人斬於洛水之上|{
	單慈淺翻新書云薛德音以移檄慢逆崔弘丹以造弩多傷士前誅之次收段逹等斬洛渚上温公避國諱改弘丹為洪丹郭什柱意當作什住}
初李世勣與單雄信友善誓同生死及洛陽平世勣言雄信驍健絶倫|{
	驍堅堯翻}
請盡輸已之官爵以贖之世民不許 |{
	考異曰舊傳云高祖不許按太宗得洛城即誅雄信何嘗稟命於高祖盖太宗時史臣叙高祖時事有不厭衆心者皆稱高祖之命以掩太宗之失如□夏縣之類皆是也}
世勣固請不能得涕泣而退雄信曰我固知汝不辦事世勣曰吾不惜餘生與兄俱死但既以此身許國事無兩遂且吾死之後誰復視兄之妻子乎|{
	復扶又翻}
乃割股肉以㗖雄信曰使此肉隨兄為土庶幾不負昔誓也|{
	幾居希翻}
士民疾朱粲殘忍競投瓦礫擊其尸須臾如冢|{
	礫郎擊翻}
囚韋節楊續長孫安世等十餘人送長安|{
	長孫之長知兩翻}
士民無罪為世充所囚者皆釋之所殺者祭而誄之|{
	古者卿大夫殁則君命有司累其功德為文以哀之曰誄今誄之者哀其無罪而死也誄魯水翻}
初秦王府屬杜如晦叔父淹事王世充淹素與如晦兄弟不協譖如晦兄殺之又囚其弟楚客餓幾死|{
	幾居依翻又音祁}
楚客終無怨色及洛陽平淹當死楚客涕泣請如晦救之如晦不從楚客曰曩者叔已殺兄今兄又殺叔一門之内自相殘而盡豈不痛哉欲自剄|{
	剄古頂翻}
如晦乃為之請於世民淹得免死|{
	為于偽翻}
秦王世民坐閶闔門|{
	晉都洛陽其城西面北來第三門曰閶闔隋營新都唐六典所載都城皇城宫城苑城諸門皆無閶闔盖唐改之也闔戶臘翻}
蘇威請見稱老病不能拜世民遣人數之曰|{
	見賢遍翻數所具翻又所主翻}
公隋室宰相危不能扶使君弑國亡見李密王世充皆拜伏舞蹈今既老病無勞相見及至長安又請見不許既老且貧無復官爵卒於家年八十二|{
	史言蘇威之壽不若早夭卒子恤翻}
秦王世民觀隋宫殿歎曰逞侈心窮人欲無亡得乎命撤端門樓焚乾陽殿毁則天門及闕|{
	唐六典東都皇城南面三門中曰端門乾陽殿唐後於此起乾元殿宮城南面三門中曰應天門盖隋之則天門也唐六典曰毁建國門隋志東都城南面二門正南曰建國}
廢諸道塲城中僧尼留有名德者各三十人餘皆返初|{
	返初服也尼女夷翻}
前真定令周法明|{
	真定縣隋帶恒山郡唐改郡為恒州}
法尚之弟也|{
	周法尚自陳入隋為將}
隋末結客襲據黄梅|{
	隋志黄梅縣舊曰永興開皇初改曰新蔡十八年改曰黄梅因黄梅山以名縣也劉昫曰黄梅漢蘄春縣地宋分置新蔡郡隋為縣屬蘄春郡}
遣族子孝節攻蘄春|{
	蘄春漢縣屬江夏郡吳為蘄春郡晋改為西陽又改為蘄陽梁改曰蘄水後齊改曰齊昌隋開皇十八年復曰蘄春帶郡蘄音渠之翻}
兄子紹則攻安陸|{
	安陸漢縣屬江夏郡宋分置安陸郡梁置南司州西魏置安州隋復為安陸郡}
子紹德攻沔陽|{
	沔陽漢□陵縣地屬江夏郡後周置復州大業初改沔州尋改為沔陽郡沔彌兖翻}
皆拔之庚午以四郡來降|{
	降戶江翻下同}
壬申齊善行以洺相魏等州來降|{
	洺音名相息亮翻考異曰革命記}


|{
	云五月七日善行等至洺州實録云壬申洺相魏等州降者盖降使到之日也月末又云裴矩等以八璽降盖璽到之日也}
時建德餘衆走至洺州欲立建德養子為主徵兵以拒唐又欲剽掠居民還向海隅為盜善行獨以為不可曰隋末喪亂|{
	剽匹妙翻喪息泿翻下同}
故吾屬相聚草野苟求生耳以夏王之英武平定河朔士馬精彊一朝為擒易如反掌豈非天命有所屬|{
	夏戶雅翻易以豉翻屬之欲翻}
非人力所能爭邪|{
	邪音耶}
今喪敗如此守亦無成逃亦不免等為亡國豈可復遺毒於民|{
	復扶又翻}
不若委心請命於唐必欲得繒帛者|{
	繒慈陵翻}
當盡散府庫之物勿復殘民也|{
	復扶又翻}
於是運府庫之帛數十萬段置萬春宫東街|{
	萬春宫竇建德所築}
以散將卒凡三晝夜乃畢|{
	將即亮翻}
仍布兵守坊巷得物者即出無得更入人家士卒散盡然後與僕射裴矩行臺曹旦帥其百官奉建德妻曹氏及傳國八璽并破宇文化及所得珍寶請降于唐|{
	武德二年建德破化及得八璽及珍寶帥讀曰率璽斯氏翻降戶江翻}
上以善行為秦王左二護軍|{
	秦王所統置左三府右三府命為統軍護軍}
仍厚賜之初竇建德之誅宇文化及也隋南陽公主有子曰禪師建德虎賁郎將於士澄問之曰|{
	何承天姓苑有於姓今浙間有此姓禪市連翻賁音奔將即亮翻於如字}
化及大逆兄弟之子皆當從坐若不能捨禪師當相為留之|{
	為于偽翻}
公主泣曰虎賁既隋室貴臣|{
	按隋書帝紀大業初造龍舟於士澄已為上儀同往江南採木}
兹事何須見問建德竟殺之公主尋請為尼及建德敗公主將歸長安與宇文士及遇於洛陽士及請與相見公主不可士及立於戶外請復為夫婦公主曰我與君仇家今所以不手刃君者但謀逆之日察君不預知耳訶令速去|{
	訶虎何翻}
士及固請公主怒曰必欲就死可相見也士及知不可屈乃拜辭而去 乙亥以周法明為黄州總管|{
	黄州治黄岡縣漢江夏郡西陵縣地齊曰南安又置齊安郡隋置黄州尋改永安郡}
戊寅王世充徐州行臺杞王世辯以徐宋等三十八州詣河南道安撫大使任瓌請降|{
	使疏吏翻任音壬瓌古回翻降戶江翻}
世充故地悉平 竇建德博州刺史馮士羨|{
	隋志武陽郡聊城縣開皇十六年置博州}
復推淮安王神通為慰撫山東使徇下三十餘州建德之地悉平己卯代州揔管李大恩擊苑君璋破之 突厥寇邊長平靖王叔良督五將擊之叔良中流矢|{
	厥九勿翻將即亮翻中竹仲翻}
師旋六月戊子卒於道|{
	卒子恤翻}
戊戌孟海公餘黨蔣善合以鄆州孟噉鬼以曹州來降|{
	鄆州隋之東平郡曹州隋之濟隂郡鄆音運噉徒濫翻降戶江翻}
噉鬼海公之從兄也|{
	從才用翻}
庚子營州人石世則執總管晉文衍|{
	營州隋志之遼西郡}
舉州叛奉靺鞨突地稽為主|{
	靺音末鞨音曷}
黄州揔管周法明攻蕭銑安州拔之|{
	蕭銑盖亦置安州於隋安陸郡界}
獲其揔管馬貴遷 乙巳以右驍衛將軍盛彦師為宋州揔管安撫河南|{
	驍堅堯翻}
乙卯海州賊帥臧君相以五州來降拜海州緫管|{
	海州隋志之東海郡宋白曰魏武定七年置海州帥所類翻}
秋七月庚申王世充行臺王弘烈王泰左僕射豆盧行褒石僕射蘇世長以襄州來降|{
	襄州隋志之襄陽郡宋白曰襄州春秋穀鄧鄾盧羅郡之地秦為南陽郡地魏置襄陽郡以其地在襄山之陽也江左置雍州西魏改襄州}
上與行褒世長皆有舊先是屢以書招之|{
	先悉薦翻}
行褒輒殺使者既至長安上誅行褒而責世長世長曰隋失其鹿天下共逐之陛下既得之矣豈可復忿同獵之徒問爭肉之罪乎|{
	使疏吏翻復扶又翻}
上笑而釋之以為諫議大夫 |{
	考異曰舊本紀及唐歷年代記唐會要皆云五年六月置諫議大夫按世長自諫議歷陜州長史天策府軍諮祭酒四年十一月已預十八學士据舊職官志四年置諫議大夫今從之余按唐六典秦漢曰諫大夫光武加議字北齊集書省置諫議大夫七人隋氏門下省亦置諫議大夫七人四年以前唐未及置今始置之耳}
嘗從校獵高陵|{
	如淳曰合軍聚衆有幡校撃鼓也周禮校人掌王田獵之馬故謂之校獵師古曰如說非也此校謂以木相貫穿為闌校耳校人職云六廄成校是則以遮閳為義也校獵者大為闌校以遮禽獸而獵取也軍之幡旗雖有校名本因部校此無豫也原父曰予謂校讀如犯而不校亦競逐獵也高陵縣屬京兆府}
大獲禽獸上顧羣臣曰今日畋樂乎世長對曰陛下遊獵薄廢萬機不滿十旬未足為樂|{
	樂音洛}
上變色既而笑曰狂態復發邪|{
	復扶又翻邪音耶}
對曰於臣則狂於陛下甚忠嘗侍宴披香殿|{
	程大昌雍録慶善宫有披香殿又云慶善宫高祖舊第也在武功渭水北余按下文世長言昔侍於武功若此殿正在武功舊宅世長縱是譎諫不應引以爲言恐此殿不在慶善宫}
酒酣謂上曰此殿煬帝之所為邪上曰卿諫似直而實多詐豈不知此殿朕所為而謂之煬帝乎對曰臣實不知但見其華侈如傾宫鹿臺|{
	紂為傾宮鹿臺}
非興王之所為故也若陛下為之誠非所宜臣昔侍陛下於武功見所居宅僅庇風雨當時亦以為足今因隋之宫室已極侈矣而又增之將何以矯其失乎上深然之 甲子秦王世民至長安世民被黄金甲齊王元吉李世勣等二十五將從其後鐵騎萬匹|{
	被皮義翻將即亮翻騎奇寄翻}
前後部鼓吹|{
	鼔吹軍樂也漢制萬人將軍得之司馬法軍中有鼓笛所以發壯勇薛居正曰義鐃問鼓吹十二案合於何所荅云周禮鼓人掌六鼓四金漢朝乃有黄門鼓吹崔豹古今注云張騫使西域得摩訶兜勒一曲李延年增之分為二十八曲梁置鼓吹清商令二人唐又有掆鼓金鉦大鼓長鳴歌簫笳笛合為鼔吹十二案吹昌瑞翻}
俘王世充竇建德及隋乘輿御物獻于太廟|{
	乘繩證翻 考異曰李勣傳云太宗為上將勣為下將與太宗俱服金甲乘戎輅告捷于太廟今從唐歷}
行飲至之禮以饗之|{
	左傳歸而飲至以數軍實杜預注曰飲於廟以數軍徒器械及所獲也}
乙丑高句麗王建武遣使入貢|{
	句音駒麗鄰知翻}
建武元之弟也|{
	高元見隋紀}
上見王世充而數之|{
	數所具翻又所主翻}
世充曰臣罪固當誅然秦王許臣不死丙寅詔赦世充為庶人與兄弟子姪處蜀|{
	處昌呂翻}
斬竇建德於市 丁卯以天下略定大赦百姓給復一年|{
	復方目翻下同}
陝鼎函虢虞芮六州轉輸勞費|{
	陜州治陜弘農縣本隋弘農郡義寧元年曰鳳林領弘農□鄉湖城武德元年曰鼎州因鼎湖為名武德三年以永寧崤置函州又義寧元年分盧氏長水桃林置虢郡武德元年曰虢州義寧元年以安邑虞郷夏置安邑郡武德元年曰虞州二年以芮城河北永樂置芮州}
幽州管内久隔寇戎並給復二年律令格式且用開皇舊制赦令既下而王竇餘黨尚有遠徙者治書侍御史孫伏伽上言兵食可去信不可去陛下已赦而復徙之|{
	治直之翻伽求加翻上時掌翻去羌呂翻復扶又翻又音如字}
是自違本心使臣民何所憑依且世充尚蒙寛宥况於餘黨所宜縱釋 |{
	考異曰伏伽表云今月二日發雲雨之制而赦書乃十二日或脱十字也又云常赦所不免咸赦除之今赦無此文豈實録録赦文不盡歟}
上從之王世充以防夫未備置雍州廨舍|{
	按雍録都城坊里圖雍州廨舍後為京兆府在光德坊雍於用翻廨古隘翻}
獨孤機之子定州刺史修德帥兄弟至其所|{
	帥讀曰率}
矯稱敕呼鄭王世充與兄世惲趨出修德等殺之|{
	武德二年正月獨孤機兄弟為世充所殺故修德報仇惲於粉翻 考異曰舊傳作獨孤修今從河洛記}
詔免修德官其餘兄弟子姪等於道亦以謀反誅 隋末錢弊濫薄|{
	言錢之弊也}
至裁皮糊紙為之民間不勝其弊|{
	勝音升}
至是初行開元通寶錢重二銖四參|{
	按漢書律歷志權輕重者不失黍絫應劭註曰十黍為絫十絫為銖師古曰絫孟音來戈翻此字讀亦音纍紲之纍二銖四絫二百四十黍也參當作絫盖筆誤也}
積十錢重一兩輕重大小最為折衷|{
	衷竹仲翻}
遠近便之命給事中歐陽詢撰其文并書迴環可讀|{
	六典漢書百官表云給事中亦加官所加或博士大夫議郎漢儀注諸給事中日上朝謁平尚書奏事分為左右以有事殿中故曰給事中魏氏或為加官或為正員晉氏隸散騎省宋齊隸集書省後周天官府置給事中士隋曰給事郎唐曰給事中屬門下省掌侍奉左右分判省事凡百司奏抄侍中審定則先讀而署之以駮正違失撰其文者撰為八分篆隸一體 考異曰薛璫唐聖運圖云初進蠟樣文德皇后搯一甲故錢上有甲㾗云夌璠唐録政要云竇皇后按時竇皇后已崩文德皇后未立今皆不取}
以屈突通為陜東道大行臺右僕射鎮洛陽以淮陽王道玄為洛州緫管李世勣父盖竟無恙而還詔復其官爵|{
	屈居勿翻陕失冉翻李盖被虜見一百八十七卷武德二年十月恙余亮翻還從宣翻}
竇軌還益州|{
	自平洛還}
軌將兵征討或經旬月不解甲性嚴酷將佐有犯無貴賤立斬之鞭撻吏民常流血滿庭所部重足屏息|{
	將即亮翻重直龍翻屏必郢翻}
癸酉置錢監於洛并幽益等諸州秦王世民齊王元吉賜三鑪裴寂賜一鑪聽鑄錢|{
	賜以官鑪也鑪音爐鑪冶也}
自餘敢盜鑄者身死家口配没 河北既平上以陳君賓為洺州刺史將軍秦武通等將兵屯洺州欲使分鎮東方諸州又以鄭善果等為慰撫大使就洺州選補山東州縣官竇建德之敗也其諸將多盗匿庫物及居閭里暴横為民患|{
	洺彌并翻將即亮翻下同使疏吏翻横戶孟翻}
唐官吏以法繩之或加捶撻|{
	捶止橤翻}
建德故將皆驚懼不安高雅賢王小胡家在洺州欲竊其家以逃官吏捕之雅賢等亡命至貝州|{
	貝州隋志之清河郡}
會上徵建德故將范願董康買曹湛及雅賢等於是願等相謂曰王世充以洛陽降唐|{
	降戶江翻}
其將相大臣段逹單雄信等皆夷滅|{
	相息亮翻單慈淺翻}
吾屬至長安必不免矣吾屬自十年以來身經百戰當死久矣今何惜餘生不以之立事且夏王得淮安王遇以客禮|{
	見一百八十七卷二年十月夏戶雅翻}
唐得夏王即殺之吾屬皆為夏王所厚今不為之報仇|{
	為于偽翻}
將無以見天下之士乃謀作亂卜之以劉氏為主吉因相與之漳南|{
	之往也舊志漳南縣屬貝州漢之東陽縣隋開皇十八年分棗彊清平二縣地置漳南縣於古東陽城}
見建德故將劉雅以其謀告之雅曰天下適安定吾將老於耕桑不願復起兵|{
	復扶又翻}
衆怒且恐泄其謀遂殺之故漢東公劉黑闥時屏居漳南|{
	漢東公竇建德所封爵也屏必郢翻}
諸將往詣之告以其謀黑闥欣然從之黑闥方種蔬即殺耕牛與之共飲食定計聚衆得百人甲戌襲漳南縣據之 |{
	考異曰革命記十月二十七日衆立黑闥為漢東王建元天造即入漳南城鏁縣官於獄發使告貝州及諸鎮戍等云今漢東王為夏王起義兵於漳南請軍會戰今據實録甲戌七月十九日又黑闥䧟相州乃稱王改元在五年正月今不取}
是時諸道有事則置行臺尚書省無事則罷之朝廷聞黑闥作亂乃置山東道行臺於洺州|{
	朝直遥翻洺彌并翻}
魏冀定滄並置揔管府|{
	滄州隋志之勃海郡}
丁丑以淮安王神通為山東道行臺右僕射 辛巳褒州道安撫使郭行方攻蕭銑鄀州拔之|{
	褒州當作襄州詳見辯誤新志武德四年以竟陵之樂鄉及襄州之率道上洪置鄀州上書鄀下書州竟陵之樂鄉盖蕭銑地也鄀音若}
孟海公與竇建德同伏誅戴州刺史孟噉鬼不自安

|{
	新志武德四年改曹州之成武宋州之單父楚丘置戴州噉徒濫翻}
挾海公之子義以曹戴二州反以禹城令蔣善合為腹心|{
	禹城縣屬齊州隋之祝阿也新舊志皆云天寶元年改祝阿為禹城此時未有禹城當考又前書蔣善合以鄆州來降此以禹城令書之亦未知為誰所命也}
善合與其左右同謀斬之 八月丙戌朔日有食之 丁亥命太子安撫北邊 丁酉劉黑闥䧟鄃縣|{
	鄃縣屬貝州鄃音輸}
魏州刺史權威|{
	魏州隋志之武陽郡}
貝州刺史戴元祥與戰皆敗死黑闥悉取其餘衆及器械竇建德舊黨稍出歸之衆至二千人為壇於漳南祭建德告以舉兵之意自稱大將軍詔發關中步騎三千使將軍秦武通定州總管藍田李玄通擊之|{
	藍田縣屬雍州騎奇寄翻}
又詔幽州總管李藝引兵會擊黑闥 癸卯突厥寇代州|{
	厥九勿翻}
總管李大恩遣行軍總管王孝基拒之舉軍皆没甲辰進圍崞縣|{
	崞縣屬代州崞音郭}
乙巳王孝基自突厥逃歸李大恩衆少|{
	少詩沼翻}
據城自守突厥不敢逼月餘引去 上以南方寇盗尚多丙午以左武侯將軍張鎮周為淮南道行軍揔管大將軍陳智略為嶺南道行軍揔管鎮撫之 丁未劉黑闥䧟歷亭|{
	舊志歷亭漢東陽地隋開皇十六年分鄃縣置隋志曰分武城置時屬貝州}
執屯衛將軍王行敏使之拜不可遂殺之 初洛陽既平徐圓朗請降拜兖州總管|{
	兖州隋志之魯郡降戶江翻}
封魯郡公劉黑闥作亂隂與圓朗通謀上使葛公盛彦師安集河南行至任城|{
	任城縣屬兖州任音壬}
辛亥圓朗執彦師舉兵反黑闥以圓朗為大行臺元帥|{
	帥所類翻}
兖鄆陳杞伊洛曹戴等八州豪右皆應之圓朗厚禮彦師使作書與其弟令舉虞城降|{
	舊志虞城縣屬宋州隋分下邑縣置時置東虞州令力丁翻}
彦師為書曰吾奉使無狀為賊所擒|{
	使疏吏翻下同}
為臣不忠誓之以死汝善侍老母勿以吾為念圓朗初色動而彦師自若圓朗乃笑曰盛將軍有壯節不可殺也待之如舊河南道安撫大使任瓌行至宋州屬圓朗反|{
	宋州治睢陽時為宋城縣使疏吏翻瓌古回翻屬之欲翻}
副使柳濬勸瓌退保汴州|{
	宋州西至汴州二百八十五里}
瓌笑曰柳公何怯也圓朗又攻䧟楚丘|{
	楚丘縣後魏之己氏縣隋開皇六年更名時屬戴州}
引兵將圍虞城瓌遣部將崔樞張公謹自鄢陵帥諸豪右質子百餘人守虞城|{
	鄢陵縣時屬洧州將即亮翻下同鄢謁晚翻又於建翻又音偃帥讀曰率質音致下同}
濬曰樞與公謹皆王世充將諸州質子父兄皆反恐必為變瓌不應樞至虞城分質子使與土人合隊共守城|{
	合音閤}
賊稍近質子有叛者樞斬其隊帥於是諸隊帥皆懼|{
	帥所類翻下同}
各殺其質子樞不禁梟其首於門外遣使白瓌|{
	梟堅堯翻使疏吏翻}
瓌陽怒曰吾所以使與質子俱者欲招其父兄耳何罪而殺之退謂濬曰吾固知崔樞能辦此也縣人既殺質子與賊深仇吾何患乎賊攻虞城果不克而去 初竇建德以鄱陽崔元遜為深州刺史|{
	翻陽縣屬饒州隋開皇十六年以定州安平置深州大業初廢新志武德四年以定州之安平瀛州之饒陽置深州盖竇建德置而唐因之耳宋白曰以州城西故深城名州}
及劉黑闥反元遜與其黨數十人謀於野伏甲士於車中以禾覆其上|{
	覆敷又翻}
直入聽事|{
	聽讀曰廳}
自禾中呼譟而出執刺史裴晞殺之傳首黑闥 九月乙卯文登賊帥淳于難請降置登州以難為刺史|{
	文登本漢牟平縣地後齊置文登縣因文登山而名隋志屬東萊郡時置登州兼領萊州之觀陽縣降戶江翻下同}
突厥寇并州|{
	厥九勿翻并卑名翻}
遣左屯衛大將軍竇琮等擊之戊午突厥寇原州遣行軍揔管尉遟敬德等擊之|{
	琮祖宗翻尉紆勿翻}
辛酉徐圓朗自稱魯王 隋末歙州賊汪華據黟歙等五州有衆一萬|{
	歙州本新安郡隋平陳置歙州黟歙二縣屬歙音攝黟音伊劉昫曰音同䃜縣南石墨嶺出石墨故也}
自稱吳王甲子遣使來降|{
	使疏吏翻降戶江翻}
拜歙州揔管 隋末弋陽盧祖尚糾合壯士以衛鄉里部分嚴整羣盜畏之及煬帝遇弑鄉人奉之為光州刺史|{
	弋陽漢縣南齊為郡梁置光州分扶問翻}
時年十九奉表於皇泰主及王世充自立祖尚來降丙子以祖尚為光州總管 |{
	考異曰實録丙子以光州豪右盧祖尚為光州揔管按舊傳世充自立祖尚遂舉州歸欵而實録至此始見之盖當時止稱刺史至此方遷總管耳}
己卯詔括天下戶口 徐圓朗寇濟州治中吳伋論擊走之|{
	濟州隋之濟北郡漢置州刺史其屬有治中從事别駕從事自是兩官唐武德元年改郡太守曰州刺史郡丞曰别駕未嘗置治中今書濟州治中吳伋論豈即以别駕為治中邪下文又書徐圓朗昌州治中盖此時官稱猶未定于一濟子禮翻}
癸未詔以太常樂工皆前代因罪配没子孫相承多

歷年所|{
	所謂樂戶也}
良可哀愍宜並蠲除為民|{
	蠲吉淵翻}
且令執事若仕官入流|{
	令力丁翻入流者為流内官}
勿更追集 甲申靈州揔管楊師道擊突厥破之師道恭仁之弟也|{
	楊㳟仁時鎮凉州厥九勿翻}
詔發巴蜀兵以趙郡王孝恭為荆湘道行軍摠管|{
	荆州南郡湘州長沙郡荆湘道以南朝荆湘所部言之下荆郢道類此}
李靖攝行軍長史|{
	長知兩翻}
統十二揔管自夔州順流東下以盧江王瑗為荆湘道行軍元帥|{
	郢州隋之竟陵郡瑗于眷翻帥所翻翻}
黔州刺史田世康出辰州道|{
	舊志漢辰陽縣隋改辰陽為辰溪縣仍分置沅陵縣沅陵郡唐改為辰州以沅陵為理所黔音琴}
黄州揔管周法明出夏口道|{
	夏口即漢口夏戶雅翻}
以擊蕭銑是月孝恭發夔州時峽江方漲|{
	蜀江逕三峽謂之峽江}
諸將請俟水落進軍|{
	將即亮翻下同}
李靖曰兵貴神速今吾兵始集銑尚未知若乘江漲倏忽抵其城下掩其不備此必成擒不可失也孝恭從之 淮安王神通將關内兵至冀州與李藝兵合又發邢洺相魏恒趙等兵合五萬餘人與劉黑闥戰於饒陽城南布陳十餘黑闥衆少依隄單行而陳以當之|{
	宋白曰饒陽漢縣在饒河之陽今縣東北二十里饒陽故城是也齊天保五年移於今理按饒陽縣則魏虜渠口置虜口鎮於此後為縣理洺音名相息亮翻降戶登翻陳讀曰陣行戶剛翻}
會風雪神通乘風擊之既而風返神通大敗士馬軍資失亡三分之二李藝居西偏擊高雅賢破之逐奔數里聞大軍不利退保藁城|{
	藁城縣本屬恒州時屬亷州}
黑闥就擊之藝亦敗薛萬均萬徹皆為所虜截髪驅之萬均兄弟亡歸藝引兵歸幽州黑闥兵勢大振 上以秦王功大前代官皆不足以稱之特置天策上將位在王公上冬十月以世民為天策上將領司徒陜東道大行臺尚書令增邑二萬戶|{
	唐爵九等王食邑萬戶今倍之陜失冉翻}
仍開天策府置官屬|{
	天策府置長史司馬各一人從事中郎二人並掌通判府事軍諮祭酒二人謀軍事贊相禮儀應接賓客典□四人掌宣傳導引之事主薄二人掌省覆教命録事二人記室參軍事二人掌書疏表啓宣行教命功倉兵騎鎧士六曹參軍各二人參軍事六人}
以齊王元吉為司空世民以海内浸平乃開館於宫西延四方文學之士出教以王府屬杜如晦記室房玄齡虞世南文學禇亮姚思廉主簿李玄道參軍蔡允恭薛元敬顔相時諮議典籖蘇朂天策府從事中郎于志寧軍諮祭酒蘇世長記室薛收倉曹李守素國子助教陸德明孔頴逹信都盖文逹宋州揔管府戶曹許敬宗|{
	諸王出命稱教相息亮翻盖古盍翻}
並以本官兼文學館學士 |{
	考異曰舊書參軍薛元敬承許敬宗下今從太宗實録諮議典籖蘇勉舊書作軍諮典籖今從實録宋州揔管府戶曹許敬宗舊書禇亮傳作著作佐郎攝記室敬宗傳擬漣州别駕今從實録}
分為三番更日直宿|{
	更工衡翻}
供給珍膳恩禮優厚世民朝謁公事之暇輒至館中引諸學士討論文籍|{
	朝直遥翻論盧昆翻}
或夜分乃寢又使庫直閻立本圖像|{
	庫直隸親事府}
禇亮為贊號十八學士士大夫得預其選者時人謂之登瀛洲|{
	自來相傳海中有三神山篷萊方丈瀛洲人不能至至則成仙矣故以為喻}
允恭大寶之弟子|{
	蔡大寶輔後梁主蕭詧}
兀敬收之從子|{
	從才用翻}
相時師古之弟|{
	顔師古以碩學名}
立本毗之子也|{
	閻毗以巧思事隋煬帝}
初杜如晦為秦王府兵曹參軍俄遷陜州長史|{
	陜失冉翻長知兩翻}
時府僚多補外官世民患之房玄齡曰餘人不足惜至於杜如晦王佐之才大王欲經營四方非如晦不可世民驚曰微公言幾失之|{
	幾居依翻}
即奏為府屬與玄齡常從世民征伐參謀帷幄軍中事如晦剖决如流世民每破軍克城諸將佐爭取寶貨玄齡獨收采人物致之幕府又將佐有勇略者玄齡必與之深相結|{
	將即亮翻}
使為世民盡死力|{
	為于偽翻下同}
世民每令玄齡入奏事上歎曰玄齡為吾兒陳事雖隔千里皆如面談李玄道嘗事李密為記室密敗官屬為王世充所虜懼死皆逹曙不寐獨玄道起居自若曰死生有命非憂可免衆服其識量 庚寅劉黑闥䧟瀛州殺刺吏盧士叡觀州人執刺史雷德備以城降之|{
	隋以東光縣置觀州大業初廢武德四年以德州之弓高胡蘇東光冀州之阜陵蓚安陵觀津置觀州觀古喚翻降戶江翻下同}
辛卯蕭銑鄂州刺史雷長頴以魯山來降|{
	隋平陳以江夏郡置鄂州治江南之江夏大業初復為郡蕭銑盖置州於魯山}
趙郡王孝恭帥戰艦二千餘艘東下|{
	帥讀曰率艦戶黯翻艘蘇遭翻}
蕭銑以江水方漲殊不為備孝恭等拔其荆門宜都二鎮進至夷陵|{
	蕭銑置宜都鎮於峽州夷道縣夷陵縣帶峽州}
銑將文士弘將精兵數萬屯清江|{
	將即亮翻按水經注清江即佷山夷水也水色清照十丈分沙蜀人見其澄清因名清江吳分漢南郡之巫縣立沙渠縣後周於縣立施州清江郡隋廢郡及州為清江縣}
癸巳孝恭擊走之獲戰艦三百餘艘殺溺死者萬計追奔至百里洲|{
	自清江而東過歸州峽州而後至百里洲}
士弘收兵復戰|{
	復扶又翻}
又敗之進入北江|{
	百里洲在枝江縣江中江水至此分流出百里洲北而東流者因謂之北江敗補邁翻}
銑江州總管盖彦舉以五州來降|{
	梁以漢夷道縣置宜都郡宜昌縣後周置江州隋廢為巴山縣屬清江郡蕭銑盖復置江州於此盖古盍翻}
毛州刺史趙元愷|{
	魏州舘陶縣舊置毛州隋大業初州廢竇建德復置唐因之領魏州之舘陶冠氏博州之堂邑具州之臨清清水}
性嚴急下不堪命丁卯州民董燈明等作亂殺元愷以應劉黑闥 盛彦師自徐圓朗所逃歸王薄因說青萊密諸州皆下之|{
	萊州東萊郡後魏之光州也密州高密郡後魏之膠州也說輸芮翻}
蕭銑之罷兵營農也|{
	見上卷三年}
纔留宿衛數千人聞唐兵至文士弘敗大懼倉猝徵兵皆在江嶺之外道塗阻遠不能遽集乃悉見兵出拒戰|{
	見賢遍翻}
孝恭將擊之李靖止之曰彼救敗之師策非素立勢不能久不若且泊南岸|{
	江陵南岸即馬頭岸}
緩之一日彼必分其兵或留拒我或歸自守兵分勢弱我乘其懈而擊之|{
	懈古隘翻}
蔑不勝矣今若急之彼則併力死戰楚兵剽鋭未易當也|{
	剽匹妙翻易以豉翻}
孝恭不從留靖守營自帥銳師出戰果敗走趣南岸|{
	帥讀曰率趣七喻翻又逡須翻}
銑衆委舟收掠軍資人皆負重靖見其衆亂縱兵奮擊大破之乘勝直抵江陵入其外郭又攻水城抜之大獲舟艦李靖使孝恭盡散之江中諸將皆曰|{
	將即亮翻}
破敵所獲當藉其用柰何棄以資敵靖曰蕭銑之地南出嶺表東距洞庭|{
	洞庭湖在岳州巴陵縣}
吾懸軍深入若攻城未拔援軍四集吾表裏受敵進退不獲雖有舟楫將安用之今弃舟艦使塞江而下|{
	塞悉則翻}
援兵見之必謂江陵已破未敢輕進往來覘伺|{
	覘丑廉翻又丑艶翻伺相吏翻}
動淹旬月吾取之必矣銑援兵見舟艦果疑不進其交州刺史丘和長史高士廉司馬杜之松將朝江陵|{
	交州隋交趾郡長知兩翻朝直遥翻}
聞銑敗悉詣孝恭降|{
	降戶江翻下同}
孝恭勒兵圍江陵銑内外阻絶問策於中書侍郎岑文本文本勸銑降銑乃謂羣下曰天不祚梁不可復支矣|{
	復扶又翻}
若必待力屈則百姓蒙患柰何以我一人之故䧟百姓於塗炭乎乙巳銑以太牢告于太廟下令開門出降 |{
	考異曰高祖實録癸已趙郡王孝恭與蕭銑將文士弘相遇於清江合口擊之獲其戰艦千餘艘下宜昌當陽枝江松滋四縣舊書孝恭傳攻其水城克之所得船散於江中諸將皆曰虜得賊船當藉其用何為弃之無乃資賊邪孝恭曰不然蕭銑偽境南極嶺外東至洞庭若攻城未拔援兵復到我則内外受敵進退不可雖有舟楫何所用之今銑緑江州鎮忽見船舸亂下必知銑敗未敢進兵來去覘伺動淹旬月用緩其救吾克之必矣銑救兵至巴陵見船被江而下果狐疑不敢輕進太宗實録孝恭傳進師至清江銑遣其將文士弘以兵拒戰擊走之追奔至于百里洲士弘收兵復戰又敗之追入北江銑悉兵以拒之孝恭將戰李靖止之曰楚人輕鋭難與爭鋒今新失荆門盡兵出戰此救敗之師也非其本圖勢不能久一日不戰賊必兩分留輕兵抗我退羸師以自守此即勢攜力弱擊之必捷孝恭不從遣靖撫營自以鋭師水戰孝恭果敗奔于南岸賊委舟大掠人皆負重靖見其軍亂進兵擊之賊大敗乘勝進軍入其郛郭攻其水城尅之悉取其舟楫散於江中賊救兵見之謂城已䧟莫敢輕進銑内外阻絶城中攜貮由是懼而出降唐歷孝恭靖乘勝進兵攻其水城尅之悉取其船艦散於江中諸將曰弃之無乃資敵靖曰不然云云如舊書所載孝恭語既而銑救兵見之謂城已䧟莫敢輕進銑由是懼而出降按十道志荆門在峽州宜都縣界夷陵峽州縣名清江在峽州巴山縣界百里洲在荆州枝江縣界江自此洲派别去江陵已近故銑悉兵死戰太宗實録近為得實今從之其餘則參取四書之語孝恭以李靖為謀主盖靖畫策而孝恭為諸將言之今從唐歷}
守城者皆哭銑帥羣臣緦縗布幘詣軍門|{
	帥讀曰率縗倉回翻}
曰當死者唯銑耳百姓無罪願不殺掠孝恭入據其城諸將欲大掠岑文本說孝恭曰江南之民自隋末以來困於虐政重以羣雄虎爭|{
	將即亮翻說輸芮翻重直用翻}
今之存者皆鋒鏑之餘跂踵延頸以望真主|{
	踵不至地曰跂音夫智翻}
是以蕭氏君臣江陵父老决計歸命庶幾有所息肩今若縱兵俘掠恐自此以南無復向化之心矣|{
	幾居希翻復扶又翻}
孝恭稱善遽禁止之諸將又言梁之將帥與官軍拒鬬死者其罪既深請籍没其家以賞將士李靖曰王者之師宜使義聲先路彼為其主鬭死|{
	將即亮翻帥所類翻先悉薦翻下同為于偽翻}
乃忠臣也豈可同叛逆之科籍其家乎於是城中安堵秋毫無犯南方州縣聞之皆望風欵附銑降數日援兵至者十餘萬聞江陵不守皆釋甲而降孝恭送銑於長安上數之銑曰隋失其鹿天下共逐之銑無天命故至此若以為罪無所逃死竟斬於都市詔以孝恭為荆州揔管李靖為上柱國賜爵永康縣公|{
	永康縣屬婺州}
仍使之安撫嶺南得承制拜授先是銑遣黄門侍郎江陵劉洎略地嶺表得五十餘城未還而銑敗洎以所得城來降除南康州都督府長史|{
	是年分端州之端溪置南康州仍置都督府督端康封新宋瀧等州時改揔管府為都督府貞觀初始分上中下州洎其冀翻長知兩翻}
戊申徐圓朗昌州治中劉善行以須昌來降|{
	圓朗盖以鄆州之須昌置昌州降戶江翻}
庚戌詔陜東道大行臺尚書省自令僕至郎中主事|{
	六典曰漢官云光禄勲有南北廬主事三署主事於諸郎之中察茂才高第者為之秩四百石次補尚書郎出宰百里謝承後漢書胡伯藩范滂公沙穆並以俊才舉孝廉除郎中光禄勲主事後魏尚書吏部儀曹三公虞曹都官二千石比部各量事置事故主事員門下置主事令史並從八品上隋初臺省並置主事令史煬帝三年並去令史之名其主事隋曹閑劇而置每十令史置一主事不滿十者亦置一人雜用才術之士至唐並用流外入流者補之陜失冉翻}
品秩皆與京師同而員數差少|{
	少詩沼翻}
山東行臺及揔管府諸州並隸焉其益州襄州山東淮南河北等道令僕以下各降京師一等員數又減焉行臺尚書令得承制補署其秦王齊王府官之外各置左右六護軍府及左右親事帳内府|{
	護軍惟秦齊二府有之他國不得置也親王親事府及帳内府各置典軍二人正五品上副典軍二人從五品上}
閠月乙卯上幸稷州|{
	武德三年以京兆之武功好時盩厔置稷州又廢郇州以郿鳳泉二縣屬焉}
己未幸武功舊墅壬戌獵于好畤|{
	武德二年分醴泉置好畤縣屬雍州因漢舊名也}
乙丑獵于九嵕|{
	九嵕山在雍州醴泉縣嵕子紅翻}
丁卯獵于仲山戊辰獵于清水谷|{
	隋志京兆宜君縣有清水}
遂幸三原|{
	三原本屬漢池陽縣界後周置建忠郡隋置三原縣唐屬雍州}
辛未幸周氏陂壬申還長安 十一月甲申上祀圓丘|{
	貞觀禮冬至祀昊天上帝於圓丘}
杜伏威使其將王雄誕擊李子通子通以精兵守獨松嶺|{
	自宣州廣德縣東南過獨松嶺至湖州嶺路險狹將即亮翻下同}
雄誕遣其將陳當|{
	陳當之下合有世字盖唐史避大宗諱去世字也}
將千餘人乘高據險以逼之多張旗幟夜則縛炬火於樹布滿山澤子通懼燒營走保杭州雄誕追擊之又敗之於城下|{
	敗補邁翻}
庚寅子通窮蹙請降|{
	降戶江翻 考異曰實録是月景申會稽賊帥李子通伏誅按子通因杜伏威入朝始謀叛伏誅於時未也舊紀是月子通以其地來降新紀庚寅李子通降丙申謀反亦不寤伏威未入朝也}
伏威執子通并其左僕射樂伯通送長安上釋之先是汪華據黟歙稱王十餘年雄誕還軍擊之華拒之於新安洞口|{
	唐歙州隋之新安郡也新安洞口即歙州隘道之口先悉薦翻黟音伊歙音攝}
甲兵甚銳雄誕伏精兵於山谷帥羸弱數千犯其陳|{
	帥讀曰率下同羸倫為翻陳讀曰陣}
戰纔合陽不勝走還營華進攻之不能克會日暮引還伏兵已據其洞口華不得入窘迫請降聞人遂安據崑山無所屬伏威使雄誕擊之雄誕以崑山險隘難以力勝乃單騎造其城下|{
	騎奇寄翻造七到翻}
陳國威靈|{
	陳唐國之威靈也}
示以禍福遂安感悅帥諸將出降於是伏威盡有淮南江東之地南至嶺東距海雄誕以功除歙州總管賜爵宜春郡公|{
	袁州宜春郡}
壬辰林州揔管劉旻擊劉仚成|{
	仚許延翻}
大破之|{
	舊志慶州華池縣隋置武德四年置林州揔管府}
仚成僅以身免部落皆降|{
	降戶江翻}
李靖度嶺遣使分道招撫諸州所至皆下|{
	使疏吏翻}
蕭銑桂州揔管李襲志帥所部來降|{
	桂州隋志之始安郡}
趙郡王孝恭即以襲志為桂州揔管明年入朝|{
	朝直遥翻}
以李靖為嶺南撫慰大使檢校桂州揔管引兵下九十六州得戶六十餘萬壬寅劉黑闥䧟定州執揔管李玄通黑闥愛其才欲以為大將|{
	將即亮翻下同}
玄通不可故吏有以酒肉饋之者玄通曰諸君哀吾幽辱幸以酒肉來相開慰當為諸君一醉|{
	為于偽翻}
酒酣為守者曰吾能劒舞願假五口刀守者與之玄通舞竟太息曰大丈夫受國厚恩鎮撫方面不能保全所守亦何面目視息世間哉即引刀自刺|{
	刺七亦翻}
潰腹而死上聞為之流涕|{
	為于偽翻}
拜其子伏護為大將 庚戌杞州人周文舉殺刺史王文矩以城應徐圓朗 幽州大饑高開道許以粟賑之|{
	賑津忍翻}
李藝遣老弱詣開道就食開道皆厚遇之藝喜於是發民三千人車數百乘驢馬千餘匹往受粟開道悉留之告絶於藝復稱燕王|{
	賑津忍翻乘繩登翻復扶又翻又音如字燕囚肩翻}
北連突厥南與劉黑闥相結引兵攻易州不克大掠而去又遣其將謝稜詐降於藝請兵援接藝出兵應之將至懷戎|{
	舊志北燕州懷戎縣後漢上谷之潘縣也北齊改為懷戎媯水經其中州所治也貞觀八年改北燕州為媯州}
稜襲擊破之開道與突厥連兵數入為寇|{
	厥九勿翻數所角翻}
恒定幽易咸被其患|{
	恒戶登翻被皮義翻}
十二月乙卯劉黑闥䧟冀州殺刺史麴稜黑闥既破淮安王神通移書趙魏|{
	以戰國時趙魏大界言之}
故竇建德將卒爭殺唐官吏以應黑闥庚申遣右屯衛大將軍義安王孝常將兵討黑闥黑闥將兵數萬進逼宗城|{
	將即亮翻}
黎州揔管李世勣先屯宗城弃城走保洺州|{
	新志武德二年以黎陽縣置黎州宗城本廣宗縣隋仁壽初改為宗城縣避煬帝名也屬清河郡時置宗州洺彌并翻 考異曰實録世勣與黑闥戰於宋州我師敗績革命記李勣為大揔管張仕貴為副領兵二萬人入宋州勣以五百騎自探聞黑闥到南宫馳至宋州不入城而西過至洺州騎馬於南門外喚陳君賓党仁弘秦武通等弃城西抜永年縣令程名振見武通狼狽走出馳馬向縣取家口入城城人恐相劫掠即閉城門自守名振乃於城北門上以繩懸下將母妻男女步走西去不踰四五里母妻等被劫散失名振脱身而免黑闥攻宋城破之仕貴以輕騎突圍而走投相州數日黑闥大軍至洺州案舊地里志武德四年置宗州於宗城縣宋字皆當作宗世勣名將必不至如革命記所云但力不能拒而弃城耳今從舊書黑闥傳}
甲子黑闥追擊世勣等破之殺步卒五千人世勣僅以身免丙寅洺州土豪翻城應黑闥黑闥於城東南告天及祭竇建德而後入後旬日引兵攻拔相州|{
	相息亮翻 考異曰實録黑闥䧟相州在來年正月乙酉盖奏到之日也今從革命記}
執刺史房晃右武衛將軍張士貴潰圍走黑闥南取黎衛二州半歲之間盡復建德舊境又遣使北連突厥頡利可汗遣俟斤宋邪那帥胡騎從之|{
	使疏吏翻下同厥九勿翻俟渠之翻邪讀曰耶帥讀曰率騎奇寄翻}
右武衛將軍秦武通洺州刺史陳君賓永寧令程名振|{
	永寧當作永年}
皆自河北遁歸長安 丁卯命秦王世民齊王元吉討黑闥 昆彌遣使内附昆彌即漢之昆明也|{
	昆明蠻在㸑蠻西以西洱河為境西洱河即葉榆河也}
嶲州治中吉弘緯通南寜|{
	嶲州漢越嶲郡地後周置嚴州開皇六年改西寧州十八年改嶲州南寜古南中味升麻諸縣之地武德四年置南寧州嶲音髓}
至其國說之遂來降|{
	說輸芮翻降戶江翻}
己巳劉黑闥䧟邢州趙州庚午䧟魏州殺揔管潘道毅辛未䧟莘州|{
	隋開皇十六年以魏州之莘縣置莘州大業初廢是年復以魏州之莘臨黄武陽博州之武水置莘州考異曰實録作華州新書作業州按地里志無業州必莘州也十道志開皇十六年於莘縣置莘州舊志武德五年置}
壬申徙宋王元嘉為徐王

資治通鑑卷一百八十九
















































































































































