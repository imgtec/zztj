<!DOCTYPE html PUBLIC "-//W3C//DTD XHTML 1.0 Transitional//EN" "http://www.w3.org/TR/xhtml1/DTD/xhtml1-transitional.dtd">
<html xmlns="http://www.w3.org/1999/xhtml">
<head>
<meta http-equiv="Content-Type" content="text/html; charset=utf-8" />
<meta http-equiv="X-UA-Compatible" content="IE=Edge,chrome=1">
<title>資治通鑒_112-資治通鑑卷一百十_112-資治通鑑卷一百十</title>
<meta name="Keywords" content="資治通鑒_112-資治通鑑卷一百十_112-資治通鑑卷一百十">
<meta name="Description" content="資治通鑒_112-資治通鑑卷一百十_112-資治通鑑卷一百十">
<meta http-equiv="Cache-Control" content="no-transform" />
<meta http-equiv="Cache-Control" content="no-siteapp" />
<link href="/img/style.css" rel="stylesheet" type="text/css" />
<script src="/img/m.js?2020"></script> 
</head>
<body>
 <div class="ClassNavi">
<a  href="/24shi/">二十四史</a> | <a href="/SiKuQuanShu/">四库全书</a> | <a href="http://www.guoxuedashi.com/gjtsjc/"><font  color="#FF0000">古今图书集成</font></a> | <a href="/renwu/">历史人物</a> | <a href="/ShuoWenJieZi/"><font  color="#FF0000">说文解字</a></font> | <a href="/chengyu/">成语词典</a> | <a  target="_blank"  href="http://www.guoxuedashi.com/jgwhj/"><font  color="#FF0000">甲骨文合集</font></a> | <a href="/yzjwjc/"><font  color="#FF0000">殷周金文集成</font></a> | <a href="/xiangxingzi/"><font color="#0000FF">象形字典</font></a> | <a href="/13jing/"><font  color="#FF0000">十三经索引</font></a> | <a href="/zixing/"><font  color="#FF0000">字体转换器</font></a> | <a href="/zidian/xz/"><font color="#0000FF">篆书识别</font></a> | <a href="/jinfanyi/">近义反义词</a> | <a href="/duilian/">对联大全</a> | <a href="/jiapu/"><font  color="#0000FF">家谱族谱查询</font></a> | <a href="http://www.guoxuemi.com/hafo/" target="_blank" ><font color="#FF0000">哈佛古籍</font></a> 
</div>

 <!-- 头部导航开始 -->
<div class="w1180 head clearfix">
  <div class="head_logo l"><a title="国学大师官网" href="http://www.guoxuedashi.com" target="_blank"></a></div>
  <div class="head_sr l">
  <div id="head1">
  
  <a href="http://www.guoxuedashi.com/zidian/bujian/" target="_blank" ><img src="http://www.guoxuedashi.com/img/top1.gif" width="88" height="60" border="0" title="部件查字,支持20万汉字"></a>


<a href="http://www.guoxuedashi.com/help/yingpan.php" target="_blank"><img src="http://www.guoxuedashi.com/img/top230.gif" width="600" height="62" border="0" ></a>


  </div>
  <div id="head3"><a href="javascript:" onClick="javascript:window.external.AddFavorite(window.location.href,document.title);">添加收藏</a>
  <br><a href="/help/setie.php">搜索引擎</a>
  <br><a href="/help/zanzhu.php">赞助本站</a></div>
  <div id="head2">
 <a href="http://www.guoxuemi.com/" target="_blank"><img src="http://www.guoxuedashi.com/img/guoxuemi.gif" width="95" height="62" border="0" style="margin-left:2px;" title="国学迷"></a>
  

  </div>
</div>
  <div class="clear"></div>
  <div class="head_nav">
  <p><a href="/">首页</a> | <a href="/ShuKu/">国学书库</a> | <a href="/guji/">影印古籍</a> | <a href="/shici/">诗词宝典</a> | <a   href="/SiKuQuanShu/gxjx.php">精选</a> <b>|</b> <a href="/zidian/">汉语字典</a> | <a href="/hydcd/">汉语词典</a> | <a href="http://www.guoxuedashi.com/zidian/bujian/"><font  color="#CC0066">部件查字</font></a> | <a href="http://www.sfds.cn/"><font  color="#CC0066">书法大师</font></a> | <a href="/jgwhj/">甲骨文</a> <b>|</b> <a href="/b/4/"><font  color="#CC0066">解密</font></a> | <a href="/renwu/">历史人物</a> | <a href="/diangu/">历史典故</a> | <a href="/xingshi/">姓氏</a> | <a href="/minzu/">民族</a> <b>|</b> <a href="/mz/"><font  color="#CC0066">世界名著</font></a> | <a href="/download/">软件下载</a>
</p>
<p><a href="/b/"><font  color="#CC0066">历史</font></a> | <a href="http://skqs.guoxuedashi.com/" target="_blank">四库全书</a> |  <a href="http://www.guoxuedashi.com/search/" target="_blank"><font  color="#CC0066">全文检索</font></a> | <a href="http://www.guoxuedashi.com/shumu/">古籍书目</a> | <a   href="/24shi/">正史</a> <b>|</b> <a href="/chengyu/">成语词典</a> | <a href="/kangxi/" title="康熙字典">康熙字典</a> | <a href="/ShuoWenJieZi/">说文解字</a> | <a href="/zixing/yanbian/">字形演变</a> | <a href="/yzjwjc/">金 文</a> <b>|</b>  <a href="/shijian/nian-hao/">年号</a> | <a href="/diming/">历史地名</a> | <a href="/shijian/">历史事件</a> | <a href="/guanzhi/">官职</a> | <a href="/lishi/">知识</a> <b>|</b> <a href="/zhongyi/">中医中药</a> | <a href="http://www.guoxuedashi.com/forum/">留言反馈</a>
</p>
  </div>
</div>
<!-- 头部导航END --> 
<!-- 内容区开始 --> 
<div class="w1180 clearfix">
  <div class="info l">
   
<div class="clearfix" style="background:#f5faff;">
<script src='http://www.guoxuedashi.com/img/headersou.js'></script>

</div>
  <div class="info_tree"><a href="http://www.guoxuedashi.com">首页</a> > <a href="/SiKuQuanShu/fanti/">四库全书</a>
 > <h1>资治通鉴</h1> <!--         下载:【右键另存为】即可 --></div>
  <div class="info_content zj clearfix">
  
<div class="info_txt clearfix" id="show">
<center style="font-size:24px;">112-資治通鑑卷一百十</center>
    資治通鑑卷一百十一<br />
<br />
  宋 司馬光 撰<br />
<br />
  胡三省 音注<br />
<br />
  晉紀三十三【起屠維大淵獻盡上章困敦凡二年】<br />
<br />
  安皇帝丙<br />
<br />
  隆安三年春正月辛酉大赦 戊辰燕昌黎尹留忠謀反誅事連尚書令東陽公根尚書段成皆坐死遣中衛將軍衛雙就誅忠弟志於凡城以衛將軍平原公元為司徒尚書令 庚午魏主珪北廵分命大將軍常山王遵等三軍從東道出長川【長川在禦夷鎮西北大漠之東垂也下所謂西道中道盖絶漠分為三路】鎮北將軍高凉王樂真等七軍從西道出牛川珪自將大軍從中道出駁髯水以襲高車【將即亮翻駁北角翻髯而占翻】 壬午燕右將軍張真城門校尉和翰坐謀反誅癸未燕大赦改元長樂【樂音洛】燕主盛每十日一自决獄不加拷掠多得其情【拷音考掠音亮史言慕容盛以聰察殺身】 武威王烏孤徙治樂都【治直之翻樂音洛】以其弟西平公利鹿孤鎮安夷【安夷縣漢屬金城郡晉分屬西平郡】廣武公傉檀鎮西平【西平治樂都縣唐鄯州之湟水縣也傉奴沃翻】叔父素渥鎮湟河若留鎮澆河從弟替引鎮嶺南【嶺南即洪池嶺之南】洛囘鎮亷川從叔吐若留鎮浩亹【從才用翻浩亹在樂都之東隋唐併入湟水縣浩音誥亹音門】夷夏俊傑【夏戶雅翻】隨才授任内居顯位外典郡縣咸得其宜烏孤謂羣臣曰隴右河西本數郡之地【漢時河西置武威張掖酒泉四郡隴右置隴西金城二郡】遭亂分裂至十餘國呂氏乞伏氏段氏最彊今欲取之三者何先楊統曰乞伏氏本吾之部落終當服從【乞伏與禿髪氏皆鮮卑也】段氏書生無能為患且結好於我攻之不義【好呼到翻】呂光衰耄嗣子微弱【謂光以子紹為嗣也】纂弘雖有才而内相猜忌若使浩亹亷川乘虛迭出彼必疲於奔命不過二年兵勞民困則姑臧可圖也【姑臧呂光所都】姑臧舉則二寇不待攻而服矣烏孤曰善 二月丁亥朔魏軍大破高車三十餘部獲七萬餘口馬三十餘萬匹牛羊百四十餘萬頭衛王儀别將三萬騎絶漠千餘里【將即亮翻】破其七部獲二萬餘口馬五萬餘匹牛羊二萬餘頭高車諸部大震 林邑王范達陷日南九真遂寇交趾太守杜瑗擊破之【瑗于眷翻】 庚戌魏征虜將軍庾岳破張超於勃海斬之【張超據南皮見上卷上年】段業即凉王位改元天璽【是為北凉璽斯氏翻】以沮渠蒙遜為尚書左丞【沮子余翻】梁中庸為右丞 魏主珪大獵於牛川之南以高車人為圍周七百餘里因驅其禽獸南抵平城使高車築鹿苑廣數十里【廣古曠翻】三月己未珪還平城甲子珪分尚書三十六曹及外署凡置三百六十曹令八部大夫主之【八部大夫恐當作八部大人魏王珪天興元年置八部大人於皇城四方四維一面置一人以擬八座謂之八國各有屬官常侍待詔直左右出入王命】吏部尚書崔宏通署三十六曹如令僕統事置五經博士增國子太學生員合三千人珪問博士李先曰天下何物最善可以益人神智對曰莫若書籍珪曰書籍凡有幾何如何可集對曰自書契以來世有滋益以至于今不可勝計苟人主所好何憂不集珪從之命郡縣大索書籍悉送平城【魏主珪之崇文如此而魏之儒風及平凉州之後始振盖代北以右武為俗雖其君尚文未能回也嗚呼平凉之後儒風雖振而北人胡服至孝文遷洛之時未盡改也用夏變夷之難如是夫勝音升好呼到翻索晋客翻】初秦王登之弟廣帥衆三千依南燕王德德以為冠軍將軍處之乞活堡【帥讀曰率冠古玩翻乞活堡晉惠帝時諸賊保聚之地處昌呂翻】會熒惑守東井或言秦當復興【復扶又翻】廣乃自稱秦王擊南燕北地王鍾破之是時滑臺孤弱【德徙滑臺事見上卷上年】土無十城衆不過一萬鍾旣敗附德者多去德而附廣德乃留魯陽王和守滑臺自帥衆討廣斬之【帥讀曰率下同】燕主寶之至黎陽也【事見上卷上年】魯陽王和長史李辯勸和納之和不從辯懼故潛引晉軍至管城【事亦見上卷上年】欲因德出戰而作亂旣而德不出辯愈不自安及德討苻廣辯復勸和反【復扶又翻下同】和不從辯乃殺和以滑臺降魏【降戶江翻】魏行臺尚書和跋在鄴帥輕騎自鄴赴之【騎奇寄翻】旣至辨悔之閉門拒守跋使尚書郎鄧暉說之【鄧暉魏之鄴臺尚書郎也說輸芮翻】辯乃開門内跋跋悉收德宫人府庫德遣兵擊跋跋逆擊破之又破德將桂陽王鎮【將即亮翻】俘獲千餘人陳頴之民多附於魏【陳頴陳郡頴川也】南燕右衛將軍慕容雲斬李辯帥將士家屬二萬餘口出滑臺赴德【帥讀曰率】德欲攻滑臺韓範曰嚮也魏為客吾為主人今也吾為客魏為主人人心危懼不可復戰【復扶又翻】不如先據一方自立基本乃圖進取【微韓範之言德若進攻滑臺必至喪敗固不待慕容超之時也】張華曰彭城楚之舊都【項羽都彭城故云然】可攻而據之北地王鍾等皆勸德攻滑臺尚書潘聰曰滑臺四通八達之地【滑臺當滑津之要魏自北渡河而南向晉從清水入河秦沿渭順河而下皆湊於滑臺又其城旁無山陵可依車騎舟師皆可以騁故謂之四通八達之地】北有魏南有晉西有秦居之未嘗一日安也彭城土曠人稀平夷無嶮且晉之舊鎮未易可取【易以豉翻】又密邇江淮夏秋多水乘舟而戰者吳之所長我之所短也青州沃野二千里精兵十餘萬左有負海之饒右有山河之固廣固城曹嶷所築【嶷魚力翻】地形阻峻足為帝王之都三齊英傑思得明主以立功於世久矣辟閭渾昔為燕臣【孝武大元十九年辟閭渾為慕容農所破遂臣於燕】今宜遣辯士馳說於前大兵繼踵於後若其不服取之如拾芥耳【兼弱攻昧取亂侮亡自三代之時仲虺已有是言夫子定書弗之刪也後人泥古專言王者之師以仁義行之若宋襄公可以為鑒矣說輸芮翻】旣得其地然後閉關養鋭伺隙而動此乃陛下之關中河内也【用荀彧說魏武之言伺相吏翻】德猶豫未决沙門竺朗素善占候【竺朗之俗姓】德使牙門蘇撫問之朗曰敬覽三策潘尚書之議興邦之言也且今歲之初彗星起奎婁掃虚危彗者除舊布新之象奎婁為魯虛危為齊【晉天文志奎婁胃魯徐州虚危齊青州彗祥歲翻又旋芮翻又徐醉翻】宜先取兖州廵撫琅邪至秋乃北徇齊地此天道也撫又密問以年世朗以周易筮之曰燕衰庚戌年則一紀世則及子【其後燕亡於義熙六年歲在上章閹茂上章庚也閹茂戌也】撫還報德德乃引師而南兖州北鄙諸郡縣皆降之【降戶江翻下同】德置守宰以撫之禁軍士無得虜掠百姓大悦牛酒屬路【屬之欲翻】 丙子魏主珪遣建義將軍庾真越騎校尉奚斤擊庫狄宥連侯莫陳三部皆破之【其後庫狄侯莫陳二姓皆貴顯而宥連之種微矣】追奔至大峨谷置戍而還【還從宣翻又如字】 己卯追尊帝所生母陳夫人為德皇太后 夏四月鮮卑疊掘河内帥戶五千降于西秦西秦王乾歸以河内為疊掘都統以宗女妻之【疊掘亦鮮卑一種也河内其名掘其月翻妻七細翻】 甲午燕大赦 會稽王道子有疾【會工外翻】且無日不醉世子元顯知朝望去之乃諷朝廷解道子司徒揚州刺史【朝直遙翻下同】乙未以元顯為揚州刺史道子醒而後知之大怒無如之何元顯以廬江太守會稽張灋順為謀主【會工外翻】多引樹親黨朝貴皆畏事之【為元顯張灋順俱被誅張本】燕散騎常侍餘超左將軍高和等坐謀反誅【散悉亶翻騎奇】<br />
<br />
  【寄翻】 凉太子紹太原公纂將兵伐北凉【河西四郡張掖在北故號北凉將即亮翻】北凉王業求救於武威王烏孤烏孤遣驃騎大將軍利鹿孤及楊軌救之【驃匹妙翻騎奇寄翻】業將戰沮渠蒙遜諫曰楊軌恃鮮卑之彊有窺窬之志【秃髪本鮮卑種也沮子余翻】紹纂深入置兵死地不可敵也今不戰則有泰山之安戰則有累卵之危業從之案兵不戰紹纂引兵歸六月烏孤以利鹿孤為凉州牧鎮西平召車騎大將軍傉檀入録府國事【傉奴沃翻】 會稽世子元顯自以少年不欲頓居重任【少詩照翻】戊子以琅邪王德文為司徒 魏前河間太守盧溥帥其部曲數千家就食漁陽遂據有數郡秋七月己未燕主盛遣使拜溥幽州刺史【為下魏黜張衮襲禽盧漙張本帥讀曰率使疏吏翻】 辛酉燕主盛下詔曰法例律公侯有罪得以金帛贖【戰國時魏文侯師李悝撰次諸國法著法經以為王者之政莫急盜賊盜賊須劾捕故著網捕二篇其輕狡越城博戲假借不亷淫侈踰制以為雜律一篇又以其律具其加減故所著六篇皆罪名之制也漢蕭何條益事律興廏戶三篇合為九篇魏陳羣等采漢律制新律十八篇集罪例為刑名冠於律首盜律有劫略恐猲和賣買人科有持質皆非盜事分以為劫略律賊律有欺謾詐偽踰封矯制囚律有詐偽生死令丙有詐自復免事類衆多分為詐律賊律有賊伐樹木殺傷人畜產及諸亡印金布律有毁傷亡失縣官財物分為毁亡律囚律有告劾傳覆廏律有告反逮受科有登聞道辭分為告劾律囚律有繫囚鞠獄斷獄之法興律有上獄之事科有考事報讞宜别為篇分為繋訊斷獄律盜律有受所監受財枉法雜律有假借不亷令乙有呵人受錢科有使者驗賂其事相類分為請賕律盜律有勃辱強賊興律有擅興徭役具律有出賣科有擅作修舍事分為興擅律興律有乏徭稽留賊律有儲峙不辦廏律有乏軍乏興及舊典有奉法不謹不承用詔書漢氏施行不宜復以為灋别為之留律秦世舊有廏置乘傳副車食厨後漢但設騎置無車馬而律猶著其文則為虚設故除廏律取其可用合科者為郵驛令告劾律上言變事令以驚事告急與興律烽燧及科令者以為驚事律盜律有還贓卑主金布律有罸贖入責以呈黄金為價科有平庸坐贓事以為償贓律律之初制無免坐之文張湯趙禹始作監臨部主見知故縱之例其見知而故不舉劾以贖論其不見不知者不坐科條免坐繁多宜總為免例以省科文故更定以為免坐律晉初賈充定法就漢九章增十一篇改舊律為刑名法例辨囚律為告劾繋訊斷獄分盜律為請賕詐偽水火毁亡因事類為衛禁違制撰周官為諸侯律合二十篇孔頴達曰古之贖罪皆用銅漢始改用黄金但少其斤兩令與金相敵漢及後魏贖罪皆用黄金後魏以金難得合金一兩收絹十匹今律乃復依古贖銅】此不足以懲惡而利於王府甚無謂也自今皆令立功以自贖勿復輸金帛【復扶又翻】 西秦丞相南川宣公出連乞都卒【南川地名宣諡也】 秦齊公崇鎮東將軍楊佛嵩寇洛陽河南太守隴西辛恭靖嬰城固守雍州刺史楊佺期遣使求救於魏常山王遵【雍於用翻使疏吏翻下同】魏主珪以散騎侍郎西河張濟為遵從事中郎以報之佺期問於濟曰魏之伐中山戎士幾何濟曰四十餘萬【事見一百八卷孝武太元二十一年】佺期曰以魏之彊小羌不足滅也且晉之與魏本為一家【謂猗盧救劉琨時也】今旣結好【好呼到翻】義無所隱此間兵弱糧寡洛陽之救恃魏而已若其保全必有厚報若其不守與其使羌得之不若使魏得之【若楊佺期者豈可使之扞禦封疆哉】濟還報八月珪遣太尉穆崇將六萬騎往救之 燕遼西太守李朗在郡十年威行境内【燕遼西郡治令支】恐燕主盛疑之累徵不赴以其家在龍城未敢顯叛陰召魏兵許以郡降魏【降戶江翻下同】遣使馳詣龍城廣張寇勢盛曰此必詐也召使者詰問【詰去吉翻】果無事實盛盡滅朗族丁酉遣輔國將軍李旱討之 初魏奮武將軍張衮以才謀為魏主珪所信重委以腹心珪問中州士人於衮衮薦盧溥及崔逞珪皆用之珪圍中山久未下軍食乏【見一百九卷隆安元年】問計於羣臣逞為御史中丞對曰桑椹可以佐糧飛鴞食椹而改音詩人所稱也珪雖用其言聽民以椹當租然以逞為侮慢心銜之【詩翩彼飛鴞集于泮林食我桑椹懷我好音註云鴞惡聲之鳥也鴞恒惡鳴今食桑椹故改其鳴歸就我以善音珪本北人而入中原故銜逞以為侮慢】秦人寇襄陽雍州刺史郗恢以書求救於魏常山王遵曰賢兄虎步中原珪以恢無君臣之禮命衮及逞為復書必貶其主衮逞謂帝為貴主珪怒曰命汝貶之而謂之貴主何如賢兄也逞之降魏也【見一百九卷隆安元年】以天下方亂恐無復遺種【種章勇翻】使其妻張氏與四子留冀州逞獨與幼子賾詣平城【賾士革翻】所留妻子遂奔南燕珪并以是責逞賜逞死盧溥受燕爵命侵掠魏郡縣殺魏幽州刺史封沓干珪謂衮所舉皆非其人黜衮為尚書令史衮乃闔門不通人事惟手校經籍歲餘而終燕主寶之敗也中書令民部尚書封懿降於魏珪以懿為給事黄門侍郎都坐大官【魏官有三都大官都坐大官外都大官内都大官坐徂卧翻】珪問懿以燕氏舊事懿應對疎慢亦坐廢於家【珪盖自疑以為衣冠之士慢之也】 武威王秃髪烏孤醉走馬傷脇而卒遺令立長君【長知兩翻】國人立其弟利鹿孤諡烏孤曰武王廟號烈祖利鹿孤大赦徙治西平 南燕王德遣使說幽州刺史辟閭渾欲下之【晉氏南渡僑立幽冀青并四州於江北秦圍幽州刺史田洛于三阿是其證也孝武太元之季復取齊地徙幽冀二州于齊是後鎮齊者率領青冀二州刺史渾領幽州刺史盖自北而南未純為晉臣使領幽州而鎮廣固也說輸芮翻】渾不從德遣北地王鍾帥步騎二萬擊之【帥讀曰率】德進據琅邪徐兖之民歸附者十餘萬德自琅邪引兵而北以南海王法為兖州刺史鎮梁父【父音甫】進攻莒城守將任安委城走德以潘聰為徐州刺史鎮莒城【莒縣前漢屬城陽國後漢屬琅邪晉分屬東莞郡將即亮翻任音壬】蘭汗之亂燕吏部尚書封孚南奔辟閭渾渾表為勃海太守及德至孚出降【降戶江翻下同】德大喜曰孤得青州不為喜喜得卿耳遂委以機密北地王鍾傳檄青州諸郡諭以禍福辟閭渾徙八千餘家入守廣固遣司馬崔誕戍薄荀固平原太守張豁戍柳泉【薄荀盖人姓名遇亂聚衆保固此地因以為名齊人率謂保聚之地為固漢書地理志北海郡有柳泉侯國後漢晉省】誕豁承檄皆降於德渾懼携妻子奔魏德遣射聲校尉劉綱追之及於莒城斬之渾子道秀自詣德請與父俱死德曰父雖不忠【以辟閭渾背燕為不忠】而子能孝特赦之渾參軍張瑛為渾作檄【瑛音英為于偽翻】辭多不遜德執而讓之瑛神色自若徐曰渾之有臣猶韓信之有蒯通通遇漢祖而生【事見十二卷高祖十二年】臣遭陛下而死比之古人竊不不幸耳德殺之遂定都廣固 燕李旱行至建安燕主盛急召之羣臣莫測其故九月辛未復遣之李朗聞其家被誅【被皮義翻】擁二千餘戶以自固及聞旱還謂有内變不復設備【復扶又翻】留其子養守令支【應劭曰令音鈴師古曰音郎定翻孟康曰支音袛裴松支其兒翻】自迎魏師於北平【前漢北平郡治平剛後漢治土垠晉治徐無後魏治盧龍】壬子旱襲令支克之遣廣威將軍孟廣平追及朗於無終斬之【無終春秋無終子之國自漢以來為縣屬右北平劉昫曰唐薊州玉田縣漢無終縣地】 秦主興以災異屢見【見賢遍翻】降號稱王下詔令羣公卿士將牧守宰各降一等【將即亮翻守式又翻下同】大赦改元弘始存問孤貧舉拔賢俊簡省灋令清察獄訟守令之有政迹者賞之貪殘者誅之遠近肅然 冬十月甲午燕中衛將軍衛雙有罪賜死李旱還聞雙死懼棄軍而亡至板陘【陘音刑】復還歸罪【復扶又翻】燕主盛復其爵位謂侍中孫勍曰旱為將而棄軍罪在不赦【勍渠京翻將即亮翻】然昔先帝蒙塵骨肉離心公卿失節惟旱以宦者忠勤不懈始終如一【事見上卷二年】故吾念其功而赦之耳 辛恭靖固守百餘日魏救未至秦兵拔洛陽獲恭靖恭靖見秦王興不拜曰吾不為羌賊臣興囚之恭靖逃歸自淮漢以北諸城多請降送任於秦【降戶江翻】魏主珪以穆崇為豫州刺史鎮野王【秦旣克洛陽魏置鎮於野王以備其渡河侵軼】 會稽世子元顯性苛刻【會工外翻】生殺任意發東土諸郡免奴為客者【奴戶者有罪沒為官奴公卿以下至九品官及宗室國賓先賢之後及士人子孫占䕃以為客戶是謂免奴為客】號曰樂屬【樂徒各翻】移置京師以充兵役東土囂然苦之孫恩因民心騷動自海島帥其黨殺上虞令【上虞縣自漢以來屬會稽郡西北距郡城百餘里帥讀曰率】遂攻會稽會稽内史王凝之羲之之子也世奉天師道【天師道即張道陵之所傳也會工外翻】不出兵亦不設備日於道室稽顙跪呪【道室奉道之室也稽音啟】官屬請出兵討恩凝之曰我已請大道借鬼兵守諸津要各數萬賊不足憂也及恩漸近乃聽出兵恩已至郡下甲寅恩陷會稽凝之出走恩執而殺之并其諸子凝之妻謝道藴弈之女也聞寇至舉措自若命婢肩輿抽刀出門手殺數人乃被執【被皮義翻】吳國内史桓謙臨海太守新秦王崇【晉書作新蔡王崇崇汝南王祐之曾孫自其祖父以來嗣新蔡國封秦當作蔡】義興太守魏隱皆棄郡走於是會稽謝鍼吳郡陸瓌吳興丘尫【鍼其亷翻瓌姑回翻尫烏光翻】義興許允之臨海周胄永嘉張永等及東陽新安凡八郡人一時起兵殺長吏以應恩旬日之中衆數十萬吳興太守謝邈永嘉太守司馬逸嘉興公顧胤南康公謝明慧黄門郎謝冲張琨中書郎孔道等皆為恩黨所殺邈冲皆安之弟子也時三吳承平日久民不習戰故郡縣兵皆望風奔潰恩據會稽自稱征東將軍逼人士為官屬號其黨曰長生人民有不與之同者戮及嬰孩死者什七八醢諸縣令以食其妻子【食祥吏翻】不肯食者輒支解之【支解者隨其支節解剥若解牛然】所過掠財物燒邑屋焚倉廪刋木堙井【孔安國曰刋槎其木也堙塞也】相帥聚於會稽【帥讀曰率下同】婦人有嬰兒不能去者投於水中曰賀汝先登仙堂我當尋後就汝恩表會稽王道子及世子元顯之罪請誅之自帝即位以來内外乖異石頭以南皆為荆江所據以西皆豫州所專【江水自荆江二州界入揚州界皆東北流歷陽在江西建康在江東孫權築石頭城盖據江津之要衝也】京口及江北皆劉牢之及廣陵相高雅之所制朝政所行惟三吳而已【朝直遙翻】及孫恩作亂八郡皆為恩有【八郡會稽臨海永嘉東陽新安吳吳興義興也】畿内諸縣盜賊處處蠭起恩黨亦有潜伏在建康者人情危懼常慮竊發於是内外戒嚴加道子黄鉞元顯領中軍將軍命徐州刺史謝琰兼督吳興義興軍事以討恩劉牢之亦發兵討恩拜表輒行【牢之鎮京口】 西秦以金城太守辛靜為右丞相 十二月甲午燕燕郡太守高湖帥戶三千降魏湖泰之子也【為後高歡簒魏張本降戶江翻】 丙午燕主盛封弟淵為章武公䖍為博陵公子定為遼西公 丁未燕太后段氏卒諡曰惠德皇后 謝琰擊斬許允之迎魏隱還郡進擊丘尫破之與劉牢之轉鬬而前所向輒克琰留屯烏程【烏程縣前漢屬會稽郡後漢屬吳郡魏晉以來屬吳興郡】遣司馬高素助牢之進臨浙江【浙之列翻】詔以牢之都督吳郡諸軍事初彭城劉裕生而母死父翹僑居京口家貧將棄之同郡劉懷敬之母裕之從母也生懷敬未朞走往救之斷懷敬乳而乳之【從才用翻斷丁管翻上乳如字下乳人喻翻】及長勇健有大志僅識文字以賣履為業好樗蒲為鄉閭所賤【長知兩翻好呼到翻】劉牢之擊孫恩引裕參軍事【晉宋之制參軍不署曹者無定員】使將數十人覘賊遇賊數千人即迎擊之從者皆死【覘丑亷翻從才用翻】裕墜岸下賊臨岸欲下裕奮長刀仰斫殺數人乃得登岸仍大呼逐之【呼火故翻】賊皆走裕所殺傷甚衆劉敬宣怪裕久不返引兵尋之見裕獨驅數千人咸共歎息因進擊賊大破之斬獲千餘人【劉裕事始此】初恩聞八郡響應謂其屬曰天下無復事矣【復扶又翻】當與諸君朝服至建康【言欲踐位也朝直遙翻】旣而聞牢之臨江曰我割浙江以東不失作句踐【欲如越王句踐保有會稽也句音鈎】戊申牢之引兵濟江恩聞之曰孤不羞走【江表傳周瑜之破魏軍也曹公曰孤不羞走故恩引以為言】遂驅男女二十餘萬口東走多棄寶物子女於道官軍競取之恩由是得脱復逃入海島【復扶又翻下同】高素破恩黨於山陰【山陰縣屬會稽郡郡城以北皆縣界】斬恩所署吳郡太守陸瓌【瓌工回翻】吳興太守丘尫餘姚令吳與沈穆夫【餘姚縣屬會稽郡在郡城東二百餘里】東土遭亂企望官軍之至【企去智翻】既而牢之等縱軍士暴掠士民失望郡縣城中無復人跡月餘乃稍有還者朝廷憂恩復至以謝琰為會稽太守都督五郡軍事帥徐州文武戍海浦【五郡會稽臨海東陽永嘉新安也今自龕山而東至蘭風石堰鳴鶴松浦蟹浦定海皆海浦也帥讀曰率】以元顯録尚書事時人謂道子為東録元顯為西録西府車騎填輳東第門可張羅矣元顯無良師友所親信者率皆佞諛之人或以為一時英傑或以為風流名士由是元顯日益驕侈諷禮官立議以已德隆望重既録百揆百揆皆應盡敬【舜納于百揆禹宅百揆周官曰唐虞稽古建官維百内有百揆四岳外有州牧侯伯皆以百揆為官名孔安國曰揆度也舜舉八凱使揆度百事是古以百揆名官之義也晉人多以百揆為百官】於是公卿以下見元顯皆拜時軍旅數起國用虚竭自司徒以下日廪七升而元顯聚斂不已富踰帝室【為元顯亡國敗家張本數所角翻斂力贍翻】 殷仲堪恐桓玄跋扈乃與楊佺期結昏為援佺期屢欲攻玄仲堪每抑止之玄恐終為殷楊所滅乃告執政求廣其所統執政亦欲交構使之乖離乃加玄都督荆州四郡軍事【執政謂元顯荆州四郡謂長沙衡楊湘東零陵也】又以玄兄偉代佺期兄廣為南蠻校尉佺期忿懼楊廣欲拒桓偉仲堪不聽出廣為宜都建平二郡太守楊孜敬先為江夏相【夏戶雅翻相息亮翻】玄以兵襲而劫之以為諮議參軍佺期勒兵建牙聲言援洛欲與仲堪共襲玄仲堪雖外結佺期而内疑其心苦止之猶慮弗能禁遣從弟遹屯于北境【遹以律翻雍州治襄陽在江陵之北】以遏佺期佺期旣不能獨舉又不測仲堪本意乃解兵仲堪多疑少決【少詩沼翻】諮議參軍羅企生謂其弟遵生曰殷侯仁而無斷必及於難【企去智翻斷丁亂翻難乃旦翻】吾蒙知遇義不可去必將死之是歲荆州大水平地三丈仲堪竭倉廪以賑饑民【賑之忍翻】桓玄欲乘其虛而伐之乃發兵西上【上時掌翻】亦聲言救洛與仲堪書曰佺期受國恩而棄山陵宜共罪之【晉復洛陽以屬雍州統内故玄以棄山陵罪佺期】今當入沔討除佺期【沔彌兖翻】已頓兵江口若見與無貳【與許也從也黨也心持兩端為貳】可收楊廣殺之如其不爾便當帥兵入江【入江則欲攻江陵帥讀曰率下同】時巴陵有積穀玄先遣兵襲取之梁州刺史郭銓當之官路經夏口【夏戶雅翻】玄詐稱朝廷遣銓為已前鋒乃授以江夏之衆使督諸軍竝進密報兄偉令為内應偉遑遽不知所為【遑急也遽亦急也】自齎疏示仲堪仲堪執偉為質【質音致】令與玄書辭甚苦至玄曰仲堪為人無決常懷成敗之計為兒子作慮【為于偽翻】我兄必無憂也仲堪遣殷遹帥水軍七千至西江口【水經江水東至長沙下雋縣北湘水從南來注之江水又東左得二夏浦註云夏浦俗謂之西江口】玄使郭銓苻宏擊之【孝武太元十年苻宏來奔處之江州玄因以為將】遹等敗走玄頓已陵食其穀仲堪遣楊廣及弟子道護等拒之皆為玄所敗【敗補邁翻】江陵震駭城中乏食以胡麻廪軍士【胡麻今謂之芝麻粒小於粟而黑可以為油九炊九曝以為飯食之使人不飢廩當作稟給也】玄乘勝至零口【零口即靈溪入江之口】去江陵二十里仲堪急召楊佺期以自救佺期曰江陵無食何以待敵可來見就共守襄陽仲堪志在全軍保境不欲棄州逆走乃紿之曰【紿待亥翻】比來收集已有儲矣【比毗至翻近也】佺期信之帥步騎八千精甲耀日至江陵仲堪唯以飯餉其軍佺期大怒曰今茲敗矣不見仲堪與其兄廣共擊玄玄畏其鋭退軍馬頭【江陵縣南有江津戍戍南對馬頭岸】明日佺期引兵急擊郭銓幾獲之【幾居依翻】會玄兵至佺期大敗單騎奔襄陽仲堪出奔鄼城【鄼縣即蕭何所封之邑漢屬南陽郡晉分屬順陽郡鄼音贊】玄遣將軍馮該追佺期及廣皆獲而殺之傳首建康佺期弟思平從弟尚保孜敬逃入蠻中【從才用翻】仲堪聞佺期死將數百人將奔長安至冠軍城【冠軍縣即霍去病所封之邑屬南陽郡其地在唐鄧州臨湍縣南界冠古玩翻】該追獲之還至柞溪【水經註柞溪水出江陵縣北盖諸池散流咸所會合積以成川東流逕驛路水上有大橋仲堪縊處也又東注船官湖柞子各翻又在各翻】逼令自殺并殺殷道護仲堪奉天師道禱請鬼神不吝財賄而嗇於周急好為小惠以悦人【好呼到翻】病者自為診脉分藥【診止忍翻按脉以候病為診為于偽翻】用計倚伏煩密而短於鑒略故至於敗仲堪之走也文武無送者惟羅企生從之路經家門弟遵生曰作如此分離何可不一執手企生旋馬授手遵生有力因牽下之曰家有老母去將何之企生揮淚曰今日之事我必死之汝等奉養【企去智翻養羊尚翻下同】不失子道一門之中有忠與孝亦復何恨【復扶又翻】遵生抱之愈急仲堪於路待之見企生無脱理策馬而去及玄至荆州人士無不詣玄者企生獨不往而營理仲堪家事或曰如此禍必至矣企生曰殷侯遇我以國士為弟所制不得隨之共殄醜逆復何面目就桓求生乎【復扶又翻下同】玄聞之怒然待企生素厚先遣人謂曰若謝我當釋汝企生曰吾為殷荆州吏荆州敗不能救尚何謝為玄乃收之復遣人問企生欲何言企生曰文帝殺嵇康嵇紹為晉忠臣【殺嵇康事見七十八卷魏元帝景元三年嵇紹死事見八十五卷惠帝永興元年】從公乞一弟以養老母玄乃殺企生而赦其弟 凉王光疾甚立太子紹為天王自號太上皇帝以太原公纂為太尉常山公弘為司徒謂紹曰今國家多難三鄰伺隙【三鄰謂秃髪乞伏段業也難乃旦翻】吾沒之後使纂統六軍弘管朝政【朝直遙翻】汝恭已無為委重二兄庶幾可濟【幾居依翻】若内相猜忌則蕭牆之變旦夕至矣又謂纂弘曰永業才非撥亂【呂紹字永業】直以立嫡有常猥居元首【君為元首】今外有彊寇人心未寧汝兄弟緝睦則祚流萬世【緝當作輯】若内自相圖則禍不旋踵矣纂弘泣曰不敢又執纂手戒之曰汝性麤暴深為吾憂善輔永業勿聽讒言是日光卒【年六十三】紹秘不發喪纂排閤入哭盡哀而出紹懼以位讓之曰兄功高年長【長知兩翻】宜承大統纂曰陛下國之冢嫡臣敢奸之【奸音干】紹固讓纂不許驃騎將軍呂超謂紹曰纂為將積年【驃匹妙翻騎奇寄翻將即亮翻】威震内外臨喪不哀步高視遠必有異志宜早除之紹曰先帝言猶在耳奈何棄之吾以弱年負荷大任【荷下可翻】方賴二兄以寧家國縱其圖我我視死如歸終不忍有此意也卿勿復言【復扶又翻下同】纂見紹於湛露堂超執刀侍側目纂請收之紹弗許【為超終殺纂張本】超光弟寶之子也弘密遣尚書姜紀謂纂曰主上闇弱未堪多難【難乃旦翻】兄威恩素著宜為社稷計不可狥小節也纂於是夜帥壯士數百踰北城攻廣夏門弘帥東苑之衆斧洪範門【王隐晉書曰凉州城東西三里南北七里本匈奴所築及張氏之世又增築四城箱各千步東城命曰講武場北城命曰玄武圃皆殖園果有宫殿廣夏門洪範門皆中城門也帥讀曰率夏戶雅翻】左衛將軍齊從守融明觀【觀古玩翻】逆問之曰誰也衆曰太原公從曰國有大故主上新立太原公行不由道夜入禁城將為亂邪因抽劔直前斫纂中額【中竹仲翻】纂左右禽之纂曰義士也勿殺紹遣虎賁中郎將呂開帥禁兵拒戰於端門呂超帥卒二千赴之衆素憚纂皆不戰而潰纂入自青角門升謙光殿【青角門盖凉州中城之東門也謙光殿張駿所起自以專制河右而世執臣節雖謙而光故以名殿】紹登紫閣自殺呂超奔廣武纂憚弘兵彊以位讓弘弘曰弘以紹弟也而承大統衆心不順是以違先帝遺命而廢之慙負黄泉【杜預曰地中之泉故曰黄泉】今復踰兄而立豈弘之本志乎纂乃使弘出告衆曰先帝臨終受詔如此羣臣皆曰苟社稷有主誰敢違者纂遂即天王位【纂字永緒光之庶長子也】大赦改元咸寧諡光曰懿武皇帝廟號太祖諡紹曰隱王以弘為大都督督中外諸軍事大司馬車騎大將軍司隸校尉録尚書事改封番禾郡公【番音盤】纂謂齊從曰卿前斫我一何甚也從泣曰隱王先帝所立陛下雖應天順人而微心未達唯恐陛下不死何謂甚也纂賞其忠善遇之纂叔父征東將軍方鎮廣武纂遣使謂方曰【使疏吏翻】超實忠臣義勇可嘉但不識國家大體權變之宜方賴其用以濟世難【難乃旦翻】可以此意諭之超上疏陳謝纂復其爵位【為超殺纂張本】 是歲燕主盛以河間公熙為都督中外諸軍事尚書左僕射領中領軍 劉衛辰子文陳降魏【降戶江翻】魏主珪妻以宗女【妻七細翻】拜上將軍賜姓宿氏【魏内入諸姓有宿六斤氏改為宿氏盖使文陳與之合族屬】四年春正月壬子朔燕主盛大赦自貶號為庶人天王魏材官將軍和跋【漢置材官將軍領郡國材官士以出征師還則省晉魏以後置材官】<br />
<br />
  【將軍主工匠土木之事則漢右校令之任也】襲盧溥於遼西戊午克之【溥附燕見上年】禽溥及其子煥送平城車裂之燕主盛遣廣威將軍孟廣平救溥不及斬魏遼西守宰而還 乙亥大赦西秦王乾歸遷都苑川【乞㐲氏本居苑川乾歸遷于金城今復都苑川】 秃髪利鹿孤大赦改元建和 高句麗王安事燕禮慢【句如字又音駒麗力知翻】二月丙申燕王盛自將兵三萬襲之【將即亮翻】以驃騎大將軍熙為前鋒拔新城南蘇二城開境七百餘里徙五千餘戶而還【還從宣翻又如字】熙勇冠諸將盛曰叔父雄果有世祖之風【慕容垂廟號世祖冠古玩翻下同】但弘略不如耳 初魏主珪納劉頭眷之女寵冠後庭生子嗣及克中山【克中山見一百九卷隆安元年】獲燕主寶之幼女將立皇后用其國故事鑄金人以卜之劉氏所鑄不成慕容氏成三月戊午立慕容氏為皇后【北史曰魏故事將立皇后必令手鑄金人以成者為吉不則不得立也】桓玄既克荆雍【雍於用翻】表求領荆江二州詔以玄為都督荆司雍秦梁益寧七州諸軍事荆州刺史以中護軍桓脩為江州刺史玄上疏固求江州於是進玄督八州及揚豫八郡諸軍事復領江州刺史【復扶又翻】玄輒以兄偉為雍州刺史朝廷不能違又以從子振為淮南太守【玄既督八州及揚豫八郡則西極岷嶓東盡歷陽蕪湖皆其統内矣漢晉淮南郡本治夀春成帝時祖約蘇峻為亂胡寇又屢至民南渡江者轉多乃於江南僑立淮南郡後又割丹陽之于湖為淮南境玄遣振守之是逼建康之漸也從才用翻】 凉王纂以大司馬弘功高地逼忌之弘亦自疑遂以東苑之兵作亂攻纂纂遣其將焦辨擊之【將即亮翻】弘衆潰出走纂縱兵大掠悉以東苑婦女賞軍弘之妻子亦在中纂笑謂羣臣曰今日之戰何如侍中房晷對曰天禍凉室憂患仍臻先帝始崩隱王廢黜山陵甫訖大司馬稱兵京師流血昆弟接刃雖弘自取夷滅亦由陛下無常棣之恩【左傳富辰曰召穆公思周德之不類糾合宗族於成周而作詩曰常棣之華鄂不韡韡几今之人莫如兄弟其四章曰兄弟閱于牆外禦其侮如是則兄弟雖有小忿不廢懿親】當省已責躬以謝百姓【省悉景翻】乃更縱兵大掠囚辱士女釁自弘起百姓何罪且弘妻陛下之弟婦弘女陛下之姪也奈何使無賴小人辱為婢妾天地神明豈忍見此遂歔欷流涕【歔音虚欷許既翻又音希】纂改容謝之召弘妻子寘於東宫厚撫之弘將犇秃髪利鹿孤道過廣武詣呂方方見之大哭曰天下甚寛汝何為至此乃執弘送獄纂遣力士康龍就拉殺之【拉盧合翻】纂立妃楊氏為后以后父桓為尚書左僕射凉都尹【凉都姑臧改武威太守為凉都尹】 辛卯燕襄平令段登等謀反誅 凉王纂將伐武威王利鹿孤中書令楊頴諫曰利鹿孤上下用命國未有釁不可伐也不從利鹿孤使其弟傉檀拒之【傉奴沃翻】夏四月傉檀敗凉兵於三堆【三堆在浩亹河南敗補邁翻】斬首二千餘級 初隴西李暠好文學有令名【暠古老翻好呼到翻】嘗與郭黁及同母弟敦煌宋繇同宿【黁奴昆翻敦徒門翻】黁起謂繇曰君當位極人臣李君終當有國家有騍馬生白額駒【騍馬牝馬也騍音課晉書作騧】此其時也及孟敏為沙州刺史以暠為效穀令【效穀縣自漢以來屬敦煌郡師古曰本魚澤障也桑欽說孝武元封六年濟南崔不意為魚澤尉教力田以勤效得穀因立為縣名後周併入敦煌縣】宋繇事北凉王業為中散常侍【以中散大夫常侍左右也散悉亶翻】孟敏卒敦煌護軍馮翊郭謙沙州治中敦煌索仙等【索昔各翻】以暠溫毅有惠政推為敦煌太守【敦徒門翻】暠初難之會宋繇自張掖告歸謂暠曰段王無遠略終必無成兄忘郭黁之言邪白額駒今已生矣暠乃從之遣使請命於業【使疏吏翻】業因以暠為敦煌太守右衛將軍敦煌索嗣言於業曰李暠不可使處敦煌【索昔各翻處昌呂翻】業遂以嗣代暠為敦煌太守使帥五百騎之官【帥讀曰率】嗣未至二十里移暠迎己【未至敦煌讒二十里移書於暠使迎已也】暠驚疑將出迎之效穀令張邈及宋繇止之曰段王闇弱正是英豪有為之日將軍據一國成資奈何拱手授人嗣自恃本郡謂人情附已不意將軍猝能拒之可一戰擒也暠從之先遣繇見嗣啗以甘言【啗徒敢翻又徒陷翻】繇還謂暠曰嗣志驕兵弱易取也【易以豉翻】暠乃遣邈繇與其二子歆讓逆撃嗣嗣敗走還張掖暠素與嗣善尤恨之表業請誅嗣沮渠男成亦惡嗣勸業除之業乃殺嗣【段業既失張掖又殺索嗣以自翦其羽翼所以終死於沮渠蒙遜之手惡烏路翻】遣使謝暠進暠都督凉興以西諸軍事鎮西將軍【段業分敦煌之凉興烏澤晉昌之宜禾為凉興郡至宇文氏併晉之廣至宜安淵泉合為凉興縣隋唐瓜州之常樂縣即其地也】 吐谷渾視羆卒世子樹洛干方九歲弟烏紇堤立妻樹洛干之母念氏生慕璝慕延【璝古回翻】烏紇堤懦弱荒淫不能治國【治直之翻】念氏專制國事有膽智國人畏服之 燕前將軍段璣太后段氏之兄子也為段登辭所連及五月壬子逃奔遼西【為後段璣等弑盛張本盛懲蘭汗嚴刑以繩下亦終於身死人手人而不仁疾之已甚亂也】 丙寅衛將軍東亭獻侯王珣卒 己巳魏主珪東如涿鹿西如馬邑觀灅源【灅力水翻】戊寅燕段璣復還歸罪【復扶又翻】燕王盛赦之賜號曰思<br />
<br />
  悔侯使尚公主入直内殿 謝琰以資望鎮會稽【資謂門地成資望謂時望會工外翻】不能綏懷又不為武備諸將咸諫曰賊近在海浦伺人形便宜開其自新之路琰不從曰苻堅之衆百萬尚送死淮南【琰與謝玄同破苻堅遂輕孫恩】孫恩小賊敗死入海何能復出【復扶又翻下同】若其果出是天欲殺之也旣而恩寇浹口【浹口今在明州定海縣虎蹲山外浹即叶翻杜佑曰浹口在明州鄮縣東北七十里】入餘姚破上虞進及邢浦【晉書曰邢浦去山陰北三十五里】琰遣參軍劉宣之擊破之恩退走少日復寇邢浦【少詩沼翻復扶又翻下同】官軍失利恩乘勝徑進己卯至會稽琰尚未食曰要當先滅此賊而後食因跨馬出戰兵敗為帳下都督張猛所殺吳興太守庾桓恐郡民復應恩殺男女數千人恩轉寇臨海朝廷大震遣冠軍將軍桓不才【冠古玩翻】輔國將軍孫無終寧朔將軍高雅之拒之 秦征西大將軍隴西公碩德將兵五千伐西秦【五千恐少當考】入自南安峽【南安峽在唐秦州隴城縣界德將即亮翻】西秦王乾歸帥諸將拒之【帥讀曰率】軍于隴西 楊軌田玄明謀殺武威王利鹿孤利鹿孤殺之【隆安二年楊軌降利鹿孤】 六月庚辰朔日有食之 以琅邪王師何澄為尚書左僕射【晉諸王置師友文學各一人初避景帝諱改師為傅後以桃廟不諱復為師】澄準之子也【何準見一百卷穆帝升平元年】 甲子燕大赦 凉王纂將襲北凉姜紀諫曰盛夏農事方殷且宜息兵今遠出嶺西【自姑臧西北出張掖其間有大嶺度嶺而西西郡當其要】秃髪氏乘虚襲京師將若之何不從進圍張掖西掠建康秃髪傉檀聞之將萬騎襲姑臧纂弟隴西公緯憑北城以自固傉檀置酒朱明門上鳴鐘鼓饗將士曜兵於青陽門【朱明門姑臧城南門也青陽門東門也傉奴沃翻】掠八千餘戶而去纂聞之引兵還 秋七月壬子太皇太后李氏崩 丁卯大赦 西秦王乾歸使武衛將軍慕兀等屯守秦軍樵采路絶秦王興潜引兵救之乾歸聞之使慕兀帥中軍二萬屯栢楊【水經註伯陽水出伯陽谷在董亭東又東有伯陽城城南謂之伯陽川盖李耳西入往逕所由故川原畎谷往往播其名後又訛為栢楊五代志天水郡秦嶺縣後魏置伯陽縣隋開皇中更名秦嶺唐併秦嶺入清水縣帥讀曰率下同】鎮軍將軍羅敦帥外軍四萬屯侯辰谷乾歸自將輕騎數千前候秦兵【將即亮翻騎奇寄翻下同】會大風昏霧與中軍相失為追騎所逼入於外軍旦與秦戰大敗走歸苑川其部衆三萬六千皆降於秦興進軍枹罕【降戶江翻枹音膚】乾歸奔金城謂諸豪帥曰【帥所類翻】吾不才叨竊名號已踰一紀【孝武太元十三年乾歸嗣國至是十三年】今敗散如此無以待敵欲西保允吾【允吾縣漢屬金城郡晉志省劉昫曰唐鄯州龍支縣漢允吾縣允吾音鈆牙】若舉國而去必不得免卿等留此各以其衆降秦以全宗族勿吾隨也皆曰死生願從陛下乾歸曰吾今將寄食於人若天未亡我庶幾異日克復舊業【幾居希翻】復與卿等相見今相隨而死無益也乃大哭而别乾歸獨引數百騎奔允吾乞降於武威王利鹿孤利鹿孤遣廣武公傉檀迎之置於晉興【張軌分西平界置晉興郡闞駰曰允吾縣西四十里有小晉興城傉奴沃翻】待以上賓之禮鎮北將軍秃髪俱延言於利鹿孤曰乾歸本吾之屬國因亂自尊今勢窮歸命非其誠欵若逃歸姚氏必為國患不如徙置乙弗之間【乙弗亦鮮卑種居西北海史曰吐谷渾北有乙弗勿敵國國有曲海海周囬千餘里種有萬落風俗與吐谷渾同北史又曰乙弗世為吐谷渾渠帥居青海號青海王】使不得去利鹿孤曰彼窮來歸我而逆疑其心何以勸來者俱延利鹿孤之弟也秦兵既退南羌梁戈等密招乾歸乾歸將應之其臣屋引阿洛以告晉興太守隂暢暢馳白利鹿孤利鹿孤遣其弟吐雷帥騎三千屯捫天嶺【捫天嶺在允吾東南】乾歸懼為利鹿孤所殺謂其太子熾磐曰【熾昌志翻】吾父子居此必不為利鹿孤所容今姚氏方強吾將歸之若盡室俱行必為追騎所及吾以汝兄弟及汝母為質【質音致】彼必不疑吾在長安彼終不敢害汝也乃送熾磐等於西平八月乾歸南奔枹罕遂降於秦 丁亥尚書右僕射王雅卒 九月癸丑地震 凉呂方降於秦廣武民三千餘戶奔武威王利鹿孤【呂方鎮廣武既降於秦其民無主故奔秃髪氏】 冬十一月高雅之與孫恩戰於餘姚雅之敗走山陰死者什七八詔以劉牢之都督會稽等五郡帥衆擊恩【會工外翻帥讀曰率】恩走入海牢之東屯上虞使劉裕戍句章【句章縣自漢以來屬會稽郡今鄞縣以東定海昌國皆其地也】吳國内史袁崧築滬瀆壘以備恩崧喬之孫也【袁崧當作袁山松滬瀆今在平江府吳縣東陸龜蒙叙矢魚之具云列竹於海澨曰滬是瀆以此得名吳都記松江東瀉海名曰滬瀆輿地記曰扈業者濱海捕魚之名挿竹列於海中以䋲編之向岸張兩翼潮上即沒潮落即出魚隨海潮礙竹不得去名曰扈瀆范成大吳郡志曰列竹於海澨曰滬吳之滬瀆是也自滬瀆泝松江至吳郡將門將門今訛為匠門袁喬見九十七卷穆帝永和二年三年】 會稽世子元顯求領徐州詔以元顯為開府儀同三司都督揚豫徐兖青幽冀并荆江司雍梁益交廣十六州諸軍事領徐州刺史【雍於用翻】封其子彦瑋為東海王 乞伏乾歸至長安秦王興以為都督河南諸軍事河州刺史歸義侯【此河南謂金城河之南】久之乞伏熾磐欲逃詣乾歸武威王利鹿孤追獲之利鹿孤將殺熾磐廣武公傉檀曰子而歸父無足深責宜宥之以示大度利鹿孤從之【秃髪傉檀勸其兄宥熾磬而卒死於熾磐之手豈非養虎自遺患乎】 秦王興遣晉將劉嵩等二百餘人來歸【劉嵩等盖因洛陽陷而沒於秦將即亮翻】北凉晉昌太守唐瑶叛移檄六郡【六郡盖敦煌酒泉晉昌凉興建康祁連也】推李暠為冠軍大將軍沙州刺史凉公領敦煌太守【暠古老翻冠古玩翻】暠赦其境内改元庚子【北凉之地至此又分為西凉】以瑤為征東將軍郭謙為軍諮祭酒索仙為左長史【索昔各翻】張邈為右長史尹建興為左司馬張體順為右司馬遣從事中郎宋繇東伐凉興并擊玉門已西諸城皆下之酒泉太守王德亦叛北凉自稱河州刺史北凉王業使沮渠蒙遜討之德焚城將部曲奔唐瑤蒙遜追至沙頭大破之【沙頭縣本屬酒泉郡惠帝分屬晉昌郡沮子余翻將子亮翻】虜其妻子部落而還【還從宣翻又如字】 十二月戊寅有星孛于天津【天文志天津九星横河中一曰天漢一曰天江主四瀆津梁所以度神通四方也孛蒲内翻】會稽世子元顯以星變解録尚書事【會工外翻】復加尚書令【復扶又翻】吏部尚書車胤以元顯驕恣【車尺遮翻】白會稽王道子請禁抑之元顯聞而未察以問道子曰車武子屏人言及何事【車胤字武子屏必郢翻】道子弗答固問之道子怒曰爾欲幽我不令我與朝士語邪【朝直遙翻】元顯出謂其徒曰車胤間我父子密遣人責之胤懼自殺【間古莧翻】 壬辰燕主盛立燕臺統諸部雜夷【二趙以來皆立單于臺以統雜夷盛仍此立之】 魏太史屢奏天文乖亂魏主珪自覽占書多云改王易政乃下詔風勵羣下以帝王繼統皆有天命不可妄干【風讀曰諷】又數變易官名欲以厭塞災異【數所角翻厭於葉翻塞悉側翻】儀曹郎董謐獻服餌仙經珪置仙人博士立仙坊煮鍊百藥封西山以供薪蒸【西山平城西山也毛晃曰麄曰薪細曰蒸】藥成令死罪者試服之多死不驗而珪猶信之訪求不已珪常以燕主垂諸子分據勢要使權柄下移遂至敗亡深非之博士公孫表希旨上韓非書【上時掌翻】勸珪以法制御下左將軍李粟性簡慢【粟或作栗】常對珪舒放不肅咳唾任情【咳口慨翻】珪積其宿過遂誅之羣下震栗【史言魏主珪悖暴于治】 丁酉燕王盛尊獻莊后丁氏為皇太后立遼西公定為皇太子大赦 是歲南燕王德即皇帝位于廣固【德字玄明皝之少子也】大赦改元建平更名備德【更工衡翻】欲使吏民易避【易以豉翻】追諡燕主暐曰幽皇帝以北地王鍾為司徒慕輿拔為司空封孚為左僕射慕輿護為右僕射立妃段氏為皇后<br />
<br />
  資治通鑑卷一百十一  <br>
   </div> 

<script src="/search/ajaxskft.js"> </script>
 <div class="clear"></div>
<br>
<br>
 <!-- a.d-->

 <!--
<div class="info_share">
</div> 
-->
 <!--info_share--></div>   <!-- end info_content-->
  </div> <!-- end l-->

<div class="r">   <!--r-->



<div class="sidebar"  style="margin-bottom:2px;">

 
<div class="sidebar_title">工具类大全</div>
<div class="sidebar_info">
<strong><a href="http://www.guoxuedashi.com/lsditu/" target="_blank">历史地图</a></strong>  
<a href="http://www.880114.com/" target="_blank">英语宝典</a>  
<a href="http://www.guoxuedashi.com/13jing/" target="_blank">十三经检索</a> 
<br><strong><a href="http://www.guoxuedashi.com/gjtsjc/" target="_blank">古今图书集成</a></strong> 
<a href="http://www.guoxuedashi.com/duilian/" target="_blank">对联大全</a> <strong><a href="http://www.guoxuedashi.com/xiangxingzi/" target="_blank">象形文字典</a></strong> 

<br><a href="http://www.guoxuedashi.com/zixing/yanbian/">字形演变</a>  <strong><a href="http://www.guoxuemi.com/hafo/" target="_blank">哈佛燕京中文善本特藏</a></strong>
<br><strong><a href="http://www.guoxuedashi.com/csfz/" target="_blank">丛书&方志检索器</a></strong> <a href="http://www.guoxuedashi.com/yqjyy/" target="_blank">一切经音义</a>  

<br><strong><a href="http://www.guoxuedashi.com/jiapu/" target="_blank">家谱族谱查询</a></strong>  <strong><a href="http://shufa.guoxuedashi.com/sfzitie/" target="_blank">书法字帖欣赏</a></strong> 
<br>

</div>
</div>


<div class="sidebar" style="margin-bottom:0px;">

<font style="font-size:22px;line-height:32px">QQ交流群9:489193090</font>


<div class="sidebar_title">手机APP 扫描或点击</div>
<div class="sidebar_info">
<table>
<tr>
	<td width=160><a href="http://m.guoxuedashi.com/app/" target="_blank"><img src="/img/gxds-sj.png" width="140"  border="0" alt="国学大师手机版"></a></td>
	<td>
<a href="http://www.guoxuedashi.com/download/" target="_blank">app软件下载专区</a><br>
<a href="http://www.guoxuedashi.com/download/gxds.php" target="_blank">《国学大师》下载</a><br>
<a href="http://www.guoxuedashi.com/download/kxzd.php" target="_blank">《汉字宝典》下载</a><br>
<a href="http://www.guoxuedashi.com/download/scqbd.php" target="_blank">《诗词曲宝典》下载</a><br>
<a href="http://www.guoxuedashi.com/SiKuQuanShu/skqs.php" target="_blank">《四库全书》下载</a><br>
</td>
</tr>
</table>

</div>
</div>


<div class="sidebar2">
<center>


</center>
</div>

<div class="sidebar"  style="margin-bottom:2px;">
<div class="sidebar_title">网站使用教程</div>
<div class="sidebar_info">
<a href="http://www.guoxuedashi.com/help/gjsearch.php" target="_blank">如何在国学大师网下载古籍?</a><br>
<a href="http://www.guoxuedashi.com/zidian/bujian/bjjc.php" target="_blank">如何使用部件查字法快速查字?</a><br>
<a href="http://www.guoxuedashi.com/search/sjc.php" target="_blank">如何在指定的书籍中全文检索?</a><br>
<a href="http://www.guoxuedashi.com/search/skjc.php" target="_blank">如何找到一句话在《四库全书》哪一页?</a><br>
</div>
</div>


<div class="sidebar">
<div class="sidebar_title">热门书籍</div>
<div class="sidebar_info">
<a href="/so.php?sokey=%E8%B5%84%E6%B2%BB%E9%80%9A%E9%89%B4&kt=1">资治通鉴</a> <a href="/24shi/"><strong>二十四史</strong></a>&nbsp; <a href="/a2694/">野史</a>&nbsp; <a href="/SiKuQuanShu/"><strong>四库全书</strong></a>&nbsp;<a href="http://www.guoxuedashi.com/SiKuQuanShu/fanti/">繁体</a>
<br><a href="/so.php?sokey=%E7%BA%A2%E6%A5%BC%E6%A2%A6&kt=1">红楼梦</a> <a href="/a/1858x/">三国演义</a> <a href="/a/1038k/">水浒传</a> <a href="/a/1046t/">西游记</a> <a href="/a/1914o/">封神演义</a>
<br>
<a href="http://www.guoxuedashi.com/so.php?sokeygx=%E4%B8%87%E6%9C%89%E6%96%87%E5%BA%93&submit=&kt=1">万有文库</a> <a href="/a/780t/">古文观止</a> <a href="/a/1024l/">文心雕龙</a> <a href="/a/1704n/">全唐诗</a> <a href="/a/1705h/">全宋词</a>
<br><a href="http://www.guoxuedashi.com/so.php?sokeygx=%E7%99%BE%E8%A1%B2%E6%9C%AC%E4%BA%8C%E5%8D%81%E5%9B%9B%E5%8F%B2&submit=&kt=1"><strong>百衲本二十四史</strong></a>  <a href="http://www.guoxuedashi.com/so.php?sokeygx=%E5%8F%A4%E4%BB%8A%E5%9B%BE%E4%B9%A6%E9%9B%86%E6%88%90&submit=&kt=1"><strong>古今图书集成</strong></a>
<br>

<a href="http://www.guoxuedashi.com/so.php?sokeygx=%E4%B8%9B%E4%B9%A6%E9%9B%86%E6%88%90&submit=&kt=1">丛书集成</a> 
<a href="http://www.guoxuedashi.com/so.php?sokeygx=%E5%9B%9B%E9%83%A8%E4%B8%9B%E5%88%8A&submit=&kt=1"><strong>四部丛刊</strong></a>  
<a href="http://www.guoxuedashi.com/so.php?sokeygx=%E8%AF%B4%E6%96%87%E8%A7%A3%E5%AD%97&submit=&kt=1">說文解字</a> <a href="http://www.guoxuedashi.com/so.php?sokeygx=%E5%85%A8%E4%B8%8A%E5%8F%A4&submit=&kt=1">三国六朝文</a>
<br><a href="http://www.guoxuedashi.com/so.php?sokeytm=%E6%97%A5%E6%9C%AC%E5%86%85%E9%98%81%E6%96%87%E5%BA%93&submit=&kt=1"><strong>日本内阁文库</strong></a> <a href="http://www.guoxuedashi.com/so.php?sokeytm=%E5%9B%BD%E5%9B%BE%E6%96%B9%E5%BF%97%E5%90%88%E9%9B%86&ka=100&submit=">国图方志合集</a> <a href="http://www.guoxuedashi.com/so.php?sokeytm=%E5%90%84%E5%9C%B0%E6%96%B9%E5%BF%97&submit=&kt=1"><strong>各地方志</strong></a>

</div>
</div>


<div class="sidebar2">
<center>

</center>
</div>
<div class="sidebar greenbar">
<div class="sidebar_title green">四库全书</div>
<div class="sidebar_info">

《四库全书》是中国古代最大的丛书,编撰于乾隆年间,由纪昀等360多位高官、学者编撰,3800多人抄写,费时十三年编成。丛书分经、史、子、集四部,故名四库。共有3500多种书,7.9万卷,3.6万册,约8亿字,基本上囊括了古代所有图书,故称“全书”。<a href="http://www.guoxuedashi.com/SiKuQuanShu/">详细>>
</a>

</div> 
</div>

</div>  <!--end r-->

</div>
<!-- 内容区END --> 

<!-- 页脚开始 -->
<div class="shh">

</div>

<div class="w1180" style="margin-top:8px;">
<center><script src="http://www.guoxuedashi.com/img/plus.php?id=3"></script></center>
</div>
<div class="w1180 foot">
<a href="/b/thanks.php">特别致谢</a> | <a href="javascript:window.external.AddFavorite(document.location.href,document.title);">收藏本站</a> | <a href="#">欢迎投稿</a> | <a href="http://www.guoxuedashi.com/forum/">意见建议</a> | <a href="http://www.guoxuemi.com/">国学迷</a> | <a href="http://www.shuowen.net/">说文网</a><script language="javascript" type="text/javascript" src="https://js.users.51.la/17753172.js"></script><br />
  Copyright &copy; 国学大师 古典图书集成 All Rights Reserved.<br>
  
  <span style="font-size:14px">免责声明:本站非营利性站点,以方便网友为主,仅供学习研究。<br>内容由热心网友提供和网上收集,不保留版权。若侵犯了您的权益,来信即刪。scp168@qq.com</span>
  <br />
ICP证:<a href="http://www.beian.miit.gov.cn/" target="_blank">鲁ICP备19060063号</a></div>
<!-- 页脚END --> 
<script src="http://www.guoxuedashi.com/img/plus.php?id=22"></script>
<script src="http://www.guoxuedashi.com/img/tongji.js"></script>

</body>
</html>
