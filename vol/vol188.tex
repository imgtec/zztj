\section{資治通鑑卷一百八十八}


宋 司馬光 撰

胡三省 音註

唐紀四|{
	起屠維單閼十二月盡重光大荒落二月凡一年有奇}


高祖神堯大聖孝皇帝中之上

武德二年十一月己卯劉武周寇浩州|{
	武周復寇西河}
秦王世民引兵自龍門乘冰堅度河屯栢壁|{
	栢壁在龍門關東北宋白曰栢壁在正平縣西南二十里正平絳州治所}
與宋金剛相持時河東州縣|{
	此河東通言大河以東非專指河東一郡}
俘掠之餘未有倉廪人情恇擾|{
	恇去王翻恇懼也}
聚入城堡徵歛無所得|{
	歛力瞻翻}
軍中乏食世民發教諭民民聞世民為帥而來莫不歸附|{
	此豈可以聲音笑貌致之帥所類翻}
自近及遠至者日多然後漸收其糧食軍食以充|{
	孔子曰足食足兵民信之矣民已信之足食足兵當知所先後也}
乃休兵秣馬唯令偏禆乘間抄掠|{
	間古莧翻抄楚交翻}
大軍堅壁不戰由是賊勢日衰世民嘗自帥輕騎覘敵|{
	帥讀曰率騎奇寄翻下同覘丑廉翻又丑艶翻下同}
騎皆四散世民獨與一甲士登丘而寢俄而賊兵四合初不之覺會有蛇逐鼠觸甲士之面甲士驚寤遂白世民俱上馬|{
	史言世民之有天命上時掌翻}
馳百餘步為賊所及世民以大羽箭射殪其驍將|{
	史言世民不惟有天命亦武藝絶人射而亦翻殪一計翻驍堅堯翻下同將即亮翻下同}
賊騎乃退 李世勣欲歸唐恐禍及其父謀於郭孝恪孝恪曰吾新事竇氏動則見疑宜先立效以取信然後可圖也世勣從之襲王世充獲嘉破之|{
	是年閏二月王世充殺李育德取獲嘉}
多所俘獲以獻建德建德由是親之初漳南人劉黑闥少驍勇狡獪|{
	舊志具州漳南縣漢東陽縣地後魏省東陽縣隋開皇六年分棗強清平二縣地復置東陽縣於東陽古城十八年改為漳南宋白曰取地居漳水之南為名少詩照翻驍堅堯翻獪古外翻}
與竇建德善後為羣盜轉事郝孝德李密王世充世充以為騎將每見世充所為竊笑之世充使黑闥守新鄉|{
	舊志隋分汲獲嘉二縣地於古新樂城置新鄉縣時屬義州後屬殷州宋白曰漢書武帝將幸緱氏至汲縣之新中鄉即此地新樂城十六國時燕樂安王臧所築}
李世勣擊虜之獻於建德建德署為將軍賜爵漢東公常使將奇兵東西掩襲或濳入敵境覘視虚實黑闥往往乘間奮擊克獲而還|{
	間古莧翻還從宣翻}
十二月庚申上獵于華山|{
	華戶化翻}
于筠說永安王孝基急攻呂崇茂|{
	說輸芮翻}
獨孤懷恩請先成攻具然後進孝基從之崇茂求救於宋金剛金剛遣其將善陽尉遲敬德尋相將兵奄至夏縣孝基表裏受敵軍遂大敗|{
	將即亮翻舊志朔州善陽縣漢定襄縣地有秦時馬邑城武周塞後魏置桑乾郡隋大業初置善陽縣尉紆勿翻尋姓也姓苑晉有尋曾相息亮翻是年十月呂崇茂據夏縣夏戶雅翻 考異曰高祖實録云戰于下邽縣按下邽乃在關中去夏縣殊遠實録之誤也今從舊書孝基傳}
孝基懷恩筠唐儉及行軍總管劉世讓皆為所虜敬德名恭以字行上徵裴寂入朝責其敗軍下吏|{
	朝直遥翻敗補邁翻下遐嫁翻}
既而釋之寵待彌厚尉遲敬德尋相將還澮州|{
	澮古外翻}
秦王世民遣兵部尚書殷開山總管秦叔寶等邀之於美良川大破之斬首二千餘級頃之敬德尋相濳引精騎援王行本於蒲反|{
	騎奇寄翻下同}
世民自將步騎三千從間道夜趨安邑|{
	安邑古縣時屬虞州將即亮翻間古莧翻趨七喻翻}
邀擊大破之敬德相僅以身免悉俘其衆復歸栢壁諸將咸請與宋金剛戰世民曰金剛懸軍深入精兵猛將咸聚於是|{
	將即亮翻下同}
武周據大原倚金剛為扞蔽軍無蓄積以虜掠為資利在速戰我閉營養銳以挫其鋒分兵汾隰衝其心腹|{
	汾隰隋龍泉西河二郡之地也孫愐曰汾州本漢西河郡兹氏縣地魏於兹氏置西河郡今州城是也左傳曰重耳居蒲即隰川縣故蒲城是也漢為蒲子縣後魏齊周之間為汾州隋為隰州爾雅曰下濕曰隰以州帶泉泊下濕故以隰名}
彼糧盡計窮自當遁走當待此機未宜速戰永安壯王孝基謀逃歸劉武周殺之 李世勣復遣人說竇建德曰曹戴二州戶口完實|{
	隋置曹州於濟陰戴州於成武大業初廢二州併為濟陰郡大業亂復為州復扶又翻說式芮翻}
孟海公竊有其地與鄭人外合内離|{
	王世充國號鄭}
若以大軍臨之指期可取既得海公以臨徐兖|{
	王世充時遣王世辯據徐州徐圓朗據兖州}
河南可不戰而定也建德以為然欲自將徇河南先遣其行臺曹旦等將兵五萬濟河 |{
	考異曰實録在來年正月今從革命記}
世勣引兵三千會之

三年春正月將軍秦武通攻王行本於蒲反行本出戰而敗糧盡援絶欲突圍走無隨之者戊寅開門出降|{
	降戶江翻}
辛巳上幸蒲州斬行本|{
	蒲州治蒲反宋白曰蒲州漢之河東郡蒲坂縣本舜都周為虞虢耿揚芮之地戰國時魏地漢置河東郡後魏初置雍州延和元年改泰州後周改蒲州}
秦王世民輕騎謁上於蒲州|{
	騎奇寄翻下同}
宋金剛圍絳州|{
	絳州治正平}
癸巳上還長安 李世勣謀俟竇建德至河南掩襲其營殺之冀得其父并建德土地以歸唐會建德妻產久之不至曹旦建德之妻兄也在河南多所侵擾諸賊羈屬者皆怨之賊帥魏郡李文相號李商胡|{
	帥所類翻相息亮翻 考異曰革命記作傷胡今從河洛記}
聚五千餘人據孟津中潬|{
	此即河陽中潬城也宋白曰中潬城東魏所築仍置河陽關潬徒旱翻}
母霍氏亦善騎射自稱霍揔管|{
	考異曰革命記商胡母張氏號女將軍今從河洛記}
世勣結商胡為昆弟入拜

商胡之母母泣謂世勣曰竇氏無道如何事之世勣曰母無憂不過一月當殺之相與歸唐耳世勣辭去母謂商胡曰東海公許我共圖此賊事久變生何必待其來不如速决是夜商胡召曹旦偏禆二十三人飲之酒盡殺之|{
	飲於鴆翻}
旦别將高雅賢阮君明尚在河北未濟商胡以巨舟四艘濟河北之兵三百人至中流悉殺之有獸醫游水得免|{
	獸醫以能醫牛馬從軍將即亮翻艘蘇遭翻}
至南岸告曹旦旦嚴警為備商胡既舉事始遣人告李世勣世勣與曹旦連營郭孝恪勸世勣襲旦世勣未决聞旦已有備遂與孝恪帥數十騎來奔|{
	帥讀曰率}
商胡復引精兵二千|{
	復扶又翻}
北襲阮君明破之高雅賢收衆去商胡追之不及而還|{
	還從宣翻又如字}
建德羣臣請誅李蓋建德曰世勣唐臣為我所虜不忘本朝乃忠臣也|{
	朝直遥翻}
其父何罪遂赦之甲午世勣孝恪至長安曹旦遂取濟州|{
	武德之初張青持據濟北濟北郡即濟州是後建德與唐相持於虎牢張青特運糧為唐所獲蓋先以濟州降曹旦也濟子禮翻}
復還洺州|{
	復扶又翻下同又音如字}
二月庚子上幸華陰|{
	華戶化翻}
劉武周遣兵寇潞州䧟長子壺關|{
	二縣皆屬潞州宋白曰潞州春秋潞子國秦漢為上黨郡後周立潞州以其浸汾潞為名}
潞州刺史郭子武不能禦上以將軍河東王行敏助之|{
	河東縣帶蒲州即蒲反也隋開皇十六年析蒲反置縣大業初并蒲反入焉}
行敏與子武不叶或言子武將叛行敏斬子武以徇乙巳武周復遣兵寇潞州行敏擊破之 壬子開州蠻冉肇則䧟通州|{
	舊志開州隋巴東郡之盛山縣盛山漢巴郡之朐䏰縣也義寧元年析巴東之盛山新浦通川之萬世西流置萬州武德元年改開州通州漢宕渠縣地梁置萬州元魏改通州隋為通州郡武德元年復為通州孫愐曰通州本漢宕渠縣内有地萬餘頃因名為萬州後魏以萬州居四達之路改為通州宋為達州}
甲寅遣將軍桑顯和等攻呂崇茂於夏縣|{
	夏戶雅翻}
初

工部尚書獨孤懷恩攻蒲反久不下失亡多上數以勑書誚讓之|{
	數所角翻誚才笑翻}
懷恩由是怨望上嘗戲謂懷恩曰姑之子皆已為天子|{
	謂隋煬帝及上也}
次應至舅之子乎懷恩亦頗以此自負或時扼腕曰我家豈女獨貴乎|{
	周明帝后隋文帝后及上母皆獨孤氏腕烏貫翻}
遂與麾下元君寶謀反會懷恩君寶與唐儉皆没於尉遲敬德|{
	尉紆勿翻}
君寶謂儉曰獨孤尚書近謀大事若能早决豈有此辱哉及秦王世民敗敬德於美良川|{
	敗補邁翻}
懷恩逃歸上復使之將兵攻蒲反|{
	復扶又翻下同將即亮翻下同}
君寶又謂儉曰獨孤尚書遂抜難得還|{
	難乃旦翻}
復在蒲反可謂王者不死儉恐懷恩遂成其謀乃說尉遲敬德|{
	說輸芮翻尉紆勿翻}
請使劉世讓還與唐連和敬德從之遂以懷恩反狀聞時王行本已降|{
	降戶江翻下同}
懷恩入據其城上方濟河幸懷恩營已登舟矣世讓適至上大驚曰吾得免豈非天也乃使召懷恩懷恩未知事露輕舟來至即執以屬吏|{
	屬之欲翻}
分捕黨與甲寅誅懷恩及其黨竇建德攻李商胡殺之建德至洺州勸課農桑境内無盜商旅野宿 突厥處羅可汙迎楊政道立為隋王|{
	楊政道齊王暕遺腹之子厥九勿翻處昌呂翻可後刋入聲汗音寒}
中國士民在北者處羅悉以配之有衆萬人置百官皆依隋制居于定襄|{
	此蓋隋之定襄郡也治大利城}
三月乙丑劉武周遣其將張萬歲寇浩州|{
	將即亮翻}
李仲文擊走之俘斬數千人 改納言為侍中内史令為中書令給事郎為給事中|{
	復舊官名也杜佑曰漢制給事中日上朝謁平尚書奏事以有事殿中故曰給事中東漢省魏復置南北朝因之後周天官之屬有給事中掌治六經給事左右其後别置給事中在六宫之外隋初於吏部置給事郎至煬帝移為門下之職置員四人以省讀奏章至是改為給事中龍朔二年改東臺舍人咸亨元年復舊掌侍從讀署奏抄較正違失分判省事給事中蓋因漢之名行周隋之職}
甲戍以内史侍郎封德彞為中書令 王世充將帥州縣來降者時月相繼|{
	帥所類翻降戶江翻}
世充乃峻其法一人亡叛舉家無少長就戮|{
	少詩照翻長知兩翻}
父子兄弟夫婦許相告而免之又使五家為保有舉家亡者四鄰不覺皆坐誅殺人益多而亡者益甚至於樵采之人出入皆有限數公私愁窘|{
	窘渠隕翻}
人不聊生又以宫城為大獄意所忌者并其家屬收擊宫中諸將出討亦質其家屬於宫中|{
	將即亮翻質音致}
禁止者常不減萬口餒死者日有數十世充又以臺省官為司鄭管原伊殷梁湊嵩谷懷德等十二州營田使|{
	世充以洛州為司州汜水為鄭州管城為管州沁水為原州襄城為伊州獲嘉為殷州睢陽為梁州湊州闕九域志鄭州古跡有湊水當置湊州於此嵩陽為嵩州大谷為谷州河内為懷州武德為德州}
丞郎得為此行者喜若登仙|{
	丞郎尚書左右丞及諸曹郎也史言王世充將敗}
甲申行軍副摠管張綸敗劉武周於浩州|{
	敗補邁翻}
俘斬千餘人 西河公張綸|{
	此張綸即上張綸上書其官此書其爵}
真鄉公李仲文引兵臨石州|{
	石州隋之離石郡}
劉季真懼而詐降乙酉以季真為石州總管賜姓李氏封彭山郡王蠻酋冉肇則寇信州|{
	按新志信州隋之巴東郡武德二年改為夔州史以舊州名書}


|{
	之杜佑曰夔州春秋時為魚國梁置信州唐武德二年避皇外祖獨孤信諱改為夔州治奉節縣酋慈由翻}
趙郡公孝㳟與戰不利李靖將兵八百襲擊斬之|{
	將即亮翻}
俘五千餘人己丑復開通二州孝恭又擊蕭銑東平王闍提斬之|{
	闍視遮翻 考異曰舊書蕭銑傳云孝恭討之拔其開通二州斬其偽東平王蕭闍提按實録云冉肇則䧟我通州又云孝恭復開通二州若二州本屬銑不當云我與復蓋肇則先據開州又䧟通州以地附銑銑使闍提助之耳}
夏四月丙申上祠華山壬寅還長安|{
	華戶化翻還從宣翻}
置益州道行臺以益利會鄜涇遂六總管隸焉|{
	益州隋之蜀郡利州隋之義城郡梁之黎州晉之晉夀蜀之漢壽漢之葭萌也會州隋之凉川縣會寧鎮西魏之會州也鄜州隋之上郡西魏之敷州後魏之北華州中部敷城郡也太和中為東秦州涇州隋之安定郡遂州隋之遂寧郡漢之廣漢縣也是時益州行臺所統起蜀跨隴而東北}
劉武周數攻浩州為李仲文所敗|{
	數所角翻敗補邁翻}
宋金剛軍中食盡丁未金剛北走秦王世民追之 羅士信圍慈澗|{
	隋志河南郡壽安縣有慈澗水經注新安有孝水孝水東十里有水世謂之慈澗}
王世充使太子玄應救之士信刺玄應墜馬|{
	刺七亦翻}
人救之得免 壬子以顯州道行臺楊士林為行臺尚書令|{
	去年正月楊士林降}
甲寅加秦王世民益州道行臺尚書令 秦王世民追及尋相於呂州|{
	新志義寜元年以晉州之霍邑趙城汾西汾州之靈石置霍山郡武德元年曰呂州呂州蓋治霍邑也相息亮翻}
大破之乘勝逐北一晝夜行二百餘里戰數十合至高壁嶺總管劉弘基執轡諫曰大王破賊逐北至此功亦足矣深入不已不愛身乎且士卒饑疲宜留壁於此俟兵糧畢集然後復進未晩也|{
	復扶又翻下同}
世民曰金剛計窮而走衆心離沮功難成而易敗機難得而易失|{
	沮在呂翻易以豉翻}
必乘此勢取之若更淹留使之計立備成不可復攻矣吾竭忠徇國豈顧身乎遂策馬而進將士不敢復言饑|{
	將即亮翻}
追及金剛於雀鼠谷一日八戰皆破之俘斬數萬人夜宿於雀鼠谷西原世民不食二日不解甲三日矣軍中止有一羊世民與將士分而食之|{
	將即亮翻}
丙辰陜州摠管于筠自金剛所逃來|{
	陜失冉翻去年十二月筠為金剛將所擒}
世民引兵趣介休|{
	介休介州治所趣七喻翻又逡須翻}
金剛尚有衆二萬出西門背城布陳|{
	背蒲妹翻陳讀曰陣}
向北七里世民遣摠管李世勣與戰小却為賊所乘世民帥精騎撃之|{
	帥讀曰率騎奇寄翻}
出其陳後金剛大敗斬首三千級金剛輕騎走世民追之數十里至張難堡|{
	張難蓋人姓名築堡自守因以名之}
浩州行軍摠管樊伯通張德政據堡自守世民免胄示之堡中喜譟且泣左右告以王不食獻濁酒脫粟飯尉遲敬德收餘衆守介休|{
	尉紆勿翻}
世民遣任城王道宗宇文士及往諭之|{
	任音壬}
敬德與尋相舉介休及永安降|{
	永安漢中陽縣也後魏更名時屬浩州雀鼠谷在永安介休二縣間}
世民得敬德甚喜以為右一府統軍使將其舊衆八千與諸營相參|{
	將即亮翻下同}
屈突通慮其變驟以為言世民不聽劉武周聞金剛敗大懼棄并州走突厥金剛收其餘衆欲復戰衆莫肯從亦與百餘騎走突厥|{
	秦王之破劉武周宋金剛與破薛仁杲宗羅方略一也復扶又翻走音奏下同厥九勿翻}
世民至晉陽武周所署僕射楊伏念以城降|{
	降戶江翻}
唐儉封府庫以待世民|{
	唐儉與于筠同被擒}
武周所得州縣皆入于唐未幾金剛謀走上谷突厥追獲腰斬之|{
	金剛本起於上谷幾居豈翻}
嵐州總管劉六兒從宋金剛在介休秦王世民擒斬之其兄季真棄石州奔劉武周將馬邑高滿政滿政殺之|{
	去年五月劉六兒降今年三月季真降而實附金剛武周今皆誅死}
武周之南寇也其内史令苑君璋諫曰唐主舉一州之衆直取長安所向無敵此乃天授非人力也晉陽以南道路險隘縣軍深入|{
	縣讀曰懸}
無繼於後若進戰不利何以自還|{
	還從宣翻}
不如北連突厥南結唐朝|{
	厥九勿翻朝直遥翻}
南面稱孤足為長策武周不聽留君璋守朔州及敗泣謂君璋曰不用君言以至於此久之武周謀亡歸馬邑事泄突厥殺之突厥又以君璋為大行臺統其餘衆仍令郁射設督兵助鎮 庚申懷州總管黄君漢擊王世充太子玄應於西濟州大破之|{
	新志武德二年王世充將丁伯德以濟源縣來降置西濟州曰西者以别濟北之濟州濟子禮翻}
熊州行軍總管史萬寶邀之於九曲又破之 辛酉王世充䧟鄧州 上聞并州平大悦壬戌宴羣臣賜繒帛使自入御府盡力取之|{
	唐御府蓋屬内侍省内府局六典内府令掌中宫府藏寶貨給納名數几朝會五品已上賜絹及雜綵金銀器於殿庭者並供之今使各稱力自取繒帛繒慈陵翻}
復唐儉官爵仍以為并州道安撫大使|{
	使疏吏翻}
所籍獨孤懷恩田宅資財悉以賜之|{
	賞其發懷恩反謀也}
世民留李仲文鎮并州劉武周數遣兵入寇|{
	此言武周未死之前數所角翻}
仲文輒擊破之下城堡百餘所|{
	謂馬邑郡界城堡也}
詔仲文檢校并州總管|{
	檢校官未為真}
五月竇建德遣高士興擊李藝於幽州不克退軍籠火城藝襲擊大破之斬首五千級建德大將軍王伏寶勇略冠軍中|{
	冠古玩翻}
諸將疾之|{
	將即亮翻下同}
言其謀反建德殺之伏寶曰大王柰何聽讒言自斬左右手乎 初尉遲敬德將兵助呂崇茂守夏縣上濳遣使赦崇茂罪拜夏州刺史|{
	蓋以夏縣為夏州使疏吏翻夏戶雅翻尉紆勿翻}
使圖敬德事泄敬德殺之敬德去崇茂餘黨復據夏縣拒守|{
	復扶又翻}
秦王世民引軍自晉州還攻夏縣壬午屠之 |{
	考異曰高祖實録帝曰平薛舉之初不殺奴賊致生叛亂若不盡誅必為後患詔勝兵者悉斬疑作實録者歸太宗之過於高祖今不取}
辛卯秦王世民至長安 是月突厥遣阿史那掲多獻馬千匹於王世充|{
	厥九勿翻掲居謁翻}
且求昏世充以宗女妻之|{
	妻七細翻}
并與之互市 六月壬辰詔以和州總管東南道行臺尚書令楚王杜伏威為使持節總管江淮以南諸軍事揚州刺史東南道行臺尚書令淮南道安撫使進封吳王賜姓李氏|{
	使疏吏翻下同}
以輔公祏為行臺左僕射封舒國公|{
	祏音石}
丙午立皇子元景為趙王元昌為魯王元亨為酆王 顯州行臺尚書令楚公楊士林雖受唐官爵而北結王世充南通蕭銑詔廬江王瑗與安撫使李弘敏討之|{
	瑗于眷翻}
兵未行長史田瓚為士林所忌甲寅瓚殺士林降於世充世充以瓚為顯州總管|{
	長知兩翻瓚藏旱翻降戶江翻}
秦王世民之討劉武周也突厥處羅可汗遣其弟步利設帥二千騎助唐|{
	厥九勿翻處昌呂翻可從刋入聲汗音寒帥讀曰率騎奇寄翻}
武周既敗是月處羅至晉陽總管李仲文不能制又留倫特勒使將數百人云助仲文鎮守自石嶺以北皆留兵戍之而去|{
	石嶺關在代州杜佑曰忻州定襄縣漢陽曲縣有石嶺關甚嶮固}
上議擊王世充世充聞之選諸州鎮驍勇皆集洛陽|{
	驍堅堯翻}
置四鎮將軍募人分守四城|{
	謂洛陽四城也}
秋七月壬戌詔秦王世民督諸軍擊世充陜東道行臺屈突通二子在洛陽|{
	時命通判陜東道行臺左僕射從秦王東征屈區勿翻}
上謂通曰今欲使卿東征如卿二子何通曰臣昔為俘囚分當就死陛下釋縛加以恩禮|{
	事見一百八十四卷義寧元年十二月分扶問翻}
當是之時臣心口相誓期以更生餘年為陛下盡節|{
	為于偽翻}
但恐不獲死所耳今得備先驅二兒何足顧乎上歎曰徇義之士一至此乎癸亥突厥遣使濳詣王世充潞州摠管李襲譽邀擊敗之|{
	使疏史翻敗補邁翻}
虜牛羊萬計 驃騎大將軍可朱渾定遠|{
	可朱渾虜三字姓驃匹妙翻騎奇寄翻}
告并州總管李仲文與突厥通謀欲俟洛陽兵交引胡騎直入長安甲戌命皇太子鎮蒲反以備之又遣禮部尚書唐儉安撫并州蹔廢并州總管府|{
	蹔與暫同}
徵仲文入朝|{
	朝直遥翻}
壬午秦王世民至新安|{
	九域志新安在洛州西七十里 考異曰高祖實録丙戌至新安蓋据奏到之日今從河洛記}
王世充遣魏王弘烈鎮襄陽|{
	襄陽襄州}
荆王行本鎮虎牢宋王泰鎮懷州齊王世惲檢校南城楚王世偉守寶城太子玄應守東城漢王玄恕守含嘉城魯王道狥守曜儀城|{
	六典東都皇城在都城之西北隅東城在皇城之東皇城在東城之内皇宫在皇城之北以地望凖之南城蓋在皇城之南端門之外曜儀城蓋在東城之東含嘉城則含嘉倉城寶城即寶城朝堂蓋皇城也惲於粉翻}
世充自將戰兵|{
	將即亮翻下同}
左輔大將軍楊公卿帥左龍驤二十八府騎兵右游擊大將軍郭善才帥内軍二十八府步兵左游擊大將軍跋野綱|{
	跋野虜複姓帥讀曰率驤思將翻騎奇寄翻下同}
帥外軍二十八府步兵摠三萬人以備唐弘烈行本世偉之子泰世充之兄子也 梁師都引突厥稽胡兵入寇|{
	厥九勿翻}
行軍總管段德操擊破之斬首千餘級 羅士信將前軍圍慈澗世充自將兵三萬救之己丑秦王將輕騎前覘世充|{
	覘丑廉翻又丑艶翻}
猝與之遇衆寡不敵道路險阨為世充所圍|{
	考異曰太宗實録云師次穀州王充以兵三萬來拒戰太宗帥輕騎挑之衆寡不敵被圍數重太宗引弓馳射皆應弦而倒獲其大將燕頎賊乃退舊書太宗紀云太宗命左右先歸獨留後殿世充驍將單雄信數百騎夾道來逼交搶競進太宗幾為所敗太宗左右射之無不應弦而倒獲其大將燕頎單雄信傳云大宗圍逼東都雄信出軍拒戰援搶而至幾及太宗徐世勣呵止之曰此秦王也雄信惶懼遂退太宗由是獲免按劉餗小說英公勣與海陵王元吉圍洛陽元吉恃膂力每親行圍王世充召雄信酌以金椀雄信盡飲馳馬而出槍不及海陵者一尺勣惶遽連呼曰阿兄此是勣王雄信乃攬轡而止顧笑曰胡兒不緣你且竟舊書蓋承此致誤耳雄信若知是秦王則取之尤切安肯惶懼而退借如小說所云雄信既受世充之命指取元吉亦安肯以勣故而捨之况元吉之圍東都勣乃從太宗在虎牢今不取}
世民左右馳射獲其左建威將軍燕琪|{
	燕因肩翻 考異曰高祖實錄作燕頃大宗實録作燕傾舊大宗紀作燕頎今從河洛記}
世充乃退世民還營塵埃覆面|{
	覆敷又翻}
軍不復識欲拒之|{
	復扶又翻}
世民免胄自言乃得入旦日帥步騎五萬進軍慈澗|{
	帥讀曰率騎奇寄翻}
世充拔慈澗之戍歸於洛陽世民遣行軍總管史萬寶自宜陽南據龍門|{
	此伊關之龍門也酈道元曰伊水北入伊闕昔大禹疏以通水兩山相對望之若闕故謂之伊闕春秋昭公二十六年趙鞅使女寛守闕塞即此傳毅反都賦曰因龍門以暢化開伊闕以達聦是後武后居東都數遊龍門正此地也}
將軍劉德威自太行東圍河内|{
	行戶剛翻}
上谷公王君廓自洛口斷其餉道|{
	斷丁管翻}
懷州總管黄君漢自河陰攻迴洛城大軍屯于北邙連營以逼之世充洧州長史繁水張公謹與刺史崔樞以州城來降|{
	世充蓋以扶溝鄢陵置洧州隋志繁水縣屬武陽郡唐貞觀十八年併入昌樂縣屬魏州洧音于軌翻降戶江翻下同}
八月丁酉南寧西㸑蠻遣使入貢初隋末蠻酋爨翫反誅諸子没為官奴棄其地|{
	爨翫見一百七十八卷隋文帝開皇十七年十八年使疏吏翻酋慈由翻}
帝即位以翫子弘達為昆州刺史|{
	新志昆州本隋置隋亂廢武德元年開南中復置領晉寧秦臧等縣}
令持其父尸歸葬益州刺史段綸因遣使招諭其部落皆來降 己亥竇建德共州縣令唐綱殺刺史以州來降|{
	新志衛州共城縣武德元年置共州去年竇建德破降李世勣取衛州故共州亦附建德唐綱當是共城縣令也共讀曰恭降戶江翻}
鄧州土豪執王世充所署刺史來降|{
	是年五月王世充䧟鄧州}
癸卯梁師都石堡留守張舉帥千餘人來降|{
	此石堡蓋在夏州東非開元天寶間與吐蕃爭之石堡城也守式又翻帥讀曰率下同}
甲辰黄君漢遣校尉張夜义以舟師襲迴洛城克之|{
	以舟師自懷州度河襲破迥洛}
獲其將達奚善定斷河陽南橋而還降其堡聚二十餘|{
	將即亮翻斷丁管翻還從宣翻聚才喻翻}
世充使太子玄應帥楊公卿等攻迴洛不克 |{
	考異曰革命記作公卿河洛記唐書作公卿今從之}
乃築月城於其西留兵戍之世充陳於青城宫秦王世民亦置陳當之|{
	今世以郊天齋宿大次為青城宫其地當在都城之南此青城宫若在洛城西北按六典洛城西禁苑北拒北邙西至孝水南帶洛水支渠穀洛二水會于其間中有合璧翠微宿羽青城等十一宫陳讀曰陣}
世充隔水謂世民曰隋室傾覆唐帝關中鄭帝河南世充未嘗西侵王忽舉兵東來何也世民使宇文士及應之曰四海皆仰皇風唯公獨阻聲教為此而來|{
	為子偽翻}
世充曰相與息兵講好不亦善乎又應之曰奉詔取東都不令講好也|{
	好呼到翻}
至暮各引兵還 上遣使與竇建德連和建德遣同安長公主隨使者俱還|{
	同安長公主上同母妺黎陽之破没於竇建德使疏吏翻長知兩翻}
乙卯劉德威襲懷州入其外郭下其堡聚 九月庚午梁師都將劉旻以華池來降以為林州總管|{
	慶州華池縣西魏之蔚州後周州廢隋仁壽初置華池縣今置林川將即亮翻降戶江翻}
癸酉王世充顯州總管田瓚以所部二十五州來降|{
	是年六月田瓚降世充瓚藏旱翻}
自是襄陽聲問與世充絶|{
	世充使王弘烈鎮襄陽自襄陽至洛路出南陽鄧州既屬唐南陽之路不可由矣則自顯州出蔡汝以至洛顯州今又降唐故襄陽聲問絶}
史萬寶進軍甘泉宫|{
	漢甘泉宫在京北醴泉縣史萬寶自新安進軍逼洛陽不應至漢之甘泉宫隋志河南夀安縣後魏之甘棠縣有顯仁宫或者以顯仁宫為甘棠宫也泉恐當作棠}
丁丑秦王世民遣右武衛將軍王君廓攻轘轅拔之|{
	新志洛州氏縣東南有轘轅故關轘音環}
王世充遣其將魏隱等撃君廓君廓偽遁設伏大破之遂東徇地至管城而還|{
	隋志榮陽郡管城縣舊曰中牟開皇十六年折置管城縣十八年省内牟入焉隋改中牟曰内牟時為管州治所將即亮翻下同還從宣翻}
先是王世充將郭士衡許羅漢掠唐境|{
	先悉薦翻}
君廓以策擊却之詔勞之曰|{
	勞力到翻}
卿以十三人破賊一萬自古以少制衆未之有也|{
	少詩沼翻}
世充尉州刺史時德叡帥所部杞夏陳隨許潁尉七州來降|{
	王世充蓋置杞州於雍丘夏州於陽夏陳州於宛丘隨州無所考意洧州之誤也許州於長社潁州於汝陰尉州於尉氏帥讀曰率降戶江翻下同}
秦王世民以便宜命州縣官並依世充所署無所變易改尉州為南汴州於是河南郡縣相繼來降劉武周降將尋相等多叛去諸將疑尉遲敬德囚之軍中|{
	相息亮翻尉紆勿翻}
行臺左僕射屈突通尚書殷開山言於世民曰敬德驍勇絶倫|{
	屈九勿翻驍堅堯翻}
今既囚之心必怨望留之恐為後患不如遂殺之世民曰不然敬德若叛豈在尋相之後邪|{
	邪音耶}
遽命釋之引入卧内賜之金曰丈夫意氣相期勿以小嫌介意吾終不信讒言以害忠良公宜體之必欲去者以此金相資表一時共事之情也辛巳世民以五百騎行戰地|{
	騎奇寄翻下同行視地形可置陳處行下孟翻}
登魏宣武陵|{
	魏宣武陵曰景陵在北邙山魏世宗謚宣武帝}
王世充帥步騎萬餘猝至圍之單雄信引槊直趨世民|{
	帥讀曰率下同槊所角翻趨七喻翻}
敬德躍馬大呼横刺雄信墜馬|{
	呼火故翻刺七亦翻}
世充兵稍却敬德翼世民出圍世民敬德更帥騎兵還戰出入世充陳往反無所礙屈突通引大兵繼至世充兵大敗僅以身免擒其冠軍大將軍陳智略|{
	陳讀曰陣下同冠古玩翻 考異曰實録丙戌太宗與世充相遇於宣武陵擊大破之斬數千級獲陳智略舊書敬德傳太宗既釋之是日從獵於榆窠世充以步騎數萬來戰單雄信直趨大宗敬德刺雄信墜馬翼太宗出圍更率騎兵交戰擒陳智略据擒智略則宣武榆窠之戰共是一事也實録据奏到日河洛記在二十一日今從之}
斬首千餘級獲排矟兵六千|{
	排矟言執排執矟者也矟與槊同}
世民謂敬德曰公何相報之速也賜敬德金銀一篋|{
	篋苦協翻}
自是寵遇日隆敬德善避矟每單騎入敵陳中敵叢矟刺之終莫能傷又能奪敵矟返刺之|{
	刺七亦翻}
齊王元吉以善馬矟自負聞敬德之能請各去刃相與校勝負敬德曰敬德謹當去之王勿去也既而元吉刺之終不能中|{
	去羌呂翻中竹仲翻}
秦王世民問敬德曰奪矟與避矟孰難敬德曰奪矟難乃命敬德奪元吉矟元吉操矟躍馬志在刺之|{
	操七刀翻}
敬德須臾三奪其矟元吉雖面相歎異内甚恥之 叛胡䧟嵐州|{
	嵐盧含翻}
初王世充以邴元真為滑州行臺僕射濮州刺史杜才幹|{
	隋志東平郡鄄城縣舊置濮陽郡開皇十六年置濮州大業初廢州以鄄城縣屬東平蓋李密復置州也濮博木翻}
李密故將也|{
	將即亮翻}
恨元真叛密|{
	叛密事見一百八十六卷元年九月}
詐以其衆降之|{
	降戶江翻下同}
元真恃其官勢自往招慰才幹出迎延入就坐|{
	坐徂卧翻}
執而數之曰|{
	數所具翻又所主翻}
汝本庸才魏公置汝元僚|{
	謂李密以為長史}
不建毫髪之功乃構滔天之禍今來送死是汝之分|{
	分扶問翻}
遂斬之遣人齎其首至黎陽祭密墓壬午以濮州來降 突厥莫賀咄設寇凉州總管楊恭仁擊之為所敗|{
	厥九勿翻咄當没翻敗補邁翻}
掠男女數千人而去丙戌以田瓚為顯州總管賜爵蔡國公|{
	瓚藏早翻}
冬十月甲午王世充大將軍張鎮周來降 甲辰行軍摠管羅士信襲王世充硤石堡拔之|{
	水經注穀水自新安縣東流逕千秋亭又東逕雍谷溪迴岫縈紆石路阻峽故亦有峽石之稱考異曰河洛記作峽山堡今從實録}
士信又圍千金堡|{
	此於古千金碣築堡也水經注穀水逕周乾祭門北東至千金堨河南境簿曰河南縣城東十五里有千金堨洛陽記曰千金堨舊堨穀水魏時更修此堨謂之千金堨}
堡中人罵之士信夜遣百餘人抱嬰兒數十至堡下使兒啼呼詐云從東都來歸羅總管既而相謂曰此千金堡也吾屬誤矣即去堡中以為士信已去來者洛陽亡人出兵追之士信伏兵於道伺其門開突入屠之|{
	伺相吏翻}
竇建德之圍幽州也|{
	是年五月建德兵攻幽州}
李藝告急于高開道開道帥二千騎救之建德兵引去開道因藝遣使來降|{
	帥讀曰率騎奇寄翻使疏史翻降戶江翻}
戊申以開道為蔚州總管|{
	蔚州隋鴈門郡之靈丘上谷郡之飛狐縣地蔚紆勿翻}
賜姓李氏封北平郡王開道有矢鏃在頰召醫出之醫曰鏃深不可出開道怒斬之别召一醫曰出之恐痛又斬之更召一醫醫曰可出乃鑿骨置楔其間|{
	楔先結翻}
骨裂寸餘竟出其鏃開道奏妓進膳不輟|{
	妓渠綺翻}
竇建德帥衆二十萬復攻幽州建德兵已攀堞|{
	復扶又翻堞達協翻}
薛萬均萬徹帥敢死士百人從地道出其背掩擊之建德兵潰走斬首千餘級李藝兵乘勝薄其營建德陳於營中|{
	陳讀曰陣}
填塹而出奮擊大破之|{
	塹七艶翻}
建德逐北至其城下攻之不克而還|{
	還從宣翻又音如字}
李密之敗也|{
	見一百八十六卷元年九月}
楊慶歸洛陽復姓楊氏|{
	楊慶歸密改姓事見一百八十四卷義寧元年十一月}
及王世充稱帝|{
	見上卷本年四月}
慶復姓郭氏世充以為管州摠管妻以兄女|{
	妻七細翻}
秦王世民逼洛陽慶濳遣人請降|{
	降戶江翻}
世民遣總管李世勣將兵往據其城|{
	將即亮翻下四將其將同}
慶欲與其妻偕來妻曰主上使妾侍巾櫛者|{
	櫛阻瑟翻梳也}
欲結君之心也今君既辜付託|{
	辜負也}
狥利求全妾將如君何若至長安則君家一婢耳君何用為願送至洛陽君之惠也慶不許慶出妻謂侍者曰若唐遂勝鄭則吾家必滅鄭若勝唐則吾夫必死人生至此何用生為遂自殺庚戌慶來降復姓楊氏拜上柱國郇國公|{
	郇音荀}
時世充太子玄應鎮虎牢軍于榮汴之間|{
	榮當作滎言軍于滎澤汴水之間}
聞之引兵趣管城|{
	趣七喻翻又逡須翻}
李世勣擊却之使郭孝恪為書說榮州刺史魏陸|{
	王世充蓋以滎陽縣置滎州作榮亦誤也}
陸密請降玄應遣大將軍張志就陸徵兵丙辰陸擒志等四將舉州來降陽城令王雄帥諸堡來降秦王世民使李世勣引兵應之以雄為嵩州刺史|{
	新志陽城縣屬洛州又云武德四年王世充偽令王雄來降以陽城嵩陽陽翟置嵩州與此所書稍差二三月}
嵩南之路始通|{
	嵩南謂嵩山以南}
魏陸使張志詐為玄應書停其東道之兵令其將張慈寶且還汴州又密告汴州刺史王要漢使圖慈寶要漢斬慈寶以降玄應聞諸州皆叛大懼奔還洛陽詔以要漢為汴州摠管賜爵郳國公|{
	郳五稽翻}
王弘烈據襄陽上令金州摠管府司馬涇陽李大亮安撫樊鄧以圖之十一月庚申大亮攻樊城鎮拔之|{
	隋志西城郡梁置梁州尋改曰南梁川西魏改東梁州尋改金州置摠管府府置長史司馬舊志襄州鄧城縣漢鄧縣屬南陽郡古樊城也宋改安養縣此時樊城鎮當在安養縣界}
斬其將國大安下其城柵十四|{
	將即亮翻}
蕭銑性褊狹多猜忌諸將恃功恣横好專誅殺|{
	横戶孟翻好呼到翻}
銑患之乃宣言罷兵營農實欲奪諸將之權大司馬董景珍弟為將軍怨望謀作亂事泄伏誅景珍時鎮長沙|{
	長沙潭州治所}
銑下詔赦之召還江陵景珍懼甲子以長沙來降|{
	降戶江翻}
詔峽州刺史許紹出兵應之 雲州摠管郭子和|{
	隋志定襄郡開皇五年置雲州摠管府治大利城}
先與突厥梁師都相連結既而襲師都寧朔城克之|{
	新志寜朔縣屬夏州後周置先悉薦翻厥九勿翻}
又詗得突厥舋隙|{
	詗休正翻舋許覲翻}
遣使以聞為突厥候騎所獲|{
	使疏吏翻騎奇寄翻}
處羅可汗大怒囚其弟子升子和自以孤危請帥其民南徙|{
	處昌呂翻可從刋入聲汙音寒帥讀曰率 考異曰子和傳云四年拔戶口南徙按處羅可汗以今年卒故置此}
詔以延州故城處之|{
	處昌呂翻}
張舉劉旻之降也|{
	是年八月張舉降九月劉旻降}
梁師都大懼遣其尚書陸季覽說突厥處羅可汗曰比者中原喪亂|{
	說輸芮翻比毗至翻喪息浪翻}
分為數國勢均力弱故皆北面歸附突厥今定楊可汗既亡|{
	是年四月劉武周敗亡}
天下將悉為唐有師都不辭灰滅亦恐次及可汗不若及其未定南取中原如魏道武所為|{
	事見晉孝武帝紀}
師都請為鄉導|{
	鄉讀曰嚮}
處羅從之謀使莫賀咄設入自原州|{
	平凉郡置原州處昌呂翻咄當没翻}
泥步設與師都入自延州突利可汗與奚霫契丹靺鞨入自幽州|{
	可從刋入聲汗音寒奚與契丹本皆東胡種保烏丸山者其後為奚保鮮卑山者其後為契丹霫與突厥同俗保冷陘山南契丹東靺鞨西拔野古靺鞨居肅慎地亦曰挹婁元魏時曰勿吉霫而立翻契欺訖翻又音喫靺莫撥翻鞨戶割翻 考異曰舊突厥傳大業中突利年數歲始畢遣領東牙之兵號泥步設頡利嗣位以為突利可汗按梁師都傳此際有泥步設又有突利可汗然則突利處羅時已為小可汗非頡利嗣位後也高祖實録云處羅欲分兵大掠中國於懷戎鴈門靈武凉州四道俱入今從舊書梁師都傳}
會竇建德之師自滏口西入會于晉絳|{
	滏口滏水之口在磁州滏陽縣界晉州隋之臨汾郡絳州隋之絳郡滏音釡}
莫賀咄者處羅之弟咄苾也突利者始畢之子什鉢苾也|{
	咄當没翻苾毗必翻}
處羅又欲取并州以居楊政道|{
	楊政道時居定襄}
其羣臣多諫處羅曰我父失國賴隋得立|{
	事見一百七十八卷隋開皇十九年}
此恩不可忘將出師而卒|{
	卒子恤翻}
義成公主以其子奥射設醜弱廢之更立莫賀咄設|{
	更工衡翻}
號頡利可汗乙酉頡利遣使告處羅之喪|{
	使疏吏翻}
上禮之如始畢之喪|{
	去年四月始畢卒}
戊子安撫大使李大亮取王世充沮華二州|{
	襄州南漳縣後周置沮州南漳漢之臨沮縣也隋廢沮州蓋王世充復置漢南縣宋置華山郡西魏廢郡王世充蓋取宋郡名而置華州也漢南縣唐貞觀八年省併入宜城沮子魚翻華戶化翻}
是月竇建德濟河擊孟海公 |{
	考異曰實録在十二月丙午盖於時唐始聞之遣劉世讓攻洛州之日也今從革命記}
初王世充侵建德黎陽建德襲破殷州以報之|{
	殷州治獲嘉此皆去年冬事}
自是二國交惡信使不通及唐兵逼洛陽世充遣使求救於建德|{
	使疏吏翻 考異曰隋季革命記曰世充亦自遣使求救於建德云夏王或率領軍師來相救援王取東都河洛之地北收并汾南盡楊越充乃取京師蒲絳以西通蜀荆襄之境並據山河之險長為弟兄之國按世充止有河洛之地豈肯遽以賂建德借有是言建德亦何由肯信今從河洛記}
建德中書侍郎劉彬說建德曰天下大亂唐得關西鄭得河南夏得河北共成鼎足之勢今唐舉兵臨鄭自秋涉冬唐兵日增鄭地日蹙唐彊鄭弱勢必不攴鄭亡則夏不能獨立矣|{
	說輸芮翻夏戶雅翻}
不如解仇除忿發兵救之夏擊其外鄭攻其内破唐必矣唐師既退徐觀其變若鄭可取則取之并二國之兵乘唐師之老天下可取也建德從之遣使詣世充許以赴援又遣其禮部侍郎李大師等詣唐請罷洛陽之兵秦王世民留之不荅十二月辛卯王世充許亳等十一州皆請降|{
	許州隋之頴州郡亳州隋之譙郡亳蒲博翻降戶江翻}
壬辰燕郡王李藝又擊竇建德軍於籠火城破之|{
	燕因肩翻}
辛丑王世充隨州摠管徐毅舉州降|{
	隨州隋之漢東郡}
癸卯峽州刺史許紹攻蕭銑荆門鎮拔之|{
	荆門在荆州長林縣}
紹所部與梁鄭鄰接|{
	峽州北境接鄭之襄州東境接梁之荆門}
二境得紹士卒皆殺之紹得二境士卒皆資給遣之敵人愧感不復侵掠境内以安|{
	復扶又翻}
蕭銑遣其齊王張繡攻長沙董景珍謂繡曰前年醢彭越往年殺韓信卿不見之乎何為相攻|{
	引漢高祖殺功臣事以恐動繡}
繡不應進兵圍之景珍欲潰圍走為麾下所殺銑以繡為尚書令繡恃功驕横|{
	横戶孟翻}
銑又殺之由是功臣諸將皆有離心兵勢益弱|{
	史言蕭銑將亡}
王世充遣其兄子代王琬長孫安世詣竇建德報聘且乞師|{
	長知兩翻}
突厥倫特勒在并州大為民患|{
	是年六月突厥留倫特勒於并州厥九勿翻}
并州摠管劉世讓設策擒之上聞之甚喜張道源從竇建德在河南|{
	去年九月道源為建德所執}
密遣人詣長安請出兵攻洺州以震山東|{
	竇建德都洺州洺音名}
丙午詔世讓為行軍摠管使將兵出土門趣洺州|{
	新志恒州鹿泉縣有故井陘關一名土門關鹿泉漢之石邑也}
己酉瓜州刺史賀拔行威執驃騎將軍達奚暠|{
	瓜州隋之燉煌郡騎奇寄翻驃匹妙翻暠古老翻}
舉兵反 是歲李子通度江攻沈法興取京口|{
	京口時屬揚州延陵縣}
法興遣其僕射蔣元超拒之戰於庱亭|{
	庱亭在毗陵西北庱丑拯翻又恥陵翻}
元超敗死法興棄毗陵奔吳郡|{
	毗陵至吳郡百八十里}
於是丹楊毗陵等郡皆降於子通|{
	降戶江翻}
子通以法興府掾李百藥為内史侍郎國子祭酒|{
	掾于絹翻}
杜伏威遣行臺左僕射輔公祏|{
	祏音石}
將卒數千攻子通|{
	將即亮翻}
以將軍闞稜王雄誕為之副公祏度江攻丹楊克之進屯溧水|{
	隋志丹陽郡治江寜溧水以縣屬焉本溧陽縣開皇十八年更名自丹陽至溧水二百四十里溧音栗}
子通帥衆數萬拒之|{
	帥讀曰率}
公祏簡精甲千人執長刀為前鋒|{
	簡選也分别也}
又使千人踵其後曰有退者即斬之自帥餘衆復居其後|{
	帥讀曰率復扶又翻下同}
子通為方陳而前|{
	陳讀曰陣}
公祏前鋒千人殊死戰公祏復張左右翼以擊之子通敗走公祏逐之反為所敗|{
	敗補邁翻}
還閉壁不出王雄誕曰子通無壁壘又狃於初勝|{
	狃女九翻}
乘其無備擊之可破也公祏不從雄誕以其私屬數百人夜出擊之|{
	私屬親兵不在大軍名籍者}
因風縱火子通大敗降其卒數千人|{
	降戶江翻}
子通食盡棄江都保京口江西之地盡入於伏威|{
	盧和等州皆江西也}
伏威徙居丹陽子通復東走太湖|{
	太湖在蘇州吳縣東南五十里}
收合亡散得二萬人襲沈法興於吳郡大破之法興帥左右數百人棄城走吳郡賊帥聞人遂安遣其將葉孝辯迎之|{
	聞人複姓今吳中亦以為著姓賊帥所類翻將即亮翻}
法興中塗而悔欲殺孝辯更向會稽|{
	會稽越州會古外翻}
孝辯覺之法興窘廹赴江溺死子通軍勢復振徙都餘杭|{
	餘杭杭州}
盡收法興之地北自太湖南至嶺|{
	嶺五嶺也}
東包會稽西距宣城|{
	按子通之地西距宣城耳南境安能至嶺哉史大而言之耳會古外翻}
皆有之 廣新二州賊帥高法澄沈寶徹殺隋官據州附於林士弘|{
	隋志南海郡廣州信安郡新興縣梁置新州宋白曰新州秦始皇所取陸梁地漢為合浦臨元縣晉置新寧郡梁置新州}
漢陽太守馮盎擊破之|{
	馮盎自大業之亂歸嶺南未受朝命故書隋官守式又翻}
既而寶徹兄子智臣復聚兵於新州盎引兵擊之賊始合盎免胄大呼曰|{
	呼火故翻}
爾識我乎賊多棄仗肉袒而拜|{
	馮盎自其祖母洗夫人以來威令行於嶺南故然}
遂潰擒寶徹智臣等嶺外遂定 竇建德行臺尚書令恒山胡大恩請降|{
	恒山恒州恒戶登翻降戶江翻}


四年春正月癸酉以大恩為代州總管|{
	代州隋之鴈門郡}
封定襄郡王賜姓李氏代州石嶺之北自劉武周之亂寇盜充斥大恩徙鎮鴈門|{
	鴈門漢廣武縣隋更名隋唐代州皆治鴈門漢鴈門郡治陰館李大恩豈徙鎮漢鴈門邪宋白曰句注在代州西北三十五里鴈門界西陘山也始皇十三年移樓煩於善無縣今句注山北下館城是也故續漢書云鴈門郡理陰館建安立新興郡陘北悉棄之其地荒廢魏文帝移鴈門郡南度句注置廣武城即今州西廣武故城是也後魏明帝又移置廣武東古上館城内即今州城是也或曰李大恩自恒山請降授代州□管故自恒山徙鎮鴈門}
討擊悉平之 稽胡酋帥劉仚成部落數萬為邊寇|{
	酋慈由翻帥所類翻仚許延翻}
辛巳詔太子建成統諸軍討之 王世充梁州摠管程嘉會以所部來降|{
	後魏置梁州於浚儀因古大梁城以名州也此時以浚儀為汴州而隋之梁郡治宋城縣宋城古睢陽也漢梁國都之後魏以來以睢陽為梁郡王世充當於此置梁州}
杜伏威遣其將陳正通徐紹宗帥精兵二千來會秦王世民擊王世充|{
	將即亮翻帥讀曰率 考異曰舊書伏威傳太宗之圍王世充遣使招之伏威請降高祖遣使就拜東南道行臺尚書令江淮以南安撫大使上柱國封吳王賜姓李氏按伏威封吳王在太宗討王世充前今從高祖太宗實録}
甲申攻梁克之|{
	梁縣屬伊州杜佑曰汝州梁縣漢舊縣戰國時謂之南梁以别大梁少梁也}
丙戌黔州刺史田世康|{
	黔州隋之黔安郡古黔中也黔音琴}
攻蕭銑五州四鎮皆克之 秦王世民選精銳千餘騎皆皁衣玄甲分為左右隊使秦叔寶程知節尉遲敬德翟長孫分將之|{
	騎奇寄翻下同尉紆勿翻翟萇伯翻長知兩翻將即亮翻下同}
每戰世民親被玄甲帥之為前鋒|{
	被皮義翻帥讀曰率下同}
乘機進擊所向無不摧破敵人畏之行臺僕射屈突通贊皇公竇軌|{
	屈九勿翻軌封贊皇縣公贊皇縣屬趙州隋開皇十六年置劉昫曰取贊皇山為名}
引兵按行營屯|{
	行下孟翻}
猝與王世充遇戰不利秦王世民帥玄甲救之世充大敗獲其騎將葛彦璋 |{
	考異曰太宗實録云初羅士信取千金堡太宗令屈突通守之世充自來攻堡通懼舉烽請救太宗度通力堪自守且緩救以驕世充通舉三烽以告急太宗方出援之左右未獲從以兩騎而進遇賊騎將葛彦璋射之應弦而墜擒之於陳後軍亦繼至通軍復振表裏奮擊世充大敗俘斬六千餘人幾獲世充今從河洛記}
俘斬六千餘人世充遁歸 李靖說趙郡王孝恭以取蕭銑十策孝恭上之|{
	說輸芮翻上時掌翻考異曰高祖實録孝恭獻平銑之策帝嘉納之太宗實録李靖傳靖說趙郡王孝恭陳伐蕭銑之計獻以十策高祖以孝恭未更戎旅三軍之任一以委靖授靖行軍摠管兼攝孝恭長史事孝恭傳時李靖亦奉使江南以策干孝恭孝恭善之委以軍事蓋靖畫策使孝恭上之耳}
二月辛卯改信州為夔州 以孝恭為摠管使大造舟艦習水戰|{
	艦戶黯翻}
以孝恭未更軍旅|{
	更工衡翻}
以靖為行軍摠管兼孝恭長史委以軍事靖說孝恭悉召巴蜀酋長子弟|{
	長知兩翻酋才由翻}
量才授任置之左右外示引擢實以為質|{
	量音良質音致}
王世充太子玄應將兵數千人自虎牢運糧入洛陽秦王世民遣將軍李君羨邀擊大破之玄應僅以身免世民使宇文士及奏請進圍東都上謂士及曰歸語爾王|{
	語牛倨翻}
今取洛陽止於息兵克城之日乘輿法物圖籍器械|{
	乘䋲證翻}
非私家所須者委汝收之其餘子女玉帛並以分賜將士|{
	將即亮翻}
辛丑世民移軍青城宫壁壘未立王世充帥衆二萬自方諸門出憑故馬坊垣塹臨穀水以拒唐兵|{
	東都城西連禁苑方諸門蓋自都城出禁苑之門也青城宫在禁苑中穀洛二水會于禁苑之中帥讀曰率下同塹七艶翻}
諸將皆懼世民以精騎陳於北邙登魏宣武陵以望之謂左右曰賊勢窘矣悉衆而出徼幸一戰|{
	騎奇寄翻陳讀曰陣下同徼古堯翻}
今日破之後不敢復出矣|{
	復扶又翻下同}
命屈突通帥步卒五千度水擊之|{
	屈居勿翻}
戒通曰兵交則縱煙煙作世民引騎南下身先士卒|{
	先悉薦翻}
與通合勢力戰世民欲知世充陳厚薄與精騎數十衝之直出其背衆皆披靡|{
	披普彼翻}
殺傷甚衆既而限以長堤與諸騎相失將軍丘行恭獨從世民世充數騎追及之世民馬中流矢而斃行恭回騎射追者發無不中|{
	中竹仲翻射而亦翻}
追者不敢前乃下馬以授世民行恭於馬前步執長刀距躍大呼|{
	距躍超距而跳躍也杜預曰距躍超越也呼火故翻}
斬數人突陳而出得入大軍世充亦帥衆殊死戰散而復合者數四自辰至午世充兵始退世民縱兵乘之直抵城下俘斬七千人遂圍之驃騎將軍段志玄與世充兵力戰深入馬倒為世充兵所擒兩騎夾持其髻|{
	驃匹妙翻騎奇寄翻髻古詣翻}
將渡洛水志玄踴身而奮二人俱墜馬志玄馳歸追者數百騎不敢逼初驃騎將軍王懷文為唐軍斥為世充所獲世充欲慰悦之引置左右壬寅世充出右門|{
	東都城南面三門中曰端門左曰左掖門右曰右掖門洛水逕其前有天津永濟中橋三橋}
臨洛水為陳懷文忽引槊刺世充世充衷甲槊折不能入|{
	刺七亦翻折而設翻}
左右猝出不意皆愕眙不知所為懷文走趣唐軍|{
	眙丑吏翻趣七喻翻又逡須翻}
至寫口|{
	洛城中水於此寫放以流其惡因名之為寫口}
追獲殺之世充歸解去衷甲|{
	去羌呂翻}
袒示羣臣曰懷文以槊刺我卒不能傷|{
	卒子恤翻}
豈非天所命乎先是御史大夫鄭頲不樂仕世充多稱疾不預事|{
	鄭頲李密之臣為世充所獲疾其多詐故不樂仕焉先悉薦翻頲他鼎翻樂音洛}
至是謂世充曰臣聞佛有金剛不壞身陛下真是也臣實多幸得生佛世願棄官削髪為沙門服勤精進以資陛下之神武|{
	詭辭以求去}
世充曰國之大臣聲望素重一旦入道將駭物聽俟兵革休息當從公志頲固請不許退謂其妻曰吾束髪從官志慕名節|{
	束髪謂幼小總角時也}
不幸遭遇亂世流離至此側身猜忌之朝累足危亡之地|{
	朝直遥翻累力委翻}
智力淺薄無以自全人生會有死早晩何殊姑從吾所好|{
	好呼到翻}
死亦無憾遂削髪被僧服|{
	被皮義翻}
世充聞之大怒曰爾以我為必敗欲苟免邪|{
	邪音耶}
不誅之何以制衆遂斬頲於市頲言笑自若觀者壯之詔贈王懷文上柱國朔州刺史|{
	朔州隋之馬邑郡}
并州安撫使唐儉密奏真鄉公李仲文與妖僧志覺有謀反語|{
	真鄉縣公也西魏置真鄉縣時屬綏州使疏吏翻妖於驕翻}
又娶陶氏之女以應桃李之謡諂事可汗甚得其意可汗許立為南面可汗|{
	可從刋入聲汗音寒}
及在并州贓賄狼籍上命裴寂陳叔達蕭瑀雜鞫之|{
	瑀音禹}
乙巳仲文伏誅庚戌王泰棄河陽走|{
	去年七月世充使泰守河陽}
其將趙夐等以城來降|{
	將即亮翻夐休正翻降戶江翻}
别將單雄信裴孝達與總管王君廓相持於洛口|{
	單慈淺翻}
秦王世民帥步騎五千援之至轘轅雄信等遁去君廓追敗之|{
	轘音環敗補邁翻}
壬子延州總管段德操擊劉仚成破之|{
	仚許延翻}
斬首千餘級 乙卯王世充懷州刺史陸善宗以城降 秦王世民圍洛陽宫城城中守禦甚嚴大礮飛石重五十斤擲二百步|{
	礮與砲同匹貌翻}
八弓弩箭如車輻鏃如巨斧射五百步|{
	八弓弩八弓共一絭也如古連弩今之划車弩亦其類也輻音幅}
世民四面攻之晝夜不息旬餘不克城中欲飜城者凡十三輩皆不果發而死唐將士皆疲弊思歸總管劉弘基等請班師世民曰今大舉而來當一勞永逸東方諸州已望風欵服唯洛陽孤城勢不能久功在垂成柰何棄之而去乃下令軍中曰洛陽未破師必不還|{
	還從宣翻}
敢言班師者斬衆乃不敢復言|{
	復扶又翻}
上聞之亦密勑世民使還世民表稱洛陽必可克又遣參謀軍事封德彞入朝面論形勢|{
	參謀之官蓋始於此朝直遥翻}
德彞言於上曰世充得地雖多率皆覊屬|{
	言覊靡屬之而已}
號令所行唯洛陽一城而已智盡力窮克在朝夕今若旋師賊勢復振|{
	復扶又翻又音如字}
更相連結後必難圖上乃從之世民遺世充書|{
	遺于季翻}
諭以禍福世充不報 戊午王世充鄭州司兵沈悦遣使詣左武大將軍李世勣請降|{
	李世勣時屯管城使疏吏翻降戶江翻}
左衛將軍王君廓夜引兵襲虎牢|{
	王君廓時屯洛口}
悦為内應遂拔之獲其荆王行本及長史戴胄|{
	長知兩翻}
悦君理之孫也|{
	沈君理仕陳為僕射}
竇建德克周橋虜孟海公

資治通鑑卷一百八十八
