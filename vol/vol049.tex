資治通鑑卷四十九   宋 司馬光 撰

胡三省 音註

漢紀四十一|{
	起柔兆敦䍧盡旃蒙單閼凡十年}


孝殤皇帝|{
	諱隆和帝少子也諡法短折不成曰殤㐲侯古今注曰隆之字曰盛}


延平元年春正月辛卯以太尉張禹為太傅司徒徐防為太尉參録尚書事太后以帝在襁褓|{
	襁居兩翻褓音保}
欲令重臣居禁内乃詔禹舍宫中五日一歸府每朝見特贊與三公絶席|{
	特贊者每朝見贊拜者先獨贊禹名既乃贊大尉名以下禹不與三公同贊也絶席者朝位獨在百僚上不與三公聯席也朝直遥翻見賢遍翻}
封皇兄勝為平原王癸卯以光禄勲梁鮪為司徒|{
	鮪于軌翻}
三月甲申葬孝

和皇帝于慎陵|{
	賢曰慎陵在雒陽東南三十里}
廟曰穆宗 丙戌清河王慶濟北王夀河間王開常山王章始就國|{
	濟子禮翻}
太后特加慶以殊禮|{
	殊異也其禮異於諸王也}
慶子祜年十三太后以帝幼弱遠慮不虞留祜與嫡母耿姬居清河邸|{
	為帝崩立祜張本}
耿姬况之曾孫也|{
	耿况以上谷從光武}
祜母犍為左姬也|{
	犍居言翻}
夏四月鮮卑寇漁陽漁陽太守張顯率數百人出塞追之兵馬掾嚴授諫曰前道險阻賊執難量|{
	掾俞絹翻量音良}
宜且結營先令輕騎偵視之顯意甚鋭怒欲斬之遂進兵遇虜伏發士卒悉走唯授力戰身被十創手殺數人而死|{
	緣邉郡曹有兵馬掾掌兵馬偵丑鄭翻創初良翻}
主簿衛福功曹徐咸皆自投赴顯俱沒於陳|{
	陳讀曰陣}
丙寅以虎賁中郎將鄧隲為車騎將軍儀同三司|{
	三司三公也晉職官志曰儀同三司之名始此隲職日翻}
隲弟黄門侍郎悝為虎賁中郎將宏閶皆侍中|{
	悝苦回翻閶齒良翻}
司空陳寵薨 五月辛卯赦天下 壬辰河東垣山崩|{
	賢曰垣縣今絳州縣也}
六月丁未以太常尹勤為司空 郡國三十七雨水 己未太后詔減大官導官尚方内署諸服御珍膳靡麗難成之物|{
	賢曰大官令周官也秩千石典天子御膳導官掌擇御米導擇也尚方掌作御刀劒諸器物内署掌内府衣物令秩皆六百石}
自非供陵廟稻米不得導擇朝夕一肉飯而已舊太官湯官經用歲且二萬萬自是裁數千萬|{
	百官志湯官丞主酒屬太官令}
及郡國所貢皆減其過半悉斥賣上林鷹犬|{
	東都亦冇上林苑在雒陽西斥開也棄也}
離宫别館儲峙米糒薪炭悉令省之|{
	峙丈里翻糒音備}
丁卯詔免遣掖庭宮人及宗室沒入者皆為庶民 秋七月庚寅敕司隸校尉部刺史曰|{
	司隸校尉及諸州部刺史也}
間者郡國或有水災妨害秋稼朝廷惟咎憂惶悼懼|{
	惟思也咎過也}
而郡國欲獲豐穰虚飾之譽遂覆蔽災害|{
	覆敷又翻}
多張墾田不揣流亡|{
	揣初委翻}
競增戶口掩匿盗賊令姦惡無懲|{
	隱蔽盗賊不以上聞弗加誅計使姦惡無所懲艾}
署用非次選舉乖宜貪苛慘毒延及平民|{
	賢曰平民謂善人也}
刺史垂頭塞耳阿私下比|{
	塞悉則翻比毗至翻}
不畏于天不愧于人|{
	詩小雅何人斯之辭}
假貸之恩不可數恃|{
	數所角翻}
自今以後將糾其罪二千石長吏其各實覈所傷害為除田租芻稾|{
	長知兩翻為于偽翻}
八月辛卯帝崩|{
	年二歲}
癸丑殯于崇德前殿|{
	賢曰雒陽南宫有崇德殿}
太后與兄車騎將軍隲虎賁中郎將悝等定策禁中|{
	悝苦回翻}
其夜使隲持節以王青蓋車迎清河王子祜|{
	賢曰續漢志曰皇太子皇子皆安車朱班輪青盖金華蚤皇子為王錫以乘之故曰王青盖車皇孫則緑車}
齋于殿中皇太后御崇德殿百官皆吉服陪位|{
	賢曰不可以凶事臨朝改吉服也}
引拜祜為長安侯|{
	賢曰不即立為天子而封侯者不欲從微即登皇位余謂先封候者用立孝宣帝故事也}
乃下詔以祜為孝和皇帝嗣又作策命有司讀策畢太尉奉上璽綬即皇帝位|{
	上時掌翻璽斯氏翻綬音受}
太后猶臨朝|{
	公羊傳曰猶者可止之辭}
詔告司隸校尉河南尹南陽太守曰每覽前代外戚賓客濁亂奉公|{
	言其挾埶恣横奉公之吏為所濁亂也}
為民患苦咎在執法怠懈不輒行其罰故也|{
	懈古隘翻}
今車騎將軍隲等雖懷敬順之志而宗門廣大姻戚不小賓客姦猾多干禁憲|{
	賢曰干犯也}
其明加檢敕勿相容護自是親屬犯罪無所假貸 九月六州大水 丙寅葬孝殤皇帝于康陵|{
	賢曰康陵在慎陵塋中庚地}
以連遭大水百姓苦役方中秘藏|{
	賢曰方中陵中也冢藏之中故言秘也孔頴逹曰凡天子之葬掘地為方壙漢書謂之方中方中之内先累椁於其方中南而為羨道以蜃車載柩至壙說而載以龍輴從羨道而入至方中乃屬紼於棺之緘從上而下棺入於椁之中方上謂覆坑方石上}
及諸工作事減約十分居一|{
	十分居一者減其九分也}
乙亥殞石于陳留|{
	陳留郡在雒陽東五百三十里}
詔以北地梁慬為西域副校尉|{
	慬音勤校戶教翻}
慬行至河西會西域諸國反攻都護任尚於疏勒尚上書求救詔慬將河西四郡羌胡五千騎馳赴之慬未至而尚已得解詔徵尚還以騎都尉段禧為都護西域長史趙博為騎都尉禧博守它乾城|{
	班超為都護居龜兹它乾城}
城小梁慬以為不可固乃譎說龜兹王白覇|{
	譎古宂翻說輸芮翻龜兹音丘慈}
欲入共保其城白覇許之吏民固諫白覇不聼慬既入遣將急迎段禧趙博合軍八九千人龜兹吏民並叛其王而與温宿姑墨數萬兵反共圍城慬等出戰大破之連兵數月胡衆敗走乘勝追撃凡斬首萬餘級獲生口數千人龜兹乃定|{
	梁慬非不健闘然終不能定西域者徒勇而無策畧也}
冬十月四州大水雨雹|{
	雨于具翻}
清河孝王慶病篤上書求葬樊濯宋貴人冢旁|{
	欲從其母也}
十二月甲子王薨 乙酉罷魚龍曼延戲|{
	武帝元封二年作魚龍曼延戲今罷之曼音萬延衍而翻}
尚書郎南陽樊凖以儒風寖衰上疏曰臣聞人君不可以不學光武皇帝受命中興東西誅戰不遑啓處|{
	處昌呂翻}
然猶投戈講藝|{
	藝六藝也}
息馬論道孝明皇帝庶政萬機無不簡心|{
	朱子曰簡閲也}
而垂情古典游意經藝每饗射禮畢正坐自講諸儒並聼四方欣欣又多徵名儒布在廊廟每宴會則論難衎衎|{
	賢曰衎衎和樂貌也難乃旦翻衎苦旱翻又苦汗翻}
共求政化期門羽林介胄之士悉通孝經|{
	期門即虎賁士}
化自聖躬流及蠻荒是以議者每稱盛時咸言永平今學者益少|{
	少詩沼翻}
遠方尤甚博士倚席不講|{
	賢曰禮記曰凡侍坐於大司成者遠間三席又曰若非飲食之客則布席席間函文注云謂講問客也倚席言不施講坐也}
儒者競論浮麗忘蹇蹇之忠習諓諓之辭|{
	賢曰諓諓謟言也音踐前書曰昔秦穆公說諓諓之言}
臣愚以為宜下明詔博求幽隱寵進儒雅以俟聖上講習之期|{
	時安帝始年十三故請求儒雅以俟講習}
太后深納其言詔公卿中二千石各舉隱士大儒務取高行以勸後進|{
	行下孟翻}
妙簡博士|{
	妙精也簡擇也}
必得其人

孝安皇帝上|{
	諱祜肅宗孫也父曰清河孝王慶諡法寛容和平曰安伏侯古今注曰祜之字曰福}


永初元年春正月癸酉朔赦天下 蜀郡徼外羌内屬|{
	徼吉弔翻下同}
二月丁卯分清河國封帝弟常保為廣川王|{
	廣川縣屬信都國賢曰故城在今冀州棗彊縣東北}
庚午司徒梁鮪薨 三月癸酉日有食之 己卯永昌徼外僬僥種夷陸類等舉種内附|{
	永昌郡在雒陽西七千二百六十里僬僥國人長不過三尺徼吉弔翻僬兹消翻僥倪么翻種章勇翻}
甲申葬清河孝王於廣丘|{
	廣丘在清河厝縣後更名甘陵}
司空宗正護喪事儀比東海恭王|{
	恭王葬見四十五卷明帝永平元年 考異曰帝紀書車騎將軍護葬今從傳}
自和帝之喪鄧隲兄弟常居禁中隲不欲久在内連求還第太后許之夏四月封太傅張禹太尉徐防司空尹勤車騎將軍鄧隲城門校尉鄧悝虎賁中郎將鄧宏黄門郎鄧閶皆為列侯|{
	禹安鄉侯防龍鄉侯隲上蔡侯悝葉侯宏西平侯閶西華侯閶音昌 考異曰袁紀前作閶後作闓盖誤}
食邑各萬戶隲以定策功增三千戶隲及諸弟辭讓不獲遂逃避使者間關詣闕|{
	賢曰間關猶崎嶇也}
上疏自陳至于五六乃許之 五月甲戌以長樂衛尉魯恭為司徒|{
	樂音洛}
恭上言舊制立秋乃行薄刑自永元十五年以來改用孟夏|{
	事見上卷}
而刺史太守因以盛夏徵召農民拘對考驗連滯無已|{
	連謂獄辭相連及也滯謂留滯不决也}
上逆時氣下傷農業案月令孟夏斷薄刑者謂其輕罪已正|{
	謂已結正也斷丁亂翻下同}
不欲令久繫故時斷之也臣愚以為今孟夏之制可從此令其决獄案考皆以立秋為斷又奏孝章皇帝欲助三正之微定律著令斷獄皆以冬至之前|{
	事見四十七卷章帝元和三年}
小吏不與國同心者率十一月得死罪賊不問曲直便即格殺雖有疑罪不復讞正|{
	復扶又翻讞魚列翻又魚戰翻又魚蹇翻議獄也}
可令大辟之科|{
	辟毗亦翻}
盡冬月乃斷朝廷皆從之 丁丑詔封北海王睦孫夀光侯普為北海王|{
	和帝永元八年北海王威自殺今復紹封夀光縣本屬北海後屬樂安國}
九真徼外夜郎蠻夷舉土内屬 西域都護段禧等

雖保龜兹而道路隔塞|{
	塞悉則翻}
檄書不通公卿議者以為西域阻遠數有背叛|{
	數所角翻背蒲妹翻}
吏士屯田其費無已六月壬戌罷西域都護|{
	和帝永元三年復置西域都護今罷}
遣騎都尉王宏發關中兵迎禧及梁慬趙博伊吾盧柳中屯田吏士而還 初燒當羌豪東號之子麻奴隨父來降|{
	東號降見四十七卷和帝永元元年降戶江翻}
居于安定時諸降羌布在郡縣皆為吏民豪右所徭役|{
	徭使也}
積以愁怨及王宏西迎段禧發金城隴西漢陽羌數百千騎與俱郡縣廹促發遣羣羌懼遠屯不還行到酒泉頗有散叛諸郡各發兵邀遮或覆其廬落于是勒姐當煎大豪東岸等愈驚遂同時犇潰|{
	姐音紫且翻又音紫}
麻奴兄弟因此與種人俱西出塞滇零與鍾羌諸種大為寇掠斷隴道|{
	零音憐續漢書曰鍾羌九千餘戶在隴西臨洮谷隴道隴坁之道也種章勇翻斷丁管翻}
時羌歸附既久無復器甲或持竹竿木枝以代戈矛或負板案以為楯|{
	楯食尹翻}
或執銅鏡以象兵|{
	銅鏡映日人遥望之以為兵也}
郡縣畏懦不能制丁卯赦除諸羌相連結謀叛逆者罪 秋九月庚午太尉徐防以災異寇賊策免三公以災異免自防始辛未司空尹勤以水雨漂流策免 仲長統昌言曰|{
	仲姓也商湯左相仲虺周冇仲山甫舜十六相有仲堪仲熊周八士有仲突仲忽}
光武皇帝愠數世之失權忿彊臣之竊命|{
	賢曰昌當也愠猶恨也數世謂元成哀平彊臣謂王莽}
矯枉過直政不任下雖置三公事歸臺閣|{
	賢曰臺閣謂尚書余謂三公失職非至光武時始然也自武帝游宴後庭用宦者處樞機至于宣帝專任恭顯而丞相御史取充位事歸臺閣其所由來者漸也}
自此以來三公之職備員而已然政有不治猶加譴責而權移外戚之家寵被近習之豎|{
	被皮義翻}
親其黨類用其私人内充京師外布州郡顛倒賢愚貿易選舉|{
	貿音茂}
疲駑守境|{
	駑音奴駑駘馬之下乘以喻不才之吏}
貪殘牧民撓擾百姓|{
	撓音火高翻}
忿怒四夷招致乖叛亂離斯瘼|{
	用詩語賢曰瘼病也}
怨氣並作隂陽失和三光虧缺怪異數至|{
	數所角翻}
蟲螟食稼水旱為災此皆戚宦之臣所致然也反以策讓三公至于死免乃足為叫呼蒼天號咷泣血者矣|{
	放聲而哭曰號咷號戶刀翻咷徒刀翻}
又中世之選三公也務於清慤謹慎循常習故者是以婦女之檢柙|{
	賢曰檢柙猶規矩也揚子曰蠢廸檢柙注云檢柙猶隱栝也毛晃曰檢柙檢束也輔也俗作撿押非}
鄉曲之常人耳惡足以居斯位邪|{
	惡音烏下同}
埶既如彼選又如此而欲望三公勲立於國家績加於生民不亦遠乎昔文帝之於鄧通可謂至愛而猶展申徒嘉之志|{
	事見十四卷文帝後二年}
夫見任如此則何患於左右小臣哉至於近世外戚宦豎請託不行意氣不滿立能陷人於不測之禍惡可得彈正者哉曩者任之重而責之輕今者任之輕而責之重光武奪三公之重至今而加甚不假后黨以權數世而不行蓋親疏之埶異也|{
	賢曰言光武奪三公重任今奪更甚光武不假后黨威權數代遂不遵行此為三公疏后黨親故也}
今人主誠專委三公分任責成而在位病民|{
	病民謂百姓受其害也}
舉用失賢百姓不安争訟不息天地多變人物多妖|{
	妖於驕翻}
然後可以分此罪矣 壬午詔太僕少府減黄門鼔吹以補羽林士|{
	漢官儀曰黄門鼔吹百四十五人羽林左監主羽林八百人右監主九百人杜佑曰漢代有黄門鼔吹享宴食舉樂十三曲與魏代鼔吹長簫伎録並云絲竹合作執節者歌又建初録云務成黄爵玄雲遠期皆騎吹曲非鼔吹曲此則列於殿庭者為鼔吹今之從行者為騎吹二曲異也孫權觀魏武軍作鼔吹而還應是此鼔吹魏晉代給鼓吹甚輕牙門督將五校悉有鼔吹齊梁至陳則重矣今代短簫鐃歌亦謂之鼓吹蔡邕曰鼔吹軍樂也黄帝岐伯所作以揚威武勸士諷敵也雍門周說孟嘗君鼔吹於不測之淵說者云鼔自一物吹自竽之屬非簫鼔合奏别為一樂之名也然則短簫鐃歌此時未名鼔吹矣宋白曰鼔吹據崔豹古今注張騫使西域得摩訶兜勒一曲李延年增之分為二十八曲梁置清商鼓吹令二人唐又有揺鼔金鉦大鼓長鳴歌簫笳笛合為鼓吹十二案大享會則設於懸外}
廐馬非乘輿常所御者皆減半食|{
	賢曰乘輿所乘車輿也不敢斥言尊者故稱乘輿見蔡邕獨斷乘繩證翻}
諸所造作非供宗廟園陵之用皆且止 庚寅以太傅張禹為太尉太常周章為司空大長秋鄭衆中常侍蔡倫等皆秉勢豫政周章數進直言|{
	數所角翻}
太后不能用初太后以平原王勝有痼疾而貪殤帝孩抱養為己子故立焉及殤帝崩羣臣以勝疾非痼意咸歸之大后以前不立勝恐後為怨乃迎帝而立之周章以衆心不附密謀閉宫門誅鄧隲兄弟及鄭衆蔡倫刼尚書廢太后於南宫封帝為遠國王而立平原王事覺冬十一月丁亥章自殺 戊子敕司隸校尉冀幷二州刺史民訛言相驚弃捐舊居老弱相携窮困道路其各敇所部長吏躬親曉喻|{
	長知兩翻}
若欲歸本郡在所為封長檄不欲勿彊|{
	為于偽翻賢曰封謂印封之也長檄猶今長牒也欲歸者皆給以長牒為驗彊音其兩翻}
十二月乙卯以頴川太守張敏為司空詔車騎將軍鄧隲征西校尉任尚將五營及諸郡兵五萬人|{
	五營北軍五校營也}
屯漢陽以備羌 |{
	考異曰帝紀在六月今從西羌傳}
是歲郡國十八地震四十一大水二十八大風雨雹 鮮卑大人燕荔陽詣闕朝賀太后賜燕荔陽王印綬赤車參駕|{
	赤車者帷裳衡軛皆赤參駕者駕二馬燕於賢翻荔力計翻}
令止烏桓校尉所居甯城下|{
	甯城屬上谷郡}
通胡市因築南北兩部質館|{
	賢曰築館以受降質質音致下同}
鮮卑邑落百二十部各遣入質

二年春正月鄧隲至漢陽諸郡兵未至鍾羌數千人擊敗隲軍于冀西|{
	冀縣之西也敗補邁翻}
殺千餘人梁慬還至燉煌|{
	自西域還也燉徒門翻}
逆詔慬留為諸軍援|{
	逆迎也}
慬至張掖|{
	張掖郡在雒陽西四千二百里應劭曰張掖者言為國張臂掖也}
破諸羌萬餘人其能脱者十二三進至姑臧羌大豪三百餘人詣慬降並慰譬遣還故地 御史中丞樊凖以郡國連年水旱民多飢困上疏請令大官尚方考功上林池籞諸官實減無事之物|{
	賢曰前書百官表少府掌山海池澤之税屬官有太官考工尚方上林中十池監太官掌御膳飲食考工主作器械尚方主作刀劒實減謂實覆其數減之也功當作工籞偶許翻}
五府調省中都官吏京師作者|{
	賢曰五府謂太傳太尉司徒司空大將軍也調徵也省減也中都官吏在京師之官吏也作謂營作者也余按是時不拜大將軍獨鄧隲為車騎將軍耳調徒弔翻}
又被災之郡百姓凋殘恐非賑給所能勝贍|{
	被皮義翻勝音升}
雖有其名終無其實可依征和元年故事|{
	賢曰武帝征和元年詔曰當今務在禁苛暴止擅賦力本農桑母乏武備而已予據此乃征和四年詔也征和元年當有遣使慰安故事}
遣使持節慰安尤困乏者徙置荆揚孰郡|{
	孰古熟字通}
今雖有西屯之役宜先東州之急|{
	西屯謂討羌之師東州謂雒陽以東冀兖諸州被水旱也先悉薦翻}
太后從之悉以公田賦與貧民|{
	賦布也}
即擢凖與議郎呂倉並守光禄大夫二月乙丑遣凖使冀州倉使兖州稟貸流民咸得蘇息|{
	稟給也貸施也死而更生曰蘇氣絶而復續曰息}
夏旱五月丙寅皇太后幸雒陽寺|{
	賢曰寺官舍也風俗通曰寺嗣也理事之吏嗣續於其中}
及若盧獄|{
	前漢有若盧獄屬少府漢舊儀曰主鞫將相大臣東都初省和帝永元九年復置}
録囚徒雒陽有囚實不殺人而被考自誣羸困輿見|{
	輿箯輿也獄囚被掠委困者以箯輿處之師古曰編竹木以為輿形如今之食輿見賢遍翻箯音編}
畏吏不敢言將去舉頭若欲自訴太后察視覺之即呼還問狀具得枉實|{
	得其見枉之實也}
即時收雒陽令下獄抵罪|{
	下遐稼翻}
行未還宮澍雨大降|{
	澍音注又殊遇翻時雨也}
六月京師及郡國四十大水大風雨雹|{
	東觀記曰雹大如芋魁鷄子風拔樹發屋雨于具翻}
秋七月太白入北斗|{
	晉書天文志北斗七星在太微北七政之樞機隂陽之元本也故運乎中央而臨制四方所以建四時而均五行也天文志曰太白入斗中為貴相凶}
閏月廣川王常保薨無子國除 癸未蜀郡徼外羌舉土内屬|{
	東觀記曰徼外羌薄申等八種舉衆降徼吉弔翻}
冬鄧隲使任尚及從事中郎河内司馬鈞率諸郡兵與滇零等數萬人戰于平襄|{
	平襄縣屬漢陽郡賢曰故襄戎邑零音憐}
尚軍大敗死者八千餘人羌衆遂大盛朝廷不能制湟中諸縣粟石萬錢百姓死亡不可勝數而轉運難劇|{
	劇甚也勝音升}
故左校令河南龎參|{
	將作大匠屬官而左右校令各一人秩六百石左校令掌左工徒右校令掌右工徒校戶教翻龎皮江翻}
先坐法輸作若盧使其子俊上書曰方今西州流民擾動而徵發不絶水潦不休地力不復|{
	賢曰言其耗損不復於舊}
重之以大軍|{
	重直用翻}
疲之以遠戍農功消於轉運資財竭於徵發田疇不得墾闢禾稼不得收入手困窮|{
	賢曰兩手相言無計也}
無望來秋百姓力屈不復堪命|{
	復扶又翻}
臣愚以為萬里運糧遠就羌戎不若總兵養衆以待其疲車騎將軍隲宜且振旅|{
	傳曰三年而治兵入而振旅書曰班師振旅振整也整衆而還也}
留征西校尉任尚使督凉州士民轉居三輔|{
	參建棄凉州之議發於此書}
休徭役以助其時止煩賦以益其財令男得耕種女得織絍|{
	賢曰絍音如深翻杜預注左傳云織絍織繒布也字釋云絍機縷也又如沁翻}
然後畜精鋭乘懈沮出其不意攻其不備則邉民之仇報犇北之耻雪矣|{
	懈古隘翻沮在呂翻}
書奏會樊凖上疏薦參太后即擢參於徒中召拜謁者使西督三輔諸軍屯十一月辛酉詔鄧隲還師留任尚屯漢陽為諸軍節度遣使迎拜隲為大將軍既至使大鴻臚親迎中常侍郊勞|{
	臚陵如翻勞力到翻}
王主以下候望於道寵靈顯赫光震都鄙|{
	王主諸王及諸公主也鄧隲西征無功而還當引罪求自貶以天下據埶持權冒受榮寵於心安乎君子是以知其不終也}
滇零自稱天子於北地招集武都參狼|{
	羌居武都者為参狼種参所簪翻}
上郡西河諸雜種羌斷隴道寇鈔三輔|{
	斷丁管翻鈔楚交翻}
南入益州殺漢中太守董炳梁慬受詔當屯金城聞羌寇三輔即引兵赴擊轉戰武功美陽間|{
	武功美陽二縣屬扶風}
連破走之羌稍退散 十二月廣漢塞外參狼羌降|{
	此與武都參狼同種而分居廣漢塞外者也}
是歲郡國十二地震三年春正月庚子皇帝加元服赦天下 遣騎都尉任仁督諸郡屯兵救三輔仁戰數不利|{
	數所角翻}
當煎勒姐羌攻沒破羌縣|{
	破羌縣屬金城縣}
鍾羌攻沒臨洮縣執隴西南部都尉|{
	臨洮縣隴西南部都尉治所洮土刀翻}
三月京師大飢民相食壬辰公卿詣闕謝詔務思變復以助不逮|{
	變改也改過以復於善也}
壬寅司徒魯恭罷恭再在公位|{
	和帝永元十二年恭代呂蓋為司徒永初元年復代梁鮪為司徒}
選辟高第至列卿郡守者數十人|{
	此謂恭府掾屬之高第也守手又翻}
而門下耆生|{
	耆老也}
或不蒙薦舉至有怨望者恭聞之曰學之不講是吾憂也|{
	論語孔子之言}
諸生不有鄉舉者乎|{
	賢曰言人患學之不習耳若能究習自有鄉里之舉豈要待三公之辟乎}
終無所言亦不借之議論學者受業必窮核問難|{
	難乃旦翻}
道成然後謝遣之學者曰魯公謝與議論不可虛得 夏四月丙寅以大鴻臚九江夏勤為司徒 三公以國用未足奏令吏民入錢穀得為關内侯虎賁羽林郎五官大夫官府吏緹騎營士各有差|{
	五官亦郎也大夫光禄太中中散諫議大夫也官府吏給事諸官府者賢曰續漢志曰執金吾緹騎二百人緹赤黄色營士謂五校營士也緹丁禮翻又丁奚翻}
甲申清河愍王虎威薨無子五月丙申封樂安王寵子延平為清河王奉孝王後 六月漁陽烏桓與右北平胡千餘寇代郡上谷 漢人韓琮隨匈奴南單于入朝|{
	漢人與匈奴錯居韓琮因事南單于琮徂宗翻}
既還說南單于云|{
	說式芮翻}
關東水人民飢餓死盡可擊也單于信其言遂反 秋七月海賊張伯路等寇濱海九郡殺二千石令長|{
	長知兩翻}
遣侍御史巴郡龎雄督州郡兵擊之伯路等乞降尋復屯聚|{
	復扶又翻}
九月鴈門烏桓率衆王無何允與鮮卑大人邱倫等及南匈奴骨都合七千騎寇五原與大守戰于高渠谷|{
	賢曰東觀記戰九原高梁谷渠梁相類必有誤}
漢兵大敗 南單于圍中郎將耿种於美稷|{
	使匈奴中郎將也种音冲}
冬十一月以大司農陳國何熙行車騎將軍事中郎將龎雄為副將五營及邉郡兵二萬餘人又詔遼東太守耿夔率鮮卑及諸郡共擊之以梁慬行度遼將軍事雄夔擊南匈奴薁鞬日逐王破之|{
	薁於六翻鞬居言翻}
十二月辛酉郡國九地震 乙亥有星孛于天苑|{
	晉書天文志天苑十六星在畢南天子之苑圃養獸之所也孛蒲内翻}
是歲京師及郡國四十一雨水幷凉二州大飢人相食 太后以隂陽不和軍旅數興|{
	數所角翻}
詔歲終饗遣衛士勿設戲作樂|{
	西都之制歲盡衛卒交代上臨饗罷遣之續漢志曰饗遣故衛士儀百官會位定謁者持節引故衛士入自端門衛司馬執幡鉦護行行定侍御史持節慰勞以詔恩問所疾苦受其章奏所欲言畢饗賜作樂觀以角扺樂闋罷遣勸以農桑}
減逐疫侲子之半|{
	賢曰侲子逐疫之人也音振薛綜注西京賦云侲之言善也善童幼子也續漢書曰大儺選中黄門子弟年十歲以上十二以下百二十人為侲子皆赤幘皂製執大鞉}


四年春正月元會徹樂不陳充庭車|{
	賢曰每大朝會必陳乘輿法物車輦於庭以年飢故不陳}
鄧隲在位頗能推進賢士薦何熙李郃等列於朝廷|{
	郃曷閤翻}
又辟弘農楊震巴郡陳禪等置之幕府天下稱之震孤貧好學明歐陽尚書通達博覽諸儒為之語曰關西孔子楊伯起|{
	楊震字伯起居弘農在函谷關之西}
教授二十餘年不答州郡禮命|{
	禮謂延聘之禮命謂辟置之命}
衆人謂之晚暮|{
	謂歲月已老而出仕遲也}
而震志愈篤隲聞而辟之時震年已五十餘累遷荆州刺史東萊太守|{
	郡國志東萊郡在雒陽東三千一百二十八里}
當之郡道經昌邑|{
	昌邑縣屬山陽郡賢曰昌邑故城在今兖州金鄉縣西北}
故所舉荆州茂才王密為昌邑令夜懷金十斤以遺震|{
	遺于季翻下同}
震曰故人知君君不知故人何也密曰暮夜無知者震曰天知地知我知子知何謂無知者密愧而出後轉涿郡太守|{
	郡國志涿郡在雒陽東北千八百里}
性公廉子孫常蔬食步行|{
	食不魚肉行不車騎也}
故舊或欲令為開產業|{
	為于偽翻}
震不肯曰使後世稱為清白吏子孫以此遺之不亦厚乎 張伯路復攻郡縣殺守令|{
	復扶又翻下同}
黨衆浸盛詔遣御史中丞王宗持節發幽冀諸郡兵合數萬人徵宛陵令扶風法雄為青州刺史|{
	宛陵縣屬河南尹法姓也齊襄王法章之後秦滅齊子孫不敢稱田姓故以法為氏}
與宗并力討之 南單于圍耿种數月梁慬耿夔撃斬其别將於屬國故城|{
	班志西河美稷縣屬國都尉治故城盖在美稷縣界將即亮翻下同}
單于自將迎戰慬等復破之單于遂引還虎澤|{
	班志西河郡穀羅縣虎澤在西北師古避唐諱以虎為武}
丙午詔減百官及州郡縣奉各有差|{
	奉讀曰俸}
二月南匈奴寇常山 滇零遣兵寇褒中|{
	褒中縣屬漢中郡古褒國也賢曰今梁州褒城縣}
漢中太守鄭勤移屯褒中任尚軍久出無功民廢農桑乃詔尚將吏民還屯長安罷遣南陽穎川汝南吏士乙丑初置京兆虎牙都尉於長安扶風都尉於雍如西京三輔都尉故事|{
	賢曰漢官儀京兆虎牙扶風都尉以凉州近羌數犯三輔將兵衛護園陵扶風都尉居雍故俗人稱雍營西京三輔京兆有京輔都尉馮翊有左輔都尉扶風有右輔都尉}
謁者龎參說鄧隲|{
	說輸芮翻}
徙邉郡不能自存者入居三輔隲然之欲棄凉州并力北邉乃會公卿集議隲曰譬若衣敗壞一以相補猶有所完|{
	壞音怪}
若不如此將兩無所保郎中陳國虞詡言於太尉張禹曰若大將軍之策不可者三先帝開拓土宇劬勞後定而今憚小費舉而棄之此不可一也凉州既棄即以三輔為塞|{
	隴西安定北地皆凉州所部凉州既弃則三輔為極邉}
則園陵單外此不可二也喭曰關西出將關東出相|{
	賢曰說文喭傳言也前書秦漢以來山西出將山東出相秦時郿白起穎陽王翦漢興義渠公孫賀傳介子成紀李廣李蔡上邽趙充國狄道辛武賢皆名將也丞相則蕭曹魏丙韋平孔翟之類也}
烈士武臣多出凉州土風壯猛便習兵事今羌胡所以不敢入據三輔為心腹之害者以凉州在後故也凉州士民所以推鋒執鋭蒙矢石於行陳|{
	行戶剛翻陳讀曰陣}
父死於前子戰於後無反顧之心者為臣屬於漢故也|{
	為于偽翻}
今推而捐之|{
	推通回翻}
割而弃之民庶安土重遷必引領而怨曰中國弃我於夷狄雖赴義從善之人不能無恨如卒然起謀|{
	卒讀曰猝}
因天下之飢敝乘海内之虚弱豪雄相聚量材立帥|{
	量音良帥所類翻}
驅氐羌以為前鋒席卷而東雖賁育為卒太公為將|{
	賁音奔將即亮翻}
猶恐不足當禦如此則函谷以西園陵舊京非復漢有|{
	復扶又翻}
此不可三也|{
	是後北宫伯玉王國閻忠馬騰韓遂之變卒如詡言賢曰席卷言無餘也前書曰雲徹席卷後無餘災也予謂席卷者言其埶便易也}
議者喻以補衣猶有所完詡恐其疽食侵淫而無限極也|{
	賢曰疽癰瘡也余謂食者言其侵食肌肉也}
禹曰吾意不及此微子之言幾敗國事|{
	微無也幾居希翻}
詡因說禹收羅凉土豪桀|{
	說輸芮翻}
引其牧守子弟於朝|{
	守式又翻朝直遥翻}
令諸府各辟數人外以勸厲答其功勤内以拘致防其邪計禹善其言更集四府皆從詡議於是辟西州豪桀為掾屬拜牧守長吏子弟為郎以安慰之|{
	掾俞絹翻長知兩翻下同}
鄧隲由是惡詡欲以吏法中傷之|{
	惡烏路翻中竹仲翻}
會朝歌賊甯季等數千人攻殺長吏屯聚連年州郡不能禁乃以詡為朝歌長|{
	朝歌縣屬河内郡賢曰朝歌故城在今衛州衛縣西}
故舊皆弔之|{
	謂其將得罪也}
詡笑曰事不避難臣之職也不遇槃根錯節無以别利器|{
	以斤斧自喻也别彼列翻}
此乃吾立功之秋也始到謁河内太守馬稜稜曰君儒者當謀謨廟堂乃在朝歌甚為君憂之|{
	為于偽翻}
詡曰此賊犬羊相聚以求温飽耳願明府不以為憂稜曰何以言之詡曰朝歌者韓魏之郊|{
	賢曰韓界上黨魏界河内相接太行故云郊也}
背太行|{
	背蒲妹翻行戶剛翻}
臨黄河去敖倉不過百里而青冀之民流亡萬數賊不知開倉招衆刼庫兵守成臯斷天下右臂|{
	賢曰右臂喻要便也余謂右臂之說祖張儀見三卷周赧王四年斷丁管翻}
此不足憂也今其衆新盛難與争鋒兵不厭權願寛假轡策勿令有所拘閡而已|{
	詡欲用度外之人以制羣盗恐郡家循常襲故以文法繩之故先以此言於稜賢曰閡與礙同}
及到官設三科以募求壯士自掾史以下各舉所知|{
	百官志縣有廷掾猶郡之五官掾也監鄉部春夏為勸農掾秋冬為制度掾史則有獄史佐史斗食令史掾史幹小史}
其攻刼者為上傷人偷盗者次之不事家業者為下收得百餘人詡為饗會悉貰其罪|{
	此三等人皆惡少年負宿罪者也悉貰之使入賊為間為于偽翻貰始制翻}
使入賊中誘令刼掠乃伏兵以待之遂殺賊數百人|{
	誘賊出刼掠而伏兵殺之誘音酉}
又潜遣貧人能縫者傭作賊衣以采線縫其裾有出市里者吏輒禽之賊由是駭散咸稱神明縣境皆平 三月何熙軍到五原曼柏暴疾不能進遣龎雄與梁慬耿种將步騎萬六千人攻虎澤連營稍前單于見諸軍並進大恐怖|{
	怖普布翻}
顧讓韓琮曰汝言漢人死盡今是何等人也|{
	賢曰顧反也讓責也反顧責韓琮也}
乃遣使乞降許之單于脱帽徒跣對龎雄等拜陳道死罪|{
	自陳謝罪言當死也}
於是赦之遇待如初乃還所鈔漢民男女|{
	鈔楚交翻}
及羌所畧轉賣入匈奴中者合萬餘人會熙卒即拜梁慬為度遼將軍龎雄還為大鴻臚先零羌復寇褒中|{
	復扶又翻下同}
鄭勤欲擊之主簿段崇諫

以為虜乘勝鋒不可當宜堅守待之勤不從出戰大敗死者三千餘人段崇及門下史王宗原展|{
	周原伯之後有原莊公又晉卿先軫邑於原子孫以為氏又孔子弟子有原憲}
以身扞刃與勤俱死|{
	郡門下有掾有史}
徙金城郡居襄武|{
	襄武縣屬隴西郡賢曰今渭州縣}
戊子杜陵園火|{
	宣帝陵園也}
癸巳郡國九地震 夏四月六州蝗|{
	東觀記曰司隸豫兖徐青冀六州}
丁丑赦天下 王宗法雄與張伯路連戰破走之會赦到|{
	赦書到也}
賊以軍未解甲不敢歸降|{
	降戶江翻}
王宗召刺史太守共議|{
	刺史青州刺史太守青州所部諸郡太守}
皆以為當遂擊之法雄曰不然兵凶器戰危事|{
	賢曰史記范蠡之辭}
勇不可恃勝不可必賊若乘船浮海深入遠島|{
	島都老翻水中有山曰島}
攻之未易也|{
	易以䜴翻}
及有赦令可且罷兵以慰誘其心埶必解散然後圖之可不戰而定也宗善其言即罷兵賊聞大喜乃還所畧人而東萊郡兵猶未解甲賊復驚恐遁走遼東止海島上|{
	果如法雄之言}
秋七月乙酉三郡大水騎都尉任仁與羌戰累敗而兵士放縱檻車徵詣廷尉死護羌校尉段禧卒復以前校尉侯覇代之移居張掖|{
	永初二年侯覇以衆羌反叛免護羌校尉時居狄道按水經注羌水出湟中西南山下逕護羌城東故護羌校尉治又東逕臨羌城西護羌校尉盖治金城郡臨羌縣界也然宣帝置護羌校尉本治金城令居東都定河隴之後護羌校尉治安夷縣既而自安夷徙臨羌侯覇先居隴西狄道以羌叛而臨羌不可居也今移居張掖以隴西殘破復度河而西}
九月甲申益州郡地震|{
	賢曰益州郡故城在今昆州晉寧縣}
皇太后母新野君病|{
	續漢志曰婦人封君儀比公主油鞬軿車帶綬以采組為緄帶各如其綬色黄金辟邪加其首為帶}
太后幸其第連日宿止三公上表固争乃還宫冬十月甲戌新野君薨使司空護喪事儀比東海恭王|{
	恭王事見四十四卷明帝永平元年}
鄧隲等乞身行服太后欲不許以問曹大家大家上疏曰妾聞謙讓之風德莫大焉今四舅深執忠孝引身自退|{
	賢曰四舅謂隲悝宏閶也}
而以方垂未静拒而不許如後有毫毛加於今日誠恐推讓之名不可再得|{
	賢曰謂有纎微之過則推讓之美失也}
太后乃許之及服除詔隲復還輔朝政|{
	朝直遥翻}
更授前封|{
	帝即阼之初封隲悝宏閶皆辭不受}
隲等叩頭固讓乃止于是並奉朝請位次三公下特進侯上|{
	賢曰在特進及侯之上請才性翻又如字}
其有大議乃詣朝堂與公卿參謀 太后詔隂后家屬皆歸故郡|{
	隂后家南徙事見上卷和帝永元十四年歸故郡歸南陽也}
還其資財五百餘萬

五年春正月庚辰朔日有食之 丙戌郡國十地震己丑太尉張禹免甲申以光禄勲潁川李修為太尉先零羌寇河東至河内|{
	零音憐}
百姓相驚多南奔度河使北軍中候朱寵將五營士屯孟津|{
	北軍中候掌監屯騎越騎步兵長水射聲五營續漢志曰舊有中壘校尉領北軍營壘之事中興省中壘但置中候以監五營洪氏隸釋曰按祝睦後碑書為北軍軍中候則知此亦省文耳}
詔魏郡趙國常山中山繕作塢候六百一十六所|{
	郡國四皆屬冀州懼羌自河東河内北入冀州界故作塢候以備之}
羌既轉盛而緣邉二千石令長多内郡人並無守戰意皆争上徙郡縣以避寇難|{
	長知兩翻上時掌翻}
三月詔隴西徙襄武|{
	隴西郡本治狄道 考異曰上云金城徙襄武此又云隴西徙襄武紀傳皆然或者二郡皆寄治於襄武歟}
安定徙美陽北地徙池陽上郡治衙|{
	賢曰安定郡今涇州也美陽故城在今武功縣北北地郡今寜州池陽縣故城在今涇陽縣北上郡今綏州也衙縣故城在今同州白水縣東北左傳秦晉戰于彭衙即此也予按郡國志美陽縣屬扶風池陽衙二縣屬馮翊衙音牙}
百姓戀土不樂去舊|{
	樂音洛}
遂乃刈其禾稼發撤室屋夷營壁破積聚|{
	積子賜翻聚慈喻翻}
時連旱蝗飢荒而驅䠞刼掠流離分散隨道死亡或弃捐老弱或為人僕妾喪其太半|{
	䠞與蹙同喪息浪翻}
復以任尚為侍御史擊羌於上黨羊頭山破之|{
	賢曰羊頭山在上黨郡穀遠縣復扶又翻下同}
乃罷孟津屯 夫餘王寇樂浪|{
	夫餘為寇始此夫音扶樂浪音洛琅}
高句驪王宫與濊貊寇玄菟|{
	句如字又音駒驪力知翻濊音穢貊莫百翻菟同都翻}
夏閏四月丁酉赦凉州河西四郡 海賊張伯路復寇東萊青州刺史法雄擊破之賊逃還遼東遼東人李久等共斬之|{
	張伯路永初三年作亂至是始平}
於是州界清静秋九月漢陽人杜琦及弟季貢同郡王信等與羌通

謀聚衆據上邽城冬十二月漢陽太守趙博遣客杜習刺殺琦|{
	刺七亦翻}
封習討姧侯杜季貢王信等將其衆據樗泉營 是歲九州蝗郡國八雨水

六年春正月甲寅詔曰凡供薦新味多非其節或欝養彊孰|{
	謂為土室蓄火使土氣蒸欝而養之彊使成熟也前書召信臣傳曰大官園種冬生葱韭菜茹覆以屋廡晝夜然藴火待温氣乃生彊其兩翻}
或穿掘萌芽味無所至而夭折生長|{
	夭於兆翻短折曰夭長知兩翻}
豈所以順時育物乎傳曰非其時不食|{
	賢曰論語曰不時不食言非其時物則不食之前書召信臣曰不時之物有傷於人不宜以奉供養}
自今當奉祠陵廟及給御者皆須時乃上|{
	時熟乃上進也上時掌翻}
凡所省二十三種|{
	種章勇翻}
三月十州蝗 夏四月乙丑司空張敏罷 己卯以太常劉愷為司空 詔建武元功二十八將皆紹封 五月旱 丙寅詔令中二千石下至黄綬一切復秩|{
	董巴輿服志中二千石青綬四百石三百石二百石黄綬四年減百官奉今復之}
六月壬辰豫章員谿原山崩|{
	豫章郡在雒陽南二千七百里屬揚州}
辛巳赦天下 侍御史唐喜討漢陽賊王信破斬之杜季貢亡從滇零是歲滇零死子零昌立年尚少同種狼莫為其計策|{
	滇音顛零音憐少詩照翻種章勇翻}
以季貢為將軍别居丁奚城|{
	按東觀記丁奚城在北地郡靈州縣}


七年春二月丙午郡國十八地震 夏四月己未平原懷王勝薨無子太后立樂安夷王寵子得為平原王丙申晦日有食之 秋護羌校尉侯覇騎都尉馬賢擊先零别部牢羌於安定|{
	零音憐}
獲首虜千人 蝗

元初元年春正月甲子改元 二月乙卯日南地坼長百餘里|{
	東觀記曰坼長百八十二里廣五十六里長直亮翻}
三月癸亥日有食之 |{
	考異曰帝紀二月己卯日南地坼三月癸酉日食本志及袁紀皆云三月己卯日南地坼按長歷是年二月壬辰朔無己卯三月壬戌朔癸酉十二日不應日食二月當是乙卯三月當是癸亥}
詔遣兵屯河内通谷衝要三十六所皆作塢壁設鳴鼔以備羌寇|{
	自太行北至恒山限隔并冀其間多有谷道以相通今於衝要之地作塢壁以備羌寇}
夏四月丁酉赦天下 京師及郡國五旱蝗 五月先零羌寇雍城|{
	右扶風雍縣城也雍於用翻}
蜀郡夷寇蠶陵殺縣令|{
	蠶陵縣屬蜀郡賢曰蠶陵故城在今翼州翼水縣西有蠶陵山因以名焉}
九月乙丑太尉李修罷 羌豪號多與諸種鈔掠武都漢中巴郡板楯蠻救之|{
	按范書板楯蠻夷者秦昭襄王時射殺白虎有功昭王復夷人頃田不租十妻不筭傷人者論殺人者得以倓錢贖死高祖為漢王發夷人以定三秦復其渠帥七姓不輸租賦餘戶乃歲入賨錢口四十世號為板楯蠻夷閬中有渝水其人多居水左右天性勁勇數䧟陳喜歌舞高祖為制巴渝舞蠻盖挾板楯而戰因以為名楯食尹翻倓徒濫翻賨藏宗翻}
漢中五官掾程信率郡兵與蠻共擊破之|{
	百官志郡冇五官掾署功曹及諸曹事掾俞絹翻}
號多走還斷隴道與零昌合|{
	斷丁管翻零音憐}
侯霸馬賢與戰於枹罕破之|{
	枹罕縣屬隴西郡唐之河州枹音膚}
辛未以大司農山陽司馬苞為大尉 冬十月戊子朔日有食之 凉州刺史皮楊|{
	姓譜皮樊仲皮之後又鄭有上卿子皮出于罕氏}
擊羌於狄道大敗死者八百餘人 |{
	考異曰紀作皮陽今從西羌傳予按隴西郡舊治狄道去年徒襄武則其地已弃而不有矣}
是歲郡國十五地震

二年春護羌校尉龎參以恩信招誘諸羌號多等帥衆降|{
	帥讀曰率降戶江翻}
參遣詣闕賜號多侯印遣之參始還治令居|{
	自張掖徙還令居也}
通河西道 零昌分兵寇益州遣中郎將尹就討之 夏四月丙午立貴人滎陽閻氏為皇后|{
	閻后之母鄧宏之妻之同產也故得立}
后性妬忌後宫李氏生皇子保后鴆殺李氏|{
	為后廢保張本保後立是為順帝}
五月京師旱河南及郡國十九蝗|{
	河南即京師也}
六月丙戌大尉司馬苞薨 秋七月辛巳以太僕泰山馬英為太尉 八月遼東鮮卑圍無慮|{
	無慮縣屬遼東郡應劭曰慮音閭師古曰即所謂毉無閭按郡國志無慮縣時屬遼東屬國}
九月又攻夫犂營殺縣令|{
	賢曰夫犂縣屬遼東屬國故城在今營州東南余按兩漢志遼東郡及遼東屬國皆無夫犂縣今言殺縣令則嘗為縣矣未知賢所據者何書也}
壬午晦日有食之尹就擊羌黨呂叔都等蜀人陳省羅横應募刺殺叔

都|{
	刺七亦翻}
皆封侯賜錢詔屯騎校尉班雄屯三輔雄超之子也以左馮翊司馬鈞行征西將軍督關中諸郡兵八千餘人龎參將羌胡兵七千餘人與鈞分道並擊零昌參兵至勇士東|{
	賢曰勇士縣屬天水郡余按天水時已改為漢陽郡}
為杜季貢所敗引退|{
	敗補邁翻}
鈞等獨進攻拔丁奚城杜季貢率衆偽逃鈞令右扶風仲光等收羌禾稼 |{
	考異曰袁紀作右扶風太守种暠今從范書}
光等違鈞節度散兵深入羌乃設伏要擊之|{
	要一遥翻}
鈞在城中怒而不救冬十月乙未光等兵敗並沒死者三千餘人鈞乃遁還龎參既失期稱病引還皆坐徵下獄|{
	下遐稼翻}
鈞自殺時度遼將軍梁慬亦坐事抵罪校書郎中扶風馬融|{
	融以郎中校蘭臺書故稱校書郎中}
上書稱參慬智能宜宥過責效詔赦參等 |{
	考異曰慬傳曰慬為度遼將軍明年安定北地上郡皆被羌寇不能自立詔慬發邉兵迎三郡吏民徙扶風界慬即遣南單于兄子優孤塗奴將兵迎之既還慬以塗奴接其家屬有勞輒授以羌侯印綬坐專擅徵下獄扺罪明年校書郎馬融上書訟慬與參按慬為度遼將軍在永初四年徙三郡民在五年參下獄在今年不得云明年融訟之也疑傳誤}
以馬賢代參領護羌校尉復以任尚為中郎將代班雄屯三輔 |{
	考異曰帝紀冬十月遣任尚屯三輔按西羌傳司馬鈞扺罪後尚乃代雄屯三輔耳復扶又翻}
懷令虞詡說尚曰|{
	說輸芮翻}
兵法弱不攻彊走不逐飛自然之埶也今虜皆馬騎日行數百里來如風雨去如絶絃以步追之埶不相及所以雖屯兵二十餘萬曠日而無功也為使君計莫如罷諸郡兵各令出錢數千二十人共市一馬以萬騎之衆逐數千之虜追尾掩截|{
	賢曰尾猶尋也予謂尾者隨其後而擊之也掩襲也截邀也}
其道自窮|{
	言虜之路自窮不能捷出而寇掠也}
便民利事大功立矣尚即上言用其計遣輕騎擊杜季貢於丁奚城破之太后聞虞詡有將帥之畧以為武都太守 |{
	考異曰詡傳曰羌寇武都太后以詡有將帥之畧遷武都太守又曰賊敗散南入益州本紀元初元年羌寇武都漢中據此似詡以元初元年為武都大守也然按西羌傳龎參抵罪後任尚屯三輔時詡猶為懷令說尚用騎兵袁紀亦云懷令虞詡說尚如范書所言又云上問何從發此計尚表之受於懷令虞詡由是知名遷武都太守以此驗之當在龎參抵罪後也}
羌衆數千遮詡於陳倉崤谷|{
	此崤谷當在陳倉縣界即今之大散關非宏農澠池縣之崤山也}
詡即停軍不進而宣言上書請兵須到當發羌聞之乃分鈔傍縣|{
	鈔楚交翻}
詡因其兵散日夜進道兼行百餘里令吏士各作兩竈日增倍之羌不敢逼或問曰孫臏減竈而君增之|{
	減竈事見二卷周顯王二十八年臏頻忍翻}
兵法日行不過三十里以戒不虞|{
	前書吉行五十里師行三十里}
而今日且二百里何也詡曰虜衆多吾兵少徐行則易為所及|{
	易以䜴翻}
速進則彼所不測虜見吾竈日增必謂郡兵來迎衆多行速必憚追我孫臏見弱|{
	見賢遍翻}
吾今示彊埶有不同故也既到郡兵不滿三千而羌衆萬餘攻圍赤亭數十日|{
	賢曰赤亭故城在今渭州襄武縣東南有赤亭水予按唐渭州漢隴西郡地漢武都唐階成州地此自是武都之赤亭非渭州之赤亭也又按郡國志武都下辨縣有赤亭即此}
詡乃令軍中彊弩勿發而濳發小弩羌以為矢力弱不能至并兵急攻詡于是使二十彊弩共射一人發無不中|{
	射而亦翻中竹仲翻}
羌大震退詡因出城奮擊多所傷殺明日悉陳其兵衆令從東郭門出北郭門入貿易衣服回轉數周羌不知其數更相恐動|{
	貿音茂更工衡翻}
詡計賊當退乃潜遣五百餘人於淺水設伏候其走路|{
	詡知賊退遇水必踏淺而度因於其處設伏以待之}
虜果大犇因掩擊大破之斬獲甚衆賊由是走散詡乃占相地埶|{
	相息亮翻}
築營壁百八十所招還流亡假賑貧民開通水運|{
	詡案行川谷自沮至下辨數十里燒石剪木開漕船道}
詡始到郡穀石千鹽石八千見戶萬三千|{
	見存之戶也見賢遍翻}
視事三年米石八十鹽石四百民增至四萬餘戶人足家給一郡遂安 十一月庚申郡國十地震 十二月武陵澧中蠻反|{
	澧中今澧州地澧音禮}
州郡討平之 己酉司徒夏勤罷|{
	夏戶雅翻}
庚戌以司空劉愷為司徒光禄勲袁敞為司空敞安之子也 前虎賁中郎將鄧宏卒|{
	宏自遭母喪去官奉朝請故曰前}
宏性儉素治歐陽尚書|{
	漢千乘歐陽生傳伏生尚書由是尚書有歐陽氏學治直之翻}
授帝禁中有司奏贈宏驃騎將軍位特進封西平侯|{
	西平縣屬汝南郡}
太后追宏雅意不加贈位衣服但賜錢千萬布萬匹兄隲等復辭不受|{
	復扶又翻下同}
詔封宏子廣德為西平侯將葬有司復奏發五營輕車騎士禮儀如霍光故事|{
	賢曰霍光薨宣帝遣太中大夫侍御史持節護喪事中二千石修莫府冢上賜玉衣梓宫便房黄腸題湊輼輬車黄屋左纛輕車材官五校士以送葬也}
太后皆不聼但白蓋雙騎門生輓送|{
	賢曰白蓋車也輓音晚騎奇寄翻}
後以帝師之重分西平之都鄉封廣德弟甫德為都鄉侯

資治通鑑卷四十九
