<!DOCTYPE html PUBLIC "-//W3C//DTD XHTML 1.0 Transitional//EN" "http://www.w3.org/TR/xhtml1/DTD/xhtml1-transitional.dtd">
<html xmlns="http://www.w3.org/1999/xhtml">
<head>
<meta http-equiv="Content-Type" content="text/html; charset=utf-8" />
<meta http-equiv="X-UA-Compatible" content="IE=Edge,chrome=1">
<title>資治通鑒_211-資治通鑑卷二百十_211-資治通鑑卷二百十</title>
<meta name="Keywords" content="資治通鑒_211-資治通鑑卷二百十_211-資治通鑑卷二百十">
<meta name="Description" content="資治通鑒_211-資治通鑑卷二百十_211-資治通鑑卷二百十">
<meta http-equiv="Cache-Control" content="no-transform" />
<meta http-equiv="Cache-Control" content="no-siteapp" />
<link href="/img/style.css" rel="stylesheet" type="text/css" />
<script src="/img/m.js?2020"></script> 
</head>
<body>
 <div class="ClassNavi">
<a  href="/24shi/">二十四史</a> | <a href="/SiKuQuanShu/">四库全书</a> | <a href="http://www.guoxuedashi.com/gjtsjc/"><font  color="#FF0000">古今图书集成</font></a> | <a href="/renwu/">历史人物</a> | <a href="/ShuoWenJieZi/"><font  color="#FF0000">说文解字</a></font> | <a href="/chengyu/">成语词典</a> | <a  target="_blank"  href="http://www.guoxuedashi.com/jgwhj/"><font  color="#FF0000">甲骨文合集</font></a> | <a href="/yzjwjc/"><font  color="#FF0000">殷周金文集成</font></a> | <a href="/xiangxingzi/"><font color="#0000FF">象形字典</font></a> | <a href="/13jing/"><font  color="#FF0000">十三经索引</font></a> | <a href="/zixing/"><font  color="#FF0000">字体转换器</font></a> | <a href="/zidian/xz/"><font color="#0000FF">篆书识别</font></a> | <a href="/jinfanyi/">近义反义词</a> | <a href="/duilian/">对联大全</a> | <a href="/jiapu/"><font  color="#0000FF">家谱族谱查询</font></a> | <a href="http://www.guoxuemi.com/hafo/" target="_blank" ><font color="#FF0000">哈佛古籍</font></a> 
</div>

 <!-- 头部导航开始 -->
<div class="w1180 head clearfix">
  <div class="head_logo l"><a title="国学大师官网" href="http://www.guoxuedashi.com" target="_blank"></a></div>
  <div class="head_sr l">
  <div id="head1">
  
  <a href="http://www.guoxuedashi.com/zidian/bujian/" target="_blank" ><img src="http://www.guoxuedashi.com/img/top1.gif" width="88" height="60" border="0" title="部件查字,支持20万汉字"></a>


<a href="http://www.guoxuedashi.com/help/yingpan.php" target="_blank"><img src="http://www.guoxuedashi.com/img/top230.gif" width="600" height="62" border="0" ></a>


  </div>
  <div id="head3"><a href="javascript:" onClick="javascript:window.external.AddFavorite(window.location.href,document.title);">添加收藏</a>
  <br><a href="/help/setie.php">搜索引擎</a>
  <br><a href="/help/zanzhu.php">赞助本站</a></div>
  <div id="head2">
 <a href="http://www.guoxuemi.com/" target="_blank"><img src="http://www.guoxuedashi.com/img/guoxuemi.gif" width="95" height="62" border="0" style="margin-left:2px;" title="国学迷"></a>
  

  </div>
</div>
  <div class="clear"></div>
  <div class="head_nav">
  <p><a href="/">首页</a> | <a href="/ShuKu/">国学书库</a> | <a href="/guji/">影印古籍</a> | <a href="/shici/">诗词宝典</a> | <a   href="/SiKuQuanShu/gxjx.php">精选</a> <b>|</b> <a href="/zidian/">汉语字典</a> | <a href="/hydcd/">汉语词典</a> | <a href="http://www.guoxuedashi.com/zidian/bujian/"><font  color="#CC0066">部件查字</font></a> | <a href="http://www.sfds.cn/"><font  color="#CC0066">书法大师</font></a> | <a href="/jgwhj/">甲骨文</a> <b>|</b> <a href="/b/4/"><font  color="#CC0066">解密</font></a> | <a href="/renwu/">历史人物</a> | <a href="/diangu/">历史典故</a> | <a href="/xingshi/">姓氏</a> | <a href="/minzu/">民族</a> <b>|</b> <a href="/mz/"><font  color="#CC0066">世界名著</font></a> | <a href="/download/">软件下载</a>
</p>
<p><a href="/b/"><font  color="#CC0066">历史</font></a> | <a href="http://skqs.guoxuedashi.com/" target="_blank">四库全书</a> |  <a href="http://www.guoxuedashi.com/search/" target="_blank"><font  color="#CC0066">全文检索</font></a> | <a href="http://www.guoxuedashi.com/shumu/">古籍书目</a> | <a   href="/24shi/">正史</a> <b>|</b> <a href="/chengyu/">成语词典</a> | <a href="/kangxi/" title="康熙字典">康熙字典</a> | <a href="/ShuoWenJieZi/">说文解字</a> | <a href="/zixing/yanbian/">字形演变</a> | <a href="/yzjwjc/">金 文</a> <b>|</b>  <a href="/shijian/nian-hao/">年号</a> | <a href="/diming/">历史地名</a> | <a href="/shijian/">历史事件</a> | <a href="/guanzhi/">官职</a> | <a href="/lishi/">知识</a> <b>|</b> <a href="/zhongyi/">中医中药</a> | <a href="http://www.guoxuedashi.com/forum/">留言反馈</a>
</p>
  </div>
</div>
<!-- 头部导航END --> 
<!-- 内容区开始 --> 
<div class="w1180 clearfix">
  <div class="info l">
   
<div class="clearfix" style="background:#f5faff;">
<script src='http://www.guoxuedashi.com/img/headersou.js'></script>

</div>
  <div class="info_tree"><a href="http://www.guoxuedashi.com">首页</a> > <a href="/SiKuQuanShu/fanti/">四库全书</a>
 > <h1>资治通鉴</h1> <!--         下载:【右键另存为】即可 --></div>
  <div class="info_content zj clearfix">
  
<div class="info_txt clearfix" id="show">
<center style="font-size:24px;">211-資治通鑑卷二百十</center>
    資治通鑑卷二百十   宋 司馬光 撰<br />
<br />
  胡三省 音注<br />
<br />
  唐紀二十六【起上章閹茂八月盡昭陽赤奮若凡三年有奇】<br />
<br />
  睿宗玄真大聖大興孝皇帝下<br />
<br />
  景靈元年八月庚寅往巽第按問【此承上卷洛陽縣官微聞其謀】重福奄至縣官馳出白留守羣官皆逃匿洛州長史崔日知獨帥衆討之【重直龍翻守式又翻長知兩翻帥讀曰率】留臺侍御史李邕遇重福於天津橋從者已數百人馳至屯營【從才用翻即洛城左右屯營也】告之曰譙王得罪先帝【言重福得罪中宗居之均州】今無故入都此必為亂君等宜立功取富貴又告皇城【東都皇城也】使閉諸門重福先趣左右屯營營中射之【趣七喻翻射而亦翻】矢如雨下乃還趣左掖門【還從宣翻掖音亦】欲取留守兵見門閉大怒命焚之火未及然左屯營兵出逼之重福窘迫策馬出上東【然與燃同窘渠隕翻上東洛城上東門也東面北來第一門】逃匿山谷明日留守大出兵搜捕重福赴漕渠溺死 【考異曰睿宗實録舊本紀皆云癸巳重福反今從太上皇實録】日知日用之從父兄也【從才用翻】以功拜東都留守鄭愔貌醜多須既敗梳髻著婦人服匿車中【愔於今翻著陟略翻】擒獲被鞫股慄不能對【被皮義翻】張靈均神氣自若顧愔曰吾與此人舉事宜其敗也與愔皆斬於東都市初愔附來俊臣得進俊臣誅附張易之易之誅附韋氏韋氏敗又附譙王重福竟坐族誅【史言張靈均雖幸禍好亂之人猶能臨死不變鄭愔者反覆於羣憸之間冒利不顧而畏死乃爾烏足以權大事乎】嚴善思免死流静州【嶺南之靜州貞觀中已改為富州此靜州屬劒南儀鳳元年以悉州之悉唐縣置南和州武后天授二年更名靜州嚴善思免死而流此夙依嬖倖今從亂又得以偷生】 萬騎恃討諸韋之功多暴横【騎奇寄翻横戶孟翻】長安中苦之詔並除外官又停以戶奴為萬騎【戶奴為萬騎盖必起於永昌之後】更置飛騎隸左右羽林【更工衡翻】姚元之宋璟及御史大夫畢構上言先朝斜封官悉<br />
<br />
  宜停廢【璟瑟影翻上時掌翻朝直遥翻】上從之癸巳罷斜封官凡數千人【斜封官見上卷中宗景龍三年】 刑部尚書同中書門下三品裴談貶蒲州刺史【舊志蒲州京師東北三百二十四里尚辰羊翻】 贈蘇安諫議大夫【蘇安恒死見二百八卷中宗景龍元年恒戶登翻】 九月辛未以太子少師致仕唐休璟為朔方道大總管【少始照翻】 冬十月甲申禮儀使姚元之宋璟奏【唐世凡有國恤皆以宰相為禮儀使掌山陵祔廟等事使疏吏翻】大行皇帝神主應祔太廟請遷義宗神主於東都别立廟從之【義宗祔廟見二百八卷中宗神龍元年】 乙未追復天后尊號為大聖天后 丁酉以幽州鎮守經略節度大使薛訥為左武衛大將軍兼幽州都督節度使之名自訥始【使疏吏翻 考異曰統紀景雲二年四月以賀拔延秀為河西節度使節度之名自此始會要云景雲二年賀拔延嗣為凉州都督充河西節度始有節度之號又云范陽節度自先天二年始除甄道一新表景雲元年置河西諸軍州節度支度營田大使按訥先己為節度大使則節度之名不始於延嗣也今從太上皇實録 是後天寶緣邉御戎之地置八節度使其任愈重受命之日賜雙旌雙節得以專制軍事行則建節樹六纛入境州縣築節楼迎以鼔角衙仗居前旌幢居中大將鳴珂金鉦鼔角居後州縣齎印迎於道左又唐之制有節度大使副大使節度使其親王領節度大使而不出閣則在鎮知節度者為副大使其異姓為節度使者有節度副使至後唐開成二年七月勑頃因本朝親王遥領方鎮其在鎮者遂云副大使知節度事但年代已深相沿未改今天下侯伯並正節旄其未落副大使者祇言節度使】 太平公主以太子年少意頗易之既而憚其英武欲更擇闇弱者立之以久其權數為流言云太子非長不當立【少詩照翻易以豉翻數所角翻】己亥制戒諭中外以息浮議公主每覘伺太子所為纎介必聞於上【覘丑亷翻又丑艶翻伺相吏翻】太子左右亦往往為公主耳目太子深不自安【為誅太平公主及其支黨張本】 諡故太子重俊曰節愍太府少卿萬年韋湊上書以為賞罰所不加者則考行立諡以褒貶之【上時掌翻行下孟翻】故太子重俊與李多祚等稱兵入宫中宗登玄武門以避之太子據鞍督兵自若及其徒倒戈多祚等死太子方逃竄曏使宿衛不守其為禍也胡可忍言明日中宗雨泣【雨泣者淚下如雨也】謂供奉官曰【中書門下兩省官謂之供奉官】幾不與卿等相見其危如此【幾居希翻】今聖朝禮葬諡為節愍臣竊惑之夫臣子之禮過廟必下【下遐嫁翻】過位必趨漢成帝之為太子不敢絶馳道【漢成帝為太子初居桂官元帝嘗急召之太子出龍楼門不敢絶馳道西至直城門得絶乃度還入作室門上遲之問其故以狀對乃著令太子得絶馳道】而重俊稱兵宫内跨馬御前無禮甚矣若以其誅武三思父子而嘉之則興兵以誅姦臣而尊君父可也今欲自取之是與三思競為逆也又足嘉乎若以其欲廢韋氏而嘉之則韋氏於時逆狀未彰大義未絶苟無中宗之命而廢之是脅父廢母也庸可乎漢戻太子困於江充之讒發忿殺充雖興兵交戰非圍逼君父也兵敗而死【事見二十二卷武帝征和二年】及其孫為天子始得改葬猶諡曰戾【見二十四卷宣帝本始元年】况重俊可諡之曰節愍乎臣恐後之亂臣賊子得引以為比開悖逆之原非所以彰善癉惡也【彰明也癉病也明其為善病其為惡者也癉丁但翻】請改其諡多祚等從重俊興兵不為無罪陛下今宥之可也名之為雪亦所未安上甚然其言而執政以為制命已行不為追改【為于偽翻】但停多祚等贈官而已 十一月戊申朔以姚元之為中書令 己酉葬孝和皇帝于定陵【定陵在雍州富平縣西北十五里】廟號中宗朝議以韋后有罪不應祔葬追諡故英王妃趙氏曰和思順聖皇后求其瘞莫有知者【妃死見二百二卷高宗上元二年】乃以禕衣招䰟【唐制皇后之服曰褘衣鞠衣䄠衣禕衣者受册助祭朝會大事之服也深青織成為之畫翬赤質五色十二等素紗中單黼領朱羅縠褾襈蔽膝隨裳色以緅領為緣用翟為章三等青衣革帶大帶隨衣色禆約紐佩綬如天子青韈舄加金舄】覆以夷衾【覆敷又翻】祔葬定陵 壬子侍中韋安石罷為太子少保左僕射同中書門下三品蘇瓌罷為少傳 甲寅追復裴炎官爵初裴伷先自嶺南逃歸復杖一百徙北庭【伷讀曰胄復扶又翻】至徙所殖貨任俠常遣客詗都下事武后之誅流人也【裴炎死伷先流嶺南見二百三卷武后光宅 元年誅流人見二百五卷長夀二年詗休正翻】伷先先知之逃奔胡中北庭都護追獲囚之以聞使者至流人盡死伷先以待報未殺既而武后下制安撫流人有未死者悉放還伷先由是得歸至是求炎後獨伷先在拜詹事丞【詹事丞正六品上掌判詹事府事】 壬戌追復王同皎官爵【王同皎死見二百八卷中宗神龍二年】庚午許文貞公蘇瓌薨制起復其子頲為工部侍郎頲固辭【頲他鼎翻】上使李日知諭旨日知終坐不言而還【坐徂卧翻】奏曰臣見其哀毁不忍發言恐其隕絶上乃聽其終制 十二月癸未上以二女西城隆昌公主為女官以資天皇太后之福仍欲於城西造觀【觀古玩翻道士所居曰觀】諫議大夫甯原悌上言以為先朝悖逆庶人以愛女驕盈而及禍新城宜都以庶孽抑損而獲全【新城公主下嫁武延暉宜城公主下嫁裴巽皆中宗女】又釋道二家皆以清淨為本不當廣營寺觀勞人費財梁武帝致敗於前先帝取災於後殷鑒不遠今二公主入道將為之置觀【觀古玩翻為于偽翻】不宜過為崇麗取謗四方又先朝所親狎諸僧尚在左右宜加屏斥【朝直遥翻屛卑郢翻】上覽而善之 宦者閻興貴以事屬長安令李朝隱【屬之欲翻朝直遥翻下同】朝隐繫於獄上聞之召見朝隐勞之曰卿為赤縣令能如此朕復何憂【勞力到翻復扶又翻下無復同】因御承天門集百官及諸州朝集使宣示以朝隐所為且下制稱宦官遇寛柔之代必弄威權朕覽前載每所歎息能副朕意實在斯人可加一階為太中大夫賜中上考及絹百匹 壬辰奚霫犯塞掠漁陽雍奴出盧龍塞而去【漁陽縣本屬幽州中宗神龍元年分屬營州雍奴縣漢以來屬漁陽郡隋屬涿郡唐屬幽州盧龍漢肥如縣也屬遼西郡隋開皇十八年更名盧龍屬北平郡唐帶平州霫而立翻】 幽州都督薛訥追擊之弗克 舊制三品以上官冊授五品以上制授六品以下敇授【唐王言之制有七一曰冊書二曰制書三曰慰勞制書四曰發勑五曰勑旨六曰論事敇書七曰敇牒】皆委尚書省奏擬文屬吏部武屬兵部尚書曰中銓侍郎曰東西銓【所謂三銓也】中宗之末嬖倖用事選舉混淆無復綱紀至是以宋璟為吏部尚書李乂盧從愿為侍郎皆不畏彊禦請謁路絶集者萬餘人留者三銓不過二千人服其公以姚元之為兵部尚書陸象先盧懷慎為侍郎武選亦治【選須絹翻治直吏翻】從愿承慶之族子【盧承慶見三百卷高宗顯慶四年】象先元方之子也【陸元方見二百五卷天后證聖元年】 侍御史藳城倪若水【藳城縣前漢屬真定國後漢以來屬鉅鹿郡唐屬恒州】奏彈國子祭酒祝欽明司業郭山惲亂常改作希旨病君【謂郊祀請以韋后亞獻也】於是左授欽明饒州刺史山惲括州長史【舊志饒州京師東南三千二百六十三里括州後為處州京師東南四千二百七十八里】 侍御史楊孚彈糾不避權貴權貴毁之上曰鷹狡兎須急救之不爾必反為所噬御史繩姦慝亦然苟非人主保衛之則亦為奸慝所噬矣孚隋文帝之姪孫也 置河西節度支度營田等使領凉甘肅伊瓜沙西七州治凉州【唐制凡天下邉軍皆有支度使以計軍資糧仗之用節度不兼支度者支度自為一司其兼支度者則節度使自支度凡邉防鎮守轉運不給則開置屯田以益軍儲於是有營田使使疏吏翻度徒洛翻】 姚州羣蠻先附吐蕃攝監察御史李知古請發兵擊之既降【降戶江翻】又請築城列置州縣重稅之黄門侍郎徐堅以為不可【句斷】不從知古發劒南兵築城因欲誅其豪傑掠子女為奴婢羣蠻怨怒蠻酋傍名引吐蕃攻知古殺之以其尸祭天由是姚嶲路絶連年不通【酋慈由翻嶲音髓】安西都護張玄表侵掠吐蕃北境吐蕃雖怨而未絶和親乃賂鄯州都督楊矩請河西九曲之地以為公主湯沐邑矩奏與之【九曲者去積石軍三百里水甘草良宜畜牧盖即漢大小榆谷之地吐蕃置洪濟大漠門等城以守之史為楊矩後悔愳自殺張本鄯時戰翻又音善】<br />
<br />
  二年春正月癸丑突厥可汗默啜遣使請和許之【厥九勿翻可從刊入聲汗音寒使疏吏翻】 己未以太僕卿郭元振中書侍郎張說並同平章事【說讀曰悦】 以温王重茂為襄王充集州刺史遣中郎將將兵五百就防之【舊志集州京師西南一千四百二十五里將即亮翻】 乙丑追立妃劉氏曰肅明皇后陵曰惠陵德妃竇氏曰昭成皇后陵曰靖陵皆招䰟葬於東都城南【二妃死在二百五卷武后長夀二年】立廟京師號儀坤廟【會要儀坤廟在親仁里】竇氏太子之母也 太平公主與益州長史竇懷貞等結為朋黨欲以危太子使其壻唐晙邀韋安石至其第【晙子峻翻】安石固辭不往上嘗密召安石謂曰聞朝廷皆傾心東宫卿宜察之對曰陛下安得亡國之言此必太平之謀耳太子有功於社稷仁明孝友天下所知願陛下無惑讒言上瞿然曰【瞿俱遇翻瞿然驚視之貌】朕知之矣卿勿言時公主在簾下竊聽之以飛語陷安石欲收按之賴郭元振救之得免公王又嘗乘輦邀宰相於光範門内【唐六典曰宣政殿前西廊曰月華門門西中書省省西南北街南直昭慶門出光範門韓愈伏光範門下上宰相書即此】諷以易置東宮衆皆失色宋璟抗言曰東宫有大功於天下真宗廟社稷之主公主奈何忽有此議璟與姚元之密言於上曰宋王陛下之元子王高宗之長孫【王守禮章懷太子賢之子長知兩翻】太平公主交構其間將使東宫不安請出宋王及王皆為刺史罷岐薛二王左右羽林使為左右率以事太子【韋氏初平二王領羽林東宮五率分為左右十率此指左右衛率】太平公主請與武攸暨皆於東都安置上曰朕更無兄弟惟太平一妹豈可遠置東都諸王惟卿所處【處昌呂翻】乃先下制云諸王駙馬自今毋得典禁兵見任者皆改它官【見賢遍翻】頃之上謂侍臣曰術者言五日中當有急兵入宫卿等為朕備之【為于季翻下為陛同】張說曰此必讒人欲離間東宫【間古莧翻】願陛下使太子監國【監古銜翻】則流言自息矣姚元之曰張說所言社稷之至計也上說【說與悦同】二月丙子朔以宋王成器為同州刺史王守禮為州刺史【舊志同州京師東北二百五十五里州京師西北四百九十三里】左羽林大將軍岐王隆範為左衛率右羽林大將軍薛王隆業為右衛率太平公主蒲州安置丁丑命太子監國六品以下除官及徒罪以下並取太子處分【處昌呂翻分扶問翻】 殿中侍御史崔蒞太子中允薛昭素言於上曰斜封官皆先帝所除恩命已布【斜封官見上卷中宗景龍二年】姚元之等建議一朝盡奪之彰先帝之過為陛下招怨【為于偽翻】今衆口沸騰徧於海内恐生非常之變太平公主亦言之上以為然戊寅制諸緣斜封别敇授官先停任者並量材叙用【量音良 考異曰朝野僉載云宋璟畢構出後見鬼人彭君卿受斜封人賄奏云孝和怒曰我與人官何因奪却於是斜封皆復舊職今不取】 太平公主聞姚元之宋璟之謀大怒以讓太子太子愳奏元之璟離間姑兄【姑謂太平公主兄謂宋王王間古莧翻】請從極法甲申貶元之為申州刺史璟為楚州刺史【舊志申州至京師一千七百九十六里楚州京師東南二千五百一里】丙戍宋王王亦寢刺史之命 中書舍人參知機務劉幽求罷為戶部尚書以太子少保韋安石為侍中安石與李日知代姚宋為政自是綱紀紊亂復如景龍之世矣【紊音問復扶又翻又如字】前右率府鎧曹參軍柳澤上疏以為斜封官皆因僕妾汲引豈出孝和之意【中宗諡孝和皇帝率所律翻上時掌翻疏所去翻】陛下一切黜之天下莫不稱明一旦忽盡收叙善惡不定反覆相攻何陛下政令之不一也議者咸稱太平公主令胡僧慧範曲引此曹誑誤陛下【誑居况翻】臣恐積小成大為禍不細上弗聽澤亨之孫也【柳亨事隋為王屋長歸高祖以女孫竇氏妻之歷事太宗位至檢校岐州刺史】 左右萬騎與左右羽林為北門四軍使葛福順等將之【騎奇寄翻將即亮翻又音如字】 三月以宋王成器女為金山公主許嫁突厥默啜【厥九勿翻啜叱劣翻】 夏四月甲申宋王成器讓司徒許之以為太子賓客以韋安石為中書令 上召羣臣三品以上謂曰朕素懷澹泊不以萬乘為貴【澹徒覽翻乘繩證翻】曩為皇嗣又為太弟皆辭不處【為皇嗣見二百四卷天授元年辭太弟見二百八卷神龍元年嗣祥吏翻處昌呂翻】今欲傳位太子何如羣臣莫對太子使右庶子李景伯固辭不許殿中侍御史和逢堯附太平公主言於上曰陛下春秋未高方為四海所依仰豈得遽爾上乃止戊子制凡政事皆取太子處分【處昌呂翻分扶問翻】其軍旅死刑及五品已上除授皆先與太子議之然後以聞辛卯以李日知守侍中 壬寅赦天下 五月太子<br />
<br />
  請讓位於宋王成器不許請召太平公主還京師許之庚戌制則天皇后父母墳仍舊為昊陵順陵量置官<br />
<br />
  屬【廢武氏二陵見上卷元年量音良】太平公主為武攸暨請之也【為于偽翻下各為同】 辛酉更以西城為金仙公主隆昌為玉真公主各為之造觀【金仙玉真二觀皆造於京城内輔興坊玉真觀本竇誕舊宅與金仙觀相對更工衡翻】逼奪民居甚多用功數百萬右散騎常侍魏知古黄門侍郎李乂諫不聽【散悉亶翻騎奇寄翻】 壬戌殿中監竇懷貞為御史大夫同平章事 僧慧範恃太平公主埶逼奪民產御史大夫薛謙光與殿中侍御史慕容珣奏彈之公主訴於上出謙光為岐州刺史 【考異曰統紀曰監察御史慕容珣奏彈西明寺僧慧範以其通宫人張氏張即太平公主乳母也侵奪百姓上以為御史當不避豪貴見公主出居蒲州乃敢弹射在日不言狀涉離間骨肉遂貶為密州員外司馬今從舊傳】 時遣使按察十道【太宗貞觀十八年遣十七道廵察武后垂供初亦嘗遣九道廵察天授二年又遣十道存撫使至是分為十道按察使以亷按州郡二周年一替使疏吏翻】議者以山南所部闊遠乃分為東西道又分隴右為河西道六月壬午又分天下置汴齊兖魏冀并蒲鄜涇秦益緜遂荆岐通梁襄揚安閩越洪潭二十四都督【武德元年改蜀郡為益州綿州漢涪縣地江左置巴西郡西魏曰潼州隋開皇改緜州大業初廢州為金山郡唐武德初復曰緜州又武德二年置閩州於閩縣開元十三年更閩州為福州鄜音膚】各糺察所部刺史以下善惡惟洛及近畿州不隸都督府【雍華同商岐為京畿洛汝為都畿】太子右庶子李景伯舍人盧俌等上言【俌音甫】都督專殺生之柄權任太重或用非其人為害不細今御史秩卑望重以時廵察奸宄自禁【宄音軌】其後竟罷都督但置十道按察使而已 秋七月癸巳追復上官昭容諡曰惠文【追復其昭容之職而加之以諡】 乙卯以高祖故宅枯柿復生赦天下【時詔以興聖寺是高祖舊宅有柿樹天授中枯死至是重生大赦天下復扶又翻又如字】 己巳以右御史大夫解琬為朔方大總管琬考按三城戍兵【三城三受降城也解戶賈翻】奏減十萬人 庚午以中書令韋安石為左僕射兼太子賓客同中書門下三品太平公主以安石不附己故崇以虚名實去其權也【去羌呂翻】 九月庚辰以竇懷貞為侍中懷貞每退朝必詣太平公主第【朝直遥翻】時脩金仙玉真二觀羣臣多諫懷貞獨勸成之身自督役時人謂懷貞前為皇后阿㸙【事見上卷中宗景龍二年㸙正奢翻】今為公主邑司【唐公主有邑司令丞掌其主家財貨出入田園徵封之事 考異曰睿宗實録云乙卯御史大夫竇懷貞為侍中太上皇實録云庚辰御史大夫同中書門下三品竇懷貞為侍中知金仙玉真公主邑司事舊紀己卯懷貞為侍中新紀新表乙亥懷貞守侍中按是月癸酉朔無乙卯又懷貞以自督修二觀之故時人語曰竇懷貞前為國㸙今為公主邑丞非真知邑司也今從舊紀】 冬十月甲辰上御承天門引韋安石郭元振竇懷貞李日知張說宣制責以政教多闕水旱為災府庫益竭吏日滋雖朕之薄德亦輔佐非才安石可左僕射東都留守【守手又翻】元振可吏部尚書懷貞可左御史大夫日知可戶部尚書說可左丞並罷政事以吏部尚書劉幽求為侍中右散騎常侍魏知古為左散騎常侍太子詹事崔湜為中書侍郎並同中書門下三品中書侍郎陸象先同平章事皆太平公主之志也象先清凈寡欲言論高遠為時人所重湜私侍太平公主公主欲引以為相【相息亮翻】湜請與象先同升公主不可湜曰然則湜亦不敢當公主乃為之幷言於上【為于偽翻】上不欲用湜公主涕泣以請乃從之 【考異曰朝野僉載云湜妻美并二女皆得幸於太子時人牓之曰託庸才於主第進豔婦於春宮今不取】 右補闕辛替否上疏以為自古失道破國亡家者口說不如身逢耳聞不如目覩臣請以陛下所目覩者言之太宗皇帝陛下之祖也撥亂返正【用太史公撥亂世返之正語意】開基立極官不虚授財無枉費不多造寺觀而有福不多度僧尼而無災【觀古玩翻下同尼女夷翻】天地垂祐風雨時若【若順也】粟帛充溢蠻夷率服享國久長名高萬古陛下何不取而法之中宗皇帝陛下之兄弃祖宗之業狥女子之意無能而禄者數千人無功而封者百餘家造寺不止費財貨者數百億度人無窮免租庸者數十萬所出日滋所入日寡奪百姓口中之食以養貪殘剝萬人體上之衣以塗土木于是人怨神怒衆叛親離水旱並臻公私俱罄享國不永禍及其身陛下何不懲而改之自頃以來水旱相繼兼以霜蝗人無所食未聞賑恤【賑津忍翻】而為二女造觀用錢百萬餘緡【指言金仙玉真二觀為于偽翻】陛下豈可不計當時府庫之蓄積有幾中外之經費有幾而輕用百餘萬緡以供無用之役乎陛下族韋氏之家而不去韋氏之惡【去羌呂翻】忍弃太宗之法不忍弃中宗之政乎且陛下與太子當韋氏用事之時日夕憂危切齒於羣兇【羣兇謂韋温宗楚客等】今幸而除之乃不改其所為臣恐復有切齒於陛下者也然則陛下又何惡於羣兇而誅之【復扶又翻惡烏路翻】昔先帝之憐悖逆也【帝追廢安樂公主為悖逆庶人故稱之悖蒲内翻又蒲沒翻】宗晉卿為之造第趙履温為之葺園【為于偽翻】殫國財竭人力第成不暇居園成不暇遊而身為戮沒今之造觀崇侈者必非陛下公主之本意殆有宗趙之徒從而勸之不可不察也陛下不停斯役臣恐人之愁怨不減前朝之時人人知其禍敗而口不敢言言則刑戮隨之矣韋月將燕欽融之徒先朝誅之陛下賞之豈非陛下知直言之有益於國乎臣今所言亦先朝之直也【朝直遥翻下同】惟陛下察之上雖不能從而嘉其切直 御史中丞和逢堯攝鴻臚卿使于突厥【臚陵如翻使疏吏翻】說默啜曰處密堅昆聞可汗結昏於唐皆當歸附可汗何不襲唐冠帶使諸胡知之豈不美哉默啜許諾明日襆頭衣紫衫南向再拜稱臣【襆頭紫衫唐三品已上之服也襆頭起於後周便武事者也太宗時馬周上議以禮無服衫之文請加襴䄂褾襈說輪芮翻襆防玉翻衣於既翻】遣其子楊我支及國相隨逢堯入朝十一月戊寅至京師逢堯以奉使功遷戶部侍郎 壬辰令天下百姓二十五入軍五十五免 十二月癸卯以興昔亡可汗阿史那獻馬招慰十姓使 上召天台山道士司馬承禎【臨海記天台山超然秀出山有八重視之如一高一萬八千丈周囘八百里又有飛泉垂流千仞時屬台州唐興縣界我朝太祖建隆元年始改唐興縣為天台縣其山在今縣西二十餘里】問以隂陽數術對曰道者損之又損以至于無為安肯勞心以學術數乎上曰理身無為則高矣如理國何對曰國猶身也順物自然而心無所私則天下理矣上歎曰廣成之言無以過也【廣成子居崆峒之上黄帝立下風而問道】承禎固請還山上許之尚書左丞盧藏用指終南山【程大昌曰終南山横亘關中南面西起秦隴東徹藍田凡維岐郿鄠長安萬年相去且八百里而連綿峙據其南者皆此山也毛公曰終南周之名山中南也中南即終南也關中記曰言居天之中都之南也】謂承禎曰此中大有佳處何必天台承禎曰以愚觀之此乃仕宦之捷徑耳藏用嘗隱終南則天時徵為左拾遺故承禎言之<br />
<br />
  玄宗至道大聖大明孝皇帝上之上<br />
<br />
  【諱隆基睿宗第三子也此諡廣德元年所定】<br />
<br />
  先天元年【是年八月方改元先天】春正月 【考異曰新紀表壬辰以陸象先同中書門下三品太上皇睿宗實録舊紀皆無之不知新書何出今不取】辛巳睿宗祀南郊初因諫議大夫賈曾議合祭天地【歐陽修曰古者祭天於圓丘在國之南祭地於澤中之方丘在國之北所以順隂陽因高下以事天地以其類也而後世有合祭之文則天天冊萬歲元年親享南郊始合祭天地至是曾議曰有虞氏禘黄帝而郊嚳夏后氏禘黄帝而郊鯀郊之與廟皆有禘也禘於廟則祖宗合食於太祖禘於郊則地祇羣望皆合食於圓丘以始祖配享盖有事之大祭非常祭也三輔故事祭於圓丘上帝后土位皆南面則漢嘗合祭也時皆以曾言為然】曾言忠之子也【言忠見二百一卷高宗總章元年】 戊子幸滻東【水經注覇水北歷藍田川又左合滻水滻水逕長樂坡西是後韋堅引為廣運潭在京師苑城之東此地又在滻水之東】耕藉田【藉在亦翻】 己丑赦天下改元太極 乙未上御安福門宴突厥楊我支以金山公主示之既而會上傳位昏竟不成 以左御史大夫竇懷貞戶部尚書岑羲並同中書門下三品 二月【考異曰太上皇實録云命皇太子送金山公主往并州令幽州都督裴懷古節度内發三萬兵赴黑山道并】<br />
<br />
  【州長史薛訥節度内發四萬兵於汾州迎皇太子右御史大夫朔方大總管解琬節度内發二萬兵赴單于道太子既親征諸軍一事以上並取處分按以軍法從事它書皆無此事按太子送公主與突厥和親安用九萬兵又豈得謂之親征今不取】辛酉廢右御史臺【武后光宅元年改御史臺為肅政臺分左右神龍元年為左右御史臺】 蒲州刺史蕭至忠自託於太平公主公主引為刑部尚書 【考異曰舊傳及劉餗小說皆云自晉州刺史入為尚書今從太上皇睿宗録】華州刺史蔣欽緒其妹夫也謂之曰如子之才何憂不逹勿為非分妄求【分扶問翻】至忠不應欽緒退歎曰九代卿族一舉滅之可哀也哉【引左傳衛太叔儀之言至忠蕭德言之曾孫故云然】至忠素有雅望嘗自公主第門出遇宋璟璟曰非所望於蕭君也至忠笑曰善乎宋生之言遽策馬而去 幽州大都督薛訥鎮幽州二十餘年【按武后聖歷元年薛訥方自藍田令擢為安東道經略】吏民安之未嘗舉兵出塞虜亦不敢犯與燕州刺史李璡有隙【武德六年自營州遷燕州於幽州城中燕因肩翻璡將鄰翻又即刃翻】璡毁之於劉幽求幽求薦左羽林將軍孫佺代之【佺此緣翻】三月丁丑以佺為幽州大都督徙訥為并州長史 夏五月益州獠反【獠魯皓翻】 戊寅上祭北郊 辛巳赦天下改元延和 六月丁未右散騎常侍武攸暨卒【卒子恤翻】追封定王 上以節愍太子之亂岑羲有保護之功【節愍之難冉祖雍誣帝及太平與太子連謀賴羲與蕭至忠保護得免】癸丑以羲為侍中 庚申幽州大都督孫佺與奚酋李大酺戰于泠陘【貞觀中奚酋可度者内附賜姓李後遂以李為姓酋慈由翻酺音蒲陘音刑 考異曰上皇録云甲子今從睿宗録】全軍覆沒是時佺帥左驍衛將軍李楷洛左威衛將軍周以悌發兵二萬騎八千分為三軍以襲奚契丹【帥讀曰率驍堅堯翻騎奇寄翻契欺訖翻又音喫】將軍烏可利諫曰道險而天熱懸軍遠襲往必敗佺曰薛訥在邉積年竟不能為國家復營州【營州陷見二百五卷武后萬歲通天元年為于偽翻】今乘其無備往必有功使楷洛將騎四千前驅遇奚騎八千楷洛戰不利佺怯懦不敢救【將即亮翻懦奴臥翻又奴亂翻】引兵欲還虜乘之唐兵大敗佺阻山為方陳以自固【陳讀曰陣】大酺使謂佺曰朝廷既與我和親今大軍何為而來佺曰吾奉敇來招慰耳楷洛不禀節度輒與汝戰請斬以謝大酺曰若然國信安在佺悉歛軍中帛得萬餘段并紫袍金帶魚袋以贈之【高宗永徽二年在京文武職事官五品已上並給隨身魚袋天后垂拱二年諸州都督並凖京官帶魚唐六典曰隨身魚符所以明貴賤應徵召其制左二右一太子以玉親王以金庶官以銅皆題云某位姓名並以袋盛其袋三品已上飾以金五品已上飾以銀】大酺曰請將軍南還勿相驚擾將士懼無復部伍【復扶又翻又如字】虜追擊之士卒皆潰佺以悌為虜所擒獻於突厥默啜皆殺之楷洛可利脱歸 秋七月彗星出西方經軒轅入太微至于大角 有相者謂同中書門下三品竇懷貞曰公有刑厄【相息亮翻】懷貞懼請解官為安國寺奴【雍録曰安國寺在朱雀街東第四街之長樂坊唐會要景雲元年勑捨潜龍舊宅為寺便以本封安國為名】敕聽解官乙亥復以懷貞為左僕射兼御史大夫平章軍國重事【復扶又翻】 太平公主使術者言於上曰彗所以除舊布新又帝座及心前星皆有變【帝座在中宫華盖之下心三星中星為明堂天子位前星為太子慧祥歲翻又音歲又音遂】皇太子當為天子上曰傳德避災吾志决矣太平公主及其黨皆力諫以為不可上曰中宗之時羣姦用事天變屢臻朕時請中宗擇賢子立之以應災異中宗不悦朕憂恐數日不食豈可在彼則能勸之在己則不能邪太子聞之馳入見【見賢遍翻】自投於地叩頭請曰臣以微功不次為嗣懼不克堪未審陛下遽以大位傳之何也上曰社稷所以再安吾之所以得天下皆汝力也今帝座有災故以授汝轉禍為福汝何疑邪太子固辭上曰汝為孝子何必待柩前然後即位邪【柩音舊】太子流涕而出壬辰制傳位於太子太子上表固辭【上時掌翻】太平公主勸上雖傳位猶宜自總大政上乃謂太子曰汝以天下事重欲朕兼理之邪昔舜禪禹猶親廵狩【舜既禪禹南廵狩而崩於蒼梧引此為據也】朕雖傳位豈忘家國其軍國大事當兼省之【省悉景翻 考異曰太上皇録全以為上皇之意睿宗錄云太子既為太平公主所構或唯遣皇帝知三品以下除授及徒罪其軍國大務并重刑獄上仍兼省之五日一受朝於太極殿今兩取之】 八月庚子玄宗即位尊睿宗為太上皇上皇自稱曰朕命曰誥五日一受朝於太極殿皇帝自稱曰予命曰制敕日受朝於武德殿【朝直遥翻】三品以上除授及大刑政决於上皇餘皆决於皇帝壬寅上大聖天后尊號曰聖帝天后 甲辰赦天下改元 乙巳於鄚州北置渤海軍【莫縣自漢以來屬涿郡唐屬瀛州景雲二年分置鄚州開元十三年復單用莫字】恒定州境置恒陽軍【杜祐曰恒陽軍在恒州城東恒戶登翻】媯蔚州境置懷柔軍屯兵五萬【媯居為翻蔚紆勿翻】 丙午立妃王氏為皇后以后父仁皎為太僕卿仁皎下邽人也戊申立皇子許昌王嗣直為郯王真定王嗣謙為郢王 以劉幽求為右僕射同中書門下三品魏知古為侍中崔湜為檢校中書令 初河内人王琚預於王同皎之謀【謂中宗神龍元年王同皎謀殺武三思也】亡命傭書於江都上之為太子也琚還長安選補諸暨主簿【諸暨越王允常故都也自漢以下為縣屬會稽】過謝太子琚至廷中故徐行高視宦者曰殿下在簾内琚曰何謂殿下當今獨有太平公主耳【用范睢故智為此言以激發太子】太子遽召見與語琚曰韋庶人弑逆人心不服誅之易耳【易以豉翻】太平公主武后之子凶猾無比大臣多為之用琚竊憂之太子引與同榻坐泣曰主上同氣唯有太平言之恐傷主上之意不言為患日深為之奈何琚曰天子之孝異於匹夫當以安宗廟社稷為事盖主漢昭帝之姊自幼供養有罪猶誅之【事見漢紀盖古盍翻供居用翻養羊尚翻】為天下者豈顧小節太子悦曰君有何藝可以與寡人遊琚曰能飛煉詼嘲【飛煉謂飛丹砂以鍊丹也舊書載琚之言曰飛丹煉砂詼諧嘲詠可與優人比肩】太子乃奏為詹事府司直【唐六典詹事府司直掌彈劾官寮糾舉職事】日與遊處【處昌呂翻】累遷太子中舍人【唐六典曰太子中舍人本漢魏太子舍人也晉惠帝在儲宫以舍人四人有文學才美者與中庶子共理文書至咸寧三年齊王攸為太傅遂加名為中舍人與中庶子共掌禁令糾正違闕侍從左右儐相威儀盡規獻納】及即位以為中書侍郎 【考異曰鄭綮開天傳信記云上於藩邸時每戲遊城南韋杜之間因逐狡兎意樂忘返與其徒十數人倦甚休息於封部大樹下適有書生延上過其家甚貧止於村妻一驢而已上坐未久書生殺驢拔䔉備饌酒肉霶霈上顧而奇之及與語磊落不凡問其姓名乃王琚也自是上每遊韋杜間必過琚家琚所諮議合上意上益親善焉及韋氏專制上憂甚獨密言於琚琚曰亂則殺之又何疑也上遂納琚之謀戡定禍難累拜為中書侍郎實預配享焉今從舊傳】是時宰相多太平公主之黨劉幽求與右羽林將軍張暐謀以羽林兵誅之使暐密言於上曰竇懷貞崔湜岑羲皆因公主得進日夜為謀不軌若不早圖一旦事起太上皇何以得安請速誅之 【考異曰舊傳云幽求自謂功在朝臣之右志求左僕射兼領中書令俄而竇懷貞為左僕射崔湜為中書令幽求心甚不平形於言色乃與張暐請誅之按幽求素盡心於玄宗湜等附太平非幽求因私忿而害之也今不取】臣已與幽求定計惟俟陛下之命上深以為然暐洩其謀於侍御史鄧光賓上大懼遽列上其狀丙辰幽求下獄有司奏幽求等離間骨肉罪當死上為言幽求有大功不可殺【列上時掌翻下遐嫁翻間古莧翻為于偽翻】癸亥流幽求于封州【封州漢廣信武陽縣地梁置成州隋改封州唐屬廣州都督府舊志封州去京師水陸四千五百一十里】張暐于峰州光賓于繡州【舊志峯州隋交趾郡之嘉寧縣武德四年置峰州去京師七千七百一十里繡州去京師六千九十里】初崔湜為襄州刺史密與譙王重福通書重福遺之金帶【遺于季翻】重福敗湜當死張說劉幽求營護得免既而湜附太平公主與公主謀罷說政事以左丞分司東都及幽求流封州湜諷廣州都督周利貞使殺之【封州屬廣州都督】桂州都督王晙知其謀留幽求不遣【晙子峻翻】利貞屢移牒索之【索山客翻】晙不應利貞以聞湜屢逼晙使遣幽求幽求謂晙曰公拒執政而保流人勢不能全徒仰累耳【累力瑞翻】固請詣廣州晙曰公所坐非可絶於朋友者也晙因公獲罪無所恨竟逗遛不遣幽求由是得免 九月丁卯朔日有食之 辛卯立皇子嗣昇為陜王【陜失冉翻 考異曰睿宗實録作甲申太子皇録作甲午今從玄宗實録】嗣昇母楊氏士逹之曾孫也【楊士逹仕隋官至訥言】王后無子母養之 冬十月庚子上謁太廟赦天下 癸卯上幸新豐獵於驪山之下【驪力知翻】 辛酉沙陀金山遣使入貢沙陀者處月之别種也姓朱邪氏【使疏吏翻種章勇翻邪音耶處月居金娑山之陽蒲類之東有大磧名沙陀故號沙陀 考異曰薛居正五代史後唐太祖紀曰太祖姓朱邪氏始祖拔野古貞觀中為墨離軍使太宗平薛延陀分同羅僕骨之人置沙陀都督府盖北庭有磧曰沙陀因以名焉永徽中以拔野古為都督其後子孫五世相承曾祖盡忠貞元中繼為沙陀府都督歐陽修五代史記曰李氏之先盖出於西突厥本號朱邪至其後世别自號曰沙陀而以朱邪為姓拔野古為始祖其自叙云沙陀者北庭之磧也當唐太宗時破西突厥諸部分同羅僕骨之人於此磧置沙陀府而以其始祖拔野古為都督且傳子孫數世皆為沙陀都督故其後世因自號沙陀然予考于傳紀其說皆非也夷狄無姓氏朱邪部族之號耳拔野古與朱邪同時人非其始祖而唐太宗時未嘗有沙陀府也唐太宗破西突厥分其諸部置三十州以同羅為龜林都督府僕骨為金微都督府拔野古為幽陵都督府未嘗有沙陀府也當是時西突厥有鐵勒薛延陀阿史那之類為最大其别部有同羅僕骨拔野古以十數盖其小者也又有處月處密諸部又其小者也朱邪者處月别部之號耳太宗二十二卷已降拔野古其明年阿史那賀魯叛至高宗永徽二年處月朱邪孤注從賀魯戰于牢山為契苾何力所敗遂沒不見後百五六十年當憲宗時有朱邪盡忠及子執宜見于中國而自號沙陀以朱邪為姓矣盖沙陀者大磧也在金莎山之陽蒲類海之東自處月以來居此磧號沙陀突厥而夷狄無文字傳記朱邪又微不足録故其後世自失其傳至盡忠孫始賜姓李氏李氏後大而夷狄之人遂以沙陀為貴種云今從之】 十一月乙酉奚契丹二萬騎寇漁陽幽州都督宋璟閉城不出虜大掠而去 上皇誥遣皇帝廵邉西自河隴東及燕薊選將練卒【燕因肩翻薊音計將即亮翻】甲午以幽州都督宋璟為左軍大總管并州長史薛訥為中軍大總管朔方大總管兵部尚書郭元振為右軍大總管 十二月刑部尚書李日知請致仕日知在官不行捶撻而事集【捶止橤翻下同】刑部有令史受敇三日忘不行【忘巫放翻】日知怒索杖集羣吏欲捶之【索山客翻】既而謂曰我欲捶汝天下人必謂汝能撩李日知嗔【撩落蕭翻取動也嗔昌真翻】受李日知杖不得比於人妻子亦將弃汝矣遂釋之吏皆感悦無敢犯者脱有稽失衆共謫之開元元年【是年十二月方改元】春正月己亥誥衛士自今二十五入軍五十免羽林飛騎並以衛士簡補【騎奇寄翻】 以吏部尚書蕭至忠為中書令 皇帝廵邊改期所募兵各散遣約八月復集【復扶又翻】竟不成行 二月庚子夜開門然燈【按舊書嚴挺之傳先天二年正月望胡僧婆陁請夜開門燃千百燈】又追作去年大酺【元年受内禅不及賜天下酺乃追為之酺音蒲】大合伎樂上皇與上御門樓臨觀或以夜繼晝凡月餘【帝之侈心盖己發露於此矣伎其綺翻】左拾遺華隂嚴挺之上疏諫以為酺者因人所利合醵為歡【醵其虐翻合錢飲酒也】今乃損萬人之力營百戲之資非所以光聖德美風化也乃止 初高麗既亡【高麗亡見二百一卷高宗總章元年】其别種大祚榮徙居營州及李盡忠反【李盡忠反見二百五卷武后萬歲通天元年風俗通大姓大庭氏之後大款為顓帝師按禮記曰大連善居喪東夷之子也盖東夷之有大姓尚矣種章勇翻】祚榮與靺鞨乞四北羽聚衆東走阻險自固【靺鞨音末曷】盡忠死武后使將軍李楷固討其餘黨楷固擊乞四北羽斬之引兵踰天門嶺逼祚榮【新書天門嶺在上護眞河北三百里】祚榮逆戰楷固大敗僅以身免祚榮遂帥其衆東據東牟山築城居之【東牟山在挹婁國界地直營州東二千里南北新羅以泥河為境東窮海西契丹帥讀曰率】祚榮驍勇善戰【驍堅堯翻下同】高麗靺鞨之人稍稍歸之地方二千里戶十餘萬勝兵數萬人【勝音升】自稱振國王附于突厥時奚契丹皆叛道路阻絶武后不能討中宗即位遣侍御史張行岌招慰之【岌魚及翻】祚榮遣子入侍至是以祚榮為左驍衛大將軍渤海郡王以其所部為忽汗州令祚榮兼都督【靺鞨自此盛矣始去靺鞨專號渤海】庚申勑以嚴挺之忠直宣示百官厚賞之 三月辛<br />
<br />
  巳皇后親蠶【舊制有皇后祀先蠶親桑之禮後周制皇后衣十二等採桑服鴇衣唐制皇后親蠶服鞠衣黄羅為之 考異曰玄宗實録脱此年二月三月事祀先蠶詔乃三月丁卯也而唐歷承其誤云正月辛巳皇后祀先蠶太上皇録云三月辛巳皇后親蠶自嗣聖光宅以來廢闕此禮至是重行太上皇睿宗實録舊本紀皆云辛卯按制書云以今月十八日祀先蠶是月甲子朔今從玄宗實録】 晉陵尉楊相如上疏言時政其略曰煬帝自恃其彊不憂時政雖制敇交行而聲實舛謬言同堯舜迹如桀紂舉天下之大一擲而弃之又曰隋氏縱欲而亡太宗抑欲而昌願陛下詳擇之又曰人主莫不好忠正而惡佞邪【好呼到翻惡烏路翻下同】然忠正者常踈佞邪者常親以至於覆國危身而不寤者何哉誠由忠正者多忤意佞邪者多順指積忤生憎積順生愛此親疎之所以分也明主則不然愛其忤以收忠賢惡其順以去佞邪【忤五故翻去羌呂翻下除去同】則太宗太平之業將何遠哉又曰夫法貴簡而能禁罸貴輕而必行陛下方興崇至德大布新政請一切除去碎密不察小過小過不察則無煩苛大罪不漏則止姧慝使簡而難犯寛而能制則善矣上覽而善之 先是修大明宫未畢【先悉薦翻】夏五月庚寅敕以農務方勤罷之以待閑月【閑月謂農功畢入之後】 六月丙辰以兵部尚書郭元振同中書門下三品 【考異曰六月辛丑郭元振同三品下注曰舊紀在丙辰今從睿宗實録余據考異則通鑑正文當改丙辰為辛丑】 太平公主依上皇之埶擅權用事與上有隙宰相七人五出其門 【考異曰唐歷曰宰相有七四出其門天子孤立而無援新舊傳皆云宰相七人五出主門下按是時竇懷貞蕭至忠岑羲崔湜與主連謀其不附主者郭元振魏知古陸象先三人也薛稷太子少保不為宰相也考新舊傳并象先數之唐歷不數象先耳】文武之臣大半附之與竇懷貞岑羲蕭至忠崔湜及太子少保薛稷雍州長史新興王晉【雍於用翻】左羽林大將軍常元楷知右羽林將軍事李慈左金吾將軍李欽中書舍人李猷右散騎常侍賈膺福鴻臚卿唐晙及僧慧範等謀廢立【晙子峻翻】又與宫人元氏謀于赤箭粉中寘毒進於上【陶弘景曰赤箭亦是芝類莖赤如箭簳葉生其端根如人足又如芋魁有十二子為衛其苖為粉久服益氣力長隂肥健輕身增年沈括曰赤箭天麻苗也根則抽苗徑直而上苗則結子成熟而落返從簳中而下至土而生赤箭則言苗用之有自表入裏之功天麻則言根用之有自内逹外之理本草圖經曰赤箭莖中空依半而上貼莖微有尖葉梢頭生成穗開花結子如豆粒大其子至夏不落却透虚入莖中潜生土内】晉德良之孫也【德良長平王叔良之弟武德初封新興王】元楷慈數往來主第相與結謀【數所角翻】王琚言于上曰事廹矣不可不速發左丞張說自東都遣人遺上佩刀意欲上斷割【遺于李翻君臣之禮當言獻佩刀此因舊史成文失於改定耳斷丁亂翻】荆州長史崔日用入奏事言於上曰太平謀逆有日陛下往在東宫猶為臣子若欲討之須用謀力今既光臨大寶但下一制書誰敢不從萬一姧宄得志悔之何及上曰誠如卿言直恐驚動上皇日用曰天子之孝在於安四海若姧人得志則社稷為墟安在其為孝乎請先定北軍【北軍謂左右羽林左右萬騎也】後收逆黨則不驚動上皇矣上以為然以日用為吏部侍郎秋七月魏知古告公主欲以是月四日作亂 【考異曰上皇録云公主謀不利於上與今上更立皇子獨專權期以是月七日作亂今上密知其事勒左右禁兵誅之按是月壬戌朔玄宗以三日甲子誅之今從玄宗録】令元楷慈以羽林兵突入武德殿【時上於武德殿受羣臣朝故欲突入為變】懷貞至忠羲等於南牙舉兵應之【西内以太極殿為正牙自北門言之曰南牙】上乃與岐王範薛王業郭元振及龍武將軍王毛仲【景雲初以左右萬騎與左右羽林為北門四軍置左右龍武將軍以領萬騎位從三品】殿中少監姜皎太僕少卿李令問尚乘奉御王守一内給事高力士【乘繩證翻内給事屬内侍省從五品下掌判省事元正冬至羣臣朝賀中宫則出入宣傳凡宫人衣服費用則具其品秩計其多少春秋二時宣送中書】果毅李守德等定計誅之皎謩之曾孫【姜謩見一百八十四卷隋恭帝義寧元年】令問靖弟客師之孫【李客師亦有戰功】守一仁皎之子力士潘州人也【潘州古西甌駱越地漢屬合浦郡界江左置定川郡隋廢郡為縣唐武德四年置南宕州貞觀八年改潘州以潘水為名】甲子上因王毛仲取閑廐馬及兵三百餘人自武德殿入䖍化門【西内太極殿北曰朱明門左曰䖍化門右曰肅章門䖍化之東曰武德西門門内則武德殿】召元楷慈先斬之禽膺福猷於内客省以出【四方館隸中書省故内客省在焉中書省在太極門之右膺福猷皆中書省官也】執至忠羲於朝堂【東西朝堂在承天門内分左右朝直遥翻】皆斬之 【考異曰玄宗實録作乙丑按僉載七月三日誅常元楷今從睿宗上皇實録唐歷新舊本紀舊王琚傳琚與岐王範薛王業姜皎王毛仲等並預誅逆以鐵騎至承天門時睿宗聞鼔譟聲召郭元振升承天楼宣詔下闕令侍御史任知古召募數百人於朝堂不得入頃間琚等從玄宗至楼上太上皇實録公主期以是月七日令常元楷以羽林兵自北門入竇懷貞等於南衙舉兵應之今上密知其事登時勒左右禁兵出北門召常元楷李慈即斬於闕下還至承天門執岑羲蕭至忠斬於朝堂舊蕭至忠傳曰至忠遽遁入山寺數日捕而伏誅盖以太平公主事為至忠事今從玄宗實録朝野僉載曰羽林將軍常元楷三家告密得官至先天二年七月三日楷以反逆誅家口配沒玄宗實録云上誅凶逆睿宗恐宫中有變御承天門號令南衙兵士以備非常郭元振帥兵侍衛登楼奏曰皇帝前奉誥誅竇懷貞等惟陛下勿憂睿宗大喜今擇其可信者取之】懷貞逃入溝中自縊死戮其尸改姓曰毒【縊於計翻】上皇聞變登承天門楼郭元振奏皇帝前奉誥誅竇懷貞等無它也上尋至樓上上皇乃下詔罪狀懷貞等因赦天下惟逆人親黨不赦薛稷賜死于萬年獄乙丑上皇誥自今軍國政刑一皆取皇帝處分【處昌呂翻分扶問翻 考異曰舊本紀云七月三日誅懷貞等睿宗明日下詔軍國政刑並取皇帝處分新本紀云乙丑始聽政唐歷亦無乙丑下誥唯玄宗實録云丙寅今從諸書】朕方無為養志以遂素心是日徙居百福殿【唐六典曰兩儀殿之右曰宜秋門宜秋之右日百福門其内百福殿】太平公主逃入山寺三日乃出賜死于家 【考異曰新傳云三日乃出太上皇實録曰公主聞難作遁入山寺數日方出禁錮終身諸子皆伏誅今從新舊傳睿宗實録】公主諸子及黨與死者數十人薛崇簡以數諫其母被撻特免死【數所角翻】賜姓李官爵如故【崇簡即崇暕】籍公主家財貨山積珍物侔於御府廐牧羊馬田園息錢收之數年不盡慧範家亦數十萬緡改新興王晉之姓曰厲【姓譜本自有厲姓漢有魏郡太守義陽侯厲温】初上謀誅竇懷貞等召崔湜將託以心腹湜弟滌謂湜曰主上有問勿有所隱湜不從懷貞等既誅湜與右丞盧藏用俱坐私侍太平公主湜流竇州【舊志竇州至京師水陸六千一百二里】藏用流瀧州【瀧閭江翻】新興王晉臨刑歎曰本為此謀者崔湜今吾死湜生不亦寃乎會有司鞫宮人元氏元氏引湜同謀進毒乃追賜死于荆州【舊志荆州京師東南一千七百三十里】薛稷之子伯陽以尚主免死流嶺南於道自殺初太平公主與其黨謀廢立竇懷貞蕭至忠岑羲崔湜皆以為然陸象先獨以為不可公主曰廢長立少【宋王成器長也長知兩翻少詩照翻】已為不順且又失德若之何不去【去羌呂翻】象先曰既以功立當以罪廢【言上平内難有大功于天下國家無罪不可廢】今實無辠象先終不敢從公主怒而去上既誅懷貞等召象先謂曰歲寒知松柏信哉【論語孔子曰歲寒然後知松柏之後凋也】時窮治公主枝黨當坐者衆象先密為申理所全甚多【治直之翻為于偽翻】然未嘗自言當時無知者百官素為公主所善及惡之者【惡烏路翻】或黜或陟終歲不盡丁卯上御承天門樓赦天下己巳賞功臣郭元振等官爵第舍金帛有差以高力士為右監門將軍知内侍省事【監古銜翻】初太宗定制内侍省不置三品官【内侍省内侍四人以久次一人知省事從四品上】黄衣廩食守門傳命而已天后雖女主宦官亦不用事中宗時嬖倖猥多宦官七品以上至千餘人然衣緋者尚寡【嬖卑義翻又博計翻衣於既翻下同】上在藩邸力士傾心奉之【力士馮益曾孫也聖歷初嶺南討擊使李千里上二閹兒曰金剛曰力士中人高延福養為子故冒姓高既壮為宫闈丞帝在藩力士傾心附結】及為太子奏為内給事至是以誅蕭岑功賞之是後宦官稍增至三千餘人除三品將軍者浸多衣緋紫至千餘人宦官之盛自此始【衣去聲】 壬申遣益州長史畢構等六人宣撫十道 乙亥以左丞張說為中書令 庚辰中書侍郎同平章事陸象先罷為益州長史劒南按察使【使疏吏翻】八月癸巳以封州流人劉幽求為左僕射平章軍國大事 丙辰突厥可汗默啜遣其子楊我支來求昏丁巳許以蜀王女南和縣主妻之【妻七細翻】中宗之崩也同中書門下三品李嶠密表韋后請出相王諸子於外【相息亮翻】上即位於禁中得其表以示侍臣嶠時以特進致仕或請誅之張說曰嶠雖不識逆順然為當時之謀則忠矣上然之九月壬戍以嶠子率更令暢為䖍州刺史【唐六典曰漢率更令丞主庶子舍人更直職似光禄勲晉率更令掌宫殿門戶之禁郎將屯衛之士北齊率更令掌周衛禁防漏刻鍾鼔更工衡翻】令嶠隨暢之官 庚午以劉幽求同中書門下三品 丙戌復置右御史臺督察諸州【去年春廢左御史臺復扶又翻】罷諸道按察使【使疏吏翻】 冬十月辛卯引見京畿縣令【唐京城兩赤縣為京縣畿内諸縣為畿縣京縣令正五品上畿縣令正六品下見賢遍翻】戒以歲饑惠養黎元之意 己亥上幸新豐癸卯講武於驪山之下徵兵二十萬旌旗連亘五十餘里【亘古鄧翻】以軍容不整坐兵部尚書郭元振於纛下將斬之劉幽求張說跪於馬前諫曰元振有大功於社稷不可殺乃流新州【舊志新州去京師五千五十二里】斬給事中知禮儀事唐紹以其制軍禮不肅故也上始欲立威亦無殺紹之意金吾衛將軍李邈遽宣敕斬之上尋罷邈官廢弃終身時二大臣得臯諸軍多震懾失次【懾之涉翻】惟左軍節度薛訥【時講武分左右軍以訥為左軍節度】朔方道大總管解琬二軍不動上遣輕騎召之皆不得入其陳【解戶買翻騎奇寄翻陳讀曰陣】上深歎美慰勉之甲辰獵于渭川【此即新豐界之渭川】上欲以同州刺史姚元之為相張說疾之使御史大夫趙彦昭彈之【彈徒丹翻】上不納又使殿中監姜皎言於上曰陛下常欲擇河東總管而難其人臣今得之矣上問為誰皎曰姚元之文武全才真其人也上曰此張說之意也汝何得面欺罪當死皎叩頭首服【首式又翻】上即遣中使召元之詣行在【使疏吏翻】既至上方獵引見【見賢遍翻】即拜兵部尚書同中書門下三品 【考異曰世傳升平源以為吴兢所撰云姚元崇初拒太平得罪上頗德之既誅太平方任元崇以相進拜同州刺史張說素不叶命趙彦昭驟彈之不許居無何上將獵于渭濱密召元崇會於行所初元崇聞上講武于驪山謂所親曰凖式車駕行幸三百里内刺史合朝覲元崇必為權臣所擠若何參軍李景初進曰某有兒母者其父即教坊長入内相公儻致厚賂使其冒法進狀可逹公然之輒効燕公說使姜皎入曰陛下久卜河東總管重難其人臣有所得何以見賞上曰誰邪如愜有萬金之賜乃曰馮翊太守姚崇文武全材即其人也上曰此張說意也卿罔上當誅皎首服萬死即詔中官追赴行在上方獵于渭濱公至拜馬首上曰卿頗知獵乎元崇曰臣少孤居廣成澤目不知書唯以射獵為事四十年方遇張憬藏謂臣當以文學備位將相無為自弃爾來折節讀書今雖官位過沗至于馳射老而猶能於是呼鷹放犬遲速稱旨上大悦上曰朕久不見卿思有顧問卿可於宰相行中行公行猶後上縱轡久之顧曰卿行何後公曰臣官疎賤不合參宰相行上曰可兵部尚書同平章事公不謝上顧訝焉至頓上命宰臣坐公跪奏臣適奉作弼之詔而不謝欲以十事上獻有不可行臣不敢奉詔上曰悉數之朕當量力而行然定可否公曰自垂拱已來朝廷以刑法理天下臣請聖政先仁義可乎上曰朕深心有望于公也又曰聖朝自喪師青海未有牽復之悔臣請三數十年不求邉功可乎上日可又曰自太后臨朝以來㗋舌之任或出于閹人之口臣請中官不預公事可乎上曰懷之久矣又曰武氏諸親猥侵清切權要之地繼以韋庶人安樂太平用事班序荒雜臣請國親不任臺省官凡有斜封待闕員外等官悉請停罷可乎上曰朕素志也又曰比來近密佞幸之徒冒犯憲網皆以寵免臣請行法可乎上曰朕切齒久矣又曰比因豪家戚里貢獻求媚延及公卿方鎮亦為之臣請除租庸賦税之外悉杜塞之可乎上曰願行之又曰太后造福先寺中宗造聖善寺上皇造金仙玉真觀皆費鉅百萬耗蠧生靈凡寺觀宮殿臣請止絶建造可乎上曰朕每覩之心即不安而况敢為者哉又曰先朝褻狎大臣或虧君臣之敬臣請陛下接之以禮可乎上曰事誠當然有何不可又曰自燕欽融韋月將獻直得罪由是諫臣沮色臣請凡在臣子皆得觸龍鱗犯忌諱可乎上曰朕非唯能容之亦能行之又曰呂氏產禄幾危西京馬竇閻梁亦亂東漢萬古寒心國朝為甚臣請陛下書之于史冊永為殷鑒作萬代法可乎上乃澘然良久曰此事真可為刻肌刻骨者也公再拜曰此誠陛下致仁政之初是臣千載一遇之日臣敢當弼諧之地天下幸甚天下幸甚又再拜蹈舞稱萬歲者三從官千萬皆出涕上曰坐公坐于燕公之下燕公讓不敢坐上問對曰元崇是先朝舊臣合首坐公曰張說是紫微宫使今臣是客宰相不合首坐上曰可紫微宫使居首座果如所言則元崇進不以正又當時天下之事止此十條須因事啓沃豈一旦可邀似好事者為之依託兢名難以盡信今不取】元之吏事明敏三為宰相皆兼兵部尚書【姚崇始相武后後相睿宗今復為相】緣邉屯戍斥候士馬儲械無不默記上初即位勵精為治【治直之翻】每事訪於元之元之應荅如響同僚唯諾而已【唯于癸翻】故上專委任之元之請抑權倖愛爵賞納諫諍却貢獻不與羣臣褻狎上皆納之【此即前所獻十事之二三也】 乙巳車駕還京師 姚元之嘗奏請序進郎吏 【考異曰此出李德裕次柳氏舊聞不知郎吏為何官若郎中員外郎則是清要官不得云秩卑恐是郎將又不敢必故仍用舊文】上仰視殿屋元之再三言之終不應元之懼趨出罷朝高力士諫曰陛下新總萬機宰臣奏事當面加可否奈何一不省察【朝直遥翻省悉景翻】上曰朕任元之以庶政大事當奏聞共議之郎吏卑秩乃一一以煩朕邪會力士宣事至省中【唐世凡機事皆使内臣宣旨于宰相】為元之道上語【為于偽翻】元之乃喜聞者皆服上識君人之體左拾遺曲江張九齡【曲江縣漢屬桂陽郡江左置始興郡唐武德四年置番州尋改東衡州貞觀元年改韶州】以元之有重望為上所信任奏記勸其遠謟躁進純厚【遠于願翻躁則到翻】其略曰任人當才為政大體與之共理無出此途而曏之用才非無知人之鑒其所以失溺在緣情之舉【溺奴狄翻】又曰自君侯職相國之事持用人之權而淺中弱植之徒已延頸企踵而至謟親戚以求譽媚賓客以取容其間豈不有才所失在於無耻元之嘉納其言新興王晉之誅也僚吏皆奔散惟司功李撝步從【從才用翻唐制諸州功曹司功參軍事掌考課假使祭祀禮樂學校表疏書啟禄食祥異醫藥卜筮陳設喪葬】不失在官之禮仍哭其尸姚元之聞之曰欒布之儔也【欒布哭彭越】及為相擢為尚書郎己酉以刑部尚書趙彦昭為朔方道大總管 十一月乙丑劉幽求兼侍中 辛巳羣臣上表請加尊號為開元神武皇帝從之戊子受冊【上時掌翻】 中書侍郎王琚為上所親厚羣臣莫及每進見侍笑語逮夜方出或時休沐往往遣中使召之或言於上曰王琚權譎縱横之才【見賢遍翻使疏吏翻譎古穴翻縱子容翻】可與之定禍亂難與之守承平上由是浸疎之是月命琚兼御史大夫按行北邉諸軍【行下孟翻考異曰朝野僉載曰琚以謟諛自進未周年為中書侍郎其母氏聞之自洛赴京戒之曰汝徒以謟媚取容色交自逹朝廷側目海内切齒吾恐汝家墳隴無人守之琚慙懼表請侍母上初大怒後許之按舊傳琚未嘗去官侍母今不取舊傳又云使琚按行天兵以北諸軍按五年始置天兵軍於并州盖琚傳追言之耳】十二月庚寅赦天下改元【改元開元】尚書左右僕射為左<br />
<br />
  右丞相中書省為紫微省門下省為黄門省侍中為監雍州為京兆府洛州為河南府長史為尹司馬為少尹【隋以京守為牧武德初因隋置牧以親王為之或不出閤以長史知府事至是改為府升長史為尹從三品專總府事魏晉以下州府皆有治中隋文帝改為司馬煬帝改為贊理又為丞武德改為治中永徽避高宗名改為司馬至是改為少尹從四品下雍於用翻】 甲午吐蕃遣其大臣來求和壬寅以姚元之兼紫微令元之避開元尊號復名崇<br />
<br />
  【姚元之本名元崇武后長安四年命以字行今復舊名而省元字復扶又翻】 敕都督刺史都護將之官皆引面辭畢側門取進止【東内有左右側門左右側門之外即金吾左右仗】 姚崇既為相紫微令張說懼乃潜詣岐王申款【款誠也】它日崇對於便殿行微蹇上問有足疾乎對曰臣有腹心之疾非足疾也上問其故對曰岐王陛下愛弟張說為輔臣而密乘車入王家恐為所誤故憂之癸丑說左遷相州刺史 【考異曰松窗雜録姚崇為相忽一日對于便殿舉右足不甚輕利上曰卿有足疾邪崇奏曰臣有心腹之疾非足疾也因前奏張說罪狀數百言上怒曰卿歸中書宜宣與御史中丞共按其事而說未之知會朱衣吏報午後三刻說乘馬先歸崇急呼御史中丞李林甫以前詔付之林甫語崇曰說多智謀是必困之宜以劇地崇曰丞相得罪未宜太逼林甫又曰公必不忍即說當無害林甫止將詔付於小御史中路以馬墜告說未遭崇奏前旬月家有教授書生通于說侍兒最寵者會擒得姦狀以聞於說說怒甚將窮獄于京兆尹書生厲聲言曰覩色不能禁人之常情也公貴為宰相豈無緩急用人何靳靳於一婢女邪說奇其言而釋之兼以侍兒與歸書生跳跡去旬餘無所聞知忽一日直訪于說憂色滿面而言曰某感公之恩當有謝者久矣今聞公為姚相所構外獄將具公不之知危將至矣某願得公平生所寶者用計于九公主必能立釋之說因自歷指狀所寶者書生皆曰未足解公之難又凝思久之忽曰近有雞林郡夜明簾為寄信者書生曰吾事濟矣因請說手筆數行懇以情言遂急趍出□夜始及九公主邸第書生具以說言之兼用夜明簾為贄且謂主曰上獨不念在東宫時思必始終恩加於張丞相乎而今反用快不利張丞相者之心邪明早公主上謁具為奏之上感動因急命高力士就御史臺宣前所按獄事並宜罷之書生迄亦不再見于張丞相也此說亦似出於好事者又元崇開元四年罷相林甫十四年始為御史中丞今從新傳】右僕射同中書門下三品劉幽求亦罷為太子少保甲寅以黄門侍郎盧懷慎同紫微黄門平章事<br />
<br />
  資治通鑑卷二百十<br />
<br />
<史部,編年類,資治通鑑>  <br>
   </div> 

<script src="/search/ajaxskft.js"> </script>
 <div class="clear"></div>
<br>
<br>
 <!-- a.d-->

 <!--
<div class="info_share">
</div> 
-->
 <!--info_share--></div>   <!-- end info_content-->
  </div> <!-- end l-->

<div class="r">   <!--r-->



<div class="sidebar"  style="margin-bottom:2px;">

 
<div class="sidebar_title">工具类大全</div>
<div class="sidebar_info">
<strong><a href="http://www.guoxuedashi.com/lsditu/" target="_blank">历史地图</a></strong>  
<a href="http://www.880114.com/" target="_blank">英语宝典</a>  
<a href="http://www.guoxuedashi.com/13jing/" target="_blank">十三经检索</a> 
<br><strong><a href="http://www.guoxuedashi.com/gjtsjc/" target="_blank">古今图书集成</a></strong> 
<a href="http://www.guoxuedashi.com/duilian/" target="_blank">对联大全</a> <strong><a href="http://www.guoxuedashi.com/xiangxingzi/" target="_blank">象形文字典</a></strong> 

<br><a href="http://www.guoxuedashi.com/zixing/yanbian/">字形演变</a>  <strong><a href="http://www.guoxuemi.com/hafo/" target="_blank">哈佛燕京中文善本特藏</a></strong>
<br><strong><a href="http://www.guoxuedashi.com/csfz/" target="_blank">丛书&方志检索器</a></strong> <a href="http://www.guoxuedashi.com/yqjyy/" target="_blank">一切经音义</a>  

<br><strong><a href="http://www.guoxuedashi.com/jiapu/" target="_blank">家谱族谱查询</a></strong>  <strong><a href="http://shufa.guoxuedashi.com/sfzitie/" target="_blank">书法字帖欣赏</a></strong> 
<br>

</div>
</div>


<div class="sidebar" style="margin-bottom:0px;">

<font style="font-size:22px;line-height:32px">QQ交流群9:489193090</font>


<div class="sidebar_title">手机APP 扫描或点击</div>
<div class="sidebar_info">
<table>
<tr>
	<td width=160><a href="http://m.guoxuedashi.com/app/" target="_blank"><img src="/img/gxds-sj.png" width="140"  border="0" alt="国学大师手机版"></a></td>
	<td>
<a href="http://www.guoxuedashi.com/download/" target="_blank">app软件下载专区</a><br>
<a href="http://www.guoxuedashi.com/download/gxds.php" target="_blank">《国学大师》下载</a><br>
<a href="http://www.guoxuedashi.com/download/kxzd.php" target="_blank">《汉字宝典》下载</a><br>
<a href="http://www.guoxuedashi.com/download/scqbd.php" target="_blank">《诗词曲宝典》下载</a><br>
<a href="http://www.guoxuedashi.com/SiKuQuanShu/skqs.php" target="_blank">《四库全书》下载</a><br>
</td>
</tr>
</table>

</div>
</div>


<div class="sidebar2">
<center>


</center>
</div>

<div class="sidebar"  style="margin-bottom:2px;">
<div class="sidebar_title">网站使用教程</div>
<div class="sidebar_info">
<a href="http://www.guoxuedashi.com/help/gjsearch.php" target="_blank">如何在国学大师网下载古籍?</a><br>
<a href="http://www.guoxuedashi.com/zidian/bujian/bjjc.php" target="_blank">如何使用部件查字法快速查字?</a><br>
<a href="http://www.guoxuedashi.com/search/sjc.php" target="_blank">如何在指定的书籍中全文检索?</a><br>
<a href="http://www.guoxuedashi.com/search/skjc.php" target="_blank">如何找到一句话在《四库全书》哪一页?</a><br>
</div>
</div>


<div class="sidebar">
<div class="sidebar_title">热门书籍</div>
<div class="sidebar_info">
<a href="/so.php?sokey=%E8%B5%84%E6%B2%BB%E9%80%9A%E9%89%B4&kt=1">资治通鉴</a> <a href="/24shi/"><strong>二十四史</strong></a>&nbsp; <a href="/a2694/">野史</a>&nbsp; <a href="/SiKuQuanShu/"><strong>四库全书</strong></a>&nbsp;<a href="http://www.guoxuedashi.com/SiKuQuanShu/fanti/">繁体</a>
<br><a href="/so.php?sokey=%E7%BA%A2%E6%A5%BC%E6%A2%A6&kt=1">红楼梦</a> <a href="/a/1858x/">三国演义</a> <a href="/a/1038k/">水浒传</a> <a href="/a/1046t/">西游记</a> <a href="/a/1914o/">封神演义</a>
<br>
<a href="http://www.guoxuedashi.com/so.php?sokeygx=%E4%B8%87%E6%9C%89%E6%96%87%E5%BA%93&submit=&kt=1">万有文库</a> <a href="/a/780t/">古文观止</a> <a href="/a/1024l/">文心雕龙</a> <a href="/a/1704n/">全唐诗</a> <a href="/a/1705h/">全宋词</a>
<br><a href="http://www.guoxuedashi.com/so.php?sokeygx=%E7%99%BE%E8%A1%B2%E6%9C%AC%E4%BA%8C%E5%8D%81%E5%9B%9B%E5%8F%B2&submit=&kt=1"><strong>百衲本二十四史</strong></a>  <a href="http://www.guoxuedashi.com/so.php?sokeygx=%E5%8F%A4%E4%BB%8A%E5%9B%BE%E4%B9%A6%E9%9B%86%E6%88%90&submit=&kt=1"><strong>古今图书集成</strong></a>
<br>

<a href="http://www.guoxuedashi.com/so.php?sokeygx=%E4%B8%9B%E4%B9%A6%E9%9B%86%E6%88%90&submit=&kt=1">丛书集成</a> 
<a href="http://www.guoxuedashi.com/so.php?sokeygx=%E5%9B%9B%E9%83%A8%E4%B8%9B%E5%88%8A&submit=&kt=1"><strong>四部丛刊</strong></a>  
<a href="http://www.guoxuedashi.com/so.php?sokeygx=%E8%AF%B4%E6%96%87%E8%A7%A3%E5%AD%97&submit=&kt=1">說文解字</a> <a href="http://www.guoxuedashi.com/so.php?sokeygx=%E5%85%A8%E4%B8%8A%E5%8F%A4&submit=&kt=1">三国六朝文</a>
<br><a href="http://www.guoxuedashi.com/so.php?sokeytm=%E6%97%A5%E6%9C%AC%E5%86%85%E9%98%81%E6%96%87%E5%BA%93&submit=&kt=1"><strong>日本内阁文库</strong></a> <a href="http://www.guoxuedashi.com/so.php?sokeytm=%E5%9B%BD%E5%9B%BE%E6%96%B9%E5%BF%97%E5%90%88%E9%9B%86&ka=100&submit=">国图方志合集</a> <a href="http://www.guoxuedashi.com/so.php?sokeytm=%E5%90%84%E5%9C%B0%E6%96%B9%E5%BF%97&submit=&kt=1"><strong>各地方志</strong></a>

</div>
</div>


<div class="sidebar2">
<center>

</center>
</div>
<div class="sidebar greenbar">
<div class="sidebar_title green">四库全书</div>
<div class="sidebar_info">

《四库全书》是中国古代最大的丛书,编撰于乾隆年间,由纪昀等360多位高官、学者编撰,3800多人抄写,费时十三年编成。丛书分经、史、子、集四部,故名四库。共有3500多种书,7.9万卷,3.6万册,约8亿字,基本上囊括了古代所有图书,故称“全书”。<a href="http://www.guoxuedashi.com/SiKuQuanShu/">详细>>
</a>

</div> 
</div>

</div>  <!--end r-->

</div>
<!-- 内容区END --> 

<!-- 页脚开始 -->
<div class="shh">

</div>

<div class="w1180" style="margin-top:8px;">
<center><script src="http://www.guoxuedashi.com/img/plus.php?id=3"></script></center>
</div>
<div class="w1180 foot">
<a href="/b/thanks.php">特别致谢</a> | <a href="javascript:window.external.AddFavorite(document.location.href,document.title);">收藏本站</a> | <a href="#">欢迎投稿</a> | <a href="http://www.guoxuedashi.com/forum/">意见建议</a> | <a href="http://www.guoxuemi.com/">国学迷</a> | <a href="http://www.shuowen.net/">说文网</a><script language="javascript" type="text/javascript" src="https://js.users.51.la/17753172.js"></script><br />
  Copyright &copy; 国学大师 古典图书集成 All Rights Reserved.<br>
  
  <span style="font-size:14px">免责声明:本站非营利性站点,以方便网友为主,仅供学习研究。<br>内容由热心网友提供和网上收集,不保留版权。若侵犯了您的权益,来信即刪。scp168@qq.com</span>
  <br />
ICP证:<a href="http://www.beian.miit.gov.cn/" target="_blank">鲁ICP备19060063号</a></div>
<!-- 页脚END --> 
<script src="http://www.guoxuedashi.com/img/plus.php?id=22"></script>
<script src="http://www.guoxuedashi.com/img/tongji.js"></script>

</body>
</html>
