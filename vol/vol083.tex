<!DOCTYPE html PUBLIC "-//W3C//DTD XHTML 1.0 Transitional//EN" "http://www.w3.org/TR/xhtml1/DTD/xhtml1-transitional.dtd">
<html xmlns="http://www.w3.org/1999/xhtml">
<head>
<meta http-equiv="Content-Type" content="text/html; charset=utf-8" />
<meta http-equiv="X-UA-Compatible" content="IE=Edge,chrome=1">
<title>資治通鑒_84-資治通鑑卷八十三_84-資治通鑑卷八十三</title>
<meta name="Keywords" content="資治通鑒_84-資治通鑑卷八十三_84-資治通鑑卷八十三">
<meta name="Description" content="資治通鑒_84-資治通鑑卷八十三_84-資治通鑑卷八十三">
<meta http-equiv="Cache-Control" content="no-transform" />
<meta http-equiv="Cache-Control" content="no-siteapp" />
<link href="/img/style.css" rel="stylesheet" type="text/css" />
<script src="/img/m.js?2020"></script> 
</head>
<body>
 <div class="ClassNavi">
<a  href="/24shi/">二十四史</a> | <a href="/SiKuQuanShu/">四库全书</a> | <a href="http://www.guoxuedashi.com/gjtsjc/"><font  color="#FF0000">古今图书集成</font></a> | <a href="/renwu/">历史人物</a> | <a href="/ShuoWenJieZi/"><font  color="#FF0000">说文解字</a></font> | <a href="/chengyu/">成语词典</a> | <a  target="_blank"  href="http://www.guoxuedashi.com/jgwhj/"><font  color="#FF0000">甲骨文合集</font></a> | <a href="/yzjwjc/"><font  color="#FF0000">殷周金文集成</font></a> | <a href="/xiangxingzi/"><font color="#0000FF">象形字典</font></a> | <a href="/13jing/"><font  color="#FF0000">十三经索引</font></a> | <a href="/zixing/"><font  color="#FF0000">字体转换器</font></a> | <a href="/zidian/xz/"><font color="#0000FF">篆书识别</font></a> | <a href="/jinfanyi/">近义反义词</a> | <a href="/duilian/">对联大全</a> | <a href="/jiapu/"><font  color="#0000FF">家谱族谱查询</font></a> | <a href="http://www.guoxuemi.com/hafo/" target="_blank" ><font color="#FF0000">哈佛古籍</font></a> 
</div>

 <!-- 头部导航开始 -->
<div class="w1180 head clearfix">
  <div class="head_logo l"><a title="国学大师官网" href="http://www.guoxuedashi.com" target="_blank"></a></div>
  <div class="head_sr l">
  <div id="head1">
  
  <a href="http://www.guoxuedashi.com/zidian/bujian/" target="_blank" ><img src="http://www.guoxuedashi.com/img/top1.gif" width="88" height="60" border="0" title="部件查字,支持20万汉字"></a>


<a href="http://www.guoxuedashi.com/help/yingpan.php" target="_blank"><img src="http://www.guoxuedashi.com/img/top230.gif" width="600" height="62" border="0" ></a>


  </div>
  <div id="head3"><a href="javascript:" onClick="javascript:window.external.AddFavorite(window.location.href,document.title);">添加收藏</a>
  <br><a href="/help/setie.php">搜索引擎</a>
  <br><a href="/help/zanzhu.php">赞助本站</a></div>
  <div id="head2">
 <a href="http://www.guoxuemi.com/" target="_blank"><img src="http://www.guoxuedashi.com/img/guoxuemi.gif" width="95" height="62" border="0" style="margin-left:2px;" title="国学迷"></a>
  

  </div>
</div>
  <div class="clear"></div>
  <div class="head_nav">
  <p><a href="/">首页</a> | <a href="/ShuKu/">国学书库</a> | <a href="/guji/">影印古籍</a> | <a href="/shici/">诗词宝典</a> | <a   href="/SiKuQuanShu/gxjx.php">精选</a> <b>|</b> <a href="/zidian/">汉语字典</a> | <a href="/hydcd/">汉语词典</a> | <a href="http://www.guoxuedashi.com/zidian/bujian/"><font  color="#CC0066">部件查字</font></a> | <a href="http://www.sfds.cn/"><font  color="#CC0066">书法大师</font></a> | <a href="/jgwhj/">甲骨文</a> <b>|</b> <a href="/b/4/"><font  color="#CC0066">解密</font></a> | <a href="/renwu/">历史人物</a> | <a href="/diangu/">历史典故</a> | <a href="/xingshi/">姓氏</a> | <a href="/minzu/">民族</a> <b>|</b> <a href="/mz/"><font  color="#CC0066">世界名著</font></a> | <a href="/download/">软件下载</a>
</p>
<p><a href="/b/"><font  color="#CC0066">历史</font></a> | <a href="http://skqs.guoxuedashi.com/" target="_blank">四库全书</a> |  <a href="http://www.guoxuedashi.com/search/" target="_blank"><font  color="#CC0066">全文检索</font></a> | <a href="http://www.guoxuedashi.com/shumu/">古籍书目</a> | <a   href="/24shi/">正史</a> <b>|</b> <a href="/chengyu/">成语词典</a> | <a href="/kangxi/" title="康熙字典">康熙字典</a> | <a href="/ShuoWenJieZi/">说文解字</a> | <a href="/zixing/yanbian/">字形演变</a> | <a href="/yzjwjc/">金 文</a> <b>|</b>  <a href="/shijian/nian-hao/">年号</a> | <a href="/diming/">历史地名</a> | <a href="/shijian/">历史事件</a> | <a href="/guanzhi/">官职</a> | <a href="/lishi/">知识</a> <b>|</b> <a href="/zhongyi/">中医中药</a> | <a href="http://www.guoxuedashi.com/forum/">留言反馈</a>
</p>
  </div>
</div>
<!-- 头部导航END --> 
<!-- 内容区开始 --> 
<div class="w1180 clearfix">
  <div class="info l">
   
<div class="clearfix" style="background:#f5faff;">
<script src='http://www.guoxuedashi.com/img/headersou.js'></script>

</div>
  <div class="info_tree"><a href="http://www.guoxuedashi.com">首页</a> > <a href="/SiKuQuanShu/fanti/">四库全书</a>
 > <h1>资治通鉴</h1> <!--         下载:【右键另存为】即可 --></div>
  <div class="info_content zj clearfix">
  
<div class="info_txt clearfix" id="show">
<center style="font-size:24px;">84-資治通鑑卷八十三</center>
    資治通鑑卷八十三   宋 司馬光 撰<br />
<br />
  胡三省 音註<br />
<br />
  晉紀五【起屠維協洽盡上章涒灘凡二年】<br />
<br />
  孝惠皇帝上之下<br />
<br />
  元康九年春正月孟觀大破氐衆於中亭【水經注扶風美陽縣有中亭水亦謂之中亭川在美陽縣西】獲齊萬年太子洗馬陳留江統【洗悉薦翻】以為戎狄亂華宜早絶其原乃作徙戎論以警朝廷曰夫夷蠻戎狄地在要荒【周禮九州之外謂之蕃國謂東夷南蠻西戎北狄也國語曰蠻夷要服戎狄荒服韋昭注曰要者要結好信而服從之荒者言荒忽無常也要一遥翻】禹平九土而西戎即叙【孔安國曰言荒服之外流沙之内皆就次叙班固曰即叙者言就而叙之】其性氣貪婪【婪盧含翻】凶悍不仁【悍侯罕翻又下罕翻】四夷之中戎狄為甚弱則畏服彊則侵叛當其彊也以漢高祖困於白登孝文軍於霸上及其弱也以元成之微而單于入朝此其巳然之效也【單音禪朝直遥翻】是以有道之君牧夷狄也惟以待之有備禦之有常雖稽顙軌贄【周禮蕃國世一見各以其所貴寶為贄稽音啟】而邉城不弛固守【漢元帝時匈奴單于請罷邉塞守備應以為不可所謂不弛固守也】彊暴為寇而兵甲不加遠征【周宣王薄伐獫狁至于太原盡境而返比于蟁蝱驅之而已不加遠征也】令境内獲安疆場不侵而已及至周室失統諸侯專征封疆不固利害異心戎狄乘閒得入中國【如戎伐魯濟西山戎病燕狄伐衛邢長狄入三國之類閒古莧翻】或招誘安撫以為已用【如申繒以西戎攻殺周幽王晉遷陸渾之戎於伊川與之掎角以敗秦師于殽楚以蠻軍與晉戰于鄢陵誘音酉】自是四夷交侵與中國錯居【如徐夷在齊晉魯宋之間鮮虞介燕晉之境赤狄居上黨之地陸渾戎居伊洛之間義渠大荔居秦晉之域戎蠻子居梁霍之地】及秦始皇并天下兵威旁達攘胡走越當是時中國無復四夷也【事見秦紀】漢建武中馬援領隴西太守討叛羌徙其餘種於關中【種章勇翻】居馮翊河東空地數歲之後族類蕃息【蕃扶元翻】既恃其肥彊且苦漢人侵之永初之元羣羌叛亂覆没將守屠破城邑鄧隲敗北侵及河内十年之中夷夏俱敝任尚馬賢僅乃克之【事並見漢紀按漢光武建武十一年馬援討羌降之安帝永初元年羌反自建武十一年至永初元年凡七十三年數歲之後當作數十歲之後將即亮翻守式又翻隲之日翻夏戶雅翻任音壬】自此之後餘燼不盡小有際會輒復侵叛【復扶又翻】中世之寇惟此為大魏興之初與蜀分隔疆場之戎一彼一此武帝徙武都氐於秦川【事見六十八卷漢獻帝建安二十三年】欲以弱寇彊國扞禦蜀虜此蓋權宜之計非萬世之利也今者當之已受其敝矣夫關中土沃物豐帝王所居【周都豐鎬秦都咸陽漢都長安皆關中之地】未聞戎狄宜在此土也非我族類其心必異而因其衰敝遷之畿服【畿服謂邦畿千里之内】士庶翫習侮其輕弱使其怨恨之氣毒於骨髓至於蕃育衆盛【蕃扶袁翻】則坐生其心以貪悍之性挾憤怒之情候隙乘便輒為横逆【横戶孟翻】而居封域之内無障塞之隔掩不備之人收散野之積【積子賜翻聚也】故能為禍滋蔓暴害不測此必然之埶已驗之事也當今之宜宜及兵威方盛衆事未罷徙馮翊北地新平安定界内諸羌著先零罕开析支之地徙扶風始平京兆之氐出還隴右著隂平武都之界【先零罕开析支之地自湟中西至賜支河首隂平武都舊白馬氐地也著直畧翻零音憐开苦堅翻】廩其道路之糧令足自致【廩當作稟給也下廩糧同】各附本種【種章勇翻】反其舊土使屬國撫夷就安集之【屬國都尉及撫夷護軍也】戎晉不雜並得其所縱有猾夏之心【孔安國曰猾亂也夏華夏也夏戶雅翻】風塵之警則絶遠中國隔閡山河【遠于願翻閡與礙同】雖有寇暴所害不廣矣難者曰氐寇新平關中饑疫百姓愁苦咸望寧息而欲使疲悴之衆徙自猜之寇恐埶盡力屈緒業不卒【難乃旦翻悴秦醉翻卒子恤翻終也】前害未及弭而後變復横出矣【復扶又翻】答曰子以今者羣氐為尚挾餘資悔惡反善懷我德惠而來柔附乎將埶窮道盡智力俱困懼我兵誅以至於此乎曰無有餘力埶窮道盡故也然則我能制其短長之命而令其進退由已矣夫樂其業者不易事【樂音洛】安其居者無遷志方其自疑危懼畏怖促遽【怖普布翻】故可制以兵威使之左右無違也迨其死亡流散離逷未鳩【逷他歷翻爾雅曰逷遠也鳩集也】與關中之人戶皆為讐【謂氐羌之反暴掠平民關中之人怨毒之戶皆為讐敵】故可遐遷遠處令其心不懷土也夫聖賢之謀事也為之於未有治之於未亂【治直之翻】道不著而平德不顯而成其次則能轉禍為福因敗為功值困必濟遇否能通【否皮鄙翻】今子遭敝事之終而不圖更制之始【更工衡翻】愛易轍之勤而遵覆車之軌何哉【車覆於前不可遵其轍當易路而行若遵覆車之迹則後車又將覆矣】且關中之人百餘萬口率其少多【率列恤翻約數也少詩沼翻】戎狄居半處之與遷必須口實【口實謂糧食也處昌呂翻】若有窮乏糝粒不繼者【糝桑頷翻以米和羮也】故當傾關中之穀以全其生生之計必無擠於溝壑而不為侵掠之害也【氐羌窮乏勢必聚而侵掠晉朝欲弭其害故當傾穀以給之擠子西翻又子細翻】今我遷之傳食而至【謂所過郡縣逓給其食也傳直戀翻】附其種族自使相贍而秦地之人得其半穀【言關中居人戎狄居半今遷使歸其舊地則秦中百姓將食其所積之穀以約率之正得常居之半穀也種章勇翻下餘種同】此為濟行者以廩糧遺居者以積倉【遺于季翻】寛關中之逼去盜賊之原【去羌呂翻】除旦夕之損建終年之益若憚蹔舉之小勞【蹔與暫同】而忘永逸之宏策惜日月之煩苦而遺累世之寇敵非所謂能創業垂統謀及子孫者也并州之胡本實匈奴桀惡之寇也建安中使右賢王去卑誘質呼廚泉聽其部落散居六郡【謂并州所統六郡也晉書匈奴傳曰匈奴與晉人雜居平陽西河大原新興上黨樂平莫不有焉質呼㕑泉事見六十七卷漢獻帝建安二十一年質音致】咸熙之際以一部太彊分為三率【率讀曰帥音所類翻】泰始之初又增為四於是劉猛内叛連結外虜【事見七十九卷武帝泰始七年八年】近者郝散之變發於穀遠【穀遠縣漢屬上黨郡晉省蓋其地猶存舊縣名也劉昫曰穀遠今沁源縣宋白曰漢穀遠故縣在沁源縣南百五十步孤遠故城是也晉地記云穀遠今名孤遠後代語訛耳郝散事見上卷四年】今五部之衆戶至數萬人口之盛過於西戎其天性驍勇弓馬便利倍於氐羌【驍堅堯翻】若有不虞風塵之慮則并州之域可為寒心【劉淵之禍江統固逆知之矣】正始中母丘儉討句驪【事見七十五卷魏邵陵厲公正始七年句如字又音駒驪力知翻】徙其餘種於滎陽【種章勇翻】始徙之時戶落百數子孫孳息【孶津之翻生也】今以千計數世之後必至殷熾【熾昌志翻】今百姓失職【民不得安于耕鑿是失職也】猶或亡叛犬馬肥充則有噬齧况於夷狄能不為變但顧其微弱埶力不逮耳【顧内顧也】夫為邦者憂不在寡而在不安【論語孔子曰丘聞有國有家者不患寡而患不均不患貧而患不安】以四海之廣士民之富豈須夷虜在内然後取足哉此等皆可申諭發遣還其本域慰彼羈旅懷土之思釋我華夏纖介之憂【夏戶雅翻】惠此中國以綏四方【詩大雅民勞之辭】德施永世於計為長也朝廷不能用 散騎常侍賈謐侍講東宫對太子倨傲成都王頴見而叱之謐怒言於賈后出頴為平北將軍鎮鄴 【考異曰帝紀云以頴為鎮北大將軍今從本傳】徵梁王肜為大將軍録尚書事以河間王顒為鎮西將軍鎮關中【肜余中翻顒魚容翻】初武帝作石函之制非至親不得鎮關中顒輕財愛士朝廷以為賢故用之【顒安平獻王孚之孫太原烈王瓌之子也初襲父爵咸寧三年改封河間為穎顒各據方鎮以阻兵張本】 夏六月高密文獻王泰薨 【考異曰帝紀云隴西王本傳云泰為尚書令改封高密紀誤】 賈后淫虐日甚私於太醫令程據等【晉志太醫令屬宗正】又以簏箱載道上年少入宫【簏盧谷翻說文竹高篋也少詩照翻】復恐其漏泄往往殺之【復扶又翻】賈模恐禍及已甚憂之裴頠與模及張華議廢后更立謝淑妃【謝淑妃太子之母也頠魚毁翻更工衡翻考異曰模與裴頠王衍謀廢之衍後悔而正今從頠傳】模華皆曰主上自無廢黜<br />
<br />
  之意而吾等專行之儻上心不以為然將若之何且諸王方彊朋黨各異恐一旦禍起身死國危無益社稷頠曰誠如公言然宫中逞其昏虐亂可立待也華曰卿二人於中宫皆親戚言或見信宜數為陳禍福之戒庶無大悖則天下尚未至於亂吾曹得以優游卒歲而已【張華處昏亂之朝位冠羣后而持心如此天殆假手於趙王倫而誅之也數所角翻為于偽翻卒子恤翻悖蒲内翻】頠旦夕說其從母廣城君【說輸芮翻從才用翻】令戒諭賈后以親厚太子賈模亦數為后言禍福后不能用反以模為毁已而疎之模不得志憂憤而卒秋八月以裴頠為尚書僕射頠雖賈后親屬然雅望素隆四海惟恐其不居權位尋詔頠專任門下事【晉制侍中與給事黄門侍郎同管門下事頠為侍中專任門下事賈后之意也】頠上表固辭以賈模適亡復以臣代之【復扶又翻】崇外戚之望彰偏私之舉為聖朝累【累力瑞翻】不聽或謂頠曰君可以言當盡言於中宫言而不從當遠引而去儻二者不立雖有十表難以免矣頠慨然久之竟不能從【史言華頠顧戀禄位以殞首亡家】帝為人戇騃【戅陟降翻愚也騃語駭翻癡也】嘗在華林園聞蝦蟆【蝦何加翻蟆謨加翻】謂左右曰此鳴者為官乎為私乎【為于偽翻】時天下荒饉百姓餓死帝聞之曰何不食肉糜【糜忙皮翻粥也】由是權在羣下政出多門埶位之家更相薦託有如互市【更工衡翻】賈郭恣横【横戶孟翻】貨賂公行南陽魯褒作錢神論以譏之曰錢之為體有乾坤之象親之如兄字曰孔方【錢圜函方天圜而地方故曰有乾坤之象孔方亦以錢體言】無德而尊無埶而熱排金門入紫闥危可使安死可使活貴可使賤生可使殺是故忿爭非錢不勝幽滯非錢不抜怨讐非錢不解令聞非錢不發【聞音問】洛中朱衣當塗之士【晉制諸王朱衣絳紗襮當塗之士謂當路柄用者】愛我家兄皆無己已執我之手抱我終始凡今之人惟錢而已又朝臣務以苛察相高每有疑議羣下各立私意刑法不壹獄訟繁滋裴頠上表曰先王刑賞相稱【稱尺證翻】輕重無二故下聽有常羣吏安業去元康四年大風廟闕屋瓦有數枚傾落免太常荀㝢事輕責重有違常典五年二月有大風蘭臺主者懲懼前事求索阿棟之間得瓦小邪十五處【蘭臺主者御史臺主者也即令史之類阿屋之隈曲棟屋穩也索山客翻】遂禁止太常復興刑獄【復扶又翻下頌復史復同】今年八月陵上荆一枝圍七寸二分者被斫司徒太常奔走道路【說文荆楚木也司徒漢丞相之職漢制丞相與太常掌園陵被皮義翻】雖知事小而按劾難測【劾戶槩翻又戶得翻】搔擾驅馳各競免負【負罪負也】于今太常禁止未解夫刑書之文有限而舛違之故無方故有臨時議處之制【言法有一定之文而罪有故誤情有輕重故制令臨時隨事情議處其罪處昌呂翻】誠不能皆得循常也至於此等皆為過當【當丁浪翻】恐姦吏因緣得為淺深也既而曲議猶不止【曲議謂曲法而議自為淺深】三公尚書劉頌復上疏曰【晉志漢成帝置三公尚書主斷獄光武以三公曹主歲盡考課州郡事】自近世以來法漸多門令甚不一吏不知所守下不知所避姦偽者因以售其情居上者難以檢其下【檢校檢束也】事同議異獄犴不平【犴魚旰翻野獄曰犴】夫君臣之分各有所司法欲必奉故令主者守文理有窮塞故使大臣釋滯事有時宜故人主權斷【塞悉則翻斷丁亂翻下宏斷同】主者守文若釋之執犯蹕之平也【事見十四卷漢文帝三年】大臣釋滯若公孫宏斷郭解之獄也【事見十八卷漢武帝元朔二年】人主權斷若漢祖戮丁公之為也【事見十一卷漢高祖五年】天下萬事自非此類不得出意妄議皆以律令從事然後法信於下人聽不惑吏不容姦可以言政矣 【考異曰刑法志叙頌奏續頠表之下而云侍中太宰汝南王亮按頠表引元康八年事時亮死已久蓋志誤也】乃下詔郎令史復出法駁案者隨事以聞然亦不能革也【郎令史尚書郎及尚書蘭臺令史也出法駁案者謂出於法之外而為駁議也駁北角翻】頌遷吏部尚書建九班之制欲令百官居職希遷考課能否明其賞罰賈郭用權仕者欲速事竟不行裴頠薦平陽韋忠於張華【魏邵陵厲公正始八年分河東郡之汾北為平陽郡】華辟之忠辭疾不起人問其故忠曰張茂先華而不實裴逸民慾而無厭【張華字茂先裴頠字逸民厭於鹽翻】棄典禮而附賊后此豈大丈夫之所為哉逸民每有心託我我常恐其溺於深淵而餘波及我况可褰裳而就之哉【溺奴狄翻】關内侯敦煌索靖知天下將亂【敦徒門翻索蘇各翻】指洛陽宫門銅駝歎曰會見汝在荆棘中耳【銅駝魏明帝景初元年自長安徙之洛陽】冬十一月甲子朔日有食之 初廣城君郭槐以賈后無子常勸后使慈愛太子賈謐驕縱數無禮於太子【數所角翻】廣城君恒切責之【恒戶登翻】廣城君欲以韓壽女為太子妃太子亦欲婚韓氏以自固壽妻賈午及后皆不聽而為太子聘王衍少女太子聞衍長女美而后為賈謐聘之【為于偽翻少詩照翻長知兩翻】心不能平頗以為言及廣城君病臨終執后手令盡心於太子言甚切至又曰趙粲賈午必亂汝家事我死後勿復聽入深記吾言【郭槐妬狠而垂没之時所以告戒其女者如此蓋多權數故其智慮能及此耳復扶又翻下同】后不從更與粲午謀害太子太子幼有令名【事見上卷武帝太康十年】及長不好學【長知兩翻好呼報翻下同】惟與左右嬉戲賈后復使黄門輩誘之為奢靡威虐【誘音酉】由是名譽浸減驕慢益彰或廢朝侍而縱遊逸【朝直遥翻】於宫中為市使人屠酤手揣斤兩【揣初委翻】輕重不差其母本屠家女也故太子好之【古者擇女必求之名門取其幽閒令淑者良有以也好呼到翻】東宫月俸錢五十萬【俸扶用翻】太子常探取二月用之猶不足【探叶南翻又他紺翻探取預取也】又令西園賣葵菜藍子雞麵等物而收其利【葵亦菜也魯相公儀休抜園葵漆室氏女曰晉客馬踐吾葵使吾終歲不食葵是也藍盧甘翻草可以染青者也本草圖經曰藍實人家蔬圃中作畦種蒔三月四月生苗高三四尺許葉似水蓼花紅白色實亦若蓼子而大黑色五月六月採實麫屑麥為之】又好隂陽小數多所拘忌【班固曰隂陽家蓋出於羲和之官敬順昊天歷象日月星辰敬授人時此其所長也及拘者為之則牽於禁忌泥於小數捨人事而任鬼神】洗馬江統上書陳五事一曰雖有微苦宜力疾朝侍【苦亦疾也朝直遥翻】二曰宜勤見保傅咨詢善道三曰畫室之功可宜減省【畫室以五采繪畫室屋也畫與盡同】後園刻鏤雜作一皆罷遣【鏤郎豆翻】四曰西園賣葵藍之屬虧敗國體貶損令聞【敗補邁翻聞音問】五曰繕牆正瓦不必拘攣小忌【攣閭緑翻】太子皆不從中舍人杜錫【晉志太子中舍人四人咸寧四年置以舍人才學美者為之與中庶子共掌文翰職如黄門侍郎在中庶子下洗馬上】恐太子不得安其位每盡忠諫勸太子修德業保令名言辭懇切太子患之置針著錫常所坐氈中【著陟畧翻】刺之血流【刺七亦翻】錫預之子也【杜預武帝時建平吴之功】太子性剛知賈謐恃中宫驕貴不能假借之謐時為侍中至東宫或捨之於後庭遊戲詹事裴權諫曰【詹事秦官掌太子家晉初未置詹事宫事無大小皆由二傅咸寧元年置詹事掌宫事二傅不復令官屬】謐后所親昵【昵尼質翻】一旦交構則事危矣不從謐譛太子於后曰太子多畜私財以結小人者為賈氏故也【為于偽翻】若宫車晏駕彼居大位依楊氏故事誅臣等廢后於金墉如反手耳【賈后殺楊駿廢太后天地之所不容也觀其姑姪之間所言若此則其心固不能一息安也】不如早圖之更立慈順者可以自安【更工衡翻】后納其言乃宣揚太子之短布於遠近又詐為有娠【娠升人翻孕也】内藁物產具取妹夫韓壽子慰祖養之欲以代太子於是朝野咸知賈后有害太子之意中護軍趙俊請太子廢后太子不聽左衛率東平劉卞以賈后之謀問張華【帝在東宫置衛率初曰中衛率泰始五年分為左右各領一軍愍懷在東宫又加前後二率謂之四率率所律翻】華曰不聞卞曰卞自須昌小吏受公成抜以至今日【須昌縣屬東平國卞自縣小吏從令入洛歷官至左衛率】士感知己是以盡言而公更有疑於卞邪華曰假令有此君欲如何卞曰東宫俊乂如林【時江統潘滔王敦等皆為東宫官屬馬融曰才過千人曰俊百人曰乂】四率精兵萬人公居阿衡之任若得公命皇太子因朝入録尚書事【朝直遥翻下同】廢賈后於金墉城兩黄門力耳華曰今天子當陽太子人子也吾又不受阿衡之命【華自言事任不可以伊尹自居】忽相與行此是無君父而以不孝示天下也况權戚滿朝威柄不一成可必乎賈后常使親黨微服聽察於外頗聞卞言乃遷卞為雍州刺史【雍於用翻】卞知言泄飲藥而死【賈后剛悍使閒卞言而張華不以告則華必死于賈后之手意卞言實華泄之也】十二月太子長子虨病【長知兩翻虨甫斤翻又方□翻】太子為虨求王爵不許虨疾篤太子為之禱祀求福【為于偽翻】賈后聞之乃詐稱帝不豫召太子入朝既至后不見置于别室遣婢陳舞以帝命賜太子酒三升使盡飲之太子辭以不能飲三升舞逼之曰不孝邪天賜汝酒而不飲【臣子以君父為天故以君父之賜為天賜】酒中有惡物邪太子不得已彊飲至盡【彊其兩翻】遂大醉后使黄門侍郎潘岳作書草【潘岳此事自當赤族其後天假手於孫秀耳】令小婢承福以紙筆及草因太子醉稱詔使書之文曰陛下宜自了不自了吾當入了之中宫又宜速自了不自了吾當手了之并與謝妃共要刻期兩發勿疑猶豫以致後患茹毛飲血於三辰之下皇天許當掃除患害立道文為王蔣氏為内主願成當以三牲祠北君太子醉迷不覺遂依而寫之【謝妃太子母也要約也言并以書與謝妃約刻期内外俱發也茹毛飲血謂盟誓也虨字道文蔣氏虨母蔣保林也内主言將立為后也三牲牛羊豕也北君北帝也按此書不惟無證佐使常人觀之亦知其偽為而不可信晉朝王公卿尚書黄散視而不敢言張華之諫實亦不敢發賈氏之姦姑引古義依違而言之耳裴頠請檢校傳書者賈氏之姦無所逃矣而亦不敢竟其說上下相蒙宜其大亂也】其字半不成后補成之以呈帝壬戍帝幸式乾殿召公卿入使黄門令董猛以太子書及青紙詔示之曰遹書如此令賜死徧示諸公王莫有言者【諸公王宗室諸王之為公者】張華曰此國之大禍自古以來常因廢黜正嫡以致喪亂【喪息浪翻】且國家有天下日淺願陛下詳之裴頠以為宜先檢校傳書者又請比校太子手書不然恐有詐妄賈后乃出太子啟事十餘紙衆人比視亦無敢言非者賈后使董猛矯以長廣公主辭白帝曰【長廣公主武帝女下嫁甄德】事宜速决而羣臣各不同其不從詔者宜以軍法從事【欲以此言脅羣臣也】議至日西不决后見華等意堅懼事變乃表免太子為庶人詔許之於是使尚書和郁等持節詣東宫廢太子為庶人太子改服出拜受詔步出承華門【承華門東宫門也陸機詩所謂振纓承華是也】乘麤犢車東武公澹以兵仗送太子及妃王氏三子虨臧尚同幽于金墉城王衍自表離婚許之妃慟哭而歸【清談之禍起於何晏何晏猶與曹爽同禍福若王衍者又不逮何晏矣】殺太子母謝淑媛及虨母保林蔣俊【保林良娣漢六宫十四等之數魏晉以下為東宫女官品秩師古曰保林言其可安衆如林也】<br />
<br />
  永康元年春正月癸亥朔 【考異曰帝紀天文志皆有己卯日食宋志無之按長歷己卯十七日安得日食】赦天下改元西戎校尉司馬閻纘【武帝置南蠻校尉於襄陽西戎校尉於長安南夷校尉於寧州各有長史司馬】輿棺詣闕上書以為漢戾太子稱兵拒命言者猶曰罪當笞耳【事見二十二卷漢武帝征和二年三年】今遹受罪之日不敢失道猶為輕於戾太子宜重選師傅【重再也重直龍翻】先加嚴誨若不悛改弃之未晩也【悛丑緑翻】書奏不省【省悉景翻】纘圃之孫也【閻圃見六十七卷漢獻帝建安二十年】賈后使黄門自首【首式救翻下同】欲與太子為逆詔以黄門首辭班示公卿遣東武公澹以千兵防衛太子幽于許昌宫令持書御史劉振持節守之【持書御史即治書侍御史】詔宫臣不得辭送洗馬江統潘滔舍人王敦杜蕤魯瑶等冒禁至伊水拜辭涕泣【晉志太子舍人十六人職比散騎中書等侍郎水經注伊水過伊闕中東北至洛陽縣南北入于洛】司隸校尉滿奮收縛統等送獄其繫河南獄者樂廣悉解遣之【樂廣時為河南尹】繫洛陽縣獄者猶未釋【付郡者河南尹得解遣之繫洛陽獄者尹不得與故未釋】都官從事孫琰說賈謐曰所以廢徙太子以其為惡故耳今宫臣冒罪拜辭而加以重辟流聞四方乃更彰太子之德也【說輸芮翻下乃說辟黜亦翻聞音問】不如釋之謐乃語洛陽令曹攄使釋之【攄抽居翻語牛倨翻】廣亦不坐敦覽之孫【王覽見七十七卷魏高貴鄉公甘露元年】攄肇之孫也【曹肇見七十四卷魏明帝景初二年】太子至許遺王妃書【遺于季翻】自陳誣枉妃父衍不敢以聞 丙子皇孫虨卒【非疾也 考異曰帝紀虨作霖按虨字道文不當作霖今從傳】 三月尉氏雨血【尉氏縣自漢以來屬陳留郡應劭曰古獄官曰尉氏鄭之别獄也臣瓚曰鄭大夫尉氏之邑故以為邑名師古曰鄭大夫尉氏亦以掌獄之官故為族耳應說是也雨于具翻】妖星見南方【星見妖而不知其名故但曰妖星妖於驕翻見賢遍翻下同】太白書見【晉天文志曰太白晝見與日爭明彊國弱小國彊女主昌】中台星拆【史記天官書曰魁下六星兩兩而比者曰三台三台色齊君臣和不齊君臣乖戾漢天文志曰三台曰泰階上階上星為天子下星為女主中階上星為諸侯三公下星為卿大夫下階上星為士下星為庶人拆者兩星不相比也】張華少子韙勸華遜位【少詩照翻韙羽委翻】華不從曰天道幽遠不如靜以待之【華所謂靜以待之者欲何所待也】 太子既廢衆情憤怒右衛督司馬雅常從督許超皆嘗給事東宫與殿中中郎士猗等【右衛督常從督殿中中郎皆屬二衛武帝甚重兵官殿中軍校多選朝廷凊望之士居之司馬雅宗室之疎屬也從才用翻】謀廢賈后復太子以張華裴頠安常保位難與行權右軍將軍趙王倫執兵柄性貪冒【冒密北翻】可假以濟事乃說孫秀曰【說輸芮翻】中宫凶妬無道與賈謐等共誣廢太子今國無嫡嗣社稷將危大臣將起大事而公名奉事中宫與賈郭親善太子之廢皆云豫知【言倫秀豫知廢太子之謀】一朝事起禍必相及何不先謀之乎秀許諾言於倫倫納焉遂告通事令史張林【通事令史中書令史也中書侍郎本通事郎官名雖改令史猶以通事冠之陸機惠帝起居注曰張林者黑山賊張燕之曾孫】及省事張衡等【省事亦吏職也賈充為尚書令以目疾表置省事吏四員省事蓋自此始省悉景翻】使為内應事將起孫秀言於倫曰太子聰明剛猛若還東宫必不受制於人明公素黨於賈后道路皆知之今雖建大功於太子太子謂公特逼於百姓之望翻覆以免罪耳【言百姓望太子復倫等畏逼故背賈氏復太子以求自免罪】雖含忍宿忿必不能深德明公若有瑕釁猶不免誅不若遷延緩期【遲其事而遷延未發也】賈后必害太子然後廢賈后為太子報讐【為于偽翻】非徒免禍而已乃更可以得志倫然之秀因使人行反間【間古莧翻】言殿中人欲廢皇后立太子【司馬雅許超士猗皆殿中人也】賈后數遣宫婢微服於民間聽察【數所角翻】聞之甚懼倫秀因勸謐等早除太子以絶衆望癸未賈后使太醫令程據和毒藥【和戶卧翻】矯詔使黄門孫慮至許昌毒太子太子自廢黜恐被毒常自煮食於前【被皮義翻】慮以告劉振振乃徙太子於小坊中絶其食宫人猶竊於牆上過食與之慮逼太子以藥太子不肯服慮以藥杵椎殺之【椎傳追翻】有司請以庶人禮葬賈后表請以廣陵王禮葬之 夏四月辛卯朔日有食之 趙王倫孫秀將討賈后告右衛佽飛督閭和【晉制右衛有佽飛虎賁二督佽飛荆人赴江斬蛟古勇士也自漢以來以為衛士之號佽日四翻】和從之期以癸巳丙夜一籌以鼓聲為應【丙夜夜三鼓丙夜一籌三更一點也】癸巳秀使司馬雅告張華曰趙王欲與公共匡社稷為天下除害【為于偽翻】使雅以告華拒之雅怒曰刃將在頸猶為是言邪不顧而出【華素有籌略雅辭氣之悖如此而無以處之蓋亦知衆怒不可遏而已為賈后用心不敢背之搏手無策待死而已】及期倫矯詔勑三部司馬曰【晉二衛有前驅由基彊弩三部司馬】中宫與賈謐等殺吾太子今使車騎入廢中宫【時趙王倫以車騎將軍領右軍將軍】汝等皆當從命事畢賜爵關中侯不從者誅三族衆皆從之又矯詔開門夜入陳兵道南【御道之南也】遣翊軍校尉齊王冏【武帝太康元年置翊軍校尉冏居永翻】將百人排閤而入華林令駱休為内應【華林令華林園令也魏起芳林園後避齊王芳諱改曰華林園有天淵池池中有魏文帝九花叢殿晉志華林令屬大鴻臚姓譜齊太公之後有公子駱子孫以為氏又秦之先有大駱】迎帝幸東堂以詔召賈謐於殿前將誅之謐走入西鍾下呼曰阿后救我【呼大故翻阿今相傳從安入聲】就斬之賈后見齊王冏驚曰卿何為來冏曰有詔收后后曰詔當從我出何詔也后至上閤遥呼帝曰陛下有婦使人廢之亦行自廢矣是時梁王肜亦預其謀后問冏曰起事者誰冏曰梁趙后曰繫狗當繫頸反繫其尾何得不然【恨不先誅梁趙也】遂廢后為庶人幽之於建始殿收趙粲賈午等付暴室考竟【晉志暴室令屬光禄勲】詔尚書收捕賈氏親黨召中書監侍中黄門侍郎八座皆夜入殿尚書始疑詔有詐郎師景露版奏請手詔【郎尚書郎也師姓景名】倫等斬之以徇倫隂與秀謀簒位欲先除朝望【朝直遥翻】且報宿怨乃執張華裴頠解系解結等於殿前【倫秀怨華頠系事見上卷元康六年結系弟也秀亂關中結議秀罪應誅故亦怨之】華謂張林曰卿欲害忠臣邪林稱詔詰之曰【詰去吉翻】卿為宰相太子之廢不能死節何也華曰式乾之議臣諫事具存可覆按也林曰諫而不從何不去位華無以對遂皆斬之仍夷三族解結女適裴氏明日當嫁而禍起裴氏欲認活之女曰家既如此我何以活為亦坐死朝廷由是議革舊制女不從死【不從父母家同坐死也】甲午倫坐端門【宫門正南門曰端門】遣尚書和郁持節送賈庶人于金墉【楊太后太子遹之廢史皆不書為庶人此獨書賈庶人者正其罪也】誅劉振董猛孫慮程據等司徒王戎及内外官坐張裴親黨黜免者甚衆閻纘撫張華尸慟哭曰早語君遜位而不肯今果不免命也【語牛倨翻】於是趙王倫稱詔赦天下自為使持節都督中外諸軍事相國侍中一依宣文輔魏故事【晉志曰丞相相國秦官也晉受魏禪並不置自惠帝之後省置無恒為之者趙王倫梁王肜成都王頴南陽王保王敦王導之徒非復人臣之職也今按宣王懿以丞相輔魏文王昭以相國輔魏皆非人臣之職】置府兵萬人以其世子散騎常侍荂領冗從僕射【荂枯花翻楊正衡音孚晉志冗從僕射屬先禄勲從才用翻】子馥為前將軍封濟陽王【濟子禮翻】䖍為黄門郎封汝隂王詡為散騎侍郎封霸城侯【黄門郎即黄門侍郎散騎侍郎魏初與散騎常侍同置自魏至晉散騎常侍侍郎與侍中黄門侍郎共平尚書奏事皆要官也】孫秀等皆封大郡並據兵權文武官封侯者數千人百官總己以聽於倫【朱氏曰總己謂總攝已職】倫素庸愚復受制于孫秀【復扶又翻】秀為中書令威權振朝廷天下皆事秀而無求於倫詔追復故太子遹位號使尚書和郁帥東宫官屬迎太子喪于許昌【帥讀曰率】追封遹子虨為南陽王封虨弟臧為臨淮王尚為襄陽王有司奏尚書令王衍備位大臣太子被誣志在苟免【謂太子遺王妃書自陳誣枉衍不敢以聞也】請禁錮終身從之相國倫欲收人望選用海内名德之士以前平陽太守李重滎陽太守荀組為左右長史東平王堪沛國劉謨為左右司馬尚書郎陽平東晳為記室【魏文帝黄初二年分魏郡置陽平郡記室主文翰束晳漢太子太傅疎廣之後廣曾孫避難因去疎字之足改姓為束續漢志曰記室主上章表報書記】淮南王文學荀崧殿中郎陸機為參軍【殿中郎尚書郎也主殿中曹】組朂之子【朂為晉初佐命之臣】崧彧之玄孫也【荀彧為魏初佐命之官】李重知倫有異志辭疾不就倫逼之不已憂憤成疾扶曳受拜數日而卒 丁酉以梁王肜為太宰左光禄大夫何劭為司徒右光禄大夫劉寔為司空【晉志左右光禄大夫假金章紫綬品秩第二禄賜班位冠幘車服佩玉置吏卒羽林後之金紫光禄大夫蓋魏晉之左右光禄大夫也但魏晉之大夫皆為專官後世則為寄禄官耳杜佑曰魏晉以來左右光禄三大夫皆銀印青綬其重者詔加金章紫綬者則謂之金紫光禄大夫重者既有金紫之號故謂本光禄為銀青光禄大夫】 太子遹之廢也將立淮南王允為太弟議者不合【言有持異議者也】會趙王倫廢賈后乃以允為驃騎將軍開府儀同三司領中護軍 己亥相國倫矯詔遣尚書劉宏齎金屑酒賜賈后死于金墉城 五月己巳詔立臨海王臧為皇太孫還妃王氏以母之【太子之廢也歸王妃于父母家】太子官屬即轉為太孫官屬相國倫行太孫太傅 己卯謚故太子曰愍懷六月壬寅葬于顯平陵 清河康王遐薨 中護軍淮南王允性沈毅【沈持林翻】宿衛將士皆畏服之允知相國倫及孫秀有異志隂養死士謀討之倫秀深憚之秋八月轉允為太尉外示優崇實奪其兵權【中護軍掌兵轉太尉則兵權去矣】允稱疾不拜秀遣御史劉機逼允收其官屬以下劾以拒詔大逆不敬【劾戶槩翻又戶得翻】允視詔乃秀手書也大怒收御史將斬之御史走免斬其令史二人【此蘭臺令史也】厲色謂左右曰趙王欲破我家遂帥國兵及帳下七百人直出【國兵淮南國兵也帳下中護軍帳下也帥讀曰率】大呼曰【呼火故翻】趙王反我將討之從我者左袒於是歸之者甚衆允將赴宫尚書左丞王輿閉掖門【宫門端門之左曰左掖門右曰右掖門】允不得入遂圍相府【時倫以東宫為相府】允所將兵皆精銳【將即亮翻】倫與戰屢敗死者千餘人太子左率陳徽勒東宫兵鼓譟於内以應允【左率即左衛率】允結陳於承華門前弓弩齊發射倫飛矢雨下【陳讀曰陣射而亦翻】主書司馬眭袐以身蔽倫【續漢志尚書三十六曹郎曹有三主書此主書司馬蓋相國府官屬倫所自署置眭息隨翻姓也】箭中其背而死【中竹仲翻下同】倫官屬皆隱樹而立每樹輒中數百箭自辰至未中書令陳淮【前有中書令陳準淮蓋準字之誤也】徽之兄也欲應允言於帝曰宜遣白虎幡以解鬬【白虎幡以麾軍進戰非以解鬬也陳準蓋以帝庸愚故請以白虎幡麾軍欲倫兵見之以為允之攻倫出於帝命將自潰也否則何以應允】乃使司馬督護伏胤將騎四百持幡從宫中出【司馬督護亦殿中將校屬二衛】侍中汝隂王䖍在門下省隂與胤誓曰富貴當與卿共之胤乃懷空板出【空板不書詔之板本無詔書而别取空板懷之以出也】詐言有詔助淮南王允不之覺開陣内之下車受詔胤因殺之并殺允子秦王郁漢王廸坐允夷滅者數千人曲赦洛陽【不普赦天下而獨赦洛陽故曰曲赦】初孫秀嘗為小吏事黄門郎潘岳岳屢撻之【孫秀琅邪人潘岳為琅邪内史秀為小吏給岳狡黠自喜岳惡其為人數撻辱之】衛尉石崇之甥歐陽建素與相國倫有隙【建表倫罪惡見上卷元康六年】崇有愛妾曰緑珠【緑珠善吹笛太平廣記曰今白州雙角山下有緑珠井昔梁氏之女有容貌石崇使交州以眞珠三斛買之梁氏之居舊井存焉汲飲者必誕美女里閭以美女無益遂以石填之】孫秀使求之崇不與及淮南王允敗秀因稱石崇潘岳歐陽建奉允為亂收之 【考異曰崇傳曰崇建潜知其計隂勸淮南王允齊王冏圖趙王倫若崇果與允同謀允敗崇應惶懼不應被收時方宴于樓上蓋倫秀以舊怨誣殺之耳 今按石崇傳孫秀索緑珠崇不許秀怒乃勸倫誅崇崇正宴於樓上介士到門崇謂緑珠曰我今為爾得罪緑珠泣曰當效死于君前因自投於樓下而死】崇歎曰奴輩利吾財爾收者曰知財為禍何不早散之崇不能答初潘岳母常誚責岳曰汝當知足而乾没不已乎【蓋戒岳乘時射利不知止也服䖍曰乾没射成敗也如淳曰得利為乾失利為没乾音干一說以水為喻也言其視利而趨雖乾而在陸没而滅頂皆所不顧也】及敗岳謝母曰負阿母【阿從安入聲】遂與崇建皆族誅籍没崇家相國倫收淮南王母弟吳王晏欲殺之光禄大夫傅祗爭之於朝堂【朝直遥翻】衆皆諫止倫乃貶晏為賓徒縣王【賓徒縣前漢屬遼西郡後漢屬遼東屬國都尉晉屬昌黎郡】齊王冏以功遷游擊將軍【晉志驍騎將軍游擊將軍並漢雜號將軍也魏置為中軍及晉以領護左右衛驍騎游擊為六軍】冏意不滿有恨色孫秀覺之且憚其在内乃出為平東將軍鎮許昌【為冏自許昌起兵討倫張本】 以光禄大夫陳準為太尉録尚書事未幾薨 孫秀議加相國倫九錫百官莫敢異議吏部尚書劉頌曰昔漢之錫魏魏之錫晉皆一時之用非可通行【謂禪代然後有九錫非常典也】周勃霍光其功至大皆不聞有九錫之命也【謂周勃霍光定策以安漢室且不聞有九錫之命所以折倫秀之姦謀也】張林積忿不已以頌為張華之黨將殺之孫秀曰殺張裴已傷時望不可復殺頌林乃止【復扶又翻下同】以頌為光禄大夫【晉志光禄大夫與卿同秩中二千石著進賢兩梁冠黑介幘五時朝服佩水蒼玉 考異曰三十國春秋云倫黨大怒謀害頌頌懼自殺頌傳云頌為光禄尋病卒今從傳】遂下詔加倫九錫復加其子荂撫軍將軍【撫軍將軍文帝以授武帝遂以代魏倫以加其世子意趣為何】䖍中軍將軍【武帝受禪置中軍將軍統宿衛七軍尋罷已而復置】詡為侍中又加孫秀侍中輔國將軍相國司馬右率如故【右率右衛率也不解此官者欲握東宫兵】張林等並居顯要增相府兵為二萬人與宿衛同并所隱匿之兵數踰三萬九月改司徒為丞相以梁王肜為之肜固辭不受倫及諸子皆頑鄙無識秀狡黠貪淫【黠下入翻】所與共事者皆邪佞之士惟競榮利無遠謀深畧志趣乖異互相憎嫉秀子會為射聲校尉形貌短陋如奴僕之下者秀使尚帝女河東公主【史言倫秀兵已在頸乃圖非望】冬十一月甲子立皇后羊氏赦天下后尚書郎泰山羊玄之之女也外祖平南將軍樂安孫旂與孫秀善故秀立之拜玄之光禄大夫特進散騎常侍封興晉侯【晉志光禄大夫假銀章青綬者品秩第三加特進則品秩與左右光禄大夫同矣晉置興晉郡在唐河州界】 詔徵益州刺史趙廞為大長秋【廞許今翻】以成都内史中山耿滕為益州刺史【晉諸王國置内史猶漢王國相也武帝大康九年改諸王國相為内史 考異曰帝紀作耿勝載記華陽國志作滕今從之】廞賈后之姻親也聞徵甚懼【懼以賈后親黨連坐】且以晉室衰亂隂有據蜀之志乃傾倉廩賑流民以收衆心以李特兄弟材武其黨類皆巴西人與廞同郡【李特黨類本巴氐趙廞亦巴西人也】厚遇之以為爪牙特等憑恃廞埶專聚衆為盜蜀人患之【特等入蜀事始上卷元康八年】滕數密表流民剛剽蜀人愞弱【數所角翻剽匹妙翻愞奴亂翻】主不能制客必為亂階宜使還本居若留之險地【蜀地阻險】恐秦雍之禍更移于梁益矣【流民本居秦雍雍於用翻】廞聞而惡之【惡烏路翻】州被詔書遣文武千餘人迎滕是時成都治少城益州治太城【二城皆秦張儀所築儀既築太城後一年又築少城太城今成都府子城也少城唯西南北三壁東即太城之西墉也少詩照翻】廞猶在太城未去滕欲入州功曹陳恂諫曰今州郡搆怨日深【州謂益州郡謂成都此言廞滕搆怨也】入城必有大禍不如留少城以觀其變檄諸縣合村保以備秦氐【李特等本巴氐蜀人以其徙居秦州界因謂之秦氐】陳西夷行至【陳西夷謂西夷校尉陳總也行至言總來領西夷校尉之職行且至成都也晉置西夷校尉於汶山平越中郎將於廣州南蠻校尉于襄陽南夷校尉于寧州】且當待之不然退保犍為西渡江源以防非常【江源縣漢屬蜀郡後李雄分立江源郡晉改為多融縣又改為晉原縣唐蜀州之晉原青城唐安三縣皆漢江源縣地犍居言翻】滕不從是日帥衆入州【帥讀曰率下同】廞遣兵逆之戰于西門滕敗死 【考異曰華陽國志曰戰于廣漢宣化亭殺傳詔按州郡俱治成都不容戰于廣漢又趙廞若已與滕戰不應欲直入州今從載記】郡吏皆竄走惟陳恂面縛詣廞請滕死【請其屍而葬之死讀曰屍】廞義而許之廞又遣兵逆西夷校尉陳摠摠至江陽【江陽縣漢屬犍為郡劉璋分江陽郡唐瀘州瀘川綿水二縣漢江陽之地也】聞廞有異志主簿蜀郡趙模曰今州郡不協必生大變當速行赴之府是兵要助順討逆【言西夷府摠蜀兵之要順謂耿滕逆謂趙廞使摠助滕討廞也】誰敢動者摠更緣道停留比至南安魚涪津【南安縣屬犍為郡有魚涪津唐眉州青神縣漢南安縣地宋白曰榮州應靈縣資官縣嘉州龍川縣皆漢安南縣比音必寐翻涪音浮】已遇廞軍模白摠散財募士以拒戰若克州軍則州可得【言破廞軍則益州可取罪人斯得矣】不克順流而退必無害也【言順流而退廞軍埶不能追必無所害】摠曰趙益州忿耿侯故殺之與吾無嫌何為如此【兵臨其前猶發是言陳摠特庸人耳】模曰今州起事必當殺君以立威雖不戰無益也言至垂涕摠不聽衆遂自潰摠逃草中模著摠服格戰【著陟畧翻】廞兵殺模見其非是更搜求得摠殺之【搜尋也 考異曰帝紀廞又殺犍為太守李密汶山太守霍固按華陽國志犍為太守李苾汶山太守楊邠非密固也載記亦作李苾蓋紀誤】廞自稱大都督大將軍益州牧 【考異曰晉春秋云建號太平元年他書無之今不取】署置僚屬改易守令王官被召無敢不往【王官謂晉朝所命者被皮義翻】李庠帥妹壻李含天水任回上官晶扶風李攀始平費他【帥讀曰率楊正衡曰晶音精武帝泰始二年分扶風置始平郡費扶沸翻他徒河翻】氐苻成隗伯等四千騎歸廞廞以庠為威寇將軍【沈約志威寇將軍四十號之第七】封陽泉亭侯委以心膂使招合六郡壯勇至萬餘人以斷北道【六郡即天水畧陽等六郡壯勇流民之壯勇者北道自關中入蜀之道斷丁管翻】<br />
<br />
  資治通鑑卷八十三  <br>
   </div> 

<script src="/search/ajaxskft.js"> </script>
 <div class="clear"></div>
<br>
<br>
 <!-- a.d-->

 <!--
<div class="info_share">
</div> 
-->
 <!--info_share--></div>   <!-- end info_content-->
  </div> <!-- end l-->

<div class="r">   <!--r-->



<div class="sidebar"  style="margin-bottom:2px;">

 
<div class="sidebar_title">工具类大全</div>
<div class="sidebar_info">
<strong><a href="http://www.guoxuedashi.com/lsditu/" target="_blank">历史地图</a></strong>  
<a href="http://www.880114.com/" target="_blank">英语宝典</a>  
<a href="http://www.guoxuedashi.com/13jing/" target="_blank">十三经检索</a> 
<br><strong><a href="http://www.guoxuedashi.com/gjtsjc/" target="_blank">古今图书集成</a></strong> 
<a href="http://www.guoxuedashi.com/duilian/" target="_blank">对联大全</a> <strong><a href="http://www.guoxuedashi.com/xiangxingzi/" target="_blank">象形文字典</a></strong> 

<br><a href="http://www.guoxuedashi.com/zixing/yanbian/">字形演变</a>  <strong><a href="http://www.guoxuemi.com/hafo/" target="_blank">哈佛燕京中文善本特藏</a></strong>
<br><strong><a href="http://www.guoxuedashi.com/csfz/" target="_blank">丛书&方志检索器</a></strong> <a href="http://www.guoxuedashi.com/yqjyy/" target="_blank">一切经音义</a>  

<br><strong><a href="http://www.guoxuedashi.com/jiapu/" target="_blank">家谱族谱查询</a></strong>  <strong><a href="http://shufa.guoxuedashi.com/sfzitie/" target="_blank">书法字帖欣赏</a></strong> 
<br>

</div>
</div>


<div class="sidebar" style="margin-bottom:0px;">

<font style="font-size:22px;line-height:32px">QQ交流群9:489193090</font>


<div class="sidebar_title">手机APP 扫描或点击</div>
<div class="sidebar_info">
<table>
<tr>
	<td width=160><a href="http://m.guoxuedashi.com/app/" target="_blank"><img src="/img/gxds-sj.png" width="140"  border="0" alt="国学大师手机版"></a></td>
	<td>
<a href="http://www.guoxuedashi.com/download/" target="_blank">app软件下载专区</a><br>
<a href="http://www.guoxuedashi.com/download/gxds.php" target="_blank">《国学大师》下载</a><br>
<a href="http://www.guoxuedashi.com/download/kxzd.php" target="_blank">《汉字宝典》下载</a><br>
<a href="http://www.guoxuedashi.com/download/scqbd.php" target="_blank">《诗词曲宝典》下载</a><br>
<a href="http://www.guoxuedashi.com/SiKuQuanShu/skqs.php" target="_blank">《四库全书》下载</a><br>
</td>
</tr>
</table>

</div>
</div>


<div class="sidebar2">
<center>


</center>
</div>

<div class="sidebar"  style="margin-bottom:2px;">
<div class="sidebar_title">网站使用教程</div>
<div class="sidebar_info">
<a href="http://www.guoxuedashi.com/help/gjsearch.php" target="_blank">如何在国学大师网下载古籍?</a><br>
<a href="http://www.guoxuedashi.com/zidian/bujian/bjjc.php" target="_blank">如何使用部件查字法快速查字?</a><br>
<a href="http://www.guoxuedashi.com/search/sjc.php" target="_blank">如何在指定的书籍中全文检索?</a><br>
<a href="http://www.guoxuedashi.com/search/skjc.php" target="_blank">如何找到一句话在《四库全书》哪一页?</a><br>
</div>
</div>


<div class="sidebar">
<div class="sidebar_title">热门书籍</div>
<div class="sidebar_info">
<a href="/so.php?sokey=%E8%B5%84%E6%B2%BB%E9%80%9A%E9%89%B4&kt=1">资治通鉴</a> <a href="/24shi/"><strong>二十四史</strong></a>&nbsp; <a href="/a2694/">野史</a>&nbsp; <a href="/SiKuQuanShu/"><strong>四库全书</strong></a>&nbsp;<a href="http://www.guoxuedashi.com/SiKuQuanShu/fanti/">繁体</a>
<br><a href="/so.php?sokey=%E7%BA%A2%E6%A5%BC%E6%A2%A6&kt=1">红楼梦</a> <a href="/a/1858x/">三国演义</a> <a href="/a/1038k/">水浒传</a> <a href="/a/1046t/">西游记</a> <a href="/a/1914o/">封神演义</a>
<br>
<a href="http://www.guoxuedashi.com/so.php?sokeygx=%E4%B8%87%E6%9C%89%E6%96%87%E5%BA%93&submit=&kt=1">万有文库</a> <a href="/a/780t/">古文观止</a> <a href="/a/1024l/">文心雕龙</a> <a href="/a/1704n/">全唐诗</a> <a href="/a/1705h/">全宋词</a>
<br><a href="http://www.guoxuedashi.com/so.php?sokeygx=%E7%99%BE%E8%A1%B2%E6%9C%AC%E4%BA%8C%E5%8D%81%E5%9B%9B%E5%8F%B2&submit=&kt=1"><strong>百衲本二十四史</strong></a>  <a href="http://www.guoxuedashi.com/so.php?sokeygx=%E5%8F%A4%E4%BB%8A%E5%9B%BE%E4%B9%A6%E9%9B%86%E6%88%90&submit=&kt=1"><strong>古今图书集成</strong></a>
<br>

<a href="http://www.guoxuedashi.com/so.php?sokeygx=%E4%B8%9B%E4%B9%A6%E9%9B%86%E6%88%90&submit=&kt=1">丛书集成</a> 
<a href="http://www.guoxuedashi.com/so.php?sokeygx=%E5%9B%9B%E9%83%A8%E4%B8%9B%E5%88%8A&submit=&kt=1"><strong>四部丛刊</strong></a>  
<a href="http://www.guoxuedashi.com/so.php?sokeygx=%E8%AF%B4%E6%96%87%E8%A7%A3%E5%AD%97&submit=&kt=1">說文解字</a> <a href="http://www.guoxuedashi.com/so.php?sokeygx=%E5%85%A8%E4%B8%8A%E5%8F%A4&submit=&kt=1">三国六朝文</a>
<br><a href="http://www.guoxuedashi.com/so.php?sokeytm=%E6%97%A5%E6%9C%AC%E5%86%85%E9%98%81%E6%96%87%E5%BA%93&submit=&kt=1"><strong>日本内阁文库</strong></a> <a href="http://www.guoxuedashi.com/so.php?sokeytm=%E5%9B%BD%E5%9B%BE%E6%96%B9%E5%BF%97%E5%90%88%E9%9B%86&ka=100&submit=">国图方志合集</a> <a href="http://www.guoxuedashi.com/so.php?sokeytm=%E5%90%84%E5%9C%B0%E6%96%B9%E5%BF%97&submit=&kt=1"><strong>各地方志</strong></a>

</div>
</div>


<div class="sidebar2">
<center>

</center>
</div>
<div class="sidebar greenbar">
<div class="sidebar_title green">四库全书</div>
<div class="sidebar_info">

《四库全书》是中国古代最大的丛书,编撰于乾隆年间,由纪昀等360多位高官、学者编撰,3800多人抄写,费时十三年编成。丛书分经、史、子、集四部,故名四库。共有3500多种书,7.9万卷,3.6万册,约8亿字,基本上囊括了古代所有图书,故称“全书”。<a href="http://www.guoxuedashi.com/SiKuQuanShu/">详细>>
</a>

</div> 
</div>

</div>  <!--end r-->

</div>
<!-- 内容区END --> 

<!-- 页脚开始 -->
<div class="shh">

</div>

<div class="w1180" style="margin-top:8px;">
<center><script src="http://www.guoxuedashi.com/img/plus.php?id=3"></script></center>
</div>
<div class="w1180 foot">
<a href="/b/thanks.php">特别致谢</a> | <a href="javascript:window.external.AddFavorite(document.location.href,document.title);">收藏本站</a> | <a href="#">欢迎投稿</a> | <a href="http://www.guoxuedashi.com/forum/">意见建议</a> | <a href="http://www.guoxuemi.com/">国学迷</a> | <a href="http://www.shuowen.net/">说文网</a><script language="javascript" type="text/javascript" src="https://js.users.51.la/17753172.js"></script><br />
  Copyright &copy; 国学大师 古典图书集成 All Rights Reserved.<br>
  
  <span style="font-size:14px">免责声明:本站非营利性站点,以方便网友为主,仅供学习研究。<br>内容由热心网友提供和网上收集,不保留版权。若侵犯了您的权益,来信即刪。scp168@qq.com</span>
  <br />
ICP证:<a href="http://www.beian.miit.gov.cn/" target="_blank">鲁ICP备19060063号</a></div>
<!-- 页脚END --> 
<script src="http://www.guoxuedashi.com/img/plus.php?id=22"></script>
<script src="http://www.guoxuedashi.com/img/tongji.js"></script>

</body>
</html>
