<!DOCTYPE html PUBLIC "-//W3C//DTD XHTML 1.0 Transitional//EN" "http://www.w3.org/TR/xhtml1/DTD/xhtml1-transitional.dtd">
<html xmlns="http://www.w3.org/1999/xhtml">
<head>
<meta http-equiv="Content-Type" content="text/html; charset=utf-8" />
<meta http-equiv="X-UA-Compatible" content="IE=Edge,chrome=1">
<title>資治通鑒_284-資治通鑑卷二百八十三_284-資治通鑑卷二百八十三</title>
<meta name="Keywords" content="資治通鑒_284-資治通鑑卷二百八十三_284-資治通鑑卷二百八十三">
<meta name="Description" content="資治通鑒_284-資治通鑑卷二百八十三_284-資治通鑑卷二百八十三">
<meta http-equiv="Cache-Control" content="no-transform" />
<meta http-equiv="Cache-Control" content="no-siteapp" />
<link href="/img/style.css" rel="stylesheet" type="text/css" />
<script src="/img/m.js?2020"></script> 
</head>
<body>
 <div class="ClassNavi">
<a  href="/24shi/">二十四史</a> | <a href="/SiKuQuanShu/">四库全书</a> | <a href="http://www.guoxuedashi.com/gjtsjc/"><font  color="#FF0000">古今图书集成</font></a> | <a href="/renwu/">历史人物</a> | <a href="/ShuoWenJieZi/"><font  color="#FF0000">说文解字</a></font> | <a href="/chengyu/">成语词典</a> | <a  target="_blank"  href="http://www.guoxuedashi.com/jgwhj/"><font  color="#FF0000">甲骨文合集</font></a> | <a href="/yzjwjc/"><font  color="#FF0000">殷周金文集成</font></a> | <a href="/xiangxingzi/"><font color="#0000FF">象形字典</font></a> | <a href="/13jing/"><font  color="#FF0000">十三经索引</font></a> | <a href="/zixing/"><font  color="#FF0000">字体转换器</font></a> | <a href="/zidian/xz/"><font color="#0000FF">篆书识别</font></a> | <a href="/jinfanyi/">近义反义词</a> | <a href="/duilian/">对联大全</a> | <a href="/jiapu/"><font  color="#0000FF">家谱族谱查询</font></a> | <a href="http://www.guoxuemi.com/hafo/" target="_blank" ><font color="#FF0000">哈佛古籍</font></a> 
</div>

 <!-- 头部导航开始 -->
<div class="w1180 head clearfix">
  <div class="head_logo l"><a title="国学大师官网" href="http://www.guoxuedashi.com" target="_blank"></a></div>
  <div class="head_sr l">
  <div id="head1">
  
  <a href="http://www.guoxuedashi.com/zidian/bujian/" target="_blank" ><img src="http://www.guoxuedashi.com/img/top1.gif" width="88" height="60" border="0" title="部件查字,支持20万汉字"></a>


<a href="http://www.guoxuedashi.com/help/yingpan.php" target="_blank"><img src="http://www.guoxuedashi.com/img/top230.gif" width="600" height="62" border="0" ></a>


  </div>
  <div id="head3"><a href="javascript:" onClick="javascript:window.external.AddFavorite(window.location.href,document.title);">添加收藏</a>
  <br><a href="/help/setie.php">搜索引擎</a>
  <br><a href="/help/zanzhu.php">赞助本站</a></div>
  <div id="head2">
 <a href="http://www.guoxuemi.com/" target="_blank"><img src="http://www.guoxuedashi.com/img/guoxuemi.gif" width="95" height="62" border="0" style="margin-left:2px;" title="国学迷"></a>
  

  </div>
</div>
  <div class="clear"></div>
  <div class="head_nav">
  <p><a href="/">首页</a> | <a href="/ShuKu/">国学书库</a> | <a href="/guji/">影印古籍</a> | <a href="/shici/">诗词宝典</a> | <a   href="/SiKuQuanShu/gxjx.php">精选</a> <b>|</b> <a href="/zidian/">汉语字典</a> | <a href="/hydcd/">汉语词典</a> | <a href="http://www.guoxuedashi.com/zidian/bujian/"><font  color="#CC0066">部件查字</font></a> | <a href="http://www.sfds.cn/"><font  color="#CC0066">书法大师</font></a> | <a href="/jgwhj/">甲骨文</a> <b>|</b> <a href="/b/4/"><font  color="#CC0066">解密</font></a> | <a href="/renwu/">历史人物</a> | <a href="/diangu/">历史典故</a> | <a href="/xingshi/">姓氏</a> | <a href="/minzu/">民族</a> <b>|</b> <a href="/mz/"><font  color="#CC0066">世界名著</font></a> | <a href="/download/">软件下载</a>
</p>
<p><a href="/b/"><font  color="#CC0066">历史</font></a> | <a href="http://skqs.guoxuedashi.com/" target="_blank">四库全书</a> |  <a href="http://www.guoxuedashi.com/search/" target="_blank"><font  color="#CC0066">全文检索</font></a> | <a href="http://www.guoxuedashi.com/shumu/">古籍书目</a> | <a   href="/24shi/">正史</a> <b>|</b> <a href="/chengyu/">成语词典</a> | <a href="/kangxi/" title="康熙字典">康熙字典</a> | <a href="/ShuoWenJieZi/">说文解字</a> | <a href="/zixing/yanbian/">字形演变</a> | <a href="/yzjwjc/">金 文</a> <b>|</b>  <a href="/shijian/nian-hao/">年号</a> | <a href="/diming/">历史地名</a> | <a href="/shijian/">历史事件</a> | <a href="/guanzhi/">官职</a> | <a href="/lishi/">知识</a> <b>|</b> <a href="/zhongyi/">中医中药</a> | <a href="http://www.guoxuedashi.com/forum/">留言反馈</a>
</p>
  </div>
</div>
<!-- 头部导航END --> 
<!-- 内容区开始 --> 
<div class="w1180 clearfix">
  <div class="info l">
   
<div class="clearfix" style="background:#f5faff;">
<script src='http://www.guoxuedashi.com/img/headersou.js'></script>

</div>
  <div class="info_tree"><a href="http://www.guoxuedashi.com">首页</a> > <a href="/SiKuQuanShu/fanti/">四库全书</a>
 > <h1>资治通鉴</h1> <!--         下载:【右键另存为】即可 --></div>
  <div class="info_content zj clearfix">
  
<div class="info_txt clearfix" id="show">
<center style="font-size:24px;">284-資治通鑑卷二百八十三</center>
    資治通鑑卷二百八十三 宋 司馬光 撰<br />
<br />
  胡三省 音註<br />
<br />
  後晋紀四【起玄黓攝提格盡閼逢執徐正月凡二年有奇】<br />
<br />
  高祖聖文章武明德孝皇帝下<br />
<br />
  天福七年春正月丁巳鎮州牙將自西郭水碾門導官軍入城【碾魚蹇翻水碾水磑也】殺守陴民二萬人【陴頻彌翻】執安重榮斬之杜重威殺導者自以為功庚申重榮首至鄴都帝命漆之函送契丹【重直龍翻】 癸亥改鎮州為恒州成德軍為順國軍【鎮州本恒州唐避穆宗名改焉今以安重榮反改州名從舊又改軍號恒胡登翻】丙寅以門下侍郎同平章事趙瑩為侍中以杜重威為順國節度使兼侍中安重榮私財及恒州府庫重威盡有之帝知而不問又表衛尉少卿范陽王瑜為副使瑜為之重斂於民恒人不勝其苦【少詩照翻為之于偽翻歛力贍翻勝音升】張式父鐸詣闕訟寃【張彦澤殺張式事見上卷上年】壬午以河陽節度使王周為彰義節度使代張彦澤 閩主曦立皇后李氏同平章事真之女也嗜酒剛愎【愎蒲逼翻】曦寵而憚之彰武節度使丁審琪養部曲千人縱之為暴於境内軍校賀行政與諸胡相結為亂攻延州帝遣曹州防禦使何重建將兵救之同鄜援兵繼至乃得免【校戶教翻鄜方無翻】二月癸巳以重建為彰武留後召審琪歸朝重建雲朔閒胡人也 唐左丞相宋齊丘固求豫政事唐主聽入中書又求領尚書省乃罷侍中壽王景遂判尚書省更領中書門下省【更工衡翻】以齊丘知尚書省事其三省事並取齊王璟參決【所以制宋齊丘】齊丘視事數月親吏夏昌圖盜官錢三千緡齊丘判貸其死唐主大怒斬昌圖【夏戶雅翻】齊丘稱疾請罷省事從之 涇州奏遣押牙陳延暉持敕書詣涼州州中將吏請延暉為節度使 三月閩主曦立長樂王亞澄為閩王【樂音洛】 張彦澤在涇州擅發兵擊諸胡兵皆敗沒調民馬千餘匹以補之【調徒釣翻】還至陜【自涇州代還至陜還從宣翻陜失冉翻】獲亡將楊洪乘醉斷其手足而斬之【斷音短】王周奏彦澤在鎮貪殘不法二十六條民散亡者五千餘戶【王周代彦澤故得奏其在鎮事】彦澤既至帝以其有軍功又與楊光遠連姻釋不問【歐史張彦澤與帝連姻又討范延光有功】夏四月己未右諫議大夫鄭受益上言楊洪所以被屠由陛下去歲送張式與彦澤使之逞志致彦澤敢肆凶殘無所忌憚見聞之人無不切齒而陛下曾不動心一無詰讓淑慝莫辨【詰去吉翻問也讓責也慝吐得翻】賞罰無章中外皆言陛下受彦澤所獻馬百匹聽其如是臣竊為陛下惜此惡名【為于偽翻受獻而釋有罪是惡名也】乞正彦澤罪法以湔洗聖德【湔則前翻】疏奏留中受益從讜之兄子也【鄭從讜見唐僖宗紀讜音黨】庚申刑部郎中李濤等伏閤極論彦澤之罪語甚切至【伏閤者伏閤門下奏事閤門使以聞】辛酉敕張彦澤削一階降爵一級【階武散階爵級封爵之級】張式父及子弟皆拜官涇州民復業者減其徭賦癸亥李濤復與兩省及御史臺官伏閤【復扶又翻兩省官中書門下省官也】奏彦澤罰太輕請論如法帝召濤面諭之濤端笏前迫殿陛聲色俱厲帝怒連叱之濤不退帝曰朕已許彦澤不死濤曰陛下許彦澤不死不可負不知范延光鐵劵安在【謂許范延光以不死而楊光遠殺之也事見上卷五年】帝拂衣起入禁中丙寅以彦澤為左龍武大將軍【為張彦澤為契丹用以殘滅晉國李濤詣彦澤而不懼張本】漢高祖寢疾以其子秦王弘度晉王弘熙皆驕恣少<br />
<br />
  子越王弘昌孝謹有智識【少詩照翻】與右僕射兼西御院使王翷謀出弘度鎮邕州弘熙鎮容州而立弘昌【為弘熙殺弘昌及翷張本翷求仁翻】制命將行會崇文使蕭益入問疾以其事訪之益曰立嫡以長違之必亂乃止【蕭益引經義以沮立弘昌之義長知兩翻】丁丑高祖殂【年五十四】高祖為人辨察多權數好自矜大常謂中國天子為洛州刺史【好呼到翻以中國天子都洛陽洛陽之地盖本洛州刺史所治也言其政今不能及遠特昔時洛州刺史之任耳】嶺南珍異所聚每窮奢極麗宫殿悉以金玉珠翠為飾用刑惨酷有灌鼻割舌支解刳剔炮炙烹蒸之法或聚毒蛇水中以罪人投之謂之水獄同平章事楊洞潜諫不聽末年尤猜忌以士人多為子孫計故專任宦官由是其國中宦者大盛【自劉龔之後專任宦者謂百官為門外人傳至于鋹而國亡矣】秦王弘度即皇帝位更名玢【更工衡翻玢府巾翻】以弘熙輔政改元光天尊母趙昭儀曰皇太妃 契丹以晉招納吐谷渾遣使來讓帝憂悒不知為計五月己亥始有疾【悒乙及翻】 己巳尊太妃劉氏為皇太后太后帝之庶母也【徐無黨曰帝所生母也】 唐丞相太保宋齊丘既罷尚書省不復朝謁【復扶又翻朝直遥翻下同】唐主遣壽王景遂勞問【勞力到翻】許鎮洪州始入朝唐主與之宴酒酣齊丘曰陛下中興臣之力也奈何忘之唐主怒曰公以遊客干朕【事見二百六十六卷梁太祖乾化二年】今為三公亦足矣乃與人言朕鳥喙如句踐難與共安樂有之乎【越范蠡遺文種書言越王為人長頸鳥喙可與同患難不可與同安樂句讀如鉤樂讀如洛】齊丘曰臣實有此言臣為遊客時陛下乃偏裨耳今日殺臣可矣明日唐主手詔謝之曰朕之褊性【褊補典翻】子嵩平昔所知少相親老相怨可乎【少詩照翻自古君臣之間豈無親故未有如宋齊丘之挾舊矜功唐主之啟寵納侮者也】丙午以齊丘為鎮南節度使【踐洪州之約宋齊丘本洪州進士寵之以衣錦也】 帝寢疾一旦馮道獨對帝命幼子重睿出拜之又令宦者抱重睿置道懷中其意蓋欲道輔立之【重直龍翻 考異曰漢高祖實録晉高祖大漸召近臣屬之曰此天下明宗之天下寡人竊而取之久矣寡人既謝當歸許王寡人之願也此說難信今從薛史】六月乙丑帝殂【年五十一五代會要殂于鄴都大内之保昌殿】道與天平節度使侍衛馬步都虞候景延廣議以國家多難宜立長君乃奉廣晉尹齊王重貴為嗣【晉高祖託孤于馮道與吳主孫休託孤於濮陽興張布之事畧同難乃旦翻】是日齊王即皇帝位延廣以為己功始用事禁都下人無得偶語【以防姦人謀為變】初高祖疾亟有旨召河東節度使劉知遠入輔政齊王寢之知遠由是怨齊王【為劉知遠不肯入援張本】 丁卯尊皇太后曰太皇太后【高祖之庶母劉氏也】皇后曰皇太后【高祖之后李氏也】 閩富沙王延政圍汀州閩主曦發漳泉兵五千救之【九域志泉州西至漳州二百九十五里漳州西至汀州五百四十里宋白曰梁山有漳浦水一名漳溪水唐垂拱二年析泉州之西南置漳州垂拱之泉州今之福州也】又遣其將林守亮入尤溪大明宫使黄敬忠屯尤口【九域志尤溪縣在南劍州南一百五十五里蓋王氏初置縣也尤口尤溪口也】欲乘虛襲建州國計使黄紹頗將步卒八千為二軍聲援 秋七月壬辰太皇太后劉氏殖 閩富沙王延政攻汀州四十二戰不克而歸其將包洪實陳望將水軍以禦福州之師丁酉遇於尤口【尤溪口】黄敬忠將戰占者言時刻未利按兵不動洪實等引兵登岸水陸夾攻之殺敬忠俘斬二千級林守亮黄紹頗皆遁歸 庚子大赦 癸卯加景延廣同平章事兼侍衛馬步都指揮使【賞其定策之功也為景延廣挾權制上搆契丹之隙張本】 勲舊皆欲復置樞密使【罷樞密使見上卷上年】馮道等三奏請以樞密舊職讓之【并樞密于中書故謂樞密院舊所典之職為舊職】帝不許 有神降於博羅縣民家【博羅漢古縣唐屬循州時為漢土郡國志循州有博羅山浮海而來傅著羅山故名博羅宋朝博羅縣屬惠州九域志在州北四十五里宋白曰博羅縣接境于羅山故曰博羅東接龍州南接西平西接增城界】與人言而不見其形閭閻人往占吉凶多驗縣吏張遇賢事之甚謹時循州盜賊羣起莫相統一賊帥共禱于神神大言曰張遇賢當為汝主於是共奉遇賢稱中天八國王改元永樂【樂音洛】置百官攻掠海隅【循州東南距潮惠二州皆海隅之地】遇賢年少【少詩照翻】無它方畧諸將但告進退而已漢主以越王弘昌為都統循王弘杲為副以討之戰于錢帛館漢兵不利二王皆為賊所圍指揮使陳道庠等力戰救之得免東方州縣多為遇賢所陷【東方州縣謂番禺以東州縣也即惠潮之地九域志廣州東至惠州三百一十五里又自惠州東至潮州八百一十里】道庠端州人也 高行周圍襄州踰年不下【去年十一月高行周圍襄州事始見上卷】城中食盡奉國軍都虞候曲周王清言於行周曰【曲周縣屬洺州宋熙寧三年省曲周縣為鎮入雞澤縣】賊城已危我師已老民力已困不早迫之尚何俟乎與奉國都指揮使元城劉詞帥衆先登【元城縣帶魏州帥讀曰率】八月拔之安從進舉族自焚 甲子以趙瑩為中書令 閩主曦遣使以手詔及金器九百錢萬緡將吏敕吿六百四十通求和於富沙王延政延政不受丙寅閩主曦宴羣臣於九龍殿從子繼柔不能飲強之【從才用翻強其兩翻】繼柔私減其酒曦怒并客將斬之【王曦之酗虐孫皓之流也將即亮翻】 閩主鑄永隆通寶大鐵錢一當鈆錢百 漢葬天皇大帝于康陵廟號高祖 唐主自為吳相興利除害變更舊法甚多【梁均王之貞明四年唐主始得吳政吳王隆演之十五年也】及即位命法官及尚書刪定為昇元條三十卷【時唐以昇元紀元】庚寅行之 閩主曦以同平章事侯官余廷英為泉州刺史廷英貪穢掠人女子詐稱受詔采擇以備後宫事覺曦遣御史按之廷英懼詣福州自歸曦詰責將以屬吏【詰去吉翻屬之欲翻】廷英退獻買宴錢萬緡曦悦明日召見謂曰宴已買矣皇后貢物安在廷英復獻錢於李后乃遣歸泉州自是諸州皆别貢皇后物未幾復召廷英為相【見賢遍翻復扶又翻幾居豈翻史言閩主曦之好貨甚于昶】 冬十月丙子張遇賢陷循州殺漢刺史劉傳 楚王希範作天策府【王舉天下大定録曰希範建天策府於州城西北造天策光政等一十六樓及造天策勤政等五堂】極棟宇之盛戶牖欄檻皆飾以金玉塗壁用丹砂數十萬斤【丹砂出辰溪溆錦等州及諸溪峒皆楚之境内也本草圖經曰丹砂生深谷石崖閒土人穴地數十尺始見其苗乃白石也謂之丹砂床砂生石上其塊大者如雞子小者如石榴顆狀若芙蓉頭箭鏃連床者紫黯若鐵色而光明瑩徹碎之嶄巖作墻壁又似雲母片可析者無石彌佳過此則淘土石中得之】地衣春夏用角簟【角簟剖竹為細篾織之藏節去筠瑩滑可愛南蠻或以白藤為之】秋冬用木棉【木棉今南方多有焉於春中作畦種之至夏秋之交結實至秋半其實之外皮四裂中踊出白如綿土人取而紡之織以為布細密厚暖宜以御冬】與子弟僚屬遊宴其間十一月庚寅葬聖文章武明德孝皇帝于顯陵【陵在河南府壽】<br />
<br />
  【安縣】廟號高祖 先是河南北諸州官自賣海鹽歲收緡錢十七萬又散蠶鹽斂民錢【蠶鹽所以裛繭唐天成二年勑每年二月内一度俵散蠶鹽依夏税限納錢宋白曰周顯德三年勅齊州蠶鹽于秋苗上俵配謂之查頭每一石徵錢三千文滄棣濱淄青每石徵絹一匹後齊州減徵一半五州所徵絹加倍先悉薦翻】言事者稱民坐私販鹽抵罪者衆不若聽自販而歲以官所賣錢直斂於民謂之食鹽錢高祖從之俄而鹽價頓賤每斤至十錢至是三司使董遇欲增求羨利【羨延面翻】而難于驟變前法乃重征鹽商過者七錢留賣者十錢由是鹽商殆絶而官復自賣【復扶又翻】其食鹽錢至今斂之如故【五代會要時言事者請將食鹽錢於諸道州府計戶每戶一貫至二百為五等配之然後任人逐便興販既不虧官又益百姓朝廷行之諸處場務且仍舊俄而鹽貨頓賤去出鹽遠處州縣每斤不過二十掌事者又難驟改其法奏請重置税焉蓋欲絶興販歸利于官場院糶鹽雖多人戶鹽錢又不放免民甚苦之】 閩鹽鐵使右僕射李仁遇敏之子【李敏閩主昶元妃梁國夫人之父】閩主曦之甥也年少美姿容得幸於曦【有龍陽之寵也】十二月以仁遇為左僕射兼中書侍郎翰林學士吏部侍郎李光準為中書侍郎兼戶部尚書並同平章事曦荒淫無度嘗夜宴光準醉忤旨【忤五故翻】命執送都市斬之吏不敢殺繫獄中明日視朝【朝直遥翻】召復其位是夕又宴收翰林學士周維岳下獄【下戶嫁翻】吏拂榻待之曰相公昨夜宿此尚書勿憂醒而釋之他日又宴侍臣皆以醉去獨維岳在曦曰維岳身甚小何飲酒之多左右或曰酒有别腸【此俚俗之常語】不必長大曦欣然命捽維岳下殿【捽昨沒翻】欲剖視其酒腸或曰殺維岳無人侍陛下劇飲者乃捨之 帝之初即位也大臣議奉表稱臣告哀於契丹景延廣請致書稱孫而不稱臣【景延廣之議因三年契丹主令高祖稱兒皇帝用家人之禮致書也】李崧曰屈身以為社稷何恥之有【為于偽翻】陛下如此他日必躬擐甲胄【擐音宦】與契丹戰於時悔無益矣【於時者於其時也】延廣固争馮道依違其間帝卒從延廣議【卒子恤翻】契丹大怒遣使來責讓且言何得不先承禀遽即帝位延廣復以不遜語荅之契丹盧龍節度使趙延壽欲代晉帝中國【趙延壽父子欲帝中國之心已見於屯圑柏之時】屢說契丹擊晉契丹主頗然之【說式芮翻為契丹入寇張本】<br />
<br />
  齊王上【諱重貴高祖兄敬儒之子】<br />
<br />
  天福八年春正月癸卯蜀主以宣徽使兼宫苑使田敬全領永平節度使敬全宦者也引前蜀王承休為比而命之【前蜀主王衍使宦者王承休帥秦州事見二百七十三卷唐莊宗同光二年詩云殷鑒不遠在夏后之世孟昶不能以前蜀之亡國為鑒乃引王承休為比以崇秩宦官其國至宋而亡晩矣】國人非之 帝聞契丹將入寇二月己未發鄴都乙丑至東京【帝即位於鄴都保昌殿柩前至是始還汴】然猶與契丹問遺相往來無虛月【遺唯季翻】 唐宣城王景逹剛毅開爽烈祖愛之屢欲以為嗣【烈祖即謂唐主唐主崩廟號烈祖通鑑因其國史成文書之】宋齊丘亟稱其才【亟去吏翻】唐主以齊王璟年長而止【長知兩翻】璟以是怨齊丘【既以贊奪嫡之謀怨之又以争權誤國怒之宋齊丘于是不得免矣】唐主幼子景逷母种氏有寵齊王璟母宋皇后稀得進見唐主如璟宫遇璟親調樂器大怒誚讓者數日种氏乘間言景逷雖幼而慧可以為嗣【逷他歷翻种直中翻見賢遍翻誚才笑翻間古莧翻】唐主怒曰子有過父訓之常事也國家大計女子何得預知即命嫁之【史言唐主明斷不牽于女寵】唐主嘗夢吞靈丹旦而方士史守沖獻丹方以為神而餌之浸成躁急【自叔孫豹以來踐妖夢以自禍者多矣】左右諫不聽嘗以藥賜李建勲建勲曰臣餌之數日已覺躁熱况多餌乎唐主曰朕服之久矣羣臣奏事往往暴怒然或有正色論辯中理者【中竹仲翻】亦斂容慰謝而從之唐主問道士王栖霞何道可致太平對曰王者治心治身乃治家國今陛下尚未能去飢嗔飽喜【治直之翻去羌呂翻嗔昌眞翻】何論太平宋后自簾中稱嘆以為至言凡唐主所賜予【子讀曰與】栖霞皆不受栖霞常為人奏章唐主欲為之築壇【為于偽翻】辭曰國用方乏何暇及此俟焚章不化乃當奏請耳【道士率奏章自謂上達於天史言王栖霞異乎挾術以干寵利者】駕部郎中馮延己為齊王元帥府掌書記性傾巧與宋齊丘及宣徽副使陳覺相結同府在己上者延己稍以計逐之延己嘗戲謂中書侍郎孫晟曰公有何能為中書郎晟曰晟山東鄙儒【孫晟密州高密縣人奔南見二百七十六卷唐明宗天成二年】文章不如公談諧不如公諂詐不如公然主上使公與齊王遊處【處昌呂翻】蓋欲以仁義輔導之也豈但為聲色狗馬之友邪晟誠無能公之能適足為國家之禍耳延己歙州人也【歙書涉翻】又有魏岑者亦在齊王府給事中常夢錫屢言陳覺馮延己魏岑皆佞邪小人不宜侍東宫司門郎中判大理寺蕭儼表稱陳覺姦回亂政唐主頗感悟未及去【去羌呂翻】會疽發背祕不令人知密令醫治之聽政如故庚午疾亟太醫吳廷裕遣親信召齊王璟入侍疾【唐主密令醫治疾猶可日欲以鎮安人心至于危殆召嫡長入侍乃出于醫師之意此可以為法手治直之翻】唐主謂璟曰吾餌金石始欲益壽乃更傷生汝宜戒之是夕殂【年五十六】祕不發喪下制以齊王監國大赦孫晟恐馮延己等用事欲稱遺詔令太后臨朝稱制翰林學士李貽業曰先帝嘗云婦人預政亂之本也安肯自為厲階此必近習姦人之詐也且嗣君春秋已長明德著聞【長知兩翻聞音問】公何得遽為亡國之言若果宣行吾必對百官毁之晟懼而止貽業蔚之從曾孫也【李蔚唐僖宗乾符中為相蔚紆勿翻從才用翻】丙子始宣遺制【庚午至丙子七日始發喪】烈祖末年卞急近臣多罹譴罰陳覺稱疾累月不入及宣遺詔乃出蕭儼劾奏覺端居私室以俟升遐【劾戶槩翻又戶得翻】請按其罪齊王不許自烈祖相吳禁壓良為賤【買良人子女為奴婢謂之壓良為賤律之所禁也】令買奴婢者通官作劵馮延已及弟禮部員外郎延魯俱在元帥府草遺詔聽民賣男女意欲自買姬妾蕭儼駮曰【駮北角翻】此必延已等所為非大行之命也【自漢以來天子升遐梓宫在殯稱曰大行皇帝】昔延魯為東都判官【東都留守判官也唐以江都為東都】已有此請先帝訪臣臣對曰陛下昔為吳相民有鬻男女者為出府金贖而歸之【為出于偽翻府金藏府之金也】故遠近歸心今即位而反之使貧人之子為富人厮役可乎先帝以為然將治延魯罪【治直之翻】臣以為延魯愚無足責先帝斜封延魯章抹三筆持入宫請求諸宫中必尚在齊王命取先帝時留中章奏千餘道皆斜封一抹【凡章奏留中不下者皆當時不行者也若其言可行者則付外施行】果得延魯疏然以遺詔已行竟不之改【聲馮延魯以已私傳益遺制之罪明底其罰而改之不亦可乎史言唐烈祖尚儒故當國有大故之時猶有能持正以斷國論者】 閩富沙王延政稱帝於建州國號大殷大赦改元天德以將樂縣為鏞州【唐武德五年分邵武置將樂縣時屬建州宋屬南劔州九域志在州南二百四十里宋白曰其地在越已有將樂之名按後漢書云永安三年析建安之校鄉置將樂縣按漢無永安年號獨吳孫休改元永安耳樂音洛】延平鎮為鐔州【鐔州今之南劔州是也吳分建安置南平縣晉改為延平縣閩王審知立延平鎮王延政置鐔州南唐改劔州取寶劔化龍於延平津以立州也宋朝混一始加南字以别蜀之劔州鐔徐林翻又讀如覃】立皇后張氏以節度判官潘承祐為吏部尚書節度巡官建陽楊思恭為兵部尚書【唐武德四年置建陽縣屬建州九域志在州西二百四十里宋白曰漢建安元年割建安縣地為桐鄉十年會稽南部都尉賀齊分上饒之地兼舊桐鄉置建平縣晉太元四年改建平為建陽縣因山之陽為名】未幾以承祐同平章事【潘承祐能諫王延政之日尋干戈而不能諫其舉大號又俛眉而為之相亦復何也】思恭遷僕射錄軍國事延政服赭袍視事然牙參及接鄰國使者猶如藩鎮禮殷國小民貧軍旅不息楊思㳟以善聚斂得幸【歛力贍翻】增田畝山澤之税至於魚鹽蔬果無不倍征【征之倍其常數】國人謂之楊剥皮三月己卯朔以中書令趙瑩為晉昌節度使兼中書令以晉昌節度使兼侍中桑維翰為侍中【桑維翰始者居藩鎮而兼侍中今入朝正為門下省長官】 唐元宗即位【本名景通改名璟後又改名景】大赦改元保大祕書郎韓熙載請俟踰年改元不從【古者人君即位踰年而後改元不忍遽改父之道也】尊皇后曰皇太后【唐烈祖后宋氏】立妃鍾氏為皇后唐主未聼政【以居喪未御正朝聽政】馮延己屢入白事一日至數四唐主曰書記有常職何為如是其煩也【馮延己時為齊王掌書記】唐主為人謙謹初即位不名大臣數延公卿論政體【數所角翻】李建勲謂人曰主上寛仁大度優于先帝但性習未定苟旁無正人但恐不能守先帝之業耳【後果如李建勲之言其僅保江南者幸也】唐主以鎮南節度使宋齊丘為太保兼中書令奉化節度使周宗為侍中【九域志南唐置奉化軍節度于江州】唐主以齊丘宗先朝勲舊故順人望召為相政事皆自決之徙夀王景遂為燕王宣城王景逹為鄂王初唐主為齊王知政事【晉天福三年唐烈祖徙吳王璟為齊王若其輔政則始于後唐潞王清泰元年吳睿皇之大和六年也此盖言知唐政事時】每有過失常夢錫常直言規正始雖忿懟【懟直類翻】終以諒直多之及即位許以為翰林學士齊丘之黨疾之坐封駮制書貶池州判官池州多遷客【以罪遷降於外州者其州人謂之為遷客】節度使上蔡王彦儔防制過甚幾不聊生【幾居依翻】惟事夢錫如在朝廷【王彦儔豈知敬常夢錫哉以其事唐主於齊府貶非其罪必將復召用故敬之耳】宋齊丘待陳覺素厚唐主亦以覺為有才遂委任之馮延己延魯魏岑雖齊邸舊僚皆依附覺與休寧查文徽【吳分歙縣置休寧縣後改曰海陽晉武帝改曰海寧隋改曰休寧唐置歙州九域志在州西六十六里查鉏加翻姓也何承天姓苑已有此姓則江南之有查姓舊矣】更相汲引侵蠧政事【更工衡翻】唐人謂覺等為五鬼延魯自禮部員外郎遷中書舍人勤政殿學士江州觀察使杜昌業聞之歎曰國家所以驅駕羣臣在官爵而已若一言稱旨遽通顯【勤政殿學士盖烈祖所置猶中朝之端明殿學士也稱尺證翻】後有立功者何以賞之未幾唐主以岑及文徽皆為樞密副使【幾居豈翻】岑既得志會覺遭母喪岑即暴揚覺過惡【暴顯也】擯斥之 唐置定遠軍於濠州 漢殤帝驕奢不親政事高祖在殯作樂酣飲夜與倡婦微行倮男女而觀之【倡音昌倮魯果翻】左右忤意輒死無敢諫者【忤五故翻】惟越王弘昌及内常侍番禺吳懷恩屢諫不聼【番音潘】常猜忌諸弟每宴集令宦者守門羣臣宗室皆露索然後入【露體而搜索之恐其挾懷兵刃也索山客翻】晉王弘熙欲圖之乃盛飾聲伎娛悦其意以成其惡【伎巨綺翻】漢主好手搏弘熙令指揮使陳道庠引力士劉思潮譚令禋林少強林少良何昌廷等五人習手搏於晉府【好呼到翻少詩照翻晉府弘熙所居第也】漢主聞而悦之丙戌與諸王宴於長春宫觀手搏至夕罷宴漢主大醉弘熙使道庠思潮等掖漢主因拉殺之【年二十四因扶掖而拉其脅殺之拉盧合翻】盡殺其左右明旦百官諸王莫敢入宫越王弘昌帥諸弟臨於寢殿【帥讀曰率臨力鴆翻】迎弘熙即皇帝位更名晟【晟漢主玢之弟也更工衡翻】改元應乾以弘昌為太尉兼中書令諸道兵馬都元帥知政事循王弘杲為副元帥參預政事陳道庠及劉思潮等皆受賞賜甚厚 閩主曦納金吾使尚保殷之女【考異曰閩錄作尚可殷今從十國紀年】立為賢妃妃有殊色曦嬖之醉中<br />
<br />
  妃所欲殺則殺之所欲有則宥之【沈酗于酒惟婦言是用商紂所以亡也嬖卑義翻又博計翻】 夏四月戊申朔日有食之 唐以中書侍郎同平章事李建勲為昭武節度使鎮撫州【九域志吳置昭武節度于撫州】 殷將陳望等攻閩福州【是年二月王延政建國于建州號曰殷】入其西郛既而敗歸 五月殷吏部尚書同平章事潘承祐上書陳十事大指言兄弟相攻逆傷天理一也賦斂煩重力役無節二也【斂力贍翻】發民為兵羇旅愁怨三也【民為兵則疲于征戍羇旅異鄉不得反其桑梓故愁怨】楊思㳟奪民衣食使歸怨於上羣臣莫敢言四也【楊思恭事見上二月】疆土狹隘多置州縣增吏困民五也【謂置鏞州鐔州也】除道裹糧將攻臨汀【臨汀汀州也唐開撫福二州山洞置汀州因長汀為名初治新羅後移治長汀白石村天寶改為臨汀郡乾元復為州九域志延平西至臨汀八百里】曾不憂金陵錢塘乘虛相襲六也【唐都金陵吳越都錢塘唐兵自撫信可以襲建州吳越兵自婺衢可以襲建州】括高貲戶財多者補官逋負者被刑七也【被皮義翻】延平諸津征果菜魚米獲利至微斂怨甚大八也與唐吳越為隣即位以來未嘗通使九也宫室臺榭崇飾無度十也殷王延政大怒【殷王當作殷主】削承祐官爵勒歸私第 漢中宗既立國中議論詾詾【言其弑兄自立也詾許拱翻】循王弘杲請斬劉思潮等以謝中外漢主不從思潮等聞之譛弘杲謀反漢主令思潮等伺之弘杲方宴客思潮與譚令禋帥衛兵突入【伺相吏翻也察也帥讀曰率突入掩不備】斬弘杲於是漢主謀盡誅諸弟以越王弘昌賢而得衆尤忌之【弘昌見忌事始上年四月】雄武節度使齊王弘弼【詳考本末雄武當作建武建武軍邕州】自以居大鎮懼禍求入朝許之 初閩主曦侍康宗宴【閩主昶廟號康宗】會新羅獻寶劔【新羅國之于閩國其地在海東通使於閩】康宗舉以示同平章事王倓曰此何所施倓對曰斬為臣不忠者時曦已蓄異志凛然變色至是宴羣臣復有獻劔者曦命發倓冡斬其尸【倓徒甘翻又徒濫翻徒敢翻復扶又翻】校書郎陳光逸謂其友曰主上失德亡無日矣吾欲死諫其友止之不從上書諫曦大惡五十事曦怒命衛士鞭之數百不死以繩繫其頸懸諸庭樹久之乃絶秋七月己丑詔以年饑國用不足分遣使者六十餘<br />
<br />
  人於諸道括民穀 吳越王弘佐初立上統軍使闞璠彊戾【闞苦鑑翻姓也璠音煩】排斥異已弘佐不能制内牙上都監使章德安數與之争【數所角翻】右都監使李文慶不附於璠乙巳貶德安於處州【章德安受託孤之寄而為闞璠所制其才不足稱也】文慶于睦州璠與右統軍使胡進思益專横【為吳越誅闞璠張本横戶孟翻】璠明州人【今明州猶祀闞璠謂之闞相公廟】文慶睦州人進思湖州人也唐主緣烈祖意【緣因也由也】以天雄節度使兼中書令金<br />
<br />
  陵尹燕王景遂為諸道兵馬元帥徙封齊王居東宫天平節度使守侍中東都留守鄂王景達為副元帥徙封燕王宣告中外約以傳位立長子弘冀為南昌王景遂景達固辭不許景遂自誓必不敢為嗣更其字曰退身【更工衡翻為弘冀毒景遂張本】 漢指揮使萬景忻敗張遇賢於循州【敗補邁翻】遇賢吿于神神曰取䖍州則大事可成遇賢帥衆踰嶺趨䖍州唐百勝節度使賈匡浩不為備【梁以百勝節度使命盧光稠淮南楊氏既并䖍州因而不改宋朝紹興初改䖍州為贑州取章貢二水以名州也帥讀曰率趣七喻翻】遇賢衆十餘萬攻陷諸縣再敗州兵【敗補邁翻】城門晝閉遇賢作宫室營署於白雲洞遣將四出剽掠【剽匹妙翻】匡浩公鐸之子也【賈公鐸見二百六十卷唐昭宗乾寧三年】 八月乙卯唐主立弟景逷為保寧王宋太后怨种夫人屢欲害景逷【种夫人欲立景逷見是年二月】唐主力保全之 夏州牙内指揮使拓跋崇斌謀作亂綏州刺史李彝敏將助之事覺辛未彞敏弃州與其弟彝俊等五人奔延州【趙珣聚米圖經綏州南至延州界三百四十里宋白曰綏州北至夏州三百六十里】 九月尊帝母秦國夫人安氏為皇太妃妃代北人也【帝既繼大宗則帝父敬儒為皇伯今尊生母安氏為皇太妃將以為誰之妃乎】帝事太后太妃甚謹待諸弟亦友愛【高祖七子此時惟重睿在耳帝敬儒之子也亦無兄弟見于史】 初河陽牙將喬榮 【考異曰漢隐帝實錄作喬熒陷蕃記作喬瑩今從晉少帝漢高祖實録景延廣傳契丹傳】從趙延夀入契丹契丹以為回圖使【凡外國與中國貿易者置回圖務今之回易場也】往來販易於晉置邸大梁及契丹與晉有隙景延廣說帝囚榮於獄【說式芮翻】悉取邸中之貨凡契丹之人販易在晉境者皆殺之奪其貨大臣皆言契丹有大功【謂救解晉陽之圍高祖遂以得中原】不可負戊子釋榮慰賜而歸之榮辭延廣延廣大言曰歸語而主【語牛倨翻而汝也】先帝為北朝所立故稱臣奉表今上乃中國所立所以降志於北朝者正以不敢忘先帝盟約故耳為鄰稱孫足矣無稱臣之理北朝皇帝勿信趙延夀誑誘【誑居况翻誘以久翻】輕侮中國中國士馬爾所目睹翁怒則來戰孫有十萬横磨劔足以相待他日為孫所敗【敗補邁翻】取笑天下毋悔也榮自以亡失貨財恐歸獲罪且欲為異時據驗乃曰公所言頗多懼有遺忘【忘巫放翻】願記之紙墨延廣命吏書其語以授之榮具以白契丹主契丹主大怒入寇之志始決【景延廣建議稱孫不稱臣猶可曰為國體也囚其邸吏而取其貨財則誤國之罪無所逃矣】晉使如契丹皆縶之幽州不得見桑維翰屢請遜辭以謝契丹每為延廣所沮【沮在呂翻】帝以延廣有定策功故寵冠羣臣【冠古玩翻】又縂宿衛兵故大臣莫能與之争河東節度使劉知遠知延廣必致寇而畏其方用事不敢言【劉知遠非不敢言盖亦有憾於帝而不欲言將坐觀成敗因而利之也】但益募兵奏置興捷武節等十餘軍以備契丹 甲午定難節度使李彝殷奏李彝敏作亂之狀【難乃旦翻】詔執彝敏送夏州斬之 冬十月戊申立吳國夫人馮氏為皇后初高祖愛少弟重胤養以為子【歐史重胤高祖弟也不知其為親疏高祖愛之養以為子故於名加重而下齒諸子少詩照翻重直龍翻】及留守鄴都娶副留守安喜馮濛女為其婦【安喜縣屬定州劉煦曰安喜漢中山之盧奴縣也慕容垂改為不連北齊改曰安喜隋改為鮮虞唐武德為安喜定州所治也】 重胤早卒馮夫人寡居有美色帝見而悦之高祖崩梓宫在殯帝遂納之羣臣皆賀帝謂馮道等曰皇太后之命與卿等不任大慶羣臣出帝與夫人酣飲過梓宫前醊而告曰皇太后之命與先帝不任大慶【任音壬醊陟衛翻祭而以酒酹地也斬焉衰絰之中觸情縱欲以亂大倫又從而狎侮其先何以能久】左右失笑【不覺發笑為失笑】帝亦自笑顧謂左右曰我今日作新壻何如夫人與左右皆大笑太后雖恚而無如之何【恚於避翻魯昭公在感而有嘉容終以失國帝與夫人淪於異域非不幸也】既正位中宫頗預政事后兄玉時為禮部郎中鹽鐵判官帝驟擢用至端明殿學士戶部侍郎與議政事 漢主命韶王弘雅致仕 唐主遣洪州營屯都虞候嚴恩將兵討張遇賢以通事舍人金陵邉鎬為監軍鎬用䖍州人白昌裕為謀主擊張遇賢屢破之遇賢禱於神神不復言【復扶又翻】其徒大懼昌裕勸鎬伐木開道出其營後襲之遇賢弃衆奔别將李台台知神無驗執遇賢以降斬於金陵市【去年七月張遇賢作亂於漢境入唐境而亡史言依託怪妄之禍敗降戶江翻】 十一月丁亥漢主祀南郊大赦改元乾和 戊子吳越王弘佐納妃仰氏仁詮之女也【仰仁詮見任于吳越王元瓘詮且緣翻】 初高祖以馬三百借平盧節度使楊光遠景延廣以詔命取之光遠怒曰是疑我也密召其子單州刺史承祚【唐末以宋州之碭山縣梁太祖鄉里也為置輝州已而徙治單父縣後唐㓕梁改為單州薛居正五代史唐莊宗同光二年六月改輝州為單州單音善】戊戌承祚稱母病夜開門奔青州庚子以左飛龍使金城何超權知單州【此應州之金城縣地】遣内班賜光遠玉帶御馬以安其意【内班盖宦者也】壬寅遣侍衛步軍都指揮使郭謹將兵戍鄆州【以防河津使楊光遠不得與契丹交通也】 唐葬光文肅武孝高皇帝于永陵廟號烈祖 十二月乙巳朔遣左領軍衛將軍蔡行遇將兵戍鄆州楊光遠遣騎兵入淄州劫刺史翟進宗歸于青州【九域志青州西南至淄州一百二十里翟萇伯翻】甲寅徙楊承祚為登州刺史以從其便【登州平盧巡屬也】光遠益驕密告契丹以晉主負德違盟境内大饑公私困竭乘此際攻之一舉可取趙延夀亦勸之契丹主乃集山後及盧龍兵合五萬人使延夀將之【山後即媯檀雲應諸州盧龍幽州軍號此皆天福之初割與契丹之土地人民也契丹用中國之將將中國之兵以攻晉藉寇兵而齎盜糧中國自此胥為夷矣將即亮翻】委延夀經畧中國曰若得之當立汝為帝又常指延夀謂晉人曰此汝主也延夀信之由是為契丹盡力畫取中國之策【趙延夀為契丹主愚弄鼔舞至死不悟嗜欲深者天機淺也是為于偽翻】朝廷頗聞其謀丙辰遣使城南樂及德清軍【時置德清軍於澶州清豐縣在州北六十里宋白曰德清軍本舊澶州地晉天福三年移澶州於德勝寨乃于舊澶州置頓丘鎮取縣為名至四年改鎮為德清軍開運元年移德清軍於陸家店在新澶州之北七十里】徵近道兵以備之 唐侍中周宗年老恭謹自守中書令宋齊丘廣樹朋黨百計傾之【宋齊丘之嫌隙開于吳唐禪代之間權利啟人争心有如此者事見二百八十卷】宗泣訴於唐主唐主由是薄齊丘既而陳覺被疎乃出齊丘為鎮海節度使【陳覺者宋齊丘之黨唐主所親任者也覺疎則齊丘無君側之助乃出被皮義翻】齊丘忿懟表乞歸九華舊隐【懟直類翻齊丘隐九華見二百七十七卷唐明宗長興二年】唐主知其詐一表即從之賜書曰明日之行昔時相許朕實知公故不奪公志仍賜號九華先生封青陽公食一縣租税齊丘乃治大第於青陽【宋白曰青陽縣本吳臨城縣地赤烏中置隋平陳廢臨城縣為南陵縣唐天寶元年分涇南陵秋浦置青陽縣屬池州以其地在青山之陽也九域志在州東南一百里治直之翻】服御將吏皆如王公而憤邑尤甚 寧州酋長莫彦殊以所部温那等十八州附于楚【寧州即唐之南寧州也天寶末没于蠻唐末復置寧州于清溪鎮去黔州二十九日行酋慈由翻長知兩翻】其州無官府惟立牌於岡阜略以恩威羈縻而已 是歲春夏旱秋冬水蝗大起東自海壖西距隴坻【壖而宣翻坻丁禮翻】南踰江淮北抵幽薊原野山谷城郭廬舍皆滿竹木葉俱盡重以官括民穀【薊音計重直用翻是年秋七月以年饑用不足括民穀】使者督責嚴急至封碓磑不留其食有坐匿穀抵死者縣令往往以督趣不辦納印自劾去民餒死者數十萬口流亡不可勝數【碓都内翻舂也磑五封翻䃺也趣讀曰促劾戶槩翻又戶得翻勝音升】於是留守節度使下至將軍各獻馬金帛芻粟以助國朝廷以恒定饑甚獨不括民穀順國節度使杜威奏稱軍食不足請如諸州例許之【杜重威平安重榮即用為恒帥帝即位避帝名去重字止稱威順國軍號亦新改恒戶登翻】威用判官王緒謀檢索殆盡【索山客翻】得百萬斛威止奏三十萬斛餘皆入其家令判官李沼稱貸於民復滿百萬斛來春糶之【稱蚩援翻舉也復扶又翻糶他弔翻】得緡錢二百萬闔境苦之定州吏欲援例為奏【援恒州例援于元翻】義武節度使馬全節不許曰吾為觀察使職在養民豈忍效彼所為乎【唐節度使率兼觀察使節度之職掌兵觀察之職掌民馬全節之不效杜威是矣鄰於善民之望也杜威曾念及此乎】 楚地多產金銀茶利尤厚由是財貨豐殖而楚王希範奢欲無厭喜自誇大【更於鹽翻喜許記翻】為長槍大槊飾之以金可執而不可用募富民年少肥澤者八千人為銀槍都【少詩照翻】宫室園囿服用之物務窮侈靡作九龍殿刻沈香為八龍【沈持林翻】飾以金寶長十餘丈【長直亮翻下同】抱柱相向希範居其中自為一龍其襆頭脚長丈餘以象龍角用度不足重為賦斂【襆防玉翻後周武帝製幞頭裁幅巾出四脚至今人服用之唐人其脚向上至宋太祖始為放脚長直亮翻斂力贍翻下同】每遣使者行田專以增頃畝為功民不勝租賦而逃【行下孟翻勝音升】王曰但令田在何憂無穀【民逃則有不耕之土何從得穀乎史言馬希範不知稼穡之艱難】命營田使鄧懿文籍逃田募民耕藝出租【藝種也】民捨故從新僅能自存自西徂東各失其業【民無安生樂業之心安能親其上而死其長乎】又聽人入財拜官以財多少為官高卑之差富商大賈【賈音古】布在列位外官還者必責貢獻【還從宣翻又如字】民有罪則富者輸財強者為兵惟貧弱受刑又置函使人投匿名書相告訐【訐居謁翻】至有滅族者是歲用孔目官周陟議令常税之外大縣貢米二千斛中千斛小七百斛無米者輸布帛天策學士拓跋恒上書曰殿下長深宫之中藉已成之業【長知兩翻藉慈夜翻】身不知稼穡之勞耳不聞鼓鼙之音【鼙部迷翻】馳騁遨遊雕牆玉食【張宴曰玉食珍食也韋昭曰諸侯備珍異之食】府庫盡矣而浮費益甚百姓困矣而厚斂不息今淮南為仇讎之國番禺懷吞噬之志荆渚日圖窺伺溪洞待我姑息【淮南謂唐番禺謂漢荆渚謂高氏溪洞彭莫諸族伺相吏翻】諺曰足寒傷心民怨傷國願罷輸米之令誅周陟以謝郡縣去不急之務減興作之役無令一旦禍敗為四方所笑王大怒他日恒請見【去羌呂翻見賢遍翻】辭以晝寢恒謂客將區弘練曰【將即亮翻區豈俱翻又音歐今湖南多此姓】王逞欲而愎諫【愎蒲邁翻】吾見其千口飄零無日矣【人多謂闔家之人曰百口今曰千口者以其諸侯盛言之】王益怒遂終身不復見之【復扶又翻】 閩主曦嫁其女取班簿閲視之【班簿者簿記朝參名員】朝士有不賀者十二人皆杖之於朝堂以御史中丞劉贊不舉劾【劾戶槩翻又戶得翻】亦將杖之贊義不受辱欲自殺諫議大夫鄭元弼諫曰古者刑不上大夫【記曲禮之言上時掌翻】中丞儀刑百僚豈宜加之箠楚【箠止橤翻】曦正色曰卿欲效魏徵邪元弼曰臣以陛下為唐太宗故敢效魏徵曦怒稍解乃釋贊贊竟以憂卒<br />
<br />
  開運元年【是年七月方改元】春正月乙亥邉藩馳告契丹前鋒將趙延夀趙延照將兵五萬入寇逼貝州【邉藩猶言邉鎮也】延照思温之子也【趙思温本中國人沒於契丹】先是朝廷以貝州水陸要衝【先悉薦翻】多聚芻粟為大軍數年之儲以備契丹軍校邵珂性凶悖【校戶教翻珂丘何翻悖蒲妺翻又蒲沒翻】永清節度使王令温黜之【時置永清軍於貝州】珂怨望密遣人亡入契丹言貝州粟多而兵弱易取也【易以豉翻】會令温入朝執政以前復州防禦使吳巒權知州事【天福初吳巒堅守雲州以拒契丹故朝廷用之】巒至推誠撫士會契丹入寇巒書生無爪牙珂自請願効死巒使將兵守南門巒自守東門契丹主自攻貝州巒悉力拒之燒其攻具殆盡己卯契丹復攻城【復扶又翻】珂引契丹自南門入巒赴井死契丹遂陷貝州所殺且萬人庚辰以歸德節度使高行周為北面行營都部署以河陽節度使苻彦卿為馬軍左廂排陳使【苻當作符鄭樵氏族畧曰魯頃公為楚所滅頃公之孫公雅為秦符節令因以為氏後漢有符融皇朝有符彦卿望出琅邪非苻秦之苻也陳讀曰陣下同】以右神武統軍皇甫遇為馬軍右廂排陳使以陜府節度使王周為步軍左廂排陳使以左羽林將軍潘環為步軍右廂排陳使【陜失冉翻】 太原奏契丹入鴈門關【鴈門關即陘嶺關】恒邢滄皆奏契丹入寇【恒戶登翻】 成德節度使杜威【自安重榮反死晉改成德軍為順國軍史以舊軍名書之耳】遣幕僚曹光裔詣楊光遠為陳禍福【為於偽翻】光遠遣光裔入奏稱承祚逃歸母疾故爾【去年十一月楊承祚自單州逃歸青州】既蒙恩宥闔族荷恩【荷下可翻】朝廷信其言遣使與光裔復往慰諭之【復扶又翻】 唐以侍中周宗為鎮南節度使左僕射兼門下侍郎同平章事張居詠為鎮海節度使 唐主決欲傳位於齊燕二王【傳位之議始於去年七月燕於賢翻】翰林學士馮延己等因之欲隔絶中外以擅權辛巳敕齊王景遂參決庶政百官惟樞密副使魏岑查文徽得白事【查鉏加翻】餘非召對不得見國人大駭給事中蕭儼上疏極論不報 【考異曰江南錄此敕在去年十二月今從十國紀年紀年云宋齊丘上疏今從江南録】侍衛都虞候賈崇叩閤求見【見賢遍翻】曰臣事先帝三十年觀其延接疎遠孜孜不怠下情猶有不通者陛下新即位所任者何人而頓與羣臣謝絶臣老矣不復得奉顔色因涕泗嗚咽【詩涕泗滂沱注云自目曰涕自鼻曰泗】唐主感悟遽收前敕唐主於宫中作高樓召侍臣觀之衆皆歎美蕭儼曰恨樓下無井唐主問其故對曰以此不及景陽樓耳【陳後主起景陽樓隋兵至自投於樓下井中蕭儼引亡國以諫也】唐主怒貶於舒州觀察使孫晟遣兵防之儼曰儼以諫諍得罪非有他志昔顧命之際君幾危社稷【謂孫晟欲使太后臨朝也幾居依翻】其罪顧不重於儼乎今日反見防邪晟慙懼遽罷之帝遣使持書遺契丹【遺唯季翻】契丹已屯鄴都【時契丹屯于鄴都城外】不得通而返壬午以侍衛馬步都指揮使景延廣為御營使前静難節度使李周為東京留守【難乃旦翻】是日高行周以前軍先發時用兵方畧號令皆出延廣宰相以下皆無所預延廣乘勢使氣陵侮諸將雖天子亦不能制【為罷景延廣張本】乙酉帝發東京丁亥滑州奏契丹至黎陽【黎陽在滑州西岸隔大河耳故奏其事】戊子帝至澶州【澶州時據德勝津】契丹主屯元城【劉昫曰魏州元城隋縣治古殷城唐貞觀十七年併入貴郷聖歷二年分貴郷萃縣置元城縣治王莽城開元十三年移治郭下古殷城在朝城東北十二里時契丹主盖屯古殷城也】趙延夀屯南樂【南樂即唐魏州之昌樂縣後唐避其祖李國昌諱改曰南樂九域志南樂縣在魏州南四十四里樂音洛】以延夀為魏博節度使封魏王【此契丹主所命也】契丹寇太原劉知遠與白承福合兵二萬擊之甲午以知遠為幽州道行營招討使杜威為副使馬全節為都虞候丙申遣右武衛上將軍張彦澤等將兵拒契丹於黎陽 戊戌蜀主復以將相遥領節度使【蜀罷將相領節見二百八十二卷高祖天福六年蜀主之廣政五年也】 帝復遣譯者孟守忠致書於契丹求修舊好【復扶又翻好呼到翻】契丹主復書曰已成之勢不可改也辛丑太原奏破契丹偉王於秀容【秀容漢汾陽縣地隋自秀容故城移於此因更縣名唐帶忻州】斬首三千級契丹自鴉鳴谷遁去【自鴉鳴谷出潞州東與契丹主大軍合】 殷鑄天德通寶大鐵錢一當百 唐主遣使遺閩主曦及殷主延政書責以兄弟尋戈【遺唯季翻左傳鄭子產曰昔高莘氏有二子伯曰閼伯季曰實沈居於曠林不相能也日尋干戈以相征討】曦復書引周公誅管蔡唐太宗誅建成元吉為比延政復書斥唐主奪楊氏國唐主怒遂與殷絶【為唐滅殷張本】 天平節度副使知鄆州顔衎遣觀察判官竇儀奏博州刺史周儒以城降契丹【九域志鄆州西北至博州一百七十里衎苦旦翻又音侃】又與楊光遠通使往還引契丹自馬家口濟河擒左武衛將軍蔡行遇【去年十二月遣蔡行遇戍鄆州】儀謂景延廣曰虜若濟河與光遠合則河南危矣延廣然之儀薊州人也【薊音計】<br />
<br />
  資治通鑑卷二百八十三<br />
<br />
<史部,編年類,資治通鑑>  <br>
   </div> 

<script src="/search/ajaxskft.js"> </script>
 <div class="clear"></div>
<br>
<br>
 <!-- a.d-->

 <!--
<div class="info_share">
</div> 
-->
 <!--info_share--></div>   <!-- end info_content-->
  </div> <!-- end l-->

<div class="r">   <!--r-->



<div class="sidebar"  style="margin-bottom:2px;">

 
<div class="sidebar_title">工具类大全</div>
<div class="sidebar_info">
<strong><a href="http://www.guoxuedashi.com/lsditu/" target="_blank">历史地图</a></strong>  
<a href="http://www.880114.com/" target="_blank">英语宝典</a>  
<a href="http://www.guoxuedashi.com/13jing/" target="_blank">十三经检索</a> 
<br><strong><a href="http://www.guoxuedashi.com/gjtsjc/" target="_blank">古今图书集成</a></strong> 
<a href="http://www.guoxuedashi.com/duilian/" target="_blank">对联大全</a> <strong><a href="http://www.guoxuedashi.com/xiangxingzi/" target="_blank">象形文字典</a></strong> 

<br><a href="http://www.guoxuedashi.com/zixing/yanbian/">字形演变</a>  <strong><a href="http://www.guoxuemi.com/hafo/" target="_blank">哈佛燕京中文善本特藏</a></strong>
<br><strong><a href="http://www.guoxuedashi.com/csfz/" target="_blank">丛书&方志检索器</a></strong> <a href="http://www.guoxuedashi.com/yqjyy/" target="_blank">一切经音义</a>  

<br><strong><a href="http://www.guoxuedashi.com/jiapu/" target="_blank">家谱族谱查询</a></strong>  <strong><a href="http://shufa.guoxuedashi.com/sfzitie/" target="_blank">书法字帖欣赏</a></strong> 
<br>

</div>
</div>


<div class="sidebar" style="margin-bottom:0px;">

<font style="font-size:22px;line-height:32px">QQ交流群9:489193090</font>


<div class="sidebar_title">手机APP 扫描或点击</div>
<div class="sidebar_info">
<table>
<tr>
	<td width=160><a href="http://m.guoxuedashi.com/app/" target="_blank"><img src="/img/gxds-sj.png" width="140"  border="0" alt="国学大师手机版"></a></td>
	<td>
<a href="http://www.guoxuedashi.com/download/" target="_blank">app软件下载专区</a><br>
<a href="http://www.guoxuedashi.com/download/gxds.php" target="_blank">《国学大师》下载</a><br>
<a href="http://www.guoxuedashi.com/download/kxzd.php" target="_blank">《汉字宝典》下载</a><br>
<a href="http://www.guoxuedashi.com/download/scqbd.php" target="_blank">《诗词曲宝典》下载</a><br>
<a href="http://www.guoxuedashi.com/SiKuQuanShu/skqs.php" target="_blank">《四库全书》下载</a><br>
</td>
</tr>
</table>

</div>
</div>


<div class="sidebar2">
<center>


</center>
</div>

<div class="sidebar"  style="margin-bottom:2px;">
<div class="sidebar_title">网站使用教程</div>
<div class="sidebar_info">
<a href="http://www.guoxuedashi.com/help/gjsearch.php" target="_blank">如何在国学大师网下载古籍?</a><br>
<a href="http://www.guoxuedashi.com/zidian/bujian/bjjc.php" target="_blank">如何使用部件查字法快速查字?</a><br>
<a href="http://www.guoxuedashi.com/search/sjc.php" target="_blank">如何在指定的书籍中全文检索?</a><br>
<a href="http://www.guoxuedashi.com/search/skjc.php" target="_blank">如何找到一句话在《四库全书》哪一页?</a><br>
</div>
</div>


<div class="sidebar">
<div class="sidebar_title">热门书籍</div>
<div class="sidebar_info">
<a href="/so.php?sokey=%E8%B5%84%E6%B2%BB%E9%80%9A%E9%89%B4&kt=1">资治通鉴</a> <a href="/24shi/"><strong>二十四史</strong></a>&nbsp; <a href="/a2694/">野史</a>&nbsp; <a href="/SiKuQuanShu/"><strong>四库全书</strong></a>&nbsp;<a href="http://www.guoxuedashi.com/SiKuQuanShu/fanti/">繁体</a>
<br><a href="/so.php?sokey=%E7%BA%A2%E6%A5%BC%E6%A2%A6&kt=1">红楼梦</a> <a href="/a/1858x/">三国演义</a> <a href="/a/1038k/">水浒传</a> <a href="/a/1046t/">西游记</a> <a href="/a/1914o/">封神演义</a>
<br>
<a href="http://www.guoxuedashi.com/so.php?sokeygx=%E4%B8%87%E6%9C%89%E6%96%87%E5%BA%93&submit=&kt=1">万有文库</a> <a href="/a/780t/">古文观止</a> <a href="/a/1024l/">文心雕龙</a> <a href="/a/1704n/">全唐诗</a> <a href="/a/1705h/">全宋词</a>
<br><a href="http://www.guoxuedashi.com/so.php?sokeygx=%E7%99%BE%E8%A1%B2%E6%9C%AC%E4%BA%8C%E5%8D%81%E5%9B%9B%E5%8F%B2&submit=&kt=1"><strong>百衲本二十四史</strong></a>  <a href="http://www.guoxuedashi.com/so.php?sokeygx=%E5%8F%A4%E4%BB%8A%E5%9B%BE%E4%B9%A6%E9%9B%86%E6%88%90&submit=&kt=1"><strong>古今图书集成</strong></a>
<br>

<a href="http://www.guoxuedashi.com/so.php?sokeygx=%E4%B8%9B%E4%B9%A6%E9%9B%86%E6%88%90&submit=&kt=1">丛书集成</a> 
<a href="http://www.guoxuedashi.com/so.php?sokeygx=%E5%9B%9B%E9%83%A8%E4%B8%9B%E5%88%8A&submit=&kt=1"><strong>四部丛刊</strong></a>  
<a href="http://www.guoxuedashi.com/so.php?sokeygx=%E8%AF%B4%E6%96%87%E8%A7%A3%E5%AD%97&submit=&kt=1">說文解字</a> <a href="http://www.guoxuedashi.com/so.php?sokeygx=%E5%85%A8%E4%B8%8A%E5%8F%A4&submit=&kt=1">三国六朝文</a>
<br><a href="http://www.guoxuedashi.com/so.php?sokeytm=%E6%97%A5%E6%9C%AC%E5%86%85%E9%98%81%E6%96%87%E5%BA%93&submit=&kt=1"><strong>日本内阁文库</strong></a> <a href="http://www.guoxuedashi.com/so.php?sokeytm=%E5%9B%BD%E5%9B%BE%E6%96%B9%E5%BF%97%E5%90%88%E9%9B%86&ka=100&submit=">国图方志合集</a> <a href="http://www.guoxuedashi.com/so.php?sokeytm=%E5%90%84%E5%9C%B0%E6%96%B9%E5%BF%97&submit=&kt=1"><strong>各地方志</strong></a>

</div>
</div>


<div class="sidebar2">
<center>

</center>
</div>
<div class="sidebar greenbar">
<div class="sidebar_title green">四库全书</div>
<div class="sidebar_info">

《四库全书》是中国古代最大的丛书,编撰于乾隆年间,由纪昀等360多位高官、学者编撰,3800多人抄写,费时十三年编成。丛书分经、史、子、集四部,故名四库。共有3500多种书,7.9万卷,3.6万册,约8亿字,基本上囊括了古代所有图书,故称“全书”。<a href="http://www.guoxuedashi.com/SiKuQuanShu/">详细>>
</a>

</div> 
</div>

</div>  <!--end r-->

</div>
<!-- 内容区END --> 

<!-- 页脚开始 -->
<div class="shh">

</div>

<div class="w1180" style="margin-top:8px;">
<center><script src="http://www.guoxuedashi.com/img/plus.php?id=3"></script></center>
</div>
<div class="w1180 foot">
<a href="/b/thanks.php">特别致谢</a> | <a href="javascript:window.external.AddFavorite(document.location.href,document.title);">收藏本站</a> | <a href="#">欢迎投稿</a> | <a href="http://www.guoxuedashi.com/forum/">意见建议</a> | <a href="http://www.guoxuemi.com/">国学迷</a> | <a href="http://www.shuowen.net/">说文网</a><script language="javascript" type="text/javascript" src="https://js.users.51.la/17753172.js"></script><br />
  Copyright &copy; 国学大师 古典图书集成 All Rights Reserved.<br>
  
  <span style="font-size:14px">免责声明:本站非营利性站点,以方便网友为主,仅供学习研究。<br>内容由热心网友提供和网上收集,不保留版权。若侵犯了您的权益,来信即刪。scp168@qq.com</span>
  <br />
ICP证:<a href="http://www.beian.miit.gov.cn/" target="_blank">鲁ICP备19060063号</a></div>
<!-- 页脚END --> 
<script src="http://www.guoxuedashi.com/img/plus.php?id=22"></script>
<script src="http://www.guoxuedashi.com/img/tongji.js"></script>

</body>
</html>
