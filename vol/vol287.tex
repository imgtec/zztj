






























































資治通鑑卷二百八十七 宋 司馬光 撰

胡三省 音註

後漢紀二|{
	起疆圉恊洽五月盡著雍涒灘二月不滿一年}


高祖睿文聖武昭肅孝皇帝中

天福十二年五月乙酉朔永康王烏雲召延壽及張礪和凝李崧馮道於所館飲酒|{
	所館者烏雲所館之地}
烏雲妻素以兄事延壽烏雲從容謂延夀曰|{
	從千容翻}
妹自上國來|{
	言其妻方自契丹中來}
寧欲見之乎延壽欣然與之俱入良久舒嚕出謂礪等曰燕王謀反適已鎖之矣又曰先帝在汴時遺我一籌|{
	遺唯季翻}
許我知南朝軍國|{
	朝直遥翻}
近者臨崩别無遺詔而燕王擅自知南朝軍國豈理邪下令延壽親黨皆釋不問間一日舒嚕至待賢館受蕃漢官謁賀笑謂張礪等曰燕王果於此禮上|{
	上時掌翻}
吾以鐵騎圍之諸公亦不免矣後數日集蕃漢之臣於府署|{
	恒州府署也}
宣契丹主遺制|{
	遺制  自為之也}
其略曰永康王大聖皇帝之嫡孫人皇王之長子太后鍾愛羣情允歸可於中京即皇帝位|{
	契丹主安巴堅謚大聖皇帝其長子東丹王托雲號人皇王托雲奔唐其子烏雲留本國不從契丹主耶律德光封之為永康王又德光取中國以恒州為中京}
於是始舉哀成服既而易吉服見羣臣不復行喪|{
	復扶又翻}
歌吹之聲不絶於内 辛巳以絳州防禦使王晏為建雄節度使|{
	王晏守絳州見上卷是年二月}
帝集羣臣庭議進取|{
	庭議者議之於庭}
諸將咸請出師井陘攻取鎮魏|{
	鎮州時為恒州契丹諸酋聚焉魏帥杜重威陘音刑}
先定河北則河南拱手自服帝欲自石會趨上黨|{
	趨七喻翻}
郭威曰虜主雖死黨衆猶盛各據堅城我出河北兵少路迂|{
	少詩沼翻下同迂音于又音紆曲也回遠也}
旁無應援若羣虜合勢共擊我軍進則遮前退則邀後糧餉路絶此危道也上黨山路險澁|{
	澁色入翻}
粟少民殘無以供億亦不可由近者陜晉二鎮相繼欵附|{
	陜晉歸附事見上卷上年陜失冉翻}
引兵從之萬無一失不出兩旬洛汴定矣帝曰卿言是也蘇逢吉等曰史弘肇大軍已屯上黨羣虜繼遁不若出天井抵孟津為便司天奏太歲在午不利南行|{
	隂陽家所謂逆太歲}
宜由晉絳抵陜|{
	九域志自晉州南至絳州一百二十五里自絳州南至陜州二百五十里自陜而東則至洛矣}
帝從之辛卯詔以十二日發北京|{
	自後唐以來以太原為北京是月乙酉朔十二日丙申}
告諭諸道 甲申以太原尹崇為北京留守以趙州刺史李存瓌為副留守河東幕僚真定李驤為少尹牙將太原蔚進為馬步指揮使以佐之|{
	李存瓌等後遂為北漢佐命瓌古回翻蔚紆勿翻姓也}
存瓌唐莊宗之從弟也|{
	從才用翻}
是日劉晞弃洛陽奔大梁|{
	以人心歸漢知不可守也}
武安節度副使天策府都尉領鎮南節度使馬希廣

|{
	鎮南軍洪州時屬唐}
楚文昭王希範之母弟也性謹順希範愛之使判内外諸司事壬辰夜希範卒將佐議所立都指揮使張少敵|{
	少詩照翻}
都押牙袁友恭以武平節度使知永州事希萼|{
	楚置武平節度於朗州朗永之疑注詳於後}
於希範諸弟為最長請立之|{
	長音知兩翻下同下齒長居長同}
長直都指揮使劉彦瑫|{
	瑫它牢翻}
天策府學士李弘臯鄧懿文小門使楊滌|{
	小門使諸鎮皆置之掌門戶之事府有宴集則執兵在門外}
皆欲立希廣張少敵曰永州齒長而性剛必不為都尉之下明矣必立都尉當思長策以制永州使帖然不動則可不然社稷危矣|{
	兄弟爭國社稷必危}
彦瑫等不從天策府學士拓跋恒曰三十五郎雖判軍府之政然三十郎居長請遣使以禮讓之不然必起爭端|{
	希廣第三十五希萼第三十藩府將吏稱府主之子為郎君}
彦瑫等皆曰今日軍政在手天與不取使它人得之異日吾輩安所自容乎希廣懦弱不能自決乙未彦瑫等稱希範遺命共立之|{
	史言劉彦瑫等為身謀以亂馬氏兄弟傳國長幼之序 考異曰十國紀年五月己丑希範得疾集國官告以傳位希廣湖湘故事希廣又不能強弱猶豫之間羣輔明日衆口勸廣乃受軍府排衙賀之以其事奏聞朝廷託以希範臨終之日遺言以付希廣按希範存時若已集國官傳位希廣則沒後將佐誰敢更有異議必彦瑫等假託希範遺令也今從湖湘故事}
張少敵退而歎曰禍其始此乎與拓跋恒皆稱疾不出|{
	為馬希萼攻殺希廣張本}
丙申帝發太原自隂地關出晉絳丁酉史弘肇奏克澤州始弘肇攻澤州刺史翟令奇固守不下|{
	翟萇伯翻}
帝以弘肇兵少欲召還|{
	還從宣翻}
蘇逢吉楊邠曰今陜晉河陽皆已向化崔廷勲耿崇美朝夕遁去|{
	時契丹之兵大勢已北還故知懷州之兵必不能久留}
若召弘肇還則河南人心動搖虜勢復壯矣帝未決使人諭指於弘肇|{
	句斷}
曰兵已及此勢如破竹可進不可退與逢吉等議合帝乃從之|{
	觀此則知帝猶憚契丹有未敢輕進之心}
弘肇遣部將李萬超說令奇|{
	說式芮翻}
令奇乃降|{
	降戶江翻}
弘肇以萬超權知澤州 崔廷勲耿崇美奚王伊喇合兵逼河陽張遇帥衆數千救之戰於南阪敗死|{
	太行南阪也帥讀曰率}
武行德出戰亦敗閉城自守伊喇欲攻之廷勲曰今北軍已去|{
	北軍謂契丹聚於恒州之軍崔廷勲等在南故謂屯恒之軍為北}
得此城何用且殺一夫猶可惜况一城乎聞弘肇已得澤州乃釋河陽還保懷州弘肇將至廷勲等擁衆北遁|{
	澤州南至懷州一百二十里耳漢兵又進而逼之故遁}
過衛州大掠而去|{
	九域志懷州東北至衛州二百九十三里}
契丹在河南者相繼北去弘肇引兵與武行德合弘肇為人沈毅寡言御衆嚴整將校小不從命立撾殺之|{
	沈持林翻將即亮翻校戶教翻撾則瓜翻}
士卒所過犯民田及繫馬於樹者皆斬之軍中惕息|{
	惕它歷翻}
莫敢犯令故所向必克帝自晉陽安行入洛及汴兵不血刃皆弘肇之力也帝由是倚愛之辛丑帝至霍邑|{
	霍邑漢彘縣後漢改曰永安隋改曰霍邑唐屬晉州九域志在州西北一百三十五里}
遣使諭河中節度使趙匡贊仍以契丹囚其父告之|{
	所以絶趙匡贊北顧之心}
滋德宫有宫人五十餘人|{
	五代會要晉天福四年改明德殿為滋德殿薛史曰以宫城南門同名故也}
蕭翰欲取之宦者張環不與翰破鎖奪宫人執環燒鐵灼之腹爛而死初翰聞帝擁兵而南欲北歸恐中國無主必大亂已不得從容而去|{
	從千容翻從容不急遽之貌}
時唐明宗子許王從益與王淑妃在洛陽|{
	王淑妃母子自晉入洛以後常居洛陽是年二月至大梁尋還洛陽}
翰遣高謨翰迎之矯稱契丹主命以從益知南朝軍國事召已赴恒州|{
	此矯契丹主烏雲之命也烏雲時尚在恒州恒戶登翻}
淑妃從益匿於徽陵下宫|{
	徽陵唐明宗陵梓宫所窆之所謂之下宫}
不得已而出至大梁翰立以為帝帥諸酋長拜之|{
	帥讀曰率酋慈秋翻長知兩翻}
又以禮部尚書王松御史中丞趙遠為宰相前宣徽使甄城翟光鄴為樞密使|{
	甄當作鄄音吉掾翻鄄城漢古縣也自唐以來帶濮州}
左金吾大將軍王景崇為宣徽使以北來指揮使劉祚權侍衛親軍都指揮使充在京巡檢|{
	北來謂先從契丹主自北而來者}
松徽之子也|{
	王徽相唐僖宗}
百官謁見淑妃|{
	見賢遍翻}
淑妃泣曰吾母子單弱如此而為諸公所推是禍吾家也翰留燕兵千人守諸門為從益宿衛|{
	燕於賢翻下同}
壬寅翰及劉晞辭行|{
	先是劉晞弃洛陽奔大梁}
從益餞於北郊遣使召高行周於宋州|{
	高行周唐明宗親將時帥歸德王淑妃欲以舊恩召之為衛}
武行德於河陽|{
	武行德并人必亦少在唐明宗麾下}
皆不至淑妃懼召大臣謀之曰吾母子為蕭翰所逼分當滅亡|{
	分扶問翻下處分同}
諸公無罪宜早迎新主|{
	以帝新舉大號擁兵南來將有中國故謂之新主}
自求多福勿以吾母子為意衆感其言皆未忍叛去或曰今集諸營不減五千與燕兵併力堅守一月北救必至|{
	此救謂契丹之救也}
淑妃曰吾母子亡國之餘|{
	後唐既亡惟王淑妃母子在耳故自謂然}
安敢與人爭天下不幸至此死生惟人所裁若新主見察當知我無所負今更為計畫則禍及它人闔城塗炭終何益乎衆猶欲拒守三司使文安劉審交曰余燕人豈不為燕兵計|{
	文安漢縣唐屬莫州以戰國七雄有固之大界言則唐之莫皆燕之南界以唐諸道節度言之則莫盧龍巡屬也故劉審交家於文安自謂燕人}
顧事有不可如何者今城中大亂之餘公私窮竭遺民無幾|{
	汴城經張彦澤剽掠契丹又席卷而北故云然幾居豈翻}
若復受圍一月無噍類矣願諸公勿復言一從太妃處分|{
	復扶又翻噍才笑翻處昌呂翻}
乃用趙遠翟光鄴策稱梁王知軍國事|{
	從益本爵許王以稱號於大梁自稱梁王是已建國更號矣今既奉表迎漢何為又更國號是當時議者禍之也}
遣使奉表稱臣迎帝請早赴京師仍出居私第 甲辰帝至晉州 契丹主烏雲以契丹主德光有子在國己以兄子襲位又無舒嚕太后之命|{
	舒嚕太后烏雲祖母也}
擅自立内不自安初契丹主安巴堅卒於勃海舒嚕太后殺酋長及諸將凡數百人|{
	事見二百七十五卷唐明宗天成元年二月}
契丹主德光復卒於境外|{
	復扶又翻}
酋長諸將懼死乃謀奉契丹主烏雲勒兵北歸契丹主以安國節度滿達勒為中京留守|{
	薛史滿達勒耶律德光之從弟其父曰薩喇安巴堅時自蕃中奔唐莊宗尋奔梁莊宗平梁獲之磔於市}
以前武州刺史高奉明為安國節度使晉文武官及士卒悉留於恒州獨以翰林學士徐台符李澣及後宫宦者教坊人自隨|{
	留文武官而以宫女宦官聲樂自隨史言烏雲無遠畧}
乙已發真定|{
	恒州建真定府}
帝之即位也絳州刺史李從朗與契丹將成霸卿等拒命|{
	成姓也何氏姓苑本自周文王子成伯之後周有成肅公又楚令尹子玉封於成是為成得臣其後亦以成為氏}
帝遣西南面招討使護國節度使白文珂攻之未下|{
	護國軍河中府時未得河中白文珂領節也珂丘何翻}
帝至城下命諸軍四布而勿攻以利害諭之戊申從朗舉城降帝命親將分護諸門士卒一人毋得入|{
	恐其入城剽掠}
以偏將薛瓊為防禦使 辛亥帝至陜州趙暉自御帝馬而入壬子至石壕|{
	九域志陜州陜縣有石壕鎮}
汴人有來迎者|{
	汴人越鄭洛而來迎可以見其苦契丹之虐政徯漢氏之來蘇惜乎卒無以副其望也}
六月甲寅朔蕭翰至恒州滿達勒以鐵騎圍張礪之第礪方卧病出見之翰數之曰汝何故言於先帝云胡人不可以為節度使|{
	張礪言見二百八十五卷晉齊王開運三年數所具翻}
又吾為宣武節度使且國舅也汝在中書乃帖我又先帝留我守汴州|{
	見上卷是年三月}
令我處宫中|{
	處昌呂翻}
汝以為不可又譛我及轄里於先帝云轄里好掠人財我好掠人子女|{
	好呼到翻}
今我必殺汝命鎖之礪抗聲曰此皆國家大體吾實言之欲殺即殺奚鎖為滿達勒以大臣不可專殺力救止之翰乃釋之是夕礪憤恚而卒|{
	恚於避翻}
崔廷勲見滿達勒趋拜跪而獻酒滿達勒踞而受之|{
	史言張礪抗直而蕭翰不敢殺崔廷勲過恭而滿達勒不為禮}
乙卯帝至新安|{
	新安縣屬西京河南府九域志在京西七十里}
西京留司官悉來迎 吳越忠獻王弘佐卒|{
	年二十}
遺令以丞相弘倧為鎮海鎮東節度使兼侍中|{
	倧徂冬翻}
丙辰帝至洛陽入居宫中汴州百官奉表來迎詔諭以受契丹補署者皆勿自疑聚其告牒而焚之趙遠更名上交|{
	避帝名也更工衡翻}
命鄭州防禦使郭從義先入大梁清宫密令殺李從益及王淑妃淑妃且死曰吾兒為契丹所立何罪而死何不留之使每歲寒食以一盂麥飯洒明宗陵乎|{
	五代會要曰人君奉先之道無寒食野祭近代莊宗每年寒食出祭謂之破散故襲而行之歐陽修曰寒食野祭而焚紙錢中國幾何其不為夷狄矣按唐開元敕寒食上墓禮經無文近世相傳寖以成俗宜許上墓同拜掃禮蓋唐許士庶之家行之而人君無此禮也}
聞者泣下|{
	為漢祖者待李從益以不死可也殺之過矣}
戊午帝發洛陽樞密院吏魏仁浦自契丹逃歸見於鞏|{
	見賢遍翻九域志鞏縣屬西京在京東一百一十里}
郭威問以兵數及故事仁浦強記精敏威由是親任之仁浦衛州人也 辛酉汴州百官竇貞固等迎於滎陽|{
	滎陽縣屬鄭州自鞏縣東至滎陽一百九十里}
甲子帝至大梁晉之藩鎮相繼來降 丙寅吳越王弘倧襲位 戊辰帝下詔大赦凡契丹所除節度使下至將吏各安職任不復變更|{
	復扶又翻}
復以汴州為東京|{
	契丹廢東京為汴州見上卷是年正月}
改國號曰漢仍稱天福年曰余未忍忘晉也復青襄汝三節度|{
	晉蓋以楊光遠反廢平盧軍以安從進反廢山南東道也汝州未嘗為節鎮恐是安州以李金全反廢安遠軍也然契丹入汴之後嘗以楊光遠子承信為平盧節度使蓋漢自以繼晉而興革契丹之政不以為著令也}
壬申以北京留守崇為河東節度使同平章事 契丹舒魯太后聞契丹主自立大怒發兵拒之契丹主以偉王為前鋒相遇於石橋|{
	胡嶠入遼録曰烏雲及舒嚕戰于沙河石橋蓋沙河之橋也南則姚家洲北則宣化館至西樓}
初晉侍衛馬軍都指揮使李彦韜從晉主北遷|{
	見上卷本年正月}
隸舒嚕太后麾下太后以為排陳使|{
	陳讀曰陣}
彦韜迎降於偉王太后兵由是大敗契丹主幽太后於安巴堅墓|{
	胡嶠入遼録曰烏雲囚舒嚕后於撲馬山又行三日始至西樓歐史曰契丹於安巴堅墓置祖州匈奴須知祖州東至上京五十里上京西樓也今並録之若其地名之同異道里之遠近必親歷然後能審其是}
改元天禄自稱天授皇帝以高勲為樞密使契丹主慕中華風俗多用晉臣而荒于酒色輕慢諸酋長由是國人不附諸部數叛|{
	數所角類}
興兵誅討故數年之間不暇南寇|{
	史言中國經喪亂之後由此得稍自安集}
初契丹主德光命奉國都指揮使南宫王繼弘|{
	南宫縣屬冀州九域志在州西南六十二里}
都虞侯樊暉以所部兵戍相州彰德節度使高唐英善待之|{
	高唐英契丹所署也見上卷是年四月相息亮翻}
戍兵無鎧仗唐英以鎧仗給之倚信如親戚唐英聞帝南下舉鎮請降使者未返繼弘暉殺唐英繼弘自稱留後遣使告云唐英反覆詔以繼弘為彰德留後庚辰以暉為磁州刺史|{
	磁墻之翻}
安國節度使高奉明聞唐英死心不自安請滿達勒署馬步都指揮使劉鐸為節度副使知軍府事身歸恒州|{
	邢相既不能守恒州安能孤立哉為諸將逐滿達勒張本}
帝遣使告諭荆南高從誨上表賀且求郢州帝不許及加恩使至拒而不受|{
	自唐以來新君踐阼則遣使加恩於諸鎮使疏吏翻}
唐主聞契丹主德光卒蕭翰弃大梁去下詔曰乃眷中原本朝故地|{
	唐主自謂出於吳王恪故云然朝直遥翻}
以左右衛聖統軍忠武節度使李金全為北面行營招討使|{
	李金全晉將也奔唐見二百八十二卷晉高祖天福五年}
議經畧北方聞帝已入大梁遂不敢出兵 秋七月甲午以馬希廣為天策上將軍武安節度使江南諸道都統兼中書令封楚王|{
	因即位加恩遂命馬希廣以其父兄官爵}
或傳趙延壽已死郭威言於帝曰趙匡贊契丹所署|{
	見上卷本年正月}
今猶在河中宜遣使弔祭因起復移鎮彼既家國無歸|{
	父死虜中無可歸之家契丹北去無可歸之國}
必感恩承命從之會鄴都留守天雄節度使兼中書令杜重威天平節度使兼侍中李守貞皆奉表歸命重威仍請移它鎮歸德節度使兼中書令高行周入朝丙申徙重威為歸德節度使以行周代之|{
	杜重威尋不受代遂命高行周攻之}
守貞為護國節度使加兼中書令|{
	為李守貞據河中張本}
徙護國節度使趙匡贊為晉昌節度使後二年延壽始卒於契丹|{
	史明傳者之妄}
吳越王弘倧以其弟台州刺史弘俶同參相府事|{
	俶昌六翻}
李達以其弟通知福州留後|{
	李仁達降唐唐賜名弘義編之屬籍及其叛唐為唐所攻求救於吳越而弘字犯吳越諱改名為達其弟先名弘通亦止名通}
自詣錢唐見吳越王弘倧弘倧承制加達兼侍中更其名曰孺贇|{
	更工衡翻}
既而孺贇悔懼|{
	悔其來且懼死也}
以金筍二十株及雜寶賂内牙統軍使胡進思求歸福州進思為之請弘倧從之|{
	為於偽翻為李孺贇叛誅胡思進不自安張本}
杜重威自以附契丹負中國|{
	事見一百八十五卷晉齊王開運三年}
内常疑懼及移鎮制下復拒而不受遣其子弘璲質於滿達勒求援|{
	璲音遂質音致}
趙延壽有幽州親兵二千在恒州|{
	趙延夀為契丹主卾納鎖之北去其親兵留恒州恒戶登翻}
指揮使張璉將之重威請以守魏|{
	為張璉助杜重威堅守張本將即亮翻}
滿達勒遣將楊衮將契丹千五百人及幽州兵赴之閏月庚午詔削奪重威官爵以高行周為招討使鎮寧節度使慕容彦超副之以討重威|{
	為慕容彦超挾勢陵轢高行周將帥不和張本}
辛未楊邠郭威王章皆為正使|{
	帝即位於太原以楊邠權樞密使郭威權樞密副使王章權三司使今皆為正使}
時兵荒之餘公私匱竭北來兵與朝廷兵合頓增數倍|{
	北來兵謂從帝及史弘肇自太原來者朝廷兵謂晉朝舊兵}
章白帝罷不急之務省無益之費以奉軍用度克贍 庚辰制建宗廟太祖高皇帝世祖光武皇帝皆百世不遷又立四親廟追尊謚號|{
	五代會要追尊高祖湍明元皇帝廟號文祖曾祖昂恭僖皇帝廟號德祖祖僎昭獻皇帝廟號翼祖考琠章聖皇帝廟號顯祖}
凡六廟滿達勒貪猾殘忍民間有珍貨美婦女必奪取之又捕村民誣以為盜披面抉目斷腕|{
	抉於決翻斷音短下即斷同腕烏貫翻}
焚炙而殺之欲以威衆常以其具自隨|{
	具謂披面抉目斷腕焚炙之具}
左右懸人肝膽手足飲食起居於其間語笑自若出入或被黄衣用乘輿服御物|{
	被皮義翻乘繩證翻}
曰兹事漢人以為不可吾國無忌也又以宰相員不足乃牒馮道判弘文館李崧判史館和凝判集賢劉昫判中書其僭妄如此|{
	宰相分判須降制勑滿達勒以牒行之史言其僭妄}
然契丹或犯法無所容貸故市肆不擾常恐漢人妄去謂門者曰漢有窺門者即斷其首以來滿達勒使督運於洺州洺州防禦使薛懷讓聞帝入大梁殺其使者舉州降帝遣郭從義將兵萬人會懷讓攻劉鐸於邢州不克|{
	劉鐸為契丹守九域志洺州西北至邢州九十里}
鐸請兵於滿達勒滿達勒遣將楊安及前義武節度使李殷將千騎攻懷讓於洺州懷讓嬰城自守安等縱兵大掠於邢洺之境契丹所留兵不滿二千|{
	謂留恒州之兵也}
滿達勒令所司給萬四千人食收其餘自入滿達勒常疑漢兵且以為無用稍稍廢省又損其食以飼胡兵|{
	飼祥吏翻}
衆心怨憤聞帝入大梁皆有南歸之志前潁州防禦使何福進控鶴指揮使太原李榮潜結軍中壯士數十人謀攻契丹然畏契丹尚彊猶豫未發會楊衮楊安等軍出|{
	楊衮赴魏州楊安攻洺州}
契丹留恒州者纔八百人福進等遂決計約以擊佛寺鍾為號|{
	約漢兵聞佛寺擊鍾則齊出攻契丹然佛寺晨昏擊鍾食時撃鍾日日然也此必以未發前預相戒約以次日食時閒佛寺鍾聲而俱發耳}
辛巳契丹主烏雲遣騎至恒州召前威勝節度使兼中書令馮道樞密使李崧左僕射和凝等會葬契丹主德光於木葉山道等未行食時鍾聲發漢兵奪契丹守門者兵擊契丹殺十餘人因突入府中李榮先據甲庫悉召漢兵及市人以鎧仗授之焚牙門與契丹戰榮召諸將并力護聖左廂都指揮使恩州團練使白再榮|{
	恩州時屬南漢境白再榮遥領也}
狐疑匿於别室軍吏以佩刀決幕引其臂|{
	白再榮以幕自蔽軍吏決幕引出之}
再榮不得已而行諸將繼至煙火四起鼓譟震地滿達勒大驚載寶貨家屬走保北城而漢兵無所統壹貪狡者乘亂剽掠懦者竄匿|{
	剽匹妙翻}
八月壬午朔契丹自北門入|{
	恒州牙城北門也}
勢復振漢民死者二千餘人前磁州刺史李穀恐事不濟請馮道李崧和凝至戰所慰勉士卒士卒見道等至爭自奮|{
	微李穀之謀漢兵殆矣}
會日暮有村民數千譟於城外欲奪契丹寶貨婦女契丹懼而遁滿達勒劉晞崔廷勲皆奔定州|{
	恒州東北至定州一百二十里}
與義武節度使耶律忠合忠即隆卾特|{
	隆卾特鎮澶州而兵亂契丹又使鎮定州}
馮道等四出安撫兵民衆推道為節度使道曰我書生也當奏事而已宜擇諸將為留後時李榮功最多|{
	李榮先據甲庫授兵與契丹戰諸將皆繼其後故論功最多}
而白再榮位在上乃以再榮權知留後具以狀聞且請援兵帝遣左飛龍使李彦從將兵赴之|{
	唐有飛龍使及小馬坊使梁改小馬坊為天驥後唐復舊長興元年改飛龍院為左飛龍院小馬坊為右飛龍院宋太平興國三年改左右天廐坊雍熙二年又改左右騏驥院使}
白再榮貪昧猜忌諸將奉國軍主華池王饒|{
	晉氏南渡以後南北兵爭各置軍主隊主之官隋唐以下無是也此書奉國軍主通鑑蓋因舊史成文猶言軍師耳非官名也慶州華池縣隋所置宋熙寧中省華池縣為寨鎮屬合水縣其地在慶州之東南宋白曰華池本漢歸德縣地即洛源縣隋仁夀二年於今縣東北二里庫多汗故城又置華池縣南有華池水故名}
恐為再榮所併詐稱足疾據東門樓嚴兵自衛司天監趙延乂善於二人往來諭釋始得解再滎以李崧和凝久為相家富|{
	晉高祖入洛即以李崧為相天福五年和凝為相}
遣軍士圍其第求賞給崧凝各以家財與之又欲殺崧凝以滅口李穀往見再榮責之曰國亡主辱公輩握兵不救今僅能逐一虜將鎮民死者幾三千人|{
	虜將滿達勒恒舊鎮州也}
豈獨公之力邪纔得脱死遽欲殺宰相新天子若詰公專殺之罪|{
	詰去吉翻}
公何辭以對再榮懼而止又欲率民財以給軍穀力爭之乃止漢人嘗事滿達勒再榮皆拘之以取其財恒人以其貪虐謂白滿達勒|{
	言其貪虐似滿達勒姓白耳然再榮以貪虐殖財郭威入汴竟以多財殞其身天道好還蓋昭昭矣}
楊衮至邢州聞滿達勒逐即日北還楊安亦遁去李殷以其衆來降 庚寅以薛懷讓為安國節度使劉鐸聞滿達勒遁舉邢州降懷讓詐云巡檢引兵向邢州鐸開門納之懷讓殺鐸以克復聞朝廷知而不問 辛卯復以恒州順國軍為鎮州成德軍|{
	改恒州及順國軍見二百八十三卷晉高祖天福七年}
乙未以白再榮為成德留後踰年始以何福進為曹州防禦使李榮為博州刺史|{
	踰年之後乃知逐滿達勒二人之功始賞之此事與晉高祖天福二年馬萬盧順密之事同}
敕盜賊母問贓多少皆抵死時四方盜賊多朝廷患之故重其法仍令命使者逐捕蘇逢吉自草詔意云應賊盜并四鄰同保皆全族處斬|{
	處昌呂翻}
衆以為盜猶不可族况鄰保乎逢吉固爭不得已但省去全族字|{
	夫羌呂翻}
由是捕賊使者張令柔殺平隂十七村民|{
	劉昫曰平隂漢肥塜縣隋為平隂縣屬濟州唐屬鄆州九域志平隂縣在鄆州東北一百二十里項安世家說曰古無村名今之村即古之鄙野也凡地在國中邑中則名之為都都美也言其人物衣制皆雅麗也凡言美者曰都曰子都都人士車騎甚都是也郊外則名之為野為鄙言其樸拙無文也曰鄙者如列子所謂鄭之鄙人是也故古語謂美好為都麄陋為鄙本此為義也隋世已有村名唐令在田野者為村置村正一人則村之為義明矣}
逢吉為人文深好殺|{
	好呼到翻}
在河東幕府|{
	謂為河東節度判官時也}
帝嘗令靜獄以祈福逢吉盡殺獄囚還報|{
	靜獄者使之決遣繫囚而蘇逢吉盡殺之以為靜}
及為相朝廷草創帝悉以軍旅之事委楊邠郭威百司庶務委逢吉及蘇禹珪二相決事皆出胸臆不拘舊制雖事無留滯而用捨黜陟惟其所欲帝方倚信之無敢言者逢吉尤貪詐公求貨財無所顧避繼母死不為服庶兄自外至不白逢吉而見諸子逢吉怒密語郭威以它事杖殺之|{
	語牛倨翻蘇逢吉之好殺固天道所不容况怙勢而殺其兄乎}
楚王希廣庶弟天策左司馬希崇性狡險隂遺兄希萼書|{
	遺唯季翻}
言劉彦瑫違先王之命|{
	先王謂楚王殷也殷遺命見二百七十七卷唐明宗長興元年}
廢長立少以激怒之|{
	希萼兄也希廣弟也捨兄立弟故云然長知兩翻少詩沼翻}
希萼自永州來奔喪|{
	歐史曰希萼自朗州來奔喪通鑑於是年正月楚王希範之卒將佐議所立亦言希萼知永州事但希萼為武平節度使武平軍置於朗州下文言希萼求還朗州又希廣欲分潭朗而治則朗州為是前此作永州誤也}
乙巳至趺石|{
	趺甫無翻}
彦瑫白希廣遣侍從都指揮使周廷誨等將水軍逆之|{
	從才用翻}
命永州將士皆釋甲而入館希萼於碧湘宫|{
	館古玩翻今潭州西北出有碧湘門馬氏盖立宫於是門之側}
成服於其次不聽入與希廣相見希萼求還朗州|{
	還從宣翻又如字}
周廷誨勸希廣殺之希廣曰吾何忍殺兄|{
	馬希廣其後唐閔帝之儔乎}
寧分潭朗而治之|{
	治直之翻}
乃厚贈希萼遣還朗州希崇常為希萼詗希廣|{
	為于偽翻詗古永翻又翾正翻}
語言動作悉以告之約為内應|{
	史言希萼之攻潭州希崇啟之也}
契丹之滅晉也驅戰馬二萬歸其國|{
	事見上卷是年正月}
至是漢兵乏馬詔市士民馬於河南諸道不經剽掠者|{
	剽匹妙翻}
制以錢弘倧為東南兵馬都元帥鎮海鎮東節度使兼中書令吳越王高從誨聞杜重威叛發水軍數千襲襄州|{
	以漢兵方北討魏州}


|{
	未暇南救也}
山南東道節度使安審琦擊却之又寇郢州刺史尹實大破之|{
	九域志荆南府北至襄州四百四十里東至郢州三百二十里}
乃絶漢附于唐蜀|{
	高從誨求郢州不許見上六月}
初荆南介居湖南嶺南福建之間|{
	此語專為三道入貢過荆南發}
地狹兵弱自武信王季興時諸道入貢過其境者多掠奪其貨幣|{
	過音戈}
及諸道移書詰讓或加以兵不得已復歸之|{
	詰去吉翻復扶又翻}
曾不為愧及從誨立唐晉契丹漢更據中原|{
	更工衡翻}
南漢閩吳蜀皆稱帝從誨利其賜予|{
	予讀曰與}
所向稱臣諸國賤之謂之高無賴|{
	俚俗語謂奪攘苟得無愧耻者為無賴}
唐主以太傅兼中書令宋齊丘為鎮南節度使 南漢主恐諸弟與其子爭國殺齊王弘弼貴王弘道定王弘益辨王弘濟同王弘簡益王弘建恩王弘偉宜王弘照盡殺其男納其女充後宫|{
	劉晟殘同氣而瀆天倫桀紂之虐不如是之甚也}
作離宫千餘間飾以珠寶設鑊湯鐵牀刳剔等刑號生地獄嘗醉戲以瓜置樂工之頸試劒遂斷其頭|{
	歐史伶人謂之尚玉樓即被斬之樂工也斷音短}
初帝與吏部尚書竇貞固俱事晉高祖雅相知重及即位欲以為相問蘇逢吉其次誰可相者逢吉與翰林學士李濤善因薦之曰昔濤乞斬張彦澤|{
	事見二百八十三卷晉高祖天福七年}
陛下在太原嘗重之此可相也會高行周慕容彦超共討杜重威於鄴都|{
	遣二將討杜重威事始上閏七月}
彦超欲急攻城行周欲緩之以待其弊行周女為重威子婦彦超揚言行周以女故愛賊不攻由是二將不協|{
	慕容彦超既以帝同產之親而陵高行周又誣行周以婚姻之故而緩賊故不協}
帝恐生它變欲自將擊重威意未決濤上疏請親征帝大悦以濤有宰相器九月甲戌加逢吉左僕射兼門下侍郎蘇禹珪右僕射兼中書侍郎貞固司空兼門下侍郎濤戶部尚書兼中書侍郎並同平章事|{
	竇貞固以司空拜相而書於二僕射之次者二蘇舊相貞固則新相也}
戊寅詔幸澶魏勞軍|{
	澶時連翻勞力到翻}
以皇子承訓為東京留守 馮道李崧和凝自鎮州還|{
	白再榮等既逐契丹馮道等乃得免而還還從宣翻又如字}
己卯以崧為太子太傅凝為太子太保 庚辰帝發大梁 晉昌節度使趙匡贊|{
	是年秋七月趙匡贊自河中徙長安}
恐終不為朝廷所容冬十月遣使降蜀請自終南山路出兵應援|{
	終南山路子午谷路也}
戊戌帝至鄴都城下舍於高行周營|{
	人主親戎不為御營而舍于元帥之營有入韓信壁奪軍之意高行周心迹無它故不發}
行周言於帝曰城中食未盡急攻徒殺士卒未易克也|{
	易以䜴翻}
不若緩之彼食盡自潰帝然之慕容彦超數因事陵轢行周|{
	數所角翻轢郎擊翻}
行周泣訴於執政掬糞壤實其口|{
	示受陵辱而不敢言也}
蘇逢吉楊邠密以白帝帝深知彦超之曲猶命二臣和解之又召彦超於帳中責之|{
	不明底彦超之罪牽於愛也}
且使詣行周謝杜重威聲言車駕至即降帝遣給事中陳觀往諭指重威復閉門拒之|{
	復扶又翻下同}
城中食浸竭將士多出降者慕容彦超固請攻城帝從之丙午親督諸將攻城自寅至辰士卒傷者萬餘人死者千餘人不克而止彦超乃不敢復言|{
	死傷者多而城不克則高行周持久以弊之之說為是慕容彦超之語遂塞}
初契丹留幽州兵千五百戍大梁|{
	即蕭翰所留也見上五月}
帝入大梁或告幽州兵將為變帝盡殺之於繁臺之下|{
	繁臺在大梁丁度曰繁臺本師曠吹臺梁孝王增築曰繁臺薛史曰繁臺即梁王吹臺其後有繁氏居其側里人乃以姓呼之}
及圍鄴都張璉將幽州兵二千助重威拒守|{
	張璉入鄴都助重威事始上七月}
帝屢遣人招諭許以不死璉曰繁臺之卒何罪而戮今守此以死為期耳由是城久不下十一月丙辰内殿直韓訓獻攻城之具帝曰城之所恃者衆心耳衆心苟離城無所保用此何為|{
	始用高行周之言}
杜重威之叛觀察判官金鄉王敏屢泣諫不聽|{
	金鄉縣唐初屬濟州後屬兖州九域志屬濟州在州東南九十里}
及食竭力盡甲戌遣敏奉表出降乙亥重威子弘璉來見|{
	見賢遍翻下同}
丙子妻石氏來見石氏即晉之宋國長公主也|{
	長知兩翻}
帝復遣入城丁丑重威開門出降城中餒死者什七八存者皆尫瘠無人狀|{
	尫烏黄翻瘠秦昔翻}
張璉先邀朝廷信誓詔許以歸鄉里及出降殺璉等將校數十人縱其士卒北歸將出境大掠而去|{
	幽州兵將出魏州之境去漢兵既遠心無所憚遂大掠逞其忿而去將即亮翻校戶教翻}
郭威請殺重威牙將百餘人并重威家貲籍之以賞戰士從之以重威為太傅兼中書令楚國公重威每出入路人往往擲瓦磔詬之|{
	以其歷藩鎮則貪黷無厭為將則賣國殄民也為殺杜重威市人噉其肉張本詬苦侯翻又許侯翻}


臣光曰漢高祖殺幽州無辜千五百人非仁也誘張璉而誅之非信也杜重威罪大而赦之非刑也仁以合衆信以行令刑以懲奸失此三者何以守國其祚運之不延也宜哉

高行周以慕容彦超在澶州固辭鄴都|{
	澶魏相去百五十里行周彦超既交惡接境而處必不相安故力辭}
己卯以忠武節度使史弘肇領歸德節度使兼侍衛馬步都指揮使義成節度使劉信領忠武節度使兼侍衛馬步副都指揮使徙彦超為天平節度使並加同平章事 吳越王弘倧大閲水軍賞賜倍於舊胡進思固諫弘倧怒投筆水中曰吾之財與士卒共之奚多少之限邪|{
	為胡進思廢弘倧張本}
十二月丙戌帝發鄴都|{
	發自鄴都而歸大梁}
蜀主遣雄武都押牙吳崇惲|{
	雄武都押牙秦州都押牙也憚於粉翻}
以樞密使王處回書招鳳翔節度使侯益|{
	處昌呂翻}
庚寅以山南西道節度使兼中書令張䖍釗為北面行營招討安撫使雄武節度使何重建副之|{
	張䖍釗以潞王之亂攻鳳翔而敗降蜀何重建以契丹入中國降蜀故蜀主用之以經略岐雍重直龍翻}
宣徽使韓保貞為都虞侯共將兵五萬䖍釗出散關重建出隴州以擊鳳翔|{
	既遣使招侯益又隨之以兵臨脅之}
奉鑾肅衛都虞侯李廷珪將兵二萬出子午谷以援長安|{
	從趙匡贊之請也}
諸軍發成都旌旗數十里 辛卯皇子開封尹承訓卒承訓孝友忠厚達於從政人皆惜之|{
	史言承訓死而漢祚蹙}
癸巳帝至大梁威武節度使李孺贇與吳越戍將鮑修讓不協謀襲

殺修讓復以福州降唐修讓覺之引兵攻府第|{
	復扶又翻府第福州府署也}
是日殺孺贇夷其族|{
	李仁達據福州事始見二百八十五卷晉齊王開運二年史言狂狡反覆者終死于人手}
乙未追立皇子承訓為魏王 侯益請降於蜀使吳崇惲持兵籍糧帳西還|{
	還從宣翻又如字}
與趙匡贊同上表請出兵平定關中 己酉鮑修讓傳李孺贇首至錢塘吳越王弘倧以丞相山隂吳程知威武節度事 吳越王弘倧性剛嚴憤忠獻王弘佐時容養諸將政非己出|{
	按歐使吳越王錢鏐以徐綰之亂使子元瓘質於宣州以胡進思戴惲等自隨元瓘嗣立用進思為大將元瓘卒而弘佐立進思以舊將自待甚見尊禮及倧立頗卑侮之進思不能平}
及襲位誅杭越侮法吏三人|{
	侮當作舞}
内牙統軍使胡進思恃迎立功干預政事弘倧惡之|{
	惡烏路翻}
欲授以一州|{
	欲奪其兵權而遠之}
進思不可進思有所謀議弘倧數面折之進思還家設忠獻王位被髪慟哭|{
	數所角翻折之舌翻被皮義翻}
民有殺牛者吏按之引人所市肉近千斤|{
	近其靳翻}
弘倧問進思牛大者肉幾何對曰不過三百斤弘倧曰然則吏妄也命按其罪進思拜賀其明弘倧曰公何能知其詳進思踧踖對曰|{
	踧子六翻踖子昔翻}
臣昔未從軍亦嘗從事於此進思以弘倧為知其素業故辱之益恨怒|{
	此禇遂良所以戒唐太宗窮張玄素也}
進思建議遣李孺贇歸福州|{
	見上七月}
及孺贇叛|{
	謂復欲降唐也}
弘倧責之進思愈不自安弘倧與内牙指揮使何承訓謀逐進思又謀於内都監使水丘昭劵|{
	按薛史吳越王鏐母水丘氏昭劵蓋外戚也}
昭劵以為進思黨盛難制不如容之弘倧猶豫未決承訓恐事洩反以謀告進思|{
	古人有言需者事之賊弘倧猶豫不決故何承訓懼而生心洩息列翻}
庚戌晦弘倧夜宴將吏進思疑其圖已與其黨謀作亂帥親兵百人|{
	帥讀曰率下同}
戎服執兵入見於天策堂曰老奴無罪王何故圖之弘倧叱之不退左右持兵者皆憤怒弘倧猝愕不暇發言|{
	乘左右之憤怒而用之以順討逆何畏乎胡進思是以人貴於有膽决}
趨入義和院進思鎖其門矯稱王命告中外云猝得風疾傳位於同參相府事弘俶進思因帥諸將迎弘俶于私第且召丞相元德昭德昭至立於簾外不拜曰俟見新君進思亟出褰簾|{
	褰起䖍翻}
德昭乃拜進思稱弘倧之命承制授弘俶鎮海鎮東節度使兼侍中弘俶曰能全吾兄乃敢承命不然當避賢路進思許之弘俶始視事進思殺水丘昭劵及進侍鹿光鉉|{
	進侍吳越所置官在王左右者也}
光鉉弘倧之舅也進思之妻曰它人猶可殺昭劵君子也奈何害之|{
	史言婦人智識有過於丈夫者}
是歲唐主以羽林大將軍王延政為安化節度使翻陽王鎮饒州|{
	唐蓋置安化軍於饒州王延政降唐見二百八十四卷晉齊王開運二年南唐之保大三年也}


乾祐元年春正月乙卯大赦改元 帝以趙匡贊侯益與蜀兵共為寇患之會回鶻入貢訴稱為党項所阻|{
	自唐長興以來西路党項部族劫掠使臣及外域進奉唐雖遣兵討之莫能遏止党底朗翻}
乞兵應接詔左衛大將軍王景崇將軍齊藏珍將禁軍數千赴之因使之經略關西|{
	因應接回鶻使者之名以出師實則經畧關右}
晉昌節度判官李恕久在趙延壽幕下延壽使之佐匡贊匡贊將入蜀恕諫曰燕王入朝豈所願哉|{
	言趙延夀受囚鎖於契丹而入北}
今漢家新得天下方務招懷若謝罪歸朝必保富貴入蜀非全計也蹄涔不容尺鯉|{
	劉曜之言岑耡針翻蹄涔謂牛馬所踐之跡因而渟水處也非盈尺之鯉所可容身以喻蜀小國勢不能容趙匡贊}
公必悔之匡贊乃遣恕奉表請入朝景崇等未行而恕至帝問恕匡贊何為附蜀對曰匡贊自以身受虜官|{
	謂先受契丹主耶律德光之命鎮河中府}
父在虜庭|{
	父謂趙延夀}
恐陛下未之察故附蜀求苟免耳臣以為國家必應存撫故遣臣來祈哀帝曰匡贊父子本吾人也不幸陷虜今延壽方墜檻穽|{
	趙延壽為契丹所鎖事見去年五月}
吾何忍更害匡贊乎即聽其入朝侯益亦請赴二月四日聖壽節上壽|{
	五代會要帝生於唐乾寧二年二月四日}
景崇等將行帝召入臥内敕之曰匡贊益之心皆未可知汝至彼彼已入朝則勿問若尚遷延顧望當以便宜從事 己未帝更名暠|{
	更工衡翻暠古老翻}
以前威勝節度使馮道為太師 壬戌吳越王弘俶遷故王弘倧於衣錦軍私第|{
	遷於臨安私第也}
遣匡武都頭薛温將親兵衛之潜戒之曰若有非常處分皆非吾意當以死拒之|{
	處昌呂翻分扶問翻弘俶知胡進思必謀殺弘倧故密約敕薛温使知所備為進思害弘倧而不克張本}
帝自魏王承訓卒悲痛過甚甲子始不豫 趙匡贊不俟李恕返命已離長安丙子入見|{
	離力智翻見賢遍翻}
王景崇等至長安聞蜀兵已入秦川|{
	自大散關以北達于岐雍夾渭川南北岸沃野千里謂之秦川}
以兵少發本道及趙匡贊牙兵千餘人同拒之|{
	本道謂晉昌一道}
景崇恐匡贊牙兵亡逸欲文其面微露風旨軍校趙思綰首請自文其面以帥下|{
	文其面以軍號則亡逸無所至校戶教翻帥讀曰率}
景崇悦齊藏珍竊言曰思綰凶暴難制不如殺之景崇不聽思綰魏州人也|{
	為趙思綰據長安反張本}
蜀李廷珪將至長安聞趙匡贊已入朝欲引歸王景崇邀之敗廷珪于子午谷|{
	敗補邁翻下追敗同}
張䖍釗至寶雞諸將議不協按兵未進侯益聞廷珪西還因閉壁拒蜀兵䖍釗勢孤引兵夜遁景崇帥鳳翔隴邠涇鄜坊之兵追敗蜀兵於散關俘將卒四百人|{
	李廷珪張䖍釗二軍皆蜀主去年十二月所遣帥讀曰率鄜音夫敗補邁翻}
丁丑帝大漸楊邠忌侍衛馬步都指揮使忠武節度使劉信立遣之鎮|{
	劉信以從弟之親典侍衛故楊邠忌之遣就鎮許州}
信不得奉辭雨泣而去|{
	涕泣如雨謂之雨泣}
帝召蘇逢吉楊邠史弘肇郭威入受顧命曰余氣息微不能多言承祐幼弱後事託在卿輩又曰善防重威|{
	諭以誅杜重威也}
是日殂于萬歲殿|{
	年五十四薛史梁受禪以大梁萬歲堂為萬歲殿}
逢吉等祕不發喪庚辰下詔稱重威父子因朕小疾謗議搖衆并其子弘璋弘璉弘璨皆斬之晉公主及内外親族一切不問|{
	晉公主石氏杜重威之妻}
磔重威尸於市|{
	磔陟格翻}
市人爭啖其肉|{
	啖徒濫翻怨杜重威賣國引虜入汴而都人被毒也}
吏不能禁斯須而盡二月辛巳朔立皇子左衛大將軍大内都點檢承祐為周王同平章事有頃發喪宣遺制令周王即皇帝位時年十八 蜀韓保貞龎福誠引兵自隴州還|{
	韓保貞亦蜀主去年十二月所遣還從宣翻又如字}
要何重建俱西是日保貞等至秦州分兵守諸門及衢路重建遂入于蜀|{
	要一遥翻天福十二年何重建附蜀至是蜀兵刧與俱西}
丁亥尊皇后曰皇太后朝廷知成德留後白再榮非將帥才庚寅以前建雄留後劉在明代之 癸巳大赦|{
	即位十三日而肆赦}
吳越内牙指揮使何承訓復請誅胡進思及其黨|{
	復扶又翻}
吳越王弘俶惡其反覆且懼召禍乙未執承訓斬之|{
	惡烏路翻何承訓泄弘倧之謀以䧟君於幽廢而又請弘俶誅胡進思誰敢復與之謀乎}
進思屢請殺廢王弘倧以絶後患弘俶不許進思詐以王命密令薛温害之温曰僕受命之日不聞此言不敢妄發進思乃夜遣其黨方安二人踰垣而入弘倧闔戶拒之大呼求救|{
	呼火故翻}
温聞之率衆而入斃安等于庭中入告弘俶|{
	自臨安入錢唐告其事}
弘俶大驚曰全吾兄汝之力也弘俶畏忌進思曲意下之|{
	下戶嫁翻}
進思亦内憂懼未幾疽發背卒|{
	幾居豈翻}
弘倧由是獲全 詔以王景崇兼鳳翔巡檢使景崇引兵至鳳翔侯益尚未行景崇以禁兵分守諸門或勸景崇殺益景崇以受先朝密旨|{
	密旨謂高祖卧内便宜從事之命也見上朝直遥翻下同}
嗣主未之知或疑於專殺猶豫未決益聞之不告景崇而去景崇悔自詬|{
	詬古侯翻又許侯翻}
戊戌益入朝隱帝問何故召蜀軍對曰臣欲誘致而殺之帝哂之|{
	誘音酉哂矢忍翻笑不壞顔為哂}
蜀張䖍釗自恨無功癸卯至興州慙忿而卒|{
	自散關還至興州也張䖍釗蓋不知可否不度利鈍而急於求功之人觀其攻王都於定州攻潞王於鳳翔皆急於求勝而敗可知已恚於避翻}
侍衛馬步都指揮使同平章事史弘肇遭母喪不數日復出朝參|{
	復扶又翻居喪而經營起復已得罪於名教未起復而自出朝參雖史弘肇武人無識亦可見朝章之紊朝直遥翻}


資治通鑑卷二百八十七














































































































































