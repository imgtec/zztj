<!DOCTYPE html PUBLIC "-//W3C//DTD XHTML 1.0 Transitional//EN" "http://www.w3.org/TR/xhtml1/DTD/xhtml1-transitional.dtd">
<html xmlns="http://www.w3.org/1999/xhtml">
<head>
<meta http-equiv="Content-Type" content="text/html; charset=utf-8" />
<meta http-equiv="X-UA-Compatible" content="IE=Edge,chrome=1">
<title>資治通鑒_35-資治通鑑卷三十四_35-資治通鑑卷三十四</title>
<meta name="Keywords" content="資治通鑒_35-資治通鑑卷三十四_35-資治通鑑卷三十四">
<meta name="Description" content="資治通鑒_35-資治通鑑卷三十四_35-資治通鑑卷三十四">
<meta http-equiv="Cache-Control" content="no-transform" />
<meta http-equiv="Cache-Control" content="no-siteapp" />
<link href="/img/style.css" rel="stylesheet" type="text/css" />
<script src="/img/m.js?2020"></script> 
</head>
<body>
 <div class="ClassNavi">
<a  href="/24shi/">二十四史</a> | <a href="/SiKuQuanShu/">四库全书</a> | <a href="http://www.guoxuedashi.com/gjtsjc/"><font  color="#FF0000">古今图书集成</font></a> | <a href="/renwu/">历史人物</a> | <a href="/ShuoWenJieZi/"><font  color="#FF0000">说文解字</a></font> | <a href="/chengyu/">成语词典</a> | <a  target="_blank"  href="http://www.guoxuedashi.com/jgwhj/"><font  color="#FF0000">甲骨文合集</font></a> | <a href="/yzjwjc/"><font  color="#FF0000">殷周金文集成</font></a> | <a href="/xiangxingzi/"><font color="#0000FF">象形字典</font></a> | <a href="/13jing/"><font  color="#FF0000">十三经索引</font></a> | <a href="/zixing/"><font  color="#FF0000">字体转换器</font></a> | <a href="/zidian/xz/"><font color="#0000FF">篆书识别</font></a> | <a href="/jinfanyi/">近义反义词</a> | <a href="/duilian/">对联大全</a> | <a href="/jiapu/"><font  color="#0000FF">家谱族谱查询</font></a> | <a href="http://www.guoxuemi.com/hafo/" target="_blank" ><font color="#FF0000">哈佛古籍</font></a> 
</div>

 <!-- 头部导航开始 -->
<div class="w1180 head clearfix">
  <div class="head_logo l"><a title="国学大师官网" href="http://www.guoxuedashi.com" target="_blank"></a></div>
  <div class="head_sr l">
  <div id="head1">
  
  <a href="http://www.guoxuedashi.com/zidian/bujian/" target="_blank" ><img src="http://www.guoxuedashi.com/img/top1.gif" width="88" height="60" border="0" title="部件查字,支持20万汉字"></a>


<a href="http://www.guoxuedashi.com/help/yingpan.php" target="_blank"><img src="http://www.guoxuedashi.com/img/top230.gif" width="600" height="62" border="0" ></a>


  </div>
  <div id="head3"><a href="javascript:" onClick="javascript:window.external.AddFavorite(window.location.href,document.title);">添加收藏</a>
  <br><a href="/help/setie.php">搜索引擎</a>
  <br><a href="/help/zanzhu.php">赞助本站</a></div>
  <div id="head2">
 <a href="http://www.guoxuemi.com/" target="_blank"><img src="http://www.guoxuedashi.com/img/guoxuemi.gif" width="95" height="62" border="0" style="margin-left:2px;" title="国学迷"></a>
  

  </div>
</div>
  <div class="clear"></div>
  <div class="head_nav">
  <p><a href="/">首页</a> | <a href="/ShuKu/">国学书库</a> | <a href="/guji/">影印古籍</a> | <a href="/shici/">诗词宝典</a> | <a   href="/SiKuQuanShu/gxjx.php">精选</a> <b>|</b> <a href="/zidian/">汉语字典</a> | <a href="/hydcd/">汉语词典</a> | <a href="http://www.guoxuedashi.com/zidian/bujian/"><font  color="#CC0066">部件查字</font></a> | <a href="http://www.sfds.cn/"><font  color="#CC0066">书法大师</font></a> | <a href="/jgwhj/">甲骨文</a> <b>|</b> <a href="/b/4/"><font  color="#CC0066">解密</font></a> | <a href="/renwu/">历史人物</a> | <a href="/diangu/">历史典故</a> | <a href="/xingshi/">姓氏</a> | <a href="/minzu/">民族</a> <b>|</b> <a href="/mz/"><font  color="#CC0066">世界名著</font></a> | <a href="/download/">软件下载</a>
</p>
<p><a href="/b/"><font  color="#CC0066">历史</font></a> | <a href="http://skqs.guoxuedashi.com/" target="_blank">四库全书</a> |  <a href="http://www.guoxuedashi.com/search/" target="_blank"><font  color="#CC0066">全文检索</font></a> | <a href="http://www.guoxuedashi.com/shumu/">古籍书目</a> | <a   href="/24shi/">正史</a> <b>|</b> <a href="/chengyu/">成语词典</a> | <a href="/kangxi/" title="康熙字典">康熙字典</a> | <a href="/ShuoWenJieZi/">说文解字</a> | <a href="/zixing/yanbian/">字形演变</a> | <a href="/yzjwjc/">金 文</a> <b>|</b>  <a href="/shijian/nian-hao/">年号</a> | <a href="/diming/">历史地名</a> | <a href="/shijian/">历史事件</a> | <a href="/guanzhi/">官职</a> | <a href="/lishi/">知识</a> <b>|</b> <a href="/zhongyi/">中医中药</a> | <a href="http://www.guoxuedashi.com/forum/">留言反馈</a>
</p>
  </div>
</div>
<!-- 头部导航END --> 
<!-- 内容区开始 --> 
<div class="w1180 clearfix">
  <div class="info l">
   
<div class="clearfix" style="background:#f5faff;">
<script src='http://www.guoxuedashi.com/img/headersou.js'></script>

</div>
  <div class="info_tree"><a href="http://www.guoxuedashi.com">首页</a> > <a href="/SiKuQuanShu/fanti/">四库全书</a>
 > <h1>资治通鉴</h1> <!--         下载:【右键另存为】即可 --></div>
  <div class="info_content zj clearfix">
  
<div class="info_txt clearfix" id="show">
<center style="font-size:24px;">35-資治通鑑卷三十四</center>
    資治通鑑卷三十四   宋 司馬光 撰<br />
<br />
  胡三省 音註<br />
<br />
  漢紀二十六【起柔兆執徐盡著雍敦牂凡三年】<br />
<br />
  孝哀皇帝中<br />
<br />
  建平二年春正月有星孛于牽牛【晋天文志牽牛六星天之關梁主犧牲事孛蒲内翻】 丁傅宗族驕奢皆嫉傅喜之恭儉又傅太后欲求稱尊號與成帝母齊尊喜與孔光師丹共執以為不可上重違大臣正議【師古曰重難也】又内迫傅太后依違者連歲【如淳曰依違不决事之言也余謂上二語即依違之意】傅太后大怒上不得已先免師丹以感動喜【師丹免見上卷上年】喜終不順朱博與孔鄉侯傅晏連結共謀成尊號事數燕見【數所角翻見賢遍翻】奏封事毀短喜及孔光【毀短者譛毀而言其短也】丁丑上遂策免喜以侯就第御史大夫官既罷【成帝綏和元年罷御史大夫置大司空事見三十二卷】議者多以為古今異制漢自天子之號下至佐史皆不同於古【漢官至斗食佐史而止言漢承秦號為皇帝下至百官稱號皆不與古同】而獨改三公職事難分明無益於治亂【治直吏翻】於是朱博奏言故事選郡國守相高第為中二千石【守式又翻相息亮翻】選中二千石為御史大夫任職者為丞相【言御史大夫能任其職則進而為相】位次有序所以尊聖德重國相也今中二千石未更御史大夫而為丞相【師古曰更經也音工衡反】權輕非所以重國政也臣愚以為大司空官可罷復置御史大夫遵奉舊制臣願盡力以御史大夫為百僚率上從之夏四月戊午更拜博為御史大夫又以丁太后兄陽安侯明為大司馬衛將軍置官属大司馬冠號如故事【復綏和以前之制也冠古玩翻】 傅太后又自詔丞相御史大夫曰高武侯喜附下罔上與故大司空丹同心背畔放命圯族【應劭曰謂放棄教令圯其族類背蒲妹反圯皮美反】不宜奉朝請【朝直遥反請才性反又如字】其遣就國 丞相孔光自先帝時議繼嗣有持異之隙又重忤傅太后指【持異事見三十二卷成帝綏和元年重忤傅太后指謂不使居北宮奏傅遷持稱尊號之議也師古曰重音直用翻忤五故翻】 由是傅氏在位者與朱博為表裏共毁譛光【表外也裏内也傅氏譛之于内朱博毁之於外也】乙亥策免光為庶人【師古曰漢舊儀云丞相有他過使者奉策書即時步出府乘棧車歸田里】以御史大夫朱博為丞相封陽鄉侯【恩澤侯表陽鄉侯國于山陽湖陵 考異曰公卿表四月乙未孔光免朱博為丞相又曰四月戊午博為御史大夫乙亥遷五行志五月乙亥朔博為丞相荀紀乙亥孔光免按長歷是月丁巳朔無乙未十九日乙亥非朔也表志皆有誤】少府趙玄為御史大夫【成帝綏和元年趙玄自太子太傅左遷今復進用皆丁傅之意也】臨延登受策【師古曰延入而登殿也漢書儀云丞相御史大夫初拜皇帝延登親詔也】有大聲如鐘鳴殿中郎吏陛者皆聞焉【師古曰陛者謂執兵列於陛側】上以問黄門侍郎蜀郡揚雄【續漢志給事黄門侍郎六百石掌侍從左右給事中關通中外及諸王朝見於殿上引王就坐揚雄解嘲所謂官不過侍郎擢纔給事黄門者也揚雄自謂其先出自有周伯僑者食采於晋之揚因氏焉不知伯僑周何别也】及李㝷尋對曰此洪範所謂鼓妖者也師灋以為人君不聰為衆所惑空名得進則有聲無形不知所從生【洪範五行傳曰聽之不聰是謂不謀時則有鼓妖君嚴猛而閉下臣戰栗而塞耳則妄聞之氣發於音聲故有鼓妖妖於驕翻】其傳曰歲月日之中則正卿受之今以四月日加辰巳有異是為中焉【以一歲三分之則四月巳為歲之中以一日三分之則辰巳巳為日之中】正卿謂執政大臣也宜退丞相御史以應天變然雖不退不出期年其人自蒙其咎【師古曰期年十二月也蒙猶被也期音基】揚雄亦以為鼓妖聽失之象也朱博為人彊毅多權謀宜將不宜相【將即亮翻相息亮翻】恐有凶惡亟疾之怒【師古曰亟急也音居力翻】上不聽朱博既為丞相上遂用其議下詔曰定陶共皇之號不宜復稱定陶【復扶又翻】尊共皇太后曰帝太太后稱永信宮共皇后曰帝太后稱中安宮為共皇立寢廟於京師【為于偽翻】比宣帝父悼皇考制度【宣帝既立八年有司言禮父為士子為天子祭以天子悼園宜稱皇考立廟因園為寢以時薦享焉然悼園在廣明之鄉長安東郭之外也定陶其王葬定陶而立廟京師則非因園為寢矣】於是四太后各置少府太僕秩皆中二千石【傅太后丁太后趙太后與太皇太后為四太后】傅太后既尊後尤驕與太皇太后語至謂之嫗【嫗威遇翻】時丁傅以一二年間暴興尤盛為公卿列侯者甚衆然帝不甚假以權埶不如王氏在成帝世也 丞相博御史大夫玄奏言前高昌侯宏首建尊號之議而為關内侯師丹所劾奏免為庶人【事見上卷綏和二年劾戶槩翻】時天子衰麤委政於丹【師古曰言新有成帝之喪斬衰麤服故天子不親政事也衰音倉回翻】丹不深惟褒廣尊號之義【惟思也】而妄稱說抑貶尊號虧損孝道不忠莫大焉陛下仁聖昭然定尊號宏以忠孝復封高昌侯丹惡逆暴著雖蒙赦令不宜有爵邑請免為庶人奏可又奏新都侯莽前為大司馬不廣尊尊之義抑貶尊號虧損孝道【事亦見上卷綏和二年】當伏顯戮幸蒙赦令不宜有爵土請免為庶人上曰以莽與皇太后有属勿免遣就國及平阿侯仁臧匿趙昭儀親屬皆遣就國【仁譚之子也臧古藏字通】天下多寃王氏者【為下元壽二年王莽復柄國張本】諫大夫楊宣上封事言孝成皇帝深惟宗廟之重稱述陛下至德以承天序【天序謂帝王正統相傳之次天所命也上時掌翻】聖策深遠恩德至厚惟念先帝之意豈不欲以陛下自代奉承東宮哉【師古曰言供養太后】太皇太后春秋七十數更憂傷【謂先罹元帝之喪而後又哭成帝也數所角翻更工衡翻】敕令親屬引領以避丁傅【師古曰引領自引首領而退也】行道之人為之隕涕【為于偽翻】况於陛下登高遠望獨不慙於延陵乎【言王氏斥逐而丁傅貴寵若登高而望成帝陵寢寧不有慙於付託乎】帝深感其言復封成都侯商中子邑為成都侯【綏和二年商子况以罪奪侯今以邑紹封中讀曰仲】 朱博又奏言漢家故事置部刺史秩卑而賞厚【漢刺史秩六百石耳居部九歲舉為守相秩二千石其有異材功效著者輒登擢】咸勸功樂進【師古曰勸功自勸勉而立功也樂音洛】前罷刺史更置州牧【事見三十二卷成帝綏和元年更工衡翻】秩眞二千石位次九卿九卿缺以高第補其中材則苟自守而已恐功效陵夷【師古曰陵夷謂漸廢替】姦軌不禁臣請罷州牧置刺史如故上從之 六月庚申帝太后丁氏崩詔歸葬定陶共皇之園【從夫也共皇葬於其國賢曰在今曹州濟隂縣北共讀曰恭】發陳留濟隂近郡國五萬人穿復土【近郡國謂郡國之近定陶者讀書音義曰穿復土謂穿壙填塞事也言下棺訖復以土為墳故曰復上近其靳翻】 初成帝時齊人甘忠可詐造天官歷包元太平經十二卷言漢家逢天地之大終當更受命於天以教渤海夏賀良等【夏戶雅翻】中壘校尉劉向奏忠可假鬼神罔上惑衆【忠可詐稱天帝使眞人赤精子下教我故向奏之】下獄治服【服其挟詐也下遐稼翻】未斷病死【斷丁亂翻】賀良等復私以相教【復扶又翻下同】上即位司隸校尉解光騎都尉李尋白賀良等皆待詔黄門【應劭曰諸以材技徵召未有正官故曰待詔董巴曰黄門禁門黄闥】數召見【數所角翻見賢遍翻】陳說漢歷中衰當更受命成帝不應天命故絶嗣今陛下久疾變異屢數【師古曰數音所角翻】天所以譴告人也宜急改元易號乃得延年益夀皇子生災異息矣得道不得行【師古曰言知道而不能行】咎殃且無不有洪水將出災火且起滌盪民人上久寢疾【班固曰上即位痿痺末年寖劇】冀其有益遂從賀良等議詔大赦天下以建平二年為太初元年號曰陳聖劉太平皇帝【李斐曰陳道也言得神道聖者劉也如淳曰陳舜後王莽陳之後謬語陳當立而不知韋昭曰敷陳聖劉之德也師古曰如韋二說是也余謂韋說不詭於正如說則流于巫顔以為二說皆是將安從乎】漏刻以百二十為度【師古曰舊漏晝夜共百刻今增其二十】 秋七月以渭城西北原上永陵亭部為初陵勿徙郡國民 上既改號月餘寢疾自若夏賀良等復欲妄變政事大臣爭以為不可許賀良等奏言大臣皆不知天命宜退丞相御史以解光李尋輔政上以其言無驗八月詔曰待詔賀良等建言改元易號增益漏刻可以永安國家朕信道不篤過聽其言【師古曰過誤也】冀為百姓護福卒無嘉應【為于偽翻卒子恤翻】夫過而不改是謂過矣【論語載孔子之言】六月甲子詔書非赦令皆蠲除之【如淳曰悔前赦令不蒙其福故赦令還之臣瓚曰改元易號大赦天下以求延祚而不蒙福哀帝悔之故更下制書諸非赦事皆除之謂改制易號今皆復故也師古曰如說非也瓚說是矣唯赦令不改餘皆除之】賀良等反道惑衆奸態當窮竟皆下獄伏誅【下遐稼翻】尋及解光減死一等徙燉煌郡【此漢法所謂减死徙邉也减死者罪至死而特為末減也减死罪一等為城旦舂】 上以寢疾盡復前世所嘗興諸神祠凡七百餘所【成帝建始初匡衡張譚奏罷諸神祠不應禮者今盡復之】一歲三萬七千祠云【神祠既多而有歲五祠者有歲四祠者故其數若是之多】 傅太后怨傅喜不已使孔鄉侯風丞相朱博令奏免喜侯【師古曰風讀曰諷】博與御史大夫趙玄議之玄言事已前决【謂前已决遣就國罪無重科也】得無不宜【師古曰得無猶言無乃也】博曰已許孔鄉侯矣匹夫相要尚相得死【要一遥翻得死謂得其死力一曰得其相為死也】何况至尊【至尊謂傅太后】博唯有死耳【大臣以道事君而博以死奉私屬貪權藉勢之心為之也】玄即許可博惡獨斥奏喜【惡烏故翻】以故大司空汜鄉侯何武前亦坐過免就國【事見上卷綏和二年】事與喜相似即并奏喜武前在位皆無益於治【治直吏翻】雖已退免爵土之封非所當也皆請免為庶人上知傅太后素嘗怨喜疑博玄承指即召玄詣尚書問狀玄辭服【丞相御史同奏而獨召問玄者以博強毅多權詐難遽得其情而玄易以窮詰也】有詔左將軍彭宣與中朝者雜問宣等奏劾博玄晏皆不道不敬【劾戶槩切】請召詣廷尉詔獄上减玄死罪三等削晏戶四分之一【减死罪三等為隸臣妾晏封五千戶削其千二百五十】假謁者節召丞相詣廷尉博自殺國除 九月以光禄勲平當為御史大夫冬十月甲寅遷為丞相以冬月故且賜爵關内侯【如淳曰漢儀注御史大夫為丞相更春乃封故先賜爵關内侯也李奇曰以冬月非封侯時故且先賜爵關内侯也師古曰李說是也】以京兆尹平陵王喜為御史大夫【按表傳喜當作嘉詳見下年及審是】 上欲令丁傅處爪牙官【處昌呂翻】是歲策免左將軍淮陽彭宣以關内侯歸家而以光禄勲丁望代為左將軍【上策宣曰前有司數奏言諸侯國人不得宿衛將軍不宜典兵馬處大位朕惟將軍任漢將之重而子又前娶淮陽王女婚姻不絶非國之制其上左將軍印綬余按彭宣以連姻藩國而免官丁傅以戚黨而見用卒之奪劉氏者非藩國乃外戚也丁傅於國有大故之時拱手授柄于王氏而彭宣乃能辭三公位於王莽專權之初任官惟賢材烏得拘小嫌乎】 烏孫畀爰疐侵盜匈奴西界單于遣兵擊之殺數百人畧千餘人牛畜去卑爰疐恐遣子趨逯為質匈奴【疐竹二翻師古曰敺與驅同逯音録質音致下同】單于受以狀聞漢遣使者責讓單于告令還歸卑爰疐質子【責以匈奴烏孫並為漢臣單于不當擅受卑爰疐質子】單于受詔遣歸<br />
<br />
  三年春正月立廣德夷王弟廣漢為廣平王【廣德夷王雲客成帝鴻嘉二年封又二年薨無後今立廣漢以奉中山靖王嗣謚法安心好静曰夷克殺秉政曰夷】 帝太太后所居桂宮正殿火 【考異曰五行志云桂宮鴻寜殿災荀紀云桂宮正殿火今從哀紀】 上使使者召丞相平當欲封之當病篤不應【不應召也】室家或謂當不可強起受侯印為子孫邪【室家當之妻子也謂受侯印而死得以封爵遺子孫也強其兩翻為于偽翻下仝】當曰吾居大位已負素餐責矣起受侯印還卧而死死有餘罪今不起者所以為子孫也遂上書乞骸骨上不許三月己酉當薨 有星孛于河鼓【天文志河鼓在牽牛北大星上將左右星左右將孛蒲内翻】 夏四月丁酉王嘉為丞相河南太守王崇為御史大夫【守式又翻】崇京兆尹駿之子也嘉以時政苛急郡國守相數有變動【數所角翻】乃上疏曰臣聞聖王之功在於得人孔子曰才難不其然與【師古曰論語載孔子之言也才難言有賢材者難得也與讀曰歟余謂才難二語古語也孔子引之謂其言之是也】故繼世立諸侯象賢也【禮記郊特牲之文師古曰象其先父祖之賢耳非必皆賢也】雖不能盡賢天子為擇臣立命卿以輔之【記王制大國三卿皆命於天子次國三卿二卿命於天子一卿命於其君小國二卿皆命於其君春秋之時如晋之六卿以中軍帥為正卿亦其君先命之而後聞于天子耳齊之高國魯之三桓皆世卿也漢之王國傅相中尉命於天子猶古之命卿也】居是國也累世尊重然後士民之衆附焉是以教化行而治功立【治直吏翻】今之郡守重於古諸侯【周初班爵五等公侯地方百里伯七十里子男五十里其後齊晋秦楚以兼并而地始廣大耳漢郡守方制千里連城以十數是重于古諸侯也守式又翻下同】往者致選賢材【致極也】賢材難得拔擢可用者或起於囚徒昔魏尚坐事繫文帝感馮唐之言遣使持節赦其罪拜為雲中太守匈奴忌之【事見十四卷文帝十四年】武帝擢韓安國於徒中拜為梁内史骨肉以安【按韓安國傳安國坐法抵罪會梁内史缺漢使使者拜安國為梁内史起徒中為二千石此景帝時事也武帝當作景帝師古曰骨肉以安言梁孝王免罪也】張敞為京兆尹有罪當免黠吏知而犯敞【黠下八翻】敞收殺之其家自寃【自言其寃也】使者覆獄劾敞賊殺人上逮捕不下【上奏請逮捕敞而天子不下其奏也上時掌翻下遐嫁翻】會免亡命十數日宣帝徵敞拜為冀州刺史卒獲其用【事見二十七卷宣帝甘露元年卒子恤翻】前世非私此三人貪其材器有益於公家也孝文時吏居官者或長子孫【長知兩翻下同】倉氏庫氏則倉庫吏之後也其二千石長吏亦安官樂職【樂音洛】然後上下相望莫有苟且之意其後稍稍變易公卿以下傳相促急又數改更政事【傳知戀翻數所角翻更工衡翻】司隸部刺史舉劾苛細發揚隂私【司隸部三輔三河弘農其餘部刺史分部諸郡國劾戶槩翻】吏或居官數月而退送故迎新交錯道路中材苟容求全【師古曰不敢操持群下也】下材懷危内顧【師古曰常恐獲罪每為私計也】壹切營私者多二千石益輕賤吏民慢易之或持其微過增加成罪言於司隸刺史或上書告之衆庶知其易危【師古曰言易可傾危易以鼓翻】小失意則有離畔之心前山陽亡徒蘇令等縱横【事見三十一卷成帝永始三年師古曰横音胡孟翻】吏士臨難【難乃旦翻】莫肯伏節死義以守相威權素奪也【師古曰守郡守也相諸侯相也素奪謂不先假之威權也】孝成皇帝悔之下詔書二千石不為故縱【孟康曰不以故縱為罪所以優之也】遣使者賜金尉厚其意誠以為國家有急取辦于二千石二千石尊重難危乃能使下孝宣皇帝愛其善治民之吏有章劾事留中會赦壹解【師古曰不即下治其事恐為擾重故每留中或經赦令壹切皆解散也余謂善治民之吏宣帝愛其材或有章劾留中不下會赦則其事得釋治直之翻劾戶槩翻】故事尚書希下章為煩擾百姓證驗繫治或死獄中章文必有敢告之字乃下【師古曰所以丁寜告者之辭絶其相誣也余謂此乃防其誣告耳下遐稼翻為于偽翻】唯陛下留神於擇賢記善忘過容忍臣子勿責以備【師古曰不求備於一人也余謂責備者求全也】二千石部刺史三輔縣令有材任職者人情不能不有過差宜可濶略【師古曰當寛恕其小罪也】令盡力者有所勸此方今急務國家之利也前蘇令發欲遣大夫使逐問狀【使之逐盜而問其狀也】時見大夫無可使者【師古曰謂見在大夫皆不堪為使見賢遍翻】召盩厔令尹逢拜為諫大夫遣之【盩厔音舟窒】今諸大夫有材能者甚少宜豫畜養可成就者則士赴難不愛其死臨事倉卒乃求非所以明朝廷也【人材當聚於朝廷事會之來無可用者倉猝求之適所以明朝廷之無人耳少詩沼翻畜許六翻難乃旦翻卒讀曰猝】嘉因薦儒者公孫光滿昌【風俗通荆蠻有暪氏音舛變為滿國語路潞泉余滿皆赤狄隗姓】及能吏蕭咸薛修皆故二千石有名稱者天子納而用之【按嘉此疏誠中當時之病然為相者在於朝夕納誨隨事矯正天下不能窺其際而自臻於治平不在著見於奏疏以騰口說也自宣帝之後為相者始加詳於奏疏而考其治迹愈不逮前相業固不在乎此也稱曲證翻】 六月立魯頃王子部鄉侯閔為王【魯共王曾孫頃王封傳國於其子文王睃睃薨無後今立閔紹封部鄉據紀表及傳當作郚鄉師古曰郚音吾又音魚睃音子緣翻地理志東海郡有郚鄉侯國】 上以寢疾未定【定猶安也】冬十一月壬子令太皇太后下詔復甘泉秦畤汾隂后土祠罷南北郊【成帝崩皇太后詔罷甘泉汾隂祠復南北郊畤音止】上亦不能親至甘泉河東遣有司行事而禮祠焉 無鹽危山土自起覆草如馳道狀【無鹽縣屬東平國危山山名言土自起覆草成路如人力開掘作馳道狀也】又瓠山石轉立【晋灼曰漢書作報山山脅石一枚轉側起立高九尺六寸旁行一丈廣四尺也師古曰報山山名也古作瓠字為其形似瓠耳晋說是也】東平王雲及后謁自之石所祭治石象瓠山立石束倍草并祠之【雲元帝子東平王宇之子也謁后名也蘇林曰於宫中作山象師古曰倍草黄倍草也倍音步賄翻原父曰立石屬上句治直之翻】河内息夫躬【息夫複姓姓譜媯姓之國為息氏公子邉受爵為大夫又有息夫氏出焉】長安孫寵相與謀共告之曰此取封侯之計也乃與中郎右師譚【張晏曰右師姓譚名余謂右師以官為氏】共因中常侍宋弘上變事告焉【上時掌翻】是時上被疾多所惡事下有司逮王后謁下獄驗治服祠祭詛祝上為雲求為天子【被皮義翻下遐稼翻詛莊助翻祝職救翻為雲于偽翻】以為石立宣帝起之表也【事見二十三卷昭帝元鳳三年】有司請誅王有詔廢徙房陵雲自殺謁及舅伍宏及成帝舅安成共侯夫人放皆棄市【安成共侯王崇時已死矣故稱帝舅及諡以别下御史大夫王崇也伍宏以醫伎得幸出入禁門盖放薦之故并得禍共音恭】事連御史大夫王崇左遷大司農擢寵為南陽太守譚潁川都尉弘躬皆光禄大夫左曹給事中<br />
<br />
  四年春正月大旱 關東民無故驚走持槀或掫一枚【如淳曰掫麻幹也師古曰藁禾稈也音工老翻掫音鄒又音側九翻】轉相付與曰行西王母籌【師古曰西王母元后夀考之象行籌又言執國家籌策行於天下】道中相過逢多至千數或被髮徒跣或夜入關或踰牆入或乘車騎犇馳以置驛傳行【被皮義翻折而設翻傳知戀翻】經郡國二十六至京師不可禁止民又聚會里巷阡陌設博具【師古曰博戲之具】歌舞祠西王母至秋乃止【五行志曰此異乃王太后莽之應也】上欲封傅太后從父弟侍中光禄大夫商【從才用翻】尚書僕射平陵鄭崇諫曰孝成皇帝封親舅五侯天為赤黄晝昏日中有黑氣【事見三十卷成帝建始元年為于偽翻】孔鄉侯皇后父高武侯以三公封尚有因緣【孔鄉侯傅晏高武侯傅喜言皇后父及三公封侯尚有漢家舊比可因緣也】今無故復欲封商壞亂制度【復扶又翻壞音怪】逆天人之心非傅氏之福也臣願以身命當國咎崇因持詔書案起【李奇曰持當受詔書案起也師古曰李說非也案者即寫詔之文余按更始時常侍奏事韓夫人起抵破書案則案非文案之案也李說是】傅太后大怒曰何有為天子乃反為一臣所顓制邪二月癸卯上遂下詔封商為汝昌侯【恩澤侯表汝昌侯國於東郡須昌之陽穀 考異曰哀紀及恩澤侯表皆云商以今年二月封而孫寶傳云制詔丞相大司空按建平二年已罷大司空闕   官疑傳誤】 駙馬都尉侍中雲陽董賢得幸於上出則參乘入御左右【乘繩證翻御侍也】賞賜累鉅萬貴震朝廷常與上卧起嘗晝寢偏藉上袖【師古曰藉謂身卧其上也】上欲起賢未覺【師古曰覺寐之寤也覺音工劾翻】不欲動賢乃斷袖而起【斷丁管翻】又詔賢妻得通引籍殿中止賢廬【師古曰廬謂殿中所止宿處】又召賢女弟以為昭儀位次皇后昭儀及賢與妻旦夕上下並侍左右【上時掌翻】以賢父恭為少府賜爵關内侯詔將作大匠為賢起大第北闕下重殿洞門【師古曰重殿謂有前後殿洞門謂闕門相當也皆僭天子之制度者也為于偽翻重直龍翻】土木之功窮極技巧【技渠綺翻】賜武庫禁兵上方珍寶【禁中謂之上方】其選物上弟盡在董氏【選物物之選其尤者上第於衆物之中等第居上也弟與第同】而乘輿所服乃其副也【乘繩證翻】及至東園秘器珠襦玉匣【師古曰東園署名屬少府漢舊儀云東園秘器作棺梓素木長三丈崇廣四尺珠襦以珠為襦如鎧狀連縫之以黄金為縷腰以下玉為柙長尺廣三寸半為甲至足亦縫以黄金縷】豫以賜賢無不備具又令將作為賢起冢塋義陵旁【義陵帝夀陵也塋余傾翻墓域】内為便房剛柏題凑【服䖍曰便房藏中便坐也蘇林曰以柏木黄心致累棺外曰黄腸木頭皆内向故曰題凑師古曰便房小曲室也】外為徼道周垣數里【徼道徼循之道師古曰徼謂遮繞也音工釣翻垣牆也】門闕罘罳甚盛【罘音浮愚音思】鄭崇以賢貴寵過度諫上由是重得罪【師古曰重音直用翻】數以職事見責【數所角翻】發疾頸癰欲乞骸骨不敢尚書令趙昌佞讇【讇古諂字】素害崇知見疏因奏崇與宗族通疑有姦請治【治直之翻下同】上責崇曰君門如市人【師古曰言請求者多交通賓客】何以欲禁切主上崇對曰臣門如市臣心如水【師古曰言至清也】願得考覆上怒下崇獄【下遐稼翻】司隸孫寶上書曰【成帝元延四年省司隸校尉綏和二年上復置但曰司隸冠進賢冠屬大司空】按尚書令昌奏僕射崇獄覆榜掠將死卒無一辭【師古曰榜掠謂笞擊而考問之也榜音彭掠音亮卒音子恤翻】道路稱寃疑昌與崇内有纎介【師古曰言有細故宿嫌也】浸潤相陷自禁門樞機近臣蒙受寃譛虧損國家為謗不小臣請治昌以解衆心書奏上下詔曰司隸寶附下罔上以春月作詆欺遂其姦心盖國之賊也免寶為庶人崇竟死獄中 三月諸吏散騎光禄勲賈延為御史大夫【延為光禄勲而加諸吏散騎也百官表諸吏得舉法散騎騎旁乘輿車師古曰騎而散從無常職也散悉亶翻】 上欲侯董賢而未有緣侍中傅嘉勸上定息夫躬孫寵告東平本章去宋弘更言因董賢以聞【更定告章刋去宋弘名而入董賢名師古曰定謂改治其章也去羌呂翻更工衡翻】欲以其功侯之皆先賜爵關内侯頃之上欲封賢等而心憚王嘉乃先使孔鄉侯晏持詔書示丞相御史於是嘉與御史大夫賈延上封事言竊見董賢等三人始賜爵衆庶匈匈咸曰賢貴其餘并蒙恩【師古曰言董賢以貴寵故妄得封而躬寵等遂蒙恩】至今流言未解陛下仁恩於賢等不已宜暴賢等本奏語言【師古曰暴謂章露也】延問公卿大夫博士議郎考合古今明正其義然後乃加爵土不然恐大失衆心海内引領而議【引領猶言引頸也項背曰領】暴評其事必有言當封者在陛下所從天下雖不說【師古曰說讀曰悦】咎有所分不獨在陛下前定陵侯淳于長初封其事亦議【事見三十一卷成帝永始二年】大司農谷永以長當封衆人歸咎於永先帝不獨蒙其譏臣嘉臣延材駑不稱【稱尺證翻】死有餘責知順指不迕【師古曰迕逆也音五故翻】可得容身須臾所以不敢者思報厚恩也上不得已且為之止【為于偽翻下同】 夏六月尊帝太太后為皇太太后【傅太后也】秋八月辛卯上下詔切責公卿曰昔楚有子玉得臣<br />
<br />
  晉文公為之側席而坐【晉文公與楚戰勝于城濮文公猶有憂色曰得臣猶在憂未歇也記曰有憂者側席而坐】近事汲黯折淮南之謀【事見十九卷武帝元狩元年】今東平王雲等至有圖弑天子逆亂之謀者是公卿股肱莫能悉心務聰明以銷厭未萌故也【師古曰悉盡也務聰明者廣視聽也厭音一涉翻】賴宗廟之靈侍中駙馬都尉賢等發覺以聞咸伏厥辜書不云乎用德章厥善【師古曰尚書盤庚之辭也】其封賢為高安侯【恩澤侯表高安侯國於朱扶而朱扶之地無所考】南陽太守寵為方陽侯【恩澤侯表方陽侯國於沛郡龍亢】左曹光禄大夫躬為宜陵侯【恩澤侯表宜陵侯國於南陽村衍】賜右師譚爵關内侯又封傅太后同母弟鄭惲子業為陽信侯【恩澤侯表陽信侯國於南陽新野惲於粉翻】息夫躬既親近數進見言事【近其靳翻數所角翻見賢遍翻】議論無所避上疏歷詆公卿大臣衆畏其口見之仄目 上使中黄門【續漢志中黄門比百石掌給事禁中】發武庫兵前後十輩送董賢及上乳母王阿舍執金吾母將隆奏言武庫兵器天下公用國家武備繕治造作皆度大司農錢【毋將複姓治直之翻蘇林曰用度皆出大司農】大司農錢自乘輿不以給共養共養勞賜一出少府【乘繩證翻師古曰共音居用翻養音弋尚翻勞郎到翻】盖不以本臧給末用不以民力共浮費【臧古藏字通音徂浪翻師古曰共讀曰供下同】别公私示正路也【别彼列翻】古者諸侯方伯得顓征伐乃賜斧鉞【禮記曰諸侯賜斧鉞然後征王制八州八伯謂之方伯各統其州之國】漢家邉吏職任距寇亦賜武庫兵皆任事然後蒙之【任音壬】春秋之誼家不臧甲【春秋公羊傳載孔子墮三都之言臧與藏通讀從平聲】所以抑臣威損私力也今賢等便僻弄臣【便頻連翻】私恩微妾而以天下公用給其私門契國威器共其家備【李奇曰契缺也晉灼曰契取也師古曰李說是也音詰結翻】民力分於弄臣武兵設於微妾建立非宜以廣驕僭非所以示四方也孔子曰奚取於三家之堂【師古曰論語云三家者以雍徹孔子曰相維辟公天子穆穆奚取於三家之堂言以雍徹食乃天子之禮何為在三家之堂也三家謂魯叔孫仲孫季孫也余謂隆引孔子之言以謂武庫兵器不當以共婦妾之家猶歌雍不當在三家之堂也】臣請收還武庫上不說【說讀曰悦】頃之傅太后使謁者賤買執金吾官婢八人隆奏言買賤請更平直【漢書作賈賤賈讀曰價下同】上於是制詔丞相御史隆位九卿既無以匡朝廷之不逮而反奏請與永信宫争貴賤之賈【傅太后稱永信宫】傷化失俗以隆前有安國之言左遷為沛郡都尉初成帝末隆為諫大夫嘗奏封事言古者選諸侯入為公卿以褒功德【如衛武公鄭武公莊公是也】宜徵定陶王使在國邸以填萬方【師古曰填讀曰鎮音竹忍翻】故上思其言而宥之 諫大夫渤海鮑宣上書曰【姓譜鮑本自夏禹之裔因封為鮑氏齊之鮑氏世為上卿】竊見孝成皇帝時外親持權人人牽引所私以充塞朝廷【塞悉則翻】妨賢人路濁亂天下奢泰亡度【亡古無字通】窮困百姓是以日食且十彗星四起【日食十注已見三十二卷元延二年建始元年星孛于營室元延元年星孛于營室元延元年星孛于東井後又晨出東方十三日又夕見西方是四起也彗祥歲翻延芮翻又徐醉翻】危亡之徵陛下所親見也今奈何反覆劇於前乎【覆當作復劇增也甚也】今民有七亡【師古曰亡謂失其作業也】隂陽不和水旱為災一亡也縣官重責更賦租税二亡也【師古曰更謂為更卒也音工衡翻】貪吏並公受取不已三亡也【師古曰並依也音步浪翻】豪族大姓蠶食亡厭四亡也【亡厭上古無字通下音於鹽翻】苛吏繇役失農桑時五亡也【繇古徭字通】部落鼓鳴男女遮列六亡也【師古曰言閒桴鼔之聲以為有盗賊皆當遮列而追捕】盗賊刼略取民財物七亡也七亡尚可又有七死酷吏毆殺一死也【師古曰毆擊也音一口翻】治獄深刻二死也【治直之翻】寃陷亡辜<br />
<br />
  三死也【亡古無字通下同】盜賊横發四死也【師古曰横音尸孟翻】怨讐相殘五死也歲惡饑餓六死也時氣疾疫七死也【天有六氣隂陽風雨晦明也分為四時序為五節過則為災而生疾疫亦非時之氣所為也】民有七亡而無一得欲望國安誠難民有七死而無一生欲望刑措誠難此非公卿守相貪殘成化之所致邪羣臣幸得居尊官食重禄豈有肯加惻隱於細民助陛下流教化者邪【師古曰惻隱皆痛也】志但在營私家稱賓客為姦利而已【師古曰務稱賓客所求也稱尺證翻】以苟容曲從為賢以拱默尸禄為智【拱默拱手而默然不言也師古曰尸主也不憂其職但主食禄而已】謂如臣宣等為愚陛下擢臣巖穴誠冀有益豪毛豈徒使臣美食大官重高門之地哉【晉灼曰高門殿名也師古曰在未央宫中余謂宣言徒知養賢為朝廷之重而不計其有益於時與否】天下乃皇天之天下也陛下上為皇天子下為黎庶父母為天牧養元元【為天之為于偽翻】視之當如一合尸鳩之詩【師古曰尸鳩曹國風之篇也其詩曰尸鳩在桑其子七兮淑人君子其儀一兮言尸鳩養其子七平均如一善人君子布德施惠亦當然也毛氏曰尸鳩秸鞠也尸鳩之養其子朝從上下暮從下上平均如一秸音居八翻又音吉】今貧民菜食不厭衣又穿空【師古曰厭飽足也空孔也穿空言破敝也】父子夫婦不能相保誠可為酸鼻陛下不救將安所歸命乎奈何獨養外親與幸臣董賢多賞賜以大萬數使奴從賓客漿酒藿肉【劉德曰視酒如漿視肉如藿也師古曰藿豆菜也貧人茹之從才用翻】蒼頭廬兒皆用致富【孟康曰黎民黔首黔黎皆黑也下民隂類故以黑為號漢名奴為蒼頭非純黑以别於良人也諸給事殿中者所居為廬蒼頭侍從因呼為廬兒臣瓚曰漢儀注官奴給事計從侍中已下為蒼頭青幘】非天意也及汝昌侯傅商亡功而封【古亡無字通下同】夫官爵非陛下之官爵乃天下之官爵也陛下取非其官官非其人【師古曰此官不當加於此人此人不當受此官也】而望天說民服豈不難哉【說讀曰悦】方陽侯孫寵宜陵侯息夫躬辯足以移衆彊可用獨立姦人之雄惑世尤劇者也宜以時罷退及外親幼童未通經術者皆宜令休就師傅急徵故大司馬傅喜使領外親故大司空何武師丹故丞相孔光故左將軍彭宣經皆更博士【言經學有師法也更工衡翻】位皆歷三公龔勝為司直郡國皆慎選舉【司直掌佐丞相舉不法勝守正不阿郡國懼為所舉奏故皆慎於選舉】可大委任也陛下前以小不忍退武等【師古曰少有不快於心不能忍也】海内失望陛下尚能容亡功德者甚衆曾不能忍武等邪治天下者當用天下之心為心【治直之翻】不得自專快意而已也宣語雖刻切上以宣名儒優容之 匈奴單于上書願朝五年【朝直遥翻下同】時帝被疾【被皮義翻】或言匈奴從上游來厭人【服䖍曰游猶流也河水從西北來故曰上游也師古曰上游亦總謂地形耳不必係於河水也厭音一涉翻厭勝也】自黄龍竟寧時單于朝中國輒有大故【師古曰大故謂國之大喪】上由是難之以問公卿亦以為虛費府帑【師古曰府物所聚也帑藏金帛之所也帑音他莽翻又音奴】可且勿許單于使辭去未發【已辭而未行也使疏吏翻】黄門郎揚雄上書諫曰臣聞六經之治貴於未亂兵家之勝貴於未戰【書周官曰制治于未亂兵法曰戰不必勝不苟接刃師古曰已亂而後治之戰鬬而後獲勝則不足貴治直吏翻】二者皆微【師古曰微謂精妙也】然而大事之本不可不察也今單于上書求朝國家不許而辭之臣愚以為漢與匈奴從此隙矣【言嫌隙從此而開也】匈奴本五帝所不能臣三王所不能制其不可使隙明甚臣不敢遠稱請引秦以來明之以秦始皇之彊蒙恬之威然不敢窺西河乃築長城以界之【蒙恬斥逐匈奴以北河為竟漢朔方郡地是也若西河則漢武威張掖燉煌酒泉地是也秦不能取築長城起臨洮以界之】會漢初興以高祖之威靈三十萬衆困於平城【事見十一卷高帝七年】時奇譎之士石畫之臣甚衆【鄧展曰石大也師古曰石言堅固如石也畫計策也音獲】卒其所以脱者世莫得而言也【師古曰卒終也莫得而言謂自免之計其事醜惡故不傳卒子恤翻】又高后時匈奴悖慢大臣權書遺之然後得解【事見十二卷惠帝三年杜佑曰以權道為書順辭以答遺于季翻】及孝文時匈奴侵暴北邉候騎至雍甘泉京師大駭發三將軍屯棘門細柳覇上以備之數月乃罷【事見十五卷文帝後六年雍於用翻】孝武即位設馬邑之權欲誘匈奴徒費財勞師一虜不可得見况單于之面乎【事見十七卷武帝元光二年言欲見匈奴一人且不可得况使單于面來獻見乎】其後深惟社稷之計規恢萬載之策【載子亥翻】乃大興師數十萬使衛青霍去病操兵前後十餘年於是浮西河絶大幕破窴顔襲王庭窮極其地追犇逐北封狼居胥山禪於姑衍以臨瀚海虜名王貴人以百數自是之後匈奴震怖益求和親然而未肯稱臣也【事並見武帝紀操千高翻窴音填怖普布翻】且夫前世豈樂傾無量之費役無罪之人快心狼望之北哉【師古曰狼望匈奴中地名余謂邉人謂舉燧為狼烟狼望謂狼烟候望之地樂音洛】以為不壹勞者不久逸不暫費者不永寧是以忍百萬之師以摧餓虎之喙【師古曰喙口也摧百萬之師於獸口也喙許穢翻余謂順文而為說其義自通唐諱虎故師古改曰獸】運府庫之財填盧山之壑而不悔也【師古曰盧山匈奴中山也余按衛青薨起冢象盧山青唯絶幕至填顔山耳或者寘顔山即盧山歟孟康曰盧山單于南庭也】至本始之初匈奴有桀心【師古曰桀竪也言其起立不順】欲掠烏孫侵公主乃發五將之師十五萬騎以擊之時鮮有所獲徒奮揚威武明漢兵若雷風耳雖空行空反尚誅兩將軍【事見二十四卷宣帝本始三年鮮息踐翻兵若雷風言師速而疾風驅霆行一過而不留也】故北狄不服中國未得高枕安寢也【枕職任翻】逮至元康神爵之間大化神明鴻恩溥洽而匈奴内亂五單于争立日逐呼韓邪攜國歸死扶伏稱臣【事並見宣帝紀歸死者歸死命于漢也扶伏猶言匍匐也師古曰伏音蒲北翻】然尚羈縻之計不顓制【師古曰顓與專同專制謂以為臣妾也】自此之後欲朝者不距【朝直遥翻】不欲者不彊【師古曰彊音其兩翻】何者外國天性忿鷙形容魁健負力怙氣難化以善易肄以惡【師古曰鷙音竹二翻鷙狠也魁大也負恃也余謂肄習也言易習於為惡也】其彊難詘【詘與屈同】其和難得故未服之時勞師遠攻傾國殚貨伏尸流血破堅拔敵如彼之難也既服之後慰薦撫循交接賂遺威儀俯仰如此之備也往時嘗屠大宛之城【事見二十一卷武帝太初三年宛於元翻】蹈烏桓之壘【事見二十三卷昭帝元鳳三年】探姑繒之壁【事見二十三卷昭帝始元四年探吐南翻】藉蕩姐之場【劉德曰蕩姐羌屬師古曰藉猶蹈也姐音紫余據元帝永光三年隴西羌彡姐反豈是邪】艾朝鮮之旃【事見二十一卷武帝元封三年師古曰艾讀曰刈刈絶也朝音潮】拔兩越之旗【見二十卷武帝元鼎六年】近不過旬月之役遠不離二時之勞【師古曰離歷也三月為一時】固已犂其庭【師占曰犂耕也】掃其閭郡縣而置之雲徹席卷後無餘災【如雲之徹如席之卷天清地浄無纎毫之塵翳也】唯北狄為不然真中國之堅敵也三垂比之懸矣【師古曰懸絶也】前世重之兹甚【師古曰兹益也余謂兹此也兹甚此為甚也】未易可輕也【易以豉翻】今單于歸義懷欵誠之心欲離其庭陳見于前【離力智翻】此乃上世之遺策神靈之所想望國家雖費不得已者也奈何距以來厭之辭【謂或言從上游來厭人也】踈以無日之期【止其來朝辭以他日而無一定之期則匈奴與漢踈】消往昔之恩開將來之隙夫疑而隙之使有恨心負前言緣往辭歸怨於漢【師古曰言單于因緣往昔和好之辭以怨漢也余謂負恃也負前言者恃前者有和好之言也】因以自絶終無北面之心威之不可諭之不能焉得不為大憂乎【焉於䖍翻】夫明者視於無形聰者聽於無聲誠先於未然【先悉薦翻】即兵革不用而憂患不生不然壹有隙之後雖智者勞心於内辯者轂擊於外【師古曰轂擊言使車交馳其轂相擊也轂戶容翻】猶不若未然之時也且往者圖西域制車師置城郭都護三十六國【事並見武帝宣帝紀】豈為康居烏孫能踰白龍堆而寇西邉哉【為于偽翻下同】乃以制匈奴也夫百年勞之一日失之費十而愛一【謂向者不惮十分之費以制匈奴今來朝之費十分之一耳乃愛惜之】臣竊為國不安也【為于偽翻】唯陛下少留意於未亂未戰【少詩沼翻】以遏邉萌之禍【萌與氓同謂邉民也】書奏天子寤焉召還匈奴使者更報單于書而許之【更工衡翻改也】賜雄帠五十匹黄金十斤單于未發會病復遣使願朝明年【復扶又翻】上許之 董賢貴幸日盛丁傅害其寵孔鄉侯晏與息夫躬謀欲求居位輔政會單于以病未朝躬因是而上奏【上時掌翻】以為單于當以十一月入塞後以病為解【師古曰自解說云病】疑有他變烏孫兩昆彌弱卑爰疐彊盛東結單于遣子往侍【事見上建平二年疐竹二翻】恐其合埶以并烏孫烏孫并則匈奴盛而西域危矣可令降胡詐為卑爰疐使者來上書欲因天子威告單于歸臣侍子因下其章【降戶江翻下遐稼翻】令匈奴客聞焉則是所謂上兵伐謀【匈奴客謂匈奴使者也服䖍曰謀者舉兵伐解之也師古曰此說非也言知敵有謀者則以事而應之沮其所為不用兵革所以為貴上兵伐謀其次伐交孫武子之言】其次伐交者也【師古曰知敵有外交連結相援者則間而誤之令其解散也】書奏上引見躬【見賢遍翻下屢見之見同】召公卿將軍大議左將軍公孫禄以為中國常以威信懷伏夷狄躬欲逆詐【逆詐者敵之詐謀未見欲迎測其情也】造不信之謀不可許且匈奴賴先帝之德保塞稱藩今單于以疾病不任奉朝賀遣使自陳不失臣子之禮【任音壬】臣禄自保沒身不見匈奴為邉竟憂也【竟讀曰境】躬掎禄曰【師古曰掎從後引之也謂引躡其言也音居綺翻】臣為國家計【為于偽翻】冀先謀將然【師古曰謂彼欲有其事則為謀策以壞之】豫圖未形【師古曰圖謀也未有形兆而圖之】為萬世慮而禄欲以其犬馬齒保目所見臣與禄異議未可同日語也上曰善乃罷羣臣獨與躬議躬因建言災異屢見恐必有非常之變可遣大將軍行邉兵敕武備【師古曰行音下孟翻敕整也】斬一郡守以立威震四夷【守手又翻】因以厭應變異【師古曰厭音一涉翻】上然之以問丞相嘉對曰臣聞動民以行不以言【行下孟翻】應天以實不以文下民微細猶不可詐况於上天神明而可欺哉天之見異【師古曰見謂顯示也】所以敕戒人君欲令覺悟反正推誠行善民心說而天意得矣【說讀曰悦】辯士見一端或妄以意傅著星歷【師古曰傅讀曰附著音治畧翻】虛造匈奴西羌之難【難乃旦翻】謀動干戈設為權變非應天之道也守相有辠【相息亮翻】車馳詣闕交臂就死恐懼如此而談說者欲動安之危【師古曰之往也言揺動安全之計往就危殆也】辯口快耳其實未可從夫議政者苦其讇諛傾險辯惠深刻也【讇古諂字】昔秦繆公不從百里奚蹇叔之言以敗其師其悔過自責疾詿誤之臣思黄髮之言名垂於後世【秦穆公欲襲鄭蹇叔百里奚諫不聽遂出師晉襄公要而敗諸殽還歸作秦誓以悔過其辭曰惟古之謀人則曰未就予忌惟今之謀人姑將以為親雖則云然尚猷詢兹黄髪則罔所愆又曰惟截截善諞言俾君子易辭我皇多有之昩昧我思之敗補邁翻詿戶卦翻】願陛下觀覽古戒反覆參考無以先入之語為主【師古曰謂躬為此計先入於帝耳】上不聽【為董賢沮躬策躬遂得罪張本】<br />
<br />
  資治通鑑卷三十四<br />
<br />
<史部,編年類,資治通鑑>  <br>
   </div> 

<script src="/search/ajaxskft.js"> </script>
 <div class="clear"></div>
<br>
<br>
 <!-- a.d-->

 <!--
<div class="info_share">
</div> 
-->
 <!--info_share--></div>   <!-- end info_content-->
  </div> <!-- end l-->

<div class="r">   <!--r-->



<div class="sidebar"  style="margin-bottom:2px;">

 
<div class="sidebar_title">工具类大全</div>
<div class="sidebar_info">
<strong><a href="http://www.guoxuedashi.com/lsditu/" target="_blank">历史地图</a></strong>  
<a href="http://www.880114.com/" target="_blank">英语宝典</a>  
<a href="http://www.guoxuedashi.com/13jing/" target="_blank">十三经检索</a> 
<br><strong><a href="http://www.guoxuedashi.com/gjtsjc/" target="_blank">古今图书集成</a></strong> 
<a href="http://www.guoxuedashi.com/duilian/" target="_blank">对联大全</a> <strong><a href="http://www.guoxuedashi.com/xiangxingzi/" target="_blank">象形文字典</a></strong> 

<br><a href="http://www.guoxuedashi.com/zixing/yanbian/">字形演变</a>  <strong><a href="http://www.guoxuemi.com/hafo/" target="_blank">哈佛燕京中文善本特藏</a></strong>
<br><strong><a href="http://www.guoxuedashi.com/csfz/" target="_blank">丛书&方志检索器</a></strong> <a href="http://www.guoxuedashi.com/yqjyy/" target="_blank">一切经音义</a>  

<br><strong><a href="http://www.guoxuedashi.com/jiapu/" target="_blank">家谱族谱查询</a></strong>  <strong><a href="http://shufa.guoxuedashi.com/sfzitie/" target="_blank">书法字帖欣赏</a></strong> 
<br>

</div>
</div>


<div class="sidebar" style="margin-bottom:0px;">

<font style="font-size:22px;line-height:32px">QQ交流群9:489193090</font>


<div class="sidebar_title">手机APP 扫描或点击</div>
<div class="sidebar_info">
<table>
<tr>
	<td width=160><a href="http://m.guoxuedashi.com/app/" target="_blank"><img src="/img/gxds-sj.png" width="140"  border="0" alt="国学大师手机版"></a></td>
	<td>
<a href="http://www.guoxuedashi.com/download/" target="_blank">app软件下载专区</a><br>
<a href="http://www.guoxuedashi.com/download/gxds.php" target="_blank">《国学大师》下载</a><br>
<a href="http://www.guoxuedashi.com/download/kxzd.php" target="_blank">《汉字宝典》下载</a><br>
<a href="http://www.guoxuedashi.com/download/scqbd.php" target="_blank">《诗词曲宝典》下载</a><br>
<a href="http://www.guoxuedashi.com/SiKuQuanShu/skqs.php" target="_blank">《四库全书》下载</a><br>
</td>
</tr>
</table>

</div>
</div>


<div class="sidebar2">
<center>


</center>
</div>

<div class="sidebar"  style="margin-bottom:2px;">
<div class="sidebar_title">网站使用教程</div>
<div class="sidebar_info">
<a href="http://www.guoxuedashi.com/help/gjsearch.php" target="_blank">如何在国学大师网下载古籍?</a><br>
<a href="http://www.guoxuedashi.com/zidian/bujian/bjjc.php" target="_blank">如何使用部件查字法快速查字?</a><br>
<a href="http://www.guoxuedashi.com/search/sjc.php" target="_blank">如何在指定的书籍中全文检索?</a><br>
<a href="http://www.guoxuedashi.com/search/skjc.php" target="_blank">如何找到一句话在《四库全书》哪一页?</a><br>
</div>
</div>


<div class="sidebar">
<div class="sidebar_title">热门书籍</div>
<div class="sidebar_info">
<a href="/so.php?sokey=%E8%B5%84%E6%B2%BB%E9%80%9A%E9%89%B4&kt=1">资治通鉴</a> <a href="/24shi/"><strong>二十四史</strong></a>&nbsp; <a href="/a2694/">野史</a>&nbsp; <a href="/SiKuQuanShu/"><strong>四库全书</strong></a>&nbsp;<a href="http://www.guoxuedashi.com/SiKuQuanShu/fanti/">繁体</a>
<br><a href="/so.php?sokey=%E7%BA%A2%E6%A5%BC%E6%A2%A6&kt=1">红楼梦</a> <a href="/a/1858x/">三国演义</a> <a href="/a/1038k/">水浒传</a> <a href="/a/1046t/">西游记</a> <a href="/a/1914o/">封神演义</a>
<br>
<a href="http://www.guoxuedashi.com/so.php?sokeygx=%E4%B8%87%E6%9C%89%E6%96%87%E5%BA%93&submit=&kt=1">万有文库</a> <a href="/a/780t/">古文观止</a> <a href="/a/1024l/">文心雕龙</a> <a href="/a/1704n/">全唐诗</a> <a href="/a/1705h/">全宋词</a>
<br><a href="http://www.guoxuedashi.com/so.php?sokeygx=%E7%99%BE%E8%A1%B2%E6%9C%AC%E4%BA%8C%E5%8D%81%E5%9B%9B%E5%8F%B2&submit=&kt=1"><strong>百衲本二十四史</strong></a>  <a href="http://www.guoxuedashi.com/so.php?sokeygx=%E5%8F%A4%E4%BB%8A%E5%9B%BE%E4%B9%A6%E9%9B%86%E6%88%90&submit=&kt=1"><strong>古今图书集成</strong></a>
<br>

<a href="http://www.guoxuedashi.com/so.php?sokeygx=%E4%B8%9B%E4%B9%A6%E9%9B%86%E6%88%90&submit=&kt=1">丛书集成</a> 
<a href="http://www.guoxuedashi.com/so.php?sokeygx=%E5%9B%9B%E9%83%A8%E4%B8%9B%E5%88%8A&submit=&kt=1"><strong>四部丛刊</strong></a>  
<a href="http://www.guoxuedashi.com/so.php?sokeygx=%E8%AF%B4%E6%96%87%E8%A7%A3%E5%AD%97&submit=&kt=1">說文解字</a> <a href="http://www.guoxuedashi.com/so.php?sokeygx=%E5%85%A8%E4%B8%8A%E5%8F%A4&submit=&kt=1">三国六朝文</a>
<br><a href="http://www.guoxuedashi.com/so.php?sokeytm=%E6%97%A5%E6%9C%AC%E5%86%85%E9%98%81%E6%96%87%E5%BA%93&submit=&kt=1"><strong>日本内阁文库</strong></a> <a href="http://www.guoxuedashi.com/so.php?sokeytm=%E5%9B%BD%E5%9B%BE%E6%96%B9%E5%BF%97%E5%90%88%E9%9B%86&ka=100&submit=">国图方志合集</a> <a href="http://www.guoxuedashi.com/so.php?sokeytm=%E5%90%84%E5%9C%B0%E6%96%B9%E5%BF%97&submit=&kt=1"><strong>各地方志</strong></a>

</div>
</div>


<div class="sidebar2">
<center>

</center>
</div>
<div class="sidebar greenbar">
<div class="sidebar_title green">四库全书</div>
<div class="sidebar_info">

《四库全书》是中国古代最大的丛书,编撰于乾隆年间,由纪昀等360多位高官、学者编撰,3800多人抄写,费时十三年编成。丛书分经、史、子、集四部,故名四库。共有3500多种书,7.9万卷,3.6万册,约8亿字,基本上囊括了古代所有图书,故称“全书”。<a href="http://www.guoxuedashi.com/SiKuQuanShu/">详细>>
</a>

</div> 
</div>

</div>  <!--end r-->

</div>
<!-- 内容区END --> 

<!-- 页脚开始 -->
<div class="shh">

</div>

<div class="w1180" style="margin-top:8px;">
<center><script src="http://www.guoxuedashi.com/img/plus.php?id=3"></script></center>
</div>
<div class="w1180 foot">
<a href="/b/thanks.php">特别致谢</a> | <a href="javascript:window.external.AddFavorite(document.location.href,document.title);">收藏本站</a> | <a href="#">欢迎投稿</a> | <a href="http://www.guoxuedashi.com/forum/">意见建议</a> | <a href="http://www.guoxuemi.com/">国学迷</a> | <a href="http://www.shuowen.net/">说文网</a><script language="javascript" type="text/javascript" src="https://js.users.51.la/17753172.js"></script><br />
  Copyright &copy; 国学大师 古典图书集成 All Rights Reserved.<br>
  
  <span style="font-size:14px">免责声明:本站非营利性站点,以方便网友为主,仅供学习研究。<br>内容由热心网友提供和网上收集,不保留版权。若侵犯了您的权益,来信即刪。scp168@qq.com</span>
  <br />
ICP证:<a href="http://www.beian.miit.gov.cn/" target="_blank">鲁ICP备19060063号</a></div>
<!-- 页脚END --> 
<script src="http://www.guoxuedashi.com/img/plus.php?id=22"></script>
<script src="http://www.guoxuedashi.com/img/tongji.js"></script>

</body>
</html>
