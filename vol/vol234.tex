<!DOCTYPE html PUBLIC "-//W3C//DTD XHTML 1.0 Transitional//EN" "http://www.w3.org/TR/xhtml1/DTD/xhtml1-transitional.dtd">
<html xmlns="http://www.w3.org/1999/xhtml">
<head>
<meta http-equiv="Content-Type" content="text/html; charset=utf-8" />
<meta http-equiv="X-UA-Compatible" content="IE=Edge,chrome=1">
<title>資治通鑒_235-資治通鑑卷二百三十四_235-資治通鑑卷二百三十四</title>
<meta name="Keywords" content="資治通鑒_235-資治通鑑卷二百三十四_235-資治通鑑卷二百三十四">
<meta name="Description" content="資治通鑒_235-資治通鑑卷二百三十四_235-資治通鑑卷二百三十四">
<meta http-equiv="Cache-Control" content="no-transform" />
<meta http-equiv="Cache-Control" content="no-siteapp" />
<link href="/img/style.css" rel="stylesheet" type="text/css" />
<script src="/img/m.js?2020"></script> 
</head>
<body>
 <div class="ClassNavi">
<a  href="/24shi/">二十四史</a> | <a href="/SiKuQuanShu/">四库全书</a> | <a href="http://www.guoxuedashi.com/gjtsjc/"><font  color="#FF0000">古今图书集成</font></a> | <a href="/renwu/">历史人物</a> | <a href="/ShuoWenJieZi/"><font  color="#FF0000">说文解字</a></font> | <a href="/chengyu/">成语词典</a> | <a  target="_blank"  href="http://www.guoxuedashi.com/jgwhj/"><font  color="#FF0000">甲骨文合集</font></a> | <a href="/yzjwjc/"><font  color="#FF0000">殷周金文集成</font></a> | <a href="/xiangxingzi/"><font color="#0000FF">象形字典</font></a> | <a href="/13jing/"><font  color="#FF0000">十三经索引</font></a> | <a href="/zixing/"><font  color="#FF0000">字体转换器</font></a> | <a href="/zidian/xz/"><font color="#0000FF">篆书识别</font></a> | <a href="/jinfanyi/">近义反义词</a> | <a href="/duilian/">对联大全</a> | <a href="/jiapu/"><font  color="#0000FF">家谱族谱查询</font></a> | <a href="http://www.guoxuemi.com/hafo/" target="_blank" ><font color="#FF0000">哈佛古籍</font></a> 
</div>

 <!-- 头部导航开始 -->
<div class="w1180 head clearfix">
  <div class="head_logo l"><a title="国学大师官网" href="http://www.guoxuedashi.com" target="_blank"></a></div>
  <div class="head_sr l">
  <div id="head1">
  
  <a href="http://www.guoxuedashi.com/zidian/bujian/" target="_blank" ><img src="http://www.guoxuedashi.com/img/top1.gif" width="88" height="60" border="0" title="部件查字,支持20万汉字"></a>


<a href="http://www.guoxuedashi.com/help/yingpan.php" target="_blank"><img src="http://www.guoxuedashi.com/img/top230.gif" width="600" height="62" border="0" ></a>


  </div>
  <div id="head3"><a href="javascript:" onClick="javascript:window.external.AddFavorite(window.location.href,document.title);">添加收藏</a>
  <br><a href="/help/setie.php">搜索引擎</a>
  <br><a href="/help/zanzhu.php">赞助本站</a></div>
  <div id="head2">
 <a href="http://www.guoxuemi.com/" target="_blank"><img src="http://www.guoxuedashi.com/img/guoxuemi.gif" width="95" height="62" border="0" style="margin-left:2px;" title="国学迷"></a>
  

  </div>
</div>
  <div class="clear"></div>
  <div class="head_nav">
  <p><a href="/">首页</a> | <a href="/ShuKu/">国学书库</a> | <a href="/guji/">影印古籍</a> | <a href="/shici/">诗词宝典</a> | <a   href="/SiKuQuanShu/gxjx.php">精选</a> <b>|</b> <a href="/zidian/">汉语字典</a> | <a href="/hydcd/">汉语词典</a> | <a href="http://www.guoxuedashi.com/zidian/bujian/"><font  color="#CC0066">部件查字</font></a> | <a href="http://www.sfds.cn/"><font  color="#CC0066">书法大师</font></a> | <a href="/jgwhj/">甲骨文</a> <b>|</b> <a href="/b/4/"><font  color="#CC0066">解密</font></a> | <a href="/renwu/">历史人物</a> | <a href="/diangu/">历史典故</a> | <a href="/xingshi/">姓氏</a> | <a href="/minzu/">民族</a> <b>|</b> <a href="/mz/"><font  color="#CC0066">世界名著</font></a> | <a href="/download/">软件下载</a>
</p>
<p><a href="/b/"><font  color="#CC0066">历史</font></a> | <a href="http://skqs.guoxuedashi.com/" target="_blank">四库全书</a> |  <a href="http://www.guoxuedashi.com/search/" target="_blank"><font  color="#CC0066">全文检索</font></a> | <a href="http://www.guoxuedashi.com/shumu/">古籍书目</a> | <a   href="/24shi/">正史</a> <b>|</b> <a href="/chengyu/">成语词典</a> | <a href="/kangxi/" title="康熙字典">康熙字典</a> | <a href="/ShuoWenJieZi/">说文解字</a> | <a href="/zixing/yanbian/">字形演变</a> | <a href="/yzjwjc/">金 文</a> <b>|</b>  <a href="/shijian/nian-hao/">年号</a> | <a href="/diming/">历史地名</a> | <a href="/shijian/">历史事件</a> | <a href="/guanzhi/">官职</a> | <a href="/lishi/">知识</a> <b>|</b> <a href="/zhongyi/">中医中药</a> | <a href="http://www.guoxuedashi.com/forum/">留言反馈</a>
</p>
  </div>
</div>
<!-- 头部导航END --> 
<!-- 内容区开始 --> 
<div class="w1180 clearfix">
  <div class="info l">
   
<div class="clearfix" style="background:#f5faff;">
<script src='http://www.guoxuedashi.com/img/headersou.js'></script>

</div>
  <div class="info_tree"><a href="http://www.guoxuedashi.com">首页</a> > <a href="/SiKuQuanShu/fanti/">四库全书</a>
 > <h1>资治通鉴</h1> <!--         下载:【右键另存为】即可 --></div>
  <div class="info_content zj clearfix">
  
<div class="info_txt clearfix" id="show">
<center style="font-size:24px;">235-資治通鑑卷二百三十四</center>
    資治通鑑卷二百三十四 宋 司馬光 撰<br />
<br />
  胡三省 音註<br />
<br />
  唐紀五十【起玄黓涒灘盡閼逢閹茂五月二年有奇始壬申終甲戌五月凡二年零五月】德宗神武聖文皇帝九<br />
<br />
  貞元八年春二月壬寅執夢衝數其罪而斬之【數所具翻又所主翻】雲南之路始通 三月丁丑山南東道節度使曹成王臯薨【使疏吏翻臯諡曰成薨呼肱翻】 宣武節度使劉玄佐有威略每李納使至玄佐厚結之故常得其隂事先為之備納憚之【孫子五間有因間因間者因其鄉人而用之張預注云因敵國人知其㡳裏就而用之可使伺候也劉玄佐之制李納正用此術】其母雖貴日織絹一匹謂玄佐曰汝本寒微天子富貴汝至此必以死報之故玄佐始終不失臣節【史言玄佐忠順母教也此言蓋本之劉氏母墓誌唐人誌墓不無溢美者然此等言語有益於世教】庚午玄佐薨 山南東道節度判官李實知留後事性刻薄裁損軍士衣食鼓角將楊清潭帥衆作亂【將即亮翻鼓角將掌軍中鼓角者也帥讀曰率】夜焚掠城中獨不犯曹王臯家【曹王臯之家蓋已出次外館不居使宅】實踰城走免明旦都將徐誠縋城而入【縋馳為翻】號令禁遏然後止收清潭等六人斬之實歸京師以為司農少卿【少詩照翻】實元慶之玄孫也【道王元慶高祖之子】丙子以荆南節度使樊澤為山南東道節度使 初竇參為度支轉運使【度徒洛翻使疏吏翻】班宏副之參許宏俟一歲以使職歸之歲餘參無歸意宏怒司農少卿張滂宏所薦也【少始照翻滂普郎翻】參欲使滂分主江淮鹽鐵宏不可滂知之亦怨宏及參為上所疎乃讓度支使於宏又不欲利權專歸於宏乃薦滂於上以滂為戶部侍郎鹽鐵轉運使仍隸於宏以悦之 竇參隂狡而愎【狡古巧翻愎弼力翻】恃權而貪每遷除多與族子給事中申議之申招權受賂時人謂之喜鵲【竇参每遷除朝士先與申議申因先報其人以招權納賂時人謂之喜鵲者以人家有喜事鵲必先噪於門庭以報之也】上頗聞之謂參曰申必為卿累【累良瑞翻】宜出之以息物議參再三保其無他申亦不悛【悛丑緣翻】左金吾大將軍虢王則之巨之子也【虢王巨肅宗上元二年為段子璋所殺】與申善左諫議大夫知制誥吴通玄與陸贄不叶竇申恐贄進用隂與通玄則之作謗書以傾贄上皆察知其狀夏四月丁亥貶則之昭州司馬【昭州漢荔浦縣地屬蒼梧郡晉置平樂縣屬始安郡武德四年置樂州貞觀八年改曰昭州宋白曰郡北有昭崗潭因山岡為名舊志昭州至京師四千四百三十六里】通玄泉州司馬【隋置泉州治閩縣南安莆田縣屬焉武后聖歷二年分泉州之南安莆田龍溪置武榮州景雲二年改武榮為泉州而閩之泉州改為閩州開元十三年又改閩州為福州舊志泉州京師東南六千二百一十六里】申道州司馬尋賜通玄死 劉玄佐之喪將佐匿之稱疾請代上亦為之隱【將即亮翻為于偽翻】遣使即軍中問以陜虢觀察使吳湊為代可乎監軍孟介行軍司馬盧瑗皆以為便然後除之【陜失冉翻監古銜翻瑗于眷翻】湊行至汜水【汜音祀汜水縣本屬鄭州時屬孟州】玄佐之柩將發軍中請備儀仗瑗不許又令留器用以俟新使將士怒玄佐之壻及親兵皆被甲擁玄佐之子士寧釋衰絰登重榻【被皮義翻衰倉回翻重直龍翻】自為留後執城將曹金㟁【城將使之領兵廵視城堞晨夕警邏】浚儀令李邁曰爾皆請吳湊者遂冎之【冎古瓦翻】盧瑗逃免士寧以財賞將士劫孟介以請於朝上以問宰相竇參曰今汴人指李納以邀制命不許將合於納庚寅以士寧為宣武節度使 【考異曰實録士寧位未定遣使通王武俊劉濟田緒以士寧未受詔有國使皆留之舊傳云以士寧未受詔於國皆留之新傳云諸鎭不直之皆執其使然則舊傳是也】士寧疑宋州刺史翟良佐不附己【翟直格翻】託言廵撫至宋州以都知兵馬使劉逸準代之 【考異曰韓愈集作逸淮今從舊傳】逸準正臣之子也【劉正臣肅宗至德初為平盧節度使】 乙未貶中書侍郎同平章事竇參為郴州别駕【舊志郴州京師南三千三百里郴丑林翻 考異曰柳珵上清傳曰貞元壬申歲春三月相國竇公居光福里第月夜閒步於中庭有常所寵青衣上清者乃曰今欲啟事郎須到堂前方敢言之竇公亟上堂上清曰庭樹上有人恐驚郎請謹避之竇公曰陸贄久欲傾奪吾權位今有人在庭樹上吾禍將至且此事奏與不奏皆受禍必竄死於道路汝在輩流中不可多得吾身死家破汝定為宫婢聖君若顧問善為我辭焉上清泣曰誠如是死生以之竇公下階大呼曰樹上君子應是陸贄使來能全老夫性命敢不厚報樹上應聲而下乃衣衰麤者也曰家有大喪貧甚不辦葬禮伏知相公推心濟物所以卜夜而來幸相公無怪公曰某罄所有堂封絹千匹而已方擬修私廟今且輟贈可乎縗者拜謝竇公答之如禮又曰便辭相公請左右齎所賜絹擲於牆外某先於街中俟之竇公依其請命僕使偵其絕蹤且方敢歸寢翌日執金吾先奏其事竇公得次又奏之德宗厲聲曰卿交通節將蓄養俠刺位崇台鼎更欲何求竇公頓首曰臣起自刀筆小才官以至貴皆陛下奬拔實不由人今不幸至此抑亦仇家所為耳陛下忽震雷霆之怒臣便合死中使下殿宣曰卿且歸私第待進止越月貶郴州别駕會宣武節度使劉士寧通好于郴州亷使條疏上聞德宗曰交通節將言而有徵流竇公于驩州沒入家資一簪不著身竟未逹流所詔自盡上清果隸名掖庭後數年以善應對能煎茶數得在帝左右德宗謂曰宫掖間人數不少汝了事從何得至此上清對曰妾本故宰相竇參家女奴竇某妻早亡故妾得陪掃灑及竇某家破幸得塡宫旣侍龍顔如在天上德宗曰竇某罪不止養俠刺亦甚有贓汙前時納官銀器至多上清流涕而言曰竇某自御史中丞歷度支戶部鹽鐵三使至宰相首尾六年月入數十萬前後非時賞賜亦不知紀極乃者郴州所送納官銀物皆是恩賜當部録日妾在郴州親見州縣希陸䞇意指刮去所進銀器上刻作藩鎭官衘姓名誣為贓物伏乞陛下驗之於是宣索竇某沒官銀器覆視其刮字處皆如上清言時貞元十二年德宗又問蓄養俠刺事上清曰本實無悉是陸贄陷害使人為之德宗怒陸贄曰這獠奴我脱却伊緑衫便與紫衫著又常喚伊作陸九我任使竇參方稱意次須教我枉殺却它及至權入伊手其為軟弱甚於泥團乃下詔雪竇參時裴延齡探知陸贄恩衰得恣行媒糵贄竟受譴後上清特勑丹書度為女道士終嫁為金忠義妻世以陸贄門生名位多顯達者不敢傳說故此事絕無人知信如此說則參為人所劫德宗豈得反云蓄養俠刺况陸贄賢相安肯為此就使欲陷參其術固多豈肯為此兒戲全不近人情今不取】貶竇申錦州司戶以尚書左丞趙憬兵部侍郎陸贄並為中書侍郎同平章事憬仁本之曾孫也【憬居永翻趙仁本見二百一卷高宗咸亨元年】 張滂請鹽鐵舊簿於斑宏宏不與共擇巡院官莫有合者闕官甚多滂言於上曰如此職事必廢臣罪無所逃丙午上命宏滂分掌天下財賦如大歷故事【大歷元年命第五琦劉晏分理天下財賦事見二百二十四卷】 壬子吐蕃寇靈州䧟水口支渠敗營田【敗補邁翻】詔河東振武救之遣神策六軍二千戍定遠懷遠城【懷遠縣屬靈州後周置隋五原郡在縣界宋白曰定遠縣在靈州東北二百里】吐蕃乃退 陸贄請令臺省長官各舉其屬【長知兩翻】著其名於詔書異日考其殿最并以升黜舉者【殿丁練翻所舉得人則升舉主以昭進賢之賞所舉非人則黜舉主以昭失舉之罰】五月戊辰詔行贄議未幾或言於上曰諸司所舉皆有情故或受貨賂不得實才上密諭贄自今除改卿宜自擇勿任諸司【諸司即謂臺省長官】贄上奏其略曰國朝五品以上制敕命之蓋宰相商議奏可者也六品以下則旨授蓋吏部銓材署職詔旨畫聞而不可否者也【六品以下告身皆畫聞字】開元中起居遺補御史等官猶並列於選曹【言起居郎舍人拾遺補闕及御史皆由吏部奏擬選須絹翻】其後倖臣專朝【朝直遥翻】捨僉議而重己權廢公舉而行私惠是使周行庶品【行戶剛翻下班行同】苟不出時宰之意則莫致也又曰宣行以來纔舉十數議其資望既不愧於班行考其行能又未聞於闕敗【行下孟翻】而議者遽以騰口上煩聖聰道之難行亦可知矣請使所言之人指陳其狀某人受賄某舉有情付之有司覈其虛實謬舉者必行其罰誣善者亦反其辜【謂反坐以罪也】何必貸其姦贓不加辯詰私其公議不出主名【主名告主之名也】使無辜見疑有罪獲縱枉直同貰人何賴焉又宰相不過數人豈能徧諳多士【諳烏含翻】若令悉命羣官理須展轉詢訪是則變公舉為私薦易明揚以闇投【公私明闇以相形而文理自見此作文之法然明揚二字本之虞書闇投二字本之漢書作文又不可無來處近世敎人為文者類此文詎止於此而已】情故必多為弊益甚所以承前命官罕不涉謗雖則秉鈞不一或自行情亦由私訪所親轉為所賣其弊非遠聖鍳明知又曰今之宰相則往日臺省長官今之臺省長官乃將來之宰相但是職名暫異固非行舉頓殊【行舉者臺省長官舉之宰相行之】豈有為長官之時則不能舉一二屬吏居宰相之位則可擇千百具僚物議悠悠其惑斯甚蓋尊者領其要卑者任其詳是以人主擇輔臣輔臣擇庶長【庶長庶官之長也】庶長擇佐僚將務得人無易於此夫求才貴廣考課貴精往者則天欲收人心進用不次【則天謂武后也】非但人得薦士亦得自舉其才然而課責既嚴進退皆速是以當代謂知人之明累朝賴多士之用又曰則天舉用之法傷易而得人【朝直遥翻易以豉翻】陛下愼簡之規【書曰愼簡乃僚】太精而失士上竟追前詔不行 癸酉平盧節度使李納薨軍中推其子師古知留後 六月吐蕃千餘騎寇涇州掠田軍千餘人而去【田軍屯田之軍也】 嶺南節度使奏近日海舶珍異多就安南市易欲遣判官就安南收市乞命中使一人與俱上欲從之陸贄上言以為遠國商販惟利是求緩之斯來擾之則去廣州素為衆舶所湊【舶音白】今忽改就安南若非侵刻過深則必招攜失所【攜離也言所以招攜離者失其道也左傳管仲曰招攜以禮】曾不内訟【論語孔子曰吾未見能見其過而内自訟者也注云訟猶責也言人有過莫能自責】更蕩上心【記月令毋或作為淫巧以蕩上心注蕩謂動揺之也】况嶺南安南莫非王土中使外使悉是王臣豈必信嶺南而絕安南重中使以輕外使所奏望寢不行秋七月甲寅朔戶部尚書判度支班宏薨【尚辰羊翻度徒洛翻薨呼肱翻】陸贄請以前湖南觀察使李巽權判度支上許之既而復欲用司農少卿裴延齡【使疏吏翻度徒洛翻復扶又翻又音如字少詩照翻】贄上言以為今之度支準平萬貨【上時掌翻】刻吝則生患寛假則容姦延齡誕妄小人用之交駭物聽尸禄之責固宜及於微臣知人之明亦恐傷於聖鑒上不從己未以延齡判度支事【為裴延齡譛贄張本】河南北江淮荆襄陳許等四十餘州大水溺死者二萬餘人陸贄請遣使賑撫上曰聞所損殊少【溺奴狄翻少詩沼翻】即議優恤恐生姦欺贄上奏其略曰流俗之弊多徇諂諛揣所悦意則侈其言度所惡聞則小其事【揣初委翻度徒洛翻惡烏路翻】制備失所恒病於斯【制備謂隨事為之制而豫備也恒戶登翻】又曰所費者財用所收者人心苟不失人何憂乏用上許為遣使【為于偽翻】而曰淮西貢賦既闕不必遣使贄復上奏【復扶又翻】以為陛下息師含垢宥彼渠魁【渠大也魁率也】惟兹下人所宜矜恤昔秦晉讎敵穆公猶救其饑【左傳晉饑秦輸之粟秦饑晉閉之糴穆公伐晉執惠公而晉又饑穆公復餼之粟曰吾怨其君而矜其民】况帝王懷柔萬邦唯德與義寧人負我無我負人【反曹操之言則有帝王氣象】八月遣中書舍人京兆奚陟等宣撫諸道水災以前青州刺史李師古為平盧節度使 韋臯攻維州【代宗廣德元年維州沒於吐蕃】獲其大將論贊熱 陸贄上言以邊儲不贍由措置失當【當丁浪翻】蓄歛乖宜其略曰所謂措置失當者戍卒不隸於守臣守臣不總於元帥至有一城之將一旅之兵各降中使監臨【監古銜翻】皆承别詔委任分鎭亘千里之地莫相率從緣邊列十萬之師不設謀主每有寇至方從中覆比蒙徵發赴援【比必利翻及也】寇已獲勝罷歸吐蕃之比中國衆寡不敵工拙不侔然而彼攻有餘我守不足蓋彼之號令由將而我之節制在朝【將即亮翻朝直遥翻】彼之兵衆合并而我之部分離析故也【分扶問翻】所謂蓄歛乖宜者陛下頃設就軍和糴之法以省運制與人加倍之價以勸農此令初行人皆悦慕【此李泌所行之法也事見卷二年】而有司競為苟且專事纎嗇歲稔則不時歛藏艱食則抑使收糴遂使豪家貪吏反操利權【歛力驗翻操士刀翻】賤取於人以俟公私之乏又有勢要近親羈遊之士委賤糴於軍城取高價於京邑又多支絺紵充直【絺丑之翻紵直呂翻】窮邊寒不可衣鬻無所售上既無信於下下亦以偽應之度支物估轉高【度徒洛翻估音古價也】軍城穀價轉貴度支以苟售滯貨為功利軍城以所得加價為羨餘【羨弋線翻】雖設巡院轉成囊槖【元和四年十二月十二日勑遠處州使率情違法臺司無由盡知轉運使度支悉有巡院委以訪察當道使司及州縣有兩税外榷率及違格勑文法等事狀報臺司蓋劉晏始置巡院自江淮以來達于河渭其後逐及緣邊諸道亦置之】至有空申簿帳偽指囷倉【囷區倫翻囷倉皆以藏穀圓曰囷方曰倉】計其數則億萬有餘考其實則百十不足又曰舊制以關中用度之多歲運東方租米至有斗錢運斗米之言習聞見而不達時宜者則曰國之大事不計費損雖知勞煩不可廢也習近利而不防遠患者則曰每至秋成之時但令畿内和糴既易集事【令力丁翻糴亭歷翻易以䜴翻】又足勸農臣以兩家之論互有長短將制國用須權重輕食不足而財有餘則弛於積財而務實倉廪【廪力錦翻毛晃曰倉有屋曰廩】食有餘而財不足則緩於積食而嗇用貨泉近歲關輔屢豐公儲委積【屢力注翻委於偽翻積子智翻】足給數年今夏江淮水潦米貴加倍人多流庸【流謂流徙庸謂庸雇】關輔以穀賤傷農宜加價以糴而無錢江淮以穀貴人困宜減價以糶而無米【糶他弔翻】而又運彼所乏益此所餘斯所謂習見聞而不達時宜者也今江淮斗米直百五十錢運至東渭橋僦直又約二百米糙且陳【僦子就翻糙七到翻米僅剥穀為糙】尤為京邑所賤據市司月估【今之市令司亦月具物價低昂之數以聞於上】斗糶三十七錢耗其九而存其一【以江淮之米合運漕之僦直率一斗為錢三百五十而京師米價斗止三十七錢是耗其九而存其一也】餒彼人而傷此農制事若斯可謂深失矣頃者每年自江湖淮浙運米百一十萬斛至河隂留四十萬斛貯河隂倉至陜州又留三十萬斛貯太原倉【貯丁呂翻】餘四十萬斛輸東渭橋今河隂太原倉見米猶有三百二十餘萬斛【見賢遍翻】京兆諸縣斗米不過直錢七十請令來年江淮止運三十萬斛至河隂河隂陜州以次運至東渭橋其江淮所停運米八十萬斛委轉運使每斗取八十錢於水災州縣糶之以救貧乏【糶它吊翻】計得錢六十四萬緡减僦直六十九萬緡請令戶部先以二十萬緡付京兆令糴米以補渭橋倉之缺數【渭橋倉即東渭橋倉】斗用百錢以利農人【增價以糴則利農】以一百二萬六千緡付邊鎭使糴十萬人一年之糧餘十萬四千緡以充來年和糴之價【糴徒歷翻】其江淮米錢僦直並委轉運使折市綾絹絁綿以輸上都【折之舌翻絁式支翻繒之似布者今謂之紬唐都長安謂之上都】償先貸戶部錢九月詔西北邊貴糴以實倉儲【考異曰實録云凡積米三十三萬斛按陸贄論守備狀坐致邊儲數逾百萬諸鎭收糴今已向終又云更經】<br />
<br />
  【一年可積十萬人三歲之糧矣蓋實録所言今年之數贄狀通計來春也】邊備浸充 冬十一月壬子朔日有食之 吐蕃雲南日益相猜每雲南兵至境上吐蕃輒亦發兵聲言相應實為之備辛酉韋臯復遺雲南王書【復扶又翻遺唯季翻】欲與共襲吐蕃驅之雲嶺之外【雲南之地本漢雲南縣也漢屬益州郡後漢分屬永昌郡南中志曰雲南縣西高山相連衆山之中又有山特高大狀如扶風太一鬱然高峻與雲氣相連視之不見其山固隂沍寒雖五月盛暑不熱所謂雲嶺也】悉平吐蕃城堡獨與雲南築大城於境上置戍相保永同一家 右庶子姜公輔久不遷官詣陸䞇求遷贄密語之曰【語牛倨翻】聞竇相屢奏擬上不允【今人謂聖旨不從所請為不允習聞唐人之言也】有怒公之言公輔懼請為道士上問其故公輔不敢泄贄語以聞參言為對上怒參歸怨於君己巳貶公輔為吉州别駕又遣中使責參【姜公輔居猜忌之朝不能安於命義而由此重竇參之罪亦陸䞇之一言也 考異曰實録初公輔罷相為左庶子以憂免復除右庶子數私謁竇參參數奏公輔以他官上不許而有怒公輔之言公輔恐乃請免官為道士久之未報因開延英奏之上問其故公輔對以參言上曉之固不已大怒貶之而詔書責參推過於上公輔傳曰陸贄知政事以有翰林之舊數告䞇求官贄密謂公輔曰予常見郴州竇相言為公奏擬數矣上旨不允有怒公之言公輔恐懼乞罷官為道士久之未報後又庭奏德宗問其故公輔不敢泄䞇言便以參言為對帝怒貶公輔為泉州别駕又遣使齎詔責參贄傳曰姜公輔奏竇參常語臣云陛下怒臣未已德宗怒再貶參竟殺之時議云公輔奏竇參語得之於贄云參之死贄有力焉按贄請令長官舉屬吏狀云亦由私訪所親轉為所賣其弊非遠聖鑒明知此乃解參之語也及參之死贄救解甚至由是觀之贄豈有殺參之意邪且䞇語公輔之時安知公輔請為道士及於上前以泄言之罪歸參此乃公輔之意非贄意也當時之人見參贄有隙遂以己意猜之史官不悦贄者因歸罪於贄耳今不取】 庚午山南西道節度使嚴震奏敗吐蕃於芳州及黑水堡【敗補邁翻芳州高宗上元二年已為吐蕃所陷酈道元曰黑水出羌中西南逕黑水城西其地蓋在隂平西北臨洮西南古沓中之地也使疏吏翻敗補邁翻堡音保】 初李納以棣州蛤有鹽利城而據之又戍德州之南三汊城以通田緒之路【棣大計翻蛤古合翻康音螺余按集韻螺字下無字同韻有垜字音都戈翻小堆也恐當作垜汊楚嫁翻李納之阻兵也李長卿以棣州入朱滔而蛤為納所據因城而戍之其後王武俊敗朱滔得德棣二州蛤猶為納戍納又於德州南跨河而城守之謂之三汊以交魏博通田緒】及李師古襲位王武俊以其年少輕之【少詩照翻】是月引兵屯德棣將取蛤及三汊城師古遣趙鎬將兵拒之上遣中使諭止之武俊乃還【鎬下老翻將即亮翻又音如字使疏吏翻還從宣翻又音如字】初劉怦薨【見二百三十二卷貞元元年怦普萌翻薨呼肱翻】劉濟在莫州其母弟澭在父側以父命召濟而以軍府授之【莫州治鄚縣在幽州南二百八十里澭於用翻】濟以澭為瀛州刺史【瀛州河間郡幽州巡屬大州也其地在幽州南】許它日代己既而濟用其子為副大使【河朔三鎭及淄青皆以其子為副大使儲帥也】澭怨之擅通表朝廷遣兵千人防秋濟怒發兵擊澭破之【朝直遥翻為劉澭歸朝張本】 左神策大將軍柏良器【唐左右神策大將軍正二品史炤曰柏皇氏古帝號後為氏顓帝師柏亮父帝嚳父柏超之裔也】募才勇之士以易販鬻者監軍竇文場惡之【惡烏路翻】會良器妻族飲醉寓宿宫舍【宫舍宫中直宿之舍也史言䆠官惡柏良器能舉其職因其妻黨犯衛禁而文致其罪】十二月丙戍良器坐左遷右領軍自是䆠官始專軍政【為䆠官挾兵權以脅天子張本右領軍十六衛之一也時南牙諸衛具位而已北軍掌禁兵權重故良器為左遷】九年春正月癸卯初稅茶【爾雅釋木云檟苦茶郭璞注云樹大小似梔子冬生葉可煮作羮飲今呼早採者為晩採者為茗一名荈蜀人為之苦茶是也今通謂之茶茶聲近故呼之春中始生嫩葉蒸焙去苦水末之乃可喫與古所食殊不同也本草衍義曰晉温嶠上表貢茶千斤茗三百斤郭璞曰早採為茶晚採為茗茗或曰荈兖葉老者也古人謂其芽為雀舌麥顆言其至嫩也又有新芽一發便長寸餘微麄如針惟芽長為上品其根榦土力皆有餘故也如雀舌麥顆又下品前人未盡識史言税茶始此遂開利孔】凡州縣產茶及茶山外要路皆估其直什稅一從鹽鐵使張滂之請也滂奏去歲水災减稅用度不足請稅茶以足之自明年以往稅茶之錢令所在别貯俟有水旱以代民田稅自是歲收茶稅錢四十萬緡未嘗以救水旱也【榷茶之稅始於趙贊至張滂而始行】滂又奏姦人銷錢為銅器以求贏請悉禁銅器銅山聽人開采無得私賣二月甲寅以義武留後張昇雲為節度使 初鹽州既䧟【鹽州陷見二百三十二卷二年】塞外無復保障吐蕃常阻絕靈武侵擾鄜坊【既阻絕靈武往來之路又侵擾鄜坊之民】辛酉詔發兵三萬五千人城鹽州 【考異曰邠志八年詔遣張公議築鹽夏二城張公奏曰師之進取切籍驍將神策散將魏茪者武藝冠絕得茪足以集事上遣之張公以茪為邠寧馬軍兵馬使三月師及諸軍赴于五原去城百里而軍茪獨以其騎徑至城下䧟城而入逐吐蕃召諸軍城之更引其軍西掠境上往復走望為師耳目蕃衆距境而不敢入官軍城二郡而歸白居易樂府鹽州注亦云貞元壬申歲特詔城之而實錄在九年二月蓋去歲詔使城之今年因命杜彦光等而言之】又詔涇原山南劒南各發兵深入吐蕃以分其勢城之二旬而畢命鹽州節度使杜彦光戍之朔方都虞楊朝晟戍木波堡【木波堡在慶州方渠縣界九域志方渠宋朝改為通遠縣置環州有木波鎮】由是靈夏河西獲安 上使人諭陸贄以要重之事勿對趙憬陳論當密封手疏以聞又苖粲以父晉卿往年攝政【寶應間連有國憂晉卿攝冢宰】嘗有不臣之言諸子皆與古帝王同名【晉卿十子發丕堅垂與帝王同名】今不欲明行斥逐兄弟亦各除外官勿使近屯兵之地【近其靳翻】又卿清愼太過諸道饋遺【遺唯季翻】一皆拒絕恐事情不通如鞭靴之類受亦無傷贄上奏其略曰昨臣所奏惟趙憬得聞陛下已至勞神委曲防護是於心膂之内尚有形迹之拘迹同事殊鮮克以濟【鮮息淺翻】恐爽無私之德【爽差也】且傷不吝之明【書曰改過不吝】又曰爵人必於朝刑人必於市惟恐衆之不覩事之不彰【記曰爵人於朝與衆共之刑人於市與衆弃之朝直遥翻】君上行之無愧心兆庶聽之無疑議受賞安之無怍色當刑居之無怨言此聖王所以宣明典章與天下公共者也凡是譖訴之事多非信實之言利於中傷【中作仲翻】懼於公辯或云歲月已久不可究尋或云事體有妨須為隱忍【為于偽翻】或云惡迹未露宜假它事為名或云但弃其人何必明言責辱詞皆近於情理【近其靳翻】意實苞於矯誣傷善售姦莫斯為甚若晉卿父子實有大罪則當公議典憲若被誣枉豈令隂受播遷夫聽訟辨讒必求情辨跡情見跡著【見賢遍翻】辭服理窮然後加刑罰焉是以下無寃人上無謬聽又曰監臨受賄盈尺有刑【律諸監臨之官受所監臨財物者一尺笞四十諸監臨主司受財而枉法者一尺杖一百監古銜翻】至於士吏之微尚當嚴禁矧居風化之首反可通行【風化之首謂宰相者風化之所自出】賄道一開展轉滋甚鞭鞾不已必及金玉【鞾與靴同】目見可欲何能自窒于心【古語有之不見可欲此心不亂】已與交私何能中絕其意【謂既受其私饋則難以絕其私謁】是以涓流不絕溪壑成災矣又曰若有所受有所却則遇却者疑乎見拒而不通矣若俱辭不受則咸知不受者乃其常理復何嫌阻之有乎【復扶又翻下同】 初竇參惡左司郎中李巽【惡烏路翻】出為常州刺史及參貶郴州巽為湖南觀察使【郴丑林翻】汴州節度使劉士寧遺參絹五十匹【遺唯季翻】巽奏參交結藩鎮上大怒欲殺參陸贄以為參罪不至死上乃止既而復遣中使謂贄曰參交結中外其意難測社稷事重卿速進文書處分【處昌呂翻分扶問翻下無分同】贄上言參朝廷大臣誅之不可無名昔劉晏之死罪不明白至今衆議為之憤邑【為于偽翻】叛臣得以為辭【見二百二十六卷建中元年二月】參貪縱之罪天下共知至於濳懷異圖事跡曖昧【曖音愛不明貌】若不推鞠遽加重辟駭動不細【辟音闢刑辟】竇參於臣無分【言無契分之雅分扶問翻】陛下所知豈欲營救其人盖惜典刑不濫三月更貶參驩州司馬男女皆配流上又命理其親黨【理治也】贄奏罪有首從法有重輕【首謂為頭者從謂隨從者為首者重隨從者輕】參既蒙宥親黨亦應末減況參得罪之初私黨並已連坐人心久定請更不問從之上又欲籍其家貲贄曰在法反逆者盡沒其財贜汚者止徵所犯皆須結正施刑然後收藉今罪法未詳陛下已存惠貸若簿錄其家恐以財傷義時䆠官左右恨參尤深謗毁不已參未至驩州竟賜死於路竇申杖殺貨財奴婢悉傳送京師【傳知戀翻】 海州團練使張昇璘昇雲之弟李納之壻也以父大祥歸于定州【海州東海郡淄青巡屬璘離珍翻壻西計翻定州義武帥治所子居父喪再朞而大祥】嘗於公座罵王武俊武俊奏之夏四月丁丑詔削其官遣中使杖而囚之【使所吏翻】定州富庶武俊常欲之因是遣兵襲取義豐掠安喜無極萬餘口徒之德棣【義豐屬定州安喜縣本定州治所盖州治徙也無極漢古縣因無極山為名唐屬定州按無極山碑云無極山與天地俱生從上至體可三里所立石為體二丈五尺所石上青下黄白所前正平可鋪兩大席在無極西南三十里景福二年以無極縣為祁州棣大計翻】昇雲閉城自守屢遣使謝之乃止上命李師古毁三汊城【李納築三汊城見上年汉楚嫁翻】師古奉詔然常招聚亡命有得罪於朝廷者皆撫而用之【朝直遥翻】 五月甲辰以中書侍郎趙憬為門下侍郎同平章事義成節度使賈耽為右僕射右丞盧邁守本官並同平章事邁翰之族子也【憬居永翻耽都含翻射寅謝翻興元時盧翰與李勉劉從一同為相】憬疑陸贄恃恩欲專大政排己置之門下【政事堂在中書省今憬遷東省故疑贄排己右僕射屬門下省】多稱疾不豫事由是與贄有隙【為趙憬附裴延齡張本 考異曰舊憬傳曰憬與陸贄同知政事贄恃久在禁庭特承恩顧以國政為己任纔周歲轉憬為門下侍郎憬由是深銜之數以目疾請告不甚當政事因是不相恊按憬遷門下猶為宰相又益以賈耽盧邁贄豈得專政盖憬以此心疑之耳】 陸贄上奏論備邊六失以為措置乖方課責虧度財匱於兵衆力分於將多怨生於不均機失於遥制【自措置乖方以下所謂六失也上時掌翻將即亮翻下同】關東戍卒不習土風身苦邊荒心畏戎虜國家資奉若驕子姑息如倩人【倩七政翻】屈指計歸張頤待哺或利王師之敗乘擾攘而東潰或拔棄城鎭搖遠近之心豈惟無益實亦有損復有犯刑讁徙者【復扶又翻下同】既是無良之類且加懷土之情思亂幸災又甚戍卒可謂措置乖方矣【此一失也】自頃權移於下柄失於朝將之號令既鮮克行之於軍國之典常又不能施之於將【朝直遥翻鮮息淺翻將即亮翻】務相遵養【遵率也言相率以養惡也】苟度歲時欲賞一有功翻慮無功者反仄欲罰一有罪復慮同惡者憂虞罪以隱忍而不彰功以嫌疑而不賞姑息之道乃至於斯故使忘身效節者獲誚於等夷【誚才笑翻】率衆先登者取怨於士卒僨軍蹙國者不懷於愧畏【僨方問翻】緩救失期者自以為智能此義士所以痛心勇夫所以解體可謂課責虧度矣【此二失也】虜每入寇將帥遞相推倚【帥所類翻推吐雷翻】無敢誰何虛張賊勢上聞則曰兵少不敵朝廷莫之省察【朝直遥翻省悉景翻】唯務徵發益師無禆備禦之功重增供億之幣【重直用翻】閭井日耗徵求日繁以編戶傾家破產之資兼有司榷鹽税酒之利【榷古岳翻】總其所入歲以事邊可謂財匱於兵衆矣【此三失也】吐蕃舉國勝兵之徒【勝音升勝兵謂人之才力堪執兵以戰者也】纔當中國十數大郡而已動則中國懼其衆而不敢抗静則中國憚其強而不敢侵厥理何哉良以中國之節制多門蕃醜之統帥專一故也【帥讀曰率】夫統帥專一則人心不分號令不貳進退可齊疾徐如意機會靡愆【愆違也】氣勢自壮斯乃以少為衆以弱為強者也開元天寶之間控禦西北兩蕃唯朔方河西隴右三節度中興以來未遑外討抗兩蕃者亦朔方涇原隴右河東四節度而已【言西北兩蕃者以别奚契丹兩蕃若開元天寶以來西則吐蕃北則突厥中興以來所謂兩蕃西則吐蕃北則囘紇】自頃分朔方之地建牙擁節者凡三使焉【事見二百二十五卷大歷十四年使疏吏翻】其餘鎮軍數且四十皆承特詔委寄各降中貴監臨人得抗衡莫相禀屬【監古銜翻史炤曰衡車上横木抗衡謂兩相抗拒有若車衡相抗也余謂衡所以揆平首尾有所偏重則衡為之昂商輕重者所必争也抗衡者言無所昂而平視之也】每俟邊書告急方令計會用兵既無軍法下臨惟以客禮相待【令力丁翻】夫兵以氣勢為用者也氣聚則盛散則消勢合則威析則弱今之邊備勢弱氣消可謂力分於將多矣【夫音扶將即亮翻此四失也】理戎之要在於練覈優劣之科以為衣食等級之制使能者企及【企去智翻】否者息心雖有厚薄之殊而無觖望之釁【觖古宂翻】今窮邊之地長鎮之兵皆百戰傷夷之餘終年勤苦之劇然衣糧所給唯止當身例為妻子所分常有凍餒之色而關東戍卒怯於應敵懈於服勞衣糧所頒厚踰數等又有素非禁旅本是邊軍將校詭為媚詞因請遥隸神策不離舊所【當丁浪翻懈俱賣翻離去智翻】唯改舊名其於廪賜之饒遂有三倍之益夫事業未異而給養有殊苟未忘懷就能無愠可謂怨生於不均矣【愠於問翻此五失也】凡欲選任將帥必先考察行能【將即亮翻帥所類翻行下孟翻】可者遣之不可者退之疑者不使使者不疑故將在軍君命有所不受【此孫子兵法之言】自頃邊軍去就裁斷多出宸衷【斷丁亂翻】選置戎臣先求易制【易以䜴翻】多其部以分其力輕其任以弱其心遂令爽於軍情亦聽命乖於事宜亦聽命戎虜馳突迅如風飊馹書上聞【馹人質翻驛傳遞馬】旬月方報守土者以兵寡不敢抗敵分鎮者以無詔不肯出師賊既縱掠退歸此乃陳功告捷其敗喪則減百而為一其捃獲則張百而成千【喪息浪翻捃居隕翻】將帥既幸於總制在朝不憂罪累【累良瑞翻】陛下又以為大權由己不究事情可謂機失於遥制矣【此六失也】臣愚謂宜罷諸道將士防秋之制令本道但供衣糧募戍卒願留及蕃漢子弟以給之又多開屯田官為收糴【為于偽翻】寇至則人自為戰時至則家自力農與夫倏來忽往者豈可同等而論哉又宜擇文武能臣為隴右朔方河東三元帥分統緣邊諸節度使有非要者隨所便近而併之然後減姦濫虛浮之費以豐財定衣糧等級之制以和衆弘委任之道以宣其用懸賞罰之典以考其成如是則戎狄威懷疆場寧謐矣上雖不能盡從心甚重之 韋臯遣大將董勔等將兵出西山【勔彌兖翻自彭州導江縣西出蠶崖關歷維茂至當悉諸州皆西山也】破吐蕃之衆拔堡柵五十餘 丙午門下侍郎同平章事董晉罷為禮部尚書 雲南王異牟尋遣使者三輩一出戎州【宋白曰戎州漢僰道南安縣地梁置戎州言以鎮戎夷也西南取曲恊州并南寧州安寧鹽井路至南詔所居羊苴哶城二千三百里舊志戎州在京師西南三千一百四里】一出黔州一出安南各齎生金丹砂詣韋臯【金礦未經鍜鍊者為生金丹砂產石中鑿石取之黔音琴】金以示堅丹砂以示赤心三分臯所與書為信皆達成都異牟尋上表請弃吐蕃歸唐并遺臯帛書【遺唯季翻】自稱唐雲南王孫吐蕃贊普義弟日東王【吐蕃以雲南王為弟見二百一十六卷天寶十載封日東王見二百二十六卷代宗大歷十四年】臯遣其使者詣長安并上表賀上賜異牟尋詔書令臯遣使慰撫之【并上時掌翻令力丁翻使疏吏翻】 賈耽陸贄趙憬盧邁為相百官白事更讓不言【耽都含翻憬居永翻相息亮翻更工衡翻互也】秋七月奏請依至德故事宰相迭秉筆以處政事【事見二百一十九卷肅宗至德元載十月處昌呂翻下同】旬日一易詔從之其後日一易之 劒南西山諸羌女王湯立志【女王亦羌别種東與吐蕃党項茂州接西屬三波訶北距于闐屬雅州羅女蠻白狼夷以女為君居康延川巗險四繚有弱水南流湯立志新書作湯立悉杜陽編女蠻國人危髻金冠瓔珞被體故謂之菩薩蠻當時倡優遂因制菩薩蠻曲】哥隣王董卧庭白狗王羅陀怱弱水王董辟和南水王薛莫庭悉董王湯悉贊清遠王蘇唐磨咄霸王董邈蓬及逋租王【自哥隣以下諸種皆散居西山西山即雪山今威州保寧縣有雪山連乳川白狗嶺有九峯積雪春夏不消白狗嶺與雪山相連威州唐之維州也】先皆役屬吐蕃至是各帥衆内附韋臯處之於維保覇州【天寶元年招附生羌置覇州】給以耕牛種糧【種章勇翻】立志陀怱辟和入朝皆拜官厚賜而遣之 癸卯戶部侍郎裴延齡奏自判度支以來檢責諸州欠負錢八百餘萬緡收諸州抽貫錢三百萬緡呈様物三十餘萬緡請别置欠負耗賸季庫以掌之【耗虧減也賸贏餘也賸以證翻又食證翻三月為一季凡三月終則入物于庫故謂之季庫】柒練物則别置月庫以掌之【每月入物故謂之月庫】詔從之欠負皆貧人無可償徒存其數者抽貫錢給用隨盡呈様染練皆左藏正物【藏徂浪翻】延齡徒置别庫虛張名數以惑上上信之以為能富國而寵之於實無所增也【於實者於其實也】虛費吏人簿書而已京城西汚濕地生蘆葦數畝延齡奏稱長安咸陽有陂澤數百頃可牧廐馬上使有司閱視無之亦不罪也左補闕權德輿上奏【權本顓頊之後為楚武王所滅子孫以國為氏上時掌翻】以為延齡取常賦支用未盡者充羨餘以為己功【羨弋線翻】縣官先所市物再給其直用充别貯【貯丁呂翻】邊軍自今春以來並不支糧陛下必以延齡孤貞獨立時人醜正流言何不遣信臣覆視究其本末明行賞罰今羣情衆口喧於朝市【朝直遥翻】豈京城士庶皆為朋黨邪陛下亦宜稍囘聖慮而察之上不從 八月庚戍太尉中書令西平忠武王李晟薨 冬十月甲子韋臯遣其節度巡官崔佐時齎詔書詣雲南并自為帛書答之【節度巡官在判官推官之下衙推之上】 十一月乙酉上祀圓丘赦天下 劉士寧既為宣武節度使【八年三月命劉士寧為宣武節度使】諸將多不服士寧淫亂殘忍出畋輒數日不返軍中苦之都知兵馬使李萬榮得衆心士寧疑之奪其兵權令攝汴州事十二月乙卯士寧帥衆二萬畋于外野【帥讀曰率】萬榮晨入使府召所留親兵千餘人詐之曰敕徵大夫入朝以吾掌留務汝輩人賜錢三十緡衆皆拜又諭外營兵皆聽命乃分兵閉城門使馳白士寧曰敕徵大夫宜速即路【即就也】少或遷延當傳首以獻士寧知衆不為用以五百騎逃歸京師北至東都【比必利翻及也】所餘僕妾而已至京師敕歸第行喪禁其出入淮西節度使吳少誠聞變發兵屯郾城【郾城漢晉之郾縣也後魏省倂入曲陽縣隋開皇初置郾城縣屬汴州時屬蔡州蔡北鄙也東有漢召陵縣故城東南有後漢征羌縣故城郾一戰翻】遣使問故【問所以逐士寧之故】且請戰萬榮以言戲之【戲之示無所畏】少誠慙而退上聞萬榮逐士寧使問陸贄贄上奏以為今軍州已定宜且遣朝臣宣勞【勞力到翻】徐察事情冀免差失其略曰今士寧見逐雖是衆情萬榮典軍且非朝旨此安危彊弱之機也願陛下審之愼之上復使謂贄【復扶又翻下同】若更淹遲恐於事非便今譏除一親王充節度使且令萬榮知留後其制即從内出贄復上奏其略曰臣雖服戎角力諒匪克堪而經武伐謀或有所見夫制置之安危由勢付授之濟否由才勢如器焉惟在所置置之夷地則平才如負焉唯在所授授踰其力則踣【踣蒲北翻】萬榮今所陳奏頗涉張皇但露徼求之情【徼一遥翻】殊無退讓之禮據兹鄙躁殊異循良又聞本是滑人【劉玄佐滑州匡城人萬榮與同里相善】偏厚當州將士【當州猶言本州謂滑州也】與之相得纔止三千諸營之兵已甚懷怨據此頗僻【頗滂何翻偏也】亦非將材若得志驕盈不悖則敗悖則犯上敗則僨軍【將即亮翻悖蒲内翻僨方問翻】又曰苟邀則不順苟允則不誠君臣之間勢必嫌阻【邀求也非所當求而求之為苟邀允從也非所當從而從之為苟允下以不順求之上以不誠應之其勢必至於嫌阻】與其圖之於滋蔓【左傳曰毋使滋蔓蔓難圖也】不若絕之於萌芽又曰為國之道以義訓人將教事君先令順長【長知兩翻下獨長同】又曰方鎮之臣事多專制欲加之罪誰則無辭若使傾奪之徒便得代居其任利之所在人各有心此源濳滋禍必難救非獨長亂之道亦關謀逆之端又曰昨逐士寧起於倉卒【卒讀曰猝】諸郡守將固非連謀【將即亮翻】一城師人亦未恊志各計度於成敗之勢【度徒洛翻】迴遑於逆順之名安肯捐軀與之同惡又曰陛下但選文武羣臣一人命為節度仍降優詔慰勞本軍【勞力到翻】奨萬榮以撫定之功别加寵任褒將士以輯睦之義厚賜資装揆其大情理必寧息萬榮縱欲跋扈勢何能為又曰儻後事有愆素【左傳曰不愆于素杜預注云不過素所慮也】臣請受敗橈之罪【橈奴敎翻】上不從壬戍以通王諶為宣武節度大使【諶氏壬翻】以萬榮為留後 丁卯納故駙馬都尉郭曖女為廣陵王淳妃淳太子之長子【是為憲宗】妃母即昇平公主也<br />
<br />
  十年春正月劒南西山羌蠻二萬餘戶來降詔加韋臯押近界羌蠻及西山八國使【八國即前女王哥鄰等弱水最弱小不得預八國數】崔佐時至雲南所都羊苴咩城【蜀注苴徐嗟翻咩彌嗟翻詳見前 考異】<br />
<br />
  【曰舊傳作羊苴咩城今從舊傳】吐蕃使者數百人先在其國雲南王異牟尋尚不欲吐蕃知之令佐時衣牂柯服而入【牂柯蠻在昆明東九百里東距辰州二千四百里其南千五百里即交州衣於既翻下同】佐時不可曰我大唐使者豈得衣小夷之服異牟尋不得已夜迎之佐時大宣詔書【大聲以宣詔書】異牟尋恐懼顧左右失色業已歸唐【事已成為業】乃歔欷流涕俯伏受詔【歔音虛欷許既翻又音希】鄭囘密見佐時敎之【鄭囘勸異牟歸唐事見二百三十二卷三年】故佐時盡得其情因勸異牟尋悉斬吐蕃使者去吐蕃所立之號獻其金印【去羌呂翻吐蕃給雲南金印見二百一十一卷玄宗天寶十載】復南詔舊名異牟尋皆從之 【考異曰舊韋臯傳四年正月臯遣判官崔佐時至苴咩城接西南夷事狀四年臯微聞異牟尋之意始因諸蠻寓書於牟尋自是比年招諭至九年牟尋始遣使分臯書以來朝廷賜之詔書臯乃遣佐時齎詔以往牟尋猶欲使佐時易服而入臯傳誤也】仍刻金契以獻異牟尋帥其子尋夢湊等與佐時盟於點蒼山神祠【帥讀曰率】先是吐蕃與囘鶻爭北庭大戰死傷甚衆【爭北庭事見上卷五年六年先悉薦翻】徵兵萬人於雲南異牟尋辭以國小請發三千人吐蕃少之【以為少也少詩紹翻】益至五千乃許之異牟尋遣五千人前行自將數萬人踵其後晝夜兼行襲擊吐蕃戰于神川大破之取鐵橋等十六城【鐵橋在施蠻東南據新書是戰也異牟尋破施順二蠻並虜其王置白崖城又據新志自戎州開邊縣南行至白崖城三千里而近南詔傳曰南詔居永昌姚州之間鐵橋之南將即亮翻】虜其五王降其衆十餘萬戊戍遣使來獻捷【降戶江翻使疏吏翻】 瀛州刺史劉澭為兄濟所逼【濟澭不恊事始上卷八年】請西扞隴坻【坻丁禮翻】遂將部兵千五百人男女萬餘口詣京師號令嚴整在道無一人敢取人雞犬者上嘉之二月丙午以為秦州刺史隴右經略軍使理普潤【理治也以普潤為治所】軍中不撃柝不設音樂士卒病者澭親視之死者哭之 乙丑義成節度使李融薨丁卯以華州刺史李復為義成節度使復齊物之子也【李齊物淮安王神通之孫】復辟河南尉洛陽盧坦為判官監軍薛盈珍數侵軍政【數所角翻】坦每據理以拒之盈珍常曰盧侍御所言公我固不違也【坦後卒能脫於盈珍之譛侍御坦之寄禄官所謂憲街也】 横海節度使程懷直入朝厚賜遣歸 夏四月庚午宣武軍亂留後李萬榮討平之先是宣武親兵三百人素驕横【先悉薦翻横戶孟翻】萬榮惡之【惡烏路翻】遣詣京西防秋親兵怨之大將韓惟清張彦琳誘親兵作亂攻萬榮萬榮擊破之親兵掠而潰多奔宋州宋州刺史劉逸準厚撫之【史言李萬榮不能制劉逸準】惟清奔鄭州彦琳奔東都萬榮悉誅亂者妻子數千人有軍士數人呼於市曰【呼火故翻】今夕兵大至城當破萬榮收斬之奏稱劉士寧所為五月庚子徙士寧於郴州【郴丑林翻】 欽州蠻酋黄少卿反圍州城【少卿者西原黄洞蠻酋也酋兹由翻】邕管經略使孫公器奏請發嶺南兵救之【上元後置邕管經略使領邕貴黨横等州】上不許遣中使諭解之 陸贄上言郊禮赦下已近半年而竄謫者尚未霑恩乃為三狀擬進上使謂之曰故事左降官準赦量移【史炤曰移徙也謂得罪遠謫者遇赦則量徙近地】不過三五百里今所擬稍似超越又多近兵馬及當路州縣【當路州縣謂其地當入京之路者近其靳翻】事恐非便贄復上言【復扶又翻下同】以為王者待人以誠有責怒而無猜嫌有懲沮而無怨忌斥遠以儆其不恪【遠于願翻】甄恕以勉其自新【甄稽延翻察也免也】不儆則浸及威刑不勉而復加黜削雖屢進退俱非愛憎行法乃暫使左遷念材而漸加進叙又知復用誰不增脩何憂乎亂常何患乎蓄憾如或以其貶黜便謂姦凶恒處防閑之中【處昌呂翻】長從擯弃之例則是悔過者無由自補蘊才者終不見伸凡人之情窮則思變含悽貪亂或起于兹【悽悲也痛也】今若所移不過三五百里則有疆域不離於本道風土反惡於舊州【離力智翻風土之同道而獨甚惡者如廣府統廣韶端康封岡新樂瀧竇義雷春高循潮等州而春循新瘴氣特重於諸州是也】徒有徙家之勞寔增移配之擾又當今郡府多有軍兵所在封疆少無舘驛示人疑慮體又非弘乞更賜裁審上性猜忌不委任臣下官無大小必自選而用之宰相進擬少所稱可【稱尺證翻稱愜也下同少詩沼翻】及羣臣一有譴責往往終身不復收用好以辯給取人【好呼到翻下同】不得敦實之士艱於進用羣材滯淹贄上奏諫其略曰夫登進以懋庸【懋勉也庸功也】黜退以懲過二者迭用理如循環進而有過則示懲懲而改脩則復進既不廢法亦無弃人雖纎介必懲而用材不匱故能使黜退者克勵以求復登進者警飭而恪居【恪居謂恪居官次也】上無滯疑下無蓄怨又曰明主不以辭盡人不以意選士如或好善而不擇所用悦言而不驗所行進退隨愛憎之情離合繫異同之趣是由捨繩墨而意裁曲直弃權衡而手揣重輕雖甚精微不能無謬【由與猶同揣初委翻】又曰中人以上迭有所長苟區别得宜【别彼列翻】付授當器【當丁浪翻下過當同】各適其性各宣其能及乎合以成功亦與全才無異但在明鑒大度御之有道而已又曰以一言稱愜為能【愜苦叶翻】而不核虛實以一事違忤為咎而不考忠邪【忤五故翻】其稱愜則付任逾涯不思其所不及其違忤則罪責過當不恕其所不能是以職司之内無成功君臣之際無定分上不聽【分扶問翻】贄又請均節財賦凡六條其一論兩税之弊其略曰舊制賦役之法曰租調庸【調徒弔翻】丁男一人受田百畝歲輸粟二石謂之租每戶各隨土宜出絹若綾若絁共二丈【絁式支翻】綿三兩不蠶之土輸布二丈五尺麻三斤謂之調每丁歲役則收其庸日準絹三尺謂之庸天下為家法制均一雖欲轉徙莫容其姦故人無揺心而事有定制及羯胡亂華【謂安禄山史思明】黎庶雲擾版圖墮於避地【墮讀曰隳】賦法壞於奉軍建中之初再造百度執事者知弊之宜革而所作兼失其原知簡之可從而所操不得其要【操七刀翻執事者謂楊炎】凡欲拯其弊須窮致弊之由時弊則但理其時法弊則全革其法所為必當其悔乃亡【易曰革而當其悔乃亡當丁浪翻】兵興以來供億無度此乃時弊非法弊也而遽更租庸調法【更工衡翻】分遣使者搜擿郡邑【擿他狄翻】校驗簿書每州取大歷中一年科率最多者以為兩税定額【事見二百二十六卷建中元年】夫財之所生必因人力故先王之制賦入必以丁夫為本不以務穡增其税不以輟稼减其租則播種多不以殖產厚其征不以流寓免其調則地著固【著直略翻】不以飭勵重其役不以窳怠蠲其庸則功力勤【窳勇主翻惰也】如是故人安其居盡其力矣兩税之立惟以資產為宗不以丁身為本曾不悟資產之中有藏於襟懷囊篋物雖貴而人莫能窺【謂商賈居寶貨待時而取利者】有積於場圃囷倉直雖輕而衆以為富【謂力田而蓄穀粟者】有流通蕃息之貨數雖寡而計日收贏【蕃讀如繁謂貸子錢而收利者】有廬舍器用之資價雖高而終歲無利【謂美居室侈服用而夸一時者】如此之比其流實繁一槩計估筭緡宜其失平長偽【長知兩偽】由是務輕資而樂轉徙者恒脱於徭税【樂音洛】敦本業而樹居產者每困於徵求此乃誘之為姦驅之避役【誘音酉】力用不得不弛賦入不得不闕復以創制之首【創制之首猶言立法之初復扶又翻】不務齊平供應有煩簡之殊牧守有能否之異所在徭賦輕重相懸所遣使臣意見各異計奏一定有加無除又大歷中供軍進奉之類既收入兩税今於兩税之外復又並存望稍行均减以救凋殘其二請二税以布帛為額不計錢數其略曰凡國之賦税必量人之力【量音良】任土之宜故所入者惟布麻繒纊【繒疾陵翻纊苦謗翻】與百糓而已先王懼物之貴賤失平而人之交易難凖又定泉布之法以節輕重之宜【班固曰太公為周立九府圜法貨寶於金利於刀流於泉布於布束於帛又鄭氏周禮注曰其藏曰泉其行曰布取名於水泉其流行無不徧】歛散弛張必由於是蓋御財之大柄為國之利權守之在官不以任下然則糓帛者人之所為也錢貨者官之所為也是以國朝著令租出穀庸出絹調出繒纊布曷嘗有禁人鑄錢而以錢為賦者哉今之兩税獨異舊章但估資產為差便以錢穀定税臨時折徵雜物【謂折錢穀之價以徵他雜物也折之舌翻】每歲色目頗殊唯計求得之利宜靡論供辦之難易【易以豉翻】所徵非所業所業非所徵遂或增價以買其所無減價以賣其所有一增一減耗損已多望勘會諸州初納兩税年絹布定估比類當今時價加賤減貴酌取其中縂計合税之錢折為布帛之數又曰夫地力之生物有大限取之有度用之有節則常足取之無度用之無節則常不足生物之豐敗由天用物之多少由人是以聖王立程量入為出雖遇災難【難乃旦翻】下無困窮理化既衰則乃反是【理化猶言治化也】量出為入不恤所無桀用天下而不足湯用七十里而有餘是乃用之盈虚在節與不節耳其三論長吏以增戶加税闢田為課績其畧曰長人者罕能推忠恕易地之情體至公徇國之意迭行小惠競誘姦甿【長知丈翻甿謨耕翻】以傾奪鄰境為智能以招萃逋逃為理化捨彼適此者既為新收而有復【萃聚也復方目翻復除也】倏往忽來者又以復業而見優唯懷土安居首末不遷者則使之日重斂之日加【斂力贍翻下同】是令地著之人恒代惰遊賦役何異驅之轉徙教之澆訛【恒戶登翻澆古堯翻】此由牧宰不克弘通各私所部之過也又曰立法齊人久無不弊理之者若不知維御損益之宜則巧偽萌生恒因沮勸而滋矣請申命有司詳定考績若當管之内人益阜殷所定税額有餘任其據戶口均减以减數多少為考課等差其當管稅物通比每戶十分减三者為上課減二者次焉減一者又次焉【此不以增戶為課最而以戶額增之税能減地著戶之税額為課最也】如或人多流亡加税見戶【見賢遍翻】比校殿罰亦如之【殿丁練翻】其四論税限迫促其畧曰建官立國所以養人也賦人取財所以資國也明君不厚其所資而害其所養故必先人事而借其暇力先家給而歛其餘財【先悉薦翻】又曰蠶事方興已輸縑税農功未艾遽斂穀租上司之繩責既嚴下吏之威暴愈促有者急賣而耗其半直無者求假而費其倍酬望更詳定徵税期限其五請以税茶錢置義倉以備水旱【税茶見上九年】其畧曰古稱九年六年之蓄者【記王制曰三年耕必有一年之食九年耕必有三年之食以三十年之通制國用量入以為出國無九年之蓄曰不足無六年之蓄曰急無三年之蓄曰國非其國也】率土臣庶通為之計耳固非獨豐公庾不及編甿也近者有司奏請稅茶歲約得五十萬貫元敕令貯戶部用救百姓凶饑今以蓄糧適副前旨其六論兼并之家私斂重於公稅其畧曰今亰畿之内每田一畝官稅五升而私家收租殆有畝至一石者是二十倍於官稅也降及中等租猶半之夫土地王者之所有耕稼農夫之所為而兼并之徒居然受利又曰望凡所占田【占之贍翻】約為條限裁減租價務利貧人法貴必行慎在深刻裕其制以便俗嚴其令以懲違微損有餘稍優不足失不損富優可賑窮此乃安富恤窮之善經不可捨也【周禮地官以保息六養萬民一曰慈幼二曰養老三曰振窮四曰恤貧五曰寛疾六曰安富】<br />
<br />
  資治通鑑卷二百三十四<br />
<br />
<史部,編年類,資治通鑑>  <br>
   </div> 

<script src="/search/ajaxskft.js"> </script>
 <div class="clear"></div>
<br>
<br>
 <!-- a.d-->

 <!--
<div class="info_share">
</div> 
-->
 <!--info_share--></div>   <!-- end info_content-->
  </div> <!-- end l-->

<div class="r">   <!--r-->



<div class="sidebar"  style="margin-bottom:2px;">

 
<div class="sidebar_title">工具类大全</div>
<div class="sidebar_info">
<strong><a href="http://www.guoxuedashi.com/lsditu/" target="_blank">历史地图</a></strong>  
<a href="http://www.880114.com/" target="_blank">英语宝典</a>  
<a href="http://www.guoxuedashi.com/13jing/" target="_blank">十三经检索</a> 
<br><strong><a href="http://www.guoxuedashi.com/gjtsjc/" target="_blank">古今图书集成</a></strong> 
<a href="http://www.guoxuedashi.com/duilian/" target="_blank">对联大全</a> <strong><a href="http://www.guoxuedashi.com/xiangxingzi/" target="_blank">象形文字典</a></strong> 

<br><a href="http://www.guoxuedashi.com/zixing/yanbian/">字形演变</a>  <strong><a href="http://www.guoxuemi.com/hafo/" target="_blank">哈佛燕京中文善本特藏</a></strong>
<br><strong><a href="http://www.guoxuedashi.com/csfz/" target="_blank">丛书&方志检索器</a></strong> <a href="http://www.guoxuedashi.com/yqjyy/" target="_blank">一切经音义</a>  

<br><strong><a href="http://www.guoxuedashi.com/jiapu/" target="_blank">家谱族谱查询</a></strong>  <strong><a href="http://shufa.guoxuedashi.com/sfzitie/" target="_blank">书法字帖欣赏</a></strong> 
<br>

</div>
</div>


<div class="sidebar" style="margin-bottom:0px;">

<font style="font-size:22px;line-height:32px">QQ交流群9:489193090</font>


<div class="sidebar_title">手机APP 扫描或点击</div>
<div class="sidebar_info">
<table>
<tr>
	<td width=160><a href="http://m.guoxuedashi.com/app/" target="_blank"><img src="/img/gxds-sj.png" width="140"  border="0" alt="国学大师手机版"></a></td>
	<td>
<a href="http://www.guoxuedashi.com/download/" target="_blank">app软件下载专区</a><br>
<a href="http://www.guoxuedashi.com/download/gxds.php" target="_blank">《国学大师》下载</a><br>
<a href="http://www.guoxuedashi.com/download/kxzd.php" target="_blank">《汉字宝典》下载</a><br>
<a href="http://www.guoxuedashi.com/download/scqbd.php" target="_blank">《诗词曲宝典》下载</a><br>
<a href="http://www.guoxuedashi.com/SiKuQuanShu/skqs.php" target="_blank">《四库全书》下载</a><br>
</td>
</tr>
</table>

</div>
</div>


<div class="sidebar2">
<center>


</center>
</div>

<div class="sidebar"  style="margin-bottom:2px;">
<div class="sidebar_title">网站使用教程</div>
<div class="sidebar_info">
<a href="http://www.guoxuedashi.com/help/gjsearch.php" target="_blank">如何在国学大师网下载古籍?</a><br>
<a href="http://www.guoxuedashi.com/zidian/bujian/bjjc.php" target="_blank">如何使用部件查字法快速查字?</a><br>
<a href="http://www.guoxuedashi.com/search/sjc.php" target="_blank">如何在指定的书籍中全文检索?</a><br>
<a href="http://www.guoxuedashi.com/search/skjc.php" target="_blank">如何找到一句话在《四库全书》哪一页?</a><br>
</div>
</div>


<div class="sidebar">
<div class="sidebar_title">热门书籍</div>
<div class="sidebar_info">
<a href="/so.php?sokey=%E8%B5%84%E6%B2%BB%E9%80%9A%E9%89%B4&kt=1">资治通鉴</a> <a href="/24shi/"><strong>二十四史</strong></a>&nbsp; <a href="/a2694/">野史</a>&nbsp; <a href="/SiKuQuanShu/"><strong>四库全书</strong></a>&nbsp;<a href="http://www.guoxuedashi.com/SiKuQuanShu/fanti/">繁体</a>
<br><a href="/so.php?sokey=%E7%BA%A2%E6%A5%BC%E6%A2%A6&kt=1">红楼梦</a> <a href="/a/1858x/">三国演义</a> <a href="/a/1038k/">水浒传</a> <a href="/a/1046t/">西游记</a> <a href="/a/1914o/">封神演义</a>
<br>
<a href="http://www.guoxuedashi.com/so.php?sokeygx=%E4%B8%87%E6%9C%89%E6%96%87%E5%BA%93&submit=&kt=1">万有文库</a> <a href="/a/780t/">古文观止</a> <a href="/a/1024l/">文心雕龙</a> <a href="/a/1704n/">全唐诗</a> <a href="/a/1705h/">全宋词</a>
<br><a href="http://www.guoxuedashi.com/so.php?sokeygx=%E7%99%BE%E8%A1%B2%E6%9C%AC%E4%BA%8C%E5%8D%81%E5%9B%9B%E5%8F%B2&submit=&kt=1"><strong>百衲本二十四史</strong></a>  <a href="http://www.guoxuedashi.com/so.php?sokeygx=%E5%8F%A4%E4%BB%8A%E5%9B%BE%E4%B9%A6%E9%9B%86%E6%88%90&submit=&kt=1"><strong>古今图书集成</strong></a>
<br>

<a href="http://www.guoxuedashi.com/so.php?sokeygx=%E4%B8%9B%E4%B9%A6%E9%9B%86%E6%88%90&submit=&kt=1">丛书集成</a> 
<a href="http://www.guoxuedashi.com/so.php?sokeygx=%E5%9B%9B%E9%83%A8%E4%B8%9B%E5%88%8A&submit=&kt=1"><strong>四部丛刊</strong></a>  
<a href="http://www.guoxuedashi.com/so.php?sokeygx=%E8%AF%B4%E6%96%87%E8%A7%A3%E5%AD%97&submit=&kt=1">說文解字</a> <a href="http://www.guoxuedashi.com/so.php?sokeygx=%E5%85%A8%E4%B8%8A%E5%8F%A4&submit=&kt=1">三国六朝文</a>
<br><a href="http://www.guoxuedashi.com/so.php?sokeytm=%E6%97%A5%E6%9C%AC%E5%86%85%E9%98%81%E6%96%87%E5%BA%93&submit=&kt=1"><strong>日本内阁文库</strong></a> <a href="http://www.guoxuedashi.com/so.php?sokeytm=%E5%9B%BD%E5%9B%BE%E6%96%B9%E5%BF%97%E5%90%88%E9%9B%86&ka=100&submit=">国图方志合集</a> <a href="http://www.guoxuedashi.com/so.php?sokeytm=%E5%90%84%E5%9C%B0%E6%96%B9%E5%BF%97&submit=&kt=1"><strong>各地方志</strong></a>

</div>
</div>


<div class="sidebar2">
<center>

</center>
</div>
<div class="sidebar greenbar">
<div class="sidebar_title green">四库全书</div>
<div class="sidebar_info">

《四库全书》是中国古代最大的丛书,编撰于乾隆年间,由纪昀等360多位高官、学者编撰,3800多人抄写,费时十三年编成。丛书分经、史、子、集四部,故名四库。共有3500多种书,7.9万卷,3.6万册,约8亿字,基本上囊括了古代所有图书,故称“全书”。<a href="http://www.guoxuedashi.com/SiKuQuanShu/">详细>>
</a>

</div> 
</div>

</div>  <!--end r-->

</div>
<!-- 内容区END --> 

<!-- 页脚开始 -->
<div class="shh">

</div>

<div class="w1180" style="margin-top:8px;">
<center><script src="http://www.guoxuedashi.com/img/plus.php?id=3"></script></center>
</div>
<div class="w1180 foot">
<a href="/b/thanks.php">特别致谢</a> | <a href="javascript:window.external.AddFavorite(document.location.href,document.title);">收藏本站</a> | <a href="#">欢迎投稿</a> | <a href="http://www.guoxuedashi.com/forum/">意见建议</a> | <a href="http://www.guoxuemi.com/">国学迷</a> | <a href="http://www.shuowen.net/">说文网</a><script language="javascript" type="text/javascript" src="https://js.users.51.la/17753172.js"></script><br />
  Copyright &copy; 国学大师 古典图书集成 All Rights Reserved.<br>
  
  <span style="font-size:14px">免责声明:本站非营利性站点,以方便网友为主,仅供学习研究。<br>内容由热心网友提供和网上收集,不保留版权。若侵犯了您的权益,来信即刪。scp168@qq.com</span>
  <br />
ICP证:<a href="http://www.beian.miit.gov.cn/" target="_blank">鲁ICP备19060063号</a></div>
<!-- 页脚END --> 
<script src="http://www.guoxuedashi.com/img/plus.php?id=22"></script>
<script src="http://www.guoxuedashi.com/img/tongji.js"></script>

</body>
</html>
