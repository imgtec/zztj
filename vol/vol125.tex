資治通鑑卷一百二十五 宋 司馬光 撰

胡三省 音註

宋紀七|{
	起彊圉大淵獻盡上章攝提格凡四年}


太祖文皇帝中之下

元嘉二十四年春正月甲戌大赦 魏吐京胡及山胡曹僕渾等反二月征東將軍武昌王提等討平之 癸未魏主如中山 魏師之克敦煌也|{
	敦煌當作姑臧事見一百二十三卷十六年}
沮渠牧犍使人斫開府庫|{
	沮子余翻犍居言翻}
取金玉及寶器因不復閉|{
	復扶又翻下同}
小民爭入盗取之有司索盜不獲|{
	索山客翻下同}
至是牧犍所親及守藏者告之|{
	藏徂浪翻}
且言牧犍父子多蓄毒藥潜殺人前後以百數况復姊妹皆學左道|{
	謂學曇無䜟之術也}
有司索牧犍家得所匿物魏主大怒賜沮渠昭儀死并誅其宗族唯沮渠祖以先降得免|{
	祖降亦見十六年}
又有告牧犍猶與故臣民交通謀反者三月魏主遣崔浩就第賜牧犍死諡曰哀王 魏人徙定州丁零三千家於平城 六月魏西征諸將|{
	西征謂討蓋吳之將也將即亮翻}
扶風公處真等八人|{
	處昌呂翻}
坐盜没軍資及虜掠各千萬計並斬之 初上以貨重物輕改鑄四銖錢|{
	元嘉七年鑄四銖錢見一百二十一卷}
民多翦鑿古錢取銅盜鑄上患之録尚書事江夏王義恭建議請以大錢一當兩|{
	夏戶雅翻}
右僕射何尚之議曰夫泉貝之興以估貨為本|{
	估音古}
事存交易豈假多鑄數少則幣重|{
	少詩沼翻下同}
數多則物重多少雖異濟用不殊况復以一當兩徒崇虛價者邪|{
	復扶又翻}
若今制遂行富人之貲自倍貧者彌增其困懼非所以使之均壹也上卒從義恭議|{
	卒子恤翻下同}
秋八月乙未徐州刺史衡陽文王義季卒義季自彭城王義康之貶|{
	義康貶見一百二十三卷十七年}
遂縱酒不事事帝以書誚責且戒之|{
	誚才笑翻}
義季猶酣飲自若以至成疾而終 魏樂安宣王範卒 冬十月壬午胡藩之子誕世殺豫章太守桓隆之據郡反|{
	胡藩家于豫章}
欲奉前彭城王義康為主前交州刺史檀和之去官歸過豫章擊斬之|{
	過工禾翻}
十一月甲寅封皇子渾為汝隂王 十二月魏晉王伏羅卒 |{
	考異曰宋索虜傳曰燾所住屠蘇為疾雷所擊屠蘇倒見壓殆死左右皆號泣晉王獨不悲燾怒賜死此出于傳聞今從後魏書}
楊文德據葭蘆城|{
	水經注羌水出隴西羌道東南流逕宕昌城東西北去仇池五百餘里又東逕葭蘆城西}
招誘氐羌武都等五郡氐皆附之|{
	魏取仇池置武都天水漢陽武階仇池五郡誘音酉}


二十五年春正月魏仇池鎮將皮豹子帥諸軍擊之|{
	將即亮翻帥讀曰率}
文德兵敗棄城奔漢中豹子收其妻子僚屬軍資及楊保宗所尚魏公主而還|{
	還從宣翻又如字}
初保宗將叛|{
	保宗叛魏見上卷二十年}
公主勸之或曰奈何叛父母之國公主曰事成為一國之母豈比小縣公主哉魏主賜之死楊文德坐失守免官削爵土|{
	宋免削之也}
二月癸卯魏主如定州罷塞圍役者|{
	築塞圍見上卷二十三年}
遂如上黨誅潞縣叛民二千餘家徙河西離石民五千餘家于平城|{
	河西當作西河}
閏月己酉帝大蒐于宣武場|{
	建康倣洛都之制築宣武場於臺城北}
初劉湛既誅|{
	湛誅見一百二十三卷十七年}
庾炳之遂見寵任累遷吏部尚書勢傾朝野炳之無文學性彊急輕淺既居選部好詬詈賓客且多納貨賂士大夫皆惡之|{
	選須絹翻好呼報翻惡烏路翻}
炳之留令史二人宿於私宅|{
	尚書令史掌省中文案不當宿尚書私家}
為有司所糾上薄其過欲不問僕射何尚之因極陳炳之之短曰炳之見人有燭盤佳驢無不乞匄選用不平不可一二|{
	言其罪不可一二數也}
交結朋黨構扇是非亂俗傷風過於范曄所少賊一事耳|{
	言所少者唯不至如范曄作賊一事少詩沼翻}
縱不加罪故宜出之上欲以炳之為丹楊尹尚之曰炳之蹈罪負恩方復有尹京赫赫之授|{
	復扶又翻引用詩赫赫師尹以諭京尹然詩所謂師尹者乃太師尹氏也}
乃更成其形勢也古人云無賞無罰雖堯舜不能為治|{
	漢宣帝詔曰有功不賞有罪不誅雖唐虞不能以化治直吏翻}
臣昔啟范曄|{
	事見一百二十三卷十七年}
亦懼犯顔苟白愚懷九死不悔|{
	言苟愚懷所欲吐者雖冒九死猶將言之而不悔}
歷觀古今未有衆過藉藉|{
	藉秦昔翻}
受貨數百萬更得高官厚禄如炳之者也上乃免炳之官以徐湛之為丹陽尹 彭城太守王玄謨上言彭城要兼水陸|{
	魏人南寇水行自清入泗陸行自歷城瑕丘皆湊彭城故云要兼水陸}
請以皇子撫臨州事夏四月乙卯以武陵王駿為安北將軍徐州刺史五月甲戌魏以交阯公韓拔為鄯善王|{
	魏書官氏志内入諸姓出大汗氏改為韓氏鄯上扇翻}
鎮鄯善賦役其民比之郡縣 當兩大錢行之經時公私不以為便己卯罷之 六月丙寅荆州刺史南譙王義宣進位司空 辛酉魏主如廣德宫|{
	魏主起殿於隂山比殿成而楊難當來朝因命曰廣德宫}
秋八月甲子封皇子彧為淮陽王|{
	彧於六翻}
西域般悦國去平城萬有餘里|{
	據北史般悦當作悦般般音鉢}
遣使詣魏|{
	使疏吏翻}
請與魏東西合擊柔然魏主許之中外戒嚴 九月辛未以尚書右僕射何尚之為左僕射領軍將軍沈演之為吏部尚書 丙戌魏主如隂山 魏成周公萬度歸擊焉耆大破之焉耆王鳩尸卑那奔龜兹|{
	龜兹音丘慈}
魏主詔唐和與前部王車伊洛帥所部兵會度歸討西域|{
	車伊洛車師大帥也世附於魏魏封為前部王帥讀曰率}
和說降柳驢等六城|{
	說輸芮翻}
因共擊波居羅城拔之 冬十月辛丑魏弘農昭王奚斤卒子它觀襲魏主曰斤關西之敗|{
	事見一百二十一卷五年}
罪固當死朕以斤佐命先朝|{
	朝直遥翻}
復其爵邑使得終天年君臣之分亦足矣|{
	分扶問翻}
乃降它觀爵為公 癸亥魏大赦 十二月魏萬度歸自焉耆西討龜兹留唐和鎮焉耆柳驢戍主乙直伽謀叛|{
	伽求迦翻}
和擊斬之由是諸胡咸服西域復平|{
	復扶又翻下復伐同}
魏太子朝于行宫|{
	隂山行宫也朝直遥翻}
遂從伐柔然至受降城|{
	即漢武帝所築受降城降戶江翻}
不見柔然因積糧於城内置戍而還|{
	還從宣翻又如字}
二十六年春正月戊辰朔魏主饗群臣於漠南甲戌復伐柔然高涼王那出東道略陽王羯兒出西道|{
	羯居謁翻}
魏主與太子出涿邪山|{
	邪讀曰耶}
行數千里柔然處羅可汗恐懼遠遁|{
	處昌呂翻可從刋入聲汗音寒}
二月己亥上如丹徒謁京陵三月丁巳大赦募諸州樂移者數千家以實京口|{
	樂音洛}
庚寅魏主還平城 夏五月壬午帝還建康 庚寅

魏主如隂山 帝欲經略中原羣臣爭獻策以迎合取寵彭城太守王玄謨尤好進言|{
	守手又翻好呼到翻}
帝謂侍臣曰觀玄謨所陳令人有封狼居胥意|{
	漢霍去病伐匈奴封狼居胥禪于姑衍以臨瀚海}
御史中丞袁淑言於上曰陛下今當席卷趙魏|{
	卷讀曰捲}
檢玉岱宗|{
	封泰山用玉檢}
臣逢千載之會願上封禪書|{
	載子亥翻上時掌翻}
上悦淑耽之曾孫也|{
	袁耽見晉成帝紀}
秋七月辛未以廣陵王誕為雍州刺史|{
	雍於用翻}
上以襄陽外接關河欲廣其資力乃罷江州軍府文武悉配雍州|{
	沈約曰晉孝武始于襄陽立雍州并立僑郡縣至是割荆州之襄陽南陽新野順陽随五郡為雍州而僑郡縣猶寄寓在諸郡界}
湘州入臺租税悉給襄陽 九月魏主伐柔然高涼王那出東道略陽王羯兒出中道柔然處羅可汗悉國内精兵圍那數十重那掘塹堅守處羅數挑戰輒為那所敗|{
	重直龍翻掘其月翻塹七艷翻數所角翻挑徒了翻敗補邁翻}
以那衆少而堅|{
	少詩沼翻}
疑大軍將至解圍夜去那引兵追之九日九夜處羅益懼棄輜重踰穹隆嶺遠遁那收其輜重|{
	重直用翻}
引軍還與魏主會於廣澤略陽王羯兒收柔然民畜凡百餘萬自是柔然衰弱屏跡不敢犯魏塞|{
	屏必郢翻}
冬十二月戊申魏主還平城 沔北諸山蠻寇雍州建威將軍沈慶之帥後軍中兵參軍柳元景随郡太守宗慤等二萬人討之|{
	帥讀曰率}
八道俱進先是諸將討蠻者皆營於山下以迫之蠻得據山矢石以擊官軍多不利|{
	乘高臨下矢石之勢所及過於平原相遇者故軍多不利先悉薦翻}
慶之曰去歲蠻田大稔積穀重巖|{
	重直龍翻}
不可與之曠日相守也不若出其不意衝其腹心破之必矣乃命諸軍斬木登山鼓譟而前羣蠻震恐因其恐而擊之所向奔潰|{
	斬木登山八道並進蠻救首救尾之不暇故震恐而奔潰若一道而進蠻聚兵據險拒戰雖欲斬木而登山庸可得乎}


二十七年春正月乙酉魏主如洛陽 沈慶之自冬至春屢破雍州蠻因蠻所聚穀以充軍食前後斬首三千級虜二萬八千餘口降者二萬五千餘戶|{
	降下江翻}
幸諸山大羊蠻憑險築城守禦甚固慶之擊之命諸軍連營於山中開門相通各穿池於營内朝夕不外汲頃之風甚蠻潜兵夜來燒營諸軍以池水沃火多出弓弩夾射之|{
	射而亦翻}
蠻兵散走蠻所據險固不可攻慶之乃置六戍以守之久之蠻食盡稍稍請降悉遷於建康以為營戶|{
	史言沈慶之又能持久以弊諸蠻降戶江翻}
魏主將入寇二月甲午大獵於梁川|{
	梁川後魏天平二年置梁城郡於其地領參合旋鴻二縣}
帝聞之勑淮泗諸郡若魏寇小至則各堅守大至則拔民歸壽陽邉戍偵候不明|{
	偵且鄭翻}
辛亥魏主自將步騎十萬奄至|{
	將即亮翻騎奇寄翻 考異曰宋書是月辛丑南平王鑠進號西平辛巳索虜寇汝南按長歷二月壬辰朔十日辛丑二十日辛亥巳當作亥}
南頓太守鄭琨|{
	南頓縣本屬汝南晉惠帝分置南頓郡}
潁川太守鄭道隱並棄城走是時豫州刺史南平王鑠鎮壽陽遣將軍行參軍陳憲行汝南郡事守懸瓠|{
	水經注汝水自汝南上蔡縣東逕懸瓠城北今豫州刺史汝南郡治汝水枝别左出西北流又屈西東轉又西南會汝形若懸瓠故以名城瓠戶故翻又音乎}
城中戰士不滿千人魏主圍之三月以軍興减内外百官俸三分之一魏人晝夜攻懸瓠多作高樓臨城以射之|{
	射而亦翻}
矢下如雨城中負戶以汲施大鉤於衝車之端以牽樓堞壞其南城|{
	堞達叶翻壞音怪}
陳憲内設女墻外立木柵以拒之魏人填塹肉薄登城|{
	薄迫也塹七豔翻}
憲督厲將士苦戰|{
	將即亮翻}
積屍與城等魏人乘屍上城短兵相接憲鋭氣愈奮戰士無不一當百殺傷萬計城中死者亦過半魏主遣永昌王仁將步騎萬餘驅所掠六郡生口北屯汝陽|{
	汝陽縣本屬汝南郡江左分立汝陽郡}
時徐州刺史武陵王駿鎮彭城帝遣間使命駿騎齎三日糧襲之|{
	間古莧翻使疏吏翻騎奇寄翻下同}
駿百里内馬得千五百匹分為五軍遣參軍劉泰之 |{
	考異曰後魏紀作劉坦之今從宋書}
帥安北騎兵行參軍垣謙之田曹行參軍臧肇之集曹行參軍尹定|{
	田曹主營田集曹主安集流散猶漢之安集掾也時駿為安北將軍謙之等皆府僚也}
武陵左常侍杜幼文|{
	晉制王國置左右常侍各一人}
殿中將軍程天祚等將之|{
	將即亮翻}
直趨汝陽|{
	趨七喻翻}
魏人唯慮救兵自壽陽來不備彭城丁酉泰之等潜進擊之殺三千餘人燒其輜重|{
	重直用翻}
魏人奔散諸生口悉得東走魏人偵知泰之等兵無繼|{
	偵丑鄭翻}
復引兵擊之|{
	復扶又翻}
垣謙之先退士卒驚亂棄仗走泰之為魏人所殺肇之溺死天祚為魏所擒謙之定幼文及士卒免者九百餘人馬還者四百匹魏主攻懸瓠四十二日帝遣南平内史臧質詣壽陽與安蠻司馬劉康祖共將兵救懸瓠|{
	時南平王鑠領安蠻校尉以康祖為司馬}
魏主遣殿中尚書任城公乞地真將兵逆拒之|{
	魏殿中尚書知殿内兵馬倉庫任音壬}
質等擊斬乞地真康祖道錫之從兄也|{
	劉道錫見一百二十三卷十八年從才用翻}
夏四月魏主引兵還|{
	還從宣翻又如字}
癸卯至平城壬子安北將軍武陵王駿降號鎮軍將軍垣謙之伏誅尹定杜幼文付上方|{
	輸作尚方也}
以陳憲為龍驤將軍汝南新蔡二郡太守|{
	驤思將翻}
魏主遺帝書曰前蓋吳反逆扇動關隴彼復使人就而誘之丈夫遺以弓矢婦人遺以環釧|{
	復扶又翻誘音酉遺于季翻通使蓋吳事見上卷二十二年二十三年釧尺絹翻臂環也}
是曹正欲譎誑取賂|{
	譎古穴翻誑居况翻}
豈有遠相服從之理為大丈夫何不自來取之而以貨誘我邉民募往者復除七年是賞姦也|{
	復方目翻}
我今來至此土所得多少孰與彼前後得我民邪彼若欲存劉氏血食者當割江以北輸之攝守南度|{
	攝收也言收江北守兵南度江也}
當釋江南使彼居之不然可善勑方鎮刺史守宰嚴供帳之具|{
	守式又翻帳當作張音竹亮翻}
來秋當往取揚州大勢已至終不相縱彼往日北通蠕蠕西結赫連沮渠吐谷渾東連馮弘高麗|{
	事並見前蠕人兖翻沮子余翻吐從暾入聲谷音浴麗力知翻}
凡此數國我皆滅之以此而觀彼豈能獨立蠕蠕吳提吐賀真皆已死我今北征先除有足之寇|{
	柔然多馬故言其有足}
彼若不從命來秋當復往取之|{
	復扶又翻下復縱復非同}
以彼無足故不先討耳我往之日彼作何計為掘塹自守為築垣以自障也|{
	塹七艷翻}
我當顯然往取揚州不若彼翳行竊步也|{
	翳於計翻蔽也言隱蔽其身而行也}
彼來偵諜我已擒之復縱還其人目所盡見委曲善問之|{
	偵丑鄭翻}
彼前使裴方明取仇池既得之疾其勇功已不能容有臣如此尚殺之|{
	事見上卷二十年}
烏得與我校邪彼非我敵也彼常欲與我一交戰我亦不癡復非苻堅何時與彼交戰|{
	觀此魏人猶有憚南兵之心蓋高祖之餘威而邉垂諸將猶為有人也}
晝則遣騎圍繞夜則離彼百里外宿|{
	騎奇計翻離力智翻}
吳人正有斫營伎|{
	伎渠綺翻}
彼募人以來不過行五十里天已明矣彼募人之首豈得不為我有哉彼公時舊臣雖老猶有智策知今已殺盡|{
	謂謝晦檀道濟輩}
豈非天資我邪取彼亦不須我兵刃此有善呪婆羅門|{
	天竺國有婆羅門善呪術}
當使鬼縛以來耳 侍中左衛將軍江湛遷吏部尚書湛性公廉與僕射徐湛之並為主上所寵信時稱江徐 魏司徒崔浩自恃才略及魏主所寵任專制朝權嘗薦冀定相幽并五州之士數十人皆起家為郡守|{
	朝直遥翻相息亮翻守式又翻}
太子晃曰先徵之人亦州郡之選也|{
	先徵之人謂游雅李靈高允等}
在職已久勤勞未答宜先補郡縣以新徵者代為郎吏且守令治民宜得更事者|{
	守手又翻治直之翻更工衡翻}
浩固爭而遣之中書侍郎領著作郎高允聞之謂東宫博士管恬曰崔公其不免乎苟遂其非而校勝於上將何以堪之魏主以浩監祕書事|{
	監工衘翻}
使與高允等共譔國記|{
	譔雛免翻譔述也}
曰務從實録著作令史閔湛郗標|{
	郗丑之翻}
性巧佞為浩所寵信浩嘗註易及論語詩書湛標上疏言馬鄭王賈不如浩之精微|{
	馬融鄭玄王肅賈逵也}
乞收境内諸書班浩所註令天下習業|{
	令習肄浩所注經以為家業}
并求勑浩註禮傳|{
	傳直戀翻}
令後生得觀正義浩亦薦湛標有著述才湛標又勸浩刋所譔國史于石以彰直筆高允聞之謂著作郎宗欽曰湛標所營分寸之間恐為崔門萬世之禍吾徒亦無噍類矣|{
	噍才笑翻}
浩竟用湛標議刋石立於郊壇東方百步|{
	據水經注平城西郭外有郊天壇}
用功三百萬浩書魏之先世事皆詳實列於衢路往來見者咸以為言北人無不忿恚|{
	北人謂其先世從拓跋氏來自北荒者恚於避翻}
相與譛浩於帝以為暴揚國惡帝大怒使有司案浩及祕書郎吏等罪狀初遼東公翟黑子有寵於帝奉使并州|{
	使疏吏翻}
受布千匹事覺黑子謀於高允曰主上問我當以實告為當諱之允曰公帷幄寵臣有罪首實|{
	首式救翻}
庶或見原|{
	原赦也}
不可重為欺罔也|{
	重直用翻}
中書侍郎崔覽公孫質曰若首實罪不可測不如諱之黑子怨允曰君奈何誘人就死地入見帝不以實對帝怒殺之|{
	誘音酉見賢遍翻}
帝使允授太子經及崔浩被收|{
	被皮義翻}
太子召允至東宫因留宿明旦與俱入朝|{
	朝直遥翻}
至宫門謂允曰入見至尊吾自導卿脱至尊有問但依吾語允曰為何等事也|{
	為于偽翻}
太子曰入自知之太子見帝言高允小心慎密且微賤制由崔浩請赦其死帝召允問曰國書皆浩所為乎對曰太祖記前著作郎鄧淵所為先帝記及今記臣與浩共為之然浩所領事多總裁而已|{
	謂摠其大綱裁其可否也}
至於著述臣多於浩帝怒曰允罪甚於浩何以得生太子懼曰天威嚴重允小臣迷亂失次耳臣曏問皆云浩所為帝問允信如東宫所言乎對曰臣罪當滅族不敢虚妄殿下以臣侍講日久哀臣欲匄其生耳|{
	匄古大翻}
實不問臣臣亦無此言不敢迷亂帝顧太子曰直哉此人情所難而允能為之臨死不易辭信也為臣不欺君貞也宜特除其罪以旌之遂赦之於是召浩前臨詰之|{
	詰去吉翻}
浩惶惑不能對允事事申明皆有條理帝命允為詔誅浩及僚屬宗欽段承根等下至僮吏凡百二十八人皆夷五族允持疑不為帝頻使催切允乞更一見然後為詔帝引使前允曰浩之所坐若更有餘舋|{
	舋許覲翻}
非臣敢知若直以觸犯罪不至死|{
	觸犯謂直書國惡不為尊者諱也}
帝怒命武士執允太子為之拜請|{
	為于偽翻下欲為同}
帝意解乃曰無斯人當有數千口死矣六月己亥詔誅清河崔氏與浩同宗者無遠近及浩姻家范陽盧氏太原郭氏河東柳氏並夷其族|{
	浩所連姻皆士望也非冇憑附屬請之罪以浩故皆赤其族擇耦可不謹哉}
餘皆止誅其身縶浩置檻内送城南|{
	檻檻車也後魏刑人必於城南縶陟立翻}
衛士數十人溲其上|{
	溲所鳩翻小便也}
呼聲嗷嗷|{
	嗷五刀翻}
聞於行路|{
	聞音問}
宗欽臨刑歎曰高允其殆聖乎它日太子讓允曰人亦當知幾吾欲為卿脱死既開端緒而卿終不從激怒帝如此每念之使人心悸|{
	悸其季翻}
允曰夫史者所以記人主善惡為將來勸戒故人主有所畏忌慎其舉措崔浩孤負聖恩以私欲没其廉潔愛憎蔽其公直此浩之責也至於書朝廷起居言國家得失此為史之大體未為多違|{
	允言浩死非其罪}
臣與浩實同其事死生榮辱義無獨殊誠荷殿下再造之慈|{
	荷下可翻}
違心苟免非臣所願也太子動容稱歎允退謂人曰我不奉東宫指導者恐負翟黑子故也初冀州刺史崔賾武城男崔模與浩同宗而别族|{
	賾士革翻别分也依宋祁國語補音彼列翻}
浩常輕侮之由是不睦及浩誅二家獨得免賾逞之子也|{
	崔逞歸魏為大祖所殺}
辛丑魏主北巡隂山魏主既誅崔浩而悔之會北部尚書李孝伯病篤|{
	魏北部尚書知北邉州郡}
或傳已卒魏主悼之曰李宣城可惜|{
	李孝伯封宣城公}
既而曰朕失言崔司徒可惜李宣城可哀孝伯順之從父弟也|{
	李順亦為魏主所寵任得罪而死從才用翻}
自浩之誅軍國謀議皆出孝伯寵眷亞於浩 初車師大帥車伊洛世服於魏|{
	帥所類翻}
魏拜伊洛平西將軍封前部王伊洛將入朝沮渠無諱斷其路|{
	沮渠無諱時屯高昌朝直遙翻斷丁管翻}
伊洛屢與無諱戰破之無諱卒|{
	卒於元嘉二十一年}
弟安周奪其子乾壽兵伊洛遣人說乾壽乾壽遂帥其民五百餘家奔魏|{
	帥讀曰率}
伊洛又說李寶弟欽等五十餘人下之皆送于魏|{
	說輸芮翻}
伊洛西擊焉耆留其子歇守城沮渠安周引柔然兵間道襲之|{
	間古莧翻}
攻拔其城歇走就伊洛共收餘衆保焉耆鎮|{
	魏破焉耆以為鎮}
遣使上書於魏主言為沮渠氏所攻首尾八年|{
	元嘉十九年無諱襲據高昌自此與車師相攻使疏吏翻}
百姓飢窮無以自存臣今棄國出奔得免者僅三分之一已至焉耆東境乞垂賑救魏主詔開焉耆倉以賑之|{
	賑諱忍翻}
吐谷渾王慕利延為魏所逼上表求入保越巂|{
	唐時吐蕃與雲南窺蜀即此路也蓋自漢武帝開昆明之後後人遂通此路耳巂音髄}
上許之慕利延竟不至上欲伐魏丹陽尹徐湛之吏部尚書江湛彭城太守

王玄謨等並勸之左軍將軍劉康祖以為歲月已晚請待明年上曰北方苦虜虐政義徒並起頓兵一周沮向義之心不可|{
	沮在呂翻}
太子步兵校尉沈慶之諫曰|{
	時東宫置兵與羽林等故亦有步兵校尉南史曰高祖永初二年置東宫屯騎步兵翊軍三校尉}
我步彼騎其勢不敵|{
	騎奇計翻}
檀道濟再行無功|{
	營陽王景平二年道濟出師元嘉七年至濟上皆無功而還}
到彦之失利而返|{
	見一百二十一卷七年}
今料王玄謨等未踰兩將|{
	將即亮翻}
六軍之盛不過往時恐重辱王師|{
	重直用翻}
上曰王師再屈别自有由道濟養寇自資彦之中塗疾動|{
	謂彦之目疾大動也}
虜所恃者唯馬今夏水浩汗河道流通汎舟北下碻磝必走滑臺小戍易可覆拔|{
	易以䜴翻}
克此二城館穀弔民|{
	館穀就食敵人所積之穀}
虎牢洛陽自然不固比及冬初|{
	比必利翻}
城守相接虜馬過河即成擒也慶之又固陳不可上使徐湛之江湛難之|{
	難乃旦翻}
慶之曰治國譬如治家耕當問奴織當訪婢|{
	治直之翻}
陛下今欲伐國而與白面書生輩謀之事何由濟上大笑太子劭及護軍將軍蕭思話亦諫上皆不從魏主聞上將北伐復與上書曰彼此和好日久而彼志無厭|{
	復扶又翻和好呼到翻厭於鹽翻}
誘我邉民|{
	誘音酉}
今春南廵聊省我民|{
	省悉井翻}
驅之使還今聞彼欲自來設能至中山及桑乾川|{
	乾音干}
随意而行來亦不迎去亦不送若厭其區宇者可來平城居我亦往揚州相與易地彼年已五十未嘗出戶雖自力而來如三歲嬰兒與我鮮卑生長馬上者果如何哉|{
	觀魏主與帝二書誠有憚江南之心大明以後北不復憚南矣長知兩翻}
更無餘物可以相與今送獵馬十二匹并氈藥等物彼來道遠馬力不足可乘或不服水土藥可自療也秋七月庚午詔曰虜近雖摧挫|{
	謂攻懸瓠不克而退也}
獸心靡革比得河朔秦雍華戎表疏|{
	比毗寐翻近也雍於用翻}
歸訴困棘|{
	棘急也}
跂望綏拯|{
	跂丘弭翻又去智翻舉踵而望脚跟不著地也}
潜相糾結以候王師芮芮亦遣間使|{
	芮芮即蠕蠕南人語轉耳間古莧翻}
遠輸誠欵誓為掎角|{
	掎居蟻翻}
經略之會實在兹日可遣寧朔將軍王玄謨帥太子步兵校尉沈慶之鎮軍諮議參軍申坦水軍入河受督於青冀二州刺史蕭斌|{
	帥讀曰率斌音彬}
太子左衛率臧質驍騎將軍王方回徑造許洛|{
	率所律翻驍堅堯翻騎奇寄翻造七到翻}
徐兖二州刺史武陵王駿豫州刺史南平王鑠各勒所部東西齊舉梁南北秦三州刺史劉秀之震盪汧隴|{
	汧苦堅翻}
太尉江夏王義恭出次彭城為衆軍節度|{
	夏戶雅翻}
坦鍾之曾孫也|{
	申鍾見九十五卷晉成帝咸和九年}
是時軍旅大起王公妃主及朝士牧守下至富民各獻金帛雜物以助國用|{
	朝直遥翻}
又以兵力不足悉發青冀徐豫二兖六州|{
	二兖南兖北兖也}
三五民丁倩使蹔行|{
	三五者三丁其一五丁其二倩七政翻}
符到十日裝束|{
	自符到之日以十日為裝束過此期即行}
緣江五郡集廣陵緣淮三郡集盱眙|{
	緣江五郡南東海南蘭陵南琅邪南東莞晉陵也緣淮三郡臨淮淮陵下邳也盱眙音吁怡}
又募中外有馬步衆藝武力之士應科者皆加厚賞有司又奏軍用不充揚南徐兖江四州|{
	此兖謂南兖州}
富民家貲滿五十萬僧尼滿二十萬並四分借一事息即還建武司馬申元吉引兵趨碻磝乙亥魏濟州刺史王買德棄城走|{
	魏明元帝泰常八年置濟州于碻磝城趨七喻翻濟子禮翻 考異曰宋畧云虜濟州刺史王淮敗走虜支解王淮傳示列戍今從宋書}
蕭斌遣將軍崔猛攻樂安魏青州刺史張淮之亦棄城走|{
	樂安千乘傅昌之地唐青州千乘縣此時樂安郡也}
斌與沈慶之留守碻磝使王玄謨進圍滑臺 |{
	考異曰宋畧九月庚申玄謨前軍次白馬與虜兖州刺史歌得跋戰破之玄謨進攻滑臺今從宋書}
雍州刺史随王誕|{
	雍於用翻}
遣中兵參軍柳元景振威將軍尹顯祖奮武將軍曾方平|{
	南史作魯方平參考水經作魯為是}
建武將軍薛安都略陽太守龎法起將兵出弘農|{
	龎皮江翻將即亮翻}
後軍外兵參軍龎季明年七十餘自以關中豪右請入長安招合夷夏|{
	夏戶雅翻}
誕許之乃自貲谷入盧氏盧氏民趙難納之|{
	貲谷在盧氏縣南山之南盧氏縣漢屬弘農郡晉分屬上洛郡唐屬虢川}
季明遂誘說士民|{
	誘音酉說輸芮翻}
應之者甚衆安都等因之自熊耳山出|{
	熊耳山在盧氏故縣東}
元景引兵繼進豫州刺史南平王鑠遣中兵參軍胡盛之出汝南梁坦出上蔡向長社 |{
	考異曰鑠傳作到坦之今從宋畧}
魏荆州刺史魯爽鎮長社棄城走爽軌之子也|{
	軌魯宗之之子}
幢主王陽兒擊魏豫州刺史僕蘭破之|{
	軍有幢主隊主總一軍者謂之軍主僕蘭亦姓拓拔魏書官氏志内入諸姓僕蘭氏改為僕氏幢傳江翻}
僕蘭奔虎牢|{
	虎牢魏豫州刺史治所也}
鑠又遣安蠻司馬劉康祖將兵助坦進逼虎牢魏羣臣初聞有宋師言於魏主請遣兵救緣河穀帛魏主曰馬今未肥天時尚熱速出必無功若兵來不止且還隂山避之國人本著羊皮袴何用綿帛|{
	國人謂同自北荒來之種人也著陟畧翻}
展至十月吾無憂矣|{
	展寛也}
九月辛卯魏主引兵南救滑臺命太子晃屯漠南以備柔然吳王余守平城庚子魏州郡兵五萬分給諸軍王玄謨士衆甚盛器械精嚴而玄謨貪愎好殺|{
	愎弼力翻好呼到翻}
初圍滑臺城中多茅屋衆請以火箭燒之|{
	杜佑曰以小瓢盛油冠矢端射城樓櫓板木上瓢敗油散因燒矢内簳中射油散處火立燃復以油瓢續之則樓櫓盡焚謂之火箭}
玄謨曰彼吾財也何遽燒之城中即撤屋穴處|{
	處昌呂翻}
時河洛之民競出租穀操兵來赴者日以千數|{
	操千高翻}
玄謨不即其長帥而以配私暱|{
	即就也言不能就其長帥而用之使各為部隊而以其人分配私所愛暱者長知兩翻帥所類翻暱尼質翻}
家付匹布責大梨八百由是衆心失望攻城數月不下聞魏救將至衆請車為營玄謨不從|{
	玄謨豈不知為車營可憑而戰哉蓋于時已有走心矣}
冬十月癸亥魏主至枋頭使關内侯代人陸真夜與數人犯圍潜入滑臺撫慰城中且登城視玄謨營曲折還報乙且魏主渡河衆號百萬鞞鼔之聲震動天地|{
	鞞部迷翻}
玄謨懼退走魏人追擊之死者萬餘人麾下散亡略盡委棄軍資器械山積先是玄謨遣鍾離太守垣護之以百舸為前鋒據石濟|{
	鍾離縣漢屬九江郡晉屬淮南郡晉安帝分立鍾離郡屬南兖州沈約志屬徐州水經曰河水逕東燕縣故城北則有濟水自北來注之注云垣護之守石濟即此處先悉薦翻舸古我翻守手又翻}
在滑臺西南百二十里護之聞魏兵將至馳書勸玄謨急攻曰昔武皇攻廣固死没者甚衆|{
	事見一百一十五卷晉安帝義熙五年六年}
况今事迫於曩日豈得計士衆傷疲願以屠城為急玄謨不從|{
	使玄謨從護之計急攻而得滑臺魏兵随至同無以善其後也}
及玄謨敗退不暇報護之魏人以所得玄謨戰艦連以鐵鎻三重斷河|{
	艦戶黯翻重直龍翻斷音短}
以絶護之還路河水迅急護之中流而下每至鐵鎖以長柯斧斷之|{
	斷音短}
魏不能禁唯失一舸餘皆完備而返|{
	舸古我翻}
蕭斌遣沈慶之將五千人救玄謨|{
	將即亮翻}
慶之曰玄謨士衆疲老寇虜已逼得數萬人乃可進小軍輕往無益也斌固遣之會玄謨遁還斌將斬之慶之固諫曰佛狸威震天下|{
	魏主小字佛狸佛音弼}
控弦百萬豈玄謨所能當且殺戰將以自弱非良計也|{
	將即亮翻}
斌乃止斌欲固守碻磝慶之曰今青冀虚弱而坐守窮城若虜衆東過清東非國家有也|{
	東過謂越碻磝而過東入青冀界清東謂清水以東也}
碻磝孤絶復作朱修之滑臺耳|{
	朱修之事見一百二十二卷八年復扶又翻下復召同}
會詔使至不聽斌等退師|{
	使疏吏翻}
斌復召諸將議之並謂宜留慶之曰閫外之事將軍得以專之詔從遠來不知事勢節下有一范增不能用|{
	引漢高帝之言}
空議何施斌及坐者並笑曰沈公乃更學問|{
	更經也歷也音工衡翻}
慶之厲聲曰衆人雖知古今不如下官耳學也|{
	耳學謂雖未嘗目覽書傳能以耳聽人所講說者而學之}
斌乃使王玄謨戍碻磝申坦垣護之據清口|{
	清水南通淮北通河此謂清水入河之口水經濟水東北過壽張縣西界安民亭南汶水從東北來注之注云戴延之所謂清口也}
自帥諸軍還歷城|{
	帥讀曰率下同自此以上皆王玄謨攻滑臺事}
閏月龐法起等諸軍入盧氏斬縣令李封以趙難為盧氏令使帥其衆為鄉導|{
	鄉讀曰嚮}
柳元景自百丈崖從諸軍於盧氏|{
	百丈崖在温谷南}
法起等進攻弘農辛未拔之擒魏弘農太守李初古拔薛安都留屯弘農丙戌龐法起進向潼關|{
	自閏月以下皆柳元景攻關陜事}
魏主命諸將分道並進永昌王仁自洛陽趨夀陽尚書長孫真趣馬頭|{
	沈約曰馬頭郡故淮南當塗縣地晉安帝立馬頭郡因山形而名屬南豫州宋屬徐州將即亮翻趣七喻翻下同}
楚王建趣鍾離高涼王那自青州趣下邳魏主自東平趣鄒山十一月辛卯魏主至鄒山 |{
	考異曰宋略云戊子至鄒山今從後魏書}
魯郡太守崔邪利為魏所擒|{
	宋魯郡時治鄒山}
魏主見秦始皇石刻使人排而仆之|{
	秦始皇二十八年上鄒嶧山立石頌德}
以太牢祠孔子楚王建自清西進屯蕭城步尼公自清東進屯留城|{
	魏收地形志沛郡蕭縣有蕭城彭城郡之留縣有留城}
武陵王駿遣參軍馬文恭將兵向蕭城江夏王義恭遣軍主嵇玄敬將兵向留城文恭為魏所敗|{
	敗補賣翻}
步尼公遇玄敬引兵趣苞橋欲渡清西沛縣民燒苞橋夜於林中擊鼓魏以為宋兵大至争渡苞水|{
	水經注苞水亦曰豐水水上承大薺陂東逕已氏及平樂縣又東逕豐縣故城南又東合黄水水上舊有梁謂之苞橋沛縣民燒苞橋魏兵溺死之地也又東逕沛縣故城南}
溺死者殆半|{
	自此以上魏主分遣諸將事也}
詔以柳元景為弘農太守元景使薛安都尹顯祖先引兵就龐法起等於陜元景於後督租陜城險固諸軍攻之不拔|{
	陜失冉翻}
魏洛州刺史張是連提 |{
	考異曰宋畧作張是連踶今從宋書}
帥衆二萬度崤救陜|{
	自洛至陜冇三崤之險帥讀曰率}
安都等與戰於城南魏人縱突騎諸軍不能敵|{
	騎奇寄翻下同}
安都怒脱兜鍪解鎧|{
	鍪音牟}
唯著絳納兩當衫|{
	著陟畧翻前當心後當背謂之兩當衫}
馬亦去具裝瞋目横矛單騎突陳所向無前魏人夾射不能中如是數四殺傷不可勝數|{
	去羌呂翻瞋七人翻陳讀曰陣射而亦翻中竹仲翻勝音升}
會日暮别將魯元保引兵自函谷關至魏兵乃退元景遣軍副柳元怙將步騎二千救安都等|{
	一軍之將謂之軍主副將謂之軍副}
夜至魏人不之知明日安都等陳於城西南|{
	陳讀曰陣}
曾方平謂安都曰今勍敵在前|{
	勍渠京翻}
堅城在後是吾取死之日卿若不進我當斬卿我若不進卿當斬我也安都曰善卿言是也遂合戰元怙引兵自南門鼓譟直出旌旗甚盛魏衆驚駭安都挺身奮擊流血凝肘矛折|{
	折而設翻}
易之更入諸軍齊奮自旦至日昃魏衆大潰斬張是連提及將卒三千餘級其餘赴河塹死者甚衆生降二千餘人|{
	昃阻力翻將即亮翻塹七艷翻降下江翻}
明日元景至讓降者曰汝輩本中國民今為虜盡力屈乃降何也|{
	為于偽翻}
皆曰虜驅民使戰後出者滅族以騎蹙步未戰先死此將軍所親見也諸將欲盡殺之元景曰今王旗北指當使仁聲先路|{
	先悉薦翻}
盡釋而遣之皆稱萬歲而去甲午克陜城龎法起等進攻潼關魏戍主婁須棄城走法起等據之關中豪傑所在蠭起及四山羌胡皆來送欵|{
	關中之地四面阻山時羌胡皆依山而居自為聚落}
上以王玄謨敗退魏兵深入柳元景等不宜獨進皆召還元景使薛安都斷後|{
	斷丁管翻}
引兵歸襄陽詔以元景為襄陽太守|{
	此以上柳元景攻關陜事}
魏永昌王仁攻懸瓠項城拔之帝恐魏兵至夀陽召劉康祖使還癸卯仁將八萬騎追及康祖於尉武|{
	將即亮翻騎奇計翻 考異曰宋略及南平王鑠傳皆作尉氏按康祖傳云去壽陽裁數十里然則非尉氏也今從康祖及索虜傳作尉武 今按沈約志秦郡有尉氏縣秦郡治堂邑屬南兖州非南平王鑠所統其地又不在壽陽北數十里温公之考覈精矣按北史拓拔崘傳尉武亭名劉康祖戰死于此}
康祖有衆八千人軍副胡盛之|{
	幢隊軍皆有主副}
欲依山險間行取至|{
	間古莧翻取至謂取至夀陽也}
康祖怒曰臨河求敵遂無所見幸其自送奈何避之乃結車營而進下令軍中曰顧望者斬首轉步者斬足魏人四面攻之將士皆殊死戰自旦至晡殺魏兵萬餘人流血没踝|{
	踝胡瓦翻足踝也}
康祖身被十創|{
	被皮義翻創初良翻}
意氣彌厲魏分其衆為三且休且戰會日暮風急魏以騎負草燒軍營康祖随補其闕有流矢貫康祖頸墜馬死餘衆不能戰遂潰魏人掩殺殆盡 |{
	考異曰康祖傳云大戰一日一夜又云虜死者大半今從宋畧}
南平王鑠使左軍行參軍王羅漢以三百人戍尉武魏兵至衆欲依卑林以自固羅漢以受命居此不去魏人攻而擒之鎖其頸使三郎將掌之|{
	三郎將蓋主内三郎魏謂衛士曰三郎將將即亮翻}
羅漢夜斷三郎將首|{
	斷丁管翻}
抱鎖亡奔盱眙|{
	盱眙音吁怡}
魏永昌王仁進逼夀陽焚掠馬頭鍾離南平王鑠嬰城固守|{
	自此以上魏兵向壽陽事}
魏兵在蕭城去彭城十餘里彭城兵雖多而食少|{
	少詩沼翻下同}
太尉江夏王義恭欲棄彭城南歸安北中兵參軍沈慶之以為歷城兵少食多欲為函箱車陳以精兵為外翼奉二王及妃女直趨歷城|{
	陳讀曰陣趨七喻翻}
分兵配護軍蕭思話使留守彭城太尉長史何朂欲席卷奔鬱洲|{
	東海郡贑榆縣東海中有鬱洲今東海軍是其地泰始三年於此僑立青州齊梁為青冀二州刺史治所卷讀曰捲鬱音聿}
自海道還京師義恭去意已判|{
	判亦决也}
惟二議彌日未决|{
	沈慶之之議自彭城趨歷城猶曰主於進何朂之議則主於奔退耳}
安北長史沛郡太守張暢曰若歷城鬱洲有可至之理下官敢不高贊|{
	時沛郡治蕭城張暢以安北長史帶沛郡太守高抗也贊助也言抗聲以助决其議也}
今城中乏食百姓咸有走志但以關扃嚴固欲去莫從耳|{
	扃古熒翻外閉之關也此言門守嚴固百姓無從得去}
一旦動足則各自逃散欲至所在何由可得今軍食雖寡朝夕猶未窘罄|{
	窘渠隕翻}
豈有捨萬安之術而就危亡之道若此計必行下官請以頸血汚公馬蹄|{
	汚烏故翻}
武陵王駿謂義恭曰阿父既為總統去留非所敢干|{
	義恭頓彭城為諸軍節度故曰總統阿讀從安入聲}
道民沗為城主而委鎮奔逃實無顔復奉朝廷|{
	義恭於駿諸父也駿小字道民徐州刺史治彭城故曰城主復扶又翻}
必與此城共其存没張長史言不可異也義恭乃止壬子魏主至彭城立氈屋於戲馬臺以望城中|{
	戲馬臺在彭城城南其高十仞廣袤百步項羽所築也}
馬文恭之敗也隊主蒯應没於魏|{
	此上蕭城之敗也蒯苦怪翻}
魏主遣應至小市門求酒及甘蔗|{
	甘蔗說文所謂諸蔗也生於南方北人嗜之蔗之夜翻}
武陵王駿與之仍就求槖駞|{
	韋昭曰槖駞背肉似槖而善負物顔師古曰言能負槖而馱物故曰槖駞爾雅翼駞外國之奇畜背有兩封如鞍其足三節色蒼褐負物至千斤日三百里凡欲椿載必先屈足受之所載未盡其量終不起古語謂之槖佗槖囊也佗負荷也今云駱駞蓋槖音之轉}
明日魏主使尚書李孝伯至南門餉義恭貂裘餉駿槖駞及騾|{
	騾盧戈翻驢父馬母堅耐健走}
且曰魏主致意安北可蹔出見|{
	蹔與暫同}
我亦不攻此城何為勞苦將士備守如此駿使張暢開門出見之曰安北致意魏主常遲面寫|{
	遲直利翻待也}
但以人臣無境外之交恨不蹔悉|{
	悉詳盡也言恨不蹔時得詳盡所懷也}
備守乃邉鎮之常悦以使之則勞而無怨耳|{
	易兌卦彖辭曰悦以先民民忘其勞}
魏主求甘橘及借博具皆與之復餉氈及九種鹽胡豉|{
	孝伯傳曰凡此諸鹽各有所宜白鹽食鹽主上日所食黑鹽療腹脹氣滿末之六銖以酒而服胡鹽療目痛戎鹽療諸瘡赤鹽駮鹽馬齒鹽四種並非食鹽豉是義翻說文曰配鹽幽尗也胡豉胡人所造尗與菽同豆也復扶又翻種章勇翻}
又借樂器義恭應之曰受任戎行|{
	行戶剛翻}
不齎樂具孝伯問暢何為怱怱閉門絶橋暢曰二王以魏主營壘未立將士疲勞此精甲十萬恐輕相陵踐故閉城耳待休息士馬然後共治戰場刻日交戲|{
	左傳晉楚將戰于城濮楚令尹子玉遣使謂晉曰請與君之士戲踐息演翻治直之翻}
孝伯曰賓有禮主則擇之|{
	左傳魯大夫羽父語薛侯之言}
暢曰昨見衆賓至門未為有禮魏主使人來言曰致意太尉安北何不遣人來至我所彼此之情雖不可盡要須見我小大知我老小觀我為人若諸佐不可遣亦可使僮幹來|{
	諸佐謂佐吏也僮幹則給使令者耳魏主此言猶知宋為有人}
暢以二王命對曰魏主形狀才力久為來往所具李尚書親自銜命不患彼此不盡故不復遣使|{
	復扶又翻下無復同}
孝伯又曰王玄謨亦常才耳南國何意作如此任使以致奔敗自入此境七百餘里主人竟不能一相拒逆鄒山之險君家所憑前鋒始接崔邪利遽藏入穴諸將倒曳出之|{
	鄒山多石穴土人謂穴為嶧相率入保藏以避兵故孝伯云然}
魏主賜其餘生今從在此暢曰王玄謨南土偏將不謂為才但以之為前驅大軍未至河氷向合玄謨因夜還軍致戎馬小亂耳崔邪利陷没何損於國魏主自以數十萬衆制一崔邪利乃足言邪知入境七百里無相拒者此自太尉神筭鎮軍聖略|{
	武陵王駿降號鎮軍將軍}
用兵有機不用相語|{
	語牛倨翻}
孝伯曰魏主當不圍此城自帥衆軍直造瓜步|{
	瓜步山在秦郡尉氏縣界尉氏隋改為六合縣南北對境圖曰今祧葉山即瓜步鎮之地帥讀曰率造七到翻}
南事若辦彭城不待圍若其不捷彭城亦非所須也我今當南飲江湖以療渴耳暢曰去留之事自適彼懷若虜馬遂得飲江便為無復天道先是童謡云|{
	先悉薦翻}
虜馬飲江水佛狸死卯年|{
	佛音弼}
故暢云然暢音容雅麗孝伯與左右皆歎息孝伯亦辯贍且去謂暢曰長史深自愛相去步武|{
	舉足而行曰步足迹曰武}
恨不執手暢曰君善自愛冀蕩定有期君若得還宋朝今為相識之始|{
	兵交使在其間史言行人善於辭令亦足以增國威朝直遥翻}
上起楊文德為輔國將軍引兵自漢中西入揺動汧隴|{
	汧苦堅翻}
文德宗人楊高帥隂平平武羣氐拒之|{
	帥讀曰率}
文德擊高斬之隂平平武悉平|{
	隂平縣漢屬廣漢屬國晉泰始中置隂平郡劉蜀分隂平置平廣縣晉太康元年更名平武隂平平武皆今龍州地也宋白曰隂平今文州平武今龍州}
梁南秦二州刺史劉秀之遣文德伐啖提氐不克執送荆州使文德從祖兄頭戍葭蘆|{
	啖徒覽翻又徒濫翻從才用翻}
丁未大赦 魏主攻彭城不克十二月丙辰朔引兵南下使中書郎魯秀出廣陵高凉王那出山陽永昌王仁出横江所過無不殘滅城邑皆望風奔潰戊午建康纂嚴己未魏兵至淮上 |{
	考異曰魏本紀云丁卯至淮按宋畧己未虜至淮西宋本紀乙丑胡崇之等敗今從之}
上使輔國將軍臧質將萬人救彭城|{
	將即亮翻}
至盱眙魏主已過淮|{
	盱眙音吁怡}
質使宂從僕射胡崇之積弩將軍臧澄之營東山|{
	宂而隴翻從才用翻 考異曰序傳作臧證之今從帝紀質傳作澄之}
建威將軍毛熙祚據前浦|{
	東山前浦皆在盱眙城左右東山在今盱眙城東南東山之北則高家山高家山之東則陡山稍南則都梁山都梁山之東北則古盱眙城城臨遇明河又東逕楊茅澗口又東逕富陵河口則君山魏太武作浮橋于此自此渡淮稍東則龜山}
質營於城南 |{
	考異曰宋畧云質屯盱眙城北今從宋書}
乙丑魏燕王譚攻崇之等三營皆敗没質按兵不敢救澄之燾之孫|{
	臧燾高祖敬皇后之兄}
熙祚修之之兄子也|{
	毛修之從高祖為將青泥之敗没于赫連後入于魏}
是夕質軍亦潰質棄輜重器械單將七百人赴城|{
	重直用翻將即亮翻}
初盱眙太守沈璞到官|{
	盱眙縣前漢屬臨淮郡後漢屬下邳國晉復屬臨淮郡晉安帝分立盱眙郡今為招信軍}
王玄謨猶在滑臺江淮無警璞以郡當衝要乃繕城浚隍積財穀儲矢石為城守之備僚屬皆非之朝廷亦以為過及魏兵南向守宰多棄城走或勸璞宜還建康璞曰虜若以城小不顧夫復何懼|{
	夫音扶復扶又翻}
若肉薄來攻此乃吾報國之秋諸君封侯之日也|{
	薄伯各翻}
奈何去之諸君嘗見數十萬人聚於小城之下而不敗者乎昆陽合肥前事之明驗也|{
	王尋王邑以百萬敗於昆陽諸葛恪以二十萬敗於合肥故曰用兵之計攻城最下}
衆心稍定璞收集得二千精兵曰足矣及臧質向城衆謂璞曰虜若不攻城則無所事衆若其攻城則城中止可容見力耳|{
	見賢遍翻}
地狹人多鮮不為患|{
	鮮息淺翻}
且敵衆我寡人所共知若以質衆能退敵完城者則全功不在我若避罪歸都會資舟楫必更相蹂踐|{
	蹂人九翻踐慈演翻}
正足為患不若閉門勿受璞歎曰虜必不能登城敢為諸君保之|{
	為于偽翻}
舟楫之計固已久息虜之殘害古今未有屠剝之苦衆所共見其中幸者不過驅還北國作奴婢耳彼雖烏合寧不憚此邪所謂同舟而濟胡越一心者也|{
	王弼曰同舟而濟則胡越何患乎異心}
今兵多則虜退速少則退緩吾寧可欲專功而留虜乎乃開門納質質見城中豐實大喜衆皆稱萬歲因與璞共守魏人之南寇也不齎糧用唯以抄掠為資及過淮民多竄匿抄掠無所得人馬飢乏|{
	抄初交翻}
聞盱眙有積粟欲以為北歸之資既破崇之等一攻城不拔即留其將韓元興以數千人守盱眙|{
	守言以兵相守也將即亮翻下同}
自帥大衆南向由是盱眙得益完所備|{
	為明年魏主還攻盱眙不克張本帥讀曰率}
庚午魏主至瓜步壞民廬舍|{
	壞音怪}
及伐葦為筏聲言欲渡江建康震懼民皆荷擔而立|{
	荷擔而立急則迸走荷戶可翻又如字擔丁濫翻}
壬午内外戒嚴丹陽統内盡戶丁|{
	凡人戶見丁無論多少盡之}
王公以下子弟皆從役命領軍將軍劉遵考等將兵分守津要遊邏上接干湖下至蔡洲陳艦列營周亘江濱自采石至于暨陽|{
	今太平州當塗縣北三十里有采石山山下有采石磯暨陽今江隂軍邏郎佐翻艦戶黯翻亘古鄧翻}
六七百里太子劭出鎮石頭總統水軍丹陽尹徐湛之守石頭倉城吏部尚書江湛兼領軍軍事處置悉以委焉|{
	處昌呂翻}
上登石頭城有憂色謂江湛曰北伐之計同議者少|{
	謂唯江徐贊北伐之計羣臣之議多不同也少詩沼翻}
今日士民勞怨不得無慙貽大夫之憂予之過也又曰檀道濟若在豈使胡馬至此上又登幕府山|{
	幕府山在今建康府城西二十五里晉元帝初度江丞相王導建幕府於其上宋白曰元帝度江秣陵荒落以府第居縣北幕府山幕府之名自此南史幕府山在臨沂縣}
觀望形勢購魏主及王公首許以封爵金帛又募人齎野葛酒置空村中欲以毒魏人竟不能傷|{
	野葛有毒食之殺人}
魏主鑿瓜步山為蟠道於其上設氈屋 |{
	考異曰魏帝紀云癸未車駕臨江起行宫於瓜步山蓋謂此也今從宋書}
魏主不飲河南水以槖駞負河北水自随餉上槖駞名馬并求和請婚上遣奉朝請田奇餉以珍羞異味|{
	奉朝請者奉朝會請召而已朝直遥翻}
魏主得黄甘即噉之|{
	甘即今之柑噉徒濫翻又徒覽翻}
并大進酒|{
	荆州記曰長沙郡縣有湖周迴二里取湖水為酒酒極甘美杜佑曰衡州衡陽縣漢縣地孟康曰音零}
左右有附耳語者疑食中有毒魏主不應舉手指天以其孫示奇曰吾遠來至此非欲為功名實欲繼好息民永結姻援宋若能以女妻此孫我以女妻武陵王自今匹馬不復南顧|{
	好呼到翻妻七細翻復扶又翻}
奇還上召太子劭及羣臣議之衆並謂宜許江湛曰戎狄無親許之無益劭怒謂湛曰今三王在阨|{
	謂江夏王義恭武陵王駿在彭城南平王鑠在壽陽也}
詎宜苟執異議聲色甚厲坐散俱出|{
	坐徂卧翻}
劭使班劍及左右排湛湛幾至僵仆|{
	班劍持劍為班列在車前幾居希翻僵居良翻}
劭又言於上曰北伐敗辱數州淪破獨有斬江湛徐湛之可以謝天下上曰北伐自是我意江徐但不異耳|{
	言不持異議也}
由是太子與江徐不平|{
	史言劭於此時已有弑逆之心}
魏亦竟不成婚 |{
	考異曰魏帝紀云甲申義隆使獻百牢貢其方物又請進女於皇孫以求和好帝以師婚非禮許和而不許婚使散騎侍郎夏侯野報之詔皇孫為書致馬通問此皆魏史夸辭今從宋書}


資治通鑑卷一百二十五
















































































































































