






























































資治通鑑卷二百六十三 宋 司馬光 撰

胡三省 音註

唐紀七十九|{
	起玄黓閹茂畫昭陽大淵獻正月凡一年有奇}


昭宗聖穆景文孝皇帝中之下

天復二年春正月癸丑朱全忠復屯三原又移軍武功|{
	將復逼鳳翔也宋白曰三原縣本漢池陽縣地苻堅於嶻嶭北置三原護軍以其地南有鄷原西有孟侯原北有白鹿原為三原後魏太平真君七年罷護軍置縣}
河東將李嗣昭周德威攻慈隰以分全忠兵勢|{
	朱全忠兼有河中慈隰二州其巡屬也}
丁卯以給事中韋貽範為工部侍郎同平章事 丙子以給事中嚴龜充岐汴和恊使賜朱全忠姓李與李茂貞為兄弟全忠不從時茂貞不出戰全忠聞有河東兵二月戊寅朔還軍河中 |{
	考異曰實録在正月按編遺録二月戊寅上以久駐兵車於三原乃議東歸蒲阪遂取高陵櫟陽左馮入于蒲津梁太祖實録正月戊申朔上總御戎馬發自三原復至武功縣駐焉貢章奉辭迴軍赴蒲阪今從唐年補録舊紀}
李嗣昭等攻慈隰下之進逼晉絳己丑全忠遣兄子友寧將兵會晉州刺史氏叔琮擊之李嗣昭襲取絳州汴將康懷英復取之|{
	康懷英即康懷貞後避梁均王友貞名始改名懷英斯時未改也史雜書之}
嗣昭等屯蒲縣乙未汴軍十萬營于蒲南|{
	蒲漢古縣唐屬隰州九域志在州東南九十五里按漢蒲反縣古蒲邑也屬河東郡河東郡又有蒲子縣春秋晉公子所居蒲城也汴州長垣縣古名蒲邑子路所治之地也古邑之以蒲名者蓋非一處宋白曰後魏孝文帝改蒲子為長壽縣隋開皇十八年改為隰川後魏孝武帝於蒲子東南置石城縣尋廢後周大象元年於廢縣置蒲子縣取古蒲子為名隋大業二年改為蒲縣移今理}
叔琮夜帥衆斷其歸路|{
	帥讀曰率斷音短}
而攻其壘破之殺獲萬餘人己亥全忠自河中赴之乙巳至晉州 盜發簡陵|{
	簡陵懿宗陵}
西川兵至利州昭武節度使李繼忠弃鎮奔鳳翔王建以劒州刺史王宗偉為利州制置使|{
	光啓二年升興鳳二州為感義軍節度使時僖宗在山南欲以捍東兵也文德元年感義軍增領利州至乾寧四年更感義軍曰昭武軍徙鎭利州李茂貞既兼山南欲以鎮兵捍王建而終不能捍也建自此遂有利州}
三月庚戌上與李茂貞及宰相學士中尉樞密宴酒酣茂貞及韓全誨亡去上問韋貽範朕何以巡幸至此對曰臣在外不知固問不對上曰卿何得於朕前妄語云不知又曰卿既以非道取宰相當於公事如法|{
	謂處事當皆如國法}
若有不可必準故事|{
	謂貶竄之也}
怒目視之|{
	怒奴古翻}
微言曰此賊兼須杖之二十顧謂韓偓曰此輩亦稱宰相貽範屢以大盃獻上上不即持貽範舉盃直及上頤|{
	史言昭宗以酣酗納侮}
戊午氏叔琮朱友寧進攻李嗣昭周德威營時汴軍横陳十里|{
	陳讀曰陣}
而河東軍不過數萬深入敵境衆心忷懼|{
	忷許拱翻}
德威出戰而敗密令嗣昭以後軍前去德威尋引騎兵亦退叔琮友寧長驅乘之河東軍驚潰禽克用子廷鸞兵仗輜重委弃畧盡|{
	重直用翻}
朱全忠令叔琮友寧乘勝遂攻河東李克用聞嗣昭等敗遣李存信以親兵逆之|{
	李克用親兵皆代北雜虜最為驍勁}
至清源|{
	清源縣在晉陽南五十里}
遇汴軍存信走還晉陽|{
	衆寡不敵故走}
汴軍取慈隰汾三州辛酉汴軍圍晉陽營於晉祠|{
	晉陽有晉王祠}
攻其西門周德威李嗣昭收餘衆依西山得還|{
	汾水過晉陽東晉陽西南接界休縣之介山綿山}
城中兵未集叔琮攻城甚急每行圍|{
	行下孟翻}
褒衣博帶以示閒暇克用晝夜乘城不得寢食召諸將議保雲州李嗣昭李嗣源周德威曰兒輩在此必能固守 |{
	考異曰唐太祖紀年録嗣昭與今上日夜入賊營斬將搴旗賊多驚擾梁太祖實録三月癸丑虜衆悉出友寜以飛騎犯其左右翼虜大敗北掩殺不知其數擒克用男廷鸞及將校健卒數人實録朱友寧圍太原營西北隅攻其西門城内大恐克用欲奔雲中弟克寧止之又遣李嗣昭與克用子存朂日夜擾賊營友寧乃燒營而遁按紀年録所謂今上者乃明宗非莊宗也實録誤}
王勿為此謀動揺人心李存信曰關東河北皆受制於朱温我兵寡地蹙守此孤城彼築壘穿塹環之|{
	環音宦}
以積久制我我飛走無路坐待困斃耳今事勢已急不若且入北虜徐圖進取嗣昭力爭之克用不能決劉夫人言於克用曰存信北川牧羊兒耳|{
	代北之地謂之北川以陘嶺之北皆平川也}
安知遠慮王常笑王行瑜輕去其城死於人手|{
	王行瑜死見二百六十卷乾寧二年}
今日反効之邪且王昔居達靼幾不自免賴朝廷多事乃得復歸|{
	事見二百五十三卷僖宗廣明元年幾居依翻}
今一足出城則禍變不測塞外可得至邪克用乃止居數日潰兵復集軍府浸安克用弟克寧為忻州刺史聞汴寇至中塗復還晉陽|{
	晉陽北至忻州一百七十餘里復扶又翻}
曰此城吾死所也去將何之衆心乃定壬戌朱全忠還河中遣朱友寧將兵西擊李茂貞軍于興平武功之間|{
	興平縣在長安西武功縣在長安西北}
李嗣昭李嗣源數將敢死士夜入氏叔琮營|{
	數所角翻將即亮翻下同}
斬首捕虜汴軍驚擾備禦不暇會大疫丁卯叔琮引兵還嗣昭與周德威將兵追之及石會關叔琮留數馬及旌旗於高岡之巔嗣昭等以為有伏兵乃引去復取慈隰汾三州自是克用不敢與全忠爭者累年|{
	兵少力疲故閉境養晦以俟時}
克用以使引咨幕府|{
	使引節度府所行文引謀事曰咨今北人以文書達於上曰咨使疏吏翻}
曰不貯軍食何以聚衆不置兵甲何以克敵不修城池何以扞禦利害之間請垂議度|{
	貯丁呂翻度徒洛翻}
掌書記李襲吉獻議略曰國富不在倉儲兵強不由衆寡人歸有德神固害盈|{
	書咸有一德曰非商求于下民惟民歸于一德易謙卦彖辭曰鬼神害盈而福謙}
聚斂寧有盜臣|{
	大學載孟獻子之言曰與其有聚斂之臣寧有盜臣歛力贍翻}
苛政有如猛虎|{
	記檀弓載孔子之言曰苛政猛於虎也}
所以鹿臺將散周武以興|{
	武王伐紂散鹿臺之財一戎衣而天下大定}
齊庫既焚晏嬰入賀|{
	韓詩外傳曰晉平公之藏臺火救火三日三夜乃勝之公子晏束帛而賀曰臣聞王者藏於天下諸侯藏於百姓農夫藏於囷庾今百姓乏於外而賦斂無已昔桀紂殘賊為天下戮今皇天降災於藏臺是君之福也李襲吉以為齊庫焚而晏嬰入賀蓋别有所据}
又曰伏以變法不若養人|{
	温公讀此語感熙豐之政蓋深有味乎其言也}
改作何如舊貫|{
	論語魯人為長府閔子騫曰仍舊貫如之何何必改作}
韓建蓄財無數首事朱温|{
	事見上卷上年十一月}
王珂變法如麻一朝降賊|{
	事見上卷上年正月珂丘何翻降戶江翻}
中山城非不峻|{
	謂王郜不能守定州城}
蔡上兵非不多|{
	謂秦宗權恃衆卒為朱温禽自韓建以下又以克用耳目之所睹記者動悟之}
前事甚明可以為戒且霸國無貧主強將無弱兵伏願大王崇德愛人去奢省役|{
	去羌呂翻}
設險固境訓兵務農定亂者選武臣制理者選文吏|{
	制理猶言制治也避唐廟諱}
錢穀有句|{
	出納之籍明則姦弊自無所容句讀曰鉤}
刑法有律|{
	依律定刑則吏手不得而輕重}
誅賞由我則下無威福之弊近密多正則人無譛謗之憂順天時而絶欺誣敬鬼神而禁淫祀則不求富而國富不求安而自安外破元凶|{
	元凶指朱温}
内康疲俗名高五霸|{
	杜預曰五霸夏昆吾商大彭豕韋周齊桓晉文又曰齊桓晉文宋襄秦穆楚莊為五霸}
道冠八元|{
	冠古玩翻高辛氏有才子八人伯奮仲堪叔獻季仲伯虎仲熊叔豹季貍忠肅恭懿宣慈惠和天下之民謂之八九}
至於率閭閻定間架增麴蘖|{
	蘖魚列翻}
檢田疇開國建邦恐未為切克用親軍皆沙陀雜虜喜侵暴良民|{
	喜許記翻}
河東甚苦之其子存朂以為言克用曰此輩從吾攻戰數十年比者帑藏空虚|{
	比毘至翻帑他朗翻藏才浪翻}
諸軍賣馬以自給今四方諸侯皆重賞以募士我若急之則彼皆散去矣吾安與同保此乎|{
	此高歡告杜弼之說也異時莊宗既得天下兒郎寒冷遮馬邀求以養成驕軍之禍得非此語誤之邪}
俟天下稍平當更清治之耳|{
	如此語則克用之意蓋有待也莊宗得天下之後豈不復記憶此語邪治直之翻}
存朂幼警敏有勇略克用為朱全忠所困封疆日蹙憂形於色存朂進言曰物不極則不返惡不極則不亡朱氏恃其詐力窮凶極暴吞滅四鄰人怨神怒今又攻逼乘輿窺覦神器|{
	乘䋲證翻覦音俞}
此其極也殆將斃矣吾家世襲忠貞|{
	謂自朱邪執宜以來皆輸力於唐室}
勢窮力屈無所愧心大人當遵養時晦|{
	詩酌之篇曰於鑠王師遵養時晦毛傳曰遵率養取晦昧也鄭箋曰文王之用師率殷之叛國以事紂養是暗昧之君以老其惡}
以待其衰奈何輕為沮喪|{
	喪息浪翻}
使羣下失望乎克用悦即命酒奏樂而罷劉夫人無子克用寵姬曹氏生存朂劉夫人待曹氏加厚克用以是益賢之諸姬有子輒命夫人母之夫人教養悉如所生 上以金吾將軍李儼為江淮宣諭使書御札賜楊行密拜行密東面行營都統中書令吳王以討朱全忠以朱瑾為平盧節度使馮弘鐸為武寧節度使朱延壽為奉國節度使|{
	平盧軍青州武寧軍徐州奉國軍蔡州朱瑾等皆遥領耳}
加武安節度使馬殷同平章事淮南宣歙湖南等道立功將士聽用都統牒承制遷補然後表聞儼張濬之子也賜姓李 |{
	考異曰唐補紀昭宗自鳳翔遣金吾將軍李儼齎御札自巫峽間道潜行宣告吳王楊行密為討伐逆賊朱全忠事李儼者宰臣張濬男其張濬先為都統討太原退軍朝貶韓建力救不赴貶所只在三峯其男留行在乃授金吾將軍昭宗差來宣告於吳王行密朱全忠探知張濬一門盡遭殺戮按此年濬未死儼賜姓見此年十月注}
夏四月丁酉崔胤自華州詣河中泣訴於朱全忠恐李茂貞刼天子幸蜀宜以時迎奉勢不可緩全忠與之宴胤親執板為全忠歌以侑酒|{
	板拍板也古樂無之玄宗時教坊散樂用横笛一拍板一腰鼓三後人因之歌舞率以板為節以木若象凡八片以韋貫之兩手各執其外一片而拍之為于偽翻}
辛丑回鶻遣使入貢請發兵赴難|{
	難乃旦翻}
上命翰林學士承旨韓偓答書許之乙巳偓上言戎狄獸心不可倚信彼見國家人物華靡而城邑荒殘甲兵彫弊必有輕中國之心啓其貪婪|{
	婪盧含翻}
且自會昌以來回鶻為中國所破|{
	事見二百四十七卷武宗會昌三年}
恐其乘危復怨所賜可汗書宜諭以小小寇竊不須赴難虛愧其意實沮其謀從之兵部侍郎參知機務盧光啓罷為太子太保 楊行密遣顧全武歸杭州以易秦裴|{
	顧全武為淮南兵所禽見上卷上年秦裴降錢鏐見二百六十一卷光化元年}
錢鏐大喜遣裴還汴將康懷貞擊鳳翔將李繼昭於莫谷|{
	莫谷即漠谷在奉天城北}


大破之繼昭蔡州人也本姓符名道昭|{
	為後繼昭降汴復舊姓張本}
五月庚戌溫州刺史朱褒卒兄敖自稱刺史|{
	薛史朱褒温州}


|{
	人兄弟皆為本州牙校刺史胡璠卒誕據郡朱褒逼誕而代之與通鑑稍異}
鳳翔人聞朱全忠且來皆懼癸丑城外居民皆遷入城己未全忠將精兵五萬發河中 |{
	考異曰全鑾記五月三日岐馬步軍敗迴戈傷中不少八日聞四面百姓盡般移入城内二十一日聞汴帥於郿縣築城及寶雞下寨二十三日聞汴帥至石鼻又至横渠二十四日聞汴帥至城南十里按編遺録六月全忠始離渭橋此蓋全忠下遊兵耳實録據金鑾記云癸亥朱全忠引軍在石鼻乙丑至横渠己巳駐師城南誤也}
至東渭横橋遇霖雨留旬日 庚午工部侍郎平章事韋貽範遭母喪|{
	平章事之上當有同字}
宦官薦翰林學士姚洎為相|{
	洎渠至翻}
洎謀於韓偓偓曰若圖永久之利則莫若未就為善儻出上意固無不可且汴軍旦夕合圍孤城難保家族在東可不慮乎洎乃移疾|{
	移文稱有疾}
上亦自不許 鎮海鎮東節度使彭城王錢鏐進爵越王|{
	自郡王進爵國王}
六月丙子以中書舍人蘇檢為工部侍郎同平章事時韋貽範在草土|{
	居喪者寢苫枕塊故曰草土}
薦檢及姚洎於李茂貞上既不用洎茂貞及宦官恐上自用人恊力薦檢遂用之 丁丑朱全忠軍于虢縣|{
	九域志虢縣在鳳翔府南三十五里宋白曰虢縣禮記注謂虢為郭在武都南一百里有虢叔城是也又案地理志云虢漢併於雍今虢縣後魏立為武都郡後周大統十三年遷同州洛邑縣城於武都城西置洛邑縣隋大業三年改洛邑為虢縣}
武寧節度使馮弘鐸介居宣揚之間|{
	宣田頵揚楊行密馮弘鐸以昇州居二鎮之間}
常不自安然自恃樓船之彊不事兩道寧國節度使田頵欲圖之|{
	頵居筠翻}
募弘鐸工人造戰艦|{
	艦戶黯翻}
工人曰馮公遠求堅木故其船堪久用今此無之頵曰第為之|{
	第但也}
吾止須一用耳弘鐸將馮暉顔建說弘鐸先擊頵弘鐸從之帥衆南上|{
	說式芮翻上時長翻}
聲言攻洪州|{
	鍾傳據洪州}
實襲宣州也楊行密使人止之不從|{
	楊行密時為南面諸道都統故欲制其行師進止}
辛巳頵帥舟師逆擊於葛山大破之|{
	新書作曷山當從之張舜民郴行録曰褐山磯在大信口稍西南去蕪湖縣四十餘里帥讀曰率}
甲申李茂貞大出兵自將之與朱全忠戰于虢縣之北大敗而還|{
	將即亮翻下同還音旋又如字}
死者萬餘人丙戌全忠遣其將孔勍出散關|{
	勍渠京翻散關在鳳翔府寶雞縣西南自諸葛亮以來多以自蜀出師為出散關今朱全忠自虢縣遣孔勍進攻鳳州為出散關彼我之說也}
攻鳳州拔之丁亥全忠進軍鳳翔城下全忠朝服嚮城而泣曰臣但欲迎車駕還宫耳|{
	朱全忠借正說以行其譎朝直遥翻}
不與岐王角勝也遂為五寨環之|{
	環音宦 考異曰梁太祖實録六月丁丑暨虢縣辛未文通涸兵驟出布陳俟敵我之將卒躍進決鬭始辰暨午寇大敗屍仆萬餘人命諸軍徙寨逼其壘自是岐人繼出師靡不喪衂六月乙亥上以盩厔有博野軍與岐人往來以窺我命李暉討平丙戌復遣孔勍領兵由大散關取鳳州按六月乙亥朔無辛未前云丁丑後云辛未又再云六月皆誤從唐實録}
馮弘鐸收餘衆沿江將入海|{
	僖宗光啟元年張雄據上元雄死弘鐸繼之至是而亡楊行密遂有昇州}
楊行密恐其為後患遣使犒軍且說之曰|{
	說式芮翻}
公徒衆猶盛胡為自弃滄海之外吾府雖小足以容公之衆使將吏各得其所如何弘鐸左右皆慟哭聽命|{
	衆心既攜馮弘鐸欲不歸楊行密其可得乎}
弘鐸至東塘行密自乘輕舟迎之從者十餘人|{
	從才用翻}
常服不持兵升弘鐸舟慰諭之舉軍感悦署弘鐸淮南節度副使館給甚厚|{
	館古玩翻}
初弘鐸遣牙將丹徒尚公迺詣行密求潤州行密不許公迺大言曰公不見聽但恐不敵樓船耳至是行密謂公迺曰頗記求潤州時否公迺謝曰將吏各為其主|{
	為于偽翻}
但恨無成耳行密笑曰爾事楊叟如事馮公無憂矣|{
	為田頵朱延壽之亂尚公迺盡忠力於楊行密張本}
行密以李神福為昇州刺史|{
	楊行密用李神福刺昇州以横制宣潤}
楊行密發兵討朱全忠以副使李承嗣權知淮南軍府事軍吏欲以巨艦運糧都知兵馬使徐溫曰運路久不行葭葦堙塞|{
	黄巢作亂高駢不臣江淮之運不復至京師故其路久不行塞悉則翻}
請用小艇庶幾易通軍至宿州會久雨重載不能進士有饑色而小艇先至|{
	艇徒鼎翻載昨代翻}
行密由是奇溫始與議軍事|{
	為徐溫竊楊氏三世國命以成養子張本}
行密攻宿州不克竟以糧運不繼引還 秋七月孔勍取成隴二州士卒無鬬者至秦州州人城守乃自故關歸|{
	九域志鳳州西至成州二百七十里北至隴州二百五十里又自隴州西至秦州亦二百五十里孔勍自鳳州西取成州自成州北取隴州又自隴州西至秦州三州時皆屬李茂貞又秦州清水縣東五十里有大震關大中六年隴州防禦使薛達徙築安戎關於隴山由是謂大震關為故關今隴州之西有故關山又西南則清水縣大中六年隴州防禦使薛達奏伏以汧源西境切在故關雖有隄防全無制置僻在重岡之上苟務高深今移要會之中實堪控扼伏乞改為安戎關}
韋貽範之為相也多受人賂許以官既而以母喪罷去日為債家所譟|{
	譟喧聒也}
親吏劉延美所負尤多故汲汲於起復日遣人詣兩中尉樞密及李茂貞求之甲戌命韓偓草貽範起復制偓曰吾腕可斷|{
	腕烏貫翻斷音短}
此制不可草即上疏論貽範遭憂未數月遽令起復實駭物聽傷國體學士院二中使怒曰學士勿以死為戲|{
	時韓誨等使二中使監學士院以防上與之密議國事兼掌傳宣回奏以偓不肯草制故怒}
偓以疏授之解衣而寢二使不得已奏之上即命罷草|{
	罷草制也}
仍賜敕襃賞之八月乙亥朔班定無白麻可宣|{
	班定謂百官立班已定也學士不草制故無麻可宣}
宦官喧言韓侍郎不肯草麻聞者大駭茂貞入見上曰|{
	見賢遍翻}
陛下命相而學士不肯草麻與反何異上曰卿輩薦貽範朕不之違學士不草麻朕亦不之違况彼所陳事理明白若之何不從茂貞不悦而出至中書見蘇檢曰姦邪朋黨宛然如舊扼腕者久之貽範猶經營不已茂貞語人曰我實不知書生禮數為貽範所誤|{
	語牛倨翻李茂貞因此乃知居喪起復之非禮}
會當於邠州安置|{
	言將出貽範}
貽範乃止 保大節度使李茂勲將兵屯三原救李茂貞朱全忠遣其將康懷貞孔勍擊之茂勲遁去茂勲茂貞之從弟也|{
	從才用翻}
初孫儒死|{
	見二百五十九卷景福元年}
其士卒多奔浙西錢鏐愛其驍悍|{
	悍下罕翻又侯旰翻}
以為中軍號武勇都行軍司馬杜稜諫曰狼子野心它日必為深患請以土人代之不從|{
	土人謂浙西人也}
鏐如衣錦軍|{
	錢鏐臨安人既貴改所居營曰衣錦營又升曰衣錦城每遊衣錦城宴故老山林皆覆以錦}
命右武勇都指揮使徐綰帥衆治溝洫|{
	治衣錦軍溝洫帥讀曰率治直之翻洫泥逼翻}
鎮海節度副使成及聞士卒怨言白鏐請罷役不從甲戌鏐臨饗諸將綰謀殺鏐於座不果稱疾先出鏐怪之丁亥命綰將所部兵先還杭州及外城縱兵焚掠武勇左都指揮使許再思以迎兵與之合|{
	迎候兵者許再思以錢鏐將還領兵迎候}
進逼牙城鏐子傳瑛|{
	瑛音英}
與三城都指揮使馬綽等閉門拒之牙將潘長擊綰綰退屯龍興寺鏐還及龍泉|{
	龍泉即龍井在杭州城西南風篁嶺上去城十五里}
聞變疾驅至城北使成及建鏐旗鼓與綰戰鏐微服乘小舟夜抵牙城東北隅踰城而入|{
	宋自高宗駐蹕杭州以杭州牙城為宫城東北隅則今之和寧門外也}
直更卒憑鼓而寐|{
	更工衡翻鼓更鼓也}
鏐親斬之城中始知鏐至武安都指揮使杜建徽自新城入援|{
	九域志新城縣在杭州西南一百三十里}
徐綰聚木將焚北門建徽悉焚之建徽稜之子也湖州刺史高彦聞難遣其子渭將兵入援至靈隱山|{
	九域志湖州南至杭州一百五十五里靈隱山在杭州城西十二里有靈隱寺難乃旦翻}
綰伏兵擊殺之初鏐築杭州羅城|{
	事見二百五十九卷景福二年}
謂僚佐曰十步一樓可以為固矣掌書記餘姚羅隱曰樓不若内向至是人以隱言為驗|{
	樓謂城上敵樓也樓外向所以禦敵今徐綰據杭州羅城而錢鏐自外攻之故人以羅隱不若内向之言為驗}
庚戌李茂貞出兵夜襲奉天虜汴將倪章邵棠以歸乙未茂貞大出兵與朱全忠戰不勝暮歸汴兵追之幾入西門|{
	幾居依翻西門鳳翔城之西門}
己亥再起復前戶部侍郎同平章事韋貽範使姚洎草制貽範不讓即表謝明日視事 西川兵請假道於興元|{
	言假道以勤王}
山南西道節度使李繼密遣兵戍三泉以拒之辛丑西川前鋒將王宗播攻之不克退保山寨親吏柳修業謂宗播曰公舉族歸人不為之死戰何以自保|{
	柳修業王宗播元從孔目官也王宗播許存也歸王建見二百六十卷乾寧二年為于偽翻}
宗播令其衆曰吾與汝曹决戰取功名不爾死於此遂破金牛黑水西縣褒城四寨|{
	武德三年分利州之綿谷置金牛縣寶歷元年省入興元府西縣今三泉縣東六十里有金牛驛輿地廣記大劒山有小石門穿山通道六丈有餘昔秦欲代蜀而不知道乃作五石牛以金置尾下言能糞金欲以遺蜀蜀王負力而貪乃令五丁開道引之秦因使張儀司馬錯引兵代蜀滅之謂之石牛道置牛之地謂之金牛驛褒城漢褒中縣古襃國也隋改曰褒城唐屬興元府九域志縣在府西四十五里又有褒城鎮}
軍校秦承厚攻西縣矢貫左目達于右目鏃不出王建自舐其創膿潰鏃出|{
	王建髣髴吳起吮疽太宗吮血之意校戶教翻舐直氏翻創初良翻}
王宗播攻馬盤寨繼密戰敗奔還漢中西川軍乘勝至城下王宗滌帥衆先登遂克之|{
	帥讀曰率}
繼密請降遷於成都|{
	光化二年李繼密得興元至是而敗王建遂并有山南西道降戶江翻}
得兵三萬騎五千宗滌入屯漢中王建曰繼密殘賊三輔|{
	李繼密從李茂貞茂貞犯獵畿甸繼密蓋預有罪故王建云然}
以其降不忍殺復其姓名曰王萬弘不時召見諸將陵易之|{
	易以豉翻}
萬弘終日縱酒俳優輩亦加戲誚萬弘不勝憂憤醉投池水而卒|{
	誚才笑翻勝音升}
詔以王宗滌為山南西道節度使宗滌有勇略得衆心王建忌之建作府門繪以朱丹蜀人謂之畫紅樓建以宗滌姓名應之|{
	宗滌本姓華名洪更姓名見二百六十一卷乾寧四年}
王宗佶等疾其功復構以飛語|{
	佶巨乙翻復扶又翻}
建召宗滌至成都詰責之宗滌曰三蜀略平|{
	東西川及漢川為三蜀詰去吉翻}
大王聽讒殺功臣可矣建命親隨馬軍都指揮使唐道襲夜飲之酒縊殺之|{
	飲於禁翻}
成都為之罷市連營涕泣如喪親戚|{
	為于偽翻華洪王建之一將耳其死也連營涕泣謂其有勇畧得士心可也而蜀人為之罷市是必有以得民者宜乎不能免於雄猜之主也為於偽翻喪息浪翻}
建以指揮使王宗賀權興元留後道襲閬州人也始以舞童事建後浸預謀畫|{
	為王建太子元膺殺唐道襲張本}
九月乙巳朱全忠以久雨士卒病召諸將議引兵歸河中親從指揮使高季昌左開道指揮使劉知俊曰天下英雄窺此舉一歲矣|{
	朱全忠自去年冬舉兵至此時幾一歲從才用翻}
今茂貞已困奈何捨之去全忠患李茂貞堅壁不出季昌請以譎計誘致之|{
	譎古宂翻誘音酉}
募有能入城為諜者|{
	諜達恊翻間也}
騎士馬景請行曰此行必死願大王録其妻子|{
	録收恤之也}
全忠惻然止之景不可時全忠遣朱友倫發兵於大梁明日將至當出兵迓之|{
	迓魚駕翻迎也}
景請因此時給駿馬雜衆騎而出全忠從之命諸軍皆秣馬飽士丁未旦偃旗幟潜伏營中寂如無人景與衆騎皆出忽躍馬西去詐為逃亡入城告茂貞曰全忠舉軍遁矣獨留傷病者近萬人守營|{
	近其靳翻}
今夕亦去矣請速擊之於是茂貞開門悉衆攻全忠營全忠鼓於中軍百營俱出縱兵擊之又遣數百騎據其城門|{
	遮其歸路也}
鳳翔軍進退失據自蹈藉|{
	藉慈夜翻}
殺傷殆盡茂貞自是喪氣|{
	喪息浪翻}
始議與全忠連和奉車駕還京不復以詔書勒全忠還鎮矣|{
	復扶又翻}
全忠表季昌為宋州團練使|{
	賞其謀也}
季昌硤石人本朱友恭之僕夫也|{
	歐史高季昌董璋皆為汴富人李讓家奴世呼為李七郎者也朱全忠養以為子更姓名曰朱友恭十國紀年以為友恭本夀州賈人李彦威通鑑從之今按歐史據薛史十國紀年與王舉天下大定録同}
戊申武定節度使李思敬以洋州降王建|{
	王建又并有洋州之地}
辛亥李茂貞盡出騎兵於鄰州就芻糧壬子朱全忠穿蚰蜒壕圍鳳翔設犬鋪鈴架以絶内外|{
	蚰與周翻蜒以然翻蚰蜒蟲也多涎天隂雨則出行地皆有跡穿壕塹如蚰蜒行地之狀故謂之蚰蜒壕凡行軍下營四面設犬鋪以犬守之敵來則羣吠使營中知所警備鈴架者繞營設架掛鈴其上敵來觸之則嗚}
癸亥以茂貞為鳳翔靜難武定昭武四鎮節度使|{
	武定昭武時已為王建所取}
或勸錢鏐度江東保越州以避徐許之難|{
	徐許徐綰許再思也難乃旦翻下同}
杜建徽按劒叱之曰事或不濟同死於此豈可復東度乎|{
	復扶又翻}
鏐恐徐綰等據越州遣大將顧全武將兵戍之全武曰越州不足往不若之廣陵|{
	之亦往也廣陵楊行密所治}
鏐曰何故對曰聞綰等謀召田頵田頵至淮南助之不可敵也建徽曰孫儒之難王嘗有德於楊公|{
	難乃旦翻事見二百五十八卷大順二年}
今往告之宜有以相報鏐命全武告急於楊行密全武曰徒往無益請得王子為質|{
	質音致}
鏐命其子傳璙為全武僕|{
	璙力弔翻又力小翻}
與偕之廣陵且求昏於行密過潤州團練使安仁義愛傳璙清麗將以千僕易之全武夜半賂閽者逃去|{
	安仁義號淮南名將居專城之任而門關出入之禁不嚴非善守者也}
綰等果召田頵頵引兵赴之先遣親吏何饒謂鏐曰請大王東如越州空府廨以相待|{
	廨古隘翻}
無為殺士卒鏐報曰軍中叛亂何方無之公為節帥乃助賊為逆戰則亟戰|{
	帥所類翻亟紀力翻}
又何大言頵築壘絶往來之道鏐患之募能奪其地者賞以州衢州制置使陳璋將卒三百出城奮擊遂奪其地鏐即以為衢州刺史|{
	觀此則當時諸州制置使在刺史下}
顧全武至廣陵說楊行密曰使田頵得志必為王患王召頵還錢王請以子傳璙為質且求昏行密許之以女妻傳璙|{
	說式苪翻妻七細翻}
冬十月李儼至揚州 |{
	考異曰十國紀年注李昊蜀書張格傳云弟休仕唐為御史奉使揚州聞長水之禍改姓名為李儼九國志云李儼本左僕射張濬之少子名播起家校書郎遷右拾遺濬為朱全忠所害播自長水奔鳳翔昭宗賜其姓名來使欲徵兵復讐行密與朱全忠書云選張述於諫省俾銜命於敝藩授秩執金賜編屬藉新舊唐書昭宗紀及濬傳皆云天復三年干二月全忠殺濬於長水然則儼來使時濬猶未死述字與休字相亂或一名播乎實録是月始以儼為江淮宣諭使以行密充吳王東面行營都統誤也據行密書則儼父在時已賜姓李宣諭行密以討全忠明年春全忠既克鳳翔儼遂留淮南不敢歸朝耳}
楊行密始建制敕院每有封拜輒以告儼於紫極宫玄宗像前陳制書再拜然後下|{
	玄宗詔天下州郡皆立紫極宫以奉玄元皇帝下戶嫁翻}
王建攻拔興州以軍使王宗浩為興州刺史|{
	王建又併有興州宋白曰興州漢武都之沮縣也蜀置武興督後魏為武興鎮後改為東益州隋改州為順政郡唐武德置興州因武興為州名}
戊寅夜李茂貞假子彦詢帥三團步兵奔於汴軍|{
	帥讀曰率下同}
己卯李彦韜繼之庚辰朱全忠遣幕僚司馬鄴奉表入城 |{
	考異曰實録庚辰司馬鄷奉表壬午對全忠使司馬鄷薛居正五代史司馬鄴傳大軍在岐下遣奏事於昭宗再入復出實録作鄷誤也}
甲申又遣使獻熊白|{
	陸佃埤雅曰熊脂一名熊白熊山居冬蟄當心有白脂如玉味甚美俗呼熊白}
自是獻食物繒帛相繼|{
	繒慈陵翻}
上皆先以示李茂貞使啓視之茂貞亦不敢啓丙戌復遣使請與茂貞議連和|{
	復扶又翻下同}
民出城樵采者皆不抄掠|{
	抄楚交翻}
丁亥全忠表請修宫闕及迎車駕己丑遣國子司業薛昌祚内使王延繢齎詔賜全忠|{
	内使即中使往往梁臣避朱全忠名改中為内耳繢戶外翻又戶對翻}
癸巳茂貞復出兵擊汴軍城西寨敗還全忠以絳袍衣降者|{
	衣於既翻降戶江翻}
使招呼城中人鳳翔軍夜縋去|{
	縋馳偽翻}
及因樵采去不返者甚衆是後茂貞或遣兵出擊汴軍多不為用散還茂貞疑上與全忠有密約壬寅更於御院北垣外增兵防衛 十一月癸卯朔保大節度使李茂勲帥其衆萬餘人救鳳翔屯於城北阪上|{
	阪音反}
與城中舉烽相應 甲辰上使趙國夫人詗學士院二使皆不在|{
	詗古迥翻又翾正翻二使二中使之直學士院者韓全誨等置之以防上密召對學士前此怒韓偓者即其人也}
亟召韓偓姚洎竊見之於土門外執手相泣洎請上速還恐為它人所見上遽去 朱全忠遣其將孔勍李暉將兵乘虛襲鄜坊|{
	鄜音夫下同}
壬子拔坊州甲寅大雪汴軍冒之夕進五鼓抵鄜州城下|{
	九域志坊州北至鄜州一百一十里}
鄜人不為備汴軍入城城中兵尚八千人格鬭至午鄜人始敗|{
	格鬭者短兵接鬭兩兩相當以力角力 考異曰編遺録十二月癸酉遣孔勍李暉領兵襲鄜州以牽李周彛之兵己亥我師攻陷鄜墻獲周彛親族遂令李暉權知鄜畤軍事不數日周彞乃遣幕賓投分通好然後上許抽兵梁太祖實録十一月癸卯鄜帥李周彛統州兵萬餘人屯于老聃祠之下上命孔勍李暉乘虚襲取之壬子勍等破中部郡甲寅大雨雪大軍冒之夕進五鼓及其壘克之按癸卯距己亥近六十日鄜汴相守豈得全不交兵今從唐梁二實録}
擒留守李繼璙勍撫存李茂勲及將士之家按堵無擾命李暉權知軍府事茂勲聞之引兵遁去|{
	重戰輕防此李茂勲之所以敗也厚撫其家以攜之茂勲所以歸心於朱全忠也}
汴軍每夜鳴鼓角城中地如動攻城者詬城上人云劫天子賊乘城者詬城下人云奪天子賊|{
	詬古候翻又許侯翻}
是冬大雪城中食盡凍餒死者不可勝計或臥未死已為人所冎|{
	勝音升冎古瓦翻}
市中賣人肉斤直錢百犬肉直五百茂貞儲偫亦竭|{
	偫文里翻}
以犬彘供御膳上鬻御衣及小皇子衣於市以充用削漬松柹以飼御馬|{
	柹方廢翻斫木札也詳見辯誤飼祥吏翻}
丙子戶部侍郎同平章事韋貽範薨 癸亥朱全忠遣人薙城外草以困城中|{
	薙它計翻除草也}
甲子李茂貞增兵守宫門|{
	行宫門也}
諸宦者自度不免互相尤怨蘇檢數為韓偓經營入相|{
	度徒洛翻數所角翻為于偽翻}
言於茂貞及中尉樞密且遣親吏告偓偓怒曰公與韋公自貶所召歸旬月致位宰相訖不能有所為今朝夕不濟乃欲以此相汚邪|{
	汚烏路翻}
田頵急攻杭州仍具舟將自西陵渡江錢鏐遣其將盛造朱郁拒破之 十二月李茂勲遣使請降於朱全忠更名周彞|{
	更工衡翻}
於是茂貞山南州鎮皆入王建關中州鎮皆入全忠坐守孤城乃密謀誅宦官以自贖遺全忠書曰|{
	遺唯季翻}
禍亂之興皆由全誨僕迎駕至此以備它盜公既志匡社稷請公迎扈還宫僕以弊甲彫兵從公陳力|{
	弊甲彫兵用戰國張儀語半殘為彫}
全忠復書曰僕舉兵至此正以乘輿播遷|{
	乘繩證翻}
公能恊力固所願也 楊行密使人召田頵曰不還吾且使人代鎮宣州|{
	顧全武之說行矣}
庚辰頵將還徵犒軍錢二十萬緡於錢鏐且求鏐子為質將妻以女|{
	質音致妻七細翻}
鏐謂諸子|{
	謂語之也句斷}
孰能為田氏婿者莫對鏐欲遣幼子傳球傳球不可鏐怒將殺之次子傳瓘請行吳夫人泣曰奈何寘兒虎口傳瓘曰紓國家之難|{
	紓緩也難乃旦翻}
安敢愛身再拜而出鏐泣送之|{
	當此之時錢鏐置後之意固已屬于傳瓘矣}
傳瓘從數人縋北門而下|{
	敵情叵測不敢開城門直出故縋而下}
頵與徐綰許再思同歸宣州鏐奪傳球内牙兵印|{
	以其不肯出質也}
越州客軍指揮使張洪以徐綰之黨自疑|{
	客軍蓋亦孫儒散卒}
帥步兵三百奔衢州刺史陳璋納之|{
	帥讀曰率}
温州將丁章逐刺史朱敖敖奔福州|{
	僖宗中和元年朱褒陷温州至是而敗王審知時據福州}
章據温州田頵遣使招之道出衢州陳璋聽其往還錢鏐由是恨璋|{
	為錢鏐圖陳璋張本按田頵時鎮宣州九域志宣州南至歙州自歙州南至睦州自睦州南至婺州自婺州南至處州自處州東至温州其路徑捷今自温州取道衢州者蓋睦州兩浙巡屬其守不與田頵通頵使不敢由此道也自衢州取婺州自婺州取處州自處州取温州更無他岐時盧約據處州亦兩浙巡屬也錢鏐不恨約而恨璋以盧約猶是羈縻而陳璋乃其部曲將故也}
丁酉上召李茂貞蘇檢李繼誨李彦弼李繼岌李繼遠李繼忠食議與朱全忠和上曰十六宅諸王以下凍餒死者日有數人在内諸王及公主妃嬪|{
	十六宅諸王上之兄弟及羣從也在内諸王及公主皇子皇女也}
一日食粥一日食湯餅|{
	湯餅者磑麥為麵以麵作餅投之沸湯煮之黄庭堅所謂煮餅深注湯是也程大昌續演繁露曰釋名餅併也溲麥使合并也蒸餅湯餅之屬各隨形名之}
今亦竭矣卿等意如何皆不對上曰速當和解耳鳳翔兵十餘人遮韓全誨於左銀臺門|{
	長安大明宫城門有左右銀臺門而鳳翔行宫亦設此門示若在長安宫中也}
諠罵曰闔境塗炭闔城餒死止為軍容輩數人耳|{
	為于偽翻}
全誨叩頭訴於茂貞茂貞曰卒輩何知命酌酒兩盃對飲而罷又訴於上上亦諭解之李繼昭謂全誨曰昔楊軍容破楊守亮一族|{
	見二百五十九卷景福元年乾寧元年}
今軍容亦破繼昭一族邪慢罵之遂出降於全忠|{
	降戶江翻}
復姓苻名道昭 是歲䖍州刺史盧光稠攻嶺南陷韶州|{
	韶䖍二州相去雖六百餘里特以大庾嶺為阻而實鄰境也 考異曰新紀是歲光稠卒牙將李圖自稱知州事按十國紀年歐陽修五代史光稠傳開平五年方卒新紀誤也}
使其子延昌守之進圍潮州清海劉隱發兵擊走之乘勝進攻韶州隱弟陟以為延昌有䖍州之援未可遽取隱不從遂圍韶州會江漲餽運不繼|{
	自廣州運糧以餽韶州行營當泝流而上江漲則水湍急不可以泝餽運由此不繼}
光稠自䖍州引兵救之其將譚全播伏精兵萬人於山谷以羸弱挑戰|{
	羸倫為翻桃徒了翻}
大破隱于城南隱奔還全播悉以功讓諸將光稠益賢之 岳州刺史鄧進思卒弟進忠自稱刺史

三年春正月甲辰遣殿中侍御史崔構供奉官郭遵誨詣朱全忠營丙午李茂貞亦遣牙將郭啓期往議和解平盧節度使王師範頗好學|{
	好呼到翻}
以忠義自許為治

有聲迹|{
	聲聞于時而治有實迹所謂名與實稱好呼到翻治直吏翻}
朱全忠圍鳳翔韓全誨以詔書徵藩鎮兵入援乘輿師範見之泣下霑衿曰吾屬為帝室藩屏|{
	乘繩證翻衿音今屏必郢翻}
豈得坐視天子困辱如此各擁彊兵但自衛乎會張濬自長水亦遺之書|{
	遺于季翻}
勸舉義兵師範曰張公言正會吾意夫復何疑|{
	夫音扶復扶又翻}
雖力不足當死生以之時關東兵多從全忠在鳳翔師範分遣諸將詐為貢獻及商販包束兵仗載以小車入汴徐兖鄆齊沂河南孟滑河中陜虢華等州|{
	諸州皆朱全忠所有之地鄆音運陜失冉翻華戶化翻}
期以同日俱發討全忠適諸州者多事泄被擒獨行軍司馬劉鄩取兖州|{
	鄩徐林翻}
時泰寧節度使葛從周悉將其兵屯邢州|{
	朱全忠攻鳳翔使葛從周悉泰寜之兵屯邢州以備河東}
鄩先遣人為販油者入城詗其虚實及兵所從入|{
	詗古永翻又翾正翻}
丙午鄩將精兵五百夜自水竇入比明軍城悉定市人皆不知|{
	比必利翻及也軍城泰寧軍牙城也以此觀之軍人與市人異處營屋之立自唐然矣 考異曰舊紀丙午青州牙將劉鄩陷全忠之兖州又令牙將張厚入奏是日亦竊發於華州殺州將婁敬思唐太祖紀年録是年四月青州帥王師範將劉鄩竊據兖州同日師範將張厚輦戈甲十乘至華州為華人所詰因竊發燔其郛殺華州指揮使婁敬思而去新紀丙午師範取兖州梁太祖實録丙辰青州綱將亂於華而敗是日劉鄩䧟我兖州唐實録亦在丙辰按長歷丙辰正月四日丙午十四日編遺録云魏師及朱友寧告急劉鄩正月四日襲䧟兖州與紀年録等同梁太祖實録多謬誤恐難據令從諸書移置丙午唐祖補紀云天復二年尤誤}
鄩據府舍拜從周母每旦省謁待其妻子甚有恩禮子弟職掌供億如故|{
	省悉景翻鄩料從周必還攻兖州故善視其家}
是日青州牙將張居厚帥壯士二百將小車至華州東城|{
	帥讀曰率下同}
知州事婁敬思疑其有異剖視之其徒大呼|{
	呼火故翻}
殺敬思攻西城崔胤在華州師衆拒之|{
	天復元年十二月崔胤帥百官遷於華州事見上卷}
不克|{
	為崔胤所拒遂不能克華州}
走至商州追獲之|{
	九域志華州南至商州一百八十里}
全忠留節度判官裴迪守大梁師範遣走卒齎書至大梁迪問以東方事走卒色動|{
	走卒謂卒之備趨走者後漢志有門闌走卒}
迪察其有變屏人問之|{
	屏必郢翻又卑正翻}
走卒具以實告迪不暇白全忠亟請馬步都指揮使朱友寧將兵萬餘人東巡兖鄆|{
	亟紀力翻將即亮翻下同}
友寧召葛從周於邢州共攻師範全忠聞變亦分兵先歸使友寧并將之|{
	為朱友寧戰死朱全忠後夷王師範張本}
戊申李茂貞獨見上|{
	見賢遍翻}
中尉韓全誨張彦弘樞密使袁易簡周敬容皆不得對|{
	易以豉翻}
茂貞請誅全誨等與朱全忠和解奉車駕還京上喜即遣内養帥鳳翔卒四十人收全誨等斬之|{
	内養亦宦者也帥讀曰率}
以御食使第五可範為左軍中尉|{
	御食使掌御膳亦唐末所置内諸司使之一也}
宣徽南院使仇承坦為右軍中尉王知古為上院樞密使楊䖍朗為下院樞密使|{
	樞密分東西院東院為上院西院為下院}
是夕又斬李繼筠李繼誨李彦弼及内諸司使韋處廷等十六人|{
	處昌呂翻}
己酉遣韓偓及趙國夫人詣全忠營又遣使囊全誨等二十餘人首以示全忠 |{
	考異曰舊紀丁巳蔣玄暉與中使押送全誨等二十人首級告諭四鎮兵士回鑾之期新紀正月戊申殺全誨等唐太祖紀年録正月甲辰鳳翔李茂貞殺其子繼筠觀軍容韓全誨張彦弘樞密使袁易簡周敬容等二十二人皆斬首囊盛押領出城以示朱温金鑾記六日誅全誨等唐年補録正月癸卯賜朱全忠詔唐補紀云天復三年二月誅全誨等八人其全誨等伏誅日今從金鑾記實録新紀按金鑾記唐年補録唐實録後唐紀年録載六日所誅宦官名可見者全誨等四人處廷等十六人而金鑾記云是夜處置内官一十九人唐年補録云全誨以下二十二人首級紀年録殺全誨等二十二人北夢瑣言亦云二十二人首新傳繼筠繼誨彦弼皆伏誅是夜誅内諸司使韋處廷等二十二人若并繼筠等數之則多一人若只數宦官則少二人若如金鑾記是夜又誅十九人則多一人或者二人名不見歟}
曰曏來脅留車駕懼罪離間|{
	間古莧翻}
不欲恊和皆此曹也今朕與茂貞決意誅之卿可曉諭諸軍以豁衆憤辛亥全忠遣觀察判官李振奉表入謝|{
	朱全忠先此以李振為天平節度副使今蓋為四鎮觀察判官}
全誨等已誅而全忠圍猶未解茂貞疑崔胤教全忠欲必取鳳翔白上急召胤令帥百官赴行在|{
	帥讀曰率}
凡四降詔三賜朱書御札|{
	薛史載莊宗朝段徊奏曰唐制或歲時災歉國用不足天子將求經濟之要則内出朱書御札以訪羣臣}
言甚切至悉復故官爵胤竟稱疾不至茂貞懼自致書于胤辭甚卑遜全忠亦以書召胤且戲之曰吾未識天子須公來辨其是非胤始來|{
	崔胤其初所以未敢來者待朱全忠之命耳然君命累召而不來朱全忠一書而遽至人臣事君者必知所先後輕重矣}
甲寅鳳翔始啟城門丙辰全忠巡諸寨至城北有鳳翔兵自北山下全忠疑其逼已遣兵擊之擒其將李繼欽上遣趙國夫人馮翊夫人詣全忠營詰其故|{
	二夫人于内命婦爵秋有國郡之殊詰者詰其已和解而復遣兵相擊}
全忠遣親吏蔣玄暉奉表入奏李茂貞請以其子侃尚平原公主又欲以蘇檢女為景王祕妃以自固平原公主何后之女也后意難之上曰且令我得出|{
	嗚呼唐昭宗惟幸于得出徐令全忠取平原茂貞必不敢距豈知夫婦委命于全忠不復有能取之者乎}
何憂爾女后乃從之壬戌平原公主嫁宋侃|{
	嫌于同姓嫁娶故復侃本姓}
納景王妃蘇氏|{
	古者猶謂師昏為非禮唏矣}
時鳳翔所誅宦官已七十二人朱全忠又密令京兆搜捕致仕不從行者誅九十人甲子車駕出鳳翔幸全忠營全忠素服待罪命客省使宣旨釋罪|{
	時客省使蓋通知閤門事故令宣旨釋罪}
去三仗止報平安|{
	唐制正衙有親勲翊三衛立仗左右金吾將軍以一人報平安去三仗者恐全忠以羽衛之嚴不敢入也 考異曰王禹偁五代史闕文曰昭宗佯為鞋糸脱呼梁祖曰全忠為吾繫鞋梁祖不得已跪而結之流汗浹背時天子扈蹕尚有衛兵昭宗意謂左右擒梁祖以殺之其如無敢動者自是梁祖被召多不至其後盡去昭宗禁衛皆用汴人矣按全忠時擁十萬之衆昭宗方脱茂貞虎口託身全忠豈敢遽為此謀或者欲効漢高祖之折黥布亦恐昭宗不能辦耳今不取去羌呂翻}
以公服入謝|{
	唐章服之制有朝服公服朝服具服也公服從省服也}
全忠見上頓首流涕上命韓偓扶起之上亦泣曰宗廟社稷賴卿再安朕與宗族賴卿再生親解玉帶以賜之少休即行全忠單騎前導十餘里上辭之|{
	此皆朱全忠繆為恭敬也}
全忠乃令朱友倫將兵扈從自留部分後隊焚撤諸寨|{
	從才用翻下同分扶問翻}
友倫存之子也|{
	存全忠仲兄也}
是夕車駕宿岐山丁卯至興平崔胤始帥百官迎謁|{
	帥讀曰率}
復以胤為司空門下侍郎同平章事領三司如故|{
	車駕至鳳翔貶崔胤官今復之}
己巳入長安庚午全忠崔胤同對胤奏國初承平之時宦官不典兵預政天寶以來宦官浸盛貞元之末分羽林衛為左右神策軍以便衛從始令宦官主之以二千人為定制|{
	神策軍入衛苑中自代宗魚朝恩始德宗貞元末始分為左右從才用翻}
自是參掌機密奪百司權上下彌縫共為不法大則構扇藩鎮傾危國家小則賣官鬻獄蠧害朝政|{
	朝直遙翻}
王室衰亂職此之由不翦其根禍終不已請悉罷諸司使其事務盡歸之省寺諸道監軍俱召還闕下上從之是日全忠以兵驅宦官第五可範等數百人於内侍省盡殺之 |{
	考異曰舊紀辛未内官第五可範已下七百人並賜死於内侍省金鑾記二十八日處置第五可範已下四百五十人太祖紀年録内諸司百餘人及隨駕鳳翔羣小二百餘人一時斬首于内侍省舊傳與紀年録同新傳胤全忠議誅第五可範等八百餘人于内侍省梁太祖實録己巳翌日誅宦官第五可範等五百餘人于内侍省仍命畿内及諸道搜索處置以盡厥類唐年補録云誅宦官七百一十人按舊紀編遺録皆云正月辛未誅可範等而梁實録唐補紀續寶運録金鑾記唐年補録薛居正五代史梁紀新唐紀或云己巳翌日或云二十八日今從之蓋全忠胤雖奏云罷諸司使務追監軍赴闕其實即日已擅誅之至二月癸酉始下詔賜死故昭宗哀而祭之耳}
寃號之聲徹於内外|{
	號戶刀翻徹敕列翻}
其出使外方者詔所在收捕誅之|{
	使疏吏翻下同}
止留黄衣幼弱者三十人以備洒掃|{
	宦官品秩之卑者衣黄洒所賣翻又如字掃蘇報翻又如字}
又詔成德節度使王鎔選進五十人充敕使取其土風深厚人性謹樸也上愍可範等或無罪為文祭之自是宣傳詔命皆令宫人出入其兩軍内外八鎮兵悉屬六軍|{
	謂左右神策所統内外八鎮兵也}
以崔胤兼判六軍十二衛事

臣光曰宦官用權為國家患其來久矣蓋以出入宫禁人主自幼及長|{
	長知兩翻}
與之親狎非如三公六卿進見有時可嚴憚也|{
	見賢遍翻}
其間復有性識儇利|{
	儇許緣翻智也疾也利也}
語言辯給伺候顔色承迎志趣|{
	伺相吏翻}
受命則無違迕之患使令則有稱㥦之効|{
	迕五故翻稱尺證翻㥦與愜同詰叶翻}
自非上智之主燭知物情慮患深遠侍奉之外不任以事則近者日親遠者日疎甘言卑辭之請有時而從浸潤膚受之愬有時而聽|{
	論語孔子曰浸潤之譛膚受之愬不行焉可謂明也已矣朱熹註云浸潤如水之浸灌滋潤漸漬而不驟也膚受謂肌膚所受利害切身者也}
於是黜陟刑賞之政潜移于近習而不自知如飲醇酒嗜其味而忘其醉也黜陟刑賞之柄移而國家不危亂者未之有也東漢之衰宦官最名驕横|{
	横戶孟翻}
然皆假人主之權依憑城社|{
	宦官在人主左右有所依憑如城狐社鼠不畏熏燒}
以濁亂天下未有能劫脇天子如制嬰兒廢置在手東西出其意使天子畏之若乘虎狼而挾蛇虺|{
	虺許鬼翻}
如唐世者也所以然者非它漢不握兵唐握兵故也太宗鑒前世之弊深抑宦官無得過四品明皇始隳舊章是崇是長|{
	宋祁曰太宗詔内侍省不立三品官以内侍為之長階第四不任以事惟門閤守禦廷内掃除稟食而已武后時稍增其人至中宗黄衣乃二千員七品以上員外置千員然衣朱紫者尚少玄宗承平日久財用富足志大事奢不愛惜賞賜爵位開元天寶中宦官黄衣以上三千員衣朱紫者千餘人其稱旨者輒拜三品將軍列戟於門其在殿頭供奉委任華重長知兩翻}
晚節令高力士省決章奏|{
	省悉景翻}
乃至進退將相時與之議自太子王公皆畏事之宦官自此熾矣及中原板蕩肅宗收兵靈武李輔國以東宫舊隸參豫軍謀寵過而驕不能復制|{
	復扶又翻}
遂至愛子慈父皆不能庇以憂悸終|{
	悸其季翻}
代宗踐阼仍遵覆轍程元振魚朝恩相繼用事竊弄刑賞壅蔽聰明視天子如委裘|{
	賈誼曰臥赤子天下之上而安植遺腹朝委裘而天下不亂孟康注云委裘若容衣天子未坐朝事天子裘衣也朝直遥翻下同}
陵宰相如奴虜是以來瑱入朝遇讒賜死吐蕃深侵郊甸匿不以聞致狼狽幸陜|{
	陜失冉翻}
李光弼危疑憤鬰以隕其生郭子儀擯廢家居不保丘壟僕固懷恩寃抑無訴遂弃勲庸更為叛亂|{
	更工衡翻改也}
德宗初立頗振綱紀宦官稍絀|{
	絀讀曰黜}
而返自興元猜忌諸將以李晟渾瑊為不可信悉奪其兵而以竇文場霍仙鳴為中尉使典宿衛自是太阿之柄落其掌握矣憲宗末年吐突承璀欲廢嫡立庶以成陳洪志之變寶歷狎䁥羣小|{
	璀七罪翻䁥尼質翻}
劉克明與蘇佐明為逆其後絳王及文武宣懿僖昭六帝皆為宦官所立勢益驕横王守澄仇士良田令孜楊復恭劉季述韓全誨為之魁傑至自稱定策國老目天子為門生根深蔕固疾成膏肓不可救藥矣|{
	左傳晉侯疾病求醫於秦秦伯使醫緩為之未至公夢疾為二豎子曰彼良醫也懼傷我焉逃之其一曰居肓之上膏之下若我何醫至曰疾不可為也在肓之上膏之下攻之不可達之不及藥不至焉不可為也肓音荒鬲也心下為膏}
文宗深憤其然志欲除之以宋申錫之賢猶不能有所為反受其殃况李訓鄭注反覆小人欲以一朝譎詐之謀|{
	譎古穴翻}
翦累世膠固之黨遂至涉血禁塗積尸省戶公卿大臣連頸就誅闔門屠滅天子陽瘖縱酒飲泣吞氣自比赧獻不亦悲乎|{
	瘖于金翻赧奴板翻}
以宣宗之嚴毅明察猶閉目摇首自謂畏之况懿僖之驕侈苟聲色毬獵足充其欲則政事一以付之呼之以父固無怪矣賊汚宫闕|{
	汚烏故翻}
兩幸梁益皆令孜所為也昭宗不勝其耻力欲清滌而所任不得其人所行不由其道始則張濬覆軍於平陽增李克用跋扈之勢復恭亡命於山南啟宋文通不臣之心|{
	李茂貞本宋文通以軍功賜姓名}
終則兵交闕庭矢及御衣漂泊莎城流寓華隂幽辱東内劫遷岐陽|{
	莎素何翻華戶化翻}
崔昌遐無如之何|{
	崔胤字昌遐通鑑稱其字避宋朝太祖廟諱也}
更召朱全忠以討之連兵圍城再罹寒暑御膳不足於糗糒|{
	糗去久翻糒音備}
王侯斃踣於飢寒|{
	踣蒲北翻}
然後全誨就誅乘輿東出翦滅其黨靡有孑遺而唐之廟社因以丘墟矣|{
	此論歷叙唐宦官之禍其事皆具見前紀乘䋲證翻}
然則宦官之禍始于明皇盛于肅代成于德宗極于昭宗易曰履霜堅氷至為國家者防微杜漸可不慎始哉|{
	易坤之初六曰履霜堅氷至象曰履霜堅氷隂始凝也馴致其道至堅氷也文言曰臣弑其君子弑其父非一朝一夕之故其所由來者漸矣}
此其為患章章尤著者也自餘傷賢害能召亂致禍賣官鬻獄沮敗師徒|{
	敗補邁翻}
蠧害烝民不可徧舉夫寺人之官|{
	寺音侍}
自三王之世具載於詩禮|{
	詩有巷伯之篇禮有寺人之職}
所以謹閨闥之禁通内外之言安可無也如巷伯之疾惡|{
	周幽王之時寺人傷于讒而作巷伯之詩記曰好賢如緇衣惡惡如巷伯}
寺人披之事君|{
	左傳晉獻公信讒使寺人披伐公子重耳於蒲城重耳踰垣而出披斬其祛及其反國披請見公使讓之曰蒲城之役君命一宿汝即至其後予從狄君以田渭濱汝為惠公來求殺余命汝三宿汝中宿至雖有君命何其速也對曰君命無二古之制也除君之惡惟力是視蒲人狄人予何有焉今君即位其無蒲狄乎公見之以呂卻之難告公由是得免}
鄭衆之辭賞|{
	事見四十八卷漢和帝永元元年}
呂彊之直諫|{
	事見五十七卷漢靈帝光和二年五十八卷中平元年}
曹日昇之救患馬存亮之弭亂楊復光之討賊嚴遵美之避權|{
	事並見前紀}
張承業之竭忠|{
	事見後梁紀}
其中豈無賢才乎顧人主不當與之謀議政事進退士大夫使有威福足以動人耳果或有罪小則刑之大則誅之無所寛赦如此雖使之專横孰敢焉|{
	横戶孟翻}
豈可不察臧否不擇是非欲草薙而禽獮之|{
	否音鄙薙它計翻獮息淺翻杜預曰獮殺也}
能無亂乎是以袁紹行之于前而董卓弱漢|{
	事見漢靈獻紀}
崔昌遐襲之于後而朱氏簒唐雖快一時之忿而國隨以亡是猶惡衣之垢而焚之|{
	惡烏路翻}
患木之蠧而伐之其為害豈不益多哉孔子曰人而不仁疾之已甚亂也|{
	見論語}
斯之謂矣

王師範遣使以起兵告李克用克用貽書褒贊之河東監軍張承業亦勸克用發兵救鳳翔克用攻晉州聞車駕東歸乃罷 楊行密承制加朱瑾東面諸道行營副都統同平章事以昇州刺史李神福為淮南行軍司馬鄂岳行營招討使舒州團練使劉存副之將兵擊杜洪洪將駱殷戍永興弃城走縣民方詔據城降神福曰永興大縣饋運所仰已得鄂之半矣|{
	永興漢鄂縣地吳分鄂置新陽縣隋改新陽曰永興唐屬鄂州九域志在鄂州東南四百五里今壽昌軍即其地降戶江翻}


資治通鑑卷二百六十三














































































































































