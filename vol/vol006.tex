










 


 
 


 

  
  
  
  
  





  
  
  
  
  
 
  

  

  
  
  



  

 
 

  
   




  

  
  


    資治通鑑卷六     宋 司馬光 撰

  胡三省 音註

  秦紀一【起柔兆敦牂盡昭陽作噩凡二十八年始丙午終癸酉也 陸德明曰秦隴西谷名也在雍州鳥鼠山之東北秦之先非子為周孝王養馬於汧渭之間封為附庸邑之秦谷非子曾孫秦仲周宣王命為大夫仲之孫襄公討西戎救周平王東遷以岐豐之地賜之列為諸侯春秋時稱秦伯}
昭襄王【名稷惠文王庶子也西周既亡天下莫適為主通鑑以秦卒併天下因以昭襄王繫年諡法昭德有勞曰昭辟地有德曰襄以沈約諡法言之則昭襄複諡也卒子恤翻諡神至翻辟讀曰闢}


  五十二年河東守王稽坐與諸侯通棄市【河東本魏地秦取之以其地在大河之東置河東郡守式又翻刑人於市與衆棄之秦法論死于市謂之弃市}
應侯日以不懌【王稽薦范睢於秦王睢既相秦稽亦進用今以罪死故睢日以不懌懌悦也不懌不悦也應於陵翻或曰范睢之初進用於秦至於為相昭襄王誠悦之也鄭安平既降趙王稽又得罪睢雖為相昭襄王臨朝接之日以不悦懌羊益翻}
王臨朝而歎【朝直遙翻}
應侯請其故王曰今武安君死而鄭安平王稽等皆畔内無良將而外多敵國吾是以憂【將即亮翻}
應侯懼不知所出燕客蔡澤聞之【燕於賢翻蔡姓也以國為氏}
西入秦先使人宣言於應侯曰蔡澤天下雄辨之士彼見王必困君而奪君之位應侯怒使人召之蔡澤見應侯禮又倨【倨居御翻傲也}
應侯不快因讓之曰子宣言欲代我相【相息亮翻}
請聞其說蔡澤曰吁【孔安國曰吁疑怪之辭孔穎達曰吁者心有所嫌而為此聲故以為疑怪之辭也}
君何見之晚也夫四時之序成功者去【謂春生夏長秋就實冬閉藏各成其功而相代謝也夫音扶下同}
君獨不見夫秦之商君楚之吳起越之大夫種何足願與【商君事見二卷周顯王三十一年吳起事見一卷安王二十一年大夫種相越王句踐以雪會稽之耻功成不退為句踐所殺種温公音章勇翻與讀曰歟句音鈎踐慈淺翻種章勇翻}
應侯謬曰何為不可此三子者義之至也忠之盡也君子有殺身以成名死無所恨蔡澤曰夫人立功豈不期於成全邪【應於陵翻謬靡幼翻邪音耶}
身名俱全者上也名可法而身死者次也名僇辱而身全者下也【僇與戮同}
夫商君吳起大夫種其為人臣盡忠致功則可願矣閎夭周公豈不亦忠且聖乎三子之可願孰與閎夭周公哉【閎夭周文王武王之賢臣閔音宏夭於驕翻又於表翻}
應侯曰善蔡澤曰然則君之主惇厚舊故不倍功臣孰與孝公楚王越王【倍與背同蒲昧翻}
曰未知何如蔡澤曰君之功能孰與三子曰不若蔡澤曰然則君身不退患恐甚於三子矣語曰日中則移月滿則虧進退嬴縮【五星早出為嬴晚出為縮嬴餘輕翻縮所六翻}
與時變化聖人之道也今君之怨已讐而德已報【怨已讐謂殺魏齊德已報謂進用王稽鄭安平等}
意欲至矣而無變計竊為君危之【為于偽翻}
應侯遂延以為上客因薦於王王召與語大悦拜為客卿應侯因謝病免王新悦蔡澤計畫遂以為相國澤為相數月免【相息亮翻}
 楚春申君以荀卿為蘭陵令【姓譜荀本姓郇後去邑為荀又晉荀林父公族隰叔之後班志蘭陵縣屬東海郡史記正義曰今沂州承縣有蘭陵山}
荀卿者趙人名況嘗與臨武君論兵於趙孝成王之前王曰請問兵要臨武君對曰上得天時下得地利觀敵之變動後之發先之至此用兵之要術也【後先皆去聲}
荀卿曰不然臣所聞古之道凡用兵攻戰之本在乎一民弓矢不調則羿不能以中六馬不和則造父不能以致遠【羿古之善射者造父古之善御者也羿音詣中竹仲翻父音甫}
士民不親附則湯武不能以必勝也故善附民者是乃善用兵者也故兵要在乎附民而已臨武君曰不然兵之所貴者勢利也所行者變詐也善用兵者感忽悠闇【楊倞曰感忽恍惚也悠闇謂遠視不分之貌}
莫知所從出孫吳用之無敵於天下豈必待附民哉荀卿曰不然臣之所道仁人之兵王者之志也君之所貴權謀勢利也仁人之兵不可詐也彼可詐者怠慢者也露袒者也【露袒如人之支體上下無衣裳以覆蔽祼露肉袒者也}
君臣上下之間滑然有離德者也【滑音骨亂也}
故以桀詐桀猶巧拙有幸焉以桀詐堯譬之以卵投石以指橈沸【橈奴巧翻又奴敎翻攪也}
若赴水火入焉焦没耳故仁人之兵上下一心三軍同力臣之於君也下之於上也若子之事父弟之事兄若手臂之扞頭目而覆胸腹也【覆敷救翻蓋也}
詐而襲之與先驚而後擊之一也且仁人用十里之國則將有百里之聽用百里之國則將有千里之聽用千里之國則將有四海之聽必將聰明警戒和傳而一【康曰將音將帥之將余據文義讀如字為通傳音附}
故仁人之兵聚則成卒【百人為卒}
散則成列延則若莫邪之長刃【莫邪吳之寶劒也說文莫邪長戟也邪音耶}
嬰之者斷兌則若莫邪之利鋒當之者潰【兌劉向新序作鋭楊倞曰兌猶聚也讀與隊同倞音諒}
圜居而方止則若盤石然觸之者角摧而退耳且夫暴國之君將誰與至哉【夫音扶}
彼其所與至者必其民也其民之親我歡若父母其好我芬若椒蘭彼反顧其上則若灼黥若仇讐人之情雖桀跖豈有肯為其所惡賊其所好者哉【字書仇讐皆匹也說文仇讐也讐猶應也左傳怨耦曰仇記曰父之讐不與共戴天蓋謂仇之初匹也至于耦而成怨則為仇讐校也兩本相對覆校是非也殺父之人一旦相對覆校是非則不共戴天矣仇讐之義至此為甚後世率以為言好呼到翻為于偽翻惡烏路翻}
是猶使人之子孫自賊其父母也彼必將來告夫又何可詐也故仁人用國日明諸侯先順者安後順者危敵之者削反之者亡詩曰武王載發有度秉鉞如火烈烈則莫我敢遏此之謂也【商頌之辭武王湯也發依商頌讀為斾古者軍將戰則建斾}
孝成王臨武君曰善請問王者之兵設何道何行而可【楊倞曰設謂制置道謂論說教令也行謂動用也}
荀卿曰凡君賢者其國治君不能者其國亂隆禮貴義者其國治簡禮賤義者其國亂治者彊亂者弱是彊弱之本也【冶直吏翻}
上足卬則下可用也上不足卬則下不可用也【卬古仰字音魚向翻楊倞曰下託上曰仰}
下可用則彊下不可用則弱是彊弱之常也齊人隆技擊【孟康曰技擊者兵家之技巧習手足便器械積機關以立攻守之勝者也楊倞曰技材力也齊人以勇力擊斬敵者號為技擊隆重也技渠綺翻}
其技也得一首者則賜贖錙金無本賞矣【楊倞曰八兩曰錙本賞謂有功同受賞也其技擊之術斬得一首則官賜以錙金贖之斬首雖戰敗亦賞不斬首雖勝亦不賞是無本賞夹錙莊持翻}
是事小敵毳則偷可用也【毳與脆同音此芮翻}
事大敵堅則渙焉離耳若飛鳥然傾側反覆無日是亡國之兵也兵莫弱是矣是其去賃市傭而戰之幾矣【賃女禁翻毛晃曰借也僦也市傭謂市人之受雇者也}
魏氏之武卒以度取之【楊倞曰選擇武勇之士號為武卒度取之謂取長短材力之中度者也}
衣三屬之甲【如淳曰三身一髀禈一脛繳一凡三屬衣于既翻屬之欲翻}
操十二石之弩【沈括曰鈞石之石五權之名石重百二十斤後人以一斛為一石自漢時已如此于定國飲酒一石不亂是也挽彊弓弩古人以鈞石率之今人乃以秔米一斛之重為一石凡石以九十二斤半為法乃漢秤三百四十一斤也今之武卒蹶弩有及九石者計其力乃古二十五石比魏之武卒當二人有餘弓有挽三石者乃古之二十四鈞比顔高之弓當五人有餘此皆近世教習所致武備之盛前古未有其比按括之論詳矣然用之則誤國喪師不知合變是趙括之談兵也操七刀翻}
負矢五十箇置戈其上【謂置戈于身之上即荷戈也荷下可翻}
冠胄帶劒贏三日之糧日中而趨百里中試則復其戶利其田宅【中試言程試而中度者復其戶不徭役也利其田宅給以田宅便利之處胄今之兜鍪冠古玩翻贏怡成翻擔也中竹仲翻復方目翻}
是其氣力數年而衰而復利未可奪也改造則不易周也【改造謂更選擇也易弋豉翻}
是故地雖大其稅必寡是危國之兵也秦人其生民也陿隘其使民也酷烈刼之以勢隱之以阨忸之以慶賞鰌之以刑罰【陿與狹同隘烏懈翻楊倞曰隱之以阨謂隱蔽以險阨使敵不能害鄭氏曰秦地多阨隱藏其民于阨中也忸與狃同串習也戰勝則與之慶賞使習以為常鰌藉也不勝則以刑罰陵藉之莊子風謂蛇曰鰌我亦勝我陸德明音義曰鰌音秋藉也李云鰌藉也藉則削也忸女九翻}
使民所以要利於上者非鬬無由也使以功賞相長五甲首而隸五家【楊倞曰有功則賞之使相長凡獲得五甲首則役隷鄉里之五家也要一遙翻長知兩翻}
是最為衆彊長久之道故四世有勝非幸也數也【四世謂秦孝公惠文王悼武王昭襄王}
故齊之技擊不可以遇魏之武卒魏之武卒不可以遇秦之鋭士秦之鋭士不可以當桓文之節制桓文之節制不可以當湯武之仁義有遇之者若以焦熬投石焉【焦熬之物至脆投石則碎熬五刀翻}
兼是數國者皆干賞蹈利之兵也傭徒鬻賣之道也未有貴上安制綦節之理也【楊倞曰干賞蹈利之兵與傭徒之人鬻賣其力而作者無異未有愛貴其上而為之致死安於制度自不踰越極于節義心不為非之理也}
諸侯有能微妙之以節則作而兼殆之耳【楊倞曰微妙精盡也節仁義也作起也殆危也諸侯有能精盡仁義則起而兼此數國使之危殆}
故招延募選隆勢詐上功利是漸之也【漸浸漬也言勢詐功利漸染以成俗漸子廉翻}
禮義敎化是齊之也故以詐遇詐猶有巧拙焉以詐遇齊譬之猶以錐刀墮泰山也【謂禮義敎化之所齊以詐遇之無不敗者墮讀曰隳}
故湯武之誅桀紂也拱挹指麾而彊暴之國莫不趨使【挹一及翻義與揖同}
誅桀紂若誅獨夫故泰誓曰獨夫紂此之謂也故兵大齊則制天下小齊則治鄰敵【冶直之翻}
若夫招延募選隆勢詐上功利之兵【夫音扶下同}
則勝不勝無常代翕代張代存代亡相為雌雄耳夫是謂之盗兵君子不由也孝成王臨武君曰善請問為將【將即亮翻}
荀卿曰知莫大於棄疑【楊倞曰不用疑謀此智之大知讀曰智}
行莫大於無過【行下孟翻}
事莫大於無悔事至無悔而止矣不可必也【言不可自以為必勝}
故制號政令欲嚴以威慶賞刑罰欲必以信處舍收藏欲周以固【楊倞曰處舍營壘也收藏財物也周密嚴固則敵不得而陵奪也處昌呂翻}
徙舉進退欲安以重欲疾以速窺敵觀變欲潜以深欲伍以參【楊倞曰謂使間諜觀敵欲潜隱深入之也伍參猶錯雜也使間諜或參之或伍之于敵之間而盡知其事韓子曰省同異之言以知朋黨之分偶參伍之驗以責陳言之實又曰參之以比物伍之以合參}
遇敵决戰必行吾所明無行吾所疑夫是之謂六術無欲將而惡廢【言欲為將而惡失權則舍己之勝筭遷就以逢君之欲矣將即亮翻}
無怠勝而忘敗無威内而輕外無見其利而不顧其害凡慮事欲熟而用財欲泰夫是之謂五權【夫音扶}
將所以不受命於主有三可殺而不可使處不完可殺而不可使擊不勝可殺而不可使欺百姓夫是之謂三至【楊倞曰至謂守一而不變處昌呂翻}
凡受命於主而行三軍三軍既定百官得序【楊倞曰百官軍之百吏也}
羣物皆正則主不能喜敵不能怒夫是謂之至臣慮必先事而申之以敬慎終如始始終如一夫是之謂大吉【楊倞曰言必無覆敗之禍}
凡百事之成也必在敬之其敗也必在慢之故敬勝怠則吉怠勝敬則滅計勝欲則從欲勝計則凶戰如守行如戰有功如幸敬謀無曠敬事無曠敬吏無曠敬衆無曠敬敵無曠夫是之謂五無曠【曠廢也夫音扶}
慎行此六術五權三至而處之以恭敬無曠夫是之謂天下之將則通於神明矣臨武君曰善請問王者之軍制荀卿曰將死鼓【將即亮翻將建旗伐鼓以令三軍之進退死不離局離力智翻}
御死轡百吏死職士大夫死行列【行戶剛翻}
聞皷聲而進聞金聲而退順命為上有功次之令不進而進猶令不退而退也【令力正翻}
其罪惟均不殺老弱不獵禾稼服者不禽格者不赦奔命者不獲【楊倞曰服謂不戰而退者不追禽之格謂相拒捍者奔命謂奔走來歸其命不獲之以為囚俘}
凡誅非誅其百姓也誅其亂百姓者也百姓有捍其賊則是亦賊也以其順刃者生傃刃者死奔命者貢【楊倞曰傃向也謂傃向格鬬者貢謂取歸命者獻於上將也傃音素}
微子開封於宋【殷紂暴虐微子奔周武王殺紂封微子於宋微子本名啓此云開者蓋漢景帝諱劉向改之也}
曹觸龍斷於軍【楊倞曰說苑云桀為天子其臣有左師觸龍者諂諛不正此云紂當是說苑誤案戰國時趙亦有左師觸龍豈姓名同乎姓譜曹姓本自顓頊玄孫陸終之子六安周武王封曹挾於邾故邾曹姓也至魏武帝始祖曹叔振鐸}
商之服民所以養生之者無異周人故近者謌謳而樂之遠者竭蹶而趨之【以上下文觀之商周二字恐或倒置楊倞曰竭蹶顛仆猶言匍匐也樂音洛蹶居月翻}
無幽閒辟陋之國莫不趨使而安樂之【閒讀曰閑辟讀曰僻}
四海之内若一家通達之屬莫不從服夫是之謂人師【夫音扶}
詩曰自西自東自南自北無思不服此之謂也【引文王有聲之詩而言}
王者有誅而無戰城守不攻兵格不擊敵上下相喜則慶之不屠城不潜軍不留衆師不越時故亂者樂其政不安其上欲其至也【亂國之民樂吾之政故不安其上惟欲吾兵之至也樂音洛}
臨武君曰善陳囂問荀卿曰【囂虛驕翻又牛刀翻}
先生議兵常以仁義為本仁者愛人義者循理然則又何以兵為凡所為有兵者為爭奪也【爲爭于僞翻}
荀卿曰非汝所知也彼仁者愛人愛人故惡人之害之也義者循理循理故惡人之亂之也【惡烏路翻}
彼兵者所以禁暴除害也非爭奪也 燕孝王薨子喜立周民東亡【義不為秦民也}
秦人取其寶器遷西周公於狐

  之聚【此西周文公也武公之子也自赧王時東西分治赧王擁虛器而已班志河南郡梁縣有□狐聚括地志汝州外古梁城即狐聚也陽人故城即陽人聚也在汝州梁縣西四十里秦遷東周所居也梁亦古梁城也在汝州梁縣西南十五里索隱曰狐聚與陽人聚在洛陽南北五十里梁新城之間也與憚同聚賢曰慈諭翻}
 楚王遷魯於莒而取其地【魯至是而亡莒居許翻}


  五十三年摎伐魏取吳城【後漢志河東郡大陽縣有吳山山上有虞城杜預曰虞國也帝王世紀曰舜妃嬪于虞虞城是也亦謂吳城秦昭王伐魏取吳城是也摎紀虬翻}
韓王入朝魏舉國聽令【朝直遙翻令力政翻}


  五十四年王郊見上帝於雍【班志雍縣屬扶風秦惠公都之有五畤故於此郊見上帝欲行天子之禮也應劭曰四方積高曰雍凡下見上之見音賢遍翻雍於用翻畤音止}
 楚遷於鉅陽【赧王三十七年楚自郢東北徙於陳今自陳徙鉅陽至始皇六年春申君以朱英之言自陳徙夀春則此時雖徙鉅陽未離陳地也赧奴版翻郢以井翻離力智翻}


  五十五年衛懷君朝於魏魏人執而殺之更立其弟是為元君【更工衡翻}
元君魏壻也【壻女夫妻謂夫亦曰壻旁從女或從士思繼翻}
五十六年秋王薨孝文王立尊唐八子為唐太后【薨呼肱翻七子八子秦宮中女官名}
以子楚為太子趙人奉子楚妻子歸之韓王衰絰入弔祠【賢曰喪服斬衰裳上曰衰下曰裳麻在首要皆曰絰首絰象緇布冠要絰象大帶絰之言實衰之言摧明中實摧痛也衰七雷翻}
 燕王喜使栗腹約歡於趙【姓譜栗姓栗陸氏之後燕因肩翻}
以五百金為趙王酒反而言於燕王曰趙壯者皆死長平【長平之敗事見上卷周赧王五十五年}
其孤未壯可伐也王召昌國君樂閒問之對曰趙四戰之國【言其四境皆鄰于彊敵四面拒戰也}
其民習兵不可王曰吾以五而伐一對曰不可王怒羣臣皆以為可乃發二千乘栗腹將而攻鄗【乘繩證翻鄗呼各翻}
卿秦攻代【姓譜卿姓也}
將渠曰與人通關約交以五百金飲人之王【姓譜將亦姓也音即良翻飲於禁翻}
使者報而攻之不祥師必無功【使疏吏翻}
王不聽自將偏軍隨之將渠引王之綬王以足蹴之【蹴子六翻蹋也綬音受}
將渠泣曰臣非自為為王也【為于偽翻}
燕師至宋子【班志宋子縣屬鉅鹿郡}
趙廉頗為將逆擊之敗栗腹於鄗敗卿秦樂乘於代【將即亮翻樂乘趙將也戰國策曰樂乘敗卿秦于代當從之敗補邁翻}
追北五百餘里遂圍燕【燕都薊趙人進圍之}
燕人請和趙人曰必令將渠處和【處昌呂翻處和者主和也}
燕王使將渠為相而處和趙師乃解去【相息亮翻}
 趙平原君卒【卒子恤翻}


  孝文王【索隱曰名柱諡法五宗安之曰孝慈惠愛民曰文}


  元年冬十月己亥王即位三日薨子楚立是為莊襄王尊華陽夫人為華陽太后夏姬為夏太后【姓譜周封夏后氏于杞非為後不得封者以夏為氏一曰陳夏徵舒之後夏姬生莊襄王故尊為太后華戶化翻夏戶雅翻}
 燕將攻齊聊城拔之【聊城在濟水之北班志聊城縣屬東郡}
或譛之燕王燕將保聊城不敢歸齊田單攻之歲餘不下魯仲連乃為書約之矢以射城中【燕因肩翻將即亮翻約之矢謂以書圜繞束縳于矢也射而亦翻}
遺燕將為陳利害【遺于季翻為于偽翻}
曰為公計者不歸燕則歸齊今獨守孤城齊兵日益而燕救不至將何為乎燕將見書泣三日猶豫不能自决欲歸燕已有隙欲降齊所殺虜於齊甚衆恐已降而後見辱喟然歎曰與人刃我寧我自刃遂自殺【降戶江翻喟丘貴翻言與其使人加刃于我寧使我拔刃而自殺也}
聊城亂田單克聊城【用大師曰克}
歸言魯仲連於齊欲爵之仲連逃之海上曰吾與富貴而詘于人【詘曲勿翻禮記不充詘于富貴詘者喜失節貌予謂此詘即屈伸之屈}
寧貧賤而輕世肆志焉魏安釐王問天下之高士於子順【釐讀曰僖}
子順曰世無其人也抑可以為次其魯仲連乎王曰魯仲連彊作之者【彊其兩翻}
非體自然也子順曰人皆作之作之不止乃成君子作之不變習與體成則自然也【朱熹曰君子成德之名}


  莊襄王【本名異人改名楚孝文王之中子也諡法勝敵志彊曰莊}


  元年呂不韋為相國【相息亮翻}
 東周君與諸侯謀伐秦王使相國帥師討滅之遷東周君於陽人聚【帥讀曰率聚慈喻翻}
周既不祀【皇甫謐曰周凡三十七王八百六十七年周有天下祀后稷以配天宗祀文王於明堂以配上帝宗廟血食八百六十餘年西周已亡猶幸東周能守其祀東周又為秦所滅則盡不祀矣索隱曰既盡也日食盡曰既言周祚盡滅無主祭祀}
周比亡【比必寐翻及也}
凡有七邑河南洛陽穀城平陰偃師鞏緱氏【班志河南縣故郟鄏地周武王遷九鼎周公營以為都是為王城平王居之洛陽周公遷殷民於此是為成周師古曰穀城即今新安應劭曰平陰在平城北故曰平陰班志河南郡之平縣即平城也括地志曰故穀城在洛州河南縣西北十八里苑中河陰縣城本漢平陰縣在洛州洛陽縣東北五十里十三州志曰在平津大河之南魏文帝改曰河陰劉昭曰偃師帝嚳所都盤庚復南亳是為西亳鞏古鞏伯國周之東周公所居緱氏周大夫劉子邑宋白曰緱氏春秋之滑國已上七邑漢皆屬河南郡緱工侯翻郟音夾鄏音辱}
 以河南洛陽十萬戶封相國不韋為文信侯【相息亮翻}
 蒙驁伐韓【驁五到翻又五刀翻}
取成臯滎陽【班志滎陽縣屬河南郡滎澤在其南唐屬鄭州}
初置三川郡 楚滅魯遷魯頃公於卞【春秋夫人姜氏會齊侯于卞即其地班志卞縣屬魯郡頃音傾}
為家人【家人猶今所謂齊民也}
二年日有食之 蒙驁伐趙取榆次狼孟等三十七城【班志榆次狼孟二縣並屬太原郡榆次即左傳涂水梗陽之地括地志狼孟故城在并州陽曲縣東北二十六里}
 楚春申君言於楚王曰淮北地邊於齊其事急請以為郡而封於江東楚王許之春申君因城吳故墟以為都邑【吳都姑蘇越王句踐滅吳王夫差而吴為墟班志吳縣太伯所邑漢為會稽郡治所句音鈎踐慈演翻}
宮室極盛【春申君相楚楚正弱秦正彊不能為國謀乃營其都而盛宮室何足道也孔穎違曰爾雅云室謂之宮宮謂之室别而言之論其四面穹隆則曰宮因其貯物則曰室室之言實也}
三年王齕攻上黨諸城悉拔之初置太原郡【齕恨勿翻}
 蒙驁帥師伐魏取高都汲【班志高都縣屬上黨郡汲縣屬河内郡括地志高都縣今澤州也汲故城在衛州所理汲縣之西南三十五里帥讀曰率}
魏師數敗【數所角翻}
魏王患之乃使人請信陵君於趙信陵君畏得罪不肯還【信陵君留趙事見上卷周赧王五十八年還從宣翻乂音如字赧奴版翻}
誡門下曰【誡居拜翻敕也}
有敢為魏使通者死【為于偽翻使疏吏翻}
賓客莫敢諫毛公薛公見信陵君曰公子所以重於諸侯者徒以有魏也【康曰重直用切余按文義當音輕重之重}
今魏急而公子不恤一旦秦人克大梁夷先王之宗廟公子當何面目立天下乎語未卒信陵君色變趣駕還魏【卒子恤翻趣讀曰促催也}
魏王持信陵君而泣以為上將軍信陵君使人求援於諸侯諸侯聞信陵君復為魏將【將即亮翻}
皆遣兵救魏信陵君率五國之師敗蒙驁於河外【自春秋至戰國率以黄河之西為河外晉賂秦以可外列城五即其證也驁五到翻敗補邁翻}
蒙驁遁走信陵君追至函谷關抑之而還安陵人縮高之子仕于秦【後漢志汝南郡征羌縣有安陵亭注云即魏安陵君所封地括地志曰陵縣西北十五里李奇云六國時為安陵還從宣翻又音如字縮所六翻}
秦使之守管【班志河南郡中牟縣有管叔邑後漢志中牟縣有管城杜預曰管國也在京縣東北}
信陵君攻之不下使人謂安陵君曰君其遣縮高吾將仕之以五大夫使為執節尉【信陵君使安陵君遣縮高欲使安陵以君諭其民以父諭其子也軍尉之執節者也周執節以使漢執節則使且可以專殺矣}
安陵君曰安陵小國也不能必使其民使者自往請之使吏導使者至縮高之所使者致信陵君之命縮高曰君之幸高也將使高攻管也夫父攻子守人之笑也見臣而下是倍主也父敎子倍亦非君之所喜【夫音扶倍蒲妹翻喜許既翻}
敢再拜辭使者以報信陵君信陵君大怒遣使之安陵君所曰安陵之地亦猶魏也【使疏吏翻之如也往也安陵本魏地魏襄王以封其弟}
今吾攻管而不下則秦兵及我社稷必危矣願君生束縮高而致之若君弗致無忌將發十萬之師以造安陵之城下【造七到翻}
安陵君曰吾先君成侯受詔襄王以守此城也手授太府之憲【太府魏國藏圖籍之府憲法也}
憲之上篇曰臣弑君子弑父有常不赦【有常謂有常法也}
國雖大赦降城亡子不得與焉【降戶江翻與讀曰預}
今縮高辭大位以全父子之義而君曰必生致之是使我負襄王之詔而廢太府之憲也雖死終不敢行縮高聞之曰信陵君為人悍猛而自用此辭必反為國禍【謂為安陵之禍也悍下罕翻又音汗}
吾已全已無違人臣之義矣豈可使吾君有魏患乎乃之使者之舍刎頸而死信陵君聞之縞素辟舍【刎扶粉翻頸居郢翻縞古老翻爾雅曰縞皓也辟讀曰避}
使使者謝安陵君曰無忌小人也困於思慮失言於君請再拜辭罪【安陵受封於魏國者也縮高受㕓於安陵者也縮高之子不為魏民逃歸秦而臣於秦為秦守管時秦加兵於魏欲取大梁安陵儻念魏為宗國縮高儻念其先為魏民見魏之危安敢坐視而不救公子無忌為魏舉師以臨之安陵君則陳太府之憲縮高則陳大臣之義以拒之雖死不避反而求之可謂得其死乎無忌為之縞素辟舍以謝安陵吾亦未知其何所處也}
王使人行萬金於魏以間信陵君求得晉鄙客【信陵君殺晉鄙事見上卷周赧王五十六年間古莧翻}
令說魏王曰公子亡在外十年矣今復為將諸侯皆屬天下徒聞信陵君而不聞魏王矣【令力丁翻說式芮翻將即亮翻}
王又數使人賀信陵君得為魏王未也【數所角翻}
魏王日聞其毁不能不信乃使人代信陵君將兵信陵君自知再以毁廢乃謝病不朝日夜以酒色自娛凡四歲而卒【朝直遙翻卒子恤翻}
韓王往弔其子榮之以告子順子順曰必辭之以禮鄰國君弔君為之主【鄭玄曰君為之主弔臣恩為已也子不敢當主中庭北面哭不拜記曰昔者衛靈公適魯遭季桓子之喪衛君請弔哀公辭不得命公為主客入弔公揖讓升自東階西鄉客升自西階弔公拜興哭}
今君不命子則子無所受韓君也其子辭之 五月丙午王薨【薨呼肱翻}
太子政立生十三年矣國事皆决於文信侯號稱仲父【呂不韋封文信侯仲父以齊桓禮管仲禮之}
 晉陽反【是年秦攻得晉陽置太原郡未久而秦有莊襄王之喪故反}


  始皇帝上【諱政莊襄王子也王并天下自以德兼三皇功過五帝故自號曰皇帝欲傳世以一至萬乃除諡法號始皇帝}


  元年蒙驁擊定之【擊定晉陽也驁五到翻}
 韓欲疲秦人使無東伐乃使水工鄭國為間於秦鑿涇水自仲山為渠【間古莧翻班志涇水出安定郡涇陽縣西开頭山東至馮翊陽陵縣入渭過郡三行千六十里淮南子曰涇水出薄落之山華戎對境圖涇水上接蔚茹水南流至笄頭山西折而東流逕原州涇州界又東流逕邠州乾州之北又東南流至雍州涇陽縣而合于渭師古曰仲山即今九嵏之東仲山也开輕烟翻蔚紆勿翻笄古兮翻雍於用翻嵏祖紅翻}
並北山東注洛【並步浪翻師古曰洛水即馮翊漆沮水程大昌曰禹貢止有漆沮秦漢以來始有洛水所謂洛者班志云源出北地歸德縣北蠻夷中今按其水自入塞後歷鄜坊同三州始入渭孔安國謂自馮翊懷德縣入渭是也漢懷德唐同州朝邑縣是也漆水自華原縣東北同官縣界來沮水自邠州東北來洛在漆沮之東至同州白水縣與漆沮合所謂洛即漆沮者言其本同也沮七余翻鄜音膚邠彼巾翻}
中作而覺【師古曰中作謂用功中道事未竟也覺露也韓之謀露也}
秦人欲殺之鄭國曰臣為韓延數年之命然渠成亦秦萬世之利也乃使卒為之【臣為于偽翻卒子恤翻終也}
注填閼之水漑舄鹵之地四萬餘頃收皆畝一鍾【師古曰注引也填閼謂壅泥也言引淤濁之水灌鹹鹵之田更令肥美故一畝之收至六斛四斗杜佑曰古者百步為畝秦漢以降即二百四十步為畝閼讀曰淤音於據翻舄與潟同音思積翻鹵也鹵亦作滷音郎古翻鹹滷}
關中由是益富饒【饒有餘裕也}
二年麃公將卒攻卷【索隱曰麃邑名麃公史失其姓名麃悲驕翻將即亮翻又音如字卷逵員翻邑名}
斬首三萬 趙以廉頗為假相國伐魏取繁陽【班志繁陽縣屬魏郡應劭曰在繁水之陽括地志繁陽故城在相州内黄縣東北二十七里相息亮翻}
趙孝成王薨子悼襄王立使武襄君樂乘代廉頗廉頗怒攻武襄君武襄君走廉頗出犇魏久之魏不能信用趙師數困於秦趙王思復得廉頗廉頗亦思復用於趙趙王使使者視廉頗尚可用否廉頗之仇郭開多與使者金令毁之廉頗見使者一飯斗米肉十斤被甲上馬以示可用使者還報曰廉將軍雖老尚善飯然與臣坐頃之三遺矢矣【數所角翻復扶用翻使使疏吏翻令力丁翻被皮義翻上時掌翻矢糞也}
趙王以為老遂不召【郭開之間亷頗以其仇也其讒殺李牧則好貨耳讒人罔極其祸國可勝言哉間古莧翻好呼到翻勝音升}
楚人陰使迎之廉頗一為楚將無功曰我思用趙人卒死于壽春【將即亮翻壽秦縣漢屬九冮郡唐為壽州治所始皇六年楚方徙都壽春史終言廉頗之事也卒子恤翻}


  三年大饑【五穀皆不熟為大饑}
 蒙驁伐韓取十二城【驁五到翻}
 趙王以李牧為將伐燕取武遂方城【班志武遂縣屬河間國方城縣屬廣陽國後漢志作方城括地志易州遂城縣戰國時武遂城也方城故城在幽州固安縣南十七里將即亮翻燕因肩翻}
李牧者趙之北邊良將也嘗居代雁門備匈奴【秦置雁門邵在代郡西南匈奴淳維之後本夏后氏之苗裔索隱曰張晏云淳緱以殷時奔北邊又樂彦括地譜曰夏桀無道湯放之鳴條三年而死其子獯粥妻桀之衆妾避居北野隨畜移徙中國謂之匈奴其言夏后苗裔或當然也故應劭風俗通曰殷時曰獯粥改曰匈奴又晉灼云堯時曰葷粥周曰獫狁秦曰匈奴韋昭曰漢曰匈奴葷粥其別名則淳維是其始祖蓋與獯粥是一也獯許云翻粥音育獫虛檢翻}
以便宜置吏市租皆輸入莫府為士卒費【康曰師出無常處所在張幕居之以將帥得稱府故曰莫府莫與幕同一曰莫大也莫府猶言大府}
日擊數牛饗士習騎射【孔潁達曰古人不騎馬故但經記正典無言騎者今言騎者當是周末時射之所起起自黄帝故易繫辭黄帝下九事章云古者弦木為弧剡木為矢弧矢之利以威天下又世本云揮作弓夷牟作矢註云揮夷牟黄帝臣是弓矢起於黄帝矣騎奇寄翻下同剡以冉翻}
謹烽火多間諜【塞上置候望之地邊有警則舉烽漢書音義烽如覆米䉛縣著桔橰頭有寇則舉之燧積薪有寇則燔然之索隱曰字林䉛漉米藪也音一六翻纂要䉛淅箕也烽見敵則舉燧有難則焚烽主晝燧主夜間諜者使之間行以伺敵觀其變動也間古莧翻諜達協翻著直畧翻桔吉屑翻橰音臯漉音鹿淅音析難乃旦翻伺相吏翻}
為約曰匈奴即入盗急入收保【收畜產而自保也}
有敢捕虜者斬匈奴每入烽火謹輒入收保不戰如是數歲亦不亡失匈奴皆以為怯雖趙邊兵亦以為吾將怯【將即亮翻}
趙王讓之【讓青也}
李牧如故王怒使佗人代之歲餘屢出戰不利多失亡邊不得田畜【說文畜許竹翻養也史記正義許又翻又音蓄聚也}
王復請李牧【復扶又翻}
李牧杜門稱病不出王彊起之【杜門塞門以拒絶來者彊其兩翻}
李牧曰必欲用臣如前乃敢奉令王許之李牧至邊如約匈奴數歲無所得終以為怯邊士日得賞賜而不用【言屢賞而不用之以戰也}
皆願一戰於是乃具選車得千三百乘選騎得萬三千匹【車騎皆選其堅良者乘繩證翻騎奇寄翻}
百金之士五萬人【管子曰能禽敵殺將者賞百金將即亮翻}
彀者十萬人【彀古候翻張弓也索隱曰彀謂能射者也}
悉勒習戰大縱畜牧人民滿野匈奴小入佯北不勝以數十人委之【委弃也委之於敵也佯音羊}
單于聞之【單于匈奴首領之稱班書曰單于者廣大之貌言其象天單于然也單音蟬稱處陵翻}
大率衆來入李牧多為奇陳【陳讀曰陣}
張左右翼擊之大破之殺匈奴十餘萬騎滅襜襤【如淳曰襜襤胡名在代地班書作澹林襜都甘翻襤路談翻類篇盧甘翻}
破東胡【東胡其後為鮮卑烏丸服䖍曰在匈奴東故曰東胡}
降林胡【如淳以澹林為東胡以此觀之似是兩種降戶江翻}
單于犇走十餘歲不敢近趙邊【近其靳翻}
先是天下冠帶之國七而三國邊於戎狄【先悉薦翻}
秦自隴以西有緜諸緄戎翟䝠之戎【班志緜諸道屬天水郡西漢之制縣有蠻夷曰道括地志緜諸城在秦州秦嶺縣北五十六里唐貞觀十七年省秦嶺入清水縣韋昭曰緄戎春秋以為大戎師古曰混云夷也史記正義曰緄音昆字當作混余謂昆戎即周之昆夷翟與狄同班志隴西郡有狄道師古曰其地有狄種故曰狄道天水郡有䝠道應劭曰䝠戎邑也狄道晉置武始郡括地志䝠道故城在渭州襄武縣東南三十七里䝠戶官翻}
岐梁涇漆之北有義渠大荔烏氏朐衍之戎【班志岐山在扶風美陽縣西北梁山在馮翊夏陽縣西北師古曰此漆水在新平後漢志扶風漆縣有漆水晉分扶風置新平郡治漆縣班志義渠道屬北地郡括地志唐寧慶二州地又班志馮翊臨晉縣古大荔城括地志同州馮翊縣及朝邑縣本漢臨晉地今朝邑縣東三十步故王城即大荔王城也宋白曰同州馮翊縣古大荔城在今州東三十七里胡邑縣界故王城是也荔力計翻班志安定郡有烏氏縣括地志烏氏故城在涇州安定縣東三十里周之故地後入戎秦惠王取之置烏氏縣氏音支班志北地郡有朐衍縣括地志鹽州古戎狄居之即朐衍戎之地應劭曰朐音煦師古音香于翻康求于翻非}
而趙北有林胡樓煩之戎燕北有東胡山戎【自漢北平無終白狼以北皆大山重谷諸戎居之春秋謂之山戎}
各分散居谿谷自有君長往往而聚者百有餘戎然莫能相一其後義渠築城郭以自守而秦稍蠶食之至惠王遂拔義渠二十五城昭王之時宣太后誘義渠王殺諸甘泉【甘泉在漢馮翊雲陽縣漢起甘泉宮於此誘羊久翻}
遂發兵伐義渠滅之始於隴西北地上郡【隴西唐渭州洮州河州之地北地唐慶州寧州鄜州靈州鹽州之地上郡唐延州綏州銀州之地}
築長城以拒胡趙武靈王北破林胡樓煩築長城自代並陰山下至高闕為塞【徐廣曰五原郡西安陽縣北有陰山陰山在河南陽山在河北酈道元曰余按南河北河及安陽縣以南悉沙阜耳無他異山故廣志云朔方郡移沙七所而無山以擬之是議志之僻也陰山在河東南斯可矣漢郎中侯應曰陰山東西千餘里單于之苑閒也孝武出師攘之于漠北匈奴過之未嘗不哭則此山蓋在沙漠之南也括地志陰山在朔州北塞外突厥界杜佑曰今安北府北山是也安北府治中受降城地志朔方郡臨戎縣北有連山險於長城其山中斷兩峰俱峻名曰高闕水經注河水自窳渾縣東屈而東流逕高闕南闕口有城跨山結局謂之高闕戍劉昫曰高闕北拒大磧口三百里杜佑曰高闕當在豐州河西厥九勿翻降戶冮翻窳以主翻渾戶昆翻磧七迹翻}
而置雲中雁門代郡【史記正義曰雲中故城趙雲中城秦雲中郡在勝州榆林縣東北四千里秦漢之雁門代郡皆在句注陘之北唐之雲翔蔚新武州即其地也若唐之代州雁門郡惟崞繫畤二縣漢雁門郡之舊縣其雁門縣則漢大原郡之廣武縣也五臺則漢太原之慮虒縣也句音鉤陘音刑蔚紆勿翻崞音郭時音止師古曰慮虒音廬夷}
其後燕將秦開為質于胡【姓譜秦本顓頊後子嬰既滅支庶為秦氏余按左傳魯有秦堇父秦姓其來尚矣燕因肩翻將即亮翻質音致父音甫堇凡隱翻}
胡甚信之歸而襲破東胡東胡却千餘里燕亦築長城自造陽至襄平【韋昭曰造陽地名在上谷余按漢書所謂上谷之斗造陽是也杜佑曰晉太康地志自北地郡北行九百里得五原塞又北出九百里得造陽即麟州銀城縣史記燕築長城自造陽至襄平韋昭曰造陽地在上谷未詳孰是史記正義曰上谷今媯州王隱地道志曰郡在谷之頭故以上谷名焉班志襄平縣遼東郡治所燕因肩翻媯居為翻}
置上谷漁陽右北平遼東郡以拒胡【漁陽唐薊州檀州北平唐平州遼東其地在大遼水之東唐嘗置遼州又嘗為安東都護府治所}
及戰國之末而匈奴始大

  四年春蒙驁伐魏取畼有詭【畼徐廣音場索隱音暢類篇又直亮翻伸郎翻}
三月軍罷 秦質子歸自趙趙太子出歸國【質音致}
 七月蝗疫【蝗子始生曰蝝翅成而飛曰蝗以食苗為災疫札瘥瘟也}
令百姓納粟千石拜爵一級 魏安釐王薨子景湣王立【釐讀曰僖湣讀曰閔}


  五年蒙驁伐魏取酸棗燕虛長平雍丘山陽等二十城【驁五到翻括地志酸棗故城在滑州酸棗縣北十五里索隱曰燕虛二邑名春秋桓十二年會于虛赧王四十二年黄歇說秦王曰拔酸棗虛桃按今東郡燕縣東三十里有桃城虛蓋與桃相近括地志南燕城古燕國骨州胙城縣是也桃虛在濮州雷澤縣東十三里燕烏田翻虛如字班志長平縣屬汝南郡括地志在陳州宛丘縣西六十六里班志雍丘縣屬陳留郡故祀國也雍於用翻史記正義曰地理志河内郡有山陽縣余考之上下文此非河内之山陽蓋班志山陽郡之地}
初置東郡 初劇辛在趙與龎煖善【赧王三年劇辛自趙適燕劇竭戟翻煖音許遠翻又許兀翻赧奴版翻}
已而仕燕燕王見趙數困於秦廉頗去而龐煖為將欲因其敝而攻之問於劇辛對曰龎煖易與耳【數所角翻將即亮翻易弋䜴翻}
燕王使劇辛將而伐趙趙龎煖禦之殺劇辛取燕師二萬 諸侯患秦攻伐無已時【以發明年合從伐秦事從子容翻}


  六年楚趙魏韓衛合從以伐秦楚王為從長【從子容翻長知兩翻}
春申君用事取壽陵【徐廣曰壽陵在常山史記正義曰本趙邑也余據五國攻秦取壽陵至函谷則壽陵不在新安宜陽之冏當在河東郡界常山無乃太遠}
至函谷秦師出五國之師皆敗走楚王以咎春申君春申君以此益疎觀津人朱英謂春申君曰【史記正義曰觀音館今魏州觀城縣余按班志觀津縣屬信都國又按隋志魏州之觀城舊曰衛國開皇六年始更名信都國則隋冀州也開皇六年置武邑縣并得觀律縣地則觀津猶屬信都也正義誤矣觀古玩翻}
人皆以楚為彊君用之而弱其於英不然先君時秦善楚二十年而不攻楚何也秦踰黽阨之塞而攻楚不便【劉昭曰江夏郡鄳縣古冥阨之塞也史記正義曰黽阨之塞在申州張守節曰申州羅山縣本漢鄳縣平靖關蓋鄳縣之阨塞括地志曰石城山在申州羅山縣東南二十一里古冥阨塞黽音盲康彌兖切非也阨音厄又於賣翻}
假道於兩周背韓魏而攻楚不可【背蒲妹翻}
今則不然魏旦暮亡不能愛許鄢陵【鄢於幰翻}
魏割以與秦秦兵去陳百六十里臣之所觀者見秦楚之日鬬也楚于是去陳徙壽春命曰郢【郢以井翻}
春申君就封於吳行相事【相息亮翻}
 秦拔魏朝歌【朝歌紂都衛康叔所封也班志朝歌縣屬河内郡}
及衛濮陽衛元君率其支屬徙居野王【班志野王縣屬河内郡濮博木翻}
阻其山以保魏之河内

  七年伐魏取汲 夏太后甍【即夏姬也夏戶雅翻薨呼肱翻}
 蒙驁卒【驁五到翻卒子恤翻}


  八年魏與趙鄴 韓桓惠王薨子安立

  九年伐魏取垣蒲【蒲晉公子重耳所居邑也班志蒲子與垣縣皆屬河東郡括地志故垣城漢縣治本魏地王垣在絳州垣縣西北二十里蒲故城在隰州蒲縣北四十五里垣干元翻重直龍翻}
夏四月寒民有凍死者 王宿雍【雍於用翻}
 己酉王冠【冠古喚翻}
帶劒 楊端和伐魏【姓譜周宣王子尚父幽王邑諸楊號曰楊侯後并於晉因以為氏又晉大夫楊食我食采於楊氏子孫以邑為氏楊食音嗣采倉代翻}
取衍氏【史記正義曰衍氏在鄭州衍羊善翻}
 初王即位年少【少始照翻}
太后時與文信侯私通王益壯文信侯恐事覺禍及已乃詐以舍人嫪毐為宦者進於太后【師古曰嫪居虯翻許慎郎到翻康盧道切毐烏改翻}
太后幸之生二子封毐為長信侯以太原為毐國政事皆决於毐客求為毐舍人者甚衆王左右有與毐爭言者告毐實非宦者王下吏治毐【下遐稼翻冶直之翻}
毐懼矯王御璽發兵欲攻蘄年宮【班志蘄年宮秦惠公所起在雍括地志在岐州城西故城内蘄巨依翻}
為亂【句斷}
王使相國昌平君昌文君發卒攻毐【相息亮翻}
戰咸陽斬首數百毐敗走獲之秋九月夷毐三族【秦有夷三族之罪張晏曰三族父母兄弟妻子也如淳曰父族母族妻族也師古曰如說是所謂參夷之誅也}
黨與皆車裂滅宗舍人罪輕者徙蜀凡四千餘家遷太后於雍萯陽宮【萯陽宮秦文王所起水經注甘水出南山甘谷北逕秦文王萯陽宮西又北逕五柞宮東又北逕甘亭西後漢志甘亭在扶風鄠縣萯音倍}
殺其二子下令曰敢以太后事諫者戮而殺之斷其四支積於闕下死者二十七人【斷丁管翻}
齊客茅焦上謁請諫【姓譜周公之子封于茅其後以國為氏又有茅戎邾大夫有茅地茅夷鴻謁猶今之刺也上謁者通名而求見也上時掌翻}
王使謂之曰若不見夫積闕下者邪【若汝也夫音扶}
對曰臣聞天有二十八宿【二十八宿角亢氐房心尾箕斗牛女虛危室壁奎婁胃昴畢觜參井鬼柳星張翼軫天之經星也日月五星之行躔次所舍故謂之宿宿音秀亢音剛觜即移翻參疏簪翻}
今死者二十七人臣之來固欲滿其數耳臣非畏死者也使者走入白之【使疏吏翻}
茅焦邑子同食者盡負其衣物而逃【邑子同邑之少年也}
王大怒曰是人也故來犯吾趣召鑊烹之【趣讀曰促鑊胡郭翻吳人謂之鍋}
是安得積闕下哉王按劒而坐口正沫出【沫莫曷翻涎也}
使者召之入茅焦徐行至前再拜謁起稱曰臣聞有生者不諱死有國者不諱亡諱死者不可以得生諱亡者不可以得存死生存亡聖主所欲急聞也陛下欲聞之乎【蔡邕獨斷曰陛階陛也與天子言不敢指斥故稱陛下應劭曰陛者升堂之陛王者必有執兵陳於階陛羣臣與至尊言不敢指斥故呼在陛下者以告之因卑以達尊之意若今稱殿下閣下之類斷丁亂翻}
王曰何謂也茅焦曰陛下有狂悖之行【悖蒲妹翻又蒲没翻行下孟翻下同}
不自知邪【邪音耶}
車裂假父【謂嫪毐}
囊撲二弟【以囊盛其人撲而殺之撲弼角翻又普卜翻}
遷母於雍殘戮諫士桀紂之行不至於是矣令天下聞之盡瓦解無嚮秦者臣竊為陛下危之【雍於用翻行下孟翻為于偽翻}
臣言已矣乃解衣伏質【質與鑕同職日翻鐵椹也}
王下殿手自接之曰先生起就衣今願受事【受事者受所敎之事也}
乃爵之上卿王自駕虛左方往迎太后歸於咸陽復為母子如初 楚考烈王無子春申君患之求婦人宜子者甚衆進之卒無子【卒子恤翻}
趙人李園持其妹欲進諸楚王聞其不宜子恐久無寵乃求為春申君舍人已而謁歸【謂謁告而歸也}
故失期而還【欲以發春申君之問也還從宣翻}
春申君問之李園曰齊王使人求臣之妹與其使者飲故失期春申君曰聘入乎【謂已入聘幣否也使疏吏翻}
曰未也春申君遂納之既而有娠【娠音身}
李園使其妹說春申君曰楚王貴幸君雖兄弟不如也今君相楚二十餘年而王無子【周赧王五十三年楚以春申君為相至是二十餘年說式芮翻相息亮翻}
即百歲後將更立兄弟【人謂死後為百歲後}
彼亦各貴其故所親君又安得常保此寵乎非徒然也【言非但如此而已也}
君貴用事久多失禮於王之兄弟兄弟立禍且及身矣今妾有娠而人莫知妾幸君未久誠以君之重進妾於王王必幸之妾賴天而有男則是君之子為王也楚國盡可得孰與身臨不測之禍哉春申君大然之乃出李園妹謹舍而言諸楚王【謹舍者别為舘舍以居之奉衛甚謹也}
王召入幸之遂生男立為太子李園妹為王后李園亦貴用事而恐春申君泄其語陰養死士欲殺春申君以滅口國人頗有知之者楚王病朱英謂春申君曰世有無望之福亦有無望之禍【史記正義曰無望者不望而忽至}
今君處無望之世【正義曰謂生死無常也處昌呂翻}
事無望之主【正義曰謂喜怒不節也}
安可以無無望之人乎【正義曰謂吉凶忽為}
春申君曰何謂無望之福曰君相楚二十餘年矣雖名相國其實王也【相息亮翻}
王今病旦暮薨薨而君相幼主因而當國王長而反政不即遂南面稱孤【長知兩翻不讀曰否}
此所謂無望之福也何謂無望之禍曰李園不治國而君之仇也【左傳曰怨耦曰仇蓋取此義治直之翻}
不為兵而養死士之日久矣王薨李園必先入據權而殺君以滅口此所謂無望之禍也【薨呼肱翻}
何謂無望之人曰君置臣郎中【班書百官表郎掌門戶出充車騎有議郎中郎侍郎郎中韓信曰吾事項王官不過郎中位不過執戟蓋戰國時置此官}
王薨李園先入臣為君殺之此所謂無望之人也【為于偽翻}
春申君曰足下置之李園弱人也僕又善之且何至此朱英知言不用懼而亡去後十七日楚王薨李園果先入伏死士於棘門之内【史記正義曰棘門壽春城門名}
春申君入死士俠刺之【俠讀曰夾蓋夾而刺之魏晉儀衛有俠轂隊亦曰夾轂隊刺七亦翻}
投其首於棘門之外于是使吏盡捕誅春申君之家太子立是為幽王揚子法言曰或問信陵平原孟嘗春申益乎曰上失其政姦臣竊國命何其益乎

  王以文信侯奉先王功大【事見上卷周赧王五十八年}
不忍誅十年冬十月文信侯免相出就國【相息亮翻文信侯國於河南洛陽}
宗室大臣議曰諸侯人來仕者皆為其主遊間耳【謂遊說以間秦之君臣為于偽翻間古莧翻}
請一切逐之于是大索逐客【索山客翻}
客卿楚人李斯亦在逐中行且上書曰昔穆公求士西取由余於戎東得百里奚於宛迎蹇叔於宋求丕豹公孫支於晉【上時掌翻史記戎王使由余使於秦穆公留由余而遺戎王以女樂戎王受而說之乃歸由余由余諫戎王而不聽穆公使人要之由余遂去戎降秦穆公用其謀伐戎并國十二開地千里晉獻公滅虞虜其大夫百里奚以媵於秦百里奚亡秦走宛穆公贖之於楚授以國政奚薦其友蹇叔穆公使人厚幣迎之以為上大夫晉惠公殺其大夫丕鄭其子豹奔秦穆公用之公孫支子桑也余使疏吏翻遺于季翻說讀為悦要一遙翻降戶江翻媵以證翻宛於元翻}
并國二十遂覇西戎孝公用商鞅之法諸侯親服至今治彊惠王用張儀之計散六國之從使之事秦昭王得范睢彊公室杜私門【事並見前治直吏翻從子容翻睢息隨翻}
此四君者皆以客之功由此觀之客何負於秦哉夫色樂珠玉不產於秦而王服御者衆【夫音扶色女色也}
取人則不然不問可否不論曲直非秦者去為客者逐是所重者在乎色樂珠玉而所輕者在乎人民也臣聞太山不讓土壤故能成其大河海不澤細流故能就其深王者不却衆庶故能明其德此五帝三王之所以無敵也今乃弃黔首以資敵國【秦謂民為黔首黔其廉翻黧黑也}
却賓客以業諸侯所謂藉寇兵齎盗糧者也【藉慈夜翻假也借也齎子兮翻持遺也或為資義亦通}
王乃召李斯復其官除逐客之令李斯至驪邑而還【班志京兆新豐縣秦之驪邑古驪戎國也驪山在其南漢高帝七年更名新豐驪呂支翻還從宣翻又音如字更工衡翻}
王卒用李斯之謀陰遣辯士齎金玉遊說諸侯諸侯名士可下以財者厚遺結之不肯者利劒刺之離其君臣之計然後使良將隨其後數年之中卒兼天下【卒子恤翻遺于季翻刺七亦翻又七賜翻將即亮翻}


  十一年趙人伐燕取貍陽【史記正義曰按燕無貍陽疑貍字誤當作漁陽故城在檀州密雲縣南十八里燕漁陽郡城也趙東界至瀛州則檀州在北趙攻燕取漁陽城也康從本字力之切余謂康音是戰國策燕昭王攻齊陽城及貍竊意貍即貍陽也其地當在齊燕境上燕因肩翻}
兵未罷將軍王翦桓齮楊端和伐趙【言伐燕之兵未罷而秦兵來伐也姓譜桓本自姜姓齊桓公後因諡為氏余按齊桓之前有周桓王魯桓公晉有桓莊之族而以姓桓者為祖齊桓亦不通矣齮丘奇翻又去倚翻諡神至翻}
攻鄴取九城王翦攻閼與轑陽【閼於曷翻與音預又音余徐廣曰轑音老在并州十三州志轑陽在上黨西北百八十里蓋唐樂平郡地今之遼州也據十三州志轑當音遼}
桓齮取鄴安陽【鄴縣有安陽城曹魏置安陽縣屬魏郡}
 趙悼襄王薨【薨呼肱翻}
子幽繆王遷立【繆靡幼翻}
其母倡也【倡音昌妓女也}
嬖於悼襄王【嬖卑義翻又博計翻}
悼襄王廢嫡子嘉而立之遷素以無行聞於國【為遷亡趙張本行下孟翻}
 文信侯就國歲餘諸侯賓客使者相望於道請之【使疏吏翻}
王恐其為變乃賜文信侯書曰君何功於秦封君河南食十萬戶何親於秦號稱仲父其與家屬徙處蜀【處昌呂翻}
文信侯自知稍侵恐誅

  十二年文信侯飲酖死【鴆鳥出南方噉蝮蛇以其羽盡酒中飲之立死酖直禁翻}
竊葬其舍人臨者皆逐遷之【臨良鴆翻哭也}
且曰自今以來操國事不道如嫪毐不韋者籍其門視此【操七刀翻嫪居蚪翻毐烏改翻}
揚子法言曰或問呂不韋其智矣乎以人易貨曰誰謂不韋智者歟以國易宗呂不韋之盗穿窬之雄乎【穿穿壁窬穿墻窬音諭又音俞}
穿窬也者吾見擔石矣【擔亦作儋齊人名小罌為儋音都濫翻石斗石也罌於耕翻}
未見雒陽也

  自六月不雨至于八月 發四郡兵助魏伐楚【發關東四郡兵也}


  十三年桓齮伐趙敗趙將扈輒於平陽【齮丘奇翻又去倚翻敗補邁翻將即亮翻扈夏有扈氏之後音戶輒陟涉翻後漢志魏郡鄴縣有平陽城括地志平陽故城在相州臨漳縣西二十五里史記正義曰平陽戰國時屬韓後屬趙若據正義所云則以此平陽為河東之平陽非也當以後漢志括地志為正}
斬首十萬殺扈輒趙王以李牧為大將軍復戰於宜安肥下【復扶又翻括地志宜安故城在常山藁城縣西南二十五里肥下即班志真定國之肥纍縣春秋肥子之國括地志肥纍故城在藁城縣西七里}
秦師敗績【大崩曰敗績}
桓齮犇還趙封李牧為武安君【還從宣翻}


  十四年桓齮伐趙取宜安平陽武城【後漢志魏郡鄴縣有武城史記正義曰即員州武城縣外城是齮丘奇翻又去倚翻}
 韓王納地効璽請為藩臣使韓非來聘【古者列國之於天子比年一小聘三年一大聘璽斯氏翻}
韓非者韓之諸公子也善刑名灋術之學【班志法家者流蓋出於理官信賞必罰以輔禮制鼂錯為申商刑名之學言人主不可不知術數張晏曰術數刑名之書也臣瓚曰術數謂法制治國之術也師古曰瓚說是也公孫弘曰擅殺生之力通雍塞之塗權輕重之數論得失之道使遠近情偽畢見於上謂之術與錯所言同灋古法字鼂古朝字錯千故翻瓚藏旱翻塞悉則翻見賢遍翻}
見韓之削弱數以書干韓王【數所角翻}
王不能用於是韓非疾治國不務求人任賢【冶直之翻}
反舉浮淫之蠧而加之功實之上寛則寵名譽之人急則用介胄之士所養非所用所用非所養悲廉直不容於邪枉之臣觀往者得失之變作孤憤五蠧内外儲說林說難五十六篇十餘萬言【自孤憤至說難皆韓非子篇名索隱曰孤憤者憤孤直不容於時也五蠧者蠧政之事有五也内外儲者韓非子有内外儲說篇内儲者言明君執術以制臣下制之在己故曰内也外儲者明君觀聽臣下之言行以斷其賞罰賞罰在彼故曰外也說林廣說諸事其多若林故曰說林也余謂說難者言游說之難温公揚子註說音税難如字}
王聞其賢欲見之非為韓使於秦因上書說王曰【為于偽翻使疏吏翻上時掌翻說式芮翻}
今秦地方數千里師名百萬號令賞罰天下不如臣昧死願望見大王言所以破天下從之計大王誠聽臣說一舉而天下之從不破【從子容翻}
趙不舉韓不亡荆魏不臣齊燕不親霸王之名不成四鄰諸侯不朝【燕因肩翻朝直遥翻}
大王斬臣以徇國以戒為王謀不忠者也王悦之未任用李斯嫉之曰韓非韓之諸公子也今欲并諸侯非終為韓不為秦此人情也【為于偽翻}
今王不用久留而歸之此自遺患也不如以法誅之王以為然下吏治非李斯使人遺非藥令早自殺【下遐稼翻冶直之翻遺于季翻令力丁翻}
韓非欲自陳不得見王後悔使人赦之非已死矣

  揚子法言曰或問韓非作說難之書而卒死乎說難敢問何反也【知說之難而卒死于說是何其所行與所言反也說式芮翻難如字卒子恤翻}
曰說難蓋其所以死乎曰何也君子以禮動以義止合則進否則退確乎不憂其不合也【確堅也言自信之堅也}
夫說人而憂其不合則亦無所不至矣或曰非憂說之不合【夫音扶此非指韓非子之名}
非邪【此非是非之非邪音耶}
曰說不由道憂也由道而不合非憂也

  臣光曰臣聞君子親其親以及人之親愛其國以及人之國是以功大名美而享有百福也今非為秦畫謀而首欲覆其宗國【謂欲亡韓}
以售其言罪固不容於死矣【言死猶有餘罪也}
烏足愍哉

  十五年王大興師伐趙一軍抵鄴一軍抵太原取狼孟番吾遇李牧而還【番音婆乂音盤還從宣翻又音如字秦軍畏李牧不敢戰而還趙之所恃者李牧而卒殺之以速其亡}
 初燕太子丹嘗質于趙與王善【王之父異人質於趙生王於邯鄲}
王即位丹為質於秦【質音致}
王不禮焉丹怒亡歸【為丹遣荆軻刺秦王張本}


  十六年韓獻南陽地【此漢南陽郡之地時秦楚韓分有之}
九月發卒受地於韓 魏人獻地 代地震自樂徐以西北至平陰【史記正義曰樂徐在晉州平陰在汾州余謂上書代地震則樂徐平陰皆代地也烏得在晉汾二州界水經注徐水出代郡廣昌縣東南大嶺下東北流逕郎山入北平郡界意樂徐之地當在徐水左右又代郡平邑縣王莽曰平湖十三州志平湖城在高柳南百八十里水經注曰代郡道人縣城北有潭淵而不注俗謂之平湖平隂之地蓋在此湖之陰也樂意當音洛}
臺居牆垣太半壞地坼東西百三十步【毛晃曰四方而高曰臺垣于元翻坼斥格翻說文裂也}


  十七年内史勝滅韓【史記本紀作内史騰班書百官表内史周官秦因之掌治京師余按秦内史兼治漢三輔之地始皇并天下置三十六郡内史其一也}
虜韓王安以其地置潁川郡【韓至是而亡潁川郡韓地也韓自平陽徙都河南新鄭韓景侯又自新鄭徙都陽翟秦滅韓遂以為陽翟縣為潁川郡治所}
 華陽太后薨【華戶化翻}
 趙大饑 衛元君薨子角立

  十八年王翦將上地兵下井陘【史記正義曰上郡上縣今綏州是也余謂上地以其地在大河上游凡上郡抵西河之地皆是也應劭曰井陘在常山郡井陘縣西唐謂之土門將即亮翻又音如字陘音刑}
端和將河内兵共伐趙【端和即楊端和此逸楊字}
趙李牧司馬尚禦之秦人多與趙王嬖臣郭開金【嬖卑義翻又博計翻}
使毁牧及尚言其欲反趙王使趙葱及齊將顏聚代之【姓譜顔姓本自魯伯禽支庶有食采顔邑者因而著族又邾武公名夷字曰顔故公羊傳稱顔公後以為氏將即亮翻采食代翻傳直戀翻}
李牧不受命趙人捕而殺之廢司馬尚十九年王翦擊趙軍大破之殺趙葱顔聚亡遂克邯鄲虜趙王遷【趙至是亡邯鄲音寒丹}
王如邯鄲故與母家有仇怨者皆殺之【王毋邯鄲美女也事見上卷周赧王五十八年怨於元翻}
還從太原上郡歸【還從宣翻又音如字}
 太后薨【薨呼肱翻}
 王翦屯中山以臨燕【中山春秋之鮮虞也戰國時為中山國趙滅之以其地為中山郡水經注曰城中有山故曰中山唐之定州即其地也燕因肩翻}
趙公子嘉帥其宗數百人犇代【帥讀曰率}
自立為代王趙之亡大夫稍稍歸之與燕合兵軍上谷【上谷燕地秦置上谷郡唐易州媯州之地括地志上谷郡故城在媯州懷戎縣東北百一十里媯居為翻}
 楚幽王薨國人立其弟郝【郝音釋康曰呵各切}
三月郝庶兄負芻殺之自立魏景湣王薨子假立【湣與閔同}
 燕太子丹怨王【怨王之不禮也}
欲報之以問其傅鞠武【鞠居六翻姓也姓譜云后稷之孫生而有文在手曰鞠因以為氏余謂此傅會之說也}
鞠武請西約三晉南連齊楚北媾匈奴以圖秦太子曰太傅之計曠日彌久令人心惽然恐不能須也【令力丁翻康曰惽音昬恐丘用翻余謂然字句絶言鞠武之計迂遠使人悶然恐如字須待也}
頃之將軍樊於期得罪亡之燕【姓譜周宣王封太王之子虞仲支孫仲山甫於樊後因氏焉}
太子受而舍之【舍如字舘也}
鞠武諫曰夫以秦王之暴而積怒於燕足為寒心又况聞樊將軍之所在乎是謂委肉當餓虎之蹊也願太子疾遣樊將軍入匈奴太子曰樊將軍窮困於天下歸身於丹是固丹命卒之時也【命卒謂命盡也丹言樊將軍以窮來歸當盡死以保匿舍藏之卒子恤翻}
願更慮之鞠武曰夫行危以求安造禍以為福計淺而怨深連結一人之後交不顧國家之大害所謂資怨而助禍矣太子不聽太子聞衛人荆軻之賢【楚國本曰荆此蓋楚未改國號之前受姓也}
卑辭厚禮而請見之謂軻曰今秦已虜韓王又舉兵南伐楚北臨趙趙不能支秦則禍必至於燕燕小弱數困於兵【燕因肩翻數所角翻}
何足以當秦諸侯服秦莫敢合從丹之私計愚以為誠得天下之勇士使於秦【從子容翻使疏吏翻}
刼秦王使悉反諸侯侵地若曹沬之與齊桓公則大善矣不可則因而刺殺之【燕丹於禮致荆軻之初畫兩端之策荆軻守其初說所以事不成要之戰國之士皆祖曹沫之故智若藺相如會秦王毛遂結從於楚之類是也沫音末又讀曰劌刺七亦翻又七賜翻}
彼大將擅兵於外而内有亂則君臣相疑以其間諸侯得合從其破秦必矣唯荆卿留意焉荆軻許之於是舍荆卿於上舍太子日造門下所以奉養荆軻無所不至【閒古莧翻造七到翻}
及王翦滅趙太子聞之懼欲遣荆軻行荆軻曰今行而無信則秦未可親也誠得樊將軍首與燕督亢之地圖【後漢志涿郡方城縣有督亢亭劉向别録曰督亢膏腴之地史記正義曰督亢陂在幽州范陽縣東南十里今固安縣南有督亢陌幽州南界唐會要涿州新城縣太和六年置古督亢地也督都毒翻亢音剛康苦浪翻}
奉獻秦王秦王必說見臣【說讀曰悦}
臣乃有以報太子曰樊將軍窮困來歸丹丹不忍也荆軻乃私見樊於期曰秦之遇將軍可謂深矣父母宗族皆為戮没今聞購將軍首金千斤邑萬家將奈何於期太息流涕曰計將安出荆卿曰願得將軍之首以獻秦王秦王必喜而見臣臣左手把其袖右手揕其胸【揕張鴆翻索隱曰揕謂以劍刺其胸也}
則將軍之仇報而燕見陵之愧除矣【燕因肩翻}
樊於期曰此臣之日夜切齒腐心也【索隱曰切齒相磨切也爾雅曰治骨曰切腐音輔腐亦爛也猶今人事不可忍云腐爛然皆奮怒之意}
遂自刎太子聞之犇往伏哭然已無奈何遂以函盛其首【刎扶粉翻盛時征翻}
太子豫求天下之利匕首使工以藥焠之【焠忽潰翻索隱曰焠染也謂以毒藥染劒鍔也水與火合為焠}
以試人血濡縷人無不立死者【言以匕首試人人血出纔足以霑濡絲縷便立死也康曰血出如絲縷也濡人余翻縷龍主翻}
乃為遣荆軻以燕勇士秦舞陽為之副使入秦

  資治通鑑卷六


    


 


 



 

 
  







 


  
  
 
 
 


  

 















	
	









































 
  



















 





 












  
  
  

 





