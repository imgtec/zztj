<!DOCTYPE html PUBLIC "-//W3C//DTD XHTML 1.0 Transitional//EN" "http://www.w3.org/TR/xhtml1/DTD/xhtml1-transitional.dtd">
<html xmlns="http://www.w3.org/1999/xhtml">
<head>
<meta http-equiv="Content-Type" content="text/html; charset=utf-8" />
<meta http-equiv="X-UA-Compatible" content="IE=Edge,chrome=1">
<title>資治通鑒_24-資治通鑑卷二十三_24-資治通鑑卷二十三</title>
<meta name="Keywords" content="資治通鑒_24-資治通鑑卷二十三_24-資治通鑑卷二十三">
<meta name="Description" content="資治通鑒_24-資治通鑑卷二十三_24-資治通鑑卷二十三">
<meta http-equiv="Cache-Control" content="no-transform" />
<meta http-equiv="Cache-Control" content="no-siteapp" />
<link href="/img/style.css" rel="stylesheet" type="text/css" />
<script src="/img/m.js?2020"></script> 
</head>
<body>
 <div class="ClassNavi">
<a  href="/24shi/">二十四史</a> | <a href="/SiKuQuanShu/">四库全书</a> | <a href="http://www.guoxuedashi.com/gjtsjc/"><font  color="#FF0000">古今图书集成</font></a> | <a href="/renwu/">历史人物</a> | <a href="/ShuoWenJieZi/"><font  color="#FF0000">说文解字</a></font> | <a href="/chengyu/">成语词典</a> | <a  target="_blank"  href="http://www.guoxuedashi.com/jgwhj/"><font  color="#FF0000">甲骨文合集</font></a> | <a href="/yzjwjc/"><font  color="#FF0000">殷周金文集成</font></a> | <a href="/xiangxingzi/"><font color="#0000FF">象形字典</font></a> | <a href="/13jing/"><font  color="#FF0000">十三经索引</font></a> | <a href="/zixing/"><font  color="#FF0000">字体转换器</font></a> | <a href="/zidian/xz/"><font color="#0000FF">篆书识别</font></a> | <a href="/jinfanyi/">近义反义词</a> | <a href="/duilian/">对联大全</a> | <a href="/jiapu/"><font  color="#0000FF">家谱族谱查询</font></a> | <a href="http://www.guoxuemi.com/hafo/" target="_blank" ><font color="#FF0000">哈佛古籍</font></a> 
</div>

 <!-- 头部导航开始 -->
<div class="w1180 head clearfix">
  <div class="head_logo l"><a title="国学大师官网" href="http://www.guoxuedashi.com" target="_blank"></a></div>
  <div class="head_sr l">
  <div id="head1">
  
  <a href="http://www.guoxuedashi.com/zidian/bujian/" target="_blank" ><img src="http://www.guoxuedashi.com/img/top1.gif" width="88" height="60" border="0" title="部件查字,支持20万汉字"></a>


<a href="http://www.guoxuedashi.com/help/yingpan.php" target="_blank"><img src="http://www.guoxuedashi.com/img/top230.gif" width="600" height="62" border="0" ></a>


  </div>
  <div id="head3"><a href="javascript:" onClick="javascript:window.external.AddFavorite(window.location.href,document.title);">添加收藏</a>
  <br><a href="/help/setie.php">搜索引擎</a>
  <br><a href="/help/zanzhu.php">赞助本站</a></div>
  <div id="head2">
 <a href="http://www.guoxuemi.com/" target="_blank"><img src="http://www.guoxuedashi.com/img/guoxuemi.gif" width="95" height="62" border="0" style="margin-left:2px;" title="国学迷"></a>
  

  </div>
</div>
  <div class="clear"></div>
  <div class="head_nav">
  <p><a href="/">首页</a> | <a href="/ShuKu/">国学书库</a> | <a href="/guji/">影印古籍</a> | <a href="/shici/">诗词宝典</a> | <a   href="/SiKuQuanShu/gxjx.php">精选</a> <b>|</b> <a href="/zidian/">汉语字典</a> | <a href="/hydcd/">汉语词典</a> | <a href="http://www.guoxuedashi.com/zidian/bujian/"><font  color="#CC0066">部件查字</font></a> | <a href="http://www.sfds.cn/"><font  color="#CC0066">书法大师</font></a> | <a href="/jgwhj/">甲骨文</a> <b>|</b> <a href="/b/4/"><font  color="#CC0066">解密</font></a> | <a href="/renwu/">历史人物</a> | <a href="/diangu/">历史典故</a> | <a href="/xingshi/">姓氏</a> | <a href="/minzu/">民族</a> <b>|</b> <a href="/mz/"><font  color="#CC0066">世界名著</font></a> | <a href="/download/">软件下载</a>
</p>
<p><a href="/b/"><font  color="#CC0066">历史</font></a> | <a href="http://skqs.guoxuedashi.com/" target="_blank">四库全书</a> |  <a href="http://www.guoxuedashi.com/search/" target="_blank"><font  color="#CC0066">全文检索</font></a> | <a href="http://www.guoxuedashi.com/shumu/">古籍书目</a> | <a   href="/24shi/">正史</a> <b>|</b> <a href="/chengyu/">成语词典</a> | <a href="/kangxi/" title="康熙字典">康熙字典</a> | <a href="/ShuoWenJieZi/">说文解字</a> | <a href="/zixing/yanbian/">字形演变</a> | <a href="/yzjwjc/">金 文</a> <b>|</b>  <a href="/shijian/nian-hao/">年号</a> | <a href="/diming/">历史地名</a> | <a href="/shijian/">历史事件</a> | <a href="/guanzhi/">官职</a> | <a href="/lishi/">知识</a> <b>|</b> <a href="/zhongyi/">中医中药</a> | <a href="http://www.guoxuedashi.com/forum/">留言反馈</a>
</p>
  </div>
</div>
<!-- 头部导航END --> 
<!-- 内容区开始 --> 
<div class="w1180 clearfix">
  <div class="info l">
   
<div class="clearfix" style="background:#f5faff;">
<script src='http://www.guoxuedashi.com/img/headersou.js'></script>

</div>
  <div class="info_tree"><a href="http://www.guoxuedashi.com">首页</a> > <a href="/SiKuQuanShu/fanti/">四库全书</a>
 > <h1>资治通鉴</h1> <!--         下载:【右键另存为】即可 --></div>
  <div class="info_content zj clearfix">
  
<div class="info_txt clearfix" id="show">
<center style="font-size:24px;">24-資治通鑑卷二十三</center>
    資治通鑑卷二十三   宋 司馬光 撰<br />
<br />
  胡三省 音註<br />
<br />
  漢紀十五【起旃蒙協洽盡柔兆敦牂凡十二年】<br />
<br />
  孝昭皇帝上【諱弗陵武帝少子也張晏曰後以二名難諱但名弗荀悅曰諱弗之字曰不應劭曰禮諡法聖聞周逹曰昭】<br />
<br />
  始元元年夏益州夷二十四邑三萬餘人皆反遣水衡都尉呂破胡募吏民及發犍為蜀郡犇命往擊大破之【犍為蜀郡皆屬益州犍為郡唐瀘戎嘉眉榮資簡州地蜀郡唐成都府彭蜀卭雅翼茂州之地應劭曰舊時郡國皆有材官騎士以赴急難今夷反常兵不足以討之故權發精勇聞命犇走故謂之犇命李奇曰平居發二十以上至五十為甲卒今者五十以上六十以下為犇命犇命言急也師古曰應說是余據左傳子重子反一歲七奔命奔命者救急之師固不拘五十以上六十以下也犍居言翻】秋七月赦天下 大雨至於十月渭橋絶 武帝初崩賜諸侯王璽書【左傳襄公在楚季武子使公冶問璽書追而與之盖君臣通用也秦漢以來惟至尊以為信】燕王旦得書不肯哭曰璽書封小【張晏曰文小則封小】京師疑有變遣幸臣夀西長孫縱之王孺等之長安【蘇林曰夀西姓長名師古曰之往也】以問禮儀為名隂刺候朝廷事【刺七亦翻探也】及有詔褒賜旦錢三十萬益封萬三千戶旦怒曰我當為帝何賜也遂與宗室中山哀王子長齊孝王孫澤等結謀【中山哀王昌靖王勝子齊孝王將閭悼惠王肥子】詐言以武帝時受詔得職吏事修武備備非常【如淳曰諸侯不得治民與職事是以詐言受詔得知職事發兵為備也】郎中成軫謂旦曰大王失職獨可起而索【姓譜成姓本自周文王成伯之後周有成肅公又楚有令尹成得臣師古曰失職謂當為漢嗣而不被用也索求也音山客翻】不可坐而得也大王壹起國中雖女子皆奮臂隨大王旦即與澤謀為姦書言少帝非武帝子大臣所共立天下宜共伐之使人傳行郡國以揺動百姓澤謀歸發兵臨菑【臨菑齊郡太守青州刺史治所】殺青州刺史雋不疑【雋辭兖翻】旦招來郡國姦人賦斂銅鐵作甲兵數閱其車騎材官卒發民大獵以講士馬須期日【師古曰講習也須待也余謂澤歸臨菑謀舉兵故旦閲兵以待期數所角翻下同】郎中韓義等數諫旦旦殺義等凡十五人會缾侯成知澤等謀【成菑川靖王之子班志缾侯國屬琅邪郡缾步丁翻】以告雋不疑八月不疑收捕澤等以聞天子遣大鴻臚丞治【續漢志大鴻臚丞秩千石臚陵如翻】連引燕王有詔以燕王至親勿治而澤等皆伏誅遷雋不疑為京兆尹【百官表武帝太初元年改右内史為京兆尹張晏曰地絶高曰京左傳曰莫之與京十億曰兆尹正也師古曰京大也兆者衆數言大衆所在故云京兆也酈道元曰尹正也所以董正京畿率先百郡也孔穎達曰釋詰文曰萬億曰兆依如筭法億之數有大小二法其小數以十為等十萬為億十億為兆也其大數以萬億為等萬至萬是萬之為億又從億而數至萬億曰兆億億曰兆億億曰秭兆在億秭之間】不疑為京兆尹吏民敬其威信每行縣録囚徒還【師古曰省録之知其情狀有寃滯與否也今云慮囚本録聲之去者耳音力具翻而近俗不曉其意訛其文遂為思慮之慮失其源也甚矣行下孟翻】其母輒問不疑有所平反活幾何人即不疑多有所平反【毛晃曰平反理正幽枉也反音幡】母喜笑異於他時或無所出母怒為不食【為於偽翻】故不疑為吏嚴而不殘 九月丙子秺敬侯金日磾薨【秺音妬磾丁奚翻】初武帝病有遺詔封金日磾為秺侯上官桀為安陽侯【恩澤侯表安陽侯食邑於河内之蕩隂水經註陜縣有安陽城武帝封上官桀為侯國】霍先為博陸侯【文頴曰博大陸平取其嘉名無此縣也食邑於北海河間東郡師古曰盖亦取郷聚之名以為國號非必縣也博陸初封食北海河間後益封食東郡】皆以前捕反者馬何羅等功封【捕馬何羅事見上卷武帝後元元年】日磾以帝少不受封【少詩照翻】光等亦不敢受及日磾病困光白封日磾卧受印綬一日薨日磾兩子賞建俱侍中與帝畧同年共卧起賞為奉車建駙馬都尉及賞嗣侯佩兩綬上謂霍將軍曰金氏兄弟兩人不可使俱兩綬邪對曰賞自嗣父為侯耳上笑曰侯不在我與將軍乎對曰先帝之約有功乃得封侯遂止 閏月遣故廷尉王平等五人持節行郡國舉賢良問民疾苦寃失職者【行下孟翻】 冬無冰<br />
<br />
  二年春正月封大將軍光為博陸侯【按師古注光初封食邑北海河間】左將軍桀為安陽侯【桀食邑蕩隂】 或說霍光曰將軍不見諸呂之事乎處伊尹周公之位【說式芮翻處昌呂翻】攝政擅權而背宗室不與共職是以天下不信卒至於滅亡【背蒲妹翻卒子恤翻】今將軍當盛位帝春秋富宜納宗室又多與大臣共事【服䖍曰共議事也師古曰每事皆與參共知之】反諸呂道如是則可以免患【師古曰言諸呂專權而滅亡今納宗室是反其道乃可免患也】光然之乃擇宗室可用者遂拜楚元王孫辟彊及宗室劉長樂皆為光禄大夫辟彊守長樂衛尉【漢長樂建章甘泉各有衛尉以掌其宫衛然不常置樂音洛】三月遣使者振貸貧民無種食者【師古曰種者五穀之種也食者所以為糧食也種之勇翻】 秋八月詔曰往年災害多今年蠺麥傷所振貸種食勿收責毋令民出今年田租 初武帝征伐匈奴深入窮追二十餘年匈奴馬畜孕重墯殰罷極苦之【師古曰孕重懷任者也墯落也殰敗也罷極困也苦之心厭苦也罷讀曰疲殰音讀鄭玄曰内敗曰殰陸云謂懷任不成也】常有欲和親意未能得狐鹿孤單于有異母弟為左大都尉賢國人鄉之【郷讀曰嚮謂悉皆附之】母閼氏恐單于不立子而立左大都尉也【閼氏音煙支】乃私使殺之左大都尉同母兄怨遂不肯復會單于庭【復扶又翻】是歲單于病且死謂諸貴人我子少不能治國立弟右谷蠡王【少詩照翻治直之翻谷蠡音鹿黎】及單于死衛律等與顓渠閼氏謀【顓渠閼氏單于之正室也位大閼氏上】匿其喪矯單于令更立子左谷蠡王為壺衍鞮單于【更工衡翻】左賢王右谷蠡王怨望率其衆欲南歸漢恐不能自致即脅盧屠王欲與西降烏孫【降戶江翻】盧屠王告之單于使人驗問右谷蠡王不服反以其罪罪盧屠王國人皆寃之于是二王去居其所不復肯會龍城【匈奴諸王長少歲正月會單于庭五月大會龍城祭其先天地鬼神今二王自居其本處不復會祭龍城也復扶又翻】匈奴始衰<br />
<br />
  三年春二月有星孛于西北【孛蒲内翻】 冬十一月壬辰朔日有食之 初霍光與上官桀相親善光每休沐出【漢制中朝官五日一下里舍休沐三署諸郎亦然】桀常代光入决事光女為桀子安妻生女年甫五歲【甫始也】安欲因光内之宫中光以為尚幼不聽盖長公主私近子客河間丁外人【地理志盖縣屬泰山郡師古曰食邑於鄂為盖侯所尚故曰盖長公主長公主儀比諸王帝姊妹乃稱之盖侯王充武帝舅王信之子襲爵盖如字又古盍翻子客子賓客也丁姓外人其名長知兩翻下同近其靳翻】安素與外人善說外人曰【說式芮翻】安子容貌端正誠因長主時得入為后以臣父子在朝而有椒房之重【師古曰椒房殿在未央宫中皇后所居以椒和泥塗壁取其温而芳朝直遥反】成之在於足下漢家故事常以列侯尚主足下何憂不封侯乎外人喜言于長主長主以為然詔召安女為倢伃【倢伃音接予】安為騎都尉【為安父子與霍光争權謀亂張本】<br />
<br />
  四年春三月甲寅立皇后上官氏赦天下 西南夷姑繒葉榆復反【姑繒葉榆皆西南夷别種其所居地在益州郡界葉榆澤名武帝開為縣繒慈陵翻葉式涉翻】遣水衡都尉呂辟胡將益州兵擊之【此益州刺史所部兵也宋昌漢武帝元鼎中分雍州之南置益州釋名曰益阨也所在之地險阨也應劭地理風俗記曰疆理益廣故曰益州班志漢中廣漢蜀郡越嶲益州牂柯巴郡皆屬益州師古曰辟音璧】辟胡不進蠻夷遂殺益州太守【武帝元封二年開滇王國置益州郡治滇池縣守式又翻】乘勝與辟胡戰士戰及溺死者四千餘人冬遣大鴻臚田廣明擊之【臚陵如翻】 廷尉李种坐故縱死罪【种音冲】棄市 是歲上官安為車騎將軍 【考異曰昭紀作驃騎今從百官表外戚傳】<br />
<br />
  五年春正月追尊帝外祖趙父為順成侯【順成侯趙父鉤弋夫人之父也父時已死追封為順成侯置園邑三百戶于扶風】順成侯有姊君姁【師古曰姁音况羽翻】賜錢二百萬奴婢第宅以充實焉諸昆弟各以親疏受賞賜【孔頴達曰五服之内大功已上服麄者為親小功已下服精者為疏疏與疎同】無在位者有男子乘黄犢車詣北闕【未央宫北闕蕭何築也師古曰未央宫雖南向而上書奏事謁見者皆詣北闕公車司馬在焉】自謂衛太子公車以聞【班表公車屬衛尉天下上事皆總領之師古曰公車主受章奏】詔使公卿將軍中二千石雜識視【師古曰雜共也有素識之者令視知其是非也】長安中吏民聚觀者數萬人右將軍勒兵闕下以備非常丞相御史中二千石至者並莫敢發言京兆尹不疑後到叱從吏收縛【從才用翻】或曰是非未可知且安之【安猶徐也】不疑曰諸君何患於衛太子昔蒯聵違命出奔輒距而不納春秋是之【師古曰蒯聵衛靈公太子輒蒯聵子也蒯聵得罪於靈公而出奔晉及靈公卒使輒嗣位晉趙鞅納蒯聵於戚欲求入衛齊國夏衛石曼姑帥師圍戚公羊傳曰曼姑受命於靈公而立輒曼姑之義固可以距蒯聵也輒之義可以立乎曰可奈何不以父命辭王父命也蒯苦怪翻瞆五怪翻】衛太子得罪先帝亡不即死【即就也】今來自詣此罪人也遂送詔獄天子與大將軍霍光聞而嘉之曰公卿大臣當用有經術明於大誼者繇是不疑名聲重於朝廷在位者皆自以不及也廷尉驗治何人【凡不知姓名及所從來皆曰何人】竟得姦詐本夏陽人姓成名方遂居湖以卜筮為事有故太子舍人嘗從方遂卜謂曰子狀貌甚似衛太子方遂心利其言冀得以富貴坐誣罔不道要斬【要與腰同 考異曰昭紀云張延年雋不疑傳云成方遂又云一姓張名延年今從不疑傳】 夏六月封上官安為桑樂侯【恩澤侯表桑樂侯食邑於千乘樂來各翻】安日以驕淫受賜殿中對賓客言與我壻飲大樂【樂音洛】見其服飾使人歸欲自燒物子病死仰而罵天其頑悖如此【悖蒲内翻】罷儋耳真番郡【武帝元鼎六年置儋耳郡元封二年置眞番郡今皆罷之儋都甘翻】秋大鴻臚廣明軍正王平擊益州斬首捕虜三萬<br />
<br />
  餘人獲畜產五萬餘頭諫大夫杜延年見國家承武帝奢侈師旅之後數為大將軍光言年歲比不登【數所角翻為於偽翻比毗至翻】流民未盡還宜修孝文時政示以儉約寛和順天心說民意【說讀曰悅】年歲宜應光納其言延年故御史大夫周之子也<br />
<br />
  六年春二月詔有司問郡國所舉賢良文學民所疾苦教化之要皆對願罷鹽鐵酒榷均輸官【鹽鐵事始見十九卷武帝元狩四年均輸事始見二十卷元鼎三年酒榷事始見上卷天漢三年榷古岳翻】毋與天下争利示以儉節然後教化可興桑弘羊難以為此國家大業所以制四夷安邉足用之本不可廢也【難乃旦翻】於是鹽鐵之議起焉【師古曰議罷鹽鐵之官百姓皆得鬻鹽鑄錢因摠論政治得失也據班史藝文志有鹽鐵論十篇今行於世】初蘇武既徙北海上【事見二十一卷天漢元年】禀食不至【禀給也】掘野鼠去草實而食之【蘇林曰掘野鼠所去草實而食之張晏曰取鼠及草實并而食之師古曰蘇說是去謂藏之也貢父曰今北方野鼠甚多皆可食也武掘野鼠得即食之其草實乃頗去藏耳去丘呂翻】杖漢節牧羊卧起操持節旄盡落【操千高翻】武在漢與李陵俱為侍中陵降匈奴不敢求武久之【降匈奴事見二十一卷天漢二年降戶江翻下同】單于使陵至海上為武置酒設樂【為于偽翻下同】因謂武曰單于聞陵與子卿素厚【子卿蘇武字】故使來說足下虚心欲相待終不得歸漢空自苦亡人之地【說式芮翻亡古無字通】信義安所見乎【見賢遍反】足下兄弟二人前皆坐事自殺來時太夫人已不幸【不幸謂死也】子卿婦年少【少詩照翻】聞已更嫁矣獨有女弟二人兩女一男今復十餘年【更工衡翻復扶又翻下同】存亡不可知人生如朝露【師古曰朝露見日則晞乾人命短促亦如之】何久自苦如此陵始降時忽忽如狂自痛負漢加以老母繫保宫【班表少府屬官有居室武帝太初元年更名保宫】子卿不欲降何以過陵且陛下春秋高法令無常大臣無罪夷滅者數十家安危不可知子卿尚復誰為乎武曰武父子無功德皆為陛下所成就位列將爵通侯兄弟親近【皆為如字將即亮翻近其靳翻】常願肝腦塗地今得殺身自效雖斧鉞湯鑊誠甘樂之【師古曰鼎大而無足曰鑊樂音洛】臣事君猶子事父也子為父死無所恨願勿復再言陵與武飲數日復曰子卿壹聽陵言武曰自分已死久矣【分扶問翻】王必欲降武【匈奴封李陵為右校王故稱之】請畢今日之驩効死於前陵見其至誠喟然嘆曰嗟乎義士陵與衛律之罪上通于天因泣下霑衿與武决去【師古曰决别也】賜武牛羊數十頭後陵復至北海上語武以武帝崩武南郷號哭歐血旦夕臨數月【語牛倨翻郷讀曰嚮號戶高翻臨哭也力禁翻】及壺衍鞮單于立母閼氏不正【閼氏音煙支】國内乖離常恐漢兵襲之於是衛律為單于謀與漢和親漢使至求蘇武等匈奴詭言武死後漢使復至匈奴常惠私見漢使教使者謂單于【謂告語也】言天子射上林中得鴈足有係帛書言武等在某澤中使者大喜如惠語以讓單于單于視左右而驚謝漢使曰武等實在乃歸武及馬宏等馬宏者前副光禄大夫王忠使西國【西國謂西域諸國使疏吏翻】為匈奴所遮忠戰死馬宏生得亦不肯降故匈奴歸此二人欲以通善意於是李陵置酒賀武曰今足下還歸【還音旋又如字】揚名於匈奴功顯於漢室雖古竹帛所載丹青所畫何以過子卿陵雖駑怯令漢貰陵罪【駑音奴貰寛也貰時夜翻】全其老母使得奮大辱之積志庶幾乎曹柯之盟【李奇曰言欲刼單于如曹劌劫齊桓公柯盟之時幾居衣翻】此陵宿昔之所不忘也收族陵家為世大戮【事見上卷天漢三年】陵尚復何所顧乎已矣令子卿知吾心耳陵泣下數行【行戶剛翻】因與武决單于召會武官屬前已降及物故凡隨武還者九人既至京師詔武奉一太牢謁武帝園廟【程大昌演繁露曰牛羊豕具為太牢有羊豕而無牛則為少牢今人獨以太牢名牛失之矣】拜為典屬國秩中二千石【班表典屬國本秦官掌歸義蠻夷漢因之今以命武以武久在匈奴中習外夷事故使為是官其後省併大鴻臚】賜錢二百萬公田二頃宅一區武留匈奴凡十九歲始以彊壯出及還須髪盡白【須與鬚同】霍光上官桀與李陵素善遣陵故人隴西任立政等三人俱至匈奴招之陵曰歸易耳【易以䜴翻】丈夫不能再辱遂死於匈奴【陵意謂降匈奴已辱矣今若歸漢漢將使刀筆吏簿責其喪師降匈奴之罪是謂再辱也故遂不歸】 夏旱 秋七月罷榷酤官從賢良文學之議也【酤古護翻】武帝之末海内虚耗戶口减半霍光知時務之要輕徭薄賦與民休息至是匈奴和親百姓充實稍復文景之業焉 詔以鉤町侯毋波【鉤町西南夷種武帝開為縣屬牂柯郡雖置官吏而仍以其君長為鉤町侯使主其種類鉤音劬町音梃母波漢書作亡波亡古無字也】率其邑君長人民擊反者有功【長如兩翻】立以為鉤町王賜田廣明爵關内侯<br />
<br />
  元鳳元年【應劭曰三年中鳳凰比下東海海西樂郡故以冠元】春武都氐人反【武都郡屬凉州氐人即白馬氐也魚豢魏畧曰其人分竄山谷或號青氐或號白氐氐丁奚翻】遣執金吾馬適建龍頟侯韓增大鴻臚田廣明將三輔太常徒皆免刑擊之【師古曰姓馬適名建也據班書功臣表弓高侯韓頹當之孫說以擊匈奴功封龍頟侯坐酎金失侯復以破東越功封按道侯後為衛太子所殺子興嗣侯坐巫蠱誅後元元年復以增嗣龍頟侯增興弟也班志龍頟侯國屬平原郡頟音洛作額者非崔浩曰今有龍頟村蘇林曰是時太常主諸陵縣治民也余謂此刑徒輸作三輔及太常者也】 夏六月赦天下 秋七月乙亥晦日有食之既 八月改元 上官桀父子既尊盛德長公主欲為丁外人求封侯霍光不許又為外人求光禄大夫欲令得召見又不許盖主大以是怨光而桀安數為外人求官爵弗能得亦慙【長知兩翻為于偽翻數所角翻】又桀妻父幸充國為太醫監【充國史失其姓大醫監屬少府】闌入殿中【闌妄也漢制諸入宫殿門皆著籍無藉而妄人謂之闌入】下獄當死冬月且盡【漢論上囚不過冬月下遐嫁翻】盖主為充國入馬二十匹贖罪乃得减死論於是桀安父子深怨光而重德盖主自先帝時桀已為九卿位在光右【武帝時桀為太僕位九卿秩中二千石光為奉車太尉光禄大夫秩比二千石是桀之位在光右也右上也】及父子並為將軍【桀為左將軍安為車騎將軍】皇后親安女光乃其外祖而顧專制朝事【師古曰顧猶反也朝直遥翻】由是與光争權燕王旦自以帝兄不得立常懷怨望及御史大夫桑弘羊建造酒榷鹽鐵為國興利伐其功【伐矜也榷古岳翻為於偽翻下同】欲為子弟得官亦怨恨光於是盖主桀安弘羊皆與旦通謀旦遣孫縱之等前後十餘輩多齎金寶走馬賂遺盖主桀弘羊等【師占曰走馬馬之善走者也遺于季翻】桀等又詐令人為燕王上書言光出都肄郎羽林【孟康曰都試也肄習也張晏曰都肄郎羽材也師古曰都大也大會試之漢光禄勲令諸當試者不會都所免之都肄謂緫閱試習武備也肄羊至翻】道上稱䟆【天子出稱䟆以清道止行人䟆與蹕同】太官先置【師古曰供飲食之具太官屬少府主膳食凡車駕所幸太官先往其處供置】又引蘇武使匈奴二十年不降乃為典屬國【實十九年而言二十者欲久其事以見寃屈故言多也使疏史翻降戶江翻】大將軍長史敞無功為搜粟都尉又擅調益莫府校尉【師古曰調選也莫府大將軍府也調音徒釣翻】光專權自姿疑有非常臣旦願歸符璽人宿衛察姦臣變【璽斯氏翻】候司光出沐日奏之桀欲從中下其事【伺光出沐不在禁中桀欲自從禁中下其事也司讀曰伺師古曰下謂下有司也下音胡稼翻下同】弘羊當與諸大臣共執退光【當者以之自任也】書奏帝不肯下明旦光聞之止畫室中不入【如淳曰近臣調計畫之室或曰雕畫之室師古曰雕畫是也】上問大將軍安在左將軍桀對曰以燕王告其罪故不敢入有詔召大將軍光入免冠頓首謝上曰將軍冠【師古曰令復著冠也】朕知是書詐也將軍無罪光曰陛下何以知之上曰將軍之廣明都郎近耳【師古曰之往也廣明亭名今據廣明亭在長安城東東都門外水經注京兆奉明縣廣成郷有廣明苑史皇孫及王夫人葬于郭北宣帝移于苑北以為悼園在東都門外】調校尉以來未能十日燕王何以得知之且將軍為非不須校尉【文頴曰帝云將軍欲反不由一校尉】是時帝年十四尚書左右皆驚【班表少府屬官有尚書等十二官令丞又有中書謁者等七官令丞續漢志尚書令千石本注曰承秦所置武帝用宦者更為中書謁者令成帝用士人復故掌凡選署及奏下尚書曹文書衆事余據表則尚書中書為兩官據續志則合為一官此時既有尚書則與中書謁者為兩官明矣沈約曰秦世少府遣吏四人在殿中主發書故謂之尚書尚猶主也漢初有尚冠尚衣尚席尚浴尚食尚書故謂之六尚秦時尚書有令有僕射有丞至漢並隸少府武帝使左右曹諸吏分平尚書事昭帝即位霍光領尚書約又曰漢武遊後庭始使宦者典尚書事謂之中書謁者置令僕射成帝改中書謁者令為中謁者令罷謁者東京省中謁者令而有中宫謁者令非其職也沈約亦以尚書中書為兩官明矣】而上書者果亡捕之甚急桀等懼白上小事不足遂【師古曰遂猶竟也言不須窮竟也】上不聽後桀黨與有譛光者上輒怒曰大將軍忠臣先帝所屬以輔朕身敢有毁者坐之自是桀等不敢復言【屬之欲翻復扶又翻】<br />
<br />
  李德裕論曰人君之德莫大於至明明以照姦則百邪不能蔽矣漢昭帝是也周成王有慙德矣高祖文景俱不如也成王聞管蔡流言遂使周公狼跋而東漢高聞陳平去魏背楚欲捨腹心臣【背蒲妹翻】漢文惑季布使酒難近罷歸股肱郡疑賈生擅權紛亂復踈賢士景帝信誅晁錯兵解遂戮三公【武王崩周公相成王管叔蔡叔流言於國曰公將不利于孺子周公于是東征成王不知周公之志公乃為䲭鴞之詩周大夫亦為賦狼跋之詩曰狼跋其胡載疐其尾毛氏註云跋躐也疐跲也老狼有胡進則躐其胡退則跲其尾進退有難然而不失其猛疏曰李廵曰跋前行曰躐跲却頓曰疐也說文云跋蹎丁千翻跲躓竹二翻躓即疐也然則跋與疐皆是顚倒之類以跋為躐者謂跋其胡而倒耳老狼有胡謂頷垂胡近則躐其胡謂躐胡而前倒也退則跲其尾謂却頓而倒於尾上也高祖疑陳平事見九卷二年文帝罷季布事見十四卷前四年疎賈生事同上景帝誅晁錯事見十六卷前三年】所謂執狐疑之心來讒賊之口【劉向之言】使昭帝得伊呂之佐則成康不足侔矣<br />
<br />
  桀等謀令長公主置酒請光伏兵格殺之因廢帝迎立燕王為天子旦置驛書往來相報許立桀為王外連郡國豪桀以千數旦以語相平【平為燕相史失其姓語牛倨翻】平曰大王前與劉澤結謀事未成而發覺者以劉澤素夸好侵陵也【好呼到翻】平聞左將軍素輕易車騎將軍少而驕【易以豉翻少詩照翻】臣恐其如劉澤時不能成又恐既成反大王也旦曰前日一男子詣闕自謂故太子長安中民趣郷之【趣七喻翻郷讀曰嚮】正讙不可止【師古曰人衆既多故讙譁讙况爰翻】大將軍恐出兵陳之以自備耳我帝長子【帝謂武帝長知兩翻】天下所信何憂見反後謂羣臣盖主報言獨患大將軍與右將軍王莽【張晏曰王莽天水人也字稚叔】今右將軍物故丞相病幸事必成徵不久令羣臣皆裝【令皆治行裝也】安又謀誘燕王至而誅之【誘音酉】因廢帝而立桀或曰當如皇后何安曰逐麋之狗當顧莬邪【師古曰言所求者大不顧小也糜鹿之大者莬讀曰兔吐故翻】且用皇后為尊一旦人主意有所移雖欲為家人亦不可得【家人謂凡庶匹夫也】此百世之一時也會盖主舍人父稻田使者燕倉知其謀【如淳曰特為諸稻田置使者假與民收其稅入也燕音煙姓譜召公封于燕其後為秦所滅子孫以為氏】以告大司農楊敞敞素謹畏事不敢言乃移病卧【師古曰移病謂移書言病一曰以病而移居尒謂前說是】以告諫大夫杜延年延年以聞九月詔丞相部中二千石逐捕孫縱之及桀安弘羊外人等并宗族悉誅之盖主自殺燕王旦聞之召相平曰事敗遂發兵乎【相息亮翻】平曰左將軍已死百姓皆知之不可發也王憂懣【師古曰懣音滿又音悶煩也】置酒與羣臣妃妾别會天子以璽書讓旦【璽斯氏翻】旦以綬自絞死后夫人隨旦自殺者二十餘人天子加恩赦王太子建為庶人賜旦諡曰刺王【刺來違翻諡法暴戾無親曰刺】皇后以年少不與謀【與讀曰豫】亦霍光外孫故得不廢 庚午右扶風王訢為御史大夫【訢與欣同】 冬十月封杜延年為建平侯【班表建平侯食邑于濟陽】燕倉為宜城侯【宜城侯食邑於濟隂】故丞相徵事任宫捕得桀為弋陽侯【文頴曰徵事丞相官屬位差尊掾屬也如淳曰時宫以時事待詔丞相府故曰丞相徵事張晏曰漢儀注徵事比六百石皆故吏二千石不以贓罪免者為徵事絳衣奉朝賀正月師古曰張說是班志弋陽侯國屬汝南郡任音壬】丞相少史王山夀誘安入府為商利侯【如淳曰漢儀注武帝置丞相少史秩四百石班表商利侯食邑於臨淮之徐少詩照翻】久之文學濟隂魏相對策【濟隂郡屬兖州唐為曹州濟子禮翻】以為日者燕王為無道韓義出身彊諫為王所殺義無比干之親而蹈比干之節【比干紂之賢臣諫紂而死】宜顯賞其子以示天下明為人臣之義乃擢義子延夀為諫大夫 大將軍光以朝無舊臣【朝無直遥翻】光禄勲張安世自先帝時為尚書令【班表少府屬官有尚書令續漢志尚書令承秦所置掌凡選署及奏下尚書曹文書衆事秩千石】志行純篤【行下孟翻】乃白用安世為右將軍兼光禄勲以自副焉安世故御史大夫湯之子也光又以杜延年有忠節【以其發燕盖上官之謀也】擢為太僕右曹給事中【太僕正卿右曹給事中加官也晉灼曰漢儀注諸吏給事中日上朝謁平尚書奏事分為左右曹班表給事中掌顧問慮對位中常待下盖得入出禁中】光持刑罰嚴延年常輔之以寛吏民上書言便宜輒下延年平處復奏【下遐嫁翻先平處其可否復奏言之處昌呂翻】可官試者至為縣令或丞相御史除用滿歲以狀聞或抵其罪法【師古曰抵至也言事之人有姦妄者則致之於罪法】 是歲匈奴發左右部二萬騎為四隊並入邊為寇漢兵追之斬首獲虜九千人生得甌脫王漢無所失亡匈奴見甌脫王在漢恐以為道擊之【道讀曰導】即西北遠去不敢南逐水草發人民屯甌脫<br />
<br />
  二年夏四月上自建章宫徙未央宫 六月赦天下是歲匈奴復遣九千騎屯受降城以備漢【復扶又翻】北橋余吾水令可度以備奔走【師古曰于余吾水上作橋擬有廹急奔走避漢從此橋度也】欲求和親而恐漢不聼故不肯先言常使左右風漢使者【風讀曰諷】然其侵盗益希遇漢使愈厚欲以漸致和親漢亦覊縻之<br />
<br />
  三年春正月泰山有大石自起立上林有柳樹枯僵自起生【僵居良翻什也】有蟲食其葉成文曰公孫病已立【此為宣帝興于民間之符】符節令魯國眭弘上書【班表符節令屬少府秩六百石續漢志曰為符節臺率主符節事漢改秦薛郡為魯國屬豫州唐兖州地師古曰眭息隨翻今河朔猶有此姓】言大石自立僵柳復起【復扶又翻下同】當有匹庶為天子者枯樹復生故廢之家公孫氏當復興乎漢家承堯之後【班贊曰春秋晉史蔡墨有言陶唐既衰其後有劉累學擾龍事孔甲范氏其後也而范宣子亦曰祖自虞以上為陶唐氏在夏為御龍氏在商為豕韋氏在周為唐杜氏晉主夏盟為范氏范氏為晉士師魯文公世奔秦後歸于晉其處者為劉氏劉向云戰國時劉氏自秦獲於魏秦滅魏遷大梁都于豐故周市說雍齒曰豐故梁徙也是以頌高祖云漢帝本系出自唐帝降及於周在秦作劉涉魏而東遂為豐公豐公蓋太上皇父及高祖即位置祠祀官則有秦晉梁荆之巫世祠天地綴之以祀豈不信哉由是言之漢承堯運協于火德得天統矣】有傳國之運當求賢人禪帝位退自封百里以順天命弘坐設妖言惑衆伏誅 匈奴單于使犂汙王窺邉【據王莽時使譯出塞誘呼右犂汙王咸則犂汙王所居地盖近塞下也】言酒泉張掖兵益弱出兵試擊冀可復得其地時漢先得降者聞其計天子詔邉警備後無幾【幾居豈翻】右賢王犂汙王四千騎分三隊入日勒屋蘭番和【班志三縣皆屬張掖郡賢曰日勒故城在今甘州刪丹縣東南師古曰番音盤】張掖太守屬國都尉【續漢志張掖屬國都尉治居延縣守式又翻】發兵擊大破之得脫者數百人屬國義渠王射殺犂汙王【義渠王屬國義渠胡之君長射而亦翻】賜黄金二百斤馬二百匹因封為犂汙王自是後匈奴不敢入張掖 燕盖之亂【燕王盖主也燕于賢翻盖古盍翻】桑弘羊子遷亡過父故吏侯史吳【侯史姓也吳其名也晉武帝時有侯史光過古禾翻】後遷捕得伏法會赦侯史吳自出繫獄廷尉王平少府徐仁雜治反事皆以為桑遷坐父謀反而侯史吳臧之【治直之翻下同臧讀曰藏】非匿反者乃匿為隨者也【言桑遷但隨坐耳非匄反也】即以赦令除吳罪後侍御史治實【師古曰重覈治其事也】以桑遷通經術知父謀反而不諫争【争與諍同】與反者身無異侯史吳故三百石吏首匿遷【師古曰首匿者言身為謀首而藏匿人也】不與庶人匿隨從者等吳不得赦奏請覆治【此深文傳致吳之罪從才用翻】劾廷尉少府縱反者【劾戶槩翻師古曰縱放也】少府徐仁即丞相車千秋女壻也【車千秋即田千秋漢以其年老得乘小車入殿中因呼為車丞相】故千秋數為侯史吳言【數所角翻】恐大將軍光不聽千秋即召中二千石博士會公車門【公車門即未央宫北闕門也】議問吳法【師古曰于法律之中吳當得何罪】議者知大將軍指皆執吳為不道明日千秋封上衆議【上時掌翻】光於是以千秋擅召中二千石以下外内異言【張晏曰外則去疾欲盡内則為其壻也師古曰非也外内謂内朝及外朝也】遂下廷尉平少府仁獄【下遐嫁翻】朝廷皆恐丞相坐之太僕杜延年奏記光曰吏縱罪人有常法今更詆吳為不道【師古曰詆誣也】恐於法深又丞相素無所守持而為好言於下盡其素行也【師古曰言非故有所執持但其素行好與在下人言議耳】至擅召中二千石甚無狀【師古曰無善狀也】延年愚以為丞相久故及先帝用事【言在位已久是為故舊又嘗及相先帝而任事也】非有大故不可棄也間者民頗言獄深吏為峻詆今丞相所議又獄事也如是以及丞相恐不合衆心羣下讙譁【讙許爰翻】庶人私議流言四布延年竊重將軍失此名於天下也【師古曰重猶難也以此為重事也】光以廷尉少府弄法輕重卒下之獄【卒子恤翻】夏四月仁自殺平與左馮翊賈勝胡皆要斬【内史周官秦因之景帝二年分置左内史武帝更名左馮翊要與腰同】而不以及丞相終與相竟【師古曰謂終丞相之身無貶黜也余謂言與千秋共事終其身】延年論議持平合和朝廷皆此類也 冬遼東烏桓反初冒頓破東胡東胡餘衆散保烏桓及鮮卑山為二族【遼東郡屬幽州唐嘗置安東都護府於其地東胡破見十一卷高祖六年後漢書烏桓之地在丁零西南烏孫東北武帝遣霍去病擊破匈奴左地因徙烏桓于上谷漁陽右北平遼東遼西五郡塞外為漢偵察匈奴動静其大人歲一朝見於是始置護烏桓校尉秩比二千石鮮卑先遠竄于遼東塞外與烏桓相接未嘗通中國至後漢稍徙遼西塞外始為中國患】世役屬匈奴武帝擊破匈奴左地因徙烏桓於上谷漁陽右北平遼東塞外【上谷漁陽北平皆屬幽州上谷唐媯州漁陽唐檀薊州北平唐平州之地】為漢偵察匈奴動静【為于偽翻偵丑鄭翻又尹貞翻候也】置護烏桓校尉監領之【監占衘翻】使不得與匈奴交通至是部衆漸彊遂反先是匈奴三千餘騎入五原【五原郡屬并州先悉薦翻】殺略數千人後數萬騎南旁塞獵【旁步浪翻】行攻塞外亭障略取吏民去是時漢邉郡熢火候望精明匈奴為邉寇者少利希復犯塞【少詩沼翻復扶又翻下同】漢復得匈奴降者言烏桓嘗發先單于冢匈奴怨之方發二萬騎擊烏桓霍光欲發兵邀擊之【師古曰邀迎而擊之】以問護軍都尉趙充國【護軍都尉秦官武帝以屬大司馬此時盖屬大將軍也】充國以為烏桓間數犯塞【師古曰間即中間也猶言比日也數所角翻】今匈奴擊之於漢便又匈奴希寇盗北邉幸無事蠻夷自相攻擊而發兵要之【要與邀同】招寇生事非計也光更問中郎將范明友明友言可擊於是拜明友為度遼將軍【度遼將軍盖使之度遼水以伐烏桓至後漢遂以為將軍之號以護匈奴】將二萬騎出遼東匈奴聞漢兵至引去初光誡明友兵不空出即後匈奴遂擊烏桓【師古曰後匈奴者言兵遲後邀匈奴不及後戶遘翻】烏桓時新中匈奴兵【師古曰為匈奴所中傷中竹仲翻】明友既後匈奴因乘烏桓敝擊之斬首六千餘級獲三王首匈奴由是恐不能復出兵<br />
<br />
  四年春正月丁亥帝加元服【如淳曰元服謂初冠加上服也師古曰如氏以為衣服之服非也元首也冠者首之所著故曰元服没黯序傳云上正元服是知謂冠為元服余按續漢志有加元服之禮】 甲戌富民定侯田千秋薨【諡法安民大慮曰定】時政事壹决大將軍光千秋居丞相位謹厚自守而已 夏五月丁丑孝文廟正殿火【人火曰火】上及群臣皆素服發中二千石將五校作治【將作大匹屬官有左右前後中五校令掌五校士校戶教翻】六日成太常及廟令丞郎吏皆劾大不敬【劾戶槩翻】會赦太常轑陽侯德免為庶人【班表轑陽侯食邑清河文穎曰轑陽在魏郡清淵轑音料又音聊】 六月赦天下 初杅罙遣太子賴丹為質于龜兹【龜兹國治延城去長安七千四百八十里杅音烏罙與彌同質音致下同龜音丘兹音慈賢曰今龜音丘劾翻兹音沮惟翻盖急言之也】貳師擊大宛還【事見二十一卷武帝太初元年宛于元翻】將賴丹入至京師霍光用桑弘羊前議以賴丹為校尉將軍田輪臺【弘羊議田輪臺見二十二卷征和元年】龜兹貴人姑翼謂其王曰賴丹本臣屬吾國今佩漢印綬來廹吾國而田必為害王即殺賴丹而上書謝漢樓蘭王死匈奴先聞之遣其質子安歸歸得立為王 【考異曰西域傳作常歸今從昭紀及傅介子傳】漢遣使詔新王令入朝王辭不至樓蘭國最在東垂【西域之東垂也】近漢當白龍堆【孟康曰龍堆形如土龍身無頭有尾高大者三四丈埤者長丈餘皆東北向而相似也近其靳翻下同】乏水草常主發導負水擔糧送迎漢使【擔都甘翻】又數為吏卒所寇懲艾不便與漢通【師古曰艾讀曰又數所角翻下同】後復為匈奴反閒【閒古莧反】數遮殺漢使其弟尉屠耆降漢具言狀駿馬監北地傅介子使大宛【班表太僕屬官有駿馬監北地郡屬凉州刺史姓譜傳說出傅嚴因以為氏】詔因令責樓蘭龜兹介子至樓蘭龜兹責其王皆謝服介子從大宛還到龜兹會匈奴使從烏孫還在龜兹介子因率其吏士共誅斬匈奴使者還奏事詔拜介子為中郎遷平樂監【平樂監監平樂觀樂音洛】介子謂大將軍霍光曰樓蘭龜兹數反覆而不誅無所懲艾介子過龜兹時其王近就人易得也願往刺之以威示諸國【易以豉翻刺七亦翻下同】大將軍曰龜兹道遠且驗之於樓蘭於是白遣之介子與士卒俱齎金幣揚言以賜外國為名至樓蘭樓蘭王意不親介子介子陽引去至其西界使譯謂曰【班表大鴻臚有譯官令典屬國有九譯令皆掌譯此譯則樓蘭國之譯人】漢使者持黄金錦繡行賜諸國王不來受我去之西國矣即出金幣以示譯譯還報王王貪漢物來見使者介子與坐飲陳物示之飲酒皆醉介子謂王曰天子使我私報王【師古曰謂密有所論】王起隨介子入帳中屏語【屛人而獨共語也屛必郢翻】壯士二人從後刺之【刺七亦翻】刃交匈立死【匈與胷同】其貴臣左右皆散走介子告諭以王負漢罪天子遣我誅王當更立王弟尉屠耆在漢者【更工衡翻】漢兵方至毋敢動自令滅國矣介子遂斬王安歸首馳傳詣闕縣首北闕下【傳張戀翻縣古懸字通】乃立尉屠耆為王更名其國為鄯善為刻印章賜以宫女為夫人備車騎輜重【更工衡翻鄯上扇翻為刻于偽翻重直用翻】丞相率百官送至横門外祖而遣之【三輔黄圖横門長安城北出西頭第一門孟康曰横音光祖祖道也】王自請天子曰身在漢久今歸單弱而前王有子在恐為所殺國中有伊循城其地肥美願漢遣一將屯田積穀令臣得依其威重於是漢遣司馬一人吏士四十人田伊循以填撫之【填讀曰鎮】秋七月乙巳封范明友為平陵侯【賞破烏桓之功也班表平陵侯食邑于南陽之武當】傅介子為義陽侯【班表義陽侯食邑于南陽之平氏】<br />
<br />
  臣光曰王者之於戎狄叛則討之服則舍之【舍讀曰捨】今樓蘭王既伏其罪又從而誅之後有叛者不可得而懷矣必以為有罪而討之則宜陳師鞠旅【毛詩注曰鞠告也將戰之日陳其師旅誓告之也】明致其罰今乃遣使者誘以金幣而殺之後有奉使諸國者復可信乎【復扶又翻】且以大漢之彊而為盜賊之謀於蠻夷不亦可羞哉論者或美介子以為奇功過矣<br />
<br />
  五年夏大旱 秋罷象郡分屬鬰林牂柯【班志鬱林故秦桂林郡】冬十一月大雷 十二月庚戍宜春敬侯王訢薨【恩澤】<br />
<br />
  【侯表宜春侯食邑於汝南訢音欣】<br />
<br />
  六年春正月募郡國徒築遼東玄菟城【菟音塗】 夏赦天下 烏桓復犯塞【復扶又翻】遣度遼將軍范明友擊之 冬十一月乙丑以楊敞為丞相少府河内蔡義為御史大夫【河内郡時屬司隷唐懷孟衛州地】<br />
<br />
  資治通鑑卷二十三  <br>
   </div> 

<script src="/search/ajaxskft.js"> </script>
 <div class="clear"></div>
<br>
<br>
 <!-- a.d-->

 <!--
<div class="info_share">
</div> 
-->
 <!--info_share--></div>   <!-- end info_content-->
  </div> <!-- end l-->

<div class="r">   <!--r-->



<div class="sidebar"  style="margin-bottom:2px;">

 
<div class="sidebar_title">工具类大全</div>
<div class="sidebar_info">
<strong><a href="http://www.guoxuedashi.com/lsditu/" target="_blank">历史地图</a></strong>  
<a href="http://www.880114.com/" target="_blank">英语宝典</a>  
<a href="http://www.guoxuedashi.com/13jing/" target="_blank">十三经检索</a> 
<br><strong><a href="http://www.guoxuedashi.com/gjtsjc/" target="_blank">古今图书集成</a></strong> 
<a href="http://www.guoxuedashi.com/duilian/" target="_blank">对联大全</a> <strong><a href="http://www.guoxuedashi.com/xiangxingzi/" target="_blank">象形文字典</a></strong> 

<br><a href="http://www.guoxuedashi.com/zixing/yanbian/">字形演变</a>  <strong><a href="http://www.guoxuemi.com/hafo/" target="_blank">哈佛燕京中文善本特藏</a></strong>
<br><strong><a href="http://www.guoxuedashi.com/csfz/" target="_blank">丛书&方志检索器</a></strong> <a href="http://www.guoxuedashi.com/yqjyy/" target="_blank">一切经音义</a>  

<br><strong><a href="http://www.guoxuedashi.com/jiapu/" target="_blank">家谱族谱查询</a></strong>  <strong><a href="http://shufa.guoxuedashi.com/sfzitie/" target="_blank">书法字帖欣赏</a></strong> 
<br>

</div>
</div>


<div class="sidebar" style="margin-bottom:0px;">

<font style="font-size:22px;line-height:32px">QQ交流群9:489193090</font>


<div class="sidebar_title">手机APP 扫描或点击</div>
<div class="sidebar_info">
<table>
<tr>
	<td width=160><a href="http://m.guoxuedashi.com/app/" target="_blank"><img src="/img/gxds-sj.png" width="140"  border="0" alt="国学大师手机版"></a></td>
	<td>
<a href="http://www.guoxuedashi.com/download/" target="_blank">app软件下载专区</a><br>
<a href="http://www.guoxuedashi.com/download/gxds.php" target="_blank">《国学大师》下载</a><br>
<a href="http://www.guoxuedashi.com/download/kxzd.php" target="_blank">《汉字宝典》下载</a><br>
<a href="http://www.guoxuedashi.com/download/scqbd.php" target="_blank">《诗词曲宝典》下载</a><br>
<a href="http://www.guoxuedashi.com/SiKuQuanShu/skqs.php" target="_blank">《四库全书》下载</a><br>
</td>
</tr>
</table>

</div>
</div>


<div class="sidebar2">
<center>


</center>
</div>

<div class="sidebar"  style="margin-bottom:2px;">
<div class="sidebar_title">网站使用教程</div>
<div class="sidebar_info">
<a href="http://www.guoxuedashi.com/help/gjsearch.php" target="_blank">如何在国学大师网下载古籍?</a><br>
<a href="http://www.guoxuedashi.com/zidian/bujian/bjjc.php" target="_blank">如何使用部件查字法快速查字?</a><br>
<a href="http://www.guoxuedashi.com/search/sjc.php" target="_blank">如何在指定的书籍中全文检索?</a><br>
<a href="http://www.guoxuedashi.com/search/skjc.php" target="_blank">如何找到一句话在《四库全书》哪一页?</a><br>
</div>
</div>


<div class="sidebar">
<div class="sidebar_title">热门书籍</div>
<div class="sidebar_info">
<a href="/so.php?sokey=%E8%B5%84%E6%B2%BB%E9%80%9A%E9%89%B4&kt=1">资治通鉴</a> <a href="/24shi/"><strong>二十四史</strong></a>&nbsp; <a href="/a2694/">野史</a>&nbsp; <a href="/SiKuQuanShu/"><strong>四库全书</strong></a>&nbsp;<a href="http://www.guoxuedashi.com/SiKuQuanShu/fanti/">繁体</a>
<br><a href="/so.php?sokey=%E7%BA%A2%E6%A5%BC%E6%A2%A6&kt=1">红楼梦</a> <a href="/a/1858x/">三国演义</a> <a href="/a/1038k/">水浒传</a> <a href="/a/1046t/">西游记</a> <a href="/a/1914o/">封神演义</a>
<br>
<a href="http://www.guoxuedashi.com/so.php?sokeygx=%E4%B8%87%E6%9C%89%E6%96%87%E5%BA%93&submit=&kt=1">万有文库</a> <a href="/a/780t/">古文观止</a> <a href="/a/1024l/">文心雕龙</a> <a href="/a/1704n/">全唐诗</a> <a href="/a/1705h/">全宋词</a>
<br><a href="http://www.guoxuedashi.com/so.php?sokeygx=%E7%99%BE%E8%A1%B2%E6%9C%AC%E4%BA%8C%E5%8D%81%E5%9B%9B%E5%8F%B2&submit=&kt=1"><strong>百衲本二十四史</strong></a>  <a href="http://www.guoxuedashi.com/so.php?sokeygx=%E5%8F%A4%E4%BB%8A%E5%9B%BE%E4%B9%A6%E9%9B%86%E6%88%90&submit=&kt=1"><strong>古今图书集成</strong></a>
<br>

<a href="http://www.guoxuedashi.com/so.php?sokeygx=%E4%B8%9B%E4%B9%A6%E9%9B%86%E6%88%90&submit=&kt=1">丛书集成</a> 
<a href="http://www.guoxuedashi.com/so.php?sokeygx=%E5%9B%9B%E9%83%A8%E4%B8%9B%E5%88%8A&submit=&kt=1"><strong>四部丛刊</strong></a>  
<a href="http://www.guoxuedashi.com/so.php?sokeygx=%E8%AF%B4%E6%96%87%E8%A7%A3%E5%AD%97&submit=&kt=1">說文解字</a> <a href="http://www.guoxuedashi.com/so.php?sokeygx=%E5%85%A8%E4%B8%8A%E5%8F%A4&submit=&kt=1">三国六朝文</a>
<br><a href="http://www.guoxuedashi.com/so.php?sokeytm=%E6%97%A5%E6%9C%AC%E5%86%85%E9%98%81%E6%96%87%E5%BA%93&submit=&kt=1"><strong>日本内阁文库</strong></a> <a href="http://www.guoxuedashi.com/so.php?sokeytm=%E5%9B%BD%E5%9B%BE%E6%96%B9%E5%BF%97%E5%90%88%E9%9B%86&ka=100&submit=">国图方志合集</a> <a href="http://www.guoxuedashi.com/so.php?sokeytm=%E5%90%84%E5%9C%B0%E6%96%B9%E5%BF%97&submit=&kt=1"><strong>各地方志</strong></a>

</div>
</div>


<div class="sidebar2">
<center>

</center>
</div>
<div class="sidebar greenbar">
<div class="sidebar_title green">四库全书</div>
<div class="sidebar_info">

《四库全书》是中国古代最大的丛书,编撰于乾隆年间,由纪昀等360多位高官、学者编撰,3800多人抄写,费时十三年编成。丛书分经、史、子、集四部,故名四库。共有3500多种书,7.9万卷,3.6万册,约8亿字,基本上囊括了古代所有图书,故称“全书”。<a href="http://www.guoxuedashi.com/SiKuQuanShu/">详细>>
</a>

</div> 
</div>

</div>  <!--end r-->

</div>
<!-- 内容区END --> 

<!-- 页脚开始 -->
<div class="shh">

</div>

<div class="w1180" style="margin-top:8px;">
<center><script src="http://www.guoxuedashi.com/img/plus.php?id=3"></script></center>
</div>
<div class="w1180 foot">
<a href="/b/thanks.php">特别致谢</a> | <a href="javascript:window.external.AddFavorite(document.location.href,document.title);">收藏本站</a> | <a href="#">欢迎投稿</a> | <a href="http://www.guoxuedashi.com/forum/">意见建议</a> | <a href="http://www.guoxuemi.com/">国学迷</a> | <a href="http://www.shuowen.net/">说文网</a><script language="javascript" type="text/javascript" src="https://js.users.51.la/17753172.js"></script><br />
  Copyright &copy; 国学大师 古典图书集成 All Rights Reserved.<br>
  
  <span style="font-size:14px">免责声明:本站非营利性站点,以方便网友为主,仅供学习研究。<br>内容由热心网友提供和网上收集,不保留版权。若侵犯了您的权益,来信即刪。scp168@qq.com</span>
  <br />
ICP证:<a href="http://www.beian.miit.gov.cn/" target="_blank">鲁ICP备19060063号</a></div>
<!-- 页脚END --> 
<script src="http://www.guoxuedashi.com/img/plus.php?id=22"></script>
<script src="http://www.guoxuedashi.com/img/tongji.js"></script>

</body>
</html>
