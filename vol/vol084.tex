






























































資治通鑑卷八十四   宋 司馬光 撰

胡三省 音註

晉紀六|{
	起重光作噩盡玄黓閹茂凡二年}


孝惠皇帝中之上

永寧元年|{
	此猶是永康二年正月乙丑趙王倫改元建始四月帝反正始改元永寧}
春正月以散騎常侍安定張軌為涼州刺史|{
	散悉置翻騎奇寄翻}
軌以時方多難隂有保據河西之志故求為涼州時州境盜賊縱横|{
	難乃旦翻縱子容翻}
鮮卑為寇軌至以宋配汜瑗為謀主|{
	楊正衡曰汜音凡姓也瑗于眷翻}
悉討破之威著西土|{
	張氏保據涼土始此嗚呼世亂則人思自全然求全而不能自全者亦多矣竇融張軌之求出河西此求全而得全者也謝晦袁顗之求鎮荆襄此求全而不能自全者也蓋竇融張軌始終一心以奉漢晉此固宜永終福禄詒及子孫者也謝晦袁顗志在據地險以全身其用心非矣天所不與也然劉焉求牧益州袁紹志圖冀部石敬塘心欲河東皆以之潛規非望至其成敗久速則有非智慮所及者}
相國倫與孫秀使牙門趙奉詐傳宣帝神語云倫宜早入西宫|{
	司馬懿追謚宣皇帝時倫以東宫為相國府謂禁中為西宫}
散騎常侍義陽王威望之孫也素諂事倫倫以威兼侍中使威逼奪帝璽綬作禪詔又使尚書令滿奮持節奉璽綬|{
	璽斯氏翻綬音受}
禪位於倫左衛將軍王輿前軍將軍司馬雅等帥甲士入殿|{
	帥讀曰率}
曉諭三部司馬示以威賞無敢違者張林等屯守諸門|{
	屯守宫城諸門也}
乙丑倫備灋駕入宫即帝位 |{
	考異曰三十國春秋云倫將篡位義陽王威執詔示嵇紹曰聖上法堯舜之舉卿其然乎紹厲聲曰有死而已終不有二威怒抜劒而出及惠帝遷于金墉城唯紹固志不從直于金墉絶不通倫時人皆為之懼晉書忠義傳云倫簒位紹為侍中惠帝復祚遂居其職二說不同今皆不取 復祚之祚當作阼}
赦天下改元建始帝自華林西門出居金墉城|{
	華林西門華林園西門也}
倫使張衡將兵守之|{
	將即亮翻}
丙寅尊帝為太上皇改金墉曰永昌宫廢皇太孫為濮陽王|{
	濮博木翻}
立世子荂為太子|{
	荂枯花翻楊正衡音孚}
封子馥為京兆王 䖍為廣平王詡為霸城王皆侍中將兵以梁王肜為宰衡|{
	肜余中翻}
何劭為太宰孫秀為侍中中書監票騎將軍儀同三司|{
	票匹妙翻}
義陽王威為中書令張林為衛將軍其餘黨與皆為卿將|{
	卿將列卿及諸中郎將也將即亮翻}
超階越次不可勝紀|{
	勝音升}
下至奴卒亦加爵位每朝會貂蟬盈座|{
	武冠一曰武弁諸武官冠之侍中中常侍加黄金璫附蟬為文貂尾為飾謂之趙惠文冠胡廣說曰趙武靈王效胡服以金璫飾首前插貂尾為貴職秦滅趙以其冠賜近臣應劭漢官曰說者以金取堅剛百鍊不耗蟬居高飲潔口在腋下貂内勁悍而外温潤此因物生義也徐廣曰趙武靈王胡服有此秦漢即而用之說者蟬取其清高飲露而不食貂紫蔚采潤而毛采不彰故於義亦取胡廣又曰意謂北方寒涼本以貂皮暖額附施於冠因遂變成首飾沈約曰貂蟬之說因物生義非其實也其實趙武靈王變胡服秦滅趙以其君冠賜侍臣故秦漢以來侍臣有貂蟬也朝直遥翻}
時人為之諺曰貂不足狗尾續|{
	史記曰狐裘雖敝不可補以黄狗之皮亦此意}
是歲天下所舉賢良秀才孝亷皆不試|{
	舊制賢良秀孝皆策試而後補官}
郡國計吏及太學生年十六以上皆署吏守令赦日在職者皆封侯|{
	守式又翻}
郡綱紀並為孝亷縣綱紀並為亷吏|{
	郡綱紀功曹之屬縣綱紀主簿録事史之屬亷吏亦選舉之一科史言倫秀欲以濫恩收衆心}
府庫之儲不足以供賜與應侯者多鑄印不給或以白板封之初平南將軍孫旂之子弼弟子髦輔琰皆附會孫秀與之合族旬月間致位通顯及倫稱帝四子皆為將軍封郡侯以旂為車騎將軍開府旂以弼等受倫官爵過差必為家禍遣幼子回責之弼等不從旂不能制慟哭而已|{
	據晉書孫旂四子並以吏才稱於當世附麗非人至於滅族擇交之難也然孫旂先與孫秀親善故諸子從而附會之擇交之不審何以詔其子哉雖慟哭無益也孫族之赤旂實為之}
癸酉殺濮陽哀王臧孫秀專執朝政倫所出詔令秀輒改更與奪|{
	朝直遥翻更工衡翻}
自書青紙為詔或朝行夕改百官轉易如流張林素與秀不相能且怨不得開府潛與太子荂牋言秀專權不合衆心而功臣皆小人撓亂朝廷|{
	撓火高翻又奴巧翻}
可悉誅之荂以書白倫倫以示秀秀勸倫收林殺之夷其三族秀以齊王冏成都王頴河間王顒各擁彊兵據方面惡之|{
	冏鎮許昌頴鎮鄴顒鎮關中惡烏路翻顒魚容翻}
乃盡用其親黨為三王參佐加冏鎮東大將軍頴征北大將軍皆開府儀同三司以寵安之 李庠驍勇得衆心趙廞浸忌之而未言|{
	驍堅堯翻廞許今翻}
長史蜀郡杜淑張粲說廞曰將軍起兵始爾而遽遣李庠握彊兵於外|{
	謂廞使庠招合壯勇以斷北道也說輸芮翻}
非我族類其心必異此倒戈授人也宜早圖之會庠勸廞稱尊號淑粲因白廞以庠大逆不道引斬之并其子姪十餘人 |{
	考異曰載記曰及其子姪宗族三十餘人今從華陽國志又國志庠死在本年冬晉春秋在今年春今從之}
時李特李流皆將兵在外廞遣人慰撫之曰庠非所宜言罪應死兄弟罪不相及復以特流為督將|{
	將即亮翻}
特流怨廞引兵歸緜竹廞牙門將涪陵許弇求為巴東監軍|{
	涪陵縣漢屬巴郡蜀分為涪陵郡涪音浮監音工銜翻}
杜淑張粲固執不許弇怒手殺淑粲於廞閤下淑粲左右復殺弇|{
	復扶又翻}
三人皆廞之腹心也廞由是遂衰|{
	腹心既死廞無所倚故其勢衰}
廞遣長史犍為費遠|{
	犍居言翻費扶沸翻}
蜀郡太守李苾督護常俊督萬餘人斷北道屯緜竹之石亭|{
	苾毘必翻緜竹縣漢屬廣漢郡晉屬新都郡唐屬漢州斷丁管翻}
李特密收兵得七千餘人夜襲遠等軍燒之死者十八九遂進攻成都費遠李苾及軍祭酒張微夜斬關走文武盡散廞獨與妻乘小船走至廣都為從者所殺|{
	從才用翻}
特入成都縱兵大掠遣使詣洛陽陳廞罪狀初梁州刺史羅尚聞趙廞反表廞非雄才蜀人不附敗亡可計日而待詔拜尚平西將軍益州刺史督牙門將王敦|{
	此别一王敦}
蜀郡太守徐儉廣漢太守辛冉等七千餘人入蜀特等聞尚來甚懼使其弟驤於道奉迎并獻珍玩尚悦以驤為騎督|{
	驤斯將翻騎奇寄翻騎督督騎兵}
特流復以牛酒勞尚於緜竹王敦辛冉說尚曰|{
	復扶又翻勞力到翻說輸芮翻}
特等專為盜賊宜因會斬之不然必為後患尚不從冉與特有舊謂特曰故人相逢不吉當凶矣特深自猜懼三月尚至成都汶山羌反尚遣王敦討之為羌所殺|{
	汶音岷 考異曰帝紀在八月疑是洛陽始知今從華陽國志}
齊王冏謀討趙王倫未發會離狐王盛|{
	離狐縣前漢屬東郡後漢晉屬濟隂郡唐天寶元年改為南華縣屬鄆州}
頴川處穆|{
	晉書作王處穆}
聚衆於濁澤|{
	濁澤在頴川長社縣}
百姓從之日以萬數倫以其將管襲為齊王軍司討盛穆斬之冏因收襲殺之|{
	考異曰齊王冏傳曰冏潛與盛穆謀起兵誅倫未發恐事泄乃與襲殺穆送首於倫以安其意今從三十國春秋}
與豫州刺史何勗龍驤將軍董艾等起兵遣使告成

都王頴河間王顒常山王乂及南中郎將新野公歆|{
	晉志曰四中郎將並後漢置武帝以來四中郎將或領刺史或持節為之歆扶風王駿之子也}
移檄征鎮州郡縣國|{
	征鎮四征四鎮居方面者}
稱逆臣孫秀迷誤趙王當共誅討有不從命者誅及三族使者至鄴成都王頴召鄴令盧志謀之志曰趙王簒逆人神共憤殿下收英俊以從人望杖大順以討之百姓必不召自至攘臂爭進蔑不克矣|{
	蔑無也}
頴從之以志為諮議參軍|{
	諮議參軍晉公府皆置之蓋取諮詢謀議軍事也其位在諸參軍之上}
仍補左長史志毓之孫也|{
	盧毓見七十三卷魏明帝景初元年}
頴以兖州刺史王彦冀州刺史李毅督護趙驤石超等為前鋒遠近響應至朝歌|{
	朝歌縣前漢屬河内郡晉分屬汲郡隋大業二年改朝歌縣為衛縣屬衛州有紂所都朝歌城在縣西}
衆二十餘萬超苞之孫也|{
	石苞事文帝武帝功參佐命}
常山王乂在其國與太原内史劉暾各帥衆為頴後繼|{
	暾他昆翻帥讀曰率}
新野公歆得冏檄未知所從嬖人王綏曰趙親而彊齊疎而弱|{
	歆父扶風王駿與趙王倫皆宣帝子歆於倫為叔姪其屬親冏於歆為從子其屬視倫為疎嬖卑義翻又博計翻}
公宜從趙參軍孫詢大言於衆曰趙王凶逆天下當共誅之何親疎彊弱之有歆乃從冏前安西參軍夏侯奭在始平合衆數千人以應冏遣使邀河間王顒顒用長史李含謀遣振武將軍河間張方討擒奭及其黨腰斬之|{
	沈約志振武將軍始於西漢之末王莽以命王況}
冏檄至顒執冏使送於倫|{
	使疏吏翻}
遣張方將兵助倫方至華隂|{
	華戶化翻}
顒聞二王兵盛復召方還更附二王|{
	二王謂齊王冏成都王頴}
冏檄至揚州州人皆欲應冏刺史郗隆慮之玄孫也|{
	郗丑之翻郗慮漢獻帝時為御史大夫}
以兄子鑒及諸子悉在洛陽疑未决悉召僚吏謀之主簿淮南趙誘前秀才虞潭皆曰趙王簒逆海内所疾今義兵四起其敗必矣為明使君計莫若自將精兵徑赴許昌上策也|{
	齊主冏時鎮許昌}
遣將將兵會之中策也量遣小軍隨形助勝下策也|{
	將息亮翻量音良}
隆退密與别駕顧彦謀之彦曰誘等下策乃上計也治中留寶主簿張褒西曹留承聞之請見曰不審明使君今當何施隆曰我俱受二帝恩|{
	二帝謂宣帝武帝或曰二帝謂惠帝及趙王倫非也}
無所偏助欲守州而已承曰天下世祖之天下也|{
	文帝廟號世祖文帝平諸葛誕滅蜀始宏晉業}
太上承代已久|{
	太上謂惠帝時號太上皇}
今上取之不平|{
	今上謂趙王倫}
齊王順時舉事成敗可見|{
	言齊王冏舉事必成趙王倫必敗也}
使君不早發兵應之狐疑遷延變難將生|{
	難乃旦翻}
此州豈可保也隆不應潭翻之孫也|{
	虞翻事吴主權以直聞}
隆停檄六日不下|{
	停冏檄不下曹下遐嫁翻}
將士憤怨參軍王邃鎮石頭將士爭往歸之隆遣從事於牛渚禁之不能止|{
	平吳之後揚州移鎮秣陵今於牛渚禁將士往石頭疑此時揚州又還治淮南也}
將士遂奉邃攻隆隆父子及顧彦皆死傳首於冏安南將軍監沔北諸軍事孟觀以為紫宫帝座無他變|{
	晉志北極五星鉤陳六星皆在紫宫中鉤陳中一星曰天皇大帝大帝上九星曰華蓋所以覆蔽大帝之座也觀徒占天象而不察諸人事此其所以死也監古銜翻沔□兖翻}
倫必不敗乃為之固守|{
	為于偽翻}
倫秀聞三王兵起大懼|{
	三王謂齊王冏成都王頴河間王顒也}
詐為冏表曰不知何賊猝見攻圍臣懦弱不能自固乞中軍見救|{
	魏晉以禁兵為中軍}
庶得歸死以其表宣示内外遣上軍將軍孫輔折衝將軍李嚴|{
	上軍將軍蓋當時所置沈約志折衝將軍始於建安中曹公以樂進為之}
帥兵七千自延壽關出|{
	晉志河南侯氏縣有延夀城帥讀曰率下同}
征虜將軍張泓左軍將軍蔡璜前軍將軍閭和帥兵九千自崿阪關出|{
	晉志河南陽城縣有崿阪關杜佑曰崿嶺在河南登封縣登封故嵩陽也崿五各翻阪音反}
鎮軍將軍司馬雅揚威將軍莫原|{
	沈約志揚威將軍魏置姓譜莫姓楚莫敖之後}
帥兵八千自成臯關出|{
	晉志河南成臯縣有成臯關}
以拒冏|{
	三路出兵以拒冏}
遣孫秀子會督將軍士猗許超帥宿衛兵三萬以拒頴召東平王楙為衛將軍都督諸軍又遣京兆王馥廣平王䖍帥兵八千為三軍繼援|{
	孫會士猗許超三人所將之軍為三軍}
倫秀日夜禱祈厭勝以求福|{
	厭益葉翻}
使巫覡選戰日|{
	覡他狄翻}
又使人於嵩山著羽衣詐稱仙人王喬作書述倫祚長久欲以惑衆|{
	嵩山中嶽在穎川陽城縣漢武帝分置崈高縣以奉中嶽東漢省併入陽城縣晉陽城縣屬河南郡著陟畧翻劉向列仙傳曰王子喬周靈王太子晉也好吹笙作鳳鳴遊伊洛間道士浮邱公接上嵩山三十餘年後來於山上告桓良曰告我家七月七日待我於氏山頭果乘白鶴駐山嶺望之不得到舉手謝時人而去故倫秀詐以惑衆著陟畧翻}
閏月丙戍朔日有食之自正月至于是月五星互經天縱横無常|{
	志曰傳曰日陽君道也星隂臣道也日出則星亡臣不得專也晝而星見午上為經天其占為不臣為更王今五星悉經天天變所未有也縱子容翻}
張泓等進據陽翟|{
	陽翟縣漢屬頴川郡晉屬河南郡}
與齊王冏戰屢破之冏軍頴隂|{
	穎隂縣在穎川郡穎隂去陽翟四十里}
夏四月泓乘勝逼之冏遣兵逆戰諸軍不動而孫輔徐建軍夜亂徑歸洛自首曰|{
	首式救翻}
齊王兵盛不可當泓等已没矣趙王倫大恐祕之而召其子䖍及許超還|{
	欲召河北之軍還以自衛}
會泓破冏露布至倫乃復遣之|{
	復扶又翻}
泓等悉帥諸軍濟頴攻冏營|{
	潁水出頴川陽城縣少室東南流過陽翟縣之北帥讀曰率下同}
冏出兵擊其别將孫髦司馬譚等破之泓等乃退孫秀詐稱已破冏營擒得冏令百官皆賀成都王頴前鋒至黄橋|{
	朝歌西有黄澤澤水右入蕩水謂之黄雀溝橋當在溝上}
為孫會士猗許超所敗|{
	敗補邁翻}
殺傷萬餘人士衆震駭頴欲退保朝歌盧志王彦曰今我軍失利敵新得志有輕我之心我若退縮士氣沮䘐不可復用|{
	沮在呂翻䘐女六翻復扶又翻}
且戰何能無勝負不若更選精兵星行倍道出敵不意此用兵之奇也|{
	星行者夜行戴星而行也}
頴從之倫賞黄橋之功士猗許超與孫會皆持節由是各不相從軍政不一且恃勝輕頴而不設備頴帥諸軍擊之大戰于溴水|{
	湨水出河内軹縣東南至温入河溴古閴翻 考異曰趙王倫傳作激水今從帝紀}
會等大敗棄軍南走頴乘勝長驅濟河自冏等起兵百官將士皆欲誅倫秀秀懼不敢出中書省及聞河北軍敗憂懣不知所為|{
	懣母木翻又莫困翻}
孫會許超士猗等至與秀謀或欲收餘卒出戰或欲焚宫室誅不附己者挾倫南就孫旂孟觀|{
	孫旂在荆州孟觀在宛}
或欲乘船東走入海計未决辛酉左衛將軍王輿與尚書廣陵公漼|{
	漼取猥翻}
帥營兵七百餘人自南掖門入宫三部司馬為應於内攻孫秀許超士猗於中書省皆斬之遂殺孫奇孫弼及前將軍謝惔等|{
	惔徒甘翻}
漼伷之子也|{
	伷音胄}
王輿屯雲龍門召八坐皆入殿中|{
	坐徂卧翻}
使倫為詔曰吾為孫秀所誤以怒三王今已誅秀其迎太上皇復位吾歸老于農畝傳詔以騶虞幡勑將士解兵|{
	傳詔者使之宣傳詔命因以為官名}
黄門將倫自華林東門出及太子荂皆還汶陽里第|{
	將如字引也荂枯花翻楊士衡音孚洛陽城中有汶陽里倫私第在焉楊正衡曰汶音問}
遣甲士數千迎帝于金墉城百姓咸稱萬歲帝自端門入升殿羣臣頓首謝罪詔送倫荂等赴金墉城廣平王䖍自河北還至九曲|{
	水經注九曲瀆在河南鞏縣西}
聞變棄軍將數十人歸里第癸亥赦天下改元|{
	改元永寧}
大酺五日|{
	酺薄乎翻}
分遣使者慰勞三王|{
	勞力到翻}
梁王肜等表趙王倫父子凶逆宜伏誅丁卯遣尚書袁敞持節賜倫死收其子荂馥䖍詡皆誅之凡百官為倫所用者皆斥免臺省府衛僅有存者|{
	尚書御史謁者臺門下中書祕書省府諸公府也衛二衛及六軍也}
是日成都王頴至己巳河間王顒至頴使趙驤石超助齊王冏討張泓等於陽翟泓等皆降|{
	降戶江翻}
自兵興六十餘日戰鬬死者近十萬人|{
	近其靳翻}
斬張衡閭和孫髦于東市蔡璜自殺五月誅義陽王威襄陽太守宗岱承冏檄斬孫旂|{
	沈約曰魏武帝分南郡編縣以北及南陽之山都立襄陽郡魚豢曰魏文帝立}
永饒冶令空桐機斬孟觀|{
	永饒冶當在南陽宛縣空桐姓機名姓譜曰漢覆姓有空桐空相二氏世本云空同子姓蓋因崆峒山也}
皆傳首洛陽夷三族 立襄陽王尚為皇太孫六月乙卯齊王冏帥衆入洛陽|{
	帥讀曰率}
頓軍通章署甲士數十萬威震京都|{
	晉避景帝諱謂京師曰京都}
戊辰赦天下 復封賓徒王晏為吳王|{
	晏貶見上卷永康元年 考異曰晏傳自賓徒徙封代王倫誅復本封今從帝紀}
甲戍詔以齊王冏為大司馬加九錫備物典策如宣景文武輔魏故事成都王頴為大將軍都督中外諸軍事假黄鉞録尚書事加九錫入朝不趨劒履上殿 |{
	考異曰頴傳曰至鄴詔王粹加九錫進位大將軍都督中外頴拜受徽號讓殊禮按頴在洛盧志已謂頴曰今當與齊王共輔朝政明已有録尚書之命但頴不受歸鄴故朝廷使粹追命之耳且頴功大於冏不應獨賞冏而頴未賞也今從帝紀}
河間王顒為侍中太尉加三賜之禮|{
	記王制諸侯賜弓矢然後征賜鈇鉞然後殺賜圭瓚然後為鬯}
常山王乂為撫軍大將軍領左軍|{
	左軍即左軍將軍所統}
進廣陵公漼爵為王領尚書加侍中進新野公歆爵為王都督荆州諸軍事加鎮南大將軍|{
	歆自南中郎將加鎮南}
齊成都河間三府各置掾屬四十人武號森列|{
	自東漢以來公府皆有掾有屬但不帶武號耳掾以絹翻}
文官備員而已識者知兵之未戢也己卯以梁王肜為太宰領司徒|{
	肜以太師領丞相之職}
光禄大夫劉蕃女為趙世子荂妻故蕃及二子散騎侍郎輿冠軍將軍琨皆為趙王倫所委任|{
	冠古玩翻}
大司馬冏以琨父子有才望特宥之以輿為中書郎|{
	中書郎即中書侍郎}
琨為尚書左丞又以前司徒王戎為尚書令劉暾為御史中丞|{
	暾他昆翻}
王衍為河南尹新野王歆將之鎮|{
	將出鎮荆州也}
與冏同乘謁陵|{
	乘繩正翻}
因說冏曰成都王至親|{
	帝弟之親故曰至親說輸芮翻}
同建大勲今宜留之與輔政若不能爾當奪其兵權常山王乂與成都王頴俱拜陵乂謂頴曰天下者先帝之業王宜維正之聞其言者莫不憂懼|{
	憂懼者以冏與乂頴必阻兵相圖將罹其禍也}
盧志謂頴曰齊王衆號百萬與張泓等相持不能决大王逕前濟河功無與貳今齊王欲與大王共輔朝政志聞兩雄不俱立宜因太妃微疾求還定省|{
	頴母程才人冊為成都太妃記曲禮凡為人子者冬温而夏凊昏定而晨省省悉景翻}
委重齊王以收四海之心|{
	委朝政之重於齊王則四海之人謂穎功大不居將歸心於頴}
此計之上也頴從之帝見頴于東堂慰勞之|{
	勞力到翻}
頴拜謝曰此大司馬冏之勲臣無豫焉因表稱冏功德宜委以萬機自陳母疾請歸藩即辭出不復還營便謁太廟出自東陽城門|{
	洛陽城東面北頭第二門曰東陽門}
遂歸鄴遣信與冏别冏大驚馳出送頴至七里澗及之|{
	水經注鴻臺陂在洛陽東北二十里其水東流左合七里澗武帝泰始十年立城東七里澗石橋}
頴住車言别流涕滂沱|{
	滂沱淚下如雨也}
惟以太妃疾苦為憂不及時事由是士民之譽皆歸頴冏辟新興劉殷為軍諮祭酒洛陽令曹攄為記室督|{
	漢建安三年曹公置軍謀祭酒晉制文武官公及諸方面征鎮府皆置軍諮祭酒漢三公及大將軍府皆有記室令史主上章表奏報書記曹公輔漢以陳琳阮瑀管記室晉諸公府皆有記室督攄抽居翻}
尚書郎江統陽平太守河内苟晞參軍事|{
	晉諸公諸從公為持節都督增參軍為六員}
吳國張翰為東曹掾孫惠為戶曹掾前廷尉正顧榮及順陽王豹為主簿|{
	晉制東曹在倉曹之上戶曹在倉曹之下廷尉屬官有正監平魏分南陽立南鄉郡武帝太康中更名順陽郡掾俞絹翻豹補敎翻}
惠賁之曾孫|{
	孫賁吴主權從兄}
榮雍之孫也|{
	顧雍吳相也}
殷幼孤貧養曾祖母以孝聞|{
	養羊亮翻}
人以穀帛遺之|{
	遺于季翻}
殷受而不謝直云待後貴當相酧耳及長博通經史性倜儻有大志|{
	長知兩翻倜他歷翻倜儻卓異也劉殷後事劉聰貴顯女充聰後宫何足尚也}
儉而不陋清而不介望之頹然而不可侵也冏以何勗為中領軍董艾典樞機又封其將佐有功者葛旟路秀衛毅劉真韓泰皆為縣公委以心膂號曰五公|{
	葛旟牟平公路秀小黄公衛毅隂平公劉真安鄉公韓泰封邱公旟音輿 考異曰路秀帝紀作路季今從齊王冏傳}
成都王頴至鄴詔遣使者就申前命頴受大將軍讓九錫殊禮表論興義功臣皆封公侯|{
	頴亦表封盧志和演董洪王彦趙驤等}
又表稱大司馬前在陽翟與賊相持既久百姓困敝乞運河北邸閣米十五萬斛以賑陽翟饑民造棺八千餘枚以成都國秩為衣服歛祭黄橋戰士|{
	歛力贍翻}
旌顯其家加常戰亡二等又命温縣瘞趙王倫戰士萬四千餘人|{
	此溴水之戰也温縣屬河内郡周司寇蘇忿生之國也瘞於計翻}
皆盧志之謀也頴貌美而神昏不知書然氣性敦厚委事於志故得成其美焉詔復遣使諭頴入輔并使受九錫|{
	復扶又翻}
頴嬖人孟玖不欲還洛|{
	嬖卑義翻又博計翻}
又程太妃愛戀鄴都故頴終辭不拜初大司馬冏疑中書郎陸機為趙王倫撰禪詔收欲殺之大將軍頴為之辯理得免死|{
	為于偽翻}
因表為平原内史以其弟雲為清河内史機友人顧榮及廣陵戴淵以中國多難|{
	難乃旦翻}
勸機還吳機以受頴全濟之恩且謂頴有時望可與立功遂留不去|{
	為陸機陸雲為穎所殺張本}
秋七月復封常山王乂為長沙王|{
	武帝太康十年封乂為長沙王楚王瑋之誅乂以同母貶為常山王今復舊封}
遷開府驃騎將軍 東萊王蕤凶暴使酒數陵侮大司馬冏|{
	數所角翻}
又從冏求開府不得而怨之密表冏專權與左衛將軍王輿謀廢冏事覺八月詔廢蕤為庶人誅輿三族徙蕤於上庸上庸内史陳鍾承冏旨潛殺之 |{
	考異曰帝紀六月庚午蕤與王輿謀廢冏事覺得罪甲戌冏為大司馬按誅輿詔已稱冏為大司馬則輿事覺不應在冏為大司馬前今從三十國春秋在八月}
赦天下 東武公澹坐不孝徙遼東九月徵其弟東安王繇復舊爵|{
	繇廢徙見八十二卷元康元年}
拜尚書左僕射繇舉東平王楙為都督徐州諸軍事鎮下邳 |{
	考異曰楙為平東督徐州九月繇復爵按楙傳繇為僕射舉楙為平東故移在繇還後}
初朝廷符下秦雍州使召還流民入蜀者|{
	下遐嫁翻雍於用翻}
又遣御史馮該張昌督之李特兄輔自略陽至蜀言中國方亂不足復還特然之累遣天水閻式詣羅尚求權停至秋又納賂於尚及馮該尚該許之朝廷論討趙廞功拜特宣威將軍弟流奮武將軍皆封侯璽書下益州條列六郡流民與特同討廞者將加封賞廣漢太守辛冉欲以滅廞為己功寢朝命|{
	寢封拜特流之命也下遐嫁翻朝直遥翻}
不以實上|{
	所謂條列者不以實上上時掌翻下同}
衆咸怨之|{
	六郡之衆也}
羅尚遣從事督遣流民限七月上道|{
	上時掌翻}
時流民布在梁益為人傭力聞州郡逼遣人人愁怨不知所為且水潦方盛年穀未登無以為行資特復遣閻式詣尚求停至冬|{
	復扶又翻下日復復值同}
辛冉及犍為太守李苾以為不可|{
	犍居言翻苾毗必翻}
尚舉别駕杜弢秀才式為弢說逼移利害弢亦欲寛流民一年|{
	弢他刀翻為于偽翻下數為同}
尚用冉苾之謀不從弢乃致秀才板出還家|{
	送至為致}
冉性貪暴欲殺流民首領取其資貨乃與苾白尚言流民前因趙廞之亂多所剽掠宜因移設關以奪取之|{
	剽匹妙翻移即移書也流民安於蜀土雖以朝命驅使還本猶恐其不去况欲設關以奪取其資財是速之為亂也}
尚移書梓潼太守張演於諸要施關搜索寶貨|{
	蜀劉民分廣漢立梓潼郡諸要者凡路所通其地當往來之津要者施關者先未嘗立關今特設之}
特數為流民請留|{
	數所角翻}
流民皆感而恃之多相帥歸特特乃結大營於綿竹以處流民|{
	帥讀曰率處昌呂翻}
移辛冉求自寛冉大怒遣人分牓通衢購募特兄弟許以重賞特見之悉取以歸與弟驤改其購云能送六郡酋豪李任閻趙上官及氐叟侯王一首賞百匹|{
	蜀叟自是一種酋慈由翻任音壬}
於是流民大懼歸特者愈衆旬月間過二萬人流亦聚衆數千人特又遣閻式詣羅尚求申期|{
	申重也求重為期限使流民得自寛也}
式見營柵衝要謀揜流民歎曰民心方危今而速之亂將作矣又知辛冉李苾意不可回乃辭尚還緜竹尚謂式曰子且以吾意告諸流民今聽寛矣式曰明公惑於姦說恐無寛理弱而不可輕者民也今趣之不以理|{
	趣讀曰促}
衆怒難犯|{
	左傳鄭子產之言}
恐為禍不淺尚曰然吾不欺子子其行矣式至緜竹言於特曰尚雖云爾然未可信也何者尚威刑不立冉等各擁彊兵一旦為變亦非尚所能制深宜為備特從之|{
	閻式已覘知冉等之情}
冬十月特分為二營特居北營流居東營繕甲厲兵戒嚴以待之冉苾相與謀曰羅侯貪而無斷|{
	斷丁亂翻}
日復一日令流民得展姦計李特兄弟並有雄才吾屬將為所虜矣宜為决計|{
	欲一戰以决之也}
羅侯不足復問也|{
	復扶又翻}
乃遣廣漢都尉曾元牙門張顯劉並等潛帥步騎三萬襲特營|{
	帥讀曰率}
羅尚聞之亦遣督護田佐助元元等至特安卧不動待其衆半入發伏擊之死者甚衆殺田佐曾元張顯傳首以示尚冉尚謂將佐曰此虜成去矣|{
	謂特雖求申行期而去計已成也}
而廣漢不用吾言以張賊勢|{
	辛冉為廣漢太守故稱之尚言冉輕用兵為特所敗使其勢愈張也張知亮翻}
今若之何於是六郡流民共推特行鎮北大將軍承制封拜以其弟流行鎮東大將軍號東督護以相鎮統又以兄輔為驃騎將軍弟驤為驍騎將軍|{
	驃匹妙翻驍堅堯翻}
進兵攻冉於廣漢|{
	廣漢郡治廣漢縣後宋置遂寧郡齊梁加東字後魏改廣漢縣為方義縣後周改東遂寧為遂州方義為遂州治所}
尚遣李苾費遠帥衆救冉|{
	費扶沸翻}
畏特不敢進冉出戰屢敗潰圍奔德陽|{
	德陽縣後漢置屬廣漢郡至唐屬劒州}
特入據廣漢以李超為太守進兵攻尚於成都尚以書諭閻式式復書曰辛冉傾巧曾元小豎李叔平非將帥之才|{
	李苾字叔平將即亮翻帥所類翻}
式前為節下及杜景文論留徙之宜|{
	晉人稱方面專征之將率曰節下杜弢字景文為于偽翻}
人懷桑梓|{
	桑梓祖父之所樹以遺子孫者故謂懷故鄉者為懷桑梓}
孰不願之但往日初至隨穀庸賃|{
	謂往日流民初至蜀之時無以自給隨所往逐糧出力為人傭作賃女禁翻亦傭也}
一室五分復值秋潦|{
	潦魯皓翻雨水大貌復扶又翻}
乞須冬熟而終不見聽繩之太過窮鹿抵虎流民不肯延頸受刀以致為變即聽式言寛使治嚴|{
	即就也治嚴猶云治裝也冶直之翻}
不過去九月盡集|{
	日月已過者為去}
十月進道令達鄉里何有如此也特以兄輔弟驤子始蕩雄及李含含子國離任回李攀攀弟恭上官晶任臧楊褒上官惇等為將帥閻式李遠等為僚佐羅尚素貪殘為百姓患特與蜀民約灋三章施捨賑貸|{
	杜預曰施恩惠捨勞役}
禮賢抜滯軍政肅然蜀民大悅尚頻為特所敗|{
	敗隋邁翻}
乃阻長圍緣郫水作營連延七百里|{
	水經注綿水西出綿竹縣又與湔水合亦謂之郫江載記曰尚緣水作營自都安至犍為七百里師古曰郫音疲}
與特相拒求救於梁州及南夷校尉|{
	南夷校尉統南中都郡}
十二月穎昌康公何邵薨 封大司馬冏子氷為樂安王英為濟陽王超為淮南王

太安元年|{
	是年十二月齊王冏死方改元太安此猶是永寧二年}
春三月冲太孫尚薨|{
	冲諡也}
夏五月己酉梁孝王肜薨 以右光禄大夫劉寔為太傅尋以老病罷 河間王顒遣督護衙博討李特|{
	姓譜秦穆公子食采於衙因氏焉衙縣漢屬馮翊}
軍于梓潼|{
	梓潼縣漢屬廣漢郡劉蜀分廣漢置梓潼郡唐劒州之梓潼普安黄安求歸武連臨津劒門皆漢梓潼縣地潼音同}
朝廷復以張微為廣漢太守軍于德陽|{
	復扶又翻下同}
羅尚遣督護張龜軍于繁城|{
	繁縣屬蜀郡劉昫曰唐彭州九隴縣漢繁縣地宋白曰益州新繁縣本漢繁縣}
特使其子鎮軍將軍蕩等襲博而自將擊龜破之蕩敗博兵於陽沔|{
	敗補邁翻下所敗同}
梓潼太守張演委城走巴西丞毛植以郡降蕩進攻博於葭萌|{
	巴西郡唐為閬果二州之地劉蜀改漢葭萌縣為漢壽縣晉又改為晉壽此本之漢舊縣名而書之唐為利州之緜谷葭萌二縣地}
博走其衆盡降|{
	降戶江翻}
河間王顒更以許雄為梁州刺史特自稱大將軍益州牧都督梁益二州諸軍事 大司馬冏欲久專大政以帝子孫俱盡|{
	太子遹死帝無子矣虨臧尚死帝無孫矣}
大將軍穎有次立之勢|{
	穎於帝諸弟之次當及}
清河王覃遐之子也方八歲乃上表請立之癸卯立覃為皇太子以冏為太子太師東海王越為司空領中書監 秋八月李特攻張微微擊破之遂進攻特營李蕩引兵救之山道險陿|{
	陿與狹同}
蕩力戰而前遂破微兵特欲還涪蕩及司馬王幸諫曰微軍已敗智勇俱竭宜乘銳氣遂禽之特復進攻微殺之生禽微子存以微喪還之特以其將寋碩守德陽|{
	寋九件翻姓也}
李驤軍毗橋|{
	今懷安軍西北有中江源從漢州彌牟雒水毗橋水三水會為一江懷安軍漢廣漢新都縣之地}
羅尚遣軍擊之屢為驤所敗驤遂進攻成都燒其門李流軍成都之北尚遣精勇萬人攻驤驤與流合擊大破之還者什一二許雄數遣軍攻特不勝|{
	數所角翻}
特勢益盛建寧大姓李叡毛詵逐太守許俊|{
	建寧古滇王國之地漢開置益州郡劉蜀更名建寧郡唐為昆州之地}
朱提大姓李猛逐太守雍約以應特|{
	朱提縣前漢屬犍為郡後漢屬犍為屬國都尉劉蜀分置朱提郡唐為曲州之地朱提蘇林音銖時雍於用翻}
衆各數萬南夷校尉李毅討破之斬詵李猛奉牋降而辭意不遜毅誘而殺之冬十一月丙戍復置寧州|{
	罷寧州見八十一卷武帝太康五年}
以毅為刺史 齊武閔王冏既得志頗驕奢擅權大起府第壞公私廬舍以百數|{
	壞音怪}
制與西宫等中外失望侍中嵇紹上疏曰存不忘亡易之善戒也|{
	易大傳子曰危者有其安者也亡者保其存者也亂者有其治者也君子安而不忘危存而不忘亡治而不忘亂然後身安而國家可保也易曰其亡其亡繫于苞桑}
臣願陛下無忘金墉大司馬無忘頴上大將軍無忘黄橋則禍亂之萌無由而兆矣|{
	齊桓公與鮑叔牙管夷吾甯戚飲酒酣叔牙為壽曰願君無忘在莒時願管子無忘束縛於魯時甯子無忘飯牛車下時嵇紹之言祖其意}
又與冏書以為唐虞茅茨夏禹卑宫|{
	唐虞采椽不斵茅茨不翦禹卑宫室}
今大興第舍及為三王立宅|{
	為于偽翻}
豈今日之急邪冏遜辭謝之然不能從冏耽於宴樂不入朝見|{
	樂音洛朝直遥翻見賢遍翻}
坐拜百官|{
	坐受百官之拜也一說天子用三公九卿諸將軍猶引而拜之今冏安坐府第拜受百官也}
符勑三臺選用不均|{
	以私意選用符勑三臺使奉行不均之大者也}
嬖寵用事|{
	凡史書其人將敗必先叙其致敗之由此左氏傳例}
殿中御史桓豹奏事不先經冏府即加考竟|{
	魏制蘭臺遣二御史居殿中伺察非法及晉置四人史言冏但欲專權考竟殿中御史不知無君之迹愈著}
南陽處士鄭方|{
	處昌呂翻}
上書諫冏曰今大王安不慮危宴樂過度一失也|{
	樂音洛}
宗室骨肉當無纎介今則不然二失也蠻夷不靜大王謂功業已隆不以為念三失也|{
	蠻夷不静謂李特等寇亂梁益也}
兵革之後百姓窮困不聞賑救四失也|{
	此一失蓋指成都王穎運米以收河南人心而不敢察察言之耳}
大王與義兵盟約事定之後賞不踰時而今猶有功未論者五失也|{
	兵法曰賞不踰時欲民速得為善之利也此言頴上之功猶有未叙者}
冏謝曰非子孤不聞過孫惠上書曰天下有五難四不可而明公皆居之冒犯鋒刃一難也|{
	冒莫北翻}
聚致英豪二難也與將士均勞苦三難也|{
	將即亮翻}
以弱勝彊四難也興復皇業五難也大名不可久荷|{
	荷下可翻}
大功不可久任大權不可久執大威不可久居大王行其難而不以為難|{
	謂在穎上時也}
處其不可而謂之可|{
	惠之此言婉而切處昌呂翻}
惠切所不安也明公宜思功成身退之道|{
	老子曰功成名遂身退天之道}
崇親推近委重長沙成都二王長揖歸藩則太伯子臧不專美於前矣|{
	吴太伯以天下讓曹子臧以國讓}
今乃忘高亢之可危|{
	亢口浪翻高極為亢}
貪權埶以受疑雖遨遊高臺之上逍遥重墉之内|{
	重直龍翻}
愚竊謂危亡之憂過於在頴翟之時也|{
	頴翟謂頴川陽翟也}
冏不能用惠辭疾去冏謂曹攄曰或勸吾委權還國何如攄曰物禁太盛大王誠能居高慮危褰裳去之斯善之善者也冏不聽張翰顧榮皆慮及禍翰因秋風起思菰菜蓴羮鱸魚鱠|{
	菰一名蔣本草曰菰又謂之茭歲久中心生白臺謂之菰米其臺中有黑者謂之茭至後結實乃雕胡黑米也蓴生水中葉似鳬茨春夏細長肥滑三月至八月為絲蓴九月至十一月為猪蓴鱸魚出吳松江者佳吳人以為鱠甚美蓴殊倫翻}
歎曰人生貴適志耳富貴何為即引去榮故酣飲不省府事|{
	省悉景翻}
長史葛旟以其廢職白冏徙榮為中書侍郎頴川處士庾衮|{
	姓譜庾姓堯時為掌庾大夫因氏焉處昌呂翻下處要同}
聞冏朞年不朝歎曰晉室卑矣禍亂將興帥妻子逃於林慮山中|{
	帥讀曰率慮音廬}
王豹致牋於冏曰伏思元康以來宰相在位未有一人獲終者|{
	元康元年楊駿誅繼而汝南王亮死永康元年張華裴頠死}
乃事埶使然非皆為不善也今公克平禍亂安國定家乃復尋覆車之軌|{
	復扶又翻}
欲冀長存不亦難乎今河間樹根於關右成都盤桓於舊魏|{
	曹魏以鄴都基王業故謂之舊魏}
新野大封於江漢三王方以方剛強盛之年並典戎馬處要害之地而明公以難賞之功挾震主之威獨據京都專執大權進則亢龍有悔|{
	易乾上九爻辭象曰亢龍有悔盈不可久也}
退則據于蒺藜|{
	易困六三爻辭陶宏景曰蒺藜多生道上而葉布地子有刺狀若菱而小有三角長安最饒人以故多著木屐今軍家乃鑄鉄作之以布敵路亦呼為蒺藜易云據于蒺藜言其凶傷也爾雅翼茨蒺藜詩曰牆有茨蒺昨失翻藜力脂翻又力兮翻}
冀此求安未見其福也因請悉遣王侯之國|{
	豹因此語掇長沙王乂之怒以殺其身}
依周召之法|{
	召讀曰邵}
以成都王為北州伯治鄴冏自為南州伯治宛分河為界各統王侯以夾輔天子|{
	周之時周召分陜而治為二伯以夾輔王室故王豹欲依以為法宛於元翻}
冏優令答之長沙王乂見豹牋謂冏曰小子離間骨肉何不銅駞下打殺冏乃奏豹讒内間外|{
	間右莧翻}
坐生猜嫌不忠不義鞭殺之豹將死曰縣吾頭大司馬門見兵之攻齊也|{
	縣讀曰懸昔伍子胥為吴王夫差所殺將死曰縣吾目於吳東門見越之入吳也豹倣此語}
冏以河間王顒本附趙王倫心常恨之梁州刺史安定皇甫商與顒長史李含不平含被徵為翊軍校尉時商參冏軍事夏侯奭兄亦在冏府含心不自安|{
	顒附趙王倫奭為顒所殺事並見上永寧元年}
又與冏右司馬趙驤有隙遂單馬犇顒詐稱受密詔使顒誅冏因說顒曰|{
	說輸芮翻}
成都王至親有大功推讓還藩甚得衆心|{
	推吐雷翻}
齊王越親而專政朝廷側目今檄長沙王使討齊齊王必誅長沙吾因以為齊罪而討之必可禽也去齊立成都|{
	去羌呂翻}
除逼建親以安社稷大勲也顒從之是時武帝族弟范陽王虓都督豫州諸軍事|{
	虓宣帝弟東武城侯馗之少子虓虛交翻}
顒上表陳冏罪狀且言勒兵十萬欲與成都王頴新野王歆范陽王虓共會洛陽請長沙王乂廢冏還第以頴代冏輔政顒遂舉兵以李含為都督帥張方等趨洛陽復遣使邀頴|{
	帥讀曰率趨七喻翻復扶又翻}
頴將應之盧志諫不聽十二月丁卯顒表至冏大懼會百官議之曰孤首唱義兵臣子之節信著神明今二王信讒作難將若之何|{
	二王謂河間王顒成都王頴難乃旦翻}
尚書令王戎曰公勲業誠大然賞不及勞故人懷貳心今二王兵盛不可當也若以王就第委權崇讓庶可求安冏從事中郎葛旟怒曰三臺納言不恤王事|{
	謂尚書也}
賞報稽緩|{
	賞以報功故曰賞報稽留也緩遟也}
責不在府|{
	自謂過不在齊府也}
讒言逆亂當共誅討奈何虛承偽書遽令公就第乎漢魏以來王侯就第寧有得保妻子者邪議者可斬百官震悚失色戎偽藥發墮厠得免李含屯隂盤|{
	魏收地形志隂盤縣漢屬安定郡晉屬京兆郡鴻門戲水皆在縣界余按漢京兆與馮翊以渭水為界安定在馮翊之北晉安得割安定之隂盤以屬京兆邪此魏收之誤也水經注冷水逕隂盤新豐兩原之間北流注于渭漢靈帝建寧三年改新豐為都鄉封段頴為侯國後立隂槃城其水際城北出謂是水為隂槃水又北絶漕槃溝注于渭是則李含所屯之隂盤也五代史志隋廢後魏平涼郡入隂盤縣地形志涇州有平原郡治隂盤縣一志之間兩隂盤並載而不覺其誤以是見史學之難精也劉昫曰唐涇州良原縣隋隂盤縣是即漢安定之隂盤縣宋白曰京兆昭應縣東十三里有漢新豐縣故城亦謂之隂盤城後漢靈帝未移安定隂盤縣寄理於此是即京兆之隂盤也}
張方帥兵二萬軍新安|{
	新安縣漢屬宏農郡晉屬河南郡帥讀曰率}
檄長沙王乂使討冏冏遣董艾襲乂乂將左右百餘人馳入宫閉諸門奉天子攻大司馬府董艾陳兵宫西縱火燒千秋神武門|{
	千秋神武門宫西門也東漢曰神虎晉及南北諸史皆唐羣臣所定唐太祖諱虎置之改為武}
冏使人執騶虞幡唱云長沙王乂矯詔乂又稱大司馬謀反是夕城内大戰飛矢雨集火光屬天|{
	屬之欲翻}
帝幸上東門|{
	此上東門非洛城之上東門宫城之上東門也}
矢集御前羣臣死者相枕|{
	枕職鴆翻}
連戰三日冏衆大敗大司馬長史趙淵殺何勗因執冏以降|{
	何勗與冏同起兵時為中領軍降戶江翻}
冏至殿前帝惻然欲活之乂叱左右趣牽出|{
	趣讀曰促}
斬於閶闔門外|{
	水經注曰按禮王有五門謂臯門庫門雉門惠門路門魏明帝上法太極於洛陽南宫起太極殿于漢崇德殿之故處改雉門曰閭闔門余按天門曰閶闔法以名門又按晉志洛陽城西有廣陽西明閶闔三門未知孰是此時怱怱奚暇牽冏出都城西門乎此必宫城之閶闔門也}
徇首六軍同黨皆夷三族死者二千餘人囚冏子超氷英於金墉城廢冏弟北海王寔赦天下改元|{
	改元太安}
李含等聞冏死引兵還長安長沙王乂雖在朝廷事無巨細皆就鄴諮大將軍頴頴以孫惠為參軍陸雲為右司馬 是歲陳留王薨謚曰魏元皇帝|{
	晉受魏禪奉魏帝為陳留王}
鮮卑宇文單于莫圭部衆彊盛遣其弟屈雲攻慕容廆廆擊其别帥素怒延破之|{
	單音蟬帥所類翻}
素怒延恥之復發兵十萬圍廆於棘城|{
	復扶又翻}
廆衆皆懼廆曰素怒延兵雖多而無灋制已在吾算中矣諸君但為力戰|{
	為于偽翻}
無所憂也遂出擊大破之追犇百里俘斬萬計 |{
	考異曰載記作素延下云素延怒率衆圍棘城按燕書紀傳皆謂之素怒延然則怒延是其名也}
遼東孟暉先没於宇文部帥其衆數千家降于廆|{
	帥讀曰率降戶江翻}
廆以為建威將軍廆以其臣慕輿句勤恪亷靖使掌府庫句心計默識|{
	識音志記也}
不按簿書始終無漏以慕輿河明敏精審使典獄訟覆訊清允|{
	慕輿蓋亦鮮卑之種别為一姓史言慕容廆善用人}


資治通鑑卷八十四














































































































































