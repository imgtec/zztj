<!DOCTYPE html PUBLIC "-//W3C//DTD XHTML 1.0 Transitional//EN" "http://www.w3.org/TR/xhtml1/DTD/xhtml1-transitional.dtd">
<html xmlns="http://www.w3.org/1999/xhtml">
<head>
<meta http-equiv="Content-Type" content="text/html; charset=utf-8" />
<meta http-equiv="X-UA-Compatible" content="IE=Edge,chrome=1">
<title>資治通鑒_92-資治通鑑卷九十一_92-資治通鑑卷九十一</title>
<meta name="Keywords" content="資治通鑒_92-資治通鑑卷九十一_92-資治通鑑卷九十一">
<meta name="Description" content="資治通鑒_92-資治通鑑卷九十一_92-資治通鑑卷九十一">
<meta http-equiv="Cache-Control" content="no-transform" />
<meta http-equiv="Cache-Control" content="no-siteapp" />
<link href="/img/style.css" rel="stylesheet" type="text/css" />
<script src="/img/m.js?2020"></script> 
</head>
<body>
 <div class="ClassNavi">
<a  href="/24shi/">二十四史</a> | <a href="/SiKuQuanShu/">四库全书</a> | <a href="http://www.guoxuedashi.com/gjtsjc/"><font  color="#FF0000">古今图书集成</font></a> | <a href="/renwu/">历史人物</a> | <a href="/ShuoWenJieZi/"><font  color="#FF0000">说文解字</a></font> | <a href="/chengyu/">成语词典</a> | <a  target="_blank"  href="http://www.guoxuedashi.com/jgwhj/"><font  color="#FF0000">甲骨文合集</font></a> | <a href="/yzjwjc/"><font  color="#FF0000">殷周金文集成</font></a> | <a href="/xiangxingzi/"><font color="#0000FF">象形字典</font></a> | <a href="/13jing/"><font  color="#FF0000">十三经索引</font></a> | <a href="/zixing/"><font  color="#FF0000">字体转换器</font></a> | <a href="/zidian/xz/"><font color="#0000FF">篆书识别</font></a> | <a href="/jinfanyi/">近义反义词</a> | <a href="/duilian/">对联大全</a> | <a href="/jiapu/"><font  color="#0000FF">家谱族谱查询</font></a> | <a href="http://www.guoxuemi.com/hafo/" target="_blank" ><font color="#FF0000">哈佛古籍</font></a> 
</div>

 <!-- 头部导航开始 -->
<div class="w1180 head clearfix">
  <div class="head_logo l"><a title="国学大师官网" href="http://www.guoxuedashi.com" target="_blank"></a></div>
  <div class="head_sr l">
  <div id="head1">
  
  <a href="http://www.guoxuedashi.com/zidian/bujian/" target="_blank" ><img src="http://www.guoxuedashi.com/img/top1.gif" width="88" height="60" border="0" title="部件查字,支持20万汉字"></a>


<a href="http://www.guoxuedashi.com/help/yingpan.php" target="_blank"><img src="http://www.guoxuedashi.com/img/top230.gif" width="600" height="62" border="0" ></a>


  </div>
  <div id="head3"><a href="javascript:" onClick="javascript:window.external.AddFavorite(window.location.href,document.title);">添加收藏</a>
  <br><a href="/help/setie.php">搜索引擎</a>
  <br><a href="/help/zanzhu.php">赞助本站</a></div>
  <div id="head2">
 <a href="http://www.guoxuemi.com/" target="_blank"><img src="http://www.guoxuedashi.com/img/guoxuemi.gif" width="95" height="62" border="0" style="margin-left:2px;" title="国学迷"></a>
  

  </div>
</div>
  <div class="clear"></div>
  <div class="head_nav">
  <p><a href="/">首页</a> | <a href="/ShuKu/">国学书库</a> | <a href="/guji/">影印古籍</a> | <a href="/shici/">诗词宝典</a> | <a   href="/SiKuQuanShu/gxjx.php">精选</a> <b>|</b> <a href="/zidian/">汉语字典</a> | <a href="/hydcd/">汉语词典</a> | <a href="http://www.guoxuedashi.com/zidian/bujian/"><font  color="#CC0066">部件查字</font></a> | <a href="http://www.sfds.cn/"><font  color="#CC0066">书法大师</font></a> | <a href="/jgwhj/">甲骨文</a> <b>|</b> <a href="/b/4/"><font  color="#CC0066">解密</font></a> | <a href="/renwu/">历史人物</a> | <a href="/diangu/">历史典故</a> | <a href="/xingshi/">姓氏</a> | <a href="/minzu/">民族</a> <b>|</b> <a href="/mz/"><font  color="#CC0066">世界名著</font></a> | <a href="/download/">软件下载</a>
</p>
<p><a href="/b/"><font  color="#CC0066">历史</font></a> | <a href="http://skqs.guoxuedashi.com/" target="_blank">四库全书</a> |  <a href="http://www.guoxuedashi.com/search/" target="_blank"><font  color="#CC0066">全文检索</font></a> | <a href="http://www.guoxuedashi.com/shumu/">古籍书目</a> | <a   href="/24shi/">正史</a> <b>|</b> <a href="/chengyu/">成语词典</a> | <a href="/kangxi/" title="康熙字典">康熙字典</a> | <a href="/ShuoWenJieZi/">说文解字</a> | <a href="/zixing/yanbian/">字形演变</a> | <a href="/yzjwjc/">金 文</a> <b>|</b>  <a href="/shijian/nian-hao/">年号</a> | <a href="/diming/">历史地名</a> | <a href="/shijian/">历史事件</a> | <a href="/guanzhi/">官职</a> | <a href="/lishi/">知识</a> <b>|</b> <a href="/zhongyi/">中医中药</a> | <a href="http://www.guoxuedashi.com/forum/">留言反馈</a>
</p>
  </div>
</div>
<!-- 头部导航END --> 
<!-- 内容区开始 --> 
<div class="w1180 clearfix">
  <div class="info l">
   
<div class="clearfix" style="background:#f5faff;">
<script src='http://www.guoxuedashi.com/img/headersou.js'></script>

</div>
  <div class="info_tree"><a href="http://www.guoxuedashi.com">首页</a> > <a href="/SiKuQuanShu/fanti/">四库全书</a>
 > <h1>资治通鉴</h1> <!--         下载:【右键另存为】即可 --></div>
  <div class="info_content zj clearfix">
  
<div class="info_txt clearfix" id="show">
<center style="font-size:24px;">92-資治通鑑卷九十一</center>
    資治通鑑卷九十一   宋 司馬光 撰<br />
<br />
  胡三省 音註<br />
<br />
  晉紀十三【起屠維單閼盡重光大荒落凡三年】<br />
<br />
  中宗元皇帝中<br />
<br />
  大興二年春二月劉遐徐龕擊周撫於寒山破斬之【魏收地形志彭城郡彭城縣有寒山龕苦含翻】初掖人蘇峻帥鄉里數千家結壘以自保遠近多附之【掖縣屬東萊郡蘇峻傳云長廣掖人據志長廣郡有挺縣無掖縣帥讀曰率】曹嶷惡其彊將攻之峻率衆浮海來奔【嶷魚力翻惡烏路翻】帝以峻為鷹揚將軍【沈約志鷹揚將軍建安中曹公以命曹洪】助劉遐討周撫有功詔以遐為臨淮太守峻為淮陵内史【惠帝元康七年分臨淮置淮陵郡其地當在唐沂州臨沂縣界宋白曰泗州招信縣本漢淮陵縣】 石勒遣左長史王修獻捷於漢漢主曜遣兼司徒郭汜授勒太宰領大將軍進爵趙王加殊禮出警入蹕如曹公輔漢故事【汜音祀】拜王修及其副劉茂皆為將軍封列侯修舍人曹平樂從修至粟邑【樂音洛】因留仕漢言於曜曰大司馬遣修等來【曜初即位以勒為大司馬故稱之】外表至誠内覘大駕彊弱俟其復命將襲乘輿【覘丑亷翻乘繩證翻】時漢兵實疲弊曜信之乃追汜還斬修於市三月勒還至襄國劉茂逃歸言修死狀勒大怒曰孤事劉氏於人臣之職有加矣彼之基業皆孤所為今既得志還欲相圖趙王趙帝孤自為之何待於彼邪乃誅曹平樂三族【為劉石相攻張木】 帝令羣臣議郊祀尚書令刁協等以為宜須還洛乃修之司徒荀組等曰漢獻帝都許即行郊祀【范書漢獻帝建安元年郊祀上帝於安邑是年七月至洛陽復郊祀上帝八月遷許無郊祀之事或别見他書也晉書禮志載組議云獻帝遷許即便立郊蓋郊祀不在遷許之年也】何必洛邑帝從之立郊丘於建康城之己地辛卯帝親祀南郊以未有北郊【按成帝咸和八年始於覆舟山南立北郊】并地祇合祭之詔琅邪恭王宜稱皇考賀循曰禮子不敢以己爵加於父【此前漢師丹引禮以為言事見三十三卷漢哀帝建平元年】乃止初蓬陂塢主陳川【蓬陂即左傳之蓬澤在浚儀縣】自稱陳留太守【守式又翻】祖逖之攻樊雅也川遣其將李頭助之頭力戰有功逖厚遇之頭每嘆曰得此人為主吾死無恨川聞而殺之頭黨馮寵帥其衆降逖川益怒大掠豫州諸郡逖遣兵擊破之夏四月川以浚儀叛降石勒【浚儀縣屬陳留郡故大梁也帥讀曰率降戶江翻下同】 周撫之敗走也徐龕部將于藥追斬之及朝廷論功而劉遐先之【先悉薦翻】龕怒以泰山叛降石勒自稱兖州刺史 漢王曜還都長安【自粟邑還長安遂定都也】立妃羊氏為皇后【即惠帝羊皇后曜納羊后見八十七卷懷帝永嘉五年】子熙為皇太子封子襲為長樂王【樂音洛】闡為太原王冲為淮南王敞為齊王高為魯王徽為楚王諸宗室皆進封郡王羊氏即故惠帝后也曜嘗問之曰吾如何司馬家兒羊氏曰陛下開基之聖主彼亡國之暗夫何可並言彼貴為帝王有一婦一子及身三耳曾不能庇妾於爾時實不欲生意謂世間男子皆然自奉巾櫛已來始知天下自有丈夫耳曜甚寵之頗干預國事 南陽王保自稱晉王改元建康置百官以張寔為征西大將軍開府儀同三司陳安自稱秦州刺史降于漢又降于成上邽大飢士衆困迫張春奉保之南安祁山【之往也】寔遣韓璞帥步騎五千救之陳安退保緜諸【緜諸道前漢屬天水郡後漢晉省水經註緜諸水歷緜諸故道北東東入清水清水東南注渭】保歸上邽未幾保復為安所逼【幾居豈翻復扶又翻】寔遣其將宋毅救之安乃退 江東大饑詔百官各上封事益州刺史應詹上疏曰【詹自益州刺史還建康】元康以來賤經尚道以玄虚宏放為夷達【夷曠也】以儒術清儉為鄙俗宜崇奬儒官以新俗化 祖逖攻陳川于蓬關石勒遣石虎將兵五萬救之戰于浚儀逖兵敗退屯梁國勒又遣桃豹將兵至蓬關逖退屯淮南【此淮南郡治夀春】虎徙川部衆五千戶于襄國留豹守川故城 石勒遣石虎擊鮮卑日六延於朔方大破之斬首二萬級俘虜三萬餘人孔萇攻幽州諸郡悉取之段匹磾士衆饑散欲移保上谷【晉志上谷郡治沮陽縣秦置郡在谷之上頭故名焉】代王鬱律勒兵將擊之匹磾棄妻子奔樂陵依邵續【樂陵郡治厭次續保之以奉晉】 曹嶷遣使賂石勒請以河為境勒許之【嶷已緣河置戌矣今賂勒請以河為境者懼勒之侵軼也】 梁州刺史周訪擊杜曾大破之馬雋等執曾以降訪斬之并獲荆州刺史第五猗送於武昌訪以猗本中朝所署【朝直遥翻】加有時望白王敦不宜殺敦不聽而斬之【猗從杜曾事始八十九年愍帝建興四年】初敦患杜曾難制謂訪曰若擒曾當相論為荆州及曾死而敦不用王廙在荆州【廙羊至翻又逸職翻】多殺陶侃將佐【將即亮翻】以皇甫方回為侃所敬責其不詣已收斬之士民怨怒上下不安帝聞之徵廙為散騎常侍以周訪代廙為荆州刺史王敦忌訪威名意難之從事中郎郭舒說敦曰鄙州雖荒弊乃用武之國不可以假人宜自領之【郭舒先在荆州歷事劉弘王澄說輸芮翻】訪為梁州足矣敦從之六月丙子詔加訪安南將軍餘如故訪大怒敦手書譬解并遺玉環玉椀以申厚意【遺于季翻】訪抵之於地曰吾豈賈豎可以寶悅邪【賈音古】訪在襄陽務農訓兵隂有圖敦之志守宰有缺輒補然後言上【上時掌翻】敦患之而不能制魏該為胡寇所逼自宜陽率衆南遷新野【魏該自懷帝末屯宜陽界一泉塢宜陽縣屬弘農郡新野縣漢屬南陽郡晉屬義陽郡】助周訪討杜曾有功拜順陽太守趙固死郭誦留屯陽翟【陽翟縣漢屬潁川郡晉屬河南郡】石生屢攻之不能克 漢主曜立宗廟社稷南北郊於長安詔曰吾之先興於北方光文立漢宗廟以從民望【見八十五卷惠帝永興元年】今宜改國號以單于為祖亟議以聞羣臣奏光文始封盧奴伯【晉成都王穎封劉淵為盧奴伯】陛下又王中山中山趙分也【王于况翻分扶問翻】請改國號為趙從之以冒頓配天【冒莫北翻】光文配上帝 徐龕寇掠濟岱【岱泰山也龕寇掠濟岱之間濟子禮翻】破東莞【沈約志武帝太康元年分琅邪立東莞郡晉志東莞故魯鄆邑劉昫曰唐沂州沂水縣漢東莞縣地宋白曰春秋莒魯爭鄆杜預註云城陽姑幕縣南有員亭即鄆也俗變其字耳十三州志云有東西二鄆魯昭公所居者為西鄆兖州東平郡是也莒魯所爭者為東鄆漢東莞縣是也莞音官】帝問將帥可以討龕者於王導【將即亮翻帥所類翻】導以為太子左衛率泰山羊鑒龕之州里冠族【冠古玩翻】必能制之鑒深辭才非將帥郗鑒亦表鑒非才不可使導不從秋八月以羊鑒為征虜將軍征討都督督徐州刺史蔡豹臨淮太守劉遐鮮卑段文鴦等討之【段文鴦時從其兄匹磾在厭次】 冬石勒左右長史張敬張賓左右司馬張屈六程遐等勸勒稱尊號勒不許十一月將佐等復請勒稱大將軍大單于領冀州牧趙王【復扶又翻單音蟬】依漢昭烈在蜀魏武在鄴故事以河内等二十四郡為趙國太守皆為内史凖禹貢復冀州之境【時以河内魏汲頓丘平原清河鉅鹿常山中山長樂樂平趙國廣平陽平章武勃海河間上黨定襄范陽漁陽武邑燕國樂陵二十四郡為趙國凖禹貢魏武復冀州之境南至孟津西達龍門東至于河北至塞垣】以大單于鎮撫百蠻罷并朔司三州【晉未嘗置朔州此罷朔州未知誰所置也】通置部司以監之勒許之戊寅即趙王位【石勒字世龍】大赦依春秋時列國稱元年初勒以世亂律令煩多命法曹令史貫志【貫姓也志其名】采集其要作辛亥制五千文施行十餘年乃用律令以理曹參軍上黨續咸為律學祭酒【姓譜帝舜七友有續牙曰晉大夫狐鞫居食采於續號續簡伯後以為氏】咸用法詳平國人稱之以中壘將軍支雄【中壘將軍後趙創置】游擊將軍王陽領門臣祭酒【勒置經學祭酒律學祭酒史學祭酒門臣祭酒】專主胡人辭訟重禁胡人不得陵侮衣冠華族【華族中華之族也勒胡人也能禁其醜類不使陵暴華人及衣冠之士晉文公初欲俘陽樊之民殆有愧焉】號胡為國人遣使循行州郡勸課農桑朝會始用天子禮樂衣冠儀物從容可觀矣【朝直遥翻從于容翻】加張賓大執灋專總朝政【朝直遥翻下同】以石虎為單于元輔都督禁衛諸軍事尋加驃騎將軍侍中開府賜爵中山公【驃匹妙翻】自餘羣臣授位進爵各有差張賓任遇優顯羣臣莫及而謙虛敬慎開懷下士屏絶阿私【屏必郢翻】以身帥物【帥讀曰率】入則盡規出則歸美勒甚重之每朝常為之正容貌簡辭令呼曰右侯而不敢名【史言張賓有大臣之節所以膺石勒之禮貌為于偽翻】 十二月乙亥大赦平州刺史崔毖自以中州人望鎮遼東【毖崔琰之曾孫琰在魏時為冀州人士之首子孫遂為冀州冠族毖音袐】而士民多歸慕容廆【廆戶罪翻】心不平數遣使招之皆不至【數所角翻】意廆拘留之乃隂說高句麗段氏宇文氏使共攻之【說輸芮翻句音如字又音駒麗力知翻】約滅廆分其地毖所親勃海高瞻力諫毖不從三國合兵代廆諸將請擊之廆曰彼為崔毖所誘欲邀一切之利軍勢初合其鋒甚鋭不可與戰當固守以挫之彼烏合而來【飛鳥見食羣集而聚啄之人或驚之則四散飛去故兵以利合無所統一者謂之烏合】既無統壹莫相歸服久必攜貳一則疑吾與毖詐而覆之二則三國自相猜忌待其人情離貳然後擊之破之必矣三國進攻棘城廆閉門自守遣使獨以牛酒犒宇文氏【使疏吏翻犒苦告翻】二國疑宇文氏與廆有謀各引兵歸【兵法所謂合則能離之慕容廆有焉】宇文大人悉獨官曰二國雖歸吾當獨取之宇文氏士卒數十萬連營口四十里廆使召其子翰於徒河【翰自愍帝建興元年鎮徒河】翰遣使白廆曰悉獨官舉國為寇彼衆我寡易以計破難以力勝今城中之衆足以禦寇翰請為奇兵於外伺其間而擊之【間古莧翻下同】内外俱奮使彼震駭不知所備破之必矣今并兵為一彼得專意攻城無復它虞【虞防也備也復扶又翻下同】非策之得者也且示衆以怯恐士氣不戰先沮矣【沮在呂翻】廆猶疑之遼東韓夀言於廆曰悉獨官有憑陵之志將驕卒惰軍不堅密若奇兵卒起【卒讀曰猝】掎其無備必破之策也【掎舉綺翻偏引曰掎又從後牽曰掎】廆乃聽翰留徒河悉獨官聞之曰翰素名驍果【驍堅堯翻】今不入城或能為患當先取之城不足憂乃分遣數千騎襲翰翰知之詐為段氏使者逆於道曰慕容翰久為吾患聞當擊之吾已嚴兵相待宜速進也使者既去翰即出城設伏以待之宇文氏之騎見使者大喜馳行不復設備進入伏中翰奮擊盡獲之乘勝徑進遣間使語廆出兵大戰【投間隙而行故謂之間使間古莧翻】廆使其子皝與長史裴嶷將精鋭為前鋒【皝呼廣翻】自將大兵繼之悉獨官初不設備聞廆至驚悉衆出戰前鋒始交翰將千騎從旁直入其營縱火焚之【將即亮翻】衆皆惶擾不知所為遂大敗悉獨官僅以身免廆盡俘其衆獲皇帝玉璽三鈕【皇帝璽即宇文大人普回出獵所得者璽斯氏翻】崔毖聞之懼使其兄子燾詣棘城偽賀會三國使者亦至請和曰非我本意崔平州教我耳廆以示燾臨之以兵燾懼首服【首式救翻】廆乃遣燾歸謂毖曰降者上策走者下策也引兵隨之毖與數十騎棄家犇高句麗其衆悉降於廆【降戶江翻】廆以其子仁為征虜將軍鎮遼東【為仁以遼東與皝爭國張本】官府市里案堵如故高句麗將如奴子據于河城廆遣將軍張統掩擊擒之俘其衆千餘家以崔燾高瞻韓恒石琮歸于棘城待以客禮恒安平人琮鑒之孫也【石鑒事武帝惠帝位通顯】廆以高瞻為將軍瞻稱疾不就廆數臨候之【數所角翻】撫其心曰君之疾在此不在它也今晉室喪亂孤欲與諸君共清世難【喪息浪翻難乃旦翻】翼戴帝室君中州望族宜同斯願柰何以華夷之異介然踈之哉【介然堅正不移之貌】夫立功立事惟問志略何如耳華夷何足問乎【以瞻薄廆起於東夷不肯委身事之故有是言】瞻猶不起廆頗不平龍驤主簿宋該與瞻有隙【廆進號龍驤將軍以該為府主簿驤思將翻】勸廆除之廆不從瞻以憂卒 初鞠羨既死【鞠羨死見八十六卷懷帝永嘉元年】苟晞復以羨子彭為東萊太守【復扶又翻】會曹嶷狥青州【事見八十七卷永嘉三年嶷魚力翻】與彭相攻嶷兵雖彊郡人皆為彭死戰【為于偽翻】嶷不能克久之彭歎曰今天下大亂彊者為雄曹亦鄉里【彭與嶷皆齊人】為天所相【相悉亮翻】苟可依憑即為民主何必與之力爭使百姓肝腦塗地吾去此則禍自息矣郡人以為不可爭獻拒嶷之策彭一無所用與鄉里千餘家浮海歸崔毖北海鄭林客於東萊彭嶷之相攻林情無彼此嶷賢之不敢侵掠彭與之俱去比至遼東【比必寐翻】毖已敗乃歸慕容廆廆以彭參龍驤軍事遺鄭林車牛粟帛【遺于季翻】皆不受躬耕于野宋該勸廆獻捷江東廆使該為表裴嶷奉之并所得三璽詣建康獻之高句麗數寇遼東【句如字又音駒麗力知翻數所角翻】廆遣慕容翰慕容仁伐之高句麗王乙弗利逆來求盟翰仁乃還【還從宣翻又如字】 是歲蒲洪降趙【考異曰三十國晉春秋洪降劉曜在大興元年案元年曜未都長安晉書洪載記無年但云曜僭號長安洪歸曜故置是年】趙主曜以洪為率義侯 屠各路松多起兵於新平扶風以附晉王保【屠直於翻】保使其將楊曼王連據陳倉張顗周庸據隂密松多據草壁【水經註隴山西南降隴城北有松多川蓋松多據此因以為地名草壁在隂密之東顗魚豈翻】秦隴氐羌多應之趙主曜遣諸將攻之不克曜自將擊之【將即亮翻】<br />
<br />
  三年春正月曜攻陳倉王連戰死楊曼犇南氐【氏種之居陳倉南者即仇池楊氏也】曜進抜草壁路松多犇隴城又抜隂密晉王保懼遷于桑城【水經註洮水自臨洮縣東北流過索西城又北出門峽又東北逕桑城東又北逕安故縣保欲自桑城奔河西也】曜還長安以劉雅為大司徒張春謀奉晉王保奔涼州張寔遣其將隂監將兵迎之聲言翼衛其實拒之 段末柸攻段匹磾破之【磾丁奚翻】匹磾謂邵續曰吾本夷狄以慕義破家君不忘久要【要一遥翻久要舊約也】請相與共擊末柸續許之遂相與追擊末柸大破之匹磾與弟文鴦攻薊【匹磾奔邵續薊為石氏所取薊音計】後趙王勒知續勢孤【是時劉石國號皆曰趙史以石趙為後趙以别之】遣中山公虎將兵圍厭次【厭於琰翻】孔萇攻續别營十一皆下之二月續自出擊虎虎伏騎斷其後【斷丁管翻】遂執續使降其城【欲使續降厭次城也降戶江翻下同】續呼兄子竺等謂曰吾志欲報國不幸至此汝等努力奉匹磾為主勿有貳心匹磾自薊還未至厭次聞續已没衆懼而散復為虎所遮【復扶又翻下同】文鴦以親兵數百力戰始得入城與續子緝兄子存竺等嬰城固守虎送續於襄國勒以為忠釋而禮之以為從事中郎因下令自今克敵獲士人毋得擅殺必生致之【勒禮續而終於殺續所以令生致士人者不過欲使之從已耳】吏部郎劉胤聞續被攻【被皮義翻】言於帝曰北方藩鎮盡矣惟餘邵續而已如使復為石虎所滅孤義士之心阻歸本之路愚謂宜發兵救之【胤續所遣也事見八十九卷愍帝建興二年】帝不能從聞續已没乃下詔以續位任授其子緝 趙將尹安宋始宋恕趙慎四軍屯洛陽叛降後趙【漢主曜改國號曰趙石勒稱趙王同在上年而勒併曜始得中原故以後趙别之】後趙將石生引兵赴之安等復叛降司州刺史李矩【復扶又翻】矩使潁川太守郭默將兵入洛石生虜宋始一軍北渡河於是河南之民皆相帥歸矩【帥讀曰率】洛陽遂空 三月裴嶷至建康【嶷魚力翻】盛稱慕容廆之威德賢雋皆為之用朝廷始重之【朝廷始以裔夷待慕容今以嶷言始重之】帝謂嶷曰卿中朝名臣【朝直遥翻】當留江東朕别詔龍驤送卿家屬嶷曰臣少蒙國恩出入省闥【嶷仕西朝歷中書侍郎給事黄門郎故云然少詩照翻】若得復奉輦轂臣之至榮但以舊京淪没山陵穿毁雖名臣宿將莫能雪恥【復扶又翻將即亮翻】獨慕容龍驤竭忠王室志除凶逆故使臣萬里歸誠今臣來而不返必謂朝廷以其僻陋而棄之孤其嚮義之心使懈體於討賊【體當依載記作怠懈居隘翻】此臣之所甚惜是以不敢狥私而忘公也【謂留江東乃是狥一身之私計歸棘城則可輔廆以討賊乃天下之公義也嶷之心蓋以廆可與共功名鄙晉之君臣宴安江沱為不足與共事而已】帝曰卿言是也乃遣使隨嶷拜廆安北將軍平州刺史【使疏吏翻】 閏月以周顗為尚書左僕射【顗魚豈翻】 晉王保將張春楊次與别將楊韜不協勸保誅之且請擊陳安保皆不從夏五月春次幽保殺之保體肥大重八百斤喜睡好讀書【喜許記翻好呼到翻】而暗弱無斷故及於難【斷丁亂翻難乃旦翻】保無子張春立宗室子瞻為世子稱大將軍保衆散犇涼州者萬餘人陳安表於趙主曜請討瞻等曜以安為大將軍擊瞻殺之張春犇枹罕【枹罕縣前漢屬金城後漢屬隴西郡張軌分屬晉興郡唐為河州枹音膚】安執楊次於保柩前斬之因以祭保安以天子禮葬保於上邽諡曰元王 羊鑒討徐龕頓兵下邳不敢前蔡豹敗龕於檀丘【檀丘在魯國卞縣東南敗補邁翻】龕求救於後趙後趙王勒遣其將王伏都救之又使張敬將兵為之後繼勒多所邀求而伏都淫暴龕患之張敬至東平龕疑其襲已乃斬伏都等三百餘人復來請降【復扶又翻降戶江翻下同】勒大怒命張敬據險以守之【據險守龕欲持久以弊之也】帝亦惡龕反覆不受其降【惡烏路翻】敕鑒豹以時進討鑒猶疑憚不進尚書令刁協劾奏鑒免死除名以蔡豹代領其兵王導以所舉失人乞自貶帝不許 六月後趙孔萇攻段匹磾【磾丁奚翻】恃勝而不設備段文鴦襲擊大破之 京兆人劉弘客居涼州天梯山【武威姑臧城南有天梯山】以妖術惑衆從受道者千餘人【妖於驕翻】西平元公張寔左右皆事之帳下閻涉牙門趙卭皆弘鄉人弘謂之曰天與我神璽應王涼州【璽斯氏翻王于况翻】涉卬信之密與寔左右十餘人謀殺寔奉弘為主寔弟茂知其謀請誅弘寔令牙門將史初收之未至涉等懷刃而入殺寔於外寢 【考異曰晉書作閻沙趙卬又云寔知其謀收劉弘殺之据晉春秋作閻涉趙卬又弘死在寔被殺後今從之】弘見史初至謂曰使君已死殺我何為初怒截其舌而囚之轘於姑臧市【轘胡悍翻車裂也涼州及武威郡皆治姑臧縣】誅其黨與數百人左司馬隂元等以寔子駿尚幼推張茂為涼州刺史西平公赦其境内以駿為撫軍將軍 丙辰趙將解虎及長水校尉尹車謀反與巴酋句徐厙彭等相結【解戶買翻酋慈由翻下同句古侯翻厙音舍皆姓也】事覺虎車皆伏誅趙主曜囚徐彭等五十餘人于阿房將殺之【阿房即秦阿房宫舊基亦謂之阿城】光禄大夫游子遠諫曰聖王用刑惟誅元惡而已不宜多殺爭之叩頭流血曜怒以為助逆而囚之盡殺徐彭等尸諸市十日乃投於水於是巴衆盡反推巴酋句渠知為主自稱大秦改元曰平趙四山氐羌巴羯應之者三十餘萬關中大亂城門晝閉子遠又從獄中上表諫爭【爭讀曰諍】曜手毁其表曰大荔奴【大荔戎種落之名子遠蓋戎出也】不憂命在須臾猶敢如此嫌死晚邪叱左右速殺之中山王雅郭汜朱紀呼延晏等諫曰子遠幽囚禍在不測猶不忘諫爭【汜音祀爭讀曰諍】忠之至也陛下縱不能用奈何殺之若子遠朝誅臣等亦當夕死以彰陛下之過天下將皆捨陛下而去陛下誰與居乎曜意解乃赦之曜敕内外戒嚴將自討渠知子遠又諫曰陛下誠能用臣策一月可定大駕不必親征也曜曰卿試言之子遠曰彼非有大志欲圖非望也【謂帝王之事非常人所望】直畏陛下威刑欲逃死耳陛下莫若廓然大赦與之更始【更工衡翻】應前日坐虎車等事其家老弱没入奚官者皆縱遣之使之自相招引聽其復業彼既得生路何為不降【降戶江翻下同】若其中自知罪重屯結不散者願假臣弱兵五千必為陛下梟之【梟不孝鳥說文日至捕梟磔之以頭掛木上故今謂掛首為梟首為于偽翻梟堅堯翻】不然今反者彌山被谷【彌滿也被皮義翻】雖以天威臨之恐非歲月可除也曜大悅即日大赦以子遠為車騎大將軍開府儀同三司都督雍秦征討諸軍事子遠屯于雍城【雍于用翻】降者十餘萬移軍安定反者皆降惟句氏宗黨五千餘家保于隂密進攻滅之遂引兵巡隴右先是氐羌十餘萬落據險不服【先悉薦翻】其酋虛除權渠自號秦王子遠進造其壁【造七到翻】權渠出兵拒之五戰皆敗權渠欲降其子伊餘大言于衆曰往者劉曜自來猶無若我何况此偏師何謂降也帥勁卒五萬晨壓子遠壘門【帥讀曰率】諸將欲擊之子遠曰伊餘勇悍當今無敵所將之兵復精于我【復扶又翻】又其父新敗怒氣方盛其鋒不可當也不如緩之使氣竭而後擊之乃堅壁不戰伊餘有驕色子遠伺其無備【伺相吏翻】夜勒兵蓐食旦值大風塵昏子遠悉衆出掩之生擒伊餘盡俘其衆權渠大懼被髪剺面請降【被皮義翻剺力之翻以刀劃面也】子遠啟曜以權渠為征西將軍西戎公【啟開也開陳其事以白于上謂之啟】分徙伊餘兄弟及其部落二十餘萬口于長安曜以子遠為大司徒録尚書事曜立太學選民之神志可教者千五百人擇儒臣以教之作酆明觀【觀古玩翻下同】及西宫起陵霄臺于滈池【司馬彪曰滈在上林苑中孟康曰長安西南有滈池古史考曰武王遷鎬長安豐亭鎬池也滈與鎬同下老翻】又于霸陵西南營夀陵侍中喬豫和苞上疏諫以為衛文公承亂亡之後節用愛民營建宫室得其時制故能興康叔之業延九百之祚【衛為狄人所滅文公徙居楚丘大布之衣大帛之冠務材訓農通商惠工始建城市而營宮室得其時制百姓悦之國家殷富衛以復興自康叔始封于衛至秦始滅延祚九百餘年】前奉詔書營酆明觀市道細民咸譏其奢曰以一觀之功足以平涼州矣【言以起一觀之功力足以平河西張氏】今又欲擬阿房而建西宫法瓊臺而起陵霄其為勞費億萬酆明若以資軍旅乃可兼吳蜀而壹齊魏矣【吳謂晉蜀謂李特齊謂曹嶷魏謂石勒】又聞營建壽陵周圍四里深三十五丈【深式禁翻】以銅為椁飾以黄金功費若此殆非國内所能辦也秦始皇下錮三泉土未乾而發毁【詳見三十一卷漢成帝永始元年劉向封事乾音干】自古無不亡之國不掘之墓故聖王之儉葬乃深遠之慮也陛下奈何於中興之日【曜平靳氏之難而自立故其臣謂之中興】而踵亡國之事乎曜下詔曰二侍中懇懇有古人之風可謂社稷之臣矣其悉罷宫室諸役夀陵制度一遵霸陵之法封豫安昌子苞平輿子【輿音豫】並領諫議大夫仍布告天下使知區區之朝欲聞其過也【朝直遥翻】又省酆水囿以與貧民【豐水出京兆南山東北流注于渭曜立囿於豐水左右】 祖逖將韓潛與後趙將桃豹分據陳川故城豹居西臺潛居東臺豹由南門潛由東門出入相守四旬逖以布囊盛土如米狀【盛時征翻】使千餘人運上臺【上時掌翻】又使數人擔米息於道豹兵逐之【擔他甘翻】棄擔而走【擔都濫翻】豹兵久饑得米以為逖士衆豐飽益懼【先以囊盛土運之潛所以疑之也又使人擔米以餌豹兵示之以實也】後趙將劉夜堂以驢千頭運糧饋豹逖使韓潛及别將馮鐵邀擊於汴水【水經注蒗渠水自中牟東流至浚儀縣分為二水南流者曰沙水東注者曰汴水汴水東流入梁郡】盡獲之豹宵遁屯東燕城【即漢東郡燕縣也後魏置東燕縣屬陳留郡隋改為胙城縣屬東郡唐屬滑州豹兵已有懼心糧又為逖所獲故宵遁也】逖使潛進屯封丘以逼之馮鐵據二臺逖鎮雍丘【封丘雍丘二縣皆屬陳留郡春秋傳敗狄于長丘在封丘界雍丘故杞國也】數遣兵邀擊後趙兵【數所角翻】後趙鎮戌歸逖者甚多境土漸蹙先是趙固上官已李矩郭默互相攻擊逖馳使和解之【先悉薦翻使疏吏翻下同】示以禍福遂皆受逖節度秋七月詔加逖鎮西將軍逖在軍與將士同甘苦約已務施【施式豉翻】勸課農桑撫納新附雖踈賤者皆結以恩禮河上諸塢先有任子在後趙者皆聽兩屬【居兩界之上者聽其兩屬因以為間】時遣游軍偽抄之【抄楚交翻】明其未附塢主皆感恩後趙有異謀輒密以告由是多所克獲自河以南多叛後趙歸于晉逖練兵積穀為取河北之計後趙王勒患之乃下幽州為逖修祖父墓置守冢二家【逖范陽人其祖父墓在焉下遐嫁翻】因與逖書求通使及互市逖不報書而聽其互市收利十倍逖牙門童建殺新蔡内史周密降于後趙【姓譜顓頊子老童之後以為氏】勒斬之送首於逖曰叛臣逃吏吾之深仇將軍之惡猶吾惡也【惡烏路翻】逖深德之自是後趙人叛歸逖者逖皆不納禁諸將不使侵暴後趙之民邊境之間稍得休息【逖聽河上諸塢兩屬此用間之智也然石勒為逖修祖父墓斬童建而送其首亦所以懈逖摧鋒越河之心】 八月辛未梁州刺史周訪卒訪善於撫士衆皆為致死【為于偽翻】知王敦有不臣之心私常切齒【切齒上下齒相磨切也】敦由是終訪之世未敢為逆敦遣從事中郎郭舒監襄陽軍【監工銜翻】帝以湘州刺史甘卓為梁州刺史督沔北諸軍事鎮襄陽【王敦憚周訪而不敢為逆至其舉兵也不以甘卓為虞亦可謂奸雄矣】舒既還帝徵為右丞敦留不遣 後趙王勒遣中山公虎帥步騎四萬擊徐龕【帥讀曰率下同】龕送妻子為質乞降勒許之【勒許龕降力未能取龕耳觀其後殺龕足以知其心質音致】蔡豹屯卞城【卞縣屬魯國劉昫曰隋於卞縣古城置泗水縣唐屬兖州】石虎將擊之豹退守下邳為徐龕所敗【敗補邁翻】虎引兵城封丘而旋徙士族三百家寘襄國崇仁里【崇仁里勒所命名以處衣冠之族】置公族大夫以領之 後趙王勒用法甚嚴諱胡尤峻【勒本胡人故以為諱】宫殿既成初有門戶之禁有醉胡乘馬突入止車門勒大怒責宫門小執法馮翥【執法御史之官也紫宫南蕃中二星曰左右執法晉之故臣為勒定宫制取此置宫門執法即以張賓為大執法總朝政故宫門置小執法翥章庶翻】翥惶懼忘諱對曰向有醉胡乘馬馳入甚呵禦之而不可與語勒笑曰胡人正自難與言恕而不罪勒使張賓領選初定五品後更定九品命公卿及州郡歲舉秀才至孝亷清賢良直言武勇之士各一人【選須絹翻石勒立國粗有綱紀石虎繼之無復有是】西平公張茂立兄子駿為世子 蔡豹既敗將詣建<br />
<br />
  康歸罪北中郎將王舒止之帝聞豹退遣使收之【使疏吏翻】舒夜以兵圍豹豹以為它寇帥麾下擊之聞有詔乃止舒執豹送建康冬十月丙辰斬之 王敦殺武陵内史向碩【史書王敦專殺以著其無君之罪】帝之始鎮江東也敦與從弟導同心翼戴帝亦推心托之敦總征討【懷帝永嘉五年帝以敦刺揚州加都督征討諸軍事其討華軼杜弢王機杜曾皆其功也從才用翻】導專機政【尚書萬機之本導録尚書事是專機政也】羣從子弟布列顯要【從才用翻】時人為之語曰王與馬共天下後敦自恃有功且宗族彊盛稍益驕恣帝畏而惡之【惡烏路翻】乃引劉隗刁協等以為腹心稍抑損王氏之權導亦漸見疎外中書郎孔愉陳導忠賢有佐命之勲宜加委任帝出愉為司徒左長史導能任真推分澹如也【分扶問翻澹杜覽翻】有識皆稱其善處興廢而敦益懷不平【史言導所以福祚流子孫敦所以隕身喪家禍及王含父子處昌呂翻】遂構嫌隙初敦辟吳興沈充為參軍充薦同郡錢鳳於敦敦以為鎧曹參軍二人皆巧諂凶狡知敦有異志隂贊成之為之畫策敦寵信之勢傾内外敦上疏為導訟屈辭語怨望導封以還敦【導録尚書先見敦疏故封還之為于偽翻下隗為同】敦復遣奏之【復扶又翻】左將軍譙王氶【氶音拯以此觀之則前作承誤也】忠厚有志行【行下孟翻】帝親信之夜召氶以敦疏示之曰王敦以頃年之功位任足矣而所求不已言至於此將若之何氶曰陛下不早裁之以至今日敦必為患劉隗為帝謀出心腹以鎮方面會敦表以宣城内史沈充代甘卓為湘州刺史帝謂氶曰王敦姦逆已著朕為惠皇其勢不遠【言當如惠帝受制於強臣也】湘州據上流之勢控三州之會【三州謂荆交廣】欲以叔父居之何如【古者同姓諸侯天子謂之伯父叔父氶宣帝之從孫而帝宣帝之曾孫於屬亦叔父也】氶曰臣奉承詔命惟力是視何敢有辭然湘州經蜀寇之餘【蜀寇謂杜弢之亂也】民物凋弊若得之部比及三年乃可即戎【用論語冉有對孔子之言即從也朱熹曰即就也戎兵也比必寐翻】苟未及此雖復灰身亦無益也【復扶又翻】十二月詔曰晉室開基方鎮之任親賢並用其以譙王氶為湘州刺史長沙鄧騫聞之歎曰湘州之禍其在斯乎氶行至武昌敦與之宴謂氶曰大王雅素佳士【雅素猶言平常也】恐非將帥才也【將即亮翻帥所類翻】氶曰公未見知耳鈆刀豈無一割之用【後漢班超之言】敦謂錢鳳曰彼不知懼而學壯語足知其不武無能為也乃聽之鎮【氶雖忠有餘而才不足敦窺見而知其無能為】時湘土荒殘公私困弊氶躬自儉約傾心綏撫甚有能名 高句麗寇遼東【句如字又音駒麗力知翻】慕容仁與戰大破之自是不敢犯仁境<br />
<br />
  四年春二月徐龕復請降【復扶又翻下同】 張茂築靈鈞臺基高九仞【高居傲翻】武陵閻曾夜叩府門【武陵疑當作武威】呼曰武公遣我來【張軌諡武公呼火故翻】言何故勞民築臺有司以為妖請殺之茂曰吾信勞民曾稱先君之命以規我何謂妖乎乃為之罷役【妖於驕翻為于偽翻下同】 三月癸亥日中有黑子【日中有黑子隂侵陽而磨蕩之也時王敦驕悖浸甚故象見于天】著作佐郎河東郭璞以帝用刑過差上疏以為隂陽錯繆皆繁刑所致赦不欲數【數所角翻】然子產知鑄刑書非政之善不得不作者須以救弊故也【左傳鄭鑄刑書叔向詒子產書曰國將亡必多制復書曰吾以救世也須待也】今之宜赦理亦如之 後趙中山公虎攻幽州刺史段匹磾於厭次【磾丁奚翻厭於琰翻】孔萇攻其統内諸城悉抜之段文鴦言於匹磾曰我以勇聞故為民所倚望今視民被掠而不救是怯也【被皮義翻下同】民失所望誰復為我致死遂帥壯士數十騎出戰【復扶又翻為于偽翻帥讀曰率】殺後趙兵甚衆馬乏伏不能起虎呼之曰兄與我俱夷狄久欲與兄同為一家今天不違願於此得相見何為復戰請釋仗文鴦罵曰汝為寇賊當死日久吾兄不用吾策【事見七十八卷懷帝永嘉六年】故令汝得至此我寧鬬死不為汝屈遂下馬苦戰槊折執刀戰不已【槊色角翻矛長丈八者曰槊折而設翻】自辰至申後趙兵四面解馬羅披自彰【馬羅披意即障泥也】前執文鴦文鴦力竭被執城内奪氣匹磾欲單騎歸朝【騎奇寄翻朝直遥翻】邵續之弟樂安内史洎勒兵不聽洎復欲執臺使王英送於虎【臺使晉朝所遣者也使疏吏翻】匹磾正色責之曰卿不能遵兄之志逼吾不得歸朝亦已甚矣復欲執天子使者我雖夷狄所未聞也洎與兄子緝竺等輿櫬出降【櫬初覲翻降戶江翻】匹磾見虎曰我受晉恩志在滅汝不幸至此不能為汝敬也後趙王勒及虎素與匹磾結為兄弟虎即起拜之勒以匹磾為冠軍將軍【冠古玩翻】文鴦為左中郎將散諸流民三萬餘戶復其本業置守宰以撫之於是幽冀并三州皆入於後趙匹磾不為勒禮常著朝服持晉節【著陟略翻】久之與文鴦邵續皆為後趙所殺【厭次既破無復後患匹磾兄弟與邵續皆被害石勒志趣從可知矣】 五月庚申詔免中州良民遭難為揚州諸郡僮客者以備征役【難乃旦翻】尚書令刁協之謀也由是衆益怨之 終南山崩【終南山長安南山也時劉曜據關中亡國之徵晉書書於曜載記】 秋七月甲戌以尚書僕射戴淵為征西將軍都督司兖豫并雍冀六州諸軍事司州刺史鎮合肥【合肥縣屬淮南郡雍於用翻】丹陽尹劉隗為鎮北將軍都督青徐幽平四州諸軍事青州刺史鎮淮隂【淮隂縣前漢屬臨淮郡後漢屬下邳郡晉屬廣陵郡】皆假節領兵名為討胡實備王敦也隗雖在外而朝廷機事進退士大夫帝皆與之密謀敦遺隗書曰【遺于季翻】頃承聖上顧眄足下今大賊未滅中原鼎沸欲與足下及周生之徒【周生謂周顗敦素憚顗見輒扇而不休故舉以為言】戮力王室共靜海内若其泰也則帝祚於是乎隆若其否也【否皮鄙翻】則天下永無望矣隗答曰魚相忘於江湖人相忘於道術【引莊子大宗師之言】竭股肱之力効之以忠貞【晉大夫荀息之言】吾之志也敦得書甚怒壬午以驃騎將軍王導為侍中司空假節録尚書領中書監【驃匹妙翻】帝以敦故并疎忌導御史中丞周嵩上疏以為導忠素竭誠輔成大業不宜聽孤臣之言惑疑似之說放逐舊德以佞伍賢【用兵列陳五人為伍伍同列也以佞伍賢言賢佞同列也】虧既往之恩招將來之患【向者親倚導而今疎忌之是虧既往之恩也導或自疑外而與敦同是招將來之患也招之遥翻】帝頗感寤導由是得全【史言周顗兄弟保護王導】 八月常山崩【常山在常山郡上曲陽縣西北其地時屬石勒】 豫州刺史祖逖以戴淵吳士【淵廣陵人廣陵故吳王濞都也】雖有才望無弘致遠識且已翦荆棘收河南地而淵雍容一旦來統之意甚怏怏【怏於兩翻】又聞王敦與劉刁構隙將有内難【難乃旦翻】知大功不遂感激發病九月壬寅卒於雍丘豫州士女若喪父母譙梁間皆為立祠【喪息浪翻為于偽翻】王敦久懷異志聞逖卒益無所憚【王敦之所忌周訪祖逖訪卒而逖繼之宜其益無所憚也然温嶠郗鑒諸人已在晉朝卒藉之以清大憝以此知上天生材以應世世變無窮而人才亦與之無窮固非姦雄所能逆睹也】冬十月壬午以逖弟約為平西將軍豫州刺史領逖之衆約無綏御之才不為士卒所附初范陽李產避亂依逖見約志趣異常謂所親曰吾以北方鼎沸故遠來就此冀全宗族今觀約所為有不可測之志吾託名姻親當早自為計無事復陷身於不義也爾曹不可以目前之利而忘長久之策乃帥子弟十餘人間行歸鄉里【李產父子後事慕容雋復扶又翻帥讀曰率間古莧翻】 十一月皇孫衍生 後趙王勒悉召武鄉耆舊詣襄國與之共坐歡飲初勒微時與李陽鄰居數爭漚麻池相歐【數所角翻漚於候翻久漬也詩云東門之池可以漚麻毛氏曰漚柔也考工記㡆氏以涗水漚其絲注云漚漸也楚人曰漚齊人曰涹涹烏禾翻然則漚是漸漬之名云漚柔者謂漸漬使之柔勒也魏收地形志鄉郡三臺領上有李陽基有麻池石石勒與李陽爭漚麻處也歐於口翻擊也】陽由是獨不敢來勒曰陽壯士也漚麻布衣之恨孤方兼容天下豈讐匹夫乎遽召與飲引陽臂曰孤往日厭卿老拳卿亦飽孤毒手因拜參軍都尉以武鄉比豐沛復之三世【勒欲並驅漢光武光武復南頓不敢遠期十歲而勒復武鄉三世多見其不知量也復方目翻】勒以民始復業資儲未豐於是重制禁釀郊祀宗廟皆用醴酒【酒一宿而熟者曰醴】行之數年無復釀者 十二月以慕容廆為都督幽平二州東夷諸軍事車騎將軍平州牧【考異曰燕書云車騎大將軍平州刺史按晉書載記先拜平州刺史尋加車騎州牧今從之】封遼東公單于如故遣謁者即授印綬聽承制置官司守宰廆於是備置僚屬以裴嶷遊邃為長史【嶷魚力翻】裴開為司馬韓壽為别駕陽耽為軍諮祭酒崔燾為主簿黄泓鄭林參軍事【鄭林不受廆車牛粟帛而躬耕于野廆蓋以是取之】廆立子皝為世子作東横【横與黌同學舍也載記作東庠皝呼廣翻】以平原劉讚為祭酒使皝與諸生同受業廆得暇亦親臨聽之【得暇者言廆惟於國事無暇財得一息之暇亦親臨東横聽其講說史言廆之能崇儒】皝雄毅多權略喜經術國人稱之【喜許記翻】廆徙慕容翰鎮遼東慕容仁鎮平郭【平郭縣漢屬遼東郡晉省唐新書曰高麗建安城古平郭縣也】翰撫安民夷甚有威惠仁亦次之拓跋猗㐌妻惟氏忌代王鬱律之彊恐不利於其子乃殺欝律而立其子賀傉【欝律立見八十九卷愍帝建興四年傉奴沃翻】大人死者數十人鬱律之子什翼犍【犍居言翻】幼在襁褓其母王氏匿於袴中祝之曰天苟存汝則勿啼久之不啼乃得免惟氏專制國政遣使聘後趙後趙人謂之女國使【以惟氏專政故謂之女國史言拓跋所以中衰使疏吏翻】<br />
<br />
  資治通鑑卷九十一<br />
<br />
<史部,編年類,資治通鑑>  <br>
   </div> 

<script src="/search/ajaxskft.js"> </script>
 <div class="clear"></div>
<br>
<br>
 <!-- a.d-->

 <!--
<div class="info_share">
</div> 
-->
 <!--info_share--></div>   <!-- end info_content-->
  </div> <!-- end l-->

<div class="r">   <!--r-->



<div class="sidebar"  style="margin-bottom:2px;">

 
<div class="sidebar_title">工具类大全</div>
<div class="sidebar_info">
<strong><a href="http://www.guoxuedashi.com/lsditu/" target="_blank">历史地图</a></strong>  
<a href="http://www.880114.com/" target="_blank">英语宝典</a>  
<a href="http://www.guoxuedashi.com/13jing/" target="_blank">十三经检索</a> 
<br><strong><a href="http://www.guoxuedashi.com/gjtsjc/" target="_blank">古今图书集成</a></strong> 
<a href="http://www.guoxuedashi.com/duilian/" target="_blank">对联大全</a> <strong><a href="http://www.guoxuedashi.com/xiangxingzi/" target="_blank">象形文字典</a></strong> 

<br><a href="http://www.guoxuedashi.com/zixing/yanbian/">字形演变</a>  <strong><a href="http://www.guoxuemi.com/hafo/" target="_blank">哈佛燕京中文善本特藏</a></strong>
<br><strong><a href="http://www.guoxuedashi.com/csfz/" target="_blank">丛书&方志检索器</a></strong> <a href="http://www.guoxuedashi.com/yqjyy/" target="_blank">一切经音义</a>  

<br><strong><a href="http://www.guoxuedashi.com/jiapu/" target="_blank">家谱族谱查询</a></strong>  <strong><a href="http://shufa.guoxuedashi.com/sfzitie/" target="_blank">书法字帖欣赏</a></strong> 
<br>

</div>
</div>


<div class="sidebar" style="margin-bottom:0px;">

<font style="font-size:22px;line-height:32px">QQ交流群9:489193090</font>


<div class="sidebar_title">手机APP 扫描或点击</div>
<div class="sidebar_info">
<table>
<tr>
	<td width=160><a href="http://m.guoxuedashi.com/app/" target="_blank"><img src="/img/gxds-sj.png" width="140"  border="0" alt="国学大师手机版"></a></td>
	<td>
<a href="http://www.guoxuedashi.com/download/" target="_blank">app软件下载专区</a><br>
<a href="http://www.guoxuedashi.com/download/gxds.php" target="_blank">《国学大师》下载</a><br>
<a href="http://www.guoxuedashi.com/download/kxzd.php" target="_blank">《汉字宝典》下载</a><br>
<a href="http://www.guoxuedashi.com/download/scqbd.php" target="_blank">《诗词曲宝典》下载</a><br>
<a href="http://www.guoxuedashi.com/SiKuQuanShu/skqs.php" target="_blank">《四库全书》下载</a><br>
</td>
</tr>
</table>

</div>
</div>


<div class="sidebar2">
<center>


</center>
</div>

<div class="sidebar"  style="margin-bottom:2px;">
<div class="sidebar_title">网站使用教程</div>
<div class="sidebar_info">
<a href="http://www.guoxuedashi.com/help/gjsearch.php" target="_blank">如何在国学大师网下载古籍?</a><br>
<a href="http://www.guoxuedashi.com/zidian/bujian/bjjc.php" target="_blank">如何使用部件查字法快速查字?</a><br>
<a href="http://www.guoxuedashi.com/search/sjc.php" target="_blank">如何在指定的书籍中全文检索?</a><br>
<a href="http://www.guoxuedashi.com/search/skjc.php" target="_blank">如何找到一句话在《四库全书》哪一页?</a><br>
</div>
</div>


<div class="sidebar">
<div class="sidebar_title">热门书籍</div>
<div class="sidebar_info">
<a href="/so.php?sokey=%E8%B5%84%E6%B2%BB%E9%80%9A%E9%89%B4&kt=1">资治通鉴</a> <a href="/24shi/"><strong>二十四史</strong></a>&nbsp; <a href="/a2694/">野史</a>&nbsp; <a href="/SiKuQuanShu/"><strong>四库全书</strong></a>&nbsp;<a href="http://www.guoxuedashi.com/SiKuQuanShu/fanti/">繁体</a>
<br><a href="/so.php?sokey=%E7%BA%A2%E6%A5%BC%E6%A2%A6&kt=1">红楼梦</a> <a href="/a/1858x/">三国演义</a> <a href="/a/1038k/">水浒传</a> <a href="/a/1046t/">西游记</a> <a href="/a/1914o/">封神演义</a>
<br>
<a href="http://www.guoxuedashi.com/so.php?sokeygx=%E4%B8%87%E6%9C%89%E6%96%87%E5%BA%93&submit=&kt=1">万有文库</a> <a href="/a/780t/">古文观止</a> <a href="/a/1024l/">文心雕龙</a> <a href="/a/1704n/">全唐诗</a> <a href="/a/1705h/">全宋词</a>
<br><a href="http://www.guoxuedashi.com/so.php?sokeygx=%E7%99%BE%E8%A1%B2%E6%9C%AC%E4%BA%8C%E5%8D%81%E5%9B%9B%E5%8F%B2&submit=&kt=1"><strong>百衲本二十四史</strong></a>  <a href="http://www.guoxuedashi.com/so.php?sokeygx=%E5%8F%A4%E4%BB%8A%E5%9B%BE%E4%B9%A6%E9%9B%86%E6%88%90&submit=&kt=1"><strong>古今图书集成</strong></a>
<br>

<a href="http://www.guoxuedashi.com/so.php?sokeygx=%E4%B8%9B%E4%B9%A6%E9%9B%86%E6%88%90&submit=&kt=1">丛书集成</a> 
<a href="http://www.guoxuedashi.com/so.php?sokeygx=%E5%9B%9B%E9%83%A8%E4%B8%9B%E5%88%8A&submit=&kt=1"><strong>四部丛刊</strong></a>  
<a href="http://www.guoxuedashi.com/so.php?sokeygx=%E8%AF%B4%E6%96%87%E8%A7%A3%E5%AD%97&submit=&kt=1">說文解字</a> <a href="http://www.guoxuedashi.com/so.php?sokeygx=%E5%85%A8%E4%B8%8A%E5%8F%A4&submit=&kt=1">三国六朝文</a>
<br><a href="http://www.guoxuedashi.com/so.php?sokeytm=%E6%97%A5%E6%9C%AC%E5%86%85%E9%98%81%E6%96%87%E5%BA%93&submit=&kt=1"><strong>日本内阁文库</strong></a> <a href="http://www.guoxuedashi.com/so.php?sokeytm=%E5%9B%BD%E5%9B%BE%E6%96%B9%E5%BF%97%E5%90%88%E9%9B%86&ka=100&submit=">国图方志合集</a> <a href="http://www.guoxuedashi.com/so.php?sokeytm=%E5%90%84%E5%9C%B0%E6%96%B9%E5%BF%97&submit=&kt=1"><strong>各地方志</strong></a>

</div>
</div>


<div class="sidebar2">
<center>

</center>
</div>
<div class="sidebar greenbar">
<div class="sidebar_title green">四库全书</div>
<div class="sidebar_info">

《四库全书》是中国古代最大的丛书,编撰于乾隆年间,由纪昀等360多位高官、学者编撰,3800多人抄写,费时十三年编成。丛书分经、史、子、集四部,故名四库。共有3500多种书,7.9万卷,3.6万册,约8亿字,基本上囊括了古代所有图书,故称“全书”。<a href="http://www.guoxuedashi.com/SiKuQuanShu/">详细>>
</a>

</div> 
</div>

</div>  <!--end r-->

</div>
<!-- 内容区END --> 

<!-- 页脚开始 -->
<div class="shh">

</div>

<div class="w1180" style="margin-top:8px;">
<center><script src="http://www.guoxuedashi.com/img/plus.php?id=3"></script></center>
</div>
<div class="w1180 foot">
<a href="/b/thanks.php">特别致谢</a> | <a href="javascript:window.external.AddFavorite(document.location.href,document.title);">收藏本站</a> | <a href="#">欢迎投稿</a> | <a href="http://www.guoxuedashi.com/forum/">意见建议</a> | <a href="http://www.guoxuemi.com/">国学迷</a> | <a href="http://www.shuowen.net/">说文网</a><script language="javascript" type="text/javascript" src="https://js.users.51.la/17753172.js"></script><br />
  Copyright &copy; 国学大师 古典图书集成 All Rights Reserved.<br>
  
  <span style="font-size:14px">免责声明:本站非营利性站点,以方便网友为主,仅供学习研究。<br>内容由热心网友提供和网上收集,不保留版权。若侵犯了您的权益,来信即刪。scp168@qq.com</span>
  <br />
ICP证:<a href="http://www.beian.miit.gov.cn/" target="_blank">鲁ICP备19060063号</a></div>
<!-- 页脚END --> 
<script src="http://www.guoxuedashi.com/img/plus.php?id=22"></script>
<script src="http://www.guoxuedashi.com/img/tongji.js"></script>

</body>
</html>
