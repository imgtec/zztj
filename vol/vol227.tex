\chapter{資治通鑑卷二百二十七}
宋 司馬光 撰

胡三省 音註

唐紀四十三|{
	起重光作噩六月盡玄黓閹茂凡一年有奇始辛酉六月終壬戌凡一年零六月}


德宗神武聖文皇帝二

建中二年六月庚寅以浙江東西觀察使蘇州刺史韓滉為潤州刺史浙江東西節度使|{
	蘇州治吴縣滉呼廣翻潤州治京口使疎吏翻}
名其軍曰鎮海 張著至襄陽|{
	是年四月遣著今至襄陽盖張著亦疑梁崇義遲遲不進也}
梁崇義益懼陳兵而見之藺杲得詔不敢|{
	得除鄧州之詔也}
馳見崇義請命|{
	請免其死}
崇義對著號泣竟不受詔|{
	號戶刀翻}
著復命癸巳進李希烈爵南平郡王|{
	渝州南平郡}
加漢南漢北兵馬招討使督諸道兵討之 |{
	考異曰德宗實録五月己巳加淮寧節度李希烈南平郡王漢南漢北通知諸道兵馬使招撫處置使希烈傳曰山南東道節度使梁崇義拒捍朝命迫脅使臣二年六月詔諸道節度率兵討之加希烈南平郡王兼漢南北都知諸道兵馬招撫處置使今從建中實録}
楊炎諫曰希烈為董秦養子親任無比卒逐秦而奪其位|{
	事見二百二十五卷代宗大歷十四年董秦賜姓名李忠臣卒子恤翻}
為人狼戾無親|{
	狼當作狠}
無功猶倔強不法|{
	倔渠勿翻強其兩翻}
使平崇義何以制之上不聽炎固爭之上益不平荆南牙門將吴少誠以取梁崇義之策干李希烈希烈以少誠為前鋒少誠幽州潞人也|{
	將即亮翻少始照翻潞縣漢屬漁陽郡晉屬燕國隋屬涿郡唐屬幽州以潞水自塞北來經縣界名縣}
時内自關中西暨蜀漢南盡江淮閩越北至太原所在出兵而李正已遣兵扼徐州甬橋渦口|{
	閩眉巾翻甬橋在徐州南界汴水上後置宿州於此渦口渦水入淮之口渦音戈}
梁崇義阻兵襄陽運路皆絶人心震恐江淮進奉船千餘艘泊渦口不敢進|{
	艘蘇遭翻}
上以和州刺史張萬福為濠州刺史|{
	使之通渦口水路}
萬福馳至渦口立馬岸上進奉船淄青將士停岸睥睨不敢動|{
	淄莊持翻睥匹詣翻睨研計翻}
辛丑汾陽忠武王郭子儀薨|{
	薨呼肱翻}
子儀為上將擁彊兵程元振魚朝恩讒毁百端詔書一紙徵之無不即日就道由是讒謗不行|{
	事並見代宗紀朝直遥翻}
嘗遣使至田承嗣所承嗣西望拜之曰此膝不屈於人若干年矣李靈曜據汴州作亂|{
	事見二百二十五卷代宗大歷十一年使疏吏翻嗣祥吏翻汴皮變翻}
公私物過汴者皆留之惟子儀物不敢近|{
	近其靳翻}
遣兵衛送出境校中書令考凡二十四月入俸錢二萬緡私產不在焉|{
	校古效翻俸扶用翻緡眉巾翻}
府庫珍貨山積家人三千人八子七壻皆為朝廷顯官|{
	郭子儀八子曜晞旴□晤曖曙映}
諸孫數十人每問安不能盡辨頷之而已僕固懷恩李懷光渾瑊皆出麾下|{
	渾戶昆翻又戶本翻瑊古咸翻}
雖貴為王公常頤指役使趨走於前家人亦以僕隸視之天下以其身為安危殆三十年|{
	殆迎也將也郭子儀奮自朔方是年肅宗至德元載也至建中二年而薨是年歲在重光作噩自柔兆涒灘至重光作噩二十六年耳故云殆三十年}
功蓋天下而主不疑位極人臣而衆不疾窮奢極欲而人不非之年八十五而終其將佐至大官為名臣者甚衆|{
	將即亮翻}
壬子以懷鄭河陽節度副使李艽為河陽懷州節度使割東畿五縣隸焉|{
	艽居包翻使疏吏翻東畿東都畿也五縣河陽河清濟源温王屋}
北庭安西自吐蕃陷河隴隔絶不通|{
	陷河隴見二百二十三卷代宗廣德元年吐從暾入聲}
伊西北庭節度使李元忠四鎮留後郭昕帥將士閉境拒守|{
	昕許斤翻帥讀曰率}
數遣使奉表皆不達聲問絶者十餘年至是遣使間道歷諸胡自囘紇中來|{
	數所角翻間古莧翻}
上嘉之秋七月戊午朔加元忠北庭大都護賜爵寧塞郡王|{
	廓州寧塞郡}
以昕為安西大都護四鎮節度使賜爵武威郡王|{
	凉州武威郡}
將士皆遷七資元忠姓名朝廷所賜也本姓曹名令忠昕子儀弟也 李希烈以久雨未進軍上怪之盧密言於上曰希烈遷延以楊炎故也|{
	因炎諫用希烈而間之}
陛下何愛炎一日之名而墮大功|{
	墮讀曰隳}
不若暫免炎相以悦之事平復用無傷也|{
	相息亮翻復扶又翻又音如字}
上以為然庚申以炎為左僕射罷政事|{
	射寅謝翻 考異曰舊傳云初炎之南來途經襄漢固勸梁崇義入朝崇義不能從己懷反側尋又使其黨李舟奉使馳說崇義因而拒命遂圖叛逆皆炎迫而成之至是德宗欲假希烈兵勢以討崇義炎又固言不可上不能平會德宗嘗訪宰相羣臣中可以大任者盧杞薦張鎰嚴郢而炎舉崔昭趙惠伯上以炎論議疎闊遂罷炎相建中實録曰炎與盧同執大政形神詭陋夙為人所褻而炎氣岸高峻罕防細故方病飲食無節或為糜餐别食閤中每登堂會食辭不能偶讒者乘之謂曰楊公鄙公不欲同食銜之舊制中書舍人分署尚書六曹以平奏報中廢其職杞議復之以疏其煩炎不可杞曰不才幸措足于斯亦當有運用以荅天造寧常拳杞之手乎因密啟中書主書有過局者有詔逐之炎怒曰中書吾局也政之不修吾自理之設不理當共議何隂訴而越官邪因不相平時淮西節度使李希烈寵任方盛上欲以之平襄陽炎以為不可上曰卿勿復言遂以希烈統之時夏潦方壯澶漫數百里故希烈軍久不得發會炎病請急累日啟免炎相以悦之上以為然使中官朱如玉就第先諭旨翌日遷左僕射謁謝之日恩旨甚渥大懼按沈既濟為炎所引故建中實録言炎罷相與德宗實録頗異今取其可信者書之然舊傳云梁崇義之反炎迫而成之亦近誣也}
以前永平節度使張鎰為中書侍郎同平章事鎰齊丘之子也|{
	使疎吏翻鎰弋質翻張齊丘玄宗時為朔方節度使}
以朔方節度使崔寧為右僕射|{
	射寅謝翻}
丙子贈故伊州刺史袁光庭工部尚書光庭天寶末為伊州刺史吐蕃陷河隴光庭堅守累年吐蕃百方誘之不下|{
	伊州治伊吾縣漢伊吾盧地尚辰羊翻吐從暾入聲誘音酉}
糧竭兵盡城且陷光庭先殺妻子然後自焚郭昕使至朝廷始知之|{
	昕許斤翻朝直遥翻}
故贈官 辛巳以邠寧節度使李懷光兼朔方節度使|{
	邠卑旻翻}
癸未河東節度使馬燧昭義節度使李抱眞神策先鋒都知兵馬使李晟大破田悦於臨洺時悦攻臨洺累月不拔城中食且盡府庫竭士卒多死傷張伾飾其愛女使出拜將士曰諸君守戰甚苦伾家無它物請鬻此女為將士一日之費衆皆哭曰願盡死力不敢言賞李抱眞告急於朝|{
	朝直遥翻}
詔馬燧將步騎二萬與抱眞討悦又遣李晟將神策兵與之俱又詔幽州留後朱滔討惟岳|{
	李惟岳也}
燧等軍未出險先遣使持書諭悦為好語悦謂燧畏之不設備燧與抱眞合兵八萬東下壺關 |{
	考異曰舊田悦傳云七月三日師自壺關東下收賊盧家砦燧傳云十一月師次邯鄲恐悮今從悦傳燕南記}
軍于邯鄲擊悦支軍破之悦方急攻臨洺分李惟岳兵五千助楊朝光明日燧等進攻朝光柵悦將萬餘人救之燧命大將李自良等禦之於雙岡|{
	雙岡在邯鄲西北臨洺之西亦名盧家疃}
令之曰悦得過必斬爾自良等力戰悦軍却燧推火車焚朝光柵|{
	推吐雷翻車尺遮翻}
斬朝光獲首虜五千餘級居五日燧等進軍至臨洺悦悉衆力戰凡百餘合悦兵大敗斬首萬餘級 |{
	考異曰舊李晟傳戰于臨洺諸軍皆却晟引兵度洺水乘水而濟横擊悦軍王師復振擊悦大破之據此則是臨洺戰在冬也與馬燧傳十一月師次邯鄲相應實録十二月庚寅馬燧加左僕射又云先是悦遣將康愔領兵圍邢州楊光朝圍臨洺燧與抱眞及神策將李晟合勢救之大敗賊於雙岡斬楊朝光禽其大將盧子昌乘勝進軍又破悦於臨洺故燧等加官按實録此戰無月日但於馬燧加官時言之今據燧傳先敗悦於雙岡斬楊朝光居五日乃進至臨洺即實録此月癸未衆軍破悦於臨洺也實録在此年冬與此相違燕南記亦云七月燧與抱眞兵八萬自潞府東下壺關先收邯鄲盧家砦朝光戰死臨洺城又大破悦悦退走在李正已死前與實録此月相應臨洺之戰疑諸軍已集燧等若未至張伾必不能獨破悦軍新本紀十一月丁丑馬燧及田悦戰于雙岡敗之不知此日何出亦與諸書相違今止從七月}
悦引兵夜遁邢州圍亦解|{
	是年五月悦使其將康愔圍邢州悦敗走而圍亦解}
時平盧節度使李正己巳薨子納祕之擅領軍務悦求救於納及李惟岳納遣大將衛俊將兵萬人惟岳遣兵三千人救之悦收合散卒得二萬餘人軍于洹水淄青軍其東成德軍其西首尾相應馬燧帥諸軍進屯鄴|{
	洹于原翻帥讀曰率洹水縣屬魏州本漢内黄地後周武帝置洹水縣因水而名淄莊持翻鄴縣屬相州}
奏求河陽兵自助詔河陽節度使李艽將兵會之|{
	艽居包翻}
八月李納始喪奏請襲父位上不許 梁崇義兵攻江陵至四望|{
	今隨州随縣之東有四望山其山最高四望皆可見}
大敗而歸乃收兵襄鄧李希烈引軍循漢而上|{
	上時掌翻}
與諸道兵會崇義遣其將翟暉杜少誠逆戰於蠻水希烈大破之追至踈口又破之|{
	將即亮翻翟萇伯翻少始照翻水經漢水自襄陽東流又屈而西南流又東南流逕黎丘故城西又南與疎水合踈水出中廬縣西南東流至邙縣北界東入漢水謂之踈口漢水又南過宜城東夷水出自房陵縣東流注之桓温以其父名彛改曰蠻水}
二將請降希烈使將其衆先入襄陽慰諭軍民|{
	將即亮翻降戶江翻使將同上音又音如字}
崇義閉城拒守守者開門爭出不可禁崇義與妻赴井死傳首京師范陽節度使朱滔將討李惟岳軍于莫州張孝忠將

精兵八千守易州|{
	范陽節度使治幽州莫州在幽州南二百八十里易州成德廵屬在幽州西二百一十四里}
滔遣判官蔡雄說張孝忠曰|{
	說式芮翻}
惟岳乳臭兒敢拒朝命今昭義河東軍已破田悦淮寧李僕射克襄陽計河南諸軍朝夕北向恒魏之亡可佇立而須也使君誠能首舉易州以歸朝廷則破惟岳之功自使君始此轉禍為福之策也|{
	朝直遥翻射寅謝翻恒戶登翻使疎吏翻}
孝忠然之遣牙官程華詣滔遣録事參軍董稹奉表詣闕|{
	稹章忍翻}
滔又上表薦之|{
	上時掌翻}
上悦九月辛酉以孝忠為成德節度使命惟岳護喪歸朝惟岳不從孝忠德滔為子茂和娶滔女|{
	為于偽翻}
深相結 壬戌加李希烈同平章事 初李希烈請討梁崇義上對朝士亟稱其忠|{
	亟去吏翻}
黜陟使李承自淮西還|{
	還從宣翻又音如字}
言於上曰希烈必立微功但恐有功之後偃蹇不臣更煩朝廷用兵耳上不以為然希烈既得襄陽遂據之為已有上乃思承言時承為河中尹甲子以承為山南東道節度使上欲以禁兵送上|{
	上時掌翻}
承請單騎赴鎮至襄陽希烈寘之外館迫脅萬方承誓死不屈希烈乃大掠闔境所有而去承治之朞年軍府稍完|{
	騎奇寄翻闔戶臘翻治直之翻}
希烈留牙將於襄州守其所掠財由是數有使者往來|{
	將即亮翻數所角翻}
承亦遣其腹心臧叔雅往來許蔡|{
	李希烈既自襄陽還蔡州尋徙鎮許州故李承隂遣人至許蔡結其諸將以圖之}
厚結希烈腹心周曾等與之隂圖希烈|{
	為周曾等圖希烈不克而死張本}
初蕭嵩家廟臨曲江玄宗以娱遊之地非神靈所宅命徙之楊炎為相惡京兆尹嚴郢|{
	相息亮翻惡烏路翻}
左遷大理卿盧欲陷炎引郢為御史大夫先是炎將營家廟|{
	先悉薦翻}
有宅在東都憑河南尹趙惠伯賣之惠伯買以為官廨郢按之以為有羨利|{
	羨于線翻}
召大理正田晉議法|{
	唐志大理正從五品下掌議獄正科條凡丞斷罪不當則以法正之}
晉以為律監臨官市買有羨利以乞取論當奪官怒|{
	監古銜翻下同}
貶晉衡州司馬|{
	衡州京師東南三千四百三里田晉自朝士貶衡州司馬}
更召它吏議法|{
	更工衡翻}
以為監主自盜罪當絞炎廟正直蕭嵩廟地因譛炎云兹地有王氣|{
	王于况翻}
故玄宗令嵩徙之炎有異志故於其地建廟冬十月乙未炎自左僕射貶崖州司馬|{
	令力丁翻射寅謝翻舊志崖州至京師七千四百六十里}
未至崖州百里縊殺之惠伯自河中尹貶費州多田尉|{
	費州漢牂柯郡隋黔安郡涪川縣地貞觀四年分思州之涪川扶陽二縣置費州多田縣武德四年務州刺史奏置以土地稍平墾田盈畛故以多田為名貞觀四年改費州為思州乾元元年復為費州京師南四千七百里至東都四千九百里因州界費水為名河中尹當作河南尹}
尋亦殺之 辛巳册太子妃蕭氏 癸卯祫太廟先是太祖既正東向之位獻懿二祖皆藏西夾室不饗至是復奉獻祖東嚮而饗之|{
	先悉薦翻復扶又翻獻祖宣皇帝熙太祖之祖也懿祖光皇帝天賜太祖之父也太祖景皇帝虎始封於唐者也唐初饗四廟宣光二帝太祖世祖也貞觀九年祔高祖於太廟朱子奢請凖禮立七廟三昭三穆各置神主太祖依晉宋已來故事虛其位待遞遷方處之東向位於始是祔弘農府君重耳及高祖為六室虛太祖之位而行禘祫至二十三年太宗祔廟遷弘農府君乃藏于西夾室文明元年高宗祔廟始遷宣皇帝于西夾室至開元十年玄宗特立九廟於是追尊宣皇帝為獻祖復列於室光皇帝為懿祖以備九室禘祫猶虛太祖之位祝文於三祖不稱臣明全廟數而已至德二載剋復後新作九室神主遂不作弘農府君神主明禘祫不及故也至寶應二年祔玄宗肅宗於廟遷獻懿二祖於西夾室始以太祖當東向位至是年將祫饗禮儀使顔眞卿奏合出獻懿二祖行事其布位次第及東向之位請凖東晉蔡謨議為定遂以獻祖當東向懿祖於昭位南向太祖於穆位北向左昭右穆陳列行事}
徐州刺史李洧正已之從父兄也李納寇宋州彭城令太原白季庚說洧舉州歸國|{
	洧于軌翻從才用翻說式芮翻}
洧從之遣攝廵官崔程奉表詣闕且使口奏并白宰相以徐州不能獨抗納乞領徐海沂三州觀察使况海沂二州今皆為納有洧與刺史王涉馬萬通素有約|{
	使疎吏翻 考異曰此據舊傳也實録萬通以密州降盖自沂移密}
苟得朝廷詔書必能成功程自外來|{
	言自外方來}
以為宰相一也先白張鎰鎰以告盧怒其不先白己不從其請|{
	相息亮翻鎰戊質翻}
戊申加洧御史大夫充招諭使十一月戊午以永樂公主適檢校比部郎中田華上不欲違先志故也|{
	永樂公主許降田華見二百二十五卷代宗大歷九年樂音洛}
蜀王傀更名遂|{
	傀苦猥翻更工衡翻}
辛酉宣武節度使劉洽神策都知兵馬使曲環滑州刺史襄平李澄朔方大將唐朝臣大破淄青魏博之兵於徐州|{
	按新書李澄傳澄遼東襄平人唐自高宗世遼東之地已弃而不有李澄時以本貫在遼東襄平耳朝直遥翻淄莊持翻}
先是李納遣其將王温會魏博將信都崇慶|{
	先悉薦翻將即亮翻信都複姓}
共攻徐州李洧遣牙官温人王智興詣闕告急|{
	温古縣唐初屬懷州顯慶二年度屬洺州}
智興善走不五日而至|{
	舊志徐州京師東二千六百四里}
上為之朔方兵五千人|{
	為于偽翻}
以朝臣將之|{
	朝直遥翻將即亮翻又音如字}
與洽環澄共救之時朔方軍資裝不至旗服弊惡宣武人嗤之曰乞子能破賊乎朝臣以其言激怒士卒且曰都統有令|{
	嗤丑之翻都統謂李勉也統他綜翻俗音從上聲}
先破賊營者營中物悉與之士皆憤怒爭奮崇慶温攻彭城二旬不能下請益兵於納納遣其將石隱金將萬人助之 |{
	考異曰實録前作隱金後作隱全今從其前}
與劉洽等相拒於七里溝日向暮洽引軍稍却朔方馬軍使楊朝晟言於唐朝臣曰公以步兵負山而陳以待兩軍我以騎兵伏於山曲賊見懸軍勢孤必之我以伏兵絶其腰必敗之|{
	使疎吏翻晟成正翻陳讀曰陣敗補邁翻騎奇寄翻}
朝臣從之崇慶等果將騎二千踰橋而西追擊官軍伏兵發横擊之崇慶等兵中斷狼狽而返阻橋以拒官軍其兵有爭橋不得涉水而度者朝晟指之曰彼可涉吾何為不涉遂涉水擊據橋者皆走崇慶等兵大潰洽等乘之斬首八千級溺死過半朔方軍盡得其輜重|{
	溺奴狄翻重直用翻}
旗服鮮華乃謂宣武人曰乞子之功孰與宋多|{
	宋指宣武兵也時以宋亳為宣武軍劉洽自宋州刺史為宣武節度使故云然}
宣武人皆慙官軍乘勝逐北至徐州城下魏博淄青軍解圍走江淮漕運始通|{
	淄莊持翻漕在到翻}
己巳詔削李惟岳官爵募所部降者赦而賞之|{
	降戶江翻}
甲申淮南節度使陳少遊遣兵擊海州其刺史王涉以州降|{
	海州李納廵屬使疏吏翻少始照翻降戶江翻}
十二月李納密州刺史馬萬通乞降丁酉以為密州刺史|{
	宋白曰密州居海得禹貢嵎夷之地春秋時為莒魯之地州理即魯之諸城也漢為高密國晉立東莞郡後魏立膠州隋改曰密州取境中密水為名}
崔漢衡至吐蕃|{
	崔漢衡使吐蕃見上卷是年三月吐從暾入聲}
贊普以勑書稱貢獻及賜全以臣禮見處|{
	處昌呂翻}
又雲州之西當以賀蘭山為境|{
	五代志靈武弘靜縣有賀蘭山弘靜縣唐改為保靜雲州當作靈州史誤也}
邀漢衡更請之丁未漢衡遣判官與吐蕃使者入奏上為之改勑書|{
	為于偽翻}
境土皆如其請|{
	關東河北方用兵不暇與吐蕃較也}
加馬燧魏博招討使

三年春正月河陽節度使李艽引兵逼衛州田悦守將任履虛詐降既而復叛|{
	衛州治汲艽居包翻將即亮翻任音壬降戶江翻復扶又翻又音如字}
馬燧等諸軍屯于漳濱田悦遣其將王光進築月城以守長橋|{
	長橋在漳水上月城兩頭抱河形如半月}
諸軍不得度燧以鐵鎖連車數百實以土囊塞其下流|{
	乘䋲證翻塞悉則翻按新書燧於長橋下流以土囊遏之}
水淺諸軍涉度時軍中乏糧悦等深壁不戰燧命諸軍持十日糧進屯倉口與悦夾洹水而軍|{
	洹于元翻洹水與漳水分流又在漳水之東}
李抱眞李艽問曰糧少而深入何也燧曰糧少則利速戰今三鎮連兵不戰|{
	三鎮謂魏博淄青成德艽居包翻少始紹翻}
欲以老我師我若分軍擊其左右悦必救之則我腹背受敵戰必不利故進軍逼悦所謂攻其所必救也|{
	兵法有是言}
彼苟出戰必為諸君破之|{
	為于偽翻}
乃為三橋逾洹水日往挑戰|{
	挑徒了翻}
悦不出燧令諸軍夜半起食濳師循洹水直趨魏州|{
	令力正翻趨逡喻翻}
令曰賊至則止為陳|{
	陳讀曰陣下結陳同}
留百騎擊鼓鳴角於營中仍抱薪持火俟諸軍畢則止鼓角匿其旁俟悦軍畢度焚其橋軍行十里所悦聞之帥淄青成德步騎四萬踰橋掩其後|{
	騎奇寄翻帥讀曰率}
乘風縱火鼓譟而進|{
	譟則竈翻}
燧按兵不動先除其前草莽百步為戰場結陳以待之募勇士五千餘人為前列悦軍至火止氣衰燧縱兵擊之悦軍大敗神策昭義河陽軍小却|{
	神策李晟軍昭義李抱眞軍河陽李艽軍}
見河東軍捷還鬭又破之|{
	還從宣翻又音如字}
追奔至三橋已焚|{
	三橋即在洹水上者}
悦軍亂赴水溺死不可勝紀斬首二萬餘級捕虜三千餘人尸相枕藉三十餘里|{
	勝音升枕職任翻藉慈夜翻 考異曰實録閏月庚戌馬燧等破田悦於洹水按舊馬燧傳洹水之戰李惟岳救兵與田悦兵猶連營相拒又燕南記惟岳見悦在圍故謀歸順然則洹水戰在惟岳死前實録誤也燕南記又曰燧與抱眞雖頻破悦聞李納助軍到乃駐軍候勢晝必取之計去悦軍三十里下營夜坐帳中使心手人濳領悦兵及小將等五千餘人立帳外燧因矯與兵馬衙官已下高語曰昨日所以頻破田悦兵馬者盖偶然之事本亦不料有此勝也看悦兵雖敗其將健皆能死戰亦天下之強敵矣今更得李納兵助其勢不小我雖頻利利則有鈍他日田悦更戰大將必須審看便宜如悦直進不可當鋒耳悦帳外兵將往往共聞燧語良久曰昨日陣上獲得田悦將健所由領過既至燧大罵曰田悦小賊菽麥未分敢肆猖狂妄動兵馬你有何所解與我相敵汝皆不自由被驅入陣又何過也今矜汝放去兵等大歡叫拜謝而去其燧前後言見悦悦召大將喜而謂曰馬燧放言懼我對人罵我此可知矣吾再戰必捷也又恃李納助軍新到乃引兵出洹水又陳燧先伏兵要處佯不勝引退悦使兵盡出逐燧燧引至伏兵處伏兵齊横截悦軍兩段與抱眞縱兵擊之大破悦軍三萬餘人今從馬燧傳}
悦收餘兵千餘人走魏州|{
	走音奏}
馬燧與李抱眞不協頓兵平邑浮圖|{
	據舊書田悦傳平邑浮圖在魏州南浮圖佛寺也}
悦夜至南郭|{
	魏州南郭也}
大將李長春閉關不内以俟官軍久之天且明長春乃開門内之悦殺長春嬰城拒守城中士卒不滿數千死者親戚號哭滿街|{
	將即亮翻號戶刀翻}
悦憂懼乃持佩刀乘馬立府門外悉集軍民流涕言曰悦不肖蒙淄青成德二丈人保薦嗣守伯父業|{
	淄青李正已成德李寶臣田悦以丈人行事之伯父田承嗣也淄莊持翻嗣祥吏翻}
今二丈人即世其子不得承襲悦不敢忘二丈人大恩不量其力|{
	量音良}
輒拒朝命喪敗至此|{
	朝直遥翻喪息浪翻}
使士大夫肝腦塗地皆悦之罪也悦有老母不能自殺願諸公以此刀斷悦首持出城降馬僕射|{
	斷音短下各斷同降戶江 翻馬僕射謂馬燧射寅謝翻}
自取富貴無為與悦俱死也因從馬上自投地將士爭前抱持悦曰尚書舉兵徇義非私己也一勝一負兵家之常某輩累世受恩何忍聞此願奉尚書一戰不勝則以死繼之|{
	尚辰羊翻}
悦曰諸公不以悦喪敗而弃之悦雖死敢忘厚恩於地下乃與諸將各斷髮約為兄弟誓同生死悉出府庫所有及歛富民之財得百餘萬以賞士卒衆心始定|{
	田悦善敗不亡所謂盗亦有道}
復召貝州刺史邢曹俊使之整部伍繕守備軍勢復振|{
	悦不用邢曹俊見上卷上年復扶又翻又音如字}
李納軍於濮陽為河南軍所逼奔還濮州 |{
	考異曰時濮州治鄄城别有濮陽縣按九域志濮陽縣東至濮州九十里濮博木翻}
徵援兵於魏州田悦遣軍使符璘將三百騎送之|{
	使疏吏翻璘離珍翻將即亮翻騎奇寄翻}
璘父令奇謂璘曰吾老矣歷觀安史輩叛亂者今皆安在田氏能久乎汝因此弃逆從順是汝揚父名於後世也齧臂而别璘遂與其副李瑶帥衆降於馬燧|{
	齧魚結翻帥讀曰率降戶江翻}
悦收族其家令奇慢罵而死瑶父再春以博州降悦從兄昂以洺州降|{
	從才用翻洺音名}
王光進以長橋降悦入城旬餘日馬燧等諸軍始至城下攻之不克 丙寅李惟岳遣兵與孟祐守束鹿|{
	束鹿本鹿城縣安禄山反玄宗改縣為束鹿以厭之屬深州九域志在州西四十五里宋白曰束鹿縣本漢西梁縣地今縣南六十里有西梁故城尚存}
朱滔張孝忠攻拔之進圍深州惟岳憂懼掌書記邵眞復說惟岳密為表先遣弟惟簡入朝|{
	復扶又翻說式芮翻朝直遥翻}
然後誅諸將之不從命者身自入朝使妻父冀州刺史鄭詵權知節度事以待朝命惟簡既行孟祐知其謀密遣告田悦悦大怒使衙官扈岌往見惟岳讓之曰尚書舉兵正為大夫求旌節耳|{
	事始見上卷二年岌魚及翻尚辰羊翻為于偽翻下非為相為同}
非為己也今大夫乃信邵眞之言遣弟奉表悉以反逆之罪歸尚書自求雪身尚書何負於大夫而至此邪|{
	尚辰羊翻邪音耶}
若相為斬邵眞則相待如初不然當與大夫絶矣判官畢華言於惟岳曰田尚書以大夫之故陷身重圍|{
	為于季翻重直龍翻}
大夫一旦負之不義甚矣且魏博淄青兵彊食富足抗天下事未可知奈何遽為二三之計乎惟岳素怯不能守前計乃引邵眞對扈岌斬之|{
	淄莊持翻岌魚及翻}
成德兵萬人與孟祐俱圍束鹿丙寅朱滔張孝忠與戰於束鹿城下惟岳大敗燒營而遁 |{
	考異曰實録及舊惟岳傳止言惟岳一敗按滔傳曰滔與孝忠征之大破惟岳於束鹿滔命偏師守束鹿進圍深州惟岳乃統萬餘衆及田悦援兵圍束鹿惟岳將王武俊以騎三千方陳横進滔繢帛為狻猊象使猛士百人蒙之鼓譟奮馳賊馬驚亂隨擊大破之惟岳焚營而遁據此則是惟岳再敗也燕南記孟祐先敗惟岳又敗與滔傳相應今從之}
兵馬使王武俊為左右所構惟岳疑之惜其才未忍除也束鹿之戰使武俊為前鋒私自謀曰我破朱滔則惟岳軍勢大振歸殺我必矣故戰不甚力而敗朱滔欲乘勝攻恒州|{
	使疏吏翻恒戶登翻}
張孝忠引軍西北軍于義豐|{
	義豐縣屬定州}
滔大驚孝忠將佐皆怪之孝忠曰恒州宿將尚多未易可輕|{
	將即亮翻易以豉翻}
迫之則并力死鬭緩之則自相圖諸君第觀之吾軍義豐坐待惟岳之殄滅耳且朱司徒言大而識淺可與共始難與共終也|{
	朱滔後卒如張孝忠所料}
於是滔亦屯束鹿不敢進惟岳將康日知以趙州歸國惟岳益疑王武俊武俊甚懼或謂惟岳曰先相公委腹心於武俊|{
	先相公謂李寶臣相息亮翻}
使之輔佐大夫又有骨肉之親|{
	謂武俊子士真壻於李氏}
武俊勇冠三軍今危難之際復加猜阻若無武俊欲使誰為大夫却敵乎|{
	冠古玩翻難乃旦翻復扶又翻為于偽翻}
惟岳以為然乃使步軍使衛常寧與武俊共擊趙州又使王士眞將兵宿府中以自衛|{
	將即亮翻又音如字}
癸未蜀王遂更名遡|{
	更工衡翻}
淮南節度使陳少遊拔海密二州李納復攻陷之|{
	使疏吏翻少始照翻復扶又翻又音如字下不復同}
王武俊既出恒州謂衛常寧曰武俊今幸出虎口不復歸矣當北歸張尚書|{
	恒戶登翻尚辰羊翻張尚書謂張孝忠也}
常寧曰大夫暗弱信任左右觀其勢終為朱滔所滅今天子有詔得大夫首者以其官爵與之中丞素為衆所服與其出亡曷若倒戈以取大夫轉禍為福特反掌耳事苟不捷歸張尚書未晚也武俊深以為然會惟岳使要籍謝遵至趙州城下|{
	要籍官亦唐時節度衙前之職中宗景雲二年解琬為朔方大總管分遣隨軍要籍官河陽丞張冠宗肥鄉令韋景駿普安令于處忠校料三城兵募則唐邊鎭有要籍官尚矣又據新書忠義傳朱泚統幽州行營為涇原鳳翔節度使詔蔡廷玉以大理少卿為司馬朱體微為要籍則要籍乃節度使之腹心也朱滔王武俊之相王改要籍官曰丞令}
武俊引遵同謀取惟岳遵還密告王士眞閏月甲辰武俊常寧自趙州引兵還襲惟岳|{
	還從宣翻又音如字}
遵與士眞矯惟岳命啟城門納之黎明武俊帥數百騎突入府門|{
	帥讀曰率騎奇寄翻}
士眞應之於内殺十餘人武俊令曰大夫叛逆將士歸順敢違拒者族衆莫敢動遂執惟岳收鄭詵畢華王它奴等皆殺之|{
	令力定翻將即亮翻詵踈臻翻}
武俊以惟岳舊使之子|{
	李寶臣已死故曰舊使使疏吏翻}
欲生送之長安常寧曰彼見天子將復以叛逆之罪歸咎於中丞|{
	復扶又翻下復榷同又音如字中丞謂王武俊}
乃縊殺之|{
	縊於賜翻又於計翻}
傳首京師|{
	卒如谷從政之言代宗廣德元年李寶臣帥成德凡二世十九年而滅}
深州刺史楊榮國惟岳姊夫也降於朱滔滔使復其位|{
	姊蔣兕翻降戶江翻}
復榷天下酒惟西京不榷|{
	罷榷酒見二百二十五卷大歷十四年七月榷古岳翻}
二月戊午李惟岳所署定州刺史楊政義降時河北略定惟魏州未下河南諸軍攻李納於濮州|{
	濮博木翻濮州治鄄城縣}
納勢日蹙朝廷謂天下不日可平甲子以張孝忠為易定滄三州節度使|{
	朝直遥翻使踈吏翻}
王武俊為恒冀都團練觀察使康日知為深趙都團練觀察使以德棣二州隸朱滔令還鎭滔固請深州不許由是怨望留屯深州|{
	朱滔討李惟岳再戰再勝及瓜分成德廵屬以賞降將尺寸之地滔不預焉又欲使之取德棣此左氏所以知桓王之失鄭也}
王武俊素輕張孝忠自以手誅李惟岳功在康日知上而孝忠為節度使已與康日知俱為都團練使又失趙定二州亦不悦又詔以糧三千石給朱滔馬五百匹給馬燧武俊以為朝廷不欲使故人為節度使|{
	王武俊恒州舊將故云然}
魏博既下必取恒冀故分其糧馬以弱之疑未肯奉詔田悦聞之遣判官王侑許士則間道至深州說朱滔曰司徒奉詔討李惟岳旬朔之間拔束鹿下深州惟岳勢䠞|{
	間古莧翻說式芮翻下說王人說同䠞與蹙同}
故王大夫因司徒勝勢得以梟惟岳之首此皆司徒之功也又天子明下詔書|{
	梟堅堯翻下遐嫁翻}
令司徒得惟岳城邑皆隸本鎮今乃割深州以與日知是自弃其信也且今上志欲掃清河朔不使藩鎮承襲將悉以文臣代武臣魏亡則燕趙為之次矣|{
	令力丁翻燕因䖍翻}
若魏存則燕趙無患然則司徒果有意矜魏博之危而救之非徒得存亡繼絶之義亦子孫萬世之利也|{
	同舟遇風則胡越可使相救是以善用兵者必先離其交}
又許以貝州賂滔|{
	貝州魏博廵屬}
滔素有異志聞之大喜即遣王侑歸報魏州使將士知有外援各自堅又遣判官王郅|{
	將即亮翻 考異曰舊傳王郅作王郢今從燕南記}
與許士則俱詣恒州說王武俊曰大夫出萬死之計誅逆首拔亂根|{
	謂誅李惟岳也}
康日知不出趙州豈得與大夫同日論功而朝廷褒賞略同誰不為大夫憤邑者|{
	朝直遥翻為于偽翻}
今又聞有詔支糧馬與鄰道朝廷之意盖以大夫善戰恐為後患先欲貧弱軍府俟平魏之日使馬僕射北首|{
	射寅謝翻馬僕射謂馬燧時攻魏州首式又翻}
朱司徒南向共相滅耳朱司徒亦不敢自保使郅等效愚計欲與大夫共救田尚書而存之|{
	尚辰羊翻田悦拒命宜削官而當時猶稱其朝銜可以見朝命之重}
大夫自留糧馬以供軍朱司徒不欲以深州與康日知願以與大夫請早定刺史以守之三鎮連兵|{
	此三鎮謂范陽恒冀魏博}
若耳目手足之相救則它日永無患矣武俊亦喜許諾|{
	利害同故說之易入}
即遣判官王巨源使於滔|{
	使疏吏翻}
且令知深州事|{
	令力丁翻}
相與刻日舉兵南向滔又遣人說張孝忠孝忠不從|{
	說武芮翻}
宣武節度使劉洽攻李納於濮州克其外城納於城上涕泣求自新李勉又遣人說之癸卯納遣其判官房說以其母弟經及子成務入見|{
	說式芮翻見賢遍翻房說讀曰悦通鑑本文作癸卯然自上文二月戊午推至下文三月乙未其間不容有癸卯當作己卯}
會中使宋鳳朝稱納勢窮蹙不可捨上乃囚說等於禁中納遂歸鄆州復與田悦等合|{
	使疏吏翻朝直遥翻鄆音運復扶又翻又音如字}
朝廷以納勢未衰三月乙未始以徐州刺史李洧兼徐海沂都團練觀察使海沂已為納所據洧竟無所得|{
	洧于軌翻史言帝鋭意削平藩鎮而不能應機撫接以自遺患}
李納之初反也其所署德州刺史李西華備守甚嚴都虞候李士眞密毁西華於納納召西華還府以士眞代之士眞又以詐召棣州刺史李長卿長卿過德州士眞劫之與同歸國夏四月戊午以士眞長卿為二州刺史|{
	德州治安德縣棣州治厭次縣本皆淄青廵屬今皆歸國棣大計翻長知丈翻 考異曰燕南記云授士眞德棣兩州觀察團練使今從實録}
士眞求援於朱滔滔已有異志遣大將李濟時將三千人聲言助士眞守德州且召士眞詣深州議軍事至則留之使濟時領州事|{
	將即亮翻時將音同上又音如字德宗以德棣與朱滔滔卒以詐力得之不知又以為王武俊之資也}
庚申吐蕃歸曏日所俘掠兵民八百人|{
	自吐蕃陷河隴入京師俘掠唐人可以數計邪德宗先歸所俘者以懷之其歸向日所俘者八百人而已狼子野心姑以此報塞中國其志果如何哉觀異日平凉劫盟之事可見也吐從暾入聲}
上遣中使盧龍恒冀易定兵萬人|{
	盧龍朱滔恒冀王武俊易定張孝忠使疎吏翻恒戶登翻}
詣魏州討田悦王武俊不受詔執使者送朱滔滔言於衆曰將士有功者吾奏求官勲皆不遂|{
	將即亮翻唐制官有品勲有級}
今欲與君敕裝|{
	敕與飭同飭治也}
共趨魏州|{
	趨逡諭翻}
擊破馬燧以取温飽何如皆不應三問乃曰幽州之人自安史之反從而南者無一人得還今其遺人痛入骨髓|{
	還從宣翻又音如字髓息委翻}
况太尉司徒皆受國寵榮|{
	太尉謂滔兄泚}
將士亦各蒙官勲誠且願保目前不敢復有僥冀|{
	復扶又翻下俊復同}
滔默然而罷乃誅大將數十人厚撫循其士卒康日知聞其謀以告馬燧燧以聞上以魏州未下王武俊復叛力未能制滔壬戌賜滔爵通義郡王冀以安之|{
	眉州通義郡}
滔反謀益甚分兵營於趙州以逼康日知|{
	將即亮翻又音如字趙州治平棘縣}
以深州授王巨源|{
	朱滔如前約以結王武俊}
武俊以其子士眞為恒冀深三州留後將兵圍趙州|{
	恒戶登翻將即亮翻又音如字}
涿州刺史劉怦|{
	代宗大歷四年朱希彩表分幽州之范陽歸義固安置涿州治范陽縣距幽州一百二十里涿竹角翻怦普耕翻}
聞滔欲救田悦以書諫之曰今昌平故里朝廷改為太尉鄉司徒里此亦丈夫不朽之名也|{
	朝直遥翻朱泚朱滔本昌平人朝廷以其官名其鄉里以寵其兄弟之功}
但以忠順自持則事無不濟竊思近日務大樂戰|{
	樂音洛}
不顧成敗而家滅身屠者安史是也怦忝密親默而無告是負重知惟司徒圖之無貽後悔滔雖不用其言亦嘉其盡忠卒無疑貳滔將起兵恐張孝忠為後患復遣牙官蔡雄往說之孝忠曰昔者司徒幽州遣人語孝忠曰|{
	卒子恤翻復扶又翻說式芮翻下解說同語牛倨翻}
李惟岳負恩為逆謂孝忠歸國即為忠臣孝忠性直用司徒之教今既為忠臣矣不復助逆也|{
	復扶又翻又音如字}
且孝忠與武俊皆出夷落|{
	張孝忠本奚乞失活種王武俊出契丹怒皆部}
深知其心最喜飜覆|{
	喜許既翻}
司徒勿忘鄙言它日必相念矣|{
	其後滔武俊交惡果如孝忠之言}
雄復欲以巧辭說之|{
	復扶又翻下復以同}
孝忠怒欲執送京師雄懼逃歸滔乃使劉怦將兵屯要害以備之|{
	怦普耕翻將即亮翻又音如字下文滔將同}
孝忠完城礪兵獨居彊寇之間莫之能屈滔將步騎二萬五千深州至束鹿詰旦將行|{
	騎奇寄翻詰起吉翻}
吹角未畢士卒忽大亂喧譟曰天子令司徒歸幽州奈何違勑南救田悦|{
	譟則竈翻令力丁翻下旨令同}
滔大懼走入驛後堂避匿蔡雄與兵馬使宗頊等矯謂士卒曰汝輩勿喧聽司徒傳令衆稍止|{
	使疏吏翻頊吁玉翻令力定翻}
雄又曰司徒將范陽恩旨令得李惟岳州縣即有之司徒以幽州少絲纊故與汝曹竭力血戰以取深州冀得其絲纊以寛汝曹賦率|{
	少詩沼翻纊苦謗翻細綿也賦率猶言賦歛也}
不意國家無信復以深州與康日知又朝廷以汝曹有功賜絹人十匹至魏州西境盡為馬僕射所奪|{
	復扶又翻又音如字朝直遥翻下各還同射寅謝翻}
司徒但處范陽富貴足矣今兹南行乃為汝曹非自為也|{
	處昌呂翻為于偽翻下行為不為同}
汝曹不欲南行任自歸北何用喧悖|{
	悖蒲昩翻又蒲没翻}
乖失軍禮衆聞言不知所為乃曰敕使何得不為軍士守護賞物遂入敕使院擘裂殺之|{
	軍中别置館舍以居敕使謂之敕使院使疏吏翻}
又呼曰雖知司徒此行為士卒終不如且奉詔歸鎮|{
	呼火故翻}
雄曰然則汝曹各還部伍詰朝復往深州休息數日相與歸鎮耳衆然後定滔即引軍還深州密令諸將訪察唱率為亂者得二百餘人悉斬之|{
	詰去吉翻復音如字令力丁翻將即亮翻}
餘衆股栗乃復引軍而南衆莫敢前却|{
	呼火故翻復扶又翻又音如字觀田庭玠之諫田悦谷從政邵眞之諫李惟岳范陽之兵不肯從朱滔南救魏州河朔三鎮之人豈皆好亂哉上之人御失其道耳}
進取寧晉|{
	寧晉縣屬趙州本癭陶縣天寶元年更名也九域志云在趙州南四十一里}
留屯以待王武俊武俊將步騎萬五千取元氏|{
	元氏縣漢為常山郡治後魏屬趙郡唐屬趙州將即亮翻又音如字}
東趣寧晉|{
	趣七喻翻宋白曰寧晉漢楊氏縣也後漢為癭陶侯國後魏為癭陶縣唐天寶元年改寧晉縣九域志寧晉縣在趙州東南四十一里}
武俊之始誅李惟岳也遣判官孟華入見|{
	見賢遍翻}
華性忠直有才略應對慷慨上悦以為恒冀團練副使|{
	慷苦廣翻恒戶登翻}
會武俊與朱滔有異謀上遽遣華歸諭旨華至武俊已出師華諫曰聖意於大夫甚厚苟盡忠義何患官爵之不崇土地之不廣不日天子必移康中丞於它鎮|{
	康中丞謂康日知}
深趙終為大夫之有何苦遽自同於逆亂乎異日無成悔之何及華曏在李寶臣幕府以直道已為同列所忌至是為副使同列尤疾之言於武俊曰華以軍中隂事奏天子請為内應故得超遷是將覆大夫之車大夫宜備之武俊以其舊人不忍殺奪職使歸私第田悦恃援兵將至遣其將康愔將萬餘人出城西與馬燧等戰於御河上|{
	將即亮翻愔於今翻愔將音同上又音如字燧音遂御河即隋煬帝所開永濟渠也開元二十八年魏州刺史盧暉徙永濟渠自石灰窰引流至城西注魏橋以通江淮之貨杜佑曰御河在魏州魏縣煬帝引白溝水為永濟渠即此}
大敗而還|{
	還從宣翻又音如字 考異曰悦傳曰悦以救軍將至盡率其衆出戰于御河之上大敗而還燧傳曰悦恃燕趙之援又出兵二萬背城而陳燧復與諸軍擊破之今從實録}
時兩河用兵月費百餘萬緍|{
	緍眉巾翻}
府庫不支數月太常博士韋都賓陳京建議以為貨利所聚皆在富商請括富商錢出萬緍者借其餘以供軍計天下不過借一二千商則數年之用足矣上從之甲子詔借商人錢令度支條上|{
	度徒洛翻上時掌翻}
判度支杜佑大索長安中商賈所有貨意其不實輒加搒捶人不勝苦有縊死者|{
	索山客翻賈音古搒音彭捶止橤翻勝音升縊於賜翻又於計翻}
長安囂然如被寇盗|{
	囂五羔翻又許驕翻被皮義翻}
計所得纔八十餘萬緍又括僦櫃質錢|{
	民間以物質錢異時贖出於母錢之外復還子錢謂之僦櫃僦即就翻}
凡蓄積錢帛粟麥者皆借四分之一封其櫃窖|{
	蓄錢帛者以櫃積粟麥者以窖窖古教翻}
百姓為之罷市|{
	為于偽翻}
相帥遮宰相馬自訴以千萬數|{
	帥讀曰率}
盧杞始慰諭之勢不可遏乃疾驅自他道歸計并借商所得纔二百萬緍 |{
	考異曰實録借商統計田宅奴婢等估纔餘八萬貫今從舊盧傳傳又曰杜佑計京師帑廩不支數月且得五百萬貫可支半歲用則兵濟矣於是戶部侍郎判度支趙贊與韋都賓等謀行括借約罷兵後以公錢還勑既下京兆少尹韋貞督責頗峻長安尉薛萃荷校乘車搜人財貨計富戶田宅奴婢等估纔及八十八萬貫又借僦櫃質錢共纔及二百萬貫今從實録}
人已竭矣京叔明之五世孫也|{
	陳叔明陳宣帝子封宜都王}
甲戌以昭義節度副使磁州刺史盧玄卿為洺州刺史兼魏博招討副使|{
	使疏吏翻磁墻之翻洺音名}
初李抱眞為澤潞節度使馬燧領河陽三城抱眞欲殺懷州刺史楊鉥鉥奔燧|{
	鉥時迄翻}
燧納之且奏其無罪抱眞怒及同討田悦數以事相恨望二人怨隙遂深不復相見由是諸軍逗撓久無成功|{
	數所角翻下上數同復扶又翻逗者逗留不進勢屈為撓撓奴教翻}
上數遣中使和解之|{
	數所角翻使疏吏翻}
及王武俊逼趙州抱眞分麾下二千人戍邢州燧大怒曰餘賊未除宜相與勠力乃分兵自守其地欲引兵歸李晟說燧曰李尚書以邢趙連壤|{
	九域志趙州南至邢州界七十四里自界首至邢州七十里晟成正翻說式芮翻尚辰羊翻}
分兵守之誠未有害今公遽自引去衆謂公何燧悦乃單騎造抱眞壘|{
	騎奇寄翻造七到翻}
相與釋憾結歡會洺州刺史田昂請入朝燧奏以洺州隸抱眞|{
	洺州自此遂屬昭義洺音名朝直遥翻}
請玄卿為刺史兼充招討之副李晟軍先隸抱眞又請兼隸燧以示協和上皆從之|{
	晟成正翻燧音遂}
盧龍節度行軍司馬蔡廷玉惡判官鄭雲逵奏貶莫州參軍雲逵妻朱滔之女也滔復奏為掌書記|{
	惡烏路翻復扶又翻又音如字}
雲逵深搆廷玉於滔廷玉又與檢校大理少卿朱體微言於泚曰|{
	蔡廷玉朱體微皆事朱泚者也校古孝翻少始照翻}
滔在幽鎮事多專擅其性非長者不可以兵權付之滔知之大怒數與泚書請殺二人者|{
	長知丈翻數所角翻}
泚不從由是兄弟頗有隙及滔拒命上欲歸罪於廷玉等以悦滔甲子貶廷玉柳州司戶體微萬州南浦尉|{
	柳州漢潭州縣地唐置柳州以分野當柳星之下而名去京師水陸相乘五千四百七十里萬州治南浦縣春秋夔國之地秦漢為朐䏰縣地後周置萬川郡唐置萬州以郡為稱京師西南一千六百二十四里}
宣武節度使劉洽攻李納之濮陽降其守將高彦昭|{
	使疏吏翻濮博木翻降戶江翻將即亮翻}
朱滔遣人以蠟書置髻中遺朱泚|{
	遺唯季翻}
欲與同反馬燧獲之并使者送長安泚不之知上驛召泚於鳳翔至以蠟書并使者示之泚惶恐頓首請罪上曰相去千里初不同謀非卿之罪也因留之長安私第|{
	大歷九年朱泚請入朝代宗為之築大第於京師事見二百二十五卷 考異曰幸奉天録云上命還私第但絶朝謁日給酒肉而已以内侍一人監之今從實録及舊傳}
賜名園腴田錦綵金銀甚厚以安其意其幽州盧龍節度太尉中書令並如故|{
	為朱泚失兵權乘時逆上張本}
上以幽州兵在鳳翔|{
	幽州兵朱泚所將以入朝防秋者}
思得重臣代之盧忌張鎰忠直為上所重欲出之於外已得專總朝政|{
	鎰弋質翻朝直遥翻}
乃對曰朱泚名位素崇鳳翔將校班秩已高非宰相信臣無以鎮撫臣請自行上俛首未言|{
	泚且禮翻又音此將即亮翻校戶教翻相息亮翻俛音免}
杞又曰陛下必以臣貌寢不為三軍所伏|{
	貌不揚曰寢}
固惟陛下神算上乃顧鎰曰才兼文武望重内外無以易卿鎰知為所排而無辭以免因再拜受命戊寅以鎰兼鳳翔尹隴右節度等使|{
	為張鎰為李楚琳所殺張本}
初盧與御史大夫嚴郢共搆楊炎趙惠伯之獄|{
	事見上二年}
炎死復忌郢|{
	復扶又翻}
會蔡廷玉等貶官殿中侍御史鄭詹誤遞文符至昭應送之廷玉等行已至藍田召還而東|{
	還從宣翻又音如字}
廷玉等以為執已送朱滔至靈寶西赴河死|{
	靈寶縣屬陜州古桃林地漢為弘農縣開元末改為靈寶縣弘農縣故城在今縣西南二十里}
上聞之駭異盧杞因奏朱泚必疑以為詔旨請遣三司使案詹|{
	此謂遣兩省官及御史臺官為三司使使案詹等獄使踈吏翻}
又言御史所為必禀大夫請并郢案之獄未具壬午杞奏杖殺詹於京兆府貶郢費州刺史|{
	費州治涪州江岸因州界費水為名舊志費州京師南四千七百里費扶未翻 考異曰舊盧傳云貶郢驩州刺史今從舊傳}
卒於貶所|{
	卒子恤翻}
上初即位崔祐甫為相務崇寛大故當時政聲藹然以為有貞觀之風|{
	相息亮翻觀古玩翻}
及盧為相知上性多忌因以疑似離間羣臣|{
	間古莧翻}
始勸上以嚴刻御下中外失望 淮南節度使陳少遊奏本道税錢每千請增二百|{
	使疏吏翻少始照翻舊志淮南道督楊滁常潤和宣歙七州此貞觀中之制也以今觀之唐中世以後當統楊楚滁和濠廬夀光蘄黄申安舒等州税錢謂田税及商税錢也}
五月丙戌詔增它道税錢皆如淮南又鹽每斗價皆增百錢|{
	鹽每斗價幾何而頓增百錢人誰堪之}
朱滔王武俊自寧晉南救魏州|{
	是年四月滔武俊進屯寧晉}
辛卯詔朔方節度使李懷光將朔方及神策步騎萬五千人東討田悦且拒滔等|{
	將即亮翻又音如字騎奇寄翻}
滔行至宗城掌書記鄭雲逵參謀田景仙弃滔來降|{
	宗城縣屬魏州漢廣宗縣地降戶江翻}
丁酉加河東節度使馬燧同平章事 辛亥置義武軍節度於定州以易定滄三州隸之|{
	以命張孝忠}
張光晟之殺突董也|{
	事見上卷元年晟成正翻}
上欲遂絶回紇召册可汗使源休還太原久之乃復遣休送突董及翳密施大小梅録等四喪還其國|{
	紇下没翻可從刋入聲汗音寒還從宣翻又音如字下同復扶又翻}
可汗遣其宰相頡子斯迦等迎之頡子斯迦坐大帳|{
	新書回鶻傳作頡于迦相息亮翻頡奚結翻迦求加翻}
立休等於帳前雪中詰以殺突董之狀欲殺者數四供待甚薄留五十餘日乃得歸可汗使人謂之曰國人皆欲殺汝以償怨我意則不然汝國已殺突董等我又殺汝如以血洗血汚益甚耳|{
	詰去吉翻汚烏故翻}
今吾以水洗血不亦善乎唐負我馬直百八十萬匹當速歸之遣其散支將軍康赤心隨休入見|{
	見賢遍翻}
休竟不得見可汗而還|{
	還從宣翻又音如字}
己卯至長安詔以帛十萬匹金銀十萬兩償其馬直休有口辯盧恐其見上得幸乘其未至先除光禄卿|{
	為源休以賞薄怨望從朱泚反張本}
朱滔王武俊軍至魏州田悦具牛酒出迎魏人懽呼

動地|{
	呼火故翻}
滔營於愜山是日李懷光軍亦至馬燧等盛軍容迎之滔以為襲己遽出陳懷光勇而無謀欲乘其營壘未就擊之燧請且休將士觀舋而動懷光曰彼營壘既立將為後患此時不可失也遂擊滔於愜山之西|{
	魏氏土地記曰勃海高城縣東北五十里有篋山余按此愜山當在魏州界近永濟渠陳讀曰陣燧音遂將即亮翻舋許覲翻愜詰叶翻}
殺步卒千餘人滔軍崩沮|{
	沮在呂翻}
懷光按轡觀之有喜色士卒爭入滔營取寶貨王武俊引二千騎横衝懷光軍軍分為二滔引兵繼之官軍大敗蹙入永濟渠溺死者不可勝數人相蹈藉其積如山水為之不流|{
	騎奇寄翻溺奴狄翻勝音升為于偽翻}
馬燧等各收軍保壘是夕滔等堰永濟渠入王莽故河|{
	酈道元曰漢溝洫志云河為中國害尤甚故禹導河自積石歷龍門釃二渠以引河一則漯川今河所流也一則北瀆王莽時絶故世俗名是瀆為王莽河}
絶官軍糧道及歸路明日水深三尺餘|{
	深式禁翻}
馬燧懼遣使卑辭謝滔求與諸節度歸本道奏天子請以河北事委五郎處之|{
	使疏吏翻朱滔第五故稱之為五郎若尊之然處昌呂翻}
滔欲許之王武俊以為不可滔不從秋七月燧與諸軍涉水而西退保魏縣以拒滔|{
	九域志魏縣在魏州城西三十五里}
滔乃謝武俊武俊由是恨滔後數日滔等亦引兵營魏縣東南與官軍隔水相拒|{
	考異曰實録六月辛巳朱滔王武俊兵至魏州是日李懷光之師亦至七月庚子馬燧等四節度兵退保魏縣又曰田悦等築堰欲决御河水灌王莽故河以絶我糧道燧令白懷光欲退軍懷光不可抱眞晟亦欲决死守之賊築堰愈急勢迫會夜乃俱引退燕南記曰六月朱滔武俊懷光俱至懷光即欲戰馬燧抱眞不得已從之七月六日懷光等擊滔勝之尋為王武俊所敗其夜决河水絶懷光等西歸之路明日水深三尺餘馬燧與朱滔有外族之親呼滔為表侄使人說滔曰老夫不度氣力與李相公等昨日先陳王大夫善戰海内所知也司徒五郎與商議放老夫等却歸太原諸節度亦各歸本道當為聞奏河北地任五郎收取滔見武俊戰勝私心忌其勝已乃謂武俊曰大夫二兄破懷光等氣已沮喪馬司徒既屈服如此且放去漸圖未晩武俊曰豈有四五節度兵逾十萬使打賊始經一陣被殺却五萬人將何面目歸見天子今窮蹙詐求退去料不過到洺州界必築壘相待悔難及也滔心明知其事竟絶水放燧等既離魏府城下退行三十里遂連魏縣河列營相拒滔雖慙謝武俊終有恨意又同進軍魏橋河東南去懷光營五里移營在七月中旬也邠志曰三年夏詔懷光率邠甲五千兼統諸軍東征六月師及魏郛戰焉陷燕人之衆師入賊營收其寶貨馬公燧曰我二年困此賊彼旦至而夕破之人其為我何乃稍抽戰卒以孤其勢田悦曰馬太原妬功也朔方軍可襲矣乃使步卒七百人負刀而趨乘我失度擠之于河死者數百人皆精騎也馬公遽命平射三百人爭橋以出我軍故步軍不敗軍勢大衂詔唐朝臣自河南引軍會之舊田悦傳曰王武俊以二千騎横擊懷光陳滔軍繼踵而進禁軍大敗人相蹈藉投尸于河二十里河水為之不流馬燧收軍保壘是夜王武俊決河水入王莽故河欲隔官軍水已深三尺糧餉路絶王師計無從出乃遣人告朱滔云云時武俊戰勝滔心忌之即曰大夫二兄已敗官軍馬司徒卑屈若此不宜迫人於險也武俊曰燧等連兵十萬皆是國之名臣一戰而北貽國之耻不知此等何面目見天子邪然吾不惜放還但不行五十里必反相拒按長歷六月壬子朔七月壬午朔然則辛巳六月三十日庚子七月十九日也滔與懷光至魏之日滔營壘猶未立懷光即與之戰豈得至七月六日邪戰于愜山之夜武俊決水明日燧等即退保魏縣豈得至十九日邪實録燕南記所載日皆不可據也然實録多據奏到之日不知戰與移營的在何日要之必在六七月之際故但記七月退保魏縣耳朱滔與王武俊同舉兵志在破馬燧軍豈有一戰纔勝遽忌武俊縱燧令去自貽後患邪直是滔無遠識謂燧等不足畏得其卑辭而縱去邪又舊悦傳云決河水若決黄河不須築堰決水經日不止三尺既決之後不可復壅今從實録決御河水灌王莽河耳}
李納求救於滔等滔遣魏博兵馬使信都承慶將兵

助之納攻宋州不克遣兵馬使李克信李欽遥戍濮陽南華以拒劉洽|{
	使疏吏翻將即亮翻又音如字濮博木翻濮陽縣時屬濮州南華縣屬曹州漢離狐縣也}
甲辰以淮寧節度使李希烈兼平盧淄青兖鄆登萊齊州節度使討李納又以河東節度使馬燧兼魏博澶相節度使加朔方邠寧節度使李懷光同平章事|{
	前此改淮西節度為淮寧軍鄆音運燧音遂澶時連翻相息亮翻邠卑旻翻}
神策行營招討使李晟請以所將兵北解趙州之圍與張孝忠合勢圖范陽上許之晟自魏州引兵北趨趙州|{
	晟成正翻將即亮翻又音如字趨逡諭翻}
王士眞解圍去晟留趙州三日與孝忠合兵北略恒州|{
	恒戶登翻}
演州司馬李孟秋舉兵反|{
	演州漢咸驩縣之地唐武德初置驩州貞觀九年改曰演州十六年省改咸驩為懷驩屬驩州廣德二年分驩州復置}
自稱安南節度使安南都護輔良交討斬之|{
	新書方鎮表乾元元年升安南營内經略使為安南節度使}
八月丁未置汴東西水陸運兩税鹽鐵使二人度支總其大要而已|{
	汴皮變翻度徒洛翻}
辛酉以涇原留後姚令言為節度使 盧惡太子太師顔眞卿欲出之於外|{
	惡烏路翻}
眞卿謂曰先中丞傳首至平原|{
	中丞謂父奕也事見二百一十七卷天寶十四載}
眞卿以舌舐面血|{
	舐直氐翻}
今相公忍不相容乎矍然起拜然恨之益甚 九月癸卯殿中少監崔漢衡自吐蕃歸|{
	去年崔漢衡使吐蕃}
贊普遣其臣區頰贊隨漢衡入見|{
	見賢遍翻}
冬十月辛亥以湖南觀察使曹王臯為江南西道節度使臯至洪州悉集將佐簡閲其才得牙將伊慎王鍔等擢為大將引荆襄判官許孟容置幕府慎兖州人孟容長安人也慎常從李希烈討梁崇義希烈愛其才欲留之慎逃歸希烈聞臯用慎恐為己患遺慎七屬甲|{
	遺唯季翻周禮函人為甲犀甲七屬鄭注云屬讀如灌注之注謂上旅下旅札屬之數凡七也}
詐為復書墜之境上上聞之遣中使即軍中斬慎臯為之論雪|{
	臯為于偽翻}
未報會江賊三千餘衆入寇|{
	江賊江中羣盗也自湖口入寇江南西道}
臯遣慎擊賊自贖慎擊破之斬首數百級而還由是得免 盧杞秉政知上必更立相|{
	相息亮翻}
恐其分己權乘間薦吏部侍郎關播儒厚可以鎮風俗|{
	間古莧翻}
丙辰以播為中書侍郎同平章事 |{
	考異曰舊播傳曰播為吏部侍郎轉刑部尚書十月拜銀青光禄大夫中書侍郎同中書門下平章事今實録自吏部侍郎為相與傳不同疑傳誤明年罷相乃改刑部尚書}
政事皆决於杞播但斂袵無所可否上嘗從容與宰相論事播意有所不可起立欲言目之而止還至中書杞謂播曰以足下端慤少言故相引至此曏者奈何口欲言邪播自是不復敢言|{
	從千容翻少詩沼翻復扶又翻}
戊辰遣都官員外郎樊澤使于吐蕃告以結盟之期 丙子肅王詳薨|{
	詳皇子也}
十一月己卯朔加淮南節度使陳少遊同平章事 田悦德朱滔之救與王武俊議奉滔為主稱臣事之滔不可曰愜山之捷皆大夫二兄之力|{
	二兄謂王武俊也武俊第二}
滔何敢獨居尊位於是幽州判官李子千恒冀判官鄭濡等 |{
	考異曰舊傳作李子牟鄭儒今從燕南記}
共議請與鄆州李大夫為四國|{
	鄆州李大夫謂李納也}
俱稱王而不改年號如昔諸侯奉周家正朔築壇同盟有不如約者衆共伐之不然豈可常為叛臣茫然無主用兵既無名有功無官爵為賞使將吏何所依歸乎滔等皆以為然滔乃自稱冀王田悦稱魏王王武俊稱趙王仍請李納稱齊王是日滔等築壇於軍中告天而受之|{
	考異曰實録於十一月末云是月朱滔僭稱大冀王燕南記云十月十一日於下營處各築壇塲設儀注告天稽首稱名同日偽立為王舊本紀朱滔王武俊傳皆云十一月而無日惟田悦傳云十一月一日今從之}
滔為盟主稱孤武俊悦納稱寡人所居堂曰殿處分曰令|{
	處昌呂翻分扶問翻令力定翻}
羣下上書曰牋妻曰妃長子曰世子各以其所治州為府置留守兼元帥以軍政委之又置東西曹視中書門下省左右内史視侍中中書令餘官皆倣天朝而易其名武俊以孟華為司禮尚書華竟不受嘔血死|{
	上時掌翻長知丈翻守式又翻帥所類翻司禮尚書視天朝禮部尚書朝直遥翻}
以兵馬使衛常寧為内史監|{
	彼所謂内史監當位于左右内史之上使疏吏翻}
委以軍事常寧謀殺武俊武俊腰斬之武俊遣其將張終葵寇趙州康日知擊斬之|{
	將即亮翻}
李希烈帥所部三萬徙鎮許州遣所親詣李納與謀共襲汴州|{
	帥讀曰率汴皮變翻九域志許州東至汴州二百一十五里}
遣使告李勉云己兼領淄青欲假道之官勉為之治橋具饌以待之|{
	為于偽翻淄莊持翻治直之翻饌雛戀翻又雛皖翻}
而嚴為之備希烈竟不至又密與朱滔等交通納亦數遣遊兵度汴以迎希烈|{
	數所角翻}
由是東南轉輸者皆不敢由汴渠自蔡水而上|{
	蔡河古之琵琶溝在浚儀縣杜佑曰漢運路出浚儀十里路入琵琶溝至陳州而合宋白曰建中初杜佑改漕路自浚儀西十里路其南涯引流入琵琶溝經蔡河至陳州合潁是秦漢故道自隋開汴河利涉揚楚故官漕不復由此道佑始開之上時掌翻}
十二月丁丑李希烈自稱天下都元帥太尉建興王時朱滔等與官軍相拒累月官軍有度支饋糧諸道益兵而滔與王武俊孤軍深入專仰給於田悦|{
	度徒洛翻仰牛向翻}
客主日益困弊|{
	客謂滔武俊之軍主謂田悦}
聞李希烈軍勢甚盛頗怨望乃相與謀遣使詣許州勸希烈稱帝希烈由是自稱天下都元帥|{
	使疎吏翻帥所類翻}
司天少監徐承嗣請更造建中正元歷從之|{
	乾元元年改太史局為司天臺以令為監正三品少監正四品上掌察天文稽歷數凡日月星辰風雲氣色之異率其屬占之肅宗時韓潁損益大衍歷為至德歷寶應元年代宗以至德歷不與天合詔司天臺官屬郭獻之等復用麟德元紀更立歲差增損遲疾交會及五星差數以寫大衍術曰五紀歷至是五紀歷氣朔加時稍後天推測星度與大衍差率頗異乃詔承嗣等雜麟德大衍之旨治新歷名建中正元歷少始照翻嗣祥吏翻更工衡翻}


資治通鑑卷二百二十七
