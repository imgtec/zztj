<!DOCTYPE html PUBLIC "-//W3C//DTD XHTML 1.0 Transitional//EN" "http://www.w3.org/TR/xhtml1/DTD/xhtml1-transitional.dtd">
<html xmlns="http://www.w3.org/1999/xhtml">
<head>
<meta http-equiv="Content-Type" content="text/html; charset=utf-8" />
<meta http-equiv="X-UA-Compatible" content="IE=Edge,chrome=1">
<title>資治通鑒_260-資治通鑑卷二百五十九_260-資治通鑑卷二百五十九</title>
<meta name="Keywords" content="資治通鑒_260-資治通鑑卷二百五十九_260-資治通鑑卷二百五十九">
<meta name="Description" content="資治通鑒_260-資治通鑑卷二百五十九_260-資治通鑑卷二百五十九">
<meta http-equiv="Cache-Control" content="no-transform" />
<meta http-equiv="Cache-Control" content="no-siteapp" />
<link href="/img/style.css" rel="stylesheet" type="text/css" />
<script src="/img/m.js?2020"></script> 
</head>
<body>
 <div class="ClassNavi">
<a  href="/24shi/">二十四史</a> | <a href="/SiKuQuanShu/">四库全书</a> | <a href="http://www.guoxuedashi.com/gjtsjc/"><font  color="#FF0000">古今图书集成</font></a> | <a href="/renwu/">历史人物</a> | <a href="/ShuoWenJieZi/"><font  color="#FF0000">说文解字</a></font> | <a href="/chengyu/">成语词典</a> | <a  target="_blank"  href="http://www.guoxuedashi.com/jgwhj/"><font  color="#FF0000">甲骨文合集</font></a> | <a href="/yzjwjc/"><font  color="#FF0000">殷周金文集成</font></a> | <a href="/xiangxingzi/"><font color="#0000FF">象形字典</font></a> | <a href="/13jing/"><font  color="#FF0000">十三经索引</font></a> | <a href="/zixing/"><font  color="#FF0000">字体转换器</font></a> | <a href="/zidian/xz/"><font color="#0000FF">篆书识别</font></a> | <a href="/jinfanyi/">近义反义词</a> | <a href="/duilian/">对联大全</a> | <a href="/jiapu/"><font  color="#0000FF">家谱族谱查询</font></a> | <a href="http://www.guoxuemi.com/hafo/" target="_blank" ><font color="#FF0000">哈佛古籍</font></a> 
</div>

 <!-- 头部导航开始 -->
<div class="w1180 head clearfix">
  <div class="head_logo l"><a title="国学大师官网" href="http://www.guoxuedashi.com" target="_blank"></a></div>
  <div class="head_sr l">
  <div id="head1">
  
  <a href="http://www.guoxuedashi.com/zidian/bujian/" target="_blank" ><img src="http://www.guoxuedashi.com/img/top1.gif" width="88" height="60" border="0" title="部件查字,支持20万汉字"></a>


<a href="http://www.guoxuedashi.com/help/yingpan.php" target="_blank"><img src="http://www.guoxuedashi.com/img/top230.gif" width="600" height="62" border="0" ></a>


  </div>
  <div id="head3"><a href="javascript:" onClick="javascript:window.external.AddFavorite(window.location.href,document.title);">添加收藏</a>
  <br><a href="/help/setie.php">搜索引擎</a>
  <br><a href="/help/zanzhu.php">赞助本站</a></div>
  <div id="head2">
 <a href="http://www.guoxuemi.com/" target="_blank"><img src="http://www.guoxuedashi.com/img/guoxuemi.gif" width="95" height="62" border="0" style="margin-left:2px;" title="国学迷"></a>
  

  </div>
</div>
  <div class="clear"></div>
  <div class="head_nav">
  <p><a href="/">首页</a> | <a href="/ShuKu/">国学书库</a> | <a href="/guji/">影印古籍</a> | <a href="/shici/">诗词宝典</a> | <a   href="/SiKuQuanShu/gxjx.php">精选</a> <b>|</b> <a href="/zidian/">汉语字典</a> | <a href="/hydcd/">汉语词典</a> | <a href="http://www.guoxuedashi.com/zidian/bujian/"><font  color="#CC0066">部件查字</font></a> | <a href="http://www.sfds.cn/"><font  color="#CC0066">书法大师</font></a> | <a href="/jgwhj/">甲骨文</a> <b>|</b> <a href="/b/4/"><font  color="#CC0066">解密</font></a> | <a href="/renwu/">历史人物</a> | <a href="/diangu/">历史典故</a> | <a href="/xingshi/">姓氏</a> | <a href="/minzu/">民族</a> <b>|</b> <a href="/mz/"><font  color="#CC0066">世界名著</font></a> | <a href="/download/">软件下载</a>
</p>
<p><a href="/b/"><font  color="#CC0066">历史</font></a> | <a href="http://skqs.guoxuedashi.com/" target="_blank">四库全书</a> |  <a href="http://www.guoxuedashi.com/search/" target="_blank"><font  color="#CC0066">全文检索</font></a> | <a href="http://www.guoxuedashi.com/shumu/">古籍书目</a> | <a   href="/24shi/">正史</a> <b>|</b> <a href="/chengyu/">成语词典</a> | <a href="/kangxi/" title="康熙字典">康熙字典</a> | <a href="/ShuoWenJieZi/">说文解字</a> | <a href="/zixing/yanbian/">字形演变</a> | <a href="/yzjwjc/">金 文</a> <b>|</b>  <a href="/shijian/nian-hao/">年号</a> | <a href="/diming/">历史地名</a> | <a href="/shijian/">历史事件</a> | <a href="/guanzhi/">官职</a> | <a href="/lishi/">知识</a> <b>|</b> <a href="/zhongyi/">中医中药</a> | <a href="http://www.guoxuedashi.com/forum/">留言反馈</a>
</p>
  </div>
</div>
<!-- 头部导航END --> 
<!-- 内容区开始 --> 
<div class="w1180 clearfix">
  <div class="info l">
   
<div class="clearfix" style="background:#f5faff;">
<script src='http://www.guoxuedashi.com/img/headersou.js'></script>

</div>
  <div class="info_tree"><a href="http://www.guoxuedashi.com">首页</a> > <a href="/SiKuQuanShu/fanti/">四库全书</a>
 > <h1>资治通鉴</h1> <!--         下载:【右键另存为】即可 --></div>
  <div class="info_content zj clearfix">
  
<div class="info_txt clearfix" id="show">
<center style="font-size:24px;">260-資治通鑑卷二百五十九</center>
    資治通鑑卷二百五十九 宋 司馬光 撰<br />
<br />
  胡三省 音註<br />
<br />
  唐紀七十五【起玄黓用敦盡閼逢攝提格凡三年】<br />
<br />
  昭宗聖穆景文孝皇帝上之中<br />
<br />
  景福元年春正月丙寅赦天下改元 鳳翔李茂貞静難王行瑜【難乃旦翻】鎮國韓建同州王行約秦州李茂莊五節度使上言楊守亮容匿叛臣楊復恭【事見上卷上年上時掌翻】請出軍討之乞加茂貞山南西道招討使朝議以茂貞得山南不可復制【復扶又翻】下詔和解之皆不聽 王鎔李匡威合兵十餘萬攻堯山李克用遣其將李嗣勲撃之大破幽鎮兵斬獲三萬 【考異曰實錄在二月恐約奏到今從唐太祖紀年錄】 楊行密謂諸將曰孫儒之衆十倍於我吾戰數不利【數所角翻】欲退保銅官何如劉威李神福曰儒掃地遠來利在速戰宜屯據險要堅壁清野以老其師時出輕騎抄其饋餉【抄楚交翻】奪其俘掠彼前不得戰退無資糧可坐擒也戴友規曰儒與我相持數年【僖宗光啓三年楊行密孫儒争揚州至是五年矣】勝負略相當今悉衆致死於我我若望風弃城正墮其計淮南士民從公度江及自儒軍來降者甚衆公宜遣將先護送歸淮南【降戶江翻將即亮翻】使復生業儒軍聞淮南安堵皆有思歸之心人心既搖安得不敗行密悦從之【以孫儒驅淮南人以攻楊行密故其謀云爾為行密擒孫儒張本】友規廬州人也 威戎節度使楊晟【僖宗文德元年置威戎軍於彭州】與楊守亮等約攻王建二月丁丑晟出兵掠新繁漢州之境使其將呂蕘將兵二千會楊守厚攻梓州【蕘如招翻梓州東川節度使顧彦暉治所】建遣行營都指揮使李簡擊蕘斬之 戊寅朱全忠出兵擊朱瑄遣其子友裕將兵前行軍於斗門【據舊書李師道傳斗門城在濮陽縣界】李茂貞王行瑜擅舉兵撃興元【不以天子之命舉兵故曰擅】茂貞表求招討使不已遺杜讓能西門君遂書【遣唯季翻杜讓能時為相西門君遂時為神策中尉此内外二大臣也】陵蔑朝廷上意不能容御延英召宰相諫官議之時宦官有隂與二鎮相表裏者宰相相顧不敢言上不悦給事中牛徽曰先朝多難茂貞誠有翼衛之功【此謂僖宗再幸山南時也難乃旦翻】諸楊阻兵亟出攻討其志亦在疾惡但不當不俟詔命耳比聞兵過山南【比毗至翻】殺傷至多陛下儻不以招討使授之使用國法約束則山南之民盡矣上曰此言是也乃以茂貞為山南西道招討使【牛徽之言上所以誘掖其君下所以彌縫悍將若以之為國謀則未也】 甲申朱全忠至衛南朱瑄將步騎萬人襲斗門朱友裕弃營走瑄據其營全忠不知乙酉引兵趣斗門【趣七喻翻】至者皆為鄆人所殺全忠退軍瓠河【九域志濮州雷澤縣有瓠河鎮】丁亥瑄擊全忠大破之全忠走張歸厚於後力戰全忠僅免 【考異曰歸厚傳云十一月誤也今從梁紀】副使李璠等皆死【璠音煩】 朱全忠奏貶河陽節度使趙克裕 【考異曰實錄在正月末云全忠欲全義得河陽乃奏克裕有誣謗之言而貶新紀云己未朱全忠陷孟州逐河陽節度使趙克裕今從編遺錄】以佑國節度使張全義兼河陽節度使【二鎮時皆屬朱全忠或貶或兼唯其所奏】 孫儒圍宣州初劉建鋒為孫儒守常州將兵從儒撃楊行密甘露鎮使陳可言帥部兵千人據常州【潤州城東角土山上有甘露寺前對北固山後枕大江寶歷中李德裕建寺適有甘露降因以名之孫儒蓋因此寺而置甘露鎮也帥讀曰率】行密將張訓引兵奄至城下可言倉猝出迎訓手刃殺之遂取常州 【考異曰新紀景福二年二月楊行密陷常州按行密自宣歸楊過常州巳歎張訓之功新紀誤也今從十國紀年】行密别將又取潤州【楊行密自此遂有潤州而與錢氏争常州矣】朱全忠連年攻時溥【光啓三年徐汴始交兵】徐泗濠三州民不<br />
<br />
  得耕穫兖鄆河東兵救之皆無功復值水災【復扶又翻】人死者什六七溥困甚請和於全忠全忠曰必移鎮乃可溥許之全忠乃奏請移溥他鎮仍命大臣鎮徐州詔以門下侍郎同平章事劉崇望同平章事充感化節度使以溥為太子太師溥恐全忠詐而殺之據城不奉詔崇望及華隂而還【華戶化翻還從宣翻又如字】 忠義節度使趙德諲薨子匡凝代之 【考異曰實錄此月以前忠義軍節度使趙匡凝起復某官不言德諲卒在何時新溥薛史但云匡凝為唐州刺史兼七州馬步軍都校及德諲卒自為襄州留後朝廷即以旄鉞授之亦下言年月今附於此】 范暉驕侈失衆心【范暉據福州見上卷上年】王潮以從弟彦復為都統弟審知為都監將兵攻福州【從才用翻監古銜翻】民自請輸米餉軍平湖洞及濱海蠻夷皆以兵船助之【平湖洞在泉州莆田縣界外几域志曰今興化軍大飛山地本平湖數頃一夕風雨暴至旦見此山聳峙一名大飛】辛丑王建遣族子嘉州刺史宗裕雅州刺史王宗侃<br />
<br />
  威信都指揮使華洪茂州刺史王宗瑤將兵五萬攻彭州【按九域志彭州距成都几十餘里此其壤地相接煙火相望所謂卧榻之側豈容他人鼾腄者也王建安得而不急攻之邪】楊晟逆戰而敗宗裕等圍之楊守亮遣其將符昭救之徑趨成都營三學山【趨七喻翻漢州金堂縣東北十里有三學山】建亟召華洪還洪疾驅而至【王建一時諸將唯華洪饒智略建所倚也故亟召之以禦符昭華戶化翻】後軍尚未集以數百人夜去昭營數里多擊更鼓昭以為蜀軍大至引兵宵遁【更工衡翻更鼓持更之鼓官府及行軍每更撃之以為節更鼓多則敵人以為營寨多故宵遁】 三月以戶部尚書鄭延昌為中書侍郎同平章事延昌從讜之從兄弟也【僖宗乾符間鄭從讜鎮河東有聲績之從才用翻】 左神策勇勝三都都指揮使楊子實子遷子釗皆守亮之假子也【勇勝三都亦神策五十二都之數】自渠州引兵救楊晟知守亮必敗壬子帥其衆二萬降於王建【帥讀曰率】 李克用王處存合兵攻王鎔癸丑拔天長鎮【天長鎮在滹沱河東北】戊午鎔與戰於新市大破之殺獲三萬餘人【新市漢古縣唐併入鎮州九門縣】辛酉克用退屯欒城詔和解河東及鎮定幽四鎮 楊晟遺楊守貞楊守忠楊守厚書【遺于季翻】使攻東川以解彭州之圍守貞等從之神策督將竇行實戍梓州守厚密誘之為内應【誘音酉】守厚至涪城行實事泄顧彦暉斬之 【考異曰實錄明年正月楊守厚攻東川以竇行實為内應事泄行實死守厚遁去因李茂貞與王建争東川追叙今年事耳今從十國紀年】守厚遁去守貞守忠軍至無所歸盤桓綿劒閣【綿劒二州名宋白曰綿州漢涪城縣地西魏置潼州隋置綿州以綿水為稱九域志綿州東北至劒州二百九十四里】王建遣其將吉諫襲守厚破之癸亥西川將李簡邀擊守忠於鍾陽【九域志綿州巴西縣有鍾陽鎮】斬獲三千餘人夏四月簡又破守厚於銅鉾【鉾亡侯翻】斬獲三千餘人降萬五千人守忠守厚皆走 乙酉置武勝軍於杭州以錢鏐為防禦使 天威軍使賈德晟以李順節之死頗怨憤【李順節死見上卷上年】西門君遂惡之【惡烏路翻】奏而殺之德晟麾下千餘騎奔鳳翔李茂貞由是益彊 李匡威出兵侵雲代壬寅李克用始引兵還【自鎮州引還】 時溥遣兵南侵至楚州楊行密將張訓李德誠敗之于夀河【敗補邁翻下同】遂取楚州執其刺史劉瓚【朱全忠以劉瓚刺楚州見二百五十七卷僖宗光啓三年張訓等既破徐兵乘勝遂取汴之楚州 考異曰新紀三月乙巳楊行密陷楚州執刺史劉瓚十國紀年三月時溥遣兵三萬南侵至楚州四月楊行密將張訓李德誠敗徐兵于夀河俘斬三千級取楚州執瓚今從之】加邠寧節度使王行瑜兼中書令楊行密屢敗孫儒兵破其廣德營【廣德營孫儒之兵營於廣德者也敗】<br />
<br />
  【補邁翻】張訓屯安吉斷其糧道【義寜二年沈法興分烏程置安吉縣唐因之屬湖州九域志在州西南百七十一里斷音短】儒食盡士卒大疫遣其將劉建鋒馬殷分兵掠諸縣六月行密聞儒疾瘧【瘧逆約翻疾而寒熱迭作謂之瘧】戊寅縱兵擊之會大雨晦冥儒軍大敗安仁義破儒五十餘寨田頵擒儒于陳【陳讀曰陣】斬之傳首京師儒衆多降於行密【光啓三年孫儒始與行密交兵至是而敗孫儒以十倍之衆攻行密其智勇亦無以大相過而卒斃於行密者儒專務殺掠人心不附又後無根本行密雖為儒所困分遣張訓李德誠略淮浙之地以自廣又斥餘廩以飼饑民既得人心又有根本所以勝也】劉建鋒馬殷收餘衆七千南走洪州【走音奏】推建鋒為帥殷為先鋒指揮使張佶為謀主比至江西衆十餘萬【帥所類翻比必利翻及也】丁酉楊行密帥衆歸揚州【帥讀曰率 考異曰十國紀年行密過常州謂左右曰常州大城也張訓以一劔下之不亦牡哉舊紀大順二年三月淮南節度使孫儒為寅州觀察使楊行密所殺初行密揚州失守據宣州孫儒以兵攻圍三年是春淮南大饑軍中疫癘是月孫儒亦病為帳下所執降行密行密乃併孫儒之衆復據廣陵薛居正五代史行密傳曰大順元年行密危蹙出據宣州儒復入揚州二年儒攻行密屬江淮疾疫師人多死儒亦卧病為部下所執送於行密殺之行密自宣城長驅入于廣陵唐補紀大順二年六月孫儒兵敗于宛陵城下楊行密進首級於西京吳錄曰景福元年六月六日太祖盡率諸將晨出擊儒田頵臨陳擒儒以獻斬儒于市傳首京師新紀實錄十國紀年皆據此舊紀薛史唐補紀皆誤】秋七月丙辰至廣陵表田頵守宣州安仁義守潤州先是揚州富庶甲天下【先悉薦翻】時人稱揚一益二【言揚州居一益州為次也】及經秦畢孫楊兵火之餘【秦彦畢師鐸孫儒楊行密也】江淮之間東西千里掃地盡矣王建圍彭州久不下民皆竄匿山谷諸寨日出俘掠<br />
<br />
  謂之淘虜都將先擇其善者餘則士卒分之以是為常【將即亮翻】有軍士王先成者新津人本書生也世亂為兵度諸將惟北寨王宗侃最賢乃往說之曰【度徒洛翻說式苪翻】彭州本西川之巡屬也陳田召楊晟割四州以授之【見二百五十七卷文德元年陳田謂陳敬瑄田令孜】偽署觀察使與之共拒朝命【朝直遙翻】今陳田已平而晟猶據之州民皆知西川乃其大府【巡屬諸州以節度使府為大府亦謂之會府】而司徒乃其主也【時朝命以王建檢校司徒故稱之】故大軍始至民不入城而入山谷避之以俟招安今軍至累月未聞招安之命軍士復從而掠之【復扶又翻】與盜賊無異奪其貲財驅其畜產分其老弱婦女以為奴婢使父子兄弟流離愁怨其在山中者暴露於暑雨殘傷於蛇虎孤危饑渇無所歸訴彼始以楊晟非其主而不從今司徒不加存恤彼更思楊氏矣宗侃惻然不覺屢移其牀前問之先成曰又有甚於是者今諸寨每旦出六七百人入山淘虜薄暮乃返【薄迫也】曾無守備之意賴城中無人耳萬一有智者為之畫策【為于偽翻】使乘虚奔突先伏精兵千人於門内登城望淘虜者稍遠出弓弩手礮手各百人【礮與砲同匹貌翻】攻寨之一面隨以役卒五百負薪土填壕為道然後出精兵奮擊且焚其寨又於三面城下各出耀兵【耀兵者以耀敵使不知所備】諸寨咸自備禦無暇相救城中得以益兵繼出如此能無敗乎宗侃矍然曰【矍居縛翻】此誠有之將若之何先成請條列為狀以白王建宗侃即命先成草之大指言今所白之事須四面通共【時西川兵圍彭州四面下寨宗裕宗侃華洪宗瑤各當一面】宗侃所司止於北面或所白可從乞以牙舉施行【牙舉謂從使牙檢舉而見之施行】事凡七條其一乞招安山中百姓其二乞禁諸寨軍士及子弟無得一人輒出淘虜仍表諸寨之旁七里内聽樵牧敢越表者斬其三乞置招安寨中容數千人以處所招百姓【處昌呂翻】宗侃請選所部將校謹幹者為招安將使將三十人書夜執兵巡衛其四招安之事須委一人總領今牓帖既下【下戶嫁翻】諸寨必各遣軍士入山招安百姓見之無不驚疑如鼠見狸誰肯來者【狸捕鼠者也鼠見狸則知必死特恨不可得而走耳詎肯前就之哉故以為諭】欲招之必有其術願降帖付宗侃專掌其事其五乞嚴勒四寨指揮使悉索前日所虜彭州男女老幼集於營場有父子兄弟夫婦自相認者即使相從牒具人數部送招安寨有敢私匿一人者斬仍乞勒府中諸營亦令嚴索【府謂成都府索山客翻】有自軍前先寄歸者量給資糧【量音良】悉部送歸招安寨其六乞置九隴行縣於招安寨中【彭州治九隴縣彭州未下故乞置行縣九隴故漢繁縣地後魏改曰九隴以州西冇九隴山為名九隴一伏隴二豆隴三秋隴四龍奔隴五走馬隴六駱駞隴七千秋隴八較車隴九横擔隴】以前南鄭令王丕攝縣令【南鄭漢古縣唐帶興元府】設置曹局撫安百姓擇其子弟之壯者給帖使自入山招其親戚彼知司徒嚴禁浸掠前日為軍士所虜者皆獲安堵必歡呼踊躍相帥下山【帥讀曰率】如子歸母不日盡出其七彭州土地宜麻【益州記彭之地號小郫言土地肥良北之郫邑也】百姓未入山時多漚藏者【漚烏候翻久漬也】宜令縣令曉諭各歸田里出所漚麻鬻之以為資糧必漸復業建得之大喜即行之悉如所申【考異曰張耆舊傳云五月二十日諸軍馬步兵士到彭州城下至七月初已經五十餘日諸軍兵士始到刈麥充糧至七月初麥盡並無顆粒兵士但託求食乃每日遠去入山虜刼逃避百姓有一軍士本是儒生乃往北面寨說于統帥云云十國紀年王先成謂王宗侃云云先成上招攜七事建皆納之先成蜀州新津人按十國紀年王建自二月辛丑遣王宗裕等擊楊晟遂圍彭州又晟遺楊守忠書云弊邑雖小圍守三年矣而張云五月二十日方圍彭州或者先圍之不克而再往歟但云有一軍士而十國紀年姓王名先成不知其本出何書也】明日牓帖至威令赫然無敢犯者三日山中民競出赴招安寨如歸市寨不能容斥而廣之浸有市井又出麻鬻之民見村落無抄暴之患【抄楚交翻】稍稍辭縣令復故業月餘招安寨皆空 己巳李茂貞克鳳州感義節度使滿存奔興元【僖宗光啓二年滿存得鳳州至是而敗奔興元就楊守亮】茂貞又取興洋二州皆表其子弟鎮之 【考異曰薛居正五代史茂貞傳曰大順二年楊復恭得罪奔山南與楊守亮據興元叛茂貞與王行瑜討平之詔以徐彦若鎮興元茂貞違詔表其假子繼徽為留後堅請旄鉞昭宗不得已而授之自是茂貞始萌問鼎之志既而逐涇原節度使張球洋州節度使楊守忠鳳州刺史滿存皆奪據其地云大順二年誤也今從新紀】 八月以楊行密為淮南節度使同平章事以田頵知宣州留後安仁義為潤州刺史孫儒降兵多蔡人【降戶江翻】行密選其尤勇健者五千人厚其稟賜以皁衣蒙甲【稟筆錦翻給也皁才早翻】號黑雲都每戰使之先登陷陳【陳讀曰陣】四鄰畏之行密以用度不足欲以茶鹽易民布帛掌書記舒城高勗曰兵火之餘十室九空又漁利以困之【記坊記諸侯不下漁色注曰象捕魚然中網取之是無所擇漁利之漁猶漁色之漁】將復離叛【復扶又翻下同】不若悉我所有易鄰道所無足以給軍選賢守令勸課農桑數年之間倉庫自實行密從之【守式又翻令力正翻】田頵聞之日賢者之言其利遠哉行密馳射武伎皆非所長【伎渠綺翻】而寛簡有智略善撫御將士與同甘苦推心待物無所猜忌嘗早出從者斷馬鞦取其金【從才用翻斷音短鞦七由翻史炤曰馬紂也】行密知而不問他日復早出如故人服其度量淮南被兵六年【光啓三年卑師鐸亂淮南始被兵被皮義翻】士民轉徙幾盡【幾居依翻下幾復同】行密初至賜與將吏【將即亮翻】帛不過數尺錢不過數百而能以勤儉足用非公宴未嘗舉樂招撫流散輕徭薄歛【歛力贍翻】未及數年公私富庶幾復承平之舊【復還也讀如字】 李克用北巡至天寧軍【代州西有天安軍天寶十二載置】聞李匡威赫連鐸將兵八萬寇雲州遣其將李君慶發兵於晉陽克用潛入新城伏兵於神堆【神堆在雲州城南新城又在神堆東南神堆即神武川之黄花堆新城在其側盖克用祖執宜保黄花堆時所築也按薛史唐紀李克用生於神武川之新城宋白曰雲州西南至神堆柵九十里】擒吐谷渾邏騎三百【邏郎佐翻】匡威等大驚丙申君慶以大軍至克用遷入雲州丁酉出撃匡威等大破之己亥匡威等燒營而遁追至天成軍【蔚州東北有天成軍】斬獲不可勝計【勝音升】 辛丑李茂貞攻拔興元楊復恭楊守亮楊守信楊守貞楊守忠滿存奔閬州【光啓三年楊守亮鎮興元至是而敗 考異曰舊紀景福元年十一月辛丑鳳翔邠寧之衆攻興元陷之節度使楊守亮前中尉楊復恭判官李巨川突圍而遁十二月辛未華州刺史韓建奏於乾元縣遇興元散兵撃敗之斬楊守亮楊復恭傳首實錄乾寜元年七月鳳翔邠寧之兵攻興元陷之楊守亮楊復恭突圍而遁新紀景福元年八月茂貞寇興元守亮滿存奔閬州乾寧元年七月茂貞陷閬州八月守亮伏誅新復恭傳景福元年茂貞攻興元破其城復恭守亮守信奔閬州十國紀年蜀史景福元年十月行瑜茂貞表守亮招納叛臣請討之感義節度使滿存救守亮為茂貞所敗奔興元十一月邠岐攻陷興元楊復恭帥守亮守貞守忠滿存同奔閬州十二月壬午華洪敗守亮等於州按實錄景福二年正月移茂貞山南於時守亮不應猶在山南今年月從新紀事則參取諸書】茂貞表其子繼密權知興元府事九月加荆南節度使成汭同平章事 時溥迫監軍奏稱將士留已【是年二月召時漙為太子太師】冬十月復以漙為侍中感化節度朱全忠奏請追溥新命詔諭解之溥初邢洺磁州留後李存孝與李存信俱為李克用假子不相睦存信有寵於克用存孝在邢州欲立大功以勝之乃建議取鎮冀【見上卷大順二年】存信從中沮之不時聽許及王鎔圍堯山存孝救之不克克用以存信為蕃漢馬步都指揮使與存孝共擊之二人互相猜忌逗留不進克用更遣李嗣勲等擊破之【事見上是年正月】存信還譖存孝無心擊賊疑與之有私約存孝聞之自以有功於克用而信任顧不及存信【顧反也】憤怨且懼及禍乃潛結王鎔及朱全忠上表以三州自歸於朝廷 【考異實錄大順元年十月太原將邢州刺史李存孝自晉州帥行營兵據邢州舊記十一月癸丑朔大原將邢州刺史李存孝自恃擒孫揆功合為昭義帥怨克用授康君立存孝自晉州帥行營兵歸邢州據城上表歸明仍致書與張濬王鎔求救援唐末見聞錄十月二十四日李存孝領兵打晉州遁歸邢州背叛與宰臣張濬狀曰某自主三郡已近二年又曰常思安知建在此之日歸順朝廷之時四鄰不有保持一家俄受塗炭以此猶豫莫敢申明遂至去年遽絶鄰好豈是某之情願蓋因李某之指揮又曰自今春戰争之後實願休罷戈鋌自九月十五日以來有李某之人使促令某南至進軍面趙州牽脅李某即土門路入直届鎮州今月十四日昭義軍人百姓等衆請某權知兵馬留後歸順朝廷大王聞存孝致逆大震雄威令下先差大將進軍速至邢州仍候指揮不得輒有鬬敵但圍小壘專俟大軍據唐太祖紀年錄薛居正五代史紀傳實錄新紀皆云景福元年十月存孝叛太原歸朝廷而紀唐末見聞錄在大順元年十月舊紀恐是連言以後事按二年三月安知建方叛太原而此書中已說知建又云自主三郡已近二年存孝大順二年方為邢洺磁節度至景福元年乃二年也然則實錄邢州刺史據邢州亦因舊記之誤見聞錄所載存孝書蓋與王鎔誤云與張濬也】乞賜旌節及會諸道兵討李克用詔以存孝為邢洺磁節度使不許會兵 十一月時溥濠州刺史張璲泗州刺史張諫以州附于朱全忠【璲徐醉翻史言時溥巡屬皆附于汴漙僅保徐州】 乙未朱全忠遣其子友裕將兵十萬攻濮州拔之執其刺史邵倫【濮州朱瑄巡屬濮博木翻】遂令友裕移兵擊時溥 孫儒將王壇陷婺州刺史蔣瓌奔越州【中和四年蔣瓌據婺州】 廬州刺史蔡儔發楊行密祖父墓【光啓三年楊行密留蔡儔守廬州明年儔以州附孫儒儒既敗儔遂阻兵以拒行密】與舒州刺史倪章連兵遣使送印於朱全忠以求救全忠惡其反覆【惡烏路翻】納其印不救且牒報行密行密謝之行密遣行營都指揮使李神福將兵討儔 宣明歷浸差【穆宗立以為累世纘緒必更歷紀乃詔日官改撰歷術名曰宣明】太子少詹事邊岡造新歷成十二月上之命曰景福崇玄歷【邊岡與司天少監胡秀林均州司馬王穉改治新歷然術一出於岡岡用算巧能馳騁反覆於乘除間由是簡捷超徑等接之術興而經制遠大衰序之法廢矣雖籌策便易然皆冥於本原上時掌翻】 壬午王建遣其將華洪擊楊守亮於閬州破之建遣節度押牙延陵鄭頊使於朱全忠【延陵漢曲阿縣地晉分置延陵郡隋移治丹徒武德三年移於舊郡治屬潤州今丹楊縣之延陵鎮即其地】全忠問劔閣頊極言其險全忠不信頊曰苟不以聞恐誤公軍機全忠大笑 是歲明州刺史鍾文季卒其將黄晟自稱刺史【路振九國志黄晟明州鄞縣人歷為將領會刺史鍾文季卒遂據其郡】<br />
<br />
  二年春正月時溥遣兵攻宿州刺史郭言戰死【大順二年朱全忠取宿州事見上卷】 東川留後顧彦暉既與王建有隙【大順二年楊守亮攻東川王建遣兵救之欲因而取之不克由是與顧彦暉有隙事亦見上卷】李茂貞欲撫之使從已奏請更賜彦暉節【大順二年朝廷遣中使賜顧彦暉節楊守厚邀而奪之故請更賜】詔以彦暉為東川節度使【申前命也】茂貞又奏遣知興元府事李繼密救梓州【梓州朱受兵而救之何也非救之也遣兵助彦暉以致西川之師耳】未幾建遣兵敗東川鳳翔之兵於利州【幾居豈翻敗補邁翻下同】彦暉求和請與茂貞絶乃許之鳳翔節度使李茂貞自請鎮興元詔以茂貞為山南西道兼武定節度使以中書侍郎同平章事徐彦若同平章事充鳳翔節度使【考異曰舊紀在七月癸未今從實錄新紀】又割果閬二州隸武定軍茂貞欲兼得鳳翔不奉詔 二月甲戍加西川節度使王建同平章事 李克用引兵圍邢州王鎔遣牙將王藏海致書解之克用怒斬藏海進兵撃鎔敗鎮兵於平山【平山漢蒲吾縣隋為房山縣至德元年改為平山縣屬鎮州九域志在州西六十五里】辛巳攻天長鎮旬日不下鎔出兵三萬救之克用逆戰於叱日嶺下大破之斬首萬餘級餘衆潰去河東軍無食脯其尸而啗之【啗徒濫翻】 時溥求救於朱瑾朱全忠遣其將霍存將騎兵三千軍曹州以備之瑾將兵二萬救徐州存引兵赴之與朱友裕合擊徐兖兵於石佛山下大破之【石佛山近彭城薛史曰石佛山在彭門南述征記彭城南有石佛山頂方二丈二尺】瑾遁歸兖州辛卯徐兵復出存戰死【霍存恃勝而不虞徐兵之復出故戰敗而死復扶又翻】 李克用進下井陘李存孝將兵救王鎔遂入鎮州與鎔計事鎔又乞師於朱全忠全忠方與時溥相攻不能救但遺克用書【遺唯季翻】言鄴下有十萬精兵抑而未進克用復書儻實屯軍鄴下顒望降臨【顒魚容翻仰也】必欲真决雌雄願角逐於常山之尾甲午李匡威引兵救鎔敗河東兵於元氐【敗補邁翻】克用引還邢州鎔犒匡威於藁城輦金帛二十萬以酬之 朱友裕圍彭城時溥數出兵友裕閉壁不戰【去年十一月朱全忠遣友裕攻彭城此言其積時相持之事數所角翻】朱瑾宵遁友裕不追【謂石佛山下戰時】都虞侯朱友恭以書譖友裕于全忠全忠怒驛書下都指揮使龎師古【下戶嫁翻】使代之將且按其事書誤達於友裕友裕大懼以二千騎逃入山中【按薛史元貞張后傳作二十騎朱友裕傳作數騎二千騎太多當以二十騎為是】潛詣碭山匿於伯父全昱之所【朱全忠兄弟本居碭山全昱全忠長兄也碭音唐】全忠夫人張氏聞之使友裕單騎詣汴州見全忠泣涕拜伏于庭人王忠命左右捽抑將斬之【捽者持其髻抑者按其頸捽昨沒翻】夫人趨就抱之泣曰汝捨兵衆束身歸罪無異志明矣全忠悟而捨之使權知許州友恭夀春人李彦威也幼為全忠家僮 【考異曰薛居正五代史高季興傳以友恭為汴之賈人李七郎十國紀年以為夀春賈人友恭傳云彦威丱角事太祖今從之】全忠養以為子張夫人碭山人多智略全忠敬憚之雖軍府事時與之謀議或將兵出中塗夫人以為不可遣一介召之全忠立為之返【為于偽翻】龎師古攻佛山寨拔之【佛山寨即石佛山寨】自是徐兵不敢出 李匡威之救王鎔也將發幽州家人會别【家人悉會於使宅以送别】弟匡籌之妻美匡威醉而淫之二月匡威自鎮州還至博野匡籌據軍府自稱留後以符追行營兵匡威衆潰歸但與親近留深州【深州在博野東南一百五十里】進退無所之遣判官李抱真入奏請歸京師京師屢更大亂【更工衡翻經也】聞匡威來坊市大恐曰金頭王來圖社稷士民或竄匿山谷王鎔德其以己故致失地【德其救己以致失幽州】迎歸鎮州為築第父事之【為于偽翻為李匡威刼王鎔而死張本薛史曰鎔館匡威於寶夀佛寺】 以渝州刺史柳玭為瀘州刺史【九域志渝州西至瀘州七百六十里玭部田翻考異曰新傳云玭坐事貶瀘州刺史卒北夢瑣言亦云謫授瀘州新舊書玭貶官無年月今据實錄此月玭自渝為瀘州刺史當是初貶渝州後移瀘州新傳北夢瑣言誤也】柳氏自公綽以來世以孝悌禮法為士大夫所宗【言自元和以來為名家】玭為御史大夫上欲以為相宦官惡之【惡烏路翻】故久謫於外玭嘗戒其子弟曰凡門地高可畏不可恃也立身行巳一事有失則得罪重於他人死無以見先人於地下此其所以可畏也門高則驕心易生族盛則為人所嫉懿行實才人未之信【行下孟翻下同】小有玼纇【玼疾移翻纇盧對翻玉病曰玼絲節曰纇】衆皆指之此其所以不可恃也故膏粱子弟學宜加勤行宜加勵僅得比他人耳【使柳氏子姪常能守玼之戒各務修飭雖至今為名家可也】 王建屢請殺陳敬瑄田令孜朝廷不許夏四月乙亥建使人告敬瑄謀作亂殺之新津【陳敬瑄居新津見上卷大順二年】又告令孜通鳳翔書下獄死【下遐稼翻】建使節度判官馮㳙草表奏之曰開匣出虎孔宣父不責他人當路斬蛇孫叔敖蓋非利己【論語孔子責冉有季路曰虎兕出於柙是誰之過歟楚孫叔敖為嬰兒出遊而還憂而不食其母問其故泣而對曰今日吾見兩頭蛇恐去死無日矣母曰今蛇安在曰吾聞見兩頭蛇者死吾恐他人復見已埋之也母曰無憂汝不死吾聞之有隂德者天報以福人聞之皆諭其為仁也】專殺不行於閫外先機恐失於彀中【彀古候翻】㳙宿之孫也【馮宿事見二百四十五卷開成元年㳙圭淵翻】 汴軍攻徐州累月不克【自去年十一月攻徐州至是五月矣】通事官張濤以書白朱全忠云進軍時日非良故無功全忠以為然敬翔曰今攻城累月所費甚多徐人已困旦夕且下使將士聞此言則懈於攻取矣全忠乃焚其書癸未全忠自將如徐州戊子龎師古拔彭城時溥舉族登鷰子樓自焚死【僖宗中和元年時溥據徐州至是而亡張建封之鎮徐也有愛妓曰盻盻建封既歿張氏舊第有小樓名燕子盻盻念舊愛而不嫁居是樓十餘年幽獨悵然出白樂天集 考異曰實錄五月汴州奏拔徐州舊紀四月汴將王重師牛存節陷徐州舊傳溥求援于兖州朱瑾出兵救之值大雪糧盡而還汴將王重師牛存節夜乘梯而入漙與妻子登樓自焚而卒景福二年也新紀四月戊子朱全忠陷徐州時漙死之薛居正五代史梁紀丁亥師古下彭門梟漙首以獻唐太祖紀年錄四月澤州李罕之上言懷孟降人報汴將龎師古於今月八日攻陷徐州徐帥時溥舉族皆沒温既下徐方詐請朝廷命帥昭宗乃以兵部尚書孫儲為徐帥既而温以他詞斥去自以其將鎮之四月八日蓋河東傳聞之誤今從編遺錄新紀】己丑全忠入彭城以宋州刺史張廷範知感化留後奏乞朝廷除文臣為節度使李匡威在鎮州為王鎔完城塹繕甲兵視之如子匡威以鎔年少且樂真定土風【為于偽翻下為之同鎮州漢之真定國也樂音洛】潛謀奪之李抱真自京師還為之畫策隂以恩施悦其將士【施式䜴翻】王氏在鎮久鎮人愛之不徇匡威匡威忌日鎔就第弔之【父母終之日子以為忌日第者李匡威寓第也】匡威素服衷甲伏兵刼之鎔趨抱匡威曰鎔為晉人所困幾亡矣【晉人謂河東李克用之兵幾居依翻】賴公以有今日公欲得四州此固鎔之願也【鎮冀深趙四州】不若與公共歸府以位讓公則將士莫之拒矣匡威以為然與鎔駢馬【駢馬並馬也】陳兵入府會大風雷雨屋瓦皆震匡威入東偏門【此鎮州牙城之東偏門也】鎮之親軍閉之【既入門而為鎮兵所閉絶其繼至者】有屠者墨君和自缺垣躍出拳敺匡威甲士【敺烏口翻】挾鎔於馬上負之登屋鎮人既得鎔攻匡威殺之并其族黨 【考異日實錄殺匡威在五月恐約奏到舊紀六月乙卯幽州李匡威謀害王鎔恒州三軍攻匡威殺之舊傳唐太祖紀年錄皆云五月新紀四月丁亥按匡籌奏云四月十九日是月己巳朔十九日丁亥也今從之】鎔時年十七體疎瘦為君和所挾頸痛頭偏者累日李匡籌奏鎔殺其兄請舉兵復寃詔不許幽州將劉仁恭將兵戍蔚州過期未代士卒思歸會李匡籌立戍卒奉仁恭為帥還攻幽州【蔚紆勿翻帥所類翻】至居庸關為府兵所敗【府兵幽州節度使府之兵也敗補邁翻】仁恭奔河東李克用厚待之【為李克用取幽州張本】 李神福圍廬州甲午楊行密自將詣廬州田頵自宣州引兵會之初蔡人張顥以驍勇事秦宗權後從孫儒儒敗歸行密行密厚待之使將兵戍廬州蔡儔叛顥更為之用及圍急顥踰城來降【降戶江翻】行密以隸銀槍都使袁稹【使疏吏翻稹止忍翻】稹以顥反覆白行密請殺之行密恐稹不能容置之親軍【為張顥殺楊渥張本】稹陳州人也【稹止忍翻】 王彦復王審知攻福州久不下【去年二月王潮遣彦復等攻福州】范暉求救於威勝節度使董昌【僖宗中和三年升浙東觀察為義勝節度光晵三年改為威勝節度】昌與陳巖昏姻發温台婺州兵五千救之彦復審知以城堅援兵且至士卒死傷多白王潮欲罷兵更圖後舉潮不許請潮自臨行營潮報曰兵盡添兵將盡添將兵將俱盡吾當自來【將即亮翻】彦復審知懼親犯矢石急攻之五月城中食盡暉知不能守夜以印授監軍弃城走援兵亦還庚子彦復等入城辛丑暉亡抵沿海都為將士所殺【文德二年范暉據福州】潮入福州自稱留後素服葬陳巖以女妻其子延晦厚撫其家【妻七細翻】汀建二州降嶺海間羣盜二十餘輩皆降潰【言羣盜或降或潰也王氏自此遂據有七閩矣降戶江翻】 閏月以武勝防禦使錢鏐為蘇杭觀察使【錢鏐以杭并蘇因以命之】又以扈蹕都頭曹誠為黔中節度使耀德都頭李鋋為鎮海軍節度使宣威都頭孫惟晟為荆南節度使【耀德宣威皆神策五十四都之數黔渠今翻鋋音蟬】六月以捧日都頭陳珮為嶺南東道節度使並同平章事時李茂貞跋扈上以武臣難制欲用諸王代之故誠等四人皆加恩解兵柄令赴鎮【後四人不聞至鎮蓋各有分據者四人不得而赴也 考異曰舊紀三月庚子以陳珮為嶺南東道節度使曹誠為黔中節度使李鋋為鎮海節度使孫惟晟為荆南節度使時朝議以茂貞傲侮王命武臣難制故罷五將之權今從實錄止是四將】 李匡籌出兵攻王鎔之樂夀武強以報殺匡威之恥 秋七月王鎔遣兵救邢州李克用敗之於平山【敗補邁翻】壬申進撃鎮州鎔懼請以兵糧二十萬助攻邢州克用許之克用治兵于欒城合鎔兵三萬進屯任縣【任漢古縣中廢唐之任漢南欒縣地武德四年置任縣治苑鄉城在邢州東南治直之翻】李存信屯琉璃陂【琉璃陂在邢州龍岡縣界】 丁亥楊行密克廬州斬蔡儔左右請發儔父母冢行密曰儔以此得罪吾何為效之【蔡儔發行密祖父冢見上年】 加天雄節度使李茂莊同平章事【時以秦州為天雄軍】 錢鏐發民夫二十萬及十三都軍士築杭州羅城周七十里【錢鏐以八都兵起後其衆日盛置十三都今杭州羅城鏐所築也】 昇州刺史張雄卒 【考異曰新紀八月庚子盖約奏到之日今從十國紀年】馮弘鐸代之為刺史 李茂貞恃功驕横上表及遺杜讓能書【横戶孟翻上時掌翻遺唯季翻】辭語不遜上怒欲討之茂貞又上表略曰陛下貴為萬乘不能庇元舅之一身【元舅謂王瓌事見上卷大順二年】尊極九州不能戮復恭之一豎又曰今朝廷但觀彊弱不計是非又曰約衰殘而行法隨盛壯以加恩【李茂貞之表辭固慢然當時之政事實亦如此】體物錙銖【言體物有錙銖之重則待之亦重有錙銖之輕則待之亦輕】看人衡纊【劉峻廣絶交論曰衡所以揣其輕重纊所以屬其鼻息注云謂操衡揣勢之輕重持纊量氣之粗細】又曰軍情易變戎馬難羈唯慮甸服生靈因兹受禍【古之王者畿方千里以為甸服】未審乘輿播越自此何之【乘䋲證翻】上益怒决討茂貞命杜讓能專掌其事讓能諫曰陛下初臨大寶國步未夷茂貞近在國門【按九域志鳳翔東距長安二百八十里耳】臣愚以為未宜與之構怨萬一不克悔之無及上曰王室日卑號令不出國門此乃志士憤痛之秋藥弗瞑眩厥疾弗瘳【書說命之辭注云如服藥必瞑眩極其病乃除瞑莫遍翻眩玄遍翻瞑眩困極也】朕不能甘心為孱懦之主【孱鉏山翻】愔愔度日【愔於禽翻愔愔深静貌】坐視陵夷卿但為朕調兵食【為干偽翻調徒釣翻】朕自委諸王用兵成敗不以責卿讓能曰陛下必欲行之則中外大臣共宜協力以成聖志不當獨以任臣上曰卿位居元輔【杜讓能時為首相】與朕同休戚無宜避事讓能泣曰臣豈敢避事况陛下所欲行者憲宗之志也顧時有所未可勢有所不能耳但恐他日臣徒受晁錯之誅不能弭七國之禍也【晁錯事見漢景帝紀】敢不奉詔以死繼之【杜讓能固巳知必死矣】上乃命讓能留中書計畫調度月餘不歸【不歸私第也調徒弔翻】崔昭緯隂結邠岐為之耳目讓能朝發一言二鎮夕必知之李茂貞使其黨糾合市人數百千人擁觀軍容使西門君遂馬訴曰岐帥無罪【岐帥謂李茂貞鳳翔本岐州帥所類翻】不宜致討使百姓塗炭君遂曰此宰相事非吾所及市人又邀崔昭緯鄭延昌肩輿訴之【舊制朝臣入朝皆乘馬宋建炎播遷以揚州街路滑始許朝士乘擔子觀此則唐末宰相亦有乘肩輿者矣】二相曰兹事主上專委杜太尉吾曹不預知市人因亂投瓦石二相下輿走匿民家僅自免喪堂印及朝服上命捕其唱帥者誅之【喪息浪翻朝直遙翻下同帥讀曰率】用兵之意益堅京師民或亡匿山谷嚴刑所不能禁八月以嗣覃王嗣周為京西招討使 【考異曰按順宗子經封郯王嗣周當是其後會昌後避武宗諱改郯作覃 按武宗諱瀍後改諱炎如考異所云蓋避郯字旁從炎字也】神策大將軍李鐬副之【鐬火外翻】 丙辰楊行密遣田頵將宣州兵二萬攻歙州歙州刺史裴樞城守久不下【歙書涉翻】時諸將為刺史者多貪暴獨池州團練使陶雅寛厚得民歙人曰得陶雅為刺史請聽命行密即以雅為歙州刺史歙人納之雅盡禮見樞送之還朝樞遵慶之曾孫也【裴遵慶見二百二十二卷肅宗上元二年】 朱全忠命龎師古移兵攻兖州與朱瑾戰屢破之 九月丁卯以錢鏐為鎮海節度使【升杭州武勝防禦使為鎮海節度使唐本置鎮海軍於潤州今以命錢鏐於杭州至光化元年鏐遂請徙軍於杭州 考異曰今年五月以李鋋為鎮海節度使令赴鎮今復除鏐者按是時安仁義已據潤州又孫惟晟除荆南時成汭已據荆南二人安得赴鎮蓋但欲罷其軍權其實不至鎮而返耳實錄云仍徙鎮海軍額于杭州按吳越備史是歲鏐初除鎮海節度使猶領潤州刺史至光化元年始移鎮海軍于杭州實錄誤也】 李存孝夜犯李存信營虜奉誠軍使孫考老李克用自引兵攻邢州掘塹築壘環之【環音宦】存孝時出兵突擊塹壘不能成河東牙將袁奉韜密使人謂存孝曰大王惟俟塹成即歸晉陽尚書所憚者獨大王耳【李克用時封隴西郡王存孝蓋亦檢校尚書】諸將非尚書敵也大王若歸咫尺之塹安能沮尚書之鋒銳邪存孝以為然按兵不出旬日塹壘成飛走不能越存孝由是遂窮汴將鄧季筠從克用攻邢州輕騎逃歸【鄧季筠被擒見上卷大順元年】朱全忠大喜使將親軍【將即亮翻】 乙亥覃王嗣周帥禁軍三萬送鳳翔節度使徐彦若赴鎮軍於興平【帥讀曰率 考異曰舊紀覃王率扈駕五十四軍進攻岐陽今從實錄】李茂貞王行瑜合兵近六萬軍于盩厔以拒之【近其靳翻】禁軍皆新募市井少年茂貞行瑜所將皆邊兵百戰之餘壬午茂貞等進逼興平禁軍皆望風逃潰茂貞等乘勝進攻三橋京城大震士民奔散市人復守闕請誅首議用兵者【復扶又翻】崔昭緯心害太尉門下侍郎同平章事杜讓能密遺茂貞書曰用兵非主上意皆出於杜太尉耳【遺于季翻】甲申茂貞陳於臨皋驛【臨皋驛在長安城西】表讓能罪請誅之讓能言於上曰臣固先言之矣請以臣為解【言歸罪於讓能以解兵也】上涕下不自禁【禁居吟翻】曰與卿訣矣是日貶讓能梧州刺史【梧州去京師五千五百里宋白曰漢武帝置蒼梧郡理廣信縣隋置蒼梧郡理蒼梧縣唐為梧州】制辭略曰弃卿士之臧謀構藩垣之深釁咨詢之際證執彌堅又流觀軍容使西門君遂于儋州内樞密使李周潼于崖州段詡于驩州乙酉上御安福門斬君遂周潼詡再貶讓能雷州司戶遣使謂茂貞曰惑朕舉兵者三人也非讓能之罪以内侍駱全瓘劉景宣為左右軍中尉壬辰以東都留守韋昭度為司徒門下侍郎同平章事御史中丞崔胤為戶部侍郎同平章事胤慎由之子也【崔慎由歷事文武宣大中間為相】外寛弘而内巧險與崔昭緯深相結故得為相季父安潛謂所親曰吾父兄刻苦以立門戶終為緇郎所壞【壞音怪 考異曰舊傳胤初拜平章事安潛有此言按安潛去年卒必先時嘗有此言也】緇郎胤小字也李茂貞勒兵不解請誅杜讓能然後還鎮崔昭緯復從而擠之【復扶又翻下同擠牋西翻又子細翻】冬十月賜讓能及其弟戶部侍郎弘徽自盡【考異曰續實運錄曰大順二年相國杜讓能孔緯值上京頻嬰離亂朝綱紊墜是時徇意諸道扈駕兵五十四都坊坊皆滿兼近藩連帥要行征討便自統軍至如岐陽李茂貞先朝封為太子本姓宋洋州牧先祖討昭義劉從諫有功子孫爵賞不絶洎夀王登位後遣禮部侍郎薛廷珪持璽書具禮冊為岐王茂貞先中和年中投判軍容使田令孜作養男姓田名彦賓蓋趨其勢也汴州朱温先朝冊東平王至今上又遣薛廷珪為禮儀使延王為冊命使封為梁王且岐王與北司人情方洽宰相甚不和睦累表章云臣今駐斾咸陽未敢入中書問罪杜讓能等請寘極法表奏上不悦遂詔孔杜二相國令往咸陽謝及見岐王戰不能言岐王大怒却令歸中書省過纔到中書上又發遣令祈謝岐王如是往來三度岐王又奏曰二相見臣並不措一言如此曠官有辱聖代請行朝典别選英賢上不樂敕罷知政事不得已除昭緯荆南節度杜讓能除河中節度三日後貶于嶺表出國門三十里並賜自盡時岐王率驍果五千人住咸陽及貶二相乃退此皆誤謬之說今從實錄】復下詔布告中外稱讓能舉枉錯直【用論語孔子之言謂枉者舉之直者錯而不用也衰亂之朝安有公是非邪錯千故翻】愛憎擊于一時鬻獄賣官聚歛踰于巨萬【歛力贍翻】自是朝廷動息皆稟於邠岐南北司往往依附二鎮以邀恩澤有崔鋋王超者【鋋音蟬】為二鎮判官凡天子有所可否其不逞者輒訴於鋋超二人則教茂貞行瑜上章論之朝廷少有依違其辭語已不遜制復以茂貞為鳳翔節度使兼山南西道節度使守中書令於是茂貞盡有鳳翔興元洋隴秦等十五州之地【鳳翔本一鎮興元山南西道又一鎮洋州武定軍又一鎮秦隴天雄軍又一鎮史言李茂貞兼有四鎮之地】以徐彦若為御史大夫 戊戍以泉州刺史王潮為福建觀察使 舒州刺史倪章弃城走【倪章與蔡儔連兵儔已敗故章走】楊行密以李神福為舒州刺史 邠寧節度使守侍中兼中書令王行瑜求為尚書令韋昭度密奏太宗以尚書令執政遂登大位自是不以授人臣惟郭子儀以大功拜尚書令終身避讓行瑜安可輕議十一月以行瑜為太師賜號尚父仍賜鐵劵 十二月朱全忠請徙鹽鐵於汴州以便供軍崔昭緯以為全忠新破徐鄆兵力倍增若更判鹽鐵不可復制【復扶又翻】乃賜詔開諭之 汴將葛從周攻齊州刺史朱威朱瑄朱瑾引兵救之【按方鎮表齊州時屬平盧節度以後乾寧三年朱瓊降汴之事觀之則齊州已為兖鄆所并矣 考異曰編遺錄云十月乙未今從薛居正五代史梁紀】 初武安節度使周岳殺閔勗據潭州【見二百五十六卷僖宗光啓二年】邵州刺史鄧處訥聞而哭之諸將入弔處訥曰吾與公等咸受僕射大恩【閔勗檢校尚書右僕射欽化節度使以處訥刺邵州故言受恩路振九國志鄧處訥自唐乾符中從閔勗征蠻于安南勗帥潭署處訥邵州兵馬留後處昌呂翻】今周岳無狀殺之吾欲與公等竭一州之力為僕射報仇可乎【為于為翻】皆曰善於是訓卒厲兵八年乃結朗州刺史雷滿【雷滿與周岳有争肉之仇】共攻潭州克之斬岳自稱留後【鄧處訥甫得潭州而劉建鋒馬殷巳擬其後矣】<br />
<br />
  乾寧元年春正月乙丑朔赦天下改元 李茂貞入朝大陳兵自衛數日歸鎮 以李匡籌為盧龍節度使二月朱全忠自將撃朱瑄軍于魚山【魚山在鄆州須昌東阿兩縣之間】瑄與朱瑾合兵攻之兖鄆兵大敗死者萬餘人 以右散騎常侍鄭綮為禮部侍郎同平章事【綮康禮翻 考異曰舊傳云光化初為相恐誤北夢瑣言曰綮雖有詩名本無廊廟之望嘗典廬州吳王楊行密為本州步奏官因有遺闕而笞責之然其儒懦清慎弘農常重之昭宗時吳王雄據淮海朝廷務行姑息因盛言鄭公之德由是登庸中外驚駭太原兵至渭北天子震恐渇於攘却相國奏對請於文宣王諡號中加一哲字其不究時病率此類也按明年李克用舉兵至渭北綮巳罷相今從實錄新紀】綮好詼諧【好呼到翻】多為歇後詩譏嘲時事【歇後者叙所以為詩而歇後語不發】上以為有所藴手注班簿命以為相【班簿注在朝者姓名】聞者大驚堂吏往告之綮笑曰諸君大誤使天下更無人未至鄭綮吏曰特出聖意綮曰果如是奈人笑何既而賀客至綮搔首言曰歇後鄭五作宰相【鄭綮第五為歇後詩時謂之歇後鄭五體】時事可知矣累讓不獲乃視事 以邵州刺史鄧處訥為武安節度使 彰義節度使張鈞薨表其兄鐇為留後【時以涇州為彰義節度鐇甫袁翻】 三月黄州刺史吳討舉州降楊行密【黄州時隷鄂岳鄂岳武昌軍也按新書杜洪傳吳討鄂州永興縣民以土團帥起據黄州】 邢州城中食盡甲申李存孝登城謂李克用曰兒蒙王恩得富貴苟非困于讒慝安肯捨父子而從仇讎乎願一見王死不恨克用使劉夫人視之夫人引存孝出見克用存孝泥首謝罪曰兒粗立微勞存信逼兒失圖至此【粗坐五翻】克用叱之曰汝遺朱全忠王鎔書毁我萬端【遺唯季翻】亦存信教汝乎囚之歸于晉陽車裂於牙門 【考異曰太祖紀年錄先獲汴將鄧筠安康八軍吏劉藕子潞州所俘供奉官韓歸範皆與存孝連坐同日誅之騎將薛阿檀懼自刺按舊紀克用擒歸範尋遣歸因附表訴寃不聞復往晉陽也薛居正五代史鄧季筠傳後復自邢州逃歸汴紀年錄誤也存孝傳曰武皇出井陘將逼真定存孝面見王鎔陳軍機武皇暴怒誅先獲汴將安康八耳】存孝驍勇克用軍中皆莫及常將騎兵為先鋒所向無敵身被重鎧腰弓髀槊獨舞鐵撾陷陳萬人辟易【被皮義翻重直龍翻撾陟瓜翻陳讀曰陣下同辟讀曰闢易如字】每以二馬自隨馬稍乏【言馬稍疲而乏力也】就陳中易之出入如飛克用惜其才意臨刑諸將必為之請【為于偽翻下同】因而釋之既而諸將疾其能竟無一人言者既死克用為之不視事者旬日私恨諸將而於李存信竟無所譴又有薛阿檀者其勇與存孝相侔諸將疾之常不得志密與存孝通存孝誅恐事泄遂自殺自是克用兵勢浸弱而朱全忠獨盛矣【史言克用自翦羽翼故不競干汴】克用表馬師素為邢洺節度使 朱全忠遣軍將張從晦慰撫夀州從晦陵侮刺史江彦温而與諸將夜飲彦温疑其謀已明日盡殺在席諸將以書謝全忠而自殺軍中推其子從頊知軍州事全忠為之腰斬從晦 五月加鎮海節度使錢鏐同平章事 劉建鋒馬殷引兵至澧陵【澧當作醴醴陵在漢臨湘縣界後漢分為醴陵縣隋廢武德四年分長沙置醴陵縣屬潭州九域志在州東一百六十里】鄧處訥遣邵州指揮使蔣勛鄧繼崇將步騎三千守龍回關殷先至關下遣使詣勛勛等以牛酒犒師殷使說勛曰劉驤智勇兼人【觀後姚彦章說馬殷曰公與劉龍驤一體之人此必逸龍字說式芮翻】術家言當興翼軫間【翼軫楚荆州分長沙入軫十六度】今將十萬衆精鋭無敵而君以鄉兵數千拒之難矣【按新書蔣勛鄧繼崇皆邵州土豪所領之兵皆其土人故謂之鄉兵】不如先下之取富貴還鄉里不亦善乎勛等然之謂衆曰東軍許吾屬還【劉建鋒等兵從東來故蔣勛等謂之東兵】士卒皆懽呼弃旗幟鎧仗遁去建鋒令前鋒衣其甲張其旗趨潭州【衣於既翻趨七喻翻】潭人以為邵州兵還不為備建鋒徑入府處訥方宴擒斬之戊辰建鋒入潭州自稱留後 王建攻彭州城中人相食彭州内外都指揮使趙章出降【降戶江翻】王先成請築龍尾道屬於女牆【自城外築墱道陂陀而上屬于城上短垣其道前高後卑後塌于地若龍之垂尾然故謂之龍尾道屬之欲翻女牆即城上短垣所謂陴也】丙子西川兵登城楊晟猶帥衆力戰【帥讀曰率】刀子都虞侯王茂權斬之【文德元年楊晟得彭州王建攻彭州踰兩朞而後克亦憊矣】獲彭州馬步使安師建建欲使為將師建泣謝曰師建誓與楊司徒同生死不忍復戴日月【復扶又翻】惟速死為惠再三諭之不從乃殺之禮葬而祭之更趙章姓名王宗勉王茂權曰宗訓又更王釗名曰宗謹李綰姓名曰王宗綰【更工衡翻】辛卯中書侍郎同平早事鄭延昌罷為右僕射 朱瑄朱瑾求救於河東李克用遣騎將安福順及弟福慶福遷督精騎五百假道於魏度河應之 武昌節度使杜洪攻黄州【以吳討叛附楊行密也】楊行密遣行營都指揮使朱延夀等救之 六月甲午以宋州刺史張廷範為武寧節度使從朱全忠之請也【徐州先時改感化軍既屬朱全忠復為武寧軍】 蘄州刺史馮敬章邀擊淮南軍朱延夀攻蘄州不克【蘄州武昌巡屬也】 戊午以翰林學士承旨禮部尚書李谿同平章事方宣制水部郎中知制誥劉崇魯出班掠麻慟哭【強奪取之為掠】上召崇魯問其故對言谿姦邪依附楊復恭西門君遂得在翰林無相業恐危社稷谿竟罷為太子少傅谿鄘之孫也【李鄘見憲宗紀】上師谿為文崔昭緯恐谿為相分已權故使崇魯沮之谿十表自訟醜詆崇魯父符受贓枉法事覺自殺弟崇望與楊復恭深交崇魯庭拜田令孜為朱玫作勸進表【為于偽翻】乃云臣交結内臣何異抱贓唱賊且故事絁巾慘帶不入禁庭【絁巾絹巾也慘淺色絁式支翻】臣果不才崇魯自應上章論列【上時掌翻下上表同】豈於正殿慟哭【豈下有宜字或當字文意乃明】為國不祥無人臣禮乞正其罪詔停崇魯見任【見賢遍翻】谿猶上表不已乞行誅竄表數千言詬詈無所不至【詬古候翻又許候翻劉崇魯固可罪李谿亦褊矣當是時強藩遙制朝廷視當朝宰相特鬼朴耳李谿急於作相將以何為此其所以有都亭驛之禍也】 李克用大破吐谷渾殺赫連鐸擒白義誠 【考異曰舊紀六月壬辰克用攻陷雲州執赫連鐸以薛志勤守雲中按唐太祖紀年錄莊宗列傳薛居正五代史武皇紀皆云大順二年武皇拔雲州鐸奔吐谷渾誤也新紀六月赫連鐸與李克用戰于雲州死之太祖紀年錄十月討李匡籌師次新城邊兵願從者衆赫連鐸白義誠數敗至是窮蹙無歸自縶膝行詣於軍門太祖微數其罪命笞而脱之薛史武皇紀吐谷渾傳亦云鐸等來歸命笞而釋之薛志勤傳云王暉據雲州判討平之以志勤為大同防禦使與舊紀異唐末見聞錄六月收雲州處置赫連鐸活擒白義誠進兵幽州界巡檢迴府新紀盖據此今從之】 秋七月李茂貞遣兵攻閬州拔之楊復恭楊守亮楊守信帥其族黨犯圍走【楊復恭等奔閬州見上景福元年帥讀曰率】 禮部侍郎同平章事鄭綮自以不合衆望累表避位詔以太子少保致仕以御史大夫徐彦若為中書侍郎兼吏部尚書同平章事 綿州刺史楊守厚卒其將常再榮舉城降王建 楊復恭守亮守信將自商山奔河東至乾元【萬歲通天元年分商州豐陽置安業縣乾元元年更名乾元縣屬商州】遇華州兵獲之八月韓建獻于闕下斬于獨柳【復恭父子至是夷矣】李茂貞獻復恭遺守亮書訴致仕之由【遺唯季翻楊復恭致仕見上卷大順二年】云承天門乃隋家舊業【承天門長安太極宫南門隋文帝使宇文愷所營本名昭陽門唐改曰承天門故復恭云然】大姪但積粟訓兵勿貢獻【守亮復光養子故呼為姪】吾於荆榛中立夀王【上本封夀王】纔得尊位廢定策國老有如此負心門生天子 昭義節度使康君立詣晉陽謁李克用己未克用會諸將飲博酒酣克用語及李存孝流涕不已君立素與李存信善一言忤旨克用拔劒斫之囚於馬步司【唐末諸鎮皆於馬步司置獄今謂之兵馬司忤五故翻】九月庚申朔出之君立已死 【考異曰薛居正五代史李存孝既死武皇深惜之怒諸將無解愠者君立以一言忤旨武皇賜酖而殂唐末見聞錄曰八月三十日相公於左街宅夜飲行劒斫損昭義節度使康君立把送馬步司收禁至九月一日放出尋已身薨薛史賜酖恐是文飾其事】克用表雲州刺史薛志誠為昭義留後 冬十月封皇子祤為棣王禊為䖍王【祤兄羽翻禊胡計翻】禋為沂王禕為遂王 劉仁恭數因蓋寓獻策於李克用【數所角翻下同蓋古盍翻】願得兵萬人取幽州克用方攻邢州分兵數千欲納仁恭于幽州不克【此言是年十一月以前事】李匡籌益驕數侵河東之境【數所角翻】克用怒十一月大舉兵攻匡籌拔武州進圍新州【新州領永興山懷安龍門四縣史失其建置之始其地在媯州西北 考異曰唐太祖紀年錄十一月壬辰大軍拔截寇進收楊門九子戊戍下武州甲寅攻新州營于西北隅按十一月己未朔無壬辰戊戍甲寅紀年錄誤今從實錄】 以涇原留後張鐇為彰義節度使 朱全忠遣使至泗州陵慢刺史張諫諫舉州降楊行密【泗州本徐州巡屬自此遂為楊行密所有】行密遣押牙唐令回持茶萬餘斤如汴宋貿易【貿音茂】全忠執令回盡取其茶楊汴始有隙【為全忠攻行密張本】 十二月李匡籌遣大將將步騎數萬救新州李克用選精兵逆戰於段莊大破之【段莊在新州東南】斬首萬餘級生擒將校三百人以練之【紤充夜翻】徇于城下是夕新州降辛亥進攻媯州【宋白曰媯州東南至幽州二百八十里西南至蔚州二百四十里媯居為翻】壬子匡籌復發兵出居庸關【復扶又翻下同】克用使精騎當其前以疲之遣步將李存審自他道出其背夾擊之幽州兵大敗殺獲萬計甲寅李匡籌挈其族奔滄州義昌節度使盧彦威利其輜重妓妾遣兵攻之于景城殺之盡俘其衆【重直用翻僖宗光啓元年李全忠得幽州三世十年而滅景城漢成平縣唐屬滄州宋廢為鎮屬瀛州樂夀縣九域志樂夀在瀛州南六十里宋白曰幽州東南至滄州五百五十里妓渠綺翻 考異曰唐太祖紀年作匡儔今從新舊紀傳實錄】存審本姓苻宛丘人克用養以為子【薛史苻存審初從李罕之罕之為諸葛爽所逼出保懷州部下分散存審乃歸李克用】丙辰克用進軍幽州其大將請降匡籌素暗懦初據軍府兄匡威聞之謂諸將曰兄失弟得不出吾家亦復何恨但惜匡籌才短不能保守得及二年幸矣【景福二年匡籌得幽州至是僅及年】二 加匡國節度使王行約檢校侍中 吳討畏杜洪之逼【杜洪攻吳討見上五月】納印請代于楊行密行密以先鋒指揮使瞿章權知黄州【為瞿章為汴兵攻執張本】 是歲黄連洞蠻二萬圍汀州【黄連洞在汀州寧化縣南今潭飛磜即其地】福建觀察使王潮遣其將李承勲將萬人擊之蠻解去承勲追擊之至漿水口破之閩地略定潮遣僚佐巡州縣勸農桑定租税交好鄰道【好呼到翻】保境息民閩人安之 封州刺史劉謙卒子隱居喪于賀江【賀水源出賀州富川縣石龍亘州城合桂嶺水謂之賀江】土民百餘人謀亂隱一夕盡誅之嶺南節度使劉崇龜召補右都押牙兼賀水鎮使未幾表為封州刺史【劉隱始此】 義勝節度使董昌苛虐【詳考下卷浙東乃威勝節度又按新書方鎮表廣明三年升浙東道觀察為義勝軍節度光啓三年改威勝軍威勝為是】於常賦之外加歛數倍以充貢獻及中外饋遺【歛力贍翻遺唯季翻】每旬發一綱金萬兩銀五千鋌越綾萬五千匹他物稱是【稱尺證翻適相等也】用卒五百人或遇雨雪風水違程則皆死【唐制陸行之程馬日七十里步及驢五十里車三十里水行之程舟之重者泝河日三十里江四十里餘水四十五里空舟泝河四十里江五十里餘水六十里沿流之舟則輕重同制河日一百五十里江一百里餘水七十里轉運徵歛送納皆準程節其遲速其三峽砥柱之類不拘此限若遇風水淺不得行者即於隨近官司申牒驗記聽折半功不及是則為違程董昌蓋計日限程以至長安又不許以雨雪風水準折也】貢奉為天下最由是朝廷以為忠寵命相繼官至司徒同平章事爵隴西郡王昌建生祠于越州制度悉如禹廟【禹廟在越州會稽縣東南七里】命民間禱賽者【賽先代翻先祈福於神其後報祠謂之賽】無得之禹廟皆之生祠昌求為越王朝廷未之許昌不悦曰朝廷欲負我矣我累年貢獻無算而惜越王邪有諂之者曰王為越王曷若為越帝於是民間訛言時世將變競相帥填門喧譟請昌為帝【帥讀曰率】昌大喜遣人謝之曰天時未至時至我自為之其僚佐吳瑤都虞候李暢之等皆勸成之吏民獻謠䜟符瑞者不可勝紀【勝音升】其始賞之以錢數百緡既而獻者日多稍減至五百三百而已昌曰䜟云兔子上金牀此謂我也我生太歲在卯明年復在卯【復扶又翻】二月卯日卯時吾稱帝之秋也【為董昌僭號錢鏐舉兵討之張本】<br />
<br />
  資治通鑑卷二百五十九<br />
<br />
<史部,編年類,資治通鑑>  <br>
   </div> 

<script src="/search/ajaxskft.js"> </script>
 <div class="clear"></div>
<br>
<br>
 <!-- a.d-->

 <!--
<div class="info_share">
</div> 
-->
 <!--info_share--></div>   <!-- end info_content-->
  </div> <!-- end l-->

<div class="r">   <!--r-->



<div class="sidebar"  style="margin-bottom:2px;">

 
<div class="sidebar_title">工具类大全</div>
<div class="sidebar_info">
<strong><a href="http://www.guoxuedashi.com/lsditu/" target="_blank">历史地图</a></strong>  
<a href="http://www.880114.com/" target="_blank">英语宝典</a>  
<a href="http://www.guoxuedashi.com/13jing/" target="_blank">十三经检索</a> 
<br><strong><a href="http://www.guoxuedashi.com/gjtsjc/" target="_blank">古今图书集成</a></strong> 
<a href="http://www.guoxuedashi.com/duilian/" target="_blank">对联大全</a> <strong><a href="http://www.guoxuedashi.com/xiangxingzi/" target="_blank">象形文字典</a></strong> 

<br><a href="http://www.guoxuedashi.com/zixing/yanbian/">字形演变</a>  <strong><a href="http://www.guoxuemi.com/hafo/" target="_blank">哈佛燕京中文善本特藏</a></strong>
<br><strong><a href="http://www.guoxuedashi.com/csfz/" target="_blank">丛书&方志检索器</a></strong> <a href="http://www.guoxuedashi.com/yqjyy/" target="_blank">一切经音义</a>  

<br><strong><a href="http://www.guoxuedashi.com/jiapu/" target="_blank">家谱族谱查询</a></strong>  <strong><a href="http://shufa.guoxuedashi.com/sfzitie/" target="_blank">书法字帖欣赏</a></strong> 
<br>

</div>
</div>


<div class="sidebar" style="margin-bottom:0px;">

<font style="font-size:22px;line-height:32px">QQ交流群9:489193090</font>


<div class="sidebar_title">手机APP 扫描或点击</div>
<div class="sidebar_info">
<table>
<tr>
	<td width=160><a href="http://m.guoxuedashi.com/app/" target="_blank"><img src="/img/gxds-sj.png" width="140"  border="0" alt="国学大师手机版"></a></td>
	<td>
<a href="http://www.guoxuedashi.com/download/" target="_blank">app软件下载专区</a><br>
<a href="http://www.guoxuedashi.com/download/gxds.php" target="_blank">《国学大师》下载</a><br>
<a href="http://www.guoxuedashi.com/download/kxzd.php" target="_blank">《汉字宝典》下载</a><br>
<a href="http://www.guoxuedashi.com/download/scqbd.php" target="_blank">《诗词曲宝典》下载</a><br>
<a href="http://www.guoxuedashi.com/SiKuQuanShu/skqs.php" target="_blank">《四库全书》下载</a><br>
</td>
</tr>
</table>

</div>
</div>


<div class="sidebar2">
<center>


</center>
</div>

<div class="sidebar"  style="margin-bottom:2px;">
<div class="sidebar_title">网站使用教程</div>
<div class="sidebar_info">
<a href="http://www.guoxuedashi.com/help/gjsearch.php" target="_blank">如何在国学大师网下载古籍?</a><br>
<a href="http://www.guoxuedashi.com/zidian/bujian/bjjc.php" target="_blank">如何使用部件查字法快速查字?</a><br>
<a href="http://www.guoxuedashi.com/search/sjc.php" target="_blank">如何在指定的书籍中全文检索?</a><br>
<a href="http://www.guoxuedashi.com/search/skjc.php" target="_blank">如何找到一句话在《四库全书》哪一页?</a><br>
</div>
</div>


<div class="sidebar">
<div class="sidebar_title">热门书籍</div>
<div class="sidebar_info">
<a href="/so.php?sokey=%E8%B5%84%E6%B2%BB%E9%80%9A%E9%89%B4&kt=1">资治通鉴</a> <a href="/24shi/"><strong>二十四史</strong></a>&nbsp; <a href="/a2694/">野史</a>&nbsp; <a href="/SiKuQuanShu/"><strong>四库全书</strong></a>&nbsp;<a href="http://www.guoxuedashi.com/SiKuQuanShu/fanti/">繁体</a>
<br><a href="/so.php?sokey=%E7%BA%A2%E6%A5%BC%E6%A2%A6&kt=1">红楼梦</a> <a href="/a/1858x/">三国演义</a> <a href="/a/1038k/">水浒传</a> <a href="/a/1046t/">西游记</a> <a href="/a/1914o/">封神演义</a>
<br>
<a href="http://www.guoxuedashi.com/so.php?sokeygx=%E4%B8%87%E6%9C%89%E6%96%87%E5%BA%93&submit=&kt=1">万有文库</a> <a href="/a/780t/">古文观止</a> <a href="/a/1024l/">文心雕龙</a> <a href="/a/1704n/">全唐诗</a> <a href="/a/1705h/">全宋词</a>
<br><a href="http://www.guoxuedashi.com/so.php?sokeygx=%E7%99%BE%E8%A1%B2%E6%9C%AC%E4%BA%8C%E5%8D%81%E5%9B%9B%E5%8F%B2&submit=&kt=1"><strong>百衲本二十四史</strong></a>  <a href="http://www.guoxuedashi.com/so.php?sokeygx=%E5%8F%A4%E4%BB%8A%E5%9B%BE%E4%B9%A6%E9%9B%86%E6%88%90&submit=&kt=1"><strong>古今图书集成</strong></a>
<br>

<a href="http://www.guoxuedashi.com/so.php?sokeygx=%E4%B8%9B%E4%B9%A6%E9%9B%86%E6%88%90&submit=&kt=1">丛书集成</a> 
<a href="http://www.guoxuedashi.com/so.php?sokeygx=%E5%9B%9B%E9%83%A8%E4%B8%9B%E5%88%8A&submit=&kt=1"><strong>四部丛刊</strong></a>  
<a href="http://www.guoxuedashi.com/so.php?sokeygx=%E8%AF%B4%E6%96%87%E8%A7%A3%E5%AD%97&submit=&kt=1">說文解字</a> <a href="http://www.guoxuedashi.com/so.php?sokeygx=%E5%85%A8%E4%B8%8A%E5%8F%A4&submit=&kt=1">三国六朝文</a>
<br><a href="http://www.guoxuedashi.com/so.php?sokeytm=%E6%97%A5%E6%9C%AC%E5%86%85%E9%98%81%E6%96%87%E5%BA%93&submit=&kt=1"><strong>日本内阁文库</strong></a> <a href="http://www.guoxuedashi.com/so.php?sokeytm=%E5%9B%BD%E5%9B%BE%E6%96%B9%E5%BF%97%E5%90%88%E9%9B%86&ka=100&submit=">国图方志合集</a> <a href="http://www.guoxuedashi.com/so.php?sokeytm=%E5%90%84%E5%9C%B0%E6%96%B9%E5%BF%97&submit=&kt=1"><strong>各地方志</strong></a>

</div>
</div>


<div class="sidebar2">
<center>

</center>
</div>
<div class="sidebar greenbar">
<div class="sidebar_title green">四库全书</div>
<div class="sidebar_info">

《四库全书》是中国古代最大的丛书,编撰于乾隆年间,由纪昀等360多位高官、学者编撰,3800多人抄写,费时十三年编成。丛书分经、史、子、集四部,故名四库。共有3500多种书,7.9万卷,3.6万册,约8亿字,基本上囊括了古代所有图书,故称“全书”。<a href="http://www.guoxuedashi.com/SiKuQuanShu/">详细>>
</a>

</div> 
</div>

</div>  <!--end r-->

</div>
<!-- 内容区END --> 

<!-- 页脚开始 -->
<div class="shh">

</div>

<div class="w1180" style="margin-top:8px;">
<center><script src="http://www.guoxuedashi.com/img/plus.php?id=3"></script></center>
</div>
<div class="w1180 foot">
<a href="/b/thanks.php">特别致谢</a> | <a href="javascript:window.external.AddFavorite(document.location.href,document.title);">收藏本站</a> | <a href="#">欢迎投稿</a> | <a href="http://www.guoxuedashi.com/forum/">意见建议</a> | <a href="http://www.guoxuemi.com/">国学迷</a> | <a href="http://www.shuowen.net/">说文网</a><script language="javascript" type="text/javascript" src="https://js.users.51.la/17753172.js"></script><br />
  Copyright &copy; 国学大师 古典图书集成 All Rights Reserved.<br>
  
  <span style="font-size:14px">免责声明:本站非营利性站点,以方便网友为主,仅供学习研究。<br>内容由热心网友提供和网上收集,不保留版权。若侵犯了您的权益,来信即刪。scp168@qq.com</span>
  <br />
ICP证:<a href="http://www.beian.miit.gov.cn/" target="_blank">鲁ICP备19060063号</a></div>
<!-- 页脚END --> 
<script src="http://www.guoxuedashi.com/img/plus.php?id=22"></script>
<script src="http://www.guoxuedashi.com/img/tongji.js"></script>

</body>
</html>
