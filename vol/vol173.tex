資治通鑑卷一百七十三 宋 司馬光 撰

胡三省 音註

陳紀七|{
	起彊圉作噩盡屠維大淵獻凡三年}


高宗宣皇帝中之下

太建九年春正月乙亥朔齊太子恒即皇帝位|{
	恒戶登翻}
生八年矣改元承光大赦尊齊主為太上皇帝皇太后為太皇太后皇后為太上皇后以廣寧王孝珩為太宰|{
	珩音行}
司徒莫多婁敬顯領軍大將軍尉相願|{
	尉紆勿翻}
謀伏兵千秋門|{
	千秋門鄴宫西門}
斬高阿那肱立廣寧王孝珩會阿那肱自它路入朝不果|{
	朝直遥翻}
孝珩求拒周師謂阿那肱等曰朝廷不賜遣擊賊豈不畏孝珩反邪孝珩若破宇文邕|{
	周主諱邕}
遂至長安反亦何預國家事以今日之急猶如此猜忌邪|{
	邪音耶}
高韓恐其為變出孝珩為滄州刺史|{
	高韓謂高阿那肱韓長鸞地形志熙平二年分瀛冀二州置滄州治餘安縣城}
相願拔佩刀斫柱歎曰大事去矣知復何言|{
	復扶又翻}
齊主使長樂王尉世辯|{
	長樂郡王五代志信都郡長樂縣舊置長樂郡樂音洛尉紆勿翻}
帥千餘騎覘周師出滏口|{
	帥讀曰率騎奇寄翻覘丑亷翻又丑艶翻滏音釡}
登高阜西望遥見羣烏飛起謂是西軍旗幟即馳還比至紫陌橋不敢回顧世辯粲之子也|{
	西軍旗幟皆黑齊人時恇懼望見烏飛以為周師已至馳歸不敢囘顧懼其及也紫陌橋在鄴城外尉粲與高歡同起於北鎮爾雅曰大陸曰阜廣雅山無石曰阜幟昌志翻比必寐翻}
於是黃門侍郎顔之推中書侍郎薛道衡侍中陳德信等勸上皇往河外募兵|{
	齊制門下省侍中給事黃門侍郎各六人中書省侍郎四人河外謂大河之外王者内京師而外諸夏齊都鄴在河北故謂河南為河外}
更為經略若不濟南投陳國從之道衡孝通之子也|{
	薛孝通始從賀拔岳後因入朝遂留仕於鄴}
丁丑太皇太后太上皇后自鄴先趣濟州|{
	濟州治碻磝城趣七喻翻濟子禮翻}
癸未幼主亦自鄴東行己丑周師至紫陌橋 辛卯上祭北郊 壬辰周師至鄴城下癸巳圍之燒城西門齊人出戰周師奮擊大破之齊上皇從百騎東走|{
	騎奇寄翻}
使武衛大將軍慕容三藏守鄴宫|{
	後齊循魏制武衛將軍副貮左右衛將軍掌左右廂所主朱華閣以外階從三品加大者進等藏徂浪翻}
周師入鄴齊王公以下皆降|{
	降戶江翻}
三藏猶拒戰周主引見禮之|{
	見賢遍翻}
拜儀同大將軍三藏紹宗之子也|{
	慕容紹宗始從爾朱氏後事高歡父子}
領軍大將軍漁陽鮮于世榮齊高祖舊將也|{
	將即亮翻}
周主先以馬腦酒鍾遺之|{
	馬腦石似玉寶石也今作碼碯先息薦翻遺唯季翻}
世榮得即碎之周師入鄴世榮在三臺前鳴鼓不輟周人執之世榮不屈乃殺之周主執莫多婁敬顯數之曰汝有死罪三前自晉陽走鄴攜妾棄母不孝也|{
	自晉陽走鄴見上卷上年數所拒翻舊所貝翻走音奏}
外為偽朝戮力内實通啟於朕不忠也|{
	為于偽翻朝直遥翻}
送款之後猶持兩端不信也用心如此不死何待遂斬之使將軍尉遲勤追齊主 |{
	考異曰北齊書勤作剛今從周書尉紆勿翻}
甲午周主入鄴齊國子博士長樂熊安生博通五經|{
	晉武帝咸寧四年初立國子學置國子祭酒博士各一人後齊置博士五人黄帝有熊氏一曰出於楚鬻熊之後以名為氏樂音洛}
聞周主入鄴遽令掃門家人怪而問之安生曰周帝重道尊儒必將見我俄而周主幸其家不聽拜親執其手引與同坐賞賜甚厚給安車駟馬以自隨又遣小司馬唐道和|{
	後周之制六官七命自小冢宰至小司徒小宗伯小司馬小司寇小司空皆上大夫七命}
就中書侍郎李德林宅宣旨慰諭曰平齊之利唯在於爾引入宫|{
	宫即鄴宫時周主居之}
使内史宇文昂訪問齊朝風俗政教人物善惡|{
	朝直遥翻}
即留内省三宿乃歸|{
	内省即齊之門下省}
乙未齊上皇渡河入濟州|{
	濟子禮翻}
是日幼主禪位於大丞相任城王湝|{
	任音壬湝戶皆翻又音皆}
又為湝詔尊上皇為無上皇幼主為宋國天王|{
	齊氏於傾危之際不應改國號為宋宋國當作宗國}
令侍中斛律孝卿送禪文及璽紱於瀛州|{
	齊制天子六璽受命璽在六璽之外紱印組也古者韍如蔽又裳繡為两已相背形謂之黻此紱直以繫璽而已璽斯氏翻紱音弗}
孝卿即詣鄴|{
	以璽紱歸周}
周主詔去年大赦所未及之處皆從赦例|{
	去年周克晉陽大赦山東河南河北之地尚為齊守今既克鄴凡齊之境内赦所未及之地令皆從去年赦例}
齊洛州刺史獨孤永業有甲士三萬聞晉州陷請出兵擊周奏寢不報永業憤慨又聞并州陷乃遣子須逹請降於周|{
	降戶江翻下同}
周以永業為上柱國封應公|{
	此去年事也因齊亡叙之於此應國公用古邗晉應韓之應以封之}
丙申周以越王盛為相州總管|{
	後魏置相州於鄴東魏都鄴改為司州以其京畿之地倣漢晉之制而置司州也周既平齊復為相州列于諸州相息亮翻}
齊上皇留胡太后於濟州使高阿那肱守濟州關|{
	濟州城北有碢磝津故關}
覘侯周師|{
	覘丑亷翻又丑艶翻}
自與穆后馮淑妃幼主韓長鸞鄧長顒等數十人奔青州|{
	顒魚容翻}
使内參田鵬鸞西出參伺動静|{
	參候也伺相吏翻}
周師獲之問齊主何在紿云已去|{
	紿徒亥翻誑言也}
計當出境|{
	謂出齊境也}
周人疑其不信捶之每折一支辭色愈厲竟折四支而死|{
	捶止橤翻折而設翻}
上皇至青州即欲入陳而高阿那肱密召周師約生致齊主屢啟云周師尚遠已令燒斷橋路上皇由是淹留自寛周師至關阿那肱即降之周師奄至青州上皇囊金繫於鞍與后妃幼主等十餘騎南走|{
	騎奇寄翻}
己亥至南鄧村尉遲勤追及盡擒之并胡太后送鄴|{
	先已擒胡太后於濟州今并齊主送鄴齊天保元年更禪歲在庚午四主二十八年而亡}
庚子周主詔故斛律光崔季舒等宜追加贈謚并為改葬|{
	斛律光死見一百七十一卷太建四年崔季舒等死見五年謚神至翻為于偽翻}
子孫各隨䕃叙錄|{
	自漢以來將相公卿皆得保任子弟若孫為官所謂門䕃者也}
家口田宅沒官者並還之周主指斛律光名曰此人在朕安得至鄴辛丑詔齊之東山南園三臺並可毁撤瓦木諸物可用者悉以賜民山園之田各還其主|{
	東山南園三臺皆高氏遊宴之地撤直列翻}
二月壬午上耕籍田|{
	籍而亦翻}
丙午周主宴從官將士於齊太極殿頒賞有差|{
	從才用翻將即亮翻下同}
丁未高緯至鄴|{
	已為俘囚不復書齊主緯于貴翻}
周主降階以賓禮見之齊廣寧王孝珩至滄州以五千人會任城王湝於信都|{
	冀州治信都湝自河間進兵至信都珩音行任音壬湝古皆翻}
共謀匡復召募得四萬餘人周主使齊王憲柱國楊堅擊之令高緯為手書招湝湝不從憲軍至趙州|{
	魏孝昌二年分定相二州置殷州治廣阿後改為趙州}
湝遣二諜覘之|{
	諜徒協翻覘丑亷翻又丑艶翻}
候騎執以白憲憲集齊舊將遍示之|{
	齊舊將從憲軍者集以示諜以攜湝軍之心騎奇寄翻}
謂曰吾所争者大不在汝曹今縱汝還仍充吾使|{
	使疏吏翻}
乃與湝書曰足下諜者為候騎所拘軍中情實具諸執事|{
	謂諜者當能具言之}
戰非上計無待卜疑守乃下策或未相許已勒諸軍分道並進相望非遠憑軾有期|{
	左傳城濮之役楚子玉遣使請戰於晉侯曰請與君之士戱君憑軾而望之故引以為言兵車之軾高三尺三寸立而憑之}
不俟終日所望知機也|{
	昜大傳曰君子見幾而作不俟終日引以諭湝使速降}
憲至信都湝陳於城南以拒之湝所署領軍尉相願詐出略陳遂以衆降|{
	鄴城之破相願盖奔瀛州湝因署為領軍尉紆勿翻陳讀曰陣降戶江翻}
相願湝心腹也衆皆駭懼湝殺相願妻子明日復戰|{
	復扶又翻}
憲擊破之俘斬三萬人執湝及廣寧王孝珩憲謂湝曰任城王何苦至此湝曰下官神武皇帝之子兄弟十五人幸而獨存逢宗社顛覆無愧墳陵|{
	聚土曰墳陵大阜也墳陵猶言山陵}
憲壯之命歸其妻子又親為孝珩洗瘡傅藥禮遇甚厚|{
	為于偽翻}
孝珩歎曰自神武皇帝以吾諸父兄弟無一人至四十者命也|{
	文襄死於盗手年二十九顯祖年三十一濟南王年十七孝昭年二十七武成年三十二其餘多不得良死}
嗣君無獨見之明宰相非柱石之寄|{
	嗣祥吏翻}
恨不得握兵符受斧钺展我心力耳|{
	史言湝孝珩志節可憐五代志後齊命將出征則授鼔旗於朝皇帝陳法駕服兖冕至廟拜於太祖徧告訖降就中階引上將操钺授柯曰從此上至天將軍制之又操斧授柯曰從此下至泉將軍制之將軍既受斧钺對曰國不可從外理軍不可從中制臣既受命有鼓旗斧钺之威願無一言之命於臣帝曰苟利社稷將軍裁之將軍就車載斧钺而出皇帝推轂度閫曰從此以外將軍制之}
齊王憲善用兵多謀略得將士心齊人憚其威聲多望風沮芻牧不擾軍無私焉|{
	將即亮翻沮在呂翻}
周主以齊降將封輔相為北朔州總管北朔州齊之重鎮|{
	降戶江翻北朔州控禦突厥齊以為重鎮}
士卒驍勇|{
	驍堅堯翻}
前長史趙穆等謀執輔相迎任城王湝於瀛州不果|{
	前長史齊官長知两翻相息亮翻}
乃迎定州刺史范陽王紹義紹義至馬邑自肆州以北二百八十餘城皆應之|{
	五代志博陵郡舊置定州魏置肆州治九原六鎮叛亂寄治樓煩郡之秀容縣其北即齊北朔州界}
紹義與靈州刺史袁洪猛引兵南出欲取并州至新興而肆州已為周守|{
	地形志魏太延二年置薄骨律鎮孝昌二年置靈州東西分治其地屬西魏天平中東魏復置靈州寄治汾州隰城縣界五代志鴈門郡繁峙縣東魏置武州寄治城中後齊改為北靈州新興漢魏古郡名以五代志考之與肆州皆在樓煩郡秀容縣宋白曰唐之嵐州古新興郡為于偽翻}
前隊二儀同以所部降周|{
	二儀同前隊之將二人官皆儀同降戶江翻}
周兵擊顯州|{
	地形志魏永安中置顯州治汾州六壁城五代志鴈門郡崞縣東魏置廓州後齊改北顯州周兵所撃即此}
執刺史陸瓊復攻拔諸城|{
	復扶又翻}
紹義還保北朔州周東平公神舉將兵逼馬邑|{
	神舉即宇文神舉}
紹義戰敗北奔突厥猶有衆三千人紹義令曰欲還者從其意於是辭去者大半突厥佗鉢可汗常謂齊顯祖為英雄天子以紹義重踝似之|{
	厥九勿翻佗徒何翻可苦曷翻汗音寒重直龍翻踝戶瓦翻腿两旁曰内外踝}
甚見愛重凡齊人在北者悉以隸之於是齊之行臺州鎮唯東雍州行臺傅伏營州刺史高寶寧不下|{
	傅伏以永橋之功遷東雍州行臺五代志絳郡後魏置東雍州遼西郡置營州治和龍城雍於用翻}
其餘皆入於周凡得州五十郡一百六十二縣三百八十戶三百三萬二千五百|{
	梁太宗大寶元年齊顯祖受魏禪五主二十七年而亡齊所有司冀趙義懷㴝建東雍汾西汾晉南朔并肆靈顯恒朔定瀛幽東燕北燕營南營安青濟光膠徐仁睢兖北徐南青海東楚潼東徐洛鄭陽宋梁南兖西兖北荆襄豫東廣秦西楚揚南頴北建羅合江和共六十州而東廣已下十州時已為陳故止言五十州 考異曰隋書地理志云州九十七郡一百六十縣一百六十五今從周書}
高寶寧者齊之踈屬有勇略久鎮和龍甚得夷夏之心|{
	夏戶雅翻}
周主於河陽幽青南兖豫徐北朔定置揔管府相并二州各置宫及六府官|{
	河陽縣屬懷州河内郡地臨河津實重鎮也幽州治薊青州治益都南兖治譙豫治汝南徐治彭城北朔治馬邑定治中山或都會之地或守禦之要也故皆置摠管府摠管猶魏晉之都督也相并二州皆有齊舊宫及省故仍置宫若别都然置六府官以代省也六府官蓋倣長安六官之府未必備官也}
周師之克晉陽也|{
	克晉陽見上卷上年}
齊使開府儀同三司紇奚永安求救於突厥比至齊已亡佗鉢可汗處永安於吐谷渾使者之下|{
	紇奚虜複姓魏收宮氏志北方諸姓有紇奚氏比必寐翻及也處昌呂翻吐如字或上鶻翻谷音浴使疏吏翻}
永安言於佗鉢曰今齊國已亡永安何用餘生欲閉氣自絶恐天下謂大齊無死節之臣乞賜一刀以顯示遠近佗鉢嘉之贈馬七十匹而歸之梁主入朝於鄴|{
	梁臣於周以周平齊故入朝朝直遥翻下同}
自秦兼天下無朝覲之禮至是始命有司草具其事致積致餼設九儐九介受享於廟三公三孤六卿致食勞賓還贄致享皆如古禮|{
	積子賜翻餼許既翻勞力到翻鄭玄曰每積有牢醴米禾芻薪又曰大禮饔餼也左傳居則具一日之積杜預曰芻米菜薪詩傳曰牲腥曰餼或曰饋客生食及芻米曰餼儐主副也導主以行禮者也介賓副也輔賓以行禮者也五代志曰梁王之朝周入畿大冢宰命有司致積其餼五牢米九十筥䤈醢各三十五甕酒十八壺米禾各五十車薪芻各百車既至大司空設九儐以致館梁王束帛乘馬設九介以待之禮成而出明日王朝受享於廟既致享大冢宰又命公一人玄冕乘車陳九儐以束帛乘馬致食於賓及賓之從各有差致食訖又命公一人弁服乘車執贄設九儐以勞賓王設九介迎於門外明日朝服乘車還贄于公公皮弁迎於大門授䞇受贄並於堂之中楹又明日王朝服設九介乘車以見於公事畢公致享明日三孤一人又執贄勞于梁王明日王還贄又明日王見三孤如三公明日卿一人又執贄勞王王見卿又如三孤於是三公三孤六卿又各餼賓並屬官之長為使牢米束帛同三公儐必刃翻勞力到翻}
周主與梁主宴酒酣|{
	酣戶甘翻}
周主自彈琵琶梁主起舞曰陛下既親撫五絃臣何敢不同百獸周主大悦賜賚甚厚|{
	舜彈五絃之琴夔曰於予擊石拊石百獸率舞梁主以舜况周主故悦賚來代翻}
乙卯周主自鄴西還三月壬午周詔山東諸軍各舉明經幹治者二人若奇才異術卓爾不羣者不拘此數|{
	時周分置諸州總管以撫鎮山東治軍政故曰諸軍}
周主之擒尉相貴也|{
	擒尉相貴見上卷上年}
招齊東雍州刺史傳伏伏不從齊人以伏為行臺右僕射周主既克并州|{
	克并州亦見上卷上年}
復遣韋孝寛招之|{
	韋孝寛鎮勲州與東雍州接境故使招之復扶又翻}
令其子以上大將軍武鄉公告身|{
	以古武鄉郡封為公也石勒置武鄉郡凡授官爵皆給以符謂之告身五代志馮翊郡華隂縣西魏改武鄉置武鄉郡周當以此封傳伏}
及金馬腦二酒鍾賜伏為信伏不受謂孝寛曰事君有死無貳此兒為臣不能竭忠為子不能盡孝人所讎疾願速斬之以令天下周主自鄴還至晉州遣高阿那肱等百餘人臨汾水召伏伏出軍隔水見之|{
	汾水逕晉絳二州之間東雍州在絳州界故隔水}
問至尊今何在阿那肱曰已被擒矣|{
	被皮義翻}
伏仰天大哭帥衆入城於聽事前北面哀號良久然後降|{
	帥讀曰率聽與廳同毛晃曰聽事治官處漢晉皆作聽事六朝以來乃始加广音他經翻號戶刀翻降戶江翻}
周主見之曰何不早下伏流涕對曰臣三世為齊臣食齊禄不能自死羞見天地周主執其手曰為臣當如此乃以所食羊肋骨賜伏|{
	肋盧則翻脅肋}
曰骨親肉疎所以相付遂引使宿衛授上儀同大將軍敕之曰若亟與公高官|{
	亟紀力翻}
恐歸附者心動努力事朕勿憂富貴他日又問前救河隂得何賞|{
	救河隂事見上卷七年}
對曰蒙一轉授特進永昌郡公|{
	勲級曰轉轉張戀翻}
周主謂高緯曰朕三年教戰决取河隂正為傳伏善守|{
	為于偽翻}
城不可動遂歛軍而退公當時賞功何其薄也夏四月乙巳周主至長安置高緯於前列其王公於後車輿旗幟器物以次陳之備大駕|{
	秦大駕屬車八十一乘漢遵用之備千乘萬騎晉之盛也大駕鹵簿見於志為尤詳開皇中大駕十二乘法駕半之其後大駕用三十六法駕用十二周氏雖設六官置司輅之職以掌公車之政以隋制參之大駕鹵簿必不能如漢晉之盛幟昌志翻}
布六軍奏凱樂|{
	周官王師大獻則令奏凱樂注云大獻獻捷於祖凱樂獻功之樂}
獻俘於太廟觀者皆稱萬歲戊申封高緯為温公齊之諸王三十餘人皆受封爵周主與齊君臣飲酒令温公起舞高延宗悲不自持屢欲仰藥其傳婢禁止之周主以李德林為内史上士|{
	後周之制内史屬春官中大夫五命下犬夫四命上士三命}
自是詔誥格式及用山東人物並以委之帝從容謂羣臣曰我常日唯聞李德林名復見其為齊朝作詔書移檄|{
	從七容翻復扶又翻為于偽翻朝直遥翻}
正謂是天上人豈言今日得其驅使神武公紇豆陵毅對曰|{
	神武郡公後魏置神武郡於神武川隋為神武縣屬馬邑郡紇豆陵毅本姓竇唐宰相世系表曰竇本竇融之後以竇武之難亡入鮮卑拓拔部使居南境號沒鹿囘部世為部落大人及勒後魏穆帝命為紇豆陵氏至曾孫巖從孝文帝徙洛陽遂為河南洛陽人復為竇氏宇文復代北舊姓又復為紇豆陵氏}
臣聞麒麟鳳凰為王者瑞可以德感不可力致麒麟鳳凰得之無用豈如德林為瑞且有用哉帝大笑曰誠如公言 己巳周主享太廟 五月丁丑周以譙王儉為大冢宰庚辰以公亮為大司徒鄭公逹奚震為大宗伯梁公侯莫陳芮為大司馬應公獨孤永業為大司寇鄭公韋孝寛為大司空己丑周主祭方丘|{
	周制方丘在國隂六里之郊以其先炎帝神農氏配}
詔以路寢會義崇信含仁雲和思齊諸殿皆晉公護專政時所為事窮壯麗有踰清廟|{
	清廟者倣周祀文王之廟而為之也毛傳曰清廟者祭有清明之德者之宫也}
悉可毁撤彫斵之物並賜貧民繕造之宜務從卑朴又詔并鄴諸堂殿壯麗者凖此|{
	并鄴諸堂殿齊氏所營也}


臣光曰周高祖可謂善處勝矣|{
	處昌呂翻}
他人勝則益奢高祖勝而愈儉

六月丁卯周主東廵秋七月丙戍幸洛州|{
	洛州治洛陽}
八月壬寅議定權衡度量頒之於四方|{
	量音亮}
初魏虜西凉之人|{
	西凉謂河西自沮渠氏據河西稱凉王宋文帝元嘉十六年魏太武帝擊而虜之}
沒為隸戶齊氏因之仍供厮役|{
	厮息移翻養也役也使也賤也蘇林曰厮取薪者也韋昭曰析薪曰厮今或讀從詵入聲}
周主滅齊欲施寛惠詔曰罪不及嗣古有定科|{
	書大禹謨臯陶曰罰不及嗣孔傳云父子罪不相及}
雜役之徒獨異常憲|{
	憲法也}
一從罪配百代不免罰既無窮刑何以措凡諸雜戶悉放為民自是無復雜戶甲子鄭州獲九尾狐|{
	此時鄭州盖猶在長社}
已死獻其骨周主曰瑞應之來必彰有德若五品時叙四海和平乃能致此|{
	此稽瑞應圖而言也孔氏書傳五品謂五常}
今無其時恐非實録命焚之九月戊寅周制庶人已上唯聽衣綢綿綢絲布圓綾紗絹綃葛布等九種|{
	衣於既翻綢與紬同直由翻大絲繒也綿綢紡綿為之今淮人能織綿綢緊厚耐久服絲布以絲裨布縷織之今謂之兼絲布圓綾土綾也亦謂之花絹紗方目紗也絹吉掾翻縑也細絲繒綃相邀翻生絲繒葛葛越宜夏服布緝麻若紵為之種章勇翻}
餘悉禁之朝祭之服不拘此制|{
	朝直遥翻}
冬十月戊申周主如鄴 上聞周人滅齊欲争徐兖|{
	此言禹迹徐兖二州之地禹貢曰海岱及淮惟徐州濟河惟兖州周之九州青州得沂泗淮三水兖州得大野無復徐州矣今之徐州春秋宋地左傳圍宋彭城是也秦屬泗水郡漢屬沛郡後分立楚國後置徐州自是之後徐州專治彭城矣}
詔南兖州刺史司空吳明徹督諸軍伐之以其世子戎昭將軍惠覺攝行州事明徹軍至呂梁周徐州總管梁士彦帥衆拒戰|{
	帥讀曰率}
戊午明徹擊破之士彦嬰城自守明徹圍之帝鋭意以為河南指麾可定中書通事舍人蔡景歷諫曰師老將驕不宜過窮遠略|{
	魏黄初中中書置通事郎晉初置舍人通事江左令舍人通事謂之通事舍人掌呈奏案又掌詔命陳氏得國國之政事並由中書省有中書舍人五人分掌二十一局事各當尚書諸曹並為上司揔國内機要尚書唯聽受而已將即亮翻}
帝怒以為沮衆|{
	沮在呂翻}
出為豫章内史未行有飛章劾景歷在省贓汙狼籍坐免官削爵土|{
	飛者不知其所自來也盖出於上意劾戶槩翻又音戶得翻}
周改葬德皇帝於冀州|{
	宇文肱者宇文泰之父也從鮮于脩禮攻定州戰死於唐河武城初追謚德皇帝其地在齊未得改葬平齊之後乃得改葬于冀州}
周主服縗|{
	縗倉囘翻}
哭於太極殿百官素服 周人誣温公高緯與宜州刺史穆提婆謀反并其宗族皆賜死衆人多自陳無之高延宗獨攘袂泣而不言以椒塞口而死|{
	塞悉則翻}
唯緯弟仁英以清狂仁雅以瘖疾得免|{
	漢張敞奏言昌邑王賀清狂不惠蘇林曰凡狂者隂陽脉盡濁今此人不狂似狂故言清狂或曰色理清徐而心不慧故曰清狂清狂如今白癡也瘖於今翻瘂也}
徙於蜀其餘親屬不殺者散配西土|{
	西土謂長安西邊州郡}
皆死於邊裔周主以高湝妻盧氏賜其將斛斯徵|{
	湝戶皆翻又音皆將即亮翻下同}
盧氏蓬首垢面長齋不言笑徵放之乃為尼|{
	盧氏山東高門史言其能守節長齎者依佛教茹蔬素不食葷肉尼女夷翻女僧}
齊后妃貧者至以賣燭為業 十一月壬申周立皇子衍為道王|{
	道古國名春秋有江黄道栢皇子當作皇孫衍周太子之長子此有可疑者後注屢及之通鑑一百七十一卷太建五年六月書周皇孫衍生}
兌為蔡王 癸酉周遣上大將軍王軌將兵救徐州初周人敗齊師於晉州乘勝逐北齊人所棄甲仗未

暇收歛|{
	事見上卷八年敗補邁翻}
稽胡乘間竊出|{
	間古莧翻}
並盜而有之仍立劉蠡升之孫沒鐸為主|{
	劉蠡升為高歡所滅見一百五十七卷梁武帝大同元年}
號聖武皇帝改元石平周人既克關東|{
	謂克齊也}
將討稽胡議欲窮其巢穴齊王憲曰步落稽種類既多|{
	種章勇翻}
又山谷險絶王師一舉未可盡除且當翦其魁首餘加慰撫周主從之以憲為行軍元帥督諸軍討之|{
	行軍元帥始此帥所類翻}
至馬邑分道俱進沒鐸分遣其黨天柱守河東穆支守河西據險以拒之|{
	此西河離石之河東河西也}
憲命譙王儉撃天柱滕王逌擊穆支|{
	逌以周翻}
並破之斬首萬餘級趙王招撃沒鐸禽之餘衆皆降|{
	降戶江翻}
周詔自永熙三年以來東土之民掠為奴婢|{
	後魏孝武帝永熙三年西入關自是宇文氏高氏交兵互相侵掠得其民口各以為奴婢}
及克江陵之日良人沒為奴婢者|{
	梁世祖承聖三年江陵破事見一百六十五卷}
並放為良又詔後宫唯置妃二人世婦三人御妻三人此外皆減之周主性節儉常服布袍寢布被後宫不過十餘人每行兵親在行陳|{
	行陳上戶剛翻下讀曰陣}
步涉山谷人所不堪撫將士有恩而明察果斷|{
	斷丁亂翻}
用法嚴峻由是將士畏威而樂為之死|{
	將即亮翻樂音洛為于偽翻}
己亥晦日有食之 周初行刑書要制羣盗贓一匹及正長隱五丁若地頃以上皆死|{
	隋因周制制人五家為保保有長保五為閭閭四為族皆有正畿外置里正比閭正黨長比族正以相檢察所謂正長也百畝為頃長知两翻}
十二月戊申新作東宫成太子徙居之 庚申周主如并州徙并州軍民四萬戶於關中戊辰廢并州宫及六府|{
	是年春周置并州宫及六府}
高寶寧自黄龍上表勸進於高紹義|{
	黄龍即和龍今之黄龍府上時掌翻}
紹義遂稱皇帝改元武平以寶寧為丞相突厥佗鉢可汗舉兵助之|{
	可從刋入聲汗音寒}


十年春正月壬午周主幸鄴辛卯幸懷州|{
	懷州治河内郡野王自此以後周陳之君書如書幸雜出其間未悉義例所安}
癸巳幸洛州置懷州宫 二月甲辰周譙孝王儉卒 丁巳周主還長安 吳明徹圍周彭城環列舟艦於城下攻之甚急|{
	艦戶黯翻}
王軌引兵輕行據淮口|{
	淮口清水入淮之口即清口也}
結長圍以鐵鎻貫車輪數百沈之清水|{
	沈持林翻酈道元曰清水即泗水之别名}
以遏陳船歸路軍中忷懼|{
	忷許勇翻}
譙州刺史蕭摩訶言於明徹曰聞王軌始鎖下流其兩端築城今尚未立公若見遣擊之彼必不敢相拒水路未斷賊勢不堅彼城若立則吾屬必為虜矣明徹奮髯曰搴旗陷陳將軍事也長筭遠略老夫事也摩訶失色而退|{
	史言明徹驕而愎諫以致敗髯而占翻搴起䖍翻拔取也陳讀曰陣}
一旬之間水路遂斷周兵益至諸將議破堰拔軍以舫載馬而去馬主裴子烈曰若破堰下船船必傾倒不如先遣馬出 |{
	考異曰南史作馬明主今從陳書馬主馬軍主也堰於建翻舫府妄翻並两船也倒都皓翻}
時明徹苦背疾甚篤蕭摩訶復請曰今求戰不得進退無路若濳軍突圍未足為恥願公帥步卒乘馬轝徐行摩訶領鐵騎數千驅馳前後必當使公安逹京邑|{
	京邑謂建康觀摩訶此言亦知軍退後周師繼至必不能守淮南復扶又翻帥讀曰率轝攷字書皆無此字唯類篇有之音羊茹切舁車也今言乘馬轝則當讀與輿字同從平聲騎奇寄翻下同}
明徹曰弟之此策乃良圖也然步軍既多吾為總督必須身居其後相帥兼行|{
	帥讀曰率下同}
弟馬軍宜速在前不可遲緩摩訶因帥馬軍夜發甲子明徹决堰乘水勢退軍冀以入淮至清口水勢漸微舟艦並礙車輪不復得過王軌引兵圍而蹙之衆潰明徹為周人所執將士三萬并器械輜重皆沒於周|{
	將即亮翻重直用翻}
蕭摩訶以精騎八十居前突圍衆騎繼之|{
	騎奇寄翻}
比旦逹淮南|{
	淮水南岸也比必寐翻}
與將軍任忠周羅㬋獨全軍得還|{
	任音壬還音旋又如字}
初帝謀取彭汴以問五兵尚書毛喜|{
	彭汴謂彭城汴水之地五兵尚書以掌中兵外兵别兵都兵騎兵各官}
對曰淮左新平邊民未輯周氏始吞齊國難與爭鋒且棄舟艥之工|{
	艥與楫同}
踐車騎之地|{
	徐兖之地四平車騎便於馳突踐慈演翻騎奇寄翻}
去長就短非吳人所便臣愚以為不若安民保境寢兵結好|{
	好呼到翻}
斯久長之術也及明徹敗帝謂喜曰卿言驗於今矣即日召蔡景歷復以為征南諮議參軍|{
	亦以其言驗也}
周主封吳明徹為懷德公|{
	懷德郡公五代志巴東郡武寧縣後周置南都郡源陽縣尋改郡曰懷德縣曰武寧}
位大將軍|{
	其朝列於大將軍無職事也}
明徹憂憤而卒|{
	卒子恤翻}
乙丑周以越王盛為大冢宰 三月戊辰周於蒲州置宫|{
	五代志河東郡後魏曰秦州後周改蒲州因蒲坂以名州也}
廢同州及長春二宫|{
	同州治馮翊宇文泰輔魏多居同州其後受魏禪遂以同州置别宫長春宫在朝邑馮翊之屬縣也是宫盖亦宇文所置}
甲戍周主初服常冠以皁紗全幅向後襆髪仍裁為

四脚|{
	今之幞頭始此制微有不同耳杜佑曰後漢末王公卿士以幅巾為雅用全幅皁而向後襆髪謂之頭巾俗人因號為襆頭後周武帝因裁幅巾為四脚襆與幞同房玉翻皁才早翻}
丙子命中軍大將軍開府儀同三司淳于量為大都督|{
	陳制中軍大將軍品第二秩中二千石開府儀同三司品第一其秩則萬石矣}
總水陸諸軍事鎮西將軍孫瑒都督荆郢諸軍|{
	瑒雉杏翻又音暢}
平北將軍樊毅都督清口上至荆山緣淮諸軍寧遠將軍任忠都督壽陽新蔡霍州諸軍以備周|{
	寧遠將軍梁置陳制擬官品第五此新蔡在弋陽郡界五代志梁置平高新蔡新城三郡於殷城後齊置新蔡郡於固始二縣皆屬弋陽任音壬}
乙酉大赦 壬辰周改元宣政 夏四月庚申突厥寇周幽州殺掠吏民|{
	厥九勿翻}
戊午樊毅遣軍渡淮北對清口築城壬戍清口城不守 五月己丑周高祖帥諸軍伐突厥|{
	周主以是役殂於軍中故書其廟號帥讀曰率}
遣柱國原公姬願|{
	原古國名}
東平公神舉等將兵五道俱入|{
	將即亮翻又音如字領也}
癸巳帝不豫留止雲陽宫|{
	五代志京兆郡雲陽縣後周置雲陽郡盖亦置别宫於此}
丙申詔停諸軍驛召宗師宇文孝伯赴行在所|{
	後周置宗師之官蓋掌諸宗室杜佑曰宗師屬天官中大夫也五命小宗師下大夫四命宇文孝伯時留長安故驛召之天子所至為行在所}
帝執其手曰吾自量必無濟理|{
	量音良}
以後事付君是夜授孝伯司衛上大夫總宿衛兵|{
	後周之制凡上大夫皆六命}
又令馳驛入京鎮守以備非常六月丁酉朔帝疾甚還長安是夕殂年三十六|{
	還從宣翻又音如字殂蘇乎翻}
戊戍太子即位尊皇后阿史那氏為皇太后|{
	阿史那氏天和三年娶于突厥者也}
宣帝初立即逞奢欲大行在殯曾無戚容|{
	在戚而有嘉容魯昭公所以不終也}
捫其杖痕大罵曰死晚矣|{
	捫以手撫摸也杖痕為太子時受杖之痕}
閲視高祖宫人逼為淫欲|{
	周武帝未祔廟而書高祖者史筆也}
超拜吏部下大夫鄭譯為開府儀同大將軍内史中大夫委以朝政|{
	鄭譯有寵事始上卷八年朝直遥翻}
己未葬武皇帝於孝陵廟號高祖既葬詔内外公除帝及六宫皆議即吉京兆郡丞樂運上疏以為葬期既促事訖即除太為汲汲帝不從|{
	樂運擢京兆郡丞見一百七十一卷五年自丁酉至己未二十三日而葬太速矣上時掌翻}
帝以齊煬王憲屬尊望重忌之|{
	齊王於周主叔父也屬尊出將入相著功名其望重謚法好内遠禮曰煬憲豈有是哉周主殺之加以惡謚耳煬余亮翻}
謂宇文孝伯曰公能為朕圖齊王|{
	為于偽翻}
當以其官相授孝伯叩頭曰先帝遺詔不許濫誅骨肉齊王陛下之叔父功高德茂社稷重臣陛下若無故害之臣又順旨曲從則臣為不忠之臣陛下為不孝之子矣帝不懌由是踈之乃與開府儀同大將軍于智鄭譯等密謀之使智就宅候憲因告憲有異謀甲子帝遣宇文孝伯語憲|{
	語牛倨翻}
欲以憲為太師|{
	太師三師之首}
憲辭讓又使孝伯召憲曰晩與諸王俱入既至殿門憲獨被引進|{
	被皮義翻}
帝先伏壯士於别室至即執之憲自辯理帝使于智證憲憲目光如炬與智相質|{
	質證也驗也}
或謂憲曰以王今日事勢何用多言憲曰死生有命寧復圖存|{
	復扶又翻}
但老母在堂恐留兹恨耳|{
	言既誣以異謀恐罪及其母也}
因擲笏於地遂縊之|{
	縊於賜翻經也絞也}
帝召憲僚屬使證成憲罪參軍勃海李綱誓之以死終無撓辭|{
	撓奴教翻曲也}
有司以露車載憲尸而出|{
	車無帷盖曰露車}
故吏皆散唯李綱撫棺號慟躬自瘞之|{
	綱在憲府先此未有聞焉而能臨難盡節於所事隋唐之間汔能自持有以也夫號戶刀翻瘞於計翻}
哭拜而去又殺上大將軍王興上開府儀同大將軍獨孤熊開府儀同大將軍豆盧紹|{
	隋書曰豆盧本姓慕容燕北地王精之後中山敗歸魏北人謂歸義為豆盧因氏焉}
皆素與憲親善者也帝既誅憲而無名|{
	無罪以加之為無名古所謂無名之師亦言無罪而加之兵也}
乃云與興等謀反時人謂之伴死|{
	伴蒲旱翻}
以于智為柱國封齊公以賞之 閠月乙亥周主立妃楊氏為皇后|{
	楊堅之女也}
辛巳周以趙王招為太師陳王純為太傅 齊范陽

王紹義聞周高祖殂以為得天助幽州人盧昌期起兵據范陽|{
	五代志幽州治薊城涿縣舊置范陽郡}
迎紹義紹義引突厥兵赴之周遣柱國東平公神舉將兵討昌期|{
	將即亮翻}
紹義聞幽州總管出兵在外欲乘虛襲薊|{
	薊音計}
神舉遣大將軍宇文恩將四千人救之半為紹義所殺會神舉克范陽擒昌期紹義聞之素衣舉哀還入突厥高寶寧帥夷夏數萬騎救范陽|{
	還從宣翻又音如字厥九勿翻帥讀曰率夏戶雅翻騎奇寄翻}
至潞水|{
	水經注鮑丘水出禦夷北塞中俗謂之大榆河南過潞縣為潞水}
聞昌期死還據和龍秋七月周主享太廟丙午祀圜丘|{
	按五代志周祭圓丘及南郊並正月上辛今用七月丙午非舊制}
庚戍周以小宗伯斛斯徵為大宗伯壬戍以亳州揔管楊堅為上柱國大司馬|{
	五代志譙郡後魏置南兖州後周置揔管府後改曰亳州亳旁各翻}
癸亥周主尊所生母李氏為帝太后|{
	嫡母阿史那氏既尊為皇太后又尊生母為帝太后}
八月丙寅周主祀西郊|{
	五代志後周五郊壇其崇及去國如其行之數其方俱百二十步内壝皆半之}
壬申如同州以大司徒公亮為安州總管上柱國長孫覧為大司徒楊公王誼為大司空|{
	長知两翻}
丙戍以永昌公椿為大司寇 九月乙巳立方明壇於婁湖戊申以揚州刺史始興王叔陵為王官伯臨盟百官|{
	周禮司盟掌盟載之法凡邦國有疑會同則掌其盟約之載北面詔明神既盟則貳之鄭玄注曰有疑不協也明神神之明察者謂日月山川也覲禮加方明於壇上所以依之也詔之者讀其載書以告之也貮之者寫其副當以授六官陳祥道曰諸侯覲於天子為宫方三百步四門壇十有二尋深四尺加方明于其上覲禮方明者木也方四尺設六色東方青南方赤西方白北方黑上玄下黃設六玉上圭下璧南方璋西方琥東方圭北方璜鄭氏曰方明者上下四方神明之象也會同而盟明神監之六色象其神六玉以禮之上宜以蒼璧下宜以黄琮而不以者則上下之神非天地之至貴者也設玉者刻其木以著之王官伯者古者天子盟諸侯使天子之老涖之如春秋踐土之盟王子虎盟諸侯於王庭是之謂王官伯時彭城喪師陳人通國上下揺心故為是盟}
庚戍周主封其弟元為荆王 周主詔諸應拜者皆以三拜成禮|{
	三拜成禮用夷禮也}
甲寅上幸婁湖誓衆乙卯分遣大使以盟誓班下四方上下相警戒|{
	使疏吏翻班下戶嫁翻班布也}
冬周主還長安以大司空王誼為襄州總管|{
	五代志襄陽郡江左僑置雍州西魏改曰襄州}
戊子以尚書左僕射陸繕為尚書僕射 十一月突厥寇周邊圍酒泉殺掠吏民|{
	五代志張掖郡福禄縣舊置酒泉郡}
十二月甲子周以畢王賢為大司空 己丑周以河陽總管滕王逌為行軍元帥帥衆入寇|{
	逌音由元帥所類翻帥衆之帥讀曰率}


十一年春正月癸巳周主受朝於露門|{
	露門當作路門路大也盖周之外朝也程泰之作雍錄以唐大明宫丹鳳門太極宫承天門皆為唐之外朝蓋識此意朝直遥翻}
始與羣臣服漢魏衣冠|{
	以此知後周之君臣前此盖胡服也}
大赦改元大成置四輔官以大冢宰越王盛為大前疑相州總管蜀公尉遲迴為大右弼申公李穆為大左輔大司馬隨公楊堅為大後承|{
	尉紆勿翻記文王世子虞夏商周皆有師保有疑丞設四輔及三公周主倣此以置官}
周主之初立也以高祖刑書要制為太重而除之|{
	周行刑書要制見上九年}
又數行赦宥|{
	數所角翻下同}
京兆郡丞樂運上疏|{
	上時掌翻}
以為虞書所稱眚災肆赦|{
	眚所景翻}
謂過誤為害當緩赦之呂刑云五刑之疑有赦謂刑疑從罰罰疑從免也謹尋經典未有罪無輕重溥天大赦之文大尊豈可數施非常之惠以肆姦宄之惡乎|{
	大尊猶言至尊也}
帝不納既而民輕犯法又自以奢淫多過失惡人規諫|{
	惡烏路翻}
欲為威虐懾服羣下|{
	懾之涉翻}
乃更為刑經聖制 |{
	考異曰周帝紀行刑經聖制在八月案隋元巖傳樂運之諫因巖納說得免及王軌之死巖遂廢于家今運書已有更嚴前制之語然則行刑經在軌死前也}
用法益深大醮於正武殿告天而行之|{
	五代志道家齋法夜中於星辰之下陳設酒脯䴵餌幣物歷祀天皇太一祀五星列宿為書燒香陳讀云奏上天曹名之為醮醮子肖翻}
密令左右伺察羣臣|{
	伺相吏翻}
小有過失輒行誅譴又居喪纔踰年輒恣聲樂魚龍百戲常陳殿前|{
	五代志齊武平中有魚龍爛漫俳優侏儒山車巨象拔井種瓜殺馬剝驢等奇怪異端百有餘物名為百戲時鄭譯有寵於周主徵齊散樂並會京師為之蓋秦角抵之流也}
累日繼夜不知休息多聚美女以實後宫增置位號不可詳録遊宴沈湎或旬日不出|{
	沈持林翻湎彌兖翻毛晃曰沈湎飲酒齊其色韓詩飲酒閉門不出客曰湎}
羣臣請事者皆因宦者奏之於是樂運輿櫬詣朝堂陳帝八失|{
	櫬初現翻空棺朝直遥翻下同}
其一以為大尊比來事多獨斷|{
	比毗至翻斷丁亂翻}
不參諸宰輔與衆共之其二搜美人以實後宫儀同以上女不許輒嫁貴賤同怨其三大尊一入後宫數日不出所須聞奏多附宦官其四下詔寛刑未及半年更嚴前制其五高祖斵雕為朴崩未踰年而遽窮奢麗其六徭賦下民以奉俳優角抵其七上書字誤者即治其罪|{
	上時掌翻治直之翻}
杜獻書之路其八玄象垂誡不能諮諏善道|{
	玄象天象也日月星辰在天成象諏子于翻又子侯翻}
脩布德政若不革兹八事臣見周廟不血食矣|{
	犧牲之薦為血食}
帝大怒將殺之朝臣恐懼莫有救者内史中大夫洛陽元巖歎曰臧洪同死人猶願之|{
	陳容願與臧洪同死事見六十一卷漢獻帝興平三年}
况比干乎|{
	以樂運忠諫况之比干}
若樂運不免吾將與之俱斃乃詣閤請見|{
	見賢遍翻}
曰樂運不顧其死欲以求名陛下不如勞而遣之|{
	勞力到翻}
以廣聖度帝頗感悟明日召運謂曰朕昨夜思卿所奏實為忠臣賜御食而罷之 癸卯周立皇子闡為魯王|{
	按李延壽北史静帝諱衍後改名闡觀此則九年周立皇子行為道王自是高祖之子邪}
甲辰周主東廵以許公宇文善為大宗伯戊午周主至洛陽立魯王闡為皇太子|{
	闡昌善翻}
二月癸亥上耕籍田|{
	籍在亦翻}
周下詔以洛陽為東京發山東諸州兵治洛陽宫|{
	治直之翻}
常役四萬人徙相州六府於洛陽|{
	周置相州六府見上九年相息亮翻}
周徐州總管王軌聞鄭譯用事自知及禍謂所親曰吾昔在先朝實申社稷至計今日之事斷可知矣|{
	朝直遥翻斷丁亂翻}
此州控帶淮南鄰近彊寇欲為身計易如反掌|{
	彊寇謂陳易以䜴翻}
但忠義之節不可虧違况荷先帝厚恩|{
	荷下可翻}
豈可以獲罪嗣主遽忘之邪|{
	邪音耶}
止可於此待死冀千載之後知吾此心耳|{
	載作亥翻}
周主從容問譯曰我脚杖痕誰所為也對曰事由烏丸軌|{
	受杖事見上卷八年王軌蓋賜姓烏丸氏故稱之從千容翻}
宇文孝伯因言軌捋須事|{
	宇文孝伯何為出此言也欲自求免死耶然終於不免也捋須事見同上}
帝使内史杜慶信就州殺軌元巖不肯署詔御正中大夫顔之儀切諫|{
	武成元年置御正四人}
帝不聽巖進繼之脱巾頓顙三拜三進|{
	顙蘇朗翻須也}
帝曰汝欲黨烏丸軌邪巖曰臣非黨軌正恐濫誅失天下之望帝怒使閹豎摶其面|{
	手擊也}
軌遂死巖亦廢於家遠近知與不知皆為軌流涕|{
	為于偽翻}
之儀之推之弟也|{
	顔之推先仕於齊齊亡入周}
周主之為太子也上柱國尉遲運為宫正|{
	此太子宫正也尉紆勿翻}
數進諫不用|{
	數所角翻}
又與王軌宇文孝伯宇文神舉皆為高祖所親待太子疑其同毁已及軌死運懼私謂孝伯曰吾徒必不免禍為之奈何孝伯曰今堂上有老母地下有武帝|{
	武帝即高祖也}
為臣為子知欲何之|{
	之往也}
且委質事人本狥名義諫而不入死焉可逃|{
	質如字焉於乾翻}
足下若為身計宜且遠之|{
	遠於願翻}
於是運求出為秦州總管|{
	天水郡舊秦州}
他日帝託以齊王憲事讓孝伯曰公知齊王謀反何以不言對曰臣知齊王忠於社稷為羣小所譛言必不用所以不言且先帝付囑微臣|{
	囑之欲翻託也}
唯令輔導陛下今諫而不從實負顧託以此為罪是所甘心帝大慙俛首不語|{
	令力丁翻俛音免}
命將出賜死於家|{
	將引也領也}
時宇文神舉為并州刺史帝遣使就州酖殺之|{
	使疏吏翻}
尉遲運至秦州亦以憂死 周罷南伐諸軍突厥佗鉢可汗請和於周|{
	厥九勿翻可從刋入聲汗音寒}
周主以趙王招女為千金公主妻之|{
	妻七細翻}
且命執送高紹義佗鉢不從 辛巳周宣帝傳位於太子闡大赦改元大象自稱天元皇帝所居稱天臺冕二十四旒車服旂鼓皆倍於前王之數皇帝稱正陽宫置納言御正諸衛等官|{
	保定四年改宗伯為納言此納言似隋官之納言為門下省長官諸衛等宫左右宫伯小宫伯左右中侍左右侍左右前侍左右後侍左右騎侍左右宗侍左右庶侍左右勲侍左右武伯小武伯左右武賁左右旅賁左右射聲左右驍騎左右羽林左右游擊也}
皆準天臺尊皇太后為天元皇太后天元既傳位驕侈彌甚務自尊大無所顧憚國之儀典率情更變|{
	更工衡翻}
每對臣下自稱為天用樽彛珪瓚以飲食|{
	周禮有六尊六彛尊有罍而彛有舟鄭玄曰彛亦尊也欎鬯曰彛彛法也言為尊之法鄭衆曰於圭頭為器可以挹鬯祼祭謂之瓚瓚藏旱翻}
令羣臣朝天臺者致齋三日清身一日|{
	朝直遥翻}
既自比上帝不欲羣臣同已常自帶綬冠通天冠加金附蟬顧見侍臣弁上有金蟬及王公有綬者並令去之|{
	五代志古者君臣佩玉綬者所以貫佩相承受也又上下施韍如蔽五覇之後戰兵不息佩非兵器韍非戰儀於是解去佩韍留其繫襚而已韍佩既廢秦乃以采組連結於襚又謂之綬周制皇帝組綬以蒼以青以朱以黄以白以玄以纁以紅以紫以緅以碧以綠十有二色諸王及三公九色自黄以下王侯以下以差降殺通天冠古制高九寸正豎頂少斜却乃直下鐵為卷梁前有展筩冠前加金博山述加金附蟬者乃侍中常侍所冠武弁也史皆言天元之率意自尊綬音受冠通之冠古玩翻令力丁翻去羌呂翻}
不聽人有天高上大之稱|{
	稱尺證翻}
官名有犯皆改之改姓高者為姜|{
	齊太公之後食采於高因以為氏本姜姓也使改從本姓}
九族稱高祖者為長祖|{
	長知两翻}
又令天下車皆以渾木為輪|{
	渾戶本翻}
禁天下婦人不得施粉黛|{
	粉以傅面黛以填額畫眉}
自非宫人皆黄眉墨糚每召侍臣議論唯欲興造變革未嘗言及政事游戲無常出入不節羽儀仗衛晨出夜還|{
	還從宣翻又音如字}
陪侍之官皆不堪命自公卿以下常被楚撻每捶人皆以百二十為度謂之天杖|{
	被皮義翻下同捶止橤翻}
其後又加至二百四十宫人内職亦如之后妃嬪御雖被寵幸亦多杖背於是内外恐怖|{
	怖蒲布翻}
人不自安皆求苟免莫有固志重足累息|{
	重足而立屏氣積而不敢息重直龍翻累力委翻}
以逮於終 戊子周以越王盛為太保尉遲迴為大前疑代王逹為大右弼辛卯徙鄴城石經於洛陽|{
	尉紆勿翻漢靈帝時蔡邕立石經於太學講堂前一曰立於鴻都門魏正始中又立古篆隸三字石經高澄遷之於鄴周今復徙之洛陽}
詔河陽幽相豫亳青徐七總管並受東京六府處分|{
	相息亮翻處昌呂翻分扶問翻}
三月庚申天元還長安大陳軍伍親擐甲胄|{
	擐音宦}
入自青門静帝備法駕以從|{
	青門漢長安城東出南來第三門也門色青故名青門法駕次於大駕從才用翻}
夏四月壬戍朔立妃朱氏為天元帝后后吳人本出寒微生静帝長於天元十餘歲|{
	長知两翻}
疎賤無寵以静帝故特尊之乙巳周主祠太廟壬午大醮於正武殿五月以襄國郡為趙國濟南郡為陳國武當安富二郡為越國上黨郡為代國新野郡為滕國邑各萬戶|{
	食邑各有實土安富當作安福五代志淅陽郡武當縣舊制武當郡又安福縣置安福郡南陽郡之新野縣舊曰棘陽置新野郡濟子禮翻}
令趙王招陳王純越王盛代王逹滕王逌並之國隨公楊堅私謂大將軍汝南公慶曰天元實無積德視其相貌壽亦不長|{
	相息亮翻}
又諸藩微弱各令就國曾無深根固本之計羽翮既翦何能及遠哉|{
	觀楊堅此言豈有簒心哉然堅處猜虐之朝而發此言其免者盖幸也}
慶神舉之弟也 突厥寇周并州|{
	厥九勿翻并卑經翻}
六月周發山東諸民脩長城|{
	修齊所築長城也齊築長城見一百六十六卷梁敬帝太平元年}
秋七月庚寅周以楊堅為大前疑柱國司馬消難為大後承|{
	難乃旦翻}
辛卯初用大貨六銖錢|{
	五代志梁武帝鑄錢肉好周郭文曰五銖而又别鑄除其肉郭謂之女錢二品並行百姓或私以古錢交易有直百五銖五銖女錢太平百錢定平一百五銖雉錢五銖對文等號輕重不一天子頻下詔書非新鑄二種之錢並不許用而私用益甚至普通中乃議盡罷銅錢更鑄鐵錢人以鐵賤昜得並皆私鑄大同已後所在鐵錢如丘山錢陌所在不等至于末年陌益少以三十五為陌陳初承喪亂之後鐵錢不行始梁末又有两柱錢及鵝眼錢两柱重而鵝眼輕雜而用之其價同私家多鎔錢又間以錫鉄兼以粟帛為貨至文帝天嘉五年改鑄五銖初出當鵝眼之十至是又鑄大貨六銖以一當五銖之十後還當一人皆不以為便未幾帝崩遂廢六銖而行五銖}
丙申周納司馬消難女為正陽宫皇后|{
	静帝后也難乃旦翻}
己酉周尊天元帝太后李氏為天皇太后壬子改天元皇后朱氏為天皇后立妃元氏為天右皇后陳氏為天左皇后凡四后云元氏開府儀同大將軍晟之女|{
	晟成正翻}
陳氏大將軍山提之女也|{
	陳山提爾朱兆蒼頭也見一百五十六卷梁武帝中大通五年}
八月庚申天元如同州 丁卯上閱武於大壯觀命都督任忠帥步騎十萬陳於玄武湖都督陳景帥樓艦五百出瓜步江振旅而還|{
	觀古玩翻釋名觀者於上觀望帥讀曰率騎奇寄翻陳讀曰陣樓艦即樓船两面施重板列戰格故謂之樓艦艦戶暗翻還音旋又如字帝自喪師於彭城設近陳以耀武所謂不足者示人有餘也}
壬申周天元還長安甲戍以陳山提元晟並為上柱國|{
	二人者皆后父也晟丞正翻}
戊寅上還宫豫章内史南康王方泰在郡秩滿縱火延燒邑居因行暴掠驅錄富人徵求財賄|{
	錄收也}
上閲武方泰當從|{
	從才用翻}
啟稱母疾不行而微服往民間淫人妻為州所錄|{
	州謂揚州也}
又帥人仗抗拒傷禁司|{
	仗兵仗禁司掌禁防姦非者帥讀曰率}
為有司所奏上大怒下方泰獄|{
	下戶嫁翻}
免官削爵土尋而復舊 壬午周以上柱國畢王賢為太師郇公韓業為大左輔九月乙卯以酆王貞為大冢宰以鄖公孝寛為行軍元帥|{
	郇音荀酆音豐鄖音云帥所類翻}
帥行軍總管公亮郕公梁士彦寇淮南|{
	帥讀曰率}
仍遣御正杜杲禮部薛舒來聘 冬十月壬戍周天元幸道會苑大醮以高祖配醮初復佛像及天尊像|{
	周毁經像見上卷六年}
天元與二像俱南面坐大陳雜戲令長安士民縱觀|{
	令力丁翻}
甲戍以尚書僕射陸繕為尚書左僕射十一月辛卯大赦 周韋孝寛分遣公亮自安陸

攻黄城梁士彦攻廣陵|{
	分两路進兵以攻淮南此廣陵在新息}
甲午士彦至肥口|{
	肥水入淮之口}
乙未周天元如温湯|{
	即驪山温湯在驪山西北十道志曰温泉有三所其一處即皇堂石井後周宇文護所造}
戊戍周軍進圍壽陽周天元如同州 詔開府儀同三司南兖州刺史淳于量為上流水軍都督中領軍樊毅都督北討諸軍事左衛將軍任忠都督北討前軍事前豐州刺史臯文奏帥步騎三千趣陽平郡|{
	五代志建安郡陳置豐州江都郡安宜縣梁置陽平郡臯姓臯陶之後左傳有越大夫臯如帥讀曰率下同騎奇寄翻下同趣七喻翻下同}
壬寅周天元還長安 癸卯任忠帥步騎七千趣秦郡|{
	趣七喻翻}
丙午仁威將軍魯廣逹帥衆入淮|{
	梁置五德將軍智仁勇信嚴五威將軍代舊征虜五武將軍代舊冠軍}
是日樊毅將水軍二萬自東關入焦湖|{
	九域志巢湖亦謂之焦湖樊毅水軍欲自此湖向合肥將即亮翻又音如字領也焦子小翻}
武毅將軍蕭摩訶帥步騎趣歷陽|{
	武毅將軍亦梁置下於五德二班}
戊申韋孝寛拔壽陽公亮拔黃城梁士彦拔廣陵辛亥又取霍州|{
	水經注蕭齊立霍州治灊縣天柱山五代志廬江郡霍山縣梁置霍州}
癸丑以揚州刺史始興王叔陵為大都督總水步衆軍 丁巳周鑄永通萬國錢一當千與五行大布並行|{
	五代志周令五行大布與五銖三品並用}
十二月戊午周天元以災異屢見舍仗衛如天興宫百官上表勸復寢膳|{
	見賢遍翻舍讀曰捨上時掌翻}
甲子還宫御正武殿集百官及宫人外命婦|{
	外命婦五命以上官之妻也}
大列伎樂|{
	伎渠綺翻}
初作乞寒胡戲|{
	杜佑曰乞寒者本西國外藩之樂也新唐書康國之俗十一月鼓舞乞寒以水交潑為樂其戲流入中國}
乙丑南北兖晉三州|{
	五代志不載北兖州所治同安郡梁置豫州後改曰晉州後齊改江州陳復曰晉州}
及盱眙山陽陽平馬頭秦歷陽沛北譙南梁等九郡民並自拔還江南 |{
	考異曰陳紀九郡作九州盖字誤五代志江都永福縣舊曰沛梁置涇城東陽二郡及涇州陳廢州併二郡為沛郡全椒縣梁置北譙郡南梁郡自宋志有之不知其實土所在梁天監二年馮道根以南梁太守戍阜陵蓋自是為實土}
周又取譙北徐州|{
	譙州治渦陽在譙郡山桑縣北徐州置於琅邪郡}
自是江北之地盡沒於周周天元如洛陽親御驛馬日行三百里四皇后及文

武侍衛數百人並乘馹以從|{
	馹人質翻亦驛馬也從才用翻}
仍令四后方駕齊驅|{
	方駕並駕也}
或有先後輒加譴責人馬頓仆相及於道 癸酉遣平北將軍沈恪電威將軍裴子烈鎮南徐州開遠將軍徐道奴鎮柵口|{
	電威開遠將軍品並第七秩六百石柵口柵江口}
前信州刺史楊寶安鎮白下戊寅以中領軍樊毅都督荆郢巴武四州水陸諸軍事|{
	五代志南郡公安縣陳置荆州江夏郡置郢州巴陵郡置巴州武陵郡置武州}
己卯周天元還長安 貞毅將軍汝南周法尚|{
	貞毅將軍班五德將軍之下}
與長沙王叔堅不相能叔堅譖之於上云其欲反上執其兄定州刺史法僧|{
	五代志永安郡麻城縣陳置定州其地時已沒於周}
發兵將擊法尚法尚奔周周天元以為儀同大將軍順州刺史|{
	五代志漢東郡順義縣西魏置順州}
上遣將軍樊猛濟江撃之法尚遣部曲督韓朗詐降於猛曰法尚部兵不願降北|{
	降戶江翻}
人皆竊議欲叛還|{
	還從宣翻又音如字}
若得軍來自當倒戈猛以為然引兵急趨之|{
	趨七喻翻又音如字}
法尚陽為畏懼自保江曲|{
	江曲江水之曲}
戰而偽走伏兵邀之猛僅以身免沒者幾八千人|{
	幾居依翻}


資治通鑑卷一百七十三
