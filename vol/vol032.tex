<!DOCTYPE html PUBLIC "-//W3C//DTD XHTML 1.0 Transitional//EN" "http://www.w3.org/TR/xhtml1/DTD/xhtml1-transitional.dtd">
<html xmlns="http://www.w3.org/1999/xhtml">
<head>
<meta http-equiv="Content-Type" content="text/html; charset=utf-8" />
<meta http-equiv="X-UA-Compatible" content="IE=Edge,chrome=1">
<title>資治通鑒_33-資治通鑑卷三十二_33-資治通鑑卷三十二</title>
<meta name="Keywords" content="資治通鑒_33-資治通鑑卷三十二_33-資治通鑑卷三十二">
<meta name="Description" content="資治通鑒_33-資治通鑑卷三十二_33-資治通鑑卷三十二">
<meta http-equiv="Cache-Control" content="no-transform" />
<meta http-equiv="Cache-Control" content="no-siteapp" />
<link href="/img/style.css" rel="stylesheet" type="text/css" />
<script src="/img/m.js?2020"></script> 
</head>
<body>
 <div class="ClassNavi">
<a  href="/24shi/">二十四史</a> | <a href="/SiKuQuanShu/">四库全书</a> | <a href="http://www.guoxuedashi.com/gjtsjc/"><font  color="#FF0000">古今图书集成</font></a> | <a href="/renwu/">历史人物</a> | <a href="/ShuoWenJieZi/"><font  color="#FF0000">说文解字</a></font> | <a href="/chengyu/">成语词典</a> | <a  target="_blank"  href="http://www.guoxuedashi.com/jgwhj/"><font  color="#FF0000">甲骨文合集</font></a> | <a href="/yzjwjc/"><font  color="#FF0000">殷周金文集成</font></a> | <a href="/xiangxingzi/"><font color="#0000FF">象形字典</font></a> | <a href="/13jing/"><font  color="#FF0000">十三经索引</font></a> | <a href="/zixing/"><font  color="#FF0000">字体转换器</font></a> | <a href="/zidian/xz/"><font color="#0000FF">篆书识别</font></a> | <a href="/jinfanyi/">近义反义词</a> | <a href="/duilian/">对联大全</a> | <a href="/jiapu/"><font  color="#0000FF">家谱族谱查询</font></a> | <a href="http://www.guoxuemi.com/hafo/" target="_blank" ><font color="#FF0000">哈佛古籍</font></a> 
</div>

 <!-- 头部导航开始 -->
<div class="w1180 head clearfix">
  <div class="head_logo l"><a title="国学大师官网" href="http://www.guoxuedashi.com" target="_blank"></a></div>
  <div class="head_sr l">
  <div id="head1">
  
  <a href="http://www.guoxuedashi.com/zidian/bujian/" target="_blank" ><img src="http://www.guoxuedashi.com/img/top1.gif" width="88" height="60" border="0" title="部件查字,支持20万汉字"></a>


<a href="http://www.guoxuedashi.com/help/yingpan.php" target="_blank"><img src="http://www.guoxuedashi.com/img/top230.gif" width="600" height="62" border="0" ></a>


  </div>
  <div id="head3"><a href="javascript:" onClick="javascript:window.external.AddFavorite(window.location.href,document.title);">添加收藏</a>
  <br><a href="/help/setie.php">搜索引擎</a>
  <br><a href="/help/zanzhu.php">赞助本站</a></div>
  <div id="head2">
 <a href="http://www.guoxuemi.com/" target="_blank"><img src="http://www.guoxuedashi.com/img/guoxuemi.gif" width="95" height="62" border="0" style="margin-left:2px;" title="国学迷"></a>
  

  </div>
</div>
  <div class="clear"></div>
  <div class="head_nav">
  <p><a href="/">首页</a> | <a href="/ShuKu/">国学书库</a> | <a href="/guji/">影印古籍</a> | <a href="/shici/">诗词宝典</a> | <a   href="/SiKuQuanShu/gxjx.php">精选</a> <b>|</b> <a href="/zidian/">汉语字典</a> | <a href="/hydcd/">汉语词典</a> | <a href="http://www.guoxuedashi.com/zidian/bujian/"><font  color="#CC0066">部件查字</font></a> | <a href="http://www.sfds.cn/"><font  color="#CC0066">书法大师</font></a> | <a href="/jgwhj/">甲骨文</a> <b>|</b> <a href="/b/4/"><font  color="#CC0066">解密</font></a> | <a href="/renwu/">历史人物</a> | <a href="/diangu/">历史典故</a> | <a href="/xingshi/">姓氏</a> | <a href="/minzu/">民族</a> <b>|</b> <a href="/mz/"><font  color="#CC0066">世界名著</font></a> | <a href="/download/">软件下载</a>
</p>
<p><a href="/b/"><font  color="#CC0066">历史</font></a> | <a href="http://skqs.guoxuedashi.com/" target="_blank">四库全书</a> |  <a href="http://www.guoxuedashi.com/search/" target="_blank"><font  color="#CC0066">全文检索</font></a> | <a href="http://www.guoxuedashi.com/shumu/">古籍书目</a> | <a   href="/24shi/">正史</a> <b>|</b> <a href="/chengyu/">成语词典</a> | <a href="/kangxi/" title="康熙字典">康熙字典</a> | <a href="/ShuoWenJieZi/">说文解字</a> | <a href="/zixing/yanbian/">字形演变</a> | <a href="/yzjwjc/">金 文</a> <b>|</b>  <a href="/shijian/nian-hao/">年号</a> | <a href="/diming/">历史地名</a> | <a href="/shijian/">历史事件</a> | <a href="/guanzhi/">官职</a> | <a href="/lishi/">知识</a> <b>|</b> <a href="/zhongyi/">中医中药</a> | <a href="http://www.guoxuedashi.com/forum/">留言反馈</a>
</p>
  </div>
</div>
<!-- 头部导航END --> 
<!-- 内容区开始 --> 
<div class="w1180 clearfix">
  <div class="info l">
   
<div class="clearfix" style="background:#f5faff;">
<script src='http://www.guoxuedashi.com/img/headersou.js'></script>

</div>
  <div class="info_tree"><a href="http://www.guoxuedashi.com">首页</a> > <a href="/SiKuQuanShu/fanti/">四库全书</a>
 > <h1>资治通鉴</h1> <!--         下载:【右键另存为】即可 --></div>
  <div class="info_content zj clearfix">
  
<div class="info_txt clearfix" id="show">
<center style="font-size:24px;">33-資治通鑑卷三十二</center>
    資治通鑑卷三十二   宋 司馬光 撰<br />
<br />
  胡三省 音註<br />
<br />
  漢紀二十四【起著雍涒灘盡昭陽赤奮若凡六年】<br />
<br />
  孝成皇帝中<br />
<br />
  永始四年春正月上行幸甘泉郊泰畤大赦天下三月行幸河東祠后土 夏大旱 四月癸未長樂臨華殿未央宫東司馬門皆災【師古曰東面之司馬門也樂音洛】六月甲午霸陵園門闕災 秋七月辛未晦日有食之 冬十一月庚申衛將軍王商病免 梁王立驕恣無度【立梁孝王武八世孫也】至一日十一犯法相禹奏立對外家怨望有惡言【梁相名禹相息亮翻】有司案驗因發其與姑園子姦事奏立禽獸行請誅【漢法内亂為禽獸行行下孟翻】太中大夫谷永上書曰臣聞禮天子外屏不欲見外也【師古曰屏謂當門之牆以屏蔽者也外屏於門外為之】是以帝王之意不窺人閨門之私聽聞中冓之言【韓詩云中冓中夜應劭曰中冓材冓在堂之中也晋灼曰魯詩以為夜也師古曰冓謂舍之交積材木也應說近之冓音工豆翻】春秋為親者諱【春秋公羊傳閔元年齊仲孫來齊仲孫者何公子慶父也公子慶父則曷為謂之齊仲孫外之也曷為外之春秋為親者諱為于偽翻下同】今梁王年少【少詩照翻下同】頗有狂病始以惡言按驗既無事實而發閨門之私非本章所指王辭又不服猥強劾立傅致難明之事【劾戶槩翻師古曰傅讀曰附】獨以偏辭成辠斷獄【斷丁亂翻】無益於治道【治直吏翻】汙衊宗室【汙烏故翻孟康曰衊音漫師古曰衊音秣謂塗染也】以内亂之惡披布宣揚於天下非所以為公族隱諱【為于偽翻下為公同】增朝廷之榮華昭聖德之風化也臣愚以為王少而父同產長【姑者父之同產長知兩翻】年齒不倫梁國之富足以厚聘美女招致妖麗【妖巧也艷也好也妖於驕翻】父同產亦有恥辱之心【師古曰言其姑亦當自恥必不與姦】案事者乃驗問惡言【師古曰本所問者怨望朝廷之言也】何故猥自發舒【言何為而自發内亂之事】以三者揆之殆非人情疑有所廹切過誤失言文吏躡尋不得轉移【躡尋者謂躡其失言之後而尋其内亂之跡也】萌牙之時加恩勿治上也【如淳曰覆盖之則計之上治直之翻下同】既已案驗舉憲【舉憲者舉以法也】宜及王辭不服詔廷尉選上德通理之吏更審考清問【上與尚同書呂刑皇帝清問下民孔安國曰清問詳問也馬曰清訊】著不然之效定失誤之法【著明也效驗也明其事之不然具有證驗也失誤謂誤入人罪為失】而反命於下吏【師古曰使者還反以清白之狀付有司也】以廣公族附疏之德【附疏者使疏屬親附也】為宗室刷汙亂之恥【師古曰刷謂拭刷除之也音所劣翻】甚得治親之誼天子由是寢而不治 是歲司隸校尉蜀郡何武為京兆尹【姓譜何出自周成王母弟唐叔虞後封於韓韓滅子孫分散江淮問音以韓為何字隨音變遂為何氏】武為吏守法盡公進善退惡所居無赫赫名去後常見思<br />
<br />
  元延元年春正月己亥朔日有食之 壬戌王商復為大司馬衛將軍【商去年以病免今復位】 三月上行幸雍祠五畤【雍於用翻畤音止】 夏四月丁酉無雲而雷【劉向曰雷當託於雲猶君之託于臣隂陽之合也人君不恤天下萬民有怨畔之心故無雲而雷】有流星從日下東南行四面燿燿如雨自晡及昏而止 赦天下 秋七月有星孛于東井【孛蒲内翻】上以災變博謀羣臣北地太守谷永對曰王者躬行道德承順天地則五徵時序【五徵即洪範之八庶徵曰雨曰暘曰寒曰燠曰風也】百姓夀考符瑞並降失道妄行逆天暴物則咎徵著郵【洪範之常雨常暘常寒常燠常風為咎徵著明也天見咎徵以明著人君之過也師古曰郵與尤同尤過也】妖孽並見【洪範五行傳說曰凡草木之類謂之妖妖猶夭胎言尚微也蟲豸之類謂之孽孽則芽孽矣見賢遍翻】饑饉薦臻終不改寤惡洽變備不復譴告更命有德【如魯哀禍大天不降譴是也復扶又翻更工衡翻】此天地之常經百王之所同也加以功德有厚薄期質有修短時世有中季【師古曰中讀曰仲】天道有盛衰陛下承八世之功業【八世高惠文景武昭宣元】當陽數之標季【孟康曰陽九之末紀也師古曰標音必遥翻】涉三七之節紀【孟康曰至平帝乃三七二百一十歲之厄今已涉向其節紀】遭無妄之卦運【應劭曰天必先雲而後雷雷而後雨而今無雲而雷無妄者無所望也萬物無所望于天災異之最大者也師古曰取易之無妄卦為義項安世曰古妄與望通秦漢言无妄皆無望也朱英之說黄歇與揚子法言皆然故太玄以去凖无妄謂其無所復望也在易則自為誠妄之妄】直百六之災阸【易九戹曰初入元百六陽九孟康曰易傳也所謂陽九之戹百六之會也初入元百六歲有戹者則前元之餘氣也師古曰直當也孔穎達曰凡水旱之歲歷運有常按律歷志云十九歲為一章四章為一部二十部為一統三統為一元則一元有四千五百六十歲初入元一歲有陽九為旱九年次三百七十四歲隂九謂水九年以一百六歲并三百七十四歲為四百八十歲注云六乘八之數次四百八十歲有陽九謂旱九年次七百二十歲隂七謂水七年次七百二十歲陽七謂旱七年又注云七百二十者九乘八之數次六百歲隂五謂水五年次六百歲陽五謂旱五年注云六百歲者以八乘八八八六十四又以七乘八七八五十六相并為一千二百歲于易七八不變氣不通故合而數之各得六百歲次四百八十歲隂三次四百八十歲陽三除入元至陽三除去災歲摠有四千五百六十年其災歲兩個陽九年一個隂九年一個隂陽各七年一個隂陽各五年一個隂陽各三年摠有五十七年并前四千五百六十年通為四千六百一十七歲此一元之氣終矣此是隂陽水旱之大數也所以正用七八九六相乘者以水數六火數七木數八金數九此交互相乘也以七八九六隂陽之數自然故有九年七年五年三年之災】三難異科雜焉同會【師古曰雜謂相參也一曰雜音先合翻雜焉總萃難乃旦翻】建始元年以來二十載間【載子亥翻】羣災大異交錯鋒起多於春秋所書内則為深宫後庭將有驕臣悍妾醉酒狂悖卒起之敗【驕臣指淳于長等悍妾指趙昭儀姊弟也悍下罕翻又侯旰翻師古曰卒讀曰猝悖蒲内翻又蒲沒翻】北宫苑囿街巷之中臣妾之家幽閒之處【苑園也孔穎達曰有蕃曰園有牆曰囿園囿大同蕃牆異耳囿者域養禽獸之處園者種菜殖果之處毛晃曰苑亦以養禽獸直曰街曲曰巷師古曰閒讀曰閑】徵舒崔杼之亂【陳靈公淫于夏姬數如其家夏姬之子徵舒病之自廐射而殺之齊莊公通於崔杼之妻姜氏數如崔氏杼伏甲殺之事並見左傳此指帝微行將有徵舒崔杼之禍也】外則為諸夏下土將有樊並蘇令陳勝項梁奮臂之禍【樊並蘇令事見上卷永始三年陳勝項梁事見七卷秦二世元年夏戶雅翻下同】安危之分界宗廟之至憂【師古曰分音扶問翻】臣永所以破膽寒心豫言之累年下有其萌然後變見於上【見賢遍翻】可不致慎禍起細微姦生所易【易輕也忽也言姦生于所輕忽也易以豉翻】願陛下正君臣之義無復與羣小媟黷宴飲【師古曰媟狎也音私列翻黷汚也復扶又翻下同】勤三綱之嚴【師古曰三綱君臣父子夫婦也余按君為臣綱父為子綱夫為婦綱所謂嚴也】修後宫之政抑遠驕妬之寵崇近婉順之行【遠於願翻近其靳翻行下孟翻】朝覲法駕而後出【朝直遥翻】陳兵清道而後行無復輕身獨出飲食臣妾之家三者既除内亂之路塞矣【三者謂微行崇飲好色也塞悉則翻】諸夏舉兵萌在民饑饉而吏不恤興於百姓困而賦歛重發於下怨離而上不知【永書曰諸夏舉兵以大角為期盖言已有其萌而將至於興發也歛力贍翻】傳曰饑而不損兹謂泰厥咎亡【師古曰洪範傳之辭余按五行志盖京房易傳之辭也】比年郡國傷於水災禾麥不收【禾粟苖也又稼之摠名比毗至翻】宜損常税之時【謂此時宜減税也】而有司奏請加賦甚繆經義逆於民心市怨趨禍之道也【趨讀曰趣與促同】臣願陛下勿許加賦之奏益减奢泰之費流恩廣施【施式豉翻】振贍困乏勅勸耕桑以慰綏元元之心諸夏之亂庶幾可息【贍而艷翻幾居希翻又巨衣翻】中壘校尉劉向【武帝置中壘校尉掌北軍壘門之内又外掌西域八校尉之首也】上書曰臣聞帝舜戒伯禹毋若丹朱傲【師古曰事見虞書后稷篇丹朱堯子也敖讀曰傲仲馮曰此禹戒舜之語非舜戒禹之辭也上時掌翻】周公戒成王毋若殷王紂【商書無逸篇周公戒成王曰毋若殷王紂之迷亂酗于酒德哉】聖帝明王常以敗亂自戒不諱廢興故臣敢極陳其愚唯陛下留神察焉謹案春秋二百四十二年日食三十六【師古曰從隱公元年至哀公十四年獲麟凡二百四十二年日食三十六謂隱三年二月己巳桓三年七月壬辰朔十七年十月朔莊十八年三月二十五年六月辛未朔二十六年十二月癸亥朔三十九九月庚午朔僖五年九月戊申朔十二年三月庚午十五年五月文元年二月己亥朔十五年六月辛丑朔宣八年七月甲子十年四月丙辰十七年六月癸卯成十六年六月丙寅朔十七年十二月丁巳朔襄十四年二月乙未朔十五年秋八月丁巳二十年冬十月丙辰朔二十一年九月庚戌朔冬十月庚辰朔二十三年二月癸酉朔二十四年秋七月甲子朔八月癸巳朔二十七年冬十二月乙亥朔昭七年夏四月甲辰朔十五年六月丁巳朔十七年六月甲戌朔二十一年秋七月壬午朔二十二年十二月癸酉朔二十四年夏五月乙未朔三十一年十二月辛亥朔定五年正月辛亥朔十二年卜一月丙寅朔十五年八月庚辰朔也】今連三年比食【比毗至翻】自建始以來二十歲間而八食率二歲六月而一發古今罕有【建始三年十二月戊申朔河平元年四月癸亥晦三年八月乙卯晦四年三月癸丑朔陽朔元年二月丁未晦永始二年二月乙酉晦三年正月己卯晦四年七月辛未晦凡八食而是年春正月己亥又不預此數】異有小大希稠占有舒疾緩急觀秦漢之易世覽惠昭之無後察昌邑之不終視孝宣之紹起皆有變異著於漢紀天之去就豈不昭昭然哉【按向書曰秦始皇之末至二世時日月薄食山林淪亡星辰出於四孟太白經天而行無雲而雷枉矢夜光熒惑襲月㜸火燒宫野禽戲庭都門内崩長人見臨洮石隕於東郡星孛大角大角以亡及項籍之敗亦孛大角漢之入秦五星聚於東井得天下之象也孝惠時有雨血日食於衝滅光星見之異孝昭時有太山卧石自立上林僵柳復起大星如月西行衆星隨之此為特異孝宣興起之表天狗夹漢而西久隂不雨者二十餘日昌邑不終之異也】臣幸得託末屬誠見陛下寛明之德冀銷大異而興高宗成王之聲【向書曰高宗成王亦有雊雉拔木之變能思其故故高宗有百年之福成王有復風之報向之所以望帝者如此】以崇劉氏【崇增高也謂增高劉氏之業愈巍巍也】故懇懇數奸死亡之誅【師古曰懇懇欲誠之意也奸犯也數所角翻奸音千】天文難以相曉臣雖圖上猶須口說然後可知願賜清燕之閒指圖陳狀上輒入之【師古曰謂召入也上時掌翻閒讀曰閑又如字上輒之上如字】然終不能用也 【考異曰向傳云星孛東井岷山崩向懷不能已上此奏按岷山崩在三年此奏云自建始以來二十歲間而食八率二歲六月而一發則上此奏當在今年也胡旦亦載之三年 余按劉向傳若以星孛東井為據則上奏當在今年若以岷山崩為據則上奏當在三年若以二十歲間日八食為據則上奏當在去年然向言日食之變率二歲六月而一發以班書考之自建始三年十二月至河平元年四月則一年五月而食至四年三月癸丑朔則纔一年而食又至陽朔元年二月丁未晦則又期年而食永始元年九月丁巳晦志書食而紀不書至二年二月乙酉晦則凡九期而志所書永始元年九月丁巳晦不計也又至永始三年正月己卯晦則未及一期而食又至四年七月辛未晦則一年六月而食向所謂率二歲六月而一發亦通二十歲而約言之耳自建始三年至今年以紀考之則九食以志考之則十食此其差異又未有所折衷也】 紅陽侯立舉陳咸方正對策拜為光禄大夫給事中丞相方進復奏咸前為九卿坐為貪邪免【咸免見上卷永始二年復扶又翻】不當蒙方正舉備内朝臣【孟康曰内朝中朝也大司馬前後左右將軍侍中常侍散騎諸吏給事中為中朝官丞相以下至六百石為外朝官也】并劾紅陽侯立選舉故不以實【漢制列侯選舉不以實削封戶劾戶槩翻下同】有詔免咸勿劾立 十二月乙未王商為大將軍辛亥商薨其弟紅陽侯立次當輔政先是立使客因南郡太守李尚占墾草田數百頃【先悉薦翻據孫寶傳占墾草田頗有民所假少府陂澤略皆開發師古曰隱度而取之也草田荒田也舊為陂澤本屬少府其後以假百姓百姓皆已田之而立摠謂為草田占云新自墾占音之贍翻百畝為頃】上書以入縣官【師古曰立上書云新墾得此田請以入官也】貴取其直一億萬以上【師古曰直價直也貴者增於時價】丞相司直孫寶發之上由是廢立而用其弟光禄勲曲陽侯根庚申以根為大司馬驃騎將軍 【考異曰荀紀云十一月成紀云十二月按是歲十一月甲子朔無乙未辛亥庚申荀悦誤 今按考異又有揚雄待詔一條注云雄傳云車騎將軍王音奇其文雅薦雄待詔按雄自序云上方郊祠甘泉泰畤召雄待詔承明之庭奏甘泉賦其十二月奏羽獵賦事在今年時王音卒已久盖王根也胡旦遂誤以為曲陽侯云 余按曲陽侯即王根也王音則封安陽侯】 特進安昌侯張禹請平陵肥牛亭地【師古曰肥牛亭名禹欲得置亭之處為冢塋】曲陽侯根争以為此地當平陵寢廟衣冠所出游道宜更賜禹它地【請别以地賜之更工衡翻】上不從卒以賜禹【卒子恤翻】根由是害禹寵數毁惡之【數所角翻下同師古曰惡謂言其過惡依顔注惡當讀如字後凡毁惡之惡皆同音】天子愈益敬厚禹每病輒以起居聞【師古曰謂其飲食寢卧之增損】車駕自臨問之上親拜禹牀下禹頓首謝恩禹小子未有官禹數視其小子上即禹牀下拜為黄門郎給事中【即就也】禹雖家居以特進為天子師國家每有大政必與定議【師古曰與讀曰豫余謂與讀如字言天子與禹定其可否也】時吏民多上書言災異之應譏切王氏專政所致【上時掌翻】上意頗然之未有以明見【未能灼見人言之當否也】乃車駕至禹弟【弟與第同舍也宅也】辟左右【師古曰辟讀曰闢】親問禹以天變因用吏民所言王氏事示禹禹自見年老子孫弱又與曲陽侯不平恐為所怨則謂上曰春秋日食地震或為諸侯相殺夷狄侵中國【為于偽翻】災變之意深遠難見故聖人罕言命不語怪神【師古曰罕稀也論語云子罕言利與命與仁又曰子不語怪力亂神】性與天道自子貢之屬不得聞【師古曰論語稱子貢曰夫子之言性與天道不可得而聞也謂孔子未嘗言性命及天道】何况淺見鄙儒之所言陛下宜修政事以善應之與下同其福喜【漢書張禹傳喜作善】此經義意也新學小生亂道誤人宜無信用以經術斷之【斷丁亂翻】上雅信愛禹由此不疑王氏【元帝師蕭望之成帝師張禹皆敬重之矣元帝不能聽望之言疎許史而去恭顯成帝則聽禹言而不疑王氏望之以此殺身禹以此苟富貴漢祚中衰實由此也又成帝之時吏民猶譏切王氏平帝之末吏民以王莽不受新野田上書者至四十八萬七千五百七十二人何元帝之時吏民猶忠於漢平帝之時吏民則附王氏也政自之出久矣人心能無從之乎有國家者尚監兹哉】後曲陽侯根及諸王子弟聞知禹言皆喜說遂親就禹【張氏安矣劉氏危矣說讀曰悦】故槐里令朱雲【元帝時雲為槐里令坐論石顯廢錮故稱故】上書求見【見賢遍翻】公卿在前雲曰今朝廷大臣上不能匡主下無以益民皆尸位素餐【師古曰尸主也素空也尸位者不舉其事但主其位而已素餐者德不稱官空食禄】當孔子所謂鄙夫不可與事君苟患失之亡所不至者也【師古曰論語所載孔子之言也苟患失其寵禄則言行僻邪無所不至也謹案孔子曰鄙夫可與事君也與哉其未得之也患得之既得之患失之苟患失之無所不至矣亡與無同】臣願賜尚方斬馬劒【師古曰尚方少府之屬官也作供御器物故有斬馬劒劒利可以斬馬】斷佞臣一人頭以厲其餘【斷丁管翻】上問誰也對曰安昌侯張禹上大怒曰小臣居下訕上【盖引用論語惡居下流而訕上之言師古曰訕謗也音所諫翻又音刪】廷辱師傅罪死不赦御史將雲下雲攀殿檻檻折【師古曰檻軒前欄也折而故翻】雲呼曰臣得下從龍逢比干遊於地下足矣【師古曰呼叫也音火故翻□龍逢桀臣王子比干紂臣皆以諫而死故云然逢音皮江翻】未知聖朝何如耳【師古曰言殺直臣其聲惡余謂雲盖言亦將如夏殷之亡也朝直遥翻下入朝同每朝同】御史遂將雲去【將如字挟也攜也】於是左將軍辛慶忌免冠解印綬叩頭殿下曰此臣素著狂直於世【師古曰著表也言此名久已章表】使其言是不可誅其言非固當容之臣敢以死争慶忌叩頭流血上意解然後得已【言殺雲之事得止也】及後當治檻【治直之翻】上曰勿易因而輯之以旌直臣【師古曰輯與集同謂補合之也旌表也】 匈奴搜諧單于將入朝未入塞病死弟且莫車立為車牙若鞮單于以囊知牙斯為左賢王【單音蟬且子余翻車尺遮翻鞮丁奚翻】 北地都尉張放到官數月復徵入侍中【復扶又翻下同】太后與上書曰前所道尚未效【張晏曰謂太后言班侍中大將軍所舉宜寵異之詳上卷永始二年】見富平侯反復來其能默虖【如淳曰富平侯張放又來太后安能默然不以為言】上謝曰請今奉詔上於是出放為天水屬國都尉【地理志天水屬國都尉治勇土縣】張少府許商光禄勲師丹為光禄大夫【姓譜師古者掌樂之官因以為氏】班伯為水衡都尉並侍中皆秩中二千石每朝東宫常從【從才用翻】及大政俱使諭指於公卿【使傳上指以諭公卿也】上亦稍厭游宴復修經書之業【上為太子時好經書及即位幸酒樂宴樂今出放等復修經書業】太后甚悦 是歲左將軍辛慶忌卒慶忌為國虎臣【爪牙扞禦之臣曰虎臣】遭世承平匈奴西域親附敬其威信<br />
<br />
  二年春正月上行幸甘泉郊泰畤三月行幸河東祠后土既祭行遊龍門【師古曰龍門山在今蒲州龍門縣北】登歷觀【晋灼曰歷觀在河東蒲反縣師古曰歷山上有觀觀音古玩翻】陟西岳而歸【陟登也師古曰西岳華山也】 夏四月立廣陵孝王子守為王【廣陵孝王霸厲王胥之子也元帝初元二年紹封傳子意孫護人薨無後今立守以紹封考異曰荀紀守作憲今從漢書】 初烏孫小昆彌安日為降民所殺諸翖侯大亂【降戶江翻翖許及翻】詔徵故金城太守段會宗為左曹中郎將光禄大夫使安輯烏孫【陽朔中會宗復為西域都護終更而還以擅發戊巳校尉兵迎康居降者不遂劾乏興詔以贖論拜金城太守以病免故曰故金城太守守式又翻】立安日弟末振將為小昆彌【服䖍曰末振將人姓名師古曰其名也昆彌之弟不可别舉姓也 考異曰烏孫傳以末振將為安日弟段會宗傳以為兄兄字誤耳】定其國而還【還從宣翻又如字】時大昆彌雌栗靡勇健末振將恐為所并使貴人烏日領詐降刺殺雌栗靡【刺七亦翻】漢欲以兵討之而未能遣中郎將段會宗立公主孫伊秩靡為大昆彌【公主謂楚主解憂也公主之孫于雌栗靡為季父】久之大昆彌翖侯難栖殺末振將安日子安犂靡代為小昆彌漢恨不自誅末振將復遣段會宗發戊巳校尉諸國兵【復扶又翻校戶教翻】即誅末振將太子番丘【即就也師古曰番音盤】會宗恐大兵入烏孫驚番丘亡逃不可得即留所發兵墊婁地【服䖍曰墊音墊阸之墊鄭氏曰婁音羸師古曰墊音丁念翻婁音樓】選精兵三十弩【李奇口三十人人持一弩】徑至昆彌所在召番丘責以末振將之罪即手劒擊殺番丘【手執劒曰手劒記檀弓曰子手弓子射諸手守又翻】官屬以下驚恐馳歸小昆彌安犁靡勒兵數千騎圍會宗會宗為言來誅之意【為言奉天子命來誅番丘之意為于偽翻】今圍守殺我如取漢牛一毛耳【司馬遷答任安書假令僕伏法受誅若九牛亡一毛與螻蟻何異自諭其身甚微也】宛王郅支頭縣槀街【宛王事見二十一卷武帝太初三年郅支事見二十九卷元帝建昭三年宛於元翻】烏孫所知也昆彌以下服曰末振將負漢誅其子可也獨不可告我令飲食之邪【師古曰飲於禁翻食讀曰飤下同】會宗曰豫告昆彌逃匿之為大罪【謂豫以誅番丘之事告昆彌昆彌以叔姪之情必使番丘逃匿漢欲誅之而昆彌匿之則於漢為有大罪也】即飲食以付我傷骨肉恩【若飲食之而使之就死則於骨肉為傷恩】故不先告昆彌以下號泣罷去【號戶刀翻】會宗還奏事天子賜會宗爵關内侯黄金百斤會宗以難栖殺末振將奏以為堅守都尉【烏孫有大將都尉各一人以難栖能為雌栗靡復讐堅守臣節異於諸翖侯故以堅守二字寵之】責大禄大監以雌栗靡見殺狀奪金印紫綬更與銅墨云【宣帝甘露三年大禄大監賜金印紫綬】末振將弟卑爰疐【師古曰疐音竹二翻】本共謀殺大昆彌將衆八萬北附康居謀欲借兵兼并昆彌【卑爰疐自此彊其後都護孫建襲殺之將即亮翻】漢復遣會宗與都護孫建并力以備之【復扶又翻下同】自烏孫分立兩昆彌漢用憂勞且無寧歲【分立兩昆彌見二十七卷宣帝甘露元年】時康居復遣子侍漢【元帝時康居遣子入侍陳湯上言其非王子今復遣子入侍】貢獻【既遣子入侍而又奉貢也】都護郭舜上言【此時郭舜為都護平帝元始閒孫建始為都護上時掌翻】本匈奴盛時非以兼有烏孫康居故也及其稱臣妾非以失二國也【言匈奴之強弱不繫二國之叛服】漢雖皆受其質子然三國内相輸遺交通如故【三國謂匈奴烏孫康居質音致遺于季翻】亦相侯司【司讀曰伺】見便則發合不能相親信離不能相臣役以今言之結配烏孫竟未有益反為中國生事【謂自武帝以來以宗室女下嫁烏孫也為于偽翻】然烏孫既結在前今與匈奴俱稱臣義不可距而康居驕黠訖不肯拜使者【師古曰訖竟也黠戶八翻】都護吏至其國坐之烏孫諸使下王及貴人先飲食已乃飲啗都尉吏【師古曰飲音于禁翻啗音徒濫翻】故為無所省以夸旁國【師古曰言故不省視漢使也余謂夸者自矜耀其能傲漢也旁國隣國也省悉并翻】以此度之何故遣子入侍其欲賈市為好辭之詐也【謂特欲行賈以市易其為好辭者詐也度徒洛翻賈音古】匈奴百蠻大國【師古曰於百蠻中最大國也】今事漢甚備聞康居不拜且使單于有悔自卑之意【師古曰言單于見康居不事漢以為高自以事漢為太卑而悔之也】宜歸其侍子絶不復使【師古曰不通使於其國也使疏吏翻】以章漢家不通無禮之國【章顯著也】漢為其新通【為于偽翻】重致遠人【師古曰以此聲名為重也】終羈縻不絶<br />
<br />
  三年春正月丙寅蜀郡岷山崩【地理志岷山在蜀郡湔氏道西徼外禹貢所謂岷山導江即此山也水經注曰岷山即瀆山水曰瀆水亦曰汶阜山在氐道徼外江水所導也大江泉源發羊膊下緣崖散漫小大百數殆未濫觴東南下百餘里至白馬西歷天彭關亦謂之天谷天彭山兩山相對其高若闕謂之天彭門江水自此以上至微弱所謂其源濫觴者也漢延元中岷山崩壅江水三日不流即其處岷音武申翻】壅江三日江水竭劉向大惡之【惡音烏路翻惡其徵異也】曰昔周岐山崩三川竭而幽王亡【周幽王二年三川竭岐山崩師古曰三川涇渭洛也洛即漆沮也余按幽王時有是異後卒為犬戎所殺】岐山者周所興也【周自太王避狄去豳而邑于岐山之下周之王業遂興于此】漢家本起於蜀漢【高帝始王漢中起兵還定三秦誅項羽遂有天下】今所起之地山崩川竭星孛又及攝提大角從參至辰【天文志房南衆星曰騎官左角理右角將大角者天王帝坐庭其兩旁各有三星鼎足句之曰攝提攝提者直斗扚所指以建時節故曰攝提格晋天文志參十星千辰在申至辰者至大火也自氐五度至尾九度為大火于辰在卯如淳曰孛星尾長及攝提大角始發于參至辰也孛蒲内翻參疏簪翻】殆必亡矣 二月丙午封淳于長為定陵侯【恩澤侯表定陵侯國於汝南】 三月上行幸雍祠五畤 上將大誇胡人以多禽獸秋命右扶風發民入南山西自褒斜【師古曰褒斜南山二谷名余按自秦川逕南山通漢中南谷曰褒北谷曰斜徑五百里斜余遮翻】東至弘農【長安南山連延東至弘農今商虢二州之山皆是也】南漢中【與驅同】張羅罔罝罘【罔與網同古字通用罝音咨邪翻兎罟也罘音房尤翻翻車大網也】捕熊羆禽獸【熊似豕而大黑色羆似熊黄白色被髪人立而絶有力】載以檻車輸之長楊射熊館【師古曰長楊宫中有射熊館】以罔為周阹【李奇曰阹遮禽獸圍陳也師古曰阹音袪】縱禽獸其中令胡人手搏之自取其獲上親臨觀焉 【考異曰成紀元延二年冬行幸長楊宫從胡客大校獵宿萯陽宫賜從官胡旦用之按揚雄傳祀甘泉河東之歲十二月羽獵雄上校獵賦明年從至射熊館還上長楊賦然則從胡客校獵當在今年紀因去年冬有羽獵事致此誤耳】<br />
<br />
  四年春正月上行幸甘泉郊泰畤 中山王興定陶王欣皆來朝【興帝少弟欣帝弟定陶共王康之子朝直遥翻】中山王獨從傅定陶王盡從傅相中尉【師古曰三官皆從王入朝相息亮翻】上怪之以問定陶王對曰令諸侯王朝得從其國二千石傅相中尉皆國二千石故盡從之上令誦詩通習能說【師古曰說其義也】佗日問中山王獨從傅在何法令不能對令誦尚書又廢【師古曰中忘之也法令力政翻令誦力呈翻】及賜食於前後飽起下韈係解【師古曰食而獨在後飽及起又韈係解也韈音武伐翻余謂賜食於君前禮主於敬食而獨後又致飽而止皆非敬也及起而降階韈係解而不知是皆不能執禮夫禮所以固人肌膚之會筋骸之束也韈足衣也係所以結韈】帝由此以為不能而賢定陶王數稱其材【數所角翻】是時諸侯王唯二人於帝為至親定陶王祖母傅太后隨王來朝【傅太后元帝傅昭儀定陶共王母也隨共王就國為定陶太后】私賂遺趙皇后昭儀及票騎將軍王根【遺于季翻票匹妙翻】后昭儀根見上無子亦欲豫自結為長久計皆更稱定陶王【迭互稱其材美也師古曰更工衡翻】勸帝以為嗣帝亦自美其材為加元服而遣之【師古曰為之冠也為于偽翻】時年十七矣 三月上行幸河東祠后土 隕石于關東二【據漢書關東當作都關師古曰都關山陽之縣】 王根薦谷永徵入為大司農【自北地太守徵入】永前後所上四十餘事【上時掌翻】略相反覆專攻上身與後宫而已黨於王氏上亦知之不甚親信也為大司農歲餘病滿三月上不賜告即時免【故事公卿病輒賜告上以其黨於王氏故即時免】數月卒【史終言之】<br />
<br />
  綏和元年春正月大赦天下 上召丞相翟方進御史大夫孔光右將軍亷褒後將軍朱博入禁中【票騎將軍王根先勸帝立定陶王為嗣漢書孔光傳先書根勸立定陶王事下即書召方進光褒博入禁中通鑑因之亦不言根今但以下文觀之根亦召入禁中也】議中山定陶王誰宜為嗣者方進根褒博皆以為定陶王帝弟之子禮曰昆弟之子猶子也為其後者為之子也【昆弟之子視猶子也以弟之子為兄後則為兄之子矣公羊春秋成十五年仲嬰齊卒此公孫嬰齊也曷為謂之仲嬰齊為兄後也為兄後則曷為謂之仲嬰齊為人後者為之子也為其子則其稱仲何孫以王父字為氏也】定陶王宜為嗣光獨以為禮立嗣以親【謂兄弟同父之親子其親親於兄弟之子】以尚書盤庚殷之及王為比兄終弟及【兄終弟及殷法也殷自外丙仲壬至於盤庚率多兄弟代立而尚書無文光所引盖今文尚書也師古曰比音必寐翻余謂當如字讀】中山王先帝之子帝親弟宜為嗣上以中山王不材又禮兄弟不得相入廟【父為昭子為穆則兄弟不得相入廟也】不從光議二月癸丑詔立定陶王欣為皇太子封中山王舅諫議大夫馮參為宜鄉侯益中山國三萬戶以慰其意【師古曰以不得繼統為帝之後恐其怨恨】使執金吾任宏守大鴻臚持節徵定陶王【大鴻臚掌諸侯故任宏守大鴻臚之官以徵定陶王守者權守也任音壬臚陵如翻】定陶王謝曰臣材質不足以假充太子之宫【師古曰謙不言為太子故云假充若元非正余謂王謝意盖以將有皇嗣今為太子特假充耳】臣願且得留國邸旦夕奉問起居【謂昏定晨省記曰文王之為世子也朝於王季日三雞初鳴而衣服至于寢門外問内竪之御者曰今日安否如何内竪曰安文王乃喜及日中又至亦如之及暮再至亦如之其有不安節則内竪以告文王文王色憂行不能正履此旦夕問起居之禮也國邸謂定陶國邸也】俟有聖嗣歸國守藩書奏天子報聞【報聞報已覽其書而不從其請也】戊午孔光以議不合意左遷廷尉何武為御史大夫【光左遷廷尉而何武自廷尉為御史大夫】 初詔求殷後分散為十餘姓【殷子姓也其後為宋為孔為華為戴為桓為向為樂等姓】推求其嫡不能得匡衡梅福皆以為宜封孔子世為湯後【匡衡議以為王者存三王後所以尊其先王而通三統也其犯誅絶之罪者絶而更封他親為始封君上承其王者之始祖春秋之義諸侯不能守其社稷者絶今宋國已不守其統而失國矣則宜更立殷後為始封君而上承湯統非當繼宋之絶侯也宜明得殷後而已今之故宋推求其嫡久遠不可得雖得其嫡嫡之先已絶不當得立禮孔子曰丘殷人也先師所傳宜以孔子世為湯後此元帝時議也是時梅福復言之】上從之封孔結為殷紹嘉侯【恩澤侯表殷紹嘉侯國于沛】三月與周承休侯皆進爵為公地各百里上行幸雍祠五畤 初何武之為廷尉也【公卿表元延三年何武自沛郡太守為廷尉是年三月戊午為御史大夫】建言末俗之敝政事煩多宰相之材不能及古而丞相獨兼三公之事所以久廢而不治也【廢謂廢事也】宜建三公官上從之夏四月賜曲陽侯根大司馬印綬置官屬罷票騎將軍官【武帝初置大司馬以冠將軍之號宣帝地節三年置大司馬不冠將軍亦無印綬官屬今賜大司馬金印紫綬置官屬而大司馬為專官故根不復領票騎將軍】以御史大夫何武為大司空封汜鄉侯【武封汜鄉侯在琅邪不其縣後改食南陽博望鄉師古曰汜音凡其音基】皆增奉如丞相【如淳曰律大司馬大將軍與丞相奉月錢六萬御史大夫奉月四萬也奉讀曰俸】以備三公焉 秋八月庚戌中山孝王興薨 匈奴車牙單于死弟囊知牙斯立為烏珠留若鞮單于烏珠留單于立以弟樂為左賢王輿為右賢王【樂呼韓邪單于大閼氏之子輿弟五閼氏之子】漢遣中郎將夏侯藩副校尉韓容使匈奴或說王根曰【說輸芮翻】匈奴有斗入漢地直張掖郡【師古曰斗絶也地之斗曲入漢界者也直當也】生奇材箭竿鷲羽【師古曰鷲大鵰也黄頭赤目其羽可為箭竿音工旱翻鷲音就余按鷲羽可為箭翎也山海經曰景山多鷲黑色多力所謂皂鵰是也】如得之於邉甚饒國家有廣地之實將軍顯功垂於無窮根為上言其利【言得此地為中國利也為于偽翻下同】上直欲從單于求之【師古曰直猶正也余謂直徑直也】為有不得傷命損威【師古曰詔命不行為傷命余謂天子之命不行於夷狄為損中國之威】根即但以上指曉藩令從藩所說而求之【師古曰自以藩意說單于而求之說輸芮翻下同】藩至匈奴以語次說單于曰【語次交語之次也】竊見匈奴斗入漢地直張掖郡漢三都尉居塞上士卒數百人寒苦候望久勞【張掖兩都尉一治日勒澤索谷一治居延又有農都尉治番和是為三都尉師古曰澤音鐸索音先各翻如淳曰番音盤】單于宜上書獻此地直斷割之【謂從直割地以其斗入者與漢也斷下管翻上時掌翻下同】省兩都尉士卒數百人以復天子厚恩【師古曰復亦報也】其報必大【師古曰漢得此地必厚報單于】單于曰此天子詔語邪【邪音耶疑未定之辭】將從使者所求也藩曰詔指也然藩亦為單于畫善計耳【為于偽翻】單于曰此温偶駼王所居地也【師古曰偶音五口翻駼音塗下同余按後漢書匈奴有温禺犢王班固燕然銘曰斬温禺以釁鼓血尸遂以染鍔意温偶即温禺也後人妄加禺旁從人耳當讀曰禺】未曉其形狀所生請遣使問之【形狀謂地形之夷險可割與不可割之狀也師古曰所生謂山之所生草木鳥獸為用者】藩容歸漢後復使匈奴【復扶又翻】至則求地單于曰父兄傳五世【呼韓邪傳其長子復株絫復株絫傳其弟搜諧搜諧又傳其弟車牙車牙傳之囊知牙斯是為五世】漢不求此地至知獨求何也【單于名囊知牙斯王莽專政諷其慕中國不二名始名知史從簡便因以單名書于此】已問温偶駼王匈奴西邉諸侯作穹廬及車皆仰此山材木【師古曰謂諸小王為諸侯效中國之言耳仰音牛向翻】且先父地不敢失也【先父謂呼韓邪】藩還遷太原太守單于遣使上書以藩求地狀聞【守式又翻使疏吏翻】詔報單于藩擅稱詔從單于求地法當死更大赦二【余按是年後至明年哀帝即位大赦又明年改元赦詔云更大赦二以此知夏侯藩再使匈奴必在建平初師古曰更經也音工衡翻】今徙藩為濟南太守不令當匈奴【濟子禮翻】 冬十月甲寅王根病免上以太子既奉大宗後不得顧私親【按禮祖父以上正嫡相傳為大】<br />
<br />
  【宗别子為祖繼别為宗繼禰者為小宗定陶王以帝弟之子入奉大宗後義不得復顧定陶共王親也】十一月立楚孝王孫景為定陶王【楚孝王囂宣帝之子】太子議欲謝少傳閻崇以為人後之禮不得顧私親不當謝【少詩照翻下少府同】太傅趙玄以為當謝太子從之詔問所以謝狀尚書劾奏玄左遷少府【劾戶槩翻】以光禄勲師丹為太傅初太子之幼也王祖母傅太后躬自養視【在定陶國時也】及為太子詔傅太后丁姬自居定陶國邸【丁姬事定陶共王實生太子】不得相見頃之王太后欲令傅太后丁姬十日一至太子家帝曰太子承正統當共養陛下【漢亦稱太后為陛下後世多稱殿下唯臨朝乃稱陛下共音居用翻養音弋尚翻】不得復顧私親【此私親謂傅太后丁姬復扶又翻下同】王太后曰太子小而傅太后抱養之今至太子家以乳母恩耳【謂抱養太子恩猶乳母也】不足有所妨於是令傅太后得至太子家丁姬以不養太子獨不得 衛尉侍中淳于長有寵於上大見信用貴傾公卿外交諸侯牧守賂遺【牧州牧也守郡守也遺于季翻下同】賞賜累鉅萬淫於聲色【淫過也放也】許后姊孊為龍雒思侯夫人【龍雒思侯韓寶增子也晋灼曰孊音靡余按韓寶已死故書謚謚法外内思索曰思追悔前過曰思】寡居長與孊私通因取為小妻【孊雖皇后之姊列侯之夫人以淫放失身於長而長自有正室故為小妻記曰聘則為妻奔則為妾婦人女子之持身不可不慎也】許后時居長定宫【許后廢徙昭臺宫歲餘還徙長定宫師古曰三輔黄圖林光宫中有長定宫】因孊賂遺長欲求復為媫妤長受許后金錢乘輿服御物前後千餘萬【乘繩證翻】詐許為白上立為左皇后【許為于偽翻】孊每入長定宫輒與孊書戲侮許后嫚易無不言【師古曰嫚䙝汙也易輕也易音弋䜴翻】交通書記賂遺連年時曲陽侯根輔政久病數乞骸骨【數所角翻】長以外親居九卿位【長太后姊子於帝室為外家之親】次第當代根侍中騎都尉光禄大夫王莽心害長寵私聞其事莽侍曲陽侯病因言長見將軍久病意喜自以當代輔政至對衣冠議語署置【衣冠當時士大夫及貴游子弟也師古曰自謂當輔政故豫言某人為某官某人主某事】具言其辠過根怒曰即如是何不白也莽曰未知將軍意故未敢言根曰趣白東宫【東宫太后宫師古曰趣讀曰促】莽求見太后具言長驕佚欲代曲陽侯私與長定貴人姊通受取其衣物太后亦怒曰兒至如此【長太后姊子故呼為兒】往白之帝莽白上上以太后故免長官勿治罪遣就國【就定陵侯國治直之翻】初紅陽侯立不得輔政疑為長毁譖常怨毒長【毒苦也痛也怨之甚也】上知之及長當就國立嗣子融從長請車騎【以長當就國所常從車騎無所用故請之師古曰嗣子謂嫡長子當為嗣者也】長以珍寶因融重賂立立因上封事為長求留【上時掌翻為于偽翻】曰陛下既託文以皇太后故【蘇林曰託於詔文也】誠不可更有佗計【師古曰言不宜遣長就國】於是天子疑焉【帝知立素怨長今為長上封事求留疑心於是而起】下有司按驗【下戶稼翻下同】吏捕融立令融自殺以滅口【恐融就吏而事泄故令其自殺以滅口】上愈疑其有大姦遂逮長繫洛陽詔獄【凡詔所繫治皆為詔獄非必洛陽先有詔獄也】窮治【考鞠以窮其姦也】長具服戲侮長定宫謀立左皇后辠至大逆死獄中妻子當坐者徙合浦母若歸故郡【長母若即王太后姊故居魏郡元城師古曰若者其母名】上使廷尉孔光持節賜廢后藥自殺丞相方進復劾奏紅陽侯立狡猾不道【師古曰狡狂也猾亂也復扶又翻】請下獄上曰紅陽侯朕之舅不忍致法遣就國于是方進復奏立黨友後將軍朱博鉅鹿太守孫閎皆免官與故光禄大夫陳咸皆歸故郡【朱博杜陵人孫閎亦京師世家陳咸本沛郡相人㨿漢書翟方進傳則博閎免官獨咸歸故鄉耳與字皆字衍元延元年咸免光禄大夫故稱故】咸自知廢錮以憂死方進智能有餘兼通文法吏事以儒雅緣飾【師古曰緣飾譬之於衣加純緣者純音之允翻】號為通明相【相息亮翻】天子器重之又善求人主微指【微指謂上意所嚮未著見於外者】奏事無不當意方淳于長用事方進獨與長交稱薦之【據方進傳長初用事方進獨與長交及長寵盛與之交者不獨一方進也】及長坐大逆誅上以方進大臣為之隱諱【為于偽翻】方進内慙上疏乞骸骨【上時掌翻】上報曰定陵侯長已伏其辜君雖交通傳不云乎朝過夕改君子與之【師古曰與許也余謂此盖論語傳音直戀翻】君何疑焉其專心壹志毋怠醫藥以自持方進起視事復條奏長所厚善京兆尹孫寶右扶風蕭育刺史二千石以上免二十餘人【孫寶蕭育皆能吏也以急於求進比匪人以得罪是以君子慎交】函谷都尉建平侯杜業素與方進不平【函谷關置都尉以譏出入業杜延年之孫 素不事權貴與翟方進淳于長皆不平】方進奏業受紅陽書聽請不敬免就國【據業傳業與淳于長不平長當就國紅陽侯立與業書屬之勿】<br />
<br />
  【復用前事相侵長出關後罪復發下洛陽獄丞相史搜得紅陽侯書奏業聽請不敬服䖍曰受立屬請為不敬】上以王莽首發大姦稱其忠直王根因薦莽自代丙寅以莽為大司馬時年三十八莽既拔出同列繼四父而輔政【師古曰鳳商音根四人皆為大司馬而莽之諸父也】欲令名譽過前人遂克己不倦聘諸賢良以為掾史賞賜邑錢悉以享士【邑錢封邑所入之錢也掾俞絹翻】愈為儉約母病公卿列侯遣夫人問疾莽妻迎之衣不曳地布蔽膝【蔽膝鞸也亦曰鄭玄曰韍太古蔽膝之象】見之者以為僮使問知其夫人【此下依漢書有皆驚二字文意乃足它本皆有此二字】其飾名如此 丞相方進大司空武奏言春秋之義用貴治賤不以卑臨尊【春秋首止之會殊會王世子世子貴也宋之盟楚駕晋而書先晋黄池之會吴主會而書先晋不以卑臨尊也治直之翻】刺史位下大夫而臨二千石【刺史六百石下大夫之秩也其朝位亦班于下大夫】輕重不相準臣請罷刺史更置州牧以應古制【古制九州一為畿内八州八伯以統諸侯之國今請置州牧以應古州伯之制更工衡翻下同】十二月罷刺史更置州牧秩二千石 犍為郡於水濱得古磬十六枚【師古曰濱水厓也音賓說文曰磬樂石也古者母句氏作磬後或以玉為之犍為言翻】議者以為善祥劉向因是說上宜興辟雍【記王制天子之學曰辟雍鄭玄曰辟明也雍和也所以明和天下說輸芮翻】設庠序【古者黨有庠遂有序庠者養也序者教也】陳禮樂隆雅頌之聲盛揖讓之名以風化天下如此而不治者未之有也【治直吏翻】或曰不能具禮【師古曰或曰者劉向設為難者之言而後答釋也】禮以養人為本如有過差【師古曰過差猶失錯也】是過而養人也刑罸之過或至死傷今之刑非臯陶之法也而有司請定法削則削筆則筆【服䖍曰言隨君意也師古曰削者言有所刪去以刀削簡牘也筆者謂有所增益以筆就而書也】救時務也至於禮樂則曰不敢是敢於殺人不敢於養人也為其俎豆管絃之間小不備【為于偽翻】因是絶而不為是去小不備而就大不備惑莫甚焉【為其不能具禮而廢禮是去小不備而就大不備也俎祭器如機盛牲體者也豆似籩亦所以盛肉籩用竹而豆用木管笙簫之屬也絃琴瑟之屬也】夫教化之比於刑法刑法輕是舍所重而急所輕也【師古曰舍廢也舍讀曰捨】教化所恃以為治也刑法所以助治也【治直吏翻】今廢所恃而獨立其所助非所以致太平也自京師有誖逆不順之子孫【師古曰誖乖也音布内翻】至於䧟大辟受刑戮者不絶由不習五常之道也【師古曰五常仁義禮智信人性之所常行也辟毗亦翻】夫承千歲之衰周繼暴秦之餘敝民漸漬惡俗貪饕險詖不閑義理【漸子亷翻師古曰貪甚曰饕言行險曰詖饕音吐高翻詖音皮義翻閑習也】不示以大化而獨以刑罰【敺讀與驅同】終已不改帝以向言下公卿議【下遐稼翻】丞相大司空奏請立辟雍按行長安城南營表未作而罷【師古曰營度地也表立標也行下孟翻】時又有言孔子布衣養徒三千人今天子太學弟子少【詩沼翻】於是增弟子員三千人歲餘復如故【元帝設弟子員千人】劉向自見得信於上故常顯訟宗室譏刺王氏及在位大臣其言多痛切發於至誠上數欲用向為九卿【數所角翻】輒不為王氏居位者及丞相御史所持【師古曰持謂扶持佐助也】故終不遷居列大夫官前後三十餘年而卒後十三歲而王氏代漢<br />
<br />
  資治通鑑卷三十二  <br>
   </div> 

<script src="/search/ajaxskft.js"> </script>
 <div class="clear"></div>
<br>
<br>
 <!-- a.d-->

 <!--
<div class="info_share">
</div> 
-->
 <!--info_share--></div>   <!-- end info_content-->
  </div> <!-- end l-->

<div class="r">   <!--r-->



<div class="sidebar"  style="margin-bottom:2px;">

 
<div class="sidebar_title">工具类大全</div>
<div class="sidebar_info">
<strong><a href="http://www.guoxuedashi.com/lsditu/" target="_blank">历史地图</a></strong>  
<a href="http://www.880114.com/" target="_blank">英语宝典</a>  
<a href="http://www.guoxuedashi.com/13jing/" target="_blank">十三经检索</a> 
<br><strong><a href="http://www.guoxuedashi.com/gjtsjc/" target="_blank">古今图书集成</a></strong> 
<a href="http://www.guoxuedashi.com/duilian/" target="_blank">对联大全</a> <strong><a href="http://www.guoxuedashi.com/xiangxingzi/" target="_blank">象形文字典</a></strong> 

<br><a href="http://www.guoxuedashi.com/zixing/yanbian/">字形演变</a>  <strong><a href="http://www.guoxuemi.com/hafo/" target="_blank">哈佛燕京中文善本特藏</a></strong>
<br><strong><a href="http://www.guoxuedashi.com/csfz/" target="_blank">丛书&方志检索器</a></strong> <a href="http://www.guoxuedashi.com/yqjyy/" target="_blank">一切经音义</a>  

<br><strong><a href="http://www.guoxuedashi.com/jiapu/" target="_blank">家谱族谱查询</a></strong>  <strong><a href="http://shufa.guoxuedashi.com/sfzitie/" target="_blank">书法字帖欣赏</a></strong> 
<br>

</div>
</div>


<div class="sidebar" style="margin-bottom:0px;">

<font style="font-size:22px;line-height:32px">QQ交流群9:489193090</font>


<div class="sidebar_title">手机APP 扫描或点击</div>
<div class="sidebar_info">
<table>
<tr>
	<td width=160><a href="http://m.guoxuedashi.com/app/" target="_blank"><img src="/img/gxds-sj.png" width="140"  border="0" alt="国学大师手机版"></a></td>
	<td>
<a href="http://www.guoxuedashi.com/download/" target="_blank">app软件下载专区</a><br>
<a href="http://www.guoxuedashi.com/download/gxds.php" target="_blank">《国学大师》下载</a><br>
<a href="http://www.guoxuedashi.com/download/kxzd.php" target="_blank">《汉字宝典》下载</a><br>
<a href="http://www.guoxuedashi.com/download/scqbd.php" target="_blank">《诗词曲宝典》下载</a><br>
<a href="http://www.guoxuedashi.com/SiKuQuanShu/skqs.php" target="_blank">《四库全书》下载</a><br>
</td>
</tr>
</table>

</div>
</div>


<div class="sidebar2">
<center>


</center>
</div>

<div class="sidebar"  style="margin-bottom:2px;">
<div class="sidebar_title">网站使用教程</div>
<div class="sidebar_info">
<a href="http://www.guoxuedashi.com/help/gjsearch.php" target="_blank">如何在国学大师网下载古籍?</a><br>
<a href="http://www.guoxuedashi.com/zidian/bujian/bjjc.php" target="_blank">如何使用部件查字法快速查字?</a><br>
<a href="http://www.guoxuedashi.com/search/sjc.php" target="_blank">如何在指定的书籍中全文检索?</a><br>
<a href="http://www.guoxuedashi.com/search/skjc.php" target="_blank">如何找到一句话在《四库全书》哪一页?</a><br>
</div>
</div>


<div class="sidebar">
<div class="sidebar_title">热门书籍</div>
<div class="sidebar_info">
<a href="/so.php?sokey=%E8%B5%84%E6%B2%BB%E9%80%9A%E9%89%B4&kt=1">资治通鉴</a> <a href="/24shi/"><strong>二十四史</strong></a>&nbsp; <a href="/a2694/">野史</a>&nbsp; <a href="/SiKuQuanShu/"><strong>四库全书</strong></a>&nbsp;<a href="http://www.guoxuedashi.com/SiKuQuanShu/fanti/">繁体</a>
<br><a href="/so.php?sokey=%E7%BA%A2%E6%A5%BC%E6%A2%A6&kt=1">红楼梦</a> <a href="/a/1858x/">三国演义</a> <a href="/a/1038k/">水浒传</a> <a href="/a/1046t/">西游记</a> <a href="/a/1914o/">封神演义</a>
<br>
<a href="http://www.guoxuedashi.com/so.php?sokeygx=%E4%B8%87%E6%9C%89%E6%96%87%E5%BA%93&submit=&kt=1">万有文库</a> <a href="/a/780t/">古文观止</a> <a href="/a/1024l/">文心雕龙</a> <a href="/a/1704n/">全唐诗</a> <a href="/a/1705h/">全宋词</a>
<br><a href="http://www.guoxuedashi.com/so.php?sokeygx=%E7%99%BE%E8%A1%B2%E6%9C%AC%E4%BA%8C%E5%8D%81%E5%9B%9B%E5%8F%B2&submit=&kt=1"><strong>百衲本二十四史</strong></a>  <a href="http://www.guoxuedashi.com/so.php?sokeygx=%E5%8F%A4%E4%BB%8A%E5%9B%BE%E4%B9%A6%E9%9B%86%E6%88%90&submit=&kt=1"><strong>古今图书集成</strong></a>
<br>

<a href="http://www.guoxuedashi.com/so.php?sokeygx=%E4%B8%9B%E4%B9%A6%E9%9B%86%E6%88%90&submit=&kt=1">丛书集成</a> 
<a href="http://www.guoxuedashi.com/so.php?sokeygx=%E5%9B%9B%E9%83%A8%E4%B8%9B%E5%88%8A&submit=&kt=1"><strong>四部丛刊</strong></a>  
<a href="http://www.guoxuedashi.com/so.php?sokeygx=%E8%AF%B4%E6%96%87%E8%A7%A3%E5%AD%97&submit=&kt=1">說文解字</a> <a href="http://www.guoxuedashi.com/so.php?sokeygx=%E5%85%A8%E4%B8%8A%E5%8F%A4&submit=&kt=1">三国六朝文</a>
<br><a href="http://www.guoxuedashi.com/so.php?sokeytm=%E6%97%A5%E6%9C%AC%E5%86%85%E9%98%81%E6%96%87%E5%BA%93&submit=&kt=1"><strong>日本内阁文库</strong></a> <a href="http://www.guoxuedashi.com/so.php?sokeytm=%E5%9B%BD%E5%9B%BE%E6%96%B9%E5%BF%97%E5%90%88%E9%9B%86&ka=100&submit=">国图方志合集</a> <a href="http://www.guoxuedashi.com/so.php?sokeytm=%E5%90%84%E5%9C%B0%E6%96%B9%E5%BF%97&submit=&kt=1"><strong>各地方志</strong></a>

</div>
</div>


<div class="sidebar2">
<center>

</center>
</div>
<div class="sidebar greenbar">
<div class="sidebar_title green">四库全书</div>
<div class="sidebar_info">

《四库全书》是中国古代最大的丛书,编撰于乾隆年间,由纪昀等360多位高官、学者编撰,3800多人抄写,费时十三年编成。丛书分经、史、子、集四部,故名四库。共有3500多种书,7.9万卷,3.6万册,约8亿字,基本上囊括了古代所有图书,故称“全书”。<a href="http://www.guoxuedashi.com/SiKuQuanShu/">详细>>
</a>

</div> 
</div>

</div>  <!--end r-->

</div>
<!-- 内容区END --> 

<!-- 页脚开始 -->
<div class="shh">

</div>

<div class="w1180" style="margin-top:8px;">
<center><script src="http://www.guoxuedashi.com/img/plus.php?id=3"></script></center>
</div>
<div class="w1180 foot">
<a href="/b/thanks.php">特别致谢</a> | <a href="javascript:window.external.AddFavorite(document.location.href,document.title);">收藏本站</a> | <a href="#">欢迎投稿</a> | <a href="http://www.guoxuedashi.com/forum/">意见建议</a> | <a href="http://www.guoxuemi.com/">国学迷</a> | <a href="http://www.shuowen.net/">说文网</a><script language="javascript" type="text/javascript" src="https://js.users.51.la/17753172.js"></script><br />
  Copyright &copy; 国学大师 古典图书集成 All Rights Reserved.<br>
  
  <span style="font-size:14px">免责声明:本站非营利性站点,以方便网友为主,仅供学习研究。<br>内容由热心网友提供和网上收集,不保留版权。若侵犯了您的权益,来信即刪。scp168@qq.com</span>
  <br />
ICP证:<a href="http://www.beian.miit.gov.cn/" target="_blank">鲁ICP备19060063号</a></div>
<!-- 页脚END --> 
<script src="http://www.guoxuedashi.com/img/plus.php?id=22"></script>
<script src="http://www.guoxuedashi.com/img/tongji.js"></script>

</body>
</html>
