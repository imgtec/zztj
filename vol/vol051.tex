資治通鑑卷五十一
宋 司馬光 撰

胡三省 音註

漢紀四十三|{
	起旃蒙赤奮若盡昭陽作噩凡九年}


孝安皇帝下

延光四年春二月乙亥下邳惠王衍薨 甲辰車駕南廵 三月戊午朔日有食之 庚申帝至宛不豫|{
	宛於元翻書金縢王有疾弗豫孔安國注曰不悦豫}
乙丑帝發自宛丁卯至葉崩于乘輿|{
	葉式涉翻乘繩證翻}
年三十二皇后與閻顯兄弟江京樊豐等謀曰今晏駕道次|{
	賢曰晏晚也臣下不敢斥言帝崩猶言晚駕而出道次猶言路次也}
濟隂王在内邂逅公卿立之還為大害|{
	邂下廨翻逅戶茂翻}
乃偽云帝疾甚徙御卧車所在上食問起居如故|{
	上時掌翻}
驅馳行四日庚午還宫|{
	自葉至雒陽六百餘里}
辛未遣司徒劉熹詣郊廟社稷告天請命|{
	武王冇疾周公為三壇同墠因太王王季文王以請命于天後世踵而行之}
其夕發喪尊皇后曰皇太后太后臨朝以顯為車騎將軍儀同三司太后欲久專國政貪立幼年與顯等定策禁中迎濟北惠王子北鄉侯懿為嗣|{
	賢曰惠王名夀章帝子也濟子禮翻考異曰東觀記續漢書作北鄉侯犢今從袁紀范書}
濟隂王以廢黜不得上殿親臨梓宫|{
	臨力鴆翻}
悲號不食|{
	號戶刀翻}
内外羣僚莫不哀之 甲戌濟南孝王香薨無子國絶|{
	香濟南安王康之孫康光武子也}
乙酉北鄉侯即皇帝位 夏四月丁酉太尉馮石為太傅司徒劉熹為太尉參録尚書事前司空李郃為司徒|{
	郃古合翻又曷閤翻}
閻顯忌大將軍耿寶位尊權重威行前朝|{
	朝直遥翻}
乃風有司奏寶及其黨與|{
	風讀曰諷}
中常侍樊豐虎賁中郎將謝惲侍中周廣野王君王聖聖女永等更相阿黨互作威福皆大不道辛卯豐惲廣皆下獄死|{
	更工衡翻下遐稼翻}
家屬徙比景貶寶及弟子林慮侯承皆為亭侯|{
	牟平侯耿舒子襲尚顯宗女隆慮公主寶嗣襲封而弟子承紹公主封為林慮侯林慮即隆慮也避殤帝諱改隆為林慮音廬}
遣就國寶於道自殺王聖母子徙雁門於是以閻景為衛尉耀為城門校尉晏為執金吾兄弟並處權要|{
	處昌呂翻}
威福自由 己酉葬孝安皇帝于恭陵|{
	賢曰恭陵在今洛陽東北二十七里}
廟曰恭宗 六月乙巳赦天下 秋七月西域長史班勇發敦煌張掖酒泉六千騎及鄯善疏勒車師前部兵擊後部王軍就大破之|{
	敦徒門翻鄯上扇翻}
獲首虜八千餘人生得軍就及匈奴持節使者將至索班沒處斬之傳首京師|{
	索班事見上卷永寧元年}
冬十月丙午越巂山崩|{
	巂音髓}
北鄉侯病篤中常侍孫程謂濟隂王謁者長興渠曰|{
	賢曰興姓渠名余按百官志王國謁者比四百石其下有禮樂長衛士長醫工長永巷長祠祀長而無謁者長竊意長興姓也}
王以嫡統本無失德先帝用讒遂至廢黜若北鄉侯不起相與共斷江京閻顯事無不成者|{
	斷丁亂翻}
渠然之又中黄門南陽王康先為太子府史|{
	太子府史掌東宫府藏}
及長樂太官丞京兆王國等|{
	長樂太官丞掌太后食膳樂音洛}
並附同於程|{
	附同者既相黨附又與之同謀}
江京謂閻顯曰北鄉侯病不解|{
	解散也言病纒於身而不散也}
國嗣宜以時定何不早徵諸王子簡所置乎|{
	簡擇也置立也}
顯以為然辛亥北鄉侯薨顯白太后祕不發喪更徵諸王子閉宫門屯兵自守十一月乙卯孫程王康王國與中黄門黄龍彭愷孟叔李建王成張賢史汎馬國王道李元楊佗陳予趙封李剛魏猛苖光等聚謀於西鍾下皆截單衣為誓丁巳京師及郡國十六地震是夜程等共會崇德殿上|{
	崇德殿在南宫水經注魏文帝於漢崇德殿故處起大極殿蓋南宮正殿也}
因入章臺門時江京劉安及李閏陳達等俱坐省門下|{
	省門即禁門也前書謂禁中為省中}
程與王康共就斬京安達以李閏權勢積為省内所服欲引為主因舉刃脅閏曰今當立濟隂王母得揺動閏曰諾于是扶閏起俱於西鍾下迎濟隂王即皇帝位時年十一召尚書令僕射以下從輦幸南宫程等留守省門遮扞内外帝登雲臺召公卿百僚使虎賁羽林士屯南北宫諸門閻顯時在禁中憂迫不知所為|{
	顯盖在北宫}
小黄門樊登勸顯以太后詔召越騎校尉馮詩虎賁中郎將閻崇將兵屯平朔門以禦程等|{
	考異曰宦者傳作朔平門今從袁紀予按百官志朔平門北宫北門也恐當以宦者傳為是}
顯誘詩入省謂曰濟隂王立非皇太后意璽綬在此|{
	誘音酉璽斯氏翻綬音受此謂天子璽綬也}
苟盡力効功封侯可得太后使授之印曰能得濟隂王者封萬戶侯得李閏者五千戶侯詩等皆許諾辭以卒被召所將衆少|{
	卒讀曰猝}
顯使與登迎吏士於左掖門外詩因格殺登歸營屯守顯弟衛尉景遽從省中還外府|{
	外府衛尉府也}
收兵至盛德門孫程傳召諸尚書使收景|{
	傳召傳詔召之也}
尚書郭鎮時卧病聞之即率直宿羽林出南止車門逢景從吏士拔白刃呼曰無干兵鎮即下車持節詔之景曰何等詔因斫鎮不中|{
	呼火故翻中竹仲翻}
鎮引劒擊景墮車左右以戟义其胷遂禽之送廷尉獄即夜死戊午遣使者入省奪得璽綬帝乃幸嘉德殿|{
	按帝紀嘉德殿在南宫}
遣侍御史持節收閻顯及其弟城門校尉耀執金吾晏並下獄誅|{
	下遐稼翻}
家屬皆徙比景遷太后於離宫己未開門罷屯兵壬戌詔司隸校尉惟閻顯江京近親當伏辜誅其餘務崇寛貸封孫程等皆為列侯程食邑萬戶王康王國食九千戶黄龍食五千戶彭愷孟叔李建食四千二百戶王成張賢史汎馬國王道李元楊佗陳予趙封李剛食四千戶魏猛食二千戶苖光食千戶是為十九侯|{
	孫程為浮陽侯王康為華容侯王國為酈侯黄龍為湘南侯彭愷為西平昌侯孟叔為中廬侯李建為復陽侯王成為廣宗侯張賢為祝阿侯史汎為臨沮侯馬國為廣平侯王道為范縣侯李元為褒信侯楊佗為山都侯陳予為下巂侯趙封為折縣侯李剛為枝江侯魏猛為夷陵侯苖光為東阿侯}
加賜車馬金銀錢帛各有差李閏以先不預謀故不封擢孫程為騎都尉初程等入章臺門苖光獨不入詔書録功臣令王康疏名康詐疏光入章臺門光未受符策|{
	漢初封王侯皆剖符至武帝封齊燕廣陵三王始作策}
心不自安詣黄門令自告|{
	黄門令主省中諸宦者故詣之自告}
有司奏康光欺詐主上詔書勿問以將作大匠來歷為衛尉祋諷閭丘弘等先卒皆拜其子為郎朱倀施延陳光趙代皆見拔用後至公卿|{
	以來歷等鴻都門之諫也事見上卷祋丁外翻又丁活翻倀丑羊翻}
徵王男邴吉家屬還京師厚加賞賜|{
	男吉家徒事見上卷上年}
帝之見廢也監太子家小黄門籍建傳高梵|{
	監古銜翻賢曰梵音扶汎翻予按來歷傳傳中傳也梵又房戌翻}
長秋長趙熹丞良賀|{
	良姓也左傳鄭良霄穆公子子良之孫}
藥長夏珍皆坐徙朔方|{
	長秋長盖即大長秋丞一人六百石中宫藥長四百石皆皇后宫官}
帝即位並擢為中常侍初閻顯辟崔駰之子瑗為吏|{
	駰音因}
瑗以北鄉侯立不以正知顯將敗欲說令廢立而顯日沉醉|{
	說輔芮翻下同沉持林翻}
不能得見乃謂長史陳禪曰中常侍江京等惑蠱先帝廢黜正統扶立疎孽|{
	孔穎達曰孽者蘖也樹木斬而復生謂之蘖以嫡子比根榦庶子比枝蘖故孽子枝庶也中候曰無易子注云樹子適子玉藻云公子曰臣孽注孽當為枿文王曰本支百世是適子比樹本庶子比技蘖也}
少帝即位發病廟中周勃之徵於斯復見|{
	賢曰呂后立惠帝後宫子為少帝周勃廢之也}
今欲與君共求見說將軍|{
	說式芮翻}
白太后收京等廢少帝引立濟隂王必上當天心下合人望伊霍之功不下席而立則將軍兄弟傳於無窮若拒違天意久曠神器則將以無辠并辜元惡|{
	元惡大惡也并辜謂與之同獲罪也}
此所謂禍福之會分功之時也|{
	史記蔡澤說范睢曰君獨不見夫博者乎或欲大投或欲分功今君相秦坐制諸侯使天下皆畏秦此亦秦之分功之時也}
禪猶豫未敢從會顯敗瑗坐被斥|{
	被皮義翻}
門生蘇祗欲上書言狀瑗遽止之時陳禪為司隸校尉召瑗謂曰弟聼祗上書|{
	賢曰弟但也司馬相如傳曰弟俱如臨卭弟讀如第}
禪請為之證瑗曰此譬猶兒妾屏語耳|{
	屏必郢翻於隱屏之處相與私語也}
願使君勿復出口|{
	禪時為司隸校尉故稱之曰使君司隸校尉部察三輔三河弘農其職猶十三部使者鮑永為司隸校尉光武曰奉使如此何如復扶又翻下同}
遂辭歸不復應州郡命 己卯以諸王禮葬北鄉侯 司空劉授以阿附惡逆辟召非其人策免|{
	辟召非人事見上卷延光二年}
十二月甲申以少府河南陶敦為司空 楊震門生虞放陳翼詣闕追訟震事|{
	事見上卷上年}
詔除震二子為郎贈錢百萬以禮改葬於華隂潼亭|{
	賢曰墓在今潼關西大道之北其碑尚存華戶化翻}
遠近畢至有大鳥高丈餘集震喪前|{
	高居號翻}
郡以狀上|{
	上時掌翻}
帝感震忠詔復以中牢具祠之|{
	中牢即少牢羊豕具也復扶又翻下同}
議郎陳禪以為閻太后與帝無母子恩宜徙别館絶朝見|{
	朝直遥翻見賢遍翻}
羣臣議者咸以為宜司徒掾汝南周舉謂李郃曰昔瞽瞍常欲殺舜舜事之逾謹|{
	瞽瞍使舜塗廩而自下焚廩使浚井既入從而揜之其欲殺者屢矣而舜事瞽瞍彌謹書曰祗載見瞽瞍夔夔齋栗掾俞絹翻郃曷閤翻又古合翻}
鄭武姜謀殺莊公莊公誓之黄泉秦始皇怨母失行|{
	行上孟翻}
久而隔絶後感潁考叔茅焦之言復修子道書傳美之|{
	鄭武姜愛少子共叔段謀襲莊公公寘姜氏於城潁而誓之曰不及黄泉無相見也潁考叔以舍肉遺母感之遂為母子如初秦始皇事見六卷九年}
今諸閻新誅太后幽在離宫若悲愁生疾一旦不虞主上將何以令於天下如從禪議後世歸咎明公宜密表朝廷令奉太后率羣臣朝覲如舊以厭天心以荅人望|{
	厭如字滿也}
郃即上疏陳之

孝順皇帝上|{
	諱保安帝之子也諡法慈和徧服曰順伏侯古今注日保之字曰守}


永建元年春正月帝朝太后於東宫太后意乃安 甲寅赦天下 辛未皇太后閻氏崩 辛巳太傅馮石太尉劉熹以阿黨權貴免司徒李郃罷 二月甲申葬安思皇后|{
	賢曰諡法謀慮不愆曰思}
丙戌以太常桓焉為太傅大鴻臚朱寵為太尉參録尚書事長樂少府朱倀為司徒|{
	臚陵如翻樂音洛倀丑羊翻}
封尚書郭鎮為定潁侯|{
	以禽閻景功也定潁侯國屬汝南郡}
隴西鍾羌反校尉馬賢擊之戰於臨洮斬首千餘級羌衆皆降由是涼州復定|{
	洮土刀翻降戶江翻復扶又翻}
六月己亥封濟南簡王錯子顯為濟南王|{
	安帝延光四年濟南國絶今紹封諡法一德不懈曰簡又臣謚恭敬行善曰簡}
秋七月庚午以衛尉來歷為車騎將軍 八月鮮卑寇代郡太守李超戰歿 司隸校尉虞詡到官數月奏馮石劉熹免之又劾奏中常侍程璜陳秉孟生李閏等|{
	劾戶概翻又戶得翻}
百官側目號為苛刻三公劾奏詡盛夏多拘繫無辜為吏民患|{
	三公欲致詡罪言盛夏當順天地長物之性不當違法拘繫無辜劾戶槩翻又戶得翻}
詡上書自訟 |{
	考異曰詡傳曰帝省其章乃為免司空陶敦按袁紀孫程就國在九月而敦免在十月盖帝由此知敦不直因事免之不然何三府共奏而獨免敦也}
曰法禁者俗之隄防刑罰者民之銜轡今州曰任郡郡曰任縣更相委遠|{
	更工衡翻遠于願翻}
百姓怨窮以苟容為賢盡節為愚臣所發舉臧罪非一|{
	臧古贓字通}
三府恐為臣所奏遂加誣罪臣將從史魚死即以尸諫耳|{
	韓詩外傳曰衛大夫史魚病且死謂其子曰我數言蘧伯玉之賢而不能進彌子瑕不肖而不能退為人臣生不能進賢退不肖死不當治喪正堂殯我於室足矣君問其故子以父言聞君乃立召蘧伯玉而貴之斥彌子瑕而退之徙殯於正堂成禮而後去}
帝省其章乃不罪詡|{
	省悉景翻}
中常侍張防賣弄權埶請託受取詡案之屢寑不報詡不勝其憤|{
	勝音升}
乃自繫廷尉奏言昔孝安皇帝任用樊豐交亂嫡統幾亡社稷|{
	事見上卷安帝延光三年幾居希翻}
今者張防復弄威柄|{
	復扶又翻下同}
國家之禍將重至矣|{
	重直用翻}
臣不忍與防同朝謹自繫以聞無令臣襲楊震之跡|{
	楊震事見上卷延光三年}
書奏防流涕訴帝詡坐論輸左校|{
	將作大匠有左校令掌左工徒輸左校者免官為徒輸作左校也校戶教翻}
防必欲害之二日之中傳考四獄獄吏勸詡自引|{
	自引謂引分自裁也傳株戀翻}
詡曰寧伏歐刀以示遠近|{
	謂寧受刑而死於市也}
喑嗚自殺|{
	類篇曰啼泣無聲謂之喑歎傷謂之嗚}
是非孰辨邪浮陽侯孫程祝阿侯張賢相率乞見|{
	浮陽侯國屬勃海郡見賢遍翻}
程曰陛下始與臣等造事之時|{
	賢曰謂帝被廢程等謀立之時也}
常疾姧臣知其傾國今者即位而復自為何以非先帝乎司隸校尉虞詡為陛下盡忠|{
	為于偽翻}
而更被拘繫常侍張防臧罪明正反搆忠良今客星守羽林|{
	史記天官書虚危南有衆星曰羽林晉書天文志羽林四十五星在營室南}
其占宫中有姧臣宜急收防送獄以塞天變|{
	塞悉則翻}
時防立在帝後程叱防曰姧臣張防何不下殿防不得已趨就東箱|{
	賢曰埤蒼云箱序也字或作廂}
程曰陛下急收防無令從阿母求請|{
	賢曰阿母宋娥也}
帝問諸尚書尚書賈朗素與防善證詡之辠帝疑焉謂程曰且出吾方思之於是詡子顗|{
	顗既豈翻}
與門生百餘人舉幡候中常侍高梵車叩頭流血訴言枉狀梵入言之|{
	梵房戎翻又房汎翻}
防坐徙邊賈朗等六人或死或黜即日赦出詡程復上書陳詡有大功語甚切激|{
	復扶又翻}
帝感悟復徵拜議郎數日遷尚書僕射詡上疏薦議郎南陽左雄曰臣見方今公卿以下類多拱默|{
	拱默言拱手而默無一言}
以樹恩為賢盡節為愚至相戒曰白璧不可為容容多後福|{
	賢曰容容猶和同也言不可為白璧之清潔常與衆人和同也}
伏見議郎左雄有王臣蹇蹇之節|{
	易曰王臣蹇蹇匪躬之故}
宜擢在喉舌之官|{
	東都謂尚書為喉舌之官以其出納王命也}
必有匡弼之益由是拜雄尚書 浮陽侯孫程等懷表上殿争功帝怒有司劾奏程等干亂悖逆|{
	劾戶概翻又戶得翻悖蒲没翻又蒲内翻}
王國等皆與程黨久留京都益其驕恣帝乃免程等官悉徙封遠縣因遣十九侯就國敇雒陽令促期發遣司徒掾周舉說朱倀曰朝廷在西鍾下時非孫程等豈立|{
	東都謂天子為國家又謂為朝廷說輸芮翻倀丑羊翻}
今忘其大德録其小過如道路夭折|{
	夭於紹翻}
帝有殺功臣之譏及令未去宜急表之倀曰今詔指方怒吾獨表此必致罪譴舉曰明公年過八十位為台輔不於今時竭忠報國惜身安寵欲以何求禄位雖全必陷佞邪之譏諫而獲罪猶有忠貞之名若舉言不足採請從此辭倀乃表諫帝果從之程徙封宜城侯|{
	宜城縣属南郡春秋之羅國也 考異曰袁紀秋七月有司奏浮陽侯孫程祝阿侯張賢為司隸校尉虞詡訶叱左右謗訕大臣妄造不祥干亂悖逆王國等皆與程黨久留京師益其驕溢詔免程等徙為都梁侯程怨恨封還印綬更封為宜城侯范書孫程傳亦云坐訟虞詡呵叱左右就國按虞詡傳程言見用上不以為怒周舉傳云程坐争功就國今從之}
到國怨恨恚懟|{
	恚於避翻懟直類翻}
封還印綬符策亡歸京師往來山中詔書追求復故爵土賜車馬衣物遣還國 冬十月丁亥司空陶敦免 朔方以西障塞多壞鮮卑因此數侵南匈奴|{
	數所角翻}
單于憂恐上書乞修復障塞庚寅詔黎陽營兵出屯中山北界|{
	賢曰黎陽先置營兵以南單于求復障塞恐入侵擾亂置屯兵於中山北界舊中山郡今之定州是也予謂移黎陽營屯中山北界不過為南部聲援耳}
令緣邊郡增置步兵列屯塞下教習戰射 以廷尉張皓為司空班勇更立車師後部故王子加特奴為王|{
	更工衡翻下同}
勇

又使别校誅斬東且彌王|{
	校戶教翻且子余翻范書東且彌國去洛陽九千二百里}
亦更立其種人為王|{
	種章勇翻}
于是車師六國悉平|{
	西域傳卑陸蒲類東且彌移支車師前後王是為六國}
勇遂發諸國兵擊匈奴呼衍王亡走其衆二萬餘人皆降生得單于從兄勇使加特奴手斬之以結車師匈奴之隙北單于自將萬餘騎入後部至金且谷|{
	且于余翻}
勇使假司馬曹俊救之單于引去俊追斬其貴人骨都侯於是呼衍王遂徙居枯梧河上是後車師無復虜跡

二年春正月中郎將張國以南單于兵擊鮮卑其至鞬破之|{
	鞬居言翻}
二月遼東鮮卑寇遼東玄菟|{
	菟同都翻}
烏桓校尉耿曅發緣邉諸郡兵及烏桓出塞擊之斬獲甚衆鮮卑三萬人詣遼東降|{
	降戶江翻下同}
三月旱 初帝母李氏瘞在雒陽北|{
	李氏死見上卷安帝元初二年瘞於計翻}
帝初不知至是左右白之帝乃發哀親到瘞所更以禮殯|{
	殯用皇后禮瘞於計翻}
六月乙酉追諡為恭愍皇后葬于恭陵之北 西域城郭諸國皆服於漢唯焉耆王元孟未降|{
	元孟和帝永元六年班超所立也}
班勇奏請攻之於是遣敦煌太守張朗將河西四郡兵三千人配勇|{
	敦徒門翻}
因發諸國兵四萬餘人分為兩道擊之勇從南道朗從北道約期俱至焉耆而朗先有罪欲徼功自贖|{
	徼一遥翻}
遂先期至爵離關|{
	釋氏西域記龜兹國北四十里山上有寺名雀離大凊浄耆悉薦翻}
遣司馬將兵前戰獲首虜二千餘人元孟懼誅逆遣使乞降張朗逕入焉耆受降而還朗得免誅勇以後期徵下獄免|{
	夏之政典曰先時者殺無赦不及時者殺無赦張朗先期以徼功法所必誅則班勇非後期也漢之用刑不審厥衷勇免之後西域事去矣下遐稼翻}
秋七月甲戌朔日有食之 壬午太尉朱寵司徒朱倀免庚子以太常劉光為太尉録尚書事光禄勲汝南許敬為司徒光矩之弟也敬仕於和安之間當竇鄧閻氏之盛無所屈橈|{
	橈奴教翻}
三家既敗士大夫多染汚者|{
	汚烏故翻}
獨無謗言及於敬當世以此貴之 初南陽樊英少有學行|{
	少詩照翻行下孟翻}
名著海内隱於壺山之陽|{
	賢曰壺山在今鄧州新城縣北即張衡南都賦所云天封大狐是也}
州郡前後禮請不應公卿舉賢良方正有道皆不行安帝賜策書徵之不赴是歲帝復以策書玄纁備禮徵英|{
	復扶又翻}
英固辭疾篤詔切責郡縣駕載上道英不得已到京稱疾不肯起彊輿入殿|{
	彊其兩翻}
猶不能屈帝使出就太醫養疾|{
	大醫令屬少府掌諸醫冇藥丞方丞}
月致羊酒其後帝乃為英設壇|{
	為于偽翻}
令公車令導尚書奉引|{
	引與靷同音羊晉翻}
賜几杖待以師傅之禮 |{
	考異曰英傳云四年三月乃設壇場見英黄瓊傳李固勸書已云樊英設壇席及瓊至上疏薦英稱光禄大夫則是瓊至之時英已嘗設壇見之而為光禄大夫矣至三年旱瓊復上疏若四年方設壇場見英則都與瓊傳異知其必不在四年也}
延問得失拜五官中郎將數月英稱疾篤詔以為光禄大夫賜告歸令所在送穀以歲時致牛酒英辭位不受有詔譬旨勿聼|{
	有詔書譬曉以上旨不聼其辭位也}
英初被詔命|{
	被皮義翻下同}
衆皆以為必不降志南郡王逸素與英善因與其書多引古譬諭勸使就聘英順逸議而至及後應對無奇謀深策談者以為失望河南張楷與英俱徵謂英曰天下有二道出與處也|{
	處昌呂翻下同}
吾前以子之出能輔是君也濟斯民也而子始以不訾之身|{
	賢曰訾量也言無量可比之貴重之極也訾音資}
怒萬乘之主|{
	按英傳英彊輿入殿猶不以禮屈帝怒謂英曰朕能生君能殺君能貴君能賤君能富君能貧君君何以慢朕命英曰臣受命於天生盡其命天也死不得其命亦天也陛下焉能生臣焉能殺臣臣見暴君如見仇讐立其朝猶不肯可得而貴乎雖在布衣之列環堵之中晏然自得不易萬乘之尊又可得而賤乎陛下焉能貴臣焉能賤臣非禮之禄雖萬鍾不受也申其志雖簞食不厭也陛下焉能富臣焉能貧臣乎帝不能屈而敬其名使出就太醫養疾月致羊酒}
及其享受爵禄又不聞匡救之術進退無所據矣臣光曰古之君子邦有道則仕邦無道則隱隱非君子之所欲也人莫已知而道不得行羣邪共處|{
	處昌呂翻}
而害將及身故深藏以避之王者舉逸民揚仄陋|{
	論語曰舉逸民天下之民歸心焉堯典曰明明揚側陋}
固為其有益於國家|{
	為于偽翻}
非以狥世俗之耳目也是故有道德足以尊主智能足以庇民被褐懷玉深藏不市|{
	聖人被褐懷玉玉至寶也被褐而懷之喻珍美不外見也良賈深藏若虛賈有善貨深藏若無所有者不得善賈則不售此皆以喻抱道懷才之士被皮義翻}
則王者當盡禮而致之屈己以訪之克己以從之然後能利澤施于四表功烈格于上下盖取其道不取其人務其實不務其名也其或禮備而不至意勤而不起則姑内自循省而不敢彊致其人|{
	省悉景翻彊其兩翻}
曰豈吾德之薄而不足慕乎政之亂而不可輔乎羣小在朝而不敢進乎誠心不至而憂其言之不用乎何賢者之不我從也苟其德已厚矣政已治矣羣小遠矣|{
	治直吏翻遠于願翻}
誠心至矣彼將扣閽而自售又安有勤求而不至者哉荀子曰耀蟬者務在明其火振其木而已火不明雖振其木無益也|{
	楊倞曰南方人照蟬取而食之禮記有蜩范是也}
今人主有能明其德則天下歸之若蟬之歸明火也或者人主耻不能致乃至誘之以高位脅之以嚴刑|{
	公孫述之待李業諸人政如此誘音酉}
使彼誠君子邪則位非所貪刑非所畏終不可得而致也可致者皆貪位畏刑之人也烏足貴哉若乃孝弟著於家庭行誼隆於鄉曲|{
	弟讀曰悌行下孟翻}
利不苟取仕不苟進潔已安分優游卒歲|{
	分扶問翻卒子恤翻}
雖不足以尊主庇民是亦清修之吉士也王者當褒優安養俾遂其志若孝昭之待韓福|{
	昭帝元鳳元年三月賜郡國所選有行義者涿郡韓福等五人帛人五十匹遣歸詔曰朕閔勞以官職之事其務修孝弟以教鄉里令郡縣常以正月賜羊酒其有不幸者賜衣一襲祠以中牢}
光武之遇周黨|{
	事見四十一卷建武五年}
以勵廉耻美風俗斯亦可矣固不當如范升之詆毁又不可如張楷之責望也至于飾偽以邀譽釣奇以驚俗不食君祿而争屠沽之利不受小官而規卿相之位名與實反心與迹違斯乃華士少正卯之流|{
	韓非子曰太公封於齊東海上有華矞華士昆弟二人太公殺}


|{
	之周公急傳而問曰二子皆賢人殺之何也太公曰是昆弟立議曰不臣天子是望不得而臣也不友諸侯是望不得而友也耕而食之掘而飲之無求於人是望不得以賞罰勸禁也且聖王所以使人非爵賞則刑罰也今四者不足以使之則望誰為君乎是以誅之也荀子曰孔子為魯相七日而誅少正卯門人進問曰夫少正卯魯之聞人也夫子為政而始誅之得無失乎孔子曰其有惡者五而盗竊不與焉一曰心逹而險二曰行僻而堅三曰言偽而辯四曰記醜而博五曰順非而澤此五者有一於人則不得免於君子之誅而少正卯兼有之}
其得免於聖王之誅幸矣尚何聘召之有哉

時又徵廣漢楊厚江夏黄瓊瓊香之子也厚既至豫陳漢有三百五十年之戹以為戒|{
	賢曰春秋命歷序曰四百年之間閉四門聼外難羣異並賊官有孽臣州有兵亂五七弱暴漸之效也宋均注云五七三百五十歲當順帝漸微四方多逆賊也}
拜議郎瓊將至李固以書逆遺之曰|{
	遺于季翻}
君子謂伯夷隘柳下惠不恭不夷不惠可否之間|{
	孟子曰伯夷隘柳下惠不恭隘與不恭君子不由也賢曰論語孔子曰伯夷叔齊不降其志不辱其身謂柳下惠少連降志辱身矣我則異於是無可無不可鄭玄注云不為夷齊之清不為惠連之屈故曰異於是也}
聖賢居身之所珍也誠欲枕山棲谷|{
	枕之鴆翻}
擬迹巢由斯則可矣若當輔政濟民今其時也自生民以來善政少而亂俗多必待堯舜之君此為士行其志終無時矣嘗聞語曰嶢嶢者易缺皦皦者易汙|{
	嶢嶢山之高也皦皦王石 之白也嶢倪么翻易以豉翻}
盛名之下其實難副近魯陽樊君被徵初至|{
	被皮義翻}
朝廷設壇席猶待神明雖無大異而言行所守亦無所缺|{
	行下孟翻}
而毁謗布流應時折減者|{
	言其名譽折減也折食列翻}
豈非觀聼望深聲名太盛乎|{
	言其聲名之盛素動人之觀聼故所望者深也}
是故俗論皆言處士純盗虚聲|{
	處昌呂翻}
願先生宏此遠謨令衆人歎服一雪此言耳瓊至拜議郎稍遷尚書僕射瓊昔隨父在臺閣|{
	瓊父香和帝時為尚書令}
習見故事及後居職逹練官曹|{
	逹明也練習也言明習尚書諸曹事也}
争議朝堂莫能抗奪|{
	莫能抗言以奪其議也朝直遥翻}
數上疏言事|{
	數所角翻}
上頗采用之李固郃之子|{
	郃曷閤翻又古合翻}
少好學|{
	少詩沼翻好呼到翻}
常改易姓名杖策驅驢|{
	策馬策也}
負笈從師不遠千里|{
	笈極曄翻書箱也不遠千里不憚千里之遠也}
遂究覽墳籍為世大儒每到太學密入公府定省父母|{
	記曰凡為人子冬温而夏凊昏定而晨省孔頴逹曰安定其牀衽也省問其安否何如省悉景翻}
不令同業諸生知其為郃子也三年春正月丙子京師地震 夏六月旱 秋七月茂陵園寑災 九月鮮卑寇漁陽 冬十二月己亥太傅桓焉免 車騎將軍來歷罷 南單于拔死弟休利立為去特若尸逐就單于 帝悉召孫程等還京師四年春正月丙寅赦天下 丙子帝加元服 夏五月壬辰詔曰海内頗有災異朝廷修政太官減膳珍玩不御|{
	御進也}
而桂陽太守文礱|{
	郡國志桂陽郡在雒陽南三千九百里礱音力公翻}
不惟竭忠宣暢本朝|{
	言不思宣暢本朝遇災修省之意也朝直遥翻}
而遠獻大珠以求幸媚今封以還之|{
	不罪礱而但封還其珠非所以昭德塞違也}
五州雨水 秋八月丁巳太尉劉光司空張皓免 尚書僕射虞詡上言安定北地上郡山川險阨沃埜千里土宜畜牧水可溉漕|{
	既可溉田又可通漕也畜許六翻}
頃遭元元之災|{
	洪氏隸釋曰東漢書鄧隲傳元二之災注云元二即元元也古書字當再讀者於上字之下為小二字當兩度言之後人不曉遂讀為元二或同之陽九或附之百六良由不悟致斯乖舛岐州石鼔銘凡重言者皆為二字明驗也趙氏云楊孟文石門碑漢威宗建和二年立其文有曰申遭元二橋梁斷絶若讀為元元則為不成文理疑當時自有此語漢注未必然也予按漢刻如北海相景君及李翊夫人碑之類凡重文皆以小二字贅其下此碑有烝烝明明蕩蕩世世勤勤亦不再出上一字然非若元二遂書為二大字也又孔耽碑云遭元二轗軻人民相食若作元元則下文不應言人民漢注之非明矣王充論衡曰今上嗣位元二之間嘉德布流三年零陵生芝草五本四年甘露降五縣則論衡所云元二者蓋謂即位之元年二年也鄧君傳云永初元年夏涼部叛羌揺蕩西州詔隲將羽林五校士擊之冬徵隲班師迎拜為大將軍帝紀班師在二年十一月傳有脱字也又云時遭元二之災人士荒饑盗賊羣起四夷侵叛隲崇節儉罷力役進賢士故天下復安則此傳所云元二者亦謂元年二年也安帝紀書兩年之間萬民饑流羌貊叛戾又與傳同此碑所云西戎虐殘橋梁斷絶正是鄧隲出師時則史傳碑碣皆與論衡合建初者章帝之始年永初者安帝之始年乃知東漢之文所謂元二者如此}
衆羌内潰郡縣兵荒二十餘年夫弃沃壤之饒捐自然之財不可謂利離河山之阻|{
	離力智翻}
守無險之處難以為固今三郡未復園陵單外|{
	賢曰園陵謂長安諸陵園也單外謂不固余謂西漢諸陵園不皆在長安單外言無蔽障}
而公卿選懦容頭過身|{
	賢曰前書音義曰選懦柔怯也懦音而掾翻}
張解設難|{
	張解者開張其說以為解設難者鋪設其辭以發難難乃旦翻}
但計所費不圖其安宜開聖聼考行所長九月詔復安定北地上郡還舊土|{
	安帝永初五年三郡内徙}
癸酉以大鴻臚龎參為太尉録尚書事|{
	臚陵如翻龎皮江翻}
太常王龔為司空 冬十一月庚辰司徒許敬免 鮮卑寇朔方 十二月乙卯以宗正弘農劉崎為司徒|{
	崎邱宜翻}
是歲于窴王放前殺拘彌王興自立其子為拘彌王|{
	拘彌王居寧彌城去長史所居柳中城四千九百里}
而遣使者貢獻敦煌太守徐由上求討之|{
	敦徒門翻上時掌翻}
帝赦于窴罪令歸拘彌國放前不肯

五年夏四月京師旱 京師及郡國十二蝗 定遠侯班超之孫始尚帝姑隂城公主|{
	公主清河孝王之女隂縣屬南陽郡宋白曰隂城縣在今穀城縣北宋乾德二年置光化軍}
主驕淫無道始積忿怒伏刃殺主冬十月乙亥始坐腰斬同產皆棄市

六年春二月庚午河間孝王開薨子政嗣政傲狠不奉法|{
	狠下墾翻}
帝以侍御史吳郡沈景有彊能擢為河間相|{
	侍御史秩六百石擢為王國相秩二千石相息亮翻}
景到國謁王王不正服箕踞殿上侍郎贊拜景峙不為禮|{
	賢曰峙立也}
問王所在虎賁曰是非王邪|{
	賁音奔}
景曰王不正服常人何别今相謁王豈謁無禮者邪王慙而更服|{
	别彼列翻更工衡翻}
景然後拜出住宫門外請王傅責之|{
	漢諸王國有太傅至成帝時更曰傅}
曰前發京師陛見受詔|{
	見賢遍翻}
以王不恭相使檢督諸君空受爵禄曾無訓導之義因奏治其罪|{
	治直之翻}
詔書讓政而詰責傳景因捕諸奸人奏案其辠殺戮尤惡者數十人出寃獄百餘人政遂為改節悔過自修|{
	為于偽翻}
帝以伊吾膏膄之地傍近西域匈奴資之以為鈔暴|{
	近其靳翻鈔楚交翻}
三月辛亥復令開設屯田如永和時事|{
	見四十七卷和帝永元二年}
置伊吾司馬一人初安帝薄於藝文博士不復講習朋徒相視怠散學

舍穨敝鞠為園蔬|{
	穨徒回翻賢曰詩小雅曰鞠為茂草注曰鞠窮也}
或牧兒蕘豎薪刈其下|{
	蕘豎刈草者也蕘如招翻}
將作大匠翟酺上疏請修繕誘進後學帝從之|{
	翟直格翻酺薄乎翻誘音酉}
秋九月繕起太學凡所造構二百四十房千八百五十室 護烏桓校尉耿曄遣兵擊鮮卑破之|{
	曄與曅同}
護羌校尉韓皓轉湟中屯田置兩河間以逼羣羌|{
	兩河謂賜支河及逢留大河也}
皓坐事徵以張掖太守馬續代為校尉兩河間羌以屯田近之|{
	近其靳翻}
恐必見圖乃解仇詛盟各自儆備|{
	詛莊助翻}
續上移田還湟中|{
	上上奏也音時掌翻}
羌意乃安 帝欲立皇后而貴人有寵者四人莫知所建議欲探籌以神定選|{
	書四人姓氏於籌禱之於神而探之得之為入選探它南翻}
尚書僕射南郡胡廣與尚書馮翊郭䖍史敞上疏諫曰竊見詔書以立后事大謙不自專欲假之籌策决疑靈神篇籍所記祖宗典故未嘗有也恃神卜筮既未必當賢就值其人猶非德選夫岐嶷形於自然|{
	賢曰詩云克岐克嶷鄭注云岐岐然意有所知其貌嶷嶷然有所識别也嶷魚力翻}
俔天必有異表|{
	賢曰俔音苦見翻說文曰俔譬諭也詩云文王嘉止大邦有子俔天之妹文王聞太姒之賢則美之言大邦有子女譬天之有女弟故求為配焉}
宜參良家簡求有德德同以年年鈞以貌稽之典經斷之聖慮|{
	斷丁亂翻}
帝從之恭懷皇后弟子乘氏侯商之女|{
	恭懷皇后和帝母梁貴人也乘氏縣屬濟隂郡春秋之乘丘也乘繩證翻}
選入掖庭為貴人常特被引御從容辭曰夫陽以博施為德|{
	被皮義翻從干容翻施式智翻}
隂以不專為義螽斯則百福所由興也|{
	言后妃不妬忌若螽斯則子孫衆多而百福興矣}
願陛下思雲雨之均澤小妾得免於罪帝由是賢之

陽嘉元年春正月乙巳立貴人梁氏為皇后 京師旱三月揚州六郡妖賊章河等寇四十九縣殺傷長吏

|{
	揚州部九江丹陽廬江會稽吳豫章等六郡妖於驕翻長知兩翻}
庚辰赦天下改元夏四月梁商加位特進頃之拜執金吾 冬耿曄遣

烏桓戎末魔等鈔擊鮮卑大獲而還|{
	前書鮮卑傳作戎末廆賢曰廆音胡罪翻鈔楚交翻}
鮮卑復寇遼東屬國耿曄移屯遼東無慮城以拒之|{
	復扶又翻郡國志遼東屬國故邯鄉西部都尉安帝時以為屬國都尉領昌遼賓徒徒河無慮險瀆房六城在雒陽東北三千二百六十里無慮因毉無慮山以名縣慮音廬}
尚書令左雄上疏曰昔宣帝以為吏數變易則下不安業久於其事則民服教化其有政治者輒以璽書勉勵增秩賜金公卿缺則以次用之是以吏稱其職民安其業漢世良吏於兹為盛|{
	謂尹翁歸韓延夀朱邑龔遂黄覇之屬也事並見宣帝紀數所角翻治直吏翻稱尺證翻}
今典城百里轉動無常各懷一切莫慮長久謂殺害不辜為威風聚歛整辦為賢能|{
	歛力贍翻}
以治已安民為劣弱|{
	治直之翻}
奉法循理為不治髠鉗之戮生於睚眥|{
	師古曰睚眥舉目眥也猶言顧曕之頃也睚音厓眥音才賜翻字書曰睚牛懈翻怒視也}
覆尸之禍成於喜怒視民如寇讐税之如豺虎監司項背相望|{
	賢曰項背相望謂前後相顧也監古銜翻背音輩}
與同疾疢|{
	言同有此病也疢丑刃翻}
見非不舉聞惡不察觀政於亭傳責成於朞月|{
	言郡縣長吏飾亭傳以夸過使客監司亦以是觀政也賢曰朞匝也謂一歲傳株戀翻}
言善不稱德論功不據實虛誕者獲譽拘檢者離毁|{
	賢曰離遭也譽音余}
或因辠而引高或色斯而求名|{
	因有罪而先自弃官以為高論語曰色斯舉矣此言見上之人顔色不善則舉而去之以求見幾之名也}
州宰不覆|{
	覆審也}
競共辟召踴躍升騰超等踰匹或考奏捕案而亡不受罪會赦行賂復見洗滌朱紫同色清濁不分故使姦猾枉濫輕忽去就拜除如流缺動百數鄉官部吏職賤禄薄車馬衣服一出於民亷者取足貪者充家特選横調|{
	曰特曰横皆出於常賦之外者也賢曰調徵也徒釣翻}
紛紛不絶送迎煩費損政傷民和氣未洽災眚不消咎皆在此臣愚以為守相長吏惠和有顯效者可就增秩勿移徙非父母喪不得去官|{
	守式又翻相息亮翻長知兩翻}
其不從法禁不式王命|{
	賢曰式用也}
錮之終身雖會赦令不得齒列若被劾奏亡不就法者|{
	劾戶㮣翻又戶得翻}
徙家邊郡以懲其後其鄉部親民之吏皆用儒生清白任從政者|{
	賢曰任堪也音人林翻}
寛其負筭|{
	賢曰負欠也筭口錢也儒生未有品秩故寛之}
增其秩禄吏職滿歲宰府州郡乃得辟舉如此威福之路塞|{
	塞悉則翻}
虚偽之端絶送迎之役損賦斂之源息循理之吏得成其化率土之民各寧其所矣帝感其言復申無故去官之禁|{
	先已有此禁今復申嚴之復扶又翻}
又下有司考吏治真偽詳所施行|{
	下遐稼翻治直吏翻}
而宦官不便終不能行雄又上言孔子曰四十不惑|{
	見論語}
禮稱彊仕|{
	曲禮曰四十曰彊而仕}
請自今孝亷年不滿四十不得察舉皆先詣公府諸生試家法|{
	賢曰儒有一家之學故稱家法也}
文吏課牋奏|{
	周成雜字曰牋表也漢雜事曰凡羣臣之書通於天子者四品一曰章二曰奏三曰表四曰駁議章者需頭稱稽首上以聞謝恩陳事詣闕通者也奏者亦需頭其京師官但言稽首言下言稽首以聞其中有所請若罪法劾案公府送御史臺卿校送謁者臺也表者不需頭上言臣某言下言誠惶誠恐頓首頓首死罪死罪左方下附曰某官臣某甲乙上}
副之端門|{
	宫之正南門曰端門尚書於此受天下章奏令舉者先詣公府課試以副本納之端門尚書審覈之}
練其虚實以觀異能以美風俗有不承科令者正其罪法若有茂材異行|{
	行下孟翻茂材即秀才賈公彦曰漢光武號秀改為茂才}
自可不拘年齒帝從之胡廣郭䖍史敞上書駁之曰凡選舉因才無拘定制六奇之策不出經學鄭阿之政非必章奏|{
	陳平六出奇計以佐高帝子產相鄭擇能而使之内無國中之亂外無諸侯之患說苑曰晏子化東阿三年景公召而數之晏子請改道易行明年上計景公迎而賀之晏子對曰臣前之化東阿也屬託不行貨賂不至君反以罪臣今則反是而更蒙賀景公下席而謝駁北角翻}
甘奇顯用年乖彊仕終賈揚聲亦在弱冠|{
	史記曰秦欲與燕伐趙以廣河間之地甘羅年十二使于趙趙王立割五城以廣河間秦乃封羅為上卿說苑子奇年十八齊君使主東阿東阿大化前書終軍年十八自請願以長纓必羇南越王而致之闕下武帝大悦擢為諫大夫賈誼年十八揚聲漢庭文帝超遷之}
前世以來貢舉之制莫或回革今以一臣之言剗戾舊章便利未明衆心不猒|{
	回轉也反也賢曰剗削也戻乖也猒滿也剗楚限翻}
矯枉變常政之所重而不訪台司不謀卿士若事下之後議者剝異|{
	下遐稼翻下同剥與駁同}
異之則朝失其便同之則王言已行|{
	言若附同雄言而與駁議者異則朝政為不便若與駁議者同而以雄言為非則上已從雄言而行之矣朝直遥翻}
臣愚以為可宣下百官參其同異然後覽擇勝否詳采厥衷|{
	衷陟仲翻下同}
帝不從辛卯初令郡國舉孝亷限年四十以上諸生通章句文吏能牋奏乃得應選其有茂才異行若顔淵子奇不拘年齒久之廣陵所舉孝亷徐淑年未四十臺郎詰之|{
	臺郎尚書郎也詰去吉翻}
對曰詔書曰有如顔回子奇不拘年齒是故本郡以臣充選郎不能屈左雄詰之曰顔回聞一知十孝亷聞一知幾邪|{
	幾居豈翻}
淑無以對乃罷却之郡守坐免

袁宏論曰夫謀事作制以經世訓物必使可為也古者四十而仕非謂彈冠之會必將是年也|{
	師古曰彈冠言入仕也}
以為可仕之時在於彊盛故舉其大限以為民衷且顔淵子奇曠代一有而欲以斯為格豈不偏乎

然雄公直精明能審覈真偽决志行之頃之胡廣出為濟隂太守|{
	濟子禮翻}
與諸郡守十餘人皆坐謬舉免黜唯汝南陳䉒潁川李膺下邳陳球等三十餘人得拜郎中自是牧守畏慄莫敢輕舉迄于永嘉察選清平多得其人閏月庚子恭陵百丈廡災|{
	賢曰廡廊屋也說文堂下周屋曰廡廡音武}


上聞北海郎顗精於隂陽之學|{
	姓譜魯懿公孫費伯城郎因居之子孫以為氏顗魚豈翻}


二年春正月詔公車徵顗問以災異顗上章曰三公上應台階下同元首|{
	賢曰春秋元命包曰魁下六星兩兩而比曰三台前書音義曰泰階三台也又黄帝泰階六符經曰泰階者天之三階也上階為天子中階為諸侯公卿大夫下階為士庶人三階平則隂陽和風雨時尚書曰君為元首臣作股肱言三公上象天之台階下與人君同體也}
政失其道則寒隂反節今之在位競託高虛納累鍾之奉|{
	奉與俸同音扶用翻}
亡天下之憂|{
	亡古無字}
棲遲偃仰|{
	小雅北山之詩曰或栖遲偃仰毛公曰棲遲遊息也偃仰卧也}
寑疾自逸被策文得賜錢即復起矣何疾之易而愈之速|{
	被皮義翻復扶又翻易以䜴翻}
以此消伏災眚興致升平其可得乎今選牧守委任三府長吏不良既咎州郡|{
	守式又翻長知兩翻}
州郡有失豈得不歸責舉者而陛下崇之彌優自下慢事愈甚所謂大網疏小網數|{
	賢曰謂緩於三公切於州郡也數趨玉翻密也孟子曰數罟不入汚池}
三公非臣之仇臣非狂夫之作所以發憤忘食懇懇不已者誠念朝廷欲致興平臣書不擇言死不敢恨因條便宜七事一園陵火災宜念百姓之勞罷繕修之役二立春以後隂寒失節宜采納良臣以助聖化三今年少陽之歲|{
	少詩沼翻}
春當旱夏必有水宜遵前典惟節惟約四去年八月熒惑出入軒轅|{
	晉書天文志軒轅十七星黄帝之神黄龍之體也后妃之主女職也}
宜簡出宫女恣其姻嫁五去年閏十月有白氣從西方天苑趨參左足入玉井|{
	續漢志曰時客星氣白廣二尺長五丈起天苑西南晉書天文志曰天苑十六星在昴畢南天子之苑囿餋獸之所也參十星白虎之體其中三星横列三將也東北曰左肩主左將西北曰右肩主右將東南曰左足主後將軍西南曰右足主偏將軍玉井四星在參左足下主水漿以給厨參所今翻}
恐立秋以後將有羌寇畔戾之患宜豫告諸郡嚴為備禦六今月十四日乙卯白虹貫日|{
	顗曰凡日氣色白而純者名為虹貫日中者侵太陽也晉志曰凡白虹者百殃之本衆亂所基}
宜令中外官司並須立秋然後考事七漢興以來三百三十九歲於時三朞|{
	賢曰謂以三朞之法推之也}
宜大蠲法令有所變更|{
	更工衡翻}
王者隨天譬猶自春徂夏改青服絳也|{
	春服青夏服絳各隨時之色}
自文帝省刑適三百年|{
	賢曰自文帝十三年除肉刑至順帝陽嘉二年合三百年也}
而輕微之禁漸已殷積王者之法譬猶江河當使易避而難犯也二月顗復上書薦黄瓊李固以為宜加擢用又言自冬涉春訖無嘉澤數有西風反逆時節|{
	賢曰春當東風也復扶又翻數所角翻}
朝廷勞心廣為禱祈薦祭山川暴龍移市|{
	董仲舒春秋繁露曰春旱以甲乙日為蒼龍一長八丈居中央為小龍各長四丈皆東向其間相去八尺小童八人皆齋三日服青衣而舞之夏以丙丁日為赤龍服赤衣季夏以戊巳日為黄龍服黄衣秋以庚辛日為白龍服白衣冬以壬癸日為黑龍服黑衣龍長與舞童各依其行數牲各依其方色皆燔雄雞煅豭豬尾於里北門及市中以祈焉禮記歲旱魯穆公問於縣子縣子曰為之徙市可也}
臣聞皇天感物不為偽動|{
	為于偽翻}
災變應人要在責已若令雨可請降水可禳止則歲無隔并太平可待然而災害不息者患不在此也書奏特拜郎中辭病不就 三月使匈奴中郎將趙稠遣從事將南匈奴兵出塞擊鮮卑破之初帝之立也乳母宋娥與其謀|{
	與讀曰預}
帝封娥為山陽君又封執金吾梁商子冀為襄邑侯|{
	襄邑縣屬陳留郡}
尚書令左雄上封事曰高帝約非劉氏不王非有功不侯孝安皇帝封江京王聖等遂致地震之異|{
	事見上卷安帝建光元年}
永建二年封隂謀之功|{
	不見于史}
又有日食之變數術之士咸歸咎於封爵今青州饑虛盗賊未息誠不宜追録小恩虧失大典詔不聼雄復諫曰臣聞人君莫不好忠正而惡讒諛然而歷世之患莫不以忠正得罪讒諛蒙倖者盖聼忠難從諛易也|{
	復扶又翻好呼到翻惡烏路翻下同易以豉翻}
夫刑罪人情之所甚惡貴寵人情之所甚欲是以時俗為忠者少而習諛者多故令人主數聞其美稀知其過迷而不悟以至於危亡臣伏見詔書顧念阿母舊德宿恩欲特加顯賞案尚書故事|{
	漢故事皆尚書主之}
無乳母爵邑之制唯先帝時阿母王聖為野王君聖造生讒賊廢立之禍|{
	事見上卷安帝延光三年}
生為天下所咀嚼|{
	咀在呂翻}
死為海内所歡快桀紂貴為天子而庸僕羞與為比者以其無義也夷齊賤為匹夫而王侯争與為伍者以其有德也今阿母躬蹈儉約以身率下羣僚蒸庶莫不向風|{
	蒸衆也}
而與王聖並同爵號懼違本操失其常願臣愚以為凡人之心理不相遠|{
	遠于願翻}
其所不安古今一也百姓深懲王聖傾覆之禍民萌之命危於累卵|{
	萌與氓同}
常懼時世復有此類|{
	復扶又翻下同}
怵惕之念未離於心|{
	怵惕悚懼也上尺律翻下他歷翻離力智翻}
恐懼之言未絶於口乞如前議歲以千萬給奉阿母|{
	盖雄先已有此議今乞行之也}
内足以盡恩愛之歡外可不為吏民所怪梁冀之封事非機急宜過災戹之運然後平議可否于是冀父商讓還冀封書十餘上|{
	上時掌翻}
帝乃從之夏四月己亥京師地震五月庚子詔羣公卿士各直言厥咎仍各舉敦樸士一人左雄復上疏曰先帝封野王君漢陽地震|{
	安帝延光二年封王聖是歲京師及郡國三地震漢陽蓋其一也}
今封山陽君而京城復震專政在隂其災尤大臣前後瞽言封爵至重王者可私人以財不可以官宜還阿母之封以塞災異|{
	塞悉則翻}
今冀已高讓山陽君亦宜崇其本節雄言切至娥亦畏懼辭讓而帝戀戀不能已卒封之|{
	卒子恤翻}
是時大司農劉據以職事被譴召詣尚書傳呼促步|{
	促步催使速行也被皮義翻}
又加以捶撲|{
	捶止蕊翻撲普卜翻}
雄上言九卿位亞三事班在大臣行有佩玉之節|{
	禮記曰公侯佩山玄玉而朱組綬大夫佩水蒼玉而緇組綬詩曰雜佩以贈之毛氏注云珩璜琚瑀衝牙之類月令章句曰佩上有雙珩下冇雙璜琚瑀以雜之衝牙蠙珠以納其間玉藻曰左徵角右宫羽進則揖之退則揚之然後玉瑲鳴也至漢明帝乃為大佩衝牙雙瑀璜皆以白玉乘輿落以白珠公卿諸侯以采絲孔穎達曰凡佩玉必上繫於衡下垂三道穿以蠙珠下端前後以懸於璜中央下端懸以衝牙動則衝牙前後觸璜而為聲所觸之玉其形似牙故曰衝牙}
動則有庠序之儀|{
	庠序之儀謂濟濟蹌蹌}
孝明皇帝始有撲罸皆非古典帝納之是後九卿無復捶撲者|{
	撲蒲卜翻又弼角翻捶止蕊翻}
戊午司空王龔免六月辛未以太常魯國孔扶為司空 丁丑雒陽宣德亭地拆長八十五丈|{
	按續漢志宣德亭近郊地光武立郊北於雒陽城南亭蓋在平城門外長直亮翻}
帝引公卿所舉敦樸之士使之對策及特問以當世之敝為政所宜李固對曰前孝安皇帝變亂舊典封爵阿母因造妖孽|{
	妖於驕翻孽魚列翻}
改亂嫡嗣至令聖躬狼狽親遇其艱既拔自困殆龍興即位天下喁喁|{
	喁魚容翻師古曰喁喁衆口向上貌}
屬望風政|{
	屬之欲翻}
積敝之後易致中興|{
	易以豉翻}
誠當沛然思惟善道|{
	賢曰沛然寛廣之意}
而論者猶云方今之事復同於前|{
	復扶又翻}
臣伏在草澤痛心傷臆實以漢興以來三百餘年賢聖相繼十有八主|{
	高惠文景武昭宣元成哀平光明章和殤安至帝凡十八主}
豈無阿乳之恩豈忘貴爵之寵然上畏天威俯案經典知義不可故不封也今宋阿母雖有大功勤謹之德但加賞賜足以酬其勞苦至于裂土開國實乖舊典聞阿母體性謙虚必有遜讓陛下宜許其辭國之高使成萬安之福夫妃后之家所以少完全者|{
	少詩沼翻}
豈天性當然但以爵位尊顯顓摠權柄天道惡盈|{
	易曰天道虧盈而益謙惡烏路翻}
不知自損故致顛仆先帝寵遇閻氏位號太疾故其受禍曾不旋踵|{
	安帝建光元年諸鄧得罪閻氏始盛延光四年閻氏誅盖不能五稔也}
老子曰其進鋭者其退速也今梁氏戚為椒房禮所不臣|{
	禮不臣妻之父母}
尊以高爵尚可然也而子弟羣從榮顯兼加|{
	從才用翻}
永平建初故事殆不如此宜令步兵校尉冀及諸侍中還居黄門之官使權去外戚政歸國家豈不休乎|{
	休美也}
又詔書所以禁侍中尚書中臣子弟不得為吏察孝亷者以其秉威權容請託故也而中常侍在日月之側聲埶振天下子弟禄任曾無限極雖外託謙默不干州郡而諂偽之徒望風進舉|{
	謂州郡阿私宦官進舉其子弟也}
今可為設常禁|{
	為于偽翻下同}
同之中臣|{
	此中臣謂中朝臣也}
昔館陶公主為子求郎明帝不許賜錢千萬|{
	事見四十五卷永平十八年為于偽翻}
所以輕厚賜重薄位者為官人失才害及百姓也竊聞長水司馬武宣|{
	百官志北軍五營校尉各有司馬秩千石}
開陽城門候羊迪等|{
	雒陽城十二門每門候一人秩六百石開陽門位在已應劭漢官曰開陽門始成未有名宿昔有一柱來樓上琅邪開陽縣上言縣南城門一柱飛去光武皇帝使來識視愴然遂堅縛之刻記其歲月因以名焉}
無他功德初拜便真|{
	漢制初拜官稱守滿歲為真續漢書曰中都官千石六百石故事先守一歲然後補真}
此雖小失而漸壞舊章|{
	壞音怪}
先聖法度所宜堅守故政教一跌百年不復|{
	跌徒結翻}
詩云上帝板板下民卒癉刺周王變祖宗法度故使下民將盡病也|{
	凡伯刺周厲王之詩賢曰板反也卒盡也音子恤翻癉病也癉音當但翻}
今陛下之有尚書猶天之有北斗也斗為天喉舌尚書亦為陛下喉舌斗斟酌元氣運乎四時|{
	天文志曰斗為帝車運乎中央臨制四方分隂陽建四時均五行移節度定諸紀皆繫于斗}
尚書出納王命賦政四海|{
	賢曰賦布也}
權尊埶重責之所歸若不平心災眚必至誠宜審擇其人以毗聖政|{
	毗輔也}
今與陛下共天下者外則公卿尚書内則常侍黄門譬猶一門之内一家之事安則共其福慶危則通其禍敗|{
	此等議論發之嬖倖盈朝之時謂之曲而當可也猶以直而不見容嗚呼}
刺史二千石外統職事内受法則夫表曲者景必邪源清者流必潔猶叩樹本百枝皆動也由此言之本朝號令豈可蹉跌|{
	蹉倉何翻}
天下之紀綱當今之急務也夫人君之有政猶水之有隄防隄防完全雖遭雨水霖潦不能為變政教一立蹔遭凶年|{
	蹔與暫同}
不足為憂誠令隄防穿漏萬夫同力不能復救政教一壞賢智馳騖不能復還|{
	復扶又翻}
今隄防雖堅漸有孔穴|{
	諭嬖倖之門也當此之時不可以言漸矣固特婉其辭耳}
譬之一人之身本朝者心腹也州郡者四支也心腹痛則四支不舉故臣之所憂在腹心之疾非四支之患也苟堅隄防務政教先安心腹整理本朝雖有寇賊水旱之變不足介意也誠令隄防壞漏心腹有疾雖無水旱之災天下固可以憂矣又宜罷退宦官去其權重裁置常侍二人方直有德者省事左右小黄門五人才智閒雅者給事殿中|{
	范曄曰漢承秦制置中常侍官然亦引用士人以參其選皆銀璫左貂給事殿省及高后稱制乃以張卿為大謁者出入卧内受宣詔命文帝時有趙談北宫伯子頗見親倖至于武帝亦愛李延年帝數宴後庭或潜遊離館故請奏機事多以宦人主之至元帝之世史游為黄門令勤身納忠有所補益其後弘恭石顯以佞險自進卒有蕭周之禍損穢帝德焉中興之初宦官悉用閹人不復雜調他士至永平中始置員數中常侍四人小黄門十人和帝即祚幼弱竇憲兄弟專摠威權所與居者閹宦而已故鄭衆得專謀禁中終除大憝遂享分土之封超登公卿之位於是中官始盛矣自明帝之後迄于延平委用漸大其員稍增中常侍至有十人小黄門二十人改以金璫右貂兼領卿署之職鄧后以女主臨政萬機殷遠朝臣國議無由參斷不得不委用刑人寄之國命手握王爵口銜天憲非復掖庭永巷之職閨牖房闥之任也去羌呂翻省悉景翻}
如此則論者厭塞升平可致也|{
	塞悉則翻}
扶風功曹馬融對曰今科條品制四時禁令所以承天順民者備矣悉矣不可加矣然而天猶有不平之效民猶有咨嗟之怨者百姓屢聞恩澤之聲而未見惠和之實也古之足民者非能家贍而人足之量其財用為之制度|{
	量音良}
故嫁娶之禮儉則婚者以時矣喪祭之禮約則終者掩藏矣不奪其時則農夫利矣夫妻子以累其心|{
	累力瑞翻}
產業以重其志舍此而為非者有必不多矣|{
	馬融之對不詭於聖人盖有得於經學故其辭氣和平而切於政體也舍讀曰捨}
太史令南陽張衡對曰|{
	百官志太史令屬太常秩六百石掌天時星歷}
自初舉孝亷迄今二百歲矣|{
	武帝元光元年初舉孝亷至是凡二百七年}
皆先孝行行有餘力始學文法辛卯詔書以能章句奏案為限|{
	去年冬十一月辛卯詔書也孝行下孟翻}
雖有至孝猶不應科此弃本而取末曾子長於孝然實魯鈍文學不若游夏政事不若冉季今欲使一人兼之苟外有可觀内必有闕則違選舉孝亷之志矣且郡國守相剖符寜境為國大臣一旦免黜十有餘人|{
	謂濟隂太守胡廣等也}
吏民罷於送迎之役|{
	罷讀曰疲}
新故交際公私放濫或臨政為百姓所便而以小過免之是為奪民父母使嗟號也|{
	號戶刀翻}
易不遠復論不憚改|{
	易曰不遠復無祗悔論語曰過則勿憚改}
朋友交接且不宿過况於帝王承天理物以天下為公者乎中間以來妖星見於上|{
	古今注曰是年四月壬寅太白晝見五月癸巳又晝見見賢遍翻}
震裂著於下|{
	謂永建三年京師地震今年宣德亭地裂也}
天誡詳矣可為寒心明者銷禍於未萌今既見矣|{
	為于為翻見賢遍翻}
修政恐懼則禍轉為福矣上覽衆對以李固為第一即時出阿母還舍諸常侍悉叩頭謝辠朝廷肅然以固為議郎而阿母宦者皆疾之詐為飛章以陷其辠事從中下|{
	從中下者不經尚書下遐稼翻}
大司農南郡黄尚等請之於梁商僕射黄瓊復救明其事|{
	復扶又翻}
久乃得釋出為洛令|{
	洛當作雒雒縣屬廣漢郡}
固弃官歸漢中融博通經籍美文辭對奏亦拜議郎衡善屬文|{
	屬之欲翻}
通貫六藝雖才高於世而無驕尚之情|{
	驕者以才驕人也尚者以才尚人也}
善機巧尤致思於天文隂陽歷算|{
	致極也思相吏翻}
作渾天儀著靈憲|{
	蔡邕曰言天體者三家一曰周髀二曰宣夜三曰渾天宣夜之學絶無師法周髀數術具存考驗天狀多所違失故史官不用唯渾天者近得其情今史官所用候臺銅儀則其法也立八尺圓體之度而具天地之象以正黄道以察發斂以行日月以步五緯精微深妙萬世不易之道也衡著靈憲曰昔在先王將步天路用之靈軌尋緒本元先凖之於渾體是為正儀立度而皇極有逌建也樞運有逌稽也乃建乃稽斯經天常聖人無心因兹以生心故靈憲作興王蕃曰天地之體狀如鷄卵天包地外猶殻之裹黄也周旋無端其形渾渾然故曰渾天周天三百六十五度五百八十九分度之百四十五半露地上半在地下其二端謂之南極北極北極出地三十六度南極入地亦三十六度兩極相去一百八十二度半強繞北極徑七十二度常見不隱謂之上規繞南極七十五度常隱不見謂之下規赤道帶天之紘去兩極各九十一度少強黄道日之所行也半在赤道外半在赤道内與赤道東交於角五弱西交於奎十四少強其出赤道外極遠者去赤道二十四度斗二十一度是也其入赤道内極遠者亦二十四度井二十五度是也日南至在斗二十一度去極百一十五度少強是也日最南去極最遠故景最長黄道斗二十一度出辰入申故日亦出辰入申日晝行地上百四十六度強故日短夜行地下二百一十九度少弱故夜長自南至之後日去極稍近故景稍短日晝行地上度稍多故日稍長夜行地下度稍少故夜稍短日所在度稍北故日稍北以至於夏至日在井二十五度去極六十七度稍強是也日最北去極最近景最短黄道井二十五度出寅入戌故日亦出寅入戌日晝行地上二百一十九度少弱故日長夜行地下百四十六度強故夜短自夏至之後日去極稍遠故景稍長日晝行地上度稍少故日稍短夜行地下度稍多故夜稍長日所在度稍南故日出入稍南以至於南至而復初焉斗二十一井二十五南北相覺四十八度春分日在奎十四稍強秋分日在角五稍弱此黄赤二道之交中也去極俱九十一度少強南北處斗二十一井二十五之中故景居二至短長之中奎十四角五出卯入酉故日亦出卯入酉日晝行地上夜行地下俱百八十度半強故日見之漏晝五十刻不見之漏五十刻而晝夜同夫天之晝夜以日出入為分人之晝夜以昏明為限日未出二刻半而明日未入二刻半而昏故損夜五刻以益晝是以春秋之漏晝五十五刻洛書甄耀度曰周天三百六十五度四分度之一一度為千九百三十二里}
性恬憺|{
	憺杜覽翻}
不慕當世所居之官輒積年不徙 太尉龎參在三公中名忠直數為左右所毁會所舉用忤帝旨|{
	數所角翻忤五故翻}
司隸承風案之時當會茂才孝亷參以被奏稱疾不會|{
	被皮義翻}
廣漢上計掾段恭因會上疏曰|{
	漢郡國歲舉茂才孝亷與上計吏皆至京師受計之日公卿皆會于廷茂孝豫焉}
伏見道路行人農夫織婦皆曰太尉參竭忠盡節徒以直道不能曲心孤立羣邪之間自處中傷之地|{
	處昌呂翻中竹仲翻}
夫以讒佞傷毁忠正此天地之大禁人臣之至誡也昔白起賜死諸侯酌酒相賀|{
	白起死事見五卷周赧王五十年}
季子來歸魯人喜其紓難|{
	賢曰紓緩也季子魯公子季友也閔公之時國家多難以季子忠賢故請齊侯復之公羊傳曰季子來歸其言季子何賢也其言來歸喜之也難乃旦翻}
夫國以賢治君以忠安|{
	治直吏翻}
今天下咸欣陛下有此忠賢願卒寵任以安社稷|{
	卒子恤翻}
書奏詔即遣小黄門視參疾太醫致羊酒後參夫人疾前妻子投於井而殺之雒陽令祝良奏參罪秋七月己未參竟以災異免 八月己巳以大鴻臚施延為太尉 鮮卑寇馬城代郡太守擊之不克頃之其至犍死|{
	犍居言翻}
鮮卑由是鈔盜差稀|{
	鈔楚交翻}


資治通鑑卷五十一
