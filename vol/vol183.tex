資治通鑑卷一百八十三 宋 司馬光 撰

胡三省 音註

隋紀七|{
	起柔兆困敦盡彊圉赤奮若五月凡一年有奇}


煬皇帝下

大業十二年春正月朝集使不至者二十餘郡|{
	漢儀正旦大朝會諸郡計吏皆覲隋之朝集使亦此類也朝直遙翻下同}
始議分遣使者十二道發兵討捕盗賊|{
	使疏吏翻}
詔毗陵通守路道德|{
	守式又翻}
集十郡兵數萬人於郡東南起宫苑周圍十二里内為十六離宫大抵倣東都西苑之制而奇麗過之又欲築宫於會稽|{
	會古外翻}
會亂不果成 三月上已帝與羣臣飲於西苑水上命學士杜寶撰水飾圖經采古水事七十二使朝散大夫黄衮以木為之間以妓航酒船|{
	朝直遙翻散悉亶翻間古莧翻妓渠綺翻航戶郎翻}
人物自動如生鐘磬箏瑟能成音曲 己丑張金稱陷平恩|{
	隋志平恩縣屬武安郡稱尺證翻}
一朝殺男女萬餘口又陷武安鉅鹿清河諸縣|{
	隋志武安縣屬武安郡鉅鹿縣屬襄國郡清河郡帶清河縣既郡城堅守則此縣不陷詳考隋志帶郭之清河本武城縣開皇初改名清河而清河縣則舊清河縣金稱所陷蓋此}
金稱比諸賊尤殘暴所過民無孑遺|{
	孑單也遺餘也言無單孑遺餘也}
夏四月丁巳大業殿西苑火帝以為盗起驚走入西苑匿草間火定乃還|{
	還從宣翻又如字}
帝自八年以後每夜眠恒驚悸|{
	恒戶登翻悸其季翻心動也}
云有賊令數婦人搖撫乃得眠 癸亥歷山飛别將甄翟兒衆十萬寇太原|{
	將即亮翻甄側鄰翻}
將軍潘長文敗死|{
	長知兩翻}
五月丙戍朔日有食之既 壬午帝於景華宫徵求螢火得數斛夜出遊山放之光遍巖谷 |{
	考異曰吳競貞觀政要貞觀八年上謂侍臣曰人君之言不可容易隋煬帝幸甘泉宫怪無螢火勑云捉取多少於宫照夜所司遽遣數千人採拾送五百轝於宫側小事尚爾况其大乎今從隋書}
帝問侍臣盗賊左翊衛大將軍宇文述曰漸少帝曰比從來少幾何對曰不能什一納言蘇威引身隱柱帝呼前問之對曰臣非所司不委多少|{
	委悉也少詩沼翻}
但患漸近帝曰何謂也威曰他日賊據長白山今近在汜水|{
	隋志汜水縣屬滎陽郡舊曰成臯開皇十八年改曰汜水汜音似}
且往日租賦丁役今皆何在豈非其人皆化為盗乎比見奏賊皆不以實|{
	比毗至翻}
遂使失於支計不時翦除又昔在鴈門許罷征遼今復徵發|{
	復扶又翻}
賊何由息帝不悦而罷尋屬五月五日|{
	屬之欲翻}
百僚多饋珍玩威獨獻尚書或譛之曰尚書有五子之歌威意甚不遜|{
	言威以帝逸豫盤遊不知返將至失邦如夏太康也尚而亮翻孔安國曰以其上古之書謂之尚書}
帝益怒頃之帝問威以伐高麗事|{
	麗力知翻}
威欲帝知天下多盗對曰今兹之役願不發兵但赦羣盗自可得數十萬遣之東征彼喜於免罪爭務立功高麗可滅|{
	麗力知翻}
帝不懌威出御史大夫裴藴奏曰此大不遜天下何處有許多賊帝曰老革多姦|{
	蜀志彭羕祗劉備曰老革荒悖注云老革皮色枯瘁之形羕罵備為老革猶言老兵也帝引此語}
以賊脅我欲批其口且復隱忍|{
	批蒲鱉翻又普迷翻復扶又翻}
藴知帝意遣河南白衣張行本|{
	隋志洛州河南郡大業二年移都改曰豫州}
奏威昔在高陽典選|{
	謂九年從帝自遼東還高陽時選宣戀翻}
濫授人官畏怯突厥請還京師|{
	事見上卷上年厥九勿翻}
帝令案驗獄成下詔數威罪狀|{
	數所具翻}
除名為民後月餘復有奏威與突厥隂圖不軌者|{
	復扶又翻厥九勿翻}
事下裴藴推之藴處威死|{
	下遐嫁翻處昌呂翻}
威無以自明但摧謝而已帝憫而釋之曰未忍即殺并其子孫三世皆除名 秋七月壬戌濟景公樊子蓋卒|{
	隋書樊子蓋傳帝以子蓋守東都平玄感之功進爵濟公謂其功濟天下封以嘉名無此郡國也濟讀當如字卒子恤翻}
江都新作龍舟成送東都宇文述勸幸江都右候衛大將軍酒泉趙才諫曰今百姓疲勞府藏空竭|{
	隋志張掖郡福祿縣舊置酒泉郡藏徂浪翻}
盗賊蜂起禁令不行願陛下還京師安兆庶帝大怒以才屬吏|{
	屬之欲翻}
旬日意解乃出之朝臣皆不欲行帝意甚堅無敢諫者建節尉任宗上書極諫|{
	置建節尉事見上卷上年朝直遙翻下同任音壬上時掌翻下同}
即日於朝堂杖殺之甲子帝幸江都命越王侗與光祿大夫段達太府卿元文都檢校民部尚書韋津右武衛將軍皇甫無逸右司郎盧楚等總留後事|{
	唐六典曰煬帝三年尚書都司始置左右司郎各一人掌都省之職品同諸曹郎從五品總留後事者帝出廵幸以後事付留臺官總之侗它紅翻又音同}
津孝寛之子也|{
	韋孝寛宇文干城之將}
帝以詩留别宫人曰我夢江都好征遼亦偶然奉信郎崔民象以盗賊充斥於建國門上表諫|{
	隋志帝置謁者臺官尋又置散騎郎二十人從五品承議郎正六品通直郎從六品各三十人宣德郎正七品宣義郎從七品各四十人徵事郎正八品將仕郎從八品常從郎正九品奉信郎從九品各五十人洛都羅城門正南曰建國上時掌翻}
帝大怒先解其頤然後斬之|{
	說文頤□也}
戊辰馮翊孫華舉兵為盗|{
	隋志馮翊郡後魏置華州西魏改曰同州帝改為郡}
虞世基以盗賊充斥請發兵屯洛口倉|{
	大業二年置洛口倉}
帝曰卿是書生定猶恇怯|{
	恇音匡}
戊辰車駕至鞏勑有司移箕山公路二府於倉内|{
	鞏縣屬河南郡新唐志河南有府三十九有鞏洛府無箕山公路二府疑移於倉内後遂并為鞏洛府也}
仍令築城以備不虞至汜水|{
	汜音似}
奉信郎王愛仁復上表請還西京帝斬之而行至梁郡|{
	帝改宋州為梁郡復扶又翻}
郡人邀車駕上書曰陛下若遂幸江都天下非陛下之有又斬之是時李子通據海陵左才相掠淮北杜伏威屯六合|{
	隋志六合縣屬江都郡舊曰尉氏置秦郡後周改郡曰六合開皇初廢郡改尉氏縣為六合縣相息亮翻}
衆各數萬帝遣光祿大夫陳稜將宿衛精兵八千討之往往克捷|{
	將即亮翻}
八月乙巳賊帥趙萬海衆數十萬自恒山寇高陽|{
	帝改恒州為恒山郡帥所類翻恒戶登翻}
冬十月己丑許恭公宇文述卒|{
	卒子恤翻}
初述子化及智及皆無賴化及事帝於東宫帝寵昵之|{
	昵尼質翻}
及即位以為太僕少卿|{
	少詩照翻}
帝幸榆林|{
	三年幸榆林見一百八十卷}
化及智及冒禁與突厥交市帝怒將斬之已解衣辮髪既而釋之賜述為奴智及弟士及以尚主之故常輕智及惟化及與之親昵述卒帝復以化及為右屯衛將軍智及為將作少監|{
	為化及兄弟為逆張本復扶又翻}
李密之亡也往依郝孝德|{
	考異曰韓昱壺關錄曰大業十一年正月歷亭鎮將王該認形狀獲李密送宇文述密佯患足疾防守者一日不行一二十里忽至一澗水深岸險密跛足寅緣佯足蹷返撲而墜乃至良久狀若未蘇防守者又無計下取之遂以手中鏘戟引之密以手援戟佯作失勢推戟向水守者以危岸手探不住遂即放却密即得鎗擉守者二人俱斃遂投郝孝德於平原按密楊玄感之黨前已詐亡防者豈得不加械繫怠慢如此今不取}
孝德不禮之又入王薄薄亦不之奇也密困乏至削樹皮而食之匿於淮陽村舍|{
	帝改陳州為淮陽郡}
變姓名聚徒教授郡縣疑而捕之密亡去抵其妹夫雍丘令丘君明|{
	隋志雍丘縣屬梁郡雍於用翻}
君明不敢舍轉寄密於遊俠王秀才家秀才以女妻之君明從姪懷義告其事|{
	妻七細翻從才用翻}
帝令懷義自齎勑書與梁郡通守楊汪相知|{
	守式又翻}
收捕汪遣兵圍秀才宅適值密出外由是獲免君明秀才皆死韋城翟讓為東都灋曹|{
	隋志韋城縣屬東郡開皇六年置劉昫曰隋分白馬縣置於古城韋氏之國城東都當作東郡帝改滑州為兖州二年改為東郡郡有西曹金戶兵法士等曹翟萇伯翻}
坐事當斬獄吏黄君漢奇其驍勇|{
	驍堅堯翻下同}
夜中濳謂讓曰翟法司天時人事抑亦可知豈能守死獄中乎讓驚喜曰讓圈牢之豕|{
	圈求遠翻}
死生唯黄曹主所命君漢即破械出之讓再拜曰讓蒙再生之恩則幸矣奈黄曹主何因泣下君漢怒曰本以公為大丈夫可救生民之命故不顧其死以奉脱奈何反效兒女子涕泣相謝乎君但努力自免勿憂吾也讓遂亡命於瓦崗為羣盗|{
	瓦崗在東郡界}
同郡單雄信|{
	考異曰唐書云雄信曹州人今從河洛記單常演翻}
驍健善用馬槊|{
	驍堅堯翻槊色角翻}
聚少年往從之|{
	少詩照翻}
離狐徐世勣家於衛南|{
	離狐漢縣後魏之北濟隂郡也時屬濟隂郡唐中世改曰南華宋白曰離狐縣初置在濮水南常為神狐所穿穴遂移水北故曰離狐衛南古楚丘也隋開皇置衛南縣屬東郡宋白曰全衛之時此地在衛之南垂故以名縣}
年十七有勇畧說讓曰|{
	說輸芮翻}
東郡於公與勣皆為鄉里人多相識不宜侵掠滎陽梁郡汴水所經|{
	帝改鄭州為滎陽郡宋州為梁郡班志汴水在滎陽西南蓋汴水所起東南入梁郡界汴皮變翻}
剽行舟掠啇旅足以自資|{
	剽匹妙翻}
讓然之引衆入二郡界掠公私船資用豐給附者益衆聚徒至萬餘人時又有外黄王當仁濟陽王伯當韋城周文舉雍丘李公逸等皆擁衆為盗|{
	外黄濟陽二縣隋志皆屬濟隂郡剽匹妙翻濟子禮翻}
李密自雍州亡命往來諸帥間說以取天下之策|{
	帥所類翻說輸芮翻下同}
始皆不信久之稍以為然相謂曰斯人公卿子弟志氣若是今人人皆云楊氏將滅李氏將興吾聞王者不死斯人再三獲濟豈非其人乎由是漸敬密密察諸帥唯翟讓最彊乃因王伯當以見讓|{
	考異曰隋唐書皆云密歸翟讓其中有知密是玄感亡將潜勸讓害之密懼因王伯當以策干讓讓始敬焉按密既亡歸羣盗必不隱其姓名誰不知玄感亡將讓得之當用以敵隋何惡於密而害之今不取革命記云密投賊帥郝孝德說之曰若能用密之策河朔可指揮而定孝德曰本緣饑荒求活性命何敢别圖國家若知公在此孝德死亡無日翟讓等徒衆絶多請將兵送公於彼是日孝德以馬一匹自送至河執袂飲酒而别軍中慕從者亦數十人仍遣兵馬將送密於翟讓今從隋書}
為讓畫策|{
	為于偽翻}
往說諸小盗皆下之讓悦稍親近密與之計事|{
	近其靳翻}
密因說讓曰劉項皆起布衣為帝王今主昏於上民怨於下鋭兵盡於遼東和親絶於突厥|{
	厥九勿翻}
方乃廵遊揚越委棄東都此亦劉項奮起之會也以足下雄才大畧士馬精鋭席卷二京誅滅暴虐隋氏不足亡也|{
	卷讀曰棬}
讓謝曰吾儕羣盗旦夕偷生草間|{
	儕士皆翻}
君之言者非吾所及也會有李玄英者自東都逃來經歷諸賊求訪李密云斯人當代隋家人問其故玄英言比來民間謠歌|{
	比毗至翻}
有桃李章曰桃李子皇后繞揚州宛轉花園裏勿浪語誰道許桃李子謂逃亡者李氏之子也皇與后皆君也宛轉花園裏謂天子在揚州無還日將轉於溝壑也莫浪語誰道許者密也既與密遇遂委身事之前宋城尉齊郡房玄藻自負其才|{
	隋志宋城縣帶梁郡舊曰睢陽開皇十八年更名}
恨不為時用預於楊玄感之謀變姓名亡命遇密於梁宋之間遂與之俱遊漢沔|{
	沔彌兖翻}
徧入諸賊說其豪傑還日從者數百人|{
	說式芮翻從才用翻}
仍為遊客處於讓營|{
	處昌呂翻}
讓見密為豪傑所歸欲從其計猶豫未决有賈雄者曉隂陽占候為讓軍師言無不用密深結於雄使之託術數以說讓雄許諾懷之未發會讓召雄告以密所言問其可否對曰吉不可言又曰公自立恐未必成若立斯人事無不濟讓曰如卿言蒲山公當自立何來從我對曰事有相因所以來者將軍姓翟翟者澤也|{
	翟萇伯翻}
蒲非澤不生故須將軍也讓然之與密情好日篤|{
	好呼到翻}
密因說讓曰今四海糜沸|{
	糜粥也言如粥之沸也}
不得耕耘公士衆雖多食無倉廩唯資野掠常苦不給若曠日持久加以大敵臨之必渙然離散未若先取滎陽休兵館穀|{
	考異曰革命記密說讓曰洛口倉米逾巨億請公發一札之令使密奉之告諸道英雄就倉食米必雲合響應受命於公然後稱帝號以定中原云云讓曰就倉食米實是上謀自顧庸賤寧敢别創餘心必如此謀願奉公為主密懷懼改容而拜讓亦拜於是言宴盡歡各恨相知之晩即日讓作書與密散告諸處賊頭並剋期定日令總會洛口倉食米今從隋書}
待士馬肥充然後與人爭利讓從之於是破金隄關|{
	金隄關當在滎陽界以漢金隄名之}
攻滎陽諸縣多下之滎陽大守郇王慶弘之子也|{
	弘高祖從祖弟也封河間王守式又翻下同}
不能討帝徙張須陀為滎陽通守以討之庚戌須陀引兵擊讓讓曏數為須陀所敗|{
	數所角翻敗補邁翻}
聞其來大懼將避之密曰須陁勇而無謀兵又驟勝既驕且狠可一戰擒也|{
	狠戶墾翻}
公但列陳以待密保為公破之|{
	陳讀曰陣下同為于偽翻下同}
讓不得已勒兵將戰密分兵千餘人伏於大海寺北林間須陁素輕讓方陳而前讓與戰不利須陁乘之逐北十餘里密發伏掩之須陁兵敗密與讓及徐世勣王伯當合軍圍之須陁潰圍出左右不能盡出須陁躍馬復入救之來往數四遂戰死所部兵晝夜號哭數日不止|{
	史言張須陁得士卒心號戶刀翻}
河南郡縣為之喪氣|{
	為于偽翻喪息浪翻}
鷹揚郎將河東賈務本為須陀之副亦被傷帥餘衆五千餘人奔梁郡務本尋卒|{
	將即亮翻被皮義翻帥讀曰率卒子恤翻}
詔以光祿大夫裴仁基為河南討捕大使代領其衆徙鎮虎牢|{
	虎牢即滎陽郡汜水縣使疏吏翻}
讓乃令密建牙别統所部號蒲山公營密部分嚴整|{
	分扶問翻}
凡號令士卒雖盛夏皆如背負霜雪躬服儉素所得金寶悉頒賜麾下由是人為之用麾下士卒多為讓士卒所陵辱以威約有素不敢報也讓謂密曰今資糧粗足|{
	粗坐五翻今人多從去聲}
意欲還向瓦崗公若不往唯公所適讓從此别矣讓帥輜重東引|{
	重直用翻}
密亦西行至康城說下數城|{
	說輸芮翻}
大獲資儲讓尋悔復引兵從密|{
	復扶又翻}
鄱陽賊帥操師乞自稱元興王建元始興|{
	帝改饒州為鄱陽郡操姓也帥所類翻 考異曰隋帝紀作操天成按唐高祖實錄林士弘傳大業末與其鄉人操師乞起為羣盗師乞僭號建元為天成攻陷豫章郡入據之唐書士弘傳操乞師自號元興王皆無操天成名此賊本一人而隋唐二史各有名號年紀今參取之}
攻陷豫章郡|{
	帝改洪州為豫章郡}
以其鄉人林士弘為大將軍詔治書侍御史劉子翊將兵討之師乞中流矢死|{
	治直之翻中竹仲翻}
士弘代統其衆與子翊戰於彭蠡湖|{
	禹貢東滙澤為彭蠡班志豫章郡彭澤縣彭蠡湖在西今在南康軍城東南西接江州德化縣界周迴四百五十里}
子翊敗死士弘兵大振至十餘萬人十二月壬辰士弘自稱皇帝國號楚建元太平 |{
	考異曰唐高祖實錄士弘自稱南越王尋僭號建元延康唐書林士弘傳操師乞攻陷豫章郡而據之以士弘為大將軍師乞既死士弘代董其衆復與劉子翊大戰於彭蠡湖隋師敗績子翊死之士弘大振兵至十餘萬十三年徙據䖍州稱帝其國號年名與此同今從隋書}
遂取九江臨川南康宜春等郡|{
	帝改江州為九江郡改撫州為臨川郡䖍州為南康郡袁州為宜春郡}
豪傑爭殺隋守令以郡縣應之其地北自九江南及番禺皆為所有|{
	南海郡治番禺隋并為南海縣番音潘禺音愚}
詔以右驍衛將軍唐公李淵為太原留守|{
	帝改并州為太原郡驍堅堯翻守式又翻}
以虎賁郎將王威虎牙郎將高君雅為之副|{
	帝改定官制十二衛府每衛置護軍四人掌副貳將軍尋改護軍為虎賁郎將正四品而置虎牙郎將六人副焉從四品將即亮翻下同賁音奔}
將兵討甄翟兒|{
	甄側鄰翻}
與翟兒遇於雀鼠谷|{
	隋志西河郡永安縣有雀鼠谷}
淵衆纔數千賊圍淵數匝|{
	匝子荅翻}
李世民將精兵救之抜淵於萬衆之中會步兵至合擊大破之|{
	考異曰新舊唐書本紀皆云十三年拜太原留守新書仍云擊高陽歷山飛甄翟兒於西河破之今從隋帝}


|{
	紀}
帝疎薄骨肉蔡王智積每不自安及病不呼醫臨終謂所親曰吾今日始知得保首領没於地矣 張金稱郝孝德孫宣雅高士達楊公卿等寇掠河北屠陷郡縣隋將帥敗亡者相繼|{
	將即亮翻帥所類翻}
唯虎賁中郎將蒲城王辯清河郡丞華隂揚善會數有功|{
	按隋官制無中郎將王辯傳辯自鷹揚郎將遷虎賁郎將中字衍隋志蒲城縣屬馮翊郡數所角翻}
善會前後與賊七百餘戰未嘗負敗帝遣太僕卿楊義臣討張金稱金稱營於平恩東北|{
	隋志平恩縣屬武安郡}
義臣引兵直抵臨清之西據永濟渠為營|{
	隋志臨清縣屬清河郡劉昫曰臨清漢清泉縣後魏改為臨清永濟渠大業初所開}
去金稱營四十里深溝高壘不與戰金稱日引兵至義臣營西義臣勒兵擐甲|{
	擐音宦}
約與之戰既而不出日暮金稱還營明旦復來|{
	復扶又翻下同}
如是月餘義臣竟不出金稱以為怯屢逼其營詈辱之|{
	詈力智翻}
義臣乃謂金稱曰汝明旦來我當必戰金稱易之不復設備|{
	易以䜴翻}
義臣簡精騎二千夜自館陶濟河|{
	隋志館陶縣屬武陽郡此河謂清河也}
伺金稱離營即入擊其累重|{
	離力智翻累力瑞翻重直用翻}
金稱聞之引兵還義臣從後擊之金稱大敗與左右逃於清河之東月餘楊善會討擒之吏立木於市懸其頭張其手足令仇家割食之未死間歌謳不輟詔以善會為清河通守|{
	守式又翻}
涿郡通守郭絢|{
	絢許縣翻}
將兵萬餘人討高士達士達自以才畧不及竇建德乃進建德為軍司馬悉以兵授之建德請士達守輜重自簡精兵七千人拒絢詐為與士達有隙而叛遣人請降於絢|{
	降戶江翻}
願為前驅擊士達以自効絢信之引兵隨建德至長河|{
	隋志長河縣屬平原郡舊曰廣川仁壽初改名}
不復設備建德襲之殺虜數千人斬絢首獻士達張金稱餘衆皆歸建德楊義臣乘勝至平原欲入高雞泊討之建德謂士達曰歷觀隋將|{
	將即亮翻}
善用兵者無如義臣今滅張金稱而來其鋒不可當請引兵避之使其欲戰不得坐費歲月將士疲倦然後乘間擊之|{
	間古莧翻}
乃可破也不然恐非公之敵士達不從留建德守營自帥精兵逆擊義臣|{
	帥讀曰率}
戰小勝因縱酒高宴建德聞之曰東海公未能破敵遽自矜大禍至不久矣後五日義臣大破士達於陳斬之乘勝逐北趣其營|{
	陳讀曰陣趣七喻翻}
營中守兵皆潰建德與百餘騎亡去至饒陽|{
	隋志饒陽縣屬河間郡騎奇寄翻}
乘其無備攻陷之收兵得三千餘人義臣既殺士達以為建德不足憂引去建德還平原收士達散兵收葬死者為士達發喪軍復大振|{
	為于偽翻 考異曰革命記曰高士達高德政與宗族鳩集離散得五萬人捺渦於四根柳樹入高雞泊中德政自號東海公以建德為長史俄而德政病死即有高攩脱繼立為東海公建德仍依舊任攩脱領兵刼抄至宴城府為城中兵所射而死賊之異姓皆欲建德為主高氏一族不欲更立别人遂分為兩軍各相猜貳然高氏兵精強建德恐被屠乃詐分為官軍告高氏併力共擊之高氏無疑即合軍共闘兵刃纔交建德自後擊之高氏兵大亂建德兩軍擁掠遣坐簡其驍勇及頭首千餘人殺之遂總統其衆建德自號長樂王寇抄州縣即大業十二年二月也今從隋唐書}
自稱將軍先是羣盗得隋官及士族子弟皆殺之|{
	先悉薦翻}
獨建德善遇之由是隋官稍以城降之聲勢日盛勝兵至十餘萬人|{
	降戶江翻下同勝音升}
内史侍郎虞世基以帝惡聞賊盗|{
	惡烏路翻}
諸將及郡縣有告敗求救者|{
	將即亮翻}
世基皆抑損表狀不以實聞但云鼠竊狗盗郡縣捕逐行當殄盡願陛下勿以介懷帝良以為然或杖其使者以為妄言由是盗賊徧海内陷没郡縣帝皆弗之知也楊義臣破降河北賊數十萬列狀上聞|{
	所降者皆張金稱高士達之衆將即亮翻使疏吏翻上時掌翻}
帝歎曰我初不聞賊頓如此義臣降賊何多也世基對曰小竊雖多未足為慮義臣克之擁兵不少|{
	少詩沼翻下同}
久在閫外此最非宜帝曰卿言是也遽追義臣放散其兵賊由是復盛|{
	復扶又翻}
治書侍御史韋雲起劾奏世基及御史大夫裴藴職典樞要維持内外四方告變不為奏聞|{
	治直之翻劾戶槩翻又戶得翻為于偽翻}
賊數實多裁減言少陛下既聞賊少發兵不多衆寡懸殊往皆不克故使官軍失利賊黨日滋請付有司結正其罪大理卿鄭善果奏雲起詆訾名臣|{
	訾將此翻}
所言不實非毁朝政妄作威權|{
	朝直遙翻}
由是左遷雲起為大理司直|{
	唐六典後魏永安三年御史中尉高穆奏置司直十人視五品隸廷尉位在正監上不置曹事唯覆理御史劾事北齊及隋因之}
帝至江都江淮郡官謁見者|{
	見賢遍翻}
專問禮餉豐薄豐則超遷丞守薄則率從停解江都郡丞王世充獻銅鏡屏風遷通守|{
	守式又翻}
歷陽郡丞趙元楷獻異味遷江都郡丞|{
	帝改和州為歷陽郡趙元楷自小郡丞遷大郡丞}
由是郡縣競務刻剝以充貢獻民外為盗賊所掠内為郡縣所賦生計無遺加之饑饉無食|{
	無穀曰饑無蔬果曰饉}
民始采樹皮葉或擣藁為末或煮土而食之諸物皆盡乃自相食而官食猶充牣吏皆畏灋莫敢賑救王世充密為帝簡閲江淮民間美女獻之由是益有寵 河間賊帥格謙擁衆十餘萬據豆子䴚|{
	帝改瀛州為河間郡姓苑格姓允格之後帥所類翻䴚各朗翻}
自稱燕王帝命王世充將兵討斬之謙將勃海高開道|{
	帝改滄州為勃海郡燕因肩翻將即亮翻下同}
收其餘衆寇掠燕地軍勢復振|{
	復扶又翻}
初帝謀伐高麗|{
	麗力知翻}
器械資儲皆積於涿郡涿郡人物殷阜屯兵數萬又臨朔宫多珍寶諸賊競來侵掠留守官虎賁郎將趙什住等不能拒唯虎賁郎將雲陽羅藝獨出戰|{
	守式又翻賁音奔將即亮翻雲陽縣屬京兆郡}
前後破賊甚衆威名日重什住等隂忌之藝將作亂先宣言以激其衆曰吾輩討賊數有功|{
	數所角翻}
城中倉庫山積制在留守之官而莫肯散施以濟貧乏|{
	施式豉翻}
將何以勸將士衆皆憤怨軍還郡丞出城候藝藝因執之陳兵而入什住等懼皆來聽命乃發庫物以賜戰士開倉廩以賑貧乏境内咸服殺不同己者勃海太守唐禕等數人|{
	守式又翻}
威振燕地柳城懷遠並歸之藝黜柳城太守楊林甫改郡為營州|{
	隋志柳城縣帶遼西郡與襄平郡蓋皆帝所置改郡為州示復開皇之舊也}
以襄平太守鄧暠為總管|{
	暠古老翻}
藝自稱幽州總管 突厥數寇北邊|{
	厥九勿翻}
詔晉陽留守李淵帥太原道兵與馬邑太守王仁恭擊之|{
	晉陽留守即太原留守也太原有晉陽宫故亦稱晉陽太守帝改朔州為馬邑郡帥讀曰率下同守式又翻}
時突厥方彊兩軍衆不滿五千仁恭患之淵選善騎射者二千人使之飲食舍止一如突厥或與突厥遇則伺便擊之前後屢捷突厥頗憚之|{
	前後屢得小捷耳曰頗憚者未深憚也厥九勿翻騎奇寄翻伺相吏翻}


恭皇帝|{
	諱侑封代王元德太子昭之子煬帝之孫也諡法尊賢讓善曰恭}


義寧元年|{
	是年十一月李淵克長安方奉代王即位改元通鑑因以繫年}
春正月右禦衛將軍陳稜討杜伏威伏威帥衆拒之稜閉壁不戰伏威遺以婦人之服謂之陳姥|{
	遺于季翻姥莫補翻}
稜怒出戰伏威奮擊大破之稜僅以身免 |{
	考異曰隋陳稜傳云往往克捷唐杜伏威傳云稜僅以身免蓋稜先破李子通等後為伏威所敗也今從唐書}
伏威乘勝破高郵|{
	隋志高郵縣屬江都郡}
引兵據歷陽自稱總管以輔公祏為長史|{
	祏音石長知兩翻}
分遣諸將徇屬縣所至輒下|{
	將即亮翻}
江淮間小盗爭附之伏威常選敢死之士五千人謂之上募寵遇甚厚有攻戰輒令上募先擊之戰罷閲視有傷在背者即殺之以其退而被擊故也|{
	被皮義翻}
所獲資財皆以賞軍士有戰死者以妻妾徇葬故人自為戰所向無敵 丙辰竇建德為壇於樂壽|{
	隋志樂壽縣屬河間郡古樂城縣仁夀初更名樂音洛下同}
自稱長樂王置百官改元丁丑 |{
	考異曰許敬宗太宗實錄舊唐帝紀皆云武德元年二月建德稱長樂王按建德改元丁丑即是今歲今從隋帝紀及建德傳}
辛巳魯郡賊徐圓朗攻陷東平分兵略地自琅邪以西北至東平盡有之|{
	煬帝改兖州為魯郡改鄆州為東平郡沂州為琅邪郡邪音耶}
勝兵二萬餘人|{
	勝音升}
盧明月轉掠河南至于淮北衆號四十萬自稱無上王帝命江都通守王世充討之世充與戰於南陽大破之|{
	隋志南陽郡舊置荆州開皇初改鄧州煬帝改為郡守式又翻}
斬明月餘衆皆散 二月壬午朔方鷹揚郎將梁師都|{
	帝改夏州為朔方郡將即亮翻}
殺郡丞唐世宗據郡自稱大丞相北連突厥 馬邑太守王仁恭|{
	煬帝改朔州為馬邑郡}
多受貨賂不能振施|{
	施式豉翻}
郡人劉武周驍勇喜任俠|{
	驍堅堯翻喜許記翻}
為鷹揚府校尉|{
	煬帝改大都督為校尉校戶教翻}
仁恭以其土豪甚親厚之令帥親兵屯閤下|{
	帥讀曰率}
武周與仁恭侍兒私通恐事泄謀作亂先宣言曰今百姓饑饉僵尸滿道王府君閉倉不賑卹豈為民父母之意乎|{
	僵居良翻賑津忍翻下同}
衆皆憤怒武周稱疾卧家豪傑來候問武周椎牛縱酒因大言曰壯士豈能坐待溝壑今倉粟爛積誰能與我共取之豪傑皆許諾己丑仁恭坐聽事|{
	聽讀曰廳}
武周上謁|{
	上時掌翻}
其黨張萬歲等隨入升階斬仁恭持其首出徇郡中無敢動者於是開倉以賑饑民馳檄境内屬城皆下之收兵得萬餘人武周自稱太守|{
	守式又翻考異曰創業注云二月己丑馬邑軍人劉武周殺太守王仁恭據其郡自稱天子國號定楊按唐書武周據}


|{
	汾陽宫乃僭號于時未也}
遣使附于突厥|{
	使疏吏翻厥九勿翻}
李密說翟讓曰|{
	說式芮翻翟萇伯翻}
今東都空虚兵不素練越王冲幼留守諸官政令不壹士民離心段達元文都闇而無謀以僕料之彼非將軍之敵若將軍能用僕計天下可指麾而定乃遣其黨裴叔方覘東都虚實|{
	守式又翻令力定翻覘丑亷翻又丑艶翻}
留守官司覺之始為守禦之備且馳表告江都密謂讓曰事勢如此不可不發兵灋曰先則制於已後則制於人今百姓饑饉洛口倉多積粟去都百里有餘|{
	都謂東都}
將軍若親帥大衆輕行掩襲|{
	帥讀曰率}
彼遠未能救又先無豫備取之如拾遺耳比其聞知|{
	比必寐翻}
吾已獲之發粟以賑窮乏遠近孰不歸附百萬之衆一朝可集枕威養鋭|{
	枕職任翻}
以逸待勞縱彼能來吾有備矣然後檄召四方引賢豪而資計策選驍悍而授兵柄|{
	驍堅堯翻悍戶旰翻又下罕翻}
除亡隋之社稷布將軍之政令豈不盛哉讓曰此英雄之畧非僕所堪惟君之命盡力從事請君先發僕為後殿|{
	殿丁甸翻}
庚寅密讓將精兵七千人|{
	將即亮翻}
出陽城北踰方山自羅口襲興洛倉破之|{
	隋志陽城縣屬河南郡陸渾縣有方山鞏縣有興洛倉魏收地形志鞏縣有長羅川羅口蓋即長羅川口水經羅水出方山西北流謂之長羅川又西北過訾城東北而入于洛括地志方山在洛州汜水縣東南三十二里汜水所出也}
開倉恣民所取老弱襁負道路相屬|{
	襁居兩翻屬之欲翻}
朝散大夫時德以尉氏應密|{
	時姓楚大夫申叔時之後隋志尉氏縣屬頴川郡尉氏漢縣也應劭曰古獄官曰尉氏鄭之别獄也臣瓚曰鄭大夫尉氏之邑故遂以名邑朝直遙翻散悉亶翻}
前宿城令祖君彦自昌平往歸之|{
	隋志宿城縣屬東平郡開皇十六年置劉昫曰漢須昌縣故城在今鄆州東南三十二里隋於此置宿城縣昌平縣隋志屬涿郡}
君彦珽之子也|{
	珽他鼎翻}
博學強記文辭贍敏著名海内吏部侍郎薛道衡嘗薦之於高祖高祖曰是歌殺斛律明月人兒邪|{
	歌殺斛律光事見一百七十一卷陳高宗太建四年邪音耶}
朕不須此輩煬帝即位尤疾其名依常調選東平書佐檢校宿城令|{
	隋制州郡皆有書佐在祭酒從事之上視正九品謂之流内視品檢校官未得為真調徒釣翻選宣戀翻}
君彦自負其才常鬱鬱思亂密素聞其名得之大喜引為上客軍中書檄一以委之越王侗遣虎賁郎將劉長恭光祿少卿房崱|{
	侗音通賁音奔將即亮翻少始照翻崱士力翻}
帥步騎二萬五千討密|{
	帥讀曰率下同騎奇寄翻}
時東都人皆以密為饑賊盗米烏合易破|{
	易以豉翻}
爭來應募國子三舘學士|{
	隋以國子太學四門為三館}
及貴勝親戚皆來從軍器械修整衣服鮮華旌旗鉦鼓甚盛長恭等當其前使河南討捕大使裴仁基等將所部兵自汜水西入以掩其後|{
	大使疏吏翻將即亮翻汜音祀}
約十一日會于倉城 |{
	按考異曰蒲山公傳云剋取二十一日會戰於河洛故曰取其月十二日會戰按下有庚子則非二十一日也當是十一日}
密讓具知其計東都兵先至士卒未朝食長恭等驅之度洛水陳於石子河西|{
	水經注洞水出南溪石泉世亦名之為石泉水過鞏東坎欿聚西而北入于洛蓋即石子河也陳讀曰陣下同}
南北十餘里密讓選驍雄分為十隊|{
	驍堅堯翻}
令四隊伏横嶺下以待仁基以六隊陳於石子河東長恭等見密兵少輕之|{
	少詩沼翻}
讓先接戰不利密帥麾下横衝之隋兵饑疲遂大敗長恭等解衣濳竄得免奔還東都士卒死者什五六越王侗釋長恭等罪慰撫之密讓盡收其輜重器甲|{
	重直用翻}
威聲大振讓於是推密為主上密號為魏公|{
	上時掌翻}
庚子設壇場即位稱元年 |{
	考異曰壺關錄云王伯當令密於西垣校射書王字於堋上如錢約中者為主其次以遠近為拜官高下使賈雄執箭仰天而誓密正中字心遂奉以為主其說鄙陋今不取河洛記云改大業十二年為永平元年今從蒲山公傳及隋唐書}
大赦其文書行下|{
	下遐稼翻}
稱行軍元帥府|{
	帥所類翻下同}
其魏公府置三司六衛元帥府置長史以下官屬拜翟讓為上柱國司徒東郡公|{
	長知兩翻 考異曰河洛記云鄧公蓋後來進封耳今從蒲山公傳及隋唐書}
亦置長史以下官減元帥府之半以單雄信為左武候大將軍|{
	單音善}
徐世勣為右武候大將軍各領所部房彦藻為元帥左長史東郡邴元真為右長史楊德方為左司馬鄭德韜為右司馬祖君彦為記室其餘封拜各有差於是趙魏以南江淮以北羣盗莫不響應孟讓郝孝德王德仁|{
	郝呼各翻}
及濟隂房獻伯上谷王君廓長平李士才淮陽魏六兒李德謙譙郡張遷魏郡李文相譙郡黑社白社濟北張青特上洛周比洮胡驢賊等皆歸密|{
	隋志長平郡舊曰建州開皇初改為澤州煬帝改為郡譙郡後魏置南兖州後周改亳州煬帝改為郡改濟州為濟北郡商州為上洛郡黑社白社蓋賊之號非人姓名也濟子禮翻相息亮翻比毗至翻洮土刀翻}
密悉拜官爵使各領其衆置百營簿以領之道路降者不絶如流|{
	降戶江翻}
衆至數十萬 |{
	考異曰二月丙辰密遣其將夜襲倉城二府兵擊退之己未又悉衆來攻而府兵敗遂入據倉然二府將士猶各固小倉城二十餘日不下既而外救不至食又盡城乃陷没死者大半於是鞏縣長柴孝和監察御史鄭頲等舉縣降賊密開倉招納降者日數百千人於是趙魏以南江淮以北莫不歸附自是賊徒滋蔓矣壬子遣劉長恭房崱等繞兵東討大敗戊午還都王慰撫不責也於是發教募士庶商旅奴等分置營壁各立將帥統領而固守其諸里居民皆移入三城之内於省寺府舍安置焉又使宋遵貴將兵鎮陜縣太原倉雜記密稱魏公改年于時倉猶自固守既而密遣翟讓將兵夜襲倉城官軍擊退之明日又引衆攻倉連戰三日陷外城官軍猶捉子城月餘外援不至城盡陷没死者十六七按二月壬午朔無丙辰等日今從隋書}
乃命其護軍田茂廣築洛口城方四十里而居之|{
	考異曰壺關錄云周四十八里今從隋書}
密遣房彦藻將兵東略地取安陸汝南淮安濟陽|{
	煬帝改安州為安陸郡隋志汝南郡後魏置豫州帝改洛州為豫州以此為溱州又改曰蔡州尋改為郡淮安郡後魏置東荆州西魏改為淮州開皇五年又改為顯州煬帝改為郡濟陽縣屬濟隂郡濟子禮翻}
河南郡縣多陷於密 鴈門郡丞河東陳孝意與虎賁郎將王智辯共討劉武周圍其桑乾鎮|{
	桑乾漢縣後魏為桑乾郡後周廢隋以為鎮在馬邑郡義陽縣界賁音奔將即亮翻乾音干}
壬寅武周與突厥合兵擊智辯殺之孝意奔還鴈門三月丁卯武周襲破樓煩郡進取汾陽宫獲隋宫人以賂突厥始畢可汗|{
	厥九勿翻可從刋入聲汗音寒}
始畢以馬報之兵勢益振又攻陷定襄|{
	煬帝改雲州為定襄郡}
突厥立武周為定楊可汗|{
	言將使之定楊州也 考異曰新舊唐書武周皆無國號惟創業起居注云國號定楊}
遺以狼頭纛|{
	隋書曰突厥本狼種牙門建狼頭纛示不忘本也遺于季翻}
武周即皇帝位立妻沮氏為皇后|{
	沮子余翻}
改元天興以衛士楊伏念為尚書左僕射妹婿同縣苑君璋為内史令武周引兵圍鴈門陳孝意悉力拒守乘間出擊武周屢破之|{
	間古莧翻下同}
既而外無救援遣間使詣江都皆不報|{
	使疏吏翻}
孝意誓以必死旦夕向詔敕庫俯伏流涕悲動左右圍城百餘日食盡校尉張倫殺孝意以降|{
	校戶教翻降戶江翻}
梁師都略定雕隂弘化延安等郡|{
	隋志雕隂郡西魏置綏州大業初改為上郡尋改為雕隂郡改慶州為弘化郡}
遂即皇帝位國號梁改元永隆始畢遺以狼頭纛號為大度毗伽可汗師都乃引突厥居河南之地攻破鹽川郡|{
	隋志鹽川郡西魏置西安州後改為鹽州煬帝改為郡}
左翊衛蒲城郭子和|{
	郭子和蓋衛士之屬左翊衛府者}
坐事徙榆林會郡中大饑子和濳結敢死士十八人攻郡門執郡丞王才數以不恤百姓斬之|{
	數所具翻}
開倉賑施自稱永樂王|{
	施式豉翻樂音洛}
改元丑平尊其父為太公以其弟子政為尚書令子端子升為左右僕射有二千餘騎南連梁師都北附突厥各遣子為質以自固|{
	騎奇寄翻厥九勿翻質音致}
始畢以劉武周為定楊天子梁師都為解事天子|{
	解戶買翻}
子和為平楊天子|{
	平楊猶定楊也}
子和固辭不敢當乃更以為屋利設 汾隂薛舉僑居金城|{
	隋志汾隂縣屬河東郡煬帝改蘭州為金城郡僑寓也}
驍勇絶倫家資鉅萬交結豪傑雄於西邊為金城府校尉|{
	新唐志金城郡有府二曰廣武金城校尉其帥驍堅堯翻校戶教翻}
時隴右盗起金城令郝瑗|{
	金城縣帶郡郝呼各翻瑗于眷翻}
募兵得數千人使舉將而討之|{
	將即亮翻}
夏四月癸未方授甲置酒饗士舉與其子仁果及同黨十三人於座劫瑗發兵 |{
	考異曰唐高祖實錄先作仁果後作仁杲新舊高祖太宗紀薛舉傳柳芳唐歷柳宗元集皆作仁杲太宗實錄吳競太宗勲吏革命記焦璐唐朝年代記陳嶽唐統記皆作仁果今醴泉昭陵前有石馬六匹其一銘曰白蹄烏平薛仁果時所乘此最可據今從之}
囚郡縣官開倉賑施|{
	賑津忍翻施式智翻}
自稱西秦覇王改元秦興以仁果為齊公少子仁越為晉公招集羣盗掠官牧馬賊帥宗羅㬋帥衆歸之以為義興公|{
	少詩照翻賊帥所類翻睺帥讀曰率}
將軍皇甫綰將兵一萬屯枹罕|{
	煬帝改河州為枹罕郡綰將即亮翻枹音膚}
舉選精鋭二千人襲之岷山羌酋鍾利俗擁衆二萬歸之|{
	隋志臨洮郡臨洮縣有岷山酋才由翻}
舉兵大振更以仁果為齊王領東道行軍元帥|{
	帥所類翻}
仁越為晉王兼河州刺史|{
	復以枹罕郡為河州}
羅㬋為興王以副仁果分兵略地取西平澆河二郡|{
	煬帝改鄯州為西平郡周武帝逐吐谷渾置廓州煬帝改為澆河郡澆古堯翻}
未幾盡有隴西之地衆至十三萬|{
	幾居豈翻}
李密以孟讓為總管齊郡公 |{
	考異曰河洛記作孟達今從隋書}
己丑夜讓帥步騎二千入東都外郭|{
	外郭羅郭也帥讀曰率騎奇計翻}
燒掠豐都市比曉而去|{
	比必寐翻}
於是東京居民悉遷入宫城臺省府寺皆滿鞏縣長柴孝和監察御史鄭頲以城降密|{
	長知兩翻下同頲它鼎翻降戶江翻下同}
密以孝和為護軍頲為右長史裴仁基每破賊得軍資悉以賞士卒監軍御史蕭懷靜不許|{
	監工銜翻}
士卒怨之懷靜又屢求仁基長短劾奏之|{
	劾戶槩翻又戶得翻}
倉城之戰仁基失期不至聞劉長恭等敗懼不敢進屯百花谷|{
	百花谷蓋在汜水縣西鞏縣東南}
固壘自守又恐獲罪於朝|{
	朝直遥翻}
李密知其狼狽|{
	集韻狽音貝狼屬生子或欠一足二足相附而行離則蹎故猝遽謂之狼狽}
使人說之|{
	說輸芮翻}
啗以厚利|{
	啗徒濫翻}
賈務本之子閏甫在軍中|{
	賈務本見上滎陽之戰}
勸仁基降密仁基曰如蕭御史何閏甫曰蕭君如棲上雞若不知機變在明公一刀耳仁基從之遣閏甫詣密請降密大喜以閏甫為元帥府司兵參軍兼直記室事使之復命遺仁基書慰納之|{
	遺于季翻}
仁基還屯虎牢蕭懷靜密表其事仁基知之遂殺懷靜帥其衆以虎牢降密密以仁基為上柱國河東公仁基子行儼驍勇善戰密亦以為上柱國絳郡公|{
	驍堅堯翻下同}
密得秦叔寶及東阿程齩金|{
	隋志東阿縣屬濟北郡齩五巧翻}
皆用為驃騎|{
	驃騎開皇官制也煬帝改為鷹揚郎將驃匹妙翻騎奇寄翻}
選軍中尤驍勇者八千人分隸四驃騎以自衛號曰内軍常曰此八千人足當百萬齩金後更名知節|{
	更工衡翻}
羅士信趙仁基皆帥衆歸密密署為總管使各統所部癸巳密遣裴仁基孟讓帥二萬餘人襲囘洛東倉破之|{
	新唐志孟州河陽有囬洛故城是地得名之由見一百五十八卷梁武帝大同九年}
遂燒天津橋|{
	煬帝使宇文愷營造東都洛水貫都有河漢之象因名其橋為天津橋}
縱兵大掠東都出兵擊之仁基等敗走密自帥衆屯囘洛倉東都兵尚二十餘萬人乘城擊柝|{
	柝他各翻}
晝夜不解甲密攻偃師金墉皆不克|{
	偃師縣屬河南郡在都城東六十里晉金墉城在洛城西北隋營東都城東去故都十八里則金墉亦在都城之東}
乙未還洛口 |{
	考異曰畧記三月辛未密遣孟讓將二千餘人夜入都郭燒豐都市比曉而去癸未密襲據都倉乙亥密部衆入自上春門於宣仁門東街立栅而住丙寅燒上春門及街南北里門樓火接宣仁門因逼門為陳與城上弓矢相接而退還倉雜記密遣格謙將兵燒豐都市三月越王侗教募人捉宫城守固官賞有差撤天津等諸橋運回洛倉米入城四月密攻偃師圍金墉東都兵出密還洛口五月裴仁基翻虎牢入賊自滎陽以東至陳譙下邳彭城梁郡皆屬密賊衆逾盛并家口百萬蒲山公傳三月丁亥密帥衆入自上東門攻宣仁門不克丙寅燒上東門而退此三書月日交錯皆不可憑今從隋唐書}
東都城内乏糧而布帛山積至以絹為汲綆|{
	綆古杏翻}
然布以㸑越王侗使人運囘洛倉米入城遣兵五千屯豐都市五千屯上春門五千屯北邙山為九營首尾相應以備密丁酉房獻伯陷汝隂|{
	煬帝改頴州為汝隂郡}
淮陽太守趙陁舉郡降密|{
	守式又翻陁徒河翻降戶江翻 考異曰隋書作趙佗今從蒲山公傳}
己亥密帥衆三萬復據囘洛倉大修營塹以逼東都|{
	復扶又翻塹七艷翻}
段達等出兵七萬拒之辛丑戰於倉北隋兵敗走丁未密使其幕府移檄郡縣數煬帝十罪|{
	數所具翻又所主翻}
且曰罄南山之竹書罪無窮决東海之波流惡難盡祖君彦之辭也越王侗遣太常丞元善達間行賊中|{
	間古莧翻}
詣江都奏稱李密有衆百萬圍逼東都據洛口倉城内無食若陛下速還烏合必散不然者東都决没因歔欷嗚咽帝為之改容|{
	歔音虚欷音希又許既翻為于偽翻}
虞世基進曰越王年少|{
	少詩照翻}
此輩誑之|{
	誑居况翻}
若如所言善達何緣來至帝乃勃然怒曰善達小人敢廷辱我因使經賊中向東陽催運|{
	此東陽蓋指婺州東陽郡}
善達遂為羣盗所殺是後人人杜口莫敢以賊聞世基容貌沈審|{
	沈持林翻}
言多合意特為帝所親愛朝臣無與為比親黨憑之鬻官賣獄賄賂公行其門如市由是朝野共疾怨之|{
	朝直遥翻}
内史舍人封德彛託附世基以世基不閑吏務|{
	閑習也}
密為指畫宣行詔命諂順帝意羣臣表疏忤旨者皆屛而不奏|{
	為于偽翻忤五故翻屛必郢翻}
鞫獄用灋多峻文深詆論功行賞則抑削就薄故世基之寵日隆而隋政益壞皆德彞所為也 初唐公李淵娶於神武肅公竇毅|{
	神武郡名隋志馬邑郡神武縣舊置神武郡}
生四男建成世民玄覇元吉一女適太子千牛備身臨汾柴紹|{
	隋志東宫左右内率府有千牛備身八人掌執千牛刀以千牛名刀者取其解千牛而芒刃不頓臨汾縣帶臨汾郡本平陽也開皇初改名}
世民聰明勇决識量過人見隋室方亂隂有安天下之志傾身下士散財結客咸得其歡心|{
	下遐稼翻}
世民娶右驍衛將軍長孫晟之女|{
	驍堅堯翻長知兩翻晟承正翻}
右勲衛長孫順德晟之族弟也與右勲侍池陽劉弘基|{
	隋開皇置親勲武三衛大業初改為親勲武三侍順德蓋開皇中為勲衛弘基則為大業勲侍也三衛三侍皆分左右劉弘基雍州池陽人隋志雍州有雲陽縣無池陽舊唐志云貞觀三年改石門為雲陽雲陽為池陽通鑑據唐書以唐州縣書之也}
皆避遼東之役亡命在晉陽依淵與世民善左親衛竇琮熾之孫也|{
	竇熾隋初三公}
亦亡命在太原素與世民有隙每以自疑世民加意待之出入卧内琮意乃安晉陽宫監猗氏裴寂|{
	隋離宫皆置宫監猗氏縣屬河東郡}
晉陽令武功劉文靜|{
	晉陽縣帶太原郡武功縣屬京兆郡}
相與同宿見城上烽火寂嘆曰貧賤如此復逢亂離|{
	復扶又翻}
將何以自存文靜笑曰時事可知吾二人相得何憂貧賤文靜見李世民而異之深自結納謂寂曰此非常人豁達類漢高神武同魏祖年雖少命世才也|{
	少詩照翻}
寂初未然之|{
	未知文靜之言為是}
文靜坐與李密連昏繫太原獄|{
	繫郡獄}
世民就省之文靜曰天下大亂非高光之才不能定也|{
	以漢高祖光武之事擿發世民省悉景翻下同}
世民曰安知其無但人不識耳我來相省非兒女子之情欲與君議大事也計將安出文靜曰今主上南廵江淮李密圍逼東都羣盗殆以數萬當此之際有真主驅駕而用之取天下如反掌耳太原百姓皆避盗入城文靜為令數年知其豪傑一旦收拾可得十萬人尊公所將之兵復且數萬|{
	將即亮翻復扶又翻}
一言出口誰敢不從以此乘虚入關號令天下不過半年帝業成矣世民笑曰君言正合吾意乃隂部署賓客淵不之知也世民恐淵不從猶豫久之不敢言淵與裴寂有舊每相與宴語或連日夜文靜欲因寂關說|{
	關白也說輸芮翻下同}
乃引寂與世民交世民出私錢數百萬使龍山令高斌亷與寂博稍以輸之|{
	按隋志後齊置龍山縣帶太原郡開皇十年改曰晉陽則此時不復有龍山矣豈斌亷在開皇中嘗為令史以舊官書之邪對慱者不勝者納物與勝者曰輸斌音彬}
寂大喜由是日從世民遊情欵益狎世民乃以其謀告之寂許諾會突厥寇馬邑淵遣高君雅將兵與馬邑太守王仁恭并力拒之仁恭君雅戰不利|{
	按王仁恭是年春已死此必去年史序李淵起兵來歷及之厥九勿翻將即亮翻守式又翻}
淵恐并獲罪甚憂之世民乘間屏人說淵曰|{
	間古莧翻屏必郢翻}
今主上無道百姓困窮晉陽城外皆為戰場大人若守小節下有寇盗上有嚴刑危亡無日不若順民心興義兵轉禍為福此天授之時也淵大驚曰汝安得為此言吾今執汝以告縣官因取紙筆欲為表世民徐曰世民觀天時人事如此故敢發言必欲執告不敢辭死淵曰吾豈忍告汝汝慎勿出口明日世民復說淵曰今盗賊日繁遍於天下大人受詔討賊賊可盡乎要之終不免罪且世人皆傳李氏當應圖䜟|{
	復扶又翻說式芮翻䜟楚譖翻}
故李金才無罪一朝族滅|{
	李渾字金才族滅事見上卷大業十一年}
大人設能盡賊則功高不賞身益危矣唯昨日之言可以救禍此萬全之策也願大人勿疑淵乃嘆曰吾一夕思汝言亦大有理今日破家亡軀亦由汝化家為國亦由汝矣先是裴寂私以晉陽宫人侍淵|{
	先悉薦翻}
淵從寂飲酒酣寂從容言曰二郎隂養士馬欲舉大事正為寂以宫人侍公|{
	世民第二從千容翻為于偽翻}
恐事覺并誅為此急計耳衆情已恊公意如何淵曰吾兒誠有此謀事已如此當復奈何正須從之耳|{
	復扶又翻下同}
帝以淵與王仁恭不能禦寇遣使者執詣江都|{
	此帝謂煬帝使疏吏翻}
淵大懼世民與寂等復說淵曰今主昏國亂盡忠無益偏禆失律而罪及明公|{
	說式芮翻禆賓彌翻}
事已迫矣宜早定計且晉陽士馬精彊宫監蓄積巨萬以茲舉事何患無成代王幼冲關中豪傑並起未知所附公若鼓行而西撫而有之如探囊中之物耳|{
	探吐南翻}
柰何受單使之囚|{
	使疏吏翻下遣使同}
坐取夷滅乎淵然之密部勒將發會帝繼遣使者馳驛赦淵及仁恭使復舊任|{
	考異曰創業注曰隋主遣司直姓名馳驛繫帝而斬仁恭帝自以姓名著於圖錄太原王氣所在恐被猜忌}


|{
	因而禍及頗有所悔時皇太子在河東獨有秦王侍側耳語謂王曰隋歷將盡吾家繼膺符命不早起兵者顧爾兄弟未集耳今遭羑里之厄爾昆季須會孟津之師不可從吾同受拏戮家破身亡為英雄笑王泣而啓帝曰芒碭山澤是處容人請同漢祖以觀時變帝曰今遇時來逢茲錮繫雖覩機變何能為也然天命有在吾應會昌未必不以此相啓今吾激厲謹當敬天之誡以卜興亡自天祐吾彼焉能害天必亡我何所逃刑乃後數日果有詔使馳驛而至釋淵而免仁恭各依舊檢校所部案煬帝若有詔斬仁恭則比後使之至仁恭已死矣又高祖身為留守且被禁繫亡去何之恐此亦非太宗之謀也今皆不取}
淵謀亦緩淵之為河東討捕使也|{
	大業十一年淵為使討捕河東使疏吏翻}
請大理司直夏侯端為副端詳之孫也|{
	梁武帝起兵荆雍夏侯詳佐命夏戶雅翻}
善占候及相人|{
	相息亮翻}
謂淵曰今玉牀搖動帝座不安|{
	晉天文志北極五星第二星主帝座太乙之座謂最赤明者紫宫門内六星曰天牀主寢舍解息燕休又大角一星在攝提間大角者天王帝座也天官書云大角北三星為帝座主宴飲酬酢也}
參墟得歲必有真人起於其分|{
	左傳參為晉星故以晉陽為參墟得歲謂歲星居參也參所今翻分扶問翻}
非公而誰乎主上猜忍尤忌諸李金才既死公不思變通必為之次矣淵心然之及留守晉陽鷹揚府司馬太原許世緒|{
	煬帝制鷹揚府有司馬及兵倉兩司}
說淵曰公姓在圖籙名應歌謠握五郡之兵|{
	五郡謂太原鴈門馬邑樓煩西河說式苪翻}
當四戰之地舉事則帝業可成端居則亡不旋踵唯公圖之行軍司鎧文水武士彠|{
	按士彠傳蓋為行軍府司鎧參軍隋志文水縣屬太原郡舊曰受陽開皇十年改名宋白曰文水縣漢大陵縣後魏省大陵於今處置受陽縣隋改曰文水彠一虢翻}
前太子左勲衛唐憲|{
	開皇之制東宫左右衛率府亦有親勲翊三衛煬帝改親衛為功曹}
憲弟儉皆勸淵舉兵儉說淵曰明公北招戎狄南收豪傑以取天下此湯武之舉也淵曰湯武非所敢擬在私則圖存在公則拯亂卿姑自重吾將思之憲邕之孫也|{
	唐邕以彊幹事高齊}
時建成元吉尚在河東|{
	淵留建成護家居河東}
故淵遷延未發劉文靜謂裴寂曰先發制人後發制於人何不早勸唐公舉兵而推遷不已|{
	推遷言推故遷延也推吐雷翻}
且公為宫監而以宫人侍客公死可爾何誤唐公也寂甚懼屢趣淵起兵|{
	趣讀曰促}
淵乃使文靜詐為敕書發太原西河鴈門馬邑民年二十已上五十已下悉為兵期歲暮集涿郡擊高麗|{
	麗力知翻}
由是人情恟恟思亂者益衆|{
	恟許拱翻}
及劉武周據汾陽宫世民言於淵曰大人為留守而盗賊竊據離宫不早建大計禍今至矣淵乃集將佐謂之曰|{
	將即亮翻下同}
武周據汾陽宫吾輩不能制罪當族滅若之何王威等皆懼再拜請計淵曰朝廷用兵動止皆稟節度今賊在數百里内江都在三千里外加以道路險要復有他賊據之|{
	復扶又翻}
以嬰城膠柱之兵當巨猾豕突之勢必不全矣進退維谷何為而可威等皆曰公地兼親賢同國休戚若俟奏報豈及事機要在平賊專之可也淵陽若不得已而從之者曰然則先當集兵乃命世民與劉文靜長孫順德劉弘基等各募兵遠近赴集旬日間近萬人仍密遣使召建成元吉於河東柴紹於長安|{
	近其靳翻使疏吏翻}
王威高君雅見兵大集疑淵有異志謂武士彠曰順德弘基皆背征三寺所犯當死|{
	二人避役亡命故曰背征背蒲妹翻}
安得將兵欲收按之士彠曰二人皆唐公客若爾必大致紛紜威等乃止留守司兵田德平欲勸威等按募人之狀|{
	隋制留守置司功倉戶兵法士曹等書佐守式又翻}
士彠曰討捕之兵悉隸唐公威君雅但寄坐耳|{
	言但寄身於留守坐間也坐徂卧翻}
彼何能為德平亦止晉陽鄉長劉世龍|{
	開皇初置保長黨長鄉長亦此類也長知兩翻}
密告淵云威君雅欲因晉祠祈雨為不利|{
	晉陽有晉王祠}
五月癸亥夜淵使世民伏兵於晉陽宫城之外甲子旦淵與威君雅共坐視事使劉文靜引開陽府司馬胙城劉政會入立庭中|{
	新唐志太原有府十八開陽其一也隋志胙城縣屬東郡舊曰東燕開皇十八年改名}
稱有密狀|{
	言有狀告密}
淵目威等取狀視之政會不與曰所告乃副留守事唯唐公得視之淵陽驚曰豈有是邪視其狀乃云威君雅濳引突厥入寇|{
	邪音耶厥九勿翻}
君雅攘袂大詬曰|{
	詬若候翻罵也}
此乃反者欲殺我耳時世民已布兵塞衢路|{
	塞悉則翻}
文靜因與劉弘基長孫順德等共執威君雅繫獄丙寅突厥數萬衆寇晉陽輕騎入外郭北門出其東門|{
	騎奇寄翻下同}
淵命裴寂等勒兵為備而悉開諸城門突厥不能測莫敢進衆以為威君雅實召之也淵於是斬威君雅以徇淵部將王康達將千餘人出戰皆死城中忷懼|{
	將即亮翻下舉將同忷許拱翻}
淵夜遣軍濳出城旦則張旗鳴鼓自他道來如援軍者突厥終疑之留城外二日大掠而去 煬帝命監門將軍涇陽龎玉|{
	隋志左右監門府各將軍一人掌宫殿門禁及守衛事涇陽縣屬京兆郡監工銜翻}
虎賁郎將霍世舉將關内兵援東都柴孝和說李密曰|{
	賁音奔將即亮翻說式芮翻}
秦地山川之固秦漢所憑以成王業者也今不若使翟司徒守洛口裴柱國守囘洛|{
	翟司徒讓也裴柱國仁基翟萇伯翻}
明公自簡精鋭西襲長安既克京邑業固兵彊然後東向以平河洛傳檄而天下定矣方今隋失其鹿豪傑競逐不早為之必有先我者|{
	先悉薦翻}
悔無及矣密曰此誠上策吾亦思之久矣但昏主尚存從兵猶衆|{
	從才用翻}
我所部皆山東人見洛陽未下誰肯從我西入諸將出於羣盗留之各競雌雄如此則大業隳矣孝和曰然則大軍既未可西上僕請間行觀舋|{
	將即亮翻上時掌翻間古莧翻}
密許之孝和與數十騎至陜縣|{
	隋志陜縣屬河南郡騎奇寄翻陜失冉翻}
山賊歸之者萬餘人時密兵鋒甚鋭每入苑與隋兵連戰|{
	苑即大業初所築西苑}
會密為流矢所中尚卧營中丁丑越王侗使段達與龎玉等夜出兵陳於囘洛倉西北|{
	陣讀曰陣}
密與裴仁基出戰達等大破之殺傷太半密乃弃囘洛奔洛口 |{
	考異曰洛記云四月戊申段達等帥關内兵陳於倉西倉南密出兵拒戰大破兇醜密還固倉五月丁丑達等又出兵陳於倉西倉北密又來拒大破之密奔洛口按隋書北史新舊唐書皆云密為流矢所中卧營中東都出兵擊之密衆大潰弃囘洛倉奔洛口俱無月日河洛記密軍失利歸於鞏縣東都復得囬洛倉蒲山公傳曰五月二十八日越王夜出師使段達等大戰於倉西北密軍敗績歸於鞏縣亦不云密連月再敗也戊申四月二十八日丁丑五月二十八日蓋趙毅承蒲山公傳誤以密一敗分為二事也}
龎玉霍世舉軍於偃師|{
	龎薄江翻杜佑曰偃師帝嚳所都古西亳也湯亦都之武王伐紂迴師息戎遂名偃師縣屬河南郡}
柴孝和之衆聞密退各散去孝和輕騎歸密楊德方鄭德韜皆死 |{
	考異曰楊德方壺關錄作王德仁今從河洛記}
密以鄭頲為左司馬|{
	頲它鼎翻}
滎陽鄭乾象為右司馬 |{
	考異曰隋唐書皆作䖍象唯壼關錄作乾象云密殺其兄乾覆乾覆之子會通後從盛彦師殺密今從之}
李建成李元吉弃其弟智雲於河東而去吏執智雲送長安殺之建成元吉遇柴紹於道與之偕行

資治通鑑卷一百八十三
