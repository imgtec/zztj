<!DOCTYPE html PUBLIC "-//W3C//DTD XHTML 1.0 Transitional//EN" "http://www.w3.org/TR/xhtml1/DTD/xhtml1-transitional.dtd">
<html xmlns="http://www.w3.org/1999/xhtml">
<head>
<meta http-equiv="Content-Type" content="text/html; charset=utf-8" />
<meta http-equiv="X-UA-Compatible" content="IE=Edge,chrome=1">
<title>資治通鑒_149-資治通鑑卷一百四十八_149-資治通鑑卷一百四十八</title>
<meta name="Keywords" content="資治通鑒_149-資治通鑑卷一百四十八_149-資治通鑑卷一百四十八">
<meta name="Description" content="資治通鑒_149-資治通鑑卷一百四十八_149-資治通鑑卷一百四十八">
<meta http-equiv="Cache-Control" content="no-transform" />
<meta http-equiv="Cache-Control" content="no-siteapp" />
<link href="/img/style.css" rel="stylesheet" type="text/css" />
<script src="/img/m.js?2020"></script> 
</head>
<body>
 <div class="ClassNavi">
<a  href="/24shi/">二十四史</a> | <a href="/SiKuQuanShu/">四库全书</a> | <a href="http://www.guoxuedashi.com/gjtsjc/"><font  color="#FF0000">古今图书集成</font></a> | <a href="/renwu/">历史人物</a> | <a href="/ShuoWenJieZi/"><font  color="#FF0000">说文解字</a></font> | <a href="/chengyu/">成语词典</a> | <a  target="_blank"  href="http://www.guoxuedashi.com/jgwhj/"><font  color="#FF0000">甲骨文合集</font></a> | <a href="/yzjwjc/"><font  color="#FF0000">殷周金文集成</font></a> | <a href="/xiangxingzi/"><font color="#0000FF">象形字典</font></a> | <a href="/13jing/"><font  color="#FF0000">十三经索引</font></a> | <a href="/zixing/"><font  color="#FF0000">字体转换器</font></a> | <a href="/zidian/xz/"><font color="#0000FF">篆书识别</font></a> | <a href="/jinfanyi/">近义反义词</a> | <a href="/duilian/">对联大全</a> | <a href="/jiapu/"><font  color="#0000FF">家谱族谱查询</font></a> | <a href="http://www.guoxuemi.com/hafo/" target="_blank" ><font color="#FF0000">哈佛古籍</font></a> 
</div>

 <!-- 头部导航开始 -->
<div class="w1180 head clearfix">
  <div class="head_logo l"><a title="国学大师官网" href="http://www.guoxuedashi.com" target="_blank"></a></div>
  <div class="head_sr l">
  <div id="head1">
  
  <a href="http://www.guoxuedashi.com/zidian/bujian/" target="_blank" ><img src="http://www.guoxuedashi.com/img/top1.gif" width="88" height="60" border="0" title="部件查字,支持20万汉字"></a>


<a href="http://www.guoxuedashi.com/help/yingpan.php" target="_blank"><img src="http://www.guoxuedashi.com/img/top230.gif" width="600" height="62" border="0" ></a>


  </div>
  <div id="head3"><a href="javascript:" onClick="javascript:window.external.AddFavorite(window.location.href,document.title);">添加收藏</a>
  <br><a href="/help/setie.php">搜索引擎</a>
  <br><a href="/help/zanzhu.php">赞助本站</a></div>
  <div id="head2">
 <a href="http://www.guoxuemi.com/" target="_blank"><img src="http://www.guoxuedashi.com/img/guoxuemi.gif" width="95" height="62" border="0" style="margin-left:2px;" title="国学迷"></a>
  

  </div>
</div>
  <div class="clear"></div>
  <div class="head_nav">
  <p><a href="/">首页</a> | <a href="/ShuKu/">国学书库</a> | <a href="/guji/">影印古籍</a> | <a href="/shici/">诗词宝典</a> | <a   href="/SiKuQuanShu/gxjx.php">精选</a> <b>|</b> <a href="/zidian/">汉语字典</a> | <a href="/hydcd/">汉语词典</a> | <a href="http://www.guoxuedashi.com/zidian/bujian/"><font  color="#CC0066">部件查字</font></a> | <a href="http://www.sfds.cn/"><font  color="#CC0066">书法大师</font></a> | <a href="/jgwhj/">甲骨文</a> <b>|</b> <a href="/b/4/"><font  color="#CC0066">解密</font></a> | <a href="/renwu/">历史人物</a> | <a href="/diangu/">历史典故</a> | <a href="/xingshi/">姓氏</a> | <a href="/minzu/">民族</a> <b>|</b> <a href="/mz/"><font  color="#CC0066">世界名著</font></a> | <a href="/download/">软件下载</a>
</p>
<p><a href="/b/"><font  color="#CC0066">历史</font></a> | <a href="http://skqs.guoxuedashi.com/" target="_blank">四库全书</a> |  <a href="http://www.guoxuedashi.com/search/" target="_blank"><font  color="#CC0066">全文检索</font></a> | <a href="http://www.guoxuedashi.com/shumu/">古籍书目</a> | <a   href="/24shi/">正史</a> <b>|</b> <a href="/chengyu/">成语词典</a> | <a href="/kangxi/" title="康熙字典">康熙字典</a> | <a href="/ShuoWenJieZi/">说文解字</a> | <a href="/zixing/yanbian/">字形演变</a> | <a href="/yzjwjc/">金 文</a> <b>|</b>  <a href="/shijian/nian-hao/">年号</a> | <a href="/diming/">历史地名</a> | <a href="/shijian/">历史事件</a> | <a href="/guanzhi/">官职</a> | <a href="/lishi/">知识</a> <b>|</b> <a href="/zhongyi/">中医中药</a> | <a href="http://www.guoxuedashi.com/forum/">留言反馈</a>
</p>
  </div>
</div>
<!-- 头部导航END --> 
<!-- 内容区开始 --> 
<div class="w1180 clearfix">
  <div class="info l">
   
<div class="clearfix" style="background:#f5faff;">
<script src='http://www.guoxuedashi.com/img/headersou.js'></script>

</div>
  <div class="info_tree"><a href="http://www.guoxuedashi.com">首页</a> > <a href="/SiKuQuanShu/fanti/">四库全书</a>
 > <h1>资治通鉴</h1> <!--         下载:【右键另存为】即可 --></div>
  <div class="info_content zj clearfix">
  
<div class="info_txt clearfix" id="show">
<center style="font-size:24px;">149-資治通鑑卷一百四十八</center>
    資治通鑑卷一百四十八 宋 司馬光 撰<br />
<br />
  胡三省 音註<br />
<br />
  梁紀四【起旃蒙協洽盡著雍閹茂凡四年】<br />
<br />
  高祖武皇帝四<br />
<br />
  天監十四年春正月乙巳朔上冠太子於太極殿【古者冠於廟冠古玩翻】大赦 辛亥上祀南郊 甲寅魏主有疾丁巳殂于式乾殿【年三十三諡宣武皇帝廟號世宗】侍中中書監太子少傅崔光侍中領軍將軍于忠詹事王顯中庶子代人侯剛迎太子詡於東宫至顯陽殿王顯欲須明行即位禮【須待也】崔光曰天位不可暫曠何待至明顯曰須奏中宫光曰帝崩太子立國之常典何須中宫令也於是光等請太子止哭立於東序于忠與黄門郎元昭扶太子西面哭十餘聲止光攝太尉奉策進璽綬【璽斯氏翻綬音受】太子跪受服衮冕之服御太極殿即皇帝位【帝諱詡宣武皇帝之第二子】光等與夜置羣官立庭中北面稽首稱萬歲【倉猝不暇集百官備高氏也稽音啟】昭遵之曾孫也【魏畧陽公遵見一百八卷晉孝武太元二十年】高后欲殺胡貴嬪中給事譙郡劉騰以告侯剛【中給事宧官也北齊之制中侍中省有中侍中中常侍中給事中蓋因魏制】剛以告于忠忠問計於崔光光使置貴嬪於别所嚴加守衛由是貴嬪深德四人【為劉騰等亂政崔光尊寵而不能矯正張本】戊午魏大赦己未悉召西伐東防兵【西伐謂伐蜀之兵東防謂防淮之兵】驃騎大將軍廣平王懷扶疾入臨【驃匹妙翻騎奇寄翻臨力鴆翻】徑至太極西廡哀慟呼侍中領軍黄門二衛【二衛始於晉初左右衛將軍統之此二衛即謂左右衛將軍廡音武】云身欲上殿哭大行又須入見主上衆皆愕然相視無敢對者崔光攘衰振杖【見賢遍翻衰吐雷翻振舉也】引漢光武崩趙熹扶諸王下殿故事【事見四十四卷光武中元二年】聲色甚厲聞者莫不稱善懷聲淚俱止曰侍中以古義裁我我敢不服遂還仍頻遣左右致謝先是高肇擅權尤忌宗室有時望者太子太傅任城王澄數為肇所譖懼不自全【懲彭城王勰之禍也先悉薦翻任音壬數所角翻】乃終日酣飲【酣戶甘翻】所為如狂朝廷機要無所關豫及世宗殂肇擁兵於外【謂高肇方擁伐蜀之兵也】朝野不安于忠與門下議【門下省侍中等官居之朝直遥翻】以肅宗幼未能親政宜使太保高陽王雍入居西柏堂省决庶政【省悉景翻】以任城王澄為尚書令總攝百揆奏皇后請即敕授【請即以手敕授二王倉猝不及下詔慮有沮閣者也】王顯素有寵於世宗恃勢使威為衆所疾恐不為澄等所容與中常侍孫伏連等密謀寢門下之奏矯皇后令以高肇録尚書事以顯與勃海公高猛同為侍中于忠等聞之託以侍療無效執顯於禁中【觀此則侍御師王顯詹事王顯又似一人】下詔削爵任顯臨執呼寃直閤以刀鐶撞其掖下【撞直江翻掖與腋同】送右衛府一宿而死庚申下詔如門下所奏百官總已聽於二王中外悦服二月庚辰尊皇后為皇太后魏主稱名為書告哀於高肇且召之還肇承變憂懼【承告哀之變也】朝夕哭泣至於羸悴【羸倫為翻悴秦醉翻】歸至瀍澗【書我乃卜澗水東瀍水西水經瀍水出河南穀城縣北山東與千金渠合又東過洛陽縣南又東過偃師縣又東入于洛澗水出新安縣南白石山東南入于洛此瀍澗直謂瀍水非如書及水經之瀍澗為二水也瀍直連翻】家人迎之不與相見辛巳至闕下衰服號哭【衰土回翻號戶刀翻】升太極殿盡哀高陽王雍與于忠密謀伏直寢邢豹等十餘人於舍人省下【舍人省即中書省通事舍人宿直之所】肇哭畢引入西廡【廡音武】清河諸王皆竊言目之肇入省豹等搤殺之【搤於革翻】下詔暴其罪惡稱肇自盡自餘親黨悉無所問削除職爵葬以士禮逮昏於厠門出尸歸其家 魏之伐蜀也軍至晉夀蜀人震恐傅豎眼將步兵三萬擊巴北上遣寧州刺史任太洪自隂平間道入其州【傅豎眼以益州刺史鎮晉夀此隂平非鄧艾所由之隂平今利州之隂平縣是也間古莧翻豎而庾翻將即亮翻任音壬】招誘氐蜀絶魏運路【氐蜀氐人及蜀人也誘音酉】會魏大軍北還太洪襲破魏東洛除口二戍【唐利州景谷縣舊白水縣也置東洛郡後周省郡入平興郡隋又廢平興為景谷縣水經漢水過大小黄金南東合蘧蒢口注云蘧蒢水出就谷南歷蘧蒢溪又南流注于漢謂之蒢口】聲言梁兵繼至氐蜀翕然從之太洪進圍關城豎眼遣統軍姜喜等擊太洪大破之太洪棄關城走還【關城即白水關城】 癸未魏以高陽王雍為太傅領太尉清河王懌為司徒廣平王懷為司空 甲午魏葬宣武皇帝于景陵廟號世宗己亥尊胡貴嬪為皇太妃三月甲辰朔以高太后為尼徙居金墉瑶光寺【子無廢母之義魏之亂亡宜矣按魏廢后率居瑶光寺馮后高后是也】非大節慶不得入宫 魏左僕射郭祚表稱蕭衍狂悖謀斷川瀆【謂築浮山堰也悖蒲妹翻斷丁管翻】役苦民勞危亡已兆宜命將出師長驅撲討【將即亮翻撲普木翻】魏詔平南將軍楊大眼督諸軍鎮荆山【水經淮水過塗山北而後至荆山今塗山在鍾離縣西九十五里荆山在鍾離縣西八十三里】 魏于忠既居門下又總宿衛【門下謂為侍中宿衛謂為領軍】遂專朝政權傾一時【朝直遥翻】初太和中軍國多事高祖以用度不足百官之禄四分減一忠悉命歸所減之禄舊制民税絹一匹别輸綿八兩布一匹别輸麻十五斤忠悉罷之乙丑詔文武羣官各進位一級【史言于忠擅魏欲收衆心】 夏四月浮山堰成而復潰或言蛟龍能乘風雨破堰其性惡鐵【復扶又翻惡烏路翻】乃運東西冶鐵器數千萬斤沈之【建康有東西二冶各置冶令以掌之沈持林翻】亦不能合乃伐樹為井幹【井幹井欄也言疊木為井幹之形幹揚子注及西都賦注音寒莊子音如字】填以巨石加土其上緣淮百里内木石無巨細皆盡負檐者肩上皆穿夏日疾疫死者相枕【檐都濫翻枕職任翻】蠅蟲晝夜聲合 魏梁州刺史薛懷吉破叛氐於沮水【水經沔水出武都沮縣東狼谷中又東南流逕沮水戍注云沔水一名沮水闞駰曰以其初出沮洳然故曰沮水師古曰沮千余翻】懷吉真度之子也【薛真度見一百四十卷齊明帝建武二年】五月甲寅南秦州刺史崔暹又破叛氐解武興之圍【魏南秦州治駱谷城領天水漢陽武都武階修武仇池郡此時蓋叛氐圍武興也】 六月魏冀州沙門法慶以妖幻惑衆【妖於驕翻幻戶辦翻】與勃海人李歸伯作亂推法慶為主法慶以尼惠暉為妻以歸伯為十住菩薩平魔軍司定漢王【魏書法慶以殺一人者為一住菩薩殺十人者為十住菩薩菩薄乎翻薩桑葛翻】自號大乘又合狂藥【合音閤】令人服之父子兄弟不復相識【復扶又翻】唯以殺害為事刺史蕭寶寅遣兼長史崔伯驎擊之伯驎敗死【驎力珍翻】賊衆益盛所在毁寺舍斬僧尼燒經像云新佛出世除去衆魔【去羌呂翻】秋七月丁未詔假右光禄大夫元遥征北大將軍以討之【假者未正以征北大將軍授之】 魏尚書裴植自謂人門不後王肅【植裴叔業之兄子也後戶遘翻】以朝廷處之不高【處昌呂翻】意常怏怏【怏於兩翻】表請解官隱嵩山世宗不許深怪之及為尚書志氣驕滿每謂人曰非我須尚書尚書亦須我每入參議論好面譏毁羣官【好呼到翻】又表征南將軍田益宗言華夷異類不應在百世衣冠之上于忠元昭見之切齒【忠昭皆北人故深諱此言】尚書左僕射郭祚冒進不已自以東宫師傅望封侯儀同【天監十年魏以祚領太子少師】詔以祚為都督雍岐華三州諸軍事征西將軍雍州刺史【魏太和十一年置岐州治雍城鎮領平秦武都武功三郡雍於用翻華戶化翻】祚與植皆惡于忠專横密勸高陽王雍使出之忠聞之大怒令有司誣奏其罪尚書奏羊祉告植姑子皇甫仲達云受植旨詐稱被詔【惡烏路翻横戶孟翻被皮義翻】帥合部曲欲圖于忠【帥讀曰率下同】臣等窮治辭不伏引【伏引猶引伏也治直之翻】然衆證明昞【昞音丙】準律當死衆證雖不見植皆言仲達為植所使植召仲達責問而不告列推論情狀不同之理不可分明不得同之常獄有所降減計同仲達處植死刑【處昌呂翻下裁處同】植親帥城衆附從王化【謂齊東昏侯永元二年植以夀陽降魏】依律上議【謂依八議之律也上時掌翻】乞賜裁處【處昌呂翻】忠矯詔曰凶謀既爾罪不當恕雖有歸化之誠無容上議亦不須待秋分【魏書制秋分然後决死刑】八月己亥植與郭祚及都水使者社陵韋儁皆賜死儁祚之昏家也忠又欲殺高陽王雍崔光固執不從乃免雍官以王還第朝野寃憤莫不切齒【使疏吏翻朝直遥翻】 丙子魏尊胡太妃為皇太后居崇訓宫于忠領崇訓衛尉劉騰為崇訓太僕加侍中侯剛為侍中撫軍將軍【三人者胡后所深德者也】又以太后父國珍為光禄大夫 庚辰定州刺史田超秀帥衆三千降魏【超秀叛魏見上卷上年帥讀曰率降戶江翻】 戊子魏大赦 己丑魏清河王懌進位太傅領太尉廣平王懷為太保領司徒任城王澄為司空庚寅魏以車騎大將軍于忠為尚書令特進崔光為車騎大將軍並加開府儀同三司【任音壬騎奇寄翻下同】 魏江陽王繼熙之曾孫也【熙道武之子】先為青州刺史坐以良人為婢奪爵繼子乂娶胡太后妹壬辰詔復繼本封以乂為通直散騎侍郎乂妻為新平郡君仍拜女侍中【為元乂囚靈太后張本散悉亶翻】羣臣奏請太后臨朝稱制九月乙未靈太后始臨朝聽政【胡太后諡曰靈】猶稱令以行事羣臣上書稱殿下太后聰悟頗好讀書屬文射能中針孔【好呼到翻屬之欲翻中竹仲翻】政事皆手筆自决加胡國珍侍中封安定公自郭祚等死詔令生殺皆出于忠王公畏之重足脅息【重直龍翻脅息者屏氣鼻不敢息唯兩脅潛動以舒氣息耳】太后既親政乃解忠侍中領軍崇訓衛尉止為儀同三司尚書令後旬餘太后引門下侍官於崇訓宫【門下侍官自侍中至散騎常侍侍郎崇訓宫蓋太后所居宮也】問曰忠在端右聲望何如咸曰不稱厥任【稱尺證翻】乃出忠為都督冀定瀛三州諸軍事征北大將軍冀州刺史以司空澄領尚書令澄奏安定公宜出入禁中參諮大務詔從之【人之老也戒之在得任城王澄血氣衰矣】 甲寅魏元遥破大乘賊擒法慶并渠帥百餘人【帥所類翻】傳首洛陽左遊擊將軍趙祖悦襲魏西硤石【水經淮水東過夀春縣北又北逕山峽中謂之峽石對岸山上結二城以防津要在淮水西岸者謂之西硤石杜佑曰潁州下蔡縣有硤石山梁築城以拒魏今縣城也】據之以逼夀陽更築外城徙緣淮之民以實城内將軍田道龍等散攻諸戍魏揚州刺史李崇分遣諸將拒之【將即亮翻】癸亥魏遣假鎮南將軍崔亮攻西硤石又遣鎮東將軍蕭寶寅决淮堰 冬十月乙酉魏以胡國珍為中書監儀同三司侍中如故 甲午弘化太守杜桂舉郡降魏【弘化地闕蓋亦緣邊蠻郡也降戶江翻】 初魏于忠用事自言世宗許其優轉太傅雍等皆不敢違加忠車騎大將軍忠又自謂新故之際有定社稷之功諷百僚令加已賞雍等議封忠常山郡公忠又難於獨受乃諷朝廷同在門下者皆加封邑雍等不得已復封崔光為博平縣公而尚書元昭等上訴不已【魏主之立也元昭亦同在門下故上訴不已復扶又翻上時掌翻】太后敕公卿再議太傅懌等上言先帝升遐【記曲禮曰告喪曰天王登假注云登上也遐已也上已者若仙去云耳登即升也假讀與遐同】奉迎乘輿侍衛省闥乃臣子常職【乘繩證翻】不容以此為功臣等前議授忠茅土正以畏其威權苟免暴戾故也若以功過相除悉不應賞請皆追奪崔光亦奉送章綬茅土表十餘上太后從之高陽王雍上表自劾稱臣初入柏堂見詔旨之行一由門下臣出君行深知其不可而不能禁【謂殺生予奪皆出於于忠之意而以詔旨行之上時掌翻劾戶槩翻又戶得翻】于忠專權生殺自恣而臣不能違忠規欲殺臣賴在事執拒【在事謂在位任事之臣此指言崔光執拒謂執不可而拒忠】臣欲出忠於外在心未行反為忠廢忝官尸禄孤負恩私請返私門伏聽司敗【司敗即司寇也】太后以忠有保護之功不問其罪十二月辛丑以忠為太師領司州牧尋復録尚書事與太傅懌太保懷侍中胡國珍入居門下同釐庶政【復扶又翻釐力之翻治也】 己酉魏崔亮至硤石趙祖悦逆戰而敗閉城自守亮進圍之 丁卯魏主及太后謁景陵 是冬寒甚淮泗盡凍浮山堰士卒死者什七八 魏益州刺史傅豎眼性清素民獠懷之龍驤將軍元法僧代豎眼為益州刺史【此魏之東益州也豎而庾翻獠魯皓翻驤思將翻】素無治幹加以貪殘【治直吏翻】王賈諸姓本州士族法僧皆召為兵葭萌民任令宗因衆心之患魏也殺魏晉夀太守以城來降民獠多應之【葭音家任音壬守式又翻降戶江翻獠魯皓翻】益州刺史鄱陽王恢遣巴西梓潼二郡太守張齊將兵三萬迎之【將即亮翻】法僧熙之曾孫也 魏岐州刺史趙王謐幹之子也【趙郡王幹見一百四十卷齊明帝建武二年此於王謐之上逸郡字】為政暴虐一旦閉城門大索執人而掠之【索山客翻掠音亮】楚毒備至又無故斬六人闔城兇懼衆遂大呼屯門【兇許拱翻呼火故翻】謐登樓毁梯以自固胡太后遣遊擊將軍王靖馳驛諭城人城人開門謝罪奉送管籥乃罷謐刺史謐妃太后從女也至洛除大司農卿【史言魏母后擅朝城狐社鼠有所依憑驛音日從才用翻】太后以魏主尚幼未能親祭欲代行祭事禮官博議以為不可太后以問侍中崔光光引漢和熹鄧太后祭宗廟故事太后大悦遂攝行祭事【史言崔光逢女主之惡】 魏南荆州刺史桓叔興表請不隸東荆州許之【南荆隸東荆見上卷十一年】<br />
<br />
  十五年春正月戊辰朔魏大赦改元熙平 魏崔亮攻硤石未下與李崇約水陸俱進崇屢違期不至【李崇時鎮夀陽】胡太后以諸將不壹乃以吏部尚書李平為使持節鎮軍大將軍兼尚書右僕射將步騎二千赴夀陽别為行臺節度諸軍【將即亮翻使疏吏翻騎奇寄翻】如有乖異以軍法從事蕭寶寅遣輕車將軍劉智文等渡淮攻破三壘二月乙巳又敗將軍垣孟孫等於淮北【敗補邁翻】李平至硤石督李崇崔亮等水陸進攻無敢乖互【乖異也互差也】戰屢有功上使左衛將軍昌義之將兵救浮山未至康絢已擊魏兵却之【絢許縣翻 考異曰絢傳十二月魏遣李曇定督衆軍來戰按魏帝紀此年正月乃遣李平節度諸軍絢傳誤也曇定即平字也】上使義之與直閤王神念泝淮救硤石【泝蘇故翻考異曰李崇傳衍遣趙祖悦襲據西硤石又遣義之神念帥水軍泝淮而上規取夀春按義之傳絢破魏軍】<br />
<br />
  【義之乃救硤石今從之】崔亮遣將軍博陵崔延伯守下蔡【下蔡縣漢屬沛郡梁置下蔡郡屬豫州水經注淮水自硤石北逕下蔡故城東本州來之城也吳季札始邑延陵後邑州來故曰延州來季子春秋襄二年蔡成公自新蔡遷于州來謂之下蔡淮之東岸又有一城下蔡新城也二城對據翼帶淮濆按蔡有三蔡州上蔡縣蔡仲始封之邑也又有新蔡蔡平侯自上蔡遷都于此又有下蔡縣蔡成公所遷也】延伯與别將伊甕生夾淮為營延伯取車輪去輞削鋭其輻兩兩接對揉竹為絙【將即亮翻去羌呂翻輞扶紡翻車之牙車輮也輻方目翻輪轑也揉人九翻絙古恒翻大索也】貫連相屬並十餘道横水為橋兩頭施大鹿盧【鹿盧圓轉之木屬之欲翻】出沒随意不可燒斫既斷趙祖悦走路又令戰艦不通義之神念屯梁城不得進【嗚呼吾國之失襄陽亦以水陸援斷而諸將不進也斷丁管翻艦戶黯翻】李平部分水陸攻硤石【分扶問翻】克其外城乙丑祖悦出降斬之盡俘其衆【降戶江翻】胡太后賜崔亮書使乘勝深入平部分諸將水陸並進攻浮山堰亮違平節度以疾請還随表輒發平奏處亮死刑【處昌呂翻】太后令曰亮去留自擅違我經畧雖有小捷豈免大咎但吾攝御萬機庶幾惡殺【幾居依翻惡烏故翻亮崔光之族弟也故平奏不行】可聽特以功補過魏師遂還【還從宣翻又如字】 魏中尉元匡奏彈于忠幸國大災專擅朝命裴郭受寃【謂裴植郭祚彈徒丹翻朝直遥翻】宰輔黜辱【謂高陽王雍被黜後又以授忠茅土乞自貶退也】又自矯旨為儀同三司尚書令領崇訓衛尉原其此意欲以無上自處【處昌呂翻下尉處同】既事在恩後【恩後謂赦思之後也】宜加顯戮請遣御史一人就州行决【于忠時為冀州刺史欲就州戮之】自去歲世宗晏駕以後皇太后未親覽以前諸不由階級或門下詔書或由中書宣敕擅相拜授者已經恩宥止可免罪並宜追奪太后令曰忠已蒙特原無宜追罪餘如奏匡又彈侍中侯剛掠殺羽林【掠音亮下同】剛本以善烹調為尚食典御【嘗食典御魏官也掌調和御食温凉寒熱以時供進則嘗之或曰嘗當作尚平去二字通用】凡三十年以有德於太后【事見上年】頗專恣用事王公皆畏附之廷尉處剛大辟【辟毗亦翻】太后曰剛因公事掠人邂逅致死於律不坐少卿陳郡袁飜曰邂逅謂情狀已露隱避不引【不引謂不引伏也邂戶介翻逅戶遘翻】考訊以理者也今此羽林問則具首【首式又翻】剛口唱打殺撾築非理安得謂之邂逅【撾則瓜翻】太后乃削剛戶三百解尚食典御 三月戊戌朔日有食之 魏論西硤石之功辛未以李崇為驃騎將軍加儀同三司【驃匹妙翻騎奇寄翻】李平為尚書右僕射崔亮進號鎮北將軍亮與平争功於禁中太后以亮為殿中尚書 魏蕭寶寅在淮堰上為手書誘之使襲彭城許送其國廟及室家諸從還北【諸從猶羣從也從才用翻】寶寅表上其書於魏朝【上時掌翻朝直遥翻】 夏四月淮堰成長九里下廣一百四十丈上廣四十五丈高二十丈樹以杞柳【長直亮翻廣古曠翻高居號翻杞柳柜柳也】軍壘列居其上或謂康絢曰四凟天所以節宣其氣不可久塞【國語周太子晉曰古之長民者不墮山不崇藪不防川不竇澤夫山土之聚也藪物之歸也川氣之導也澤水之鍾也天地成而聚於高歸物於下疏為川谷以導其氣陂塘汚庳以鍾其美塞悉則翻】若鑿湬東注則游波寛緩堰得不壞絢乃開湬東注又縱反間於魏曰梁人所懼開湬不畏野戰蕭寶寅信之鑿山深五丈開湬北注水日夜分流猶不減【丁度集韻湬與湫同將由翻間古莧翻深式浸翻】魏軍竟罷歸水之所及夾淮方數百里李崇作浮橋於硤石戍間又築魏昌城於八公山東南以備夀陽城壞居民散就岡隴【山脊為岡高丘為隴】其水清徹俯視廬舍冢墓了然在下初堰起於徐州境内【浮山在鍾離郡界梁置徐州於鍾離】刺史張豹子宣言謂已必掌其事既而康絢以他官來監作【監古銜翻】豹子甚慙俄而敕豹子受絢節度豹子遂譖絢與魏交通上雖不納猶以事畢徵絢還【絢還則堰壞矣】 魏胡太后追思于忠之功曰豈宜以一謬棄其餘勲【胡后以于忠擁護為功若忠之專横其謬固非一也】復封忠為靈夀縣公【復扶又翻】亦封崔光為平恩縣侯 魏元法僧遣其子景隆將兵拒張齊【將即亮翻下同】齊與戰於葭萌大破之屠十餘城遂圍武興法僧嬰城自守境内皆叛法僧遣使間道告急於魏【使疏吏翻間古莧翻】魏驛召鎮南軍司傅豎眼於淮南以為益州刺史西征都督將兵騎三千以赴之【豎而庾翻騎奇寄翻】豎眼入境轉戰三日行二百餘里九遇皆捷五月豎眼擊殺梁州刺史任太洪民獠聞豎眼至皆喜迎拜於路者相繼【民獠惡法僧而懷豎眼故迎之者屬路任音壬獠魯皓翻】張齊退保白水豎眼入州【入武興也】白水以東民皆安業魏梓潼太守苟金龍領關城戍主梁兵至金龍疾病不堪部分【分扶問翻】其妻劉氏帥厲城民乘城拒戰【帥讀曰率】百有餘日士卒死傷過半戍副高景謀叛劉氏斬景及其黨與數千人【數千當作數十】自餘將士分衣減食勞逸必同莫不畏而懷之井在城外為梁兵所據會天大雨劉氏命出公私布絹及衣服懸之絞而取水城中所有雜物悉儲之【雜物謂瓶罌甕盎之屬】豎眼至梁兵乃退魏人封其子為平昌縣子 六月庚子以尚書令王瑩為左光禄大夫開府儀同三司尚書右僕射袁昂為左僕射吏部尚書王暕為右僕射暕儉之子也【暕古限翻王儉齊初佐命】 張齊數出白水侵魏葭萌傳豎眼遣虎威將軍強蚪攻信義將軍楊興起殺之復取白水【數所角翻強其兩翻又如字復扶又翻下州復不復同】寧朔將軍王光昭又敗於隂平張齊親帥驍勇二萬餘人與傳豎眼戰【驍堅堯翻】秋七月齊軍大敗走還小劍大劍諸戍皆棄城走【今劍州劍門縣有大劍山又有小劍山在其西北三十里又有小劍故城在益昌縣西南五十里大劍雖號天險有阨塞可守崇墉之間徑路頗夷小劍則鑿石架閣有不容越李白所謂一夫當關萬夫莫開者是也】東益州復入于魏 八月乙巳魏以胡國珍為驃騎大將軍開府儀同三司雍州刺史【驃匹妙翻騎奇寄翻雍於用翻】國珍年老太后實不欲令出止欲示以方面之榮竟不行 康絢既還張豹子不復修淮堰九月丁丑淮水暴漲堰壞其聲如雷聞三百里【聞音問】緣淮城戍村落十餘萬口皆漂入海初魏人患淮堰以任城王澄為大將軍大都督南討諸軍事勒衆十萬將出徐州來攻堰尚書右僕射李平以為不假兵力終當自壞及聞破太后大喜賞平甚厚澄遂不行 壬辰大赦 魏胡太后數幸宗戚勲貴之家【數所角翻】侍中崔光表諫曰禮諸侯非問疾弔喪而入諸臣之家謂之君臣為謔【記禮運之辭也注云無故而相之是戲謔也陳靈公與孔寧儀行父數如夏氏以取弑焉】不言王后夫人明無適臣家之義夫人父母在有歸寧沒則使卿寧【左傳莊二十七年冬杞伯姬來歸寧也杜預注曰寧問父母安否襄十二年楚司馬子庚聘于秦為夫人寧禮也注曰諸侯大夫父母既沒歸寧使卿故曰禮】漢上官皇后將廢昌邑霍光外祖也親為宰輔后猶御武帳以接羣臣【事見二十四卷漢昭帝元平元年】示男女之别也【别彼列翻】今帝族方衍勲貴增遷祇請遂多將成彞式【方衍謂生子也增遷謂增秩遷官也祇敬也謂宗戚勲貴之家凡有吉慶皆請太后臨幸】願陛下簡息遊幸則率土屬賴含生仰悦矣【屬之欲翻下請屬同】任城王澄以北邊鎮將選舉彌輕【將即亮翻】恐賊虜闚邉山陵危迫【魏自顯祖以上山陵皆在雲中】奏求重鎮將之選修警備之嚴詔公卿議之廷尉少卿袁翻議【秦漢以來九卿各一卿魏太和十五年九卿各置少卿蓋倣周官六卿有小宰小司徒小宗伯小司馬小司寇小司空之遺制也】以為比緣邊州郡官不擇人唯論資級或值貪汚之人廣開戍邏多置帥領或用其左右姻親或受人貨財請屬皆無防寇之心唯有聚斂之意【比毗至翻邏郎佐翻帥所類翻屬之欲翻斂力贍翻】其勇力之兵驅令抄掠若遇彊敵即為奴虜如有執獲奪為己富其羸弱老小之輩微解金鐵之工少閑草木之作【抄楚交翻羸倫為翻解戶買翻曉也閑習也】無不搜營窮壘苦役百端自餘或伐木深山或芸草平陸販貿往還相望道路此等禄既不多貲亦有限皆收其實絹給其虚粟窮其力薄其衣用其功節其食綿冬歷夏加之疾苦死於溝瀆者什常七八【自古至今守邊之兵皆病於此貿音茂】是以鄰敵伺間擾我疆場皆由邊任不得其人故也愚謂自今已後南北邊諸藩及所統郡縣府佐統軍至于戍主皆令朝臣王公己下各舉所知必選其才不拘階級若稱職及敗官并所舉之人随事賞罰太后不能用【間古莧翻場音亦朝直遥翻稱尺證翻敗補邁翻】及正光之末北邊盜賊羣起遂逼舊都犯山陵如澄所慮【正光四年破六韓拔陵衛可孤等反孝昌初年雲中沒矣】 冬十一月交州刺史李畟斬交州反者阮宗孝傳首建康【畟察色翻】 初魏世宗作瑶光寺未就是歲胡太后又作永寧寺【水經注穀渠南流出太尉司徒兩坊間水西有永寧寺】皆在宫側又作石窟寺於伊闕口皆極土木之美而永寧尤盛有金像高丈八者一如中人者十玉像二為九層浮圖掘地築基下及黄泉【杜預曰地中之泉故曰黄泉】浮圖高九十丈上刹復高十丈【高居報翻刹所轄翻刹柱也浮圖上柱今謂之相輪復扶又翻】每夜静鈴鐸聲聞十里【聞音問】佛殿如太極殿南門如端門僧房千間珠玉錦繡駭人心目自佛法入中國塔廟之盛未之有也【漢永明中佛法入中國佛弟子收奉舍利建宫宇號為塔亦胡言猶宗廟也故世稱塔廟】揚州刺史李崇上表以為高祖遷都垂三十年【遷都見一百三十八卷齊世宗永明十一年】明堂未修太學荒廢城闕府寺頗亦頹壞非所以追隆堂構【書大誥曰若考作室既底法厥子乃弗肯堂矧肯構】儀刑萬國者也今國子雖有學官之名而無教授之實何異兔絲燕麥南箕北斗【爾雅曰唐蒙女蘿兔絲釋名曰唐也蒙也女蘿也兔絲也一物四名毛氏詩傳曰女蘿兔絲兔絲松蘿也陸機草木疏曰兔絲蔓連草上生黄赤如金合藥兔絲子是也松蘿自蔓松上生枝正青與兔絲殊異本草曰免絲生川澤田野蔓延草木之上瞿麥一名燕麥又名雀麥其苗與麥同但穗細長而疎言兔絲有絲之名而不可以織燕麥有麥之名而不可以食古歌曰田中兔絲如何可絡道邊燕麥何嘗可穫詩云維南有箕不可以簸揚維北有斗不可以挹酒漿皆謂有名無實也】事不兩興須有進退宜罷尚方雕靡之作省永寧土木之功減瑶光材瓦之力分石窟䥴琢之勞及諸事役非急者於三事農隙修此數條使國容嚴顯禮化興行不亦休哉太后優令答之<br />
<br />
  而不用其言【闕】          太后好事佛民多絶戶為沙門【家有一子出為沙門其戶絶矣好呼到翻】高陽王友李瑒上言三千之罪莫大於不孝不孝之大無過於絶祀【孔子曰五行之屬三千其罪莫大於不孝孟子曰不孝有三無後為大瑒杖梗翻又音暢】豈得輕縱背禮之情肆其向法之意一身親老棄家絶養缺當世之禮而求將來之益【佛法以今世修種為來生因果背蒲妹翻養余亮翻】孔子云未知生焉知死【論語載孔子答子路之言焉於䖍翻】安有棄堂堂之政而從鬼教乎又今南服未静衆役仍煩百姓之情實多避役若復聽之恐捐棄孝慈比屋皆為沙門矣【復扶又翻比毗必翻又毗至翻】都統僧暹等忿瑒謂之鬼教以為謗佛【魏有沙門統謂之都統猶今都僧録】泣訴於太后太后責之瑒曰天曰神地曰祗人曰鬼傳曰明則有禮樂幽則有鬼神然則明者為堂堂幽者為鬼教佛本出於人名之為鬼愚謂非謗太后雖知瑒言為允難違暹等之意罰瑒金一兩 魏征南大將軍田益宗求為東豫州刺史以招二子太后不許竟卒於洛陽【田益宗二子叛見上卷十三年卒子恤翻】 柔然伏跋可汗壯健善用兵【可從刋入聲汗音寒】是歲西擊高車大破之執其王彌俄突繫其足於駑馬頓曳殺之漆其頭為飲器【彌俄突殺柔然佗汗見上卷七年】鄰國先羈屬柔然後叛去者伏跋皆擊滅之其國復彊【復扶又翻下賊復同】<br />
<br />
  十六年春正月辛未上祀南郊 魏大乘餘賊復相聚【沙門法慶之餘黨也】突入瀛州刺史宇文福之子員外散騎侍郎延帥奴客拒之【散悉亶翻騎奇寄翻帥讀曰率】賊燒齋閤延突火抱福出外肌髮皆焦勒衆苦戰賊遂散走追討平之 甲戌魏大赦 魏初民間皆不用錢高祖太和十九年始鑄太和五銖錢遣錢工在所鼓鑄民有欲鑄錢者聽就官鑪銅必精練無得殽雜世宗永平三年又鑄五銖錢禁天下用錢不依準式者既而洛陽及諸州鎮所用錢各不同商賈不通尚書令任城王澄上言以為不行之錢律有明式指謂雞眼鐶鑿【雞眼者謂錢薄小其眼如雞眼也鐶鑿云者謂鑿好以取銅僅存其肉也鐶戶關翻】更無餘禁計河南諸州今所行悉非制限昔來繩禁愚竊惑焉又河北既無新錢復禁舊者專以單絲之縑疎縷之布狹幅促度不中常式【復扶又翻中竹仲翻】裂匹為尺以濟有無徒成杼軸之勞【杼直呂翻說文機之持緯者】不免饑寒之苦殆非所以救恤凍餒子育黎元之意也錢之為用貫繦相屬【繦居兩翻亦錢貫也屬之欲翻】不假度量平均簡易濟世之宜謂為深允乞並下諸方州鎮【易以豉翻下遐稼翻】其太和與新鑄五銖及古諸錢方俗所便用者但内外全好雖有大小之異並得通行貴賤之差自依鄉價庶貨環海内公私無壅其雞眼鐶鑿及盗鑄毁大為小生新巧偽不如法者據律罪之詔從之然河北少錢【少詩沼翻】民猶用物交易錢不入市 魏人多竊冒軍功尚書左丞盧同閲吏部勲書因加檢覈【覈戶革翻】得竊階者三百餘人乃奏乞集吏部中兵二局勲簿對句奏案【句古侯翻考也稽也】更造兩通【更工衡翻】一關吏部一留兵局又在軍斬首成一階以上者即令行臺軍司給券當中豎裂一支付勲人一支送門下【此韓愈寄崔立之詩所謂當如合分支者也今人亦謂析產文契為分支帳豎而庾翻】以防偽巧太后從之同玄之族孫也【盧玄見一百二十二卷宋文帝元嘉八年】中尉元匡奏取景明元年已來内外考簿吏部除書中兵勲案并諸殿最【殿丁練翻】欲以案校竊階盜官之人太后許之尚書令任城王澄表以為法忌煩苛治貴清約【治直吏翻】御史之體風聞是司若聞有冒勲妄階止應攝其一簿研檢虚實繩以典刑豈有移一省之案【取尚書省之案赴御史臺所謂移也】尋兩紀之事【自景明元年至是年凡十八年今言兩紀之事蓋景明初所叙階勲皆太和末准漢用兵所上勲人名籍也】如此求過誰堪其罪斯實聖朝所宜重慎也太后乃止又以匡所言數不從慮其辭解【朝直遥翻數所角翻辭解者辭職解官也】欲奨安之乃加鎮東將軍二月丁未立匡為東平王【為匡治棺攻澄張本】 三月丙子敕織官文錦不得為仙人鳥獸之形【織官猶漢之織室令丞也】為其裁翦有乖仁恕【為于偽翻】丁亥魏廣平文穆王懷卒 夏四月戊申魏以中書<br />
<br />
  監胡國珍為司徒 詔以宗廟用牲有累冥道【冥幽也幽則有鬼神冥道鬼神之道也累力瑞翻】宜皆以麪為之 【考異曰梁帝紀此詔在四月甲子南史云在二月云祈告天地宗廟以去殺之理欲被之含識郊廟牲牷皆代以麪其山川諸祀則否按長歷是月辛卯朔無甲子隋志但云四月亦不云郊祀去牲今從之】於是朝野諠譁以為宗廟去牲乃是不復血食【去羌呂翻復扶又翻】帝竟不從八坐乃議以大脯代一元大武【記曲禮牛曰一元大武鄭玄曰元頭也武迹也坐徂卧翻】 秋八月丁未詔魏太師高陽王雍入居門下參决尚書奏事【魏字當在詔字之上】 冬十月詔以宗廟猶用脯脩【鄭玄曰脯乾肉也脩鍜脩也薄析曰脯棰之而施薑桂曰鍜脩】更議代之於是以大餅代大脯其餘盡用蔬果又起至敬殿景陽臺置七廟座每月中再設浄饌【饌雛戀翻又雛朊翻】 乙卯魏詔北京士民未遷者悉聽留居為永業【魏以代都為北京】 十一月甲子巴州刺史牟漢寵叛降魏【五代志巴西郡梁置南梁州北巴州降戶江翻】 十二月柔然伏跋可汗遣俟斤尉比建等請和於魏【俟斤柔然大臣之號俟渠希翻尉紆勿翻】用敵國之禮 是歲以右衛將軍馮道根為豫州刺史道根謹厚木訥行軍能檢敕士卒諸將争功道根獨默然為政清簡吏民懷之上嘗歎曰道根所在令朝廷不復憶有一州【復扶又翻】 魏尚書崔亮奏請於王屋等山採銅鑄錢從之【五代志河内郡王屋縣有王屋山】是後民多私鑄錢稍薄小用之益輕【是時錢輕南北皆然豈天時邪】<br />
<br />
  十七年春正月甲子魏以氐酋楊定為隂平王【酋慈由翻】魏秦州羌反 二月癸巳安成康王秀卒【卒子恤翻】秀雖與上布衣昆弟及為君臣小心畏敬過於疎賤上益以此賢之秀與弟始興王憺尤相友愛憺久為荆州常中分其禄以給秀【秀憺皆吳太妃之子齊和帝中興元年憺督雍州天監元年進督荆州五年徵至郡荆州總西夏之寄俸入優厚憺徒敢翻又徒覽翻】秀稱心受之【稱尺證翻】亦不辭多也 甲辰大赦 己酉魏大赦改元神龜 魏東益州氐反 魏主引見柔然使者【見賢遍翻使疏吏翻】讓之以藩禮不備議依漢待匈奴故事遣使報之【漢宣帝待呼韓邪位在諸侯王上蓋稱臣也按張倫表諫與為昆弟蓋用漢文景故事】司農少卿張倫上表以為太祖經啟帝圖日有不暇遂令豎子遊魂一方【謂道武南略社崘得以雄跨漠北】亦由中國多虞急諸華而緩夷狄也高祖亦事南轅未遑北伐【謂孝文南都洛陽用兵淮漢未暇伐柔然也】世宗遵述遺志虜使之來受而弗答【見百四十六卷六年】以為大明臨御國富兵彊抗敵之禮何憚而為之何求而行之今虜雖慕德而來亦欲觀我彊弱若使王人銜命虜庭與為昆弟恐非祖宗之意也苟事不獲已應為制詔示以上下之儀命宰臣致書論以歸順之道觀其從違徐以恩威進退之則王者之體正矣豈可以戎狄兼并【謂伏跋新破高車及滅鄰國之叛者也】而遽虧典禮乎不從倫白澤之子也【張白澤見一百三十四卷宋順帝昇明元年】 三月辛未魏靈夀武敬公于忠卒 魏南秦州氐反遣龍驤將軍崔襲持節諭之【驤思將翻】 夏四月丁酉魏秦文宣公胡國珍卒贈假黄鉞相國都督中外諸軍事太師號曰太上秦公葬以殊禮贈禭儀衛事極優厚又迎太后母皇甫氏之柩與國珍合葬謂之太上秦孝穆君【柩音舊】諫議大夫常山張普惠以為前世后父無稱太上者太上之名不可施於人臣詣闕上疏陳之左右莫敢為通【為于偽翻】會胡氏穿壙下有磐石乃密表以為天無二日土無二王太上者因上而生名也皇太后稱令以繫敕下蓋取三從之道遠同文母列於十亂【武王曰予有亂臣十人孔子曰才難有婦人焉九人而已先儒以為十亂太公望周公旦召公奭畢公高榮公太顛閎天散宜生南宫适及文母】今司徒為太上恐乖繫敕之意孔子稱必也正名乎【論語載孔子答子路之言】比克吉定兆【比毗至翻】而以淺改卜亦或天地神靈所以垂至戒啟聖情也伏願停逼上之號以邀謙光之福太后乃親至國珍宅召集五品以上博議王公皆希太后意爭詰難普惠【詰去吉翻難乃旦翻】普惠應機辯析無能屈者太后使元乂宣令於普惠曰朕之所行孝子之志卿之所陳忠臣之道羣公已有成議卿不得苦奪朕懷後有所見勿難言也太后為太上君造寺壯麗埒於永寧尚書奏復徵民綿麻之税【為于偽翻埒力輟翻復扶又翻下浸復復欲無復欲復復見同】張普惠上疏以為高祖廢大斗去長尺改重稱【去羌呂翻下去天下同稱尺證翻下稱尺同事見一百四十卷齊明帝建武二年】以愛民薄賦知軍國須綿麻之用故於絹增税綿八兩於布增税麻十五斤民以稱尺所減不啻綿麻故鼓舞供調【調徒釣翻】自兹以降所税絹布浸復長濶百姓嗟怨聞於朝野【聞音問朝直遥翻】宰輔不尋其本在於幅廣度長遽罷綿麻【于忠罷綿麻見上十四年】既而尚書以國用不足復欲徵斂去天下之大信棄已行之成詔【斂力贍翻去羌呂翻】追前之非遂後之失不思庫中大有綿麻而羣臣共竊之也何則所輸之物或斤羨百銖【羨延面翻餘也】未聞有司依律以罪州郡或小有濫惡則坐戶主連及三長【戶主者一家之長則為一戶之主三長見一百三十六卷齊世祖永明四年】是以在庫絹布踰制者多羣臣受俸人求長濶厚重無復準極未聞以端幅有餘還求輸官者也【布帛六丈為端爾雅倍丈謂之端倍端謂之兩倍兩謂之匹杜預曰二丈為端二端為兩所謂匹也說文幅布帛廣也俸扶用翻】今欲復調綿麻當先正稱尺明立嚴禁無得放溢使天下知二聖之心愛民惜法如此則太和之政復見於神龜矣普惠又以魏主好遊騁苑囿【好呼到翻】不親視朝【朝直遥翻】過崇佛法郊廟之事多委有司上疏切諫以為殖不思之冥業損巨費於生民減禄削力近供無事之僧崇飾雲殿遠邀未然之報昧爽之臣稽首於外【謂羣臣入朝者也孔安國曰昧冥爽明早旦稽音啟】玄寂之衆遨遊於内愆禮忤時【忤五故翻】人靈未穆愚謂修朝夕之因求祗劫之果【祗巨支翻釋氏之言祗劫猶云無數劫也】未若收萬國之懽心以事其親使天下和平災害不生也【用孝經文意】伏願淑慎威儀【淑善也】為萬邦作式躬致郊廟之䖍親紆朔朢之禮【紆縈也屈也】釋奠成均【五帝之學曰成均鄭玄曰釋菜奠幣禮先師也又曰釋奠者設薦饌酌奠而已孔穎達曰釋奠有牲牢有幣帛釋菜則惟釋蘋藻而已】竭心千畝【千畝謂藉田也】量撤僧寺不急之華還復百官久折之秩【量音良折而設翻】已造者務令簡約速成未造者一切不復更為則孝弟可以通神明【弟謂曰悌】德教可以光四海節用愛人法俗俱賴矣尋敕外議釋奠之禮又自是每月一陛見羣臣皆用普惠之言也普惠復表論時政得失太后與帝引普惠於宣光殿随事詰難【復扶又翻見賢遍翻難乃旦翻】 臨川王宏妾弟吳法夀殺人而匿於宏府中上敕宏出之即日伏辜有司奏免宏官【御史臺曰南臺亦曰南司】上注曰愛宏者兄弟私親免宏者王者正法所奏可五月戊寅司徒驃騎大將軍揚州刺史臨川王宏免【驃匹妙翻騎奇寄翻】宏自洛口之敗【事見一百四十六卷五年】常懷愧憤都下每有竊輒以宏為名屢為有司所奏上每赦之上幸光宅寺【帝以三橋舊宅為光宅寺三橋在秣陵縣同夏里】有盜伏於驃騎航【宏府面秦淮於府前為浮橋謂之驃騎航以宏官名航也】待上夜出上將行心動乃於朱雀航過事稱為宏所使上泣謂宏曰我人才勝汝百倍當此猶恐不堪汝何為者我非不能為漢文帝【謂誅淮南厲王長也】念汝愚耳宏頓首稱無之故因匿法夀免宏官宏奢僭過度殖貨無厭【厭於鹽翻】庫屋垂百間【垂幾及也】在内堂之後關籥甚嚴【籥與鑰同關牡也】有疑是鎧仗者密以聞【鎧可亥翻】上於友愛甚厚殊不悦它日送盛饌與宏愛妾江氏曰當來就汝懽宴【饌雛戀翻又雛晥翻】獨攜故人射聲校尉丘佗卿往與宏及江大飲半醉後謂曰我今欲履行汝後房【行下孟翻】即呼輿徑往堂後宏恐上見其貨賄顔色怖懼【賄呼罪翻怖普布翻】上意益疑之於是屋屋檢視每錢百萬為一聚黄榜標之千萬為一庫懸一紫標如此三十餘間上與佗卿屈指計見錢三億餘萬【見賢遍翻】餘屋貯布絹絲綿漆蜜紵蠟等雜貨但見滿庫不知多少【貯丁呂翻紵直呂翻紵麻屬而細於麻少詩沼翻】上始知非仗大悦謂曰阿六汝生計大可【宏於諸弟次第六阿從安入聲】乃更劇飲至夜舉燭而還【還從宣翻又如字】兄弟方更敦睦宏都下有數十邸出懸錢立券每以田宅邸店懸上文契【上時掌翻】期訖便驅券主奪其宅都下東土百姓失業非一上後知之制懸券不得復驅奪自此始【復扶又翻】侍中領軍將軍吳平侯昺雅有風力為上所重軍國大事皆與議决以為安右將軍監揚州【安右將軍帝所置百號將軍之一也昺音丙監王銜翻】昺自以越親居揚州【昺帝從父弟揚州京邑昺自以為越同氣之親而居之故懇讓】涕泣懇讓上不許在州尤稱明斷【斷丁亂翻】符教嚴整辛巳以宏為中軍將軍中書監六月乙酉又以本號行司徒【本號中軍將軍號也】<br />
<br />
  臣光曰宏為將則覆三軍【將即亮翻】為臣則涉大逆高祖貸其死罪可矣數旬之間還為三公於兄弟之恩誠厚矣王者之法果安在哉<br />
<br />
  初洛陽有漢所立三字石經【石經事見四十七卷漢靈帝熹平四年酈道元曰蔡邕正定五經文字刻石立於太學門外魏正始中又立古篆隸三字石經】雖屢經喪亂而初無損失【喪息浪翻】及魏馮熙常伯夫相繼為洛州刺史【魏都平城以洛陽為洛州既遷洛始改為司州】毁取以建浮圖精舍遂大致頹落所存者委於榛莽道俗随意取之侍中領國子祭酒崔光請遣官守視命國子博士李郁等補其殘缺胡太后許之會元乂劉騰作亂事遂寢【乂騰作亂事見下卷普通二年】 秋七月魏河州羌卻鐵忽反自稱水池王【河州治枹罕領金城武始洪和臨洮郡水池縣魏真君四年置郡後改為縣屬洪和郡隋併洪和郡為當夷縣其地在唐洮州臨潭縣界】詔以主客郎源子恭為行臺以討之【曹魏置尚書主客郎】子恭至河州嚴勒州郡及諸軍毋得犯民一物亦不得輕與賊戰然後示以威恩使知悔懼八月鐵忽等相帥詣子恭降首尾不及二旬【言自子恭至河州及于賊降首尾不及二旬也帥讀曰率降戶江翻】子恭懷之子也【源懷源賀之子歷事文成獻文孝文宣武】 魏宦者劉騰手不解書【解戶買翻】而多姦謀善揣人意【揣初委翻】胡太后以其保護之功【事見上十四年】累遷至侍中右光禄大夫遂干預政事納賂為人求官無不效者【為于偽翻】河間王琛簡之子也【齊郡王簡見一百三十七卷齊武帝永明九年琛丑林翻】為定州刺史以貪縱著名及罷州還太后詔曰琛在定州唯不將中山宫來【後燕都中山建宫室魏克中山因以為中山宫】自餘無所不致何可更復敘用【復扶又翻】遂廢于家琛乃求為騰養息【養息即養子也】賂騰金寶鉅萬計騰為之言於太后【為于偽翻】得兼都官尚書出為秦州刺史會騰疾篤太后欲及其生而貴之九月癸未朔以騰為衛將軍加儀同三司 魏胡太后以天文有變欲以崇憲高太后當之戊申夜高太后暴卒冬十月丁卯以尼禮葬於北邙【尼女夷翻下同】諡曰順皇后百官單衣邪巾【古者二十成人士冠庶人巾邪巾者邪厭於首捨衰絰喪冠而單衣邪巾示不成喪也】送至墓所事訖而除 乙亥以臨川王宏為司徒 魏胡太后遣使者宋雲與比丘惠生如西域求佛經【比丘僧也比毗至翻下連比同】司空任城王澄奏昔高祖遷都制城内唯聽置僧尼寺各一餘皆置於城外蓋以道俗殊歸欲其浄居塵外故也正始三年沙門統惠深始違前禁自是卷詔不行【卷讀曰捲】私謁彌衆都城之中寺踰五百占奪民居三分且一屠沽塵穢連比雜居【占之贍翻比毗至翻】往者代北有法秀之謀【事見一百三十五卷齊太祖建元三年】冀州有大乘之變太和景明之制非徒使緇素殊途蓋亦以防微杜漸昔如來闡教多依山林【如來佛也】今此僧徒戀著城邑【著直略翻】正以誘於利欲不能自已【誘音酉】此乃釋氏之糟糠法王之社鼠【法王謂佛】内戒所不容【釋氏有五戒】國典所共棄也臣謂都城内寺未成可從者宜悉徙於郭外僧不滿五十者併小從大外州亦準此然卒不能行【卒子恤翻】 是歲魏太師雍等奏鹽池天藏【藏徂浪翻】資育羣生先朝為之禁限【朝直遥翻】亦非苟與細民爭利但利起天池取用無法或豪貴封護或近民吝守貧弱遠來邈然絶望因置主司令其裁察彊弱相兼務令得所什一之税自古有之所務者遠近齊平公私兩宜耳及甄琛啟求禁集【事見一百四十六卷五年甄之人翻琛丑林翻】乃為繞池之民尉保光等擅自固護語其障禁倍於官司取與自由貴賤任口【言鹽價賤貴任其口之所出也】請依先朝禁之為便詔從之<br />
<br />
  資治通鑑卷一百四十八  <br>
   </div> 

<script src="/search/ajaxskft.js"> </script>
 <div class="clear"></div>
<br>
<br>
 <!-- a.d-->

 <!--
<div class="info_share">
</div> 
-->
 <!--info_share--></div>   <!-- end info_content-->
  </div> <!-- end l-->

<div class="r">   <!--r-->



<div class="sidebar"  style="margin-bottom:2px;">

 
<div class="sidebar_title">工具类大全</div>
<div class="sidebar_info">
<strong><a href="http://www.guoxuedashi.com/lsditu/" target="_blank">历史地图</a></strong>  
<a href="http://www.880114.com/" target="_blank">英语宝典</a>  
<a href="http://www.guoxuedashi.com/13jing/" target="_blank">十三经检索</a> 
<br><strong><a href="http://www.guoxuedashi.com/gjtsjc/" target="_blank">古今图书集成</a></strong> 
<a href="http://www.guoxuedashi.com/duilian/" target="_blank">对联大全</a> <strong><a href="http://www.guoxuedashi.com/xiangxingzi/" target="_blank">象形文字典</a></strong> 

<br><a href="http://www.guoxuedashi.com/zixing/yanbian/">字形演变</a>  <strong><a href="http://www.guoxuemi.com/hafo/" target="_blank">哈佛燕京中文善本特藏</a></strong>
<br><strong><a href="http://www.guoxuedashi.com/csfz/" target="_blank">丛书&方志检索器</a></strong> <a href="http://www.guoxuedashi.com/yqjyy/" target="_blank">一切经音义</a>  

<br><strong><a href="http://www.guoxuedashi.com/jiapu/" target="_blank">家谱族谱查询</a></strong>  <strong><a href="http://shufa.guoxuedashi.com/sfzitie/" target="_blank">书法字帖欣赏</a></strong> 
<br>

</div>
</div>


<div class="sidebar" style="margin-bottom:0px;">

<font style="font-size:22px;line-height:32px">QQ交流群9:489193090</font>


<div class="sidebar_title">手机APP 扫描或点击</div>
<div class="sidebar_info">
<table>
<tr>
	<td width=160><a href="http://m.guoxuedashi.com/app/" target="_blank"><img src="/img/gxds-sj.png" width="140"  border="0" alt="国学大师手机版"></a></td>
	<td>
<a href="http://www.guoxuedashi.com/download/" target="_blank">app软件下载专区</a><br>
<a href="http://www.guoxuedashi.com/download/gxds.php" target="_blank">《国学大师》下载</a><br>
<a href="http://www.guoxuedashi.com/download/kxzd.php" target="_blank">《汉字宝典》下载</a><br>
<a href="http://www.guoxuedashi.com/download/scqbd.php" target="_blank">《诗词曲宝典》下载</a><br>
<a href="http://www.guoxuedashi.com/SiKuQuanShu/skqs.php" target="_blank">《四库全书》下载</a><br>
</td>
</tr>
</table>

</div>
</div>


<div class="sidebar2">
<center>


</center>
</div>

<div class="sidebar"  style="margin-bottom:2px;">
<div class="sidebar_title">网站使用教程</div>
<div class="sidebar_info">
<a href="http://www.guoxuedashi.com/help/gjsearch.php" target="_blank">如何在国学大师网下载古籍?</a><br>
<a href="http://www.guoxuedashi.com/zidian/bujian/bjjc.php" target="_blank">如何使用部件查字法快速查字?</a><br>
<a href="http://www.guoxuedashi.com/search/sjc.php" target="_blank">如何在指定的书籍中全文检索?</a><br>
<a href="http://www.guoxuedashi.com/search/skjc.php" target="_blank">如何找到一句话在《四库全书》哪一页?</a><br>
</div>
</div>


<div class="sidebar">
<div class="sidebar_title">热门书籍</div>
<div class="sidebar_info">
<a href="/so.php?sokey=%E8%B5%84%E6%B2%BB%E9%80%9A%E9%89%B4&kt=1">资治通鉴</a> <a href="/24shi/"><strong>二十四史</strong></a>&nbsp; <a href="/a2694/">野史</a>&nbsp; <a href="/SiKuQuanShu/"><strong>四库全书</strong></a>&nbsp;<a href="http://www.guoxuedashi.com/SiKuQuanShu/fanti/">繁体</a>
<br><a href="/so.php?sokey=%E7%BA%A2%E6%A5%BC%E6%A2%A6&kt=1">红楼梦</a> <a href="/a/1858x/">三国演义</a> <a href="/a/1038k/">水浒传</a> <a href="/a/1046t/">西游记</a> <a href="/a/1914o/">封神演义</a>
<br>
<a href="http://www.guoxuedashi.com/so.php?sokeygx=%E4%B8%87%E6%9C%89%E6%96%87%E5%BA%93&submit=&kt=1">万有文库</a> <a href="/a/780t/">古文观止</a> <a href="/a/1024l/">文心雕龙</a> <a href="/a/1704n/">全唐诗</a> <a href="/a/1705h/">全宋词</a>
<br><a href="http://www.guoxuedashi.com/so.php?sokeygx=%E7%99%BE%E8%A1%B2%E6%9C%AC%E4%BA%8C%E5%8D%81%E5%9B%9B%E5%8F%B2&submit=&kt=1"><strong>百衲本二十四史</strong></a>  <a href="http://www.guoxuedashi.com/so.php?sokeygx=%E5%8F%A4%E4%BB%8A%E5%9B%BE%E4%B9%A6%E9%9B%86%E6%88%90&submit=&kt=1"><strong>古今图书集成</strong></a>
<br>

<a href="http://www.guoxuedashi.com/so.php?sokeygx=%E4%B8%9B%E4%B9%A6%E9%9B%86%E6%88%90&submit=&kt=1">丛书集成</a> 
<a href="http://www.guoxuedashi.com/so.php?sokeygx=%E5%9B%9B%E9%83%A8%E4%B8%9B%E5%88%8A&submit=&kt=1"><strong>四部丛刊</strong></a>  
<a href="http://www.guoxuedashi.com/so.php?sokeygx=%E8%AF%B4%E6%96%87%E8%A7%A3%E5%AD%97&submit=&kt=1">說文解字</a> <a href="http://www.guoxuedashi.com/so.php?sokeygx=%E5%85%A8%E4%B8%8A%E5%8F%A4&submit=&kt=1">三国六朝文</a>
<br><a href="http://www.guoxuedashi.com/so.php?sokeytm=%E6%97%A5%E6%9C%AC%E5%86%85%E9%98%81%E6%96%87%E5%BA%93&submit=&kt=1"><strong>日本内阁文库</strong></a> <a href="http://www.guoxuedashi.com/so.php?sokeytm=%E5%9B%BD%E5%9B%BE%E6%96%B9%E5%BF%97%E5%90%88%E9%9B%86&ka=100&submit=">国图方志合集</a> <a href="http://www.guoxuedashi.com/so.php?sokeytm=%E5%90%84%E5%9C%B0%E6%96%B9%E5%BF%97&submit=&kt=1"><strong>各地方志</strong></a>

</div>
</div>


<div class="sidebar2">
<center>

</center>
</div>
<div class="sidebar greenbar">
<div class="sidebar_title green">四库全书</div>
<div class="sidebar_info">

《四库全书》是中国古代最大的丛书,编撰于乾隆年间,由纪昀等360多位高官、学者编撰,3800多人抄写,费时十三年编成。丛书分经、史、子、集四部,故名四库。共有3500多种书,7.9万卷,3.6万册,约8亿字,基本上囊括了古代所有图书,故称“全书”。<a href="http://www.guoxuedashi.com/SiKuQuanShu/">详细>>
</a>

</div> 
</div>

</div>  <!--end r-->

</div>
<!-- 内容区END --> 

<!-- 页脚开始 -->
<div class="shh">

</div>

<div class="w1180" style="margin-top:8px;">
<center><script src="http://www.guoxuedashi.com/img/plus.php?id=3"></script></center>
</div>
<div class="w1180 foot">
<a href="/b/thanks.php">特别致谢</a> | <a href="javascript:window.external.AddFavorite(document.location.href,document.title);">收藏本站</a> | <a href="#">欢迎投稿</a> | <a href="http://www.guoxuedashi.com/forum/">意见建议</a> | <a href="http://www.guoxuemi.com/">国学迷</a> | <a href="http://www.shuowen.net/">说文网</a><script language="javascript" type="text/javascript" src="https://js.users.51.la/17753172.js"></script><br />
  Copyright &copy; 国学大师 古典图书集成 All Rights Reserved.<br>
  
  <span style="font-size:14px">免责声明:本站非营利性站点,以方便网友为主,仅供学习研究。<br>内容由热心网友提供和网上收集,不保留版权。若侵犯了您的权益,来信即刪。scp168@qq.com</span>
  <br />
ICP证:<a href="http://www.beian.miit.gov.cn/" target="_blank">鲁ICP备19060063号</a></div>
<!-- 页脚END --> 
<script src="http://www.guoxuedashi.com/img/plus.php?id=22"></script>
<script src="http://www.guoxuedashi.com/img/tongji.js"></script>

</body>
</html>
