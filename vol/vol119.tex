<!DOCTYPE html PUBLIC "-//W3C//DTD XHTML 1.0 Transitional//EN" "http://www.w3.org/TR/xhtml1/DTD/xhtml1-transitional.dtd">
<html xmlns="http://www.w3.org/1999/xhtml">
<head>
<meta http-equiv="Content-Type" content="text/html; charset=utf-8" />
<meta http-equiv="X-UA-Compatible" content="IE=Edge,chrome=1">
<title>資治通鑒_120-資治通鑑卷一百十九_120-資治通鑑卷一百十九</title>
<meta name="Keywords" content="資治通鑒_120-資治通鑑卷一百十九_120-資治通鑑卷一百十九">
<meta name="Description" content="資治通鑒_120-資治通鑑卷一百十九_120-資治通鑑卷一百十九">
<meta http-equiv="Cache-Control" content="no-transform" />
<meta http-equiv="Cache-Control" content="no-siteapp" />
<link href="/img/style.css" rel="stylesheet" type="text/css" />
<script src="/img/m.js?2020"></script> 
</head>
<body>
 <div class="ClassNavi">
<a  href="/24shi/">二十四史</a> | <a href="/SiKuQuanShu/">四库全书</a> | <a href="http://www.guoxuedashi.com/gjtsjc/"><font  color="#FF0000">古今图书集成</font></a> | <a href="/renwu/">历史人物</a> | <a href="/ShuoWenJieZi/"><font  color="#FF0000">说文解字</a></font> | <a href="/chengyu/">成语词典</a> | <a  target="_blank"  href="http://www.guoxuedashi.com/jgwhj/"><font  color="#FF0000">甲骨文合集</font></a> | <a href="/yzjwjc/"><font  color="#FF0000">殷周金文集成</font></a> | <a href="/xiangxingzi/"><font color="#0000FF">象形字典</font></a> | <a href="/13jing/"><font  color="#FF0000">十三经索引</font></a> | <a href="/zixing/"><font  color="#FF0000">字体转换器</font></a> | <a href="/zidian/xz/"><font color="#0000FF">篆书识别</font></a> | <a href="/jinfanyi/">近义反义词</a> | <a href="/duilian/">对联大全</a> | <a href="/jiapu/"><font  color="#0000FF">家谱族谱查询</font></a> | <a href="http://www.guoxuemi.com/hafo/" target="_blank" ><font color="#FF0000">哈佛古籍</font></a> 
</div>

 <!-- 头部导航开始 -->
<div class="w1180 head clearfix">
  <div class="head_logo l"><a title="国学大师官网" href="http://www.guoxuedashi.com" target="_blank"></a></div>
  <div class="head_sr l">
  <div id="head1">
  
  <a href="http://www.guoxuedashi.com/zidian/bujian/" target="_blank" ><img src="http://www.guoxuedashi.com/img/top1.gif" width="88" height="60" border="0" title="部件查字,支持20万汉字"></a>


<a href="http://www.guoxuedashi.com/help/yingpan.php" target="_blank"><img src="http://www.guoxuedashi.com/img/top230.gif" width="600" height="62" border="0" ></a>


  </div>
  <div id="head3"><a href="javascript:" onClick="javascript:window.external.AddFavorite(window.location.href,document.title);">添加收藏</a>
  <br><a href="/help/setie.php">搜索引擎</a>
  <br><a href="/help/zanzhu.php">赞助本站</a></div>
  <div id="head2">
 <a href="http://www.guoxuemi.com/" target="_blank"><img src="http://www.guoxuedashi.com/img/guoxuemi.gif" width="95" height="62" border="0" style="margin-left:2px;" title="国学迷"></a>
  

  </div>
</div>
  <div class="clear"></div>
  <div class="head_nav">
  <p><a href="/">首页</a> | <a href="/ShuKu/">国学书库</a> | <a href="/guji/">影印古籍</a> | <a href="/shici/">诗词宝典</a> | <a   href="/SiKuQuanShu/gxjx.php">精选</a> <b>|</b> <a href="/zidian/">汉语字典</a> | <a href="/hydcd/">汉语词典</a> | <a href="http://www.guoxuedashi.com/zidian/bujian/"><font  color="#CC0066">部件查字</font></a> | <a href="http://www.sfds.cn/"><font  color="#CC0066">书法大师</font></a> | <a href="/jgwhj/">甲骨文</a> <b>|</b> <a href="/b/4/"><font  color="#CC0066">解密</font></a> | <a href="/renwu/">历史人物</a> | <a href="/diangu/">历史典故</a> | <a href="/xingshi/">姓氏</a> | <a href="/minzu/">民族</a> <b>|</b> <a href="/mz/"><font  color="#CC0066">世界名著</font></a> | <a href="/download/">软件下载</a>
</p>
<p><a href="/b/"><font  color="#CC0066">历史</font></a> | <a href="http://skqs.guoxuedashi.com/" target="_blank">四库全书</a> |  <a href="http://www.guoxuedashi.com/search/" target="_blank"><font  color="#CC0066">全文检索</font></a> | <a href="http://www.guoxuedashi.com/shumu/">古籍书目</a> | <a   href="/24shi/">正史</a> <b>|</b> <a href="/chengyu/">成语词典</a> | <a href="/kangxi/" title="康熙字典">康熙字典</a> | <a href="/ShuoWenJieZi/">说文解字</a> | <a href="/zixing/yanbian/">字形演变</a> | <a href="/yzjwjc/">金 文</a> <b>|</b>  <a href="/shijian/nian-hao/">年号</a> | <a href="/diming/">历史地名</a> | <a href="/shijian/">历史事件</a> | <a href="/guanzhi/">官职</a> | <a href="/lishi/">知识</a> <b>|</b> <a href="/zhongyi/">中医中药</a> | <a href="http://www.guoxuedashi.com/forum/">留言反馈</a>
</p>
  </div>
</div>
<!-- 头部导航END --> 
<!-- 内容区开始 --> 
<div class="w1180 clearfix">
  <div class="info l">
   
<div class="clearfix" style="background:#f5faff;">
<script src='http://www.guoxuedashi.com/img/headersou.js'></script>

</div>
  <div class="info_tree"><a href="http://www.guoxuedashi.com">首页</a> > <a href="/SiKuQuanShu/fanti/">四库全书</a>
 > <h1>资治通鉴</h1> <!--         下载:【右键另存为】即可 --></div>
  <div class="info_content zj clearfix">
  
<div class="info_txt clearfix" id="show">
<center style="font-size:24px;">120-資治通鑑卷一百十九</center>
    資治通鑑卷一百十九<br />
<br />
  宋 司馬光 撰<br />
<br />
  胡三省 音注<br />
<br />
  宋紀一【起上章涒灘盡昭陽大淵獻凡四年 劉氏世居彭城彭城於春秋之時宋土也故帝之始建國號曰宋】<br />
<br />
  高祖武皇帝【諱裕字德輿小字寄奴姓劉氏彭城縣綏德里人漢楚元王交二十一世孫也彭城楚都故苗裔家焉晉氏東遷劉氏移居晉陵丹徒之京口里】<br />
<br />
  永初元年【是年六月改元】春正月己亥魏主還宫【晉有天下通鑑於魏主率兼書名是年宋主受禪】 秦王熾磐立其子乞伏暮末為大子【熾昌志翻 考異曰晉書作乞伏慕末宋書又作乞佛茂蔓今從崔鴻十六國春秋作暮末】仍領撫軍大將軍都督中外諸軍事大赦改元建弘 宋王欲受禪而難於發言乃集朝臣宴飲【此宋朝之臣也朝直遥翻】從容言曰【從千容翻】桓玄簒位鼎命已移我首唱大義興復帝室南征北伐平定四海功成業著遂荷九錫【荷下可翻】今年將衰暮崇極如此物忌盛滿非可久安今欲奉還爵位歸老京師羣臣惟盛稱功德莫諭其意日晩坐散【坐徂卧翻】中書令傅亮還外乃悟而宫門已閉亮叩扉請見【見賢遍翻】王即開門見之亮入但曰臣暫宜還都王解其意無復他言【解戶買翻曉也復扶又翻】直云須幾人自送亮曰數十人可也即時奉辭亮出已夜見長星竟天拊髀歎曰我常不信天文今始驗矣【長星所以除舊布新故云然】亮至建康夏四月徵王入輔王留子義康為都督豫司雍并四州諸軍事豫州刺史鎮夀陽【豫州後漢治譙魏治汝南安成晉平吳治陳國江左治夀陽蕪湖邾城牛渚歷陽馬頭夀春姑孰不常厥居安帝之末帝欲開拓河南綏定豫土割揚州大江以西大雷以北悉屬豫州豫州基址因此而立帝既平關洛置司州刺史治虎牢領河南滎陽弘農實土三郡河内東京兆二僑郡雍州仍僑治襄陽秦并州刺史鎮蒲阪毛德祖既自蒲阪退屯虎牢則并州當寄治虎牢也雍於用翻】義康尚幼以相國參軍南陽劉湛為長史決府州事【府州都督府及豫州也】湛自弱年即有宰物之情常自比管葛【謂管仲諸葛亮也】博涉書史不為文章不喜談議【喜許記翻】王甚重之 五月乙酉魏更諡宣武帝曰道武帝【魏主嗣永興二年諡父珪曰宣武皇帝更古衡翻】 魏淮南公司馬國璠池陽子司馬道賜謀外叛司馬文思告之庚戌魏主殺國璠道賜賜文思爵鬱林公【國璠等降魏見上卷晉安帝義熙十三年璠孚袁翻】國璠等連引平城豪桀坐族誅者數十人章安侯封懿之子玄之當坐魏主以玄之燕朝舊族【慕容廆興於昌黎封氏依之遂世仕於燕貴顯】欲宥其一子玄之曰弟子磨奴早孤乞全其命乃殺玄之四子而宥磨奴 六月壬戌王至建康傅亮諷晉恭帝禪位於宋具詔草呈帝使書之帝欣然操筆謂左右曰桓玄之時晉氏已無天下重為劉公所延將二十載【晉安帝元興三年裕討桓玄至是凡十七年操千高翻重直龍翻載子亥翻】今日之事本所甘心遂書赤紙為詔甲子帝遜於琅邪第百官拜辭祕書監徐廣流涕哀慟【晉武帝泰始元年受禪歲在乙酉建興四年長安陷歲在丙子凡五十二年次年元帝建號於江東改元建武至是年歲在庚申凡一百單三年西東享國共一百五十七年而亡】丁卯王為壇於南郊即皇帝位禮畢自石頭備法駕入建康宫徐廣又悲感流涕侍中謝晦謂之曰徐公得無小過廣曰君為宋朝佐命【朝直遙翻】身是晉室遺老悲歡之事固不可同廣邈之弟也【徐邈為晉孝武所親重】帝臨太極殿大赦改元其犯鄉論清議一皆蕩滌與之更始【犯郷論清議盖得罪於名教者更工衡翻】<br />
<br />
  裴子野論曰昔重華受終四凶流放【書堯使舜嗣位正月上日受終於文祖流共工于幽州放驩兜于崇山竄三苗于三危殛鯀于羽山四罪而天下咸服重直龍翻】武王克殷頑民遷洛【武王克殷遷頑民于洛邑】天下之惡一也鄉論清議除之過矣<br />
<br />
  奉晉恭帝為零陵王優崇之禮皆倣晉初故事即宫于故秣陵縣【沈約曰秣陵縣本治去京邑六十里今故治村是也晉安帝義熙九年移治京邑在鬬場恭帝元元年省楊州禁防參軍縣移治其處】使冠軍將軍劉遵考將兵防衛【冠古亮翻將即亮翻】降禇后為王妃追尊皇考為孝穆皇帝皇妣趙氏為孝穆皇后尊王太后蕭氏為皇太后【帝父翹娶趙氏生帝而徂繼室以蕭氏】上事蕭太后素謹及即位春秋已高每旦入朝太后【朝直遙翻】未嘗失時刻詔晉氏封爵當隨運改獨置始興廬陵始安長沙康樂五公降爵為縣公及縣侯【始興廬陵始安長沙皆郡公獨康樂縣公耳據南史降始興郡公為華容縣公廬陵公為柴桑縣公始安公為荔浦縣侯長沙公為醴陵縣侯樂音洛】以奉王導謝安温嶠陶侃謝玄之祀其宣力義熙豫同艱難者一仍本秩庚午以司空道憐為太尉封長沙王追封司徒道規為臨川王以道憐子義慶襲其爵其餘功臣徐羨之等增位進爵各有差追封劉穆之為南康郡公王鎮惡為龍陽縣侯上每歎念穆之曰穆之不死當助我治天下【治直之翻】可謂人之云亡邦國殄瘁【詩曕卬之辭瘁秦醉翻】又曰穆之死人輕易我【易以豉翻】立皇子桂陽公義真為廬陵王彭城公義隆為宜都王義康為彭城王己卯改泰始歷為永初歷【以元改歷】 魏主如翳犢山遂至馮滷池【據北史翳犢山在平城之西五原之東馮滷池即五原鹽池唐屬鹽州界滷龍五翻】聞上受禪驛召崔浩告之曰卿往年之言驗矣【浩言見上卷晉安帝義熙十四年】朕於今日始信天道 秋七月丁酉魏主如五原 甲辰詔以涼公歆為都督高昌等七郡諸軍事征西大將軍酒泉公秦王熾磐為安西大將軍【熾昌志翻】 交州刺史杜慧度擊林邑大破之【林邑屢為寇故慧度撃之】所殺過半林邑乞降前後為所鈔掠者皆遣還【降戶江翻鈔楚交翻】慧度在交州為政纎密一如治家【治直之翻】吏民畏而愛之城門夜開道不拾遺 己未魏主如雲中 河西王蒙遜欲伐涼先引兵攻秦浩亹【浩亹音告門】旣至濳師還屯川巖涼公歆欲乘虛襲張掖宋繇張體順切諫不聽太后尹氏謂歆曰汝新造之國地狹民希自守猶懼不足何暇伐人先王臨終【李暠卒見上卷晉安帝義熙十三年】殷勤戒汝深慎用兵保境寧民以俟天時言猶在耳奈何棄之蒙遜善用兵非汝之敵數年以來常有兼并之志汝國雖小足為善政修德養民靜以待之彼若昏暴民將歸汝若其休明【休美也】汝將事之豈得輕為舉動僥冀非望以吾觀之非但喪師【喪息浪翻】殆將亡國亦不聽宋繇歎曰今兹大事去矣歆將步騎三萬東出【將即亮翻騎奇寄翻】蒙遜聞之曰歆已入吾術中然聞吾旋師必不敢前乃露布西境云已克浩亹將進攻黄谷【此露布非必建之漆竿如魏晉告捷之制但露檄布言其事耳】歆聞之喜進入都瀆澗蒙遜引兵擊之戰於懷城歆大敗或勸歆還保酒泉歆曰吾違老母之言以取敗不殺此胡何面目復見我母【復扶又翻】遂勒兵戰於蓼泉為蒙遜所殺歆弟酒泉太守翻新城太守預領羽林右監密左將軍眺右將軍亮西奔敦煌蒙遜入酒泉【安帝隆安四年李嵩據敦煌凡二主二十一年而滅敦徒門翻】禁侵掠士民安堵以宋繇為吏部郎中委之選舉涼之舊臣有才望者咸禮而用之以其子牧犍為酒泉太守【犍居言翻守式又翻】敦煌太守李恂翻之弟也與翻等棄敦煌奔北山蒙遜以索嗣之子元緒行敦煌太守【索嗣死事見一百十二卷晉安帝隆安四年索昔各翻】蒙遜還姑臧見涼太后尹氏而勞之尹氏曰李氏為胡所滅知復何言【蒙遜張掖盧水胡也勞力到翻復扶又翻下可復同】或謂尹氏曰今母子之命在人掌握奈何傲之且國亡子死曾無憂色何也尹氏曰存亡死生皆有天命奈何更如凡人為兒女子之悲乎吾老婦人國亡家破豈可復惜餘生為人臣妾乎惟速死為幸耳蒙遜嘉而赦之娶其女為牧犍婦 八月辛未追諡妃臧氏為敬皇后癸酉立王太子義符為皇太子 閏月壬午詔晉帝諸陵悉署守衛 九月秦振武將軍王基等襲河西王蒙遜胡園戍俘二千餘人而還【還徒宣翻又如字】 李恂在敦煌有惠政索元緒麤險好殺大失人和【好呼到翻】郡人宋承張弘密信招恂冬恂帥數十騎入敦煌元緒東奔涼興【涼興郡在唐瓜州常樂縣界帥讀曰率】承等推恂為冠軍將軍涼州刺史【冠古玩翻】改元永建河西王蒙遜遣世子政德攻敦煌恂閉城不戰 十二月丁亥杏城羌酋狄温子帥三千餘家降魏【背夏降魏也酋慈由翻帥讀曰率降戶江翻】 是歲魏姚夫人卒追諡昭哀皇后【姚夫人歸魏見一百十七卷晉安帝義熙十一年】二年春正月辛酉上祀南郊大赦<br />
<br />
  裴子野論曰夫郊祀天地修歲事也赦彼有罪夫何為哉<br />
<br />
  以揚州刺史廬陵王義真為司徒尚書僕射徐羨之為尚書令揚州刺史中書令傅亮為尚書僕射辛未魏主嗣行如公陽 河西王蒙遜帥衆二萬攻李恂于敦煌 秦王熾磐遣征北將軍木弈干輔國將軍元基攻上邽遇霖雨而還 三月甲子魏陽平王熙卒 魏主發代都六千人築苑東包白登周三十餘里 河西王蒙遜築隄壅水以灌敦煌李恂乞降不許恂將宋承等舉城降恂自殺蒙遜屠其城獲恂弟子寶囚于姑臧【李氏滅矣李寶卒由此開有唐之基天之所啓誰能廢之】於是西域諸國皆詣蒙遜稱臣朝貢【朝直遥翻】 夏四月己卯朔詔所在淫祠自蔣子文以下皆除之其先賢及以勲德立祠者不在此例 吐谷渾王阿柴遣使降秦【使疏吏翻降戶江翻】秦王熾磐以阿柴為征西大將軍開府儀同三司安州牧白蘭王【秦蓋以吐谷渾之地為安州】 六月乙酉魏主北巡至蟠羊山【蟠羊山在參合陂東】秋七月西巡至河 河西王蒙遜遣右衛將軍沮渠鄯善建節將軍沮渠苟生帥衆七千伐秦秦王熾磐遣征北將軍木奕干等帥步騎五千拒之敗鄯善等于五澗【五澗在洪池嶺北水經註云五澗水出姑臧城東而西北流注馬城河敗補邁翻】虜苟生斬首二千而還 初帝以毒酒一甖【甖於耕翻瓦器也】授前琅邪郎中令張偉使酖零陵王偉歎曰酖君以求生不如死乃於道自飲而卒【卒子恤翻】偉邵之兄也【初帝領揚州牧辟邵為僚屬】太常褚秀之侍中褚淡之皆王之妃兄也王每生男帝輒令秀之兄弟方便殺之【方便者隨宜處分不令其事彰露也】王自遜位深慮禍及與禇妃共處一室【處昌呂翻】自煮食於牀前飲食所資皆出禇妃故宋人莫得伺其隙【伺相吏翻】九月帝令淡之與兄右衛將軍叔度往視妃妃出就别室相見兵人踰垣而入進藥於王王不肯飲曰佛教自殺者不復得人身兵人以被掩殺之【復扶又翻 考異曰宋本紀九月己丑零陵王薨晉本紀九月丁丑據長歷九月丙午朔無己丑丁丑今不書日】帝帥百官臨於朝堂三日【自是之後禪讓之君罕得全矣帥讀曰率臨力鴆翻朝直遙翻】 庚戌魏主還宫 冬十月己亥詔以河西王蒙遜為鎮軍大將軍開府儀同三司涼州刺史 己亥魏主如代 十一月辛亥葬晉恭帝於沖平陵帝帥百官瞻送 十二月丙申魏主西巡至雲中秦王熾磐遣征西將軍孔子等帥騎二萬擊契汗秃真於羅川【契欺訖翻汗音寒】 河西王蒙遜所署晉昌太守唐契據郡叛蒙遜遣世子政德討之契瑶之子也【唐瑶見一百十一卷晉安帝隆安四年】 上之為宋公也謝瞻為宋臺中書侍郎其弟晦為右衛將軍時晦權遇已重自彭城還都迎家【上為宋公時建宋臺於彭城】賓客輻湊門巷填咽瞻在家驚駭謂晦曰汝名位未多而人歸趣乃爾【趣七喻翻】吾家素以恬退為業不願干豫時事交遊不過親朋而汝遂勢傾朝野【朝直遙翻】此豈門戶之福邪乃以籬隔門庭曰吾不忍見此及還彭城言於宋公曰臣本素士父祖位不過二千石【瞻晦晉太常謝裒之玄孫於謝安為從孫是其高曾與謝安同其所自出但名位不及耳】弟年始三十志用凡近榮冠臺府【冠古玩翻】位任顯密福過災生其應無遠特乞降黜以保衰門前後屢陳之晦或以朝廷密事語瞻【語牛倨翻】瞻故向親舊陳說用為戲笑以絶其言及上即位晦以佐命功位任益重瞻愈憂懼是歲瞻為豫章太守遇病不療臨終遺晦書曰吾得啓體幸全亦何所恨【遺于季翻曾子有疾召門弟子曰啓予足啓予手詩云戰戰兢兢如臨深淵如履薄氷而今而後吾知免夫小子孔子曰身體髪膚受之父母不敢毁傷父母全而生之子全而歸之】弟思自勉勵為國為家【居寵思危謝瞻有焉為謝晦殺身亡家張本為于偽翻】<br />
<br />
  三年春正月甲辰朔魏主自雲中西巡至屋竇城【據北史屋竇城在薛林山東】 癸丑以徐羨之為司空録尚書事刺史如故江州刺史王弘為衛將軍開府儀同三司中領軍謝晦為領軍將軍兼散騎常侍入直殿省總統宿衛【散悉亶翻騎奇寄翻】徐羨之起自布衣【徐羨之為桓修撫軍中兵參軍與帝同府深相親結及起義兵益見親任】又無術學直以志力局度一旦居廊廟朝野推服咸謂有宰臣之望沈密寡言不以憂喜見色【朝直遙翻沈持林翻見賢遍翻】頗工奕棊觀戲常若未解【解戶買翻曉也】當世倍以此推之傅亮蔡廓常言徐公曉萬事安異同常與傅亮謝晦宴聚亮晦才學辯博羨之風度詳整時然後言鄭鮮之歎曰觀徐傅言論不復以學問為長【復扶又翻】 秦征西將軍孔子等大破契汗秃真【契欺訖翻汗音寒秃吐谷翻】獲男女二萬口牛羊五十餘萬頭禿真帥騎數千西走其别部樹奚帥戶五千降秦【帥讀曰率降戶江翻】 二月丁丑詔分豫州淮以東為南豫州治歷陽以彭城王義康為刺史【義熙之初帝欲開拓河南綏定豫土至九年割揚州大江以西大雷以北悉屬豫州至是以淮西之地為北豫州治汝南沈約志南豫州領歷陽南譙盧江南汝隂南梁晉熙弋陽安豐南汝南新蔡東郡南潁潁川西汝隂汝陽陳留南陳左郡邊城左郡光城左郡十九郡案徐志及永初郡國志止領十三郡盖沈志有景平以後續置郡在其間也】又分荆州十郡置湘州治臨湘【晉安帝義熙十三年省湘州今復置臨湘漢舊縣唐為潭州長沙縣】以左衛將軍張邵為刺史丙戌魏主還宫 三月上不豫大尉長沙王道憐司空徐羨之尚書僕射傅亮領軍將軍謝晦護軍將軍檀道濟並入侍醫藥羣臣請祈禱神祇上不許唯使侍中謝方明以疾告宗廟而已上性不信奇怪微時多符瑞及貴史官審以所聞上拒而不答檀道濟出為鎮北將軍南兖州刺史鎮廣陵悉監淮南諸軍【晉成帝立南兖州治京口自此治廣陵領廣陵海陵山陽盱眙秦川南沛等郡監工銜翻】皇太子多狎羣小謝晦言於上曰陛下春秋既高宜思存萬世神器至重不可使負荷非才【晦發此言已有廢昏立明之意荷下可翻又如字】上曰廬陵何如晦曰臣請觀焉出造廬陵王義真【造七到翻】義真盛欲與談晦不甚答還曰德輕於才非人主也丁未出義真為都督南豫豫雍司秦并六州諸軍事車騎將軍開府儀同三司南豫州刺史【為晦等殺義真張本雍於用翻騎奇寄翻】是後大州率加都督多者或至五十州不可復詳載矣【迄宋之季境内惟二十二州至梁武帝時沿邊分置諸州始有五十州復扶又翻】帝疾瘳【瘳丑留翻】己未大赦 秦雍流民南入梁州庚申遣使送絹萬匹且漕荆雍之穀以賑之【秦雍之雍古雍州也關中之地荆雍之雍晉末所置南雍州也治襄陽使疏吏翻賑之忍翻】 刁逵之誅也【事見一百十三卷晉安帝元興三年】其子彌亡命辛酉彌帥數十人入京口【帥讀曰率】太尉留府司馬陸仲元撃斬之【時長沙王道憐以太尉鎮京口入侍醫藥故有留府】乙丑魏河南王曜卒 夏四月甲戌魏立皇子燾為<br />
<br />
  太平王拜相國加大將軍丕為樂平王彌為安定王範為樂安王健為永昌王崇為建寧王俊為新興王 乙亥詔封仇池公楊盛為武都王 秦王熾磐以折衝將軍乞伏是辰為西胡校尉築列渾城於汁羅以鎮之【汁羅盖即羅川之地】 五月帝疾甚召太子誡之曰檀道濟雖有幹畧而無遠志非如兄韶有難御之氣也徐羨之傅亮當無異圖謝晦數從征伐頗識機變若有同異必此人也【帝固有疑晦之心矣數所角翻】又為手詔曰後世若有幼主朝事一委宰相母后不煩臨朝【朝直遙翻】司空徐羨之中書令傅亮領軍將軍謝晦鎮北將軍檀道濟同被顧命癸亥帝殂於西殿【年六十自是以後南北朝之君没皆稱殂被皮義翻顧音古】帝清簡寡欲嚴整有法度被服居處儉於布素遊宴甚稀嬪御至少【處昌呂翻少詩沼翻】嘗得後秦高祖從女【後秦王興廟號高祖從才用翻】有盛寵頗以廢事謝晦微諫即時遣出財帛皆在外府内無私藏【藏徂浪翻】嶺南嘗獻入筒細布一端八丈帝惡其精麗勞人【揚雄蜀都賦曰布則蜘蛛作絲不可見風䇶中黄潤一端數金言其細也惡烏路翻】即付有司彈太守【彈徒丹翻】以布還之并制嶺南禁作此布公主出適遣送不過二十萬無錦繡之物内外奉禁莫敢為侈靡太子即皇帝位年十七大赦尊皇太后曰太皇太后立妃司馬氏為皇后晉恭帝女海鹽公主也 魏主服寒食散頻年藥發災異屢見【見賢遍翻】頗以自憂遣中使密問白馬公崔浩曰【使疏吏翻】屬者日食趙代之分【屬者猶言比者近者屬之欲翻分扶問翻】朕疾彌年不愈恐一旦不諱諸子竝少【少詩照翻】將若之何其為我思身後之計【為于偽翻】浩曰陛下春秋富盛行就平愈必不得已請陳瞽言自聖代龍興不崇儲貳是以永興之始社稷幾危【事見一百十五卷晉安帝義熙五年幾居希翻又音祁】今宜早建東宫選賢公卿以為師傅左右信臣以為賓友入總萬幾出撫戎政如此則陛下可以優游無為頤神養夀萬歲之後國有成主民有所歸姦宄息望禍無自生矣皇子燾年將周星【歲星十二年一周天】明叡温和立子以長禮之大經若必待成人然後擇之倒錯天倫則召亂之道也【倒錯謂廢長立少長知兩翻下同】魏主復以問南平公長孫嵩【復扶又翻】對曰立長則順置賢則人服燾長且賢天所命也帝從之立太平王燾為皇太子使之居正殿臨朝為國副主【朝直遙翻】以長孫嵩及山陽公奚斤【魏收官氏志後魏獻帝弟為達奚氏孝文改為奚氏】北新公安同為左輔坐東廂西面崔浩與太尉穆觀散騎常侍代人丘堆為右弼【後魏孝文以獻帝第五兄敦丘氏為丘氏】坐西廂東面百官總已以聽焉【坐東廂者西面坐西廂者東面皆朝拱皇太子】帝避居西宫時隱而窺之【自隱蔽其身而窺之也】聽其決斷【斷丁亂翻】大悦謂侍臣曰嵩宿德舊臣歷事四世功存社稷【嵩事昭成帝及道武帝明元帝及太子燾為四世】斤辯捷智謀名聞遐邇【聞音問】同曉解俗情明練於事觀達於政要識吾旨趣浩博聞彊識精察天人堆雖無大用然在公專謹以此六人輔相太子吾與汝曹巡行四境伐叛柔服足以得志於天下矣【解胡買翻行下孟翻 嵩實姓拔拔斤姓達奚觀姓丘穆陵堆姓丘敦是時魏之羣臣出于代北者姓多重複及高祖遷洛始皆改之舊史惡其煩雜難知故皆從後姓以就簡易今從之】魏主又以典東西部劉絜門下奏事代人古弼【重直龍翻惡烏路翻易以䜴翻道武天賜四年置侍官侍直左右出納詔命魏書官氏志内入諸姓吐奚氏為古氏】直郎徒河盧魯元【拓跋與慕容段氏同出鮮卑其後強盛謂東種為徒河官氏志内入諸姓吐復伏盧氏為盧氏】忠謹公勤使之給侍東宫分典機要宣納辭令太子聰明有大度羣臣時奏所疑帝曰此非我所知當決之汝曹國主也六月壬申以尚書僕射傅亮為中書監尚書令以領軍將軍謝晦領中書令侍中謝方明為丹陽尹方明善治郡【治直之翻】所至有能名承代前人不易其政必宜改者則以漸移變使無迹可尋 戊子長沙景王道憐卒 魏建義將軍刁雍寇青州州兵擊破之雍收散卒走保大鄉山【魏收地形志濟隂郡乘氏縣有大鄉城雍於容翻】 秋七月己酉葬武皇帝於初寧陵【陵在丹陽建康縣蔣山】廟號高祖 河西王蒙遜遣前將軍沮渠成都帥衆一萬耀兵嶺南遂屯五澗【盖耀兵于洪池嶺南而還屯五澗也沮子余翻帥讀曰率下同】九月秦王熾磐遣征北將軍出連䖍等帥騎六千擊之 初魏主聞高祖克長安【見上卷晉安帝義熙十三年】大懼遣使請和自是每歲交聘不絶及高祖殂殿中將軍沈範等奉使在魏【晉置二衛仍置殿中將軍使疏吏翻下同】還及河魏主遣人追執之議發兵取洛陽虎牢滑臺崔浩諫曰陛下不以劉裕欻起納其使貢【欻許勿翻】裕亦敬事陛下不幸今死遽乘喪伐之雖得之不足為美且國家今日亦未能一舉取江南也而徒有伐喪之名【禮不伐喪】竊為陛下不取【為于偽翻】臣謂宜遣人弔祭存其孤弱恤其凶災使義聲布於天下則江南不攻自服矣况裕新死黨與未離兵臨其境必相帥拒戰【帥讀曰率】功不可必不如緩之待其彊臣爭權變難必起【難乃旦翻】然後命將出師可以兵不疲勞坐收淮北也【將即亮翻】魏主曰劉裕乘姚興之死而滅之【事見一百十七卷晉安帝義熙十二年及上卷義熙十三年】今我乘裕喪而伐之何為不可浩曰不然姚興死諸子交爭故裕乘舋伐之今江南無舋不可比也【舋與釁同許覲翻】魏主不從假司空奚斤節加晉兵大將軍行揚州刺史使督宋兵將軍交州刺史周幾【後魏孝文以獻帝次兄普氏之後為周氏】吳兵將軍廣州刺史公孫表同入寇【晉兵宋兵吳兵鄭兵楚兵等將軍皆魏所置】 乙巳魏主如灅南宫遂如廣甯【晉愍帝建興元年猗盧築新平城於灅北其後築宫於灅南酈道元曰廣甯去平城五十里廣甯縣漢屬上谷郡晉太康中立為郡灅力水翻】 辛亥魏人築平城外郭周圍三十二里 魏主如喬山【五代志喬山在涿郡懷戎縣劉昫曰唐媯州懷戎縣後漢上谷之潘縣也】遂東如幽州冬十月甲戌還宫魏軍將發公卿集議於監國之前【監工銜翻】以先攻城與先略地奚斤欲先攻城崔浩曰南人長於守城昔苻氏攻襄陽經年不拔【事見一百四卷晉孝武太元三年四年】今以大兵坐攻小城若不時克挫傷軍勢敵得徐嚴而來我怠彼鋭此危道也不如分軍略地至淮為限列置守宰收歛租穀則洛陽滑臺虎牢更在軍北絶望南救必沿河東走不則為囿中之物【不讀曰否】何憂其不獲也公孫表固請攻城魏主從之於是奚斤等帥步騎二萬【帥讀曰率騎奇寄翻】濟河營於滑臺之東時司州刺史毛德祖戍虎牢東郡太守王景度告急於德祖【王景度以東郡太守戍滑臺】德祖遣司馬翟廣等將步騎三千救之【翟長伯翻將即亮翻下同】先是司馬楚之聚衆在陳留之境聞魏兵濟河遣使迎降魏【先悉薦翻使疏吏翻降戶江翻】以楚之為征南將軍荆州刺史使侵擾北境【北謂宋之北境】德祖遣長社令王法政將五百人戍邵陵【邵陵縣漢屬汝南郡晉以後屬潁川郡杜佑曰蔡州郾陵縣有古召陵城】將軍劉憐將二百騎戍雍丘以備之楚之引兵襲憐不克會臺送軍資憐出迎之酸棗民王玉馳以告魏【酸棗縣自漢以來屬陳留郡唐屬滑州】丁酉魏尚書滑稽引兵襲倉垣【康曰滑戶入切姓也本滑伯國姬姓其後因國為氏漢有詹事滑典】兵吏悉踰城走陳留太守馮翊嚴稜詣斤降魏以王玉為陳留太守給兵守倉垣【魏收地形志陳留郡治浚儀縣有倉垣城】奚斤等攻滑臺不拔求益兵魏主怒切責之壬辰自將諸國兵五萬餘人南出天關踰恒嶺【此即晉孝武大元二十一年燕主垂襲魏平城之路魏主珪既平中山自望都鐵閼鑿恒嶺至代五百餘里將即亮翻下同】為斤等聲援 秦出連䖍與河西沮渠成都戰禽之【沮渠成都時屯五澗】 十一月魏太子燾將兵出屯塞上【魏主南援攻河南之兵故大子屯塞上以備柔然】使安定王彌與安同居守【守式又翻】庚戌奚斤等急攻滑臺拔之王景度出走景度司馬陽瓚為魏所執不降而死【瓚藏旱翻降戶江翻】魏主以成皋侯苟兒為兖州刺史鎮滑臺【魏書官氏志内入諸姓若干氏為苟氏】斤等進擊翟廣等於土樓破之【上樓在虎牢東九域志澶州臨河縣有土樓鎮】乘勝進逼虎牢毛德祖與戰屢破之魏主别遣黑矟將軍于栗磾將三千人屯河陽謀取金墉【矟色角翻磾于奚翻】德祖遣鎮威將軍竇晃等緣河拒之十二月丙戌魏主至冀州遣楚兵將軍徐州刺史叔孫建將兵自平原濟河徇青兖豫州刺史劉粹遣治中高道瑾將步騎五百據項城【宋豫州領汝南新蔡譙梁陳南頓潁川汝陽汝隂陳留等郡】徐州刺史王仲德將兵屯湖陸【徐州領彭城沛下邳蘭陵東海東莞東安琅邪淮陽陽平濟隂北濟隂鍾離馬頭等郡】于栗磾濟河與奚斤并力攻竇晃等破之魏主遣中領軍代人娥清期思侯柔然閭大肥將兵七千人會周幾叔孫建南渡河軍於碻磝【碻磝城臨河津後魏為濟州治所水經注曰城即故荏平縣也】癸未兖州刺史徐琰棄尹卯南走【水經濟水自須昌縣西北涇漁山東又北過穀城縣西註云濟水側岸有尹卯壘南去漁山四十餘里是穀城縣界故春秋之小穀城也】於是泰山高平金鄊等郡皆没於魏【金鄊縣漢屬山陽晉屬高平盖晉末分置郡也】叔孫建等東入青州司馬愛之季之先聚衆於濟東皆降於魏【濟水之東則青州界濟子禮翻】戊子魏兵逼虎牢青州刺史東莞竺夔鎮東陽城【青州自曹嶷以來治廣固武帝克慕容超夷其城青州遷治東陽城在廣縣西南宋白曰今青州治益都縣州東城即東陽城晉武帝太康初分琅邪立東莞郡青州領齊濟南高密樂安平昌北海東萊太原長廣等郡莞音官】遣使告急【使疏吏翻】己丑詔南兖州刺史檀道濟監征討諸軍事【監工銜翻】與王仲德共救之廬陵王義真遣龍驤將軍沈叔狸將三千人就劉粹量宜赴援【義真時鎮夀陽劉粹時鎮懸瓠驤思將翻量音良】 秦王熾磐徵秦州牧曇達為左丞相征東大將軍【曇徒含翻】<br />
<br />
  營陽王【諱義符小字車兵武帝長子也 考異曰宋本紀高氏小史皆作滎陽臧后謝晦蔡廓傳作營陽營陽南方郡名也今徒之】<br />
<br />
  景平元年春正月己亥朔大赦改元 辛丑帝祀南郊魏于栗磾攻金墉癸卯河西太守王涓之棄城走魏<br />
<br />
  主以栗磾為豫州刺史鎮洛陽【磾丁奚翻】 魏主南巡恒嶽【恒戶登翻】丙辰至鄴【去年十二月已書魏主至冀州今又書南巡恒嶽必有一誤也】 己未詔徵豫章太守蔡廓為吏部尚書【自晉以來謂吏部尚書為大尚書以其在諸曹之右且其權任要重也】廓謂傅亮曰選事若悉以見付不論【不論者不復置議論于辭受之際也】不然不能拜也亮以語錄事尚書徐羨之【語牛倨翻錄事尚書當作録尚書事】羨之曰黄散以下悉以委蔡吾徒不復措懷【黄散謂黄門侍郎及散騎常侍侍郎也復扶又翻】自此以上故宜共參同異廓曰我不能為徐干木署紙尾【為于偽翻】遂不拜干木羨之小字也選案黄紙錄尚書與吏部尚書連名【選案選曹文案也洪邁日葉石林言制敕用黄紙始高宗時非也晉㳟帝時王韶之遷黄門侍郎凡諸詔黄皆其辭也則東晉時已用黄紙寫詔矣又宋明帝時吏部尚書禇淵就赭圻行選是役也皆先戰授位版檄不供由是有黄紙札則宋世就軍補官賞功又多用黄紙矣又徐羨之召蔡廓為吏部尚書廓曰我不能為徐干木署紙尾則是宋世以黄紙為案矣至齊世立左右丞書案之制曰白案則右丞書名在上左丞次書黄案則左丞上書右丞下書雖世遠莫知何者之為黄案何者之為白案所可知者其紙已分黄白二色決矣至東晉時閹人以紙包裹魚肉還家並是五省黄案然則文書之用黄紙其來已久高宗時凡謄寫詔制以下州縣始皆用黄紙耳槩言詔書用黄紙始於高宗不審也選須絹翻】故廓云然<br />
<br />
  沈約論曰蔡廓固辭銓衡恥為志屈豈不知選錄同體義無偏斷乎【吏部典選録尚書兼録諸曹尚書事斷丁亂翻】良以主闇時難不欲居通塞之任【銓衡之任得其人則賢路通不得其人則賢路塞塞悉則翻】遠矣哉<br />
<br />
  庚申檀道濟軍於彭城魏叔孫建入臨淄所向城邑皆潰 【考異曰索虜傳云虜又遣楚兵將軍徐州刺史安平公涉歸幡能健越兵將軍青州刺史臨淄侯薛道千陳兵將軍淮州刺史夀張子張模所向城邑皆奔走本紀亦云安平公涉歸寇青州按後魏書無涉歸等姓名盖皆胡中舊名即叔孫建等也】竺夔聚民保東陽城其不入城者使各依據山險芟夷禾稼【芟所銜翻】魏軍至無所得食濟南太守垣苗帥衆依夔【垣苗棄歷城依夔濟子禮翻帥讀曰率下同】刁雍見魏主於鄴魏主曰叔孫建等入青州民皆藏避攻城不下彼素服卿威信【雍先聚兵河濟之間雍于容翻】今遣卿助之乃以雍為青州刺史給雍騎使行募兵以取青州魏兵濟河向青州者凡六萬騎【騎奇寄翻】刁雍募兵得五千人撫慰土民皆送租供軍 柔然寇魏邊二月戊辰魏築長城自赤城西至五原延袤二千餘里【袤音茂】備置戍卒以備柔然 丁丑太皇太后蕭氏殂 河西王蒙遜及吐谷渾王阿柴皆遣使入貢【使疏吏翻】庚辰詔以蒙遜為都督涼秦河沙四州諸軍事驃騎太將軍涼州牧河西王以阿柴為督塞表諸軍事安西將軍沙州刺史澆河公【吐谷渾據塞外沙漒之地故令督塞表諸軍事澆堅堯翻】 三月壬子葬孝懿皇后于興寜陵【興寜陵在晉陵丹徒縣諫壁里雩山】 魏奚斤公孫表等共攻虎牢魏主自鄴遣兵助之毛德祖於城内穴地入七丈分為六道出魏圍外募敢死之士四百人使參軍范道基等帥之從穴中出掩襲其後魏軍驚擾斬首數百級焚其攻具而還【還從宣翻又如字】魏兵雖退散隨復更合【復扶又翻下復嬰未復復作復戰同】攻之益急奚斤自虎牢將步騎三千攻潁川太守李元德等於許昌元德等敗走魏以潁川人庾龍為潁川太守戍許昌毛德祖出兵與公孫表大戰從朝至晡殺魏兵數百會奚斤自許昌還合撃德祖大破之亡甲士千餘人復嬰城自守魏主又遣萬餘人從白沙渡河屯濮陽南【濮陽對岸則頓丘之境白沙當在今澶州之界】朝議以項城去魏不遠【朝直遥翻下同】非輕軍所抗使劉粹召高道瑾還夀陽若沈叔狸已進亦宜且追粹奏虜攻虎牢未復南向若遽攝軍捨項城則淮西諸郡無所凴依沈叔狸已頓肥口【肥口肥水入淮之口】又不宜遽退時李元德帥散卒二百至項劉粹使助高道瑾戍守請宥其奔敗之罪朝議並許之乙已魏主畋于韓陵山【魏郡鄴縣有韓陵山】遂如汲郡至枋頭初毛德祖在北【毛德祖本滎陽人武帝未取關洛德祖自北來歸】與公孫表有舊表有權略德祖患之乃與交通音問密遣人說奚斤云表與之連謀每答表書多所治定【此曹操間韓馬之智也說輸芮翻治直之翻】表以書示斤斤疑之以告魏主先是表與太史令王亮少同營署好輕侮亮亮奏表置軍虎牢東不得便地故令賊不時滅魏主素好術數以為然積前後忿使人夜就帳中縊殺之【先悉薦翻少詩照翻好呼到翻】乙卯魏主濟自靈昌津【靈昌津古延津也石勒襲劉曜塗出於此以河氷為神靈之助改曰靈昌津】遂如東郡陳留叔孫建將三萬騎逼東陽城城中文武纔一千五百人竺夔垣苗悉力固守時出奇兵擊魏破之魏步騎繞城列陳十餘里大治攻具夔作四重塹魏人填其三重為橦車以攻城【陳讀曰陣重直龍翻治直之 翻橦與撞同傳江翻擣也】夔遣人從地道中出以大麻絙挽之令折【絙居曾翻大索也又居鄧翻折而設翻】魏人復作長圍進攻逾急歷時浸久城轉墮壞【墮讀曰隳下隳其同】戰士多死傷餘衆困乏旦暮且陷檀道濟至彭城以司青二州並急而所領兵少【少詩沼翻】不足分赴青州道近竺夔兵弱乃與王仲德兼行先敕之甲子劉粹遣李元德襲許昌斬庾龍元德因留綏撫并上租糧【上時掌翻】魏主至盟津于栗磾造浮橋於冶阪津【郭緣生述征記曰踐土今冶阪城是水經註河陽縣故城在冶阪西北魏土地記云冶阪城舊名漢祖渡城險固南臨孟津在洛陽西北四十二里盟讀曰孟】乙丑魏主引兵北濟西如河内娥清周幾閭大肥徇地至湖陸高平民屯聚而射之【射而亦翻】清等盡攻破高平諸縣滅數千家虜掠萬餘口兖州刺史鄭順之戍湖陸以兵少不敢出魏主又遣并州刺史伊樓拔助奚斤攻虎牢【伊婁虜複姓樓與婁同】毛德祖隨方抗拒頗殺魏兵而將士稍零落夏四月丁卯魏主如成臯絶虎牢汲河之路【北史虎牢乏水城内懸綆汲河魏主令連艦上施轒輼絶其汲路】停三日自督衆攻城竟不能下遂如洛陽觀石經【石經後漢蔡邕所書者注詳見五十七卷漢靈帝熙平四年】遣使祀嵩高【使疏吏翻】叔孫建攻東陽墮其北城三十許步【墮讀曰隳】刁雍請速入建不許遂不克及聞檀道濟等將至雍又謂建曰賊畏官軍突騎以鎖連車為函陳【函陳方陳也陳讀曰陣】大峴已南處處狹隘車不得方軌雍請將所募兵五千據險以邀之破之必矣【將即亮翻】時天暑魏軍多疫建曰兵人疫病過半若相持不休兵自死盡何須復戰今全軍而返計之上也己巳道濟軍于臨朐【朐音劬 考異曰裴子野宋略依乙巳按長歷是月丁卯朔興乙巳必己巳也】壬申建等燒營及器械而遁道濟至東陽粮盡不能追竺夔以東陽城壞不可守移鎮不其城【不其縣前漢屬琅邪郡後漢屬東萊晉屬長廣郡如淳曰其音基賢曰不其故城在今萊州即墨縣西南】叔孫建自東陽趨滑臺【趨七喻翻】道濟分遣王仲德向尹卯道濟停軍湖陸仲德未至尹卯聞魏兵己遠還就道濟刁雍遂留鎮尹卯招集譙梁彭沛民五千餘家置二十七營以領之 蠻王梅安帥渠帥數十人入貢於魏【帥渠帥上讀曰率下所類翻】初諸蠻本居江淮之間其後種落滋蔓【種章勇翻蔓音萬】布於數州東連夀春西通巴蜀北接汝潁往往有之在魏世不甚為患及晉稍益繁昌漸為寇暴及劉石亂中原諸蠻無所忌憚漸復北徙伊闕以南滿於山谷矣【據史此諸蠻乃盤瓠之後也復扶又翻】 河西世子政德攻晉昌克之唐契及弟和甥李寶同奔伊吾【唐契以晉昌叛河西見武帝永初二年】招集遺民歸附者至二千餘家臣於柔然柔然以契為伊吾王秦王熾磐謂其羣臣曰今宋雖奄有江南夏人雄據關中皆不足與也獨魏主奕世英武賢能為用且䜟云恒代之北當有真人吾將舉國而事之【䜟楚譛翻】乃遣尚書郎莫者阿胡等入見於魏【見賢遍翻】貢黄金二百斤并陳伐夏方略 閏月丁未魏主如河内登太行至高都【高都縣自漢以來屬上黨郡劉昫曰唐澤州晉城縣漢高都縣地行戶剛翻】叔孫建自滑臺西就奚斤共攻虎牢虎牢被圍二百日【被皮義翻】無日不戰勁兵戰死殆盡而魏增兵轉多魏人毁其外城毛德祖於其内更築三重城以拒之魏人又毁其二重德祖唯保一城晝夜相拒將士眼皆生創【重直龍翻人夜不得睡則眼眊燥以手揩之則生創創初良翻下同】德祖撫之以恩終無離心時檀道濟軍湖陸劉粹軍項城沈叔貍軍高橋皆畏兵彊不敢進丁已魏人作地道以洩虎牢城中井井深四十文【深式禁翻】山勢峻峭不可得防城中人馬渴乏被創者不復出血重以饑疫【被皮義翻復扶又翻重直用翻】魏仍急攻之己未城陷將士欲扶德祖出走德祖曰我誓與此城俱斃義不使城亡而身存也魏主命將士得德祖者必生致之將軍代人豆代田執德祖以獻【豆姓也漢書有校尉豆如意】將佐在城中者皆為魏所虜唯參軍范道基將二百人突圍南還【將即亮翻】魏士卒疫死者亦什二三奚斤等悉定司兖豫諸郡縣置守宰以撫之【是時司州之地盡入於魏兖州之地自湖陸以南豫州之地自項城以南皆為宋守魏未能悉定諸郡縣也】魏主命周幾鎮河南河南人安之徐羨之傅亮謝晦以亡失境土上表自劾【劾戶槩翻又戶德翻】詔勿問 徐羨之兄子吳郡太守珮之頗豫政事與侍中王韶之程道惠中書舍人邢安泰潘盛結為黨友時謝晦久病不堪見客珮之等疑其詐疾有異圖乃稱羨之意以告傅亮欲令亮作詔誅之【亮時進中書監中書掌詔命】亮曰我等三人同受顧命豈可自相誅戮諸君果行此事亮當角巾步出掖門耳【宫門正南門日端門左右二門謂之左掖門右掖門】珮之等乃止 五月魏主還平城【考異曰後魏帝紀五月庚寅還次雁門庚寅車駕至自南廵必一誤今皆不取】 六月已亥<br />
<br />
  魏宜都文成王穆觀卒 丙辰魏主北巡至參合陂秋七月尊帝母張夫人為皇太后 魏主如三會屋侯泉【魏收地形志秀容郡肆盧縣治新會城真君七年併三會城屬焉】八月辛丑如馬邑觀灅源【灅力水翻】 柔然寇河西河西王蒙遜命世子政德擊之政德輕騎進戰【騎奇寄翻】為柔然所殺蒙遜立次子興為世子 九月乙亥魏主還宫召奚斤還平城留兵守虎牢使娥清周幾鎮枋頭以司馬楚之所將戶口置汝南南陽南頓新蔡四郡【晉惠帝分汝隂立新蔡郡分汝南立南頓郡魏未能有四郡之地僑置之耳】以益豫州 冬十月癸卯魏人廣西宫外垣周二十里【平城西宫也魏主珪天賜元年所築】 秃髪傉檀之死也【事見一百十六卷晉安帝義熙十年傉奴沃翻】河西王蒙遜遣人誘其故太子虎臺許以番禾西安二郡處之【誘音酉番音盤處昌呂翻】且借之兵使伐秦報其父讎復取故地虎臺隂許之事泄而止秦王熾磐之后虎臺之妹也熾磐待之如初后密與虎臺謀曰秦本我之仇讎雖以婚姻待之盖時宜耳先王之薨又非天命遺令不治者欲全濟子孫故也【治直之翻不治謂被鴆而不解也事見一百十六卷晉安帝義熙十年】為人子者豈可臣妾於仇讎而不思報復乎乃與武衛將軍越質洛城謀弑熾磐后妹為熾磐左夫人知其謀而告之熾磐殺后及虎臺等十餘人十一月魏周幾寇許昌許昌潰潁川太守李元德奔<br />
<br />
  項戊辰魏人圍汝陽汝陽太守王公度亦奔項【沈約曰晉太康地志王隱地道無汝陽郡應是江左分汝南立汝陽漢舊縣属汝南郡】劉粹遣其將姚聳夫等將兵助守項城【將即亮翻】魏人夷許昌城毁鍾城以立封疆而還【鍾城在泰山界夷許昌以立豫州封疆毁鍾城以立兖州封疆也還從宣翻又如字】己巳魏太宗殂【年三十二】壬申世祖即位【世祖諱燾明元皇帝之長子也】<br />
<br />
  【蕭子顯曰燾字佛狸】大赦十二月庚子魏葬明元帝于金陵【此雲中之金陵據北史道武帝葬盛樂金陵蓋魏諸陵皆曰金陵杜佑曰後魏盛樂縣在雲中郡】廟號太宗魏主追尊其母杜貴嬪為密皇后【密諡也】自司徒長孫嵩以下普增爵位以襄城公盧魯元為中書監會稽公劉絜為尚書令【會工外翻】司衛監尉眷散騎侍郎劉庫仁等八人分典四部【司衛監盖魏所置以掌宿衛此又一劉庫仁非什翼犍所用之劉庫仁也尉音紆勿翻散悉亶翻騎奇寄翻四部東西南北四部也】眷古真之弟子也【尉古真見一百六卷晉孝武太元十年】以河内鎮將代人羅結為侍中外都大官【魏書官氏志内入諸姓叱羅氏為羅氏魏有外都大官内都大官將即亮翻】摠三十六曹事結時年一百七精爽不衰【杜預曰爽明也】魏主以其忠慤親任之使兼長秋卿監典後宫出入卧内【監工銜翻】年一百一十乃聽歸老朝廷每有大事遣騎訪焉【騎奇寄翻】又十年乃卒左光禄大夫崔浩研精經術練習制度【魏晉以來左右光禄大夫在光禄大夫上假金章紫綬研精者窮其精力】凡朝廷禮儀軍國書詔無不關掌浩不好老莊之書曰此矯誣之說不近人情【託聖賢以神其說謂之矯聖賢無是事寓言而加詆謂之誣好呼到翻近其靳翻】老聃習禮仲尼所師【史記及大戴記皆云仲尼問禮於老聃聃他甘翻】豈肯為敗法之書以亂先王之治乎【敗補邁翻治直吏翻】尤不信佛法曰何為事此胡神及世祖即位左右多毁之帝不得已命浩以公歸第然素知其賢每有疑議輒召問之浩纎妍潔白如美婦人【纎細也妍美好也】常自謂才比張良而稽古過之既歸第因修服食養性之術初嵩山道士寇謙之讃之弟也修張道陵之術自言嘗遇老子降命謙之繼道陵為天師【張道陵後漢人修五斗米道俗所謂天師也】授以辟穀輕身之術及科戒二十卷【今道家科戒盖始於此】使之清整道教又遇神人李譜文【譜博古翻】云老子之玄孫也授以圖籙真經六十餘卷使之輔佐北方太平真君出天宫靜輪之法其中數篇李君之手筆也謙之奉其書獻於魏主朝野多未之信【朝直遙翻】崔浩獨師事之從受其術且上書贊明其事曰臣聞聖王受命必有天應河圖洛書皆寄言於蟲獸之文【河出圖伏羲象以畫八卦洛出書禹得之以叙九疇故曰龍圖授羲龜書畀姒又尚書中候曰堯沈璧於洛玄龜負書背中赤文朱字止于壇畔舜禮壇于河畔黄龍負卷舒圖出于水】未若今日人神接對手筆粲然辭旨深妙自古無比豈可以世俗常慮而忽上靈之命臣竊懼之帝欣然使謁者奉玉帛牲牢祭嵩嶽迎致謙之弟子在山中者以崇奉天師顯揚新法宣布天下起天師道場於平城之東南重壇五層【水經註濕水南逕平城之東水左有大道壇寇謙之所建也濕水即灅水】給道士百二十人衣食每月設厨會數千人臣光曰老莊之書大指欲同死生輕去就而為神僊者服餌修鍊以求輕舉鍊草石為金銀【谷永說漢成帝曰諸言世有仙人服食不終之藥遙興輕舉登遐倒景覽觀縣圃浮遊蓬萊黄冶變化皆姦人惑衆挾左道懷詐偽以欺罔世主服餌修鍊以求輕舉即谷永所謂服食不終之藥遙興輕舉者也鍊草石以為金銀即谷永所謂黄冶變化者也】其為術正相戾矣是以劉歆七略叙道家為諸子神仙為方技【以其相戾故七略不得合為一】其後復有符水禁呪之術【符水禁呪即張道陵之術】至謙之遂合而為一至今循之其訛甚矣崔浩不喜佛老之書而信謙之之言其故何哉【喜許記翻】昔臧文仲祀爰居孔子以為不智【海鳥爰居避風止于魯東門之外臧文仲使國人祀之孔子以為臧文仲不智者三祀爰居其一也】如謙之者其為爰居亦大矣詩三百一言以蔽之曰思無邪君子之於擇術可不慎哉<br />
<br />
  資治通鑑卷一百十九  <br>
   </div> 

<script src="/search/ajaxskft.js"> </script>
 <div class="clear"></div>
<br>
<br>
 <!-- a.d-->

 <!--
<div class="info_share">
</div> 
-->
 <!--info_share--></div>   <!-- end info_content-->
  </div> <!-- end l-->

<div class="r">   <!--r-->



<div class="sidebar"  style="margin-bottom:2px;">

 
<div class="sidebar_title">工具类大全</div>
<div class="sidebar_info">
<strong><a href="http://www.guoxuedashi.com/lsditu/" target="_blank">历史地图</a></strong>  
<a href="http://www.880114.com/" target="_blank">英语宝典</a>  
<a href="http://www.guoxuedashi.com/13jing/" target="_blank">十三经检索</a> 
<br><strong><a href="http://www.guoxuedashi.com/gjtsjc/" target="_blank">古今图书集成</a></strong> 
<a href="http://www.guoxuedashi.com/duilian/" target="_blank">对联大全</a> <strong><a href="http://www.guoxuedashi.com/xiangxingzi/" target="_blank">象形文字典</a></strong> 

<br><a href="http://www.guoxuedashi.com/zixing/yanbian/">字形演变</a>  <strong><a href="http://www.guoxuemi.com/hafo/" target="_blank">哈佛燕京中文善本特藏</a></strong>
<br><strong><a href="http://www.guoxuedashi.com/csfz/" target="_blank">丛书&方志检索器</a></strong> <a href="http://www.guoxuedashi.com/yqjyy/" target="_blank">一切经音义</a>  

<br><strong><a href="http://www.guoxuedashi.com/jiapu/" target="_blank">家谱族谱查询</a></strong>  <strong><a href="http://shufa.guoxuedashi.com/sfzitie/" target="_blank">书法字帖欣赏</a></strong> 
<br>

</div>
</div>


<div class="sidebar" style="margin-bottom:0px;">

<font style="font-size:22px;line-height:32px">QQ交流群9:489193090</font>


<div class="sidebar_title">手机APP 扫描或点击</div>
<div class="sidebar_info">
<table>
<tr>
	<td width=160><a href="http://m.guoxuedashi.com/app/" target="_blank"><img src="/img/gxds-sj.png" width="140"  border="0" alt="国学大师手机版"></a></td>
	<td>
<a href="http://www.guoxuedashi.com/download/" target="_blank">app软件下载专区</a><br>
<a href="http://www.guoxuedashi.com/download/gxds.php" target="_blank">《国学大师》下载</a><br>
<a href="http://www.guoxuedashi.com/download/kxzd.php" target="_blank">《汉字宝典》下载</a><br>
<a href="http://www.guoxuedashi.com/download/scqbd.php" target="_blank">《诗词曲宝典》下载</a><br>
<a href="http://www.guoxuedashi.com/SiKuQuanShu/skqs.php" target="_blank">《四库全书》下载</a><br>
</td>
</tr>
</table>

</div>
</div>


<div class="sidebar2">
<center>


</center>
</div>

<div class="sidebar"  style="margin-bottom:2px;">
<div class="sidebar_title">网站使用教程</div>
<div class="sidebar_info">
<a href="http://www.guoxuedashi.com/help/gjsearch.php" target="_blank">如何在国学大师网下载古籍?</a><br>
<a href="http://www.guoxuedashi.com/zidian/bujian/bjjc.php" target="_blank">如何使用部件查字法快速查字?</a><br>
<a href="http://www.guoxuedashi.com/search/sjc.php" target="_blank">如何在指定的书籍中全文检索?</a><br>
<a href="http://www.guoxuedashi.com/search/skjc.php" target="_blank">如何找到一句话在《四库全书》哪一页?</a><br>
</div>
</div>


<div class="sidebar">
<div class="sidebar_title">热门书籍</div>
<div class="sidebar_info">
<a href="/so.php?sokey=%E8%B5%84%E6%B2%BB%E9%80%9A%E9%89%B4&kt=1">资治通鉴</a> <a href="/24shi/"><strong>二十四史</strong></a>&nbsp; <a href="/a2694/">野史</a>&nbsp; <a href="/SiKuQuanShu/"><strong>四库全书</strong></a>&nbsp;<a href="http://www.guoxuedashi.com/SiKuQuanShu/fanti/">繁体</a>
<br><a href="/so.php?sokey=%E7%BA%A2%E6%A5%BC%E6%A2%A6&kt=1">红楼梦</a> <a href="/a/1858x/">三国演义</a> <a href="/a/1038k/">水浒传</a> <a href="/a/1046t/">西游记</a> <a href="/a/1914o/">封神演义</a>
<br>
<a href="http://www.guoxuedashi.com/so.php?sokeygx=%E4%B8%87%E6%9C%89%E6%96%87%E5%BA%93&submit=&kt=1">万有文库</a> <a href="/a/780t/">古文观止</a> <a href="/a/1024l/">文心雕龙</a> <a href="/a/1704n/">全唐诗</a> <a href="/a/1705h/">全宋词</a>
<br><a href="http://www.guoxuedashi.com/so.php?sokeygx=%E7%99%BE%E8%A1%B2%E6%9C%AC%E4%BA%8C%E5%8D%81%E5%9B%9B%E5%8F%B2&submit=&kt=1"><strong>百衲本二十四史</strong></a>  <a href="http://www.guoxuedashi.com/so.php?sokeygx=%E5%8F%A4%E4%BB%8A%E5%9B%BE%E4%B9%A6%E9%9B%86%E6%88%90&submit=&kt=1"><strong>古今图书集成</strong></a>
<br>

<a href="http://www.guoxuedashi.com/so.php?sokeygx=%E4%B8%9B%E4%B9%A6%E9%9B%86%E6%88%90&submit=&kt=1">丛书集成</a> 
<a href="http://www.guoxuedashi.com/so.php?sokeygx=%E5%9B%9B%E9%83%A8%E4%B8%9B%E5%88%8A&submit=&kt=1"><strong>四部丛刊</strong></a>  
<a href="http://www.guoxuedashi.com/so.php?sokeygx=%E8%AF%B4%E6%96%87%E8%A7%A3%E5%AD%97&submit=&kt=1">說文解字</a> <a href="http://www.guoxuedashi.com/so.php?sokeygx=%E5%85%A8%E4%B8%8A%E5%8F%A4&submit=&kt=1">三国六朝文</a>
<br><a href="http://www.guoxuedashi.com/so.php?sokeytm=%E6%97%A5%E6%9C%AC%E5%86%85%E9%98%81%E6%96%87%E5%BA%93&submit=&kt=1"><strong>日本内阁文库</strong></a> <a href="http://www.guoxuedashi.com/so.php?sokeytm=%E5%9B%BD%E5%9B%BE%E6%96%B9%E5%BF%97%E5%90%88%E9%9B%86&ka=100&submit=">国图方志合集</a> <a href="http://www.guoxuedashi.com/so.php?sokeytm=%E5%90%84%E5%9C%B0%E6%96%B9%E5%BF%97&submit=&kt=1"><strong>各地方志</strong></a>

</div>
</div>


<div class="sidebar2">
<center>

</center>
</div>
<div class="sidebar greenbar">
<div class="sidebar_title green">四库全书</div>
<div class="sidebar_info">

《四库全书》是中国古代最大的丛书,编撰于乾隆年间,由纪昀等360多位高官、学者编撰,3800多人抄写,费时十三年编成。丛书分经、史、子、集四部,故名四库。共有3500多种书,7.9万卷,3.6万册,约8亿字,基本上囊括了古代所有图书,故称“全书”。<a href="http://www.guoxuedashi.com/SiKuQuanShu/">详细>>
</a>

</div> 
</div>

</div>  <!--end r-->

</div>
<!-- 内容区END --> 

<!-- 页脚开始 -->
<div class="shh">

</div>

<div class="w1180" style="margin-top:8px;">
<center><script src="http://www.guoxuedashi.com/img/plus.php?id=3"></script></center>
</div>
<div class="w1180 foot">
<a href="/b/thanks.php">特别致谢</a> | <a href="javascript:window.external.AddFavorite(document.location.href,document.title);">收藏本站</a> | <a href="#">欢迎投稿</a> | <a href="http://www.guoxuedashi.com/forum/">意见建议</a> | <a href="http://www.guoxuemi.com/">国学迷</a> | <a href="http://www.shuowen.net/">说文网</a><script language="javascript" type="text/javascript" src="https://js.users.51.la/17753172.js"></script><br />
  Copyright &copy; 国学大师 古典图书集成 All Rights Reserved.<br>
  
  <span style="font-size:14px">免责声明:本站非营利性站点,以方便网友为主,仅供学习研究。<br>内容由热心网友提供和网上收集,不保留版权。若侵犯了您的权益,来信即刪。scp168@qq.com</span>
  <br />
ICP证:<a href="http://www.beian.miit.gov.cn/" target="_blank">鲁ICP备19060063号</a></div>
<!-- 页脚END --> 
<script src="http://www.guoxuedashi.com/img/plus.php?id=22"></script>
<script src="http://www.guoxuedashi.com/img/tongji.js"></script>

</body>
</html>
