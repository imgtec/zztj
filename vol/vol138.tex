資治通鑑卷一百三十八 宋 司馬光 撰

胡三省 音註

齊紀四|{
	昭陽作噩一年}


世祖武皇帝下

永明十一年春正月以驃騎大將軍王敬則為司空|{
	驃匹妙翻騎奇寄翻}
鎮軍大將軍陳顯逹為江州刺史顯逹自以門寒位重|{
	陳顯逹南彭城人起於卒伍}
每遷官常有愧懼之色戒其子勿以富貴陵人而諸子多事豪侈顯逹聞之不悦|{
	以陳顯逹之居寵思畏終不能自免於猜暴之朝至於稱兵而死豈非繫於所遇之時哉}
子休尚為郢府主簿過九江|{
	自建康至郢府先過九江}
顯逹曰麈尾蠅拂|{
	麈腫庾翻麈麋屬尾能生風辟蠅蜹陸佃埤雅曰麈似鹿而大其尾辟塵以置舊帛中能令歲久紅色不黦又以拂氊令氊不蠧名苑曰鹿之大者曰麈羣鹿隨之皆視麈所往麈尾所轉為凖於文主鹿為麈古之談者揮焉良為是也}
是王謝家物汝不須捉此|{
	言不須以風流自標置也捉執也}
即取於前燒之 初上於石頭造露車三千乘欲步道取彭城魏人知之劉昶數泣訴于魏主乞處邊戌招集遺民以雪私耻|{
	以蕭氏簒宋夷滅劉氏故也數所角翻處昌呂翻}
魏主大會公卿於經武殿|{
	魏書帝紀太和十二年起經武殿}
以議南伐於淮泗間大積馬芻上聞之以右衛將軍崔慧景為豫州刺史以備之|{
	為下魏入寇張本}
魏遣員外散騎侍郎邢巒等來聘|{
	散悉亶翻騎奇寄翻}
巒頴之

孫也|{
	穎曹魏大常邢貞之後邢穎見一百二十二卷宋文帝元嘉八年}
丙子文惠太子長懋卒太子風韻甚和上晩年好遊宴|{
	好呼到翻}
尚書曹事分送太子省之由是威加内外|{
	省悉景翻}
太子性奢靡治堂殿園囿過於上宫|{
	治直之翻}
費以千萬計恐上望見之乃傍門列修竹凡諸服玩率多僭侈啓於東田起小苑使東宫將吏更番築役|{
	將即亮翻更工衡翻更番分番更作也}
營城包巷彌亘華遠|{
	言其彌極華麗而延亘又遼遠也}
上性雖嚴多布耳目太子所為人莫敢以聞上嘗過太子東田見其壯麗大怒收監作主帥太子皆藏之由是大被誚責|{
	監工衘翻帥所類翻被皮義翻誚才笑翻}
又使嬖人徐文景造輦及乘輿御物|{
	嬖卑義翻又博計翻乘繩證翻}
上嘗幸東宫怱怱不暇藏輦|{
	怱怱者急遽之意}
文景乃以佛像内輦中故上不疑文景父陶仁謂文景曰我正當掃墓待喪耳|{
	掃墓謂掃除墓地也}
仍移家避之後文景竟賜死陶仁遂不哭及太子卒上履行東宫|{
	行下孟翻}
見其服玩大怒敕有司隨事毁除以竟陵王子良與太子善而不啓聞并責之太子素惡西昌侯鸞嘗謂子良曰我意中殊不喜此人不解其故|{
	惡烏路翻喜許記翻解戶買翻曉也}
當由其福薄故也子良為之救解|{
	為于偽翻}
及鸞得政太子子孫無遺焉|{
	西昌侯夷滅太子子孫事見後按鸞翦除高武諸子及太子子孫以成簒事文惠雖不惡之其子孫亦不能免也觀隆昌建武時事君子謂文惠知所惡矣}
二月魏主始耕藉田於平城南|{
	魏起於北荒未嘗講古者天子親耕之禮今孝文始行之藉在亦翻}
雍州刺史王奐惡寜蠻長史劉興祖收繫獄|{
	雍於用翻惡烏路翻蕭子顯齊志寜蠻府屬雍州别領西新安義寜南襄北建武蔡陽永安安定懷化武寜新陽義安高安左義陽南襄城廣昌東襄城北襄城懷安北弘農西弘農析陽北義陽漢廣中襄城等蠻郡}
誣其搆扇山蠻欲為亂敕送興祖下建康|{
	自襄陽順流東至建康故曰下}
奐於獄中殺之詐云自經上大怒遣中書舍人呂文顯直閤將軍曹道剛將齋仗五百人收奐|{
	齋仗齋庫精仗以給禁衛勇力之士將即亮翻}
敕鎮西司馬曹虎從江陵步道會襄陽奐子彪素凶險奐不能制長史殷叡奐之壻也謂奐曰曹呂來既不見真敕恐為奸變正宜錄取|{
	錄收也攝也}
馳啓聞耳奐納之 |{
	考異曰南史奐子彪議閉門拒命叡諫曰今開門白服接臺使不過隳官免爵耳彪堅執不囘叡又請遣典籖間道送啓奐從之典籖出城為文顯所執叡曰忠不背國勇不逃死勸奐仰藥叡與彪同誅今從齊書}
彪輒發州兵千餘人開庫配甲仗出南堂陳兵閉門拒守奐門生鄭羽叩頭啓奐乞出城迎臺使|{
	使疏吏翻}
奐曰我不作賊欲先遣啓自申正恐曹呂等小人相陵藉|{
	陵者侮之而出其上藉者蹈之使薦於下藉慈夜翻}
故且閉門自守耳彪遂出與虎軍戰兵敗走歸三月乙亥司馬黄瑶起寜蠻長史河東裴叔業於城内起兵攻奐斬之|{
	為後奐子肅食瑶起之肉張本}
執彪及弟爽弼殷叡皆伏誅彪兄融琛死於建康琛弟祕書丞肅獨得脱犇魏|{
	為王肅屢引魏兵入寇張本琛丑林翻考異曰南史奐弟份自拘請罪帝宥之肅屢引魏人至邊帝謂份曰比有北信不份曰肅近忘墳柏寜遠憶有臣按奐以三月死帝以七月殂是冬肅始見魏主於鄴南史誤也齊書無此語}
夏四月甲午立南郡王昭業為皇太孫東宫文武悉改為太孫官屬|{
	東宫官屬文則太傅少傅詹事率更令家令僕門大夫中庶子中舍人庶子洗馬舍人武則左右衛率翊軍步兵屯騎三校尉旅賁中郎將左右積弩將軍殿中將軍員外殿中將軍常從虎賁督}
以太子妃琅邪王氏為皇太孫太妃南郡王妃何氏為皇太孫妃妃戢之女也|{
	何戢見一百三十五卷高帝建元二年戢則立翻又疾立翻}
魏太尉丕等請建中宫戊戌立皇后馮氏后熙之女也|{
	為後馮后以讒廢張本}
魏主以白虎通云|{
	漢章帝集諸儒於白虎觀議五經同異作白虎通}
王者不臣妻之父母下詔令太師上書不稱臣入朝不拜|{
	朝直遥翻}
熙固辭 光城蠻帥征虜將軍田益宗帥部落四千餘戶叛降於魏|{
	沈約曰光城郡疑大明中分弋陽所立五代史志曰光州光山縣舊置光城郡蠻帥所類翻宗帥讀曰率降戶江翻}
五月壬戌魏主宴四廟子孫於宣文堂親與之齒用家人禮|{
	四廟子孫謂世祖恭宗高宗顯祖之子孫也太和十二年起宣文堂經武殿用家人禮者畧君臣之敬而序長幼之齒}
甲子魏主臨朝堂|{
	朝直遥翻}
引公卿以下决疑政錄囚徒帝謂司空穆亮曰自今朝廷政事日中以前卿等先自論議日中以後朕與卿等共决之 丙子以宜都王鏗為南豫州刺史|{
	鏗丘耕翻}
先是廬陵王子卿為南豫州刺史|{
	先悉薦翻}
之鎮道中戲部伍為水軍上聞之大怒殺其典籖以鏗代之子卿還第上終身不與相見 襄陽蠻酋雷婆思等帥戶千餘求内徙於魏魏人處之沔北|{
	是時沔北之地猶為齊境雷婆思等蓋居沔南徙處沔北則稍近魏境耳酋慈由翻帥讀曰率處昌呂翻}
魏主以平城地寒六月雨雪|{
	極隂之地盛夏雨雪雨王遇翻自上而下曰雨}
風沙常起|{
	風沙大風揚沙也}
將遷都洛陽恐羣臣不從乃議大舉伐齊欲以脅衆齋於明堂左个|{
	鄭玄曰明堂左个大寢南堂東偏也个古賀翻}
使太常卿王諶筮之遇革帝曰湯武革命應乎天而順乎人|{
	此革卦之彖辭也諶氐壬翻}
吉孰大焉羣臣莫敢言尚書任城王澄曰陛下奕葉重光帝有中土|{
	任音壬重直龍翻}
今出師以征未服而得湯武革命之象未為全吉也帝厲聲曰繇云大人虎變何言不吉|{
	繇直又翻大人虎變革九五爻辭九五君位也故引以難澄}
澄曰陛下龍興已久何得今乃虎變帝作色曰社稷我之社稷任城欲沮衆邪澄曰社稷雖為陛下之有臣為社稷之臣安可知危而不言帝久之乃解曰各言其志夫亦何傷既還宫|{
	自明堂左个還宫}
召澄入見逆謂之曰嚮者革卦今當更與卿論之明堂之忿恐人人競言沮我大計故以聲色怖文武耳想識朕意|{
	見賢遍翻沮在呂翻怖普布翻}
因屏人謂澄曰今日之舉誠為不易|{
	屛必郢翻易以䜴翻}
但國家興自朔土徙居平城此乃用武之地非可文治今將移風易俗其道誠難朕欲因此遷宅中原卿以為何如|{
	魏主始與任城王澄言其情}
澄曰陛下欲卜宅中土以經畧四海此周漢所以興隆也|{
	比之周成康漢光明也}
帝曰北人習常戀故必將驚擾奈何|{
	後穆泰等之謀卒如帝所慮}
澄曰非常之事故非常人之所及陛下斷自聖心|{
	斷丁亂翻}
彼亦何所能為帝曰任城吾之子房也|{
	張良贊漢高帝遷都長安故以為比}
六月丙戌命作河橋欲以濟師祕書監盧淵上表以為前代承平之主未嘗親御六軍决勝行陳之間|{
	行戶剛翻陳讀曰陣}
豈非勝之不足為武不勝有虧威望乎昔魏武以弊卒一萬破袁紹|{
	事見六十三卷漢獻帝建安五年}
謝玄以步兵三千摧苻秦|{
	事見一百五卷晉孝武帝太元八年}
勝負之變决於須臾不在衆寡也詔報曰承平之主所以不親戎事或以同軌無敵或以懦劣偷安今謂之同軌則未然|{
	天下混一則車同軌書同文}
比之懦劣則可耻必若王者不當親戎則先王制革輅何所施也|{
	周制五輅革輅龍勒條纓五就建大白以即戎鄭氏注革輅輓之以革而漆之無他飾條讀為絛}
魏武之勝蓋由仗順苻氏之敗亦由失政豈寡必能勝衆弱必能制彊邪丁未魏主講武命尚書李冲典武選|{
	時欲用兵命冲典武選銓擇才勇之士選須絹翻}
建康僧法智與徐州民周盤龍等作亂|{
	此又一周盤龍非周奉叔之父}
夜攻徐州城入之刺史王玄邈討誅之|{
	徐州城即鍾離城}
秋七月癸丑魏立皇子恂為太子|{
	為魏主後廢恂張本}
戊子魏中外戒嚴露布及移書稱當南伐|{
	用兵尚神密魏主今露其事以布告四方故亦曰露布移書則移書於齊境也}
詔揚徐州民丁廣設召募以備之中書郎王融自恃人地|{
	王融有俊才故以人身自高且王弘曾孫故以門地自高}
三十内望為公輔嘗夜直省中撫案歎曰為爾寂寂|{
	爾如此也寂寂言冷寞也}
鄧禹笑人|{
	鄧禹年二十四為漢司徒融年已過之故云然}
行逢朱雀桁開喧湫不得進|{
	朱雀桁當建康朱雀門跨秦淮南北岸以渡行人大路所由也桁開則行者填咽湫子小翻隘也經典釋文曰湫徐音秋又在酒翻}
搥車壁歎曰車前無八騶何得稱丈夫|{
	搥傳追翻車前有油壁自晉以來諸公諸從公車前紿騶八人騶側鳩翻}
竟陵王子良愛其文學特親厚之融見上有北伐之志數上書奬勸|{
	奬者推助以成其事數所角翻}
因大習騎射|{
	騎奇寄翻}
及魏將入寇子良於東府募兵板融寜朔將軍|{
	宋泰始初南攻義嘉軍功者衆板不能供始用黄紙今板授融蓋重於黄紙也或曰未經勑用者謂之板授}
使典其事融傾意招納得江西傖楚數百人並有幹用|{
	傖助庚翻}
會上不豫詔子良甲仗入延昌殿侍醫藥子良以蕭衍范雲等皆為帳内軍主戊辰遣江州刺史陳顯逹鎮樊城上慮朝野憂遑|{
	遑急也遽也}
力疾召樂府奏正聲伎|{
	江左以清商為正聲伎伎渠綺翻}
子良日夜在内太孫間日參承|{
	間日隔一日也間古莧翻參候也承奉也}
戊寅上疾亟蹔絶|{
	氣暫絶而不息也}
太孫未入内外惶懼百皆已變服王融欲矯詔立子良詔草已立蕭衍謂范雲曰道路籍籍皆云將有非常之舉王元長非濟世才|{
	王融字元長}
視其敗也雲曰憂國家者惟有王中書耳衍曰憂國欲為周召邪欲為豎刁邪|{
	召讀曰邵按左傳齊桓公既立子昭為太子易牙有寵於衛姫衛姫生無虧易牙因豎刁以薦羞於桓公遂有寵公許之立無虧公卒易牙入與豎刁殺羣吏而立無虧昭奔宋宋襄公伐齊殺無虧而立昭是為孝公}
雲不敢荅及太孫來王融戎服絳衫於中書省閤口斷東宫仗不得進|{
	斷音短}
頃之上復蘇|{
	復扶又翻}
問太孫所在因召東宫器甲皆入以朝事委尚書左僕射西昌侯鸞|{
	朝直遥翻}
俄而上殂|{
	年五十四}
融處分以子良兵禁諸門|{
	處昌呂翻分扶問翻}
鸞聞之急馳至雲龍門不得進鸞曰有敕召我排之而入奉太孫登殿命左右扶出子良指麾部署音響如鐘殿中無不從命融知不遂釋服還省|{
	釋戎服還中書省也}
歎曰公誤我由是鬰林王深怨之|{
	太孫即位尋見廢弑史以追廢之號書之為後殺王融張本}
遺詔曰太孫進德日茂社稷有寄子良善相毗輔思弘治道|{
	治直吏翻}
内外衆事無大小悉與鸞參懷共下意|{
	參豫也懷思也命鸞參豫其事而詳思其可否也共下意者令降心相從以濟國事也}
尚書中事軄務根本悉委右僕射王晏吏部尚書徐孝嗣軍旅之畧委王敬則陳顯逹王廣之王玄邈沈文季張瓌薛淵等|{
	自此以上皆遺詔之辭瓌古回翻}
世祖留心政事務總大體嚴明有斷郡縣久於其職長吏犯法封刃行誅故永明之世百姓豐樂賊盜屏息然頗好遊宴華靡之事常言恨之未能頓遣|{
	遣袪也逐也言未能袪逐遊宴之失也自此以上史述帝平生之大略斷丁亂翻長知兩翻樂音洛屏必郢翻好呼到翻}
鬱林王之未立也衆皆疑立子良口語喧騰武陵王曅於衆中大言曰若立長則應在我|{
	世祖諸弟存者曅為長長知兩翻}
立嫡則應在太孫|{
	鬰林王諱昭業字元尚小字法身文惠太子長子也以世嫡立為皇太孫}
由是帝深憑賴之|{
	太孫已即位故書帝}
直閤周奉叔曹道剛素為帝心膂並使監殿中直衛少日復以道剛為黄門郎|{
	監古衘翻少詩沼翻復扶又翻為西昌侯鸞欲弑帝先除周奉叔曹道剛張本}
初西昌侯鸞為太祖所愛|{
	事見一百三十五卷高帝建元二年}
鸞性儉素車服儀從同於素士所居官名為嚴能故世祖亦重之|{
	鸞初為安吉令有嚴能之名王子侯舊乘纒帷車鸞獨乘下帷車儀從如素士從才用翻}
世祖遺詔使竟陵王子良輔政鸞知尚書事子良素仁厚不樂世務|{
	樂音洛}
乃更推鸞故遺詔云事無大小悉與鸞參懷子良之志也|{
	史言子良無奪嫡之志}
帝少養於子良妃袁氏|{
	少詩沼翻}
慈愛甚著及王融有謀|{
	史言奪適之謀出于王融}
遂深忌子良大行出太極殿子良居中書省帝使虎賁中郎將潘敞領二百人仗屯太極殿西階以防之|{
	中書省蓋在太極殿西故使屯於西階以防子良賁音奔將即亮翻}
既成服諸王皆出子良乞停至山陵不許|{
	乞停中書省俟梓宫出葬而後出也}
壬午稱遺詔以武陵王曅為衛將軍與征南大將軍陳顯逹並開府儀同三司尚書左僕射西昌侯鸞為尚書令太孫詹事沈文季為護軍|{
	史言遺詔本無此段除授當時稱遺詔行之}
癸未以竟陵王子良為太傅蠲除三調及衆逋|{
	三調謂調粟調帛及雜調也逋欠負也}
省御府及無用池田邸治|{
	治據蕭子顯齊書當作冶謂冶鑄之所也}
減關市征稅先是蠲原之詔多無事實督責如故|{
	所謂黄放白催也先悉荐翻}
是時西昌侯鸞知政恩信兩行衆皆悦之|{
	史為西昌侯鸞簒國張本}
魏山陽景桓公尉元卒|{
	尉紆勿翻}
魏主使錄尚書事廣陵王羽持節安撫六鎮發其突騎|{
	騎奇寄翻下同}
丁亥魏主辭永固陵己丑平城南伐步騎三千餘萬使太尉丕與廣陵王羽留守平城並加使持節|{
	晉制使持節得殺二千石以下杜佑曰留守周之君陳似其任也此後無聞漢和帝南廵祠園廟張禹以太尉兼衛留守晉惠帝幸長安僕射荀藩等與遺官在洛者為留臺承制行事其後安帝播遷劉裕亦置留臺後魏孝文帝南伐以太尉丕廣陵王羽留守京師留守之制因此}
羽曰太尉宜專節度臣止可為副魏主曰老者之智少者之决|{
	言老者經事多故智慮深遠少者氣盛故臨事有斷少詩沼翻}
汝無辭也以河南王幹為車騎大將軍都督關右諸軍事又以司空穆亮安南將軍盧淵平南將軍薛胤皆為幹副衆合七萬出子午谷|{
	欲攻梁益也}
胤辯之曾孫也|{
	薛辯見一百一十八卷晉安帝義熙十三年}
鬰林王性辨慧美容止善應對哀樂過人|{
	樂音洛}
世祖由是愛之而矯情飾詐隂懷鄙慝與左右羣小共衣食同卧起始為南郡王從竟陵王子良在西州|{
	帝少養於子良妃袁氏子良為揚州刺史故帝從在西州}
文惠太子每禁其起居節其用度王密就富人求錢無敢不與别作鑰鉤|{
	鉤所以啓鑰今謂之鑰匙}
夜開西州後閤與左右至諸營署中淫宴師史仁祖侍書胡天翼|{
	王國有師掌導之教訓侍書掌教之書輪}
相謂曰若言之二宫|{
	二宫謂上宫及東宫也}
則其事未易|{
	易以䜴翻}
若於營署為異人所敺|{
	敺手口翻}
及犬物所傷豈直罪止一身亦當盡室及禍年各七十餘生豈足吝邪數日間二人相繼自殺二宫不知也|{
	人莫知其子之惡其斯之謂歟}
所愛左右皆逆加官爵疏於黄紙使囊盛帶之|{
	盛時征翻}
許南面之日依此施行侍太子疾及居喪憂容號毀|{
	號戶刀翻}
見者嗚咽裁還私室即歡笑酣飲常令女巫楊氏禱祀速求天位及太子卒|{
	文惠太子卒於是年正月}
謂由楊氏之力倍加敬信既為太孫|{
	是年夏四月自南郡王為太孫}
世祖有疾又令楊氏禱祀時何妃猶在西州|{
	太孫居東宫何妃尚留西州}
世祖疾稍危太孫與何妃書紙中央作一大喜字而作三十六小喜字繞之侍世祖疾言淚下世祖以為必能負荷大業|{
	荷下可翻又讀如字}
謂曰五年中一委宰相汝弗措意五年外勿復委人|{
	復扶又翻}
若自作無成無所多恨臨終執其手曰若憶翁當好作|{
	作音佐韓愈方橋詩曰非閣復非船可居兼可過若欲問方橋方橋如此作注云白樂天皮日休詩皆自注曰音佐朱元晦曰今按廣韻作造也荀子肉腐出蟲魚枯生蠧貪利忘身禍烖乃作音將祚翻及廉范五袴之謡皆以為此音矣然讀為佐音者又將祚之訛也而世俗所用從人從故而切為將祚者又字之俗體也}
遂殂大斂始畢悉呼世祖諸伎備奏衆樂即位十餘日即收王融下廷尉|{
	斂力瞻翻伎渠綺翻下戶嫁翻}
使中丞孔稚珪奏融險躁輕狡招納不逞誹謗朝政|{
	朝直遥翻}
融求援於竟陵王子良子良憂懼不敢救遂於獄賜死時年二十七初融欲與東海徐勉相識每託人召之勉謂人曰王君名高望促|{
	言名雖高而輕躁人知其必及禍故望促}
難可輕衣|{
	類篇毗祭翻弊或從衣此云者義與弊同}
俄而融及禍勉由是知名太學生會稽魏凖以才學為融所賞|{
	會工外翻}
融欲立子良凖鼓成其事|{
	鼓以作氣言鼓作融氣以成其事}
太學生虞羲丘國賓竊相謂曰竟陵才弱王中書無斷|{
	斷丁亂翻}
敗在眼中矣及融誅召凖入舍人省詰問|{
	詰去吉翻}
惶懼而死舉體皆青時人以為膽破 壬寅魏主至肆州|{
	魏收志肆州治九原天賜二年為鎮真君七年置州領永安秀容雁門郡而永安郡定襄縣注云真君七年并雲中九原晉昌屬焉則知魏肆州蓋治定襄之九原也然此定襄亦非漢之定襄縣地蓋曹魏所置新昌郡之定襄縣其地在陘嶺之南古定襄在陘嶺之北隋志雁門郡後周置肆州隋改曰代州又有定襄郡開皇五年置雲州總管府此蓋因古定襄以名郡參考可知矣宋白曰後魏置肆州於九原非古九原漢末曹公所置定襄郡之九原縣也唐為秀容縣忻州定襄郡治焉後魏書云太平四年置肆州治秀容州領靈丘等八郡}
見道路民有跛眇者停駕慰勞|{
	勞力到翻}
給衣食終身|{
	此亦可謂惠而不知為政矣見者則給衣食目所不見者豈能徧給其衣食哉古之為政者孤獨廢疾者皆有以養之豈必待身親見而後養之也跛補火翻跛者一足偏短眇者一目偏盲眇亡沼翻}
大司馬安定王休執軍士為盜者三人以狥於軍將斬之魏主行軍遇之|{
	行下孟翻循行也}
命赦之休不可曰陛下親御六師將遠清江表今始行至此而小人已為攘盜不斬之何以禁奸帝曰誠如卿言然王者之體時有非常之澤三人罪雖應死而因緣遇朕雖違軍法可特赦之既而謂司徒馮誕曰大司馬執法嚴諸君不可不慎|{
	馮誕后戚既親且貴故語之以儆百司}
於是軍中肅然臣光曰人主之於其國譬猶一身視遠如視邇在境如在庭舉賢才以任百官修政事以利百姓則封域之内無不得其所矣是以先王黈纊塞耳前旒蔽明欲其廢耳目之近用推聰明於四遠也|{
	東方朔曰冕而前旒所以蔽明黈纊充耳所以塞聰如淳注曰黈音主苟翻謂以玉為填用黈纊懸之也師古曰如說非也黈黄色也纊綿也以黄綿為丸用組懸之垂兩耳邊示不外聽非玉瑱之懸也塞悉則翻}
彼廢疾者宜養當命有司均之於境内今獨施於道路之所遇則所遺者多矣其為仁也不亦微乎况赦罪人以橈有司之法|{
	橈奴教翻}
尤非人君之體也惜也孝文魏之賢君而猶有是乎

戊申魏主至并州并州刺史王襲治有聲跡|{
	治直吏翻}
境内安靜帝嘉之襲教民多立銘置道側虛稱其美帝聞而問之襲對不以實帝怒降襲號二等|{
	號者所領將軍號也}
九月壬子魏遣兼員外散騎常侍勃海高聦等來聘 丁巳魏主詔車駕所經傷民秋稼者畝給穀五斛 辛酉追尊文惠太子為文皇帝廟號世宗 世祖梓宫下渚|{
	渚在東府前秦淮之渚也}
帝於端門内奉辭輼輬車未出端門亟稱疾還内|{
	端門宮之正南門内大内也輼音温輬音凉}
裁入閤即於内奏胡伎鞞鐸之聲響震内外|{
	伎渠綺翻鞞頻迷翻}
丙寅葬武皇帝於景安陵廟號世祖|{
	景安陵亦在武進帝遺詔所命陵名也在休安陵東所卜第三處休安陵蓋帝祖宋大常樂子所葬高帝受禪尊為休安陵}
戊辰魏主濟河庚午至洛陽壬申詣故太學觀石經|{
	故太學漢魏所營者}
乙亥鄧至王像舒彭遣其子舊朝于魏|{
	朝直遥翻}
且請傳位於舊魏主許之 魏主自平城至洛陽霖雨不止丙子詔諸軍前丁丑帝戎服執鞭乘馬而出羣臣稽顙于馬前|{
	稽顙於前將諫南伐也稽音啓}
帝曰廟算已定大軍將進諸公更欲何云尚書李冲等曰今者之舉天下所不願唯陛下欲之臣不知陛下獨行竟何之也|{
	言違衆南伐無異獨行}
臣等有其意而無其辭敢以死請帝大怒曰吾方經營天下期于混壹而卿等儒生屢疑大計斧钺有常卿勿復言|{
	此亦所以怖羣臣而决遷都之計也復扶又翻}
策馬將出于是安定王休等並慇勤泣諫帝乃諭羣臣曰今者興不小動而無成何以示後朕世居幽朔欲南遷中土苟不南伐當遷都於此王公以為何如欲遷者左不欲者右南安王楨進曰成大功者不謀於衆|{
	引秦商鞅之言}
今陛下苟輟南伐之謀遷都洛邑此臣等之願蒼生之幸也羣臣皆呼萬歲時舊人雖不願内徙|{
	舊人謂與魏同起于北荒之子孫即所謂國人}
而憚於南伐無敢言者遂定遷都之計李冲言於上曰陛下將定鼎洛邑宗廟宫室非可馬上遊行以待之願陛下暫還代都俟羣臣經營畢功然後備文物鳴和鸞而臨之帝曰朕將巡省州|{
	省悉景翻}
至鄴小停春首即還未宜歸北|{
	不肯歸北蓋慮北人歸代復戀土重遷也}
乃遣任城王澄還平城諭留司百官以遷都之事曰今日真所謂革也|{
	謂前筮之遇革今之遷都真以革北方之俗易說卦曰革去故也}
王其勉之帝以羣臣意多異同謂衛尉卿鎮南將軍于烈曰卿意如何烈曰陛下聖略淵遠非愚淺所測若隱心而言|{
	隱度也度徒洛翻}
樂遷之與戀舊適中半耳|{
	樂音洛中竹仲翻}
帝曰卿既不唱異|{
	言不唱為異論也}
即是肯同深感不言之益使還鎮平城曰留臺庶政一以相委烈栗磾之孫也|{
	于栗磾事魏道武帝健將也磾丁奚翻}
先是北地民支酉聚衆數千起兵于長安城北石山|{
	北地郡魏孝文帝太和十一年置班州十四年改邠州按水經注石山當在長安城東北有敷谷敷水出焉北流注于渭先悉薦翻}
遣使告梁州刺史隂智伯|{
	欲邀結齊師以為應援使疏吏翻}
秦州民王廣亦起兵應之攻執魏刺史劉藻秦雍間七州民皆響震|{
	七州雍岐秦南秦涇邠華也雍於用翻}
衆至十萬各守堡壁以待齊救魏河南王幹引兵擊之幹兵大敗支酉進至咸陽北濁谷穆亮與戰又敗 |{
	考異曰齊書穆亮作繆老生今從魏書}
隂智伯遣軍主席德仁等將兵數千與相應接酉等進向長安盧淵薛胤等拒擊大破之降者數萬口|{
	降戶江翻}
淵唯誅首惡餘悉不問獲酉廣並斬之 冬十月戊寅朔魏主如金墉城徵穆亮|{
	徵穆亮於關右}
使與尚書李冲將作大匠董爾經營洛都|{
	董爾北史作董爵}
己卯如河南城乙酉如豫州|{
	自金墉西如河南又自河南東如豫州此豫州謂虎牢城也魏明元帝改虎牢置豫州獻文帝取懸弧又置豫州以虎牢為北豫州今主太和十九年罷北豫州置東中府}
癸巳舍于石濟乙未魏解嚴設壇于滑臺城東告行廟以遷都之意|{
	遷都之議既定停南伐之師故解嚴奉神主而行故有行廟}
大赦起滑臺宫任城王澄至平城衆始聞遷都莫不驚駭澄援引古今徐以曉之衆乃開伏|{
	開也伏厭伏也言北人安土重遷蔽於此說不肯降心以相從澄援引曉喻以其蒙莫不厭伏也}
澄還報於滑臺魏主喜曰非任城朕事不成 壬寅尊皇太孫太妃為皇太后|{
	即文惠太子妃王氏也}
立妃為皇后|{
	即何妃也}
癸卯魏主如鄴城王肅見魏主於鄴|{
	是年三月王肅奔魏今方得見}


|{
	魏主}
陳伐齊之策魏主與之言不覺促席移晷|{
	降人初至君臣情分甚為闊疎言有當心故促席近前以聽之不覺其分之踈也與之言而弗厭倦日為之移晷不覺其久也}
自是器遇日隆親舊貴臣莫能間也魏主或屏左右與肅語至夜分不罷|{
	間古莧翻屏必郢翻}
自謂君臣相得之晩尋除輔國將軍大將軍長史時魏主方議興禮樂變華風凡威儀文物多肅所定 乙巳魏主遣安定王休帥從官迎家於平城|{
	帥讀曰率從才用翻}
辛亥封皇弟昭文為新安王昭秀為臨海王昭粲為永嘉王 魏主築宫於鄴西十一月癸亥徙居之 御史中丞江淹劾奏前益州刺史劉悛梁州刺史隂智伯貨巨萬皆抵罪初悛罷廣司二州|{
	按齊書劉悛傳悛出督廣州世祖自尋陽東下遇悛舟於渚間是時齊未受禪也罷廣州計當在世祖居東宫時世祖即位悛自廣陵遷督司州徵入為長兼侍中悛七倫翻又丑緣翻}
傾貲以獻世祖家無留儲在益州作金浴盆餘物稱是及鬰林王即位悛所獻減少|{
	稱尺證翻少詩沼翻悛傳云悛作金浴盆等欲以獻世祖還都而世祖晏駕鬰林新立遂減其所獻}
帝怒收悛付廷尉欲殺之西昌侯鸞救之得免猶禁錮終身悛勔之子也|{
	劉勔死于桂陽之難}


資治通鑑卷一百三十八
