資治通鑑卷一百五十二 宋 司馬光 撰

胡三省 音註

梁紀八|{
	著雍涒灘一年}


高祖武皇帝八

大通二年春正月癸亥魏以北海王顥為驃騎大將軍開府儀同三司相州刺史|{
	驃匹妙翻騎奇寄翻相息亮翻}
魏北道行臺楊津守定州城居鮮于修禮杜洛周之間迭來攻圍津蓄薪糧治器械|{
	治直之翻}
隨機拒擊賊不能克津濳使人以鐵劵說賊黨賊黨有應津者遺津書曰賊所以圍城正為取北人耳|{
	說式芮翻遺于季翻為于偽翻}
城中北人宜盡殺之不然必為患津悉收北人内子城中而不殺衆無不感其仁及葛榮代修禮統衆|{
	榮得修禮之衆見上卷普通七年}
使人說津許以為司徒津斬其使固守三年|{
	普通七年春津守定州至是三年說式芮翻其使疏吏翻}
杜洛周圍之魏不能救津遣其子遁突圍出詣柔然頭兵可汗求救遁日夜泣請頭兵遣其從祖吐豆發帥精騎一萬南出前鋒至廣昌賊塞隘口|{
	廣昌縣自漢以來屬代郡自廣昌東南山南出倒馬關至中山上曲陽縣關山險隘實為深峭石嶝逶迤沿塗九曲可從刋入聲汗音寒從才用翻帥讀曰率下同塞悉則翻}
柔然遂還乙丑津長史李裔引賊入執津欲烹之既而捨之|{
	史言城無糧援雖善守者不能支久}
瀛州刺史元寧以城降洛周|{
	降戶江翻}
乙丑魏潘嬪生女胡太后詐言皇子|{
	為後胡后立女張本嬪毗賓翻}
丙寅大赦改元武泰 蕭寶寅圍馮翊未下長孫稚軍至恒農|{
	長知兩翻恒戶登翻}
行臺左丞楊侃謂稚曰昔魏武與韓遂馬超據潼關相拒遂超之才非魏武敵也然而勝負久不决者扼其險要故也|{
	事見六十六卷漢獻帝建安十六年}
今賊守禦已固雖魏武復生無以施其智勇|{
	復扶又翻}
不如北取蒲反|{
	此用前漢書地理志蒲反字}
度河而西入其腹心|{
	此亦魏武之故智也}
置兵死地則華州之圍不戰自解|{
	五代志馮翊郡後魏置華州華戶化翻}
潼關之守必内顧而走支節既解長安可坐取也若愚計可取願為明公前驅稚曰子之計則善矣然今薛修義圍河東薛鳳賢據安邑宗正珍孫守虞坂不得進|{
	水經注曰虞坂即左傳所謂顛軨在傳巖東北十餘里東西絶澗於中築以成道指南北之路謂之軨橋橋之東北有虞原上道東有虞城其城北對長坂二十餘里謂之虞坂戰國策曰昔騏驥駕鹽車上虞坂遷延不能進正此處也坂音反}
如何可往侃曰珍孫行陳一夫|{
	行戶剛翻陳讀曰陣}
因緣為將|{
	將即亮翻}
可為人使安能使人河東治在蒲反|{
	治謂治所也}
西逼河漘|{
	漘船倫翻水厓也上平坦而下水深曰漘}
封疆多在郡東修義驅帥士民西圍郡城其父母妻子皆留舊村一旦聞官軍來至皆有内顧之心必望風自潰矣稚乃使其子子彦與侃帥騎兵自恒農北渡據石錐壁|{
	五代志河東郡虞鄉縣有石錐山於此築壘壁也}
侃聲言今且停此以待步兵且觀民情向背命送降名者各自還村俟臺軍舉三烽當亦舉烽相應其無應烽者乃賊黨也當進擊屠之以所獲賞軍於是村民轉相告語|{
	背蒲妹翻語牛倨翻降戶江翻下同}
雖實未降者亦詐舉烽一宿之間火光遍數百里賊圍城者不測其故各自散歸修義亦逃還與鳳賢俱請降丙子稚克潼關遂入河東會有詔廢鹽池税|{
	魏朝盖謂弛鹽利以與民可以得民也}
稚上表以為鹽池天產之貨密邇京畿唯應寶而守之均贍以理今四方多虞府藏罄竭冀定擾攘|{
	冀定二州時為葛榮杜洛周攻圍}
常調之絹不復可收唯仰府庫|{
	藏徂浪翻調徒弔翻復扶又翻仰魚向翻}
有出無入略論鹽税一年之中凖絹而言不下三十萬匹乃是移冀定二州置於畿甸今若廢之事同再失|{
	前此宣武帝用甄琛之言廢鹽池税已為失計今又廢之是為再失}
臣前仰違嚴旨不先討關賊徑解河東者非緩長安而急蒲反一失鹽池三軍乏食天助大魏兹計不爽|{
	爽乖也}
昔高祖昇平之年無所乏少|{
	少詩沼翻}
猶創置鹽官而加典護非與物競利恐由利而亂俗也况今國用不足租徵六年之粟調折來歲之資|{
	調徒弔翻折之列翻}
此皆奪人私財事不獲己臣輒符同監將尉|{
	謂監鹽池之將尉也監工銜翻將即亮翻下同}
還帥所部依常收税|{
	帥讀曰率下同}
更聽後敕|{
	謂合罷與否更聽後番敕下也}
蕭寶寅遣其將侯終德擊毛遐會郭子恢等屢為魏軍所敗|{
	敗補邁翻下穆敗同}
終德因其勢挫還軍襲寶寅至白門|{
	長安城東出北來第三門曰青門意白門即西出南來第三門也}
寶寅始覺丁丑與終德戰敗擕其妻南陽公主及其少子帥麾下百餘騎自後門出奔万俟醜奴|{
	少詩照翻騎奇寄翻万莫北翻俟渠之翻}
醜奴以寶寅為太傅二月魏以長孫稚為車騎大將軍開府儀同三司雍州刺史尚書僕射西道行臺|{
	雍於用翻}
羣盜李洪攻燒鞏西闕口以東|{
	謂鞏縣以西伊闕口以東也鞏縣漢屬河南尹晉分屬滎陽郡}
南結諸蠻魏都督李神軌武衛將軍費穆討之穆敗洪于闕口南遂平之 葛榮擊杜洛周殺之併其衆 魏靈太后再臨朝以來|{
	再臨朝見一百五十卷普通六年朝直遥翻}
嬖倖用事|{
	嬖卑義翻又博計翻}
政事縱弛恩威不立盗賊蠭起封疆日蹙|{
	謂秦隴以西冀并以北皆為盗區淮汝沂泗之間皆為梁所侵也}
魏肅宗年浸長太后自以所為不謹恐左右聞之於帝凡帝所愛信者太后輒以事去之|{
	長知兩翻去羌呂翻}
務為壅蔽不使帝知外事通直散騎常侍昌黎谷士恢有寵於帝使領左右|{
	散悉亶翻騎奇寄翻}
太后屢諷之欲用為州士恢懷寵不願出外太后乃誣以罪而殺之有蜜多道人能胡語帝常置左右太后使人殺之於城南而懸賞購賊由是母子之間嫌隙日深是時車騎將軍儀同三司并肆汾廣恒雲六州討虜大都督爾朱榮|{
	廣當作唐魏收志孝昌中置唐州高歡建義改唐州曰晉州按爾朱榮時駐兵於晉陽}
兵勢彊盛魏朝憚之|{
	胡直遥翻}
高歡段榮尉景蔡儁先在杜洛周黨中|{
	高歡等歸杜洛周見一百五十卷普通六年}
欲圖洛周不果逃奔葛榮又亡歸爾朱榮劉貴先在爾朱榮所屢薦歡於榮榮見其憔悴未之奇也|{
	憔而遥翻悴秦醉翻}
歡從榮之馬廐廐有悍馬榮命歡翦之|{
	髦馬而鬄落之為翦悍侯旰翻}
歡不加羈絆而翦之|{
	馬絡首曰羈繫足曰絆}
竟不蹄齧起謂榮曰御惡人亦猶是矣榮奇其言坐歡於牀下屏左右訪以時事歡曰文公有馬十二谷|{
	榮畜牧蕃庶以谷量馬屏必郢翻}
色别為羣畜此竟何用也|{
	畜吁玉翻}
榮曰但言爾意歡曰今天子闇弱太后淫亂嬖孽擅命朝政不行|{
	孽魚列翻朝直遥翻}
以明公雄武乘時奮發討鄭儼徐紇之罪以清帝側霸業可舉鞭而成此賀六渾之意也|{
	高歡字賀六渾}
榮大悦語自日中至夜半乃出自是每參軍謀并州刺史元天穆孤之五世孫也|{
	孤拓跋鬱律第四子}
與榮善榮兄事之榮常與天穆及帳下都督賀拔岳密謀欲舉兵入洛内誅嬖倖外清羣盗二人皆勸成之榮上書以山東羣盗方熾|{
	熾尺志翻}
冀定覆没官軍屢敗請遣精騎三千東援相州|{
	騎奇寄翻相息亮翻}
太后疑之報以念生梟戮寶寅就擒醜奴請降關隴已定費穆大破羣蠻絳蜀漸平又北海王顥帥衆二萬出鎮相州不須出兵|{
	梟堅堯翻降戶江翻帥讀曰率下同}
榮復上書|{
	復扶又翻下帝復同}
以為賊勢雖衰官軍屢敗人情危怯恐實難用若不更思方略無以萬全臣愚以為蠕蠕主阿那瓌荷國厚恩|{
	蠕蠕之亂魏援立阿那瓌事見一百四十九卷普通元年至三年蠕人兖翻荷下可翻}
未應忘報宜遣發兵東趣下口以躡其背|{
	言阿那瓌荷魏保護之思雖叛歸塞北未應漠然忘報恩之心下口盖指飛狐口趣七喻翻}
北海之軍嚴加警備以當其前臣麾下雖少輒盡力命自井陘以北滏口以西分據險要攻其肘腋|{
	少詩沼翻陘音刑滏音釡腋音亦}
葛榮雖并洛周威恩未著人類差異形勢可分|{
	杜洛周柔玄鎮民葛榮鮮于修禮之黨本非同 類吞并為一及其新合亟加招討則形勢可分也}
遂勒兵召集義勇北捍馬邑東塞井陘徐紇說太后以鐵劵間榮左右榮聞而恨之魏肅宗亦惡儼紇等逼於太后不能去|{
	塞悉則翻說輸芮翻間古莧翻惡烏路翻去羌呂翻}
密詔榮舉兵内向欲以脅太后榮以高歡為前鋒行至上黨帝復以私詔止之儼紇恐禍及己隂與太后謀酖帝癸丑帝暴殂|{
	年十九}
甲寅太后立皇女為帝大赦既而下詔稱潘充華本實生女|{
	潘充華即前所謂潘嬪也生女見上正月乙丑}
故臨洮王寶暉世子釗體自高祖|{
	臨洮王寶暉高祖之孫}
宜膺大寶百官文武加二階宿衛加三階乙卯釗即位釗始生三歲太后欲久專政故貪其幼而立之爾朱榮聞之大怒謂元天穆曰主上晏駕春秋十九海内猶謂之幼君况今奉未言之兒以臨天下欲求治安其可得乎|{
	治直吏翻}
吾欲帥鐵騎赴哀山陵翦除姦佞更立長君|{
	更工衡翻長知兩翻}
何如天穆曰此伊霍復見於今矣乃抗表稱大行皇帝背棄萬方|{
	復扶又翻背蒲妹翻}
海内咸稱酖毒致禍豈有天子不豫初不召醫貴戚大臣皆不侍側安得不使遠近怪愕又以皇女為儲兩|{
	太子謂之儲君易曰明兩作離大人以繼明照四方故稱儲兩}
虛行赦宥上欺天地下惑朝野己乃選君於孩提之中實使姦豎專朝隳亂綱紀|{
	朝直遥翻}
此何異掩目捕雀塞耳盗鍾|{
	塞悉則翻}
今羣盗沸騰鄰敵窺窬而欲以未言之兒鎮安天下不亦難乎願聽臣赴闕參預大議問侍臣帝崩之由訪侍衛不知之狀以徐鄭之徒付之司敗雪同天之恥|{
	君父之仇義不同天}
謝遠近之怨然後更擇宗親以承寶祚榮從弟世隆時為直閣|{
	更工衡翻從才用翻下從子之從同}
太后遣詣晉陽慰諭榮榮欲留之世隆曰朝廷疑兄故遣世隆來今留世隆使朝廷得預為之備非計也乃遣之 三月癸未葛榮陷魏滄州|{
	魏肅宗熙平二年分瀛冀二州置滄州治饒安城領浮陽樂陵郡}
執刺史薛慶之居民死者什八九 乙酉魏葬孝明皇帝于定陵廟號肅宗 爾朱榮與元天穆議以彭城武宣王有忠勲|{
	彭城王勰諡武宣忠勲謂侍孝文帝疾立宣武帝備極忠勤也}
其子長樂王子攸素有令望欲立之|{
	樂音洛}
又遣從子天光及親信奚毅倉頭王相入洛與爾朱世隆密議天光見子攸具論榮心子攸許之天光等還晉陽榮猶疑之乃以銅為顯祖諸孫各鑄像唯長樂王像成|{
	魏人立后皆鑄像以卜之慕容氏謂冉閔以金鑄己像不成胡人鑄像以卜君其來尚矣故爾朱榮效之}
榮乃起兵晉陽世隆逃出會榮於上黨靈太后聞之甚懼悉召王公等入議宗室大臣皆疾太后所為莫肯致言徐紇獨曰爾朱榮小胡敢稱兵向闕文武宿衛足以制之但守險要以逸待勞彼懸軍千里士馬疲弊破之必矣太后以為然以黄門侍郎李神軌為大都督帥衆拒之别將鄭季明鄭先護將兵守河橋|{
	帥讀曰率將即亮翻}
武衛將軍費穆屯小平津先護儼之從祖兄弟也榮至河内復遣王相密至洛迎長樂王子攸|{
	復扶又翻下榮復同}
夏四月丙申子攸與兄彭城王劭弟霸城公子正濳自高渚渡河 |{
	考異曰楊衒之洛陽伽藍記高渚作霤波今從魏書}
丁酉會榮於河陽將士咸稱萬歲戊戍濟河子攸即帝位|{
	帝彭城王勰之第三子}
以劭為無上王|{
	劭彭城嫡嗣且魏主兄也封為無上王言其尊無上也有君而言無上君子是以知魏主之不終也}
子正為始平王以榮為侍中都督中外諸軍事大將軍尚書令領軍將軍領左右封太原王|{
	領左右領左右千牛備身也}
鄭先護素與敬宗善聞帝即位與鄭季明開城納之李神軌至河橋聞北中不守|{
	晉杜預建河橋於富平津河北側岸有二城相對魏高祖置北中郎府徙諸從隸府戶并羽林虎賁領隊防之北中不守可以平行至洛陽矣宋白曰北中城即今河陽城}
即遁還費穆棄衆先降於榮|{
	降江}
徐紇矯詔夜開殿門取驊驑廐御馬十匹|{
	驊驑駿馬也故魏以名御馬廐}
東奔兖州|{
	將依羊侃也為侃與紇南歸張本}
鄭儼亦走還鄉里|{
	鄭儼滎陽開封人}
太后盡召肅宗後宫皆令出家太后亦自落髪榮召百官迎車駕己亥百官奉璽綬備法駕迎敬宗於河橋|{
	璽斯氏翻綬音受 考異曰伽藍記云十二日爾朱榮軍于邙山之北河隂之野十三日召百官迎駕至者盡誅之長歷是月戊子朔十二日己亥也今從魏書}
庚子榮遣騎執太后及幼主送至河隂|{
	騎奇寄翻}
太后對榮多所陳說榮拂衣而起沈太后及幼主於河費穆密說榮曰公士馬不出萬人今長驅向洛前無横陳既無戰勝之威羣情素不厭服|{
	沈持林翻說式芮翻陳讀曰陣厭於葉翻}
以京師之衆百官之盛知公虛實有輕侮之心若不大行誅罰更樹親黨恐公還北之日未度太行而内變作矣|{
	更工衡翻行戶剛翻}
榮心然之謂所親慕容紹宗曰洛中人士繁盛驕侈成俗不加芟翦終難制馭|{
	芟所銜翻}
吾欲因百官出迎悉誅之何如紹宗曰太后荒淫失道嬖倖弄權殽亂四海|{
	殽雜也錯也嬖卑義翻又博計翻}
故明公興義兵以清朝廷今無故殱夷多士不分忠佞|{
	殱息廉翻}
恐大失天下之望非長策也榮不聽乃請帝循河西至淘渚|{
	水經注孟津又曰富平津又謂之淘河杜畿試樓船於孟津覆於陶河即此也按爾朱榮傳陶渚在河隂西北三里南北長堤之西魏紀淘作陶杜佑曰河南河陽縣西南十三里有古遮馬堤即其處}
引百官於行宫西北云欲祭天百官既集列胡騎圍之責以天下喪亂肅宗暴崩皆由朝臣貪虐不能匡弼|{
	騎奇寄翻喪息浪翻朝直遥翻}
因縱兵殺之自丞相高陽王雍司空元欽儀同三司義陽王略以下死者二千餘人 |{
	考異曰北史云榮惑費穆之言謂天下乘機可取乃譎朝士共為盟誓將向河隂西北三里至南北長堤悉命下馬西度即遣胡騎圍之妄言丞相高陽王反殺王公二千餘人榮傳一千三百餘人今從魏紀}
前黃門郎王遵業兄弟居父喪其母敬宗之從母也相帥出迎俱死遵業慧龍之孫也|{
	劉裕得晉權殺王愉其孫慧龍奔魏著功名於南鄙從才用翻帥讀曰率}
儁爽涉學時人惜其才而譏其躁|{
	躁則到翻}
有朝士百餘人後至榮復以胡騎圍之令曰有能為禪文者免死侍御史趙元則出應募遂使為之 |{
	考異曰北史曰時隴西李神儁頓丘李諧太原温子昇並當世辭人皆在圍中恥從是命俯伏不應按神儁等不應何得不死魏書本傳皆無其事}
榮又令其軍士言元氏既滅爾朱氏興皆稱萬歲榮又遣數十人拔刀向行宫帝與無上王劭始平王子正俱出帳外榮先遣并州人郭羅刹西部高車叱列殺鬼侍帝側|{
	刹初轄翻}
詐言防衛抱帝入帳餘人即殺劭及子正又遣數十人遷帝於河橋置之幕下帝憂憤無計使人諭旨於榮曰帝王迭興盛衰無常今四方瓦解將軍奮袂而起所向無前此乃天意非人力也我本相投志在全生豈敢妄希天位|{
	希望也}
將軍見逼以至於此若天命有歸將軍宜時正尊號若推而不居|{
	推吐雷翻}
存魏社稷亦當更擇親賢而輔之|{
	更工衡翻}
時都督高歡勸榮稱帝 |{
	考異曰魏爾朱榮傳曰於是獻武王與外兵參軍司馬子如等切諫陳不可之理榮曰愆誤若是唯當以死謝朝廷今日安危之機計將何出獻武王等曰未若還奉長樂以安天下於是還奉莊帝北齊書神武紀云榮將簒位神武諫恐不聽請鑄像卜之鑄不成乃止盖魏收與北齊史官欲為神武掩此惡故云爾今從周書賀拔岳傳}
左右多同之榮疑未决賀拔岳進曰將軍首舉義兵志除姦逆大勲未立遽有此謀正可速禍未見其福榮乃自鑄金為像凡四鑄不成功曹參軍燕郡劉靈助善卜筮|{
	漢高祖定天下燕仍為國昭帝改為廣陽郡後漢光武併上谷和帝復為廣陽郡晉為廣陽國魏為燕郡燕因肩翻}
榮信之靈助言天時人事未可榮曰若我不吉當迎天穆立之靈助曰天穆亦不吉唯長樂王有天命耳榮亦精神恍惚不自支持久而方寤|{
	史言天位不可以詐力奸恍呼廣翻惚音忽}
深思愧悔曰過誤若是唯當以死謝朝廷賀拔岳請殺高歡以謝天下左右曰歡雖復愚疎言不思難|{
	復扶又翻下更復同難乃旦翻}
今四方多事須藉武將|{
	將即亮翻}
請捨之收其後效榮乃止夜四更復迎帝還營|{
	更工衡翻}
榮望馬首叩頭請死榮所從胡騎殺朝士既多不敢入洛城即欲向北為遷都之計榮狐疑甚久武衛將軍汎禮固諫|{
	皇甫謐云汎本姓凡遭秦亂避地於汜水因氏焉汎音凡}
辛丑榮奉帝入城帝御太極殿下詔大赦改元建義從太原王將士普加五階在京文官二階武官三階百姓復租役三年|{
	復方目翻除也}
時百官蕩盡存者皆竄匿不出唯散騎常侍山偉一人拜赦於闕下|{
	山偉稱頌元乂而得進其人可知特乾沒以苟一時之利禄耳}
洛中士民草草|{
	詩巷伯曰勞人草草注云草草勞心也箋云草草憂將妄得罪也}
人懷異慮或云榮欲縱兵大掠或云欲遷都晉陽富者弃宅貧者襁負率皆逃竄什不存一二|{
	襁居兩翻}
直衛空虛官守曠廢|{
	守手又翻曠空也}
榮乃上書稱大兵交際難可齊壹諸王朝貴横死者衆|{
	横戶孟翻}
臣今粉軀不足塞咎|{
	塞悉則翻}
乞追贈亡者微申私責無上王稱追尊為無上皇帝自餘死于河隂者王贈三司三品贈令僕五品贈刺史七品已下白民贈郡鎭|{
	身無官爵謂之白民猶言白丁也郡鎭郡守鎮將也}
死者無後聽繼即授封爵又遣使者循城勞問|{
	勞力到翻}
詔從之於是朝士稍出人心粗安|{
	粗坐五翻}
封無上王之子韶為彭城王榮猶執遷都之議帝亦不能違都官尚書元諶爭之以為不可|{
	漢成帝置三公曹尚書主斷獄光武改三公曹主考課置中都官曹尚書主水火盗賊事魏置尚書都官郎佐督軍事晉以三公尚書掌刑獄宋三公比部主刑法又置都官尚書主軍事刑獄至隋乃改都官尚書為刑部尚書諶氏壬翻}
榮怒曰何關君事而固執也且河隂之事君應知之諶曰天下事當與天下論之奈何以河隂之酷而恐元諶諶國之宗室位居常伯|{
	尚書古常伯之任}
生既無益死復何損|{
	復扶又翻}
正使今日碎首流腸亦無所懼榮大怒欲抵諶罪爾朱世隆固諫乃止見者莫不震悚諶顔色自若後數日帝與榮登高見宫闕壯麗列樹成行|{
	行戶剛翻}
乃歎曰臣昨愚闇有北遷之意今見皇居之盛熟思元尚書言深不可奪由是罷遷都之議諶謐之兄也癸亥以江陽王繼為太師北海王顥為太傅光禄大夫李延寔為太保賜爵濮陽王并州刺史元天穆為太尉賜爵上黨王前侍中楊椿為司徒車騎大將軍穆紹為司空領尚書令進爵頓丘王雍州刺史長孫稚為驃騎大將軍開府儀同三司賜爵馮翊王|{
	雍於用翻長知兩翻驃匹妙翻騎奇寄翻}
殿中尚書元諶為尚書右僕射賜爵魏郡王金紫光禄大夫廣陵王恭加儀同三司其餘起家暴貴者不可勝數延寔冲之子也|{
	李冲柄用於文明孝文之時勝音升}
以帝舅故得超拜徐紇弟獻伯為北海太守季產為青州長史紇使人告之皆將家屬逃去與紇俱奔泰山|{
	泰山郡屬兖州所謂東奔兖州也}
鄭儼與從兄滎陽太守仲明謀據郡起兵|{
	從才用翻}
為部下所殺丁未詔内外解嚴 魏郢州刺史元顯逹請降|{
	降戶江翻}
詔郢州刺史元樹迎之|{
	魏克義陽以梁之司州為郢州梁之郢州治江夏郡}
夏侯夔亦自楚城往會之遂留鎮焉改魏郢州為北司州以夔為刺史兼督司州|{
	梁初義陽䧟僑置司州于關南今黄州黄陂縣之地既復義陽因以為北司州}
夔進攻毛城逼新蔡豫州刺史夏侯亶圍南頓攻陳項魏行臺源子恭拒之 庚戍魏賜爾朱榮子乂羅爵梁郡王 柔然頭兵可汗數入貢于魏|{
	可從刋入聲汗音寒數所角翻}
魏詔頭兵贊拜不名上書不稱臣 魏汝南王悦及東道行臺臨淮王彧聞河隂之亂皆來奔先是魏人降者皆稱魏官為偽|{
	先悉薦翻}
彧表啓獨稱魏臨淮王上亦體其雅素不之責魏北海王顥將之相州至汲郡聞葛榮南侵及爾朱榮縱暴隂為自安之計盤桓不進以其舅殷州刺史范遵行相州事代前刺史李神守鄴行臺甄密知顥有異志相帥廢遵|{
	甄之人翻帥讀曰率}
復推李神攝州事|{
	復扶又翻}
遣兵迎顥且察其變顥聞之帥左右來奔|{
	為後送顥北還張本帥讀曰率}
密琛之從父弟也|{
	甄琛事魏宣武帝從才用翻}
北青州刺史元世儁南荆州刺史李志皆舉州來降|{
	魏北青州治東陽去梁境甚遠五代志東海郡梁置南北二青州郡領懷仁縣又注云梁置南北二青州意者元世儁以懷仁之地來降也志又曰春陵郡後魏置南荆州降戶江翻}
五月丁巳朔魏加爾朱榮北道大行臺以尚書右僕

射元羅為東道大使|{
	使疏吏翻}
光禄勲元欣副之巡方黜陟先行後聞欣羽之子也|{
	羽孝文帝之弟封廣陵王}
爾朱榮入見魏主於明光殿|{
	見賢遍翻下朝見同}
重謝河橋之事|{
	重直用翻}
誓言無復貳心帝自起止之因復為榮誓言無疑心|{
	復扶又翻下不復同為于偽翻}
榮喜因求酒飲之熟醉帝欲誅之左右苦諫乃止即以牀轝向中常侍省|{
	轝羊茹翻轝車為轝}
榮夜半方寤遂逹旦不眠自此不復禁中宿矣榮女先為肅宗嬪榮欲敬宗立以為后帝疑未决黄門侍郎祖瑩曰昔文公在秦懷嬴入侍事有反經合義|{
	左傳晉世子圉質于秦秦伯以女妻之子圉逃歸公子重耳入秦秦納女五人懷嬴與焉秦嬴氏也圉謚懷公故曰懷嬴漢儒以反經合道為權祖瑩本此嬪毗賓翻}
陛下獨何疑焉帝遂從之榮意甚悦榮舉止輕脱喜馳射|{
	喜許記翻}
每入朝見更無所為唯戲上下馬於西林園宴射恒請皇后出觀并召王公妃主共在一堂每見天子射中輒自起舞叫將相卿士悉皆盤旋|{
	上時掌翻恒戶登翻中竹仲翻將即亮翻相息亮翻}
乃至妃主亦不免隨之舉袂及酒酣耳熱必自匡坐唱虜歌|{
	匡坐正坐也虜歌胡歌也}
日暮罷歸與左右連手蹋地唱囘波樂而出|{
	此所謂蹋歌也囘波樂曲名樂音洛}
性甚嚴暴喜愠無常刀槊弓矢不離於手|{
	槊色角翻離力智翻}
每有瞋嫌|{
	瞋昌真翻}
輒行擊射|{
	射而亦翻}
左右恒有死憂|{
	恒戶登翻}
嘗見沙彌重騎一馬|{
	去俗為僧受度而未受戒者謂之沙彌重騎者二人共騎也重直龍翻}
榮即令相觸力窮不能復動|{
	復扶又翻}
遂使傍人以頭相擊死而後己辛酉榮還晉陽帝餞之於邙隂|{
	邙隂邙山之北也}
榮令元天穆入洛陽加天穆侍中録尚書事京畿大都督兼領軍將軍以行臺郎中桑乾朱瑞為黄門侍郎兼中書舍人朝廷要官悉用其腹心為之 丙寅魏主詔孝昌以來凡有寃抑無訴者悉集華林東門當親理之時承喪亂之後|{
	喪息浪翻}
倉廩虛竭始詔入粟八千石者賜爵散侯|{
	此有官入栗者之賜也魏制散侯降開國侯一品散悉亶翻}
白民輸五百石者賜出身沙門授本州統及郡縣維那|{
	維那各管其郡縣之僧}
爾朱榮之趣洛也|{
	趣七喻翻}
遣其都督樊子鵠取唐州唐州刺史崔元珍行臺酈惲拒守不從乙亥子鵠拔平陽斬元珍及惲|{
	惲於粉翻}
元珍挺之從父弟也|{
	從才用翻}
將軍曹義宗圍魏荆州堰水灌城不沒者數板時魏

方多難不能救|{
	難乃旦翻}
城中糧盡刺史王羆煮粥與將士均分食之每出戰不擐甲胄仰天大呼曰|{
	擐音宦呼火故翻}
荆州城孝文皇帝所置|{
	魏孝文太和中置荆州於穰城}
天若不祐國家令箭中王羆額不爾王羆必當破賊彌歷三年前後戰甚衆亦不被傷|{
	中作仲翻被皮義翻}
癸未魏以中軍將軍費穆都督南征諸軍事將兵救之|{
	將即亮翻}
魏臨淮王彧聞魏主定位乃以母老求還辭情懇至上惜其才而不能違六月丁亥遣彧還魏以彧為侍中驃騎大將軍加儀同三司|{
	驃匹妙翻騎奇寄翻}
魏員外散騎常侍高乾祐之從子也|{
	高祐允之從祖弟從才用翻}
與弟敖曹季式皆喜輕俠|{
	喜許記翻}
與魏主有舊爾朱榮之向洛也逃奔齊州聞河隂之亂遂集流民起兵於河濟之間受葛榮官爵頻破州軍魏主使元欣諭旨乾等乃降|{
	濟子禮翻降戶江翻}
以乾為給事黃門侍郎兼武衛將軍敖曹為通直散騎侍郎榮以乾兄弟前為叛亂不應復居近要魏主乃聽解官歸鄉里敖曹復行抄掠|{
	復扶又翻抄楚交翻}
榮誘執之與薛修義同拘於晉陽|{
	薛修義為龍門鎮將附蕭寶寅既降而反側故亦被拘誘音酉}
敖曹名昂以字行|{
	敖曹之父以其子昂藏敖曹故以為名字}
葛榮軍乏食遣其僕射任褒將兵南掠至沁水|{
	沁水縣自漢以來屬河南郡任音壬沁午侵翻}
魏以元天穆為大都督東北道諸軍事帥宗正珍孫等討之|{
	帥讀曰率下同}
前幽州平北府主簿河間邢杲帥河北流民十萬餘戶反於青州之北海自稱漢王改元天統戊申魏以征東將軍李叔仁為車騎大將軍儀同三司帥衆討之辛亥魏主詔曰朕當親御六戎|{
	戎兵也六戎猶言六軍也}
掃静燕代|{
	燕因肩翻}
以大將軍爾朱榮為左軍上黨王天穆為前軍司徒楊椿為右軍司空穆紹為後軍葛榮退屯相州之北|{
	相息亮翻}
秋七月乙丑魏加爾朱榮柱國大將軍録尚書事|{
	魏初置柱國大將軍長孫嵩以開國元勲加此號}
壬子魏光州民劉舉聚衆反於濮陽|{
	濮博木翻}
自稱皇武

大將軍 是月万俟醜奴自稱天子置百官會波斯國獻師子於魏|{
	波斯國都宿利城在忸密西古條支國也去代都二萬四千二百二十八里}
醜奴留之改元神獸 魏泰山太守羊侃|{
	守式又翻}
以其祖規嘗為宋高祖祭酒從事常有南歸之志徐紇往依之因勸侃起兵侃從之兖州刺史羊敦侃之從兄也|{
	從才用翻}
密知之據州拒侃八月侃引兵襲敦弗克|{
	魏兖州刺史治瑕丘泰山太守治博}
築十餘城守之且遣使來降|{
	使疏吏翻降戶江翻}
詔廣晉縣侯泰山羊鴉仁等將兵應接|{
	沈約志鄱陽郡有廣晉縣本吳所置廣昌縣晉武帝太康元年更名廣晉}
魏以侃為驃騎大將軍泰山公兖州刺史侃斬其使者不受將軍王弁侵魏徐州蕃郡民續靈珍擁衆萬人攻蕃郡以應梁|{
	魏徐州治彭城領彭城南陽平蕃沛蘭陵北濟隂碭郡蕃縣漢晉屬魯國魏孝昌三年置蕃郡治蕃城五代志徐州滕縣舊曰蕃置蕃郡隋開皇十六年改曰滕郡尋廢郡為縣蕃音皮又音翻}
魏徐州刺史楊昱擊靈珍斬之弁引還 甲辰魏大都督宗正珍孫擊劉舉於濮陽滅之 葛榮引兵圍鄴衆號百萬遊兵已過汲郡|{
	汲郡隋唐之衛州}
所至殘掠爾朱榮啟求討之九月爾朱榮召從子肆州刺史天光留鎮晉陽曰我身不得至處非汝無以稱我心|{
	從才用翻稱尺證翻}
自帥精騎七千馬皆有副|{
	魏收魏書云帥騎七萬帥讀曰率騎奇寄翻}
倍道兼行東出滏口以侯景為前驅|{
	滏音釡}
葛榮為盗日久|{
	梁普通七年葛榮得鮮于修禮之衆寇掠河北}
横行河北爾朱榮衆寡非敵議者謂無取勝之理葛榮聞之喜見於色|{
	見賢遍翻}
令其衆曰此易與耳|{
	易弋豉翻}
諸人俱辦長繩至則縳取自鄴以北列陳數十里箕張而進|{
	如箕之張也陳讀曰陣}
爾朱榮濳軍山谷為奇兵分督將已上三人為一處處有數百騎令所在揚塵鼓譟使賊不測多少|{
	將即亮翻少詩沼翻}
又以人馬逼戰刀不如棒勒軍士齎袖棒一枚置於馬側至戰時慮廢騰逐不聽斬級|{
	斬級者斬首以計功級}
以棒棒之而已|{
	棒蒲項翻}
分命壯勇所向衝突號令嚴明戰士同奮爾主榮身自陷陳出於賊後表裏合擊大破之於陳擒葛榮|{
	陳讀曰陣}
餘衆悉降|{
	降戶江翻}
以賊徒既衆若即分割恐其疑懼或更結聚乃下令各從所樂|{
	樂音洛}
親屬相隨任所居止於是羣情大喜登即四散|{
	登者登時也}
數十萬衆一朝散盡待出百里之外乃始分道押領隨便安置咸得其宜擢其渠帥量才授任新附者咸安時人服其處分機速|{
	帥所類翻量音良處昌呂翻分扶問翻}
以檻車送葛榮赴洛冀定滄瀛殷五州皆平時上黨王天穆軍於朝歌之南穆紹楊椿猶未發而葛榮已滅乃皆罷兵|{
	是年夏魏主將北征以爾朱榮為左軍楊椿為右軍穆紹為後軍}
初宇文肱從鮮于修禮攻定州戰死于唐河|{
	魏收志定州中山郡唐縣有唐水水經唐水導源盧奴縣西北東流至唐城西北隅堨而為河其水南入小溝下注水}
其子泰在修禮軍中修禮死從葛榮葛榮敗爾朱榮愛泰之才以為統軍|{
	宇文泰事始此}
乙亥魏大赦改元永安辛巳以爾朱榮為大丞相都督河北畿外諸軍事榮子平昌公文殊昌樂公文暢並進爵為王|{
	樂音洛}
以楊椿為太保城陽王徽為司徒冬十月丁亥葛榮至洛魏主御閶闔門|{
	洛城西面有廣陽西明閶闔三門又洛陽宫城門曰閶闔注已見八十四卷晉惠帝太安元年}
引見斬於都市 帝以魏北海王顥為魏王遣東宫直閣將軍陳慶之將兵送之還北|{
	將即亮翻 考異曰梁魏帝紀皆云以顥為魏主唯顥傳作魏王按魏封劉昶為宋王蕭寶寅為齊王蕭詧為梁王皆俟得國然後使稱帝耳若顥在南已稱魏帝當行即位之禮又梁朝應以客禮待之又顥不應再即帝位於渙水盖由王字與主字止欠一點故多致謬誤今從顥傳}
丙申魏以太原王世子爾朱菩提為驃騎大將軍開府儀同三司|{
	菩薄乎翻}
丁酉以長樂等七郡各萬戶通前十萬戶為太原王榮國|{
	樂音洛}
戊戍又加榮太師皆賞擒葛榮之功也 壬子魏江陽武烈王繼卒 魏使征虜將軍韓子熙招論邢杲杲詐降而復反|{
	降戶江翻復扶又翻}
李叔仁擊杲於惟水|{
	惟水當作濰水水經濰水出琅琊箕縣東北過東武城縣西又北過平昌縣東又北過高密縣西又北過淳于縣東又東北過下密縣故城西又東北過都昌縣東又東北入于海五代志後魏北海郡膠東縣隋改曰濰水縣後又改曰下密縣濰音維}
失利而還|{
	還從宣翻又如字}
魏費穆奄至荆州曹義宗軍敗為魏所擒荆州之圍始解|{
	荆州受圍三年始解}
元顥襲魏銍城而據之|{
	銍縣漢屬沛郡魏晉屬譙郡宋白曰宿州臨渙縣漢銍縣地銍竹乙翻}
魏行臺尚書左僕射于暉等兵數十萬擊羊侃於瑕丘|{
	劉昫曰瑕丘春秋時魯之瑕邑宋以為兖州治所隋始置瑕丘縣}
徐紇恐事不濟說侃請乞師於梁|{
	說式芮翻}
侃信之紇遂來奔暉等圍侃十餘重|{
	重直龍翻}
柵中矢盡南軍不進十一月癸亥夜侃潰圍出且戰且行一日一夜乃出魏境至渣口|{
	水經引郡國志曰偪陽有柤水南亂于沂而注于沐謂之相口春秋諸侯會吳于柤即此渣側加翻}
衆尚萬餘人馬二千匹士卒皆竟夜悲歌侃乃謝曰卿等懷土理不能相隨幸適去留|{
	言或去或留各從其意也}
於此為别各拜辭而去魏復取泰山|{
	復扶又翻又如字}
暉勁之子也|{
	于勁事魏孝文帝}
戊寅魏以上黨王天穆為大將軍開府儀同三司世襲并州刺史 十二月庚子魏詔于暉還師討邢杲 葛榮餘黨韓樓復據幽州反|{
	為爾朱榮遣將平韓樓張本復扶又翻}
北邊被其患|{
	被皮義翻}
爾朱榮以撫軍將軍賀拔勝為大都督鎮中山樓畏勝威名不敢南出

資治通鑑卷一百五十二
