






























































資治通鑑卷五十七   宋 司馬光 撰

胡三省 音註

漢紀四十九|{
	起玄武困敦盡上章君灘凡九年}


孝靈皇帝上之下

熹平元年春正月車駕上原陵司徒掾陳留蔡邕曰吾聞古不墓祭朝廷有上陵之禮始謂可損今見威儀察其本意乃知孝明皇帝至孝惻隱不易奪也|{
	據禮儀志西都舊有上陵至東都則其儀文愈備其畧見四十四卷永平元年丄時掌翻掾俞絹翻易以䜴翻}
禮有煩而不可省者此之謂也 三月壬戌太傅胡廣薨年八十二廣周流四公|{
	太傅太尉司徒司空}
三十餘年|{
	賢曰廣以順帝漢安元年為司空至熹平元年薨三十一年也}
歷事六帝|{
	安順冲質桓靈}
禮任極優罷免未嘗滿歲輒復升進|{
	復扶又翻}
所辟多天下名士與故吏陳蕃李咸竝為三司|{
	三司即三公}
練達故事明解朝章|{
	解戶買翻曉也朝直遙翻}
故

京師諺曰萬事不理問伯始天下中庸有胡公|{
	胡廣字伯始夫既曰萬事不理問伯始則當時之責望亦重矣豈可以三十餘年周流四公為榮哉賢曰中和也庸常也中和可常行之德也}
然温柔謹慤常遜言恭色以取媚於時無忠直之風天下以此薄之 五月己巳赦天下改元 長樂太僕侯覽坐專權驕奢策收印綬自殺|{
	長樂太僕太后宫官也主馭宦者為之秩二千石樂音洛}
六月京師大水 竇太后母卒於比景太后憂思感疾癸巳崩於雲臺宦者積怨竇氏以衣車載太后尸置城南市舍數日曹節王甫欲用貴人禮殯帝曰太后親立朕躬統承大業豈宜以貴人終乎於是發喪成禮節等欲别葬太后而以馮貴人配祔|{
	賢曰祔謂新死之主祔于先死者之廟婦祔于其夫所祔之妃妾祔于妾祖姑也}
詔公卿大會朝堂|{
	朝直遥翻}
令中常侍趙忠監議|{
	監古衘翻}
太尉李咸時病扶輿而起擣椒自隨|{
	孔頴達曰釋木云檓大椒郭璞曰今椒樹叢生實大者名為檓陸璣疏云椒樹如茱萸有針刺葉堅而滑澤蜀人作茶吳人作茗皆合煮其葉以為香今成臯山間有椒謂之竹葉椒其樹亦如蜀椒少毒熱不中合藥也可著飲食中又用烝雞豚最佳香東海諸鳥亦有椒樹枝葉皆相似子長而不圓甚香其味似橘皮本草亦云椒大熱有毒按李咸擣椒自隨齊明帝將殺高武諸孫敇太官煮椒二斛盖其毒能殺人也}
謂妻子曰若皇太后不得配食桓帝吾不生還矣|{
	欲以死爭之也}
既議坐者數百人各瞻望良久莫肯先言趙忠曰議當時定廷尉陳球曰皇太后以盛德良家母臨天下宜配先帝是無所疑忠笑而言曰陳廷尉宜便操筆|{
	操千高翻}
球即下議曰皇太后自在椒房有聰明母儀之德遭時不造援立聖明承繼宗廟功烈至重先帝晏駕因遇大獄遷居空宫不幸早世家雖獲罪事非太后今若别葬誠失天下之望且馮貴人冢嘗被發掘骸骨暴露與賊併尸魂靈汙染|{
	賢曰段熲為河南尹坐盜發馮貴人冢左遷諫議大夫余據熲以延熹三年入為侍中轉執金吾河南尹則發冢之事于是年近耳被皮義翻}
且無功於國何宜上配至尊忠省球議|{
	省悉井翻下同}
作色俛仰蚩球曰陳廷尉建此議甚健|{
	蚩笑也}
球曰陳竇既寃皇太后無故幽閉臣常痛心天下憤歎今日言之退而受罪宿昔之願也李咸曰臣本謂宜爾誠與意合於是公卿以下皆從球議曹節王甫猶爭以為梁后家犯惡逆别葬懿陵|{
	梁后先桓帝崩葬懿陵梁冀誅始廢陵為貴人冢}
武帝黜廢衛后而以李夫人配食|{
	戾太子之亂武帝策廢其母衛后后自殺武帝崩霍光緣帝雅意以李夫人配食}
今竇氏罪深豈得合葬先帝李咸復上疏曰臣伏惟章德竇后虐害恭懷安思閻后家犯惡逆|{
	竇后事見四十六卷章帝建初八年閻后事見五十卷五十一卷安帝延光三年四年復扶又翻}
而和帝無異葬之議順朝無貶降之文|{
	朝直遥翻}
至於衛后孝武皇帝身所廢棄不可以為比今長樂太后尊號在身親嘗稱制且援立聖明光隆皇祚太后以陛下為子陛下豈得不以太后為母子無黜母臣無貶君宜合葬宣陵一如舊制帝省奏從之|{
	省悉景翻考異曰袁紀云河南尹李咸執藥上書曰昔秦始皇幽閉母后感茅焦之言立駕迎母供養如初夫以秦后之惡始皇之悖尚納直臣之語不失母子之恩豈况皇太后不以罪殁陛下之過有重始皇臣謹左手齎章右手執藥詣闕自聞如遂不省臣當飲鴆自裁下覲先帝具陳得失章省上感其言使公卿更議廷尉陳球乃下議與范不同今從范書}
秋七月甲寅葬桓思皇后於宣陵 有人書朱雀闕|{
	古今注永平二年十一月初作北宫朱爵南司馬門闕在宫門之外}
言天下大亂曹節王甫幽殺太后 |{
	考異曰舊云常侍侯覽多殺黨人按時覽已死恐誤今去之}
公卿皆尸祿無忠言者詔司隸校尉劉猛逐捕十日一會猛以誹書言直不肯急捕月餘主名不立|{
	賢曰不得書闕主名}
猛坐左轉諫議大夫以御史中丞段熲代之熲乃四出逐捕及太學游生繫者千餘人節等又使熲以他事奏猛論輸左校|{
	校戶教翻}
初司隸校尉王寓依倚宦官求薦于太常張奐奐拒之寓遂陷奐以黨罪禁銅奐嘗與段熲争擊羌不相平|{
	事見上卷建寧元年}
熲為司隸欲逐奐歸敦煌而害之|{
	奐徙屬弘農事見上卷桓帝永康元年敦徒門翻}
奐奏記哀請於熲乃得免初魏郡李暠為司隸校尉|{
	暠古老翻}
以舊怨殺扶風蘇謙謙子不韋瘞而不葬|{
	瘞於計翻}
變姓名結客報仇暠遷大司農不韋匿于廥中鑿地旁達暠之寢室|{
	說文曰廥芻槀藏音工外翻}
殺其妾并小兒暠大懼以板藉地一夕九徙又掘暠父冢斷取其頭|{
	斷丁管翻}
標之于市暠求捕不獲憤恚嘔血死|{
	恚于避翻}
不韋遇赦還家乃葬父行喪張奐素睦於蘇氏而段熲與暠善熲辟不韋為司隸從事不韋懼稱病不詣熲怒使從事張賢就家殺之先以鴆與賢父曰若賢不得不韋便可飲此賢遂收不韋并其一門六十餘人盡誅之勃海王悝之貶癭陶也|{
	悝苦回翻癭於郢翻}
因中常侍王甫求

復國許謝錢五千萬既而桓帝遺詔復悝國|{
	悝復國事見上卷永康元年}
悝知非甫功不肯還謝錢中常侍鄭颯中黄門董騰數與悝交通|{
	颯音立數所角翻}
甫密司察以告段熲|{
	司讀曰伺}
冬十月收颯送北寺獄使尚書令亷忠誣奏颯等謀迎立悝大逆不道遂詔冀州刺史收悝考實迫責悝令自殺妃妾十一人子女七十人伎女二十四人皆死獄中|{
	伎渠綺翻}
傅相以下悉伏誅甫等十二人皆以功封列侯 十一月會稽妖賊許生起句章|{
	句章縣屬會稽郡賢曰故城在今越州鄮縣西十三州志曰句踐之地南至句無其後併吳因大城之章霸功以示子孫故曰句章妖于驕翻句音章句之句}
自稱陽明皇帝衆以萬數遣揚州刺史臧旻丹陽太守陳寅討之 十二月司徒許栩罷以大鴻臚袁隗為司徒|{
	隗五罪翻 考異曰袁紀在四年今從范書}
鮮卑寇并州 是歲單于車兒死子屠特若尸逐就單于立|{
	車昌遮翻}


二年春正月大疫 丁丑司空宗俱薨 二月壬午赦天下 以光祿勲楊賜為司空 三月太尉李咸免夏五月以司隷校尉段熲為太尉 六月北海地震秋七月司空楊賜免以太常潁川唐珍為司空珍衡之弟也 冬十二月太尉段熲罷 鮮卑寇幽并二州癸酉晦日有食之

三年春二月己巳赦天下 以太常東海陳耽為太尉三月中山穆王暢薨無子國除|{
	暢中山簡王焉之曾孫焉光武子 考異}


|{
	曰本傳云子節王稚嗣無子國除與帝紀異未知孰是又不知稚薨在何年今且從帝紀}
夏六月封河間王利子康為濟南王奉孝仁皇祀|{
	帝入繼太宗故以康奉孝仁皇祀利帝從兄弟也濟子禮翻}
吳郡司馬富春孫堅召募精勇得千餘人助州郡討許生|{
	百官志郡有丞長史而無司馬蓋是時以盜起置司馬以主兵也富春縣屬吳郡賢曰今杭州富陽縣也避晉簡文帝母鄭太后諱改曰富陽}
冬十一月臧旻陳寅大破生于會稽斬之|{
	會工外翻}
任城王博薨無子國絶|{
	桓帝延熹四年博紹封任城國}
十二月鮮卑入北地太守夏育率屠各追擊破之|{
	守式又翻夏戶雅翻屠直于翻}
遷育為護烏桓校尉鮮卑又寇并州 司空唐珍罷以永樂少府許訓為司空|{
	永樂少府董太后宫官也樂音洛}


四年春三月詔諸儒正五經文字命議郎蔡邕為古文篆隸三體書之刻石立于太學門外|{
	雒陽記曰太學在雒陽城南開陽門外講堂長十丈廣二丈堂前石經四部本碑凡四十六枚西行尚書周易公羊傳十六碑存十二碑毁南行禮記十五碑悉崩壞東行論語三碑毁禮記碑上有諫議大夫馬日磾議郎蔡邕名古文科斗書也大也隸今謂之八分書後魏江式曰伏羲氏作而八卦形其畫軒轅氏興而靈龜彰其采古史蒼頡覽二象之爻觀鳥獸之迹别剏文字以代結繩迄于三代厥體頗異雖依類取制未能殊蒼氏矣周禮保氏教國子以六書一曰指事二曰象形三曰諧聲四曰會意五曰轉注六曰假借蓋是蒼頡之遺法及宣王太史史籕著大十五篇與古文或同或異時人即謂之籕書孔子修六經左丘明述春秋皆以古文七國殊軌文字乖别秦兼天下李斯奏罷不合秦文者斯作蒼頡篇車府令趙高作爰歷篇太史令胡母敬作博學篇皆取史籕或頗有省改所謂小者也秦燒經書滌除舊典官獄繁多以趣約易始用隷書古文由此息矣隷書者始皇使下杜人程邈附于小所作也世人以邈徒隷即謂之隷書故秦有八體一曰大二曰小三曰符書四曰蟲書五曰摹印六曰署書七曰殳書八曰隷書漢興有尉律學教以籕書乂習八體乂有草書莫知誰始其形書雖無厥誼亦是一時之變通也孝宣時召通蒼頡讀者獨張敞從受之凉州刺史杜業沛人爰禮講學大夫秦近亦能言之孝平時徵禮等百餘人說文字于未央宫中黄門侍郎揚雄採以作訓纂亡新居攝使大司馬甄豐校文字之部頗改定古文時有六書一曰古文孔子壁中書也二曰奇字即古文而異者三曰書云小也四曰佐書秦隷書也五曰繆所以摹印也六曰鳥蟲所以書幡信也壁中書者魯恭王壞孔子宅而得尚書春秋論語孝經也又北平侯張蒼獻春秋左氏傳書體與孔氏相類即前代之古文也後漢郎中扶風曹喜號曰工小異斯法而甚精巧自是後學皆其法也又詔侍中賈逵修理舊文殊藝異術王教一端苟有可以加於國者靡不悉集逵即汝南許慎古學之師也慎嗟時人之好奇歎俗儒之穿鑿撰說文解字十五篇類聚羣分雜而不越最可得而論也左中郎將陳留蔡邕採李斯曹喜之灋為古今雜形詔于太學立石經刋載五經題書楷灋多是邕書後開鴻都書畫奇能莫不雲集時諸方獻無出邕者魏初博士清河張揖著埤蒼廣雅古今字詁綴拾遺漏增長事類抑于文為益然其字詁方之許篇古今體用或得或失陳留邯鄲淳亦與揖同特善倉雅許氏字指八體六書精究開理有名于揖又建三字石經於漢碑西較之說文隷大同而古字小異又有京兆韋誕河東衛覬二家並號能當時臺觀榜題寶器之名悉是誕書咸傳之子孫世稱其妙晉世義陽王典祠令呂忱表上字林六卷尋其况趣附託許慎說文而按偶章句隱别古籕奇惑之字文得正隷不差意也忱弟靜别放故左校令李登聲韻之法作韻集五卷使宫商龣徵羽各為一篇而文字與兄便是魯衛音讀楚夏時有不同皇魏承百王之季世易風移文字改變形繆錯隷體失真俗學鄙習復加虛造巧談辯士以意為疑炫惑于時難以釐改乃曰追來為歸巧言為辯小兔為䨲神蟲為蠶如斯甚衆皆不合孔氏古書史籕大許氏說文石經三字也式言字學本末頗詳故備著之趙明誠金石録曰石經蓋漢靈帝熹平四年所立其字則蔡邕小字八分書也後漢書儒林傳叙云為古文隷三體者非也蓋邕所書乃八分而三體石經乃魏時所建也洪氏隷續曰石經見于范史帝紀及儒林宦者傳皆云五經蔡邕張馴傳則曰六經惟儒林傳云為古文隸三體書法酈氏水經云漢立石經于太學魏正始中又刻古文隸三字石經唐志有三字石經古兩種曰尚書曰左傳獨隋志所書異同其目有一字石經七種三字石經三種既以七經為蔡邕書矣又云魏立一字石經乃其誤也范蔚宗時三體石經與熹平所䥴並列于學宮故史筆誤書其事後人襲其譌錯或不見石刻無以考正趙氏雖以一字為中郎所書而未見三體者歐陽氏以三體為漢碑而未嘗見一字者近世方勺作泊宅編載其弟匋所跋石經亦為范史隋志所惑指三體為漢字至公羊碑有馬日磾等名乃云世用其所正定之本因存其名可謂謬論北史江式云魏邯鄲淳以書教皇子建三字石經于漢碑西按此碑以正始年中立漢書云元嘉元年度尚命邯鄲淳作曹娥碑時淳已弱冠自元嘉至正始亦九十餘年式以三字為魏碑則是謂之邯鄲淳所書非也}
使後儒晩學咸取正焉碑始立其觀視及摹寫者車乘日千餘兩填塞街陌|{
	乘繩證翻兩音亮塞悉則翻}
初朝議以州郡相黨人情比周|{
	比毗至翻下同}
乃制昏姻之家及兩州人士不得對相監臨|{
	監古䘖翻}
至是復有三互灋|{
	賢曰三互謂婚姻之家及兩州人不得交互為官也復扶又翻下同}
禁忌轉密選用艱難幽冀二州久缺不補蔡邕上疏曰伏見幽冀舊壤鎧馬所出|{
	賢曰鎧甲也周禮考工記曰燕無函函亦甲也言幽燕之地家家皆能為函故無函匠也左傳曰冀之北土馬之所生}
比年兵饑漸至空耗今者闕職經時吏民延屬|{
	比毗至翻延屬者延頸而屬望也屬之欲翻}
而三府選舉踰月不定臣怪問其故云避三互十一州有禁當取二州而已又二州之士或復限以歲月|{
	復扶又翻下同}
疑遲淹兩州懸空萬里蕭條無所管繫愚以為三互之禁禁之薄者今但申以威靈明其憲令對相部主|{
	冀州之人刺幽州幽州之人刺冀州是為對相部主}
尚畏懼不敢營私况乃三互何足為嫌昔韓安國起自徒中|{
	韓安國梁人坐法抵罪梁内史缺天子遣使拜為梁内史起徒中為二千石}
朱買臣出於幽賤|{
	朱買臣吳人家貧賣薪以自給後隨計吏至長安拜會稽太守}
竝以才宜還守本邦豈復顧循三互繫以末制乎臣願陛下上則先帝蠲除近禁其諸州刺史器用可換者無拘日月三互以差厥中朝廷不從

臣光曰叔向有言國將亡必多制|{
	左傳叔向詒子產書之言也}
明王之政謹擇忠賢而任之凡中外之臣有功則賞有罪則誅無所阿私灋制不煩而天下大治|{
	治直吏翻}
所以然者何哉執其本故也及其衰也百官之任不能擇人而禁令益多防閑益密有功者以閡文不賞|{
	閡與礙同}
為姦者以巧灋免誅上下勞擾而天下大亂所以然者何哉逐其末故也孝靈之時刺史二千石貪如豺虎暴殄烝民而朝廷方守三互之禁以今視之豈不適足為笑而深可為戒哉

封河間王建孫佗為任城王|{
	佗帝從兄弟之子也佗徒河翻任音壬}
夏四月郡國七大水 五月丁卯赦天下 延陵園災|{
	延陵成帝陵也}
鮮卑寇幽州 六月弘農三輔螟 于窴王安國攻拘彌大破之殺其王|{
	窴徒賢翻}
戊巳校尉西域長史各發兵輔立拘彌侍子定興為王人衆裁千口

五年夏四月癸亥赦天下 益州郡夷反太守李顒討平之|{
	顒魚容翻}
大雩 五月太尉陳耽罷以司空許訓為太尉 閏月永昌太守曹鸞上書曰夫黨人者或耆年淵德或衣冠英賢皆宜股肱王室左右大猷者也而久被禁錮辱在塗泥|{
	被皮義翻}
謀反大逆尚蒙赦宥黨人何罪獨不開恕乎所以災異屢見|{
	見賢遍翻}
水旱洊臻皆由于斯宜加沛然以副天心帝省奏大怒|{
	省悉井翻}
即詔司隷益州檻車收鸞送槐里獄掠殺之|{
	永昌郡屬益州刺史而扶風槐里縣屬司隷蓋詔益州收鸞而司隷送槐里獄掠音亮}
于是詔州郡更考黨人門生故吏父子兄弟在位者悉免官禁錮爰及五屬|{
	賢曰謂斬衰齊衰小功大功緦麻也}
六月壬戌以太常南陽劉逸為司空 秋七月太尉許訓罷以光祿勲劉寛為太尉 冬十月司徒袁隗罷 十一月丙戌以光祿大夫楊賜為司徒 是歲鮮卑宼幽州

六年春正月辛丑赦天下 夏四月大旱七州蝗令三公條奏長吏苛酷貪汚者罷免之平原相漁陽陽球坐嚴酷徵詣廷尉|{
	姓譜齊人遷陽子孫以國為氏一曰周景王封少子于陽樊因邑命氏 考異曰本傳司空張顥條奏按顥光和元年為太尉未嘗為司空球光和元年陷蔡邕時已為將作大匠不知被徵果在何年唯熹平五年六年大旱故附于此}
帝以球前為九江太守討賊有功|{
	球傳云九江山賊起三府上球有理姦才拜九江太守球到設方畧凶賊殄破}
特赦之拜議郎 鮮卑寇三邊|{
	鮮卑強盛東西北三邊皆被寇也}
市賈小民相聚為宣陵孝子者數十人詔皆除太子舍人|{
	宣陵桓帝陵百官志太子舍人秩二百石更直宿衛如三署郎中賈音古}
秋七月司空劉逸免以衛尉陳球為司空 初帝好文學|{
	好呼到翻}
自造皇羲篇五十章因引諸生能為文賦者竝待制鴻都門下後諸為尺牘及工書鳥者 |{
	賢曰按說文曰牘書板也長二尺藝文志曰六體者古文奇字書隷書繆蟲書音義曰古文謂孔子壁中書也奇字即古文而異者也書謂小蓋秦始皇使程邈所作也隷書亦程邈所獻主于徒隷從簡易也繆謂其文屈曲繞所以摹印章蟲書謂為蟲鳥之形所以書旛信也}
皆加引召遂至數十人侍中祭酒樂松賈護多引無行趣埶之徒置其間|{
	百官志侍中有僕射一人中興轉為祭酒行下孟翻趣七喻翻}
憙陳閭里小事|{
	憙許記翻}
帝甚悦之待以不次之位又久不親行郊廟之禮會詔羣臣各陳政要蔡邕上封事曰夫迎氣五郊清廟祭祀養老辟雍|{
	迎氣五郊及養老辟雍註並見四十四卷明帝永平二年漢宗廟一歲五祀春以正月夏以四月秋以七月冬以十月及臘}
皆帝者之大業祖宗所祗奉也而有司數以蕃國疎喪宫内產生及吏卒小汙|{
	疎喪謂疎屬之喪也賢曰小汙謂病及死也數所角翻}
廢闕不行忘禮敬之大任禁忌之書拘信小故以虧大典自今齋制宜如故典|{
	漢制凡齋天地七日宗廟山川五日小祀三日齋日内有汙染解齋副倅行禮先齋一日有汙穢災變齋祀如儀}
庶答風霆災妖之異|{
	妖于驕翻}
又古者取士必使諸侯歲貢|{
	尚書大傳曰古者諸侯之于天子三歲一貢士}
孝武之世郡舉孝亷又有賢良文學之選于是名臣輩出文武竝興漢之得人數路而已|{
	賢曰數路謂孝亷賢良文學之類也}
夫書畫辭賦才之小者匡國治政未有其能|{
	治直之翻下同}
陛下即位之初先涉經術聽政餘日觀省篇章|{
	省悉井翻}
聊以游意當代博奕非以為教化取士之本而諸生競利作者鼎沸其高者頗引經訓風喻之言下則連偶俗語有類俳優或竊成文虛冒名氏臣每受詔于盛化門差次録第其未及者亦復隨輩皆見拜擢既加之恩難復收改但守奉祿于義已弘不可復使治民|{
	復扶又翻}
及在州郡昔孝宣會諸儒于石渠|{
	事見二十七卷甘露三年}
章帝集學士于白虎|{
	事見四十六卷建初四年}
通經釋義其事優大文武之道所宜從之若乃小能小善雖有可觀孔子以為致遠則泥君子固當志其大者|{
	賢曰子夏曰雖小道必有可觀者焉致遠恐泥鄭玄註云小道如今諸子書也泥謂滯陷不通邕以為孔子之言當别有所據也泥乃計翻}
又前一切以宣陵孝子為太子舍人臣聞孝文皇帝制喪服三十六日|{
	事見十四卷文帝後七年}
雖繼體之君父子至親公卿列臣受恩之重皆屈情從制不敢踰越今虛偽小人本非骨肉既無幸私之恩又無祿仕之實惻隱之心義無所依至有姦軌之人通容其中桓思皇后祖載之時|{
	鄭玄曰祖謂將葬祖祭于庭載謂升柩于車也}
東郡有盜人妻者亡在孝中本縣追捕乃伏其辜虛偽雜穢難得勝言|{
	勝音升}
太子官屬宜搜選令德豈有但取丘墓凶醜之人其為不祥莫與大焉|{
	言雖它有不祥莫與比並大也}
宜遣歸田里以明詐偽書奏帝乃親迎氣北郊及行辟雍之禮又詔宣陵孝子為舍人者悉改為丞尉焉|{
	漢縣置丞尉丞署文書典知倉獄尉主盜賊}
護烏桓校尉夏育上言|{
	校戶教翻夏戶雅翻上時掌翻}
鮮卑宼邊自春以來三十餘發請徵幽州諸郡兵出塞擊之一冬二春必能禽滅先是護羌校尉田晏坐事論刑|{
	先悉薦翻}
被原|{
	被皮義翻}
欲立功自効乃請中常侍王甫求得為將甫因此議遣兵與育并力討賊帝乃拜晏為破鮮卑中郎將大臣多有不同乃召百官議於朝堂蔡邕議曰征討殊類所由尚矣然而時有同異埶有可否故謀有得失事有成敗不可齊也夫以世宗神武將帥良猛財賦充實所括廣遠數十年間官民俱匱猶有悔焉|{
	謂輪臺哀痛之詔也}
况今人財竝乏事劣昔時乎自匈奴遁逃鮮卑強盛據其故地|{
	事見四十七卷和帝永元五年}
稱兵十萬才力勁健意智益生加以關塞不嚴禁網多漏精金良鐵皆為賊有漢人逋逃為之謀主兵利馬疾過于匈奴昔段熲良將習兵善戰有事西羌猶十餘年|{
	段熲自桓帝延熹二年擊西羌至建寧二年始成功凡十一年}
今育晏才策未必過熲鮮卑種衆不弱曩時|{
	種章勇翻}
而虛計二載|{
	載子亥翻}
自許有成若禍結兵連豈得中休當復徵發衆人轉運無已|{
	復扶又翻}
是為耗竭諸夏并力蠻夷夫邊垂之患手足之疥搔中國之困胷背之瘭疽|{
	賢曰疥音介搔新到翻埤蒼曰瘭必燒翻杜預註左傳曰疽惡瘡也}
方今郡縣盜賊尚不能禁况此醜虜而可伏乎昔高祖忍平城之恥呂后棄慢書之詬|{
	詬古候翻恥也}
方之於今何者為盛天設山河秦築長城漢起塞垣所以别内外異殊俗也|{
	别彼列翻}
苟無䠞國内侮之患則可矣|{
	䠞與蹙同}
豈與蟲螘之虜|{
	螘與蟻翻}
校往來之數哉雖或破之豈可殄盡而方令本朝為之旰食乎|{
	為于偽翻下同旰晩也音古按翻}
昔淮南王安諫伐越曰如使越人蒙死以逆執事厮輿之卒有一不備而歸者|{
	前書音義曰厮微也輿衆也}
雖得越王之首猶為大漢羞之而欲以齊民易醜虜皇威辱外夷就如其言猶已危矣况乎得失不可量邪|{
	量音良}
帝不從八月遣夏育出高柳田晏出雲中匈奴中郎將臧旻率南單于出鴈門各將萬騎三道出塞二千餘里檀石槐命三部大人各帥衆逆戰|{
	檀石槐分其國為三部見五十五卷桓帝延熹九年帥讀曰帥}
育等大敗喪其節傳輜重|{
	喪息浪翻傳株戀翻重直用翻}
各將數十騎犇還死者什七八三將檻車徵下獄|{
	下遐稼翻}
贖為庶人 冬十月癸丑朔日有食之 太尉劉寛免 辛丑京師地震 十一月司空陳球免 十二月甲寅以太常河南孟?為太尉|{
	?音乙六翻}
庚辰司徒楊賜免 以太常陳耽為司空 遼西太守甘陵趙苞到官遣使迎母及妻子垂當到郡道經柳城|{
	杜佑曰漢遼西郡故城在盧龍城東柳城縣屬遼西郡賢曰故城在今營州南}
值鮮卑萬餘人入塞寇鈔|{
	鈔楚交翻}
苞母及妻子遂為所刼質|{
	質音致刼以為質也}
載以擊郡苞率騎二萬與賊對陳|{
	陳讀曰陣}
賊出母以示苞苞悲號謂母曰為子無狀欲以微祿奉養朝夕不圖為母作禍|{
	號戶刀翻養羊亮翻為于偽翻}
昔為母子今為王臣義不得顧私恩毁忠節唯當萬死無以塞罪|{
	塞悉則翻}
母遥謂曰威豪|{
	趙苞字威豪}
人各有命何得相顧以虧忠義爾其勉之苞即時進戰賊悉摧破其母妻皆為所害苞自上歸葬|{
	自上奏乞歸葬也上時掌翻}
帝遣使弔慰封鄃侯|{
	鄃音輸}
苞葬訖謂鄉人曰食祿而避難非忠也|{
	難乃旦翻}
殺母以全義非孝也如是有何面目立于天下遂歐血而死

光和元年|{
	是年三月改元}
春正月合浦交趾烏滸蠻反|{
	滸呼古翻}
招引九真日南民攻没郡縣 太尉孟?罷 二月辛亥朔日有食之 癸丑以光祿勲陳國袁滂為司徒 己未地震 置鴻都門學其諸生皆敇州郡三公舉用辟召或出為刺史太守入為尚書侍中有封侯賜爵者|{
	賜爵關内侯以下也}
士君子皆恥與為列焉 三月辛丑赦天下改元 以太常常山張顥為太尉顥中常侍奉之弟也夏四月丙辰地震 侍中寺雌雞化為雄 司空陳耽免以太常來豔為司空 六月丁丑有黑氣墮帝所御温德殿東庭中長十餘丈似龍|{
	長直亮翻}
秋七月壬子青虹見玉堂後殿庭中|{
	洛陽宫殿名南宫有玉堂前後殿見賢遍翻}
詔召光祿大夫楊賜等詣金商門|{
	洛陽記曰南宫有崇德殿太極殿殿西有金商門}
問以災異及消復之術|{
	消復者消變而復其常也}
賜對曰春秋䜟曰天投蜺天下怨海内亂|{
	春秋演孔圖曰霓者斗之亂精也失度投蜺見郭璞註爾雅曰雙出色鮮盛者為雄曰虹闇者為雌曰蜺}
加四百之期亦復垂及|{
	春秋演孔圖曰劉四百歲之際褒漢王輔皇王以期有名不就宋均註曰雖褒族人為漢王以自輔以當有應期名見攝録者故名不就也復扶又翻}
今妾媵閹尹之徒共專國朝|{
	媵以證翻朝直遥翻}
欺罔日月又鴻都門下招會羣小造作賦說見寵於時更相薦說|{
	更工衡翻}
旬月之間竝各拔擢樂松處常伯任芝居納言|{
	常伯侍中納言尚書處昌呂翻}
郤儉梁鵠各受豐爵不次之寵|{
	姓譜郤晉卿郤氏之後}
而令搢紳之徒委伏畎畮|{
	畮古畝字}
口誦堯舜之言身蹈絶俗之行|{
	行下孟翻}
棄捐溝壑不見逮及冠履倒易陵谷代處幸賴皇天垂象譴告周書曰天子見怪則修德諸侯見怪則修政卿大夫見怪則修職士庶人見怪則修身|{
	此逸書也}
唯陛下斥遠佞巧之臣|{
	遠于願翻}
速徵鶴鳴之士|{
	易曰鶴鳴在隂其子和之我有好爵吾與爾縻之繫辭曰君子居室言善則千里之外應之鶴鳴之士言士之修身踐言為時所稱者也}
斷絶尺一|{
	斷丁管翻}
抑止槃游冀上天還威衆變可弭議郎蔡邕對曰臣伏思諸異皆亡國之怪也天於大漢殷勤不已故屢出祅變以當譴責|{
	祅與妖同於驕翻}
欲令人君感悟改危即安今蜺墮雞化皆婦人干政之所致也前者乳母趙嬈貴重天下|{
	嬈奴鳥翻}
讒諛驕溢續以永樂門史霍玉|{
	永樂門史董太后宫官樂音洛}
依阻城社又為姦邪今道路紛紛復云有程大人者|{
	宫中耆宿皆稱中大人復扶又翻}
察其風聲將為國患宜高為隄防明設禁令深惟趙霍以為至戒今太尉張顥為王所進光祿勲偉璋有名貪濁|{
	偉姓璋名}
又長水校尉趙玹|{
	玹音玄}
屯騎校尉蓋升|{
	蓋古合翻}
竝叨時幸榮富優足宜念小人在位之咎退思引身避賢之福伏見廷尉郭禧純厚老成光禄大夫橋玄聰達方直故太尉劉寵忠實守正竝宜為謀主數見訪問|{
	數所角翻}
夫宰相大臣君之四體委任責成優劣已分不宜聽納小吏雕琢大臣也|{
	賢曰雕琢謂䥴削以成其罪也}
又尚方工技之作|{
	續漢志尚方掌上手工作御刀劔諸好器物技巨綺翻}
鴻都篇賦之文可且消息以示惟憂|{
	惟思也}
宰府孝亷士之高選近者以辟召不慎切責三公而今竝以小文超取選舉開請託之門違明王之典衆心不厭|{
	賢曰厭伏也音一葉翻}
莫之敢言臣願陛下忍而絶之思惟萬機以答天望聖朝既自約厲左右近臣亦宜從化人自抑損以塞咎戒|{
	塞悉則翻}
則天道虧滿鬼神福謙矣|{
	易曰天道虧盈而益謙鬼神害盈而福謙以盈為滿者避惠帝諱也}
夫君臣不密上有漏言之戒下有失身之禍|{
	易曰君不密則失臣臣不密則失身}
願寢臣表無使盡忠之吏受怨姦仇章奏帝覽而歎息因起更衣|{
	更工衡翻}
曹節於後竊視之悉宣語左右|{
	語牛倨翻}
事遂漏露其為邕所裁黜者側目思報初邕與大鴻臚劉郃素不相平|{
	臚陵如翻郃古合翻又曷閣翻}
叔父衛尉質又與將作大匠陽球有隙球即中常侍程璜女夫也璜遂使人飛章言邕質數以私事請託於郃郃不聽邕含隱切志欲相中|{
	賢曰中傷也郃古合翻數所角翻中竹仲翻}
於是詔下尚書召邕詰狀|{
	下遐稼翻下是下同詰去吉翻}
邕上書曰臣實愚戇|{
	戇陟降翻}
不顧後害陛下不念忠臣直言宜加掩蔽誹謗卒至|{
	卒讀曰猝}
便用疑怪臣年四十有六孤特一身得託名忠臣死有餘榮恐陛下於此不復聞至言矣|{
	復扶又翻}
於是下邕質於雒陽獄劾以仇怨奉公議害大臣大不敬棄市|{
	誣邕以請託不聽志欲中傷為仇怨奉公之吏三公九卿皆大臣也劾戶槩翻又戶得翻}
事奏中常侍河南呂強愍邕無罪力為伸請|{
	為于偽翻}
帝亦更思其章有詔減死一等與家屬髠鉗徙朔方不得以赦令除陽球使客追路刺邕|{
	刺七亦翻}
客感其義皆莫為用球又賂其部主|{
	部主州牧郡守也}
使加毒害所賂者反以其情戒邕由是得免 八月有星孛于天市|{
	孛蒲内翻}
九月太尉張顥罷以太常陳球為太尉 司空來豔薨 |{
	考異曰袁紀云豔以久病罷今從范書}
冬十月以屯騎校尉袁逢為司空 宋皇后無寵後宫幸姬衆共譖毁渤海王悝妃宋氏即后之姑也中常侍王甫恐后怨之|{
	悝被誅事見上熹平元年悝苦回翻}
因譖后挾左道祝詛|{
	祝職救翻詛莊助翻}
帝信之遂策收璽綬|{
	璽斯氏翻綬音受}
后自致暴室以憂死父不其鄉侯酆及兄弟竝被誅|{
	不其縣前漢屬琅邪郡後漢併省為鄉賢曰故城在今萊州即墨縣西南蓋其縣之鄉也其音基被皮義翻下同}
丙子晦日有食之尚書盧植上言凡諸黨錮多非其罪可加赦恕申宥回枉又宋后家屬竝以無辜委骸横尸不得歛葬|{
	歛力贍翻}
宜敇收拾以安遊魂又郡守刺史一月數遷宜依黜陟以章能否縱不九載可滿三歲|{
	賢曰書曰三載考績三考黜陟幽明孔安國注曰三年考功三考九年能否幽明有别升進其明者黜退其幽者此皆唐堯之法也載子亥翻}
又請謁希求一宜禁塞|{
	塞悉則翻}
選舉之事責成主者又天子之體理無私積宜弘大務蠲略細微帝不省|{
	省悉井翻}
十一月太尉陳球免 十二月丁巳以光禄大夫橋玄為太尉 鮮卑寇酒泉種衆日多|{
	種章勇翻}
緣邊莫不被毒|{
	被皮義翻}
詔中尚方|{
	即尚方也屬少府}
為鴻都文學樂松江覽等三十二人圖象立贊以勸學者|{
	為于偽翻}
尚書令陽球諫曰臣案松覽等皆出於微蔑|{
	蔑者微之甚幾於無也}
斗筲小人|{
	筲竹器容斗二升音所交翻}
依憑世戚附託權豪俛眉承睫|{
	睫即涉翻目毛也}
徼進明時|{
	徼一遥翻}
或獻賦一篇或鳥盈簡|{
	賢曰八體書有鳥象形以為字也}
而位升郎中形圖丹青亦有筆不點牘辭不辨心假手請字妖偽百品莫不蒙被殊恩蟬蜕滓濁|{
	賢曰說文曰蜕蟬蛇所解皮也音式鋭翻或音他外翻}
是以有識掩口|{
	謂掩口而笑也}
天下嗟歎臣聞圖象之設以昭勸戒欲令人君動鑒得失未聞豎子小人詐作文頌而可妄竊天官垂象圖素者也今太學東觀|{
	東觀在南宫觀古玩翻}
足以宣明聖化願罷鴻都之選以銷天下之謗書奏不省|{
	省悉井翻}
是歲初開西邸賣官|{
	開邸舍於西園因謂之西邸}
入錢各有差二千石二千萬四百石四百萬其以德次應選者半之或三分之一於西園立庫以貯之|{
	貯丁呂翻}
或詣闕上書占令長隨縣好醜豐約有賈|{
	占章瞻翻長知兩翻賈讀曰價}
富者則先入錢貧者到官然後倍輸又私令左右賣公卿公千萬卿五百萬初帝為侯時常苦貧及即位每歎桓帝不能作家居|{
	居積也}
曾無私錢故賣官聚錢以為私藏|{
	藏徂浪翻}
帝嘗問侍中楊奇曰朕何如桓帝對曰陛下之於桓帝亦猶虞舜比德唐堯帝不悦曰卿強項|{
	賢曰強項言不低屈也}
直楊震子孫死後必復致大鳥矣|{
	大鳥事見五十一卷安帝延光四年復扶又翻}
奇震之曾孫也 南匈奴屠特若尸逐就單于死子呼徵立

二年春大疫 三月司徒袁滂免以大鴻臚劉郃為司徒 |{
	考異曰袁紀二月丁巳滂免劉郃作劉邵今從范書}
乙丑太尉橋玄罷拜太中大夫以太中大夫段熲為太尉玄幼子遊門次為人所刼登樓求貨|{
	所謂刼質也}
玄不與司隸校尉河南尹圍守玄家不敢迫玄瞋目呼曰|{
	瞋七人翻呼火故翻}
姦人無狀玄豈以一子之命而縱國賊乎促令攻之玄子亦死玄因上言天下凡有刼質皆并殺之不得贖以財寶開張姦路由是刼質遂絶|{
	質音致}
京兆地震 司空袁逢罷以太常張濟為司空 夏四月甲戌朔日有食之 王甫曹節等姦虐弄權扇動内外太尉段熲阿附之節甫父兄子弟為卿校牧守令長者布滿天下所在貪暴|{
	校戶教翻守式又翻長知兩翻}
甫養子吉為沛相尤殘酷凡殺人皆磔尸車上隨其罪目宣示屬縣|{
	賢曰罪目罪名也磔陟格翻}
夏月腐爛則以繩連其骨周徧一郡乃止見者駭懼視事五年凡殺萬餘人尚書令陽球常拊髀發憤曰若陽球作司隷此曹子安得容乎既而球果遷司隷甫使門生於京兆界榷官財物七千餘萬|{
	前書音義曰□障也榷專也謂障餘人買賣而自取其利榷古岳翻}
京兆尹楊彪發其姦言之司隷|{
	京兆屬司隷所部}
彪賜之子也時甫休沐里舍|{
	里舍私第也}
熲方以日食自劾球詣闕謝恩因奏甫熲及中常侍淳于登袁赦封等罪惡|{
	姓譜封夏封父之後音吐嗑翻}
辛巳悉收甫熲等送雒陽獄及甫子永樂少府萌沛相吉|{
	樂音洛}
球自臨考甫等五毒備極萌先嘗為司隸乃謂球曰父子既當伏誅亦以先後之義少以楚毒假借老父|{
	少詩照翻}
球曰爾罪惡無狀死不滅責乃欲論先後求假借邪萌乃罵曰爾前奉事吾父子如奴奴敢反汝主乎今日臨阬相擠|{
	擠子細翻又則兮翻}
行自及也球使以土窒萌口箠扑交至|{
	箠止橤翻扑普卜翻}
父子悉死於杖下熲亦自殺乃僵磔甫尸於夏城門大署牓曰賊臣王甫盡没入其財產妻子皆徙比景球既誅甫欲以次表曹節等乃敇中都官從事曰|{
	中都官從事即都官從事主察舉百官犯法者中興以後專令掊擊貴戚}
且先去權貴大猾|{
	去羌呂翻}
乃議其餘耳公卿豪右若袁氏兒輩|{
	時諸袁以與袁赦同宗貴寵於世}
從事自辦之何須校尉邪權門聞之莫不屏氣|{
	屏必郢翻}
曹節等皆不敢出沐會順帝虞貴人葬|{
	虞貴人順帝母}
百官會喪還曹節見磔甫尸道次慨然抆淚曰|{
	賢曰抆拭也音亡粉翻}
我曹可自相食何宜使犬舐其汁乎|{
	舐池爾翻 考異曰袁紀云球會虞貴人葬還入夏城門曹節見謁於道旁球大罵曰賊臣曹節節收淚于車下而有是語今從范書}
語諸常侍今且俱入勿過里舍也|{
	語牛倨翻}
節直入省白帝曰陽球故酷暴吏前三府奏當免官以九江微功復見擢用|{
	事見上建寧六年復扶乂翻下同}
愆過之人好為妄作|{
	好呼到翻}
不宜使在司隸以騁毒虐帝乃徙球為衛尉時球出謁陵|{
	諸陵皆在司部故司隸出謁陵}
節敇尚書令召拜不得稽留尺一球被召急因求見帝曰臣無清高之行横蒙鷹犬之任|{
	謂司隸主噬姦非猶鷹犬也行下孟翻横戶孟翻}
前雖誅王甫段熲蓋狐狸小醜未足宣示天下願假臣一月必令豺狼鴟梟各服其辜|{
	梟堅堯翻}
叩頭流血殿上呵叱曰衛尉扞詔邪至于再三乃受拜於是曹節朱瑀等權勢復盛節領尚書令郎中梁人審忠上書曰|{
	審姓也漢初有審食其}
陛下即位之初未能萬機皇太后念在撫育權時攝政故中常侍蘇康管霸應時誅殄太傅陳蕃大將軍竇武考其黨與志清朝政|{
	朝直遥翻}
華容侯朱瑀知事覺露禍及其身遂興造逆謀作亂王室撞蹹省闥|{
	撞直江翻蹹與踏同}
執奪璽綬迫脅陛下聚會羣臣離間骨肉母子之恩|{
	間古莧翻}
遂誅蕃武及尹勲等因共割裂城社自相封賞|{
	事見上卷建寧元年}
父子兄弟被蒙尊榮|{
	被皮義翻}
素所親厚布在州郡或登九列或據三司|{
	九列九卿也三司三公也}
不惟祿重位尊之責|{
	惟思也}
而苟營私門多蓄財貨繕修第舍連里竟巷盜取御水以作漁釣|{
	賢曰水入宫苑為御水}
車馬服玩擬於天家|{
	天家猶王家也君天也故謂之天家}
羣公卿士杜口吞聲莫敢有言州牧郡守承順風旨辟召選舉釋賢取愚故蟲蝗為之生夷寇為之起|{
	為于偽翻}
天意憤盈積十餘年故頻歲日食於上地震於下所以譴戒人主欲令覺悟誅鉏無狀昔高宗以雉雊之變故獲中興之功|{
	高宗肜日有飛雉升鼎耳而雊懼而修德殷以中興}
近者神祇啟悟陛下發赫斯之怒|{
	詩云王赫斯怒}
故王甫父子應時馘路人士女莫不稱善若除父母之讎誠怪陛下復忍孽臣之類不悉殄滅|{
	忍謂含忍也隱忍也孽魚列翻}
昔秦信趙高以危其國|{
	事見八卷秦二世紀}
吳使刑臣身遘其禍|{
	左傳吳伐越獲俘焉以為閽使守舟吳子餘祭觀舟閽以刀弑之}
今以不忍之恩赦夷族之罪姦謀一成悔亦何及臣為郎十五年皆耳目聞見瑀之所為誠皇天所不復赦願陛下留漏刻之聽|{
	漏之度晝夜百刻留漏刻之聽言少須臾留聽也}
裁省臣表|{
	省悉井翻}
埽滅醜類以答天怒與瑀考驗有不如言願受湯鑊之誅妻子并徙以絶妄言之路章寢不報中常侍呂強清忠奉公帝以衆例封為都鄉侯強固辭不受因上疏陳事曰臣聞高祖重約非功臣不侯所以重天爵明勸戒也中常侍曹節等宦官祐薄品卑人賤讒諂媚主佞邪徼寵|{
	徼一遥翻又古堯翻}
有趙高之禍未被轘裂之誅|{
	賢曰轘裂以車裂也}
陛下不悟妄授茅土開國承家小人是用|{
	易曰開國承家小人勿用}
又并及家人重金兼紫|{
	賢曰金印紫綬重兼言累積也重直龍翻}
交結邪黨下比羣佞|{
	比毗至翻}
隂陽乖刺|{
	刺盧達翻}
稼穡荒蕪人用不康罔不由兹臣誠知封事已行言之無逮|{
	封事謂封爵之事也}
所以冒死干觸陳愚忠者實願陛下損改既謬從此一止臣又聞後宫采女數千餘人衣食之費日數百金比穀雖賤而戶有饑色|{
	比頻寐翻言近者也}
案法當貴而今更賤者由賦發繁數以解縣官|{
	數所角翻賢曰縣官調發既多故賤糶穀以供之解居隘翻發也}
寒不敢衣飢不敢食民有斯戹而莫之卹宫女無用填積後庭天下雖復盡力耕桑猶不能供|{
	復扶又翻下同}
又前召議郎蔡邕對問於金商門邕不敢懷道迷國|{
	盖引論語迷邦之言不曰邦者避高帝諱}
而切言極對毁刺貴臣譏呵宦官陛下不密其言至令宣露羣邪項領膏唇拭舌|{
	賢曰詩曰駕彼四牡四牡項領註云項大也四牡者人所駕今但養大其領不肯為用諭大臣自恣王不能使也膏唇拭舌謂欲讒毁故也}
競欲咀嚼造作飛條|{
	賢曰飛條飛書也}
陛下回受誹謗致邕刑罪室家徙放老幼流離豈不負忠臣哉今羣臣皆以邕為戒上畏不測之難下懼劒客之害|{
	賢曰謂陽球使客追刺邕也難乃旦翻}
臣知朝廷不復得聞忠言矣故太尉段熲武勇冠世|{
	冠古玩翻}
習於邊事垂髪服戎|{
	賢曰垂髪謂童子也}
功成皓首歷事二主|{
	二主靈帝桓帝}
勲烈獨昭陛下既已式序|{
	式用也式序者用叙其功也}
位登台司而為司隸校尉陽球所見誣脅一身既斃而妻子遠播|{
	播遷也}
天下惆悵|{
	惆丑鳩翻}
功臣失望宜徵邕更加授任反熲家屬則忠貞路開衆怨以弭矣帝知其忠而不能用 丁酉赦天下 上禄長和海|{
	賢曰上祿縣屬武都郡今成州縣姓譜和本自羲和之後一云卞和之後}
上言禮從祖兄弟别居異財恩義已輕服屬疎末而今黨人錮及五族既乖典訓之文有謬經常之灋帝覽之而悟於是黨錮自從祖以下皆得解釋|{
	從祖緦麻服從才用翻}
五月以衛尉劉寛為太尉 護匈奴中郎將張修與南單于呼徵不相能修擅斬之更立右賢王羌渠為單于|{
	更工衡翻}
秋七月修坐不先請而擅誅殺檻車徵詣廷尉死 初司徒劉郃兄侍中鯈與竇武同謀俱死|{
	鯈直留翻桓紀作劉鯈}
永樂少府陳球說郃曰|{
	賢曰桓帝母孝崇皇后宫曰永樂置太僕少府余據此時帝母孝仁董太后居永樂宫非孝崇后也說輸芮翻}
公出自宗室位登台鼎天下瞻望社稷鎮衛豈得雷同容容無違而已今曹節等放縱為害而久在左右又公兄侍中受害節等今可表徙衛尉陽球為司隸校尉以次收節等誅之政出聖主天下太平可翹足而待也郃曰凶豎多耳目恐事未會先受其禍尚書劉納曰為國棟梁傾危不持焉用彼相邪|{
	論語孔子曰危而不持顛而不扶則將焉用彼相矣焉於䖍翻}
郃許諾亦與陽球結謀球小妻程璜之女由是節等頗得聞知乃重賂璜且脅之璜懼迫以球謀告節節因共白帝曰郃與劉納陳球陽球交通書疏謀議不軌帝大怒冬十月甲申劉郃陳球劉納陽球皆下獄死|{
	下遐稼翻}
巴郡板楯蠻反|{
	楯食尹翻}
遣御史中丞蕭瑗督益州刺史討之不克|{
	瑗于眷翻}
十二月以光祿勲楊賜為司徒 鮮卑寇幽并二州三年春正月癸酉赦天下 夏四月江夏蠻反|{
	夏戶雅翻}
秋酒泉地震 冬有星孛于狼弧|{
	晉書天文志狼一星在東井東南弧九星在狼東南孛蒲内翻}
鮮卑寇幽并二州 十二月立貴人何氏為皇后 |{
	考異曰袁紀在十一月今從范書}
徵后兄潁川太守進為侍中后本南陽屠家以選入掖庭生皇子辨故立之|{
	為何進謀誅宦官敗國亡家張本}
是歲作畢圭靈昆苑|{
	賢曰畢圭苑有二東畢圭苑周一千五百步中有魚梁臺西畢圭苑周三千三百步並在雒陽宣平門外}
司徒楊賜諫曰先帝之制左開鴻池右作上林不奢不約以合禮中今猥規郊城之地以為苑囿壞沃衍|{
	杜預注左傳曰衍沃平美之地也壞音怪}
廢田園驅居民畜禽獸|{
	畜許六翻}
殆非所謂若保赤子之義|{
	書曰若保赤子惟民其康乂}
今城外之苑已有五六|{
	賢曰陽嘉元年起西苑延熹二年造顯陽苑雒陽宫殿名有平樂苑上林苑恒帝延熹元年置鴻德苑}
可以逞情意順四節也|{
	賢曰逞快也四節謂春蒐夏苗秋獮冬狩}
宜惟夏禹卑宫太宗露臺之意|{
	惟思也}
以尉下民之勞書奏帝欲止以問侍中任芝樂松|{
	任音壬考異曰范書云中常侍樂松松本鴻都文學必非中常侍袁紀云侍中今從之}
對曰昔文王之囿百里人以為小齊宣五里人以為大|{
	齊宣王問於孟子曰文王之囿方七十里人猶以為小寡人之囿方四十里人以為大何也對曰文王之囿方七十里芻蕘者往焉雉兎者往焉與人同之人以為小不亦宜乎今王之囿殺其麋鹿者如殺人之罪人以為大不亦宜乎此云五里微與孟子異}
今與百姓共之無害於政也帝悦遂為之 巴郡板楯蠻反 蒼梧桂陽賊攻郡縣零陵太守楊琁制馬車數十乘以排囊盛石灰於車上|{
	賢曰排囊即今囊袋也排音蒲拜翻盛時征翻}
繫布索於馬尾|{
	索昔各翻}
乂為兵車專彀弓弩及戰令馬車居前順風鼓灰賊不得視因以火燒布然馬驚犇突賊陣因使後車弓弩亂發征鼓鳴震羣盜波駭破散|{
	波駭者蓋喻以物擊水一波動萬波隨而駭動}
追逐傷斬無數梟其渠帥|{
	梟者斬首而梟之木上也梟堅堯翻帥所類翻}
郡境以清荆州刺史趙凱誣奏琁實非身親破賊而妄有其功琁與相章奏凱有黨助遂檻車徵琁防禁嚴密無由自訟乃噬臂出血書衣為章具陳破賊形埶及言凱所誣狀潛令親屬詣闕通之詔書原琁拜議郎凱受誣人之罪琁喬之弟也|{
	楊喬見上卷桓帝永康元年}


資治通鑑卷五十七














































































































































