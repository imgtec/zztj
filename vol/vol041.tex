<!DOCTYPE html PUBLIC "-//W3C//DTD XHTML 1.0 Transitional//EN" "http://www.w3.org/TR/xhtml1/DTD/xhtml1-transitional.dtd">
<html xmlns="http://www.w3.org/1999/xhtml">
<head>
<meta http-equiv="Content-Type" content="text/html; charset=utf-8" />
<meta http-equiv="X-UA-Compatible" content="IE=Edge,chrome=1">
<title>資治通鑒_42-資治通鑑卷四十一_42-資治通鑑卷四十一</title>
<meta name="Keywords" content="資治通鑒_42-資治通鑑卷四十一_42-資治通鑑卷四十一">
<meta name="Description" content="資治通鑒_42-資治通鑑卷四十一_42-資治通鑑卷四十一">
<meta http-equiv="Cache-Control" content="no-transform" />
<meta http-equiv="Cache-Control" content="no-siteapp" />
<link href="/img/style.css" rel="stylesheet" type="text/css" />
<script src="/img/m.js?2020"></script> 
</head>
<body>
 <div class="ClassNavi">
<a  href="/24shi/">二十四史</a> | <a href="/SiKuQuanShu/">四库全书</a> | <a href="http://www.guoxuedashi.com/gjtsjc/"><font  color="#FF0000">古今图书集成</font></a> | <a href="/renwu/">历史人物</a> | <a href="/ShuoWenJieZi/"><font  color="#FF0000">说文解字</a></font> | <a href="/chengyu/">成语词典</a> | <a  target="_blank"  href="http://www.guoxuedashi.com/jgwhj/"><font  color="#FF0000">甲骨文合集</font></a> | <a href="/yzjwjc/"><font  color="#FF0000">殷周金文集成</font></a> | <a href="/xiangxingzi/"><font color="#0000FF">象形字典</font></a> | <a href="/13jing/"><font  color="#FF0000">十三经索引</font></a> | <a href="/zixing/"><font  color="#FF0000">字体转换器</font></a> | <a href="/zidian/xz/"><font color="#0000FF">篆书识别</font></a> | <a href="/jinfanyi/">近义反义词</a> | <a href="/duilian/">对联大全</a> | <a href="/jiapu/"><font  color="#0000FF">家谱族谱查询</font></a> | <a href="http://www.guoxuemi.com/hafo/" target="_blank" ><font color="#FF0000">哈佛古籍</font></a> 
</div>

 <!-- 头部导航开始 -->
<div class="w1180 head clearfix">
  <div class="head_logo l"><a title="国学大师官网" href="http://www.guoxuedashi.com" target="_blank"></a></div>
  <div class="head_sr l">
  <div id="head1">
  
  <a href="http://www.guoxuedashi.com/zidian/bujian/" target="_blank" ><img src="http://www.guoxuedashi.com/img/top1.gif" width="88" height="60" border="0" title="部件查字,支持20万汉字"></a>


<a href="http://www.guoxuedashi.com/help/yingpan.php" target="_blank"><img src="http://www.guoxuedashi.com/img/top230.gif" width="600" height="62" border="0" ></a>


  </div>
  <div id="head3"><a href="javascript:" onClick="javascript:window.external.AddFavorite(window.location.href,document.title);">添加收藏</a>
  <br><a href="/help/setie.php">搜索引擎</a>
  <br><a href="/help/zanzhu.php">赞助本站</a></div>
  <div id="head2">
 <a href="http://www.guoxuemi.com/" target="_blank"><img src="http://www.guoxuedashi.com/img/guoxuemi.gif" width="95" height="62" border="0" style="margin-left:2px;" title="国学迷"></a>
  

  </div>
</div>
  <div class="clear"></div>
  <div class="head_nav">
  <p><a href="/">首页</a> | <a href="/ShuKu/">国学书库</a> | <a href="/guji/">影印古籍</a> | <a href="/shici/">诗词宝典</a> | <a   href="/SiKuQuanShu/gxjx.php">精选</a> <b>|</b> <a href="/zidian/">汉语字典</a> | <a href="/hydcd/">汉语词典</a> | <a href="http://www.guoxuedashi.com/zidian/bujian/"><font  color="#CC0066">部件查字</font></a> | <a href="http://www.sfds.cn/"><font  color="#CC0066">书法大师</font></a> | <a href="/jgwhj/">甲骨文</a> <b>|</b> <a href="/b/4/"><font  color="#CC0066">解密</font></a> | <a href="/renwu/">历史人物</a> | <a href="/diangu/">历史典故</a> | <a href="/xingshi/">姓氏</a> | <a href="/minzu/">民族</a> <b>|</b> <a href="/mz/"><font  color="#CC0066">世界名著</font></a> | <a href="/download/">软件下载</a>
</p>
<p><a href="/b/"><font  color="#CC0066">历史</font></a> | <a href="http://skqs.guoxuedashi.com/" target="_blank">四库全书</a> |  <a href="http://www.guoxuedashi.com/search/" target="_blank"><font  color="#CC0066">全文检索</font></a> | <a href="http://www.guoxuedashi.com/shumu/">古籍书目</a> | <a   href="/24shi/">正史</a> <b>|</b> <a href="/chengyu/">成语词典</a> | <a href="/kangxi/" title="康熙字典">康熙字典</a> | <a href="/ShuoWenJieZi/">说文解字</a> | <a href="/zixing/yanbian/">字形演变</a> | <a href="/yzjwjc/">金 文</a> <b>|</b>  <a href="/shijian/nian-hao/">年号</a> | <a href="/diming/">历史地名</a> | <a href="/shijian/">历史事件</a> | <a href="/guanzhi/">官职</a> | <a href="/lishi/">知识</a> <b>|</b> <a href="/zhongyi/">中医中药</a> | <a href="http://www.guoxuedashi.com/forum/">留言反馈</a>
</p>
  </div>
</div>
<!-- 头部导航END --> 
<!-- 内容区开始 --> 
<div class="w1180 clearfix">
  <div class="info l">
   
<div class="clearfix" style="background:#f5faff;">
<script src='http://www.guoxuedashi.com/img/headersou.js'></script>

</div>
  <div class="info_tree"><a href="http://www.guoxuedashi.com">首页</a> > <a href="/SiKuQuanShu/fanti/">四库全书</a>
 > <h1>资治通鉴</h1> <!--         下载:【右键另存为】即可 --></div>
  <div class="info_content zj clearfix">
  
<div class="info_txt clearfix" id="show">
<center style="font-size:24px;">42-資治通鑑卷四十一</center>
    資治通鑑卷四十一   宋 司馬光 撰<br />
<br />
  胡三省 音註<br />
<br />
  漢紀三十三【起強圉大淵獻盡屠維赤奮若凡三年】<br />
<br />
  世祖光武皇帝上之下<br />
<br />
  建武三年春正月甲子以馮異為征西大將軍【晉書職官志曰四征起於漢代謂此】鄧禹慙於受任無功數以飢卒徼赤眉戰輒不利【數所角翻徼一遥翻】乃率車騎將軍鄧弘等自河北度至湖【地理志河北縣屬河東郡湖縣屬京兆賢曰湖縣故城在今虢州湖城縣西南】要馮異共攻赤眉【要一遙翻下同】異曰異與賊相拒數十日雖虜獲雄將餘衆尚多可稍以恩信傾誘難卒用兵破也【卒讀曰猝下卒起同】上今使諸將屯澠池要其東【澠彌兖翻】而異擊其西一舉取之此萬成計也【十事九成猶有一不中萬事萬成言筭無遺計也要一遥翻】禹弘不從弘遂大戰移日【言日景移也】赤眉陽敗棄輜重走【重直用翻】車皆載土以豆覆其上【覆敷救翻】兵士飢爭取之赤眉引還擊弘弘軍潰亂異與禹合兵救之赤眉小卻異以士卒飢倦可且休禹不聽復戰大為所敗【復扶又翻敗補邁翻】死傷者三千餘人禹以二十四騎脫歸宜陽異棄馬奔走上回谿阪【杜佑通典曰冋谿在河南永寧縣東北俗名回坑長四里闊二丈深二丈五尺自漢以前道皆由此酈道元云曹公西討惡南路之險更開北道】與麾下數人歸營收其散卒復堅壁自守【復扶又翻】 辛巳立四親廟於雒陽祀父南頓君以上至舂陵節侯【禮天子立親廟四今依以立舂陵節侯鬰林太守鉅鹿都尉南頓令廟】 壬午大赦 閏月乙巳鄧禹上大司徒梁侯印綬【上時掌翻下同】詔還梁侯印綬以為右將軍 馮異與赤眉約期會戰使壯士變服與赤眉同伏於道側旦日赤眉使萬人攻異前部異少出兵以救之【所以示弱也】賊見埶弱遂悉衆攻異異乃縱兵大戰日昃賊氣衰伏兵卒起衣服相亂赤眉不復識别【卒讀曰猝復扶又翻别彼列翻】衆遂驚潰追擊大破之於崤底【崤谷之底也賢曰即崤阪也在今洛州永寜縣西北】降男女八萬人【降戶江翻】帝降璽書勞異曰【勞力到翻】始雖垂翅回谿終能奮翼澠池可謂失之東隅收之桑榆【賢曰淮南子曰至於衡陽是謂隅中又前書谷永曰太白出西方六十日法當參天今已過期尚在桑榆間桑榆謂晚也余按淮南子曰西日垂景在樹端謂之桑榆】方論功賞以荅大勲赤眉餘衆東向宜陽甲辰帝親勒六軍嚴陳以待之【陳讀曰陣】赤眉忽遇大軍驚震不知所謂乃遣劉恭乞降曰盆子將百萬衆降陛下何以待之帝曰待汝以不死耳丙午盆子及丞相徐宣以下三十餘人肉袒降上所得傳國璽綬【璽斯氏翻綬音受】積兵甲宜陽城西與熊耳山齊【賢曰宜陽縣故城韓國城也在今洛州福昌縣東水經註曰洛水之北有熊耳山雙巒競舉狀同熊耳在宜陽西宋白曰宜陽故城在福昌縣東十三里】赤眉衆尚十餘萬人帝令縣厨皆賜食【宜陽縣廚也】明旦大陳兵馬臨雒水【帝改洛為雒】令盆子君臣列而觀之帝謂樊崇等曰得無悔降乎朕今遣卿歸營勒兵鳴鼔相攻決其勝負不欲彊相服也【彊其兩翻】徐宣等叩頭曰臣等出長安東都門君臣計議歸命聖德百姓可與樂成難與圖始【樂音洛】故不告衆耳今日得降猶去虎口歸慈母誠歡誠喜無所恨也帝曰卿所謂鐵中錚錚傭中佼佼者也【賢曰說文曰錚錚金地鐵之錚言微有剛利也錚初耕翻佼古巧翻詩佼人僚兮今相傳胡巧翻言佼佼者凡庸之人稍為勝也】戊申還自宜陽帝令樊崇等各與妻子居雒陽賜之田宅其後樊崇逢安反誅楊音徐宣卒於鄉里帝憐盆子以為趙王郎中【趙王良帝叔父也以盆子為其國郎中】後病失明賜滎陽均輸官地使食其税終身【賢曰均輸官名屬司農桓寛鹽鐵論云郡國諸侯各以其方物貢輸往來物多苦惡不償其費故郡國置均輸官以相紹運故曰均輸】劉恭為更始報仇殺謝禄【禄殺更始事見上卷元年為于偽翻】自繫獄帝赦不誅二月劉永立董憲為海西王【賢曰海西縣屬琅邪郡】永聞伏隆至劇【地理志劇縣屬北海郡春秋紀國之地杜佑曰漢劇縣故城在夀光縣南】亦遣使立張步為齊王步貪王爵猶豫未決隆曉譬曰高祖與天下約非劉氏不王今可得為十萬戶侯耳步欲留隆與共守二州【二州青州徐州也】隆不聽求得反命步遂執隆而受永封隆遣間使上書曰【間古莧翻使疏吏翻】臣隆奉使無狀【賢曰言罪大也】受執凶逆雖在困阨授命不顧又吏民知步反畔心不附之願以時進兵無以臣隆為念臣隆得生到闕廷受誅有司此其大願若令沒身寇手以父母昆弟長累陛下【賢曰累托也音力偽翻】陛下與皇后太子永享萬國與天無極帝得隆奏召其父湛流涕示之曰恨不且許而遽求還也其後步遂殺之帝方北憂漁陽南事梁楚故張步得專集齊地據郡十二焉【步據城陽琅邪高密膠東東萊北海齊千乘濟南平原泰山菑川十二郡】 帝幸懷 吳漢率耿弇蓋延擊青犢於軹西大破降之【賢曰軹縣屬河内郡故城在今洛州濟源縣東南蓋古盍翻】 三月壬寅以司直伏湛為大司徒涿郡太守張豐反【郡國志涿郡在雒陽東北千八百里】自稱無上大將軍與彭寵連兵朱浮以帝不自征彭寵上疏求救詔報曰往年赤眉跋扈長安【賢曰跋扈猶言暴横也】吾策其無穀必東果來歸附今度此反虜【度徒洛翻】埶無久全其中必有内相斬者今軍資未充故須後麥耳【須待也】浮城中糧盡人相食會耿況遣騎來救浮乃得脱身走薊城遂降於彭寵【考異曰朱浮傳尚書令侯霸奏浮敗亂幽州構成寵罪徒勞軍師不能死節罪當伏誅按霸明年乃為尚書令蓋追劾之】寵自稱燕王攻拔右北平上谷數縣賂遺匈奴【遺于季翻】借兵為助又南結張步及富平獲索諸賊皆與交通帝自將征鄧奉至堵陽【堵陽縣屬南陽郡杜佑曰唐州方城縣漢堵陽縣應劭曰】<br />
<br />
  【堵陽景帝改為順陽二說不同】奉逃歸淯陽董訢降【訢音欣降戶江翻下同】夏四月帝追奉至小長安與戰大破之奉肉袒因朱祜降【去年奉禽祜今因祜而降】帝憐奉舊功臣【奉鄧晨之兄子也】且釁起吳漢【事見上卷上年】欲全宥之岑彭耿弇諫曰鄧奉背恩反逆【背蒲妹翻】暴師經年陛下旣至不知悔善而親在行陳【陳讀曰陣】兵敗乃降若不誅奉無以懲惡於是斬之復朱祜位 延岑旣破赤眉即拜置牧守欲據關中時關中衆宼猶盛岑據藍田王歆據下邽【賢曰秦武公伐邽戎置以隴西有上邽故此云下】芳丹據新豐【芳姓也風俗通有漢幽州刺史芳乘】蔣震據霸陵張邯據長安公孫守據長陵楊周據谷口呂鮪據陳倉角閎據汧【角姓也漢有角善叔汧苦堅翻】駱延據盩厔【盩厔音舟窒姓譜齊太公之後有公子駱子孫以為氏史記秦之先有大駱】任良據鄠【鄠音戶】汝章據槐里【汝姓也商有汝鳩汝方春秋晉有汝齊汝寛】各稱將軍擁兵多者萬餘人少者數千人轉相攻擊馮異且戰且行屯軍上林苑中【異自崤谷之勝引兵而西且戰且行進屯上林苑中】延岑引張邯任良共擊異異擊大破之諸營保附岑者皆來降【保與堡同】岑遂自武關走南陽【走音奏】時百姓饑餓黄金一斤易豆五升道路斷隔委輸不至【委於偽翻輸春遇翻】馮異軍<br />
<br />
  士悉以果實為糧詔拜南陽趙匡為右扶風將兵助異并送縑穀異兵穀漸盛乃稍誅擊豪傑不從令者襃賞降附有功勞者悉遣諸營渠帥詣京師【帥所類翻】散其衆歸本業威行關中唯呂鮪張邯蔣震遣使降蜀【鮪于軌翻邯下甘翻】其餘悉平 吳漢率驃騎大將軍杜茂等七將軍圍蘇茂於廣樂周建招集得十餘萬人救之【周建劉永將也】漢迎與之戰不利墮馬傷䣛還營【䣛與膝同】建等遂連兵入城諸將謂漢曰大敵在前而公傷卧衆心懼矣【三軍之氣以將為主故云然】漢乃勃然裹創而起【創初良翻】椎牛饗士慰勉之士氣自倍旦日蘇茂周建出兵圍漢漢奮擊大破之茂走還湖陵睢陽人反城迎劉永蓋延率諸將圍之【反音幡蓋古盍翻】吳漢留杜茂陳俊守廣樂自將兵助延圍睢陽【睢音雖】車駕自小長安引還令岑彭率傅俊臧宫劉宏等三萬餘人南擊秦豐五月己酉車駕還宫 乙卯晦日有食之 六月壬戌大赦 延岑攻南陽得數城建威大將軍耿弇與戰于穰【地理志穰縣屬南陽郡】大破之岑與數騎走東陽與秦豐合豐以女妻之【走音奏妻七細翻】建義大將軍朱祜率祭遵等與岑戰於東陽破之【賢曰東陽聚名也故城在今鄧州南臨淮郡復有東陽縣非此地也余據郡國志南陽淯陽縣有東陽聚】岑走歸秦豐祜遂南與岑彭等軍合延岑護軍鄧仲況擁兵據隂縣【賢曰隂縣屬南陽郡故城在今襄州穀城縣界北水經注沔水南逕穀城東又南過隂縣西宋白曰今光化軍本隂縣地】而劉歆孫龔為其謀主前侍中扶風蘇竟以書說之【前此帝嘗用竟為侍中說翰芮翻】仲況與龔降竟終不伐其功隱身樂道夀終於家【樂音洛】秦豐拒岑彭於鄧【地理志鄧縣屬南陽郡春秋之鄧國也】秋七月彭擊破之進圍豐於黎丘别遣積弩將軍傅俊將兵徇江東揚州悉定蓋延圍睢陽百日劉永蘇茂周建突出將走鄼【此沛郡之鄼縣也賢曰今亳州縣音在何翻】延追擊之急永將慶吾斬永首降【姓譜齊大夫慶氏之後】蘇茂周建犇垂惠【郡國志沛郡山桑縣有垂惠聚賢曰在今亳州山桑縣西北一名禮城杜佑通典曰垂惠聚在亳州蒙城縣西北】共立永子紆為梁王佼彊犇保西防【佼古巧翻又音効】冬十月壬申上幸舂陵祠園廟【舂陵節侯以下四世園廟也】 耿弇從容言於帝【從千容翻】自請北收上谷兵未發者定彭寵於漁陽取張豐於涿郡還收富平獲索東攻張步以平齊地帝壯其意許之十一月乙未帝還自舂陵 是歲李憲稱帝置百官擁九城衆十餘萬【廬江十二城憲所得者九城耳】 帝謂太中大夫來歙曰【姓譜郲子姓商之支孫食采於郲因以為氏後避難去邑漢功臣表有侯來蒼歙許及翻】今西州未附【西州謂隗囂也】子陽稱帝【子陽公孫述字】道里阻遠諸將方務關東思西州方畧未知所在歙曰臣嘗與隗囂相遇長安其人始起以漢為名【事見三十九卷更始元年】臣願得奉威命開以丹青之信【揚子曰聖人之言炳若丹青】囂必束手自歸則述自亡之埶不足圖也帝然之始令歙使於囂【使疏吏翻下同】囂旣有功於漢又受鄧禹爵署【事見上卷元年】其腹心議者多勸通使京師囂乃奉奏詣闕帝報以殊禮言稱字用敵國之儀所以慰藉之甚厚【賢曰慰安也藉薦也言安慰而薦藉之也】<br />
<br />
  四年正月甲申大赦 二月壬子上行幸懷壬申還雒陽 延岑復宼順陽【郡國志順陽縣屬南陽郡順水東南入蔡括地志順陽故城在鄧州穰縣西三十里楚之郇邑也復扶又翻下同】遣鄧禹將兵擊破之岑犇漢中公孫述以岑為大司馬封汝寧王 田戎聞秦豐破恐懼欲降【降戶江翻下同】其妻兄辛臣圖彭寵張步董憲公孫述等所得郡國以示戎曰雒陽地如掌耳【如掌喻其狹也】不如且按甲以觀其變戎曰以秦王之彊猶為征南所圍吾降決矣【岑彭時為征南大將軍故戎云然】乃留辛臣使守夷陵自將兵沿江泝沔上黎丘【自夷陵沿江而下至沔口自沔口泝沔而上可至黎丘也上時掌翻】辛臣於後盜戎珍寶從間道先降于岑彭【間古莧翻】而以書招戎曰宜以時降無拘前計戎疑臣賣已灼龜卜降兆中坼【周禮菙氏凡卜以明火爇燋吹其焌契以授卜師鄭玄曰燋焌用荆菙之類兆者灼龜發於火其形可占者菙時髓翻燋哉約翻焌音俊又子寸翻】遂復反與秦豐合岑彭擊破之戎亡歸夷陵 夏四月丁巳上行幸鄴己巳幸臨平【賢曰縣名屬鉅鹿郡故城在今定州鼔城縣東南】遣吳漢陳俊王梁擊破五校於臨平鬲縣五姓共逐守長據城而反【賢曰鬲縣屬平原郡故城在今德州西北五姓蓋當土強宗豪右鬲音革余謂守長者守鬲縣長非正官也長知兩翻下同】諸將爭欲攻之吳漢曰使鬲反者守長罪也敢輕冒進兵者斬乃移檄告郡使收守長而使人謝城中五姓大喜即相率降諸將乃服曰不戰而下城非衆所及也 五月上幸元氏辛巳幸盧奴將親征彭寵伏湛諫曰今兖豫青冀中國之都而宼賊從横未及從化【從子容翻】漁陽邊外荒耗【邊外者邊於外夷也】豈足先圖陛下捨近務遠棄易求難【易以䜴翻】誠臣之所惑也上乃還 帝遣建議大將軍朱祜建威大將軍耿弇征虜將軍祭遵【祭則界翻】驍騎將軍劉喜討張豐於涿郡祭遵先至急攻豐禽之初豐好方術有道士言豐當為天子【西都有方士東都因稱為道士好呼到翻】以五綵囊裹石繫豐肘云石中有玉璽豐信之遂反旣執當斬猶曰肘石有玉璽傍人為椎破之【為于偽翻下同】豐乃知被詐仰天歎曰當死無恨上詔耿弇進擊彭寵弇以父况與寵同功【况與寵同有助漢之功事見上第三十九卷更始二年】又兄弟無在京師者不敢獨進求詣雒陽詔報曰將軍舉宗為國功効尤著何嫌何疑而欲求徵况聞之更遣弇弟國入侍時祭遵屯良鄉劉喜屯陽鄉【賢曰良鄉陽郷皆縣名並屬涿郡陽鄉故城在今幽州故安縣西北宋白曰良鄉在燕為中都漢為良鄉縣】彭寵引匈奴兵欲擊之耿况使其子舒襲破匈奴兵斬兩王寵乃退走 六月辛亥車駕還宫 秋七月丁亥上幸譙 【考異曰袁紀六月幸譙今從范書】遣捕虜將軍馬武騎都尉王霸圍劉紆周建於垂惠 董憲將賁休以蘭陵降【賢曰前書賁赫音肥今姓作賁音奔蘭陵縣屬東海郡故城在今沂州丞縣東】憲聞之自郯圍之【賢曰郯縣屬東海郡故城在今泗州下邳縣東北郯音談】蓋延及平狄將軍山陽龎萌在楚【楚彭城也】請往救之帝敕曰可直往擣郯【賢曰擣擊也余謂擣擣虚也此兵法所謂攻其必救也】則蘭陵自解延等以賁休城危遂先赴之憲逆戰而陽敗退延等因拔圍入城明日憲大出兵合圍延等懼遽出突走因往攻郯帝讓之曰間欲先赴郯者以其不意故耳今旣犇走賊計已立圍豈可解乎延等至郯果不能克而董憲遂拔蘭陵殺賁休八月戊午上幸夀春【地理志壽春縣屬九江郡賢曰今夀州縣】遣揚武將軍南陽馬成率誅虜將軍南陽劉隆等三將軍發會稽丹陽九江六安四郡兵擊李憲九月圍憲於舒【地理志廬江郡治舒縣賢曰故城在今廬州廬江縣西會古外翻】王莽末天下亂臨淮大尹河南侯霸獨能保全其郡【郡國志臨淮郡在雒陽東千四百里】帝徵霸會夀春拜尚書令時朝廷無故典又少舊臣【少詩沼翻】霸明習故事收録遺文條奏前世善政法度施行之 冬十月甲寅車駕還宫 隗囂使馬援往觀公孫述援素與述同里閈相善【援與述皆茂陵人說文曰閈閭也侯旰翻】以為旣至當握手歡如平生而述盛陳陛衛以延援入交拜禮畢使出就舘更為援制都布單衣【何承天纂文曰都致錯履無極皆布名方言曰單衣江淮南楚之間謂之褋關之東西謂之禪衣為于偽翻】交讓冠會百官於宗廟中立舊交之位述鸞旗旄騎【鸞旗註見十三卷文帝元年旄騎旄頭騎也秦穆公伐南山大梓有一青牛出走入豐水中其後牛出豐水中使騎擊之不勝有騎墮地復上髪解牛畏之入不出故置旄頭騎以前驅】警蹕就車磬折而入【賢曰磬折屈身如磬之曲折敬也孔穎達曰磬折者屈身如磬之折殺案考工記云磬氏為磬倨句一矩有半鄭云必先度一矩為句一矩為股而求其弦既以一矩有半觸其弦則磬之倨句也是磬之折殺其形必曲人之倚式亦當然也】禮饗官屬甚盛欲授援以封侯大將軍位賓客皆樂留【樂音洛】援曉之曰天下雄雌未定公孫不吐哺走迎國士【周公一飯三吐哺以下天下之士】與圖成敗反修飾邊幅【賢曰言若布帛修整其邊幅也】如偶人形此子何足久稽天下士乎【稽留也】因辭歸謂囂曰子陽井底蛙耳【言志識褊狹如坎井之蛙】而妄自尊大不如專意東方【東方謂雒陽也】囂乃使援奉書雒陽援初到良久中黄門引入【中黄門宦者也屬少府】帝在宣德殿南廡下【廡音武堂下周屋也】但幘坐迎笑【董巴曰古者有冠無幘其戴也加首有頍所以安物詩曰有頍者弁謂此也秦加武將首飾為絳袙其後稍稍作顔題漢興續其顔却摞之施巾連題却覆之今喪幘是其制也名之曰幘幘者頭首嚴賾也至孝文乃高顔題續之為耳崇其巾為屋合後施收貴賤皆服之蔡邕曰幘古者卑賤執事不冠者之所服元帝頟有壯髪不欲使人見始進幘服之】謂援曰卿遨遊二帝間今見卿使人大慙援頓首辭謝因曰當今之世非但君擇臣臣亦擇君矣臣與公孫述同縣少相善【少詩照翻】臣前至蜀述陛戟而後進臣【陛戟謂衛者持戟夾陛也】臣今遠來陛下何知非刺客姦人而簡易若是帝復笑曰卿非刺客顧說客耳【說去聲】援曰天下反覆盜名字者不可勝數【盜名字謂僭竊位號稱帝稱王也易以䜴翻復扶又翻說輸芮翻勝音升】今見陛下恢廓大度同符高祖乃知帝王自有眞也 太傅卓茂薨 十一月丙申上行幸宛【宛於元翻】岑彭攻秦豐三歲斬首九萬餘級豐餘兵裁千人食且盡十二月丙寅帝幸黎丘遣使招豐豐不肯降【降戶江翻下同】乃使朱祜等代岑彭圍黎丘使岑彭傅俊南擊田戎 公孫述聚兵數十萬人積粮漢中又造十層樓船多刻天下牧守印章【守式又翻】遣將軍李育程烏將數萬衆出屯陳倉就呂鮪將徇三輔馮異迎擊大破之育烏俱犇漢中 【考異曰公孫述傳使李育程烏與呂鮪徇三輔三年馮異擊鮪育於陳倉大敗之按本紀四年馮異與述將程馬戰陳倉破之馮異傳亦在今年蓋述傳誤以四年為三年焉作烏耳】異還擊破呂鮪營保降者甚衆是時隗囂遣兵佐異有功遣使上狀帝報以手書曰慕樂德義思相結納【樂音洛】昔文王三分猶服事殷【孔子曰三分天下有其二以服事殷周之德可謂至德也已矣】但駑馬鉛刀不可強扶【賢曰周禮校人掌六馬駑馬最下者也說文鉛青金也似錫而包青言駑馬鉛刀不可強扶而用也強其兩翻】數蒙伯樂一顧之價【戰國策蘇代謂淳于髠曰人有賣駿馬者比三旦立於市市人莫之知往見伯樂曰臣有駿馬欲賣之比三旦立市市人莫與言願子還而視之去而顧之臣請獻一朝之價伯樂如其言一旦而價十倍也數所角翻樂音洛】將軍南拒公孫之兵北御羌胡之亂【御讀曰禦】是以馮異西征得以數千百人躑躅三輔【賢曰躑躅猶踟蹰也毛晃曰躑躅跳也躑直炙翻躅直録翻】微將軍之助則咸陽已為他人禽矣如令子陽到漢中三輔願因將軍兵馬鼔旗相當儻肯如言即智士計功割地之秋也【賢曰秋一歲中功成之時故舉以為言】管仲曰生我者父母成我者鮑子【賢曰事見史記】自今以後手書相聞勿用傍人間構之言【間古莧翻下同】其後公孫述數遣將間出囂輒與馮異合勢共摧挫之述遣使以大司空扶安王印綬授囂【扶安謂相扶助而安也】囂斬其使出兵擊之以故蜀兵不復北出【復扶又翻】 泰山豪傑多與張步連兵吳漢薦強弩大將軍陳俊為泰山太守擊破步兵遂定泰山【郡國志泰山郡在雒陽東千四百里】<br />
<br />
  五年春正月癸巳車駕還宫 帝使來歙持節送馬援歸隴右 【考異曰袁紀曰援與拒蜀侯國遊先俱奉使遊先至長安為仇家所殺其弟為囂雲旗將軍來歙恐其怨恨與援俱還長安按囂使被殺者周遊也不在此時】隗囂與援共卧起問以東方事曰前到朝廷上引見數十【東觀記曰凡十四見】每接燕語自夕至旦才明勇畧非人敵也且開心見誠無所隱伏闊達多大節畧與高帝同經學博覽政事文辨前世無比囂曰卿謂何如高帝援曰不如也高帝無可無不可【賢曰此論語孔子自言己之所行也】今上好吏事【好呼到翻】動如節度又不喜飲酒【喜許記翻】囂意不懌曰如卿言反復勝邪【復扶又翻】 二月丙午大赦 蘇茂將五校兵救周建於垂惠【校戶敎翻】馬武為茂建所敗犇過王霸營大呼求救【敗補邁翻呼火故翻】霸曰賊兵盛出必兩敗努力而已乃閉營堅壁軍吏皆爭之霸曰茂兵精鋭其衆又多吾吏士心恐而捕虜與吾相恃【馬武為捕虜將軍】兩軍不一此敗道也今閉營固守示不相援賊必乘勝輕進捕虜無救其戰自倍【人各致死則一人倍二人之力】如此茂衆疲勞吾承其敝乃可克也茂建果悉出攻武【悉兵而出攻也】合戰良久霸軍中壯士數十人斷髪請戰【斷丁管翻】霸乃開營後出精騎襲其背茂建前後受敵驚亂敗走霸武各歸營茂建復聚兵挑戰【復扶又翻挑徒了翻下同】霸堅卧不出方饗士作倡樂【倡音昌】茂雨射營中【射矢如雨也射而亦翻】中霸前酒樽【中竹仲翻】霸安坐不動軍吏皆曰茂前日已破今易擊也【易以豉翻】霸曰不然蘇茂客兵遠來粮食不足故數挑戰以徼一時之勝【數所角翻徼堅堯翻又一遙翻】今閉營休士所謂不戰而屈人兵者也【孫子曰百戰百勝非善之善者也不戰而屈人兵善之善者也霸蓋引其言】茂建旣不得戰乃引還營其夜周建兄子誦反閉城拒之建於道死茂犇下邳【地理志下邳縣屬東海郡】與董憲合劉紆犇佼彊 乙丑上行幸魏郡 彭寵妻數為惡夢【數所角翻】又多見怪變卜筮望氣者皆言兵當從中起寵以子后蘭卿質漢歸不信之【子后蘭卿歸見上卷二年質音致】使將兵居外無親於中寵齋在便室【賢曰便坐之室非正室也】蒼頭子密等三人【賢曰秦呼民為黔首謂奴為蒼頭者以别于良人也】因寵卧寐共縛著牀【著直畧翻】告外吏云大王齋禁皆使吏休偽稱寵命收縛奴婢各置一處又以寵命呼其妻妻入驚曰奴反奴乃捽其頭擊其頰【捽昨沒翻】寵急呼曰趣為諸將軍辦裝【趣讀曰促賢曰呼奴為將軍者欲其赦已也呼火故翻為于偽翻】於是兩奴將妻入取寶物留一奴守寵寵謂守奴曰若小兒吾素所愛也今為子密所廹刼耳解我縛當以女珠妻汝【妻七細翻】家中財物皆以與若【若亦汝也】小奴意欲解之視戶外見子密聽其語遂不敢解於是收金玉衣物至寵所裝之被馬六匹【被皮義翻加馬以鞍勒曰被馬】使妻縫兩縑囊昏夜後解寵手令作記告城門將軍云今遣子密等至子后蘭卿所勿稽留之書成斬寵及妻頭置囊中便持記馳出城因以詣闕明旦閤門不開官屬踰牆而入見寵尸驚怖【怖普布翻】其尚書韓立等共立寵子午為王國師韓利斬午首詣祭遵降【國師以寵所署置也蓋遵王莽之制】夷其宗族帝封子密為不義侯<br />
<br />
  權德輿議曰伯通之叛命【伯通彭寵字也】子密之戕君同歸于亂罪不相蔽宜各致於法昭示王度【王度猶言王法也】反乃爵于五等又以不義為名且舉以不義莫可侯也此而可侯漢爵為不足勸矣春秋書齊豹盜三叛人名之義【衛司宼齊豹以私怨殺衛侯之兄孟縶春秋書之曰盜三叛人名謂襄二十一年邾庶其以漆閭丘來奔昭五年莒牟夷以牟婁及防兹來奔哀十四年小邾射以句繹來奔句音鉤】無乃異於是乎<br />
<br />
  帝以扶風郭伋為漁陽太守【郡國志漁陽郡在雒陽東北二千里】伋承離亂之後養民訓兵開示威信盜賊銷散匈奴遠迹在職五年戶口增倍 帝使光禄大夫樊宏持節迎耿况于上谷【郡國志上谷郡在雒陽東北二千里】曰邊郡寒苦不足久居况至京師賜甲第奉朝請封牟平侯【地理志牟平縣屬東萊郡唐宋屬登州宋白曰牟平縣以在牟山之陽其地平坦故曰牟平漢牟平故城在今黄縣東百三十里朝直遥翻請音才性翻又如字】吳漢率耿弇王常擊富平獲索賊于平原【郡國志平原郡在雒陽北一千三百里】大破之追討餘黨至勃海【郡國志勃海郡在雒陽北一千六百里】降者四萬餘人上因詔弇進討張步 平敵將軍龐萌為人遜順【前作平狄將軍】帝信愛之常稱曰可以託六尺之孤寄百里之命者【論語孔子之言呂與叔曰託六尺之孤謂輔幼主寄百里之命謂為諸侯】龐萌是也使與蓋延共擊董憲時詔書獨下延而不及萌【下遐稼翻】萌以為延譛已自疑遂反襲延軍破之 【考異曰東觀記漢書皆云萌攻延延與戰破之詔書勞延曰龐萌一夜反畔相去不遠營壁不堅殆令人齒欲相擊而將軍有不可動之節吾甚美之延傳言幸而得免與彼不同今從延傳】與董憲連和自號東平王屯桃鄉之北【東平國任城縣有桃鄉賢曰故城在今兖州龔丘縣西北】帝聞之大怒自將討萌與諸將書曰吾嘗以龐萌為社稷之臣將軍得無笑其言乎老賊當族其各厲兵馬會睢陽【睢陽梁國都郡國志在雒陽東南八百五十里】龐萌攻破彭城將殺楚郡太守孫萌【郡國志楚郡在雒陽東千二百二十里考異曰袁紀作楚相孫萌今從范書】郡吏劉平伏太守身上號泣請代其死身被七創【號戶刀翻下同被皮義翻創初良翻】龐萌義而捨之太守已絶復蘇【孔頴達曰更息曰蘇言氣絶而更息也】渇求飲平傾創血以飲之 岑彭攻拔夷陵田戎亡入蜀盡獲其妻子士衆數萬人公孫述以戎為翼江王岑彭謀伐蜀以夾川穀少【夾川猶言夾江也江大川也】水險難漕留威虜將軍馮駿軍江州都尉田鴻軍夷陵領軍李玄軍夷道【地理志夷道縣屬南郡】自引兵還屯津鄉【郡國志南郡江陵縣有津郷賢曰所謂江津也】當荆州要會喻告諸蠻夷降者奏封其君長 夏四月旱蝗 隗囂問於班彪曰往者周亡戰國並爭數世然後定意者從横之事將復起于今乎【從子容翻復扶又翻】將承運迭興在于一人也彪曰周之廢興與漢殊異昔周爵五等諸侯從政【師古曰言諸侯之國各自為政】本根旣微枝葉彊大【本根謂王室枝葉謂諸侯】故其末流有從横之事埶數然也漢承秦制改立郡縣主有專已之威臣無百年之柄至于成帝假借外家【師古曰假音工暇翻又工雅翻】哀平短祚國嗣三絶故王氏擅朝能竊號位危自上起傷不及下【賢口成帝威權借於外家是危自上起也漢不得罪于百姓是傷不及下也朝直遥翻】是以即真之後天下莫不引領而歎十餘年間中外騷擾遠近俱發假號雲合咸稱劉氏不謀同辭方今雄桀帶州域者皆無六國世業之資而百姓謳吟思仰漢必復興已可知矣囂曰生言周漢之埶可也至于但見愚人習識劉氏姓號之故而謂漢復興疎矣昔秦失其鹿劉季逐而掎之【師古曰掎偏持其足也居蟻翻】時民復知漢乎彪乃為之著王命論以風切之【為于偽翻風讀曰諷】曰昔堯之禪舜曰天之歷數在爾躬舜亦以命禹【論語所載】洎于稷契咸佐唐虞至湯武而有天下【洎其冀翻契息列翻】劉氏承堯之祚堯據火德而漢紹之有赤帝子之符【事見七卷秦二世元年】故為鬼神所福饗天下所歸往由是言之未見運世無本功德不紀【師古曰不紀言不為人所記】而得屈起在此位者也【屈起特起也屈求勿翻】俗見高祖興于布衣不達其故至比天下於逐鹿幸捷而得之不知神器有命不可以智力求也【劉德曰神器璽也李奇曰帝王賞罰之柄也師古曰李說是也仲馮曰神器聖人之大寶曰位是也】悲夫此世所以多亂臣賊子者也夫餓饉流隸【師古曰隷賤隸】飢寒道路所願不過一金然終轉死溝壑何則貧窮亦有命也况乎天子之貴四海之富神明之祚可得而妄處哉【處昌呂翻】故雖遭罹阨會竊其權柄勇如信布彊如梁籍【謂項梁項籍也】成如王莽然卒潤鑊伏質【師古曰質鍖也伏於鍖上而斬之也卒子恤翻】亨醢分裂【亨與烹同】又况么麽尚不及數子【師古曰么麽皆微小之稱也么音一堯翻麽音莫可翻】而欲闇奸天位者乎【奸音干】昔陳嬰之母以嬰家世貧賤卒富貴不祥止嬰勿王【卒讀曰猝事見八卷秦二世二年】王陵之母知漢王必得天下伏劔而死以固勉陵【事見九卷高祖元年】夫以匹婦之明猶能推事理之致探禍福之機而全宗祀于無窮垂策書於春秋【師古曰凡言匹夫匹婦謂凡庶之人春秋史書記事之總稱】而况大丈夫之事虖是故窮達有命吉凶由人嬰母知廢陵母知興審此二者帝王之分決矣【分扶問翻下同】加之高祖寛明而仁恕知人善任使當食吐哺納子房之策拔足揮洗揖酈生之說舉韓信于行陳【洗息典翻行戶剛翻陳讀曰陣】收陳平於亡命【事並見高帝紀】英雄陳力羣策畢舉此高祖之大畧所以成帝業也若乃靈瑞符應其事甚衆故淮隂留侯謂之天授非人力也英雄誠知覺寤超然遠覽淵然深識收陵嬰之明分絶信布之覬覦【覬音冀覦音俞】距逐鹿之瞽說審神器之有授毋貪不可冀為二母之所笑則福祚流于子孫天禄其永終矣囂不聽彪遂避地河西竇融以為從事【漢制將軍府及司隸刺史郡守皆有從事】甚禮重之彪遂為融畫策【為于偽翻】使之專意事漢焉 初竇融等聞帝威德心欲東向以河西隔遠未能自通乃從隗囂受建武正朔囂皆假其將軍印綬囂外順人望内懷異心使辯士張玄說融等曰更始事已成尋復亡滅【說輸芮翻復扶又翻】此一姓不再興之效也今即有所主便相係屬一旦拘制自令失柄後有危敗雖悔無及方今豪桀競逐雌雄未決當各據土宇與隴蜀合從【從子容翻】高可為六國下不失尉佗【尉佗事見十二卷高帝十一年佗徒何翻】融等召豪桀議之其中識者皆曰今皇帝姓名見於圖書【見賢遍翻】自前世博物道術之士谷子雲夏賀良等皆言漢有再受命之符【谷永書見三十一卷成帝永始二年夏賀良事見三十三卷哀帝建元二年】故劉子駿改易名字冀應其占【劉歆改名事見三十三卷成帝綏和二年歆字子駿意在改名之後】及莽末西門君惠謀立子駿事覺被殺出謂觀者曰䜟文不誤劉秀眞汝主也【事見三十九卷更始元年讖楚諧翻】此皆近事暴著【暴步木翻毛見曰顯示也又如字義同】衆所共見者也况今稱帝者數人而雒陽土地最廣甲兵最強號令最明觀符命而察人事它姓殆未能當也衆議或同或異融遂決策東向遣長史劉鈞等奉書詣雒陽【時衆推融為大將軍故置長史】先是帝亦發使遺融書以招之【先悉薦翻遺于季翻】遇鈞於道即與俱還帝見鈞歡甚禮饗畢乃遣令還賜融璽書曰今益州有公孫子陽天水有隗將軍方蜀漢相攻權在將軍舉足左右便有輕重【言左投則蜀重右投則漢重也】以此言之欲有厚豈有量哉欲遂立桓文輔微國當勉卒功業欲三分鼎足連衡合從亦宜以時定【開兩說以觀融去就量音良卒子恤翻衡讀曰横從子容翻】天下未并吾與爾絶域非相吞之國今之議者必有任囂教尉佗制七郡之計【事見十二卷高帝十一年賢曰七郡蒼梧鬰林合浦交趾九眞南海日南也余謂尉佗之時未置七郡光武據後來置郡言之】王者有分土無分民【分扶問翻】自適己事而已因授融為涼州牧璽書至河西河西皆驚以為天子明見萬里之外 朱祜急攻黎丘六月秦豐窮困出降轞車送雒陽吳漢劾祜廢詔命受豐降上誅豐不罪祜董憲與劉紆蘇茂佼彊去下邳還蘭陵使茂彊助龐萌圍桃城【桃城即桃郷之城也賢曰在今兖州任城縣北】帝時幸蒙聞之乃留輜重【重直用翻】自將輕兵晨夜馳赴至亢父【賢曰蒙縣名屬梁國故城在今宋州北地理志亢父縣屬東平國師古曰音抗甫】或言百官疲倦可且止宿上不聽復行十里宿任城【復扶又翻任音壬欲度亢父之險故進而宿任城】去桃城六十里旦日諸將請進龐萌等亦勒兵挑戰【挑徒了翻】帝令諸將不得出休士養鋭以挫其鋒時吳漢等在東郡【郡國志東郡去雒陽八百里】馳使召之萌等驚曰數百里晨夜行以為至當戰而堅坐任城致人城下眞不可往也乃悉兵攻桃城城中聞車駕至衆心益固萌等攻二十餘日衆疲困不能下吳漢王常蓋延王梁馬武王霸等皆至帝乃率衆軍進救桃城親自搏戰大破之龐萌蘇茂佼彊夜走從董憲秋七月丁丑帝幸沛進幸湖陵董憲與劉紆悉其兵數萬人屯昌慮【地理志昌慮縣屬東海郡宋白曰徐州滕縣漢蕃昌慮二縣地應劭註蕃縣即小邾國又有邾國濫城在今縣東南即漢之昌慮縣也師古曰慮音廬】憲招誘五校餘賊與之拒守建陽【賢曰建陽縣屬東海郡故城在今沂州丞縣北丞時證翻】帝至蕃【賢曰蕃音皮又音婆地理志蕃縣屬魯國應劭曰小邾國也師古曰白裒云陳蕃為魯相國人為諱改曰皮此說非也郡縣之名土俗各有别稱不必皆依本字杜佑通典蕃音反余謂皮字乃傳寫反字之誤當從通典反音孚袁翻】去憲所百餘里諸將請進帝不聽知五校乏食當退敕各堅壁以待其敝頃之五校果引去帝乃親臨四面攻憲三日大破之佼彊將其衆降【降戶江翻】蘇茂犇張步憲及龐萌走保郯八月己酉帝幸郯留吳漢攻之車駕轉徇彭城下邳吳漢拔郯董憲龐萌走保朐【賢曰朐縣屬東海郡今海州朐山縣西有故朐城朐音劬宋白曰朐故城在朐山縣西九十里】劉紆不知所歸其軍士高扈斬之以降吳漢進圍朐 冬十月帝幸魯【魯國本屬徐州帝改屬豫州】 張步聞耿弇將至使其大將軍費邑軍歷下【賢曰歷下城在今齊州歷城縣 考異曰袁紀作濟南王費邑今從耿弇傳】又令兵屯祝阿【地理志祝阿縣屬平原郡賢曰今齊州縣故城在今山荏縣東北天寶元年改祝阿為禹城以縣西有禹息故城也】别于泰山鍾城列營數十以待之弇渡河先擊祝阿自旦攻城日未中而拔之故開圍一角令其衆得奔歸鍾城鍾城人聞祝阿已潰大恐懼遂空壁亡去費邑分遣弟敢守巨里【郡國志濟南歷城有巨里聚賢曰一名巨合城在今齊州全節縣東南】弇進兵先脅巨里嚴令軍中趣修攻具【趣讀曰促】宣敕諸部後三日當悉力攻巨里城隂緩生口令得亡歸以弇期告邑邑至日果自將精兵三萬餘人來救之弇喜謂諸將曰吾所以修攻具者欲誘致之耳【誘音酉】野兵不擊何以城為即分三千人守巨里自引精兵上岡阪【爾雅曰山脊曰岡坡者曰坂上時掌翻】乘高合戰大破之臨陳斬邑【陳讀曰陣】旣而收首級以示城中城中兇懼【賢曰兇恐懼聲音呼勇翻】費敢悉衆亡歸張步弇復收其積聚【復扶又翻積子賜翻聚才喻翻】縱兵擊諸未下者平四十餘營遂定濟南【郡國志濟南郡在雒陽東千八百里濟子禮翻】時張步都劇使其弟藍將精兵二萬守西安【賢曰西安縣名屬齊郡故城在今青州臨菑縣西北】諸郡太守合萬餘人守臨菑【臨菑縣屬齊郡】相去四十里弇進軍畫中【賢曰畫中邑名也畫音胡麥翻故城在今西安城東南有澅水因名馬水經注澅水東去臨菑城十八里】居二城之間弇視西安城小而堅且藍兵又精臨菑名雖大而實易攻【易以䜴翻】乃敕諸校後五日會攻西安【校戶教翻】藍聞之晨夜警守至期夜半弇敕諸將皆蓐食【前書音義曰未起而牀蓐中食也】會明至臨菑城護軍荀梁等爭之以為攻臨菑西安必救之攻西安臨菑不能救不如攻西安弇曰不然西安聞吾欲攻之日夜為備方自憂何暇救人臨菑出不意而至必驚擾吾攻之一日必拔拔臨菑即西安孤與劇隔絶必復亡去【復扶又翻】所謂擊一而得二者也若先攻西安不能卒下【卒讀曰猝】頓兵堅城死傷必多縱能拔之藍引軍還奔臨菑并兵合埶觀人虛實吾深入敵地後無轉輸旬月之間不戰而困矣遂攻臨菑半日拔之入據其城張藍聞之懼遂將其衆亡歸劇弇乃令軍中無得虜掠須張步至乃取之以激怒步步聞大笑曰以尤來大彤十餘萬衆吾皆即其營而破之今大耿兵少於彼【即就也賢曰弇況之長子故呼為大耿少詩沼翻】又皆疲勞何足懼乎乃與三弟藍弘夀及故大彤渠帥重異等兵【賢曰重姓異名重直龍翻姓譜南正重之後】號二十萬至臨菑大城東將攻弇弇上書曰臣據臨菑深塹高壘張步從劇縣來攻疲勞飢渇欲進誘而攻之欲去隨而擊之臣依營而戰精鋭百倍以逸待勞以實擊虚旬日之間步首可獲於是弇先出菑水上【水經淄水出泰山萊蕪縣原山東北過臨菑縣東】與重異遇突騎欲縱弇恐挫其鋒令步不敢進故示弱以盛其氣乃引歸小城陳兵于内使都尉劉歆泰山太守陳俊分陳于城下【陳讀曰陣下同】步氣盛直攻弇營與劉歆等合戰弇升王宫壞臺望之【賢曰臨菑本齊國所都即齊王宫中有壞臺也】視歆等鋒交乃自引精兵以横突步陳于東城下大破之飛矢中弇股【中竹仲翻】以佩刀截之左右無知者至暮罷弇明旦復勒兵出【復扶又翻下同】是時帝在魯聞弇為步所攻自往救之未至陳俊謂弇曰劇虜兵盛可且閉營休士以須上來弇曰乘輿且到【乘繩證翻】臣子當擊牛釃酒以待百官【釃山宜翻陸德明曰以筐酒賢曰濾也】反欲以賊虜遺君父邪【遺于季翻】乃出兵大戰自旦及昏復大破之殺傷無數溝塹皆滿弇知步困將退豫置左右翼為伏以待之【兩旁伏兵如鳥之舒翼】人定時步果引去【昏後謂之人定時】伏兵起縱擊追至鉅昩水上【賢曰鉅昩水名一名巨洋水在今青州壽光縣西水經註巨洋水出朱虚縣東泰山袁宏謂之鉅昧王韶之以為巨蔑北過臨朐縣東又北過臨朐縣西又東北過夀光縣西昩音莫葛翻】八九十里僵尸相屬【屬之欲翻】收得輜重二千餘兩【重直用翻兩音亮下同風俗通車一乘為一兩箱轅及輪兩兩而偶故稱兩】步還劇兄弟各分兵散去後數日車駕至臨菑自勞軍【勞力到翻】羣臣大會帝謂弇曰昔韓信破歷下以開基【事見十卷高祖四年】今將軍攻祝阿以發迹此皆齊之西界功足相方而韓信襲擊已降將軍獨拔勍敵其功又難於信也【勍渠京翻】又田横亨酈生及田横降高祖詔衛尉不聽為仇【事見十一卷高帝五年亨與烹同】張步前亦殺伏隆【事見上三年】若步來歸命吾當詔大司徒釋其怨又事尤相類也將軍前在南陽建此大策【謂三年冬弇從帝幸舂陵自請平齊也】常以為落落難合【賢曰落落猶疎闊也】有志者事竟成也帝進幸劇耿弇復追張步步犇平夀【賢曰平壽縣名屬北海郡故城在今青州北海縣復扶又翻】蘇茂將萬餘人來救之茂讓步曰以南陽兵精延岑善戰而耿弇走之【事見丄三年】大王奈何就攻其營旣呼茂不能待邪步曰負負無可言者【賢曰負愧也再言負者愧之甚也】帝遣使告步茂能相斬降者封為列侯【降戶江翻】步遂斬茂詣耿弇軍門肉袒降弇傳詣行在所【傳直戀翻】而勒兵入據其城【平夀城也】樹十二郡旗鼔令步兵各以郡人詣旗下衆尚十餘萬輜重七千餘兩皆罷遣歸鄉里張步三弟各自繫所在獄詔皆赦之封步為安丘侯【地理志安丘侯國屬琅邪郡又北海郡有安丘縣宋白曰密州有安丘縣古根牟國城漢為安丘縣有渠丘亭故莒渠丘公所居也】與妻子居雒陽於是琅邪未平【郡國志琅邪郡在雒陽東千五百里】上徙陳俊為琅邪太守始入境盜賊皆散耿弇復引兵至城陽【地理志城陽國都莒賢曰城陽故城在今沂州臨沂縣南】降五校餘黨【校戶教翻】齊地悉平振旅還京師弇為將凡所平郡四十六屠城三百未嘗挫折焉 初起太學車駕還宫幸太學【陸機洛陽記曰太學在洛陽城故開陽門外去宫八里講堂長十丈廣三丈】稽式古典修明禮樂煥然文物可觀矣 十一月大司徒伏湛免以侯霸為大司徒霸聞太原閔仲叔之名而辟之旣至霸不及政事徒勞苦而已【賢曰勞其勤苦也勞音力到翻】仲叔恨曰始蒙嘉命且喜且懼今見明公喜懼皆去以仲叔為不足問邪不當辟也辟而不問是失人也遂辭出投劾而去【賢曰案罪曰劾自投其劾狀而去也投猶下也今冇投辭投牒之言也劾戶槩翻】 初五原人李興隨昱【姓譜隨侯之後又杜伯之玄孫為晉大夫食采于隨曰隨會】朔方人田颯【颯音立】代郡人石鮪閔堪各起兵自稱將軍匈奴單于遣使與興等和親欲令盧芳還漢地為帝興等引兵至單于庭迎芳十二月與俱入塞都九原縣【賢曰九原縣名屬五原郡故城在今勝州銀城縣】掠有五原朔方雲中定襄鴈門五郡並置守令與胡兵侵苦北邊 馮異治關中出入三歲上林成都【賢曰成都言歸附之多也史記曰一年成邑二年成都治直之翻】人有上章言異威權至重百姓歸心號為咸陽王帝以章示異異惶懼上書陳謝詔報曰將軍之于國家義為君臣恩猶父子何嫌何疑而有懼意 隗囂矜已飾智每自比西伯【西伯文王也】與諸將議欲稱王鄭興曰昔文王三分天下有其二尚服事殷【論語載孔子之言】武王八百諸侯不謀同會猶還兵待時【武王觀兵孟津諸侯不期而會者八百皆曰紂可伐矣王曰汝未知天命乃還師】高帝征伐累年猶以沛公行師今令德雖明世無宗周之祚威畧雖振未有高祖之功而欲舉未可之事昭速【速不速之速明召也】禍患【昭明也】無乃不可乎囂乃止後又廣置職位以自尊高鄭興曰夫中郎將太中大夫使持節官皆王者之器非人臣所當制也無益於實有損于名非尊上之意也囂病之而止【賢曰病猶難也】時關中將帥數上書言蜀可擊之狀【數所角翻下同】帝以書示囂因使擊蜀以效其信【效驗也】囂上書盛言三輔單弱劉文伯在邊【盧芳自稱劉文伯】未宜謀蜀帝知囂欲持兩端不願天下統一于是稍黜其禮正君臣之儀【帝與囂書初用敵國禮今黜其禮】帝以囂與馬援來歙相善數使歙援奉使往來勸令入朝【朝直遥翻】許以重爵囂連遣使深持謙辭言無功德須四方平定退伏閭里帝復遣來歙說囂遣子入侍【復扶又翻說輸芮翻下同】囂聞劉永彭寵皆已破滅乃遣長子恂隨歙詣闕帝以為胡騎校尉封鐫羌侯【胡騎校尉武帝置秩二千石賢曰鐫謂鐫鑿也鐫子全翻】鄭興因恂求歸葬父母囂不聽而徙興舍益其秩禮興入見曰今為父母未葬乞骸骨【為于偽翻】若以增秩徙舍中更停留是以親為餌也【賢曰猶釣餌也】無禮甚矣將軍焉用之【焉於䖍翻】願留妻子獨歸葬將軍又何猜焉囂乃令與妻子俱東馬援亦將家屬隨恂歸雒陽以所將賓客猥多求屯田上林苑中帝許之【將如字】囂將王元以為天下成敗未可知不願專心内事說囂曰昔更始西都四方響應天下喁喁【賢曰喁喁魚口向上也音魚客翻】謂之太平一旦壞敗將軍幾無所厝【事見上卷元年】今南有子陽北有文伯江湖海岱王公十數而欲牽儒生之說【賢曰儒生謂馬援說囂歸光武余謂儒生指鄭興班彪等】棄千乘之基【列國之賦兵車千乘乘䋲證翻】羈旅危國以求萬全此循覆車之軌者也今天水完富士馬㝡彊元請以一丸泥為大王東封函谷關【為于偽翻】此萬世一時也若計不及此且畜養士馬【畜許六翻】據隘自守曠日持久以待四方之變圖王不成其敝猶足以霸【前書徐樂之言】要之魚不可脱于淵神龍失埶與蚯蚓同【老子曰魚不可脱于泉脱失也失泉則涸矣慎子曰騰蛇游霧飛龍乘雲雲罷霧除與蚯蚓同失其所乘故也】囂心然元計雖遣子入質【質音致】猶負其險阨欲專制方面申屠剛諫曰愚聞人所歸者天所與人所畔者天所去也本朝誠天之所福非人力也【賢曰本朝謂光武也】今璽書數到【數所角翻】委國歸信欲與將軍共同吉凶布衣相與尚有沒身不負然諾之信況于萬乘者哉今何畏何利而久疑若是【賢曰言從漢何畏附蜀何利而久疑不決】卒有非常之變【卒讀曰猝】上負忠孝下愧當世夫未至豫言固常為虚及其已至又無所及是以忠言至諫希得為用誠願反覆愚老之言囂不納于是游士長者稍稍去之 王莽末交趾諸郡閉境自守【賢曰交趾郡今交州縣也南濱大海輿地志云其夷足大指開析兩足並立則相交應劭曰始開北方遂交於南方為子孫基阯地余按武帝元鼎六年置交趾州治廣信時已開朔方遂交於南方為子孫基趾也七郡謂南海蒼梧鬱林合浦交趾九真日南並屬交州余謂唐之交州峯州皆漢交趾郡之地固不可指唐交趾一縣而言也】岑彭素與交趾牧鄧讓厚善與讓書陳國家威德又遣偏將軍屈充移檄江南班行詔命【屈其勿翻】於是讓與江夏太守侯登武陵太守王堂長沙相韓福桂陽太守張隆零陵太守田翕蒼梧太守杜穆交趾太守錫光等相率遣使貢獻【郡國志江夏郡在雒陽南千五百里武陵郡在雒陽南二千一百里長沙郡在雒陽南二千八百里桂陽郡在雒陽南三千九百里零陵郡在雒陽南三千三百里蒼梧郡在雒陽南六千四百一十里交趾郡在雒陽南一萬一千里夏戶雅翻守式又翻下同錫姓光名】悉封為列侯錫光者漢中人在交趾教民夷以禮義帝復以宛人任延為九眞太守【郡國志九真郡在雒陽南萬一千五百八十里復扶又翻任音壬】延教民耕種嫁娶故嶺南華風始于二守焉 是歲詔徵處士太原周黨會稽嚴光等至京師【處昌呂翻會古外翻】黨入見伏而不謁【凡朝謁者必拜稽首以姓名自言見賢遍翻】自陳願守所志博士范升奏曰伏見太原周黨東海王良山陽王成等蒙受厚恩使者三聘乃肯就車及陛見帝廷【見賢遍翻】黨不以禮屈伏而不謁偃蹇驕悍同時俱逝黨等文不能演義武不能死君釣采華名庶幾三公之位【幾居希翻】臣願與坐雲臺之下【續漢志曰雲臺周家之所造圖書術籍珍玩寶怪皆藏焉】考試圖國之道不如臣言伏虚妄之罪而敢私竊虚名誇上求高皆大不敬書奏詔曰自古明王聖主必有不賓之士伯夷叔齊不食周粟太原周黨不受朕禄亦各有志焉其賜帛四十匹罷之帝少與嚴光同遊學【少詩照翻】及即位以物色訪之【賢曰以其形貌求之】得于齊國累徵乃至拜諫議大夫不肯受去耕釣於富春山中【地理志富春縣屬會稽郡賢曰今杭州富陽縣本漢富春縣避晉簡文帝鄭太后諱改曰富陽】以夀終於家王良後歷沛郡太守大司徒司直在位恭儉布被瓦器妻子不入官舍後以病歸一歲復徵【復扶又翻】至滎陽疾篤不任進道【言以疾篤稽留道上不進于行也任音壬】過其友人友人不肯見曰不有忠言奇謀而取大位何其往來屑屑不憚煩也【揚雄方言曰屑屑不安也秦晉曰屑屑郭景純曰往來貌】遂拒之良慙自後連徵不應卒于家【卒子恤翻】 元帝之世莎車王延嘗為侍子京師慕樂中國【樂音洛】及王莽之亂匈奴畧有西域唯延不肯附屬常敕諸子當世奉漢家不可負也延卒子康立康率旁國拒匈奴【旁國猶鄰國也】擁衛故都護吏士妻子千餘口【王莽之亂西域攻沒都護其吏士妻子皆不得還】檄書河西問中國動静竇融乃承制立康為漢莎車建功懷德王西域大都尉五十五國皆屬焉<br />
<br />
  資治通鑑卷四十一<br />
<br />
<史部,編年類,資治通鑑>  <br>
   </div> 

<script src="/search/ajaxskft.js"> </script>
 <div class="clear"></div>
<br>
<br>
 <!-- a.d-->

 <!--
<div class="info_share">
</div> 
-->
 <!--info_share--></div>   <!-- end info_content-->
  </div> <!-- end l-->

<div class="r">   <!--r-->



<div class="sidebar"  style="margin-bottom:2px;">

 
<div class="sidebar_title">工具类大全</div>
<div class="sidebar_info">
<strong><a href="http://www.guoxuedashi.com/lsditu/" target="_blank">历史地图</a></strong>  
<a href="http://www.880114.com/" target="_blank">英语宝典</a>  
<a href="http://www.guoxuedashi.com/13jing/" target="_blank">十三经检索</a> 
<br><strong><a href="http://www.guoxuedashi.com/gjtsjc/" target="_blank">古今图书集成</a></strong> 
<a href="http://www.guoxuedashi.com/duilian/" target="_blank">对联大全</a> <strong><a href="http://www.guoxuedashi.com/xiangxingzi/" target="_blank">象形文字典</a></strong> 

<br><a href="http://www.guoxuedashi.com/zixing/yanbian/">字形演变</a>  <strong><a href="http://www.guoxuemi.com/hafo/" target="_blank">哈佛燕京中文善本特藏</a></strong>
<br><strong><a href="http://www.guoxuedashi.com/csfz/" target="_blank">丛书&方志检索器</a></strong> <a href="http://www.guoxuedashi.com/yqjyy/" target="_blank">一切经音义</a>  

<br><strong><a href="http://www.guoxuedashi.com/jiapu/" target="_blank">家谱族谱查询</a></strong>  <strong><a href="http://shufa.guoxuedashi.com/sfzitie/" target="_blank">书法字帖欣赏</a></strong> 
<br>

</div>
</div>


<div class="sidebar" style="margin-bottom:0px;">

<font style="font-size:22px;line-height:32px">QQ交流群9:489193090</font>


<div class="sidebar_title">手机APP 扫描或点击</div>
<div class="sidebar_info">
<table>
<tr>
	<td width=160><a href="http://m.guoxuedashi.com/app/" target="_blank"><img src="/img/gxds-sj.png" width="140"  border="0" alt="国学大师手机版"></a></td>
	<td>
<a href="http://www.guoxuedashi.com/download/" target="_blank">app软件下载专区</a><br>
<a href="http://www.guoxuedashi.com/download/gxds.php" target="_blank">《国学大师》下载</a><br>
<a href="http://www.guoxuedashi.com/download/kxzd.php" target="_blank">《汉字宝典》下载</a><br>
<a href="http://www.guoxuedashi.com/download/scqbd.php" target="_blank">《诗词曲宝典》下载</a><br>
<a href="http://www.guoxuedashi.com/SiKuQuanShu/skqs.php" target="_blank">《四库全书》下载</a><br>
</td>
</tr>
</table>

</div>
</div>


<div class="sidebar2">
<center>


</center>
</div>

<div class="sidebar"  style="margin-bottom:2px;">
<div class="sidebar_title">网站使用教程</div>
<div class="sidebar_info">
<a href="http://www.guoxuedashi.com/help/gjsearch.php" target="_blank">如何在国学大师网下载古籍?</a><br>
<a href="http://www.guoxuedashi.com/zidian/bujian/bjjc.php" target="_blank">如何使用部件查字法快速查字?</a><br>
<a href="http://www.guoxuedashi.com/search/sjc.php" target="_blank">如何在指定的书籍中全文检索?</a><br>
<a href="http://www.guoxuedashi.com/search/skjc.php" target="_blank">如何找到一句话在《四库全书》哪一页?</a><br>
</div>
</div>


<div class="sidebar">
<div class="sidebar_title">热门书籍</div>
<div class="sidebar_info">
<a href="/so.php?sokey=%E8%B5%84%E6%B2%BB%E9%80%9A%E9%89%B4&kt=1">资治通鉴</a> <a href="/24shi/"><strong>二十四史</strong></a>&nbsp; <a href="/a2694/">野史</a>&nbsp; <a href="/SiKuQuanShu/"><strong>四库全书</strong></a>&nbsp;<a href="http://www.guoxuedashi.com/SiKuQuanShu/fanti/">繁体</a>
<br><a href="/so.php?sokey=%E7%BA%A2%E6%A5%BC%E6%A2%A6&kt=1">红楼梦</a> <a href="/a/1858x/">三国演义</a> <a href="/a/1038k/">水浒传</a> <a href="/a/1046t/">西游记</a> <a href="/a/1914o/">封神演义</a>
<br>
<a href="http://www.guoxuedashi.com/so.php?sokeygx=%E4%B8%87%E6%9C%89%E6%96%87%E5%BA%93&submit=&kt=1">万有文库</a> <a href="/a/780t/">古文观止</a> <a href="/a/1024l/">文心雕龙</a> <a href="/a/1704n/">全唐诗</a> <a href="/a/1705h/">全宋词</a>
<br><a href="http://www.guoxuedashi.com/so.php?sokeygx=%E7%99%BE%E8%A1%B2%E6%9C%AC%E4%BA%8C%E5%8D%81%E5%9B%9B%E5%8F%B2&submit=&kt=1"><strong>百衲本二十四史</strong></a>  <a href="http://www.guoxuedashi.com/so.php?sokeygx=%E5%8F%A4%E4%BB%8A%E5%9B%BE%E4%B9%A6%E9%9B%86%E6%88%90&submit=&kt=1"><strong>古今图书集成</strong></a>
<br>

<a href="http://www.guoxuedashi.com/so.php?sokeygx=%E4%B8%9B%E4%B9%A6%E9%9B%86%E6%88%90&submit=&kt=1">丛书集成</a> 
<a href="http://www.guoxuedashi.com/so.php?sokeygx=%E5%9B%9B%E9%83%A8%E4%B8%9B%E5%88%8A&submit=&kt=1"><strong>四部丛刊</strong></a>  
<a href="http://www.guoxuedashi.com/so.php?sokeygx=%E8%AF%B4%E6%96%87%E8%A7%A3%E5%AD%97&submit=&kt=1">說文解字</a> <a href="http://www.guoxuedashi.com/so.php?sokeygx=%E5%85%A8%E4%B8%8A%E5%8F%A4&submit=&kt=1">三国六朝文</a>
<br><a href="http://www.guoxuedashi.com/so.php?sokeytm=%E6%97%A5%E6%9C%AC%E5%86%85%E9%98%81%E6%96%87%E5%BA%93&submit=&kt=1"><strong>日本内阁文库</strong></a> <a href="http://www.guoxuedashi.com/so.php?sokeytm=%E5%9B%BD%E5%9B%BE%E6%96%B9%E5%BF%97%E5%90%88%E9%9B%86&ka=100&submit=">国图方志合集</a> <a href="http://www.guoxuedashi.com/so.php?sokeytm=%E5%90%84%E5%9C%B0%E6%96%B9%E5%BF%97&submit=&kt=1"><strong>各地方志</strong></a>

</div>
</div>


<div class="sidebar2">
<center>

</center>
</div>
<div class="sidebar greenbar">
<div class="sidebar_title green">四库全书</div>
<div class="sidebar_info">

《四库全书》是中国古代最大的丛书,编撰于乾隆年间,由纪昀等360多位高官、学者编撰,3800多人抄写,费时十三年编成。丛书分经、史、子、集四部,故名四库。共有3500多种书,7.9万卷,3.6万册,约8亿字,基本上囊括了古代所有图书,故称“全书”。<a href="http://www.guoxuedashi.com/SiKuQuanShu/">详细>>
</a>

</div> 
</div>

</div>  <!--end r-->

</div>
<!-- 内容区END --> 

<!-- 页脚开始 -->
<div class="shh">

</div>

<div class="w1180" style="margin-top:8px;">
<center><script src="http://www.guoxuedashi.com/img/plus.php?id=3"></script></center>
</div>
<div class="w1180 foot">
<a href="/b/thanks.php">特别致谢</a> | <a href="javascript:window.external.AddFavorite(document.location.href,document.title);">收藏本站</a> | <a href="#">欢迎投稿</a> | <a href="http://www.guoxuedashi.com/forum/">意见建议</a> | <a href="http://www.guoxuemi.com/">国学迷</a> | <a href="http://www.shuowen.net/">说文网</a><script language="javascript" type="text/javascript" src="https://js.users.51.la/17753172.js"></script><br />
  Copyright &copy; 国学大师 古典图书集成 All Rights Reserved.<br>
  
  <span style="font-size:14px">免责声明:本站非营利性站点,以方便网友为主,仅供学习研究。<br>内容由热心网友提供和网上收集,不保留版权。若侵犯了您的权益,来信即刪。scp168@qq.com</span>
  <br />
ICP证:<a href="http://www.beian.miit.gov.cn/" target="_blank">鲁ICP备19060063号</a></div>
<!-- 页脚END --> 
<script src="http://www.guoxuedashi.com/img/plus.php?id=22"></script>
<script src="http://www.guoxuedashi.com/img/tongji.js"></script>

</body>
</html>
