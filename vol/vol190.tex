<!DOCTYPE html PUBLIC "-//W3C//DTD XHTML 1.0 Transitional//EN" "http://www.w3.org/TR/xhtml1/DTD/xhtml1-transitional.dtd">
<html xmlns="http://www.w3.org/1999/xhtml">
<head>
<meta http-equiv="Content-Type" content="text/html; charset=utf-8" />
<meta http-equiv="X-UA-Compatible" content="IE=Edge,chrome=1">
<title>資治通鑒_191-資治通鑑卷一百九十_191-資治通鑑卷一百九十</title>
<meta name="Keywords" content="資治通鑒_191-資治通鑑卷一百九十_191-資治通鑑卷一百九十">
<meta name="Description" content="資治通鑒_191-資治通鑑卷一百九十_191-資治通鑑卷一百九十">
<meta http-equiv="Cache-Control" content="no-transform" />
<meta http-equiv="Cache-Control" content="no-siteapp" />
<link href="/img/style.css" rel="stylesheet" type="text/css" />
<script src="/img/m.js?2020"></script> 
</head>
<body>
 <div class="ClassNavi">
<a  href="/24shi/">二十四史</a> | <a href="/SiKuQuanShu/">四库全书</a> | <a href="http://www.guoxuedashi.com/gjtsjc/"><font  color="#FF0000">古今图书集成</font></a> | <a href="/renwu/">历史人物</a> | <a href="/ShuoWenJieZi/"><font  color="#FF0000">说文解字</a></font> | <a href="/chengyu/">成语词典</a> | <a  target="_blank"  href="http://www.guoxuedashi.com/jgwhj/"><font  color="#FF0000">甲骨文合集</font></a> | <a href="/yzjwjc/"><font  color="#FF0000">殷周金文集成</font></a> | <a href="/xiangxingzi/"><font color="#0000FF">象形字典</font></a> | <a href="/13jing/"><font  color="#FF0000">十三经索引</font></a> | <a href="/zixing/"><font  color="#FF0000">字体转换器</font></a> | <a href="/zidian/xz/"><font color="#0000FF">篆书识别</font></a> | <a href="/jinfanyi/">近义反义词</a> | <a href="/duilian/">对联大全</a> | <a href="/jiapu/"><font  color="#0000FF">家谱族谱查询</font></a> | <a href="http://www.guoxuemi.com/hafo/" target="_blank" ><font color="#FF0000">哈佛古籍</font></a> 
</div>

 <!-- 头部导航开始 -->
<div class="w1180 head clearfix">
  <div class="head_logo l"><a title="国学大师官网" href="http://www.guoxuedashi.com" target="_blank"></a></div>
  <div class="head_sr l">
  <div id="head1">
  
  <a href="http://www.guoxuedashi.com/zidian/bujian/" target="_blank" ><img src="http://www.guoxuedashi.com/img/top1.gif" width="88" height="60" border="0" title="部件查字,支持20万汉字"></a>


<a href="http://www.guoxuedashi.com/help/yingpan.php" target="_blank"><img src="http://www.guoxuedashi.com/img/top230.gif" width="600" height="62" border="0" ></a>


  </div>
  <div id="head3"><a href="javascript:" onClick="javascript:window.external.AddFavorite(window.location.href,document.title);">添加收藏</a>
  <br><a href="/help/setie.php">搜索引擎</a>
  <br><a href="/help/zanzhu.php">赞助本站</a></div>
  <div id="head2">
 <a href="http://www.guoxuemi.com/" target="_blank"><img src="http://www.guoxuedashi.com/img/guoxuemi.gif" width="95" height="62" border="0" style="margin-left:2px;" title="国学迷"></a>
  

  </div>
</div>
  <div class="clear"></div>
  <div class="head_nav">
  <p><a href="/">首页</a> | <a href="/ShuKu/">国学书库</a> | <a href="/guji/">影印古籍</a> | <a href="/shici/">诗词宝典</a> | <a   href="/SiKuQuanShu/gxjx.php">精选</a> <b>|</b> <a href="/zidian/">汉语字典</a> | <a href="/hydcd/">汉语词典</a> | <a href="http://www.guoxuedashi.com/zidian/bujian/"><font  color="#CC0066">部件查字</font></a> | <a href="http://www.sfds.cn/"><font  color="#CC0066">书法大师</font></a> | <a href="/jgwhj/">甲骨文</a> <b>|</b> <a href="/b/4/"><font  color="#CC0066">解密</font></a> | <a href="/renwu/">历史人物</a> | <a href="/diangu/">历史典故</a> | <a href="/xingshi/">姓氏</a> | <a href="/minzu/">民族</a> <b>|</b> <a href="/mz/"><font  color="#CC0066">世界名著</font></a> | <a href="/download/">软件下载</a>
</p>
<p><a href="/b/"><font  color="#CC0066">历史</font></a> | <a href="http://skqs.guoxuedashi.com/" target="_blank">四库全书</a> |  <a href="http://www.guoxuedashi.com/search/" target="_blank"><font  color="#CC0066">全文检索</font></a> | <a href="http://www.guoxuedashi.com/shumu/">古籍书目</a> | <a   href="/24shi/">正史</a> <b>|</b> <a href="/chengyu/">成语词典</a> | <a href="/kangxi/" title="康熙字典">康熙字典</a> | <a href="/ShuoWenJieZi/">说文解字</a> | <a href="/zixing/yanbian/">字形演变</a> | <a href="/yzjwjc/">金 文</a> <b>|</b>  <a href="/shijian/nian-hao/">年号</a> | <a href="/diming/">历史地名</a> | <a href="/shijian/">历史事件</a> | <a href="/guanzhi/">官职</a> | <a href="/lishi/">知识</a> <b>|</b> <a href="/zhongyi/">中医中药</a> | <a href="http://www.guoxuedashi.com/forum/">留言反馈</a>
</p>
  </div>
</div>
<!-- 头部导航END --> 
<!-- 内容区开始 --> 
<div class="w1180 clearfix">
  <div class="info l">
   
<div class="clearfix" style="background:#f5faff;">
<script src='http://www.guoxuedashi.com/img/headersou.js'></script>

</div>
  <div class="info_tree"><a href="http://www.guoxuedashi.com">首页</a> > <a href="/SiKuQuanShu/fanti/">四库全书</a>
 > <h1>资治通鉴</h1> <!--         下载:【右键另存为】即可 --></div>
  <div class="info_content zj clearfix">
  
<div class="info_txt clearfix" id="show">
<center style="font-size:24px;">191-資治通鑑卷一百九十</center>
    資治通鑑卷一百九十  宋 司馬光 撰<br />
<br />
  胡三省 音注<br />
<br />
  唐紀六【起玄黓敦牂盡閼逢涒灘五月凡二年有奇】<br />
<br />
  高祖神堯大聖光孝皇帝中之下<br />
<br />
  武德五年春正月劉黑闥自稱漢東王改元天造定都洺州以范願為左僕射董康買為兵部尚書高雅賢為右領軍徵王琮為中書令劉斌為中書侍郎【琮藏宗翻斌音彬】竇建德時文武悉復本位其設法行政悉師建德而攻戰勇決過之 丙戌同安賊帥殷恭邃以舒州來降【舒州隋之同安郡宋白曰舒州漢皖縣屬廬江郡晉置晉熙郡梁置南豫州復為晉州北齊改江州隋初改熙州大業改同安郡唐改舒州以其地古舒子之國也帥所類翻降戶江翻】 丁亥濟州别駕劉伯通執刺史竇務本以州附徐圓朗【濟子禮翻】 庚寅東鹽州治中王才藝殺刺史田華以城應劉黑闥【滄州鹽山縣本漢高成縣地去年置東鹽州以清池縣隸之】 秦王世民軍至獲嘉【按宋白續通典隋開皇四年移獲嘉縣於修武故城】劉黑闥弃相州退保洺州【宋白曰洺州城春秋晉曲梁之地相息亮翻】丙申世民復取相州 【考異曰實録云禄州人殺刺史獨孤徹以城應黑闥按地里志無禄州蓋字誤耳新書作相州尤誤也 按劉黑闥攻拔相州執刺史房晃秦王兵至乃弃相州故秦王復取之新書帝紀拔相州殺房晃在正月乙酉相州人殺獨孤徹叛附黑闥在丙申其誤明矣禄州既莫考其地然黑闥之拔相州與秦王之復相州本末甚明與禄州全不相干而新書所書殺房晃拔相州月日亦誤復扶又翻又音如字】進軍肥鄉【肥鄉縣漢魏郡邯溝縣地曹魏置肥鄉縣屬廣平郡武德初屬紫州去年廢紫州以肥鄉屬磁州九域志肥鄉在洺州東南三十五里】列營洺水之上以逼之【水經洺水東逕曲梁城曲梁即永年洺州所治也】蕭銑既敗散兵多歸林士弘軍勢復振【林士弘為張善安所敗兵勢自此衰見一百八十四卷隋義寧元年】己酉嶺南俚帥楊世畧以循潮二州來降【循州龍川郡秦漢龍川博羅縣地南朝置郡隋置州潮州義安郡漢南海郡掲陽縣地梁置東揚州尋改瀛州隋平陳改潮州俚音里帥所類翻降戶江翻下同】唐使者王義童下泉睦建三州【睦州遂安郡漢富春歙縣地劉昫曰武德四年以建安郡之建安縣置建州蓋隋置泉州建安郡治閩縣景雲元年改為閩州開元十三年改為福州聖歷二年分泉州之南安龍溪莆田三縣置武榮州景雲二年更武榮州為泉州是今之福州乃唐初之泉州今之泉州乃景雲二年之泉州也使疏吏翻】幽州摠管李藝將所部兵數萬會秦王世民討劉黑闥黑闥聞之留兵萬人使范願守洺州自將兵拒藝夜宿沙河【隋以漢襄國縣為龍岡縣又分龍岡置沙河縣時屬邢州宋白曰沙河即湡水也水經云湡水出趙郡襄國縣西山東過沙河縣沙河在縣南五里范成大曰臨洺鎮東去洺州三十五里過洺河三十里至沙河縣二十五里至邢州將即亮翻下同】程名振載鼔六十具於城西二里隄上急擊之城中地皆震動范願驚懼馳告黑闥黑闥遽還遣其弟十善與行臺張君立將兵一萬擊藝於鼓城壬子戰於徐河【鼓城縣舊曰曲陽後齊廢隋開皇十六年分置晉陽縣十八年改為鼔城按水經註下曲陽縣有鼓聚此因春秋鼓子之國以名縣也唐屬趙州水經徐水出廣昌縣東南大嶺下東北逕五回縣又逕北平縣界東南出山又東逕清苑城北又東至高陽入河范成大曰徐河在清苑北十里】十善君立大敗所失亡八千人 洺水人李去惑【隋志洺水縣舊曰斥漳後齊省入平恩開皇六年分置曲周大業初廢曲周入洺水唐末省洺水入曲周去羌呂翻】據城來降秦王世民遣彭公王君廓將千五百騎赴之入城共守二月劉黑闥引兵還攻洺水癸亥行至列人【列人縣漢屬廣平國後漢屬魏郡晉復屬廣平後魏復屬魏郡隋廢九域志洺州有列人城在漢斥丘縣東北武安縣西南杜佑曰漢列人縣故城在肥鄉縣東北騎奇寄翻】秦王世民使秦叔寶邀擊破之 【考異曰實録癸亥秦王擊劉黑闥於列人大破之革命記十一月太宗度河入相州劉黑闥從洺州勒兵拒王師置營於鄴縣東三十里每日兩軍皆挑戰而大兵皆不出經十日餘洺水縣人李去惑李潘買李開弼等為車騎驃騎領兵在劉黑闥營去惑等背賊營來入洺州城誑人云劉黑闥已敗先走得歸乃喚得宗室子弟二百餘人守城定遣使間道以告太宗太宗遣彭國公王君廓領馬軍一千五百騎入洺州經十許日黑闥引兵攻洺州行至故列人城西秦叔寶等以五千騎擊之叔寶等為闥所敗又以伏兵從河下起横擊黑闥敗之會日暮收軍其夜三更賊兵摠至洺州城東營即於城兩門掘壕豎柵防王廓之走洺州城四面有水闊五十步已上深皆三四尺黑闥於東北角兩處填柴運土作甬道以撞車攻城太宗三度將兵擊之賊置陣拒官軍攻城愈急按高祖太宗實録皆以去年十二月命太宗討黑闥今年正月始至河北無十一月度河之事太宗實録亦無列人戰事蓋叔寶破賊秦王奏之耳又按洺水洺州屬縣去惑君廓所據者洺水縣城水字誤作州耳】 豫章賊帥張善安以䖍吉等五州來降拜洪州總管【洪州豫章郡䖍州南康郡吉州廬陵郡皆隋平陳所置州帥所類翻】 戊辰金鄉人陽孝誠叛徐圓朗以城來降【金鄉縣漢屬山陽郡晉屬高平郡隋屬曹州武德四年晉金州是年廢州以縣屬戴州】 己巳秦王世民復取邢州辛未井州人馮伯讓以城來降【隋開皇十六年以恒州井陘縣置井州大業初廢武德元年復置 考異曰實録作并州按并州未嘗失城蓋是時於井陘縣置井州字之誤也】 丙子李藝取劉黑闥定欒亷趙四州【隋開皇十六年分趙州廣阿縣置欒州大業初州廢併為趙州新志武德五年改趙州為欒州按趙州治平棘欒州治廣阿竇建德劉黑闥相繼跨有山東蓋自置欒州是年黑闥破走之後始併趙州為欒州也武德元年分恒州藁城縣置亷州 考異曰實録作定率亷隋四州按河北無率隋二州今從唐統紀】獲黑闥尚書劉希道引兵與秦王世民會洺州【洺彌并翻】 劉黑闥攻洺水甚急城四旁皆有水廣五十餘步【廣古曠翻】黑闥於城東北築二甬道以攻之世民三引兵救之黑闥拒之不得進世民恐王君廓不能守召諸將謀之【將即亮翻】李世勣曰若甬道達城下城必不守行軍總管郯勇公羅士信請代君廓守之世民乃登城南高冢以旗招君廓君廓帥其徒力戰潰圍而出士信帥左右二百人乘之入城代君廓固守【帥讀曰率】黑闥晝夜急攻會大雪救兵不得往凡八日丁丑城陷黑闥素聞其勇欲生之士信詞色不屈乃殺之時年二十 【考異曰高祖實録王君廓知不可守潰圍而出秦王謂諸將曰誰能代者士信曰願以死守因遣之按君廓若已潰圍而出則黑闥圍守益固士信何以復得入城革命記曰太宗知賊勢盛恐王君廓不能固以問諸將士信以為無慮太宗使士信入守之太宗登段王墓以旗招王君廓從南門突闈不得即向北門併兵攻捉門人少退得出士信亦以左右二百人入城經八日晝夜被攻木石俱盡士信被左右執之以降賊五年正月城陷李去惑以數十人突圍出歸太宗去惑後授秦州都督李潘買拜檀州刺史李開弼城陷而没贈上柱國以公禮葬今從之高祖實録士信死時年二十八舊傳云年二十按士信始從張須陁擊王薄等時年十四若死時年二十八則在大業四年於時王薄未為盜年二十則在大業十二年是歲須陁死今從之】 戊寅汴州總管王要漢攻徐圓朗杞州拔之獲其將周文舉【將即亮翻下同】 庚辰延州道行軍總管段德操【延州漢上郡膚施之地元魏之末置東夏州西魏改曰延州隋曰延安郡】擊梁師都石堡城師都自將救之【將即亮翻】德操與戰大破之師都以十六騎遁去上益其兵使乘勝進攻夏州克其東城【騎奇寄翻夏戶雅翻】師都以數百人保西城會突厥救至【厥九勿翻】詔德操引還【還從宣翻】 辛巳秦王世民拔洺水【洺彌并翻】三月世民與李藝營於洺水之南分兵屯水北黑闥數挑戰【數所角翻挑徒了翻】世民堅壁不應别遣奇兵絶其糧道壬辰黑闥以高雅賢為左僕射軍中高會李世勣引兵逼其營雅賢乘醉單騎逐之世勣部將潘毛刺之墜馬【將即亮翻刺七亦翻】左右繼至扶歸未至營而卒【卒子恤翻】甲午諸將復往逼其營【復扶又翻】潘毛為王小胡所擒黑闥運糧於冀貝滄瀛諸州水陸俱進程名振以千餘人邀之沈其舟焚其車【沈持林翻】 宋州總管盛彦師帥齊州總管王薄攻須昌【帥讀曰率】徵軍糧於潭州刺史李義滿與薄有隙閉倉不與及須昌降【降戶江翻】彦師收義滿繫齊州獄詔釋之使者未至義滿憂憤死獄中【使疏吏翻】薄還過潭州戊戌夜義滿兄子武意執薄殺之【潭州當作譚州武德二年李義滿以齊州之章丘縣來降於平陵置譚州并置平陵縣蓋因春秋譚子之國以名州也】彦師亦坐死【坐死者朝廷以義滿之死為彦師罪而殺之】上遣使賂突厥頡利可汗且許結昏【使疏吏翻厥九勿翻可從刋入】<br />
<br />
  【聲汗音寒】頡利乃遣漢陽公瓌鄭元璹長孫順德等還【瓌等被留事見上卷四年瓌工回翻璹殊玉翻長知兩翻】庚子復遣使來修好【復扶又翻好呼到翻】上亦遣其使者特勒熱寒阿史那德等還【還從宣翻】并州總管劉世讓屯鴈門頡利與高開道苑君璋合衆攻之月餘乃退 【考異曰舊世讓傳云時鴻臚卿鄭元璹先使在蕃可汗令元璹來說之世讓厲聲曰大丈夫乃為夷狄作說客邪經月餘虜乃退及元璹還述世讓忠貞勇幹高祖下制褒美之按高祖稱元璹蘇武弗之過安肯為可汗遊說脱或果爾則元璹唯恐帝知之安肯稱世讓忠貞說之不下邪據實録世讓傳無此事今不取】 甲辰以隋交趾太守丘和為交州總管【以交趾郡為交州宋白曰交州周為越裳重譯之地漢交趾日南二郡界按交趾之稱今南方夷人其足大指開廣若並足而立其指相交故曰交趾吳黄武中以交趾縣遠分為二州割合浦以北海東四郡立廣州交趾以南立交州守式又翻】和遣司馬高士亷奉表請入朝【朝直遙翻】詔許之遣其子師利迎之【義師之向長安丘師利以兵來附】 秦王世民與劉黑闥相持六十餘日黑闥潛師襲李世勣營世民引兵掩其後以救之為黑闥所圍尉遲敬德帥壯士犯圍而入【尉紆勿翻帥讀曰率】世民與畧陽公道宗乘之得出道宗帝之從子也【從才用翻】世民度黑闥糧盡必來決戰【度徒洛翻】乃使人堰洺水上流【洺彌并翻】謂守吏曰待我與賊戰乃決之丁未黑闥帥步騎二萬南度洺水壓唐營而陳【陳讀曰陣】世民自將精騎擊其騎兵破之乘勝蹂其步兵【蹂人九翻】黑闥帥衆殊死戰【帥讀曰率】自午至昏戰數合黑闥勢不能支王小胡謂黑闥曰智力盡矣宜蚤亡去遂與黑闥先遁餘衆不知猶格戰守吏決堰洺水大至深丈餘【洺彌并翻深式禁翻】黑闥衆大潰斬首萬餘級溺死數千人黑闥與范願等二百騎奔突厥山東悉平【騎奇寄翻厥九勿翻按秦王之討黑闥前後接戰黑闥之衆皆決死確鬭特秦王大展方畧黑闥智力俱困而敗走耳秦王之平羣盜黑闥最為堅敵】 高開道寇易州殺刺史慕容孝幹 夏四月己未隋鴻臚卿甯長真以寧越鬱林之地請降於李靖【臚陵如翻降戶江翻】交愛之道始通以長真為欽州總管【寧越郡欽州漢合浦縣地宋為宋夀宋廣郡鬱林郡鬱州漢古郡隋置州愛州九真郡漢古郡梁置州劉昫曰交州至京師七千五百二十三里愛州至京師八千八百里】 以夔州總管趙郡王孝恭為荆州總管 徐圓朗聞劉黑闥敗大懼不知所出河間人劉復禮說圓朗曰有劉世徹者其才不世出名高東夏【說輸芮翻夏戶雅翻】且有非常之相【相息亮翻】真帝王之器將軍若自立恐終無成若迎世徹而奉之天下指揮可定圓朗然之使復禮迎世徹於浚儀【浚儀縣漢晉後魏屬陳留郡周隋唐為汴州治所】或說圓朗曰將軍為人所惑欲迎劉世徹而奉之世徹若得志將軍豈有全地乎僕不敢遠引前古將軍獨不見翟讓之於李密乎【李密翟讓事始一百八十三卷隋大業十二年終一百八十四卷義寧元年翟萇伯翻 考異曰革命記云盛彦師以世徹有虛名於徐兖恐二人相得為患益深因說圓朗使不納按實録彦師奔王薄與薄共殺李義滿三月戊戌王薄死丁未黑闥乃敗彦師在圓朗所時黑闥未敗也今稱或說以闕疑】圓朗復以為然【復扶又翻下同】世徹至已有衆數千人頓於城外以待圓朗出迎圓朗不出使人召之世徹知事變欲亡走恐不免乃入謁圓朗悉奪其兵以為司馬使徇譙杞二州東人素聞其名所向皆下圓朗遂殺之秦王世民自河北引兵將擊圓朗會上召之使馳傳入朝乃以兵屬齊王元吉【傳株戀翻屬之欲翻付也】庚申世民至長安上迎之於長樂【長樂坂在長安城東樂音洛】世民具陳取圓朗形勢上復遣之詣黎陽會大軍趨濟隂【曹州治濟隂縣趨七喻翻濟子禮翻】 丁卯廢山東行臺【劉黑闥敗走故也】壬申代州總管定襄王李大恩為突厥所殺【厥九勿翻】先<br />
<br />
  是大恩奏稱突厥饑饉馬邑可取【先悉薦翻】詔殿内少監獨孤晟將兵與大恩共擊苑君璋期以二月會馬邑失期不至大恩不能獨進頓兵新城【新城當在朔州南郡佑曰魏都平城於馬邑郡北三百餘里置懷朔鎮及遷洛後於郡北三百餘里置朔州魏初雲州郡在今郡北三百餘里定襄故城北北齊置朔州在故都西南新城一名平城後移於馬邑即今郡城也馬邑郡治善陽縣亦漢定襄縣地新城即魏之新平城也晟承正翻將即亮翻】頡利可汗遣數萬騎與劉黑闥共圍大恩上遣右驍衛大將軍李高遷救之【騎奇寄翻驍堅堯翻】未至大恩糧盡夜遁突厥邀之衆潰而死上惜之獨孤晟坐減死徙邊 丙子行臺民部尚書史萬寶攻徐圓朗陳州拔之【宋白曰陳州楚襄王自郢徙此謂之西楚漢以為淮陽國之地後魏立陳郡天平二年置北揚州隋改陳州治宛丘春秋陳國都也】 戊寅廣州賊帥鄧文進隋合浦太守甯宣日南太守李晙並來降【合浦郡越州貞觀改亷州日南郡德州貞觀改驩州帥所類翻守式又翻晙子峻翻】五月庚寅瓜州土豪王幹斬賀拔行威以降【瓜州晉昌郡是年分瓜州之常樂縣置瓜州舊瓜州為西沙州賀拔行威反事始見一百八十八卷三年】瓜州平突厥寇忻州【忻州新興郡義寧元年以樓煩郡秀容縣置此秀容漢汾陽縣地非後魏之北秀容也】李高遷擊破之 六月辛亥劉黑闥引突厥寇山東詔燕郡王李藝擊之【厥九勿翻燕因肩翻】 癸丑吐谷渾寇洮旭疊三州岷州總管李長卿擊破之【後周武帝逐吐谷渾置疊州於疊川旭州於洮源岷州於臨洮義寧元年改臨洮於溢樂後周書曰於河州雞鳴防置旭州於宕州渠株川置岷州吐從暾入聲谷音浴洮土刀翻旭吁玉翻長知兩翻】 乙卯遣淮安王神通擊徐圓朗 丁卯劉黑闥引突厥寇定州 秋七月甲申為秦王世民營弘義宫【弘義宫後改為大安宫在宫城外西偏為于偽翻】使居之世民擊徐圓朗下十餘城聲振淮泗杜伏威懼請入朝【朝直遙翻下同】世民以淮濟之間畧定【濟子禮翻】使淮安王神通行軍總管任瓌李世勣攻圓朗乙酉班師 丁亥杜伏威入朝延升御榻拜太子太保仍兼行臺尚書令留長安位在齊王元吉上以寵異之以闞稜為左領軍將軍【漢建安中魏武為丞相始置中領軍北齊置領軍府煬帝改為禦衛唐改為領軍衛】李子通謂樂伯通曰伏威既來江東未定我往收舊兵可以立大功遂相與亡至藍田關【雍州藍田縣有藍田關】為吏所獲俱伏誅 劉黑闥至定州其故將曹湛董康買亡命在鮮虞【鮮虞縣舊曰盧奴開皇初更名以其地春秋鮮虞子之國也定州治所】復聚兵應之【復扶又翻】甲午以淮陽王道玄為河北道行軍總管以討之丙申遷州人鄧士政執刺史李敬昂以反【西魏以房陵置遷州大業初改曰房州武德初復曰遷州房陵郡】 丁酉隋漢陽太守馮盎承李靖檄帥所部來降【守式又翻帥讀曰率降戶江翻下同】以其地為高羅春白崖儋林振八州【高州高涼郡羅州石城郡春州陽春郡白州南昌郡崖州珠崖郡林州桂林郡振州臨振郡羅州今化州白州今鬱林州之博白縣崖儋振皆在海外振州今吉陽軍地林州後改繡州投荒録高州高涼郡土厚而山環遶高而稍汙故名羅州宋將檀道濟於陵羅江口築城因置羅州今廢陵羅縣在化州北一百二十里宋白曰州在陵羅二水之間春州治陽春縣故名白州以博白江名崖州以珠崖名儋州以儋耳名林州以綏懷林邑名振州以漢臨振古縣而名儋丁甘翻】以盎為高州總管封耿國公先是或說盎曰唐始定中原未能及遠公所領二十州地已廣於趙佗【趙佗見漢紀先悉薦翻說輸芮翻佗徒何翻】宜自稱南越王盎曰吾家居此五世矣【馮氏居高涼事始見一百六十三卷梁簡文帝大寶元年】為牧伯者不出吾門富貴極矣常懼不克負荷為先人羞敢効趙佗自王一方乎【荷下可翻又如字王於况翻】遂來降於是嶺南悉平 八月辛亥以洺荆交并幽五州為大總管府【洺彌并翻并卑名翻】改葬隋煬帝於揚州雷塘【雷塘漢所謂雷陂也在今揚州城北平岡上 考異曰實録武德三年六月癸巳已有詔葬隋帝及子孫此又云葬煬帝蓋三年李子通猶據江都雖有是詔不果葬也】 甲戌吐谷渾寇岷州敗總管李長卿【吐從暾入聲谷音浴敗補邁翻長知兩翻】詔益州行臺右僕射竇軌渭州刺史且洛生救之【渭州隴西郡且姓也音子余翻】 乙卯突厥頡利可汗寇邊【厥九勿翻可從刊入聲汗音寒】遣左武衛將軍段德操雲州總管李子和將兵拒之子和本姓郭【郭子和武德三年以榆林郡降榆林之地本屬雲州隋割置勝州榆林郡子和以榆林降因命之為雲州總管】以討劉黑闥有功賜姓丙辰頡利十五萬騎入鴈門【頡奚結翻騎奇寄翻】己未寇并州别遣兵寇原州庚子命太子出幽州道【幽州當作豳州】秦王世民出秦州道以禦之【秦州當作泰州出豳州以禦原州之寇出泰州以禦并州之寇泰州時治龍門】李子和趨雲中掩擊可汗【漢雲中古城在榆林郡東北四十里趨七喻翻】段德操趨夏州邀其歸路【夏戶雅翻】辛酉上謂羣臣曰突厥入寇而復求和【厥九勿翻復扶又翻】和與戰孰利太常卿鄭元璹曰戰則怨深不如和利【璹殊玉翻】中書令封德彞曰突厥恃犬羊之衆有輕中國之意若不戰而和示之以弱明年將復來【復扶又翻】臣愚以為不如擊之既勝而後與和則恩威兼著矣上從之己巳并州大總管襄邑王神符破突厥於汾東汾州刺史蕭顗破突厥【浩州西河郡武德三年更名汾州顗魚豈翻】斬首五千餘級 吐谷渾寇洮州遣武州刺史賀亮禦之【武都郡西魏置武州唐後改曰階州洮土刀翻】 丙子突厥寇亷州戊寅陷大震關【大震關在隴州汧源縣當隴山之路程大昌曰漢武帝至此遇雷大震因以為名按寇亷州者并州之寇陷大震關者原州之寇也】上遣鄭元璹詣頡利是時突厥精騎數十萬自介休至晉州數百里間填溢山谷元璹見頡利責以負約與相辯詰【詰去吉翻】頡利頗慙元璹因說頡利曰唐與突厥風俗不同突厥雖得唐地不能居也今虜掠所得皆入國人於可汗何有不如還師復修和親【說輸芮翻復扶又翻】可無跋涉之勞坐受金幣又皆入可汗府庫【可從刋入聲汗音寒】孰與弃昆弟積年之歡而結子孫無窮之怨乎頡利悦引兵還【還從宣翻】元璹自義寧以來五使突厥幾死者數焉【使疏吏翻幾居依翻數所角翻】 九月癸巳交州刺史權士通【西魏置北秦州於上郡廢帝三年改曰交州】弘州總管宇文歆【慶州弘化縣開皇十八年置弘州大業初州廢蓋唐復置也】靈州總管楊師道擊突厥於三觀山破之【觀古玩翻】乙未太子班師丙申宇文歆邀突厥於崇崗鎮大破之斬首千餘級【歆許今翻】壬寅定州總管雙士洛擊突厥於恒山之南【雙姓也姓苑曰黄帝後封於雙蒙城恒戶登翻】丙午領軍將軍安興貴擊突厥於甘州皆破之【甘州張掖郡】 劉黑闥陷瀛州殺刺史馬匡武鹽州人馬君德以城叛附黑闥【此鹽州即東鹽州】高開道寇蠡州【武德五年以瀛州之博野清苑定州之義豐置蠡州因漢蠡吾亭以名州也】 冬十月己酉詔齊王元吉討劉黑闥於山東壬子以元吉為領軍大將軍并州大總管癸丑貝州刺史許善護與黑闥弟十善戰於鄃縣【鄃音輸】善護全軍皆没甲寅右武候將軍桑顯和擊黑闥於晏城破之【隋開皇十六年分冀州之鹿城置晏城縣大業初廢入鹿城蓋其縣名猶在】觀州刺史劉會以城叛附黑闥【觀古喚翻】 契丹寇北平【北平郡平州契欺訖翻又音喫】 甲子以秦王世民領左右十二衛大將軍【左右衛左右驍衛左右武衛左右屯衛左右領軍衛左右候衛為十二衛】 乙丑行軍總管淮陽壯王道玄與劉黑闥戰於下博【下博縣屬冀州 考異曰高祖實録謚曰忠本傳諡曰壯蓋後來改諡也】軍敗為黑闥所殺時道玄將兵三萬與副將史萬寶不協道玄帥輕騎先出犯陳【將即亮翻帥讀曰率騎奇寄翻陳讀曰陣下同】使萬寶將大軍繼之萬寶擁兵不進謂所親曰我奉手敕云淮陽小兒軍事皆委老夫今王輕脱妄進若與之俱必同敗没不如以王餌賊王敗賊必爭進我堅陳以待之破之必矣由是道玄獨進敗没萬寶勒兵將戰士卒皆無鬭志軍遂大潰萬寶逃歸道玄數從秦王世民征伐死時年十九【數所角翻】世民深惜之謂人曰道玄常從吾征伐見吾深入賊陳心慕効之以至於此為之流涕【為於偽翻】世民自起兵以來前後數十戰常身先士卒【先悉薦翻】輕騎深入雖屢危殆而未嘗為矢刃所傷【史言秦王有天命】 林士弘遣其弟鄱陽王藥師攻循州刺史楊畧與戰斬之其將王戎以南昌州降士弘懼己巳請降尋復走保安成山洞袁州人相聚應之【是年以洪州建昌縣置南昌州吉州安復本吳所置安成縣也唐後改為安福袁州宜春郡將即亮翻降戶江翻復扶又翻】洪州總管若于則遣兵擊破之會士弘死其衆遂散【隋大業十三年林士弘起為盜至是死散】淮陽王道玄之敗也山東震駭洺州總管廬江王瑗弃城西走【洺彌并翻瑗於眷翻】州縣皆叛附於黑闥旬日間黑闥盡復故地乙亥進據洺州十一月庚辰滄州刺史程大買為黑闥所迫棄城走齊王元吉畏黑闥兵彊不敢進上之起兵晉陽也皆秦王世民之謀【事見一百八十二卷隋義寧元年】上謂世民曰若事成則天下皆汝所致當以汝為太子世民拜且辭及為唐王將佐亦請以世民為世子【將即亮翻】上將立之世民固辭而止太子建成性寛簡喜酒色【喜許記翻】遊畋【句斷】齊王元吉多過失皆無寵於上世民功名日盛上常有意以代建成建成内不自安乃與元吉協謀共傾世民各引樹黨友上晩年多内寵小王且二十人【尹德妃生酆王元亨莫嬪生荆王元景孫嬪生漢王元昌宇文昭儀生韓王元嘉魯王靈夔瞿嬪生鄧王元裕楊嬪生江王元祥小楊嬪生舒王元名郭婕妤生徐王元禮劉婕妤生道王元慶楊美人生虢王鳳張美人生霍王元軌張寶林生鄭王元懿柳寶林生滕王元嬰王才人生彭王元則魯才人生密王元曉張氏生周王元方凡十七人且者將及未及之辭】其母競交結諸長子以自固建成與元吉曲意事諸妃嬪諂諛賂遺無所不至【遺於季翻】以求媚於上或言蒸於張婕妤尹德妃宫禁深祕莫能明也是時東宫諸王公妃主之家及後宫親戚横長安中【横戶孟翻下同】恣為非法有司不敢詰世民居承乾殿【閣本太極宫圖月華門内有承慶殿無承乾殿按新書承乾殿在西宫又按王溥會要承乾殿在宫中蓋皆指太極官】元吉居武德殿後院【武德殿在東宫西按閣本太極宫圖武德殿在䖍化門東入門過内倉廪立政殿萬春殿即東上閭門】與上臺東宫晝夜通行無復禁限【上臺謂帝居復扶又翻】太子二王出入上臺皆乘馬攜弓刀雜物相遇如家人禮太子令秦齊王教與詔敕並行有司莫知所從唯據得之先後為定【使唐之政終於如此亡隋之續耳】世民獨不奉事諸妃嬪諸妃嬪爭譽建成元吉而短世民【譽音余】世民平洛陽上使貴妃等數人詣洛陽選閲隋宫人及收府庫珍物貴妃等私從世民求寶貨及為親屬求官【唐制皇后而下有貴妃淑妃德妃賢妃是為夫人昭儀昭容昭媛脩儀脩容脩媛充儀充容充媛是為九嬪婕妤美人才人各九合二十七是為世婦寶林御女采女各二十七合八十一是為御妻為于偽翻】世民曰寶貨皆已籍奏官當授賢才有功者皆不許由是益怨世民以淮安王神通有功給田數十頃張婕妤之父因婕妤求之於上【婕妤音接予】上手敕賜之神通以教給在先不與婕妤訴於上曰敕賜妾父田秦王奪之以與神通上遂發怒責世民曰我手敕不如汝教邪【邪音耶】它日謂左僕射裴寂曰此兒久典兵在外為書生所教非復昔日子也尹德妃父阿鼠驕横【阿烏葛翻】秦王府屬杜如晦過其門阿鼠家童數人曳如晦墜馬敺之折一指【敺烏口翻折而設翻】曰汝何人敢過我門而不下馬阿鼠恐世民訴於上先使德妃奏云秦王左右陵暴妾家上復怒【復扶又翻】責世民曰我妃嬪家猶為汝左右所陵况小民乎世民深自辯析上終不信世民每侍晏宫中對諸妃嬪思太穆皇后早終【竇皇后謚太穆帝未即位先崩建成世民玄霸元吉皆其所生也】不得見上有天下或歔欷流涕上顧之不樂【歔音虚欷音希又許既翻樂音洛下同】諸妃嬪因密共譖世民曰海内幸無事陛下春秋高唯宜相娛樂而秦王每獨涕泣正是憎疾妾等陛下萬歲後妾母子必不為秦王所容無孑遺矣【言必皆誅翦無有孑然見遺者也】因相與泣且曰皇太子仁孝陛下以妾母子屬之【屬之欲翻】必能保全上為之愴然【為于偽翻】由是無易太子意待世民浸疎而建成元吉日親矣 【考異曰高祖實録曰建成幼不拘細行荒色嗜酒好畋獵常與博徒遊故時人稱為任俠高祖起義於太原建成時在河東本既無寵又以今上首建大計高祖不之思也而今上白高祖遣使召之盤游不即往今上急難情切遽以手書諭之建成乃與元吉間行赴太原隋人購求之幾為所獲及義旗建而方至高祖亦喜其獲免因授以兵又曰建成帷簿不修有禽獸之行聞於遠邇今上以為恥嘗流涕諫之建成慙而成憾又曰太宗每摠戎律惟以撫接才賢為務至於參請妃媛素所不行太宗實録曰隱太子始則流宕河曲逸遊是好素無才畧不預經綸於後雖統左軍非衆所附既陞儲兩坐構猜嫌太宗雖備禮竭誠以希恩睦而妬害之心日以滋甚又巢刺王性本兇愎志識庸下行同禽獸兼以弃鎮失守罪戾尤多反害太宗之能於是潛苞毁譖同惡相濟膚受日聞雖大名徽號禮冠羣后而情疎意隔寵異曩時按建成元吉雖為頑愚既為太宗所誅史臣不無抑揚誣諱之辭今不盡取】太子中允王珪洗馬魏徵【太子左春坊左庶子為之長掌侍從贊相敷正啟奏中允為之貳洗馬漢官掌前馬唐為司經局長官掌四庫圖籍繕寫刋緝之事唐六典曰後漢太子官屬有中允在中庶子下洗馬上其後無聞唐始置太子中允洗悉薦翻】說太子曰秦王功蓋天下中外歸心殿下但以年長位居東宫【說輸芮翻長音知兩翻】無大功以鎮服海内今劉黑闥散亡之餘衆不滿萬資糧匱乏以大軍臨之勢如拉朽【拉音盧合翻】殿下宜自擊之以取功名因結納山東豪傑庶可自安太子乃請行於上上許之珪頍之兄子也【王頍僧辯之子死於隋漢王諒反時頍音丘弭翻】甲申詔太子建成將兵討黑闥其陜東道大行臺及山東道行軍元帥河南河北諸州並受建成處分【將即亮翻陜失冉翻帥所類翻處昌呂翻分扶問翻】得以便宜從事 乙酉封宗室畧陽公道宗等十八人為郡王道宗道玄從父弟也【從才用翻】為靈州總管梁師都遣弟洛兒引突厥數萬圍之道宗乘間出擊大破之【厥九勿翻間古莧翻】突厥與師都相結遣其郁射設入居故五原【五原縣屬鹽州武德初寄治靈州故地為突厥所居杜佑曰鹽州西魏五原郡地漢五原縣城在今榆林郡界】道宗逐出之斥地千餘里【毛晃曰斥開拓也】上以道宗武幹如魏任城王彰【魏任城王彰曹操之子擊烏丸有功任音壬】乃立為任城郡王 丙申上幸宜州【義寧二年以京兆之華原宜君同官置宜君郡武德元年曰宜州】 己亥齊王元吉遣兵擊劉十善於魏州破之 癸卯上校獵於富平【富平縣屬雍州漢富平縣治唐靈州迴樂縣界後漢移寧州彭原縣界晉移懷德城魏移於懷德城東北今耀州東南富平縣即其地】 劉黑闥擁兵而南自相州以北州縣皆附之【相息亮翻】唯魏州摠管田留安勒兵拒守黑闥攻之不下引兵南拔元城復還攻之【元城縣治古郡城在朝城東北十二里時魏州治貴鄉縣相息亮翻復扶又翻 考異曰實録十二月甲子黑闥攻魏州蓋留安破黑闥奏到之日也案革命記黑闥攻魏州在十一月今從之】 十二月庚戌立宗室孝友等八人為郡王孝友神通之子也 丙辰上校獵於華池【京兆三原縣武德四年改池陽六年改華池】 戊午劉黑闥陷恒州殺刺史王公政【恒戶登翻恒州漢常山郡唐置恒州因恒山為名】 庚申車駕至長安 癸亥幽州大摠管李藝復亷定二州 甲子田留安擊劉黑闥破之獲其莘州刺史孟柱【魏州莘縣隋開皇十六年置莘州大業二年廢唐復置】降將卒六千人【降戶江翻將即亮翻】是時山東豪傑多殺長吏以應黑闥【長知兩翻】上下相猜人益離怨留安待吏民獨坦然無疑白事者無問親疎皆聽直入卧内每謂吏民曰吾與爾曹俱為國禦賊【為于偽翻】固宜同心協力必欲弃順從逆者但自斬吾首去吏民皆相戒曰田公推至誠以待人當共竭死力報之必不可負有苑竹林者本黑闥之黨潛有異志留安知之不發其事引置左右委以管鑰竹林感激遂更歸心卒收其用【卒子恤翻】以功進封道國公乙丑并州刺史成仁重擊范願破之【并卑名翻】 劉黑闥攻魏州未下太子建成齊王元吉大軍至昌樂【劉昫曰晉置昌樂縣屬陽平郡今縣西古城是也隋廢縣入繁水武德元年復置仍築今治所樂音洛】黑闥引兵拒之再陳皆不戰而罷【陳讀曰陣】魏徵言於太子曰前破黑闥其將帥皆懸名處死【言亡命者先書其名處以死罪也將即亮翻帥所類翻下同處昌呂翻】妻子係虜故齊王之來雖有詔書赦其黨與之罪皆莫之信今宜悉解其囚俘慰諭遣之則可坐視離散矣太子從之黑闥食盡衆多亡或縛其渠帥以降【帥所類翻降戶江翻】黑闥恐城中兵出與大軍表裏擊之遂夜遁至館陶永濟橋未成不得度【館陶縣屬魏州在州北隋煬帝鑿永濟渠所經也宋白曰館陶春秋時晉冠氏邑陶丘在縣西北七里趙時置館于其側因為縣名】壬申太子齊王以大軍至黑闥使王小胡背水而陳【背蒲妹翻陳讀曰陣】自視作橋成即過橋西衆遂大潰 【考異曰高祖實録壬申太子與黑闥戰於魏州城下破之闥抽軍北遁甲戌追闥於毛州賊背永濟渠而陳接戰又破之舊傳六年二月太子破黑闥于館陶革命記闥遁至館陶二十五日官軍至闥敗走按館陶即毛州也長歷十二月壬申二十五日甲戌二十七日蓋實録據奏到之日也舊傳尤疎今從革命記太宗實録云黑闥重反高祖謂太宗曰前破黑闥欲令盡殺其黨使空山東不用吾言致有今日及隱太子征闥平之將遣唐儉往使男子年十五已上悉阬之小弱及婦女摠驅入關以實京邑太宗諫曰臣聞惟德動天唯恩容衆山東人物之所河北蠶綿之鄉而天府委輸待以成績今一旦見其反覆盡戮無辜流離寡弱恐以殺不能止亂非行弔伐之道其事遂寢新書隱太子傳云黑闥敗於洺水太子建成問於洗馬魏徵曰山東其定乎對曰黑闥雖敗殺傷太甚其魁黨皆縣名處死妻子係虜欲降無繇雖有赦令獲者必戮不大蕩宥恐殘賊嘯結民來可安既而黑闥復振廬江王瑗弃洺州山東亂命齊王元吉討之有詔降者赦罪衆不信建成至獲俘皆撫遣之百姓欣悦賊懼夜奔兵追戰黑闥衆猶盛乃縱囚使相告曰禠而甲還鄉里若妻子獲者既已釋矣衆乃散或縛其渠長降遂禽黑闥按高祖雖不仁亦不至有欲空山東之理史臣專欲歸美太宗其於高祖亦太誣矣今采革命記及新書】捨仗來降大軍度橋追黑闥度者纔千餘騎橋壞由是黑闥得與數百騎亡去【降戶江翻下同騎奇寄翻下同】 上以隋末戰士多没於高麗是歲賜高麗王建武書使悉遣還亦使州縣索高麗人在中土者遣歸其國【麗力知翻索山客翻】建武奉詔遣還中國民前後以萬數<br />
<br />
  六年春正月己卯劉黑闥所署饒州刺史諸葛德威執黑闥舉城降時太子遣騎將劉弘基追黑闥黑闥為官軍所迫奔走不得休息至饒陽【饒陽縣前漢屬涿郡後漢屬安平國晉魏屬博陵郡隋屬河間郡唐屬深州黑闥分置饒州將即亮翻】從者纔百餘人【從才用翻】餒甚德威出迎延黑闥入城黑闥不可德威涕泣固請黑闥乃從之至城旁市中憩止【憩去例翻】德威饋之食食未畢德威勒兵執之送詣太子并其弟十善斬於洺水【洺彌并翻 考異曰革命記劉黑闥走至深州崔元愻為偽深州摠管黑闥欲至城中陳列三千餘兵擬納黑闥據城拒守北勾突厥諸葛德威為車騎領當城之兵有張善護者先任鄉長來就軍中語三五少年曰可捉黑闥以取富貴今若不捉在後終是擾亂山東廢我等作生活諸少年咸云非諸葛車騎不可善護知德威非得酒食不肯出師乃于家宰一肥猪出酒一石延德威而語之德威許諾黑闥至元愻乃請之入城而不許唯就市中遣鋪設而坐食元愻請以城中兵呈閲言並精鋭必堪拒守黑闥食而許之元愻乃召兵以呈之德威以前領軍卒出即就市擒劉黑闥送於洺州皇太子所元愻與數十人奔突厥斬黑闥於洺州城西臨刑乃嘆云云今從實録亦兼采革命記】黑闥臨刑歎曰我幸在家鉏菜為高雅賢輩所誤至此 壬午巂州人王摩沙舉兵【巂音髓沙讀曰莎蘇何翻】自稱元帥改元進通遣驃騎將軍衛彦討之【帥所類翻驃匹妙翻騎奇寄翻下同】 庚子以吳王杜伏威為太保【唐制太師太傅太保謂之三師正一品天子所師法無所統職功德崇重者乃使居之】 二月庚戌上幸驪山温湯【驪山在雍州新豐有湯泉天寶起華清宫於此驪力知翻】甲寅還宫 平陽昭公主薨戊午葬公主詔加前後部鼓吹班劒四十人【班列也持劒成列夾道而行也吹昌瑞翻】武賁甲卒【武賁虎賁也唐諱虎字改為武賁音奔】太常奏禮婦人無鼓吹上曰鼓吹軍樂也公主親執金鼓興義兵以輔成大業【事見一百八十五卷隋義寧元年】豈與常婦人比乎 丙寅徐圓朗窮蹙與數騎弃城走為野人所殺其地悉平林邑王梵志遣使入貢初隋人破林邑【見一百八十卷隋大業九年梵扶汎翻使疏吏翻】分其地為三郡【三郡比景海隂林邑也】及中原喪亂【喪息浪翻】林邑復國至是始入貢 幽州摠管李藝請入朝【朝直遙翻】庚午以藝為左翊衛大將軍 廢參旗等十二軍【十二軍詳見一百八十八卷二年參疏簪翻】 三月癸未高開道掠文安魯城【文安縣前漢屬勃海郡後漢屬河間國晉屬章武郡隋唐屬瀛州魯城縣開皇十六年置亦屬瀛州】驃騎將軍平善政邀擊破之【驃匹妙翻騎奇寄翻】 庚子梁師都將賀遂索同以所部十二州來降【將即亮翻降戶江翻】 乙巳前洪州摠管張善安反遣舒州摠管張鎮周等擊之【舒州同安郡隋為熙州武德四年改舒州以古羣舒之國也】 夏四月吐谷渾寇芳州刺史房當樹奔松州【吐從暾入聲谷音浴西魏逐吐谷渾置同昌郡及封德等縣後周以縣立芳州隋大業初廢武德元年以同昌之常芬縣置芳州省封德松州交川枯治嘉誠縣生羌之地後魏時白水羌舒彭遣使朝貢始置甘松縣魏亂而絶後周復招慰之於此置龍涸防天和六年改置扶州隋改甘松縣為嘉誠縣屬同昌郡武德初置松州取甘松嶺為名】 張善安陷孫州【舊唐志武德五年分洪州置南昌州摠管府管南昌西吳靖米孫五州南昌州領建昌龍安永修三縣八年廢南昌州及孫州以南昌州新吳永修龍安入建昌縣以孫州之建昌入豫章縣而以豫章屬洪州新志武德五年以洪州南昌縣置孫州以建昌縣置南昌州】執總管王戎而去 乙丑鄜州道行軍摠管段德操擊梁師都至夏州俘其民畜而還【鄜音膚夏戶雅翻還從宣翻又如字】 丙寅吐谷渾寇洮岷二州【洮土刀翻】 丁卯南州刺史龎孝恭南越州民甯道明高州首領馮暄俱反陷南越州進攻姜州合州刺史甯純引兵救之【武德四年以合浦郡之南昌合浦地置南州六年改白州合浦郡舊置越州隋改合州武德四年復曰越州加南字以别會稽之越州也舊志桂州摠管府所管有姜州武德五年以合州之封山縣置姜州貞觀十年廢入越州雷州海康郡本合州徐聞郡武德四年置貞觀更州郡名龎孝恭新書作龎孝泰】 壬申立皇子元軌為蜀王鳳為豳王元慶為漢王 【考異曰實録以皇子元真為邵王鶴為豳王新本記封元璹為蜀王按高祖子無名元真鶴元璹及封邵王者今從舊傳及唐歷】 癸酉以裴寂為左僕射蕭瑀為右僕射【瑀音禹】楊恭仁為吏部尚書兼中書令封德彞為中書令 五月庚辰遣岐州刺史柴紹救岷州【岐州扶風郡】 庚寅吐谷渾及党項寇河州【吐從暾入聲谷音浴河州枹罕郡党底朗翻】刺史盧士良擊破之 丙申梁師都將辛獠兒引突厥寇林州【將即亮翻下同獠盧皓翻厥九勿翻舊志慶州華池縣武德四年置林州摠管府】 戊戌苑君彰將高滿政寇代州驃騎將軍林寶言擊走之【驃匹妙翻騎奇寄翻】 癸卯高開道引奚騎寇幽州長史王詵擊破之【長知兩翻詵疏臻翻】劉黑闥之叛也突地稽引兵助唐徙其部落於幽州之昌平城【昌平城在軍都關南】高開道引突厥寇幽州突地稽將兵邀擊破之 六月戊午高滿政以馬邑來降【降戶江翻】先是前并州總管劉世讓除廣州總管將之官上問以備邊之策【先悉薦翻】世讓對曰突厥比數為寇【比毗至翻數所角翻下同】良以馬邑為之中頓故也【中頓者謂中道有城有糧可以頓食也置食之所曰頓唐人多言置頓】請以勇將戍崞城【崞音郭】多貯金帛【貯直呂翻】募有降者厚賞之數出騎兵掠其城下蹂其禾稼敗其生業不出歲餘彼無所食必降矣【數所角翻蹂人九翻敗補邁翻】上然其計曰非公誰為勇將即命世讓戍崞城馬邑病之是時馬邑人多不願屬突厥上復遣人招諭苑君璋高滿政說苑君璋盡殺突厥戍兵降唐【復扶又翻說輸芮翻厥九勿翻】君璋不從滿政因衆心所欲夜襲君璋君璋覺之亡奔突厥滿政殺君璋之子及突厥戍兵二百人而降 壬戌梁師都以突厥寇匡州【武德分綏州延福縣地置北吉州羅州匡州】 丁卯苑君璋與突厥吐屯設寇馬邑高滿政與戰破之以滿政為朔州總管封榮國公【朔州馬邑郡】 瓜州總管賀若懷廣按部至沙州【隋以敦煌郡置瓜州武德五年改沙州分沙州之常樂縣為瓜州晉昌郡宋白曰瓜州西至沙州二百八十里若人者翻】值州人張護李通反懷廣以數百人保子城涼州總管楊恭仁遣兵救之為護等所敗【敗補邁翻】 癸酉柴紹與吐谷渾戰【紹救岷州遂與吐谷渾戰吐從暾入聲谷音浴】為其所圍虜乘高射之【射而亦翻】矢下如雨紹遣人彈胡琵琶二女子對舞虜怪之駐弓矢相與聚觀紹察其無備潛遣精騎出虜陳後擊之虜衆大潰【騎奇寄翻陳讀曰陣】 秋七月丙子苑君璋以突厥寇馬邑右武候大將軍李高遷及高滿政禦之戰於臘河谷破之 張護李通殺賀拔懷廣 【考異曰實録上云張護此云高護今從上余按賀拔意亦當從上作賀若】立汝州别駕竇伏明為主【汝當作】進逼瓜州長史趙孝倫擊却之【長知兩翻】高開道掠赤岸鎮及靈壽九門行唐三縣而去【九域志定州唐縣有赤岸鎮三縣皆屬恒州時以靈夀屬并州】 丁丑崗州刺史馮士翽據新會反【隋以新會郡置岡州以地有金岡故名大業初廢武德四年復以廣州新會義寜二縣置岡州翽呼會翻】廣州刺史劉感討降之使復其位【降戶江翻】 辛巳高開道所部弘陽統漢二鎮來降 癸未突厥寇原州【原州漢高平縣地後魏立原州取高平曰原以名州厥九勿翻】乙酉寇朔州李高遷為虜所敗行軍總管尉遲敬德將兵救之己亥遣太子將兵屯北邊【備原州之寇敗補邁翻尉紆勿翻將即亮翻】秦王世民屯并州【備朔州之寇】以備突厥八月丙辰突厥寇真州【舊志武德二年置綏州總管府管雲銀真等十一州真州蓋置於銀州真鄉縣也】又寇馬邑 壬子淮南道行臺僕射輔公祏反 【考異曰舊傳云沈法興據毗陵公祏擊破之按法興武德三年已為李子通所滅舊傳誤也】初杜伏威與公祏相友善公祏年長【長知兩翻】伏威兄事之軍中謂之伯父畏敬與伏威等伏威浸忌之乃署其養子闞稜為左將軍王雄誕為右將軍潛奪其兵權公祏知之怏怏不平【怏於兩翻】與其故人左遊仙陽為學道辟穀以自晦及伏威入朝【入朝見上年朝直遙翻】留公祏守丹陽【此南朝之舊丹陽郡】令雄誕典兵為之副隂謂雄誕曰吾至長安苟不失職勿令公祏為變伏威既行左遊仙說公祏謀反【說輸芮翻】而雄誕握兵公祏不得發乃詐稱得伏威書疑雄誕有貳心雄誕聞之不悦稱疾不視事公祏因奪其兵使其黨西門君儀諭以反計雄誕始寤而悔之曰今天下方平吳王又在京師【杜伏威封吳王】大唐兵威所向無敵奈何無故自求族滅乎雄誕有死而已不敢聞命今從公為逆不過延百日之命耳大丈夫安能愛斯須之死而自陷於不義乎公祏知不可屈縊殺之【縊於賜翻又於計翻】雄誕善撫士卒得其死力又約束嚴整每破城邑秋毫無犯死之日江南軍中及民間皆為之流涕【為于偽翻】公祏又詐稱伏威不得還江南貽書令其起兵大修鎧仗【鎧可亥翻】運糧儲尋稱帝於丹陽國號宋修陳故宫室而居之署置百官以左遊仙為兵部尚書東南道大使越州總管【使疏吏翻下同】與張善安連兵以善安為西南道大行臺 己未突厥寇原州【厥九勿翻】 乙丑詔襄州道行臺僕射趙郡王孝恭以舟師趣江州【江州南朝之尋陽郡隋改為九江郡趣七喻翻又逡須翻 考異曰實録八月乙丑已云遣孝恭率兵趣江州至九月戊子又云蓋因徐紹宗等侵邊而言之也】嶺南道大使李靖以交廣泉桂之衆趣宣州【宣州宣城郡】懷州總管黄君漢出譙亳齊州總管李世勣出淮泗【自泗水入淮也】以討輔公祏孝恭將發與諸將宴集命取水忽變為血在坐者皆失色【坐徂卧翻】孝恭舉止自若曰此乃公祏授首之徵也飲而盡之衆皆悦服 丙寅吐谷渾内附【吐從暾入聲谷音浴】 辛未突厥陷原州之善和鎮癸酉又寇渭州【渭州隴西郡】 高開道以奚侵幽州州兵擊却之九月太子班師【自豳州道班師】 戊子輔公祏遣其將徐紹<br />
<br />
  宗寇海州陳政通寇夀陽【祏音石將即亮翻夀陽夀州治所】 邛州獠反【邛州臨邛郡武德元年析雅州置邛渠容翻獠盧皓翻】遣沛公鄭元璹討之【璹殊玉翻】 庚寅突厥寇幽州【厥九勿翻】 壬辰詔以秦王世民為江州道行軍元帥【欲使之討輔公祏也帥所類翻】 乙未竇伏明以沙州降【降戶江翻 考異曰實録云伏明斬賀拔威以城來降按五年五月實録瓜州人王幹殺賀拔威以降則威死久矣此誤也按上文作賀若懷廣死而立竇伏明為沙州主當考】 高昌王麴伯雅卒【卒子恤翻】子文泰立 丙申渝州人張大智反【渝州巴郡漢江州縣地】刺史薛敬仁弃城走 壬寅高開道引突厥二萬騎寇幽州【騎奇寄翻】 突厥惡弘農公劉世讓為己患【惡烏路翻】遣其臣曹般陁來言世讓與可汗通謀欲為亂【般補末翻可從刋入聲汗音寒】上信之冬十月丙午殺世讓籍其家【聽言之道必以其事觀之突厥用間高祖遽信之而殺干城之將不明甚矣】 秦王世民猶在并州己未詔世民引兵還【太子與秦王分道備突厥太子先已班師故亦詔秦王引還】 上幸華隂【華戶化翻】 張大智侵涪州【涪州涪陵郡武德元年以渝州之涪陵鎮置涪音浮】刺史田世康等討之大智以衆降 初上遣右武候大將軍李高遷助朔州摠管高滿政守馬邑【宋白曰朔州春秋北狄之地曹魏立為馬邑縣西晉米其地為猗盧所據都代郡此後魏為畿内之地亦曾為懷朔鎮孝文遷洛之後於定襄故城置朔州高齊又於新平郡置朔州天保八年乃移於馬邑城】苑君璋引突厥萬餘騎至城下滿政擊破之頡利可汗怒大發兵攻馬邑高遷懼帥所部二千人斬關宵遁虜邀之失亡者半頡利自帥衆攻城【帥讀曰率】滿政出兵禦之或一日戰十餘合上命行軍摠管劉世讓救之至松子嶺不敢進【松子嶺地闕】還保崞城會頡利遣使求婚【使疏吏翻】上曰釋馬邑之圍乃可議婚頡利欲解兵義成公主固請攻之頡利以高開道善為攻具召開道與之攻馬邑甚急頡利誘滿政使降滿政罵之糧且盡救兵未至滿政欲潰圍走朔州右虞候杜士遠【隋文帝於東宫置左右虞候府掌斥候是後州鎮各置虞候以為衙前之職以備候不虞名官】以虜兵盛恐不免壬戌殺滿政降於突厥【按通鑑據新唐書高祖本紀自丙午至壬戌排日書之但十七日間先書殺劉世讓後復書命世讓救馬邑及退保事蓋通鑑序突厥陷馬邑事書之不詳排日之近遠也】苑君璋復殺城中豪傑與滿政同謀者三十餘人上以滿政子玄積為上柱國襲爵丁卯突厥復請和親【復扶又翻下同】以馬邑歸唐上以將軍秦武通為朔州摠管 突厥數為邊患【數所角翻】并州大摠管府長史竇靜表請於太原置屯田以省餽運議者以為煩擾不許靜切論不已敕徵靜入朝【長知兩翻朝直遙翻】使與裴寂蕭瑀封德彞相論難於上前【瑀音禹難乃旦翻】寂等不能屈乃從靜議歲收穀數千斛上善之命檢校并州大總管靜抗之子也【竇抗榮定之子】十一月辛巳秦王世民復請增置屯田於并州之境從之 黄州摠管周法明將兵擊輔公祏【黄州漢邾縣地蕭齊置齊安郡隋置黄州祏音石】張善安據夏口拒之【將即亮翻夏戶雅翻】法明屯荆口鎮【蓋當荆江之口置鎮其地在漢陽界】壬午法明登戰艦飲酒【艦戶黯翻】善安遣刺客數人詐乘魚艓而至【艓達協翻丁度曰舟名】見者不以為虞遂殺法明而去 甲申舒州摠管張鎮周等擊輔公祏將陳當世於猷州之黄沙大破之【武德三年以宣州之涇縣置南徐州尋改曰猷州】 丁亥上校獵於華隂【華戶化翻】己丑迎勞秦王世民於忠武頓【秦王自并州還勞力到翻】 十二月癸卯安撫使李大亮誘張善安執之【使疏吏翻誘音酉】大亮擊善安於洪州與善安隔水而陳【陳讀曰陣】遙相與語大亮諭以禍福善安曰善安初無反心正為將士所誤欲降又恐不免【將即亮翻】大亮曰張摠管有降心則與我一家耳因單騎度水入其陳與善安執手共語示無猜間【降下江翻間古莧翻】善安大悦遂許之降既而善安將數十騎詣大亮營大亮止其騎於門外引善安入與語久之善安辭去大亮命武士執之從騎皆走【從才用翻】善安營中聞之大怒悉衆而來將攻大亮大亮使人諭之曰吾不留摠管摠管赤心歸國謂我曰若還營恐將士或有異同為其所制故自留不去耳卿輩何怒於我其黨復大罵曰張摠管賣我以自媚於人遂皆潰去大亮追擊多所虜獲送善安於長安善安自稱不與輔公祏交通【祏音石】上赦其罪善遇之及公祏敗得所與往還書乃殺之 甲寅車駕至長安 己巳突厥寇定州州兵擊走之 庚申白簡白狗羌並遣使入貢【白簡恐當作白蘭隋書附國有白蘭白狗等種風俗畧同党項或役屬於吐谷渾或附附國新書白蘭羌吐蕃謂之丁零左屬党項右與多彌接勝兵萬人勇戰鬭善作兵器武德六年使者入朝明年以其地為維恭二州白狗羌與東會州接勝兵纔千人使疏吏翻】<br />
<br />
  七年春正月依周齊舊制每州置大中正一人【州置大中正周齊又因魏晉之制】掌知州内人物品量望第以本州門望高者領之無品秩 壬午趙郡王孝恭擊輔公祏别將於樅陽破之【樅陽縣漢屬廬江郡梁置樅陽郡隋廢郡改為同安縣屬廬州祏音石將即亮翻樅七容翻】庚寅鄒州人鄧同頴殺刺史李士衡反【唐初以齊州之鄒平長山】<br />
<br />
  【置鄒州】 丙申以白狗等羌地置維恭二州【維州維川郡以白狗羌降戶姜維故城置其地乃漢蜀郡徼外冉駹之地蜀將姜維馬忠討汶山羌於此故壘在焉恭州即西恭州後改曰笮州又戎州都督府所領羈縻州有曲州本隋之恭州隋亂廢武德元年開南中復置八年改曲州故朱提郡地非此也劉昫曰維州即古西戎地也其地南界江陽岷山連嶺而西不知其極北望隴山積雪如玉東望成都若在井底地接石紐山夏禹生於石紐是也其城在岷山之孤峯三面臨江距成都四百里許杜佑曰維州在當州北二百六十里】二月輔公祏遣兵圍猷州刺史左難當嬰城自守【宋白曰宣州涇縣唐武德二年置南徐州於此其年改為猷州】安撫使李大亮引兵擊公祏破之【使疏吏翻下同】趙郡王孝恭攻公祏鵲頭鎮拔之【新志宣州南陵縣有鵲頭鎮】 丁未高麗王建武遣使來請班歷遣使冊建武為遼東郡王高麗王以百濟王扶餘璋為帶方郡王新羅王金真平為樂浪郡王【麗力知翻樂浪音洛郎】始州獠反【始州普安郡後改劔州獠魯皓翻】遣行臺僕射竇軌討之己酉詔諸州有明一經以上未仕者咸以名聞州縣<br />
<br />
  及鄉皆置學 壬子行軍副總管權文誕破輔公祏之黨於猷州拔其枚洄等四鎮 丁巳上幸國子監釋奠詔諸王公子弟各就學 戊午改大總管為大都督府己未高開道將張金樹殺開道來降【將即亮翻降戶江翻下同】開<br />
<br />
  道見天下皆定欲降自以數反覆不敢【高開道既降而復叛自知有反覆之罪數所角翻】且恃突厥之衆遂無降意其將卒皆山東人思鄉里咸有離心開道選勇敢士數百謂之假子常直閤内使金樹領之故劉黑闥將張君立亡在開道所與金樹密謀取開道金樹遣其黨數人入閤内與假子遊戲向夕潛斷其弓弦【斷丁管翻】藏刀槊於牀下合瞑抱之趨出【人睡則目合而瞑槊色角翻瞑莫定翻】金樹帥其黨大譟攻開道閤【帥讀曰率譟蘇到翻】假子將禦之弓弦皆絶刀槊已失爭出降君立亦舉火於外與相應内外惶擾開道知不免乃擐甲持兵坐堂上與妻妾奏樂酣飲【擐音宦酣戶甘翻】衆憚其勇不敢逼天且明開道縊【縊於賜翻又於計翻】妻妾及諸子乃自殺金樹陳兵悉收假子斬之并殺君立死者五百餘人遣使來降詔以其地置媯州【以媯水名州媯俱為翻】壬戌以金樹為北燕州都督【媯州媯川郡治懷戎縣北齊置北燕州本治懷戎唐既以懷戎之地置媯州又以北燕州都督之名寵金樹也燕因肩翻】 戊辰洋集二州獠反陷隆州晉城【洋州洋川郡治西鄉縣西鄉漢成固縣蜀立西鄉後魏於此置洋州以水為名洋音祥集州符陽郡武德元年析梁州之難江巴州之符陽長池白石置隆州巴西郡漢閬中地劉昫曰西魏置隆州於閬中隋為巴西郡唐復為隆州宋白曰取其連岡地勢高隆為名後魏典畧云此州古有隆城堅險因置隆州晉城縣亦閬中地梁置木蘭郡西魏廢郡改西充國曰晉城獠魯皓翻】 是月太保吳王杜伏威薨輔公祏之反也詐稱伏威之命以紿其衆【紿蕩亥翻】及公祏平趙郡王孝恭不知其詐以狀聞詔追除伏威名籍没其妻子及太宗即位知其寃赦之復其官爵 三月初定令以太尉司徒司空為三公次尚書門下中書祕書殿中内侍為六省次御史臺次太常至太府為九寺【太常光禄衛尉宗正太僕大理鴻臚司農太府凡九寺】次將作監次國子學次天策上將府【將即亮翻】次左右衛至左右領衛為十四衛【十二衛及左右監門衛為十四衛】東宫置三師三少詹事及兩坊三寺十率府【兩坊門下坊典書坊三寺家令寺率更寺僕寺十率府左右衛率左右宗衛率左右虞候率左右監門率左右内率少詩沼翻率讀如字】王公置府佐國官公主置邑司並為京職事官【王府有傳諮議參軍友文學東西閤祭酒長史司馬掾屬主簿史記室録事參軍録事功倉戶兵騎法士等七曹參軍參軍事行參軍典籖王國有國令大農尉丞録事典衛舍人學官長食官長廏牧長典府長公主邑司有令丞主簿謁者舍人家吏掌主家財出入田園徵封之事】州縣鎮戍為外職事官自開府儀同三司至將仕郎二十八階為文散官【開府儀同三司從一品特進正二品光禄大夫從二品金紫光禄大夫正三品銀青光禄大夫從三品正議大夫正四品太中大夫從四品上中大夫從四品下中散大夫正五品上朝議大夫正五品下朝請大夫從五品上朝散大夫從五品下朝議郎正六品上承議郎正六品下奉議郎從六品上通直郎從六品下朝議郎正七品上宣德郎正七品下朝散郎從七品上宣義郎從七品下給事郎正八品上徵事郎正八品下承奉郎從八品上承務郎從八品下儒林郎正九品上登仕郎正九品下文林郎從九品上將仕郎從九品下散悉亶翻下同】驃騎大將軍至陪戎副尉三十一階為武散官【驃騎大將軍從一品輔國大將軍正二品鎮軍大將軍從二品冠軍大將軍懷化大將軍正三品上懷化將軍正三品下雲麾將軍歸德大將軍從三品上歸德將軍從三品下忠武將軍正四品上壯武將軍懷化中郎將正四品下宣威將軍從四品上明威將軍歸德中郎將從四品下定遠將軍正五品上寧遠將軍懷化郎將正五品下游騎將軍從五品上游擊將軍歸德郎將從五品下昭武校尉正六品上昭武副尉懷化司階正六品下振威校尉從六品上振威副尉歸德司階從六品下致果校尉正七品上致果副尉懷化中候正七品下翊麾校尉從七品上翊麾副尉德歸中候從七品下宣節校尉正八品上宣節副尉懷化司戈正八品下禦侮校尉從八品上禦侮副尉歸德司戈從八品下仁勇校尉正九品上仁勇副尉懷化執戟長上九品下陪戎校尉從九品上陪戎副尉歸德執戟長上從九品下驃匹妙翻騎奇寄翻下同】上柱國至武騎尉十二等為勲官【勲級十有二轉為上柱國視正二品十有一轉為柱國視從二品十轉為上護軍視正三品九轉為護軍視從三品八轉為上輕車都尉視正四品七轉為輕車都尉視從四品六轉為上騎都尉視正五品五轉為騎都尉視從五品四轉為驍騎校視正六品三轉為飛騎尉視從六品二轉為雲騎尉視正七品一轉為武騎尉視從七品】 丙戌趙郡王孝恭破輔公祏於蕪湖拔梁山等三鎮【蕪湖時在宣州當塗縣界梁山在和州歷陽縣南七十里臨江祏音石】辛卯安撫使任瓌拔揚子城廣陵城主龍龕降【揚子城在揚州江都縣界揚州治江都古廣陵城也姓苑龍姓古龍伯氏之後任音壬瓌古回翻龕口含翻】 丁酉突厥寇原州 戊戌趙郡王孝恭克丹陽先是輔公祏遣其將馮慧亮陳當世將舟師三萬屯博望山【天門山在宣州當塗縣西南三十里又名蛾眉山夾江對峙東曰博望西曰梁山先悉薦翻將即亮翻下同 考異曰舊趙郡王孝恭傳作陳當時舊李靖傳云屯當塗今皆從高祖實録】陳正通徐紹宗將步騎三萬屯青林山【水經注即水出廬江郡之東陵鄉禹貢所謂過九江至於東陵者也西南流水積為湖湖西有青林山又今當塗縣東南有青山】仍於梁山連鐵鏁以斷江路【斷丁管翻】築却月城延袤十餘里【袤音茂】又結壘江西以拒官軍孝恭與李靖帥舟師次舒州李世勣帥步卒一萬度淮拔夀陽次硤石【帥讀曰率】慧亮等堅壁不戰孝恭遣奇兵絶其糧道慧亮等軍乏食夜遣兵薄孝恭營孝恭堅卧不動孝恭集諸將議軍事皆曰慧亮等擁彊兵據水陸之險攻之不可猝拔不如直指丹陽掩其巢穴丹陽既潰慧亮等自降矣【降戶江翻】孝恭將從其議李靖曰公祏精兵雖在此水陸二軍然所自將亦不為少今博望諸柵尚不能拔公祏保據石頭豈易取哉【易以豉翻】進攻丹陽旬月不下慧亮躡吾後腹背受敵此危道也【李靖此議與長孫無忌安市之議畧同然李靖决勝而太宗無功及安市班師靖咎其不能用道宗之策此用兵之所以難也】慧亮正通皆百戰餘賊其心非不欲戰正以公祏立計使之持重欲以老我師耳我今攻其城以挑之【挑徒了翻】一舉可破也孝恭然之使羸兵先攻賊營而勒精兵結陳以待之【羸倫為翻陳讀曰陣】攻壘者不勝而走賊出兵追之行數里遇大軍與戰大破之【此左傳楚五大夫破吳師以滅舒鳩之故智也】闞稜免胄謂賊衆曰汝曹不識我邪【邪音耶】何敢來與我戰賊多稜故部曲皆無鬭志或有拜者由是遂敗孝恭靖乘勝逐北轉戰百餘里博山青林兩戍皆潰慧亮正通等遁歸殺傷及溺死者萬餘人【溺奴狄翻】李靖兵先至丹楊公祏大懼擁兵數萬弃城東走欲就左遊仙於會稽李世勣追之公祏至句容【句容縣漢屬丹陽郡時屬蔣州在建康城東九十里】從兵能屬者纔五百人【從才用翻屬之欲翻】夜宿常州其將吳騷等謀執之公祏覺之弃妻子獨將腹心數十人斬關走至武康【吳分烏程餘抗立永安縣晉改為永康又改為武康屬湖州在州西南一百七里】為野人所攻西門君儀戰死執公祏送丹陽梟首【梟堅堯翻】分捕餘黨悉誅之江南皆平己亥以孝恭為東南道行臺右僕射李靖為兵部尚書頃之廢行臺以孝恭為揚州大都督靖為府長史上深美靖功曰靖蕭輔之膏肓也【謂蕭銑輔公祏皆為靖所殺也肓呼光翻】闞稜功多頗自矜伐公祏誣稜與己通謀會趙郡王孝恭籍没賊黨田宅【籍没者舉籍賊黨所有田宅没而入官】稜及杜伏威王雄誕田宅在賊境者孝恭并籍没之稜自訴理忤孝恭【忤五故翻】孝恭怒以謀反誅之 夏四月庚子朔赦天下是日頒新律令比開皇舊制增新格五十三條 初定均田租庸調法【調徒釣翻】丁中之民給田一頃篤疾減什之六寡妻妾減七皆以什之二為世業八為口分【分扶問翻】每丁歲入租粟二石調隨土地所宜綾絹絁布歲役二旬不役則收其傭日三尺【新志凡授田丁歲輸粟二斛謂之租丁隨鄉所出歲輸絹二匹綾絁二丈布加五之一綿三兩麻三斤非蠶郷則輸銀十四兩謂之調用人之力歲二十日閏加三日不役收其庸日三尺絁式支翻繒似布】有事而加役者旬有五日免其調三旬租調俱免水旱蟲霜為災什損四以上免租損六以上免調損七已上課役俱免凡民貲業分九等【上中下各有三等也】百戶為里五里為鄉四家為鄰四鄰為保 【考異曰唐歷云四家為鄰五家為保按通典四鄰為保唐歷誤也】在城邑者為坊田野者為村食禄之家無得與民爭利工商雜類無預士伍男女始生為黄四歲為小十六為中二十為丁六十為老歲造計帳三年造戶籍 丁未党項寇松州 庚申通事舍人李鳳起擊萬州反獠平之【後魏分朐䏰置魚泉縣後周改為萬川隋改為南浦屬信州武德元年分置萬州南浦郡獠魯皓翻下同】 五月辛未突厥寇朔州 甲戌羌與吐谷渾同寇松州【吐從暾入聲谷音浴】遣益州行臺左僕射竇軌自翼州道扶州刺史蔣善合自芳州道擊之【西魏逐吐谷渾置鄧州隋開皇七年改曰扶州同昌郡武德元年分會州之左封翼斜置翼州臨翼郡唐制上州刺史從三品中正四品上下正四品下】 丙戌作仁智宫於宜君【宜君縣置於古祋祤城隋屬京兆郡時屬宜州】丁亥竇軌破反獠於方山俘二萬餘口<br />
<br />
  資治通鑑卷一百九十  <br>
   </div> 

<script src="/search/ajaxskft.js"> </script>
 <div class="clear"></div>
<br>
<br>
 <!-- a.d-->

 <!--
<div class="info_share">
</div> 
-->
 <!--info_share--></div>   <!-- end info_content-->
  </div> <!-- end l-->

<div class="r">   <!--r-->



<div class="sidebar"  style="margin-bottom:2px;">

 
<div class="sidebar_title">工具类大全</div>
<div class="sidebar_info">
<strong><a href="http://www.guoxuedashi.com/lsditu/" target="_blank">历史地图</a></strong>  
<a href="http://www.880114.com/" target="_blank">英语宝典</a>  
<a href="http://www.guoxuedashi.com/13jing/" target="_blank">十三经检索</a> 
<br><strong><a href="http://www.guoxuedashi.com/gjtsjc/" target="_blank">古今图书集成</a></strong> 
<a href="http://www.guoxuedashi.com/duilian/" target="_blank">对联大全</a> <strong><a href="http://www.guoxuedashi.com/xiangxingzi/" target="_blank">象形文字典</a></strong> 

<br><a href="http://www.guoxuedashi.com/zixing/yanbian/">字形演变</a>  <strong><a href="http://www.guoxuemi.com/hafo/" target="_blank">哈佛燕京中文善本特藏</a></strong>
<br><strong><a href="http://www.guoxuedashi.com/csfz/" target="_blank">丛书&方志检索器</a></strong> <a href="http://www.guoxuedashi.com/yqjyy/" target="_blank">一切经音义</a>  

<br><strong><a href="http://www.guoxuedashi.com/jiapu/" target="_blank">家谱族谱查询</a></strong>  <strong><a href="http://shufa.guoxuedashi.com/sfzitie/" target="_blank">书法字帖欣赏</a></strong> 
<br>

</div>
</div>


<div class="sidebar" style="margin-bottom:0px;">

<font style="font-size:22px;line-height:32px">QQ交流群9:489193090</font>


<div class="sidebar_title">手机APP 扫描或点击</div>
<div class="sidebar_info">
<table>
<tr>
	<td width=160><a href="http://m.guoxuedashi.com/app/" target="_blank"><img src="/img/gxds-sj.png" width="140"  border="0" alt="国学大师手机版"></a></td>
	<td>
<a href="http://www.guoxuedashi.com/download/" target="_blank">app软件下载专区</a><br>
<a href="http://www.guoxuedashi.com/download/gxds.php" target="_blank">《国学大师》下载</a><br>
<a href="http://www.guoxuedashi.com/download/kxzd.php" target="_blank">《汉字宝典》下载</a><br>
<a href="http://www.guoxuedashi.com/download/scqbd.php" target="_blank">《诗词曲宝典》下载</a><br>
<a href="http://www.guoxuedashi.com/SiKuQuanShu/skqs.php" target="_blank">《四库全书》下载</a><br>
</td>
</tr>
</table>

</div>
</div>


<div class="sidebar2">
<center>


</center>
</div>

<div class="sidebar"  style="margin-bottom:2px;">
<div class="sidebar_title">网站使用教程</div>
<div class="sidebar_info">
<a href="http://www.guoxuedashi.com/help/gjsearch.php" target="_blank">如何在国学大师网下载古籍?</a><br>
<a href="http://www.guoxuedashi.com/zidian/bujian/bjjc.php" target="_blank">如何使用部件查字法快速查字?</a><br>
<a href="http://www.guoxuedashi.com/search/sjc.php" target="_blank">如何在指定的书籍中全文检索?</a><br>
<a href="http://www.guoxuedashi.com/search/skjc.php" target="_blank">如何找到一句话在《四库全书》哪一页?</a><br>
</div>
</div>


<div class="sidebar">
<div class="sidebar_title">热门书籍</div>
<div class="sidebar_info">
<a href="/so.php?sokey=%E8%B5%84%E6%B2%BB%E9%80%9A%E9%89%B4&kt=1">资治通鉴</a> <a href="/24shi/"><strong>二十四史</strong></a>&nbsp; <a href="/a2694/">野史</a>&nbsp; <a href="/SiKuQuanShu/"><strong>四库全书</strong></a>&nbsp;<a href="http://www.guoxuedashi.com/SiKuQuanShu/fanti/">繁体</a>
<br><a href="/so.php?sokey=%E7%BA%A2%E6%A5%BC%E6%A2%A6&kt=1">红楼梦</a> <a href="/a/1858x/">三国演义</a> <a href="/a/1038k/">水浒传</a> <a href="/a/1046t/">西游记</a> <a href="/a/1914o/">封神演义</a>
<br>
<a href="http://www.guoxuedashi.com/so.php?sokeygx=%E4%B8%87%E6%9C%89%E6%96%87%E5%BA%93&submit=&kt=1">万有文库</a> <a href="/a/780t/">古文观止</a> <a href="/a/1024l/">文心雕龙</a> <a href="/a/1704n/">全唐诗</a> <a href="/a/1705h/">全宋词</a>
<br><a href="http://www.guoxuedashi.com/so.php?sokeygx=%E7%99%BE%E8%A1%B2%E6%9C%AC%E4%BA%8C%E5%8D%81%E5%9B%9B%E5%8F%B2&submit=&kt=1"><strong>百衲本二十四史</strong></a>  <a href="http://www.guoxuedashi.com/so.php?sokeygx=%E5%8F%A4%E4%BB%8A%E5%9B%BE%E4%B9%A6%E9%9B%86%E6%88%90&submit=&kt=1"><strong>古今图书集成</strong></a>
<br>

<a href="http://www.guoxuedashi.com/so.php?sokeygx=%E4%B8%9B%E4%B9%A6%E9%9B%86%E6%88%90&submit=&kt=1">丛书集成</a> 
<a href="http://www.guoxuedashi.com/so.php?sokeygx=%E5%9B%9B%E9%83%A8%E4%B8%9B%E5%88%8A&submit=&kt=1"><strong>四部丛刊</strong></a>  
<a href="http://www.guoxuedashi.com/so.php?sokeygx=%E8%AF%B4%E6%96%87%E8%A7%A3%E5%AD%97&submit=&kt=1">說文解字</a> <a href="http://www.guoxuedashi.com/so.php?sokeygx=%E5%85%A8%E4%B8%8A%E5%8F%A4&submit=&kt=1">三国六朝文</a>
<br><a href="http://www.guoxuedashi.com/so.php?sokeytm=%E6%97%A5%E6%9C%AC%E5%86%85%E9%98%81%E6%96%87%E5%BA%93&submit=&kt=1"><strong>日本内阁文库</strong></a> <a href="http://www.guoxuedashi.com/so.php?sokeytm=%E5%9B%BD%E5%9B%BE%E6%96%B9%E5%BF%97%E5%90%88%E9%9B%86&ka=100&submit=">国图方志合集</a> <a href="http://www.guoxuedashi.com/so.php?sokeytm=%E5%90%84%E5%9C%B0%E6%96%B9%E5%BF%97&submit=&kt=1"><strong>各地方志</strong></a>

</div>
</div>


<div class="sidebar2">
<center>

</center>
</div>
<div class="sidebar greenbar">
<div class="sidebar_title green">四库全书</div>
<div class="sidebar_info">

《四库全书》是中国古代最大的丛书,编撰于乾隆年间,由纪昀等360多位高官、学者编撰,3800多人抄写,费时十三年编成。丛书分经、史、子、集四部,故名四库。共有3500多种书,7.9万卷,3.6万册,约8亿字,基本上囊括了古代所有图书,故称“全书”。<a href="http://www.guoxuedashi.com/SiKuQuanShu/">详细>>
</a>

</div> 
</div>

</div>  <!--end r-->

</div>
<!-- 内容区END --> 

<!-- 页脚开始 -->
<div class="shh">

</div>

<div class="w1180" style="margin-top:8px;">
<center><script src="http://www.guoxuedashi.com/img/plus.php?id=3"></script></center>
</div>
<div class="w1180 foot">
<a href="/b/thanks.php">特别致谢</a> | <a href="javascript:window.external.AddFavorite(document.location.href,document.title);">收藏本站</a> | <a href="#">欢迎投稿</a> | <a href="http://www.guoxuedashi.com/forum/">意见建议</a> | <a href="http://www.guoxuemi.com/">国学迷</a> | <a href="http://www.shuowen.net/">说文网</a><script language="javascript" type="text/javascript" src="https://js.users.51.la/17753172.js"></script><br />
  Copyright &copy; 国学大师 古典图书集成 All Rights Reserved.<br>
  
  <span style="font-size:14px">免责声明:本站非营利性站点,以方便网友为主,仅供学习研究。<br>内容由热心网友提供和网上收集,不保留版权。若侵犯了您的权益,来信即刪。scp168@qq.com</span>
  <br />
ICP证:<a href="http://www.beian.miit.gov.cn/" target="_blank">鲁ICP备19060063号</a></div>
<!-- 页脚END --> 
<script src="http://www.guoxuedashi.com/img/plus.php?id=22"></script>
<script src="http://www.guoxuedashi.com/img/tongji.js"></script>

</body>
</html>
