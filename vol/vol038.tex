<!DOCTYPE html PUBLIC "-//W3C//DTD XHTML 1.0 Transitional//EN" "http://www.w3.org/TR/xhtml1/DTD/xhtml1-transitional.dtd">
<html xmlns="http://www.w3.org/1999/xhtml">
<head>
<meta http-equiv="Content-Type" content="text/html; charset=utf-8" />
<meta http-equiv="X-UA-Compatible" content="IE=Edge,chrome=1">
<title>資治通鑒_39-資治通鑑卷三十八_39-資治通鑑卷三十八</title>
<meta name="Keywords" content="資治通鑒_39-資治通鑑卷三十八_39-資治通鑑卷三十八">
<meta name="Description" content="資治通鑒_39-資治通鑑卷三十八_39-資治通鑑卷三十八">
<meta http-equiv="Cache-Control" content="no-transform" />
<meta http-equiv="Cache-Control" content="no-siteapp" />
<link href="/img/style.css" rel="stylesheet" type="text/css" />
<script src="/img/m.js?2020"></script> 
</head>
<body>
 <div class="ClassNavi">
<a  href="/24shi/">二十四史</a> | <a href="/SiKuQuanShu/">四库全书</a> | <a href="http://www.guoxuedashi.com/gjtsjc/"><font  color="#FF0000">古今图书集成</font></a> | <a href="/renwu/">历史人物</a> | <a href="/ShuoWenJieZi/"><font  color="#FF0000">说文解字</a></font> | <a href="/chengyu/">成语词典</a> | <a  target="_blank"  href="http://www.guoxuedashi.com/jgwhj/"><font  color="#FF0000">甲骨文合集</font></a> | <a href="/yzjwjc/"><font  color="#FF0000">殷周金文集成</font></a> | <a href="/xiangxingzi/"><font color="#0000FF">象形字典</font></a> | <a href="/13jing/"><font  color="#FF0000">十三经索引</font></a> | <a href="/zixing/"><font  color="#FF0000">字体转换器</font></a> | <a href="/zidian/xz/"><font color="#0000FF">篆书识别</font></a> | <a href="/jinfanyi/">近义反义词</a> | <a href="/duilian/">对联大全</a> | <a href="/jiapu/"><font  color="#0000FF">家谱族谱查询</font></a> | <a href="http://www.guoxuemi.com/hafo/" target="_blank" ><font color="#FF0000">哈佛古籍</font></a> 
</div>

 <!-- 头部导航开始 -->
<div class="w1180 head clearfix">
  <div class="head_logo l"><a title="国学大师官网" href="http://www.guoxuedashi.com" target="_blank"></a></div>
  <div class="head_sr l">
  <div id="head1">
  
  <a href="http://www.guoxuedashi.com/zidian/bujian/" target="_blank" ><img src="http://www.guoxuedashi.com/img/top1.gif" width="88" height="60" border="0" title="部件查字,支持20万汉字"></a>


<a href="http://www.guoxuedashi.com/help/yingpan.php" target="_blank"><img src="http://www.guoxuedashi.com/img/top230.gif" width="600" height="62" border="0" ></a>


  </div>
  <div id="head3"><a href="javascript:" onClick="javascript:window.external.AddFavorite(window.location.href,document.title);">添加收藏</a>
  <br><a href="/help/setie.php">搜索引擎</a>
  <br><a href="/help/zanzhu.php">赞助本站</a></div>
  <div id="head2">
 <a href="http://www.guoxuemi.com/" target="_blank"><img src="http://www.guoxuedashi.com/img/guoxuemi.gif" width="95" height="62" border="0" style="margin-left:2px;" title="国学迷"></a>
  

  </div>
</div>
  <div class="clear"></div>
  <div class="head_nav">
  <p><a href="/">首页</a> | <a href="/ShuKu/">国学书库</a> | <a href="/guji/">影印古籍</a> | <a href="/shici/">诗词宝典</a> | <a   href="/SiKuQuanShu/gxjx.php">精选</a> <b>|</b> <a href="/zidian/">汉语字典</a> | <a href="/hydcd/">汉语词典</a> | <a href="http://www.guoxuedashi.com/zidian/bujian/"><font  color="#CC0066">部件查字</font></a> | <a href="http://www.sfds.cn/"><font  color="#CC0066">书法大师</font></a> | <a href="/jgwhj/">甲骨文</a> <b>|</b> <a href="/b/4/"><font  color="#CC0066">解密</font></a> | <a href="/renwu/">历史人物</a> | <a href="/diangu/">历史典故</a> | <a href="/xingshi/">姓氏</a> | <a href="/minzu/">民族</a> <b>|</b> <a href="/mz/"><font  color="#CC0066">世界名著</font></a> | <a href="/download/">软件下载</a>
</p>
<p><a href="/b/"><font  color="#CC0066">历史</font></a> | <a href="http://skqs.guoxuedashi.com/" target="_blank">四库全书</a> |  <a href="http://www.guoxuedashi.com/search/" target="_blank"><font  color="#CC0066">全文检索</font></a> | <a href="http://www.guoxuedashi.com/shumu/">古籍书目</a> | <a   href="/24shi/">正史</a> <b>|</b> <a href="/chengyu/">成语词典</a> | <a href="/kangxi/" title="康熙字典">康熙字典</a> | <a href="/ShuoWenJieZi/">说文解字</a> | <a href="/zixing/yanbian/">字形演变</a> | <a href="/yzjwjc/">金 文</a> <b>|</b>  <a href="/shijian/nian-hao/">年号</a> | <a href="/diming/">历史地名</a> | <a href="/shijian/">历史事件</a> | <a href="/guanzhi/">官职</a> | <a href="/lishi/">知识</a> <b>|</b> <a href="/zhongyi/">中医中药</a> | <a href="http://www.guoxuedashi.com/forum/">留言反馈</a>
</p>
  </div>
</div>
<!-- 头部导航END --> 
<!-- 内容区开始 --> 
<div class="w1180 clearfix">
  <div class="info l">
   
<div class="clearfix" style="background:#f5faff;">
<script src='http://www.guoxuedashi.com/img/headersou.js'></script>

</div>
  <div class="info_tree"><a href="http://www.guoxuedashi.com">首页</a> > <a href="/SiKuQuanShu/fanti/">四库全书</a>
 > <h1>资治通鉴</h1> <!--         下载:【右键另存为】即可 --></div>
  <div class="info_content zj clearfix">
  
<div class="info_txt clearfix" id="show">
<center style="font-size:24px;">39-資治通鑑卷三十八</center>
    資治通鑑卷三十八   宋 司馬光 撰<br />
<br />
  胡三省 音註<br />
<br />
  漢紀三十【起旃蒙大淵獻盡玄黓敦牂凡八年】<br />
<br />
  王莽下<br />
<br />
  天鳳三年春二月大赦天下 民訛言黄龍墮死黄山宫中【晉灼曰黄山宫在槐里黄圖黄山宫在興平縣西三十里】百姓犇走往觀者有萬數莽惡之【師古曰莽自謂黄德故有此妖惡烏路翻】捕繫問所從起不能得 單于咸既和親求其子登屍莽欲遣使送致恐咸怨恨害使者乃收前言當誅侍子者故將軍陳欽以他辠殺之【誅侍子事見上卷始建國三年使疏吏翻】莽選辯士濟南王咸為大使夏五月莽復遣和親侯歙與咸等送右厨唯姑夕王因奉歸前所斬侍子登及諸貴人從者喪【濟子禮翻歙許及翻復扶又翻從才用翻喪息琅翻】單于遣云當子男大且渠奢等至塞迎之【且子余翻】咸到單于庭陳莽威德莽亦多遺單于金珍因諭說改其號【遺于季翻說輸芮翻】號匈奴曰恭奴單于曰善于賜印綬封骨都侯當為後安公當子男奢為後安侯單于貪莽金幣故曲聽之然寇盗如故 莽意以為制定則天下自平故銳思於地理制禮作樂講合六經之說【思相吏翻】公卿旦入暮出論議連年不决不暇省獄訟寃結【省悉井翻】民之急務縣宰缺者數年守兼【師古曰不拜正官權令人守兼】一切貪殘日甚中郎將繡衣執灋在郡國者並秉權勢傳相舉奏【傳知戀翻】又十一公士分布勸農桑班時令按諸章【應劭曰士掾也余按漢公府各有掾屬莽置十一公改掾曰士】冠蓋相望交錯道路召會吏民逮捕證左【左音佐】郡縣賦歛【歛力贍翻】逓相賕賂白黑紛然【師古曰白黑謂清濁也紛然亂之意言清濁不分也余謂白黑色之易别者且紛然不能分可謂繆亂之甚】守關告訴者多莽自見前顓權以得漢政故務自攬衆事有司受成苟免【師古曰莽事事自决成就乃以付吏吏苟免罪責而已】諸寶物名帑藏錢穀官皆宦者領之【帑他朗翻藏徂浪翻】吏民上封事宦官左右開發尚書不得知【舊上封事者先由尚書乃奏御莽恐尚書壅蔽令宦官左右發其封自省之上時掌翻】其畏備臣下如此又好變改制度【好呼到翻】政令煩多當奉行者輒質問乃以從事前後相乘憒眊不渫【師古曰乘積也登也憒眊不明也渫散也徹也憒音工内翻眊音莫報翻余謂前者省决未了而後者復來謂之相乘渫音泄清也】莽常御燈火至明猶不能勝【勝音升】尚書因是為姦寢事【上書者尚書不以聞而竊寢其事】上書待報者連年不得去拘繫郡縣者逢赦而後出衛卒不交代者至三歲穀糴常貴邊兵二十餘萬人仰衣食縣官【師古曰仰音牛句翻】五原代郡尤被其毒【被皮義翻】起為盗賊數千人為輩轉入旁郡莽遣捕盗將軍孔仁將兵與郡縣合擊歲餘乃定邯鄲以北大雨水出深者數丈流殺數千人【邯鄲音寒丹】三年春二月乙酉地震大雨雪【雨于具翻】關東尤甚深者一丈竹柏或枯【竹柏冬青或至於枯言常寒之咎】大司空王邑上書以地震乞骸骨莽不許曰夫地有動有震震者有害動者不害春秋記地震易繫坤動動靜辟翕萬物生焉【師古曰辟音闢闢開也易上繫之辭曰夫坤其動也闢其靜也翕是以廣生焉故莽引之也】其好自誣飾【好呼到翻】皆此類也 先是莽以制作未定【先悉薦翻】上自公侯下至小吏皆不得俸禄【俸扶用翻】夏五月莽下書曰予遭陽九之阨【傳曰三統之元有隂陽之九焉天地之常數也】百六之會國用不足民人騷動自公卿以下一月之禄十緵布二匹【孟康曰緵八十縷也師古曰緵音子公翻】或帛一匹【帛繒也】予每念之未嘗不戚焉今阨會已度府帑雖未能充【帑他朗翻】略頗稍給其以六月朔庚寅始賦吏禄皆如制度【賦布也與也】四輔公卿大夫士下至輿僚凡十五等【左傳曰人有十等王臣公公臣大夫大夫臣士士臣皁皁臣輿輿臣隸隸臣僚僚臣僕僕臣臺今莽自四輔以下分為十五等】僚禄一歲六十六斛稍以差稱【稱尺證翻】上至四輔而為萬斛云莽又曰古者歲豐穰則充其禮【師古曰穰音人掌翻又音如羊翻】有災害則有所損與百姓同憂喜也其用上計時通計【上時掌翻】天下幸無災害者太官膳羞備其品矣即有災害以什率多少而損膳焉【以十為率視災害所減多少而制分數】自十一公六司六卿以下【六司即前所置六監也】各分州郡國邑保其災害【東嶽太師立國將軍保東方三州一部二十五郡南嶽太傅前將軍保南方二州一部二十五郡西嶽國師寧始將軍保西方一州二部三十五郡北嶽國將衛將軍保北方二州一部二十五郡大司馬保納卿言卿仕卿作卿京尉扶尉兆隊右隊中部左洎前七部大司徒保樂卿典卿宗卿秩卿翼尉光尉左隊前隊中部右部有五郡大司空保予卿虞卿共卿工卿師尉烈尉祈隊後遂中部洎後十郡及六司六卿皆隨所屬之公保其災害】亦以十率多少而損其禄郎從官中都官吏食禄都内之委者【從才用翻委於偽翻委積也】以太官膳羞備損而為節冀上下同心勸進農業安元元焉莽之制度煩碎如此課計不可理吏終不得禄各因官職為姦受取賕賂以自共給焉【師古曰共讀曰供】 戊辰長平館西岸崩壅涇水不流毁而北行【長平館即長平觀在涇水之南原涇水東南流入渭為岸所壅故毁而北行】羣臣上夀以為河圖所謂以土填水【師古曰填讀與鎮同】匈奴滅亡之祥也莽乃遣并州牧宋弘游擊都尉任萌等將兵擊匈奴【任音壬將即亮翻】至邊止屯 秋七月辛酉霸城門災【黄圖霸城門長安城東出南頭第一門亦曰青門】 戊子晦日有食之大赦天下 平蠻將軍馮茂擊句町【句町音劬挺】士卒疾疫死者什六七賦歛民財什取五【歛力贍翻下同】益州虚耗而不克徵還下獄死【下遐稼翻】冬更遣寧始將軍廉丹與庸部牧史熊【孟康曰莽改益州為庸部余按莽置州牧部監州自是州部自是部今史熊為庸部牧則又若州部牧為一】大發天水隴西騎士廣漢巴蜀犍為吏民十萬人【犍居言翻】轉輸者合二十萬人擊之始至頗斬首數千其後軍糧前後不相及士卒飢疫莽徵丹熊丹熊願益調度必克乃還復大賦斂【調徒弔翻復扶又翻下同斂力贍翻】就都大尹馮英不肯給【莽於蜀郡廣都縣置就都大尹】上言自西南夷反叛以來積且十年郡縣距擊不已【上時掌翻】續用馮茂苟施一切之政僰道以南【地理志僰道縣屬犍為郡僰音蒲北翻】山險高深茂多敺衆遠居【敺與驅同】費以億計吏士罹毒氣死者什七今丹熊懼於自詭期會調發諸郡兵穀復訾民取其什四【師古曰發人訾財十取其四也訾與貲同】空破梁州功終不遂【莽改益州曰梁州師古曰遂成也爾雅註梁州以西方金氣剛強強梁也】宜罷兵屯田明設購賞莽怒免英官後頗覺寤曰英亦未可厚非復以英為長沙連率【率所類翻】粤嶲蠻夷任貴亦殺太守枚根【師古曰枚根者太守之姓名嶲音髓任音壬】 翟義黨王孫慶捕得莽使太醫尚方與巧屠共刳剥之【師古曰刳剖也音口胡翻】量度五臧【五臟心肺肝脾腎也周禮有九藏註曰正藏五又有胃旁胱大腸小腸疏曰五藏五者肺脾心肝腎又有胃旁胱大腸小腸者此乃六府中取此四者以益五藏為九藏也六府胃小腸大腸旁胱膽三焦以其受盛故謂之為府亦有藏稱故入九藏之數然六府取此四者案黄帝八十一難經說胃為水穀之府小腸為受盛之府大腸為行道之府旁胱為精液之府氣之所生下氣象天故放寫而不實實不滿若然則正府也故入九藏其餘膽者清淨之府三焦為孤府非正府故不入九藏師古曰度音大各翻臧讀曰臟】以竹筳導其脉知所終始云可以治病【師古曰筳竹挺也音庭按醫書脉有三部六經心部在左手寸口屬手少隂經與小腸手太陽經合肝部在左手關上屬足厥隂經與膽足少陽經合腎部在左手尺中屬足少隂經與膀胱足太陽經合肺部在右手寸口屬手太隂經與大腸手陽明經合脾部在右手關上屬足太隂經與胃足陽明經合右腎在右手尺中屬手厥隂心包經與三焦手少陽經合手少隂之脉起於心中出屬心系下膈絡小腸其支者從心系上俠咽系目其直者復從心系却上肺出腋下下循臑内後廉行太隂心主之後下肘内廉循臂内後廉抵掌後兑骨之端入掌内廉循小指之内出其端手太陽之脉起於小指之端循手外側上腕出踝中直上循臂骨下廉出肩解繞肩脾交肩上入缺盆絡心循咽下膈抵胃屬小腸其支别者從缺盆循頸上頰至目兑却入耳中其支者别頰上䪼抵鼻至目内眥足厥隂之脉起於大指聚毛之際上循足跗上廉去内踝一寸上踝八寸交出太隂之後上膕内亷循股入隂毛中環隂器抵小腹俠胃屬肝絡膽上貫膈布脇肋循㗋龍之後上入頑顙連目系上出額與督脉會於巔其支從目系下頰裏環脣内其支復從肝貫膈上注肺足少陽之脉起於目兌眥上抵頭角下耳後循頸行手少陽之脉前至肩上却交出少陽之後入缺盆其支别者從耳中出走耳前至目兌眥後其支别者自兌眥下大迎合手少陽於䪼下交頰車下頸合缺盆下胷中貫膈絡肝屬膽循脇裏出氣街繞髮際横入髀厭中直者從缺盆下腋循胷過季脇下合髀厭中以下循髀太陽出膝外亷下外輔骨之前直下抵絶骨之端下出外踝之前循足跗上入小指次指之間其支者從跗上入大指循岐骨出其端還貫入爪甲出三毛足少隂之脉起於小指之下斜趣足心出然谷之下循内踝之後别入跟中上腨内出膕内亷上股内後亷貫脊屬腎絡膀胱其直者從腎上貫肝膈入肺中循㗋嚨俠舌本其支從肺出絡心注胷中足太陽之脉起於目内眥上額交巔上其支别者從巔至耳上角其直行者從巔入絡腦還出别下項循肩膊内俠脊抵腰中入循膂絡腎屬膀胱其支别者從腰中下貫臀入膕中其支别者從膊内左右别下貫脾俠脊内過髀樞循髀外後廉下合膕中下貫膈内出外踝之後循京骨至小指外端手太隂之脉起於中焦下絡大腸還循胃口上膈屬肺從肺系横出腋下下循臑肉行少隂心主之前下肘中循臂内上骨下廉入寸口上魚循魚際出大指之端其支者從腕後直出次指内廉出其端手陽明之脉起於大指次指之端循指上廉出合谷兩肩之間上入兩筋之中循臂上廉入肘外廉循臑内前廉上肩出髃肩之前廉上出柱骨之會上下入缺盆絡肺下膈屬大腸其支者從缺盆上頸貫頰下入齒縫中還出俠口交人中左之右右之左上俠鼻孔足太隂之脉起於大指之端循指内側白肉際過竅骨後上内踝前廉上臑肉循䯒骨後交出厥隂之前上循膝股内前廉入腹屬脾絡腎上膈俠咽連舌本散舌下其支復從胃别上膈注心中足陽明之脉起於鼻交頞中下循鼻外入上齒中還出俠口環脣下交承漿却循頤後下廉出大迎承頰車上耳前過客主入循髮際至額顱其支者從人迎前下人迎循㗋嚨入缺盆下膈屬胃絡脾其直行者從缺盆下乳内廉下俠臍入氣充中其支者起胃下口循腹裹下至氣充而合以下髀關抵伏兎下入膝臏中下循䯒外廉下足跗入中指内間其支者下膝三寸而别以下入大指間出其端手厥隂之脉起於胷中出屬心包下膈歷絡三焦其支者循胷出脇下腋三寸上抵腋下下循臑内行太隂少隂之間入肘内行兩筋之間入掌中循中指出其端手少隂之脉起於小指次指之端上出次指之間循出表腕出臂外兩骨之間上貫肘循臑内上肩交出足少陽之後入缺盆交膻中散絡心包下膈徧屬三焦其支者從膻中上出缺盆上項俠耳後直出上耳上角以屈下頰至䪼其支者從耳後入耳中却出至目兌師古曰云可以治病者以知血脉之原則盡攻療之道也】 是歲遣大使五威將王駿西域都護李崇戊巳校尉郭欽出西域諸國皆郊迎送兵穀駿欲襲擊之焉耆詐降而聚兵自備【降戶江翻】駿等將莎車龜兹兵七千餘人分為數部【將即亮翻下同莎素何翻龜兹音丘慈】命郭欽及佐帥何封别將居後【帥所類翻】駿等入焉耆焉耆伏兵要遮駿及姑墨封犂危須國兵為反間【姑墨國王治南城去長安八千一百五十里封犁漢書作尉犁要一遥翻間古莧翻】還共襲駿皆殺之欽後至焉耆焉耆兵未還欽襲擊殺其老弱從車師還入塞莽拜欽為填外將軍【師古曰填音竹刃翻】封劋胡子【師古曰劋絶也音子小翻】何封為集胡男李崇收餘士還保龜兹【龜兹音丘慈】及莽敗崇沒西域遂絶<br />
<br />
  四年夏六月莽更授諸侯王茅土於明堂親設文石之平陳菁茅四色之土【師古曰尚書禹貢包匭菁茅儒者以為菁菜名也茅三脊茅也而莽此言以菁茅為一物則是謂善茅為菁茅也土有五色而此云四者中央之土不以封也春秋大傳曰天子之國有泰社東方青南方赤西方白北方黑上方黄故將封於東方者取青土封於南者取赤土封於西者取白土封於北者取黑土各取其方土裹以白茅封以為社此始受封於天子者也此之謂主土主土者立社以奉之也菁音精】告於岱宗泰社后土先祖先妣以班授之莽好空言【好呼到翻】慕古灋多封爵人性實吝嗇託以地理未定故且先賦茅土用慰喜封者 秋八月莽親之南郊鑄作威斗以五石銅為之【李奇曰以五色藥石及銅為之蘇林曰以五色銅鑛治之師古曰李說是也若今作鍮石之為】若北斗長二尺五寸【長直亮翻】欲以厭勝衆兵【師古曰厭音一葉翻】既成令司命負之莽出在前入在御旁莽置羲和命士以督五均六筦【鹽一也酒二也鐵三也名山大澤四也五均賖貸五也鐵布銅冶六也】郡有數人皆用富賈為之【賈音古】乘傳求利【傳知戀翻】交錯天下因與郡縣通姦多張空簿【師古曰簿計簿也音步戶翻】府藏不實【藏徂浪翻】百姓愈病是歲莽復下詔申明六筦【下遐稼翻】每一筦為設科條防禁【為于偽翻】犯者罪至死姦民猾吏並侵衆庶各不安生又一切調上公以下諸有奴婢者【調復釣翻】率一口出三千六百天下愈愁納言馮常以六筦諫莽大怒免常官灋令煩苛民揺手觸禁不得耕桑繇役煩劇【師古曰繇讀曰徭】而枯旱蝗蟲相因獄訟不决吏用苛暴立威旁緣莽禁【師古曰旁依也音步浪翻】侵刻小民富者不能自别【别彼列翻】貧者無以自存於是並起為盜賊依阻山澤吏不能禽而覆蔽之浸淫日廣【覆敷救翻師古曰漫淫猶漸染也余謂此以水為諭漸浸而至於淫溢也】臨淮瓜田儀依阻會稽長州【服䖍曰姓瓜田名儀師古曰長州即枚乘所云長州之苑余謂今蘇州長洲縣即其地會工外翻】琅邪呂母聚黨數千人殺海曲宰入海中為盜【莽改縣令長曰宰初呂母子為縣吏為宰所寃殺母散家財以酤酒買弓弩隂厚貧窮少年得百餘人遂攻海曲縣殺其宰以祭于墓地理志海曲縣屬琅邪郡賢曰故城在密州莒縣東】其衆浸多至萬數荆州饑饉民衆入野澤掘鳬茈而食之【荆州部南陽南郡桂陽武陵零陵江夏等郡爾雅曰芍鳬茈郭璞曰生下田中苗似龍鬚而細根如指根黑色可食茈音才支翻芍音胡了翻】更相侵奪【更工衡翻】新市人王匡王鳳為平理諍訟【地理志新市縣屬江夏郡為于偽翻諍與爭同晉王沈釋時論闒茸勇敢於饕諍叶韻平聲古字多假借用也】遂推為渠帥衆數百人【孔安國曰渠大也帥所類翻】於是諸亡命者南陽馬武潁川王常成丹等皆往從之共攻離鄉聚臧於緑林山中【賢曰離鄉聚謂諸鄉聚離散去城郭遠者大曰鄉小曰聚前書曰收合離鄉置大城中即其義也緑林山在今荆州當陽縣東北余按郡國志新市侯國有離鄉聚緑林山則以離鄉為聚名聚才喻翻臧古藏字】數月間至七八千人又有南郡張霸江夏羊牧等與王匡俱起衆皆萬人莽遣使者即赦盜賊【即就也就其相聚為盜處而赦之也】還言盜賊解輒復合【復扶又翻】問其故皆曰愁灋禁煩苛不得舉手力作所得不足以給貢稅閉門自守又坐鄰伍鑄錢挾銅姦吏因以愁民民窮悉起為盜賊莽大怒免之其或順指言民驕黠當誅【黠下八翻】及言時運適然且滅不久莽說輒遷官【說讀曰悦】<br />
<br />
  五年春正月朔北軍南門災【北軍壘門之南出者也】 以大司馬司允費興為荆州牧見問到部方略【引見而問其方畧也見賢遍翻】興對曰荆揚之民率依阻山澤以漁采為業【師古曰漁謂捕魚也采謂采取蔬菓之屬】間者國張六筦稅山澤妨奪民之利連年久旱百姓饑窮故為盜賊興到部欲令明曉告盜賊歸田里假貸犁牛種食【種章勇翻】濶其租賦【師古曰濶寛也】冀可以解釋安集莽怒免興官 天下吏以不得俸禄【俸扶用翻】並為姦利郡尹縣宰家累千金莽乃考始建國二年胡虜猾夏以來諸軍吏及緣邊吏大夫以上為姦利增產致富者收其家所有財產五分之四以助邊急【助邊費之急也】公府士馳傳天下【傳知戀翻】考覆貪饕【師古曰饕音士高翻】開吏告其將【將即亮翻】奴婢告其主冀以禁姦而姦愈甚 莽孫功崇公宗坐自畫容貌被服天子衣冠刻三印發覺自殺【莽傳功崇公國於穀城郡三印一曰維祉冠存已夏處南山臧薄氷二曰肅聖寶繼三曰德封昌圖畫古畫通被皮義翻】宗姊妨為衛將軍王興夫人坐祝詛姑【祝職救翻詛莊徂翻】殺婢以絶口與興皆自殺 是歲揚雄卒初成帝之世雄為郎給事黄門與莽及劉秀並列哀帝之初又與董賢同官莽賢為三公權傾人主所薦莫不拔擢而雄三世不徙官及莽簒位雄以耆老久次轉為大夫恬於勢利【師古曰恬安也】好古樂道【好呼到翻樂音洛】欲以文章成名於後世乃作太玄以綜天地人之道【桓譚曰揚雄作玄書以為玄者天也道也言聖賢制法作事皆引天道以為本統而因附屬萬類王政人事法度故伏羲氏謂之易老子謂之道孔子謂之元揚雄謂之玄玄經三篇以紀天地人之道立三體有上中下如禹貢之陳三品三三而九因以九九八十一故為八十一卦以四為數數從一至四重累變易竟八十一而徧不可增損以三十五蓍揲之玄經五千餘言而傳十二篇】又見諸子各以其智舛馳【師古曰舛相背】大抵詆訾聖人即為怪迂析辯詭辭以撓世事【師古曰大抵大歸也詆訾毁也迂遠也析分也詭異也言諸子之書大歸皆非毁周孔之教為巧辯異辭以撓亂時政也訾音紫迂音于撓音火高翻】雖小辯終破大道而惑衆使溺於所聞而不自知其非也故人時有問雄者常用法應之號曰法言用心於内不求於外於時人皆忽之【師古曰忽謂輕也】唯劉秀及范逡敬焉而桓譚以為絶倫【師古曰無比類】鉅鹿侯芭師事焉【服䖍曰芭音葩】大司空王邑納言嚴尤聞雄死謂桓譚曰子常稱揚雄書豈能傳於後世乎譚曰必傳顧君與譚不及見也凡人賤近而貴遠親見揚子雲禄位容貌不能動人【揚雄字子雲】故輕其書昔老聃著虚無之言兩篇【師古曰謂道德輕也】薄仁義非禮學然後好之者尚以為過於五經【好呼到翻】自漢文景之君及司馬遷皆有是言今揚子之書文義至深而論不詭於聖人【師古曰詭違也聖人謂周公孔子】則必度越諸子矣 琅邪樊崇起兵於莒【莒縣班志屬城陽國續漢志屬琅邪國邪音耶】衆百餘人轉入太山羣盜以崇勇猛皆附之一歲間至萬餘人崇同郡人逢安【賢曰逢音龐】東海人徐宣謝禄楊音各起兵合數萬人復引從崇【復扶又翻】共還攻莒不能下轉掠青徐間又有東海刁子都【刁一作力姓譜力黄帝佐力牧之後漢有力子都】亦起兵鈔擊徐兖【鈔楚交翻】莽遣使者發郡國兵擊之不能克 烏累單于死【累力追翻】弟左賢王輿立為呼都而尸道臯若鞮單于【鞮丁奚翻】輿既立貪利賞賜遣大且渠奢與伊墨居次云女弟之子醯櫝王【且子余翻師古曰櫝音讀】俱奉獻至長安莽遣和親侯歙與奢等俱至制虜塞下與云及須卜當會因以兵迫脅云當將至長安云當小男從塞下得脫歸匈奴當至長安莽拜為須卜單于欲出大兵以輔立之兵調度亦不合【調徒弔翻】而匈奴愈怒並入北邊為寇<br />
<br />
  六年春莽見盜賊多乃令太史推三萬六千歲歷紀六歲一改元布天下下書自言己當如黄帝僊升天欲以誑燿百姓銷解盜賊衆皆笑之 初獻新樂於明堂太廟【新樂莽所作也】更始將軍廉丹擊益州不能克【丹蓋自寧始將軍遷更始將軍更工衡翻】益州夷棟蠶若豆等起兵殺郡守越嶲夷人大牟亦叛殺略吏人【按後漢書棟蠶若豆益州夷兩種也大牟越嶲姑復縣夷人嶲音髓】莽召丹還更遣大司馬護軍郭興庸部牧李曅擊蠻夷若豆等【更工衡翻】太傅羲叔士孫喜清潔江湖之盜賊【莽以太傅主夏故置羲叔官士孫複姓姓譜漢平陵士孫張為博士明梁丘易】而匈奴寇邊甚莽乃大募天下丁男及死罪囚吏民奴名曰豬突豨勇以為銳卒【服䖍曰豬性觸突人故以為諭師古曰東方人名豕曰豨或曰豨豕走也音許豈翻】一切稅天下吏民訾三十取一【訾與貲同】縑帛皆輸長安令公卿以下至郡縣黄綬皆保養軍馬【續漢志四百石三百石二百石黄綬師古曰保者不許其死傷】多少各以秩為差吏盡復以與民【師古曰轉令百姓養之】又博募有奇技術可以攻匈奴者【技渠綺翻】將待以不次之位言便宜者以萬數或言能度水不用舟楫【師古曰楫所以刺舟也】連馬接騎濟百萬師或言不持斗糧服食藥物三軍不饑或言能飛一日千里可窺匈奴莽輒試之取大鳥翮為兩翼【師古曰羽本曰翮音胡隔翻】頭與身皆著毛【著側略翻】通引環紐【紐女九翻】飛數百步墮莽知其不可用苟欲獲其名皆拜為理軍賜以車馬待發初莽之欲誘迎須卜當也大司馬嚴尤諫曰當在匈奴右部兵不侵邊單于動靜輒語中國【語牛倨翻】此方面之大助也于今迎當置長安槀街一胡人耳不如在匈奴有益莽不聽既得當欲遣尤與廉丹擊匈奴皆賜姓徵氏號二徵將軍令誅單于輿而立當代之出車城西横廐未發尤素有智略非莽攻伐四夷數諫不從【數所角翻】及當出廷議尤固言匈奴可且以為後先憂山東盜賊莽大怒策免尤 大司空議曹史代郡范升【漢公府諸曹有掾有史有屬皆公自辟置】奏記王邑曰升聞子以人不間於其父母為孝【間古莧翻】臣以下不非其君上為忠【賢曰論語孔子曰孝哉閔子騫人不間於其父母之言間非也言子騫之孝化其父母言人無非之者忠臣事君有過即諫在下無有非其君者是忠臣也】今衆人咸稱朝聖【朝直遥翻下同】皆曰公明蓋明者無不見聖者無不聞今天下之事昭昭於日月震震於靁霆而朝云不見公云不聞則元元焉所呼天【元元民也良善之民師古曰元元善意也焉於䖍翻】公以為是而不言則過小矣知而從令則過大矣二者於公無可以免宜乎天下歸怨於公矣朝以遠者不服為至念升以近者不悦為重憂【遠者不服謂四夷也近者不悦謂人心不便於莽之法令也】今動與時戾事與道反馳騖覆車之轍踵循敗事之後後出益可怪晚發愈可懼耳方春歲首而動發遠役藜藿不充田荒不耕穀價騰躍斛至數千吏民䧟於湯火之中非國家之民也如此則胡貊守闕青徐之寇在於帷帳矣【謂京輔之民亦將為變也】升有一言可以解天下倒縣【縣讀曰懸】免元元之急不可書傳願蒙引見極陳所懷邑不聽 翼平連率田况奏郡縣訾民不實【地理志北海壽光縣莽曰翼平師古曰言舉百姓訾財不以實數率所類翻訾與貲同】莽復三十取一以况忠言憂國進爵為伯賜錢二百萬衆庶皆詈之青徐民多棄鄉里流亡老弱死道路壯者入賊中 夙夜連率韓博【地理志東萊不夜縣莽曰夙夜】上言有奇士長丈大十圍【長直亮翻】來至臣府欲奮擊胡虜自謂巨毋霸出於蓬萊東南五城西北昭如海瀕【師古曰昭如海名瀕厓也神仙家言蓬萊有五城十二樓】軺車不能載【軺音遥小車】三馬不能勝【勝音升】即日以大車四馬建虎旗載霸詣闕霸卧則枕鼓以鐵箸食【枕職任翻箸遟倨翻】此皇天所以輔新室也願陛下作大甲高車賁育之衣遣大將一人與虎賁百人【賁音奔】迎之於道京師門戶不容者開高大之以示百蠻鎮安天下博意欲以風莽【以莽字巨君諷言毋得簒盜而霸風讀曰諷】莽聞惡之【惡烏路翻】留霸在所新豐【師古曰在所謂其見到之處】更其姓曰巨母氏謂因文母太后而霸王符也【師古曰莽字巨君若言文母出此人而使我致霸王更音古衡翻】徵博下獄【下遐稼翻】以非所宜言棄市 關東饑旱連年刁子都等黨衆寖多至六七萬<br />
<br />
  地皇元年春正月乙未赦天下改元曰地皇從三萬六千歲歷號也 莽下書曰方出軍行師敢有趨讙犯灋者輒論斬毋須時【師古曰趨讙謂趨走而讙譁也須待也讙許元翻】於是春夏斬人都市百姓震懼道路以目【韋昭曰不敢發言以目相眄而已】 莽見四方盜賊多復欲厭之【復扶又翻師古曰厭音一葉翻】下書又曰予之皇初祖考黄帝定天下將兵為大將軍内設大將外置大司馬五人大將軍至士吏凡七十三萬八千九百人士千三百五十萬人予受符命之文稽前人將條備焉於是置前後左右中大司馬之位賜諸州牧至縣宰皆有大將軍偏裨校尉之號焉【州牧為大將軍卒正連率大尹為偏將軍屬令長為禆將軍縣宰為校尉】乘傳使者經歷郡國日且十輩倉無見穀以給【師古曰見謂見在也傳知戀翻見賢遍翻】傳車馬不能足賦取道中車馬【師古曰於道中行者即執取之以充事也】取辦於民 秋七月大風毁王路堂【莽改未央宫前殿曰王路堂服䖍曰如言路寢也路大也】莽下書曰乃壬午餔時有烈風靁雨發屋折木之變【餔食也餔時食時也或曰餔即晡時日加申為晡師古曰烈風烈暴之風折而設翻】予甚恐焉伏念一旬迷乃解矣【師古曰先言烈風雷雨後言迷乃解矣盖取舜烈風雷雨弗迷以為言也】昔符命立安為新遷王臨國洛陽為統義陽王議者皆曰臨國洛陽為統謂據土中為新室統也宜為皇太子自此後臨久病雖瘳不平【言疾雖有瘳不能平復如其初也】臨有兄而稱太子名不正惟即位以來隂陽未和穀稼鮮耗【師古曰鮮少也耗減也鮮音先踐翻】蠻夷猾夏寇賊姦宄【夏戶雅翻】人民征營無所錯手足【師古曰征營惶恐不自安之意也錯七故翻】深惟厥咎在名不正焉其立安為新遷王【服䖍曰安莽第三子也遷音仙莽改汝南新蔡曰新遷師古曰遷猶僊耳不勞假借音】臨為統義陽王莽又下書曰寶黄厮赤【服䖍曰以黄為寶自用其行氣也厮赤厮役賤者皆衣赤】<br />
<br />
  【賤漢行也厮音斯】其令郎從官皆衣絳【從才用翻衣於既翻】 望氣為數者多言有土功象九月甲申莽起九廟於長安城南【九廟祖廟五親廟四】黄帝廟方四十丈高十七丈【高居傲翻】餘廟半之制度甚盛博徵天下工匠及吏民以義入錢穀助作者駱驛道路【師古曰駱驛言不絶】窮極百工之巧功費數百餘萬卒徒死者萬數 是月大雨六十餘日 鉅鹿男子馬適求等謀舉燕趙兵以誅莽【師古曰馬適姓也求名也】大司空士王丹發覺以聞莽遣三公大夫逮治黨與連及郡國豪桀數千人皆誅死封丹為輔國侯 莽以私鑄錢死及非沮寶貨投四裔【事見上卷始建國二年沮在呂翻】犯灋者多不可勝行【勝音升】乃更輕其灋私鑄作泉布者與妻子沒入為官奴婢吏及比伍知而不舉告與同罪【師古曰比音頻寐翻又頻脂翻】非沮寶貨民罰作一歲吏免官 太傅平晏死以予虞唐尊為太傅尊曰國虛民貧咎在奢泰乃身短衣小褏乘牝馬柴車藉槀以瓦器飲食【師古曰柴車即棧車藉槀去蒲蒻也褏古袖字余按漢氏之盛乘牸牝者禁不得會聚至鄉閭阡陌皆然朝市之間從可知矣尊為上公而乘牝亦以矯世也】又以歷遺公卿【遺于季翻】出見男女不異路者尊自下車以象刑赭幡汚染其衣【師古曰赭幡以赭汁潰巾幡汚烏故翻】莽聞而說之【說讀曰悦】下詔申敕公卿思與厥齊【師古曰令與尊同此操行也論語稱孔子曰見賢思齊故莽云然】封尊為平化侯 汝南郅惲明天文歷數以為漢必再受命上書說莽曰【惲於粉翻說輸芮翻】上天垂戒欲悟陛下令就臣位取之以天還之以天可謂知命矣莽大怒繋惲詔獄踰冬會赦得出<br />
<br />
  二年春正月莽妻死諡曰孝睦皇后初莽妻以莽數殺其子【數所角翻莽殺子獲見三十四卷哀帝建平二年通鑑書於三十五卷天夀元年殺子宇見三十六卷平帝元始三年】涕泣失明莽令太子臨居中養焉【養余亮翻】莽妻旁侍者原碧莽幸之臨亦通焉恐事泄謀共殺莽臨妻愔國師公女【師古曰愔音一尋翻】能為星語臨宫中且有白衣會【晉書天文志木與金合為白衣之會土與金合亦為白衣之會言宫中者以所會之舍占而知之語牛倨翻】臨喜以為所謀且成後貶為統義陽王出在外第愈憂恐會莽妻病困臨予書曰【予讀曰與】上於子孫至嚴前長孫中孫年俱三十而死【宇字長孫獲字中孫獲先死安得俱年三十乎長知兩翻中讀曰仲】今臣臨復適三十誠恐一旦不保中室則不知死命所在【李奇曰中室臨之母也晉灼曰長樂宫中殿也師古曰二說皆非也臨自言欲於室中自保全不可得耳復扶又翻下同】莽候妻疾見其書大怒疑臨有惡意不令得會喪既葬收原碧等考問具服姦謀殺狀莽欲祕之使殺案事使者司命從事埋獄中【司命從事司命之屬官也】家不知所在賜臨藥臨不肯飲自刺死【刺七亦翻】又詔國師公臨本不知星事從愔起愔亦自殺 是月新遷王安病死初莽為侯就國時【哀帝初莽就國元夀元年召還京師】幸侍者增秩懷能生子興匡皆留新都國以其不明故也【師古曰言侍者或與外人私通所生子不可分明也】及安死莽乃以王車遣使者迎興匡封興為功脩公匡為功建公 卜者王况謂魏成大尹李焉曰【莽改魏郡曰魏成】漢家當復興李氏為輔因為焉作䜟書合十餘萬言【因為于偽翻】事發莽皆殺之 莽遣太師羲仲景尚【莽以太師主春其屬置羲仲官】更始將軍護軍王黨【諸將軍皆置護軍】將兵擊青徐賊國師和仲曹放助郭興擊句町【莽以國師主秋故置和仲句町音劬挺】皆不能克軍師放縱百姓重困【重直用翻】 莽又轉天下穀帛詣西河五原朔方漁陽每一郡以百萬數欲以擊匈奴須卜當病死莽以庶女妻其子後安公奢【莽女捷侍者開明所生也以妻奢李奇曰奢本為侯莽以女妻之故進爵為公妻七組翻】所以尊寵之甚厚終欲為出兵立之者【師古曰言為此計意不止為于偽翻下同】會莽敗云奢亦死秋隕霜殺菽關東大饑蝗 莽既輕私鑄錢之灋犯<br />
<br />
  者愈衆及伍人相坐沒入為官奴婢其男子檻車女子步以鐵瑣琅當其頸傳詣鍾官以十萬數【師古曰琅當長瑣也鍾官主鑄錢之官也】到者易其夫婦【師古曰改相配匹不依其舊也】愁苦死者什六七 上谷儲夏自請說瓜田儀降之【儲夏人姓名戰國時齊有儲子】儀未出而死莽求其尸葬之為起冢祠室謚曰瓜寧殤男【此殤非未成人之殤強死者也楚辭所謂國殤者】 閏月丙辰大赦 郎陽成脩獻符命【姓陽成名脩而官為郎也】言繼立民母又曰黄帝以百二十女致神僊【漢儒言天子三夫人九嬪二十七世婦八十一御妻則亦百二十女】莽於是遣中散大夫謁者各四十五人【百官志中散大夫秩六百石時屬司中】分行天下【行下孟翻】博采鄉里所高有淑女者上名【上時掌翻】 莽惡漢高廟神靈【惡烏路翻】遣虎賁武士入高廟四面提擊【師古曰謂夢見譴責提擲也音徒計翻】斧壞戶牖【師古曰以斧斫壞之壞音怪】桃湯赭鞭鞭灑屋壁【師古曰桃湯灑之赭鞭鞭之也赭赤也】令輕車校尉居其中【漢以虎賁校尉主輕車此輕車校尉莽所置也】 是歲南郡秦豐聚衆且萬人平原女子遟昭平亦聚數千人在河阻中【姓譜遟姓也樊遟之後以王父字為氏一曰古賢人遟任之後】莽召問羣臣禽賊方畧皆曰此天囚行尸命在漏刻【言其得罪於天死在須臾其猖狂為盜特尸行耳】故左將軍公孫禄徵來與議【師古曰與讀曰豫】禄曰太史令宗宣【宗姓晉伯宗之後伯宗本出於宋桓公】典星歷候氣變以凶為吉亂天文誤朝廷太傅平化侯尊飾虚偽以媮名位賊夫人之子【夫音扶】國師嘉信公秀【信當作新】顛倒五經毁師灋令學士疑惑明學男張邯地理侯孫陽造井田使民棄土業【邯下甘翻】羲和魯匡設六筦以窮工商說符侯崔發阿諛取容令下情不上通宜誅此數子以慰天下又言匈奴不可攻當與和親臣恐新室憂不在匈奴而在封域之中也莽怒使虎賁扶禄出【禄之言則直矣然以漢舊臣而與莽朝之議出處語默於義得乎事君若龔勝者可也】然頗采其言左遷魯匡為五原卒正以百姓怨誹故也六筦非匡所獨造莽厭衆意而出之【師古曰厭滿也音一艶翻】初四方皆以饑寒窮愁起為盜賊稍羣聚常思歲熟得歸鄉里衆雖萬數不敢略有城邑日闋而已【師古曰闋盡也隨口而盡也此言羣盜攻剽所得日給口體而已闋空穴翻】諸長吏牧守皆自亂鬬中兵而死【師古曰中傷也音竹仲翻】賊非敢欲殺之也而莽終不諭其故【師古曰不曉此意也】是歲荆州牧發犇命二萬人討緑林賊賊帥王匡等相率迎擊於雲杜【賢曰雲杜縣名屬江夏郡故城在今復州沔陽縣西北杜佑曰安州應城縣漢雲杜縣地】大破牧軍殺數千人盡獲輜重【重直用翻】牧欲北歸馬武等復遮擊之【復扶又翻】鈎牧軍屏泥【屏泥緹油飾之在軾前】刺殺其驂乘【刺七亦翻乘繩證翻】然終不敢殺牧賊遂攻拔竟陵【賢曰竟陵縣名屬江夏郡故城在郢州長夀縣南】轉撃雲杜安陸【賢曰安陸縣屬江夏郡今安州縣】多略婦女還入緑林中至有五萬餘口州郡不能制又大司馬士按章豫州【師古曰有上章相告者就而按治之豫州部頴州汝南沛郡梁國魯國】為賊所獲賊送付縣士還上書具言狀【還從宣翻又如字】莽大怒以為誣罔因下書責七公曰【七公謂四輔三公】夫吏者理也宣德明恩以牧養民仁之道也抑彊督姦【師古曰督謂察視也】捕誅盜賊義之節也今則不然盜發不輒得至成羣黨遮畧乘傳宰士【傳知戀翻】士得脫者又妄自言我責數賊何為如是【數所具翻】賊曰以貧窮故耳賊護出我今俗人議者率多若此惟貧困饑寒犯灋為非大者羣盜小者偷宂【師古曰宂謂穿墻為盜也】不過二科今乃結謀連黨以千百數是逆亂之大者豈饑寒之謂邪七公其嚴敕卿大夫卒正連率庶尹謹牧養善民急捕殄盜賊有不同心并力疾惡黠賊【惡烏路翻黠下八翻】而妄曰饑寒所為輒捕繫請其罪【請治其罪也】於是羣下愈恐莫敢言賊情者州郡又不得擅發兵賊由是遂不制【言不可制也】唯翼平連率田况素果敢發民年十八以上四萬餘人授以庫兵與刻石為約樊崇等聞之不敢入界况自劾奏【劾戶槩翻】莽讓况未賜虎符而擅發兵此弄兵也厥罪乏興【師古曰擅發之罪與乏軍興同科也】以况自詭必禽滅賊故且勿治【治直之翻下同】後况自請出界擊賊所嚮皆破莽以璽書令况領青徐二州牧事【璽斯氏翻】况上言盜賊始發其原甚微部吏伍人所能禽也【部吏部盜賊之吏郡賊曹縣游徼郷亭長之類是也師古曰伍人同伍之人若今伍保者也】咎在長吏不為意縣欺其郡郡欺朝廷實百言十實千言百朝廷忽畧不輒督責遂至延蔓連州【師古曰延音亦戰翻】乃遣將帥多使者傳相監趣【傳知戀翻監古衘翻師古曰趣讀曰促】郡縣力事上官應塞詰對【師古曰力勤也塞當也塞悉則翻詰去吉翻】共酒食具資用以救斷斬【師古曰交懼斬死之刑也共讀曰供斷丁管翻又丁亂翻】不暇復憂盜賊治官事【復扶又翻治直之翻】將帥又不能躬率吏士戰則為賊所破吏氣浸傷徒費百姓前幸蒙赦令賊欲解散或反遮擊恐入山谷轉相告語【語牛倨翻】故郡縣降賊皆更驚駭恐見詐滅因饑饉易動【降戶江翻易以豉翻】旬日之間更十餘萬人此盜賊所以多之故也今洛陽以東米石二千竊見詔書欲遣太師更始將軍二人爪牙重臣多從人衆道上空竭【言牢禀不給也】少則無以威示遠方宜急選牧尹以下明其賞罰收合離鄉小國無城郭者【小國諸列侯國也】徙其老弱置大城中積臧穀食【臧讀曰藏】并力固守賊來攻城則不能下所過無食埶不得羣聚如此招之必降擊之則滅今空復多出將帥【復扶又翻】郡縣苦之反甚於賊宜盡徵還乘傳諸使者【傳知戀翻】以休息郡縣委任臣况以二州盜賊必平定之莽畏惡况【畏惡其能也惡烏路翻】隂為發代【為于偽翻】遣使者賜况璽書使者至見况因令代監其兵【監古衘翻】遣况西詣長安拜為師尉大夫况去齊地遂敗<br />
<br />
  三年春正月九廟成納神主【木主也】莽謁見大駕乘六馬以五采毛為龍文衣著角長三尺【師古曰以被馬上也著陟畧翻長直亮翻】又造華蓋九重【古今注曰華盖黄帝所作也黄帝與蚩尤戰于涿鹿之野常有雲氣金枝玉葉因而作華盖重直龍翻】高八丈一尺【高居傲翻】載以四輪車輓者皆呼登仙莽出令在前百官竊言此似輀車【輀音而喪車也】非僊物也 二月樊崇等殺景尚【景尚去年所遣】 關東人相食 夏四月遣太師王匡更始將軍廉丹東討衆賊【更工衡翻】初樊崇等衆既寖盛乃相與為約殺人者死傷人者償創【創初良翻】其中最尊號三老次從事次卒史【師古曰言不為大號余謂三老從事卒史皆郡縣史也崇等起於民伍所識止此耳其後黨衆日盛氣勢日張則攻長安立盆子非其初不為大號也】及聞太師更始將討之恐其衆與莽兵亂乃皆朱其眉以相識别【别彼列翻】由是號曰赤眉匡丹合將銳士十餘萬人【將即亮翻下遣將同】所過放縱東方為之語曰寧逢赤眉不逢太師太師尚可更始殺我卒如田况之言【卒子恤翻】 莽又多遣大夫謁者分教民煮草木為酪【服䖍曰煮木實或曰如今餌木之屬也如淳曰作杏酪之屬也師古曰如說是也】酪不可食重為煩費【師古曰重直用翻】 緑林賊遇疾疫死者且半乃各分散引去王常成丹西入南郡號下江兵王鳳王匡馬武及其支黨朱鮪張卬等北入南陽號新市兵【郡國志新市縣屬江夏郡水經注新市縣治杜城屬竟陵郡杜佑曰漢新市縣故城在郢州富水縣東北】皆自稱將軍莽遣司命大將軍孔仁部豫州納言大將軍嚴尤秩宗大將軍陳茂擊荆州【莽賜司卿及六卿號皆大將軍】各從吏士百餘人乘傳到部募士【傳知戀翻】尤謂茂曰遣將不與兵符必先請而後動是猶紲韓盧而責之獲也【師古曰紲繫也韓盧古韓國之名犬也黑色曰盧紲音私列翻】 蝗從東方來飛蔽天 流民入關者數十萬人乃置養贍官禀食之【師古曰禀給也音彼甚翻食讀曰飤】使者監領【監古衘翻】與小吏共盜其禀【師古曰盜其禀者盜所給之物】餓死者什七八先是莽使中黄門王業領長安市買【先悉薦翻】賤取於民民甚患之業以省費為功賜爵附城莽聞城中饑饉以問業業曰皆流民也乃市所賣粱飯肉羹持入示莽曰居民食咸如此莽信之 秋七月新市賊王匡等進攻隨【賢曰隨縣屬南陽郡今隨州縣】平林人陳牧廖湛復聚等千餘人號平林兵以應之【姓譜廖周文王子伯廖之後風俗通古有廖叔安左傳作颶蓋其後也水經注章水南逕隨郡平林縣故城西俗謂之將陂城與新市接界賢曰廖音力弔翻平林地名在今隨州隨縣東北復扶又翻】 莽詔書讓廉丹曰倉廪盡矣府庫空矣可以怒矣可以戰矣將軍受國重任不捐身於中野無以報恩塞責【塞悉則翻】丹惶恐夜召其掾馮衍以書示之【掾俞絹翻】衍因說丹曰張良以五世相韓椎秦始皇博浪之中【事見七卷秦始皇二十九年說輸芮翻相息亮翻】將軍之先為漢信臣【賢曰廉褒襄武人宣帝時為後將軍即丹之先】新室之興英俊不附今海内潰亂人懷漢德甚於周人思召公也【召公之教明於南國周人思之為賦甘棠召音邵】人所歌舞天必從之【賢曰詩小雅曰雖無德與女式歌且舞言漢氏之德人歌舞之也書曰人之所欲天必從之】今方為將軍計莫若屯據大郡鎮撫吏士砥厲其節納雄桀之士詢忠智之謀興社稷之利除萬人之害則福祚流于無窮功烈著於不滅何與軍覆於中原身膏於草野功敗名喪恥及先祖哉【師古曰與猶知也喪息浪翻】丹不聽衍左將軍奉世曾孫也冬無鹽索盧恢等舉兵反城附賊【師古曰索盧姓也恢名也呂氏春秋禽滑釐有門人索盧參反城據城以反也一曰反音幡今語賊猶曰幡城索音先各翻余謂一說是賢曰無鹽縣屬東平郡故城在今鄆州須昌縣之東】廉丹王匡攻拔之斬首萬餘級莽遣中郎將奉璽書勞丹匡【勞力到翻】進爵為公封吏士有功者十餘人赤眉别校董憲等衆數萬人在梁郡【校戶教翻下同梁國時除為郡】王匡欲進擊之廉丹以為新抜城罷勞【罷讀曰疲】當且休士養威匡不聽引兵獨進丹隨之合戰成昌【師古曰成昌地名也余據後漢書亦當在無鹽縣界】兵敗匡走丹使吏持其印韍節付匡【韍音弗】曰小兒可走吾不可遂止戰死校尉汝雲王隆等二十餘人别鬬【汝姓也左傳晉大夫女齊陸德明曰女音汝】聞之皆曰廉公已死吾誰為生【為于偽翻】馳犇賊皆戰死國將哀章自請願平山東【將即亮翻】莽遣章馳東與太師匡并力又遣大將軍楊浚守敖倉司徒王尋將十餘萬屯洛陽鎮南宮大司馬董忠養士習射中軍北壘【恐當作北軍中壘】大司空王邑兼三公之職 初長沙定王發生舂陵節侯買買生戴侯熊渠熊渠生考侯仁仁以南方卑濕徙封南陽之白水鄉與宗族往家焉【賢曰舂陵鄉名本屬零陵泠道縣在今永州唐興縣北元帝時徙南陽仍號舂陵故城在今隨州棗陽縣東杜佑曰棗陽後漢蔡陽縣漢舂陵故城在今縣東】仁卒子敞嗣值莽簒位國除節侯少子外為鬱林太守【賢曰鬱林郡今郴州縣余按唐郴州無鬱林縣而唐之桂柳鬱邕象鷰潯南尹樂融賓等州皆漢鬱林郡地】外生鉅鹿都尉回【賢曰鉅鹿郡今邢州縣余按唐邢州固有鉅鹿縣而唐邢趙二州皆漢鉅鹿郡地】囘生南頓令欽【賢曰南頓縣屬汝南郡故城在今陳州項城縣西括地志陳州南頓縣古頓子國逼於陳南徙故曰南頓】欽娶湖陽樊重女【湖陽縣屬南陽郡宋白曰湖陽縣古蓼國地】生三男縯仲秀【縯音衍】兄弟早孤養於叔父良縯性剛毅慷慨有大節自莽簒漢常憤憤懷復社稷之慮不事家人居業傾身破產交結天下雄俊秀隆準日角【賢曰隆高也許負云鼻頭為準鄭玄尚書中候註云日角謂庭中骨起狀如日】性勤稼穡【賢曰種曰稼歛曰穡】縯常非笑之比於高祖兄仲【賢曰仲郃陽侯喜也能為產業高祖為太上皇夀曰始大人常以臣不能治產業不如仲力今其業所就孰與仲多】秀姊元為新野鄧晨妻秀嘗與晨俱過穰人蔡少公少公頗學圖讖言劉秀當為天子【少詩照翻讖楚譛翻】或曰是國師公劉秀乎秀戲曰何用知非僕邪坐者皆大笑晨心獨喜宛人李守好星歷讖記【宛於元翻好呼到翻】為莽宗卿師【賢曰平帝五年王莽攝政郡國置宗師以主宗室蓋時尊之故曰宗卿師也余按莽置宗師主漢宗室耳此宗卿師莽簒時所置也】嘗謂其子通曰劉氏當興李氏為輔及新市平林兵起南陽騷動通從弟軼謂通曰【從才用翻軼音逸又徒結翻】今四方擾亂漢當復興【復秩用翻又如字】南陽宗室獨劉伯升兄弟汎愛容衆可與謀大事【縯字伯升】通笑曰吾意也會秀賣穀於宛【宛於元翻】通遣軼往迎秀與相見因具言讖文事與相約結定計議通欲以立秋材官都試騎士日劫前隊大夫甄阜及屬正梁丘賜【莽改南陽曰前隊置大夫職如太守屬正職如都尉師古曰隊音遂甄之人翻】因以號令大衆使軼與秀歸舂陵舉兵以相應於是縯召諸豪桀計議曰王莽暴虐百姓分崩今枯旱連年兵革並起此亦天亡之時復高祖之業定萬世之秋也【言定天下傳之萬世此其時也】衆皆然之於是分遣親客於諸縣起兵縯自發舂陵子弟諸家子弟恐懼皆亡匿曰伯升殺我及見秀絳衣大冠【董巴輿服志曰大冠者武官冠之東觀記上時絳衣大冠將軍服也】皆驚曰謹厚者亦復為之乃稍自安凡得子弟七八千人部署賓客自稱柱天都部【賢曰柱天若天之柱也都部者都統其衆也】秀時年二十八李通未發事覺亡走父守及家屬坐死者六十四人縯使族人嘉招說新市平林兵【說輸芮翻】與其帥王鳳陳牧西擊長聚【帥所類翻聚才喻翻下同】進屠唐子鄉【賢曰多所誅殺曰屠唐子鄉有唐子山在今唐州湖陽縣西南】又殺湖陽尉軍中分財物不均衆恚恨欲反攻諸劉【恚於避翻】劉秀斂宗人所得物悉以與之衆乃悦進拔棘陽【賢曰棘陽縣名屬南陽郡在棘水之陽古謝國也故城在今唐州湖陽縣西北棘音紀力翻】李軼鄧晨皆將賓客來會【將即亮翻】 嚴尤陳茂破下江兵成丹王常張卬等收散卒入蔞谿略鍾龍間【賢曰蔞音力于翻盛弘之荆州記曰永陽縣北有石龍山在今安州應山縣東北又隨州隨縣東北有三鍾山】衆復振【復扶又翻下同】引軍與荆州牧戰于上唐【賢曰上唐鄉名故城在今隨州棗陽縣東北水經注上唐本蔡陽縣之上唐鄉春秋時唐國也】大破之 十一月有星孛于張【賢曰張南方宿續漢志曰張為周地晉書天文志張六星在天廟北孛蒲内翻】 劉縯欲進攻宛至小長安聚【賢曰續漢書淯陽縣有小長安聚故城在今鄧州南陽縣南杜佑曰南陽漢宛縣縣南三十七里有小長安】與甄阜梁丘賜戰時天密霧漢軍大敗秀單馬走遇女弟伯姫與共騎而犇前行復見姊元趣令上馬【騎奇計翻復扶又翻下同趣讀曰促上時掌翻】元以手揮曰行矣不能相救無為兩沒也會追兵至元及三女皆死縯弟仲及宗從死者數十人【從才用翻】縯復收會兵衆還保棘陽阜賜乘勝留輜重於藍鄉【賢曰北陽縣有藍鄉重直用翻】引精兵十萬南度潢淳【賢曰酈道元注水經曰諸水二湖流注合為黄水又南經棘陽縣之潢淳聚又謂之潢淳水在今唐州湖陽縣】臨沘水【水經注沘水出比陽縣東北大胡山南與澧水會謂之?水昔漢光武破甄阜梁丘賜於沘水西斬之於斯水也杜佑曰漢舞陽故城在唐州泌陽縣北有泌水在縣南光武破阜賜處】阻兩川間為營絶後橋示無還心新市平林見漢兵數敗【數所角翻】阜賜軍大至各欲解去縯甚患之會下江兵五十餘人至宜秋【賢曰宜秋聚名在沘陽縣余按續漢志南陽平氏縣有宜秋聚】縯即與秀及李通造其壁曰【造七到翻】願見下江一賢將議大事衆推王常縯見常說以合從之利【賢曰以利合曰從說輸芮翻下同從子容翻】常大悟曰王莽殘虐百姓思漢今劉氏復興【復扶又翻下同】即真主也誠思出身為用輔成大功縯曰如事成豈敢獨饗之哉遂與常深相結而去常還具為餘將成丹張卬言之【常與縯會餘二將在軍為于偽翻將即亮翻】丹卬負其衆曰大丈夫既起當各自為主何故受人制乎常乃徐曉說其將帥曰王莽苛酷積失百姓之心民之謳吟思漢非一日也故使吾屬因此得起夫民所怨者天所去也民所思者天所與也舉大事必當下順民心上合天意功乃可成若負強恃勇觸情恣欲雖得天下必復失之以秦項之勢尚至夷覆况今布衣相聚草澤以此行之滅亡之道也今南陽諸劉舉兵觀其來議者皆有深計大慮王公之才與之并合必成大功此天所以祐吾屬也下江諸將雖屈彊少識【屈彊梗戾貌屈音居勿翻彊音其兩翻少詩沼翻】然素敬常乃皆謝曰無王將軍吾屬幾陷於不義【幾居希翻】即引兵與漢軍新市平林合於是諸部齊心同力銳氣益壯縯大饗軍士設盟約休卒三日分為六部十二月晦潛師夜起襲取藍鄉盡獲其輜重【重直用翻】<br />
<br />
  資治通鑑卷三十八<br />
<br />
<史部,編年類,資治通鑑>  <br>
   </div> 

<script src="/search/ajaxskft.js"> </script>
 <div class="clear"></div>
<br>
<br>
 <!-- a.d-->

 <!--
<div class="info_share">
</div> 
-->
 <!--info_share--></div>   <!-- end info_content-->
  </div> <!-- end l-->

<div class="r">   <!--r-->



<div class="sidebar"  style="margin-bottom:2px;">

 
<div class="sidebar_title">工具类大全</div>
<div class="sidebar_info">
<strong><a href="http://www.guoxuedashi.com/lsditu/" target="_blank">历史地图</a></strong>  
<a href="http://www.880114.com/" target="_blank">英语宝典</a>  
<a href="http://www.guoxuedashi.com/13jing/" target="_blank">十三经检索</a> 
<br><strong><a href="http://www.guoxuedashi.com/gjtsjc/" target="_blank">古今图书集成</a></strong> 
<a href="http://www.guoxuedashi.com/duilian/" target="_blank">对联大全</a> <strong><a href="http://www.guoxuedashi.com/xiangxingzi/" target="_blank">象形文字典</a></strong> 

<br><a href="http://www.guoxuedashi.com/zixing/yanbian/">字形演变</a>  <strong><a href="http://www.guoxuemi.com/hafo/" target="_blank">哈佛燕京中文善本特藏</a></strong>
<br><strong><a href="http://www.guoxuedashi.com/csfz/" target="_blank">丛书&方志检索器</a></strong> <a href="http://www.guoxuedashi.com/yqjyy/" target="_blank">一切经音义</a>  

<br><strong><a href="http://www.guoxuedashi.com/jiapu/" target="_blank">家谱族谱查询</a></strong>  <strong><a href="http://shufa.guoxuedashi.com/sfzitie/" target="_blank">书法字帖欣赏</a></strong> 
<br>

</div>
</div>


<div class="sidebar" style="margin-bottom:0px;">

<font style="font-size:22px;line-height:32px">QQ交流群9:489193090</font>


<div class="sidebar_title">手机APP 扫描或点击</div>
<div class="sidebar_info">
<table>
<tr>
	<td width=160><a href="http://m.guoxuedashi.com/app/" target="_blank"><img src="/img/gxds-sj.png" width="140"  border="0" alt="国学大师手机版"></a></td>
	<td>
<a href="http://www.guoxuedashi.com/download/" target="_blank">app软件下载专区</a><br>
<a href="http://www.guoxuedashi.com/download/gxds.php" target="_blank">《国学大师》下载</a><br>
<a href="http://www.guoxuedashi.com/download/kxzd.php" target="_blank">《汉字宝典》下载</a><br>
<a href="http://www.guoxuedashi.com/download/scqbd.php" target="_blank">《诗词曲宝典》下载</a><br>
<a href="http://www.guoxuedashi.com/SiKuQuanShu/skqs.php" target="_blank">《四库全书》下载</a><br>
</td>
</tr>
</table>

</div>
</div>


<div class="sidebar2">
<center>


</center>
</div>

<div class="sidebar"  style="margin-bottom:2px;">
<div class="sidebar_title">网站使用教程</div>
<div class="sidebar_info">
<a href="http://www.guoxuedashi.com/help/gjsearch.php" target="_blank">如何在国学大师网下载古籍?</a><br>
<a href="http://www.guoxuedashi.com/zidian/bujian/bjjc.php" target="_blank">如何使用部件查字法快速查字?</a><br>
<a href="http://www.guoxuedashi.com/search/sjc.php" target="_blank">如何在指定的书籍中全文检索?</a><br>
<a href="http://www.guoxuedashi.com/search/skjc.php" target="_blank">如何找到一句话在《四库全书》哪一页?</a><br>
</div>
</div>


<div class="sidebar">
<div class="sidebar_title">热门书籍</div>
<div class="sidebar_info">
<a href="/so.php?sokey=%E8%B5%84%E6%B2%BB%E9%80%9A%E9%89%B4&kt=1">资治通鉴</a> <a href="/24shi/"><strong>二十四史</strong></a>&nbsp; <a href="/a2694/">野史</a>&nbsp; <a href="/SiKuQuanShu/"><strong>四库全书</strong></a>&nbsp;<a href="http://www.guoxuedashi.com/SiKuQuanShu/fanti/">繁体</a>
<br><a href="/so.php?sokey=%E7%BA%A2%E6%A5%BC%E6%A2%A6&kt=1">红楼梦</a> <a href="/a/1858x/">三国演义</a> <a href="/a/1038k/">水浒传</a> <a href="/a/1046t/">西游记</a> <a href="/a/1914o/">封神演义</a>
<br>
<a href="http://www.guoxuedashi.com/so.php?sokeygx=%E4%B8%87%E6%9C%89%E6%96%87%E5%BA%93&submit=&kt=1">万有文库</a> <a href="/a/780t/">古文观止</a> <a href="/a/1024l/">文心雕龙</a> <a href="/a/1704n/">全唐诗</a> <a href="/a/1705h/">全宋词</a>
<br><a href="http://www.guoxuedashi.com/so.php?sokeygx=%E7%99%BE%E8%A1%B2%E6%9C%AC%E4%BA%8C%E5%8D%81%E5%9B%9B%E5%8F%B2&submit=&kt=1"><strong>百衲本二十四史</strong></a>  <a href="http://www.guoxuedashi.com/so.php?sokeygx=%E5%8F%A4%E4%BB%8A%E5%9B%BE%E4%B9%A6%E9%9B%86%E6%88%90&submit=&kt=1"><strong>古今图书集成</strong></a>
<br>

<a href="http://www.guoxuedashi.com/so.php?sokeygx=%E4%B8%9B%E4%B9%A6%E9%9B%86%E6%88%90&submit=&kt=1">丛书集成</a> 
<a href="http://www.guoxuedashi.com/so.php?sokeygx=%E5%9B%9B%E9%83%A8%E4%B8%9B%E5%88%8A&submit=&kt=1"><strong>四部丛刊</strong></a>  
<a href="http://www.guoxuedashi.com/so.php?sokeygx=%E8%AF%B4%E6%96%87%E8%A7%A3%E5%AD%97&submit=&kt=1">說文解字</a> <a href="http://www.guoxuedashi.com/so.php?sokeygx=%E5%85%A8%E4%B8%8A%E5%8F%A4&submit=&kt=1">三国六朝文</a>
<br><a href="http://www.guoxuedashi.com/so.php?sokeytm=%E6%97%A5%E6%9C%AC%E5%86%85%E9%98%81%E6%96%87%E5%BA%93&submit=&kt=1"><strong>日本内阁文库</strong></a> <a href="http://www.guoxuedashi.com/so.php?sokeytm=%E5%9B%BD%E5%9B%BE%E6%96%B9%E5%BF%97%E5%90%88%E9%9B%86&ka=100&submit=">国图方志合集</a> <a href="http://www.guoxuedashi.com/so.php?sokeytm=%E5%90%84%E5%9C%B0%E6%96%B9%E5%BF%97&submit=&kt=1"><strong>各地方志</strong></a>

</div>
</div>


<div class="sidebar2">
<center>

</center>
</div>
<div class="sidebar greenbar">
<div class="sidebar_title green">四库全书</div>
<div class="sidebar_info">

《四库全书》是中国古代最大的丛书,编撰于乾隆年间,由纪昀等360多位高官、学者编撰,3800多人抄写,费时十三年编成。丛书分经、史、子、集四部,故名四库。共有3500多种书,7.9万卷,3.6万册,约8亿字,基本上囊括了古代所有图书,故称“全书”。<a href="http://www.guoxuedashi.com/SiKuQuanShu/">详细>>
</a>

</div> 
</div>

</div>  <!--end r-->

</div>
<!-- 内容区END --> 

<!-- 页脚开始 -->
<div class="shh">

</div>

<div class="w1180" style="margin-top:8px;">
<center><script src="http://www.guoxuedashi.com/img/plus.php?id=3"></script></center>
</div>
<div class="w1180 foot">
<a href="/b/thanks.php">特别致谢</a> | <a href="javascript:window.external.AddFavorite(document.location.href,document.title);">收藏本站</a> | <a href="#">欢迎投稿</a> | <a href="http://www.guoxuedashi.com/forum/">意见建议</a> | <a href="http://www.guoxuemi.com/">国学迷</a> | <a href="http://www.shuowen.net/">说文网</a><script language="javascript" type="text/javascript" src="https://js.users.51.la/17753172.js"></script><br />
  Copyright &copy; 国学大师 古典图书集成 All Rights Reserved.<br>
  
  <span style="font-size:14px">免责声明:本站非营利性站点,以方便网友为主,仅供学习研究。<br>内容由热心网友提供和网上收集,不保留版权。若侵犯了您的权益,来信即刪。scp168@qq.com</span>
  <br />
ICP证:<a href="http://www.beian.miit.gov.cn/" target="_blank">鲁ICP备19060063号</a></div>
<!-- 页脚END --> 
<script src="http://www.guoxuedashi.com/img/plus.php?id=22"></script>
<script src="http://www.guoxuedashi.com/img/tongji.js"></script>

</body>
</html>
