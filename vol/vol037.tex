<!DOCTYPE html PUBLIC "-//W3C//DTD XHTML 1.0 Transitional//EN" "http://www.w3.org/TR/xhtml1/DTD/xhtml1-transitional.dtd">
<html xmlns="http://www.w3.org/1999/xhtml">
<head>
<meta http-equiv="Content-Type" content="text/html; charset=utf-8" />
<meta http-equiv="X-UA-Compatible" content="IE=Edge,chrome=1">
<title>資治通鑒_38-資治通鑑卷三十七_38-資治通鑑卷三十七</title>
<meta name="Keywords" content="資治通鑒_38-資治通鑑卷三十七_38-資治通鑑卷三十七">
<meta name="Description" content="資治通鑒_38-資治通鑑卷三十七_38-資治通鑑卷三十七">
<meta http-equiv="Cache-Control" content="no-transform" />
<meta http-equiv="Cache-Control" content="no-siteapp" />
<link href="/img/style.css" rel="stylesheet" type="text/css" />
<script src="/img/m.js?2020"></script> 
</head>
<body>
 <div class="ClassNavi">
<a  href="/24shi/">二十四史</a> | <a href="/SiKuQuanShu/">四库全书</a> | <a href="http://www.guoxuedashi.com/gjtsjc/"><font  color="#FF0000">古今图书集成</font></a> | <a href="/renwu/">历史人物</a> | <a href="/ShuoWenJieZi/"><font  color="#FF0000">说文解字</a></font> | <a href="/chengyu/">成语词典</a> | <a  target="_blank"  href="http://www.guoxuedashi.com/jgwhj/"><font  color="#FF0000">甲骨文合集</font></a> | <a href="/yzjwjc/"><font  color="#FF0000">殷周金文集成</font></a> | <a href="/xiangxingzi/"><font color="#0000FF">象形字典</font></a> | <a href="/13jing/"><font  color="#FF0000">十三经索引</font></a> | <a href="/zixing/"><font  color="#FF0000">字体转换器</font></a> | <a href="/zidian/xz/"><font color="#0000FF">篆书识别</font></a> | <a href="/jinfanyi/">近义反义词</a> | <a href="/duilian/">对联大全</a> | <a href="/jiapu/"><font  color="#0000FF">家谱族谱查询</font></a> | <a href="http://www.guoxuemi.com/hafo/" target="_blank" ><font color="#FF0000">哈佛古籍</font></a> 
</div>

 <!-- 头部导航开始 -->
<div class="w1180 head clearfix">
  <div class="head_logo l"><a title="国学大师官网" href="http://www.guoxuedashi.com" target="_blank"></a></div>
  <div class="head_sr l">
  <div id="head1">
  
  <a href="http://www.guoxuedashi.com/zidian/bujian/" target="_blank" ><img src="http://www.guoxuedashi.com/img/top1.gif" width="88" height="60" border="0" title="部件查字,支持20万汉字"></a>


<a href="http://www.guoxuedashi.com/help/yingpan.php" target="_blank"><img src="http://www.guoxuedashi.com/img/top230.gif" width="600" height="62" border="0" ></a>


  </div>
  <div id="head3"><a href="javascript:" onClick="javascript:window.external.AddFavorite(window.location.href,document.title);">添加收藏</a>
  <br><a href="/help/setie.php">搜索引擎</a>
  <br><a href="/help/zanzhu.php">赞助本站</a></div>
  <div id="head2">
 <a href="http://www.guoxuemi.com/" target="_blank"><img src="http://www.guoxuedashi.com/img/guoxuemi.gif" width="95" height="62" border="0" style="margin-left:2px;" title="国学迷"></a>
  

  </div>
</div>
  <div class="clear"></div>
  <div class="head_nav">
  <p><a href="/">首页</a> | <a href="/ShuKu/">国学书库</a> | <a href="/guji/">影印古籍</a> | <a href="/shici/">诗词宝典</a> | <a   href="/SiKuQuanShu/gxjx.php">精选</a> <b>|</b> <a href="/zidian/">汉语字典</a> | <a href="/hydcd/">汉语词典</a> | <a href="http://www.guoxuedashi.com/zidian/bujian/"><font  color="#CC0066">部件查字</font></a> | <a href="http://www.sfds.cn/"><font  color="#CC0066">书法大师</font></a> | <a href="/jgwhj/">甲骨文</a> <b>|</b> <a href="/b/4/"><font  color="#CC0066">解密</font></a> | <a href="/renwu/">历史人物</a> | <a href="/diangu/">历史典故</a> | <a href="/xingshi/">姓氏</a> | <a href="/minzu/">民族</a> <b>|</b> <a href="/mz/"><font  color="#CC0066">世界名著</font></a> | <a href="/download/">软件下载</a>
</p>
<p><a href="/b/"><font  color="#CC0066">历史</font></a> | <a href="http://skqs.guoxuedashi.com/" target="_blank">四库全书</a> |  <a href="http://www.guoxuedashi.com/search/" target="_blank"><font  color="#CC0066">全文检索</font></a> | <a href="http://www.guoxuedashi.com/shumu/">古籍书目</a> | <a   href="/24shi/">正史</a> <b>|</b> <a href="/chengyu/">成语词典</a> | <a href="/kangxi/" title="康熙字典">康熙字典</a> | <a href="/ShuoWenJieZi/">说文解字</a> | <a href="/zixing/yanbian/">字形演变</a> | <a href="/yzjwjc/">金 文</a> <b>|</b>  <a href="/shijian/nian-hao/">年号</a> | <a href="/diming/">历史地名</a> | <a href="/shijian/">历史事件</a> | <a href="/guanzhi/">官职</a> | <a href="/lishi/">知识</a> <b>|</b> <a href="/zhongyi/">中医中药</a> | <a href="http://www.guoxuedashi.com/forum/">留言反馈</a>
</p>
  </div>
</div>
<!-- 头部导航END --> 
<!-- 内容区开始 --> 
<div class="w1180 clearfix">
  <div class="info l">
   
<div class="clearfix" style="background:#f5faff;">
<script src='http://www.guoxuedashi.com/img/headersou.js'></script>

</div>
  <div class="info_tree"><a href="http://www.guoxuedashi.com">首页</a> > <a href="/SiKuQuanShu/fanti/">四库全书</a>
 > <h1>资治通鉴</h1> <!--         下载:【右键另存为】即可 --></div>
  <div class="info_content zj clearfix">
  
<div class="info_txt clearfix" id="show">
<center style="font-size:24px;">38-資治通鑑卷三十七</center>
    資治通鑑卷三十七   宋 司馬光 撰<br />
<br />
  胡三省 音註<br />
<br />
  漢紀二十九【起屠維大荒落盡閼逢閹蔑鹿六年】<br />
<br />
  王莽中<br />
<br />
  始建國元年春正月朔【去年十二月莽改元以十二月為歲首逋鑑不書不與其改正朔也】莽帥公侯卿士奉皇太后璽韍【帥讀曰率璽斯氏翻師古曰韍謂璽之組音弗】上太皇太后順符命去漢號焉【上時掌翻去羌呂翻】初莽娶故丞相王訢孫宜春侯咸女為妻【師古曰王訢為丞相初封宜春侯傳爵至孫咸恩澤侯表宜春侯國於汝南】立以為皇后生四男宇獲前誅死安頗荒忽【師古曰荒音呼廣翻】乃以臨為皇太子安為新嘉辟【師古曰辟君也謂之辟者取為國君之義辟音壁】封宇子六人皆為公【千為功隆公夀為功明公吉為功成公宗為功崇公世為功昭公利為功著公】大赦天下莽乃策命孺子為定安公封以萬戶地方百里立漢祖宗之廟於其國與周後並行其正朔服色【此皆空言耳】以孝平皇后為定安太后讀策畢莽親執孺子手流涕歔欷【師古曰歔音虚欷音許氣翻又音希】曰昔周公攝位終得復子明辟今予獨迫皇天威命不得如意哀嘆良久中傅將孺子下殿北面而稱臣【漢諸侯王國有太傅中傅太傅秩二千石中傅則在宫中傅王者耳賢曰前書音義曰中傅宦者也】百僚陪位莫不感動又按金匱封拜輔臣【哀章所獻金匱圖金策書也】以太傅左輔王舜為太師封安新公大司徒平晏為太傅就新公少阿羲和劉秀為國師嘉新公廣漢梓潼哀章為國將美新公【梓潼縣時屬廣漢郡將即亮翻下同】是為四輔位上公太保後丞甄邯為大司馬承新公丕進侯王尋為大司徒章新公步兵將軍王邑為大司空隆新公是為三公太阿右拂大司空甄豐為更始將軍廣新公京兆王興為衛將軍奉新公輕車將軍孫建為立國將軍成新公京兆王盛為前將軍崇新公是為四將凡十一公王興者故城門令史【城門令史事城門校尉掌文書】王盛者賣餅【釋名餅併也溲麥使合并也蒸餅湯餅之屬隨形而名之】莽按符命求得此姓名十餘人兩人容貌應卜相【相息亮翻】徑從布衣登用以示神焉是日封拜卿大夫侍中尚書官凡數百人諸劉為郡守者皆徙為諫大夫改明光宫為定安館定安太后居之以大鴻臚府為定安公第皆置門衛使者監領【臚陵如翻監古䘖翻】勑阿乳母不得與嬰語常在四壁中【孟康曰令定安公居四壁中不得有所見漢官典職曰省中皆糊粉壁紫素界之畫古烈士釋名曰壁辟也辟禦風寒也】至於長大不能名六畜【長知兩翻六畜牛馬羊犬豕雞也養之曰畜用之曰牲畜者許救翻】後莽以女孫宇子妻之【莽長男宇之子則女孫也妻七細翻】莽策命羣司各以其職如典誥之文置大司馬司允大司徒司直大司空司若位皆孤卿【師古曰允信也若順也余按古之三孤位六卿爵秩同六卿曰孤卿】更名大司農曰羲和後更為納言【更工衡翻下同】大理曰作士太常曰秩宗大鴻臚曰典樂少府曰共工【師古曰共音恭】水衡都尉曰予虞【皆放唐虞建官也】與三公司卿分屬三公【司卿即司允司直司若】置二十七大夫八十一元士分主中都官諸職又更光禄勲等名為六監皆上卿【光禄勲曰司中太僕曰太御衛尉曰太衛執金吾曰奮武中尉曰軍正又置大贅官主乘輿服御物後又典兵】改郡太守曰大尹都尉曰大尉縣令長曰宰【守式又翻長知兩翻】長樂宫曰常樂室【樂音洛】長安曰常安其餘百官宫室郡縣盡易其名不可勝紀【勝音升】封王氏齊縗之屬為侯【齊音咨縗裳而緶其下縗倉囘翻】大功為伯小功為子緦麻為男其女皆為任【師古曰任充也男服之義男亦任也音壬】男以睦女以隆為號焉【師古曰睦隆皆其受封邑之號取嘉名也】又曰漢氏諸侯或稱王至於四夷亦如之違於古典繆於一統【王大一統王者有天下之號也諸侯及四夷稱之非古也繆戾也】其定諸侯王之號皆稱公及四夷僭號稱王者皆更為侯於是漢諸侯王三十二人皆降為公王子侯者百八十一人皆降為子其後皆奪爵焉 【考異曰諸侯王表皆云莽簒位貶為公明年廢王子侯表但云絶或云免皆在今年按明年立國將軍建奏諸劉為諸侯者以戶多少就五等之差亦不云奪爵也後漢城陽王祉傳云劉氏侯者皆降為子後奪爵不知奪在幾年】 莽又封黄帝少昊顓頊帝嚳堯舜夏商周及臯陶伊尹之後皆為公侯使各奉其祭祀【姚恂為初睦侯奉黄帝後梁護為修遠伯奉少昊後皇孫功隆公千奉帝嚳後劉歆為祁烈伯奉顓頊後國師劉歆子疊為伊休侯奉堯後媯昌為始睦侯奉虞帝後山遵為褒謀侯奉臯陶後伊玄為褒衡子奉伊尹後周後衛公姫黨更封為章平公殷後宋公孔弘更封為章昭侯夏後遼西姒豐封為章功侯】 莽因漢承平之業府庫百官之富百蠻賓服天下晏然莽一朝有之其心意未滿【朱滿未饜足也】陿小漢家制度欲更為疏闊【欲變改制度以從古也陿與狹同更工衡翻】乃自謂黄帝虞舜之後至齊王建孫濟北王安失國齊人謂之王家因以為氏故以黄帝為初祖虞帝為始祖追尊陳胡公為陳胡王田敬仲為田敬王濟北王安為濟北愍王【以黄帝之後分為有虞氏有虞之後封於陳田敬仲自陳奔齊為田氏田安之後為王家故也濟子禮翻】立祖廟五親廟四天下姚媯陳田王五姓皆為宗室世世復無所與【復方目翻與讀曰豫下其復同】封陳崇田豐為侯以奉胡王敬王後天下牧守皆以前有翟義趙朋等作亂【事見上卷居攝元年守式又翻下同】領州郡懷忠孝封牧為男守為附城以漢高廟為文祖廟【師古曰欲法舜受終于文祖】漢氏園寢廟在京師者勿罷祠薦如故諸劉勿解其復【復方目翻】各終厥身州牧數存問【數所角翻】勿令有侵寃 莽以劉之為字卯金刀也詔正月剛卯金刀之利皆不得行【服䖍曰剛卯以正月卯日作佩之長三寸廣一寸四方或用玉或用金或用桃著革帶佩之今有玉在者銘其一而曰正月剛卯金刀莽所鑄之錢也晉灼曰剛卯長一寸廣五分四方當中央從穿作孔以采絲茸其底如冠纓頭蕤刻其上面作两行書文曰正月剛卯既央靈殳四方青赤白黄四色是當帝令祝融以教夔龍庶疫剛癉莫我敢當其一銘曰疾日嚴卯帝令夔化順爾固伏化兹靈殳既正既直既觚既方庶疫剛癉莫我敢當師古曰今往往有土中得玉剛卯者案大小及文服說是也莽以劉字上有卯下有金旁有刀故禁剛卯及金刀也余按劉字上本從丣莽以丣字近卯故云爾】乃罷錯刀契刀【孔頴逹曰古文有刀刀有二種一是契刀一是錯刀契刀直五百錯刀直一千契刀無鏤而錯刀用金鏤之刀形如錢而邊作刀字形也故世猶呼錢為錢刀】及五銖錢更作小錢【更工衡翻】徑六分重一銖文曰小錢直一與前大錢五十者為二品並行欲防民盜鑄乃禁不得挾銅炭夏四月徐鄉侯劉快結黨數千人起兵於其國【師古曰快膠東恭王子也而王子侯表作炔從火與此不同疑表誤也】快兄殷故漢膠東王時為扶崇公快舉兵攻即墨【即墨膠東國都殷膠東康王寄玄孫之子也】殷閉城門自繫獄吏民距快快敗走至長廣死【地理志長廣縣屬琅邪郡】莽赦殷益其國滿萬戶地方百里 莽曰古者一夫田百畒什一而稅【孟子曰周人百畒而徹此言周制也】則國給民富而頌聲作秦壞聖制廢井田【見二卷周顯王十九年壞音怪】是以兼并起貪鄙生彊者規田以千數弱者曾無立錐之居又置奴婢之市與牛馬同闌【師古曰闌謂遮闌之若牛馬闌圈也】制於民臣顓斷其命繆於天地之性人為貴之義【孝經孔子曰天地之性人為貴斷丁亂翻】減輕田租三十而稅一常有更賦罷癃咸出【師古曰更音工衡翻罷讀曰疲癃音隆晉灼曰雖老病者皆復出口筭】而豪民侵陵分田劫假【師古曰分田謂貧者無田而取富人田耕種共分其所收也假亦謂貧人賃富人之田也劫者富人劫奪其稅侵欺之也】厥名三十稅一實什稅五也故富者犬馬餘菽粟驕而為邪貧者不厭糟糠窮而為姦俱陷于辜刑用不錯【師古曰錯置也音千故翻】今更名天下田曰王田奴婢曰私屬【更工衡翻下同】皆不得賣買其男口不盈八而田過一井者分餘田予九族鄰里鄉黨【予讀曰與】故無田今當受田者如制度敢有非井田聖制無灋惑衆者投諸四裔以禦魑魅【師古曰魑山神也魅老物精也魑音螭魅音媚】如皇始祖考虞帝故事【舜投四凶於四裔以禦魑魅】 秋遣五威將王奇等十二人【五威將分左右前後中帥衣冠車服駕馬各如其方面色數將即亮翻帥所類翻下同】班符命四十二篇於天下德祥五事符命二十五福應十二五威將奉符命齎印綬王侯以下及吏官名更者【師古曰更改也】外及匈奴西域徼外蠻夷【徼古弔翻下同】皆即授新室印綬因收故漢印綬大赦天下五威將乘乾文車【鄭氏曰畫天文於車也】駕坤六馬【鄭氏曰坤為牝馬六地數】背負鷩鳥之毛服飾甚偉【師古曰鷩烏雉屬即鵕䴊也今俗呼云山雞非也鷩音鼈】每一將各置五帥將持節帥持幢【帥所類翻幢傳江翻旛也】其東出者至玄菟樂浪高句驪夫餘【菟音塗樂浪音洛琅陸德明曰句俱付翻又音駒驪力支翻師古曰夫音扶范曄曰武帝滅朝鮮開高句驪為縣使屬玄菟其人有五部在遼東之東千里夫餘在玄菟北千里東明之後也高句驪朱蒙之後以高為氏】南出者隃徼外歷益州改句町王為侯【徼外邊徼之外益州武帝所置益州郡也昭帝時姑繒葉榆夷反句町侯亡波擊反者有功立為王隃與踰同徼工釣翻句町音劬挺】西出至西域盡改其王為侯北出至匈奴庭授單于印改漢印文去璽言章【印符也信也亦因也封物相因付漢官儀曰諸侯王黄金槖駝鈕文曰璽列侯黄金龜鈕文曰章御史大夫金印中二千石銀印龜鈕文曰章千石至四百石皆銅印文曰印為莽以更印綬撓亂四夷張本去羌呂翻璽斯氏翻】 冬靁桐華以統睦侯陳崇為司命主司察上公以下又以說符侯崔發等為中城四關將軍主十二城門及繞霤羊頭肴黽汧隴之固【中城將軍主十二城門四關將軍分主繞霤羊頭肴黽汧隴四處三輔黄圖長安城東出南頭第一門曰霸城門莽改曰仁夀門第二門曰清明門莽曰宣德門東出北頭第一門曰宣平門莽曰春王門南出東頭第一門曰覆盎城門莽曰更清門第二門曰安門莽曰光禮門第三門曰便門莽曰信平門西出南頭第一門曰章城門莽曰萬秋門第二門曰直城門莽曰正道門第三門曰雍門莽曰章儀門北出東頭第一門曰洛門第二門曰厨城門莽曰建子城門第三門曰横門服䖍曰繞霤隘險之道師古曰謂之繞霤者言四面阸塞其道屈曲谿谷之水囘繞而霤也其處即今之商州界七盤十二繞是也霤音力救翻羊頭山名在上黨長子縣肴肴山也黽黽池也皆在陜縣之東也汧扶風汧縣有吳山汧水之阻隴謂隴坻也汧隴相連黽彌充翻汧口堅翻坻丁禮翻】皆以五威冠其號【冠古玩翻】 又遣諫大夫五十人分鑄錢於郡國 是歲真定常山大雨雹【雨于其翻】<br />
<br />
  二年春二月赦天下 五威將帥七十二人還奏事漢諸侯王為公者悉上璽綬為民【上時掌翻】無違命者獨故廣陽王嘉以獻符命【廣陽王嘉燕王旦之玄孫】魯王閔以獻神書中山王成都以獻書言莽德皆封列侯<br />
<br />
  班固論曰昔周封國八百同姓五十有餘所以親親賢賢關諸盛衰深根固本為不可拔者也故盛則周召相其治致刑錯【相息亮翻治直吏翻錯于故翻】衰則五伯扶其弱與共守【師古曰伯讀曰霸此五霸謂齊桓宋襄秦穆晉文吳子夫差也】天下謂之共主【如淳曰雖至微弱猶共以之為主】彊大弗之敢傾【師古曰言諸侯雖彊大不敢傾滅周也】歷載八百餘年數極德盡降為庶人用天年終【謂赧王也】秦訕笑三代竊自號為皇帝而子弟為匹夫内無骨肉本根之輔外無尺土藩翼之衛陳吳奮其白梃【應劭曰白梃大杖也孟子曰可使制梃以撻秦楚是也師古曰梃音徒鼎翻】劉項隨而斃之故曰周過其歷秦不及期國埶然也【應劭曰武王克商卜世三十卜年八百今乃三十六世八百六十七年此謂過其歷者也秦以諡法少恐後世相襲自稱始皇帝子曰二世欲以一迄萬今至子而亡此之謂不及期也】漢興之初懲戒亡秦孤立之敗於是尊王子弟【王于况翻】大啓九國自鴈門以東盡遼陽為燕代【師古曰遼陽遼水之陽也】常山以南大行左轉度河濟漸于海為齊趙【師古曰太行山名也左轉亦謂自太行而東也漸入也一曰浸也濟子禮翻漸音子廉翻亦讀如字】穀泗以往奄有龜蒙為梁楚【晉灼曰水經云泗水出魯國卞縣臣瓚曰穀在彭城泗之下流為穀水師古曰奄覆也龜蒙二山名】東帶江湖薄會稽為荆吳【文頴曰即今吳也高祖六年為荆國十六年更名吳師古曰荆吳同是一國薄伯各翻會工外翻】北界淮瀕略廬衡為淮南【師古曰瀕水厓也盧衡二山名】波漢之陽亘九嶷為長沙【鄭氏曰波音陂澤之陂孟康曰亘竟也音古贈翻師古曰波漢之陽者循漢水而往也水北曰陽波音彼皮翻又音彼義翻九嶷山名有九峰在零陵營道嶷音疑】諸侯比境周匝三垂外接胡越【師古曰比謂相次也三垂謂東北南也比音頻寐翻接連也】天子自有三河東郡潁川南陽自江陵以西至巴蜀北自雲中至隴西與京師内史凡十五郡公主列侯頗邑其中【師古曰十五郡中又往往有列侯公主之邑】而藩國大者夸州兼郡連城數十宫室百官同制京師可謂矯枉過其正矣【師古曰夸音跨枉曲也正曲曰矯言矯秦孤立之敗而大封子弟過於彊盛以至失其中也】雖然高祖創業日不暇給孝惠享國又淺高后女主攝位而海内晏如亡狂狡之憂卒折諸呂之難【亡古無字通卒子恤翻難乃旦翻下同】成太宗之業者亦賴之於諸侯也然諸侯原本以大末流濫以致溢小者淫荒越灋大者暌孤横逆以害身喪國【師古曰暌孤乖剌之意暌音工攜翻横戶孟翻喪息浪翻】故文帝分齊趙【事見十四卷文帝二年又見十六年】景帝削吳楚【事見十六卷景帝三年】武帝下推恩之令而藩國自析【事見十八卷武帝元朔二年】自此以來齊分為七【師古曰謂齊城陽濟北濟南甾川膠西膠東也】趙分為六【謂趙平原真定中山廣川河間也】梁分為五【師古曰謂梁濟川濟東山陽濟隂也】淮南分為三【師古曰謂淮南衡山廬江】皇子始立者大國不過十餘城長沙燕代雖有舊名皆亡南北邊矣【如淳曰長沙之南更置郡燕代以北更置緣邊郡其所有饒利兵馬器械三國皆失之也亡古無字通下同】景遭七國之難抑損諸侯減黜其官【師古曰謂改丞相曰相省御史大夫廷尉少府宗正博士損大夫謁者諸官長丞員等也】武有衡山淮南之謀【事見十九卷武帝元朔五年及元狩三年】作左官之律【服䖍曰仕於諸侯為左官絶不使得仕於王朝也應劭曰人道尚右今舍天子而仕諸侯故謂之左官也師古曰左官猶言左道也蓋僻左不正應說是也漢時依上古灋朝廷之列以右為尊故謂降秩者為左遷仕諸侯者為左官】設附益之灋【張晏曰律鄭氏說封諸侯過限曰附益或曰阿媚王侯有重法也師古曰附益者蓋取孔子云求也為之聚歛而附益之之義也皆背正法而厚於私家也】諸侯惟得衣食稅租不與政事【與讀曰預】至於哀平之際皆繼體苗裔親屬踈遠【師古曰言非始封之君皆其後裔也故於天子益疎遠矣】生於帷牆之中不為士民所尊埶與富室亡異而本朝短祚國統三絶【師古曰謂成哀平皆早崩又無繼嗣亡古無字通】是故王莽知漢中外殫微【師古曰殫盡也音單】本末俱弱無所忌憚生其姦心因母后之權假伊周之稱顓作威福廟堂之上不降階序而運天下【師古曰序謂東西廂】詐謀既成遂據南面之尊分遣五威之吏馳傳天下【傳知戀翻】班行符命漢諸侯王厥角稽首【應劭曰厥者頓也角者額角也稽首首至地也言王莽漸潰威福日久亦值漢之單弱王侯見莽簒弑莫敢怨望皆厥角稽首至地而上其璽綬也晉灼曰厥猶竪也叩頭則額角竪師古曰應說是也】奉上璽韍惟恐在後或乃稱美頌德以求容媚豈不哀哉<br />
<br />
  國師公劉秀言周有泉府之官收不售與欲得【師古曰言賣不售者官收取之無而欲得者官出與之】即易所謂理財正辭禁民為非者也【易下繫辭曰理財正辭禁民為非曰義】莽乃下詔曰周禮有賖貸【師古曰周禮泉府之職曰凡賖者祭祀無過旬日喪紀無過三月凡人之貸者與其有司辨而授之以國服為之息謂以祭祀喪紀故從官賖買物不過旬日及三月而償之其從官貸物者以共其所屬吏定價而後與之各以其國服之稅而輸息謂若受園㕓之田而貸萬錢者一朞之月出息五百貸音土戴翻】樂語有五均【鄧展曰樂語樂元語河間獻王所傳道五均事臣瓚曰其文云天子取諸侯之上以立五均則市無二價四民常均彊者不得困弱富者不得要貧則公家有餘恩及小民矣】傳記各有筦焉【筦古緩翻】今開賖貸張五均設諸筦者所以齊衆庶抑并兼也遂於長安及洛陽邯鄲臨菑宛成都立五均司市錢府官司市常以四時仲月定物上中下之賈【邯鄲音寒丹宛於元翻師古曰賈讀曰價下同】各為其市平民賣五穀布帛絲綿之物不售者均官考檢厥實用其本賈取之物貴過平一錢則以平賈賣與民賤減平者聽民自相與市又民有乏絶欲賖貸者錢府予之【予讀曰與】每月百錢收息三錢又以周官稅民凡田不耕為不殖出三夫之稅城郭中宅不樹藝者為不毛【師古曰樹藝謂種樹果木及菜蔬】出三夫之布民浮游無事出夫布一疋其不能出布者宂作縣官衣食之【師古曰宂散也音人勇翻衣音於既翻食讀曰飤】諸取金銀連錫鳥獸魚鱉於山林水澤【孟康曰連錫之别名也李奇曰鉛錫應劭曰連似銅師古曰孟李二說皆非也許慎曰鏈銅屬也一曰丱也鏈抽延翻又陵延翻】及畜牧者【畜許六翻】嬪婦桑蠶織紝紡績補縫【師古曰機縷曰紝音人禁翻】王匠醫巫卜祝及他方技商販賈人皆各自占所為於其所之【技渠綺翻賈音古占之瞻翻下同之往也】縣官除其本計其利十分之而以其一為貢敢不自占自占不以實者盡沒入所采取而作縣官一歲羲和魯匡復奏請榷酒酤【復扶又翻榷古岳翻酤音故】莽從之又禁民不得挾弩鎧犯者徙西海 初莽既班四條於匈奴【四條見三十五卷平帝元始二年】後護烏桓使者告烏桓民毋得復與匈奴皮布稅匈奴遣使者責稅【護烏桓使者即護烏桓校尉范曄曰烏桓自為昌頓所破常臣伏匈奴歲輸牛馬羊皮過時不具輒沒其妻子武帝遣霍去病擊破匈奴左地因徙烏桓於上谷漁陽右北平遼西遼東五郡塞外後置護烏桓校尉秩二千石擁節監領之師古曰故時常稅是以求之】收烏桓酋豪縛倒懸之酋豪兄弟怒共殺匈奴使【酋慈由翻】單于聞之發左賢王兵入烏桓攻擊之頗殺人民婦女弱小且千人去置左地【讀曰驅】告烏桓曰持馬畜皮布來贖之烏桓持財畜往贖匈奴受留不遣【師古曰受其財畜而留人不遣】及五威將王駿等六人至匈奴【六人一將五帥也】重遺單于金帛【遺于季翻下同】諭曉以受命代漢狀因易單于故印故印文曰匈奴單于璽莽更曰新匈奴單于章【師古曰新者莽自係其國號更工衡翻】將率既至【率讀與帥同所類翻】授單于印韍【韍音弗】詔令上故印綬【上時掌翻下同綬音受】單于再拜受詔譯前欲解取故印韍【譯通中國之語於匈奴者也】單于舉掖授之左姑夕侯蘇從旁謂單于曰【蘇者姑夕侯之名】未見新印文宜且勿與單于止不肯與請使者坐穹廬單于欲前為夀【奉酒為使者夀】五威將曰故印紱當以時上單于曰諾復舉掖授譯【復扶又翻下同】蘇復曰未見印文且勿與單于曰印文何由變更【更工衡翻】遂解故印紱奉上將帥受著新紱【著陟略翻】不解視印飲食至夜乃罷右帥陳饒謂諸將帥曰曏者姑夕侯疑印文幾令單于不與人【幾居希翻】如令視印見其變改必求故印此非辭說所能距也既得而復失之辱命莫大焉不如椎破故印以絶禍根將帥猶與莫有應者【師古曰與讀曰豫】饒燕士果悍【師古曰果决也悍勇也悍音胡幹翻又下罕翻】即引斧椎壞之【壞音怪下同】明日單于果遣右骨都侯當白將帥曰漢單于印言璽不言章又無漢字諸王己下乃有漢言章【言諸王已下印文有漢字最下文言章字】今去璽加新【言去璽字為章字又加新字也去羌呂翻】與臣下無别【言無所别異也别彼列翻】願得故印將帥示以故印謂曰新室順天制作故印隨將帥所自為破壞單于宜承天命奉新室之制當還白單于知己無可柰何又多得賂遺即遣弟右賢王輿奉馬牛隨將帥入謝因上書求故印將帥還左犂汙王咸所居地見烏桓民多以問咸咸具言狀【其言前所以略鳥桓民之狀】將帥曰前封四條不得受烏桓降者亟還之【降戶江翻】咸曰請密與單于相聞得語歸之【謂得單于遣歸之語然後歸之也】單于使咸報曰當從塞内還之邪從塞外還之邪將帥不敢顓决以聞詔報從塞外還之莽悉封五威將為子帥為男獨陳饒以破璽之功封威德子【饒帥也以功為子】單于始用夏侯藩求地有拒漢語【見三十二卷成帝綏和元年】後以求稅烏桓不得因寇掠其人民釁由是生重以印文改易故怨恨【重直用翻】乃遣右大且渠蒲呼盧訾等十餘人【且子余翻訾子斯翻】將兵衆萬騎以護送烏桓為名勒兵朔方塞下【師古曰陽云護送烏桓人實來為寇】朔方太守以聞莽以廣新公甄豐為右伯【莽以符命分陜立二伯豐為右伯平晏為左伯】當出西域車師後王須置離聞之憚於供給煩費謀亡入匈奴都護但欽召置離斬之【但姓欽名】置離兄輔國侯狐蘭支將置離衆二千餘人亡降匈奴【車師國有輔國侯猶相也擊胡侯猶將也將即亮翻降戶江翻】單于受之遣兵與狐蘭支共入寇擊車師殺後城長【後城即車師後王城也長知兩翻】傷都護司焉及狐蘭兵復還入匈奴【匈奴兵既殺傷漢吏復與狐蘭支兵還入匈奴也】時戊巳校尉刁護病【姓譜刁齊大夫豎刁之後余按豎刁寺人安得有後史記貨殖傳有刁間校戶校翻】史陳良終帶司馬丞韓玄右曲候任商【史校尉之史也司馬丞司馬之丞也右曲侯軍分左右部部下有曲曲有候】相與謀曰西域諸國頗背叛【背蒲妹翻】匈奴大侵要死可殺校尉帥人衆降匈奴【如淳曰言匈奴來侵會當死耳可降匈奴也師古曰要音一遥翻言要之必死也帥讀曰率】遂殺護及其子男昆弟盡脅略戊巳校尉吏士男女二千餘人入匈奴單于號良帶曰烏賁都尉【師古曰賁音奔 考異曰匈奴傳云烏桓都將軍西域傳云烏賁都尉今從之】冬十一月立國將軍孫建奏九月辛巳陳良終帶自稱廢漢大將軍亡入匈奴【廢漢言漢氏已廢滅也孫建之言云爾】又今月癸酉不知何一男子遮臣建車前自稱漢氏劉子輿成帝下妻子也【師古曰下妻猶言小妻】劉氏當復趣空宫【師古曰復音扶福翻趣讀曰促】收繫男子即常安姓武字仲【莽改長安曰常安】皆逆天違命大逆無道漢氏宗廟不當在常安城中及諸劉當與漢俱廢陛下至仁久未定前故安衆侯劉崇等更聚衆謀反【劉崇事見上卷居攝元年更反謂崇敗後劉信劉快又起兵師古曰更音工衡翻】今狂狡之虜復依託亡漢【復扶又翻】至犯夷滅連未止者此聖恩不蚤絶其萌芽故也臣請漢氏諸廟在京師者皆罷諸劉為吏者皆待除於家【師古曰罷黜其職各使退歸而言在家待遷除】莽曰可嘉新公國師以符命為予四輔明德侯劉龔率禮侯劉嘉等【考異曰燕王旦傳廣陽王嘉封扶美侯莽傳云率禮侯劉嘉未知其改封或别一人也 余按率禮侯劉嘉】<br />
<br />
  【安衆侯劉崇之族父也事見上卷居攝元年】凡三十二人皆知天命或獻天符或貢昌言或捕告反虜厥功茂焉諸劉與三十二人同宗共祖者勿罷賜姓曰王唯國師公以女配莽子故不賜姓【國師公秀女愔配莽之子臨】定安公太后自劉氏之廢常稱疾不朝會【朝直遥翻】時年未二十莽敬憚傷哀欲嫁之乃更號曰黄皇室主【師古曰莽自謂土德故云黄皇室主若漢之稱公主余謂室主若言未嫁在室者也更工衡翻下同】欲絶之於漢令孫建世子盛飾將醫往問疾后大怒鞭笞其傍侍御因發病不肯起莽遂不復彊也【漢孝平王后周天元楊后猶有婦人内夫家外父母家之意然夷考二后本末天元楊后不逮孝平后也復扶又翻彊其兩翻】 十二月靁 莽恃府庫之富欲立威匈奴乃更名匈奴單于曰降奴服于【降戶江翻】下詔遣立國將軍孫建等率十二將分道並出五威將軍苖訢【姓譜引風俗通曰楚大夫伯棼之後賁皇奔晉食采於苗因而氏焉訢音欣】虎賁將軍王况出五原厭難將軍陳欽震狄將軍王廵出雲中【師古曰厭音一涉翻難乃旦翻】振武將軍王嘉平狄將軍王萌出代郡相威將軍李棽鎮遠將軍李翁出西河【相息亮翻師古曰棽音所林翻】誅貉將軍楊俊討濊將軍嚴尤出漁陽【貉音陌莫百翻濊音穢】奮武將軍王駿定胡將軍王晏出張掖及偏裨以下百八十人募天下囚徒丁男甲卒三十萬人轉輸衣裘兵器糧食自負海江淮至北邊使者馳傳督趣以軍興灋從事【言事誅斬也傳知戀翻趣讀曰促】先至者屯邊郡須畢具乃同時出窮追匈奴内之丁令【師占曰逐之遣入丁零也令音零】分其國土人民以為十五立呼韓邪子孫十五人皆為單于 莽以錢幣訖不行復下書曰寶貨皆重則小用不給皆輕則僦載煩費【復扶又翻下同師古曰僦□也一曰貨也音子就翻】輕重大小各有差品則用便而民樂【樂音洛】於是更作金銀龜貝錢布之品名曰寶貨【更工衡翻】錢貨六品【小錢徑六分重一銖名曰小錢直一次七分三銖曰么錢二十次八分五銖曰幼錢二十次九分七銖曰中錢三十次一寸九銖曰壯錢四十因前大錢五十為六品師古曰么音一遥翻】金貨一品【黄金一斤直錢萬是為一品】銀貨二品【朱提銀重八兩為一流直一千五百八十他銀一流直千是為二品師古曰朱音殊提音土支翻】龜貨四品【元龜岠冉長尺二寸直二千六百一十公龜九寸直五百侯龜七寸以上直三百子龜五寸以上直百是為四品孟康曰冉龜甲緣也岠至也度背兩邊緣尺二寸也元者大也】貝貨五品【大貝四寸八分以上二枚為一朋直二百一十六壯貝三寸六分以上一朋直五十幺貝二寸四分以上一明直三十小貝寸二分以上一朋直十不盈寸二分漏度不得為朋率枚直錢三是為五品貝紫貝也】布貨十品【大布次布弟布壯布中布差布厚布幼布么布小布小布長寸五分重十五銖文曰小布一百自小布以上各相長一分相重一銖文各為其布名直各加一百上至大布長二寸四分重一兩而直千錢矣師古曰市亦錢耳謂之布者言其分布流行也】凡寶貨五物六名二十八品鑄作錢布皆用銅殽以連錫百姓潰亂【潰漢書作憒】其貨不行莽知民愁乃但行小錢直一與大錢五十二品並行龜貝布屬且侵盜鑄錢者不可禁乃重其灋一家鑄錢五家坐之沒入為奴婢吏民出入持錢以副符傳不持者厨傳勿舍關津苛留【師古曰舊法行者持符傳即不稽留今更令持錢與符相副乃得過也厨行道飲食處傳置驛之舍也苛問也傳音張戀翻苛音何】公卿皆持以入宫殿門欲以重而行之是時百姓便安漢五銖錢以莽錢大小兩行難知又數變改不信【數所角翻】皆私以五銖錢市買訛言大錢當罷莫肯挾莽患之復下書諸挾五銖錢言大錢當罷者比非井田制投四裔及坐賣買田宅奴婢鑄錢自諸侯卿大夫至于庶民抵罪者不可勝數【勝音升】於是農商失業食貨俱廢民人至涕泣於市道 莽之謀簒也吏民爭為符命皆得封侯其不為者相戲曰獨無天帝除書乎司命陳崇白莽曰此開姦臣作福之路而亂天命宜絶其原莽亦厭之遂使尚書大夫趙並驗治【莽分九卿每一卿置三大夫尚書大夫蓋屬共工也冶直之翻】非五威將帥所班皆下獄【下遐稼翻】初甄豐劉秀王舜為莽腹心唱導在位褒揚功德安漢宰衡之號及封莽母兩子兄子【事並見上卷】皆豐等所共謀而豐舜秀亦受其賜並富貴矣非復欲令莽居攝也【復扶又翻下同】居攝之萌出於泉陵侯劉慶前煇光謝嚻【事見上卷元始五年周申伯邑於謝其後子孫以謝為氏】長安令田終術【事不見於史】莽羽翼己成意欲稱攝豐等承順其意莽輒復封舜秀豐等子孫以報之豐等爵位已盛心意既滿又實畏漢宗室天下豪桀而疏遠欲進者並作符命莽遂據以即真【謂哀章等也疏與踈同】舜秀内懼而已豐素剛彊莽覺其不說故託符命文徙豐為更始將軍【說讀曰悦更工衡翻】與賣餅兒王盛同列豐父子默默時子尋為侍中兆兆大尹茂德侯即作符命新室當分陜立二伯以豐為右伯太傳平晏為左伯如周召故事【師古曰自陜以東周公主之自陜以西召公主之陜即今陜州是其地也伯長也陜音式冉翻召讀曰邵】莽即從之拜豐為右伯當述職西出未行尋復作符命言故漢氏平帝后黄皇室主為尋之妻莽以詐立心疑大臣怨謗欲震威以懼下因是發怒曰黄皇室主天下母此何謂也收捕尋尋亡豐自殺尋隨方士入華山【華山在華隂縣南華戶化翻】歲餘捕得辭連國師公秀子隆威侯棻【師古曰棻亦分字也音扶云翻】棻弟右曹長水校尉伐虜侯泳大司空邑弟左關將軍掌威侯奇【莽置左關將軍主函谷】及秀門人侍中騎都尉丁隆等牽引公卿黨親列侯以下死者數百人乃流棻于幽州放尋于三危殛隆于羽山【師古曰放舜之罰共工等也】皆驛車傳致其屍云【傳知戀翻】 是歲莽始興神仙事以方士蘇樂言起八風臺臺成萬金【師古曰費直萬金也】又種五粱禾於殿中【師古曰五色禾也谷永所云耕耘五德者也晉灼曰翼氏風角五德東方甲南方丙西方庚北方壬中央戊種五色禾於此地而耕耘也汜勝之曰粱是秫粟】先以寶玉漬種【煮鶴髓瑇瑁犀玉二十餘物取汁以漬種種章勇翻】計粟斛成一金<br />
<br />
  三年遣田禾將軍趙並發戍卒屯田五原北假以助軍糧 莽遣中郎將藺苞副校尉戴級將兵萬騎多齎珍寶至雲中塞下招誘呼韓邪諸子欲以次拜為十五單于苞級使譯出塞誘呼左犂汙王咸咸子登助三人至【誘音酉】至則脅拜咸為孝單于助為順單于皆厚加賞賜傳送助登長安【傳張戀翻】莽封苞為宣威公拜為虎牙將軍封級為揚威公拜為虎賁將軍單于聞之怒曰先單于受漢宣帝恩不可負也【先單于諸呼韓邪單于】今天子非宣帝子孫何以得立遣左骨都侯右伊秩訾王呼盧訾及左賢王樂將兵入雲中益夀塞大殺吏民【訾子斯翻】是後單于歷告左右部都尉【即左右大都尉也】諸邊王【諸王庭近漢邊者】入塞寇盜大輩萬餘中輩數千少者數百殺鴈門朔方太守都尉略吏民畜產不可勝數【勝音升】緣邊虚耗是時諸將在邊以大衆未集未敢出擊匈奴討濊將軍嚴尤諫曰【濊音穢】臣聞匈奴為害所從來久矣未聞上世有必征之者也後世三家周秦漢征之然皆未有得上策者也周得中策漢得下策秦無策焉當周宣王時獫狁内侵至于涇陽【地理志安定郡有涇陽縣賢曰涇陽故城在今原州平凉縣南】命將征之盡境而還其視戎狄之侵譬猶蟁蝱之而已【蟁古蚊字蝱音盲與驅同】故天下稱明是為中策漢武帝選將練兵約齎輕糧【師古日約少也少齎衣裝】深入遠戍雖有克獲之功胡輒報之兵連禍結三十餘年中國罷耗匈奴亦創艾【師古曰罷讀曰疲耗損也創音初向翻艾讀曰乂】而天下稱武是為下策秦始皇不忍小恥而輕民力築長城之固延袤萬里【師古曰袤長也音茂】轉輸之行起於負海疆境既完中國内竭以喪社稷是為無策【喪息浪翻】今天下遭陽九之厄比年饑饉【比毗至翻】西北邊尤甚發三十萬衆具三百日糧東援海代【師古曰援音爰引也余謂代當作岱岱山也】南取江淮然後乃備計其道里一年尚未集合兵先至者聚居暴露師老械弊埶不可用此一難也邊既空虛不能奉軍糧内調郡國不相及屬此二難也【師古曰調發也音徒釣翻屬音之欲翻】計一人三百日食用糒十八斛【糒音備乾飯也】非牛力不能勝【勝音升】牛又自當齎食加二十斛重矣胡地沙鹵多乏水草以往事揆之軍出未滿百日牛必物故且盡餘糧尚多人不能負此三難也胡地秋冬甚寒春夏甚風多齎釜鍑薪炭重不可勝【師古曰鍑釜之大口者也音富勝音升】食糒飲水以歷四時【糒音備】師有疾疫之憂是故前世伐胡不過百日非不欲久埶力不能此四難也輜重自隨則輕銳者少不得疾行虜徐遁逃埶不能及幸而逢虜又累輜重【謂幸而逢虜得與之戰又為輜重所累也重直用翻累力瑞翻】如遇險阻䘖尾相隨【師古曰銜馬銜也尾馬尾也言前後單行不得並驅】虜要遮前後危殆不測此五難也【要讀曰邀】大用民力功不可必立臣伏憂之今既發兵宜縱先至者令臣尤等深入霆擊且以創艾胡虜【師古曰請率見到之兵且以擊虜艾音乂】莽不聽尤言轉兵穀如故天下騷動咸既受莽孝單于之號馳出塞歸庭具以見脅狀白單于單于更以為於栗置支侯匈奴賤官也【漢書栗作粟】後助病死莽以登代助為順單于吏士屯邊者所在放縱而内郡愁於徵發民棄城郭始流亡為盜賊并州平州尤甚【并州部太原上黨上郡西河朔方五原雲中定襄鴈門等郡余按此時未有平州漢末公孫度自號平州牧魏始分幽州置平州平字誤也】莽令七公六卿號皆兼稱將軍【七公四輔及三公也六卿羲和作士秩宗典樂共工予虞】遣著武將軍逯並等鎭名都中郎將繡衣執灋各五十五人分鎭緣邊大郡督大姦猾擅弄兵者皆乘便為姦於外撓亂州郡【師古曰撓音火高翻其字從手下同】貨賂為市侵漁百姓莽下書切責之曰自今以來敢犯此者輒捕繫以名聞然猶放縱自若北邊自宣帝以來數世不見煙火之警【謂匈奴欵塞之後也】人民熾盛牛馬布野及莽撓亂匈奴與之搆難【難乃旦翻】邊民死亡係獲數年之間北邊虛空野有暴骨矣【暴步卜翻】 太師王舜自莽簒位後病悸寖劇死【師古曰心動曰悸寖漸也悸音葵季翻】 莽為太子置師友各四人秩以大夫以故大司徒馬宫等為師凝傅丞阿輔保拂是為四師【馬宫為師凝宗伯鳳為傅丞袁聖為阿輔王嘉為保拂也師古曰拂讀曰弼】故尚書令唐林等為胥附犇走先後禦侮是為四友【唐林為胥附李充為犇走趙襄為先後廉丹為禦侮走讀曰奏先悉薦翻後胡茂翻】又置師友侍中諫議六經祭酒各一人凡九祭酒秩皆上卿【師友侍中諫議三祭酒并六經六祭酒為九祭酒】遣使者奉璽書印綬安車駟馬迎龔勝即拜為師友祭酒【師古曰就家迎之因拜官璽斯氏翻】使者與郡太守縣長吏三老官屬行義諸生千人以上入勝里致詔【龔勝楚人史逸其所居縣百官表縣令丞尉為長吏師古曰行義謂鄉邑有行義之人諸生學徒也行下孟翻】使者欲令勝起迎久立門外勝稱病篤為牀室中戶西南牖下【師古曰牖窗也於戶之西室之南牖下也】東首加朝服拖紳【師古曰拖引也卧著朝衣故云加引大帶於體也論語稱孔子疾君視之東首加朝服拖紳故放之也首式又翻拖音吐賀翻】使者付璽書奉印綬内安車駟馬進謂勝曰聖朝未嘗忘君制作未定待君為政思聞所欲施行以安海内勝對曰素愚加以年老被病命在朝夕隨使君上道【師古曰示若尊敬使者故謂之使君被皮義翻】必死道路無益萬分使者要說【師古曰要音一遥翻說音式芮翻】至以印綬就加勝身勝輒推不受【推吐雷翻】使者上言方盛夏暑熱勝病少氣【少詩沼翻】可須秋凉乃發【師古曰須待也】有詔許之使者五日壹與太守俱問起居為勝兩子及門人高暉等言【為于偽翻】朝廷虚心待君以茅土之封雖疾病宜移動至傳舍【傳知戀翻】示有行意必為子孫遺大業【大業謂封邑也】暉等白使者語勝自知不見聽即謂暉等吾受漢家厚恩無以報今年老矣旦暮入地誼豈以一身事二姓下見故主哉勝因敕以棺歛喪事【勑誡也師古曰棺音工喚翻歛音力贍翻】衣周於身棺周於衣勿隨俗動吾冢種柏作祠堂【師古曰若葬多設器備則恐被掘故云動吾冢也亦不得種柏及作祠堂皆不隨俗貢父曰勝意謂一葬之後更不隨俗動冢土種柏作祠堂】語畢遂不復開口飲食【復扶又翻】積十四日死死時七十九矣是時清名之士又有琅琊紀逡齊薛方太原郇越郇相【師古曰逡音于旬翻郇音荀又音胡頑翻今郇荀二姓並有之俱稱周武王之後也】沛唐林唐尊皆以明經飭行顯名於世【師古曰飭謹也行下孟翻】紀逡兩唐皆仕莽封侯貴重歷公卿位唐林數上疏諫正有忠直節唐尊衣敝履空【師古曰衣音於既翻著敝衣躡空履也空穿也】被虚偽名【師古曰被音皮義翻】郇相為莽太子四友病死莽太子遣使裞以衣衾【師古曰贈喪衣服曰裞音式芮翻其字從衣】其子攀棺不聽曰死父遺言師友之送勿有所受今於皇太子得託友官故不受也京師稱之莽以安車迎薛方方因使者辭謝曰堯舜在上下有巢由【巢父許由也】今明主方隆唐虞之德小臣欲守箕山之節【張晏曰許由隱於箕山在陽城有許由祠】使者以聞莽說其言【說讀曰悦】不彊致【彊其兩翻】初隃麋郭欽為南郡太守【師古曰隃麋扶風之縣也隃音踰】杜陵蔣詡為兖州刺史【姓譜周公之子封於蔣後以為氏】亦以廉直為名莽居攝欽詡皆以病免官歸鄉里臥不出戶卒於家哀平之際沛國陳咸以律令為尚書【中興之後沛方為國此由范史以後來所見書之也陳咸後漢陳寵之曾祖也】莽輔政多改漢制咸心非之及何武鮑宣死【事見上卷平帝元始三年】咸歎曰易稱見幾而作不俟終日【易下繫之辭也幾居希翻】吾可以逝矣即乞骸骨去職及莽簒位召咸為掌寇大夫【掌寇大夫當屬作士】咸謝病不肯應時三子參欽豐皆在位咸悉令解官歸鄉里閉門不出入猶用漢家祖臘人問其故咸曰我先人豈知王氏臘乎悉收歛其家律令書文壁藏之【按三十二卷成帝綏和元年陳咸以淳于長事廢歸故郡以憂死咸沛郡相人也此書沛國陳咸木之後漢書陳寵傳光武始改沛郡為沛國二陳咸雖同居沛各是一人】又齊栗融北海禽慶蘇章【禽姓也墨子弟子有禽滑釐又有犀首禽息】山陽曹竟皆儒生去官不仕於莽<br />
<br />
  班固贊曰春秋列國卿大夫及至漢興將相名臣耽寵以失其世者多矣【師古曰言不能去】是故清節之士於是為貴然大率多能自治而不能治人【治直之翻】王貢之材優於龔鮑守死善道勝實蹈焉【師古曰論語孔子曰守死善道今龔勝不受莽官蹈斯迹也】貞而不諒薛方近之【師古曰論語孔子曰君子貞而不諒謂君子之人正其道耳言不必信也薛方引避亂朝詭引巢許為諭近此義近其靳翻】郭欽蔣詡好遯不汙絶紀唐矣【師古曰欽詡不仕於莽遯逃濁亂不汙其節殊於紀逡及兩唐好呼到翻通鑑書龔勝之死遂及一時人士又書班固之論其為監也不亦昭乎】<br />
<br />
  是歲瀕河郡蝗生 河决魏郡泛清河以東數郡先是莽恐河决為元城冢墓害【莽曾祖賀以下冢墓在魏郡元城先悉薦翻】及决東去元城不憂水故遂不堤塞【塞悉則翻】<br />
<br />
  四年春二月赦天下 厭難將軍陳欽震狄將軍王巡上言捕得虜生口驗問言虜犯邊者皆孝單于咸子角所為莽乃會諸夷斬咸子登於長安市 大司馬甄邯死【甄之人翻邯戶甘翻】 莽至明堂下書以洛陽為東都常安為西都邦畿連體各有采任【男食邑於畿内曰采女食邑於畿内曰任師古曰采采服也任男服也】州從禹貢為九爵從周氏為五【禹貢冀兖青徐揚豫荆雍梁凡九州周爵公侯伯子男凡五等】諸侯之員千有八百【八州州二百一十國并畿内凡千七百七十三國言千八百國舉成數也】附城之數亦如之以俟有功諸公一同【地方百里曰同】有衆萬戶其餘以是為差今已受封者公侯以下凡七百九十六人附城千五百五十一人以圖簿未定未授國邑且令受奉都内【奉讀曰俸都内積錢之府屬大司農】月錢數千諸侯皆困乏至有傭作者 莽性躁擾不能無為【躁則到翻】每有所興造動欲慕古不度時宜【度徒洛翻】制度又不定吏緣為姦天下謷謷陷刑者衆【師古曰謷謷衆口愁聲音敖】莽知民愁怨乃下詔諸食王田皆得賣之勿拘以灋犯私買賣庶人者且一切勿治【治直之翻】然他政誖亂刑罰深刻賦歛重數猶如故焉【誖蒲内翻數所角翻】 初五威將帥出西南夷改句町王為侯王邯怨怒不附【師古曰邯句町王之名也音下甘翻】莽諷牂柯大尹周歆詐殺邯【牂柯音臧哥 考異曰西南夷傳作周欽莽傳作周歆今從之】邯弟承起兵殺歆州郡擊之不能服莽又發高句驪兵擊匈奴高句驪不欲行郡彊迫皆亡出塞因犯灋為寇【彊其兩翻】遼西大尹田譚追擊之為所殺州郡歸咎於高句驪侯騶嚴尤奏言貉人犯灋不從騶起【貉與貊同莫百翻後漢書句驪一名貊耳】正有他心宜令州郡且慰安之【師古曰假令騶有惡心亦當且慰安】今猥被以大罪【師古曰猥多也厚也被加也音皮義翻余謂猥積也曲也】恐其遂畔夫餘之屬必有和者【和胡卧翻】匈奴未克夫餘濊貉復起此大憂也【後漢書濊與句驪同種言語法俗大抵相類各有部界復扶又翻】莽不尉安濊貉遂反詔尤擊之尤誘高句驪侯騶至而斬焉傳首長安【騶側尤翻】莽大說更名高句驪為下句驪【說讀曰悦更工衡翻下同】於是貉人愈犯邊東北與西南夷皆亂【東濊貊北匈奴也】莽志方盛以為四夷不足吞滅專念稽古之事復下書以此年二月東巡狩具禮儀調度【復扶又翻調徒弔翻】既而以文母太后體不安且止待後 初莽為安漢公時欲諂太皇太后以斬郅支功奏尊元帝廟為高宗【事見上卷元始四年】太后晏駕後當以禮配食云及莽改號太后為新室文母絶之於漢不令得體元帝墮壞孝元廟【師古曰夫婦一體也墮毁也音火規翻壞音怪】更為文母太后起廟獨置孝元廟故殿以為文母篹食堂【置捨也留地孟康曰篹音撰晉灼曰篹具也】既成名曰長夀宫以太后在故未謂之廟莽置酒長夀宫請太后既至見孝元廟廢徹塗地太后驚泣曰此漢家宗廟皆有神靈與何治而壞之【師古曰與讀曰預言此何罪於汝無所干預何為毁壞之壞音怪】且使鬼神無知又何用廟為如令有知我乃人之妃妾豈宜辱帝之堂以陳饋食哉【釋名吳人謂祭為饋】私謂左右曰此人慢神多矣能久得祐乎【祐福也神助也】飲酒不樂而罷【樂音洛】自莽簒位後知太后怨恨求所以媚太后者無不為然愈不說【說讀曰悦】莽更漢家黑貂著黄貂【孟康曰侍中所著貂也莽改漢制服黄更音工衡翻著音陟畧翻】又改漢正朔伏臘日太后令其官屬黑貂至漢家正臘日獨與其左右相對飲食<br />
<br />
  五年春二月文母皇太后崩年八十四葬渭陵與元帝合而溝絶之【如淳曰葬於司馬門内作溝絶之】新室世世獻祭其廟元帝配食坐於牀下莽為太后服喪三年【為于偽翻】 烏孫大小昆彌遣使貢獻莽以烏孫國人多親附小昆彌見匈奴諸邊並侵意欲得烏孫心乃遣使者引小昆彌使坐大昆彌使上【使疏吏翻下同】師友祭酒滿昌劾奏使者曰夷狄以中國有禮誼故詘而服從【劾戶槩翻詘與屈同】大昆彌君也今序臣使於君使之上非所以有夷狄也奉使大不敬莽怒免昌官【師友祭酒龔勝不肯就而滿昌為之鳳皇翔于千仭烏鳶彈射不去非虚言也】 西域諸國以莽積失恩信焉耆先叛【焉耆國治員渠城去長安七千三百里】殺都護但欽西域遂瓦解 十一月彗星出二十餘日不見【彗祥歲翻旋芮翻又徐醉翻見賢遍翻】 是歲以挾銅炭者多除其灋 匈奴烏珠留單于死用事大臣右骨都侯須卜當【史記正義匈奴須卜氏掌獄訟】即王昭君女伊墨居次云之壻也云常欲與中國和親又素與伊栗置支侯咸厚善【云於咸為季父也】見咸前後為莽所拜故遂立咸為烏累若鞮單于【師古曰累音力追翻】烏累單于咸立以弟輿為右谷蠡王【谷音鹿蠡鹿奚翻】烏珠留單于子蘇屠胡本為左賢王後更謂之護于【烏珠留單于以左賢王數死不祥更易命左賢王為護于】欲傳以國咸怨烏珠留單于貶已號乃貶護于為左屠耆王<br />
<br />
  天鳳元年春正月赦天下 莽下詔將以是歲四仲月徧行廵狩之禮太官齎糒乾肉内者行張坐臥【乾音干内者令時屬共工續漢志内者令掌布張諸衣物師古曰張坐臥者謂帷帳茵席也】所過毋有所給【師古曰言自齎食及帷帳以行在路所經過不須供費也】俟畢北廵狩之禮即于土中居洛陽之都羣公奏言皇帝至孝新遭文母之喪顔色未復飲食損少今一歲四廵道路萬里春秋尊非糒乾肉之所能堪【糒音備乾音干】且無廵狩須闋大服以安聖體【師古曰闋盡也音口决翻】莽從之要期以天鳳七年廵狩厥明年即土之中遣太傅平晏大司空王邑之洛陽營相宅兆【相息亮翻宅居也壇域瑩界皆曰兆】圖起宗廟社稷郊兆云 三月壬申晦日有食之大赦天下以災異策大司馬逯並就侯氏朝位【免官以侯爵就朝位朝直遥翻】太傅平晏勿領尚書事以利苖男訢為大司馬【如浮曰利苖邑名訢音欣】莽即眞尤備大臣抑奪下權朝臣有言其過失者輒拔擢孔仁趙博費興等以敢擊大臣故見信任【洪氏隸釋曰余家所收姓氏文字粗備以諸書參攷頗多牴牾不合姓苑云費氏禹後音父位翻李利涉編古命氏云費氏出自魯桓公少子季友受邑於費元和姓纂費氏亦音秘姓林云費氏音蜚夏禹之後余嘗攷之此字有兩姓音讀不同源流亦異其一音蜚嬴姓出於伯益之後史記所載費昌費中楚費無極漢費將軍費直費長房費禕之徒是其後也其一音祕姫姓出於魯季友姓苑所載琅邪費氏是其後也然則姓苑姓林姓纂皆云夏禹之後姓纂又云亦音祕及謂琅琊費氏為直之後皆其差誤而編古命氏以費將軍費禕之徒出於魯季友亦非也師古曰費音扶味翻】擇名官而居之國將哀章頗不清莽為選置和叔【師古曰特為置此官余謂莽以國將主冬故置和叔之官將即亮翻下同】敕曰非但保國將閨門當保親屬在西州者【章梓潼人其親屬在西州】諸公皆輕賤而章尤甚【言十一公皆為莽所輕賤而章尤甚也】 夏四月隕霜殺草木海瀕尤甚【師古曰邊海之地也】六月黄霧四塞【塞悉則翻】秋七月大風拔樹飛北闕直城門屋瓦【直城門長安城西出南頭第二門】雨雹殺牛羊【雨于具翻】 莽以周官王制之文置卒正連率大尹職如太守【王制三十國為卒卒有正十國為連連有率率所類翻守式又翻】又置州牧部監二十五人分長安城旁六鄉置帥各一人分三輔為六尉郡【師古曰三輔黄圖云渭城安陵以西北至栒邑義渠十縣屬京尉大夫府居故長安寺高陵以北十縣屬師尉大夫府居故廷尉府新豐以東至湖十縣屬翊尉大夫府居城東霸陵杜陵東至藍田西至武功都夷十縣屬光尉大夫府居城南茂陵槐里至汧十縣屬扶尉大夫府居城西長陵池陽以北至雲陽祋祤十縣屬烈尉大夫府居城北帥所類翻】河内河東弘農河南潁川南隊為六隊郡【師古曰隊音遂仲馮曰河南當為滎陽莽所分為六遂之一也下文自有河南大尹更為保忠信卿河東兆隊河内後隊弘農右隊滎陽祈隊潁川左隊南陽前隊】更名河南大尹曰保忠信卿【更工衡翻下同】益河南屬縣滿三十置六郊州長各一人人主五縣及他官名悉改大郡至分為五合百二十有五郡九州之内縣二千二百有三又倣古六服為惟城惟寧惟翰惟屏惟垣惟藩各以其方為稱【公作甸服是為惟城諸在侯服是為惟寜在采任諸侯是為惟翰在賓服是為惟屏在揆文教奮武衛是為惟垣在九州之外是為惟藩屏必郢翻稱尺證翻】總為萬國焉其後歲復變更一郡至五易名而還復其故吏民不能紀每下詔書輒繫其故名云【按莽傳詔曰祈隊故滎陽是也】 匈奴右骨都侯須卜當伊墨居次云勸單于和親遣人之西虎猛制虜塞下【漢書作西河虎猛制虜塞下師古曰虎猛縣名制虜塞在其界此逸河字之往也】告塞吏云欲見和親侯和親侯者王昭君兄子歙也【師古曰歙音翕】中部都尉以聞【漢邊郡置五部都尉分治屬縣】莽遣歙歙弟騎都尉展德侯颯使匈奴【師古曰颯音立】賀單于初立賜黄金衣被繒帛紿言侍子登在因購求陳良終帶等單于盡收陳良等二十七人皆械檻付使者遣厨唯姑夕王富等四十人送歙颯莽作焚如之刑燒殺陳良等【應劭曰易有焚如死如棄如之言莽依此作刑名也如淳曰焚如死如棄如者謂不孝子也不畜於父母不容於朋友故燒殺棄之莽依此作刑名也】緣邊大饑人相食諫大夫如普行邊兵還言軍士久<br />
<br />
  屯寒苦邊郡無以相贍今單于新和宜因是罷兵校尉韓威進曰以新室之威而吞胡虜無異口中蚤蝨臣願得勇敢之士五千人不齎斗糧飢食虜肉渴飲其血可以横行莽壯其言以威為將軍然采普言徵還諸將在邊者免陳欽等十八人又罷四關鎮都尉諸屯兵【莽置四關各有鎮都尉領屯兵】單于貪莽賂遺【遺于季翻】故外不失漢故事然内利寇掠又使還知子登前死怨恨【使疏吏翻還從宣翻又如字】寇虜從左地入不絶【師古曰入為寇而虜掠】使者問單于輒曰烏相與匈奴無狀黠民共為寇入塞譬如中國有盜賊耳【黠下入翻】咸初立持國威信尚淺盡力禁止不敢有二心莽復發軍屯【復扶又翻下同】 益州蠻夷愁擾盡反復殺益州大尹程隆【武帝開滇國為益州郡屬益州莽屬梁州】莽遣平蠻將軍馮茂發巴蜀犍為吏士賦歛取足於民以擊之【犍居言翻歛力贍翻】 莽復申下金銀龜貝之貨頗增減其賈直【賈讀曰價】而罷大小錢改作貨布貨泉二品並行【貨布長二寸五分廣一寸首長八分有奇廣八分其圜好徑二分半足枝長八分間廣二分其文右曰貨左曰布重二十五銖直貨泉二十五貨泉徑一寸重五銖文右曰貨左曰泉文直一孔潁達曰案食貨志今世謂之笇錢是也邊猶為貨泉之字大泉即今四文錢也四邊並有文也貨布之形今世難識世人耕地猶有得者古時一箇凖二十五錢也余按古所謂泉布者其藏曰泉其行曰布取名於水泉其流行無不徧無不徧則布之義也王莽以為貨二品非古義 考異曰食貨志改作貨布在天鳳元年莽傳在地皇元年盖以大錢盡之年至地皇元年乃絶不行耳非其年始作貨布也】又以大錢行久罷之恐民挾不止乃令民且獨行大錢盡六年毋得復挾大錢矣每一易錢民用破業而大陷刑<br />
<br />
  資治通鑑卷三十七  <br>
   </div> 

<script src="/search/ajaxskft.js"> </script>
 <div class="clear"></div>
<br>
<br>
 <!-- a.d-->

 <!--
<div class="info_share">
</div> 
-->
 <!--info_share--></div>   <!-- end info_content-->
  </div> <!-- end l-->

<div class="r">   <!--r-->



<div class="sidebar"  style="margin-bottom:2px;">

 
<div class="sidebar_title">工具类大全</div>
<div class="sidebar_info">
<strong><a href="http://www.guoxuedashi.com/lsditu/" target="_blank">历史地图</a></strong>  
<a href="http://www.880114.com/" target="_blank">英语宝典</a>  
<a href="http://www.guoxuedashi.com/13jing/" target="_blank">十三经检索</a> 
<br><strong><a href="http://www.guoxuedashi.com/gjtsjc/" target="_blank">古今图书集成</a></strong> 
<a href="http://www.guoxuedashi.com/duilian/" target="_blank">对联大全</a> <strong><a href="http://www.guoxuedashi.com/xiangxingzi/" target="_blank">象形文字典</a></strong> 

<br><a href="http://www.guoxuedashi.com/zixing/yanbian/">字形演变</a>  <strong><a href="http://www.guoxuemi.com/hafo/" target="_blank">哈佛燕京中文善本特藏</a></strong>
<br><strong><a href="http://www.guoxuedashi.com/csfz/" target="_blank">丛书&方志检索器</a></strong> <a href="http://www.guoxuedashi.com/yqjyy/" target="_blank">一切经音义</a>  

<br><strong><a href="http://www.guoxuedashi.com/jiapu/" target="_blank">家谱族谱查询</a></strong>  <strong><a href="http://shufa.guoxuedashi.com/sfzitie/" target="_blank">书法字帖欣赏</a></strong> 
<br>

</div>
</div>


<div class="sidebar" style="margin-bottom:0px;">

<font style="font-size:22px;line-height:32px">QQ交流群9:489193090</font>


<div class="sidebar_title">手机APP 扫描或点击</div>
<div class="sidebar_info">
<table>
<tr>
	<td width=160><a href="http://m.guoxuedashi.com/app/" target="_blank"><img src="/img/gxds-sj.png" width="140"  border="0" alt="国学大师手机版"></a></td>
	<td>
<a href="http://www.guoxuedashi.com/download/" target="_blank">app软件下载专区</a><br>
<a href="http://www.guoxuedashi.com/download/gxds.php" target="_blank">《国学大师》下载</a><br>
<a href="http://www.guoxuedashi.com/download/kxzd.php" target="_blank">《汉字宝典》下载</a><br>
<a href="http://www.guoxuedashi.com/download/scqbd.php" target="_blank">《诗词曲宝典》下载</a><br>
<a href="http://www.guoxuedashi.com/SiKuQuanShu/skqs.php" target="_blank">《四库全书》下载</a><br>
</td>
</tr>
</table>

</div>
</div>


<div class="sidebar2">
<center>


</center>
</div>

<div class="sidebar"  style="margin-bottom:2px;">
<div class="sidebar_title">网站使用教程</div>
<div class="sidebar_info">
<a href="http://www.guoxuedashi.com/help/gjsearch.php" target="_blank">如何在国学大师网下载古籍?</a><br>
<a href="http://www.guoxuedashi.com/zidian/bujian/bjjc.php" target="_blank">如何使用部件查字法快速查字?</a><br>
<a href="http://www.guoxuedashi.com/search/sjc.php" target="_blank">如何在指定的书籍中全文检索?</a><br>
<a href="http://www.guoxuedashi.com/search/skjc.php" target="_blank">如何找到一句话在《四库全书》哪一页?</a><br>
</div>
</div>


<div class="sidebar">
<div class="sidebar_title">热门书籍</div>
<div class="sidebar_info">
<a href="/so.php?sokey=%E8%B5%84%E6%B2%BB%E9%80%9A%E9%89%B4&kt=1">资治通鉴</a> <a href="/24shi/"><strong>二十四史</strong></a>&nbsp; <a href="/a2694/">野史</a>&nbsp; <a href="/SiKuQuanShu/"><strong>四库全书</strong></a>&nbsp;<a href="http://www.guoxuedashi.com/SiKuQuanShu/fanti/">繁体</a>
<br><a href="/so.php?sokey=%E7%BA%A2%E6%A5%BC%E6%A2%A6&kt=1">红楼梦</a> <a href="/a/1858x/">三国演义</a> <a href="/a/1038k/">水浒传</a> <a href="/a/1046t/">西游记</a> <a href="/a/1914o/">封神演义</a>
<br>
<a href="http://www.guoxuedashi.com/so.php?sokeygx=%E4%B8%87%E6%9C%89%E6%96%87%E5%BA%93&submit=&kt=1">万有文库</a> <a href="/a/780t/">古文观止</a> <a href="/a/1024l/">文心雕龙</a> <a href="/a/1704n/">全唐诗</a> <a href="/a/1705h/">全宋词</a>
<br><a href="http://www.guoxuedashi.com/so.php?sokeygx=%E7%99%BE%E8%A1%B2%E6%9C%AC%E4%BA%8C%E5%8D%81%E5%9B%9B%E5%8F%B2&submit=&kt=1"><strong>百衲本二十四史</strong></a>  <a href="http://www.guoxuedashi.com/so.php?sokeygx=%E5%8F%A4%E4%BB%8A%E5%9B%BE%E4%B9%A6%E9%9B%86%E6%88%90&submit=&kt=1"><strong>古今图书集成</strong></a>
<br>

<a href="http://www.guoxuedashi.com/so.php?sokeygx=%E4%B8%9B%E4%B9%A6%E9%9B%86%E6%88%90&submit=&kt=1">丛书集成</a> 
<a href="http://www.guoxuedashi.com/so.php?sokeygx=%E5%9B%9B%E9%83%A8%E4%B8%9B%E5%88%8A&submit=&kt=1"><strong>四部丛刊</strong></a>  
<a href="http://www.guoxuedashi.com/so.php?sokeygx=%E8%AF%B4%E6%96%87%E8%A7%A3%E5%AD%97&submit=&kt=1">說文解字</a> <a href="http://www.guoxuedashi.com/so.php?sokeygx=%E5%85%A8%E4%B8%8A%E5%8F%A4&submit=&kt=1">三国六朝文</a>
<br><a href="http://www.guoxuedashi.com/so.php?sokeytm=%E6%97%A5%E6%9C%AC%E5%86%85%E9%98%81%E6%96%87%E5%BA%93&submit=&kt=1"><strong>日本内阁文库</strong></a> <a href="http://www.guoxuedashi.com/so.php?sokeytm=%E5%9B%BD%E5%9B%BE%E6%96%B9%E5%BF%97%E5%90%88%E9%9B%86&ka=100&submit=">国图方志合集</a> <a href="http://www.guoxuedashi.com/so.php?sokeytm=%E5%90%84%E5%9C%B0%E6%96%B9%E5%BF%97&submit=&kt=1"><strong>各地方志</strong></a>

</div>
</div>


<div class="sidebar2">
<center>

</center>
</div>
<div class="sidebar greenbar">
<div class="sidebar_title green">四库全书</div>
<div class="sidebar_info">

《四库全书》是中国古代最大的丛书,编撰于乾隆年间,由纪昀等360多位高官、学者编撰,3800多人抄写,费时十三年编成。丛书分经、史、子、集四部,故名四库。共有3500多种书,7.9万卷,3.6万册,约8亿字,基本上囊括了古代所有图书,故称“全书”。<a href="http://www.guoxuedashi.com/SiKuQuanShu/">详细>>
</a>

</div> 
</div>

</div>  <!--end r-->

</div>
<!-- 内容区END --> 

<!-- 页脚开始 -->
<div class="shh">

</div>

<div class="w1180" style="margin-top:8px;">
<center><script src="http://www.guoxuedashi.com/img/plus.php?id=3"></script></center>
</div>
<div class="w1180 foot">
<a href="/b/thanks.php">特别致谢</a> | <a href="javascript:window.external.AddFavorite(document.location.href,document.title);">收藏本站</a> | <a href="#">欢迎投稿</a> | <a href="http://www.guoxuedashi.com/forum/">意见建议</a> | <a href="http://www.guoxuemi.com/">国学迷</a> | <a href="http://www.shuowen.net/">说文网</a><script language="javascript" type="text/javascript" src="https://js.users.51.la/17753172.js"></script><br />
  Copyright &copy; 国学大师 古典图书集成 All Rights Reserved.<br>
  
  <span style="font-size:14px">免责声明:本站非营利性站点,以方便网友为主,仅供学习研究。<br>内容由热心网友提供和网上收集,不保留版权。若侵犯了您的权益,来信即刪。scp168@qq.com</span>
  <br />
ICP证:<a href="http://www.beian.miit.gov.cn/" target="_blank">鲁ICP备19060063号</a></div>
<!-- 页脚END --> 
<script src="http://www.guoxuedashi.com/img/plus.php?id=22"></script>
<script src="http://www.guoxuedashi.com/img/tongji.js"></script>

</body>
</html>
