<!DOCTYPE html PUBLIC "-//W3C//DTD XHTML 1.0 Transitional//EN" "http://www.w3.org/TR/xhtml1/DTD/xhtml1-transitional.dtd">
<html xmlns="http://www.w3.org/1999/xhtml">
<head>
<meta http-equiv="Content-Type" content="text/html; charset=utf-8" />
<meta http-equiv="X-UA-Compatible" content="IE=Edge,chrome=1">
<title>資治通鑒_277-資治通鑑卷二百七十六_277-資治通鑑卷二百七十六</title>
<meta name="Keywords" content="資治通鑒_277-資治通鑑卷二百七十六_277-資治通鑑卷二百七十六">
<meta name="Description" content="資治通鑒_277-資治通鑑卷二百七十六_277-資治通鑑卷二百七十六">
<meta http-equiv="Cache-Control" content="no-transform" />
<meta http-equiv="Cache-Control" content="no-siteapp" />
<link href="/img/style.css" rel="stylesheet" type="text/css" />
<script src="/img/m.js?2020"></script> 
</head>
<body>
 <div class="ClassNavi">
<a  href="/24shi/">二十四史</a> | <a href="/SiKuQuanShu/">四库全书</a> | <a href="http://www.guoxuedashi.com/gjtsjc/"><font  color="#FF0000">古今图书集成</font></a> | <a href="/renwu/">历史人物</a> | <a href="/ShuoWenJieZi/"><font  color="#FF0000">说文解字</a></font> | <a href="/chengyu/">成语词典</a> | <a  target="_blank"  href="http://www.guoxuedashi.com/jgwhj/"><font  color="#FF0000">甲骨文合集</font></a> | <a href="/yzjwjc/"><font  color="#FF0000">殷周金文集成</font></a> | <a href="/xiangxingzi/"><font color="#0000FF">象形字典</font></a> | <a href="/13jing/"><font  color="#FF0000">十三经索引</font></a> | <a href="/zixing/"><font  color="#FF0000">字体转换器</font></a> | <a href="/zidian/xz/"><font color="#0000FF">篆书识别</font></a> | <a href="/jinfanyi/">近义反义词</a> | <a href="/duilian/">对联大全</a> | <a href="/jiapu/"><font  color="#0000FF">家谱族谱查询</font></a> | <a href="http://www.guoxuemi.com/hafo/" target="_blank" ><font color="#FF0000">哈佛古籍</font></a> 
</div>

 <!-- 头部导航开始 -->
<div class="w1180 head clearfix">
  <div class="head_logo l"><a title="国学大师官网" href="http://www.guoxuedashi.com" target="_blank"></a></div>
  <div class="head_sr l">
  <div id="head1">
  
  <a href="http://www.guoxuedashi.com/zidian/bujian/" target="_blank" ><img src="http://www.guoxuedashi.com/img/top1.gif" width="88" height="60" border="0" title="部件查字,支持20万汉字"></a>


<a href="http://www.guoxuedashi.com/help/yingpan.php" target="_blank"><img src="http://www.guoxuedashi.com/img/top230.gif" width="600" height="62" border="0" ></a>


  </div>
  <div id="head3"><a href="javascript:" onClick="javascript:window.external.AddFavorite(window.location.href,document.title);">添加收藏</a>
  <br><a href="/help/setie.php">搜索引擎</a>
  <br><a href="/help/zanzhu.php">赞助本站</a></div>
  <div id="head2">
 <a href="http://www.guoxuemi.com/" target="_blank"><img src="http://www.guoxuedashi.com/img/guoxuemi.gif" width="95" height="62" border="0" style="margin-left:2px;" title="国学迷"></a>
  

  </div>
</div>
  <div class="clear"></div>
  <div class="head_nav">
  <p><a href="/">首页</a> | <a href="/ShuKu/">国学书库</a> | <a href="/guji/">影印古籍</a> | <a href="/shici/">诗词宝典</a> | <a   href="/SiKuQuanShu/gxjx.php">精选</a> <b>|</b> <a href="/zidian/">汉语字典</a> | <a href="/hydcd/">汉语词典</a> | <a href="http://www.guoxuedashi.com/zidian/bujian/"><font  color="#CC0066">部件查字</font></a> | <a href="http://www.sfds.cn/"><font  color="#CC0066">书法大师</font></a> | <a href="/jgwhj/">甲骨文</a> <b>|</b> <a href="/b/4/"><font  color="#CC0066">解密</font></a> | <a href="/renwu/">历史人物</a> | <a href="/diangu/">历史典故</a> | <a href="/xingshi/">姓氏</a> | <a href="/minzu/">民族</a> <b>|</b> <a href="/mz/"><font  color="#CC0066">世界名著</font></a> | <a href="/download/">软件下载</a>
</p>
<p><a href="/b/"><font  color="#CC0066">历史</font></a> | <a href="http://skqs.guoxuedashi.com/" target="_blank">四库全书</a> |  <a href="http://www.guoxuedashi.com/search/" target="_blank"><font  color="#CC0066">全文检索</font></a> | <a href="http://www.guoxuedashi.com/shumu/">古籍书目</a> | <a   href="/24shi/">正史</a> <b>|</b> <a href="/chengyu/">成语词典</a> | <a href="/kangxi/" title="康熙字典">康熙字典</a> | <a href="/ShuoWenJieZi/">说文解字</a> | <a href="/zixing/yanbian/">字形演变</a> | <a href="/yzjwjc/">金 文</a> <b>|</b>  <a href="/shijian/nian-hao/">年号</a> | <a href="/diming/">历史地名</a> | <a href="/shijian/">历史事件</a> | <a href="/guanzhi/">官职</a> | <a href="/lishi/">知识</a> <b>|</b> <a href="/zhongyi/">中医中药</a> | <a href="http://www.guoxuedashi.com/forum/">留言反馈</a>
</p>
  </div>
</div>
<!-- 头部导航END --> 
<!-- 内容区开始 --> 
<div class="w1180 clearfix">
  <div class="info l">
   
<div class="clearfix" style="background:#f5faff;">
<script src='http://www.guoxuedashi.com/img/headersou.js'></script>

</div>
  <div class="info_tree"><a href="http://www.guoxuedashi.com">首页</a> > <a href="/SiKuQuanShu/fanti/">四库全书</a>
 > <h1>资治通鉴</h1> <!--         下载:【右键另存为】即可 --></div>
  <div class="info_content zj clearfix">
  
<div class="info_txt clearfix" id="show">
<center style="font-size:24px;">277-資治通鑑卷二百七十六</center>
    資治通鑑卷二百七十六 宋  司馬光 撰<br />
<br />
  胡三省 音註<br />
<br />
  後唐紀【起彊圉大淵獻七月盡屠維赤奮若凡二年有奇】<br />
<br />
  明宗聖德和武欽孝皇帝中之上<br />
<br />
  天成二年秋七月以歸德節度使王晏球為北面副招討使【烏震既死以王晏球代之按薛史是年七月甲辰詔曰本朝親王遥領方鎮遂有副大使知節度事傳代已深相沿未改其西川東川今後落副大使只云節度使尋諸鎮皆正授節度使】 丙寅升夔州為寧江軍以西方鄴為節度使【賞破高季興軍復夔忠萬之功也蜀以夔州為鎮江軍今改為寧江軍】 癸巳以與高季興夔忠萬三州為豆盧革韋說之罪【元年以三州與季興革說猶為相因以此罪之】皆賜死 流段凝於遼州温韜於德州劉訓於濮州【自唐末以來流貶者皆不至其地遼德濮皆唐境也此三人皆使至流所】 任圜請致仕居磁州【磁墻之翻】許之八月己卯朔日有食之 冊禮使至長沙楚王殷始<br />
<br />
  建國【封楚王殷為國王見上卷是年六月】立宫殿置百官皆如天子或微更其名【示不敢擬天朝也更工行翻】翰林學士曰文苑學士知制誥曰知辭制樞密院曰左右機要司羣下稱之曰殿下令曰教以姚彦章為左丞相許德勲為右丞相李鐸為司徒崔潁為司空拓跋恒為僕射張彦瑶張迎判機要司【馬殷所恃以為國者高郁也建國置官郁不與焉何也豈殷諸子已有忌郁之心歟】然管内官屬皆稱攝惟朗桂節度使先除後請命【朗武平軍桂靜江軍時皆屬楚】恒本姓元避殷父諱改焉 九月帝謂安重誨曰從榮左右有矯宣朕旨令勿接儒生恐弱人志氣者朕以從榮年少臨大藩【是年三月從榮鎮鄴都事見上卷少詩照翻】故擇名儒使輔導之今姧人所言乃如此欲斬之重誨請嚴戒而已【安重誨非儒也故寛言者之罪獨不思矯宣上言國有常刑邪】 北都留守李彦超請復姓符從之【彦超李存審子存審本姓符】 丙寅以樞密使孔循兼東都留守【帝欲東巡使孔循留守洛陽莊宗同光三年復以洛陽為東都】 壬申契丹來請修好【好呼到翻】遣使報之 冬十月乙酉帝發洛陽將如汴州丁亥至滎陽【九域志滎陽縣在鄭州西六十里東至大梁一百四十里】民間譌言帝欲自擊吳又云欲制置東方諸侯宣武節度使檢校侍中朱守殷疑懼判官高密孫晟勸守殷反【高密漢古縣隋亂廢唐武德三年置于義城堡六年移就故夷安城即高密古縣也屬密州九域志在州東北一百二十里 考異曰江南録作孫忌今從王溥周世宗實録晟承正翻】守殷遂乘城拒守帝遣宣徽使范延光往諭之延光曰不早擊之則汴城堅矣願得五百騎與俱帝從之延光暮發未明行二百里抵大梁城下與汴人戰汴人大驚戊子帝至京水【京水在滎陽之東索水之西】遣御營使石敬瑭將親兵倍道繼之【自梁以來有侍衛親軍侍衛馬軍侍衛步軍】或謂安重誨曰失職在外之人乘賊未破或能為患不如除之重誨以為然奏遣使賜任圜死【任圜罷相見上卷是年六月】端明殿學士趙鳳哭謂重誨曰任圜義士安肯為逆公濫刑如此何以贊國使者至磁州圜聚其族酣飲然後死神情不撓【撓奴教翻】 己丑帝至大梁四面進攻吏民縋城出降者甚衆【縋馳偽翻】守殷知事不濟盡殺其族引頸命左右斬之乘城者望見乘輿【乘承正翻】相帥開門降【帥讀曰率下同】孫晟奔吳徐知誥客之【為孫晟盡節於江南張本】 戊戌詔免三司逋負近二百萬緡【近其靳翻】 辛丑吳大丞相都督中外諸軍事諸道都統鎮海寧國節度使兼中書令東海王徐温卒初温子行軍司馬忠義節度使同平章事知詢以其兄知誥非徐氏子【徐温養知誥為子見二百六十卷唐昭宗乾寧二年】數請代之執吳政【數所角翻】温曰汝曹皆不如也嚴可求及行軍副使徐玠屢勸温以知詢代知誥【徐知誥之於嚴可求結之以婚姻而可求之心不為之變徐温之門忠于所事者嚴可求陳彦謙而已】温以知誥孝謹不忍也陳夫人曰知誥自我家貧賤時養之【陳夫人徐温之妻子畜知誥者也】奈何富貴而弃之可求等言之不已温欲帥諸藩鎮入朝勸吳王稱帝【帥讀曰率】將行有疾乃遣知詢奉表勸進因留代知誥執政知誥草表欲求洪州節度使俟旦上之【上時兩翻】是夕温凶問至乃止【史言徐知誥得吳國之政亦有數存乎其間篡吳之業自此成矣】知詢亟歸金陵【為知誥知詢不相容張本】吳主贈温齊王諡曰忠武 山南西道節度使張筠久疾將佐請見不許副使符彦琳等疑其已死恐左右有奸謀請權交符印筠怒收彦琳及判官都指揮使下獄誣以謀反【下遐嫁翻】詔取彦琳等詣闕按之無狀釋之【觀于可洪張筠之事帝之廟號曰明亦有以也】徙筠為西都留守【莊宗同光三年復以長安為西都】 癸卯以保義節度使石敬瑭為宣武節度使【朱守殷反死以石敬瑭代之】兼侍衛親軍馬步都指揮使十一月庚戌吳王即皇帝位追尊孝武王曰武皇帝景王曰景皇帝宣王曰宣皇帝【孝武王忠武王行密也景王威王渥也宣王者隆演也】 安重誨議伐吳【安重誨欲乘徐温之死而伐之且問其舉大號之罪】帝不從【根本不固而伐人之國莊宗覆車可鑒也故不許】甲子吳大赦改元乾貞 丙子吳主尊太妃王氏曰皇太后以徐知詢為諸道副都統鎮海寧國節度使兼侍中【若使之嗣徐温之官職者】加徐知誥都督中外諸軍事【吳國中外大權實皆歸於徐知誥】 十二月戊寅朔孟知祥發民丁二十萬修成都城 吳主立兄廬江公濛為常山王弟鄱陽公激為平原王【激勅列翻】兄子南昌公珙為建安王【珙居勇翻吳主稱帝封其兄弟及其兄子皆自公陞王】 初晉陽相者周玄豹【相息亮翻】嘗言帝貴不可言帝即位欲召詣闕趙鳳曰玄豹言陛下當為天子今已驗矣無所復詢【復扶又翻】若置之京師則輕躁狂險之人必輻輳其門爭問吉凶自古術士妄言致人族滅者多矣非所以靖國家也【史言趙鳳有識】帝乃就除光禄卿致仕厚賜金帛而已中書舍人馬縞【縞工老翻】請用漢光武故事七廟之外别立親廟【見四十一卷漢光武建武三年】中書門下奏請如漢孝德孝仁皇例稱皇不稱帝【孝德皇見五十卷漢安帝建光元年孝仁皇見五十六卷靈帝建寧元年】帝欲兼稱帝羣臣乃引德明玄元興聖皇帝例皆立廟京師【唐尊臯陶為德明皇帝老子為玄元皇帝凉武昭王為興聖皇帝例時詣翻】帝令立於應州舊宅自高祖考妣以下皆追諡曰皇帝皇后墓曰陵【五代會要帝追尊高祖聿為孝恭皇帝廟號惠祖陵曰順陵妣崔氏曰昭皇后曾祖教曰孝質皇帝廟號毅祖陵曰衍陵妣張氏曰順皇后祖琰曰孝靖皇帝廟號烈祖陵曰奕陵妣何氏曰穆皇后父霓曰孝成皇帝廟號德祖陵曰慶陵歐史曰高祖妣劉氏曾祖諱敖父孝成妣劉氏諡懿皇后四陵皆在應州金城縣按帝之先本夷狄既無姓氏其名必當時有司所製也】 漢主如康州【九域志廣州南至康州一百九十里】 是歲蔚代緣邊粟斗不過十錢【蔚紆勿翻】<br />
<br />
  三年春正月丁巳吳主立子璉為江都王璘為江夏王璆為宜春王宣帝子廬陵公玢為南陽王【璉力展翻璠離珍翻璆音求玢悲巾翻吳主諡兄隆演曰宣皇帝】 昭義節度使毛璋所為驕僭時服赭袍【赭袍天子所服赭音者】縱酒為戲左右有諫者剖其心而視之帝聞之徵為右金吾衛上將軍【毛璋在邠州以驕僭徵及在潞州復然謂之不軌可也然一詔徵之則束手入衛蓋其人冥頑驕虐本無它心不知僭擬之為非然亦明宗能容之耳】 契丹陷平州【元年冬盧文進來奔唐得平州至是復為契丹所陷】二月丁丑朔日有食之 帝將如鄴都時扈駕諸軍家屬甫遷大梁又聞將如鄴都皆不悦詾詾有流言【說讀曰悦詾許拱翻】帝聞之不果行 吳自莊宗滅梁以來使者往來不絶庚辰吳使者至安重誨以為楊溥敢與朝廷抗禮【並立為帝是抗禮也】遣使窺覘【覘丑亷翻又丑艶翻】拒而不受自是遂與吳絶 張筠至長安【去年徙張筠留守西都】守兵閉門拒之【上意也】筠單騎入朝以為左衛上將軍 壬辰寧江節度使西方鄴攻拔歸州未幾荆南復取之【歸州高季興巡屬也九域志夔州東至歸州三百三十里幾居豈翻復扶又翻下宜復同】 樞密使同平章事孔循性狡佞安重誨親信之帝欲為皇子娶重誨女【為于偽翻】循謂重誨曰公職居近密不宜復與皇子為昏重誨辭之久之或謂重誨曰循善離間人【間古莧翻】不可置之密地循知之隂遣人結王德妃求納其女德妃請娶循女為從厚婦帝許之【王德妃有寵於帝言無不行後進拜淑妃】重誨大怒乙未以循同平章事充忠武節度使兼東都留守【解其近密之職】重誨性強愎【愎蒲逼翻】秦州節度使華温琪入朝請留闕下帝嘉之【當時諸帥皆樂在方鎮得自恣獨華温琪入朝請留故嘉之華戶化翻】除左驍衛上將軍月别賜錢穀【俸給之外别賜錢穀】歲餘帝謂重誨曰温琪舊人宜擇一重鎮處之【華温琪仕梁已為節鎮故云然處昌呂翻】重誨對以無闕它日帝屢言之重誨愠曰臣累奏無闕惟樞密使可代耳帝曰亦可重誨無以對【華温琪之才誠不足以當重鎮安重誨以君臣相得之雅詳明敷奏明宗宜無不從今則上下之言交不能暢其意相厲而已斯不學至此也】温琪聞之懼數月不出重誨惡成德節度使同平章事王建立奏建立與王都交結有異志【惡烏路翻初帝為代州刺史王建立已為虞侯將後從鎮真定帝自鄴為亂兵所逼舉兵南向建立殺真定監軍帝家屬得全由是愛之及帝即位擢為真定帥安重誨亦帝潛躍之時所親信者也即位自中門使擢樞密使重誨之所以惡建立權寵之間耳又是時王都在中山有異志數以書通建立約為兄弟故重誨言之】建立亦奏重誨專權求入朝面言其狀帝召之既至言重誨與宣徽使判三司張延朗結昏相表裏弄威福三月辛亥帝見重誨氣色甚怒謂曰今與卿一鎮自休息以王建立代卿張延朗亦除外官重誨曰臣披荆棘事陛下數十年值陛下龍飛承乏機密【承乏者承人之乏也言適時乏人故已得任機密】數年間天下幸無事今一旦棄之外鎮臣願聞其罪帝不懌而起【此段自孔循以下言重誨與孔循相傾自華温琪以下言其君臣嫌隙之所自來蓋重誨挾依乘之舊戀權而不肯退明宗積受浸潤之譖欲遠之而不能至於决裂則不可救矣】以語宣徽使朱弘昭【語牛倨翻】弘昭曰陛下平日待重誨如左右手奈何以小忿棄之願垂三思【朱弘昭今日之言知重誨之眷未衰也鳳翔之奏知重誨之權已去也小人之智隨時而反覆可畏也哉】帝尋召重誨慰撫之明日建立辭歸鎮帝曰卿比奏欲入分朕憂【比毗至翻近也】今復去何之【復扶又翻下不復同】會門下侍郎兼刑部尚書同平章事鄭珏請致仕己未以珏為左僕射致仕癸亥以建立為右僕射兼中書侍郎同平章事判三司 孟知祥屢與董璋爭鹽利【蜀中井鹽東西川巡屬之内皆有之各欲障固以專其利故爭按唐盛時卬嘉眉有井十三劍南西川院領之梓遂綿合昌渝瀘資榮陵簡有井四百六十劍南東川院領之東川鹽利多于西川矣】璋誘商旅販東川鹽入西川知祥患之乃於漢州置三場重征之【漢州東南與東川接界故列置三場以征鹽商】歲得錢七萬緡商旅不復之東川【之往也】 楚王殷如岳州遣六軍使袁詮【詮丑緣翻】副使王環監軍馬希瞻將水軍擊荆南高季興以水軍逆戰至劉郎洑【江陵府石首縣沙步有劉郎浦蜀先主納吳女處也洑房六翻洄流曰洑】希瞻夜匿戰艦數十艘於港中【艦戶黯翻艘疎留翻港古項翻】詰旦兩軍合戰希瞻出戰艦横擊之季興大敗俘斬以千數進逼江陵季興請和歸史光憲于楚【高季興執史光憲見上卷上年】軍還【還從宣翻又如字】楚王殷讓環不遂取荆南環曰江陵在中朝及吳蜀之間【中朝謂唐也既在中原且天朝也】四戰之地也【四面受敵謂之四戰之地】宜存之以為吾扞蔽【宋時趙韓王勸太祖緩取太原意亦如此】殷悦環每戰身先士卒【先悉薦翻】與衆同甘苦常置鍼藥於座右戰罷索傷者於帳前自傅治之【鍼諸深翻索山客翻治直之翻】士卒隸環麾下者相賀曰吾屬得死所矣故所向有功【史言為將得士卒之死力者勝】 楚大舉水軍擊漢圍封州【宋白曰封州即漢蒼梧郡之廣信縣也梁置梁信郡隋置封州在豐水之陽】漢主以周易筮之遇大有【龜為卜策為筮以四十九策信手分開視其奇耦三變而成爻十有八變而成封】於是大赦改元大有命左右街使蘇章將神弩三千戰艦百艘救封州【漢都番禺倣唐上京置左右街使九域志廣州西至封州六百一十里】章至賀江沈鐵絙於水【沈持林翻絙居登翻】兩岸作巨輪挽絙築長堤以隱之伏壯士於堤中章以輕舟逆戰陽不利楚人逐之入堤中挽輪舉絙楚艦不能進退以強弩夾水射之【射而亦翻】楚兵大敗解圍遁去漢主以章為封州團練使 夏四月以鄴都留守從榮為河東節度使北都留守以客省使太原馮贇為副留守【贇於倫翻】夾馬指揮使新平楊思權為步軍都指揮使以佐之戊寅以宣武節度使石敬瑭為鄴都留守天雄節度使加同平章事以樞密使范延光為成德節度使丙戌以樞密使安重誨兼河南尹以河南尹從厚為宣武節度使仍判六軍諸衛事【從厚本以河南尹判六軍諸衛事今易鎮汴州而判六軍諸衛事如故】 吳右雄武軍使苗璘靜江統軍王彦章將水軍萬人攻楚岳州至君山【岳州治巴陵洞庭湖在巴陵西君山在洞庭湖中方六十里】楚王殷遣右丞相許德勲將戰艦千艘禦之德勲曰吳人掩吾不備見大軍必懼而走乃潛軍角子湖使王環夜帥戰艦三百絶吳歸路【帥讀曰率】遲明吳人進軍荆江口【遲直二翻荆江口洞庭湖與大江會處】將會荆南兵攻岳州丁亥至道人磯德勲命戰棹都虞侯詹信以輕舟三百出吳軍後德勲以大軍當其前夾擊之吳軍大敗虜璘及彦章以歸 初義武節度使兼中書令王都鎮易定十餘年【梁均王龍德元年王都得定州至是九年】自除刺史以下官租賦皆贍本軍及安重誨用事稍以法制裁之帝亦以都篡父位惡之【王都囚其父處直而篡其位見二百七十一卷後梁均王龍德元年惡烏路翻】時契丹數犯塞【數所角翻】朝廷多屯兵於幽易間【瓦橋盧臺皆在幽易之間】大將往來都隂為之備浸成猜阻都恐朝廷移之它鎮腹心和昭訓勸都為自全之計都乃求昏於盧龍節度使趙德鈞又知成德節度使王建立與安重誨有隙遣使結為兄弟隂與之謀復河北故事【欲復如唐河北諸鎮世襲不輸朝廷貢賦不受朝廷徵發】建立陽許而密奏之都又以蠟書遺青徐潞益梓五帥離間之【是時青帥霍彦威徐帥房知温潞帥毛璋益帥孟知祥梓帥董璋皆倔強難制者也遺唯季翻下金遺同間古莧翻】又遣人說北面副招討使歸德節度使王晏球【說式芮翻】晏球不從乃以金遺晏球帳下使圖之不克【遺唯季翻】癸巳晏球以都反狀聞詔宣徽使張延朗與北面諸將議討之【北面諸將謂招討王晏球及所部戍幽易間諸將及幽州帥趙德鈞也】 戊戌吳徙常山王濛為臨川王 庚子詔削奪王都官爵壬寅以王晏球為北面招討使權知定州行州事以横海節度使安審通為副招討使以鄭州防禦使張䖍釗為都監【監古銜翻】發諸道兵會討定州是日晏球攻定州拔其北關城【權知定州行州事者以未得定州城使王晏球權知行州事於城外以招撫定州之民蓋此命未頒晏球之兵已至定州城下矣】都以重賂求救於奚酋托諾【托諾即圍莊宗者虜酋之桀也酋慈秋翻】五月托諾以萬騎突入定州晏球退保曲陽【曲陽漢之上曲陽縣隋改為恒陽唐元和十五年更名曲陽避穆宗名也屬定州九域志縣在州西六十里】都與托諾就攻之晏球與戰於嘉山下大破之托諾以二千騎奔還定州晏球追至城門因進攻之得其西關城定州城堅不可攻晏球增修西關城以為行府【置招討使行府及定州行州于西關城】使三州民輸税供軍食而守之【三州定祁易也王晏球之攻定州以持久弊之此其先定之計也】 辛酉以天雄節度副使趙敬怡為樞密使 王晏球聞契丹發兵救定州將大軍趣望都【趣七喻翻】遣張延朗分兵退保新樂【九域志望都縣在定州東北六十里新樂縣在州西南五十里】延朗遂之真定【之往也同光初建北都於鎮州以鎮州為真定府尋廢北都而真定府不廢九域志自新樂縣西南至真定七十里】留趙州刺史朱建豐將兵修新樂城契丹已自它道入定州與王都夜襲新樂破之殺建豐乙丑王晏球張延朗會於行唐【九域志行唐縣在真定府北五十五里】丙寅至曲陽【自行唐西北至曲陽三十許里】王都乘勝悉其衆與契丹五千騎合萬餘人邀晏球等於曲陽丁卯戰于城南晏球集諸將校令之曰王都輕而驕【將即亮翻校戶教翻令魯定翻輕牽正翻】可一戰擒也今日諸君報國之時也悉去弓矢【去羌呂翻】以短兵擊之回顧者斬於是騎兵先進奮檛揮劍直衝其陳大破之僵尸蔽野【用短兵則將士齊致死直衝其陳則敵不及拒北人所恃者弓矢既入其陳皆不得用而檛劍所及不死則傷是以甚敗檛則瓜翻僵居良翻陳讀曰陣】契丹死者過半【過音戈】餘衆北走都與托諾得數騎僅免盧龍節度使趙德鈞邀擊契丹北走者殆無子遺【孑吉列翻單也言無單孑得遺也】 吳遣使求和於楚請苗璘王彦章楚王殷歸之使許德勲餞之德勲謂二人曰楚國雖小舊臣宿將猶在願吳朝勿以措懷【朝直遥翻】必俟衆駒爭皁棧【皁才早翻棧土限翻皁馬櫪也棧以竹木藉之】然後可圖也時殷多内寵嫡庶無别諸子驕奢故德勲語及之【别彼列翻其後馬氏諸子爭國南唐乘而取之卒如許德勲之言然德勲相楚知其將亂不以告戒其主而以語鄰國之人非忠也左傳鄭子太叔謂晉張趯有智然猶在君子之後者正此類也】 六月辛巳高季興復請稱藩于吳【吳徐温議不受高季興稱臣見上卷上年五月】吳進季興爵秦王帝詔楚王殷討之殷遣許德勲將兵攻荆南以其子希範為監軍次沙頭【次沙頭則已逼江陵矣】季興從子雲猛指揮使從嗣單騎造楚壁請與希範挑戰决勝副指揮使廖匡齊出與之鬭拉殺之【從子才用翻造七到翻挑徒了翻廖力救翻拉盧合翻】季興懼明日請和德勲還匡齊贑人也【還從宣翻又如字贑縣屬䖍州贑音紺】王晏球知定州有備未易急攻【易以豉翻】朱弘昭張䖍釗宣言大將畏怯有詔促令攻城晏球不得已乙未攻之殺傷將士三千人【張䖍釗不知鑒定州之事其後急攻鳳翔以致敗國身為亡虜其誤明宗之社稷多矣】 先是詔發西川兵戍夔州【備高季興也先昔薦翻】孟知祥遣左肅邊指揮使毛重威將三千人往頃之知祥奏夔忠萬三州已平請召戍兵還【還從宣翻又如字】以省饋運【孟知祥恐戍兵為唐所留坐自削弱故請召還】帝不許知祥隂使人誘之【誘音酉】重威帥其衆鼓譟逃歸帝命按其罪知祥請而免之【史言唐之威令不行於蜀中】 陜州行軍司馬王宗壽請葬故蜀主王衍【王衍死于長安見二百七十四卷元年陜失冉翻】秋七月贈衍順正公以諸侯禮葬之【王宗壽許州民家子也王建以其同姓録之為子事王衍數直諫衍不聽以至亡國衍死宗壽東遷至澠池聞莊宗遇弑逃入熊耳山至是復出詣京師求衍宗族葬之帝嘉其忠為保義行軍司馬得衍等十八喪葬長安南三趙村】 北面招討使安審通卒【招討之下當有副字】 東都民有犯私麴者留守孔循族之或請聽民造麴而於秋税畝收五錢己未勅從之【按唐初無榷酒之法德宗建中三年初榷天下酒悉令官釀斛收直三千米雖賤不得减二千委州縣綜領醨薄私釀罪有差京師特免榷元和六年京兆府奏榷酒錢除出正酒戶外一切隨兩税青苗據貫均率會昌六年勅楊州八道置榷麴并置官店沽酒代百姓納榷酒并充資助軍用有人私沽酒及置私麴者罪止一身至是以孔循過行酷法勅應三京鄴都諸道州府鄉村人戶於夏秋田苗上每畝納麴錢五文足陌任百姓造麴醖酒供家其錢隨夏秋徵納並不折色其京都父諸道縣鎮坊界及關城草市内應逐年賣官麴酒戶便許自造麴醖酒貨賣應諸處麴務仰十分減八分價錢出賣不得更請官本踏造麴音曲】壬戌契丹復遣酋長特哩衮將七千騎救定州【復扶又翻】王晏球逆戰於唐河北【惕它力翻水經注滱水出代郡靈丘縣高氏山東南過中山上曲陽縣又東過唐縣謂之唐河】大破之甲子追至易州時久雨水漲契丹為唐所俘斬及陷溺死者不可勝數【勝音升】 戊辰以威武節度使王延鈞為閩王契丹北走道路泥濘【濘乃定翻】人馬饑疲入幽州境八月壬戌趙德鈞遣牙將武從諫將精騎邀擊之分兵扼險要生擒特哩衮數百人餘衆散投村落村民以白梃擊之【梃徒頂翻】其得脱歸國者不過數十人自是契丹沮氣不敢輕犯塞【沮在呂翻】 初莊宗徇地河北獲小兒畜之宫中及長【畜吁玉翻長知兩翻】賜姓名李繼陶帝即位縱遣之王都得之使衣黄袍坐堞間【歐史曰帝即位安重誨出繼陶以乞段徊徊亦惡而逐之都使人求得之衣於既翻堞達協翻】謂王晏球曰此莊宗皇帝子也已即帝位公受先朝厚恩曾不念乎【王晏球即杜晏球莊宗之滅梁也晏球以軍降莊宗賜以姓名而用之王都欲以此動晏球】晏球曰公作此小數竟何益吾今教公二策不悉衆决戰則束手出降耳自餘無以求生也 王建立以目不知書請罷判三司不許 乙未吳大赦 吳越王鏐欲立中子傳瓘為嗣【中讀曰仲】謂諸子曰各言汝功吾擇多者而立之【言欲擇功多者立以為嗣】傳瓘兄傅璹傳璙傳璟皆推傳瓘【璹殊六翻璙力弔翻又力小翻璟于景翻又古永翻】乃奏請以兩鎮授傳瓘閏月丁未詔以傳瓘為鎮海鎮東節度使 戊申趙德鈞獻契丹俘特哩衮諸將皆請誅之帝曰此曹皆虜中之驍將殺之則虜絶望不若存之以紓邊患【紓商居翻緩也】乃赦特哩衮酋長五十人置之親衛【後唐蓋倣盛唐之制朝會立仗有親勲翊三衛】餘六百人悉斬之【為契丹屢求特哩衮張本】 契丹遣美楞濟蘇等入貢 初盧文進來降【事見上卷元年】契丹以蕃漢都提舉使張希崇代之為盧龍節度使守平州遣親將以三百騎監之【監工銜翻】希崇本書生為幽州牙將没於契丹【歐史曰劉守光使張希崇戍平州契丹陷平州得之】性和易契丹將稍親信之【易以豉翻將即亮翻】因與其部曲謀南歸部曲泣曰歸固寢食所不忘也然虜衆我寡奈何希崇曰吾誘其將殺之【誘音酉】兵必潰去此去虜帳千餘里比其知而徵兵【比必利翻及也】吾屬去遠矣衆曰善乃先為穽實以石灰【穽才性翻石灰鑿取山石煅之為灰今在處有之】明日召虜將飲醉并從者殺之投諸穽中【從才用翻】其營在城北亟發兵攻之【此所發者漢兵也】契丹衆皆潰去希崇悉舉其所部二萬餘口來奔詔以為汝州刺史【歐史曰以為汝州防禦使】 吳王太后殂【吳主之母王氏也】 九月辛巳荆南敗楚兵于白田執楚岳州刺史李廷規歸于吳【九域志岳州巴陵縣有白田鎮時荆南稱藩干吳敗補賣翻】乙未勅以温韜發諸陵段凝反覆令所在賜死【去年温韜】<br />
<br />
  【流德州段凝流遼州】 己亥以武寧節度使房知温兼荆南行營招討使知荆南行府事分遣中使發諸道兵赴襄陽以討高季興【前年劉訓討荆南不克今復招討之】 辛丑徙慶州防禦使竇廷琬為金州刺史冬十月廷琬據慶州拒命 丙午以横海節度使李從敏兼北面行營副招討使【代安審通也】從敏帝之從子也【從子才用翻】 戊申詔靜難節度使李敬通發兵討竇廷琬【慶州靜難軍巡屬也故使討之難乃旦翻】 王都據定州守備固伺察嚴【伺相吏翻】諸將屢有謀翻城應官軍者皆不果帝遣使者促王晏球攻城晏球與使者聯騎巡城【騎奇計翻】指之曰城高峻如此借使主人聽外兵登城亦非梯衝所及【梯雲梯衝衝車】徒多殺精兵無損於賊如此何為不若食三州之租愛民養兵以俟之彼必内潰帝從之【用兵之術攻城最難然攻城有二術城有外援則須悉力急攻以求必克城無外援則持久以弊之在我者兵力不損而坐收全勝古之善用兵者皆知此術也】 十一月有司請為哀帝立廟詔立廟於曹州【為于偽翻梁太祖開平二年弑唐哀帝于曹州事見二百六十六卷】 平盧節度使晉忠武公霍彦威卒 忠州刺史王雅取歸州【忠州時屬夔州寧江軍西方鄴所部也歸州時屬荆南軍高季興所部也】 庚寅皇子從厚納孔循女為妃循因之得之大梁【時孔循兼留守東都帝在大梁得之者得往也有職守者不得擅離職守今循因嘉禮得至行在所得之本或作得至按唐都洛陽以大梁為東都孔循職守在東都而曰得之大梁者蓋安重誨怒孔循自樞密出為忠武帥兼東都留守時帝在大梁循未得領留守之職今因嫁女得至東都耳以下文促令歸鎮明之可以知矣】厚結王德妃之黨乞留安重誨具奏其事力排之禮畢【嘉禮畢也】促令歸鎮【復歸忠武軍所鎮】 甲午以中書侍郎同平章事王建立同平章事充平盧節度使 丙申上問趙鳳帝王賜人鐵劵何也對曰與之立誓令其子孫長享爵禄耳上曰先朝受此賜者止三人【薛居正五代史莊宗同光二年正月甲寅帝御中興殿面賜郭崇韜鐵劵二月丁亥賜李嗣源鐵劵三年賜朱友謙姓名李繼麟入屬籍賜鐵劵】崇韜繼麟尋皆族滅【二人族滅事見二百七十四卷元年朝直遥翻】朕得脱如毫釐耳【帝為莊宗所猜忌又困於讒事始於二百七十三卷同光三年取鄴都細鎧之時訖于二百七十四卷元年出鄴都在魏縣之日】因歎息久之趙鳳曰帝王心存大信固不必刻之金石也 十二月甲辰李敬周奏拔慶州族竇廷琬 荆南節度使高季興寢疾命其子行軍司馬忠義節度使同平章事從誨權知軍府事丙辰季興卒 【考異曰唐明宗實録天成三年十一月壬午房知温奏高季興卒烈祖實錄亦云乾貞二年十一月季興卒蓋傳聞之誤按陶穀季興神道碑及勃海行年記皆云十二月十五日卒今從之】吳主以從誨為荆南節度使兼侍中【高從誨字遵聖季興長子也】 史館修撰張昭遠上言臣竊見先朝時皇弟皇子皆喜俳優【喜許計翻】入則飾姬妾出則誇僕馬習尚如此何道能賢【言何道而能為賢人也】諸皇子宜精擇師傅令皇子屈身師事之講禮義之經論安危之理古者人君即位則建太子所以明嫡庶之分塞禍亂之源今卜嗣建儲臣未敢輕議至於恩澤賜與之間昏姻省侍之際嫡庶長幼宜有所分示以等威絶其僥冀【分扶問翻塞昔則翻省昔井翻長知兩翻僥堅堯翻】帝賞歎其言而不能用【自梁開平以來至于天成惟張昭遠一疏能以所學而論時事耳不有儒者其能國乎惜其言之不用也史言賞歎而不能用嗚呼帝之賞歎者亦由時人言張昭遠儒學而賞歎之耳豈知所言深有益於人之國哉】 閩王延鈞度民二萬為僧由是閩中多僧 河東節度使北都留守從榮年少驕狠【少詩照翻狠戶懇翻】不親政務帝遣左右素與從榮善者往與之處使從容諷導之【處昌呂翻從千容翻】其人私謂從榮曰河南相公恭謹好善親禮端士有老成之風【從厚時為河南尹故稱之為河南相公端士正士也好音呼到翻】相公齒長【長知兩翻言從榮之年長於從厚也】宜自策勵勿令聲問出河南之下從榮不悦退告步軍都指揮使楊思權曰朝廷之人皆推從厚而短我我其廢乎思權曰相公手握強兵且有思權在何憂因勸從榮多募部曲繕甲兵隂為自固之備【觀從榮之問與楊思權之對其所以求自安者乃所以自危也】又謂帝左右曰君每譽弟而抑其兄【譽音余】我輩豈不能助之邪其人懼以告副留守馮贇贇密奏之【帝遣左右諷導從榮是其密受上指最為親切從榮之不悦楊思權之脅持凡此情狀其人當密以奏聞安得以告馮贇而待贇奏之也此其間必有曲折】帝召思權詣闕以從榮故亦弗之罪也【帝不罪楊思權其後遂為從厚之禍然二子嫌隙已搆雖罪思權亦末如之何矣】四年春正月馮贇入為宣徽使謂執政曰從榮剛僻而輕易【易以䜴翻】宜選重德輔之 王都托諾欲突圍走不得出二月癸丑定州都指揮使馬讓能開門納官軍都舉族自焚擒托諾及契丹二千人【王晏球自去年四月攻王都至是克之】辛亥以王晏球為天平節度使與趙德鈞並加兼侍中【賞王晏球以平王都之功也賞趙德鈞以擒特哩衮功也】托諾至大梁斬於市 樞密使趙敬怡卒 甲子帝發大梁 丁卯門下侍郎同平章事崔協卒於須水【唐初置須水縣貞觀中併入鄭州管城縣九域志鄭州滎陽縣有須水鎮卒音子恤翻】 庚午帝至洛陽【二年冬十月帝如大梁至是還洛陽】 王晏球在定州城下日以私財饗士自始攻至克城未嘗戮一卒三月辛巳晏球入朝帝美其功晏球謝久煩饋運而已【史言王晏球有功而不伐】 皇子右衛大將軍從璨性剛安重誨用事從璨不為之屈【為于偽翻】帝東巡【即謂如大梁時也】以從璨為皇城使從璨與客宴於會節園【會節園在洛陽城中張全義鎮洛歲久私第在會節坊室宇園池為一時巨麗輸之官以為會節園】酒酣戲登御榻【凡御園設御榻遊幸之所御也】重誨奏請誅之丙戌賜從璨死 横山蠻寇邵州【邵州漢為昭陵縣屬長沙國東漢屬長沙零陵二郡又改昭陵為昭陽縣吳立邵陵郡晉武帝改昭陽曰邵陽縣隋廢郡唐置南梁州改為邵州時屬楚境】 楚王殷命其子武安節度副使判長沙府希聲知政事總録内外諸軍事自是國政先歷希聲乃聞於殷【希聲字若訥殷次子也為殺高郁張本】 夏四月庚子朔禁鐵錫錢時湖南專用錫錢銅錢一直錫錢百流入中國法不能禁【馬殷得湖南鑄錫為錢本用之境内其後遂流入中國五代會要同光二年三月勅泉布之弊雜以鉛錫江湖之外盜鑄尤多市肆之間公行無畏因是縱商挾帶舟載往來換易好錢藏貯富室實為蠧弊須有條流宜令京城及諸道于市行使錢内點檢雜惡鉛錫並宜禁斷沿江州縣每有舟船到岸嚴加覺察若私載往來並宜收納天成元年十二月敕行使銅錢之内如聞挾帶鐵錢若不嚴加科流轉恐私加鑄造應中外所使銅錢内鐵鑞錢即宜毁棄不得輒更有行使如違其所使錢不計多少並納入官仍科深罪蓋鐵錫錢之禁舊矣今又申嚴之而不能禁也】 丙午楚六軍副使王環敗荆南兵于石首【敗補賣翻】 初令緣邊置場市党項馬不令詣闕先是党項皆詣闕以貢馬為名國家約其直酬之加以館穀賜與歲費五十餘萬緡有司苦其耗蠧故止之【五代會要曰自上御極以來党項之衆競赴闕下賣馬常賜食於禁廷醉則連袂歌其土風凡將到馬無駑良並云上進雖約給價直然館給賜賚耗蠧為多雖降敕止之竟不能行党底朗翻】 壬子以皇子從榮為河南尹判六軍諸衙事從厚為河東節度使北都留守【兩易二子之任】 契丹寇雲州 甲寅以端明殿學士兵部侍郎趙鳳為門下侍郎同平章事 五月乙酉中書言太常改諡衷帝曰昭宣光烈孝皇帝廟號景宗既稱宗則應入太廟在别廟則不應稱宗【哀帝廟在曹州】乃去廟號【去羌呂翻】帝將祀南郊遣客省使李仁矩以詔諭兩川令西川獻錢一百萬緡東川五十萬緡皆辭以軍用不足西川獻五十萬緡東川獻十萬緡仁矩帝在藩鎮時客將也為安重誨所厚恃恩驕慢至梓州【東川節度治梓州】董璋置宴召之日中不往方擁妓酣飲【妓渠綺翻】璋怒從卒徒執兵入驛立仁矩於階下而詬之曰公但聞西川斬李客省【詬古候翻又許侯翻李客省謂李嚴也斬李嚴見上卷二年】謂我獨不能邪仁矩流涕拜請僅而得免既而厚賂仁矩以謝之【欲以賂絶其口】仁矩還言璋不法未幾【幾居啟翻】帝復遣通事舍人李彦珣詣東川【復扶又翻】入境失小禮璋拘其從者【從才用翻】彦珣奔還【還從宣翻又如字】 高季興之叛也【見上卷二年】其子從誨切諫不聽從誨既襲位謂僚佐曰唐近而吳遠非計也乃因楚王殷以謝罪於唐又遺山南東道節度使安元信書【遺惟季翻】求保奏復修職貢丙申元信以從誨書聞帝許之 契丹寇雲州【一月之間再寇雲州者契丹主耶律德光漸西徙也】 六月戊申復以鄴都為魏州【莊宗同光元年即位于魏州以魏州為興唐府建東京既遷洛同光三年復唐之舊以洛陽為東都改魏州之東京為鄴都今復以為魏州】留守皇城使並停庚申高從誨自稱前荆南行軍司馬歸州刺史上表<br />
<br />
  求内附秋七月甲申以從誨為荆南節度使兼侍中己丑罷荆南招討使【討京南事始上卷二年今以其内附罷兵】 八月吳武昌節度使兼侍中李簡以疾求還江都【揚州治江都縣吳所都也】癸丑卒于採石徐知詢簡壻也擅留簡親兵二千人于金陵【徐知詢時代父温鎮金陵】表薦簡子彦忠代父鎮鄂州【武昌節度使治鄂州】徐知誥以龍武統軍柴再用為武昌節度使知詢怒曰劉崇俊兄之親三世為濠州【吳初用劉金為濠州刺史金卒子仁規代之仁規卒子崇俊代之】彦忠吾妻族獨不得邪 初楚王殷用都軍判官高郁為謀主【馬殷初得潭州即用高郁為謀主】國賴以富強【如收茶征令民種桑以繒纊充賦之類】鄰國皆疾之莊宗入洛殷遣其子希範入貢【見二百七十二卷莊宗同光元年】莊宗愛其警敏曰比聞馬氏當為高郁所奪今有子如此郁安能得之【此言所以間高郁也比毗至翻】高季興亦以流言間郁於殷【間古莧翻】殷不聽乃遣使遺節度副使知政事希聲書【遺惟季翻】盛稱郁功名願為兄弟使者言於希聲曰高公常云馬氏政事皆出高郁此子孫之憂也希聲信之行軍司馬楊昭遂希聲之妻族也謀代郁任日譖之於希聲希聲屢言於殷稱郁奢僭且外交鄰藩請誅之殷曰成吾功業皆郁力也汝勿為此言希聲固請罷其兵柄乃左遷郁行軍司馬郁謂所親曰亟營西山吾將歸老【西山即長沙西岸嶽麓諸山也】猘子漸大能咋人矣【猘征例翻犬強為猘咋鉏陌翻齧也】希聲聞之益怒明日矯以殷命殺郁於府舍【府舍荆南軍府署舍也】牓諭中外誣郁謀叛并誅其族黨至暮殷尚未知是日大霧殷謂左右曰吾昔從孫儒度淮【唐昭宗光啟三年馬殷從孫儒度淮事見二百五十七卷】每殺不辜多致茲異馬步院豈有寃死者乎【時諸鎮皆有馬步司置獄院以鞫囚今大藩亦有兵馬司】明日吏以郁死告殷撫膺大慟曰吾老耄政非已出使我勲舊横罹寃酷【横戶孟翻】既而顧左右曰吾亦何可久處此乎【蓋是時馬殷尸居而已不復能制其子處昌呂翻】 九月上與馮道從容語及年穀屢登【從千容翻屢龍遇翻】四方無事道曰臣常記昔在先皇幕府【謂為河東掌書記時也】奉使中山歷井陘之險【自太原使中山經井陘之道陘音刑】臣憂馬蹶執轡甚謹幸而無失逮至平路放轡自逸俄至顛隕凡為天下者亦猶是也上深以為然上又問道今歲雖豐百姓贍足否道曰農家歲凶則死於流殍【殍被表翻】歲豐則傷於穀賤豐凶皆病者惟農家為然臣記進士聶夷中詩云二月賣新絲五月糶新穀醫得眼下瘡剜却心頭肉語雖鄙俚曲盡田家之情狀【謂絲穀未熟農家艱食先稱貸以自給至於賣絲糶穀僅足以償債耳聶尼輒翻糶它弔翻剜烏丸翻】農於四人之中最為勤苦【士農工商是謂四民唐避太宗諱率謂民為人】人主不可不知也上悦命左右録其詩常諷誦之 鄜州兵戍東川者歸本道【鄜音夫】董璋擅留其壯者選羸老歸之【羸倫為翻】仍收其甲兵 癸巳西川右都押牙孟容弟為資州税官坐自盜抵死【律監臨自盜贓重者至死抵至也】觀察判官馮瑑中門副使王處回為之請【瑑柱兖翻為于偽翻】孟知祥曰雖吾弟犯法亦不可貸况它人乎 吳越王鏐居其國好自大朝廷使者曲意奉之則贈遺豐厚不然則禮遇疎薄【好呼到翻遺惟季翻下同】嘗遺安重誨書辭禮頗倨【薛史曰錢鏐致書安重誨云吳越國王致書干某官執事不叙寒暄重誨怒其無禮】帝遣供奉官烏昭遇 【考畢曰吳越備史十國紀年皆云監門衛上將軍蓋借官耳今從實録等諸書】韓玫使吳越【玫莫杯翻】昭遇與玫有隙使還【使疏吏翻還從宣翻又如字】玫奏昭遇見鏐稱臣拜舞謂鏐為殿下及私以國事告鏐安重誨奏賜昭遇死癸巳制鏐以太師致仕自餘官爵皆削之凡吳越進奏官使者綱吏令所在繫治之【治直之翻】鏐令子傳瓘等上表訟寃皆不省【省悉井翻】 初朔方節度使韓洙卒【梁均王乾化四年韓洙嗣鎮朔方】弟澄為留後未幾定遠軍使李匡賓聚黨據保靜鎮作亂【幾居豈翻保靜隋之弘靜縣也唐神龍元年改曰安靜至德元載改曰保靜縣屬靈州宋白保靜鎮在黄河北岸】朔方不安冬十月丁酉韓澄遣使齎絹表乞朝廷命帥【帥所類類】前磁州刺史康福善胡語上退朝多召入便殿訪以時事福以胡語對安重誨惡之【惡其以胡語奏事在左右者莫之曉也惡烏路翻】常戒之曰康福汝但妄奏事會當斬汝福懼求外補重誨以靈州深入胡境為帥者多遇害戊戌以福為朔方河西節度使【唐之盛時河西節度使治凉州與朔方隴西並為緣邊大鎮肅代以後淪陷宣宗大中間收復然隔以吐蕃党項朝廷懸屬而已至于唐末以朔方兼節度河西然亦聲勢不接趙珣聚米圖經靈州西至凉州九百里】福見上涕泣辭之上命重誨為福更它鎮【為干偽翻更工行翻】重誨曰福自刺史無功建節尚復何求【復扶又翻】且成命已行難以復改上不得已謂福曰重誨不肯非朕意也福辭行上遣將軍牛知柔河中都指揮使衛審等將兵萬人衛送之審徐州人也【與都同】 辛亥割閬果二州置保寧軍壬子以内客省使李仁矩為節度使【欲以制兩川也為李仁矩敗没張本按職官分紀五代有内客省使客省使副使各一官通鑑於天成元年三月書客省使李仁矩今書内客省使豈自客省使陞為内客省使邪】 先是西川常發芻糧饋峽路【先悉薦翻】孟知祥辭以本道兵自多難以奉它鎮【峽路時别為寧江軍故云然】詔不許屢督之甲寅知祥奏稱財力乏不奉詔 吳諸道副都統鎮海寧國節度使兼侍中徐知詢自以握兵據上流【金陵在廣陵上流】意輕徐知誥數與知誥爭權内相猜忌【數所角翻】知誥患之内樞密使王令謀曰公輔政日久挾天子以令境内誰敢不從知詢年少恩信未洽於人無能為也【少詩照翻】知詢待諸弟薄諸弟皆怨之徐玠知知詢不可輔反持其短以附知誥【徐玠本勸徐温以知詢代知誥者也其事見本卷上年十月】吳越王鏐遺知詢金玉鞍勒器皿皆飾以龍鳳知詢不以為嫌乘用之【錢鏐以此間徐知詢知詢不之覺其庸昧如此路振九國志以為錢弘佐所遺非也】知詢典客周廷望說知詢曰公誠能捐寶貨以結朝中勲舊使皆歸心於公則彼誰與處【說式芮翻朝直遥翻處昌呂翻彼謂徐知誥也】知詢從之使廷望如江都諭意【諭音喻】廷望與知誥親吏周宗善密輸欵於知誥【欵誠也】亦以知誥隂謀告知詢【周廷望處人兄弟之間而反覆兩端固取死之道也】知詢召知誥詣金陵除父温喪知誥稱吳主之命不許周宗謂廷望曰人言侍中有不臣七事【徐知詢之代父鎮金陵也加侍中故以稱之】宜亟入謝【誘之入朝徐知誥之計也】廷望還以告知詢十一月知詢入朝知誥留知詢為統軍領鎮海節度使遣右雄武都指揮使柯厚徵金陵兵還江都【姓譜柯姓吳公子柯盧之後又拓拔興諸姓有柯拔氏改為柯氏】知誥自是始專吳政【史言徐知誥之篡事至此方成】知詢責知誥曰先王違世【先王謂徐温也】兄為人子初不臨喪可乎知誥曰爾挺劍待我【挺待鼎翻拔也】我何敢往爾為人臣畜乘輿服御物亦可乎【畜敕六翻乘繩證翻謂知詢用錢鏐所遺龍鳳飾鞍勒器皿也天子服用之物謂之乘輿物】知詢又以廷望所言詰知誥【詰去吉翻】知誥曰以爾所為告我者亦廷望也遂斬廷望 壬辰吳主加尊號曰睿聖文明光孝皇帝大赦改元大和 康福行至方渠羌胡出兵邀福福擊走之至青剛峽【自方渠槖駞路出青岡峽過旱海至靈州趙珣聚米圖經曰環州洪德寨歸德青剛兩川歸德川在洪德東透入鹽州青剛川在洪德西北本靈州大路自此過美利寨入浦洛河至耀德清邊鎮入靈州自過美利寨後漸入平夏經旱海中難得水泉】遇吐蕃野利大蟲二族數千帳皆不覺唐兵至福遣衛審掩擊大破之殺獲殆盡由是威聲大振遂進至靈州自是朔方始受代 十二月吳加徐知誥兼中書令領寧國節度使【徐知誥奪知詢寧國節而自領之】知誥召徐知詢飲以金鍾酌酒賜之曰願弟壽千歲知詢疑有毒引它器均之跽獻知誥曰願與兄各享五百歲【跽其几翻䠆跽也】知誥變色左右顧不肯受知詢捧酒不退左右莫知所為伶人申漸高徑前為詼諧語掠二酒合飲之【不以禮取之為掠合音閤】懷金鍾趨出知誥密遣人以良藥解之己腦潰而卒 【考異曰鄭文寶南唐近事烈祖曲宴便殿引酖賜周本本疑而不飲佯醉别引一巵均酒之半跪捧而進曰陛下千萬歲陛下若不飲此非君臣同心同德之義也臣不敢奉詔上色變無言久之左右皆相顧流汗莫知所從伶倫申漸高有機智者竊諭其旨乃乘詼諧盡併兩盞以飲之内杯于懷中亟趍而出上密使親信持良藥詣其私第解之已不及矣漸高腦潰而卒江表志烈祖曲宴引金鍾賜知詢酒曰願我弟百千長壽知詢疑懼引它器均之曰願與兄各享五百歲知誥不飲久之樂工申漸高乘詼諧併而飲之至家腦潰而卒二書皆出文寶而不同乃爾按知誥既即位欲除周本自應多方不須如此云酖知詢近是今從之】 奉國節度使知建州王延稟稱疾退居里第請以建州授其子繼雄庚子詔以繼雄為建州刺史【時王延稟既與王延鈞弑其君延翰兵強權重建州又居福州上流勢陵延鈞故不復稟命于延鈞而專達洛陽】 安重誨既以李仁矩鎮閬州使與綿州刺史武䖍裕皆將兵赴治【赴治者赴治所也】䖍裕帝之故吏重誨之外兄也重誨使仁矩詗董璋反狀【詗火迥翻又翾正翻】仁矩增飾而奏之朝廷又使武信節度使夏魯奇治遂州城隍【治直之翻】繕甲兵益兵戍之璋大懼時道路傳言又將割綿龍為節鎮孟知祥亦懼【分閬遂為節鎮欲以制東川也故董璋懼綿州逼近成都而前州又鄧艾入蜀之道也武䖍裕既刺綿州是亦有分鎮之漸矣重以傳聞故孟知祥亦懼】璋素與知祥有隙未嘗通問至是璋遣使詣成都請為其子娶知祥女【為于偽翻】知祥許之謀併力以拒朝廷【為兩川連兵攻陷遂閬張本】<br />
<br />
  資治通鑑卷二百七十六  <br>
   </div> 

<script src="/search/ajaxskft.js"> </script>
 <div class="clear"></div>
<br>
<br>
 <!-- a.d-->

 <!--
<div class="info_share">
</div> 
-->
 <!--info_share--></div>   <!-- end info_content-->
  </div> <!-- end l-->

<div class="r">   <!--r-->



<div class="sidebar"  style="margin-bottom:2px;">

 
<div class="sidebar_title">工具类大全</div>
<div class="sidebar_info">
<strong><a href="http://www.guoxuedashi.com/lsditu/" target="_blank">历史地图</a></strong>  
<a href="http://www.880114.com/" target="_blank">英语宝典</a>  
<a href="http://www.guoxuedashi.com/13jing/" target="_blank">十三经检索</a> 
<br><strong><a href="http://www.guoxuedashi.com/gjtsjc/" target="_blank">古今图书集成</a></strong> 
<a href="http://www.guoxuedashi.com/duilian/" target="_blank">对联大全</a> <strong><a href="http://www.guoxuedashi.com/xiangxingzi/" target="_blank">象形文字典</a></strong> 

<br><a href="http://www.guoxuedashi.com/zixing/yanbian/">字形演变</a>  <strong><a href="http://www.guoxuemi.com/hafo/" target="_blank">哈佛燕京中文善本特藏</a></strong>
<br><strong><a href="http://www.guoxuedashi.com/csfz/" target="_blank">丛书&方志检索器</a></strong> <a href="http://www.guoxuedashi.com/yqjyy/" target="_blank">一切经音义</a>  

<br><strong><a href="http://www.guoxuedashi.com/jiapu/" target="_blank">家谱族谱查询</a></strong>  <strong><a href="http://shufa.guoxuedashi.com/sfzitie/" target="_blank">书法字帖欣赏</a></strong> 
<br>

</div>
</div>


<div class="sidebar" style="margin-bottom:0px;">

<font style="font-size:22px;line-height:32px">QQ交流群9:489193090</font>


<div class="sidebar_title">手机APP 扫描或点击</div>
<div class="sidebar_info">
<table>
<tr>
	<td width=160><a href="http://m.guoxuedashi.com/app/" target="_blank"><img src="/img/gxds-sj.png" width="140"  border="0" alt="国学大师手机版"></a></td>
	<td>
<a href="http://www.guoxuedashi.com/download/" target="_blank">app软件下载专区</a><br>
<a href="http://www.guoxuedashi.com/download/gxds.php" target="_blank">《国学大师》下载</a><br>
<a href="http://www.guoxuedashi.com/download/kxzd.php" target="_blank">《汉字宝典》下载</a><br>
<a href="http://www.guoxuedashi.com/download/scqbd.php" target="_blank">《诗词曲宝典》下载</a><br>
<a href="http://www.guoxuedashi.com/SiKuQuanShu/skqs.php" target="_blank">《四库全书》下载</a><br>
</td>
</tr>
</table>

</div>
</div>


<div class="sidebar2">
<center>


</center>
</div>

<div class="sidebar"  style="margin-bottom:2px;">
<div class="sidebar_title">网站使用教程</div>
<div class="sidebar_info">
<a href="http://www.guoxuedashi.com/help/gjsearch.php" target="_blank">如何在国学大师网下载古籍?</a><br>
<a href="http://www.guoxuedashi.com/zidian/bujian/bjjc.php" target="_blank">如何使用部件查字法快速查字?</a><br>
<a href="http://www.guoxuedashi.com/search/sjc.php" target="_blank">如何在指定的书籍中全文检索?</a><br>
<a href="http://www.guoxuedashi.com/search/skjc.php" target="_blank">如何找到一句话在《四库全书》哪一页?</a><br>
</div>
</div>


<div class="sidebar">
<div class="sidebar_title">热门书籍</div>
<div class="sidebar_info">
<a href="/so.php?sokey=%E8%B5%84%E6%B2%BB%E9%80%9A%E9%89%B4&kt=1">资治通鉴</a> <a href="/24shi/"><strong>二十四史</strong></a>&nbsp; <a href="/a2694/">野史</a>&nbsp; <a href="/SiKuQuanShu/"><strong>四库全书</strong></a>&nbsp;<a href="http://www.guoxuedashi.com/SiKuQuanShu/fanti/">繁体</a>
<br><a href="/so.php?sokey=%E7%BA%A2%E6%A5%BC%E6%A2%A6&kt=1">红楼梦</a> <a href="/a/1858x/">三国演义</a> <a href="/a/1038k/">水浒传</a> <a href="/a/1046t/">西游记</a> <a href="/a/1914o/">封神演义</a>
<br>
<a href="http://www.guoxuedashi.com/so.php?sokeygx=%E4%B8%87%E6%9C%89%E6%96%87%E5%BA%93&submit=&kt=1">万有文库</a> <a href="/a/780t/">古文观止</a> <a href="/a/1024l/">文心雕龙</a> <a href="/a/1704n/">全唐诗</a> <a href="/a/1705h/">全宋词</a>
<br><a href="http://www.guoxuedashi.com/so.php?sokeygx=%E7%99%BE%E8%A1%B2%E6%9C%AC%E4%BA%8C%E5%8D%81%E5%9B%9B%E5%8F%B2&submit=&kt=1"><strong>百衲本二十四史</strong></a>  <a href="http://www.guoxuedashi.com/so.php?sokeygx=%E5%8F%A4%E4%BB%8A%E5%9B%BE%E4%B9%A6%E9%9B%86%E6%88%90&submit=&kt=1"><strong>古今图书集成</strong></a>
<br>

<a href="http://www.guoxuedashi.com/so.php?sokeygx=%E4%B8%9B%E4%B9%A6%E9%9B%86%E6%88%90&submit=&kt=1">丛书集成</a> 
<a href="http://www.guoxuedashi.com/so.php?sokeygx=%E5%9B%9B%E9%83%A8%E4%B8%9B%E5%88%8A&submit=&kt=1"><strong>四部丛刊</strong></a>  
<a href="http://www.guoxuedashi.com/so.php?sokeygx=%E8%AF%B4%E6%96%87%E8%A7%A3%E5%AD%97&submit=&kt=1">說文解字</a> <a href="http://www.guoxuedashi.com/so.php?sokeygx=%E5%85%A8%E4%B8%8A%E5%8F%A4&submit=&kt=1">三国六朝文</a>
<br><a href="http://www.guoxuedashi.com/so.php?sokeytm=%E6%97%A5%E6%9C%AC%E5%86%85%E9%98%81%E6%96%87%E5%BA%93&submit=&kt=1"><strong>日本内阁文库</strong></a> <a href="http://www.guoxuedashi.com/so.php?sokeytm=%E5%9B%BD%E5%9B%BE%E6%96%B9%E5%BF%97%E5%90%88%E9%9B%86&ka=100&submit=">国图方志合集</a> <a href="http://www.guoxuedashi.com/so.php?sokeytm=%E5%90%84%E5%9C%B0%E6%96%B9%E5%BF%97&submit=&kt=1"><strong>各地方志</strong></a>

</div>
</div>


<div class="sidebar2">
<center>

</center>
</div>
<div class="sidebar greenbar">
<div class="sidebar_title green">四库全书</div>
<div class="sidebar_info">

《四库全书》是中国古代最大的丛书,编撰于乾隆年间,由纪昀等360多位高官、学者编撰,3800多人抄写,费时十三年编成。丛书分经、史、子、集四部,故名四库。共有3500多种书,7.9万卷,3.6万册,约8亿字,基本上囊括了古代所有图书,故称“全书”。<a href="http://www.guoxuedashi.com/SiKuQuanShu/">详细>>
</a>

</div> 
</div>

</div>  <!--end r-->

</div>
<!-- 内容区END --> 

<!-- 页脚开始 -->
<div class="shh">

</div>

<div class="w1180" style="margin-top:8px;">
<center><script src="http://www.guoxuedashi.com/img/plus.php?id=3"></script></center>
</div>
<div class="w1180 foot">
<a href="/b/thanks.php">特别致谢</a> | <a href="javascript:window.external.AddFavorite(document.location.href,document.title);">收藏本站</a> | <a href="#">欢迎投稿</a> | <a href="http://www.guoxuedashi.com/forum/">意见建议</a> | <a href="http://www.guoxuemi.com/">国学迷</a> | <a href="http://www.shuowen.net/">说文网</a><script language="javascript" type="text/javascript" src="https://js.users.51.la/17753172.js"></script><br />
  Copyright &copy; 国学大师 古典图书集成 All Rights Reserved.<br>
  
  <span style="font-size:14px">免责声明:本站非营利性站点,以方便网友为主,仅供学习研究。<br>内容由热心网友提供和网上收集,不保留版权。若侵犯了您的权益,来信即刪。scp168@qq.com</span>
  <br />
ICP证:<a href="http://www.beian.miit.gov.cn/" target="_blank">鲁ICP备19060063号</a></div>
<!-- 页脚END --> 
<script src="http://www.guoxuedashi.com/img/plus.php?id=22"></script>
<script src="http://www.guoxuedashi.com/img/tongji.js"></script>

</body>
</html>
