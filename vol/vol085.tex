










 


 
 


 

  
  
  
  
  





  
  
  
  
  
 
  

  

  
  
  



  

 
 

  
   




  

  
  


    資治通鑑卷八十五   宋 司馬光 撰

  胡三省 音註

  晉紀七【起昭陽大淵獻盡閼逢困敦凡二年】

  孝惠皇帝中之下

  太安二年春正月李特潛渡江擊羅尚水上軍皆散走【郫水上軍也】蜀郡太守徐儉以少城降【少詩昭翻降戶江翻下同】特入據之惟取馬以供軍餘無侵掠赦其境内改元建初 【考異曰帝紀太安元年五月特自號大將軍載記太安元年特稱大將軍改元後魏書李雄傳云昭帝七年特稱大將軍號年建初昭帝七年太安元年也祖孝徵脩文殿御覽云太安二年特大赦改元建初元年特見殺三十國晉春秋云太安二年正月特僭位改元今從御覽等書】羅尚保太城遣使求和於特蜀民相聚為塢者皆送欵於特特遣使就撫之【使疏吏翻】以軍中糧少【少詩沼翻】乃分六郡流民於諸塢就食李流言於特曰諸塢新附人心未固宜質其大姓子弟【質音致】聚兵自守以備不虞又與特司馬上官惇書曰納絳如受敵不可易也【恐其詐降當嚴為之備如待敵然易以豉翻】前將軍雄亦以為言特怒曰大事已定但當安民何為更逆加疑忌使之離叛乎朝廷遣荆州刺史宗岱建平太守孫阜帥水軍三萬以救羅尚【朝直遥翻守式又翻帥讀曰率下同】岱以阜為前鋒進逼德陽特遣李蕩及蜀郡太守李璜就德陽太守任臧共拒之【李特蓋又分廣漢立德陽郡任音壬下同】岱阜軍勢甚盛諸塢皆有貳志益州兵曹從事蜀郡任叡言於尚曰李特散衆就食驕怠無備此夭亡之時也宜密約諸塢刻期同發内外擊之破之必矣尚使叡夜縋出城【任音壬縋馳偽翻】宣旨於諸塢期以二月十日同擊特叡因詣特詐降特問城中虛實叡曰糧儲將盡但餘貨帛耳叡求出省家特許之遂還報尚 【考異曰載記作任明羅尚傳作任銳今從華陽國志省悉景翻】二月尚遣兵掩襲特營諸塢皆應之特兵大敗斬特及李輔李遠皆焚尸傳首洛陽流民大懼李蕩李雄收餘衆還保赤祖【赤祖地名當在緜竹東祖子邪翻】流自稱大將軍大都督益州牧保東營蕩雄保北營孫阜破德陽獲寋碩【寋姓也與蹇同】任臧退屯涪陵【此涪陵乃漢廣漢郡之涪縣晉梓潼郡之涪城縣非涪陵郡之涪陵廣漢梓潼之涪今緜州今人猶謂緜州為涪陵涪陵郡之涪陵則今涪州涪陵縣也】三月羅尚遣督護何冲常深攻李流涪陵民藥紳亦起兵攻流流與李驤拒紳何冲乘虛攻北營氐符成隗伯在營中叛應之蕩母羅氏擐甲拒戰【擐音䆠】伯手刃傷其目羅氏氣益壯會流等破深紳引兵還與冲戰大破之成伯率其黨突出詣尚流等乘勝進抵成都尚復閉城自守【復扶又翻】蕩馳馬逐北中矛而死【中竹仲翻】朝廷遣侍中劉沈假節統羅尚許雄等軍【羅尚帥益州兵許雄帥梁州兵沈持林翻】討李流行至長安河間王顒留沈為軍師遣席薳代之【薳羽委翻】李流以李特李蕩繼死宗岱孫阜將至甚懼李含勸流降流從之李驤李雄迭諫不納夏五月流遣其子世及含子胡為質於阜軍【質音致】胡兄離為梓潼太守聞之自郡馳還欲諫不及退與雄謀襲阜軍雄曰為今計當如是而二翁不從柰何【二翁謂李流李含也】離曰當劫之耳雄大喜乃共說流民曰【說輸芮翻】吾屬前已殘暴蜀民今一旦束手便為魚肉惟有同心襲阜以取富貴耳衆皆從之雄遂與離襲擊阜軍大破之會宗岱卒於墊江【墊音疊墊江縣自漢來屬巴郡唐為合州之地】荆州軍遂退流甚慙由是奇雄才軍事悉以任之 新野莊王歆為政嚴急失蠻夷心義陽蠻張昌聚黨數千人欲為亂【劉昫曰義陽本漢平氏縣之義陽鄉魏文帝黄初中分義陽縣蓋治石城後分南陽郡立義陽郡治安昌城領安昌平林平氏陽平春闕五縣唐為闕】荆州以壬午詔書發武勇赴益州討李流號壬午兵民憚遠征皆不欲行詔書督遣嚴急所經之界停留五日者二千石免官由是郡縣官長皆親出驅逐【長知兩翻】展轉不遠輒復屯聚為羣盜【復扶又翻下同】時江夏大稔民就食者數千口張昌因之誑惑百姓【夏戶雅翻誑居况翻】更姓名曰李辰【更工衡翻】募衆於安陸石巖山【晉書張昌傳云石巖山去安陸郡八十里水經注溳水過江夏安陸縣西又南逕石巖山北今德安府南十里有石巖山】諸流民及避戍役者多從之太守弓欽遣兵討之不勝【姓譜弓姓魯大夫叔弓之後余按孔子弟子有仲弓又有馯臂子弓而獨以魯叔弓後殊為未通】昌遂攻郡欽兵敗與部將朱伺犇武昌【伺相吏翻】歆遣騎督靳滿討之滿復敗走【騎奇寄翻靳居焮翻】昌遂據江夏【杜佑曰漢江夏郡故城在安州雲夢縣柬南】造妖言云當有聖人出為民主得山都縣吏邱沈【山都縣漢屬南陽郡晉屬襄陽郡其地屬唐襄州穀城縣界杜佑曰山都縣故城在襄州義清縣東南沈持林翻】更其姓名曰劉尼詐云漢後奉以為天子曰此聖人也昌自為相國詐作鳳皇玉璽之瑞【璽斯氏翻】建元神鳳郊祀服色悉依漢故事有不應募者族誅之士民莫敢不從又流言江淮已南皆反官軍大起當悉誅之互相扇動人情惶懼江沔間所在起兵以應昌【沔彌兖翻】旬月間衆至三萬皆著絳㡌【著陟畧翻】以馬尾作髯詔遣監軍華宏討之敗于障山【今安陸縣東四十里有障山監工銜翻華戶化翻】歆上言妖賊犬羊萬計絳頭毛面挑刀走戟其鋒不可當【挑徒了翻挑刀舞刀也今鄉落悍民兩手運雙刀坐作進退為擊刺之勢擲刀空中高一二丈以手接之又善舞戟左犇右赴為刺敵之勢又環身盤戟回轉如縈又以戟矜柱地跳過矜上特為儇捷此所謂走戟也妖於嬌翻】請臺勑諸軍三道救助朝廷以屯騎校尉劉喬為豫州刺史寧朔將軍沛國劉宏為荆州刺史【寧朔將軍始見於此】又詔河間王顒遣雍州刺史劉沈將州兵萬人并征西府五千人出藍田關以討昌【藍田關在京兆藍田縣即秦之嶢關也雍於用翻沈持林翻】顒不奉詔沈自領州兵至藍田顒又逼奪其衆於是劉喬屯汝南劉宏及前將軍趙驤平南將軍羊伊屯宛【宛於元翻】昌遣其將黄林帥二萬人向豫州劉喬擊却之【帥讀曰率】初歆與齊王冏善【事見上卷永寧元年】冏敗歆懼自結於大將軍頴及張昌作亂歆表請討之時長沙王乂已與頴有隙疑歆與頴連謀不聽歆出兵昌衆日盛從事中郎孫洵謂歆曰公為岳牧【古有四岳十二牧各統其方諸侯之國故後人謂專方面者為岳牧】受閫外之託拜表輒行有何不可而使姦凶滋蔓禍釁不測豈藩翰王室鎮靜方夏之義乎【毛萇曰藩樊也籬也翰榦也夏戶雅翻】歆將出兵王綏曰昌等小賊偏禆自足制之何必違詔命親矢石也昌至樊城歆乃出拒之衆潰為昌所殺詔以劉宏代歆為鎮南將軍都督荆州諸軍事六月宏以南蠻長史陶侃為大都護【南蠻校尉有長史司馬】參軍蒯恒為義軍督護【義軍蓋民兵也督護之官蓋創置於此時蒯苦怪翻】牙門將皮初為都戰帥進據襄陽【杜佑曰襄陽漢中廬縣也】張昌并軍圍宛敗趙驤軍殺羊伊劉宏退屯梁【梁縣屬汝南郡唐為汝州治所敗補邁翻】昌進攻襄陽不克 李雄攻殺汶山太守陳圖【汶音民 考異曰華陽國志作陳旹今從載記】遂取郫城【郫縣屬蜀郡李膺益州記郫縣故城在今縣北劉昫曰唐益州温江縣漢郫縣地郫音疲】秋七月李流徙屯郫蜀民皆保險結塢或南入寧州或東下荆州城邑皆空野無煙火流虜掠無所得士衆饑乏唯涪陵千餘家依青城山處士范長生【青城山在汶山郡都安縣今在永康軍青城縣北三十二里杜光庭作青城山記曰岷山連峯接岫千里不絶青城乃第一峯也范長生涪陵人率衆保之處昌呂翻考異曰華陽國志作范賢今從載記】平西參軍涪陵徐轝說羅尚求為汶山太守邀結長生與共討流【尚為平西將軍以徐轝為參軍 考異曰華陽國志作徐輿今從載記】尚不許轝怒出降於流【降戶江翻】流以轝為安西將軍轝說長生使資給流軍糧【說輸芮翻下含說同】長生從之流軍由是復振 初李含以長沙王乂微弱必為齊王冏所殺因欲以為冏罪而討之遂廢帝立大將軍頴以河間王顒為宰相已得用事既而冏為乂所殺【事見上卷上年】頴顒猶守藩不如所謀頴恃功驕奢百度弛廢甚於冏時猶嫌乂在内不得逞其欲欲去之【去羌呂翻】時皇甫商復為乂參軍商兄重為秦州刺史含說顒曰商為乂所任重終不為人用宜早除之【商含不平事見上卷元年】可表遷重為内職因其過長安執之重知之露檄上尚書【上時掌翻】發隴上兵以討含【自隴以西六郡統於秦州】乂以兵方少息遣使詔重罷兵徵含為河南尹 【考異曰含傳云河間王顒表含為河南尹今從皇甫重傳】含就徵而重不奉詔顒遣金城太守游楷隴西太守韓稚等合四郡兵攻之【秦州刺史鎮冀城】顒密使含與侍中馮蓀中書令卞粹謀殺乂皇甫商以告乂收含蓀粹殺之【蓀音孫】驃騎從事琅邪諸葛玟前司徒長史武邑牽秀皆出犇鄴【驃匹妙翻從事從事中郎也玟莫杯翻武邑縣前漢屬信都郡後漢晉屬安平國武帝分立武邑郡唐為縣屬冀州】張昌黨石氷寇揚州敗刺史陳徽【敗補邁翻】諸郡盡没又攻破江州别將陳貞攻武陵零陵豫章武昌長沙皆陷之臨淮人封雲起兵寇徐州以應氷【江州時治豫章漢置臨淮郡章帝以合下邳國晉太康元年復置臨淮郡姓譜封姓夏封父之後將即亮翻】於是荆江徐揚豫五州之境多為昌所據昌更置牧守【更工衡翻守式又翻】皆桀盜小人專以劫掠為務劉宏遣陶侃等攻昌於竟陵【竟陵縣屬江夏郡孫宗鑑曰自蔡州南至信陽軍始有山路迤邐至安陸又兩驛至復州皆平地南至大江並無邱陵之阻渡江至石首始有淺山謂之竟陵者陵至此而竟謂之石首石至此而首也古竟陵今復州】劉喬遣其將李楊等向江夏侃等屢與昌戰大破之前後斬首數萬級昌逃于下儁山其衆悉降【長沙下儁縣之山也師古曰儁字兖翻又辭兖翻 考異曰帝紀八月庚申劉宏及張昌戰于清水斬之昌傳云昌敗竄于下儁山明年秋禽斬之按宏斬張奕表云張昌姦黨初平昌未梟首故從昌本傳】初陶侃少孤貧【少詩照翻】為郡督郵長沙太守萬嗣過廬江見而異之命其子結友而去後察孝亷至洛陽豫章國郎中令楊薦之於顧榮【帝弟熾封豫章王丁角翻】侃由是知名既克張昌劉宏謂侃曰吾昔為羊公參軍【羊公謂羊祜也】謂吾後當居身處【晉人多自謂為身】今觀卿必繼老夫矣宏之退屯於梁也征南將軍范陽王虓遣前長水校尉張奕領荆州【范陽王虓鎮豫州】宏至奕不受代舉兵拒宏宏討奕斬之時荆部守宰多缺【守式又翻下同】宏請補選詔許之宏叙功銓德【銓量也選也】隨才授任人皆服其公當【當丁浪翻】宏表皮初補襄陽太守【姓譜皮姓樊仲皮之後】朝廷以初雖有功而望淺更以宏壻前東平太守夏侯陟為襄陽太守宏下教曰夫治一國者宜以一國為心必若親姻然後可用則荆州十郡【按晉志荆州統二十二郡時已分襄陽武昌安成三郡屬江州尚統十九郡又分新城魏興上庸三郡屬梁州尚統十六郡至懷帝分長沙衡陽湘東零陵邵陵桂陽六郡屬湘州此時荆州猶統十一郡此蓋言當時缺守者十郡也治直之翻】安得十女壻然後為政哉乃表陟姻親舊制不得相監【監古銜翻】皮初之勲宜見酬報詔聽之宏於是勸課農桑寛刑省賦公私給足百姓愛悅 河間王顒聞李含等死即起兵討長沙王乂大將軍頴上表請討張昌許之聞昌已平因欲與顒共攻乂盧志諫曰公前有大功而委權辭寵時望美矣【事見上卷永寧元年】今若頓軍關外【關外謂郊關之外】文服入朝此霸主之事也【朝直遥翻】參軍魏郡邵續曰人之有兄弟如左右手明公欲當天下之敵而先去其一手可乎【去羌呂翻】頴皆不從八月顒頴共表乂論功不平與右僕射羊玄之左將軍皇甫商專擅朝政殺害忠良【謂殺李含等朝直遥翻】請誅玄之商遣乂還國詔曰顒敢舉大兵内向京輦吾當親率六軍以誅姦逆其以乂為太尉都督中外諸軍事以禦之【考異曰帝紀太安元年十二月乂誅齊王冏即以乂為太尉都督中外晉春秋二年七月顒頴起兵乃以乂為太尉都督以討之按齊王死後頴懸執朝政乂未應都督中外又顒見為太尉乂不應更為太尉今從晉春秋】顒以張方為都督將精兵七萬自函谷東趨洛陽【將即亮翻趨七喻翻】頴引兵屯朝歌以平原内史陸機為前將軍前鋒都督督北中郎將王粹冠軍將軍牽秀【沈約曰楚懷王以宋義為卿子冠軍冠軍之號自此始魏以文欽為冠軍將軍冠古玩翻】中護軍石超等軍二十餘萬南向洛陽機以羇旅事頴一旦頓居諸將之右王粹等心皆不服白沙督孫惠【白沙在鄴城東南】與機親勸機讓都督於粹機曰彼將謂吾首鼠兩端【首鼠兩端漢田蚡語服䖍曰首鼠一前一却也陸佃埤雅曰舊說鼠性疑出穴多不果故持兩端謂之首鼠】適所以速禍也遂行頴列軍自朝歌至河橋鼓聲聞數百里【河橋即富平津河橋聞音問】乙丑帝如十三里橋【橋在洛城西去城十三里因以為名】太尉乂使皇甫商將萬餘人拒張方於宜陽己巳帝還軍宣武場【水經注大夏門東宣武觀憑城結構南望天淵池北矚宣武場場西故賈充宅】庚午舍于石樓九月丁丑屯于河橋壬子張方襲皇甫商敗之【敗補邁翻下敗牽同】甲申帝軍于芒山丁亥帝幸偃師【偃師縣漢屬河南郡晉省隋復置在洛城東北】辛卯舍于豆田【據晉書五行志洛陽城東有豆田壁】大將軍頴進屯河南阻清水為壘【此河南謂黄河之南非河南縣也清水蓋清濟之水】癸巳羊玄之憂懼而卒帝旋軍城東丙申幸緱氏擊牽秀走之【緱工侯翻】大赦張方入京城大掠死者萬計 李流疾篤謂諸將曰驍騎仁明固足以濟大事然前軍英武殆天所相可共受事於前軍【李特以弟驤為驍騎將軍少子雄為前將軍相息亮翻驍堅堯翻】流卒衆推李雄為大都督大將軍益州牧治郫城【郫音皮】雄使武都朴泰紿羅尚【朴姓也板楯七姓蠻之種也孫盛曰朴音浮】使襲郫城云已為内應尚使隗伯將兵攻郫泰約舉火為應李驤伏兵於道泰出長梯於外隗伯兵見火起爭緣梯上【隗五罪翻上時掌翻】驤縱兵擊大破之追犇夜至城下詐稱萬歲曰已得郫城矣入少城尚乃覺之退保太城隗伯創甚雄生獲之赦不殺【隗伯本亦流民之豪帥叛歸羅尚創初良翻】李驤攻犍為斷尚運道【犍居言翻斷丁管翻】獲太守龔恢殺之 石超進逼緱氏【緱工侯翻】冬十月壬寅帝還宫丁未敗牽秀於東陽門外【水經注曰東陽門漢洛陽城之中東門也敗補邁翻】大將軍頴遣將軍馬咸助陸機戊申太尉乂奉帝與機戰于建春門【水經注建春門漢雒城之上東門也穀水逕其前水上有石橋 考異曰陸機傳云戰于鹿苑今從帝紀】乂司馬王瑚使數千騎繫戟於馬以突咸陳【騎奇寄翻陳讀曰陣】咸軍亂執而斬之機軍大敗赴七里澗死者如積水為之不流【為于偽翻】斬其大將賈崇等十六人石超遁去初䆠人孟玖有寵於大將軍頴玖欲用其父為邯鄲令【邯鄲縣漢屬趙國魏晉屬廣平郡隋唐屬磁州邯鄲音寒丹】左長史盧志等皆不敢違右司馬陸雲固執不許曰此縣公府掾資【言歷此縣者其資級可得公府掾掾以絹翻】豈有黄門父居之邪玖深怨之玖弟超領萬人為小督【機為都督與黄門之弟共事可以辭去矣】未戰縱兵大掠陸機録其主者【録收也】超將鐵騎百餘人直入機麾下奪之顧謂機曰貉奴能作督不【楊正衡曰貉音鶴獸名善睡似狐余謂超蓋詈機為貉奴不讀曰否】機司馬吳郡孫拯勸機殺之機不能用【使機能用孫拯之言斬孟超是穰苴之戮莊賈也由此為頴所殺豈不光明俊偉哉】超宣言於衆曰陸機將反又還書與玖言機持兩端故軍不速决及戰超不受機節度輕兵獨進敗没玖疑機殺之譖之於頴曰機有二心於長沙牽秀素諂事玖將軍王闡郝昌帳下督陽平公師藩【諸王公領兵及任方面者皆有帳下督統帳下兵魏文帝黄初二年分魏郡置陽平郡公師複姓也】皆玖所引用相與共證之頴大怒使秀將兵收機參軍事王彰諫曰今日之舉彊弱異勢庸人猶知必克况機之明達乎但機吳人殿下用之太過北土舊將皆疾之耳頴不從機聞秀至釋戎服著白帢【著陟畧翻帢苦洽翻帽也弁缺四隅謂之帢晉志曰魏武以天下凶荒資財乏匱擬古皮弁裁縑帛以為帢以色辯其貴賤本施軍飾非為國容徐爰曰俗說帢本未有岐荀文若巾之行觸樹枝成岐謂之為善今通為慶弔服】與秀相見為牋辭頴既而歎曰華亭鶴唳可復聞乎【機發此言有咸陽市上歎黄犬之意華亭時屬吳郡嘉興縣界有華亭谷華亭水至唐始分嘉興縣為華亭縣今縣東七十里其地出鶴土人謂之曰鶴窠復扶又翻】秀遂殺之頴又收機弟清河内史雲平東祭酒耽及孫拯皆下獄【按晉書陸雲傳雲自清河内史轉大將軍右司馬此當書右司馬雲下戶嫁翻 考異曰孫拯晉春秋作孫承今從晉書傳】記室江統陳留蔡克頴川棗嵩等上疏以為陸機淺

  謀致敗殺之可也至於反逆則衆共知其不然宜究檢校機反狀若有徵驗誅雲等未晩也統等懇請不已頴遲迴者三日蔡克入至頴前叩頭流血曰雲為孟玖所怨遠近莫不聞今果見殺竊為明公惜之【竊為于偽翻下為陳同】僚屬隨克入者數十人流涕固請頴惻然有宥雲色孟玖扶頴入催令殺雲耽夷機三族獄吏考掠孫拯數百兩踝骨見【掠音亮踝戶瓦翻腿兩旁曰内外踝見賢遍翻】終言機寃吏知拯義烈謂拯曰二陸之枉誰不知之君可不愛身乎拯仰天歎曰陸君兄弟世之奇士吾蒙知愛今既不能救其死忍復從而誣之乎【復扶又翻】玖等知拯不可屈乃令獄吏詐為拯辭頴既殺機意常悔之及見拯辭大喜謂玖等曰非卿之忠不能窮此姦遂夷拯三族拯門人費慈宰意二人詣獄明拯寃【費扶沸翻宰以官為氏春秋周有宰咺孔子弟子有宰予】拯譬遣之【譬喻也】曰吾義不負二陸死自吾分【分扶問翻】卿何為爾邪曰君既不負二陸僕又安可負君固言拯寃玖又殺之太尉乂奉帝攻張方方兵望見乘輿皆退走【乘繩證翻】方遂大敗死者五千餘人方退屯十三里橋 【考異曰河間王顒傳云駛水橋今從帝紀】衆懼欲夜遁方曰勝負兵家之常善用兵者能因敗為成今我更前作壘出其不意此奇策也乃夜潛逼洛城七里築壘數重【重直龍翻】外引廩穀以足軍食乂既戰勝以為方不足憂聞方壘成十一月引兵攻之不利朝議以為乂頴兄弟可辭說而釋【朝直遥翻】乃使中書令王衍等往說頴令與乂分陜而居【周公召公分陜為二伯陜在宏農此言分陜引周召事欲令頴乂為二伯耳非分陜地而居也往說輸芮翻陜式冉翻】頴不從乂因致書於頴為陳利害欲與之和解頴復書請斬皇甫商等首則引兵還鄴乂不可頴進兵逼京師張方决千金堨【水經注河南縣城東十五里有千金堨洛陽記曰千金堨舊堰穀水魏時更脩此堰謂之千金堨堨阿葛翻】水碓皆涸【碓都内翻】乃發王公奴婢手舂給兵一品已下不從征者男子十三以上皆從役又發奴助兵公私窮踧【踧與蹙同子六翻】米石萬錢詔命所行一城而已【京師危蹙如此乂雖戰勝安得久邪】驃騎主簿范陽祖逖【乂為驃騎將軍以逖為主簿驃匹妙翻】言於乂曰劉沈忠義果毅雍州兵力足制河間【雍於用翻】宜啟上為詔與沈使發兵襲顒顒窘急必召張方以自救此良策也【窘渠隕翻】乂從之沈奉詔馳檄四境諸郡多起兵應之沈合七郡之衆凡萬餘人趣長安【雍州統七郡治安定或曰時治新平沈持林翻趣七喻翻】乂又使皇甫商間行齎帝手詔命游楷等罷兵勅皇甫重進軍討顒商間行至新平遇其從甥從甥素憎商以告顒【間古莧翻從才用翻顒魚容翻】顒捕商殺之 十二月議郎周玘前南平内史長沙王矩【吳置南郡於江南晉平吳改曰南平以别江北之南郡玘口紀翻】起兵江東以討石氷推前吳興太守吳郡顧祕都督揚州九郡諸軍事【揚州統郡十八帝割豫章鄱陽廬陵臨川建安南康晉安屬江州揚州統十一郡今止推祕督丹陽宣城毗陵吳吳興會稽東陽新安臨海九郡淮南廬江在江北不與也】傳檄州郡殺氷所署將吏【將即亮翻】於是前侍御史賀循起兵於會稽【會工外翻】廬江内史廣陵華譚及丹陽葛洪甘卓皆起兵以應祕【華戶化翻】玘處之子循邵之子卓寧之曾孫也【周處見八十二卷元康六年七年賀邵事吳主皓為皓所殺甘寧事吳主權為相以勇聞三家皆吳之彊宗也】氷遣其將羌毒【姓譜羌姓也】帥兵數萬拒玘玘擊斬之氷自臨淮趨壽春【帥讀曰率趨七喻翻】征東將軍劉準聞氷至惶懼不知所為廣陵度支廬江陳敏統衆在壽春【陳敏自尚書令史出為合肥度支漕運南方米穀以濟中州遷廣陵度支度徒洛翻】謂準曰此等本不樂遠戍逼迫成賊烏合之衆其勢易離敏請督運兵為公破之【樂音洛易以鼔翻為干偽翻下為叡同】準乃益敏兵使擊之 閏月李雄急攻羅尚尚軍無食留牙門張羅守城 【考異曰載記作羅特今從華陽國志】夜由牛鞞水東走【水經曰牛鞞水在犍為牛鞞縣劉昫曰洛水一名牛鞞水杜佑曰簡州陽安縣漢牛鞞縣地盂康曰鞞音髀師古曰音必爾翻】羅開門降【降戶江翻】雄入成都軍士饑甚乃帥衆就穀於郪【帥讀曰率郪縣漢屬廣漢郡晉省立五代史志郪縣舊曰伍城隋大業改曰郪縣唐為梓州治所宋白曰漢舊郪縣城在今縣南九十里臨江郪王城基址見在以郪江為縣名郪音妻又干私翻】掘野芋而食之【芋羊遇翻所謂㟭山之下有蹲鴟也】許雄坐討賊不進徵即罪【許雄刺梁州見上卷太安元年即就也】 安北將軍都督幽州諸軍事王浚以天下方亂欲結援夷狄乃以一女妻鮮卑段務勿塵一女妻素怒延【妻七細翻宇文國有别帥曰素怒延】又表以遼西郡封務勿塵為遼西公【為王浚用段氏以攻成都王頴及石勒張本】浚沈之子也【王沈比晉以弑魏高貴鄉公】 毛銑之死也【事見上卷太安元年】李叡犇五苓夷帥于陵丞于陵丞詣李毅為叡請命【五苓夷寧州附塞部落之名帥所類翻為于偽翻】毅許之叡至毅殺之于陵丞怒帥諸夷反攻毅 尚書令樂廣女為成都王妃或譖諸太尉乂乂以問廣廣神色不動徐曰廣豈以五男易一女哉【謂附頴則五男被誅】乂猶疑之

  永興元年【長沙王乂之死改元永安西遷長安方改元永興】春正月丙午樂廣以憂卒 長沙厲王乂屢與大將軍頴戰破之【長沙王乂不得其死頴顒之黨加以惡謚耳】前後斬獲六七萬人而乂未嘗虧奉上之禮城中糧食日窘而士卒無離心【窘渠隕翻】張方以為洛陽未可克欲還長安而東海王越慮事不濟癸亥潛與殿中諸將夜收乂送别省 【考異曰越傳云殿中諸將及三部司馬疲於戰守密與左衛將軍朱默夜收乂别省逼越為主今從乂傳】甲子越啟帝下詔免乂官置金墉城大赦改元【改元永安 考異曰帝紀太安二年十二月甲子大赦永興元年正月大赦改元疑是一事】城既開殿中將士見外兵不盛悔之更謀劫出乂以拒頴越懼欲殺乂以絶衆心黄門侍郎潘淊曰不可將自有靜之者乃遣人密告張方丙寅方取乂於金墉城至營炙而殺之方軍士亦為之流涕【為于偽翻考異曰帝紀三十國晉春秋云大安二年十二月殺乂乂傳曰初乂執權之始洛下謡曰草木萌芽殺長沙乂以正月二十五日廢二十七日死如謡言焉樂廣傳云成都王頴廣之壻也及與長沙王乂遘難而廣既處朝望羣小讒謗之廣以憂卒惠帝紀永興九年正月丙午樂廣卒若廣卒時乂未死即乂傳正月二十九日廢為是合移在永興元年正月而晉春秋太安二年八月樂廣自裁按帝紀今年正月以頴為丞相遣兵屯城門代宿衛者疑此皆乂初死時事又今年正月末亦有甲子丙寅今從乂傳】公卿皆詣鄴謝罪大將軍頴入京師復還鎮于鄴詔以頴為丞相加東海王越守尚書令頴遣奮武將軍石超等率兵五萬屯十二城門【洛陽城東有建春東陽清明三門南有開陽津陽平昌宣陽四門西有廣陽西明閶闔三門北有大夏廣莫二門凡十二門帥讀曰率】殿中宿所忌者頴皆殺之悉代去宿衛兵【去羌呂翻】表盧志為中書監留鄴參署丞相府事 河間王顒頓軍於鄭【鄭縣屬京兆郡周宣王弟鄭桓公封邑唐屬華州】為東軍聲援聞劉沈兵起還鎮渭城【渭城縣故秦咸陽也前漢屬扶風後漢省而地名猶在石勒置石安縣唐復為咸陽縣屬京兆】遣督護虞夔逆戰於好畤【好畤縣前漢屬扶風後漢晉省唐武德二年復分醴泉置好畤縣屬京兆】夔兵敗顒懼退入長安急召張方方掠洛中官私奴婢萬餘人而西軍中乏食殺人雜牛馬肉食之劉沈渡渭而軍與顒戰顒屢敗沈使安定太守衙博功曹皇甫澹以精甲五千襲長安入其門力戰至顒帳下沈兵來遲馮翊太守張輔見其無繼引兵横擊之殺博及澹【澹徒覽翻又徒濫翻】兵遂敗收餘卒而退張方遣其將敦偉夜擊之【敦徒渾翻姓也】沈軍驚潰沈與麾下南走追獲之沈謂顒曰知已之惠輕【顒留沈為軍師遂為雍州刺史】君臣之義重沈不可以違天子之詔量彊弱以苟全【量音良】投袂之日期之必死【左傳宋殺楚使楚子聞之投袂而起】葅醢之戮其甘如薺【詩云誰謂荼苦其甘如薺薺在禮翻】顒怒鞭之而後腰斬新平太守江夏張光數為沈畫計【夏戶雅翻數所角翻為于偽翻】顒執而詰之光曰劉雍州不用鄙計故令大王得有今日顒壯之引與歡宴表為右衛司馬 羅尚逃至江陽【華陽國志曰瀘州瀘川縣本漢江陽縣又江安縣亦漢江陽縣也】遣使表狀詔尚權統巴東巴郡涪陵以供軍賦【三郡本屬梁州尚權統之涪音浮】尚遣别駕李興詣鎮南將軍劉宏求糧宏綱紀以運道阻遠【綱紀謂宏參佐操持一府之綱紀者】且荆州自空乏欲以零陵米五千斛與尚宏曰天下一家彼此無異吾今給之則無西顧之憂矣【謂尚在巴涪則為荆州屛蔽無西顧之憂】遂以三萬斛給之尚賴以自存李興願留為宏參軍宏奪其手版而遣之【手版即古笏也參佐施敬府公故持手版今奪興手版遣之不許其去尚而事已也】又遣治中何松領兵屯巴東為尚後繼于時流民在荆州者十餘萬戶羈旅貧乏多為盜賊宏大給其田及種糧【種章勇翻】擢其賢才隨資叙用流民遂安 三月乙酉丞相頴表廢皇后羊氏幽于金墉城廢皇太子覃為清河王【羊后立見八十三卷永康元年覃立見上卷太安元年】 陳敏與石氷戰數十合氷衆十倍於敏敏擊之所向皆捷遂與周玘合攻氷於建康三月氷北走投封雲【封雲徐州賊應氷者】雲司馬張統斬氷及雲以降楊徐二州平周玘賀循皆散衆還家不言功賞朝廷以陳敏為廣陵相 河間王顒表請立丞相頴為太弟戊申詔以頴為皇太弟都督中外諸軍事丞相如故大赦乘輿服御皆遷于鄴【天子在洛而建儲于鄴則既非矣乘輿服御亦遷而就之何居乘䋲正翻】制度一如魏武帝故事以顒為太宰大都督雍州牧【雍於用翻】前太傅劉寔為太尉寔以老固讓不拜 太弟頴僭侈日甚嬖倖用事大失衆望【時人望頴以匡輔帝室今乃若此故大失衆望嬖卑義翻又博計翻】司空東海王越與右衛將軍陳聄【楊正衡曰聄止忍翻又音真】及長沙故將上官已等謀討之秋七月丙申朔陳聄勒兵入雲龍門以詔召三公百僚及殿中【殿中者三部諸將也】戒嚴討頴石超犇鄴戊戌大赦復皇后羊氏及太子覃己亥越奉帝北征以越為大都督徵前侍中稽紹詣行在【長沙王乂當國以紹為侍中乂死紹黜免為庶人今討頴故復徵詣行在】侍中秦凖謂紹曰今往安危難測卿有佳馬乎紹正色曰臣子扈衛乘輿死生以之佳馬何為越檄召四方兵赴者雲集比至安陽【晉志安陽縣屬魏郡魏土地記曰鄴城南四十里有安陽城乘䋲證翻比必寐翻】衆十餘萬鄴中震恐頴會羣僚問計東安王繇曰天子親征宜釋甲縞素出迎請罪頴不從遣石超帥衆五萬拒戰【帥讀曰率】折衝將軍喬智明勸頴奉迎乘輿頴怒曰卿名曉事投身事孤今主上為羣小所逼卿奈何欲使孤束手就刑邪陳聄二弟匡規自鄴赴行在云鄴中皆已離散由是不甚設備己未石超軍奄至乘輿敗績於蕩隂【蕩隂縣漢屬河内郡晉屬魏郡唐為相州蕩隂縣按水經注湯隂縣因湯水為名宋白曰古湯隂縣在湯水南漢初廢安陽縣入湯隂隋又廢湯隂入安陽則安陽湯隂二縣接境也師古曰蕩音湯】帝傷頰中三矢【中竹仲翻】百官侍御皆散嵇紹朝服下馬登輦以身衛帝兵人引紹於轅中斫之【轅輈也方言曰楚衛謂之輈朝直遥翻】帝曰忠臣也勿殺對曰奉太弟令惟不犯陛下一人耳遂殺紹血濺帝衣【濺子賤翻】帝墯於草中亡六璽石超奉帝幸其營帝餒甚超進水左右奉秋桃【桃以夏熟者進御秋桃非所以奉至尊而奉之恤所無也】頴遣盧志迎帝庚申入鄴大赦改元曰建武左右欲浣帝衣【浣戶管翻濯也】帝曰嵇侍中血勿浣也【孰謂帝為戇愚哉】陳聄上官已等奉太子覃守洛陽司空越奔下邳徐州都督東平王楙不納越徑還東海太弟頴以越兄弟宗室之望【越騰畧模皆有聲稱於諸宗室中】下令招之越不應命前奮威將軍孫惠上書勸越要結藩方【要一遥翻】同奬王室越以惠為記室參軍與參謀議北軍中候苟晞奔范陽王虓虓承制以晞行兖州刺史【范陽王虓時鎮許昌虓許交翻】 初三王之起兵討趙王倫也【事見上卷永寧元年】王浚擁衆挾兩端禁所部士民不得赴三王召募太弟頴欲討之而未能【使頴兄弟不自内相圖聲浚之罪而討之固有餘力矣何未能邪】浚心亦欲圖頴頴以右司馬和演為幽州刺史【和演與頴謀起兵討趙王倫頴之腹心也】密使殺浚演與烏桓單于審登謀與浚游薊城南清泉因而圖之【單音蟬】會天暴雨兵器霑濕不果而還【還從宣翻又如字下同】審登以為浚得天助乃以演謀告浚浚與審登密嚴兵約幷州刺史東嬴公騰共圍演殺之自領幽州營兵【幽州刺史營兵也】騰越之弟也太弟頴稱詔徵浚浚與鮮卑段務勿塵烏桓羯朱及東嬴公騰同起兵討頴【羯居謁翻】頴遣北中郎將王斌及石超擊之【斌音彬】 太弟頴怨東安王繇前議【怨其使已縞素迎天子請罪也】八月戊辰收繇殺之初繇兄琅邪恭王覲薨子睿嗣睿沈敏有度量【沈持林翻】為左將軍與東海參軍王導善【導參東海王越軍事】導敦之從父弟也【從才用翻下從者之從同】識量清遠以朝廷多故每勸睿之國及繇死睿從帝在鄴恐及禍將逃歸頴先勑關津無得出貴人【關立於經塗要會處以譏出入津者濟渡江河所必由之處】睿至河陽為津吏所止從者宋典自後來以鞭拂睿而笑曰舍長【舍長守舍之長也長丁丈翻今知兩翻】官禁貴人汝亦被拘邪【被皮義翻】吏乃聽過至洛陽迎太妃夏侯氏俱歸國【元帝中興事始此夏戶雅翻】 丞相從事中郎王澄發孟玖姦利事勸太弟頴誅之頴從之 上官已在洛陽殘暴縱横【横戶孟翻】守河南尹周馥浚之從父弟也【周浚從王渾伐吳有戰功】與司隸滿奮等謀誅之事洩奮等死馥走得免司空越之討太弟頴也太宰顒遣右將軍馮翊太守張方將兵二萬救之聞帝已入鄴因命方鎮洛陽已與别將苗願拒之大敗而還太子覃夜襲已願已願出走方入洛陽覃於廣陽門迎方而拜【洛城西面南頭第一門曰廣陽門】方下車扶止之復廢覃及羊后 初太弟頴表匈奴左賢王劉淵為冠軍將軍監五部軍事【楊駿輔政以劉淵為五部大都督元康未坐部人叛出塞免官頴鎮鄴表監五部軍事冠古玩翻監工銜翻】使將兵在鄴淵子聰驍勇絶人博涉經史善屬文【驍堅堯翻屬之欲翻】彎弓三百斤弱冠游京師【記曲禮曰人生十年曰幼學二十曰弱冠冠古玩翻】名士莫不與交頴以聰為積弩將軍淵從祖右賢王宣謂其族人曰自漢亡以來我單于徒有虛號無復尺土【事見六十七卷漢獻帝建安二十一年從才用翻單音蟬】自餘王侯降同編戶【編相聯次也民謂之編民亦謂之編戶者言比屋聯次而居編於民籍無高下之差】今吾衆雖衰猶不減二萬奈何歛首就役奄過百年【奄忽也遽也】左賢王英武超世天苟不欲興匈奴必不虛生此人也今司馬氏骨肉相殘四海鼎沸復呼韓邪之業此其時矣【漢宣帝時稽侯狦來朝稱呼韓邪單于光武時日逐王比内附亦稱呼韓邪單于】乃相與謀推淵為大單于使其黨呼延攸詣鄴告之【師古曰漢書匈奴中貴種有呼衍氏即今之呼延氏】淵白頴請歸會葬頴弗許淵令攸先歸告宣等使招集五部及雜胡聲言助頴實欲叛之及王浚東嬴公騰起兵淵說頴曰今二鎮跋扈衆十餘萬【二鎮謂幽并說輸芮翻下同】恐非宿衛及近郡士衆所能禦也請為殿下還說五部以赴國難【為于偽翻說輸芮翻難乃旦翻】頴曰五部之衆果可發否就能發之鮮卑烏桓未易當也【易以豉翻】吾欲奉乘輿還洛陽以避其鋒【乘繩證翻】徐傳檄天下以逆順制之【言見力不足以制二鎮欲檄徵天下兵杖順制逆】君意何如淵曰殿下武皇帝之子有大勲於王室威恩遠著四海之内孰不願為殿下盡死力者【為于偽翻下請為同】何難發之有王浚豎子東嬴疎屬【東嬴公騰宣帝弟東武侯馗之孫故云疎屬】豈能與殿下爭衡邪殿下一發鄴宫示弱於人洛陽不可得而至雖至洛陽威權不復在殿下也【頴奔敗而失權卒如淵之言】願殿下撫勉士衆靖以鎮之淵請為殿下以二部摧東嬴三部梟王浚【梟堅堯翻】二豎之首可指日而懸也頴悅拜淵為北單于參丞相軍事淵至左國城【左國城蓋匈奴左部所居城也據晉書載記光武建武之初南單于入居西河之美稷今離石左國城單于所徙庭也水經注曰左國城在汾州之右介休縣西南杜佑曰左國城在石州離石縣宋白曰離石縣東北有離石水因以為名】劉宣等上大單于之號【上時掌翻】二旬之間有衆五萬都於離石【離石縣自漢以來屬西河郡】以聰為鹿蠡王【師古曰蠡音盧奚翻鹿蠡王即仍漢時谷蠡王號也谷鹿字雖不同而音則同耳】遣左於陸王宏帥精騎五千【帥讀曰率騎奇寄翻下同】會頴將王粹拒東嬴公騰粹已為騰所敗【敗補邁翻下敗石同】宏無及而歸王浚東嬴公騰合兵擊王斌大破之浚以主簿祁宏為前鋒【姓譜祁姓黄帝二十五子之一也又晉獻侯四世孫奚食邑於祁曰祁奚】敗石超于平棘【平棘縣漢屬常山郡晉屬趙國劉昫曰漢平棘縣在今趙州平棘縣南】乘勝進軍候騎至鄴鄴中大震百僚犇走士卒分散盧志勸頴奉帝還洛陽時甲士尚有萬五千人志夜部分【分扶問翻】至曉將發而程太妃戀鄴不欲去頴狐疑未决俄而衆潰頴遂將帳下數十騎與志奉帝御犢車【晉制皁輪犢車諸公乘之】南犇洛陽倉猝上下無齎中黄門被囊中齎私錢三千詔貸之【貸愓德翻又敵德翻又他代翻假借也】於道中買飯夜則御中黄門布被食以瓦盆至温將謁陵【帝之先河内温縣孝敬里人自京兆尹防以上皆葬于温】帝喪履納從者之履【喪息浪翻從才用翻】下拜流涕及濟河張方自洛陽遣其子羆帥騎三千以所乘車奉迎帝至芒山下方自帥萬餘騎迎帝【帥讀曰率】方將拜謁帝下車自止之【甚於夷王下堂而見諸侯矣】帝還宫犇散者稍還百官粗備【粗坐五翻】辛已大赦王浚入鄴士衆暴掠死者甚衆使烏桓羯朱追太弟頴至朝歌不及浚還薊以鮮卑多掠人婦女命敢有挾藏者斬於是沈於易水者八千人【王浚進不成勤王而縱鮮卑烏桓猾夏亂華其死於石勒之手晚矣沈持林翻】 東嬴公騰乞師於拓抜猗㐌以擊劉淵猗㐌與弟猗盧合兵擊淵於西河破之與騰盟于汾東而還【自此拓跋氏屢以兵助并州㐌徒河翻還從宣翻又如字】劉淵聞太弟頴去鄴歎曰不用吾言逆自犇潰真奴才也然吾與之有言矣不可以不救將發兵擊鮮卑烏桓劉宣等諫曰晉人奴隸御我今其骨肉相殘是天棄彼而使我復呼韓邪之業也鮮卑烏桓我之氣類【鮮卑烏桓東胡之種與匈奴同稟北方剛強之氣又同類也】可以為援奈何擊之淵曰善大丈夫當為漢高魏武呼韓邪何足効哉宣等稽首曰非所及也【稽音啟】荆州兵擒斬張昌同黨皆夷三族【去年昌逃于下雋山至是方禽滅】李雄以范長生有名德為蜀人所重欲迎以為君而臣之長生不可諸將固請雄即尊位冬十月雄即成都王位大赦改元建興除晉灋約灋七章以其叔父驤為太傅兄始為太保李離為太尉李雲為司徒李璜為司空李國為太宰閻式為尚書令楊褒為僕射尊母羅氏為王太后追尊父特為成都景王雄以李國李離有智謀凡事必咨而後行然國離事雄彌謹【史言諸李守君臣之分以相保固所謂盜亦有道也】 劉淵遷都左國城 【考異曰下云離石大饑遷于黎亭則是淵猶在離石也按杜佑通興離石有南單于庭左國城然則淵雖遷左國猶在離石縣境内也】胡晉歸之者愈衆淵謂羣臣曰昔漢有天下久長恩結於民吾漢氏之甥約為兄弟兄亡弟紹不亦可乎乃建國號曰漢劉宣等請上尊號【上時掌翻】淵曰今四方未定且可依高祖稱漢王於是即漢王位【劉淵字元海 考異曰帝紀李雄劉淵稱王皆在十一月惠帝入長安後華陽國志李雄十月稱王一本作十二月三十國晉春秋十六國鈔皆在十月今從之】大赦改元曰元熙追尊安樂公禪為孝懷皇帝【樂音洛】作漢三祖五宗神主而祭之【淵以漢高祖世祖昭烈為三祖太宗世宗中宗顯宗肅宗為五宗】立其妻呼延氏為王后以右賢王宣為丞相崔游為御史大夫左於陸王宏為太尉范隆為大鴻臚【臚陵如翻】朱紀為太常上黨崔懿之後部人陳元達皆為黄門郎【劉淵皆用漢官制後部即匈奴北部也居新興】族子曜為建武將軍游固辭不就【崔游淵之師也范隆朱紀同門生崔游既能以師道不為淵屈且又得不變於夷之義沈約志魏置建武將軍】元達少有志操【少詩照翻】淵嘗招之元達不答及淵為漢王或謂元達曰君其懼乎元達笑曰吾知其人久矣彼亦亮吾之心但恐不過三二日驛書必至其暮淵果徵元達元達事淵屢進忠言退而削草雖子弟莫得知也【草奏藁也】曜生而眉白目有赤光幼聰慧有膽量早孤養於淵及長儀觀魁偉性拓落高亮【拓恢拓也落磊落也長知兩翻觀古玩翻】與衆不羣好讀書善屬文【好呼報翻屬之欲翻】鐵厚一寸射而洞之【厚胡茂翻射而亦翻洞貫也】常自比樂毅及蕭曹時人莫之許也惟劉聰重之曰永明漢世祖魏武之流數公何足道哉【劉曜字永明數公謂樂毅蕭曹】 帝既還洛陽張方擁兵專制朝政【朝直遥翻下同】太弟頴不得復豫事【復扶又翻】豫州都督范陽王虓徐州都督東平王楙等上言頴弗克負荷宜降封一邑特全其命【荷下可翻又如字 考異曰虓傳云與鎮東將軍劉馥同上言按馥傳帝自長安還馥出為平東將軍都督揚州代劉凖為鎮東据此表張方猶存盖自鄴還洛陽時也】太宰宜委以關右之任【河間王顒時為太宰故稱之】自州郡以下選舉授任一皆仰成朝之大事廢興損益每輒疇咨【言關右州郡聽顒選舉朝政亦咨而後行仰牛向翻孔安國曰疇誰也顔師古曰疇誰也咨謀也言謀於衆人誰可為事也余按此所謂疇咨恐非孔顔注義蓋疇類也咨問也言朝之大事類以問顒朝直遥翻】張方為國效節而不達變通未即西還宜遣還郡【方本為馮翊太守為于偽翻】所加方官請悉如舊司徒戎司空越並忠國小心宜幹機事委以朝政【幹讀曰管又如字】王浚有定社稷之勲【謂舉兵討頴也】宜特崇重遂撫幽朔長為北藩臣等竭力扞城藩屏皇家【屛必郢翻】則陛下垂拱四海自正矣張方在洛既久兵士剽掠殆竭衆情喧喧無復留意【剽匹妙翻衆情謂方之軍情也】議欲奉帝遷都長安恐帝及公卿不從欲須帝出而劫之【須待也】乃請帝謁廟帝不許十一月乙未方引兵入殿以所乘車迎帝帝馳避後園竹中軍人引帝出逼使上車【上時掌翻】帝垂泣從之方於馬上稽首曰今寇賊縱横宿衛單少【稽音啟縱子容翻少詩沼翻】願陛下幸臣壘臣盡死力以備不虞時羣臣皆逃匿唯中書監盧志侍側曰陛下今日之事當一從右將軍【張方時為右將軍】帝遂幸方壘令方具車載宫人寶物軍人因妻畧後宫分争府藏【藏徂浪翻】割流蘇武帳為馬帴【毛晃曰流蘇盤線繪繡之毬五采錯為之同心而下垂者是也蘇猶鬚也又散貌以其橤下垂故曰蘇今人謂絛頭橤為蘇孟康曰今御武帳置兵闌五兵於帳中帴將先翻馬藉也】魏晉以來蓄積掃地無遺方將焚宗廟宫室以絶人返顧之心盧志曰董卓無道焚燒洛陽【事見五十九卷漢獻帝初平元年】怨毒之聲百年猶存何為襲之乃止帝停方壘三日方擁帝及太弟頴豫章王熾等趨長安【趨七喻翻】王戎出奔郟【郟縣前漢屬頴川郡後漢省晉屬襄城郡隋唐為汝州郟城縣郟音夾】太宰顒帥官屬步騎三萬迎于覇上【帥讀曰率】顒前拜謁帝下車止之帝入長安以征西府為宫【征西府征西將軍府顒所居也】唯尚書僕射荀藩司隸劉暾河南尹周馥在洛陽為留臺承制行事號東西臺【洛陽為東臺長安為西臺暾他昆翻】藩勗之子也【荀勗朋比賈充貴顯於晉初】丙午留臺大赦改元復為永安辛丑復皇后羊氏 羅尚移屯巴郡遣兵掠蜀中獲李驤妻昝氏及子壽【昝子感翻姓也】 十二月丁亥詔太弟頴以成都王還第更立豫章王熾為皇太弟【更工衡翻】帝兄弟二十五人時存者惟頴熾及吳王晏晏材資庸下熾冲素好學【好呼報翻】故太宰顒立之詔以司空越為太傅與顒夾輔帝室王戎參錄朝政【朝直遥翻】又以光祿大夫王衍為尚書左僕射高密王略為鎮南將軍領司隸校尉權鎮洛陽 【考異曰惠紀作高密王簡按宗室傳高密孝王畧字元簡時都督青州後遷都督荆州未嘗鎮洛陽蓋簡即畧也時雖有朝命而畧不至或嘗鎮洛陽而本傳遺脱耳 以余觀之時朝廷命令不行於方鎮畧蓋未嘗赴洛也】東中郎將模為寧北將軍都督冀州諸軍事鎮鄴【晉書帝紀作東中郎將模傳作北中郎將又按晉制方面之任有四征四鎮四安四平無四寧也寧恐當作安】百官各還本職令州郡蠲除苛政愛民務本清通之後當還東京【謂阻兵者解兵道路清通之後也帝時在長安故謂洛陽為東京】大赦改元【改元永興】略模皆越之弟也王浚既去鄴越使模鎮之顒以四方乖離禍難不已故下此詔和解之冀獲少安【顒欲和解兄弟當在乂頴搆隙之時及事不可為而顒亦不免矣難乃旦翻】越辭太傅不受又詔以太宰顒都督中外諸軍事張方為中領軍錄尚書事領京兆太守【時帝在長安京兆太守實掌輦轂下張方握兵顒所親倚故使領京兆】 東嬴公騰遣將軍聶玄擊漢王淵戰於大陵【大陵縣自漢以來屬太原郡魏收地形志太原郡統内受陽縣有大陵城其地蓋在唐遼并二州界杜佑曰文水縣漢太陵縣聶尼輒翻】玄兵大敗淵遣劉曜寇太原取泫氏屯留長子中都【泫氏屯留長子屬上黨郡中都屬太原郡賢曰泫氏今澤州高平縣劉昫曰澤州陵川縣漢泫氏縣高平漢泫氏縣地屯留長子唐皆屬潞州師古曰泫音工玄翻屯音純長讀如長短之長陸德明讀如長幼之長】又遣冠軍將軍喬晞寇西河取介休介休令賈渾不降晞殺之【介休縣漢屬太原郡晉屬西河郡唐屬汾州冠古玩翻降戶江翻】將納其妻宗氏宗氏罵晞而哭晞又殺之淵聞之大怒曰使天道有知喬晞望有種乎【種章勇翻】追還降秩四等收渾屍葬之

  資治通鑑卷八十五


    


 


 



 

 
  







 


  
  
 
 
 


  

 















	
	









































 
  



















 





 












  
  
  

 





