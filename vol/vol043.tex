\chapter{資治通鑑卷四十三}
宋 司馬光 撰

胡三省 音註

漢紀三十五|{
	起柔兆涺灘盡柔兆敦牂凡十一年}


世祖光武皇帝中之下

建武十二年春正月吳漢破公孫述將魏黨公孫永於魚涪津|{
	續漢書曰犍為郡南安縣有魚涪津在縣北臨大江南中志曰漁涪津廣數百步涪音浮}
遂闈武陽述遣子婿史興救之漢迎擊破之因入犍為界諸縣皆城守詔漢直取廣都據其心腹漢乃進軍攻廣都拔之|{
	武帝元朔二年置廣都縣屬蜀郡}
遣輕騎燒成都市橋|{
	賢曰市橋即七星橋之一橋也李膺益州記冲星橋舊市橋也在今成都縣西南四里水經註成都中兩江有七橋西南石牛門外曰市橋}
公孫述將帥恐懼日夜離叛述雖誅滅其家猶不能禁|{
	將即亮翻帥所類翻}
帝必欲降之|{
	降戶江翻下同}
又下詔諭述曰勿以來歙岑彭受害自疑|{
	二人受害見上卷上年歙許及翻}
今以時自詣則宗族完全詔書手記不可數得|{
	數所角翻}
述終無降意秋七月馮駿拔江州獲田戎 帝戒吳漢曰成都十

餘萬衆不可輕也但堅據廣都待其來攻勿與爭鋒若不敢來公轉營迫之須其力疲乃可擊也漢乘利遂自將步騎二萬進逼成都去城十餘里阻江北營作浮橋使副將武威將軍劉尚將萬餘人屯於江南為營相去二十餘里帝聞之大驚讓漢曰比敕公千條萬端何意臨事勃亂|{
	比毗至翻千條萬端言詳細也勃與悖同}
旣輕敵深入又與尚别營事有緩急不復相及|{
	復扶又翻}
賊若出兵綴公以大衆攻尚尚破公即敗矣幸無它者|{
	言幸而無它虞不至喪敗也}
急引兵還廣都詔書未到九月述果使其大司徒謝豐執金吾袁吉將衆十許萬|{
	十許萬者約言之也}
分為二十餘營出攻漢使别將將萬餘人劫劉尚令不得相救漢與大戰一日兵敗走入壁豐因圍之漢乃召諸將厲之曰|{
	厲勉也毛晃曰勉厲之厲有修飾振起之意}
吾與諸君踰越險阻轉戰千里遂深入敵地至其城下而今與劉尚二處受圍埶旣不接其禍難量|{
	量音良}
欲濳師就尚於江南并兵禦之若能同心一力人自為戰大功可立如其不然敗必無餘成敗之機在此一舉諸將皆曰諾於是饗士秣馬閉營三日不出乃多

樹旛旗使煙火不絶夜銜枚引兵與劉尚合軍豐等不覺明日乃分兵拒水北自將攻江南漢悉兵迎戰自旦至晡|{
	日加申為晡奔謨翻}
遂大破之斬豐吉於是引還廣都留劉尚拒述具以狀上|{
	上時掌翻}
而深自譴責帝報曰公還廣都甚得其宜述必不敢略尚而擊公也|{
	賢曰略猶過也}
若先攻尚公從廣都五十里悉步騎赴之適當值其危困破之必矣自是漢與述戰於廣都成都之間八戰八克遂軍于其郭中|{
	成都郭中也}
臧宮拔緜竹破涪城|{
	涪縣屬廣漢郡賢曰涪城今緜竹縣宋白曰緜州巴西縣本漢涪縣}
斬公孫恢|{
	恢述弟也}
復攻拔繁郫與吳漢會於成都|{
	賢曰繁縣名屬蜀郡繁江名因以為縣名故城在今益州新繁縣北郫縣名屬蜀郡故城在今益州郫縣北郫音皮}
李通欲避權埶乞骸骨積二歲帝乃聽上大司空印綬|{
	上時掌翻}
以特進奉朝請後有司奏封皇子帝感通首創大謀|{
	事見三十八卷王莽地皇三年}
即日封通少子雄為召陵侯|{
	召讀與邵同}
公孫述困急謂延岑曰事當奈何岑曰男兒當死中求生可坐窮乎財物易聚耳|{
	易以豉翻}
不宜有愛述乃悉散金帛募敢死士五千餘人以配岑岑於市橋偽建旗幟鳴鼓挑戰|{
	幟昌志翻挑徒了翻下同}
而濳遣奇兵出吳漢軍後襲擊破漢漢墮水緣馬尾得出漢軍餘七日糧隂具船欲遁去蜀郡太守南陽張堪聞之|{
	時成都未破先署蜀郡太守以招懷蜀人}
馳往見漢說述必敗不宜退師之策|{
	說如字}
漢從之乃示弱以挑敵冬十一月臧宮軍咸陽門|{
	臧宫傳作咸門賢曰成都城北面東頭門此衍陽字東或作西}
戊寅述自將數萬人攻漢使延岑拒宮大戰岑三合三勝自旦及日中軍士不得食並疲漢因使護軍高午唐邯將鋭卒數萬擊之|{
	邯戶甘翻}
述兵大亂高午犇陳刺述|{
	陳讀曰陣刺七亦翻}
洞胷墮馬左右輿入城述以兵屬延岑|{
	屬之欲翻}
其夜死明旦延岑以城降|{
	降戶江翻}
辛巳吳漢夷述妻子盡滅公孫氏并族延岑遂放兵大掠焚述宫室帝聞之怒以譴漢又讓劉尚曰城降三日吏民從服孩兒老母口以萬數一旦放兵縱火聞之可為酸鼻尚宗室子孫更嘗吏職|{
	更工衡翻}
何忍行此仰視天俯視地觀放麑啜羹二者孰仁|{
	韓子曰孟孫獵得麑使秦西巴持之其母隨而呼秦西巴不忍放而與其母孟孫怒而逐西巴既而復之使傅其子戰國策曰樂羊為將為魏文侯攻中山中山之君烹其子而遺之羹樂羊坐於幕下而啜之盡一杯文侯謂禇師贊曰樂羊以我故而食其子之肉答曰子且食之其誰不食旣拔中山文侯賞其功而疑其心}
良失斬將弔民之義也|{
	將即亮翻}
初述徵廣漢李業為博士業固稱疾不起|{
	業平帝元始中除為郎會王莽居攝以病去官杜門不應州郡之命王莽以業為酒士病不之官遂隱藏山谷絶匿名迹夫旣不仕於莽其肯為述起乎}
述羞不能致使大鴻臚尹融奉詔命以劫業若起則受公侯之位不起賜以毒酒融譬旨曰方今天下分崩孰知是非而以區區之身試於不測之淵乎朝廷貪慕名德曠官缺位于今七年四時珍御不以忘君|{
	珍御謂食珍之供進者}
宜上奉知己下為子孫|{
	為于偽翻下同}
身名俱全不亦優乎業乃歎曰古人危邦不入亂邦不居|{
	論語戴孔子之言也}
為此故也君子見危授命|{
	論語載子張之言也}
何乃誘以高位重餌哉|{
	誘音酉}
融曰宜呼室家計之業曰丈夫斷之於心久矣|{
	斷丁亂翻}
何妻子之為遂飲毒而死述恥有殺賢之名遣使弔祠賻贈百匹業子翬逃辭不受|{
	翬音暉}
述又聘巴郡譙玄|{
	姓譜曹大夫食采於譙因氏焉玄平帝元始四年為繡衣使者分行天下觀省風俗會莽居攝弃使者車歸家隱遁}
玄不詣亦遣使者以毒藥劫之太守自詣玄廬勸之行玄曰保志全高死亦奚恨遂受毒藥玄子瑛泣血叩頭於太守願奉家錢千萬以贖父死太守為請|{
	為于偽翻}
述許之述又徵蜀郡王皓王嘉|{
	平帝時皓為美陽令嘉為郎王莽簒位並弃官西歸}
恐其不至先繫其妻子使者謂嘉曰速裝妻子可全對曰犬馬猶識主况於人乎|{
	言身為漢臣豈不念故主乎}
王皓先自刎以首付使者|{
	刎武粉翻}
述怒遂誅皓家屬王嘉聞而嘆曰後之哉乃對使者伏劍而死犍為費貽不肯仕述漆身為癩陽狂以避之|{
	犍居言翻費音祕又父沸翻}
同郡任永馮信皆託青盲以辭徵命|{
	青盲者其瞳子不精明不能睹物任音壬}
帝旣平蜀詔贈常少為太常張隆為光祿勲|{
	少隆死見上卷上年}
譙玄已卒祠以中牢|{
	師古曰中牢即少牢謂羊豕也}
勑所在還其家錢而表李業之閭徵費貽任永馮信會永信病卒獨貽仕至合浦太守|{
	郡國志合浦郡在雒陽南九千一百九十一里}
上以述將程烏李育有才幹皆擢用之於是西土咸悦莫不歸心焉初王莽以廣漢文齊為益州太守|{
	郡國志益州郡在雒陽西五千五百里}
齊訓農治兵|{
	治直之翻}
降集羣夷甚得其和|{
	降戶江翻下同}
公孫述時齊固守拒險述拘其妻子許以封侯齊不降聞上即位間道遣使自聞|{
	間古莧翻使疏吏翻}
蜀平徵為鎮遠將軍封成義侯 十二月辛卯揚武將軍馬成行大司空事 是歲參狼羌與諸種寇武都|{
	參狼羌無弋爰劒之後也爰劍孫卭將其種人南出賜支河曲之西數千里其後子孫分别各自為種或為氂牛種越巂羌是也或為白馬種廣漢羌是也或為参狼種武都羌是也爰劍曾孫忍及弟舞留湟中是為湟中諸種羌種章勇翻}
隴西太守馬援擊破之降者萬餘人於是隴右清静援務開恩信寛以待下任吏以職但緫大體而賓客故人日滿其門諸曹時白外事援輒曰此丞掾之任何足相煩|{
	百官志郡守有丞一人有諸曹掾史有功曹史主選署功勞有五官掾署功曹及諸曹事其餘有議曹法曹賊曹決曹金曹倉曹等掾俞絹翻}
頗哀老子使得遨遊若大姓侵小民黠吏不從令|{
	黠丁八翻}
此乃太守事耳傍縣嘗有報讐者吏民驚言羌反百姓奔入城狄道長詣門請閉城發兵|{
	賢曰狄道縣屬隴西郡今蘭州縣余據隴西郡治狄道故得詣門白太守長知兩翻}
援時與賓客飲大笑曰虜何敢復犯我曉狄道長歸守寺舍|{
	賢曰曉喻也寺舍官舍也}
良怖急者可牀下伏|{
	怖普布翻}
後稍定郡中服之 詔邊吏力不足戰則守追虜料敵不拘以逗留法|{
	賢曰漢法軍行逗留畏愞者斬追虜或近或遠量敵進退不拘以軍法直取勝敵為務}
山桑節侯王常牟平烈侯耿况東光成侯耿純皆薨|{
	謚法好亷自克曰節有功安民曰烈賀琛曰佐相克終曰成惇厖惇固曰成}
况疾病乘輿數自臨幸復以弇弟廣舉並為中郎將|{
	乘繩證翻數所角翻復扶又翻}
弇兄弟六人|{
	弇舒國廣舉霸兄弟六人}
皆垂青紫省侍醫藥|{
	省悉景翻}
當世以為榮盧芳與匈奴烏桓連兵數寇邊帝遣驃騎大將軍杜茂等將兵鎮守北邊治飛狐道|{
	治飛狐道以通趙魏應援北邉之兵}
築亭障修烽燧凡與匈奴烏桓大小數十百戰終不能克 上詔竇融與五郡太守入朝融等奉詔而行官屬賓客相隨駕乘千餘兩馬牛羊被野|{
	乘繩證翻兩音亮被皮義翻}
旣至詣城門上印綬|{
	上時掌翻}
詔遣使者還侯印綬引見賞賜恩寵傾動京師尋拜融冀州牧|{
	冀州部魏郡鉅鹿常山中山信都河間清河趙國勃海}
又以梁統為太中大夫姑臧長孔奮為武都郡丞姑臧在河西最為富饒|{
	姑臧縣屬武威郡劉昫曰姑臧縣秦月氏戎所處匈奴名盖藏城語訛為姑臧城長知兩翻}
天下未定士多不脩檢操居縣者不盈數月輒致豐積奮在職四年力行清潔為衆人所笑以為身處脂膏不能自潤|{
	說文戴角者脂無角者膏處昌呂翻}
及從融入朝諸守令財貨連轂彌竟川澤|{
	轂戶谷翻}
唯奮無資單車就路帝以是賞之帝以睢陽令任延為武威太守|{
	睢音雖任音壬}
帝親見戒之曰善事上官無失名譽延對曰臣聞忠臣不和和臣不忠 |{
	考異曰延傳作忠臣不私私臣不忠按高峻小史作忠臣不和和臣不忠意思為長又與上語相應今從之}
履正奉公臣子之節上下雷同非陛下之福|{
	曲禮曰毋雷同鄭氏註曰雷之發聲物無不同時應者人之言當各由已不當然也}
善事上官臣不敢奉詔帝歎息曰卿言是也

十三年春正月庚申大司徒侯霸薨 戊子詔曰郡國獻異味其令太官勿復受|{
	百官志太官令一人秩六百石掌御膳飲食復扶又翻}
遠方口實所以薦宗廟自如舊制|{
	漢官儀曰口實膳羞之事也}
時異國有獻名馬者日行千里又進寶劒價直百金詔以劒賜騎士馬駕鼔車|{
	輿服志乘輿法駕後有金鉦黄鉞黄門鼔車}
上雅不喜聽音樂|{
	喜許旣翻}
手不持珠玉嘗出獵車駕夜還上東門候汝南郅惲拒關不開|{
	賢曰上東門洛陽城東面北頭門也惲於粉翻}
上令從者見面於門間|{
	見賢遍翻}
惲曰火明遼遠遂不受詔上乃回從東中門入|{
	賢曰東面中門也}
明日惲上書諫曰昔文王不敢盤于遊田以萬民惟正之供|{
	尚書無逸之辭盤樂也}
而陛下遠獵山林夜以繼晝其如社稷宗廟何書奏賜惲布百匹貶東中門候為參封尉|{
	雒陽十二城門每門候一人秩六百石參封縣屬琅邪郡}
二月遣捕虜將軍馬武屯虖沱河以備匈奴|{
	虖讀曰呼}
盧芳攻雲中久不下其將隨昱留守九原欲脅芳來降芳知之與十餘騎亡入匈奴其衆盡歸隨昱昱乃詣闕降詔拜昱五原太守封鐫胡侯|{
	鐫子全翻}
朱祜奏古者人臣受封不加王爵丙辰詔長沙王興眞定王得河間王邵中山王茂皆降爵為侯|{
	高帝封諸侯王其子孫無有與漢俱存亡者文帝封梁王城陽菑川景帝封河間長沙中山常山昭帝封廣陽廣陵高密此數國至王莽簒漢而廢但封長沙眞定河間中山者與帝同出於景帝也長沙舂陵之太宗眞定常山王憲之後改封者今復降爵為侯以服屬已疏也}
丁巳以趙王良為趙公太原王章為齊公魯王興為魯公|{
	良帝叔父章興帝兄子也}
是時宗室及絶國封侯者凡一百三十七人富平侯張純安世之四世孫也歷王莽世以敦謹守約保全前封建武初先來詣闕為侯如故於是有司奏列侯非宗室不宜復國上曰張純宿衛十有餘年其勿廢更封武始侯食富平之半|{
	賢曰武始縣屬魏郡富平縣屬平原郡}
庚午以紹嘉公孔安為宋公承休公姬常為衛公|{
	平帝元始四年改紹嘉公曰宋公承休公曰鄭公今又改鄭曰衛}
三月辛未以沛郡太守韓歆為大司徒|{
	郡國志沛郡在雒陽東南一千二百里}
丙子行大司空馬成復為揚武將軍 吳漢自蜀振旅而還至宛|{
	宛於元翻}
詔過家上冢賜穀二萬斛|{
	上時掌翻}
夏四月至京師於是大饗將士功臣增邑更封|{
	更工衡翻}
凡三百六十五人其外戚恩澤封者四十五人定封鄧禹為高密侯食四縣|{
	禹食昌安夷安淳于高密四縣賢曰高密國名今密州縣余據西漢以高密為王國東漢為侯國屬北海國賢所云盖侯國也}
李通為固始侯賈復為膠東侯食六縣|{
	固始侯國屬汝南郡故寢縣也帝更名史記正義曰孫叔敖以寢丘土寢薄取為封邑李通又慕叔敖受邑光武嘉之改名固始膠東西漢以為王國帝以為侯國併屬北海食郁秋壯武下密即墨挺胡觀陽凡六縣}
餘各有差已歿者益封其子孫或更封支庶帝在兵間久厭武事且知天下疲耗思樂息肩自隴蜀平後非警急未嘗復言軍旅|{
	樂音洛復扶又翻}
皇太子嘗問攻戰之事帝曰昔衛靈公問陳孔子不對|{
	論語衛靈公問陳於孔子孔子曰俎豆之事則嘗聞之矣軍旅之事未之學也陳讀曰陣}
此非爾所及鄧禹賈復知帝偃干戈修文德不欲功臣擁衆京師乃去甲兵敦儒學|{
	去羌呂翻}
帝亦思念欲完功臣爵土不令以吏職為過|{
	恐其以職事有過而失爵邑也}
遂罷左右將軍官耿弇等亦上大將軍將軍印綬|{
	上時掌翻}
皆以列侯就第加位特進奉朝請|{
	朝直遥翻請才性翻又如字}
鄧禹内行淳備|{
	行下孟翻}
有子十三人各使守一藝修整閨門教養子孫皆可以為後世灋資用國邑不修產利|{
	凡用度皆資於國邑不事生產作業及營利也}
賈復為人剛毅方直多大節旣還私第闔門養威重朱祜等薦復宜為宰相帝方以吏事責三公故功臣並不用是時列侯唯高密固始膠東三侯與公卿參議國家大事恩遇甚厚帝雖制御功臣而每能回容|{
	回容猶今言回護也賢曰回曲也曲法以容也}
宥其小失遠方貢珍甘必先徧賜諸侯而太官無餘故皆保其福禄無誅譴者 益州傳送公孫述瞽師郊廟樂器葆車輿輦於是灋物始備|{
	賢曰瞽無目之人也為樂師取其無所見於音審也郊廟之器樽彛之屬也樂器鐘磬之屬也葆車謂上建羽葆也合聚五采羽名為葆孔穎達曰羽葆者以鳥羽注於柄頭如盖謂之羽葆謂盖也輿者車之總名也輦者駕人以行法物謂大駕鹵簿儀式也時草創未暇今得之始備余謂法物即上樂器葆車輿輦之類傳直戀翻}
時兵革旣息天下少事文書調役|{
	調徒弔翻}
務從簡寡至乃十存一焉 甲寅以冀州牧竇融為大司空融自以非舊臣一旦入朝|{
	朝直遥翻下同}
在功臣之右每朝會進見容貌辭氣卑恭已甚帝以此愈親厚之融小心久不自安數辭爵位|{
	數所角翻}
上疏曰臣融有子朝夕教導以經藝不令觀天文見䜟記|{
	䜟楚譛翻}
誠欲令恭肅畏事恂恂守道不願其有才能何况乃當傳以連城廣土享故諸侯王國哉因復請間求見|{
	復扶又翻間古莧翻見賢遍翻下以意推}
帝不許後朝罷逡廵席後|{
	逡廵却退貌}
帝知欲有讓遂使左右傳出|{
	傳旨使融出也}
它日會見迎詔融曰日者知公欲讓職還土|{
	賢曰日者猶往日也}
故命公暑熱且自便今相見宜論它事勿得復言融不敢重陳請|{
	重直用翻}
五月匈奴寇河東

十四年夏卭穀王任貴遣使上三年計即授越巂太守|{
	郡國志越巂郡在雒陽西四千八百里巂音髓卭渠容翻任音壬上時掌翻}
秋會稽大疫|{
	郡國志會稽郡在雒陽東三千八百里會古外翻}
莎車王賢鄯善王安皆遣使奉獻|{
	莎素禾翻鄯時戰翻}
西域苦匈奴重歛|{
	歛力贍翻}
皆願屬漢復置都護上以中國新定不許 太中大夫梁統上疏曰臣竊見元帝初元五年輕殊死刑三十四事哀帝建平元年輕殊死刑八十一事其四十二事手殺人者减死一等自是之後著為常準故人輕犯法吏易殺人|{
	易以䜴翻}
臣聞立君之道仁義為主仁者愛人義者正理愛人以除殘為務正理以去亂為心|{
	去羌呂翻}
刑罰在衷無取於輕|{
	衷中也適也}
高帝受命約令定律誠得其宜|{
	高帝入關約法三章後蕭何定律九章}
文帝唯除省肉刑相坐之灋|{
	文帝元年除收孥相坐法十三年除肉刑}
自餘皆率由舊章至哀平繼體即位日淺聽斷尚寡|{
	斷丁亂翻}
丞相王嘉輕為穿鑿虧除先帝舊約成律|{
	按嘉傳及刑法志並無其事統與嘉時代相接所引固不妄矣但班固畧而不載也}
數年之間百有餘事或不便於理或不厭民心|{
	厭於葉翻}
謹表其尤害於體者傅奏於左|{
	體政體也傅音附}
願陛下宣詔有司詳擇其善定不易之典事下公卿|{
	下遐嫁翻}
光禄勲杜林奏曰大漢初興蠲除苛政海内歡欣及至其後漸以滋章|{
	老子曰法令滋章盗賊多有}
果桃菜茹之饋集以成贓小事無妨於義以為大戮至於灋不能禁令不能止上下相遁為敝彌深|{
	賢曰遁猶回避也前書曰上下相匿以文避法焉}
臣愚以為宜如舊制不合翻移統復上言曰|{
	復扶又翻}
臣之所奏非曰嚴刑經曰爰制百姓于刑之衷|{
	尚書呂刑之言}
衷之為言不輕不重之謂也自高祖至于孝宣海内稱治|{
	治直吏翻}
至初元建平而盜賊浸多皆刑罰不衷愚人易犯之所致也|{
	易以豉翻}
由此觀之則刑輕之作反生大患惠加姦軌而害及良善也事寢不報

十五年春正月辛丑大司徒韓歆免歆好直|{
	好呼到翻}
言無隱諱帝每不能容歆於上前證歲將饑凶指天畫地言甚剛切故坐免歸田里帝猶不釋復遣使宣詔責之|{
	復扶又翻}
歆及子嬰皆自殺歆素有重名死非其罪衆多不猒|{
	猒一葉翻}
帝乃追賜錢穀以成禮葬之|{
	賢曰成禮具禮也言不以非命而降其葬禮}


臣光曰昔高宗命說曰若藥弗瞑眩厥疾弗瘳|{
	說傅說也音悦孔安國曰如服藥必瞑眩極其病乃除欲其出切言以自警陸德明音瞑莫遍翻眩玄遍翻徐又呼縣翻瞑眩困極也}
夫切直之言非人臣之利乃國家之福也是以人君日夜求之唯懼弗得聞惜乎以光武之世而韓歆用直諫死豈不為仁明之累哉|{
	累力瑞翻}


丁未有星孛於昴|{
	昂七星西方之宿也主獄事又為旄頭胡星也昴畢間為天街黄道之所經也孛蒲内翻}
以汝南太守歐陽歙為大司徒|{
	郡國志汝南郡在雒陽南六百五十里歙許及翻}
匈奴寇鈔日盛|{
	鈔楚交翻}
州郡不能禁二月遣吳漢率馬成馬武等北擊匈奴徙雁門代郡上谷吏民六萬餘口置居庸常山關以東以避胡寇|{
	郡國志雁門郡在雒陽北一千五百里代郡在雒陽東北二千五百里前書曰代郡有常山關上谷郡居庸縣有關}
匈奴左部遂復轉居塞内|{
	復扶又翻}
朝廷患之增緣邊兵部數千人|{
	每部各數千人也}
夏四月丁巳封皇子輔為右翊公英為楚公陽為東海公康為濟南公|{
	濟子禮翻}
蒼為東平公延為淮陽公荆為山陽公衡為臨淮公焉為左翊公京為琅邪公|{
	邪音耶}
癸丑追諡兄縯為齊武公兄仲為魯哀公帝感縯功業不就|{
	事見三十九卷更始元年}
撫育二子章興恩愛甚篤以其少貴|{
	少詩照翻}
欲令親吏事使章試守平隂令興緱氏令|{
	平隂緱氏二縣皆屬河南尹緱工侯翻}
其後章遷梁郡太守|{
	梁郡在雒陽東南八百五十里}
興遷弘農太守|{
	郡國志弘農郡在雒陽西南四百五十里}
帝以天下墾田多不以實自占|{
	占之贍翻}
又戶口年紀互有增減乃詔下州郡檢覈|{
	覈者考其實也下戶稼翻}
於是刺史太守多為詐巧苟以度田為名聚民田中并度廬屋里落民遮道啼呼|{
	度徒洛翻呼火故翻}
或優饒豪右侵刻羸弱|{
	羸倫為翻}
時諸郡各遣使奏事帝見陳留吏牘上有書視之云潁川弘農可問河南南陽不可問|{
	宋白曰漢割秦南陽河南二郡之西境置弘農郡義取弘大農桑為名}
帝詰吏由趣|{
	由從也問是書之所從來也趣向也問是書之意其所向為何如也}
吏不肯服抵言於長夀街上得之|{
	抵欺也賢曰長夀街在洛陽城中}
帝怒時東海公陽年十二在幄後言曰吏受郡敕|{
	敕教戒也}
當欲以墾田相方耳|{
	敕教也戒也相方求問其墾田之數以相比也}
帝曰即如此何故言河南南陽不可問對曰河南帝城多近臣南陽帝鄉多近親田宅踰制不可為凖帝令虎賁將詰問吏|{
	虎賁將虎賁中郎將也將即亮翻}
吏乃實首服如東海公對|{
	首式救翻}
上由是益奇愛陽|{
	為立陽為太子張本}
遣謁者考實二千石長吏阿枉不平者|{
	長知兩翻}
冬十一月甲戌大司徒歙坐前為汝南太守度田不實贓罪千餘萬下獄|{
	下遐稼翻}
歙世授尚書八世為博士|{
	自歐陽生傳伏生尚書至歙八世皆為博士}
諸生守闕為歙求哀者千餘人|{
	為于偽翻}
至有自髠剔者|{
	毛晃曰剃髪曰髠盡及身毛曰剔}
平原禮震年十七|{
	禮姓也左傳衛有大夫禮孔}
求代歙死帝竟不赦歙死獄中 十二月庚午以關内侯戴涉為大司徒 盧芳自匈奴復入居高柳|{
	復扶又翻}
是歲驃騎大將軍杜茂坐使軍吏殺人免使揚武將軍馬成代茂繕治障塞十里一候以備匈奴|{
	治直之翻}
使騎都尉張堪領杜茂營擊破匈奴於高柳|{
	杜佑曰雲洲治雲中縣縣界有高柳城闞駰曰高柳在狋氏縣北百三十里酈道元曰高柳縣故城舊代郡治高柳在代中其山重巒疊巘霞舉雲高連山隱隱東出遼塞}
拜堪漁陽太守堪視事八年匈奴不敢犯塞勸民耕稼以致殷富百姓歌曰桑無附枝麥秀兩岐|{
	蠶月既採桑斫去繁枝留其特長者則來年桑葉茂盛麥率一莖一穗罕有兩岐者故以為瑞}
張君為政樂不可支|{
	樂音洛}
安平侯盖延薨|{
	盖古盍翻}
交趾麊泠縣雒將女子徵側甚雄勇|{
	師古曰麊泠音麋零交州外域記曰交趾昔未有郡縣之時土地有雒田民墾食其田因名為雒民設雒王雒侯主諸郡縣縣有雒將銅印青綬宋白曰峯州漢麊泠縣地}
交趾太守蘇定以灋䋲之徵側忿怨

十六年春二月徵側與其妹徵貳反九眞日南合浦蠻俚皆應之|{
	郡國志曰南郡秦象郡地在雒陽南萬三千四百里賢曰俚蠻之别號今呼為俚人宋白曰愛州漢九真郡治胥浦縣驩州漢日南郡治朱吾縣}
凡略六十五城自立為王都麊泠交趾刺史及諸太守僅得自守 三月辛丑晦日有食之 秋九月河南尹張伋及諸郡守十餘人皆坐度田不實下獄死後上從容謂虎賁中郎將馬援曰|{
	武帝置期門郎掌執兵送從平帝元始元年更名虎賁郎置中郎將漢儀虎賁騎鶡冠虎文單衣度徒洛翻從千容翻賁音奔}
吾甚恨前殺守相多也|{
	守式又翻相息亮翻}
對曰死得其罪何多之有但死者旣往不可復生也|{
	復扶又翻}
上大笑郡國羣盜處處並起郡縣追討到則解散去復屯結青徐幽冀四州尤甚冬十月遣使者下郡國|{
	下遐稼翻}
聽羣盜自相糾擿|{
	賢曰擿猶發也他狄翻}
五人共斬一人者除其罪吏雖逗留回避故縱者皆勿問聽以禽討為效其牧守令長坐界内有盜賊而不收捕者又以畏愞捐城委守者皆不以為負|{
	賢曰委守謂棄其所守也負罪負也愞而戀翻又奴亂翻}
但取獲賊多少為殿最|{
	殿丁甸翻}
唯蔽匿者乃罪之於是更相追捕|{
	更工衡翻}
賊並解散徙其魁帥於它郡賦田受禀|{
	禀給也帥所類翻}
使安生業自是牛馬放牧不收邑門不閉 盧芳與閔堪使使請降帝立芳為代王堪為代相賜繒二萬匹|{
	繒慈陵翻}
因使和集匈奴芳上疏謝自陳思望闕庭詔報芳朝明年正月|{
	朝直遥翻下同}
初匈奴聞漢購求芳貪得財帛故遣芳還降旣而芳以自歸為功不稱匈奴所遣單于復耻言其計|{
	復扶又翻下同}
故賞遂不行由是大恨入寇尤深 馬援奏宜如舊鑄五銖錢|{
	廢五銖錢事見三十七卷王莽始建國元年}
上從之天下賴其便 盧芳入朝南及昌平|{
	昌平縣属上谷郡賢曰故城在今幽州昌平縣東南}
有詔止令更朝明歲

十七年春正月趙孝公良薨|{
	諡法慈惠愛親曰孝}
初懷縣大姓李子春二孫殺人懷令趙憙窮治其姦|{
	憙許記翻又讀曰熹治直之翻}
二孫自殺收繫子春京師貴戚為請者數十|{
	為于偽翻}
憙終不聽及良病上臨視之問所欲言良曰素與李子春厚今犯罪懷令趙憙欲殺之願乞其命帝曰吏奉法律不可枉也更道它所欲良無復言|{
	復扶又翻}
旣薨上追思良乃貰出子春|{
	貰時夜翻赦也}
遷憙為平原太守|{
	郡國志平原郡在雒陽北千三百里}
二月乙未晦日有食之 |{
	考異曰帝紀乙亥晦袁紀乙未據長歷三月丙申朔帝紀誤}
夏四月乙卯上行幸章陵|{
	章陵故舂陵帝更名}
五月乙卯還宫六月癸巳臨淮懷公衡薨 妖賊李廣攻沒皖城|{
	賢曰}


|{
	皖縣名屬廬江郡故城在今舒州有皖水妖於驕翻皖音下板翻}
遣虎賁中郎將馬援驃騎將軍段志討之秋九月破皖城斬李廣 郭后寵衰數懷怨懟|{
	數所角翻懟直類翻}
上怒之冬十月辛巳廢皇后郭氏立貴人隂氏為皇后詔曰異常之事非國休福不得上夀稱慶郅惲言於帝曰臣聞夫婦之好|{
	好呼到翻}
父不能得之於子况臣能得之於君乎|{
	賢曰得猶制御也司馬遷曰妃匹之愛君不能得之臣父不能得之子况卑下乎}
是臣所不敢言雖然願陛下念其可否之計無令天下有議社稷而已帝曰惲善恕已量主|{
	量音良}
知我必不有所左右而輕天下也|{
	賢曰左右猶向背也言其齊等}
帝進郭后子右翊公輔為中山王以常山郡益中山國|{
	郡國志中山國在雒陽北一千四百里}
郭后為中山太后其餘九國公皆為王 甲申帝幸章陵修園廟祠舊宅觀田廬置酒作樂賞賜時宗室諸母因酣悦相與語曰文叔少時謹信|{
	少詩照翻}
與人不欵曲唯直柔耳今乃能如此帝聞之大笑曰吾治天下亦欲以柔道行之|{
	治直之翻}
十二月還自章陵是歲莎車王賢復遣使奉獻請都護|{
	復扶又翻}
帝賜賢西域都護印綬及車旗黄金錦繡敦煌太守裴遵上言夷狄不可假以大權|{
	唐氏族志伯益之後封於鄉因以為氏後徙封解邑乃去邑從衣郡國志敦煌郡在雒陽西五千里敦徒門翻}
又令諸國失望詔書收還都護印綬更賜賢以漢大將軍印綬其使不肯易遵迫奪之賢由是始恨而猶詐稱大都護移書諸國諸國悉服屬焉匈奴鮮卑赤山烏桓數連兵入塞|{
	鮮卑亦東胡也别依鮮卑山故因號焉漢初為冒頓所破遠竄遼東塞外與烏桓相接未嘗通中國至是始入塞為寇烏桓傳赤山在遼東西北數千里數所角翻下同}
殺略吏民詔拜襄賁令祭肜為遼東太守|{
	賢曰襄賁縣名屬東海郡故城在今沂州臨沂縣南賁音肥郡國志遼東郡在雒陽東北三千六百里祭則介翻肜當作彤}
肜有勇力虜每犯塞常為士卒鋒數破走之肜遵之從弟也|{
	從才用翻}
徵側等寇亂連年詔長沙合浦交趾具車船修道橋通障谿|{
	障與嶂同山也山谿為阻則治橋道以通之}
儲糧穀拜馬援為伏波將軍以扶樂侯劉隆為副|{
	賢曰扶樂縣名屬九真郡余謂賢說誤矣九真郡未嘗有扶樂縣隆初封亢父侯以度田不實免次年封為扶樂鄉侯則扶樂乃鄉名非縣名賢考之不詳也水經註扶樂城在扶溝縣砂水逕其北}
南擊交趾

十八年二月蜀郡守將史歆反攻太守張穆穆踰城走宕渠楊偉等起兵以應歆|{
	宕渠縣屬巴郡宕渠故城在今渠州流江縣東北七十里賢曰宕渠山名因以名縣故城在今渠州流江縣東北俗名車騎城是也師古曰宕音徒浪翻}
帝遣吳漢等將萬餘人討之 甲寅上行幸長安三月幸蒲坂|{
	蒲坂縣屬河東郡}
祠后土 馬援緣海而進隨山刋道千餘里至浪泊上|{
	浪泊在交趾封溪縣界按馬援旣平交趾奏分西里置封溪望海二縣水經曰葉榆水過交趾麊泠縣北分為五水絡交趾郡中其南水自麊泠縣東逕封溪縣北又東逕浪泊馬援以其地高自西里進屯焉宋白曰馬援自九真以南隨山刋木至日南}
與徵側等戰大破之追至禁谿|{
	禁谿水經註及越志皆作金溪其地盖在麊泠縣西南水經註曰徵側走入金谿究三歲乃得之竺芝扶南記曰山溪瀬中謂之究賢曰其地今岑州新昌縣也余按唐志新昌縣屬豐州岑字誤}
賊遂散走夏四月甲戌車駕還宫 戊申上行幸河内戊子還

宫 五月旱 盧芳自昌平還内自疑懼遂復反|{
	復扶又翻}
與閔堪相攻連月匈奴遣數百騎迎芳出塞芳留匈奴中十餘年病死 吳漢發廣漢巴蜀三郡兵|{
	郡國志廣漢郡在雒陽西三千里巴郡在雒陽西二千七百里蜀郡在雒陽西三千一百里}
圍成都百餘日秋七月拔之斬史歆等漢乃乘桴|{
	編竹木以渡水大曰筏小曰桴}
沿江下巴郡楊偉等惶恐解散漢誅其渠帥徙其黨與數百家於南郡長沙而還|{
	帥所類翻還從宣翻又如字}
冬十月庚辰上幸宜城|{
	賢曰宜城縣屬南郡楚之鄢邑也故城在今襄州率道縣南}
還祠章陵十二月還宫 是歲罷州牧置刺史|{
	置州牧事始見三十二卷成帝綏和元年至哀帝建平二年復為刺史元夀二年復為牧}
五官中郎將張純與太僕朱浮奏議禮為人子事大宗降其私親當除今親廟四以先帝四廟代之大司徒涉等奏立元成哀平四廟上自以昭穆次第當為元帝後|{
	昭讀為佋音韶}


十九年春正月庚子追尊宣帝曰中宗始祠昭帝元帝於太廟|{
	賢曰漢官儀曰光武第雖十二於父子之次於成帝為兄弟於哀帝為諸父於平帝為祖父皆不可為之後上至元帝於光武為父故上繼元帝而為九代故河圖云赤九會昌謂光武也然則宣帝為祖昭帝為曾祖故追尊及祠之}
成帝哀帝平帝於長安舂陵節侯以下於章陵其長安章陵皆太守令長侍祠|{
	祭祀志曰時詔曰宗廟處所未定且祫祭高廟其成哀平且祠祭長安故高廟其南陽舂陵歲時且各因故園廟祭祀園廟去太守治所遠者在所令長行太守事侍祠如淳曰宗廟在章陵者南陽太守稱使者往祭不使侯王祭者諸侯不得祖天子凡臨祭宗廟皆為侍祠}
馬援斬徵側徵貳 妖賊單臣傅鎮等相聚入原武城|{
	妖於驕翻單音善原武縣屬河南尹}
自稱將軍詔太中大夫臧宫將兵圍之數攻不下|{
	數所角翻}
士卒死傷帝召公卿諸侯王問方略皆曰宜重其購賞東海王陽獨曰妖巫相劫埶無久立其中必有悔欲亡者但外圍急不得走耳宜小挺緩令得逃亡|{
	賢曰挺解也余據禮記月令挺重囚挺寛也音待鼎翻}
逃亡則一亭長足以禽矣帝然之即敕宫撤圍緩賊賊衆分散夏四月拔原武斬臣鎮等 馬援進擊徵側餘黨都陽等至居風降之|{
	賢曰居風縣名屬九真郡今愛州交州記曰居風冇山出金牛往往夜見光耀十里山有風門常有風}
嶠南悉平|{
	賢曰嶠嶺嶠也爾雅曰山鋭而高曰嶠居廟翻考異曰援傳作都羊帝紀作都陽今從紀又帝紀十八年四月遣援擊交趾十九年四月斬側貳等因擊都陽等降之援傳十七年拜伏波將軍討側貳十八年春軍至浪泊明年正月斬側貳盖紀之所書者援奏破側貳及傳側貳首至雒之時也沈懷遠南越志云徵側奔入金溪穴中二年乃得之援傳近是今從之}
援與越人申明舊制以約束之自後駱越奉行馬將軍故事|{
	賢曰駱者越别名林邑記曰日南盧容浦通銅鼓外越銅鼓即越駱也有銅鼔因得其名馬援取其鼓以鑄銅馬}
閏月戊申進趙齊魯三公爵皆為王 郭后旣廢太子彊意不自安郅惲說太子曰久處疑位上違孝道下近危殆|{
	說輸芮翻處昌呂翻近其靳翻}
不如辭位以奉養母氏太子從之數因左右及諸王陳其懇誠願備藩國|{
	數所角翻}
上不忍遲回者數歲六月戊申詔曰春秋之義立子以貴|{
	春秋公羊傳曰立嫡以長不以賢立子以貴不以長桓公何以貴母貴也母貴則子貴子以母貴母以子貴}
東海王陽皇后之子宜承大統皇太子彊崇執謙願備藩國父子之情重久違之|{
	重難也}
其以彊為東海王立陽為皇太子改名莊

袁宏論曰夫建太子所以重宗統一民心也非有大惡於天下不可移也世祖中興漢業宜遵正道以為後灋今太子之德未虧於外内寵旣多嫡子遷位可謂失矣然東海歸藩謙恭之心彌亮明帝承統友于之情愈篤|{
	論語孔子曰惟孝友于兄弟}
雖長幼易位興廢不同父子兄弟至性無間夫以三代之道處之|{
	間古莧翻處昌呂翻}
亦何以過乎

帝以太子舅隂識守執金吾隂興為衛尉皆輔導太子識性忠厚入雖極言正議及與賓客語未嘗及國事帝敬重之常指識以敕戒貴戚激厲左右焉興雖禮賢好施而門無遊俠|{
	西都之季萭章樓護陳遵等皆俠遊於貴近之門至於此時亦有杜保王磐之徒好呼到翻施式豉翻俠戶頰翻}
與同郡張宗上谷鮮于裒不相好|{
	姓譜鮮于本子姓周武王封箕子于朝鮮支子仲食采於于因以鮮于為氏裒蒲侯翻}
知其有用猶稱所長而達之友人張汜杜禽與興厚善以為華而少實但私之以財終不為言|{
	少詩沼翻為于偽翻}
是以世稱其忠上以沛國桓榮為議郎|{
	沛國即沛郡建武二十年中山王輔徙封沛始為國續漢志凡郎官皆主更直執戟宿衛諸殿門出充車騎惟議郎不在直中議郎秩六百石}
使授太子經車駕幸太學會諸博士論難於前|{
	難乃旦翻}
榮辨明經義每以禮讓相厭|{
	厭服也一葉翻}
不以辭長勝人儒者莫之及特加賞賜又詔諸生雅歌擊磬盡日乃罷帝使左中郎將汝南鍾興授皇太子及宗室諸侯春秋|{
	鍾興為公羊春秋嚴氏學也}
賜興爵關内侯興辭以無功帝曰生教訓太子及諸王侯非大功耶興曰臣師少府丁恭於是復封恭|{
	復扶又翻}
而興遂固辭不受 陳留董宣為雒陽令湖陽公主蒼頭白日殺人因匿主家吏不能得及主出行以奴驂乘宣於夏門亭候之|{
	雒陽十二城門夏門位在亥蔡質漢儀曰雒陽十二城門門一亭賢曰夏門雒陽城北面西頭門門外有萬夀亭乘䋲證翻}
駐車叩馬|{
	叩近也}
以刀畫地大言數主之失|{
	數所具翻}
叱奴下車因格殺之主即還宫訴帝帝大怒召宣欲箠殺之|{
	箠止蕊翻}
宣叩頭曰願乞一言而死帝曰欲何言宣曰陛下聖德中興而縱奴殺人將何以治天下乎|{
	治直之翻}
臣不須箠請得自殺即以頭擊楹|{
	楹柱也}
流血被面|{
	被皮義翻}
帝令小黄門持之|{
	小黄門宦者也屬少府}
使宣叩頭謝主宣不從彊使頓之|{
	彊其兩翻}
宣兩手據地終不肯俯主曰文叔為白衣時藏亡匿死|{
	亡謂亡命死謂犯死罪者}
吏不敢至門今為天子威不能行一令乎帝笑曰天子不與白衣同因敕彊項令出|{
	賢曰彊項言不低屈也}
賜錢三十萬宣悉以班諸吏由是能摶擊豪彊京師莫不震慓|{
	慓當作慄慓音匹妙翻前書音義曰慓疾也非此義}
九月壬申上行幸南陽進幸汝南南頓縣舍置酒會

賜吏民復南頓田租一歲|{
	復芳目翻下同}
父老前叩頭言皇考居此日久陛下識知寺舍|{
	賢曰光武嘗從皇考至南頓故識知官府舍宇風俗通曰寺者嗣也理事之吏嗣續於其中也又曰寺司也諸官府所止皆曰寺}
每來輒加厚恩願賜復十年帝曰天下重器常恐不任|{
	任音壬勝也}
日復一日|{
	日復之復扶又翻下復增同}
安敢遠期千歲乎吏民又言陛下實惜之何言謙也帝大笑復增一歲進幸淮陽梁沛 西南夷棟蠶反殺長吏詔武威將軍劉尚討之路由越巂|{
	嶲音髓}
卭穀王任貴恐尚旣定南邊威灋必行已不得自放縱即聚兵起營多釀毒酒欲先勞軍|{
	勞力到翻}
因襲擊尚尚知其謀即分兵先據卭都|{
	越巂郡治卭都任貴所據宋白曰漢卭都縣唐為巂州越巂縣}
遂掩任貴誅之

二十年春二月戊子車駕還宫 夏四月庚辰大司徒戴涉坐入故太倉令奚涉罪下獄死|{
	無罪加之以罪曰入百官志太倉令屬大司農主受郡國漕轉穀秩六百石下遐稼翻}
帝以三公連職策免大司空竇融 廣平忠侯吳漢病篤車駕親臨問所欲言對曰臣愚無所知識惟願陛下慎無赦而已五月辛亥漢薨詔送葬如大將軍霍光故事|{
	事見二十四卷宣帝地節二年}
漢性彊力每從征伐帝未安常側足而立諸將見戰陳不利|{
	陳讀曰陣}
或多惶懼失其常度漢意氣自若方整厲器械激揚吏士帝時遣人觀大司馬何為還言方修戰攻之具乃歎曰吳公差彊人意隱若一敵國矣|{
	賢曰隱威重之貌言其威重若敵國}
每當出師朝受詔夕則引道初無辦嚴之日|{
	辦皮莧翻具也賢曰即装也避明帝諱改之}
及在朝廷斤斤謹質形於體貌|{
	爾雅曰明明斤斤察也李廵曰斤精詳之察也孫炎曰謹慎之察也斤音靳}
漢嘗出征妻子在後買田業漢還讓之曰軍師在外吏士不足何多買田宅乎遂盡以分與昆弟外家故能任職以功名終 匈奴寇上黨天水遂至扶風|{
	郡國志上黨郡在雒陽北一千五百里天水郡在雒陽西二千里}
帝苦風眩疾甚以隂興領侍中受顧命於雲臺廣室|{
	賢曰尚書曰成王將崩命召公作顧命孔安國註云臨終之命曰顧命顧音古雒陽南宫有雲臺廣德殿余謂廣室者寢殿也據晉書元帝紀有司奏太極殿廣室施絳帳帝令夏施青練帷冬施青布則廣室之為寢殿明矣}
會疾瘳召見興|{
	見賢遍翻}
欲以代吳漢為大司馬興叩頭流涕固讓曰臣不敢惜身誠虧損聖德不可苟冒至誠發中感動左右帝遂聽之太子太傅張湛自郭后之廢稱疾不朝帝彊起之欲以為司徒湛固辭疾篤不能復任朝事|{
	彊其兩翻復扶又翻任音壬朝直遥翻}
遂罷之六月庚寅以廣漢太守河内蔡茂為大司徒太僕朱浮為大司空 壬辰以左中郎將劉隆為驃騎將軍行大司馬事 乙未徙中山王輔為沛王以郭况為大鴻臚帝數幸其第賞賜金帛豐盛莫比|{
	况郭后弟也數恩况者以慰后心耳數所角翻}
京師號况家為金穴秋九月馬援自交阯還平陵孟冀迎勞之|{
	勞力到翻}
援曰

方今匈奴烏桓尚擾北邊欲自請擊之男兒要當死於邉野以馬革裹尸還葬耳何能卧牀上在兒女子手中邪冀曰諒為烈士當如是矣 冬十月甲午上行幸魯東海楚沛國|{
	皆諸皇子封國也後東海王彊兼食魯郡而都於魯時猶為魯王興國}
十二月匈奴寇天水扶風上黨 壬寅車駕還宫 馬援自請擊匈奴帝許之使出屯襄國|{
	賢曰襄國縣名屬趙國今邢州龍岡縣}
詔百官祖道援謂黄門郎梁松竇固曰凡人富貴當使可復賤也如卿等欲不可復賤|{
	復扶又翻}
居高堅自持勉思鄙言松統之子固友之子也 劉尚進兵與棟蠶等連戰皆破之

二十一年春正月追至不韋|{
	孫盛蜀譜曰初秦徙呂不韋子弟宗族於蜀漢武帝開西南夷置郡縣徙呂氏以充之因置不韋縣華陽國志曰武帝通博南山置不韋縣徙南越相呂嘉子孫宗族實之因名不韋以章其先人惡行也郡國志本屬益州郡明帝永平二年分置永昌郡治不韋史記正義不韋縣北去葉榆六百里}
斬棟蠶帥西南諸夷悉平|{
	帥所類翻}
烏桓與匈奴鮮卑連兵為寇代郡以東尤被烏桓之害|{
	被皮義翻}
其居止近塞|{
	近其靳翻}
朝發穹廬暮至城郭五郡民庶家受其辜|{
	五郡謂代郡上谷漁陽右北平遼西也}
至於郡縣損壞百姓流亡邊陲蕭條無復人迹秋八月帝遣馬援與謁者分築堡塞稍興立郡縣或空置太守令長招還人民烏桓居上谷塞外白山者最為彊富援將三千騎擊之無功而還 |{
	考異曰劉昭注補後漢書志亦謂之續漢志其郡國志注云中郎將馬援誤也帝紀冬十月遣援出塞擊烏桓援傳十二月出屯襄國明年秋將三千騎出高柳袁紀在八月祭肜事前今從之}
鮮卑萬餘騎寇遼東太守祭肜率數千人迎擊之自被甲陷陳|{
	被皮義翻陳讀曰陣}
虜大犇投水死者過半遂窮追出塞虜急皆棄兵裸身散走是後鮮卑震怖畏肜不敢復闚塞|{
	裸郎果翻怖普布翻復扶又翻}
冬匈奴寇上谷中山 莎車王賢浸以驕横欲兼并西域數攻諸國|{
	横戶孟翻數所角翻}
重求賦税諸國愁懼車師前王鄯善焉耆等十八國俱遣子入侍獻其珍寶及得見皆流涕稽首|{
	鄯上扇翻稽音啓}
願得都護帝以中國初定北邊未服皆還其侍子|{
	史所謂量時度力也}
厚賞賜之諸國聞都護不出而侍子皆還大憂恐乃與敦煌太守檄|{
	敦徒門翻}
願留侍子以示莎車言侍子見留都護尋出冀且息其兵裴遵以狀聞帝許之

二十二年春閏正月丙戌上幸長安二月己巳還雒陽夏五月乙未晦日有食之 秋九月戊辰地震 冬

十月壬子大司空朱浮免 癸丑以光禄勲杜林為大司空 初陳留劉昆為江陵令縣有火災昆向火叩頭火尋滅|{
	江陵縣屬南郡}
後為弘農太守虎皆負子渡河帝聞而異之徵昆代林為光禄勲帝問昆曰前在江陵反風滅火後守弘農虎北渡河行何德政而致是事對曰偶然耳左右皆笑帝歎曰此乃長者之言也顧命書諸策|{
	策簡策編簡為之漢制天子策書長二尺國史亦用簡策此書諸策即史策也尚書古文書以八寸策}
是歲青州蝗|{
	青州郡濟南平原樂安北海東萊齊國}
匈奴單于輿死子左賢王烏達鞮侯立復死|{
	鞮丁奚翻}
弟左賢王蒲奴立匈奴中連年旱蝗赤地數千里|{
	赤地言在地之物皆盡}
人畜饑疫死耗太半|{
	賢曰三分損一為太半}
單于畏漢乘其敝乃遣使詣漁陽求和親帝遣中郎將李茂報命 |{
	考異曰帝紀是歲匈奴日逐王比遣使詣漁陽請和親使茂報命按明年又有比遣使詣西河内附然則茂所報者非比也今從南匈奴傳}
烏桓乘匈奴之弱擊破之匈奴北徙數千里幕南地空詔罷諸邊郡亭候吏卒以幣帛招降烏桓|{
	降戶江翻}
西域諸國侍子久留敦煌皆愁思亡歸莎車王賢知都護不至擊破鄯善攻殺龜兹王|{
	龜兹前書音丘慈賢曰今龜音丘勿翻兹音沮惟翻盖急言耳}
鄯善王安上書願復遣子入侍|{
	復扶又翻下同}
更請都護都護不出誠廹於匈奴帝報曰今使者大兵未能得出如諸國力不從心東西南北自在也|{
	任其所從}
於是鄯善車師復附匈奴班固論曰孝武之世圖制匈奴患其兼從西國結黨南羌|{
	南羌即湟中諸羌從子容翻}
乃表河曲列四郡|{
	四郡武威張掖酒泉敦煌也}
開玉門通西域以斷匈奴右臂隔絶南羌月氏|{
	斷丁管翻氏音支}
單于失援由是遠遁而幕南無王庭遭值文景玄默養民五世|{
	高惠及呂后文景為五世}
財力有餘士馬彊盛故能睹犀布瑇瑁|{
	瑇音代瑁音妹}
則建珠厓七郡感蒟醬竹杖則開牂牁越巂|{
	蒟音矩牂音臧巂音髓}
聞天馬蒲萄則通大宛安息自是殊方異物四面而至於是開苑囿廣宫室盛帷帳美服玩設酒池肉林以饗四夷之客作魚龍角抵之戲以觀視之|{
	觀古玩翻師古曰視讀曰示觀視之者示之令觀也}
及賂遺贈送|{
	遺于季翻}
萬里相奉師旅之費不可勝計|{
	勝音升}
至於用度不足乃榷酒酤|{
	榷古岳翻酤古暮翻}
筦鹽鐵鑄白金造皮幣筭至車船租及六畜|{
	畜許救翻}
民力屈|{
	屈其勿翻}
財用竭因之以凶年寇盜並起道路不通直指之使始出衣繡杖斧斷斬於郡國|{
	使疏吏翻衣於旣翻斷丁亂翻}
然後勝之是以末年遂棄輪臺之地而下哀痛之詔豈非仁聖之所悔哉|{
	事並見武帝紀}
且通西域近有龍堆遠則葱嶺身熱頭痛懸度之阨淮南杜欽揚雄之論皆以為此天地所以界别區域絶外内也|{
	别彼列翻}
西域諸國各有君長兵衆分弱無所統一雖屬匈奴不相親附匈奴能得其馬畜旃罽而不能統率與之進退|{
	畜許救翻}
與漢隔絶道里又遠得之不為益棄之不為損盛德在我無取於彼故自建武以來西域思漢威德咸樂内屬數遣使置質于漢|{
	樂音洛數所角翻質音致謂侍子也}
願請都護聖上遠覽古今因時之宜辭而未許雖大禹之序西戎周公之讓白雉太宗之卻走馬義兼之矣|{
	禹貢曰西戎即序即就也序次也禹就而序之非尚威服致其貢物也師古曰昔周公相成王越裳氏重九譯而獻白雉成王問周公公曰德不加焉則君子不饗其質政不施焉則君子不臣其遠吾何以獲此物也譯曰吾受命國之黄耉曰久矣天之無烈風雷雨也意者中國有聖人乎然後歸之王稱先王之神所致以薦宗廟却走馬事見十三卷文帝元年}


資治通鑑卷四十三
