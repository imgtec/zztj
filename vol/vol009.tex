<!DOCTYPE html PUBLIC "-//W3C//DTD XHTML 1.0 Transitional//EN" "http://www.w3.org/TR/xhtml1/DTD/xhtml1-transitional.dtd">
<html xmlns="http://www.w3.org/1999/xhtml">
<head>
<meta http-equiv="Content-Type" content="text/html; charset=utf-8" />
<meta http-equiv="X-UA-Compatible" content="IE=Edge,chrome=1">
<title>資治通鑒_10-資治通鑑卷九_10-資治通鑑卷九</title>
<meta name="Keywords" content="資治通鑒_10-資治通鑑卷九_10-資治通鑑卷九">
<meta name="Description" content="資治通鑒_10-資治通鑑卷九_10-資治通鑑卷九">
<meta http-equiv="Cache-Control" content="no-transform" />
<meta http-equiv="Cache-Control" content="no-siteapp" />
<link href="/img/style.css" rel="stylesheet" type="text/css" />
<script src="/img/m.js?2020"></script> 
</head>
<body>
 <div class="ClassNavi">
<a  href="/24shi/">二十四史</a> | <a href="/SiKuQuanShu/">四库全书</a> | <a href="http://www.guoxuedashi.com/gjtsjc/"><font  color="#FF0000">古今图书集成</font></a> | <a href="/renwu/">历史人物</a> | <a href="/ShuoWenJieZi/"><font  color="#FF0000">说文解字</a></font> | <a href="/chengyu/">成语词典</a> | <a  target="_blank"  href="http://www.guoxuedashi.com/jgwhj/"><font  color="#FF0000">甲骨文合集</font></a> | <a href="/yzjwjc/"><font  color="#FF0000">殷周金文集成</font></a> | <a href="/xiangxingzi/"><font color="#0000FF">象形字典</font></a> | <a href="/13jing/"><font  color="#FF0000">十三经索引</font></a> | <a href="/zixing/"><font  color="#FF0000">字体转换器</font></a> | <a href="/zidian/xz/"><font color="#0000FF">篆书识别</font></a> | <a href="/jinfanyi/">近义反义词</a> | <a href="/duilian/">对联大全</a> | <a href="/jiapu/"><font  color="#0000FF">家谱族谱查询</font></a> | <a href="http://www.guoxuemi.com/hafo/" target="_blank" ><font color="#FF0000">哈佛古籍</font></a> 
</div>

 <!-- 头部导航开始 -->
<div class="w1180 head clearfix">
  <div class="head_logo l"><a title="国学大师官网" href="http://www.guoxuedashi.com" target="_blank"></a></div>
  <div class="head_sr l">
  <div id="head1">
  
  <a href="http://www.guoxuedashi.com/zidian/bujian/" target="_blank" ><img src="http://www.guoxuedashi.com/img/top1.gif" width="88" height="60" border="0" title="部件查字,支持20万汉字"></a>


<a href="http://www.guoxuedashi.com/help/yingpan.php" target="_blank"><img src="http://www.guoxuedashi.com/img/top230.gif" width="600" height="62" border="0" ></a>


  </div>
  <div id="head3"><a href="javascript:" onClick="javascript:window.external.AddFavorite(window.location.href,document.title);">添加收藏</a>
  <br><a href="/help/setie.php">搜索引擎</a>
  <br><a href="/help/zanzhu.php">赞助本站</a></div>
  <div id="head2">
 <a href="http://www.guoxuemi.com/" target="_blank"><img src="http://www.guoxuedashi.com/img/guoxuemi.gif" width="95" height="62" border="0" style="margin-left:2px;" title="国学迷"></a>
  

  </div>
</div>
  <div class="clear"></div>
  <div class="head_nav">
  <p><a href="/">首页</a> | <a href="/ShuKu/">国学书库</a> | <a href="/guji/">影印古籍</a> | <a href="/shici/">诗词宝典</a> | <a   href="/SiKuQuanShu/gxjx.php">精选</a> <b>|</b> <a href="/zidian/">汉语字典</a> | <a href="/hydcd/">汉语词典</a> | <a href="http://www.guoxuedashi.com/zidian/bujian/"><font  color="#CC0066">部件查字</font></a> | <a href="http://www.sfds.cn/"><font  color="#CC0066">书法大师</font></a> | <a href="/jgwhj/">甲骨文</a> <b>|</b> <a href="/b/4/"><font  color="#CC0066">解密</font></a> | <a href="/renwu/">历史人物</a> | <a href="/diangu/">历史典故</a> | <a href="/xingshi/">姓氏</a> | <a href="/minzu/">民族</a> <b>|</b> <a href="/mz/"><font  color="#CC0066">世界名著</font></a> | <a href="/download/">软件下载</a>
</p>
<p><a href="/b/"><font  color="#CC0066">历史</font></a> | <a href="http://skqs.guoxuedashi.com/" target="_blank">四库全书</a> |  <a href="http://www.guoxuedashi.com/search/" target="_blank"><font  color="#CC0066">全文检索</font></a> | <a href="http://www.guoxuedashi.com/shumu/">古籍书目</a> | <a   href="/24shi/">正史</a> <b>|</b> <a href="/chengyu/">成语词典</a> | <a href="/kangxi/" title="康熙字典">康熙字典</a> | <a href="/ShuoWenJieZi/">说文解字</a> | <a href="/zixing/yanbian/">字形演变</a> | <a href="/yzjwjc/">金 文</a> <b>|</b>  <a href="/shijian/nian-hao/">年号</a> | <a href="/diming/">历史地名</a> | <a href="/shijian/">历史事件</a> | <a href="/guanzhi/">官职</a> | <a href="/lishi/">知识</a> <b>|</b> <a href="/zhongyi/">中医中药</a> | <a href="http://www.guoxuedashi.com/forum/">留言反馈</a>
</p>
  </div>
</div>
<!-- 头部导航END --> 
<!-- 内容区开始 --> 
<div class="w1180 clearfix">
  <div class="info l">
   
<div class="clearfix" style="background:#f5faff;">
<script src='http://www.guoxuedashi.com/img/headersou.js'></script>

</div>
  <div class="info_tree"><a href="http://www.guoxuedashi.com">首页</a> > <a href="/SiKuQuanShu/fanti/">四库全书</a>
 > <h1>资治通鉴</h1> <!--         下载:【右键另存为】即可 --></div>
  <div class="info_content zj clearfix">
  
<div class="info_txt clearfix" id="show">
<center style="font-size:24px;">10-資治通鑑卷九</center>
    資治通鑑卷九      宋司馬光 撰<br />
<br />
  胡三省 音註<br />
<br />
  漢紀一【起旃蒙協洽盡柔兆涒灘凡二年項羽之分天下王諸將也王沛公於巴蜀漢中曰漢王王怒欲攻羽蕭何諫曰語曰天漢其稱甚美於是就國及滅項羽冇天下遂因始封國名而號曰漢】太祖高皇帝上之上【姓劉氏諱邦字季沛豐邑中陽里人張晏曰諡法無高以帝為功最高而為帝之太祖故特起此名也】<br />
<br />
  元年冬十月【古有三正子為天正周用之以十一月為歲首丑為地正殷用之以十二月為歲首寅為人正夏用之以十三月為歲首秦水德謂建亥之月水得位故以十月為歲首高祖以十月至霸上因而不革至武帝太初元年定歷改用夏正始以寅為歲首至於今因之】沛公至霸上【考異曰史記漢書荀悦漢紀皆云是月五星聚東井按魏收後魏書高允傳崔浩集諸術士考校漢元以來日月薄蝕五星行度并譏前史之失别為魏歷以示允允曰善言遠者必先驗於近且漢元年冬十月五星聚於東井此乃歷術之淺事今譏漢史而不覺此謬恐後之譏今猶今之譏古浩曰所繆云何允曰按星傳金水二星常附日而行冬十月日旦在尾箕昏沒於申南而東井方出於寅北何因背日而行是史官欲神其事不復推之於理浩曰欲為變者何所不可君獨不疑三星之聚而怪二星之來允曰此不可以空言争宜更審之時坐者咸怪東宫少傅游雅曰高君長於歷當不虚言也後歲餘浩謂允曰先所論者本不經心及更考究果如君語以前三月聚於東井非十月也今從之十月不言五星聚】秦王子嬰素車白馬係頸以組封皇帝璽符節降軹道旁【應劭曰子嬰不敢襲帝號但稱王耳素車白馬喪人之服組者天子韍也係頸言欲自殺也師古曰此組謂綬也所以帶璽也組總五翻今綬分絛是也應劭曰璽信也古者尊卑共之左傳襄公在楚季武子使公冶問璽書追而與之秦漢尊者以為信羣下乃避之漢官儀曰子嬰上始皇璽因服御之代代傳受號漢傳國璽沈約曰高祖入關得秦始皇藍田玉璽螭虎紐文曰受天之命皇帝夀昌後代名傳國璽史記正義曰天子有六璽皇帝行璽皇帝之璽皇帝信璽天子行璽天子之璽天子信璽皇帝信璽凡事皆用之璽令施行天子信璽以遷拜封諸侯之璽以發兵皆以武都紫泥封虞喜志林曰傳國璽自在六璽之外天子凡七璽符說文曰信也韋昭曰符發兵符也師古曰符諸所合符以為契者也周禮地官之屬有掌節鄭玄注云節猶信也行者所執之信三禮義宗曰節長尺二寸秦漢以下改為旌之形韋昭曰節者使所擁也釋名云為號令賞罰之節也師古曰節以毛為之上下相重取象竹節將命者持之以為信徐廣曰軹道在霸陵蘇林曰亭名也在長安東十三里漢宫殿疏曰軹道亭東去霸城觀四里觀東去霸水百步括地志軹音紙軹道在雍州萬年縣東北十六里苑中】諸將或言誅秦王沛公曰始懷王遣我固以能寛容【事見上卷秦二世二年】且人已降殺之不祥乃以屬吏【屬付也屬吏者付之於吏使監守之也屬之欲翻】<br />
<br />
  賈誼論曰秦以區區之地致萬乘之權招八州而朝同列【蘇林曰招音翹舉也秦國周職方雍州之地耳既破六國乃舉豫兖青揚荆幽冀并八州有之六國與秦俱稱王是為同列朝直遥翻】百有餘年然後以六合為家【六合謂天地東西南北】殽函為宫一夫作難而七廟墮【墮讀曰隳記天子七廟三昭三穆與太祖之廟而七】身死人手為天下笑者何也仁誼不施而攻守之勢異也<br />
<br />
  沛公西入咸陽諸將皆争走金帛財物之府分之【走音奏】蕭何獨先入收秦丞相府圖藉藏之以此沛公得具知天下阨塞【阨乙革翻塞悉則翻】戶口多少彊弱之處沛公見秦宫室帷帳狗馬重寶婦女以千數意欲留居之樊噲諫曰沛公欲有天下邪將為富家翁邪凡此奢麗之物皆秦所以亡也沛公何用焉願急還霸上無留宫中【樊噲起於狗屠識見如此予謂噲之功當以諫留秦宫為上鴻門誚讓項羽次之姓譜周宣王封仲山甫於樊後因氏焉】沛公不聽張良曰秦為無道故沛公得至此夫為天下除殘賊宜縞素為資【縞素有喪之服謂弔民也為音于偽翻縞音工老翻】今始入秦即安其樂【樂音洛】此所謂助桀為虐且忠言逆耳利於行毒藥苦口利於病願沛公聽樊噲言沛公乃還軍霸上十一月沛公悉召諸縣父老豪傑謂曰父老苦秦苛法久矣【苛音柯細也】吾與諸侯約先入關者王之【王於况翻又如字】吾當王關中與父老約法三章耳殺人者死傷人及盜抵罪【服䖍曰隨輕重制法也李奇曰傷人冇曲直盜贓有多少罪名不可預定凡言抵罪未知抵何罪也師古曰抵至也當也服李二說並得之抵丁禮翻】餘悉除去秦法諸吏民皆案堵如故【案次第也堵墻堵也言不遷動也去羌呂翻】凡吾所以來為父老除害非有所侵暴無恐且吾所以還軍霸上待諸侯至而定約束耳乃使人與秦吏行縣鄉邑告諭之【秦制縣大率方百里十里一亭十亭一鄉所封食邑為於偽翻行下孟翻】秦民大喜争持牛羊酒食獻饗軍士沛公又讓不受曰倉粟多非乏不欲費民民又益喜唯恐沛公不為秦王項羽既定河北率諸侯兵欲西入關先是諸侯吏卒繇使屯戍過秦中者秦中吏卒遇之多無狀【言無善狀也先悉薦翻繇讀曰徭】及章邯以秦軍降諸侯諸侯吏卒乘勝多奴虜使之輕折辱秦吏卒秦吏卒多怨竊言曰章將軍等詐吾屬降諸侯今能入關破秦大善即不能諸侯虜吾屬而東秦又盡誅吾父母妻子奈何諸將微聞其計以告項羽項羽召黥布蒲將軍計曰秦吏卒尚衆其心不服至關不聽事必危不如撃殺之而獨與章邯長史欣都尉翳入秦於是楚軍夜擊阬秦卒二十餘萬人新安城南【班志縣屬弘農郡師古曰今穀州縣括地志新安故城在洛州澠池縣東一十二里】 或說沛公曰秦富十倍天下地形彊聞項羽號章邯為雍王王關中【雍於用翻王關之王於况翻下欲王同】今則來沛公恐不得有此可急使兵守函谷關無内諸侯軍【内音納又如字今傳内從人奴對翻從入者讀為納】 稍徵關中兵以自益距之沛公然其計從之已而項羽至關關門閉聞沛公已定關中大怒使黥布等攻破函谷關十二月項羽進至戲【戲許宜翻】沛公左司馬曹無傷使人言項羽曰沛公欲王關中令子嬰為相珍寶盡有之欲以求封項羽大怒饗士卒期旦日擊沛公軍當是時項羽兵四十萬號百萬在新豐鴻門【新豐縣本秦驪邑高祖七年方置史以後來縣名書之應劭曰太上皇思東歸於是高祖築城市街里以象豐徙豐民以實之故號新豐孟康曰鴻門在新豐東十七里舊大道下阪口名也姚察云在新豐古城東末至戲水道南有斷原南北洞門是也水經注今新豐古城東冇阪長二里餘塹原通道南北洞門有同門汏謂之鴻門孟康言在新豐東十七里無之蓋指縣治而言非謂城也自新豐古城西至霸城五十里霸城西十里則霸水又西二十里則長安城】沛公兵十萬號二十萬在霸上范增說項羽曰沛公居山東時貪財好色今入關財物無所取婦女無所幸此其志不在小吾令人望其氣皆為龍虎成五采此天子氣也【周禮眡祲氏掌十煇之法以觀妖祥辨吉凶即後世所謂望氣者也晉天文志天子氣内赤外黄四方所發之處當有王者若天子欲有遊往處其地亦先發此氣或如城門隐隱在氣霧中或氣象青衣人無手在日西或如龍馬或雜色欝欝衝天者皆帝王之氣】急擊勿失楚左尹項伯者【楚官有左尹右尹】項羽季父也素善張良乃夜馳之沛公軍私見張良具告以事欲呼與俱去曰毋俱死也張良曰臣為韓王送沛公沛公今有急亡去不義不可不語【為於偽翻語牛居翻】良乃入具告沛公沛公大驚良曰料公士卒足以當項羽乎沛公默然曰固不如也且為之奈何張良曰請往謂項伯言沛公之不敢叛也沛公曰君安與項伯有故張良曰秦時與臣游嘗殺人臣活之今事有急故幸來告良沛公曰孰與君少長良曰長於臣【少詩照翻長知两翻】沛公曰君為我呼入吾得兄事之張良出固要項伯【要一遥翻】項伯即入見沛公沛公奉巵酒為夀約為婚姻曰吾入關秋毫不敢有所近【文穎曰毫秋乃成好舉盛而言也師古曰毫成之時端極纎細適足喻小非言其盛近其靳翻】籍吏民封府庫而待將軍所以遣將守關者備它盜之出入與非常也日夜望將軍至豈敢反乎願伯具言臣之不敢倍德也【倍讀曰背】項伯許諾謂沛公曰旦日不可不蚤自來謝沛公曰諾於是項伯復夜去至軍中具以沛公言報項羽因言曰沛公不先破關中公豈敢入乎今人有大功而擊之不義也不如因善遇之項羽許諾沛公旦日從百餘騎來見項羽鴻門謝曰臣與將軍勠力而攻秦將軍戰河北臣戰河南不自意能先入關破秦得復見將軍於此今者有小人之言令將軍與臣有隙項羽曰此沛公左司馬曹無傷言之不然籍何以至此項羽因留沛公與飲范增數目項羽【數所角翻】舉所佩玉玦以示之者三【玦如環而有缺增舉以示羽盖欲其决意殺沛公也】項羽默然不應范增起出召項莊謂曰君王為人不忍若入前為夀【若汝也師古曰凡言為夀者謂進爵於尊者而獻無疆之夀】夀畢請以劒舞因擊沛公於坐殺之【坐徂卧翻】不者若屬皆且為所虜莊則入為夀夀畢曰軍中無以為樂【樂音洛】請以劒舞項羽曰諾項莊拔劒起舞項伯亦拔劒起舞常以身翼蔽沛公莊不得擊於是張良至軍門見樊噲噲曰今日之事何如良曰今項莊拔劒舞其意常在沛公也噲曰此廹矣臣請入與之同命噲即帶劒擁盾入【盾所以蔽身者也盾食尹翻】軍門衛士欲止不内樊噲側其盾以撞衛士仆地【撞丈江翻擊也】遂入披帷立【在旁曰帷釋名曰帷圍也以自障圍也】瞋目視項羽【瞋怒目也昌真翻】頭髮上指目眦盡裂【眦才賜翻又在計翻目際也】項羽按劒而跽曰【跽其紀翻長跪也】客何為者張良曰沛公之參乘樊噲也項羽曰壯士賜之巵酒則與斗巵酒噲拜謝起立而飲之項羽曰賜之彘肩則與一生彘肩樊噲覆其盾於地加彘肩其上拔劒切而㗖之項羽曰壯士復能飲乎【復扶又翻】樊噲曰臣死且不避巵酒安足辭夫秦有虎狼之心殺人如不能舉刑人如恐不勝天下皆叛之懷王與諸將約曰先破秦入咸陽者王之今沛公先破秦入咸陽毫毛不敢有所近【近其靳翻】還軍霸上以待將軍勞苦而功高如此未有封爵之賞而聽細人之說欲誅有功之人此亡秦之續耳竊為將軍不取也項羽未有以應曰坐樊噲從良坐坐須臾沛公起如厠因招樊噲出沛公曰今者出未辭也為之奈何樊噲曰如今人方為刀俎我方為魚肉何辭為於是遂去鴻門去霸上四十里沛公則置車騎【置留也留車騎於鴻門不以自隨】脫身獨騎樊噲夏侯嬰靳彊紀信等四人【姓譜夏侯出自夏后之後杞簡公為楚所滅其弟佗奔魯魯悼公以佗出自夏后氏受爵為侯謂之夏侯因而命氏紀春秋紀侯之後以國為姓】持劒盾步走從驪山下道芷陽間行趣霸上【班志京兆霸陵縣故芷陽也文帝更名間空也投空隙而行間古莧翻趣讀如趣嚮之趨逡須翻後以義推又七喻翻】留張良使謝項羽以白璧獻羽玉斗與亞父沛公謂良曰從此道至吾軍不過二十里耳度我至軍中公乃入【度徒洛翻】沛公已去間至軍中張良入謝曰沛公不勝桮杓不能辭【勝音升】謹使臣良奉白璧一雙再拜獻將軍足下玉斗一雙再拜奉亞父足下項羽曰沛公安在良曰聞將軍有意督過之【師古曰謂視責也】脫身獨去已至軍矣項羽則受璧置之坐上【坐徂卧翻】亞父受玉斗置之地拔劒撞而破之曰唉【歎恨之聲音烏開翻又於其翻】豎子不足與謀奪將軍天下者必沛公也吾屬今為之虜矣沛公至軍立誅殺曹無傷居數日項羽引兵西屠咸陽殺秦降王子嬰燒秦宫室火三月不滅收其貨寶婦女而東秦民大失望【秦民初見沛公無所侵暴而悦及為項羽殘滅失其初所望也】韓生說項羽曰關中阻山帶河四塞之地地肥饒可都以霸項羽見秦宫室皆已燒殘破又心思東歸曰富貴不歸故鄉如衣繡夜行誰知之者韓生曰人言楚人沐猴而冠耳果然【師古曰沐猴獮猴也言雖著人衣冠其心不類人也果然如人之言也】項羽聞之烹韓生項羽使人致命懷王懷王曰如約【言如前約使沛公王關中】項羽怒曰懷王者吾家所立耳非有功伐【張晏曰積功曰伐】何以得專主約天下初發難時【謂初起兵時難乃旦翻】假立諸侯後以伐秦然身被堅執銳首事暴露於野【史記正義曰暴蒲北翻又如字】三年滅秦定天下者皆將相諸君與籍之力也懷王雖無功固當分其地而王之諸將皆曰善春正月羽陽尊懷王為義帝曰古之帝者地方千里必居上游【游即流也言居水之上流】乃徙義帝於江南都郴【史記曰長沙郴縣班志郴縣屬桂陽郡盖高祖定天下方分長沙為桂陽郡也郴丑林翻】二月羽分天下王諸將羽自立為西楚霸王【文頴曰史記貨殖傳淮以北沛陳汝南南郡為西楚彭城以東吴廣陵為東楚衡山九江江南長沙豫章為南楚羽欲都彭城故自稱西楚孟康曰舊名江陵為南楚吴為東楚彭城為西楚師古曰孟說是】王梁楚地九郡都彭城【班志縣屬楚國史記正義曰徐州縣】羽與范增疑沛公而業已講解又惡負約【惡烏路翻】乃隂謀曰巴蜀道險秦之遷人皆居之乃曰巴蜀亦關中地也故立沛公為漢王王巴蜀漢中都南鄭【巴蜀漢中秦所置三郡地也班志南鄭縣屬漢中括地志南鄭縣今梁州治所近世有季文子者蜀人也著蜀鑑曰南鄭自南鄭漢中自漢中南鄭乃古褒國秦未得蜀以前先取之漢中乃金洋均房等州六百里是也秦既得漢中乃分南鄭以隸之而置郡焉南鄭與漢中為一自此始春秋楚人巴人㓕庸即今均房兩州地班志漢中郡治西城今金州上庸郡是也】而三分關中王秦降將以距塞漢路【塞悉則翻】章邯為雍王王咸陽以西都廢邱【班志扶風槐里縣周曰犬邱懿王所都也秦曰廢邱高祖三年更名韋昭曰犬邱周懿王所都秦欲廢周故曰廢邱括地志廢邱故城在雍州始平縣東南一十里】長史欣者故為櫟陽獄掾嘗有德於項梁都尉董翳者本勸章邯降楚故立欣為塞王王咸陽以東至河都櫟陽【韋昭曰塞在長安名桃林塞史記正義曰桃林塞今華州潼關師古曰取河華之固為阨塞耳非桃林也塞先代翻櫟陽縣屬馮翊括地志漢七年分櫟陽城内為萬年縣隋改為大興縣唐復萬年秦獻公所城櫟陽故城在今雍州櫟陽縣東北二十五里項梁嘗有櫟陽逮請蘄獄掾曹咎書以抵欣而事得已所謂有德於梁也櫟音藥】立翳為翟王王上郡都高奴【以上郡北近戎翟因以名國班志高奴縣屬上郡索隱曰今鄜州有高奴城括地志延州城即漢高奴縣杜佑曰延州春秋白翟之地漢為膚施高奴臨河縣地後魏置東夏州後改延州以界内延水為名董翳都高奴今金明縣是也】項羽欲自取梁地乃徙魏王豹為西魏王王河東都平陽【班志縣屬河東郡】瑕邱申陽者張耳嬖臣也先下河南郡迎楚河上故立申陽為河南王都洛陽【括地志洛陽故城在洛州洛陽縣東北二十六里周公所築即成周城也輿地志成周之地秦莊襄王以為洛陽縣三川守治焉後漢都雒陽改為雒漢以火德忌水故去洛旁水而加佳魏於行次為土土水之忌也水得土而流土得水而柔故除佳而加水】韓王成因故都都陽翟趙將司馬卬定河内數有功故立卬為殷王王河内都朝歌【河内郡朝歌縣故殷都也因以名國】徙趙王歇為代王趙相張耳素賢又從入關故立耳為常山王王趙地治襄國【括地志邢州本漢襄國縣秦置三十六郡於此置信都縣屬鉅鹿郡項羽改曰襄國子據班志襄國縣屬趙國信都縣屬信都國漢盖又分為二縣宋白曰趙王歇都襄國今邢州所理龍岡縣城是也】當陽君黥布為楚將常冠軍【冠音古玩翻】故立布為九江王都六【班志當陽縣屬南郡九江應劭曰江自廬江尋陽分為九地理志九江在尋陽縣南皆束合為大江史記正義曰九江郡即夀州楚自陳徙夀春號曰郢秦滅楚于此置九江郡】番君吴芮率百越佐諸侯又從入關故立芮為衡山王都邾【班志邾縣屬江夏郡括地志曰邾故城在黄州黄岡縣東南二十里番音婆】義帝柱國共敖將兵擊南郡功多因立敖為臨江王都江陵【共音龔人姓也姓譜共商諸侯之國晉有左行共華又云鄭共叔段後臨江孟康曰本南郡漢改為臨江國江陵縣屬焉】徙燕王韓廣為遼東王都無終【故無終子之國班志無終縣屬北平郡非遼東郡界盖羽令韓廣都于無終而令併王遼東之地故也】燕將臧荼從楚救趙【姓譜臧姓魯孝公子臧僖伯之後】因從入關故立荼為燕王都薊【班志薊縣屬廣陽國師古曰今幽州縣水經注薊城西北隅有薊丘故名薊音計】徙齊王田市為膠東王都即墨齊將田都從楚救趙因從入關故立都為齊王都臨菑項羽方渡河救趙田安下濟北數城引其兵降項羽故立安為濟北王都博陽【史記正義曰博陽在濟北班志太山郡盧縣濟北王都豈博陽即此地邪余據濟北有博關博陽蓋在博關之南也濟子禮翻】田榮數負項梁【數所角翻】又不肯將兵從楚擊秦以故不封成安君陳餘棄將印去不從入關亦不封客多說項羽曰張耳陳餘一體有功於趙今耳為王餘不可以不封羽不得已聞其在南皮【班志南皮縣屬勃海郡闞駰曰章武有北皮亭故此云南括地志南皮故城在滄州南皮縣北四里】因環封之三縣【環音宦】番君將梅鋗功多封十萬戶侯漢王怒欲攻項羽周勃灌嬰樊噲皆勸之【灌風俗通曰斟灌氏之後】蕭何諫曰雖王漢中之惡不猶愈於死乎漢王曰何為乃死也何曰今衆弗如百戰百敗不死何為夫能詘於一人之下而信於萬乘之上者湯武是也【詘與屈同信與伸同】臣願大王王漢中養其民以致賢人收用巴蜀還定三秦【雍翟塞為三秦】天下可圖也漢王曰善乃遂就國以何為丞相漢王賜張良金百鎰珠二斗良具以獻項伯漢王亦因令良厚遺項伯【遺于季翻】使盡請漢中地項王許之夏四月諸侯罷戲下兵【師古曰戲謂軍之旌麾也先是諸侯從項羽入關者各帥其兵聽命於羽今既受封爵各使就國故總言罷戲下也一說云時從羽在戲水之上故言罷戲下此說非也羽見高祖於鴻門此時已過戲矣又入燒宮室不復在戲也漢書通以戲為麾許宜翻】各就國項王使卒三萬人從漢王之國楚與諸侯之慕從者數萬人從杜南入蝕中【漢京兆杜縣之南也如淳曰蝕入漢中道川谷名近世有程大昌者著雍錄曰以地望求之關中南面背礙南山其冇微徑可逹漢中者唯子午谷在長安正南其次向西則駱谷此蝕中若非駱谷即是子午谷李奇蝕音力】張良送至褒中【地理志褒中縣屬漢中郡師古曰褒中言居褒谷之中括地志褒谷在梁州褒城縣北五十里南中山李文子曰褒谷在褒城北南谷曰褒北谷曰斜同為一谷自褒谷至鳳州界一百三十里始通斜谷斜谷在鳳翔府郿縣谷中褒水所流穴山架木而行】漢王遣良歸韓良因說漢王燒絶所過棧道以備諸侯盗兵【師古曰棧即閣也今謂之閣道蓋架木為之棧士限翻公休士諫翻】且示項羽無東意 田榮聞項羽徙齊王市於膠東而以田都為齊王大怒五月榮發兵距擊田都都亡走楚【走音奏】榮留齊王市不令之膠東市畏項羽竊亡之國榮怒六月追擊殺市於即墨自立為齊王是時彭越在鉅野有衆萬餘人無所屬榮與越將軍印使擊濟北秋七月越擊殺濟北王安榮遂并王三齊之地【三齊謂齊及濟北膠東也王于况翻】又使越擊楚項王命蕭公角將兵擊越越大破楚軍 張耳之國陳餘益怒曰張耳與餘功等也今張耳王餘獨侯此項羽不平乃隂使張同夏說說齊王榮曰【夏說讀曰悦】項羽為天下宰不平盡王諸將善地徙故王於醜地今趙王乃北居代餘以為不可聞大王起兵不聽不義願大王資餘兵擊常山復趙王請以趙為扞蔽【師古曰扞蔽猶言藩屛也】齊王許之遣兵從陳餘項王以張良從漢王韓王成又無功故不遣之國與俱至彭城廢以為穰侯【班志穰縣屬南陽郡】已又殺之 初淮隂人韓信家貧無行【班志武帝元狩六年置臨淮郡淮隂縣屬焉史記正義曰今楚州縣無行言無善行可推擇也行下孟翻】不得推擇為吏又不能治生商賈【行賣曰商坐販曰賈治直之翻】常從人寄食飲人多厭之信釣於城下有漂母見信飢飯信【漂匹妙翻以水擊絮曰漂飯扶晚翻】信喜謂漂母曰吾必有以重報母母怒曰大丈夫不能自食吾哀王孫而進食豈望報乎淮隂屠中少年有侮信者曰若雖長大好帶刀劒中情怯耳因衆辱之曰信能死刺我【刺七亦翻】不能死出我袴下【徐廣曰袴一作胯胯股也漢書作跨同耳師古曰跨兩股之間索隱曰胯枯化翻然尋此文作袴欲依字讀何為不通袴下乃胯下也何必須要作胯下】于是信孰視之俛出袴下蒲伏【俛音免俯首也伏蒲北翻】一市人皆笑信以為怯及項梁渡淮信杖劒從之居麾下無所知名項梁敗又屬項羽羽以為郎中數以策干羽羽不用【數所角翻】漢王之入蜀信亡楚歸漢未知名為連敖坐當斬【據史記表信為連敖典客班表作栗客索隱以為誤徐廣於周竈表以連敖為典客蓋以信表為據李奇曰楚官名如淳曰連敖楚官左傳楚有連尹莫敖其後合為一官號】其輩十三人皆已斬次至信信乃仰視適見滕公曰【滕公即夏侯嬰初從高祖為滕令故號滕公】上不欲就天下乎何為斬壯士滕公奇其言壯其貌釋而不斬與語大說之【說讀曰悦】言於王王拜以為治粟都尉【班表治粟内史秦官掌穀貨都尉盖其屬也至漢改内史為大司農】亦未之奇也信數與蕭何語何奇之漢王至南鄭諸將及士卒皆歌謳思東歸多道亡者信度何等已數言王王不我用即亡去【數所角翻】何聞信亡不及以聞自追之人有言王曰丞相何亡王大怒如失左右手居一二日何來謁王王且怒且喜罵何曰若亡何也何曰臣不敢亡也臣追亡者耳王曰若所追者誰何曰韓信也王復罵曰諸將亡者以十數公無所追追信詐也何曰諸將易得耳至如信者國士無雙【師古曰為國家之奇士予謂何言漢國之士僅有信一人它無與比也】王必欲長王漢中無所事信【長王于况翻】必欲争天下非信無可與計事者顧王策安所决耳王曰吾亦欲東耳安能欝欝久居此乎何曰計必欲東能用信信即留不能用信終亡耳王曰吾為公以為將【吾為于偽翻】何曰雖為將信不留王曰以為大將何曰幸甚於是王欲召信拜之何曰王素慢無禮今拜大將如呼小兒此乃信所以去也王必欲拜之擇良日齋戒設壇場具禮乃可耳王許之諸將皆喜人人各自以為得大將至拜大將乃韓信也一軍皆驚信拜禮畢上坐【上時掌翻坐徂卧翻】王曰丞相數言將軍將軍何以教寡人計策信辭謝因問王曰今東鄉争權天下豈非項王邪【鄉讀曰嚮】漢王曰然曰大王自料勇悍仁彊孰與項王漢王默然良久曰不如也信再拜賀曰惟信亦以為大王不如也【惟史記作惟漢書作唯師古曰唯弋癸翻應辭仲馮曰惟字當屬下句讀如本字予謂如漢書本文則當如師古如史記本文則當如仲馮賀曰句斷】然臣嘗事之請言項王之為人也項王喑噁叱咤【喑於鴆翻噁烏路翻懷怒氣也叱昌栗翻咤卓嫁翻發怒聲也】千人皆廢【晉灼曰廢不收也】然不能任屬賢將【屬之欲翻】此特匹夫之勇耳項王見人恭敬慈愛言語嘔嘔【索隱曰嘔嘔猶姁姁同音吁鄧展曰和好貌】人有疾病涕泣分食飲至使人有功當封爵者印刓敝忍不能予【蘇林曰手弄角訛不忍授也予謂角訛者刓之義敝舊敝也師古曰刓五丸翻蘇林大官翻又音專】此所謂婦人之仁也項王雖覇天下而臣諸侯不居關中而都彭城背義帝之約而以親愛王諸侯不平【背蒲妹翻王于况翻下而王威王王王當王同】逐其故主而王其將相又遷逐義帝置江南所過無不殘滅百姓不親附特刼於威彊耳名雖為覇實失天下心故其彊易弱今大王誠能反其道任天下武勇何所不誅以天下城邑封功臣何所不服以義兵從思東歸之士何所不散【散謂四散而立功劉氏曰用東歸之兵擊東方之敵此敵無不敗散也貢父曰何不散者言義兵無敵諸侯之兵無不離散以敗也】且三秦王為秦將【謂章邯司馬欣董翳三人】將秦子弟數歲矣所殺亡不可勝計【勝音升】又欺其衆降諸侯至新安項王詐坑秦降卒二十餘萬唯獨邯欣翳得脫秦父兄怨此三人痛入骨髓今楚強以威王此三人秦民莫愛也大王之入武關秋毫無所害除秦苛法與秦民約法三章秦民無不欲得大王王秦者於諸侯之約大王當王關中關中民咸知之大王失職入漢中秦民無不恨者今大王舉而東三秦可傳檄而定也于是漢王大喜自以為得信晩遂聽信計部署諸將所擊【師古曰部分而署置之】留蕭何收巴蜀租給軍糧食八月漢王引兵從故道出襲雍【春秋釋例掩其不備曰襲班志故道縣屬武都郡括地志故道今鳳州兩當縣杜佑通典曰故道鳳州梁泉兩當縣地】雍王章邯迎擊漢陳倉雍兵敗還走止戰好畤又敗【班志陳倉縣屬扶風唐之岐州寶鷄縣是也杜佑曰故城在縣東二十里班志好畤縣屬扶風孟康曰畤音止神靈之所止也師古曰即今雍州好畤縣宋白曰漢好畤故縣在今縣東南四十三里奉天縣界好畤故城是也李文子曰在今鳳翔天興縣界】走廢丘漢王遂定雍地東至咸陽引兵圍雍王於廢丘而遣諸將略地塞王欣翟王翳皆降以其地為渭南河上上郡【渭南後曰京兆河上後曰馮翊】令將軍薛歐王吸出武關【歐惡后翻吸音翕】因王陵兵以迎太公呂后項王聞之發兵距之陽夏不得前【夏音賈】王陵者沛人也先聚黨數千人居南陽至是始以兵屬漢項王取陵母置軍中陵使至則東鄉坐陵母欲以招陵【古以東鄉之位為尊沛公見羽於鴻門羽東鄉坐韓信東鄉坐李左車而師事之是也鄉讀曰嚮】陵母私送使者泣曰願為老妾語陵善事漢王漢王長者終得天下毋以老妾故持二心妾以死送使者遂伏劒而死項王怒亨陵母【為于偽翻語牛倨翻亨讀曰烹】項王以故吴令鄭昌為韓王以距漢【班志吴縣屬會稽郡】 張良遺項王書曰漢王失職欲得關中如約即止不敢東又以齊梁反書遺項王曰【遺于季翻】齊欲與趙并滅楚項王以此故無西意而北擊齊 燕王廣不肯之遼東臧荼擊殺之并其地 是歲以内史沛周苛為御史大夫【班表御史大夫秦官位上卿掌副宰相應劭曰侍御史之率故稱大夫】 項王使趣義帝行其羣臣左右稍稍叛之【趣讀曰促】<br />
<br />
  二年冬十月項王密使九江衡山臨江王擊義帝殺之江中【九江王黥布衡山王吴芮臨江王共敖】 陳餘悉三縣兵與齊兵共襲常山常山王張耳敗走漢謁漢王於廢丘漢王厚遇之陳餘迎趙王於代復為趙王趙王德陳餘立以為代王陳餘為趙王弱國初定不之國【為于偽翻】留傅趙王而使夏說以相國守代 張良自韓間行歸漢【間古莧翻】漢王以為成信侯良多病未嘗特將【特將未嘗獨將兵也將即亮翻】常為畫策臣時時從漢王 漢王如陜【陜失冉翻】鎮撫關外父老 河南王申陽降置河南郡 漢王以韓襄王孫信為韓太尉將兵略韓地信急擊韓王昌於陽城昌降十一月立信為韓王常將韓兵從漢王 漢王還都櫟陽 諸將拔隴西 春正月項王北至城陽齊王榮將兵會戰敗走平原平原民殺之項王復立田假為齊王遂北至北海燒夷城郭室屋坑田榮降卒係虜其老弱婦女所過多所殘滅齊民相聚叛之 漢將拔北地虜雍王弟平【章平也雍於用翻】 三月漢王自臨晉渡河【臨晉注見三卷赧王五年師古曰其地在河之西濱東臨晉境即今之同州朝邑界也史記正義曰臨晉即蒲津關】魏王豹降將兵從下河内虜殷王卬置河内郡 初陽武人陳平家貧好讀書里中社【孔隸達曰按祭法曰大夫以下成羣立社曰置社注云大夫不得特立社與民族居百家以上則共立一社今時里社是也如鄭此言則周之政法百家以上得立社其秦漢以來雖非大夫民二十五家以上則得立社故云今之里社又鄭志云月令命民社謂秦社也自秦以下民始得立社】平為宰【師古曰宰主切割肉也】分肉甚均父老曰善陳孺子之為宰平曰嗟乎使平得宰天下亦如是肉矣及諸侯叛秦平事魏王咎於臨濟為太僕【班表大僕秦官掌輿馬應劭曰周穆王所置盖太御衆僕之長也濟子禮翻】說魏王不聽人或讒之平亡去後事項羽賜爵為卿【張晏曰禮秩如卿不治事】殷王反項羽使平擊降之還拜為都尉賜金二十鎰居無何【師古曰言無幾時】漢王攻下殷項王怒將誅定殷將吏平懼乃封其金與印使使歸項王而挺身間行【挺待鼎翻拔也言平拔身間出而行也】杖劒亡渡河歸漢王於修武因魏無知求見漢王漢王召入賜食遣罷就舍平曰臣為事來【為于偽翻】所言不可以過今日于是漢王與語而說之【說讀曰悦】問曰子之居楚何官曰為都尉是日即拜平為都尉使為參乘典護軍【使平典護軍而監護諸將也】諸將盡讙曰【讙音喧譁然不服之聲】大王一日得楚之亡卒未知其高下而即與同載反使監護長者【監古銜翻】漢王聞之愈益幸平 漢王南渡平隂津至洛陽新城【班志平隂縣屬河南郡水經河水逕平隂縣北魏文帝改平隂曰河隂洛陽縣屬河南郡新城時屬縣界惠帝四年始置新城縣括地志洛州伊闕縣在州南七十里本漢新城也隋文帝改新城為伊闕取伊闕山為名】三老董公遮說王曰【班表十里一亭亭有長十亭一鄉鄉有三老掌教化秦制横道自言曰遮說式芮翻】臣聞順德者昌逆德者亡兵出無名【伐有罪則兵出有名】事故不成故曰明其為賊敵乃可服項羽為無道放殺其主【放謂遷義帝于郴殺謂殺之江中殺讀曰弑】天下之賊也夫仁不以勇義不以力【文頴曰以用也已有仁天下歸之可不用勇而天下自服已有義天下奉之可不用力而天下自定】大王宜率三軍之衆為之素服以告諸侯而伐之則四海之内莫不仰德此三王之舉也于是漢王為義帝發喪袒而大哭哀臨三日【如淳曰袒亦如禮袒踊也師古曰袒謂脱衣之袖也袒徒旱翻衆哭曰臨力禁翻】發使告諸侯曰天下共立義帝北面事之今項羽放殺義帝江南大逆無道寡人悉發關中兵收三河士【韋昭曰河南河東河内也】南浮江漢以下願從諸侯王擊楚之殺義帝者【史記正義曰南收三河士發關内兵從雍州入子午道至漢中歷漢水而下東行至徐州擊楚予謂正義之說迂矣三河在彭城之北已不可謂南收三河士若發關内兵南浮江漢獨不能出武關而浮江漢而必入子午谷至漢中而下漢水邪况子午道此時亦未通鑿其可引之而為說乎此特言發三河士以攻其北又南浮江漢下兵以夾攻之也服䖍曰漢名王為諸侯師古曰非也當時漢未有此稱號直言諸侯及王耳】使者至趙陳餘曰漢殺張耳乃從於是漢王求人類張耳者斬之持其頭遺陳餘【遺于季翻】餘乃遣兵助漢 田榮弟横收敗卒得數萬人起城陽【史記正義曰城陽濮州雷澤是子考正義所謂城陽乃班志濟隂郡之城陽縣田榮初與項羽會戰之地榮既敗而北走死於平原羽遂至北海燒夷城郭室屋則濟隂之城陽已隔在羽軍之後田横所起蓋班志城陽國之地春秋莒之故墟也羽既連戰未能克横而漢入彭城遂南從魯出胡陵至蕭以擊漢莒魯舊為鄰國則此城陽為莒之故墟明矣】夏四月立榮子廣為齊王以拒楚項王因留連戰未能下雖聞漢東既擊齊欲遂破之而後擊漢漢王以故得率諸侯兵凡五十六萬人伐楚到外黄彭越將其兵三萬餘人歸漢漢王曰彭將軍收魏地得十餘城【項羽併王梁楚徙魏王豹於河東號西魏王今越所下外黄十餘城皆梁地也】欲急立魏後今西魏王豹真魏後乃拜彭越為魏相國擅將其兵略定梁地漢王遂入彭城收其貨寶美人日置酒高會項王聞之令諸將擊齊而自以精兵三萬人南從魯出胡陵至蕭【魯即伯禽所都秦置魯縣屬薛郡漢後以薛郡為魯國史記正義曰魯今兖州曲阜縣蕭縣秦屬泗水郡唐徐州蕭縣是也】晨擊漢軍而東至彭城日中大破漢軍漢軍皆走相隨入穀泗水死者十餘萬人漢卒皆南走山楚又追擊至靈壁東睢水上【臣瓚曰穀泗二水皆在沛郡彭城水經注睢水出陳留縣西蒗蕩渠東過沛郡相縣又逕彭城郡之靈壁東而東南流項羽敗漢王處也漢書又云東逼穀泗服䖍曰水名也在沛國相縣界又詳睢水逕穀熟而兩分而睢水為蘄水故二水所在枝分通為兼稱穀水之名蓋因地變然則穀水即睢水也睢水又東南至下相而入于泗謂之睢口泗水又東南過彭城縣東北南至下邳入淮孟康曰靈壁故小縣在彭城南史記正義曰靈壁在徐州符離縣西北九十里】漢軍却為楚所擠【擠子詣翻排也又子奚翻】卒十餘萬人皆入睢水水為之不流圍漢王三匝會大風從西北起折木發屋揚沙石窈冥晝晦逢迎楚軍大亂壞散而漢王乃得與數十騎遁去欲過沛收家室而楚亦使人之沛取漢王家家皆亡不與漢王相見漢王道逢孝惠魯元公主【魯元公主帝女也服䖍曰元長也食邑于魯韋昭曰元謚也師古曰公主惠帝姊也以其最長故號曰元不得為謚貢父曰韋昭是也】載以行楚騎追之漢王急推墮二子車下【推吐雷翻】滕公為太僕【滕公夏侯嬰也史記曰嬰從擊秦軍洛陽東賜爵封轉為滕公漢書曰嬰為滕令奉車故號滕公班表太僕秦官掌輿馬應劭曰周穆王所置蓋大御衆僕之長中大夫也】常下收載之如是者三日今雖急不可以驅奈何棄之故徐行漢王怒欲斬之者十餘滕公卒保護脫二子【卒子恤翻】審食其從太公呂后間行求漢王不相遇反遇楚軍【審姓食其名食其音異基將間行以避楚軍乃反與楚軍相遇也間古莧翻下同】楚軍與歸項王常置軍中為質【質音致】是時呂后兄周呂侯為漢將兵居下邑【班志下邑縣屬梁國梁國秦碭郡漢改焉宋白曰今宋州碭山縣即古下邑城】漢王間往從之稍稍收其士卒諸侯皆背漢復與楚【背蒲妺翻】塞王欣翟王翳亡降楚 田横進攻田假假走楚楚殺之横遂復定三齊之地 漢王問羣臣曰吾欲捐關以東等弃之誰可與共功者【師古曰捐關以東謂不自有其地將以與人令其立功共破楚也予謂等弃之者言捐以與人與弃等也】張良曰九江王布楚梟將【師古曰梟謂最勇健也】與項王有隙彭越與齊反梁地此兩人可急使而漢王之將獨韓信可屬大事當一面【師古曰屬委也音之欲翻】即欲捐之捐之此三人則楚可破也初項王擊齊徵兵九江九江王布稱病不往遣將將軍數千人行漢之破楚彭城布又稱病不佐楚楚王由此怨布數使使者誚讓【數所角翻以辭相責曰誚讓誚才笑翻】召布布愈恐不敢往項王方北憂齊趙西患漢所與者獨九江王又多布材【師古曰多者猶重也】欲親用之以故未之擊漢王自下邑徙軍碭遂至虞【班志虞縣屬梁國師古曰今宋州虞城縣宋白曰古虞國舜禪禹封其子商均於虞少康奔虞即此】謂左右曰如彼等者無足與計天下事謁者隨何進曰不審陛下所謂【姓譜隨姓隨侯之後又云杜伯之玄孫會為晉大夫食采于隨曰隨武子後因以為姓】漢王曰孰能為我使九江令之發兵倍楚【倍蒲妹翻】留項王數月我之取天下可以百全隨何曰臣請使之漢王使與二十人俱 五月漢王至滎陽諸敗軍皆會蕭何亦發關中老弱未傅者【傅讀曰附孟康曰古者二十而傅三年耕有一年儲故二十三而後役之如淳曰律言二十三傳之疇官各從其父疇學之高不滿六尺二寸以下為罷癃漢儀注云民年二十三為正一歲為衛士一歲為材官騎士習射御馳戰陳又曰年五十六乃得免為庶民就田里今老弱未傅者皆發之未二十為弱過五十六為老師古曰傅著也言著名籍給公家徭役也】悉詣滎陽漢軍復大振楚起於彭城常乘勝逐北與漢戰滎陽南京索閒【京縣秦屬三川郡漢改曰河南郡即鄭共叔所居京城也應劭曰京縣今有大索小索亭括地志京縣城在鄭州滎陽縣東南二十里滎陽縣即大索城杜預曰成臯城東有大索城又有小索故城在滎陽縣北四里宋白曰滎陽縣故城在鄭州滎澤縣南十七里平原上索水逕其東即項羽圍漢王處秦三川郡亦曾移理於此括地志所謂滎陽縣即大索城乃唐之滎陽縣晉灼曰索音冊師古音求索之索】楚騎來衆漢王擇軍中可為騎將者皆推故秦騎士重泉人李必駱甲【班志重泉縣屬馮翊括地志重泉故城在同州蒲城縣東南四十五里姓譜齊太公之後有公子駱子孫以為氏史記惡來革之玄孫曰大駱】漢王欲拜之必甲曰臣故秦民恐軍不信臣願得大王左右善騎者傅之【如淳曰傅音附猶言隨從者】乃拜灌嬰為中大夫令李必駱甲為左右校尉將騎兵擊楚騎於滎陽東大破之楚以故不能過滎陽而西漢王軍滎陽築甬道屬之河以取敖倉粟【括地志敖倉在鄭州滎陽西北十五里縣門之東北臨汴水南帶三皇山屬之欲翻】 周勃灌嬰等言于漢王曰陳平雖美如冠玉【孟康曰飾冠以玉光好外見中無所有也】其中未必有也臣聞平居家時盜其嫂事魏不客亡歸楚不中【中竹仲翻】又亡歸漢今日大王尊官之令護軍臣聞平受諸將金金多者得善處金少者得惡處平反覆亂臣也願王察之漢王疑之召讓魏無知無知曰臣所言者能也陛下所問者行也今有尾生孝已之行【尾生古之信士或曰即微生高孝已商高宗之子以孝行著行下孟翻】而無益勝負之數陛下何暇用之乎楚漢相距臣進奇謀之士顧其計誠足以利國家不耳【不讀曰否】盜嫂受金又何足疑乎漢王召讓平曰先生事魏不中事楚而去今又從吾游信者固多心乎平曰臣事魏王魏王不能用臣說故去事項王項王不能信人其所任愛非諸項即妻之昆弟雖有奇士不能用聞漢王能用人故歸大王臣躶身來【躶郎果翻赤身也】不受金無以為資誠臣計畫有可采者願大王用之使無可用者金具在請封輸官得請骸骨漢王乃謝厚賜拜為護軍中尉盡護諸將諸將乃不敢復言 魏王豹謁歸視親疾【謁歸謂謁告而歸也】至則絶河津反為楚【豹都平陽在河東故斷其津濟以拒漢軍為于偽翻】六月漢王還櫟陽壬午立子盈為太子赦罪人 漢兵引水灌廢丘廢<br />
<br />
  丘降章邯自殺盡定雍地以為中地北地隴西郡【自置中地郡後至九年罷屬内史武帝建元六年分為右内史太初元年更名主爵都尉為右扶風】 關中大飢米斛萬錢人相食令民就食蜀漢初秦之亡也豪桀争取金玉宣曲任氏獨窖倉粟【漢有長水宣曲胡騎高祖功臣有宣曲侯蓋地名也張楫曰宣曲宫名在昆明池西師古曰宣曲觀名索隱曰上林賦云西馳宣曲當在京輔今闕其地窖工孝翻穿地以藏粟也】及楚漢相距滎陽民不得耕種而豪桀金玉盡歸任氏任氏以此起富者數世 秋八月漢王如滎陽命蕭何守關中侍太子為法令約束立宗廟社稷宫室縣邑事有不及奏决者輒以便宜施行上來以聞計關中戶口轉漕調兵以給軍未嘗乏絶【調徒弔翻】 漢王使酈食其往說魏王豹且召之豹不聽曰漢王慢而侮人罵詈諸侯羣臣如罵奴耳吾不忍復見也【復扶又翻】於是漢王以韓信為左丞相與灌嬰曹參俱擊魏漢王問食其魏大將誰也對曰柏直【姓譜柏柏皇氏之後顓頊師柏招帝嚳師柏景春秋柏國為楚所滅】王曰是口尚乳臭【言其少不經事弱不任事若未離乳保之懷者】安能當韓信騎將誰也曰馮敬曰是秦將馮無擇子也雖賢不能當灌嬰步卒將誰也曰項它曰不能當曹參吾無患矣韓信亦問酈生魏得無用周叔為大將乎酈生曰柏直也信曰豎子耳遂進兵魏王盛兵蒲坂以塞臨晉【塞悉則翻】信乃益為疑兵陳船欲渡臨晉而伏兵從夏陽以木罌渡軍襲安邑【班志夏陽縣屬馮翊秦之少梁也秦惠文王十一年更名史記正義曰夏陽在同州止韓城界木罌服䖍曰以木押縛罌缶以渡也韋昭曰以木為器如罌缶以渡軍無船且尚密也師古曰服說是罌缶謂瓶之大腹小口者也罌一政翻康於耕翻】魏王豹驚引兵迎信九月信擊虜豹傳詣滎陽【傳直戀翻言以驛馬傳送詣漢王所】悉定魏地置河東上黨太原郡 漢之敗於彭城而西也陳餘亦覺張耳不死即背漢【背蒲妺翻】韓信既定魏使人請兵三萬人願以北舉燕趙東擊齊南絶楚粮道漢王許之乃遣張耳與俱引兵東北擊趙代【時趙王歇王趙陳餘王代】後九月信破代兵禽夏說於閼與信之下魏破代漢輒使人收其精兵詣滎陽以距楚<br />
<br />
  資治通鑑卷九  <br>
   </div> 

<script src="/search/ajaxskft.js"> </script>
 <div class="clear"></div>
<br>
<br>
 <!-- a.d-->

 <!--
<div class="info_share">
</div> 
-->
 <!--info_share--></div>   <!-- end info_content-->
  </div> <!-- end l-->

<div class="r">   <!--r-->



<div class="sidebar"  style="margin-bottom:2px;">

 
<div class="sidebar_title">工具类大全</div>
<div class="sidebar_info">
<strong><a href="http://www.guoxuedashi.com/lsditu/" target="_blank">历史地图</a></strong>  
<a href="http://www.880114.com/" target="_blank">英语宝典</a>  
<a href="http://www.guoxuedashi.com/13jing/" target="_blank">十三经检索</a> 
<br><strong><a href="http://www.guoxuedashi.com/gjtsjc/" target="_blank">古今图书集成</a></strong> 
<a href="http://www.guoxuedashi.com/duilian/" target="_blank">对联大全</a> <strong><a href="http://www.guoxuedashi.com/xiangxingzi/" target="_blank">象形文字典</a></strong> 

<br><a href="http://www.guoxuedashi.com/zixing/yanbian/">字形演变</a>  <strong><a href="http://www.guoxuemi.com/hafo/" target="_blank">哈佛燕京中文善本特藏</a></strong>
<br><strong><a href="http://www.guoxuedashi.com/csfz/" target="_blank">丛书&方志检索器</a></strong> <a href="http://www.guoxuedashi.com/yqjyy/" target="_blank">一切经音义</a>  

<br><strong><a href="http://www.guoxuedashi.com/jiapu/" target="_blank">家谱族谱查询</a></strong>  <strong><a href="http://shufa.guoxuedashi.com/sfzitie/" target="_blank">书法字帖欣赏</a></strong> 
<br>

</div>
</div>


<div class="sidebar" style="margin-bottom:0px;">

<font style="font-size:22px;line-height:32px">QQ交流群9:489193090</font>


<div class="sidebar_title">手机APP 扫描或点击</div>
<div class="sidebar_info">
<table>
<tr>
	<td width=160><a href="http://m.guoxuedashi.com/app/" target="_blank"><img src="/img/gxds-sj.png" width="140"  border="0" alt="国学大师手机版"></a></td>
	<td>
<a href="http://www.guoxuedashi.com/download/" target="_blank">app软件下载专区</a><br>
<a href="http://www.guoxuedashi.com/download/gxds.php" target="_blank">《国学大师》下载</a><br>
<a href="http://www.guoxuedashi.com/download/kxzd.php" target="_blank">《汉字宝典》下载</a><br>
<a href="http://www.guoxuedashi.com/download/scqbd.php" target="_blank">《诗词曲宝典》下载</a><br>
<a href="http://www.guoxuedashi.com/SiKuQuanShu/skqs.php" target="_blank">《四库全书》下载</a><br>
</td>
</tr>
</table>

</div>
</div>


<div class="sidebar2">
<center>


</center>
</div>

<div class="sidebar"  style="margin-bottom:2px;">
<div class="sidebar_title">网站使用教程</div>
<div class="sidebar_info">
<a href="http://www.guoxuedashi.com/help/gjsearch.php" target="_blank">如何在国学大师网下载古籍?</a><br>
<a href="http://www.guoxuedashi.com/zidian/bujian/bjjc.php" target="_blank">如何使用部件查字法快速查字?</a><br>
<a href="http://www.guoxuedashi.com/search/sjc.php" target="_blank">如何在指定的书籍中全文检索?</a><br>
<a href="http://www.guoxuedashi.com/search/skjc.php" target="_blank">如何找到一句话在《四库全书》哪一页?</a><br>
</div>
</div>


<div class="sidebar">
<div class="sidebar_title">热门书籍</div>
<div class="sidebar_info">
<a href="/so.php?sokey=%E8%B5%84%E6%B2%BB%E9%80%9A%E9%89%B4&kt=1">资治通鉴</a> <a href="/24shi/"><strong>二十四史</strong></a>&nbsp; <a href="/a2694/">野史</a>&nbsp; <a href="/SiKuQuanShu/"><strong>四库全书</strong></a>&nbsp;<a href="http://www.guoxuedashi.com/SiKuQuanShu/fanti/">繁体</a>
<br><a href="/so.php?sokey=%E7%BA%A2%E6%A5%BC%E6%A2%A6&kt=1">红楼梦</a> <a href="/a/1858x/">三国演义</a> <a href="/a/1038k/">水浒传</a> <a href="/a/1046t/">西游记</a> <a href="/a/1914o/">封神演义</a>
<br>
<a href="http://www.guoxuedashi.com/so.php?sokeygx=%E4%B8%87%E6%9C%89%E6%96%87%E5%BA%93&submit=&kt=1">万有文库</a> <a href="/a/780t/">古文观止</a> <a href="/a/1024l/">文心雕龙</a> <a href="/a/1704n/">全唐诗</a> <a href="/a/1705h/">全宋词</a>
<br><a href="http://www.guoxuedashi.com/so.php?sokeygx=%E7%99%BE%E8%A1%B2%E6%9C%AC%E4%BA%8C%E5%8D%81%E5%9B%9B%E5%8F%B2&submit=&kt=1"><strong>百衲本二十四史</strong></a>  <a href="http://www.guoxuedashi.com/so.php?sokeygx=%E5%8F%A4%E4%BB%8A%E5%9B%BE%E4%B9%A6%E9%9B%86%E6%88%90&submit=&kt=1"><strong>古今图书集成</strong></a>
<br>

<a href="http://www.guoxuedashi.com/so.php?sokeygx=%E4%B8%9B%E4%B9%A6%E9%9B%86%E6%88%90&submit=&kt=1">丛书集成</a> 
<a href="http://www.guoxuedashi.com/so.php?sokeygx=%E5%9B%9B%E9%83%A8%E4%B8%9B%E5%88%8A&submit=&kt=1"><strong>四部丛刊</strong></a>  
<a href="http://www.guoxuedashi.com/so.php?sokeygx=%E8%AF%B4%E6%96%87%E8%A7%A3%E5%AD%97&submit=&kt=1">說文解字</a> <a href="http://www.guoxuedashi.com/so.php?sokeygx=%E5%85%A8%E4%B8%8A%E5%8F%A4&submit=&kt=1">三国六朝文</a>
<br><a href="http://www.guoxuedashi.com/so.php?sokeytm=%E6%97%A5%E6%9C%AC%E5%86%85%E9%98%81%E6%96%87%E5%BA%93&submit=&kt=1"><strong>日本内阁文库</strong></a> <a href="http://www.guoxuedashi.com/so.php?sokeytm=%E5%9B%BD%E5%9B%BE%E6%96%B9%E5%BF%97%E5%90%88%E9%9B%86&ka=100&submit=">国图方志合集</a> <a href="http://www.guoxuedashi.com/so.php?sokeytm=%E5%90%84%E5%9C%B0%E6%96%B9%E5%BF%97&submit=&kt=1"><strong>各地方志</strong></a>

</div>
</div>


<div class="sidebar2">
<center>

</center>
</div>
<div class="sidebar greenbar">
<div class="sidebar_title green">四库全书</div>
<div class="sidebar_info">

《四库全书》是中国古代最大的丛书,编撰于乾隆年间,由纪昀等360多位高官、学者编撰,3800多人抄写,费时十三年编成。丛书分经、史、子、集四部,故名四库。共有3500多种书,7.9万卷,3.6万册,约8亿字,基本上囊括了古代所有图书,故称“全书”。<a href="http://www.guoxuedashi.com/SiKuQuanShu/">详细>>
</a>

</div> 
</div>

</div>  <!--end r-->

</div>
<!-- 内容区END --> 

<!-- 页脚开始 -->
<div class="shh">

</div>

<div class="w1180" style="margin-top:8px;">
<center><script src="http://www.guoxuedashi.com/img/plus.php?id=3"></script></center>
</div>
<div class="w1180 foot">
<a href="/b/thanks.php">特别致谢</a> | <a href="javascript:window.external.AddFavorite(document.location.href,document.title);">收藏本站</a> | <a href="#">欢迎投稿</a> | <a href="http://www.guoxuedashi.com/forum/">意见建议</a> | <a href="http://www.guoxuemi.com/">国学迷</a> | <a href="http://www.shuowen.net/">说文网</a><script language="javascript" type="text/javascript" src="https://js.users.51.la/17753172.js"></script><br />
  Copyright &copy; 国学大师 古典图书集成 All Rights Reserved.<br>
  
  <span style="font-size:14px">免责声明:本站非营利性站点,以方便网友为主,仅供学习研究。<br>内容由热心网友提供和网上收集,不保留版权。若侵犯了您的权益,来信即刪。scp168@qq.com</span>
  <br />
ICP证:<a href="http://www.beian.miit.gov.cn/" target="_blank">鲁ICP备19060063号</a></div>
<!-- 页脚END --> 
<script src="http://www.guoxuedashi.com/img/plus.php?id=22"></script>
<script src="http://www.guoxuedashi.com/img/tongji.js"></script>

</body>
</html>
