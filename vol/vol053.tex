\section{資治通鑑卷五十三}
宋 司馬光 撰

胡三省 音註

漢紀四十五|{
	起柔兆閹茂盡柔兆涒灘凡十一年}


孝質皇帝|{
	諱纘章帝曾孫渤海孝王鴻之子也諡法忠正無邪曰質伏侯古今注曰纘之字曰繼}


本初元年夏四月庚辰令郡國舉明經詣太學自大將軍以下皆遣子受業歲滿課試拜官有差又千石六百石四府掾屬三署郎|{
	三署郎五官署郎及左右署郎也屬光祿勲掾俞絹翻}
四姓小侯先能通經者各令隨家法其高第者上名牒|{
	此時蓋以梁氏入四姓隂竇諸后族衰廢者未必得豫也名牒者書名於牒上之上時掌翻}
當以次賞進自是遊學增盛至三萬餘生|{
	此鄧后臨朝之故智梁后踵而行之耳遊學增盛亦干名蹈利之徒何足尚也 或問曰太學諸生三萬人漢末互相標榜清議此乎出子盡以為干名蹈利之徒可乎答曰積水成淵蛟龍生焉謂其間無其人則不可然互相標榜者實干名蹈利之徒所為也禍李膺諸人者非太學諸生諸生見其立節從而標榜以重清議耳不然則郭泰仇香亦游太學泰且拜香而欲師之泰為八顧之首仇香曾不預標榜之列豈清議不足尚歟抑香隱德無能名歟}
五月庚寅徙樂安王鴻為勃海王 海水溢漂沒民居 六月丁巳赦天下 帝少而聰慧|{
	少詩照翻}
嘗因朝會目梁冀曰|{
	目者眨目而注視之朝直遙翻}
此跋扈將軍也|{
	賢曰跋扈猶彊梁也余按爾雅山卑而大扈跋者不由蹊隧而行言彊梁之人行不由正路山卑而大且欲跋而踰之故曰跋扈蜀本注甚鄙淺兹不復録詳見辯誤}
冀聞深惡之|{
	惡烏路翻下同}
閏月甲申冀使左右置毒於煮餅而進之|{
	煮餅今湯餅也釋名餅并也搜麥麪使合并也束晳曰禮仲春之月天子食麥而朝事之籩煮麥為麪内則諸饌不設䴵䴵之作也其來近矣湯餅煮麪也黄庭堅文煮麥深注湯}
帝苦煩盛使促召太尉李固固入前問帝得患所由帝尚能言曰食煮餅今腹中悶得水尚可活時冀亦在側曰恐吐不可飲水|{
	吐土故翻嘔也}
語未絶而崩|{
	年九歲}
固伏尸號哭|{
	言伏地而號哭其狀如尸也號戶高翻}
推舉侍醫冀慮其事泄大惡之|{
	推舉者劾舉其侍疾無狀而推究其姦也設於此時固能窮冀弑君之罪儻不能正其誅以身死之豈不忠壯既不能然又且俛首於其間欲以立長之議矯而正之卒死於兇豎之手可謂忠有餘而才不足矣惡烏路翻}
將議立嗣固與司徒胡廣司空趙戒先與冀書曰天下不幸頻年之間國祚三絶|{
	賢曰順帝崩冲帝立一年崩質帝立一年崩凡三絶}
今當立帝天下重器誠知太后垂心將軍勞慮詳擇其人務存聖明然愚情眷眷竊獨有懷遠尋先世廢立舊儀近見國家踐祚前事未嘗不詢訪公卿廣求羣議令上應天心下合衆望傳曰以天下與人易為天下得人難|{
	孟子之言為于偽翻}
昔昌邑之立昏亂日滋霍光憂愧發憤悔之折骨|{
	折而設翻}
自非博陸忠勇延年奮發大漢之祀幾將傾矣|{
	事見二十四卷昭帝元平元年幾居希翻}
至憂至重可不熟慮悠悠萬事唯此為大|{
	就冀而言萬事皆可付之悠悠至於立嗣關天下國家之大}
國之興衰在此一舉冀得書乃召三公中二千石列侯大議所立固廣戒及大鴻臚杜喬皆以為清河王蒜明德著聞又屬最尊親|{
	蒜於質帝為兄尊也同出樂安王寵親也臚陵如翻}
宜立為嗣朝廷莫不歸心而中常侍曹騰嘗謁蒜蒜不為禮宦者由此惡之|{
	惡烏路翻}
初平原王翼既貶歸河間|{
	事見五十卷安帝建光元年}
其父請分蠡吾縣以侯之|{
	蠡吾縣前漢屬涿郡時屬河間國賢曰蠡吾故城在今瀛州博野縣西蠡音禮翼父河間孝王開也}
順帝許之翼卒子志嗣梁太后欲以女弟妻志|{
	妻七細翻}
徵到夏門亭會帝崩梁冀欲立志衆論既異憤憤不得意而未有以相奪|{
	賢曰未有别理而易奪之}
曹騰等聞之夜往說冀曰將軍累世有椒房之親|{
	說輸芮翻下同累世椒房謂恭懷后及太后也}
秉攝萬機賓客縱横|{
	横戶孟翻}
多有過差清河王嚴明若果立則將軍受禍不久矣不如立蠡吾侯富貴可長保也冀然其言明日重會公卿|{
	重直用翻再也}
冀意氣凶凶|{
	凶凶言意氣惡暴也}
言辭激切自胡廣趙戒以下莫不懾憚|{
	懾之舌翻}
皆曰惟大將軍令獨李固杜喬堅守本議冀厲聲曰罷會固猶望衆心可立|{
	以衆心屬於清河王猶望可立也}
復以書勸冀|{
	復扶又翻}
冀愈激怒丁亥冀說太后先策免固|{
	為殺李固杜喬張本}
戊子以司徒胡廣為太尉司空趙戒為司徒與大將軍冀參録尚書事太僕袁湯為司空湯安之孫也庚寅使大將軍冀持節以王青蓋車迎蠡吾侯志入南宫其日即皇帝位時年十五太后猶臨朝政 秋七月乙卯葬孝質皇帝於靜陵|{
	賢曰靜陵在雒陽東南三十里}
大將軍掾朱穆奏記勸戒梁冀曰明年丁亥之歲刑德合於乾位|{
	賢曰歷法太歲在丁壬歲德在北宫太歲在亥卯歲刑亦在北宫故曰合於乾位掾俞絹翻}
易經龍戰之會|{
	易坤卦上六龍戰于野隂疑于陽也}
陽道將勝隂道將負願將軍專心公朝|{
	朝直遙翻}
割除私欲廣求賢能斥遠佞惡為皇帝置師傅|{
	遠于願翻為于偽翻}
得小心忠篤敦禮之士將軍與之俱入參勸講授師賢法古此猶倚南山坐平原也|{
	喻其安而無傾}
誰能傾之議郎大夫之位本以式序儒術高行之士|{
	式用也}
今多非其人九卿之中亦有乖其任者惟將軍察焉又薦种暠欒巴等冀不能用穆暉之孫也|{
	朱暉事章帝}
九月戊戌追尊河間孝王為孝穆皇夫人趙氏曰孝穆后|{
	諡法布德執義曰穆中情見貌曰穆}
廟曰清廟陵曰樂成陵|{
	樂成縣屬河間國}
蠡吾先侯曰孝崇皇|{
	沈約曰諡法所不載者如孝崇皇之類是也}
廟曰烈廟陵曰博陵|{
	賢曰博陵本蠡吾縣之地也陵在今瀛州博野縣西}
皆置令丞使司徒持節奉策書璽綬祠以太牢|{
	璽斯氏翻綬音受}
冬十月甲午尊帝母氏為博園貴人|{
	音偃史記姓咎繇之後貴人諱明本蠡吾侯之媵妾博園博陵寢園}
滕撫性方直不交權勢為宦官所惡論討賊功當封|{
	討揚徐賊之功也惡烏路翻}
太尉胡廣承旨奏黜之卒於家

孝桓皇帝上之上|{
	諱志章帝曾孫蠡吾侯翼之子諡法克敵服遠曰桓伏侯古今注志之字曰意}


建和元年春正月辛亥朔日有食之 戊午赦天下三月龍見譙|{
	譙縣屬沛國見賢遍翻}
夏四月庚寅京師地震立阜陵王代兄勃遒亭侯便為阜陵王|{
	阜陵王延傳國五世至代代薨無子國絶今以便紹封遒才由翻}
六月太尉胡廣罷光禄勲杜喬為太尉 |{
	考異曰帝紀云大司農杜喬喬傳喬自司農累遷為大鴻臚光禄勲乃為太尉袁紀亦然荀淑傳云光禄勲杜喬舉淑方正今從之}
自李固之廢朝野喪氣|{
	喪息浪翻}
羣臣側足而立唯喬正色無所回橈|{
	賢曰回邪也橈曲也橈音奴高翻}
由是朝野皆倚望焉 秋七月渤海孝王鴻薨無子太后立帝弟蠡吾侯悝為渤海王以奉鴻祀|{
	悝苦回翻}
詔以定策功益封梁冀萬三千戶封冀弟不疑為潁陽侯|{
	潁陽縣屬潁川郡}
蒙為西平侯冀子為襄邑侯胡廣為安樂侯|{
	按廣傳封淯陽縣之安樂鄉樂音洛}
趙戒為厨亭侯袁湯為安國侯|{
	安國亦亭侯}
又封中常侍劉廣等皆為列侯|{
	按曹騰傳廣騰及州輔等七人皆封亭侯}
杜喬諫曰古之明君皆以用賢賞罰為務失國之主其朝豈無貞幹之臣|{
	貞與楨同幹與榦同築垣墻必須楨榦以喻立國必須賢才朝直遙翻}
典誥之篇哉|{
	謂封爵之典策詔誥以授有功具有故事}
患得賢不用其謀韜書不施其教聞善不信其義聽讒不審其理也陛下自藩臣即位天人屬心|{
	屬之欲翻下冀屬同}
不急忠賢之禮而先左右之封|{
	先悉薦翻}
梁氏一門宦者微孽並帶無功之紱裂勞臣之土|{
	孽魚列翻紱音弗}
其為乖濫胡可勝言|{
	勝音升}
夫有功不賞為善失其望姦回不詰為惡肆其凶|{
	詰去吉翻}
故陳資斧而人靡畏|{
	前書音義曰資利也}
班爵位而物無勸苟遂斯道豈伊傷政為亂而已喪身亡國可不慎哉書奏不省|{
	喪息浪翻省悉景翻 考異曰喬傳此章在為太尉前袁紀在為太尉後今從袁紀}
八月乙未立皇后梁氏|{
	考異曰皇后紀袁紀皆云八月而無日帝紀云七月乙未以長歷考之七月戊申朔無乙未乙未八月十八日也蓋帝紀脱八月字}
梁冀欲以厚禮迎之杜喬據執舊典不聽|{
	漢書舊儀聘皇后黄金萬斤呂后為惠帝娶魯元公主女特優其禮為二萬斤儀禮納采用鴈鄭玄注云納其采擇之禮用鴈取順隂陽往來也周禮王者穀圭以聘女鄭玄曰士大夫以上乃以玄纁束帛天子加以穀圭諸侯加以大璋禮言以圭而漢用璧形制雖異為玉同也時依孝惠納后故事聘黄金二萬斤納采鴈璧乘馬束帛一依舊典乘馬馬四匹也雜記曰納幣一束束五兩兩五尋蓋每端二丈也}
冀屬喬舉汜宫為尚書|{
	屬之欲翻汜符咸翻姓也皇甫謐曰木姓凡氏遭秦亂避地於汜水因氏焉}
喬以宫為臧罪不用|{
	臧古贓字通}
由是日忤於冀|{
	忤五故翻}
九月丁卯京師地震喬以災異策免冬十月以司徒趙戒為太尉司空袁湯為司徒前大尉胡廣為司空 宦者唐衡左悺共譖杜喬於帝|{
	賢曰悺音工喚翻又音綰}
曰陛下前當即位喬與李固抗議以為不堪奉漢宗祀|{
	賢曰抗舉也宗祀大宗之祀也}
帝亦怨之十一月清河劉文與南郡妖賊劉鮪交通|{
	鮪于軌翻}
妄言清河王當統天下欲共立蒜事覺文等遂刼清河相謝暠曰當立王為天子以暠為公暠罵之文刺殺暠於是捕文鮪誅之有司劾奏蒜|{
	暠工老翻刺七亦翻劾戶槩翻又戶得翻}
坐貶爵為尉氏侯|{
	尉氏縣屬陳留郡應劭曰古獄官曰尉氏鄭之别獄也臣瓚曰鄭大夫尉氏之邑故以為邑名}
徙桂陽自殺梁冀因誣李固杜喬云與文鮪等交通請逮按罪太后素知喬忠不許 |{
	考異曰喬傳云策免而已喬前已免官傳誤也}
冀遂收固下獄|{
	下遐稼翻}
門生渤海王調貫械上書證固之枉河内趙承等數十人亦要鈇鑕詣闕通訴|{
	要讀曰腰鈇斧也鑕音質椹也}
太后詔赦之及出獄京師市里皆稱萬歲冀聞之大驚畏固名德終為己害乃更據奏前事|{
	前事即文鮪事也}
大將軍長史吳祐傷固之枉與冀爭之冀怒不從從事中郎馬融主為冀作章表融時在坐|{
	為于偽翻坐才卧翻}
祐謂融曰李公之罪成於卿手李公若誅卿何面目視天下人|{
	言為冀誣陷忠良將無顏以見人也}
冀怒起入室祐亦徑去固遂死于獄中臨命與胡廣趙戒書曰固受國厚恩是以竭其股肱不顧死亡志欲扶持王室比隆文宣|{
	賢曰文帝宣帝皆羣臣迎立能興漢祚}
何圖一朝梁氏迷謬公等曲從以吉為凶成事為敗乎漢家衰微從此始矣公等受主厚禄顛而不扶傾覆大事後之良史豈有所私固身已矣於義得矣夫復何言|{
	復扶又翻}
廣戒得書悲慙皆長歎流涕而已冀使人脅杜喬曰早從宜|{
	賢曰從宜令其自盡也}
妻子可得全喬不肯明日冀遣騎至其門|{
	騎奇寄翻}
不聞哭者遂白太后收繫之亦死獄中冀暴固喬尸於城北四衢令有敢臨者加其罪|{
	爾雅曰四達謂之衢城北即夏門亭也臨力鴆翻哭也}
固弟子汝南郭亮尚未冠左提章鉞|{
	冠古玩翻賢曰章謂所上章也鉞斧也}
右秉鈇鑕詣闕上書乞收固尸不報與南陽董班俱往臨哭守喪不去夏門亭長呵之曰卿曹何等腐生|{
	賢曰腐生猶言腐儒也}
公犯詔書欲干試有司乎亮曰義之所動豈知性命何為以死相懼邪太后聞之皆赦不誅杜喬故掾陳留楊匡號泣星行|{
	掾俞絹翻號戶刀翻星行者見星而行見星而舍或曰星行者言戴星而行夜不遑息也}
到雒陽著故赤幘託為夏門亭吏|{
	吏著赤幘著則畧翻}
守護尸喪積十二日都官從事執之以聞|{
	都官從事司隸校尉之屬官也掌舉中都官非法者}
太后赦之匡因詣闕上書并乞李杜二公骸骨使得歸葬太后許之匡送喬喪還家|{
	喬家河内}
葬訖行服遂與郭亮董班皆隱匿終身不仕梁冀出吳祐為河間相祐自免歸卒於家|{
	卒子恤翻}
冀以劉鮪之亂思朱穆之言於是請种暠為從事中郎薦欒巴為議郎舉穆高第為侍御史|{
	穆於大將軍府掾為高第也}
是歲南單于兜樓儲死伊陵尸逐就單于車兒立|{
	車音尺遮翻}
二年春正月甲子帝加元服庚午赦天下 三月戊辰帝從皇太后幸大將軍冀府 白馬羌寇廣漢屬國|{
	安帝以蜀郡北部都尉為廣漢屬國都尉}
殺長吏益州刺史率板楯蠻討破之|{
	楯食尹翻}
夏四月丙子封帝弟顧為平原王奉孝崇皇祀尊孝崇皇夫人為孝崇園貴人 五月癸丑北宫掖庭中德陽殿及左掖門火車駕移幸南宫 六月改清河為甘陵|{
	以孝德皇陵為國名}
立安平孝王得子經侯理為甘陵王|{
	經縣屬安平國賢曰今貝州經城縣}
奉孝德皇祀 秋七月京師大水三年夏四月丁卯晦日有食之 秋八月乙丑有星孛于天市|{
	前書天文志旗星中四星曰天市又晉書天文志天市垣二十二星在房心東彗星除之為徙市易都孛蒲内翻}
京師大水 九月己卯地震庚寅地又震郡國五山崩 冬十月太尉趙戒免以司徒袁湯為

太尉大司農河内張歆為司徒 是歲前朗陵侯相荀淑卒|{
	朗陵侯國屬汝南郡}
淑少博學有高行|{
	少詩照翻行下孟翻}
當世名賢李固李膺皆師宗之在朗陵涖事明治|{
	治直吏翻}
稱為神君有子八人儉緄靖燾汪爽肅專|{
	賢曰緄音昆燾音導汪烏光翻專本或作尃音敷}
並有名稱時人謂之八龍|{
	稱尺證翻}
所居里舊名西豪潁隂令渤海苑康以為昔高陽氏有才子八人更名其里曰高陽里|{
	杜佑曰潁川郡城西南有荀淑故宅相傳云即西豪里更工衡翻潁隂縣屬汝南郡淑縣人也姓譜商武丁子子文受封於苑因以為氏左傳冇齊大夫苑何忌趙明誠金石録有漢荆州從事苑鎮碑曰其先苑柏何為晉樂正世掌朝禮又有苑子園寔能掌隂陽之理皆其胄也按姓氏志皆以為出於齊大夫苑何忌之後今此碑所謂苑柏何與子園左傳國語皆無其人故録之以待知者左傳曰昔高陽氏有才子八人蒼舒隤敳檮戭大臨厖降庭堅仲容叔達隤徒回翻敳五才翻一音五回翻韋昭音瑰檮直由翻韋昭音桃戭以善翻韋昭以震翻厖莫江翻降戶江翻}
膺性簡亢|{
	亢口浪翻高也}
無所交接唯以淑為師以同郡陳寔為友荀爽嘗就謁膺因為其御既還喜曰今日乃得御李君矣其見慕如此陳寔出於單微|{
	單獨也孤也薄也}
為郡西門亭長同郡鍾皓以篤行稱|{
	行下孟翻}
前後九辟公府年輩遠在寔前引與為友皓為郡功曹辟司徒府臨辭太守問誰可代卿者皓曰明府欲必得其人西門亭長陳寔可寔聞之曰鍾君似不察人不知何獨識我太守遂以寔為功曹時中常侍侯覽託太守高倫用吏倫教署為文學掾|{
	郡守所出命曰教百官志注郡有文學守助掾六十人掾俞絹翻}
寔知非其人懷檄請見|{
	賢曰檄板書以高倫之教書之於檄而懷之者懼洩事也}
言曰此人不宜用而侯常侍不可違寔乞從外署|{
	功曹主選署寔乞從外自署用若不出於倫者賢曰不欲陷倫於請託也}
不足以塵明德倫從之於是鄉論怪其非舉寔終無所言倫後被徵為尚書郡中士大夫送至綸氏|{
	賢曰綸氏縣屬潁川郡今嵩陽縣是}
倫謂衆人曰吾前為侯常侍用吏|{
	為于偽翻}
陳君密持教還而於外白署比聞議者以此少之|{
	比毗至翻少詩沼翻}
此咎由故人畏憚彊禦|{
	故人倫自謂也漢人於門生故吏之前率自稱故人楊震謂王密曰故人知君君不知故人是也詩曰不畏彊禦}
陳君可謂善則稱君過則稱己者也|{
	禮記坊記曰善則稱君過則稱己則民作忠坊音防}
寔固自引愆聞者方歎息由是天下服其德後為太丘長|{
	賢曰太丘縣屬沛國故城在今亳州永城縣西北}
修德清靜百姓以安鄰縣民歸附者寔輒訓導譬解發遣各令還本司官行部|{
	賢曰司官謂主司之官也行下孟翻}
吏慮民有訟者白欲禁之寔曰訟以求直禁之理將何申其勿有所拘司官聞而歎息曰陳君所言若是豈有寃於人乎亦竟無訟者以沛相賦斂違法解印綬去|{
	相息亮翻斂力贍翻}
吏民追思之鍾皓素與荀淑齊名李膺常歎曰荀君清識難尚鍾君至德可師皓兄子瑾母膺之姑也瑾好學慕古有退讓風|{
	好呼到翻}
與膺同年俱有聲名膺祖太尉修常言瑾似我家性|{
	瑾李氏之出而退讓故修云然}
邦有道不廢邦無道免於刑戮|{
	論語孔子以此言與南容}
復以膺妹妻之|{
	妻七細翻}
膺謂瑾曰孟子以為人無是非之心非人也弟於是何太無皁白邪|{
	皁白易分無皁白言無分别也}
瑾嘗以膺言白皓皓曰元禮祖父在位|{
	李膺字元禮膺祖修為太尉父益為趙相}
諸宗並盛故得然乎昔國子好招人過以致怨惡|{
	國語齊國佐見單襄公其語盡單子曰立於淫亂之國而好盡言以招人過怨之本也其後齊殺國武子招音翹}
今豈其時邪必欲保身全家爾道為貴

和平元年春正月甲子赦天下改元 乙丑太后詔歸政於帝始罷稱制二月甲寅太后梁氏崩 三月車駕徙幸北宫 甲午葬順烈皇后增封大將軍冀萬戶并前合三萬戶封冀妻孫壽為襄城君兼食陽翟租|{
	襄城陽翟二縣皆屬潁川郡}
歲入五千萬加賜赤紱比長公主|{
	漢制公主儀服同公侯紫紱長公主儀服同諸王赤紱四采赤黄縹紺長二丈一尺三百首紱音弗長知兩翻}
壽善為妖態以蠱惑冀冀甚寵憚之|{
	壽作愁眉啼粧墮馬髻折腰步齲齒笑妖於驕翻}
冀愛監奴秦宫官至太倉令|{
	太倉令秩六百石主受郡國傳漕穀屬大司農}
得出入壽所威權大震刺史二千石皆謁辭之冀與壽對街為宅殫極土木互相誇競金玉珍怪充積藏室|{
	藏徂浪翻下守藏同}
又廣開園圃採土築山十里九阪深林絶澗有若自然|{
	冀傳云築山以象二崤十里九阪阪音反}
奇禽馴獸飛走其間冀壽共乘輦車遊觀第内|{
	晉志曰羊車一名輦車毛晃曰輦步挽車也漢書注駕人以行曰輦}
多從倡伎|{
	倡音昌伎渠綺翻}
酣謳竟路或連日繼夜以騁娛恣客到門不得通皆請謝門者門者累千金又多拓林苑周遍近縣起兔苑於河南城西經亘數十里移檄所在調發生兔刻其毛以為識|{
	調徒弔翻識職吏翻}
人有犯者罪至死刑嘗有西域賈胡|{
	賈音古}
不知禁忌誤殺一兔轉相告言坐死者十餘人又起别第於城西以納姦亡|{
	謂姦民及亡命者}
或取良人悉為奴婢至數千口名曰自賣人冀用壽言多斥奪諸梁在位者外以示謙讓而實崇孫氏孫氏宗親冒名為侍中卿校郡守長吏者十餘人皆貪饕凶淫|{
	校戶教翻饕士刀翻}
各使私客籍屬縣富人|{
	賢曰籍謂疏録之也}
被以他罪|{
	被皮義翻}
閉獄掠拷|{
	掠音亮拷音考}
使出錢自贖貲物少者至於死使扶風人士孫奮居富而性吝|{
	士孫姓也奮名也}
冀以馬乘遺之|{
	乘繩證翻遺于季翻}
從貸錢五千萬奮以三千萬與之冀大怒乃告郡縣認奮母為其守藏婢|{
	藏徂浪翻}
云盜白珠十斛紫金千斤以叛|{
	紫金紫磨金也亦謂之鏐}
遂收考奮兄弟死於獄中悉沒其貲財億七千餘萬|{
	摯虞三輔决録曰士孫奮家貲一億七千餘萬余按此以萬萬為億也}
冀又遣客周流四方遠至塞外廣求異物而使人復乘埶横暴妻略婦女毆擊吏卒|{
	使疏吏翻復扶又翻妻者私他人之婦女若己妻然不以道妻之曰略横戶孟翻毆烏口翻}
所在怨毒|{
	毒痛也}
侍御史朱穆自以冀故吏奏記諫曰明將軍地有申伯之尊|{
	賢曰申國之伯周宣王之元舅}
位為羣公之首|{
	賢曰冀絶席于三公}
一日行善天下歸仁終朝為惡四海傾覆頃者官民俱匱加以水蟲為害|{
	賢曰水災及蝗蟲也}
京師諸官費用增多詔書發調或至十倍|{
	調徒弔翻}
各言官無見財|{
	見賢遍翻}
皆當出民搒掠割剝彊令充足|{
	搒音彭掠音亮彊其兩翻}
公賦既重私斂又深|{
	斂力贍翻}
牧守長吏多非德選貪聚無猒|{
	猒於鹽翻}
遇民如虜或絶命於箠楚之下或自賊於迫切之求|{
	賢曰賊殺也箠止橤翻}
又掠奪百姓皆託之尊府|{
	尊府指大將軍府}
遂令將軍結怨天下吏民酸毒道路歎嗟昔永和之末綱紀少弛頗失人望四五歲耳而財空戶散下有離心馬勉之徒乘敝而起荆揚之間幾成大患|{
	事見上卷幾居希翻}
幸賴順烈皇后初政清靜内外同力僅乃討定今百姓戚戚困於永和内非仁愛之心可得容忍外非守國之計所宜久安也夫將相大臣均體元首共輿而馳同舟而濟輿傾舟覆患實共之豈可以去明即昧|{
	賢曰即就也}
履危自安主孤時困而莫之卹乎宜時易宰守非其人者減省第宅園池之費拒絶郡國諸所奉送内以自明外解人惑使挾姦之吏無所依託司察之臣得盡耳目憲度既張遠邇清壹則將軍身尊事顯德燿無窮矣冀不納冀雖專朝縱横|{
	朝直遙翻横戶孟翻}
而猶交結左右宦官任其子弟賓客為州郡要職欲以自固恩寵穆又奏記極諫冀終不悟報書云如此僕亦無一可邪然素重穆亦不甚罪也冀遣書詣樂安太守陳蕃|{
	樂安郡本千乘郡和帝永元七年改為樂安國屬青州}
有所請託不得通使者詐稱他客求謁蕃蕃怒笞殺之坐左轉修武令|{
	修武縣屬河内郡}
時皇子有疾下郡縣市珍藥|{
	下遐稼翻}
而冀遣客齎書詣京兆并貨牛黄|{
	吳晉本草曰牛黄牛出入呻者有之夜有光走角中牛死入膽中如雞子黄神農本草曰療驚癎除邪逐鬼陶弘景曰舊云神牛出入鳴吼者有之伺其出角上以盆水盛而吐之即墮落水中今人多就膽中得之藥中之貴莫復過此本草圖經曰伺其吐出乃喝迫即落水中既得之隂乾百日一云子如雞子黄其重叠可掲輕虚而氛香為佳又云此有四種喝迫而得者名生黄其殺死而在角中得者名角中黄心中剥得者名心黄肝膽中得之者名肝黄大抵不及喝迫得者最勝}
京兆尹南陽延篤發書收客曰大將軍椒房外家而皇子有疾必應陳進醫方豈當使客千里求利乎遂殺之冀慙而不得言有司承旨求其事篤以病免 夏五月庚辰尊博園匽貴人曰孝崇后宫曰永樂|{
	續漢志曰德陽前殿西北入門内有永樂宫樂音洛下長樂同}
置太僕少府以下皆如長樂宫故事分鉅鹿九縣為后湯沐邑 秋七月梓潼山崩|{
	梓潼縣屬廣漢郡賢曰今始州縣也有梓潼水}


元嘉元年春正月朔羣臣朝會大將軍冀帶劍入省|{
	省即禁中也}
尚書蜀郡張陵呵叱令出敇虎賁羽林奪劍冀跪謝陵不應即劾奏冀請廷尉論罪|{
	劾戶槩翻又戶得翻}
有詔以一歲俸贖百僚肅然河南尹不疑嘗舉陵孝廉乃謂陵曰昔舉君適所以自罰也陵曰明府不以陵不肖誤見擢序今申公憲以報私恩不疑有愧色 癸酉赦天下改元 梁不疑好經書喜待士|{
	好呼到翻喜許記翻}
梁冀疾之轉不疑為光禄勲以其子為河南尹年十六容貌甚陋不勝冠帶|{
	勝音升}
道路見者莫不蚩笑不疑自恥兄弟有隙遂讓位歸第與弟蒙閉門自守冀不欲令與賓客交通隂使人變服至門記往來者南郡太守馬融江夏太守田明初除過謁不疑|{
	言過其門因而謁之禮不專也夏戶雅翻}
冀諷有司奏融在郡貪濁及以他事陷明皆髠笞徙朔方融自刺不殊|{
	刺七亦翻}
明遂死於路 夏四月己丑上微行幸河南尹梁府舍 |{
	考異曰袁紀作梁不疑府今從范書}
是日大風拔樹晝昏尚書楊秉上疏曰臣聞天不言語以災異譴告王者至尊出入有常警蹕而行靜室而止|{
	賢曰蹕止行人也靜室謂先使清宫也前書音義曰漢有靜室令}
自非郊廟之事則鑾旗不駕|{
	漢官儀曰前驅有雲䍐皮軒鑾旗車}
故諸侯入諸臣之家春秋尚列其誡|{
	左傳陳靈公如夏徵舒之家為徵舒所弑齊莊公如崔杼之家亦為杼所弑}
况於以先王法服而私出槃游降亂尊卑等威無序|{
	賢曰等威謂威儀有等差也左氏傳曰貴有常尊賤有等威}
侍衛守空宫璽紱委女妾|{
	璽斯氏翻紱音弗}
設有非常之變任章之謀|{
	宣帝時任宣坐謀反誅宣子章亡在渭城界中夜玄服入廟居廊間執戟立于廟門待上至欲為逆發覺伏誅任音壬}
上負先帝下悔靡及帝不納秉震之子也京師旱任城梁國饑民相食|{
	任音壬}
司徒張歆罷以

光祿勲吳雄為司徒 北匈奴呼衍王寇伊吾敗伊吾司馬毛愷|{
	敗蒲邁翻}
攻伊吾屯城詔敦煌太守馬達將兵救之|{
	敦徒門翻}
至蒲類海呼衍王引去 秋七月武陵蠻反冬十月司空胡廣致仕 十一月辛巳京師地震詔百官舉獨行之士涿郡舉崔寔詣公車稱病不對策退而論世事名曰政論其辭曰凡天下所以不治者常由人主承平日久俗漸敝而不悟政寖衰而不改習亂安危怢不自覩|{
	賢曰怢音他沒翻怢忽忘也}
或荒耽耆欲|{
	耆讀曰嗜}
不恤萬機或耳蔽箴誨厭偽忽真|{
	賢曰厭飫姦偽輕忽至真}
或猶豫岐路莫適所從|{
	爾雅路二達謂之岐郭璞曰岐道旁出也此言人主見道不明於人之邪正事之是非莫知所適從也適丁歷翻}
或見信之佐括囊守祿|{
	賢曰易曰括囊無咎無譽括結也結囊不言持禄而已}
或疎遠之臣言以賤廢是以王綱縱弛於上智士鬱伊於下|{
	賢曰鬱伊不申之貌楚辭曰獨伊鬱而誰語}
悲夫自漢興以來三百五十餘歲矣政令垢翫上下怠懈|{
	懈古隘翻}
百姓囂然咸復思中興之救矣|{
	復扶又翻}
且濟時拯世之術在於補䘺决壞枝拄邪傾|{
	賢曰䘺直莧翻禮記衣裳䘺裂紉箴請補綴余謂綻裂之綻非此義此䘺釋補縫也韓詩云破襖請來䘺是其義也拄陟柱翻}
隨形裁割要措斯世於安寧之域而已故聖人執權遭時定制|{
	賢曰權謂變也遭遇其時而定法制不循於舊也余謂權秤錘也執權者隨物之輕重為權之進退以取平也}
步驟之差各有云設不彊人以不能背急切而慕所聞也|{
	賢曰背當時之急切而慕所閒之事則非濟時之要彊其兩翻背蒲妹翻}
蓋孔子對葉公以來遠哀公以臨人景公以節禮|{
	賢曰韓子曰葉公問政於孔子孔子曰政在悦近而來遠魯哀公問政於孔子孔子曰政在選賢齊景公問政於孔子孔子曰政在節財此云臨人節禮文不同也葉式涉翻}
非其不同所急異務也俗人拘文牽古不達權制奇偉所聞簡忽所見烏可與論國家之大事哉故言事者雖合聖聽輒見掎奪|{
	賢曰掎居蟻翻賈逵注國語曰從後牽曰掎}
何者其頑士闇於時權安習所見不知樂成|{
	樂音洛}
况可慮始苟云率由舊章而已其達者或矜名妒能|{
	妒與妬同}
恥策非已舞筆奮辭以破其義寡不勝衆遂見擯棄雖稷契復存猶將困焉|{
	契息列翻復扶又翻}
斯賢智之論所以常憤鬱而不伸者也凡為天下者自非上德嚴之則治寛之則亂|{
	治直吏翻}
何以明其然也近孝宣皇帝明於君人之道審於為政之理故嚴刑峻法破姦軌之膽|{
	左傳曰亂在外為姦在内為軌}
海内清肅天下密如|{
	賢曰密如也}
筭計見效優於孝文|{
	見賢遍翻}
及元帝即位多行寛政卒以墮損|{
	卒子恤翻墮讀曰隳}
威權始奪遂為漢室基禍之主政道得失於斯可鑒昔孔子作春秋褒齊桓懿晉文歎管仲之功|{
	懿美也}
夫豈不美文武之道哉誠達權救敝之理也聖人能與世推移|{
	楚辭聖人不凝滯於物而能與世推移}
而俗士苦不知變以為結繩之約可復治亂秦之緒干戚之舞足以解平城之圍|{
	上古結繩而治後世聖人易之以書契亂秦之後俗益澆薄非結繩之約所能理也干盾也戚鉞也記曰朱干玉戚冕而舞大武所以象武王之伐功也書禹舞干羽於兩階而有苖格高帝為匈奴圍於平城用陳平秘計得出非舞干戚所能解也治直之翻下治亂同治平亦同}
夫熊經鳥伸雖延歷之術非傷寒之理呼吸吐納雖度紀之道非續骨之膏|{
	賢曰莊子曰吹呴呼吸吐故納新熊經鳥伸此道引之士養形之人也黄帝素問曰人傷於寒而轉為熱何也夫寒盛則生熱也度紀猶延年也言鳥伸不能療傷寒吸氣不能續斷骨也成公英莊子疏曰如熊縣木而自經鳥飛空而伸足爾雅翼曰熊類大豕人足黑色好緣高木見人自投而下亦以革厚而筋駑用此自快故稱熊經}
蓋為國之法有似理身平則致養疾則攻焉夫刑罰者治亂之藥石也德教者興平之粱肉也夫以德教除殘是以粱肉養疾也以刑罰治平是以藥石供養也|{
	供音恭養余兩翻}
方今承百王之敝值戹運之會自數世以來政多恩貸馭委其轡馬駘其銜|{
	說文曰駘馬鈍也音達來翻毛晃曰駘脱也}
四牡横犇皇路險傾|{
	賢曰皇路天路也}
方將柑勒鞬輈以救之豈暇鳴和鑾調節奏哉|{
	賢曰何休注公羊傳曰柑以木銜其口也柑音巨炎翻勒馬轡車轅鞬猶束也說苑曰鑾設於鑣和設於軾馬動鑾鳴鑾鳴則和應也}
昔文帝雖除肉刑當斬右趾者棄市笞者往往至死|{
	見十五卷文帝十三年景帝元年}
是文帝以嚴致平非以寛致平也寔瑗之子也|{
	崔瑗見五十一卷安帝延光四年瑗于眷翻}
山陽仲長統嘗見其書歎曰凡為人主宜寫一通置之坐側|{
	坐才卧翻}


臣光曰漢家之法已嚴矣而崔寔猶病其寛何哉蓋衰世之君率多柔懦凡愚之佐唯知姑息|{
	姑且也息安也且苟目前之安也}
是以權幸之臣有罪不坐豪猾之民犯法不誅仁恩所施止於目前姦宄得志紀綱不立故崔寔之論以矯一時之枉非百世之通義也孔子曰政寛則民慢慢則糾之以猛猛則民殘殘則施之以寛寛以濟猛猛以濟寛政是以和|{
	左傳載孔子善子太叔之辭杜預曰糾攝也}
斯不易之常道矣

閏月庚午任城節王崇薨無子國絶|{
	章帝元和元年分東平國為任城國以封東平王蒼之少子尚崇尚之姪也諡法好廉自克曰節}
以太常黄瓊為司空帝欲褒崇梁冀使中朝二千石以上會議其禮|{
	西都中世}


|{
	以後以三公九卿為外朝官東都無中外朝之别也此中朝直謂朝廷朝直遙翻}
特進胡廣太常羊溥司隸校尉祝恬太中大夫邊韶等咸稱冀之勲德宜比周公錫之山川土田附庸|{
	此西都諸臣所以尊王莽者今廣復欲以崇冀微黄瓊之言殆哉}
黄瓊獨曰冀前以親迎之勞增邑萬三千戶又其子亦加封賞今諸侯以戶邑為制不以里數為限冀可比鄧禹合食四縣朝廷從之於是有司奏冀入朝不趨劒履上殿謁讚不名禮儀比蕭何|{
	蕭何唯劒履上殿入朝不趨何嘗謁贊不名也君前臣名禮也冀何如人而寵秩之至此乎讚與擯贊之贊同}
悉以定陶陽成餘戶增封為四縣比鄧禹|{
	賢曰冀初封襄邑縣襲封乘氏更增以定陶陽城是為四縣余謂陽城當作成陽與定陶乘氏皆屬濟隂郡}
賞賜金錢奴婢綵帛車馬衣服甲第比霍光以殊元勲每朝會與三公絶席|{
	賢曰絶席别也}
十日一入平尚書事宣布天下為萬世法冀猶以所奏禮薄意不悦

二年春正月西域長史王敬為于窴所殺初西域長史趙評在于窴病癰死|{
	按西域傳評元嘉元年死窴徒賢翻}
評子迎喪道經拘彌拘彌王成國與于窴王建素有隙謂評子曰于窴王令胡醫持毒藥著創中|{
	著陟畧翻創初良翻}
故致死耳評子信之還以告敦煌太守馬達|{
	敦徒門翻 考異曰車師傳作司馬達今從于窴傳}
會敬代為長史馬達令敬隱覈于窴事|{
	隱度也覈考也實也}
敬先過拘彌成國復說云|{
	復扶又翻說輸芮翻}
于窴國人欲以我為王今可因此罪誅建|{
	謂以評死為建罪也}
于窴必服矣敬貪立功名前到于窴設供具請建而隂圖之|{
	供具宴饗之具也}
或以敬謀告建建不信曰我無罪王長史何為欲殺我旦日建從官屬數十人詣敬坐定建起行酒敬叱左右執之吏士並無殺建意官屬悉得突走時成國主簿秦牧隨敬在會持刀出曰大事已定何為復疑即前斬建于窴侯將輸僰等遂會兵攻敬|{
	按前書西域諸國各置輔國侯左右將復扶又翻僰蒲北翻}
敬持建頭上樓宣告曰天子使我誅建耳輸僰不聽上樓斬敬縣首於市|{
	縣讀曰懸}
輸僰自立為王國人殺之而立建子安國馬達聞王敬死欲將諸郡兵出塞擊于窴帝不聽徵達還而以宋亮代為敦煌太守亮到開募于窴令自斬輸僰|{
	開于窴國人自新之路仍募使斬輸僰也僰蒲北翻}
時輸僰死已經月乃斷死人頭送敦煌而不言其狀|{
	斷丁管翻}
亮後知其詐而竟不能討也|{
	史言漢之威令不復行於西域}
丙辰京師地震 夏四月甲辰孝崇皇后匽氏崩以帝弟平原王石為喪主斂送制度比恭懷皇后|{
	恭懷皇后和帝母梁氏斂力贍翻}
五月辛卯葬于博陵 秋七月庚辰日有食之 冬十月乙亥京師地震十一月司空黄瓊免十二月以特進趙戒為司空

永興元年春三月丁亥帝幸鴻池|{
	百官志注鴻池在雒陽東二十里水經注穀水東注鴻池陂池東西千步南北千一百步}
夏四月丙申赦天下改元丁酉濟南悼王廣薨無子國除|{
	廣濟南王顯之子也紹封見五十一卷順帝永建元年濟子禮翻}
秋七月郡國三十二蝗河水溢百姓饑窮流冗者數十萬戶|{
	冗散也而隴翻}
冀州尤甚詔以侍御史朱穆為冀州刺史冀部令長聞穆濟河解印綬去者四十餘人及到奏劾諸郡貪汙者|{
	劾戶槩翻又戶得翻}
有至自殺或死獄中宦者趙忠喪父歸葬安平|{
	安平國屬冀州喪息浪翻}
僭為玉匣穆下郡案驗|{
	下遐稼翻}
吏畏其嚴遂發墓剖棺陳尸出之帝聞大怒徵穆詣廷尉輸作左校|{
	不以趙忠玉匣為僭而以朱穆發墓為罪昏暗之君豈有真是非哉賢曰左校署名屬將作掌左工徒校戶教翻}
太學書生潁川劉陶等數千人詣闕上書訟穆曰伏見弛刑徒朱穆處公憂國|{
	處昌呂翻}
拜州之日志清姦惡誠以常侍貴寵父子兄弟布在州郡競為虎狼噬食小民故穆張理天綱補綴漏目羅取殘禍以塞天意|{
	塞悉則翻}
由是内官咸共恚疾|{
	内官即中官恚於避翻}
謗讟煩興讒隙仍作極其刑讁輸作左校天下有識皆以穆同勤禹稷而被共鯀之戾|{
	共音恭}
若死者有知則唐帝怒於崇山重華忿於蒼墓矣|{
	賢曰尚書放驩兜于崇山孔安國注曰崇山南裔也山海經曰有驩頭之國帝堯葬焉郭璞注曰驩頭驩兜也禮記曰舜葬蒼梧之野}
當今中官近習竊持國柄手握王爵口銜天憲|{
	天憲王法也謂刑戮出于其口也}
運賞則使餓隸富於季孫|{
	賢曰運行也論語曰季氏富于周公}
呼噏則令伊顏化為桀跖|{
	噏與吸同}
而穆獨亢然不顧身害|{
	亢音抗}
非惡榮而好辱惡生而好死也|{
	惡烏路翻好呼到翻}
徒感王綱之不攝|{
	賢曰攝接也余謂攝飭整也}
懼天網之久失故竭心懷憂為上深計臣願黥首繫趾|{
	賢曰黥首謂鑿額湼墨也繫趾謂釱其足也以鐵著足曰釱}
代穆輸作帝覽其奏乃赦之 冬十月太尉袁湯免以太常胡廣為太尉司徒吳雄司空趙戒免以太僕黄瓊為司徒光祿勲房植為司空 武陵蠻詹山等反武陵太守汝南應奉招降之 車師後部王阿羅多與戊部候嚴皓不相得|{
	戊巳兩部各置校尉各有部候西域傳曰和帝置戊部侯居車師後部侯城}
忿戾而反攻圍屯田殺傷吏士後部侯炭遮領餘民畔阿羅多詣漢吏降|{
	前書車師後國有擊胡侯漢賜印綬降戶江翻下同}
阿羅多迫急從百餘騎亡入北匈奴敦煌太守宋亮上立後故王軍就質子卑君為王|{
	上時掌翻上奏而立之安帝延光四年班勇斬後部王軍就其質子在敦煌質音致}
後阿羅多復從匈奴中還與卑君爭國|{
	復扶又翻}
頗收其國人戊校尉嚴詳慮其招引北虜將亂西域乃開信告示|{
	開信者開以丹青之信}
許復為王阿羅多乃詣詳降|{
	降戶江翻}
於是更立阿羅多為王將卑君還敦煌以後部人三百帳與之|{
	西域傳曰帳者猶中國之戶數也將如字}


二年春正月甲午赦天下 二月辛丑復聽刺史二千石行三年喪|{
	安帝建光元年斷行三年喪事見四十九卷}
癸卯京師地震夏蝗 東海朐山崩|{
	賢曰朐山在今海州朐山縣南}
乙卯封乳母

馬惠子初為列侯 秋九月丁卯朔日有食之 太尉胡廣免以司徒黄瓊為太尉閏月以光禄勲尹頌為司徒 冬十一月甲辰帝校獵上林苑遂至函谷關|{
	校戶教翻闌校也所以遮獸而獵取之謂之校獵東漢開上林苑於雒陽西函谷關在河南穀城縣}
泰山琅邪賊公孫舉東郭竇等反殺長吏

永壽元年春正月戊申赦天下改元 二月司隸冀州饑人相食 太學生劉陶上疏陳事曰夫天之與帝帝之與民猶頭之與足相須而行也陛下目不視鳴條之事耳不聞檀車之聲|{
	賢曰鳴條地名在安邑之西湯與桀戰于鳴條之野檀車兵車也詩曰檀車嘽嘽余按大雅大明之詩曰牧野洋洋檀車煌煌維師尚父時維鷹揚凉彼武王肆伐大商陶蓋用此檀車事言桀紂貴為天子得罪於天流毒於民而湯武伐之亡國之事不接于帝之耳目帝不知以為戒也毛氏詩傳曰檀彊靭之木陸璣疏檀木皮正青滑澤與檕迷相似又似駁馬駁馬梓檎故里語斫檀不諦得檕迷檕迷尚可得駮馬檕迷一名挈橀故齊人諺曰上山伐檀挈橀先殫蓋檀木彊靭可為兵車嘽吐丹翻凉力尚翻}
天災不有痛於肌膚震食不即損於聖體|{
	震食謂地震日食也}
故蔑三光之謬輕上天之怒伏念高祖之起始自布衣合散扶傷克成帝業勤亦至矣流福遺祚至於陛下陛下既不能增明烈考之軌而忽高祖之勤妄假利器委授國柄使羣醜刑隸芟刈小民|{
	芟所銜翻}
虎豹窟於麑場|{
	賢曰鹿子曰麑音研奚翻}
豺狼乳於春囿|{
	乳人喻翻產也}
貨殖者為窮寃之䰟貧餒者作饑寒之鬼|{
	言無貧富皆不得其死}
死者悲於窀穸|{
	杜預曰窀厚也穸夜也厚夜猶長夜也窀株倫翻}
生者戚於朝野是愚臣所為咨嗟長懷歎息者也|{
	朝直遙翻為于偽翻}
且秦之將亡正諫者誅諛進者賞嘉言結於忠舌國命出於讒口擅閻樂於咸陽授趙高以車府|{
	閻樂為咸陽令趙高為中車府令}
權去已而不知威離身而不顧|{
	離力智翻}
古今一揆成敗同埶願陛下遠覽彊秦之傾近察哀平之變得失昭然禍福可見臣又聞危非仁不扶亂非智不救竊見故冀州刺史南陽朱穆前烏桓校尉臣同郡李膺皆履正清平貞高絶俗斯實中興之良佐國家之柱臣也宜還本朝夾輔王室|{
	前年朱穆得罪李膺時亦免居綸氏}
臣敢吐不時之義於諱言之朝|{
	賢曰不時謂不合於時也}
猶冰霜見日必至消滅臣始悲天下之可悲今天下亦悲臣之愚惑也書奏不省|{
	省悉景翻}
夏南陽大水 司空房植免以太常韓縯為司空|{
	縯以淺翻}
巴郡益州郡山崩 秋南匈奴左薁鞬臺耆且渠伯德等反|{
	薁於六翻鞬居言翻且子余翻 考異曰帝紀作左臺且渠伯德等叛今從張奐傳}
寇美稷東羌復舉種應之|{
	復扶又翻種章勇翻}
安定屬國都尉敦煌張奐初到職|{
	賢曰屬國都尉其秩比二千石水經注安定屬國都尉治三水縣}
壁中唯有二百許人聞之即勒兵而出軍吏以為力不敵叩頭爭止之奐不聽遂進屯長城|{
	此即秦蒙恬所築長城在上郡界}
收集兵士遣將王衛招誘東羌因據龜兹縣|{
	前書上郡龜兹縣上郡屬國都尉治所師古曰龜兹國人來降附者處之於此故以名云}
使南匈奴不得交通東羌諸豪遂相率與奐共擊薁鞬等破之伯德惶恐將其衆降郡界以寧羌豪遺奐馬二十匹金鐻八枚|{
	遺于季翻賢曰郭璞注山海經云鐻音渠金食器名未詳形制也韻書曰鐻戎夷貫耳}
奐於諸羌前以酒酹地|{
	賢曰以酒沃地謂之酹音力外翻余謂蓋自誓也}
曰使馬如羊不以入厩使金如粟不以入懷悉以還之前此八都尉率好財貨|{
	好呼到翻}
為羌所患苦及奐正身潔已無不悦服威化大行

二年春三月蜀郡屬國夷反|{
	延光元年以蜀郡西部都尉為屬國都尉}
初鮮卑檀石槐勇健有智略部落畏服乃施法禁平曲直無敢犯者遂推以為大人檀石槐立庭于彈汙山歠仇水上|{
	汙范書作汗歠音昌悦翻}
去高柳北三百餘里兵馬甚盛東西部大人皆歸焉因南抄緣邊北拒丁零東卻夫餘|{
	抄楚交翻夫音扶}
西擊烏孫盡據匈奴故地東西萬四千餘里秋七月檀石槐寇雲中以故烏桓校尉李膺為度遼將軍膺到邊羌胡皆望風畏服先所掠男女悉詣塞下送還之|{
	考異曰袁紀延熹二年鮮卑寇遼東度遼將軍李膺擊破之今從范書}
公孫舉東郭竇

等聚衆至三萬人寇青兖徐三州破壞郡縣|{
	壞音怪}
連年討之不能克尚書選能治劇者以司徒掾潁川韓韶為嬴長|{
	嬴縣屬泰山郡賢曰故城在今兖州博城縣東北治直之翻掾俞絹翻長知兩翻}
賊聞其賢相戒不入嬴境餘縣流民萬餘戶入縣界韶開倉賑之主者爭謂不可|{
	主者主倉粟之吏也}
韶曰長活溝壑之人而以此伏罪含笑入地矣太守素知韶名德竟無所坐韶與同郡荀淑鍾皓陳寔皆嘗為縣長所至以德政稱時人謂之潁川四長|{
	賢曰謂荀淑為當塗長韓韶為嬴長陳寔為太丘長鍾皓為林慮長也長知兩翻}
初鮮卑寇遼東屬國都尉段熲率所領馳赴之|{
	熲古迥翻}
既而恐賊驚去乃使驛騎詐齎璽書召熲熲於道偽退潛於還路設伏虜以為信然乃入追熲熲因大縱兵悉斬獲之坐詐為璽書當伏重刑以有功論司寇刑竟拜議郎|{
	司寇二歲刑璽斯氏翻}
至是詔以東方盜賊昌熾|{
	熾尺志翻}
令公卿選將帥有文武材者司徒尹頌薦熲|{
	段熲傳作訟帝紀作頌作頌為是}
拜中郎將擊舉竇等大破斬之獲首萬餘級餘黨降散|{
	降戶江翻}
封熲為列侯 冬十二月地震 封梁不疑子馬為潁隂侯梁子桃為城父侯|{
	城父縣屬汝南郡考異曰袁紀馬桃封在建和元年馬作焉桃作祧今從范書}


資治通鑑卷五十三
