資治通鑑卷六十六   宋 司馬光 撰

胡三省 音註

漢紀五十八|{
	起屠維赤奮若盡昭陽大荒落凡五年}


孝獻皇帝辛

建安十四年春三月曹操軍至譙|{
	自赤壁還也}
孫權圍合肥久不下權率輕騎欲身往突敵長史張紘諫曰夫兵者凶器戰者危事也|{
	兵凶器戰危事前書鼂錯之言}
今麾下恃盛壯之氣忽彊暴之虜|{
	以權在軍中故稱麾下}
三軍之衆莫不寒心雖斬將搴旗威震敵場此乃偏將之任非主將之宜也|{
	將即亮翻}
願抑賁育之勇|{
	賁音奔}
懷霸王之計權乃止曹操遣將軍張喜將兵解圍久而未至揚州别駕楚國蔣濟密白刺史偽得喜書云步騎四萬已到雩婁|{
	雩婁縣屬廬江郡師古曰雩音許于翻婁音力于翻晉地道記雩婁在安豐縣西南}
遣主簿迎喜三部使齎書語城中守將|{
	語牛倨翻}
一部得入城二部為權兵所得權信之遽燒圍走 |{
	考異曰魏志武紀十二月權圍合肥劉馥傳云攻圍百餘日孫權傳云踰月不能下由此言之權退必在今年明矣}
秋七月曹操引水軍自渦入淮|{
	班志淮陽扶溝縣渦水首受狼湯渠東至向入淮過郡三行千里水經註曰至下邳睢陵縣入淮師古曰渦音戈又音瓜狼音浪湯音徒浪翻}
出肥水軍合肥開芍陂屯田|{
	水經注肥水出九江成德縣廣陽鄉西西北入芍陂陂周一百二十許里在夀春縣南八十里楚相孫叔敖所造也自芍陂上施水則至合肥肥水又北過夀春縣北入于淮師古曰芍音鵲}
冬十月荆州地震 十二月操軍還譙 廬江人陳蘭梅成據灊六叛|{
	灊六二縣皆屬廬江郡賢曰灊今夀州霍山縣灊音潛}
操遣盪寇將軍張遼討斬之|{
	盪徒朗翻考異曰遼傳無年按繁欽征天山賦云建安十四年十二月甲辰丞相武平侯曹公東征臨川未濟羣舒蠢動割有灊六乃俾上將盪寇將軍張遼治兵南岳之陽又云陟天柱而南徂故置於此}
因使遼與樂進李典等將七千餘人屯合肥 周瑜攻曹仁歲餘所殺傷甚衆仁委城走權以瑜領南郡太守屯據江陵|{
	守式又翻}
程普領江夏太守治沙羨|{
	夏戶雅翻羨音夷}
呂範領彭澤太守|{
	範傳云範領彭澤太守以彭澤柴桑歷陽為奉邑}
呂蒙領尋陽令劉備表權行車騎將軍領徐州牧會劉琦卒權以備領荆州牧周瑜分南岸地以給備|{
	荆江之南岸則零陵桂陽武陵長沙四郡地也}
備立營於油口改名公安|{
	水經南平郡孱陵縣有油水西北注于江曰油口即劉備立營之處也}
權以妹妻備|{
	妻七細翻}
妹才捷剛猛有諸兄風侍婢百餘人皆執刀侍立傋每入心常凜凜|{
	恐為所圖也}
曹操密遣九江蔣幹往說周瑜|{
	說輸芮翻下同}
幹以才辨獨步于江淮之間|{
	言江淮人士無能敵其才辨者}
乃布衣葛巾自託私行詣瑜瑜出迎之立謂幹曰子翼良苦遠涉江湖|{
	蔣幹字子翼}
為曹氏作說客邪|{
	為于偽翻}
因延幹與周觀營中行視倉庫軍資器仗訖還飲宴示之侍者服飾珍玩之物因謂幹曰丈夫處世|{
	處昌呂翻下同}
遇知己之主外託君臣之義内結骨肉之恩言行計從禍福共之假使蘇張更生|{
	謂蘇秦張儀也}
能移其意乎幹但笑終無所言還白操稱瑜雅量高致非言辭所能間也|{
	間古莧翻}
丞相掾和洽言于曹操曰天下之人材德各殊不可以一節取也儉素過中自以處身則可以此格物所失或多|{
	格正也掾俞絹翻}
今朝廷之議吏有著新衣|{
	著陟略翻}
乘好車者謂之不清形容不飾衣裘敝壞者謂之亷潔至令士大夫故汚辱其衣藏其輿服朝府大吏或自挈壺飱以入官寺|{
	朝直遙翻飱蘇昆翻熟食曰飱}
夫立教觀俗貴處中庸為可繼也|{
	中者正道庸者常道程子曰不偏之謂中不易之謂庸}
今崇一概難堪之行以檢殊塗|{
	檢束也檢柙也概與槩同行下孟翻下同}
勉而為之必有疲瘁|{
	瘁秦醉翻}
古之大教務在通人情而已凡激詭之行則容隱偽矣操善之

十五年春下令曰孟公綽為趙魏老則優不可以為滕薛大夫|{
	論語載孔子之言朱子曰公綽魯大夫趙魏晉卿之家老家臣之長大家埶重而無諸侯之事家老望尊而無官守之責優有餘也滕薛二國名大夫任國政者滕薛國小政繁大夫位高責重然則公綽蓋亷靜寡欲而短于才者}
若必亷士而後可用則齊桓其何以霸世|{
	管仲富擬公室築三歸之臺塞門反玷鏤簋朱紘桓公用之而霸}
二三子其佐我明揚仄陋|{
	書堯典曰明明楊仄陋揚舉也}
唯才是舉吾得而用之 二月乙巳朔日有食之 冬曹操作銅爵臺於鄴|{
	水經注銅爵臺在鄴城西北因城為之高十丈有屋百餘間}
十二月己亥操下令曰孤始舉孝亷|{
	操年二十舉孝亷為郎}
自以本非巖穴知名之士恐為世人之所凡愚|{
	恐時人以凡愚待之也}
欲好作政教以立名譽故在濟南除殘去穢|{
	操為濟南相國有十餘縣長吏多阿附貴戚贓汙狼籍於是奏免其八姦宄逃竄境内肅然濟子禮翻去羌呂翻}
平心選舉以是為彊豪所忿恐致家禍故以病還鄉里時年紀尚少|{
	少詩照翻}
乃於譙東五十里築精舍欲秋夏讀書冬春射獵為二十年規待天下清乃出仕耳然不能得如意徵為典軍校尉|{
	見五十九卷靈帝中平五年}
意遂更欲為國家討賊立功|{
	為于偽翻}
使題墓道言漢故征西將軍曹侯之墓此其志也而遭值董卓之難興舉義兵|{
	見五十九卷初平元年難乃旦翻}
後領兖州破降黄巾三十萬衆|{
	見六十卷初平三年降戶江翻}
又討擊袁術使窮沮而死|{
	見六十三卷建安四年沮在呂翻}
摧破袁紹|{
	見六十三卷建安五年}
梟其二子|{
	斬譚見六十四卷十年斬尚見上卷十二年梟堅堯翻}
復定劉表|{
	見上卷上年復扶又翻}
遂平天下身為宰相人臣之貴已極意望已過矣設使國家無有孤不知當幾人稱帝幾人稱王或者人見孤彊盛又性不信天命恐妄相忖度言有不遜之志|{
	言其將簒也度徒落翻}
每用耿耿|{
	耿古幸翻毛公曰耿耿猶儆儆也又憂也}
故為諸君陳道此言|{
	為于偽翻}
皆肝鬲之要也|{
	鬲胷鬲也}
然欲孤便爾委捐所典兵衆以還執事歸就武平侯國實不可也何者誠恐已離兵為人所禍|{
	離力智翻}
既為子孫計又已敗則國家傾危是以不得慕虛名而處實禍也|{
	處昌呂翻}
然兼封四縣食戶三萬何德堪之江湖未靜|{
	謂孫劉也}
不可讓位至於邑土可得而辭今上還陽夏柘苦三縣戶二萬但食武平萬戶|{
	上時掌翻武平陽夏柘苦四縣皆屬陳國夏音賈}
且以分損謗議少減孤之責也|{
	少詩沼翻下同}
劉表故吏士多歸劉備備以周瑜所給地少不足以容其衆乃自詣京見孫權|{
	京京口城也權時居京故劉備周瑜皆詣京見之後都秣陵於京口置京督又曰徐陵督爾雅絶高曰京其城因山為壘緣江為境因謂之京口}
求都督荆州|{
	荆州八郡瑜既以江南四郡給備備又欲兼得江漢間四郡也}
瑜上疏於權曰劉備以梟雄之姿|{
	梟堅堯翻}
而有關羽張飛熊虎之將必非久屈為人用者愚謂大計宜徙備置吴盛為築宫室多其美女玩好以娱其耳目|{
	好呼到翻}
分此二人各置一方使如瑜者得挾與攻戰大事可定也今猥割土地以資業之|{
	謂資之土地使成霸業}
聚此三人俱在疆場|{
	場音亦}
恐蛟龍得雲雨終非池中物也呂範亦勸留之權以曹操在北方當廣擥英雄|{
	擥魯敢翻手取也}
不從|{
	不從範瑜之言也}
備還公安久乃聞之歎曰天下智謀之士所見略同時孔明諫孤莫行其意亦慮此也孤方危急不得不往此誠險塗殆不免周瑜之手周瑜詣京見權曰今曹操新敗憂在腹心|{
	謂操以赤壁之敗威望頓損中國之人或欲因其敗而圖之是憂在腹心}
未能與將軍連兵相事也|{
	相事謂相與從事於戰攻也}
乞與奮威俱進取蜀而并張魯因留奮威固守其地與馬超結援瑜還與將軍據襄陽以䠞操|{
	䠞子六翻}
北方可圖也權許之奮威者孫堅弟子奮威將軍丹陽太守瑜也周瑜還江陵為行裝於道病困與權牋曰修短命矣誠不足惜但恨微志未展不復奉教命耳|{
	復扶又翻下同}
方今曹操在北疆場未靜劉備寄寓有似養虎|{
	言養虎將自遺患}
天下之事未知終始此朝士旰食之秋|{
	旰古旦翻晚也朝直遙翻}
至尊垂慮之日也魯肅忠烈臨事不苟可以代瑜儻所言可采瑜死不朽矣卒於巴丘|{
	裴松之曰瑜欲取蜀還江陵治嚴所卒之處應在今之巴陵與前所鎮巴江名同處異也據水經註巴丘山在湘水右岸晉武帝太康元年立巴陵縣宋文帝元嘉十六年置巴陵郡今岳州也 考異曰按江表傳瑜與策同年策以建安五年死年二十六瑜死時年三十六故知在今年也}
權聞之哀慟曰公瑾有王佐之資今忽短命孤何賴哉自迎其喪于蕪湖|{
	蕪湖縣屬丹陽郡}
瑜有一女二男權為長子登娶其女|{
	為于偽翻長知兩翻}
以其男循為騎都尉妻以女胤為興業都尉妻以宗女|{
	妻七細翻}
初瑜見友于孫策太夫人又使權以兄奉之是時權位為將軍諸將賓客為禮尚簡而瑜獨先盡敬便執臣節程普頗以年長數陵侮瑜瑜折節下之|{
	長知兩翻數所角翻折而設翻下遐稼翻}
終不與校普後自敬服而親重之乃告人曰與周公瑾交若飲醇醪不覺自醉|{
	酒不澆為醇醪滓汁酒}
權以魯肅為奮武校尉代瑜領兵令程普領南郡太守魯肅勸權以荆州借劉備與共拒曹操權從之|{
	為孫劉爭荆州張本 考異曰肅傳曰曹公聞權以土地業備方作書落筆于地恐操不至于是今不取}
乃分豫章為番陽郡|{
	番蒲何翻}
分長沙為漢昌郡|{
	鄱陽今饒州地沈約志長沙郡有吴昌縣漢末之漢昌也吴更名至隋廢吴昌入羅縣唐武德八年又省羅縣入湘隂則知吴立漢昌郡在唐岳州湘隂縣界}
復以程普領江夏太守|{
	復扶又翻}
魯肅為漢昌太守屯陸口|{
	水經江水左逕烏林南又東右岸得蒲磯口即陸口也水出下雋縣西三山溪入蒲圻縣北逕呂蒙城西孫權征長沙零桂所鎮也}
初權謂呂蒙曰卿今當塗掌事|{
	當塗猶言當路也}
不可不學蒙辭以軍中多務權曰孤豈欲卿治經為博士邪但當涉獵見往事耳|{
	師古曰涉若涉水獵若獵獸言歷覽之不專精也治直之翻}
卿言多務孰若孤孤嘗讀書自以為大有所益蒙乃始就學及魯肅過尋陽與蒙論議大驚曰卿今者才畧非復吳下阿蒙蒙曰士别三日即更刮目相待大兄何見事之晚乎肅遂拜蒙母結友而别劉備以從事龎統守耒陽令|{
	耒陽縣屬桂陽郡宋白曰郡國志云鼇山口即耒陽縣耒盧對翻}
在縣不治免官魯肅遺備書曰龎士元非百里才也使處治中别駕之任始當展其驥足耳|{
	遺于季翻處昌呂翻百官志司隸校尉從事史十二人功曹從事主選署及衆事别駕從事校部行部則奉引録衆事州牧則改功曹從事為治中從事杜佑曰别駕從事史從刺史行部别乘一乘傳車故謂之别駕治中從事史居中治事主衆曹功曹主選用}
諸葛亮亦言之備見統與善譚大器之|{
	善譚者劇論當世事也譚與談同}
遂用統為治中親待亞于諸葛亮與亮並為軍師中郎將 初蒼梧士燮為交阯太守交州刺史朱符為夷賊所殺州郡擾亂燮表其弟壹領合浦太守䵋領九真太守|{
	䵋胡悔翻又于鄙翻}
武領南海太守燮體器寛厚中國士人多往依之雄長一州偏在萬里威尊無上|{
	天下殽亂燮雄據偏州人但知威尊無復知有天子也長知兩翻}
出入儀衛甚盛震服百蠻朝廷遣南陽張津為交州刺史津好鬼神事嘗著絳帕頭|{
	好呼到翻著陟畧翻帕莫白翻項安世家說頭巾一名□音隖一名帕陸游曰袙頭者巾幘之類猶今言幞頭韓文公云以紅袙首為已失之東坡云絳袙蒙頭讀道書增一蒙字其誤尤甚}
鼓琴燒香讀道書云可以助化為其將區景所殺|{
	區烏侯翻姓也又虧于翻據史自賈琮以前皆為交阯刺史未得為交州晉志永和九年交阯太守周敞求立為州朝議不許即拜敞為交阯刺史建安八年張津為刺史士燮為交阯太守共表立為州乃拜津為交州牧十五年移居番禺}
劉表遣零陵賴恭代津為刺史|{
	姓譜賴為楚所滅子孫以國為氏風俗通漢有交阯太守賴先}
是時蒼梧太守史璜死表又遣吳巨代之朝廷賜燮璽書以燮為綏南中郎將董督七郡領交阯太守如故巨與恭相失巨舉兵逐恭恭走還零陵孫權以番陽太守臨淮步騭為交州刺史|{
	姓譜晉有步揚食采于步因氏焉番蒲荷翻騭職日翻}
士燮率兄弟奉承節度吳巨外附内違騭誘而斬之|{
	誘音酉}
威聲大震權加燮左將軍燮遣子入質|{
	質音致}
由是嶺南始服屬於權

十六年春正月以曹操世子丕為五官中郎將置官屬為丞相副|{
	漢五官中郎將主五官郎而已未嘗置官屬也領屬光祿勲未嘗為丞相副也}
三月操遣司隸校尉鍾繇討張魯使征西護軍夏侯淵等將兵出河東與繇會|{
	淵之族操所自出也付以西征先馳之任以資序未得為征西將軍故以護軍為名}
倉曹屬高柔諫曰|{
	公府倉曹主倉穀事有掾有屬}
大兵西出韓遂馬超疑為襲已必相扇動宜先招集三輔三輔苟平漢中可傳檄而定也操不從關中諸將果疑之|{
	操舍關中而遠征張魯伐虢取虞之計也盖欲討超遂而無名先張討魯之埶以速其反然後加兵耳}
馬超韓遂侯選程銀楊秋李堪張横梁興成宜馬玩等十部皆反其衆十萬屯據潼關|{
	潼關在弘農華隂縣水經註曰河在關内南流潼激關山因謂之潼關晉所謂桃林之塞秦所謂楊華是也}
操遣安西將軍曹仁督諸將拒之|{
	晉百官志曰四安起于魏初謂安東安西安南安北四將軍也}
敕令堅壁勿與戰命五官將丕留守鄴以奮武將軍程昱參丕軍事|{
	沈約曰奮武將軍始于漢末}
門下督廣陵徐宣為左護軍|{
	門下督督將之居門下者}
留統諸軍樂安國淵為居府長史統留事|{
	姓譜齊有國氏世為上卿又鄭七穆子國之後為國氏}
秋七月操自將擊超等|{
	將即亮翻下同}
議者多言關西兵習長矛非精選前鋒不可當也操曰戰在我非在賊也賊雖習長矛將使不得以刺諸君但觀之|{
	在我而不在敵故可以制勝此未易與常人言也刺七亦翻下同}
八月操至潼關與超等夾關而軍操急持之而潛遣徐晃朱靈以步騎四千人渡蒲阪津據河西為營|{
	蒲阪津在蒲阪縣西河西即唐之蒲津關 考異曰晃傳曰太祖至潼關恐不得渡召問晃晃曰公盛兵于此而賊不復别守蒲阪知其無謀也今假臣精兵渡蒲阪津為軍先以截其裏賊可禽也太初曰善按武帝紀潛遣二將渡蒲阪皆太祖之謀而晃傳云皆晃之策盖陳氏各欲稱其功美不相顧耳}
閏月操自潼關北渡河兵衆先渡操獨與虎士百餘人留南岸斷後|{
	斷丁管翻}
馬超將步騎萬餘人攻之矢下如雨操猶據胡床不動許褚扶操上船船工中流矢死|{
	中竹仲翻}
禇左手舉馬鞍以蔽操右手刺船校尉丁斐放牛馬以餌賊賊亂取牛馬操乃得渡遂自蒲阪渡西河循河為甬道而南超等退拒渭口|{
	前書渭水至船司空入河後漢省船司空屬華隂縣渭口之東即潼關也}
操乃多設疑兵潛以舟載兵入渭為浮橋夜分兵結營於渭南超等夜攻營伏兵擊破之超等屯渭南遣使求割河以西請和操不許九月操進軍悉渡渭超等數桃戰又不許固請割地求送任子賈詡以為可偽許之操復問計策|{
	數所角翻挑徒了翻復扶又翻}
詡曰離之而已操曰解|{
	解戶買翻曉也}
韓遂請與操相見操與遂有舊於是交馬語移時|{
	遂與樊稠交馬語而得以斃稠與曹操交馬語乃以自斃然後知遂之所以遇稠者非用數也若馬超等之疑遂則猶李傕之疑稠也}
不及軍事但說京都舊故拊手歡笑時秦胡觀者前後重沓|{
	重直龍翻}
操笑謂之曰爾欲觀曹公邪亦猶人也非有四目兩口但多智耳既罷超等問遂公何言遂曰無所言也超等疑之他日操又與遂書多所點竄如遂改定者超等愈疑遂|{
	二者皆所以離之也 考異曰許禇傳曰太祖與韓遂馬超等會語左右皆不得從唯將禇超負其力隂欲前突大祖素聞禇勇疑從騎是禇乃問曰公有虎侯者安在太祖顧指禇禇瞋目眄之超不敢動按時超不與遂同在彼故疑此說妄也}
操乃與克日會戰|{
	克日者尅定其日也}
先以輕兵挑之|{
	挑徒了翻}
戰良久乃縱虎騎夾擊大破之斬成宜李堪等遂超犇凉州楊秋犇安定諸將問操曰初賊守潼關渭北道缺|{
	缺謂缺而不備}
不從河東擊馮翊而反守潼關引日而後北渡何也操曰賊守潼關若吾入河東賊必引守諸津則西河未可渡吾故盛兵向潼關賊悉衆南守西河之備虛故二將得擅取西河然後引軍北渡賊不能與吾爭西河者以二將之軍也|{
	二將徐晃朱靈也將即亮翻}
連車樹柵為甬道而南既為不可勝|{
	兵法先為不可勝以待敵之可勝}
且以示弱渡渭為堅壘虜至不出所以驕之也故賊不為營壘而求割地吾順言許之所以從其意使自安而不為備因畜士卒之力一旦擊之所謂疾雷不及掩耳|{
	淮南子之言}
兵之變化固非一道也始關中諸將每一部到操輒有喜色諸將問其故操曰關中長遠若賊各依險阻征之不一二年不可定也今皆來集其衆雖多莫相歸服軍無敵主|{
	適丁歷翻}
一舉可滅為功差易吾是以喜|{
	當此之時關西之兵最為精彊而破於操者法制不一也易以豉翻}
冬十月操自長安北征楊秋圍安定秋降|{
	降戶江翻}
復其爵位使留撫其民十二月操自安定還留夏侯淵屯長安以議郎張既為京兆尹既招懷流民興復縣邑百姓懷之遂超之叛也弘農馮翊縣邑多應之河東民獨無異心操與超等夾渭為軍軍食一仰河東|{
	仰牛向翻}
及超等破餘畜尚二十餘萬斛|{
	畜讀曰蓄}
操乃增河東太守杜畿秩中二千石 扶風灋正為劉璋軍議校尉|{
	軍議校尉使之議軍事盖時議必推正之善謀璋能官之而不能用耳}
璋不能用又為其州里俱僑客者所鄙正邑邑不得志|{
	僑寄也寓也鄙薄也邑邑不樂之意}
益州别駕張松與正善自負其才忖璋不足與有為|{
	忖度也思也忖寸本翻}
常竊歎息松勸璋結劉備璋曰誰可使者松乃舉正璋使正往正辭謝佯為不得已而行還為松說備有雄略|{
	為於偽翻}
密謀奉戴以為州主會曹操遣鍾繇向漢中璋聞之内懷恐懼松因說璋曰|{
	說輸芮翻}
曹公兵無敵于天下若因張魯之資以取蜀土誰能禦之劉豫州使君之宗室而曹公之深讎也|{
	使疏吏翻}
善用兵若使之討魯魯必破矣魯破則益州彊曹公雖來無能為也今州諸將龎羲李異等皆恃功驕豪|{
	據裴松之註龎羲免璋諸子於難而李異殺趙韙故各恃功}
欲有外意|{
	謂其意欲附外也}
不得豫州則敵攻其外民攻其内必敗之道也璋然之遣灋正將四千人迎備主簿巴西黄權諫曰|{
	譙周巴記曰劉璋分巴郡墊江已上為巴西郡}
劉左將軍有驍名|{
	曹操表備為左將軍故稱之驍堅堯翻}
今請到欲以部曲遇之則不滿其心欲以賓客禮待則一國不容二君若客有泰山之安則主有累卵之危不若閉境以待時清璋不聽出權為廣漢長|{
	廣漢縣屬廣漢郡長知兩翻}
從事廣漢王累自倒懸于州門以諫璋一無所納灋正至荆州隂獻策於劉備曰以明將軍之英才乘劉牧之懦弱張松州之股肱|{
	别駕州之上佐故曰股肱}
響應於内以取益州猶反掌也 |{
	考異曰韋曜吴書曰備前見張松後得灋正皆厚以恩德接納盡其殷勤之歡因問蜀中濶狹兵器府庫人馬衆寡及諸要害道里遠近松等具言之按劉璋劉備傳松未嘗先見備吴書誤也}
備疑未决龎統言於備曰荆州荒殘人物殫盡東有孫車騎|{
	備表權為車騎將軍故以稱之}
北有曹操難以得志今益州戶口百萬土沃財富誠得以為資大業可成也備曰今指與吾為水火者曹操也|{
	言水火者以其性相反也}
操以急吾以寛操以暴吾以仁操以譎吾以忠|{
	譎古六翻}
每與操反事乃可成耳今以小利而失信義於天下柰何統曰亂離之時固非一道所能定也且兼弱攻昧|{
	尚書仲虺之言}
逆取順守|{
	前書陸賈曰湯武逆取而順守之}
古人所貴若事定之後封以大國何負於信今日不取終為人利耳備以為然乃留諸葛亮關羽等守荆州以趙雲領留營司馬|{
	留營司馬掌留營軍事也}
備將步卒數萬人入益州孫權聞備西上|{
	上時掌翻}
遣舟船迎妹而夫人欲將備子禪還吳張飛趙雲勒兵截江乃得禪還劉璋敕在所供奉備備入境如歸前後贈遺以巨億計|{
	遺于季翻}
備至巴郡巴郡太守嚴顔拊心歎曰此所謂獨坐窮山放虎自衛者也備自江州北由墊江水詣涪|{
	巴郡治江州墊江縣屬巴郡涪縣屬廣漢郡墊江水盖即涪内水也庾仲雍曰江州縣對二水口右則涪内水左則蜀外水墊音疊涪音浮賢曰涪縣故城今綿州城墊江縣唐之合州}
璋率步騎三萬餘人車乘帳幔|{
	乘繩證翻幔莫半翻幕也}
精光耀日往會之張松令灋正白備便於會襲璋備曰此事不可倉卒|{
	卒讀曰猝}
龎統曰今因會執之則將軍無用兵之勞而坐定一州也備曰初入他國恩信未著此不可也璋推備行大司馬領司隸校尉備亦推璋行征西大將軍領益州牧|{
	晉百官志曰四鎮通於柔遠謂鎮東鎮西鎮南鎮北四將軍也}
所將吏士更相之適|{
	之往也更工衡翻}
歡飲百餘日璋增備兵厚加資給使擊張魯又令督白水軍|{
	白水關在廣漢白水縣劉璋置軍屯守即楊懷高沛之軍也杜佑曰梁州金牛縣漢葭萌縣地縣南有故白水關}
備并軍三萬餘人車甲器械資貨甚盛璋還成都備北到葭萌|{
	葭萌縣屬廣漢郡賢曰葭萌今利州益昌縣應劭曰葭音家師古曰萌音氓蜀王封其弟葭萌於此因以名邑先主改曰漢夀}
未即討魯厚樹恩德以收衆心十七年春正月曹操還鄴詔操贊拜不名入朝不趨劒履上殿如蕭何故事 操之西征也河間民田銀蘇伯反扇動幽冀五官將丕欲自討之功曹常林曰|{
	據林傳時為五官將功曹}
北方吏民樂安厭亂|{
	樂音洛}
服化已久守善者多銀伯犬羊相聚不能為害方今大軍在遠外有彊敵將軍為天下之鎮|{
	謂留守鄴也}
輕動遠舉雖克不武乃遣將軍賈信討之應時克滅餘賊千餘人請降議者皆曰公有舊法圍而後降者不赦|{
	降戶江翻}
程昱曰此乃擾攘之際權時之宜今天下略定不可誅之縱誅之宜先啟聞議者皆曰軍事有專無請昱曰凡專命者謂有臨時之急耳今此賊制在賈信之手故老臣不願將軍行之也丕曰善即白操操果不誅既而聞昱之謀甚悦曰君非徒明於軍計又善處人父子之間|{
	以勸丕不專殺也處昌呂翻}
故事破賊文書以一為十國淵上首級皆如其實數|{
	國淵時統留事上時掌翻}
操問其故淵曰夫征討外寇多其斬獲之數者欲以大武功聳民聽也河間在封域之内銀等叛逆雖克捷有功淵竊恥之操大悦 夏五月癸未誅衛尉馬騰夷三族|{
	騰詣鄴見上卷十三年}
六月庚寅晦日有食之 秋七月螟 馬超等餘衆屯藍田夏侯淵擊平之鄜賊梁興|{
	鄜縣前漢屬左馮翊後漢省師古曰鄜音敷}
寇略馮翊諸縣恐懼皆寄治郡下議者以為當移就險阻左馮翊鄭渾曰興等破散藏竄山谷雖有隨者率脅從耳今當廣開降路|{
	降戶江翻下同}
宣諭威信而保險自守此示弱也乃聚吏民治城郭為守備|{
	治直之翻}
募民逐賊得其財物婦女十以七賞民大悦皆願捕賊賊之失妻子者皆還求降渾責其得他婦女然後還之於是轉相寇盜黨與離散又遣吏民有恩信者分布山谷告諭之出者相繼乃使諸縣長吏各還本治以安集之|{
	長知兩翻}
興等懼將餘衆聚鄜城操使夏侯淵助渾討之遂斬興餘黨悉平渾泰之弟也|{
	鄭泰見用于董卓而欲圖卓者也}
九月庚戌立皇子熙為濟隂王懿為山陽王邈為濟北王敦為東海王|{
	時許靖在蜀聞立諸王曰將欲翕之必始張之將欲奪之必姑與之其孟德之謂乎濟子禮翻}
初張紘以秣陵山川形勝勸孫權以為治所及劉備東過秣陵亦勸權居之權於是作石頭城徙治秣陵改秣陵為建業|{
	秣陵屬丹陽郡本金陵也秦始皇改孫權改曰建業後避晉愍帝諱改曰建康石頭城在今建康城西二里金陵志石頭城去臺城九里南合秦淮水張舜民曰石頭城者天生城壁有如城然在清涼寺北覆舟山上江行自北來者循石頭城轉入秦淮陸游曰龍灣望石頭山不甚高然峭立江中繚繞如垣牆清涼寺距石頭里餘西望宣化渡及歷陽諸山宋白曰晉平吴分為二邑自淮水南為秣陵北為建業江表傳紘謂權曰秣林楚武王所置名為金陵地勢岡阜連石頭昔秦始皇東巡經此縣望氣者云金陵地形有王者都邑之氣故掘斷連岡改名秣陵今處所具存宜為都邑獻帝春秋又載權曰秣陵有小江百餘里可以安大船吾方理水軍當移據之又據晉書郗隆傳隆為揚州刺史鎮秣陵齊王冋檄令赴討趙王倫隆停檄不下時王邃鎮石頭隆軍西赴邃者甚衆隆遣從事于牛渚禁之不得止將士卷邃攻殺隆則石頭在牛渚西詳考是事秣陵軍將赴邃欲自牛渚而西勤王也石頭自在牛渚東}
呂蒙聞曹操欲東兵說孫權夾濡須水口立塢|{
	說輸芮翻賢曰濡須水名在今和州歷陽縣西南孫權夾水立塢狀如偃月杜佑曰濡須水在歷陽西南百八十里余據濡須水出巢湖在今無為軍北二十五里濡須塢在今巢縣東南四十里}
諸將皆曰上岸擊賊|{
	上時掌翻}
洗足入船何用塢為蒙曰兵有利鈍戰無百勝如有邂逅敵步騎蹙人不暇及水其得入船乎權曰善遂作濡須塢冬十月曹操東擊孫權 董昭言於曹操曰自古以來人臣匡世未有今日之功有今日之功未有久處人臣之埶者也|{
	處昌呂翻下同}
今明公恥有慙德樂保名節|{
	樂音洛}
然處大臣之埶使人以大事疑已誠不可不重慮也|{
	重直用翻}
乃與列侯諸將議以丞相宜進爵國公九錫備物以彰殊勲|{
	賢曰禮含文嘉曰九錫一曰車馬二曰衣服三日樂器四曰朱戶五曰納陛六曰虎賁百人七曰斧鉞八曰弓矢九曰秬鬯謂之九錫錫予也九錫皆如其德左傳曰分魯公以大路大旂夏后氏之璜封父之繁弱祝宗卜史備物典策}
荀彧以為曹公本興義兵以匡朝寧國|{
	朝直遙翻}
秉忠貞之誠守退讓之實君子愛人以德|{
	記檀弓曾子曰君子之愛人也以德細人之愛人也以姑息}
不宜如此操由是不悦及擊孫權表請彧勞軍於譙|{
	勞力到翻}
因輒留彧以侍中光祿大夫持節參丞相軍事|{
	輒言專輒也}
操軍向濡須彧以疾留夀春飲藥而卒|{
	彧傳云操饋之食發視乃器也於是飲藥而卒考異曰陳志彧傳曰以憂薨范書彧傳曰操饋之食發視乃空器也於是飲藥而卒孫盛魏氏春秋亦同按彧之死操隱其誅陳夀云以憂卒盖闕疑也今不正言其飲藥恐後世為人上者謂隱誅可得而行也}
彧行義修整而有智謀好推賢進士故時人皆惜之|{
	行下孟翻好呼到翻}


臣光曰孔子之言仁也重矣自子路冉求公西赤門人之高第令尹子文陳文子諸侯之賢大夫皆不足以當之而獨稱管仲之仁豈非以其輔佐齊桓大濟生民乎|{
	論語孟武伯問子路仁乎子曰不知也又問子曰由也千乘之國可使治其賦也不知其仁也求也何如曰求也千室之邑百乘之家可使為之宰也不知其仁也赤也何如曰赤也束帶立于朝可使與賓客言也不知其仁也子張問曰令尹子文三仕為令尹無喜色三已之無慍色舊令尹之政必以告新令尹何如子曰忠矣曰仁矣乎曰未知焉得仁崔子弑齊君陳文子有馬十乘棄而違之至於他邦則曰猶吾大夫崔子也違之之一邦則又曰猶吾大夫崔子也違之何如子曰清矣曰仁矣乎曰未知焉得仁子貢曰管仲非仁者與桓公殺公子糾不能死又相之子曰管仲相桓公霸諸侯一匡天下民到于今受其賜微管仲吾其被髪左衽矣豈若匹夫匹婦之為諒也自經于溝瀆而莫之知也子路曰桓公殺公子糾召忽死之管仲不死曰未仁乎子曰桓公九合諸侯不以兵車管仲之力也如其仁如其仁}
齊桓之行若狗彘管仲不羞而相之|{
	行下孟翻相息亮翻}
其志蓋以非桓公則生民不可得而濟也漢末大亂羣生塗炭自非高世之才不能濟也然則荀彧捨魏武將誰事哉齊桓之時周室雖衰未若建安之初也建安之初四海蕩覆尺土一民皆非漢有荀彧佐魏武而興之舉賢用能訓卒厲兵决機發策征伐四克遂能以弱為彊化亂為治|{
	治直吏翻}
十分天下而有其八其功豈在管仲之後乎管仲不死子糾而荀彧死漢室其仁復居管仲之先矣|{
	復扶又翻}
而杜牧乃以為彧之勸魏武取兖州則比之高光官渡不令還許則比之楚漢及事就功畢乃欲邀名於漢代譬之教盜穴牆發匱而不與同挈得不為盜乎臣以為孔子稱文勝質則史|{
	見論語}
凡為史者記人之言必有以文之然則比魏武於高光楚漢者史氏之文也豈皆彧口所言邪用是貶彧非其罪矣且使魏武為帝則彧為佐命元功與蕭何同賞矣彧不利此而利於殺身以邀名豈人情乎

十二月有星孛於五諸侯|{
	晉天文志曰五諸侯五星在東井北又太微南蕃左執法東北一星曰謁者謁者東北三星曰三公三公北三星曰九卿九卿西五星曰内五諸侯内侍天子不之國也孛蒲内翻}
劉備在葭萌龎統言于備曰今隂選精兵晝夜兼道徑襲成都劉璋既不武又素無豫備大軍卒至|{
	卒讀曰猝}
一舉便定此上計也楊懷高沛璋之名將各仗彊兵據守關頭|{
	即白水關頭也}
聞數有牋諫璋|{
	數所角翻}
使發遣將軍還荆州將軍遣與相聞說荆州有急欲還救之並使裝束外作歸形此二子既服將軍英名又喜將軍之去計必乘輕騎來見將軍因此執之進取其兵乃向成都此中計也退還白帝|{
	白帝即巴東魚復縣城也公孫述據成都自稱白帝改魚復曰白帝城}
連引荆州徐還圖之此下計也若沈吟不去|{
	沈持林翻}
將致大困不可久矣備然其中計及曹操攻孫權權呼備自救備貽璋書曰孫氏與孤本為脣齒而關羽兵弱今不往救則曹操必取荆州轉侵州界|{
	州界謂益州界}
其憂甚於張魯魯自守之賊不足慮也因求益萬兵及資糧璋但許兵四千其餘皆給半備因激怒其衆曰吾為益州征彊敵師徒勤瘁|{
	瘁秦醉翻}
而積財吝賞何以使士大夫死戰乎張松書與備及灋正曰今大事垂立如何釋此去乎松兄廣漢太守肅恐禍及已因發其謀於是璋收斬松敕關戍諸將文書皆勿復得與備關通|{
	復扶又翻}
備大怒召璋白水軍督楊懷高沛責以無禮斬之|{
	責其無客主之禮也}
勒兵徑至關頭并其兵進據涪城|{
	此用龎統之中計也}


十八年春正月曹操進軍濡須口號步騎四十萬攻破孫權江西營|{
	大江東北流故自歷陽至濡須口皆謂之江西而建業謂之江東}
其都督公孫陽權率衆七萬禦之相守月餘操見其舟船器仗軍伍整肅歎曰生子當如孫仲謀|{
	孫權字仲謀}
如劉景升兒子豚犬耳權為牋與操說春水方生公宜速去别紙言足下不死孤不得安操語諸將曰|{
	語牛倨翻}
孫權不欺孤乃徹軍還 庚寅詔并十四州復為九州|{
	十四州司豫冀兖徐青荆揚益梁雍并幽交也復為九州者割司州之河東河内馮翊扶風及幽并二州皆入冀州凉州所統悉入雍州又以司州之京兆入焉又以司州之弘農河南入豫州交州并入荆州則省司凉幽并而復禹貢之九州矣此曹操自領冀州牧欲廣其所統以制天下耳}
夏四月曹操至鄴 初曹操在譙恐濱江郡縣為孫權所略欲徙令近内|{
	近其靳翻}
以問揚州别駕蔣濟曰昔孤與袁本初對軍官渡徙燕白馬民民不得走賊亦不敢鈔|{
	事見六十三卷建安五年燕縣白馬縣皆屬東郡燕春秋之南燕國也賢曰燕故城今滑州胙城縣鈔楚交翻燕于賢翻}
今欲徙淮南民何如對曰是時兵弱賊彊不徙必失之自破袁紹以來明公威震天下民無他志人情懷土實不樂徙|{
	樂音洛}
懼必不安操不從既而民轉相驚自廬江九江鄿春廣陵戶十餘萬皆東渡江|{
	鄿春縣本屬江夏郡沈約曰吳立鄿春郡此據吳志書之也鄿音祁}
江西遂虛合淝以南惟有皖城|{
	皖縣屬廬江郡賢曰今舒州懷寧縣師古曰皖音胡管翻}
濟後奉使詣鄴|{
	使疏吏翻}
操迎見大笑曰本但欲使避賊乃更驅盡之拜濟丹陽太守|{
	丹陽郡已屬孫權濟不得之郡也}
五月丙申以冀州十郡封曹操為魏公|{
	時以冀州之河東河内魏郡趙國中山常山鉅鹿安平甘陵平原凡十郡為魏國}
以丞相領冀州牧如故又加九錫大輅戎輅各一玄牡二駟衮冕之服赤舄副焉|{
	毛萇曰赤舄人君之盛屨也釋舄複履也鄭玄曰複下曰舄鄭衆曰舄有三等赤舄為上冕服之舄}
軒縣之樂八佾之舞|{
	周禮樂縣之位王宫縣諸侯軒縣鄭衆曰宫縣四而縣軒縣去其一面縣讀曰懸舞佾之數天子八諸侯六杜預曰八佾八八六十四人六佾六六三十六人服䖍曰天子八八諸侯六八大夫四八士二八宋傳隆曰鄭伯納晉悼公女樂二八晉以一八賜魏絳此樂以八人為列之證也佾音逸}
朱戶以居納陛以登虎賁之士三百人鈇鉞各一彤弓一彤矢百玈弓十玈矢千|{
	玈與盧同黑色也}
秬鬯一卣珪瓚副焉 大雨水益州從事廣漢鄭度聞劉備舉兵謂劉璋曰左將軍懸軍襲我兵不滿萬士衆未附軍無輜重|{
	重直用翻}
野糓是資其計莫若盡驅巴西梓潼民内涪水以西|{
	梓潼縣屬廣漢郡漢武帝元鼎元年置以縣倚梓林而枕潼水為名建安二十二年劉備分立梓潼郡班志梓潼有五婦山駞水所出南入涪應劭曰涪水出廣漢南入漢水經曰涪水出廣漢涪縣西北東至廣漢與梓潼水合又西南流又南入于墊江註云涪水出廣漢屬國剛氏道徼外梓潼水即五婦水也同入于墊江即所謂内水也}
其倉廩野穀一皆燒除高壘深溝靜以待之彼至請戰勿許久無所資不過百日必將自走走而擊之此必禽耳劉備聞而惡之|{
	惡烏路翻}
以問法正正曰璋終不能用無憂也璋果謂其羣下曰吾聞拒敵以安民未聞動民以避敵也不用度計璋遣其將劉璝冷苞張任鄧賢吳懿等拒備皆敗退保緜竹|{
	璝姑回翻又胡隈翻冷魯杏翻姓也按本或作泠泠音魯經翻緜竹縣屬廣漢郡唐屬漢州九域志在州東北九十三里}
懿詣軍降|{
	降戶江翻下同}
璋復遣護軍南陽李嚴江夏費觀督緜竹諸軍|{
	復扶又翻下同夏戶雅翻費父沸翻}
嚴觀亦率其衆降於備備軍益彊分遣諸將平下屬縣劉璝張任與璋子循退守雒城|{
	雒縣屬廣漢郡雒水所出唐為漢州治所}
備進軍圍之任勒兵出戰於鴈橋|{
	鴈江在雒縣南曾有金鴈故名為鴈橋}
軍敗任死 秋七月魏始建社稷宗廟 魏公操納三女為貴人|{
	自此以後曹操不書姓而冠以國操三女長憲次節次華節後立為皇后}
初魏公操追馬超至安定聞田銀蘇伯反引軍還參凉州軍事楊阜言於操曰超有信布之勇甚得羌胡心若大軍還不設備隴上諸郡非國家之有也|{
	隴西南安漢陽永陽皆隴上諸郡也獻帝起居注初平四年分漢陽上郡為永陽}
操還超果率羌胡擊隴上諸郡縣郡縣皆應之惟冀城奉州郡以固守|{
	冀縣屬漢陽郡郡及凉州刺史治焉}
超盡兼隴右之衆張魯復遣大將楊昂助之|{
	復扶又翻}
凡萬餘人攻冀城自正月至八月救兵不至刺史韋康遣别駕閻溫出告急於夏侯淵|{
	夏侯淵時屯長安}
外圍數重|{
	重直龍翻}
溫夜從水中潛出明日超兵見其迹遣追獲之超載溫詣城下使告城中云東方無救|{
	隴右在西方操在關東故曰東方}
溫向城大呼曰|{
	呼火故翻}
大軍不過三日至勉之城中皆泣稱萬歲超雖怒猶以攻城久不下徐徐更誘溫冀其改意|{
	誘音酉}
溫曰事君有死無二而卿乃欲令長者出不義之言乎超遂殺之已而外救不至韋康及太守欲降|{
	降戶江翻}
楊阜號哭諫曰阜等率父兄子弟以義相勵有死無二以為使君守此城|{
	號戶刀翻為於偽翻}
今柰何棄垂成之功陷不義之名乎刺史太守不聽開城門迎超超入遂殺刺史太守自稱征西將軍領并州牧督凉州軍事魏公操使夏侯淵救冀未到而冀敗淵去冀二百餘里超來逆戰淵軍不利氐王千萬反應超屯興國|{
	氐王千萬畧陽清水氐種也其後是為仇池之楊興國城名}
淵引軍還會楊阜喪妻就超求假以葬之|{
	喪息浪翻假居訝翻休假也求假猶古之請告請急也}
阜外兄天水姜叙為撫夷將軍擁兵屯歷城|{
	水經註歷城在西縣去仇池一百二十里後改為建安城杜佑曰歷城在今同谷郡西七里去仇池九十里宋白曰晉置仇池郡於歷城今為成州}
阜見叙及其母歔欷悲甚|{
	歔音虛欷許既翻又音希泣餘聲也}
叙曰何為乃爾阜曰守城不能完君亡不能死亦何面目以視息於天下|{
	目之視物一出入息之頃則一瞬}
馬超背父叛君虐殺州將|{
	背蒲妹翻將即亮翻}
豈獨阜之憂責一州士大夫皆蒙其恥君擁兵專制而無討賊心此趙盾所以書弑君也|{
	趙盾晉卿趙宣子也左傳趙穿攻靈公于桃園宣子未出疆而復太史書曰趙盾弑其君以示于朝宣子曰不然對曰子為正卿亡不越境反不討賊非子而誰}
超彊而無義多釁易圖耳|{
	易以䜴翻}
叙母慨然曰咄伯奕韋使君遇難亦汝之負豈獨義山哉|{
	咄當没翻姜叙字伯奕楊阜字義山負罪負也難乃旦翻}
人誰不死死於忠義得其所也但當速發勿復顧我我自為汝當之|{
	復扶又翻為于偽翻}
不以餘年累汝也|{
	累力瑞翻}
叙乃與同郡趙昂尹奉武都李俊等合謀討超又使人至冀結安定梁寛南安趙衢使為内應超取趙昂子月為質|{
	質音致}
昂謂妻異曰|{
	據皇甫謐列女傳異士氏女也}
吾謀如是事必萬全當柰月何異厲聲應曰雪君父之大恥喪元不足為重|{
	喪息浪翻}
况一子哉九月阜與叙進兵入鹵城|{
	鹵城當在西縣冀縣之間}
昂奉據祁山以討超|{
	水經註祁山在嶓冢之西七十許里山上有城極為險固漢水逕其南又曰祁山在上邽西南二百四十里杜佑曰祁山在今同谷郡長道縣東十里余據今西和州長道縣南十里有祁山古來南北二岈有萬餘家諸葛亮表言祁山去沮五百里有人萬戶者此也}
超聞之大怒趙衢因譎說超使自出擊之|{
	譎古穴翻說輸芮翻}
超出衢與梁寛閉冀城門盡殺超妻子超進退失據乃襲歷城得叙母叙母罵之曰汝背父之逆子殺君之桀賊|{
	背父謂馬騰在鄴不顧而反殺君謂殺韋康也背蒲妹翻}
天地豈久容汝而不早死敢以面目視人乎超殺之又殺趙昂之子月楊阜與超戰身被五創超兵敗遂南犇張魯|{
	被皮義翻創初良翻 考異曰楊阜傳云十七年九月武帝紀十八年超在漢陽復因羌胡為害十九年正月趙衢等討超超犇漢中按姜叙九月起兵超即應出討超出衢等即應閉門不應至來年正月盖魏史書捷音到鄴之月耳楊阜傳誤也}
魯以超為都講祭酒|{
	魯為五斗米道自號師君其來學者初名鬼卒後號祭酒各領部衆都講祭酒者魯使學者都習老子五千文置都講祭酒位次師君}
欲妻之以女|{
	妻七細翻}
或謂魯曰有人若此不愛其親焉能愛人|{
	焉於䖍翻}
魯乃止操封討超之功侯者十一人賜楊阜爵關内侯 冬十一月魏初置尚書侍中六卿以荀攸為尚書令凉茂為僕射|{
	凉姓茂名}
毛玠崔琰常林徐奕何夔為尚書|{
	魏置五曹尚書吏部左民客曹五兵度支}
王粲杜襲衛覬和洽為侍中|{
	自是以後侍中遂以四人為定員}
鍾繇為大理|{
	大理漢廷尉之職}
王修為大司農袁渙為郎中令行御史大夫事|{
	郎中令漢光禄勲之職}
陳羣為御史中丞|{
	時以御史大夫為三公以中丞為御史臺主}
袁渙得賞賜皆散之家無所儲乏則取之於人不為皦察之行|{
	皦吉了翻行下孟翻}
然時人皆服其清時有傳劉備死者羣臣皆賀惟渙獨否魏公操欲復肉刑令曰昔陳鴻臚以為死刑有可加於仁恩者|{
	臚陵如翻}
御史中丞能申其父之論乎|{
	陳羣父紀為漢大鴻臚}
陳羣對曰臣父紀以為漢除肉刑而增加於笞|{
	事見十五卷文帝十三年}
本興仁惻而死者更衆所謂名輕而實重者也名輕則易犯|{
	易以䜴翻}
實重則傷民且殺人償死合於古制至於傷人或殘毁其體而裁剪毛髪非其理也若用古刑使淫者下蠶室盜者刖其足則永無淫放穿踰之姦矣|{
	下遐稼翻刖音月穿者穿穴隙踰者踰垣牆}
夫三千之屬|{
	周穆王作甫刑墨罰之屬千劓罰之屬千剕罰之屬五百宫罰之屬三百大辟之罰其屬二百五刑屬三千}
雖未可悉復若斯數者時之所患宜先施用漢律所殺殊死之罪仁所不及也其餘逮死者可易以肉刑如此則所刑之與所生足以相貿矣|{
	貿易也}
今以笞死之灋易不殺之刑是重人支體而輕人軀命也當時議者唯鍾繇與羣議同餘皆以為未可行操以軍事未罷顧衆議而止

資治通鑑卷六十六














































































































































