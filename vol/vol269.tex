<!DOCTYPE html PUBLIC "-//W3C//DTD XHTML 1.0 Transitional//EN" "http://www.w3.org/TR/xhtml1/DTD/xhtml1-transitional.dtd">
<html xmlns="http://www.w3.org/1999/xhtml">
<head>
<meta http-equiv="Content-Type" content="text/html; charset=utf-8" />
<meta http-equiv="X-UA-Compatible" content="IE=Edge,chrome=1">
<title>資治通鑒_270-資治通鑑卷二百六十九_270-資治通鑑卷二百六十九</title>
<meta name="Keywords" content="資治通鑒_270-資治通鑑卷二百六十九_270-資治通鑑卷二百六十九">
<meta name="Description" content="資治通鑒_270-資治通鑑卷二百六十九_270-資治通鑑卷二百六十九">
<meta http-equiv="Cache-Control" content="no-transform" />
<meta http-equiv="Cache-Control" content="no-siteapp" />
<link href="/img/style.css" rel="stylesheet" type="text/css" />
<script src="/img/m.js?2020"></script> 
</head>
<body>
 <div class="ClassNavi">
<a  href="/24shi/">二十四史</a> | <a href="/SiKuQuanShu/">四库全书</a> | <a href="http://www.guoxuedashi.com/gjtsjc/"><font  color="#FF0000">古今图书集成</font></a> | <a href="/renwu/">历史人物</a> | <a href="/ShuoWenJieZi/"><font  color="#FF0000">说文解字</a></font> | <a href="/chengyu/">成语词典</a> | <a  target="_blank"  href="http://www.guoxuedashi.com/jgwhj/"><font  color="#FF0000">甲骨文合集</font></a> | <a href="/yzjwjc/"><font  color="#FF0000">殷周金文集成</font></a> | <a href="/xiangxingzi/"><font color="#0000FF">象形字典</font></a> | <a href="/13jing/"><font  color="#FF0000">十三经索引</font></a> | <a href="/zixing/"><font  color="#FF0000">字体转换器</font></a> | <a href="/zidian/xz/"><font color="#0000FF">篆书识别</font></a> | <a href="/jinfanyi/">近义反义词</a> | <a href="/duilian/">对联大全</a> | <a href="/jiapu/"><font  color="#0000FF">家谱族谱查询</font></a> | <a href="http://www.guoxuemi.com/hafo/" target="_blank" ><font color="#FF0000">哈佛古籍</font></a> 
</div>

 <!-- 头部导航开始 -->
<div class="w1180 head clearfix">
  <div class="head_logo l"><a title="国学大师官网" href="http://www.guoxuedashi.com" target="_blank"></a></div>
  <div class="head_sr l">
  <div id="head1">
  
  <a href="http://www.guoxuedashi.com/zidian/bujian/" target="_blank" ><img src="http://www.guoxuedashi.com/img/top1.gif" width="88" height="60" border="0" title="部件查字,支持20万汉字"></a>


<a href="http://www.guoxuedashi.com/help/yingpan.php" target="_blank"><img src="http://www.guoxuedashi.com/img/top230.gif" width="600" height="62" border="0" ></a>


  </div>
  <div id="head3"><a href="javascript:" onClick="javascript:window.external.AddFavorite(window.location.href,document.title);">添加收藏</a>
  <br><a href="/help/setie.php">搜索引擎</a>
  <br><a href="/help/zanzhu.php">赞助本站</a></div>
  <div id="head2">
 <a href="http://www.guoxuemi.com/" target="_blank"><img src="http://www.guoxuedashi.com/img/guoxuemi.gif" width="95" height="62" border="0" style="margin-left:2px;" title="国学迷"></a>
  

  </div>
</div>
  <div class="clear"></div>
  <div class="head_nav">
  <p><a href="/">首页</a> | <a href="/ShuKu/">国学书库</a> | <a href="/guji/">影印古籍</a> | <a href="/shici/">诗词宝典</a> | <a   href="/SiKuQuanShu/gxjx.php">精选</a> <b>|</b> <a href="/zidian/">汉语字典</a> | <a href="/hydcd/">汉语词典</a> | <a href="http://www.guoxuedashi.com/zidian/bujian/"><font  color="#CC0066">部件查字</font></a> | <a href="http://www.sfds.cn/"><font  color="#CC0066">书法大师</font></a> | <a href="/jgwhj/">甲骨文</a> <b>|</b> <a href="/b/4/"><font  color="#CC0066">解密</font></a> | <a href="/renwu/">历史人物</a> | <a href="/diangu/">历史典故</a> | <a href="/xingshi/">姓氏</a> | <a href="/minzu/">民族</a> <b>|</b> <a href="/mz/"><font  color="#CC0066">世界名著</font></a> | <a href="/download/">软件下载</a>
</p>
<p><a href="/b/"><font  color="#CC0066">历史</font></a> | <a href="http://skqs.guoxuedashi.com/" target="_blank">四库全书</a> |  <a href="http://www.guoxuedashi.com/search/" target="_blank"><font  color="#CC0066">全文检索</font></a> | <a href="http://www.guoxuedashi.com/shumu/">古籍书目</a> | <a   href="/24shi/">正史</a> <b>|</b> <a href="/chengyu/">成语词典</a> | <a href="/kangxi/" title="康熙字典">康熙字典</a> | <a href="/ShuoWenJieZi/">说文解字</a> | <a href="/zixing/yanbian/">字形演变</a> | <a href="/yzjwjc/">金 文</a> <b>|</b>  <a href="/shijian/nian-hao/">年号</a> | <a href="/diming/">历史地名</a> | <a href="/shijian/">历史事件</a> | <a href="/guanzhi/">官职</a> | <a href="/lishi/">知识</a> <b>|</b> <a href="/zhongyi/">中医中药</a> | <a href="http://www.guoxuedashi.com/forum/">留言反馈</a>
</p>
  </div>
</div>
<!-- 头部导航END --> 
<!-- 内容区开始 --> 
<div class="w1180 clearfix">
  <div class="info l">
   
<div class="clearfix" style="background:#f5faff;">
<script src='http://www.guoxuedashi.com/img/headersou.js'></script>

</div>
  <div class="info_tree"><a href="http://www.guoxuedashi.com">首页</a> > <a href="/SiKuQuanShu/fanti/">四库全书</a>
 > <h1>资治通鉴</h1> <!--         下载:【右键另存为】即可 --></div>
  <div class="info_content zj clearfix">
  
<div class="info_txt clearfix" id="show">
<center style="font-size:24px;">270-資治通鑑卷二百六十九</center>
    資治通鑑卷二百六十九 宋 司馬光 撰<br />
<br />
  胡三省 音註<br />
<br />
  後梁紀四【起昭陽作噩十二月盡彊圉赤奮若六月凡三年有奇】<br />
<br />
  均王上下<br />
<br />
  乾化三年十二月吳鎮海節度使徐温平盧節度使朱瑾帥諸將拒之【拒王景仁也帥讀曰率】遇于趙步【趙步瀕淮津濟之處南直壽春紫金山】吳徵兵未集温以四千餘人與景仁戰不勝而却景仁引兵乘之將及於隘【隘烏介翻險狹之處為隘】吳吏士皆失色左驍衛大將軍宛丘陳紹援槍大呼【援于元翻呼火故翻】曰誘敵太深可以進矣【誘音酉】躍馬還鬭衆隨之梁兵乃退温拊其背曰非子之智勇吾幾困矣【幾居依翻】賜之金帛紹悉以分麾下吳兵旣集復戰於霍丘梁兵大敗王景仁以數騎殿吳人不敢逼【殿丁練翻王景仁本吳之名將吳人素畏之故不敢逼】梁之渡淮而南也表其可涉之津【立表以記淺】霍丘守將朱景浮表於木徙置深淵【朱景霍丘土豪也吳用以為將守霍丘浮表於木者徙梁所立之表其下接之以木立諸深淵以誤之】及梁兵敗還【還從宣翻】望表而涉溺死者大半吳人聚梁尸為京觀於霍丘【觀古玩翻】 庚午晉王以周德威為盧龍節度使兼侍中以李嗣本為振武節度使【先是周德威以破夾寨之功帥振武今以平燕之功徙帥盧龍以李嗣本代帥振武歐史義兒傳嗣本本鴈門張氏子】燕主守光將奔滄州就劉守奇【劉守奇藉兵於梁以取滄州事見上卷上年】涉寒足腫【史炤曰釋名曰腫鍾也寒熱氣聚也】且迷失道至燕樂之境【燕樂縣後魏置治白檀古城唐長壽二年徙治新興城屬檀州宋白曰燕樂密雲二縣皆漢虒奚縣地樂音洛】晝匿阬谷數日不食令妻祝氏乞食於田父張師造家師造怪婦人異狀詰知守光處【詰去吉翻】并其三子擒之癸酉晉王方晏將吏擒守光適至王語之曰【語牛倨翻】主人何避客之深邪并仁恭置之館舍以器服膳飲賜之王命掌書記王緘草露布緘不知故事書之於布遣人曳之【魏晉以來每戰勝則書捷狀建之漆竿使天下皆知之謂之露布露布者暴白其事而布告天下未嘗書之於布而使人曳之也文心雕龍曰露布者蓋露板不封布諸觀聽也】晉王欲自雲代歸【自幽州取山後路歷雲代等州至晉陽】趙王鎔及王處直請由中山真定趣井陘【王處直王鎔欲晉王取道中山真定各展迎賀之禮趣七喻翻】王從之庚辰晉王發幽州劉仁恭父子皆荷校於露布之下【荷下可翻又音何校爻教翻易曰荷校滅耳注云校者以木絞校者也即械也校者取其通名也】守光父母唾其面而罵之曰逆賊破我家至此守光俛首而已【俛音免】甲申至定州舍於關城丙戌晉王與王處直謁北嶽廟【北嶽廟在恒山之大茂山恒山在定州曲陽縣西北】是日至行唐【行唐漢南行唐縣後魏曰行唐唐屬鎮州九域志在州北五十五里】趙王鎔迎謁于路<br />
<br />
  四年春正月戊戌朔趙王鎔詣晉王行帳上壽置酒鎔願識劉太師面【上時掌翻劉守光旣囚其父仁恭請於梁以太師致仕故王鎔因而稱之】晉王命吏脱仁恭及守光械引就席同宴鎔答其拜又以衣服鞍馬酒饌贈之【饌雛戀翻又雛晥翻】己亥晉王與鎔畋于行唐之西鎔送境上而别 丙子蜀主命太子判六軍開崇勲府置僚屬後更謂之天策府【更工衡翻】 壬子晉王以練劉仁恭父子凱歌入于晉陽【充夜翻縶縛之也戰勝得國而歸故奏凱歌】丙辰獻于太廟自臨斬劉守光守光呼曰守光死不恨【呼火故翻】然教守光不降者李小喜也【事見上卷上年】王召小喜證之小喜瞋目叱守光曰【瞋昌真翻】汝内亂禽獸行亦我教邪【行下孟翻】王怒其無禮先斬之【怒其無禮於舊君也】守光曰守光善騎射王欲成霸業何不留之使自効其二妻李氏祝氏讓之曰【讓責也】皇帝事已如此生亦何益即伸頸就戮守光至死號泣哀祈不已【史言劉守光畏死婦人之不若號戶高翻】王命節度副使盧汝弼等械仁恭至代州刺其心血以祭先王墓然後斬之【以劉仁恭叛其父也晉王葬其先王於代州鴈門縣後名為建極陵刺七亦翻】或說趙王鎔曰【說式芮翻】大王所稱尚書令乃梁官也大王旣與梁為讎不當稱其官且自太宗踐阼已來無敢當其名者【唐太宗自尚書令即帝位後之臣下率不敢當其名唐之將亡始以授藩帥】今晉王為盟主勳高位卑不若以尚書令讓之【讓遜也】鎔曰善乃與王處直各遣使推晉王為尚書令晉王三讓然後受之始開府置行臺如太宗故事【唐太宗置行臺事見高祖紀】 高季昌以蜀夔萬忠涪四州舊隸荆南興兵取之【涪音浮】先以水軍攻夔州時鎮江節度使兼侍中嘉王宗壽鎮忠州【蜀置鎮江軍節度領夔忠萬三州】夔州刺史王成先請甲宗壽但以白布袍給之成先帥之逆戰【帥讀曰率】季昌縱火船焚蜀浮橋招討副使張武舉鐵絙拒之【唐昭宗天祐元年張武以鐵絙鏁峽絙戶登翻】船不得進會風反荆南兵焚溺死者甚衆【乘順風以縱火船風反故自焚】季昌乘戰艦【艦戶黯翻】蒙以牛革飛石中之折其尾【中竹仲翻折而設翻】季昌易小舟而遁荆南兵大敗俘斬五千級成先密遣人奏宗壽不給甲之狀宗壽獲之召成先斬之 帝以岐人數為寇【數所角翻】二月徙感化節度使康懷英為永平節度使鎮長安【感化軍陜州梁初徙佑國軍於長安改為永平軍】懷英即懷貞也避帝名改焉 夏四月丙子蜀主徙鎮江軍治夔州丁丑司空兼門下侍郎同平章事于兢坐挾私遷補軍校【校戶教翻】罷為工部侍郎再貶萊州司馬 吳袁州刺史劉崇景叛附於楚崇景威之子也【劉威與楊行密同起於合肥有戰功歷方鎮】楚將許貞將萬人援之吳都指揮使柴再用米志誠帥諸將討之【此都指揮使盡統諸將非一都之指揮使帥讀曰率】 楚岳州刺史許德勲將水軍巡邊【楚之岳州東北皆邊於吳】夜分【夜半為夜分】南風暴起都指揮使王環乘風趣黄州【王環乃一州之都指揮使趣七喻翻下同】以繩梯登城徑趣州署執吳刺史馬鄴大掠而還【還從宣翻又如字】德勲曰鄂州將邀我宜備之【自黄州還岳州舟過鄂州城外故許德勲畏之】環曰我軍入黄州鄂人不知奄過其城【奄忽也】彼自救不暇安敢邀我乃展旗鳴鼓而行【以示不恐】鄂人不敢逼 五月朔方節度使兼中書令潁川王韓遜卒軍中推其子洙為留後癸丑詔以洙為節度使 吳柴再用等與劉崇景許貞戰於萬勝岡大破之崇景貞弃袁州遁去晉王旣克幽州乃謀入寇秋七月會趙王鎔及周德威於趙州南寇邢州李嗣昭引昭義兵會之楊師厚引兵救邢州軍於漳水之東【楊師厚自魏州引兵救邢州】晉軍至張公橋【晉軍出青山口至張公橋在邢州龍岡縣界按薛史唐末葛從周敗晉軍於沙河追至張公橋沙河縣在邢州南二十五里而邢州治龍岡則可知矣】裨將曹進金來奔晉軍退諸鎮兵皆引歸【諸鎮兵謂燕趙潞之兵】八月晉王還晉陽 蜀武泰節度使王宗訓鎮黔州【黔其今翻又其炎翻】貪㬥不法擅還成都庚辰見蜀主多所邀求言辭狂悖【悖蒲昧翻又蒲没翻】蜀主怒命衛士敺殺之【敺烏口翻】戊子以内樞密使潘峭為武泰節度使同平章事【峭七笑翻】翰林學士承旨毛文錫為禮部尚書判樞密院峽上有堰或勸蜀主乘夏秋江漲決之以灌江陵毛文錫諫曰高季昌不服其民何罪陛下方以德懷天下忍以鄰國之民為魚鼈食乎蜀主乃止 帝以福王友璋為武寧節度使前節度使王殷友珪所置也懼不受代叛附於吳九月命淮南西北面招討應接使牛存節及開封尹劉鄩將兵討之冬十月存節等軍于宿州【九域志徐州南至宿州一百四十五里牛存節不徑攻徐州而南屯宿州據埇橋之要所以絶淮南之援也】吳平盧節度使朱瑾等將兵救徐州存節等逆擊破之吳兵引歸 十一月乙巳南詔寇黎州蜀主以夔王宗範兼中書令宗播嘉王宗壽為三招討以擊之丙辰敗之於潘倉嶂斬其酋長趙嵯政等【敗補邁翻酋慈秋翻長知兩翻嵯才何翻】壬戌又敗之於山口城十二月乙亥破其武侯嶺十三寨【黎州南界有潘倉武侯等十一城路振九域志王宗播出卭崍關至潘倉大破蠻衆追奔至山口城則是潘倉在卭崍之南山口城又在潘倉之南也】辛巳又敗之於大渡河【按九域志黎州三面阻大渡河南面至大渡河一百里東南面至大渡河一百二十里西南面至大渡河三百里】俘斬數萬級蠻爭走度水橋絶溺死者數萬人宗範等將作浮梁濟大渡河攻之蜀主召之令還【蠻地深阻不欲勞師遠攻驅之出境而已此蜀主之志也】 癸未蜀興州刺史兼北路制置指揮使王宗鐸攻岐階州【九域志興州西南至階州五百一十里】及固鎮【固鎮在青泥嶺東北薛史地里志鳳州固鎮之地周顯德六年升為雄勝軍】破細沙等十一寨斬首四千級甲申指揮使王宗儼破岐長城關等四寨斬首二千級 岐靜難節度使李繼徽【難乃旦翻】為其子彥魯所毒而死彦魯自為留後<br />
<br />
  貞明元年【是年十一月方改元貞明】春正月己亥蜀主御得賢門受蠻俘大赦初黎雅蠻酋劉昌嗣郝玄鑒楊師泰雖内屬於唐受爵賞號㚋金堡三王【史炤曰㚋大也多也今按㚋音丁么翻蠻語多也大也唐書黎卭二州之西有三王蠻蓋莋都夷白馬氐之遺種楊劉郝三姓世為長襲封王謂之三王部落疊甓而居號㚋舍至宋又有趙王二族并劉郝楊謂之五部落居黎州之西去州百餘里限以飛越嶺其居疊石為㚋積糗糧器甲於上族無君長惟老宿之聽往來漢地悉能華言故比諸羌尤桀黠】而濳通南詔為之詗導鎮蜀者多文臣雖知其情不敢詰【詗古迥翻又翾正翻詰去吉翻窮問也】至是蜀主數以漏洩軍謀【數所具翻】斬於成都市毁㚋金堡自是南詔不復犯邊【復扶又翻】 二月牛存節等拔彭城王殷舉族自焚 【考異曰莊宗列傳朱友貞傳云乾化四年十一月拔徐州殷自燔死五代通錄薛史紀及王殷傳皆云貞明元年者今從之】 三月丁卯以右僕射兼門下侍郎同平章事趙光逢為太子太保致仕 天雄節度使兼中書令鄴王楊師厚卒師厚晚年矜功恃衆擅割財賦選軍中驍勇置銀槍効節都數千人給賜優厚欲以復故時牙兵之盛【魏博自田承嗣置牙兵至羅紹威而除楊師厚復置之】帝雖外加尊禮内實忌之及卒私於宫中受賀【畏其偪而幸其死】租庸使趙巖【租庸使自唐中世以來有之五代會要梁置租庸使其班在崇政使之下宣徽使之上】判官邵贊【判官租庸判官】言於帝曰魏博為唐腹心之蠧二百餘年不能除去者【去羌呂翻】以其地廣兵強之故也羅紹威楊師厚據之朝廷皆不能制陛下不乘此時為之計所謂彈疽不嚴必將復聚【言彈疽者必不畏病疽者之疼盡彈去其膿血然後新肉生而病已否則將復結聚也醫工彈疽用砭石】安知來者不為師厚乎宜分六州為兩鎮以弱其權【考異曰莊宗列傳宰相敬翔租庸使趙巖判官邵贊等為友貞畫策分魏博六州為兩鎮薛史無敬翔名今從之】帝以為然以平盧節度使賀德倫為天雄節度使置昭德軍於相州割澶衛二州隸焉【相息亮翻澶時連翻】以宣徽使張筠為昭德節度使仍分魏州將士府庫之半於相州筠海州人也二人旣赴鎮朝廷恐魏人不服遣開封尹劉鄩將兵六萬自白馬濟河【白馬津在滑州】以討鎮定為名實張形勢以脅之魏兵皆父子相承數百年【曰數百年者言其來也久非必實經歷數百年也】族姻磐結不願分徙德倫屢趣之【趣讀曰促】應行者皆嗟怨連營聚哭己丑劉鄩屯南樂【南樂本唐魏州昌樂縣後唐避獻祖諱改曰南樂史因而書之九域志南樂縣在魏州南四十四里】先遣澶州刺史王彦章將龍驤五百騎入魏州屯金波亭魏兵相與謀曰朝廷忌吾軍府彊盛欲設策使之殘破耳吾六州歷代藩鎮兵未嘗遠出河門【按舊唐書魏州城外有河門舊隄樂彦禎築羅城約河門舊隄周八十里】一旦骨肉流離生不如死是夕軍亂 【考異曰莊宗列傳二十七日劉鄩屯南樂遣龍驤都將王彦章以五百騎入魏州是夜三鼓魏軍亂是月辛酉朔薛史紀云己丑魏傅軍作亂蓋莊宗列傳九字誤為七字耳】縱火大掠圍金波亭王彦章斬關而走詰旦亂兵入牙城殺賀德倫之親兵五百人刼德倫置樓上有効節軍校張彦者自帥其黨拔白刃止剽掠【校戶教翻剽匹妙翻】夏四月帝遣供奉官扈異撫諭魏軍許張彦以刺史彦請復相澶衛三州如舊制【請罷昭德軍復以相澶衛三州隸天雄如舊制】異還言張彦易與【還從宣翻又如字易以䜴翻】但遣劉鄩加兵立當傳首帝由是不許但以優詔答之使者再返彥裂詔書抵於地戟手南向詬朝廷【左傳公戟其手杜預注曰抵徙屈肘如戟形陸德明曰抵音紙鄭玄曰人挾弓矢戟其肘孔穎達正義曰謂射者左手弣弓而右手彎之則戟其手】謂德倫曰天子愚暗聽人穿鼻【諭之以牛為人穿鼻旋轉前却一聽命於人以鼻為所制也】今我兵甲雖彊苟無外援不能獨立宜投款於晉【款誠也】遂逼德倫以書求援於晉李繼徽假子保衡殺李彦魯 【考異曰蜀書劉知俊傳保衡作彦康今從薛】<br />
<br />
  【史】自稱靜難留後【難乃旦翻】舉邠寧二州來附【叛岐附梁】詔以保衡為感化節度使以河陽留後霍彦威為靜難節度使吳徐温以其子牙内都指揮使知訓為淮南行軍副<br />
<br />
  使内外馬步諸軍副使【為徐知訓以驕横不終張本】 晉王得賀德倫書命馬步副總管李存審自趙州進據臨清五月存審至臨清劉鄩屯洹水【臨清在魏州北洹水在魏州西】賀德倫復遣使告急于晉【復扶又翻】晉王引大軍自黄澤嶺東下【魏收志樂平郡遼陽縣有黄澤嶺隋改遼陽為遼山縣唐帶遼州】與存審會於臨清猶疑魏人之詐按兵不進德倫遣判官司空頲犒軍【頲它鼎翻犒苦到翻】密言於晉王曰除亂當除根因言張彦凶狡之狀勸晉王先除之則無虞矣王默然【巳諭其意而不形於言慮有窺聽而洩軍機也】頲貝州人也晉王進屯永濟【永濟縣在魏州北數十里】張彦選銀槍効節五百人皆執兵自衛詣永濟謁見王登驛樓語之曰【語牛倨翻】汝陵脅主帥殘虐百姓【帥所類翻】數日中迎馬訴寃者百餘輩我今舉兵而來以安百姓非貪人土地汝雖有功於我不得不誅以謝魏人遂斬彦及其黨七人餘衆股栗王召諭之曰罪止八人餘無所問自今當竭力為吾爪牙衆皆拜伏呼萬歲明日王緩帶輕裘而進令張彦之卒擐甲執兵翼馬而從【擐音宦從才用翻翼者翼馬左右而從行】仍以為帳前銀槍都【晉王遂以銀槍効節軍取梁而亦以銀槍効節軍取禍】衆心由是大服劉鄩聞晉軍至選兵萬餘人自洹水趣魏縣【趣七喻翻】晉王留李存審屯臨清遣史建瑭屯魏縣以拒之【九域志魏縣在魏州西三十五里】王自引親軍至魏縣與鄩夾河為營【河漳河也漳河過魏縣亦謂之魏河】帝聞魏博叛大悔懼遣天平節度使牛存節將兵屯楊劉 【考異曰牛存節傳楊劉作陽留或陽劉今從唐裴度傳及薛史諸人傳】為鄩聲援會存節病卒以匡國節度使王檀代之 岐王遣彰義節度使劉知俊圍邠州霍彦威固守拒之【先是李保衡叛岐附梁梁以霍彦威代鎮邠州】 六月庚寅朔賀德倫帥將吏請晉王入府城慰勞旣入德倫上印節【帥讀曰率下同勞力到翻上時掌翻印天雄軍府印節天雄旌節】請王兼領天雄軍王固辭曰比聞汴寇侵逼貴道【比毗至翻】故親董師徒遠來相救又聞城中新罹塗炭故暫入存撫明公不垂鑒信乃以印節見推誠非素懷德倫再拜曰今寇敵密邇【謂劉鄩之兵逼魏州也】軍城新有大變人心未安德倫心腹紀綱【左傳秦伯納三千人以衛晉文公實紀綱之僕】為張彥所殺殆盡形孤勢弱安能統衆一旦生事恐負大恩王乃受之德倫帥將吏拜賀王承制以德倫為大同節度使遣之官德倫至晉陽張承業留之【大同軍北臨極邊賀德倫新附張承業不欲使其有城有兵故留之為承業後殺德倫張本】時銀槍効節都在魏城猶驕横【魏城魏州城横戶孟翻】晉王下令自今有朋黨流言及暴掠百姓者殺無赦以沁州刺史李存進為天雄都巡按使【沁午鴆翻 考異曰莊宗實錄云為軍城使存進傳云都部署莊宗列傳及薛史存進傳皆云天雄軍都巡按使今從之】有訛言搖衆及強取人一錢已上者存進皆梟首磔尸於市【梟堅堯翻磔陟格翻】旬日城中肅然無敢喧譁者存進本姓孫名重進振武人也晉王多出征討天雄軍府事皆委判官司空頲決之頲恃才挾勢睚眦必報【睚五戒翻頲他鼎翻眦士戒翻】納賄驕侈頲有從子在河南【從才用翻此河南謂大河之南也】頲密使人召之都虞候張裕執其使者以白王王責頲曰自吾得魏博庶事悉以委公公何得見欺如是獨不可先相示邪揖令歸第是日族誅於軍門【兩敵對壘而越境通私書誅之宜也族之過也】以判官王正言代之正言鄆州人也魏州孔目吏孔謙勤敏多計數善治簿書晉王以為支度務使【唐節鎮多兼支度等使至其末世藩鎮署官有為支計官者有為支度務使者治直之翻】謙能曲事權要由是寵任彌固【為孔謙以掊克亂唐張本】魏州新亂之後府庫空竭民間疲弊而聚三鎮之兵戰於河上殆將十年【三鎮并魏鎮也】供億軍須未嘗有闕謙之力也然急徵重斂【斂力贍翻】使六州愁苦歸怨於王亦其所為也【史卒言之】張彦之以魏博歸晉也貝州刺史張源德不從北結滄德【乾化三年楊師厚劉守奇北略滄德遂附於梁】南連劉鄩以拒晉數斷鎮定糧道或說晉王【數所角翻斷都管翻說式芮翻】請先發兵萬人取源德然後東兼滄景則海隅之地皆為我有晉王曰不然貝州城堅兵多未易猝攻【易以豉翻】德州隸於滄州而無備若得而戍之則滄貝不得往來【九域志德州西南至貝州二百三十里東北至滄州亦二百三十里】二壘旣孤然後可取【二壘謂滄與貝也】乃遣騎兵五百晝夜兼行襲德州刺史不意晉兵至踰城走遂克之以遼州守捉將馮通為刺史秋七月晉人夜襲澶州陷之【九域志魏州南至澶州一百四十里按九域志之澶州乃漢乾祐元年所徙之澶州也宋白曰澶州本漢頓丘縣地在魏州南當兩河之驛路唐武德四年分魏州之觀城頓丘兩縣置澶州取古澶淵為名貞觀元年州廢大歷七年田承嗣又奏置漢乾祐元年移就德勝寨舊基頓丘縣隨州移於郭下此時澶州猶治頓丘舊州城今德清軍之頓丘鎮即其地】刺史王彦章在劉鄩營晉人獲其妻子待之甚厚遣間使誘彦章【間古莧翻誘音酉】彦章斬其使晉人盡滅其家晉王以魏州將李巖為澶州刺史 【考異曰莊宗實錄作李嚴今從薛史】晉王勞軍於魏縣因帥百餘騎循河而上覘劉鄩營【勞力到翻帥讀曰率下同上時掌翻覘丑廉翻又丑艷翻】會天隂晦鄩伏兵五千於河曲叢林間鼔譟而出圍王數重王躍馬大呼帥騎馳突所向披靡禆將夏魯奇等操短兵力戰【重直龍翻呼火故翻披普彼翻操七到翻】自午至申乃得出亡其七騎魯奇手殺百餘人傷夷遍體會李存審救兵至乃得免王顧謂從騎曰幾為虜嗤【用漢光武之言幾居依翻嗤丑之翻】皆曰適足使敵人見大王之英武耳魯奇青州人也王以是益愛之賜姓名曰李紹奇劉鄩以晉兵盡在魏州晉陽必虛欲以奇計襲取之乃濳引兵自黄澤西去晉人怪鄩軍數日不出寂無聲迹遣騎覘之城中無煙火但時見旗幟循堞往來【騎奇寄翻覘丑廉翻堞達恊翻】晉王曰吾聞劉鄩用兵一步百計此必詐也更使覘之乃縛芻為人執旗乘驢在城上耳得城中老弱者詰之云軍去已二日矣晉王曰劉鄩長於襲人【劉鄩取兖州克潼關皆以掩襲得之故云然然以智遇智則必有窮者若鄩之襲晉陽則智窮矣】短於決戰計彼行纔及山下【相魏之西皆連山】亟發騎兵追之會隂雨積旬黄澤道險堇泥深尺餘士卒援藤葛而進【堇泥黏土也深式禁翻援于元翻】皆腹疾足腫死者什二三晉將李嗣恩倍道先入晉陽城中知之勒兵為備鄩至樂平糗糧且盡【樂平拒晉陽二百五十里耳糗去久翻】又聞晉有備追兵在後衆懼將潰鄩諭之曰今去家千里深入敵境腹背有兵山谷高深如墜井中去將何之惟力戰庶幾可免不則以死報君親耳衆泣而止【幾居希翻不讀曰否】周德威聞鄩西上【上時掌翻】自幽州引千騎救晉陽至土門鄩已整衆下山自邢州陳宋口踰漳水而東屯於宗城【九域志宗城縣在魏州西北一百七十里】鄩軍往還馬死殆半時晉軍乏食鄩知臨清有蓄積欲據之以絶晉糧道【自宗城東行邪趣臨清數十里宋白曰臨清本漢清泉縣地後魏太和二十一年於此置臨清縣】德威急追鄩再宿至南宫【南宫縣在冀州西南六十二里東南趣臨清亦數十里】遣騎擒其斥候者數十人斷腕而縱之【斷音短腕烏貫翻】使言曰周侍中已據臨清矣 【考異曰薛史德威聞劉鄩東還急趨南宫知鄩軍在宗城遣十餘騎迫其營擒斥候者數十人皆剚刃其背縶而遣之旣至謂鄩曰周侍中已據宗城矣鄩軍大駭按剚刃於背其人豈能復活而言今從莊宗實錄及薛史莊宗紀又鄩見在宗城而云周侍中據宗城蓋臨清字誤耳】鄩軍大駭詰朝德威略鄩營而過入臨清鄩引軍趨貝州時晉王出師屯博州劉鄩軍堂邑【趨七喻翻九域志博州在魏州東一百八十里堂邑在博州西四十里宋白曰堂邑屬博州本漢清縣發干二縣地隋置堂邑因縣西北有漢堂邑故城以名縣】周德威攻之不克翌日鄩軍於莘縣【九域志莘縣在魏州東九十里劉鄩見晉軍在博州移軍而西漸逼魏州宋白曰莘本春秋之衛邑漢為陽平縣後周改陽平為清邑縣大業改清邑為莘縣因古地名也】晉軍踵之鄩治莘城塹而守之自莘及河築甬道以通饋餉【莘縣東距大河二十餘里度河而東南即鄆濮之境故築甬道屬河以通饋餉甬道夾築垣牆以防晉人之衝突抄截治直之翻】晉王營於莘西三十里煙火相望一日數戰晉王愛元行欽驍健從代州刺史李嗣源求之嗣源不得已獻之以為散員都部署【都部署之名始見於通鑑後遂為行軍總帥之稱薛史曰時有散指揮名為散員命行欽為都部署】賜姓名曰李紹榮紹榮嘗力戰深入劒中其面未解【中竹仲翻】高行周救之得免王復欲求行周重於發言密使人以官祿啗之【復扶又翻啗徒濫翻】行周辭曰代州養壯士亦為大王耳【為于偽翻】行周事代州亦猶事大王也 【考異曰周太祖實錄晉王密令人㗖之利祿行周辭曰總管用人亦為國家事總管猶事王也予家昆仲脱難再生承總管之厚恩安忍背之按明宗實錄此年猶為代州刺史天祐十八年始為副總管此言總管蓋周太祖實錄之誤】代州脱行周兄弟於死【事見上卷乾化三年】行周不忍負之乃止 絳州刺史尹皓攻晉之隰州八月又攻慈州皆不克【按九域志絳州西北至隰州五百一十四里隰州西南至慈州一百六十里】王檀與昭義留後賀瓌攻澶州拔之執李巖送東都【按歐史職方考梁無昭義軍參考賀壞傳蓋為宣義留後也昭當作宣先是晉襲取澶州以李巖守之】帝以楊師厚故將楊延直為澶州刺史使將兵萬人助劉鄩且招誘魏人【誘音酉】 晉王遣李存審將兵五千擊貝州張源德有卒三千每夕分出剽掠【剽匹妙翻】州民苦之請塹其城以安耕耘存審乃發八縣丁夫塹而圍之【貝州管清河清陽武城經城臨清漳南歷亭夏津八縣】劉鄩在莘久饋運不給晉人數抵其寨下挑戰【數所角翻桃徒了翻】鄩不出晉人乃攻絶其甬道以千餘斧斬寨木梁人驚擾而出因俘獲而還帝以詔書讓鄩老師費糧失亡多不速戰鄩奏臣比欲以奇兵擣其腹心【比毗志翻近也擣其腹心謂欲襲取晉陽也】還取鎮定期以旬時再清河朔【十日謂之旬時】無何天未厭亂淫雨積旬糧竭士病又欲據臨清斷其饋餉而周楊五奄至馳突如神【斷音短周德威小字楊五】臣今退保莘縣享士訓兵以俟進取觀其兵數甚多便習騎射誠為勍敵未易輕也【勍渠京翻易以䜴翻】苟有隙可乘臣豈敢偷安養寇帝復問鄩決勝之策【復扶又翻下同】鄩曰臣今無策惟願人給十斛糧賊可破矣【劉鄩欲以持久制晉】帝怒責鄩曰將軍蓄米欲破賊邪欲療飢邪乃遣中使往督戰鄩集諸將問曰主上深居禁中不知軍旅徒與少年新進輩謀之夫兵在臨機制變不可預度【少詩照翻度徒洛翻】今敵尚彊與戰必不利奈何諸將皆曰勝負當一決曠日何待鄩默然不悦退謂所親曰主暗臣諛將驕卒惰吾未知死所矣【劉鄩量敵慮勝未為失計特掣其肘使不得遂其本謀耳】他日復集諸將於軍門人置河水一器於前令飲之衆莫之測鄩諭之曰一器猶難滔滔之河可勝盡乎【勝音升】衆失色後數日鄩將萬餘人薄鎮定營鎮定人驚擾晉李存審以騎兵二千横擊之李建及以銀槍千人助之鄩大敗奔還晉人逐之及寨下俘斬千計【劉鄩欲掩鎮定之不備而為晉人所敗鄩之計又窮矣】 劉巖逆婦于楚楚王殷遣永順節度使存送之 乙未蜀主以兼中書令王宗綰為北路行營都制置使兼中書令王宗播為招討使攻秦州兼中書令王宗瑤為東北面招討使同平章事王宗翰為副使攻鳳州【秦鳳二州時皆屬岐】 庚戌吳以鎮海節度使徐温為管内水陸馬步諸軍都指揮使兩浙都招討使守侍中齊國公鎮潤州以昇潤常宣歙池六州為巡屬軍國庶務參決如故【史言徐温外據重鎮内制吳國之權】留徐知訓居廣陵秉政【此速徐知訓之死也】 初帝為均王娶河陽節度使張歸霸女為妃即位欲立為后后以帝未南郊固辭【古人相傳以為郊見上帝然後代天子民】九月壬午妃疾甚冊為德妃是夕卒康王友敬目重瞳子【重直龍翻瞳音童】自謂當為天子遂謀作亂冬十月辛亥夜德妃將出葬友敬使腹心數人匿於寢殿帝覺之跣足踰垣而出召宿衛兵索殿中【索山客翻】得而手刃之壬子捕友敬誅之帝由是疎忌宗室專任趙巖及德妃兄弟漢鼎漢傑從兄弟漢倫漢融咸居近職參預謀議每出兵必使之監護【監古衘翻】巖等依勢弄權賣官鬻獄離間舊將相【間古莧翻】敬翔李振雖為執政所言多不用振每稱疾不預事以避趙張之族政事日紊【紊音問】以至於亡【史言梁有自亡之由非晉能亡之也】 劉鄩遣卒詐降於晉謀賂膳夫以毒晉王事泄晉王殺之并其黨五人 十一月己未夜蜀宫火自得成都以來寶貨貯於百尺樓悉為煨燼【貯丁呂翻煨烏回翻】諸軍都指揮使兼中書令宗侃等帥衛兵欲入救火蜀主閉門不内【恐有乘救火為變者史言蜀主之猜防】庚申旦火猶未熄蜀主出義興門見羣臣【以安衆心】命有司聚太廟神主分巡都城言訖復入宫閉門【史未熄未敢弛備復扶又翻】將相皆獻帷幕飲食 壬戌蜀大赦 乙丑改元【此書梁改元貞明也 考異曰吳越備史云正月壬辰朔改元大赦今從薛史末帝紀】 己巳蜀王宗翰引兵出青泥嶺克固鎮【九域志鳳州河池縣有固鎮】與秦州將郭守謙戰於泥陽川【九域志成州栗亭縣有泥陽鎮】蜀兵敗退保鹿臺山【今成州東十里有鹿玉山】辛未王宗綰等敗秦州兵於金沙谷【敗補邁翻】擒其將李彦巢等乘勝趣秦州【趣七喻翻】興州刺史王宗鐸克階州降其刺史李彦安甲戌王宗綰克成州擒其刺史李彦德蜀軍至上染坊秦州節度使李繼崇遣其子彦秀奉牌印迎降宗絳入秦州【九域志秦州東南至鳳州三百二十里西南至成州二百六十五里成州西南至階州二百五十里宗絳當作宗綰】表排陳使王宗儔為留後【陳讀曰陣】劉知俊攻霍彦威於邠州半歲不克【是年五月劉知俊攻邠州】聞秦州降蜀知俊妻子皆遷成都知俊解圍還鳳翔終懼及禍夜帥親兵七十人斬關而出庚辰奔于蜀軍【帥讀曰率為劉知俊為蜀所殺張本 考異曰十國紀年知俊奔秦州庚戌來降按上有甲戌下有癸未必庚辰也】王宗綰自河池兩當進兵會王宗瑤攻鳳州癸未克之【蜀遂有秦鳳成三州之地宋白曰河池縣漢屬武都華陽國志河池一名仇池按仇池山在成州界今河池縣屬鳳州去縣稍遠今縣所處謂之河池水故以名縣兩當漢故道縣水經云兩當水出陳倉縣之大散嶺西南流入故道川又河池縣有兩當水西北自成州界入東南流入故道水縣取水為名或曰縣西界有兩山相當故名九域志河池在鳳州西一百五十五里兩當在鳳州西八十五里】 岐義勝節度使同平章事李彦韜知岐王衰弱十二月舉耀鼎二州來降【岐置義勝軍以授温韜見二百六十八卷太祖乾化元年】彦韜即温韜也乙未詔改耀州為崇州鼎州為裕州義勝軍為靜勝軍復彦韜姓温氏名昭圖官任如故 丁未蜀大赦改明年元曰通正置武興軍於鳳州割文興二州隸之以前利州團練使王宗魯為節度使 是歲清海建武節度使兼中書令劉巖【時以邕州為建武軍】以吳越王鏐為國王而已獨為南平王【南平王郡王也】表求封南越王及加都統帝不許巖謂僚屬曰今中國紛紛孰為天子安能梯航萬里【梯航謂梯山航海】遠事偽庭乎自是貢使遂絶【使疏吏翻】<br />
<br />
  二年春正月宣武節度使守中書令廣德靖王全昱卒【廣國名德靖謚也全昱帝之伯父】 帝聞前河南府參軍李愚學行【行下孟翻】召為左拾遺充崇政院直學士衡王友諒貴重李振等見皆拜之愚獨長揖帝聞而讓之曰衡王於朕兄也朕猶拜之卿長揖可乎對曰陛下以家人禮見衡王拜之宜也振等陛下家臣臣於王無素【謂先無過從之雅】不敢妄有所屈久之竟以抗直罷為鄧州觀察判官 蜀主以李繼崇為武泰節度使兼中書令隴西王 二月辛丑夜吳宿衛將馬謙李球刼吳王登樓發庫兵討徐知訓知訓將出走嚴可求曰軍城有變公先弃衆自去衆將何依知訓乃止衆猶疑懼可求闔戶而寢鼾息聞於外【鼾下旦翻鼻息也聞音問】府中稍安壬寅謙等陳于天興門外【楊行密以揚州牙城南門為天興門】諸道副都統朱瑾自潤州至【至自徐温所】視之曰不足畏也返顧外衆舉手大呼【呼火故翻】亂兵皆潰【史言吳兵畏服朱瑾】擒謙球斬之 帝屢趣劉鄩戰【趣讀曰促】鄩閉壁不出晉王乃留副總管李存審守營【守莘西之營也】自勞軍於貝州【勞力到翻勞圍張源德之軍也】聲言歸晉陽鄩聞之奏請襲魏州帝報曰今掃境内以屬將軍【屬之欲翻】社稷存亡繫兹一舉將軍勉之鄩令澶州刺史楊延直引兵萬人會於魏州延直夜半至城南城中選壯士五百濳出擊之延直不為備潰亂而走詰旦鄩自莘縣悉衆至城東與延直餘衆合李存審以營中兵踵其後李嗣源以城中兵出戰晉王亦自貝州至與嗣源當其前鄩見之驚曰晉王邪引兵稍却晉王躡之【躡尼輒翻】至故元城西【隋元城縣治古殷城唐貞觀十七年併入貴鄉聖歷二年又分貴鄉莘縣置元城縣治王莾城開元十三年移元城治魏州郭下故有故元城古殷城在朝城東北十二里】與李存審遇晉王為方陳於西北存審為方陳於東南鄩為圓陳於其中間【陳讀曰陣】四面受敵合戰良久梁兵大敗鄩引數十騎突圍走梁步卒凡七萬晉兵環而擊之敗卒登木木為之折【環音宦為于偽翻折而設翻】追至河上殺溺殆盡鄩收散卒自黎陽度河保滑州匡國節度使王檀密疏請發關西兵襲晉陽【去年五月王檀代牛存節屯河上】帝從之發河中陜同華諸鎮兵合三萬出隂地關奄至晉陽城下晝夜急攻城中無備發諸司丁匠及驅市人乘城拒守城幾陷者數四【幾居依翻】張承業大懼代北故將安金全退居太原【安金全從晉王克用起於代北故云故將】往見承業曰晉陽根本之地若失之則大事去矣僕雖老病憂兼家國【言晉陽若陷則國破家亡】請以庫甲見授為公擊之【為于偽翻】承業即與之金全帥其子弟及退將之家得數百人【帥讀曰率將即亮翻】夜出北門擊梁兵於羊馬城内梁兵大驚引却昭義節度使李嗣昭聞晉陽有寇遣牙將石君立將五百騎救之君立朝發上黨夕至晉陽【按九域志上黨至晉陽五百餘里輕騎疾馳朝發夕至何其速也】梁兵扼汾河橋【汾橋在晉陽城東南汾水上】君立擊破之徑至城下大呼曰昭義侍中大軍至矣【呼火故翻李嗣昭鎮昭義官侍中故稱之】遂入城夜與安金全等分出諸門擊梁兵梁兵死傷什二三詰朝王檀引兵大掠而還【詰去吉翻還從宣翻又如字】晉王性矜伐以策非已出故金全等賞皆不行【虞書曰汝惟不矜天下莫與汝爭能汝惟不伐天下莫與汝爭功晉王矜伐而有功者不賞此其所以能取天下而不能守天下也】梁兵之在晉陽城下也大同節度使賀德倫部兵多逃入梁軍張承業恐其為變收德倫斬之【張承業之權略烏可以宦者待之哉】帝聞劉鄩敗又聞王檀無功歎曰吾事去矣 三月乙卯朔晉王攻衛州壬戌刺史米昭降之又攻惠州刺史靳紹走擒斬之復以惠州為磁州【唐天祐三年以磁慈聲相近改磁州為惠州是時政在朱氏晉旣取之因復舊州名靳居焮翻】晉王還魏州 上屢召劉鄩不至己巳即以鄩為宣義節度使【劉鄩旣喪師懼罪不敢入朝梁亦懼其反側就以滑帥命之為明年鄩入朝左遷張本】使將兵屯黎陽 夏四月晉人拔洺州以魏州都巡檢使袁建豐為洺州刺史劉鄩旣敗河南大恐鄩復不應召【復扶又翻】由是將卒皆搖心帝遣捉生都指揮使李霸帥所部千人戍楊劉癸卯出宋門【宋門大梁城東面南來第二門梁改名觀化門而時人不改其舊呼曰宋門晉天福三年改仁和門】其夕復自水門入大譟縱火剽掠【剽匹妙翻】攻建國門【建國門大梁宫城正南門太祖所起也宋白曰大梁皇城南為建國門】帝登樓拒戰【樓謂建國門樓也】龍驤四軍都指揮使杜晏球【按歐史晏球本洛陽王氏子少遇亂為盜所掠汴州富人杜氏得之養以為子冒姓杜氏後歸唐賜姓名曰李紹虔尋復本姓名曰王晏球】以五百騎屯毬場賊以油沃幕長木揭之【揭其列翻舉也】欲焚樓勢甚危晏球於門隙窺之見賊無甲胄乃出騎擊之決力死戰俄而賊潰走帝見騎兵擊賊呼曰非吾龍驤之士乎誰為亂首晏球曰亂者惟李霸一都餘軍不動陛下但帥控鶴守宫城遲明臣必破之【帥讀曰率遲直利翻待也】旣而晏球討亂者闔營皆族之以功除單州刺史【唐末以太祖生於碭山改單州為輝州是時復以輝州為單州單音善】 五月吳越王鏐遣浙西安撫判官皮光業自建汀虔郴潭岳荆南道入貢【吳越界西南盡衢州按九域志自衢州界西南至建州四百四十五里自建州西至汀州九百三十里自汀州西至虔州五百五十里自虔州西至郴州六百六十里自郴州東北至潭州四百九十八里自潭州東北至岳州三百八十五里自岳州西北至荆南四百三十里】光業日休之子也【皮日休見二百五十四卷唐僖宗廣明元年郴丑林翻】 六月晉人攻邢州保義節度使閻寶拒守帝遣捉生都指揮使張温將兵五百救之温以其衆降晉 秋七月甲寅朔晉王至魏州 上嘉吳越王鏐貢獻之勤【以其取道回遠數千里至大梁也】壬戌加鏐諸道兵馬元帥朝議多言鏐之入貢利於市易【市易者以所有易所無相與為市也朝直遙翻】不宜過以名器假之翰林學士竇夢徵執麻以泣坐貶蓬萊尉【蓬萊本漢黄縣唐神龍三年更名帶登州】夢徵棣州人也 甲子吳潤州牙將周郊作亂入府殺大將秦師權等大將陳祐等討斬之 八月丁酉以太子少保致仕趙光逢為司空兼門下侍郎同平章事 丙午蜀主以王宗綰為東北面都招討集王宗翰嘉王宗壽為第一第二招討將兵十萬出鳳州以王宗播為西北面都招討武信軍節度使劉知俊天雄節度使王宗儔【蜀天雄節度使鎮秦州】匡國軍使唐文裔為第一第二第三招討將兵十二萬出秦州以伐岐【出鳳州之兵指寶雞以攻鳳翔出秦州之兵指隴州】 晉王自將攻邢州昭德節度使張筠弃相州走晉人復以相州隸天雄軍【去年梁分相州為昭德軍相息亮翻】以李嗣源為刺史 【考異曰劉恕廣本云筠奔東都授左衛上將軍莊宗實錄命李存審入城招撫除昭德軍額仍舊隸魏州徙洺州刺史袁建豐為相州刺史按上四月筠已遣人納款于晉此復云走者蓋始者文降今為晉兵所迫故走耳筠旣降晉今還猶得將軍者蓋濳通款於晉梁朝不知耳明宗實錄云八月張筠走移帝為相州刺史九月為安國節度使而莊宗實錄云袁建豐為相州刺史按明宗實錄建豐傳云戰胡柳改建豐猶為相州乃是天祐十五年十二月蓋明宗初為相州移邢州後方除建豐莊宗錄誤書在張筠走下耳】晉王遣人告閻寶以相州已拔又遣張温帥援兵至城下諭之寶舉城降【告之以相州已拔則彼知邢州之勢孤示之以張温已降則彼知援兵之望絶閻寶於是不能守矣帥讀曰率下同】晉王以寶為東南面招討使領天平節度使同平章事【天平時屬梁晉命閻寶遙領】以李存審為安國節度使鎮邢州【邢州梁保義軍旣入于晉自此遂改為安國軍 考異曰王溥五代會要薛史地理志樂史寰宇記皆云梁建保義軍唐同光元年改為安國軍而莊宗明宗實錄列傳薛史存審傳皆云此年授安國節度使恐是纔屬晉即改軍額會要等書誤云同光元年】 契丹主安巴堅帥諸部兵三十萬號百萬自麟勝攻晉蔚州陷之虜振武節度使李嗣本【契丹攻蔚州自麟勝出詭道以掩晉不備也按麟勝至蔚州中間懸隔雲朔蔚州恐當作朔州 考異曰開元中振武軍在朔州西北三百五十里單于都護府城内隸朔方節度使乾元元年置振武節度使領鎮北大都護麟勝二州後唐振武節度使亦帶安北都護麟勝等州觀察等使石晉以後皆帶朔州刺史據此乃治蔚州不知遷徙年月】遣使以木書求貨於大同防禦使李存璋存璋斬其使契丹進攻雲州存璋悉力拒之【雲州即大同軍】 九月晉王還晉陽王性仁孝故雖經營河北而數還晉陽省曹夫人歲再三焉【數所角翻省悉景翻曹夫人實生晉王晉王事生母者重事嫡母者輕異日太后太妃尊號倒置皆根於心而發於事者】 晉人以兵逼滄州順化節度使戴思遠弃城奔東都【河朔盡歸於晉滄州孤絶戴思遠不能守】滄州將毛璋據城降晉晉王命李嗣源將兵鎮撫之嗣源遣璋詣晉陽晉王徙李存審為横海節度使鎮滄州【滄德自此屬晉復改順化為横海從唐舊也】以嗣源為安國節度使嗣源以安重誨為中門使【晉王封内凡節鎮皆有中門使其任即天朝樞密使也】委以心腹重誨亦為嗣源盡力重誨應州胡人也【為于偽翻為安重誨為嗣源佐命張本薛史曰安重誨其先本北部酋豪父福遷於河東將救兖鄆而沒】 晉王自將兵救雲州行至代州契丹聞之引去王亦還以李存璋為大同節度使 晉人圍貝州踰年【去年八月晉圍貝州】張源德聞河北諸州皆為晉有欲降謀於其衆衆以窮而後降恐不免死不從共殺源德嬰城固守城中食盡噉人為糧乃謂晉將曰出降懼死請擐甲執兵而降【噉徒濫翻擐音宦】事定而釋之晉將許之其衆三千出降旣釋甲圍而殺之盡殪【殪壹計翻 考異曰莊宗實錄賊將張源德固守貝州旣聞河北皆平而有翻然之志詢謀於衆羣賊皆河南人懼其歸罪不從因殺源德噉人為糧固守其城王師歷年攻圍賊旣食竭呼我大將曰今欲請罪懼晉王不我赦我將衿甲持兵而見已即解之如何報曰無便於此者賊衆三千衿甲出降我將甘言諭之俱釋兵解甲旣而四面陳兵皆殺之歐陽史死事傳曰晉王入魏河北六鎮數十州之地皆歸晉獨貝一州圍之踰年不可下城中食且盡貝人勸源德出降源德不從遂見殺按源德若以不降而死其衆當即降於晉豈得猶拒守與晉邀約而後出哉明是衆懼死不降耳今從莊宗實錄 余謂若如通鑑之去取則張源德非一心守死者不得與於死事傳】晉王以毛璋為貝州刺史於是河北皆入於晉惟黎陽為梁守【黎陽臨河梁兵聲援猶接又劉鄩守之所以能自固為于偽翻】 晉王如魏州 吳光州將王言殺刺史載肇【載恐當作戴】吳王遣楚州團練使李厚討之廬州觀察使張崇不俟命引兵趣光州【趣七喻翻】言弃城走以李厚權知光州崇慎縣人也 庚申蜀新宫成在舊宫之北 天平節度使兼中書令琅邪忠毅王王檀多募羣盜置帳下為親兵己卯盜乘檀無備突入府殺檀節度副使裴彦帥府兵討誅之軍府由是獲安【帥讀曰率】 冬十月甲申蜀王宗綰等出大散關大破岐兵俘斬萬計遂取寶雞己丑王宗播等出故關至隴州【故關大震故關】庚寅保勝節度使兼侍中李繼岌畏岐王猜忌【岐置保勝軍於隴州】帥其衆二萬【帥讀曰率】弃隴州奔于蜀軍蜀兵進攻隴州以繼岌為西北面行營第四招討劉知俊會王宗綰等圍鳳翔岐兵不出會大雪蜀主召軍還【還從宣翻又如字】復李繼岌姓名曰桑弘志弘志黎陽人也 丁酉以禮部侍郎鄭珏為中書侍郎同平章事珏綮之姪孫也【鄭綮見二百五十九卷唐昭宗乾寧元年 考異曰薛史梁末帝紀無珏初拜相年月此年十月丁酉以中書侍郎平章事鄭珏兼刑部尚書平章事至貞明四年四月己酉又云以中書侍郎平章事鄭珏兼刑部尚書疑貞明二年拜相四年轉刑部尚書也本傳云累遷禮部侍郎貞明中拜平章事唐餘錄均帝紀貞明二年十月丁酉禮部侍郎鄭珏為中書侍郎平章事今從之又高若拙後史補云珏應一十九舉方捷姓名為第十九人第行亦同自登第凡十九年為宰相今按珏光化三年及第自光化三年至此年纔十七年矣又不可合】 己亥蜀大赦 晉王遣使如吳會兵以擊梁十一月吳以行軍副使徐知訓為淮北行營都招討使及朱瑾等將兵趣宋亳與晉相應【趣七喻翻】旣渡淮移檄州縣進圍潁州 十二月戊申蜀大赦改明年元曰天漢國號大漢 楚王殷聞晉王平河北遣使通好【好呼到翻】晉王亦遣使報之 是歲慶州叛附于岐【慶州本岐地也蓋因去年李保衡以邠寧附梁遂為梁有】岐將李繼陟據之詔以左龍虎統軍賀瓌為西面行營馬步都指揮使將兵討之破岐兵下寧衍二州【衍州岐李茂貞置在寧慶之間宋廢衍州為定平鎮屬邠州 考異曰薛史賀瓌傳貞明二年慶州叛為李繼陟所據帝命左龍虎統軍賀瓌為西面行營馬步軍都指揮使兼諸軍都虞候與張筠破涇鳳之衆三萬下寧衍二州此非小事而末帝紀李茂貞傳皆無惟瓌傳有之今以為據】河東監軍張承業旣貴用事其姪瓘等五人自同州<br />
<br />
  往依之晉王以承業故皆擢用之承業治家甚嚴有姪為盜殺販牛者承業立斬之王亟使救之已不及王以瓘為麟州刺史承業謂瓘曰汝本車度一民與劉開道為賊【劉開道必指劉知俊也知俊為梁開道指揮使又嘗鎮同州車尺遮翻】慣為不法今若不悛【慣古患翻悛丑緣翻】死無日矣由此瓘所至不敢貪暴 吳越牙内先鋒都指揮使錢傳珦逆婦於閩自是閩與吳越通好【珦虛亮翻好呼到翻】 閩鑄鈆錢與銅錢並行 初燕人苦劉守光殘虐軍士多歸於契丹及守光被圍於幽州【事見上卷】其北邊士民多為契丹所掠契丹日益彊大契丹王安巴堅自稱皇帝國人謂之天皇王以妻舒嚕氏為皇后置百官至是改元神冊 【考異曰紀年通譜云舊史不記安巴堅建元事今契丹中有歷日通紀百二十年臣景祐三年冬北使幽州得其歷因閲年次以乙亥為首次年始著神策之元其後復有天贊按五代契丹傳自耶律德光乃記天顯之名疑當時未得其傳不然虜人恥安巴堅無號追為之耳保機虜中又號天皇王虜庭雜記太祖一舉併吞奚國仍立奚人依舊為奚王命契丹監督兵甲又滅勃海虜其王大諲譔立長子為勃海東丹王號人皇王自號天皇王始立年號曰天贊國稱大遼於所居大部落置樓謂之西樓今謂之上京又於其南木葉山置樓謂之南樓又於其東千里置樓謂之東樓又於其北三百里置樓謂之北樓大祖四季常遊獵於四樓之間又曰安巴堅變家為國之後始以王族號為横帳姓錫里錫里以漢語誤譯謂之耶律氏賜后族姓曰蕭氏王族惟與后族同昏其諸部若不奉北主之命不得與二部落通昏歐陽史曰安巴堅用其妻舒嚕策使人告諸部大人曰我有鹽池諸部所食然諸部知食鹽之利而不知鹽有主人可乎當來犒我諸部以為然共以酒會鹽池安巴堅伏兵其旁酒酣伏發盡殺諸部大人遂立不復代安巴堅稱皇帝前史不見年月莊宗列傳契丹傳在莊宗即帝位李存審守范陽後漢高祖實錄唐餘錄皆云安巴堅設策併諸族遂稱帝在乾寧中劉仁恭鎮幽州前薛史在莊宗天祐末按紀元通譜安巴堅神策元年歲在丙子乃莊宗天祐十三年梁貞明二年似不在天祐末及莊宗即位後編遺錄開平二年五月太祖賜安巴堅記事猶呼之為卿及言臣事我朝望國家降使冊立必未稱帝安得在劉仁恭鎮幽州前唐餘錄全取漢高祖實錄契丹事作傳最為差錯不知其稱帝實在何年今因其改年號置於此】舒嚕后勇決多權變安巴堅行兵御衆舒嚕后常預其謀安巴堅嘗度磧擊党項【党項在磧西磧七迹翻党底朗翻】留舒嚕后守其帳黄頭錫伯二室韋乘虛合兵掠之【黄頭室韋彊部也臭泊室韋以所居地名其部】舒嚕后知之勒兵以待其至奮擊大破之由是名震諸夷舒嚕后有母有姑皆踞榻受其拜曰吾惟拜天不拜人也晉王方經營河北欲結契丹為援常以叔父事安巴堅以叔母事舒嚕后【以晉王克用與安巴堅結為兄弟也】劉守光末年衰困遣參軍韓延徽求援於契丹 【考異曰漢高祖實錄延徽傳云天祐中連帥劉守光攻中山不利欲結北戎遣延徽將命入虜劉恕以為劉守光據幽州後未嘗攻定州惟唐光化三年汴將張存敬拔瀛莫攻定州劉仁恭遣守光救定州為存敬所敗恐是此時仁恭方為幽帥非守光也按劉仁恭父子彊盛之時常陵暴契丹豈肯遣使與之相結乾化元年守光攻易定王處直求救於晉故晉王遣周德威伐之其遣延徽結契丹蓋在此時然事無顯據故但云衰困附於此】契丹主怒其不拜使牧馬於野延徽幽州人有智略頗知屬文【屬之欲翻】舒嚕后言於契丹主曰延徽能守節不屈此今之賢者奈何辱以牧圉宜禮而用之契丹主召延徽與語悅之遂以為謀主舉動訪焉延徽始教契丹建牙開府築城郭立市里以處漢人【處昌呂翻】使各有配偶墾藝荒田由是漢人各安生業逃亡者益少契丹威服諸國延徽有助焉頃之延徽逃奔晉陽晉王欲置之幕府掌書記王緘疾之延徽不自安求東歸省母【自晉陽歸幽州自西徂東也省悉景翻】過真定止於鄉人王德明家【王德明為趙王鎔養子即燕人張文禮也】德明問所之延徽曰今河北皆為晉有當復詣契丹耳德明曰叛而復往得無取死乎【言旣叛契丹歸中國今復往詣契丹恐為所殺也復扶又翻下同】延徽曰彼自吾來如喪手目【喪息浪翻】今往詣之彼手目復完安肯害我旣省母遂復入契丹契丹主聞其至大喜如自天而下拊其背曰曏者何往延徽曰思母欲告歸恐不聽故私歸耳契丹主待之益厚及稱帝以延徽為相累遷至中書令【歐史四夷附錄曰□僄檅以延徽為相號政事令契丹謂之崇文相公】晉王遣使至契丹延徽寓書於晉王叙所以北去之意且曰非不戀英主非不思故鄉所以不留正懼王緘之讒耳因以老母為託且曰延徽在此契丹必不南牧【賈誼過秦論胡人不敢南下而牧馬】故終同光之世契丹不深入為寇延徽之力也【按莊宗之世契丹圍周德威救張文禮曷嘗不欲深入為寇哉晉之兵力方彊能折其鋒耳豈延徽之力邪】三年春正月詔宣武節度使袁象先救潁州旣至吳軍引還【去年十一月吳圍穎州】 二月甲申晉王攻黎陽劉鄩拒之數日不克而去 晉王之弟威塞軍防禦使存矩在新州【晉置威塞軍於新州後遂為節鎮新州領永興一縣薛居正曰唐莊宗同光二年七月昇新州為威塞軍節度使以媯儒武三州隸之】驕惰不治【治直之翻】侍婢預政晉王使募山北部落驍勇者及劉守光亡卒以益南討之軍又率其民出馬民或鬻十牛易一戰馬期會迫促邊人嗟怨存矩得五百騎自部送之以壽州刺史盧文進為裨將【壽州屬吳盧文進遥領刺史耳】行者皆憚遠役存矩復不存恤【復扶又翻下同】甲午至祁溝關小校宫彥璋與士卒謀曰聞晉王與梁人確鬭【確堅也凡戰者隨兵勢而為進退離合至於確鬭則兩敵相當用實力而鬬惟堅耐而用長技乃勝耳校戶教翻】騎兵死傷不少吾儕捐父母妻子為人客戰【儕士皆翻為于偽翻千里行役戰於異鄉是為客戰】千里送死而使長復不矜恤奈何衆曰殺使長【防禦使為一州之長故曰使長使疏吏翻長知兩翻】擁盧將軍還新州據城自守其如我何因執兵大譟趣傳舍【趣七喻翻傳株戀翻】詰朝存矩寢未起就殺之【詰去吉翻】文進不能制撫膺哭其尸曰奴輩旣害郎君使我何面復見晉王因為衆所擁還新州守將楊全章拒之又攻武州鴈門以北都知防禦兵馬使李嗣肱擊敗之【敗補邁翻】周德威亦遣兵追討文進帥其衆奔契丹【帥讀曰率】晉王聞存矩不道以致亂殺侍婢及幕僚數人 初幽州北七百里有渝關【渝關入營州界及平州石城縣界漢書音義渝音喻今讀如榆】下有渝水通海自關東北循海有道道狹處纔數尺旁皆亂山高峻不可越比至進牛口【比當作北】舊置八防禦軍募土兵守之【歐史曰渝關東臨海北有兎耳覆舟山山皆斗絶並海東北有路狹僅通車其旁地可耕植唐時置東硤石西硤石淥疇米磚長楊黄花紫蒙白狼城以阨之宋白曰渝關關城下有渝水入大海其關東臨海北有兎耳山覆舟山山皆斗峻山下尋海岸東北行狹處纔通一軌三面皆海北連陸關西亂山至進牛柵凡六口柵戍相接此所以天限戎狄者也】田租皆供軍食不入於薊【薊音計】幽州歲致繒纊以供戰士衣每歲早穫清野堅壁以待契丹契丹至輒閉壁不戰俟其去選驍勇據隘邀之【幽州盧龍節度治薊縣繒慈陵翻纊苦謗翻隘烏懈翻】契丹常失利走土兵皆自為田園力戰有功則賜勲加賞【勲勲級也】由是契丹不敢輕入寇及周德威為盧龍節度使恃勇不脩邊備遂失渝關之險契丹每芻牧於營平之間【金虜節要曰燕山之地易州西北乃金坡關昌平縣之西乃居庸關順州之地乃古北口景州之東北乃松亭關平州之東乃渝關渝關之東即金人來路也此數關皆天造地設以分蕃漢之限一夫守之可以當百本朝復燕之役若得諸關則燕山之境可保然關内之地平欒營三州自後唐陷於安巴堅改平州為遼興府以營欒二州隸之號為平州路至石晉之初耶律德光又得燕山檀順景薊涿易諸州建燕山為燕京以轄六郡號燕京路而與平州自成兩路海上議割地但云燕雲兩路而已初謂燕山路盡得關内之地殊不知燕山平州盡在關内而異路也破遼之後金人復得平州路據之故斡里雅布後由平州入寇乃當時議燕雲不明地理之故又金虜行程云灤州古無之唐末安巴堅攻陷平營劉守光據幽州暴虐民多亡入虜中乃築此城營州古柳城郡舜所築也乃殷之孤竹國漢唐遼西地其城外多大山高下皆石不產草木地當營室故以為名自營州東至渝關並無保障沃野千里北限大山重岡複嶺中有五關唯渝關居庸可以通餉饋松亭金坡古北口止通人馬不可行車其山之南則五穀百果良材美木無所不有出關未數里則地皆瘠鹵豈天設此以限華夷乎】德威又忌幽州舊將有名者往往殺之吳主遣使遺契丹主以猛火油曰攻城以此油然火焚樓櫓敵以水沃之火愈熾【南蕃志猛火油出占城國蠻人水戰用之以焚敵舟遺于季翻】契丹主大喜即選騎三萬欲攻幽州舒嚕后哂之曰【哂失忍翻】豈有試油而攻一國乎因指帳前樹謂契丹主曰此樹無皮可以生乎契丹主曰不可舒嚕后曰幽州城亦猶是矣吾但以三千騎伏其旁掠其四野使城中無食不過數年城自困矣何必如此躁動輕舉萬一不勝為中國笑吾部落亦解體矣契丹主乃止【婦人智識若此丈夫愧之多矣此特安巴堅因其能勝室韋從而張大之以威鄰敵耳就使能爾曷為不能止德光之南牧旣内虛其國又不能為根本之計而終有木葉山之囚乎】三月盧文進引契丹兵急攻新州刺史安金全不能守弃城走文進以其部將劉殷為刺史使守之晉王使周德威合河東鎮定之兵攻之旬日不克契丹主帥衆三十萬救之德威衆寡不敵大為契丹所敗【帥讀曰率敗補邁翻】奔歸 楚王殷遣其弟存攻吳上高俘獲而還【還從宣翻又如字】 契丹乘勝進圍幽州聲言有衆百萬氊車毳幕彌漫山澤【毳充芮翻獸毛縟細者為毳】盧文進教之攻城為地道晝夜四面俱進城中宂地然膏以邀之又為土山以臨城城中鎔銅以灑之日殺千計而攻之不止周德威遣間使詣晉王告急【間古莧翻】王方與梁相持河上欲分兵則兵少欲勿救恐失之謀於諸將獨李嗣源李存審閻寶勸王救之王喜曰昔太宗得一李靖猶擒頡利【事見一百九十二卷貞觀四年】今吾有猛將三人復何憂哉【褒而期之以作三臣之氣復扶又翻】存審寶以為虜無輜重【重直用翻】勢不能久俟其野無所掠食盡自還然後踵而擊之李嗣源曰周德威社稷之臣今幽州朝夕不保恐變生於中何暇待虜之衰臣請身為前鋒以赴之王曰公言是也即日命治兵【治直之翻下同】夏四月晉王命嗣源將兵先進軍于淶水【淶水縣屬易州淶音來宋白曰李嗣源時屯淶水扼祁溝諸關以伺賊勢】閻寶以鎮定之兵繼之 吳昇州刺史徐知誥治城市府舍甚盛五月徐温行部至昇州【吳以昇常宣歙池為徐温巡屬行下孟翻】愛其繁富潤州司馬陳彦謙勸温徙鎮海軍治所於昇州【鎮海軍本治潤州】温從之徙知誥為潤州團練使知誥求宣州温不許知誥不樂【樂音洛】宋齊丘密言於知誥曰三郎驕縱敗在朝夕潤州去廣陵隔一水耳此天授也知誥悦即之官三郎謂温長子知訓也【為知訓死知誥得權張本知訓第三】温以陳彦謙為鎮海節度判官温但舉大綱細務悉委彦謙江淮稱治【稱治者時人稱之耳治直吏翻】彦謙常州人也【為陳彦謙垂死請於徐温立巳子張本】 高季昌與孔勍脩好復通貢獻【高季昌為孔勍所敗事見上卷太祖乾化二年好呼到翻復扶又翻】<br />
<br />
  資治通鑑卷二百六十九<br />
<br />
<史部,編年類,資治通鑑>  <br>
   </div> 

<script src="/search/ajaxskft.js"> </script>
 <div class="clear"></div>
<br>
<br>
 <!-- a.d-->

 <!--
<div class="info_share">
</div> 
-->
 <!--info_share--></div>   <!-- end info_content-->
  </div> <!-- end l-->

<div class="r">   <!--r-->



<div class="sidebar"  style="margin-bottom:2px;">

 
<div class="sidebar_title">工具类大全</div>
<div class="sidebar_info">
<strong><a href="http://www.guoxuedashi.com/lsditu/" target="_blank">历史地图</a></strong>  
<a href="http://www.880114.com/" target="_blank">英语宝典</a>  
<a href="http://www.guoxuedashi.com/13jing/" target="_blank">十三经检索</a> 
<br><strong><a href="http://www.guoxuedashi.com/gjtsjc/" target="_blank">古今图书集成</a></strong> 
<a href="http://www.guoxuedashi.com/duilian/" target="_blank">对联大全</a> <strong><a href="http://www.guoxuedashi.com/xiangxingzi/" target="_blank">象形文字典</a></strong> 

<br><a href="http://www.guoxuedashi.com/zixing/yanbian/">字形演变</a>  <strong><a href="http://www.guoxuemi.com/hafo/" target="_blank">哈佛燕京中文善本特藏</a></strong>
<br><strong><a href="http://www.guoxuedashi.com/csfz/" target="_blank">丛书&方志检索器</a></strong> <a href="http://www.guoxuedashi.com/yqjyy/" target="_blank">一切经音义</a>  

<br><strong><a href="http://www.guoxuedashi.com/jiapu/" target="_blank">家谱族谱查询</a></strong>  <strong><a href="http://shufa.guoxuedashi.com/sfzitie/" target="_blank">书法字帖欣赏</a></strong> 
<br>

</div>
</div>


<div class="sidebar" style="margin-bottom:0px;">

<font style="font-size:22px;line-height:32px">QQ交流群9:489193090</font>


<div class="sidebar_title">手机APP 扫描或点击</div>
<div class="sidebar_info">
<table>
<tr>
	<td width=160><a href="http://m.guoxuedashi.com/app/" target="_blank"><img src="/img/gxds-sj.png" width="140"  border="0" alt="国学大师手机版"></a></td>
	<td>
<a href="http://www.guoxuedashi.com/download/" target="_blank">app软件下载专区</a><br>
<a href="http://www.guoxuedashi.com/download/gxds.php" target="_blank">《国学大师》下载</a><br>
<a href="http://www.guoxuedashi.com/download/kxzd.php" target="_blank">《汉字宝典》下载</a><br>
<a href="http://www.guoxuedashi.com/download/scqbd.php" target="_blank">《诗词曲宝典》下载</a><br>
<a href="http://www.guoxuedashi.com/SiKuQuanShu/skqs.php" target="_blank">《四库全书》下载</a><br>
</td>
</tr>
</table>

</div>
</div>


<div class="sidebar2">
<center>


</center>
</div>

<div class="sidebar"  style="margin-bottom:2px;">
<div class="sidebar_title">网站使用教程</div>
<div class="sidebar_info">
<a href="http://www.guoxuedashi.com/help/gjsearch.php" target="_blank">如何在国学大师网下载古籍?</a><br>
<a href="http://www.guoxuedashi.com/zidian/bujian/bjjc.php" target="_blank">如何使用部件查字法快速查字?</a><br>
<a href="http://www.guoxuedashi.com/search/sjc.php" target="_blank">如何在指定的书籍中全文检索?</a><br>
<a href="http://www.guoxuedashi.com/search/skjc.php" target="_blank">如何找到一句话在《四库全书》哪一页?</a><br>
</div>
</div>


<div class="sidebar">
<div class="sidebar_title">热门书籍</div>
<div class="sidebar_info">
<a href="/so.php?sokey=%E8%B5%84%E6%B2%BB%E9%80%9A%E9%89%B4&kt=1">资治通鉴</a> <a href="/24shi/"><strong>二十四史</strong></a>&nbsp; <a href="/a2694/">野史</a>&nbsp; <a href="/SiKuQuanShu/"><strong>四库全书</strong></a>&nbsp;<a href="http://www.guoxuedashi.com/SiKuQuanShu/fanti/">繁体</a>
<br><a href="/so.php?sokey=%E7%BA%A2%E6%A5%BC%E6%A2%A6&kt=1">红楼梦</a> <a href="/a/1858x/">三国演义</a> <a href="/a/1038k/">水浒传</a> <a href="/a/1046t/">西游记</a> <a href="/a/1914o/">封神演义</a>
<br>
<a href="http://www.guoxuedashi.com/so.php?sokeygx=%E4%B8%87%E6%9C%89%E6%96%87%E5%BA%93&submit=&kt=1">万有文库</a> <a href="/a/780t/">古文观止</a> <a href="/a/1024l/">文心雕龙</a> <a href="/a/1704n/">全唐诗</a> <a href="/a/1705h/">全宋词</a>
<br><a href="http://www.guoxuedashi.com/so.php?sokeygx=%E7%99%BE%E8%A1%B2%E6%9C%AC%E4%BA%8C%E5%8D%81%E5%9B%9B%E5%8F%B2&submit=&kt=1"><strong>百衲本二十四史</strong></a>  <a href="http://www.guoxuedashi.com/so.php?sokeygx=%E5%8F%A4%E4%BB%8A%E5%9B%BE%E4%B9%A6%E9%9B%86%E6%88%90&submit=&kt=1"><strong>古今图书集成</strong></a>
<br>

<a href="http://www.guoxuedashi.com/so.php?sokeygx=%E4%B8%9B%E4%B9%A6%E9%9B%86%E6%88%90&submit=&kt=1">丛书集成</a> 
<a href="http://www.guoxuedashi.com/so.php?sokeygx=%E5%9B%9B%E9%83%A8%E4%B8%9B%E5%88%8A&submit=&kt=1"><strong>四部丛刊</strong></a>  
<a href="http://www.guoxuedashi.com/so.php?sokeygx=%E8%AF%B4%E6%96%87%E8%A7%A3%E5%AD%97&submit=&kt=1">說文解字</a> <a href="http://www.guoxuedashi.com/so.php?sokeygx=%E5%85%A8%E4%B8%8A%E5%8F%A4&submit=&kt=1">三国六朝文</a>
<br><a href="http://www.guoxuedashi.com/so.php?sokeytm=%E6%97%A5%E6%9C%AC%E5%86%85%E9%98%81%E6%96%87%E5%BA%93&submit=&kt=1"><strong>日本内阁文库</strong></a> <a href="http://www.guoxuedashi.com/so.php?sokeytm=%E5%9B%BD%E5%9B%BE%E6%96%B9%E5%BF%97%E5%90%88%E9%9B%86&ka=100&submit=">国图方志合集</a> <a href="http://www.guoxuedashi.com/so.php?sokeytm=%E5%90%84%E5%9C%B0%E6%96%B9%E5%BF%97&submit=&kt=1"><strong>各地方志</strong></a>

</div>
</div>


<div class="sidebar2">
<center>

</center>
</div>
<div class="sidebar greenbar">
<div class="sidebar_title green">四库全书</div>
<div class="sidebar_info">

《四库全书》是中国古代最大的丛书,编撰于乾隆年间,由纪昀等360多位高官、学者编撰,3800多人抄写,费时十三年编成。丛书分经、史、子、集四部,故名四库。共有3500多种书,7.9万卷,3.6万册,约8亿字,基本上囊括了古代所有图书,故称“全书”。<a href="http://www.guoxuedashi.com/SiKuQuanShu/">详细>>
</a>

</div> 
</div>

</div>  <!--end r-->

</div>
<!-- 内容区END --> 

<!-- 页脚开始 -->
<div class="shh">

</div>

<div class="w1180" style="margin-top:8px;">
<center><script src="http://www.guoxuedashi.com/img/plus.php?id=3"></script></center>
</div>
<div class="w1180 foot">
<a href="/b/thanks.php">特别致谢</a> | <a href="javascript:window.external.AddFavorite(document.location.href,document.title);">收藏本站</a> | <a href="#">欢迎投稿</a> | <a href="http://www.guoxuedashi.com/forum/">意见建议</a> | <a href="http://www.guoxuemi.com/">国学迷</a> | <a href="http://www.shuowen.net/">说文网</a><script language="javascript" type="text/javascript" src="https://js.users.51.la/17753172.js"></script><br />
  Copyright &copy; 国学大师 古典图书集成 All Rights Reserved.<br>
  
  <span style="font-size:14px">免责声明:本站非营利性站点,以方便网友为主,仅供学习研究。<br>内容由热心网友提供和网上收集,不保留版权。若侵犯了您的权益,来信即刪。scp168@qq.com</span>
  <br />
ICP证:<a href="http://www.beian.miit.gov.cn/" target="_blank">鲁ICP备19060063号</a></div>
<!-- 页脚END --> 
<script src="http://www.guoxuedashi.com/img/plus.php?id=22"></script>
<script src="http://www.guoxuedashi.com/img/tongji.js"></script>

</body>
</html>
