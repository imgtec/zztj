<!DOCTYPE html PUBLIC "-//W3C//DTD XHTML 1.0 Transitional//EN" "http://www.w3.org/TR/xhtml1/DTD/xhtml1-transitional.dtd">
<html xmlns="http://www.w3.org/1999/xhtml">
<head>
<meta http-equiv="Content-Type" content="text/html; charset=utf-8" />
<meta http-equiv="X-UA-Compatible" content="IE=Edge,chrome=1">
<title>資治通鑒_133-資治通鑑卷一百三十二_133-資治通鑑卷一百三十二</title>
<meta name="Keywords" content="資治通鑒_133-資治通鑑卷一百三十二_133-資治通鑑卷一百三十二">
<meta name="Description" content="資治通鑒_133-資治通鑑卷一百三十二_133-資治通鑑卷一百三十二">
<meta http-equiv="Cache-Control" content="no-transform" />
<meta http-equiv="Cache-Control" content="no-siteapp" />
<link href="/img/style.css" rel="stylesheet" type="text/css" />
<script src="/img/m.js?2020"></script> 
</head>
<body>
 <div class="ClassNavi">
<a  href="/24shi/">二十四史</a> | <a href="/SiKuQuanShu/">四库全书</a> | <a href="http://www.guoxuedashi.com/gjtsjc/"><font  color="#FF0000">古今图书集成</font></a> | <a href="/renwu/">历史人物</a> | <a href="/ShuoWenJieZi/"><font  color="#FF0000">说文解字</a></font> | <a href="/chengyu/">成语词典</a> | <a  target="_blank"  href="http://www.guoxuedashi.com/jgwhj/"><font  color="#FF0000">甲骨文合集</font></a> | <a href="/yzjwjc/"><font  color="#FF0000">殷周金文集成</font></a> | <a href="/xiangxingzi/"><font color="#0000FF">象形字典</font></a> | <a href="/13jing/"><font  color="#FF0000">十三经索引</font></a> | <a href="/zixing/"><font  color="#FF0000">字体转换器</font></a> | <a href="/zidian/xz/"><font color="#0000FF">篆书识别</font></a> | <a href="/jinfanyi/">近义反义词</a> | <a href="/duilian/">对联大全</a> | <a href="/jiapu/"><font  color="#0000FF">家谱族谱查询</font></a> | <a href="http://www.guoxuemi.com/hafo/" target="_blank" ><font color="#FF0000">哈佛古籍</font></a> 
</div>

 <!-- 头部导航开始 -->
<div class="w1180 head clearfix">
  <div class="head_logo l"><a title="国学大师官网" href="http://www.guoxuedashi.com" target="_blank"></a></div>
  <div class="head_sr l">
  <div id="head1">
  
  <a href="http://www.guoxuedashi.com/zidian/bujian/" target="_blank" ><img src="http://www.guoxuedashi.com/img/top1.gif" width="88" height="60" border="0" title="部件查字,支持20万汉字"></a>


<a href="http://www.guoxuedashi.com/help/yingpan.php" target="_blank"><img src="http://www.guoxuedashi.com/img/top230.gif" width="600" height="62" border="0" ></a>


  </div>
  <div id="head3"><a href="javascript:" onClick="javascript:window.external.AddFavorite(window.location.href,document.title);">添加收藏</a>
  <br><a href="/help/setie.php">搜索引擎</a>
  <br><a href="/help/zanzhu.php">赞助本站</a></div>
  <div id="head2">
 <a href="http://www.guoxuemi.com/" target="_blank"><img src="http://www.guoxuedashi.com/img/guoxuemi.gif" width="95" height="62" border="0" style="margin-left:2px;" title="国学迷"></a>
  

  </div>
</div>
  <div class="clear"></div>
  <div class="head_nav">
  <p><a href="/">首页</a> | <a href="/ShuKu/">国学书库</a> | <a href="/guji/">影印古籍</a> | <a href="/shici/">诗词宝典</a> | <a   href="/SiKuQuanShu/gxjx.php">精选</a> <b>|</b> <a href="/zidian/">汉语字典</a> | <a href="/hydcd/">汉语词典</a> | <a href="http://www.guoxuedashi.com/zidian/bujian/"><font  color="#CC0066">部件查字</font></a> | <a href="http://www.sfds.cn/"><font  color="#CC0066">书法大师</font></a> | <a href="/jgwhj/">甲骨文</a> <b>|</b> <a href="/b/4/"><font  color="#CC0066">解密</font></a> | <a href="/renwu/">历史人物</a> | <a href="/diangu/">历史典故</a> | <a href="/xingshi/">姓氏</a> | <a href="/minzu/">民族</a> <b>|</b> <a href="/mz/"><font  color="#CC0066">世界名著</font></a> | <a href="/download/">软件下载</a>
</p>
<p><a href="/b/"><font  color="#CC0066">历史</font></a> | <a href="http://skqs.guoxuedashi.com/" target="_blank">四库全书</a> |  <a href="http://www.guoxuedashi.com/search/" target="_blank"><font  color="#CC0066">全文检索</font></a> | <a href="http://www.guoxuedashi.com/shumu/">古籍书目</a> | <a   href="/24shi/">正史</a> <b>|</b> <a href="/chengyu/">成语词典</a> | <a href="/kangxi/" title="康熙字典">康熙字典</a> | <a href="/ShuoWenJieZi/">说文解字</a> | <a href="/zixing/yanbian/">字形演变</a> | <a href="/yzjwjc/">金 文</a> <b>|</b>  <a href="/shijian/nian-hao/">年号</a> | <a href="/diming/">历史地名</a> | <a href="/shijian/">历史事件</a> | <a href="/guanzhi/">官职</a> | <a href="/lishi/">知识</a> <b>|</b> <a href="/zhongyi/">中医中药</a> | <a href="http://www.guoxuedashi.com/forum/">留言反馈</a>
</p>
  </div>
</div>
<!-- 头部导航END --> 
<!-- 内容区开始 --> 
<div class="w1180 clearfix">
  <div class="info l">
   
<div class="clearfix" style="background:#f5faff;">
<script src='http://www.guoxuedashi.com/img/headersou.js'></script>

</div>
  <div class="info_tree"><a href="http://www.guoxuedashi.com">首页</a> > <a href="/SiKuQuanShu/fanti/">四库全书</a>
 > <h1>资治通鉴</h1> <!--         下载:【右键另存为】即可 --></div>
  <div class="info_content zj clearfix">
  
<div class="info_txt clearfix" id="show">
<center style="font-size:24px;">133-資治通鑑卷一百三十二</center>
    資治通鑑卷一百三十二 宋 司馬光 撰<br />
<br />
  胡三省 音註<br />
<br />
  宋紀十四【起彊圉協洽盡上章閹茂凡四年】<br />
<br />
  太宗明皇帝中<br />
<br />
  泰始三年春正月張永等棄城夜遁 【考異曰宋本紀去年冬永攸之大敗遂失淮北四州及豫州淮西地宋畧今年正月永攸之師次彭城虜掩其輜重敗王穆之於武原薛安都開彭城以納虜永等引退虜追之王師敗績畢捺亦舉兖州歸虜遂失淮北之地魏帝紀去年九月常珍奇薛安都内属張永沈攸之擊安都詔尉元救彭城西河公石救懸瓠十一月畢衆敬内属十二月己未次于秺周凱張永沈攸之相繼退走今年正月癸巳尉元破永攸之於呂梁東閏月沈文秀崔道固舉州内属按青冀今歲始叛宋去年豈得已失淮北安都為永攸之所逼故降魏豈得今年永攸之始次彭城安都始納魏兵乎盖去年穆之等已敗退今春永大敗耳今從後魏帝紀】會天大雪泗水氷合永等棄船步走士卒凍死者大半手足斷者什七八尉元邀其前薛安都乘其後大破永等於呂梁之東【水經注泗水自彭城東南過呂縣南泗水之上有石梁焉故曰呂梁尉紆勿翻】死者以萬數枕尸六十餘里【枕職任翻】委棄軍資器械不可勝計【勝音升】永足指亦墮與沈攸之僅以身免梁南秦二州刺史垣恭祖等為魏所虜上聞之召蔡興宗以敗書示之曰我愧卿甚永降號左將軍攸之免官以貞陽公領職還屯淮隂【漢志桂陽郡有湞陽縣沈約志宋泰始三年改湞為貞屬廣興公相】由是失淮北四州及豫州淮西之地【淮北四州青冀徐兖豫州淮西汝南新蔡譙梁陳南頓潁川汝南汝隂諸郡也 考異曰後魏帝紀閏月沈文秀崔道固舉州内屬宋索虜傳曰永攸之敗退虜攻青冀二州執文秀道固又下書曰淮北三州民自天安二年正月三十日壬寅昧爽已前罪一切原免按青州破在五年淮北三州盖謂徐司豫壬寅二十日壬子三十日也】<br />
<br />
  裴子野論曰昔齊桓矜於葵丘而九國叛【公羊傳曰貫澤之會齊桓公有憂中國之心不召而至者江人黃人也葵丘之會桓公震而矜之叛者九國】曹公不禮張松而天下分【見六十五卷漢獻帝建安十三年】一失豪釐其差遠矣太宗之初威令所被不滿百里【被皮義翻】卒有離心士無固色而能開誠心布欵實莫不感恩服德致命效死故西摧北蕩宇内褰開既而六軍獻捷方隅束手天子欲賈其餘威【賈音古】師出無名長淮以北倏忽為戎惜乎若以嚮之虛懷不驕不伐則三叛奚為而起哉【三叛薛安都畢衆敬常珍奇也】高祖蟣虱生介冑【蟣居豨翻蝨子也蝨色櫛翻】經啓疆場【場音亦】後之子孫日蹙百里【大雅召旻之詩曰昔先王受命有如召公日辟國百里今也日蹙國百里】播穫堂構豈云易哉【書大誥曰若考作室既底法厥子乃弗肯堂矧肯構厥父菑厥子乃弗肯播矧肯穫易以䜴翻】<br />
<br />
  魏尉元以彭城兵荒之後公私困竭請冀相濟兖四州粟【相息亮翻濟子禮翻】取張永所弃船九百艘沿清運載以賑新民【新民謂新取徐州之民謂沿清水而運載也艘蘇遭翻賑津忍翻】魏朝從之【朝直遥翻】魏東平王道符反於長安殺副將駙馬都尉萬古真<br />
<br />
  等【副將副鎭將也將即亮翻下同】丙午司空和其奴等將殿中兵討之丁未道符司馬段太陽攻道符斬之以安西將軍陸眞為長安鎮將以撫之道符翰之子也【秦王翰死於正平宗愛之禍】閏月魏以頓丘王李峻為太宰 沈文秀崔道固為土人所攻【土人謂青冀二州之人】遣使乞降於魏【使疏吏翻降戶江翻下同】且請兵自救 二月魏西河公石自懸瓠引兵攻汝隂太守張超不克【此汝隂郡盖猶治汝隂也當在隋蔡州新蔡縣界守式又翻 考異曰宋帝紀云索虜寇汝隂太守張景遠擊破之景遠即超也宋畧七月張景遠先卒汝隂城又䧟亦誤也今從後魏書】退屯陳項【陳項本二邑時陳郡治項因曰陳項】議還長社待秋擊之鄭羲曰張超蟻聚窮命糧食已盡不降當走可翹足而待也今弃之遠去超修城浚隍積薪儲穀更來恐難圖矣石不從遂還長社【為石再攻汝隂不克張本】 初尋陽既平帝遣沈文秀弟文炳以詔書諭文秀又遣輔國將軍劉懷珍將馬步三千人與文炳偕行未至值張永等敗退懷珍還鎮山陽文秀攻青州刺史明僧暠【僧暠起兵見上卷上年暠古老翻】帝使懷珍帥龍驤將軍王廣之將五百騎步卒二千人浮海救之【帥讀曰卒驤思將翻將即亮翻騎奇寄翻】至東海僧暠已退保東萊懷珍進據朐城衆心兇懼欲且保郁洲【兇許拱翻兇恐懼聲朐城漢東海之胊縣城也水經註曰胊山西側有朐縣故城東北海中有大洲故謂之郁洲胊音劬】懷珍曰文秀欲以青州歸索虜【索昔各翻】計齊之士民安肯甘心左祍邪今揚兵直前宣布威德諸城可飛書而下柰何守此不進自為沮撓乎【沮在呂翻撓奴教翻屈也】遂進至黔陬【黔陬縣前漢屬琅邪郡後漢屬東萊郡晉屬城陽郡宋屬高密郡陬子侯翻隋志膠州膠西縣舊曰黔陬杜佑曰漢黔陬縣故城在密州諸城縣東北】文秀所署高密平昌二郡太守弃城走【高密漢郡平昌郡魏文帝分城陽立宋志高密郡領黔陬淳于高密夷安營陵昌安平昌郡領安邱平昌東武琅邪朱虛五代志黔陬縣舊置平昌郡】懷珍送致文炳逹朝廷意文秀猶不降【降戶江翻下同】百姓聞懷珍至皆喜文秀所署長廣太守劉桃根將數千人戍不其城【不其縣前漢屬琅邪郡後漢屬東萊郡晉分屬長廣郡唐萊州膠水縣即長廣郡地章懷太子賢曰不其故城在今萊州即墨縣西南其音基】懷珍軍於洋水【水經巨洋水出朱虛縣東泰山北過縣西北過臨朐縣東又北過劇縣西又東北過壽光縣西又東北入于海師古曰洋音祥】衆謂且宜堅壁伺隙懷珍曰今衆少糧竭【少詩沼翻】懸軍深入正當以精兵速進掩其不備耳乃遣王廣之將百騎襲不其城拔之文秀聞諸城皆敗乃遣使請降【使疏吏翻】帝復以為青州刺史崔道固亦請降復以為冀州刺史懷珍引還【懷珍既還兵勢不接故青冀二州尋為魏有復扶又翻】 魏濟隂王小新成卒【濟子禮翻】 沈攸之之自彭城還也留長水校尉王玄載守下邳【校戶教翻】積射將軍沈韶守宿豫睢陵淮陽皆留兵戍之玄載玄謨之從弟也【王玄謨以功名著於太祖世祖二朝從才用翻】時東平太守申纂守無鹽幽州刺史劉休賓守梁鄒【梁鄒縣漢屬濟南府】幷州刺史清河房崇吉守升城【魏收志東太原郡太原縣治升城其地在唐濟州長清縣界】輔國將軍清河張讜守團城【水經注琅邪郡東莞縣春秋之鄆邑今有鄆亭在團城東北四十里】及兖州刺史王整蘭陵太守桓忻【沈約志蘭陵太守治昌慮漢舊縣也】肥城糜溝垣苖等戍皆不附於魏【肥城縣前漢属泰山郡後漢屬濟北郡晉罷宋復置濟北郡于肥城魏收志糜溝垣苖二城亦在東太原郡太原縣界又據水經注濟水自平隂城西東北流逕垣苖城西東武帝西征長安令垣苖鎮此故俗以為稱濟水又東北過盧縣北賢曰肥城縣故城在今濟州平隂縣東南則此三戍皆在漢太山郡盧縣及肥城縣界至後漢和帝永安二年始分太山為濟北郡】休賓乘民之兄子也【劉乘民見上卷上年】魏遣平東將軍長孫陵等將兵赴青州【長知兩翻】征南大將軍慕容白曜將騎五萬為之繼援白曜燕太祖之玄孫也【燕王皝廟號太祖】白曜至無鹽欲攻之將佐皆以為攻具未備不宜遽進左司馬范陽酈範曰今輕軍遠襲深入敵境豈宜淹緩且申纂必謂我軍來速不暇攻圍將不為備【師速而疾者畧也畧謂畧地也無暇於攻城圍邑白曜以形形申纂故料其不為備也】今若出其不意可一鼓而克白曜曰司馬策是也乃引兵偽退申纂不復設備白曜夜中部分【復扶又翻分扶問翻】三月甲寅旦攻城食時克之纂走追擒殺之 【考異曰宋畧云七月纂戰死蓋贈官之月今從魏帝紀】白曜欲盡以無鹽人為軍賞酈範曰齊形勝之地宜遠為經略今王師始入其境人心未洽連城相望咸有拒守之志苟非以德信懷之未易平也【易以豉翻】白曜曰善皆免之白曜將攻肥城酈範曰肥城雖小攻之引日勝之不能益軍勢不勝足以挫軍威彼見無鹽之破死傷塗地不敢不懼若飛書告諭縱使不降亦當逃散【此即李左車教韓信以破趙之勢而喻燕故智也降戶江翻下同】白曜從之肥城果潰獲粟三十萬斛白曜謂範曰此行得卿三齊不足定也遂取垣苖糜溝二戍一旬中連拔四城威震齊土【史言慕容白曜能用酈範之計以取勝】 丙子以尚書左僕射蔡興宗為郢州刺史房崇吉守升城勝兵者不過七百人【勝音升任也勝兵者謂能操五兵而戰也】慕容白曜築長圍以攻之自二月至于夏四月乃克之白曜忿其不降欲盡阬城中人參軍事昌黎韓麒麟諫曰今勍敵在前而阬其民【勍渠京翻】自此以東諸城人自為守不可克也師老粮盡外寇乘之此危道也白曜乃慰撫其民各使復業崇吉脱身走崇吉母傅氏申纂妻賈氏與濟州刺史盧度世有中表親然已踈遠【濟子禮翻】及為魏所虜度世奉事甚恭贍給優厚【親者毋失其為親故者毋失其為故其盧度世之謂乎】度世閨門之内和而有禮雖世有屯夷【屯陟倫翻多難也夷平易也】家有貧富百口怡怡豐儉同之【史言盧度世有行】崔道固閉門拒魏沈文秀遣使迎降於魏請兵援接白曜欲遣兵赴之酈範曰文秀室家墳墓皆在江南【沈文秀吳興武康人】擁兵數萬城固甲堅彊則拒戰屈則遁去我師未逼其城無朝夕之急何所畏忌而遽求援軍且觀其使者視下而色愧語煩而志怯此必挾詐以誘我不可從也【春秋之時諸侯交兵謀人之軍師者多能以此覘敵酈範亦祖其故智耳誘音酉】不若先取歷城克般陽【般陽縣漢属濟南郡應劭曰有般水之陽師古曰般音盤劉煦曰唐淄州淄川縣漢盤陽縣也】下梁鄒平樂陵【晉武帝分平原立樂陵郡宋文帝置樂陵郡於故千乘地皆在隋唐青州界】然後案兵徐進不患其不服也白曜曰崔道固等兵力單弱不敢出戰吾通行無礙直抵東陽彼自知必亡故望風求服夫又何疑範曰歷城兵多糧足非朝夕可拔文秀坐據東陽為諸城根本今多遣兵則無以攻歷城少遣兵則不足以制東陽【少詩沼翻】若進為文秀所拒退為諸城所邀腹背受敵必無全理願更審計無墮賊彀中【彀古翻張子厚曰彀指拇也彀弓旣持滿以指拇為度而矢以志於中墮彀中者言敵彀弓指我而我不知避則矢必集于我而受其害】白曜乃止文秀果不降魏尉元上表稱彭城賊之要藩不有重兵積粟則不可固守若資儲旣廣雖劉彧師徒悉起不敢窺淮北之地【彧於六翻】又言若賊向彭城必由清泗過宿豫歷下邳趨青州亦由下邳泝水經東安【東安縣前漢屬城陽國後漢屬琅琊郡晉屬東莞郡惠帝元康七年分東莞置東安郡唐沂州沂水縣即東安郡地趨七喻翻】此數者皆為賊用師之要今若先定下邳平宿豫鎮淮陽戍東安則青冀諸鎮可不攻而克若四城不服青冀雖拔百姓狼顧猶懷僥倖之心【僥堅堯翻】臣愚以為宜釋青冀之師先定東南之地斷劉彧北顧之意【斷丁管翻】絶愚民南望之心夏水雖盛無津途可由冬路雖通無高城可固如此則淮北自舉暫勞永逸兵貴神速久則生變若天雨旣降彼或因水通運粮益衆規為進取恐近淮之民翻然改圖青冀二州猝未可拔也 【考異曰尉元傳先上表論取四城利害後乃云沈攸之欲援下邳遣孔道恭擊破之按元以泰始二年九月受詔救薛安都此表云受命出疆再離寒暑又云今雖向熱猶可行師則似上表時在四年春末夏初也又按沈攸之以三年八月出師尋即敗退則上表當在攸之敗後今此表但言陳顯逹循宿豫不言攸之救下邳又慕容白曜以四年二月十七日拔歷城而此表欲釋青冀之師先定東南之地則此表不在其年春末夏初决矣盖再當作載是語助之辭非謂兩經寒暑也故置於此】 五月壬戌以太子詹事袁粲為尚書右僕射 沈攸之自送運米至下邳【沈攸之自淮隂至下邳】魏人遣清泗間人詐攸之云薛安都欲降【降戶江翻】求軍迎接軍副吳喜請遣千人赴之攸之不許旣而來者益多喜固請不已攸之乃集來者告之曰君諸人旣有誠心若能與薛徐州子弟俱來者皆即假君以本鄉縣唯意所欲如其不爾無為空勞往還自是一去不返【以攸之知其情也】攸之使軍主彭城陳顯逹將千人助戍下邳而還【將即亮翻下同還從宣翻又如字】薛安都子令伯亡命梁雍之間【雍於用翻】聚黨數千人攻䧟郡縣秋七月雍州刺史巴陵王休若遣南陽太守張敬兒等擊斬之 上復遣中領軍沈攸之等擊彭城攸之以為清泗方涸糧運不繼固執以為不可使者七返上怒強遣之【復扶又翻強其兩翻 考異曰宋沈攸之傳宋畧皆云帝怒攸之云卿若不行便可使吳喜獨去按喜傳乃無與攸之討彭城事後魏書作吳僖公不知即吳喜為别一人也按南史亦有謂吳喜為吳喜公者】八月壬寅以攸之行南兖州刺史將兵北出使行徐州事蕭道成將千人鎮淮隂【去年僑立徐州於鍾離今使蕭道成屯淮隂為沈攸之後鎮】道成收養豪俊賓客始盛【為後蕭道成取宋張本】魏之入彭城也【事見上卷上年】垣崇祖將部曲奔朐山據之【魏收曰胊縣漢屬東海晉曰臨朐屬琅琊郡有朐山臨海濱今按晉志臨朐屬東莞郡後魏復曰朐屬琅邪郡朐音劬】遣使來降蕭道成以為朐山戍主朐山瀕海孤絶人情未安崇祖浮舟水側欲有急則逃入海魏東徐州刺史成固公戍圂城【魏收地形志魏置南青州於圂城圂城當在唐沂州沂水縣界圂戶困翻】崇祖部將有罪亡降魏成固公遣步騎二萬襲朐山【將即亮翻降戶江翻騎奇寄翻下同】去城二十里崇祖方出送客城中人驚懼皆下船欲去崇祖還謂腹心曰虜非有宿謀承叛者之言而來耳易誑也【易以豉翻誑居况翻】今得百餘人還事必濟矣但人情一駭不可歛集卿等可亟去此一里外大呼而來云艾塘義人已得破虜【呼火故翻艾塘當在唐海州懷仁縣界北齊於此置義塘郡宋人謂淮北起兵拒魏者為義人】須戍軍速往相助逐之舟中人果喜爭上岸【上時掌翻】崇祖引入據城遣羸弱入島【島海中山也羸倫為翻】人持兩炬火登山鼔譟魏參騎以為軍備甚盛乃退【參騎候騎也】上以崇祖為北琅邪蘭陵二郡太守垣榮祖亦自彭城奔朐山以奉使不效【奉使說薛安都見上卷上年】畏罪不敢出往依蕭道成於淮隂榮祖少學騎射或謂之曰武事可畏【謂矢石交乎前生死在於須臾也少詩照翻】何不學書榮祖曰昔曹公父子上馬橫槊下馬談詠【謂魏王操及文帝兄弟也上時掌翻槊色角翻】此於天下可不負飲食矣君輩無自全之伎【伎渠綺翻能也藝也】何異犬羊乎【此犬羊直謂其無防身之術耳】劉善明從弟僧副將部曲二千人避魏居海島道成亦召而撫之【從才用翻】 魏於天宫寺作大像高四十三尺【高居傲翻】用銅十萬斤黃金六百斤 魏尉元遣孔伯恭帥步騎一萬拒沈攸之【帥讀曰率騎奇寄翻】又以攸之前敗所喪士卒瘃墮膝行者悉還攸之以沮其氣【喪息浪翻瘃陟玉翻寒瘡也沮在呂翻】上尋悔遣攸之等復召使還【復扶又翻】攸之至焦墟去下邳五十餘里陳顯逹引兵迎攸之至睢清口【清水合於泗水故泗水亦得清水之名水經注泗水過下邳縣西又東南得睢水口泗水又東南入于淮水故謂之睢清口睢音雖】伯恭擊破之攸之引兵退伯恭追擊之攸之大敗龍驤將軍姜彦之等戰没攸之創重【驤思將翻創初良翻】入保顯逹營丁酉夜衆潰攸之輕騎南走委弃軍資器械以萬計還屯淮隂尉元以書諭徐州刺史王玄載玄載弃下邳走【沈攸之留王玄載戍下邳因領徐州刺史】魏以隴西辛紹先為下邳太守【守式又翻】紹先不尚苛察務舉大綱教民治生禦寇而已由是下邳安之【治直之翻】孔伯恭進攻宿豫宿豫戍將魯僧遵亦弃城走魏將孔大恒等將千騎南攻淮陽淮陽太守崔武仲焚城走【尉元取下邳等城不愆于素使沈攸之不敗於再舉元亦未敢輕動也淮陽太守治角城角城在唐泗州宿遷縣界宿遷即宿豫唐避諱改焉】慕容白曜進屯瑕邱崔道固之未降也【降戶江翻下同】綏邉將軍房法壽為王玄邈司馬屢破道固軍歷城人畏之【崔道固鎮歷城其軍皆歷城人】及道固降皆罷兵道固畏法壽扇動百姓廹遣法壽使還建康會從弟崇吉自升城來【從才用翻】以母妻為魏所獲謀於法壽法壽雅不欲南行【雅素也】怨道固廹之時道固遣兼治中房靈賓督清河廣川二郡事戍磐陽法壽乃與崇吉謀襲磐陽據之降於慕容白曜以贖崇吉母妻道固遣兵攻之白曜自瑕邱遣將軍長孫觀救磐陽【磐陽即盤陽】道固兵退白曜表冠軍將軍韓麒麟與法壽對為冀州刺史【冠古玩翻】以法壽從弟靈民思順靈悦伯憐伯玉叔玉思安幼安等八人皆為郡守【此冀州即宋所置冀州以命法夀郡守即守冀州所領廣川平原清河樂陵魏郡河間頓丘高陽勃海皆僑郡也】白曜自瑕邱引兵攻崔道固於歷城遣平東將軍長孫陵等攻沈文秀於東陽道固拒守不降白曜築長圍守之陵等至東陽文秀請降陵等入其西郭縱士卒暴掠文秀悔怒閉城拒守擊陵等破之 【考異曰文秀傳云八月虜蜀郡公拔式入西郭今從慕容白曜傳】陵等退屯清西屢進攻城不克 癸卯大赦 戊申魏主李夫人生子宏夫人惠之女也【李惠盖李貴人兄弟貴人魏主之母賜死見一百二十八卷宋孝建三年夫人則惠之女而宏之母也宏是為魏孝文帝】馮太后自撫養宏頃之還政於魏主魏主始視國事勤於為治【治直吏翻】賞罰嚴明拔清節黜貪汙於是魏之牧守始有以廉潔著聞者 太中大夫徐爰自太祖時用事【徐爰事始一百二十六卷文帝元嘉二十八年】素不禮於上上銜之詔數其姦佞之罪【數所其翻】徙交州 冬十月辛巳詔徙義陽王昶為晉熙王【昶丑兩翻考異曰宋帝紀在十一月今從宋畧】使員外郎李豐以金千兩贖昶於魏【昶奔魏見一百三十卷元年】魏人弗許使昶與上書為兄弟之儀上責其不稱臣不荅魏主復使昶與上書【復扶又翻下復攻復以同】昶辭曰臣本實彧兄【昶文帝第九子帝文帝第十一子】未經為臣若改前書事為二敬【既稱臣於魏復稱臣於宋是為二敬也】苟或不改彼所不納臣不敢奉詔乃止魏人愛重昶凡三尚公主 十一月乙卯分徐州置東徐州以輔國將軍張讜為刺史【張讜時守團城就置東徐州以刺史命之讜音黨】十二月庚戌以幽州刺史劉休賓為兖州刺史【時兖州之境已沒於魏劉休賓守梁鄒就以刺史命也】休賓之妻崔邪利之女也生子文曄與邪利皆沒於魏【邪利没於魏見一百二十五卷文帝元嘉二十七年】慕容白曜將其妻子至梁鄒城下示之休賓密遣主簿尹文逹至歷城見白曜且視其妻子休賓欲降而兄子聞慰不可白曜使人至城下呼曰劉休賓數遣人來見僕射約降【呼火故翻數所角翻降戶江翻】何故違期不至由是城中皆知之共禁制休賓不得降魏兵圍之 魏西河公石復攻汝隂【今年春石攻汝隂不克】汝隂有備無功而還【還從宣翻又如字】常珍奇雖降於魏實懷貳心劉勔復以書招之會西河公石攻汝隂珍奇乘虛燒劫懸瓠驅掠上蔡安成平輿三縣民屯於灌水【水經注灌水導源廬江金蘭縣西北東陵郷大蘇山東北逕蓼縣故城西而北注决水許愼曰灌水出雩婁縣輿音預】四年春正月己未上祀南郊大赦 魏汝陽司馬趙懷仁帥衆寇武津【汝陽司馬汝陽郡司馬也沈約曰武津縣屬汝陽郡何志不注置立在隋唐上蔡縣界帥讀曰率】豫州刺史劉勔遣龍驤將軍申元德擊破之又斬魏于都公閼于拔於汝陽臺東【魏收地形志汝陽郡汝陽縣有章華臺此書汝陽臺者盖以别南郡之章華臺也】獲運車千三百乘【乘䋲證翻】魏復寇義陽【復扶又翻】勔使司徒參軍孫臺瓘擊破之【臺瓘當作曇瓘】淮西民賈元友上書陳伐魏取陳蔡之策【宋豫州淮西之地春秋陳蔡之地也】上以其書示劉勔勔上言元友稱虜主幼弱内外多難【難乃旦翻】天亡有期臣以為虜自去冬蹈藉王土磐據數郡百姓殘亡今春以來連城圍逼國家未能復境何暇滅虜元友所陳率多夸誕狂謀皆無事實言之甚易行之甚難【易以豉翻】臣竊尋元嘉以來傖荒遠人多干國議負檐歸闕【傖助庚翻檐丁濫翻】皆勸討虜從來信納皆貽後悔境上之人唯視強弱王師至彼必壺漿候塗裁見退軍便抄截蜂起【抄楚交翻】此前後所見明驗非一也上乃止【史言劉勔諳識邊情】魏尉元遣使說東徐州刺史張讜讜以團城降魏【魏已】<br />
<br />
  【得彭城又得團城故因宋所置東徐州以命讜據水經注東莞郡治團城城在春秋之鄆邑西南四十里魏後徙東徐州治下邳說輸芮翻】魏以中書侍郎高閭與讜對為東徐州刺史【考異曰尉元傳沈攸之旣走元以書諭王玄載玄載與魯僧遵崔武仲楊繼皆走遂以高閭與張讜對為東徐州刺史按三年十一月乙卯始以讜為東徐州刺史則於時未降魏也故置於此】李璨與畢衆敬對為東兖州刺史【宋兖州治瑕邱畢衆敬以瑕邱降魏魏以為東兖州盖先已有兖州也】元又說兖州刺史王整【說輸芮翻沈約曰宋失淮北僑立兖州寄治淮隂時蕭道成鎮淮隂王整盖屯徐州界領兖州刺史耳此時宋魏交兵疆吏能自守者即以州刺史命之無常處也】蘭陵太守桓忻整忻皆降於魏魏以元為開府儀同三司都督徐南北兖三州諸軍事徐州刺史鎮彭城【兖州治瑕邱以王整新降故分南北兖】召薛安都畢衆敬入朝【朝直遥翻】至平城魏以上客待之羣從皆封侯賜第宅資給甚厚【從才用翻】慕容白曜圍歷城經年二月庚寅拔其東郭癸巳崔道固面縛出降 【考異曰宋畧云丙申索虜䧟歷城執崔道固按後魏列傳道固表云以今月十四日臣東郭失守以十七日面縛請罪長歷是月丁丑朔今從之】白曜遣道固之子景業與劉文曄同至梁鄒劉休賓亦出降白曜送道固休賓及其僚属於平城 辛丑以前龍驤將軍常珍奇為都督司北豫二州諸軍事司州刺史魏西河公石攻之珍奇單騎奔壽陽【常珍奇自灌水奔壽陽驤思將翻騎奇寄翻下同】 乙巳車騎大將軍曲江莊公王玄謨卒【曲江縣漢屬桂陽郡宋屬廣興公相謚法屢征殺伐曰莊武征而不遂曰莊】 三月魏慕容白曜進圍東陽【白曜旣得歷城始進圍東陽用酈範之計也】上以崔道固兄子僧祐為輔國將軍將兵數千從海道救歷城至不其聞歷城已沒遂降於魏【將即亮翻其音基降戶江翻】交州刺史劉牧卒州人李長仁殺牧北來部曲據州反自稱刺史 廣州刺史羊希使晉康太守沛郡劉思道伐俚【晉穆帝永和七年分蒼梧立晉康郡今端州即其地李延壽曰記云南方曰蠻有不火食者矣然其種類非一與華人錯居其流曰蜑曰獽曰俚曰獠曰㐌居無君長隨山洞而居俚音里】思道違節度失利希遣收之思道帥所領攻州【帥讀曰率】希兵敗而死龍驤將軍陳伯紹將兵伐俚還擊思道擒斬之【驤思將翻將即亮翻】希玄保之兄子也【羊玄保見一百二十三卷文帝元嘉十七年】 夏四月己卯復減郡縣田租之半【復扶又翻】 徙東海王禕為廬江王山陽王休祐為晉平王上以廢帝謂禕為驢王【見一百三十卷元年】故以廬江封之 劉勔敗魏兵於許昌【敗補邁翻】 魏以南郡公李惠為征南大將軍儀同三司都督關右諸軍事雍州刺史進爵為王【雍於用翻】 五月乙卯魏主畋于崞山遂如繁畤【崞山縣即漢晉鴈門之崞縣魏曰崞山天平二年分屬繁畤郡隋志鴈門崞縣有崞山據水經注山在繁畤之西灅水之南孟康曰崞音郭九域志繁畤縣在代州之東六十里】辛酉還宫 六月魏以昌黎王馮熙為太傳熙太后之兄也 秋七月庚申以驍騎將軍蕭道成為南兖州刺史【代沈攸之也南兖州治廣陵驍堅堯翻騎奇寄翻】 八月戊子以南康相劉勃為交州刺史【相息亮翻】 上以沈文秀之弟征北中兵參軍文靜為輔國將軍統高密等五郡軍事【五郡盖高密平昌長廣東海東莞也】自海道救東陽至不其城為魏所斷【其音基斷丁管翻】因保城自固魏人攻之不克辛卯分青州置東青州以文靜為刺史 九月辛亥魏立皇叔楨為南安王【楨音貞】長壽為城陽王太洛為章武王休為安定王 冬十月癸酉朔日有食之發諸州兵北伐 十一月李長仁遣使請降【使疏吏翻降戶江翻】自貶行州事許之 十二月魏人拔不其城殺沈文靜入東陽西郭 義嘉之亂巫師請修寧陵戮玄宫為厭勝是歲改葬昭太后【昭太后陵曰修寧晉安王子勛於昭后孫也改元義嘉南史曰修寜陵在孝武陵東南厭衣檢翻又益涉翻禳也】 先是中書侍郎舍人皆以名流為之【史曰先是謂元嘉以前通鑑因而書之先悉薦翻】太祖始用寒士秋當【秋當人姓名姓譜秋姓秋胡之後】世祖猶雜選士庶巢尚之戴法興皆用事【士謂巢尚之庶謂戴法興皆俱也】及上即位盡用左右細人【細微也纎也小也細人言纎微小人也】遊擊將軍阮佃夫中書通事舍人王道隆員外散騎侍郎楊運長等並參預政事權亞人主巢戴所不及也佃夫尤恣橫人有順迕禍福立至【佃音田散悉亶翻騎奇寄翻横下孟翻迕五故翻謂順之則有福迕之則有禍】大納貨賂所餉減二百匹絹則不報書園宅飲饌過於諸王妓樂服飾宫掖不如也朝士貴賤莫不自結僕隸皆不次除官捉車人至虎賁中郎將馬士至員外郎【捉車人持車者馬士控馬者員外郎謂員外散騎郎也饌離戀翻又雛皖翻妓渠織翻朝直遥翻賁音奔將即亮翻】<br />
<br />
  五年春正月癸亥上耕藉田大赦 沈文秀守東陽魏人圍之三年【泰始三年魏始攻文秀至此時首尾涉三年也】外無救援士卒晝夜拒戰甲胄生蟣蝨無離叛之志乙丑魏人拔東陽【史言沈文秀善守以援兵不接而没蟣居豨翻】文秀解戎服正衣冠取所持節坐齋内魏兵交至問沈文秀何在文秀厲聲曰身是【身猶今人言我】魏人執之去其衣【去羌呂翻】縛送慕容白曜使之拜文秀曰各兩國大臣何拜之有白曜還其衣為之設饌鎻送平城魏主數其罪而宥之【爲于偽翻數所具翻數其罪者以文秀既迎降復拒守也】待為下客給惡衣疏食【疏食粗飯也食祥吏翻】旣而重其不屈稍嘉禮之拜外都下大夫【外都下大夫外都大官之屬僚也拓跋氏所置】於是青冀之地盡入於魏矣 戊辰魏平昌宣王和其奴卒二月己卯魏以慕容白曜為都督青齊東徐三州諸軍事征南大將軍開府儀同三司青州刺史【宋置冀州於歷城魏既得之改為齊州統東魏東平原東清河廣川濟南東太原六郡東徐州統東安東莞二郡】進爵濟南王【濟子禮翻】白曜撫御有方東人安之【荀卿有言兼并易也堅凝之難魏并青徐淮北四州之民未忘宋也惟其撫御有方民安其生不復引領南望矣書云撫我則后虐我則仇信哉】魏自天安以來【泰始二年魏改元天安】比歲旱饑重以青徐用兵山東之民疲於賦役【比昆至翻重直用翻師之所聚荆棘生焉大兵之後必有凶年豈謂是邪】顯祖命因民貧富為三等輸租之法等為三品上三品輸平城中輸它州下輸本州又魏舊制常賦之外有雜調十五至是悉罷之由是民稍贍給【調徒釣翻史言魏能紓民力】河東柳欣慰等謀反欲立太尉廬江王禕禕自以於帝為兄而帝及諸兄弟皆輕之遂與欣慰等通謀相酬和【禕吁韋翻和戶卧翻】征北諮議參軍杜幼文告之丙申詔降禕為車騎將軍開府儀同三司南豫州刺史出鎭宣城帝遣腹心楊運長領兵防衛欣慰等並伏誅 三月魏人寇汝隂太守楊文萇擊却之 夏四月丙申魏大赦 五月魏徙青齊民於平城置升城歷城民望於桑乾立平齊郡以居之【據崔道固傳立平齊郡於平城西北北新城五代志馬邑郡雲内縣後魏立平齊郡乾音干】自餘悉為奴婢分賜百官魏沙門統曇曜奏【沙門統猶今之僧錄曇徒含翻】平齊戶及諸民有能歲輸穀六十斛入僧曹者即為僧祇戶粟為僧祇粟遇凶歲賑給飢民又請民犯重罪及官奴以為佛圖戶以供諸寺洒掃魏主並許之於是僧祇戶粟及寺戶徧於州鎭矣【魏自北方并有諸夏亦依魏晉制置諸州刺史其西北被邊夷晉雜居之地則置鎮將以鎮之祇翹移翻史言魏割民力以奉釋氏】 六月魏立皇子宏為太子 癸酉以左衛將軍沈攸之為郢州刺史 上又令有司奏廬江王禕忿懟有怨言【懟直類翻】請窮治不許【治直之翻】丁丑免禕官爵遣大鴻臚持節奉詔責禕【臚陵如翻】因逼令自殺子輔國將軍充明廢徙新安冬十月丁卯朔日有食之 魏頓邱王李峻卒 十<br />
<br />
  一月丁未魏復遣使來修和親自是信使歲通【自元嘉之末南北不復通好帝即位之三年四年再遣聘使是歲魏使來復通好復扶又翻使疏吏翻】 閏月戊子以輔師將軍孟陽為兖州刺史始治淮隂【是歲改輔國將軍為輔師將軍兖州本治瑕邱既入於魏始治淮隂蕭子顯曰淮隂縣前漢屬臨淮郡後漢屬下邳國晉屬廣陵郡穆帝永和中荀羨北討鮮卑以淮隂舊鎮地形都要水陸交通乃營邱城池是時既失淮北遂為重鎮後為北兖州治所九域志楚州淮隂縣在州西四十里】 十二月戊戌司徒建安王休仁解揚州休仁年與上鄰亞素相友愛景和之世上賴其力以脱禍【事見上卷元年】及泰始初四方兵起休仁親當矢石克成大功【事亦見上卷元年二年】任揔百揆親寄甚隆由是朝野輻凑上漸不悦【為帝殺休仁張本朝直遥翻】休仁悟其旨故表解揚州己未以桂陽王休範為揚州刺史 分荆州之巴東建平益州之巴西梓潼郡置三巴校尉治白帝【校戶教翻】先是三峽蠻獠歲為抄暴故立府以鎭之【先悉薦翻抄楚交翻府謂三巴校尉府也】上以司徒參軍東莞孫謙為巴東建平二郡太守謙將之官敕募千人自隨謙曰蠻夷不賓盖待之失節耳何煩兵役以為國費固辭不受至郡開布恩信蠻獠翕然懷之【獠魯皓翻】競餉金寶謙皆慰諭不受 臨海賊帥田流自稱東海王剽掠海鹽殺鄞令【帥所類翻剽匹妙翻鄞縣自漢以來屬會稽郡師古曰鄞音牛斤翻今屬明州】東土大震<br />
<br />
  六年春正月乙亥初制間二年一祭南郊間一年一祭明堂【昔周公郊祀后稷以配天宗祀文王於明堂以配上帝三歲一郊始於漢武帝平帝元始中始行祫祭明堂之禮明帝永平初始盛其儀亦曰宗祀公羊傳曰古者五年而再殷祭謂祫祭也然古之所謂祫者合祭於太祖之廟而明堂宗祀則嚴父以配帝此先儒之說所以異也蔡邕謂明堂即太廟盖有見於此歟然明堂九室而太廟七室則又不得而合也間年一祭非古也故曰初制間古莧翻】 二月壬寅以司徒休仁為太尉領司徒固辭【辭者以上不悦也然終不能免於禍】 癸丑納江智淵孫女為太子妃甲寅大赦令百官皆獻物始興太守孫泰伯止獻琴書上大怒封藥賜死既而原之 魏以東郡王陸定國為司空定國麗之子也【宗愛之亂陸麗有立嫡之功乙渾之亂麗死之】 魏主遣征西大將軍上黨王長孫觀擊吐谷渾【長知两翻吐從暾入聲谷音浴】 夏四月辛丑魏大赦戌申魏長孫觀與吐谷渾王拾寅戰於曼頭山【隋伐吐谷渾置河源郡有曼頭城盖因山得名也考異曰宋本紀作拾虔今從後魏書】拾寅敗走遣别駕康盤龍入貢魏主囚之 癸亥立皇子爕為晉熙王奉晉熙王昶後【昶時在魏】 五月魏立皇弟長樂為建昌王【樂音洛】六月癸卯以江州刺史王景文為尚書左僕射揚州<br />
<br />
  刺史以尚書僕射袁粲為右僕射上宫中大宴祼婦人而觀之【祼即果翻】王后以扇障面上怒曰外舍寒乞【寒乞猶言寒陋也】今共為樂何獨不視后曰為樂之事其方自多豈有姑姊妹集而祼婦人以為笑外舍之樂雅異於此【樂音洛】上大怒遣后起后兄景文聞之曰后在家劣弱今段遂能剛正如此 南兖州刺史蕭道成在軍中久【據蕭子顯齊書文帝元嘉十九年遣道成討竟陵蠻則在軍中久矣】民間或言道成有異相當為天子【齊書言道成姿表奇異龍顙鐘聲鱗文遍體相息亮翻】上疑之徵為黃門侍郎越騎校尉【騎奇寄翻校戶教翻】道成懼不欲内遷而無計得留冠軍參軍廣陵荀伯玉勸道成遣數十騎入魏境安置標榜【道成時假冠軍將軍以伯玉為參軍冠古玩翻】魏果遣遊騎數百履行境上【行下孟翻】道成以聞上使道成復本任秋九月命道成遷鎮淮隂【按三年八月蕭道成以行徐州事鎮淮隂以沈攸之北伐使為後鎮也攸之北還道成代為南兖州刺史鎮廣陵今復使遷鎮淮隂】以侍中中領軍劉勔為都督南徐兖等五州諸軍事鎭廣陵 戊寅立總明觀置祭酒一人儒玄文史學士各十人【文帝元嘉十五年立儒玄文史四學今置總明觀祭酒以撫之觀古玩翻】 柔然部眞可汗侵魏【可從刋入聲汗音寒】魏主引羣臣議之尚書右僕射南平公目辰曰若車駕親征京師危懼不如持重固守虜懸軍深入糧運無繼不久自退遣將追擊破之必矣【將即亮翻】給事中張白澤曰蠢爾荒愚輕犯王略【杜預曰畧界也毛晃曰畧封界也】若鑾輿親行必望麾崩散豈可坐而縱敵以萬乘之尊嬰城自守非所以威服四夷也【乘繩證翻】魏主從之白澤衮之孫也【魏道武之建國也張衮有功焉】魏主使京兆王子推等督諸軍出西道任城王雲等督諸軍出東道【任音壬】汝隂王天賜等督諸軍為前鋒隴西王源賀等督諸軍為後繼鎭西將軍呂羅漢等掌留臺事諸將會魏主於女水之濱【將即亮翻】與柔然戰柔然大敗乘勝逐北斬首五萬級降者萬餘人獲戎馬器械不可勝計【勝音升】旬有九日往返六千餘里改女水曰武川【按魏紀女水當在長川之西赤城之西北後魏置武川鎮隋書宇文述代郡武川人代郡指代郡平城也魏都平城謂之代都代都以北列置鎮將其後罷鎭置州則武川屬代郡】司徒東安王劉尼坐昏醉軍陳不整免官【陳讀曰陣】壬申還至平城是時魏百官不給禄少能以廉白自立者【前言魏主拔清節黜貪汙魏之牧守始有以廉潔著聞者此言魏之百官少能以亷白自立盖法行於州郡未行於朝廷也少詩沼翻】魏主詔吏受所監臨羊一口酒一斛者死與者以從坐論【監工銜翻】有能糾告尚書已下罪狀者隨所糾官輕重授之張白澤諫曰昔周之下士尚有代耕之禄【孟子曰周室班爵祿下士與庶人在官者同祿祿足以代其耕也】今皇朝貴臣服勤無報若使受禮者刑身【受禮謂受羊酒之禮刑身謂刑加其身朝直遥翻】糾之者代職臣恐姦人闚望忠臣懈節【懈居隘翻】如此而求事簡民安不亦難乎請依律令舊法仍班祿以酬亷吏魏主乃為之罷新法【為于偽翻】冬十月辛卯詔以世祖繼體䧟憲無遺【事見上卷二年】以皇子智隨為世祖子立為武陵王【孝武廟號世祖 考異曰宋本紀作智贊宋畧作贊列傳作智隨按太宗生子皆筮之以卦為其字今從列傳】 初魏乙渾專政【事見上卷元年二年】慕容白曜頗附之魏主追以為憾遂稱白曜謀反誅之及其弟如意初魏南部尚書李敷儀曹尚書李訢【蕭子顯曰魏南部尚書知南邊】<br />
<br />
  【州郡儀曹尚書盖知禮儀按魏初有殿中樂部駕部南部北部五尚書儀曹選曹等尚書中世所置訢許斤翻】少相親善【少詩沿翻】與中書侍郎盧度世皆以才能為世祖顯祖所寵任參豫機密出納詔命其後訢出為相州刺史【相息亮翻】受納貨賂為人所告敷掩蔽之顯祖聞之檻車徵訢案驗服罪當死是時敷弟奕得幸於馮太后帝意已踈之有司以中旨諷訢告敷兄弟隂事可以得免訢謂其壻裴攸曰吾與敷族世雖遠恩踰同生今在事勸吾為此【在事謂有司也言在官而止案敷之事】吾情所不忍每引簪自刺【刺七亦翻】解帶自絞終不得死且吾安能知其隂事將若之何攸曰何為為人死也【為為上如字下于偽翻】有馮闡者先為敷所敗【敗補邁翻】其家深怨之今詢其弟敷之隂事可得也訢從之又趙郡范條列敷兄弟事狀凡三十餘條【與標同音卑遥翻】有司以聞帝大怒誅敷兄弟【為馮太后鴆魏主張本告李敷兄弟者范其後告李訢者亦范也】訢得減死鞭髠配役未幾復為太倉尚書攝南部事【魏中世分殿中尚書所掌倉庫置太倉尚書掌倉粟事也幾居豈翻復狀又翻】敷順之子也【李順以才能事魏太武而為崔浩所告而誅】 魏陽平王新成卒是歲命龍驤將軍義興周山圖將兵屯浹口討田流<br />
<br />
  平之【驤思將翻浹即協翻】 柔然攻于闐于闐遣使者素目伽奉表詣魏求救魏主命公卿議之皆曰于闐去京師幾萬里【北史曰于闐國去代九千八百里闐徒賢翻又堂見翻使疏吏翻伽求迦翻幾居希翻】蠕蠕唯習野掠【蠕人兖翻】不能攻城若其可攻尋已亡矣雖欲遣師勢無所及魏主以議示使者使者亦以為然乃詔之曰朕應急敇諸軍以拯汝難【難乃旦翻】但去汝遐阻必不能救當時之急汝宜知之朕今練甲養士一二歲間當躬帥猛將為汝除患【帥讀曰率為于偽翻】汝其謹脩警堠以待大舉<br />
<br />
  資治通鑑卷一百三十二<br />
<br />
<史部,編年類,資治通鑑>  <br>
   </div> 

<script src="/search/ajaxskft.js"> </script>
 <div class="clear"></div>
<br>
<br>
 <!-- a.d-->

 <!--
<div class="info_share">
</div> 
-->
 <!--info_share--></div>   <!-- end info_content-->
  </div> <!-- end l-->

<div class="r">   <!--r-->



<div class="sidebar"  style="margin-bottom:2px;">

 
<div class="sidebar_title">工具类大全</div>
<div class="sidebar_info">
<strong><a href="http://www.guoxuedashi.com/lsditu/" target="_blank">历史地图</a></strong>  
<a href="http://www.880114.com/" target="_blank">英语宝典</a>  
<a href="http://www.guoxuedashi.com/13jing/" target="_blank">十三经检索</a> 
<br><strong><a href="http://www.guoxuedashi.com/gjtsjc/" target="_blank">古今图书集成</a></strong> 
<a href="http://www.guoxuedashi.com/duilian/" target="_blank">对联大全</a> <strong><a href="http://www.guoxuedashi.com/xiangxingzi/" target="_blank">象形文字典</a></strong> 

<br><a href="http://www.guoxuedashi.com/zixing/yanbian/">字形演变</a>  <strong><a href="http://www.guoxuemi.com/hafo/" target="_blank">哈佛燕京中文善本特藏</a></strong>
<br><strong><a href="http://www.guoxuedashi.com/csfz/" target="_blank">丛书&方志检索器</a></strong> <a href="http://www.guoxuedashi.com/yqjyy/" target="_blank">一切经音义</a>  

<br><strong><a href="http://www.guoxuedashi.com/jiapu/" target="_blank">家谱族谱查询</a></strong>  <strong><a href="http://shufa.guoxuedashi.com/sfzitie/" target="_blank">书法字帖欣赏</a></strong> 
<br>

</div>
</div>


<div class="sidebar" style="margin-bottom:0px;">

<font style="font-size:22px;line-height:32px">QQ交流群9:489193090</font>


<div class="sidebar_title">手机APP 扫描或点击</div>
<div class="sidebar_info">
<table>
<tr>
	<td width=160><a href="http://m.guoxuedashi.com/app/" target="_blank"><img src="/img/gxds-sj.png" width="140"  border="0" alt="国学大师手机版"></a></td>
	<td>
<a href="http://www.guoxuedashi.com/download/" target="_blank">app软件下载专区</a><br>
<a href="http://www.guoxuedashi.com/download/gxds.php" target="_blank">《国学大师》下载</a><br>
<a href="http://www.guoxuedashi.com/download/kxzd.php" target="_blank">《汉字宝典》下载</a><br>
<a href="http://www.guoxuedashi.com/download/scqbd.php" target="_blank">《诗词曲宝典》下载</a><br>
<a href="http://www.guoxuedashi.com/SiKuQuanShu/skqs.php" target="_blank">《四库全书》下载</a><br>
</td>
</tr>
</table>

</div>
</div>


<div class="sidebar2">
<center>


</center>
</div>

<div class="sidebar"  style="margin-bottom:2px;">
<div class="sidebar_title">网站使用教程</div>
<div class="sidebar_info">
<a href="http://www.guoxuedashi.com/help/gjsearch.php" target="_blank">如何在国学大师网下载古籍?</a><br>
<a href="http://www.guoxuedashi.com/zidian/bujian/bjjc.php" target="_blank">如何使用部件查字法快速查字?</a><br>
<a href="http://www.guoxuedashi.com/search/sjc.php" target="_blank">如何在指定的书籍中全文检索?</a><br>
<a href="http://www.guoxuedashi.com/search/skjc.php" target="_blank">如何找到一句话在《四库全书》哪一页?</a><br>
</div>
</div>


<div class="sidebar">
<div class="sidebar_title">热门书籍</div>
<div class="sidebar_info">
<a href="/so.php?sokey=%E8%B5%84%E6%B2%BB%E9%80%9A%E9%89%B4&kt=1">资治通鉴</a> <a href="/24shi/"><strong>二十四史</strong></a>&nbsp; <a href="/a2694/">野史</a>&nbsp; <a href="/SiKuQuanShu/"><strong>四库全书</strong></a>&nbsp;<a href="http://www.guoxuedashi.com/SiKuQuanShu/fanti/">繁体</a>
<br><a href="/so.php?sokey=%E7%BA%A2%E6%A5%BC%E6%A2%A6&kt=1">红楼梦</a> <a href="/a/1858x/">三国演义</a> <a href="/a/1038k/">水浒传</a> <a href="/a/1046t/">西游记</a> <a href="/a/1914o/">封神演义</a>
<br>
<a href="http://www.guoxuedashi.com/so.php?sokeygx=%E4%B8%87%E6%9C%89%E6%96%87%E5%BA%93&submit=&kt=1">万有文库</a> <a href="/a/780t/">古文观止</a> <a href="/a/1024l/">文心雕龙</a> <a href="/a/1704n/">全唐诗</a> <a href="/a/1705h/">全宋词</a>
<br><a href="http://www.guoxuedashi.com/so.php?sokeygx=%E7%99%BE%E8%A1%B2%E6%9C%AC%E4%BA%8C%E5%8D%81%E5%9B%9B%E5%8F%B2&submit=&kt=1"><strong>百衲本二十四史</strong></a>  <a href="http://www.guoxuedashi.com/so.php?sokeygx=%E5%8F%A4%E4%BB%8A%E5%9B%BE%E4%B9%A6%E9%9B%86%E6%88%90&submit=&kt=1"><strong>古今图书集成</strong></a>
<br>

<a href="http://www.guoxuedashi.com/so.php?sokeygx=%E4%B8%9B%E4%B9%A6%E9%9B%86%E6%88%90&submit=&kt=1">丛书集成</a> 
<a href="http://www.guoxuedashi.com/so.php?sokeygx=%E5%9B%9B%E9%83%A8%E4%B8%9B%E5%88%8A&submit=&kt=1"><strong>四部丛刊</strong></a>  
<a href="http://www.guoxuedashi.com/so.php?sokeygx=%E8%AF%B4%E6%96%87%E8%A7%A3%E5%AD%97&submit=&kt=1">說文解字</a> <a href="http://www.guoxuedashi.com/so.php?sokeygx=%E5%85%A8%E4%B8%8A%E5%8F%A4&submit=&kt=1">三国六朝文</a>
<br><a href="http://www.guoxuedashi.com/so.php?sokeytm=%E6%97%A5%E6%9C%AC%E5%86%85%E9%98%81%E6%96%87%E5%BA%93&submit=&kt=1"><strong>日本内阁文库</strong></a> <a href="http://www.guoxuedashi.com/so.php?sokeytm=%E5%9B%BD%E5%9B%BE%E6%96%B9%E5%BF%97%E5%90%88%E9%9B%86&ka=100&submit=">国图方志合集</a> <a href="http://www.guoxuedashi.com/so.php?sokeytm=%E5%90%84%E5%9C%B0%E6%96%B9%E5%BF%97&submit=&kt=1"><strong>各地方志</strong></a>

</div>
</div>


<div class="sidebar2">
<center>

</center>
</div>
<div class="sidebar greenbar">
<div class="sidebar_title green">四库全书</div>
<div class="sidebar_info">

《四库全书》是中国古代最大的丛书,编撰于乾隆年间,由纪昀等360多位高官、学者编撰,3800多人抄写,费时十三年编成。丛书分经、史、子、集四部,故名四库。共有3500多种书,7.9万卷,3.6万册,约8亿字,基本上囊括了古代所有图书,故称“全书”。<a href="http://www.guoxuedashi.com/SiKuQuanShu/">详细>>
</a>

</div> 
</div>

</div>  <!--end r-->

</div>
<!-- 内容区END --> 

<!-- 页脚开始 -->
<div class="shh">

</div>

<div class="w1180" style="margin-top:8px;">
<center><script src="http://www.guoxuedashi.com/img/plus.php?id=3"></script></center>
</div>
<div class="w1180 foot">
<a href="/b/thanks.php">特别致谢</a> | <a href="javascript:window.external.AddFavorite(document.location.href,document.title);">收藏本站</a> | <a href="#">欢迎投稿</a> | <a href="http://www.guoxuedashi.com/forum/">意见建议</a> | <a href="http://www.guoxuemi.com/">国学迷</a> | <a href="http://www.shuowen.net/">说文网</a><script language="javascript" type="text/javascript" src="https://js.users.51.la/17753172.js"></script><br />
  Copyright &copy; 国学大师 古典图书集成 All Rights Reserved.<br>
  
  <span style="font-size:14px">免责声明:本站非营利性站点,以方便网友为主,仅供学习研究。<br>内容由热心网友提供和网上收集,不保留版权。若侵犯了您的权益,来信即刪。scp168@qq.com</span>
  <br />
ICP证:<a href="http://www.beian.miit.gov.cn/" target="_blank">鲁ICP备19060063号</a></div>
<!-- 页脚END --> 
<script src="http://www.guoxuedashi.com/img/plus.php?id=22"></script>
<script src="http://www.guoxuedashi.com/img/tongji.js"></script>

</body>
</html>
