<!DOCTYPE html PUBLIC "-//W3C//DTD XHTML 1.0 Transitional//EN" "http://www.w3.org/TR/xhtml1/DTD/xhtml1-transitional.dtd">
<html xmlns="http://www.w3.org/1999/xhtml">
<head>
<meta http-equiv="Content-Type" content="text/html; charset=utf-8" />
<meta http-equiv="X-UA-Compatible" content="IE=Edge,chrome=1">
<title>資治通鑒_223-資治通鑑卷二百二十二_223-資治通鑑卷二百二十二</title>
<meta name="Keywords" content="資治通鑒_223-資治通鑑卷二百二十二_223-資治通鑑卷二百二十二">
<meta name="Description" content="資治通鑒_223-資治通鑑卷二百二十二_223-資治通鑑卷二百二十二">
<meta http-equiv="Cache-Control" content="no-transform" />
<meta http-equiv="Cache-Control" content="no-siteapp" />
<link href="/img/style.css" rel="stylesheet" type="text/css" />
<script src="/img/m.js?2020"></script> 
</head>
<body>
 <div class="ClassNavi">
<a  href="/24shi/">二十四史</a> | <a href="/SiKuQuanShu/">四库全书</a> | <a href="http://www.guoxuedashi.com/gjtsjc/"><font  color="#FF0000">古今图书集成</font></a> | <a href="/renwu/">历史人物</a> | <a href="/ShuoWenJieZi/"><font  color="#FF0000">说文解字</a></font> | <a href="/chengyu/">成语词典</a> | <a  target="_blank"  href="http://www.guoxuedashi.com/jgwhj/"><font  color="#FF0000">甲骨文合集</font></a> | <a href="/yzjwjc/"><font  color="#FF0000">殷周金文集成</font></a> | <a href="/xiangxingzi/"><font color="#0000FF">象形字典</font></a> | <a href="/13jing/"><font  color="#FF0000">十三经索引</font></a> | <a href="/zixing/"><font  color="#FF0000">字体转换器</font></a> | <a href="/zidian/xz/"><font color="#0000FF">篆书识别</font></a> | <a href="/jinfanyi/">近义反义词</a> | <a href="/duilian/">对联大全</a> | <a href="/jiapu/"><font  color="#0000FF">家谱族谱查询</font></a> | <a href="http://www.guoxuemi.com/hafo/" target="_blank" ><font color="#FF0000">哈佛古籍</font></a> 
</div>

 <!-- 头部导航开始 -->
<div class="w1180 head clearfix">
  <div class="head_logo l"><a title="国学大师官网" href="http://www.guoxuedashi.com" target="_blank"></a></div>
  <div class="head_sr l">
  <div id="head1">
  
  <a href="http://www.guoxuedashi.com/zidian/bujian/" target="_blank" ><img src="http://www.guoxuedashi.com/img/top1.gif" width="88" height="60" border="0" title="部件查字,支持20万汉字"></a>


<a href="http://www.guoxuedashi.com/help/yingpan.php" target="_blank"><img src="http://www.guoxuedashi.com/img/top230.gif" width="600" height="62" border="0" ></a>


  </div>
  <div id="head3"><a href="javascript:" onClick="javascript:window.external.AddFavorite(window.location.href,document.title);">添加收藏</a>
  <br><a href="/help/setie.php">搜索引擎</a>
  <br><a href="/help/zanzhu.php">赞助本站</a></div>
  <div id="head2">
 <a href="http://www.guoxuemi.com/" target="_blank"><img src="http://www.guoxuedashi.com/img/guoxuemi.gif" width="95" height="62" border="0" style="margin-left:2px;" title="国学迷"></a>
  

  </div>
</div>
  <div class="clear"></div>
  <div class="head_nav">
  <p><a href="/">首页</a> | <a href="/ShuKu/">国学书库</a> | <a href="/guji/">影印古籍</a> | <a href="/shici/">诗词宝典</a> | <a   href="/SiKuQuanShu/gxjx.php">精选</a> <b>|</b> <a href="/zidian/">汉语字典</a> | <a href="/hydcd/">汉语词典</a> | <a href="http://www.guoxuedashi.com/zidian/bujian/"><font  color="#CC0066">部件查字</font></a> | <a href="http://www.sfds.cn/"><font  color="#CC0066">书法大师</font></a> | <a href="/jgwhj/">甲骨文</a> <b>|</b> <a href="/b/4/"><font  color="#CC0066">解密</font></a> | <a href="/renwu/">历史人物</a> | <a href="/diangu/">历史典故</a> | <a href="/xingshi/">姓氏</a> | <a href="/minzu/">民族</a> <b>|</b> <a href="/mz/"><font  color="#CC0066">世界名著</font></a> | <a href="/download/">软件下载</a>
</p>
<p><a href="/b/"><font  color="#CC0066">历史</font></a> | <a href="http://skqs.guoxuedashi.com/" target="_blank">四库全书</a> |  <a href="http://www.guoxuedashi.com/search/" target="_blank"><font  color="#CC0066">全文检索</font></a> | <a href="http://www.guoxuedashi.com/shumu/">古籍书目</a> | <a   href="/24shi/">正史</a> <b>|</b> <a href="/chengyu/">成语词典</a> | <a href="/kangxi/" title="康熙字典">康熙字典</a> | <a href="/ShuoWenJieZi/">说文解字</a> | <a href="/zixing/yanbian/">字形演变</a> | <a href="/yzjwjc/">金 文</a> <b>|</b>  <a href="/shijian/nian-hao/">年号</a> | <a href="/diming/">历史地名</a> | <a href="/shijian/">历史事件</a> | <a href="/guanzhi/">官职</a> | <a href="/lishi/">知识</a> <b>|</b> <a href="/zhongyi/">中医中药</a> | <a href="http://www.guoxuedashi.com/forum/">留言反馈</a>
</p>
  </div>
</div>
<!-- 头部导航END --> 
<!-- 内容区开始 --> 
<div class="w1180 clearfix">
  <div class="info l">
   
<div class="clearfix" style="background:#f5faff;">
<script src='http://www.guoxuedashi.com/img/headersou.js'></script>

</div>
  <div class="info_tree"><a href="http://www.guoxuedashi.com">首页</a> > <a href="/SiKuQuanShu/fanti/">四库全书</a>
 > <h1>资治通鉴</h1> <!--         下载:【右键另存为】即可 --></div>
  <div class="info_content zj clearfix">
  
<div class="info_txt clearfix" id="show">
<center style="font-size:24px;">223-資治通鑑卷二百二十二</center>
    資治通鑑卷二百二十二 宋 司馬光 撰<br />
<br />
  胡三省 音註<br />
<br />
  唐紀三十八【起重光赤奮盡昭陽單閼六月凡二年有奇】<br />
<br />
  肅宗文明武德大聖大宣孝皇帝下之下<br />
<br />
  上元二年春正月癸卯史思明改元應天 張景超引兵攻杭州敗李藏用將李彊於石夷門【敗補邁翻將即亮翻下同】孫待封自武康南出【吴分烏程餘杭二縣置永安縣晋改為永康又改為武康唐属杭州】將會景超攻杭州【自武康南出過狗頭嶺至杭州五十里】温晁據險擊敗之【去年李藏用使温晁屯餘杭餘杭東至杭州錢塘縣界一十八里又東二十七里則至杭州此陸路也故温晁得趨而據險以敗孫待封】待封脫身奔烏程李可封以常州降丁未田神功使特進楊惠元等將千五百人西擊王暅【暅戶登翻】辛亥夜神功先遣特進范知新等將四千人自白沙濟西趣下蜀鄧景山將千人自海陵濟東趣常州【趣七喻翻下同】神功與邢延恩將三千人軍于瓜洲壬子濟江展將步騎萬餘陳于蒜山【陳讀曰陣蒜山在潤州城西三里其上多蒜故名蒜蘇貫翻】神功以舟載兵趣金山會大風五舟飄抵金山下【金山在大江中南直西津渡口去潤州城七里】展屠其二舟沈其三舟【沈持林翻】神功不得度還軍瓜洲而范知新等兵已至下蜀展擊之不勝弟殷勸展引兵逃入海可延歲月展曰若事不濟何用多殺人父子乎死早晩等耳遂更率衆力戰【帥讀曰率】將軍賈隐林射展中目而仆遂斬之【射而亦翻中竹仲翻 考異曰實錄云乙卯平盧兵馬使田神功生擒逆賊劉展舊神功傳亦然今從劉展亂紀】劉殷許嶧等皆死【嶧音亦】隱林滑州人也楊惠元等擊破王暅于淮南暅引兵東走至常熟乃降【王暅東走度江而至常熟晉分关縣置海虞縣梁立信義郡南沙縣隋平陳廢郡并海虞南沙海陽前京信義興國等縣為常熟縣属蘇州】孫待封詣李藏用降張景超聚兵至七千餘人聞展死悉以兵授張灋雷使攻杭州景超逃入海灋雷至杭州李藏用擊破之餘黨皆平平盧軍大掠十餘日【田神功所將平盧兵也】安史之亂亂兵不及江淮至是其民始罹荼毒矣 【考異曰劉展亂紀孫待封降以下事在二月今因展敗終言之】 荆南節度使呂諲奏請以江南之潭岳郴邵永道連黔中之涪州皆隸荆南從之【邵州漢召陵都梁之地召陵後漢改為昭陽晋改為邵陽吳立邵陵郡隋廢郡為邵陽縣属潭州唐武德四年分置南梁州貞觀十年更名邵州郴丑林翻黔其今翻】 二月奴刺党項寇寶雞【奴刺西羌種落之名刺來葛翻至德二載改陳倉縣為寶雞縣以其地有秦時寶雞祠故也時属鳳翔府】燒大散關南侵鳳州殺刺史蕭■〈忄曳〉【■〈忄曳〉余世翻】大掠而西鳳翔節度使李鼎追擊破之 戊辰新羅王金嶷入朝【嶷魚力翻】因請宿衛 或言洛中將士皆燕人【燕於肩翻】久戍思歸上下離心擊之可破也陜州觀軍容使魚朝恩以為信然屢言於上上勅李光弼等進取東京光弼奏稱賊鋒尚銳未可輕進朔方節度使僕固懷恩勇而愎【愎蒲逼翻】麾下皆蕃漢勁卒恃功多不法郭子儀寛厚曲容之每用兵臨敵倚以集事李光弼性嚴一裁之以法無所假貸懷恩憚光弼而心惡之乃附朝恩言東都可取【史言僕固懷恩欲覆李光弼之軍以便其私惡烏故翻朝直遥翻】由是中使相繼督光弼使出師光弼不得已使鄭陳節度使李抱玉守河陽與懷恩將兵會朝恩及神策節度使衛伯玉攻洛陽戊寅陳於邙山光弼命依險而陳懷恩陳於平原光弼曰依險則可以進可以退若平原戰而不利則盡矣思明不可忽也命移于險懷恩復止之史思明乘其陳未定【陳讀曰陣】進兵薄之官軍大敗死者數千人軍資器械盡棄之 【考異曰實録云史思明潜遣間諜反說官軍曰洛中將士久成思歸士多不睦魚朝恩以為然乃告光弼及僕固懷恩衛伯玉等曰可速出軍以掃殘寇光弼等然之今從舊光弼傳實録曰光弼懷恩敗績步兵死者數萬今從舊思明傳】光弼懷恩度河走保聞喜朝恩伯玉奔還陜抱玉亦弃河陽走河陽懷州皆沒于賊朝廷聞之大懼益兵屯陜【相州之敗邙山之敗皆魚朝恩為之也唐不以覆軍之罪罪朝恩而罷郭李兵柄失刑甚矣】 李揆與呂諲同為相不相悦【乾元二年李揆與呂諲同相上元元年諲罷】諲在荆南以善政聞揆恐其復入相奏言置軍湖南非便【軍郴邵水道連皆在洞庭湖之南呂諲請兼領之故揆言非其便復扶又翻】又隂使人如荆湖【荆謂荆南湖謂湖南】求諲過失諲上疏頌揆罪癸未貶揆袁州長史以河中節度使蕭華為中書侍郎同平章事 史思明猜忍好殺【好呼到翻】羣下小不如意動至族誅人不自保朝義其長子也【朝直遙翻下朝清同長知兩翻】常從思明將兵頗謙謹愛士卒將士多附之無寵於思明思明愛少子朝清使守范陽【朝凊守范陽事始上卷上年少詩照翻】常欲殺朝義立朝清為太子左右頗泄其謀思明既破李光弼欲乘勝西入關使朝義將兵為前鋒自北道襲陜城思明自南道將大軍繼之【南道出二崤之間漢建安中曹公西討巴蜀惡南路之險更開北道】三月甲午朝義兵至礓子嶺【即礓子阪也按舊書礓子嶺在陜城東】衛伯玉逆擊破之 【考異曰實錄作甲子按長歷此月丙戌朔下有戊戌當作甲午】朝義數進兵皆為陜兵所敗【數所角翻敗補邁翻】思明退屯永寜以朝義為怯曰終不足成吾事欲按軍法斬朝義及諸將戊戌命朝義築三隅城【新書作三角城盖一角依山止築其三角也】欲貯軍糧【貯丁呂翻】期一日畢朝義築畢未泥思明至詬怒之令左右立馬監泥斯須而畢【詬古候翻監古銜翻】思明又曰俟克陜州終斬此賊朝義憂懼不知所為思明在鹿橋驛【鹿橋驛永寧傳舍也貞觀十七年嘗徙永寜縣於此】令腹心曹將軍將兵宿衛朝義宿于逆旅【水經注陜城東有漫澗澗北有逆旅亭謂之漫口客舍此酈道元以一時經由所見者言之耳自元魏至唐乾元上元間三百許年矣漫口客舍必不復存此逆旅特汎言旅舍耳】其部將駱悦蔡文景說朝義曰【說式芮翻】悦等與王死無日矣自古有廢立請召曹將軍謀之朝義俛首不應【俛音免】悦等曰王苟不許悦等今歸李氏王亦不全矣朝義泣曰諸君善為之勿驚聖人【當時臣子謂其君父為聖人】悦等乃令許叔冀之子季常召曹將軍至則以其謀告之曹將軍知諸將盡怨恐禍及己不敢違【令力丁翻將即亮翻】是夕悦等以朝義步兵三百被甲詣驛【被皮義翻】宿衛兵怪之畏曹將軍不敢動悦等引兵入至思明寢所值思明如厠問左右未及對已殺數人左右指示之思明聞有變踰垣至廐中自鞴馬乘之【鞴平祕翻】悦傔人周子俊射之【傔丁念翻射而亦翻】中臂墜馬遂擒之【中竹仲翻】思明問亂者為誰悦曰奉懷王命【思明封朝義為懷王】思明曰我朝來語失【謂欲斬朝義也】宜其及此然殺我太早何不待我克長安今事不成矣悦等送思明于柳泉驛囚之【唐制三十里一驛柳泉驛又當在鹿橋驛東三十里 考異曰河洛春秋曰思明混諸嫡庶以少者為尊惟愛所鍾即為繼嗣欲殺朝義追朝清為偽太子左右泄之父子之隙自此始構邠志曰三月思明乘勝欲下陜城使朝義帥銳卒北路先往己自宜陽引衆繼之今從實録舊傳】還報朝義曰事成矣【還從宣翻朝直通翻】朝義曰不驚聖人乎悦曰無時周摯許叔冀將後軍在福昌【將即亮翻又音如字福昌又在柳泉驛之東宋白曰福昌縣唐属洛州占宜陽地今縣治魏一泉塢城】悦等使許季常往告之摯驚倒于地朝義引軍還摯叔冀來迎悦等勸朝義執摯殺之軍至柳泉悦等恐衆心未壹遂縊殺思明【縊一計翻又於賜翻】以氈裹其尸槖駝負歸洛陽朝義即皇帝位改元顯聖密使人至范陽勅散騎常侍張通儒等【散昔亶翻騎奇計翻】殺朝清及朝清母辛氏并不附己者數十人其黨自相攻擊戰城中數月死者數千人范陽乃定【去年若果能使郭子儀自朔方直擣范陽通值城中自相攻擊可馳檄而下也】朝義以其將柳城李懷仙為范陽尹燕京留守【燕因肩翻守式又翻 考異曰實錄曰朝義既殺思明密遣使馳至范陽殺偽太子朝英及偽皇后辛氏并不附己者數十人偽范陽留守張通儒知有變遂引兵戰於城中數日戰不利死者數千人通儒被斬于亂兵中薊門紀亂曰思明既王有數十州之地年餘朝興遂為皇太子朝興辛氏之長男特為思明所愛嗜酒好色凶獷頑戾招集幽薊惡少與其年齒相類者百餘人為左右皆彎弓利劍飾以丹雘珠玉帶佩印雕鏤金銀控弦揮刃常如見敵以南行大將子弟統之每與其黨飲宴酒酣爇燎其鬚髮或以銅弹丸擊之以頤顙為的血流至地無楚痛之色則賞巵酒少似嚬蹙乃鞭之從脛至腫或至數千困絶將殞方捨之候稍愈復鞭之有杖六七千不死者姬妾皆思明所掠良家子有不稱命則殺之亦有以湯鑊死者既火盛湯沸令壯士抱而投之初宛轉叫呼須臾骨肉糜爛旁人皆毛竪股慄朝興笑臨而觀之以所策毬杖于鑊中撞擊顔色自若上元二年三月甲寅使使告捷云王師敗績于洛北斬首萬餘級勒其六宫及朝興備車馬為赴洛之計賊庭之黨相慶踊躍叫唤聲振天地十餘日又宦者二人傳思明偽勑云收兵陜虢以朝興為周京留守仍勒馳驛速并辛氏已下續行朝興大喜其宦者朝義偽遣之人莫知也時朝義已殺思明僭位濳勒偽左散騎常侍張通儒戶部尚書康孝忠與朝興衙將高鞫仁高如震等謀誅朝興其日朝興速召工匠與其母妻造寶鈿鞍勒搜索庫藏修乘騎之具並命左右各備行装唯數十人侍衛思明留駿馬百餘匹在其厩中朝興出入馳驟每日則於桑乾河飲之通儒將入潜令康孝忠從數十人持兵詣飲處馳取其馬閉於城南毗沙門神之院通儒與鞫仁領步兵十餘人入其日華門偽皇城留守劉象昌逢之驚問其故通儒顧左右斬之俄而朝興腹心衛鳴鶴又問亦斬之子城擾亂朝興惶怖猶能擐甲持兵與親信二三十人出拒奔走亍厩中取馬馬盡矣唯病馬一匹朝興乘而策之不前遂步戰通儒立白旗招朝興之黨降者捨罪復官爵惡少等雖沐朝興之錫賚亦怨其無道鞭捶降者大半朝興猶從十餘人接戰弓矢所無不中者中者皆應弦沒羽通儒軍披靡所傷者數十百人退出子城外人不知甲兵之故皆惶恐潜匿通儒于城門拒戰良久日已云暮朝興衆寡不敵走匿城上之逍遙樓遂失所在通儒兵入禁中劫掠金帛思明朝興妻衣服皆盡夜丰蕃將曹閔之於樓上擒獲之朝興曰我兄弟六七人朝興一身斬之何益高如雲對曰以殿下殘酷人各有怨心朝興曰乞放此一度後更不敢執者皆笑又謂閔之曰此腰帶三十兩黄金新造謹奉將軍閔之曰殿下但死腰帶閔之自解取左右益笑縊以弓弦斷其首函送洛陽偽侍中向閏客特受思明委託朝興亦甚敬憚至是惶怖走入私第不自安匍匐待罪通儒頷之勒馳驛赴洛通儒收朝興黨與悉誅之思明驍將辛萬年特有寵于朝興叉與鞫仁如震等友善為兄弟當誅朝興之黨也通儒有意於萬年及令行刑遂忘之至是勑鞫仁如震斬萬年首送鞫仁置酒與萬年同飲謂曰張尚書令殺弟故相報萬年稽首但乞快死鞫仁抗聲曰只可兄弟謀取通儒終不肯殺弟於是如震萬年領其部曲百餘人入子城斬通儒於子城南廊下城中擾亂又殺其素不快者軍將數人共推偽中書令阿史那承慶為留守函通儒等首使萬年送洛陽誣其欲以薊城歸順朝義聞之使使令向閏客所在却回為留守鞫仁如震等各從數百人被甲廵城城中人心彌懼承慶為留守一兩日又不自安逓相疑阻於是領蕃兵數十騎出子城至如震宅門立令屈將軍暫要相見如震不虞有難馳至馬前承慶斬之應聲而殞承慶入東軍與偽尚書康孝忠招集蕃羯鞫仁聞如震遇害驚而且怒統麾下軍討之相逢于宴設樓下接戰自午至酉鞫仁兵皆城旁少年驍勇勁捷馳射如飛承慶兵雖多不敵大敗殺傷甚衆積尸成丘承慶孝忠出城收散卒東保潞縣又南掠屬縣野營月餘徑詣洛陽自陳其事城中蕃軍家口盡踰城相繼而去鞫仁令城中殺胡者皆重賞於是羯胡俱殪小兒皆擲於空中以戈承之高鼻類胡而濫死者甚衆時鞫仁在城中最尊使使奏朝義以承慶等反向閏客行至貝州承朝義命迴將至衆官迎之鞫仁嚴兵不出閏客甚懼戒其子弟從者無帶兵器從數人而入鞫仁待之於日華門閏客望見下馬執手相慰鞫仁亦抗禮還營閏客但專守子城端坐餘不敢輒有所問奏承慶等使迴朝義以鞫仁為燕京都知兵馬使五月甲戌朔朝義以偽大常卿李懷仙為御史大夫范陽節度使燕州頗有兵甲故委腹心鞫仁聞之意不快也無何懷仙至從羸馬數千自薊城南門入鞫仁不出迎之於日華門懷仙至卑身過禮立談約為兄弟結盟相固期同保燕邦以奨其主鞫仁意少解懷仙以薊縣為節度院雖任節制鞫仁兵五千餘人皆不受命十數日懷仙待之彌厚每衙皆降階交接鞫仁亦不為之屈既而懷仙命饗軍士中宴鞫仁疑有變兵皆驚走還營被甲懷仙憂懼無計遂囚其牙將朱希彩賁以驚軍中之罪其夜鞫仁將襲懷仙遇大雨持疑未决徹明遂止单騎至節度門懷仙已潛備壯士待之鞫仁趨入懷仙亦不改常禮與坐良久乃問驚軍之罪門已關顧左右拉殺之立捨希彩自暮春至夏中兩月間城中相攻殺凡四五死者數千戰闘皆在坊市閭巷間但兩敵相向不入人家剽劫一物盖家家自有軍人之故又百姓至於婦人小童皆閑習弓矢以此無虞六月丙申宣思明遺誥喪將相百僚縞素哭于其聽政樓前卑幼相視而笑笑聲與哭聲参半焉朝義又追向閏客赴洛陽加懷仙燕京留守河洛春秋初朝義令人與向貢并阿史那王殺朝清朝凊既受父命常有君臨之心惟以毬獵為務車下勇敢之士僅三千人每日教習然其殘酷頗有父風而加婬亂幽州士庶無不吁嗟向貢高久仁等既見諸將之書又聞思明已死因說朝清曰昨有密旨令大王主器承祧其事尤重今敵國猶在上人未還倘更移恩于人誠恐自貽窘廹朝凊然之是日顧左右各令辭訣便自飭装高久仁高如震等及其無備率壯士數百人濳入子城門阿史那王向貢等共率三百人繼至朝清時在卧内僕妾侍側忽聞兵士問是何人門人曰三軍叛乃擐甲登樓賁讓向貢等高如震乃於樓下佯戰朝清自援弓射之凡斃數人阿史那軍佯北朝清下樓向貢等令人擒殺之向貢攝知軍事經四十日阿史那又殺向貢阿史那自稱長史三日後斬高久仁以其首梟之殺朝清故也高如震還固守與阿史那相持城中分兩軍經五日以燕州街為界各自禦備逓相捉搦不得往來阿史那從經略軍領諸蕃部落及漢兵三萬人至宴設樓前與如震會戰如震不利乃使輕兵二千人于子城東出直至經略軍南街腹背而擊之并招漢軍萬餘人阿史那軍敗走于武清縣界野營後朝義使招之盡歸東都應是胡面不擇少長盡誅之於是朝義偽受李懷仙幽州節度高如震旅拒之中承阿史那遁逸之後野行卓次人各持兵糗糧芻茭非戮不應朝義令兵士悉為高賈白衣先行至幽州盡被捉為團練懷仙方自統五千餘騎直叩薊門高如震方欲出師以抗其命慮其卒叛因出迎之懷仙實内圖之且外示寛宥天行誘募咸捨厥?於是士衆帖然競皆欣戴乃大賞設經三日因衆前却乃斬高如震幽州遂平舊傳亦云朝義令人殺偽太子朝英新傳作朝清今從河洛春秋及新傳餘從薊門紀亂】時洛陽四面數百里州縣皆為丘墟而朝義所部節度使皆安禄山舊將與思明等夷朝義召之多不至【朝直遥翻使疏吏翻將即亮翻】略相覊縻而已不能得其用 李光弼上表固求自貶【上時掌翻】制以開府儀同三司侍中領河中節度使 術士長塞鎮將朱融【據新書長塞鎮當在蔚州界唐制上鎮將正六品下中鎮將正七品上下鎮將正七品下】與左武衛將軍竇如玢等【玢音彬】謀奉嗣岐王珍作亂金吾將軍邢濟告之夏四月乙卯朔廢珍為庶人溱州安置其黨皆伏誅【嗣祥史翻溱兹詵翻】珍業之子也【岐王業上皇之弟】丙辰左散騎常侍張鎬貶辰州司戶鎬嘗買珍宅故也【散昔亶翻騎奇計翻鎬下老翻辰州瀘溪郡漢辰陵沅陵義陵縣地舊志辰州京師南微東三千四百五里】 己未以吏部侍郎裴遵慶為黄門侍郎同平章事 乙亥青密節度使尚衡破史朝義兵斬首五千餘級 丁丑兖鄆節度使能元皓破史朝義兵【鄆音運能奴代翻】 壬午梓州刺史段子璋反子璋驍勇從上皇在蜀有功【梓州梓潼郡漢鄠縣廣漢氐道地驍堅堯翻】東川節度使李奐奏替之【替代也】子璋舉兵襲奐于綿州【綿州治巴西漢之涪縣也】道過遂州刺史虢王巨蒼黄修屬郡禮迎之【梓遂二州並属東川節度蓋列郡也巨修屬郡禮以迎子璋示卑服之意】子璋殺之李奐戰敗奔成都子璋自稱梁王改元黄龍以綿州為龍安府置百官又陷劒州【劒州治普安之梓潼縣也】  五月己丑李光弼自河中入朝 初李輔國與張后同謀遷上皇於西内【遷上皇見上卷上年】是日端午山人李唐見上上方抱幼女謂唐曰朕念之卿勿怪也對曰太上皇思見陛下計亦如陛下之念公主也上泫然泣下【泫胡畎翻】然畏張后尚不敢詣西内 癸巳党項寇寶雞 初史思明以其博州刺史令狐彰為滑鄭汴節度使將數千兵戍滑臺【滑州古滑臺也】彰密因中使楊萬定通表請降徙屯杏園度思明疑之遣其將薛岌圍之彰與岌戰大破之因随萬定入朝甲午以彰為滑衛等六州節度使【滑衛相貝魏博六州】 戊戍平盧節度使侯希逸擊史朝義范陽兵破之 乙未西川節度使崔光遠與東川節度使李奐共攻綿州庚子拔之斬段子璋 復以李光弼為河南副元帥太尉兼侍中都統河南淮南東西山南東荆南江南西浙江東西八道行營節度 【考異曰實録舊紀皆云光弼都統河南淮南山南東江東五道唐歷會要為河南淮南東西山南東荆南五道劉展亂紀又有江西浙東浙西凡八道按袁晁亂浙東光弼討平之則是浙東亦其統内也今從之復扶又翻帥所類翻統他綜翻】出鎮臨淮【臨淮郡泗州】 六月甲寅青密節度使能元皓敗史朝義將李元遇【按上卷五年冬書兖鄆節度使能元皓詳考本末青密恐當作兖鄆敗補賣翻】 江淮都統李峘畏失守之罪【失守事見上卷上年峘戶登翻】歸咎於浙西節度使侯令儀丙子令儀坐除名長流康州【康州因晉康郡而名治端溪縣至京師五千七百五十里東都五千一百五十里】加田神功開府儀同三司【賞平劉展之功也】徙徐州刺史【自平盧兵馬使徙刺徐州】徵李峘鄧景山還京師【還從宣翻又音如字】戊寅党項寇好畤【党底朗翻畤音止好畤縣自漢至後魏屬扶風後周省隋開皇十七】<br />
<br />
  【年置上宜縣屬京兆又有舊莫西縣十八年改曰好畤大業三年廢入上宜武德二年分醴泉置好畤貞觀八年廢上宜入岐陽二十一年省好畤岐陽復置上宜更上宜曰好畤】 秋七月癸未朔日有食之既大星皆見【見賢遍翻】 以試少府監李藏用為浙西節度副使【少詩照翻】 八月癸丑朔加開府儀同三司李輔國兵部尚書乙未輔國赴上【僕射尚書赴省供職曰赴上上時掌翻】宰相朝臣皆送之御厨具饌太常設樂輔國驕縱日甚求為宰相上曰以卿之功何官不可為其如朝望未允何【朝息亮翻朝直遥翻】輔國乃諷僕射裴冕等使薦已上密謂蕭華曰輔國求為宰相若公卿表來不得不與華出問冕曰初無此事吾臂可斷【斷丁管翻】宰相不可得華入言之上大悦輔國銜之【銜戶緘翻】 己巳李光弼赴河南行營 辛巳以殿中監李若幽為鎮西北庭興平陳鄭等節度行營及河中節度使【代李光弼】鎮絳州賜名國貞 九月甲申天成地平節【上於景雲二年九月三日生以九月三日為天成地平節】上於三殿置道塲【南部新書大明宫中有麟德殿在仙居殿之西北此殿三面亦以三殿為名雍録麟德殿在翰林院之東】以宫人為佛菩薩【釋典曰菩普也薩濟也言能普濟衆生也】武士為金剛神王【范成大曰在處寺門有兩金剛神是千佛數中最後者一名婁至德一名青葉髻】召大臣膜拜圍繞【膜莫呼翻】 壬寅制去尊號但稱皇帝去年號但稱元年【去羌呂翻】以建子月為歲首月皆以所建為數因赦天下停京兆河南太原鳳翔四京及江陵南都之號【四京見二百二十卷至德元載南都見上卷上年】自今每除五品以上清望官及郎官御史刺史令舉一人自代觀其所舉以行殿最【令力丁翻殿丁甸翻】 江淮大饑人相食 冬十月江淮都統崔圓署李藏用為楚州刺史【統他綜翻或從上聲 考異曰劉展亂紀曰初劉展既平諸將争功疇賞未及李藏用崔圓乃署藏用為楚州刺史領二城而居盱眙按實録七月藏用已除浙西節度副使盖恩命木到耳】會支度租庸使以劉展之亂諸州用倉庫物無準奏請徵驗【唐六典度攴郎掌支度國用租賦多少之數凡天下邉軍皆有攴度之使以計軍資糧仗之用每歲皆申度攴而會計之此支度租庸使盖使之支度江淮租庸者也使疏史翻】時倉猝募兵物多散亡徵之不足諸將往往賣產以償之藏用恐其及已嘗與人言頗有悔恨其牙將高幹挾故怨使人詣廣陵告藏用反先以兵襲之藏用走幹追斬之崔圓遂簿責藏用將吏以驗之將吏畏皆附成其狀【將即亮翻附高幹之言以成李藏用反狀】獨孫待封堅言不反圓命引出斬之或曰子何不從衆以求生待封曰吾始從劉大夫奉詔書來赴鎮【劉大夫謂劉展赴鎮事見上卷上年】人謂吾反李公起兵滅劉大夫今又以李公為反如此誰則非反者庸有極乎吾寜就死不能誣人以非罪遂斬之【史言兵興之時多濫刑】 建子月壬午朔上受朝賀如正旦儀【以其月為歲首也朝直遥翻下同】 或告鴻臚卿康謙與史朝義通事連司農卿嚴莊俱下獄【臚陵如翻朝直遥翻下遐嫁翻獄宜欲翻】京兆尹劉晏遣吏防守莊家上尋勑出莊引見【見賢遍翻】莊怨晏因言晏與臣言常道禁中語矜功怨上丁亥貶晏通州刺史【通州通川郡漢宕渠縣地舊志通州京師西南二千五百里】莊難江尉【難江縣漢宕渠地梁置東巴州後魏恭帝改集州後周改宕渠為難江縣以水為名也隋廢集州以縣屬梁州唐初復置集州以難江為治所】謙伏誅戊子御史中丞元載為戶部侍郎充句當度支鑄錢鹽鐵兼江淮轉運等使【句古候翻當丁浪翻度徒洛翻使疏史翻】載初為度支郎中敏悟善奏對上愛其才委以江淮漕運數月遂代劉晏專掌財利戊戍冬至己亥上朝上皇於西内【朝直遥翻下同】 神策節度使衛伯玉攻史朝義拔永寜破澠池福昌長水等縣【永寧澠池福昌三縣時属河南府澠彌兖翻後魏分陜縣置南陜縣西魏改曰長淵屬弘農郡唐初改名長水避高祖諱也初屬糓州後属洛州宋白曰長水縣本盧氏地後魏延昌二年分盧氏東境庫谷以西沙渠谷以東為南陜縣餘同上注】 己酉上朝獻太清宫庚戍享太廟元獻廟【太凊宫在丹鳳門之左南出第二坊太廟在朱雀街東第二街北來第二坊元獻廟上母元獻楊后廟也】建丑月辛亥朔祀圓丘太一壇【乾元元年立太一壇見二百二十卷 考異曰實録建子月戊戌冬至其日祀昊天上帝己亥詔以來月一日祀圓丘及太一壇又云建丑月辛亥以河南節度使來瑱為太子少保下又有丁未己酉庚戌日事又云建丑月辛亥朔拜南郊祭太一壇按瑱傳未嘗為河南節度使及少保實錄剩此一日事其冬至祀上帝盖有司行事非親祀也】 平盧節度使侯希逸與范陽相攻連年救援既絶又為奚所侵乃悉舉其軍二萬餘人襲李懷仙破之因引兵而南<br />
<br />
  寶應元年【以楚州表言上帝賜寶玉改元改元實在四月】建寅月【去年九月勑以建子月為歲首而通鑑仍以建寅月為歲首者以是年四月制復月數皆如其舊也改元亦在是月】甲申追尊靖德太子琮為奉天皇帝妃竇氏為㳟應皇后丁酉葬于齊陵【琮上皇之長子天寶十載薨謚曰靖德太子新書地理志齊陵在京兆昭應縣東十六里琮徂宗翻】 甲辰吐蕃遣使請和【使疏吏翻】 李光弼拔許州擒史朝義所署潁川太守李春朝義將史参救之【許州潁川郡唐已復郡為州安史猶仍天寶舊名守式又翻將即亮翻】丙午戰于城下又破之 戊申平盧節度使侯希逸於青州北度河而會田神功能元皓於兖州【使疏史翻能奴代翻】 租庸使元載以江淮雖經兵荒其民比諸道猶有貲產乃按籍舉八年租調之違負及逋逃者計其大數而徵之【八年自天寶十三載止上元二年天寶十三載天下未亂租調之入為盛十四載而祿山反租調始有違負逋逃自是迄于去年大難未平戰兵不止違負逋逃年甚一年今不問有無計其天數而徵之調徒釣翻】擇豪吏為縣令而督之不問負之有無貲之高下察民有粟帛者發徒圍之籍其所有而中分之甚者什取八九謂之白著【著直略翻今人猶謂無故而費散財物者為白著勃海高雲有白著歌曰上元官史務剝削江淮之人多白著】有不服者嚴刑以威之民有蓄穀十斛者則重足以待命【重直龍翻】或相聚山澤為羣盗州縣不能制 建卯月辛亥朔赦天下復以京兆為上都河南為東都鳳翔為西都江陵為南都太原為北都【去年罷西京及南都復扶又翻又如字】 奴刺寇成固【成固縣自漢以來屬漢中刺來違翻】初王思禮為河東節度使資儲豐衍贍軍之外積米百萬斛奏請輸五十萬斛於京師思禮薨【乾元元年王思禮鎮太原其薨當在去年】管崇嗣代之為政寛弛信任左右數月間耗散殆盡惟陳腐米萬餘斛在【言見在米止此數】上聞之以鄧景山代之景山至則鉤校所出入將士輩多有隱沒皆懼有禆將抵罪當死諸將請之不許其弟請代兄死亦不許請入一馬以贖死乃許之諸將怒曰我輩曾不及一馬乎遂作亂【將即亮翻】癸丑殺景山上以景山撫御失所以致亂不復推䆒亂者遣使慰諭以安之【復扶又翻】諸將請以都知兵馬使代州刺史辛雲京為節度使雲京奏張光晟為代州刺史【張光晟有德於辛雲京見上卷乾元二年】 絳州素無儲蓄民間饑不可賦歛【歛力贍翻】將士糧賜不充朔方等諸道行營都統李國貞屢以狀聞朝廷未報軍中咨怨【咨嗟憂愁而怨上也】突將王元振將作亂【突將以領驍勇馳突之士突將即亮翻】矯令於衆曰來日修都統宅各具畚鍤【畚布衮翻織竹為器鍤測洽翻鍫也】待命于門士卒皆怒曰朔方健兒豈修宅夫邪乙丑元振帥其徒作亂燒牙城門【師讀曰率】國貞逃于獄元振執之置卒食於前曰食此而役其力可乎國貞曰修宅則無之軍食則屢奏而未報諸君所知也衆欲退元振曰今日之事何必更問都統不死則我輩死矣遂拔刃殺之鎮西北庭行營兵屯于翼城【翼城縣属絳州本漢絳縣後魏曰北絳縣隋開皇十八年改曰翼城以春秋翼侯邑於此也】亦殺節度使茘非元禮推禆將白孝德為節度使朝廷因而授之 戊辰淮西節度使王仲昇與史朝義將謝欽讓戰于申州城下為賊所虜淮西震駭會侯希逸田神功能元皓攻汴州朝義召欽讓兵救之 絳州諸軍剽掠不已【剽匹妙翻】朝廷憂其與太原亂軍合從連賊【從子容翻】非新進諸將所能鎮服辛未以郭子儀為汾陽王知朔方河中北庭潞澤節度行營兼興平定國等軍副元帥發京師絹四萬匹布五萬端米六萬石以給絳軍建辰月庚寅子儀將行時上不豫羣臣莫得進見【見賢遍翻】子儀請曰老臣受命將死於外不見陛下目不瞑矣上召入卧内謂曰河東之事一以委卿史朝義遣兵圍李抱玉於澤州子儀發定國軍救之乃去上召山南東道節度使來瑱赴京師瑱樂在襄陽其<br />
<br />
  將士亦愛之乃諷所部將吏上表留之行及鄧州復令還鎮【樂音洛復扶又翻】荆南節度使呂諲淮西節度使王仲昇及中使往來者言瑱曲收衆心恐久難制上乃割商金均房别置觀察使令瑱止領六州【山南東道領襄鄧随唐安均房金商九州今分四州餘五州耳曰領六州無亦於郢復二州增領一州邪】會謝欽讓圍王仲昇於申州數月瑱怨之按兵不救仲昇竟敗沒行軍司馬裴茙謀奪瑱位【茙如融翻】密表瑱倔彊難制【倔其勿翻彊其兩翻】請以兵襲取之上以為然癸巳以瑱為淮西河南十六州節度使 【考異曰舊傳無汝云十五州今從實錄】外示寵任實欲圖之密勅以茙代瑱為襄鄧等州防禦使 甲午奴刺寇梁州【刺來達翻】觀察使李勉棄城走【時山南西道觀察使置司梁州】以邠州刺史河西臧希讓為山南西道節度使【山南西道節度領梁洋集壁文通巴興鳳利開渠蓬十三州邠彌頻翻使疏史翻 考異曰肅宗實録作希液代宗實録有傳作希讓今從之】 丙申党項寇奉天【党底朗翻奉天縣属雍州】 李輔國以求宰相不得怨蕭華【求相不得事見上上元二年八月相息亮翻】庚午以戶部侍郎元載為京兆尹載詣輔國固辭輔國識其意壬寅以司農卿陶銳為京兆尹輔國言蕭華專權請罷其相上不許輔國固請不已乃從之仍引元載代華 【考異曰舊華傳云肅宗寢疾輔國矯命罷華相今從輔國傳】戊申華罷為禮部尚書以載同平章事領度支轉運使如故【尚辰羊翻度徒洛翻】 建巳月庚戍朔【建巳月四月也】澤州刺史李抱玉破史朝義兵於城下【朝直遥翻】 壬子楚州刺史崔侁表稱有尼真如恍惚登天見上帝賜以寶玉十三枚【侁疎臻翻恍呼廣翻尼女夷翻惚音忽唐會要十三寶一曰玄黄天符如笏長八寸䦚三寸上圓下方近圓冇孔黄玉也二曰玉難毛文悉備白玉也三曰穀璧白玉也徑可五六寸其文粟粒無雕鑴之迹四曰西王母白環二枚白玉也徑六七寸五曰碧色寶圓而有光六曰如意寶珠圓如雞卵光如月七曰紅靺鞨大如巨栗赤如櫻桃八曰琅玕珠二枚長一寸二分九曰玉玦形如玉環四分缺十曰玉印大如半手斜長理如鹿形陷入印中以印物則鹿形著焉十一曰皇后採桑鉤長五六寸細如筯屈其末似金又似銀十二曰雷公石斧長四寸闊二寸無孔細緻如青玉十三曰缺凡十三寶置于日中皆白氣連天侁所臻翻】云中國有災以此鎮之羣臣表賀 甲寅上皇崩于神龍殿【神龍殿盖中宗於神龍間居之遂以名殿】年七十八乙卯遷坐于太極殿【坐徂卧翻神御坐也】上以寢疾發哀於内殿羣臣發哀於太極殿【内殿上居大明宫之寢殿也太極殿西内前殿大行所御】蕃官剺面割耳者四百餘人【剺里之翻】丙辰命苖晉卿攝冢宰上自仲春寢疾聞上皇登遐哀慕疾轉劇乃命太子監國【冢而隴翻監古銜翻】甲子制改元【改元寶應】復以建寅為正月月數皆如其舊【復扶又翻】赦天下 初張后與李輔國相表裏專權用事【見上卷乾元二年】晚年更有隙内射生使三原程元振黨於輔國【以宦官領射生手故曰内射生使】上疾篤后召太子謂曰李輔國久典禁兵制勑皆從之出擅逼遷聖皇【玄宗尊號曰聖皇天帝】其罪甚大所忌者吾與太子今主上彌留【書顧命曰病日臻既彌留言病日至一日愈留而不去體也】輔國陰與程元振謀作亂不可不誅太子泣曰陛下疾甚危二人皆陛下勲舊之臣一旦不告而誅之必致震驚恐不能堪也后曰然則太子姑歸吾更徐思之太子出后召越王係謂曰太子仁弱不能誅賊臣汝能之乎對曰能係乃命内謁者監段恒俊選宦官有勇力者二百餘人授甲於長生殿後【恒戶登翻】乙丑后以上命召太子元振知其謀密告輔國伏兵於陵霄門以俟之【雍録六典大明宫圖宫城北面玄武門之西有青霄門閣本大明宫圖作陵雲門】太子至以難告【難乃旦翻】太子曰必無是事主上疾亟召我我豈可畏死而不赴乎元振曰社稷事大太子必不可入乃以兵送太子於飛龍廐【飛龍厩仗内六閑之一也程大昌曰在玄武門外】且以甲卒守之是夜輔國元振勒兵三殿收捕越王係段恒俊及知内侍省事朱光輝等百餘人繫之以太子之命遷后于别殿時上在長生殿使者逼后下殿并左右數十人幽於後宫宦官宫人皆驚駭逃散丁卯上崩【年五十二】輔國等殺后并係及兖王僴【僴下赧翻 考異曰肅宗實錄曰張后因太子監國謀誅輔國其日使人以上命召太子語之太子不可乙丑后矯上命將喚太子程元振知之密告輔國丙寅元振與輔國夜勒兵於三殿前使人收捕越王及同謀内侍朱光輝段恒俊等百餘人繫之移皇后于别殿其夜六宫内人中官等驚駭奔走及明上崩代宗實錄曰乙丑皇后召上既夜輔國元振勒兵捕係幽后丁卯肅宗崩係傳乙丑后召太子丙寅夜元振輔國勒兵捕係幽后是日俱為輔國所害舊肅宗紀丁卯宣遺詔是日上崩代宗紀乙丑皇后矯詔召大子輔國元振衛從太子入飛龍厩以俟變是夕勒兵於三殿收係及朱光輝馬英俊等丁卯肅宗崩新本紀丙寅閑厩使李輔國飛龍厩副使程元振遷皇后于别殿殺越王係兖王僴是夜皇帝崩代宗録唐歷統紀係傳皆以段恒俊為馬英俊按張后以乙丑日召太子迨夜不至時必知冇變矣輔國等安能待至來夜然後勒兵收係等乎盖收係等在乙丑之夜也今從代宗實録舊代宗紀新舊傳皆云兖王僴寶應元年薨而代宗實錄羣臣議係僴之罪云二王同惡共扇姦謀盖僴亦預謀也今從之】是日輔國始引太子素服於九仙門與宰相相見【閣本大明宫圖宫城西面右銀臺門之北有九仙門又北轉東則凌雲門相息亮翻】敘上皇晏駕拜哭【敘自上皇晏駕後宫中多故不見輔臣】始行監國之令【命太子監國在甲子前而乙丑即有内變既定乃始行令監古咸翻】戊辰發大行皇帝喪於兩儀殿宣遺詔【肅宗崩于東内寢殿喪于西内内朝從上皇也上皇梓宫在西内前殿】己巳代宗即位 高力士遇赦還至朗州【上元元年高力士流巫州遇赦還至朗州自朗至京師尚二千二百五十九里赦者甲子赦也還從宣翻又音如字】聞上皇崩號慟嘔血而卒【力士流見上卷上元元年號戶刀翻卒子恤翻】 甲戍以皇子奉節王适為天下兵馬元帥【奉節縣名蜀先主改魚復縣為奉節縣帥所類翻】 李輔國恃功益横明謂上曰大家但居禁中外事聽老奴處分【於上前明言之無所忌惮横戶孟翻處昌呂翻分扶問翻】上内不能平以其方握禁兵外尊禮之乙亥號輔國為尚父而不名【齊太公輔周武王號師尚父今以其號寵中人】事無大小皆咨之羣臣出入皆先詣輔國亦晏然處之【史言李輔國凶愚處昌呂翻】以内飛龍廐副使程元振為左監門衛將軍知内侍省事朱光輝及内常侍啖庭瑶山人李唐等二十餘人皆流黔中【自朱光輝以下皆火行左右監古銜翻啖徒覧翻黔其今翻】 初李國貞治軍嚴【泊直之翻】朔方將士不樂【樂音洛】皆思郭子儀故王元振因之作亂子儀至軍元振自以為功子儀曰汝臨賊境【絳州東與河南接界時賊又據河陽河内故云然】輒害主將若賊乘其釁【釁隙也將即亮翻】無絳州矣吾為宰相豈受一卒之私邪五月庚辰收元振及其同謀四十人皆殺之【相息亮翻 考異曰實錄云子儀至軍撫循士衆潜問罪人得害國貞者王元禮等四十人為首者斬餘並决殺邠志曰七月郭公到朔方行營舊傳曰三月子儀辭赴鎮汾陽家傳曰建辰月十一日上都二十七日至絳州五月二日斬元振等三十人今元振名從諸書月日從家傳人數從實録】辛雲京聞之亦推按殺鄧景山者數十人誅之由是河東諸鎮率皆奉法【郭子儀誅王元振而河東諸鎮皆奉法僕固懷恩分河北諸州授田承嗣等以成藩鎮之禍用人可不謹哉】 壬午以李輔國為司空兼中書令 党項寇同官華原 甲申以平盧節度使侯希逸為平盧青淄等六州節度使【青淄齊沂密海六州淄州治淄川本漢般陽縣宋僑立清河郡及貝丘縣魏為東清河郡隋置淄州取淄水為名】由是青州節度有平盧之號 乙酉徙奉節王适為魯王【适古活翻】 追尊上母吳妃為皇太后【吳妃事肅宗于東宫生上而薨】 壬辰貶禮部尚書蕭華為峽州司馬元載希李輔國意【輔國以不相銜華】以罪誣之也 勑乾元大小錢皆一當一民始安之【民不便乾元二品錢見上卷乾元二年】 史朝義自圍宋州數月城中食盡將陷刺史李岑不知所為遂城果毅開封劉昌曰【易州有遂城府開封漢縣唐屬汴州漢故縣在今縣南五十里杜佑曰天寶以後邉帥怙寵便請署官易州遂城府坊州安臺府别將果毅之類每一制則同授千餘人】倉中猶有麴數千斤請屑食之不過二十日李太尉必救我【李太尉謂光弼】城東南隅最危昌請守之李光弼至臨淮諸將以朝義兵尚彊請南保揚州光弼曰朝廷倚我以為安危我復退縮朝廷何望【復扶又翻】且吾出其不意賊安知吾之衆寡遂徑趣徐州【趣七喻翻】使兖鄆節度使田神功進擊朝義大破之先是田神功既克劉展【去年正月神功克劉展先悉薦翻下同】留連揚州未還太子賓客尚衡與左羽林大將軍殷仲卿相攻于兖鄆 【考異曰衡上元元年為淄青節度使此年五月田神功自淄青移兖鄆六月衡自賓客為常侍七月仲卿自左羽林大將軍為光祿卿而得相攻於兖鄆者盖衡猶未離淄青仲卿亦在彼雖有新除官皆未肯入朝也】聞光弼至憚其威名神功遽還河南【此河南總言河南道】衡仲卿相繼入朝 【考異曰舊傳云朝義乘北印之勝寇申光等十三州自領精兵圍李岑於宋州將士皆懼請南保揚州光弼徑赴徐州以鎮之遣田神功擊敗之又曰初光弼將赴臨淮在道舁疾而行監軍使以袁晁方擾江淮光弼兵少請保潤州以避其鋒光弼不從徑往泗州光弼木至河南也田神功平劉展後逗留于楊府尚衡殷仲卿相攻于兖鄆來瑱旅拒于襄陽及光弼輕騎至徐州史朝義退走田神功遽歸河南尚衡殷仲卿來瑱皆懼其威名相繼赴闕按光弼既使田神功擊敗朝義則是神功已還也實録今年八月袁晁始陷台州借使當時已擾冮淮則自泗州往潤州不得謂避其鋒也今從新書本傳】光弼在徐州惟軍旅之事自决之其餘衆務悉委判官張傪【傪七感翻又倉含翻】傪吏事精敏區處如流【處昌呂翻】諸將白事光弼多令與傪議之諸將事傪如光弼由是軍中肅然東夏以寜【夏戶雅翻】先是田神功起偏禆為節度使【去年六月田神功自平盧兵馬使節度兖鄆】留前使判官劉位等於幕府神功皆平受其拜及見光弼與傪抗禮乃大驚徧拜位等曰神功出於行伍不知禮儀諸君亦胡為不言成神功之過乎【史言武大悍將可以禮化居其上者當以身作則行戶剛翻】丁酉赦天下 立皇子益昌王邈為鄭王【天寶改利州為益昌郡】延為慶王迥為韓王 來瑱聞徙淮西大懼上言淮西無糧請俟收麥而行又諷將吏留己【來瑱挾衆以要君欲再求免得乎】上欲姑息無事壬寅復以瑱為山南東道節度使【復扶又翻】飛龍副使程元振謀奪李輔國權密言於上請稍加<br />
<br />
  裁制六月己未解輔國行軍司馬及兵部尚書餘如故以元振代判元帥行軍司馬仍遷輔國出居外第【自肅宗時李輔國常居禁中内宅】於是道路相賀輔國始懼上表遜位辛酉罷輔國兼中書令進爵博陸王輔國入謝憤咽而言曰老奴事郎君不了請歸地下事先帝上猶慰諭而遣之【考異曰舊傳輔國欲入中書作謝表閽吏止之曰尚父罷相不應復入此門輔國氣憤而言曰老奴死罪事】<br />
<br />
  【郎君不了請歸地下事先帝上猶優詔答之按此乃對上之語非對閽吏之言也今從唐紀】 壬戌以兵部侍郎嚴武為西川節度使 襄鄧防禦使裴茙屯穀城【穀城漢筑陽縣地晉置義城郡及義城縣隋開皇十六年廢郡改縣曰穀城以其地有穀城山也】既得密勅即帥麾下二千人沿漢趣襄陽【帥讀曰率趣七喻翻】己巳陳于穀水北瑱以兵逆之問其所以來對曰尚書不受朝命故來若受代謹當釋兵瑱曰吾己蒙恩復留鎮此【復扶又翻】何受代之有因取勑及告身示之茙驚惑瑱與副使薛南陽縱兵夾擊大破之追擒茙於申口【金州洵陽縣有申口鎮】送京師賜死 【考異曰舊茙傳云瑱設具于江津以俟之茙初聲言假道入朝及見瑱即云奉代且欲視事瑱報曰瑱已奉恩命復任此茙惶惑喻其麾下曰此言必妄遂引射瑱軍因與瑱兵交戰茙軍大敗按瑱若設具相見則茙豈得遽射瑱軍而交戰今從瑱傳】乙亥以通州刺史劉晏【去年十一月晏貶通州】為戶部侍郎兼京兆尹充度支轉運鹽鐵鑄錢等使 秋七月壬辰以郭子儀都知朔方河東北庭潞儀澤沁陳鄭等節度行營【時以潞儀澤沁陳鄭為一鎮以李抱玉為節度使盖抱玉先以陳鄭節度使討賊在行營李光弼邙山之敗抱玉奔澤州陳鄭為賊所隔朝廷因使之節度潞儀沁澤四州】及興平等軍副元帥 癸巳劒南兵馬使徐知道反以兵守要害拒嚴武武不得進 八月桂州刺史邢濟討西原賊帥吳功曹等平之【帥所類翻】 己未徐知道為其將李忠厚所殺劒南悉平 乙丑山南東道節度使來瑱入朝謝罪上優待之 己巳郭子儀自河東入朝時程元振用事忌子儀功高任重數譛之於上子儀不自安表請解副元帥節度使上慰撫之子儀遂留京師 台州賊帥袁晁攻陷浙東諸州改元寶勝【考異曰柳璨正閏位歷宋庠紀元通譜皆改元昇國今從新書】民疲於賦歛者多歸之李光弼遣兵擊晁於衢州【衢州春秋時越始蔑之地秦以為太末縣漢分立新安縣晉改信安唐置衢州以三衢山名昔洪水?山為三道故曰三衢歛力贍翻】破之 乙亥徙魯王适為雍王【雍於用翻】 九月庚辰以來瑱為兵部尚書同平章事知山南東道節度使 乙未加程元振驃騎大將軍兼内侍監【驃匹妙翻騎奇寄翻】 左僕射裴冕為山陵使【方上之役唐置山陵使以宰相為之】議事有與程元振相違者丙申貶冕施州刺史 【考異曰代宗實録祕書監韓潁中書舍人劉烜善候星歷乾元中待詔翰林頗承恩顧又與李輔國昵狎時上軫憂山陵廣詢卜兆潁等不能精慎妄有否臧因是得罪配流嶺南既行賜死于路初冕為僕射數論時政遂兼御史大夫充山陵使以李輔國權重有恩乃奏輔國所親信劉烜為判官潜結輔國烜得罪乃連坐焉今從舊程元振傳】 上遣中使劉清潭使於回紇修舊好【好呼到翻】且徵兵討史朝義清潭至其庭回紇登里可汗已為朝義所誘云唐室繼有大喪今中原無主【因連有玄宗肅宗之喪遂誑以中原無主】可汗宜速來共收其府庫可汗信之清潭致勑書曰先帝雖棄天下今上繼統乃昔日廣平王與葉護共收兩京者也【事見二百二十卷至德二載】回統業已起兵至三城【即朔方三受降城】見州縣皆為丘墟有輕唐之志乃困辱清潭清潭遣使言狀且曰回紇舉國十萬衆至矣京師大駭上遣殿中監藥子昂往勞之於忻州南初毗伽闕可汗為登里求婚肅宗以僕固懷恩女妻之為登里可敦【勞力到翻為于偽翻妻七細翻】可汗請與懷恩相見懷恩時在汾州上令往見之懷恩為可汗言唐家恩信不可負【為于偽翻】可汗悦遣使上表請助國討朝義可汗欲自蒲關入由沙苑出潼關東向藥子昂說之曰關中數遭兵荒【說式芮翻數所角翻】州縣蕭條無以供擬恐可汗失望賊兵盡在洛陽請自土門略邢洺懷衛而南得其資財以充軍裝可汗不從又請自太行南下據河陰扼賊咽㗋【可從刋入聲汗音寒行戶剛翻咽音煙】亦不從又請自陜州太陽津度河【陜州陜縣北有太陽關黄河津濟之要也即左傳秦孟明伐晋自茅津濟封殽尸之路也亦曰陜津陜失冉翻】食太原倉粟與諸道俱進乃從之【隋置太原倉在河東界史言回紇所利在中國財寶而不敢輕與賊遇】袁晁陷信州【信州本吴鄱陽郡之葛陽縣陳改葛陽為弋陽唐乾元元年分饒州之弋陽衢】<br />
<br />
  【州之常山玉山及割建撫之地置信州治上饒縣以其旁下饒州故以名縣晁馳遥翻】 冬十月袁晁陷温州明州【温州永嘉郡治永嘉縣明州餘姚郡治鄮縣今之鄞縣是也】 以雍王适為天下兵馬元帥辛酉辭行以兼御史中丞藥子昂魏琚為左右廂兵馬使以中書舍人韋少華為判官給事中李進為行軍司馬會諸道節度使及回紇于陜州進討史朝義【雍於用翻适古活翻帥所類翻少始照翻使疏吏翻朝直遥翻】上欲以郭子儀為适副程元振魚朝恩等沮之而止【沮在呂翻】加朔方節度使僕固懷恩同平章事兼絳州刺史領諸軍節度行營以副适【絳州絳郡時朔方軍屯絳州故以懷恩領刺史為懷恩恃功畔援張本】上在東宫以李輔國專横心甚不平【横下孟翻】及嗣位以<br />
<br />
  輔國有殺張后之功不欲顯誅之壬戍夜盗入其第竊輔國之首及一臂而去 【考異曰舊傳云盗殺李輔國擕首臂而去統紀曰輔國悖于明皇上在東宫聞而頗怒及踐阼輔國又立功難于顯戮密令人刺之斷其首棄之圂中又斷其右臂馳祭泰陵中外莫測後杭州刺史杜濟話於人曰嘗識一武人為牙門將曰某即害尚父者今從舊傳】勑有司捕盗遣中使存問其家為刻木首葬之【為于偽翻】仍贈太傅【太傳三公】 丙寅上命僕固懷恩與母妻俱詣行營【時登里與懷恩之女俱來故使懷恩母妻詣行營以親結之】雍王适至陜州回紇可汗屯於河北【雍於用翻适古活翻陜失冉翻紇下沒翻可從刋入聲汗音寒陜州之河北也括地志曰陜州河北縣本漢大陽縣天寶元年太守李齊物開三門以利漕運得古刃有篆文曰平陸因更名平陸縣】适與僚屬從數十騎往見之可汗責适不拜舞藥子昂對以禮不當然回紇將軍車鼻曰唐天子與可汗約為兄弟可汗與雍王叔父也何得不拜舞子昂曰雍王天子長子今為元帥【騎奇計翻長知兩翻帥所類翻】安有中國儲君向外國可汗拜舞乎且兩宫在殯【兩宫謂上皇先帝時皆未葬】不應舞蹈力争久之車鼻遂引子昂魏琚韋少華李進各鞭一百以适年少未諳事遣歸營【應於陵翻少詩照翻諳烏含翻考異曰代宗實錄云雍王恭行詔命辭色不屈虜亦不敢失禮時人難之時官軍合圍將誅無禮王以東略之故止之又曰會中數萬人駭愕失色雍王正色叱之可汗遂退建中實錄曰上堅立不屈此盖史官虚美耳今從舊回紇傳】琚少華一夕而死戊辰諸軍發陜州僕固懷恩與回紇左殺為前鋒陜西節度使郭英乂【方鎮表上元元年改陜虢華節度為陜西節度使使疏吏翻】神策觀軍容使魚朝恩為殿【殿丁練翻】自澠池入【澠彌兖翻】潞澤節度使李抱玉自河陽入河南等道副元帥李光弼自陳留入【分道並入以攻洛陽】雍王留陜州 【考異曰代宗實錄戊辰元帥雍王帥僕固懷恩等諸軍及回紇兵馬進陜州東討留英乂朝恩為後殿是日又詔河東道節度使自澤州路入今從唐歷及舊朝義傳】辛未懷恩等軍于同軌【河南永寜縣後周之同軌縣地有同軌城】史朝義聞官軍將至謀於諸將阿史那承慶曰唐若獨與漢兵來宜悉衆與戰若與回紇俱來其鋒不可當宜退守河陽以避之朝義不從【朝直遥翻諸將即亮翻紇下沒翻】壬申官軍至洛陽北郊分兵取懷州癸酉拔之【去年邙山之敗河陽懷州皆陷于賊洛陽北郊在邙山外】乙亥官軍陳于横水【按舊書横水在洛陽北郊金人疆域圖孟津縣有横水店陳讀曰陣下恩陳陳於賊陳陳亦同】賊衆數萬立栅自固懷恩陳于西原以當之遣驍騎及回紇並南山出栅東北【驍堅堯翻騎奇計翻並步浪翻柵測革翻】表裏合擊大破之朝義悉其精兵十萬救之陳于昭覺寺官軍驟擊之殺傷甚衆而賊陳不動魚朝恩遣射生五百人力戰賊雖多死者陳亦如初鎮西節度使馬璘曰事急矣【犯陳而不能陷引退必敗故曰事急使疏吏翻璘離珍翻】遂單騎奮擊奪賊兩牌【牌古謂之楯晋宋之間謂之彭排南方以皮編竹為之以捍敵北人以木為之左傳樂祁以揚楯賈禍盖北方之用木也尚矣】突入萬衆中賊左右披靡【披普彼翻】大軍乘之而入賊衆大敗轉戰於石榴園老君廟賊又敗人馬相蹂踐填尚書谷【蹂人九翻尚辰羊翻】斬首六萬級捕虜二萬人朝義將輕騎數百東走【將即亮翻又音如字下等將同】懷恩進克東京及河陽城獲其中書令許叔冀王伷等【伷音胄】承制釋之懷恩留回紇可汗營于河陽使其子右廂兵馬使瑒及朔方兵馬使高輔成帥步騎萬餘乘勝逐朝義【紇下沒翻可從刋入聲汗音寒使疏吏翻瑒音暢又雉杏翻帥讀曰率下同騎奇計翻朝直遥翻】至鄭州再戰皆捷朝義至汴州其陳留節度使張獻誠閉門拒之朝義奔濮州獻城開門出降【汴皮面翻濮博木翻降戶剛翻下同】回紇入東京肆行殺略死者萬計火累旬不滅朔方神策軍亦以東京鄭汴汝州皆為賊境所過虜掠三月乃已【使郭李為帥安有是禍邪】比屋蕩盡士民皆衣紙【比毗必翻又毗至翻衣於既翻】回紇悉置所掠寶貨於河陽留其將安恪守之十一月丁丑露布至京師朝義自濮州北度河懷恩進攻滑州拔之追敗朝義於衛州【將即亮翻敗補邁翻】朝義睢陽節度使田承嗣等將兵四萬餘人與朝義合復來拒戰【睢音雖嗣祥吏翻復扶又翻】僕固瑒擊破之長驅至昌樂東朝義帥魏州兵來戰又敗走【昌樂漢古縣属魏州後唐避諱改為南樂樂音洛帥讀曰率】於是鄴郡節度使薛嵩以相衛洺邢四州降于陳鄭澤潞節度使李抱玉恒陽節度使張忠志以趙恒深定易五州降于河東節度使辛雲京【洺音名恒戶登翻 考異曰舊懷恩傳云嵩以相衛洺邢趙州降于李抱玉李寶臣以深恒定易四州降于雲京代宗實錄曰張忠志以趙定深恒易五州歸順又曰史思明授忠志恒趙節度使今從舊王武俊傳】嵩楚玉之子也【楚玉薛訥之弟】抱玉等已進軍入其營按其部伍嵩等皆受代居無何僕固懷恩皆令復位由是抱玉雲京疑懷恩有貳心各表言之朝廷密為之備懷恩亦上疏自理上慰勉之【令力丁翻朝直遥翻上時掌翻疏所去翻】辛巳制東京及河南北受偽官者一切不問【為青冀魏幽各據所有州縣以傳世張本】 己丑以戶部侍郎劉晏兼河南道水陸轉運都使【睿宗先天二年以李傑為陜州刺史充水陸運使水陸運使自此始也至開元二年間以傑除河南少尹充水陸運使天寶十二載陜郡太守崔無詖充使楊國忠充使水陸轉運都使始此】丁酉以張忠志為成德軍節度使統恒趙深定易五州賜姓李名寶臣初辛雲京引兵將出井陘常山禆將王武俊說寶臣曰今河東兵精銳出境遠闘不可敵也且吾以寡當衆以曲遇直戰則必離守則必潰公其圖之寶臣乃撤守備舉五州來降及復為節度使以武俊之策為善【說式芮翻復扶又翻】擢為先鋒兵馬使武俊本契丹也初名沒諾干【為王武俊夷張氏得成德張本契欺訖翻又音喫】郭子儀以僕固懷恩有平河朔功請以副元帥讓之己亥以懷恩為河北副元帥加左僕射兼中書令單于鎮北大都護朔方節度使史朝義走至貝州與其大將薛忠義等兩節度合僕固瑒追之至臨清【臨清漢清淵縣後魏改曰臨清唐属貝州九域志在魏州北一百五十里】朝義自衡水引兵三萬還攻之瑒設伏擊走之回紇又至官軍益振遂逐之大戰于下博東南【下博漢縣時属深州】賊大敗積尸擁流而下朝義奔莫州 【考異曰河洛春秋曰朝義戰敗走歸范陽途經衡水僕固瑒領蕃漢兵一十五萬追及朝義接戰敗之是夏涉秋苦雨陂湖流注河東兵馬使李竭誠成德軍將李令崇咸統精兵亦革面來王競為犄角其漳河及諸津渡船悉是虜獲朝義遣人致命竟不應續令散雇舟船並皆掠盡四路俱絶諸將或請戰或請降朝義不悦田承嗣上疏與朝義曰臣聞兵勢兩均成敗由將衆寡不敵全滅在權昔劉主破于白帝曹公敗于赤壁陸遜黄盖皆以權道取之今部統之師皆自疲頓主客勢倍勞逸力殊若驅而令戰未見其利請用車五十乘於古夏康王城北作三箇車營車上皆設棚排倒戈為禦每車甲士二人持兵而伏随軍子女羅於帳中每營輜重分列其次營後選二萬人布偃月陣凡敵衆我寡則設此陣左右有險亦設此陣左右奇軍亦設此陣各令猛將主之左者東南行右者西南行令去車營十里餘營前選精卒五千人雁行陣使之接戰不勝則退於偃月陣後前軍既却敵必至車營愛其珍玩必將攻取候其兵縱陣勢已分然後桴鼔齊鳴前後俱至貔貅奮勇鹵楯争先左軍西行右軍東邁皆取古城之南令首尾相屬伏兵之料敵必驚後軍之來自然斷絶前後既不相救中軍又遇精兵服色相亂不敗何待令文景義主左軍逹干義感主右軍足下自主中軍若其不捷老臣請以弱卒五千為足下吞之朝義覧疏大悦因用其計官軍敗績喪師三千餘級僕固瑒大震退師數十里由是朝義得達莫州朝義既敗官軍威聲復振凡所追集人莫已違鳩集舟舫并連牌栰先濟輜重兼及老弱方以軍南行若有攻擊僕固瑒令吏士各顧所部以抗其鋒朝義乃整師徒一時北濟僕固瑒亦連船艦宵濟趨之今從舊懷恩傳】懷恩都知兵馬使薛兼訓兵馬使郝庭玉與田神功辛雲京會於下博進圍朝義於莫州青淄節度使侯希逸繼至 十二月庚申初以太祖配天地【高祖武德元年制每歲圓丘方丘之祀以太祖景皇帝配高宗乾封二年以高祖太宗並配是時太常卿杜鴻漸等議以神堯為受命之主非始封之君不得為太祖以配天也太祖景皇帝始受封于唐即殷之契周之后稷也請以郊配天地從之】<br />
<br />
  代宗睿文孝武皇帝上之上<br />
<br />
  【初名俶後改名豫肅宗長子也登遐之後議上廟號曰世宗避太宗諱改曰代宗】<br />
<br />
  廣德元年【是年七月方改元事見下卷】春正月己卯追諡吳太后曰章敬皇后【吳太后上生母也】 癸未以國子祭酒劉晏為吏部尚書同平章事度支等使如故 初來瑱在襄陽程元振有所請託不從及為相【去年加來瑱同平章事】元振譛瑱言涉不順王仲昇在賊中以屈服得全賊平得歸與元振善奏瑱與賊合謀致仲昇陷賊壬寅瑱坐削官爵流播州賜死於路由是藩鎮皆切齒於元振【為諸鎮忌程元振不敢勤王張本】史朝義屢出戰皆敗田承嗣說朝義令親往幽州發<br />
<br />
  兵還救莫州承嗣自請留守莫州【說式芮翻守式又翻】朝義從之選精騎五千自北門犯圍而出朝義既去承嗣即以城降送朝義母妻子於官軍於是僕固瑒侯希逸薛兼訓等帥衆三萬追之及於歸義【歸義漢易縣地属涿郡北齊省入鄭縣武德五年置歸義及北義州貞觀元年州縣皆省八年復置歸義縣属幽州按宋白續通典唐歸義縣在瓦橋關北】與戰朝義敗走時朝義范陽節度使李懷仙已因中使駱奉仙請降遣兵馬使李抱忠將兵三千鎮范陽縣【范陽縣漢涿縣也為涿郡治所曹魏文帝改涿郡為范陽郡隋廢范陽郡復置涿郡於薊縣以涿縣属焉武德七年改涿縣為范陽縣仍属幽州天寶元年改幽州涿郡為范陽郡故薊城亦曰范陽史以縣字别之其地在薊南宋白曰南至莫州一百六十里】朝義至范陽不得入官軍將至朝義遣人諭抱忠以大軍留莫州輕騎來發兵救援之意因責以君臣之義抱忠對曰天不祚燕唐室復興【復扶又翻】今既歸唐矣豈可更為反覆獨不愧三軍邪大丈夫恥以詭計相圖願早擇去就以謀自全且田承嗣必已叛矣不然官軍何以得至此朝義大懼曰吾朝來未食獨不能以一餐相餉乎抱忠乃令人設食於城東於是范陽人在朝義麾下者並拜辭而去朝義涕泣而已獨與胡騎數百既食而去東奔廣陽【餉式亮翻令力丁翻朝直遥翻騎奇計翻檀州燕樂縣後魏置廣陽郡後齊廢郡而舊郡名猶存】廣陽不受欲北入奚契丹至温泉柵【據新舊書懷恩傳温泉柵在平州界石城縣東北契欺紇翻又音喫柵測革翻】李懷仙遣兵追及之朝義窮蹙縊於林中懷仙取其首以獻僕固懷恩與諸軍皆還甲辰朝義首至京師【縊於賜翻又於計翻還音旋又如字 考異曰河洛春秋曰朝義東投廣陽郡不受北取潞縣漁陽擬投兩蕃至榆關李懷仙使使招回却至漁陽過從潞縣至幽州城東阿婆門外於巫閭神廟中兄弟同被絞縊而死乃授首與駱奉仙經一日諸軍方知歸莫州城下舊僕固懷恩傳曰寶應二年三月朝義至平州石城縣温泉柵窮蹙走入長林自縊懷仙使妻弟徐有濟傳其首以獻史朝義傳二年正月李懷仙於莫州生擒之送欵來降梟首至闕下實錄寶應元年十一月己亥僕固懷恩上言幽州平河北州縣盡平史朝義為亂兵所戮傳首上都舊紀寶應二年十月河北州郡悉平李懷仙以幽州降田承嗣以魏州降沈既濟建中實錄二年正月賊將李懷仙擒朝義以降山東平唐歷正月甲辰李懷仙擒史朝義梟首獻至闕下盡以所管來降年代記寶應元年十二月己亥僕固懷恩上言史朝義為亂兵所殺傳首上都二年正月甲申朝義梟首至闕新紀廣德元年正月甲申朝義自殺其將李懷仙以幽州降按諸軍圍朝義於莫州已在去年十一月末而河洛春秋云圍城四十日懷恩舊傳亦云攻守月餘日然則朝義之死必在今年正月明矣諸書皆云朝義此年正月被殺而實錄在元年十一月舊記因之又脫十一月字懷恩傳誤以正月為三月甲申正月十日甲辰三十日也新本紀盖據年代記但年代記元年冬十一月己亥朝義死亦與實錄同若正月被殺不應十月首級已至長安疑甲申自殺甲辰傳首至闕新紀正用年代記甲辰至闕為自殺日未知何所據今從唐歷以甲辰傳首至京師】 閏月己酉夜有回紇十五人犯含光門突入鴻臚寺【紇下沒翻唐太極宫南面三門中曰朱雀門東曰安上門西曰含光門按朱雀門太極宫端門也雍錄曰承天門之南朱雀門之北宗廟社稷百僚廨舍列乎其間六省九寺一臺兩監十八衛以坊里準之此兩門内外南北各占兩坊不為民居自朱雀門南即市井邑厔各立坊巷以此觀之則朱雀街西兩坊百司庶府居之其門曰含光門朱雀街東兩坊亦百司庶府居之其門曰安上門也臚陵如翻】門司不敢遏 癸亥以史朝義降將薛嵩為相衛邢洺貝磁六州節度使【宋白曰磁州本漢廣平縣地周武帝於此置陽縣及成安郡隋開皇十年廢郡置磁州唐武德元年分相州置磁州貞觀元年州廢薛嵩既得節復表以相州之陽洺州之邯郸武安置磁州磁墻之翻相息亮翻】田承嗣為魏博德滄瀛五州都防禦使【魏州漢魏郡之地博州漢為東郡聊城縣德州漢平原郡地隋置德州因安德縣名之】李懷仙仍故地為幽州盧龍節度使【改范陽節度使為幽州節度使時平盧已䧟又兼盧龍節度使】時河北諸州皆已降嵩等迎僕固懷恩拜於馬首乞行間自効【降戶江翻行戶剛翻】懷恩亦恐賊平寵衰故奏留嵩等及李寶臣分帥河北自為黨援【帥所類翻下同】朝廷亦厭苦兵革苟冀無事因而授之【河北藩鎮自此強傲不可制矣】回紇登里可汗歸國其部衆所過抄掠【紇下沒翻可從刋入聲汗】<br />
<br />
  【音寒抄楚交翻】廪給小不如意輒殺人無所忌憚陳鄭澤潞節度使李抱玉欲遣官屬置頓人人辭憚【使疏吏翻】趙城尉馬燧獨請行【隋義寜元年分霍邑置趙城縣属晋州】比回紇將至【比必利翻及也】燧先遣人賂其渠帥約毋暴掠【師所類翻下同】帥遺之旗曰【遺惟季翻】有犯令者君自戮之燧取死囚為左右小有違令立斬之回紇相顧失色涉其境者皆拱手遵約束抱玉奇之燧因說抱玉曰燧與回紇言頗得其情僕固懷恩恃功驕蹇其子瑒好勇而輕【說式芮翻好呼到翻瑒音暢又雉杏翻輕區正翻】今内樹四帥【四帥謂田承嗣李寶臣李懷仙薛嵩】外交回紇必有窺河東澤潞之志宜深備之抱玉然之 初長安人梁崇義以羽林射生從來瑱鎮襄陽累遷右兵馬使崇義有勇力能卷鐵舒鈎沈毅寡言【瑱他甸翻卷讀曰捲沈持林翻】得衆心瑱之入朝也命諸將分戍諸州【朝直遥翻將即亮翻】瑱死戍者皆奔歸襄陽行軍司馬龐充將兵二千赴河南【盖先是來瑱使龎充赴河南行營會討史朝義】至汝州聞瑱死引兵還襲襄州左兵馬使李昭拒之充奔房州崇義自鄧州引戍兵歸與昭及副使薛南陽相讓為長【長知兩翻】久之不决衆皆曰兵非梁卿主之不可遂推崇義為帥崇義尋殺昭及南陽以其狀聞上不能討三月甲辰以崇義為襄州刺史山南東道節度留後【唐藩鎮命帥未授旌節者先以為節度留後為梁崇義以襄陽拒命而死張本】崇義奏改葬瑱為之立祠不居瑱聽事及正堂【為于偽翻聽讀曰廳】 辛酉葬至道大聖大明孝皇帝於泰陵【泰陵在同州奉先縣東北二十里金粟山】廟號玄宗庚午葬文明武德大聖大宣孝皇帝于建陵【建陵在京兆醴泉縣東北十八里武將山】廟號肅宗夏四月庚辰李光弼奏擒袁晁浙東皆平時晁聚衆近二十萬【近其靳翻】轉攻州縣光弼使部將張伯儀將兵討平之伯儀魏州人也 郭子儀數上言吐蕃党項不可忽宜早為之備【不能用郭子儀之言為二虜入京師張本數所角翻上時掌翻下同】 辛丑遣兼御史大夫李之芳等使於吐蕃為虜所留二年乃得歸【史䆒言之】 羣臣三上表請立太子五月癸卯詔許俟秋成議之 丁卯制分河北諸州以幽莫媯檀平薊為幽州管恒定趙深易為成德軍管相貝邢洺為相州管魏博德為魏州管滄棣冀瀛為青淄管懷衛河陽為澤潞管【自田承嗣李靈耀相繼叛亂諸鎮所管不復守此制】 六月癸酉禮部侍郎華隂楊綰上疏以為古之選士必取行實【行下孟翻下行著同】近世專尚文辭自隋煬帝始置進士科猶試策而已至高宗時考功員外郎劉思立始奏進士加雜文明經加帖從此積弊轉而成俗朝之公卿以此待士【朝直遥翻】家之長老以此訓子【長知兩翻】其明經則誦帖括以求僥倖【帖括者舉人因試帖遂括取稡會為一書相傳習誦之以應試謂之帖括】又舉人皆令投牒自應如此欲其返淳朴崇亷讓何可得也請令縣令察孝亷取行著鄉閭學知經術者薦之於州【行下孟翻】刺史考試升之於省任各占一經【占之贍翻】朝廷擇儒學之士問經義二十條對策三道上第即注官中第得出身下第罷歸又道舉亦非理國【置道舉見二百十四卷開元二十五年】望與明經進士並停上命諸司通議給事中李栖筠左丞賈至京兆尹嚴武並與綰同至議以為今試學者以帖字為精通考文者以聲病為是非【聲病謂以平上去入四聲緝而成文音從文順謂之聲反是則謂之病】風流頹敝誠當釐改然自東晉以來人多僑寓士居鄉土百無一二請兼廣學校保桑梓者鄉里舉焉在流寓者庠序推焉勑禮部具條目以聞綰又請置五經秀才科 庚寅以魏博都防禦使田承嗣為節度使承嗣舉管内戶口壯者皆籍為兵惟使老弱者耕稼數年間有衆十萬又選其驍健者萬人自衛謂之牙兵【魏牙兵始此迄於梁唐魏以之強亦以之亡】 同華節度使李懷讓為程元振所譛恐懼自殺【乾元元年置陜虢華節度使上元元年改陜西節度使分河中之同州與華州為同華節度使華戶化翻】<br />
<br />
  資治通鑑卷二百二十二<br />
<br />
<史部,編年類,資治通鑑>  <br>
   </div> 

<script src="/search/ajaxskft.js"> </script>
 <div class="clear"></div>
<br>
<br>
 <!-- a.d-->

 <!--
<div class="info_share">
</div> 
-->
 <!--info_share--></div>   <!-- end info_content-->
  </div> <!-- end l-->

<div class="r">   <!--r-->



<div class="sidebar"  style="margin-bottom:2px;">

 
<div class="sidebar_title">工具类大全</div>
<div class="sidebar_info">
<strong><a href="http://www.guoxuedashi.com/lsditu/" target="_blank">历史地图</a></strong>  
<a href="http://www.880114.com/" target="_blank">英语宝典</a>  
<a href="http://www.guoxuedashi.com/13jing/" target="_blank">十三经检索</a> 
<br><strong><a href="http://www.guoxuedashi.com/gjtsjc/" target="_blank">古今图书集成</a></strong> 
<a href="http://www.guoxuedashi.com/duilian/" target="_blank">对联大全</a> <strong><a href="http://www.guoxuedashi.com/xiangxingzi/" target="_blank">象形文字典</a></strong> 

<br><a href="http://www.guoxuedashi.com/zixing/yanbian/">字形演变</a>  <strong><a href="http://www.guoxuemi.com/hafo/" target="_blank">哈佛燕京中文善本特藏</a></strong>
<br><strong><a href="http://www.guoxuedashi.com/csfz/" target="_blank">丛书&方志检索器</a></strong> <a href="http://www.guoxuedashi.com/yqjyy/" target="_blank">一切经音义</a>  

<br><strong><a href="http://www.guoxuedashi.com/jiapu/" target="_blank">家谱族谱查询</a></strong>  <strong><a href="http://shufa.guoxuedashi.com/sfzitie/" target="_blank">书法字帖欣赏</a></strong> 
<br>

</div>
</div>


<div class="sidebar" style="margin-bottom:0px;">

<font style="font-size:22px;line-height:32px">QQ交流群9:489193090</font>


<div class="sidebar_title">手机APP 扫描或点击</div>
<div class="sidebar_info">
<table>
<tr>
	<td width=160><a href="http://m.guoxuedashi.com/app/" target="_blank"><img src="/img/gxds-sj.png" width="140"  border="0" alt="国学大师手机版"></a></td>
	<td>
<a href="http://www.guoxuedashi.com/download/" target="_blank">app软件下载专区</a><br>
<a href="http://www.guoxuedashi.com/download/gxds.php" target="_blank">《国学大师》下载</a><br>
<a href="http://www.guoxuedashi.com/download/kxzd.php" target="_blank">《汉字宝典》下载</a><br>
<a href="http://www.guoxuedashi.com/download/scqbd.php" target="_blank">《诗词曲宝典》下载</a><br>
<a href="http://www.guoxuedashi.com/SiKuQuanShu/skqs.php" target="_blank">《四库全书》下载</a><br>
</td>
</tr>
</table>

</div>
</div>


<div class="sidebar2">
<center>


</center>
</div>

<div class="sidebar"  style="margin-bottom:2px;">
<div class="sidebar_title">网站使用教程</div>
<div class="sidebar_info">
<a href="http://www.guoxuedashi.com/help/gjsearch.php" target="_blank">如何在国学大师网下载古籍?</a><br>
<a href="http://www.guoxuedashi.com/zidian/bujian/bjjc.php" target="_blank">如何使用部件查字法快速查字?</a><br>
<a href="http://www.guoxuedashi.com/search/sjc.php" target="_blank">如何在指定的书籍中全文检索?</a><br>
<a href="http://www.guoxuedashi.com/search/skjc.php" target="_blank">如何找到一句话在《四库全书》哪一页?</a><br>
</div>
</div>


<div class="sidebar">
<div class="sidebar_title">热门书籍</div>
<div class="sidebar_info">
<a href="/so.php?sokey=%E8%B5%84%E6%B2%BB%E9%80%9A%E9%89%B4&kt=1">资治通鉴</a> <a href="/24shi/"><strong>二十四史</strong></a>&nbsp; <a href="/a2694/">野史</a>&nbsp; <a href="/SiKuQuanShu/"><strong>四库全书</strong></a>&nbsp;<a href="http://www.guoxuedashi.com/SiKuQuanShu/fanti/">繁体</a>
<br><a href="/so.php?sokey=%E7%BA%A2%E6%A5%BC%E6%A2%A6&kt=1">红楼梦</a> <a href="/a/1858x/">三国演义</a> <a href="/a/1038k/">水浒传</a> <a href="/a/1046t/">西游记</a> <a href="/a/1914o/">封神演义</a>
<br>
<a href="http://www.guoxuedashi.com/so.php?sokeygx=%E4%B8%87%E6%9C%89%E6%96%87%E5%BA%93&submit=&kt=1">万有文库</a> <a href="/a/780t/">古文观止</a> <a href="/a/1024l/">文心雕龙</a> <a href="/a/1704n/">全唐诗</a> <a href="/a/1705h/">全宋词</a>
<br><a href="http://www.guoxuedashi.com/so.php?sokeygx=%E7%99%BE%E8%A1%B2%E6%9C%AC%E4%BA%8C%E5%8D%81%E5%9B%9B%E5%8F%B2&submit=&kt=1"><strong>百衲本二十四史</strong></a>  <a href="http://www.guoxuedashi.com/so.php?sokeygx=%E5%8F%A4%E4%BB%8A%E5%9B%BE%E4%B9%A6%E9%9B%86%E6%88%90&submit=&kt=1"><strong>古今图书集成</strong></a>
<br>

<a href="http://www.guoxuedashi.com/so.php?sokeygx=%E4%B8%9B%E4%B9%A6%E9%9B%86%E6%88%90&submit=&kt=1">丛书集成</a> 
<a href="http://www.guoxuedashi.com/so.php?sokeygx=%E5%9B%9B%E9%83%A8%E4%B8%9B%E5%88%8A&submit=&kt=1"><strong>四部丛刊</strong></a>  
<a href="http://www.guoxuedashi.com/so.php?sokeygx=%E8%AF%B4%E6%96%87%E8%A7%A3%E5%AD%97&submit=&kt=1">說文解字</a> <a href="http://www.guoxuedashi.com/so.php?sokeygx=%E5%85%A8%E4%B8%8A%E5%8F%A4&submit=&kt=1">三国六朝文</a>
<br><a href="http://www.guoxuedashi.com/so.php?sokeytm=%E6%97%A5%E6%9C%AC%E5%86%85%E9%98%81%E6%96%87%E5%BA%93&submit=&kt=1"><strong>日本内阁文库</strong></a> <a href="http://www.guoxuedashi.com/so.php?sokeytm=%E5%9B%BD%E5%9B%BE%E6%96%B9%E5%BF%97%E5%90%88%E9%9B%86&ka=100&submit=">国图方志合集</a> <a href="http://www.guoxuedashi.com/so.php?sokeytm=%E5%90%84%E5%9C%B0%E6%96%B9%E5%BF%97&submit=&kt=1"><strong>各地方志</strong></a>

</div>
</div>


<div class="sidebar2">
<center>

</center>
</div>
<div class="sidebar greenbar">
<div class="sidebar_title green">四库全书</div>
<div class="sidebar_info">

《四库全书》是中国古代最大的丛书,编撰于乾隆年间,由纪昀等360多位高官、学者编撰,3800多人抄写,费时十三年编成。丛书分经、史、子、集四部,故名四库。共有3500多种书,7.9万卷,3.6万册,约8亿字,基本上囊括了古代所有图书,故称“全书”。<a href="http://www.guoxuedashi.com/SiKuQuanShu/">详细>>
</a>

</div> 
</div>

</div>  <!--end r-->

</div>
<!-- 内容区END --> 

<!-- 页脚开始 -->
<div class="shh">

</div>

<div class="w1180" style="margin-top:8px;">
<center><script src="http://www.guoxuedashi.com/img/plus.php?id=3"></script></center>
</div>
<div class="w1180 foot">
<a href="/b/thanks.php">特别致谢</a> | <a href="javascript:window.external.AddFavorite(document.location.href,document.title);">收藏本站</a> | <a href="#">欢迎投稿</a> | <a href="http://www.guoxuedashi.com/forum/">意见建议</a> | <a href="http://www.guoxuemi.com/">国学迷</a> | <a href="http://www.shuowen.net/">说文网</a><script language="javascript" type="text/javascript" src="https://js.users.51.la/17753172.js"></script><br />
  Copyright &copy; 国学大师 古典图书集成 All Rights Reserved.<br>
  
  <span style="font-size:14px">免责声明:本站非营利性站点,以方便网友为主,仅供学习研究。<br>内容由热心网友提供和网上收集,不保留版权。若侵犯了您的权益,来信即刪。scp168@qq.com</span>
  <br />
ICP证:<a href="http://www.beian.miit.gov.cn/" target="_blank">鲁ICP备19060063号</a></div>
<!-- 页脚END --> 
<script src="http://www.guoxuedashi.com/img/plus.php?id=22"></script>
<script src="http://www.guoxuedashi.com/img/tongji.js"></script>

</body>
</html>
