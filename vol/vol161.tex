<!DOCTYPE html PUBLIC "-//W3C//DTD XHTML 1.0 Transitional//EN" "http://www.w3.org/TR/xhtml1/DTD/xhtml1-transitional.dtd">
<html xmlns="http://www.w3.org/1999/xhtml">
<head>
<meta http-equiv="Content-Type" content="text/html; charset=utf-8" />
<meta http-equiv="X-UA-Compatible" content="IE=Edge,chrome=1">
<title>資治通鑒_162-資治通鑑卷一百六十一_162-資治通鑑卷一百六十一</title>
<meta name="Keywords" content="資治通鑒_162-資治通鑑卷一百六十一_162-資治通鑑卷一百六十一">
<meta name="Description" content="資治通鑒_162-資治通鑑卷一百六十一_162-資治通鑑卷一百六十一">
<meta http-equiv="Cache-Control" content="no-transform" />
<meta http-equiv="Cache-Control" content="no-siteapp" />
<link href="/img/style.css" rel="stylesheet" type="text/css" />
<script src="/img/m.js?2020"></script> 
</head>
<body>
 <div class="ClassNavi">
<a  href="/24shi/">二十四史</a> | <a href="/SiKuQuanShu/">四库全书</a> | <a href="http://www.guoxuedashi.com/gjtsjc/"><font  color="#FF0000">古今图书集成</font></a> | <a href="/renwu/">历史人物</a> | <a href="/ShuoWenJieZi/"><font  color="#FF0000">说文解字</a></font> | <a href="/chengyu/">成语词典</a> | <a  target="_blank"  href="http://www.guoxuedashi.com/jgwhj/"><font  color="#FF0000">甲骨文合集</font></a> | <a href="/yzjwjc/"><font  color="#FF0000">殷周金文集成</font></a> | <a href="/xiangxingzi/"><font color="#0000FF">象形字典</font></a> | <a href="/13jing/"><font  color="#FF0000">十三经索引</font></a> | <a href="/zixing/"><font  color="#FF0000">字体转换器</font></a> | <a href="/zidian/xz/"><font color="#0000FF">篆书识别</font></a> | <a href="/jinfanyi/">近义反义词</a> | <a href="/duilian/">对联大全</a> | <a href="/jiapu/"><font  color="#0000FF">家谱族谱查询</font></a> | <a href="http://www.guoxuemi.com/hafo/" target="_blank" ><font color="#FF0000">哈佛古籍</font></a> 
</div>

 <!-- 头部导航开始 -->
<div class="w1180 head clearfix">
  <div class="head_logo l"><a title="国学大师官网" href="http://www.guoxuedashi.com" target="_blank"></a></div>
  <div class="head_sr l">
  <div id="head1">
  
  <a href="http://www.guoxuedashi.com/zidian/bujian/" target="_blank" ><img src="http://www.guoxuedashi.com/img/top1.gif" width="88" height="60" border="0" title="部件查字,支持20万汉字"></a>


<a href="http://www.guoxuedashi.com/help/yingpan.php" target="_blank"><img src="http://www.guoxuedashi.com/img/top230.gif" width="600" height="62" border="0" ></a>


  </div>
  <div id="head3"><a href="javascript:" onClick="javascript:window.external.AddFavorite(window.location.href,document.title);">添加收藏</a>
  <br><a href="/help/setie.php">搜索引擎</a>
  <br><a href="/help/zanzhu.php">赞助本站</a></div>
  <div id="head2">
 <a href="http://www.guoxuemi.com/" target="_blank"><img src="http://www.guoxuedashi.com/img/guoxuemi.gif" width="95" height="62" border="0" style="margin-left:2px;" title="国学迷"></a>
  

  </div>
</div>
  <div class="clear"></div>
  <div class="head_nav">
  <p><a href="/">首页</a> | <a href="/ShuKu/">国学书库</a> | <a href="/guji/">影印古籍</a> | <a href="/shici/">诗词宝典</a> | <a   href="/SiKuQuanShu/gxjx.php">精选</a> <b>|</b> <a href="/zidian/">汉语字典</a> | <a href="/hydcd/">汉语词典</a> | <a href="http://www.guoxuedashi.com/zidian/bujian/"><font  color="#CC0066">部件查字</font></a> | <a href="http://www.sfds.cn/"><font  color="#CC0066">书法大师</font></a> | <a href="/jgwhj/">甲骨文</a> <b>|</b> <a href="/b/4/"><font  color="#CC0066">解密</font></a> | <a href="/renwu/">历史人物</a> | <a href="/diangu/">历史典故</a> | <a href="/xingshi/">姓氏</a> | <a href="/minzu/">民族</a> <b>|</b> <a href="/mz/"><font  color="#CC0066">世界名著</font></a> | <a href="/download/">软件下载</a>
</p>
<p><a href="/b/"><font  color="#CC0066">历史</font></a> | <a href="http://skqs.guoxuedashi.com/" target="_blank">四库全书</a> |  <a href="http://www.guoxuedashi.com/search/" target="_blank"><font  color="#CC0066">全文检索</font></a> | <a href="http://www.guoxuedashi.com/shumu/">古籍书目</a> | <a   href="/24shi/">正史</a> <b>|</b> <a href="/chengyu/">成语词典</a> | <a href="/kangxi/" title="康熙字典">康熙字典</a> | <a href="/ShuoWenJieZi/">说文解字</a> | <a href="/zixing/yanbian/">字形演变</a> | <a href="/yzjwjc/">金 文</a> <b>|</b>  <a href="/shijian/nian-hao/">年号</a> | <a href="/diming/">历史地名</a> | <a href="/shijian/">历史事件</a> | <a href="/guanzhi/">官职</a> | <a href="/lishi/">知识</a> <b>|</b> <a href="/zhongyi/">中医中药</a> | <a href="http://www.guoxuedashi.com/forum/">留言反馈</a>
</p>
  </div>
</div>
<!-- 头部导航END --> 
<!-- 内容区开始 --> 
<div class="w1180 clearfix">
  <div class="info l">
   
<div class="clearfix" style="background:#f5faff;">
<script src='http://www.guoxuedashi.com/img/headersou.js'></script>

</div>
  <div class="info_tree"><a href="http://www.guoxuedashi.com">首页</a> > <a href="/SiKuQuanShu/fanti/">四库全书</a>
 > <h1>资治通鉴</h1> <!--         下载:【右键另存为】即可 --></div>
  <div class="info_content zj clearfix">
  
<div class="info_txt clearfix" id="show">
<center style="font-size:24px;">162-資治通鑑卷一百六十一</center>
    資治通鑑卷一百六十一 宋 司馬光 撰<br />
<br />
  胡三省 音註<br />
<br />
  梁紀十七【著雍執徐一年】<br />
<br />
  高祖武皇帝十七<br />
<br />
  太清二年春正月己亥慕容紹宗以鐵騎五千夹擊侯景【承上卷上年紹宗與景相持事故不書東魏】景誑其衆曰汝輩家屬已為高澄所殺衆信之【盖前乎此時景以此言誑衆也誑居况翻】紹宗遥呼曰【呼火故翻】汝輩家屬並完若歸官勲如舊【歸謂復歸東魏官者各人先所居之官勲勲階也】被髮向北斗為誓【質北斗為誓以明其言之不欺被皮義翻】景士卒不樂南渡【樂音洛】其將暴顯等各帥所部降於紹宗【暴顯去年春為侯景所執將即亮翻帥讀曰率降戶江翻】景衆大潰争赴渦水【渦音戈】水為之不流【為于偽翻】景與腹心數騎自硖石濟淮稍收散卒得步騎八百人【騎奇寄翻】南過小城人登陴詬之曰跛奴【侯景右足偏短故詬為跛奴陴頻彌翻詬苦侯翻跛普我翻】欲何為邪景怒破城殺詬者而去晝夜兼行追軍不敢逼 【考異曰典略云晝息夜行追軍漸逼今從梁書】使謂紹宗曰景若就擒公復何用紹宗乃縱之【人臣苟有才必養寇以自資東魏之世彭樂慕容紹宗同一轍耳復扶又翻】 辛丑以尚書僕射謝舉為尚書令守吏部尚書王克為僕射 甲辰豫州刺史羊鴉仁以東魏軍漸逼稱糧運不繼棄懸瓠還義陽殷州刺史羊思達亦棄項城走【去年使羊鴉仁鎮懸瓠羊思達鎮項城 考異曰典畧在六月今從梁帝紀】東魏人皆據之上怒責讓鴉仁鴉仁懼啟申後期頓軍淮上【不敢歸義陽】 侯景既敗不知所適時鄱陽王範除南豫州刺史未至【去年遣蕭淵明攻彭城以範代鎮夀陽時猶未至】馬頭戍主劉神茂素為監州事韋黯所不容【監工衘翻】聞景至故往候之【有意見之為故鄭玄曰古者謂候為進孔穎達曰古時謂迎客為進漢時謂迎客為候今按經傳迎客為進則進使者而問故之類是也迎客為候則鄭注周禮人云候候迎賓客之來是也】景問曰夀陽去此不遠城池險固欲往投之韋黯其納我乎神茂曰黯雖據城是監州耳王若馳至近郊彼必出迎因而執之可以集事得城之後徐以啟聞朝廷喜王南歸必不責也景執其手曰天教也神茂請帥步騎百人先為鄉導【帥讀曰率鄉讀曰嚮】壬子景夜至夀陽城下韋黯以為賊也授甲登陴【陴頻彌翻】景遣其徒告曰河南王戰敗來投此鎮願速開門黯曰既不奉敇不敢聞命景謂神茂曰事不諧矣神茂曰黯懦而寡智可說下也【說式芮翻】乃遣夀陽徐思玉入見黯曰【徐思玉本夀陽人仕於東魏今随侯景北來】河南王朝廷所重君所知也今失利來投何得不受黯曰吾之受命惟知守城河南自敗何預吾事思玉曰國家付君以閫外之略今君不肯開城若魏兵來至河南為魏所殺君豈能獨存何顔以見朝廷黯然之思玉出報景大悦曰活我者卿也癸丑黯開門納景景遣其將分守四門詰責黯將斬之【將即亮翻下同詰去吉翻】既而撫手大笑置酒極歡黯叡之子也【合肥之役黯請叡下城避箭其懦闇可知矣然使黯能拒景梁朝亦將敇黯納之】朝廷聞景敗未得審問或云景與將士盡沒上下咸以為憂侍中太子詹事何敬容詣東宫太子曰淮北始更有信侯景定得身免不如所傳敬容曰得景遂死深為朝廷之福太子失色問其故敬容曰景翻覆叛臣終當亂國太子於玄圃自講老莊【自蕭齊以來東宫有玄圃崑崙之山三級下曰樊桐二曰玄圃三曰層城太帝之所居東宫次於帝居故立玄圃】敬容謂學士吳孜曰【梁祕書省有撰史學士】昔西晉祖尚玄虚使中原淪於胡羯【事見晉紀】今東宫復爾江南亦將為戎乎【何敬容雖不能優游於文義其識則過於梁朝諸臣矣復扶又翻下景復復敕乃復故復同】甲寅景遣儀同三司于子悦馳以敗聞併自求貶削優詔不許景復求資給上以景兵新破未忍移易乙卯即以景為南豫州牧本官如故更以鄱陽王範為合州刺史鎮合肥【更工衡翻】光禄大夫蕭介上表諫曰竊聞侯景以渦陽敗績隻馬歸命【渦音戈】陛下不悔前禍復敇容納臣聞凶人之性不移天下之惡一也昔呂布殺丁原以事董卓終誅董而為賊【事見漢靈獻二帝紀】劉牢反王恭以歸晉還背晉以構妖【事見晉安帝紀妖於驕翻】何者狼子野心終無馴狎之性養虎之喻必見飢噬之禍矣侯景以凶狡之才荷高歡卵翼之遇【左傳楚令尹子西曰勝如卵予翼而長之荷下可翻】位忝台司任居方伯然而高歡墳土未乾【乾音干】即還反噬逆力不逮乃復逃死關西宇文不容故復投身於我陛下前者所以不逆細流【李斯上秦王書曰江海不擇細流故能就其深】正欲比屬國降胡以討匈奴【漢邉郡置屬國以處降胡使偵伺匈奴降戶江翻】冀獲一戰之效耳今既亡師失地直是境上之匹夫陛下愛匹夫而棄與國【與國謂東魏】若國家猶待其更鳴之晨歲暮之效臣竊惟侯景必非歲暮之臣【惟思也】棄鄉國如脱屣背君親如遺芥【背蒲妹翻】豈知遠慕聖德為江淮之純臣乎事迹顯然無可致惑臣朽老疾侵不應干預朝政【朝直遥翻】但楚囊將死有城郢之忠【左傳楚令尹子囊將死遺言子庚必城郢君子謂子囊忠將死不忘衛社稷】衛魚臨亡亦有尸諫之節【孔子家語曰衛大夫蘧伯玉賢靈公不用彌子瑕不肖反任之史魚驟諫不從將卒命其子曰吾不能進蘧伯玉退彌子瑕是不能正君也生不能正君死無以成禮我死汝置屍牖下於我畢矣其子從之靈公弔焉怪而問之其子以告公曰是寡人之過也命之殯於客位進蘧伯玉退彌子瑕孔子聞之曰古之烈諫者死則己矣未有若史魚死而屍諫忠感其君者也】臣忝為宗室遺老敢忘劉向之心【劉向事見三十卷漢成帝陽朔二年】上歎息其忠然不能用介思話之孫也【宋元嘉間蕭思話歷當方任按新唐書宰相世系表介與帝同十三世祖後漢中山相苞】 己未東魏大將軍澄朝于鄴【朝直遥翻下同】 魏以開府儀同三司趙貴為司空 魏皇孫生大赦 二月東魏殺其南兖州刺史石長宣討侯景之黨也【石長宣書官者表其以南兖州附侯景也不可以春秋書法觀之】其餘為景所脅從者皆赦之 東魏既得懸瓠項城悉復舊境大將軍澄數遣書移【移謂移檄也數所角翻】復求通好朝廷未之許澄謂貞陽侯淵明曰先王與梁主和好十有餘年【復扶又翻好呼到翻下舊好同】聞彼禮佛文云奉為魏主并及先王【為于偽翻言為魏主君臣祈福也】此乃梁主厚意不謂一朝失信致此紛擾知非梁主本心當是侯景扇動耳宜遣使諮論【使疏吏翻下同】若梁主不忘舊好吾亦不敢違先王之意諸人並即遣還侯景家屬亦當同遣淵明乃遣省事夏侯僧辯奉啟於上稱勃海王弘厚長者若更通好當聽淵明還上得啟流涕【此所謂婦人之仁也帝於是堕高澄數中矣】與朝臣議之右衛將軍朱异御史中丞張綰等皆曰静寇息民和實為便司農卿傳岐獨曰高澄何事須和必是設間【异羊至翻間古莧翻】故命貞陽遣使欲令侯景自疑景意不安必圖禍亂若許通好正堕其計中【侯景之反覆何敬容蕭介知之高澄之姦詐傅岐知之梁朝非果無人也武帝不能决擇而用之耳】异等固執宜和上亦厭用兵乃從异言賜淵明書曰知高大將軍禮汝不薄省啟甚以慰懷當别遣行人重敦鄰睦【省悉景翻重直用翻】僧辯還過夀陽侯景竊訪知之攝問具服【攝問收録其人而問之也】乃寫荅淵明之書陳啟於上曰高氏心懷鴆毒怨盈北土人願天從歡身殞越【謂人所祝願天從而殺之】子澄嗣惡計滅待時所以昧此一勝者【謂渦陽之勝也】盖天蕩澄心以盈凶毒耳【左傳楚武王將死告其夫人鄧曼曰余心蕩鄧曼曰王禄盡乎盈而蕩天之道也杜預注曰蕩動散也】澄苟行合天心【行下孟翻又如字】腹心無疾又何急急奉璧求和豈不以秦兵扼其喉【秦兵謂西魏之兵西魏據有關西故曰秦兵】胡騎迫其背【胡騎謂柔然之兵】故甘辭厚幣取安大國臣聞一日縱敵數世之患【晉先軫之言】何惜高澄一豎以弃億兆之心竊以北魏安彊莫過天監之始鍾離之役匹馬不歸【鍾離之戰見一百四十六卷天監六年】當其彊也陛下尚伐而取之及其弱也反慮而和之舍已成之功縱垂死之虜使其假命彊梁以遺後世【舍讀曰捨遺于季翻】非直愚臣扼腕實亦志士痛心昔伍相奔吳楚邦卒滅【左傳楚殺伍奢其子奔吳吳王闔閭用之破楚入郢腕烏貫翻相息亮翻卒子恤翻】陳平去項劉氏用興【見漢高帝紀】臣雖才劣古人心同往事誠知高澄忌賈在翟惡會居秦【左傳晉靈公之初賈季奔翟随會奔秦秦人用其謀晉人患之六卿相見於諸浮趙宣子曰隨會在秦賈季在翟難日至矣將若之何翟與狄同惡烏路翻】求盟請和冀除其患若臣死有益萬殞無辭唯恐千載有穢良史【觀景此言其氣悖矣】景又致書於朱异餉金三百兩异納金而不通其啟【史言朱异昧利而不顧患】己卯上遣使弔澄景又啟曰臣與高氏舋隙已深仰憑威靈期雪讐恥今陛下復與高氏連和使臣何地自處【此乃侯景由衷之言使疏吏翻舋許覲翻復扶又翻處昌呂翻】乞申後戰宣暢皇威上報之曰朕與公大義已定豈有成而相納敗而相弃乎今高氏有使求和朕亦更思偃武進退之宜國有常制公但清静自居無勞慮也景又啟曰臣今蓄糧聚衆秣馬潜戈指日計期克清趙魏不容軍出無名故願以陛下為主耳【觀景此言亦那可忍】今陛下弃臣遐外南北復通將恐微臣之身不免高氏之手【景言至此辭意迫切獸窮則搏能無及乎復扶又翻下勞復同】上又報曰朕為萬乘之主豈可失信於一物想公深得此心不勞復有啟也【既斷來章景又生心矣乘繩證翻】景乃詐為鄴中書求以貞陽侯易景上將許之舍人傅岐曰【傅岐先兼中書通事舍人累遷太僕司農卿兼舍人如故】侯景以窮歸義弃之不祥且百戰之餘寧肯束手就縶謝舉朱异曰景奔敗之將一使之力耳【將即亮翻使疏吏翻】上從之復書曰貞陽旦至侯景夕返景謂左右曰我固知吴老公薄心腸【帝之情態於此畢露而帝不自知也嗚呼】王偉說景曰今坐聽亦死【言坐而聽梁朝所為亦必至于死說式芮翻】舉大事亦死唯王圖之於是始為反計屬城居民悉召募為軍士輒停責市估及田租【市估應商旅之物入市者估其直而收税田租計畝所出常租】百姓子女悉以配將士【景之反謀彰灼如此梁之君臣若罔聞知其亡宜矣】 三月癸巳東魏以太尉襄城王旭為大司馬【旭吁玉翻】開府儀同三司高岳為太尉辛亥大將軍澄南臨黎陽自虎牢濟河至洛陽魏同軌防長史裴寛與東魏將彭樂等戰為樂所擒澄禮遇甚厚寛得間逃歸【將即亮翻間古莧翻】澄由太行返晉陽【行戶剛翻】 屈獠洞斬李賁【賁竄屈獠洞見一百五十九卷中大同元年獠魯皓翻考異曰陳高祖紀云太清元年盖謂破賁之年今從梁帝紀 按通鑑破賁書于中大同元年】傳首建康賁兄天寶遁入九真收餘兵二萬圍愛州【五代志九真郡梁置愛州】交州司馬陳霸先帥衆討平之【帥讀曰率】詔以霸先為西江督護高要太守督七郡諸軍事 夏四月甲子東魏吏部令史張永和等偽假人官事覺糾檢首者六萬餘人【糾檢官所糾檢而發之者也首自首者也史言喪亂之際吏因為奸濫胃者不勝其多首手又翻】 甲戌東魏遣太尉高岳行臺慕容紹宗大都督劉豐生等將步騎十萬攻魏王思政於潁川【王思政守潁川事始上卷上年將即亮翻騎奇寄翻】思政命卧皷偃旗若無人者岳恃其衆四面陵城思政選驍勇開門出戰【驍堅堯切】岳兵敗走岳更築土山晝夜攻之思政随方拒守奪其土山置樓堞以助防守【堞徒協翻守式又翻】 五月魏以丞相泰為太師廣陵王欣為太傅李弼為大宗伯趙貴為大司寇于謹為大司空【宇文相魏倣成周之制建官】太師泰奉太子巡撫西境登隴至原州歷北長城【此盖秦所築長城也】東趣五原至蒲州【自五原還至蒲州也五代志河東郡後魏置秦州後周改曰蒲州因蒲坂為名也趣七喻翻】聞魏主不豫而還【還從宣翻又如字】及至已愈泰還華州【華戶化翻】上遣建康令謝挺散騎常侍徐陵等聘于東魏【按梁官制】<br />
<br />
  【建康令秩千石散騎常侍秩二千石謝挺不當在徐陵之上盖徐陵將命而使謝挺特輔行耳散悉亶翻騎奇寄翻】復修前好【愎扶又翻好呼到翻】陵摛之子也【徐摛見一百五十五卷中大通三年摛丑知翻】 六月東魏大將軍澄巡北邊 秋七月庚寅朔日有食之 乙卯東魏大將軍澄朝于鄴【朝直遥翻】以道士多偽濫始罷南郊道壇【魏太武帝崇信寇謙之置南郊道壇】八月庚寅澄還晉陽遣尚書辛術帥諸將略江淮之北凡獲二十三州【侯景既亂梁明年東魏始盡有淮南之地史究其終言之帥讀曰率將即亮翻】 侯景自至夀陽徵求無己朝廷未嘗拒絶景請娶於王謝上曰王謝門高非偶可於朱張以下訪之【朱張謂朱异張綰之族也】景恚【恚於避翻】曰會將吳兒女配奴又啟求錦萬匹為軍人作袍中領軍朱异議以青布給之又以臺所給仗多不能精啟請東冶鍜工欲更營造【鍜丁貫翻】景以安北將軍夏侯夔之子譒為長史【譒補過翻】徐思玉為司馬譒遂去夏稱侯託為族子【夏侯祥為梁朝佐命功臣其子亶夔皆宣力邊陲並著聲績至譒不克負荷矣】上既不用景言與東魏和親是後景表疏稍稍悖慢【悖蒲内翻又蒲妹翻】又聞徐陵等使魏反謀益甚【使疏吏翻】元貞知景有異志累啟還朝【景求輔貞見上卷上年朝直遥翻】景謂曰河北事雖不果江南何慮失之何不小忍貞懼逃歸建康具以事聞上以貞為始興内史亦不問景【帝既不問景又不為之備盖耄期倦勤直付之無可柰何】臨賀王正德所至貪暴不法屢得罪於上【正德既奔魏而逃歸上復其本封正德志行無悛常公行刼掠及随豫章王北侵又委軍而走為有司所奏上詔曰汝往年在蜀昵近小人猶謂少年情志未定更於吳都殺戮無辜刼盜財物及還京師專為逋逃乃至江乘要道湖頭斷路奪人妻妾畧人子女我每加覆掩冀汝自新了無悛革怨讐逾甚匹馬奔亡志懷反噬汝既來歸又令仗節董戎前驅豈謂汝狼心不改志欲覆敗國計以快汝心今宥汝以遠於是免官削爵徙臨海未至徙所追赦之復以朱异之言封臨賀王為丹陽尹坐所部多刼盜去職出為南兖州在任苛刻人不堪命從是黜廢轉增憤恨】由是憤恨隂養死士儲米積貨幸國家有變景知之正德在北與徐思玉相知【謂奔魏時也】景遣思玉致牋於正德曰今天子年尊姦臣亂國以景觀之計日禍敗大王屬當儲貳中被廢黜【詳見一百四十九卷普通三年被皮義翻】四海業業歸心大王景雖不敏實思自效願王允副蒼生鑒斯誠欵正德大喜曰侯公之意闇與吾同天授我也報之曰朝廷之事如公所言僕之有心為日久矣今僕為其内公為其外何有不濟機事在速今其時矣鄱陽王範密啟景謀反時上以邊事專委朱异動静皆關之异以為必無此理上報範曰景孤危寄命譬如嬰兒仰人乳哺【仰牛向翻】以此事勢安能反乎範重陳之曰不早剪撲禍及生民【重直用翻撲普卜翻】上曰朝廷自有處分不須汝深憂也【此亦報範之言非面語之也處昌呂翻分扶問翻】範復請以合肥之衆討之上不許【範非景敵也使上許範而進兵討景肉投餧虎耳復扶又翻下不復同】朱异謂範使曰鄱陽王遂不許朝廷有一客自是範啟异不復為通【使疏吏翻下同為于偽翻】景邀羊鴉仁同反鴉仁執其使以聞【羊鴉仁自懸瓠還頓軍淮上】异曰景數百叛虜何能為敕以使者付建康獄俄解遣之景益無所憚啟上曰若臣事是實應罹國憲如蒙照察請戮鴉仁 【考異曰梁書南史皆云並抑不奏典畧朱异拒之云云今從太清紀】景又言高澄狡猾寧可全信陛下納其詭語求與連和臣亦竊所笑也臣寧堪粉骨投命讎門【讐門謂高氏也】乞江西一境受臣控督如其不許即帥甲騎臨江上向閩越非唯朝廷自恥亦是三公旰食【帥讀日率騎奇寄翻旰古按翻】上使朱异宣語荅景使曰譬如貧家畜十客五客尚能得意【畜吁玉翻】朕惟有一客致有忿言亦朕之失也益加賞賜錦綵錢布信使相望【史言帝養成侯景之禍以敗國亡身】戊戌景反於夀陽以誅中領軍朱异少府卿徐驎太子右衛率陸驗制局監周石珍為名【驎離珍翻李延夀曰制局小司專典兵力雲陛天啟亘設蘭錡羽林精卒重屯廣衛至於元戎啟轍武侯遮迾清道晨行按轡督察往來馳驚輦轂驅投分部親承几案領護所攝手總成規蕭子顯曰尚書外司領武官有制局監内器仗兵役亦用寒人被恩倖者率所律翻】异等皆以姦佞驕貪蔽主弄權為時人所疾故景託以興兵驎驗吳郡人石珍丹陽人驎驗迭為少府丞以苛刻為務百賈怨之【賈音古】异尤與之暱【暱尼質翻】世人謂之三蠧司農卿傅岐梗直士也嘗謂异曰卿任參國鈞榮寵如此比日所聞【比毗至翻】鄙穢狼籍若使聖主發悟欲免得乎异曰外間謗黷知之久矣心苟無愧何恤人言岐謂人曰朱彦和將死矣【朱异字彦和】恃謟以求容肆辯以拒諫聞難而不懼【難乃旦翻】知惡而不改天奪之鑒其能久乎景西攻馬頭【景自渦陽之敗南走馬頭戍主劉神茂迎候之以入夀陽當塗之馬頭也今又自夀陽西攻馬頭則此馬頭在夀陽之西當淮津濟渡之要縛馬頭以登舟又非當塗之馬頭也當塗之馬頭郡在夀陽東考異曰梁書云執太守劉神茂按神茂素附於景無煩攻執今從太清紀典畧】遣其將宋子仙東攻木柵【木柵在荆山西】執戍主曹璆等【璆音求又渠幽翻】上聞之笑曰是何能為吾折箠笞之【此即朱异謂景數百叛虜何能為之說也君驕昏而臣貪昧禍至不懼以自取敗亡折之舌翻】敇購斬景者封三千戶公除州刺史甲辰詔以合州刺史鄱陽王範為南道都督北徐州刺史封山侯正表為北道都督【五代志封山縣屬合浦郡】司州刺史柳仲禮為西道都督通直散騎常侍裴之高為東道都督以侍中開府儀同三司邵陵王綸持節董督衆軍以討景正表宏之子仲禮慶遠之孫之高邃之兄子也【宏上之弟正表正德兄弟皆其子也柳慶遠裴邃皆天監名臣】九月東魏濮陽武公婁昭卒【濮博木翻】 侯景聞臺軍討<br />
<br />
  之問策於王偉偉曰邵陵若至彼衆我寡必為所困不如弃淮南【夀陽古淮南郡治所】决志東向帥輕騎直掩建康【帥讀曰率】臨賀反其内大王攻其外天下不足定也兵貴拙速宜即進路景乃留外弟中軍大都督王顯貴守夀陽癸未詐稱遊獵出夀陽人不之覺冬十月庚寅景揚聲趨合肥而實襲譙州【此譙州非渦陽之譙州魏收志梁置譙州於新昌城領高塘臨徐南梁新昌郡其地當在唐廬和二州之間宋白曰梁大同三年割北徐州之新昌南譙州之北譙立為南譙州居桑根山西今滁州城是也】助防董紹先開城降之 【考異曰太清紀云十三年陷譙城下又云十三日以王質巡江遏訪典畧上作庚戍下作庚子按此月戊子朔盖三日庚寅也】執刺史豐城侯泰泰範之弟也先為中書舍人【先悉薦翻】傾財以事時要超授譙州刺史至州徧發民丁使擔腰輿扇繖等物【腰輿者人舉之而行其高纔至腰繖蘇旰翻又蘇早翻盖也】不限士庶恥為之者重加杖責多輸財者即縱免之由是人皆思亂及侯景至人無戰心故敗庚子詔遣寧遠將軍王質帥衆三千巡江防遏景攻歷陽太守莊鐵丁未鐵以城降【降戶江翻】因說景曰國家承平歲久人不習戰聞大王舉兵内外震駭宜乘此際速趨建康【說式芮翻趨七喻翻】可兵不血刃而成大功若使朝廷徐得為備内外小安遣羸兵千人直據采石【羸倫為翻】大王雖有精甲百萬不得濟矣景乃留儀同三司田英郭駱守歷陽以鐵為導引兵臨江江上鎮戍相次啟聞上問討景之策於都官尚書羊侃侃請以二千人急據采石令邵陵王襲取夀陽使景進不得前退失巢穴烏合之衆自然瓦解朱异曰景必無度江之志遂寢其議侃曰今兹敗矣戊申以臨賀王正德為平北將軍都督京師諸軍事屯丹陽郡【盧循之寇建康也徐赤特敗於張侯橋循兵大上至丹陽郡則丹陽郡治盖近江渚】正德遣大船數十艘詐稱載荻密以濟景【艘蘇遭翻荻音狄】景將濟慮王質為梗使諜視之會臨川太守陳昕啟稱采石急須重鎮王質水軍輕弱恐不能濟【恐其不能濟國事也諜徒協翻昕許斤翻】上以昕為雲旗將軍代質戍采石徵質知丹陽尹事昕慶之之子也【陳慶之有入洛之功】質去采石而昕猶未下渚諜告景云質已退【未下渚者未下秦淮渚也諜徒協翻】景使折江東樹枝為驗諜如言而返景大喜曰吾事辦矣己酉自横江濟于采石有馬數百匹兵八千人是夕朝廷始命戒嚴景分兵襲姑孰執淮南太守文成侯寧【晉成帝初于姑孰僑立淮南郡五代志丹陽郡當塗縣舊置淮南郡】南津校尉江子一帥舟師千餘人欲於下流邀景【帥讀曰率】其副董桃生家在江北與其徒先潰走子一收餘衆步還建康子一子四之兄也太子見事急戎服入見上【入見賢遍翻】禀受方略上曰此自汝事何更問為内外軍事悉以付汝 【考異曰太清紀云太宗見事急乃入面啟高祖曰請以軍事並以垂付願不勞聖心南史云帝曰此自汝事何更問為今從典略】太子乃停中書省指授軍事物情惶駭莫有應募者朝廷猶不知臨賀王正德之情命正德屯朱雀門寧國公大臨屯新亭大府卿韋黯屯六門繕修宫城為受敵之備大臨大器之弟也【大臨大器皆太子綱之子】己酉景至慈湖建康大駭御街人更相劫掠【更工衡翻】不復通行【復扶又翻】赦東西冶尚方錢署及建康繫囚以揚州刺史宣城王大器都督城内諸軍事以羊侃為軍師將軍副之南浦侯推守東府【劉禪建興八年立南浦縣屬巴東郡】西豐公大春守石頭【沈約曰吴立豐縣屬臨川郡晉武帝太康元年更名西豐】輕車長史謝禧始興太守元貞守白下韋黯與右衛將軍柳津等分守宫城諸門及朝堂【朝直遥翻】推秀之子【安成王秀上弟也】大春大臨之弟津仲禮之父也攝諸寺庫公藏錢聚之德陽堂以充軍實【攝收也諸寺謂十二寺也藏徂浪翻天監六年改閲武堂為德陽堂在南闕前】庚戍侯景至板橋【張舜民曰出秦淮西南行循東岸行小夹中十里過板橋店】遣徐思玉來求見上實欲觀城中虚實上召問之思玉詐稱叛景請間陳事上將屏左右【屏必郢翻】舍人高善寶曰思玉從賊中來情偽難測安可使獨在殿上朱异侍坐曰徐思玉豈刺客邪思玉出景啟言异等弄權乞帶甲入朝除君側之惡异甚慚悚【朝直遥翻】景又請遣了事舍人出相領解【了事猶言暁事也領總録也解分判也領解言總錄景所欲言之事而分判是非也凡此皆侯景詭言以怠梁朝君臣使無戰心】上遣中書舍人賀季主書郭寶亮随思玉勞景于板橋【勞力到翻】景北面受敇季曰今者之舉何名景曰欲為帝也王偉進曰朱异等亂政除姦臣耳景既出惡言遂留季獨遣寶亮還宫百姓聞景至競入城公私混亂無復次第【復扶又翻】羊侃區分防擬皆以宗室間之【間古莧翻】軍人争入武庫自取器甲所司不能禁【所司謂武庫令之屬】侃命斬數人方止是時梁興四十七年【天監十八普通七大通二中大通六大同十一中大同一至是年太清二年通四十七年】境内無事公卿在位及閭里士大夫罕見兵甲賊至猝迫公私駭震宿將已盡後進少年並出在外【將即亮翻少詩沼翻】軍旅指撝一决於侃侃膽力俱壮太子深仗之【仗除兩翻憑仗也】辛亥景至朱雀桁南【桁戶剛翻】太子以臨賀王正德守宣陽門東宫學士新野庾信守朱雀門帥宫中文武三千餘人營桁北太子命信開大桁以挫其鋒正德曰百姓見開桁必大驚駭可且安物情太子從之俄而景至信帥衆開桁始除一舶【帥讀曰率下同舶旁陌翻大舟曰舶】見景軍皆著鐵面【著陟略翻】退隱于門信方食甘蔗【甘蔗生於南方狀如紫竹圍數寸高丈餘以刀去皮切食其味甘冷解煩析酲楚辭所謂泰尊柘漿析朝酲司馬相如子虚賦所謂諸柘者也蔗之夜翻】有飛箭中門柱信手甘蔗應弦而落遂弃軍走南塘遊軍沈子睦臨賀王正德之黨也復閉桁度景【景至秦淮南岸子睦領遊軍在南塘庾信既走北岸無兵子睦因得閉桁以度景兵中竹仲翻復扶又翻】太子使王質將精兵三千援信至領軍府遇賊未陳而走正德帥衆於張侯橋迎景馬上交揖既入宣陽門望闕而拜歔欷流涕随景度淮景軍皆著青袍正德軍並著絳袍碧裏【陳讀曰陣歔音虚欷許既翻又音希著陟畧翻】既與景合悉反其袍景乘勝至闕下城中恟懼【恟許拱翻】羊侃詐稱得射書云邵陵王西昌侯援兵已至近路【邵陵王綸兵時已度江向鍾離西昌侯淵藻時鎮京口】衆乃小安西豐公大春弃石頭奔京口劉禧元貞弃白下走津主彭文粲等以石頭城降景【降戶江翻】景遣其儀同三司于子悦守之壬子景列兵繞臺城旗旛皆黑射啟於城中曰【射而亦翻】朱异等蔑弄朝權輕作威福【朝直遥翻】臣為所陷欲加屠戮陛下若誅朱异等臣則斂轡北歸上問太子有是乎對曰然上將誅之太子曰賊以异等為名耳今日殺之無救於急適足貽笑將來俟賊平誅之未晚上乃止景繞城既帀【帀作答翻周也】百道俱攻鳴鼓吹脣喧聲震地縱火燒大司馬東西華諸門羊侃使鑿門上為竅【竅苦弔翻空也穴也】下水沃火太子自捧銀鞍往賞戰士直閣將軍朱思帥戰士數人踰城出外灑水久之方滅賊又以長柯斧斫東掖門門將開羊侃鑿扇為孔【扇門扇也】以槊刺殺二人斫者乃退【刺七亦翻】景據公車府【蕭子顯齊志公車令屬領軍以受天下章奏梁制公車令屬衛尉其署舍在臺城門外故景得據之府者署舍之通稱】正德據左衛府景黨宋子仙據東宫范桃棒據同泰寺【棒部項翻】景取東宫妓數百分給軍士【妓渠綺翻女樂也】東宫近城【近臺城也】景衆登其墻射城内【射而亦翻下臨射亦射弓射同】至夜景於東宫置酒奏樂太子遣人焚之臺殿及所聚圖書皆盡景又燒乘黄廏士林館太府寺【大同中於臺城西立士林館使朱异顧琛孔子祛等逓互講述乘繩證翻】癸丑景作木驢數百攻城城上投石碎之景更作尖項木驢石不能破【更工衡翻杜佑曰以木為眷長一丈徑一尺五寸下安六脚下濶而上尖高七尺内可容六人以濕牛皮蒙之人蔽其下舁直抵城下木石鐵火所不能敗用以攻城謂之木驢】羊侃使作雉尾炬灌以膏蠟叢擲焚之俄盡【杜佑曰鷰尾炬縳葦草為之分為兩岐如鷰尾狀以油臘灌之加火從城墜下使人騎木驢而燒之侃之作雉尾炬也施鐵鏃以油灌之擲驢上焚之】景又作登城樓高十餘丈欲臨射城中【高居傲翻射而亦翻】侃曰車高塹虚彼來必倒可卧而觀之及車動果倒【塹七艶翻】景攻既不克士卒死傷多乃築長圍以絶内外又啟求誅朱异等城中亦射賞格出外曰【射而亦翻下同】有能送景首者授以景位并錢一億萬布絹各萬匹朱异張綰議出兵擊之上問羊侃侃曰不可今出人若少【少詩沼翻】不足破賊徒挫鋭氣若多則一旦失利門隘橋小必大致失亡异等不從使千餘人出戰鋒未及交退走争橋赴水死者大半侃子鷟為景所獲【鷟士角翻】執至城下以示侃侃曰我傾宗報主猶恨不足豈計一子幸早殺之數日復持來【復扶又翻】侃謂鷟曰久以汝為死矣猶在邪引弓射之景以其忠義亦不之殺莊鐵慮景不克託稱迎母與左右數十趣歷陽【趣七喻翻】先遣書紿田英郭駱曰【紿待多翻】侯王己為臺軍所殺國家使我歸鎮駱等大懼弃城奔夀陽鐵入城不敢守奉其母奔尋陽十一月戊午朔刑白馬祀蚩尤於太極殿前【應劭曰蚩尤亦古天子好五兵故祭之求福祥薛瓚曰蚩尤庶人之貧者非天子也管仲曰割廬山發而出水金從之蚩尤受之以作劍戟】臨賀王正德即帝位於儀賢堂【天監六年改聽訟堂為儀賢堂在南闕前】下詔稱普通以來姦邪亂政上久不豫社稷將危河南王景釋位來朝【左傳王子朝曰諸侯釋位以間王政朝直遥翻】猥用朕躬紹兹寶位可大赦改元正平立其世子見理為皇太子以景為丞相妻以女【妻七喻翻】并出家之寶貨悉助軍費於是景營於闕前分其兵二千人攻東府南浦侯推拒之三日不克景自往攻之矢石雨下宣城王防閣許伯衆潛引景衆登城【宣城王大器太子之長子也許伯衆為其防閣在東府故得為景内應姚思亷梁書作許鬰華時為東府東北樓主】辛酉克之殺南浦侯推及城中戰士三千人載其尸聚於杜姥宅遥語城中人曰【語牛倨翻】若不早降正當如此【降戶江翻】景聲言上已晏駕雖城中亦以為然壬戌太子請上巡城上幸大司馬門城上聞蹕聲皆皷譟流涕衆心粗安【粗坐五翻】江子一之敗還也【謂自采石下流敗還之時】上責之子一拜謝曰臣以身許國常恐不得其死今所部皆弃臣去臣以一夫安能擊賊若賊遂能至此臣誓當碎首以贖前罪不死闕前當死闕後乙亥子一啟太子與弟尚書左丞子四東宫主帥子五帥所領百餘人開承明門出戰【主帥所類翻五帥讀曰率】子一直抵賊營賊伏兵不動【未測其情故不動】子一呼曰賊輩何不速出久之賊騎出夾攻之子一徑前引槊刺賊從者莫敢繼賊解其肩而死子四子五相謂曰與兄俱出何面獨旋皆免胄赴賊子四中矟洞胷而死【呼火故翻刺七亦翻從才用翻中竹仲翻矟與槊同色角翻】子五傷脰還至塹一慟而絶【江子一兄弟駢肩以死於闕下而不足以衛社稷悲夫古人所以重折衝千里之外者也塹七艶翻】景初至建康謂朝夕可拔號令嚴整士卒不敢侵暴及屢攻不克人心離沮景恐援兵四集一旦潰去又食石頭常平諸倉既盡軍中乏食乃縱士卒掠奪民米及金帛子女是後米一升至七八萬錢人相食餓死者什五六乙丑景於城東西起土山驅迫士民不限貴賤亂加敺捶疲羸者因殺以填山號哭動地【敺烏口翻捶止蘂翻羸倫為翻號戶刀翻】民不敢竄匿並出從之旬日間衆至數萬城中亦築土山以應之太子宣城王已下皆親負土執畚鍤【畚布衮翻所以盛土鍤側洽翻所以鍪土】於山上起芙蓉層樓高四丈飾以錦罽【芙蓉層樓下施㭿拱層層疊出若芙蓉花然罽毳布也織毛為之高居傲翻罽音居倒翻】募敢死士二千人厚衣袍鎧謂之僧騰客【衣於既翻下衣錦同】分配二山【二山謂東土山西土山也】晝夜交戰不息會大雨城内土山崩賊乘之垂入苦戰不能禁羊侃令多擲火為火城以斷其路【斷音短】徐於内築城賊不能進景募人奴降者悉免為良【降戶江翻】得朱异奴以為儀同三司异家貲產悉與之奴乘良馬衣錦袍於城下仰詬异曰【詬苦候翻】汝五十年仕宦方得中領軍我始事侯王已為儀同矣於是三日之中羣奴出就景者以千數景皆厚撫以配軍人人感恩為之致死【凡為奴者皆羣不逞也一旦免之為良固己踴躍况又資之以金帛安得不為賊致死乎士大夫承平之時虐用奴婢豈特誤其身誤其家亦以誤國事可不戒哉為于偽翻】荆州刺史湘東王繹聞景圍臺城丙寅戒嚴移檄所督湘州刺史河東王譽雍州刺史岳陽王詧【雍于用翻】江州刺史當陽公大心郢州刺史南平王恪等發兵入援大心大器之弟恪偉之子也【南平王偉上弟也】朱异遺景書為陳禍福【遺于季翻為于偽翻】景報書并告城中士民以為梁自近歲以來權倖用事割剥齊民以供嗜欲如曰不然公等試觀今日國家池苑王公第宅僧尼寺塔及在位庶僚姬姜百室僕從數千不耕不織【從才用翻】錦衣玉食不奪百姓從何得之【景書及此异等其何辭以對】僕所以趨赴闕庭指誅權佞非傾社稷今城中指望四方入援吾觀王侯諸將志在全身誰能竭力致死與吾争勝負哉【將即亮翻】長江天險二曹所歎【事見魏文帝紀】吾一葦杭之【詩國風曰誰謂河廣一葦杭之注杭渡也箋云誰謂河水廣與一葦加之則可以渡之喻狭也】日明氣浄自非天人允協何能如是幸各三思自求元吉景又奉啟於東魏主稱臣進取夀春暫欲停憩而蕭衍識此運終自辭寶位臣軍未入其國已投同泰捨身去月二十九日【去月謂前此月也】届此建康江海未蘇干戈暫止永言故鄉人馬同戀尋當整轡以奉聖顔臣之母弟久謂屠滅近奉明敇始承猶在【承猶奉也言奉近敇始知母弟猶在也】斯乃陛下寛仁大將軍恩念臣之弱劣知何仰報今輒齎啟迎臣母弟妻兒伏願聖慈特賜裁放【景欲卑辭以迎其家高澄兄弟詎能堕其數中邪】己巳湘東王繹遣司馬吳曅天門太守樊文皎等將兵發江陵陳昕為景所擒景與之極飲使昕收集部曲欲用之昕不可景使其儀同三司范桃棒囚之昕因說桃棒【說式芮翻】使帥所部襲殺王偉宋子仙詣城降桃棒從之潛遣昕夜縋入城【帥讀曰率降戶江翻下同縋馳偽翻】上大喜敇鐫銀劵賜桃棒曰【鐫子全翻刻也雕也】事定之日封汝河南王即有景衆并給金帛女樂太子恐其詐猶豫不决上怒曰受降常理何忽致疑太子召公卿會議朱异傅岐曰桃棒降必非謬桃棒既降賊景必驚乘此擊之可大破也太子曰吾堅城自守以俟外援援兵既至賊豈足平此萬全策也今開門納桃棒桃棒之情何易可知【易以豉翻】萬一為變悔無所及社稷事重須更詳之异曰殿下若以社稷之急宜納桃棒如其猶豫非异所知太子終不能决桃棒又使昕啟曰今止將所領五百人若至城門皆自脱甲乞朝廷開門賜容事濟之後保擒侯景 【考異曰太清紀南史皆云桃棒求以甲士二千人來降以景首應購今從典略】太子見其懇切愈疑之朱异撫膺曰失此社稷事去矣【太子綱固多疑少斷朱异撫膺於此時何其晚也】俄而桃棒為部下所告景拉殺之【拉盧合翻拉其幹而殺之】陳昕不知如期而出景邀得之逼使射書城中曰【射而亦翻】桃棒且輕將數十人先入【將即亮翻】景欲衷甲随之昕不肯期以必死乃殺之景使蕭見理與儀同三司盧暉略戌東府見理凶險夜與羣盗剽刼於大桁中流矢而死【㔄匹妙翻中竹仲翻】邵陵王綸行至鍾離聞侯景已度采石綸晝夜兼道旋軍入援濟江中流風起人馬溺者什一二【盧循之亂劉裕冒風濟江而風止侯景之亂綸濟江而風起豈天之欲亡梁邪是以善觀人之國者必觀於天人祐助之際也】遂帥寧遠將軍西豐公大春【沈約志西豐縣屬臨川郡吳立】新塗公大成【帥讀曰率新塗或作新淦沈約志新淦縣漢屬豫章郡】永安侯確安南侯駿【是皆以古縣名為侯國吳分烏程餘杭立永安縣晉已改為武康晉武帝分江安立安南縣五代志無之】前譙州刺史趙伯超武州刺史蕭弄璋等【武陵郡梁置武州】步騎三萬自京口西上【上時掌翻】大成大春之弟確綸之子駿懿之孫也景遣軍至江乘拒綸軍趙伯超曰若從黄城大路必與賊遇不如徑指鍾山【鍾山即蒋山吳孫權立蒋子文廟於是山又以其祖諱鍾改名蒋山】突㨿廣莫門出賊不意賊圍必解矣綸從之夜行失道迂二十餘里【迂音于又音紆曲也遠也】庚辰旦營于蒋山景見之大駭悉送所掠婦女珍貨於石頭具舟欲走分兵三道攻綸綸與戰破之時山巔寒雪乃引軍下愛敬寺【帝事文皇帝獻皇后孝於鍾山造大愛敬寺以資福】景陳兵於覆舟山北乙酉綸進軍玄武湖側 【考異曰太清紀云二十九日典略云壬午今從梁帝紀】與景對陳不戰【陳讀曰陣】至暮景更約明日會戰綸許之安南侯駿見景軍退以為走即與壮士逐之景旋軍擊之駿敗走趨綸軍趙伯超望見亦引兵走景乘勝追撃之諸軍皆潰綸收餘兵近千人【近其靳翻】入天保寺景追之縱火燒寺綸奔朱方【冊徒春秋朱方之地時為蘭陵武進縣】士卒踐氷雪往往堕足景悉收綸輜重【重直用翻】生擒西豐公大春安前司馬莊丘慧主帥霍俊等而還【帝置二百四十號將軍有安前將軍置長史司馬帥所類翻還從宣翻又如字 考異曰典略作廣陵令崔俊南史作直閣將軍胡子約廣陵令霍雋今從太清紀】丙戍景陳所獲綸軍首虜鎧仗及大春等於城下使言曰邵陵王已為亂兵所殺霍俊獨曰王小失利己全軍還京口城中但堅守援軍尋至賊以刀敺其背【敺烏口翻】俊辭色彌厲景義而釋之臨賀王正德殺之是日晚鄱陽王範遣其世子嗣與西豫州刺史裴之高建安太守趙鳳舉【晉安帝分廬江郡立晉熙郡及懷寜縣梁置西豫州隋為同安郡唐為舒州五代志沔陽郡竟陵縣舊有京山縣齊置建安郡但其地在漢陽與舒州勢不相接夀陽義陽之間有建安戍蕭子顯齊志及五代志皆不言于此置郡五代志又云黄州麻城縣梁置建寧軍或者史以建寧為建安與更考】各將兵入援軍於蔡洲【將即亮翻 考異曰梁帝紀作張公洲今從太清紀】以待上流諸軍範以之高督江右援軍事景悉驅南岸居民於水北【此謂秦淮水也】焚其廬舍大街以西掃地俱盡北徐州刺史封山侯正表鎮鍾離【隋志有封山縣屬合浦郡盖梁置也】上召之入援正表託以船糧未集不進景以正表為南兖州刺史封南郡王正表乃於歐陽立柵以斷援軍【斷音短水經註邗溝水上承歐陽引江入埭六十里至廣陵城以地望攷之此歐陽在今真州界按江淮之間地名歐陽見於史者非一處裴邃移長孫稚欲營歐陽在夀春境上吳喜使蕭道成留軍歐陽在淮隂界】帥衆一萬聲言入援實欲襲廣陵密書誘廣陵令劉詢使燒城為應【誘音酉】詢以告南兖州刺史南康王會理十二月會理使詢帥步騎千人夜襲正表大破之【帥讀曰率騎奇寄翻下同】正表走還鍾離詢收其兵糧歸就會理與之入援癸巳侍中都官尚書羊侃卒城中益懼侯景大造攻具陳於闕前大車高數丈一車二十輪丁酉復進攻城【高居報翻復扶又翻】以蝦蟇車運土填塹湘東王繹遣世子方等將步騎一萬入援建康【將即亮翻下同】庚子發公安繹又遣竟陵太守王僧辯將舟師萬人出自漢川載糧東下【漢水經竟陵郡入江 考異曰太清紀云僧辯將精卒二萬今從梁書】方等有俊才善騎射每戰親犯矢石以死節自任【為人臣子固當以身許國然存其身者所以存國也两陳相向勝負未分危機交急親犯矢石以帥厲將士可一用之耳豈可以為常哉方等以死節自任以親犯矢石為常此其所以敗死于湘川也若方等者謂之必死之將可也若論臣子大節則全其身以全國家斯得謂之忠孝矣】壬寅侯景以火車焚臺城東南樓材官吳景有巧思【思相吏翻】於城内構地為樓火纔滅新樓即立賊以為神景因火起潜遣人於其下穿城城將崩乃覺之【詳觀上下文景因火起作賊因火起則于當時事勢文理為明順盖侯景與吴景殽亂也讀者難以明辨】吳景于城内更築迂城【迂憂俱翻迃曲也】狀如却月以擬之兼擲火焚其攻具賊乃退走太子遣洗馬元孟恭將千人自大司馬門出盪孟恭與左右奔降於景己酉景土山稍逼城樓柳津命作地道以取其土外山崩壓賊且盡又於城内作飛橋懸罩二土山景衆見飛橋迥出崩騰而走城内擲雉尾炬焚其東山樓柵蕩盡賊積死於城下【死於城下者豈真賊哉侯景驅民以攻城以其黨迫蹙于後攻城之人退則死於賊手進則死于矢石嗚呼積死於城下者得非梁之赤子乎】乃棄土山不復脩自焚其攻具【復扶又翻】材官將軍宋嶷降於景【嶷魚力翻降戶江翻】教之引玄武湖水以灌臺城闕前皆為洪流上徵衡州刺史韋粲為散騎常侍【吴孫亮太平二年分長沙東部都尉立湘東郡今之衡州按五代志梁置衡州於南海郡含洭縣湘東之衡州隋平陳所置】以都督長沙歐陽頠監州事粲放之子也【韋放見一百五十一卷大通元年監古銜翻】還至廬陵聞侯景亂粲簡閲部下得精兵五千倍道赴援至豫章聞景己出横江粲就内史劉孝儀謀之孝儀曰必如此當有敇豈可輕信人言妄相驚動或恐不然時孝儀置酒粲怒以杯抵地曰賊已度江便逼宫闕水陸俱斷何暇有報假令無敇豈得自安韋粲今日何情飲酒即馳馬出部分將發【分扶問翻】會江州刺史當陽公大心遣使邀粲【使疏吏翻】粲乃馳往見大心曰上游藩鎮江州去京最近【按沈約志江州去京水行一千四百里】殿下情計誠宜在前但中流任重當須應接不可闕鎮今宜且張聲勢移鎮湓城【張知兩翻】遣偏將賜随【將即亮翻下在將同】於事便足大心然之遣中兵柳昕帥兵二千人随粲粲至南洲外弟司州刺史柳仲禮亦帥步騎萬餘人至横江【帥讀曰率】粲即送糧仗贍給之并散私金帛以賞其戰士西豫州刺史裴之高自張公洲遣船度仲禮【攷之粲傳張公洲盖即蔡洲】丙辰夜粲仲禮及宣猛將軍李孝欽 【考異曰梁帝紀作李遷仕今從太清紀】前司州刺史羊鴉仁南陵太守陳文徹【五代志宣州南陵縣梁置南陵郡】合軍屯新林王遊苑粲議推仲禮為大都督報下流衆軍【下流衆軍張公洲之兵也】裴之高自以年位恥居其下議累日不决粲抗言於衆曰今者同赴國難【難乃旦翻】義在除賊所以推柳司州者正以久捍邉疆先為侯景所憚且士馬精鋭無出其前若論位次柳在粲下語其年齒亦少於粲直以社稷之計不得復論今日形勢貴在將和若人心不同大事去矣裴公朝之舊德豈應復挟私情以沮大計粲請為諸軍解之【語牛倨翻少詩沼翻復扶又翻沮在侣翻為于偽翻】乃單舸至之高營切讓之曰【舸古我翻】今二宫危逼猾寇滔天臣子當戮力同心豈可自相矛楯【韓非子有鬻矛楯者曰吾矛之利物無不陷也又曰吾楯之堅物莫能陷也或問之曰以子之矛陷子之楯可乎鬻者不能對後世矛楯之說祖此】豫州必欲立異鋒鏑便有所歸【言將攻之高也】之高垂泣致謝遂推仲禮為大都督宣城内史楊白華遣其子雄將郡兵繼至【華讀曰花將即亮翻下同】援軍大集衆十餘萬緣淮樹柵景亦於北岸樹柵以應之裴之高與弟之横以舟師一萬屯張公洲景囚之高弟姪子孫臨水陳兵連鏁列於陳前以鼎鑊刀鋸随其後謂曰裴公不降今即烹之【於陳讀曰陣降戶江翻】之高召善射者使射其子再發皆不中【使射而亦翻中竹仲翻】景帥步騎萬人於後渚挑戰【據韋粲傳後渚在中興寺前挑徒了翻】仲禮欲出擊之韋粲曰日晚我勞未可戰也仲禮乃堅壁不出景亦引退湘東王繹將鋭卒三萬發江陵留其子綏寧侯方諸居守【守手又翻沈約志廣州南海郡有綏寧縣宋文帝立】諮議參軍劉之迡等三上牋請留荅教不許【湘東王繹非有自將入援之志也陽為不許耳迡與遲同又音奴寄翻】鄱陽王範遣其將梅伯龍攻王顯貴於夀陽克其羅城攻中城不克而退範益其衆使復攻之【復扶又翻】 東魏大將軍澄患民錢濫惡議不禁民私鑄但懸稱市門【稱尺證翻】錢不重五銖毋得入市朝議以為年穀不登請俟他年乃止【朝直遥翻】 魏太師泰殺安定國臣王茂而非其罪【泰封安定公故有國臣】尚書左丞柳慶諫泰怒曰卿黨罪人亦當坐執慶于前慶辭色不撓【撓奴教翻】曰慶聞君蔽於事為不明臣知而不争為不忠慶既竭忠不敢愛死但懼公為不明耳泰寤亟使赦茂不及乃賜茂家錢帛曰以旌吾過 丙辰晦柳仲禮夜入韋粲營部分衆軍【分扶問翻】旦日會戰諸將各有據守令粲頓青塘粲以青塘當石頭中路【粲傳曰青塘迫近淮渚據陳霸先之言青塘即青溪塘也】賊必争之頗憚之仲禮曰青塘要地非兄不可若疑兵少當更遣軍相助乃使直閣將軍劉叔胤助之【為下韋粲敗死張本少詩沼翻】<br />
<br />
  資治通鑑卷一百六十一<br />
<br />
<史部,編年類,資治通鑑>  <br>
   </div> 

<script src="/search/ajaxskft.js"> </script>
 <div class="clear"></div>
<br>
<br>
 <!-- a.d-->

 <!--
<div class="info_share">
</div> 
-->
 <!--info_share--></div>   <!-- end info_content-->
  </div> <!-- end l-->

<div class="r">   <!--r-->



<div class="sidebar"  style="margin-bottom:2px;">

 
<div class="sidebar_title">工具类大全</div>
<div class="sidebar_info">
<strong><a href="http://www.guoxuedashi.com/lsditu/" target="_blank">历史地图</a></strong>  
<a href="http://www.880114.com/" target="_blank">英语宝典</a>  
<a href="http://www.guoxuedashi.com/13jing/" target="_blank">十三经检索</a> 
<br><strong><a href="http://www.guoxuedashi.com/gjtsjc/" target="_blank">古今图书集成</a></strong> 
<a href="http://www.guoxuedashi.com/duilian/" target="_blank">对联大全</a> <strong><a href="http://www.guoxuedashi.com/xiangxingzi/" target="_blank">象形文字典</a></strong> 

<br><a href="http://www.guoxuedashi.com/zixing/yanbian/">字形演变</a>  <strong><a href="http://www.guoxuemi.com/hafo/" target="_blank">哈佛燕京中文善本特藏</a></strong>
<br><strong><a href="http://www.guoxuedashi.com/csfz/" target="_blank">丛书&方志检索器</a></strong> <a href="http://www.guoxuedashi.com/yqjyy/" target="_blank">一切经音义</a>  

<br><strong><a href="http://www.guoxuedashi.com/jiapu/" target="_blank">家谱族谱查询</a></strong>  <strong><a href="http://shufa.guoxuedashi.com/sfzitie/" target="_blank">书法字帖欣赏</a></strong> 
<br>

</div>
</div>


<div class="sidebar" style="margin-bottom:0px;">

<font style="font-size:22px;line-height:32px">QQ交流群9:489193090</font>


<div class="sidebar_title">手机APP 扫描或点击</div>
<div class="sidebar_info">
<table>
<tr>
	<td width=160><a href="http://m.guoxuedashi.com/app/" target="_blank"><img src="/img/gxds-sj.png" width="140"  border="0" alt="国学大师手机版"></a></td>
	<td>
<a href="http://www.guoxuedashi.com/download/" target="_blank">app软件下载专区</a><br>
<a href="http://www.guoxuedashi.com/download/gxds.php" target="_blank">《国学大师》下载</a><br>
<a href="http://www.guoxuedashi.com/download/kxzd.php" target="_blank">《汉字宝典》下载</a><br>
<a href="http://www.guoxuedashi.com/download/scqbd.php" target="_blank">《诗词曲宝典》下载</a><br>
<a href="http://www.guoxuedashi.com/SiKuQuanShu/skqs.php" target="_blank">《四库全书》下载</a><br>
</td>
</tr>
</table>

</div>
</div>


<div class="sidebar2">
<center>


</center>
</div>

<div class="sidebar"  style="margin-bottom:2px;">
<div class="sidebar_title">网站使用教程</div>
<div class="sidebar_info">
<a href="http://www.guoxuedashi.com/help/gjsearch.php" target="_blank">如何在国学大师网下载古籍?</a><br>
<a href="http://www.guoxuedashi.com/zidian/bujian/bjjc.php" target="_blank">如何使用部件查字法快速查字?</a><br>
<a href="http://www.guoxuedashi.com/search/sjc.php" target="_blank">如何在指定的书籍中全文检索?</a><br>
<a href="http://www.guoxuedashi.com/search/skjc.php" target="_blank">如何找到一句话在《四库全书》哪一页?</a><br>
</div>
</div>


<div class="sidebar">
<div class="sidebar_title">热门书籍</div>
<div class="sidebar_info">
<a href="/so.php?sokey=%E8%B5%84%E6%B2%BB%E9%80%9A%E9%89%B4&kt=1">资治通鉴</a> <a href="/24shi/"><strong>二十四史</strong></a>&nbsp; <a href="/a2694/">野史</a>&nbsp; <a href="/SiKuQuanShu/"><strong>四库全书</strong></a>&nbsp;<a href="http://www.guoxuedashi.com/SiKuQuanShu/fanti/">繁体</a>
<br><a href="/so.php?sokey=%E7%BA%A2%E6%A5%BC%E6%A2%A6&kt=1">红楼梦</a> <a href="/a/1858x/">三国演义</a> <a href="/a/1038k/">水浒传</a> <a href="/a/1046t/">西游记</a> <a href="/a/1914o/">封神演义</a>
<br>
<a href="http://www.guoxuedashi.com/so.php?sokeygx=%E4%B8%87%E6%9C%89%E6%96%87%E5%BA%93&submit=&kt=1">万有文库</a> <a href="/a/780t/">古文观止</a> <a href="/a/1024l/">文心雕龙</a> <a href="/a/1704n/">全唐诗</a> <a href="/a/1705h/">全宋词</a>
<br><a href="http://www.guoxuedashi.com/so.php?sokeygx=%E7%99%BE%E8%A1%B2%E6%9C%AC%E4%BA%8C%E5%8D%81%E5%9B%9B%E5%8F%B2&submit=&kt=1"><strong>百衲本二十四史</strong></a>  <a href="http://www.guoxuedashi.com/so.php?sokeygx=%E5%8F%A4%E4%BB%8A%E5%9B%BE%E4%B9%A6%E9%9B%86%E6%88%90&submit=&kt=1"><strong>古今图书集成</strong></a>
<br>

<a href="http://www.guoxuedashi.com/so.php?sokeygx=%E4%B8%9B%E4%B9%A6%E9%9B%86%E6%88%90&submit=&kt=1">丛书集成</a> 
<a href="http://www.guoxuedashi.com/so.php?sokeygx=%E5%9B%9B%E9%83%A8%E4%B8%9B%E5%88%8A&submit=&kt=1"><strong>四部丛刊</strong></a>  
<a href="http://www.guoxuedashi.com/so.php?sokeygx=%E8%AF%B4%E6%96%87%E8%A7%A3%E5%AD%97&submit=&kt=1">說文解字</a> <a href="http://www.guoxuedashi.com/so.php?sokeygx=%E5%85%A8%E4%B8%8A%E5%8F%A4&submit=&kt=1">三国六朝文</a>
<br><a href="http://www.guoxuedashi.com/so.php?sokeytm=%E6%97%A5%E6%9C%AC%E5%86%85%E9%98%81%E6%96%87%E5%BA%93&submit=&kt=1"><strong>日本内阁文库</strong></a> <a href="http://www.guoxuedashi.com/so.php?sokeytm=%E5%9B%BD%E5%9B%BE%E6%96%B9%E5%BF%97%E5%90%88%E9%9B%86&ka=100&submit=">国图方志合集</a> <a href="http://www.guoxuedashi.com/so.php?sokeytm=%E5%90%84%E5%9C%B0%E6%96%B9%E5%BF%97&submit=&kt=1"><strong>各地方志</strong></a>

</div>
</div>


<div class="sidebar2">
<center>

</center>
</div>
<div class="sidebar greenbar">
<div class="sidebar_title green">四库全书</div>
<div class="sidebar_info">

《四库全书》是中国古代最大的丛书,编撰于乾隆年间,由纪昀等360多位高官、学者编撰,3800多人抄写,费时十三年编成。丛书分经、史、子、集四部,故名四库。共有3500多种书,7.9万卷,3.6万册,约8亿字,基本上囊括了古代所有图书,故称“全书”。<a href="http://www.guoxuedashi.com/SiKuQuanShu/">详细>>
</a>

</div> 
</div>

</div>  <!--end r-->

</div>
<!-- 内容区END --> 

<!-- 页脚开始 -->
<div class="shh">

</div>

<div class="w1180" style="margin-top:8px;">
<center><script src="http://www.guoxuedashi.com/img/plus.php?id=3"></script></center>
</div>
<div class="w1180 foot">
<a href="/b/thanks.php">特别致谢</a> | <a href="javascript:window.external.AddFavorite(document.location.href,document.title);">收藏本站</a> | <a href="#">欢迎投稿</a> | <a href="http://www.guoxuedashi.com/forum/">意见建议</a> | <a href="http://www.guoxuemi.com/">国学迷</a> | <a href="http://www.shuowen.net/">说文网</a><script language="javascript" type="text/javascript" src="https://js.users.51.la/17753172.js"></script><br />
  Copyright &copy; 国学大师 古典图书集成 All Rights Reserved.<br>
  
  <span style="font-size:14px">免责声明:本站非营利性站点,以方便网友为主,仅供学习研究。<br>内容由热心网友提供和网上收集,不保留版权。若侵犯了您的权益,来信即刪。scp168@qq.com</span>
  <br />
ICP证:<a href="http://www.beian.miit.gov.cn/" target="_blank">鲁ICP备19060063号</a></div>
<!-- 页脚END --> 
<script src="http://www.guoxuedashi.com/img/plus.php?id=22"></script>
<script src="http://www.guoxuedashi.com/img/tongji.js"></script>

</body>
</html>
