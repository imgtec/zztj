<!DOCTYPE html PUBLIC "-//W3C//DTD XHTML 1.0 Transitional//EN" "http://www.w3.org/TR/xhtml1/DTD/xhtml1-transitional.dtd">
<html xmlns="http://www.w3.org/1999/xhtml">
<head>
<meta http-equiv="Content-Type" content="text/html; charset=utf-8" />
<meta http-equiv="X-UA-Compatible" content="IE=Edge,chrome=1">
<title>資治通鑒_293-資治通鑑卷二百九十二_293-資治通鑑卷二百九十二</title>
<meta name="Keywords" content="資治通鑒_293-資治通鑑卷二百九十二_293-資治通鑑卷二百九十二">
<meta name="Description" content="資治通鑒_293-資治通鑑卷二百九十二_293-資治通鑑卷二百九十二">
<meta http-equiv="Cache-Control" content="no-transform" />
<meta http-equiv="Cache-Control" content="no-siteapp" />
<link href="/img/style.css" rel="stylesheet" type="text/css" />
<script src="/img/m.js?2020"></script> 
</head>
<body>
 <div class="ClassNavi">
<a  href="/24shi/">二十四史</a> | <a href="/SiKuQuanShu/">四库全书</a> | <a href="http://www.guoxuedashi.com/gjtsjc/"><font  color="#FF0000">古今图书集成</font></a> | <a href="/renwu/">历史人物</a> | <a href="/ShuoWenJieZi/"><font  color="#FF0000">说文解字</a></font> | <a href="/chengyu/">成语词典</a> | <a  target="_blank"  href="http://www.guoxuedashi.com/jgwhj/"><font  color="#FF0000">甲骨文合集</font></a> | <a href="/yzjwjc/"><font  color="#FF0000">殷周金文集成</font></a> | <a href="/xiangxingzi/"><font color="#0000FF">象形字典</font></a> | <a href="/13jing/"><font  color="#FF0000">十三经索引</font></a> | <a href="/zixing/"><font  color="#FF0000">字体转换器</font></a> | <a href="/zidian/xz/"><font color="#0000FF">篆书识别</font></a> | <a href="/jinfanyi/">近义反义词</a> | <a href="/duilian/">对联大全</a> | <a href="/jiapu/"><font  color="#0000FF">家谱族谱查询</font></a> | <a href="http://www.guoxuemi.com/hafo/" target="_blank" ><font color="#FF0000">哈佛古籍</font></a> 
</div>

 <!-- 头部导航开始 -->
<div class="w1180 head clearfix">
  <div class="head_logo l"><a title="国学大师官网" href="http://www.guoxuedashi.com" target="_blank"></a></div>
  <div class="head_sr l">
  <div id="head1">
  
  <a href="http://www.guoxuedashi.com/zidian/bujian/" target="_blank" ><img src="http://www.guoxuedashi.com/img/top1.gif" width="88" height="60" border="0" title="部件查字,支持20万汉字"></a>


<a href="http://www.guoxuedashi.com/help/yingpan.php" target="_blank"><img src="http://www.guoxuedashi.com/img/top230.gif" width="600" height="62" border="0" ></a>


  </div>
  <div id="head3"><a href="javascript:" onClick="javascript:window.external.AddFavorite(window.location.href,document.title);">添加收藏</a>
  <br><a href="/help/setie.php">搜索引擎</a>
  <br><a href="/help/zanzhu.php">赞助本站</a></div>
  <div id="head2">
 <a href="http://www.guoxuemi.com/" target="_blank"><img src="http://www.guoxuedashi.com/img/guoxuemi.gif" width="95" height="62" border="0" style="margin-left:2px;" title="国学迷"></a>
  

  </div>
</div>
  <div class="clear"></div>
  <div class="head_nav">
  <p><a href="/">首页</a> | <a href="/ShuKu/">国学书库</a> | <a href="/guji/">影印古籍</a> | <a href="/shici/">诗词宝典</a> | <a   href="/SiKuQuanShu/gxjx.php">精选</a> <b>|</b> <a href="/zidian/">汉语字典</a> | <a href="/hydcd/">汉语词典</a> | <a href="http://www.guoxuedashi.com/zidian/bujian/"><font  color="#CC0066">部件查字</font></a> | <a href="http://www.sfds.cn/"><font  color="#CC0066">书法大师</font></a> | <a href="/jgwhj/">甲骨文</a> <b>|</b> <a href="/b/4/"><font  color="#CC0066">解密</font></a> | <a href="/renwu/">历史人物</a> | <a href="/diangu/">历史典故</a> | <a href="/xingshi/">姓氏</a> | <a href="/minzu/">民族</a> <b>|</b> <a href="/mz/"><font  color="#CC0066">世界名著</font></a> | <a href="/download/">软件下载</a>
</p>
<p><a href="/b/"><font  color="#CC0066">历史</font></a> | <a href="http://skqs.guoxuedashi.com/" target="_blank">四库全书</a> |  <a href="http://www.guoxuedashi.com/search/" target="_blank"><font  color="#CC0066">全文检索</font></a> | <a href="http://www.guoxuedashi.com/shumu/">古籍书目</a> | <a   href="/24shi/">正史</a> <b>|</b> <a href="/chengyu/">成语词典</a> | <a href="/kangxi/" title="康熙字典">康熙字典</a> | <a href="/ShuoWenJieZi/">说文解字</a> | <a href="/zixing/yanbian/">字形演变</a> | <a href="/yzjwjc/">金 文</a> <b>|</b>  <a href="/shijian/nian-hao/">年号</a> | <a href="/diming/">历史地名</a> | <a href="/shijian/">历史事件</a> | <a href="/guanzhi/">官职</a> | <a href="/lishi/">知识</a> <b>|</b> <a href="/zhongyi/">中医中药</a> | <a href="http://www.guoxuedashi.com/forum/">留言反馈</a>
</p>
  </div>
</div>
<!-- 头部导航END --> 
<!-- 内容区开始 --> 
<div class="w1180 clearfix">
  <div class="info l">
   
<div class="clearfix" style="background:#f5faff;">
<script src='http://www.guoxuedashi.com/img/headersou.js'></script>

</div>
  <div class="info_tree"><a href="http://www.guoxuedashi.com">首页</a> > <a href="/SiKuQuanShu/fanti/">四库全书</a>
 > <h1>资治通鉴</h1> <!--         下载:【右键另存为】即可 --></div>
  <div class="info_content zj clearfix">
  
<div class="info_txt clearfix" id="show">
<center style="font-size:24px;">293-資治通鑑卷二百九十二</center>
    資治通鑑卷二百九十二 宋 司馬光 撰<br />
<br />
  胡三省 音註<br />
<br />
  後周紀三【起閼逢攝提格五月盡柔兆執徐二月凡一年有奇】<br />
<br />
  太祖聖神恭肅文武孝皇帝下<br />
<br />
  顯德元年五月甲戌朔王逵自潭州遷于朗州以周行逢知潭州事以潘叔嗣爲岳州團練使【已而潘叔嗣殺王逵而周行逢收田父漁者之功矣】 丙子帝至晉陽城下【帝自上黨趣晉陽七日而至】旗幟環城四十里【史言周兵之盛幟昌志翻環音宦】楊衮疑北漢代州防禦使鄭處謙貳于周召與計事欲圖之處謙知之不往衮使胡騎數十守其城門處謙殺之因閉門拒衮衮奔歸契丹契丹主怒其無功囚之處謙舉城來降丁丑置靜塞軍於代州以鄭處謙爲節度使【創置方鎭以懷撫鄭處謙處昌呂翻】契丹數千騎屯忻代之間爲北漢聲援庚辰遣符彦卿等將步騎萬餘擊之彦卿入忻州契丹退保忻口【九域志忻州忻容縣有忻口寨在石嶺關南】丁亥置寧化軍於汾州以石沁二州隸之代州將桑珪解文遇殺鄭處謙【沁七鴆翻解戶買翻姓也姓苑自唐叔虞食邑於解晉有解狐解揚】誣奏云潜通契丹符彦卿奏請益兵癸巳遣李筠張宋德將兵三千赴之契丹游騎時至忻州城下丙申彦卿與諸將陳以待之【陳讀曰陣】史彦超將二十騎爲前鋒【二十太少恐當作二千】遇契丹與戰李筠引兵繼之殺契丹二千人彦超恃勇輕進去大軍浸遠衆寡不敵爲契丹所殺筠僅以身免周兵死傷甚衆彦卿退保忻州尋引兵還晉陽【還從宣翻又如字下同】府州防禦使折德扆將州兵來朝【將即亮翻】辛丑復置永安軍於府州【復扶又翻漢乾祐三年罷永安軍見二百八十九卷】以德扆爲節度使時大發兵夫東自懷孟西及蒲陜以攻晉陽不克會久雨士卒疲病乃議引還【陜失冉翻考異曰世宗實録徵懷孟蒲陜丁夫數萬攻城旦夕之間期於必取會大雨軍士勞苦又聞忻口之師不振】<br />
<br />
  【帝數日憂沮不食遂决還京之意晉陽見閒錄六月旦周師南轅返斾惟數百騎間之以步卒千人長槍赤甲衍趫捷跳梁於城隅晡晚殺行而抽退今從世宗實録】初王得中返自契丹【北漢主遣王得中求救於契丹見上卷本年三月】值周兵圍晉陽留止代州及桑珪殺鄭處謙囚得中送於周軍帝釋之賜以帶馬問虜兵何時當至得中白臣受命送楊衮他無所求或謂得中曰契丹許公發兵公不以實告契丹兵即至公得無危乎得中太息曰吾食劉氏禄有老母在圍中若以實告周人必發兵據險以拒之如此家國兩亡吾獨生何益不若殺身以全家國所得多矣甲辰帝以得中欺罔縊殺之【王得中之死知所惡有甚於死者也】乙巳帝發晉陽匡國節度使藥元福言於帝曰進軍易退軍難【進軍者或乘初至之銳或乘屢勝之勢敵人畏讋自守不敢迎戰故易退軍者士有歸志敵人據險遮其前率衆躡其後輜重老弱皆足爲吾之累故難易以䜴翻】帝曰朕一以委卿元福乃勒兵成列而殿【殿丁練翻】北漢果出兵追躡元福擊走之然軍還怱遽【還從宣翻】芻糧數十萬在城下悉焚弃之軍中訛言相驚或相剽掠軍須失亡不可勝計【剽匹妙翻凡行軍所欲得以爲用者皆謂之軍須勝音升】所得北漢州縣周所置刺史等皆弃城走惟代州桑珪既叛北漢又不敢歸周嬰城自守北漢遣兵攻拔之【前所謂都府未拔雖得属郡而無益者要其終也】乙酉帝至潞州甲子至鄭州【以乙巳發晉陽甲子至鄭州考之中間不應以乙酉至潞州恐是乙卯】丙寅謁嵩陵【嵩陵復土帝適有軍旅之事不獲親之此其謁陵與彞制謁陵其情有不同者】庚午至大梁 帝違衆議破北漢自是政事無大小皆親决百官受成於上而已河南府推官高錫上書諫以爲四海之廣萬機之衆雖堯舜不能獨治【治直之翻】必擇人而任之今陛下一以身親之天下不謂陛下聰明睿智足以兼百官之任皆言陛下褊迫疑忌舉不信羣臣也【褊輔典翻】不若選能知人公正者以爲宰相能愛民聽訟者以爲守令【守式又翻】能豐財足食者使掌金穀能原情守法者使掌刑獄陛下但垂拱明堂視其功過而賞罰之天下何憂不治何必降君尊而代臣職屈貴位而親賤事無乃失爲政之本乎帝不從錫河中人也 北漢主憂憤成疾悉以國事委其子侍衛都指揮使承鈞 河西節度使申師厚不俟詔擅棄鎭入朝【太祖廣順元年申師厚鎭河西事見二百九十卷】署其子爲留後秋七月癸酉朔責授率府副率【唐制東宫十率府皆有副率其後逐以爲冗散之官申師厚以藩府失職牙將而得節棄鎭擅歸雖加責授猶勝故吾】 丁丑加吳越王錢弘俶天下兵馬都元帥 癸巳加門下侍郎同平章事范質守司徒以樞密直學士工部侍郎長山景範爲中書侍郎同平章事判三司【長山漢於陵縣地江左僑置廣川縣及武彊縣隋廢郡改武彊曰長山唐屬淄州九域志在州北五十五里景姓也姓苑云齊景公之後余姑以春秋時言之晉宋皆有景公何獨齊哉】加樞密使同平章事鄭仁誨兼侍中乙未以樞密副使魏仁浦爲樞密使范質既爲司徒司徒竇貞固歸洛陽府縣以民視之【府縣謂河南府及洛陽縣也】課役皆不免貞固訴於留守向訓訓不聽【以竇貞固漢之舊臣故也考古驗今今何足怪】初帝與北漢主相距於高平命前澤州刺史李彦崇將兵守江猪嶺遏北漢主歸路彦崇聞樊愛能等南遁引兵退北漢主果自其路遁去八月己酉貶彦崇率府副率 己巳廢鎭國軍【唐末以華州爲鎭國軍】 初太祖以建雄節度使王晏有拒北漢之功【王晏拒北漢事見二百九十卷太祖廣順元年】其鄉里在滕縣徙晏爲武寧節度使【武寧軍徐州膝縣屬焉九域志滕縣在州北一百九十里】晏少時嘗爲羣盜【少詩照翻】至鎭悉召故黨贈之金帛鞍馬謂曰吾鄉素名多盜昔吾與諸君皆嘗爲之想後來者無能居諸君之右諸君幸爲我語之使勿復爲【幸爲于僞翻下請爲同語牛倨翻復扶又翻】爲者吾必族之於是一境清肅九月徐州人請爲之立衣錦碑【衣於既翻】許之 冬十月甲辰左羽林大將軍孟漢卿坐納藁税【藁禾稈也】塲官擾民多取耗餘【塲官藁塲之官耗餘者於納藁束正數之外又多取之言以備耗折也】賜死有司奏漢卿罪不至死上曰朕知之欲以懲衆耳 己酉廢安遠永清軍【唐以安州爲安遠軍晉以貝州爲永清軍】 初宿衛之士累朝相承務求姑息不欲簡閲恐傷人情由是羸老者居多【羸倫爲翻】但驕蹇不用命實不可用每遇大敵不走即降其所以失國亦多由此【如唐閔帝潞王是也】帝因高平之戰始知其弊癸亥謂侍臣曰凡兵務精不務多今以農夫百未能養甲士一奈何浚民之膏澤養此無用之物乎且健懦不分衆何所勸乃命大簡諸軍精銳者升之上軍羸者斥去之【去羌呂翻】又以驍勇之士多爲藩鎮所蓄詔募天下壯士咸遣詣闕命太祖皇帝選其尤者爲殿前諸班【今之班直是也五代會要曰時詔募天下豪傑不以草澤爲阻送於闕下躬親閱試選武藝超絶及有身首者分署爲殿前諸班因有散員散指揮使内殿直散都頭鐵騎控鶴之號】其騎步諸軍各命將帥選之【帥所類翻】由是士卒精彊近代無比征伐四方所向皆捷選練之力也【史言周世宗彊兵之效】 戊辰帝謂侍臣曰諸道盜賊頗多討捕終不能絶蓋由累朝分命使臣巡檢致藩侯守令皆不致力宜悉召還專委節鎭州縣責其清肅 河自楊劉至于博州百二十里連年東潰分爲二?匯爲大澤【?普拜翻匯戶罪翻水回合也】彌漫數百里又東北壞古堤而出【壞音怪古隄前代所築以防河者河屢徙故古隄在平地】灌齊棣淄諸州至于海涯漂没民田廬不可勝計流民採菰稗捕魚以給食【勝音升菰音孤蔣也稗旁卦翻草似穀者】朝廷屢遣使者不能塞【塞悉則翻下同】十一月戊戌帝遣李穀詣澶鄆齊按視隄塞役徒六萬三十日而畢 北漢主疾病命其子承鈞監國【疾甚曰病】尋殂【年六十 考異曰劉恕云世宗實錄薛史帝紀僭僞傳皆云顯德二年十二月劉崇卒大定錄云顯德二年春旻病死紀年通譜顯德二年崇之乾祜八年冬崇死顯德三年承鈞改元天會開寶元年承鈞之天會十三年死開寶二年繼元改元廣運興國四年繼元之廣運十一年也河東劉氏有國全無記録惟其舊臣中書舍人直翰林院王保衡歸朝後所纂晉陽偽署見聞要録云甲寅年春南伐敗歸夏周師攻圍旻積憂勞成心疾是冬卒鈞即位丁巳年正月日改乾祐十年爲天會元年又云鈞丙戌年二十九承位年四十三卒右諫議大夫楊夢申奉敕撰大漢都紇追封定王劉繼顒神道碑云天會十二年今皇帝踐阼之初年也十七年繼顒卒末題廣運元年歲次甲戌九月丙午朔今按周廣順元年辛亥旻即帝位稱乾祐四年顯德元年甲寅旻之乾祐七年也旻卒釣改元顯德四年丁巳鈞改乾祐十年爲天會元年宋開寶元年戊辰鈞之十二年也鈞卒繼元立開寶七年甲戌繼元改天會十八年爲廣運元年據歷是歲九月丙午朔興國四年己卯繼元之廣運六年也鈞以唐天成元年丙戌生至顯聽元年甲寅嗣位乃二十九歲矣釣及繼元踰年未改元盖孟蜀後主漢隱帝周世宗之比也諸書皆傳聞相因前後相戾惟晉陽見聞錄劉繼顒碑歲月最可考證故以爲據】遣使告哀于契丹契丹遣驃騎大將軍知内侍省事劉承訓冊命承鈞爲帝更名鈞【鈞漢主旻次子也更工衡翻】北漢孝和帝性孝謹既嗣位勤於爲政愛民禮士境内粗安【粗坐五翻】每上表於契丹主稱男契丹主賜之詔謂之兒皇帝 馬希萼之帥羣蠻破長沙也【事見二百八十九卷漢隱帝乾祐三年帥讀曰率】府庫累世之積皆爲溆州蠻酋苻彦通所掠【溆音叙酋慈由翻】彦通由是富彊稱王於谿洞間王逵既得湖南【去年六月王逵殺劉言始盡得湖南故地事見上卷】欲遣使撫之募能往者其將王䖍朗請行既至彦通盛侍衛而見之禮貌甚倨䖍朗厲聲責之曰足下自稱苻秦苖裔【苻秦之亡苻宏奔晉從諸桓於荆楚其後無聞彦通自以爲苻秦苗裔蓋言出於宏之後】宜知禮義有以異於羣蠻昔馬氏在湖南足下祖父皆北面事之今王公盡得馬氏之地足下不早往乞盟致使者先來又不接之以禮異日得無悔乎【言大兵若至雖悔無及】彦通慙懼起執䖍朗手謝之䖍朗知其可動因說之曰溪洞之地隋唐之世皆爲州縣著在圖籍【說式苪翻溪洞之地隋唐列爲郡縣皆屬黔中道】今足下上無天子之詔下無使府之命【使府謂湖南都府】雖自王於山谷之間【王于况翻】不過蠻夷一酋長耳【酋慈秋翻長知兩翻】曷若去王號自歸於王公王公必以天子之命授足下節度使與中國侯伯等夷豈不尊榮哉彦通大喜即日去王號【去羌呂翻】因䖍朗獻銅鼓數枚於王逵【谿洞諸蠻鑄銅爲大鼔初成懸於庭中置酒以招同類豪富子女則以金銀爲大釵執以扣鼔竟乃留遺主人名爲銅鼔釵俗好相殺多搆仇怨欲相攻則鳴此鼔至者如雲】逵曰䖍朗一言勝數萬兵真國士也承制以彦通爲黔中節度使【黔中自唐末至二蜀爲武泰軍節度黔其今翻】以䖍朗爲都指揮使預聞府政【預聞湖南都府之政】逵慮西界鎭遏使錦州刺史劉瑫爲邊患【王逵之逐邊鎬也以劉瑫鎭遏羣蠻】表爲鎭南節度副使【鎭南軍洪州属唐王逵表以其號寵劉瑫耳】充西界都招討使 是歲湖南大饑民食草木實武清節度使知潭州事周行逢【自彭師暠等擁立馬希萼於衡山自署武清節度使王逵因之以授周行逢】開倉以賑之全活甚衆行逢起於微賤知民間疾苦勵精爲治嚴而無私【治直吏翻】辟署僚屬皆取廉介之士約束簡要其自奉甚薄或譏其太儉行逢曰馬氏父子窮奢極靡不恤百姓今子孫乞食於人又足效乎【爲行逢跨有潭朗張本】世宗睿武孝文皇帝上【諱榮本姓柴氏邢州龍岡人柴氏女適太祖是爲聖穆皇后后兄守禮生帝從姑長於太祖家以謹厚見愛太祖遂以爲子】<br />
<br />
  顯德二年春正月庚辰上以漕運自晉漢以來不給斗耗綱吏多以虧欠抵死詔自今每斛給耗一斗 定難節度使李彛興【李彛興即彛殷也避宋朝宣祖廟諱始改名彛興史以後來所更名書之難乃旦翻】以折德扆亦爲節度使與已並列恥之【夏州自唐以來爲緣邊大鎭李氏又世襲節度使府州漢氏方置節鎭折氏父子又晚出故恥與並列】塞路不通周使【塞悉則翻】癸未上謀於宰相對曰夏州邊鎭朝廷向來每加優借府州褊小得失不繫重輕且宜撫諭彛興庶全大體上曰德扆數年以來盡忠戮力以拒劉氏奈何一旦棄之且夏州惟產羊馬貿易百貨悉仰中國【貿音茂仰牛向翻】我若絶之彼何能爲乃遣供奉官齊藏珍齎詔書責之【風俗通云凡氏之興九事氏於國者齊魯宋衛是也余按左傳衛有大夫齊氏此豈氏於國乎】彛興惶恐謝罪 戊子蜀置威武軍於鳳州 辛卯初令翰林學士兩省官舉令錄除官之日仍署舉者姓名若貪穢敗官並當連坐【敗補邁翻】契丹自晉漢以來屢寇河北輕騎深入無藩籬之限郊野之民每困殺掠言事者稱深冀之間有胡盧河横亘數百里可浚之以限其奔突【胡盧河俗謂之葫盧河即衡漳水在東光縣西三十里】是月詔忠武節度使王彦超彰信節度使韓通【周改曹州威信軍爲彰信軍避太祖諱也】將兵夫浚胡盧河築城於李晏口留兵戍之【冀州蓚縣東北有李晏鎭時築城屯軍以爲靜安軍按薛史其軍南距冀州百里北距深州三十里夾胡盧河爲壘將即亮翻】帝召德州刺史張藏英問以備邊之策藏英具陳地形要害請列置戍兵募邊人驍勇者厚其稟給自請將之隨便宜討擊帝皆從之以藏英爲沿邊巡檢招牧都指揮使藏英到官數月募得千餘人王彦超等行視役者【行下孟翻】嘗爲契丹所圍藏英引所募兵馳擊大破之自是契丹不敢涉胡盧河河南之民始得休息【此河南謂胡盧河之南也】 二月庚子朔日有食之 蜀夔恭孝王仁毅卒【仁毅蜀主之弟也】壬戌詔羣臣極言得失其畧曰朕於卿大夫才不能盡知面不能盡識若不采其言而觀其行【行下孟翻】審其意而察其忠則何以見器略之淺深知任用之當否【當丁浪翻】若言之不入罪實在予苟求之不言咎將誰執唐主以中書侍郎知尚書省嚴續爲門下侍郎同平章事 三月辛未以李晏口爲静安軍 帝常憤廣明以來中國日蹙【唐僖宗廣明元年黄巢入長安自此之後彊藩割據中國日蹙矣】及高平既捷慨然有削平天下之志會秦州民夷有詣大梁獻策請恢復舊疆者【以唐全盛版圖言之蜀亦舊疆也以漢晉事言之則契丹入中原重以王景崇之亂階成秦鳳遂入於蜀】帝納其言【爲取階成秦鳳張本】蜀主聞之遣客省使趙季札按視邊備季札素以文武才畧自任使還奏稱雄武節度使韓繼勲【蜀置雄武節度於秦州使疏吏翻還從宣翻又如字】鳳州刺史王萬迪非將帥才不足以禦大敵蜀主問誰可往者季札請自行丙申以季札爲雄武監軍使仍以宿衛精兵千人爲之部曲 帝以大梁城中迫隘【隘烏懈翻】夏四月乙卯詔展外城先立標幟【幟昌志翻】俟今冬農隙興板築東作動則罷之更俟次年以漸成之且令自今葬埋皆出所標七里之外其標内俟縣官分畫街衢倉塲營廨之外【廨古隘翻】聽民隨便築室丙辰蜀主命知樞密院王昭遠按行北邊城寨及甲<br />
<br />
  兵【以備周也行下孟翻】 上謂宰相曰朕每思致治之方未得其要寢食不忘【治直吏翻】又自唐晉以來吳蜀幽幷皆阻聲教未能混一【吳李氏蜀孟氏幽入於契丹并爲北漢】宜命近臣著爲君難爲臣不易論及開邊策各一篇朕將覽焉比部郎中王朴獻策【比音毗】以爲中國之失吳蜀幽幷皆由失道【梁失吳後唐得蜀而復失之晉失幽周失幷】今必先觀所以失之之原然後知所以取之之術其始失之也莫不以君暗臣邪兵驕民困姦黨内熾武夫外横【横戶孟翻】因小致大積微成著今欲取之莫若反其所爲而已夫進賢退不肖所以收其才也恩隱誠信所以結其心也【隱卹也】賞功罰罪所以盡其力也去奢節用所以豐其財也【去羌呂翻】時使薄斂所以阜其民也【時使使之以時歛力贍翻】俟羣才既集政事既治財用既充士民既附然後舉而用之功無不成矣彼之人觀我有必取之勢則知其情狀者願爲間諜知其山川者願爲鄉導【間古莧翻下伺間同諜達協翻鄉讀曰嚮】民心既歸天意必從矣凡攻取之道必先其易者唐與吾接境幾二千里其勢易擾也【唐與中國以淮爲境自淮源東至海幾二千里易以䜴翻】擾之當以無備之處爲始備東則擾西備西則擾東彼必奔走而救之奔走之間可以知其虚實彊弱然後避實擊虚避彊擊弱未須大舉且以輕兵擾之南人懦怯聞小有警必悉師以救之師數動則民疲而財竭【數所角翻】不悉師則我可以乘虚取之如此江北諸州將悉爲我有【帝之取江北王朴之計也】既得江北則用彼之民行我之法江南亦易取也得江南則嶺南巴蜀可傳檄而定【時劉氏據嶺南孟氏據巴蜀王朴欲乘勝勢以先聲下之】南方既定則燕地必望風内附【時契丹跨有燕地燕於賢翻】若其不至移兵攻之席卷可平矣【卷讀如捲凡兵之動知敵之主此以其時契丹主言之也】惟河東必死之寇【言北漢據河東與周為世仇也】不可以恩信誘【誘音酉】當以彊兵制之然彼自高平之敗【事見上卷上年三月】力竭氣沮必未能為邊患宜且以為後圖俟天下既平然後伺間一舉可擒也【是後世宗用兵以至宋朝削平諸國皆如王朴之言惟幽燕不可得而取至於宣和則舉國以殉之矣伺相吏翻】今士卒精練甲兵有備羣下畏法諸將効力期年之後可以出師【期讀曰朞】宜自夏秋蓄積實邊矣【蓄積於邊上以為用兵之備】上欣然納之時羣臣多守常偷安所對少有可取者【少詩沼翻】惟朴神峻氣勁有謀能斷【斷丁亂翻】凡所規畫皆稱上意【稱尺證翻】上由是重其器識未幾遷左諫議大夫知開封府事【開封在輦轂下事繁職重史言世宗屬任王朴自此而重然朴先事上於濳藩其君臣相得亦有素矣】 上謀取秦鳳求可將者【將即亮翻】王溥薦宣徽南院使鎭安節度使向訓【五代會要漢天福十二年廢陳州鎭安軍周廣順二年復】上命訓與鳳翔節度使王景客省使高唐昝居潤偕行【高唐縣屬博州九域志縣在州東北一百七十里昝姓也音子感翻】五月戊辰朔景出兵自散關趣秦州【趣七喻翻】 勑天下寺院非勑額者悉廢之【勑額者勑賜寺額如慈恩安國興唐之類】禁私度僧尼凡欲出家者必俟祖父母父母伯叔之命惟兩京大名府【唐以魏州為鄴都興唐府晉改為廣晉府大名府盖漢所改也】京兆府青州聽設戒壇【戒壇僧尼受戒之所】禁僧俗捨身斷手足煉指掛燈帶鉗之類幻惑流俗者【煉指者束香於指而燃之掛燈者裸體以小鐵鈎徧鈎其膚凡鈎皆掛小燈圈燈盞貯油而燃之俚俗謂之燃肉身燈今人帶布枷以化誘流俗者亦幻惑之餘蔽斷音短幻戶辦翻】令兩京及諸州每歲造僧帳有死亡歸俗皆隨時開落是歲天下寺院存者二千六百九十四廢者三萬三百三十六見僧四萬二千四百四十四尼一萬八千七百五十六【見賢遍翻】 王景等拔黄牛八寨【黄牛等八寨皆當在秦州界】戊寅蜀主以捧聖控鶴都指揮使保寧節度使李廷珪為北路行營都統【蜀以秦鳳為北路】左衛聖步軍都指揮使高彦儔為招討使武寧節度使呂彦珂副之【武寧軍徐州屬周呂彦珂遙領也珂何翻】客省使趙崇韜為都監蜀趙季札至德陽聞周師入境懼不敢進【德陽縣屬漢州去成】<br />
<br />
  【都未遠已懼而不敢進】上書求解邊任還奏事先遣輜重及妓妾西歸【重直用翻妓渠綺翻】丁亥單騎馳入成都衆以為奔敗莫不震恐蜀主問以機事皆不能對蜀主怒繫之御史臺庚午斬之於崇禮門【趙季札雖誅無救於秦鳳之喪失是以用人當審之於其初也】 六月庚子上親録囚於内苑有汝州民馬遇父及弟為吏所寃死屢經覆按不能自伸上臨問始得其實人以為神由是諸長吏無不親察獄訟【史言帝明謹於庶獄長知兩翻】 壬寅西師與蜀李廷珪等戰于威武城東不利【威武城前蜀所築也在鳳州東北】排陳使濮州刺史胡立等為蜀所擒【陳讀曰陣濮博木翻】丁未蜀主遣間使如北漢及唐【間古莧翻使疏吏翻】欲與之俱出兵以制周北漢主唐主皆許之己酉以彰信節度使韓通充西南行營馬步軍都虞候 戊午南漢主殺禎州節度使通王弘政【禎州漢博羅縣之地梁置梁化郡隋置循州治歸善縣唐因之至南唐改唐之河源縣為龍川縣徙循州治焉以循州舊治歸善縣置禎州宋朝避仁宗諱改曰惠州九域志循州南至惠州三百里】於是高祖之諸子盡矣【南漢主龑廟號高祖】 壬戌以樞密院承旨清河張美為右領軍大將軍權點檢三司事【清河縣帶貝州權點檢三司事未除為正使】初帝在澶州美掌州之金穀隸三司者帝或私有所求美曲為供副【供副者供辦以應副所求澶時連翻】太祖聞之怒恐傷帝意但徙美為濮州馬步軍都虞候【濮博木翻】美治財精敏當時鮮及【治直之翻鮮息淺翻】故帝以利權授之然思其在澶州所為終不以公忠待之【自漢以來能如此者吳主孫權及周世宗而已】 秋七月丁卯朔以王景兼西南行營都招討使向訓兼行營兵馬都監宰相以景等久無功饋運不繼固請罷兵帝命太祖皇<br />
<br />
  帝往視之還言秦鳳可取之狀【還從宣翻又如字】帝從<br />
<br />
  之 八月丁未中書侍郎同平章事景範罷判三司尋以父喪罷政事 王景等敗蜀兵【敗補邁翻】獲將卒三百己未蜀主遣通奏使知樞密院武泰節度使伊審徵如行營慰撫仍督戰 帝以縣官久不鑄錢而民間多銷錢為器皿及佛像錢益少九月丙寅朔敇始立監采銅鑄錢自非縣官法物軍器及寺觀鐘磬鈸鐸之類【觀古玩翻鈸蒲撥翻】聽留外【句斷】自餘民間銅器佛像五十日内悉令輸官給其直過期隱匿不輸五觔以上其罪死不及者論刑有差【輸舂遇翻時勅有隱藏銅器及埋窖使用者一兩至一觔徙二年一觔至五觔處死若納到熟銅每觔官給錢一百五十生銅每觔一百】上謂侍臣曰卿輩勿以毁佛為疑夫佛以善道化人苟志於善斯奉佛矣彼銅像豈所謂佛邪且吾聞佛在利人雖頭目猶捨以布施【施式䜴翻】若朕身可以濟民亦非所惜也<br />
<br />
  臣光曰若周世宗可謂仁矣不愛其身而愛民若周世宗可謂明矣不以無益廢有益<br />
<br />
  蜀李廷珪遣先鋒都指揮使李進據馬嶺寨又遣奇兵出斜谷屯白澗【九域志鳳州梁泉縣有白澗鎭】又分兵出鳳州之北唐倉鎭及黄花谷絶周糧道【黄花鎭亦在梁泉縣界有黄花川大散水入焉】閏月王景遣裨將張建雄將兵二千抵黄花又遣千人趣唐倉扼蜀歸路【趣七喻翻】蜀染院使王巒將兵出唐倉與建雄戰於黄花蜀兵敗奔唐倉遇周兵又敗虜巒及其將士三千人馬嶺白澗兵皆潰李廷珪高彦儔等退保青泥嶺蜀雄武節度使兼侍中韓繼勲棄秦州奔還成都觀察判官趙玭舉城降斜谷援兵亦潰【玭蒲眠翻 考異曰十國紀年玭召官屬告之曰周兵無敵今朝廷所遣勇將精兵不死即逃我輩不能去危就安禍且至矣衆皆聽命舉城叛降周斜谷援兵亦潰五代通録官軍之圍鳳州偽秦州節度使高處儔引兵往復援之中塗聞黄花之敗奔秦州玭與城中將校閉門不納處儔遂西奔玭即以城歸國今從實録】成階二州皆降蜀人震恐玭澶州人也【澶時連翻】帝欲以玭為節度使范質固爭以為不可乃以為郢州刺史壬子百官入賀帝舉酒屬王溥曰邊功之成卿擇帥之力也【屬之欲翻擇帥事見上四月帥所類翻】甲子上與將相食於萬歲殿因言兩日大寒朕於宫<br />
<br />
  中食珍膳深愧無功於民而坐享天禄既不能躬耕而食惟當親冒矢石為民除害差可自安耳【冒莫北翻為于偽翻差之為言稍也】 乙丑蜀李廷珪上表待罪冬十月壬申伊審徵至成都請罪【皆以蹙國喪帥也】蜀主致書於帝請和自稱大蜀皇帝帝怒其抗禮不答蜀主愈恐聚兵糧於劒門白帝為守禦之備【守劒門以備北兵之自岐雍來者守白帝以備北兵之泝峽而上者】募兵既多用度不足始鑄鐵錢榷境内鐵器民甚苦之【榷古岳翻】唐主性和柔好文章而喜人佞已【好呼到翻喜許記翻】由是諂諛之臣多進用【諂諛之臣謂馮延己兄弟魏岑陳覺等】政事日亂既克建州破湖南【克建州見二百八十四卷晉開運二年破湖南見二百九十卷太祖廣順二年】益驕有吞天下之志李守貞慕容彦超之叛皆為之出師遙為聲援【援李守貞見二百八十八卷漢隱帝乾祐元年乂援慕容彦超見二百九十卷太祖廣順二年皆為于偽翻】又遣使自海道通契丹及北漢約共圖中國值中國多事未暇與之校先是每冬淮水淺涸唐人常發兵戍守謂之把淺【先悉薦翻把淺之處自霍以上西盡光州界】壽州監軍吳廷紹以為疆場無事坐費資糧悉罷之【史先叙唐所以蹙國之由場音亦】清淮節度使劉仁贍上表固爭不能得十一月乙未朔帝以李穀為淮南道前軍行營都部署兼知廬壽等行府事以忠武節度使王彦超副之督侍衛馬軍都指揮使韓令坤等十二將以伐唐令坤磁州武安人也【磁詳之翻】汴水自唐末潰决自埇橋東南悉為汚澤上謀擊唐<br />
<br />
  先命武寧節度使武行德發民夫因故堤疏導之【自埇橋東南抵唐境皆武寧廵屬也埇余拱翻】東至泗上議者皆以為難成上曰數年之後必獲其利【謂淮南既平藉以通漕將獲其利也】 丁未上與侍臣論刑賞上曰朕必不因怒刑人因喜賞人 先是大梁城中民侵街衢為舍通大車者蓋寡【先悉薦翻】上命悉直而廣之廣者至三十步【此言横廣也】又遷墳墓於標外【立標幟見上四月】上曰近廣京城於存没擾動誠多怨謗之語朕自當之他日終為人利【世宗志識宏遠不顧人言然仁人不忍為也】 王景等圍鳳州韓通分兵城固鎭以絶蜀之援兵戊申克鳳州擒蜀威武節度使王環【是年正月蜀置威武節度於鳳州】及都監趙崇溥等將士五千人崇溥不食而死環真定人也【蜀將士多中原人盖後唐遣之戍蜀為孟知祥所留者也】乙卯制曲赦秦鳳階成境内所獲蜀將士願留者優其俸賜願去者給資裝而遣之詔曰用慰衆情免違物性其四州之民二税徵科之外凡蜀人所立諸色科徭悉罷之 唐人聞周兵將至而懼劉仁贍神氣自若部分守禦無異平日衆情稍安【劉仁贍之善守於此已見其方畧分扶問翻】唐主以神武統軍劉彦貞為北面行營都部署將兵二萬趣壽州【趣七喻翻】奉化節度使同平章事皇甫暉為應援使【唐置奉化軍於江州】常州團練使姚鳳為應援都監將兵三萬屯定遠【舊唐書地理志曰定遠漢曲陽縣地隋置定遠縣唐屬濠州九域志在州南八十里】召鎭南節度使宋齊丘還金陵謀國難【還從宣翻又如字難乃旦翻】以翰林承旨戶部尚書殷崇義為吏部尚書知樞密院 李穀等為浮梁自正陽濟淮十二月甲戌穀奏王彦超敗唐兵二千餘人於壽州城下【敗補邁翻下同】己卯又奏先鋒都指揮使白延遇敗唐兵千餘人於山口鎭【此時唐蓋置鎭於六安山口按薛史本紀顯德四年劉重遇奏殺紫金山潰兵三千人於壽州東山口又疑置鎭於此地未知孰是】 丙戌樞密使兼侍中韓忠正公鄭仁誨卒上臨其喪近臣奏稱歲道非便上曰君臣義重何日時之有往哭盡哀 吳越王弘俶遣元帥府判官陳彦禧入貢【朝廷授弘俶天下都元帥故置元帥府判官】帝以詔諭弘俶使出兵擊唐【使出兵常州以擊之則唐有反顧之憂為吳越兵為唐所敗張本】<br />
<br />
  三年春正月丙午以王環為右驍衛大將軍賞其不降也【以王環堅守鳳州城陷而後就擒也】 丁酉李穀奏敗唐兵千餘人於上窑【窑餘招翻又作窯】 戊戌發開封府曹滑鄭州之民十餘萬築大梁外城【曹滑鄭皆近京之州九域志開封府西至鄭州界一百一十五里北至滑州界一百里東北至曹州界一百四十五里陳許亦近郡而不發者以方征淮南道上供億故也】 庚子帝下詔親征淮南以宣徽南院使鎭安節度使向訓權東京留守端明殿學士王朴副之彰信節度使韓通權點檢侍衛司及在京内外都廵檢命侍衛都指揮使歸德節度使李重進將兵先赴正陽河陽節度使白重贊將親兵三千屯潁上【潁上縣隋置唐屬潁州九域志在州東一百一十七里宋白曰潁上縣漢慎縣也南北畫淮為守關防莫謹於此隋大業二年於今縣南故鄭城置潁上縣以地枕潁水上游為名】壬寅帝發大梁李穀攻壽州久不克唐劉彦貞引兵救之至來遠鎭【九域志壽州安豐縣有來遠鎭今按來遠鎭即東正陽西至渒河十里】距壽州二百里又以戰艦數百艘趣正陽【趣七喻翻下同】為攻浮梁之勢李穀畏之召將佐謀曰我軍不能水戰若賊斷浮梁【斷音短】則腹背受敵皆不歸矣不如退守浮梁以待車駕上至圉鎭【九域志開封雍丘縣有圉城鎭】聞其謀亟遣中使乘驛止之比至【比必利翻及也】已焚芻糧退保正陽丁未帝至陳州【九域志開封府南至陳州三百三十里】亟遣李重進引兵趣淮上辛亥李穀奏賊艦中流而進弩礮所不能及【艦戶黯翻礮與砲同普教翻】若浮梁不守則衆心動搖須至退軍今賊艦日進淮水日漲【奏水方生故李穀慮淮水日漲】若車駕親臨萬一糧道阻絶其危不測願陛下且駐蹕陳潁【陳潁二州名】俟李重進至臣與之共度賊艦可禦浮梁可完立具奏聞【度徒洛翻先為不可勝以待敵之可勝李穀之退未為失計也】但若厲兵秣馬春去冬來足使賊中疲弊取之未晩帝覧奏不悦劉彦貞素驕貴無才畧不習兵所歷藩鎭專為貪暴積財巨億以賂權要【萬萬為億億億為巨億詩所謂萬億及秭孔頴達所謂大數也】由是魏岑等爭譽之【譽音余】以為治民如龔黄用兵如韓彭【龔遂黄覇漢之良吏韓信彭越漢之良將治直之翻】故周師至唐主首用之其禆將咸師朗等皆勇而無謀聞李穀退喜引兵直抵正陽旌旗輜重數百里【重直用翻】劉仁贍及池州刺史張全約固止之仁贍曰公軍未至而敵人先遁是畏公之威聲也安用速戰萬一失利則大事去矣彦貞不從既行仁贍曰果遇必敗乃益兵乘城為備【以城中戰兵乘城益守兵】李重進度淮逆戰於正陽東大破之【淮水西岸謂之西正陽屬潁州潁上縣界東岸謂之東正陽屬夀州下蔡縣界此據九域志地里】斬彦貞生擒咸師朗等斬首萬餘級伏尸三十里收軍資器械三十餘萬是時江淮久安民不習戰彦貞既敗唐人大恐張全約收餘衆奔壽州劉仁贍表全約為馬步左廂都指揮使皇甫暉姚鳳退保清流關【梁置南譙州於桑根山之陽在滁州清流縣西南八十里隋始置清流縣唐為滁州治所清流關在縣西南二十餘里南唐所置也】滁州刺史王紹顔委城走壬子帝至永寧鎮【九域志黄州麻城縣有永寧鎮此非也麻城在夀州西南數百里帝猶未度淮安得至麻城之永寧邪又考九域志潁州汝隂縣有永寧鎮又東百餘里至正陽此則是也】謂侍臣曰聞夀州圍解農民多歸村落今聞大軍至必復入城憐其聚為餓殍【復扶又翻殍被表翻】宜先遣使存撫各令安業甲寅帝至正陽以李重進代李穀為淮南道行營都招討使以穀判夀州行府事【宋敏求曰凡節度州為三品刺史州為五品國初曹翰以觀察使判潁州是以四品臨五品州也同品為知隔品為判自後唯輔臣宣徽使太子太保僕射為判餘並為知州】丙辰帝至夀州城下營於淝水之陽【淝水自安豐縣界流入夀春縣界經夀春城北入於淮去城二里水北為陽】命諸軍圍夀州徙正陽浮梁於下蔡鎮【唐潁州有下蔡縣時廢縣為鎮西抵正陽五十五里】丁巳徵宋亳陳潁徐宿許蔡等州丁夫數十萬以攻城晝夜不息唐兵萬餘人維舟於淮營於塗山之下【塗山在濠州本塗山氏之邑禹會諸侯處也今在鍾離縣西九十五里濱淮有漢當塗縣故城南北朝兵爭之際為馬頭郡城淮水逕城北而東流渦水自西北來注於淮謂之渦口南岸正對馬頭城】庚申帝命太祖皇帝擊之太祖皇帝遣百餘騎薄其營而偽遁伏兵邀之<br />
<br />
  大敗唐兵於渦口【敗補邁翻渦音戈】斬其都監何延錫<br />
<br />
  等奪戰艦五十餘艘【艘蘇遭翻】 詔以武平節度使<br />
<br />
  兼中書令王逵為南面行營都統使攻唐之鄂<br />
<br />
  州逵引兵過岳州岳州團練使潘叔嗣厚具燕<br />
<br />
  犒奉事甚謹逵左右求取無厭【犒苦到翻厭於鹽翻】不滿<br />
<br />
  望者譖叔嗣於逵云其謀叛逵怒形於詞色叔<br />
<br />
  嗣由是懼而不自安【為潘叔嗣殺王逵張本】唐主聞湖南<br />
<br />
  兵將至命武昌節度使何敬洙徙民入城為固<br />
<br />
  守之計敬洙不從使除地為戰場曰敵至則與<br />
<br />
  軍民俱死於此耳【何敬洙為將亦唐之良也因王逵有潘叔嗣之難又以成】<br />
<br />
  【其名】唐主善之 二月丙寅下蔡浮梁成上自往<br />
<br />
  視之戊辰廬夀光黄巡檢使司超【司姓也左傳鄭有司臣】<br />
<br />
  奏敗唐兵三千餘人於盛唐【敗補邁翻盛唐本唐初之霍山縣也】<br />
<br />
  【開元二十七年更名盛唐屬夀州宋朝開寶四年改為六安縣九域志六安縣在夀州南二百】<br />
<br />
  【一十里】擒都監高弼等獲戰艦四十餘艘上命太<br />
<br />
  祖皇帝倍道襲清流關皇甫暉等陳於山下【陳讀】<br />
<br />
  【曰陣】方與前鋒戰太祖皇帝引兵出山後暉等大<br />
<br />
  驚走入滁州【宋白曰滁州之地劉宋為新昌郡梁立南譙州於桑根山西今州西】<br />
<br />
  【南十八里南譙故城是也北齊自南譙徙新昌郡今州城是也隋廢州以其地為清流縣唐為滁州】欲斷橋自守【斷音短】太祖皇帝躍馬麾兵涉水<br />
<br />
  直抵城下暉曰人各為其主【為于偽翻】願容成列而<br />
<br />
  戰【皇甫暉受唐莊宗畜養之恩一旦作亂莊宗以之喪亡棄中國而奔江南委質於唐乃言】<br />
<br />
  【人各為其主蓋兵鋒所迫倉皇失措為是言以欵敵耳】太祖皇帝笑而許<br />
<br />
  之【太祖自審智勇足以辦皇甫暉故許之】暉整衆而出太祖皇帝<br />
<br />
  擁馬頸突陳而入【陳讀曰陣】大呼曰吾止取皇甫暉<br />
<br />
  它人非吾敵也手劍擊暉中腦生擒之【呼火故翻手式】<br />
<br />
  【又翻中竹仲翻】并擒姚鳳遂克滁州後數日宣祖皇帝<br />
<br />
  為馬軍副都指揮使【宣祖諱弘殷】引兵夜半至滁州<br />
<br />
  城下傳呼開門太祖皇帝曰父子雖至親城門<br />
<br />
  王事也不敢奉命【史言太祖勇於戰謹於守】上遣翰林學士<br />
<br />
  竇儀籍滁州帑藏【帑它朗翻藏徂浪翻下同】太祖皇帝遣親<br />
<br />
  吏取藏中絹儀曰公初克城時雖傾藏取之無<br />
<br />
  傷也今既籍為官物非有詔書不可得也【竇儀有守】<br />
<br />
  太祖皇帝由是重儀【太祖之識度豈一時將帥所能及哉】詔左金<br />
<br />
  吾衛將軍馬崇祚知滁州初永興節度使劉<br />
<br />
  詞遺表薦其幕僚薊人趙普有才可用會滁<br />
<br />
  州平范質薦普為滁州軍事判官太祖皇帝<br />
<br />
  與語悦之時獲盜百餘人皆應死普請先訊<br />
<br />
  鞫然後决所活十七八太祖皇帝益奇之【太祖】<br />
<br />
  【重竇儀奇趙普皆在潛躍之時普自此為佐命元功儀乃為普所忌而不至相位】太祖<br />
<br />
  皇帝威名日盛每臨陳必以繁纓飾馬鎧仗鮮<br />
<br />
  明或曰如此為敵所識【陳讀曰陣繁蒲官翻】太祖皇帝曰<br />
<br />
  吾固欲其識之耳唐主遣泗州牙將王知朗齎<br />
<br />
  書抵徐州稱唐皇帝奉書大周皇帝請息兵修<br />
<br />
  好願以兄事帝歲輸貨財以助軍費【好呼到翻輸舂遇翻】甲戌徐州以聞【九域志泗州西北至徐州七百五十里王知朗不敢詣軍前而扺徐州恐犯兵鋒而死也】帝不答【以唐主猶敢抗禮欲為兄弟之國也】戊寅命前武勝節度使侯章等攻夀州水寨决其壕之西北隅導壕水入于淝太祖皇帝遣使獻皇甫暉等暉傷甚見上卧而言曰臣非不忠於所事但士卒勇怯不同耳臣曏日屢與契丹戰【皇甫暉本魏兵唐莊宗使戍瓦橋拒契丹因而作亂其自謂屢與契丹戰蓋戍瓦橋時也】未嘗見兵精如此因盛稱太祖皇帝之勇上釋之後數日卒帝詗知揚州無備【詗古永翻又翾正翻】己卯命韓令坤等將兵襲之戒以毋得殘民其李氏陵<br />
<br />
  寢遣人與李氏人共守護之唐主兵屢敗懼亡<br />
<br />
  乃遣翰林學士戶部侍郎鍾謨工部侍郎文理院學士李德明奉表稱臣來請平獻御服湯藥及金器千兩銀器五千兩繒錦二千匹【繒慈陵翻】犒軍牛五百頭酒二千斛壬午至夀州城下謨德明素辯口上知其欲遊說【犒苦到翻說式芮翻下欲說同】盛陳甲兵而見之曰爾主自謂唐室苗裔【南唐祖唐太宗之子吳王恪】宜知禮義異於他國與朕止隔一水【謂南唐與周以淮為界】未嘗遣一介修好【好呼到翻】惟泛海通契丹捨華事夷禮義安在【自徐温執吳政屢泛海使契丹欲與共圖中國至唐烈祖及今主皆然】且汝欲說我令罷兵邪我非六國愚主豈汝口舌所能移邪【言六國皆愚主故蘇張得行其遊說使遇英明之君雖辯如蘇張不能移也】可歸語汝主【語牛倨翻】亟來見朕再拜謝過則無事矣不然朕欲觀金陵城借府庫以勞軍【勞力到翻】汝君臣得無悔乎謨德明戰栗不敢言 吳越王弘俶遣兵屯境上以俟周命蘇州營田指揮使陳滿言於丞相吳程曰周師南征唐舉國驚擾常州無備易取也【九域志蘇州西北至常州一百八十餘里易以䜴翻】會唐主有詔撫安江隂吏民【江隂縣本晉毗陵之暨陽縣也江左分置蘭陵縣梁敬帝時常置江隂郡及江隂縣隋廢唐置暨州南唐始置江隂軍九域志在常州東北九十里】滿告程云周詔書已至程為之言於弘俶【為于偽翻】請亟發兵從其策丞相元德昭曰唐大國未可輕也若我入唐境而周師不至誰與并力能無危乎請姑俟之程固爭以為時不可失弘俶卒從程議【卒子恤翻】癸未遣程督衢州刺史鮑修讓中直都指揮使羅晟趣常州【趣七喻翻】程謂將士曰元丞相不欲出師將士怒流言欲擊德昭【吳越將士狃福州之勝謂唐之可乘也兵驕者破豈虚言哉】弘俶匿德昭于府中令捕言者歎曰方出師而士卒欲擊丞相不祥甚哉 乙酉韓令坤奄至揚州平旦先遣白延遇以數百騎馳入城城中不之覺令坤繼至唐東都營屯使賈崇焚官府民舍棄城南走副留守工部侍郎馮延魯髠髮被僧服匿于佛寺【唐以揚州為東都故置留守髠苦昆翻被皮義翻】軍士執之令坤慰撫其民使皆安堵庚寅王逵奏拔鄂州長山寨<br />
<br />
  【長山在鄂州南界唐立寨以備潭朗】執其將陳澤等獻之辛卯太祖皇帝奏唐天長制置使耿謙降【唐天寶元年分江都六合高郵三縣地置千秋縣七載改名天長九域志天長縣在揚州西一百一十里】獲芻糧二十餘萬唐主遣園苑使尹延範如泰州【梁有宫苑使又有内園栽接使唐置園苑使亦猶是也】遷吳讓皇之族于潤州【晉天福四年唐烈祖自潤州遷讓皇之族于泰州今以周師攻逼復遷潤州】延範以道路艱難恐楊氏為變盡殺其男子六十人還報唐主怒腰斬之 韓令坤等攻泰州拔之【南唐升海陵鎮為泰州九域志揚州東至泰州一百一十五里】刺史方訥奔金陵【自泰州南奔泰興縣度江取潤州至金陵】 唐主遣人以蠟丸求救于契丹壬辰靜安軍使何繼筠獲而獻之【去年帝置靜安軍于李晏口】 以給事中高防權知泰州 癸巳吳越王弘俶遣上直都指揮使路彦銖攻宣州羅晟帥戰艦屯江隂唐靜海制置使姚彦洪帥兵民萬人奔吳越【帥讀曰率南唐于海陵之東境置靜海都鎮制置院周取其地置靜海軍尋升為通州通州南至大江二十四里絶江而南即吳越之蘇州界】 潘叔嗣屬將士而告之曰吾事令公至矣【屬之欲翻集會也王逵兼中書令故稱為令公】今乃信讒疑怒軍還必擊我吾不能坐而待死汝輩能與吾俱西乎衆憤怒請行叔嗣帥之西襲朗州【九域志岳州西至朗州五百五十里帥讀曰率】逵聞之還軍追之及于武陵城外【朗州武陵郡】與叔嗣戰逵敗死 【考異曰湖湘故事云王逵奉詔伐吳有蜜蜂無萬數集逵繖蓋周行逢内喜潛與潘叔嗣張文表等謀曰我覩王公妖怪入繖他時忽落别人之手我輩處身何地我等若三人同心共保馬氏舊基同處富貴豈不是男兒哉叔嗣文表聞行逢之言已會深意遂乃拜受此語各散歸營廣本逵命行營副使毛立為袁州營統軍使潘叔嗣張文表為前鋒軍次醴陵縣吏請具牛酒犒軍立不許叔嗣文表因士卒之怒縛立送于行逢以兵叛逵逵大懼乘輕舟奔朗州叔嗣追至朗州殺之湖湘故事逵連夜走歸朗州去經數日潘叔嗣始到潭州既聞王逵走歸朗州亦以舟楫倍程而趨至朗州殺之今按世宗實録顯德三年二月丙寅朗州王逵言領大軍入淮南界庚寅言入鄂州界攻下長山寨癸巳荆南高保融言進逵自鄂州領兵復歸本道又云潘叔嗣為先鋒行及鄂州叔嗣囘戈襲武陵進逵聞之倍道先入武陵叔嗣攻其城進逵敗走為叔嗣所殺又云三月壬寅進逵差牙將押送淮南將陳澤等蓋進逵未敗前奏事三月始逵行在與薛史承襲傳及湖南傳記略同惟湖湘故事及丁璹馬氏行事記載逵攻袁州叔嗣叛之丁璹云逵三月至潭州四月叔嗣叛丁璹云五月五日叔嗣殺逵于朗州皆妄也周行逢據湖南仕進尚門䕃衍屢獻文章不得調退居鄉里教授及張文表叛辟為幕職事敗逃遁會赦乃敢出窮困無以自進採摭故事撰湖湘馬氏故事二十卷如京師獻之太宗憫其窮且老授將作監丞衍本小人言辭鄙俚非有意著書故叙事顛倒前後自相違背以無為有不可勝數素怨周行逢尤多誣毁不欲行逢不預叔嗣之謀乃妄造此說凡載行逢罪惡之甚者皆出于衍云璹亦國初人疑其說得于衍書皆不可為據今從十國紀年】或勸叔嗣遂據朗州叔嗣曰吾救死耳安敢自尊宜以督府歸潭州太尉【時湖湘以朗州為督府潭州太尉謂周行逢也】豈不以武安見處乎【言行逢必將以潭州處已處昌呂翻】乃歸岳州使團練判官李簡【潘叔嗣為岳州團練使判官其屬也】帥朗州將吏迎武安節度使周行逢【帥讀曰率下同】衆謂行逢必以潭州授叔嗣【謂告也語也】行逢曰叔嗣賊殺主帥罪當族所可恕者得武陵而不有以授吾耳若遽用為節度使天下謂我與之同謀何以自明宜且以為行軍司馬俟踰年授以節鉞可也【觀此則周行逢本有奉辭討潘叔嗣之心以其迎已故不發耳】乃以衡州刺史莫弘萬權知潭州帥衆入朗州自稱武平武安留後告于朝廷【行逢欲兼有潭朗也】以叔嗣為行軍司馬叔嗣怒稱疾不至行逢曰行軍司馬吾嘗為之【周行逢為武安行軍司馬見上卷太祖廣順三年】權與節度使相埒耳【埒龍輟翻等也】叔嗣猶不滿望更欲圖我邪或說行逢授叔嗣武安節鉞以誘之【說式芮翻誘以九翻】令至都府受命此乃机上肉耳行逢從之叔嗣將行其所親止之叔嗣自恃素以兄事行逢相親善【行逢叔嗣親善事始見二百九十一卷太祖廣順二年九月】遂行不疑行逢遣使迎候道路相望既至自出郊勞【勞力到翻】相見甚懽叔嗣入謁未至聽事【聽讀曰廳】遣人執之立于庭下責之曰汝為小校無大功【校戶教翻】王逵用汝為團練使一旦反殺主帥【帥所類翻】吾以疇昔之情未忍斬汝以為行軍司馬乃敢違拒吾命而不受乎叔嗣知不免以宗族為請遂斬之<br />
<br />
  資治通鑑卷二百九十二<br />
<br />
<史部,編年類,資治通鑑>  <br>
   </div> 

<script src="/search/ajaxskft.js"> </script>
 <div class="clear"></div>
<br>
<br>
 <!-- a.d-->

 <!--
<div class="info_share">
</div> 
-->
 <!--info_share--></div>   <!-- end info_content-->
  </div> <!-- end l-->

<div class="r">   <!--r-->



<div class="sidebar"  style="margin-bottom:2px;">

 
<div class="sidebar_title">工具类大全</div>
<div class="sidebar_info">
<strong><a href="http://www.guoxuedashi.com/lsditu/" target="_blank">历史地图</a></strong>  
<a href="http://www.880114.com/" target="_blank">英语宝典</a>  
<a href="http://www.guoxuedashi.com/13jing/" target="_blank">十三经检索</a> 
<br><strong><a href="http://www.guoxuedashi.com/gjtsjc/" target="_blank">古今图书集成</a></strong> 
<a href="http://www.guoxuedashi.com/duilian/" target="_blank">对联大全</a> <strong><a href="http://www.guoxuedashi.com/xiangxingzi/" target="_blank">象形文字典</a></strong> 

<br><a href="http://www.guoxuedashi.com/zixing/yanbian/">字形演变</a>  <strong><a href="http://www.guoxuemi.com/hafo/" target="_blank">哈佛燕京中文善本特藏</a></strong>
<br><strong><a href="http://www.guoxuedashi.com/csfz/" target="_blank">丛书&方志检索器</a></strong> <a href="http://www.guoxuedashi.com/yqjyy/" target="_blank">一切经音义</a>  

<br><strong><a href="http://www.guoxuedashi.com/jiapu/" target="_blank">家谱族谱查询</a></strong>  <strong><a href="http://shufa.guoxuedashi.com/sfzitie/" target="_blank">书法字帖欣赏</a></strong> 
<br>

</div>
</div>


<div class="sidebar" style="margin-bottom:0px;">

<font style="font-size:22px;line-height:32px">QQ交流群9:489193090</font>


<div class="sidebar_title">手机APP 扫描或点击</div>
<div class="sidebar_info">
<table>
<tr>
	<td width=160><a href="http://m.guoxuedashi.com/app/" target="_blank"><img src="/img/gxds-sj.png" width="140"  border="0" alt="国学大师手机版"></a></td>
	<td>
<a href="http://www.guoxuedashi.com/download/" target="_blank">app软件下载专区</a><br>
<a href="http://www.guoxuedashi.com/download/gxds.php" target="_blank">《国学大师》下载</a><br>
<a href="http://www.guoxuedashi.com/download/kxzd.php" target="_blank">《汉字宝典》下载</a><br>
<a href="http://www.guoxuedashi.com/download/scqbd.php" target="_blank">《诗词曲宝典》下载</a><br>
<a href="http://www.guoxuedashi.com/SiKuQuanShu/skqs.php" target="_blank">《四库全书》下载</a><br>
</td>
</tr>
</table>

</div>
</div>


<div class="sidebar2">
<center>


</center>
</div>

<div class="sidebar"  style="margin-bottom:2px;">
<div class="sidebar_title">网站使用教程</div>
<div class="sidebar_info">
<a href="http://www.guoxuedashi.com/help/gjsearch.php" target="_blank">如何在国学大师网下载古籍?</a><br>
<a href="http://www.guoxuedashi.com/zidian/bujian/bjjc.php" target="_blank">如何使用部件查字法快速查字?</a><br>
<a href="http://www.guoxuedashi.com/search/sjc.php" target="_blank">如何在指定的书籍中全文检索?</a><br>
<a href="http://www.guoxuedashi.com/search/skjc.php" target="_blank">如何找到一句话在《四库全书》哪一页?</a><br>
</div>
</div>


<div class="sidebar">
<div class="sidebar_title">热门书籍</div>
<div class="sidebar_info">
<a href="/so.php?sokey=%E8%B5%84%E6%B2%BB%E9%80%9A%E9%89%B4&kt=1">资治通鉴</a> <a href="/24shi/"><strong>二十四史</strong></a>&nbsp; <a href="/a2694/">野史</a>&nbsp; <a href="/SiKuQuanShu/"><strong>四库全书</strong></a>&nbsp;<a href="http://www.guoxuedashi.com/SiKuQuanShu/fanti/">繁体</a>
<br><a href="/so.php?sokey=%E7%BA%A2%E6%A5%BC%E6%A2%A6&kt=1">红楼梦</a> <a href="/a/1858x/">三国演义</a> <a href="/a/1038k/">水浒传</a> <a href="/a/1046t/">西游记</a> <a href="/a/1914o/">封神演义</a>
<br>
<a href="http://www.guoxuedashi.com/so.php?sokeygx=%E4%B8%87%E6%9C%89%E6%96%87%E5%BA%93&submit=&kt=1">万有文库</a> <a href="/a/780t/">古文观止</a> <a href="/a/1024l/">文心雕龙</a> <a href="/a/1704n/">全唐诗</a> <a href="/a/1705h/">全宋词</a>
<br><a href="http://www.guoxuedashi.com/so.php?sokeygx=%E7%99%BE%E8%A1%B2%E6%9C%AC%E4%BA%8C%E5%8D%81%E5%9B%9B%E5%8F%B2&submit=&kt=1"><strong>百衲本二十四史</strong></a>  <a href="http://www.guoxuedashi.com/so.php?sokeygx=%E5%8F%A4%E4%BB%8A%E5%9B%BE%E4%B9%A6%E9%9B%86%E6%88%90&submit=&kt=1"><strong>古今图书集成</strong></a>
<br>

<a href="http://www.guoxuedashi.com/so.php?sokeygx=%E4%B8%9B%E4%B9%A6%E9%9B%86%E6%88%90&submit=&kt=1">丛书集成</a> 
<a href="http://www.guoxuedashi.com/so.php?sokeygx=%E5%9B%9B%E9%83%A8%E4%B8%9B%E5%88%8A&submit=&kt=1"><strong>四部丛刊</strong></a>  
<a href="http://www.guoxuedashi.com/so.php?sokeygx=%E8%AF%B4%E6%96%87%E8%A7%A3%E5%AD%97&submit=&kt=1">說文解字</a> <a href="http://www.guoxuedashi.com/so.php?sokeygx=%E5%85%A8%E4%B8%8A%E5%8F%A4&submit=&kt=1">三国六朝文</a>
<br><a href="http://www.guoxuedashi.com/so.php?sokeytm=%E6%97%A5%E6%9C%AC%E5%86%85%E9%98%81%E6%96%87%E5%BA%93&submit=&kt=1"><strong>日本内阁文库</strong></a> <a href="http://www.guoxuedashi.com/so.php?sokeytm=%E5%9B%BD%E5%9B%BE%E6%96%B9%E5%BF%97%E5%90%88%E9%9B%86&ka=100&submit=">国图方志合集</a> <a href="http://www.guoxuedashi.com/so.php?sokeytm=%E5%90%84%E5%9C%B0%E6%96%B9%E5%BF%97&submit=&kt=1"><strong>各地方志</strong></a>

</div>
</div>


<div class="sidebar2">
<center>

</center>
</div>
<div class="sidebar greenbar">
<div class="sidebar_title green">四库全书</div>
<div class="sidebar_info">

《四库全书》是中国古代最大的丛书,编撰于乾隆年间,由纪昀等360多位高官、学者编撰,3800多人抄写,费时十三年编成。丛书分经、史、子、集四部,故名四库。共有3500多种书,7.9万卷,3.6万册,约8亿字,基本上囊括了古代所有图书,故称“全书”。<a href="http://www.guoxuedashi.com/SiKuQuanShu/">详细>>
</a>

</div> 
</div>

</div>  <!--end r-->

</div>
<!-- 内容区END --> 

<!-- 页脚开始 -->
<div class="shh">

</div>

<div class="w1180" style="margin-top:8px;">
<center><script src="http://www.guoxuedashi.com/img/plus.php?id=3"></script></center>
</div>
<div class="w1180 foot">
<a href="/b/thanks.php">特别致谢</a> | <a href="javascript:window.external.AddFavorite(document.location.href,document.title);">收藏本站</a> | <a href="#">欢迎投稿</a> | <a href="http://www.guoxuedashi.com/forum/">意见建议</a> | <a href="http://www.guoxuemi.com/">国学迷</a> | <a href="http://www.shuowen.net/">说文网</a><script language="javascript" type="text/javascript" src="https://js.users.51.la/17753172.js"></script><br />
  Copyright &copy; 国学大师 古典图书集成 All Rights Reserved.<br>
  
  <span style="font-size:14px">免责声明:本站非营利性站点,以方便网友为主,仅供学习研究。<br>内容由热心网友提供和网上收集,不保留版权。若侵犯了您的权益,来信即刪。scp168@qq.com</span>
  <br />
ICP证:<a href="http://www.beian.miit.gov.cn/" target="_blank">鲁ICP备19060063号</a></div>
<!-- 页脚END --> 
<script src="http://www.guoxuedashi.com/img/plus.php?id=22"></script>
<script src="http://www.guoxuedashi.com/img/tongji.js"></script>

</body>
</html>
