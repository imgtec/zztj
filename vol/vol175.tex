










 


 
 


 

  
  
  
  
  





  
  
  
  
  
 
  

  

  
  
  



  

 
 

  
   




  

  
  


    資治通鑑卷一百七十五 宋 司馬光 撰

  胡三省 音注

  陳紀九【起重光赤奮若盡昭陽單閼凡三年】

  高宗宣皇帝下之下

  太建十三年春正月壬午以晉安王伯恭為尚書左僕射吏部尚書袁憲為右僕射憲樞之弟也 周改元大定 二月甲寅隋王始受相國百揆九錫【自初命至是五十一日乃受】建臺置官【置百官也】丙辰詔進王妃獨孤氏為王后世子勇為太子開府儀同大將軍庾季才勸隋王宜以今月甲子應天受命【庾季才持正於宇文護擅權之時而勸進於楊氏革命之日巫史之學自信其術耳非胸中眞有所見也】太傅李穆開府儀同大將軍盧賁亦勸之於是周主下詔遜居别宮甲子命兼太傅公椿奉册大宗伯趙煚奉皇帝璽紱禪位于隋【册册書也周制皇帝八璽有神璽有傳國璽皆寶而不用神璽明受之於天傳國璽明受之於運皇帝負扆則置神璽於筵前之右置傳國璽於筵前之左又有六璽其一皇帝行璽封命諸侯及三公用之其二皇帝之璽與諸侯及三公書用之其三皇帝信璽發諸夏之兵用之其四天子行璽封命蕃國之君用之其五天子之璽與蕃國之君書用之其六天子信璽徵蕃國之兵用之六璽皆白玉為之方一寸五分高寸螭虎鈕梁敬帝太平元年周閔帝受魏禪五主二十四年而亡隋主本襲封隨公故國號曰隋以周齊不遑寧處故去辶作隋以辶訓走故也辶音綽煚俱永翻璽斯氏翻紱音弗】隋主冠遠遊冠【遠遊冠制似通天冠而前無山述有展筩横于冠前皇太子及王者後諸王服之主冠古玩翻】受册璽改服紗㡌黄袍【紗㡌白紗㡌也名高頂帽皇帝服絳紗袍志曰開皇初高祖常服烏紗㡌紀云秋七月上始服黄百寮畢賀蓋以黄為常服】入御臨光殿服衮冕如元會之儀【元會正旦大朝會也文物充庭羣官各入就位再拜上公一人詣西階解劒升賀降階帶劒復位而拜羣官在位者又再拜搢笏三稱萬歲】大赦改元開皇命有司奏册祀于南郊【告天以受命】遣少冢宰元孝矩代太子勇鎭洛陽【少冢宰當作小冢宰】孝矩名矩以字行天賜之孫也【按隋書元孝矩傳祖修義不言以字行汝隂王天賜當魏太和之世距此時百餘年當考】女為太子妃少内史崔仲方勸隋主除周六官【周定六官事始一百六十六卷梁敬帝紹泰元年少内史當作小内史】依漢魏之舊從之置三師三公及尚書門下内史祕書内侍五省【隋志三師不主事不置府僚蓋與天子坐而論道者也三公參議國之大事依後齊置府僚無其人則闕祭祀則太尉亞獻司徒奉俎司空行掃除其位多曠皆攝行事尋省府及僚佐置公則坐於尚書都省朝之衆務總歸於臺閣尚書省事無不總置令左右僕射各一人總吏部禮部兵部都官度支工部六曹事屬官左右丞各一人都事八人分司管轄六曹尚書分統三十六侍郎各司曹務直宿禁省如漢之制門下省置納言給事黄門侍郎散騎常侍侍郎通直員外諫議大夫等官内史省置監令侍郎舍人等官祕書省置監丞郎等官領著作太史二曹内史省即中書省避武元諱改曰内史門下内史二省主出納朝直代言猶有職事祕書省較優閑内侍省則皆宦官也】御史都水二臺【御史臺置大夫治書侍御史侍御史殿内侍御史監察御史等官都水臺置使者及丞參軍河堤謁者又領掌船局及諸津都水尉津尉丞長等官】太常等十一寺【太常光禄衛尉宗正太僕大理鴻臚司農太府九寺並置卿少卿丞主簿録事等員國子寺置祭酒屬官有主簿録事國子太學四門書等學各置博士助教將作寺置大匠丞主簿録事統左右校署令】左右衛等十二府【左右衛左右武衛左右武候左右領左右左右監門左右領軍各置大將軍將軍長史司馬録事功倉兵騎等曹參軍法曹鎧曹行參軍行參軍等員】以分司統職又置上柱國至都督十一等勲官以酬勤勞【隋採後周之制置上柱國柱國上大將軍大將軍上開府儀同三司開府儀同三司上儀同三司儀同三司大都督帥都督都督總十一等】特進至朝散大夫七等散官【特進左右光禄大夫金紫光禄大夫銀青光禄大夫朝議大夫朝散大夫總七等朝直遥翻散悉亶翻】以加文武官之有德聲者改侍中為納言【以考諱忠故改侍中為納言】以相國司馬高熲為尚書左僕射兼納言【熲古迥翻射音夜】相國司録京兆虞慶則為内史監兼吏部尚書相國内郎李德林為内史令乙丑追尊皇考為武元皇帝【相息亮翻相國内郎相國府從事中郎避諱改為内郎皇考周隨國桓公楊忠】廟號太祖皇妣呂氏為元明皇后丙寅修廟社【時自高祖以下置四親廟同殿異室而已無受命之祧社稷並列于含光門内之右】立王后獨孤為皇后【獨孤之下逸氏字】王太子勇為皇太子丁卯以太尉趙煚為尚書右僕射己巳封周靜帝為介公【煚居永翻周主雖禪死乃有謚通鑑先以謚書之介古國名】周氏諸王皆降爵為公初劉鄭矯詔以隋主輔政【劉鄭劉昉鄭譯也矯詔事見上卷上年】楊后雖不預謀然以嗣子幼冲恐權在它族聞之甚喜後知其父有異圖意頗不平形於言色及禪位憤惋逾甚【嗣祥吏翻惋烏慣翻】隋主内甚愧之改封樂平公主【樂平郡公主五代志太原郡樂平縣舊置樂平郡樂音洛】久之欲奪其志公主誓不許乃止隋主與周載下大夫北平榮建緒有舊【載下逸師字後周置載師之官屬地官有中大夫有下大夫北平郡治盧龍榮姓出周榮公莊子有榮啟期】隋主將受禪建緒為息州刺史【五代志汝南郡新息縣後魏置東豫州梁改西豫州又改淮州東魏復曰東豫州後周改曰息州】將之官隋主謂曰且躊躇【躊直由翻躇陳如翻住足也】當共取富貴建緒正色曰明公此旨非僕所聞及即位來朝帝謂之曰卿亦悔不【朝直遥翻不讀曰否】建緒稽首曰臣位非徐廣情類楊彪【稽音啟徐廣事見一百一十九卷宋高祖永初元年楊彪事見六十九卷魏文帝黄初二年】帝笑曰朕雖不曉書語亦知卿此言不遜上柱國竇毅之女聞隋受禪自投堂下撫膺太息曰恨我不為男子救舅氏之患【撫與拊同拍也膺胸也太息憤而舒氣長也】毅及襄陽公主掩其口曰汝勿妄言滅吾族毅由是奇之及長以適唐公李淵淵昞之子也【昞周柱國李虎之子李淵始見于此長知兩翻昞音丙】虞慶則勸隋主盡滅宇文氏高熲楊惠亦依違從之【依違者不敢言其不可而心不以為可】李德林固爭以為不可隋主作色曰君書生不足與議此於是周太祖孫譙公乾惲冀公絢【惲於粉翻絢許縣翻】閔帝子紀公湜【湜常職翻】明帝子酆公貞宋公實高祖子漢公贊秦公贄曹公允道公充蔡公兑荆公元宣帝子萊公衍郢公術皆死【通鑑書宣帝子衍始終備但目録書大成元年立太子衍亦自背馳】德林由此品位不進 乙亥上耕藉田【藉在亦翻】 隋主封其弟邵公慧為滕王安公爽為衛王【邵安皆以州為封國】子鴈門公廣為晉王俊為秦王秀為越王諒為漢王 隋主賜李穆詔曰公既舊德且又父黨【李穆與隋主之父忠比肩事周皆為功臣】敬惠來旨義無有違【謂穆勸之受命也】即以今月十三日恭膺天命【孔安國曰膺當也】俄而穆入朝【自并州入朝朝直遥翻】帝以穆為太師贊拜不名子孫雖在襁褓【襁居兩翻負兒衣褓博抱翻抱兒衣】悉拜儀同一門執象笏者百餘人【隋志曰案禮笏諸侯以象凡有指畫於君前用笏受命書於笏笏畢用也五經要義曰所以紀事防忽忘禮圖云度二尺有六寸中博二寸其殺六分去一晉宋以來謂之手板此乃不經今還謂之笏以法古名自西魏以降五品以上通用象牙六品以下兼用竹木笏呼骨翻】貴盛無比又以上柱國竇熾為太傅幽州摠管于翼為太尉李穆上表乞骸骨【人臣致身以事君身非已有故求閑者自言乞骸骨上時掌翻】詔曰呂尚以期頤佐周【記百年曰期頤呂尚遇文王年八十矣佐文王以及武王則是期頤之年也】張蒼以華皓相漢【華皓謂白首也張蒼免相後口中無齒食乳年百餘歲乃卒】高才命世不拘常禮仍以穆年耆敕蠲朝集【蠲免也朝集猶言朝會也朝直遥翻】有大事就第詢訪【用古人欲有謀焉則就之之意隋主姑以是恩李穆耳非欲與之大有為也】美陽公蘇威【美陽古縣名漢晉屬扶風五代志不見蓋已省廢姑以古縣名封爵之耳】綽之子也【蘇綽佐宇文泰以興周】少有令名周晉公護強以女妻之【少詩照翻強其兩翻妻七細翻】威見護專權恐禍及已屏居山寺以諷讀為娯【屏必郢翻】周高祖聞其賢除車騎大將軍儀同三司【騎奇寄翻】又除稍伯下大夫【稍所教翻】皆辭疾不拜宣帝就除開府儀同大將軍隋主為丞相高熲薦之隋主召見與語大悦居月餘聞將受禪遁歸田里【觀蘇威之初其立身何可議哉至于末節展轉于宇文化及李密王世充之朝何其可鄙也君子是以知令終之難】熲請追之【追者尋其後而召之】隋主曰此不欲預吾事耳置之及受禪徵拜太子少保追封其父為邳公【邳亦以州名為公國少始照翻】以威襲爵丁丑隋以晉王廣為并州摠管三月戊子以上開府

  儀同三司賀若弼為吳州摠管鎭廣陵【考異曰隋書帝紀云楚州今從弼傳】和州刺史河南韓擒虎為廬州摠管鎭廬江【廣陵為吳州仍周舊也歷陽為和州仍齊舊也隋書韓擒虎河東垣人河南當作河東五代志廬江郡梁置南豫州又改合州開皇初改廬州蓋梁之南豫合州皆治合肥合州因合肥而名也廬江在合肥東五十里既徙治廬江故以廬名州若人者翻】隋主有并吞江南之志問將帥於高熲【將即亮翻帥所類翻】熲薦弼與擒虎故置於南邊使潜為經略戊戌以太子少保蘇威兼納言度支尚書【度支尚書統度支戶部金部倉部度徒洛翻】初蘇綽在西魏以國用不足制征税法頗重【後周太祖作相置載師掌任土之法辨夫家田里之數會六畜車乘之稽審賦役斂弛之節制畿疆修廣之域頒施惠之要審牧產之政司均掌田里之政令凡人口十已上宅五畝口九已上宅四畝口五已下宅三畝有室者田百四十畝丁者田百畝司賦掌功賦之政令凡人自十八以至六十有四與輕癃者皆賦之其賦之法有室者歲不過絹一疋綿八兩粟五斛丁者半之其非桑土有室者布一疋麻十斤丁者又半之豐年則全賦中年半之下年一之皆以時徵焉若艱荒凶札則不徵其賦又有市門之税自今觀之亦不為重矣而蘇綽猶望後之人弛之可謂有志于民矣】既而歎曰今所為者譬如張弓非平世法也後之君子誰能弛之威聞其言每以為己任至是奏減賦役務從輕簡隋主悉從之【蘇威為度支尚書居可言可行之地】漸見親重與高熲參掌朝政【朝直遥翻】帝嘗怒一人將殺之威入閤進諫帝不納將自出斬之威當帝前不去帝避之而出威又遮止帝拂衣而入良久乃召威謝曰公能若是吾無憂矣賜馬二匹錢十餘萬尋復兼大理卿京兆尹御史大夫本官悉如故治書侍御史安定梁毗以威兼領五職【漢宣帝幸宣室齋居決事令侍御史二人治書侍側魏晉因别置治書侍御史安定郡涇州五職納言度支尚書大理卿京兆尹御史大夫也治平聲】安繁戀劇無舉賢自代之心抗表劾威【劾戶慨翻又戶得翻】帝曰蘇威朝夕孜孜【孜孜不怠也】志存遠大何遽迫之因謂朝臣曰蘇威不値我無以措其言我不得蘇威何以行其道楊素才辯無雙至於斟酌古今助我宣化非威之匹也【匹偶也】威若逢亂世南山四皓豈易屈哉【四皓東園公綺里季夏黄公甪里先生遭秦之亂隱于商山鬚眉皓白故曰四皓商山在長安南故曰南山隋主以蘇威隱遯於周世故云然易以豉翻】威嘗言於帝曰臣先人每戒臣云【先人謂威父綽】唯讀孝經一卷足以立身治國何用多為帝深然之【治直之翻下同】高熲深避權勢上表遜位【上時掌翻】讓於蘇威帝欲成其美【成其讓賢之美】聽解僕射數日帝曰蘇威高蹈前朝【前朝謂周朝高蹈謂其隱遯不仕蹈踐也履也高蹈言踐履之高朝直遥翻】熲能推舉吾聞進賢受上賞【漢武帝詔曰進賢受上賞蔽賢蒙顯戮古之道也】寧可使之去官命熲復位熲威同心協贊政刑大小帝無不與之謀議然後行之故革命數年天下稱平太子左庶子盧賁以熲威執政心甚不平時柱國劉昉亦被踈忌賁因諷昉及上柱國元諧李詢華州刺史張賓等謀黜熲威【五代志京兆郡鄭縣後魏置東雍州并華山郡西魏改曰華州昉甫兩翻被皮義翻華戶化翻】五人相與輔政又以晉王廣有寵於帝私謂太子曰賁欲數謁殿下【數所角翻】恐為上所譴願察區區之心謀洩帝窮治其事【治直之翻】昉等委辠於賓賁公卿奏二人當死帝以故舊不忍誅並除名為民【二人皆翼戴隋主于潜躍者也張賓道上也隋主作輔賓自言洞曉星歷盛言有代謝之徵且言上儀表非大臣之相由是大被知遇常在幕府】庚子隋詔前代品爵皆依舊不降【此普謂中外官也】 丁未梁主遣其弟太宰巖入賀于隋【賀受命也】 夏四月辛巳隋大赦戊戌悉放太常散樂為民仍禁雜戲【後齊之季有散樂周天元即位悉徵詣長安隸太常隋今放之】 散騎常侍韋鼎兼通直散騎常侍王瑳聘于周【散悉亶翻騎奇寄翻瑳七何翻又七可翻】辛丑至長安隋已受禪隋主致之介國【說文致送詣也周主時封介公】 隋主召汾州刺史韋冲為兼散騎常侍【五代志文城郡東魏置南汾州後周改為汾州散悉亶翻】時發稽胡築長城【按隋紀時修築長城二旬而罷】汾州胡千餘人在塗亡叛帝召冲問計對曰夷狄之性易為反覆【易以䜴翻】皆由牧宰不稱之所致【稱尺證翻】臣請以理綏靜可不勞兵而定帝然之命冲綏懷叛者月餘皆至並赴長城之役冲夐之子也【夐見一百六十七卷周高祖永定三年】 五月戊午隋封邘公雄為廣平王【按隋書此即䢴公惠也改名雄開皇中改封清漳王仁夀初改封安德王大業中從征吐谷渾還進封觀王薨謚曰德後所謂觀德王雄者是也邘當作䢴音寒】永康公弘為河間王【永康縣公也五代志清化郡永穆縣梁置曰永康】雄高祖之族子也 隋主潜害周靜帝而為之舉哀【為于偽翻】葬于恭陵以其族人洛為嗣六月癸未隋詔郊廟冕服必依禮經【隋制冕服採用東齊之法乘輿衮冕垂白珠十有二旒以組為纓色如其綬黈纊充耳玉笄玄衣纁裳衣山龍華蟲火宗彝五章裳藻粉米黼黻四章衣重宗彝裳重黼黻為十二等衣褾領織成升龍白紗内單黼領青褾襈裾革帶玉鈎䚢大帶素帶朱裏紕其外上以朱下以緑韍隨裳色龍火山三章鹿盧玉具劒火珠鏢首白玉雙佩玄組雙大綬六采玄黄青白縹緑純玄質長二丈四尺五百首廣一尺小雙綬長二尺六寸色同大綬而首半之間施三玉環朱韈赤舄舄加金飾凡綬先合單紡為一絲絲四為一扶扶五為一首首五成一文褾皮小翻襈雛免翻䚢丑例翻又敕列翻紕音卑緣也鏢紕招翻說刀削末銅也縹匹沼翻紡甫罔翻】其朝會之服旗幟犧牲皆尚赤【隋自以為得火德故尚赤色朝直遥翻下同幟昌志翻】戎服以黄常服通用雜色秋七月乙卯隋主始服黄百僚畢賀于是百官常服同於庶人皆著黄袍【著則咯翻】隋主朝服亦如之唯以十三鐶帶為異八月壬午隋廢東京官【周徙相州六府於東京事見一百七十三卷大建十一年】吐谷渾寇凉州【凉州武威郡吐從暾入聲谷音浴】隋主遣行軍元帥

  樂安公元諧等步騎數萬擊之諧擊破吐谷渾於豐利山【豐利山在青海東帥所類翻騎奇寄翻】又敗其太子可博汗於青海【青海在吐谷渾國都伏俟城之東十五里周迴千餘里中有小山唐時謂之龍駒島敗補邁翻可從刋入聲汗音寒】俘斬萬計吐谷渾震駭其王侯三十人各帥所部來降吐谷渾可汗夸呂帥親兵遠遁【帥讀曰率降戶江翻】隋主以其高寧王移兹裒為河南王【裒薄侯翻】使統降衆以元諧為寧州刺史【五代志北地郡後魏置豳州西魏改為寧州】留行軍摠管賀婁子幹鎭凉州 九月庚午將軍周羅睺攻隋故墅拔之【故墅當作胡墅胡墅在大江北岸對石頭城睺音侯墅承與翻】蕭摩訶攻江北 隋奉車都尉于宣敏【漢武帝置三都尉奉車駙馬騎也】奉使巴蜀還【使疏吏翻還音旋又如字】奏稱蜀土沃饒人物殷阜周德之衰遂成戎首【謂王謙以益州起兵也】宜樹建藩屏封殖子孫【屏必郢翻】隋主善之辛未以越王秀為益州總管改封蜀王【為秀在蜀以奢僭得罪張本】宣敏謹之孫也【于謹周之功臣】 隋以上柱國長孫覽元景山並為行軍元帥【長知兩翻帥所類翻】發兵入寇命尚書左僕射高熲節度諸軍初周齊所鑄錢凡四等及民間私錢名品甚衆【五代志齊】

  【文宣受禪改鑄常平五銖重如其文其錢甚貴且制造甚精至乾明皇建之間往往私鑄鄴中用錢有赤熟青熟細眉赤生之異河南所用有青薄鉛錫之别青齊徐兖梁豫州輩類各殊武平已後私鑄轉甚或以生鐵和銅至于齊亡卒不能禁後周之初尚用魏錢及武帝保定元年乃更鑄布泉之錢以一當五與五銖並行時梁益之境又雜用古錢交易河西諸郡或用西域金銀之錢而官不禁建德三年更鑄五行大布錢以一當十與布泉並行五年以布泉漸賤遂廢之齊平已後山東猶雜用齊氏舊錢宣帝大象元年又鑄永通萬國錢以一當干與五行大布及五銖凡三品並用】輕重不等隋主患之更鑄五銖錢背面肉好皆有周郭【錢之文為面其漫為背錢體為肉錢孔為好外圓周之以規内方周之以矩曰周郭更工衡翻肉而牧翻好虛到翻】每一千重四斤二兩悉禁古錢及私錢置樣於關不如樣者没官銷毁之自是錢幣始壹民間便之 隋鄭譯以上柱國歸第賞賜豐厚譯自以被疎呼道士醮章祈福【道士有消災度厄之法依隂陽五行數術推人年命書之如章表之儀并具贄幣燒香陳讀云奏上天曹請為除厄謂之上章夜中於星辰之下陳設酒果䴵餌幣物歷祀天皇太乙五星列宿為書如上章之儀以奏之名為醮被皮義翻醮子肖翻】為婢所告以為巫蠱譯又與母别居為憲司所劾【憲司御史臺官劾戶槩翻又戶得翻】由是除名隋主下詔曰譯若留之於世在人為不道之臣戮之於朝入地為不孝之鬼有累幽顯無所置之【朝直遥翻累力瑞翻】宜賜以孝經令其熟讀仍遣與母共居 初周法比於齊律煩而不要隋主命高熲鄭譯及上柱國楊素率更令裴政等【太子率更令魏晉之制主宮殿門戶及掌罰事職如光禄勲衛尉隋制掌伎樂漏刻率如字更工衡翻】更加修定政練習典故達於從政乃采魏晉舊律下至齊梁沿革重輕【累世循襲者為沿中有變更者為革】取其折衷【衷竹仲翻】時同修者十餘人凡有疑滯皆取決於政于是去前世梟轘及鞭法【梟者斬首掛之木上轘者車裂于市梁制有制鞭法鞭常鞭凡三等之差制鞭生革廉成法鞭生革去廉常鞭熟靼不去廉皆作鶴頭紐長一尺一寸梢長二尺七寸廣三寸靶長二尺五寸去羌呂翻梟古堯翻轅戶關翻又戶慣翻】自非謀叛以上無收族之罪始制死刑二絞斬流刑三自二千里至三千里【按隋志流刑三有千里千五百里二千里應配者一千里居作二年一千五百里居作二年半二千里居作三年應住居作者三流俱役三年近流加杖一百一等加三十此云自二千里至三千里不同】徒刑五自一年至三年【徒刑有一年有一年半有二年有二年半有三年】杖刑五自六十至百笞刑五自十至五十又制議請減贖官當之科以優士大夫【議即周禮八議之法請者凡在八議之科則請之減者官品第七已上犯罪皆例減一等其品第九已上犯者聽贖應贖者皆以銅代絹贖銅一斤為負負卜為殿笞十者銅一斤加至杖百則十斤徒一年贖銅二十斤每等則加銅十斤三年則六十斤矣流一千里贖銅八十斤每等則加銅十斤二千里則百斤矣二死皆贖銅百二十斤犯私罪以官當徒者五品以上一官當徒二年九品以上一官當徒一年當流者三流同比徒三年若犯公罪者徒各加一年當流者各加一等其累徒過九年者流三千里孔穎達曰古之贖罪用銅漢始改用黄金但少其斤兩令與銅相敵後魏以金難得令金一兩收絹十匹隋復依古贖銅】除前世訊囚酷法考掠不得過二百【時有司用前世訊囚之法用大棒束杖車輻鞵底壓踝杖桄之屬考擊也掠音亮笞也】枷杖大小咸有程式民有枉屈縣不為理者聽以次經郡及州若仍不為理聽詣闕伸訴【枷居牙翻為于為翻】冬十月戊子始行新律詔曰夫絞以致斃斬則殊形【夫音扶】除惡之體於斯已極梟首轘身義無所取不益懲肅之理徒表安忍之懷【忍殘忍也安忍安于為殘忍之事】鞭之為用殘剥膚體徹骨侵肌【徹敕列翻】酷均臠切雖云往古之式【舜典曰鞭作官刑故云往古之式】事乖仁者之刑梟轘及鞭並令去之貴帶礪之書不當徒罸【令力丁翻使也去羌呂翻漢高帝分封功臣與之剖符作誓曰使黄河如帶泰山若礪國以永存爰及苗裔】廣軒冕之䕃旁及諸親【服冕乘軒貴仕也】流役六年改為五載【載作亥翻】刑徒五歲變從三祀【祀亦年也】其餘以輕代重化死為生條目甚多備於簡策雜格嚴科並宜除削自是法制遂定後世多遵用之【宋朝所行之刑統舊所傳者也】隋主嘗怒一郎於殿前笞之諫議大夫劉行本進曰此人素清其過又小願少寛之【笞丑之翻少詩沼翻】帝不顧行本於是正當帝前曰陛下不以臣不肖置臣左右臣言若是陛下安得不聽若非當致之於理【送詣大理寺治其罪】因置笏於地而退帝斂容謝之遂原所笞者行本璠之兄子也【劉璠自梁入西魏見一百六十四卷梁元年承聖元年笞丑之翻璠音煩又扶元翻】獨孤皇后家世貴盛【后父獨孤信仕西魏以及周列於元功后姊為周明帝后女為周宣帝后】而能謙恭雅好讀書【好呼到翻】言事多與隋主意合帝甚寵憚之宮中稱為二聖帝每臨朝【朝直遥翻下同】后輒與帝方輦而進【方輦并兩輦也】至閤乃止使宦官伺帝政有所失隨即匡諫【伺相吏翻】候帝朝同反燕寢【朝直遥翻燕寢燕居之寢】有司奏稱周禮百官之妻命於王后請依古制后曰婦人與政或從此為漸不可開其源也【與讀曰預】大都督崔長仁后之中外兄弟也犯法當斬帝以后故欲免其辠后曰國家之事焉可顧私【焉於䖍翻】長仁竟坐死后性儉約帝嘗合止利藥【泄瀉不禁者曰利合音閤】須胡粉一兩宮内不用求之竟不得又欲賜柱國劉嵩妻織成衣領宮内亦無之然帝懲周氏之失不以權任假借外戚后兄弟不過將軍刺史帝外家呂氏濟南人【五代志齊郡歷城縣置濟南郡濟子禮翻】素微賤齊亡以來帝求訪不知所在【齊之未亡濟南之地屬齊不可得而求訪故齊亡始訪之】及即位始求得舅子呂永吉追贈外祖雙周為太尉封齊郡公以永吉襲爵永吉從父道貴性尤頑騃【從才用翻騃五駭翻癡也】言詞鄙陋帝厚加供給而不許接對朝士【朝直遥翻】拜上儀同三司出為濟南太守【守式又翻】後郡廢終于家 壬辰隋主如岐州【隋志扶風郡舊置岐州】岐州刺史安定梁彦光有惠政隋主下詔褒美賜束帛及御傘【傘與繖同先盰翻又蘇旱翻蓋也】以厲天下之吏久之徙相州刺史【相悉亮翻下同】岐俗質厚彦光以靜鎭之奏課連為天下最【奏課奏計帳及輸籍也】及居相部如岐州法鄴自齊亡衣冠士人多遷入關唯工商樂戶移實州郭【城外曰郭釋名郭廓也廓落在城外也】風俗險詖好興謡訟【詖彼義翻好呼到翻】目彦光為著㡌?【?飴也?軟而甘言彦光為人軟美如團?特著㡌耳孔穎達曰凡飴謂之?關東之通語也方言曰?謂之張皇或云滑餹著則略翻?徐盈翻】帝聞之免彦光官歲餘拜趙州刺史【五代志趙郡大陸縣舊曰廣阿置殷州及南鉅鹿郡後改南趙郡改州為趙州】彦光自請復為相州帝許之豪猾聞彦光再來皆嗤之【嗤丑之翻笑也】彦光至發擿姦伏【擿他狄翻發也動也】有若神明豪猾潛竄闔境大治【治直吏翻】於是招致名儒每鄉立學親臨策試褒勤黜怠及舉秀才祖道於郊以財物資之於是風化大變吏民感悦無復訟者【史因岐州之政終言彦光歷刺他州事復扶又翻】時又有相州刺史陳留樊叔畧有異政帝以璽書褒美班示天下徵拜司農【按樊叔略傳徵拜司農卿璽斯氏翻】新豐令房恭懿【新豐縣自漢以來屬京兆】政為三輔之㝡帝賜以粟帛雍州諸縣令朝謁【雍於用翻朝直遥翻下同】帝見恭懿必呼至榻前咨以治民之術累遷德州司馬【五代志平原郡開皇九年置德州治直之翻】帝謂諸州朝集使曰【隋志每元會諸州悉遣使赴京師朝集謂之朝集使使疏吏翻】房恭懿志存體國愛養我民此乃上天宗廟之所祐朕若置而不賞上天宗廟必當責我卿等宜師範之因擢為海州刺史【海州東海郡】由是州縣吏多稱職百姓富庶【樊叔略房恭懿之被褒擢非必皆是年事通鑑因梁彦光事悉書於此以見開皇之治以賞良吏而成稱尺證翻】 十一月丁卯隋遣兼散騎侍郎鄭撝來聘【散悉亶翻騎奇寄翻撝許為翻】 十二月庚子隋主還長安復鄭譯官爵廣州刺史馬靖【廣州治番禺】得嶺表人心兵甲精練數有戰功【數所角翻】朝廷疑之遣吏部侍郎蕭引觀靖舉措諷令送質【朝直遥翻令力丁翻質音致】外託收督賧物【賧吐濫翻蠻蜑所輸貨物曰賧一曰夷人以財贖罪曰賧】引至番禺【番禺音潘愚】靖即遣子弟入質 是歲隋主詔境内之民任聽出家仍令計口出錢營造經像於是時俗隨風而靡民間佛書多於六經數十百倍 突厥佗鉢可汗病且卒【厥九勿翻佗徒何翻可從刋入聲汗音寒卒子恤翻】謂其子菴邏曰吾兄不立其子委位於我【事見一百七十一卷太建四年邏即佐翻突厥傳作菴羅】我死汝曹當避大邏便【大邏便者木杆之子杜佑曰突厥以勇健者為莫賀弗肥麄者為大羅便大羅便酒器也似角而麄短體貌似之故以為號此官特貴唯其子弟為之】及卒國人將立大邏便以其母賤衆不服菴邏實貴【墮書作菴羅母貴當從之】突厥素重之攝圖最後至謂國人曰若立菴邏者我當帥兄弟事之【帥讀曰率】若立大邏便我必守境利刃長矛以相待攝圖長且雄勇國人莫敢拒【攝圖為小可汗統東面部落又逸可汗之子故長長知兩翻】竟立菴邏為嗣大邏便不得立心不服菴邏每遣人詈辱之【詈力智翻】菴邏不能制因以國讓攝圖國中相與議曰四可汗子【四可汗謂逸可汗及木杆可汗褥但可汗佗鉢可汗】攝圖㝡賢共迎之 【考異曰隋突厥傳云木杆在位二十年卒佗鉢在位十年卒按周傳魏廢帝二年三月科羅獻馬木杆猶未立建德二年佗鉢獻馬然則木杆以承聖二年立太建四年卒佗鉢以其年立十三年卒也】號沙鉢略可汗居都斤山菴邏降居獨洛水稱第二可汗【都斤山獨洛水皆突厥中地名第二可汗言其位次沙鉢略也】大邏便乃謂沙鉢略曰我與爾俱可汗子各承父後爾今極尊我獨無位何也沙鉢略患之以為阿波可汗還領所部又沙鉢略從父玷厥居西面號達頭可汗【從才用翻玷丁念翻厥九勿翻】諸可汗各統部衆分居四面沙鉢略勇而得衆北方皆畏附之隋主既立待突厥禮薄突厥大怨千金公主傷其宗祀覆滅日夜言於沙鉢略請為周室復讐【遣千金公主嫁突厥見上卷十二年為于偽翻】沙鉢略謂其臣曰我周之親也今隋主自立而不能制復何面目見可賀敦乎【復扶又翻突厥之君長稱可汗其妻稱可賀敦】乃與故齊營州刺史高寶寧合兵為寇隋主患之敕緣邊修保障峻長城命上柱國武威隂夀鎭幽州京兆尹虞慶則鎭并州屯兵數萬以備之初奉車都尉長孫晟送千金公主入突厥【長知兩翻晟成正翻】突厥可汗愛其善射留之竟歲命諸子弟貴人與之親友冀得其射法沙鉢略弟處羅侯號突利設尤得衆心為沙鉢略所忌密託心腹隂與晟盟晟與之遊獵因察山川形埶部衆彊弱靡不知之及突厥入寇晟上書曰今諸夏雖安【上時掌翻夏戶雅翻】戎虜尚梗興師致討未是其時弃於度外又相侵擾【此二語明指出當時利病今人多上書言時事滕口說耳】故宜密運籌策有以攘之【下方是晟獻策】玷厥之於攝圖兵彊而位下外名相屬内隙已彰鼓動其情必將自戰又處羅侯者攝圖之弟姦多勢弱【言其心多奸巧而形勢甚弱】曲取衆心國人愛之因為攝圖所忌其心殊不自安迹示彌縫實懷疑懼又阿波首鼠介在其間【漢書首鼠兩端】頗畏攝圖受其牽率【左傳牽率老夫】唯彊是與未有定心今宜遠交而近攻【史記范睢說秦王之言】離彊而合弱通使玷厥說合阿波【使疏吏翻說輸芮翻今人言說合二字說音如字合音閤】則攝圖迴兵自防右地【右地突厥西面地也】又引處羅遣連奚霫【奚庫莫奚霫 又一種霫音習】則攝圖分衆還備左方【左方突厥東面地也】首尾猜嫌腹心離阻十數年後乘舋討之【舋許覲翻】必可一舉而空其國矣帝省表大悦【省悉景翻】因召與語晟復口陳形勢【復扶又翻】手畫山川寫其虛實皆如指掌帝深嗟異皆納用之遣太僕元暉出伊吾道詣達頭賜以狼頭纛【太僕太僕卿也伊吾即漢伊吾盧之地突厥之先狼種也子孫為君長牙門建狼頭纛示不忘本也纛徒到翻】達頭使來引居沙鉢略使上【使疏吏翻】以晟為車騎將軍出黄龍道【黄龍即和龍今黄龍府即其地時為高寶寧所據騎奇寄翻】齎幣賜奚霫契丹【奚本曰庫莫奚東部胡之種也為慕容氏所破遺落竄匿松漠之間後稍強盛霫匈奴之别種也居潢水北契丹之先與奚同種而異類並為慕容氏所破俱竄松漠之間其後稍大居黄龍之北數百里契欺訖翻又音喫齎則兮翻】遣為鄉導【鄉讀曰嚮】得至處羅侯所深布心腹誘之内附【誘音酉】反間既行【間古莧翻】果相猜貳 始興王叔陵太子之次弟也與太子異母母曰彭貴人叔陵為江州刺史性苛刻狡險新安王伯固以善諧謔有寵於上及太子【謔虚約翻】叔陵疾之隂求其過失欲中之以法【中竹仲翻】叔陵入為揚州刺史事務多關涉省閤【省閤謂中書尚書二省】執事承意順旨即諷上進用之微致違忤必抵以大辠重者至殊死【身首異處為殊死忤五故翻】伯固憚之乃謟求其意叔陵好發古冢伯固好射雉【好呼到翻射而亦翻】常相從郊野大相欵狎因密圖不軌伯固為侍中每得密語必告叔陵 十四年春正月己酉上不豫太子與始興王叔陵長沙王叔堅並入侍疾叔陵隂有異志命典藥吏曰切藥刀甚鈍可礪之甲寅上殂倉猝之際叔陵命左右於外取劒左右弗悟取朝服木劒以進【朝服帶劒以為儀飾非求其適用故為木劒朝直遥翻】叔陵怒【怒其不能會已意】叔堅在側聞之疑有變伺其所為【伺相吏翻】乙卯小斂【斂力贍翻】太子哀哭俯伏叔陵抽剉藥刀斫太子中項【中竹仲翻】太子悶絶于地母柳皇后走來救之又斫后數下乳媪吳氏自後掣其肘【媪烏皓翻掣昌列翻】太子乃得起叔陵持太子衣太子自奮得免叔堅手搤叔陵奪去其刀仍牽就柱以其褶袖縛之【搤於革翻去羌呂翻褶音習布褶衣也今之寛袖山海經註魏毌丘儉破高句麗遣王頎窮追過汗沮干餘里彼人言海中有長臂人近於海中得布褶衣兩袖各長三丈有餘則知所謂褶衣有自來矣】時吳媪已扶太子避賊叔堅求太子所在欲受生殺之命叔陵多力奮袖得脱突走出雲龍門馳車還東府召左右斷青溪道【斷音短】赦東城囚以充戰士【東城即東府城】散金帛賞賜又遣人往新林追所部兵仍自被甲著白布㡌【被皮義翻著則略翻】登城西門招募百姓又召諸王將帥【將即亮翻帥所類翻】莫有至者唯新安王伯固單馬赴之助叔陵指揮叔陵兵可千人欲據城自守時衆軍並緣江防守臺内空虛叔堅白柳后使太子舍人河内司馬申以太子命召右衛將軍蕭摩訶入見受敕【見賢遍翻】帥馬步數百趣東府【帥讀曰率下同趣七喻翻下同】屯城西門叔陵惶恐遣記室韋諒送其鼓吹與摩訶【吹昌瑞翻】謂曰事捷必以公為台輔摩訶紿報之曰須王心膂節將自來方敢從命【紿徒亥翻將即亮翻】叔陵遣其所親戴温譚騏驎詣摩訶【譚徒含翻春秋齊滅譚子孫以國為氏驎離珍翻】摩訶執以送臺斬其首徇東城叔陵自知不濟入内沈其妃張氏及寵妾七人于井【沈持林翻】帥步騎數百自小航度【六朝都建業航秦淮而度者非一處當朱雀門者為大航當東府門者為小航騎奇寄翻下同航戶剛翻】欲趣新林乘舟奔隋行至白楊路為臺軍所邀伯固見兵至旋避入巷叔陵馳騎拔刃追之伯固復還叔陵部下多弃甲潰去摩訶馬容陳智深迎刺叔陵僵仆陳仲華就斬其首【軍行擇便於鞍馬軀幹壯偉者乘馬居前以壯軍容謂之馬容刺七亦翻】伯固為亂兵所殺自寅至已乃定叔陵諸子並賜死伯固諸子宥為庶人韋諒及前衡陽内史彭暠【五代志長沙郡衡山縣舊置衡陽郡陳為王國故置内史暠古老翻】諮議參軍兼記室鄭信典籖俞公喜並伏誅暠叔陵舅也信諒有寵於叔陵常參謀議諒粲之子也【韋粲梁臣死於侯景之難】丁巳太子即皇帝位大赦 辛酉隋置河北道行臺於并州以晉王廣為尚書令【并州治晉陽】置西南道行臺於益州以蜀王秀為尚書令隋主懲周氏孤弱而亡故使二子分蒞方面以二王年少【少詩照翻】盛選貞良有才望者為之僚佐以霛州刺史王韶為并省右僕射【五代志靈武郡後魏置靈州按靈州漢北地郡富平縣地赫連勃勃之果園後魏置靈州取靈武郡名之注又見前】鴻臚卿趙郡李雄為兵部尚書【五代志趙郡治平棘李雄趙郡高邑人臚陵如翻】左武衛將軍朔方李徹摠晉王府軍事【朔方郡夏州李徹朔方巖緑人】兵部尚書元巖為益州摠管府長史王韶李雄元巖俱有骨鯁名李徹前朝舊將【李徹事周征吐谷渾平齊定淮南皆有功朝直遥翻】故用之初李雄家世以學業自通雄獨習騎射【騎奇寄翻下同】其兄子旦讓之曰非士大夫之素業也雄曰自古聖賢文武不備而能成其功業者鮮矣【鮮息淺翻】雄雖不敏頗觀前志但不守章句耳既文且武兄何病焉及將如并省帝謂雄曰吾兒更事未多【更工衡翻】以卿兼文武才吾無北顧之憂矣二王欲為奢侈非法韶巖輒不奉教或自鎖或排閤切諫二王甚憚之每事諮而後行不敢違法度帝聞而賞之【隋文帝擇人以輔其子可謂用心矣而二子皆不克令終何也中人以下之性束縛之雖急一縱則不可復收也】又以秦王俊為河南道行臺尚書令洛州刺史領關東兵【洛州治洛陽】 癸亥以長沙王叔堅為驃騎將軍開府儀同三司揚州刺史【驃匹妙翻】蕭摩訶為車騎將軍南徐州刺史封綏遠公始興王家金帛累巨萬悉以賜之以司馬申為中書通事舍人乙丑尊皇后為皇太后時帝病創【創初良翻下同】卧承香殿不能聽政太后居柏梁殿百司衆務皆決於太后帝創愈乃歸政焉丁卯封皇弟叔重為始興王奉昭烈王祀【叔陵既誅以叔重奉昭烈王祀】 隋元景山出漢口【漢口漢水入江之口】遣上開府儀同三司鄧孝儒將卒四千攻甑山鎭將軍陸綸以舟師救之為孝儒所敗【敗補邁翻】溳口甑山沌陽守將皆弃城走【漢水記自漢口入二百里得溳口有村又三百里得溳城楚邑也漢安陸縣居之沌陽在沌水之北五代志沔陽郡漢陽縣有沌水溳音云沌柱兖翻將即亮翻】戊辰遣使請和於隋歸其胡墅【去年周羅睺拔胡墅使疏吏翻】 己巳立妃沈氏為皇后辛未立皇弟叔儼為尋陽王叔愼為岳陽王叔達為義陽王叔熊為巴山王叔虞為武昌王【宣帝諸子唯叔達後仕於唐貴顯】 隋高熲奏禮不伐喪【春秋公羊傳襄公十九年晉士匄帥師侵齊至穀聞齊侯卒乃還還者何善辭也何善爾大其不伐喪也】二月己丑隋主詔熲等班師 三月己巳以尚書左僕射晉安王伯恭為湘州刺史【湘州治長沙】永陽王伯智為尚書僕射 夏四月庚寅隋大將軍韓僧夀破突厥於雞頭山【雞頭山涇水所出在原州平高縣西】上柱國李充破突厥於河北山【此山蓋在北河之北】 丙申立皇子永康公胤為太子胤孫姬之子沈后養以為子五月己未高寶寧引突厥寇隋平州【五代志北平郡舊置平州治盧】

  【龍】突厥悉發五可汗控弦之士四十萬入長城【沙鉢略可汗第二可汗達頭可汗阿波可汗貪汗可汗凡五可汗】 壬戌隋任穆公于翼卒【任古國名謚法布德執義曰穆中情見貌曰穆任音壬卒子恤翻】 甲子隋更命傳國璽曰受命璽【更工衡翻璽斯氏翻】六月甲申隋遣使來弔【使疏吏翻】 乙酉隋上柱國李光敗突厥於馬邑【李光當作李充馬邑朔州治所大業初改馬邑縣為善陽縣】突厥又寇蘭州【五代志金城郡開皇初置蘭州總管府】凉州摠管賀婁子幹敗之於可洛峐【山無草木曰峐峐古哀翻】 隋主嫌長安城制度狹小又宮内多妖異【妖於驕翻】納言蘇威勸帝遷都帝以初受命難之夜與威及高熲共議明旦通直散騎庾季才奏曰臣仰觀乾象俯察圖記必有遷都之事且漢營此城將八百歲【漢高帝五年徙都長安歲在己亥是年歲在壬寅凡八百四歲惠帝元年城長安歲在丁未距是年七百九十六年】水皆鹹鹵不甚宜人【京都地大人衆加以歲久壅底墊隘穢惡聚而不泄則水多鹹鹵鹵郎古翻】願陛下協天人之心為遷徙之計帝愕然謂熲威曰是何神也太師李穆亦上表請遷都帝省表曰天道聰明已有徵應【上時掌翻省悉景翻徵證也】太師人望復抗此請無不可矣【復扶又翻】丙申詔高熲等創造新都於龍首山【三秦記龍首山長六十里首入渭水尾達樊川頭高二十丈尾漸下可六七丈色赤舊傳有黑龍從南山出飲渭水其行道因行成迹】以太子左庶子宇文愷有巧思領營新都副監【晉志太子庶子四人職比散騎常侍中書監令隋分置門下坊左庶子二人典書坊右庶子二人監者監領營新都事思相吏翻】愷忻之弟也 秋七月辛未大赦 九月丙午設無㝵大會於太極殿【㝵與礙同釋氏書也】捨身及乘輿御服【乘繩證翻】大赦 丙午以長沙王叔堅為司空將軍刺史如故【驃騎將軍揚州刺史】 冬十月癸酉隋太子勇屯兵咸陽以備突厥【咸陽在長安西北隔渭水耳屯兵於此以備突厥蓋其兵勢強盛欲窺長安此亦猶漢霸上棘門細柳之屯耳】 十二月丙子隋命新都曰大興城 乙酉隋遣沁源公虞慶則屯弘化以備突厥【沁源縣公五代志上黨郡有沁源縣後魏置弘化郡治合水開皇六年置慶州沁七鴆翻】行軍摠管達奚長儒將兵二千與突厥沙鉢略可汗遇於周槃【據慶則傳長儒别道邀賊為虜所圍慶則案營不救則周槃亦當在弘化縣界長儒當作長孺】沙鉢略有衆十餘萬軍中大懼長儒神色慷慨且戰且行為虜所衝散而復聚【復扶又翻】四面抗拒轉鬬三日晝夜凡十四戰五兵咸盡士卒以拳敺之【敺烏口翻】手皆骨見【見賢遍翻孟子曰盡信書不如無書五兵咸盡士卒奮拳擊虜以言死鬬則可若虜以全師四面蹙之安能免乎史但極筆叙長儒力戰之績耳觀者不以辭害意可也】殺傷萬計虜氣稍奪於是解去長儒身被五瘡通中者二【被皮義翻中竹仲翻】其戰士死者什八九詔以長儒為上柱國餘勲囘授一子時柱國馮昱屯乙弗泊【乙弗泊當在鄯州之西】蘭州摠管叱列長乂守臨洮【五代志後周武帝逐吐谷渾置洮陽郡尋立洮州大業初置臨洮郡洮士刀翻】上柱國李崇屯幽州皆為突厥所敗【敗蒲邁翻】於是突厥縱兵自木硤石門兩道入寇武威天水金城上郡弘化延安六畜咸盡【木峡石門兩關皆在弘化郡平高縣界此由虞慶則按營不戰達奚長儒孤軍摧衂故沙鉢略縱兵兩道而入然五可汗之兵東西齊舉西自乙弗泊東至幽州盡隋西北二邊無不被寇若武威至延安則達頭沙鉢略之兵耳天水上郡皆古郡天水則秦州上郡則敷州也延安郡後魏置東夏州西魏改為延州畜許又翻】沙鉢略更欲南入達頭不從引兵而去長孫晟又說沙鉢略之子染干【說輸芮翻】詐告沙鉢略曰鐵勒等反欲襲其牙【鐵勒之先本匈奴苗裔種類最多自西海之東依據山谷往往不絶至北海之南雖姓氏不同總謂之鐵勒】沙鉢略懼迴兵出塞 隋主既立待遇梁主恩禮彌厚是歲納梁主女為晉王妃【按隋書蕭后傳及蕭巋傳初皆云巋女詳考之則后本巋生江南風俗二月生子者不舉后以二月生故季父岌收而養之未幾岌夫妻俱死轉養於舅氏張軻家高祖為晉王選妃於梁徧占諸女皆不吉巋迎后於舅氏占之曰吉遂為王妃】又欲以其子瑒尚蘭陵公主【㻛雉杏翻又音暢】由是罷江陵摠管【西魏遷梁主詧於江陵置助防曰防主後遂置總管今罷之】梁主始得專制其國

  長城公上【諱叔寶字元秀小字黄奴宣帝嫡長子也】

  至德元年春正月庚子隋將入新都大赦 壬寅大赦改元 初上病創【創初良翻】不能視事政無大小皆決於長沙王叔堅權傾朝廷叔堅頗驕縱上由是忌之都官尚書山隂孔範【山隂漢古縣屬會稽郡】中書舍人施文慶皆惡叔堅而有寵於上【惡烏路翻下同】日夕求其短【日夕猶言朝夕也】搆之於上上乃即叔堅驃騎將軍本號用三司之儀出為江州刺史以祠部尚書江總為吏部尚書 癸卯立皇子深為始安王 二月己巳朔日有食之 癸酉遣兼散騎常侍賀徹等聘于隋【散悉亶翻騎奇寄翻】 突厥寇隋北邊【厥九勿翻】癸巳葬孝宣皇帝于顯寧陵廟號高宗 右衛將軍兼中書通事舍人司馬申既掌機密頗作威福多所譛毁能候人主顔色有忤已者必以微言譛之【忤五故翻】附已者因機進之是以朝廷内外皆從風而靡上欲用侍中吏部尚書毛喜為僕射申惡喜彊直言於上曰喜臣之妻兄高宗時稱陛下有酒德【周公戒成王曰無若殷王受之迷亂酗于酒德哉注云言紂心迷政亂以酗酒為德】請逐去宮臣【去羌呂翻】陛下寧忘之邪【邪音耶】上乃止上創愈置酒於殿以自慶引吏部尚書江總以下展樂賦詩【展舒而陳之也創初良翻】既醉而命毛喜于時山陵初畢喜見之不懌欲諫則上已醉喜升階陽為心疾仆于階下移出省中上醒謂江總曰我悔召毛喜彼實無疾但欲阻我歡宴非我所為耳【言喜以帝所為為非】乃與司馬申謀曰此人負氣吾欲乞鄱陽兄弟聽其報讐可乎【鄱陽兄弟世祖諸子也高宗之簒殺劉師知韓子高到仲舉父子以及始興王伯茂皆毛喜之謀後主怒喜欲以喜乞鄱陽兄弟聽其報讐於臣為不君於父為不子乞音氣與也】對曰彼終不為官用【陳之臣子率稱其君曰官】願如聖旨中書通事舍人北地傅縡爭之【傳稱縡少依蕭循蓋循自闕中歸縡與之俱南也縡作代翻】曰不然若許報讐欲置先皇何地上曰當乞一小郡勿令見人事耳乃以喜為永嘉内史【考異曰司馬申傳云右僕射沈君理卒朝議以毛喜代之按君理卒在太建五年非後主時又毛喜傳云時山陵初畢未及踰年按高祖殂過朞乃葬而云未及踰年恐誤】 三月丙辰隋遷于新都 【考異曰隋食貨志正月帝入新宮今從帝紀】初令民二十一成丁減役者每歲十二番為二十日役減調絹一匹為二丈【後周之制民年十八成丁今增三歲每歲十二番則三十日役今減為二十日役及調絹減半調徒弔翻】周末榷酒坊鹽池鹽井至是皆罷之【周末官制酒坊收利鹽池鹽井皆禁百姓採用池鹽則河東池鹽井鹽則蜀中處處有之榷古岳翻】祕書監牛弘【隋書牛弘傳弘安定鶉觚人本姓尞氏父允仕魏賜姓牛氏】上表以典籍屢經喪亂【上時掌翻喪息浪翻】率多散逸周氏聚書僅盈萬卷平齊所得除其重雜裁益五千興集之期屬膺聖世【重直龍翻裁與纔同屬之欲翻膺當也】為國之本莫此為先豈可使之流落私家不歸王府必須勒之以天威引之以微利則異典必臻觀閣斯積【漢東觀及天禄石渠等閣皆藏書之所故云觀古玩翻】隋主從之丁巳詔購求遺書於天下每獻書一卷賚縑一匹【賚洛代翻賜也與也縑絹也說文曰并絲繒】 夏四月庚午吐谷渾寇隋臨洮洮州刺史皮子信出戰敗死汶州摠管梁遠擊走之又寇廓州州兵擊走之【五代志後周武帝逐吐谷渾置洮陽郡尋立洮州汶山郡後周置汶州宋白曰晉置廣陽縣於茂州汶山縣西北五十里今不詳其處所後周又立廣陽縣于石鏡山面六十里至舊廣陽即今縣也又置汶州於此汶讀曰珉隋改會州澆河郡亦周逐吐谷渾以置廓州】 壬申隋以尚書右僕射趙煚兼内史令【煚古迥翻】突厥數為隋寇【厥几勿翻數所角翻】隋主下詔曰往者周齊抗

  衡分割諸夏【夏戶雅翻】突厥之虜俱通二國周人東慮恐齊好之深【好呼到翻】齊氏西虞懼周交之厚謂虜意輕重國遂安危蓋並有大敵之憂思減一邊之防也朕以為厚斂兆庶【斂力贍翻】多惠犲狼未嘗感恩資而為賊節之以禮不為虛費省徭薄賦國用有餘因入賊之物加賜將士【將即亮翻下同】息道路之民務為耕織清邊制勝成策在心凶醜愚闇未知深旨將大定之日比戰國之時乘昔世之驕結今時之恨近者盡其巢窟俱犯北邊蓋上天所忿驅就齊斧【齊讀曰齋言齋戒而授斧鉞於將帥一讀曰資應劭曰利斧也】諸將今行義兼含育有降者納【降戶江翻】有違者死使其不敢南望永服威刑何用侍子之朝寧勞渭橋之拜【匈奴遣子入侍及來朝渭橋並見漢宣帝紀朝直遥翻】於是命衛王爽等為行軍元帥【帥所類翻】分八道出塞擊之爽督摠管李充等四將出朔州道【自馬邑出塞也】己卯與沙鉢略可汗遇於白道【白道在長城北有白道嶺白道溪】李充言於爽曰突厥狃於驟勝【狙女久翻】必輕我而無備以精兵襲之可破也諸將多以為疑唯長史李徹贊成之遂與充帥精騎五千掩擊突厥大破之【帥讀曰率騎奇寄翻並下】沙鉢畧弃所服金甲潜艸中而遁其軍中無食粉骨為糧加以疾疫死者甚衆幽州摠管隂夀帥步騎十萬出盧龍塞擊高寶寧寶寧求救於突厥突厥方禦隋師不能救庚辰寶寧弃城奔磧北【磧七迹翻】和龍諸縣悉平夀設重賞以購寶寧又遣人離其腹心寶寧奔契丹為其麾下所殺【高寶寧自齊末據和龍至是敗滅契欺訖翻又音喫】 己丑郢州城主張子譏遣使請降於隋【郢州治江夏中流之重鎭今欲降隋史言陳之邊將已離心使疏吏翻下同降戶江翻】隋主以和好不納【好呼到翻】 辛卯隋主遣兼散騎常侍薛舒兼散騎常侍王劭來聘劭松年之子也【王松年仕齊為通直散騎侍郎人在下中散悉亶翻騎奇寄翻】 癸巳隋主大雩【隋雩壇在國南十三里啟夏門外道左】甲子突厥遣使入見于隋【見賢遍翻】 隋改度支尚書為

  民部【度徒洛翻】都官尚書為刑部命左僕射判吏禮兵三部事右僕射判民刑工三部事廢光禄衛尉鴻臚寺及都水臺【臚陵如翻】 五月癸卯隋行軍摠管李晃破突厥於摩那度口 乙巳梁太子琮入朝于隋賀遷都【朝直遥翻】 辛酉隋主祀方澤【隋為方丘於宮城之北十四里】 隋秦州摠管竇榮定帥九摠管步騎三萬出凉州【帥讀曰率】與突厥阿波可汗相拒於高越原阿波屢敗榮定熾之兄子也【厥九勿翻可從刋入聲汗音寒竇熾時為太傅】前上大將軍京兆史萬歲坐事配敦煌為戌卒【敦煌郡瓜州敦徒門翻】詣榮定軍門請自効榮定素聞其名見而大悦壬戌將戰榮定遣人謂突厥曰士卒何罪而殺之但當各遣一壯士決勝負耳突厥許諾因遣一騎挑戰【騎奇寄翻下同挑徒了翻】榮定遣萬歲出應之萬歲馳斬其首而還【還音旋又如字】突厥大驚不敢復戰【復扶又翻】遂請盟引軍而去長孫晟時在榮定軍中為偏將【將即亮翻】使謂阿波曰攝圖每來戰皆大勝阿波纔入遽即奔敗此乃突厥之恥也且攝圖之與阿波兵勢本敵今攝圖日勝為衆所崇阿波不利為國生辱攝圖必當以辠歸阿波成其宿計滅北牙矣【阿波建牙在攝圖之北】願自量度【量音良度徒洛翻】能禦之乎阿波使至【使疏吏翻下同】晟又謂之曰今達頭與隋連和而攝圖不能制可汗何不依附天子連結達頭相合為彊此萬全計也豈若喪兵負辠【喪息浪翻】歸就攝圖受其戮辱邪【邪音耶】阿波然之遣使随晟入朝沙鉢略素忌阿波驍悍【朝直遥翻驍堅堯翻悍侯旰翻】自白道敗歸又聞阿波貳於隋因先歸襲擊北牙大破之殺阿波之母阿波還無所歸西奔達頭達頭大怒遣阿波帥兵而東【帥讀曰率】其部落歸之者將十萬騎遂與沙鉢略相攻屢破之復得故地兵埶益彊貪汗可汗素睦於阿波沙鉢略奪其衆而廢之貪汗亡奔達頭沙鉢從弟地勤察别統部落與沙鉢略有隙復以衆叛歸阿波【從才用翻復扶又翻】連兵不已各遣使詣長安請和求援隋主皆不許 六月庚辰隋行軍摠管梁遠破吐谷渾於爾汗山【吐從暾入聲谷音浴汗音寒】 突厥寇幽州隋幽州摠管廣宗壯公李崇帥步騎三千拒之【廣宗縣公廣宗漢古縣五代志屬清河郡厥九勿翻帥讀曰率騎奇寄翻】轉戰十餘日師人多死遂保砂城突厥圍之城荒頹不可守禦曉夕力戰又無所食每夜出掠虜營得六畜以繼軍糧【畜許又翻】突厥畏之厚為其備每夜中結陳以待之【陳讀曰陣下同】崇軍苦飢出輒遇敵死亡畧盡及明奔還城者尚百許人然多重傷不堪更戰突厥意欲降之【降戶江翻】遣使謂崇曰若來降者封為特勒【特勒突厥達官新書突厥子弟曰特勒使疏吏翻】崇知不免令其士卒曰崇喪師徒【令立定翻喪息浪翻】罪當萬死今日効命以謝國家汝俟吾死且可降賊便散走努力還鄉若見至尊道崇此意乃挺刃突陳復殺二人【挺拔也】突厥亂射殺之【射而亦翻】秋七月以豫州刺史代人周揺為幽州摠管命李崇子敏襲爵【襲爵廣宗公】敏娶樂平公主之女娥英【隋主受禪周天元后改封樂平公主】詔假一品羽儀禮如尚帝女既而將侍宴公主謂敏曰我以四海與至尊唯一壻當為爾求柱國【為于偽翻】若餘官汝愼勿謝及進見【見賢遍翻】帝授以儀同及開府皆不謝帝曰公主有大功於我我何得於其壻而惜官乎今授汝柱國敏乃拜而蹈舞 八月丁卯朔日有食之 【考異曰隋紀作七月丁卯蓋歷差】長沙王叔堅未之江州復留為司空【復扶又翻】實奪之權壬午隋遣尚書左僕射高熲出寧州道内史監虞慶

  則出原州道以擊突厥【五代志平凉郡舊置原州監甲暫翻厥九勿翻】 九月癸丑隋大赦 冬十月甲戌隋廢河南道行臺省【去年二月隋置河南道行臺省】以秦王俊為秦州摠管隴右諸州盡隸焉【秦州天水郡】 丁酉立皇弟叔平為湘東王叔敖為臨賀王叔宣為陽山王叔穆為西陽王 戊戌侍中建昌侯徐陵卒【卒子恤翻】 癸丑立皇弟叔儉為安南王叔澄為南郡王叔興為沅陵王叔韶為岳山王叔純為新興王【自湘東以下皆以郡名疏爵今著後人之所未能徧知者五代志始安郡富川縣舊置臨賀郡熙平郡桂陽縣梁置陽山郡永安郡黄岡縣置西陽郡巴陵郡華容縣舊曰安南置安南郡沅陵縣置沅陵郡信安郡新興縣梁置新興郡岳山郡闕郡縣志巴陵一名天岳山岳山蓋即巴陵以封叔韶沅音元】 十一月遣散騎常侍周墳通直散騎常侍袁彦聘于隋帝聞隋主狀貌異人使彦畫像而歸帝見大駭曰吾不欲見此人亟命屛之【散悉亶翻騎奇寄翻亟紀力翻屏必郢翻】 隋既班律令【前年十月隋行新律】蘇威屢欲更易事條【更工衡翻下數更同】内史令李德林曰修律令時公何不言今始頒行且宜專守自非大為民害不可數更【數所角翻】河南道行臺兵部尚書楊尚希曰竊見當今郡縣倍多於古或地無百里數縣並置或戶不滿千二郡分領具僚已衆資費日多吏卒增倍租調歲減【調徒弔翻】民少官多【少詩沼翻下同】十羊九牧今存要去【上聲】閒併小為大國家則不虧粟帛選舉則易得賢良【易以豉翻】蘇威亦請廢郡帝從之甲午悉罷諸郡為州 十二月乙卯隋遣兼散騎常侍曹令則通直散騎常侍魏澹來聘【散悉亶翻騎奇寄翻】澹收之族也【魏收事齊以文名澹徒覽翻】 丙辰司空長沙王叔堅免叔堅既失恩心不自安乃為厭媚【厭一琰翻】醮日月以求福或上書告其事 【考異曰南史云上隂令人造其厭媚又令人告之今從陳書】帝召叔堅囚于西省【門下省為東省中書省為西省】將殺之令近侍宣敕數之【數責其罪也數所具翻又所主翻】叔堅對曰臣之本心非有佗故但欲求親媚耳臣既犯天憲辠當萬死臣死之日必見叔陵願宣明詔責之於九泉之下帝乃赦之免官而已【因叔堅之言始念其擁護之功此豈少恩而已哉不明故爾】 隋以上柱國竇榮定為右武衛大將軍榮定妻隋主姊安成公主也隋主欲以榮定為三公辭曰衛霍梁鄧若少自貶損不至覆宗帝乃止【四姓皆漢外戚也衛氏夷於武帝之末霍族赤於宣帝之時桓帝怒而梁宗誅滅安帝長而鄧門衰廢事並見漢紀少詩紹翻】帝以李穆功大詔曰法備小人不防君子太師申公自今雖有辠但非謀逆縱有百死終不推問禮部尚書牛弘請立明堂帝以時事草創不許帝覽刑部奏斷獄數猶至萬【斷丁亂翻】以為律尚嚴密故人多陷罪又敕蘇威牛弘等更定新律除死罪八十一條流罪一百五十四條徒杖等千餘條唯定留五百條凡十二卷【一曰名例二曰衛禁三曰職制四曰戶婚五曰廐庫六曰擅興七曰盜賊八曰鬬訟九曰詐偽十曰雜律十一曰捕亡十二曰斷獄】自是刑網簡要疎而不失仍置律博士弟子員【大理寺之屬有律博士八人】 隋主以長安倉廩尚虛是歲詔西自蒲陜東至衛汴【河東郡蒲州恒農郡陜州汲郡衛州陳留郡汴州陜式冉翻汴皮變翻】水次十三州募丁運米【華陜穀洛管汴汾晉蒲絳懷衛相凡十三州】又於衛州置黎陽倉陜州置常平倉華州置廣通倉【五代志京兆郡鄭縣後魏置東雍州並華山郡西魏改曰華州華戶化翻】轉相灌輸【輸平聲】漕關東及汾晉之粟以給長安【水運曰漕關東自函谷關以東州郡五代志文城郡東魏置南汾州後周改為汾州晉州臨汾郡舊平陽郡也漕在到翻】時刺史多任武將類不稱職【將即亮翻下同稱尺澄翻】治書侍御史柳彧上表曰【治直之翻上時掌翻】昔漢光武與二十八將披荆棘定天下及功成之後無所任職【事見漢光武紀】伏見詔書以上柱國和千子為州刺史【五代志梁郡雍丘縣隋置州】千子前任趙州【趙州時治廣阿】百姓歌之曰老禾不早殺【今人猶呼割稻為殺稻】餘種穢良田【種章勇翻】千子弓馬武用是其所長治民涖衆非其所解【治直之翻解戶買翻曉也】如謂優老尚年【尚尊也高也】自可厚賜金帛若令刺舉【漢置刺史掌刺舉郡縣吏故云然】所損殊大帝善之千子竟免彧見上勤於聽受百僚奏請多有煩碎上疏諫曰臣聞上古聖帝莫過唐虞不為樷脞是謂欽明【書元首叢脞哉孔安國曰叢脞細碎無大略馬云叢總也脞小也堯典曰放勲欽明文思脞倉果翻】舜任五臣【孔子曰舜有臣五人而天下治孔安國曰五臣禹稷契臯陶伯益】堯咨四岳【孔安國曰四岳即羲和之四子分掌四岳之諸侯】垂拱無為天下以治【治直吏翻下同】所謂勞於求賢逸於任使比見陛下留心治道【比毗至翻治直吏翻】無憚疲勞亦由羣官懼罪不能自決取判天旨【判決也】聞奏過多乃至營造細小之事出給輕微之物一日之内酧荅百司至乃日旰忘食夜分未寢【旰古按翻日晏也夜分半夜也】動以文簿憂勞聖躬伏願察臣至言少減煩務【少詩沼翻】若經國大事非臣下裁斷者伏願詳決【斷丁亂翻】自餘細務責成所司則聖體盡無疆之夀臣下蒙覆育之賜【覆敷救翻】上覽而嘉之因曰柳彧直士國之寶也彧以近世風俗每正月十五夜然燈遊戲奏請禁之【上元燃燈或云以漢祠太一自昏至晝故事此說非也梁簡文帝有列燈詩陳後主有光璧殿遥詠山燈詩則柳彧所謂近世風俗是也】曰竊見京邑爰及外州每以正月望夜【望夜月之十五夜也月旦以日月合謂之朔十五夜以日月相望謂之望】充街塞陌【塞悉則翻】聚戲朋遊鳴鼓聒天燎炬照地竭貲破產競此一時盡室并孥無問貴賤男女混雜緇素不分穢行因此而成盜賊由斯而起【觀此則上元遊戲之弊其來久矣後之當路者能不惑於世俗奮然革之亦所謂豪傑之士也】因循弊風曾無先覺無益於化實損於民請頒天下並即禁斷【斷音短】詔從之

  資治通鑑卷一百七十五  
    


 


 



 

 
  







 


  
  
 
 
 


  

 















	
	









































 
  



















 





 












  
  
  

 





