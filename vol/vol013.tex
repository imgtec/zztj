資治通鑑卷十三    宋 司馬光 撰

胡三省 音註

漢紀五|{
	起閼逢攝提格盡昭陽大淵獻凡十年}


高皇后|{
	荀悦曰諱雉之字曰野雞索隱曰字娥姁應劭曰禮婦人從夫諡故稱高也師古曰諱雉故臣下諱雉也姁許于翻}


元年冬太后議欲立諸呂為王問右丞相陵陵曰高帝刑白馬盟曰|{
	高祖刑白馬與羣臣盟曰非劉氏不王非有功不侯}
非劉氏而王天下共擊之今王呂氏非約也太后不說|{
	說讀曰悦}
問左丞相平太尉勃對曰高帝定天下王子弟今太后稱制王諸呂無所不可|{
	王于况翻}
太后喜罷朝|{
	朝直遥翻}
王陵讓陳平絳侯曰始與高帝啑血盟諸君不在耶|{
	啑所甲翻小啜也索隱引鄒氏音使接翻}
今高帝崩太后女主欲王呂氏諸君縱欲阿意背約|{
	背蒲妹翻}
何面目見高帝於地下乎陳平絳侯曰於今面折廷爭|{
	謂當朝廷而諫諍}
臣不如君全社稷定劉氏之後君亦不如臣陵無以應之十一月甲子太后以王陵為帝太傅實奪之相權陵遂病免歸乃以左丞相平為右丞相|{
	此時尚右故陳平自左丞相遷右丞相}
以辟陽侯審食其為左丞相不治事|{
	治直之翻}
令監宫中如郎中令|{
	言食其不董丞相職事常監宫中若郎中令監古銜翻}
食其故得幸於太后公卿皆因而決事太后怨趙堯為趙隱王謀乃抵堯罪|{
	堯為趙王謀事見上卷高祖十年趙王如意諡隱諡法隱拂不成曰隱不顯尸國曰隱見美堅長曰隱為于偽翻}
上黨守任敖嘗為沛獄吏有德於太后乃以為御史大夫|{
	任敖沛人少為獄吏高祖常避吏吏繫呂后遇之不謹敖擊傷主呂后吏故后德之}
太后又追尊其父臨泗侯呂公為宣王兄周呂令武侯澤為悼武王欲以王諸呂為漸|{
	臨泗侯班表以后父賜號索隱曰應劭云周呂國也按周及呂皆國名濟隂有呂都縣晉灼曰呂縣名以為侯國予據班志呂縣屬楚國令武諡也}
春正月除三族罪妖言令|{
	秦為威虐罪之重者戮及三族過誤之語以為妖言故皆除之}
夏四月魯元公主薨封公主子張偃為魯王諡公主曰魯元太后 辛卯封所名孝惠子山為襄城侯|{
	班志襄城縣屬潁川郡}
朝為軹侯|{
	軹縣屬河内郡}
武為壺關侯|{
	壺關縣屬上黨郡}
太后欲王呂氏乃先立所名孝惠子彊為淮陽王不疑為恒山王|{
	惠帝元年淮陽王友徙王趙今以封彊恒山郡本屬趙國今割以封不疑恒戶登翻}
使大謁者張釋風大臣|{
	風讀曰諷漢書惠景間侯者表及匈奴傳皆作澤史記呂后本紀八年中大謁者張釋漢書紀作釋卿恩澤侯表及周勃傳皆云張釋顔師古註曰荆燕吳傳云張擇今從史記呂后本紀漢書恩澤侯}
大臣乃請立悼武王長子酈侯台為呂王|{
	蘇林曰台音胞胎之台索隱曰鄭鄒並音怡 考異曰漢書外戚侯表及高五王傳皆作鄜侯今從史記本紀功臣侯表}
割齊之濟南郡為呂國|{
	濟子禮翻}
五月丙申趙王宫叢臺災|{
	劉昭志趙國邯鄲縣有叢臺}
秋桃李華

二年冬十一月呂肅王台薨 |{
	考異曰史記本紀高后元年立孝惠子不疑為恒山王呂台為呂王二年恒山王薨十一月呂王台薨年表二人皆以元年薨漢書本紀元年立不疑呂台產禄通為王二年不疑薨年表元年不疑及呂台為王二年皆薨盖史記年表薨字應在二年誤書於元年耳其實二人皆以二年薨漢書本紀云產禄通為王亦誤也}
春正月乙卯地震羌道武都道山崩|{
	羌道班志縣蜀隴西郡武都時為縣漢志縣雜蠻夷曰道武帝置武都郡}
夏五月丙申封楚元王子郢客為上邳侯齊悼惠王子章為朱虚侯|{
	班志東海下邳縣應劭曰邳在薛其後徙此故曰下邳臣瓚曰有上邳故曰下邳師古曰瓚說是也班志朱虚縣屬琅邪郡括地志朱虛故城在青州臨朐縣東六十里漢朱虚也十三州志丹朱遊故虛故云朱虚也虚猶丘也朱猶丹也索隱虚音墟 考異曰史記高后紀在元年今從漢書王子侯表}
令入宿衛又以呂禄女妻章|{
	妻千細翻}
六月丙戌晦日有食之秋七月恒山哀王不疑薨|{
	恒戶登翻}
行八銖錢|{
	應劭曰本秦錢質如周錢文曰半兩重如其文即八銖也漢以其太重更鑄莢錢今民間名榆莢錢是也民患其太輕至是復行八銖錢}
癸丑立襄成侯山為恒山王更名義|{
	更工衡翻}


三年夏江水漢水溢流四千餘家|{
	班志江水出蜀郡湔氐道徼外㟭山東南至江都入海禹貢嶓冢導漾東流為漢孔安國註曰泉始出山為漾水東南流為沔水至漢中東行為漢水班志隴西氐道縣禹貢漾水所出至武都為漢又於武都註曰東漢水受氐道水一名沔過江夏謂之夏水入江又漢中郡有沔陽縣如淳註曰此方人謂漢水為沔水師古曰漢上曰沔水經則以為沔漾異源漾出隴西氐道嶓冢山東至武都沮縣為漢水其流東南歷白水葭萌又東南過巴郡閬中至江津縣而入於江涪水注之庾仲雍所謂内水者也沔水出武都沮縣東狼谷中一名沮水東逕漢中郡沔陽南鄭成固等縣又東逕西城錫縣又東逕南郡襄陽中廬即宜城郡當陽縣又東逕江夏雲杜縣又南至沙羨縣入江予據禹貢導漾東流為漢又東為滄浪之水過三澨至大别南入於江則漢水源出於漾據水經則漾會於涪沔入於江所出異源所入異派據班志則漾出隴西氐道至武都為漢水而東漢水受氐道水通謂之沔過江夏而入于江則漾沔似合為一矣然又言沮水出沮縣南至沙羨入江與水經所謂沔水即沮水說似不合而實合也}
秋星晝見|{
	見賢遍翻}
伊水洛水溢流千六百餘家|{
	班志伊水出弘農郡熊耳山東北入洛水水經伊水出南陽縣荀渠山酈道元註即大同陵巒牙别耳又班志洛水出弘農上洛縣東北至河南鞏縣入河}
汝水溢流八百餘家|{
	應劭曰汝水出弘農縣入淮水經汝水出南陽魯陽縣之大孟山東南逕潁川之郏定陵郾又東南過汝南之上蔡平輿南入於淮}


四年春二月癸未立所名孝惠子太為昌平侯|{
	班志昌平縣屬上谷郡}
夏四月丙申太后封女弟為臨光侯|{
	音須}
少帝寖長自知非皇后子|{
	惠帝張皇后魯元公主之女太后以其無子使陽為有身取後宫美人子名之而殺其母少帝及義朝彊不疑皆是也長知兩翻}
乃出言曰后安能殺吾母而名我我壯即為變太后聞之幽之永巷中言帝病左右莫得見太后語羣臣曰|{
	語牛倨翻}
今皇帝病久不已失惑昏亂不能繼嗣治天下|{
	治直之翻}
其代之羣臣皆頓首言皇太后為天下齊民計所以安宗廟社稷甚深羣臣頓首奉詔遂廢帝幽殺之五月丙辰立恒山王義為帝更名曰弘|{
	更工衡翻}
不稱元年以太后制天下事故也以軹侯朝為恒山王|{
	恒戶登翻}
是歲以平陽侯曹窋為御史大夫|{
	窋張律翻}
有司請禁南越關市鐵器|{
	漢於邉關與蠻夷通市謂之關市}
南越王佗曰高帝立我通使物今高后聽讒臣别異蠻夷隔絶器物此必長沙王計欲倚中國擊滅南越而并王之自為功也|{
	使疏吏翻别彼列翻并王于况翻}


五年春佗自稱南越武帝|{
	韋眧曰生以武為號不稽古也師古曰此說非也湯曰吾武甚自號曰武王佗言武帝亦猶是耳何謂其不稽古乎貢父曰顔雖引成湯之言然未知湯自號武王乎聖人者人與之名耳詩謂湯為武王亦猶書謂文王為寧王耳史記之言未可信也}
發兵攻長沙敗數縣而去|{
	敗補邁翻}
秋八月淮陽懷王彊薨以壺關侯武為淮陽王九月發河東上黨騎屯北地 初令戍卒歲更|{
	秦虐用其民南戍五嶺北築長城戌卒連年不歸而死者多矣至此始令一歲而更更工衡翻}
六年冬十月太后以呂王嘉居處驕恣廢之|{
	嘉台之子也二年台薨嘉嗣處昌呂翻}
十一月立肅王弟產為呂王|{
	台諡曰肅}
春星晝見|{
	見賢遍翻}
夏四月丁酉赦天下 封朱虚侯章弟興居為東牟侯|{
	班志東牟縣屬東萊郡賢曰東牟故城在今萊州文登縣西北}
亦入宿衛匈奴寇狄道攻阿陽|{
	班志狄道縣屬隴西郡阿陽縣屬天水郡}
行五分錢|{
	應劭曰所謂莢錢者}
宣平侯張敖卒 |{
	考異曰史記呂后本紀敖卒在明年六月按史記功臣表高后六年敖卒漢書功臣表敖以高祖九年封十七年薨盖本紀之誤}
賜諡曰魯元王|{
	張敖本嗣父耳爵為趙王貫高之謀發敖廢為宣平侯仍尚魯元公主及惠帝之世齊悼惠王獻城陽郡以奉魯元敖之卒也因公主而賜諡曰魯元王}


七年冬十二月匈奴寇狄道畧二千餘人 春正月太后召趙幽王友|{
	惠帝元年友自淮陽徙王趙}
友以諸呂女為后弗愛愛他姬諸呂女怒去讒之於太后曰王言呂氏安得王太后百歲後吾必擊之太后以故召趙王趙王至置邸不得見|{
	言置之趙邸也師古曰郡國朝宿之舍在京師者率名邸邸至也言所歸至也邸丁禮翻}
令衛圍守之弗與食其羣臣或竊饋輒捕論之|{
	捕其饋者以罪論之}
丁丑趙王餓死以民禮葬之長安民冢次 己丑日食晝晦太后惡之謂左右曰此為我也|{
	惡烏路翻為于偽翻下為之同}
二月徙梁王恢為趙王呂王產為梁王梁王不之國為帝太傅 秋七月丁巳立平昌侯太為濟川王|{
	四年封太為昌平侯班表亦作昌平此誤以平字在上濟川即濟南濟北之地盖割齊封之時太年幼未嘗之國濟子禮翻}
呂女為將軍營陵侯劉澤妻|{
	班志營陵縣屬北海郡或曰營丘應劭曰}


|{
	師尚父封於營丘陵亦丘也臣瓚曰營丘即臨淄營陵春秋謂之緣陵師古曰臨菑營陵皆故營丘地括地志營陵故城在青州北海縣南三十里}
澤者高祖從祖昆弟也|{
	從才用翻下后從同}
齊人田生為之說大謁者張卿曰|{
	張卿即前大謁者張釋也說式芮翻}
諸呂之王也諸大臣未大服今營陵侯澤諸劉最長今卿言太后王之|{
	長知兩翻王于况翻}
呂氏王益固矣張卿入言太后太后然之乃割齊之琅邪郡封澤為琅邪王|{
	秦滅齊以瀕海之地置琅邪郡漢因之 考異曰史記世家漢書列傳皆云田生先說張卿令風大臣立呂產為呂王然後說令王澤按太后自以呂王嘉驕恣廢之以產代為呂王非始封於呂又諸呂之王已久何必待田生之謀以此不取}
趙王恢之徙趙心懷不樂|{
	樂音洛}
太后以呂產女為王后王后從官皆諸呂擅權微伺趙王|{
	從才用翻}
趙王不得自恣王有所愛姬王后使人酖殺之六月王不勝悲憤自殺|{
	勝音升}
太后聞之以為王用婦人棄宗廟禮|{
	諸侯王有國所以奉宗廟也今恢以愛姬之故至於自殺故以棄宗廟禮罪之}
廢其嗣是時諸呂擅權用事朱虚侯章年二十有氣力忿劉氏不得職嘗入侍太后燕飲太后令章為酒吏章自請曰臣將種也|{
	將即亮翻種章勇翻下其種同}
請得以軍法行酒太后曰可酒酣章請為耕田歌太后許之章曰深耕穊種立苗欲疏非其種者鋤而去之|{
	師古曰穊稠也穊種言多生子孫也疏立者四散置之令為藩輔也非其種者鋤而去之以斥諸呂也穊音冀去羌呂翻}
太后默然頃之諸呂有一人醉亡酒章追拔劒斬之而還報曰有亡酒一人臣謹行法斬之|{
	師古曰亡酒避酒而逃亡也}
太后左右皆大驚業已許其軍法無以罪也因罷自是之後諸呂憚朱虚侯雖大臣皆依朱虛侯劉氏為益彊|{
	為于偽翻下因為同}
陳平患諸呂力不能制恐禍及己嘗燕居深念|{
	師古曰以國家不安故靜居獨慮其方策}
陸賈往直入坐而陳丞相不見|{
	師古曰言不因門人將命而徑自入座平方深思不覺其至坐徂卧翻}
陸生曰何念之深也陳平曰生揣我何念|{
	揣初委翻度也}
陸生曰足下極富貴無欲矣然有憂念不過患諸呂少主耳陳平曰然為之奈何陸生曰天下安注意相天下危注意將將相和調則士豫附|{
	師古曰豫素也余謂豫順也}
天下雖有變權不分為社稷計在兩君掌握耳臣嘗欲謂太尉絳侯絳侯與我戲易吾言|{
	謂告語也言絳侯素與之戲狎輕易其言也周勃封絳侯班志絳縣屬河東郡晉之舊都}
君何不交驩太尉深相結因為陳平畫呂氏數事陳平用其計乃以五百金為絳侯夀厚具樂飲|{
	師古曰厚為其具而與太尉樂飲樂音洛}
太尉報亦如之兩人深相結呂氏謀益衰陳平以奴婢百人車馬五十乘錢五百萬遺陸生為飲食費|{
	遺于季翻}
太后使使吿代王|{
	高祖七年立子恒為代王}
欲徙王趙|{
	王于况翻}
代王謝之願守代邊太后乃立兄子呂禄為趙王追尊禄父建成康侯釋之為趙昭王 九月燕靈王建薨有美人子太后使人殺之國除|{
	高祖初封盧綰於燕綰入匈奴乃立建為燕王美人子美人所生之子也}
遣隆慮侯周竈將兵擊南越|{
	班志隆慮縣屬河内郡至後漢避殤帝諱改曰林慮慮音閭}


八年冬十月辛丑立呂肅王子東平侯通為燕王|{
	東平地名在濟東宣帝甘露二年為東平國}
封通弟莊為東平侯 三月太后袚還過軹道|{
	師古曰袚者除惡之祭袚音廢又敷勿翻}
見物如蒼犬撠太后掖|{
	師古曰撠謂拘持之也撠音戟拘居足翻掖與腋同}
忽不復見卜之云趙王如意為祟|{
	祟雖遂翻神禍也鬼厲也}
太后遂病掖傷太后為外孫魯王偃年少孤弱|{
	偃張敖子為于偽翻}
夏四月丁酉封張敖前姬兩子侈為新都侯|{
	班志新都縣屬南陽郡}
夀為樂昌侯|{
	徐廣曰樂昌今細陽之池陽鄉余據班志細陽縣屬汝南郡又東郡有樂昌縣 考異曰史記惠景間侯者表新都作信都夀作受今從本紀}
以輔魯王又封中大謁者張釋為建陵侯|{
	如淳曰灌嬰為中謁者後常以閹人為之諸官加中者多閹人也班志建陵縣屬東海郡}
以其勸王諸呂賞之也江漢水溢流萬餘家 秋七月太后病甚乃令趙王禄為上將軍居北軍呂王產居南軍|{
	班表中壘校尉掌北軍壘門外又有中尉掌徼循京師屬官有中壘寺互等令丞至後漢始置北軍中掌監五營劉昭註曰舊有中壘校尉領北軍營壘之事中興省中壘但置中以監五營又據班表中壘以下八校尉皆武帝置意者武帝以前北軍屬中尉故領中壘令丞等官南軍盖衛尉所統班表衛尉掌宫門衛屯兵周勃之入北軍也尚有南軍乃先使曹窋告衛尉毋入呂產殿門然後使朱虚侯逐產殺之未央宫郎中府吏廁中以此知南軍屬衛尉也}
太后誡產禄曰呂氏之王大臣弗平我即崩帝年少大臣恐為變必據兵衛宫慎毋送喪為人所制辛巳太后崩遺詔大赦天下以呂王產為相國以呂禄女為帝后高后已葬以左丞相審食其為帝太傅 |{
	考異曰史記將相表八年七月辛巳食其為太傅九月丙戌復為丞相後九月免漢書公卿表七年七月辛巳食其為太傅八年九月復為丞相後九月免以長歷推之八年七月無辛巳九月無丙戌閏月羣臣代邸上議無食其名二表皆誤今從史記本紀免相在此月本紀又云八月壬戌食其復為左丞相亦誤}
諸呂欲為亂畏大臣絳灌等未敢發朱虚侯以呂禄女為婦故知其謀乃隂令人吿其兄齊王欲令發兵西朱虚侯東牟侯為内應以誅諸呂立齊王為帝齊王乃與其舅駟鈞郎中令祝午中尉魏勃隂謀發兵齊相召平弗聽|{
	班表諸侯王高祖初置有太傅輔王内史治國民中尉掌武職丞相統衆官如漢朝景帝中五年令諸侯王不得復治國天子為置吏改丞相曰相武帝分漢内史為左右後又更右為京兆尹左為馮翊中尉為執金吾郎中令為光禄勲故王國如故損其郎中令秩千石改太僕曰僕秩亦千石成帝綏和元年更令相治民如郡太守中尉如郡都尉康曰廣陵人召平與東陵侯召平及此召平凡三人此召平之子奴以平死事封黎侯見功臣表召與邵同姓譜駟鄭七穆駟氏之後祝周武王封黄帝之後於祝後以為氏}
八月丙午齊王欲使人誅相相聞之乃發卒衛王宫魏勃紿邵平曰王欲發兵非有漢虎符驗也|{
	應劭曰銅虎符第一至第五國家當發兵遣使者至郡合符符合乃聽受之張晏曰符以代古之圭璋從簡易也予據史記文帝紀三年九月初與郡國守相為銅虎符既有初字則前乎文帝之時當未有銅虎符也召平魏勃事在三年之前何緣有虎符發兵班史于文紀三年只書初與郡守為銅虎符汰去國相二字温公則但書勃語於此而文紀不復書豈亦有疑於此邪}
而相君圍王固善勃請為君將兵衛王|{
	為于偽翻}
召平信之勃既將兵遂圍相府召平自殺於是齊王以駟鈞為相魏勃為將軍祝午為内史悉發國中兵使祝午東詐琅邪王曰|{
	琅邪王劉澤也三年割齊琅邪封之}
呂氏作亂齊王發兵欲西誅之齊王自以年少不習兵革之事願舉國委大王大王自高帝將也|{
	言澤自高帝時為將}
請大王幸之臨菑見齊王計事|{
	臨菑即古營丘齊國所都}
琅邪王信之西馳見齊王齊王因留琅邪王而使祝午盡發琅邪國兵并將之 |{
	考異曰史記澤世家漢書傳皆以為澤與齊王合謀盖誤今從史記呂后本紀齊王世家漢書呂后紀齊王傳}
琅邪王說齊王曰大王高皇帝適長孫也當立|{
	適讀曰嫡齊王襄悼惠王之子高帝之長孫也長知兩翻下同}
今諸大臣狐疑未有所定而澤於劉氏㝡為長年大臣固待澤決計今大王留臣無為也不如使我入關計事齊王以為然乃益具車送琅邪王琅邪王既行齊遂舉兵西攻濟南|{
	濟南本屬齊元年割以封呂台台卒產嗣封}
遺諸侯王書|{
	遺于季翻}
陳諸呂之罪欲舉兵誅之相國呂產等聞之乃遣潁隂侯灌嬰將兵擊之|{
	班志潁隂縣屬潁川郡}
灌嬰至滎陽謀曰諸呂擁兵關中欲危劉氏而自立今我破齊還報此益呂氏之資也乃留屯滎陽使使諭齊王及諸侯與連和以待呂氏變共誅之齊王聞之乃還兵西界待約呂禄呂產欲作亂内憚絳侯朱虚等外畏齊楚兵又恐灌嬰畔之欲待灌嬰兵與齊合而發猶豫未決當是時濟川王太淮陽王武常山王朝及魯王張偃皆年少未之國居長安趙王禄梁王產各將兵居南北軍皆呂氏之人也列侯羣臣莫自堅其命太尉絳侯勃不得主兵曲周侯酈商老病|{
	班志曲周縣屬廣平國}
其子寄與呂禄善絳侯乃與丞相陳平謀使人刼酈商令其子寄往紿說呂禄曰高帝與呂后共定天下劉氏所立九王|{
	楚王交高祖弟代王恒淮南王長高祖子吳王濞高祖姪琅邪王澤劉氏疏屬齊王襄高祖孫常山王朝淮陽王武濟川王太惠帝子說式芮翻}
呂氏所立三王|{
	梁王呂產趙王呂禄燕王呂通也}
皆大臣之議事已布吿諸侯皆以為宜今太后崩帝少而足下佩趙王印不急之國守藩乃為上將將兵留此為大臣諸侯所疑足下何不歸將印以兵屬太尉|{
	屬之欲翻下同}
請梁王歸相國印與大臣盟而之國齊兵必罷大臣得安足下高枕而王千里此萬世之利也|{
	而王于况翻}
呂禄信然其計欲以兵屬太尉使人報呂產及諸呂老人或以為便或曰不便計猶豫未有所決呂禄信酈寄時與出游獵過其姑呂嬃大怒曰|{
	呂后之妹樊噲之妻於禄姑也過工禾翻}
若為將而棄軍呂氏今無處矣乃悉出珠玉寶器散堂下曰毋為他人守也九月庚申旦 |{
	考異曰史記本紀八月庚申旦上有八月丙午漢書高后紀亦云八月庚申今以長歷推之下八月當為九月}
平陽侯窋行御史大夫事見相國產計事郎中令賈夀使從齊來|{
	姓譜周康王封唐叔虞少子公明於賈城子孫以國為氏又晉大夫賈季食邑于賈其後以邑為氏}
因數產曰王不早之國今雖欲行尚可得邪|{
	數所具翻}
具以灌嬰與齊楚合從欲誅諸呂吿產|{
	師古曰齊楚俱在山東連兵西鄉欲誅諸呂亦猶六國為從以敵秦故謂之合從也從子容翻}
且趣產急入宫|{
	趣讀曰促}
平陽侯頗聞其語馳吿丞相太尉太尉欲入北軍不得入襄平侯紀通尚符節乃令持節矯内太尉北軍|{
	班志襄平縣屬遼東郡張晏曰紀通紀信子也尚主也今符節令也晉灼曰紀信焚死不見其後功臣表云通紀成之子以成死事故封侯貢父曰漢祖以善用人得天下豈忘紀信之功哉疑成者即信之一名也通尚符節故使持節矯以帝命内勃北軍内讀曰納}
太尉復令酈寄與典客劉揭先說呂禄曰|{
	復扶又翻班志典客秦官掌諸侯歸義蠻夷景帝中六年更名大行令武帝太初元年更名大鴻臚掲音竭}
帝使太尉守北軍欲足下之國急歸將印辭去不然禍且起呂禄以為酈况不欺己遂解印屬典客而以兵授太尉太尉至軍呂禄已去太尉入軍門行令軍中曰為呂氏右袒為劉氏左袒|{
	師古曰袒者脱衣袖而肉袒也左右袒者偏脱其一耳袒徒旱翻鄭氏注覲禮云凡為禮事者左袒若請罪待刑則右袒}
軍中皆左袒太尉遂將北軍然尚有南軍丞相平乃召朱虚侯章佐太尉太尉令朱虚侯監軍門|{
	監古銜翻}
令平陽侯吿衛尉毋入相國產殿門|{
	衛尉掌宫門衛屯兵平陽侯時為御史大夫盖將丞相之命以告衛尉使毋納產也}
呂產不知呂禄已去北軍乃入未央宫欲為亂至殿門弗得入徘徊往來平陽侯恐弗勝馳語太尉|{
	語牛倨翻}
太尉尚恐不勝諸呂未敢公言誅之乃謂朱虚侯曰急入宫衛帝朱虚侯請卒太尉予卒千餘人|{
	予讀曰與}
入未央宫門見產廷中日餔時|{
	申時食為餔餔奔謨翻}
遂擊產產走天風大起以故其從官亂莫敢鬬逐產殺之郎中府吏廁中|{
	如淳曰郎中令掌宫殿門戶故府在宫中從才用翻}
朱虚侯已殺產帝命謁者持節勞朱虛侯|{
	勞力到翻}
朱虚侯欲奪其節謁者不肯朱虚侯則從與載因節信馳走|{
	師古曰因謁者所持之節用為信也章與謁者同車故為門者所信得入長樂宫}
斬長樂衛尉呂更始|{
	更工衡翻}
還馳入北軍報太尉太尉起拜賀朱虚侯曰所患獨呂產今已誅天下定矣遂遣人分部悉捕諸呂男女無少長皆斬之|{
	分扶問翻}
辛酉捕斬呂禄而笞殺呂使人誅燕王呂通而廢魯王張偃戊辰徙濟川王王梁|{
	呂產既誅故徙太王梁}
遣朱虚侯章以誅諸呂事吿齊王令罷兵灌嬰在滎陽聞魏勃本教齊王舉兵使使召魏勃至責問之勃曰失火之家豈暇先言丈人而後救火乎因退立股戰而栗|{
	師古曰言以社稷將危故舉兵而正之不暇待有詔命也股脚也戰者懼之甚也栗與慄同}
恐不能言者終無他語灌將軍熟視笑曰人謂魏勃勇妄庸人耳何能為乎乃罷魏勃灌嬰兵亦罷滎陽歸

班固贊曰孝文時天下以酈寄為賣友|{
	言寄與禄友善詭說之出游因奪其兵而誅之是寄賣友也}
夫賣友者謂見利而忘義也若寄父為功臣而又執刼雖摧呂禄以安社稷誼存君親可也|{
	師古曰周勃刼其父令其子行說予謂刼者刼質也盖刼寄父商為質諭以不行說禄將殺之也盖當時皆以寄為賣友故固發明父子朋友各有其倫為人臣子者當知所緩急先後也}


諸大臣相與隂謀曰少帝及梁淮陽恒山王皆非真孝惠子也呂后以計詐名他人子殺其母養後宫令孝惠子之立以為後及諸王以彊呂氏今皆已夷滅諸呂而所立即長用事吾屬無類矣|{
	長知兩翻下同}
不如視諸王㝡賢者立之或言齊王高帝長孫可立也大臣皆曰呂氏以外家惡而幾危宗廟亂功臣|{
	幾居衣翻}
今齊王舅駟鈞虎而冠|{
	言駟鈞惡戾如虎而著冠}
即立齊王復為呂氏矣代王方今高帝見子㝡長仁孝寛厚太后家薄氏謹良|{
	言高帝見在諸子惟代王為㝡長也見賢遍翻代王高帝姬薄氏所生薄姓戰國已有之風俗通衛有賢人薄疑}
且立長固順况以仁孝聞天下乎乃相與共隂使人召代王代王問左右郎中令張武等曰漢大臣皆故高帝時大將習兵多謀詐此其屬意非止此也|{
	師古曰言常有異志也屬意猶言注意也屬音之欲翻}
特畏高帝呂太后威耳今已誅諸呂新啑血京師|{
	索隱曰漢書作喋音跕下牒翻陳湯杜業皆言喋血無盟㰱事廣雅曰喋履也予據類篇啑字有色甲色洽二翻既從啑字音義當與㰱同若從喋字則有履之義公羊傳曰京大也師衆也天子之居必以衆大之詞言之}
此以迎大王為名實不可信願大王稱疾母往以觀其變中尉宋昌進曰羣臣之議皆非也夫秦失其政諸侯豪傑並起人人自以為得之者以萬數然卒踐天子之位者劉氏也|{
	卒子恤翻下同}
天下絶望一矣高帝封王子弟地犬牙相制此所謂磐石之宗也|{
	師古曰言地形如犬之牙交而相入也石大而下平磐據地面不可得而移動故以為喻也王于况翻}
天下服其彊二矣漢興除秦苛政約法令施德惠人人自安難動揺三矣夫以呂太后之嚴立諸呂為三王擅權專制然而太尉以一節入北軍一呼|{
	呼火故翻}
士皆左袒為劉氏叛諸呂卒以滅之此乃天授非人力也今大臣雖欲為變百姓弗為使|{
	為于偽翻使如字}
其黨寧能專一邪方今内有朱虚東牟之親外畏吳楚淮陽琅邪齊代之彊|{
	淮陽史記作淮南當從之}
方今高帝子獨淮南王與大王大王又長|{
	長知兩翻}
賢聖仁孝聞於天下|{
	聞音問}
故大臣因天下之心而欲迎立大王大王勿疑也代王報太后計之猶豫未定卜之兆得大横|{
	應劭曰龜曰兆筮曰卦卜者以荆灼龜文正横也}
占曰大横庚庚余為天王夏啟以光|{
	服䖍曰庚庚横貌李奇曰庚庚其繇文也占謂其繇也張晏曰先是五帝官天下老則嬗賢至夏啟始傳嗣能光先君之業文帝亦襲父迹言似啟也師古曰繇丈救翻本作籕籕書也謂讀卜詞孔頴達曰兆者龜之舋坼繇者卜之文詞}
代王曰寡人固已為王矣又何王卜人曰所謂天王者乃天子也於是代王遣太后弟薄昭往見絳侯絳侯等具為昭言所以迎立王意|{
	為于偽翻}
薄昭還報曰信矣毋可疑者|{
	毋與無通}
代王乃笑謂宋昌曰果如公言乃命宋昌參乘|{
	師古曰戎事則稱車右其餘則曰參乘參者三也盖取三人為義乘繩證翻}
張武等六人乘傳從詣長安至高陵休止|{
	傳株戀翻班志高陵縣屬左馮翊括地志高陵故城在雍州高陵縣西一里從才用翻}
而使宋昌先馳之長安觀變昌至渭橋|{
	蘇林曰渭橋在長安北三里索隱曰咸陽宫在渭北興樂宫在渭南秦昭王通兩宫之間作渭橋長三百八十步關中記云石柱以北屬扶風石柱以南屬京兆}
丞相以下皆迎昌還報代王馳至渭橋羣臣拜謁稱臣代王下車答拜太尉勃進曰願請間|{
	包愷曰間音閑言欲向空閑處師古曰間容也猶今言中間也請容暇之頃當有所陳不欲於衆中顯論也他皆類此}
宋昌曰所言公公言之所言私王者無私太尉乃跪上天子璽符|{
	上時掌翻}
代王謝曰至代邸而議之後九月己酉晦代王至長安舍代邸羣臣從至邸丞相陳平等皆再拜言曰子弘等皆非孝惠子不當奉宗廟大王高帝長子宜為嗣|{
	長知兩翻}
願大王即天子位代王西鄉讓者三南鄉讓者再|{
	如淳曰讓羣臣也或曰賓主位東西面君臣位南北面故西鄉坐三讓不受羣臣猶稱宜乃更南鄉坐示變即君位之漸也余謂如說以代王南鄉坐為即君位之漸恐非代王所以再讓之意盖王入代邸而漢廷羣臣繼至王以賓主禮接之故西鄉羣臣勸進王凡三讓羣臣遂扶王正南面之位王又讓者再則南鄉非王之得已也羣臣扶之使南鄉耳遽以為南鄉坐可乎鄉讀曰嚮}
遂即天子位羣臣以禮次侍東牟侯興居曰誅呂氏臣無功請得除宫|{
	除宫清宫也應劭曰舊典天子行幸所至必遣靜室令先按行清淨殿中以備非常余謂此時羣臣雖奉帝即位而少帝猶居禁中盖有所屏除也}
乃與太僕汝隂侯滕公入宫前謂少帝曰足下非劉氏子不當立乃顧麾左右執戟者掊兵罷去|{
	掊芳遇翻類篇曰頓也}
有數人不肯去兵宦者令張釋諭吿亦去兵|{
	班表宦者令屬少府張釋即大謁者封建陵侯者釋本宦者故兼是官去羌呂翻}
滕公乃召乘輿車載少帝出|{
	康曰天子以天下為家不以宫室為常處當乘輿以行天下故託乘輿言余謂康說乘輿本不與古義相悖但此所謂乘輿車不當以此解之漢乘輿之制輪朱班重牙貳轂兩轄金薄繆龍為輿倚較文虎伏軾龍首銜軛左右吉陽筩鸞雀立衡文畫輈羽盖華蚤建大旗十二斿畫日月升龍駕六馬象鑣鏤錫金鍐方釳插翟尾朱兼繁纓赤罽易茸金就十有二左纛以犛牛尾為之在左騑馬軛上大如斗此即法駕文帝已立少帝安得乘此出宫乎沈約禮志云魏晉御小出多乘輿車輿車今之小輿滕公職為太僕與東牟侯除宫亦無緣召乘輿金根以載少帝意者此輿車盖天子常所乘輿車即魏晉間小輿也}
少帝曰欲將我安之乎滕公曰出就舍舍少府乃奉天子法駕迎代王於邸|{
	漢官儀天子鹵簿有大駕法駕小駕大駕公卿奉引大將軍驂乘屬車八十一乘法駕公卿不在鹵簿中惟京兆尹執金吾長安令奉引侍中驂乘屬車三十六乘蔡邕曰法駕乘金根車駕六馬有五時副車駕四馬侍中驂乘屬車三十六乘沈約禮志漢制乘輿金根車輪皆朱班重轂兩轄飛軨轂外復有轂施轄其外復設轄銅貫其中飛軨以赤油為之廣八寸長注地繫軸頭謂之飛軨金金薄繆龍為輿倚較較在箱上文畫藩藩箱也文虎伏軾鸞雀立衡文畫轅翠羽盖黄裏所謂黄屋也金華施橑末建太常十二斿畫日月升龍駕六黑馬施十二鸞金為义髦插以翟尾又加左纛所謂左纛輿也路如周玉路之制應劭漢官鹵簿圖乘輿大駕則御鳳凰車以金根為副又五色安車五色文車各五乘建龍旗駕四車施八鸞餘如金根之制猶周金路也車各如方色所謂五時副車白馬者朱其鬛安車者坐乘又有建華盖九重甘泉鹵簿者道車五乘斿車九乘在乘輿車前又有象車㝡在前試橋道宋明帝時建安王休仁議曰秦改周輅制為金根通以金薄周匝四面漢魏二晉因循莫改}
報曰宫謹除代王即夕入未央宫有謁者十人持戟衛端門|{
	郎謁者皆執戟以宿衛宫殿前所書少帝左右執戟者亦中郎郎中謁者之官也端門未央宫前殿之正南門也}
曰天子在也足下何為者而入代王乃謂太尉太尉往諭謁者十人皆掊兵而去代王遂入夜拜宋昌為衛將軍|{
	班表前後左右將軍皆周末官秦因之漢不常置蔡質漢儀漢興置大將軍驃騎將軍位次丞相車騎將軍衛將軍左右前後將軍皆金紫位次上卿余據大將軍始於灌嬰驃騎車騎左右前後將軍景武之後方有其官衛將軍則始置於此}
鎮撫南北軍以張武為郎中令行殿中|{
	行謂案行也行下更翻}
有司分部誅滅梁淮陽恒山王及少帝於邸|{
	分扶問翻}
文帝還坐前殿夜下詔書赦天下

太宗孝文皇帝上|{
	荀悦曰諱恒之字曰常高祖中子也母曰薄姬禮祖有功而宗有德漢之子孫以為功莫盛於高帝故為帝者太祖之廟德莫盛於文帝故為帝者太宗之廟自唐以來諸帝廟號莫不稱宗而此義泯矣諡法經緯天地曰文}


元年冬十月庚戌徙琅邪王澤為燕王封趙幽王子遂為趙王|{
	澤以呂后七年自營陵侯封琅邪王齊王起兵誅諸呂澤失國西至京師與大臣共立帝以功徙封燕王趙王友幽死於呂后七年徙梁王恢王趙恢尋以逼死以其國封呂禄禄誅乃復封友長子遂為趙王}
陳平謝病上問之平曰高祖時勃功不如臣及誅諸呂臣功亦不如勃願以右丞相讓勃十一月辛巳上徙平為左丞相太尉勃為右丞相大將軍灌嬰為太尉諸呂所奪齊楚故地皆復與之|{
	呂后封呂台為呂王得梁地奪齊楚之地以傅益之}
論誅諸呂功右丞相勃以下益戶賜金各有差絳侯朝罷趨出意得甚上禮之恭常目送之|{
	上禮勃恭甚其罷朝也常目送之待其既出然後肆體自如朝直遥翻下同}
郎中安陵袁盎諫曰|{
	安陵屬右扶風惠帝所起陵邑按姓譜轅袁爰三姓皆出陳轅濤塗之後按史記作爰盎漢書作袁盎則袁爰通也}
諸呂悖逆大臣相與共誅之|{
	悖蒲内翻}
是時丞相為太尉本兵柄適會其成功今丞相如有驕主色|{
	如似也}
陛下謙讓臣主失禮竊為陛下弗取也|{
	為于偽翻下同}
後朝上益莊丞相益畏十二月詔曰法者治之正也|{
	治直吏翻}
今犯法已論而使

無罪之父母妻子同產坐之及為收帑朕甚不取其除收帑諸相坐律令|{
	應劭曰帑子也秦法一人有罪并坐其室家今除此律帑音奴}
春正月有司請蚤建太子上曰朕既不德縱不能博求天下賢聖有德之人而禪天下焉而曰豫建太子是重吾不德也其安之|{
	師古曰重謂增益也安猶徐也言不宜汲汲耳重直用翻他皆類此}
有司曰豫建太子所以重宗廟社稷不忘天下也上曰楚王季父也吳王兄也淮南王弟也豈不豫哉今不選舉焉而曰必子人其以朕為忘賢有德者而專於子非所以優天下也有司固請曰古者殷周有國治安皆千餘歲用此道也|{
	師古曰所以能爾者以承嗣相傳故也治直吏翻}
立嗣必子所從來遠矣高帝平天下為太祖子孫繼嗣世世不絶今釋宜建|{
	釋舍也宜建謂嗣也}
而更選於諸侯及宗室非高帝之志也更議不宜|{
	師古曰不當更議}
子啟㝡長|{
	啟景帝名長知兩翻}
純厚慈仁請建以為太子上乃許之 三月立太子母竇氏為皇后|{
	春秋之法母以子貴風俗通夏帝相遭有窮氏之難其妃方娠逃出自竇而生少康其後氏焉}
皇后清河觀津人|{
	班志觀津縣屬信都國清河郡無觀津盖信都清河本皆趙也景帝二年為廣川國四年為信都郡而清河郡則高帝置此在未分置之前故繫之清河杜佑曰漢觀津縣在德州蓨縣東北}
有弟廣國字少君幼為人所略賣傳十餘家|{
	傳直戀翻}
聞竇后立乃上書自陳召見驗問得實乃厚賜田宅金錢與兄長君家於長安絳侯灌將軍等曰吾屬不死命乃且縣此兩人|{
	縣讀曰懸}
兩人所出微不可不為擇師傅賓客又復效呂氏大事也於是乃選士之有節行者與居|{
	為于偽翻行下孟翻}
竇長君少君由此為退讓君子不敢以尊貴驕人|{
	觀絳灌所以處二竇後世大臣以文義自持者其智識及此乎}
詔振貸鰥寡孤獨窮困之人|{
	師古曰振起也為給貸之令其存立也諸振救振贍其義皆同今流俗作字從具者非也自别有訓貸吐戴翻}
又令八十已上月賜米肉酒九十已上加賜帛絮賜物當稟鬻米者|{
	稟給也鬻讀曰粥之六翻糜也}
長吏閲視丞若尉致|{
	師古曰長吏縣之令長也若者豫及之辭致者送至也或丞或尉自致之也班表縣令長皆秦官掌治其縣萬戶以上為令秩千石至六百石減萬戶為長秩五百石至三百石皆有丞尉秩四百石至二百石是為長吏長知兩翻}
不滿九十嗇夫令史致|{
	漢制十里一亭十亭一鄉鄉有嗇夫職聽訟收賦税風俗通曰嗇者省也夫賦也言消息百姓均其賦役又漢制縣長吏百石以下有所謂斗食佐史漢官云斗食佐史即斗食令史}
二千石遣都吏循行不稱者督之|{
	蘇林曰取其都吏有德也如淳曰律說都吏今督郵是也閑惠曉事即為文無害都吏師古曰如說是其循行有不如詔意者二千石察視責罰之行下孟翻稱尺證翻}
楚元王交薨 夏四月齊楚地震二十九山同日崩大水潰出 時有獻千里馬者帝曰鸞旗在前|{
	劉昭志乘輿大駕法駕前驅有九斿雲䍐鳳凰闟戟皮軒鸞旗皆大夫載鸞旗者編羽毛列繫幢旁民或謂之雞翹非也胡廣曰鸞旗以銅作鸞鳥車衡上與本志不同晉志曰鸞旗車駕四馬先輅所載也}
屬車在後|{
	漢制大駕屬車八十一乘備千乘萬騎劉昭曰古者諸侯貳車九乘秦滅六國兼其車服古大駕屬車八十一乘法駕半之沈約曰屬車皆皁盖黄裏師古曰屬之欲翻}
吉行日五十里師行三十里朕乘千里馬獨先安之於是還其馬與道里費而下詔曰朕不受獻也其令四方毋求來獻 帝既施惠天下諸侯四夷遠近驩洽乃脩代來功封宋昌為壯武侯|{
	班志壯武屬膠東國括地志壯武故城在萊州即墨縣西六十里古萊夷之國}
帝益明習國家事朝而問右丞相勃曰天下一歲決獄幾何|{
	朝直遥翻}
勃謝不知又問一歲錢穀入幾何勃又謝不知惶愧汗出沾背上問左丞相平平曰有主者上曰主者謂誰曰陛下即問決獄責廷尉|{
	廷尉掌刑辟故決獄當問之}
問錢穀責治粟内史|{
	班表治粟内史秦官掌穀貨故錢穀出入當問之武帝太初元年改為大司農}
上曰苟各有主者而君所主者何事也平謝曰陛下不知其駑下|{
	師古曰駑凡馬之稱非駿者也故以自喻駑音奴}
使待罪宰相宰相者上佐天子理隂陽順四時下遂萬物之宜外鎮撫四夷諸侯内親附百姓使卿大夫各得任其職焉帝乃稱善右丞相大慚出而讓陳平曰君獨不素教我對陳平笑曰君居其位不知其任邪且陛下即問長安中盜賊數君欲彊對邪|{
	彊其兩翻}
於是絳侯自知其能不如平遠矣居頃之人或說勃曰君既誅諸呂立代王威震天下而君受厚賞處尊位|{
	說式芮翻處昌呂翻}
久之即禍及身矣勃亦自危乃謝病請歸相印上許之秋八月辛未右丞相勃免左丞相平專為丞相 初隆慮侯竈擊南越|{
	事見高后七年}
會暑濕士卒大疫兵不能隃領|{
	師古曰隃與踰同}
歲餘高后崩即罷兵趙佗因此以兵威財物賂遺閩越西甌駱役屬焉|{
	遺于季翻下同劉昫曰唐黨州古西甌所居也漢屬鬱林郡界駱越也唐貴州鬱平縣古西甌駱越所居漢為鬱林廣鬱縣地又潘州亦西甌駱越地漢合浦郡地也又高州茂名縣及鬱林軍亦古西甌之地宋白曰秦象林郡皆西甌地師古曰西甌者即駱越也言西者以别東甌也廣州記曰交趾有駱田仰潮水上下人食其田名為駱侯諸縣自名為駱將銅印青綬即今之令後蜀王子將兵討駱侯自稱為安陽王治封溪縣南越王尉佗攻破安陽王令二使典主交趾九真二郡即甌駱也}
東西萬餘里乘黄屋左纛稱制與中國侔帝乃為佗親冢在真定者置守邑|{
	為于偽翻}
歲時奉祀召其昆弟尊官厚賜寵之復使陸賈使南越|{
	復扶又翻}
賜佗書曰朕高皇帝側室之子也|{
	師古曰言非正嫡所生}
棄外奉北藩於代道里遼遠壅蔽樸愚未嘗致書高皇帝棄羣臣孝惠皇帝即世高后自臨事不幸有疾諸呂為變賴功臣之力誅之已畢朕以王侯吏不釋之故|{
	孟康曰辭讓帝位不見置也}
不得不立今即位乃者聞王遺將軍隆慮侯書求親昆弟請罷長沙兩將軍|{
	佗真定人親昆弟皆在真定故來求之呂后七年佗反攻長沙故遣兩將軍屯於長沙以備之遺于季翻}
朕以王書罷將軍博陽侯|{
	博陽齊地高祖功臣表有博陽侯陳濞盖於此時為將軍也索隱曰博陽縣在汝南}
親昆弟在真定者已遣人存問脩治先人冢|{
	治直之翻}
前日聞王發兵於邊為寇災不止當其時長沙苦之南郡尤甚雖王之國庸獨利乎|{
	師古曰言越兵寇邊長沙南郡皆厭苦之而漢兵亦當拒戰其於越亦非利也}
必多殺士卒傷良將吏寡人之妻孤人之子獨人父母得一亡十朕不忍為也朕欲定地犬牙相入者以問吏吏曰高皇帝所以介長沙土也|{
	介隔也}
朕不得擅變焉今得王之地不足以為大得王之財不足以為富服領以南|{
	蘇林曰山領名也如淳曰長沙南界予謂服領者自五嶺以南荒服之外因以稱之}
王自治之雖然王之號為帝兩帝並立亡一乘之使以通其道|{
	亡與無同乘繩證翻}
是爭也爭而不讓仁者不為也願與王分棄前惡|{
	師古曰彼此共棄故曰分}
終今以來通使如故|{
	師古曰從今通使至於終久故曰終今以來也}
賈至南越南越王恐頓首謝罪願奉明詔長為藩臣奉貢職於是下令國中曰吾聞兩雄不俱立兩賢不並世漢皇帝賢天子自今以來去帝制黄屋左纛|{
	去羌呂翻}
因為書稱蠻夷大長老夫臣佗昧死再拜上書皇帝陛下|{
	長知兩翻下同}
曰老夫故越吏也高皇帝幸賜臣佗璽以為南越王孝惠皇帝即位義不忍絶所以賜老夫者厚甚高后用事别異蠻夷|{
	别彼列翻}
出令曰毋與蠻夷越金鐵田器馬牛羊|{
	以越為蠻夷故曰蠻夷越}
即予予牡毋予牝|{
	予讀曰與牡雄也牝雌也恐其蕃息故不予牝}
老夫處僻馬牛羊齒已長|{
	師古曰齒已長謂老也處冒呂翻下同}
自以祭祀不脩有死罪使内史藩中尉高御史平凡三輩上書謝過皆不反又風聞老夫父母墳墓已壞削兄弟宗族已誅論|{
	師古曰風聞謂風聲傳聞也誅論者以罪論死也壞音怪}
吏相與議曰今内不得振於漢|{
	言為漢所貶削不得振起也}
外亡以自高異|{
	亡讀曰無}
故更號為帝自帝其國非敢有害於天下|{
	更工衡翻}
高皇后聞之大怒削去南越之籍使使不通|{
	去羌呂翻使使上如字下疏吏翻}
老夫竊疑長沙王讒臣故發兵以伐其邊老夫處越四十九年于今抱孫焉然夙興夜寐寢不安席食不甘味目不視靡曼之色|{
	張揖曰靡細也曼澤也}
耳不聽鐘鼓之音者以不得事漢也今陛下幸哀憐復故號通使漢如故老夫死骨不腐改號不敢為帝矣 齊哀王襄薨|{
	諡法恭仁短折曰哀}
上聞河南守吳公治平為天下第一|{
	守式又翻治直吏翻}
召以為廷尉吳公薦洛陽人賈誼|{
	班志洛陽縣屬河南郡}
帝召以為博士|{
	班表博士秦官掌通古今秩比六百石員多至數十人武帝建元五年初置五經博士宣帝黄龍元年增員十二人屬奉常}
是時賈生年二十餘帝愛其辭博|{
	言其贍於文辭而博識也}
一歲中超遷至太中大夫|{
	班表太中大夫掌論議無員多至數十人秩比千石屬郎中令}
賈生請改正朔易服色定官名興禮樂以立漢制更秦法|{
	正朔謂夏建寅為人正商建丑為地正周建子為天正秦之建亥非三統也而漢因之此當改也周以火德王色尚赤漢繼周者也以土繼火色宜尚黄此當易也唐虞官百夏商官倍周官則備矣六卿各率其屬凡三百六十秦立百官之職名漢因循而不革此當定也高祖之時叔孫通采秦儀以制朝廷之禮因秦樂人以作宗廟之樂此當興也誼之說雖未為盡醇而其志則可尚矣}
帝謙讓未遑也

二年冬十月曲逆獻侯陳平薨 詔列侯各之國為吏及詔所止者遣太子|{
	李奇曰為吏謂為卿大夫者詔所止謂特以恩愛見留余謂當時如周勃者是也}
十一月乙亥周勃復為丞相 癸卯晦日有食之詔羣臣悉思朕之過失及知見之所不及匄以啟告朕|{
	匄音丐乞也}
及舉賢良方正能直言極諫者|{
	賢良方正之舉昉此}
以匡朕之不逮因各敕以職任務省繇費以便民|{
	省所景翻減也繇讀曰徭役也}
罷衛將軍|{
	按班紀詔曰朕既不能遠德故然念外人之有非是以設備未息今縱不能罷邊屯戍又飭兵厚衛其罷衛將軍軍通鑑傳寫逸一軍字耳}
太僕見馬遺財足餘皆以給傳置|{
	班表太僕掌輿馬見馬見在之馬也遺留也財與纔同少也僅也言減見在之馬所留財足充事而已置者置傳驛之所因名置也傳張戀翻}
潁隂侯騎賈山|{
	潁隂侯灌嬰也騎者盖在侯家為騎從也}
上書言治亂之道曰臣聞雷霆之所擊無不摧折者|{
	折而設翻}
萬鈞之所壓無不糜滅者今人主之威非特雷霆也埶重非特萬鈞也開道而求諫和顔色而受之用其言而顯其身士猶恐懼而不敢自盡又况於縱欲恣暴惡聞其過乎|{
	惡烏路翻}
震之以威壓之以重雖有堯舜之智孟賁之勇豈有不摧折者哉|{
	賁音奔}
如此則人主不得聞其過社稷危矣昔者周盖千八百國|{
	周爵五等而土三等公侯百里伯七十里子男五十里不滿為附庸九州州方千里八州州二百一十國天子之縣内九十三國凡九州千七百七十三國曰千八百國者舉成數也}
以九州之民|{
	周改禹貢徐梁二州合之於青雍分冀州之地以為幽并職方氏所掌曰揚州荆州豫州青州兖州雍州幽州冀州并州}
養千八百國之君君有餘財民有餘力而頌聲作|{
	頌者美盛德之形容也}
秦皇帝以千八百國之民自養力罷不能勝其役財盡不能勝其求|{
	罷讀曰疲勝音升}
一君之身耳所自養者馳騁弋獵之娛|{
	弋羊職翻繳射也}
天下弗能供也秦皇帝計其功德度其後嗣世世無窮|{
	度徒洛翻}
然身死纔數月耳天下四面而攻之宗廟滅絶矣秦皇帝居滅絶之中而不自知者何也天下莫敢告也其所以莫敢告者何也亡養老之義亡輔弼之臣|{
	亡古無字通}
退誹謗之人殺直諫之士是以道諛媮合苟容|{
	道讀曰導言為諂諛導迎主意納之於邪也媮與偷同}
比其德則賢於堯舜課其功則賢於湯武天下已潰|{
	師古曰潰水旁決也言天下已壞如水之潰也}
而莫之告也今陛下使天下舉賢良方正之士天下皆訢訢焉|{
	訢讀曰欣}
曰將興堯舜之道三王之功矣天下之士莫不精白以承休德|{
	師古曰厲精而為潔白也}
今方正之士皆在朝廷矣又選其賢者使為常侍諸吏|{
	班表左右曹諸吏散騎常侍中常侍皆加官}
與之馳驅射獵一日再三出臣恐朝廷之解弛百官之墮於事也|{
	解讀曰懈弛式氏翻放也墮與惰同}
陛下即位親自勉以厚天下節用愛民平獄緩刑天下莫不說喜|{
	說讀曰悦}
臣聞山東吏布詔令民雖老羸癃疾|{
	癃音隆病也老也疲病也}
扶杖而往聽之願少須臾毋死思見德化之成也|{
	少詩沼翻}
今功業方就名聞方昭|{
	聞音問後以義推}
四方鄉風|{
	鄉讀曰嚮}
而從豪俊之臣方正之士直與之日日射獵擊兔伐狐以傷大業絶天下之望臣竊悼之古者大臣不得與宴游|{
	師古曰安息曰宴與讀曰豫}
使皆務其方而高其節|{
	師古曰方道也一方謂廉隅也}
則羣臣莫敢不正身脩行|{
	行下孟翻}
盡心以稱大禮夫士脩之於家而壞之於天子之廷|{
	稱尺證翻壞音怪}
臣竊愍之陛下與衆臣宴游與大臣方正朝廷論議游不失樂朝不失禮|{
	樂音洛朝直遥翻下同}
軌事之大者也|{
	師古曰軌謂法度也軌居洧翻}
上嘉納其言上每朝郎從官上書疏|{
	從才用翻}
未嘗不止輦受其言言不可用置之言可用采之未嘗不稱善帝從霸陵上|{
	班志霸陵縣屬京兆故芷陽也帝起陵邑因更名}
欲西馳下峻阪中郎將袁盎騎並車擥轡|{
	並蒲浪翻擥與攬同魯敢翻}
上曰將軍怯邪盎曰臣聞千金之子坐不垂堂|{
	師古曰言富人之子則自愛也垂堂謂坐堂外邊恐墜墮也}
聖主不乘危不徼幸|{
	徼工堯翻}
今陛下騁六飛馳下峻山有如馬驚車敗陛下縱自輕奈高廟太后何上乃止上所幸慎夫人在禁中常與皇后同席坐及坐郎署袁盎引郤慎夫人坐|{
	蘇林曰郎署上林中直衛之署也如淳曰盎時為中郎將天子幸署豫設供張待之故得引郤慎夫人坐也坐徂卧翻慎姓也古有慎到}
慎夫人怒不肯坐上亦怒起入禁中盎因前說曰臣聞尊卑有序則上下和今陛下既已立后慎夫人乃妾妾主豈可與同坐哉且陛下幸之即厚賜之陛下所以為慎夫人適所以禍之也|{
	為于偽翻}
陛下獨不見人彘乎|{
	人彘事見上卷惠帝元年}
於是上乃說|{
	說讀曰悦}
召語慎夫人|{
	語牛倨翻}
慎夫人賜盎金五十斤 賈誼說上曰管子曰|{
	管子管仲之書}
倉廩實而知禮節衣食足而知榮辱民不足而可治者自古及今未之嘗聞|{
	治直之翻下同}
古之人曰一夫不耕或受之飢一女不織或受之寒生之有時而用之亡度|{
	亡古無字通}
則物力必屈|{
	屈其勿翻盡也下大屈同}
古之治天下至纎至悉故其畜積足恃今背本而趨末者甚衆是天下之大殘也|{
	師古曰本農業也末工商也言人棄農業而務工商者甚衆殘謂傷害天下也背蒲妹翻}
淫侈之俗日日以長|{
	長知兩翻}
是天下之大賊也殘賊公行莫之或止大命將泛莫之振救|{
	孟康曰泛方勇翻覆也師古曰字本作覂此通用振舉也}
生之者甚少而靡之者甚多|{
	靡讀曰糜散也}
天下財產何得不蹷|{
	蹷音厥傾竭也}
漢之為漢幾四十年矣|{
	幾居衣翻}
公私之積猶可哀痛失時不雨民且狼顧|{
	鄭氏曰民欲有畔意若狼之顧望也李奇曰狼性怯走喜還顧言民見天不雨心亦恐也師古曰李說是}
歲惡不入請賣爵子|{
	如淳曰賣爵級又賣子也余謂請賣爵子猶言請爵賣子也入粟得以拜爵故曰請爵富者有粟以儌上之急至於請爵貧者無以自活至於賣子}
既聞耳矣|{
	如淳曰聞於天子之耳}
安有為天下阽危者若是而上不驚者|{
	阽危欲墜之意阽音閻又丁念翻}
世之有饑穰天之行也|{
	李奇曰天之行氣不能常熟也或曰行道也師古曰穰豐也人羊翻}
禹湯被之矣|{
	被皮義翻}
即不幸有方二三千里之旱國胡以相恤卒然邊境有急|{
	卒讀曰猝}
數十百萬之衆國胡以餽之兵旱相乘天下大屈有勇力者聚徒而衡擊|{
	衡讀曰横}
罷夫羸老易子齩其骨|{
	罷讀曰疲齩五巧翻齧也}
政治未畢通也|{
	治直吏翻}
遠方之能僭擬者並舉而爭起矣乃駭而圖之豈將有及乎夫積貯者天下之大命也|{
	貯丁呂翻}
苟粟多而財有餘何為而不成以攻則取以守則固以戰則勝懷敵附遠何招而不至今民而歸之農皆著於本|{
	與驅同著直略翻}
使天下各食其力末技游食之民轉而緣南畮則畜積足而人樂其所矣|{
	樂音洛}
可以為富安天下而直為此廩廩也|{
	廩與凜同廩廩危懼之意師古曰言務耕農厚畜積則天下富安何乃不為而常不足直廩廩若此也}
竊為陛下惜之|{
	為于偽翻}
上感誼言春正月丁亥詔開藉田上親耕以率天下之民|{
	應劭曰古者天子耕藉田千畝為天下先藉者典藉之常也韋昭日藉借也借民以治之以奉宗廟且以勸率天下使務農也臣瓚曰景帝詔曰朕親耕后親桑為天下先本以躬親為義不得以假借為稱也藉謂蹈藉也師古曰國語云宣王即位不藉千畝虢文公諫則藉非假借明矣瓚說是也陸德明經典釋文藉在亦翻}
三月有司請立皇子為諸侯王詔先立趙幽王少子

辟彊為河間王|{
	師古曰辟彊言辟禦彊梁亦猶辟兵辟非耳辟必亦翻彊其良翻一說辟讀曰闢彊讀曰疆闢疆言開土地也賈誼書曰衛侯朝於周周行人問其名衛侯曰辟彊行人還之曰啟彊辟彊天子之號也諸侯弗得用更其名曰燬其義兩說並通他皆倣此河間本屬趙國元年以幽王子遂為趙王至是又分河間以王遂之弟辟彊}
朱虚侯章為城陽王東牟侯興居為濟北王|{
	城陽濟北本皆屬齊今分以王章興居二人皆悼惠王子濟子禮翻}
然後立皇子武為代王參為太原王揖為梁王 五月詔曰古之治天下朝有進善之旌|{
	應劭曰旌幡也堯設之五達之道令民進善也如淳曰欲有進者於旌下言之}
誹謗之木|{
	服䖍曰堯作之橋梁交午柱頭也應劭曰橋梁邊版所以書政治之愆失也至秦去之今乃復施也索隱曰尸子云堯立誹謗之木誹音非音沸韋昭曰慮政有闕失使書於木此堯時然也後代因以為飾今宫外橋頭四柱木是鄭玄注禮云一縱一横為午謂以木貫表柱四出即今之華表崔浩以為木貫柱四出名桓陳楚俗桓聲近和又云和表則華又與和相訛也}
所以通治道而來諫者也今法有誹謗訞言之罪|{
	師古曰高后元年詔除妖言令今猶有妖言罪則是中間重設此條訞與妖同}
是使衆臣不敢盡情而上無由聞過失也將何以來遠方之賢良其除之 九月詔曰農天下之大本也民所恃以生也而民或不務本而事末故生不遂朕憂其然故今兹親率羣臣農以勸之其賜天下民今年田租之半 燕敬王澤薨|{
	諡法合善典法曰敬}


資治通鑑卷十三
















































































































































