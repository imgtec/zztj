\section{資治通鑑卷一百五十三}
宋 司馬光 撰

胡三省 音註

梁紀九|{
	屠維作噩一年}


高祖武皇帝九

中大通元年|{
	是年十月方改元}
春正月甲寅魏于暉所部都督彭樂帥二千餘騎叛犇韓樓暉引還|{
	不敢復進軍討邢杲帥讀曰率騎奇寄翻}
辛酉上祀南郊大赦 甲子魏汝南王悦求還國許之|{
	悦來奔見上卷上年}
辛巳上祀明堂 二月甲午魏主尊彭城武宣王為文穆皇帝廟號肅祖母李妃為文穆皇后將遷神主於太廟以高祖為伯考大司馬兼録尚書臨淮王彧表諫以為漢高祖立太上皇帝於香街|{
	香街在漢長安故城内左馮翊府東北}
光武祀南頓君於舂陵|{
	事見四十三卷建武十九年}
元帝之於光武已疏絶服|{
	服至袒免則無服謂之絶服}
猶身奉子道入繼大宗|{
	漢元帝以太宗則上距景帝五世以祖孫世數數之則上距景帝七世光武上接景帝亦七世五服之次親盡無服而光武中興以赤劉之九之符繼元帝為九世而别為舂陵節侯以下立四親廟於舂陵}
高祖德洽寰中道超無外肅祖雖勲格宇宙猶北面為臣又二后皆將配饗乃是君臣並筵嫂叔同室竊謂不可吏部尚書李神儁亦諫不聽彧又請去帝著皇|{
	請去帝著皇亦引漢悼皇共皇為據去羌呂翻著則畧翻}
亦不聽 詔更定二百四十號將軍為四十四班|{
	天監七年定將軍為二十四班是年有司奏移寧遠將軍班中明威將軍進輕車班中以輕車班中征遠度入寧遠班中又置安遠將軍代貞武宣遠代明烈其戎夷之號亦加附擬選序則依此承用遂以定制轉則進一班黜則退一班班即階也同班以優劣為前後有鎮衛驃騎車騎同班四中四征同班八安同班四平四翊同班忠武軍師同班武臣爪牙龍騎雲麾冠軍同班鎮兵翊師宣毅宣惠四將軍東西南北四中郎將同班智威仁威勇威信威嚴威同班智武仁武勇武信武嚴武同班謂為五德將軍輕車振朔武旅貞毅明威同班寧遠安遠征遠振遠宣遠同班威雄威猛威烈威振威信威勝威略威風威力威光同班武猛武略武勝武力武毅武健武烈武威武鋭武勇同班猛毅猛烈猛威猛鋭猛震猛進猛智猛略猛勝猛駿同班壯武壯勇壯烈壯猛壯鋭壯盛壯毅壯志壯意壯力同班驍雄驍桀驍猛驍烈驍武驍勇驍鋭驍名驍勝驍迅同班雄猛雄威雄明雄烈雄信雄武雄勇雄毅雄壯雄健同班忠勇忠烈忠猛忠鋭忠壯忠毅忠勇忠信忠義忠勝同班明智明略明遠明勇明烈明威明勝明進明鋭明毅同班光烈光明光英光遠光勝光鋭光命光勇光戎光野同班飆勇飆猛飆烈飆鋭飆奇飆决飆起飆略飆勝飆出同班龍驤虎視雲旗風烈電威雷音馳鋭追鋒羽騎突騎同班開遠略遠貞威决勝清野堅鋭輕鋭拔山雲勇振旅同班超武鐵騎樓船宣猛樹功克敵平虜稜威昭威威戎同班伏波雄戟長劒衝冠雕騎佽飛勇騎破敵克敵威虜同班前鋒武毅開邊招遠全威破陣蕩寇殄虜横野馳射同班牙門期門同班侯騎熊渠同班中堅典戎同班執訊行陣同班伏武懷奇同班偏禆將軍同班凡二百四十號為四十四班}
壬寅魏詔濟隂王暉業兼行臺尚書|{
	濟子禮翻 考異曰梁書作徽業今從魏書}
都督丘大千等鎮梁國暉業小新成之曾孫也|{
	小新成見一百二十九卷宋孝武帝大明五年}
三月壬戍魏詔上黨王天穆討邢杲以費穆為前鋒大都督夏四月癸未魏遷肅祖及文穆皇后神主于太廟又

追尊彭城王劭為孝宣皇帝臨淮王彧諫曰兹事古所未有|{
	言自古未有以皇帝追尊其兄者今按自唐高宗以後率多追謚其子弟為皇帝作俑者魏敬宗也}
陛下作而不法後世何觀|{
	用左傳曹劌語意}
弗聽 魏元天穆將擊邢杲以北海王顥方入寇集文武議之衆皆曰杲衆彊盛宜以為先行臺尚書薛琡曰|{
	琡昌六翻}
邢杲兵衆雖多鼠竊狗偷非有遠志顥帝室近親|{
	顥北海王詳之子於魏主從兄弟也}
來稱義舉其勢難測宜先去之|{
	去羌呂翻}
天穆以諸將多欲擊杲又魏朝亦以顥為孤弱不足慮|{
	將即亮翻朝直遥翻}
命天穆等先定齊地還師擊顥遂引兵東出顥與陳慶之乘虛自銍城進拔滎城遂至梁國|{
	水經注春秋沙隨之地杜預注以為即梁國寧陵縣北之沙陽亭俗謂之堂城滎堂字相近意即此地而字訛也銍陟栗翻}
魏丘大千有衆七萬分築九城以拒之慶之攻之自旦至申拔其三壘大千請降|{
	降戶江翻下同}
顥登壇燔燎即帝位於睢陽城南改元孝基|{
	睢音雖 考異曰魏帝紀去年十月蕭衍以顥為魏主號年孝基入據銍城顥傳永安二年四月于梁國城南登壇燔燎年號孝基今從之}
濟隂王暉業帥羽林兵二萬軍考城|{
	前漢梁國有甾縣後漢章帝更名考城屬陳留郡晉省宋屬濟陽郡五代志曰梁郡考城縣後魏曰考陽置北梁郡隋復為考城縣属宋州帥讀曰率}
慶之攻拔其城禽暉業 |{
	考異曰魏書帝紀克考城在辛丑後今從梁帝紀}
辛丑魏上黨王天穆及爾朱兆破邢杲於濟南|{
	濟子禮翻}
杲降送洛陽斬之兆榮之從子也|{
	從才用翻}
五月丁巳魏以東南道大都督楊昱鎮滎陽尚書僕射爾朱世隆鎮虎牢侍中爾朱世承鎮崿岅|{
	崿五各翻岅與坂同音反}
乙丑内外戒嚴戊辰北海王顥克梁國顥以陳慶之為衛將軍徐州刺史引兵而西|{
	引兵而西直指洛陽}
楊昱擁衆七萬據滎陽慶之攻之未拔顥遣人說昱使降昱不從|{
	說式芮翻}
天穆與驃騎將軍爾朱吐沒兒將大軍前後繼至|{
	驃匹妙翻騎奇寄翻將即亮翻下同}
梁士卒皆恐慶之解鞍秣馬諭將士曰吾至此以來屠城略地實為不少|{
	少詩沼翻}
君等殺人父兄掠人子女亦無筭矣天穆之衆皆是仇讐我輩衆纔七千虜衆三十餘萬今日之事唯有必死乃可得生耳虜騎多不可與之野戰當及其未盡至急攻取其城而據之諸君勿或狐疑自取屠膾乃鼓之使登城將士即相帥蟻附而入|{
	帥讀曰率}
癸酉拔滎陽執楊昱|{
	楊昱輕慶之兵少不料其肉薄急攻故城陷傳曰敵無小不可輕也又曰不備不虞不可以師}
諸將三百餘人伏顥帳前請曰陛下渡江三千里無遺鏃之費昨滎陽城下一朝殺傷五百餘人願乞楊昱以快衆意顥曰我在江東聞梁主言初舉兵下都袁昂為吳郡不降每稱其忠節|{
	事見一百四十四卷齊和帝中興元年降戶江翻}
楊昱忠臣奈何殺之此外唯卿等所取於是斬昱所部統帥三十七人皆刳其心而食之|{
	帥所類翻}
俄而天穆等引兵圍城慶之帥騎三千背城力戰大破之|{
	帥讀曰率背蒲妹翻}
天穆吐沒兒皆走慶之進擊虎牢爾朱世隆弃城走獲魏東中郎將辛纂|{
	魏東中郎府在虎牢}
魏主將出避顥未知所之或勸之長安中書舍人高道穆曰關中荒殘何可復往|{
	復扶又翻}
顥士衆不多乘虛深入由將帥不得其人故能至此陛下親帥宿衛|{
	將即亮翻帥所類翻親帥讀曰率}
高募重賞背城一戰臣等竭其死力破顥孤軍必矣或恐勝負難期則車駕不若渡河徵大將軍天穆大丞相榮各使引兵來會犄角進討|{
	犄居蟻翻}
旬月之間必見成功此萬全之策也魏主從之甲戍魏主北行夜至河内郡北|{
	河内郡治野王魏主自洛北如河内當夜至郡城南不應至郡城北恐誤}
命高道穆於燭下作詔書數十紙布告遠近於是四方始知魏主所在乙亥魏主入河内臨淮王彧安豐王延明帥百僚封府庫備法駕迎顥 |{
	考異曰彧傳無迎顥事而梁陳慶之北齊宋遊道傳有之蓋魏史為彧諱也}
丙子顥入洛陽宫改元建武大赦以陳慶之為侍中車騎大將軍增邑萬戶楊椿在洛陽椿弟順為冀州刺史兄子侃為北中郎將從魏主在河北顥意忌椿而以其家世顯重恐失人望未敢誅也|{
	楊播楊椿兄弟仕魏一門貴盛子姪通顯累朝榮赫侃播之子也}
或勸椿出亡椿曰吾内外百口何所逃匿正當坐待天命耳顥後軍都督侯暄守睢陽為後援|{
	睢陽即梁國睢音雖}
魏行臺崔孝芬大都督刁宣馳往圍暄晝夜急攻戊寅暄突走禽斬之上黨王天穆等帥衆四萬攻拔大梁|{
	大梁即陳留浚儀縣}
分遣費穆將兵二萬攻虎牢顥使陳慶之擊之天穆畏顥將北渡河謂行臺郎中濟隂温子昇曰卿欲向洛為隨我北渡|{
	天穆開兩端以問子昇濟子禮翻}
子昇曰主上以虎牢失守|{
	守式又翻}
致此狼狽元顥新入人情未安今往擊之無不克者大王平定京邑奉迎大駕此桓文之舉也捨此北渡竊為大王惜之|{
	為于偽翻}
天穆善之而不能用遂引兵渡河費穆攻虎牢將拔聞天穆北渡自以無後繼遂降於慶之|{
	降戶江翻}
慶之進擊大梁梁國皆下之|{
	睢陽即梁國下遐稼翻}
慶之以數千之衆自發銍縣至洛陽凡取三十二城四十七戰所向皆克顥使黄門郎祖瑩作書遺魏主曰|{
	遺于季翻}
朕泣請梁朝誓在復耻正欲問罪於爾朱出卿於桎梏|{
	朝直遥翻桎之日翻梏苦沃翻}
卿託命豺狼委身虎口假獲民地本是榮物固非卿有|{
	顥言爾朱榮擅命顥所得一民尺地皆爾朱榮之物非魏主之有}
今國家隆替在卿與我若天道助順則皇魏再興脱或不然在榮為福於卿為禍卿宜三復|{
	三蘇暫翻復扶又翻}
富貴可保顥既入洛自河以南州郡多附之齊州刺史沛郡王欣集文武議所從曰北海長樂俱帝室近親|{
	顥北海王詳之子魏主彭城王勰之子同出於顯祖樂音洛}
今宗祏不移|{
	杜預曰宗祏今廟中藏主石室也祏音石}
我欲受赦諸君意何如在坐莫不失色|{
	坐徂臥翻}
軍司崔光韶獨抗言曰元顥受制於梁引寇讐之兵以覆宗國此魏之亂臣賊子也豈唯大王家事所宜切齒下官等皆受朝眷|{
	朝眷謂朝廷恩眷也朝直遥翻}
未敢仰從長史崔景茂等皆曰軍司議是欣乃斬顥使|{
	使疏吏翻下同}
光韶亮之從父弟也|{
	崔亮貴顯於延昌熙平之間從才用翻}
于是襄州刺史賈思同|{
	魏孝昌中置襄州領襄城舞隂南安期城宣義建城等郡治赭陽}
廣州刺史鄭先護|{
	魏主置廣州治魯陽領南陽順陽定陵魯陽汝南漢廣襄城郡}
南兖州刺史元暹|{
	魏正光中置南兖州冶譙城領陳留梁郡下蔡譙郡北梁郡沛郡馬頭郡}
亦不受顥命思同思伯之弟也|{
	賈思伯見一百四十九卷普通四年}
顥以冀州刺史元孚為東道行臺彭城郡王孚封送其書於魏主平陽王敬先起兵於河橋以討顥不克而死魏以侍中車騎將軍尚書右僕射爾朱世隆為使持節行臺僕射大將軍相州刺史鎮鄴城|{
	相息亮翻}
魏主之出也單騎而去侍衛後宫皆案堵如故顥一旦得之號令已出四方人情想其風政而顥自謂天授遽有驕怠之志宿㫺賓客近習咸見寵待千擾政事日夜縱酒不恤軍國所從南兵陵暴市里朝野失望高道穆兄子儒自洛陽出從魏主魏主問洛中事子儒曰顥敗在旦夕不足憂也爾朱榮聞魏主北出即時馳傳見魏主於長子行且部分魏主即日南還|{
	爾朱榮既至魏主有所倚以攻顥故即日南還傳張戀翻分扶問翻}
榮為前驅旬日之間兵衆大集資粮器仗相繼而至六月壬午魏大赦榮既南下幷肆不安乃以爾朱天光為幷肆等九州行臺|{
	九州幷肆恒朔雲蔚顯汾晉也}
仍行幷州事天光至晉陽部分約勒所部皆安己丑費穆至洛陽顥引入責以河隂之事而殺之|{
	費穆勸爾朱榮殺王公事見上卷}
顥使都督宗正珍孫與河内太守元襲據河内爾朱榮攻之上黨王天穆引兵會之壬寅拔其城斬珍孫及襲辛亥魏淮隂太守晉鴻以湖陽來降|{
	五代志舂陵郡湖陽縣後魏置西淮安郡及南襄州淮隂當作淮安}
閏月己未南康簡王績卒 魏北海王顥既得志密與臨淮王彧安豐王延明謀叛梁以事難未平|{
	難乃旦翻}
藉陳慶之兵力故外同内異言多猜忌慶之亦密為之備說顥曰|{
	說式芮翻}
今遠來至此未服者尚多彼若知吾虛實連兵四合將何以禦之宜啟天子|{
	天子謂梁武帝}
更請精兵幷敕諸州有南人沒此者悉須部送顥欲從之延明曰慶之兵不出數千已自難制今更增其衆寧肯復為人用乎|{
	復扶又翻下更復敢復遽復顥復時復彧復同}
大權一去動息由人魏之宗廟於斯墜矣顥乃不用慶之言又慮慶之密啓|{
	慮慶之密啓其事於上}
乃表於上曰今河北河南一時克定唯爾朱榮尚敢跋扈臣與慶之自能禽討州郡新服正須綏撫不宜更復加兵揺動百姓上乃詔諸軍繼進者皆停於境上|{
	陳慶之非爾朱榮敵也是時梁之諸將又皆出慶之下使相與繼進至洛與元顥互相猜阻亦必同歸于陷沒梁兵之不進梁之幸也武帝不務自治而務遠略所以有侯景之禍}
洛中南兵不滿一萬而羌胡之衆十倍軍副馬佛念為慶之曰|{
	凡一軍有主有副為于偽翻為慶之謀而言蜀本為作謂}
將軍威行河洛聲震中原功高埶重為魏所疑一旦變生不測可無慮乎不若乘其無備殺顥據洛此千載一時也慶之不從|{
	馬佛念有戰國策士之氣然必有非常之才然後可以行非常之事陳慶之烏足以辦此載子亥翻}
顥先以慶之為徐州刺史因固求之鎮顥心憚之不遣曰主上以洛陽之地全相任委忽聞捨此朝寄|{
	朝寄謂魏朝所寄託也朝直遥翻}
欲往彭城謂君遽取富貴不為國計|{
	此國計謂為梁國計}
非徒有損於君恐僕并受其責慶之不敢復言爾朱榮與顥相持於河上慶之守北中城顥自據南岸|{
	河橋南岸也}
慶之三日十一戰殺傷甚衆有夏州義士為顥守河中渚|{
	夏戶雅翻水經注曰河中渚上有河平侯祠河之南岸有一碑題曰洛陽北界意此中渚即唐時河陽之中潬城也為于偽翻}
隂與榮通謀求破橋立効榮引兵赴之及橋破榮應接不逮顥悉屠之榮帳然失望又以安豐王延明緣河固守而北軍無船可渡議欲還北更圖後舉黄門郎楊侃曰大王并州之日已知夏州義士之謀指來應之乎為欲廣施經略匡復帝室乎夫用兵者何嘗不散而更合瘡愈更戰况今未有所損豈可以一事不諧而衆謀頓廢乎今四方顒顒|{
	顒顒仰望也顒魚容翻}
視公此舉若未有所成遽復引歸民情失望各懷去就勝負所在未可知也不若徵發民材多為桴筏|{
	編竹木以渡水大者曰桴小者曰筏桴方無翻筏音伐}
閒以舟楫|{
	閒古莧翻下閒行同}
緣河布列數百里中皆為渡勢首尾既遠使顥不知所防一旦得渡必立大功高道穆曰今乘輿飄蕩主憂臣辱|{
	乘繩證翻}
大王擁百萬之衆輔天子而令諸侯若分兵造筏所在散渡指掌可克奈何捨之北歸使顥復得完聚徵兵天下此所謂養虺成蛇悔無及矣|{
	逸書曰為虺不摧為蛇奈何以文義觀之盖以虺為小蛇}
榮曰楊黄門已陳此策當相與議之劉靈助言於榮曰不出十日河南必平伏波將軍正平楊檦|{
	魏以聞喜曲沃二縣置正平郡隋廢郡為正平縣今絳州治所檦與標同}
與其族居馬渚自言有小船數艘求為鄉導|{
	艘蘇遭翻鄉讀曰嚮}
戊辰榮命車騎將軍爾朱兆與大都督賀拔勝縛材為筏自馬渚西硤石夜渡|{
	五代志河南熊耳縣有後魏崤縣又有硤石山唐志陜州硤石縣本崤縣有硤石塢}
襲擊顥子領軍將軍冠受擒之安豐王延明之衆聞之大潰顥失據帥麾下數百騎南走|{
	帥讀曰率騎奇寄翻下同}
慶之收步騎數千結陳東還|{
	陳讀曰陣}
顥所得諸城一時復降於魏爾朱榮自追陳慶之會嵩高水漲|{
	潁水出少室山五渡水出太室山入于潁水嵩高水漲指此水也}
慶之軍士死散略盡乃削須髪為沙門閒行出汝隂還建康|{
	慶之所以得免者亦由嵩高水漲追兵不急於軍士死散之時得以挺身逸去否則必為爾朱榮所擒矣}
猶以功除右衛將軍封永興縣侯|{
	句斷五代志會稽郡會稽縣舊有永興縣}
中軍大都督兼領軍大將軍楊津入宿殿中掃洒宫庭|{
	掃素報翻洒所賣翻又並上聲}
封閉府庫出迎魏主於北邙流涕謝罪帝慰勞之|{
	勞力到翻}
庚午帝入居蕐林園大赦以爾朱兆為車騎大將軍儀同三司|{
	賞硤石之功也}
北來軍士及隨駕文武諸立義者加五級河北報事之官及河南立義者加二級|{
	報事謂報敵情曲折者}
壬申加大丞相榮天柱大將軍增封通前二十萬戶|{
	天柱前無此號魏主以爾朱榮功高特置以寵之榮先以平葛榮之功增封至十萬戶今又增為二十萬戶以賞之}
北海王顥自轘轅南出至臨潁|{
	臨潁縣自漢以來屬潁川郡轘音環}
從騎分散|{
	從才用翻騎奇寄翻}
臨潁縣卒江豐斬之癸酉傳首洛陽臨淮王彧復自歸于魏主安豐王延明攜妻子來奔陳慶之之入洛也蕭贊送啓求還|{
	贊即豫章王綜也奔魏事見一百五十卷普通六年}
時吳淑媛尚在|{
	媛于眷翻}
上使以贊幼時衣寄之信未逹而慶之敗慶之自魏還特重北人朱异怪而問之|{
	异羊至翻}
慶之曰吾始以為大江以北皆戎狄之鄉比至洛陽|{
	比必利翻及也}
乃知衣冠人物盡在中原非江東所及也奈何輕之|{
	陳慶之特有見於洛陽華靡之俗而為是言耳}
甲戍魏以上黨王天穆為太宰城陽王徽為大司馬兼太尉乙亥魏主宴勞爾朱榮上黨王天穆及北來督將於都亭|{
	勞力到翻將即亮翻}
出宫人三百繒錦雜綵數萬匹班賜有差|{
	繒慈陵翻}
凡受元顥爵賞階復者悉追奪之|{
	復方目翻復除賦役也}
秋七月辛巳魏主始入宫以高道穆為御史中尉帝姊壽陽公主行犯清路赤棒卒呵之不止|{
	中尉前驅之卒執赤棒即清路者也棒部項翻呵虎何翻}
道穆令卒擊破其軍公主泣訢於帝帝曰高中尉清直之士彼所行者公事豈可以私責之也道穆見帝|{
	見賢遍翻}
帝曰家姊行路相犯極以為愧道穆免冠謝帝曰朕以愧卿卿何謝也於是魏多細錢米斗幾直一千|{
	幾巨依翻又居希翻}
高道穆上表以為在市銅價八十一錢得銅一斤私造薄錢斤贏二百|{
	言銅一斤造薄錢二百而贏也}
既示之以深利又隨之以重刑抵罪雖多姦鑄彌衆今錢徒有五銖之名而無二銖之實置之水上殆欲不沈|{
	沈持林翻}
此乃因循有漸科防不切朝廷失之彼復何罪|{
	復扶又翻下况復同}
宜改鑄大錢文載年號以記其始則一斤所成止七十錢計私鑄所費不能自潤直置無利自應息心况復嚴刑廣設也|{
	言置私鑄直使無利自應息心而不為况又廣設科禁有嚴刑之可畏邪}
金紫光禄大夫楊侃亦奏乞聽民與官並鑄五銖錢使民樂為而弊自改|{
	樂音洛}
魏主從之始鑄永安五銖錢 辛卯魏以車騎將軍楊津為司空 初魏以梁益二州境土荒遠更立巴州以統諸獠凡二十餘萬戶|{
	獠曾皓翻}
以巴酋嚴始欣為刺史又立隆城鎮|{
	宋白曰取其連岡地埶高隆為名後為隆州}
以始欣族子愷為鎮將|{
	將即亮翻}
始欣貪暴孝昌初諸獠反圍州城行臺魏子建撫諭之乃散始欣恐獲罪隂來請降帝遣使以詔書鐵劵衣冠等賜之|{
	降戶江翻使疏吏翻}
為愷所獲以送子建子建奏以隆城鎮為南梁州|{
	五代志巴西郡舊置南梁州西魏典畧曰此州舊有隆城故又謂之南隆治古閬中城今之閬中即其地}
用愷為刺史囚始欣於南鄭魏以唐永為東益州刺史代子建以梁州刺史傅豎眼為行臺|{
	豎而庾翻}
子建去東益而氐蜀尋反|{
	氐蜀氐人與蜀人也}
唐永弃城走東益州遂沒|{
	魏置東益州於武興時為氐蜀所攻沒梁不能有也}
傳豎眼之初至梁州也州人相賀|{
	事見一百四十有八卷天監十五年}
既而久病不能親政事其子敬紹奢淫貪暴州人患之嚴始欣重賂敬紹得還巴州遂舉兵擊嚴愷滅之以巴州來降|{
	降戶江翻}
帝遣將軍蕭玩等援之傅敬紹見魏室方亂隂有保據南鄭之志使其妻兄唐崑崙於外扇誘山民相與圍城欲為内應|{
	崙盧崑翻誘音酉}
圍合而謀洩城中將士共執敬紹以白豎眼而殺之豎眼耻恚而卒|{
	恚於避翻卒子恤翻}
八月己未魏以太傅李延寔為司徒甲戍侍中太保楊椿致仕 九月癸巳上幸同泰寺設無遮大會上釋御服持法衣行清淨大捨以便省為房|{
	便省在同泰寺上臨幸時居之故曰便省}
素牀瓦器乘小車私人執役甲子升講堂法座為四部大衆開湼槃經題|{
	四部大衆僧尼及善男子善女人也湼奴結翻}
癸卯羣臣以錢一億萬祈白三寶|{
	釋書以佛陀耶衆為佛寶逹摩耶衆為法寶僧迦耶衆為僧寶}
奉贖皇帝菩薩|{
	釋典曰菩普也薩濟也菩薩言能普濟衆生菩薄胡翻薩桑葛翻}
僧衆默許乙巳百辟詣寺東門奉表請還臨宸極|{
	唐韵曰宸屋宇也天子所居毛晃曰帝居北辰之宫故從宀從辰}
三請乃許上三答書前後並稱頓首 魏爾朱榮使大都督尖山侯淵|{
	按五代志後魏置神武郡於桑乾水上領尖山殊頹二縣}
討韓樓於薊配卒甚少騎止七百|{
	薊音計少詩沼翻騎奇寄翻}
或以為言榮曰侯淵臨機設變是其所長若摠大衆未必能用今以此衆擊此賊必能取之淵遂廣張軍聲多設供具親帥數百騎深入樓境去薊百餘里值賊帥陳周馬步萬餘|{
	親帥讀曰率賊帥所類翻}
淵潜伏以乘其背大破之虜其卒五千餘人尋還其馬伏縱令入城左右諫曰既獲賊衆何為復資遣之|{
	馬仗馬及兵仗也復扶又翻}
淵曰我兵既少不可力戰須為奇計以離間之乃可克也淵度其已至|{
	間古莧翻度徒洛翻}
遂帥騎夜進昧旦叩其城門韓樓果疑降卒為淵内應遂走追禽之幽州平以淵為平州刺史鎮范陽|{
	魏平州本治肥如今徙鎮幽州之范陽}
先是魏使征東將軍劉靈助兼尚書左僕射慰勞幽州流民於濮陽頓丘|{
	先悉薦翻勞力到翻}
因帥流民北還與侯淵共滅韓樓仍以靈助行幽州事加車騎將軍又為幽平營安四州行臺|{
	為劉靈助以營州叛爾朱張本}
万俟醜奴攻魏東秦州拔之殺刺史高子朗|{
	五代志上郡後魏置東秦州後改為北秦州西魏改為敷州隋大業二年改為敷城郡後改為上郡唐為鄜州洛交縣万莫北翻俟渠之翻}
冬十月己酉上又設四部無遮大會道俗五萬餘人會畢上御金輅還宫御太極殿大赦改元|{
	改中大通元年}
魏以前司空蕭贊為司徒十一月己卯就德興請降於魏營州平|{
	就德興反見一百五十卷普通二年降戶江翻下同}
丙午魏以城陽王徽為太保丹楊王蕭贊為太尉雍州刺史長孫稚為司徒|{
	雍於用翻}
十二月辛亥兖州刺史張景邕荆州刺史李靈起雄信將軍蕭進明叛降魏|{
	三人者皆梁魏境上民豪以刺史將軍寵授之耳}
以陳慶之為北兖州刺史|{
	此北兖州當治淮隂}
有妖賊僧彊自稱天子土豪蔡伯龍起兵應之衆至三萬攻陷北徐州|{
	此北徐州治鍾離妖於驕翻}
慶之討斬之 魏以岐州刺史王羆行南秦州事羆誘捕州境羣盗悉誅之|{
	誘音酉}


資治通鑑卷一百五十三
