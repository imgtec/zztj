<!DOCTYPE html PUBLIC "-//W3C//DTD XHTML 1.0 Transitional//EN" "http://www.w3.org/TR/xhtml1/DTD/xhtml1-transitional.dtd">
<html xmlns="http://www.w3.org/1999/xhtml">
<head>
<meta http-equiv="Content-Type" content="text/html; charset=utf-8" />
<meta http-equiv="X-UA-Compatible" content="IE=Edge,chrome=1">
<title>資治通鑒_90-資治通鑑卷八十九_90-資治通鑑卷八十九</title>
<meta name="Keywords" content="資治通鑒_90-資治通鑑卷八十九_90-資治通鑑卷八十九">
<meta name="Description" content="資治通鑒_90-資治通鑑卷八十九_90-資治通鑑卷八十九">
<meta http-equiv="Cache-Control" content="no-transform" />
<meta http-equiv="Cache-Control" content="no-siteapp" />
<link href="/img/style.css" rel="stylesheet" type="text/css" />
<script src="/img/m.js?2020"></script> 
</head>
<body>
 <div class="ClassNavi">
<a  href="/24shi/">二十四史</a> | <a href="/SiKuQuanShu/">四库全书</a> | <a href="http://www.guoxuedashi.com/gjtsjc/"><font  color="#FF0000">古今图书集成</font></a> | <a href="/renwu/">历史人物</a> | <a href="/ShuoWenJieZi/"><font  color="#FF0000">说文解字</a></font> | <a href="/chengyu/">成语词典</a> | <a  target="_blank"  href="http://www.guoxuedashi.com/jgwhj/"><font  color="#FF0000">甲骨文合集</font></a> | <a href="/yzjwjc/"><font  color="#FF0000">殷周金文集成</font></a> | <a href="/xiangxingzi/"><font color="#0000FF">象形字典</font></a> | <a href="/13jing/"><font  color="#FF0000">十三经索引</font></a> | <a href="/zixing/"><font  color="#FF0000">字体转换器</font></a> | <a href="/zidian/xz/"><font color="#0000FF">篆书识别</font></a> | <a href="/jinfanyi/">近义反义词</a> | <a href="/duilian/">对联大全</a> | <a href="/jiapu/"><font  color="#0000FF">家谱族谱查询</font></a> | <a href="http://www.guoxuemi.com/hafo/" target="_blank" ><font color="#FF0000">哈佛古籍</font></a> 
</div>

 <!-- 头部导航开始 -->
<div class="w1180 head clearfix">
  <div class="head_logo l"><a title="国学大师官网" href="http://www.guoxuedashi.com" target="_blank"></a></div>
  <div class="head_sr l">
  <div id="head1">
  
  <a href="http://www.guoxuedashi.com/zidian/bujian/" target="_blank" ><img src="http://www.guoxuedashi.com/img/top1.gif" width="88" height="60" border="0" title="部件查字,支持20万汉字"></a>


<a href="http://www.guoxuedashi.com/help/yingpan.php" target="_blank"><img src="http://www.guoxuedashi.com/img/top230.gif" width="600" height="62" border="0" ></a>


  </div>
  <div id="head3"><a href="javascript:" onClick="javascript:window.external.AddFavorite(window.location.href,document.title);">添加收藏</a>
  <br><a href="/help/setie.php">搜索引擎</a>
  <br><a href="/help/zanzhu.php">赞助本站</a></div>
  <div id="head2">
 <a href="http://www.guoxuemi.com/" target="_blank"><img src="http://www.guoxuedashi.com/img/guoxuemi.gif" width="95" height="62" border="0" style="margin-left:2px;" title="国学迷"></a>
  

  </div>
</div>
  <div class="clear"></div>
  <div class="head_nav">
  <p><a href="/">首页</a> | <a href="/ShuKu/">国学书库</a> | <a href="/guji/">影印古籍</a> | <a href="/shici/">诗词宝典</a> | <a   href="/SiKuQuanShu/gxjx.php">精选</a> <b>|</b> <a href="/zidian/">汉语字典</a> | <a href="/hydcd/">汉语词典</a> | <a href="http://www.guoxuedashi.com/zidian/bujian/"><font  color="#CC0066">部件查字</font></a> | <a href="http://www.sfds.cn/"><font  color="#CC0066">书法大师</font></a> | <a href="/jgwhj/">甲骨文</a> <b>|</b> <a href="/b/4/"><font  color="#CC0066">解密</font></a> | <a href="/renwu/">历史人物</a> | <a href="/diangu/">历史典故</a> | <a href="/xingshi/">姓氏</a> | <a href="/minzu/">民族</a> <b>|</b> <a href="/mz/"><font  color="#CC0066">世界名著</font></a> | <a href="/download/">软件下载</a>
</p>
<p><a href="/b/"><font  color="#CC0066">历史</font></a> | <a href="http://skqs.guoxuedashi.com/" target="_blank">四库全书</a> |  <a href="http://www.guoxuedashi.com/search/" target="_blank"><font  color="#CC0066">全文检索</font></a> | <a href="http://www.guoxuedashi.com/shumu/">古籍书目</a> | <a   href="/24shi/">正史</a> <b>|</b> <a href="/chengyu/">成语词典</a> | <a href="/kangxi/" title="康熙字典">康熙字典</a> | <a href="/ShuoWenJieZi/">说文解字</a> | <a href="/zixing/yanbian/">字形演变</a> | <a href="/yzjwjc/">金 文</a> <b>|</b>  <a href="/shijian/nian-hao/">年号</a> | <a href="/diming/">历史地名</a> | <a href="/shijian/">历史事件</a> | <a href="/guanzhi/">官职</a> | <a href="/lishi/">知识</a> <b>|</b> <a href="/zhongyi/">中医中药</a> | <a href="http://www.guoxuedashi.com/forum/">留言反馈</a>
</p>
  </div>
</div>
<!-- 头部导航END --> 
<!-- 内容区开始 --> 
<div class="w1180 clearfix">
  <div class="info l">
   
<div class="clearfix" style="background:#f5faff;">
<script src='http://www.guoxuedashi.com/img/headersou.js'></script>

</div>
  <div class="info_tree"><a href="http://www.guoxuedashi.com">首页</a> > <a href="/SiKuQuanShu/fanti/">四库全书</a>
 > <h1>资治通鉴</h1> <!--         下载:【右键另存为】即可 --></div>
  <div class="info_content zj clearfix">
  
<div class="info_txt clearfix" id="show">
<center style="font-size:24px;">90-資治通鑑卷八十九</center>
    資治通鑑卷八十九   宋 司馬光 撰<br />
<br />
  胡三省 音註<br />
<br />
  晉紀十一【起閼達閹茂盡柔兆困敦凡三年】<br />
<br />
  孝愍皇帝下<br />
<br />
  建興二年春正月辛未有如日隕于地又有三日相承出西方而東行【天文占曰三四五六日俱出並爭天下兵作又曰三日並出不過三旬諸侯爭為帝】 丁丑大赦 有流星出牽牛入紫微【晉天文志牽牛六星在河鼓南】光燭地墜于平陽北化為肉長三十步廣二十七步【長直亮翻廣右曠翻後倣此】漢主聰惡之【惡烏路翻】以問公卿陳元達以為女寵太盛亡國之徵 【考異曰載記元達等曰臣恐後庭有三后之事按立三后在明年於時未也】聰曰此隂陽之理何關人事聰后劉氏賢明聰所為不道劉氏每規正之己丑劉氏卒諡曰武宣自是嬖寵競進後宫無序矣【嬖卑義翻又博計翻】 聰置丞相等七公【七公見下自晉王粲至中山王曜是也】又置輔漢等十六大將軍【輔漢都護中軍上軍撫軍鎮衛京前後左右上下軍輔國冠軍龍驤虎牙等大將軍】各配兵二千以諸子為之又置左右司隸各領戶二十餘萬萬戶置一内史單于左右輔各主六夷十萬落【六夷蓋胡羯鮮卑氐羌巴蠻或曰烏丸非巴蠻也單音蟬】萬落置一都尉左右選曹尚書並典選舉自司隸以下六官皆位亞僕射以其子粲為丞相領大將軍録尚書事進封晉王江都王延年録尚書六條事【録尚書六條事始見於此沈約志曰晉康帝世何充讓録表云咸康中分置三録王導録其一荀崧陸曄各録六條事然則似有二十四條若止有十二條則荀陸各録六條導又何所司乎若導總録荀陸分掌則不得復云導録其一也其後每置二録輒云各録六條事又似止有十二條十二條者不知悉何條也江右張華江左庾亮並經關尚書七條則亦不知皆何事也余按宋元嘉以後江夏王義恭始興王濬南譙王義宣皆録尚書六條事沈氏世仕江左歷位通顯且不知為何事後之人何所取徵杜佑曰何充讓録表曰咸康中分置三録王導録其一荀松陸曄各録二條事晉氏度江有吏部祠部左民五兵度支五尚書是五條也晉初有吏部三公客曹駕部屯田度支六曹太康有吏部殿中五兵田曹度支左民六曹蓋六條也如杜佑之言則六條蓋六曹也沈約以何充表各録二條為各録六條致有此誤】汝隂王景為太師王育為太傅任顗為太保馬景為大司徒朱紀為大司空中山王曜為大司馬 壬辰王子春等及王浚使者至襄國石勒匿其勁卒精甲羸師虛府以示之【羸倫為翻】北面拜使者而受書浚遺勒麈尾【遺于季翻麈腫庾翻麋屬尾能生風辟蠅蜹晉王公貴人多執麈尾以玉為柄】勒陽不敢執懸之於壁朝夕拜之曰我不得見王公見其所賜如見公也復遣董肇奉表于浚期以三月中旬親詣幽州奉上尊號【復扶又翻上時掌翻】亦修牋于棗嵩求并州牧廣平公勒問浚之政事於王子春子春曰幽州去歲大水人不粒食【五穀不登故不粒食】浚積粟百萬不能賑贍刑政苛酷賦役殷煩忠賢内離夷狄外叛人皆知其將亡而浚意氣自若曾無懼心方更置立臺閣布列百官自謂漢高魏武不足比也勒撫几笑曰王彭祖真可擒也俊使者還薊【薊音計】具言石勒形埶寡弱欵誠無二浚大悦益驕怠不復設備 楊虎掠漢中吏民以奔成梁州人張咸等起兵逐楊難敵【楊虎楊難敵攻漢中事始上卷上年】難敵去咸以其地歸成於是漢嘉涪陵漢中之地【漢嘉本前漢青衣縣屬蜀郡後漢順帝陽嘉二年改曰漢嘉蜀分立漢嘉郡】皆為成有成主雄以李鳳為梁州刺史任囘為寧州刺史李恭為荆州刺史雄虛已好賢隨才授任【好呼到翻】命太傅驤養民於内李鳳等招懷於外刑政寛簡獄無滯囚興學校置史官【校戶教翻】其賦民男丁歲穀三斛女丁半之疾病又半之戶調絹不過數丈綿數兩【調徒釣翻賦也】事少役希【少詩沼翻】民多富實新附者皆給復除【復方目翻】是時天下大亂而蜀獨無事年穀屢熟乃至閭門不閉路不拾遺漢嘉夷王冲歸朱提審炤建寧爨畺皆歸之【朱提音銖時炤與照同爨取亂翻夷人姓也畺與疆同居良翻】巴郡嘗告急云有晉兵雄曰吾常憂琅邪微弱遂為石勒所滅以為耿耿【耿古幸翻耿耿憂也】不圖乃能舉兵使人欣然然雄朝無儀品爵位濫溢【朝直遥翻】吏無禄秩取給於民軍無部伍號令不肅此其所短也 二月壬寅以張軌為太尉涼州牧封西平郡公王浚為大司馬都督幽冀諸軍事荀組為司空領尚書左僕射兼司隸校尉行留臺事劉琨為大將軍都督并州諸軍事朝廷以張軌老病拜其子寔為副刺史【副刺史前此未有也】 石勒纂嚴將襲王浚而猶豫未發張賓曰夫襲人者當出其不意今軍嚴經日而不行豈非畏劉琨及鮮卑烏桓為吾後患乎勒曰然為之奈何賓曰彼三方智勇無及將軍者將軍雖遠出彼必不敢動且彼未謂將軍便能懸軍千里取幽州也輕軍往返不出二旬藉使彼雖有心比其謀議出師【比必寐翻】吾已還矣且劉琨王浚雖同名晉臣實為仇敵若修牋于琨送質請和【質音致下同】琨必喜我之服而快浚之亡終不救浚而襲我也用兵貴神速勿後時也勒曰吾所未了右侯已了之【了决也】吾復何疑【後扶又翻下敢復同】遂以火宵行至柏人【柏人縣屬趙國唐為邢州堯山縣】殺主簿游綸以其兄統在范陽恐泄軍謀故也遣使奉牋送質于劉琨自陳罪惡請討浚以自效琨大喜移檄州郡稱已與猗盧方議討勒勒走伏無地求抜幽都以贖罪今便當遣六修南襲平陽除僭偽之逆類降知死之逋羯【逆類謂劉聰逋羯謂石勒降戶江翻羯居謁翻】順天副民翼奉皇家斯乃曩年積誠靈祐之所致也【浚琨為勒所玩弄而不自覺宜其相繼而覆亡也 考異曰琨集檄首云三月庚午朔五月甲戌按石勒以壬申克幽州蓋時晉陽尚未知也欲叙琨事畢然後叙勒事故置此】三月勒軍達易水王浚督護孫緯馳遣白浚【緯于貴翻】將勒兵拒之游統禁之浚將佐皆曰胡貪而無信必有詭計請擊之浚怒曰石公來正欲奉戴我耳敢言擊者斬衆不敢復言浚設饗以待之壬申勒晨至薊【薊音計 考異曰三十國春秋先言癸酉勒取幽州後言壬午勒晨至薊按劉琨表曰勒以三月三日徑掩薊城然則當言壬午是也】叱門者開門猶疑有伏兵先驅牛羊數千頭聲言上禮【言欲以牛羊上浚以為禮上時掌翻】實欲塞諸街巷【塞悉則翻】浚始懼或坐或起勒既入城縱兵大掠浚左右請禦之浚猶不許勒升其聽事【中庭曰聽事言受事察訟於是漢晉皆作聽事六朝以來乃始加厂作廳並他經翻】浚乃走出堂皇【堂無四壁曰皇】勒衆執之勒召浚妻與之並坐執浚立於前浚罵曰胡奴調乃公【調田聊翻戲也】何凶逆如此勒曰公位冠元台【冠古玩翻】手握彊兵坐觀本朝傾覆【朝直遥翻下同】曾不救援乃欲自尊為天子非凶逆乎又委任姦貪殘虐百姓賊害忠良毒徧燕土【燕於賢翻】此誰之罪也使其將王洛生以五百騎送浚于襄國浚自投于水束而出之斬于襄國市勒殺浚麾下精兵萬人浚將佐爭詣軍門謝罪饋賂交錯前尚書裴憲從侍中郎荀綽獨不至勒召而讓之曰王浚暴虐孤討而誅之諸人皆來慶謝二君獨與之同惡將何以逃其戮乎對曰憲等世仕晉朝荷其榮禄【荷下可翻】浚雖凶麤猶是晉之藩臣故憲等從之【裴憲奔幽州見八十七卷懷帝永嘉五年】不敢有貳明公苟不修德義專事威刑則憲等死自其分【分扶問翻】又何逃乎請就死不拜而出勒召而謝之待以客禮綽朂之孫也勒數朱碩棗嵩等以納賄亂政為幽州患【事見上卷上年數所具翻】責游統以不忠所事皆斬之【以統欲以范陽私附之也】籍浚將佐親戚家貲皆至巨萬惟裴憲荀綽止有書百餘鹽米各十餘斛而已【與帙同直質翻書卷編次成帙】勒曰吾不喜得幽州喜得二子以憲為從事中郎綽為參軍分遣流民各還鄉里勒停薊二日焚浚宫殿以故尚書燕國劉翰行幽州刺史戍薊置守宰而還【還從宣翻又如字】孫緯遮擊之勒僅而得免勒至襄國遣使奉王浚首獻捷于漢漢以勒為大都督督陜東諸軍事【此陜東亦取分陜之義而授之耳】驃騎大將軍東單于【驃疋妙翻單音蟬】增封十二郡勒固辭受二郡而已劉琨請兵於拓跋猗盧以擊漢會猗盧所部雜胡萬餘家謀應石勒猗盧悉誅之不果赴琨約琨知石勒無降意乃大懼【降戶江翻】上表曰東北八州勒滅其七【勒入鄴殺都督東燕王騰寇信都殺冀州刺史王斌襲鄄城殺兖州刺史袁孚攻新蔡殺豫州刺史新蔡王確襲蒙城擒青州都督苟晞克上白斬青州刺史李惲攻信都殺冀州刺史王象攻定陵殺兖州刺史田徽襲幽州擒王浚除李惲田徽王浚承制所授是滅其七也】先朝所授存者惟臣【朝直遥翻】勒據襄國與臣隔山【山自太行恒山至于幽碣連延不斷襄國在山東晉陽在山西】朝發夕至城塢駭懼雖懷忠憤力不從願耳劉翰不欲從石勒乃歸段匹磾匹磾遂據薊城【磾丁奚翻】王浚從事中郎陽裕躭之兄子也逃犇令支【令支縣漢屬遼西故孤竹君之國晉省段氏據之為國都應劭曰令音鈴裴松之曰支其兒翻師古曰令又音郎定翻杜佑曰令支今北平郡盧龍縣即其地】依段疾陸眷會稽朱左車魯國孔纂泰山胡母翼自薊逃奔昌黎依慕容廆【會工外翻廆戶罪翻】是時中國流民歸廆者數萬家廆以冀州人為冀陽郡【據魏收地形志冀陽郡當置於漢北平平剛縣界】豫州人為成周郡【成周屬豫州之地故以為郡名】青州人為營丘郡【前漢志遼西臨渝縣有渝水首受白狼水南流逕營丘城西廆所置郡也】并州人為唐國郡【并州古唐國也廆因以名郡成周唐國二郡所置地闕】 初王浚以邵續為樂陵太守屯厭次【厭次本前漢平原郡之富平縣後漢明帝更名厭次晉分屬樂陵為治所丁度集韻厭於琰翻九域志曰相傳秦始皇東遊厭氣碣石次舍於此因以為名魏收曰樂陵郡厭次縣有富城邵續居之】浚敗續附於石勒勒以續子乂為督護浚所署勃海太守東萊劉胤棄郡依續謂續曰凡立大功必杖大義君晉之忠臣奈何從賊以自汙乎【汙烏故翻】會段匹磾以書邀續同歸左丞相睿續從之其人皆曰今棄勒歸匹磾其如乂何續泣曰我豈得顧子而為叛臣哉殺異議者數人勒聞之殺乂續遣劉胤使江東【使疏吏翻】睿以胤為參軍以續為平原太守石勒遣兵圍續匹磾使其弟文鴦救之勒引去襄國大饑穀二升直銀一斤肉一斤直銀一兩 杜弢將王真襲陶侃於林障【水經註材障在江夏沌陽縣沔水逕沌陽縣北又東逕林障故城北宋白曰晉江夏郡治林障義熙元年方徙夏口】侃犇灄中周訪救侃擊弢兵破之【灄書涉翻丁度曰灄水名在西陽水經註溳水過江夏安陸縣而東南流分為二水東通灄水西入于沔】 夏五月西平武穆公張軌寢疾遺令文武將佐務安百姓上思報國下以寧家己丑軌薨 【考異曰帝紀作壬辰今從前涼録鈔前涼録鈔又曰葬建陵蓋張祚僭號後追尊其墓耳】長史張璽等表世子寔攝父位【璽斯氏翻】 漢中山王曜趙染寇長安六月曜屯渭汭【春秋左氏傳曰虢公敗戎于渭汭杜預曰水之隈曲曰汭王肅曰汭入也呂忱曰汭者水相入也即渭水入河處汭儒稅翻】染屯新豐索綝將兵出拒之【索昔各翻綝丑林翻】染有輕綝之色長史魯徽曰晉之君臣自知強弱不敵將致死於我不可輕也染曰以司馬模之彊吾取之如拉朽【事見八十七卷懷帝永嘉五年拉落合翻】索綝小豎豈能汙吾馬蹄刀刃邪晨帥輕騎數百逆之【汙烏故翻帥讀曰率下同】曰要當獲綝而後食綝與戰于城西【新豐城西也】染兵敗而歸悔曰吾不用魯徽之言以至此何面目見之先命斬徽徽曰將軍愚愎以取敗乃復忌前害勝【復扶又翻下同愎弼力翻忌前忌人在前害勝害勝已者】誅忠良以逞忿猶有天地將軍其得死於枕席乎詔加索綝驃騎大將軍尚書左僕射録尚書承制行事【驃匹妙翻】曜染復與將軍殷凱帥衆數萬向長安 【考異曰晉春秋作段凱今從麴允傳】麴允逆戰於馮翊允敗收兵夜襲凱營凱敗死曜乃還攻河内太守郭默于懷列三屯圍之默食盡送妻子為質請糴於曜糴畢復嬰城固守曜怒沈默妻子于河而攻之【質音致沈持林翻】默欲投李矩於新鄭【新鄭縣漢屬河南郡晉省其地當在滎陽郡界周宣王弟鄭桓公本封京兆之鄭縣其子武公邑于虢鄫之間遂為鄭國左傳鄭莊公曰吾先君新邑于此後遂為新鄭縣以别京兆之鄭】矩使其甥郭誦迎之兵少不敢進【少詩沼翻】會劉琨遣參軍張肇帥鮮卑五百餘騎詣長安道阻不通還過矩營矩說肇使擊漢兵【說輸芮翻】漢兵望見鮮卑不戰而走默遂率衆歸矩漢主聰召曜還屯蒲坂秋趙染攻北地麴允拒之染中弩而死【中竹仲翻】 石勒始命州郡閱實戶口戶出帛二匹穀二斛 冬十月以張寔為都督涼州諸軍事涼州刺史西平公 十一月漢主聰以晉王粲為相國大單于摠百揆粲少有俊才自為宰相驕奢專恣遠賢親佞嚴刻愎諫國人始惡之【為後粲為靳準所弑張本少詩照翻遠于願翻惡烏路翻】 周勰以其父遺言【見上卷建興元年勰音協】因吳人之怨謀作亂使吳興功曹徐馥矯稱叔父丞相從事中郎札之命收合徒衆以討王導刁協豪傑翕然附之孫皓族人弼亦起兵於廣德以應之【沈約曰廣德縣疑是吳所立屬宣城郡按今廣德軍即其地宋白曰廣德縣本秦鄣郡地漢以為故鄣縣】<br />
<br />
  三年春正月徐馥殺吳興太守袁琇【琇音秀又音酉】有衆數千欲奉周札為主札聞之大驚以告義興太守孔侃勰知札意不同不敢發馥黨懼攻馥殺之孫弼亦死札子續亦聚衆應馥左丞相睿議發兵討之王導曰今少發兵則不足以平寇【少詩沼翻】多發兵則根本空虛續族弟黄門侍郎莚【莚夷然翻】忠果有謀請獨使莚往足以誅續睿從之莚晝夜兼行至郡將入遇續於門謂續曰當與君共詣孔府君有所論續不肯入莚牽逼與俱坐定莚謂孔侃曰府君何以置賊在坐【坐徂卧翻】續衣中常置刀即操刀逼莚【操千高翻】莚叱郡傳教吳曾格殺之【傳教郡吏也宣傳教令者也】莚因欲誅勰札不聽委罪於從兄邵而誅之莚不歸家省母【從才用翻省悉景翻】遂長驅而去母狼狽追之睿以札為吳興太守莚為太子右衛率以周氏吳之豪望故不窮治撫勰如舊【率如字治直之翻】 詔平東將軍宋哲屯華隂【華隂縣前漢屬京兆後漢晉屬弘農郡華戶化翻】 成王雄立后任氏 二月丙子以琅邪王睿為丞相大都督督中外諸軍事南陽王保為相國荀組為太尉領豫州牧劉琨為司空都督并冀幽三州諸軍事琨辭司空不受 南陽王模之敗也【見八十七卷懷帝永嘉五年】都尉陳安往歸世子保於秦州保命安將千餘人討叛羌寵待甚厚保將張春疾之譖安云有異志請除之保不許春輒伏刺客以刺安安被創馳還隴城【隴縣城也前漢屬天水後漢改天水為漢陽晉省以刺七亦翻被皮義翻創初良翻】遣使詣保貢獻不絶【使疏吏翻】 詔進拓跋猗盧爵為代王置官屬食代常山二郡【常山已為石勒所有拓跋氏建國曰代始此】猗盧請并州從事鴈門莫含於劉琨【姓譜莫姓楚莫敖之後】琨遣之含不欲行琨曰以并州單弱吾之不材而能自存於胡羯之間者代王之力也吾傾身竭貲以長子為質而奉之者庶幾為朝廷雪大恥也【琨以長子遵質於猗盧長知兩翻幾居希翻】卿欲為忠臣奈何惜共事之小誠而忘狥國之大節乎往事代王為之腹心乃一州之所賴也含遂行猗盧甚重之常與參大計猗盧用灋嚴國人犯灋者或舉部就誅老幼相攜而行人問何之曰往就死無一人敢逃匿者 王敦遣陶侃甘卓等討杜弢前後數十戰弢將士多死乃請降於丞相睿【弢土刀翻將即亮翻降戶江翻】睿不許弢遺南平太守應詹書【遺于季翻】自陳昔與詹共討樂鄉本同休戚後在湘中懼死求生遂相結聚【見八十七卷懷帝永嘉五年】儻以舊交之情為明枉直【為于偽翻下同】使得輸誠盟府【時琅邪王睿為東南方鎮盟主故曰盟府】厠列義徒或北清中原或西取李雄以贖前愆雖死之日猶生之年也詹為啟呈其書且言弢益州秀才【羅尚刺益州舉弢秀才為于偽翻】素有清望為鄉人所逼今悔惡歸善宜命使撫納以息江湘之民【使疏吏翻】睿乃使前南海太守王運受弢降赦其反逆之罪以弢為巴東監軍【監工銜翻】弢既受命諸將猶攻之不已弢不勝憤怒遂殺運復反【勝音升復扶又翻】遣其將杜弘張彦殺臨川内史謝摛【吳孫亮太平二年分豫章東部都尉立臨川郡摛丑之翻】遂陷豫章三月周訪擊彦斬之弘犇臨賀【臨賀縣漢屬蒼梧郡吴分立臨賀郡】 漢大赦改元建元 【考異曰十六國春秋建元元年在晉建興二年同編修劉恕言今晉州臨汾縣嘉泉村有漢太宰劉雄碑云嘉平五年歲在乙亥二月六日立然則改建元在乙亥二月後也】 雨血於漢東宫延明殿太弟乂惡之以問太傅崔瑋太保許遐瑋遐說乂曰【雨于具翻惡烏路翻說輸芮翻】主上往日以殿下為太弟者欲以安衆心耳其志在晉王久矣【聰子粲封晉王】王公已下莫不希旨附之今復以晉王為相國羽儀威重踰於東宫萬機之事無不由之諸王皆置營兵以為羽翼【事見上年】事勢已去殿下非徒不得立也朝夕且有不測之危不如早為之計今四衛精兵不減五千【謂東宫左右前後四衛率所統兵也】相國輕佻正煩一刺客耳【佻他彫翻】大將軍無日不出其營可襲而取【粲弟勃海王敷時為大將軍】餘王並幼固易奪也【易以䜴翻】苟殿下有意二萬精兵指顧可得鼓行入雲龍門宿衛之士孰不倒戈以迎殿下者大司馬不慮其為異也【大司馬謂中山王曜】乂弗從東宫舍人荀裕告瑋遐勸乂謀反【東宫舍人太子舍人之職又以太弟居東宫】漢主聰收瑋遐於詔獄假以他事殺之使冠威將軍卜抽將兵監守東宫【冠古玩翻監工銜翻】禁乂不聽朝會【朝直遥翻】乂憂懼不知所為上表乞為庶人并除諸子之封褒美晉王請以為嗣抽抑而弗通漢青州刺史曹嶷盡得齊魯間郡縣【嶷魚力翻】自鎮臨菑有衆十餘萬臨河置戌石勒表稱嶷有專據東方之志請討之漢主聰恐勒滅嶷不可復制【復扶又翻下同】弗許聰納中護軍靳凖二女月光月華【靳居惞翻】立月光為上皇后劉貴妃為左皇后月華為右皇后左司隸陳元達極諫【聰置左右司隸】以為並立三后非禮也聰不悅以元達為右光禄大夫外示優崇實奪其權於是太尉范隆等皆請以位讓元達聰乃復以元達為御史大夫儀同三司月光有穢行【行下孟翻】元達奏之聰不得已廢之月光慙恚自殺【恚於避翻】聰恨元達 夏四月大赦 六月盜發漢霸杜二陵及薄太后陵【漢薄太后葬南陵在霸陵之南】得金帛甚多詔收其餘以實内府 辛巳大赦 漢大司馬曜攻上黨八月癸亥敗劉琨之衆於襄垣曜欲進攻陽曲【襄垣縣屬上黨郡陽曲琨所居也敗補邁翻】漢主聰遣使謂之曰【使疏吏翻】長安未平宜以為先曜乃還屯蒲坂 陶侃與杜弢相攻弢使王貢出挑戰【挑徒了翻】侃遥謂之曰杜弢為益州小吏盜用庫錢父死不犇喪卿本佳人何為隨之天下寧有白頭賊邪【言為賊者不得至老】貢初横脚馬上聞侃言歛容下脚侃知可動復遣使諭之截髪為信貢遂降於侃【貢叛侃見上卷元年降戶江翻下同】弢衆潰遁走道死 【考異曰弢傳云弢逃遁不知所在晉春秋云城潰弢投水死今從帝紀】侃與南平太守應詹進克長沙【長沙杜弢之巢宂也】湘州悉平丞相睿承制赦其所部進王敦鎮東大將軍加都督江揚荆湘交廣六州諸軍事江州刺史敦始自選置刺史以下寖益驕横【横戶孟翻】初王如之降也【見上卷懷帝永嘉六年】敦從弟稜愛如驍勇【從才用翻驍堅堯翻】請敦配已麾下敦曰此輩險悍難畜【悍下罕翻又侯旰翻畜許六翻】汝性狷急【狷吉掾翻】不能容養更成禍端稜固請乃與之稜置左右甚加寵遇如數與敦諸將角射爭鬬【數所角翻】稜杖之如深以為恥及敦潛畜異志稜每諫之敦怒其異已密使人激如令殺稜如因閒宴請劒舞為歡稜許之如舞劒漸前稜惡而呵之【惡烏路翻】如直前殺稜敦聞之陽驚亦捕如誅之 初朝廷聞張光死【光死見上卷元年】以侍中第五猗為安南將軍 【考異曰周訪傳云征南大將軍今從杜曾傳】監荆梁益寧四州諸軍事荆州刺史【監工銜翻】自武關出杜曾迎猗於襄陽為兄子娶猗女【為于偽翻】遂聚兵萬人與猗分據漢沔陶侃既破杜弢乘勝進擊曾有輕曾之志司馬魯恬諫曰凡戰當先料其將今使君諸將無及曾者未易可逼也【易以豉翻】侃不從進圍曾於石城【水經註沔水南逕石城西城因山為固晉羊祜鎮荆州立晉惠帝元康九年分江夏西部都尉置竟陵郡治石城今郢州長夀縣即其地】曾軍多騎兵【騎奇寄翻】密開門突侃陳【陳讀曰陣】出其後反擊之侃兵死者數百人曾將趨順陽【趨七喻翻】下馬拜侃告辭而去時荀崧都督荆州江北諸軍事屯宛【江當作沔宛於元翻】曾引兵圍之崧兵少食盡【少詩沼翻】欲求救於故吏襄城太守石覽崧小女灌年十三帥勇士數十人【帥讀曰率下同】踰城突圍夜出且戰且前遂達覽所又為崧書求救於南中郎將周訪訪遣子撫帥兵三千與覽共救崧曾乃遁去曾復致牋於崧求討丹水賊以自效【丹水縣前漢屬弘農郡後漢屬南陽郡晉屬順陽郡賢曰丹水故城在今鄧州内鄉縣西南臨丹水復扶又翻下同】崧許之陶侃遺崧書曰杜曾凶狡所謂鴟梟食母之物【梟堅堯翻一曰流離爾雅作鶹陸璣草木疏曰梟也關西人謂之流離大則食其母爾雅有茅鴟今鴟鳩也似鷹而白怪䲭即䲭鵂也梟䲭土梟也孔穎達曰鴞惡聲之鳥一名鵬與梟一名䲭詩瞻卬云為鴟為梟是也俗說以為土梟非也陸璣疏云鴞大如班鳩緑色惡聲之鳥也入人家凶賈誼所謂服鳥是也其肉甚美可為羮又可為炙漢供御物各隨其時唯鴞冬夏尚施之以其美故也遺于季翻】此人不死州土未寧足下當識吾言【識職吏翻記也】崧以宛中兵少藉曾為外援不從曾復帥流亡二千餘人圍襄陽數日不克而還 王敦嬖人吳興錢鳳疾陶侃之功屢毁之【嬖卑義翻又博計翻】侃將還江陵欲詣敦自陳朱伺及安定皇甫方囘諫曰公入必不出侃不從既至敦留侃不遣左轉廣州刺史以其從弟丞相軍諮祭酒廙為荆州刺史【從才用翻廙逸職翻又羊至翻】荆州將吏鄭攀馬雋等詣敦上書留侃敦怒不許攀等以侃始滅大賊而更被黜【謂滅杜弢也】衆情憤惋【惋烏貫翻】又以廙忌戾難事遂帥其徒三千人屯溳口【水經注溳水出蔡陽縣東南過隨縣又南過江夏安陸縣又東南分為二水西入于沔者謂之溳口溳音云】西迎杜曾廙為攀等所襲犇于江安【江安縣屬南平郡武帝太康元年分孱陵置】杜曾與攀等北迎第五猗以拒廙廆督諸軍討曾復為曾所敗敦意攀承侃風旨被甲持矛將殺侃出而復還者數四【敗補邁翻被皮義翻復扶又翻】侃正色曰使君雄斷當裁天下【斷丁亂翻】何此不决乎因起如厠諮議參軍梅陶長史陳頒言於敦曰周訪與侃親姻如左右手【訪與侃結友以女妻侃子瞻】安有斷人左手而右手不應者乎【斷丁管翻】敦意解乃設盛饌以餞之【饌雛戀翻又雛晥翻】侃便夜發敦引其子瞻為參軍初交州刺史顧祕卒州人以祕子壽領州事帳下督梁碩起兵攻壽殺之碩遂專制交州王機自以盜據廣州【見上卷懷帝永嘉六年】恐王敦討之更求交州會杜弘詣機降【杜弘杜弢將也弢敗弘走降機降戶江翻下同】敦欲因機以討碩乃以降杜弘為機功轉交州刺史機至鬱林【鬱林秦桂林郡地漢武帝平南越更置鬱林郡唐潯州桂平縣古鬱林郡所治布山縣地也】碩迎前刺史修則子湛行州事以拒之機不得進乃更與杜弘及廣州將温邵交州秀才劉沈謀復還據廣州陶侃至始興【吴孫皓甘露元年分桂陽南部都尉立始興郡治漢曲江縣唐為韶州沈持林翻】州人皆言宜觀察形勢不可輕進侃不聽直至廣州【廣州治南海郡番禺縣】諸郡縣皆已迎機矣杜弘遣使偽降侃知其謀進擊弘破之遂執劉沈於小桂【秦置桂林郡漢武帝改曰鬱林郡治布山桂林為縣屬焉吴孫皓鳳凰三年分立桂林郡因謂桂林為小桂陶弘景曰始興桂陽縣即是小桂】遣督護許高討王機走之機病死于道高掘其尸斬之諸將皆請乘勝擊温邵侃笑曰吾威名已著何事遣兵但一函紙自定耳乃下書諭之邵懼而走追獲於始興杜弘詣王敦降廣州遂平侃在廣州無事輒朝運百甓於齋外【甓蒲歷翻瓴甋也】暮運於齋内人問其故答曰吾方致力中原過爾優逸恐不堪事故自勞耳王敦以杜弘為將寵任之 九月漢主聰使大鴻臚賜石勒弓矢策命勒為陜東伯【陜失冉翻】得專征伐拜刺史將軍守宰封列侯歲盡集上【上時掌翻集其所授官爵及其人之姓名而上之】 漢大司馬曜寇北地詔以麴允為大都督驃騎將軍以禦之【驃匹妙翻騎奇寄翻】冬十月以索綝為尚書僕射都督宫城諸軍事曜進拔馮翊太守梁肅犇萬年【秦櫟陽縣漢高祖改曰萬年屬馮翊晉屬京兆】曜轉寇上郡麴允去黄白城軍于靈武【漢北地郡之靈武縣也】以兵弱不敢進帝屢徵兵於丞相保保左右皆曰蝮虵螫手壯士斷腕【漢書齊王曰蝮蠚手則斬手蓋以為不如此則流毒於一身至於死也螫音釋斷丁管翻下同腕烏貫翻】今胡寇方盛且宜斷隴道以觀其變從事中郎裴詵曰今虵已螫頭頭可斷乎保乃以鎮軍將軍胡崧行前鋒都督須諸軍集乃發麴允欲奉帝往就保索綝曰保得天子必逞其私志乃止於是自長安以西不復貢奉朝廷【復扶又翻】百官饑乏採稆以自存【稆音呂禾自生曰稆】 涼州軍士張冰得璽文曰皇帝行璽獻於張寔僚屬皆賀寔曰是非人臣所得留遣使歸于長安【晉諸征鎮能知君臣之分者張氏父子而已璽斯氏翻】<br />
<br />
  四年春正月司徒梁芬議追尊吳王晏右僕射索綝等引魏明帝詔以為不可【魏明帝詔見七十一卷太和三年】乃贈太保諡曰孝 【考異曰本傳晏諡敬王今從愍帝紀】 漢中常侍王沈宣懷中宫僕射郭猗等【沈持林翻】皆寵幸用事漢主聰游宴後宫或三日不醒或百日不出自去冬不視朝政事一委相國粲唯殺生除拜乃使沈等入白之沈等多不白而自以其私意决之故勲舊或不叙而姦佞小人有數日至二千石者軍旅歲起將士無錢帛之賞而後宫之家賜及僮僕動至數千萬沈等車服第舍踰於諸王子弟中表為守令者三十餘人皆貪殘為民害【謂他姓與沈等子弟有中表親者沈持林翻守式又翻】靳凖闔宗諂事之郭猗與凖皆有怨於太弟义猗謂相國粲曰殿下光文帝之世孫主上之嫡子四海莫不屬心【屬之欲翻】奈何欲以天下與太弟乎且臣聞太弟與大將軍謀因三月上已大宴作亂事成許以主上為太上皇大將軍為皇太子又許衛軍為大單于【聰以子驥為大將軍子勱為衛大將軍皆粲弟也又按時以子敷為大將軍敷卒後乃以驥為之】三王處不疑之地【處昌呂翻】並握重兵以此舉事無不成者然三王貪一時之利不顧父兄事成之後主上豈有全理殿下兄弟固不待言東宫相國單于當在武陵兄弟何肯與人也【武陵兄弟當是乂之諸子相息亮翻單音蟬】今禍期甚迫宜早圖之臣屢言於主上主上篤於友愛以臣刀鋸之餘終不之信願殿下勿泄密表其狀殿下儻不信臣可召大將軍從事中郎王皮衛軍司馬劉惇假之恩意許其歸首【首式救翻】以問之必可知也粲許之猗密謂皮惇曰三王逆狀主上及相國具知之矣卿同之乎二人驚曰無之猗曰兹事已决吾憐卿親舊并見族耳因歔欷流涕【歔音虚欷許既翻又音希】二人大懼叩頭求哀猗曰吾為卿計卿能用之乎相國問卿卿但云有之若責卿不先啟卿即云臣誠負死罪然仰惟主上寛仁殿下敦睦苟言不見信則陷於誣譖不測之誅故不敢言也皮惇許諾粲召問之二人至不同時而其辭若一粲以為信然靳凖復說粲曰殿下宜自居東宫【時又居東宫復扶又翻】以領相國使天下早有所繫今道路之言皆云大將軍衛將軍欲奉太弟為變期以季春若使太弟得天下殿下無容足之地矣粲曰為之奈何凖曰人告太弟為變主上必不信宜緩東宫之禁使賓客得往來太弟雅好待士【好呼到翻】必不以此為嫌輕薄小人不能無迎合太弟之意為之謀者然後下官為殿下露表其罪【為于偽翻】殿下收其賓客與太弟交通者考問之獄辭既具則主上無不信之理也粲乃令卜抽引兵去東宫【去年聰令卜抽將兵監守東宫】少府陳休左衛將軍卜崇為人清直素惡沈等【惡烏路翻下同】雖在公座未嘗與語沈等深疾之侍中卜幹謂休崇曰王沈等勢力足以囘天地卿輩自料親賢孰與竇武陳蕃【言陳蕃之賢竇武之親且為宦官所困况休崇等乎】休崇曰吾輩年踰五十職位已崇唯欠一死耳死於忠義乃為得所安能俛首仾眉以事閹豎乎【仾與低同音都黎翻】去矣卜公勿復有言【復扶又翻下同】二月漢主聰出臨上秋閤【殿之西閤也】命收陳休卜崇及特進綦毋達大中大夫公師彧尚書王琰田歆大司農朱諧並誅之皆宦官所惡也卜幹泣諫曰陛下方側席求賢而一旦戮卿大夫七人皆國之忠良無乃不可乎藉使休等有罪陛下不下之有司【下戶稼翻】暴明其狀天下何從知之詔尚在臣所未敢宣露【卜幹為侍中詔經門下因留之而諫】願陛下熟思之因叩頭流血王沈叱幹曰卜侍中欲拒詔乎聰拂衣而入免幹為庶人太宰河間王易 【考異曰晉春秋易作士通今從載記】大將軍勃海王敷御史大夫陳元達金紫光禄大夫西河王延等皆詣闕表諫曰王沈等矯弄詔旨欺誣日月内諂陛下外佞相國威權之重侔於人主多樹姦黨毒流海内知休等忠臣為國盡節【為于偽翻】恐發其姦狀故巧為誣陷陛下不察遽加極刑痛徹天地【徹敕列翻通也】賢愚傷懼今遺晉未殄巴蜀不賓石勒謀據趙魏曹嶷欲王全齊【嶷魚力翻王于况翻】陛下心腹四支何處無患乃復以沈等助亂誅巫咸戮扁鵲臣恐遂成膏肓之疾【馬螎曰巫咸殷巫也扁鵲古良醫也秦醫緩視晉侯曰疾不可為也居膏之上肓之下攻之不可達之不及藥不至焉杜預曰心下為膏肓鬲也徐曰肓音荒說文曰心下鬲上也扁補典翻】後雖救之不可及已請免沈等官付有司治罪【治直之翻】聰以表示沈等笑曰羣兒為元達所引遂成癡也沈等頓首泣曰臣等小人過蒙陛下識抜得洒掃閨閤而王公朝士疾臣等如讐又深恨陛下願以臣等膏鼎鑊【膏居號翻潤也】則朝廷自然雍穆矣聰曰此等狂言常然卿何足恨乎聰問沈等於相國粲粲盛稱沈等忠清聰悅封沈等為列侯太宰易又詣闕上疏極諫聰大怒手壞其疏【壞音怪】三月易忿恚而卒易素忠直陳元達倚之為援得盡諫諍及卒元達哭之慟曰人之云亡邦國殄悴【詩大雅瞻卭之辭悴秦醉翻】吾既不復能言安用默默苟生乎歸而自殺 初代王猗盧愛其少子比延欲以為嗣使長子六修出居新平城而黜其母【建興元年猗盧築新平城新平城唐謂之新城在朔州界少詩照翻長知兩翻】六修有駿馬日行五百里猗盧奪之以與比延六修來朝【朝直遥翻】猗盧使拜比延六修不從猗盧乃坐比延於其步輦【步輦不駕馬使人輓之】使人導從【從才用翻】出遊六修望見以為猗盧伏謁路左至乃比延六修慙怒而去猗盧召之不至大怒帥衆討之為六修所敗【帥讀曰率敗補邁翻】猗盧微服逃民間有賤婦人識之遂為六修所弑拓抜普根先守外境聞難來赴攻六修滅之【難乃旦翻】普根代立國中大亂新舊猜嫌迭相誅滅左將軍衛雄信義將軍箕澹【澹徒覽翻又徒濫翻】久佐猗盧為衆所附謀歸劉琨乃言於衆曰聞舊人忌新人悍戰【舊人索頭部人也新人晉人及烏桓人也悍侯旰翻又下罕翻】欲盡殺之將奈何晉人及烏桓皆驚懼曰死生隨二將軍乃與琨質子遵帥晉人及烏桓三萬家馬牛羊十萬頭歸于琨【質音致帥讀曰率下同】琨大喜親詣平城撫納之琨兵由是復振夏四月普根卒其子始生普根母惟氏立之【惟氏猗㐌之妻】張寔下令所部吏民有能舉其過者賞以布帛羊米<br />
<br />
  賊曹佐高昌隗瑾曰【自漢以來公府方州郡國諸曹有掾有屬有左史前漢書西域傳車師國有高昌壁唐書曰高昌國漢車師前王庭也後破高昌置西州觀此則河西張氏固嘗於高昌之地置郡縣至後魏時始為高昌國也隗五罪翻】今明公為政事無巨細皆自决之或興師發令府朝不知【府朝謂僚佐所集之處朝直遥翻】萬一違失謗無所分羣下畏威受成而已如此雖賞之千金終不敢言也謂宜少損聰明【少詩沼翻】凡百政事皆延訪羣下使各盡所懷然後採而行之則嘉言自至何必賞也寔悦從之增瑾位三等寔遣將軍王該帥步騎五千入援長安且送諸郡貢計【貢土物也計計帳也】詔拜寔都督陜西諸軍事【陜失冉翻】以寔弟茂為秦州刺史 石勒使石虎攻劉演于廩丘幽州刺史段匹磾使其弟文鴦救之【磾丁奚翻】虎抜廩丘演犇文鴦軍虎獲演弟唘以歸 寧州刺史王遜嚴猛喜誅殺【喜許記翻】五月平夷太守雷炤【懷帝永嘉五年遜表分牂牁朱提建寧立平夷郡即漢平夷鄨二縣之地鄨孟康音鱉】平樂太守董霸【平樂郡證以隋志蓋置於越巂郡之卭部川然不知誰所置也樂音洛】帥三千餘家叛降於成【帥讀曰率降戶江翻】六月丁巳朔日有食之 秋七月漢大司馬曜圍北<br />
<br />
  地太守麴昌【晉北地郡領泥陽富平二縣耳】大都督麴允將步騎三萬救之曜遶城縱火烟起蔽天使反間紿允曰郡城已陷往無及也衆懼而潰曜追敗允於磻石谷【間古莧翻敗補邁翻磻蒲官翻魏收地形志比地郡銅官縣有石槃山】允犇還靈武曜遂取北地允性仁厚無威斷【斷丁亂翻】喜以爵位悦人【喜許記翻】新平太守竺恢始平太守楊像扶風太守竺爽安定太守焦嵩皆領征鎮杖節加侍中常侍村塢主帥小者猶假銀青將軍之號【征鎮四征四鎮將軍號也銀青將軍加將軍號而假以銀印青綬帥音所類翻】然恩不及下故諸將驕恣而士卒離怨關中危亂允告急於焦嵩嵩素侮允曰須允困當救之曜進至涇陽渭北諸城悉潰【涇陽縣前漢屬安定郡班志曰开頭山在縣西禹貢涇水所出東北至陽陵入渭過郡三行于六十里此言曜至涇陽渭北諸城悉潰則其兵已在池陽陽陵二縣間言在涇水之陽非安定之涇陽縣也】曜獲建威將軍魯充散騎常侍梁緯少府皇甫陽曜素聞充賢募生致之既見賜之酒曰吾得子天下不足定也充曰身為晉將國家喪敗不敢求生【緯于貴翻將子亮翻喪息浪翻】若蒙公恩速死為幸曜曰義士也賜之劒令自殺梁緯妻辛氏美色曜召見將妻之【妻如字】辛氏大哭曰妾夫已死義不獨生且一婦人而事二夫明公又安用之曜曰貞女也亦聽自殺皆以禮葬之 漢主聰立故張后侍婢樊氏為上皇后三后之外佩皇后璽綬者復有七人【璽斯氏翻綬音受復扶又翻下同據載記三后二靳氏及劉氏樊氏為四 考異曰劉聰載記曰四后之外按時靳上皇后已死唯三后耳云四誤也】嬖寵用事刑賞紊亂【嬖卑義翻又博計翻紊音問】大將軍敷數涕泣切諫【數所角翻】聰怒曰汝欲乃公速死邪何以朝夕生來哭人敷憂憤發病卒河東平陽大蝗民流殍者什五六石勒遣其將石越帥騎二萬屯并州招納流民【殍被表翻餓死於中野者曰殍散而之他方者曰流時勒蓋遣越屯上黨招納并州統内也帥讀曰率騎奇計翻】民歸之者二十萬戶聰遣使讓勒勒不受命潛與曹嶷相結【嶷魚力翻】 八月漢大司馬曜逼長安 九月漢主宴羣臣於光極殿引見太弟乂【見賢遍翻】乂容貌憔悴鬢髪蒼然涕泣陳謝聰亦為之慟哭【悴秦醉翻為于偽翻】乃縱酒極歡待之如初 焦嵩竺恢宋哲皆引兵救長安散騎常侍華輯監京兆馮翊弘農上洛四郡兵屯霸上【華戶化翻監古銜翻】皆畏漢兵彊不敢進相國保遣胡崧將兵入援擊漢大司馬曜於靈臺破之【三輔黄圖周文王靈臺在長安西四十里高二丈周圍百二十步】崧恐國威復振則麴索埶盛【麴允索綝也索昔各翻】乃帥城西諸郡兵屯渭北不進遂還槐里【槐里縣漢屬扶風晉屬始平郡】曜攻陷長安外城麴允索綝退保小城以自固内外斷絶城中饑甚米斗直金二兩人相食死者大半亡逃不可制唯涼州義衆千人守死不移【涼州義衆張軌父子所遣兵也】太倉有麴數十䴵【䴵必郢翻】麴允屑之為粥以供帝既而亦盡冬十一月帝泣謂允曰今窮厄如此外無救援當忍恥出降以活士民【降戶江翻下同】因歎曰誤我事者麴索二公也使侍中宗敞送降牋於曜 【考異曰帝紀作宋敞今從晉春秋】索綝潛留敞使其子說曜曰今城中食猶足支一年未易克也【說輸芮翻易以豉翻】若許綝以儀同萬戶郡公者請以城降曜斬而送之曰帝王之師以義行也孤將兵十五年未嘗以詭計敗人【敗補邁翻】必窮兵極勢然後取之今索綝所言如此天下之惡一也輒相為戮之【為于偽翻】若兵食審未盡者便可勉強固守【強其兩翻】如其糧竭兵微亦宜早寤天命甲午宗敞至曜營乙未帝乘羊車肉袒銜璧輿櫬出東門降【櫬初覲翻】羣臣號泣攀車執帝手帝亦悲不自勝【號戶刀翻勝音升】御史中丞馮翊吉朗歎曰吾智不能謀勇不能死何忍君臣相隨北面事賊虜乎乃自殺曜焚櫬受璧使宗敞奉帝還宫丁酉遷帝及公卿以下於其營辛丑送至平陽壬寅漢主聰臨光極殿帝稽首於前【稽音啟】麴允伏地慟哭扶不能起聰怒囚之允自殺聰以帝為光禄大夫封懷安侯以大司馬曜為假黄鉞大都督督陜西諸軍事太宰封秦王大赦改元麟嘉以麴允忠烈贈車騎將軍諡節愍侯【允則忠矣然猶在吉朗之後乎】以索綝不忠斬于都市【平陽都市也】尚書梁允侍中梁濬等及諸郡守皆為曜所殺華輯犇南山干寶論曰昔高祖宣皇帝以雄才碩量應時而起性深阻有若城府而能寛綽以容納行數術以御物而知人善采抜【言胷中有城府者多不能寛容任數用術者多不能用人而宣帝能之也】於是百姓與能【謂天下皆推其能莫與爭也】大象始構【劉良曰象法也言晉之興成大法從此始立也】世宗承基太祖繼業咸黜異圖用融前烈【謂内誅李豐夏侯玄外平毌丘儉文欽諸葛誕】至于世祖遂享皇極【呂延濟曰享當也皇極天子位也】仁以厚下儉以足用和而不弛寛而能斷【斷丁亂翻】掩唐虞之舊域班正朔於八荒【八荒謂八方之外戎荒之地】于時有天下無窮人之諺【呂尚曰言百姓盡富】雖太平未洽亦足以明民樂其生矣【樂音洛】武皇既崩山陵未乾而變難繼起宗子無維城之助【乾音干宗子謂八王搆難詩曰宗子維城】師尹無具瞻之貴【詩曰赫赫師尹民具爾瞻】朝為伊周夕成桀跖【謂楊駿衛瓘張華等】國政迭移於亂人禁兵外散於四方方岳無鈞石之鎮關門無結草之固【三十斤為鈞四鈞為石左傳秦伐晉晉魏顆敗秦師獲杜囘顆夢老人結草以亢杜回杜回躓而顛故獲之】戎羯稱制二帝失尊何哉樹立失權託付非才四維不張而苟且之政多也【賈誼策曰禮義廉恥是謂四維四維不張國乃滅亡】夫基廣則難傾根深則難抜理節則不亂膠結則不遷【李周翰曰理節謂政教有條理節度也膠固也言君布仁惠之根基深廣又不失理節則人心固結而不可遷也】昔之有天下者所以能長久用此道也周自后稷愛民十六王而武始君之【后稷子不窋不窋子鞠鞠子公劉公劉子慶節慶節子皇僕皇僕子差弗差弗子毁隃毁隃子公非公非子高圉高圉子亞圉亞圉子公叔祖類公叔祖類子古公亶父古公亶父子季歷季歷子文王文王子武王凡十六王】其積基樹本如此其固今晉之興也其創基立本固異於先代矣加以朝寡純德之人鄉乏不貳之老【周官有鄉老不貳謂不貳過者朝直遥翻】風俗淫僻恥尚失所【言所恥者非所恥所尚者非所尚也】學者以莊老為宗而黜六經談者以虚蕩為辯而賤名檢行身者以放濁為通而狹節信進仕者以苟得為貴而鄙居正當官者以望空為高而笑勤恪【呂延濟曰望空謂不識是非但望空署名而已】是以劉頌屢言治道傅咸每糾邪正【頌咸事並見武紀惠紀治直吏翻】皆謂之俗吏其倚杖虛曠依阿無心者皆名重海内若夫文王日昃不暇食仲山甫夙夜匪懈者蓋共嗤黜以為灰塵矣【文王自朝至于日中昃不遑暇食用咸和萬民仲山甫夙夜匪懈以事一人】由是毁譽亂於善惡之實情慝犇於貨欲之塗選者為人擇官官者為身擇利【呂延濟曰言選者不復為官擇賢為官者但擇所利而趨譽音余為于偽翻】世族貴戚之子弟陵邁超越不拘資次悠悠風塵皆奔競之士列官千百無讓賢之舉子真著崇讓而莫之省【劉寔字子真崇讓論見八十二卷武帝太康十年省悉景翻】子雅制九班而不得用【劉頌字子雅九班之制見同上】其婦女不知女工任情而動有逆于舅姑有殺戮妾媵【二事皆賈后為之倡】父兄弗之罪也天下莫之非也禮法刑政于此大壞國之將亡本必先顛其此之謂乎【左傳曰國將亡本必先顛而後枝葉從之】故觀阮籍之行而覺禮教崩弛之所由【事見七十八卷魏元帝景元三年行下孟翻】察庾純賈充之爭而見師尹之多僻【事見七十九卷武帝泰始七年八年】考平吳之功而知將帥之不讓思郭欽之謀而寤戎狄之有釁【平吳爭功及郭欽疏並見八十一卷武帝太康元年】覽傅玄劉毅之言而得百官之邪【傅玄劉毅武帝時為司隸前後糾核不避貴游因其所言而得百官之邪也】核傅咸之奏錢神之論而覩寵賂之彰【傅咸奏見八十二卷惠帝元康四年神錢論見八十三卷元康九年】民風國勢既已如此雖以中庸之才守文之主治之【劉良曰中庸謂非賢非愚之才守文謂守常平治世之主也治直之翻】猶懼致亂况我惠帝以放蕩之德臨之哉懷帝承亂即位覊於彊臣愍帝犇播之後徒守虛名天下之勢既去非命世之雄材不能復取之矣<br />
<br />
  石勒圍樂平太守韓據于坫城【楊正衡曰坫丁念翻余按武帝泰始中分上黨太原置樂平郡治沾縣沾縣漢屬上黨郡魏收地形志樂平縣有沾城師古曰沾音他兼翻載記誤作坫當讀從顔音】據請救於劉琨琨新得拓跋猗盧之衆欲因其鋭氣以討勒箕澹衛雄諫曰【澹徒覽翻又徒濫翻】此雖晉民久淪異域未習明公之恩信恐其難用不若且内收鮮卑之餘穀【拓跋鮮卑也】外抄胡賊之牛羊【胡謂劉石也抄禁交翻】閉關守險務農息兵待其服化感義然後用之則功無不濟矣琨不從悉發其衆命澹帥步騎二萬為前驅琨屯廣牧為之聲援【廣牧縣漢屬朔方郡漢末省朔方置廣牧縣於陘南屬新興郡非廣牧縣故地也帥讀曰率】石勒聞澹至將逆擊之或曰澹士馬精彊其鋒不可當不若且引兵避之深溝高壘以挫其銳必獲萬全勒曰澹兵雖衆遠來疲弊號令不齊何精彊之有今寇敵垂至何可捨去大軍一動豈易中還【易以豉翻】若澹乘我之退而逼之顧逃潰不暇焉得深溝高壘乎【焉於䖍翻】此自亡之道也立斬言者以孔萇為前鋒都督令三軍後出者斬勒據險要設疑兵於山上前設二伏出輕騎與澹戰陽為不勝而走澹縱兵追之入伏中勒前後夾擊澹軍大破之獲鎧馬萬計雄澹帥騎千餘犇代郡【帥讀曰率下同】韓據棄城走并土震駭 十二月乙卯朔日有食之 【考異曰帝紀天文志皆誤作甲申朔宋志乙卯朔與長歷合今從之】 司空長史李弘以并州降石勒【劉琨為司空以弘為長史并州時治陽曲】劉琨進退失據不知所為段匹磾遣信邀之己未琨帥衆從飛狐犇薊【恒山在常山上曲陽縣西北有阪號飛狐口磾丁奚翻薊音計】匹磾見琨甚相親重與之結婚約為兄弟勒分徙陽曲樂平民于襄國置守宰而還孔萇攻箕澹于代郡殺之【據載記萇攻澹於桑乾則此代郡乃後魏之代郡非漢晉之代郡也】萇等攻賊帥馬嚴馮䐗【帥所類翻䐗張如翻嚴䐗蓋為盗於幽冀之間】久而不克司冀并兖流民數萬戶在遼西迭相招引民不安業勒問計於濮陽侯張賓賓曰嚴䐗本非公之深仇流民皆有戀本之志今班師振旅選良牧守使招懷之則幽冀之寇可不日而清遼西流民將相帥而至矣勒乃召萇等歸以武遂令李回為易北督護兼高陽太守【武遂縣前漢屬河間國後漢晉屬安平國易北易水以北也高陽縣前漢屬涿郡後漢屬河間國武帝泰始元年分置高陽國應劭曰在高河之陽】馬嚴士卒素服回威德多叛嚴歸之嚴懼而出走赴水死馮䐗帥其衆降【帥讀曰率】回徙居易京【易京公孫瓚所築】流民歸之者相繼於道勒喜封回為弋陽子增張賓邑千戶進位前將軍賓固辭不受 丞相睿聞長安不守出師露次【露次者出宿于野上無屋宇】躬擐甲胄【擐音宦】移檄四方刻日北征以漕運稽期斬督運令史淳于伯刑者以刀拭柱血逆流上至柱末二丈餘而下【上時掌翻下同】觀者咸以為寃丞相司直劉隗上言伯罪不至死請免從事中郎周莚等官於是右將軍王導等上疏引咎請解職睿曰政刑失中皆吾闇塞所致【塞悉則翻】一無所問隗性剛訐當時名士多被彈劾【訐居謁翻被皮義翻彈徒丹翻劾戶槩翻又戶得翻】睿率皆容貸由是衆怨皆歸之南中郎將王含敦之兄也以族彊位顯驕傲自恣一請參佐及守長至二十許人多非其才【守式又翻長知兩翻】隗劾奏含文致甚苦【深文以致其罪】事雖被寢而王氏深忌疾之【為王敦請誅劉隗張本】 丞相睿以邵續為冀州刺史續女壻廣平劉遐聚衆河濟之間【濟子禮翻】睿以遐為平原内史 托跋普根之子乂卒【托與拓通魏改魏書本作托跋】國人立其從父欝律【從才用翻】<br />
<br />
  資治通鑑卷八十九<br />
<br />
<史部,編年類,資治通鑑>  <br>
   </div> 

<script src="/search/ajaxskft.js"> </script>
 <div class="clear"></div>
<br>
<br>
 <!-- a.d-->

 <!--
<div class="info_share">
</div> 
-->
 <!--info_share--></div>   <!-- end info_content-->
  </div> <!-- end l-->

<div class="r">   <!--r-->



<div class="sidebar"  style="margin-bottom:2px;">

 
<div class="sidebar_title">工具类大全</div>
<div class="sidebar_info">
<strong><a href="http://www.guoxuedashi.com/lsditu/" target="_blank">历史地图</a></strong>  
<a href="http://www.880114.com/" target="_blank">英语宝典</a>  
<a href="http://www.guoxuedashi.com/13jing/" target="_blank">十三经检索</a> 
<br><strong><a href="http://www.guoxuedashi.com/gjtsjc/" target="_blank">古今图书集成</a></strong> 
<a href="http://www.guoxuedashi.com/duilian/" target="_blank">对联大全</a> <strong><a href="http://www.guoxuedashi.com/xiangxingzi/" target="_blank">象形文字典</a></strong> 

<br><a href="http://www.guoxuedashi.com/zixing/yanbian/">字形演变</a>  <strong><a href="http://www.guoxuemi.com/hafo/" target="_blank">哈佛燕京中文善本特藏</a></strong>
<br><strong><a href="http://www.guoxuedashi.com/csfz/" target="_blank">丛书&方志检索器</a></strong> <a href="http://www.guoxuedashi.com/yqjyy/" target="_blank">一切经音义</a>  

<br><strong><a href="http://www.guoxuedashi.com/jiapu/" target="_blank">家谱族谱查询</a></strong>  <strong><a href="http://shufa.guoxuedashi.com/sfzitie/" target="_blank">书法字帖欣赏</a></strong> 
<br>

</div>
</div>


<div class="sidebar" style="margin-bottom:0px;">

<font style="font-size:22px;line-height:32px">QQ交流群9:489193090</font>


<div class="sidebar_title">手机APP 扫描或点击</div>
<div class="sidebar_info">
<table>
<tr>
	<td width=160><a href="http://m.guoxuedashi.com/app/" target="_blank"><img src="/img/gxds-sj.png" width="140"  border="0" alt="国学大师手机版"></a></td>
	<td>
<a href="http://www.guoxuedashi.com/download/" target="_blank">app软件下载专区</a><br>
<a href="http://www.guoxuedashi.com/download/gxds.php" target="_blank">《国学大师》下载</a><br>
<a href="http://www.guoxuedashi.com/download/kxzd.php" target="_blank">《汉字宝典》下载</a><br>
<a href="http://www.guoxuedashi.com/download/scqbd.php" target="_blank">《诗词曲宝典》下载</a><br>
<a href="http://www.guoxuedashi.com/SiKuQuanShu/skqs.php" target="_blank">《四库全书》下载</a><br>
</td>
</tr>
</table>

</div>
</div>


<div class="sidebar2">
<center>


</center>
</div>

<div class="sidebar"  style="margin-bottom:2px;">
<div class="sidebar_title">网站使用教程</div>
<div class="sidebar_info">
<a href="http://www.guoxuedashi.com/help/gjsearch.php" target="_blank">如何在国学大师网下载古籍?</a><br>
<a href="http://www.guoxuedashi.com/zidian/bujian/bjjc.php" target="_blank">如何使用部件查字法快速查字?</a><br>
<a href="http://www.guoxuedashi.com/search/sjc.php" target="_blank">如何在指定的书籍中全文检索?</a><br>
<a href="http://www.guoxuedashi.com/search/skjc.php" target="_blank">如何找到一句话在《四库全书》哪一页?</a><br>
</div>
</div>


<div class="sidebar">
<div class="sidebar_title">热门书籍</div>
<div class="sidebar_info">
<a href="/so.php?sokey=%E8%B5%84%E6%B2%BB%E9%80%9A%E9%89%B4&kt=1">资治通鉴</a> <a href="/24shi/"><strong>二十四史</strong></a>&nbsp; <a href="/a2694/">野史</a>&nbsp; <a href="/SiKuQuanShu/"><strong>四库全书</strong></a>&nbsp;<a href="http://www.guoxuedashi.com/SiKuQuanShu/fanti/">繁体</a>
<br><a href="/so.php?sokey=%E7%BA%A2%E6%A5%BC%E6%A2%A6&kt=1">红楼梦</a> <a href="/a/1858x/">三国演义</a> <a href="/a/1038k/">水浒传</a> <a href="/a/1046t/">西游记</a> <a href="/a/1914o/">封神演义</a>
<br>
<a href="http://www.guoxuedashi.com/so.php?sokeygx=%E4%B8%87%E6%9C%89%E6%96%87%E5%BA%93&submit=&kt=1">万有文库</a> <a href="/a/780t/">古文观止</a> <a href="/a/1024l/">文心雕龙</a> <a href="/a/1704n/">全唐诗</a> <a href="/a/1705h/">全宋词</a>
<br><a href="http://www.guoxuedashi.com/so.php?sokeygx=%E7%99%BE%E8%A1%B2%E6%9C%AC%E4%BA%8C%E5%8D%81%E5%9B%9B%E5%8F%B2&submit=&kt=1"><strong>百衲本二十四史</strong></a>  <a href="http://www.guoxuedashi.com/so.php?sokeygx=%E5%8F%A4%E4%BB%8A%E5%9B%BE%E4%B9%A6%E9%9B%86%E6%88%90&submit=&kt=1"><strong>古今图书集成</strong></a>
<br>

<a href="http://www.guoxuedashi.com/so.php?sokeygx=%E4%B8%9B%E4%B9%A6%E9%9B%86%E6%88%90&submit=&kt=1">丛书集成</a> 
<a href="http://www.guoxuedashi.com/so.php?sokeygx=%E5%9B%9B%E9%83%A8%E4%B8%9B%E5%88%8A&submit=&kt=1"><strong>四部丛刊</strong></a>  
<a href="http://www.guoxuedashi.com/so.php?sokeygx=%E8%AF%B4%E6%96%87%E8%A7%A3%E5%AD%97&submit=&kt=1">說文解字</a> <a href="http://www.guoxuedashi.com/so.php?sokeygx=%E5%85%A8%E4%B8%8A%E5%8F%A4&submit=&kt=1">三国六朝文</a>
<br><a href="http://www.guoxuedashi.com/so.php?sokeytm=%E6%97%A5%E6%9C%AC%E5%86%85%E9%98%81%E6%96%87%E5%BA%93&submit=&kt=1"><strong>日本内阁文库</strong></a> <a href="http://www.guoxuedashi.com/so.php?sokeytm=%E5%9B%BD%E5%9B%BE%E6%96%B9%E5%BF%97%E5%90%88%E9%9B%86&ka=100&submit=">国图方志合集</a> <a href="http://www.guoxuedashi.com/so.php?sokeytm=%E5%90%84%E5%9C%B0%E6%96%B9%E5%BF%97&submit=&kt=1"><strong>各地方志</strong></a>

</div>
</div>


<div class="sidebar2">
<center>

</center>
</div>
<div class="sidebar greenbar">
<div class="sidebar_title green">四库全书</div>
<div class="sidebar_info">

《四库全书》是中国古代最大的丛书,编撰于乾隆年间,由纪昀等360多位高官、学者编撰,3800多人抄写,费时十三年编成。丛书分经、史、子、集四部,故名四库。共有3500多种书,7.9万卷,3.6万册,约8亿字,基本上囊括了古代所有图书,故称“全书”。<a href="http://www.guoxuedashi.com/SiKuQuanShu/">详细>>
</a>

</div> 
</div>

</div>  <!--end r-->

</div>
<!-- 内容区END --> 

<!-- 页脚开始 -->
<div class="shh">

</div>

<div class="w1180" style="margin-top:8px;">
<center><script src="http://www.guoxuedashi.com/img/plus.php?id=3"></script></center>
</div>
<div class="w1180 foot">
<a href="/b/thanks.php">特别致谢</a> | <a href="javascript:window.external.AddFavorite(document.location.href,document.title);">收藏本站</a> | <a href="#">欢迎投稿</a> | <a href="http://www.guoxuedashi.com/forum/">意见建议</a> | <a href="http://www.guoxuemi.com/">国学迷</a> | <a href="http://www.shuowen.net/">说文网</a><script language="javascript" type="text/javascript" src="https://js.users.51.la/17753172.js"></script><br />
  Copyright &copy; 国学大师 古典图书集成 All Rights Reserved.<br>
  
  <span style="font-size:14px">免责声明:本站非营利性站点,以方便网友为主,仅供学习研究。<br>内容由热心网友提供和网上收集,不保留版权。若侵犯了您的权益,来信即刪。scp168@qq.com</span>
  <br />
ICP证:<a href="http://www.beian.miit.gov.cn/" target="_blank">鲁ICP备19060063号</a></div>
<!-- 页脚END --> 
<script src="http://www.guoxuedashi.com/img/plus.php?id=22"></script>
<script src="http://www.guoxuedashi.com/img/tongji.js"></script>

</body>
</html>
