\section{資治通鑑卷二百二十九}
宋 司馬光 撰

胡三省 音註

唐紀四十五|{
	起昭陽大淵獻十一月盡閼逢困敦正月不滿一年始癸亥十一月終甲子正月一卷所紀財三月耳}


德宗神武聖文皇帝四

建中四年十一月乙亥以隴州為奉義軍擢臯為節度使泚又使中使劉海廣許臯鳳翔節度使臯斬之|{
	史言韋臯以此身使疎吏翻泚且禮翻}
靈武留後杜希全鹽州刺史戴休顔夏州刺史時常春會渭北節度使李建徽合兵萬人入援|{
	靈武節度使治靈州夏州治朔方縣鹽州治五原縣皆鄰境相接渭北節度使本治坊州時徙治鄜州夏戶雅翻}
將至奉天上召將相議道所從出關播渾瑊曰漠谷道險狹|{
	召將即亮翻相息亮翻渾戶昆翻又戶本翻瑊古衘翻漠谷在奉天城西北}
恐為賊所邀不若自乾陵北過附柏城而行|{
	山陵樹柏成行以遮迾陵寢故謂之柏城宋白曰唐諸陵皆栽柏環之貞元六年十一月敕諸陵柏城四面各三里内不得安葬過古禾翻又古卧翻}
營於城東北雞子堆與城中犄角相應|{
	犄居蟻翻}
且分賊勢盧曰漠谷道近若為賊所邀則城中出兵應接可也儻出乾陵恐驚陵寢瑊曰自泚攻城斬乾陵松栢以夜繼晝其驚多矣今城中危急諸道救兵未至惟希全等來所繫非輕若得營據要地則泚可破也曰陛下行師豈比逆賊若令希全等過之是自驚陵寢|{
	泚且禮翻又音此令力丁翻}
上乃命希全等自漠谷進丙子希全等軍至漠谷果為賊所邀乘高以大弩巨石擊之死傷甚衆城中出兵應接為賊所敗是夕四軍潰退保邠州泚閲其輜重於城下從官相視失色|{
	自兩河兵興以至乘輿播遷盧杞之言無一不誤國而德宗信之如故庸昏甚矣敗補邁翻從才用翻邠卑旻翻泚且禮翻又音此輜莊持翻重直用翻}
休顔夏州人也|{
	夏戶雅翻}
泚攻城益急穿塹環之泚移帳於乾陵下視城中動静皆見之時遣使環城|{
	塹士艶翻使疏吏翻環音宦}
招誘士民笑其不識天命|{
	誘音酉}
神策河北行營節度使李晟疾愈|{
	前年五月李晟疾甚自易州還保定州事見上卷晟成正翻}
聞上幸奉天帥衆將奔命|{
	帥讀曰率}
張孝忠迫於朱滔王武俊倚晟為援不欲晟行數沮止之|{
	數所角翻沮在呂翻}
晟乃留其子憑使娶孝忠女為婦又解玉帶賂孝忠親信使說之|{
	說式芮翻}
孝忠乃聽晟西歸遣大將楊榮國將鋭兵六百與晟俱晟引兵出飛狐道晝夜兼行至代州|{
	沈存中曰北岳常岑謂之大茂山者是也半屬契丹以大茂山脊為界飛狐路在茂之西自銀冶寨北出倒馬關度虜界却自石門子令水鋪入缾形梅囘兩寨之間至代州今此路已不通惟北寨西出承天關路可至河東然路極峭狹按存中所謂地界乃石晉與契丹所分地界也}
丁丑加晟神策行營節度使|{
	史言李晟前只節度河北神策出征兵行營今又加節度神策行營兵出征河南者此其所以得誅劉德信也}
王武俊馬寔攻趙州不克辛巳寔歸瀛州武俊送之五里犒贈甚厚武俊亦歸恒州|{
	恒戶登翻}
上之出幸奉天也陜虢觀察使姚明|{
	陜失冉翻與揚同}
以軍事委都防禦副使張勸去詣行在勸募兵得數萬人甲申以勸為陜虢節度使 朱泚攻圍奉天經月|{
	是年十月上出奉天纔至奉天數日而朱泚繼至攻圍至是月為經月}
城中資糧俱盡上嘗遣健步出城覘賊|{
	健步今之急脚子是也覘五亷翻}
其人懇以苦寒為辭跪奏乞一襦袴|{
	襦汝朱翻短衣}
上為之尋求不獲|{
	為于偽翻}
竟憫默而遣之|{
	憫者矜其寒默者無以為辭也}
時供御纔有糲米二斛每伺賊之休息夜縋人於城外采蕪菁根而進之|{
	本草曰蕪菁及蘆菔南北通有之蕪菁即蔓菁蘆菔即蘿蔔也陶隱居云蘆菔是今温菘其根可食葉不中噉蕪菁根乃細於温菘而葉似菘好食日華子曰梗長葉瘦高者為菘闊厚短肥而及梗細者為蕪菁葉也陸佃埤雅曰舊說菘菜北種初年半為蕪菁二年菘種都絶蕪菁南種亦然菘之不生北土猶橘柚之變於淮北矣蕪菁似菘而小有臺一名葑一名須史炤曰本草注云蕪菁北人又名蔓菁根葉及子乃是菘類詩云采葑采菲疏云陸璣云葑蕪菁幽州人或謂之芥方言云蘴蕘蕪菁也陳楚謂之葑齊魯謂之蕘關西謂之蕪菁趙魏之郊謂之大芥糲盧達翻伺相吏翻縋馳偽翻}
上召公卿將吏謂曰朕以不德自䧟危亡固其宜也公輩無罪宜早降以救家室|{
	降戶江翻}
羣臣皆頓首流涕期盡死力故將士雖困急而鋭氣不衰上之幸奉天也糧料使崔縱勸李懷光令入援|{
	崔縱為魏縣行營糧料使}
懷光從之縱悉歛軍資與懷光皆來懷光晝夜倍道至河中力疲休兵三日河中尹李齊運傾力犒宴|{
	犒口到翻}
軍尚欲遷延崔縱先輦貨財度河謂衆曰至河西悉以分賜|{
	開元八年析河東縣自蒲津以西為河西縣}
衆利之西屯蒲城有衆五萬齊運惲之孫也|{
	蔣王惲太宗子也惲於粉翻}
李晟行且收兵亦自蒲津濟軍於東渭橋其始有卒四千晟善於撫御與士卒同甘苦人樂從之|{
	晟成正翻樂音洛}
旬月間至萬餘人神策兵馬使尚可孤討李希烈將三千人在襄陽自武關入援軍于七盤|{
	使踈吏翻將即亮翻又音如字七盤即古繞霤之險}
敗泚將仇敬|{
	仇敬即仇敬忠此因舊史書之敗補邁翻}
遂取藍田可孤宇文部之别種也|{
	種章勇翻}
鎮國軍副使駱元光|{
	肅宗上元元年置鎮國軍於華州}
其先安息人駱奉先養以為子將兵守潼關近十年為衆所服|{
	潼音童近其靳翻}
朱泚遣其將何望之襲華州刺史董晉弃州走行在|{
	華戶化翻下同走音奏}
望之據其城將聚兵以絶東道元光引關下兵襲望之走還長安|{
	還從宣翻又音如字}
元光遂軍華州召募士卒數日得萬餘人泚數遣兵攻元光|{
	泚且禮翻又音此數所角翻}
元光皆擊却之賊由是不能東出上即以元光為鎮國軍節度使|{
	鎮國軍節度治華州}
元光乃將兵二千西屯昭應馬燧遣其行軍司馬王權及其子彚將兵五千人入援屯中渭橋|{
	燧音遂彚于季翻宋敏求長安志引三輔黄圖曰渭水貫都以象天漢横橋南度以法牽牛盖指此之中橋而為若言也橋之廣至及六丈其柱之多至於七百五十約其地望即唐太極宫之西而太倉之北也程大昌曰此橋舊止單名渭橋水經叙渭曰水上有梁謂之橋者是也後世加中以冠橋上者為長安之西别有便門橋度渭萬年縣之東更有東渭橋故不得不以中别之也}
於是泚黨所據惟長安而已援軍遊騎時至望春樓下李忠臣等屢出兵皆敗求援於泚泚恐民間乘弊抄之|{
	望春樓近長樂城臨廣運潭玄宗所立騎奇寄翻抄楚交翻}
所遣兵皆晝伏夜行泚内以長安為憂乃急攻奉天使僧法堅造雲梯高廣各數丈|{
	高居傲翻廣古曠翻近世學者多各以音如字讀之 考異曰劇談録曰高九十餘尺下瞰城中今從實録}
裹以兕革|{
	史炤曰兕色如野牛而青一說雌犀也余按山海經兕角重百斤身重千斤黄帝得之以其皮冒鼓聲震百里其說固誕矣國語叔向曰唐叔殪兕以為大甲周官考工記犀甲壽百年兕甲壽二百年兕甲固堅於犀甲矣左傳宋華元之言曰犀兕尚多則兕者世之常有也然兕者今不常見史言朱泚裹雲梯以兕革不過用牛皮耳兕序姊翻}
下施巨輪上容壯士五百人城中望之忷懼上以問羣臣渾瑊侯仲莊對曰臣觀雲梯勢甚重重則易陷|{
	忷許拱翻渾戶昆翻又戶本翻瑊古銜翻易以豉翻}
臣請迎其所來鑿地道積薪蓄火以待之神武軍使韓澄曰|{
	開元二十六年分左右羽林置左右神武軍使疏吏翻}
雲梯小伎不足上勞聖慮|{
	伎渠綺翻}
臣請禦之乃度梯之所傃|{
	度徒洛翻傃桑故翻向也鄭玄曰攻城攻其所傃傃猶嚮也}
廣城東北隅三十步多儲膏油松脂薪葦於其上丁亥泚盛兵鼓譟攻南城韓遊瓌曰此欲分吾力也乃引兵嚴備東北戊子北風甚迅|{
	譟則竈翻環古囘翻迅疾也}
泚推雲梯|{
	推吐雷翻}
上施濕氈懸水囊載壯士攻城翼以轒輼|{
	轒扶云翻輼於云翻轒輼攻城車也兵法修轒輼距堙者三月而後成}
置人其下抱薪負土填塹而前矢石火炬所不能傷賊併兵攻城東北隅矢石如雨城中死傷者不可勝數|{
	塹七艶翻勝音升}
賊已有登城者上與渾瑊對泣羣臣惟仰首祝天上以無名告身自御史大夫實食五百戶以下千餘通授瑊|{
	無名告身即空名告身有功者則書填姓名以授之實食食實封也}
使募敢死士禦之仍賜御筆使視其功之大小書名給之告身不足則書其身|{
	謂若立功者多所給告身千餘通酬功而不足則書陳前所喝轉階勲於其身以為照驗出給告身}
且曰今便與卿别|{
	期望渾瑊死戰也}
瑊俯伏流涕上拊其背歔欷不自勝|{
	瑊古衘翻歔音虛欷許旣翻又音希勝音升}
時士卒凍餒又乏甲胄瑊撫諭激以忠義皆鼓譟力戰瑊中流矢|{
	譟則竈翻中竹仲翻}
進戰不輟初不言痛會雲梯輾地道一輪偏陷不能前却|{
	輾猪輦翻又尼展翻地道者渾瑊等所鑿以迎雲梯者也}
火從地中出|{
	火亦渾瑊等所蓄以待雲梯者}
風勢亦回城上人投葦炬散松脂沃以膏油讙呼震地|{
	讙許元翻}
須臾雲梯及梯上人皆為灰燼臭聞數里|{
	聞音問}
賊乃引退於是三門皆出兵|{
	時朱泚攻奉天城東南北三面故三門皆出兵與戰}
太子親督戰賊徒大敗死者數千人將士傷者太子親為裹瘡|{
	將即亮翻為于偽翻}
入夜泚復來攻城|{
	泚且禮翻又音此復扶又翻又音如字}
矢及御前三步而墜上大驚李懷光自蒲城引兵趣涇陽|{
	趣七喻翻}
並北山而西|{
	並讀曰傍步浪翻}
先遣兵馬使張韶微服間行詣行在|{
	間古莧翻下得間同}
藏表於蠟丸詔至奉天值賊方攻城見韶以為賤人驅之使與民俱填塹韶得間踰塹抵城下呼曰我朔方軍使者也|{
	塹七艷翻呼火故翻使疏吏翻}
城上人下繩引之比登|{
	比必利翻及也}
身中數十矢|{
	中竹仲翻}
得表於衣中而進之上大喜舁韶以徇城四隅歡聲如雷|{
	舁音余又羊茹翻}
癸巳懷光敗泚兵於灃泉|{
	敗補邁翻}
泚聞之懼引兵遁歸長安衆以為懷光復三日不至則城不守矣|{
	史言李懷光解奉天之圍不為無功泚且禮翻又音此}
泚旣退從臣皆賀|{
	從才用翻}
汴滑行營兵馬使賈隱林進言陛下性太急不能容物若此性未改雖朱泚敗亡憂未艾也上不以為忤甚稱之|{
	使踈吏翻汴皮變翻使德宗果能以此心而受諫何至追仇陸贄之盡言乎忤五故翻}
侍御史萬俟著開金商運路|{
	萬當作万莫北翻俟渠之翻万俟虜複姓也開金商運路轉江淮財賦以至奉天}
重圍既解|{
	重直龍翻}
諸道貢賦繼至用度始振朱泚至長安但為城守之計時遣人自城外來周走呼曰|{
	呼火故翻}
奉天破矣欲以惑衆泚旣據府庫之富不愛金帛以悦將士公卿家屬在城者皆給月俸|{
	將即亮翻俸扶用翻}
神策及六軍從車駕及哥舒曜李晟者泚皆給其家糧加以繕完器械日費甚廣及長安平府庫尚有餘蓄見者皆追怨有司之暴歛焉|{
	以此觀之趙贊輩不足責也杜佑判度支安能逃其罪乎歛力贍翻}
或謂泚曰陛下旣受命唐之陵廟不宜復存|{
	復扶又翻}
泚曰朕嘗北面事唐豈忍為此又曰百官多缺請以兵脅士人補之泚曰強授之則人懼|{
	強其兩翻}
但欲仕者則與之何必叩戶拜官邪|{
	邪音耶}
泚所用者惟范陽神策團練兵|{
	團練兵即團結兵事見二百二十五卷代宗大歷十二年}
涇原卒驕皆不為用但守其所掠資貨不肯出戰又密謀殺泚不果而止李懷光性粗疎|{
	泚且禮翻又音此粗讀與麤同}
自山東來赴難|{
	自魏縣行營來赴奉天之難魏縣屬魏州其地在河山之東難乃旦翻下同}
數與人言盧趙贊白志貞之姦佞|{
	數所角翻}
且曰天下之亂皆此曹所為也吾見上當請誅之旣解奉天之圍自矜其功謂上必接以殊禮|{
	禮絶百僚謂之殊禮}
或說王翃趙贊曰|{
	說式芮翻翃戶萌翻}
懷光緣道憤歎以為宰相謀議乖方|{
	乖方猶言失所也}
度支賦歛煩重京尹犒賜刻薄致乘輿播遷者三臣之罪也|{
	宰相指盧度支指趙贊京尹指王翃度徒洛翻歛力贍翻乘繩證翻}
今懷光新立大功上必披襟布誠詢得失使其言入豈不殆哉|{
	殆危也}
翃贊以告盧懼從容言於上曰|{
	從千容翻下同}
懷光勲業社稷是賴賊徒破膽皆無守心若使之乘勝取長安則一舉可以滅賊此破竹之勢也今聽其入朝必當賜宴留連累日使賊入京城得從容成備恐難圖矣上以為然|{
	懷光矜功厚望其上而求逞其欲德宗欲速逼使其下而不閔其勞盧之心自營免罪而柙闔於其間是以雖急於平賊而不知更生一賊也朝直遥翻}
詔懷光直引軍屯便橋與李建徽李晟及神策兵馬使楊惠元刻期共取長安懷光自以數千里竭誠赴難破朱泚解重圍而咫尺不得見天子意殊怏怏|{
	晟成正翻使疏吏翻難乃旦翻泚且禮翻又音此重直龍翻怏於兩翻}
曰吾今已為姦臣所排事可知矣遂引兵去至魯店|{
	魯店在奉天東南咸陽陳濤斜西北}
留二日乃行|{
	為李懷光反與朱泚連兵張本}
劒南西山兵馬使張朏以所部兵作亂入成都|{
	使疏吏翻劒南宿重兵于西山以備吐蕃崔寧以是兵殺郭英乂張朏以是兵逐張延賞朏敷尾翻}
西川節度使張延賞弃城奔漢州|{
	武后垂拱二年分益州置漢州九域志成都北至漢州九十五里}
鹿頭戍將叱干遂等討之|{
	鹿頭關在漢州德陽縣劉昫曰成都北一百五十里有鹿頭山扼兩川之要將即亮翻叱尺栗翻}
斬朏及其黨延賞復歸成都 淮南節度使陳少遊將兵討李希烈屯盱眙|{
	盱眙漢縣唐初屬楚州建中四年度屬泗州少始照翻盱音吁眙音怡}
聞朱泚作亂歸廣陵修塹壘繕甲兵浙江東西節度使韓滉閉關梁禁馬牛出境築石頭城穿井近百所繕館第數十修塢壁|{
	泚且禮翻又音此塹七艶翻滉呼廣翻近其靳翻通俗文營居曰塢壁壘也釋名曰壁辟也所以辟禦寇盗也}
起建業抵京峴|{
	京峴山在潤州州治東五里峴戶蹇翻}
樓堞相屬|{
	屬之欲翻聨屬也堞逹恊翻}
以備車駕度江且自固也少遊兵三千大閲於江北滉亦舟師三千曜武於京江以應之|{
	大江逕京口城北謂之京江}
鹽鐵使包佶|{
	佶巨乙翻}
有錢帛八百萬將輸京師陳少遊以為賊據長安未期收復|{
	言收復未有期也}
欲彊取之|{
	彊如字}
佶不可少遊欲殺之佶懼匿妻子於案牘中急濟江少遊悉收其錢帛 |{
	考異曰奉天記曰佶以財幣一百八十萬欲轉輸入城少遊彊收之今從舊傳}
佶有守財卒三千少遊亦奪之佶纔與數十人俱至上元復為韓滉所奪|{
	上元縣時帶昇州宋白曰上元縣晉江寧縣地貞觀七年移還舊郭即今所置縣也九年改為江寧縣玄宗置昇州因縣宇為州城縣元治鳳凰山南今移治會府時包佶盖在楊子廵院也史言天子播遷籓鎮阻兵陵轢王人復扶又翻}
時南方藩鎮各閉境自守惟曹王臯數遣使間道貢獻|{
	曹王臯時節度江南西道史言曹王臯悉心于帝室數所角翻使疏吏翻間古莧翻}
李希烈攻逼汴鄭江淮路絶朝貢皆自宣饒荆襄趣武關|{
	汴皮變翻朝直遥翻趣逡喻翻下同}
臯治郵驛平道路由是往來之使通行無阻|{
	此為江浙往來之使治直之翻郵音尤}
上問陸䞇以當今切務䞇以曏日致亂由上下之情

不通勸上接下從諫乃上疏|{
	上時掌翻疏所據翻}
其略曰臣謂當今急務在於審察羣情若羣情之所甚欲者陛下先行之所甚惡者陛下先去之|{
	此即孟子所欲與之聚之所惡勿施之意惡烏路翻下同去羌呂翻}
欲惡與天下同而天下不歸者自古及今未之有也夫理亂之本繫於人心|{
	夫音扶}
况乎當變故動揺之時在危疑向背之際|{
	背蒲妺翻}
人之所歸則植|{
	植立也}
人之所去則傾陛下安可不審察羣情同其欲惡使億兆歸趣以靖邦家乎|{
	趣嚮也}
此誠當今之所急也又曰頃者竊聞輿議|{
	輿衆也}
頗究羣情四方則患於中外意乖百辟又患於君臣道隔郡國之志不逹於朝廷朝廷之誠不升於軒陛上澤闕於下布下情壅於上聞實事不必知知事不必實上下否隔於其際眞偽雜糅於其間|{
	朝直遥翻否皮鄙翻糅女救翻}
聚怨囂囂騰謗籍籍欲無疑阻其可得乎又曰總天下之智以助聰明順天下之心以施教令則君臣同志何有不從遠邇歸心孰與為亂又曰慮有愚而近道|{
	近其靳翻}
事有要而似迃疏奏旬日上無所施行亦不詰問|{
	詰去吉翻}
䞇又上疏|{
	上疏音並同前}
其略曰臣聞立國之本在乎得衆得衆之要在乎見情|{
	言洞見人情也}
故仲尼以謂人情者聖王之田|{
	記禮運以為仲尼之言}
言理道所生也|{
	理道猶言治道也唐人避高宗諱率以治為理}
又曰易乾下坤上曰泰坤下乾上曰否損上益下曰益損下益上曰損夫天在下而地處上於位乖矣而反謂之泰者上下交故也君在上而臣處下|{
	否皮鄙翻下同夫音扶處昌呂翻}
於義順矣而反謂之否者上下不交故也上約已而裕於人人必說而奉上矣|{
	說讀曰悦}
豈不謂之益乎上蔑人而肆諸己人必怨而叛上矣豈不謂之損乎|{
	陸贄此言深究否泰損益之義誠足以箴砭德宗之失}
又曰舟即君道水即人情舟順水之道乃浮違則没君得人之情乃固失則危是以古先聖王之居人上也必以其欲從天下之心而不敢以天下之人從其欲|{
	祖左傳臧文仲所謂以欲從人則可以人從欲鮮濟之語之意}
又曰陛下憤習俗以妨理|{
	理治也言德宗憤強藩之跋扈習以成俗有妨為治}
任削平而在躬以明威照臨以嚴法制斷|{
	斷丁亂翻}
流弊自久浚恒太深|{
	易恒之初六曰浚恒貞凶無攸利象曰浚恒之凶始求深也王弼注曰始求深者求深窮底令物無餘藴漸以至此人猶不堪而况始求深者乎以此為恒無所施而利也}
遠者驚疑而阻命逃死之禍作近者畏懾而偷容避罪之態生|{
	懾質涉翻}
君臣意乖上下情隔君務致理而下防誅夷臣將納忠又上慮欺誕|{
	此數語亦深中當時君臣之病誕妄也}
故睿誠不布於羣物物情不逹於睿聰臣於往年曾任御史|{
	德宗初年陸贄為監察御史}
獲奉朝謁僅欲半年|{
	朝直遥翻}
陛下嚴邃高居未嘗降旨臨問|{
	此可以見德宗初年臨朝氣象}
羣臣跼蹐趨退|{
	跼音局蹐音脊}
亦不列事奏陳軒陛之間且未相諭宇宙之廣何由自通雖復例對使臣别延宰輔|{
	復扶又翻又音如字使疏吏翻例對使臣謂功臣節度及諸軍使待制者得隨例以次對也别延宰輔謂朝謁之外别延之與議天下事也復扶又翻}
既殊師錫|{
	書堯典師錫帝曰孔安國注云師衆也錫與也}
且異公言未行者則戒以樞密勿論己行者又謂之遂事不諫|{
	論語載孔子責宰我之言}
漸生拘礙動涉猜嫌由是人各隱情以言為諱至於變亂將起億兆同憂獨陛下恬然不知方謂太平可致|{
	德宗致亂之事誠如贄言}
陛下以今日之所覩驗往時之所聞孰眞孰虛何得何失則事之通塞備詳之矣|{
	塞悉則翻}
人之情偽盡知之矣上乃遣中使諭之曰朕本性甚好推誠|{
	好呼到翻}
亦能納諫將謂君臣一體全不隄防緣推誠不疑多被姦人賣弄今所致患害朕思亦無它其失反在推誠|{
	此德宗猜防之心於言而不能自掩者也被皮義翻}
又諫官論事少能慎密例自矜衒|{
	少詩沼翻衒音炫}
歸過於朕以自取名朕從卽位以來見奏對論事者甚多大抵皆是雷同道聽塗說|{
	孔子有言道聽而塗說德之棄也馬融注曰謂聞於道路則傳而說之}
試加質問遽即辭窮若有奇才異能在朕豈惜拔擢朕見從前已來事祗如此所以近來不多取次對人|{
	言次對人敷奏緣此多不取用其言或曰取次唐人語也}
亦非倦於接納卿宜深悉此意|{
	悉詳也}
贄以人君臨下當以誠信為本諫者雖辭情鄙拙亦當優容以開言路若震之以威折之以辯則臣下何敢盡言乃復上疏|{
	折之舌翻復扶又翻上時掌翻疏所據翻}
其略曰天子之道與天同方天不以地有惡木而廢生天子不以時有小人而廢聽納又曰唯信與誠有失無補|{
	言人君所為有失於誠信則無補於治道}
一不誠則心莫之保一不信則言莫之行陛下所謂失於誠信以致患害者臣竊以斯言為過矣又曰馭之以智則人詐示之以疑則人偷上行之則下從之上施之則下報之|{
	施式豉翻或讀如字}
若誠不盡於己而望盡於人衆必怠而不從矣不誠於前而曰誠於後衆必疑而不信矣是知誠信之道不可斯須而去身願陛下慎守而行之有加恐非所以為悔者也|{
	因德宗之言以為失在推誠故陸贄極言誠信之不可去身以開廣上意}
又曰臣聞仲虺贊揚成湯不稱其無過而稱其改過|{
	書仲虺之誥曰惟王改過不吝虺許偉翻}
吉甫歌誦周宣不美其無闕而美其補闕|{
	詩烝民曰衮職有闕惟仲山甫補之尹吉甫所以美宣王之任賢使能也}
是則聖賢之意較然著明惟以改過為能不以無過為貴蓋為人之行己必有過差|{
	蓋為于偽翻}
上智下愚俱所不免智者改過而遷善愚者恥過而遂非遷善則其德日新遂非則其惡彌積又曰諫官不密自矜信非忠厚其於聖德固亦無虧陛下若納諫不違則傳之適足增美陛下若違諫不納又安能禁之勿傳|{
	陸贄告君之言可謂深切著明}
又曰侈言無驗不必用|{
	德宗之信裴延齡以侈言也}
質言當理不必違|{
	德宗之罷柳渾以質言也當丁浪翻}
辭拙而效速者不必愚|{
	如蕭復之諫幸鳳翔是也}
言甘而利重者不必智|{
	趙贊竇滂之苛征重歛是也}
是皆考之以實慮之以終其用無它唯善所在又曰陛下所謂比見奏對論事皆是雷同道聽塗說者|{
	比毘至翻}
臣竊以衆多之議足見人情必有可行亦有可畏恐不宜一槩輕侮而莫之省納也|{
	省悉景翻察也}
陛下又謂試加質問即便辭窮臣但以陛下雖窮其辭而未窮其理能服其口而未服其心|{
	但以若依上文作竊以又覺文從字順}
又曰為下者莫不願忠為上者莫不求理然而下每苦上之不理上每苦下之不忠若是者何兩情不通故也下之情莫不願逹於上上之情莫不求知於下然而下恒苦上之難逹上恒苦下之難知|{
	恒戶登翻}
若是者何九弊不去故也所謂九弊者上有其六而下有其三好勝人|{
	好呼到翻下同}
恥聞過騁辯給眩聰明厲威嚴恣彊愎|{
	愎符逼翻狠也}
此六者君上之弊也諂諛顧望畏愞|{
	愞奴亂翻}
此三者臣下之弊也上好勝必甘於佞辭上恥過必忌於直諫如是則下之諂諛者順指而忠實之語不聞矣上騁辯必勦說而折人以言|{
	勦初交翻又初教翻此所謂勦說者人言未竟勦絶其說而伸己之說也折之舌翻}
上眩明必臆度而虞人以詐|{
	度徒洛翻以胸臆之見料度人}
如是則下之顧望者自便而切磨之辭不盡矣上厲威必不能降情以接物上恣愎必不能引咎以受規如是則下之畏愞者避辜而情理之說不申矣夫以區域之廣大生靈之衆多宫闕之重深|{
	夫音扶重直龍翻}
高卑之限隔自黎獻而上獲覩至尊之光景者踰億兆而無一焉|{
	黎獻衆賢也}
就獲覩之中得接言議者又千萬不一幸而得接者猶有九弊居其間則上下之情所通鮮矣|{
	鮮息淺翻}
上情不通於下則人惑下情不通於上則君疑疑則不納其誠惑則不從其令誠而不見納則應之以悖令而不見從則加之以刑下悖上刑不敗何待|{
	悖蒲内翻又蒲没翻}
是使亂多理少從古以然|{
	少始紹翻或為從古以然當作從古而然今觀文意陸宣公所謂從古至今亂多治少者正以下悖上刑故也以之與而辭義相去遠矣}
又曰昔趙武呐呐而為晉賢臣|{
	晉趙文子名武其言呐呐然如不出其口為晉正卿晉國以彊諸侯不叛呐呐舒小貌音如悦翻又奴劣翻}
絳侯木訥而為漢元輔|{
	絳侯事見漢文帝紀程氏曰木者質樸訥者遲鈍}
然則口給者事或非信辭屈者理或未窮人之難知堯舜所病|{
	書臯陶曰在知人在安民禹曰吁惟帝其難之}
胡可以一詶一詰而謂盡其能哉|{
	詰去吉翻}
以此察天下之情固多失實以此輕天下之士必有遺才|{
	德宗所以成段平仲之名者正如此}
又曰諫者多表我之能好諫者直示我之能容諫者之狂誣明我之能恕諫者之漏泄彰我之能從|{
	極言納諫之美以誘掖其君上也好呼到翻}
是則人君與諫者交相益之道也諫者有爵賞之利君亦有理安之利諫者得獻替之名君亦得采納之名然猶諫者有失中而君無不美唯恐讜言之有不切天下之不聞如此則納諫之德光矣|{
	讜音黨}
上頗采用其言 李懷光頓兵不進數上表暴揚盧等罪惡|{
	數所角翻上時掌翻}
衆論諠騰亦咎等上不得已十二月壬戌貶為新州司馬白志貞為恩州司馬|{
	恩州屬漢合浦郡地蕭齊為齊安郡隋廢郡為海安縣唐貞觀二十三年以高州之西平海安杜陵置恩州海安改曰恩平天寶曰恩平郡乾元復為恩州宋平王則改貝州曰恩州遂以此州為南恩州宋白為此恩州瀕海最為蒸濕當海南五郡汎海路此路自廣汎海行數日方登陸人惮海波不由此路多由新州陸去唯健步出使與逓符牒經過耳新州治新興縣秦取陸梁地置象郡今州即其地晉永和分蒼梧郡於此置新寧郡梁武帝立新州所謂新興縣漢合浦郡臨元縣也又按舊志云恩州京師東南六千六百里西北六十里接廣州界新州至京師五千五十二里}
趙贊為播州司馬|{
	播州隋牂柯縣京師南四千四百五十里}
宦者翟文秀上所信任也|{
	翟萇伯翻}
懷光又言其罪上亦為殺之|{
	亦為于偽翻}
乙丑以翰林學士祠部員外郎陸䞇為考功郎中金部員外郎吴通微為職方郎中|{
	祠部屬禮部掌祠部考功屬吏部掌文武官功過考法以官職言之祠部比考功職方為清要郎中正五品上員外以從六品上}
䞇上奏辭以初到奉天扈從將吏|{
	上時掌翻從才用翻}
例加兩階今翰林獨遷官|{
	唐自至德以後勲階輕而職事官重故云然}
夫行罰先貴近而後卑遠則令不犯行賞先卑遠而後貴近則功不遺|{
	夫音扶先悉薦翻後戶構翻}
望先録大勞次徧羣品則臣亦不敢獨辭上不許上在奉天使人說田悦王武俊李納赦其罪|{
	說式芮翻 考}


|{
	異曰燕南記十二月二十四日前已云赦武俊等罪而實録明年正月改元乃赦武俊等蓋上先已諭旨赦罪及赦書出始明言之耳}
厚賂以官爵悦等皆密歸欵而猶未敢絶朱滔各稱王如故滔使其虎牙將軍王郅說悦曰|{
	朱滔等倣漢官置虎牙將軍按唐書滔等之稱王也以左將軍曰虎牙右將軍曰豹畧徵以新書虎牙將軍盖王郅也}
日者八郎有急淊與趙王不敢愛其死竭力赴救幸而解圍|{
	田悦第八解圍事見二百一十七卷三年}
今太尉三兄受命關中|{
	朱泚第三}
淊欲與回紇共往助之願八郎治兵與淊度河共取大梁|{
	紇下没翻治直之翻大梁汴州宣武節度治所}
悦心不欲行而未忍絶淊乃許之淊復遣其内史舍人李琯見悦審其可否|{
	内史舍人猶天朝中書舍人復扶又翻琯古緩翻}
悦猶豫不决密召扈㠋議之司武侍郎許士則曰|{
	司武侍郎猶天朝兵部侍郎也}
朱淊昔事李懷仙為牙將與兄泚及朱希彩共殺懷仙而立希彩|{
	將即亮翻泚且禮翻又音如字殺李懷仙事見二百二十四卷代宗大歷三年 考異曰燕南記作朱寀今從舊傳}
希彩所以寵信其兄弟至矣淊又與判官李子瑗謀殺希彩而立泚|{
	事見二百二十四卷大歷七年瑗于眷翻}
泚旣為帥|{
	帥所類翻}
淊乃勸泚入朝而自為留後|{
	事見二百二十五卷大歷九年}
雖勸以忠義實奪之權也平生與之同謀共功如李子瑗之徒負而殺之者二十餘人今又與泚東西相應使滔得志泚亦不為所容况同盟乎滔為人如此大王何從得其肺腑而信之邪|{
	觀時審勢量度彼已世不為無其人特其言有用不用耳泚且禮翻又音此邪音耶}
彼引幽陵回紇十萬之兵屯於郊坰|{
	紇下没翻幽陵即幽州坰古熒翻邑外謂之郊野外謂之林林外謂之坰}
大王出迎則成擒矣彼囚大王兼魏國之兵南向渡河與關中相應天下其孰能當之大王於時悔之無及為大王計不若陽許偕行而隂為之備厚加迎勞|{
	勞力到翻}
至則託以它故遣將分兵而隨之如此大王外不失報德之名而内無倉猝之憂矣扈㠋等皆以為然王武俊聞李琯適魏遣其司刑員外郎田秀馳見悦曰|{
	㠋五各翻琯古瑗翻司刑員外郎猶天朝刑部員外郎}
武俊曏以宰相處事失宜|{
	相息亮翻處昌呂翻}
恐禍及身又八郎困於重圍|{
	重直龍翻}
故與滔合兵救之今天子方在隱憂以德綏我我曹何得不悔過而歸之邪捨九葉天子不事而事滔乎|{
	自高祖太宗高宗中宗睿宗玄宗肅宗代宗至帝凡九葉}
且泚未稱帝之時滔與我曹比肩為王固已輕我曹矣|{
	事見上卷本年}
况使之南平汴洛與泚連衡|{
	汴皮變翻汴州宣武軍洛州東都也衡讀曰横}
吾屬皆為虜矣八郎慎勿與之俱南但閉城拒守武俊請伺其隙連昭義之兵擊而滅之|{
	伺相吏翻}
與八郎再清河朔復為節度使共事天子不亦善乎|{
	復扶又翻又音如字使疏吏翻}
悦意遂决紿滔云從行必如前約丁卯滔將范陽步騎五萬人私從者復萬餘人|{
	紿蕩亥翻將即亮翻又音如字騎奇寄翻私從才用翻復扶又翻}
回紇三千人河間而南輜重首尾四十里|{
	紇下没翻瀛州治河間縣重直用翻}
李希烈攻李勉於汴州|{
	李勉以宣武節度使鎮汴州}
驅民運土木築壘道以攻城忿其未就并人填之謂之濕薪勉城守累月外救不至將其衆萬餘人奔宋州|{
	將音同上勉奔宋州依劉洽也}
庚午希烈䧟大梁滑州刺史李澄以城降希烈希烈以澄為尚書令兼永平節度使勉上表請罪|{
	滑州治白馬縣降戶江翻尚辰羊翻上時掌翻}
上謂其使者曰朕猶失守宗廟勉宜自安待之如初劉洽遣其將高翼將精兵五千保襄邑|{
	九域志襄邑在汴州東南一百七十里}
希烈攻拔之翼赴水死希烈乘勝攻寧陵|{
	九域志寧陵縣在宋州西四十五里}
江淮大震陳少遊遣參謀温述送欵於希烈曰濠壽舒廬已令弛備韜戈卷甲伏俟指麾又遣廵官趙詵結李納於鄆州|{
	少始照翻濠夀舒廬四州之地在淮蔡東南送欵遂言使弛備令力丁翻使也卷讀與捲同詵疎臻翻鄆音運}
中書侍郎同平章事關播罷為刑部尚書 以給事中孔巢父為淄青宣慰使國子祭酒董晉為河北宣慰使|{
	宣慰者宣上命以慰安反側也父音甫淄莊持翻}
陸䞇言於上曰今盗遍天下輿駕播遷陛下宜痛自引過以感人心昔成湯以罪已勃興|{
	左傳臧文仲曰禹湯罪已其興也勃焉}
楚昭以善言復國|{
	楚昭王遭闔閭之禍國滅出亡父老送之王曰父老反矣何患無君父老曰有君如是其賢也相與從之或奔走赴秦號哭請救秦人憐之為之出兵二國并力遂走吴師昭王復國}
陛下誠能不吝改過以言謝天下使書詔無所避忌臣雖愚陋可以仰副聖情庶令反側之徒革心向化上然之故奉天所下書詔雖驕將悍卒聞之無不感激揮涕|{
	令力丁翻下遐稼翻將即亮翻}
術者上言國家厄運宜有變更以應時數|{
	上時掌翻下䞇上音同更工衡翻}
羣臣請更加尊號一二字上以問䞇䞇上奏以為不可其略曰尊號之興本非古制|{
	上尊號事始於開元五年}
行於安泰之日已累謙冲|{
	累力瑞翻}
襲乎喪亂之時|{
	喪息浪翻}
尤傷事體又曰嬴秦德衰兼皇與帝始總稱之|{
	見七卷秦始皇二十六年}
流及後代昏僻之君乃有聖劉天元之號|{
	聖劉見三十四卷漢哀帝建平二年天元見一百七十三卷陳宣帝太建十一年}
是知人主輕重不在名稱|{
	稱尺證翻下美稱同}
損之有謙光稽古之善崇之獲矜能納謟之譏又曰必也俯稽術數須有變更|{
	更工衡翻}
與其增美稱而失人心不若黜舊號以祗天戒上納其言但改年號而已|{
	謂改明年號為興元也}
上又以中書所撰赦文示䞇|{
	撰如免翻}
䞇上言以為動人以言所感己淺言又不切人誰肯懷今兹德音悔過之意不得不深引咎之辭不得不盡洗刷疵垢宣暢鬱堙|{
	疵才支翻}
使人人各得所欲則何有不從者乎應須改革事條謹具别狀同進捨此之外尚有所虞竊以知過非難改過為難言善非難行善為難假使赦文至精止於知過言善猶願聖慮更思所難上然之

興元元年春正月癸酉朔赦天下改元制曰致理興化必在推誠忘已濟人不吝改過朕嗣服丕構|{
	丕大也構立屋也書大誥曰若考作室既底法厥子乃弗肯堂矧肯構丕構之語本諸此}
君臨萬邦失守宗祧|{
	宗者百世不毁之廟遠廟為祧祧他彫翻}
越在草莾|{
	用左傳語}
不念率德誠莫追於既往永言思咎期有復於將來明徵其義以示天下|{
	徵證也明徵其義言無所掩覆也}
小子懼德弗嗣|{
	懼已德弗能嗣先業嗣祥吏翻}
罔敢怠荒然以長于深宫之中|{
	用禮記魯哀公之言長知丈翻}
暗於經國之務積習易溺|{
	易以豉翻}
居安忘危不知稼穡之艱難|{
	書無逸周公告成王之語}
不恤征戍之勞苦澤靡下究情未上通事既擁隔人懷疑阻猶昩省已|{
	擁恐當作壅省悉景翻}
遂用興戎|{
	戎兵也}
徵師四方轉餉千里賦車籍馬遠近騷然行齎居送衆庶勞止或一日屢交鋒刃或連年不解甲胄祀奠乏主室家靡依死生流離怨氣凝結力役不息田萊多荒|{
	鄭玄曰田萊多荒茨棘不除也陸德明曰田廢生草曰萊}
暴令峻於誅求疲甿空於杼軸|{
	詩小東大東杼軸其空杼持緯器布帛已織成者以機軸卷之}
轉死溝壑離去鄉閭|{
	離力智翻}
邑里丘墟人煙斷絶天譴於上而朕不寤人怨於下而朕不知馴致亂階變興都邑|{
	馴從也言從此而致亂也}
萬品失序九廟震驚|{
	歐陽修曰書云七世之廟可以觀德而禮家之說世數不同然自禮記王制祭法禮器大儒荀卿劉歆班固王肅之徒以為七廟者多盖自漢魏以來創業之君特起其上世又微無功德以備祖宗故其初皆不能立七廟唐武德元年始立四廟高祖崩朱子奢請立七廟虛太祖之室以待尚書入座議禮曰天子三昭三穆與太祖之廟而七晉宋齊梁皆立親廟六此故事也於是宣簡公懿王景元二帝四廟更祔弘農府君及高祖為六室太宗崩弘農以世遠毁而祔太宗高宗崩又遷宣簡而祔高宗皆為六室中宗神龍初以景帝為始祖而元帝不遷而祔孝敬帝由是爲七室中宗崩孝敬别立廟而祔中宗遂為七室至睿宗崩中宗立别廟而祔睿宗開元十年詔宣皇帝復祔正室諡爲獻祖并諡光帝爲懿祖又以中宗遷祔太廟於是太廟爲九室寶應二年祧獻懿而祔玄宗肅宗代宗崩又遷元皇帝而祔代宗自是常為九室}
上累于祖宗|{
	累力瑞翻}
下負于蒸庶痛心貌|{
	他典翻慙恧也}
罪實在予永言愧悼若墜泉谷|{
	唐避高祖諱改淵為泉}
自今中外所上書奏不得更言聖神文武之號|{
	建中元年羣臣上尊號曰聖神文武皇帝見二百二十六卷}
李希烈田悦王武俊李納等咸以勲舊各守藩維朕撫御乖方致其疑懼皆由上失其道而下罹其災朕實不君人則何罪|{
	此等言語強藩悍將聞之宜其感服易心}
宜并所管將吏等一切待之如初朱滔雖緣朱泚連坐路遠必不同謀念其舊勲務在弘貸|{
	弘大也}
如能效順亦與惟新朱泚反易天常|{
	君臣上下天秩有典之常也}
盗竊名器暴犯陵寢所不忍言獲罪祖宗朕不敢赦|{
	此等言語可與誥誓相表裏}
其脅從將吏百姓等但官軍未到京城以前去逆效順并散歸本道本軍者並從赦例|{
	所以携從逆之黨將即亮翻下同}
諸軍諸道應赴奉天及進收京城將士並賜名奉天定難功臣|{
	所以作勤王之心難乃旦翻}
其所加墊陌錢税間架竹木茶漆榷鐵之類悉宜停罷|{
	所以順人情之欲惡墊陌錢即趙贊所行除陌錢也墊丁念翻榷古岳翻}
赦下四方人心大悦及上還長安明年|{
	上還長安之明年貞元元年也下遐嫁翻還從宣翻又音如字}
李抱眞入朝為上言|{
	朝直遥翻為于偽翻}
山東宣布赦書士卒皆感泣臣見人情如此知賊不足平也|{
	史究言興元赦書感動人心之效}
命兵部員外郎李充為恒冀宣慰使|{
	唐兵部員外郎二人一人掌貢舉雜請一人判南曹歲選出使非本職命以即官出使耳恒戶登翻使疏吏翻}
朱泚更國號曰漢|{
	泚且禮翻音此朱泚初僭號國號秦更工衡翻}
自號漢元天皇改元天皇 王武俊田悦李納見赦令皆去王號|{
	去羌呂翻}
上表謝罪|{
	上時掌翻}
惟李希烈自恃兵彊財富遂謀稱帝遣人問儀於顔眞卿眞卿曰老夫嘗為禮官所記惟諸侯朝天子禮耳|{
	顔眞卿所以答李希烈者辭不廹切而義甚嚴正朝直遥翻}
希烈遂即皇帝位 |{
	考異曰希烈稱帝實録舊希烈傳顔眞卿傳皆無年月今據奉天記幸奉天録皆云赦令既行諸方莫不向化惟李希烈長惡不悛國號大楚又實録今年閏月庚午詔曰朕苟存拯物不憚屈身故於歲首特布新令赦其殊死待以初誠使臣纔及於郊畿巨猾已聞於僭竊然則希烈稱帝必在正月初也}
國號大楚改元武成置百官以其黨鄭賁為侍中孫廣為中書令李緩李元平同平章事|{
	李緩新書作李綬}
以汴州為大梁府分其境内為四節度希烈遣其將辛景臻謂顔眞卿曰不能屈節當自焚積薪灌油於其庭眞卿趨赴火景臻遽止之希烈又遣其將楊峯|{
	將即亮翻 考異曰舊傳作楊豐今從奉天記}
齎赦賜陳少遊及夀州刺史張建封建封執峯徇於軍腰斬於市少遊聞之駭懼建封具以少遊與希烈交通之狀聞上悦以建封為濠壽廬三州都團練使|{
	少始照翻使疏吏翻}
希烈乃以其將杜少誠為淮南節度使使將步騎萬餘人先取壽州後之江都|{
	使將即亮翻又音如字騎奇寄翻夀州治夀春縣之往也淮南節度治江都}
建封遣其將賀蘭元均邵怡守霍丘秋柵|{
	後周書賀蘭祥傳其先與後魏俱起有紇伏者為賀蘭莫弗遂以為氏霍丘漢廬江松滋縣地梁置安豐郡東魏廢郡隋開皇十六年置霍丘縣唐屬夀州九域志在州東一百二十里宋白曰霍丘本春秋時蓼國梁置霍丘戍隋廢戍為縣}
少誠竟不能過遂南寇蘄黄欲斷江路|{
	蘄渠希翻斷音短}
時上命包佶自督江淮財賦泝江詣行在至蘄口|{
	水經注蘄水源出蘄春縣北大浮山南過其縣西又南至蘄口入于江佶其吉翻泝蘇故翻}
遇少誠入寇曹王臯遣蘄州刺史伊慎將兵七千拒之戰於永安戍|{
	永安戍在黄州黄岡縣界梁置永安郡後廢為戍}
大破之少誠脱身走斬首萬級包佶乃得前後佶入朝具奏陳少遊奪財賦事|{
	奪財賦事見上年佶巨乙翻朝直遥翻}
少遊懼厚歛所部以償之|{
	歛力贍翻}
李希烈以夏口上流要地|{
	卾州治夏口當江漢之會夏戶雅翻}
使其驍將董侍募死士七千襲卾州刺史李兼偃旗卧鼓閉門以待之侍撤屋材以焚門兼帥士卒出戰大破之|{
	驍堅堯翻將即亮翻鄂逆各翻鄂州治江夏縣即夏口帥讀曰率}
上以兼為鄂岳沔都團練使|{
	沔彌兖翻使疏吏翻}
於是希烈東畏曹王臯西畏李兼不敢復有窺江淮之志矣|{
	史言李希烈兵勢稍挫復扶又翻}
朱滔引兵入趙境王武俊大具犒享|{
	犒口到翻}
入魏境田

悦供承倍豐使者迎候相望於道丁丑滔至永濟|{
	宋白曰永濟縣本漢貝丘縣地隋已後為臨清縣地大歷七年田承嗣奏分臨清置永濟縣屬貝州以縣西臨永濟渠為名}
遣王郅見悦約會館陶偕行度河|{
	館陶縣屬魏州在州城東稍北}
悦見郅曰悦固願從五兄南行昨日將出軍將士勒兵不聽悦出曰國兵新破|{
	謂先為馬燧等所破也}
戰守踰年資儲竭矣|{
	謂守魏州與馬燧等相持也}
今將士不免凍餒何以全軍遠征大王日自撫循猶不能安若捨城邑而去朝出暮必有變悦之志非敢有貳也如將士何已令孟祐備步騎五千從五兄供芻牧之役|{
	騎奇寄翻}
因遣其司禮侍郎裴抗等往謝滔|{
	司禮侍郎猶天朝禮部侍郎}
滔聞之大怒曰田悦逆賊曏在重圍|{
	重直龍翻}
命如絲髪使我叛君弃兄兵晝夜赴之|{
	事見二百二十七卷建中三年}
幸而得存許我貝州我辭不取尊我為天子我辭不受|{
	事見同上年}
今乃負恩誤我遠來飾辭不出即日遣馬寔攻宗城經城|{
	經城漢古縣時屬貝州宋白曰後漢分前漢堂陽縣於今縣西北二十里置經縣後魏書併南宫縣太和十年又於今理置經縣尋置廣宗郡於此北齊省郡及縣移武彊縣於此後周復於此置廣宗郡隋開皇三年罷郡復於此置經城縣宋省縣為鎮入宗城}
楊榮國攻冠氏|{
	去年張孝宗遣其將楊榮國與李晟俱赴國難及晟收京城諸將中獨楊榮國不見於史今朱滔遣楊榮國攻冠氏乃建中三年以深州降于朱滔者冠氏春秋邑名隋分館陶東界置冠氏縣唐屬魏州九域志在州東北六十里}
皆拔之又縱囘紇掠館陶頓幄帟器皿車牛以去|{
	紇下没翻帟音亦三禮圖在上曰帟四旁及上曰帷上下四旁悉周曰幄又曰帟平帳也帟主在幕若幄中坐上承塵}
悦閉城自守壬午滔遣裴抗等還|{
	還從宣翻又音如字}
分兵置吏守平恩永濟|{
	平恩縣屬洺州治平恩州}
丙戌以吏部侍郎盧翰為兵部侍郎同平章事 |{
	考異曰實録新舊紀表皆同盖翰罷領選故自吏部遷兵部耳}
翰義僖之七世孫也|{
	盧義僖仕元魏當靈后臨朝時不附徐鄭}
朱滔引兵北圍貝州引水環之|{
	環音宦}
刺史邢曹俊嬰城拒守縱范陽及囘紇兵大掠諸縣|{
	滔縱兵大掠}
又拔武城|{
	武城即漢東武城縣地唐屬貝州九域志在州東五十里}
通德棣二州使給軍食|{
	建中二年朱滔據有德棣}
遣馬寔將步騎五千屯冠氏以逼魏州 以給事中杜黄裳為江淮宣慰副使 |{
	考異曰實録去年十二月癸酉已云黄裳使江淮此又有之按舊紀去年十二月黄裳為給事耳實録誤也}
上於行宫廡下貯諸道貢獻之物牓曰瓊林大盈庫|{
	貯直呂翻}
陸䞇以為戰守之功賞賚未行而遽私别庫則士卒怨望無復鬭志上疏諫|{
	復扶又翻又音如字上時掌翻疏所據翻}
其略曰天子與天同德以四海為家何必橈廢公方|{
	橈奴教翻屈曲也方法也}
崇聚私貨降至尊而代有司之守辱萬乘以效匹夫之藏|{
	乘繩證翻}
虧法失人誘姦聚慝以斯制事豈不過哉|{
	誘羊又翻慝吐得翻}
又曰頃者六師初降|{
	降讀如字天子之行必有六師以為營衛不敢指言自京師出居奉天故微其辭曰六師初降}
百物無儲外扞兇徒内防危堞晝夜不息殆將五旬凍餒交侵死傷相枕|{
	堞逹協翻枕職任翻}
畢命同力竟夷大艱良以陛下不厚其身不私其欲絶甘以同卒伍輟食以㗖功勞|{
	㗖徒濫翻又徒覽翻}
無猛制而人不擕懷所感也無厚賞而人不怨悉所無也|{
	悉詳體也}
今者攻圍已解衣食已豐而謡讟方興|{
	讟怨謗也}
軍情稍阻豈不以勇夫恒性嗜利矜功|{
	恒戶登翻}
其患難既與之同憂而好樂不與之同利|{
	難乃旦翻好呼到翻樂音洛}
苟異恬默能無怨咨|{
	咨咨嗟也}
又曰陛下誠能近想重圍之殷憂|{
	重直用翻殷於謹翻}
追戒平居之專欲凡在二庫貨賄盡令出賜有功每獲珍華|{
	令力丁翻珍華猶言珍麗也}
先給軍賞如此則亂必靖賊必平徐駕六龍旋復都邑天子之貴豈當憂貧是乃散其小儲而成其大儲損其小寶而固其大寶也上即命去其牓|{
	去羌呂翻}
蕭復嘗言於上曰宦官自艱難以來多為監軍恃恩縱横|{
	監工銜翻横戶孟翻}
此屬但應掌宫掖之事不宜委以兵權國政上不悦又嘗言陛下踐阼之初聖德光被|{
	應乙陵翻當也掖音亦被皮義翻}
自楊炎盧黷亂朝政以致今日|{
	朝直遥翻}
陛下誠能變更睿志臣敢不竭力|{
	此必盧貶逐之後蕭復方有是言更工衡翻}
儻使臣依阿苟免臣實不能|{
	蕭復盖撲而直者}
又嘗與盧同奏事順上旨復正色曰盧言不正上愕然退謂左右曰蕭復輕朕|{
	此事必在蕭復盧同列之時史因德宗命復出使而序其事於此耳}
戊子命復充山南東西荆湖淮南江西鄂岳浙江東西福建嶺南等道宣慰安撫使實踈之也|{
	鄂五各翻使疏吏翻}
既而劉從一及朝士往往奏留復上謂陸䞇曰朕思遷幸以來江淮遠方或傳聞過實欲遣重臣宣慰謀於宰相及朝士僉謂宜然今乃反覆如是朕為之悵恨累日|{
	朝直遥翻相息亮翻為于偽翻}
意復悔行使之論奏邪|{
	意者以意度之也此亦德宗猜防臣下之一事}
卿知蕭復何如人其不欲行意趣安在䞇上奏以為復痛自脩勵慕為清貞用雖不周行則可保|{
	上時掌翻行下孟翻}
至於輕詐如此復必不為借使復欲逗留從一安肯附會今所言矛楯|{
	韓非子有鬻矛楯者自譽其矛曰吾矛之利物無不䧟也又自譽其楯曰吾楯之堅物莫能䧟也或謂之曰以子之矛䧟子之楯可乎其人不能答故後世謂議論自相反及為事自相反者為自相矛楯楯食尹翻}
願陛下明加辯詰|{
	詰去吉翻}
若蕭復有所請求則從一何容為隱|{
	為于偽翻}
若從一自有囘互則蕭復不當受疑陛下何憚而不辯明乃直為此悵恨也夫明則罔惑辯則罔寃惑莫甚於逆詐而不與明|{
	夫音扶逆者未至而迎之也詐謂人欺己也未見其詐而逆以為詐謂之逆詐}
寃莫痛於見疑而不與辯是使情偽相糅|{
	糅女救翻}
忠邪靡分兹實居上御下之要樞惟陛下留意上亦竟不復辯|{
	復扶又翻}
辛卯以王武俊為恒冀深趙節度使壬辰加李抱眞張孝忠並同平章事丙申加田悦檢校左僕射|{
	恒戶登翻使疏吏翻校古效翻射寅謝翻}
以山南東道行軍司馬樊澤為本道節度使前深趙觀察使康日知為同州刺史奉誠軍節度使|{
	以趙州與王武俊故徙康日知乾元初以同州為匡國軍節度使今又為奉誠軍}
曹州刺史李納為鄆城刺史平盧節度使|{
	李納本為曹州刺史建中二年其父正已卒納自領軍務未有朝命今方命以旌節故先叙其本職而加以新命鄆音運}
戊戍加劉洽汴滑宋亳都統副使知都統事李勉悉以其衆授之|{
	李勉既失守汴州命劉洽知都統事汴皮變翻統他綜翻俗多從上聲}
辛丑六軍各置統軍|{
	此北門左右羽林龍武神武六軍也 考異曰實録云詔六軍各置軍使一員又云因置統軍按舊紀獨置統軍耳今從之}
秩從三品以寵勲臣|{
	從才用翻}
吐蕃尚結贊請出兵助唐收京城庚子遣祕書監崔漢衡使吐蕃其兵|{
	吐從暾入聲}


資治通鑑卷二百二十九
