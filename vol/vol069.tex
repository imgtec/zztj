資治通鑑卷六十九   宋 司馬光 撰

胡三省 音註

魏紀一|{
	起上章困敦盡玄黓攝提格凡三年 操破袁尚得冀州遂居於鄴鄴漢之魏郡治所魏大名也遂封為魏公又䜟云代漢者當塗高當塗高者魏也文帝受漢禪國遂號魏}


世祖文皇帝上|{
	諱丕字子桓武王操長子也謚法學勤好問曰文世祖廟號也禮祖有功而宗有德諡法景物四方曰世承命不遷曰世靖民則法曰皇明一德者曰皇明一合道曰皇德象天地曰帝案道無為曰帝}


黄初元年|{
	魏受漢禪推五德之運以土繼火土色黄故紀元曰黄初是年十月受禪方改元}
春正月武王至洛陽庚子薨|{
	魏王操謚曰武}
王知人善察難眩以偽|{
	眩者目無常主難眩以偽謂人不能亂其明}
識拔奇才不拘微賤隨能任使皆獲其用與敵對陳|{
	陳讀曰陣}
意思安閒|{
	思相吏翻}
如不欲戰然及至決機乘勝氣埶盈溢勲勞宜賞不吝千金無功望施|{
	施式䜴翻}
分豪不與|{
	豪即亳字}
用法峻急有犯必戮或對之流涕然終無所赦雅性節儉不好華麗|{
	好呼到翻}
故能芟刈羣雄幾平海内|{
	曰幾者以不能并吳蜀也芟所銜翻幾居希翻}
是時太子在鄴軍中騷動羣僚欲祕不發喪諫議大夫賈逵以為事不可祕乃發喪或言宜易諸城守悉用譙沛人|{
	曹氏沛國譙人小見者以鄉人為可信也守式又翻下同}
魏郡太守廣陵徐宣厲聲曰今者遠近一統人懷效節何必專任譙沛以沮宿衛者之心乃止|{
	沮在呂翻}
青州兵擅擊鼓相引去|{
	青州兵獻帝初平三年操破黄巾所降者}
衆人以為宜禁止之不從者討之賈逵曰不可為作長檄令所在給其稟食|{
	為于偽翻下上為下為同稟讀曰廩食如字長檄猶今軍行所至幇劵也}
鄢陵侯彰從長安來赴|{
	操自漢中還師而東彰定代而西迎操因留彰長安鄢陸德明謁晩翻又於建翻師古音偃}
問逵先王璽綬所在|{
	璽斯氏翻綬音受}
逵正色曰國有儲副先王璽綬非君侯所宜問也凶問至鄴太子號哭不巳|{
	號戶刀翻}
中庶子司馬孚諫曰|{
	續漢志太子中庶子秩六百石職如侍中}
君王晏駕天下恃殿下為命當上為宗廟下為萬國奈何效匹夫孝也太子良久乃止曰卿言是也時羣臣初聞王薨相聚哭無復行列|{
	行戶剛翻}
孚厲聲於朝曰|{
	朝直遙翻}
今君王違世天下震動當早拜嗣君以鎮萬國而但哭邪乃罷羣臣備禁衛治喪事孚懿之弟也|{
	治直之翻}
羣臣以為太子即位當須詔命|{
	謂須待漢帝詔命也}
尚書陳矯曰王薨於外天下惶懼太子宜割哀即位以繫遠近之望且又愛子在側|{
	愛子謂鄢陵侯彰也}
彼此生變則社稷危矣即具官備禮一日皆辨|{
	辨與辦同蜀本作辦}
明旦以王后令策太子即王位大赦漢帝尋遣御史大夫華歆奉策詔授太子丞相印綬魏王璽綬領冀州牧|{
	華戶化翻}
於是尊王后曰王太后改元延康|{
	此漢改元魏志也}
二月丁未朔日有食之 壬戌以太中大夫賈詡為太尉御史大夫華歆為相國大理王朗為御史大夫 丁卯葬武王于高陵|{
	高陵在鄴城西操遺令曰汝等時時登銅雀臺望吾西陵墓田魏紀載操令曰規西門豹祠西原上為陵}
王弟鄢陵侯彰等皆就國臨菑監國謁者灌均希指奏臨菑侯植醉酒悖慢刼脅使者|{
	時禁切藩侯使謁者監其國監古銜翻悖蒲内翻又蒲没翻}
王貶植為安鄉侯誅右刺姦掾沛國丁儀|{
	王莽置左右刺姦以督姦猾光武中興亦置刺姦將軍然公府掾無其員也魏晉公府始有營軍刺姦等員掾俞絹翻}
及弟黄門侍郎廙并其男口|{
	并男口誅之絶其世也廙逸職翻又羊至翻}
皆植之黨也魚豢論曰諺言貧不學儉卑不學恭非人性分殊也|{
	分扶問翻}
埶使然耳假令太祖防遏植等在於疇昔此賢之心何緣有窺望乎彰之挾恨尚無所至至於植者豈能興難|{
	難乃旦翻}
乃令楊修以倚注遇害丁儀以希意族滅哀夫

初置散騎常侍侍郎各四人|{
	散騎常侍秦官也秦置散騎又置中常侍散騎騎從乘輿車後中常侍得入禁中皆以為加官漢東京初省散騎而中常侍用宦者至是初置散騎合之於中常侍為一官曰散騎常侍掌規諫不典事貂璫揷右騎而散從後遂為顯職散騎侍郎自魏至晉與散騎常侍侍中黄門侍郎共平尚書奏事江左乃罷}
其宦人為官者不得過諸署令|{
	謂左右中尚方中黄左右藏左校甄官奚官黄門掖庭永巷御府鉤盾中藏府内者等署也}
為金策藏之石室時當選侍中常侍王左右舊人諷主者便欲就用不調餘人|{
	調徒弔翻}
司馬孚曰今嗣王新立當進用海内英賢如何欲因際會自相薦舉邪官失其任得者亦不足貴也遂他選 尚書陳羣以天朝選用不盡人才|{
	天朝謂漢朝也朝直遙翻}
乃立九品官人之灋州郡皆置中正以定其選擇州郡之賢有識鑒者為之區别人物第其高下|{
	九品中正自此始九品上上上中上下中上中中中下下上下中下下也别彼列翻}
夏五月戊寅漢帝追尊王祖太尉曰太王|{
	王祖漢太尉曹嵩也}
夫人丁氏曰太王后 王以安定太守鄒岐為涼州刺史西平麴演結旁郡作亂以拒岐張掖張進執太守杜通酒泉黄華不受太守辛機皆自稱太守以應演|{
	誅韓遂者麴演也蓋威行涼部久矣故進等皆應之}
武威三種胡復叛|{
	種章勇翻復扶又翻}
武威太守母丘興|{
	母丘複姓也}
告急於金城太守護羌校尉扶風蘇則則將救之郡人皆以為賊埶方盛宜須大軍時將軍郝昭魏平先屯金城受詔不得西度|{
	金城與武威張掖酒泉隔河}
則乃見郡中大吏及昭等謀曰今賊雖盛然皆新合或有脅從未必同心因釁擊之善惡必離離而歸我我增而彼損矣既獲益衆之實且有倍氣之勢率以進討破之必矣若待大軍曠日彌久善人無歸必合於惡善惡既合勢難卒離|{
	卒讀曰猝}
雖有詔命違而合權專之可也昭等從之乃發兵救武威降其三種胡|{
	降戶江翻下同}
與母丘興擊張進於張掖麴演聞之將步騎三千迎則辭來助軍實欲為變則誘而斬之|{
	誘音酉}
出以狗軍其黨皆散走則遂與諸軍圍張掖破之斬進黄華懼乞降|{
	據裴松之註華即後為兖州刺史奏王淩者也事見七十五卷邵陵厲公嘉平三年}
河西平初燉煌太守馬艾卒官|{
	燉徒門翻卒子恤翻下同}
郡人推功曹張恭行長史事恭遣其子就詣朝廷請太守會黄華張進叛欲與燉煌并勢執就刼以白刃就終不囘私與恭疏曰大人率厲燉煌忠義顯然豈以就在困厄之中而替之哉今大軍垂至但當促兵以掎之耳|{
	掎舉綺翻從後牽曰掎又云偏引曰掎}
願不以下流之愛使就有恨於黄壤也|{
	論語曰君子惡居下流天下之惡皆歸焉謂下流當惡居而不當愛也一曰流輩也牽於父子之愛而廢君臣之義是常人之流下一等見識故曰下流之愛}
恭即引兵攻酒泉别遣鐵騎二百及官屬緣酒泉北塞東迎太守尹奉黄華欲救張進而西顧恭兵恐擊其後故不得往而降就卒平安奉得之郡詔賜恭爵關内侯 六月庚午王引軍南巡秋七月孫權遣使奉獻 蜀將軍孟達屯上庸與副

軍中郎將劉封不協封侵陵之達率部曲四千餘家來降達有容止才觀|{
	觀工玩翻}
王甚器愛之引與同輦以達為散騎常侍建武將軍封平陽亭侯合房陵上庸西城三郡為新城|{
	蜀分三郡見上卷漢獻帝建安二十四年}
以達領新城太守委以西南之任行軍長史劉曄曰|{
	時魏王引軍南巡以曄為長史}
達有苟得之心而恃才好術|{
	好呼到翻}
必不能感恩懷義新城與孫劉接連|{
	蜀之漢中吳之宜都皆與新城接連}
若有變態為國生患王不聽|{
	為孟達叛魏張本為于偽翻}
遣征南將軍夏侯尚右將軍徐晃與達共襲劉封上庸太守申耽叛封來降封破走還成都初封本羅侯寇氏之子漢中王初至荆州以未有繼嗣養之為子諸葛亮慮封剛猛易世之後終難制御勸漢中王因此際除之遂賜封死 武都氐王楊僕率種人内附|{
	種章勇翻}
甲午王次于譙大饗六軍及譙父老于邑東設伎樂百戲|{
	伎巨綺翻}
吏民上壽日夕而罷

孫盛曰三年之喪自天子達于庶人故雖三季之末|{
	謂三代之季也}
七雄之敝|{
	秦趙韓魏齊楚燕為戰國七雄}
猶未有廢衰斬於旬朔之間釋麻杖於反哭之日者也|{
	麻絰也居父喪苴杖禮既葬而反哭檀弓曰反哭升堂反諸其所作也反哭之弔也哀之至也反而亡焉失之矣於是為甚衰倉囘翻}
逮于漢文變易古制|{
	事見十五卷文帝後七年}
人道之紀一旦而廢固巳道薄於當年風頹於百代矣魏王既追漢制替其大禮處莫重之哀|{
	處昌呂翻}
而設饗宴之樂居貽厥之始而墮王化之基|{
	夏書曰有典有則貽厥子孫墮讀曰隳}
及至受禪顯納二女|{
	獻帝之禪也册詔魏王曰漢承堯運有傳聖之義釐降二女以嬪于魏}
是以知王齡之不遐卜世之期促也

王以丞相祭酒賈逵為豫州刺史|{
	豫州統潁川汝隂汝南梁國沛郡譙郡魯郡弋陽安豐等郡晉地理志曰魏武分沛郡立譙郡分汝南立汝隂郡合陳郡於梁國沈約志曰弋陽縣本屬汝南魏文帝分立郡又分廬江為安豐郡}
是時天下初定刺史多不能攝郡|{
	攝總録也}
逵曰州本以六條詔書察二千石以下|{
	舉漢制也}
故其狀皆言嚴能鷹揚有督察之才不言安靜寛仁有愷悌之德也今長吏慢灋盜賊公行州知而不糾天下復何取正乎|{
	復扶又翻}
其二千石以下阿縱不如灋者皆舉奏免之外脩軍旅内治民事|{
	治直之翻}
興陂田通運渠吏民稱之王曰逵真刺史矣布告天下當以豫州為灋賜逵爵關内侯 左中郎將李伏太史丞許芝表言魏當代漢見于圖緯其事衆甚|{
	據獻帝傳李伏引孔子玉板許芝引春秋漢含孳玉板䜟佐助期孝經中黄䜟易運期䜟}
羣臣因上表勸王順天人之望|{
	時勸進者辛毗劉曄傅巽衛臻桓階陳矯陳羣蘇林董巴繼之者司馬懿鄭渾羊祕鮑勛}
王不許冬十月乙卯漢帝告祠高廟使行御史大夫張音持節奉璽綬詔册禪位于魏王三上書辭讓乃為壇於繁陽|{
	時南巡至潁川潁隂縣築壇干曲蠡之繁陽亭述征記曰其地在許南七十里東有臺高七丈方五十步南有壇高二丈方三十步即受終之壇也是年以繁陽為繁昌縣}
辛未升壇受璽綬即皇帝位 |{
	考異曰陳志云丙午行至曲蠡漢帝禪位庚午升壇即祚袁紀亦云庚午魏王即位按獻帝紀乙卯始發禪册二十九日登壇受命又文帝受禪碑至今尚在亦云辛未受禪陳志袁紀誤也范書云魏遣使求璽綬曹皇后不與如此數輩后乃呼使者以璽抵軒下因涕泣横流曰天不祚爾左右皆莫能仰視案此乃前漢元后事且璽綬無容在曹后之所此說妄也}
燎祭天地嶽瀆改元大赦十一月癸酉奉漢帝為山陽公|{
	山陽縣屬河内郡}
行漢正朔用天子禮樂封公四子為列侯追尊太王曰太皇帝武王曰武皇帝廟號太祖尊王太后曰皇太后以漢諸侯王為崇德侯列侯為關中侯羣臣封爵增位各有差改相國為司徒御史大夫為司空|{
	漢獻帝建安十三年罷三公官今復舊}
山陽公奉二女以嬪于魏帝欲改正朔侍中辛毗曰魏氏遵舜禹之統應天順民至於湯武以戰伐定天下乃改正朔孔子曰行夏之時左氏傳曰夏數為得天正何必期於相反帝善而從之|{
	自是之後遂皆以建寅為正傳直戀翻}
時羣臣並頌魏德多抑損前朝|{
	朝直遙翻}
散騎常侍衛臻獨明禪授之義稱揚漢美帝數目臻曰|{
	數所角翻}
天下之珍當與山陽共之帝欲追封太后父母尚書陳羣奏曰陛下以聖德應運受命創業革制當永為後式案典籍之文無婦人分土命爵之制在禮典婦因夫爵|{
	禮記婦人無爵從夫之爵}
秦違古灋漢氏因之非先王之令典也帝曰此議是也其勿施行仍著定制藏之臺閣|{
	臺閣尚書中藏故事之處}
十二月初營洛陽宮戊午帝如洛陽|{
	裴松之曰案諸書記是時帝居北宫以建始殿朝羣臣門曰承明陳思王植詩謁帝承明廬是也至明帝時始於漢南宮崇德殿處起太極昭陽諸殿魏畧曰漢火行也火忌水故洛去水而加佳魏於行次為土土水之牡也水得土而流土得水而柔故除佳加水變雒為洛}
帝謂侍中蘇則曰前破酒泉張掖西域通使燉煌|{
	使疏吏翻燉徒門翻}
獻徑寸大珠可復求市益得不|{
	復扶又翻不讀曰否}
則對曰若陛下化洽中國德流沙漠即不求自至求而得之不足貴也帝嘿然 帝召東中郎將蔣濟為散騎常侍時有詔賜征南將軍夏侯尚曰卿腹心重將|{
	重將即亮翻}
特當任使作威作福殺人活人尚以示濟濟至帝問以所聞見對曰未有他善但見亡國之語耳帝忿然作色而問其故濟具以答因曰夫作威作福書之明誡|{
	書洪範曰臣無有作威作福臣而有作威作福其害于而家凶于而國}
天子無戲言古人所慎惟陛下察之帝即遣追取前詔 帝欲徙冀州士卒家十萬戶實河南|{
	時營洛陽故欲徙冀州士卒家以實之}
時天旱蝗民饑羣司以為不可而帝意甚盛侍中辛毗與朝臣俱求見|{
	見賢遍翻}
帝知其欲諫作色以待之皆莫敢言毗曰陛下欲徙士家其計安出帝曰卿謂我徙之非邪毗曰誠以為非也帝曰吾不與卿議也毗曰陛下不以臣不肖置之左右厠之謀議之官|{
	侍中於周為常伯之任在天子左右備切問近對拾遺補闕}
安能不與臣議邪臣所言非私也乃社稷之慮也安得怒臣帝不答起入内毗隨而引其裾帝遂奮衣不還良久乃出曰佐治卿持我何太急邪|{
	辛毗字佐治治直吏翻}
毗曰今徙既失民心又無以食也故臣不敢不力爭帝乃徙其半帝嘗出射雉顧羣臣曰射雉樂哉毗對曰於陛下甚樂於羣下甚苦帝默然後遂為之稀出|{
	射而亦翻樂音洛為于偽翻}


二年 |{
	考異曰陳志正月乙亥朝日于東郊裴松之以為朝日在二月按二月辛丑朔無乙亥}
春正月以議郎孔羨為宗聖侯奉孔子祀|{
	漢平帝元始元年封褒成君孔霸曾孫均為褒成侯奉孔子祀王莽敗失國光武建武十三年復封均子志為褒成侯志子損和帝永元四年徙封褒亭侯世世相傳至獻帝初國絶魏封孔子二十一世孫羨為宗聖侯邑百戶晉封二十三世孫震為奉聖亭侯後魏封二十七世孫乘為崇聖大夫孝文太和十九年幸魯又改封二十八世孫珍為崇聖侯北齊改封三十一世孫某為恭聖侯周武帝平齊改封鄒國公隋文帝仍舊封鄒國公煬帝改封為紹聖侯唐太宗貞觀十一年封孔子裔孫倫為褒聖侯}
三月加遼東太守公孫恭車騎將軍|{
	恭公孫度次子康之弟也}
初復五銖錢|{
	漢獻帝初平元年董卓壞五銖錢今復之}
蜀中傳言漢帝已遇害於是漢中王發喪制服諡曰孝愍皇帝羣下競言符瑞勸漢中王稱尊號前部司馬費詩上疏曰|{
	時費詩為益州前部司馬費父沸翻}
殿下以曹操父子偪主簒位故乃羈旅萬里糾合士衆將以討賊今大敵未克而先自立恐人心疑惑昔高祖與楚約先破秦者王之|{
	王于況翻}
及屠咸陽獲子嬰猶懷推讓|{
	推吐雷翻}
況今殿下未出門庭便欲自立邪愚臣誠不為殿下取也|{
	為于偽翻}
王不悅左遷詩為部永昌從事|{
	為益州刺史部從事部永昌郡}
夏四月丙午漢中王即皇帝位於武擔之南|{
	蜀木紀曰武都有丈夫化為女子顔色美好蓋山精也蜀王娶以為妻不習水土疾病欲歸國蜀王留之無幾物故蜀王發卒之武都擔土於成都郭中葬蓋地數畝高十丈號曰武擔也裴松之曰案武擔山在成都西北蓋以乾位在西北故就之以即祚杜佑曰武擔山在蜀郡西}
大赦改元章武以諸葛亮為丞相許靖為司徒

臣光曰天生烝民其勢不能自治必相與戴君以治之|{
	治直之翻}
苟能禁暴除害以保全其生賞善罰惡使不至於亂斯可謂之君矣|{
	温公之說正祖周書所謂撫我則后虐我則讐之意白虎通曰君者羣也羣下之所歸心也}
是以三代之前海内諸侯何啻萬國|{
	黄帝置左右大監監于萬國禹會諸侯于塗山執玉帛者萬國}
有民人社稷者通謂之君合萬國而君之立灋度班號令而天下莫敢違者乃謂之王王德既衰彊大之國能帥諸侯以尊天子者則謂之霸|{
	帥讀曰率}
故自古天下無道諸侯力爭或曠世無王者固亦多矣|{
	如共工氏在伏羲神農之間秦在周漢之間皆謂之霸而不王所謂曠世無王也又如有窮之於夏共和之於周亦曠世而無王也}
秦焚書坑儒漢興學者始推五德生勝以秦為閏位在木火之間霸而不王於是正閏之論興矣|{
	孟康曰秦推五勝以周為火用水勝之漢儒以庖犧繼天而王為百王首德始於木共工氏霸九域雖有水德在木火之間非其序也故霸而不王神農氏以火承木故為炎帝神農氏没黄帝氏作火生土故為土德少昊黄帝之子土生金故為金德少昊之衰顓頊受之金生水故為水德顓頊之所建帝嚳受之水生木故為木德高辛氏衰天下歸堯木生火故為火德堯嬗舜火生土故為土德舜嬗禹土生金故為金德湯伐桀纘禹金生水故為水德周伐商水生木故為木德漢伐秦繼周木生火故為火德共工及秦不在五德相生之正運故曰閏位}
及漢室顛覆三國鼎峙晉氏失馭五胡雲擾宋魏以降南北分治各有國史互相排黜南謂北為索虜北謂南為島夷|{
	索虜者以北人辮髪謂之索頭也島夷者以東南際海土地卑下謂之島中也}
朱氏代唐四方幅裂朱邪入汴比之窮新|{
	唐莊宗自以為繼唐比朱梁於有窮簒夏新室簒漢}
運歷年紀皆棄而不數此皆私已之偏辭非大公之通論也臣愚誠不足以識前代之正閏竊以為苟不能使九州合為一統皆有天子之名而無其實者也雖華夏仁暴大小強弱或時不同|{
	夏戶雅翻}
要皆與古之列國無異豈得獨尊奬一國謂之正統而其餘皆為僭偽哉若以自上相授受者為正邪則陳氏何所受拓拔氏何所受若以居中夏者為正邪|{
	夏戶雅翻}
則劉石慕容苻姚赫連所得之土皆五帝三王之舊都也若以有道德者為正邪則蕞爾之國必有令主|{
	蕞徂外翻小貌}
三代之季豈無僻王是以正閏之論自古及今未有能通其義確然使人不可移奪者也臣今所述止欲叙國家之興衰著生民之休戚使觀者自擇其善惡得失以為勸戒非若春秋立褒貶之法撥亂世反諸正也正閏之際非所敢知但據其功業之實而言之周秦漢晉隋唐皆嘗混壹九州傳祚於後子孫雖微弱播遷猶承祖宗之業有紹復之望四方與之爭衡者皆其故臣也故全用天子之制以臨之其餘地醜德齊|{
	醜類也言地之廣狹相類也}
莫能相壹名號不異本非君臣者皆以列國之制處之|{
	處昌呂翻}
彼此均敵無所抑揚庶幾不誣事實近於至公|{
	近其靳翻}
然天下離析之際不可無歲時月日以識事之先後|{
	識音誌}
據漢傳於魏而晉受之晉傳于宋以至於陳而隋取之唐傳於梁以至於周而大宋承之故不得不取魏宋齊梁陳後梁後唐後晉後漢後周年號以紀諸國之事|{
	魏下當有晉字}
非尊此而卑彼有正閏之辨也昭烈之於漢雖云中山靖王之後而族屬疎遠不能紀其世數名位亦猶宋高祖稱楚元王後|{
	宋高祖彭城人自謂漢楚元王交二十一世孫蓋以彭城楚都故其苗裔家於此地也}
南唐烈祖稱吳王恪後|{
	南唐初欲祖吳王恪或請祖鄭王元懿唐主命考二王苗裔以吳王孫禕有功禕子峴為丞相遂祖吳王}
是非難辨故不敢以光武及晉元帝為比使得紹漢氏之遺統也|{
	温公紀年之意具于此論中矣}


孫權自公安徙都卾更名卾曰武昌|{
	更工衡翻}
五月辛巳漢主立夫人吳氏為皇后后偏將軍懿之妹故劉璋兄瑁之妻也|{
	瑁莫報翻}
立子禪為皇太子娶車騎將軍張飛女為皇太子妃 太祖之入鄴也|{
	入鄴見六十四卷漢建安十年}
帝為五官中郎將見袁熙妻中山甄氏美而悅之|{
	甄之人翻}
太祖為之聘焉|{
	為于偽翻}
生子叡及即皇帝位安平郭貴嬪有寵|{
	據陳壽志郭嬪安平廣宗人漢廣宗縣屬鉅鹿郡晉志廣宗始屬安平蓋魏氏制度也六宫置貴嬪始此孔穎達曰嬪婦人之美稱可賓敬也嬪毗賓翻}
甄夫人留鄴不得見失意有怨言郭貴嬪譛之帝大怒六月丁卯遣使賜夫人死|{
	為明帝立郭太后以憂崩張木}
帝以宗廟在鄴|{
	武王之封魏王建宗廟於鄴}
祀太祖於洛陽建始殿如家人禮|{
	建始殿帝所起以建國之始命名父為士子為天子祭以天子安有用家人禮者哉}
戉辰晦日有食之有司奏免太尉|{
	仍東漢中世之制也}
詔曰災異之作以譴元首而歸過股肱豈禹湯罪已之義乎|{
	左傳臧文仲曰禹湯罪已其興也勃焉}
其令百官各䖍厥職後有天地之眚勿復劾三公|{
	復扶又翻}
漢主立其子永為魯王理為梁王|{
	晉書地理志劉備以郡國封建諸王或遙采嘉名不由檢其土地所出孫權亦取中州嘉號封建諸王自此迄於南北朝大率類此}
漢主恥關羽之没將擊孫權翊軍將軍趙雲曰國賊曹操非孫權也若先滅魏則權自服今操身雖斃子丕簒盜當因衆心早圖關中居河渭上流以討凶逆關東義士必裏糧策馬以迎王師不應置魏先與吳戰兵勢一交不得卒解非策之上也|{
	趙雲之言可謂知所先後矣卒讀曰猝}
羣臣諫者甚衆漢主皆不聽廣漢處士秦宓|{
	處昌呂翻宓莫必翻通作密不應州郡辟命故曰處士}
陳天時必無利坐下獄幽閉然後貸出|{
	貸原也赦也下遐稼翻}
初車騎將軍張飛雄壯威猛亞於關羽羽善待卒伍而驕於士大夫飛愛禮君子而不恤軍人漢主常戒飛曰卿刑殺既過差|{
	差次也過差猶今人言過次也}
又日鞭檛健兒而令在左右|{
	檛陟加翻箠也}
此取禍之道也飛猶不悛|{
	悛丑緣翻改也}
漢主將伐孫權飛當率兵萬人自閬中會江州|{
	閬中縣屬巴西郡此亦由内水下江州也杜佑曰漢江州縣故城在巴縣西}
臨發其帳下將張達范彊殺飛以其首順流犇孫權漢主聞飛營都督有表曰噫飛死矣|{
	表當自飛上而都督越次上之故知其必死也凡用兵必觀人事既失關羽又喪張飛兵可以無出矣}


陳壽評曰關羽張飛皆稱萬人之敵為世虎臣羽報效曹公|{
	事見六十三卷獻帝建安五年}
飛義釋嚴顔|{
	事見六十七卷建安十九年}
並有國士之風然羽剛而自矜飛暴而無恩以短取敗理數之常也

秋七月漢主自率諸軍擊孫權權遣使求和於漢南郡太守諸葛瑾遺漢主牋曰|{
	遺于季翻}
陛下以關羽之親何如先帝|{
	時蜀人傳漢帝已遇害因稱之為先帝}
荆州大小孰與海内俱應仇疾誰當先後若審此數易於反掌矣漢主不聽|{
	諸葛瑾之言天下之公也使漢主因此與吳解仇繼好魏氏其旰食乎易以䜴翻}
時或言瑾别遣親人與漢主相聞者權曰孤與子瑜有死生不易之誓子瑜之不負孤猶孤之不負子瑜也然謗言流聞于外陸遜表明瑾必無此宜有以散其意權報曰子瑜與孤從事積年恩如骨肉深相明究其為人非道不行非義不言玄德昔遣孔明至吳|{
	蓋謂亮至吳求救時也}
孤嘗語子瑜曰|{
	語牛倨翻}
卿與孔明同產且弟隨兄於義為順何以不留孔明孔明若留從卿者孤當以書解玄德意自隨人耳|{
	意料度也權自言料度備意必當相從}
子瑜答孤言弟亮已失身於人委質定分|{
	質如字分扶問翻}
義無二心弟之不留猶瑾之不往也其言足貫神明今豈當有此乎前得妄語文疏即封示子瑜并手筆與之孤與子瑜可謂神交非外言所間|{
	間古莧翻}
知卿意至輒封來表以示子瑜使知卿意|{
	觀孫權君臣之間推誠相與讒間不行於其間所以能保有江東也}
漢主遣將軍吳班馮習攻破權將李異劉阿等於巫|{
	巫縣漢屬南郡吳初屬宜都郡後孫休分立建平郡巫屬焉賢曰巫故城在今夔州巫山縣北杜佑曰巫歸州巴東縣是又曰巫山縣楚之巫郡漢為巫縣故城在今縣北晉置建平郡於此}
進兵秭歸兵四萬餘人武陵蠻夷皆遣使往請兵權以鎮西將軍陸遜為大都督假節督將軍朱然潘璋宋謙韓當徐盛鮮于丹孫桓等五萬人拒之|{
	孫權始命呂蒙為大督以取關羽今又復命陸遜為大都督以拒劉備大都督之號蓋昉此}
皇弟鄢陵侯彰宛侯據魯陽侯宇譙侯林贊侯衮襄邑侯峻弘農侯幹壽春侯彪歷城侯徽平輿侯茂皆進爵為公|{
	鄢謁晩翻又於建翻又音偃宛於元翻魯陽縣屬南陽郡譙縣鄼縣屬譙郡襄邑屬陳留郡壽春屬淮南郡歷城屬濟南郡平輿屬汝南郡應劭曰輿音預}
安鄉侯植改封甄城侯|{
	植以見忌貶侯今乃改封縣侯甄城屬東郡蜀本作鄄城當從之鄄音絹}
築陵雲臺|{
	據水經註陵雲臺在洛陽城中金市之東}
初帝詔羣臣令料劉備當為關羽出報孫權否|{
	為于偽翻下同}
衆議咸云蜀小國耳名將唯羽羽死軍破國内憂懼無緣復出|{
	復扶又翻}
侍中劉曄獨曰蜀雖陿弱|{
	陿即狹字}
而備之謀欲以威武自彊勢必用衆以示有餘且關羽與備義為君臣恩猶父子羽死不能為興軍報敵於終始之分不足矣|{
	分扶問翻}
八月孫權遣使稱臣卑辭奉章并送于禁等還|{
	權破南郡得于禁事見上卷獻帝建安二十四年}
朝臣皆賀|{
	朝直遙翻}
劉曄獨曰權無故求降|{
	降戶江翻下同}
必内有急權前襲殺關羽劉備必大興師伐之外有彊寇衆心不安又恐中國往乘其釁故委地求降一以却中國之兵二假中國之援以彊其衆而疑敵人耳|{
	劉曄之言曲盡權之情偽}
天下三分中國十有其八吳蜀各保一州|{
	約而言之謂吳保揚蜀保益也}
阻山依水有急相救此小國之利也今還自相攻天亡之也宜大興師徑渡江襲之蜀攻其外我襲其内吳之亡不出旬日矣吳亡則蜀孤若割吳之半以與蜀蜀固不能久存況蜀得其外我得其内乎帝曰人稱臣降而伐之疑天下欲來者心不若且受吳降而襲蜀之後也對曰蜀遠吳近又聞中國伐之便還軍不能止也今備已怒興兵擊吳聞我伐吳知吳必亡將喜而進與我爭割吳地必不改計抑怒救吳也|{
	抑按止也}
帝不聽遂受吳降|{
	若魏用劉曄之言吳其殆矣}
于禁須髮皓白形容憔顇|{
	顇與悴同秦醉翻}
見帝泣涕頓首帝慰諭以荀林父孟明視故事|{
	晉大夫荀林父與楚戰敗于邲晉景公復用之以取赤狄秦大夫孟明為晉禽于殽秦穆公復用之以霸西戎父音甫}
拜安遠將軍|{
	安遠將軍號亦前此未有也}
令北詣鄴謁高陵帝使豫於陵屋畫關羽戰克龎德憤怒禁降伏之狀|{
	畫古畫字通}
禁見慙恚發病死|{
	恚於避翻}


臣光曰于禁將數萬衆敗不能死生降於敵既而復歸文帝廢之可也殺之可也乃畫陵屋以辱之斯為不君矣|{
	賞慶刑威曰君}


丁巳遣太常邢貞奉策即拜孫權為吳王加九錫|{
	即就也}
劉曄曰不可先帝征伐天下十兼其八威震海内陛下受禪即真德合天地聲暨四遠權雖有雄才故漢票騎將軍南昌侯耳|{
	票騎南昌操挾漢而命之也事見上卷漢建安二十四年}
官輕勢卑況士民有畏中國心不可彊迫與成所謀也|{
	彊其兩翻}
不得已受其降可進其將軍號封十萬戶侯不可即以為王也夫王位去天子一階耳其禮秩服御相亂也|{
	漢自景武以後裁削藩王不使與京師同制自曹操為魏王加九錫禮秩服御與天子相亂矣}
彼直為侯江南士民未有君臣之分|{
	分扶問翻}
我信其偽降就封殖之|{
	封增土以培之殖養之使蕃茂也}
崇其位號定其君臣是為虎傅翼也|{
	傅讀曰附}
權既受王位却蜀兵之後外盡禮以事中國使其國内皆聞内為無禮以怒陛下陛下赫然發怒興兵討之乃徐告其民曰我委身事中國不愛珍貨重寶隨時貢獻不敢失臣禮而無故伐我必欲殘我國家俘我人民以為僕妾吳民無緣不信其言也信其言而感怒上下同心戰加十倍矣又不聽|{
	史言帝再不聽劉曄之言為後伐吳無功張本}
諸將以吳内附意皆縱緩獨征南大將軍夏侯尚益修攻守之備山陽曹偉素有才名|{
	此山陽郡也屬兖州}
聞吳稱藩以白衣與吳王交書求賂欲以交結京師帝聞而誅之 吳又城武昌|{
	既城石頭又城武昌此吳人保江之根本也}
初帝欲以楊彪為太尉彪辭曰嘗為漢朝三公|{
	朝直遙翻}
值世衰亂不能立尺寸之益若復為魏臣|{
	復扶又翻}
於國之選亦不為榮也帝乃止冬十月己亥公卿朝朔旦并引彪待以客禮賜延年椐杖|{
	詩其檉其椐傳云椐樻孫炎云樻腫節可以作杖陸璣疏云節中腫以扶老今人以為馬鞭及杖弘農共北山甚有之陸曰即今靈壽杖是也師古曰木似竹有枝節長不過八九尺圍三四寸自然冇合杖制不煩削治陳藏器云生劒南山谷圓長皮紫作杖令人延年益壽}
馮几使著布單衣皮弁以見|{
	馮讀曰憑著直畧翻見賢遍翻}
拜光禄大夫秩中二千石|{
	漢制光禄大夫比二千石晉志曰光禄大夫漢置無定員多以為拜假賻贈之使及監護喪事魏氏以來轉復優重不復以為使命之官其諸公告老者皆家拜此位及在朝顯職復用加之}
朝見位次三公|{
	朝直遙翻見賢遍翻}
又令門施行馬|{
	魏晉之制三公及位從公門施行馬程大昌曰行馬者一木横中兩木互穿以施四角施之於門以為約禁也周禮謂之梐枑今官府前乂子是也}
置吏卒以優崇之年八十四而卒|{
	楊彪有愧於龔勝多矣}
以穀貴罷五銖錢|{
	復五銖錢無幾何而罷}
涼州盧水胡治元多等反河西大擾帝召鄒岐還以京兆尹張既為涼州刺史遣護軍夏侯儒將軍費曜等繼其後|{
	費父沸翻}
胡七千餘騎逆拒既於鸇隂口|{
	鸇隂縣前漢屬安定郡後漢屬武威郡鸇隂口鸇隂河口也}
既揚聲軍從鸇隂乃潜由且次出武威|{
	二漢志武威有楈次縣孟康曰楈音子如翻次音咨即且次也}
胡以為神引還顯美|{
	顯美縣前漢屬張掖郡後漢及魏晉屬武威郡}
既已據武威曜乃至儒等猶未達既勞賜將士|{
	勞力到翻}
欲進軍擊胡諸將皆曰士卒疲倦虜衆氣銳難與爭鋒既曰今軍無見糧|{
	見賢遍翻}
當因敵為資若虜見兵合退依深山追之則道險窮餓兵還則出候寇鈔|{
	鈔楚交翻}
如此兵不得解所謂一日縱敵患在數世也|{
	左傳先軫曰一日縱敵數世之患也}
遂前軍顯美十一月胡騎數千因大風欲放火燒營將士皆恐既夜藏精卒三千人為伏使參軍成公英督千餘騎挑戰|{
	姓譜衛成公之後為成公氏余不敢謂之傳信}
敕使陽退胡果爭犇之因發伏截其後首尾進擊大破之斬首獲生以萬數河西悉平後西平麴光反殺其郡守諸將欲擊之既曰唯光等造反郡人未必悉同若便以軍臨之吏民羌胡必謂國家不别是非|{
	别彼列翻}
更使皆相持著|{
	著直畧翻}
此為虎傅翼也|{
	為于偽翻傅讀曰附}
光等欲以羌胡為援今先使羌胡鈔擊|{
	鈔楚交翻}
重其賞募所虜獲者皆以畀之外沮其勢|{
	沮在呂翻}
内離其交必不戰而定乃移檄告諭諸羌為光等所詿誤者原之|{
	詿古賣翻}
能斬賊帥送首者當加封賞|{
	帥所類翻}
於是光部黨斬送光首其餘皆安堵如故 邢貞至吳吳人以為宜稱上將軍九州伯|{
	王制九州其一州為天子之縣内八州八伯}
不當受魏封吳王曰九州伯於古未聞也昔沛公亦受項羽封為漢王|{
	事見九卷漢高帝元年}
蓋時宜耳復何損邪|{
	復扶又翻後同}
遂受之吳王出都亭侯貞貞入門不下車張昭謂貞曰夫禮無不敬法無不行而君敢自尊大豈以江南寡弱無方寸之刃故乎貞即遽下車中郎將琅邪徐盛忿憤顧謂同列曰盛等不能奮身出命為國家并許洛呑巴蜀|{
	為于偽翻}
而令吾君與貞盟不亦辱乎因涕泣横流貞聞之謂其徒曰江東將相如此非久下人者也|{
	觀貞此言善覘國者也使還之日嘗以復於魏主否然觀貞以張昭之言而下車則其氣已奪矣}
吳主遣中大夫南陽趙咨入謝帝問曰吳主何等主也對曰聰明仁智雄畧之主也帝問其狀對曰納魯肅於凡品是其聰也拔呂蒙於行陳是其明也|{
	行戶剛翻陳讀曰陣}
獲于禁而不害是其仁也取荆州兵不血刃是其智也據三州虎視于天下|{
	三州荆揚交也}
是其雄也屈身於陛下是其畧也帝曰吳王頗知學乎咨曰吳王浮江萬艘|{
	艘蘇刀翻}
帶甲百萬任賢使能志存經畧雖有餘閒博覽書傳|{
	傳直戀翻}
歷史籍采奇異不效書生尋章摘句而已|{
	帝好文章故趙咨以此言譏之摘蜀本作擿}
帝曰吳可征否對曰大國有征伐之兵小國有備禦之固|{
	此二語本之管子}
帝曰吳難魏乎對曰帶甲百萬江漢為池何難之有帝曰吳如大夫者幾人對曰聰明特達者八九十人如臣之比車載斗量不可勝數|{
	量音良勝音升}
帝遣使求雀頭香大貝明珠象牙犀角玳瑁孔雀翡翠鬭鴨長鳴雞於吳|{
	本草以香附子為雀頭香此物處處有之非珍也恐别是一物貝質白如玉紫點為文皆行列相當明珠出合浦大者徑寸象出交趾雄者有兩長牙長丈餘犀亦出交趾惟通天犀最貴角有白理如線置米羣雞中雞往啄米見犀輒驚却南人呼為駭雞犀玳瑁狀如龜腹背甲有烘點其大者如盤盂諸蕃志玳瑁形如龜黿背甲十三片黑白班文間錯邊欄缺齧如鋸無足而有四鬛前長後短以鬛棹水而行鬛與首班文如甲老者甲厚而黑白分明少者甲薄而花字糢糊世傳鞭血成班者妄也孔雀生羅州雄者尾金翠色光耀可愛埤雅曰博物志云孔雀尾多變色或紅或黄諭如雲霞其色不定人拍其尾則舞尾有金翠五年而後成始生三年金翠尚小初春乃生三四月後復凋與花萼俱衰榮人採其尾以飾扇拂生取則金翠之色不減南人取其尾者握刀蔽于叢竹潜隱之處伺過急斬其尾若不即斷囘首一顧金翠無復光彩每欲小棲先擇置尾之地故欲生捕候雨甚則往擒之尾霑而重不能高翔人雖至且愛其尾不復騫揚也翡翠大小一如雀雄赤曰翡雌青曰翠羽可為飾鴨馴狎能鬬者難得長鳴雞者其鳴聲長也}
吳羣臣曰荆揚二州貢有常典|{
	禹别九州任土作貢此常典也}
魏所求珍玩之物非禮也宜勿與吳王曰方有事於西北|{
	謂與蜀相距復須備魏也}
江表元元恃主為命彼所求者於我瓦石耳孤何惜焉且彼在諒闇之中|{
	闇音隂}
而所求若此寧可與言禮哉皆具以與之|{
	史言帝為敵國所窺}
吳王以其子登為太子妙選師友以南郡太守諸葛瑾之子恪綏遠將軍張昭之子休|{
	沈約志四十號將軍綏遠第十四}
大理吳郡顧雍之子譚偏將軍廬江陳武之子表皆為中庶子入講詩書出從騎射|{
	騎奇寄翻}
謂之四友登接待僚屬畧用布衣之禮 十二月帝行東巡 帝欲封吳王子登為萬戶侯吳王以登年幼上書辭不受復遣西曹掾吳興沈珩入謝|{
	姓譜沈姓出吳興本自周文王第十子聃季食采於沈即汝南平輿沈亭是也子孫以國為氏又楚莊王之子公子真封于沈鹿其後有沈尹戍沈諸梁珩音行}
并獻方物帝問曰吳嫌魏東向乎珩曰不嫌曰何以曰信恃舊盟言歸于好|{
	好呼到翻}
是以不嫌若魏渝盟自有豫備又問聞太子當來寧然乎珩曰臣在東朝朝不坐宴不與|{
	吳在江東故曰東朝朝不坐宴不與記檀弓記尹商陽之言朝直遙翻}
若此之議無所聞也帝善之吳王於武昌臨釣臺|{
	水經武昌有樊山北背大江江上有釣臺}
飲酒大醉使人以水灑羣臣|{
	醉者以水灑之醒然後能飲}
曰今日酣飲惟醉墮臺中乃當止耳張昭正色不言出外車中坐王遣人呼昭還入謂曰為共作樂耳|{
	樂音洛下同}
公何為怒乎昭對曰昔紂為糟丘酒池長夜之飲|{
	紂以酒為池糟丘足以望七里一鼓而牛飲者三千人懸肉為林使男女倮逐於其間為長夜之飲}
當時亦以為樂不以為惡也王默然慙遂罷酒吳王與羣臣飲自起行酒虞翻伏地陽醉不持王去翻起坐|{
	翻為是者所以諫也}
王大怒手劒欲擊之|{
	手劒手援劒也記曰子手弓手守又翻}
侍坐者莫不惶遽|{
	坐徂卧翻}
惟大司農劉基起抱王諫曰大王以三爵之後手殺善士雖翻有罪天下孰知之|{
	古者臣侍君宴不過三爵懼其失節也}
且大王以能容賢蓄衆故海内望風今一朝棄之可乎王曰曹孟德尚殺孔文舉|{
	事見六十五卷漢獻帝建安十三年}
孤於虞翻何有哉基曰孟德輕害士人天下非之大王躬行德義欲與堯舜比隆何得自喻於彼乎翻由是得免王因敕左右自今酒後言殺皆不得殺基繇之子也|{
	劉繇為孫策所襲走死}
初太祖既克蹋頓|{
	事見六十五卷漢獻帝建安十二年蹋徒臘翻}
而烏桓浸衰鮮卑大人步度根軻比能素利彌加厥機等因閻柔上貢獻求通市|{
	通關市以其土物與中國互市也上時掌翻}
太祖皆表寵以為王軻比能本小種鮮卑|{
	種章勇翻}
以勇健廉平為衆所服由是能威制諸部最為彊盛|{
	徒勇健而不廉平未必能制諸部也}
自雲中五原以東抵遼水皆為鮮卑庭軻比能與素利彌加割地統御各有分界|{
	分扶問翻}
軻比能部落近塞|{
	近其靳翻}
中國人多亡叛歸之素利等在遼西右北平漁陽塞外道遠故不為邊患帝以平虜校尉牽招為護鮮卑校尉南陽太守田豫為護烏桓校尉使鎮撫之

三年春正月丙寅朔日有食之 庚午帝行如許昌|{
	晉志曰漢獻帝都許魏受禪徙都洛陽許宫室武庫存焉改為許昌}
詔曰今之計孝|{
	計孝上計吏及孝廉也}
古之貢士也若限年然後取士是呂尚周晉不顯於前世也|{
	呂尚年八十餘文王以為師周太子晉少有令名}
其令郡國所選勿拘老幼儒通經術吏達文法到皆試用有司糾故不以實者|{
	故不以實謂用意為姦欺者}
二月鄯善龜兹于闐王各遣使奉獻|{
	鄯上扇翻龜兹音丘慈闐徒賢翻又徒見翻}
是後西域復通置戊巳校尉|{
	漢自安帝以後未嘗不欲通西域訖不能通今雖置戊巳校尉亦不能如漢之屯田車師也復扶又翻}
漢主自秭歸將進擊吳治中從事黄權諫曰吳人悍戰而水軍沿流進易退難|{
	悍下罕翻下旰翻易以䜴翻}
臣請為先驅以當寇陛下宜為後鎮漢主不從以權為鎮北將軍使督江北諸軍|{
	為漢主兵敗權不能自反張本}
自率諸將自江南緣山截領|{
	領古嶺字通}
軍於夷道猇亭|{
	裴松之曰猇許交翻夷道縣漢屬南郡吳屬宜都郡}
吳將皆欲迎擊之|{
	將即亮翻}
陸遜曰備舉軍東下銳氣始盛且乘高守險難可卒攻|{
	卒讀曰猝}
攻之縱下猶難盡克若有不利損我大勢非小故也今但且奬厲將士廣施方畧以觀其變若此間是平原曠野當恐有顛沛交逐之憂今緣山行軍勢不得展自當罷於木石之間徐制其敝耳|{
	罷讀曰疲魏人言陸議見兵勢正由此耳}
諸將不解|{
	解古買翻曉也}
以為遜畏之各懷憤恨漢人自佷山通武陵|{
	佷山縣前漢屬武陵郡後漢屬南郡吳屬宜都郡孟康曰佷音桓唐峽州辰陽縣有佷山佷音銀杜佑曰峽州長楊縣漢佷山縣余按唐志辰陽誤也當作長陽}
使侍中襄陽馬良以金錦賜五谿諸蠻夷授以官爵|{
	為馬良不得還蜀張本}
三月乙丑立皇子齊公叡為平原王皇弟鄢陵公彰等皆進爵為王甲戌立皇子霖為河東王 甲午帝行如襄邑 夏四月戊申立鄄城侯植為鄄城王|{
	鄄音絹}
是時諸侯王皆寄地空名而無其實王國各有老兵百餘人以為守衛隔絶千里之外不聽朝聘為設防輔監國之官以伺察之|{
	防輔者言防其為非而輔之以正也監國即監國謁者也朝直遙翻為于偽翻}
雖有王侯之號而儕於匹夫|{
	儕士皆翻}
皆思為布衣而不能得灋既峻切諸侯王過惡日聞獨北海王衮謹慎好學|{
	好呼到翻}
未嘗有失文學防輔相與言曰|{
	晉百官志王國置師友文學各一人防輔不書者魏氏防制藩國過差晉武帝懲其失而不置也}
受詔察王舉措有過當奏有善亦宜以聞遂共表稱陳衮美衮聞之大驚懼責讓文學曰修身自守常人之行耳|{
	行下孟翻}
而諸君乃以上聞是適所以增其負累也|{
	累力瑞翻}
且如有善何患不聞而遽共如是是非所以為益也|{
	衮之言漢北海王睦之故智也}
癸亥帝還許昌 五月以江南八郡為荆州江北諸郡為郢州|{
	既以孫權為荆州牧統江南八郡故以江北諸郡置郢州吳自立則郢州廢矣}
漢人自巫峽建平連營至夷陵界|{
	水經注巫峽首尾一百六十里巫縣}


|{
	屬建平郡則巫峽正在建平郡界至夷陵則為宜都郡界然孫休永安三年始分宜都立建平郡此時未有建平也史追書耳杜佑曰吳建平今巴東郡}
立數十屯以馮習為大督張南為前部督自正月與吳相拒至六月不決漢主遣吳班將數千人於平地立營吳將帥皆欲擊之陸遜曰此必有譎且觀之|{
	譎古宂翻}
漢主知其計不行乃引伏兵八千從谷中出遜曰所以不聽諸君擊班者揣之必有巧故也|{
	揣初委翻}
遜上疏於吳王曰夷陵要害國之關限|{
	自三峽下夷陵連山疊嶂江行其中迴旋湍激至西陵峽口始漫為平流夷陵正當峽口故以為吳之關限}
雖為易得亦復易失|{
	易以䜴翻復扶又翻下同}
失之非徒損一郡之地荆州可憂今日爭之當令必諧備干天常不守窟穴而敢自送臣雖不材憑奉威靈以順討逆破壞在近無可憂者臣初嫌之水陸俱進今反捨船就步處處結營察其布置必無他變伏願至尊高枕不以為念也|{
	枕職任翻}
閏月遜將進攻漢軍諸將並曰攻備當在初今乃令入五六百里相守經七八月其諸要害皆已固守擊之必無利矣遜曰備是猾虜更嘗事多|{
	更工衡翻}
其軍始集思慮精專未可干也今住已久不得我便兵疲意沮計不復生掎角此寇|{
	左傳晉人角之諸戎掎之角者當前與之角掎者從後掎其足也沮在呂翻掎居蟻翻}
正在今日乃先攻一營不利諸將皆曰空殺兵耳遜曰吾已曉破之之術乃敕各持一把茅以火攻拔之一爾勢成|{
	言一拔營之頃而兵之勝勢成也一爾猶言一如此也}
通率諸軍同時俱攻斬張南馮習及胡王沙摩相等首破其四十餘營漢將杜路劉寧等窮逼請降|{
	降戶江翻下同}
漢主升馬鞌山|{
	今峽州夷陵縣有馬鞌山}
陳兵自繞遜督促諸軍四面蹙之土崩瓦解死者萬數漢主夜遁驛人自擔燒鐃鎧斷後僅得入白帝城|{
	漢主初連兵入夷陵界沿路置驛以達于白帝及兵敗諸軍潰散惟驛人自擔所棄鐃鎧燒之于隘以斷後僅得脫也據水經注燒鎧斷道處地名石門在秭歸縣西杜佑曰歸州巴東縣有石門山劉備斷道處鐃尼交翻如鈴無舌而有秉周禮以金鐃止鼔軍中所用也斷丁管翻}
其舟船器械水步軍資一時畧盡尸骸塞江而下|{
	塞悉則翻}
漢主大慙恚曰吾乃為陸遜所折辱豈非天耶|{
	依險行兵敵扼其衝情見勢屈敵乘其懈至於失師此非天也}
將軍義陽傅肜為後殿|{
	魏文帝分南陽郡立義陽郡又立義陽縣屬焉此在肜入蜀之後史追書也肜余中翻殿丁練翻}
兵衆盡死肜氣益烈吳人諭之使降肜罵曰吳狗安有漢將軍而降者遂死之從事祭酒程畿泝江而退|{
	從事祭酒諸從事之長也}
衆曰後追將至宜解舫輕行|{
	舫甫妄翻方舟曰舫又並兩舟曰舫}
畿曰吾在軍未習為敵之走也亦死之|{
	言擐甲執兵以臨敵固欲就死未嘗習走也}
初吳安東中郎將孫桓别擊漢前鋒於夷道|{
	夷道縣漢屬南郡吳屬宜都郡}
為漢所圍求救於陸遜遜曰未可諸將曰孫安東公族見圍已困柰何不救遜曰安東得士衆心城牢糧足無可憂也待吾計展欲不救安東安東自解及方畧大施漢果犇潰桓後見遜曰前實怨不見救定至今日|{
	言至今日而事始定}
乃知調度自有方耳|{
	調徒弔翻}
初遜為大都督諸將或討逆時舊將|{
	討逆謂孫策也}
或公室貴戚各自矜恃不相聽從遜按劒曰劉備天下知名曹操所憚今在疆界此彊對也|{
	彊對猶言彊敵}
諸君並荷國恩|{
	荷下可翻}
當相輯睦共翦此虜上報所受|{
	高爵厚禄受恩多矣總兵扞敵受任重矣皆當有以上報}
而不相順何也僕雖書生受命主上國家所以屈諸君使相承望者以僕尺寸可稱能忍辱負重故也|{
	忍辱言能容諸將負重則自任也}
各任其事豈復得辭|{
	復扶又翻}
軍令有常不可犯也|{
	言將行軍法也}
及至破備計多出遜諸將乃服吳王聞之曰公何以初不啟諸將違節度者邪對曰受恩深重此諸將或任腹心或堪爪牙或是功臣皆國家所當與共克定大事者臣竊慕相如寇恂相下之義以濟國事|{
	相如事見四卷周赧王三十六年寇恂事見四十卷漢光武建武二年}
王大笑稱善加遜輔國將軍|{
	晉職官志輔國大將軍位從公其號蓋始于漢獻帝以命伏完然猶未加大}
領荆州牧改封江陵侯初諸葛亮與尚書令灋正好尚不同|{
	好呼到翻}
而以公義相取亮每奇正智術及漢主伐吳而敗時正已卒亮歎曰孝直若在必能制主上東行就使東行必不傾危矣|{
	觀孔明此言不以漢主伐吳為可然而不諫者以漢主怒盛而不可阻且得上流可以勝也兵勢無常在於觀變出奇故曰孝直在必不傾危}
漢主在白帝徐盛潘璋宋謙等各競表言備必可禽乞復攻之|{
	復扶又翻}
吳王以問陸遜遜與朱然駱統上言曰曹丕大合士衆外託助國討備内實有姦心謹決計輒還|{
	曹公不追關羽陸遜不再攻劉備其所見固同也以智遇智三國所以鼎立歟}
初帝聞漢兵樹栅連營七百餘里謂羣臣曰備不曉兵豈有七百里營可以拒敵者乎苞原隰險阻而為軍者為敵所禽此兵忌也孫權上事今至矣|{
	上事謂上奏言兵事也上時掌翻}
後七日吳破漢書到 秋七月冀州大蝗饑 漢主既敗走黄權在江北道絶不得還八月率其衆來降|{
	降戶江翻下同}
漢有司請收權妻子漢主曰孤負黄權權不負孤也|{
	以不能用權言也}
待之如初帝謂權曰君捨逆效順欲追蹤陳韓邪|{
	陳韓謂韓信陳平去楚歸漢}
對曰臣過受劉主殊遇降吳不可還蜀無路是以歸命且敗軍之將免死為幸何古人之可慕也帝善之拜為鎮南將軍封育陽侯|{
	自此以後皆名號侯不復註其國邑其地名難知者猶為之註}
加侍中使陪乘|{
	陪乘猶驂乘也乘繩證翻}
蜀降人或云漢誅權妻子帝詔權發喪權曰臣與劉葛推誠相信|{
	葛謂諸葛孔明}
明臣本志竊疑未實請須|{
	須待也}
後得審問果如所言馬良亦死於五谿 九月甲午詔曰夫婦人與政亂之本也|{
	與讀曰豫}
自今以後羣臣不得奏事太后后族之家不得當輔政之任又不得横受茅土之爵|{
	横戶孟翻}
以此詔傳之後世若有背違|{
	背蒲妹翻}
天下共誅之卞太后每見外親不假以顔色常言居處當節儉|{
	處昌呂翻}
不當望賞念自佚也外舍當怪吾遇之太薄|{
	后妃謂其外家為外舍}
吾自有常度故也吾事武帝四五十年行儉日久不能自變為奢有犯科禁者吾且能加罪一等耳|{
	言罪加於常人犯法者一等也}
莫望錢米恩貸也帝將立郭貴嬪為后中郎棧潜上疏曰|{
	漢三署中郎及虎賁羽林中郎皆秩比六百石魏文帝自五官中郎將登極省五官將惟左右中郎及虎賁羽林中郎棧仕限翻丁度曰姓也何氏姓苑棧姓出任城棧潜任城人也蓋自潜始著棧仕限翻}
夫后妃之德盛衰治亂所由生也是以聖哲慎立元妃必取先代世族之家擇其令淑以統六宫䖍奉宗廟易曰家道正而天下定|{
	易家人曰夫夫婦婦而家道正家道正而天下定矣}
由内及外先王之令典也春秋書宗人釁夏云無以妾為夫人之禮|{
	賈公彦曰襄二十四年公子荆之母嬖將以為夫人使宗人釁夏獻其立夫人之禮對曰無之公怒曰汝為宗司立夫人國之大禮也何故無之對曰周公武公娶于薛孝公惠公娶于商自桓以下娶于齊此禮也則有若以妾為夫人則固無其禮也公卒立之}
齊桓誓命于葵丘亦曰無以妾為妻|{
	見孟子}
令後宫嬖寵當亞乘輿|{
	嬖卑義翻又博計翻乘繩證翻}
若因愛登后使賤人暴貴臣恐後世下陵上替開張非度|{
	非度猶言非法}
亂自上起也帝不從庚子立皇后郭氏初吳王遣于禁護軍浩周|{
	浩姓也姓譜漢有青州刺史浩賞}
軍司馬東里衮|{
	東里之先以居里為氏}
詣帝自陳誠欵辭甚恭慤帝問周等權可信乎周以為權必臣服而衮謂其不可必服帝悅周言以為有以知之故立為吳王復使周至吳|{
	復扶又翻}
周謂吳王曰陛下未信王遣子入侍周以闔門百口明之吳王為之流涕霑襟|{
	為于偽翻}
指天為誓周還而侍子不至但多設虛辭帝欲遣侍中辛毗尚書桓階往與盟誓并責任子吳王辭讓不受帝怒欲伐之劉曄曰彼新得志上下齊心而阻帶江湖不可倉卒制也|{
	卒讀曰猝}
帝不從九月命征東大將軍曹休前將軍張遼鎮東將軍臧霸出洞口|{
	據張遼傳帝遣遼與曹休至海陵臨江與諸將破呂範又據賀齊傳齊督扶州以上至皖黄武初魏使曹休來伐齊住新市會洞口諸軍遭風流溺賴齊未濟諸將倚以為勢休等憚之遂引軍還又據王淩傳遼等至廣陵臨江蓋廣陵即海陵也蕭子顯曰南兖州刺史每以秋月出海陵觀濤與京口對岸又據晉書譙王尚之傳桓玄攻尚之于歷陽使馮該斷洞浦焚舟艦則洞口在歷陽江邊明矣}
大將軍曹仁出濡須上軍大將軍曹真征南大將軍夏侯尚左將軍張郃右將軍徐晃圍南郡|{
	郃古合翻}
吳建威將軍呂範督五軍以舟軍拒休等左將軍諸葛瑾平北將軍潘璋將軍楊粲救南郡禆將軍朱桓以濡須督拒曹仁 冬十月甲子表首陽山東為壽陵|{
	首陽山在洛陽東北}
作終制務從儉薄不臧金玉|{
	臧讀曰藏}
一用瓦器令以此詔藏之宗廟副在尚書祕書三府|{
	其副本在尚書及祕書及三公府也前臧字因舊史後藏字用今字}
吳王以楊越蠻夷多未平集乃卑辭上書求自改厲若罪在難除必不見置當奉還土地民人寄命交州以終餘年又與浩周書云欲為子登求昏宗室|{
	為于偽翻}
又云以登年弱欲遣孫長緒張子布隨登俱來|{
	孫邵字長緒吳王稱尊號以邵為丞相}
帝報曰朕之與君大義已定豈樂勞師遠臨江漢|{
	樂音洛}
若登身朝到夕召兵還耳於是吳王改元黄武|{
	吳改元黄武亦以五德之運承漢為土德也}
臨江拒守帝自許昌南征復郢州為荆州|{
	是年二月置郢州吳畔復為荆州}
十一月辛丑帝如宛|{
	宛於元翻}
曹休在洞口自陳願將銳卒|{
	將即亮翻}
虎步江南因敵取資事必克捷若其無臣不須為念帝恐休便渡江驛馬止之侍中董昭侍側曰竊見陛下有憂色獨以休濟江故乎今者渡江人情所難就休有此志勢不獨行當須諸將臧霸等既富且貴無復他望|{
	復扶又翻}
但欲終其天年保守禄祚而已何肯乘危自投死地以求徼倖|{
	徼堅堯翻}
苟霸等不進休意自沮|{
	沮在呂翻}
臣恐陛下雖有敕渡之詔猶必沈吟未便從命也|{
	沈持林翻}
頃之會暴風吹吳呂範等船綆纜悉斷|{
	綆古杏翻纜盧瞰翻皆索也所以維舟者也}
直詣休等營下斬首獲生以千數吳兵迸散|{
	迸北孟翻}
帝聞之敕諸軍促渡軍未時進吳救船遂至收軍還江南曹休使臧霸追之不利將軍尹盧戰死 庚申晦日有食之 吳王使太中大夫鄭泉聘于漢漢太中大夫宗瑋報之吳漢復通 漢主聞魏師大出遺陸遜書曰賊今已在江漢吾將復東|{
	遺于季翻復扶又翻下同}
將軍謂其能然否遜答曰但恐軍新破創夷未復始求通親|{
	通親謂通使而交親也創初良翻復如字}
且當自補未暇窮兵耳若不推算欲復以傾覆之餘遠送以來者無所逃命 漢漢嘉太守黄元叛|{
	漢嘉郡本前漢青衣縣地屬蜀郡後漢順帝陽嘉二年改為漢嘉縣屬蜀郡屬國蜀分為漢嘉郡}
吳將孫盛督萬人據江陵中洲|{
	據潘璋傳則江陵中洲即百里洲也其洲自枝江縣西至上明東及江津江津北岸即江陵故城}
以為南郡外援

資治通鑑卷六十九
