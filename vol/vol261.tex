<!DOCTYPE html PUBLIC "-//W3C//DTD XHTML 1.0 Transitional//EN" "http://www.w3.org/TR/xhtml1/DTD/xhtml1-transitional.dtd">
<html xmlns="http://www.w3.org/1999/xhtml">
<head>
<meta http-equiv="Content-Type" content="text/html; charset=utf-8" />
<meta http-equiv="X-UA-Compatible" content="IE=Edge,chrome=1">
<title>資治通鑒_262-資治通鑑卷二百六十一_262-資治通鑑卷二百六十一</title>
<meta name="Keywords" content="資治通鑒_262-資治通鑑卷二百六十一_262-資治通鑑卷二百六十一">
<meta name="Description" content="資治通鑒_262-資治通鑑卷二百六十一_262-資治通鑑卷二百六十一">
<meta http-equiv="Cache-Control" content="no-transform" />
<meta http-equiv="Cache-Control" content="no-siteapp" />
<link href="/img/style.css" rel="stylesheet" type="text/css" />
<script src="/img/m.js?2020"></script> 
</head>
<body>
 <div class="ClassNavi">
<a  href="/24shi/">二十四史</a> | <a href="/SiKuQuanShu/">四库全书</a> | <a href="http://www.guoxuedashi.com/gjtsjc/"><font  color="#FF0000">古今图书集成</font></a> | <a href="/renwu/">历史人物</a> | <a href="/ShuoWenJieZi/"><font  color="#FF0000">说文解字</a></font> | <a href="/chengyu/">成语词典</a> | <a  target="_blank"  href="http://www.guoxuedashi.com/jgwhj/"><font  color="#FF0000">甲骨文合集</font></a> | <a href="/yzjwjc/"><font  color="#FF0000">殷周金文集成</font></a> | <a href="/xiangxingzi/"><font color="#0000FF">象形字典</font></a> | <a href="/13jing/"><font  color="#FF0000">十三经索引</font></a> | <a href="/zixing/"><font  color="#FF0000">字体转换器</font></a> | <a href="/zidian/xz/"><font color="#0000FF">篆书识别</font></a> | <a href="/jinfanyi/">近义反义词</a> | <a href="/duilian/">对联大全</a> | <a href="/jiapu/"><font  color="#0000FF">家谱族谱查询</font></a> | <a href="http://www.guoxuemi.com/hafo/" target="_blank" ><font color="#FF0000">哈佛古籍</font></a> 
</div>

 <!-- 头部导航开始 -->
<div class="w1180 head clearfix">
  <div class="head_logo l"><a title="国学大师官网" href="http://www.guoxuedashi.com" target="_blank"></a></div>
  <div class="head_sr l">
  <div id="head1">
  
  <a href="http://www.guoxuedashi.com/zidian/bujian/" target="_blank" ><img src="http://www.guoxuedashi.com/img/top1.gif" width="88" height="60" border="0" title="部件查字,支持20万汉字"></a>


<a href="http://www.guoxuedashi.com/help/yingpan.php" target="_blank"><img src="http://www.guoxuedashi.com/img/top230.gif" width="600" height="62" border="0" ></a>


  </div>
  <div id="head3"><a href="javascript:" onClick="javascript:window.external.AddFavorite(window.location.href,document.title);">添加收藏</a>
  <br><a href="/help/setie.php">搜索引擎</a>
  <br><a href="/help/zanzhu.php">赞助本站</a></div>
  <div id="head2">
 <a href="http://www.guoxuemi.com/" target="_blank"><img src="http://www.guoxuedashi.com/img/guoxuemi.gif" width="95" height="62" border="0" style="margin-left:2px;" title="国学迷"></a>
  

  </div>
</div>
  <div class="clear"></div>
  <div class="head_nav">
  <p><a href="/">首页</a> | <a href="/ShuKu/">国学书库</a> | <a href="/guji/">影印古籍</a> | <a href="/shici/">诗词宝典</a> | <a   href="/SiKuQuanShu/gxjx.php">精选</a> <b>|</b> <a href="/zidian/">汉语字典</a> | <a href="/hydcd/">汉语词典</a> | <a href="http://www.guoxuedashi.com/zidian/bujian/"><font  color="#CC0066">部件查字</font></a> | <a href="http://www.sfds.cn/"><font  color="#CC0066">书法大师</font></a> | <a href="/jgwhj/">甲骨文</a> <b>|</b> <a href="/b/4/"><font  color="#CC0066">解密</font></a> | <a href="/renwu/">历史人物</a> | <a href="/diangu/">历史典故</a> | <a href="/xingshi/">姓氏</a> | <a href="/minzu/">民族</a> <b>|</b> <a href="/mz/"><font  color="#CC0066">世界名著</font></a> | <a href="/download/">软件下载</a>
</p>
<p><a href="/b/"><font  color="#CC0066">历史</font></a> | <a href="http://skqs.guoxuedashi.com/" target="_blank">四库全书</a> |  <a href="http://www.guoxuedashi.com/search/" target="_blank"><font  color="#CC0066">全文检索</font></a> | <a href="http://www.guoxuedashi.com/shumu/">古籍书目</a> | <a   href="/24shi/">正史</a> <b>|</b> <a href="/chengyu/">成语词典</a> | <a href="/kangxi/" title="康熙字典">康熙字典</a> | <a href="/ShuoWenJieZi/">说文解字</a> | <a href="/zixing/yanbian/">字形演变</a> | <a href="/yzjwjc/">金 文</a> <b>|</b>  <a href="/shijian/nian-hao/">年号</a> | <a href="/diming/">历史地名</a> | <a href="/shijian/">历史事件</a> | <a href="/guanzhi/">官职</a> | <a href="/lishi/">知识</a> <b>|</b> <a href="/zhongyi/">中医中药</a> | <a href="http://www.guoxuedashi.com/forum/">留言反馈</a>
</p>
  </div>
</div>
<!-- 头部导航END --> 
<!-- 内容区开始 --> 
<div class="w1180 clearfix">
  <div class="info l">
   
<div class="clearfix" style="background:#f5faff;">
<script src='http://www.guoxuedashi.com/img/headersou.js'></script>

</div>
  <div class="info_tree"><a href="http://www.guoxuedashi.com">首页</a> > <a href="/SiKuQuanShu/fanti/">四库全书</a>
 > <h1>资治通鉴</h1> <!--         下载:【右键另存为】即可 --></div>
  <div class="info_content zj clearfix">
  
<div class="info_txt clearfix" id="show">
<center style="font-size:24px;">262-資治通鑑卷二百六十一</center>
    資治通鑑卷二百六十一 宋 司馬光 撰<br />
<br />
  胡三省 音註<br />
<br />
  唐紀七十七【起彊圉大荒落盡屠維協洽凡三年】<br />
<br />
  昭宗聖穆景文孝皇帝中之上<br />
<br />
  乾寧四年春正月甲申韓建奏防城將張行思等【張行思華州防城將也將即亮翻】告睦濟韶通彭韓儀陳八王【皆嗣王也睦韶韓代宗之後彭肅宗之後陳文宗之後史皆逸其名及其世系】謀殺臣刼車駕幸河中建惡諸王典兵【惡烏路翻】故使行思等告之上大驚召建諭之建稱疾不入令諸王詣建自陳建表稱諸王忽詣臣理所不測事端【建言諸王為變事出不測也】臣詳酌事體不應與諸王相見又稱諸王當自避嫌疑不可輕為舉措陛下若以友愛含容請依舊制令歸十六宅妙選師傅教以詩書不令典兵預政【援開元天實舊制不令諸王出閭】且曰乞散彼烏合之兵用光麟趾之化【詩序曰關雎之化行雖衰世之公子皆信厚如麟趾之時】建慮上不從引麾下精兵圍行宫【以兵脇君】表疏連上【上時掌翻】上不得已是夕詔諸王所領軍士並縱歸田里諸王勒歸十六宅其甲兵並委韓建收掌建又奏陛下選賢任能足清禍亂何必别置殿後四軍【四軍即安聖捧宸保寧宣化也置見上卷上年】顯有厚薄之恩乖無偏無黨之道【書曰無編無黨王道蕩蕩韓建安識書語李巨川教之耳宜其不免於誅也一本厚下更有有字】且所聚皆坊市無賴姧猾之徒平居猶思禍變臨難必不為用【難乃旦翻】而使之張弓挾刃密邇皇輿臣竊寒心乞皆罷詔亦從之於是殿後四軍二萬餘人悉散天子之親軍盡矣捧日都頭李筠石門扈從功第一【石門扈從事見上卷二年從才用翻】建復奏斬於大雲橋【復扶又翻大雲橋在華州大雲寺前武后時令天下諸州各置大雲寺以藏大雲經皆受命之符也】建又奏玄宗之末永王璘暫出江南遽謀不軌【事見肅宗紀至德元載二載】代宗時吐蕃入寇光啓中朱玫亂常皆援立宗支以繫人望【謂吐蕃立廣武王承宏朱玫立襄王煴也事各見前紀援于元翻】今諸王銜命四方者乞皆召還【指言延王戒丕等】又奏諸方士出入禁庭眩惑聖聰宜皆禁止無得入宫【指言許巖士等】詔悉從之建既幽諸王於别第知上意不悦乃奏請立德王為太子欲以解之丁亥詔立德王祐為皇太子仍更名裕【更工衡翻 考異曰勤王錄曰公以儲副之設國之大本上表云云敕宜從允時正月十一日也當四日之間而儲君奉冢祀宗室歸藩邸蓬頭突鬢之士不入于禁門文成五利之徒不陳其左道君父開悟遐邇詠歌人不震驚市無易肆公之力也李巨川著書矯誣善惡乃至於此今從實錄】 龎師古葛從周併兵攻鄆州朱瑄兵少食盡不復出戰但引水為深壕以自固辛卯師古等營於水西南命為浮梁癸巳潛決濠水丙申浮梁成師古夜以中軍先濟瑄聞之弃城奔中都【按九域志中都縣在鄆州東南六十里】葛從周逐之野人執瑄及妻子以獻【僖宗中和二年朱瑄得鄆州至是而亡 考異曰薛居正五代史梁太祖紀辛卯營于濟水之次龎師古令諸將撤木為橋乙未夜師古以中軍先濟朱瑄弃壁夜走葛從周擒瑄并妻男以獻按濟水自王莽時大旱不復能絶河而南自是河南無濟水編遺錄曰五月遣騎於鄆州軍前追從周徑往洹水董師以代侯言師古留攻鄆梁太祖實錄四年正月復以洹水之師大舉伐鄆十五日辛卯營其西南河外龎師古命諸將撤木為橋以圖宵濟癸巳前軍以心膂百人盜决河口甲午浮橋集水次乙未夜師古中軍先濟聲振壁内朱瑄聞之弃壁走編遺錄四年正月己卯朱瑄兵少糧盡不敢出戰然深溝高壘難越也從周師古乃取清河内小舟採野葛草茅索之以為巨纜乃於其牆南建浮橋丙申功就我師渡橋朱瑄奔遁皆不云濟水師古去年三月已敗鄆兵于馬頰追至西門據故洛亭子為寨乙未夜先濟蓋鄆城下清河水疑朱瑄引之以環城固守故師古等為浮橋以濟師河既可决明非自然之水也舊紀癸未龎師古陷鄆州朱瑄與妻榮氏潰圍走瑄至中都為野人所殺榮氏俘於軍新紀丙申全忠陷鄆州實錄二月丙午朔陷鄆州瑄至中都為亂兵所殺妻榮至汴為尼據薛史辛卯營於濟水則癸未鄆未破也新紀云丙申陷鄆實錄二月蓋約奏到今從編遺錄新紀】 己亥罷孫偓鳳翔四面行營節度等使【赦李茂貞故罷鳳翔西面行營】以副都統李思諫為寧塞節度使【按方鎮表光化元年更延州保塞節度為寧塞節度】 錢鏐使行軍司馬杜稜救婺州安仁義移兵攻睦州不克而還【安仁義攻婺州見上卷上年還從宣翻又如字】 朱全忠入鄆州以龎師古為天平留後 【考異曰舊紀梁大祖實錄薛居正五代史師古傳皆云師古為鄆州留後編遺錄薛史梁紀皆云友裕按編遺錄三月丙子以友裕為鄆州留後師古為徐州留後蓋初以師古守鄆州後以友裕代之而徙師古於徐州也】朱瑾留大將康懷貞守兖州與河東將史儼李承嗣掠徐州之境以給軍食【九域志兖州南一百一十里即徐州界】全忠聞之遣葛從周將兵襲兖州懷貞聞鄆州已失守汴兵奄至遂降二月戊申從周入兖州獲瑾妻子朱瑾還無所歸帥其衆趨沂州刺史尹處賓不納走保海州【降戶江翻帥讀曰率趨七喻翻九域志兖州三百四十五里東至沂州沂古琅邪也沂州東至海州一百八十里】為汴兵所逼與史儼李承嗣擁州民度淮奔楊行密【光啓二年朱瑾取兖州至是而敗】行密逆之於高郵表瑾領武寧節度使【領遙領也】全忠納瑾之妻引兵還張夫人逆於封丘【九域志封丘縣在汴州北六十里】全忠以得瑾妻告之夫人請見之瑾妻拜夫人答拜且泣曰兖鄆與司空同姓約為兄弟以小故恨望起兵相攻使吾姒辱於此它日汴州失守吾亦如吾姒之今日乎【姒詳里翻長婦曰姒又兄弟之妻相呼曰姒互相尊稱之辭也】全忠乃送瑾妻於佛寺為尼斬朱瑄于汴橋于是鄆齊曹棣兖沂密徐宿陳許鄭滑濮皆入于全忠【鄆齊曹棣天平軍兖沂密泰寧軍徐宿感化軍陳許忠武軍鄭滑濮宣義軍此五鎮之地也】惟王師範保淄青一道亦服於全忠李存信在魏州聞兖鄆皆陷引兵還淮南舊善水戰不知騎射及得河東兖鄆兵軍聲大振史儼李承嗣皆河東驍將李克用深惜之遣使間道詣楊行密請之【間古莧翻】行密許之亦遣使詣克用修好【好呼到翻】戊午王建遣卭州刺史華洪彭州刺史王宗祐將兵五萬攻東川【卭渠容翻華戶化翻】以戎州刺史王宗謹為鳳翔西面行營先鋒使敗鳳翔將李繼徽等於玄武【玄武漢氐道縣晉改曰玄武唐初屬益州時屬梓州宋朝改曰中江在梓州西九十里敗蒲邁翻】繼徽本姓楊名崇本茂貞之假子也 己未赦天下 【考異曰實錄降德音曲赦天下云德音即非赦既云曲赦即不及天下實錄誤也】 上饗行廟【時駐蹕華州太常禮院請權立行廟以備告饗】 庚申王建以决雲都知兵馬使王宗侃為應援開峽都指揮使將兵八千趨渝州决勝都知兵馬使王宗阮為開江防送進奉使將兵七千趨瀘州辛酉宗侃取渝州降刺史牟崇厚癸酉宗阮拔瀘州斬刺史馬敬儒峽路始通【渝瀘皆東川巡屬王建志在廣地假通峽路進奉以為名耳趨七喻翻】鳳翔將李繼昭救梓州留偏將守劔門西川將王宗播撃擒之乙亥門下侍郎同平章事孫偓罷守本官中書侍郎<br />
<br />
  同平章事朱朴罷為袐書監朴既秉政所言皆不效【朱朴自詭月餘可致太平見上卷上年】外議沸騰太子詹事馬道殷以天文將作監許巖士以毉得幸於上韓建誣二人以罪而殺之且言偓朴與二人交通故罷相【馬道殷許巖士在上左右二相因之以白事此必有之】 詔以楊行密為江南諸道行營都統以討武昌節度使杜洪【按新書洪傳洪附朱全忠絶東南貢獻路命楊行密討之者以此】 張佶克邵州擒蔣勛【潭兵攻蔣勛事始上卷三年正月佶巨乙翻】 三月丙子朱全忠表曹州刺史葛從周為泰寧留後朱友裕為天平留後龎師古為武寧留後【朱全忠表以三鎮授三將以樹黨此時蓋復改感化為武寧】 保義節度使王珙攻護國節度使王珂珂求援於李克用珙求援於朱全忠宣武將張存敬楊師厚敗河中兵於猗氏南河東將李嗣昭敗陜兵於猗氏又敗之於張店遂解河中之圍【敗補邁翻陜失冉翻】師厚斤溝人【九域志穎州萬夀縣有斤溝鎮萬夀唐汝隂縣之百口鎮也宋朝開寶六年置縣】嗣昭克用弟克柔之假子也 更名感義軍曰昭武【更工衡翻】治利州以前静難節度使蘇文建為節度使【難乃旦翻】 夏四月以同州防禦使李繼瑭為匡國節度使 【考異曰實錄賜同州號匡國軍以防禦使李繼塘為匡國節度使按新方鎮表乾寧二年賜同州號匡國軍王行約已嘗為匡國節度使蓋行約死繼瑭但為防禦使今始復舊名耳】繼瑭茂貞之養子也 以右諫議大夫李洵為兩川宣諭使和解王建及顧彦暉 辛亥錢鏐遣顧全武等將兵三千自海道救嘉興己未至城下擊淮南兵大破之【淮南圍嘉興始上卷二年】 杜洪為楊行密所攻求救於朱全忠全忠遣其將聶金掠泗州【聶尼輒翻姓也】朱友恭攻黄州行密遣右黑雲都指揮使馬珣等救黄州黄州刺史瞿章聞友恭至弃城擁衆南保武昌寨【武昌漢古縣唐屬鄂州九域志在州東北一百八十里今置夀昌軍】 癸亥兩浙將顧全武等破淮南十八營虜淮南將士魏約等三千人淮南將田頵屯驛亭埭【埭徒耐翻】兩浙兵乘勝逐之甲戌頵自湖州奔還【自嘉興退軍取道湖州還宣州】兩浙兵追敗之【敗補邁翻】頵衆死者千餘人 韓建惡刑部尚書張禕等數人皆誣奏貶之【惡烏路翻禕吁韋翻考異曰實錄貶刑部尚書張禕趙崇蘇循等為衡州司馬韓建惡之誣奏貶焉禕等必不皆為刑部尚書皆貶衡州司馬實錄誤也】 五月加奉國節度使崔洪同平章事 辛巳朱友恭為浮梁於樊港【武昌西三里有樊山山下有樊溪注于江謂之樊口朱友恭蓋跨江為浮梁抵樊口以攻武昌也】進攻武昌寨壬午拔之執瞿章遂取黄州 【考異曰薛居正五代史梁紀五月丁丑朱友恭遣使上言大破淮寇於昌收復黄鄂二州新紀壬午全忠陷黄州刺史瞿章死之朱友恭傳云瞿章十國紀年作瞿章吳錄云執刺史瞿章當可据】馬珣等皆敗走 丙戍王建以節度副使張琳守成都【張琳王建之腹心建之攻陳敬瑄也亦使之守卭州】自將兵五萬攻東川更華洪姓名曰王宗滌【華洪累戰有功王建於是養以為子以收其力用然殺洪之心蓋已萌於此時矣華戶化翻】六月己酉錢鏐如越州受鎮東節钺 李茂貞表王<br />
<br />
  建攻東川連兵累歲不聽詔命【王建豈特攻東川哉李茂貞山南巡屬諸州建取之亦多矣力不能制欲挾天子之令以臨之】甲寅貶建南州刺史【新志武德三年開黔南蠻置南州宋白曰南州戰國時為巴國界秦則巴郡之地漢為江州之境唐武德三年割渝州之東界置南州】乙卯以茂貞為西川節度使以覃王嗣周為鳳翔節度使癸亥王建克梓州南寨執其將李繼寧丙寅宣諭使李洵至梓州【四月李洵受命而使六月始至】己巳見建于張杷砦建指執旗者曰戰士之情不可奪也 覃王赴鎮李茂貞不受代【李茂貞之狡悍豈肯以鳳翔授人涉險而争蜀邪】圍覃王於奉天 置寧遠軍於容州以李克用大將蓋寓領節度使【李克用之平王行瑜也蓋寓以功領容管觀察使今升領節度使】 秋七月加荆南節度使成汭兼侍中 韓建移書李茂貞茂貞解奉天之圍覃王歸華州 以天雄節度使李繼徽為静難節度使【李繼徽自秦州徙邠州邠寧亦為李茂貞有矣難乃旦翻】 庚戍錢鏐還杭州【自越還杭】遣顧全武取蘇州乙未拔松江【松江在蘇州南四十里淮南立寨以守之】戊戌拔無錫【無錫漢縣唐屬常州九域志在州東九十一里】辛丑拔常熟華亭【宋白曰常熟縣後漢至吳為司鹽都尉晉置南沙縣梁置常熟縣今崑山縣東一百三十里常熟故城是也九域志在蘇州北七十五里天寶十載分嘉興置華亭縣屬蘇州在州西南今屬秀州】 初李克用取幽州【見二百五十九卷乾寧元年】表劉仁恭為節度使留戌兵及腹心將十人典其機要租賦供軍之外悉輸晉陽及上幸華州克用徵兵於仁恭又遺成德節度使王鎔義武節度使王郜書【遺唯季翻郜音告】欲與之共定關中奉天子還長安仁恭辭以契丹入寇【洪遵曰契丹之讀如喫惟新唐書有音今從欺訖翻】須兵扞禦請俟虜退然後承命克用屢趣之【趣讀日促】使者相繼數月兵不出克用移書責之仁恭抵書於地慢罵囚其使者欲殺河東戍將戍將遁逃獲免【克用留兵戍幽州見上卷乾寧二年】克用大怒八月自將擊仁恭【為克用計者先舉河東之甲以勤王事定之後然後移兵臨燕以問罪劉仁恭安所逃其死乎不知出此遽興忿兵其敗宜矣】 上欲幸奉天親討李茂貞令宰相議之宰相切諫乃止【議者率謂昭宗忿不思難然亦可悲矣】延王戒丕還自晉陽【戒丕使晉陽見上卷上年】韓建奏自陛下即<br />
<br />
  位以來與近輔交惡【近輔邠岐同華也】皆因諸王典兵凶徒樂禍【樂音洛】致鑾輿不安比者臣奏罷兵權【比毗至翻近也】實慮不測之變今聞延王覃王尚苞隂計願陛下聖斷不疑【斷丁亂翻】制於未亂【周書云制治於未亂建奏引之李巨川之辭也】則社稷之福【韓建欲殺諸王久矣憚李克用故未敢發延王既還知克用之兵不能至故决請殺之】上曰何至於是數日不報建乃與知樞密劉季述矯制發兵圍十六宅諸王被髪或緣垣或升屋呼曰宅家救兒【被皮義翻呼火故翻唐末宫中率稱天子曰宅家】建擁通沂睦濟韶彭韓陳覃延丹十一王至石隄谷盡殺之【石隄谷在華州西歐陽脩集古錄云殽阬君祠今謂之五部神廟其象有石隄西戍樹谷五樓先生東臺御史王剪將軍莫曉其義其碑云石隄樹谷南通商雒又云前世通利吏民興貴有御史大夫將軍牧伯故為立祠以報其功乃知五部之號自漢有之如此則石隄者石隄谷之神五部神之一也唐韓建殺諸王於石隄谷蓋此谷也殽阬神祠在華州鄭縣 考異曰舊紀是日通覃以下十一王并其侍者皆為建兵所擁至石隄谷無長少皆殺之唐補紀曰六宅諸王准前商量請置殿後都韓建怨怒進狀争論與諸王互說短長上乃縛韓王克良已下十人送韓建府建以棘刺圍於大廳經宿不與相見軍吏諫遂請諸王歸宫散却殿後都新紀八月韓建殺通王滋沂王湮韶王彭王嗣韓王嗣陳王嗣覃王嗣周嗣延王戒丕嗣丹王允按舊紀韓建奏睦王濟王韶王通王彭王韓王儀王陳王八人新宗室傳初帝遣嗣延王戒丕嗣丹王允往見李克用又有覃王嗣周則是十一人新紀傳儀作沂按昭宗子禋封沂王不應更封宗室舊紀儀禋恐可据】以謀反聞 貶禮部尚書孫偓為南州司馬祕書監朱朴先貶夔州司馬再貶郴州司戶 【考異曰實錄朴貶郴州司戶薛廷珪鳳閣書詞有朴自祕書監責除蜀王傅分司東都制云苞藏莫顧於朝綱進見不由於相府復云猶希顧問之間來撓澄清之化又貶渠州司馬制云争臣條奏憲府極言指陳負固之謀忿嫉崇姦之計與此稍異今從實錄】朴之為相何迎驟遷至右諫議大夫至是亦貶湖州司馬【何迎薦朱朴見上卷上年】 鍾傳欲討吉州刺史襄陽周琲琲帥其衆奔廣陵【琲部凂翻又蒲昧翻帥讀曰率】 王建與顧彦暉五十餘戰九月癸酉朔圍梓州蜀州刺史周德權言於建曰公與彦暉争東川三年士卒疲於矢石百姓困於輸輓【輓音晩】東川羣盜多據州縣彦暉懦而無謀欲為偷安之計皆啗以厚利【啗徒監翻】恃其救援故堅守不下今若遣人諭賊帥以禍福【帥所類翻】來者賞之以官不服者威之以兵則彼之所恃反為我用矣建從之彦暉勢益孤德權許州人也 丁丑李克用至安塞軍【安塞軍在蔚州之東媯州之西新志幽州丁零川西南有安塞軍】辛巳攻之幽州將單可及引騎兵至【單上演翻姓也又都寒翻亦姓也元魏孝文帝改代北内入諸渇單氏為單氏】克用方飲酒前鋒白賊至矣克用醉曰仁恭何在對曰但見可及輩克用瞋目曰【瞋昌真翻】可及輩何足為敵亟命撃之是日大霧不辨人物幽州將楊師侃伏兵於木瓜澗【據新書木瓜澗亦在蔚州界】河東兵大敗失亡大半【史言李克用輕敵又不得天時故敗】會大風雨震電幽州兵解去【史言克用敗而得免者亦天也】克用醒而後知敗責大將李存信等曰吾以醉廢事汝曹何不力争【邢州之叛莘縣之潰木瓜澗之敗皆李存信之罪也克用終親任之可謂失刑矣】 湖州刺史李彦徽欲以州附於楊行密【去年楊行密表彦徽知湖州故欲附之】其衆不從彦徽奔廣陵都指揮使沈攸以州歸錢鏐【錢鏐自此遂有湖州】 以彰義節度使張璉為鳳翔西北行營招討使以討李茂貞復以王建為西川節度使同平章事加義武節度使<br />
<br />
  王郜同平章事削奪新西川節度使李茂貞官爵復姓名宋文通【始以李茂貞之請而討王建既而又以茂貞拒命赦王建而討茂貞朝廷號令朝出而暮改諸侯其孰尊而信之皇威不振自取之也李茂貞賜姓名見二百五十六卷光啓二年】 朱全忠既得兖鄆甲兵益盛乃大舉擊楊行密遣龎師古以徐宿宋滑之兵七萬壁清口【清口即今之清河口】將趨揚州【趨七喻翻下同】葛從周以兖鄆曹濮之兵壁安豐將趨夀州【安豐漢六縣故城在縣南後漢置安豐縣至唐屬夀州九域志曰安豐縣在州東南六十餘里蓋唐之夀州治夀春縣即六朝夀陽之地五代之末周世宗克夀州徙治下蔡故宋朝安豐在夀州東南】全忠自將屯宿州淮南震恐 匡國節度使李繼瑭聞朝廷討李茂貞而懼韓建復從而搖之【復扶又翻】繼瑭奔鳳翔冬十月以建為鎮國匡國兩軍節度使【韓建始兼有同華】 壬子知遂州侯紹帥衆二萬乙卯知合州王仁威帥衆千人戊午鳳翔李繼溥以援兵二千皆降于王建【帥讀曰率】建攻梓州益急庚申顧彦暉聚其宗族及假子共飲遣王宗弼自歸于建【王宗弼為東川兵所擒事見上卷二年】酒酣命其假子瑤殺已及同飲者然後自殺【僖宗光啓三年顧彦朗得東川傳至弟彦暉至是而滅】建入梓州【乾寧二年王建始攻東川蓋三年而後克之】城中兵尚七萬人建命王宗綰分兵徇昌普等州以王宗滌為東川留後 劉仁恭奏稱李克用無故稱兵見討本道大破其黨于木瓜澗請自為統帥以討克用【帥所類翻】詔不許又遺朱全忠書【遺唯季翻】全忠奏加仁恭同平章事朝廷從之仁恭又遣使謝克用陳去就不自安之意克用復書略曰今公仗钺控兵理民立法擢士則欲其報德選將則望彼酬恩已尚不然人何足信【克用言仁恭背已人亦將背之也將即亮翻】僕料猜忌出於骨肉嫌忌生於屏帷持干將而不敢授人捧盟盤而何詞著誓甲子立皇子祕為景王祚為輝王祺為祁王 加彰義節度使張璉同平章事 楊行密與朱瑾將兵三萬拒汴軍於楚州别將張訓自漣水引兵會之【乾寧二年楊行密始取漣水令張訓守之】行密以為前鋒龎師古營於清口或曰營地汙下不可久處不聽【汙烏瓜翻處昌呂翻】師古恃衆輕敵居常奕棊朱瑾壅淮上流欲灌之或以告師古師古以為惑衆斬之十一月癸酉瑾與淮南將侯瓚【瓚才旱翻】將五千騎潛度淮用汴人旗幟自北來趣其中軍【趣七喻翻】張訓踰柵而入士卒蒼黄拒戰淮水大至汴軍駭亂行密引大軍濟淮與瑾等夾攻之汴軍大敗斬師古及將士首萬餘級餘衆皆潰葛從周營於夀州西北夀州團練使朱延夀擊破之退屯濠州聞師古敗奔還行密瑾延夀乘勝追之及於渒水【水經注渒水出廬江潛縣西南霍山東北又東北過六縣東又西北過安豐縣故城西北入于淮類篇渒必至切水名在弋陽按今渒河在來遠鎮西十里來遠鎮即東正陽也東至夀州二百里】從周半濟淮南兵擊之殺溺殆盡從周走免遏後都指揮使牛存節弃馬步鬭諸軍稍得濟淮凡四日不食會大雪汴卒緣道凍餒死還者不滿千人全忠聞敗亦奔還行密遺全忠書曰龎師古葛從周非敵也公宜自來淮上决戰【遺于季翻】行密大會諸將謂行軍副使李承嗣曰始吾欲先趣夀州【趣七喻翻】副使云不如先向清口師古敗從周自走今果如所料賞之錢萬緡【賞其勝算先定】表承嗣領鎮海節度使行密待承嗣及史儼甚厚第舍姬妾咸選其尤者賜之故二人為行密盡力屢立功竟卒於淮南【為于偽翻卒于恤翻史言楊行密能用人安仁義亦涉陁也行密待之非不厚而終叛於行密狼子野心固自有難馴養者】行密由是遂保據江淮之間全忠不能與之争戊寅立淑妃何氏為皇后后東川人生德王輝王 威武節度使王潮弟審知為觀察副使有過潮猶加捶撻【捶止橤翻】審知無怨色潮寢疾捨其子延興延虹延豐延休命審知知軍府事十二月丁未潮薨審知以讓其兄泉州刺史審邽審邽以審知有功辭不受審知自稱福建留後表于朝廷 壬戌王建自梓州還戊辰至成都是歲南詔驃信舜化有上皇帝書函及督爽牒中書木夾年號中興朝廷欲以詔書報之王建上言南詔小夷不足辱詔書臣在西南彼必不敢犯塞從之黎雅間有淺蠻曰劉王郝王楊王各有部落【黎雅西南大山長谷皆蠻居之所在深遠而三王部落居近漢界故曰淺蠻】西川歲賜繒帛三千匹使覘南詔亦受南詔賂詗成都虛實【繒慈陵翻覘丑廉翻又丑艶翻詗古迥翻又翾正翻】每節度使到官三王帥酋長詣府【帥讀曰率酋慈由翻長知兩翻】節度使自謂威德所致表于朝廷而三王隂與大將相表裏節度使或失大將心則教諸蠻紛擾先是節度使多文臣【先悉薦翻】不欲生事故大將常藉此以邀姑息而南詔亦憑之屢為邊患及王建鎮西川絶其舊賜斬都押牙山行章以懲之【山行章陳田舊將王建因其與淺蠻表裏而斬之既以威示諸蠻亦除舊務盡】卭崍之南不置障候不戍一卒【謂卭崍關以南也】蠻亦不敢侵盜其後遣王宗播擊南詔三王【漏】洩軍事召而斬之【史言安邊之術惟洞知近塞蕃落情偽而折其姦則外夷不敢有所侮而動】 右拾遺張道古上疏稱國家有五危二亂昔漢文帝即位未幾明習國家事【幾居豈翻見十三卷漢文帝元年】今陛下登極已十年【帝文德元年踐阼至此十年若以即位踰年改元數之則九年】而曾不知為君馭臣之道太宗内安中原外開四夷海表之國莫不入臣今先朝封域日蹙幾盡【幾居依翻】臣雖微賤竊傷陛下朝廷社稷始為姧臣所弄終為賊臣所有也【漢唐之亡誠如張道古之言】上怒貶道古施州司戶【宋白曰施州漢巫縣地吳大帝分巫縣立沙渠縣後周建德三年於此置施州唐因之舊志施州京師南二千七百九里】仍下詔罪狀道古宣示諫官【昭宗處艱危之中猶罪言者其亡宜矣】道古青州人也【張道古見於通鑑者惟此事著其州里蓋傷之】<br />
<br />
  光化元年【是年八月還京方改元】春正月兩浙江西武昌淄青各遣使詣闕請以朱全忠為都統討楊行密【兩浙錢鏐江西鍾傳武昌杜洪淄青王師範皆憚楊行密之強而黨附朱全忠者也】詔不許 加平盧節度使王師範同平章事【淄青平盧軍】 以兵部尚書劉崇望同平章事充東川節度使以昭信防禦使馮行襲為昭信節度使【方鎮表光化元年置昭信防禦使治金州與此異】 上下詔罪已息兵復李茂貞姓名官爵應諸道討鳳翔兵皆罷之【韓建之志也】壬辰河中節度使王珂親迎於晉陽【迎魚敬翻】李克用遣其將李嗣昭守河中 李茂貞韓建皆致書於李克用言大駕出幸累年乞修和好【好呼到翻】同奬王室兼乞丁匠助修宫室克用許之 初王建攻東川顧彦暉求救於李茂貞茂貞命將出兵救之【事見上】不暇東逼乘輿詐稱改過與韓建共翼戴天子又聞朱全忠營洛陽宫累表迎車駕茂貞韓建懼請修復宫闕奉上歸長安詔以韓建為修宫闕使 【考異曰實錄建以行宫卑庳無眺覽之所表獻城南别墅建初修南莊起樓觀疏池沼欲為南内行廢立之事其叔父豐見其跋扈謂建曰汝陳許間一民乘時危亂位至方鎮不能感君父之惠而欲以同華兩州百里之地行廢立覆族在旦莫矣吾不如先自裁免為汝所累由是建稍弭其志及李茂貞表請助營宫苑又聞朱全忠繕治洛陽累表迎駕建懼故急營葺長安率諸道助役而又親程功焉按建若欲廢立何必先營南内今不取】諸道皆助錢及工材建使都將蔡敬思督其役既成二月建自往視之 錢鏐請徙鎮海軍於杭州從之【鎮海軍木治潤州今徙軍額於杭州】 復以李茂貞為鳳翔節度使 三月己丑以王審知充威武留後 朱全忠遣副使萬年韋震入奏事求兼鎮天平朝廷未之許震力争之【穆敬以後威令已不振然藩鎮所遣奏事官不敢力争於朝也】朝廷不得已以全忠為宣武宣義天平三鎮節度使全忠以震為天平留後以前台州刺史李振為天平節度副使【按歐史李振傳振為金吾衛將軍拜台州刺史盜起浙東不果行乃西歸過汴以策于朱全忠全忠留之遂為全忠用】振抱真之曾孫也【代德之間李抱真鎮昭義有大功】 淮南將周本救蘇州兩浙將顧全武擊破之淮南將秦裴以兵三千人拔崑山而戍之【崑山漢婁縣地梁分婁縣置信義縣又分信義置崑山縣取縣界崑山為名唐屬蘇州九域志在州東七十里】 以潭州刺史判湖南軍府事馬殷知武安留後時湖南管内七州賊帥楊師遠據衡州唐世旻據永州蔡結據道州陳彦謙據郴州魯景仁據連州【路振九國志唐旻蔡結皆以郡人聚兵據郡陳彦謙桂陽人殺刺史黄岳據郴州魯景仁本從黄巢以病留連州遂據之帥所類翻郴丑林翻】殷所得惟潭邵而已【為馬殷盡取諸州張本】 義昌節度使盧彦威性殘虐又不禮於鄰道與盧龍節度使劉仁恭争鹽利仁恭遣其子守文將兵襲滄州彦威弃城挈家奔魏州羅弘信不納乃奔汴州【光啓元年盧彦威得滄景至是而亡】仁恭遂取滄景德三州以守文為義昌留後仁恭兵勢益盛【併幽滄兩鎮之兵故勢益盛】自謂得天助有併吞河朔之志為守文請旌節【為于偽翻下為吾同】朝廷未許會中使至范陽仁恭語之曰【語牛倨翻】旌節吾自有之但欲得長安本色耳何為累章見拒為吾言之其悖慢如此【悖蒲内翻又蒲沒翻】 朱全忠與劉仁恭修好【好呼到翻】會魏博兵擊李克用夏四月丁未全忠至鉅鹿城下敗河東兵萬餘人逐北至青山口【五代志邢州龍岡縣隋文帝開皇十六年置青山縣煬帝大業初省入龍岡敗補邁翻】以護國節度使王珂兼侍中【珂丘何翻】 丁卯朱全忠遣葛從周分兵攻洺州戊辰拔之斬刺史邢善益 五月己巳朔赦天下葛從周攻邢州刺史馬師素弃城走辛未磁州刺史袁奉滔自剄全忠以從周為昭義留後守邢洺磁三州而還【并汴自此歲争邢洺磁三州】 以武定節度使李繼密為山南西道節度使【李繼密自洋州徙興元】 朝廷聞王建已用王宗滌為東川留後乃召劉崇望還為兵部尚書仍以宗滌為留後 湖南將姚彦章言於馬殷請取衡永道連郴五州仍薦李瓊為將殷以瓊及秦彦睴為嶺北七州游奕使張圖英李唐副之【五州併潭邵為七】將兵攻衡州斬楊師遠引兵趣永州【趣七喻翻】圍之月餘唐世旻走死殷以李唐為永州刺史 六月以濠州刺史趙珝為忠武節度使珝犨之弟也【珝况羽翻犨昌牛翻】 秋七月加武貞節度使雷滿同平章事【方鎮表光化元年置武貞節度領澧朗溆三州治澧州】加鎮南節度使鍾傳兼侍中 忠義節度使趙匡凝聞朱全忠有清口之敗【忠義軍山南東道清口之敗見去年十一月】隂附於楊行密全忠遣宿州刺史尉氏氏叔琮將兵伐之丙申拔唐州擒隨州刺史趙匡璘敗襄州兵於鄧城【敗補邁翻】 八月庚戍改華州為興德府【以車駕駐蹕故也】 戊午汴將康懷貞襲鄧州克之擒刺史國湘趙匡凝懼遣使請服於朱全忠全忠許之【為朱全忠再攻趙匡凝張本】 己未車駕發華州壬戌至長安甲子赦天下改元【改元光化】上欲藩鎮相與輯睦以太子賓客張有孚為河東汴州宣慰使賜李克用朱全忠詔又令宰相與之書使之和解克用欲奉詔而恥於先自屈乃致書王鎔使通於全忠全忠不從【朱全忠兵力方強故不從】 九月乙亥加韓建守太傅興德尹【先是上駐蹕華州因以華州為興德府】加王鎔兼中書令羅弘信守侍中 己丑東川留後王宗滌言於王建以東川封疆五千里文移往還動踰數月請分遂合瀘渝昌五州别為一鎮建表言之 顧全武攻蘇州城中及援兵食皆盡甲申淮南所署蘇州刺史臺濛弃城走【金城湯池非粟不守臺濛雖淮南良將奈之何哉】援兵亦遁全武克蘇州【乾寧三年淮南陷蘇州今果如顧全武之言而復之】追敗周本等于望亭【九域志常州無錫縣有望亭鎮在蘇州北四十五里又四十五里至無錫敗補邁翻】獨秦裴守崑山不下全武帥萬餘人攻之【帥讀曰率】裴屢出戰使病者被甲執矛壯者彀弓弩全武每為之却【見其弓弩之力及遠故為之却彀居侯翻為于偽翻】全武檄裴令降【降戶江翻下同】全武嘗為僧裴封函納款全武喜召諸將發函乃佛經一卷全武大慙曰裴不憂死何暇戲予益兵攻城引水灌之城壞食盡裴乃降錢鏐設千人饌以待之【饌雛晥翻又雛戀翻】乃出【乃當作及】羸兵不滿百人【觀通鑑上文秦裴以三千人取崑山而守之及其降也羸兵不滿百人則其兵死於戰守者殆盡其存者僅三十之一耳】鏐怒曰單弱如此何敢久為旅拒【旅衆也怙衆而拒捍曰旅拒】對曰裴義不負楊公今力屈而降耳非心降也鏐善其言顧全武亦勸鏐宥之鏐從之時人稱全武長者【顧全武甚識而度所以能佐錢鏐保據一方長知兩翻】 魏博節度使羅弘信薨 【考異曰薛居正五代史梁紀弘信傳太祖紀年錄皆云弘信八月卒按八月昭宗還京弘信猶加官舊紀傳九月卒今從之實錄十月約奏到也】軍中推其子節度副使紹威知留後 汴將朱友恭將兵還自江淮過安州【朱友恭克黄州還過安州九域志黄州西至安州三百里】或告刺史武瑜潛與淮南通謀取汴軍冬十月己亥友恭攻而殺之 李克用遣其將李嗣昭周德威將步騎二萬出青山將復山東三州【山東三州邢洺磁也是年五月葛從周取之】壬寅進攻邢州葛從周出戰大破之嗣昭等引兵退入青山從周追之將扼其歸路步兵自潰嗣昭不能制會横衝都將李嗣源以所部兵至【薛史明宗紀曰羅弘信襲破李存信於萃縣帝奮命殿軍而還武皇嘉其功即以所屬五百騎號曰横衝都】謂嗣昭曰吾輩亦去則勢不可支矣我試為公擊之【為于偽翻】嗣昭曰善我請從公後嗣源乃解鞍厲鏃乘高布陳【陳讀曰陣】左右指畫邢人莫之測嗣源直前奮擊嗣昭繼之從周乃退德威馬邑人也【馬邑秦漢舊縣名久廢開元五年分朔州善陽縣置馬邑縣於古大同軍城屬朔州】 癸卯以威武留後王審知為節度使 以羅紹威知魏博留後丁巳以東川留後王宗滌為節度使 加佑國節度<br />
<br />
  使張全義兼侍中 王珙引汴兵寇河中【珙居勇翻】王珂告急於李克用克用遣李嗣昭救之敗汴兵於胡壁【九域志河中府榮河縣有胡壁鎮榮河唐寶鼎縣也宋祥符中更名敗補邁翻】汴人走前常州刺史王柷【柷之六翻】性剛介有時望詔徵之時人以為且入相【相息亮翻】過陜【王柷就徵道過陜州陜夫冉翻】王珙延奉甚至請叙子姪之禮拜之柷固辭不受珙怒【珙以柷同姓年輩在前行且入相請叙子姪之禮以親結之而柷辭不受珙以柷薄其門地本出寒微而絶之也故怒】使送者殺之 【考異曰柷為給事中并遇害皆無年月今因珙伐河中事附此】并其家人悉投諸河掠其資裝以覆舟聞朝廷不敢詰【史言唐之威令不行藩鎮暴横王柷罹其虐殺而不敢問詰去吉翻】閏月錢鏐以其將曹圭為蘇州制置使遣王球攻婺<br />
<br />
  州 十一月甲寅立皇子禎為雅王祥為瓊王 以魏博留後羅紹威為節度使 衢州刺史陳岌請降于楊行密錢鏐使顧全武討之【降戶江翻】 朱全忠以奉國節度使崔洪與楊行密交通【淮西淮南鄰道也】遣其將張存敬攻之洪懼請以弟都指揮使賢為質【質音致 考異日十國紀年洪託以將士不受節制遣兄賢質於汴按舊紀十月汴將張存敬以兵襲蔡州刺史崔洪納欵請以弟賢質于汴許之實錄亦云弟賢今從之】且言將士頑悍【悍下罕翻又侯旰翻】不受節制請遣二千人詣麾下從征伐全忠許之【為朱全忠遣崔賢徵兵蔡將殺賢刼洪奔淮南張本】召存敬還存敬曹州人也 十二月昭義節度使薛志勤薨李克用之平王行瑜也【見上卷乾寧二年】李罕之求邠寧於克用克用曰行瑜恃功邀君故吾與公討而誅之昨破賊之日吾首奏趣蘇文建赴鎮【事見同上趣讀曰促】今纔達天聽遽復二三【復扶又翻下同】朝野之論必喧然謂吾輩復如行瑜所為也吾與公情如同體固無所愛俟還鎮當更為公論功賞耳【為于偽翻下同】罕之不悦而退私於蓋寓曰罕之自河陽失守依託大庇【蓋古盍翻李罕之失河陽見二百五十七卷僖宗文德元年】歲月已深比來衰老【比毗至翻近也】倦於軍旅若蒙吾王與太傅哀愍【王謂李克用太傳謂蓋寓】賜一小鎮使數年之間休兵養疾然後歸老閭閻幸矣寓為之言克用不應每藩鎮缺議不及罕之罕之甚鬱鬱寓恐其有它志亟為之言【亟去吏翻頻也數也】克用曰吾於罕之豈愛一鎮但罕之鷹也饑則為用飽則背飛【祖曹操駕御呂布之意而言之背蒲妹翻】及志勤薨旬日無帥罕之擅引澤州兵夜入潞州據之【九域志澤州北至潞州一百六十五里帥所類翻】以狀白克用曰薛鐵山死【薛志勤從克用起代北初名鐵山】州民無主慮不逞者為變故罕之專命鎮撫取王裁旨【裁旨者旨裁其可否也】克用怒遣人讓之罕之遂遣其子請降于朱全忠執河東將馬溉等及沁州刺史傅瑤送汴州克用遣李嗣昭將兵討之【自此李克用不能與朱全忠争邢洺磁而争澤潞矣】嗣昭先取澤州收罕之家屬送晉陽【先取澤州既掩李罕之不備且俘其家史言李嗣昭用兵有方略】 楊行密遣成及歸兩浙以易魏約等【淮南擒成及見上卷乾寧三年兩浙擒魏約見上去年四月楊錢争蘇州臺濛周本秦裴皆淮南名將也為浙人所困終不能守楊行密知錢鏐未易可輕故歸成及以易魏約意在講解也】錢鏐許之【錢鏐亦自知不如楊行密之強故許之之速】 韶州刺史曾衮舉兵攻廣州州將王瓌帥戰艦應之【帥讀曰率】清海行軍司馬劉隱一戰破之韶州將劉潼復據湞浛【復扶又翻湞浛當在韶州湞昌縣界或曰劉潼據湞陽浛洭二縣之間湞癡貞翻浛胡南翻】隱討斬之<br />
<br />
  二年春正月丁未中書侍郎兼吏部尚書崔胤罷守本官以兵部尚書陸扆同平章事【崔胤陸扆迭為拜罷】 朱全忠表李罕之為昭義節度使又表權知河陽留後丁會武寧留後王敬蕘【蕘如招翻】彰義留後張珂並為節度使【河陽武寧皆附屬朱全忠獨張珂在涇州而為之請節钺亦所以結之也】 楊行密與朱瑾將兵數萬攻徐州軍于呂梁朱全忠遣騎將張歸厚救之 劉仁恭發幽滄等十二州兵十萬【十二州幽涿瀛莫平營薊媯檀滄景德也幽州巡屬更有蔚新武三州劉仁恭留以備河東不發其兵】欲兼河朔攻貝州拔之城中萬餘戶盡屠之投尸清水【清水即清河之水】由是諸城各堅守不下仁恭進攻魏州營于城北魏博節度使羅紹威求救於朱全忠 朱全忠遣崔賢還蔡州【崔洪以弟賢為質見上年】發其兵二千詣大梁二月蔡將崔景思等殺賢刼崔洪悉驅兵民度淮奔楊行密兵民稍稍遁歸【安土重遷人情之常也】至廣陵者不滿二千人全忠命許州刺史朱友裕守蔡州 朱全忠自將救徐州楊行密聞之引兵去汴人追及之於下邳【下邳古縣唐屬徐州九域志在徐州東一百八十里】殺千餘人全忠行至輝州【是年朱全忠表以宋州之碭山虞城單父曹州之成武置輝州即單州之封域也】聞淮南兵已退乃還 三月朱全忠遣其將李思安張存敬將兵救魏博屯于内黄【九域志内黄縣在魏州西南二百一十四里】癸卯全忠以中軍軍于滑州劉仁恭謂其子守文曰汝勇十倍於思安當先虜鼠輩後擒紹威耳乃遣守文及其妹壻單可及將精兵五萬擊思安於内黄丁未思安使其將袁象先伏兵於清水之右【淇水東過内黄謂之白溝水亦謂之清河水】思安逆戰於繁陽【繁陽漢古縣唐併省入内黄杜佑曰漢繁陽縣故城在内黄縣西北】陽不勝而却守文逐之及内黄之北思安勒兵還戰伏兵發夾擊之幽州兵大敗斬可及殺獲三萬人守文僅以身免可及幽州驍將號單無敵燕軍失之喪氣【李克用輕單可及而有木瓜澗之敗劉仁恭輕李思安而單可及喪元是以用兵者戒於輕敵喪息浪翻】思安陳留人也時葛從周自邢州將精騎八百已入魏州戊申仁恭攻上水關館陶門【館陶門魏州城北門由此門出趣館陶縣因以為門名】從周與宣義牙將賀德倫出戰顧門者曰前有大敵不可返顧命其扉【轄臘翻扉門扇也】從周等殊死戰仁恭復大敗【復扶又翻下同】擒其將薛突厥王鄶郎【鄶古外翻】明日汴魏乘勝合兵擊仁恭破其八寨仁恭父子燒營而遁汴魏之人長驅追之至臨清擁其衆入永濟渠殺溺不可勝紀【勝音升】鎮人亦出兵邀擊於東境【鎮人王鎔之兵深冀趙之東境】自魏至滄五百里間僵尸相枕【僵居良翻枕職任翻】仁恭自是不振而全忠益横矣【幽并之兵勢皆挫故全忠益横横戶孟翻】德倫河西胡人也【薛史賀德倫其先河西部落人父懷慶隸滑州軍為小校故德倫少為滑牙將】劉仁恭之攻魏州也羅紹威遣使修好於河東【好呼到翻】且求救壬午李克用遣李嗣昭將兵救之會仁恭已為汴兵所敗【敗補邁翻】紹威復與河東絶【自李存信莘縣之敗魏與并絶矣今因求救而通好并兵未至而汴人有功故復與并絶】嗣昭引還葛從周乘破幽州之勢自土門攻河東拔承天軍别<br />
<br />
  將氏叔琮自馬嶺入【馬嶺在太原府太谷縣東南八十里】拔遼州樂平【晉分漢沾縣置樂平縣唐屬遼州】進軍榆次【榆次古縣唐屬太原府】李克用遣内牙軍副周德威擊之叔琮有驍將陳章號陳夜义為前鋒【俗言隂府有鬼使日夜叉時人以陳章鷙悍可畏如夜义然因稱之】請於叔琮曰河東所恃者周楊五【周德威小字楊五】請擒之求一州為賞克用聞之以戒德威德威曰彼大言耳戰于洞渦【洞渦水出沾縣北山東流南屈過受陽縣故城東西過榆次縣南此据水經注也魏收地形志洞渦水一出木瓜嶺一出沾嶺一出大廉山一出原過祠下五水合流故曰同過後語轉為洞渦按高歡建大丞相府于晉陽魏收已策名霸府齊受魏禪以晉陽為别都魏收多從其主往來晉陽宫宜知地名之的】德威微服往挑戰【挑徒了翻】謂其屬曰汝見陳夜义即走章果逐之德威奮鐵檛擊之墜馬【檛側瓜翻】生擒以獻因擊叔琮大破之斬首三千級叔琮弃營走德威追之出石會關又斬千餘級從周亦引還 丁巳朱全忠遣河陽節度使丁會攻澤州下之【去年十二月河東兵取澤州 考異曰實錄丁巳葛從周復取澤州按編遺錄丁巳河橋丁會收復澤州實錄云從周誤也唐太祖紀年錄三月周德威敗氏叔琮於洞渦驛先是逆温令丁會將兵助李罕之戍潞州至是葛從周復入潞州以代丁會賊復陷我澤州梁實錄薛史梁紀皆云六月方遣從周入潞州紀年錄于此連言後事耳】婺州刺史王壇為兩浙所圍【去年閏月兩浙兵攻婺州】求救於宣歙觀察使田頵【按宣歙觀察先是已升寧國軍以田頵為節度使歙書涉翻】夏四月頵遣行營都指揮使康儒等救之 五月甲午置武信軍於遂州以遂合等五州隸之【王建之志也】 李克用遣蕃漢馬步都指揮使李君慶將兵攻李罕之己亥圍潞州朱全忠出屯河陽辛丑遣其將張存敬救之壬寅又遣丁會將兵繼之大破河東兵君慶解圍去克用誅君慶及其禆將伊審李弘襲以李嗣昭為蕃漢馬步都指揮使代之攻潞州 庚戌康儒等敗兩浙兵於龍丘【龍丘本漢太末縣貞觀八年更名龍丘即今龍遊縣九域志屬衢州在州東七十五里敗補邁翻】擒其將王球【王球為主將以攻婺州而見擒於龍丘蓋以浙兵逆與宣兵戰也】遂取婺州【景福元年王壇得婺州至是失之】 六月乙丑李罕之疾亟【亟紀力翻】丁卯全忠表罕之為河陽節度使以丁會為昭義節度使未幾【幾居豈翻】又以其將張歸霸守邢州遣葛從周代會守潞州 【考異曰編遺錄六月乙丑李罕之疾甚請歸河陽丁卯上令抽大軍迴以丁會權制置綏懷上黨上乃東歸不言遣從周入潞薛居正五代史梁紀六月帝表丁會為潞州節度使以李罕之疾亟故也又遣葛從周由固鎮路入于潞州以援丁會梁實錄後唐紀皆云代會自此至潞州破賀德倫走不復見會名或者李罕之既卒復召會守河陽以從周代之不可知也今因會鎮潞終言之】 以西川大將王宗佶為武信節度使【王建之請也】宗佶本姓甘洪州人也 丁丑李罕之薨于懷州【李罕之自潞州赴河懷至懷州而卒】 保義節度使王珙性猜忍雖妻子親近常不自保至是軍亂為麾下所殺【僖宗中和初王重盈鎮陜傳子珙至是而亡】推都將李璠為留後【璠音煩】 秋七月朱全忠海州戍將陳漢賓請降于楊行密淮海遊奕使張訓以漢賓心未可知與漣水防遏使廬江王綰將兵二千直趣海州遂據其城【楊行密自此遂有海州趣七喻翻】加荆南節度使成汭兼中書令 馬殷遣其將李唐攻道州蔡結聚羣蠻伏兵于隘以擊之大破唐兵唐曰蠻所恃者山林耳若戰平地安能敗我【敗補邁翻】乃命因風燔林火燭天地羣蠻驚遁遂拔道州擒結斬之 朱全忠召葛從周於潞州使賀德倫守之八月丙寅李嗣昭引兵至潞州城下分兵攻澤州己巳汴將劉玘弃澤州走河東兵進拔天并關以李孝璋為澤州刺史【李孝璋當作李存璋】賀德倫閉城不出李嗣昭日以鐵騎環其城捕芻牧者【環音宦】附城三十里禾黍皆刈之乙酉德倫等弃城宵遁趣壺關【賀德倫之兵既不得出城芻牧城外禾黍又空粮援俱絶宜其遁也九域志壺關縣在潞州東二十五里趣七喻翻】河東將李存審【李存審即苻存審】伏兵邀擊之殺獲甚衆葛從周以援兵至聞德倫等已敗乃還 九月癸卯以鳳翔節度使李茂貞為鳳翔彰義節度使【是年春正月朱全忠表張珂為彰義節度使張氏鎮涇州凡三帥矣今命李茂貞兼領之】 李克用表汾州刺史孟遷為昭義留後【為孟遷以潞州叛李克用張本】 淄青節度使王師範以沂密内叛【當是時未全忠盡有河南一道之地王師範亦附屬焉若沂密内叛將安歸邪又不乞師於全忠而乞師於楊行密此事當考】乞師于楊行密冬十月行密遣海州刺史臺濛副使王綰將兵助之拔密州歸于師範將攻沂州先使覘之【覘丑廉翻】曰城中皆偃旗息鼓綰曰此必有備而救兵近不可擊也諸將曰密已下矣沂何能為綰不能止乃伏兵林中以待之諸將攻沂州不克救兵至引退【此救兵果誰兵歟】州兵乘之綰發伏擊敗之【敗補邁翻】 十一月陜州都將朱簡殺李璠自稱留後附朱全忠仍請更名友謙預於子姪【朱全忠又兼有陜虢更工衡翻】 加忠義節度使趙匡凝兼中書令 馬殷遣其將李瓊攻郴州執陳彦謙斬之進攻連州魯景仁自殺湖南皆平【馬殷始盡有湖南之地】十二月加魏博節度使羅紹威同平章事<br />
<br />
  資治通鑑卷二百六十一<br />
<br />
<史部,編年類,資治通鑑>  <br>
   </div> 

<script src="/search/ajaxskft.js"> </script>
 <div class="clear"></div>
<br>
<br>
 <!-- a.d-->

 <!--
<div class="info_share">
</div> 
-->
 <!--info_share--></div>   <!-- end info_content-->
  </div> <!-- end l-->

<div class="r">   <!--r-->



<div class="sidebar"  style="margin-bottom:2px;">

 
<div class="sidebar_title">工具类大全</div>
<div class="sidebar_info">
<strong><a href="http://www.guoxuedashi.com/lsditu/" target="_blank">历史地图</a></strong>  
<a href="http://www.880114.com/" target="_blank">英语宝典</a>  
<a href="http://www.guoxuedashi.com/13jing/" target="_blank">十三经检索</a> 
<br><strong><a href="http://www.guoxuedashi.com/gjtsjc/" target="_blank">古今图书集成</a></strong> 
<a href="http://www.guoxuedashi.com/duilian/" target="_blank">对联大全</a> <strong><a href="http://www.guoxuedashi.com/xiangxingzi/" target="_blank">象形文字典</a></strong> 

<br><a href="http://www.guoxuedashi.com/zixing/yanbian/">字形演变</a>  <strong><a href="http://www.guoxuemi.com/hafo/" target="_blank">哈佛燕京中文善本特藏</a></strong>
<br><strong><a href="http://www.guoxuedashi.com/csfz/" target="_blank">丛书&方志检索器</a></strong> <a href="http://www.guoxuedashi.com/yqjyy/" target="_blank">一切经音义</a>  

<br><strong><a href="http://www.guoxuedashi.com/jiapu/" target="_blank">家谱族谱查询</a></strong>  <strong><a href="http://shufa.guoxuedashi.com/sfzitie/" target="_blank">书法字帖欣赏</a></strong> 
<br>

</div>
</div>


<div class="sidebar" style="margin-bottom:0px;">

<font style="font-size:22px;line-height:32px">QQ交流群9:489193090</font>


<div class="sidebar_title">手机APP 扫描或点击</div>
<div class="sidebar_info">
<table>
<tr>
	<td width=160><a href="http://m.guoxuedashi.com/app/" target="_blank"><img src="/img/gxds-sj.png" width="140"  border="0" alt="国学大师手机版"></a></td>
	<td>
<a href="http://www.guoxuedashi.com/download/" target="_blank">app软件下载专区</a><br>
<a href="http://www.guoxuedashi.com/download/gxds.php" target="_blank">《国学大师》下载</a><br>
<a href="http://www.guoxuedashi.com/download/kxzd.php" target="_blank">《汉字宝典》下载</a><br>
<a href="http://www.guoxuedashi.com/download/scqbd.php" target="_blank">《诗词曲宝典》下载</a><br>
<a href="http://www.guoxuedashi.com/SiKuQuanShu/skqs.php" target="_blank">《四库全书》下载</a><br>
</td>
</tr>
</table>

</div>
</div>


<div class="sidebar2">
<center>


</center>
</div>

<div class="sidebar"  style="margin-bottom:2px;">
<div class="sidebar_title">网站使用教程</div>
<div class="sidebar_info">
<a href="http://www.guoxuedashi.com/help/gjsearch.php" target="_blank">如何在国学大师网下载古籍?</a><br>
<a href="http://www.guoxuedashi.com/zidian/bujian/bjjc.php" target="_blank">如何使用部件查字法快速查字?</a><br>
<a href="http://www.guoxuedashi.com/search/sjc.php" target="_blank">如何在指定的书籍中全文检索?</a><br>
<a href="http://www.guoxuedashi.com/search/skjc.php" target="_blank">如何找到一句话在《四库全书》哪一页?</a><br>
</div>
</div>


<div class="sidebar">
<div class="sidebar_title">热门书籍</div>
<div class="sidebar_info">
<a href="/so.php?sokey=%E8%B5%84%E6%B2%BB%E9%80%9A%E9%89%B4&kt=1">资治通鉴</a> <a href="/24shi/"><strong>二十四史</strong></a>&nbsp; <a href="/a2694/">野史</a>&nbsp; <a href="/SiKuQuanShu/"><strong>四库全书</strong></a>&nbsp;<a href="http://www.guoxuedashi.com/SiKuQuanShu/fanti/">繁体</a>
<br><a href="/so.php?sokey=%E7%BA%A2%E6%A5%BC%E6%A2%A6&kt=1">红楼梦</a> <a href="/a/1858x/">三国演义</a> <a href="/a/1038k/">水浒传</a> <a href="/a/1046t/">西游记</a> <a href="/a/1914o/">封神演义</a>
<br>
<a href="http://www.guoxuedashi.com/so.php?sokeygx=%E4%B8%87%E6%9C%89%E6%96%87%E5%BA%93&submit=&kt=1">万有文库</a> <a href="/a/780t/">古文观止</a> <a href="/a/1024l/">文心雕龙</a> <a href="/a/1704n/">全唐诗</a> <a href="/a/1705h/">全宋词</a>
<br><a href="http://www.guoxuedashi.com/so.php?sokeygx=%E7%99%BE%E8%A1%B2%E6%9C%AC%E4%BA%8C%E5%8D%81%E5%9B%9B%E5%8F%B2&submit=&kt=1"><strong>百衲本二十四史</strong></a>  <a href="http://www.guoxuedashi.com/so.php?sokeygx=%E5%8F%A4%E4%BB%8A%E5%9B%BE%E4%B9%A6%E9%9B%86%E6%88%90&submit=&kt=1"><strong>古今图书集成</strong></a>
<br>

<a href="http://www.guoxuedashi.com/so.php?sokeygx=%E4%B8%9B%E4%B9%A6%E9%9B%86%E6%88%90&submit=&kt=1">丛书集成</a> 
<a href="http://www.guoxuedashi.com/so.php?sokeygx=%E5%9B%9B%E9%83%A8%E4%B8%9B%E5%88%8A&submit=&kt=1"><strong>四部丛刊</strong></a>  
<a href="http://www.guoxuedashi.com/so.php?sokeygx=%E8%AF%B4%E6%96%87%E8%A7%A3%E5%AD%97&submit=&kt=1">說文解字</a> <a href="http://www.guoxuedashi.com/so.php?sokeygx=%E5%85%A8%E4%B8%8A%E5%8F%A4&submit=&kt=1">三国六朝文</a>
<br><a href="http://www.guoxuedashi.com/so.php?sokeytm=%E6%97%A5%E6%9C%AC%E5%86%85%E9%98%81%E6%96%87%E5%BA%93&submit=&kt=1"><strong>日本内阁文库</strong></a> <a href="http://www.guoxuedashi.com/so.php?sokeytm=%E5%9B%BD%E5%9B%BE%E6%96%B9%E5%BF%97%E5%90%88%E9%9B%86&ka=100&submit=">国图方志合集</a> <a href="http://www.guoxuedashi.com/so.php?sokeytm=%E5%90%84%E5%9C%B0%E6%96%B9%E5%BF%97&submit=&kt=1"><strong>各地方志</strong></a>

</div>
</div>


<div class="sidebar2">
<center>

</center>
</div>
<div class="sidebar greenbar">
<div class="sidebar_title green">四库全书</div>
<div class="sidebar_info">

《四库全书》是中国古代最大的丛书,编撰于乾隆年间,由纪昀等360多位高官、学者编撰,3800多人抄写,费时十三年编成。丛书分经、史、子、集四部,故名四库。共有3500多种书,7.9万卷,3.6万册,约8亿字,基本上囊括了古代所有图书,故称“全书”。<a href="http://www.guoxuedashi.com/SiKuQuanShu/">详细>>
</a>

</div> 
</div>

</div>  <!--end r-->

</div>
<!-- 内容区END --> 

<!-- 页脚开始 -->
<div class="shh">

</div>

<div class="w1180" style="margin-top:8px;">
<center><script src="http://www.guoxuedashi.com/img/plus.php?id=3"></script></center>
</div>
<div class="w1180 foot">
<a href="/b/thanks.php">特别致谢</a> | <a href="javascript:window.external.AddFavorite(document.location.href,document.title);">收藏本站</a> | <a href="#">欢迎投稿</a> | <a href="http://www.guoxuedashi.com/forum/">意见建议</a> | <a href="http://www.guoxuemi.com/">国学迷</a> | <a href="http://www.shuowen.net/">说文网</a><script language="javascript" type="text/javascript" src="https://js.users.51.la/17753172.js"></script><br />
  Copyright &copy; 国学大师 古典图书集成 All Rights Reserved.<br>
  
  <span style="font-size:14px">免责声明:本站非营利性站点,以方便网友为主,仅供学习研究。<br>内容由热心网友提供和网上收集,不保留版权。若侵犯了您的权益,来信即刪。scp168@qq.com</span>
  <br />
ICP证:<a href="http://www.beian.miit.gov.cn/" target="_blank">鲁ICP备19060063号</a></div>
<!-- 页脚END --> 
<script src="http://www.guoxuedashi.com/img/plus.php?id=22"></script>
<script src="http://www.guoxuedashi.com/img/tongji.js"></script>

</body>
</html>
