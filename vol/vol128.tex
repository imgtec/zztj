<!DOCTYPE html PUBLIC "-//W3C//DTD XHTML 1.0 Transitional//EN" "http://www.w3.org/TR/xhtml1/DTD/xhtml1-transitional.dtd">
<html xmlns="http://www.w3.org/1999/xhtml">
<head>
<meta http-equiv="Content-Type" content="text/html; charset=utf-8" />
<meta http-equiv="X-UA-Compatible" content="IE=Edge,chrome=1">
<title>資治通鑒_129-資治通鑑卷一百二十八_129-資治通鑑卷一百二十八</title>
<meta name="Keywords" content="資治通鑒_129-資治通鑑卷一百二十八_129-資治通鑑卷一百二十八">
<meta name="Description" content="資治通鑒_129-資治通鑑卷一百二十八_129-資治通鑑卷一百二十八">
<meta http-equiv="Cache-Control" content="no-transform" />
<meta http-equiv="Cache-Control" content="no-siteapp" />
<link href="/img/style.css" rel="stylesheet" type="text/css" />
<script src="/img/m.js?2020"></script> 
</head>
<body>
 <div class="ClassNavi">
<a  href="/24shi/">二十四史</a> | <a href="/SiKuQuanShu/">四库全书</a> | <a href="http://www.guoxuedashi.com/gjtsjc/"><font  color="#FF0000">古今图书集成</font></a> | <a href="/renwu/">历史人物</a> | <a href="/ShuoWenJieZi/"><font  color="#FF0000">说文解字</a></font> | <a href="/chengyu/">成语词典</a> | <a  target="_blank"  href="http://www.guoxuedashi.com/jgwhj/"><font  color="#FF0000">甲骨文合集</font></a> | <a href="/yzjwjc/"><font  color="#FF0000">殷周金文集成</font></a> | <a href="/xiangxingzi/"><font color="#0000FF">象形字典</font></a> | <a href="/13jing/"><font  color="#FF0000">十三经索引</font></a> | <a href="/zixing/"><font  color="#FF0000">字体转换器</font></a> | <a href="/zidian/xz/"><font color="#0000FF">篆书识别</font></a> | <a href="/jinfanyi/">近义反义词</a> | <a href="/duilian/">对联大全</a> | <a href="/jiapu/"><font  color="#0000FF">家谱族谱查询</font></a> | <a href="http://www.guoxuemi.com/hafo/" target="_blank" ><font color="#FF0000">哈佛古籍</font></a> 
</div>

 <!-- 头部导航开始 -->
<div class="w1180 head clearfix">
  <div class="head_logo l"><a title="国学大师官网" href="http://www.guoxuedashi.com" target="_blank"></a></div>
  <div class="head_sr l">
  <div id="head1">
  
  <a href="http://www.guoxuedashi.com/zidian/bujian/" target="_blank" ><img src="http://www.guoxuedashi.com/img/top1.gif" width="88" height="60" border="0" title="部件查字,支持20万汉字"></a>


<a href="http://www.guoxuedashi.com/help/yingpan.php" target="_blank"><img src="http://www.guoxuedashi.com/img/top230.gif" width="600" height="62" border="0" ></a>


  </div>
  <div id="head3"><a href="javascript:" onClick="javascript:window.external.AddFavorite(window.location.href,document.title);">添加收藏</a>
  <br><a href="/help/setie.php">搜索引擎</a>
  <br><a href="/help/zanzhu.php">赞助本站</a></div>
  <div id="head2">
 <a href="http://www.guoxuemi.com/" target="_blank"><img src="http://www.guoxuedashi.com/img/guoxuemi.gif" width="95" height="62" border="0" style="margin-left:2px;" title="国学迷"></a>
  

  </div>
</div>
  <div class="clear"></div>
  <div class="head_nav">
  <p><a href="/">首页</a> | <a href="/ShuKu/">国学书库</a> | <a href="/guji/">影印古籍</a> | <a href="/shici/">诗词宝典</a> | <a   href="/SiKuQuanShu/gxjx.php">精选</a> <b>|</b> <a href="/zidian/">汉语字典</a> | <a href="/hydcd/">汉语词典</a> | <a href="http://www.guoxuedashi.com/zidian/bujian/"><font  color="#CC0066">部件查字</font></a> | <a href="http://www.sfds.cn/"><font  color="#CC0066">书法大师</font></a> | <a href="/jgwhj/">甲骨文</a> <b>|</b> <a href="/b/4/"><font  color="#CC0066">解密</font></a> | <a href="/renwu/">历史人物</a> | <a href="/diangu/">历史典故</a> | <a href="/xingshi/">姓氏</a> | <a href="/minzu/">民族</a> <b>|</b> <a href="/mz/"><font  color="#CC0066">世界名著</font></a> | <a href="/download/">软件下载</a>
</p>
<p><a href="/b/"><font  color="#CC0066">历史</font></a> | <a href="http://skqs.guoxuedashi.com/" target="_blank">四库全书</a> |  <a href="http://www.guoxuedashi.com/search/" target="_blank"><font  color="#CC0066">全文检索</font></a> | <a href="http://www.guoxuedashi.com/shumu/">古籍书目</a> | <a   href="/24shi/">正史</a> <b>|</b> <a href="/chengyu/">成语词典</a> | <a href="/kangxi/" title="康熙字典">康熙字典</a> | <a href="/ShuoWenJieZi/">说文解字</a> | <a href="/zixing/yanbian/">字形演变</a> | <a href="/yzjwjc/">金 文</a> <b>|</b>  <a href="/shijian/nian-hao/">年号</a> | <a href="/diming/">历史地名</a> | <a href="/shijian/">历史事件</a> | <a href="/guanzhi/">官职</a> | <a href="/lishi/">知识</a> <b>|</b> <a href="/zhongyi/">中医中药</a> | <a href="http://www.guoxuedashi.com/forum/">留言反馈</a>
</p>
  </div>
</div>
<!-- 头部导航END --> 
<!-- 内容区开始 --> 
<div class="w1180 clearfix">
  <div class="info l">
   
<div class="clearfix" style="background:#f5faff;">
<script src='http://www.guoxuedashi.com/img/headersou.js'></script>

</div>
  <div class="info_tree"><a href="http://www.guoxuedashi.com">首页</a> > <a href="/SiKuQuanShu/fanti/">四库全书</a>
 > <h1>资治通鉴</h1> <!--         下载:【右键另存为】即可 --></div>
  <div class="info_content zj clearfix">
  
<div class="info_txt clearfix" id="show">
<center style="font-size:24px;">129-資治通鑑卷一百二十八</center>
    資治通鑑卷一百二十八 宋 司馬光 撰<br />
<br />
  胡三省 音註<br />
<br />
  宋紀十【起閼逢敦牂盡著雍閹茂凡五年】<br />
<br />
  世祖孝武皇帝上【諱駿字休龍小字道民文帝第三子也】<br />
<br />
  孝建元年春正月己亥朔上祀南郊改元大赦【上既平元兇之亂依故事即位踰年而後改元孝建者蓋欲以孝建平禍亂安宗廟之功】 甲辰以尚書令何尚之為左光禄大夫護軍將軍以左衛將軍顔竣為吏部尚書領驍騎將軍【竣七倫翻驍堅堯翻騎奇寄翻】 壬戌更鑄孝建四銖錢【更工衡翻】 乙丑魏以侍中伊馛為司空【馛蒲撥翻】丙子立皇子子業為太子 初江州刺史臧質自謂人才足為一世英雄太子劭之亂質潛有異圖以荆州刺史南郡王義宣庸闇易制【易以豉翻】欲外相推奉因而覆之質於義宣為内兄【臧質武敬皇后之姪年長於義宣故為内兄】既至江陵【質初起兵與魯爽同詣江陵事見上卷上年】即稱名拜義宣義宣驚愕問故質曰事中宜然【謂國家多事之中宜相推奉也】時義宣已奉帝為主故其計不行及至新亭【去年五月朔質至新亭】又拜江夏王義恭【夏戶雅翻】曰天下屯危禮異常日【屯陟倫翻】劭既誅義宣與質功皆第一由是驕恣事多專行凡所求欲無不必從【必上之從已】義宣在荆州十年【文帝元嘉二十年義宣鎮荆州】財富兵彊朝廷所下制度意有不同一不遵承【史歷言義宣質驕横之由下遐稼翻】質自建康之江州舫千餘乘部伍前後百餘里【舫甫妄翻乘繩證翻】帝方自攬威權而質以少主遇之【少詩沼翻】政刑慶賞一不咨禀擅用湓口鉤圻米【湓口米荆湘郢三州之運所積也鉤圻米南江之運所積也水經注灨水自南昌歷郴丘城下又歷鉤圻邸閣下而後至彭澤圻音畿】臺符屢加檢詰漸致猜懼【檢詰謂檢校米斛而詰問擅用之由也詰去吉翻】帝淫義宣諸女義宣由是恨怒質乃遣密信說義宣【密信密吏也說輸芮翻下說誘同】以為負不賞之功挾震主之威自古能全者有幾今萬物係心於公聲迹已著見幾不作將為它人所先【幾居希翻先悉薦翻】若命徐遺寶魯爽驅西北精兵來屯江上【徐遺寶刺兖州直建康北魯爽刺南豫直建康西魯爽素奉義宣徐遺寶由義宣府參軍起故欲命之同逆】質帥九江樓船為公前驅【帥讀曰率】已為得天下之半公以八州之衆徐進而臨之【義宣都督荆雍梁益湘交廣寧八州】雖韓白更生不能為建康計矣【韓白謂韓信白起】且少主失德聞于道路【聞音問同】沈柳諸將亦我之故人【沈慶之與質同以武幹事文帝質為雍州柳元景其部曲將也將即亮翻下同】誰肯為少主盡力者【為于偽翻下為公同】夫不可留者年也不可失者時也質常恐溘先朝露【溘苦荅翻又苦合翻溘奄也朝露言其易晞溘先朝露言奄然而死在朝露未晞之先先悉荐翻】不得展其旅力【毛萇曰旅衆也考孔安國書注亦然】為公掃除於時悔之何及義宣腹心將佐諮議參軍蔡超司馬竺超民等咸有富貴之望【蔡超等以江州將佐從帝起義以得富貴故懷非望】欲倚質威名以成其業共勸義宣從其計質女為義宣子採之婦義宣謂質無復異同【復扶又翻】遂許之超民夔之子也【景平元嘉間竺夔守東陽有功】臧敦時為黄門侍郎帝使敦至義宣所道經尋陽質更令敦說誘義宣【誘音酉】義宣意遂定豫州刺史魯爽有勇力義宣素與之相結義宣密使人報爽及兖州刺史徐遺寶期以今秋同舉兵使者至壽陽爽方飲醉失義宣指即日舉兵 【考異曰宋本紀二月庚午爽臧質南郡王義宣徐遺寶舉兵反義宣傳云其年正月便反宋畧云二月義宣等反按爽之反帝猶遣質收魯弘則非同日反明矣又按長歷是月戊辰朔然則庚午三日也義宣傳起兵在二月二十六日但不知爽反在正月與二月耳】爽弟瑜在建康聞之逃叛爽使其衆戴黄標【戴黄以為標識】竊造法服登壇自號建平元年疑長史韋處穆中兵參軍楊元駒治中庾騰之不與已同皆殺之【處昌呂翻】徐遺寶亦勒兵向彭城二月義宣聞爽已反狼狽舉兵魯瑜弟弘為質府佐帝敕質收之質即執臺使舉兵【使疏吏翻】義宣與質皆上表言為左右所讒疾欲誅君側之惡義宣進爽號征北將軍爽於是送所造輿服詣江陵使征北府戶曹板義宣等【晉宋之制藩方權宜授官者謂之板授】文曰丞相劉今補天子名義宣車騎臧今補丞相名質平西朱今補車騎名修之【先是臧質進號車騎將軍鎮尋陽朱修之進號平西將軍鎮襄陽進義宣丞相辭不受】皆板到奉行義宣駭愕爽所送法物並留竟陵不聽進質加魯弘輔國將軍下戍大雷義宣遣諮議參軍劉諶之將萬人就弘【諶氏壬翻將即亮翻】召司州刺史魯秀欲使為諶之後繼秀至江陵見義宣出拊膺曰吾兄誤我乃與癡人作賊今年敗矣義宣兼荆江兖豫四州之力威震遠近帝欲奉乘輿法物迎之【乘繩證翻】竟陵王誕固執不可曰奈何持此座與人乃止【竟陵王誕時為揚州刺史】 己卯以領軍將軍柳元景為撫軍將軍辛卯以左衛將軍王玄謨為豫州刺史【欲以代魯爽】命元景統玄謨等諸將以討義宣癸巳進據梁山洲【時梁山江中有洲玄謨等舟師據之】於兩㟁築偃月壘水陸待之義宣自稱都督中外諸軍事命僚佐悉稱名 甲午魏主詣道壇受圖籙【寇謙之之遺教也】 丙申以安北司馬夏侯祖歡為兖州刺史【代徐遣寶】三月己亥内外戒嚴 【考異曰宋畧宋本紀皆作癸亥下有辛丑按長歷是月戊戌朔癸亥二十六日辛丑乃四日也當作己亥】辛丑以徐州刺史蕭思話為江州刺史【欲以代臧質】柳元景為雍州刺史【欲以代朱修之雍於用翻】癸卯以太子左衛率龎秀之為徐州刺史【欲以代蕭思話】義宣移檄州郡加進位號使同發兵雍州刺史朱修之偽許之而遣使陳誠於帝【遣使疏吏翻下同】益州刺史劉秀之斬義宣使者遣中兵參軍韋崧將萬人襲江陵【將即亮翻下使將同】戊申義宣帥衆十萬江津舳艫數百里【帥讀曰率舳音逐艫音盧】以子慆為輔國將軍與左司馬竺超民留鎮江陵【慆土刀翻】檄朱修之使兵萬人繼進修之不從義宣知修之貳於己乃以魯秀為雍州刺史使將萬餘人擊之王玄謨聞秀不來【魯秀善戰故王玄謨憚之】喜曰臧質易與耳【易以豉翻】冀州刺史垣護之妻徐遺寶之姊也遺寶邀護之同反護之不從發兵擊之遺寶遣兵襲徐州長史明胤於彭城不克【蕭思話已離彭城長史明胤守之】胤與夏侯祖歡垣護之共擊遺寶於湖陸【宋兖州治湖陸】遺寶棄衆焚城奔魯爽義宣至尋陽以質為前鋒而進爽亦引兵直趣歷陽【趣七喻翻】與質水陸俱下殿中將軍沈靈賜將百舸破質前軍於南陵擒軍主徐慶安等【舸古我翻】質至梁山夾陳兩岸與官軍相拒【陳讀曰陣】夏四月戊辰以後將軍劉義綦為湘州刺史甲申以朱修之為荆州刺史【義宣為荆湘二州刺史而反故二州皆命代以朱修之效順使制其後故命以荆州】上遣左軍將軍薛安都龍驤將軍南陽宗越等戍歷陽【驤思將翻】與魯爽前鋒楊胡興等戰斬之 【考異曰安都傳作胡與今從宗越傳】爽不能進留軍大峴使魯瑜屯小峴【小峴在合肥之東大峴又在小峴之東峴戶典翻】上復遣鎮軍將軍沈慶之濟江督諸將討爽【復扶又翻】爽食少引兵稍退自留斷後【少詩沼翻斷音短斷後古之所謂殿也】慶之使薛安都帥輕騎追之【帥讀曰率騎奇寄翻】丙戌及爽於小峴爽將戰飲酒過醉安都望見爽即躍馬大呼直往刺之【呼火故翻刺七亦翻】應手而倒左右范雙斬其首爽衆奔散瑜亦為部下所殺遂進攻壽陽克之【爽為南豫州刺史鎮壽陽】徐遺寶奔東海東海人殺之<br />
<br />
  李延壽論曰凶人之濟其身非世亂莫由焉魯爽以亂世之情而行之於平日【平日謂安平無事之日】其取敗也宜哉【考異曰此語本出沈約宋書吳喜黄回傳贊而延壽取之以爽施用失所故絀其名】<br />
<br />
  南郡王義宣至鵲頭慶之送爽首示之并與書曰僕荷任一方而舋生所統【去年慶之鎮盱眙今使之專征蓋兼督兖豫荷下可翻舋許覲翻】近聊帥輕師指往翦撲【輕師言非重兵撲普卜翻】軍鋒裁及賊爽授首公情契異常【言義宣與爽相結情契異于常人】或欲相見及其可識指送相呈爽累世將家【魯爽父軌軌父宗之三世將家】驍猛善戰號萬人敵【驍堅堯翻】義宣與質聞其死皆駭懼柳元景軍于采石王玄謨以臧質衆盛遣使來求益兵【使疏吏翻】上使元景進屯姑孰 【考異曰垣護之傳作南州蓋南州即姑孰也愚按宋白續通典曰桓玄居南州以在國南故曰南州載之宜州之下晉書云桓玄於南州起齋號曰盤龍齋劉毅小字盤龍玄既敗毅以豫州刺史出鎮姑孰正居是齋桓玄既誅司馬元顯出鎮姑孰起盤龍齋蓋是時也晉書正指姑孰為南州宋白誤矣】太傅義恭與義宣書曰往時仲堪假兵靈寶尋害其族孝伯推誠牢之旋踵而敗【假兵推誠事並見一百一十卷晉安帝隆安二年桓玄殺殷仲堪見一百一十一卷三年桓玄字靈寶王恭字孝伯】臧質少無美行弟所具悉【質少輕薄無檢為文帝所嫌少詩沼翻行下孟翻】今藉西楚之彊力圖濟其私凶謀若果【果勝也克也决也】恐非復池中物也【復扶又翻】義宣由此疑之五月甲辰義宣至蕪湖質進計曰今以萬人取南州則梁山中絶【柳元景屯南州為梁山後鎮若取之則梁山之路中絶】萬人綴梁山則玄謨必不敢動下官中流鼓棹直趣石頭此上策也【沈慶之薛安都等在江西柳元景王玄謨等與義宣相持若質計得行建康殆矣趣七喻翻】義宣將從之劉諶之密言于義宣曰質求前驅此志難測不如盡鋭攻梁山事克然後長驅此萬安之計也義宣乃止冗從僕射胡子反等守梁山西壘會西南風急質遣其將尹周之攻西壘【因西南風急而攻西壘東壘之兵難以逆風赴救冗而隴翻從才用翻將即亮翻】子反方度東岸就玄謨計事聞之馳歸偏將劉季之帥水軍殊死戰【帥讀曰率】求救於玄謨玄謨不遣大司馬參軍崔勲之固爭乃遣勲之與積弩將軍垣詢之救之比至城已陷勲之詢之皆戰死【比必寐翻及也考異曰義宣傳曰五月十九日西南風猛宋畧曰己亥質遣尹周之攻梁山西壘陷之按長歷是月丁酉朔三日己亥八日甲辰十八日甲寅宋畧於己亥上有甲辰下有甲寅然則决非十九日與己亥或者是己酉與辛亥也今宋書日闕疑】詢之護之之弟也子反等奔還東岸質又遣其將龎法起將數千兵趨南浦欲自後掩玄謨【龎皮江翻趨七喻翻時玄謨使其將鄭琨武念戍南浦其地則令之大信港也俗謂之扁擔河】游擊將軍垣護之引水軍與戰破之【此以上皆梁山交戰事】朱修之斷馬鞍山道【水經注檀溪水出襄陽西柳子山下東為鴨湖湖在馬鞍山東北按馬鞍山今謂之望楚山晉劉弘所改名也高處有三磴斷丁管翻】據險自守魯秀攻之不克屢為修之所敗【敗補邁翻】乃還江陵修之引兵躡之或勸修之急追修之曰魯秀驍將也獸窮則攫不可迫也【兵法有云知彼知已百戰不殆朱修之此戰近之驍堅堯翻將即亮翻】王玄謨使垣護之告急於柳元景曰西城不守唯餘東城萬人賊軍數倍彊弱不敵欲退還姑孰就節下協力當之更議進取元景不許曰賊勢方盛不可先退吾當卷甲赴之護之曰賊謂南州有三萬人而將軍麾下裁十分之一若往造賊壘【造士到翻】則虛實露矣王豫州必不可來不如分兵援之元景曰善乃留羸弱自守悉遣精兵助玄謨多張旗幟【羸倫為翻幟昌志翻】梁山望之如數萬人皆以為建康兵悉至衆心乃安質自請攻東城諮議參軍顔樂之說義宣曰質若復克東城【樂音洛說輸芮翻復扶又翻】則大功盡歸之矣宜遣麾下自行義宣乃遣劉諶之與質俱進甲寅義宣至梁山頓兵西岸【義宣自鵲頭至梁山西岸】質與劉諶之進攻東城玄謨督諸軍大戰薛安都帥突騎先衝其陳之東南䧟之【帥讀曰率騎奇寄翻陳讀曰陣】斬諶之首劉季之宗越又䧟其西北質等兵大敗垣護之燒江中舟艦烟焰覆水【艦戶黯翻覆敷又翻下覆頭同】延及西岸營壘殆盡諸軍乘勢攻之義宣兵亦潰義宣單舸迸走閉戶而泣【舸古我翻迸北孟翻戶艦戶也】荆州人随之者猶百餘舸質欲見義宣計事而義宣已去質不知所為亦走其衆皆降散【降戶江翻】己未解嚴 癸亥以吳興太守劉延孫為尚書右僕射【守手又翻】 六月丙寅魏主如隂山 臧質至尋陽焚燒府舍載妓妾西走使嬖人何文敬領餘兵居前至西陽西陽太守魯方平紿文敬曰【妓渠綺翻嬖卑義翻又博計翻紿蕩亥翻】詔書唯捕元惡餘無所問不如逃之文敬棄衆亡去質先以妹夫羊冲為武昌郡【晉起居注武帝太康元年改江夏為武昌郡又按晉志吳主權以東鄂置武昌郡今夀昌軍是也】質往投之冲已為郡丞胡庇之所殺質無所歸乃逃于南湖【南湖今在壽昌軍武昌縣東八里】掇蓮實噉之【掇丁括翻噉徒濫翻又徒覧翻】追兵至以荷覆頭自沈於水出其鼻【沈持林翻】戊辰軍主鄭俱兒望見射之中心【射而亦翻中竹仲翻】兵刃亂至腸胃縈水草斬首送建康子孫皆棄市并誅其黨樂安太守任薈之【任音壬薈烏外翻】臨川内史劉懷之鄱陽太守杜仲儒仲儒驥之兄子也【杜驥元嘉中刺青州】功臣柳元景等封賞各有差丞相義宣走至江夏聞巴陵有軍【巴陵之軍蓋韋崧之兵也或曰湘州刺史劉遵考之兵也夏戶雅翻】囘向江陵衆散且盡與左右十許人徒步脚痛不能前僦民露車自載【僦即就翻賃也】緣道求食至江陵郭外遣人報竺超民超民具羽儀兵衆迎之時荆州帶甲尚萬餘人左右翟靈寶誡義宣使撫慰將佐【翟萇伯翻將即亮翻】以臧質違指授之宜用致失利今治兵繕甲更為後圖【治直之翻】昔漢高百敗終成大業而義宣忘靈寶之言誤云項羽千敗衆咸掩口【掩口而笑也】魯秀竺超民等猶欲收餘兵更圖一决而義宣惽沮無復神守入内不復出【沮在呂翻復扶又翻下夜復冋】左右腹心稍稍離叛魯秀北走 【考異曰宋畧云秀自襄陽敗退將及江陵聞敗北走今從宋書】義宣不能自立欲從秀去乃息慆【息子也】及所愛妾五人著男子服相随【著陟畧翻】城内擾亂白刃交横義宣懼墜馬遂步進竺超民送至城外更以馬與之歸而守城【守手又翻】義宣求秀不得左右盡棄之夜復還南郡空廨【南郡太守廨舍蓋在江陵城外廨古隘翻】旦日超民收送刺姧【自漢以來公府有刺姧掾】義宣止獄戶坐地歎曰臧質老奴誤我五妾尋被遣出義宣號泣語獄吏曰常日非苦今日分别始是苦【被皮義翻號戶高翻語牛倨翻别如字分别猶分離也】魯秀衆散不能去還向江陵城上人射之【射而亦翻】秀赴水死就取其首詔右僕射劉孝孫使荆江二州旌别枉直【使疏吏翻别彼列翻】就行誅賞且分割二州之地議更置新州【由是遂分荆湘江豫之地置郢州】初晉氏南遷以揚州為京畿穀帛所資皆出焉以荆江為重鎮甲兵所聚盡在焉常使大將居之【將即亮翻】三州戶口居江南之半上惡其彊大故欲分之【惡烏路翻下同】癸未分揚州浙東五郡置東揚州治會稽【五郡會稽東陽永嘉臨海新安會工外翻】分荆湘江豫州之八郡置郢州治江夏【分荆州之江夏竟陵随武陵天門湘州巴陵江州武昌豫州西陽凡八郡永初郡國志及何承天志江夏太守本治安陸自此之後徙治夏口今鄂州治江夏縣即其地夏戶雅翻下夏口同】罷南蠻校尉遷其營於建康【晉武帝置護南蠻校尉於襄陽江左初省尋又置於江陵水經注南蠻校尉府在方城自油口以東屯營相接悉是南蠻府屯兵校戶教翻】太傅義恭議使郢州治巴陵尚書令何尚之曰夏口在荆江之中正對沔口通接雍梁寔為津要【自夏口入沔泝流而上至襄陽又泝流而上至漢中故云通接雍梁雍於用翻】由來舊鎮根基不易【夏口自吳以來為重鎮】既有見城【見賢遍翻】浦大容舫於事為便【守江之備船艦為急故以浦大容舫為便舫甫妄翻】上從之既而荆揚因此虚耗尚之請復合二州【復扶人翻】上不許 戊子省録尚書事上惡宗室彊盛不欲權在臣下太傅義恭知其指故請省之 上使王公八座與荆州刺史朱修之書【晉志曰五曹尚書一僕射二令為八座宋蓋二僕射一令】令丞相義宣自為計書未達庚寅修之入江陵殺義宣并誅其子十六人及同黨竺超民從事中郎蔡超諮議參軍顔樂之等超民兄弟應從誅何尚之上言賊既遁走一夫可擒若超民反覆昧利即當取之非唯免愆亦可要不義之賞而超民曾無此意微足觀過知仁【何尚之此言為竺超民兄弟道地耳要一遥翻】且為官保全城府謹守庫藏【為于偽翻藏徂浪翻】端坐待縛今戮及兄弟則與其餘逆黨無異於事為重上乃原之 秋七月丙申朔日有食之 庚子魏皇子弘生辛丑大赦改元興光 丙辰大赦 八月甲戌魏趙王深卒 乙亥魏主還平城【是年夏書魏主如隂山】冬十一月戊戍魏主如中山遂如信都十二月丙子<br />
<br />
  還幸靈丘【靈丘縣自漢以來屬代郡唐為蔚州】至温泉宫庚辰還平城二年春正月魏車騎大將軍樂平王拔有辠賜死【騎奇寄翻】鎮北大將軍南兖州刺史沈慶之請老二月丙寅以<br />
<br />
  為左光禄大夫開府儀同三司慶之固讓表疏數十上又面自陳乃至稽顙泣涕【上時掌翻稽音啟】上不能奪聽以始興公就第厚加給奉頃之上復欲用慶之【復扶又翻下同】使何尚之往起之尚之累陳上意慶之笑曰沈公不效何公往而復返【尚之不能固志見一百二十六卷文帝元嘉二十八年】尚之慙而止辛巳以尚書右僕射劉延孫為南兖州刺史 夏五月戊戌以湘州刺史劉遵考為尚書右僕射 六月壬戌魏改元太安 甲子大赦 甲申魏主還平城【史亦不書所如之地】秋七月癸巳立皇弟休祐為山陽王休茂為海陵王<br />
<br />
  休業為鄱陽王 丙辰魏主如河西 雍州刺史武昌王渾【朱修之已赴江陵柳元景又留建康以渾刺雍州雍於用翻】與左右作檄文自號楚王改元永光備置百官以為戱笑長史王翼之封呈其手迹八月庚申廢渾為庶人徙始安郡上遣員外散騎侍郎東海戴明寶詰責渾【散悉亶翻騎奇寄翻詰去吉翻】因逼令自殺時年十七 丁亥魏主還平城 詔祀郊廟初設備樂從前殿中曹郎荀萬秋之議也【晉氏南渡草創二郊無樂宗廟雖有登歌亦無二舞及破苻堅得樂工始有金石之樂文帝元嘉二十二年南郊始設登歌此所謂備樂非能備雅樂魏晉以來世俗之樂耳順帝昇明二年王僧䖍所謂朝廷禮樂多違舊典蓋指此類】 上欲削弱王侯冬十月己未江夏王義恭竟陵王誕奏裁王侯車服器用樂舞制度凡九事上因諷有司奏增廣為二十四條【聽事不得南面坐施帳并幡蕃國官正冬不得徒跣登國殿及夾侍國師傳令及油戟公主妃傳令不得朱服輿不得重掆鄣扇不得雉尾槊毦不得孔雀白氅夾轂隊不得絳襖平乘但馬不得過二匹胡伎不得綵衣舞妓正冬著袿衣不得莊面蔽花正冬會不得劍舞杯柈舞長蹻伎䞬舒丸劍博山伎緣大橦伎五案伎自非正冬會奏舞曲不得舞諸妃主不得著緄帶信幡非臺省官悉用絳郡縣内史相及封内官長於其封君既非在三罷官則不復追敬不合稱臣止宜上下官敬而已諸鎮常行車前後不得過六隊白直夾轂不在其限刀不得過銀銅飾諸王女封縣主諸王子孫襲封王之妃及封侯者夫人行並不得鹵簿諸王子繼體為王婚葬吉凶悉依諸國公侯之禮不得同皇弟皇子車輿非軺車不得油幢平乘船皆平兩頭作露平形不得擬象龍舟悉不得朱油帳錯不得作五花及竪筍形】聽事不得南向坐劍不得為鹿盧形【晉灼曰古長劍首以玉作井鹿盧形】内史相及封内官長止稱下官不得稱臣罷官則不復追敬【長知兩翻復扶又翻】詔可 庚午魏以遼西王常英為太宰 壬午以太傅義恭領揚州刺史竟陵王誕為司空領南徐州刺史建平王宏為尚書令 是歲以故氐王楊保宗子元和為征虜將軍楊頭為輔國將軍頭文德之從祖兄也元和雖楊氏正統【從才用翻楊保宗氐王楊玄之子故元和為楊氏正統】朝廷以其年幼才弱未正位號部落無定主頭先戍葭蘆母妻子弟並為魏所執【文帝元嘉二十年魏克仇池楊文德敗走頭母妻子弟為魏所執當在是年二十七年始使頭戍葭蘆】而頭為宋堅守無貳心【為于偽翻】雍州刺史王玄謨上言【雍於用翻上時掌翻】請以頭為假節西秦州刺史用安輯其衆俟數年之後元和稍長使嗣故業若元和才用不稱【長知兩翻稱尺證翻】便應歸頭頭能藩扞漢川使無虜患彼四千戶荒州殆不足惜若葭蘆不守漢川亦無立理上不從<br />
<br />
  三年春正月庚寅立皇弟休範為順陽王休若為巴陵王戊戌立皇子子尚為西陽王 壬子納右衛將軍何瑀女為太子妃瑀澄之曾孫也甲寅大赦 乙卯魏立貴人馮氏為皇后后遼西郡公朗之女也【馮朗降魏見一百二十二卷文帝元嘉九年】朗為秦雍二州刺史【雍於用翻】坐事誅后由是没入宫【為馮后專魏政張本】 二月丁巳魏主立子弘為皇太子先使其母李貴人條記所付託兄弟然後依故事賜死甲子以廣州刺史宗慤為豫州刺史故事府州部内論事皆籖前直叙所論之事置典籖以主之宋世諸皇子為方鎮者多幼時主皆以親近左右領典籖【近其靳翻】典籖之權稍重至是雖長王臨藩【長知兩翻】素族出鎮典籖皆出納教命執其樞要刺史不得專其職任及慤為豫州臨安吳喜為典籖【吳分餘杭為臨水縣晉武帝太康元年更名臨安屬吳興郡】慤刑政所施喜每多違執慤大怒曰宗慤年將六十為國竭命【為于偽翻】正得一州如斗大【正一作止】不能復與典籖共臨之喜稽顙流血乃止【復扶又翻稽音啟】 丁零數千家匿井陘山中為盜【陘音刑】魏選部尚書陸真【初學記漢成帝置列曹尚書四人其一曰常侍曹光武改常侍曹曰吏部主選舉靈帝改吏部為選部後魏初有殿中樂部駕部南部北部五尚書選部尚書蓋此時方置】與州郡合兵討滅之 閏月戊午以尚書左僕射劉遵考為丹陽尹 癸酉鄱陽哀王休業卒 太傅義恭以南兖州刺史西陽王子尚有寵將避之乃辭揚州秋七月解義恭揚州丙子以子尚為揚州刺史時熒惑守南斗上廢西州舊館使子尚移治東城以厭之【厭於葉翻又於琰翻斗楊州分故厭之】揚州别駕從事沈懷文曰天道示變宜應之以德今雖空西州恐無益也不從懷文懷遠之兄也 八月魏平西將軍漁陽公尉眷擊伊吾克其城大獲而還【李寶以伊吾敦煌降魏寶既入朝伊吾復叛故擊之尉紆勿翻還從宣翻又如字】 九月壬戌以丹陽尹劉遵考為尚書右僕射 冬十月甲申魏主還平城【亦不書所如之地】 丙午太傅義恭進位太宰領司徒 十一月魏以尚書西平王源賀為冀州刺史更賜爵隴西王【更工衡翻】賀上言今北虜遊魂南寇負險疆場之間猶須防戍【場音亦】臣愚以為自非大逆赤手殺人其坐盗及過誤應入死者皆可原宥讁使守邉則是已斷之體受更生之恩徭役之家蒙休息之惠魏高宗從之久之謂羣臣曰吾用賀言一歲所活不少【少詩沼翻】增戍兵亦多卿等人人如賀朕何憂哉會武邑人石華告賀謀反【武邑縣前漢屬信都後漢屬安平晉武帝分立武邑郡至隋唐為武邑武彊衡水三縣地】有司以聞帝曰賀竭誠事國朕為卿等保之【為于偽翻】無此明矣命精加訊驗華果引誣【自引服誣告之罪】帝誅之因謂左右曰以賀忠誠猶不免誣謗不及賀者可無慎哉 十二月濮陽太守姜龍駒新平太守楊自倫棄郡奔魏【按沈約志濮陽新平皆屬兖州而不載治所蓋僑郡也新平郡又明帝泰始七年立當考又按五代志鄄城縣舊置濮陽郡濮博木翻】 上欲移青冀二州倂鎮歷城議者多不同青冀二州刺史垣護之曰青州北有河濟【濟子禮翻】又多陂澤非虜所向每來寇掠必由歷城二州并鎮此經遠之略也北又近河歸順者易【近其靳翻易以豉翻】近息民患遠申主威安邉之上計也由是遂定【青州本治東陽冀州治歷城今并為一鎮】 元嘉中官鑄四銖錢輪郭形制與五銖同用費無利【言鑄一錢之費適當一錢之用無羸利也】故民不盜鑄及上即位又鑄孝建四銖形式薄小輪郭不成【錢外圓為輪内方為郭】於是盜鑄者衆雜以鉛錫翦鑿古錢錢轉薄小守宰不能禁坐死免者相繼盜鑄益甚物價踴貴朝廷患之去歲春詔錢薄小無輪郭者悉不得行民間喧擾是歲始興郡公沈慶之建議以為宜聽民鑄錢郡縣置錢署樂鑄之家皆居署内【樂音洛】平其凖式去其雜偽【去羌呂翻】去春所禁新品一時施用今鑄悉依此格萬稅三千嚴檢盜鑄【檢柬也勘察也】丹陽尹顔竣駁之【竣七倫翻駁多角翻】以為五銖輕重定於漢世【漢武帝元狩五年行五銖錢】魏晉以降莫之能改誠以物貨既均改之偽生故也今云去春所禁一時施用若巨細摠行而不從公鑄利已既深情偽無極私鑄翦鑿盡不可禁財貨未贍大錢已竭數歲之間悉為塵土矣今新禁初行品式未一須臾自止不足以垂聖慮唯府藏空匱實為重憂【藏徂浪翻下同】今縱行細錢官無益賦之理百姓雖贍【贍昌豔翻】無解官乏唯簡費去華【去羌呂翻】專在節儉求贍之道莫此為貴耳議者又以為銅轉難得欲鑄二銖錢竣曰議者以為官藏空虛宜更改鑄天下銅少宜減錢式以救交弊【官藏空虛無錢以贍用而天下銅少又無以鑄錢是交弊也議者緣此欲改鑄小錢以救之少詩沼翻】賑國舒民【賑富也又舉救也舒緩也寛也賑津忍翻】愚以為不然今鑄二銖恣行新細於官無解於乏而民間姦巧大興天下之貨將糜碎至盡空嚴立禁而利深難絶不一二年其弊不可復救【言不待一二年而弊甚也復扶又翻】民懲大錢之改兼畏近日新禁市井之間必生紛擾遠利未聞切患猥及富商得志貧民困窘此皆甚不可者也乃止【窘渠隕翻】 魏定州刺史高陽許宗之求取不節深澤民馬超謗毁宗之【深澤縣前漢屬涿郡後漢屬安平晉以來屬博陵郡後魏博陵郡屬定州唐以博陵郡為定州後分定州置祁州深澤縣屬焉】宗之毆殺超【毆烏口翻擊也】恐其家人告狀上超詆訕朝政【上時掌翻】魏高宗曰此必妄也【魏字衍】朕為天下主何惡於超而有此言【惡烏路翻】必宗之懼罪誣超案驗果然斬宗之於都南 金紫光禄大夫顔延之卒延之子竣貴重凡所資供【供居用翻】延之一無所受布衣茅屋蕭然如故常乘羸牛笨車【笨部本翻竹裏也一曰不精也】逢竣鹵簿即屏住道側【導從之次第曰鹵簿屛必郢翻】常語竣曰吾平生不憙見要人【語牛倨翻憙許記翻】今不幸見汝竣起宅延之謂曰善為之無令後人笑汝拙也延之嘗早詣竣見賓客盈門竣尚未起延之怒曰汝出糞土之中升雲霞之上遽驕傲如此其能久乎【物忌盛滿顔竣之禍其父知之矣】竣丁父憂【丁當也郭璞曰值也】裁踰月起為右將軍丹陽尹如故竣固辭表十上【上時掌翻】上不許遣中書舍人戴明寶抱竣登車載之郡舍【之往也郡舍丹陽尹廨也】賜以布衣一襲絮以綵綸遣主衣就衣諸體【主衣主御衣服唐尚衣奉御之職也就衣於既翻】<br />
<br />
  大明元年春正月辛亥朔改元大赦 壬戌魏主畋於崞山【崞山在雁門郡崞縣崞古博翻】戊辰還平城 魏以漁陽王尉眷為太尉録尚書事【尉紆勿翻】 二月魏人寇兖州向無鹽敗東平太守南陽劉胡【無鹽縣自漢以來屬東平郡敗補邁翻】詔遣太子左衛率薛安都將騎兵東陽太守沈法系將水軍向彭城以禦之【率所律翻將即亮翻騎奇寄翻】並受徐州刺史申坦節度比至【比必利翻及也】魏兵已去先是羣盗聚任城荆榛中累世為患謂之任榛【先悉薦翻任音壬任城縣前漢屬東平郡後漢分為任城國後遂為郡宋省郡為任城縣屬高平郡】申坦請回軍討之上許之任榛聞之皆逃散時天旱人馬渴乏無功而還【還從宣翻又如字】安都法系坐白衣領職坦當誅羣臣為請莫能得【為于偽翻】沈慶之抱坦哭於市曰汝無辠而死我哭汝於市行當就汝矣有司以聞上乃免之 三月庚申魏主畋於松山己巳還平城 魏主立其弟新成為陽平王 上自即吉之後【三年之喪既除而即吉】奢淫自恣多所興造丹陽尹顔竣以藩朝舊臣【上為藩王時竣為僚佐是藩朝舊臣也晉宋之間郡曰郡朝府曰府朝藩王曰藩朝宋武帝為宋王齊高帝為齊王時曰霸朝朝直遥翻下同】數懇切諫爭【數所角翻爭則迸翻】無所回避上浸不悦竣自謂才足幹時恩舊莫比當居中永執朝政而所陳多不納疑上欲踈之乃求外出以占上意夏六月丁亥詔以竣為東揚州刺史竣始大懼【為帝殺竣張本】 癸卯魏主如隂山 雍州所統多僑郡縣【雍於用翻下同】刺史王玄謨上言僑郡縣無有境土新舊錯亂租課不時請皆土斷【斷丁亂翻】秋七月辛未詔并雍州三郡十六縣為一郡郡縣流民不願屬籍【屬土著之籍也】訛言玄謨欲反時柳元景宗彊羣從多為雍部二千石【柳元景河東解人南徙僑居于雍部羣從羣從兄弟從才用翻】乘聲皆欲討玄謨玄謨令内外晏然以解衆惑馳使啓上具陳本末【使疏吏翻】上知其虚遣主書吳喜撫慰之【主書後漢尚書令史之職漢尚書曹有主書令史二十一人江左以來中書省有主書】且報曰七十老公反欲何求君臣之際足以相保聊復為笑伸卿眉頭耳【復扶又翻】玄謨性嚴未嘗妄笑故上以此戱之 八月己亥魏主還平城 甲辰徙司空南徐州刺史竟陵王誕為南兖州刺史以太子詹事劉延孫為南徐州刺史初高祖遺詔以京口要地去建康密邇自非宗室近親不得居之延孫之先雖與高祖同源而高祖屬彭城延孫屬莒縣【南史延孫傳作呂縣呂縣屬彭城郡而莒縣屬東莞郡詳而考之呂縣為是彭城呂二縣並屬彭城郡延孫與帝室同源同郡特異縣耳】從來不序昭穆【昭讀如字】上既命延孫鎮京口仍詔與延孫合族使諸王皆序長幼【長知兩翻】上閨門無禮不擇親疎尊卑流聞民間無所不至誕寛而有禮又誅太子劭丞相義宣皆有大功【誕起兵討劭見上卷文帝元嘉三十年勸止上迎義宣事見上】人心竊向之誕多聚才力之士蓄精甲利兵上由是畏而忌之不欲誕居中使出鎮京口猶嫌其逼更徙之廣陵【南兖州時治廣陵】以延孫腹心之臣使鎮京口以防之【為帝討誕張本】 魏主將東廵冬十月詔太宰常英起行宫於遼西黄山【魏收地形志遼西郡肥如縣有黄山】 十二月丁亥更以順陽王休範為桂陽王【休範孝建三年封順陽王更工衡翻】<br />
<br />
  二年春正月丙午朔魏設酒禁釀酤飲者皆斬之【釀者酤者飲者皆斬】吉凶之會聽開禁有程日魏主以士民多因酒致鬪及議國政故禁之增置内外候官伺察諸曹及州鎮【魏自道武帝以來有候官今增其員伺相吏翻】或微服雜亂於府寺間以求百官過失有司窮治訊掠取服【治直之翻】百官滿二丈者皆斬又增律七十九章 乙卯魏主如廣甯温泉宫遂廵平州【魏平州之地止遼西北平二郡】庚午至黄山宫二月丙子登碣石山觀滄海戊寅南如信都畋於廣川【廣川縣前漢屬廣川國後漢屬清河郡晉屬勃海郡魏收地形志屬長樂郡長樂即信都也五代志曰北齊廢廣川入棗彊劉昫曰隋於舊縣東八十里置新縣尋改為長河縣屬德州】 乙酉以金紫光禄大夫禇湛之為尚書左僕射 丙戌建平宣簡王宏以疾解尚書令三月丁未卒 丙辰魏高宗還平城起太華殿【酈道元曰魏太和十六年破太華安昌諸殿造太極殿東西堂及朝堂】是時給事中郭善明性傾巧說帝大起宫室【說輸芮翻】中書侍郎高允諫曰太祖始建都邑其所營立必因農隙况建國已久永安前殿足以朝會【朝直遥翻】西堂温室足以宴息紫樓足以臨望縱有修廣亦宜馴致【易曰馴致其道向秀曰馴從也程願曰馴謂習習而漸至於盛馴似遵翻】不可倉猝今計所當役凡二萬人老弱供餉又當倍之期半年可畢一夫不耕或受之飢况四萬人之勞費可勝道乎【勝音升】此陛下所宜留心也帝納之允好切諫朝廷事有不便允輒求見帝常屏左右以待之【好呼到翻屏必郢翻屏左右者欲其言無不盡】或自朝至暮或連日不出羣臣莫知其所言語或痛切帝所不忍聞命左右扶出然終善遇之時有上事為激訐者帝省之【上時掌翻訐居謁翻省悉景翻】謂羣臣曰君父一也父有過子何不作書於衆中諫之而於私室屏處諫者【屏蔽也屏處隱蔽之處屏必郢翻】豈非不欲其父之惡彰於外邪至於事君何獨不然君有得失不能面陳而上表顯諫欲以彰君之短明己之直此豈忠臣所為乎如高允者乃忠臣也朕有過未嘗不面言至有朕所不堪聞者允皆無所避朕知其過而天下不知可不謂忠乎允所與同徵者游雅等【徵允等見一百二十二卷文帝元嘉八年】皆至大官封侯部下吏至刺史二千石者亦數十百人【部下吏謂中書之吏嘗事允在部下者】而允為郎二十七年不徙官【魏世祖神䴥四年允徵拜中書博士領著作郎至是年二十五年耳】帝謂羣臣曰汝等雖執弓刀在朕左右徒立耳【言徒能侍立而不能規諫】未嘗有一言規正唯伺朕喜悦之際【伺相吏翻】祈官乞爵今皆無功而至王公允執筆佐我國家數十年為益不小不過為郎汝等不自愧乎乃拜允中書令【上云二十七年不徙官意允拜中書令不在是年】時魏百官無禄允常使諸子樵采以自給司徒陸麗言於帝曰高允雖蒙寵待而家貧妻子不立【立成也置也建也謂不能建置家業也】帝曰公何不先言今見朕用之乃言其貧乎即日至允第惟草屋數間布被緼袍【孔安國曰緼枲著也謂雜用枲麻以著袍禮記曰緼為袍鄭康成注曰緼舊絮也又亂麻緼於粉翻】厨中鹽菜而已帝歎息賜帛五百匹粟千斛拜長子悦為長樂太守【樂音洛守手又翻】允固辭不許帝重允常呼為令公而不名游雅常曰前史稱卓子康劉文饒之為人【卓茂字子康劉寛字文饒】心者或不之信【補典翻】余與高子游處四十年【處昌呂翻】未嘗見其喜愠之色【愠於問翻】乃知古人為不誣耳高子内文明而外柔順其言呐呐不能出口【呐如悦翻又奴劣翻呐呐言緩也】昔崔司徒嘗謂余云高生豐才博學一代佳士所乏者矯矯風節耳余亦以為然及司徒得辠起於纎微詔指臨責司徒聲嘶股栗殆不能言【嘶先齊翻聲破曰嘶】宗欽已下伏地流汗皆無人色高子獨敷陳事理申釋是非辭義清辯音韵高亮人主為之動容聽者無不神聳【事見一百二十五卷文帝元嘉二十七年為于偽翻】此非所謂矯矯者乎宗愛方用事威振四海嘗召百官於都坐【魏有都坐大官魏之都坐猶唐之朝堂也或曰都坐尚書都坐即唐之政事堂坐徂卧翻】王公已下皆趨庭望拜高子獨升階長揖由此觀之汲長孺可以卧見衛青何抗禮之有【言以高允之揖宗愛觀之則汲黯可以卧見衛青與之抗禮未為過也汲黯字長孺抗禮事見十九卷漢武帝元朔五年】此非所謂風節者乎夫人固未易知【易以䜴翻】吾既失之於心崔又漏之於外【之於言則是漏之於外】此乃管仲所以致慟於鮑叔也【管仲曰生我者父母知我者鮑子也致慟蓋感其知已之深】 乙丑魏東平成王陸俟卒 夏四月甲申立皇子子綏為安陸王帝不欲權在臣下六月戊寅分吏部尚書置二人【吏部】<br />
<br />
  【尚書掌銓選以其權重江左謂之大尚書言其位任與諸曹殊絶也今置二人以分其權】以都官尚書謝莊度支尚書吳郡顧覬之為之【漢置六曹尚書中都官曹主水火盗賊事魏晉省宋復置隋改都官為刑部尚書改度支為民部尚書唐避太宗諱改民部為戶部度徒洛翻覬音冀下同】又省五兵尚書【曹魏置五兵尚書隋改曰兵部尚書】初晉世散騎常侍選望甚重【上之所遴簡為選時之所瞻屬為望散悉亶翻騎奇寄翻】與侍中不異其後職任閒散【散悉但翻】用人漸輕上欲重其選乃用當時名士臨海太守孔覬司徒長史王彧為之【彧於六翻】侍中蔡興宗謂人曰選曹要重常侍閒淡改之以名而不以實雖主意欲為輕重人心豈可變邪既而常侍之選復卑選部之貴不異【言選部貴重與前時無以異也選部須絹翻復扶又翻】覬琳之之孫【孔琳之事桓玄不務迎合諫其廢錢用穀帛復肉刑】彧謐之兄孫興宗廓之子也【王謐識武帝於龍潛蔡廓以方直著于宋初】<br />
<br />
  裴子野論曰官人之難先王言之尚矣【書臯陶曰在知人禹曰惟帝其難之知人則哲能官人】周禮始於學校【校戶敎翻】論之州里告諸六事而後貢于王庭【六事周之六卿也】其在漢家州郡積其功能五府舉為掾屬【掾以絹翻】三公參其得失尚書奏之天子一人之身所閲者衆【閲更歷也】故能官得其才鮮有敗事【鮮息淺翻】魏晉易是所失弘多【弘大也】夫厚貌深衷險如谿壑擇言觀行猶懼弗周况今萬品千群俄折乎一面【行下孟翻下戒行同折之列翻斷也一面一覿面之頃也】庶僚百位專斷於一司【一司謂選部斷丁亂翻】于是囂風遂行不可抑止【囂風謂喧競之風】干進務得兼加諂凟【易大傳曰君子上交不諂下交不凟】無復廉恥之風謹厚之操官邪國敗【左傳曰國家之敗由官邪也復扶又翻】不可紀綱假使龍作納言【尚書古之納言也】舜居南面而治致平章不可必也【堯典曰平章百姓孔注曰百姓百官平和章明不可必言不可必致也治直吏翻】况後之官人者哉孝武雖分曹為兩【謂吏部置兩尚書】不能反之於周漢朝三暮四其庸愈乎【莊子曰狙公賦芧曰朝三而暮四衆狙皆怒曰然則朝四而暮三衆狙皆喜名實未虧而喜怒為用】<br />
<br />
  丙申魏主畋于松山庚午如河西 南彭城民高闍【晉氏南渡僑立南彭城郡於晉陵界闍視遽翻】沙門曇標以妖妄相扇【曇徒含翻妖於遥翻】與殿中將軍苗允等謀作亂立闍為帝事覺甲辰皆伏誅死者數十人於是下詔沙汰諸沙門設諸科禁嚴其誅坐自非戒行精苦並使還俗【行下孟翻】而諸尼多出入宫掖此制竟不能行中書令王僧達幼聰警能文而跌蕩不拘【跌徒結翻蕩徒浪翻不拘言其不拘常檢也】帝初踐阼擢為僕射居顔劉之右【顔竣劉延孫帝之腹心也】自負才地【地謂門地】謂當時莫及一二年間即望宰相既而遷護軍怏怏不得志【怏於兩翻】累啓求出上不悦由是稍稍下遷五歲七徙【七徙官也】再被彈削僧逹既恥且怨所上表奏辭旨抑揚又好非議朝政【上時掌翻好呼到翻朝直遙翻】上已積憤怒路太后兄子嘗詣僧達趨升其榻僧達令舁棄之【路太后兄慶之嘗為王氏門下騶故僧達麾其子舁音余又羊茹翻對舉也孔光屈身於董賢以保其禄位人以為諂王僧達抗意於路瓊之以殺其身人以為躁遠小人不惡而嚴君子蓋必有道也】太后大怒固邀上令必殺僧達會高闍反上因誣僧達與闍通謀八月丙戌收付廷尉賜死<br />
<br />
  沈約論曰夫君子小人類物之通稱蹈道則為君子違之則為小人【稱尺證翻】是以太公起屠釣為周師傅說去板築為殷相【太公屠牛于朝歌釣于渭濱周文王迎以為師傅說築于傅巖之野殷高宗求以為相說於悦翻相息亮翻】明敭幽仄【書曰明明揚側陋敭與揚同】唯才是與逮于二漢兹道未革胡廣累世農夫致位公相黄憲牛醫之子名重京師非若晩代分為二途也魏武始立九品蓋以論人才優劣【詳見八十一卷晉武帝太康五年】非謂世族高卑而都正俗士随時俯仰憑藉世資用相陵駕【都正謂諸州中正也】因此相沿遂為成法周漢之道以智役愚魏晉以來以貴役賤士庶之科較然有辨矣 裴子野論曰古者德義可尊無擇負販苟非其人何取世族名公子孫還齊布衣之伍士庶雖分本無華素之隔【華榮也輝也故榮貴之族謂之華宗其子弟謂之華胄素白也質也故白屋謂之素門寒士謂之素士】自晉以來其流稍改草澤之士猶顯清途降及季年專限閥閱【史記明其等曰閥積其功曰閱又門在左曰閥在右曰閲閥音伐】自是三公之子傲九棘之家【周禮朝士掌外朝之法左九棘孤卿大夫位焉右九棘公侯伯子男位焉後世直謂九棘為九卿】黄散之孫蔑令長之室【散悉亶翻長知兩翻】轉相驕矜互爭銖兩唯論門戶不問賢能以謝靈運王僧達之才華輕躁使其生自寒宗猶將覆折【折而設翻】重以怙其庇廕召禍宜哉【重直用翻】<br />
<br />
  九月乙巳魏主還平城【自河西還】 丙寅魏大赦 冬十月甲戌魏主北廵欲伐柔然至隂山會雨雪魏主欲還太尉尉眷曰今動大衆以威北狄去都不遠而車駕遽還虜必疑我有内難【尉尉下紆勿翻難乃旦翻】將士雖寒不可不進魏主從之辛卯軍于車崘山【車崘山北史作車輪山魏收地形志秀容郡敷城縣有車輪泉神】 積射將軍殷孝祖築兩城於清水之東【文帝元嘉九年置積射彊弩等將軍沈約曰晉太康十年置】魏鎮西將軍封敕文攻之清口戍主振威將軍傅乾愛拒破之孝祖羨之曾孫也【殷羨殷浩之父】上遣虎賁主龎孟蚪救清口【虎賁主主虎賁士賁音奔 考異曰宋顔師伯傳云魏遣清水公捨賁敕文寇清口世祖遣孟蚪及殷孝祖赴討魏本紀孝祖修兩城於清水東詔封敕文擊之今從之】青冀二州刺史顔師伯遣中兵參軍苟思達助之敗魏兵於沙溝【按此清口非清水入淮之口乃濟水與汶水合之口水經濟水東北過夀張縣西安民亭南汶水從東北來注之注云戴延之所謂清口也濟水又北過須昌穀城臨邑盧縣又東北與中川水合注云中川水與賓溪水合而北流逕盧縣故城東又北流入濟俗謂之沙溝水敗補邁翻】師伯竣之族兄也上遣司空參軍卜天生將兵會傅乾愛及中兵參軍江方興共擊魏兵屢破之【江方興蓋司空中兵參軍將即亮翻下同】斬魏將窟瓌公等數人【窟苦骨翻瓌姑回翻】十一月魏征西將軍皮豹子等將三萬騎助封敕文寇青州顔師伯禦之輔國將軍焦度刺豹子墜馬獲其鎧矟具裝手殺數十人度本南安氐也【刺七亦翻鎧苦亥翻矟邑角翻】 魏主自將騎十萬車十五萬兩擊柔然度大漠旌旗千里柔然處羅可汗遠遁其别部烏朱駕頹等帥數千落降于魏【騎奇寄翻兩音亮可從刋入聲汗音寒帥讀曰率降戶江翻】魏主刻石紀功而還【還從宣翻】 初上在江州山隂戴法興戴明寶蔡閑為典籖及即位皆以為南臺侍御史兼中書通事舍人【御史臺謂之南臺晉初置中書舍人通事各一人江左令舍人通事謂之通事舍人掌呈奏案又掌詔命】是歲三典籖並以初舉兵預密謀賜爵縣男閑已卒追賜之時上親覽朝政【朝直遥翻】不任大臣而腹心耳目不得無所委寄法興頗知古今素見親待魯郡巢尚之人士之末涉獵文史為上所知亦以為中書通事舍人凡選授誅賞大處分【處昌呂翻分扶問翻下定分同】上皆與法興尚之參懷【宋齊之間凡參决機務率皆謂之參懷】内外雜事多委明寶三人權重當時而法興明寶大納貨賄凡所薦達言無不行天下輻凑門外成市家產並累千金吏部尚書顧覬之獨不降意於法興等蔡興宗與覬之善嫌其風節太峻覬之曰辛毗有言孫劉不過使吾不為三公耳【魏明帝時劉放孫資制斷時政大臣莫不交好而辛毗不與往來毗子敞諫曰劉孫用事衆皆影附大人宜少降意不然必有謗言毗正色曰吾之立身自有本末就與孫劉不平不過不為三公大丈夫欲為公而毁其高節邪覬音冀】覬之常以為人禀命有定分【分扶問翻】非智力可移唯應恭已守道而闇者不達妄意僥倖【僥堅堯翻】徒虧雅道無關得喪【喪息浪翻】乃以其意命弟子原著定命論以釋之【原南史作愿】<br />
<br />
  資治通鑑卷一百二十八  <br>
   </div> 

<script src="/search/ajaxskft.js"> </script>
 <div class="clear"></div>
<br>
<br>
 <!-- a.d-->

 <!--
<div class="info_share">
</div> 
-->
 <!--info_share--></div>   <!-- end info_content-->
  </div> <!-- end l-->

<div class="r">   <!--r-->



<div class="sidebar"  style="margin-bottom:2px;">

 
<div class="sidebar_title">工具类大全</div>
<div class="sidebar_info">
<strong><a href="http://www.guoxuedashi.com/lsditu/" target="_blank">历史地图</a></strong>  
<a href="http://www.880114.com/" target="_blank">英语宝典</a>  
<a href="http://www.guoxuedashi.com/13jing/" target="_blank">十三经检索</a> 
<br><strong><a href="http://www.guoxuedashi.com/gjtsjc/" target="_blank">古今图书集成</a></strong> 
<a href="http://www.guoxuedashi.com/duilian/" target="_blank">对联大全</a> <strong><a href="http://www.guoxuedashi.com/xiangxingzi/" target="_blank">象形文字典</a></strong> 

<br><a href="http://www.guoxuedashi.com/zixing/yanbian/">字形演变</a>  <strong><a href="http://www.guoxuemi.com/hafo/" target="_blank">哈佛燕京中文善本特藏</a></strong>
<br><strong><a href="http://www.guoxuedashi.com/csfz/" target="_blank">丛书&方志检索器</a></strong> <a href="http://www.guoxuedashi.com/yqjyy/" target="_blank">一切经音义</a>  

<br><strong><a href="http://www.guoxuedashi.com/jiapu/" target="_blank">家谱族谱查询</a></strong>  <strong><a href="http://shufa.guoxuedashi.com/sfzitie/" target="_blank">书法字帖欣赏</a></strong> 
<br>

</div>
</div>


<div class="sidebar" style="margin-bottom:0px;">

<font style="font-size:22px;line-height:32px">QQ交流群9:489193090</font>


<div class="sidebar_title">手机APP 扫描或点击</div>
<div class="sidebar_info">
<table>
<tr>
	<td width=160><a href="http://m.guoxuedashi.com/app/" target="_blank"><img src="/img/gxds-sj.png" width="140"  border="0" alt="国学大师手机版"></a></td>
	<td>
<a href="http://www.guoxuedashi.com/download/" target="_blank">app软件下载专区</a><br>
<a href="http://www.guoxuedashi.com/download/gxds.php" target="_blank">《国学大师》下载</a><br>
<a href="http://www.guoxuedashi.com/download/kxzd.php" target="_blank">《汉字宝典》下载</a><br>
<a href="http://www.guoxuedashi.com/download/scqbd.php" target="_blank">《诗词曲宝典》下载</a><br>
<a href="http://www.guoxuedashi.com/SiKuQuanShu/skqs.php" target="_blank">《四库全书》下载</a><br>
</td>
</tr>
</table>

</div>
</div>


<div class="sidebar2">
<center>


</center>
</div>

<div class="sidebar"  style="margin-bottom:2px;">
<div class="sidebar_title">网站使用教程</div>
<div class="sidebar_info">
<a href="http://www.guoxuedashi.com/help/gjsearch.php" target="_blank">如何在国学大师网下载古籍?</a><br>
<a href="http://www.guoxuedashi.com/zidian/bujian/bjjc.php" target="_blank">如何使用部件查字法快速查字?</a><br>
<a href="http://www.guoxuedashi.com/search/sjc.php" target="_blank">如何在指定的书籍中全文检索?</a><br>
<a href="http://www.guoxuedashi.com/search/skjc.php" target="_blank">如何找到一句话在《四库全书》哪一页?</a><br>
</div>
</div>


<div class="sidebar">
<div class="sidebar_title">热门书籍</div>
<div class="sidebar_info">
<a href="/so.php?sokey=%E8%B5%84%E6%B2%BB%E9%80%9A%E9%89%B4&kt=1">资治通鉴</a> <a href="/24shi/"><strong>二十四史</strong></a>&nbsp; <a href="/a2694/">野史</a>&nbsp; <a href="/SiKuQuanShu/"><strong>四库全书</strong></a>&nbsp;<a href="http://www.guoxuedashi.com/SiKuQuanShu/fanti/">繁体</a>
<br><a href="/so.php?sokey=%E7%BA%A2%E6%A5%BC%E6%A2%A6&kt=1">红楼梦</a> <a href="/a/1858x/">三国演义</a> <a href="/a/1038k/">水浒传</a> <a href="/a/1046t/">西游记</a> <a href="/a/1914o/">封神演义</a>
<br>
<a href="http://www.guoxuedashi.com/so.php?sokeygx=%E4%B8%87%E6%9C%89%E6%96%87%E5%BA%93&submit=&kt=1">万有文库</a> <a href="/a/780t/">古文观止</a> <a href="/a/1024l/">文心雕龙</a> <a href="/a/1704n/">全唐诗</a> <a href="/a/1705h/">全宋词</a>
<br><a href="http://www.guoxuedashi.com/so.php?sokeygx=%E7%99%BE%E8%A1%B2%E6%9C%AC%E4%BA%8C%E5%8D%81%E5%9B%9B%E5%8F%B2&submit=&kt=1"><strong>百衲本二十四史</strong></a>  <a href="http://www.guoxuedashi.com/so.php?sokeygx=%E5%8F%A4%E4%BB%8A%E5%9B%BE%E4%B9%A6%E9%9B%86%E6%88%90&submit=&kt=1"><strong>古今图书集成</strong></a>
<br>

<a href="http://www.guoxuedashi.com/so.php?sokeygx=%E4%B8%9B%E4%B9%A6%E9%9B%86%E6%88%90&submit=&kt=1">丛书集成</a> 
<a href="http://www.guoxuedashi.com/so.php?sokeygx=%E5%9B%9B%E9%83%A8%E4%B8%9B%E5%88%8A&submit=&kt=1"><strong>四部丛刊</strong></a>  
<a href="http://www.guoxuedashi.com/so.php?sokeygx=%E8%AF%B4%E6%96%87%E8%A7%A3%E5%AD%97&submit=&kt=1">說文解字</a> <a href="http://www.guoxuedashi.com/so.php?sokeygx=%E5%85%A8%E4%B8%8A%E5%8F%A4&submit=&kt=1">三国六朝文</a>
<br><a href="http://www.guoxuedashi.com/so.php?sokeytm=%E6%97%A5%E6%9C%AC%E5%86%85%E9%98%81%E6%96%87%E5%BA%93&submit=&kt=1"><strong>日本内阁文库</strong></a> <a href="http://www.guoxuedashi.com/so.php?sokeytm=%E5%9B%BD%E5%9B%BE%E6%96%B9%E5%BF%97%E5%90%88%E9%9B%86&ka=100&submit=">国图方志合集</a> <a href="http://www.guoxuedashi.com/so.php?sokeytm=%E5%90%84%E5%9C%B0%E6%96%B9%E5%BF%97&submit=&kt=1"><strong>各地方志</strong></a>

</div>
</div>


<div class="sidebar2">
<center>

</center>
</div>
<div class="sidebar greenbar">
<div class="sidebar_title green">四库全书</div>
<div class="sidebar_info">

《四库全书》是中国古代最大的丛书,编撰于乾隆年间,由纪昀等360多位高官、学者编撰,3800多人抄写,费时十三年编成。丛书分经、史、子、集四部,故名四库。共有3500多种书,7.9万卷,3.6万册,约8亿字,基本上囊括了古代所有图书,故称“全书”。<a href="http://www.guoxuedashi.com/SiKuQuanShu/">详细>>
</a>

</div> 
</div>

</div>  <!--end r-->

</div>
<!-- 内容区END --> 

<!-- 页脚开始 -->
<div class="shh">

</div>

<div class="w1180" style="margin-top:8px;">
<center><script src="http://www.guoxuedashi.com/img/plus.php?id=3"></script></center>
</div>
<div class="w1180 foot">
<a href="/b/thanks.php">特别致谢</a> | <a href="javascript:window.external.AddFavorite(document.location.href,document.title);">收藏本站</a> | <a href="#">欢迎投稿</a> | <a href="http://www.guoxuedashi.com/forum/">意见建议</a> | <a href="http://www.guoxuemi.com/">国学迷</a> | <a href="http://www.shuowen.net/">说文网</a><script language="javascript" type="text/javascript" src="https://js.users.51.la/17753172.js"></script><br />
  Copyright &copy; 国学大师 古典图书集成 All Rights Reserved.<br>
  
  <span style="font-size:14px">免责声明:本站非营利性站点,以方便网友为主,仅供学习研究。<br>内容由热心网友提供和网上收集,不保留版权。若侵犯了您的权益,来信即刪。scp168@qq.com</span>
  <br />
ICP证:<a href="http://www.beian.miit.gov.cn/" target="_blank">鲁ICP备19060063号</a></div>
<!-- 页脚END --> 
<script src="http://www.guoxuedashi.com/img/plus.php?id=22"></script>
<script src="http://www.guoxuedashi.com/img/tongji.js"></script>

</body>
</html>
