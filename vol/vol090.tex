<!DOCTYPE html PUBLIC "-//W3C//DTD XHTML 1.0 Transitional//EN" "http://www.w3.org/TR/xhtml1/DTD/xhtml1-transitional.dtd">
<html xmlns="http://www.w3.org/1999/xhtml">
<head>
<meta http-equiv="Content-Type" content="text/html; charset=utf-8" />
<meta http-equiv="X-UA-Compatible" content="IE=Edge,chrome=1">
<title>資治通鑒_91-資治通鑑卷九十_91-資治通鑑卷九十</title>
<meta name="Keywords" content="資治通鑒_91-資治通鑑卷九十_91-資治通鑑卷九十">
<meta name="Description" content="資治通鑒_91-資治通鑑卷九十_91-資治通鑑卷九十">
<meta http-equiv="Cache-Control" content="no-transform" />
<meta http-equiv="Cache-Control" content="no-siteapp" />
<link href="/img/style.css" rel="stylesheet" type="text/css" />
<script src="/img/m.js?2020"></script> 
</head>
<body>
 <div class="ClassNavi">
<a  href="/24shi/">二十四史</a> | <a href="/SiKuQuanShu/">四库全书</a> | <a href="http://www.guoxuedashi.com/gjtsjc/"><font  color="#FF0000">古今图书集成</font></a> | <a href="/renwu/">历史人物</a> | <a href="/ShuoWenJieZi/"><font  color="#FF0000">说文解字</a></font> | <a href="/chengyu/">成语词典</a> | <a  target="_blank"  href="http://www.guoxuedashi.com/jgwhj/"><font  color="#FF0000">甲骨文合集</font></a> | <a href="/yzjwjc/"><font  color="#FF0000">殷周金文集成</font></a> | <a href="/xiangxingzi/"><font color="#0000FF">象形字典</font></a> | <a href="/13jing/"><font  color="#FF0000">十三经索引</font></a> | <a href="/zixing/"><font  color="#FF0000">字体转换器</font></a> | <a href="/zidian/xz/"><font color="#0000FF">篆书识别</font></a> | <a href="/jinfanyi/">近义反义词</a> | <a href="/duilian/">对联大全</a> | <a href="/jiapu/"><font  color="#0000FF">家谱族谱查询</font></a> | <a href="http://www.guoxuemi.com/hafo/" target="_blank" ><font color="#FF0000">哈佛古籍</font></a> 
</div>

 <!-- 头部导航开始 -->
<div class="w1180 head clearfix">
  <div class="head_logo l"><a title="国学大师官网" href="http://www.guoxuedashi.com" target="_blank"></a></div>
  <div class="head_sr l">
  <div id="head1">
  
  <a href="http://www.guoxuedashi.com/zidian/bujian/" target="_blank" ><img src="http://www.guoxuedashi.com/img/top1.gif" width="88" height="60" border="0" title="部件查字,支持20万汉字"></a>


<a href="http://www.guoxuedashi.com/help/yingpan.php" target="_blank"><img src="http://www.guoxuedashi.com/img/top230.gif" width="600" height="62" border="0" ></a>


  </div>
  <div id="head3"><a href="javascript:" onClick="javascript:window.external.AddFavorite(window.location.href,document.title);">添加收藏</a>
  <br><a href="/help/setie.php">搜索引擎</a>
  <br><a href="/help/zanzhu.php">赞助本站</a></div>
  <div id="head2">
 <a href="http://www.guoxuemi.com/" target="_blank"><img src="http://www.guoxuedashi.com/img/guoxuemi.gif" width="95" height="62" border="0" style="margin-left:2px;" title="国学迷"></a>
  

  </div>
</div>
  <div class="clear"></div>
  <div class="head_nav">
  <p><a href="/">首页</a> | <a href="/ShuKu/">国学书库</a> | <a href="/guji/">影印古籍</a> | <a href="/shici/">诗词宝典</a> | <a   href="/SiKuQuanShu/gxjx.php">精选</a> <b>|</b> <a href="/zidian/">汉语字典</a> | <a href="/hydcd/">汉语词典</a> | <a href="http://www.guoxuedashi.com/zidian/bujian/"><font  color="#CC0066">部件查字</font></a> | <a href="http://www.sfds.cn/"><font  color="#CC0066">书法大师</font></a> | <a href="/jgwhj/">甲骨文</a> <b>|</b> <a href="/b/4/"><font  color="#CC0066">解密</font></a> | <a href="/renwu/">历史人物</a> | <a href="/diangu/">历史典故</a> | <a href="/xingshi/">姓氏</a> | <a href="/minzu/">民族</a> <b>|</b> <a href="/mz/"><font  color="#CC0066">世界名著</font></a> | <a href="/download/">软件下载</a>
</p>
<p><a href="/b/"><font  color="#CC0066">历史</font></a> | <a href="http://skqs.guoxuedashi.com/" target="_blank">四库全书</a> |  <a href="http://www.guoxuedashi.com/search/" target="_blank"><font  color="#CC0066">全文检索</font></a> | <a href="http://www.guoxuedashi.com/shumu/">古籍书目</a> | <a   href="/24shi/">正史</a> <b>|</b> <a href="/chengyu/">成语词典</a> | <a href="/kangxi/" title="康熙字典">康熙字典</a> | <a href="/ShuoWenJieZi/">说文解字</a> | <a href="/zixing/yanbian/">字形演变</a> | <a href="/yzjwjc/">金 文</a> <b>|</b>  <a href="/shijian/nian-hao/">年号</a> | <a href="/diming/">历史地名</a> | <a href="/shijian/">历史事件</a> | <a href="/guanzhi/">官职</a> | <a href="/lishi/">知识</a> <b>|</b> <a href="/zhongyi/">中医中药</a> | <a href="http://www.guoxuedashi.com/forum/">留言反馈</a>
</p>
  </div>
</div>
<!-- 头部导航END --> 
<!-- 内容区开始 --> 
<div class="w1180 clearfix">
  <div class="info l">
   
<div class="clearfix" style="background:#f5faff;">
<script src='http://www.guoxuedashi.com/img/headersou.js'></script>

</div>
  <div class="info_tree"><a href="http://www.guoxuedashi.com">首页</a> > <a href="/SiKuQuanShu/fanti/">四库全书</a>
 > <h1>资治通鉴</h1> <!--         下载:【右键另存为】即可 --></div>
  <div class="info_content zj clearfix">
  
<div class="info_txt clearfix" id="show">
<center style="font-size:24px;">91-資治通鑑卷九十</center>
    資治通鑑卷九十    宋 司馬光 撰<br />
<br />
  胡三省 音註<br />
<br />
  晉紀十二【起彊圉赤奮若盡著雍攝提格凡二年】<br />
<br />
  中宗元皇帝上【諱睿字景文宣帝曾孫琅邪武王伷之孫恭王靚之子諡法始建國都曰元】<br />
<br />
  建武元年【是年三月方改元】春正月漢兵東略弘農太守宋哲犇江東【哲屯華隂漢兵自長安東略故棄城來犇守式又翻】 黄門郎史淑侍御史王冲自長安犇涼州稱愍帝出降前一日【降戶江翻】使淑等齎詔賜張寔拜寔大都督涼州牧侍中司空承制行事且曰朕已詔琅邪王時攝大位君其協贊琅邪共濟多難淑等至姑臧寔大臨三日【難乃旦翻臨力鴆翻】辭官不受初寔叔父肅為西海太守【王莽置西海郡光武中興棄之至獻帝興平二年武威太守張雅請置西海郡分張掖之居延一縣以屬之雖郡名同而非王莽西海郡之地】聞長安危逼請為先鋒入援寔以其老弗許及聞長安不守肅悲憤而卒【卒子恤翻】寔遣太府司馬韓璞【時張氏保據河西有太府司馬太府少府主簿等官蓋都督府為太府涼州府為少府也璞匹角翻】撫戎將軍張閬等帥步騎一萬東擊漢【撫戎將軍蓋張氏創置帥讀曰率騎奇寄翻】命討虜將軍陳安【沈約志魏置將軍四十號討虜第十九】安故太守賈騫【晉志曰張茂分武興金城西平安故四郡為定州蓋張氏分金城西平二郡地置安故郡也按安故縣二漢屬隴西郡水經注洮水自臨洮縣東流又屈而北流逕安故縣故城西又北逕狄道縣故城西狄道時已置武始郡安故郡蓋即漢之一縣置郡】隴西太守吳紹各統郡兵為前驅又遺相國保書曰王室有事不忘投軀前遣賈騫瞻公舉動中被符命敕騫還軍【符命蓋保符下寔也遺于季翻被皮義翻】俄聞寇逼長安胡崧不進麴允持金五百請救於崧遂决遣騫等進軍度嶺【自涼州濟河度沃于嶺至狄道】會聞朝廷傾覆為忠不遂憤痛之深死有餘責今更遣璞等唯公命是從璞等卒不能進而還至南安【南安郡治䝠道縣卒子恤翻還從宣翻又如宇】諸羌斷路【斷丁管翻】相持百餘日糧竭矢盡璞殺車中牛以饗士泣謂之曰汝曹念父母乎曰念念妻子乎曰念欲生還乎曰欲從我令乎曰諾乃鼓譟進戰會張閬帥金城兵繼至夾擊大破之斬首數千級【帥讀曰率下同】先是長安謡曰秦川中血没腕唯有涼州倚柱觀【腕烏貫翻】及漢兵覆關中氐羌掠隴右雍秦之民死者什八九【雍於用翻】獨涼州安全 二月漢主聰使從弟暢【從才用翻】帥步騎三萬攻滎陽太守李矩屯韓王故壘相去七里【李矩屯新鄭則韓王故壘亦在新鄭也戰國時韓滅鄭徙都之故有故壘在也】遣使招矩【使疏吏翻】時暢兵猝至矩未及為備乃遣使詐降於暢暢不復設備大饗渠帥皆醉【降戶江翻復扶又翻帥所類翻】矩欲夜襲之士卒皆恇懼【恇去王翻】矩乃遣其將郭誦禱於子產祠【子產相鄭鄭人懷其惠為之立祠】使巫揚言曰子產有教當遣神兵相助衆皆踊躍爭進矩選勇敢千人使誦將之【將即亮翻】掩擊暢營斬首數千級暢僅以身免 辛巳宋哲至建康【沈約曰建康本秣陵縣漢獻帝建安十六年置孫權改秣陵為建業武帝平吴還為秣陵太康三年分秣陵之水北為建業愍帝即位避帝諱改為建康】稱受愍帝詔令丞相琅邪王睿統攝萬機三月琅邪王素服出次【杜預曰出次避正寢】舉哀三日於是西陽王羕及官屬等共上尊號【西陽王羕汝南王亮之子羕余亮翻上時掌翻】王不許羕等固請不已王慨然流涕曰孤罪人也諸賢見逼不已當琅邪耳呼私奴命駕將歸國【私奴謂私所畜養而給使令之奴非以罪没官者】羕等乃請依魏晉故事稱晉王許之辛卯即晉王位大赦改元始備百官立宗廟建社稷有司請立太子王愛次子宣城公裒欲立之【裒蒲侯翻】謂王導曰立子當以德導曰世子宣城俱有朗雋之美而世子年長【長知兩翻】王從之丙辰立世子紹為王太子封裒為琅邪王奉恭王後【帝後太宗故以裒奉琅邪國祀】仍以裒都督青徐兖三州諸軍事鎮廣陵以西陽王羕為太保封譙剛王遜之子氶為譙王【一本作譙王氶音拯】遜宣帝之弟子也又以征南大將軍王敦為大將軍江州牧揚州刺史王導為驃騎將軍都督中外諸軍事領中書監録尚書事【驃匹妙翻】丞相左長史刁協為尚書左僕射右長史周顗為吏部尚書軍諮祭酒【顗魚豈翻】賀循為中書令右司馬戴淵王邃為尚書司直劉隗為御史中丞行參軍劉超為中書舍人【晉志曰中書晉初置舍人通事各一人江左令舍人通事謂之通事舍人掌呈奏案】參軍事孔愉長兼中書郎【長兼蓋始於此】自餘參軍悉拜奉車都尉掾屬拜駙馬都尉行參軍舍人拜騎都尉【三都尉皆漢武帝置奉車都尉掌御乘輿車駙馬都尉掌駙馬騎都尉掌監羽林騎師古曰駙副馬也非正駕車皆為副馬一曰駙近也疾也晉武帝以宗室外戚為三都尉江左後罷奉車騎二都尉唯留駙馬都尉奉朝請諸尚公主者為之掾俞絹翻】王敦辭州牧王導以敦統六州辭中外都督賀循以老病辭中書令王皆許之以循為太常是時承喪亂之後江東草創【廣雅曰草造也創始也喪息浪翻】刁協久宦中朝諳練舊事【諳烏含翻悉也記也朝直遥翻】賀循為世儒宗明習禮學凡有疑議皆取决焉 劉琨段匹磾相與歃血同盟【磾丁奚翻歃色洽翻歠也】期以翼戴晉室辛丑琨檄告華夷遣兼左長史右司馬温嶠匹磾遣左長史榮邵奉表及盟文詣建康勸進【漢之禪于魏也文帝三讓魏朝羣臣累表請順天人之望此則勸進之造端也晉受魏禪何曾等亦然是時愍帝蒙塵四海無君琨等勸進為得其正】嶠羨之弟子也【温羨見八十六卷惠帝永興二年】嶠之從母為琨妻【母之姊妹為從母從才用翻】琨謂嶠曰晉祚雖衰天命未改吾當立功河朔使卿延譽江南行矣勉之王以鮮卑大都督慕容廆為都督遼左雜夷流民諸軍事龍驤將軍大單于昌黎公廆不受【遼左即遼東流民謂中州之民流移入遼東者廆戶罪翻驤思將翻】征虜將軍魯昌說廆曰今兩京覆没天子蒙塵【左傳叔帶之難襄王出居于鄭使告難于魯臧文仲對曰天子蒙塵于外敢不奔問官守說輸芮翻】琅邪王承制江東為四海所係屬【屬之欲翻】明公雖雄據一方而諸部猶阻兵未服者蓋以官非王命故也謂宜通使琅邪【使疏吏翻下同】勸承大統然後奉詔令以伐有罪誰敢不從處士遼東高詡曰【處昌呂翻】霸王之資非義不濟今晉室雖微人心猶附之宜遣使江東示有所尊然後仗大義以征諸部不患無辭矣【晉室雖衰慕容苻姚之興其初皆借王命以自重】廆從之遣長史王濟浮海詣建康勸進 漢相國粲使其黨王平謂太弟义曰適奉中詔云京師將有變宜衷甲以備非常义信之命宫臣皆衷甲以居【粲固忌刻而义亦愚甚矣甲在衣中為衷甲】粲馳遣告靳凖王沈【靳居惞翻沈持林翻】凖以白漢主聰曰太弟將為亂已衷甲矣聰大驚曰寧有是邪王沈等皆曰臣等聞之久矣屢言之而陛下不之信也聰使粲以兵圍東宫粲使凖沈收氐羌酋長十餘人窮問之【义為大單于氐羌酋長屬焉故皆服事東宫酋慈由翻長知兩翻】皆懸首高格【格以木為之周禮牛人祭祀共其牛牲之互鄭玄曰互若今屠家之懸肉格左思吴都賦曰峭格周施呂向曰格懸網木也】燒鐵灼目酋長自誣與义謀反聰謂沈等曰吾今而後知卿等之忠也當念知無不言勿恨往日言而不用也於是誅東宫官屬及乂素所親厚凖沈等素所憎怨者大臣數十人阬士卒萬五千餘人【所阬者東宫四衛之兵也】夏四月廢义為北部王【北部即匈奴後部居新興】粲尋使凖賊殺之义形神秀爽寛仁有器度故士心多附之聰聞其死哭之慟曰吾兄弟止餘二人而不相容【漢主淵諸子此時惟聰义二人在耳】安得使天下知吾心邪氐羌叛者甚衆以靳凖行車騎大將軍討平之 五月壬午日有食之 【考異曰帝紀天文志皆云五月丙子日食按長歷是月壬午朔無丙子今以歷為据】 六月丙寅温嶠等至建康王導周顗庾亮等皆愛嶠才爭與之交是時太尉豫州牧荀組冀州刺史邵續青州刺史曹嶷寧州刺史王遜東夷校尉崔毖等皆上表勸進【顗魚豈翻嶷魚力翻毖音祕】王不許 初流民張平樊雅各聚衆數千人在譙為塢主王之為丞相也遣行參軍譙國桓宣往說平雅平雅皆請降【說輸芮翻降戶江翻下同】及豫州刺史祖逖出屯蘆洲遣參軍殷乂詣平雅乂意輕平視其屋曰可作馬廐見大鑊曰可鑄鐵器平曰此乃帝王鑊天下清平方用之奈何毁之乂曰卿未能保其頭而愛鑊邪【鑊胡郭翻鼎而無足曰鑊說文云鑊江淮人謂之鍋浙人謂之鑊】平大怒於坐斬义【坐徂卧翻】勒兵固守逖攻之歲餘不下乃誘其部將謝浮使殺之【誘音酉將即亮翻】逖進據太丘【太丘縣後漢屬沛郡晉省賢曰太丘故城在今亳州永城縣西北】樊雅猶據譙城與逖相拒逖攻之不克請兵於南中郎將王含桓宣時為含參軍含遣宣將兵五百助逖逖謂宣曰卿信義已著於彼今復為我說雅【復扶又翻為于偽翻】宣乃單馬從兩人詣雅曰祖豫州方欲平蕩劉石倚卿為援前殷乂輕薄非豫州意也雅即詣逖降【降戶江翻】逖既入譙城石勒遣石虎圍譙王含復遣桓宣救之虎解去逖表宣為譙國内史已已晉王傳檄天下稱石虎敢帥犬羊渡河縱毒今遣琅邪王裒等九軍【帥讀曰率裦蒲侯翻】銳卒三萬水陸四道徑造賊場【造七到翻】受祖逖節度尋復召裒還建康【復扶又翻】秋七月大旱司冀并青雍州大蝗河汾溢漂千餘家<br />
<br />
  【皆漢境也雍於用翻】 漢王聰立晉王粲為皇太子領相國大單于總攝朝政如故【朝直遥翻】大赦 段匹磾推劉琨為大都督【磾丁奚翻】檄其兄遼西公疾陸眷及叔父涉復辰弟末柸等會于固安【固安縣漢屬涿郡魏晉改涿郡曰范陽固安曰故安劉昫曰唐易州易縣古故安縣地】共討石勒末柸說疾陸眷涉復辰曰【說輸芮翻】以父兄而從子弟恥也且幸而有功匹磾獨收之吾屬何有哉各引兵還琨匹磾不能獨留亦還薊【薊音計】 以荀組為司徒 八月漢趙固襲衛將軍華薈於臨潁殺之【臨潁縣屬潁川郡華戶化翻薈烏外翻】 初趙固與長史周振有隙振密譖固於漢主聰李矩之破劉暢也於帳中得聰詔令暢既克矩還過洛陽收固斬之以振代固矩送以示固固斬振父子帥騎一千來降【帥讀曰率騎奇寄翻降戶江翻】矩復令固守洛陽鄭攀等相與拒王廙【廙羊至翻又逸職翻】衆心不壹散還横桑口【水經沔水東南逕江夏雲杜縣又東逕左桑周昭王溺死處也村老云百姓佐昭王喪事於此故曰佐桑左桑字失體耳又東謂之横桑言得昭王喪處也】欲入杜曾王敦遣武昌太守趙誘襄陽太守朱軌擊之攀等懼請降杜曾亦請擊第五猗於襄陽以自贖廙將赴荆州留長史劉浚鎮揚口壘【水經註龍陂水逕郢城東北流謂之揚水水北逕竟陵縣西又北注于沔曰揚口中夏口也】竟陵内史朱伺謂廙曰【伺相吏翻】曾猾賊也外示屈服欲誘官軍使西然後兼道襲揚口耳宜大部分【言當大為部分以備曾掩襲分扶問翻】未可便西廙性矜厲自用以伺為老怯遂西行曾等果還趨揚口【趨七喻翻】廙乃遣伺歸裁至壘即為曾所圍劉浚自守北門使伺守南門馬雋從曾來攻壘雋妻子先在壘中【馬雋本與鄭攀同距王廙】或欲皮其面以示之【皮面者剥其面皮】伺曰殺其妻子未能解圍但益其怒耳乃止曾攻陷北門伺被傷【被皮義翻】退入船開船底以出沈行五十步乃得免【沈持林翻潛行水底曰沈行】曾遣人說伺曰【說輸芮翻】馬雋德卿全其妻子今盡以卿家内外百口付雋雋已盡心收視卿可來也伺報曰吾年六十餘不能復與卿作賊【復扶又翻】吾死亦當南歸妻子付汝裁之乃就王廙於甑山病創而卒【甑山在竟陵界隋置甑山縣屬沔陽郡創初良翻】戊寅趙誘朱軌及陵江將軍黄峻與曾戰於女觀湖【水經註柞溪水出江陵縣北東注船官湖湖水又東北入女觀湖湖水又東入于揚水】誘等皆敗死曾乘勝逕造沔口【造七到翻】威震江沔王使豫章太守周訪擊之訪有衆八千進至沌陽【沈約曰沌陽縣江左立屬江夏郡水經沔水逕沌陽縣北又東逕林障故城北沌陽者沌水之陽也沈持林翻】曾鋭氣甚盛訪使將軍李恒督左甄許朝督右甄訪自領中軍曾先攻左右甄【楊正衡曰甄音堅戰陳有左拒右拒拒方陳也有左甄右甄甄左右翼也左右拒見於周鄭繻葛之戰左右甄之義見於楚穆王孟諸之田盂諸之田宋公為右盂鄭伯為左盂杜預注曰將獵張兩甄蓋晉人以左右翼為左右甄杜預取當時之言以釋左右盂也】訪於陣後射雉以安衆心【射而亦翻】令其衆曰一甄敗鳴三鼓兩甄敗鳴六鼓趙誘子胤將父餘兵屬左甄【將即亮翻】力戰敗而復合馳馬告訪訪怒叱令更進胤號哭還戰【號戶刀翻】自旦至申兩甄皆敗訪選精銳八百人自行酒飲之【飲於鴆翻】敕不得妄動聞鼓音乃進曾兵未至三十步訪親鳴鼓將士皆騰躍犇赴曾遂大潰殺千餘人訪夜追之諸將請待明日訪曰曾驍勇能戰【驍堅堯翻】向者彼勞我逸故克之宜及其衰乘之可滅也乃鼓行而進遂定漢沔曾走保武當【武當縣漢屬南陽郡晉屬順陽郡縣以武當山得名唐為均州武當郡杜佑曰郡城延岑所築】王廙始得至荆州訪以功遷梁州刺史屯襄陽【胡子序之敗梁州陷没故令訪領梁州而屯襄陽】 冬十月丁未琅邪王裒薨 十一月己酉朔日有食之 【考異曰帝紀天文志皆云十一月丙子日食按長歷十月十二月皆己卯朔是月己酉朔二十八日丙子晉書元帝紀十一月有甲子丁卯若丙子朔則甲子丁卯乃在十月又劉琨集是年三月癸未朔八月庚辰朔皆與長歷合今以為据】 丁卯以劉琨為侍中太尉 征南軍司戴邈上疏以為喪亂以來【喪息浪翻下久喪同】庠序隳廢議者或謂平世尚文遭亂尚武此言似之而實不然夫儒道深奥不可倉猝而成比天下平泰然後修之則廢墜已久矣【比必寐翻】又貴遊之子未必有斬將搴旗之才【將即亮翻搴抜取也】從軍征戌之役不及盛年使之講肄道義良可惜也【肄羊至翻習也】世道久喪禮俗日弊猶火之消膏莫之覺也今王業肇建萬物權輿【爾雅曰權輿始也】謂宜篤道崇儒以勵風化王從之始立太學 漢主聰出畋以愍帝行車騎將軍戎服執戟前導見者指之曰此故長安天子也聚而觀之故老有泣者太子粲言於聰曰昔周武王豈樂殺紂乎【樂音洛】正恐同惡相求為患故也今興兵聚衆者皆以子業為名不如早除之聰曰吾前殺庾珉輩【殺庾珉事見八十八卷建興元年】而民心猶如是吾未忍復殺也【復扶又翻】且小觀之十二月聰饗羣臣于光極殿使愍帝行酒洗爵已而更衣又使之執蓋晉臣夕夕涕泣有失聲者尚書郎隴西辛賓起抱帝大哭聰命引出斬之【使之執戟前導使之行酒洗爵使之執蓋所以屈辱之至此極矣戎狄狡計正以此觀晉舊臣及遺黎之心也更工衡翻】趙固與河内太守郭默侵漢河東至絳【絳縣故晉都也漢屬河東郡晉屬平陽郡劉昫曰唐絳州曲沃縣漢絳縣地】右司隸部民犇之者三萬餘人【聰分司隸為左右】騎兵將軍劉勲追擊之【騎奇寄翻】殺萬餘人固默引歸太子粲帥將軍劉雅生等步騎十萬屯小平津【帥讀曰率】固揚言曰要當生縛劉粲以贖天子粲表於聰曰子業若死民無所望則不為李矩趙固之用不攻而自滅矣戊戌愍帝遇害於平陽【年十八】粲遣雅生攻洛陽固犇陽城山【河南陽城縣有陽城山】 是歲王命課督農功二千石長吏以入穀多少為殿最【長知兩翻少詩沼翻殿丁練翻】諸軍各自佃作即以為稟【佃音田稟給也】 氐王楊茂搜卒長子難敵立與少子堅頭分領部曲【少詩照翻】難敵號左賢王屯下辨堅頭號右賢王屯河池【下辨河池二縣皆屬武都郡師古曰辨皮莧翻劉昫曰辨步莧翻下辨唐為成州同谷縣河池唐為武州盤隄縣】 河南王吐谷渾卒【吐谷渾史家傳讀吐從暾入聲谷音欲】吐谷渾者慕容廆之庶兄也父涉歸分戶一千七百以隸之及廆嗣位二部馬鬬廆遣使讓吐谷渾曰先公分建有别【廆戶罪翻别彼列翻】奈何不相遠異【遠異者言遠去以相别異】而令馬有鬭傷吐谷渾怒曰馬是六畜【六畜馬牛羊犬豕雞畜許六翻】鬬乃其常何至怒及於人欲遠别甚易恐後會為難耳今當去汝萬里之外遂帥其衆西徙【易以豉翻帥讀曰率】廆悔之遣其長史乙郍婁馮追謝之【郍與那同乙郍婁虜三字姓】吐谷渾曰先公嘗稱卜筮之言云吾二子皆當彊盛祚流後世我孽子也【孽魚列翻庶出為孽】理無並大今因馬而别殆天意乎遂不復還西傅隂山而居【復扶又翻傅讀曰附】屬永嘉之亂【屬之欲翻會也】因度隴而西據洮水之西極于白蘭地方數千里【沙州記曰洮水出嵹臺山東北流逕吐谷渾中又東北流入塞此洮西塞外洮水之西也即沙漒沓中之地白蘭山名羌所居也至唐時丁零羌居之左屬黨項右與多彌接杜佑曰白蘭羌之别種東北接吐谷渾西至叱利模徒南界郡鄂風俗物產與宕昌同】鮮卑謂兄為阿于廆追思之為之作阿于之歌吐谷有子六十人長子吐延嗣【為于偽翻長知兩翻】吐延長大有勇力羌胡皆畏之【吐谷渾事始此】<br />
<br />
  大興元年【是年三月方改元】春正月遼西公疾陸眷卒其子幼叔父涉復辰自立段匹磾自薊往犇喪段末柸宣言匹磾之來欲為簒也匹磾至右北平【劉昫曰唐薊州漁陽縣古右北平郡治所磾丁奚翻】涉復辰發兵拒之末柸乘虚襲涉復辰殺之并其子弟黨與自稱單于迎擊匹磾敗之【單音蟬敗補邁翻】匹磾走還薊【薊音計】 三月癸丑愍帝凶問至建康王斬縗居廬【縗倉回翻儀禮斬衰倚廬孟康曰倚廬倚牆至地為之無楣柱喪服大記父母之喪居倚廬不塗君為廬宫之大夫士䄠之既葬柱楣塗廬不於顯者君大夫士皆宫之正義曰居倚廬者謂於中門之外東牆下倚木為廬不塗者但以章夾障不塗之也宫之者謂廬外以帷障之如宫牆䄠之言袒也其廬袒露不帷障也既葬柱楣者既葬情殺故柱楣稍舉以納日光又以泥塗辟風寒不於顯者塗廬不塗廬外顯處君大夫士皆宫之者既葬故得皆宫之】百官請上尊號【上時掌翻】王不許紀瞻曰晉氏統絶於今二年陛下當承大業顧望宗室誰復與讓若光踐大位則神民有所憑依苟為逆天時違人事大勢一去不可復還【復扶又翻】今兩都燔蕩宗廟無主劉聰竊號於西北而陛下方高讓於東南此所謂揖讓而救火也王猶不許使殿中將軍韓績徹去御坐【殿中將軍屬二衛晉初置朝會宴饗則戎服直侍左右夜開諸城門則執白虎幡監之坐徂卧翻下帝坐同】瞻叱績曰帝坐上應列星【天文志帝坐在紫宫中】敢動者斬王為之改容【為于偽翻】奉朝請周嵩上疏曰【王為丞相以嵩為參軍及為晉王拜奉朝請晉志曰奉朝請者奉朝會請召而已】古之王者義全而後取讓成而後得是以享世長久重光萬載也【重直龍翻載子亥翻】今梓宫未返舊京未清義夫泣血士女遑遑宜開延嘉謀訓卒厲兵先雪社稷大恥副四海之心則神器將安適哉由是忤旨出為新安太守【孫權分丹陽立新都郡武帝太康元年改名新安郡劉昫曰新安郡唐之歙州忤五故翻】又坐怨望抵罪嵩顗之弟也【顗魚豈翻】丙辰王即皇帝位百官皆剖列帝命王導升御床共坐導固辭曰若太陽下同萬物蒼生何由仰照帝乃止大赦改元文武增位二等帝欲賜諸吏投刺勸進者加位一等民投刺者皆除吏凡二十餘萬人【毛晃曰書姓名於奏白曰刺】散騎常侍熊遠曰陛下應天繼統率土歸戴豈獨近者情重遠者情輕不若依漢法徧賜天下爵於恩為普【漢自惠帝嗣位賜民爵一級有官秩者以歲數為差其後諸帝初即位率賜民爵一級】且可以息檢覈之煩塞巧偽之端也【塞悉則翻】帝不從 庚午立王太子紹為皇太子太子仁孝喜文辭善武藝好賢禮士【喜許記翻好呼到翻】容受規諫與庾亮温嶠等為布衣之交亮風格峻整善談老莊帝器重之聘亮妹為太子妃帝以賀循行太子太傅周顗為少傅庾亮以中書郎侍講東宫帝好刑名家以韓非書賜太子庾亮諫曰申韓刻薄傷化不足留聖心太子納之 帝復遣使授慕容廆龍驤將軍大單于昌黎公廆辭公爵不受【廆辭公爵不受外為謙讓其志不肯欝欝於昌黎也復扶又翻使疏吏翻驤思將翻】廆以游邃為龍驤長史劉翔為主簿命邃創定府朝儀灋【朝直遥翻】裴嶷言於廆曰晉室衰微介居江表【介獨也嶷魚力翻】威德不能及遠中原之亂非明公不能拯也【拯救也】今諸部雖各擁兵然皆頑愚相聚宜以漸并取以為西討之資【西討謂自遼東進兵西入中州也】廆曰君言大非孤所及也然君中朝名德不以孤僻陋而教誨之是天以君賜孤而祐其國也乃以嶷為長史委以軍國之謀諸部弱小者稍稍擊取之 李矩使郭默郭誦救趙固屯于洛汭【水經洛水東北過鞏縣東又北入于河夏五子徯太康于洛汭即其地】誦潛遣其將耿稚等夜濟河襲漢營【據李矩傳時粲營于孟津北岸】漢貝丘王翼光覘知之【覘丑亷翻又丑艶翻】以告太子粲請為之備粲曰彼聞趙固之敗自保不暇安敢來此邪毋為驚動將士俄而稚等奄至十道進攻粲衆驚潰死傷大半粲走保陽鄉【陽鄉蓋春秋陽樊之地在汲郡修武縣界】稚等據其營獲器械軍資不可勝數【勝音升】及旦粲見稚等兵少更與劉雅生收餘衆攻之漢主聰使太尉范隆帥騎助之【少詩沼翻帥讀曰率騎奇寄翻】與稚等相持苦戰二十餘日不能下李矩進兵救之漢兵臨河拒守矩兵不得濟稚等殺其所獲牛馬焚其軍資突圍犇虎牢【河南成臯縣鄭之虎牢也穆天子傳曰七萃之士生捕虎即獻天子天子畜之東虢號曰虎牢其後劉裕復中原置河南四鎮虎牢其一也】詔以矩都督河南三郡諸軍事【三郡河南滎陽弘農也】 漢螽斯則百堂災【螽斯則百堂取螽斯子孫衆多思齊則百斯男之義】燒殺漢主聰之子會稽王康等二十一人【會工外翻】 聰以其子濟南王驥為大將軍都督中外諸軍事録尚書齊王勱為大司徒【濟子禮翻勱音邁】焦嵩陳安舉兵逼上邽相國保遣使告急於張寔寔<br />
<br />
  遣金城太守竇濤督步騎二萬赴之軍至新陽【晉志新陽縣屬天水郡何承天曰魏立水經註渭水過冀縣又東出岑峽入新陽川新陽縣蓋置于此】聞愍帝崩保謀稱尊號破羌都尉張詵言於寔曰南陽王國之疏屬忘其大恥而亟欲自尊【君父皆死於賊手保之大恥也】必不能成功晉王近親且有名德當帥天下以奉之【保宣帝之從曾孫故曰疏屬帝宣帝之曾孫故曰近親帥讀曰率】寔從之遣牙門蔡忠奉表詣建康比至帝已即位【比必寐翻】寔不用江東年號猶稱建興【河西張氏用建興年號歷九世四十九年至孝宗升平五年張天錫乃奉升平年號】 夏四月丁丑朔日有食之 加王敦江州牧王導驃騎大將軍開府儀同三司導遣八部從事行揚州郡國【揚州時統丹陽會稽吳吳興宣城東陽臨海新安八郡故分遣部從事八人行下孟翻】還同時俱見諸從事各言二千石官長得失【長知兩翻】獨顧和無言導問之和曰明公作輔寧使網漏吞舟【漢書刑法志曰漢興之初雖有約法三章網漏吞舟之魚師古曰言疏濶吞舟謂大魚也】何緣採聽風聞以察察為政邪導咨嗟稱善和榮之族子也 成丞相范長生卒成主雄以長生子侍中賁為丞相長生博學多藝能年近百歲蜀人奉之如神【近其靳翻】 漢中常侍王沈養女有美色【沈持林翻】漢主聰立以為左皇后尚書令王鑒中書監崔懿之中書令曹恂諫曰臣聞王者立后比德乾坤【乾父道也君比德焉坤母道也后比德焉】生承宗廟没配后土必擇世德名宗幽閑令淑【詩關雎窈窕淑女毛註云窈窕幽閑也淑善也令亦善也】乃副四海之望稱神祇之心【稱尺證翻】孝成帝以趙飛鷰為后使繼嗣絶滅社稷為墟此前鑑也【事見三十二卷漢哀帝建平元年】自麟嘉以來【愍帝建興之四年漢麟嘉之元年】中宫之位不以德舉借使沈之弟女刑餘小醜猶不可以塵汙椒房【汙烏路翻】况其家婢邪六宫妃嬪皆公子公孫奈何一旦以婢主之臣恐非國家之福也聰大怒使中常侍宣懷謂太子粲曰鑒等小子狂言侮慢無復君臣上下之禮其速考實於是收鑒等送市皆斬之金紫光禄大夫王延馳將入諫門者弗通鑒等臨刑王沈以杖叩之曰庸奴復能為惡乎乃公何與汝事【與讀曰預】鑒瞋目叱之曰豎子滅大漢者正坐汝鼠輩與靳凖耳【瞋七人翻靳居惞翻】要當訴汝於先帝取汝於地下治之【汝直之翻】凖謂鑒曰吾受詔收君有何不善君言漢滅由吾也鑒曰汝殺皇太弟使主上獲不友之名國家畜養汝輩何得不滅【畜許六翻】懿之謂凖曰汝心如梟獍【梟食母破獍食父破獍如貙而虎身身一作眼】必為國患汝既食人人亦當食汝聰又立宣懷養女為中皇后 司徒荀組在許昌逼於石勒帥其屬數百人渡江【帥讀曰率】詔組與太保西陽王羕並録尚書事 段匹磾之犇疾陸眷喪也劉琨使其世子羣送之匹磾敗羣為段末柸所得末柸厚禮之許以琨為幽州刺史欲與之襲匹磾密遣使齎羣書請琨為内應為匹磾邏騎所得【磾丁奚翻邏郎佐翻】時琨别屯征北小城不知也【征北小城蓋征北將軍所治】來見匹磾匹磾以羣書示琨曰意亦不疑公是以白公耳琨曰與公同盟庶雪國家之恥若兒書密達亦終不以一子之故負公而忘義也匹磾雅重琨【雅素也】初無害琨意將聽還屯其弟叔軍謂匹磾曰我胡夷耳所以能服晉人者畏吾衆也今我骨肉乖離【謂與末柸相攻也】是其良圖之日若有奉琨以起吾族盡矣匹磾遂留琨琨之庶長子遵懼誅與琨左長史楊橋等閉門自守【長知兩翻】匹磾攻抜之代郡太守辟閭嵩【姓譜衛文公支孫居楚丘營辟閭里因為辟閭氏】後將軍韓據復潛謀襲匹磾事泄匹磾執嵩據及其徒黨悉誅之五月癸丑匹磾稱詔收琨縊殺之并殺其子姪四人【縊於賜翻又於計翻】琨從事中郎盧諶崔悦等帥琨餘衆犇遼西【諶氏壬翻帥讀曰率下同】依段末柸奉劉羣為主將佐多犇石勒悦林之曾孫也【崔林仕魏位至司空】朝廷以匹磾尚彊冀其能平河朔乃不為琨舉哀【為于偽翻下同】温嶠表琨盡忠帝室家破身亡宜在褒恤盧諶崔悦因末柸使者亦上表為琨訟寃後數歲乃贈琨太尉侍中諡曰愍於是夷晉以琨死皆不附匹磾末柸遣其弟攻匹磾匹磾帥其衆數千將犇邵續勒將石越邀之於鹽山【鹽山在勃海高城縣隋改高城曰鹽山縣宋白曰鹽山在縣東南八十里匹磾與琨結盟同奬晉室既殺琨而匹磾之勢亦衰終為石勒禽矣】大敗之【敗補邁翻】匹磾復還保薊末柸自稱幽州刺史初温嶠為劉琨奉表詣建康其母崔氏固止之嶠絶裾而去既至屢求返命朝廷不許會琨死除散騎侍郎嶠聞母亡阻亂不得犇喪臨葬固讓不拜苦請北歸詔曰凡行禮者當使理可經通【經常也】今桀逆未梟【梟堅堯翻】諸軍奉迎梓宫猶未得進嶠以一身於何濟其私難【難乃旦翻】而不從王命邪嶠不得已受拜 初曹嶷既據青州乃叛漢來降【謂遣使詣建康奉表勸進也嶷魚力翻】又以建康懸遠勢援不接復與石勒相結勒授嶷東州大將軍青州牧封琅邪公【曹嶷反側二國之間終為人禽而已矣復扶又翻】 六月甲申以刁協為尚書令荀崧為左僕射協性剛悍與物多忤【悍侯旰翻又下罕翻忤五故翻】與侍中劉隗俱為帝所寵任【隗五罪翻】欲矯時弊每崇上抑下排沮豪彊【沮在呂翻】故為王氏所疾諸刻碎之政皆云隗協所建協又使酒放肆侵毁公卿見者皆側目憚之【為刁協見殺張本】 戊戌封皇子晞為武陵王 劉虎自朔方侵拓抜欝律西部【虎徙朔方見八十七卷懷帝永嘉四年】秋七月欝律擊虎大破之虎走出塞從弟路孤帥其部落降于欝律【帥讀曰率降戶江翻】於是欝律西取烏孫故地東兼勿吉以西【唐書北狄列傳曰黑水靺羯居肅慎地亦曰挹婁元魏謂之勿吉通鑑蓋因魏收魏書書之欝律所取者勿吉以西之地未能兼勿吉也徒河慕容令支段氏及宇文部高句麗亦非欝律所能制伏】士馬精彊雄於北方 漢主聰寢疾徵大司馬曜為丞相石勒為大將軍皆録尚書事受遺詔輔政曜勒固辭乃以曜為丞相領雍州牧【雍於用翻】勒為大將軍領幽冀二州牧勒辭不受以上洛王景為太宰濟南王驥為大司馬【濟子禮翻】昌國公顗為太師【顗魚豈翻】朱紀為太傅呼延宴為太保並録尚書事范隆守尚書令儀同三司靳凖為大司空領司隸校尉皆迭决尚書奏事癸亥聰卒甲子太子粲即位【粲字士光】尊皇后靳氏為皇太后樊氏號弘道皇后武氏號弘德皇后王氏號弘孝皇后立其妻靳氏為皇后子元公為太子大赦改元漢昌葬聰於宣光陵諡曰昭武皇帝廟號烈宗靳太后等皆年未盈二十粲多行無禮無復哀戚靳凖隂有異志私謂粲曰如聞諸公欲行伊霍之事先誅太保及臣以大司馬統萬機陛下宜早圖之粲不從凖懼復使二靳氏言之【二靳氏聰后與粲后靳居惞翻復扶又翻】粲乃從之收其太宰景大司馬驥驥母弟車騎大將軍吳王逞太師顗大司徒齊王勱皆殺之【顗魚豈翻勱音邁】朱紀范隆犇長安【犇劉曜也】八月粲治兵於上林謀討石勒【蓋起上林苑於下陽治直之翻】以丞相曜為相國都督中外諸軍事仍鎮長安靳凖為大將軍録尚書事粲常遊宴後宫軍國之事一决於凖凖矯詔以從弟明為車騎將軍康為衛將軍【從才用翻】凖將作亂謀於王延延弗從馳將告之【將以凖謀告粲】遇靳康劫延以歸凖遂勒兵升光極殿使甲士執粲數而殺之【數所具翻】諡曰隱帝劉氏男女無少長皆斬東市【少詩照翻長知兩翻】發永光宣光二陵【淵墓號永光陵】斬聰屍焚其宗廟凖自號大將軍漢天王稱制置百官謂安定胡嵩曰自古無胡人為天子者今以傳國璽付汝還如晉家【洛陽之陷傳國璽遷于平陽如往也璽斯氏翻】嵩不敢受凖怒殺之遣使告司州刺史李矩曰劉淵屠各小醜【屠直於翻】因晉之亂矯稱天命使二帝幽没輒率衆扶侍梓宫請以上聞矩馳表于帝帝遣太常韓胤等奉迎梓宫漢尚書北宫純等招集晉人堡於東宫靳康攻滅之【北宫純降漢見八十七卷懷帝永嘉五年】凖欲以王延為左光禄大夫延罵曰屠各逆奴何不速殺我以吾左目置西陽門觀相國之入也【以劉曜將自西進兵也】右目置建春門觀大將軍之入也【以石勒將自東進兵也】凖殺之相國曜聞亂自長安赴之石勒帥精鋭五萬以討凖據襄陵北原【帥讀曰率襄陽縣漢屬河東郡晉屬平陽郡師古曰晉襄公之陵因以名縣據水經註襄陵在平陽東南】凖數挑戰【數所角翻挑徒了翻】勒堅壁以挫之冬十月曜至赤壁【水經註河東皮氏縣西北有赤石川】大保呼延晏等自平陽歸之與太傅朱紀等共上尊號【上時掌翻】曜即皇帝位【曜字永明淵之族子】大赦惟靳凖一門不在赦例改元元光初以朱紀領司徒呼延晏領司空太尉范隆以下悉復本位以石勒為大司馬大將軍加九錫增封十郡進爵為趙公勒進攻凖於平陽巴及羌羯降者十餘萬落【巴巴氏也魏武平漢中遷巴氏于關中其後種類滋蔓河東平陽皆有之羯居謁翻】勒皆徙之於所部郡縣漢主曜使征北將軍劉雅鎮北將軍劉策屯汾隂【汾隂縣漢屬河東郡晉省】與勒共討凖 十一月乙卯日夜出高三丈【高居奥翻】 詔以王敦為荆州牧加陶侃都督交州諸軍事敦固辭州牧乃聽為刺史 庚申詔羣公卿士各陳得失御史中丞熊遠上疏以為胡賊猾夏【孔安國日猾亂也夏華夏夏戶雅翻】梓宫未返而不能遣軍進討一失也羣官不以讐賊未報為恥務在調戲酒食而已二失也【諧謔以相調戲】選官用人不料實德惟在白望不求才幹惟事請託當官者以治事為俗吏【治直之翻】奉灋為苛刻盡禮為諂諛從容為高妙【從千容翻】放蕩為達士驕蹇為簡雅三失也世之所惡者陸沈泥滓【惡烏路翻司馬彪曰陸沈謂無水而沈之沈持林翻】時之所善者翺翔雲霄是以萬機未整風俗偽薄朝廷羣司以從順為善相違見貶安得朝有辨爭之臣士無禄仕之志乎【朝直遥翻】古之取士敷奏以言【舜典曰敷奏以言孔安國註曰敷陳奏進也各使陳進治體之言】今光禄不試甚違古義【此即謂秀孝不試而署吏】又舉賢不出世族用法不及權貴是以才不濟務姦無所懲若此道不改求以救亂難矣先是帝以離亂之際欲慰悅人心州郡秀孝至者不試普皆署吏【秀孝謂州郡所舉秀才及孝亷先悉薦翻】尚書陳頵亦上言宜漸循舊制試以經策【晉初秀孝以經策中第者若華譚之類是也頵於倫翻又居筠翻】帝從之仍詔不中科者刺史太守免官【欲罪舉主也中竹仲翻】於是秀孝皆不敢行其有到者亦皆託疾比三年無就試者【比必寐翻】帝欲特除孝亷已到者官尚書郎孔坦奏議以為近郡懼累君父皆不敢行【累力瑞翻君父謂刺史太守】遠近冀於不試冒昧來赴今若偏加除署是為謹身奉灋者失分僥倖投射者得官【分扶問翻投射謂投機而射利也】頹風蕩教恐從此始不若一切罷歸而為之延期【為于偽翻延遠也】使得就學則灋均而令信矣帝從之聽孝亷申至七年乃試【申寛展也】坦愉之從子也【從才用翻】 靳凖使侍中卜泰送乘輿服御請和於石勒【乘繩證翻】勒囚泰送於漢主曜曜謂泰曰先帝末年實亂大倫【先帝謂粲也亂倫謂烝其諸母】司空行伊霍之權使朕及此其功大矣若早迎大駕者當悉以政事相委况免死乎卿為朕入城具宣此意【為于偽翻】泰還平陽凖自以殺曜母兄沈吟未從【曜母胡氏為凖所殺兄則史失其名沈吟猶豫不决之意沈持林翻】十二月左右車騎將軍喬泰王騰衛將軍靳康等相與殺凖推尚書令靳明為主遣卜泰奉傳國六璽降漢【降戶江翻】石勒大怒進軍攻明明出戰大敗乃嬰城固守丁丑封皇子煥為琅邪王煥鄭夫人之子生二年矣<br />
<br />
  帝愛之以其疾篤故王之己卯薨帝以成人之禮葬之備吉凶儀服營起園陵功費甚廣琅邪國右常侍會稽孫霄【晉志王國置左右常侍各一人】上疏諫曰古者凶荒殺禮【殺所戒翻降也減也】况今海内喪亂【喪息浪翻】憲章舊制猶宜節省而禮典所無顧崇飾如是乎【葬無服之殤以成人之禮古典所無也】竭已罷之民營無益之事【罷讀曰疲】殫已困之財修無用之費此臣之所不安也帝不從 彭城内史周撫殺沛國内史周默以其衆降石勒詔下邳内史劉遐領彭城内史與徐州刺史蔡豹泰山太守徐龕共討之豹質之玄孫也【蔡質漢人蔡邕之叔父龕口含翻】 石虎帥幽冀之兵會石勒攻平陽靳明屢敗遣使求救於漢漢主曜使劉雅劉策迎之明帥平陽士女萬五千人犇漢【帥讀曰率】曜西屯粟邑【粟邑縣屬馮翊郡】收靳氏男女無少長皆斬之【少詩照翻長知兩翻】曜迎其母胡氏之喪於平陽葬于粟邑號曰陽陵諡曰宣明皇太后石勒焚平陽宫室使裴憲石會修永光宣光二陵收漢主粲已下百餘口葬之置戌而歸 成梁州刺史李鳳數有功【數所角翻】成主雄兄子稚在晉壽疾之【晉壽縣屬梓潼郡何承天曰晉惠帝立晉夀縣沈約曰案晉起居注武帝太康元年改梓潼之漢壽曰晉壽漢夀之名疑是蜀立云惠帝立非也】鳳以巴西叛雄自至涪使太傅驤討鳳斬之以李壽為前將軍督巴西軍事【涪音浮】<br />
<br />
  資治通鑑卷九十  <br>
   </div> 

<script src="/search/ajaxskft.js"> </script>
 <div class="clear"></div>
<br>
<br>
 <!-- a.d-->

 <!--
<div class="info_share">
</div> 
-->
 <!--info_share--></div>   <!-- end info_content-->
  </div> <!-- end l-->

<div class="r">   <!--r-->



<div class="sidebar"  style="margin-bottom:2px;">

 
<div class="sidebar_title">工具类大全</div>
<div class="sidebar_info">
<strong><a href="http://www.guoxuedashi.com/lsditu/" target="_blank">历史地图</a></strong>  
<a href="http://www.880114.com/" target="_blank">英语宝典</a>  
<a href="http://www.guoxuedashi.com/13jing/" target="_blank">十三经检索</a> 
<br><strong><a href="http://www.guoxuedashi.com/gjtsjc/" target="_blank">古今图书集成</a></strong> 
<a href="http://www.guoxuedashi.com/duilian/" target="_blank">对联大全</a> <strong><a href="http://www.guoxuedashi.com/xiangxingzi/" target="_blank">象形文字典</a></strong> 

<br><a href="http://www.guoxuedashi.com/zixing/yanbian/">字形演变</a>  <strong><a href="http://www.guoxuemi.com/hafo/" target="_blank">哈佛燕京中文善本特藏</a></strong>
<br><strong><a href="http://www.guoxuedashi.com/csfz/" target="_blank">丛书&方志检索器</a></strong> <a href="http://www.guoxuedashi.com/yqjyy/" target="_blank">一切经音义</a>  

<br><strong><a href="http://www.guoxuedashi.com/jiapu/" target="_blank">家谱族谱查询</a></strong>  <strong><a href="http://shufa.guoxuedashi.com/sfzitie/" target="_blank">书法字帖欣赏</a></strong> 
<br>

</div>
</div>


<div class="sidebar" style="margin-bottom:0px;">

<font style="font-size:22px;line-height:32px">QQ交流群9:489193090</font>


<div class="sidebar_title">手机APP 扫描或点击</div>
<div class="sidebar_info">
<table>
<tr>
	<td width=160><a href="http://m.guoxuedashi.com/app/" target="_blank"><img src="/img/gxds-sj.png" width="140"  border="0" alt="国学大师手机版"></a></td>
	<td>
<a href="http://www.guoxuedashi.com/download/" target="_blank">app软件下载专区</a><br>
<a href="http://www.guoxuedashi.com/download/gxds.php" target="_blank">《国学大师》下载</a><br>
<a href="http://www.guoxuedashi.com/download/kxzd.php" target="_blank">《汉字宝典》下载</a><br>
<a href="http://www.guoxuedashi.com/download/scqbd.php" target="_blank">《诗词曲宝典》下载</a><br>
<a href="http://www.guoxuedashi.com/SiKuQuanShu/skqs.php" target="_blank">《四库全书》下载</a><br>
</td>
</tr>
</table>

</div>
</div>


<div class="sidebar2">
<center>


</center>
</div>

<div class="sidebar"  style="margin-bottom:2px;">
<div class="sidebar_title">网站使用教程</div>
<div class="sidebar_info">
<a href="http://www.guoxuedashi.com/help/gjsearch.php" target="_blank">如何在国学大师网下载古籍?</a><br>
<a href="http://www.guoxuedashi.com/zidian/bujian/bjjc.php" target="_blank">如何使用部件查字法快速查字?</a><br>
<a href="http://www.guoxuedashi.com/search/sjc.php" target="_blank">如何在指定的书籍中全文检索?</a><br>
<a href="http://www.guoxuedashi.com/search/skjc.php" target="_blank">如何找到一句话在《四库全书》哪一页?</a><br>
</div>
</div>


<div class="sidebar">
<div class="sidebar_title">热门书籍</div>
<div class="sidebar_info">
<a href="/so.php?sokey=%E8%B5%84%E6%B2%BB%E9%80%9A%E9%89%B4&kt=1">资治通鉴</a> <a href="/24shi/"><strong>二十四史</strong></a>&nbsp; <a href="/a2694/">野史</a>&nbsp; <a href="/SiKuQuanShu/"><strong>四库全书</strong></a>&nbsp;<a href="http://www.guoxuedashi.com/SiKuQuanShu/fanti/">繁体</a>
<br><a href="/so.php?sokey=%E7%BA%A2%E6%A5%BC%E6%A2%A6&kt=1">红楼梦</a> <a href="/a/1858x/">三国演义</a> <a href="/a/1038k/">水浒传</a> <a href="/a/1046t/">西游记</a> <a href="/a/1914o/">封神演义</a>
<br>
<a href="http://www.guoxuedashi.com/so.php?sokeygx=%E4%B8%87%E6%9C%89%E6%96%87%E5%BA%93&submit=&kt=1">万有文库</a> <a href="/a/780t/">古文观止</a> <a href="/a/1024l/">文心雕龙</a> <a href="/a/1704n/">全唐诗</a> <a href="/a/1705h/">全宋词</a>
<br><a href="http://www.guoxuedashi.com/so.php?sokeygx=%E7%99%BE%E8%A1%B2%E6%9C%AC%E4%BA%8C%E5%8D%81%E5%9B%9B%E5%8F%B2&submit=&kt=1"><strong>百衲本二十四史</strong></a>  <a href="http://www.guoxuedashi.com/so.php?sokeygx=%E5%8F%A4%E4%BB%8A%E5%9B%BE%E4%B9%A6%E9%9B%86%E6%88%90&submit=&kt=1"><strong>古今图书集成</strong></a>
<br>

<a href="http://www.guoxuedashi.com/so.php?sokeygx=%E4%B8%9B%E4%B9%A6%E9%9B%86%E6%88%90&submit=&kt=1">丛书集成</a> 
<a href="http://www.guoxuedashi.com/so.php?sokeygx=%E5%9B%9B%E9%83%A8%E4%B8%9B%E5%88%8A&submit=&kt=1"><strong>四部丛刊</strong></a>  
<a href="http://www.guoxuedashi.com/so.php?sokeygx=%E8%AF%B4%E6%96%87%E8%A7%A3%E5%AD%97&submit=&kt=1">說文解字</a> <a href="http://www.guoxuedashi.com/so.php?sokeygx=%E5%85%A8%E4%B8%8A%E5%8F%A4&submit=&kt=1">三国六朝文</a>
<br><a href="http://www.guoxuedashi.com/so.php?sokeytm=%E6%97%A5%E6%9C%AC%E5%86%85%E9%98%81%E6%96%87%E5%BA%93&submit=&kt=1"><strong>日本内阁文库</strong></a> <a href="http://www.guoxuedashi.com/so.php?sokeytm=%E5%9B%BD%E5%9B%BE%E6%96%B9%E5%BF%97%E5%90%88%E9%9B%86&ka=100&submit=">国图方志合集</a> <a href="http://www.guoxuedashi.com/so.php?sokeytm=%E5%90%84%E5%9C%B0%E6%96%B9%E5%BF%97&submit=&kt=1"><strong>各地方志</strong></a>

</div>
</div>


<div class="sidebar2">
<center>

</center>
</div>
<div class="sidebar greenbar">
<div class="sidebar_title green">四库全书</div>
<div class="sidebar_info">

《四库全书》是中国古代最大的丛书,编撰于乾隆年间,由纪昀等360多位高官、学者编撰,3800多人抄写,费时十三年编成。丛书分经、史、子、集四部,故名四库。共有3500多种书,7.9万卷,3.6万册,约8亿字,基本上囊括了古代所有图书,故称“全书”。<a href="http://www.guoxuedashi.com/SiKuQuanShu/">详细>>
</a>

</div> 
</div>

</div>  <!--end r-->

</div>
<!-- 内容区END --> 

<!-- 页脚开始 -->
<div class="shh">

</div>

<div class="w1180" style="margin-top:8px;">
<center><script src="http://www.guoxuedashi.com/img/plus.php?id=3"></script></center>
</div>
<div class="w1180 foot">
<a href="/b/thanks.php">特别致谢</a> | <a href="javascript:window.external.AddFavorite(document.location.href,document.title);">收藏本站</a> | <a href="#">欢迎投稿</a> | <a href="http://www.guoxuedashi.com/forum/">意见建议</a> | <a href="http://www.guoxuemi.com/">国学迷</a> | <a href="http://www.shuowen.net/">说文网</a><script language="javascript" type="text/javascript" src="https://js.users.51.la/17753172.js"></script><br />
  Copyright &copy; 国学大师 古典图书集成 All Rights Reserved.<br>
  
  <span style="font-size:14px">免责声明:本站非营利性站点,以方便网友为主,仅供学习研究。<br>内容由热心网友提供和网上收集,不保留版权。若侵犯了您的权益,来信即刪。scp168@qq.com</span>
  <br />
ICP证:<a href="http://www.beian.miit.gov.cn/" target="_blank">鲁ICP备19060063号</a></div>
<!-- 页脚END --> 
<script src="http://www.guoxuedashi.com/img/plus.php?id=22"></script>
<script src="http://www.guoxuedashi.com/img/tongji.js"></script>

</body>
</html>
