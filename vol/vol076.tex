<!DOCTYPE html PUBLIC "-//W3C//DTD XHTML 1.0 Transitional//EN" "http://www.w3.org/TR/xhtml1/DTD/xhtml1-transitional.dtd">
<html xmlns="http://www.w3.org/1999/xhtml">
<head>
<meta http-equiv="Content-Type" content="text/html; charset=utf-8" />
<meta http-equiv="X-UA-Compatible" content="IE=Edge,chrome=1">
<title>資治通鑒_77-資治通鑑卷七十六_77-資治通鑑卷七十六</title>
<meta name="Keywords" content="資治通鑒_77-資治通鑑卷七十六_77-資治通鑑卷七十六">
<meta name="Description" content="資治通鑒_77-資治通鑑卷七十六_77-資治通鑑卷七十六">
<meta http-equiv="Cache-Control" content="no-transform" />
<meta http-equiv="Cache-Control" content="no-siteapp" />
<link href="/img/style.css" rel="stylesheet" type="text/css" />
<script src="/img/m.js?2020"></script> 
</head>
<body>
 <div class="ClassNavi">
<a  href="/24shi/">二十四史</a> | <a href="/SiKuQuanShu/">四库全书</a> | <a href="http://www.guoxuedashi.com/gjtsjc/"><font  color="#FF0000">古今图书集成</font></a> | <a href="/renwu/">历史人物</a> | <a href="/ShuoWenJieZi/"><font  color="#FF0000">说文解字</a></font> | <a href="/chengyu/">成语词典</a> | <a  target="_blank"  href="http://www.guoxuedashi.com/jgwhj/"><font  color="#FF0000">甲骨文合集</font></a> | <a href="/yzjwjc/"><font  color="#FF0000">殷周金文集成</font></a> | <a href="/xiangxingzi/"><font color="#0000FF">象形字典</font></a> | <a href="/13jing/"><font  color="#FF0000">十三经索引</font></a> | <a href="/zixing/"><font  color="#FF0000">字体转换器</font></a> | <a href="/zidian/xz/"><font color="#0000FF">篆书识别</font></a> | <a href="/jinfanyi/">近义反义词</a> | <a href="/duilian/">对联大全</a> | <a href="/jiapu/"><font  color="#0000FF">家谱族谱查询</font></a> | <a href="http://www.guoxuemi.com/hafo/" target="_blank" ><font color="#FF0000">哈佛古籍</font></a> 
</div>

 <!-- 头部导航开始 -->
<div class="w1180 head clearfix">
  <div class="head_logo l"><a title="国学大师官网" href="http://www.guoxuedashi.com" target="_blank"></a></div>
  <div class="head_sr l">
  <div id="head1">
  
  <a href="http://www.guoxuedashi.com/zidian/bujian/" target="_blank" ><img src="http://www.guoxuedashi.com/img/top1.gif" width="88" height="60" border="0" title="部件查字,支持20万汉字"></a>


<a href="http://www.guoxuedashi.com/help/yingpan.php" target="_blank"><img src="http://www.guoxuedashi.com/img/top230.gif" width="600" height="62" border="0" ></a>


  </div>
  <div id="head3"><a href="javascript:" onClick="javascript:window.external.AddFavorite(window.location.href,document.title);">添加收藏</a>
  <br><a href="/help/setie.php">搜索引擎</a>
  <br><a href="/help/zanzhu.php">赞助本站</a></div>
  <div id="head2">
 <a href="http://www.guoxuemi.com/" target="_blank"><img src="http://www.guoxuedashi.com/img/guoxuemi.gif" width="95" height="62" border="0" style="margin-left:2px;" title="国学迷"></a>
  

  </div>
</div>
  <div class="clear"></div>
  <div class="head_nav">
  <p><a href="/">首页</a> | <a href="/ShuKu/">国学书库</a> | <a href="/guji/">影印古籍</a> | <a href="/shici/">诗词宝典</a> | <a   href="/SiKuQuanShu/gxjx.php">精选</a> <b>|</b> <a href="/zidian/">汉语字典</a> | <a href="/hydcd/">汉语词典</a> | <a href="http://www.guoxuedashi.com/zidian/bujian/"><font  color="#CC0066">部件查字</font></a> | <a href="http://www.sfds.cn/"><font  color="#CC0066">书法大师</font></a> | <a href="/jgwhj/">甲骨文</a> <b>|</b> <a href="/b/4/"><font  color="#CC0066">解密</font></a> | <a href="/renwu/">历史人物</a> | <a href="/diangu/">历史典故</a> | <a href="/xingshi/">姓氏</a> | <a href="/minzu/">民族</a> <b>|</b> <a href="/mz/"><font  color="#CC0066">世界名著</font></a> | <a href="/download/">软件下载</a>
</p>
<p><a href="/b/"><font  color="#CC0066">历史</font></a> | <a href="http://skqs.guoxuedashi.com/" target="_blank">四库全书</a> |  <a href="http://www.guoxuedashi.com/search/" target="_blank"><font  color="#CC0066">全文检索</font></a> | <a href="http://www.guoxuedashi.com/shumu/">古籍书目</a> | <a   href="/24shi/">正史</a> <b>|</b> <a href="/chengyu/">成语词典</a> | <a href="/kangxi/" title="康熙字典">康熙字典</a> | <a href="/ShuoWenJieZi/">说文解字</a> | <a href="/zixing/yanbian/">字形演变</a> | <a href="/yzjwjc/">金 文</a> <b>|</b>  <a href="/shijian/nian-hao/">年号</a> | <a href="/diming/">历史地名</a> | <a href="/shijian/">历史事件</a> | <a href="/guanzhi/">官职</a> | <a href="/lishi/">知识</a> <b>|</b> <a href="/zhongyi/">中医中药</a> | <a href="http://www.guoxuedashi.com/forum/">留言反馈</a>
</p>
  </div>
</div>
<!-- 头部导航END --> 
<!-- 内容区开始 --> 
<div class="w1180 clearfix">
  <div class="info l">
   
<div class="clearfix" style="background:#f5faff;">
<script src='http://www.guoxuedashi.com/img/headersou.js'></script>

</div>
  <div class="info_tree"><a href="http://www.guoxuedashi.com">首页</a> > <a href="/SiKuQuanShu/fanti/">四库全书</a>
 > <h1>资治通鉴</h1> <!--         下载:【右键另存为】即可 --></div>
  <div class="info_content zj clearfix">
  
<div class="info_txt clearfix" id="show">
<center style="font-size:24px;">77-資治通鑑卷七十六</center>
    資治通鑑卷七十六    宋 司馬光 撰<br />
<br />
  胡三省 音註<br />
<br />
  魏紀八【起昭陽作噩盡旃蒙大淵獻凡三年】<br />
<br />
  邵陵厲公下<br />
<br />
  嘉平五年春正月朔蜀大將軍費禕與諸將大會於漢夀郭循在坐【費父沸翻坐徂卧翻蜀先主改葭萌為漢夀】禕歡飲沈醉【循當作修下同沈持林翻】循起刺禕殺之【刺七亦翻】禕資性汎愛【汎孚梵翻廣也言無所不愛也】不疑於人越嶲太守張嶷【嶲音髓嶷魚力翻】嘗以書戒之曰昔岑彭率師來歙杖節咸見害於刺客【岑彭來歙事見四十二卷漢光武建武十一年歙許及翻】今明將軍位尊權重待信新附太過宜鑒前事少以為警【少詩沼翻】禕不從故及禍詔追封郭循為長樂鄉侯【樂音洛】使其子襲爵 王昶毋丘儉聞東軍敗【時三道伐吳東關最在東故曰東軍昶丑兩翻】各燒屯走朝議欲貶黜諸將【朝直遥翻下同】大將軍師曰我不聽公休【諸葛誕字公休】以至於此此我過也諸將何罪悉宥之師弟安東將軍昭時為監軍唯削昭爵而已【監工銜翻】以諸葛誕為鎮南將軍都督豫州母丘儉為鎮東將軍都督揚州是歲雍州刺史陳泰求敕并州并力討胡【雍於用翻】師從之未集而新興鴈門二郡胡以遠役遂驚反【雍州在并州西南而鴈門新興二郡并州北鄙也其道里相去遠漢末曹公集塞下荒地為新興郡宋白曰曹公立新興郡於樓煩郡唐為嵐州漢為汾陽縣地】師又謝朝士曰此我過也非陳雍州之責是以人皆愧悅【司馬師承父懿之後大臣未附引咎責躬所以愧服天下之心而固其權耳盜亦有道况盗國乎】<br />
<br />
  習鑿齒論曰司馬大將軍引二敗以為己過【二敗謂東關師敗及并州故反也】過消而業隆可謂智矣若乃諱敗推過【推吐雷翻】歸咎萬物常執其功而隱其喪【喪息浪翻】上下離心賢愚解體謬之甚矣【嗚呼此賈相國之所以敗也】君人者苟統斯理以御國行失而名揚【行下孟翻】兵挫而戰勝雖百敗可也况於再乎<br />
<br />
  光禄大夫張緝言於師曰恪雖克捷見誅不久師曰何故緝曰威震其主功蓋一國求不死得乎【緝料恪雖中緝亦卒為師所殺師方專政忌才智而疾異已况以緝而耀明於師乎】 二月吳軍還自東興進封太傅恪陽都侯加荆揚州牧督中外諸軍事恪遂有輕敵之心復欲出軍【復扶又翻】諸大臣以為數出罷勞【數所角翻罷讀曰疲】同辭諫恪恪不聽中散大夫蔣延固爭【漢制大夫議郎皆掌顧問應對無常事中散大夫秩六百石在諫議大夫上按中散大夫王莽所置後漢因之散悉亶翻】恪命扶出因著論以諭衆曰凡敵國欲相吞即仇讎欲相除也有讎而長之【長知兩翻左傳晉先軫曰堕軍實而長寇讎】禍不在已則在後人不可不為遠慮也昔秦但得關西耳【謂函谷關以西也】尚以并吞六國今以魏比古之秦土地數倍以吳與蜀比古六國不能半也然今所以能敵之者但以操時兵衆於今適盡而後生者未及長大正是賊衰少未盛之時【是時魏興三十餘年生聚教訓精兵良將分鎮方面諸葛蔣費陸遜朱然相繼凋謝吳蜀蓋小懦矣恪不能兢懼以保勝恃一戰之捷遽謂魏人為衰少未盛之時其輕敵甚矣長知兩翻少詩沼翻】加司馬懿先誅王凌續自隕斃【事見上卷嘉平三年】其子幼弱而專彼大任雖有智計之士未得施用當今伐之是其厄會【既以司馬師為幼弱又謂其未能用人兹可謂不善料敵者矣】聖人急於趨時【趨七喻翻】誠謂今日若順衆人之情懷偷安之計以為長江之險可以傳世不論魏之終始而以今日遂輕其後此吾所以長歎息者也【恪自謂其才足以辦魏不欲以賊遺後人吾不知其自視與叔父亮果何如也孔明累出師以攻魏每言一州之地不足以與賊支久卒無成功齎志以没恪無孔明之才而輕用其民不唯不足以強吳適足以滅其身滅其家而已】今聞衆人或以百姓尚貧欲務閒息此不知慮其大危而愛其小勤者也昔漢祖幸已自有三秦之地何不閉關守險以自娛樂空出攻楚身被創痍【事見漢高帝紀樂音洛被皮義翻創初良翻】介胄生蟣蝨【蟣居豈翻】將士厭困苦豈甘鋒刃而忘安寧哉慮於長久不得兩存者耳每鑒荆邯說公孫述以進取之圖【事見四十二卷漢光武建武六年邯下甘翻說輸芮翻】近見家叔父表陳與賊爭競之計【家叔父謂諸葛亮亮表見七十一卷明帝太和二年】未嘗不喟然歎息也夙夜反側所慮如此故聊疏愚言以逹一二君子之末若一朝隕没志畫不立貴令來世知我所憂可思於後耳衆人雖皆心以為不可然莫敢復難【復扶又翻下同難乃旦翻】丹陽太守聶友素與恪善以書諫恪曰大行皇帝本有遏東關之計【吳主之喪未踰年故稱之為大行皇帝聶尼輒翻】計未施行寇遠自送【謂魏兵遠來而自送死也】將士憑賴威德出身用命一旦有非常之功豈非宗廟神靈社稷之福邪【聶友此言所以抑恪之盛氣者婉而當有古朋友切偲之義焉】宜且案兵養鋭【案抑也】觀釁而動今乘此勢欲復大出【復扶又翻】天時未可而苟任盛意私心以為不安恪題論後為書答友曰【即前所著以喻衆之論也】足下雖有自然之理然未見大數【謂勝負存亡之大數也】熟省此論可以開悟矣【恪之所以待舊友者驕倨如此吳主權嫌其剛狠自用蓋已見之矣省悉景翻】滕胤謂恪曰君受伊霍之託入安本朝【朝直遥翻】出摧彊敵名聲振於海内天下莫不震動萬姓之心冀得蒙君而息今猥以勞役之後【勞役謂内有山陵營作外有東關之師也】興師出征民疲力屈遠主有備【左傳秦大夫蹇叔諫穆公曰勞師以襲遠師勞力屈遠主備之無乃不可乎】若攻城不克野略無獲是喪前勞而招後責也【喪息浪翻】不如案甲息師觀隙而動且兵者大事【左傳曰國之大事在祀與戎】事以衆濟衆苟不悦君獨安之【胤之言可謂深切矣】恪曰諸云不可皆不見計筭懷居苟安者也而子復以為然【復扶又翻下同】吾何望乎夫以曹芳闇劣【劣弱也】而政在私門【私門謂司馬氏】彼之民臣固有離心今吾因國家之資藉戰勝之威則何往而不克哉【談何容易】三月恪大發州郡二十萬衆復入寇【復扶又翻】以滕胤為都下督掌統留事 夏四月大赦 漢姜維自以練西方風俗【姜維本天水冀人故自以為練西方風俗練習也】兼負其才武欲誘諸羌胡以為羽翼【誘音酉】謂自隴以西可斷而有【斷丁管翻】每欲興軍大舉費禕嘗裁制不從與其兵不過萬人曰吾等不如丞相亦已遠矣【丞相謂諸葛亮】丞相猶不能定中夏况吾等乎不如且保國治民謹守社稷【治直之翻】如其功業以俟能者無為希冀徼倖【徼堅堯翻】决成敗於一舉若不如志悔之無及及禕死維得行其志【費禕死蜀諸臣皆出維下故不能裁制之】乃將數萬人出石營圍狄道【石營在董亭西南維蓋自武都出石營也狄道縣屬隴西郡為維以勞民亡蜀張本】 吳諸葛恪入寇淮南驅畧民人諸將或謂恪曰今引軍深入疆場之民必相率遠遁恐兵勞而功少【場音亦少詩沼翻】不如止圍新城【合肥新城也】新城困救必至至而圖之乃可大獲【此即諸葛誕言於司馬師之計也見上卷上年】恪從其計五月還軍圍新城詔太尉司馬孚督軍二十萬往赴之大將軍師問於虞松曰今東西有事二方皆急【謂吳攻淮南蜀攻隴西也】而諸將意沮若之何【沮在呂翻】松曰昔周亞夫堅壁昌邑而吳楚自敗【事見十六卷漢景帝三年】事有似弱而彊不可下察也今恪悉其鋭衆足以肆暴而坐守新城欲以致一戰耳【致者猶古所謂致師也】若攻城不拔請戰不可師老衆疲勢將自走諸將之不徑進乃公之利也姜維有重兵而縣軍應恪【縣讀曰懸】投食我麥【謂維軍後無轉餉投兵魏地擬其麥以為食耳】非深根之寇也且謂我并力於東西方必虛是以徑進今若使關中諸軍倍道急赴出其不意殆將走矣師曰善乃使郭淮陳泰悉關中之衆解狄道之圍敕毋丘儉案兵自守以新城委吳【毋音無】陳泰進至洛門【即天水冀縣落門聚】姜維糧盡退還【果如虞松所料】揚州牙門將涿郡張特守新城吳人攻之連月城中兵合三千人疾病戰死者過半而恪起土山急攻城將䧟不可護特乃謂吳人曰今我無心復戰也【復扶又翻】然魏灋被攻過百日而救不至者雖降家不坐【言雖身降而其家不坐罪也被皮義翻降戶江翻】自受敵以來已九十餘日矣此城中本有四千餘人戰死者已過半城雖䧟尚有半人不欲降我當還為相語條别善惡【為于偽翻語牛倨翻别彼列翻】明日早送名且以我印綬去為信乃投其印綬與之吳人聽其辭而不取印綬【綬音受】特乃投夜徹諸屋材栅補其缺為二重【重直龍翻】明日謂吳人曰我但有鬬死耳吳人大怒進攻之不能拔會大暑吳士疲勞飲水泄下流腫病者太半死傷塗地諸營吏日白病者多恪以為詐欲斬之自是莫敢言恪内惟失計【惟思也】而耻城不下忿形于色將軍朱異以軍事迕恪【迕五故翻逆也】恪立奪其兵斥還建業都尉蔡林數陳軍計【數所角翻】恪不能用策馬來犇諸將伺知吳兵已疲乃進救兵【伺相吏翻】秋七月恪引軍去士卒傷病流曳道路或頓仆坑壑【流者放而不能自收也曳者羸困不能自扶相牽引而行顛仆顛頓而僵仆也壑溝也】或見略獲存亡哀痛大小嗟呼而恪晏然自若出住江渚一月【渚水中洲也】圖起田於潯陽【漢尋陽故縣地也在大江之北尋陽記曰尋陽春秋為吳之西境楚之東境本在大江之北今鄿州界古蘭城是也】詔召相銜【言召命相繼也舟行以舳艫不絶為相銜陸行以馬首尾相接為相銜】徐乃旋師由是衆庶失望怨讟興矣【痛怨而謗曰讟讟徒木翻】汝南太守鄧艾言於司馬師曰孫權已没大臣未附吳名宗大族皆有部曲阻兵仗埶足以違命諸葛恪新秉國政而内無其主不念撫恤上下以立根基競於外事虐用其民悉國之衆頓於堅城死者萬數載禍而歸此恪獲罪之日也昔子胥吳起商鞅樂毅皆見任時君主没猶敗【伍子胥見任於吳王闔閭闔閭死夫差不能用其言而殺之吳起事見一卷周安王二十一年商鞅事見二卷顯王三十一年樂毅事見四卷赧王三十六年】况恪才非四賢而不慮大患其亡可待也【張緝鄧艾皆料諸葛恪必誅緝死而艾存者緝附李豐而艾為師用也然艾不死於師而死於昭者功名之際難居重以鍾會之搆間也】八月吳軍還建業諸葛恪陳兵導從【從才用翻】歸入府舘【府舘即府舍也】即召中書令孫嘿厲聲謂曰卿等何敢數妄作詔【怒其數作詔召之也數所角翻】嘿惶懼辭出因病還家恪征行之後曹所奏署令長職司一更罷選【曹選曹也罷選者罷而更選也長知兩翻】愈治威嚴多所罪責當進見者無不竦息【治直之翻】又改易宿衛用其親近復敕兵嚴欲向青徐【凡此者皆恪所以速死也敕兵嚴者戒兵士使嚴装也復扶又翻】孫峻因民之多怨衆之所嫌構恪於吳主云欲為變冬十月孫峻與吳主謀置酒請恪恪將入之夜精爽擾動【左傳鄭子產曰人生始化曰魄既生魄陽曰魂用物精多則魂魄強是以有精爽至於神明杜預曰爽明也擾動言不安也】通夕不寐【死期將至故然】又家數有妖怪【數所角翻】恪疑之旦日駐車宫門峻已伏兵於帷中恐恪不時入事泄乃自出見恪曰使君若尊體不安自可須後【須待也】峻當具白主上欲以嘗知恪意【嘗試也】恪曰當自力入【言當自力疾而入見吳主也】散騎常侍張約朱恩等密書與恪曰今日張設非常【張竹亮翻】疑有他故恪以書示滕胤胤勸恪還恪曰兒輩何能為正恐因酒食中人耳【中竹仲翻 考異曰恪傳曰恪省張約等書而去未出路門逢太常滕胤恪曰卒腹痛不任入胤不知峻隂計謂恪曰君自行旋未見上今上置酒請君君已至門宜當力進恪躊躇而還孫盛以為不然今從吳歷】恪入劒履上殿進謝還坐設酒恪疑未飲孫峻曰使君病未善平【言病未良已也】有常服藥酒可取之恪意乃安别飲所齎酒數行吳主還内峻起如厠解長衣著短服【著陟畧翻】出曰有詔收諸葛恪恪驚起拔劒未得而峻刀交下張約從旁斫峻裁傷左手峻應手斫約斷右臂【斷丁管翻】武衛之士皆趨上殿【武衛之士武衛將軍領之】峻曰所取者恪也今已死悉令復刃【令内刃於鞘也】乃除地更飲恪二子竦建聞難【難乃旦翻】載其母欲來犇峻使人追殺之以葦席裹恪尸篾束腰投之石子岡【恪傳曰建業南有長陵名石子岡葬者依焉按今高座寺後即石子岡寺在建康城南門宋白曰石子岡在臺城南四十里蓋今建康城非臺城也】又遣無難督施寛就將軍施績孫壹軍【施績時在江陵孫壹時在夏口】殺恪弟奮威將軍融於公安及其三子恪外甥都鄉侯張震常侍朱恩皆夷三族臨淮臧均表乞收葬恪曰震雷電激不崇一朝【鄭康成曰崇終也言不終一朝也】大風衝發希有極日然猶繼之以雲雨因以潤物是則天地之威不可經日浹辰【浹即協翻周也辰十二辰也十二日辰一周曰浹辰】帝王之怒不宜訖情盡意【訖亦盡也音居乞翻】臣以狂愚不知忌諱敢冒破滅之罪【謂破家滅身之罪】以邀風雨之會伏念故太傳諸葛恪罪積惡盈自致夷滅父子三首梟市積日【梟堅堯翻】觀者數萬詈聲成風國之大刑無所不震長老孩幼無不畢見【長知兩翻】人情之於品物【品衆也庶也】樂極則哀生【樂音洛】見恪貴盛世莫與貳身處台輔【處昌呂翻】中間歷年今之誅夷無異禽獸觀訖情反能不憯然【憯七感翻痛也】且已死之人與土壤同域鑿掘斫刺無所復加【刺七亦翻復扶又翻下同】願聖朝稽則乾坤【稽考也法則也】怒不極旬使其鄉邑若故吏民收以士伍之服【秦漢之制奪官爵者為士伍】惠以三寸之棺【禮記曰夫子制於中都四寸之棺五寸之椁鄭康成注云此庶人之制也按禮上大夫棺八寸椁六寸下大夫棺六寸椁四寸無三寸棺制也孟子曰中古棺七寸椁稱之墨子尚儉桐棺三寸左傳趙簡子曰桐棺三寸不設屬辟下卿之罰也】昔項籍受殯葬之施韓信獲收歛之恩斯則漢高發神明之譽也【葬項籍事見十一卷漢高帝五年歛韓信事今史無所考史云帝聞信死且死且憐之是必收歛之也施式智翻歛力贍翻】惟陛下敦三皇之仁【上古送死棄之中野後世聖人易之以棺椁此所謂三皇之仁也】埀哀矜之心使國澤加於辜戮之骸復受不已之恩於以揚聲遐方沮勸天下豈不大哉【沮在呂翻】昔欒布矯命彭越【事見十二卷漢高帝十一年】臣竊恨之不先請主上而專名以肆情其得不誅實為幸耳今臣不敢章宣愚情以露天恩謹伏手書冒昧陳聞【古之人臣進言於君率曰冒死曰昧死謂人君之威難犯冒昧其死罪而言也】乞聖明哀察於是吳主及孫峻聽恪故吏斂葬【斂力贍翻】初恪少有盛名【少詩照翻】大帝深器重之而恪父瑾常以為戚曰非保家之主也【戚憂也瑾渠吝翻】父友奮威將軍張承亦以為恪必敗諸葛氏【敗補邁翻】陸遜嘗謂恪曰在我前者吾必奉之同升在我下者則扶接之今觀君氣陵其上意蔑乎下【蔑者視之若無】非安德之基也漢侍中諸葛瞻亮之子也恪再攻淮南越嶲太守張嶷與瞻書曰東主初崩【吳在蜀東故謂其君為東主嶲音髓嶷魚力翻】帝實幼弱【帝謂吳主亮】太傅受寄託之重【諸葛恪為吳太傅故稱之】亦何容易【易以豉翻】親有周公之才猶有管蔡流言之變【謂周公之才而有叔父之親且不能免於管蔡之流言】霍光受任亦有燕蓋上官逆亂之謀【事見二十三卷漢昭帝元鳳元年】賴成昭之明以免斯難耳【難乃旦翻】昔每聞東主殺生賞罰不任下人又今以埀没之命卒召太傳屬以後事【卒讀曰猝屬之欲翻】誠實可慮加吳楚剽急乃昔所記【周亞夫曰吳楚剽輕太史公曰楚俗剽輕易發怒自漢以來皆有是言剽匹妙翻】而太傅離少主【離力智翻少詩沼翻】履敵庭恐非良計長筭也雖云東家綱紀肅然上下輯睦【東家亦謂吳立國於東也】百有一失非明者之慮也取古則今今則古也【則子德翻則刌剫也様也言取古事以刌剫今之事今猶古也】自非郎君進忠言於太傅【自漢以來門生故吏率稱恩門子弟為郎君】誰復有盡言者邪【復扶又翻】旋軍廣農務行德惠數年之中東西竝舉實為不晩願深採察恪果以此敗吳羣臣共議上奏推孫峻為太尉滕胤為司徒【上時掌翻】有媚峻者言曰萬機宜在公族若承嗣為亞公【滕胤字承嗣司徒位亞太尉故曰亞公】聲名素重衆心所附不可量也【量音良】乃表峻為丞相大將軍督中外諸軍事又不置御史大夫由是士人失望【漢承秦制置御史大夫以副丞相理衆事今峻為丞相而不置御史大夫則專吳國之政故國人失望】滕胤女為恪子竦妻胤以此辭位孫峻曰鯀禹罪不相及【舜之罪也殛鯀其功也興禹】滕侯何為峻與胤雖内不沾洽【言其情不浹洽也】而外相苞容進胤爵高密侯共事如前齊王奮聞諸葛恪誅下住蕪湖欲至建業觀變傳相謝慈等諫奮殺之坐廢為庶人徙章安【章安前漢治縣也故閩越地光武更名章安屬會稽郡沈約宋志曰臨海太守本會稽東部都尉前漢治鄞後漢分會稽為吳郡疑是都尉徙治章安也晉太康記曰章安本鄞縣南之回浦郷余謂太康志所云即吳臨海郡之章安縣地今台州黄巖縣章安鎮是也奮徙章安即臨海之章安也】南陽王和妃張氏諸葛恪之甥也先是恪有遷都之意【先悉薦翻】使治武昌宫【治直之翻】民間或言恪欲迎和立之及恪被誅丞相峻因此奪和璽綬【南陽王璽綬也璽斯氏翻綬音受】徙新都又遣使者追賜死初和妾何氏生子皓諸姬子德謙俊和將死與張妃别妃曰吉凶當相隨終不獨生亦自殺何姬曰若皆從死誰當字孤【從才用翻說文曰字乳也愛也】遂撫育皓及其三弟皆賴以獲全【為後吳人立皓張本】<br />
<br />
  高貴鄉公上【諱髦字彦士文帝孫東海定王霖子也正始五年封高貴鄉公高貴鄉屬郯縣】<br />
<br />
  正元元年【是年嘉平六年也冬十月高貴鄉公方改元正元通鑑以是年繫之高貴鄉公因書正元元年】春二月殺中書令李豐初豐年十七八已有清名海内翕然稱之其父太僕恢不願其然敕使閉門斷客【斷讀曰短】曹爽專政司馬懿稱不出【事見上卷邵陵厲公正始八年九年】豐為尚書僕射依違二公間故不與爽同誅豐子韜以選尚齊長公主【帝之姊妹曰長公主齊主蓋明帝女長知兩翻】司馬師秉政以豐為中書令是時太常夏侯玄有天下重名以曹爽親不得在埶任居常怏怏【邵陵厲公嘉平元年玄自關右召詣京師埶任權埶之任也怏於兩翻】張緝以后父去郡家居【緝自東莞召見上卷嘉平四年】亦不得意豐皆與之親善師雖擢用豐豐私心常在玄豐在中書二歲帝數召豐與語【數所角翻】不知所說師知其議已請豐相見以詰豐【詰去吉翻】豐不以實告師怒以刀鐶築殺之【鐶戶關翻刀把上有鐶築擣也】送尸付廷尉遂收豐子韜及夏侯玄張緝等皆下廷尉【下遐稼翻下及下同】鍾毓案治云豐與黄門監蘇鑠永寧署令樂敦【漢有黄門令宦者為之黄門監蓋魏置也永寧宫魏太后宫名永寧署令太后宫官也亦宦者為之治直之翻】冗從僕射劉賢等【漢制中宫冗從僕射宦者為之主黄門冗從秩六百石沈約志曰漢東京有中黄門冗從僕射魏世因其名而置冗從僕射冗而隴翻散也】謀曰拜貴人日諸營兵皆屯門【屯宫城門也】陛下臨軒【檐宇之末曰軒促御坐前臨殿陛曰臨軒】因此同奉陛下將羣僚人兵就誅大將軍【下將即亮翻】陛下儻不從人便當刼將去耳又云謀以玄為大將軍緝為車騎將軍玄輯皆知其謀【此上皆獄辭也】庚戍誅韜玄緝鑠敦賢皆夷三族夏侯霸之入蜀也【見上卷嘉平元年】邀玄欲與之俱玄不從及司馬懿薨中領軍高陽許允謂玄曰無復憂矣【復扶又翻】玄歎曰士宗卿何不見事乎【許允字士宗不見事猶今人言不曉事也】此人猶能以通家年少遇我【少詩照翻】子元子上不吾容也【司馬師字子元司馬昭字子上】及下獄玄不肯下辭鍾毓自臨治之【治直之翻】玄正色責毓曰吾當何罪卿為令史責人也【自漢以來公府有令史廷尉則有獄史耳玄蓋責毓以身為九卿乃承公府指自臨治我是為公府令史而責人也】卿便為吾作【為于偽翻下同】毓以玄名士節高不可屈而獄當竟【竟結竟也】夜為作辭令與事相附【為作獄辭使與所案之事相附合也】流涕以示玄玄視頷之而已及就東市顔色不變舉動自若李豐弟翼為兖州刺史司馬師遣使收之翼妻荀氏謂翼曰中書事發可及詔書未至赴吳何為坐取死亡左右可同赴水火者為誰【赴水火者入必焦没自非誓同生死安肯相從故以為言】翼思未答妻曰君在大州不知可與同生死者雖去亦不免翼曰二兒小吾不去今但從坐身死耳【謂從兄坐罪止一身若奔吳不逹禍及妻子也】二兒必免乃止死初李恢與尚書僕射杜畿及東安太守郭智善【東安縣前漢屬城陽國後漢屬琅邪國魏分為郡沈約曰晉惠帝分東莞為東安郡蓋魏既分而又省併既省併而晉又分屬東莞又自東莞分為郡也】智子冲有内實而無外觀州里弗稱也冲嘗與李豐俱見畿既退畿歎曰孝懿無子非徒無子殆將無家君謀為不死也其子足繼其業【李恢字孝懿郭智字君謀】時人皆以畿為誤及豐死冲為代郡太守卒繼父業【卒子恤翻】正始中夏侯玄何晏鄧颺俱有盛名欲交尚書郎傳嘏嘏不受嘏友人荀粲怪而問之嘏曰太初志大其量能合虚聲而無實才【夏侯玄字太初】何平叔言遠而情近好辯而無誠所謂利口覆邦國之人也【論語孔子曰惡利口之覆邦家者何晏字平叔好呼到翻】鄧玄茂有為而無終外要名利内無關鑰貴同惡異多言而妬前多言多釁妬前無親【鄧颺字玄茂要一遥翻妬前者忌前也人忌勝已則無親之者要一遥翻惡烏路翻】以吾觀此三人者皆將敗家遠之猶恐禍及【敗補邁翻遠于願翻】况昵之乎【昵尼質翻近也比也】嘏又與李豐不善謂同志曰豐飾偽而多疑矜小智而昧於權利若任機事其死必矣辛亥大赦 三月廢皇后張氏【曹操殺漢后伏氏而司馬師殺魏后張氏】<br />
<br />
  【此不惟天道亦操之有以教之也】夏四月立皇后王氏奉車都尉夔之之女也 狄道長李簡密書請降於漢【長知兩翻降戶江翻】六月姜維寇隴西 中領軍許允素與李豐夏侯玄善秋允為鎮北將軍假節都督河北諸軍事【晉志假節都督者與四征鎮加大將軍不開府為都督者同四征鎭安平加大將軍不開府持節都督者品秩第二】帝以允當出詔會羣臣帝特引允以自近【近其靳翻】允當與帝别涕泣歔欷【君臣不密遂竝蹈失臣失身之禍歔音虚欷音希又許既翻】允未發有司奏允前放散官物收付廷尉徙樂浪【樂浪音洛琅】未至道死 吳孫峻驕矜淫暴國人側目司馬桓慮謀殺峻立太子登之子吳侯英不克皆死 帝以李豐之死意殊不平安東將軍司馬昭鎭許昌詔召之使擊姜維九月昭領兵入見帝幸平樂觀以臨軍過【見賢遍翻樂音洛觀古玩翻】左右勸帝因昭辭殺之勒兵以退大將軍已書詔於前帝懼不敢發昭引兵入城大將軍師乃謀廢帝【平樂觀在洛陽城西昭已過軍復引入城帝事去矣】甲戍師以皇太后令召羣臣會議【矯太后令以召羣臣】以帝荒淫無度䙝近倡優【倡齒良翻倡優女樂也近其靳翻】不可以承天緒羣臣皆莫敢違乃奏收帝璽綬歸藩于齊【璽斯氏翻綬音受】使郭芝入白太后太后方與帝對坐芝謂帝曰大將軍欲廢陛下立彭城王據【彭城王據文帝子此何等語芝太后之從父也故使之入脅太后】帝乃起去太后不悦芝曰太后有子不能教今大將軍意已成又勒兵于外以備非常但當順旨將復何言【復扶又翻】太后曰我欲見大將軍口有所說芝曰何可見邪但當速取璽綬【王莽簒漢遣王舜求璽於元后其辭氣何至如此】太后意折【折屈也音之列翻】乃遣傍侍御取璽綬著坐側【太后侍御非止一人傍侍御謂當時侍御之在傍側者著直畧翻坐徂卧翻】芝出報師師甚喜【王莽司馬師蕭鸞同是心也國之姦賊必有羽翼有天下者其戒之哉】又遣使者授帝齊王印綬出就西宫帝與太后垂涕而别遂乘王車從太極殿南出【王車諸王所乘青盖車也】羣臣送者數十人司馬孚悲不自勝【勝音升】餘多流涕【廢帝時年二十一】師又使使者請璽綬於太后太后曰彭城王我之季叔也今來立我當何之【之往也】且明皇帝當永絕嗣乎高貴鄉公文帝之長孫明皇帝之弟子【太后謂明帝絕嗣蓋謂以據為後則兄死弟及又禮兄弟不得相入廟也文帝黄初三年初制封王之庶子為鄉公嗣王之庶子為侯公侯之庶子為亭伯】於禮小宗有後大宗之義其詳議之【出嫡為大宗支子之子各宗其父為小宗禮王后無嗣擇建支子以繼大宗】丁丑師更召羣臣以太后令示之乃定迎高貴鄉公髦於元城【定迎者議始定而迎之也元城縣漢屬魏郡魏屬陽平郡時魏王公皆録置鄴故出髦而就元城迎之】髦者東海定王霖之子也時年十四使太常王肅持節迎之師又使請璽綬太后曰我見高貴鄉公小時識之【太后欲立高貴鄉公必見其小時意氣異於諸王子故欲立之豈知禄去帝室而終無益乎】我自欲以璽綬手授之冬十月癸丑高貴鄉公至玄武舘【酈道元曰魏氏立玄武舘於芒垂蓋館在芒山之尾其地直洛城北】羣臣奏請舍前殿【玄武館之前殿也】公以先帝舊處避止西廂羣臣又請以灋駕迎公不聽庚寅公入于洛陽羣臣迎拜西掖門南公下輿答拜儐者請曰儀不拜【儐必刃翻贊導者也儀不拜者謂於儀不當答拜也】公曰吾人臣也遂答拜至止車門下輿左右曰舊乘輿入公曰吾被皇太后徵未知所為【言唯天子可乘輿入止車門吾方被徵未知何如不可以天子自居也以余觀高貴鄉公蓋小慧而知書故能為此若以為習於禮則余以為猶魯昭公也被皮義翻】遂步至太極東堂見太后其日即皇帝位於太極前殿百僚陪位者皆欣欣焉【謂公之足與有為也而卒死於權臣之手嗚呼余觀漢文帝入立之後夜拜宋昌為衛將軍領南北軍張武為郎中令行殿中周勃陳平朱虚東牟雖有大功其權去矣夫然後能自固魏朝百官皆欣欣者果何所見邪】大赦改元【自此方是正元元年】為齊王築宫于河内【為于偽翻】漢姜維自狄道進拔河間臨洮【河間當作河關河關縣前漢屬金城郡後漢屬隴西郡以地里考之河關臨洮在狄道西姜維自狄道西拔河關臨洮意欲收魏之邉縣以自廣耳】將軍徐質與戰殺其盪寇將軍張嶷【沈約志四十號將軍盪寇第二十二嶷魚力翻】漢兵乃還 初揚州刺史文欽驍果絕人曹爽以其鄉里故愛之【欽爽邑人也驍堅堯翻】欽恃爽埶多所陵傲及爽誅【爽誅見上卷嘉平元年】又好增虜級以邀功賞【好呼到翻】司馬師常抑之由是怨望鎭東將軍毋丘儉素與夏侯玄李豐善玄等死儉亦不自安乃以計厚待欽儉子治書侍御史甸謂儉曰大人居方嶽重任【古者天子廵狩四方其方之諸侯各會朝于方嶽之下堯舜有四岳之官孔安國曰堯命羲和四子分掌四方之諸侯故曰四岳魏晉之時征鎭安平總督諸軍任專方面時因謂之方嶽重任】國家傾覆而晏然自守將受四海之責矣儉然之<br />
<br />
  二年春正月儉欽矯太后詔起兵於壽春移檄州郡以討司馬師乃表言相國懿忠正有大勲於社稷宜宥及後世請廢師以侯就第以弟昭代之太尉孚忠孝小心護軍望忠公親事皆宜親寵授以要任望孚之子也儉又遣使邀鎭南將軍諸葛誕誕斬其使【時誕都督豫州】儉欽將五六萬衆渡淮西至項儉堅守使欽在外為游兵司馬師問計於河南尹王肅肅曰昔關羽虜于禁於漢濱有北向爭天下之志後孫權襲取其將士家屬羽士衆一旦瓦解【事見六十八卷漢獻帝建安二十四年】今淮南將士父母妻子皆在州内【魏制諸將出征及鎮守方面皆留質任時淮南將士皆自内州出戍故家屬皆留内】但急往禦衛【禦儉欽之衆使不得進又衛其家屬】使不得前必有關羽土崩之勢矣時師新割目瘤創甚【瘤音留肬也肉起疾腫曰瘤創初良翻】或以為大將軍不宜自行不如遣太尉孚拒之唯王肅與尚書傳嘏中書侍郎鍾會【魏初中書既置監令又置通事郎次黄門郎黄門郎已署事過通事郎乃署名已署奏以入為帝省讀書可後改曰中書侍郎】勸師自行師疑未决嘏曰淮楚兵勁【壽春故楚都時為淮南重鎮以南備吳勁兵聚焉】而儉等負力遠鬭其鋒未易當也【易以豉翻】若諸將戰有利鈍大勢一失則公事敗矣師蹶然起曰我請輿疾而東【蹶然急遽而起之貌蹶音厥又音姑衛翻】戊午師率中外諸軍以討儉欽【中謂中軍外謂城外諸營兵】以弟昭兼中領軍留鎭洛陽召三方兵會于陳許【三方東西北也】師問計於光禄勲鄭袤【袤莫翻】袤曰毋丘儉好謀而不逹事情【好呼到翻】文欽勇而無筭今大軍出其不意江淮之卒鋭而不能固宜深溝高壘以挫其氣此亞夫之長策也【漢周亞夫堅壁以破吳楚】師稱善師以荆州刺史王基為行監軍假節統許昌軍【魏晉之制使持節都督諸軍為上假節都督次之假節監諸軍又次之假節行監軍又次之魏受漢禪以許昌為别宫屯重兵以為東南二方根本監古衘翻】基言於師曰淮南之逆非吏民思亂也儉等誑誘迫脅畏目下之戮是以尚屯聚耳【誑居况翻誘音酉】若大兵一臨必土崩瓦解儉欽之首不終朝而致於軍門矣師從之以基為前軍既而復敕基停駐【復扶又翻】基以為儉等舉軍足以深入而久不進者是其詐偽己露衆心疑沮也【沮在呂翻】今不張示威形以副民望而停軍高壘有似畏懦非用兵之埶也若儉欽虜畧民人以自益又州郡兵家為賊所得者更懷離心【言州郡兵其家有為賊所得者必懷反顧而有離散之心也】儉等所迫脅者自顧罪重不敢復還此為錯兵無用之地【錯倉故翻置也停軍不進是置之於無用之地】而成姦宄之源吳寇因之則淮南非國家之有譙沛汝豫危而不安【豫即潁川也豫州時治潁川故曰譙沛汝豫四郡皆屬豫州】此計之大失也軍宜速進據南頓【南頓縣屬汝南郡故頓子國應劭曰頓迫於陳其後南徙故號南頓】南頓有大邸閣計足軍人四十日糧保堅城因積穀先人有奪人之心【左傳楚令尹孫叔敖之言也杜預注曰奪敵戰心先悉薦翻】此平賊之要也基屢請乃聼進據㶏水【水經注汝水東南過定陵縣又東南逕奇雒城枝分别出世謂之大㶏水㶏水東流至南頓縣北入于潁師古曰㶏於謹翻又音殷】閏月甲申師次于㶏橋儉將史招李續相次來降王基復言於師曰【復扶又翻】兵聞拙速未覩為巧之久也【孫子之言】方今外有彊寇内有叛臣若不時决則事之深淺未可測也【言儉欽之變若不以時定恐吳寇乘之而來則禍之深淺有未可測者】議者多言將軍持重將軍持重是也停軍不進非也持重非不行之謂也進而不可犯耳今保壁壘以積實資虜而遠運軍糧甚非計也師猶未許基曰將在軍君令有所不受【孫子及司馬穰苴皆有是言】彼得亦利我得亦利是謂爭地【孫子之言所謂九地爭地其一也】南頓是也遂輒進據南頓儉等從項亦欲往爭發十餘里【發兵而行十餘里】聞基先到乃復還保項 癸未征西將軍郭淮卒以雍州刺史陳泰代之【雍於用翻】 吳丞相峻率驃騎將軍呂據左將軍會稽留贊襲壽春【驃匹妙翻會工外翻】司馬師命諸軍皆深壁高壘以待東軍之集【東軍青徐兖之軍也】諸將請進軍攻項師曰諸軍知其一未知其二【諸軍當作諸君】淮南將士本無反志儉欽說誘與之舉事【說輸芮翻】謂遠近必應而事起之日淮北不從【淮北謂豫兖也】史招李續前後瓦解内乖外叛自知必敗困獸思鬬【左傳吳夫槩王曰困獸猶鬬】速戰更合其志雖云必克傷人亦多且儉等欺誑將士詭變萬端小與持久詐情自露此不戰而克之術也乃遣諸葛誕督豫州諸軍自安風向壽春【安風縣前漢屬六安國後漢併屬廬江郡魏分安風等五縣置安豐郡屬豫州】征東將軍胡遵督青徐諸軍出譙宋之間【宋謂梁國之地梁國都睢陽故宋都也】絕其歸路師屯汝陽【汝陽縣屬汝南郡在汝水之北】毋丘儉文欽進不得鬬退恐壽春見襲計窮不知所為淮南將士家皆在北衆心沮散降者相屬【果如王肅之計屬之欲翻】惟淮南新附農民為之用儉之初起遣健步齎書至兖州【健步能走者今謂之急脚子又謂之快行子】兖州刺史鄧艾斬之將兵萬餘人兼道前進先趨樂嘉城【水經注潁水過汝陽縣北又東南過南頓縣㶏水注之又南逕博陽故城東城在南頓縣北四十里漢宣帝封丙吉為侯國王莾更名樂嘉趨七喻翻】作浮橋以待師儉使文欽將兵襲之師自汝陽潛兵就艾於樂嘉欽猝見大軍驚愕未知所為欽子鴦年十八勇力絕人謂欽曰及其未定擊之可破也於是分為二隊夜夾攻軍鴦帥壮士先至鼔譟【帥讀曰率】軍中震擾師驚駭所病目突出恐衆知之囓被皆破【齧被以忍疼齧魚結翻】欽失期不應會明鴦見兵盛乃引還【還從宣翻又如字】師與諸將曰賊走矣可追之諸將曰欽父子驍猛未有所屈何苦而走師曰夫一鼓作氣再而衰【左傳魯曹劌之言】鴦鼓譟失應其埶已屈不走何待欽將引而東鴦曰不先折其埶不得去也乃與驍騎十餘摧鋒䧟陳【陳讀曰陣】所向皆披靡【披普彼翻】遂引去師使左長史司馬班率驍騎八千翼而追之【魏公府及諸大將軍位從公者各置長史一人惟大將軍府及司徒府加置左右長史各一人翼者張左右翼而追之】鴦以匹馬入數千騎中輒殺傷百餘人乃出如此者六七追騎莫敢逼殿中人尹大目小為曹氏家奴常在天子左右【大目時為殿中校尉】師將與俱行【將讀如鳳將雛鷄冠距鳴將之將音如字】大目知師一目已出啟云文欽本是明公腹心但為人所誤耳又天子鄉里【文欽譙人故曰天子鄉里】素與大目相信乞為公追解語之【謂追欽而為師自解釋言之也為于偽翻語牛倨翻】令還與公復好【復還也反也好善也謂還復相善也好讀如字】師許之大目單身乘大馬被鎧胄【被皮義翻】追欽遥相與語大目心實欲為曹氏【為于偽翻】謬言君侯何苦不可復忍數日中也【蓋謂文欽何不堅忍數日與師相持師病已篤必當有變也復扶又翻】欲使欽解其旨【解胡買翻喻也曉也】欽殊不悟乃更厲聲罵大目曰汝先帝家人不念報恩反與司馬師作逆不顧上天天不祐汝張弓傅矢欲射大目【傅讀曰附射而亦翻】大目涕泣曰世事敗矣善自努力是日毋丘儉聞欽退恐懼夜走衆遂大潰欽還至項以孤軍無繼不能自立欲還壽春壽春已潰遂犇吳吳孫峻至東興聞儉等敗壬寅進至槖臯【春秋會吳于槖臯杜預曰在九江逡遒縣東南今其地在巢縣界亦謂之柘臯槖音託又讀為柘】文欽父子詣軍降【降戶江翻】毋丘儉走北至愼縣【愼縣漢屬汝南郡魏分屬汝隂郡賢曰愼縣故城在今潁州潁上縣西北余按儉自項走至愼愼在項南非北也北乃比字之誤比必寐翻】左右人兵稍棄儉去儉藏水邉草中甲辰安風津民張屬就殺儉【水經注淮水東過安豐縣東北又東為安豐津水南有城故安豐都尉治後立霍丘戍杜佑曰安風津在壽州霍丘城北】傳首京師封屬為侯諸葛誕至壽春壽春城中十餘萬口懼誅或流迸山澤或散走入吳【迸北孟翻】詔以誕為鎭東大將軍儀同三司都督揚州諸軍事夷毋丘儉三族儉黨七百餘人繫獄侍御史杜友治之【治直之翻】惟誅首事者十餘人餘皆奏免之儉孫女適劉氏當死以孕繫廷尉司隸主簿程咸議曰【魏晉之制列卿各置丞功曹主簿五官等員】女適人者若已產育則成他家之母於防不足以懲姦亂之源【防謂禁防也】於情則傷孝子之恩男不遇罪於他族而女獨嬰戮於二門【嬰當也二門謂父母之家及夫家也】非所以哀矜女弱【女隂類禀氣柔弱在室從父母既嫁從夫故曰女弱】均灋制之大分也【分扶問翻】臣以為在室之女可從父母之刑既醮之婦使從夫家之戮【毛晃曰醮冠娶祭名酌而無酬酢曰醮禮記曰醮於客位冠禮也父親醮子而命之迎婚禮也晉志曰古者昏冠皆有醮鄭氏醮文三首具存醮子肖翻】朝廷從之仍著於律令舞陽忠武侯司馬師疾篤還許昌留中郎將參軍事賈充監諸軍事充逵之子也【賈逵事武帝文帝監古衘翻】衛將軍昭自洛陽往省師【魏制衛將軍班車騎將軍下位從公省悉景翻】師令昭總統諸軍辛亥師卒于許昌【卒子恤翻】中書侍郎鍾會從師典知密事中詔敕尚書傅嘏【詔自中出上意也是時詔命皆以司馬氏之意行之此詔出於禁中之意故曰中詔】以東南新定權留衛將軍昭屯許昌為内外之援令嘏率諸軍還會與嘏謀使嘏表上【上時掌翻】輒與昭俱發還到洛水南屯住二月丁巳詔以司馬昭為大將軍録尚書事會由是常有自矜之色嘏戒之曰子志大其量而勲業難為也可不愼哉【為後鍾會作亂張本】 吳孫峻聞諸葛誕已據壽春乃引兵還以文欽為都護鎭北大將軍幽州牧【漢置都護於西域於西域稱都護將軍然未嘗以為將軍號至光武遂有都護將軍之官三國位從公晉制在撫軍下鎭軍上吳置左右都護亦不以為將軍號今以欽為都護蓋又在左右都護之上矣】三月立皇后卞氏大赦后武宣皇后弟秉之曾孫女也秋七月吳將軍孫儀張怡林恂謀殺孫峻不克死者<br />
<br />
  數十人全公主譖朱公主於峻曰與儀同謀峻遂殺朱公主【朱公主吳主權之女適朱據者也】峻使衛尉馮朝城廣陵【魏之廣陵郡治淮隂漢之廣陵故城廢棄不治】功費甚衆舉朝莫敢言唯滕胤諫止之峻不從功卒不成【卒子恤翻】 漢姜維復議出軍【復扶又翻下同】征西大將軍張翼廷爭【爭讀曰諍】以為國小民勞不宜黷武維不聼率車騎將軍夏侯霸及翼同進八月維將數萬人至枹罕【枹罕縣前漢屬金城郡後漢屬隴西郡魏時廢省枹音膚】趨狄道【趨七喻翻】征西將軍陳泰敕雍州刺史王經進屯狄道須泰軍到東西合埶乃進泰軍陳倉經所統諸軍於故關與漢人戰不利【故關謂漢時故邊關也在洮水西】經輒渡洮水泰以經不堅據狄道必有他變率諸軍以繼之經已與維戰於洮西大敗【洮土刀翻】以萬餘人還保狄道城餘皆犇散死者萬計張翼謂維曰可以止矣不宜復進或毁此大功為蛇畫足【戰國策曰昭陽為楚伐魏覆軍殺將移師攻齊陳軫為齊王使見昭陽曰楚有祠者賜其舍人酒一巵舍人相謂曰數人飲之不足一人飲之有餘請各畫地為蛇先成者飲酒一人先成引酒飲之乃左手持巵右手畫蛇曰吾能為之足為足未成一人之蛇後成奪其巵曰蛇固無足子安能為之足遂飲酒今君攻魏既勝復移師攻齊是為蛇足者也昭陽悟乃還軍】維大怒遂進圍狄道辛未詔長水校尉鄧艾行安西將軍與陳泰并力拒維【晉志曰四安起於魏初在四鎮之下】戊辰復以太尉孚為後繼泰進軍隴西諸將皆曰王經新敗賊衆大盛將軍以烏合之衆繼敗軍之後當乘勝之鋒殆必不可古人有言蝮蛇螫手壮士解腕【漢書田榮傳曰蝮蠚手則斬手蠚足則斬足應劭曰蝮一名虺螫人手足則割去其肉不然則死師古曰爾雅及說文皆以為蝮即虺也博三寸首大如擘而郭璞云各自一種蛇其蝮蛇細頸大頭焦尾色如綬文文間有毛如猪鬛鼻上有針大者長七八尺一名反鼻非虺之類也今以俗名證之郭說得矣虺若土色所在有之蝮蛇唯出南方蝮芳六翻螫式亦翻腕烏貫翻陸佃埤雅蝮蛇怒時毒在頭尾螫手則手斷螫足則足斷蛇之尤毒烈者也】孫子曰兵有所不擊地有所不守蓋小有所失而大有所全故也不如據險自保觀釁待敝然後進救此計之得者也泰曰姜維提輕兵深入正欲與我爭鋒原野求一戰之利王經當高壁深壘挫其鋭氣今乃與戰使賊得計經既破走維若以戰克之威進兵東向據櫟陽積穀之實【櫟陽縣前漢屬左馮翊後漢魏省余謂櫟陽在長安東北維兵方至狄道安得便可東據櫟陽泰盖言畧陽耳櫟音藥藥畧聲相近因語訛而致傳寫字訛耳】放兵收降【降戶江翻】招納羌胡東爭關隴傳檄四郡【四郡謂隴西南安天水畧陽略陽時為廣魏郡及晉乃更名畧陽】此我之所惡也【惡烏路翻】而乃以乘勝之兵挫峻城之下鋭氣之卒屈力致命攻守埶殊客主不同兵書曰脩櫓轒輼三月乃成拒堙三月而後已【此孫子之言也孫子之說以攻城為不得已魏武注曰修治也櫓大楯也轒輼者轒也轒伏其下四輪從中推之至城下也杜佑曰攻城戰具作四輪車車上以繩為脊生牛皮蒙之下可藏十人塡隍推之直抵城下可以攻掘金火木石所不能敗謂之轒輼車注又曰距堙者踊土稍高而前以拊其城也杜佑曰土山即孫子所謂距闉也應劭曰轒輼匈奴車非也蓋攻城之車耳師古曰轒扶云翻輼於云翻】誠非輕軍遠入之利也今維孤軍遠僑【僑音喬寄也客也】糧穀不繼是我速進破賊之時所謂疾雷不及掩耳【文子之言淮南子亦有是言】自然之埶也洮水帶其表維等在其内今乘高據埶臨其項領不戰必走寇不可縱圍不可久君等何言如是遂進軍度高城嶺【水經注曰隴西首陽縣有高城嶺嶺上有城曰渭源城】濳行夜至狄道東南高山上多舉烽火鳴鼓角狄道城中將士見救至皆憤踊維不意救兵卒至【卒讀曰猝】緣山急來攻之泰與交戰維退泰引兵揚言欲向其還路維懼九月甲辰維遁走城中將士乃得出王經歎曰糧不至旬向非救兵速至舉城屠裂覆喪一州矣【隴西畧陽天水南安秦州也喪息浪翻】泰慰勞將士前後遣還更差軍守【差擇也遣還王經所統將士更擇軍以守狄道勞力到翻差初佳翻】并治城壘【治直之翻】還屯上邽泰每以一方有事輒以虚聲擾動天下故希簡上事【上時掌翻】驛書不過六百里【狄道東至洛陽二千二百餘里而驛書不過六百里蓋傳入近裏郡縣使如常郵筒以逹洛陽也】大將軍昭曰陳征西沈勇能斷【沈持林翻】荷方伯之重【荷下河翻】救將䧟之城而不求益兵又希簡上事必能辦賊者也都督大將不當爾邪姜維退駐鍾提【鍾提當在羌中蜀之凉州界也】 初吳大帝不立太廟以武烈嘗為長沙太守立廟於臨湘【吳大帝謚其父堅曰武烈皇帝長沙郡治臨湘縣】使太守奉祠而已冬十月始作太廟於建業尊大帝為太祖 【考異曰吳歷太平元年正月立太祖廟沈約宋書孫亮立明年正月立權廟今從吳志】<br />
<br />
  資治通鑑卷七十六  <br>
   </div> 

<script src="/search/ajaxskft.js"> </script>
 <div class="clear"></div>
<br>
<br>
 <!-- a.d-->

 <!--
<div class="info_share">
</div> 
-->
 <!--info_share--></div>   <!-- end info_content-->
  </div> <!-- end l-->

<div class="r">   <!--r-->



<div class="sidebar"  style="margin-bottom:2px;">

 
<div class="sidebar_title">工具类大全</div>
<div class="sidebar_info">
<strong><a href="http://www.guoxuedashi.com/lsditu/" target="_blank">历史地图</a></strong>  
<a href="http://www.880114.com/" target="_blank">英语宝典</a>  
<a href="http://www.guoxuedashi.com/13jing/" target="_blank">十三经检索</a> 
<br><strong><a href="http://www.guoxuedashi.com/gjtsjc/" target="_blank">古今图书集成</a></strong> 
<a href="http://www.guoxuedashi.com/duilian/" target="_blank">对联大全</a> <strong><a href="http://www.guoxuedashi.com/xiangxingzi/" target="_blank">象形文字典</a></strong> 

<br><a href="http://www.guoxuedashi.com/zixing/yanbian/">字形演变</a>  <strong><a href="http://www.guoxuemi.com/hafo/" target="_blank">哈佛燕京中文善本特藏</a></strong>
<br><strong><a href="http://www.guoxuedashi.com/csfz/" target="_blank">丛书&方志检索器</a></strong> <a href="http://www.guoxuedashi.com/yqjyy/" target="_blank">一切经音义</a>  

<br><strong><a href="http://www.guoxuedashi.com/jiapu/" target="_blank">家谱族谱查询</a></strong>  <strong><a href="http://shufa.guoxuedashi.com/sfzitie/" target="_blank">书法字帖欣赏</a></strong> 
<br>

</div>
</div>


<div class="sidebar" style="margin-bottom:0px;">

<font style="font-size:22px;line-height:32px">QQ交流群9:489193090</font>


<div class="sidebar_title">手机APP 扫描或点击</div>
<div class="sidebar_info">
<table>
<tr>
	<td width=160><a href="http://m.guoxuedashi.com/app/" target="_blank"><img src="/img/gxds-sj.png" width="140"  border="0" alt="国学大师手机版"></a></td>
	<td>
<a href="http://www.guoxuedashi.com/download/" target="_blank">app软件下载专区</a><br>
<a href="http://www.guoxuedashi.com/download/gxds.php" target="_blank">《国学大师》下载</a><br>
<a href="http://www.guoxuedashi.com/download/kxzd.php" target="_blank">《汉字宝典》下载</a><br>
<a href="http://www.guoxuedashi.com/download/scqbd.php" target="_blank">《诗词曲宝典》下载</a><br>
<a href="http://www.guoxuedashi.com/SiKuQuanShu/skqs.php" target="_blank">《四库全书》下载</a><br>
</td>
</tr>
</table>

</div>
</div>


<div class="sidebar2">
<center>


</center>
</div>

<div class="sidebar"  style="margin-bottom:2px;">
<div class="sidebar_title">网站使用教程</div>
<div class="sidebar_info">
<a href="http://www.guoxuedashi.com/help/gjsearch.php" target="_blank">如何在国学大师网下载古籍?</a><br>
<a href="http://www.guoxuedashi.com/zidian/bujian/bjjc.php" target="_blank">如何使用部件查字法快速查字?</a><br>
<a href="http://www.guoxuedashi.com/search/sjc.php" target="_blank">如何在指定的书籍中全文检索?</a><br>
<a href="http://www.guoxuedashi.com/search/skjc.php" target="_blank">如何找到一句话在《四库全书》哪一页?</a><br>
</div>
</div>


<div class="sidebar">
<div class="sidebar_title">热门书籍</div>
<div class="sidebar_info">
<a href="/so.php?sokey=%E8%B5%84%E6%B2%BB%E9%80%9A%E9%89%B4&kt=1">资治通鉴</a> <a href="/24shi/"><strong>二十四史</strong></a>&nbsp; <a href="/a2694/">野史</a>&nbsp; <a href="/SiKuQuanShu/"><strong>四库全书</strong></a>&nbsp;<a href="http://www.guoxuedashi.com/SiKuQuanShu/fanti/">繁体</a>
<br><a href="/so.php?sokey=%E7%BA%A2%E6%A5%BC%E6%A2%A6&kt=1">红楼梦</a> <a href="/a/1858x/">三国演义</a> <a href="/a/1038k/">水浒传</a> <a href="/a/1046t/">西游记</a> <a href="/a/1914o/">封神演义</a>
<br>
<a href="http://www.guoxuedashi.com/so.php?sokeygx=%E4%B8%87%E6%9C%89%E6%96%87%E5%BA%93&submit=&kt=1">万有文库</a> <a href="/a/780t/">古文观止</a> <a href="/a/1024l/">文心雕龙</a> <a href="/a/1704n/">全唐诗</a> <a href="/a/1705h/">全宋词</a>
<br><a href="http://www.guoxuedashi.com/so.php?sokeygx=%E7%99%BE%E8%A1%B2%E6%9C%AC%E4%BA%8C%E5%8D%81%E5%9B%9B%E5%8F%B2&submit=&kt=1"><strong>百衲本二十四史</strong></a>  <a href="http://www.guoxuedashi.com/so.php?sokeygx=%E5%8F%A4%E4%BB%8A%E5%9B%BE%E4%B9%A6%E9%9B%86%E6%88%90&submit=&kt=1"><strong>古今图书集成</strong></a>
<br>

<a href="http://www.guoxuedashi.com/so.php?sokeygx=%E4%B8%9B%E4%B9%A6%E9%9B%86%E6%88%90&submit=&kt=1">丛书集成</a> 
<a href="http://www.guoxuedashi.com/so.php?sokeygx=%E5%9B%9B%E9%83%A8%E4%B8%9B%E5%88%8A&submit=&kt=1"><strong>四部丛刊</strong></a>  
<a href="http://www.guoxuedashi.com/so.php?sokeygx=%E8%AF%B4%E6%96%87%E8%A7%A3%E5%AD%97&submit=&kt=1">說文解字</a> <a href="http://www.guoxuedashi.com/so.php?sokeygx=%E5%85%A8%E4%B8%8A%E5%8F%A4&submit=&kt=1">三国六朝文</a>
<br><a href="http://www.guoxuedashi.com/so.php?sokeytm=%E6%97%A5%E6%9C%AC%E5%86%85%E9%98%81%E6%96%87%E5%BA%93&submit=&kt=1"><strong>日本内阁文库</strong></a> <a href="http://www.guoxuedashi.com/so.php?sokeytm=%E5%9B%BD%E5%9B%BE%E6%96%B9%E5%BF%97%E5%90%88%E9%9B%86&ka=100&submit=">国图方志合集</a> <a href="http://www.guoxuedashi.com/so.php?sokeytm=%E5%90%84%E5%9C%B0%E6%96%B9%E5%BF%97&submit=&kt=1"><strong>各地方志</strong></a>

</div>
</div>


<div class="sidebar2">
<center>

</center>
</div>
<div class="sidebar greenbar">
<div class="sidebar_title green">四库全书</div>
<div class="sidebar_info">

《四库全书》是中国古代最大的丛书,编撰于乾隆年间,由纪昀等360多位高官、学者编撰,3800多人抄写,费时十三年编成。丛书分经、史、子、集四部,故名四库。共有3500多种书,7.9万卷,3.6万册,约8亿字,基本上囊括了古代所有图书,故称“全书”。<a href="http://www.guoxuedashi.com/SiKuQuanShu/">详细>>
</a>

</div> 
</div>

</div>  <!--end r-->

</div>
<!-- 内容区END --> 

<!-- 页脚开始 -->
<div class="shh">

</div>

<div class="w1180" style="margin-top:8px;">
<center><script src="http://www.guoxuedashi.com/img/plus.php?id=3"></script></center>
</div>
<div class="w1180 foot">
<a href="/b/thanks.php">特别致谢</a> | <a href="javascript:window.external.AddFavorite(document.location.href,document.title);">收藏本站</a> | <a href="#">欢迎投稿</a> | <a href="http://www.guoxuedashi.com/forum/">意见建议</a> | <a href="http://www.guoxuemi.com/">国学迷</a> | <a href="http://www.shuowen.net/">说文网</a><script language="javascript" type="text/javascript" src="https://js.users.51.la/17753172.js"></script><br />
  Copyright &copy; 国学大师 古典图书集成 All Rights Reserved.<br>
  
  <span style="font-size:14px">免责声明:本站非营利性站点,以方便网友为主,仅供学习研究。<br>内容由热心网友提供和网上收集,不保留版权。若侵犯了您的权益,来信即刪。scp168@qq.com</span>
  <br />
ICP证:<a href="http://www.beian.miit.gov.cn/" target="_blank">鲁ICP备19060063号</a></div>
<!-- 页脚END --> 
<script src="http://www.guoxuedashi.com/img/plus.php?id=22"></script>
<script src="http://www.guoxuedashi.com/img/tongji.js"></script>

</body>
</html>
