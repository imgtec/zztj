<!DOCTYPE html PUBLIC "-//W3C//DTD XHTML 1.0 Transitional//EN" "http://www.w3.org/TR/xhtml1/DTD/xhtml1-transitional.dtd">
<html xmlns="http://www.w3.org/1999/xhtml">
<head>
<meta http-equiv="Content-Type" content="text/html; charset=utf-8" />
<meta http-equiv="X-UA-Compatible" content="IE=Edge,chrome=1">
<title>資治通鑒_48-資治通鑑卷四十七_48-資治通鑑卷四十七</title>
<meta name="Keywords" content="資治通鑒_48-資治通鑑卷四十七_48-資治通鑑卷四十七">
<meta name="Description" content="資治通鑒_48-資治通鑑卷四十七_48-資治通鑑卷四十七">
<meta http-equiv="Cache-Control" content="no-transform" />
<meta http-equiv="Cache-Control" content="no-siteapp" />
<link href="/img/style.css" rel="stylesheet" type="text/css" />
<script src="/img/m.js?2020"></script> 
</head>
<body>
 <div class="ClassNavi">
<a  href="/24shi/">二十四史</a> | <a href="/SiKuQuanShu/">四库全书</a> | <a href="http://www.guoxuedashi.com/gjtsjc/"><font  color="#FF0000">古今图书集成</font></a> | <a href="/renwu/">历史人物</a> | <a href="/ShuoWenJieZi/"><font  color="#FF0000">说文解字</a></font> | <a href="/chengyu/">成语词典</a> | <a  target="_blank"  href="http://www.guoxuedashi.com/jgwhj/"><font  color="#FF0000">甲骨文合集</font></a> | <a href="/yzjwjc/"><font  color="#FF0000">殷周金文集成</font></a> | <a href="/xiangxingzi/"><font color="#0000FF">象形字典</font></a> | <a href="/13jing/"><font  color="#FF0000">十三经索引</font></a> | <a href="/zixing/"><font  color="#FF0000">字体转换器</font></a> | <a href="/zidian/xz/"><font color="#0000FF">篆书识别</font></a> | <a href="/jinfanyi/">近义反义词</a> | <a href="/duilian/">对联大全</a> | <a href="/jiapu/"><font  color="#0000FF">家谱族谱查询</font></a> | <a href="http://www.guoxuemi.com/hafo/" target="_blank" ><font color="#FF0000">哈佛古籍</font></a> 
</div>

 <!-- 头部导航开始 -->
<div class="w1180 head clearfix">
  <div class="head_logo l"><a title="国学大师官网" href="http://www.guoxuedashi.com" target="_blank"></a></div>
  <div class="head_sr l">
  <div id="head1">
  
  <a href="http://www.guoxuedashi.com/zidian/bujian/" target="_blank" ><img src="http://www.guoxuedashi.com/img/top1.gif" width="88" height="60" border="0" title="部件查字,支持20万汉字"></a>


<a href="http://www.guoxuedashi.com/help/yingpan.php" target="_blank"><img src="http://www.guoxuedashi.com/img/top230.gif" width="600" height="62" border="0" ></a>


  </div>
  <div id="head3"><a href="javascript:" onClick="javascript:window.external.AddFavorite(window.location.href,document.title);">添加收藏</a>
  <br><a href="/help/setie.php">搜索引擎</a>
  <br><a href="/help/zanzhu.php">赞助本站</a></div>
  <div id="head2">
 <a href="http://www.guoxuemi.com/" target="_blank"><img src="http://www.guoxuedashi.com/img/guoxuemi.gif" width="95" height="62" border="0" style="margin-left:2px;" title="国学迷"></a>
  

  </div>
</div>
  <div class="clear"></div>
  <div class="head_nav">
  <p><a href="/">首页</a> | <a href="/ShuKu/">国学书库</a> | <a href="/guji/">影印古籍</a> | <a href="/shici/">诗词宝典</a> | <a   href="/SiKuQuanShu/gxjx.php">精选</a> <b>|</b> <a href="/zidian/">汉语字典</a> | <a href="/hydcd/">汉语词典</a> | <a href="http://www.guoxuedashi.com/zidian/bujian/"><font  color="#CC0066">部件查字</font></a> | <a href="http://www.sfds.cn/"><font  color="#CC0066">书法大师</font></a> | <a href="/jgwhj/">甲骨文</a> <b>|</b> <a href="/b/4/"><font  color="#CC0066">解密</font></a> | <a href="/renwu/">历史人物</a> | <a href="/diangu/">历史典故</a> | <a href="/xingshi/">姓氏</a> | <a href="/minzu/">民族</a> <b>|</b> <a href="/mz/"><font  color="#CC0066">世界名著</font></a> | <a href="/download/">软件下载</a>
</p>
<p><a href="/b/"><font  color="#CC0066">历史</font></a> | <a href="http://skqs.guoxuedashi.com/" target="_blank">四库全书</a> |  <a href="http://www.guoxuedashi.com/search/" target="_blank"><font  color="#CC0066">全文检索</font></a> | <a href="http://www.guoxuedashi.com/shumu/">古籍书目</a> | <a   href="/24shi/">正史</a> <b>|</b> <a href="/chengyu/">成语词典</a> | <a href="/kangxi/" title="康熙字典">康熙字典</a> | <a href="/ShuoWenJieZi/">说文解字</a> | <a href="/zixing/yanbian/">字形演变</a> | <a href="/yzjwjc/">金 文</a> <b>|</b>  <a href="/shijian/nian-hao/">年号</a> | <a href="/diming/">历史地名</a> | <a href="/shijian/">历史事件</a> | <a href="/guanzhi/">官职</a> | <a href="/lishi/">知识</a> <b>|</b> <a href="/zhongyi/">中医中药</a> | <a href="http://www.guoxuedashi.com/forum/">留言反馈</a>
</p>
  </div>
</div>
<!-- 头部导航END --> 
<!-- 内容区开始 --> 
<div class="w1180 clearfix">
  <div class="info l">
   
<div class="clearfix" style="background:#f5faff;">
<script src='http://www.guoxuedashi.com/img/headersou.js'></script>

</div>
  <div class="info_tree"><a href="http://www.guoxuedashi.com">首页</a> > <a href="/SiKuQuanShu/fanti/">四库全书</a>
 > <h1>资治通鉴</h1> <!--         下载:【右键另存为】即可 --></div>
  <div class="info_content zj clearfix">
  
<div class="info_txt clearfix" id="show">
<center style="font-size:24px;">48-資治通鑑卷四十七</center>
    資治通鑑卷四十七   宋 司馬光 撰<br />
<br />
  胡三省 音註<br />
<br />
  漢紀三十九【起旃蒙作噩盡重光單閼凡七年】<br />
<br />
  肅宗孝章皇帝下<br />
<br />
  元和二年春正月乙酉詔曰令云民有產子者復勿筭三歲【復方目翻復其夫勿輸筭也】今諸懷者【賢曰孕也音壬】賜胎養穀人三斛復其夫勿筭一歲著以為令又詔三公曰安靜之吏悃愊無華【說文曰悃愊至誠也悃音苦本翻愊音孚逼翻】日計不足月計有餘【莊子有是言此謂以日計功若不足者然久而計之則民安其生家給人足固有餘矣】如襄城令劉方【襄城縣屬潁川郡】吏民同聲謂之不煩雖未有他異斯亦殆近之矣【近其靳翻】夫以苛為察以刻為明以輕為德以重為威四者或興則下有怨心吾詔書數下冠蓋接道【冠蓋接道謂奉詔出使者相接於道也數所角翻】而吏不加治民或失職其咎安在勉思舊令稱朕意焉【舊令謂故府之籍所疏載者稱尺證翻】北匈奴大人車利涿兵等【車昌遮翻】亡來入塞凡七十三輩時北虜衰耗黨衆離畔南部攻其前丁零寇其後鮮卑擊其左西域侵其右不復自立【復扶又翻】乃遠引而去 南單于長死單于汗之子宣立為伊屠於閭鞮單于【屠直於翻鞮丁奚翻】太初歷施行百餘年歷稍後天【謂七曜之行在歷家所推步躔次之前晦】<br />
<br />
  【朔弦望不合也】上命治歷編訢李梵等綜校其狀【治直之翻訢音欣梵扶中翻】作四分歷 【考異曰按王莽初已廢太初用三統歷今云太初歷失天益遠蓋光武中興廢莽歷復用太初也續漢志又云自太初元年始用三統歷按三統歷劉歆所造云太初元年始用誤也】二月甲寅始施行之 帝之為太子也受尚書於東郡太守汝南張酺【續漢志東郡去雒陽八百餘里酺薄乎翻】丙辰帝東廵幸東郡引酺及門生并郡縣掾史並會庭中【東郡庭也掾俞絹翻】帝先備弟子之儀使酺講尚書一篇然後脩君臣之禮賞賜殊特莫不沾洽行過任城幸鄭均舍賜尚書祿以終其身時人號為白衣尚書【先是均事帝為尚書數納忠言帝敬重之謝病歸任城今祿以尚書任音壬】 乙丑帝耕於定陶辛未幸泰山柴告岱宗【書舜典至于岱宗柴孔安國注曰泰山為四岳所宗燔柴祭天告至】進幸奉高壬申宗祀五帝于汶上明堂【汶上明堂武帝所作在奉高縣西南四里汶音問】丙子赦天下進幸濟南【濟南國在雒陽東千八百里賢曰濟南故城在淄州長山縣西北濟子禮翻】三月己丑幸魯庚寅祀孔子於闕里【續漢志魯縣古曲阜有闕里孔子所居】及七十二弟子【自顔回以下七十餘人】作六代之樂【黄帝曰雲門堯曰咸池舜曰大韶禹曰大夏湯曰大濩周曰大武】大會孔氏男子二十以上者六十二人帝謂孔僖曰今日之會寧於卿宗有光榮乎對曰臣聞明王聖主莫不尊師貴道今陛下親屈萬乘辱臨敝里此乃崇禮先師增煇聖德【先師謂孔子】至於光榮非所敢承帝大笑曰非聖者子孫焉有斯言乎【焉於䖍翻】拜僖郎中 壬辰帝幸東平追念獻王謂其諸子曰思其人至其鄉其處在其人亡因泣下沾襟遂幸獻王陵【賢曰陵在今鄆州峗山南峗音魚委翻】祠以太牢親拜祠坐【坐徂臥翻】哭泣盡哀獻王之歸國也【事見四十二卷明帝永平四年】驃騎府吏丁牧周栩以獻王愛賢下士不忍去之遂為王家大夫數十年事祖及孫【獻王及子懷王忠及今王敞栩况羽翻下遐稼翻】帝聞之皆引見【見賢遍翻】既愍其淹滯且欲揚獻王德美即皆擢為議郎乙未幸東阿北登太行山至天井關【行戶剛翻】夏四月乙卯還宫庚申假于祖禰【虞書一歲廵四岳歸格于藝祖孔安國注曰廵狩四岳然後歸告至文祖之廟賢曰假至也音格禰父廟】 五月徙江陵王恭為六安王【恭封六安王以廬江郡為國在雒陽東一千七百里】秋七月庚子詔曰春秋重三正慎三微【賢曰三正謂天地人之正所以有三者由有三微之月王者所當奉而成之禮記曰正朔三而改文質再而復三微者三正之始萬物皆微物色不同故王者取法焉十一月時陽氣始施於黄泉之下色皆赤赤者陽氣故周為天正色尚赤十二月萬物始芽而色白白者隂氣故殷為地正色尚白十三月萬物莩甲而出其色皆黑人得加功展業故夏為人正色尚黑尚書大傳曰夏以十三月為正平旦為朔殷以十二月為正鷄鳴為朔周以十一月為正夜半為朔必以三微之月為正者當爾之時物皆尚微王者受命當扶微理弱奉承之義也】其定律無以十一月十二月報囚止用冬初十月而已 冬南單于遣兵與北虜温禺犢王戰於涿邪山斬獲而還武威太守孟雲上言北虜以前既和親而南部復往抄掠【復扶又翻】北單于謂漢欺之謀欲犯塞謂宜還南所掠生口以慰安其意詔百官議於朝堂【朝直遥翻】太尉鄭弘司空第五倫以為不可許司徒桓虞及太僕袁安以為當與之弘因大言激厲虞曰諸言當還生口者皆為不忠虞廷叱之倫及大鴻臚韋彪皆作色變容【臚陵如翻】司隸校尉舉奏弘等弘等皆上印綬謝詔報曰久議沈滯【沈持林翻】各有所志蓋事以議從策由衆定誾誾衎衎得禮之容【賢曰誾誾忠正貌衎衎和樂貌誾魚巾翻衎音侃又苦旦翻】寑嘿抑心更非朝廷之福【寑息也】君何尤而深謝其各冠履帝乃下詔曰江海所以長百川者以其下之也【老子曰江海所以為百谷王者以其善下也長知兩翻下遐稼翻】少加屈下尚何足病况今與匈奴君臣分定【少詩沼翻分扶問翻】辭順約明貢獻累至豈宜違信自受其曲其敕度遼及領中郎將龎奮倍雇南部所得生口以還北虜【領中郎將領護匈奴中郎將也賢曰雇賞報也】其南部斬首獲生計功受賞如常科<br />
<br />
  三年春正月丙申帝北廵辛丑耕于懷二月乙丑敇侍御史司空曰方春所過毋得有所伐殺車可以引避引避之騑馬可輟解輟解之【侍御史掌舉劾司空掌土功車駕行幸則侍御史掌舉劾道路之不如法司空帥工徒治道路修橋梁故皆敕之賢曰夹轅為服馬服馬外為騑馬孔頴達曰車有一轅而四馬駕之中央兩馬夹轅者名服馬兩邊名騑馬亦曰驂馬騑音非】戊辰進幸中山出長城【賢曰史記蒙恬為秦築長城西自臨洮東至海余謂此非秦長城蓋趙所築長城也】癸酉還幸元氏三月己卯進幸趙【趙國在雒陽北一千一百里】辛卯還宫太尉鄭弘數陳侍中竇憲權勢太盛【數所角翻】言甚苦切憲疾之會弘奏憲黨尚書張林雒陽令楊光在官貪殘書奏吏與光故舊因以告之光報憲憲奏弘大臣漏泄密事帝詰讓弘【詰去吉翻】夏四月丙寅收弘印綬弘自詣廷尉詔敇出之因乞骸骨歸未許病篤上書陳謝曰竇憲姦惡貫天達地海内疑惑賢愚疾惡【惡烏路翻】謂憲何術以迷主上近日王氏之禍昞然可見【謂王氏以戚屬而成簒國之禍昞音炳】陛下處天子之尊【處昌呂翻】保萬世之祚而信讒佞之臣不計存亡之機臣雖命在晷刻死不忘忠願陛下誅四凶之罪以猒人鬼憤結之望【猒一艷翻滿也 考異曰袁紀云弘為尚書僕射烏孫王遣子入侍上問弘當答其使否弘對曰烏孫前為大單于所攻陛下使小單于往救之尚未賞今如答之小單于不當怨乎上以弘議問侍中竇憲對曰禮有往來弘章句諸生不達國體上遂答烏孫小單于忿恚攻金城郡殺太守任昌上謂弘曰朕前不從君議果如此弘對曰竇憲姦臣也有少正卯之行未被兩觀之誅陛下前何為用其議按肅宗時無小單于寇金城事今不取】帝省章遣醫視弘病比至已薨【省悉景翻比必寐翻】 以大司農宋由為太尉 司空第五倫以老病乞身【委身以事君則身非我有故於其老而乞退也謂之乞身猶言乞骸骨也】五月丙子賜策罷以二千石俸終其身倫奉公盡節言事無所依違【若依若違兩可不決之論也】性質慤少文采【少詩沼翻】在位以貞白稱或問倫曰公有私乎對曰昔人有與吾千里馬者吾雖不受每三公有所選舉心不能忘亦終不用也若是者豈可謂無私乎 以太僕袁安為司空 秋八月乙丑帝幸安邑觀鹽池【安邑縣屬河東郡鹽池在縣西南楊佺期洛陽記曰河東鹽池長七十里廣七里水氣紫色許慎曰河東鹽池袤五十一里廣七里周百一十六里酈道元曰安邑鹽池上承鹽水水出東南薄山西北流逕巫咸山北又逕安邑故城南又西流注於鹽池水出石鹽自然即成朝取夕復終無減損唯山暴雨澍甘澤潢潦奔逸則鹽池用耗故公私共堨水逕防其淫濫故謂之鹽水亦為堨水也池西又有一池謂之女鹽澤東西二十五里南北二十里在猗氏故城南土人鄉俗引水裂沃麻分灌川野畦水耗竭土自成鹽即所謂鹹醝也而味苦賢曰在今蒲州虞鄉縣西】九月還宫 燒當羌迷吾復與弟號吾及諸種反【復扶又翻種章勇翻】號吾先輕入寇隴西界督烽掾季章追之【督烽掾郡掾之督烽燧者】生得號吾將詣郡號吾曰獨殺我無損於羌誠得生歸必悉罷兵不復犯塞隴西太守張紆放遣之羌即為解散【為于偽翻】各歸故地迷吾退居河北歸義城【河北逢留大河之北也歸義城本漢所築以招來諸羌之歸義者】 疏勒王忠從康居王借兵還據損中【忠叛見上卷元年賢曰損中未詳東觀記作頓中續漢書及華嶠書並作損中本或作楨未知孰是余按西域傳靈帝建寧三年凉州刺史孟佗遣兵討疏勒攻楨中城楨中是也】遣使詐降於班超超知其姦而偽許之忠從輕騎詣超超斬之因擊破其衆南道遂通楚許太后薨【楚王英之徙也許太后留楚宫】詔改葬楚王英追爵諡曰楚厲侯【諡法殺戮無辜曰厲】 帝以潁川郭躬為廷尉決獄斷刑【斷丁亂翻】多依矜恕條諸重文可從輕者四十一奏之事皆施行 博士魯國曹褒上疏以為宜定文制著成漢禮太常巢堪【巢姓有巢氏之後春秋有巢牛臣】以為一世大典非褒所定【言非褒所能定】不可許帝知諸儒拘攣【攣呂員翻】難與圖始【賢曰拘攣由拘束也】朝廷禮憲宜以時立乃拜褒侍中玄武司馬班固以為宜廣集諸儒共議得失【百官志玄武司馬主南宫玄武門秩比千石】帝曰諺言作舍道邊三年不成會禮之家名為聚訟【會禮言會而議禮賢曰聚訟言相争不定也】互生疑異筆不得下昔堯作大章一夔足矣【堯作樂曰大章記曰大章章之也賢曰夔堯樂官呂氏春秋曰魯哀公問於孔子曰樂正夔一足矣皇侃曰章明也民樂堯德大明故名樂曰大章】<br />
<br />
  章和元年【是年七月改元】春正月帝召褒受以叔孫通漢儀十二篇【通制漢儀見十卷高帝六年七年其書與律令同藏於理官】曰此制散畧多不合經今宜依禮條正使可施行 護羌校尉傅育欲伐燒當羌為其新降【為于偽翻】不欲出兵乃募人鬭諸羌胡【募人間搆諸羌使之自鬬也】羌胡不肯遂復叛出塞【復扶又翻】更依迷吾育請發諸郡兵數萬人共擊羌未及會三月育獨進軍迷吾聞之徙廬落去【廬穹廬落居也】育遣精騎三千窮追之夜至三兜谷【三兜谷在建威南】不設備迷吾襲擊大破之殺育及吏士八百八十人及諸郡兵到羌遂引去詔以隴西太守張紆為校尉將萬人屯臨羌【紆邕俱翻】 夏六月戊辰司徒桓虞免癸卯以司空袁安為司徒光禄勲任隗為司空隗光之子也【任音壬隗五罪翻】 齊王晃及弟利侯剛【班志利縣屬齊郡晃齊武王縯之曾孫殤王石之子】與母太姬更相誣告【更工衡翻】秋七月癸卯詔貶晃爵為蕪湖侯【賢曰蕪湖縣名屬丹陽郡其故城在今宣州當塗縣東南】削剛戶三千收太姬璽綬【璽斯氏翻綬音受】 壬子淮陽頃王昞薨【昞明帝子】 鮮卑入左地【匈奴左地也】擊北匈奴大破之斬優留單于而還【還從宣翻又如字】 羌豪迷吾復與諸種寇金城塞【復扶又翻種章勇翻下同】張紆遣從事河内司馬防【百官志使匈奴中郎將置從事二人護羌校尉蓋亦置二人也】與戰於木乘谷迷吾兵敗走因譯使欲降紆納之迷吾將人衆詣臨羌紆設兵大會【譯通夷言使之將命因謂之譯使設兵陳兵也使疏吏翻降戶江翻】施毒酒中伏兵殺其酋豪八百餘人【酋慈由翻】斬迷吾頭以祭傅育冢復放兵擊其餘衆斬獲數千人迷吾子迷唐與諸種解仇結婚交質【質音致】據大小榆谷以叛【水經河水逕西海郡南又東逕允川而歷大榆谷小榆谷北二榆土地肥美羌所依阻也】種衆熾盛張紆不能制 壬戌詔以瑞物仍集改元章和【章明也明和氣之致祥也】是時京師四方屢有嘉瑞前後數百千言事者咸以為美而太尉掾平陵何敞獨惡之【惡烏路翻杜佑曰漢武帝割槐里置茂陵邑昭帝又割置平陵邑】謂宋由袁安曰夫瑞應依德而至災異緣政而生今異鳥翔於殿屋怪草生於庭際不可不察由安懼不敢荅 八月癸酉帝南廵戊子幸梁乙未晦幸沛【梁沛二國】 日有食之 九月庚子帝幸彭城辛亥幸壽春【夀春縣屬九江郡】復封阜陵侯延為阜陵王【延貶事見上卷建初元年】己未幸汝陰【汝隂縣屬汝南郡賢曰今潁州縣】冬十月丙子還宫 北匈奴大亂屈蘭儲等五十八部口二十八萬詣雲中五原朔方北地降 曹褒依凖舊典雜以五經䜟記之文撰次天子至於庶人冠婚吉凶終始制度【撰次制度備其終始也䜟楚譛翻撰雛免翻冠古玩翻】凡百五十篇奏之帝以衆論難一故但納之不復令有司平奏【平奏者平其可行與否而奏之復扶又翻】 是歲班超發于窴諸國兵共二萬五千人擊莎車【元和元年超擊莎車未克故也窴徒賢翻莎素禾翻】龜兹王發温宿姑墨尉頭兵合五萬人救之【龜兹音丘慈】超召將校及于窴王議曰【將即亮翻校戶教翻】今兵少不敵其計莫若各散去于窴從是而東長史亦於此西歸【班超時為將兵長史蓋西歸疏勒也】可須夜鼓聲而發【須待也夜鼓聲鼓鼜之聲也周禮軍旅夜鼓鼜注云鼜夜戒守鼓也司馬法曰昏鼔四通為大鼜夜半三通為晨戒旦明五通為發昫所謂三鼜也此則待夜半鼔聲也鼜干歷翻昫休具翻劉休武翻】隂緩所得生口【使生口得歸言將散去也】龜兹王聞之大喜自以萬騎於西界遮超温宿王將八千騎於東界徼于窴【徼一遥翻】超知二虜已出密召諸部勒兵馳赴莎車營胡大驚亂犇走追斬五千餘級莎車遂降【降戶江翻】龜兹等因各退散自是威震西域<br />
<br />
  二年春正月濟南王康阜陵王延中山王焉來朝上性寛仁篤於親親故叔父濟南中山二王每數入朝【濟子禮翻數所角翻朝直遥翻】特加恩寵及諸昆弟並留京師不遣就國【漢制諸藩王朝會之禮畢各就國不得留京師】又賞賜羣臣過於制度倉帑為虚【帑他朗翻為于偽翻】何敞奏記宋由曰比年水旱民不收穫凉州緣邊家被凶害【賢曰時西羌犯邊為害也比毗至翻被皮義翻】中州内郡公私屈竭此實損膳節用之時國恩覆載【言恩同天地也覆敷救翻】賞賚過度但聞臘賜自郎官以上公卿王侯以下至於空竭帑藏【藏徂浪翻】損耗國資尋公家之用皆百姓之力明君賜賚宜有品制忠臣受賞亦應有度【賢曰漢官儀臘賜大將軍三公錢各二十萬牛肉二百斤粳米二百斛特進侯十五萬卿十萬校尉五萬尚書三萬侍中將大夫各二萬千石六百石各七千虎賁羽林郎二人共三千以為祀門戶直】是以夏禹玄圭【書禹貢曰禹錫玄圭】周公束帛【賢曰尚書曰召公出取幣入錫周公】今明公位尊任重責深負大上當匡正綱紀下當濟安元元豈但空空無違而已哉【空當作悾悾悾謹慤也】宜先正已以率羣下還所得賜因陳得失奏王侯就國除苑囿之禁節省浮費賑卹窮孤則恩澤下暢黎庶悦豫矣由不能用 【考異曰敞傳此事在肅宗崩後云竇氏專政外戚奢侈賞賜過制敞奏記云云袁紀在元和三年按敞記云明公視事出入再朞又言臘賜知在此時】尚書南陽宋意上疏曰陛下至孝烝烝【烝進也烝烝進進也】恩愛隆深禮寵諸王同之家人車入殿門【漢制太子諸王至司馬門皆下車故謂止車門】即席不拜【臣於君前拜而後就席】分甘損膳賞賜優渥【損御膳以分甘也】康焉幸以支庶享食大國陛下恩寵踰制禮敬過度春秋之義諸父昆弟無所不臣【君君臣臣不以親厭殺天地之大經也春秋尊王故以為春秋之義】所以尊尊卑卑彊幹弱枝者也陛下德業隆盛當為萬世典法不宜以私恩損上下之序失君臣之正又西平王羨等六王皆妻子成家【謂有妻有子自成一家也】官屬備具【謂王國官已具也】當早就蕃國為子孫基阯而室第相望久磐京邑【賢曰磐謂磐桓不去】驕奢僭擬寵祿隆過宜割情不忍以義斷恩【賢曰禮記曰門内之政恩掩義門外之政義斷恩斷丁亂翻】發遣康焉各歸蕃國令羨等速就便時以塞衆望【賢曰行日取便利之時也塞悉則翻】帝未及遣 壬辰帝崩于章德前殿年三十一遺詔無起寑廟一如先帝法制<br />
<br />
  范曄論曰魏文帝稱明帝察察章帝長者章帝素知人厭明帝苛切事從寛厚奉承明德太后盡心孝道平徭簡賦而民賴其慶又體之以忠恕文之以禮樂謂之長者不亦宜乎<br />
<br />
  太子即位年十歲尊皇后曰皇太后 三月用遺詔徙西平王羨為陳王六安王恭為彭城王【改淮陽為陳國楚郡為彭城國西平併汝南郡六安復為廬江郡】 癸卯葬孝章皇帝于敬陵【敬陵在雒陽城東南三十九里】 南單于宣死單于長之弟屯屠何立為休蘭尸逐侯鞮單于【鞮丁奚翻】 太后臨朝【蔡邕獨斷曰少帝即位太后即代攝政臨前殿朝羣臣太后東面少帝西面羣臣上書奏事皆為兩通一詣太后一詣少帝】竇憲以侍中内幹機密【賢曰幹主也或曰幹古管字也】出宣誥命弟篤為虎賁中郎將篤弟景瓌並為中常侍兄弟皆在親要之地憲客崔駰【駰音因】以書戒憲曰傳曰生而富者驕生而貴者慠【傳直戀翻慠五到翻】生富貴而能不驕慠者未之有也今寵祿初隆百僚觀行【行下孟翻】豈可不庶幾夙夜以永終譽乎【詩周頌振鷺之辭言庶幾于夙夜匪懈以終保令名於有永也】昔馮野王以外戚居位稱為賢臣【馮野王妹為元帝昭儀於九卿中野王行能第一】近陰衛尉克己復禮終受多福【陰衛尉興也謂讓侯爵又讓大司馬也】外戚所以獲譏於時垂愆於後者蓋在滿而不挹位有餘而仁不足也漢興以後迄于哀平外家二十保族全身四人而已【外家二十者呂氏張氏薄氏竇氏王氏陳氏衛氏李氏趙氏上官氏史氏許氏霍氏卬成王氏元后王氏趙氏傅氏丁氏馮氏衛氏也唯文帝薄太后竇后景帝王后卬成王后四人保族全家武帝夫人李氏雖追配武帝昌邑王立未幾而廢非外家當以史皇孫王夫人足二十之數】書曰鑒于有殷【書召誥曰我不可不鑒於有夏亦不可不鑒於有殷】可不慎哉 庚戌皇太后詔以故太尉鄧彪為太傅賜爵關内侯錄尚書事百官總己以聽竇憲以彪有義讓先帝所敬【彪父邯封鄳鄉侯父卒彪讓國於弟鳳顯宗高其節】而仁厚委隨【賢曰委隨猶順從也】故尊崇之其所施為輒外令彪奏内白太后事無不從【王莽用孔光之故智也】彪在位修身而已不能有所匡正憲性果急睚眦之怨莫不報復【賢曰睚音語懈翻眦音仕懈翻廣雅曰睚裂也或謂裂眦瞋目貌也】永平時謁者韓紆考劾憲父勲獄【勲下獄死事見四十五卷明帝永平五年劾戶槩翻又戶得翻】憲遂令客斬紆子以首祭勲冢癸亥陳王羨彭城王恭樂成王黨下邳王衍梁王暢始就國 夏四月戊寅以遺詔罷郡國鹽鐵之禁縱民煮鑄【自武帝以來鹽鐵有禁光武中興收而未罷今縱民得煮鹽鑄鐵】 五月京師旱北匈奴饑亂降南部者歲數千人【降戶江翻下同】秋七月南單于上言宜及北虜分争出兵討伐破北成南共為一國【考異曰袁紀章和元年十月南單于上書求出兵破北成南宋意諫不聽師未出而帝寑疾范書南匈奴傳】<br />
<br />
  【事並在此年七月按單于書云孝章皇帝聖思遠慮則范書是也今從之】令漢家長無北念【謂北部既滅南部保塞則漢家無復北顧以為念也】臣等生長漢地【長知兩翻】開口仰食【仰魚向翻】歲時賞賜動輒億萬雖垂拱安枕慙無報效之義願發國中及諸郡故胡新降精兵【故胡南部舊衆也新降新從北部來降者】分道並出期十二月同會虜地臣兵衆單少不足以防内外【少詩沼翻】願遣執金吾耿秉度遼將軍鄧鴻及西河雲中五原朔方上郡太守【守式又翻】并力而北冀因聖帝威神一舉平定臣國成敗要在今年已勅諸部嚴兵馬唯裁哀省察【省悉景翻】太后以示耿秉【以南單于書示之也】秉上言昔武帝單極天下【單與殫同】欲臣虜匈奴未遇天時事遂無成【謂不能使匈奴臣服也】今幸遭天授北虜分争以夷伐夷【謂以南部伐北部也】國家之利宜可聽許秉因自陳受恩分當出命效用【分扶問翻】太后議欲從之尚書宋意上書曰夫戎狄簡賤禮義無有上下彊者為雄弱即屈服自漢興以來征伐數矣【數所角翻】其所克獲曾不補害光武皇帝躬服金革之難深昭天地之明因其來降羈縻畜養【畜許六翻】邊民得生勞役休息於兹四十餘年矣【建武二十四年受南單于降至是四十一年】今鮮卑奉順斬獲萬數【謂破殺優留單于也】中國坐享大功而百姓不知其勞漢興功烈於斯為盛所以然者夷虜相攻無損漢兵者也臣察鮮卑侵伐匈奴正是利其抄掠及歸功聖朝實由貪得重賞【洞見鮮卑之情抄楚交翻】今若聽南虜還都北庭則不得不禁制鮮卑鮮卑外失暴掠之願内無功勞之賞豺狼貪婪【婪盧含翻方言殺人而取其財曰婪】必為邊患今北虜西遁請求和親宜因其歸附以為外扞巍巍之業無以過此若引兵費賦以順南虜則坐失上畧去安即危矣誠不可許會齊殤王子都鄉侯暢來弔國憂【齊殤王石齊武王縯之孫哀王章之子 考異曰袁紀作郁鄉侯暢今從范書】太后數召見之【范書曰暢素行邪僻因鄧疊母元自通長樂宫得幸太后數所角翻】竇憲懼暢分宫省之權遣客刺殺暢於屯衛之中【何敞傳曰刺殺暢於城門屯衛之中刺七亦翻】而歸罪於暢弟利侯剛乃使侍御史與青州刺史雜考剛等【青州刺史部齊國暢見殺於京師而令青州刺史考竟欲移獄以絶蹤也】尚書潁川韓稜以為賊在京師不宜捨近問遠恐為姦臣所笑太后怒以切責稜稜固執其議何敞說宋由曰【說輸苪翻】暢宗室肺府【府與腑同】茅土藩臣來弔大憂上書須報【賢曰須待也】親在武衛致此殘酷奉憲之吏莫適討捕【賢曰適音的謂無指的討捕也】蹤跡不顯主名不立敞備數股肱職典賊曹【賢曰股肱謂手臂也公府有賊曹主知盜賊余按字書股髀幹肱臂幹股肱言手足之要以為手臂誤矣】欲親至發所以糾其變【發所賊發之所糾督察也】而二府執事以為三公不與賊盜【賢曰敞在太尉府二府謂司徒司空邴吉為丞相不案事遂以為故事與讀曰預】公縱姦慝莫以為咎敞請獨奏案之由乃許焉二府聞敞行皆遣主者隨之【賢曰主者謂主知賊盜之曹也】於是推舉其得事實太后怒閉憲於内宫憲懼誅因自求擊匈奴以贖死冬十月乙亥以憲為車騎將軍伐北匈奴以執金吾耿秉為副發北軍五校黎陽雍營緣邊十二郡騎士及羌胡兵出塞【北軍五校屯騎越騎步兵長水射聲五校尉所掌宿衛兵也黎陽營注見前扶風校尉部在雍縣以凉州近羌數犯三輔將兵衛護園陵故俗稱雍營緣邊十二郡上郡西河五原雲中定襄鴈門朔方代郡上谷漁陽安定北地也校戶教翻雍於用翻】 公卿舉故張掖太守鄧訓代張紆為護羌校尉迷唐率兵萬騎來至塞下未敢攻訓先欲脅小月氏胡【匈奴破月氏月氏西徙其餘衆保南山不得去者號小月氏氏音支】訓擁衛小月氏胡令不得戰議者咸以羌胡相攻縣官之利不宜禁護訓曰張紆失信衆羌大動凉州吏民命縣絲髪【縣讀曰懸】原諸胡所以難得意者皆恩信不厚耳今因其廹急以德懷之庶能有用遂令開城及所居園門【護羌校尉所居寺舍後園之門也】悉驅羣胡妻子内之嚴兵守衛羌掠無所得又不敢逼諸胡因即解去由是湟中諸胡皆言漢家常欲鬭我曹【賢曰湟中月氏胡所居今鄯州湟水縣也】今鄧使君待我以恩信開門内我妻子乃是得父母也咸歡喜叩頭曰唯使君所命訓遂撫養教諭大小莫不感悦於是賞賂諸羌種使相招誘【誘音酉】迷唐叔父號吾將其種人八百戶來降【種章勇翻】訓因發湟中秦胡羌兵四千人出塞【秦威服四夷故夷人率謂中國人為秦人】掩擊迷唐於寫谷破之【賢曰東觀記曰寫作鴈】迷唐乃去大小榆【大小榆谷杜佑曰大小榆谷在漢榆中縣今在蘭州五泉縣界按水經大小榆谷在漢金城郡塞外河水過大小榆谷北又東過河關縣北又東過允吾縣北又東過榆中縣北榆中縣與大小榆相去甚遠杜佑說非】居頗巖谷衆悉離散<br />
<br />
  孝和皇帝上【諱肇肅宗第四子也竇后養以為子廢長立之諡法不剛不柔曰和伏侯古今註曰肇之字曰始音兆賢曰案許慎說文肇音大可翻上諱也但伏侯許慎並漢時人而帝諱音不同蓋應别有所據】<br />
<br />
  永元元年春迷唐欲復歸故地鄧訓發湟中六千人令長史任尚將之【將即亮翻】縫革為船置於箄上以度河【賢曰箄木筏也音步佳翻】掩擊迷唐大破之斬首前後一千八百餘級獲生口二千人馬牛羊三萬餘頭一種殆盡【賢曰一種謂迷唐也種章勇翻 考異曰西羌傳永元元年張紆坐徵以訓代為校尉鄧訓傳章和二年紆誘誅羌羌謀報怨公卿舉訓代紆擊破之其春迷唐復欲歸訓又破之按訓傳下云永元二年則其春永元元年春也今從訓傳】迷唐收其餘衆西徙千餘里諸附落小種皆畔之【附落羌部落之附迷唐者】燒當豪帥東號稽顙歸死【歸死自歸而請死也帥所類翻】餘皆欵塞納質【質音致】於是訓綏接歸附威信大行遂罷屯兵各令歸郡【以羌反發諸郡兵屯於塞上今羌已破罷令各歸其郡】唯置弛刑徒二千餘人分以屯田修理塢壁而已 竇憲將征匈奴三公九卿詣朝堂上書諫以為匈奴不犯邊塞而無故勞師遠涉損費國用徼功萬里【徼一遥翻】非社稷之計書連上輒寑【上時掌翻下同】宋由懼遂不敢復署議【復扶又翻】而諸卿稍自引止唯袁安任隗守正不移至免冠朝堂固争前後且十上衆皆為之危懼【為于偽翻下同】安隗正色自若侍御史魯恭上疏曰國家新遭大憂陛下方在諒闇【闇音隂】百姓闕然三時不聞警蹕之音【賢曰三時夏秋冬也天子出警入蹕沈約曰漢制曰出稱警入稱蹕而今則并稱之史臣以為警者警戒也蹕者止行也今從乘輿而出者並警戒以備非常也從外而入與乘輿相干者蹕而止之也和帝章和二年二月即位明年春議擊匈奴帝在諒闇不出故三時不聞警蹕之音】莫不懷思皇皇若有求而不得【禮記顔丁善居喪始死皇皇如有求而不得此言百姓思慕之意】今乃以盛春之月興發軍役擾動天下以事戎夷誠非所以垂恩中國改元正時由内及外也萬民者天之所生天愛其所生猶父母愛其子一物有不得其所則天氣為之舛錯况於人乎故愛民者必有天報夫戎狄者四方之異氣與鳥獸無别【别彼列翻】若雜居中國則錯亂天氣汙辱善人【汙烏故翻】是以聖王之制羈縻不絶而已【字書曰羈馬絡頭也蒼頡篇曰縻牛韁也】今匈奴為鮮卑所破遠藏於史侯河西去塞數千里而欲乘其虛耗利其微弱是非義之所出也今始徵發而大司農調度不足【調徒弔翻賢曰度音大各翻余據今人多讀如本字】上下相廹民間之急亦已甚矣羣僚百姓咸曰不可陛下奈何以一人之計棄萬人之命不卹其言乎上觀天心下察人志足以知事之得失臣恐中國不為中國豈徒匈奴而已哉尚書令韓稜騎都尉朱暉議郎京兆樂恢皆上疏諫太后不聽又詔使者為憲弟篤景並起邸第勞役百姓【爲于偽翻下同】侍御史何敞上疏曰臣聞匈奴之為桀逆久矣平城之圍【事見十一卷高帝七年】嫚書之恥【事見十二卷惠帝三年】此二辱者臣子所為捐軀而必死高祖呂后忍怒含忿舍而不誅【舍讀曰捨】今匈奴無逆節之罪漢朝無可慙之恥【朝直遥翻下同】而盛春東作【賢曰歲起於東人始就耕故曰東作】興動大役元元怨恨咸懷不悦又猥為衛尉篤奉車都尉景繕修館第彌街絶里篤景親近貴臣當為百僚表儀今衆軍在道朝廷焦唇百姓愁苦縣官無用【無財用也】而遽起大第崇飾玩好【好呼到翻】非所以垂令德示無窮也宜且罷工匠專憂北邊恤民之困書奏不省【省悉景翻】竇憲嘗使門生齎書詣尚書僕射郅壽有所請託壽即送詔獄前後上書陳憲驕恣引王莽以誡國家又因朝會刺譏憲等以伐匈奴起第宅事厲音正色辭旨甚切憲怒䧟壽以買公田誹謗下吏當誅【下遐稼翻】何敞上疏曰壽機密近臣匡救為職若懷默不言其罪當誅今壽違衆正議以安宗廟豈其私邪臣所以觸死瞽言【論語曰侍於君子有三愆未見顔色而言謂之瞽】非為壽也忠臣盡節以死為歸臣雖不知壽度其甘心安之【度徒洛翻】誠不欲聖朝行誹謗之誅以傷晏晏之化【鄭玄注尚書考靈曜曰寛容覆載謂之晏晏】杜塞忠直【塞悉則翻】垂譏無窮臣敞謬與機密【與讀曰預】言所不宜罪名明白當填牢獄先壽僵仆【先悉薦翻】萬死有餘書奏壽得減死論徙合浦未行自殺壽惲之子也【郅惲事光武惲於粉翻】夏六月竇憲耿秉出朔方雞鹿塞【賢曰今在朔方窳渾縣北闞駰十三州志曰窳渾縣有大道西北出雞鹿塞窳音羊主翻】南單于出滿夷谷【賢曰滿夷谷闕余按南單于庭在西河美稷滿夷谷當在美稷縣西北後鄧鴻討逢侯兵至美稷逢侯乘冰度隘向滿夷谷可以知矣】度遼將軍鄧鴻出稒陽塞【賢曰稒陽縣屬九原郡故城在今勝州銀城縣界稒音固】皆會涿邪山憲分遣副校尉閻盤司馬耿夔耿譚將南匈奴精騎萬餘與北單于戰於稽洛山【余按唐太宗以斛薩部地置稽落州蓋因山以名之】大破之單于遁走追擊諸部遂臨私渠北鞮海【鞮丁奚翻】斬名王已下萬三千級獲生口甚衆雜畜百餘萬頭諸裨小王率衆降者前後八十一部二十餘萬人憲秉出塞三千餘里登燕然山【唐太宗又以多濫葛部地置燕然州又按北史燕然山在菟園水北燕於賢翻】命中護軍班固刻石勒功【西都有護軍都尉今始有中護軍】紀漢威德而還【還從宣翻又如字下同】遣軍司馬吳汜梁諷奉金帛遺北單于【遺于季翻】時虜中乖亂汜諷及北單于於西海上宣國威信以詔致賜單于稽首拜受【稽音啟】諷因說令修呼韓邪故事【謂臣服于漢為北藩說輸芮翻】單于喜悦即將其衆與諷俱還到私渠海聞漢軍已入塞乃遣弟右温禺鞮王奉貢入侍隨諷詣闕憲以單于不自身到奏還其侍弟 秋七月乙未會稽山崩【會工外翻】 九月庚申以竇憲為大將軍中郎將劉尚為車騎將軍封憲武陽侯【郡國志東郡有東武陽縣泰山郡有南武陽侯國憲其封南武陽歟】食邑二萬戶憲固辭封爵詔許之舊大將軍位在三公下至是詔憲位次太傅下三公上長史司馬秩中二千石【太傳位上公則憲亦班於上公矣大將軍長史司馬秩千石今秩中二千石則亦比九卿矣】封耿秉為美陽侯【美陽縣屬扶風】竇氏兄弟驕縱而執金吾景尤甚奴客緹騎強奪人財貨簒取罪人妻畧婦女【賢曰漢官儀執金吾緹騎二百人說文曰緹丹黄色也言奴客及緹騎並為縱横也緹社兮翻又他禮翻】商賈閉塞【賈音古塞悉則翻】如避寇讐又擅發緣邊諸郡突騎有才力者有司莫敢舉奏袁安劾景擅發邊民驚惑吏民二千石不待符信【符信謂虎符以為信也劾戶槩翻又戶得翻下同】而輒承景檄當伏顯誅又奏司隸校尉河南尹阿附貴戚不舉劾請免官案罪並寢不報駙馬都尉瓌獨好經書節約自修【瓌古回翻好呼到翻】尚書何敞上封事曰昔鄭武姜之幸叔段【賢曰鄭武姜愛少子叔段鄭莊公立武姜請以京封共叔段謂之京城太叔後武姜引以襲鄭莊公伐之出奔共】衛莊公之寵州吁【賢曰衛莊公寵庶子州吁州吁好兵公弗禁石碏諫不聽及桓公立州吁乃弑桓公而簒】愛而不教終至凶戾由是觀之愛子若此猶飢而食之以毒【食讀曰飤】適所以害之也伏見大將軍憲始遭大憂公卿比奏【賢曰比頻也音毗至翻】欲令典幹國事憲深執謙退固辭盛位懇懇勤勤言之深至天下聞之莫不說喜今踰年未幾【說讀曰悦幾居豈翻】入禮未終卒然中改【禮事君方喪三年時遭國憂纔踰年故曰入禮未終卒讀曰猝】兄弟專朝【朝直遥翻】憲秉三軍之重篤景總宫衛之權而虐用百姓奢侈僭偪誅戮無罪肆心自快今者論議訩訩【訩許容翻又許勇翻】咸謂叔段州吁復生於漢【復扶又翻】臣觀公卿懷持兩端不肯極言者以為憲等若有匪懈之志則已受吉甫褒申伯之功【賢曰申伯周宣王元舅有令德故尹吉甫作詩以美之】如憲等䧟於罪辜則自取陳平周勃順呂后之權【事見高后紀】終不以憲等吉凶為憂也【此言曲盡當時廷臣之情嗚呼豈特當時哉】臣敞區區誠欲計策兩安絶其緜緜塞其㳙㳙【周金人銘曰㳙㳙不壅終為江河緜緜不絶或成網羅塞悉則翻㳙主淵翻】上不欲令皇太后損文母之號陛下有誓泉之譏【詩曰思齊太任文王之母左傳武姜啟叔段襲鄭莊公寘姜氏於城潁而誓之曰不及黄泉無相見也】下使憲等得長保其福祐也駙馬都尉比請退身願抑家權【願抑其家不與之以權比毗至翻】可與參謀聽順其意【漢之外戚傅喜竇鄧康咸能履盛滿而思謙挹然終不能全其家門十分之一蓋一杯水不能救車薪之火也】誠宗廟至計竇氏之福時濟南王康尊貴驕甚【康光武少子】憲乃白出敞為濟南太傅康有違失敞輒諫争【争則迸翻】康雖不能從然素敬重敞無所嫌牾焉【牾五故翻逆也】 冬十月庚子阜陵質王延薨【諡法名實不爽曰質】 是歲郡國九大水<br />
<br />
  二年春正月丁丑赦天下 二月壬午日有食之 夏五月丙辰封皇弟壽為濟北王開為河間王淑為城陽王【濟北河間城陽皆漢舊國也光武省濟北并泰山省河間并信都省城陽并琅邪今復泰山為濟北國在雒陽東千一百五十里分樂成涿郡勃海為河間國在雒陽北二千五百里分琅邪為城陽國濟子禮翻】紹封故淮陽頃王子側為常山王【章和元年淮陽頃王昞薨未及立嗣而國有大喪今乃紹封】 竇憲遣副校尉閻礱將二千餘騎掩擊北匈奴之守伊吾者復取其地【礱盧紅翻復扶又翻 考異曰西域傳作閻槃今從帝紀余謂副校尉閻槃即前戰于稽落山恐當作盤西域傳章帝建初元年罷伊吾屯田北匈奴遣兵守其地今】<br />
<br />
  【復擊取之】車師震慴前後王各遣子入侍【慴之涉翻】 月氏求尚公主班超拒還其使【氏音支使疏吏翻】由是怨恨遣其副王謝將兵七萬攻超超衆少皆大恐超譬軍士曰【譬喻也少詩沼翻】月氏兵雖多然數千里踰蔥嶺來非有運輸何足憂邪【言糧盡自當降也】但當收穀堅守彼飢窮自降不過數十日決矣【謂勝負決也降戶江翻】謝遂前攻超不下又鈔掠無所得超度其糧將盡【鈔楚交翻度大各翻】必從龜兹求食乃遣兵數百於東界要之【要一遥翻】謝果遣騎齎金銀珠玉以賂龜兹超伏兵遮擊盡殺之持其使首以示謝謝大驚即遣使請罪願得生歸超縱遣之月氏由是大震歲奉貢獻 初北海哀王無後【章帝元和三年北海哀王基薨無後】肅宗以齊武王首創大業而後嗣廢絶心常愍之遺詔令復齊北海二國丁卯封蕪湖侯無忌為齊王【無忌齊王晃子章和元年晃貶】北海敬王庶子威為北海王【北海敬王睦也】 六月辛卯中山簡王焉薨【諡法一德不懈曰簡】焉東海恭王之母弟而竇太后恭王之甥也【竇太后母沘陽公主東海恭王彊女也】故加賻錢一億【賻音附】大為修冢塋【為于偽翻】平夷吏民冢墓以千數作者萬餘人凡徵發揺動六州十八郡 詔封竇憲為冠軍侯篤為郾侯瓌為夏陽侯【冠軍縣屬南陽郡郾縣屬潁川郡夏陽縣屬馮翊郡冠古玩翻夏戶雅翻】憲獨不受封 秋七月乙卯竇憲出屯凉州【凉州部隴西漢陽武都金城安定北地武威張掖燉煌酒泉等郡】以侍中鄧疊行征西將軍事為副 北單于以漢還其侍弟九月復遣使欵塞稱臣欲入朝見【復扶又翻下同朝直遥翻見賢遍翻】冬十月竇憲遣班固梁諷迎之會南單于復上書求滅北庭於是遣左谷蠡王師子等將左右部八千騎出雞鹿塞【谷音鹿蠡盧奚翻】中郎將耿譚遣從事將護之【耿譚為使匈奴中郎將將領也護監也】襲擊北單于夜至圍之北單于被創【被皮義翻創初良翻】僅而得免獲閼氏及男女五人【氏音支下同】斬首八千級生虜數千口班固至私渠海而還是時南部黨衆益盛領戶三萬四千勝兵五萬【勝音升】<br />
<br />
  三年春正月甲子帝用曹褒新禮加元服【禮儀志正月甲子若丙子為吉日可加元服儀從冠禮乘輿初緇布進賢次爵弁次武弁次通天以據皆於高祖廟如禮謁賢曰元首也謂加冠於首】擢褒監羽林左騎【百官志羽林左監秩六百石主羽林左騎屬光祿勲褒監古銜翻】 竇憲以北匈奴微弱欲遂滅之二月遣左校尉耿夔司馬任尚出居延塞圍北單于於金微山【賢曰居延縣屬張掖郡居延澤在東北武帝使路博德築遮虜障於居延北余按唐太宗以僕固部置金微都督府】大破之獲其母閼氏名王已下五千餘級北單于逃走不知所在出塞五千餘里而還自漢出師所未嘗至也封夔為粟邑侯【賢曰粟邑縣名屬左馮翊故城在今同州白水縣西北】 竇憲既立大功威名益盛以耿夔任尚等為爪牙鄧疊郭璜為心腹班固傅毅之徒典文章刺史守令多出其門賦斂吏民共為賂遺【斂力贍翻遺于季翻】司徒袁安司空任隗舉奏諸二千石并所連及貶秩免官四十餘人竇氏大恨但安隗素行高亦未有以害之【行下孟翻】尚書僕射樂恢刺舉無所回避憲等疾之恢上書曰陛下富於春秋【賢曰春秋謂年也言年少春秋尚多故稱富】纂承大業諸舅不宜幹正王室以示天下之私方今之宜上以義自割下以謙自引四舅可長保爵土之榮【四舅謂憲篤景】皇太后永無慙負宗廟之憂誠策之上者也書奏不省【省悉景翻】恢稱疾乞骸骨歸長陵【恢京兆長陵人】憲風厲州郡廹脅恢飲藥死於是朝臣震慴【慴之涉翻】望風承旨無敢違者袁安以天子幼弱外戚擅權每朝會進見【見賢遍翻】及與公卿言國家事未嘗不喑嗚流涕【范書作噫嗚賢曰噫音醫又一戒翻嗚一故翻歎傷之貌也】自天子及大臣皆恃賴之 冬十月癸未上行幸長安詔求蕭曹近親宜為嗣者紹其封邑 詔竇憲與車駕會長安憲至尚書以下議欲拜之伏稱萬歲尚書韓稜正色曰夫上交不諂下交不凟【易下繫之辭】禮無人臣稱萬歲之制議者皆慙而止尚書左丞王龍私奏記上牛酒於憲【百官志尚書左丞右丞各一人掌録文書期會左丞主吏民章報及騶伯史右丞假署印綬及紙筆墨諸財用庫藏秩皆四百石蔡質漢儀曰總典臺中綱紀無所不統上時掌翻】稜舉奏龍論為城旦 龜兹姑墨温宿諸國皆降【降戶江翻】十二月復置西域都護騎都尉戊巳校尉官【章帝建初元年罷西域都護及戊巳校尉官今復置復扶又翻】以班超為都護徐幹為長史拜龜兹侍子白霸為龜兹王遣司馬姚光送之超與光共脅龜兹廢其王尤利多而立白霸使光將尤利多還詣京師【將如字】超居龜兹它乾城徐幹屯疏勒惟焉耆危須尉犁以前没都護【事見四十五卷明帝永平十八年】猶懷二心【為班超誅焉耆尉犂王張本】其餘悉定【言其餘諸國皆臣服於漢也】 初北單于既亡其弟右谷蠡王於除鞬自立為單于【鞬九言翻】將衆數千人止蒲類海遣使欵塞竇憲請遣使立於除鞬為單于置中郎將領護如南單于故事事下公卿議【下遐稼翻下同】宋由等以為可許袁安任隗奏以為光武招懷南虜非謂可永安内地正以權時之算可得扞禦北狄故也今朔漠已定宜令南單于反其北庭并領降衆【降戶江翻下同】無緣更立於除鞬以增國費事奏未以時定【言其議雖已奏上而上意從否未定也】安懼憲計遂行乃獨上封事曰南單于屯先父舉衆歸德【屯即屯屠何】自蒙恩以來四十餘年三帝積累以遺陛下【遺于季翻】陛下深宜追述先志成就其業况屯首創大謀空盡北虜輟而弗圖更立新降以一朝之計違三世之規失信於所養建立於無功【所養謂南單于無功謂於除鞬】論語曰言忠信行篤敬雖蠻貊行焉【孔子答子張之言行下孟翻貊莫北翻】今若失信於一屯則百蠻不敢復保誓矣【誓謂漢與夷人信誓之言復扶又翻】又烏桓鮮卑新殺北單于【謂章和元年斬優留單于】凡人之情咸畏仇讐今立其弟則二虜懷怨且漢故事供給南單于費直歲一億九十餘萬西域歲七千四百八十萬今北庭彌遠其費過倍是乃空盡天下而非建策之要也詔下其議安又與憲更相難折【難乃旦翻折之舌翻】憲險急負埶言辭驕訐【賢曰訐謂發人之惡音居謁翻】至詆毁安稱光武誅韓歆戴涉故事【韓歆死見四十三卷建武十五年戴涉死見同卷二十年】安終不移然上竟從憲策【考異曰袁安傳云憲請立左鹿蠡王阿佟為北單于安以為不可憲竟立右鹿蠡王於除鞬據此則阿佟與】<br />
<br />
  【於除鞬是二人袁紀作阿脩南匈奴傳止有右谷蠡王於除鞬無阿佟名今從之袁紀又云宋由丁鴻尹睦以為阿脩誅君之子又與烏丸鮮卑為父兄之讎不可立南單于先帝所置今首破北虜新建大功宜令并領降衆與范書不同又云卒從安議蓋誤今從袁安傳】<br />
<br />
  資治通鑑卷四十七<br />
<br />
<史部,編年類,資治通鑑>  <br>
   </div> 

<script src="/search/ajaxskft.js"> </script>
 <div class="clear"></div>
<br>
<br>
 <!-- a.d-->

 <!--
<div class="info_share">
</div> 
-->
 <!--info_share--></div>   <!-- end info_content-->
  </div> <!-- end l-->

<div class="r">   <!--r-->



<div class="sidebar"  style="margin-bottom:2px;">

 
<div class="sidebar_title">工具类大全</div>
<div class="sidebar_info">
<strong><a href="http://www.guoxuedashi.com/lsditu/" target="_blank">历史地图</a></strong>  
<a href="http://www.880114.com/" target="_blank">英语宝典</a>  
<a href="http://www.guoxuedashi.com/13jing/" target="_blank">十三经检索</a> 
<br><strong><a href="http://www.guoxuedashi.com/gjtsjc/" target="_blank">古今图书集成</a></strong> 
<a href="http://www.guoxuedashi.com/duilian/" target="_blank">对联大全</a> <strong><a href="http://www.guoxuedashi.com/xiangxingzi/" target="_blank">象形文字典</a></strong> 

<br><a href="http://www.guoxuedashi.com/zixing/yanbian/">字形演变</a>  <strong><a href="http://www.guoxuemi.com/hafo/" target="_blank">哈佛燕京中文善本特藏</a></strong>
<br><strong><a href="http://www.guoxuedashi.com/csfz/" target="_blank">丛书&方志检索器</a></strong> <a href="http://www.guoxuedashi.com/yqjyy/" target="_blank">一切经音义</a>  

<br><strong><a href="http://www.guoxuedashi.com/jiapu/" target="_blank">家谱族谱查询</a></strong>  <strong><a href="http://shufa.guoxuedashi.com/sfzitie/" target="_blank">书法字帖欣赏</a></strong> 
<br>

</div>
</div>


<div class="sidebar" style="margin-bottom:0px;">

<font style="font-size:22px;line-height:32px">QQ交流群9:489193090</font>


<div class="sidebar_title">手机APP 扫描或点击</div>
<div class="sidebar_info">
<table>
<tr>
	<td width=160><a href="http://m.guoxuedashi.com/app/" target="_blank"><img src="/img/gxds-sj.png" width="140"  border="0" alt="国学大师手机版"></a></td>
	<td>
<a href="http://www.guoxuedashi.com/download/" target="_blank">app软件下载专区</a><br>
<a href="http://www.guoxuedashi.com/download/gxds.php" target="_blank">《国学大师》下载</a><br>
<a href="http://www.guoxuedashi.com/download/kxzd.php" target="_blank">《汉字宝典》下载</a><br>
<a href="http://www.guoxuedashi.com/download/scqbd.php" target="_blank">《诗词曲宝典》下载</a><br>
<a href="http://www.guoxuedashi.com/SiKuQuanShu/skqs.php" target="_blank">《四库全书》下载</a><br>
</td>
</tr>
</table>

</div>
</div>


<div class="sidebar2">
<center>


</center>
</div>

<div class="sidebar"  style="margin-bottom:2px;">
<div class="sidebar_title">网站使用教程</div>
<div class="sidebar_info">
<a href="http://www.guoxuedashi.com/help/gjsearch.php" target="_blank">如何在国学大师网下载古籍?</a><br>
<a href="http://www.guoxuedashi.com/zidian/bujian/bjjc.php" target="_blank">如何使用部件查字法快速查字?</a><br>
<a href="http://www.guoxuedashi.com/search/sjc.php" target="_blank">如何在指定的书籍中全文检索?</a><br>
<a href="http://www.guoxuedashi.com/search/skjc.php" target="_blank">如何找到一句话在《四库全书》哪一页?</a><br>
</div>
</div>


<div class="sidebar">
<div class="sidebar_title">热门书籍</div>
<div class="sidebar_info">
<a href="/so.php?sokey=%E8%B5%84%E6%B2%BB%E9%80%9A%E9%89%B4&kt=1">资治通鉴</a> <a href="/24shi/"><strong>二十四史</strong></a>&nbsp; <a href="/a2694/">野史</a>&nbsp; <a href="/SiKuQuanShu/"><strong>四库全书</strong></a>&nbsp;<a href="http://www.guoxuedashi.com/SiKuQuanShu/fanti/">繁体</a>
<br><a href="/so.php?sokey=%E7%BA%A2%E6%A5%BC%E6%A2%A6&kt=1">红楼梦</a> <a href="/a/1858x/">三国演义</a> <a href="/a/1038k/">水浒传</a> <a href="/a/1046t/">西游记</a> <a href="/a/1914o/">封神演义</a>
<br>
<a href="http://www.guoxuedashi.com/so.php?sokeygx=%E4%B8%87%E6%9C%89%E6%96%87%E5%BA%93&submit=&kt=1">万有文库</a> <a href="/a/780t/">古文观止</a> <a href="/a/1024l/">文心雕龙</a> <a href="/a/1704n/">全唐诗</a> <a href="/a/1705h/">全宋词</a>
<br><a href="http://www.guoxuedashi.com/so.php?sokeygx=%E7%99%BE%E8%A1%B2%E6%9C%AC%E4%BA%8C%E5%8D%81%E5%9B%9B%E5%8F%B2&submit=&kt=1"><strong>百衲本二十四史</strong></a>  <a href="http://www.guoxuedashi.com/so.php?sokeygx=%E5%8F%A4%E4%BB%8A%E5%9B%BE%E4%B9%A6%E9%9B%86%E6%88%90&submit=&kt=1"><strong>古今图书集成</strong></a>
<br>

<a href="http://www.guoxuedashi.com/so.php?sokeygx=%E4%B8%9B%E4%B9%A6%E9%9B%86%E6%88%90&submit=&kt=1">丛书集成</a> 
<a href="http://www.guoxuedashi.com/so.php?sokeygx=%E5%9B%9B%E9%83%A8%E4%B8%9B%E5%88%8A&submit=&kt=1"><strong>四部丛刊</strong></a>  
<a href="http://www.guoxuedashi.com/so.php?sokeygx=%E8%AF%B4%E6%96%87%E8%A7%A3%E5%AD%97&submit=&kt=1">說文解字</a> <a href="http://www.guoxuedashi.com/so.php?sokeygx=%E5%85%A8%E4%B8%8A%E5%8F%A4&submit=&kt=1">三国六朝文</a>
<br><a href="http://www.guoxuedashi.com/so.php?sokeytm=%E6%97%A5%E6%9C%AC%E5%86%85%E9%98%81%E6%96%87%E5%BA%93&submit=&kt=1"><strong>日本内阁文库</strong></a> <a href="http://www.guoxuedashi.com/so.php?sokeytm=%E5%9B%BD%E5%9B%BE%E6%96%B9%E5%BF%97%E5%90%88%E9%9B%86&ka=100&submit=">国图方志合集</a> <a href="http://www.guoxuedashi.com/so.php?sokeytm=%E5%90%84%E5%9C%B0%E6%96%B9%E5%BF%97&submit=&kt=1"><strong>各地方志</strong></a>

</div>
</div>


<div class="sidebar2">
<center>

</center>
</div>
<div class="sidebar greenbar">
<div class="sidebar_title green">四库全书</div>
<div class="sidebar_info">

《四库全书》是中国古代最大的丛书,编撰于乾隆年间,由纪昀等360多位高官、学者编撰,3800多人抄写,费时十三年编成。丛书分经、史、子、集四部,故名四库。共有3500多种书,7.9万卷,3.6万册,约8亿字,基本上囊括了古代所有图书,故称“全书”。<a href="http://www.guoxuedashi.com/SiKuQuanShu/">详细>>
</a>

</div> 
</div>

</div>  <!--end r-->

</div>
<!-- 内容区END --> 

<!-- 页脚开始 -->
<div class="shh">

</div>

<div class="w1180" style="margin-top:8px;">
<center><script src="http://www.guoxuedashi.com/img/plus.php?id=3"></script></center>
</div>
<div class="w1180 foot">
<a href="/b/thanks.php">特别致谢</a> | <a href="javascript:window.external.AddFavorite(document.location.href,document.title);">收藏本站</a> | <a href="#">欢迎投稿</a> | <a href="http://www.guoxuedashi.com/forum/">意见建议</a> | <a href="http://www.guoxuemi.com/">国学迷</a> | <a href="http://www.shuowen.net/">说文网</a><script language="javascript" type="text/javascript" src="https://js.users.51.la/17753172.js"></script><br />
  Copyright &copy; 国学大师 古典图书集成 All Rights Reserved.<br>
  
  <span style="font-size:14px">免责声明:本站非营利性站点,以方便网友为主,仅供学习研究。<br>内容由热心网友提供和网上收集,不保留版权。若侵犯了您的权益,来信即刪。scp168@qq.com</span>
  <br />
ICP证:<a href="http://www.beian.miit.gov.cn/" target="_blank">鲁ICP备19060063号</a></div>
<!-- 页脚END --> 
<script src="http://www.guoxuedashi.com/img/plus.php?id=22"></script>
<script src="http://www.guoxuedashi.com/img/tongji.js"></script>

</body>
</html>
