<!DOCTYPE html PUBLIC "-//W3C//DTD XHTML 1.0 Transitional//EN" "http://www.w3.org/TR/xhtml1/DTD/xhtml1-transitional.dtd">
<html xmlns="http://www.w3.org/1999/xhtml">
<head>
<meta http-equiv="Content-Type" content="text/html; charset=utf-8" />
<meta http-equiv="X-UA-Compatible" content="IE=Edge,chrome=1">
<title>資治通鑒_292-資治通鑑卷二百九十一_292-資治通鑑卷二百九十一</title>
<meta name="Keywords" content="資治通鑒_292-資治通鑑卷二百九十一_292-資治通鑑卷二百九十一">
<meta name="Description" content="資治通鑒_292-資治通鑑卷二百九十一_292-資治通鑑卷二百九十一">
<meta http-equiv="Cache-Control" content="no-transform" />
<meta http-equiv="Cache-Control" content="no-siteapp" />
<link href="/img/style.css" rel="stylesheet" type="text/css" />
<script src="/img/m.js?2020"></script> 
</head>
<body>
 <div class="ClassNavi">
<a  href="/24shi/">二十四史</a> | <a href="/SiKuQuanShu/">四库全书</a> | <a href="http://www.guoxuedashi.com/gjtsjc/"><font  color="#FF0000">古今图书集成</font></a> | <a href="/renwu/">历史人物</a> | <a href="/ShuoWenJieZi/"><font  color="#FF0000">说文解字</a></font> | <a href="/chengyu/">成语词典</a> | <a  target="_blank"  href="http://www.guoxuedashi.com/jgwhj/"><font  color="#FF0000">甲骨文合集</font></a> | <a href="/yzjwjc/"><font  color="#FF0000">殷周金文集成</font></a> | <a href="/xiangxingzi/"><font color="#0000FF">象形字典</font></a> | <a href="/13jing/"><font  color="#FF0000">十三经索引</font></a> | <a href="/zixing/"><font  color="#FF0000">字体转换器</font></a> | <a href="/zidian/xz/"><font color="#0000FF">篆书识别</font></a> | <a href="/jinfanyi/">近义反义词</a> | <a href="/duilian/">对联大全</a> | <a href="/jiapu/"><font  color="#0000FF">家谱族谱查询</font></a> | <a href="http://www.guoxuemi.com/hafo/" target="_blank" ><font color="#FF0000">哈佛古籍</font></a> 
</div>

 <!-- 头部导航开始 -->
<div class="w1180 head clearfix">
  <div class="head_logo l"><a title="国学大师官网" href="http://www.guoxuedashi.com" target="_blank"></a></div>
  <div class="head_sr l">
  <div id="head1">
  
  <a href="http://www.guoxuedashi.com/zidian/bujian/" target="_blank" ><img src="http://www.guoxuedashi.com/img/top1.gif" width="88" height="60" border="0" title="部件查字,支持20万汉字"></a>


<a href="http://www.guoxuedashi.com/help/yingpan.php" target="_blank"><img src="http://www.guoxuedashi.com/img/top230.gif" width="600" height="62" border="0" ></a>


  </div>
  <div id="head3"><a href="javascript:" onClick="javascript:window.external.AddFavorite(window.location.href,document.title);">添加收藏</a>
  <br><a href="/help/setie.php">搜索引擎</a>
  <br><a href="/help/zanzhu.php">赞助本站</a></div>
  <div id="head2">
 <a href="http://www.guoxuemi.com/" target="_blank"><img src="http://www.guoxuedashi.com/img/guoxuemi.gif" width="95" height="62" border="0" style="margin-left:2px;" title="国学迷"></a>
  

  </div>
</div>
  <div class="clear"></div>
  <div class="head_nav">
  <p><a href="/">首页</a> | <a href="/ShuKu/">国学书库</a> | <a href="/guji/">影印古籍</a> | <a href="/shici/">诗词宝典</a> | <a   href="/SiKuQuanShu/gxjx.php">精选</a> <b>|</b> <a href="/zidian/">汉语字典</a> | <a href="/hydcd/">汉语词典</a> | <a href="http://www.guoxuedashi.com/zidian/bujian/"><font  color="#CC0066">部件查字</font></a> | <a href="http://www.sfds.cn/"><font  color="#CC0066">书法大师</font></a> | <a href="/jgwhj/">甲骨文</a> <b>|</b> <a href="/b/4/"><font  color="#CC0066">解密</font></a> | <a href="/renwu/">历史人物</a> | <a href="/diangu/">历史典故</a> | <a href="/xingshi/">姓氏</a> | <a href="/minzu/">民族</a> <b>|</b> <a href="/mz/"><font  color="#CC0066">世界名著</font></a> | <a href="/download/">软件下载</a>
</p>
<p><a href="/b/"><font  color="#CC0066">历史</font></a> | <a href="http://skqs.guoxuedashi.com/" target="_blank">四库全书</a> |  <a href="http://www.guoxuedashi.com/search/" target="_blank"><font  color="#CC0066">全文检索</font></a> | <a href="http://www.guoxuedashi.com/shumu/">古籍书目</a> | <a   href="/24shi/">正史</a> <b>|</b> <a href="/chengyu/">成语词典</a> | <a href="/kangxi/" title="康熙字典">康熙字典</a> | <a href="/ShuoWenJieZi/">说文解字</a> | <a href="/zixing/yanbian/">字形演变</a> | <a href="/yzjwjc/">金 文</a> <b>|</b>  <a href="/shijian/nian-hao/">年号</a> | <a href="/diming/">历史地名</a> | <a href="/shijian/">历史事件</a> | <a href="/guanzhi/">官职</a> | <a href="/lishi/">知识</a> <b>|</b> <a href="/zhongyi/">中医中药</a> | <a href="http://www.guoxuedashi.com/forum/">留言反馈</a>
</p>
  </div>
</div>
<!-- 头部导航END --> 
<!-- 内容区开始 --> 
<div class="w1180 clearfix">
  <div class="info l">
   
<div class="clearfix" style="background:#f5faff;">
<script src='http://www.guoxuedashi.com/img/headersou.js'></script>

</div>
  <div class="info_tree"><a href="http://www.guoxuedashi.com">首页</a> > <a href="/SiKuQuanShu/fanti/">四库全书</a>
 > <h1>资治通鉴</h1> <!--         下载:【右键另存为】即可 --></div>
  <div class="info_content zj clearfix">
  
<div class="info_txt clearfix" id="show">
<center style="font-size:24px;">292-資治通鑑卷二百九十一</center>
    資治通鑑卷二百九十一  宋 司馬光 撰<br />
<br />
  胡三省 音註<br />
<br />
  後周紀二【起玄黓困敦九月盡閼逢攝提格四月凡一年有奇】<br />
<br />
  太祖聖神恭肅文武孝皇帝中<br />
<br />
  廣順二年九月甲寅朔吳越丞相裴堅卒以台州刺史吳延福同參相府事 庚午敇北邊吏民毋得入契丹境俘掠 契丹將高謨翰以葦栰度胡盧河入寇【胡盧河在深冀之間横亘數百里丁度曰胡盧河即衡漳之别名】至冀州成德節度使何福進遣龍捷都指揮使劉誠誨等屯貝州以拒之【九域志貝州北至冀州一百二十里】契丹聞之遽引兵北度所掠冀州丁壯數百人望見官軍爭鼓譟欲攻契丹官軍不敢應契丹盡殺之 蜀山南西道節度使李廷珪奏周人聚兵關中請益兵為備蜀主遣奉鑾肅衛都虞候趙進將兵趣利州【趣七喻翻】既而聞周人聚兵以備壮漢乃引還【還從宣翻又如字】唐武安節度使邊鎬昏懦無斷【斷丁亂翻】在湖南政出多門不合衆心吉水人歐陽廣上書【吉水古吉陽縣地久廢唐置吉水縣屬吉州九域志在州東壮四十里宋白曰隋開皇十年廢吉陽縣入廬陵縣大業分廬陵縣水東十一鄉為吉水縣】言鎬非將帥才必喪湖南【將即亮翻帥所類翻喪息浪翻】宜别擇良帥益兵以救其敗不報唐主使鎬經略朗州有自朗州來者多言劉言忠順鎬由是不為備唐主召劉言入朝【朝直遥翻】言不行謂王逵曰唐必伐我奈何逵曰武陵負江湖之險【朗州武陵郡】帶甲數萬安能拱手受制於人邊鎬撫御無方士民不附可一戰擒也言猶豫未决周行逢曰機事貴速緩則彼為之備不可圖也言乃以逵行逢及牙將何敬真張倣蒲公益朱全琇【琇音秀】宇文瓊彭萬和潘叔嗣張文表十人皆為指揮使部分發兵【分扶問翻】叔嗣文表皆朗州人也行逢能謀文表善戰叔嗣果敢三人多相須成功情欵甚昵【昵尼質翻】諸將欲召溆州酋長苻彦通為援【溆音叙苻讀曰蒲酋慈由翻長知兩翻苻彦通自謂苻秦苗裔】行逢曰蠻貪而無義前年從馬希萼入潭州焚掠無遺【見二百八十九卷漢隱帝乾祐二年】吾兵以義舉往無不克烏用此物使暴殄百姓哉乃止然亦畏彦通為後患以蠻酋土團都指揮使劉瑫為羣蠻所憚【瑫它牢翻】補西境鎮遏使以備之冬十月逵等將兵分道趣長沙【趣七喻翻】以孫朗曹進為先鋒使【孫朗曹進奔朗州見上卷是年正月】邊鎬遣指揮使郭再誠等將兵屯益陽以拒之戊子逵等克沅江【沅音元沅江漢益陽縣地隋改為安樂又改為沅江乾寧中改為橋江楚復為沅江屬朗州九域志在岳州西南一百二十六里】執都監劉承遇禆將李師德帥衆五百降之【帥讀曰率降戶江翻】壬辰逵等命軍士舉小舟自蔽直造益陽【造七到翻】四面斧寨而入遂克之殺戍兵二千人邊鎬告急於唐甲午逵等克橋口及湘隂【九域志潭州長沙縣有橋口鎮】乙未至潭州邊鎬嬰城自守救兵未至城中兵少丙申夜鎬棄城走吏民俱潰醴陵門橋折【醴陵門潭州城東門折而設翻】死者萬餘人道州刺史廖偃為亂兵所殺丁酉旦王逵入城自稱武平節度副使權知軍府事【武平當作武安軍府謂潭州軍府也】以何敬真為行軍司馬遣敬真等追鎬不及斬首五百級蒲公益攻岳州【風俗通漢有詹事蒲昌又晉書載紀氐酋蒲洪之先其家池中蒲生長五丈如竹形時咸謂之蒲家因以為氏其後改姓苻則蒲之所自出有二焉】唐岳州刺史宋德權走劉言以公益權知岳州唐將守湖南諸州者聞長沙陷相繼遁去劉言盡復馬氏嶺北故地惟郴連入于南漢【郴尹林翻】 契丹瀛莫幽州大水流民入塞散居河北者數十萬口契丹州縣亦不之禁詔所在賑給存處之【賑津忍翻處昌呂翻】中國民先為所掠得歸者什五六 丁未李穀以病臂久未愈【李穀病臂始上卷是年六月穀上須有李字文乃明】三表辭位帝遣中使諭指曰卿所掌至重【謂李穀掌三司金穀也】朕難其人苟事功克集何必朝禮【朝直遥翻】朕今於便殿待卿可暫入相見穀入見于金祥殿【見賢遍翻】面陳悃欵【悃苦本翻誠也】帝不許穀不得已復視事【復扶又翻】穀未能執筆詔以三司務繁令刻名印用之 辛亥敇民有訴訟必先歷縣州及觀察使處决不直【處昌呂翻】乃聽訟於臺省或自不能書牒倩人書者必書所倩姓名居處若無可倩聽執素紙所訴必須已事毋得挾私客訴【倩七政翻假倩也事不干己妄興詞訟謂之客訴】慶州刺史郭彦欽性貪野雞族多羊馬【五代會要黨項野雞族居慶州北】彦欽故擾之以求賂野雞族遂反剽掠綱商【剽匹妙翻綱商往沿邊販易者薛史慶州北十五里寡婦山有蕃部曰野雞族刺史郭彦欽擅加糴鹽錢民夷流怨蕃族獷悍好為不法彦欽乃奏野雞族掠奪綱商】帝命寧環二州合兵討之【唐於古鳴沙之地置威州周改曰環州九 志寧州北至慶州一百二十里環州南至 州一百八十里】 劉言遣使來告稱湖南世事朝廷不幸為鄰寇所陷【鄰寇謂唐也】臣雖不奉詔輒糾合義兵削平舊國【言削平湖南舊楚之地】唐主削邊鎬官爵流饒州初鎬以都虞候從查文徽克建州【事見二百八十五卷晉齊王開運二年唐主之保大三年也】凡所俘獲皆全之建人謂之邊佛子及克潭州市不易肆【事見上卷元年唐之保大九年也】潭人謂之邊菩薩【菩薄乎翻薩桑葛翻釋典菩普也薩濟也言能普濟衆生也】既而為節度使政無綱紀惟日設齋供【供居用翻】盛修佛事潭人失望謂之邊和尚矣左僕射同平章馮延己右僕射同平章事孫晟上表請罪皆釋之晟陳請不已乃與延己皆罷守本官唐主以比年出師無功【比毗至翻】乃議休兵息民或曰願陛下數十年不用兵可小康矣唐主曰將終身不用何數十年之有【人非金石唐主自謂真能享無疆之夀乎然欲終身不用兵而周兵已至淮上矣】唐主思歐陽廣之言拜本縣令【以歐陽廣言邊鎬必敗其言驗也】十一月辛未徙保義節度使折從阮為靜難節度使<br />
<br />
  【折從阮自陜州徙邠州難乃旦翻下同】討野雞族癸酉敇約每歲民間所輸牛皮三分減二計田十頃税取一皮餘聽民自用及賣買惟禁賣于敵國先是兵興以來【先悉薦翻】禁民私賣買牛皮悉令輸官受直唐明宗之世有司止償以鹽晉天福中并鹽不給漢法犯私牛皮一寸抵死然民間日用實不可無帝素知其弊至是李穀建議均於田畝公私便之 十二月丙戌河決鄭滑遣使行視修塞【行下孟翻塞悉則翻】 甲午前靜難節度使侯章獻買宴絹千匹銀五百兩帝不受曰諸侯入覲天子宜有宴犒豈待買邪【五代之時不特方鎮入朝買宴唐明宗天成二年三月幸會節園羣臣買宴則在朝之臣亦買宴矣犒苦到翻】自今如此比者皆不受 王逵將兵及洞蠻五萬攻郴州【郴丑林翻】南漢將潘崇徹救之遇于蠔石【蠔石在郴州義章縣蠔音豪】崇徹登高望湖南兵曰疲而不整可破也縱擊大破之伏尸八十里 翰林學士徐台符請誅誣告李崧者葛延遇及李澄【誣李崧事見二百八十八卷漢乾祐元年徐台符素與李崧善故為請誅誣告者】馮道以為屢更赦不許【更工衡翻】王峻嘉台符之義白於帝癸卯收延遇澄誅之 劉言表稱潭州殘破乞移使府治朗州【使疏吏翻】且請貢獻賣茶悉如馬氏故事許之 唐江西觀察使楚王馬希萼入朝唐主留之後數年卒於金陵諡曰恭孝 初麟州土豪楊信自為刺史受命于周信卒子重訓嗣 【考異曰崇訓或作崇勲世宗實録作崇訓後蓋避梁王崇訓改名也按考異則重訓當作崇訓】以州降北漢至是為羣羌所圍復歸欵【復扶又翻】求救於夏府二州【夏州李殷府州折德扆九域志麟州西北至夏州一百二十里東北至府州一百二十里】<br />
<br />
  三年春正月丙辰以武平留後劉言為武平節度使制置武安靜江等軍事同平章事以王逵為武安節度使何敬真為靜江節度使周行逢為武安行軍司馬【王逵既得潭州則殺何敬真既殺何敬真則攻劉言而併朗州】 詔折從阮野雞族能改過者拜官賜金帛不則進兵討之壬戌從阮奏酋長李萬全等受詔立誓外【酋慈秋翻長知兩翻】自餘猶不服方討之 前世屯田皆在邊地使戍兵佃之【佃亭年翻】唐末中原宿兵所在皆置營田以耕曠土其後又募高貲戶使輸課佃之【輸舂遇翻下歲輸同】戶部别置官司總領不隸州縣或丁多無役或容庇奸盜州縣不能詰【詰去吉翻】梁太祖擊淮南掠得牛以千萬計【朱全忠大掠淮南見二百六十五卷唐昭宗天祐元年】給東南諸州農民使歲輸租自是歷數十年牛死而租不除民甚苦之帝素知其弊會閤門使知青州張凝上便宜請罷營田務李穀亦以為言乙丑敇悉罷戶部營田務以其民隸州縣其田廬牛農器並賜見佃者為永業【見賢遍翻】悉除租牛課是歲戶部增三萬餘戶民既得為永業始敢葺屋植木獲地利數倍或言營田有肥饒者不若鬻之可得錢數十萬緡以資國帝曰利在於民猶在國也朕用此錢何為 萊州刺史葉仁魯帝之故吏也【按葉仁魯漢高祖之親將也天福十二年嘗破契丹于承天軍今曰帝之故吏必嘗事帝於樞密院或討河中鎮鄴都時也】坐贓絹萬五千匹錢千緡庚午賜死帝遣中使賜以酒食曰汝自抵國法吾無如之何當存恤汝母仁魯感泣 帝以河決為憂王峻自請往行視許之【行下孟翻】鎮寧節度使榮屢求入朝峻忌其英烈每沮止之【沮在呂翻】閏月榮復求入朝【復扶又翻】會峻在河上帝乃許之 契丹寇定州圍義豐軍【時置義豐軍於定州義豐縣】定和都指揮使楊弘裕夜擊其營大獲契丹遁去又寇鎮州本道兵擊走之 丙申鎮寧節度使榮入朝故李守貞騎士馬全乂從榮入朝帝召見補殿前指揮使謂左右曰全乂忠於所事昔在河中屢挫吾軍【謂漢乾祐間帝討李茂貞時也】汝輩宜効之王峻聞榮入朝遽自河上歸戊戌至大梁 彰武節度使高允權卒其子牙内指揮使紹基謀襲父位詐稱允權疾病表已知軍府事觀察判官李彬切諫紹基怒斬之辛巳以彬謀反聞 王峻固求領藩鎮帝不得已以峻兼平盧節度使 高紹基屢奏雜虜犯邊冀得承襲帝遣六宅使張仁謙詣延州巡檢【職官分紀曰唐置十宅六宅使以諸王所屬為名或總云十六宅後止曰六宅】紹基不能匿始發父喪 戊申折從阮奏降野鷄二十一族 唐草澤邵棠上言【布衣未有朝命者謂之草澤上時掌翻】近游淮上聞周主恭儉增修德政吾兵新破於潭朗【謂邊鎬潭州之敗也】恐其有南征之志宜為之備【智識之士何國無之顧用與不用耳】 初王逵既得潭州【事見上卷十月】以指揮使何敬真為靜江節度副使朱全琇為武安節度副使張文表為武平節度副使周行逢為武安行軍司馬敬真全琇各置牙兵與逵分廳視事吏民莫知所從每宴集諸將使酒紛挐如市無復上下之分【挐奴加翻分扶問翻】唯行逢文表事逵盡禮逵親愛之敬真與逵不協辭歸朗州又不能事劉言與全琇謀作亂言素忌逵之彊疑逵使敬真伺已將討之逵聞之甚懼【伺相吏翻】行逢曰劉言素不與吾輩同心何敬真朱全琇恥在公下公宜早圖之逵喜曰與公共除凶黨同治潭朗【周行逢之據有潭朗自此造端矣治直之翻】夫復何憂【夫音扶復扶又翻】會南漢寇全道永州行逢請身至朗州說言【說式芮翻】遣敬真全琇南討【南討者拒南漢之兵】俟至長沙以計取之如掌中物耳逵從之行逢至朗州言以敬真為南面行營招討使全琇為先鋒使將牙兵百餘人會潭州兵以禦南漢二人至長沙逵出郊迎相見甚歡宴飲連日多以美妓餌之【妓渠綺翻】敬真因淹留不進朗州指揮使李仲遷部兵三千人久戍潭州敬真使之先發趣嶺壮【全道永三州皆在大庾嶺之北趣七喻翻】都頭符會等因士卒思歸劫仲遷擅還朗州逵乘敬真醉使人詐為言使者責敬真以南寇深侵不亟捍禦而專務荒宴太師命械公歸西府【太師謂劉言朗府在潭州之西故謂之西府】因收繫獄全琇逃去遣兵追捕之二月辛亥朔斬敬真以狥未幾【幾居豈翻】獲全琇及其黨十餘人皆斬之 癸丑鎮寜節度使榮歸澶州【澶時連翻】初契丹主德光北還【見二百八十六卷天福十二年】以晉傳國寶自隨至是更以玉作二寶【傳國寶及受命寶也五代會要曰時製寶兩座用白玉方六寸螭虎紐馮道書寶文其一以皇帝承天受命之寶為文其一以皇帝神寶為文宋白曰時内司製二寶詔太常具制度以閒有司言唐六典符寶郎掌天子八璽其一曰神寶二曰受命寶其神寶方六寸高四寸六分厚一寸七分蟠龍紐文與傳國璽同傳國璽秦皇以藍田玉刻之李斯篆方四寸面文曰受命于天既夀永昌紐盤五龍二寶歷代相傳以為神器别有六寶一曰皇帝行璽二曰皇帝之璽三曰皇帝信璽四曰天子行璽五曰天子之璽六曰天子信璽此六璽因文為名並白玉螭虎紐歷代傳受或亡失則補之北朝鑄之以金貞觀十六年别製玄璽一座文曰皇天景命有德者昌白玉螭虎紐同光中製寶一座文曰皇帝受命之寶天福四年製寶一座文曰皇帝神寶其同光天福二寶内司製造不見紐篆分寸制度勑今製國寶兩座其一宜以皇帝承天受命之寶為文其一以皇帝神寶為文命中書令馮道書寶議者曰國以玉璽為傳授神器邃古無聞運計樞曰舜禹天子黄龍負璽世本曰魯昭公始作璽秦兼六國稱皇帝禮取藍田之玉玉工孫夀刻之方四寸李斯為大篆書之形制如龍魚鳳鳥之狀希世之至寶也秦亡子嬰以璽降漢漢世世傳寶之王莽之簒求璽於元后后投之於堦一角微缺莽誅歸之更始更始敗歸之盆子及熊耳之敗盆子以璽降光武漢末黄巾亂投璽於井孫堅入洛見井有五色氣取得之以歸袁術術敗荆州刺史徐璆得之詣許以進獻帝魏受漢得之以傳于晉洛陽之陷劉聰得之劉曜為石勒所禽璽歸于鄴石氏之亂冉閔得之閔敗晉將戴施入鄴得之送江東傳之宋齊梁臺城之破侯景得之景敗其將侯子鑒以璽走為追兵所迫投於栖霞寺井中僧永杼得而匿之陳永定二年永杼弟子普智以璽上陳文帝隋平陳始得秦真傳國璽煬帝江都之禍宇文化及得之化及敗璽歸竇建德建德敗其妻曹氏以璽獻于唐唐禪楊涉送寶于大梁莊宗滅梁得之同光末内難作寶為火灼文字訛缺明宗得之清泰敗以寶隨身自焚而死寶遂亡失其神寶者方六寸厚一寸七分高四寸六分蟠龍隱起文與秦璽同但玉色不及形制高大耳不知何代製造東晉孝武十九年雍州刺史郗恢得之慕容永送於金陵傳之宋齊梁臺城之破侯景得之景敗侍中趙思齊攜走江北獻之齊文宣帝宇文滅齊得之宇文亡入隋隋文帝改號傳國璽又改為受命璽及平陳始得秦真傳國璽仍以秦璽後出得於亡陳以北朝所傳神璽為第一秦璽次之隋亡竇建德妻與神璽俱獻長安唐末不知所在其說頗有源委因載于此更工衡翻】 王逵遣使以斬何敬真告劉言言不得己庚申斬符會等數人【以符會等擅歸召變也】樞密使平盧節度使同平章事王峻晚節益狂躁奏請以端明殿學士顔衎【衎苦旱翻又苦旦翻】樞密直學士陳觀代范質李穀為相帝曰進退宰輔不可倉猝俟朕更思之峻力論列語浸不遜日向中帝尚未食峻爭之不已帝曰今方寒食俟假開如卿所奏峻乃退【舊制寒食節休假前後共五日假居訝翻】癸亥帝亟召宰相樞密使入幽峻於别所帝見馮道等泣曰王峻陵朕太甚欲盡逐大臣翦朕羽翼朕惟一子專務間阻暫令詣闕已懷怨望【間古莧翻令詣闕謂聽皇子榮自澶州入朝也】豈有身典樞機復兼宰相又求重鎮【復扶又翻峻求領藩鎮見上月】觀其志趣殊未盈厭【厭於艷翻又於鹽翻】無君如此誰則堪之甲子貶峻商州司馬制辭略曰肉視羣后孩撫朕躬【言視朝臣如几上肉撫天子如嬰孩】帝慮鄴都留守王殷不自安【王峻王殷佐命有功一體之人峻得罪故慮殷猜懼】命殷子尚食使承誨詣殷【尚食使唐尚食奉御之職】諭以峻得罪之狀峻至商州得腹疾帝猶愍之命其妻往視之未幾而卒【幾居豈翻】 帝命折從阮分兵屯延州【折從阮時為靜難帥帥兵討野鷄族而還師】高紹基始懼屢有貢獻又命供奉官張懷貞將禁兵兩指揮屯鄜延【鄜方無翻】紹基乃悉以軍府事授副使張匡圖甲戌以客省使向訓權知延州 三月甲申以鎮寧節度使榮為開封尹晉王【王峻既貶始召榮入】丙戌以樞密副使鄭仁誨為鎮寜節度使 初殺牛族與野鷄族有隙聞官軍討野鷄饋餉迎奉官軍利其財畜而掠之殺牛族反與野雞合敗寜州刺史張建武于包山【敗補邁翻】帝以郭彦欽擾羣胡致其作亂【事見上年十月】黜廢於家 初解州刺史浚儀郭元昭與榷鹽使李温玉有隙【漢隱帝分河中之解安邑聞喜為解州解戶買翻榷古岳翻】温玉壻魏仁浦為樞密主事【晉有尚書都令史八人秩二百石與左右丞總知都臺事梁五人謂之五都令史隋開皇初改都令史為都事置八人後魏於尚書諸司置主事令史隋於諸省又各置主事令史煬帝並去令史之名更曰主事初雜用士人至唐並用流外至五代樞密院亦置主事】元昭疑仁浦庇之會李守貞反温玉有子在河中元昭收繫温玉奏言其叛事連仁浦帝時為樞密使知其誣釋不問至是仁浦為樞密承旨元昭代歸甚懼過洛陽以告仁浦弟仁滌仁滌曰吾兄平生不與人為怨况肯以私害公乎既至丁亥仁浦白帝以元昭為慶州刺史 己丑以棣州團練使太原王仁鎬為宣徽北院使兼樞密副使 唐主復以左僕射馮延己同平章事【去年十月唐失潭州馮延己罷相】 周行逢惡武平節度副使張倣【惡烏路翻】言於王逵曰何敬真倣之親戚臨刑以後事屬倣公宜備之【屬之欲翻】夏四月庚申逵召倣飲醉而殺之 丙寅歸德節度使兼侍中常思入朝戊辰徙平盧節度使將行奏曰臣在宋州舉絲四萬餘兩在民間謹以上進請徵之【舉絲者以貨物貸與民至絲熟而徵其絲上時掌翻】帝頷之五月丁亥敇牓宋州凡常思所舉悉蠲之思亦無怍色【怍疾各翻】 自唐末以來所在學校廢絶蜀毋昭裔出私財百萬營學館【校戶教翻毋音無姓也齊宣王封母弟於毋鄉其後因以為氏】且請刻板印九經蜀主從之由是蜀中文學復盛【自漢司馬相如揚雄以來蜀中號為多士而斯文之盛衰則繫乎上之人】 六月壬子滄州奏契丹知盧臺軍事范陽張藏英來降 初唐明宗之世宰相馮道李愚請令判國子監田敏校正九經刻板印賣朝廷從之丁巳板成獻之【雕印九經始二百七十七卷唐明宗長興三年至是而成凡涉二十八年】由是雖亂世九經傳布甚廣【史言聖人之道所以不墜者以其有方策之傳也】 王逵以周行逢知潭州自將兵襲朗州克之殺指揮使鄭珓【珓古孝翻】執武安節度使同平章事劉言幽于别館【劉言為武平節度使鎮朗州非武安也安當作平言以元年七月得朗州至是而敗】 秋七月王殷三表請入朝帝疑其不誠遣使止之 唐大旱井泉涸淮水可涉饑民度淮而北者相繼濠夀發兵禦之民與兵鬬而北來【觀民心之向背唐之君臣可以岌岌矣】帝聞之曰彼我之民一也聽糴米過淮唐人遂築倉多糴以供軍八月己未詔唐民以人畜負米者聽之以舟車運載者勿予【予讀曰與】 王逵遣使上表誣劉言謀以朗州降唐又欲攻潭州其衆不從廢而囚之臣已至朗州撫安軍府訖且請復移使府治潭州【去年劉言表移使府於朗州】甲戌遣通事舍人翟光裔詣湖南宣撫從其所請【翟萇伯翻又徒歷翻】逵還長沙以周行逢知朗州事又遣潘叔嗣殺劉言於朗州【為潘叔嗣殺王逵周行逢殺叔嗣張本】 九月己亥武成節度使白重贊奏塞决河【滑州自唐以來置義成節度宋朝太平興國元年以太宗舊名始改為武成軍於此時武當作義塞悉則翻】 契丹寇樂夀齊州戍兵右保寜都頭劉漢章殺都監杜延熙謀應契丹不克并其黨伏誅 南漢主立其子繼興為衛王璇興為桂王慶興為荆王保興為禎王崇興為梅王 東自青徐南至安復西至丹慈【丹州在龍門河之西慈州在龍門河之東宋朝熙寧五年廢慈州以吉鄉縣屬隰州九域志吉鄉縣在隰州西南一百六十里】北至貝鎮皆大水 帝自入秋得風痺疾【痺必至翻又毗至翻】害於食飲及步趨術者言宜散財以禳之帝欲祀南郊又以自梁以來郊祀常在洛陽疑之執政曰天子所都則可以祀百神何必洛陽於是始築圓丘社稷壇作太廟於大梁【自梁都大梁以來建立郊廟皆所未遑晉天福四年太常禮院奏唐廟制度請以至德宫正殿隔為五室而已今始作太廟】癸亥遣馮道迎太廟社稷神主于洛陽 南漢大赦 冬十一月己丑太常請凖洛陽築四郊諸壇從之十二月丁未朔神主至大梁帝迎于西郊祔享于太廟 鄴都留守天雄節度使兼侍衛親軍都指揮使同平章事王殷恃功專横【恃佐命之功也横戶孟翻】凡河北鎮戍兵應用敇處分者殷即以帖行之又多掊歛民財【處昌呂翻分扶問翻掊蒲侯翻】帝聞之不悦使人謂曰卿與國同體鄴都帑庾甚豐【帑他朗翻】卿欲用則取之何患無財成德節度使何福進素惡殷【惡烏路翻】甲子福進入朝密以殷隂事白帝帝由是疑之乙丑殷入朝詔留殷充京城内外巡檢戊辰府州防禦使折德扆奏北漢將喬贇入寇【贇於倫翻】<br />
<br />
  擊走之 王殷每出入從者常數百人殷請量給鎧仗以備巡邏【從才用翻量音良邏郎佐翻因充京城内外巡檢遂有此請】帝難之時帝體不平將行郊祀而殷挾震主之勢在左右衆心忌之壬申帝力疾御滋德殿殷入起居遂執之下制誣殷謀以郊祀日作亂流登州出城殺之命鎭寧節度使鄭仁誨詣鄴都安撫仁誨利殷家財擅殺殷子遷其家屬於登州 唐祠部郎中知制誥徐鉉言貢舉初設不宜遽罷乃復行之【唐罷貢舉事見上卷上年】先是楚州刺史田敬洙請修白水塘溉田以實邊【先悉薦翻白水塘在楚州實應縣西八十里鄧艾所築也】馮延己以爲便李德明因請大闢曠土爲屯田修復所在渠塘堙廢者吏因緣侵擾大興力役奪民田甚衆民愁怨無訴徐鉉以白唐主唐主命鉉按視之鉉籍民田悉歸其主或譖鉉擅作威福唐主怒流鉉舒州然白水塘竟不成唐主又命少府監馮延魯廵撫諸州右拾遺徐鍇表延魯無才多罪舉措輕淺不宜奉使唐主怒貶鍇校書郎分司東都鍇鉉之弟也【錯口駭翻唐以揚州爲東都史言唐主惑於二馮而罪二徐路振九國志鉉鍇皆徐延休之子】 道州盤容洞蠻酋盤崇聚衆自稱盤容州都統屢寇郴道州【酋慈秋翻盤姓也即盤瓠之後郴道二州時皆屬南漢】 乙亥帝朝享太廟被衮冕左右掖以登堦【朝直遥翻被皮義翻掖羊益翻】纔及一室酌獻俛首不能拜而退【俛音免】命晉王榮終禮是夕宿南郊疾尤劇幾不救夜分小愈【劇甚也增也幾居希翻】<br />
<br />
  顯德元年春正月丙子朔帝祀圓丘僅能瞻仰致敬而已進爵奠幣皆有司代之大赦改元聽蜀境通商【晉天福初蜀猶與中國通開運以後中國多事蜀有吞併關西之志不復與中國通矣】 戊寅罷鄴都【唐莊宗始以魏州爲東京後罷東京以爲鄴都】但爲天雄軍 庚辰加晉王榮兼侍中判内外兵馬事時羣臣希得見帝【見賢遍翻】中外恐懼聞晉王典兵人心稍安 軍士有流言郊賞薄於唐明宗時者【唐明宗以軍士流言濫賞養成其驕莫肯效命何足法也】帝聞之壬午召諸將至寢殿讓之曰朕自即位以來惡衣菲食專以贍軍爲念府庫蓄積四方貢獻贍軍之外鮮有嬴餘【瞻力艶翻鮮息善翻嬴餘經翻】汝輩豈不知之今乃縱凶徒騰口不顧人主之勤儉察國之貧乏又不思已有何功而受賞惟知怨望於汝輩安乎皆惶恐謝罪退索不逞者戮之流言乃息【驕兵於分外希賞苟非以法齊之其無厭之心庸有極乎索山客翻】 初帝在鄴都【漢隱帝天祐三年帝在鄴都】奇愛小吏曹翰之才使之事晉王榮榮鎭澶州以爲牙將榮入爲開封尹【去年三月榮爲開封尹】未即召翰翰自至榮怪之翰請間【間古莧翻】言曰大王國之儲嗣今主上寢疾大王當入侍醫藥奈何猶决事於外邪榮感悟即日入止禁中丙戌帝疾篤停諸司細務皆勿奏有大事則晉王榮禀進止宣行之 以鎭寧節度使鄭仁誨爲樞密使同平章事 戊子以義武留後孫行友保義留後韓通朔方留後馮繼業皆爲節度使通太原人也帝屢戒晉王曰昔吾西征【謂討李守貞王景崇趙思綰時】見唐十八<br />
<br />
  陵無不發掘者【唐高祖太宗高宗中宗睿宗玄宗肅宗代宗德宗順宗憲宗穆宗敬宗文宗武宗宣宗懿宗僖宗凡十八帝皆葬關中陵名各見前紀】此無他惟多藏金玉故也我死當衣以紙衣歛以瓦棺速營葬勿久留宫中壙中無用石以甓代之【當衣於旣翻歛力贍翻甓蒲歷翻摶埴而陶之今謂之甎】工人役徒皆和雇勿以煩民葬畢募近陵民三十戶蠲其雜徭使之守視勿修下宫勿置守陵宫人勿作石羊虎人馬惟刻石置陵前云周天子平生好儉約遺令用紙衣瓦棺嗣天子不敢違也汝或吾違吾不福汝又曰李洪義當與節鉞【以李洪義發漢隱帝密詔也事見二百八十九卷乾祐三年】魏仁浦勿使離樞密院【離力智翻】 庚寅詔前登州刺史周訓等塞决河先是河决靈河魚池酸棗陽武常樂驛河隂六明鎭原武凡八口【九域志滑州白馬縣有靈河鎭魚池亦在滑州界酸棗津在大梁東北陽武在鄭州河隂在孟州東南六明鎭在大通軍大通軍即胡梁渡也晉天福四年建浮橋置大通軍原武在鄭州之北塞悉則翻先悉薦翻】至是分遣使者塞之 帝命趣草制【趣讀曰促】以端明殿學士戶部侍郎王溥爲中書侍郎同平章事壬辰宣制畢左右以聞帝曰吾無恨矣以樞密副使王仁鎬爲永興軍節度使以殿前都指揮使李重進領武信節度使【重直龍翻】馬軍都指揮使樊愛能領武定節度使步軍都指揮使何徽領昭武節度使【殿前都指揮使總殿前諸班馬軍都指揮使總侍衛司馬軍步軍都指揮使總侍衛司步軍宋朝三衙之職昉於此武信軍遂州武定軍洋州昭武軍利州三鎭皆屬蜀李重進等遙領也】重進年長於晉王榮帝召入禁中屬以後事仍命拜榮以定君臣之分【長知兩翻屬之欲翻分扶問翻】是日帝殂于滋德殿【年五十一】袐不發喪乙未宣遺制丙申晉王即皇帝位 【考異曰太祖實録乙未宣遺制晉王榮可於柩前即皇帝位世宗實錄丙申内出太祖遺制羣臣奉帝即皇帝位蓋以乙未宣遺制丙申即位也】 初靜海節度使吳權卒【吳權據交州見二百八十一卷晉高祖天福三年南漢高祖之大有十一年也】子昌岌立昌岌卒【岌魚及翻】弟昌文立是月始請命於南漢南漢以昌文爲靜海節度使兼安南都護 北漢主聞太祖晏駕甚喜謀大舉入寇遣使請兵于契丹二月契丹遣其武定節度使政事令楊衮將萬餘騎如晉陽 【考異曰晉陽見聞録衮帥騎五六七萬號十萬來會今從世宗實鋉】北漢主自將兵三萬以義成節度使白從暉爲行軍都部署武寧節度使張元徽爲前鋒都指揮使【義成軍滑州武寧軍徐州皆屬周白從暉等亦遙領考異曰世宗實録賊將張暉領三千騎爲前鋒今從晉陽聞見實録】與契丹自團柏南趣潞州【趣七喻翻】 蜀左匡聖馬步都指揮使保寧節度使安思謙譖殺張業廢趙廷隱【二事並見二百八十八卷漢乾祐元年】蜀人皆惡之【惡烏路翻】蜀主使將兵救王景崇思謙逗橈無功【見二百八十八卷漢乾祐二年蜀之明德十二年也橈奴教翻】内慙懼不自安自張業之誅宫門守衛加嚴思謙以爲疑已言多不遜思謙典宿衛多殺士卒以立威蜀主閲衛士有年尚壯而爲思謙所斥者復留隸籍思謙殺之蜀主不能平思謙三子扆嗣裔倚父勢暴横爲國人患【横戶孟翻】翰林使王藻【職官分紀唐有翰林使掌伎術之待詔者五代有翰林茶酒使蜀蓋仍唐舊制】屢言思謙怨望將反丁巳思謙入朝蜀主命壯士擊殺之及其三子藻亦坐擅啓邊奏并誅之 北漢兵屯梁侯驛昭義節度使李筠遣其將穆令均將步騎二千逆戰筠自將大軍壁於太平驛【宋白曰梁侯驛在團柏谷南太平驛西北太平驛東南距潞州八十里】張元徽與令均戰陽不勝而北令均逐之伏發殺令均俘斬士卒千餘人筠遁歸上黨【潞州治上黨闕】嬰城自守筠即李榮也【天福十二年李榮有逐滿達勒功見二百八十】<br />
<br />
  【七卷】避上名改焉世宗聞北漢主入寇欲自將兵禦之羣臣皆曰劉崇自平陽遁走以來【謂廣順元年劉崇圍晉州不克而歸也事見上卷】勢蹙氣沮必不敢自來【沮在呂翻】陛下新即位山陵有日人心易搖【易以䜴翻】不宜輕動宜命將禦之帝曰崇幸我大喪輕朕年少新立【少詩照翻】有吞天下之心此必自來朕不可不往馮道固爭之帝曰昔唐太宗定天下未嘗不自行朕何敢偷安道曰未審陛下能爲唐太宗否帝曰以吾兵力之彊破劉崇如山壓卵耳道曰未審陛下能爲山否【馮道歷事八姓身爲宰輔不聞獻替唯諫世宗親征一事】帝不悅惟王溥勸行帝從之 三月乙亥朔蜀主加捧聖控鶴都指揮使兼中書令孫漢韶武信節度使賜爵樂安郡王罷軍職【罷其掌禁兵之職也】蜀主懲安思謙之跋扈命山南西道節度使李廷珪等十人分典禁兵 北漢乘勝進逼潞州【乘梁侯驛之勝也】丁丑詔天雄節度使符彦卿引兵自磁州固鎭出北漢軍後【磁州武安縣有固鎭自此西北行至遼州北漢軍時已攻潞州符彦卿若至遼州界則出其後矣磁墻之翻】以鎭寧節度使郭崇副之又詔河中節度使王彦超引兵自晉州東出邀北漢【九域志晉州東至潞州三百八十五里】以保義節度使韓通副之又命馬軍都指揮使寧江節度使樊愛能步軍都指揮使清淮節度使何徽【寧江軍夔州屬蜀清淮軍壽州屬唐樊何亦遙領也】義成節度使白重贊【重直龍翻】鄭州防禦使史彦超前耀州團練使符彦能將兵先趣澤州【趣七喻翻】宣徽使向訓監之【監古衘翻】重贊憲州人也 辛巳大赦 癸未帝命馮道奉梓宫赴山陵【山陵在鄭州新鄭縣】以鄭仁誨爲東京留守乙酉帝發大梁庚寅至懷州【九域志大梁至懷州三百二十五里】帝欲兼行速進控鶴都指揮使眞定趙晁私謂通事舍人鄭好謙曰【晁直遙翻好呼到翻】賊勢方盛宜持重以挫之好謙言於帝帝怒曰汝安得此言必爲人所使言其人則生不然必死好謙以實對帝命并晁械於州獄【懷州獄也】壬辰帝過澤州【九域志懷州北至澤州一百二十里】宿於州東北北漢主不知帝至過潞州不攻引兵而南是夕軍於高平之南【劉昫曰高平漢泫氏縣地宋白曰漢泫氏縣後魏改泫氏北齊改高平九域志高平縣在澤州東北六十五里】癸巳前鋒與北漢軍遇擊之 【考異曰世宗實録甲午賊陳於高平南之高原按下又有甲午此必癸巳誤也今從十國紀年】北漢兵却帝慮其遁去趣諸軍亟進【趣讀曰促】北漢主以中軍陳於巴公原【陳讀曰陣下同巴公鎭在晉城縣東北】張元徽軍其東楊衮軍其西衆頗嚴整時河陽節度使劉詞將後軍未至衆心危懼而帝志氣益銳命白重進與侍衛馬步都虞候李重進將左軍居西【白重進當作白重贊】樊愛能何徽將右軍居東向訓史彦超將精騎居中央殿前都指揮使張永德將禁兵衛帝帝介馬自臨陳督戰北漢主見周軍少悔召契丹謂諸將曰吾自用漢軍可破也【北漢主未戰而先有輕敵之心宜其敗也】何必契丹今日不惟克周亦可使契丹心服諸將皆以爲然楊衮策馬前望周軍退謂北漢主曰勍敵也【勍渠京翻北人望塵知敵數又觀敵人置陳而知其強弱楊衮必有見於此】未可輕進北漢主奮曰【如占翻】時不可失請公勿言試觀我戰衮默然不悅時東北風方盛俄而忽轉南風北漢副樞密使王延嗣使司天監李義白北漢主云時可戰矣北漢主從之樞密直學士王得中扣馬諫曰義可斬也風勢如此豈助我者邪北漢主曰吾計已决老書生勿妄言且斬汝麾東軍先進張元徽將千騎擊周右軍合戰未幾【幾居豈翻】樊愛能何徽引騎兵先遁右軍潰步兵千餘人解甲呼萬歲降于北漢帝見軍勢危自引親兵犯矢石督戰太祖皇帝時爲宿衛將謂同列曰主危如此吾屬何得不致死又謂張永德曰賊氣驕力戰可破也公麾下多能左射者請引兵乘高出爲左翼我引兵爲右翼以擊之國家安危在此一舉永德從之各將二千人進戰太祖皇帝身先士卒馳犯其鋒士卒死戰無不一當百【先悉薦翻太祖皇帝自此肇基皇業】 北漢兵披靡【披普彼翻】内殿直夏津馬仁瑀謂衆曰使乘輿受敵安用我輩躍馬引弓大呼連斃數十人士氣益振【内殿直周所置殿前諸班之號夏津漢鄃縣唐天寶元年改曰夏津屬貝州九域志屬大名府在府東北二百五十里瑀王矩翻乘䋲證翻呼火故翻】殿前右番行首馬全乂【去年馬全乂自澶州從帝入朝已補殿前指揮使未至散員指揮使也右番行首居殿前右番班行之首其官猶在散員指揮使之下行戶剛翻】言於帝曰賊勢極矣將爲我擒願陛下按轡勿動徐觀諸將破之即引數百騎進陷陳北漢主知帝自臨陳【陳讀曰陣下同】褒賞張元徽趣使乘勝進兵【趣讀曰促】元徽前略陳馬倒爲周兵所殺元徽北漢之驍將也北軍由是奪氣時南風益盛周兵爭奮北漢兵大敗北漢主自舉赤幟以收兵不能止【北漢雖出於沙陀自謂劉氏纂高光之緒故旗幟尚赤】楊衮畏周兵之彊不敢救且恨北漢主之語全軍而退 【考異曰五代史補劉崇求援於契丹得飛騎數千及覩世宗兵少悔之召諸將謀曰吾觀周師易與耳契丹之衆宜勿使但以本軍决戰不唯破敵亦足使契丹見而心服諸將皆以爲然乃使人謂契丹主將曰柴氏與吾主客之勢已見必不煩足下餘刃敢請勒兵登高觀之可也契丹不知其謀從之洎世宗之入陳也三軍皆賈勇爭進莫不一當百契丹望而畏之故不敢救而崇敗今從世宗實録薛史】樊愛能何徽引數千騎南走控弦露刃剽掠輜重【剽匹妙翻重直用翻】役徒驚走失亡甚多帝遣近臣及親軍校追諭止之【校戶教翻】莫肯奉詔使者或爲軍士所殺揚言契丹大至官軍敗績餘衆已降虜矣劉詞遇愛能等於塗愛能等止之詞不從引兵而北時北漢主尚有餘衆萬餘人阻澗而陳薄暮詞至復與諸軍擊之【薄迫也復扶又翻下卒復同】北漢兵又敗殺王延嗣追至高平僵尸滿山谷委棄御物及輜重器械雜畜不可勝紀【僵居良翻勝音升】是夕帝宿於野次得步兵之降敵者皆殺之樊愛能等聞周兵大捷與士卒稍稍復還【還從宣翻】有逹曙不至者甲午休兵于高平選北漢降卒數千人爲効順指揮命前武勝行軍司馬唐景思將之使戍淮上餘二千餘人賜貲裝縱遣之【貲當作資】李穀爲亂兵所迫潛竄山谷數日乃出丁酉帝至潞州北漢主自高平被褐戴笠【被皮義翻褐毛衫也無柄曰笠有柄曰簦】乘契丹所贈黄騮【騮力求翻詩駉注赤馬黑髦曰騮黄色近於赤】帥百餘騎由雕窠嶺遁歸【雕窠嶺在高平西北由江猪嶺路入帥讀曰率】宵迷【夜行而迷失道也】俘村民爲導誤之晉州行百餘里乃覺之殺導者晝夜北走所至得食未舉筯或傳周兵至輒蒼黄而去北漢主衰老力憊【憊蒲拜翻】伏於馬上晝夜馳驟殆不能支僅得入晉陽帝欲誅樊愛能等以肅軍政猶豫未决己亥晝卧行宫帳中張永德侍側帝以其事訪之【張永德太祖壻既親且專掌殿前兵侍衛左右故訪以其事决可否】對曰愛能等素無大功忝冒節鉞望敵先逃死未塞責【塞悉則翻】且陛下方欲削平四海苟軍法不立雖有熊羆之士百萬之衆安得而用之帝擲枕於地大呼稱善【呼火故翻】即收愛能徽及所部軍使以上七十餘人【使疏吏翻】責之曰汝曹皆累朝宿將非不能戰【朝直遙翻將即亮翻下同】今望風奔遁者無他正欲以朕爲奇貨賣與劉崇耳悉斬之帝以何徽先守晉州有功【事見二百九十卷太祖廣順元年】欲免之既而以法不可廢遂并誅之而給槥車歸葬【槥車小棺也槥音衛】自是驕將惰卒始知所懼不行姑息之政矣庚子賞高平之功以李重進兼忠武節度使向訓兼義成節度使張永德兼武信節度使【兼者以本職兼節鎭禄賜優於遙領者】史彦超爲鎭國節度使【此正除節鎭】張永德盛稱太祖皇帝之智勇帝擢太祖皇帝為殿前都虞<br />
<br />
  候【後魏之末宇文置虞候都督以主候騎虞候之官蓋始於此五代殿前都虞候在副都指】<br />
<br />
  【揮使之下與都副指揮使同掌殿前班直】領嚴州刺史【嚴州隸嶺南時為南漢所】<br />
<br />
  【有遙領刺史耳今武臣所領遙郡刺史正此類而落階官正除刺史者謂之正任刺史然亦未】<br />
<br />
  【嘗臨郡治民也劉昫曰嚴州秦桂林郡地唐乾封間招致生獠置嚴州宋開寶七年廢嚴州以來賓縣隸象州】以馬仁瑀爲控鶴弓箭直指揮使馬全又爲散員指揮使自餘將校遷拜者凡數十人【校戶教翻】士卒有自行間擢主軍廂者【行戶剛翻時諸軍皆分左右廂廂各有主帥按薛史自五季至宋武官有軍主廂主曹威爲奉國軍主遷本軍廂主劉延欽爲控鶴主是其徵也】釋趙晁之囚【囚趙晁所以威衆戰勝則釋之】北漢主收散卒繕甲兵完城塹以備周楊衮將其衆北屯代州北漢主遣王得中送衮因求救於契丹契丹主遣得中還報許發兵救晉陽壬寅以符彦卿爲河東行營都部署兼知太原行府事以郭崇副之向訓爲都監李重進爲馬步都虞候史彦超爲先鋒都指揮使將步騎二萬發潞州仍詔王彦超韓通自隂地關入【北漢既敗走移晉州東出之師北攻汾并】與彦卿合軍而進又以劉詞爲隨駕部署保大節度使白重贊副之【乘勝進攻晉陽隨駕之下當有都字】 漢昭聖皇太后李氏殂于西宫【周太祖既踐阼漢太后李氏遷居西宫事見上卷廣順元年】 夏四月北漢盂縣降【盂古縣唐屬太原府九域志在府東北二百里然宋下太原徙治陽曲宋白曰盂縣本晉大夫盂丙之邑漢爲盂縣按前此盂縣在今縣西陽曲縣東北八十里故盂縣城是也後魏省屬石艾縣隋開皇十六年分石艾縣置原仇縣屬遼州因原仇故城爲名即今縣是也大業二年改原仇爲盂縣從漢舊名】符彦卿軍晉陽城下王彦超攻汾州【九域志晉州北至汾州三百五十里】北漢防禦使董希顔降帝遣萊州防禦使康延沼攻遼州【遼州唐之儀州也梁開平三年勑兖州管内已有沂州其儀州改爲遼州九域志潞州東北至遼州二百四十三里】密州防禦使田瓊攻沁州皆不下【唐置沁州至宋太平興國五年廢沁州以和川縣隸晉州熙寧五年省和川縣入冀氏九域志冀氏縣在晉州東二百八十里沁音七鴆翻】供備庫副使太原李謙溥單騎說遼州刺史張漢超【說式苪翻】漢超即降 乙卯葬聖神恭肅文武孝皇帝于嵩陵【三月乙酉梓宫赴山陵四月乙卯方葬與北漢交兵葬備多闕故緩】廟號太祖 南漢主以高王弘邈爲雄武節度使鎭邕州弘邈以齊鎭二王相繼死於邕州固辭【齊王弘弼死見二百八十三卷晉天福八年鎭王弘澤死見二百八十四卷晉開運元年】求宿衛不許至鎭委政僚佐日飲酒禱鬼神或上書誣弘邈謀作亂戊午南漢主遣甘泉宫使林延遇賜酖殺之 初帝遣符彦卿等北征但欲耀兵於晉陽城下未議攻取既入北漢境其民爭以食物迎周師泣訴劉氏賦役之重願供軍須助攻晉陽北漢州縣繼有降者帝聞之始有兼并之意【史言謀不先定者非廟勝之策】遣使往與諸將議之諸將皆言芻糧不足請且班師以俟再舉帝不聽【師有歸志宜其無功】既而諸軍數十萬聚於太原城下軍士不免剽掠北漢民失望【剽匹妙翻】稍稍保山谷自固帝聞之馳詔禁止剽掠安撫農民止徵今歲租税及募民入粟拜官有差仍發澤潞晉絳慈隰及山東近便諸州民運糧以饋軍【山東近便諸州謂邢趙鎭定】己未遣李穀詣太原計度芻糧【度徒洛翻】 庚申太師中書令瀛文懿王馮道卒 【考異曰五代通録謚曰文愍今從世宗實録薛史】道少以孝謹知名【以此知名人所難能也少詩照翻】唐莊宗世始貴顯【馮道事劉守光位不過參軍入唐始貴顯】自是累朝不離將相三公三師之位【離力智翻唐制太師太傅太保爲三師太尉司徒司空爲三公】爲人清儉寛弘人莫測其喜愠滑稽多智浮沈取容【滑音骨沈持林翻】嘗著長樂老叙自述累朝榮遇之狀【馮道長樂老叙既自陳其榮遇又自謂孝於家忠於國爲子爲弟爲人臣爲師長爲夫爲父有子有孫時開一卷時飲一杯食味别聲被色老安於當代老而自樂何樂如之其自述如此】時人往往以德量推之<br />
<br />
  歐陽修論曰禮義廉恥國之四維四維不張國乃滅亡【管子之言】禮義治人之大法【治直之翻】廉恥立人之大節况爲大臣而無廉恥天下其有不亂國家其有不亡者乎予讀馮道長樂老叙見其自述以爲榮其可謂無廉恥者矣則天下國家可從而知也予於五代得全節之士三【謂王彦章裴約劉仁瞻】死事之人十有五【謂張源德夏魯奇姚洪王思同張敬逹翟進宗沈斌王清史彦超孫晟馬彦超宋令珣李遐張彦卿鄭昭業凡十五人】皆武夫戰卒豈於儒者果無其人哉得非高節之士惡時之亂【惡烏路翻】薄其世而不肯出歟抑君天下者不足顧而莫能致之歟予嘗聞五代時有王凝者家青齊之間爲虢州司戶參軍以疾卒于官凝家素貧一子尚幼妻李氏攜其子負其遺骸以歸東過開封止於旅舍主人不納李氏顧天已暮不肯去主人牽其臂而出之李氏仰天慟曰我爲婦人不能守節而此手爲人所執邪即引斧自斷其臂見者爲之嗟泣【斷音短爲之于偽翻】開封尹聞之白其事於朝厚卹李氏而笞其主人【此事歐陽公得之於五代小說朝直遙翻下同】嗚呼士不自愛其身而忍恥以偷生者聞李氏之風宜少知愧哉【少詩沼翻】<br />
<br />
  臣光曰天地設位聖人則之以制禮立法内有夫婦外有君臣婦之從夫終身不改臣之事君有死無貳此人道之大倫也苟或廢之亂莫大焉范質稱馮道厚德稽古宏才偉量雖朝代遷貿人無間言【貿音茂易也間古莧翻】屹若巨山不可轉也【夷考范質之爲人盖學馮道者也□與屹同魚迄翻】臣愚以爲正女不從二夫忠臣不事二君爲女不正雖復華色之美織絍之巧不足賢矣爲臣不忠雖復材智之多治行之優不足貴矣【絍汝鴆翻治直吏翻行下孟翻】何則大節已虧故也道之為相歷五朝八姓【五朝唐晉遼漢周八姓唐莊宗明宗潞王各為一姓石晉邪律劉漢周太祖世宗各為一姓】若逆旅之視過客朝為仇敵暮為君臣易面變辭曾無愧怍【怍疾各翻】大節如此雖有小善庸足稱乎或以爲自唐室之亡羣雄力爭帝王興廢遠者十餘年近者四三年雖有忠智將若之何當是之時失臣節者非道一人豈得獨罪道哉臣愚以爲忠臣憂公如家見危致命君有過則彊諫力爭國敗亡則竭節致死智士邦有道則見【見賢遍翻】邦無道則隱或滅迹山林或優遊下僚今道尊寵則冠三師權任則首諸相【冠古玩翻相息亮翻】國存則依違拱默竊位素餐國亡則圖全苟免迎謁勸進君則興亡接踵道則富貴自如兹乃奸臣之尤安得與他人爲比哉或謂道能全身遠害於亂世斯亦賢已【遠于願翻】臣謂君子有殺身成仁無求生害仁【引論語夫子之言】豈專以全身遠害為賢哉然則盜跖病終而子路醢果誰賢乎【盜跖從卒九千横行天下而以壽終子路仕衛孔悝之難子路死之菹於衛東門之上】抑此非特道之愆也【愆過也】時君亦有責焉【時君謂五朝八姓之君】何則不正之女中士羞以爲家不忠之人中君羞以爲臣【中士中君以人品言謂識見不及上而可以語上者】彼相前朝語其忠則反君事讐語其智則社稷為墟後來之君不誅不棄乃復用以爲相【復扶又翻】彼又安肯盡忠於我而能獲其用乎故曰非特道之愆亦時君之責也【温公以此警後世之君臣深矣】<br />
<br />
  辛酉符彦卿奏北漢憲州刺史太原韓光愿嵐州刺史郭言皆舉城降【屬郡雖降而都府未克終於無益大軍既退則其地復為敵有矣】 初符彦卿有女適李守貞之子崇訓相者言其貴當爲天下母守貞喜曰吾婦猶母天下况我乎反意遂决及敗崇訓先刃其弟妹次及符氏符氏匿幃下崇訓倉猝求之不獲遂自剄【剄古頂翻】亂兵既入符氏安坐堂上叱亂兵曰吾父與郭公爲昆弟汝曹勿無禮太祖遣使歸之於彦卿及帝鎭澶州【廣順元年帝鎭澶州三年入為開封尹】太祖為帝娶之【為于偽翻】壬戌立爲皇后后性和惠而明决帝甚重之 王彦超韓通攻石州克之執刺史安彦進癸亥沁州刺史李廷誨降庚午帝發潞州趣晉陽【趣七喻翻】癸酉北漢忻州監軍李勍殺刺史趙臯及契丹通事楊訥呼【勍渠京翻訥奴臘翻】舉城降以勍爲忻州刺史 王逵表請復徙使府治朗州【去年王逵移使府於潭州復扶又翻治直之翻】<br />
<br />
  資治通鑑卷二百九十一  <br>
   </div> 

<script src="/search/ajaxskft.js"> </script>
 <div class="clear"></div>
<br>
<br>
 <!-- a.d-->

 <!--
<div class="info_share">
</div> 
-->
 <!--info_share--></div>   <!-- end info_content-->
  </div> <!-- end l-->

<div class="r">   <!--r-->



<div class="sidebar"  style="margin-bottom:2px;">

 
<div class="sidebar_title">工具类大全</div>
<div class="sidebar_info">
<strong><a href="http://www.guoxuedashi.com/lsditu/" target="_blank">历史地图</a></strong>  
<a href="http://www.880114.com/" target="_blank">英语宝典</a>  
<a href="http://www.guoxuedashi.com/13jing/" target="_blank">十三经检索</a> 
<br><strong><a href="http://www.guoxuedashi.com/gjtsjc/" target="_blank">古今图书集成</a></strong> 
<a href="http://www.guoxuedashi.com/duilian/" target="_blank">对联大全</a> <strong><a href="http://www.guoxuedashi.com/xiangxingzi/" target="_blank">象形文字典</a></strong> 

<br><a href="http://www.guoxuedashi.com/zixing/yanbian/">字形演变</a>  <strong><a href="http://www.guoxuemi.com/hafo/" target="_blank">哈佛燕京中文善本特藏</a></strong>
<br><strong><a href="http://www.guoxuedashi.com/csfz/" target="_blank">丛书&方志检索器</a></strong> <a href="http://www.guoxuedashi.com/yqjyy/" target="_blank">一切经音义</a>  

<br><strong><a href="http://www.guoxuedashi.com/jiapu/" target="_blank">家谱族谱查询</a></strong>  <strong><a href="http://shufa.guoxuedashi.com/sfzitie/" target="_blank">书法字帖欣赏</a></strong> 
<br>

</div>
</div>


<div class="sidebar" style="margin-bottom:0px;">

<font style="font-size:22px;line-height:32px">QQ交流群9:489193090</font>


<div class="sidebar_title">手机APP 扫描或点击</div>
<div class="sidebar_info">
<table>
<tr>
	<td width=160><a href="http://m.guoxuedashi.com/app/" target="_blank"><img src="/img/gxds-sj.png" width="140"  border="0" alt="国学大师手机版"></a></td>
	<td>
<a href="http://www.guoxuedashi.com/download/" target="_blank">app软件下载专区</a><br>
<a href="http://www.guoxuedashi.com/download/gxds.php" target="_blank">《国学大师》下载</a><br>
<a href="http://www.guoxuedashi.com/download/kxzd.php" target="_blank">《汉字宝典》下载</a><br>
<a href="http://www.guoxuedashi.com/download/scqbd.php" target="_blank">《诗词曲宝典》下载</a><br>
<a href="http://www.guoxuedashi.com/SiKuQuanShu/skqs.php" target="_blank">《四库全书》下载</a><br>
</td>
</tr>
</table>

</div>
</div>


<div class="sidebar2">
<center>


</center>
</div>

<div class="sidebar"  style="margin-bottom:2px;">
<div class="sidebar_title">网站使用教程</div>
<div class="sidebar_info">
<a href="http://www.guoxuedashi.com/help/gjsearch.php" target="_blank">如何在国学大师网下载古籍?</a><br>
<a href="http://www.guoxuedashi.com/zidian/bujian/bjjc.php" target="_blank">如何使用部件查字法快速查字?</a><br>
<a href="http://www.guoxuedashi.com/search/sjc.php" target="_blank">如何在指定的书籍中全文检索?</a><br>
<a href="http://www.guoxuedashi.com/search/skjc.php" target="_blank">如何找到一句话在《四库全书》哪一页?</a><br>
</div>
</div>


<div class="sidebar">
<div class="sidebar_title">热门书籍</div>
<div class="sidebar_info">
<a href="/so.php?sokey=%E8%B5%84%E6%B2%BB%E9%80%9A%E9%89%B4&kt=1">资治通鉴</a> <a href="/24shi/"><strong>二十四史</strong></a>&nbsp; <a href="/a2694/">野史</a>&nbsp; <a href="/SiKuQuanShu/"><strong>四库全书</strong></a>&nbsp;<a href="http://www.guoxuedashi.com/SiKuQuanShu/fanti/">繁体</a>
<br><a href="/so.php?sokey=%E7%BA%A2%E6%A5%BC%E6%A2%A6&kt=1">红楼梦</a> <a href="/a/1858x/">三国演义</a> <a href="/a/1038k/">水浒传</a> <a href="/a/1046t/">西游记</a> <a href="/a/1914o/">封神演义</a>
<br>
<a href="http://www.guoxuedashi.com/so.php?sokeygx=%E4%B8%87%E6%9C%89%E6%96%87%E5%BA%93&submit=&kt=1">万有文库</a> <a href="/a/780t/">古文观止</a> <a href="/a/1024l/">文心雕龙</a> <a href="/a/1704n/">全唐诗</a> <a href="/a/1705h/">全宋词</a>
<br><a href="http://www.guoxuedashi.com/so.php?sokeygx=%E7%99%BE%E8%A1%B2%E6%9C%AC%E4%BA%8C%E5%8D%81%E5%9B%9B%E5%8F%B2&submit=&kt=1"><strong>百衲本二十四史</strong></a>  <a href="http://www.guoxuedashi.com/so.php?sokeygx=%E5%8F%A4%E4%BB%8A%E5%9B%BE%E4%B9%A6%E9%9B%86%E6%88%90&submit=&kt=1"><strong>古今图书集成</strong></a>
<br>

<a href="http://www.guoxuedashi.com/so.php?sokeygx=%E4%B8%9B%E4%B9%A6%E9%9B%86%E6%88%90&submit=&kt=1">丛书集成</a> 
<a href="http://www.guoxuedashi.com/so.php?sokeygx=%E5%9B%9B%E9%83%A8%E4%B8%9B%E5%88%8A&submit=&kt=1"><strong>四部丛刊</strong></a>  
<a href="http://www.guoxuedashi.com/so.php?sokeygx=%E8%AF%B4%E6%96%87%E8%A7%A3%E5%AD%97&submit=&kt=1">說文解字</a> <a href="http://www.guoxuedashi.com/so.php?sokeygx=%E5%85%A8%E4%B8%8A%E5%8F%A4&submit=&kt=1">三国六朝文</a>
<br><a href="http://www.guoxuedashi.com/so.php?sokeytm=%E6%97%A5%E6%9C%AC%E5%86%85%E9%98%81%E6%96%87%E5%BA%93&submit=&kt=1"><strong>日本内阁文库</strong></a> <a href="http://www.guoxuedashi.com/so.php?sokeytm=%E5%9B%BD%E5%9B%BE%E6%96%B9%E5%BF%97%E5%90%88%E9%9B%86&ka=100&submit=">国图方志合集</a> <a href="http://www.guoxuedashi.com/so.php?sokeytm=%E5%90%84%E5%9C%B0%E6%96%B9%E5%BF%97&submit=&kt=1"><strong>各地方志</strong></a>

</div>
</div>


<div class="sidebar2">
<center>

</center>
</div>
<div class="sidebar greenbar">
<div class="sidebar_title green">四库全书</div>
<div class="sidebar_info">

《四库全书》是中国古代最大的丛书,编撰于乾隆年间,由纪昀等360多位高官、学者编撰,3800多人抄写,费时十三年编成。丛书分经、史、子、集四部,故名四库。共有3500多种书,7.9万卷,3.6万册,约8亿字,基本上囊括了古代所有图书,故称“全书”。<a href="http://www.guoxuedashi.com/SiKuQuanShu/">详细>>
</a>

</div> 
</div>

</div>  <!--end r-->

</div>
<!-- 内容区END --> 

<!-- 页脚开始 -->
<div class="shh">

</div>

<div class="w1180" style="margin-top:8px;">
<center><script src="http://www.guoxuedashi.com/img/plus.php?id=3"></script></center>
</div>
<div class="w1180 foot">
<a href="/b/thanks.php">特别致谢</a> | <a href="javascript:window.external.AddFavorite(document.location.href,document.title);">收藏本站</a> | <a href="#">欢迎投稿</a> | <a href="http://www.guoxuedashi.com/forum/">意见建议</a> | <a href="http://www.guoxuemi.com/">国学迷</a> | <a href="http://www.shuowen.net/">说文网</a><script language="javascript" type="text/javascript" src="https://js.users.51.la/17753172.js"></script><br />
  Copyright &copy; 国学大师 古典图书集成 All Rights Reserved.<br>
  
  <span style="font-size:14px">免责声明:本站非营利性站点,以方便网友为主,仅供学习研究。<br>内容由热心网友提供和网上收集,不保留版权。若侵犯了您的权益,来信即刪。scp168@qq.com</span>
  <br />
ICP证:<a href="http://www.beian.miit.gov.cn/" target="_blank">鲁ICP备19060063号</a></div>
<!-- 页脚END --> 
<script src="http://www.guoxuedashi.com/img/plus.php?id=22"></script>
<script src="http://www.guoxuedashi.com/img/tongji.js"></script>

</body>
</html>
