\chapter{資治通鑑卷一百三十六}
宋 司馬光 撰

胡三省 音註

齊紀二|{
	起閼達困敦盡屠維大荒落凡六年}


世祖武皇帝上之下

永明二年春正月乙亥以後將軍柳世隆為尚書右僕射竟陵王子良為護軍將軍兼司徒領兵置佐鎮西州子良少有清尚|{
	少詩照翻}
傾意賓客才雋之士皆遊集其門開西邸|{
	據子良傳西邸在雞籠山}
多聚古人器服以充之記室參軍范雲蕭琛樂安任昉法曹參軍王融衛軍東閤祭酒蕭衍|{
	記室參軍掌書記法曹參軍掌刑法此皆子良府屬也時王儉為衛將軍辟蕭衍為東閤祭酒自晉以來公府屬長史之下有東西閤祭酒琛丑林翻任音壬昉孚往翻}
鎮西功曹謝朓|{
	朓土了翻}
步兵校尉沈約揚州秀才吳郡陸倕|{
	校戶教翻倕是為翻}
竝以文學尤見親待號曰八友法曹參軍柳惲|{
	惲於粉翻}
太學博士王僧孺|{
	晉武帝置太學博士太常博士國子博士}
南徐州秀才濟陽江革|{
	革南徐州所舉秀才也濟陽郡時屬南徐州濟子禮翻}
尚書殿中郎范縝|{
	魏晉以來尚書諸曹殿中郎為諸曹之首縝章忍翻}
會稽孔休源亦預焉|{
	會工外翻}
琛惠開之從子|{
	蕭惠開見一百三十一卷宋明帝泰始元年二年從才用翻下同}
惲元景之從孫融僧達之孫|{
	柳元景以武功顯於宋文武二朝王僧達以世資才俊進}
衍順之之子|{
	蕭順之太祖族弟}
朓述之孫約璞之子僧孺雅之曾孫|{
	謝遽見一百二十三卷宋文帝元嘉十七年沈璞守盱眙有功元嘉二十七年孝武孝建之初以不迎義師戮王雅見一百七卷晉孝武太元十五年}
縝雲之從兄也|{
	縝章忍翻}
子良篤好釋氏招致名僧講論佛法道俗之盛江左未有或親為衆僧賦食行水|{
	好呼到翻為于偽翻賦分畀也}
世頗以為失宰相體范縝盛稱無佛子良曰君不信因果|{
	釋氏冇因緣果報之說}
何得有富貴貧賤縝曰人生如樹花同發隨風而散或拂簾幌墜茵席之上|{
	幌呼廣翻}
或關籬牆落糞溷之中墜茵席者殿下是也落糞溷者下官是也貴賤雖復殊途|{
	復扶又翻}
因果竟在何處子良無以難|{
	難乃旦翻下難之同}
縝又著神滅論以為形者神之質神者形之用也神之于形猶利之於刀未聞刀沒而利存豈容形亡而神在哉此論出朝野諠譁難之終不能屈|{
	朝直遥翻}
太原王琰著論譏縝曰嗚呼范子曾不知其先祖神靈所在欲以杜縝後對縝對曰嗚呼王子知其先祖神靈所在而不能殺身以從之子良使王融謂之曰以卿才美何患不至中書郎而故乖刺為此論|{
	中書郎即謂中書侍郎也刺來葛翻}
甚可惜也宜急毀棄之縝大笑曰使范縝賣論取官已至令僕矣|{
	令僕謂尚書令及兩僕射}
何但中書郎邪蕭衍好籌畧有文武才幹|{
	好呼到翻}
王儉深器異之曰蕭郎出三十貴不可言|{
	蕭衍事始此}
壬寅以柳世隆為尚書左僕射丹陽尹李安民為右僕射王儉領丹陽尹 夏四月甲寅魏主如方山戊午還宫庚申如鴻池|{
	鴻池即旋鴻池也水經注凉城郡旋鴻縣東山下水積成池東西二里南北四里又太祖天與二年穿鴻鴈池於平城}
丁卯還宫五月甲申魏遣員外散騎常侍李彪等來聘|{
	散悉亶翻騎奇寄翻}
六月壬寅朔中書舍人吳興茹法亮封望蔡男|{
	康曰茹人}


|{
	諸切姓也望蔡縣屬豫章郡沈約曰漢靈帝中平中汝南上蔡民分徙此城立縣名曰上蔡晉武帝太康元年更名望蔡宋白曰望蔡縣本漢建成縣靈帝分置上蔡縣晉武帝以上蔡人思本土改為望蔡今為高安縣瑞州治所}
時中書舍人四人各住一省謂之四戶以法亮及臨海呂文顯等為之既總重權勢傾朝廷守宰數遷換去來四方餉遺歲數百萬法亮嘗於衆中語人曰何須求外祿此一戶中年辦百萬蓋約言之也|{
	數所角翻遺于季翻語牛倨翻李延夀曰中書所司掌在機務漢元以令僕用事魏明以監令專權至宋孝武以來士庶雜選及明帝世胡母顥阮佃夫之徒專為侯倖矣齊初亦用久勞及以親信關讞表啓發署詔勑頗涉詞輸亦為詔文侍郎之局復見侵矣建武詔命始不關中書專出舍人省内舍人四人所直四省據此四戶則舍人分住四省自法壳等始}
後因天文有變王儉極言文顯等專權徇私上天見異禍由四戶|{
	見賢遍翻}
上手詔酬答而不能改也 魏舊制戶調帛二匹絮二斤絲一斤穀二十斛又入帛一匹二丈委之州庫以供調外之費所調各隨土之所出丁卯詔曰置官班祿行之尚矣自中原喪亂兹制中絶朕憲章舊典始班俸祿戶增調帛三匹穀二斛九斗以為官司之祿增調外帛二匹祿行之後滿一匹者死變法改度宜為更始|{
	調徒弔翻俸扶用翻更工衡翻}
其大赦天下 秋七月甲申立皇子子倫為巴陵王 乙未魏主如武州山石窟寺 九月魏詔班祿以十月為始季别受之|{
	三月為一季}
舊律枉法十匹義二十匹罪死至是義一匹枉法無多少皆死|{
	枉法謂受賕枉法而出入人罪者義贓謂人私情相饋遺雖非乞取亦計所受論贓}
仍分命使者糾按守宰之貪者秦益二州刺史恒農李洪之以外戚貴顯|{
	魏顯祖高祖皆李氏出魏避顯祖諱改弘農為恒農}
為治貪暴|{
	治直吏翻}
班祿之後洪之首以敗魏主命鎖赴平城集百官親臨數之|{
	數所具翻數其罪也}
猶以其大臣聽在家自裁自餘守宰坐死者四十餘人受祿者無不跼蹐|{
	跼音局蹐音脊}
賕賂殆絶然吏民犯它罪者魏主率寛之疑罪奏讞多減死徙邊歲以千計都下決大辟歲不過五六人州鎮亦簡久之淮南王佗奏請依舊斷祿|{
	讞魚列翻又魚蹇翻辟毗亦翻斷丁管翻}
文明太后召羣臣議之中書監高閭以為飢寒切身慈母不能保其子今給祿則廉者足以無濫貪者足以勸慕不給則貪者得肆其姦廉者不能自保淮南之議不亦謬乎詔從閭議閭又上表以為北狄悍愚難以防遏|{
	悍侯旰翻又下罕翻}
所長者野戰所短者攻城|{
	北狄指蠕蠕也}
若以狄之所短奪其所長則雖衆不能成患雖來不能深入又狄散居野澤隨逐水草戰則與家業並至奔則與畜牧俱逃不齎資糧而飲食自足是以歷代能為邊患六鎮勢分倍衆不鬬|{
	謂敵人衆力加倍則鎮人不敢鬬也}
互相圍逼難以制之請依秦漢故事於六鎮之北築長城|{
	魏世祖破蠕蠕列置降人於漢南東至濡源西暨五原隂山竟三千里分為六鎮今武川撫冥懷朔懷荒柔玄禦夷也下云六鎮東西不過千里則當自代都北塞而東至濡源耳杜佑曰後魏六鎮並在馬邑雲中單于府界}
擇要害之地往往開門造小城於其側置兵扞守狄既不攻城野掠無獲草盡則走終必懲艾計六鎮東西不過千里一夫一月之功可城三步之地彊弱相兼不過用十萬人一月可就雖有暫勞可以永逸凡長城有五利罷遊防之苦一也北部放牧無抄掠之患二也|{
	抄楚交翻}
登城觀敵以逸待勞三也息無時之備四也歲常遊運|{
	遊行也行運芻糧以實塞下}
永得不匱五也魏主優詔答之 冬十月丁巳以南徐州刺史長沙王晃為中書監初太祖臨終以晃屬帝使處於輦下或近藩|{
	屬之欲翻處昌呂翻}
勿令遠出且曰宋氏若非骨肉相殘它族豈得乘其弊汝深誡之舊制諸王在都唯得置捉刀左右四十人|{
	捉刀執刀以衛左右者也}
晃好武飾及罷南徐州私載數百人仗還建康為禁司所覺投之江水|{
	禁司主防禁諸王好呼到翻下同}
帝聞之大怒將糾以法豫章王嶷叩頭流涕曰晃罪誠不足宥陛下當憶先朝念晃帝亦垂泣由是終無異意然亦不被親寵論者謂帝優於魏文減於漢明|{
	魏文防禁任城陳諸王漢明友愛東海東干諸王嶷魚力翻朝直遥翻被皮義翻}
武陵王曅多材藝而踈悻|{
	悻直也狠也音胡頂翻}
亦無寵於帝嘗侍宴醉伏地貂抄肉柈|{
	抄楚交翻柈薄官翻}
帝笑曰肉汙貂|{
	汙烏故翻}
對曰陛下愛羽毛而踈骨肉帝不悦曅輕財好施|{
	施式智翻}
故無蓄積名後堂山曰首陽蓋怨貧薄也 高麗王璉遣使入貢於魏亦入貢於齊時高麗方彊魏置諸國使邸齊使第一高麗次之|{
	麗力知翻使疏吏翻下同}
益州大度獠恃險驕恣|{
	水經注南安縣有濛水即大度水東入于江寰宇記大度河自吐蕃界經雅州諸部落至黎州東界流入通莖界獠魯皓翻}
前後刺史不能制及陳顯逹為刺史遣使責其租賧|{
	賧吐濫翻夷人以財贖罪曰賧}
獠帥曰兩眼刺史尚不敢調我|{
	帥所類翻調徒弔翻}
况一眼乎遂殺其使顯逹分部將吏聲言出獵夜往襲之男女無少長皆斬之|{
	分扶問翻將即亮翻下同少詩照翻長知兩翻}
晉氏以來益州刺史皆以名將為之十一月丁亥帝始以始興王鑑為督益寜諸軍事益州刺史徵顯逹為中護軍先是劫帥韓武方聚黨千餘人斷流為暴|{
	先悉薦翻斷丁管翻}
郡縣不能禁鑑行至上明武方出降|{
	降戶江翻下同}
長史虞悰等咸請殺之|{
	悰徂宗翻}
鑑曰殺之失信且無以勸善乃啓臺而宥之於是巴西蠻夷為寇暴者皆望風降附鑑時年十四行至新城|{
	新城今房州}
道路籍籍云陳顯逹大選士馬不肯就徵乃停新城遣典籖張曇晢往觀形勢俄而顯逹遣使詣鑑咸勸鑑執之鑑曰顯逹立節本朝必自無此|{
	曇徒含翻晢先擊翻使疏吏翻朝直遥翻下同}
居二日曇晢還具言顯逹已遷家出城日夕望殿下至於是乃前鑑喜文學|{
	喜許記翻}
器服如素士蜀人悦之 乙未魏員外散騎常侍李彪等來聘|{
	歲悉亶翻騎奇寄翻 考異曰齊紀十二月庚申虜使李道固至今從後魏帝紀}
是歲詔增豫章王嶷封邑為四千戶宋元嘉之世諸王入齋閣得白服帬㡌見人主|{
	宋齊之間制高屋㡌下帬蓋帬渠云翻見賢遍翻}
唯出太極四廂乃備朝服|{
	太極殿前殿也有四廂}
自後此制遂絶上於嶷友愛宫中曲宴聽依元嘉故事嶷固辭不敢唯車駕至其第乃白服烏紗㡌以侍宴至於衣服器用制度動皆陳啓事無專制務從減省上竝不許嶷常慮盛滿求解揚州以授竟陵王子良上終不許曰畢汝一世無所多言嶷長七尺八寸善修容範文物衛從禮冠百僚|{
	長直亮翻從才用翻冠古玩翻}
每出入殿省瞻望者無不肅然 交州刺史李叔獻既受命|{
	命叔獻為交州刺史見上卷太祖建元元年}
而斷割外國貢獻上欲討之|{
	斷丁管翻}


三年春正月丙辰以大司農劉楷為交州刺史發南康廬陵始興兵以討叔獻叔獻聞之遣使乞更申數年獻十二隊純銀兜鍪及孔雀毦|{
	毦仍吏翻以孔雀毛為飾也}
上不許叔獻懼為楷所襲間道自湘州還朝|{
	不敢取道南康始興避劉楷之兵故也間古莧翻}
戊寅魏詔曰圖䜟之興出於三季|{
	三代之季也䜟楚譖翻}
既非經國之典徒為妖邪所憑自今圖䜟祕緯一皆焚之|{
	妖於遥翻緯于貴翻}
留者以大辟論|{
	律凡言以論者罪同真犯辟毗亦翻}
又嚴禁諸巫覡及委巷卜筮非經典所載者|{
	直曰街曲曰巷委即曲也鄭玄曰委巷猶街里委曲所為也覡刑狄翻}
魏馮太后作皇誥十八篇癸未大饗羣臣于太華殿班皇誥|{
	魏高宗興光四年起太華殿}
辛卯上祀南郊大赦 詔復立國學|{
	罷國學見上卷高帝建元四年李延夀曰江左草創日不暇給以迄宋齊國學時或開置而勸課未博建之不能十年蓋取文具而已復扶又翻}
釋奠先師用上公禮 二月己亥魏制皇子皇孫有封爵者歲祿各有差 辛丑上祀北郊 三月丙申魏封皇弟禧為咸陽王幹為河南王羽為廣陵王雍為潁川王勰為始平王|{
	勰音協}
詳為北海王|{
	自禧以下皆魏主之弟}
文明太后令置學舘選師傅以教諸王勰於兄弟最賢敏而好學善屬文|{
	好呼到翻屬之欲翻}
魏主尤奇愛之 夏四月癸丑魏主如方山甲寅還宫 初宋太宗置總明觀以集學士亦謂之東觀上以國學既立五月乙未省總明觀時王儉領國子祭酒詔於儉宅開學士舘以總明四部書充之|{
	分經史子集為甲乙丙丁四部又據宋紀明帝泰始六年立總明觀徵學士以充之舉士二十人分為儒道文史隂陽五部學言隂陽者遂無其人然則四部書者其儒道文史之書歟觀古玩翻}
又詔儉以家為府自宋世祖好文章士大夫悉以文章相尚無以專經為業者儉少好禮學及春秋言論造次必於儒者|{
	好呼到翻造七到翻}
由是衣冠翕然更尚儒術儉撰次朝儀國興自晉宋以來故事無不諳憶|{
	憶記也朝直遥翻下同諳烏含翻}
故當朝理事斷决如流每博議引證八坐丞郎無能異者|{
	八坐丞郎自八坐至左右丞諸曹郎也斷丁亂翻坐徂卧翻}
令史諮事常數十人賓客滿席儉應接辨析傍無留滯言下筆皆有音彩十日一還學監試諸生巾卷在庭|{
	監工銜翻卷巨員翻冠武也鄭註禮記云武冠卷也音起權翻}
劒衛令史儀容甚盛作解散髻|{
	據南史儉傳作解散幘蕭子顯齊書作解散髻斜挿幘簪}
斜挿簪朝野慕之相與倣效儉常謂人曰江左風流宰相唯有謝安意以自比也上深委仗之士流選用奏無不可 六月庚戌進河南王度易侯為車騎將軍遣給事中吳興丘冠先使河南并送柔然使|{
	騎奇寄翻冠古玩翻使疏吏翻}
辛亥魏主如方山丁巳還宫 秋七月癸未魏遣使拜宕昌王梁彌機兄子彌承為宕昌王|{
	宕徒浪翻 考異曰齊書是歲八月丁巳以行宕昌王梁彌頡為河梁二州刺史六年五月甲午以彌承為河涼二州刺史令從魏書}
初彌機死子彌博立為吐谷渾所逼奔仇池|{
	吐從暾入聲谷音浴}
仇池鎮將穆亮以彌機事魏素厚矜其滅亡彌博凶悖所部惡之|{
	將即亮翻悖蒲内翻又蒲沒翻惡烏路翻}
彌承為衆所附表請納之詔許之亮帥騎三萬軍于龍鵠|{
	龍鵠蓋即龍涸也在甘松界宇文氏於此置龍涸防隋為扶州嘉誠縣唐為松州杜佑曰龍涸城吐谷渾南界也去成都千餘里周武帝天和初其主率衆降以為扶州帥讀曰率騎奇寄翻}
擊走吐谷渾立彌承而還|{
	還從宣翻又如字}
亮崇之曾孫也|{
	穆崇見一百一十一卷晉安帝隆安三年}
戊子魏主如魚池|{
	魏太宗永興五年穿魚池於平城北苑}
登青原岡甲午還宫八月己亥如彌澤甲寅登牛頭山甲子還宫 魏初民多䕃附|{
	䕃附者自附於豪強之家以求䕃庇}
䕃附者皆無官役而豪彊徵斂倍於公賦|{
	斂力贍翻}
給事中李安世上言歲飢民流田業多為豪右所占奪|{
	占之贍翻}
雖桑井難復|{
	桑井謂古者井田之制五畝之宅樹墻下以桑也}
宜更均量使力業相稱又所爭之田宜限年斷|{
	量音良稱尺證翻斷丁亂翻}
事久難明悉歸今主以絶詐妄魏主善之由是始議均田冬十月丁未詔遣使者循行州郡|{
	行下孟翻}
與牧守均給天下之田|{
	守式又翻}
諸男夫十五以上受露田四十畝婦人二十畝|{
	杜佑通典注曰不栽樹萟者謂之露田}
奴婢依良丁|{
	良丁謂良人成丁者}
牛一頭受田三十畝限止四牛所授之田率倍之三易之田再倍之以供耕作及還受之盈縮|{
	倍之者合受四十畝授以八十畝此一易之田也三易之田三年耕然後復故故再倍以授之}
人年及課則受田老免及身沒則還田奴婢牛隨有無以還受初受田者男夫給二十畝課種桑五十株桑田皆為世業身終不還恒計見口有盈者無受無還不足者受種如法盈者得賣其盈|{
	恒戶登翻見賢遍翻口分世業之法始此}
諸宰民之官各隨近給公田有差更代相付|{
	更工衡翻}
賣者坐如律 辛酉魏魏郡王陳建卒 魏員外散騎常侍李彪等來聘 十二月乙卯魏以侍中淮南王佗為司徒 柔然犯魏塞魏任城王澄帥衆拒之柔然遁去澄雲之子也|{
	任城王雲見一百三十三卷宋明帝泰始七年任音壬帥讀曰率}
氐羌反詔以澄為都督梁益荆三州諸軍事梁州刺史|{
	魏高祖始置梁益二州於仇池}
澄至州討叛柔服氐羌皆平 初太祖命黄門郎虞玩之等檢定黄籍|{
	見上卷太祖建元二年}
上即位别立校籍官置令史限人一日得數巧|{
	巧謂姦偽也}
既連年不已民愁怨不安外監會稽呂文度|{
	外監屬中領軍而親任過於領軍會工外翻}
啓上籍被却者悉充遠戍|{
	被皮義翻}
民多逃亡避罪富陽民唐㝢之因以妖術惑衆作亂攻陷富陽|{
	富陽即漢富春縣也本屬會稽後屬吳郡晉簡文鄭太后諱春孝武改曰富陽妖於驕翻}
三吳却籍者奔之衆至三萬文度與茹法亮呂文顯皆以姦諂有寵於上|{
	茹音如}
文度為外監專制兵權領軍守虛位而已法亮為中書通事舍人權勢尤盛王儉常曰我雖有大位權寄豈及茹公邪 是歲柔然部真可汗卒子豆崙立|{
	可從刋入聲汗音寒崘盧昆翻}
號伏名敦可汗|{
	魏收曰伏名敦魏言恒也}
改元太平

四年春正月癸亥朔魏高祖朝會始服衮冕|{
	史言魏孝文用夏變夷朝直遥翻}
壬午柔然寇魏邊 唐㝢之攻陷錢唐吳郡諸縣令多棄城走㝢之稱帝於錢唐立太子置百官遣其將高道度等攻陷東陽|{
	將即亮翻}
殺東陽太守蕭崇之崇之太祖族弟也又遣其將孫泓寇山隂至浦陽江|{
	據水經注浦陽江即今曹娥江也水剡溪皆西流至曹娥鎮始折而東流入海}
浹口戍主湯休武擊破之|{
	浹即叶翻}
上禁兵數千人馬數百匹東擊㝢之臺軍至錢唐㝢之衆烏合畏騎兵|{
	騎奇寄翻}
一戰而潰擒斬㝢之進平諸郡縣臺軍乘勝頗縱抄掠|{
	抄楚交翻}
軍還|{
	還從宣翻又如字}
上聞之收軍主前軍將軍陳天福棄市左軍將軍劉明徹免官削爵付東冶|{
	建康有東西二冶今冶城即其地亦曰東冶亭}
天福上寵將也|{
	將即亮翻}
既伏誅内外莫不震肅使通事舍人丹楊劉係宗隨軍慰勞|{
	勞力到翻}
遍至遭賊郡縣百姓被驅逼者悉無所問 閏月癸巳立皇太子貞為邵陵王皇孫昭文為臨汝公 氐王楊後起卒丁未詔以白水太守楊集始為北秦州刺史武都王集始文弘之子也後起弟後明為白水太守魏亦以集始為武都王集始入朝于魏|{
	朝直遥翻}
魏以為南秦州刺史 辛亥帝耕籍田 二月己未立皇弟銶為晉熙王|{
	銶音求}
鉉為河東王 魏無鄉黨之法唯立宗主督護民多隱冒三五十家始為一戶内祕書令李冲上言|{
	祕書省在禁中故謂之内祕書令亦謂之中祕上時掌翻}
宜凖古法五家立鄰長五鄰立里長五里立黨長取鄉人彊謹者為之鄰長復一夫里長二夫黨長三夫|{
	長知兩翻復方目翻}
三載無過則升一等其民調一夫一婦帛一匹粟二石大率十匹為公調二匹為調外費三匹為百官俸此外復有雜調|{
	調徒弔翻俸扶用翻復扶又翻}
民年八十已上聽一子不從役孤獨癃老篤疾貧窮不能自存者三長内迭養食之|{
	食讀曰飤}
書奏詔百官通議中書令鄭羲等皆以為不可太尉丕曰臣謂此法若行於公私有賴但方有事之月校比戶口民必勞怨請過今秋至冬乃遣使者於事為宜冲曰民可使由之不可使知之|{
	論語孔子之言}
若不因調時|{
	調時所謂調課之月}
民徒知立長校戶之勤未見均徭省賦之益心必生怨宜及調課之月令知賦稅之均既識其事又得其利行之差易|{
	易以䜴翻}
羣臣多言九品差調為日已久|{
	九品上中下各分為三品事見一百三十二卷宋明帝泰始五年}
一旦改法恐成擾亂文明太后曰立三長則課調有常凖苞䕃之戶可出僥倖之人可止何為不可|{
	僥堅堯翻}
甲戌初立黨里鄰三長定民戶籍民始皆愁苦豪彊者尤不願既而課調省費十餘倍上下安之 三月丙申柔然遣使者牟提如魏時敕勒叛柔然柔然伏名敦可汗自將討之追奔至西漠|{
	西漠者大漠之西偏也將即亮翻}
魏左僕射穆亮等請乘虛擊之中書監高閭曰秦漢之世海内一統故可遠征匈奴今南有吳寇何可捨之深入虜庭魏主曰兵者凶器聖人不得已而用之|{
	老子之言}
先帝屢出征伐者以有未賓之虜故也今朕承太平之業奈何無故動兵革乎厚禮其使者而歸之 夏四月辛酉朔魏始制五等公服甲子初以法服御輦祀南郊|{
	公服朝廷之服五等朱紫緋緑青法服衮冕以見郊廟之服}
癸酉魏主如靈泉池|{
	魏於方山之南起靈泉宮引如渾水為靈泉池東西一百步南北二百步}
戊寅還宫 湘州蠻反刺史呂安國有疾不能討丁亥以尚書左僕射柳世隆為湘州刺史討平之 六月辛酉魏主如方山 |{
	考異曰魏帝紀是日幸方山七月戊戌又云幸方山皆不言還宫盖闕文耳}
己卯魏文明太后賜皇子恂名大赦 秋七月戊戌魏主如方山 八月乙亥魏給尚書五等爵已上朱衣玉佩大小組綬|{
	組綬者組織以成綬鄭玄曰綬所以貫佩玉相承受者也漢制印綬先合單紡為一系四系為一扶五扶為一首五首成一文文采淳為一圭首多者系細少者系麤皆廣一尺六寸組則古翻絞音受}
九月辛卯魏作明堂辟雍 冬十一月魏議定民官

依戶給俸|{
	以所領民戶之多少為給俸之差也}
十二月柔然寇魏邊是歲魏改中書學曰國子學|{
	魏先置中書博士及中書學生今改曰國子學從晉制也}
分置州郡凡三十八州二十五在河南十三在河北|{
	河南二十五州青南青兖齊濟光豫洛徐東徐雍秦南秦梁益荆凉河沙時又置華陜夏岐班郢凡二十五河北十三州司并肆定相冀幽燕營平安時又置瀛汾凡十三蕭子顯曰雍凉秦沙涇華岐河西華寜陜洛荆郢北豫東荆南豫西兖東兖南徐東徐青齊濟光二十五州在河南相汾懷并東雍肆定瀛朔并冀幽平司等十三州在河北}


五年春正月丁亥朔魏主詔定樂章非雅者除之 戊子以豫章王嶷為大司馬竟陵王子良為司徒臨川王映衛將軍王儉中軍將軍王敬則竝加開府儀同三司子良啓記室范雲為郡上曰聞其常相賣弄朕不復窮法當宥之以遠|{
	復扶又翻}
子良曰不然雲動相規誨諫書具存遂取以奏凡百餘紙辭皆切直上歎息謂子良曰不謂雲能爾方使弼汝何宜出守|{
	守式又翻}
文惠太子嘗出東田觀穫|{
	時太子作東田於東宫之東緜恒華遠壯麗極目又齊紀太子立樓舘于鍾山下號曰東田}
顧謂衆賓曰刈此亦殊可觀衆皆曰唯唯|{
	唯于癸翻}
雲獨曰三時之務實為長勤|{
	三時之務謂春耕夏耘秋穫也}
伏願殿下知稼穡之艱難無徇一朝之宴逸 荒人桓天生自稱桓玄宗族與雍司二州蠻相扇動|{
	雍於用翻}
據南陽故城請兵於魏將入寇丁酉詔假丹楊尹蕭景先節總帥步騎直指義陽司州諸軍皆受節度|{
	帥讀曰率騎奇寄翻}
又假護軍將軍陳顯逹節帥征虜將軍戴僧静等水軍向宛葉|{
	宛於元翻葉式涉翻}
雍司諸軍皆受顯逹節度以討之 魏光祿大夫咸陽文公高允歷事五帝|{
	太武景穆文成獻文及高祖為五帝}
出入三省|{
	三省尚書省中書省祕書省也}
五十餘年未嘗有譴馮太后及魏主甚重之常命中黄門蘇興夀扶侍允仁恕簡静雖處貴重|{
	處昌呂翻}
情同寒素執書吟覽晝夜不去手誨人以善恂恂不倦|{
	楊中立曰恂恂一於誠也朱元晦曰恂恂信實之貌}
篤親念故無所遺棄顯祖平青徐悉徙其望族於代|{
	事見一百三十二卷宋明帝泰始五年}
其人多允之婚媾流離飢寒允傾家賑施|{
	賑之忍翻施式智翻}
咸得其所又隨其才行薦之於朝|{
	行下孟翻朝直遥翻}
議者多以初附間之|{
	間古莧翻}
允曰任賢使能何有新舊必若有用豈可以此抑之允體素無疾至是微有不適猶起居如常數日而卒年九十八贈侍中司空賻襚甚厚|{
	布帛曰賻衣被曰襚賻音附襚徐醉翻}
魏初以來存亡蒙賚皆莫及也 桓天生引魏兵萬餘人至沘陽|{
	漢沘陽縣屬南陽郡應劭曰沘水所出魏太和中置東荆州於沘陽故城宋白曰今唐州沘陽縣即州故城九域志沘陽縣在唐州東北七十五里}
陳顯逹遣戴僧静等與戰於深橋|{
	戴僧靜傳深橋距沘陽四十里沘音比}
大破之殺獲萬計天生退保沘陽僧静圍之不克而還|{
	還從宣翻又如字}
荒人胡丘生起兵懸瓠以應齊魏人擊破之丘生來奔天生又引魏兵寇舞隂舞隂戍主殷公愍拒擊破之殺其副張麒麟天生被創退走|{
	被皮義翻創初良翻}
三月丁未以陳顯逹為雍州刺史|{
	雍於用翻}
顯逹進據舞陽城 夏五月壬辰魏主如靈泉池癸巳魏南平王渾卒 甲午魏主還平城詔復七廟子孫及外戚緦麻服已上賦役無所與|{
	復方月翻七廟子孫自太祖已下緦麻三月服五服至緦麻而服盡與讀當曰預}
魏南部尚書公孫邃上谷公張儵帥衆與桓天生復寇舞隂殷公愍擊破之|{
	儵式竹翻帥讀曰率復扶又翻 考異曰齊書魏虜傳云偽安南將軍遼東公平南將軍上谷公又攻武隂魏書帝紀云詔南部尚書公孫文慶上谷公張伏于南討舞隂按公孫邃傳邃字文慶與内都幢將上谷公張儵討蕭賾舞隂戍盖伏于亦儵字也}
天生還竄荒中邃表之孫也|{
	公孫表事魏明元為將}
魏春夏大旱代地尤甚加以牛疫民餒死者多六月癸未詔内外之臣極言無隱齊州刺史韓麒麟上表曰古先哲王儲積九稔|{
	古者三年耕餘一年食九年耕餘三年食以三十年之通制國用則當有九年之蓄國無九年之蓄曰不足無六年之蓄曰急無三年之蓄曰國非其國也稔而廩翻}
逮於中代亦崇斯業入粟者與斬敵同爵力田者與孝悌均賞|{
	漢令民入粟拜爵又有孝悌力田之科}
今京師民庶不田者多遊食之口參分居二自承平日久豐穰積年競相矜夸遂成侈俗貴富之家童妾袨服|{
	袨黄練翻袨服美衣也}
工商之族僕隸玉食|{
	張晏曰玉食珍食也}
而農夫闕糟糠蠶婦乏短褐故令耕者日少|{
	少詩沼翻下同}
田有荒蕪穀帛罄於府庫寶貨盈於市里衣食匱於室麗服溢於路飢寒之本寔在於斯愚謂凡珍異之物皆宜禁斷|{
	斷丁管翻}
吉凶之禮備為格式勸課農桑嚴加賞罰數年之中必有盈贍往年校比戶貫|{
	毛晃曰貫鄉籍也}
租賦輕少臣所統齊州租粟纔可給俸略無入倉|{
	俸扶用翻}
雖於民為利而不可長久脫有戎役或遭天災恐供給之方無所取濟可減絹布增益穀租年豐多積歲儉出賑|{
	歲入約少為儉賑之忍翻下同}
所謂私民之穀寄積於官官有宿積則民無荒年矣|{
	宿積子智翻}
秋七月己丑詔有司開倉賑貸聽民出關就食|{
	魏都平城郊畿之外置關於要路以譏征}
遣使者造籍分遣去留所過給糧廪所至三長贍養之 柔然伏名敦可汗殘暴|{
	可從刋入聲汗音寒}
其臣侯醫垔石洛候數諫止之|{
	垔伊真翻數所角翻}
且勸其與魏和親伏名敦怒族誅之由是部衆離心八月柔然寇魏邊魏以尚書陸叡為都督擊柔然大破之叡麗之子也|{
	陸麗陸俟之子於乙渾之難死也}
初高車阿伏至羅有部落十餘萬役屬柔然伏名敦之侵魏也阿伏至羅諫不聽阿伏至羅怒與從弟窮奇帥部落西走至前部西北|{
	從才用翻帥讀曰率前部漢車師前王地也}
自立為王 |{
	考異曰魏書高車傳云在太和十一年蠕蠕在十六年今從高車傳按蠕蠕下當有傳字}
國人號曰候婁匐勒夏言天子也號窮奇曰候倍夏言太子也|{
	夏言謂中華之言夏戶雅翻}
二人甚親睦分部而立阿伏至羅居北窮奇居南伏名敦追擊之屢為阿伏至羅所敗乃引衆東徙|{
	史言柔然寖衰敗補邁翻}
九月辛未魏詔罷起部無益之作|{
	起部掌百工之事書曰百工起哉}


出宫人不執機杼者冬十月丁未又詔罷尚方錦繡綾羅之工四民欲造任之無禁|{
	四民士農工商也}
是時魏久無事府藏盈積詔盡出御府衣服珍寶太官雜器太僕乘具内庫弓矢刀鈐十分之八|{
	藏徂浪翻乘繩證翻鈐與鉗同其廉翻刃也唐有玉鈐衛}
外府衣物繒布絲纊|{
	繒慈陵翻帛也纊苦謗翻纊絮也}
非供國用者以其大半班賚百司下至工商皁隸逮于六鎮邊戍畿内鰥寡孤獨貧癃皆有差|{
	劉熙釋名曰無妻曰鰥憂悒不能寐目常鰥鰥然其字從魚魚目常不閉無夫曰寡寡倮也倮然單獨也無父曰孤孤顧也顧望無所瞻見也無子曰獨獨鹿也鹿鹿無所依也無財曰貧疲病曰癃}
魏袐書令高祐丞李彪奏請改國書編年為紀傳表志|{
	傳直戀翻}
魏主從之祐允之從祖弟也十二月詔彪與著作郎崔光改脩國書光道固之從孫也|{
	從才用翻宋明帝泰始五年崔道固降魏}
魏主問高祐曰何以止盜對曰昔宋均立德猛虎渡河卓茂行化蝗不入境|{
	宋均事見四十五卷漢明帝永平七年卓茂為密令教化大行漢平帝時天下大蝗獨不入密縣界}
况盜賊人也苟守宰得人治化有方止之易矣|{
	守式又翻治直吏翻易以豉翻}
祐又上疏言今之選舉不採識治之優劣專簡年勞之多少|{
	少詩沼翻}
斯非盡才之謂宜停此薄藝棄彼朽勞唯才是舉則官方斯穆|{
	方道也穆和也清也}
又勲舊之臣雖年勤可錄而才非撫民者可加之以爵賞不宜委之以方任所謂王者可私人以財不私人以官者也|{
	王者不私人以官前漢書佞幸傳贊之辭}
帝善之祐出為西兖州刺史鎮滑臺以郡國雖有學縣黨亦宜有之乃命縣立講學黨立小學

六年春正月乙未魏詔犯死刑者父母祖父母年老更無成人子孫旁無朞親者具狀以聞|{
	朞親為之服朞者}
初皇子右衛將軍子響出繼豫章王嶷|{
	嶷魚力翻}
嶷後有子表留為世子子響每入朝|{
	朝直遥翻}
以車服異於諸王每拳擊車壁上聞之詔車服與皇子同於是有司奏子響宜還本三月己亥立子響為巴東王 角城戍將張蒲因大霧乘船入清中採樵|{
	清中清水中也將即亮翻}
濳納魏兵戍主皇甫仲賢覺之帥衆拒戰於門中僅能却之魏步騎三千餘人已至塹外|{
	帥讀曰率騎奇寄翻塹七艶翻}
淮隂軍主王僧慶等引兵救之魏人乃退 夏四月桓天生復引魏兵出據隔城|{
	復扶又翻}
詔游擊將軍下邳曹虎督諸軍討之輔國將軍朱公恩將兵蹹伏|{
	將即亮翻下同蹹與踏同}
遇天生遊軍與戰破之遂進圍隔城天生引魏兵步騎萬餘人來戰虎奮擊大破之俘斬二千餘人明日拔隔城斬其襄城太守帛烏祝復俘斬二千餘人天生棄平氏城走|{
	平氏漢縣屬南陽郡晉宋屬義陽郡縣西南有桐柏山淮源所出也五代志淮安郡平氏縣魏置漢廣郡至我朝開寶五年省平氏縣為鎮入唐州泌陽縣}
陳顯逹侵魏甲寅魏遣豫州刺史拓跋斤將兵拒之 甲子魏大赦 乙丑魏主如靈泉池丁卯如方山己巳還宫 魏築城於醴陽|{
	醴陽蓋在醴水之北水經注醴水出桐柏山與淮同源而别流西注逕平氏縣東北又西流注于沘水}
陳顯逹攻拔之進攻沘陽城中將士皆欲出戰鎮將韋珍曰|{
	魏樂陵鎮將鎮沘陽將即亮翻}
彼初至氣銳未可與爭且共堅守待其力攻疲弊然後擊之乃憑城拒戰旬有二日珍夜開門掩擊顯逹還|{
	還從宣翻又如字}
五月甲午以宕昌王梁彌承為河涼二州刺史|{
	宕徒浪翻}
秋七月己丑魏主如靈泉池遂如方山己亥還宫

九月壬寅上如琅邪城講武 |{
	蕭子顯曰南琅邪郡本治金城永明乃徙治白下沈約曰晉亂琅邪國人隨元帝過江者千餘戶太興三年立懷德縣丹陽雖有琅邪郡而無其地成帝成康}


|{
	元年桓温領郡鎮江乘之蒲洲金城上求割丹陽之江乘縣境立郡}
癸卯魏淮南靖王佗卒|{
	佗徒河翻}
魏主方享宗廟始薦聞之為廢祭臨視哀慟|{
	為于偽翻}
冬十月庚申立冬初臨太極殿讀時令|{
	漢儀太史每歲上其年歷先立春立夏大暑立秋立冬常讀五時令皇帝所服各隨五時之色帝升御座尚書令以下就席位尚書三公郎以令置按上奏以入就席伏讀訖賜酒一巵}
閏月辛酉以尚書僕射王奐為領軍將軍 辛未魏主如靈泉池癸酉還宫十二月柔然伊吾戍主高羔子帥衆三千以城附魏|{
	帥讀曰率}
上以中外穀帛至賤用尚書右丞江夏李珪之議|{
	夏戶雅翻}
出上庫錢五千萬及出諸州錢皆令糴買 西陵戍主杜元懿建言吳興無秋會稽豐登|{
	會工外翻}
商旅往來倍多常歲西陵牛埭稅官格日三千五百如臣所見日可增倍|{
	西陵在今越州蕭山縣西十二里西興渡是也吳越王錢鏐以西陵非吉語改曰西興牛埭即今西興堰用牛挽船因曰牛埭埭徒耐翻}
并浦陽南北津柳浦四埭|{
	浦陽江南津埭則今之梁湖堰是也北津埭則今之曹俄堰是也柳浦埭則今杭州江干浙江亭北跨浦橋埭是也}
乞為官領攝一年|{
	為于偽翻}
格外可長四百許萬|{
	長丁丈翻今知兩翻增也又音直亮翻多也}
西陵戌前檢稅無妨戌事餘三埭自舉腹心上以其事下會稽|{
	下戶嫁翻}
會稽行事吳郡顧憲之議以為始立牛埭之意非苟逼蹴以取稅也|{
	蹴子六翻}
乃以風濤迅險濟急利物耳後之監領者不逹其本各務已功|{
	監古銜翻}
或禁遏佗道或空稅江行案吳興頻歲失稔今兹尤甚去之從豐良由飢棘|{
	去之當作去乏棘急也}
埭司責稅依格弗降舊格新減尚未議登格外加倍將以何術皇慈恤隱振廪蠲調|{
	左傳楚大饑振廩同食杜預注曰振發也廪倉也調徒釣翻}
而元懿幸災榷利重增困瘼|{
	榷古岳翻瘼病也重直用翻}
人而不仁古今共疾若事不副言懼貽譴詰|{
	譴去戰翻詰去吉翻}
必百方侵苦為公賈怨元懿禀性苛刻已彰往效任以物土譬以狼將羊其所欲舉腹心亦當虎而冠耳|{
	為于偽翻賈音古狼將羊虎而冠皆漢書語以狼將羊則羊必為狼所噬食虎而冠者言其人惡戾如虎著冠}
書云與其有聚斂之臣寜有盜臣|{
	記大學記孟獻子之言斂力贍翻}
此言盜公為損蓋微斂民所害乃大也愚又以便宜者蓋謂便於公宜於民也竊見頃之言便宜者非能於民力之外用天分地|{
	用天之道分地之利此孝經第六章之言}
率皆即日不宜於民方來不便於公名與實反有乖政體凡如此等誠宜深察上納之而止 魏主訪羣臣以安民之術祕書丞李彪上封事以為豪貴之家奢僭過度第宅車服宜為之等制又國之興亡在冢嗣之善惡冢嗣之善惡在教諭之得失|{
	冢大也周禮疏曰冢大之上也冢知隴翻}
高宗文成皇帝嘗謂羣臣曰朕始學之日年尚幼冲情未能專既臨萬機不遑温習今日思之豈唯予咎抑亦師傅之不勤尚書李訢免冠謝|{
	訢許斤翻}
此近事之可鍳者也臣謂宜凖古立師傅之官以訓導太子|{
	蓋此時恂之失德已著故彪有是言}
又漢置常平倉以救匱乏|{
	見二十七卷漢宣帝五鳳四年}
去歲京師不稔移民就豐既廢營生困而後逹又於國體實有虚損曷若豫儲倉粟安而給之豈不愈於驅督老弱餬口千里之外哉|{
	餬音胡說文曰寄食鬻也余據正考父鼎銘饘於是粥於是以餬余口則餬者食饘粥之義許慎所謂寄食者蓋因左傳餬口於四方以為說今此當依許義}
宜析州郡常調九分之二京師度支歲用之餘|{
	調徒弔翻度徒洛翻}
各立官司年豐糴粟積之於倉儉則加私之二糶之於人|{
	糶他弔翻}
如此民必力田以取官絹積財以取官粟年登則常積歲凶則直給數年之中穀積而人足雖災不為害矣又宜於河表七州人中擢其門才引令赴闕依中州官比隨能序之|{
	河表七州秦雍岐華陜河涼也以下文懷江漢歸有道之情證之則七州當謂荆兖豫洛青徐齊也河表直謂大河之外門才者因其世家叙其才用中州謂代都東至海南距大河諸州比毗至翻比例也}
一可以廣聖朝均新舊之義|{
	朝直遥翻}
二可以懷江漢歸有道之情又父子兄弟異體同氣罪不相及乃君上之厚恩至於憂懼相連固自然之恒理也|{
	恒戶登翻}
無情之人父兄繫獄子弟無慘惕之容子弟逃刑父兄無愧恧之色|{
	恧女六翻}
宴安榮位遊從自若車馬衣冠不變華飾骨肉之恩豈當然也臣愚以為父兄有犯宜令子弟素服肉袒詣闕請罪子弟有坐宜令父兄露板引咎乞解所司若職任必要不宜許者慰勉留之如此足以敦厲凡薄使人知所恥矣又朝臣遭親喪者假滿赴職|{
	朝直遥翻假古訝翻時魏不聽朝臣終喪給假而已}
衣錦乘軒從郊廟之祀|{
	衣於既翻}
鳴玉垂緌同慶賜之燕|{
	緌如佳翻}
傷人子之道虧天地之經愚謂凡遭大父母父母喪者皆聽終服若無其人職業有曠者則優旨慰諭起令視事但綜司出納敷奏而已國之吉慶一令無預其軍旅之警墨縗從役|{
	春秋時晉襄公居文公之喪墨縗絰以敗秦師於殽自是之後以墨縗從戎縗倉回翻}
雖愆於禮事所宜行也魏主皆從之由是公私豐贍雖時有水旱而民不困窮 魏遣兵擊百濟為百濟所敗|{
	陳夀曰三韓凡七十八國百濟其一也據李延夀史其先以百家濟海後浸強盛以立國故曰百濟晉世句麗畧有遼東百濟亦據有遼西晉平二郡地}


七年春正月辛亥上祀南郊大赦 魏主祀南郊始備大駕|{
	漢儀大駕公卿奉引太僕御大將軍驂乘屬車八十一乘備千乘萬騎祀天郊甘泉乃備之謂之甘泉鹵簿東都惟大行備大駕晉中朝大駕鹵簿先象車鼓吹一部十三人中道次静室令駕一中道式道候二人駕一分左右次洛陽尉二人騎分左右次洛陽亭長九人赤車駕一分三道各次正二人引次洛陽令駕一中道次河南中部掾中道河橋掾在左功曹史在右並駕一次河南尹駕駟戟吏六人次河南主簿駕一中道次司隸部河南從事中道都部從事居左别駕從事居右並駕一次司隸校尉駕三戟吏八人次司隸主簿駕一中道次司隸主記駕一中道次廷尉明法掾中道五官掾居左功曹史居右並駕一次廷尉卿駕駟戟吏六人次廷尉主簿主記並駕一在左太僕引從如廷尉在中宗正引從如廷尉在右次太常駕駟中道戟吏六人大外部掾居左五官掾功曹史居右並駕一次光祿引從中道太常主簿主記居左衛尉引從居右並駕一次太尉外督令史駕一中道次西東賊倉戶等曹並駕一引從次太尉駕駟中道太尉主簿舍人各一人祭酒二人並駕一在左右次司徒引從駕駟中道次司空引從駕駟中道三公騎令史戟各八人鼓吹各一部七人次中護軍中道駕駟鹵簿左右各三行戟楯在外弓矢在内鼓吹一部七人次步兵校尉在左長水校尉在右並駕一各鹵簿左右二行戟楯在外刀楯在内鼓吹各一部七人次財校尉在左地軍校尉在右並駕一鹵簿各左右二行次驍騎將軍在左游擊將軍在右並駕一皆鹵簿左右引各二行戟楯刀楯鼓吹眂步兵長水騎隊五在左五在右隊各五十四命中督二人分領左右各有戟吏二人麾幢揭鼓在隊前次左將軍在左前將軍在右並駕一鹵簿左右引戟楯刀楯鼓吹亦如之次黄門麾騎中道次黄門前部鼓吹左右各一部十三人駕駟八校尉佐仗左右各四行外大戟楯次九尺楯次弓矢次弩並熊渠佽飛督領之次司南車駕駟中道護駕御史騎夾左右次謁者僕射駕駟中道次御史中承駕一中道次虎賁中郎將騎中道次九斿車中道武剛車夾左右並駕駟次雲罕車駕駟中道次闟戟車駕駟中道長戟邪偃向後次皮軒車駕駟中道次鸞旗車中道建華車分左右並駕駟次護駕尚書郎三人都官郎中道駕部在左中兵在右並騎又有護駕尚書一人督攝前後無常次相風中道次司馬督在前中道左右各司馬史三人引仗左右各六行外大戟楯二行次九尺楯次刀楯次弓矢次弩次五時車左右有遮騎次典兵中郎中道督攝前後無常左殿中御史右殿中監並騎次高蓋中道左畢右次御史中道左右節郎各四人次華蓋中道次殿中司馬中道殿中都尉在左殿中校尉在右左右各四行細楯一行在弩内又殿中司馬一行殿中都尉一行殿中校尉一行次鼓中道次金根車駕六馬中道太僕卿御大將軍參乘左右又各增三行為九行司馬史九人戟楯二行九尺楯一行刀楯一行由基一行細弩一行跡禽一行樵斧一行力人刀楯一行連細楯殿中司馬殿中都尉殿中校尉為左右各十二行金根連建青旗十二左將軍騎在左右將軍騎在右殿中將軍鑿斧夾車車後衣書主職步從六行合左右三十二行次曲華蓋中道侍中散騎常侍黄門侍郎並騎分左右次黄鉞車駕一在左御麾騎在右次相風中道次中書監騎左祕書監騎右次殿中御史騎左殿中監騎右次五牛旗赤青在左黄在中白黑在右次夫輦中道夫輦謂當作大輦太官令丞左太醫令丞右次金根車駕駟不建次青立車次青安車次赤立車次赤安車次黄立車次黄安車次白立車次白安車次黑立車次黑安車合十乘並駕駟建旗十二如車色立車正竪旗安車邪拖之次蹋猪車駕駟中道無旗次耕根車駕駟中道赤旗十二熊渠督左佽飛督右次御軺車次御四望車次御衣車次御書車次御藥車並駕牛中道次尚書令在左尚書僕射在右又尚書郎六人分次左右並駕一又治書侍御史二人分左右又侍御史二人分次左右又蘭臺令史分次左右並騎次豹尾車駕一自豹尾車後而鹵簿盡矣但以神弩二十張夾道至後部鼓吹其五張神弩置一將左右各二將次輕車二十乘左右分駕次流蘇馬六十四次金鉞車駕三中道左右護駕尚書郎并令史並騎各一人次金鉦車駕三中道左右護駕侍御史并令史並騎各一人次黄門後部鼓吹左右各三十人次戟鼓車駕牛二乘分左右次左大鴻臚外部彖右五官掾功曹史並駕次大鴻臚駕駟鉞吏六人次大司農引從中道左大鴻臚主簿主記右少府引從次三卿並騎吏四人鈴下四人執馬鞭辟車六人執方扇羽林十八朱衣次領軍將軍中道鹵簿左右各一行九尺楯在外弓矢在内鼓吹如護軍次後軍將軍在左後將軍在右各鹵簿鼓吹如左軍前軍次越騎校尉左屯騎校尉右各鹵簿鼓吹如步兵射聲次領護驍游軍校尉皆騎吏四人乘馬夾道都督兵曹各一人乘馬在騎將軍四人騎校鞉角金鼓鈴下信幡軍校並駕一功曹史主簿並騎從繖扇幢麾各一騎鼓吹一部七騎次領護軍加大車斧五官掾騎從次騎十隊隊各五十四將一人持幢一人鞉一人並騎在前督戰伯長各一人並騎在後羽林騎督幽州突騎督分領之郎簿十隊隊各五十人絳袍將一人騎鞉各一人在前督戰伯長一人步在後騎皆持矟次大戟一隊九尺楯一隊弓一隊弩一隊隊各五十人黑袴褶將一人騎校鞉角各一人步在前督戰伯長各一人步在後金顔督將并領之魏之大駕蓋參取漢晉之制而官名鹵簿則微有不同者}
壬戌臨川獻王映卒 初上為鎮西長史主簿王晏以傾諂為上所親|{
	宋蒼梧王元徽四年帝為鎮西長史行郢州事收晏為主簿}
自是常在上府上為太子晏為中庶子上之得罪於太祖也|{
	事見上卷元年}
晏稱疾自踈及即位為丹陽尹意任如舊朝夕一見|{
	見賢遍翻}
議論朝事自豫章王嶷及王儉皆降意接之|{
	論朝直遥翻嶷魚力翻}
二月壬寅出為江州刺史晏不願外出復留為吏部尚書|{
	復扶又翻}
三月甲寅立皇子子岳為臨賀王子峻為廣漢王子琳為宣城王子珉為義安王 夏四月丁丑魏主詔曰升樓散物以賚百姓至使人馬騰踐多有傷毁今可斷之|{
	斷讀如短}
以本所費之物賜老疾貧獨者 丁亥魏主如靈泉池遂如方山己丑還宫 上優禮南昌文憲公王儉詔三日一還朝|{
	還當作造音七到翻朝直遥翻}
尚書令史出外諮事上猶以往來煩數復詔儉還尚書下省|{
	數所角翻復扶又翻}
月聽十日出外儉固求解選詔改中書監參掌選事|{
	選須絹翻}
五月乙巳儉卒王晏既領選權行臺閣與儉頗不平禮官欲依王導諡儉為文獻晏啟上曰導乃得此諡但宋氏以來不加異姓出謂親人曰平頭憲事已行矣|{
	平頭謂憲字也諡神至翻}
徐湛之之死也|{
	湛之死見一百三十七卷宋文帝元嘉三十年}
其孫孝嗣在孕得免|{
	孕以證翻}
八歲襲爵枝江縣公尚宋康樂公主|{
	樂音洛}
及上即位孝嗣為御史中丞風儀端簡玉儉謂人曰徐孝嗣將來必為宰相上嘗問儉誰可繼卿者儉曰臣東都之日其在徐孝嗣乎|{
	謂周公既定洛請明農也周都豐鎬以洛為東都}
儉卒孝嗣時為吳興太守徵為五兵尚書 庚戌魏主祭方澤|{
	方澤者為方丘於澤中以祭地祇}
上欲用領軍王奐為尚書令以問王晏晏與奐不相能對曰柳世隆有勲望恐不宜在奐後甲子以尚書左僕射柳世隆為尚書令王奐為左僕射 六月丁亥上如琅邪城 魏懷朔鎮將汝隂靈王天賜|{
	魏置懷朔鎮於漢五原郡界是後六鎮叛改為朔州而不能有舊鎮之地杜佑曰魏都平城於馬邑郡北三百餘里置懷朔鎮及遷洛後置朔州將即亮翻下同}
長安鎮都大將雍州刺史南安惠王楨|{
	雍於用翻}
皆坐當死馮太后及魏主臨皇信堂|{
	水經注曰太極殿南對承賢門門南即皇信堂也魏書帝紀太和七年十月皇信堂成十六年以安昌殿為内寢皇信堂為中寢}
引見王公|{
	見賢遍翻}
太后令曰卿等以為當存親以毁令邪當滅親以明法邪羣臣皆言二王景穆皇帝之子|{
	景穆皇帝世祖之子薨諡曰景穆皇帝未即尊位也二王於高祖為叔祖}
宜蒙矜恕太后不應魏主乃下詔稱二王所犯難恕而太皇太后追惟高宗孔懷之恩|{
	二王於文成帝為兄弟詩曰兄弟孔懷惟思也}
且南安王事母孝謹聞於中外|{
	聞音問}
竝特免死削奪官爵禁錮終身初魏朝聞楨貪暴遣中散閭文祖詣長安察之|{
	中散中散大夫也散悉亶翻}
文祖受楨賂為之隱|{
	為于偽翻}
事覺文祖亦扺罪馮太后謂羣臣曰文祖前自謂廉今竟犯法以此言之人心信不可知魏主曰古有待放之臣|{
	春秋公羊傳晉放其大夫胥甲父于衛放之者何猶曰無去是云爾然則何言爾近正也此其為近正奈何古者大夫已去三年待放君放之非也大夫待放正也}
卿等自審不勝貪心者聽辭位歸第宰官中散慕容契進曰|{
	契蓋以宰官帶中散大夫也}
小人之心無常而帝王之法有常以無常之心奉有常之法非所克堪乞從退黜魏主曰契知心不可常則知貪之可惡矣|{
	惡烏路翻}
何必求退遷宰官令契白曜之弟子也|{
	慕容白曜有平齊之功}
秋七月丙寅魏主如靈泉池 魏主使羣臣議久與齊絶今欲通使何如|{
	使疏吏翻下同}
尚書游明根曰朝廷不遣使者又築醴陽深入彼境皆直在蕭賾今復遣使不亦可乎|{
	復扶又翻}
魏主從之八月乙亥遣兼員外散騎常侍邢產等來聘 九月魏出宫人以賜北鎮人貧無妻者|{
	北鎮六鎮也一曰懷朔鎮直平城北}
冬十一月己未魏安豐匡王猛卒 十二月丙子魏河東王苟頹卒 平南參軍顔幼明等聘於魏 魏以尚書令尉元為司徒左僕射穆亮為司空 豫章王嶷自以地位隆重深懷退素是歲啓求還第上令其世子子廉代鎮東府 太子詹事張緒領揚州中正長沙王晃屬用吳興聞人邕為州議曹|{
	屬之欲翻州議曹自漢以來率儒士為之}
緒不許晃使書佐固請緒正色曰此是身家州鄉殿下何得見逼|{
	自魏晉以來中正率用本州人望為之}
侍中江斆為都官尚書|{
	斆音効}
中書舍人紀僧真得幸於上容表有士風請於上曰臣出自本縣武吏邀逢聖時|{
	邀南史江斆傳作憿說文曰幸也集韻憿僥徼通音堅堯翻}
階榮至此為兒昏得荀昭光女即時無復所須|{
	為于偽翻復扶又翻}
唯就陛下乞作士大夫上曰此由江斆謝瀹我不得措意可自詣之僧真承旨詣斆登榻坐定斆顧命左右曰移吾牀遠客|{
	遠于願翻}
僧真喪氣而退告上曰士大夫故非天子所命斆湛之孫瀹朏之弟也|{
	二家以名義自持至於甄别流品雖萬乘之主不可得而奪喪息浪翻朏敷尾翻}
柔然别帥叱呂勤帥衆降魏|{
	别帥所類翻勤帥讀曰率降戶江翻}


資治通鑑卷一百三十六
