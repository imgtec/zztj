






























































資治通鑑卷一百四十三 宋 司馬光 撰

胡三省 音註

齊紀九|{
	上章執徐一年}


東昏侯下

永元二年春正月元會帝食後方出朝賀裁竟即還殿西序寢|{
	孔安國曰東西廂謂之序朝直遥翻下同}
自己至申百僚陪位皆僵仆飢甚|{
	僵居良翻}
比起就會|{
	比及也禮記檀弓孟獻子比御而不入陸德明經典釋文曰比必利翻下比及同以此知比及之比皆音必利翻比近之比毗至翻兩音故自不同也}
怱遽而罷乙巳魏大赦改元景明 豫州刺史裴叔業聞帝數

誅大臣|{
	數所角翻下數遣同}
心不自安登壽陽城北望肥水謂部下曰卿等欲富貴乎我能辦之及除南兖州|{
	事見上卷上年}
意不樂内徙|{
	樂音洛}
會陳顯逹反|{
	亦見上卷上年}
叔業遣司馬遼東李元護將兵救建康|{
	將即亮翻}
實持兩端顯逹敗而還|{
	還從宣翻又如字}
朝廷疑叔業有異志叔業亦遣使參察建康消息|{
	使疏吏翻}
衆論益疑之叔業兄子植颺粲皆為直閤在殿中懼棄母奔壽陽說叔業以朝廷必相掩襲宜早為計|{
	颺余章翻說輸芮翻下等說同}
徐世等以叔業在邊|{
	與標同}
急則引魏自助力未能制白帝遣叔業宗人中書舍人長穆宣旨許停本任|{
	宗人同宗之人也}
叔業猶憂畏而植等說之不已叔業遣親人馬文範至襄陽|{
	親人所親信者}
問蕭衍以自全之計曰天下大勢可知恐無復自存之理|{
	復扶又翻下可復復奔同}
不若回面向北不失作河南公|{
	言若降魏不失爵賞也}
衍報曰羣小用事豈能及遠計慮回惑自無所成唯應送家還都以安慰之|{
	蕭衍密呼諸弟而令裴叔業送家還都此亦華言耳}
若意外相逼當勒馬步二萬直出横江以斷其後|{
	自夀陽南至歷陽出横江斷丁管翻}
則天下之事一舉可定若欲北向彼必遣人相代以河北一州相處|{
	處昌呂翻}
河南公寧可復得邪如此則南歸之望絶矣|{
	裴叔業之問蕭衍之報雖二人者所志有大小而齊之邊鎮皆有異心矣帝誰與立哉}
叔業沈疑未决|{
	沈持林翻沈疑沈吟疑慮也}
乃遣其子芬之入建康為質|{
	質音致}
亦遣信詣魏豫州刺史薛真度|{
	魏豫州治懸瓠城領汝南新蔡弋陽等郡}
問以入魏可不之宜|{
	不讀曰否}
真度勸其早降|{
	降戶江翻下同}
曰若事廹而來則功微賞薄矣數遣密信往來相應和|{
	和戶卧翻}
建康人傳叔業叛者不已芬之懼復奔壽陽叔業遂遣芬之及兄女壻杜陵韋伯昕奉表降魏|{
	昕許斤翻}
丁未魏遣驃騎大將軍彭城王勰車騎將軍王肅帥步騎十萬赴之|{
	驃匹妙翻騎奇寄翻勰音協帥讀曰率}
以叔業為使持節都督豫雍等五州諸軍事征南將軍豫州刺史封蘭陵郡公|{
	使疏吏翻雍於用翻}
庚午下詔討叔業二月丙戌以衛尉蕭懿為豫州刺史戊戌魏以彭城王勰為司徒領揚州刺史鎮壽陽|{
	壽陽自東漢以來為揚州治所宋始為豫州治所今復其舊勰音協}
魏人遣大將軍李醜楊大眼將二千騎入壽陽又遣奚康生將羽林一千馳赴之大眼難當之孫也|{
	楊難當氐王也宋元嘉中據仇池眼下二將字皆息亮翻}
魏兵未渡淮己亥裴叔業病卒僚佐多欲推司馬李元護監州一二日謀不定|{
	卒子恤翻監工銜翻}
前建安戌主安定席法友等|{
	北史曰魏正光中羣蠻出山居邊城建安者八九千戶邊城郡治期思則建安戍亦當相近隋改期思縣為殷城縣取縣東古殷城為名至我宋建隆元年改殷城為商城避宣祖諱也後省為鎮入光州固始縣}
以元護非其郷曲恐有異志共推裴植監州|{
	裴叔業本河東人席法友安定人不同州部盖並僑居襄陽遂為郷曲}
袐叔業喪問教命處分皆出於植|{
	處昌呂翻分扶問翻}
奚康生至植乃開門納魏兵城庫管籥悉付康生康生集城内耆舊宣詔撫賚之魏以植為兖州刺史李元護為齊州刺史席法友為豫州刺史軍主京兆王世弼為南徐州刺史 巴西民雍道晞聚衆萬餘逼郡城|{
	巴西郡治閬中縣今之閬州即其地也雍於用翻}
巴西太守魯休烈嬰城自守三月劉季連遣中兵參軍李奉伯帥衆五千救之|{
	帥讀曰率}
與郡兵合撃道晞斬之奉伯欲進討郡東餘賊涪令李膺止之曰卒惰將驕乘勝履險非完策也|{
	完全也言非全勝之策涪音浮將即亮翻下同}
不如少緩更思後計|{
	少詩沼翻}
奉伯不從悉衆入山大敗而還|{
	還從宣翻又知字}
乙卯遣平西將軍崔慧景將水軍討壽陽帝屏除出琅邪城送之|{
	蕭子顯曰琅邪太守本治江乘蒲州上之金城永明徙治白下屏必鄙翻}
帝戎服坐樓上召慧景單騎進圍内|{
	圍内即屏除長圍之内也騎奇寄翻}
無一人自隨者裁交數言拜辭而去慧景既得出甚喜豫州刺史蕭懿將步軍三萬屯小峴|{
	峴戶典翻}
交州刺史李叔獻屯合肥|{
	武帝永明三年李叔獻自交州入朝至今猶帶交州刺史盖以其阻險不庭逼以兵威而後至廢棄不用也}
懿遣裨將胡松李居士帥衆萬餘屯死虎|{
	杜佑通典曰死虎地名在壽州壽春縣東四十餘里以此證之足知宋明帝泰始三年劉勔破劉順於宛唐宛唐即死虎字之誤也}
驃騎司馬陳伯之將水軍泝淮而上|{
	上時掌翻}
以逼壽陽軍于硤石壽陽士民多謀應齊者魏奚康生防禦内外閉城一月援軍乃至丙申彭城王勰王肅擊松伯之等大破之進攻合肥生擒叔獻統軍宇文福言於勰曰建安淮南重鎮彼此要衝|{
	魏兵南來齊兵北向建安皆為要衝之地故曰彼此}
得之則義陽可圖不得則壽陽難保|{
	魏得建安則西南可圖義陽齊司州治義陽若增建安之兵北斷魏援東臨壽陽則壽陽難保}
勰然之使福攻建安建安戌主胡景略面縛出降|{
	降戶江翻}
己亥魏皇弟恌卒|{
	恌他彫翻}
崔慧景之建康也其子覺為直閤將軍密與之約|{
	約為變也}
慧景至廣陵覺走從之慧景過廣陵數十里召會諸軍主曰吾荷三帝厚恩|{
	三帝高帝武帝明帝也荷下可翻下人荷同}
當顧託之重|{
	明帝遺詔慧景與劉悛蕭惠休同任心膂}
幼主昏狂朝廷壞亂危而不扶責在今日欲與諸君共建大功以安社禝何如衆皆響應於是還軍向廣陵司馬崔恭祖守廣陵城|{
	崔恭祖為慧景平西司馬}
開門納之帝聞變壬子假右衛將軍左興盛節都督建康水陸諸軍以討之慧景停廣陵二日即收衆濟江初南徐兖二州刺史江夏王寶玄娶徐孝嗣女為妃孝嗣誅|{
	誅事見上卷上年}
詔令離昏寶玄恨望慧景遣使奉寶玄為主寶玄斬其使因將吏守城|{
	使疏吏翻將即亮翻下同}
帝遣馬軍主戚平外監黄林夫助鎮京口|{
	戚姓也姓譜衛大夫食邑于戚因以為姓漢有戚夫人又有臨轅侯戚鰓助鎮者助寶玄守}
慧景將渡江寶玄密與相應殺司馬孔矜典籖呂承緒及平林夫開門納慧景使長史沈佚之諮議柳憕分部軍衆|{
	憕署陵翻}
寶玄乘八掆輿|{
	掆古郎翻又居浪翻掆舉也八掆輿盖八人舉之即今之平肩輿輿不帷不盖蕭子顯曰輿車形如軺車下施八掆人舉之字林曰捎掆舁也}
手執絳麾隨慧景向建康臺遣驍騎將軍張佛護直閤將軍徐元稱等六將據竹里為數城以拒之|{
	驍堅堯翻騎奇寄翻}
寶玄遣信謂佛護曰身自還朝君何意苦相斷遏|{
	朝直遥翻斷音短下所斷同}
佛護對曰小人荷國重恩使於此創立小戍殿下還朝但自直過豈敢斷遏遂射慧景軍|{
	射而亦翻}
因合戰崔覺崔恭祖將前鋒皆荒傖善戰又輕行不㸑食|{
	傖助庚翻㸑即㸑字取亂翻}
以數舫緣江載酒食為軍糧|{
	舫甫妄翻下同}
每見臺軍城中煙火起輒盡力攻之臺軍不復得食|{
	復扶又翻下乃復帝復同}
以此飢困元稱等議欲降|{
	降戶江翻下同}
佛護不可恭祖等進攻城拔之斬佛護徐元稱降餘四軍主皆死乙卯遣中領軍王瑩都督衆軍據湖頭築壘上帶蔣山西巖實甲數萬瑩誕之從曾孫也|{
	王誕見寵信於司馬元顯及宋武帝從才用翻}
慧景至查硎|{
	查鉏加翻硎戶經翻}
竹塘人萬副兒|{
	萬副兒善射獵能捕虜來投慧景}
說慧景曰|{
	說輸芮翻}
今平路皆為臺軍所斷不可議進唯宜從蔣山龍尾上出其不意耳|{
	築道陂陁以上蔣山若龍尾之垂地因曰龍尾上時掌翻}
慧景從之分遣千餘人魚貫緣山自西巖夜下鼓叫臨城中|{
	城中即湖頭所築壘中也鼓叫者既擊鼓又叫呼也柳元景曰鼓繁氣易衰叫數力易竭鼓叫即鼓譟也}
臺軍驚恐即時奔散帝又遣右衛將軍左興盛帥臺内三萬人拒慧景於北籬門|{
	帥讀曰率下同 考異曰紀云王屯北籬門傳云左興盛今從傳}
興盛望風退走甲子慧景入樂游苑|{
	樂遊苑在玄武湖南樂音洛}
崔恭祖帥輕騎十餘突入北掖門乃復出|{
	掖音亦}
宫門皆閉慧景引衆圍之於是東府石頭白下新亭諸城皆潰左興盛走不得入宫逃淮渚荻舫中|{
	淮渚秦淮渚也}
慧景擒殺之宫中遣兵出盪不克|{
	盪度朗翻又他浪翻}
慧景燒蘭臺府署為戰場|{
	蘭臺御史臺也}
守御尉蕭暢屯南掖門處分城内|{
	處昌呂翻分扶問翻}
隨方應拒衆心稍安慧景稱宣德太后令廢帝為吴王|{
	文惠太子妃王氏欎林之立尊為皇太后海陵之廢出居鄱陽王故第號宣德宫稱宣德皇太后}
陳顯逹之反也帝復召諸王入宫巴陵王昭胄懲永泰之難|{
	明帝永泰元年王敬則反帝召諸王入宫欲殺之而中止事見一百四十一卷陳顯逹反帝復召之故昭胄懼禍而逃難乃旦翻}
與弟永新侯昭頴詐為沙門逃於江西|{
	江西横江以西之地宋白曰永新縣本漢廬陵縣地吴寶鼎中立永新縣屬安成郡}
昭胄子良之子也|{
	竟陵王子良武帝次子}
及慧景舉兵昭胄兄弟出赴之慧景意更向昭胄|{
	寶玄明帝之子昭胄武帝之孫武帝高帝之大宗故慧景意向之}
猶豫未知所立竹里之捷崔覺與崔恭祖爭功慧景不能决恭祖勸慧景以火箭燒北掖樓慧景以大事垂定後若更造費用功多不從|{
	言費功力為多也}
慧景性好談義兼解佛理|{
	好呼到翻義亦理也佛理諸有皆空之說解曉也音戶買翻}
頓法輪寺對客高談|{
	客謂何點}
恭祖深懷怨望時豫州刺史蕭懿將兵在小峴|{
	懿將兵討壽陽屯小峴將即亮翻峴所典翻}
帝遣密使告之懿方食投箸而起|{
	使疏吏翻箸除據翻}
帥軍主胡松李居士等數千人|{
	帥讀曰率}
自採石濟江頓越城舉火城中鼓叫稱慶|{
	城中臺城中也以援兵至而喜}
恭祖先勸慧景遣二千人斷西岸兵令不得度|{
	斷音短西岸兵謂蕭懿兵入援自江西來也}
慧景以城旦夕降外救自然應散不從|{
	降戶江翻}
至是恭祖請擊懿軍又不許獨遣崔覺將精手數千人渡南岸|{
	精手軍中事藝高強者南岸秦淮南岸也}
懿軍昧旦進戰數合|{
	昧旦天微明之時}
士皆致死覺大敗赴淮死者二千餘人覺單馬退開桁阻淮|{
	開朱雀桁以斷懿兵阻秦淮水為固}
恭祖掠得東宫女伎覺逼奪之恭祖積忿恨其夜與慧景驍將劉靈運詣城降|{
	崔覺以是日敗恭祖等以其夜降伎渠綺翻驍堅堯翻}
衆心離壞夏四月癸酉慧景將腹心數人潜去欲北渡江城北諸軍不知猶為拒戰|{
	為于偽翻為慧景戰也}
城中出盪殺數百人懿軍渡北岸|{
	秦淮北岸即臺城}
慧景餘衆皆走慧景圍城凡十二日而敗從者於道稍散單騎至蠏浦|{
	從才用翻蠏戶買翻}
為漁人所斬 |{
	考異曰齊本紀四月丁未以張冲為南兖州刺史崔慧景於廣陵起兵襲京師壬子左興盛督衆軍寶玄以京口納慧景乙卯王瑩屯北籬門壬戌慧景至瑩等敗甲子慧景入京師蕭懿入援癸酉慧景棄衆走死慧景傳四月至廣陵回軍十二日攻䧟竹里按長歷是歲三月辛丑朔四月庚午朔丁未三月七日壬子十二日乙卯十五日壬戌二十二日甲子二十四日四月皆無也盖四月當作三月至癸酉乃四月四日耳南史云時江夏王寶玄鎮京口聞慧景北行遣左右余文興說之曰江劉徐沈君之所見今擁強兵北取廣陵收吴楚勁卒身舉州以相應取大功如反掌耳慧景常不自安聞言響應于是廬陵王長史蕭寅司馬崔恭祖守廣陵城慧景以寶玄事告恭祖恭祖口雖相和心實不同俄而慧景至恭祖閉門不敢出慧景密遣軍主劉靈運間行突入慧景俄至遂據其城子覺至仍使領兵襲京口寶玄本謂大軍併來及見人少極失所望拒覺擊走之恭祖及覺精兵八千濟江恭祖心本不同及至蒜山欲斬覺以兵降京口事既不果而止覺等軍器精嚴柳憕沈佚等謂寶玄曰崔護軍威名既重乃誠可見既已脣齒忽中道立異彼以樂歸之衆亂江而濟誰能拒之於是登北固樓並千蠟燭為烽火舉以應覺慧景停二日便率大衆一時俱濟趣京口寶玄仍以覺為前鋒恭祖次之慧景領大都督為衆軍節度又云時柳憕别推寶玄崔恭祖為寶玄羽翼不復承奉慧景慧景嫌之巴陵王昭胄先逃人間出投慧景意更向之故猶豫未知所立此聲頗泄憕恭祖始貳於慧景又云慧景單馬至蠏浦投漁人太叔榮之榮之故為慧景門人時為蠏浦戍斬慧景送都按恭祖始若閉城拒慧景慧景襲得其城而據之豈肯更授以兵柄又慧景若不立寶玄柳憕豈能别推又榮之既云漁人又云為戍自相違錯今並從齊書}
以頭内鰌籃擔送建康|{
	鰌即由翻鰌魚今江淮間湖蕩河港皆有之春二月時人取食之其味甘美至三月後人不甚食謂之楊花鰌鰌籃所以盛鰌者}
恭祖繫尚方少時殺之|{
	少時言不多時也}
覺亡命為道人捕獲伏誅寶玄初至建康軍於東城|{
	東城即東府城}
士民多往投集|{
	往投寶玄而集於東城也}
慧景敗收得朝野投寶玄及慧景人名|{
	朝直遥翻}
帝令燒之曰江夏尚爾豈可復罪餘人|{
	昏暴之君豈無一言之幾乎理東昏侯此語是也復扶又翻}
寶玄逃亡數日乃出帝召入後堂以步障裹之令左右數十人鳴皷角馳繞其外|{
	晉志曰皷按周禮以鼖皷皷軍事角說者云蚩尤氏帥魑魅與黄帝戰于涿鹿黄帝乃始吹角為龍鳴以禦之其後魏武北征烏丸越沙漠而士卒思歸於是減為中嗚尤更悲矣}
遣人謂寶玄曰汝近圍我亦如此耳初慧景欲交處士何點|{
	處昌呂翻}
點不顧及圍建康逼召點點往赴其軍|{
	何點門世信佛齊朝累徵不就從弟遁以東籬門園居之故為慧景逼召往赴其軍}
終日談義不及軍事慧景敗帝欲殺點蕭暢謂茹法珍曰|{
	茹音如}
點若不誘賊共講未易可量|{
	言何點若不與慧景講義則慧景日以攻城為事安危未可量也誘音酉易以豉翻量音良}
以此言之乃應得封帝乃止點胤之兄也|{
	何胤隱於會稽若邪山}
蕭懿既去小峴王肅亦還洛陽荒人往來者妄云肅復謀歸國|{
	復扶又翻下當復同}
五月乙巳詔以肅為都督豫徐司三州諸軍事豫州刺史西豐公 己酉江夏王寶玄伏誅|{
	夏戶雅翻}
壬子大赦 六月丙子魏彭城王勰進位大司馬領司徒王肅加開府儀同三司|{
	賞取壽陽之功也}
太陽蠻田育丘等二萬八千戶附於魏|{
	太陽當作大陽}
魏置四郡十八縣 乙丑曲赦建康南徐兖二州|{
	崔慧景自南兖州還兵而南徐州之人從之進圍建康而建康之人又多從之既大赦而誅縱失實故又曲赦三處}
先是崔慧景既平|{
	先悉薦翻}
詔赦其黨而嬖倖用事不依詔書|{
	嬖卑義翻又博計翻}
無罪而家富者皆誣為賊黨殺而籍其貲實附賊而貧者皆不問或謂中書舍人王咺之云赦書無信人情大惡|{
	咺呪晚翻惡如字不善也}
咺之曰正當復有赦耳由是再赦既而嬖倖誅縱亦如初是時帝所寵左右凡三十一人黄門十人直閤驍騎將軍徐世素為帝所委任凡有殺戮皆在其手及陳顯逹事起加輔國將軍雖用護軍崔慧景為都督而兵權實在世世亦知帝昏縱密謂其黨茹法珍梅蟲兒曰何世天子無要人但儂貨主惡耳|{
	儂吴語我也茹音如}
法珍等與之爭權以白帝帝稍惡其凶彊|{
	惡烏路翻}
遣禁兵殺之世拒戰而死自是法珍蟲兒用事並為外監口稱詔敕王咺之專掌文翰與相脣齒帝呼所幸潘貴妃父寶慶及茹法珍為阿丈|{
	前漢書匈奴傳曰漢天子我丈人行也師古注丈人尊老之稱阿烏葛翻下同}
梅蟲兒俞靈韻為阿兄帝與法珍等俱詣寶慶家躬自汲水助廚人作膳寶慶恃勢作姦富人悉誣以罪田宅貲財莫不啟乞|{
	啟上而多所求乞}
一家被䧟禍及親隣又慮後患盡殺其男口|{
	被皮義翻}
帝數往諸刀敕家游宴|{
	數所角翻時人謂捉刀應敕之徒為刀敕}
有吉凶輒往慶弔奄人王寶孫年十三四|{
	周禮注奄精氣閉藏者今謂之宦人陸德明曰奄於檢翻劉曰於驗翻徐曰於劒翻今讀作閹音於炎翻}
號為倀子|{
	倀褚羊翻狂也}
最有寵參預朝政雖王咺之梅蟲兒之徒亦下之|{
	朝直遥翻下遐嫁翻}
控制大臣移易詔敕乃至騎馬入殿詆訶天子公卿見之莫不懾息焉|{
	懾息猶言惕息也懾懼也屏氣而息詆丁禮翻訶虎何翻懾之涉翻}
吐谷渾王伏連籌事魏盡禮|{
	言盡藩臣之禮吐從暾入聲谷音浴}
而居其國置百官皆如天子之制稱制於其隣國|{
	稱制於其隣國示君臨之}
魏主遣使責而宥之|{
	使疏吏翻}
冠軍將軍驃騎司馬陳伯之再引兵攻壽陽|{
	是年春伯之攻壽陽敗退今再攻之冠古玩翻驃匹妙翻騎奇寄翻}
魏彭城王勰拒之援軍未至汝隂太守傅永將郡兵三千救壽陽|{
	勰音協將即亮翻下同}
伯之防淮口甚固|{
	此汝水入淮之口也水經汝水東至汝隂原鹿縣入于淮}
永去淮口二十餘里牽船上汝水南岸|{
	上時掌翻下同}
以水牛挽之|{
	水牛形力倍於黄牛挽音晚}
直南趣淮|{
	趣七喻翻}
下船即渡適上南岸齊兵亦至會夜永潜入城勰喜甚曰吾北望已久恐洛陽難可復見|{
	守壽陽而援兵不至其心孤危故云然復扶又翻}
不意卿能至也勰令永引兵入城永曰永之此來欲以却敵若如教旨|{
	諸王與任專方州者皆得下教於其屬故云教旨}
乃是與殿下同受攻圍豈救援之意遂軍于城外秋八月乙酉勰部分將士與永并勢擊伯之於肥口|{
	分扶問翻水經淮水東過壽春縣北肥水自黎漿北過壽春城東又北流而入于淮謂之肥口時陳伯之盖軍於肥口以逼壽陽也}
大破之斬首九千俘獲一萬伯之脱身遁還淮南遂入于魏|{
	壽春縣自漢以來為淮南郡治所史言伯之既敗建康尋受兵遂不能爭壽陽}
魏遣鎮南將軍元英將兵救淮南未至伯之已敗魏主召勰還洛陽勰累表辭大司馬領司徒乞還中山|{
	中山定州也去年魏命勰刺定州今年春赴壽陽故乞還本任還從宣翻又如字下同}
魏主不許以元英行揚州事尋以王肅為都督淮南諸軍事揚州刺史持節代之 甲辰夜後宫火時帝出未還|{
	出市里遊走未還也}
宫内人不得出外人不敢輒開|{
	謂不敢輒開後宫門}
比及開死者相枕|{
	比必利翻枕之任翻}
燒三十餘間時嬖倖之徒皆號為鬼有趙鬼者能讀西京賦言於帝曰柏梁既災建章是營|{
	後漢張衡作東京西京賦柏梁災營建章事見二十一卷漢武帝太初元年}
帝乃大起芳樂玉壽等諸殿|{
	樂音洛}
以麝香塗壁|{
	麝狀如小麋其臍有香華山之隂多有之陸佃曰商洛山中多麝所遺糞常就一處雖遠逐食必還走其地不敢遺迹它所慮為人所獲人反以是蹤迹其所在必掩羣而取之麝絶愛其香每為人所迫逐勢且急即自投高巖舉爪剔出其香就縶且死猶拱四足抱其臍麝神夜翻}
刻畫裝飾窮極綺麗役者自夜逹暁猶不副速|{
	副稱也不能稱其欲速之意也}
後宫服御極選珍奇府庫舊物不復周用|{
	復扶又翻}
貴市民間金寶價皆數倍建康酒租皆折使輸金|{
	使以金折錢輸官折之舌翻}
猶不能足鑿金為蓮華以帖地令潘妃行其上曰此步步生蓮華也|{
	華讀曰花}
又訂出雉頭鶴氅白鷺縗|{
	訂丁定翻平議也齊梁之時謂賦民為訂蓋取平議而賦之之義雉頭上毛細而色紅鮮如錦晉程據緝以為裘鶴氅鶴翎毛也白鷺縗鷺頭上毦也鶴氅鷺縗皆取其潔白詩疏曰鷺水鳥毛白而潔頂上有毛毿毿然此即縗也爾雅釋名曰鷺春鉏郭璞曰白鷺也頭翅背上皆有長翰毛今江東人取以為睫攡名之曰白鷺縗陸機曰鷺頭上有毛十數枚長尺餘毿毿然與衆毛異氅音齒兩翻縗音倉回翻}
嬖倖因緣為姦利課一輸十又各就州縣求為人輸準取見直|{
	為人于偽翻下不為同見賢遍翻}
不為輸送守宰皆不敢言重更科歛|{
	重直用翻更居孟翻再也}
如此相仍前後不息百姓困盡號泣道路|{
	號戶高翻}
軍主吳子陽等出三關侵魏九月與魏東豫州刺史

田益宗戰於長風城|{
	左傳定公四年蔡侯與吳子唐侯伐楚還塞大隧直轅冥阸所謂大隧即黄峴關直轅冥阸乃武陽平靖二關也黄峴今名九里關在義陽郡南百里武陽在今大寨嶺郡東南九十里平靖今名行者坡郡南七十五里魏太和十七年田益宗降魏十九年置東豫州於新息廣陵城以益宗為刺史長風城在隂山關南隂山關在弋陽縣界宋文帝元嘉二十五年以豫部蠻民立十八縣長風其一也屬西陽郡九域志舒州懷寧縣有長風鎮懷寧漢皖縣地晉安帝立晉熙郡仍立懷寜縣為郡治所盖以懷寧蠻左名縣也}
子陽等敗還 |{
	考異曰此一事齊書紀傳皆無之魏帝紀九月乙丑東豫州刺史田益宗破寶卷將吳子陽鄧元起於長風梁書鄧元起傳蠻帥田孔明附于魏自號郢州刺史宼掠三關規襲夏口元起帥鋭卒攻之旬月之間頻䧟六城斬獲萬計餘黨皆散走仍戍三關二書勝敗不同如此今從魏紀}
蕭懿之入援也蕭衍馳使所親虞安福說懿曰|{
	說輸芮翻}


|{
	下說帝同}
誅賊之後則有不賞之功當明君賢主尚或難立况於亂朝何以自免|{
	朝直遥翻下同}
若賊滅之後仍勒兵入宫行伊霍故事|{
	使之廢立也}
此萬世一時若不欲爾便放表還歷陽託以外拒為事則威振内外誰敢不從一朝放兵受其厚爵高而無民必生後悔|{
	謂官爵雖高而兵權去已必將束手就死}
長史徐矅甫苦勸之懿並不從崔慧景死懿為尚書令有弟九人敷衍暢融宏偉秀憺恢|{
	憺徒敢翻又徒濫翻}
懿以元勲居朝右暢為衛尉掌管籥時帝出入無度或勸懿因其出門|{
	謂出臺城門而遊走也}
舉兵廢之懿不聽嬖臣茹法珍王喧之等憚懿威權說帝曰懿將行隆昌故事|{
	謂隆昌廢欎林王也嬖卑義翻又博計翻茹音如咺况晚翻}
陛下命在晷刻帝然之徐矅甫知之密具舟江渚勸懿西奔襄陽懿曰自古皆有死豈有叛走尚書令邪|{
	史言蕭懿忠於齊室}
懿弟姪咸為之備冬十月己卯帝賜懿藥於省中懿且死曰家弟在雍深為朝廷憂之|{
	雍於用翻時以襄陽為雍州治所言衍必將舉兵也為于偽翻}
懿弟姪皆亡匿於里巷無人之者|{
	史言人心皆為蕭懿兄弟覆護}
唯融捕得誅之 丁亥魏以彭城王勰為司徒録尚書事勰固辭不免勰雅好恬素不樂勢利高祖重其事幹|{
	好呼到翻樂音洛幹用也謂臨事有幹用也}
故委以權任雖有遺詔|{
	遺詔見上卷上年}
復為世宗所留|{
	謂出當方面復入為司徒録尚書也復扶又翻}
勰每乖情願常悽然歎息為人美風儀端嚴若神折旋合度|{
	記曰周旋中規折旋中矩注云折旋曲行也}
出入言笑觀者忘疲敦尚文史物務之暇披覧不輟小心謹慎初無過失雖閒居獨處|{
	處昌呂翻}
亦無惰容愛敬儒雅傾心禮待清正儉素門無私謁|{
	史言彭城王勰為魏宗室諸王之秀}
十一月己亥魏東荆州刺史桓暉入寇拔下笮戍|{
	下笮戍在沔北直襄陽東北笮側百翻又在各翻}
歸之者二千餘戶暉誕之子也|{
	宋明帝泰豫元年桓誕降魏}
初帝疑雍州刺史蕭衍有異志直後滎陽鄭植弟紹

叔為衍寧蠻長史帝使植以紹叔為名往刺衍|{
	使為刺客刺七亦翻}
紹叔知之密以白衍衍置酒紹叔家戲植曰朝廷遣卿見圖今日閒宴是可取良會也賓主大笑又令植歷觀城隍府庫士馬器械舟艦|{
	艦戶黯翻}
植退謂紹叔曰雍州實力未易圖也紹叔曰兄還具為天子言之|{
	易以豉翻為于偽翻}
若取雍州紹叔請以此衆一戰送植於南峴|{
	南峴盖即馬鞍山道}
相持慟哭而别|{
	各盡力於所事恐不復相見故慟哭而别}
及懿死衍聞之夜召張弘策呂僧珍長史王茂别駕柳慶遠功曹吉士瞻等入宅定議|{
	宅謂州宅也 考異曰南史云茂與梁武帝不睦諸腹心並勸除之而茂少有驍名帝又惜其用令腹心鄭紹叔往之告以欲起義茂因擲枕起即袴褶隨紹叔入見武帝大喜下牀迎因結兄弟披推腹心按茂若與武帝不睦梁武何敢豫告以大事茂亦安能便響應今不取}
茂天生之子|{
	王天生事齊高帝攻袁粲見一百三十四卷宋順帝昇明元帝}
慶遠元景之弟子也|{
	諸柳雍州豪望世不乏人}
乙巳衍集僚佐謂曰昏主暴虐惡踰於紂當與卿等共除之是日建牙集衆 |{
	考異曰齊帝紀十二月梁王起義兵於襄陽誤也今從梁書高祖紀}
得甲士萬餘人馬千餘匹船三千艘|{
	艘蘇遭翻}
出檀溪竹木裝艦葺之以茅事皆立辦諸將爭㯭呂僧珍出先所具者每船付二張爭者乃息|{
	㯭與櫓同僧珍具櫓事見上卷元年然僧珍所具者數百張櫓耳安能給三千艘邪每船付二張盖給諸將所乘之船耳}
是時南康王寶融為荆州刺史西中郎長史蕭頴胄行府州事|{
	南康王以西中郎將鎮荆州頴胄為長史行事}
帝遣輔國將軍巴西梓潼二郡太守劉山陽將兵三千之官|{
	守又式翻將即亮翻}
就頴胄兵使襲襄陽衍知其謀遣參軍王天虎詣江陵徧與州府書|{
	州謂荆州官屬府謂西中郎府官屬}
聲云山陽西上并襲荆雍|{
	衍書宣布此聲也上時掌翻雍於用翻}
衍因謂諸將佐曰荆州素畏襄陽人|{
	襄陽被邊人皆習兵故荆州人畏之}
加以脣亡齒寒寧不闇同邪我合荆雍之兵鼓行而東雖韓白復生不能為建康計|{
	復扶又翻下衍復非復復不州復豈復佐復同}
况以昏主役刀敕之徒哉頴胄得書疑未能决山陽至巴陵|{
	晉武帝太康元年立巴陵縣屬長沙郡宋武帝元嘉十六年分立巴陵郡時屬郢州今岳州即其地}
衍復令天虎齎書與頴胄及其弟南康王友頴逹|{
	王國官有師有友}
天虎既行衍謂張弘策曰用兵之道攻心為上|{
	孫武子兵法有是言}
近遣天虎往荆州人皆有書今段乘驛甚急|{
	今段猶云今來一段事也}
止有兩函與行事兄弟|{
	行事兄弟謂頴胄頴逹}
云天虎口具|{
	書中不言事但云天虎口具所以疑之}
及問天虎而口無所說|{
	盖天虎之行衍亦未嘗以一語屬之}
天虎是行事心膂|{
	據頴胄傳天虎頴胄親人故云然}
彼間必謂行事與天虎共隱其事則人人生疑山陽惑於衆口判相嫌貳|{
	判决也嫌疑也貳持兩端也}
則行事進退無以自明必入吾謀内是持兩空函定一州矣|{
	蕭衍舉事於襄陽智計横出及遇侯景庸夫之不若豈耄邪抑天奪其鑒也}
山陽至江安|{
	晉武帝太康元年立江安縣屬南平郡水經注江安即公安晉平江南杜預罷華容置江安縣以吳之南郡為南平郡治焉}
遲回十餘日不上|{
	自江安至江陵泝江北上而後至上時掌翻蕭子顯齊書曰至巴陵遲回十餘日不進}
頴胄大懼計無所出夜呼西中郎城局參軍安定席闡文諮議參軍柳忱閉齋定議|{
	定議以决其所從忱氏壬翻}
闡文曰蕭雍州畜養士馬非復一日|{
	畜許六翻}
江陵素畏襄陽人又衆寡不敵取之必不可制就能制之歲寒復不為朝廷所容|{
	四時運而成歲歲至極寒而終矣歲寒以喻世事終極處孔子曰歲寒然後知松柏之後彫亦此意}
今若殺山陽與雍州舉事立天子以令諸侯則覇業成矣山陽持疑不進是不信我今斬送天虎則彼疑可釋至而圖之罔不濟矣忱曰朝廷狂悖日滋京師貴人莫不重足累息|{
	悖蒲内翻又蒲没翻重直龍翻重足而立累息而不敢出氣懼之甚也}
今幸在遠得假日自安雍州之事且藉以相斃耳|{
	藉借也音慈夜翻}
獨不見蕭令君乎|{
	蕭懿為尚書令故呼為令君}
以精兵數千破崔氏十萬衆竟為羣邪所䧟禍酷相尋前事之不忘後事之師也|{
	史記鄭世家太史公之言}
且雍州士鋭糧多蕭使君雄姿冠世|{
	冠古玩翻}
必非山陽所能敵若破山陽荆州復受失律之責進退無可宜深慮之蕭頴逹亦勸頴胄從闡文等計詰旦|{
	詰去吉翻}
頴胄謂天虎曰卿與劉輔國相識今不得不借卿頭乃斬天虎送示山陽發民車牛聲云起步軍征襄陽山陽大喜甲寅山陽至江津單車白服從左右數十人詣頴胄頴胄使前汶陽太守劉孝慶等伏兵城内山陽入門|{
	入城門也}
即於車中斬之副軍主李元履收餘衆請降|{
	降戶江翻}
柳忱世隆之子也|{
	柳世隆為高武佐命功臣}
頴胄慮西中郎司馬夏侯詳不同以告忱忱曰易耳|{
	易以豉翻}
近詳求昏未之許也乃以女嫁詳子夔而告之謀詳從之乙卯以南康王寶融敎纂嚴又敎赦囚徒施惠澤頒賞格|{
	纂集也嚴装也纂嚴纂集行装也纂嚴一敎赦囚徒施惠澤頒賞格又一敎}
丙辰以蕭衍為使持節都督前鋒諸軍事|{
	使疏吏翻下同}
丁巳以蕭頴胄為都督行留諸軍事|{
	行謂東下之軍留謂留守之軍}
頴胄有器局既舉大事虛心委己衆情歸之以别駕南陽宗夬|{
	夬古邁翻}
及同郡中兵參軍劉垣諮議參軍樂藹為州人所推信軍府經略每事諮焉頴胄夬各獻私錢穀及換借富貲以助軍長沙寺僧素富鑄黄金為金龍數千兩埋土中|{
	長沙寺在江陵宋元嘉中臨川王義慶鎮江陵起寺為其本生父長沙王道憐資福因名長沙寺}
頴胄取之以資軍費頴胄遣使送劉山陽首於蕭衍且言年月未利當須明年二月進兵衍曰舉事之初所藉者一時驍鋭之心|{
	驍堅堯翻}
事事相接猶恐疑怠若頓兵十旬必生悔吝|{
	兵以氣勢為用者也是以巧遲不若拙速}
且坐甲十萬糧用自竭若童子立異則大事不成况處分已定安可中息哉|{
	處昌呂翻分扶問翻}
昔武王伐紂行逆太歲豈復待年月乎戊午衍上表勸南康王寶融稱尊號不許十二月頴胄與夏侯詳移檄建康百官及州郡牧守數帝及梅蟲兒茹法珍罪惡|{
	數所具翻}
頴胄遣冠軍將軍天水楊公則向湘州|{
	使攻張寶積也冠古玩翻}
西中郎參軍南郡鄧元起向夏口|{
	使助蕭衍攻張冲也夏戶雅翻}
軍主王法度坐不進軍免官乙亥荆州將佐復勸寶融稱尊號不許夏侯詳之子驍騎將軍亶為殿中主帥|{
	帥所類翻}
詳密召之亶自建康亡歸壬辰至江陵稱奉宣德皇太后令南康王宜纂承皇祚方俟清宫未即大號可封十郡為宣城王|{
	時以宣城南琅邪南東海東陽臨海新安尋陽南郡竟陵宜都十郡為宣城王國盖以明帝自宣城王入纂大統故假宣德太后令以是肇封}
相國荆州牧加黄鉞選百官西中郎府南康國如故須軍次近路|{
	須待也}
主者備法駕奉迎竟陵太守新野曹景宗遣親人說蕭衍|{
	說輸芮翻}
迎南康王都襄陽先正尊號然後進軍衍不從王茂私謂張弘策曰今以南康置人手中彼挾天子以令諸侯節下前進為人所使此豈他日之長計乎弘策以告衍衍曰若前塗大事不捷故自蘭艾同焚|{
	蘭有國香人貴之艾蕭艾也人賤之言若事不捷則無貴無賤同於死也}
若其克捷則威振四海豈碌碌受人處分者邪|{
	蕭衍此言已有代齊之心特權宜推奉南康以舉兵耳處昌呂翻分扶問翻}
初陳顯逹崔慧景之亂人心不安或問時事於上庸太守杜陵韋叡|{
	杜陵自漢以來屬京兆晉僑立京兆太守及杜陵令寄治襄陽宋大明土斷割襄陽西界為實土}
叡曰陳雖舊將非命世才崔頗更事懦而不武|{
	將即亮翻更工衡翻}
其赤族宜矣定天下者殆必在吾州將乎|{
	州刺史當方面摠兵權故曰州將將即亮翻}
乃遣二子自結於蕭衍及衍起兵叡帥郡兵二千倍道赴之|{
	帥讀曰率下同}
華山太守藍田康絢帥郡兵三千赴衍|{
	藍田縣漢屬京兆宋置僑縣屬華山郡康絢傳云其先本康居侍子待詔河西因留不去其後遂氏焉晉亂遷于藍田絢祖穆帥郷族三千餘家入襄陽宋為置華山郡藍田縣於襄陽宋白曰宋大明元年立華山郡於大堤村後魏改華山郡為宜城郡唐為宜城縣屬襄州華戶化翻絢翾縣翻}
馮道根時居母喪帥鄉人子弟勝兵者悉往赴之|{
	馮道根酇人酇縣時屬廣平僑縣勝書烝翻}
梁南秦二州刺史柳惔亦起兵應衍惔忱之兄也|{
	惔徒甘翻}
帝聞劉山陽死發詔討荆雍戊寅以冠軍長史劉澮為雍州刺史|{
	欲以代蕭衍雍於用翻冠古玩翻澮古外翻}
遣驍騎將軍薛元嗣制局監暨榮伯將兵及運糧百四十餘船送郢州刺史張冲使拒西師|{
	荆雍在西故謂之西師暨姓也音居乙翻又泉暨二音將即亮翻驍堅堯翻騎奇寄翻}
元嗣等懲劉山陽之死疑冲不敢進停夏口浦聞西師將至乃相帥入郢城前竟陵太守房僧寄將還建康至郢帝敕僧寄留守魯山除驍騎將軍張冲與之結盟遣軍主孫樂祖將數千人助僧寄守魯山|{
	水經注江水東逕魯山南與沔水會山左即沔水口沔左有偃月城漢陽志大别山在沔陽縣東一名魯山}
蕭頴胄與武寧太守鄧元起書招之|{
	晉安帝隆安五年桓玄以沮漳降蠻立武寧郡屬荆州五代志竟陵郡樂鄉縣舊置武寜郡劉昫曰樂鄉漢郡縣地至我宋廢縣為樂鄉鎮入長林縣}
張冲待元起素厚衆皆勸其還郢|{
	還郢州也}
元起大言於衆曰朝廷暴虐誅戮宰輔羣小用事衣冠道盡荆雍二州同舉大事何患不克且我老母在西若事不成止受戮昏朝幸免不孝之罪|{
	時鄧元起之母盖在江陵元起南郡人也守武寜其母留鄉里朝直遥翻}
即日治嚴上道至江陵為西中郎中兵參軍|{
	是時西臺方遣元起向夏口觀者不以史文先後之次而害意可也治直之翻上時掌翻}
湘州行事張寶積兵自守未知所附楊公則克巴陵進軍白沙|{
	水經注白沙戍在黄陵廟北黄陵廟舜二妃廟也羅含湘中記曰湘川白沙如霜雪赤崖若朝霞}
寶積懼請降|{
	降戶江翻}
公則入長沙撫納之 是時北秦州刺史楊集始將衆萬餘自漢中北出規復舊地|{
	楊集始失國事見一百四十一卷明帝建武四年將即亮翻下同}
魏梁州刺史楊椿將步騎五千出頓下辯|{
	辯皮莧翻}
遺集始書開以利害集始遂復將其部曲千餘人降魏|{
	遺于季翻復扶又翻}
魏人還其爵位使歸守武興|{
	集始降齊魏人削其所授爵位而所領北秦州刺史則齊所授也今降魏魏人還其元授爵位也}


資治通鑑卷一百四十三
















































































































































