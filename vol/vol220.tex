資治通鑑卷二百二十
宋 司馬光 撰

胡三省 音註

唐紀三十六|{
	起彊圉作噩九月盡著雍閹茂凡一年有奇}


肅宗文明武德大聖大宣孝皇帝中之下

至德二載九月丁丑希德以輕騎至城下挑戰千里帥百騎開門突出欲擒之會救至收騎退還橋壞墜塹中反為希德所擒|{
	為將者不可恃勇輕脫程千里欲擒蔡希德反為希德所擒恃勇輕脫之禍也騎奇寄翻挑徒了翻帥讀曰率}
仰謂從騎曰吾不幸至此天也歸語諸將|{
	從才用翻語牛倨翻}
善為守備寜失帥不可失城|{
	帥所類翻}
希德攻城竟不克送千里於洛陽安慶緒以為特進囚之客省郭子儀以回紇兵精勸上益徵其兵以擊賊懷仁可

汗遣其子葉護及將軍帝德等將精兵四千餘人來至鳳翔上引見葉護宴勞賜賚惟其所欲|{
	見賢遍翻勞力到翻}
丁亥元帥廣平王俶將朔方等軍及回紇西域之衆十五萬號二十萬發鳳翔俶見葉護約為兄弟葉護大喜謂俶為兄回紇至扶風郭子儀留宴三日葉護曰國家有急遠來相助何以食為宴畢即行日給其軍羊二百口牛二十頭米四十斛庚子諸軍俱發壬寅至長安西陳於香積寺北灃水之東|{
	此皆漢上林苑地也地說云豐水出鄠南豐谷北流逕漢龍臺觀東南與渭水會于短隂山程大昌曰香積寺呂圖在子午谷正北微西郭子儀收長安陳于寺北距澧水臨大川大川者沉水交水唐永安渠也蓋寺在澧水之東交水之西也呂圖云在鎬水發源之北則近昆明池矣子儀先敗于清渠至此則循南山出都城後據地勢以待之也陳讀曰陣下陳於其陳於陳陳乃賊陳同}
李嗣業為前軍郭子儀為中軍王思禮為後軍賊衆十萬陳於其北李歸仁出挑戰官軍逐之逼於其陳賊軍齊進官軍却為賊所乘軍中驚亂賊争趣輜重|{
	重直用翻}
李嗣業曰今日不以身餌賊軍無孑遺矣乃肉袒執長刀立於陳前大呼奮擊|{
	呼火故翻}
當其刀者人馬俱碎殺數十人陳乃稍定於是嗣業帥前軍各執長刀如牆而進身先士卒|{
	先悉薦翻}
所向摧靡都知兵馬使王難得救其禆將|{
	王難得為鳳翔都知兵馬使時上在鳳翔蓋御營大將也}
賊射之中眉皮垂鄣目難得自拔箭掣去其皮血流被面|{
	射而亦翻中竹仲翻掣昌列翻去羌呂翻被皮義翻}
前戰不已賊伏精騎於陳東欲襲官軍之後偵者知之|{
	騎奇寄翻偵丑鄭翻}
朔方左廂兵馬使僕固懷恩引回紇就擊之翦滅殆盡賊由是氣索|{
	索昔各翻盡也}
李嗣業又與回紇出賊陳後與大軍夾擊自午及酉斬首六萬級填溝塹死者甚衆賊遂大潰餘衆走入城迨夜囂聲不止|{
	塹七艶翻囂五羔翻}
僕固懷恩言於廣平王俶曰賊弃城走矣請以二百騎追之縛取安守忠李歸仁等|{
	俶昌六翻騎奇寄翻}
俶曰將軍戰亦疲矣且休息俟明旦圖之懷恩曰歸仁守忠賊之驍將驟勝而敗此天賜我也奈何縱之使復得衆|{
	驍堅堯翻將即亮翻復扶又翻下而復可復復脩復為敢復同}
還為我患悔之無及戰尚神速何明旦也|{
	言何用俟明旦}
俶固止之使還營|{
	還從宣翻又音如字}
懷恩固請往而復反一夕四五起遲明諜至|{
	遲直二翻諜逹叶翻}
守忠歸仁與張通儒田乾真皆已遁矣|{
	廣平王若用僕固懷恩之言固不假新店之戰可以逕取東京矣}
癸卯大軍入西京初上欲速得京師與回紇約曰克城之日土地士庶歸唐金帛子女皆歸回訖至是葉護欲如約廣平王俶拜於葉護馬前曰今始得西京若遽俘掠則東京之人皆為賊固守|{
	紇下没翻為于偽翻下當為同}
不可復取矣願至東京乃如約葉護驚躍下馬答拜跪捧王足|{
	夷禮以拜跪捧足為敬}
曰當為殿下徑往東京即與僕固懷恩引回紇西域之兵自城南過營於滻水之東|{
	過京城南歷安化門明德門啟夏門外遶京城東南角轉北歷延興春明通化三門之外至滻水滻水出藍田縣境之西北行過白鹿原西又北入于霸水滻音產}
百姓軍士胡虜見俶拜皆泣曰廣平王真華夷之主上聞之喜曰朕不及也俶整衆入城百姓老幼夾道歡呼悲泣俶留長安鎮撫三日引大軍東出|{
	東出京城門取洛陽俶昌六翻}
以太子少傅虢王巨為西京留守|{
	少始照翻守式又翻}
甲辰捷書至鳳翔百寮入賀上涕泗交頤即日遣中使啖庭瑶入蜀奏上皇|{
	使疏吏翻啖徒敢翻姓也}
命左僕射裴冕入京師告郊廟及宣慰百姓上以駿馬召李泌於長安|{
	射寅謝翻泌毗必翻李泌時從軍在長安}
既至上曰朕已表請上皇東歸朕當還東宫復脩臣子之職泌曰表可追乎上曰已遠矣泌曰上皇不來矣上驚問故泌曰理勢自然上曰為之奈何泌曰今請更為羣臣賀表言自馬嵬請留靈武勸進|{
	更古孟翻嵬五回翻請留勸進事並見二百十八卷至德元載}
及今成功聖上思戀晨昏請速還京以就孝養之意則可矣|{
	養羊尚翻}
上即使泌草表上讀之泣曰朕始以至誠願歸萬機今聞先生之言乃寤其失立命中使奉表入蜀因就泌飲酒同榻而寢而李輔國請取契鑰付泌泌請使輔國掌之上許之|{
	泌掌契鑰見二百十八卷上年九月今付輔國宫禁之權盡歸之矣為輔國專擅張本}
泌曰臣今報德足矣復為閒人何樂如之上曰朕與先生累年同憂患今方相同娯樂|{
	樂音洛}
奈何遽欲去乎泌曰臣有五不可留願陛下聽臣去免臣於死上曰何謂也對曰臣遇陛下太早陛下任臣太重寵臣太深臣功太高迹太奇此其所以不可留也上曰且眠矣異日議之對曰陛下今就臣榻卧猶不得請况異日香案之前乎|{
	唐制凡朝日殿上設黼扆躡席熏爐香按皇帝升御座宰執當香案前奏事}
陛下不聽臣去是殺臣也上曰不意卿疑朕如此豈有如朕而辦殺卿邪是直以朕為句踐也|{
	邪音耶范蠡既與越王句踐報吳之恥蠡乃扁舟五湖遺大夫文種書以為句踐長頸烏喙可與同患難不可與同安樂文種見書遂稱疾句踐賜文種死句音鉤}
對曰陛下不辦殺臣故臣求歸若其既辦臣安敢復言|{
	復扶又翻}
且殺臣者非陛下也乃五不可也陛下曏日待臣如此臣於事猶有不敢言者况天下既安臣敢言乎上良久曰卿以朕不從卿北伐之謀乎|{
	謂不從使建寜王自媯檀取范陽之策也肅宗以意言之}
對曰非也所不敢言者乃建寜耳上曰建寜朕之愛子性英果艱難時有功|{
	謂馬嵬勸留及北赴靈武血戰以衛上也事見二百十八卷元載六月}
朕豈不知之但因此為小人所教欲害其兄圖繼嗣朕以社稷大計不得已而除之|{
	事見上卷本年正月嗣祥吏翻}
卿不細知其故邪對曰若有此心廣平當怨之廣平每與臣言其寃輒流涕嗚咽臣今必辭陛下去始敢言之耳上曰渠嘗夜捫廣平意欲加害對曰此皆出讒人之口豈有建寜之孝友聦明肯為此乎且陛下昔欲用建寜為元帥臣請用廣平|{
	事見二百十八卷元載九月帥所類翻}
建寜若有此心當深憾於臣而以臣為忠益相親善陛下以此可察其心矣上乃泣下曰先生言是也既往不咎|{
	引論語孔子之言}
朕不欲聞之泌曰臣所以言之者非咎既往乃欲使陛下慎將來耳昔天后有四子長曰太子弘天后方圖稱制惡其聦明酖殺之|{
	見二百二卷高宗上元二年}
立次子雍王賢賢内憂懼作黄臺瓜辭冀以感悟天后天后不聽賢卒死於黔中|{
	賢廢見二百二卷永隆元年死見二百三卷武后光宅元年卒子恤翻黔音禽}
其辭曰種瓜黄臺下瓜熟子離離一摘使瓜好再使瓜稀三摘猶為可四抱蔓歸今陛下已一矣慎無再上愕然曰安有是哉卿錄是辭朕當書紳對曰陛下但識之於心|{
	識職吏翻記也}
何必形於外也是時廣平王有大功良娣忌之潜搆流言故泌言及之|{
	李泌歷事肅代德三朝皆能言人所難言奇士也}
郭子儀引蕃漢兵追賊至潼關斬首五千級克華陰弘農二郡關東獻俘百餘人敕皆斬之監察御史李勉言於上曰今元惡未除為賊所汚者半天下|{
	汚烏故翻}
聞陛下龍興咸思洗心以承聖化今悉誅之是驅之使從賊也上遽使赦之 冬十月丁未啖庭瑶至蜀 壬子興平軍奏破賊於武關克上洛郡|{
	時王難得領興平軍}
吐蕃陷西平|{
	西平郡鄯州}
尹子奇久圍睢陽城中食盡議棄城東走張廵許遠謀以為睢陽江淮之保障若棄之去賊必乘勝長驅是無江淮也|{
	考異曰唐人皆以全江淮為廵遠功按睢陽雖當江淮之路城既被圍賊若欲取江淮繞出其外睢陽豈能障之哉蓋廵善用兵賊畏廵為後患不滅廵則不敢越過其南耳}
且我衆饑羸走必不達古者戰國諸侯尚相救恤|{
	謂春秋列國同盟有急則相救恤也}
况密邇羣帥乎|{
	羣帥謂張鎬尚衡許叔冀等帥所類翻}
不如堅守以待之茶紙既盡遂食馬馬盡羅雀掘鼠雀鼠又盡廵出愛妾殺以食士|{
	食祥史翻}
遠亦殺其奴然後括城中婦人食之繼以男子老弱人知必死莫有叛者所餘纔四百人癸丑賊登城將士病不能戰廵西向再拜曰臣力竭矣不能全城生既無以報陛下死當為厲鬼以殺賊|{
	鬼無所歸者為厲}
城遂陷廵遠俱被執尹子奇問廵曰聞君每戰眥裂齒碎何也|{
	眥疾智翻又才詣翻目眥也}
廵曰吾志吞逆賊但力不能耳子奇以刀抉其口視之|{
	扶一决翻}
所餘纔三四子奇義其所為欲活之其徒曰彼守節者也終不為用且得士心存之將為後患乃并南霽雲雷萬春等三十六人皆斬之 |{
	考異曰新傳曰虢王巨之走臨淮廵有妹嫁陸氏遮巨勸勿行不納賜百縑弗受為廵補縫行間軍中號陸家姑先廵被害按巨在彭城若走臨淮陸姊在睢陽城何以得遮之今不取}
廵且死顔色不亂揚揚如常生致許遠於洛陽廵初守睢陽時卒僅萬人城中居人亦且數萬廵一見問姓名其後無不識者前後大小戰凡四百餘殺賊卒十二萬人廵行兵不依古法教戰陳令本將各以其意教之|{
	本將謂本部之將陳讀曰陣}
人或問其故廵曰今與胡虜戰雲合鳥散變態不恒數步之間勢有同異臨機應猝在於呼吸之間而動詢大將事不相及非知兵之變者也故吾使兵識將意將識士情投之而往如手之使指兵將相習人自為戰不亦可乎自興兵器械甲仗皆取之於敵未嘗自修每戰將士或退散廵立於戰所謂將士曰我不離此|{
	離力智翻}
汝為我還决之將士莫敢不還死戰卒破敵|{
	為于偽翻卒子恤翻}
又推誠待人無所疑隐臨敵應變出奇無窮號令明賞罰信與衆共甘苦寒暑故下争致死力張鎬聞睢陽圍急倍道亟進|{
	張鎬代賀蘭進明見上卷八月}
檄浙東浙西淮南北海諸節度|{
	按新書方鎮表浙東浙西明年方置節度使時崔渙在浙東李希言在浙西皆非節度使淮南則李成式北海尚為賊將能元皓所據然去年已置北海節度使是雖未復北海而已置北海帥矣}
及譙郡太守閭丘曉使共救之曉素傲狠不受鎬命比鎬至|{
	比必利翻及也}
睢陽城已陷三日鎬召曉杖殺之 |{
	考異曰舊傳作豪州刺史新傳作濠州刺史統紀作亳州刺史按濠州在淮南去睢陽遠亳州與睢陽接境必亳州也今從統紀 余按通鑑改統紀之亳州為譙郡以此時未復郡為州也讀者宜知之}
張通儒等收餘衆走保陜|{
	自長安東走保陜}
安慶緒悉發洛陽兵使其御史大夫嚴莊將之就通儒以拒官軍并舊兵步騎猶十五萬|{
	舊兵謂張通儒等所領自西京東走之兵}
己未廣平王至曲沃|{
	此非春秋晉莊叔所封之曲沃按其地在弘農靈寶二縣之間水經注弘農縣東十三里有好陽亭又東有曲沃城}
回紇葉護使其將軍鼻施吐撥裴羅等引軍旁南山搜伏因駐軍嶺北|{
	旁步浪翻}
郭子儀等與賊遇於新店|{
	據舊書新店在陜城西}
賊依山而陳子儀等初與之戰不利賊逐之下山回紇自南山襲其背於黃埃中發十餘矢賊驚顧曰回紇至矣遂潰官軍與回紇夾擊之賊大敗僵尸蔽野嚴莊張通儒等弃陜東走廣平王俶郭子儀入陜城僕固懷恩等分道追之嚴莊先入洛陽告安慶緒庚申夜慶緒帥其黨自苑門出|{
	東都苑門也}
走河北 |{
	考異曰實録無新店戰日但云子儀與嗣業等至新店遇賊大破之逐北五十餘里人馬相枕藉器械戈甲自陜至洛城委弃道路無空地庚申慶緒走其夜自東都苑門帥其衆黨奔河北壬戍元帥廣平王與子儀取陜郡汾陽家傳九月安慶緒自洛疾使諸將至陜兼收敗卒猶十五萬十月四日於陜西依山而陳彼則憑高下擊此乃進軍上衝賊屹立不動公使偽退引令下山使回紇驀澗走險以襲其背賊乃敗績斬九萬級擒一萬人汾陽家傳十月四日破賊於陜西八日收洛陽年代記十月己未破賊于新店辛酉慶緒聞軍敗率其黨投相州舊紀庚申慶緒奔河北壬戍廣平王入東京新紀戊申敗賊新店克陜郡壬子復東京按陜洛之間幾三百里汾陽傳新紀太早實録壬戍收陜郡太晚今從年代記幸蜀記}
殺所獲唐將哥舒翰程千里等三十餘人而去許遠死于偃師 |{
	考異曰實錄舊傳皆曰尹子奇執送洛陽與哥舒翰程千里俱囚於客省及安慶緒敗度河北走使嚴莊皆害之張中丞傳相里造誄曰唐故御史中丞張許二君以守城睢陽陷張君遇害許君為羯賊所擒求死不得降逼至偃師縣亦被兵焉今從之}
壬戌廣平王俶入東京回紇意猶未厭俶患之父老請率羅錦萬匹以賂回紇回紇乃止 成都使還|{
	此還者啖庭瑶也還音旋}
上皇誥曰當與我劒南一道自奉不復來矣|{
	復扶又翻下嗣復同}
上皇懼不知所為後使者至|{
	此奉羣臣賀表中使繼還也}
言上皇初得上請歸東宫表彷徨不能食欲不歸及羣臣表至乃大喜命食作樂下誥定行日|{
	定東行歸京之日也}
上召李泌告之曰皆卿力也泌求歸山不已上固留之不能得乃聽歸衡山|{
	衡山在衡陽郡衡山縣西三十里南嶽也漢武帝以霍山為南嶽隋文帝以衡山為南嶽按泌傳泌願隱衡山詔聽之}
敕郡縣為之築室於山中|{
	為于偽翻}
給三品料 癸亥上發鳳翔遣太子太師韋見素入蜀奉迎上皇 乙丑郭子儀遣左兵馬使張用濟右武鋒使渾釋之將兵取河陽及河内嚴莊來降陳留人殺尹子奇舉郡降田承嗣圍來瑱於頴川亦遣使來降郭子儀應之緩承嗣復叛與武令珣皆走河北|{
	走音奏 考異曰舊魯炅傳云炅保南陽賊使武令珣攻之令珣死又令田承嗣攻之下又曰王師收兩京承嗣令珣奔河北唐歷慶緒據鄴武令珣自唐鄧至炅傳云武令珣死誤也}
制以瑱為河南節度使 丙寅上至望賢宫|{
	雍錄望賢宫在咸陽縣東數里}
得東京捷奏丁卯上入西京百姓出國門奉迎二十里不絶舞躍呼萬歲有泣者上入居大明宫|{
	高宗咸亨元年改蓬萊宫為大明宫即東内}
御史中丞崔器令百官受賊官爵者皆脫巾徒跣立於含元殿前|{
	含元殿東内前殿也當丹鳳門内}
膺頓首請罪環之以兵|{
	環音宦}
使百官臨視之太廟為賊所焚上素服向廟哭三日是日上皇發蜀郡 安慶緒走保鄴郡改鄴郡為安成府改元天成 |{
	考異曰唐歷曰改元天和薊門紀亂曰改元至成與實録年號不同紀年通譜兩存之今從實錄}
從騎不過三百步卒不過千人諸將阿史那承慶等散投常山趙郡范陽旬日間蔡希德自上黨田承嗣自頴川武令珣自南陽各帥所部兵歸之又召募河北諸郡人衆至六萬軍聲復振|{
	復扶又翻}
廣平王俶之入東京也百官受安祿山父子官者陳希烈等三百餘人皆素服悲泣請罪俶以上旨釋之尋勒赴西京己巳崔器令詣朝堂請罪|{
	此東内之朝堂也在含元殿左右左曰束朝堂右曰西朝堂朝直遥翻}
如西京百官之儀然後收繫大理京兆獄其府縣所由祗承人等受賊驅使追捕者皆收繫之|{
	所由人有所監典祗承人聽指呼給使令而已}
初汲郡甄濟有操行隐居青巖山|{
	五代志汲郡隋興縣有倉巖山隋興縣唐時當省入汲縣甄之人翻操七到翻行下孟翻}
安禄山為采訪使奏掌書記|{
	此天寶間事}
濟察祿山有異志詐得風疾舁歸家禄山反使蔡希德引行刑者二人封刀召之濟引首待刀希德以實病白祿山後安慶緒亦使人強舁至東京|{
	強其兩翻}
月餘會廣平王俶平東京濟起詣軍門上謁|{
	上時掌翻}
俶遣詣京師上命館之於三司|{
	時令三司按受賊官爵者因館濟於三司署舍使受賊官爵者羅拜之館音貫}
令受賊官爵者列拜以愧其心|{
	以愧受賊官爵者之心}
以濟為秘書郎國子司業蘇源明稱病不受禄山官上擢為考功郎中知制誥壬申上御丹鳳門下制士庶受賊官禄為賊用者令三司條件聞奏其因戰被虜或所居密近因與賊往來者皆聽自首除罪其子女為賊所汚者勿問|{
	東内端門曰丹鳳門樓曰丹鳳樓首手又翻汚烏故翻}
癸酉回紇葉護自東京還上命百官迎之於長樂驛|{
	長樂驛在滻東長樂坡}
上與宴於宣政殿|{
	自含元殿入宣政門為宣政殿東内之中朝也}
葉護奏以軍中馬少請留其兵於沙苑|{
	沙苑在馮翊渭曲李吉甫郡國圖沙苑一名沙阜在同州馮翊縣南十二里東西八十里南北三十里余靖曰唐沙苑監今之同州少詩沼翻}
自歸取馬還為陛下掃除范陽餘孽|{
	為于偽翻}
上賜而遣之 十一月廣平王俶郭子儀來自東京上勞子儀曰吾之家國由卿再造|{
	勞力到翻}
張鎬帥魯炅來瑱吳王祗李嗣業李奐五節度徇河南河東郡縣皆下之惟能元皓據北海高秀巖據大同未下|{
	能奴代翻姓也北海屬河南道大同屬河東道}
己丑以回紇葉護為司空忠義王歲遺回紇絹二萬匹|{
	遺于季翻}
使就朔方軍受之 以嚴莊為司農卿 上之在彭原也更以栗為九廟主|{
	禮虞主用桑練主用栗作栗主則埋桑主上皇幸蜀九廟之主委之賊手故彭原更以栗為之}
庚寅朝享於長樂殿|{
	長樂殿攷雍録及呂圖皆無之以下文上皇入大明宫御含元殿見百官次詣長樂殿謝九廟主則是殿亦在大明宫中也大明宫圖有長樂門則長樂殿盖在長樂門内}
丙申上皇至鳳翔從兵六百餘人|{
	從才用翻}
上皇命悉以甲兵輸郡庫上發精騎三千奉迎十二月丙午上皇至咸陽上備法駕迎於望賢宫上皇在宫南樓上釋黄著紫望樓下馬趨進拜舞於樓下上皇降樓撫上而泣上捧上皇足嗚咽不自勝上皇索黄自為上著之|{
	著陟略翻勝音升索山客翻為于偽翻}
上伏地頓首固辭上皇曰天數人心皆歸於汝使朕得保養餘齒汝之孝也上不得已受之父老在仗外歡呼且拜上令開仗|{
	車駕所在衛士立仗}
縱千餘人入謁上皇曰臣等今日復覩二聖相見死無恨矣|{
	復扶又翻}
上皇不肯居正殿|{
	此行宫正殿也}
曰此天子之位也上固請自扶上皇登殿尚食進食上品嘗而薦之|{
	品品必嘗而後進之}
丁未將發行宫上親為上皇習馬而進之上皇上皇上馬上親執鞚行數步|{
	為于偽翻鞚苦貢翻 考異曰幸蜀記云執轡鞚出宫門上皇令左右扶上馬今從實錄}
上皇止之上乘馬前引不敢當馳道上皇謂左右曰吾為天子五十年未為貴今為天子父乃貴耳左右皆呼萬歲|{
	玄宗失國得反宜痛自刻責以謝天下乃以為天子父之貴誇左右是全無心膓矣}
上皇自開遠門入大明宫|{
	開遠門長安城西面北來第一門}
御含元殿慰撫百官乃詣長樂殿謝九廟主慟哭久之|{
	樂音洛}
即日幸興慶宫遂居之上累表請避位還東宫上皇不許辛亥以禮部尚書李峴兵部侍郎呂諲為詳理使|{
	因按}


|{
	獄特置此官}
與御史大夫崔器共按陳希烈等獄峴以殿中侍御史李栖筠為詳理判官栖筠多務平恕故人皆怨諲器之刻深而峴獨得美譽 戊午上御丹鳳樓赦天下惟與安祿山同反及李林甫王鉷楊國忠子孫不在免例立廣平王俶為楚王加郭子儀司徒李光弼司空|{
	鉷戶公翻俶昌六翻 考異曰實錄光弼舊守司徒按舊傳光弼檢校司徒耳實録誤也}
自餘蜀郡靈武扈從立功之臣|{
	從才用翻}
皆進階賜爵加食邑有差李憕盧奕顔杲卿袁履謙許遠張廵張介然蔣清龎堅等皆加贈官|{
	差初加翻憕時陵翻李憕盧奕蔣清以守洛死顔杲卿袁履謙以守常山死許遠張廵以守睢陽死張介然以守滎陽死龎堅以守頴川死}
其子孫戰亡之家給復二載|{
	復方目翻除其賦役也載祖亥翻}
郡縣來載租庸三分蠲一近所改郡名官名一依故事|{
	天寶元年改兩省長官為左右相州為郡刺史為太守十一載又改吏部為文部兵部為武部刑部為憲部今皆復舊蠲圭淵翻}
以蜀郡為東京鳳翔為西京西京為中京|{
	以長安在洛陽鳳翔蜀郡太原之中故為中京}
以張良娣為淑妃立皇子南陽王係為趙王新城王僅為彭王頴川王僴為兖王東陽王侹為涇王僙為襄王倕為杞王偲為召王佋為興王侗為定王|{
	娣大計翻僴戶簡翻侹他頂翻僙戶剛翻召讀曰邵佋時昭翻侗吐公翻}
議者或罪張廵以守睢陽不去與其食人曷若全人其友人李翰為之作傳表上之|{
	睢音雖為于偽翻傳直戀翻上時掌翻下同}
以為廵以寡擊衆以弱制強保江淮以待陛下之師師至而廵死|{
	謂張鎬之師至而睢陽之城已陷三日也}
廵之功大矣而議者或罪廵以食人愚廵以守死|{
	以廵食人為廵罪守死為廵愚}
善遏惡揚錄瑕弃用臣竊痛之廵所以固守者以待諸軍之救救不至而食盡食既盡而及人乖其素志設使廵守城之初已有食人之心損數百之衆以全天下臣猶曰功過相掩况非其素志乎今廵死大難|{
	難乃旦翻}
不睹休明唯有令名是其榮祿若不時紀錄恐遠而不傳使廵生死不遇誠可悲焉臣敢撰傳一卷獻上乞編列史官衆議由是始息是後赦令無不及李憕等而程千里獨以生執賊庭不沾褒贈|{
	史言唐褒忠之典有遺恨}
甲子上皇御宣政殿以傳國寶授上上始涕泣而受之|{
	上不敢受傳國寶見二百一十八卷元載九月}
安慶緒之北走也|{
	謂自東京北走度河}
其大將北平王李歸仁及精兵曳落河同羅六州胡數萬人皆潰歸范陽所過俘掠人物無遺史思明厚為之備且遣使逆招之范陽境曳落河六州胡皆降同羅不從思明縱兵擊之同羅大敗悉奪其所掠餘衆走歸其國慶緒忌思明之強遣阿史那承慶安守忠往徵兵因密圖之判官耿仁智|{
	耿仁智盖為范陽節度判官}
說思明曰大夫崇重人莫敢言仁智願一言而死思明曰何也仁智曰大夫所以盡力於安氏者廹於凶威耳今唐室中興天子仁聖大夫誠帥所部歸之|{
	帥讀曰率}
此轉禍為福之計也禆將烏承玭亦說思明曰今唐室再造慶緒葉上露耳|{
	朝日一出葉上之露即晞故以為諭說式芮翻}
大夫奈何與之俱亡若歸欵朝廷以自湔洗易於反掌耳|{
	易以豉翻}
思明以為然承慶守忠以五千勁騎自隨|{
	考異曰舊傳云三千騎今從實録}
至范陽思明悉衆數萬逆之相距一里所使人謂承慶等曰相公及王遠至將士不勝其喜|{
	勝音升}
然邊兵怯懦懼相公之衆不敢進願弛弓以安之承慶等從之思明引承慶入内廳樂飲|{
	樂音洛}
别遣人收其甲兵諸郡兵皆給糧縱遣之願留者厚賜分隸諸營明日囚承慶等遣其將竇子昂奉表以所部十三郡及兵八萬來降|{
	十三郡范陽北平媯川密雲漁陽柳城文安河間上谷博陵勃海饒陽常山}
并帥其河東節度使高秀巖亦以所部來降乙丑子昂至京師|{
	考異曰河洛春秋乾元元年四月烏承恩受命入幽州陳禍福思明乃有表今從實録實錄曰明日遂拘承}


|{
	慶斬守忠之首以徇舊傳曰遂拘承慶斬守忠李立節之二首以徇新烏承玭傳曰思明斬承慶按實録明年二月承慶守忠遣人齎表狀歸順舊郭子儀傳明年七月破賊河上擒安守忠以獻則此際未死也蓋二入既被拘則降於思明復為之用耳}
上大喜以思明為歸義王范陽節度使|{
	考異曰河洛春秋及舊傳皆云河北節度使按安禄山為范陽節度使兼河北采訪使思明盖襲禄山舊官}


|{
	耳今從實錄}
子七人皆除顯官遣内侍李思敬與烏承恩往宣慰|{
	句斷}
使將所部兵討慶緒|{
	將即亮翻}
先是慶緒以張忠志為常山太守|{
	先悉薦翻}
思明召忠志還范陽以其將薛萼攝桓州刺史開井陘路|{
	開太原兵自井陘出常山之路}
招趙郡太守陸濟降之命其子朝義將兵五千人攝冀州刺史以其將令狐彰為博州刺史烏承恩所至宣布詔旨滄瀛安深德棣等州皆降|{
	後魏置安州治方城唐檀州即其地也唐無安州在河北或者安史以莫州文安郡為安州歟}
雖相州未下|{
	謂安慶緒據鄴也}
河北率為唐有矣|{
	因史思明降史言一時之事}
上皇加上尊號曰光天文武大聖孝感皇帝郭子儀還東都經營河北 崔器呂諲上言諸陷賊

官背國從偽準律皆應處死|{
	上時掌翻背蒲妹翻處昌呂翻}
上欲從之李峴以為賊陷兩京天子南廵人自逃生此屬皆陛下親戚或勲舊子孫今一槩以叛法處死恐乖仁恕之道且河北未平羣臣陷賊者尚多若寛之足開自新之路若盡誅是堅其附賊之心也書曰殱厥渠魁脅從罔理|{
	書胤征之辭李峴避唐諱改治為理}
諲器守文不達大體惟陛下圖之争之累日上從峴議以六等定罪重者刑之於市次賜自盡次重杖一百次三等流貶壬申斬逹奚珣等十八人於城西南獨柳樹下|{
	劉昫曰獨柳樹在長安子城西南隅}
陳希烈等七人賜自盡於大理寺應受杖者於京兆府門上欲免張均張垍死上皇曰均垍事賊皆任權要均仍為賊毁吾家事|{
	為于偽翻下垍為同}
罪不可赦上叩頭再拜曰臣非張說父子無有今日|{
	上皇之為太子也太平公主忌之東宫左右持兩端纎悉必聞於主元獻楊后方娠上皇不自安密語侍讀張說曰用事者不欲吾多子奈何命說挾劑而入上皇於曲室自煮之夢若有介而戈者環鼎三而三煮盡覆以告說說曰天命也乃止遂生帝及帝在東宫李林甫動揺者數矣均垍保護得免}
臣不能活均垍使死者有知何面目見說於九泉因俯伏流涕上皇命左右扶上起曰張垍為汝長流嶺表張均必不可活汝更勿救上泣而從命 |{
	考異曰柳珵常侍言旨云太上皇召肅宗謂曰張均弟兄皆與逆賊作權要官就中張均更與賊毁阿奴三哥家事雖犬彘之不若也其罪無赦肅宗下殿叩頭再拜曰臣比在東宫被人誣譖三度合死皆張說保護得全首領以至今日說兩男一度合死臣不能力爭儻死者有知臣亦何面目見張說於地下嗚咽俯伏太上皇命左右曰扶皇帝起乃曰與阿奴處置張垍宜長流遠惡處張均宜弃市阿奴更不要苦救這賊也肅宗掩泣奉詔按肅宗為李林甫所危時說已死乃得均垍之力均垍以說遺言盡心於肅宗耳今略取其意}
安禄山所署河南尹張萬頃獨以在賊中能保庇百姓不坐頃之有自賊中來者言唐羣臣從安慶緒在鄴者聞廣平王赦陳希烈等皆自悼恨失身賊庭及聞希烈等誅乃止上甚悔之臣光曰為人臣者策名委質有死無貳希烈等或貴為卿相或親連肺腑於承平之日無一言以規人主之失救社稷之危迎合苟容以竊富貴及四海横潰乘輿播越偷生苟免顧戀妻子媚賊稱臣為之陳力|{
	為于偽翻}
此乃屠酟之所羞犬馬之不如儻各全其首領復其官爵是謟諛之臣無往而不得計也彼顔杲卿張廵之徒世治則擯斥外方沈抑卞僚|{
	治宜吏翻沈持林翻}
世亂則委弃孤城韲粉寇手|{
	韲牋西翻}
何為善者之不幸而為惡者之幸朝廷待忠義之薄而保姦邪之厚邪至於微賤之臣廵徼之隸|{
	隸吉弟翻}
謀議不預號令不及朝聞親征之詔夕失警蹕之所|{
	事見二百十八卷至德元載}
乃復責其不能扈從不亦難哉|{
	復扶又翻從才用翻}
六等議刑斯亦可矣又何悔焉

故妃韋氏既廢為尼居禁中是歲卒|{
	韋妃廢見二百十五卷天寶六載}
置左右神武軍取元從子弟充|{
	元從子弟謂從帝馬嵬北行及自靈武還}


|{
	京師者從才用翻}
其制皆如四軍緫謂之北牙六軍|{
	左右羽林左右龍武左右神武謂之北牙六軍}
又擇善騎射者千人為殿前射生手分左右廂號曰英武軍|{
	騎奇寄翻}
升河中防禦使為節度領蒲絳等七州|{
	至德元載置河中防禦守捉蒲關使今升為節度領蒲絳隰慈晉虢同七州至蒲州 考異曰諸地理書皆云某郡乾元元年復為某州不見在何月日是歲十二月戊午赦云近日所改百官額及郡名官名一切依故事蓋此即復以郡為州之文也此頒下四方已涉明年矣故皆云乾元元年也}
分劔南為東西川節度東川領梓遂等十二州|{
	東川領梓遂綿劒龍閬普陵瀘榮資簡十二州治梓州}
又置荆灃節度領荆灃等五州夔峽節度領夔峽等五州|{
	荆南節度本領十州今分兩鎮荆灃兼領朗郢復共五州夔峽兼領涪忠萬共五州}
更安西曰鎮西|{
	更工衡翻}


乾元元年|{
	是年二月改元}
春正月戊寅上皇御宣政殿授冊加上尊號 |{
	考異曰實録戊寅玄宗御宣政殿授上傳國寶禮畢冊上加尊號上上言讓曰伏奉聖旨䁑臣典冊曰光天文武大聖孝感皇帝授傳國寶符受命寶符各一按去年十二月癸亥上已授國璽告太清宫甲子玄宗御宣政殿授上傳國璽於殿下涕泣拜受今又云授寶事似複重唐歷統紀年代記舊記皆云去年十二月授傳國璽此年五月戊寅冊尊號今從之}
上固辭大聖之號上皇不許上尊上皇曰太上至道聖皇天帝|{
	寇逆未平九廟未復而父子之間迭加徽稱此何為者也}
先是官軍既克京城|{
	先悉薦翻}
宗廟之器及府庫資財多散在民間遣使檢括頗有煩擾乙酉敕盡停之乃命京兆尹李峴安撫坊市 二月癸卯朔以殿中監李輔國兼太僕卿輔國依附張淑妃判元帥府行軍司馬勢傾朝野|{
	為輔國得權與淑妃交惡張本朝直遥翻}
安慶緒所署北海節度使能元皓舉所部來降|{
	能奴代翻降戶江翻}
以為鴻臚卿充河北招討使 丁未上御明鳳門|{
	唐會要曰至德三載改丹鳳門曰明鳳門通化門為逹禮門安上門為先天門凡坊名有安者悉改尋却如舊}
赦天下改元|{
	改元乾元}
盡免百姓今載租庸復以載為年|{
	改年為載自上皇天寶三載始復扶又翻}
庚午以安東副大都護王玄志為營州刺史充平盧節度使 三月甲戍徙楚王俶為成王 戊寅立張淑妃為皇后 鎮西北庭行營節度使李嗣業屯河内|{
	行營節度使始此}
癸巳北庭兵馬使王惟良謀作亂嗣業與禆將荔非元禮討誅之|{
	荔非虜複姓姓譜荔非西羌種隋有荔非雄涇州人}
安慶緒之北走也其平原太守王□|{
	□古限翻}
清河太守宇文寛皆殺其使者來降慶緒使其將蔡希德安太清攻拔之生擒以歸冎於鄴市凡有謀歸者|{
	冎古瓦翻歸字下當有國字}
誅及種族|{
	胡人種誅之華人族誅之種章勇翻}
乃至部曲州縣官屬連坐死者甚衆又與其羣臣歃血盟於鄴南而人心益離慶緒聞李嗣業在河内夏四月與蔡希德崔乾祐將步騎二萬涉沁水攻之|{
	沁水出沁州沁源縣東南出山而東流過河内縣北慶緒自鄴屯河内須度沁水沁七鴆翻}
不勝而還|{
	還從宣翻又如字}
癸卯以太子少師虢王巨為河南尹充東京留守 辛卯|{
	辛卯當作辛亥傳寫誤也新書肅宗紀作四月辛亥此又逸四月二字}
新主入太廟|{
	奉栗主自長樂殿入太廟}
甲寅上享太廟遂祀昊天上帝乙卯御明鳳門赦天下 五月壬午制停采訪使改黜陟使為觀察使|{
	觀察使始此貞觀初遣大使十人廵省天下諸州水旱則有廵察安撫存撫之名神龍二年以五品以上二十人為十道廵察使按舉州縣再周而代景雲二年置都督二十四人察刺史以下善惡當時以為權重難制罷之置十道按察使開元二年曰十道按察采訪處置使四年罷八年復置按察使秋冬廵視州縣二十年曰採訪處置使分十五道天寶末又兼黜陟使是年改曰觀察處置使}
張鎬性簡澹不事中要|{
	中要謂中人居權要者如李輔國之類}
聞史思明請降上言思明凶險因亂竊位力強則衆附勢奪則人離彼雖人面心如野獸難以德懷願勿假以威權又言滑州防禦使許叔冀狡猾多詐臨難必變請徵入宿衛|{
	思明叔冀後皆如鎬言滑州靈昌郡使疏吏翻難乃旦翻}
時上以寵納思明|{
	以當作已唐人多通用以已二字但於此作以文意不通}
會中使自范陽及白馬來皆言思明叔冀忠懇可信|{
	思明在范陽滑州治白馬縣漢古縣也許叔冀屯馬}
上以鎬為不切事機戊子罷為荆州防禦使以禮部尚書崔光遠為河南節度使|{
	尚辰羊翻}
張后生興王佋|{
	佋音韶}
纔數歲欲以為嗣上疑未決從容謂考功郎中知制誥李揆曰|{
	嗣祥吏翻從千容翻}
成王長且有功|{
	長知兩翻}
朕欲立為太子卿意何如揆再拜賀曰此社稷之福臣不勝大慶|{
	勝音升}
上喜曰朕意决矣庚寅立成王俶為皇太子揆玄道之玄孫也|{
	俶昌六翻李玄道武德中為天策府學士}
乙未以崔圓為太子少師李麟為少傅皆罷政事上頗好鬼神|{
	少始照翻好呼到翻}
太常少卿王璵|{
	璵音余}
專依鬼神以求媚每議禮儀多雜以巫祝俚俗上悦之以璵為中書侍郎同平章事|{
	俚音里 考異曰舊傳云三年七月今從實録}
贈故常山太守顔杲卿太子太保諡曰忠節以其子威明為太僕丞杲卿之死也|{
	顔杲郷死事見二百十七卷至德元載守式又翻}
楊國忠用張通幽之譛竟無褒贈上在鳳翔顔真卿為御史大夫泣訴於上上乃出通幽為普安太守|{
	劒州普安郡}
具奏其狀於上皇上皇杖殺通幽杲卿子泉明為王承業所留因寓居夀陽|{
	晉置夀陽縣屬樂平郡後魏廢樂平郡以夀陽縣屬太原郡九域志在太原府東一百八十里然本朝太原府已移治陽曲宋白曰夀陽縣本漢榆次縣地後魏風土記晉末山戎内侵太原之民來向山東戎即居之真君十年出徙夀陽之戶於太陵城南置夀陽縣隋開皇改夀陽為文水縣又於夀陽故城置夀陽縣即今縣是也}
為史思明所虜|{
	去年史思明攻太原因虜泉明}
裹以牛革送於范陽會安慶緒初立有赦得免思明降乃得歸求其父尸於東京得之遂并袁履謙尸棺歛以歸|{
	棺古玩翻歛力贍翻}
杲卿姊妹女及泉明之子皆流落河北真卿時為蒲州刺史使泉明往求之泉明號泣求訪哀感路人久乃得之泉明詣親故乞索|{
	號戶高翻索山客翻}
隨所得多少贖之先姑姊妹而後其子姑女為賊所掠泉明有錢二百緡欲贖己女閔其姑愁悴|{
	先悉薦翻後戶遘翻悴奏醉翻}
先贖姑女比更得錢|{
	比必利翻及也}
求其女已失所在遇羣從姊妹|{
	從才用翻}
及父時將吏袁履謙等妻子流落者皆與之歸凡五十餘家三百餘口均減資糧|{
	資糧則均分之其或有不足則减常數而均之}
一如親戚至蒲州真卿悉加贍給久之隨其所適而資送之袁履謙妻疑履謙衣衾儉薄發棺視之與杲卿無異乃始慙服|{
	顔杲卿之忠節固照映千古而其子之孝義亦非人所及也}
六月己酉立太一壇於南郊之東|{
	漢武帝始祀太一至唐復祀之盖參用九宫貴神之說項安世曰中宫天極一星其神太一列宿之中最尊所臨之方則嘉應洊臻漢武帝始祠之}
從王璵之請也上嘗不豫卜云山川為祟|{
	祟雖遂翻神禍也}
璵請遣中使與女巫乘驛分禱天下名山大川巫恃勢所過煩擾州縣干求受贓黄州有巫盛年美色從無賴少年數十為蠧尤甚|{
	使疏吏翻少始照翻}
至黃州宿於驛舍刺史左震晨至驛門扃鏁不可啟|{
	扃古營翻鏁蘇果翻}
震怒破鏁而入曳巫於堦下斬之所從少年悉斃之籍其贓數十萬具以狀聞且請以其贓代貧民租遣中使還京師上無以罪也 以開府儀同三司李嗣業為懷州刺史充鎮西北庭行營節度使|{
	李嗣業以鎮西北庭兵屯懷州就用為刺史征調以給軍嗣祥吏翻}
山人韓頴改造新歷丁巳初行頴歷|{
	時韓頴上言大衍歷或誤帝疑之以頴直司天臺損益其術每節增二日更名至德歷}
戊午敕兩京陷賊官三司推䆒未畢者皆釋之貶降者續處分|{
	去年十二月始命三司推究陷賊官處昌呂翻分扶問翻}
太子少師房琯既失職|{
	謂罷相也}
頗怏怏多稱疾不朝|{
	怏於}


|{
	兩翻朝直遥翻}
而賓客朝夕盈門其黨為之揚言於朝云琯有文武才宜大用上聞而惡之下制數琯罪貶州刺史|{
	為于偽翻惡烏路翻數所具翻又所主翻}
前祭酒劉秩貶閬州刺史京兆尹嚴武貶巴州刺史皆琯黨也|{
	閬州閬中郡巴州清化郡漢巴郡宕渠縣地閬音浪}
初史思明以列將事平盧軍使烏知義 |{
	考異曰舊傳知義為節}


|{
	度使按安禄山始為平盧節度使舊傳誤也}
知義善待之知義子承恩為信都太守以郡降思明|{
	事見上卷至德元載}
思明思舊恩而全之及安慶緒敗承恩勸思明降唐|{
	去年十二月事}
李光弼以思明終當叛亂而承恩為思明所親信陰使圖之又勸上以承恩為范陽節度副使賜阿史那承慶鐵劵令共圖思明上從之承恩多以私財募部曲又數衣婦人服詣諸將營說誘之|{
	數所角翻衣於既翻說輸芮翻誘音酉}
諸將以白思明思明疑未察會承恩入京師上使内侍李思敬與之俱至范陽宣慰承恩既宣旨思明留承恩館於府中|{
	按經典釋文館古玩翻}
帷其床伏二人於床下承恩少子在范陽思明使省其父|{
	少詩照翻省悉景翻思明雖伏二人以察承恩然不使其子與父共處則謀無自而露姦雄之智數固非人所及也}
夜中承恩密謂其子曰吾受命除此逆胡當以吾為節度使二人於牀下大呼而出|{
	呼火故翻}
思明乃執承恩索其装囊|{
	凡行者之裝盛以囊槖故曰裝囊有底曰囊無底曰槖索山客翻}
得鐵劵及光弼牒牒云承慶事成則付鐵劵不然不可付也又得簿書數百紙皆先從思明反者將士名|{
	烏承恩持鉄劵入不測之虜使阿史那承慶之事不成承恩其能奉鉄劵以還天子乎使思明果授首則宜宥其同惡而先籍其姓名果能悉誅之乎余謂李光弼之明智必不為此盖思明因承恩言偽為此牒抗表以罪狀光弼又偽為簿書籍將士姓名以激怒之使與已同反而無他志}
思明責之曰我何負於汝而為此承恩謝曰死罪此皆李光弼之謀也思明乃集將佐吏民西向大哭曰臣以十三萬衆降朝廷何負陛下而欲殺臣遂榜殺承恩父子|{
	榜音彭 考異曰唐歷舊傳皆云四月殺承恩今據河洛春秋四月始為節度副使六月死}
連坐死者二百餘人承恩弟承玼走免|{
	玼音此又且禮翻}
思明囚思敬表上其狀|{
	上時掌翻}
上遣中使慰諭思明曰此非朝廷與光弼之意皆承恩所為殺之甚善會三司議陷賊官罪狀至范陽思明謂諸將曰陳希烈輩皆朝廷大臣上皇自弃之幸蜀今猶不免於死況吾屬本從安禄山反乎|{
	思明又以此激怒其將士}
諸將請思明表求誅光弼思明從之命判官耿仁智與其僚張不矜為表云陛下不為臣誅光弼|{
	不為于偽翻}
臣當自引兵就太原誅之不矜草表以示思明及將入函|{
	用表皆函封}
仁智悉削去之寫表者以白思明思明命執二人斬之仁智事思明久思明憐欲活之復召入|{
	去羌呂翻復扶又翻}
謂曰我任使汝垂三十年今日非我負汝仁智大呼曰人生會有一死得盡忠義死之善者也今從大夫反不過延歲月豈若速死之愈乎思明怒亂捶之腦流於地|{
	史言耿仁智去逆從順以死全節呼火故翻}
烏承玼奔太原李光弼表為昌化郡王充石嶺軍使|{
	石嶺軍在忻州秀容縣}
秋七月丙戌初鑄當十大錢文曰乾元重寶|{
	乾元錢徑一寸每緡重十斤與開元通寶並行}
從御史中丞第五琦之謀也丁亥冊命回紇可汗曰英武威遠毗伽闕可汗以上

幼女寧國公主妻之|{
	妻七細翻}
以殿中監漢中王瑀為冊禮使右司郎中李巽副之命左僕射裴冕送公主至境上戊子又以司勲員外郎鮮于叔明為瑀副叔明仲通之弟也|{
	天寶中鮮于仲通黨附楊國忠致位通顯}
甲子上送寜國公主至咸陽公主辭訣曰國家事重死且無恨上流涕而還瑀等至回紇牙帳可汗衣赭胡㡌|{
	衣於既翻}
坐帳中榻上儀衛甚盛引瑀等立於帳外瑀不拜而立可汗曰我與天可汗兩國之君君臣有禮何得不拜瑀與叔明對曰曏者唐與諸國為昏皆以宗室女為公主今天子以可汗有功自以所生女妻可汗|{
	妻七細翻}
恩禮至重可汗奈何以子壻傲婦翁坐榻上受冊命邪可汗改容起受冊命明日立公主為可敦|{
	自突厥有國以來可汗號其正室曰可賀敦}
舉國皆喜 乙未郭子儀入朝 |{
	考異曰實錄郭子儀擒逆賊將安太清送闕下按上元元年李光弼拔懷州始擒太清實録誤也唐歷本紀等皆無之舊子儀傳七月破賊河上擒安守忠以獻諸書亦無之今不取}
八月壬寅以青登等五州節度使許叔冀為滑濮等六州節度使 |{
	考異曰實録云青徐等五州節度使季廣琛青登等五州節度使許叔冀按青州豈可屬兩節度又廣琛先為荆州長史今年五月為右常侍九月討安慶緒時實錄稱鄭蔡節度使汾陽家傳稱淮西荆澧舊紀稱荆州未嘗鎮青徐實錄於此稱青徐恐誤也余按新書方鎮表至德元載置青密節度使領北海高密東牟東萊四郡乾元元年青密節度增領滑濮二州青密節度即前所云北海節度也領青密登萊四州增領滑濮是為六州節度使若以青登五州增滑濮二州則七州矣其數不合}
庚戍李光弼入朝丙辰以郭子儀為中書令光弼為侍中丁巳子儀詣行營 回紇遣其臣骨啜特勒及帝德將驍騎三千助討安慶緒上命朔方左武鋒使僕固懷恩領之 九月庚午朔以右羽林大將軍趙玼為蒲同虢三州節度使|{
	去年置河中節度使領蒲絳等七州今趙玼節度䈬同虢三州而已蓋兵興之際分命節帥以扼險要其所統之增减離合隨時制宜耳}
丙子招討党項使王仲昇斬党項酋長拓拔戎德傳首|{
	貞觀以後吐蕃浸盛党項拓拔諸部畏偪請内徙詔慶州置静邊軍州處之又置芳池都督府於慶州懷安縣界管小州十以處党項野利氏部落至德以來中國亂党項因寇邠寧二州}
安慶緒之初至鄴也雖枝黨離析猶據七郡六十餘城|{
	汲鄴趙魏平原清河博平凡七郡}
甲兵資糧豐備慶緒不親政事專以繕臺沼樓船酣飲為事其大臣高尚張通儒等争權不叶無復綱紀蔡希德有才略部兵精銳而性剛好直言通儒譖而殺之|{
	復扶又翻好呼到翻下好殺同考異曰河洛春秋十月蔡希德有密欵歸國將襲殺慶緒以為内應左右泄之慶緒斬希德於鄴中又曰慶緒既殺希德始有土崩之兆矣薊門紀亂史思明常畏希德自知謀策果斷英武皆不及之時希德在相州為慶緒竭節展効思明未敢顯背無何希德為慶緒所殺思明初聞驚疑不信及知其實大喜見於顔色焉今從實録}
麾下數千人皆逃散諸將怨怒不為用以崔乾祐為天下兵馬使緫中外兵乾祐愎戾好殺|{
	將即亮翻愎弼力翻狠也}
士卒不附庚寅命朔方郭子儀淮西魯炅興平李奐滑濮許叔冀鎮西北庭李嗣業鄭蔡季廣琛河南崔光遠七節度使及平盧兵馬使董秦將步騎二十萬討慶緒|{
	炅火迴翻濮博木翻嗣祥吏翻琛丑林翻將即亮翻又音如字騎奇寄翻}
又命河東李光弼關内澤潞王思禮|{
	王思禮先為關内節度使時兼領澤潞節度使鎮潞州}
二節度使將所部兵助之|{
	考異曰實録有李奐無崔光遠而云凡九節度汾陽家傳有光遠無奐又有河東兵馬使薛兼訓盖實録脱光遠汾陽傳脱奐名耳兼訓盖光弼禆將光弼未至間先遣赴鄴城也汾陽傳又以炅為襄鄧廣琛為淮西荆澧舊本紀廣琛為荆州今從實録汾陽傳又云公九月十二日出洛師涉河而東今從實録涉庚二十一日也 余按涉庚當作庚寅}
上以子儀光弼皆元勲難相統屬故不置元帥|{
	諸軍並行步騎數十萬而不置元帥號令不一所以有安陽之敗}
但以宦官開府儀同三司魚朝恩為觀軍容宣慰處置使|{
	處昌呂翻}
觀軍容之名自此始 癸巳廣州奏大食波斯圍州城|{
	廣州治南海縣本漢番禺縣}
刺史韋利見踰城走二國兵掠倉庫焚廬舍浮海而去 冬十月甲辰冊太子 |{
	考異曰實録云可大赦天下頃者頻興大典累洽殊私率土之間屢經蕩滌猶慮近者或滯狴牢其天下見禁囚徒已下罪一切放免按既云大赦則死罪皆免豈有但免徒以下罪邪恐可大赦天下是衍字耳今不書赦}
更名曰豫|{
	初太子生之歲豫州獻嘉禾於是以為祥更名豫更工衡翻}
自中興以來羣下無復賜物|{
	復扶又翻}
至是始有新鑄大錢|{
	乾元重寶錢也}
百官六軍霑賚有差 郭子儀引兵自杏園濟河東至獲嘉|{
	九域志衛州汲縣有杏園鎮獲嘉縣本汲縣之新中鄉漢武帝行幸至此聞獲呂嘉因置獲嘉縣唐屬懷州九域志獲嘉縣在衛州西九十里}
破安太清斬首四千級捕虜五百人太清走保衛州子儀進圍之丙午遣使告捷魯炅自陽武濟季廣琛崔光遠自酸棗濟|{
	陽武縣武德四年置於故原武城屬鄭州}
與李嗣業兵皆會子儀於衛州慶緒悉舉鄴中之衆七萬救衛州分三軍以崔乾祐將上軍田承嗣將下軍慶緒自帥中軍子儀使善射者三千人伏於壘垣之内令曰我退賊必逐我汝乃登壘鼔譟而射之既而與慶緒戰偽退賊逐之至壘下伏兵起射之|{
	射而亦翻}
矢如雨注賊還走子儀復引兵逐之|{
	復扶又翻}
慶緒大敗獲其弟慶和殺之遂拔衛州慶緒走子儀等追之至鄴許叔冀董秦王思禮及河東兵馬使薛兼訓皆引兵繼至慶緒收餘兵拒戰於愁思岡|{
	愁思岡在鄴城西據歐史在相州湯隂縣薛居正曰湯隂縣界有一岡上人謂之愁死岡 考異曰汾陽家傳十月五日戰愁岡據實録癸丑子儀破賊擒安慶和癸丑十四日也盖捷奏始到}
又敗前後斬首三萬級捕虜千人慶緒乃入城固守子儀等圍之慶緒窘急遣薛嵩求救於史思明且請以位讓之思明發范陽兵十三萬欲救鄴觀望未敢進先遣李歸仁將步騎一萬軍于陽|{
	磁州治陽南至鄴城六十里}
遥為慶緒聲勢 甲寅上皇幸華清宫十一月丁丑還京師 崔光遠拔魏州|{
	魏州治漢元城縣郭下又置貴鄉縣與元城為二縣}
丙戍以前兵部侍郎蕭華為魏州防禦使會史思明分軍為三一出邢洺一出冀貝一自洹水趣魏州|{
	洹水縣漢長樂縣地魏郡國志曰周建德六年分臨漳縣東北置洹水縣在魏州西趣七喻翻}
郭子儀奏以崔光遠代華十二月癸卯敇以光遠領魏州刺史 甲辰置浙江西道節度使領蘇潤等十州以昇州刺史韋黃裳為之|{
	浙西道節度使兼江寧軍使領昇潤宣歙饒江蘇常杭湖十州治昇州}
庚戌置浙江東道節度使領越睦等八州以戶部尚書李峘為之|{
	浙東道節度使領越睦衢婺台明處温八州治越州}
兼淮南節度使|{
	此宜參考下卷上元元年都統李峘注}
己未羣臣請上尊號曰乾元大聖光天文武孝感皇帝許之 史思明乘崔光遠初至引兵大下光遠使將軍李處崟拒之|{
	崟魚金翻}
賊勢盛處崟連戰不利還趣城|{
	趣七喻翻}
賊追至城下揚言曰處崟召我來何為不出光遠信之腰斬處崟處崟驍將衆所恃既死衆無鬬志|{
	姚聳夫若在未必能為宋保守河南而聳夫之死宋人惜之李處崟若在未必能為唐保守魏州而處崟之死唐人惜之以兩敵相持而自戮鬭將乃自翦其手足也}
光遠脫身走還汴州丁卯思明陷魏州所殺三萬人 平盧節度使王玄志薨上遣中使往撫將士且就察軍中所欲立者授以旌節高麗人李懷玉為禆將殺玄志之子推侯希逸為平盧軍使希逸之母懷玉姑也故懷玉立之|{
	立侯希逸者李懷玉而逐侯希逸者亦李懷玉也懷玉後賜名正巳}
朝廷因以希逸為節度副使節度使由軍士廢立自此始

臣光曰夫民生有欲無主則亂|{
	書仲虺之誥之言}
是故聖人制禮以治之|{
	治直之翻}
自天子諸侯至於卿大夫士庶人尊卑有分|{
	分扶問翻}
大小有倫若綱條之相維|{
	書說命曰若綱在綱有條而不紊}
臂指之相使|{
	賈誼曰如身之使臂臂之使指莫不制從}
是以民服事其上而下無覬覦|{
	覬音冀覦音俞}
其在周易上天下澤履象曰君子以辨上下定民志此之謂也凡人君所以能有其臣民者以八柄存乎己也|{
	周禮王以八柄馭羣臣一曰爵以馭其貴二曰禄以馭其富三曰予以馭其幸四曰置以馭其行五曰生以馭其福六曰奪以馭其貧七曰廢以馭其罪八曰誅以馭其過}
苟或捨之則彼此之勢均何以使其下哉肅宗遭唐中衰幸而復國是宜正上下之禮以綱紀四方而偷取一時之安不思永久之患彼命將帥統藩維國之大事也乃委一介之使徇行伍之情|{
	行戶剛翻}
無問賢不肖惟其所欲與者則授之自是之後積習為常君臣循守以為得策謂之姑息|{
	姑且也息安也且求目前之安也}
乃至偏禆士卒殺逐主帥亦不治其罪因以其位任授之然則爵禄廢置殺生予奪|{
	此即周禮所謂八柄也治直之翻予讀曰與}
皆不出於上而出於下亂之生也庸有極乎且夫有國家者賞善而誅惡故為善者勸為惡者懲彼為人下而殺逐其上惡孰大焉乃使之擁旄秉鉞師長一方|{
	長知兩翻}
是賞之也賞以勸惡惡其何所不至乎書云遠乃猷|{
	書康誥之言猷謀也}
詩云猷之未遠是謂大諫|{
	詩大雅板之辭}
孔子曰人無遠慮必有近憂|{
	見論語}
為天下之政而專事姑息其憂患可勝校乎|{
	勝音升}
由是為下者常眄眄焉伺其上|{
	眄眠見翻目徧合而袤視也}
苟得閒則攻而族之為上者常惴惴焉畏其下苟得閒則掩而屠之|{
	二語曲盡唐末藩鎮將卒之情狀問古莧翻惴之睡翻憂懼貌}
争務先發以逞其志非有相保養為俱利久存之計也如是而求天下之安其可得乎迹其厲階肇於此矣|{
	言其禍肇於命侯希逸帥平盧也毛萇曰厲惡也鄭氏曰犯政為惡曰厲}
盖古者治軍必本於禮故晉文公城濮之戰見其師少長有禮知其可用|{
	左傳晉楚戰于城濮晉侯登有莘之虛以觀師曰少長有禮其可用也遂戰楚師敗績治直之翻下同少詩照翻長知兩翻}
今唐治軍而不顧禮使士卒得以陵偏禆偏禆得以陵將帥則將帥之陵天子自然之勢也|{
	賈誼亷陛之論正此意}
由是禍亂繼起兵革不息民墜塗炭無所控訴凡二百餘年然後大宋受命太祖始制軍法使以階級相承小有違犯咸伏斧質是以上下有叙令行禁止四征不庭|{
	庭直也不庭諸侯之不直者近世儒者以不朝為不庭謂其不來庭也}
無思不服宇内乂安兆民允殖以迄于今皆由治軍以禮故也豈非詒謀之遠哉

是歲置振武節度使領鎮北大都護府麟勝二州|{
	鎮北大都護府領大同長寧二縣振武節度使治單于都護府因舊振武軍而建節鎮兼押蕃落使宋白曰振武軍舊為單于都護府即漢定襄郡之盛樂縣也在隂山之陽黄河之北後魏所都盛樂是也唐平突厥於此置雲中都督府麟德三年改為單于大都護府至德後振武節度治焉}
又置陜虢華及豫許汝二節度使安南經略使為節度使領交陸等十一州|{
	安南節度使領交陸峯愛驩長福禄芝武莪演武安十一州治交州宋白曰陸州玉山郡本玉州上元二年改為陸州以州界有陸水為名}
吐蕃陷河源軍

資治通鑑卷二百二十
