<!DOCTYPE html PUBLIC "-//W3C//DTD XHTML 1.0 Transitional//EN" "http://www.w3.org/TR/xhtml1/DTD/xhtml1-transitional.dtd">
<html xmlns="http://www.w3.org/1999/xhtml">
<head>
<meta http-equiv="Content-Type" content="text/html; charset=utf-8" />
<meta http-equiv="X-UA-Compatible" content="IE=Edge,chrome=1">
<title>資治通鑒_279-資治通鑑卷二百七十八_279-資治通鑑卷二百七十八</title>
<meta name="Keywords" content="資治通鑒_279-資治通鑑卷二百七十八_279-資治通鑑卷二百七十八">
<meta name="Description" content="資治通鑒_279-資治通鑑卷二百七十八_279-資治通鑑卷二百七十八">
<meta http-equiv="Cache-Control" content="no-transform" />
<meta http-equiv="Cache-Control" content="no-siteapp" />
<link href="/img/style.css" rel="stylesheet" type="text/css" />
<script src="/img/m.js?2020"></script> 
</head>
<body>
 <div class="ClassNavi">
<a  href="/24shi/">二十四史</a> | <a href="/SiKuQuanShu/">四库全书</a> | <a href="http://www.guoxuedashi.com/gjtsjc/"><font  color="#FF0000">古今图书集成</font></a> | <a href="/renwu/">历史人物</a> | <a href="/ShuoWenJieZi/"><font  color="#FF0000">说文解字</a></font> | <a href="/chengyu/">成语词典</a> | <a  target="_blank"  href="http://www.guoxuedashi.com/jgwhj/"><font  color="#FF0000">甲骨文合集</font></a> | <a href="/yzjwjc/"><font  color="#FF0000">殷周金文集成</font></a> | <a href="/xiangxingzi/"><font color="#0000FF">象形字典</font></a> | <a href="/13jing/"><font  color="#FF0000">十三经索引</font></a> | <a href="/zixing/"><font  color="#FF0000">字体转换器</font></a> | <a href="/zidian/xz/"><font color="#0000FF">篆书识别</font></a> | <a href="/jinfanyi/">近义反义词</a> | <a href="/duilian/">对联大全</a> | <a href="/jiapu/"><font  color="#0000FF">家谱族谱查询</font></a> | <a href="http://www.guoxuemi.com/hafo/" target="_blank" ><font color="#FF0000">哈佛古籍</font></a> 
</div>

 <!-- 头部导航开始 -->
<div class="w1180 head clearfix">
  <div class="head_logo l"><a title="国学大师官网" href="http://www.guoxuedashi.com" target="_blank"></a></div>
  <div class="head_sr l">
  <div id="head1">
  
  <a href="http://www.guoxuedashi.com/zidian/bujian/" target="_blank" ><img src="http://www.guoxuedashi.com/img/top1.gif" width="88" height="60" border="0" title="部件查字,支持20万汉字"></a>


<a href="http://www.guoxuedashi.com/help/yingpan.php" target="_blank"><img src="http://www.guoxuedashi.com/img/top230.gif" width="600" height="62" border="0" ></a>


  </div>
  <div id="head3"><a href="javascript:" onClick="javascript:window.external.AddFavorite(window.location.href,document.title);">添加收藏</a>
  <br><a href="/help/setie.php">搜索引擎</a>
  <br><a href="/help/zanzhu.php">赞助本站</a></div>
  <div id="head2">
 <a href="http://www.guoxuemi.com/" target="_blank"><img src="http://www.guoxuedashi.com/img/guoxuemi.gif" width="95" height="62" border="0" style="margin-left:2px;" title="国学迷"></a>
  

  </div>
</div>
  <div class="clear"></div>
  <div class="head_nav">
  <p><a href="/">首页</a> | <a href="/ShuKu/">国学书库</a> | <a href="/guji/">影印古籍</a> | <a href="/shici/">诗词宝典</a> | <a   href="/SiKuQuanShu/gxjx.php">精选</a> <b>|</b> <a href="/zidian/">汉语字典</a> | <a href="/hydcd/">汉语词典</a> | <a href="http://www.guoxuedashi.com/zidian/bujian/"><font  color="#CC0066">部件查字</font></a> | <a href="http://www.sfds.cn/"><font  color="#CC0066">书法大师</font></a> | <a href="/jgwhj/">甲骨文</a> <b>|</b> <a href="/b/4/"><font  color="#CC0066">解密</font></a> | <a href="/renwu/">历史人物</a> | <a href="/diangu/">历史典故</a> | <a href="/xingshi/">姓氏</a> | <a href="/minzu/">民族</a> <b>|</b> <a href="/mz/"><font  color="#CC0066">世界名著</font></a> | <a href="/download/">软件下载</a>
</p>
<p><a href="/b/"><font  color="#CC0066">历史</font></a> | <a href="http://skqs.guoxuedashi.com/" target="_blank">四库全书</a> |  <a href="http://www.guoxuedashi.com/search/" target="_blank"><font  color="#CC0066">全文检索</font></a> | <a href="http://www.guoxuedashi.com/shumu/">古籍书目</a> | <a   href="/24shi/">正史</a> <b>|</b> <a href="/chengyu/">成语词典</a> | <a href="/kangxi/" title="康熙字典">康熙字典</a> | <a href="/ShuoWenJieZi/">说文解字</a> | <a href="/zixing/yanbian/">字形演变</a> | <a href="/yzjwjc/">金 文</a> <b>|</b>  <a href="/shijian/nian-hao/">年号</a> | <a href="/diming/">历史地名</a> | <a href="/shijian/">历史事件</a> | <a href="/guanzhi/">官职</a> | <a href="/lishi/">知识</a> <b>|</b> <a href="/zhongyi/">中医中药</a> | <a href="http://www.guoxuedashi.com/forum/">留言反馈</a>
</p>
  </div>
</div>
<!-- 头部导航END --> 
<!-- 内容区开始 --> 
<div class="w1180 clearfix">
  <div class="info l">
   
<div class="clearfix" style="background:#f5faff;">
<script src='http://www.guoxuedashi.com/img/headersou.js'></script>

</div>
  <div class="info_tree"><a href="http://www.guoxuedashi.com">首页</a> > <a href="/SiKuQuanShu/fanti/">四库全书</a>
 > <h1>资治通鉴</h1> <!--         下载:【右键另存为】即可 --></div>
  <div class="info_content zj clearfix">
  
<div class="info_txt clearfix" id="show">
<center style="font-size:24px;">279-資治通鑑卷二百七十八</center>
    資治通鑑卷二百七十八 宋 司馬光 撰<br />
<br />
  胡三省 音註<br />
<br />
  後唐紀七【起玄黓執徐七月盡閼逢敦牂閩正月凡一年有奇】<br />
<br />
  明宗聖德和武欽孝皇帝下<br />
<br />
  長興三年秋七月朔朔方奏夏州党項入寇擊敗之【夏戶雅翻敗補邁翻】追至賀蘭山【賀蘭山在靈州保靜縣】 己丑加鎮海鎮東軍節度使錢元瓘守中書令 庚寅李存瓌至成都【是年六月遣李存瓌諭孟知祥事始見上卷】孟知祥拜泣受詔【孟知祥之拜泣豈其本心之誠然邪】 武安靜江節度使馬希聲以湖南比年大旱命閉南嶽及境内諸神祠門【比毗至翻下比者同舊以霍山為南嶽今灊中天桂山是也蓋漢武帝以衡山遐遠遂徙南嶽於灊山耳至唐復以衡山為南嶽】竟不雨辛卯希聲卒六軍使袁詮潘約等迎鎮南節度使希範於朗州而立之【詮且袁翻鎮南軍洪州時屬吳馬希範領節耳希範字寶規殷第四子】 乙未孟知祥遣李存瓌還【還從宣翻又如字】上表謝罪且告福慶公主之喪【是年春正月主卒】自是復稱藩【復扶又翻】 庚子以西京留守同平章事李從珂為鳳翔節度使【為李從珂自鳳翔奪嫡張本】 廢武興軍復以鳳興文三州隸山南西道【鳳興文本山南西道巡屬唐末始分鳳州置感義軍尋廢前蜀王氏復置武興軍今廢之州還舊屬】 丁未以門下侍郎同平章事趙鳳同平章事充安國節度使 八月庚申馬希範至長沙辛酉襲位 甲子孟知祥令李昊為武泰趙季良等五留後草表【為于偽翻下同】請以知祥為蜀王行墨制仍自求旌節昊曰比者諸將攻取方鎮即有其地【比毗至翻謂李仁罕克遂州即為武信留後趙廷隱克梓州遂爭東川也】今又自求節鉞及明公封爵然則輕重之權皆在羣下矣借使明公自請豈不可邪知祥大悟更令昊為己草表請行墨制補兩川刺史已下【更工衡翻】又表請以季良等五留後為節度使【武泰留後趙季良武信留後李仁罕保寧留後趙廷隱寧江留後張鄴昭武留後李肇】初安重誨欲圖兩川自知祥殺李嚴【見二百七十五卷天成二年】每除刺史皆以東兵衛送之小州不減五百人夏魯奇李仁矩武䖍裕各數千人皆以牙隊為名【按天成二年李敬周為武信留後四年使節度使夏魯奇治遂州城魯奇蓋三年四年間至遂州也李仁矩鎮閬州武䖍裕刺綿州見上卷天成四年】及知祥克遂閬利夔黔梓六鎮得東兵無慮三萬人恐朝廷徵還表請其妻子 吳徐知誥廣金陵城周圍二十里【徐温先已築金陵今知誥復廣之將以貽子孫也】 初契丹既強寇抄盧龍諸州皆徧【抄楚交翻】幽州城門之外虜騎充斥每自涿州運糧入幽州虜多伏兵於閻溝掠取之【據水經漢涿郡故安縣有閻鄉其西山則易水所出也歐史作鹽溝】及趙德鈞為節度使城閻溝而戍之為良鄉縣【良鄉漢古縣趙德鈞移之于閻溝耳匈奴須知閻溝縣北至燕六十里古良鄉空城南至涿州四十里蓋契丹得燕之後改良鄉縣為閻溝縣而所謂古良鄉空城即趙德鈞未移縣之前古城也】糧道稍通幽州東十里之外人不敢樵牧德鈞於州東五十里城潞縣而戍之【潞漢古縣唐屬幽州匈奴須知潞縣東二里有潞河自潞縣西至燕六十里】近州之民始得稼穡至是又於州東北百餘里城三河縣以通薊州運路【唐開元四年分潞縣置三河縣屬薊州匈奴須知三河縣西至燕一百七十里薊州西至三河縣七十里】虜騎來爭德鈞擊却之九月庚辰朔奏城三河畢邊人賴之 壬午以鎮南節度使馬希範為武安節度使兼侍中【馬希範以領鎮南節自朗州入嗣今使為武安節度使嗣封楚王之漸也】孟知祥命其子仁贊攝行軍司馬兼都總轄兩川牙内馬步都軍事 冬十月己酉朔帝復遣李存瓌如成都【是年七月李存瓌還自成都今復遣之復扶又翻下不復同】凡劍南自節度使刺史以下官聽知祥差署訖奏聞朝廷更不除人唯不遣戍兵妻子然其兵亦不復徵也 秦王從榮喜為詩聚浮華之士高輦等於幕府與相唱和【喜許記翻下同和戶卧翻】頗自矜伐每置酒輒令僚屬賦詩有不如意者面毁裂抵棄壬子從榮入謁帝語之曰吾雖不知書然喜聞儒生講經義開益人智思【語牛倨翻思相吏翻】吾見莊宗好為詩將家子文非素習徒取人竊笑汝勿效也【明宗之誨其子可謂名言好呼到翻將即亮翻】丙辰幽州奏契丹屯納喇泊【時幽州有備契丹寇掠不得其志契丹主西徙】<br />
<br />
  【横帳居納喇泊出寇雲朔之間薛史本紀是年十一月雲州奏契丹主在黑榆林南納喇泊治造攻城之具是後石敬瑭鎮河東因契丹部落近在雲應遂資其兵力以取中國而燕雲十六州之地遂皆為北方引弓之民納奴葛翻喇來達翻】 前彰義節度使李金全屢獻馬【李金全先嘗鎮涇州】上不受曰卿在鎮為治何如勿但以獻馬為事【唐明宗雖出於胡人斯言也君人之言也治直吏翻】金全吐谷渾人也 壬申大理少卿康澄上書曰臣聞童謡非禍福之本妖祥豈隆替之源故雊雉升鼎而桑榖生朝不能止殷宗之盛【雊古候翻殷王太戊時亳有祥桑榖共生於朝武丁祭成湯有飛雉升鼎耳而雊二君懼而修德殷道復興太戊廟號中宗武丁廟號高宗朝直遥翻】神馬長嘶而玉龜告兆不能延晉祚之長【晉懷帝永嘉六年二月神馬嘶南城門魏明帝時張掖柳谷水涌有石馬石牛石龜之祥人以為晉興應之】是知國家有不足懼者五有深可畏者六隂陽不調不足懼三辰失行不足懼小人譌言不足懼山崩川涸不足懼蟊賊傷稼不足懼【蟊莫侯翻食根曰蟊食節曰賊皆害稼者也】賢人藏匿深可畏四民遷業深可畏上下相徇深可畏亷恥道消深可畏毁譽亂真深可畏【譽音余】直言蔑聞深可畏不足懼者願陛下存而勿論深可畏者願陛下修而靡忒【康澄所謂不足懼非果不足懼也直言人事之不得其可畏有甚于所懼者然其詞氣之間抑揚太過將使人君忽于變異災傷而不知警省非篤論也】優詔奬之 秦王從榮為人鷹視輕佻峻急【鷹視者如飛鷹欲攫俯而側目視物佻土雕翻】既判六軍諸衛事復參朝政【復扶又翻】多驕縱不法初安重誨為樞密使上專屬任之【屬之欲翻】從榮及宋王從厚自襁褓與之親狎雖典兵常為重誨所制畏事之重誨死【誅安重誨見上卷二年】王淑妃與宣徽使孟漢瓊宣傳帝命范延光趙延壽為樞密使從榮皆輕侮之河陽節度使同平章事石敬瑭兼六軍諸衛副使其妻永寧公主與從榮異母素相憎疾【明宗諸子史皆不載其母誰氏惟許王從益為王淑妃所子是時尚幼外此子女之年長者皆微時所生也】從榮以從厚聲名出已右尤忌之【事始見二百七十六卷天成三年】從厚善以卑弱奉之故嫌隙不外見【見賢遍翻】石敬瑭不欲與從榮共事【從榮判六軍諸衛事石敬瑭為副使是共事也】常思外補以避之范延光趙延壽亦慮及禍屢辭機要請與舊臣迭為之上不許會契丹欲入寇上命擇帥臣鎮河東延光延壽皆曰當今帥臣可往者獨石敬瑭康義誠耳【康義誠起代北事晉王及莊宗及帝三世在兵間不聞有功但以鄴都兵亂之時贊帝舉兵南向為功耳帥所類翻下同】敬瑭亦願行上即命除之既受詔不落六軍副使敬瑭復辭【復扶又翻】上乃以宣徽使朱弘昭知山南東道代義誠詣闕【康義誠時為山南東道節度使今召令詣闕命朱弘昭往知節度事以代之未正授以旌節也】 十一月辛巳以三司使孟鵠為忠武節度使以忠武節度使馮贇充宣徽南院使判三司鵠本刀筆吏與范延光鄉里厚善【范延光相州臨漳人孟鵠魏州人相魏隣接言二人居鄉里時相與厚善】數年間引擢至節度使上雖知其太速然不能違也 乙酉上以胡寇浸逼北邊命趣議河東帥【趣讀曰促下同】石敬瑭欲之而范延光趙延壽欲用康義誠議久不决權樞密直學士李崧以為非石太尉不可延光曰僕亦累奏用之上欲留之宿衛耳會上遣中使趣之衆乃從崧議丁亥以石敬瑭為北京留守河東節度使兼大同振武彰國威塞等軍蕃漢馬步總管【唐末移大同軍於雲州振武軍於朔州帝應州人即位置彰國軍於應州以興唐軍為寰州隸之莊宗同光元年置威塞軍於新州以媯儒武三州隸之四軍皆節鎮也】加兼侍中【為石敬瑭以河東倚契丹之援而得中國張本】 己丑加樞密使趙延壽同平章事吳以諸道都統徐知誥為大丞相太師加領得勝節度使【得勝當作德勝吳之先王楊行密起於廬州故因置德勝節度於廬州言以德而勝也】知誥辭丞相太師 大同節度使張敬達聚兵要害契丹竟不敢南下而還【按薛史時契丹帥族帳自黑榆林納喇泊至默音泊云借漢界水草張敬達聚兵遏其衝要虜竟不敢南牧還從宣翻又如字】敬達代州人也 蔚州刺史張彦超本沙陁人嘗為帝養子與石敬瑭有隙聞敬瑭為總管舉城附於契丹契丹以為大同節度使【蔚紆勿翻】 石敬瑭至晉陽以部將劉知遠周瓌為都押衙委以心腹軍事委知遠【為劉知遠為石敬瑭佐命又以是而基漢業張本】帑藏委瓌【帑它朗翻藏徂浪翻】瓌晉陽人也 十二月戊午以康義誠為河陽節度使兼侍衛親軍馬步都指揮使【葉夢得石林燕語云自梁置在京馬步軍都指揮使後唐遂置侍衛親軍都指揮使】以朱弘昭為山南東道節度使【朱弘昭始正授襄陽旌節】 是歲漢主立其子耀樞為雍王【雍於用翻】龜圖為康王弘度為賓王弘熙為晉王弘昌為越王弘弼為齊王弘雅為韶王弘澤為鎮王弘操為萬王弘杲為循王弘暐為思王弘邈為高王弘簡為同王弘建為益王弘濟為辯王弘道為貴王弘昭為宜王弘政為通王弘益為定王未幾徙弘度為秦王【幾居豈翻漢諸王皆以州為名】<br />
<br />
  四年春正月戊子加秦王從榮守尚書令兼侍中庚寅以端明殿學士歸義劉昫為中書侍郎同平章事【歸義縣屬涿州昫吁句翻又許羽翻】 閩人有言真封宅龍見者【真封宅蓋王延鈞未得國之時所居也見賢遍翻】更命其宅曰龍躍宫【更工衡翻下更名同】遂詣寶皇宫受冊備儀衛入府即皇帝位國號大閩大赦改元龍啟更名璘追尊父祖立五廟以其僚屬李敏為左僕射門下侍郎其子節度副使繼鵬為右僕射中書侍郎並同平章事以親吏吳朂為樞密使唐冊禮使裴傑程侃適至海門【海門即今福清縣之海門鎮是也】閩主以傑為如京使侃固求北還不許閩主自以國小地僻常謹事四隣由是境内差安【史言閩主雖惑於神仙妖妄而能粗安者以善鄰而然】 二月戊申孟知祥墨制以趙季良等為五鎮節度使【孟知祥為五帥請節鉞朝廷依違不報而許之墨制署授故知祥因而授五帥昔唐之季也強藩悍將猶知長安本色之為貴若趙季良等知稟命於孟知祥而已豈復知重朝命哉】 涼州大將拓拔承謙及耆老上表請以權知留後孫超為節度使上問使者超為何人對曰張義潮在河西【張義潮以河西來歸事始二百四十九卷唐宣宗大中五年】朝廷以天平軍二千五百人戍凉州自黄巢之亂凉州為党項所隔鄆人稍稍物故皆盡超及城中之人皆其子孫也 乙卯以馬希範為武安武平節度使【馬希範席父兄之業故朝廷仍命以潭朗兩鎮】兼中書令 戊午定難節度使李仁福卒庚申軍中立其子彞超為留後 癸亥以孟知祥為東西川節度使蜀王 先是河西諸鎮皆言李仁福潛通契丹【是時河西止有凉州沙州二鎮然使命不常通也竊意河西當作關西歐史只作邊將多言仁福通於契丹尤為櫽括先悉薦翻】朝廷恐其與契丹連兵併吞河右南侵關中會仁福卒三月癸未以其子彞超為彰武留後【唐末以延州置保塞軍岐改為忠義軍後唐改為彰武軍】徙彰武節度使安從進為定難留後仍命靜塞節度使藥彦稠將兵五萬以宫苑使安重益為監軍送從進赴鎮從進索葛人也【難乃但翻索葛部居振武宋白曰安從進本貫振武軍索葛府索葛村索蘇各翻】 乙酉始下制除趙季良等為五鎮節度使【孟知祥既以墨制命之朝廷不能違遂為之下制】 丁亥敕諭夏銀綏宥將士吏民以夏州窮邊李彞超年少【少詩照翻】未能扞禦故徙之延安【延州延安郡】從命則有李從曮高允韜富貴之福【季從曮事見上卷長興元年又是年高允韜自鄜延徙安國】違命則有王都李匡賓覆族之禍【王都事見二百七十六卷天成四年李匡賓事見上卷元年】夏四月彞超上言為軍士百姓擁留未得赴鎮詔遣使趣之【趣讀曰促】 言事者請為親王置師傅【為于偽翻】宰相畏秦王從榮不敢除人請令王自擇秦王府判官太子詹事王居敏薦兵部侍郎劉瓚於從榮【瓚才旱翻歐史作劉贊時為刑部侍郎】從榮表請之癸丑以瓚為祕書監秦王傅前襄州支使山陽魚崇遠為記室【漢之山陽郡唐為曹濟之地此山陽唐楚州之山陽縣也舊唐書地理志曰山陽縣漢臨淮郡之射陽縣地晉置山陽郡改為山陽縣唐為楚州治所】瓚自以左遷泣訴不得免【唐制六部侍郎除吏部之外餘皆從四品下王傅從三品然六部侍郎為嚮用王傅為左遷以職事有閑劇之不同也當是時從榮地居儲副則秦王傅不可以閑官言蓋以從榮輕佻峻急恐預其禍求自脱耳】王府參佐皆新進少年輕脱諂諛瓚獨從容規諷【從千容翻】從榮不悦瓚雖為傅從榮一槩以僚屬待之瓚有難色從榮覺之自是戒門者勿為通【勿為于偽翻】月聽一至府或竟日不召亦不得食 李彞超不奉詔【詔趣李彞超赴延安而不奉詔】遣其兄阿囉王守青嶺門【囉魯何翻青嶺門蓋漢上郡橋山之長城門也東北過奢延澤至夏州】集境内党項諸胡以自救藥彦稠等進屯蘆關【蘆子關在延州延昌縣北趙珣聚米圖經曰蘆關在延州塞門寨北十五里】彞超遣党項抄糧運及攻具【抄楚交翻】官軍自蘆關退保金明【金明漢膚施縣地後魏太平真君十二年置金明郡隋廢郡為縣縣又尋廢唐武德二年分膚施縣復置金明縣宋熙寧五年省金明縣為寨屬膚施縣趙珣聚米圖經曰自蘆關南入塞門即金明路陳執中曰塞門至金明二百里】閩主璘立子繼鵬為福王充寶皇宫使 五月戊寅<br />
<br />
  立皇子從珂為潞王從益為許王從子天平節度使從温為兖王護國節度使從璋為洋王成德節度使從敏為涇王【從子才用翻】 庚辰閩地震閩主璘避位修道命福王繼鵬權總萬機初閩王審知性節儉府舍皆庳陋【庳皮靡翻】至是大作宫殿極土木之盛 甲申帝暴得風疾庚寅小愈見羣臣於文明殿【薛史梁開平三年改西京貞觀殿為文明殿】 壬辰夜夏州城上舉火比明雜虜數千騎救之【夜舉火於城上及明而雜虜至蓋先約以舉烽為號欲内外夾擊唐兵也比必利翻】安從進遣先鋒使宋温擊走之 吳宋齊丘勸徐知誥徙吳主都金陵知誥乃營宫城於金陵 帝旬日不見羣臣都人忷懼【忷許勇翻】或潛竄山野或寓止軍營【寓止軍營者恐軍中起變欲依之以自全】秋七月庚辰帝力疾御廣壽殿【廣壽殿不知其創造之始薛史本紀長興四年重修廣壽殿帝曰此殿經焚不可不修蓋焚于同光之末也】人情始安 安從進攻夏州州城赫連勃勃所築【夏州城赫連勃勃蒸土所築統萬城也事見一百一十七卷晉安帝義熙九年宋白曰統萬城在朔方之北黑水之南其城土白而堅南有亢敵峻險非人力所攻迄今雉堞雖久崇墉若新】堅如鐵石斸鑿不能入【斸株玉翻斫也】又党項萬餘騎徜徉四野抄掠糧餉【徜常羊翻徉音羊抄楚交翻】官軍無所芻牧山路險狹關中民輸斗粟束藳費錢數緡民間困竭不能供李彞超兄弟登城謂從進曰夏州貧瘠非有珍寶蓄積可以充朝廷貢賦也但以祖父世守此土【唐僖宗時拓跋思恭據夏州傳思諫彞昌仁福以至彞超】不欲失之蕞爾孤城【蕞徂外翻】勝之不武何足煩國家勞費如此幸為表聞【為于偽翻】若許其自新或使之征伐願為衆先上聞之壬午命從進引兵還【還從宣翻下同】其後有知李仁福隂事者云仁福畏朝廷除移揚言結契丹為援【除移謂除移它鎮揚言者播其言使人知】契丹實不與之通也致朝廷誤興是役無功而還自是夏州輕朝廷每有叛臣必隂與之連以邀賂遺【遺唯季翻】上疾久未平征夏州無功軍士頗有流言乙酉賜在京諸軍優給有差既賞賚無名士卒由是益驕【唐兵之驕始于同光甚于長興極於清泰至廣運之末契丹入汴晉兵有不得食者矣】 丁亥賜錢元瓘爵吳王元瓘於兄弟甚厚其兄中吳建武節度使元璙自蘇州入見【元璙即傳璙元瓘嗣國兄弟名從傳者並改為元吳越于蘇州置中吳節度薛史曰唐莊宗三年升蘇州為中吳軍見賢遍翻】元瓘以家人禮事之奉觴為壽曰此兄之位也而小子居之兄之賜也【元璙讓位於元瓘見二百七十六卷天成三年】元璙曰先王擇賢而立之君臣位定元璙知忠順而已因相與對泣【元瓘篤友悌之義元璙知忠順之節兄弟輯睦以保其國異乎夫己氏者矣】 戊子閩主璘復位【王璘避位六十五日特以厭地震之異耳】初福建中軍使薛文傑性巧佞璘喜奢侈文傑以聚斂求媚【喜許記翻斂力瞻翻】璘以為國計使親任之文傑隂求富民之罪籍没其財被榜捶者胷背分受仍以銅斗火熨之【榜音彭捶止蘂翻熨紆勿翻又紆胃翻】建州土豪吳光入朝文傑利其財求其罪將治之【治直之翻】光怨怒帥其衆且萬人叛奔吳【為吳光引吳兵攻建州而文傑誅張本帥讀曰率】 帝以工部尚書盧文紀禮部郎中呂琦為蜀王冊禮使并賜蜀王一品朝服知祥自作九旒冕九章衣車服旌旗皆擬王者【王者謂天子也唐制真王正一品朝廷既賜孟知祥以一品朝服知祥又作九旒九章之法服是為王矣所謂車服旌旗皆擬王者是擬天子也】八月乙巳朔文紀等至成都戊申知祥服衮冕備儀衛詣驛【時館盧文紀等于成都驛舍】降階北面受冊升玉輅至府門乘步輦以歸【玉輅天子之輅步輦以人挽之】文紀簡求之孫也【盧簡求綸之子也唐宣宗懿宗之時内歷臺閣外踐節鎮】 戊申羣臣上尊號曰聖明神武廣道法天文德恭孝皇帝大赦在京及諸道將士各等第優給時一月之間再行優給【乙酉至戊申二十四日耳】由是用度益窘【明宗之優給懲莊宗之過也給之愈濫士心愈驕由是有到鳳翔更請一分之事窘巨隕翻】 太僕少卿何澤見上寢疾秦王從榮權勢方盛冀已復進用【歐史曰何澤外雖直言而内實邪佞與宰相趙鳳有舊數私于鳳鳳薄其為人以為太常少卿勅未出澤先知之即稱新官上章自訴章下中書鳳等言澤未拜命而稱新官輕侮朝廷請坐以法乃以太僕少卿致仕居于河陽】表請立從榮為太子上覽表泣下私謂左右曰羣臣請立太子朕當歸老太原舊第耳【此唐宣宗所謂若立太子則朕便為閑人之見也富有天下不思貽後之謀而為此論意趣凡近良可憫笑帝事太祖莊宗起於晉陽有舊第在焉】不得已丙戌詔宰相樞密使議之丁卯從榮見上【見賢遍翻】言曰竊聞有姦人請立臣為太子臣幼少且願學治軍民不願當此名【治直之翻】上曰羣臣所欲也從榮退見范延光趙延壽曰執政欲以吾為太子是欲奪我兵柄幽之東宫耳【從榮之言與明宗之言同一戀權之心耳】延光等知上意且懼從榮之言即具以白上辛未制以從榮為天下兵馬大元帥 九月甲戌朔吳主立德妃王氏為皇后戊寅加范延光趙延壽兼侍中 癸未中書奏節度<br />
<br />
  使見元帥儀雖帶平章事亦以軍禮廷參從之【時中書門下奏自歷朝以來無天下兵馬大元帥公事儀注或專一面之權或總諸道之帥其儀注規程公事載詳故實未見明文臣等謹沿近事伏見招討使總管兼受副使已下櫜庭禮今望令諸道節度使以下凡帶兵權者見元帥堦下具軍禮參見皆申公狀其使相者初相見亦以軍禮一度已後客禮相見應天下諸軍務公事元帥府行指揮其判六軍諸衛事則行公牒往來其元帥府所置官屬補奏軍職則委元帥奏請署署按是時執政畏從榮崇秩太過】 帝欲加宣徽使判三司馮贇同平章事贇父名章【贇父璋事帝於潛躍為閽者】執政誤引故事庚寅加贇同中書門下二品充三司使【唐制中書門下二省惟中書令侍中正二品侍郎則正三品以兩省侍郎兼宰相之職則謂之同中書門下平章事而官則自依本品今同中書門下二品則其品同兩省長官是誤也】 秦王從榮請嚴衛捧聖步騎兩指揮為牙兵【五代會要應順元年三月改左右羽林四十指揮為嚴衛左右軍龍武神武四十指揮為捧聖左右軍按是年帝殂明年正月閔帝改元應順四月潞王入立改元清泰數月之間乃宋潞二王兵爭之際何暇改屯衛諸軍號乎是必改于天成長興之間會要誤也】每入朝從數百騎張弓挾矢馳騁衢路令文士試草檄淮南書陳已將廓清海内之意從榮不快於執政私謂所親曰吾一旦南面必族之范延光趙延壽懼屢求外補以避之上以為見己病而求去甚怒曰欲去自去奚用表為齊國公主復為延壽言於禁中云延壽實有疾不堪機務【趙延壽尚帝女齊國公主復扶又翻下同為于偽翻】丙申二人復言於上曰臣等非敢憚勞願與勲舊迭為之亦不敢俱去願聽一人先出若新人不稱職【稱尺證翻】復召臣臣即至矣上乃許之戊戌以延壽為宣武節度使以山南東道節度使朱弘昭為樞密使同平章事制下弘昭復辭【亦懼從榮之禍也下戶嫁翻】上叱之曰汝輩皆不欲在吾側吾蓄養汝輩何為弘昭乃不敢言 吏部侍郎張文寶泛海使杭州【使疏吏翻】船壞水工以小舟濟之風飄至天長【天長縣在揚州西一百一十里其地北不至淮東不至海豈小舟隨風所能至今通州海門縣崇明鎮東海中有大洲謂之天賜鹽場舟人揚帆遇順東南可以徑至明州定海西南可以至許浦達蘇州恐是此處宋之通州吳之靜海軍也】從者二百人所存者五人吳主厚禮之資以從者儀服錢幣數萬仍為之牒錢氏【從才用翻為于偽翻】使於境上迎文寶獨受飲食餘皆辭之曰本朝與吳久不通問今既非君臣又非賓主若受茲物何辭以謝吳主嘉之竟達命於杭州而還【還從宣翻又如字】 庚子以前義成節度使李贊華為昭信節度使留洛陽食其俸【去年以李贊華帥義成事見上卷按唐末於金州置昭信節度五代兵爭不復以為節鎮又按五代會要長興二年升䖍州為昭信節度時䖍州屬吳吳以為百勝節度贊華所領節抑䖍州之昭信軍歟又是年十一月庚辰改慎州懷化軍為昭化軍慎州在幽州之北唐盛時所置以處突厥降者抑以贊華領昭化節而信字乃化字之誤歟】 辛丑詔大元帥從榮位在宰相上 吳徐知誥以國中水火屢為災曰兵民困苦吾安可獨樂【樂音洛】悉縱遣侍妓【妓渠綺翻】取樂器焚之 閩内樞密使薛文傑說閩王抑挫諸宗室從子繼圖不勝忿【說式苪翻從才用翻勝音升】謀反坐誅連坐者千餘人 冬十月乙卯范延光馮贇奏西北諸胡賣馬者往來如織日用絹無慮五千匹計耗國用什之七【天成四年沿邊置場市馬禁党項賣馬者到闕而卒不能禁今掌兵掌計之臣復以其耗費而奏言之】請委緣邊鎮戍擇諸胡所賣馬良者給劵具數以聞從之 戊午以前武興節度使孫岳為三司使【代馮贇也】 范延光屢因孟漢瓊王淑妃以求出庚申以延光為成德節度使以馮贇為樞密使帝以親軍都指揮使河陽節度使同平章事康義誠為朴忠親任之時要近之官多求出以避秦王之禍義誠度不能自脱乃令其子事秦王務以恭順持兩端冀得自全【帝以康義誠為朴忠朴忠者能持兩端乎是後康義誠事閔帝自請將兵拒潞王而遂迎降亦所以自全也乃所以自斃若此者果可親任耶度徒洛翻】權知夏州事李彞超上表謝罪求昭雪【去年秋討李彞超昭者明其無它雪者澡洗其罪】壬戌以彞超為定難軍節度使【難乃旦翻】 十一月甲戌上錢范延光酒罷上曰卿今遠去事宜盡言對曰朝廷大事願陛下與内外輔臣參决勿聽羣小之言【内輔臣謂樞密使外輔臣謂宰相羣小指孟漢瓊之黨】遂相泣而别【語而相泣死期將至不知泣下也】時孟漢瓊用事附之者共為朋黨以蔽惑上聽故延光言及之 庚辰改慎州懷化軍【九域志慎州昭化軍節度五代會要是年十一月庚辰改鎮州懷化軍為昭化軍史于此蓋逸為昭化軍四字】置保順軍於洮州領洮鄯等州【自唐肅代以來洮鄯没於吐蕃是時必有西戎首領來歸附故置節鎮以寵授之洮土刀翻鄯音善】 戊子帝疾復作【歐史戊子雪帝幸宫西士和亭得傷寒疾復扶又翻】己丑大漸秦王從榮入問疾帝俛首不能舉【俛音免】王淑妃曰從榮在此帝不應從榮出聞宫中皆哭從榮意帝已殂明旦稱疾不入是夕帝實小愈【歐史從榮與朱弘昭馮贇入問起居於廣壽殿帝不能知人從榮等出乃遷于雍和殿宫中皆慟哭至夜半帝蹶然自興于榻而侍疾者皆去顧殿上守漏宫女曰夜漏幾何對曰四更矣帝即唾肉如胏者數片溺便液斗餘守漏者曰大家省事乎曰吾不知也有頃六宫皆至曰大家還魂矣因進粥一器至旦疾少愈薛史作便溺升餘當改斗字從升字】而從榮不知從榮自知不為時論所與【事始見二百七十六卷天成三年人患不自知耳既自知矣而與小人謀為自全之計此其所以敗也】恐不得為嗣與其黨謀欲以兵入侍先制權臣【權臣謂孟漢瓊朱弘昭馮贇等】辛卯從榮遣都押牙馬處鈞謂朱弘昭馮贇曰吾欲帥牙兵入宫中侍疾【帥讀曰率】且備非常當止於何所二人曰王自擇之既而私於處鈞曰主上萬福【今人言起居無它者為萬福】王宜竭心忠孝不可妄信人浮言從榮怒復遣處鈞謂二人曰公輩殊不愛家族邪何敢拒我【以參夷之罪臨之復扶又翻】二人患之入告王淑妃及宣徽使孟漢瓊咸曰茲事不得康義誠不可濟【康義誠時總侍衛親軍言不得義誠與之合謀拒從榮則事不可成】乃召義誠謀之義誠竟無言但曰義誠將校耳【將即亮翻校戶教翻】不敢預議惟相公所使弘昭疑義誠不欲衆中言之夜邀至私第問之其對如初【康義誠之初計欲持兩端以自全故其對如此】壬辰從榮自河南府常服將步騎千人陳於天津橋【從榮時以河南尹判六軍諸衛居河南府】是日黎明從榮遣馬處鈞至馮贇第語之曰吾今曰决入且居興聖宫【語牛倨翻帝之嗣位也先入居興聖宫故從榮欲效之】公輩各有宗族處事亦宜詳允【處昌呂翻下處置同又使處鈞以前說臨馮贇】禍福在須臾耳又遣處鈞詣康義誠義誠曰王來則奉迎【言來則奉迎不來則不敢輕動此即義誠遣子事秦府之初計也】贇馳入右掖門見弘昭義誠漢瓊及三司使孫岳方聚謀於中興殿門外【五代會要唐莊宗同光二年改洛陽崇勲殿為中興殿萬春門為中興門】贇具道處鈞之言因讓義誠曰秦王言禍福在須臾其事可知公勿以兒在秦府左右顧望主上拔擢吾輩自布衣至將相苟使秦王兵得入此門置主上何地吾輩尚有遺種乎【種章勇翻】義誠未及對監門白秦王已將兵至端門外【監門監門衛將軍也端門宫城正南門】漢瓊拂衣起曰今日之事危及君父公猶顧望擇利邪【公謂康義誠】吾何愛餘生當自帥兵拒之耳【帥讀曰率下同】即入殿門弘昭贇隨之義誠不得已亦隨之入漢瓊見帝曰從榮反兵已攻端門須臾入宫則大亂矣宫中相顧號哭【號戶刀翻】帝曰從榮何苦乃爾問弘昭等有諸對曰有之適已令門者闔門矣帝指天泣下謂義誠曰卿自處置【處昌呂翻】勿驚百姓控鶴指揮使李重吉從珂之子也時侍側帝曰吾與爾父冒矢石定天下【冒莫北翻】數脱吾於厄【數所角翻】從榮輩得何力今乃為人所教為此悖逆【悖蒲内翻又蒲没翻】我固知此曹不足付大事當呼爾父授以兵柄耳【時從珂鎮鳳翔帝言欲召之】汝為我部閉諸門【為于偽翻下嘗為同】重吉即帥控鶴兵守宫門【帥讀曰率下同】孟漢瓊被甲乘馬【被皮義翻】召馬軍都指揮使朱洪實使將五百騎討從榮從榮方據胡床坐橋上【胡床即今之交床自晉人已來用之】遣左右召康義誠端門已閉叩左掖門【端門之東門曰左掖門西門曰右掖門言在端門之左右若臂掖之左右然也】從門隙中窺之見朱洪實引騎兵北來【端門宫城南門兵從宫中出自掖門外窺之見其兵北來】走白從榮從榮大驚命取鐵掩心擐之【甲在前者謂之掩心擐戶慣翻】坐調弓矢俄而騎兵大至從榮走歸府【走歸河南府也】僚佐皆竄匿牙兵掠嘉善坊潰去從榮與妃劉氏匿牀下皇城使安從益就斬之并殺其子以其首獻初孫岳頗得豫内廷密謀馮朱患從榮狼伉【馮朱謂馮贇朱弘昭晉周嵩謂王敦曰處仲狼抗無上】岳嘗為之極言禍福之歸康義誠恨之至是乘亂密遣騎士射殺之【射而亦翻】帝聞從榮死悲駭幾落御榻絶而復蘇者再由是疾復劇【幾居依翻復扶又翻】從榮一子尚幼養宫中諸將請除之帝泣曰此何罪不得已竟與之癸巳馮道帥羣臣入見帝於雍和殿帝雨泣嗚咽【見賢遍翻雨泣者淚下如雨】曰吾家事至此慙見卿等宋王從厚為天雄節度使甲午遣孟漢瓊徵從厚且權知天雄軍府事【使孟漢瓊徵從厚入侍疾因使漢瓊權知天雄軍府】丙申追廢從榮為庶人執政共議從榮官屬之罪馮道曰從榮所親者高輦劉陟王說而已【說讀為悦】任贊到官纔半月王居敏司徒詡在病告已半年【司徒詡其先以官為氏在病告者以病謁告家居】豈豫其謀居敏尤為從榮所惡【惡烏路翻】昨舉兵向闕之際與輦陟並轡而行指日景曰來日及今已誅王詹事矣【日景日之晷景及今猶言及此時也王詹事謂王居敏稱其官也】自非與之同謀者豈得一切誅之乎朱弘昭曰使從榮得入光政門【唐昭宗之遷洛陽也改長樂門為光政門】贊等當如何任使而吾輩猶有種乎【種章勇翻】且首從差一等耳今首已孥戮而從皆不問【從才用翻孥音奴凡定罪從減為首一等】主上能不以吾輩為庇姦人乎馮贇力爭之始議流貶時諮議高輦已伏誅丁酉元帥府判官兵部侍郎任贊祕書監兼王傅劉瓚友蘇瓚記室魚崇遠河南少尹劉陟判官司徒詡推官王說等八人並長流【唐法長流人謂之長流百姓】河南巡官李澣江文蔚等六人勒歸田里【蔚紆勿翻】六軍判官太子詹事王居敏推官郭晙並貶官【從榮判六軍諸衛事其府僚有判官推官晙祖峻翻】澣回之族曾孫也【李回唐武宗朝為宰相】詡貝州人文蔚建安人也文蔚奔吳徐知誥厚禮之初從榮失道六軍判官司諫郎中趙遠諫曰大王地居上嗣【上嗣言齒居諸子之上當嗣有大業】當勤修令德奈何所為如是勿謂父子至親為可恃獨不見恭世子戾太子乎【春秋晉獻公殺其世子而非其罪後諡曰恭戾太子事見二十二卷漢武帝征和二年】從榮怒出為涇州判官及從榮敗遠以是知名遠字上交【趙遠後事漢高祖避高祖名以字行故史著其名】幽州人也 戊戌帝殂【年六十七按下文云登極之年已踰六十則是年年六十八】帝性不猜忌與物無競登極之年已踰六十每夕於宫中焚香祝天曰某胡人因亂為衆所推【事見二百七十四卷天成元年】願天早生聖人為生民主【范仲淹曰我太祖皇帝應期而生】在位年穀屢豐【屢力住翻】兵革罕用校於五代粗為小康【校比也小康小安也粗坐五翻】辛丑宋王至洛陽【自魏州至洛陽】閩主尊魯國太夫人黄氏為皇太后 閩王好鬼神【好呼到翻】巫盛韜等皆有寵薛文傑言於閩主曰陛下左右多姦臣非質諸鬼神【質正也】不能知也盛韜善視鬼宜使察之閩主從之文傑惡樞密使吳朂【吳朂本閩主親吏故任之以機密文傑以是惡之惡烏路翻】朂有疾文傑省之【省悉景翻】曰主上以公久疾欲罷公近密僕言公但小苦頭痛耳將愈矣主上或遣使來問慎勿以它疾對也朂許諾明日文傑使韜言於閩主曰適見北廟崇順王訊吳朂謀反【閩主信北廟崇順王事始見上卷三年】以銅釘釘其腦【上釘如字下釘丁定翻】金椎擊之閩主以告文傑文傑曰未可信也宜遣使問之果以頭痛對即收下獄遣文傑及獄吏雜治之朂自誣服并其妻子誅之【吳朂歐史作吳英下遐稼翻治直吏翻】由是國人益怒吳光請兵於吳【吳光奔吳見上七月】吳信州刺史蔣延徽不俟朝命引兵會光攻建州【信州在漢時其地界于豫章餘汗會稽太末二縣之間三國時為鄱陽郡葛陽縣之地晉宋以至于隋屬東陽鄱陽二郡陳改葛陽為弋陽縣唐乾元元年析饒州之弋陽衢州之常山玉山及建撫之地置信州九域志信州南至建州四百里朝直遥翻】閩主遣使求救於吳越 十二月癸卯朔始發明宗喪【戊戌至癸卯六日始發喪亂故也】宋王即皇帝位【諱從厚明宗第五子也明宗殂四日而後宋王至至三日始發喪即位】 秦王從榮既死朱洪實妻入宫司衣王氏語及秦王【唐制内職有六尚猶外朝之六尚書也有二十四司猶二十四曹郎也司衣屬尚服局掌宫内御服首飾整比以時進奉】王氏曰秦王為人子不在左右侍疾致人歸禍是其罪也若云大逆則厚誣矣朱司徒最受王恩【朱洪實蓋加檢校司徒故稱之】當時不辯為之惜哉【為力于偽翻下同】洪實聞之大懼與康義誠以其語白閔帝且言王氏私於從榮為之詗宫中事辛亥賜王氏死事連王淑妃淑妃素厚於從榮【歐史曰初明宗後宫有生子者命妃母之是為許王從益從益乳母司衣王氏見明宗已老而秦王握兵心欲自託為後計乃曰兒思秦王是時從益已四歲又數教從益自言求見秦王明宗遣乳嫗將兒往來秦府遂與從榮私通從榮因使伺察宫中動靜事連王淑妃由是故也詗火迥翻又休正翻】帝由是疑之【從榮已死往事何足復論况國難甫定人心疑阻宜示寛大使各自安帝多疑而少斷此其所以不得令終也】 丙辰以天雄左都押牙宋令詢為磁州刺史【磁墻之翻】朱弘昭以誅秦王立帝為己功欲專朝政令詢侍帝左右最久雅為帝所親信【雅素也】弘昭不欲舊人在帝側故出之帝不悦而無如之何孟知祥聞明宗殂謂僚佐曰宋王幼弱為政者皆胥吏小人【朱弘昭馮贇先皆以胥史事明宗於潛躍遂階柄用故為孟知祥所侮易】其亂可坐俟也 辛未帝始御中興殿帝自終易月之制【循漢晉喪制以日易月二十七日而檡服】即召學士讀貞觀政要太宗實録有致治之志然不知其要寛柔少斷【治直吏翻斷丁亂翻】李愚私謂同列曰吾君延訪鮮及吾輩位高責重事亦堪憂衆惕息不敢應【李愚時為相言帝不謀政于宰相而專與樞密宣徽等議事也鮮息淺翻惕它歷翻】 順化節度使同平章事判明州錢元珦驕縱不法【以吳越於台州置德化節度槩觀之蓋置順化節度于明州也又按薛史長興三年昇楚州為順化軍以明州刺史錢元珦為本州節度使楚州時屬楊氏元珦蓋鎮明州而領楚州節耳珦許亮翻】每請事於王府不獲【王府謂吳越國王府】輒上書悖慢【悖蒲昧翻又蒲没翻】嘗怒一吏置錢牀炙之【炙之石翻】臭滿城郭吳王元瓘遣牙將仰仁詮詣明州召之仁詮左右慮元珦難制勸為之備仁詮不從常服徑造聽事【姓苑有仰姓詮且緣翻造七到翻聽讀曰廳】元珦見仁詮至股慄遂還錢塘幽于别第仁詮湖州人也 閩王改福州為長樂府【樂音洛】親從都指揮使王仁達有擒王延稟之功【從才用翻王仁達擒延稟事見上卷長興二年】性慷慨言事無所避閩主惡之【惡烏路翻】嘗私謂左右曰仁達智有餘吾猶能御之非少主臣也【少詩照翻】至是竟誣以叛族誅之 初馬希聲希範同日生希聲母曰袁德妃希範母曰陳氏希範怨希聲先立不讓及嗣位不禮於袁德妃【按歐史楚王殷有子十餘人嫡子希振長而賢其次希聲與希範同日生希聲以母袁夫人有色而寵盛得立而希振棄官為道士希聲以長幼之序當讓希振未當讓希範也】希聲母弟希旺為親從都指揮使【從才用翻】希範多譴責之袁德妃請納希旺官為道士不許解其軍職使居竹屋草門不得預兄弟燕集德妃卒希旺憂憤而卒潞王上【諱從珂鎮州平山人本姓王氏明宗為將時過平山掠得之養以為子】<br />
<br />
  清泰元年【是年四月入立始改元清泰】春正月戊寅閔帝大赦改元應順【取應天順人為義非繼體之君所以紀元也】壬午加河陽節度使兼侍衛都指揮使康義誠兼侍中判六軍諸衛事 朱弘昭馮贇忌侍衛馬軍都指揮使安彦威侍衛步軍都指揮使忠正節度使張從賓【五代會要天成二年十月升壽州為忠正節度時壽州屬吳後唐蓋升節鎮以寵授其臣遥領之耳】甲申出彦威為護國節度使以捧聖馬軍都指揮使朱洪實代之出從賓為彰義節度使以嚴衛步軍都指揮使皇甫遇代之【米馮之多忌所以速禍也薛史明宗長興三年以神捷神威雄威廣捷已下指揮改為左右羽林軍閔帝即位改左右羽林軍為嚴衛左右龍武神武軍為捧聖按薛史之誤與會要同】彦威崞人【崞音郭】遇真定人也 戊子樞密使同平章事朱弘昭同中書門下二品馮贇河東節度使兼侍中石敬瑭並兼中書令贇以超遷太過堅辭不受己丑改兼侍中 壬辰以荆南節度使高從誨為南平王武安武平節度使馬希範為楚王 甲午以鎮海鎮東節度使吳王元瓘為吳越王 吳徐知誥别治私第於金陵【治直吏翻】乙未遷居私第虛府舍以待吳主【此吳主楊溥】 鳳翔節度使兼侍中潞王從珂與石敬瑭少從明帝征伐有功名得衆心【少詩照翻】朱弘昭馮贇位望素出二人下遠甚一旦執朝政皆忌之【忌從珂及敬瑭也朝直遥翻】明宗有疾潞王屢遣其夫人入省侍【省悉景翻】及明宗殂潞王辭疾不來【以主少國疑也其相猜阻之迹見矣】使臣至鳳翔者或自言伺得潞王隂事【此小人之交鬬者迎合執政意嚮使疏吏翻伺相吏翻】時潞王長子重吉為控鶴都指揮使朱馮不欲其典禁兵己亥出為亳州團練使潞王有女惠明為尼在洛陽亦召入禁中潞王由是疑懼【為潞王舉兵張本】 吳蔣延徽敗閩兵於浦城【漢末以會稽南部置漢興縣吳更曰吳興為建安郡治所隋廢郡為縣唐載初元年分建安縣置唐興縣天授二年改曰武寧神龍元年復曰唐興天寶元年改曰浦城屬建州宋白曰浦城本東候官之北鄉也漢末置漢興縣吳曰吳興唐曰唐興天寶改浦城有二浦其城臨浦故曰浦城九域志在州東北三百三十里敗補邁翻】遂圍建州閩主璘遣上軍使張彦柔【閩置上軍使中軍使下軍使】驃騎大將軍王延宗將兵萬人救建州延宗軍及中途士卒不進曰不得薛文傑不能討賊延宗馳使以聞國人震恐太后及福王繼鵬【繼鵬閩主長子也】泣謂璘曰文傑盜弄國權枉害無辜上下怨怒久矣今吳兵深入士卒不進社稷一旦傾覆留文傑何益文傑亦在側互陳利害璘曰五吾無如卿何卿自為謀文傑出繼鵬伺之於啟聖門外以笏擊之仆地檻車送軍前市人爭持瓦礫擊之【礫郎撃翻】文傑善術數自云過三日則無患部送者聞之倍道兼行二日而至士卒見之踴躍臠食之【臠力兖翻】閩主亟遣赦之不及初文傑以為古制檻車疎濶更為之【更工衡翻改也】形如木匱攅以鐵鋩内向動輒觸之車成文傑首自入焉并誅盛韜【盛韜以鬼神事黨附薛文傑為姦者也】蔣延徽攻建州垂克徐知誥以延徽吳太祖之壻【吳尊楊行密廟號太祖】與臨川王濛素善恐其克建州奉濛以圖興復【濛為徐氏父子所忌事始二百七十一卷梁均王貞明五年】遣使召之延徽亦聞閩兵及吳越兵將至引兵歸閩人追擊敗之士卒死亡甚衆歸罪於都虞候張重進斬之【敗補邁翻重直龍翻】知誥貶延徽為右威衛將軍遣使求好于閩【好呼到翻】 閏月以左諫議大夫唐汭膳部郎中知制誥陳乂皆為給事中充樞密直學士汭以文學從帝歷三鎮在幕府【帝以開成三年鎮宣武明年徙鎮河東長興元年徙鎮天雄】及即位將佐之有才者朱馮皆斥逐之汭性迂疎朱馮恐帝含怒有時而發乃引汭於密近以其黨陳乂監之【監古銜翻】丙午尊皇后為皇太后【皇后明宗曹皇后也】安遠節度使苻彦超奴王希全任賀兒【任音壬】見朝廷多事謀殺彦超據安州附於吳夜叩門稱有急遞【軍期緊急文書入遞不容稽違晷刻者謂之急遞遞郵傳也遞者言郵置遞以相付而達其所】彦超出至聽事二奴殺之因以彦超之命召諸將有不從己者輒殺之己酉旦副使李端帥州兵討誅之并其黨【副使者節度副使也帥讀曰率】 甲寅以王淑妃為太妃【不曰尊而曰以史言閔帝之薄王淑妃】 蜀將吏勸蜀王知祥稱帝己巳知祥即皇帝位于成都【孟知祥字保胤邢州龍岡人】<br />
<br />
  資治通鑑卷二百七十八  <br>
   </div> 

<script src="/search/ajaxskft.js"> </script>
 <div class="clear"></div>
<br>
<br>
 <!-- a.d-->

 <!--
<div class="info_share">
</div> 
-->
 <!--info_share--></div>   <!-- end info_content-->
  </div> <!-- end l-->

<div class="r">   <!--r-->



<div class="sidebar"  style="margin-bottom:2px;">

 
<div class="sidebar_title">工具类大全</div>
<div class="sidebar_info">
<strong><a href="http://www.guoxuedashi.com/lsditu/" target="_blank">历史地图</a></strong>  
<a href="http://www.880114.com/" target="_blank">英语宝典</a>  
<a href="http://www.guoxuedashi.com/13jing/" target="_blank">十三经检索</a> 
<br><strong><a href="http://www.guoxuedashi.com/gjtsjc/" target="_blank">古今图书集成</a></strong> 
<a href="http://www.guoxuedashi.com/duilian/" target="_blank">对联大全</a> <strong><a href="http://www.guoxuedashi.com/xiangxingzi/" target="_blank">象形文字典</a></strong> 

<br><a href="http://www.guoxuedashi.com/zixing/yanbian/">字形演变</a>  <strong><a href="http://www.guoxuemi.com/hafo/" target="_blank">哈佛燕京中文善本特藏</a></strong>
<br><strong><a href="http://www.guoxuedashi.com/csfz/" target="_blank">丛书&方志检索器</a></strong> <a href="http://www.guoxuedashi.com/yqjyy/" target="_blank">一切经音义</a>  

<br><strong><a href="http://www.guoxuedashi.com/jiapu/" target="_blank">家谱族谱查询</a></strong>  <strong><a href="http://shufa.guoxuedashi.com/sfzitie/" target="_blank">书法字帖欣赏</a></strong> 
<br>

</div>
</div>


<div class="sidebar" style="margin-bottom:0px;">

<font style="font-size:22px;line-height:32px">QQ交流群9:489193090</font>


<div class="sidebar_title">手机APP 扫描或点击</div>
<div class="sidebar_info">
<table>
<tr>
	<td width=160><a href="http://m.guoxuedashi.com/app/" target="_blank"><img src="/img/gxds-sj.png" width="140"  border="0" alt="国学大师手机版"></a></td>
	<td>
<a href="http://www.guoxuedashi.com/download/" target="_blank">app软件下载专区</a><br>
<a href="http://www.guoxuedashi.com/download/gxds.php" target="_blank">《国学大师》下载</a><br>
<a href="http://www.guoxuedashi.com/download/kxzd.php" target="_blank">《汉字宝典》下载</a><br>
<a href="http://www.guoxuedashi.com/download/scqbd.php" target="_blank">《诗词曲宝典》下载</a><br>
<a href="http://www.guoxuedashi.com/SiKuQuanShu/skqs.php" target="_blank">《四库全书》下载</a><br>
</td>
</tr>
</table>

</div>
</div>


<div class="sidebar2">
<center>


</center>
</div>

<div class="sidebar"  style="margin-bottom:2px;">
<div class="sidebar_title">网站使用教程</div>
<div class="sidebar_info">
<a href="http://www.guoxuedashi.com/help/gjsearch.php" target="_blank">如何在国学大师网下载古籍?</a><br>
<a href="http://www.guoxuedashi.com/zidian/bujian/bjjc.php" target="_blank">如何使用部件查字法快速查字?</a><br>
<a href="http://www.guoxuedashi.com/search/sjc.php" target="_blank">如何在指定的书籍中全文检索?</a><br>
<a href="http://www.guoxuedashi.com/search/skjc.php" target="_blank">如何找到一句话在《四库全书》哪一页?</a><br>
</div>
</div>


<div class="sidebar">
<div class="sidebar_title">热门书籍</div>
<div class="sidebar_info">
<a href="/so.php?sokey=%E8%B5%84%E6%B2%BB%E9%80%9A%E9%89%B4&kt=1">资治通鉴</a> <a href="/24shi/"><strong>二十四史</strong></a>&nbsp; <a href="/a2694/">野史</a>&nbsp; <a href="/SiKuQuanShu/"><strong>四库全书</strong></a>&nbsp;<a href="http://www.guoxuedashi.com/SiKuQuanShu/fanti/">繁体</a>
<br><a href="/so.php?sokey=%E7%BA%A2%E6%A5%BC%E6%A2%A6&kt=1">红楼梦</a> <a href="/a/1858x/">三国演义</a> <a href="/a/1038k/">水浒传</a> <a href="/a/1046t/">西游记</a> <a href="/a/1914o/">封神演义</a>
<br>
<a href="http://www.guoxuedashi.com/so.php?sokeygx=%E4%B8%87%E6%9C%89%E6%96%87%E5%BA%93&submit=&kt=1">万有文库</a> <a href="/a/780t/">古文观止</a> <a href="/a/1024l/">文心雕龙</a> <a href="/a/1704n/">全唐诗</a> <a href="/a/1705h/">全宋词</a>
<br><a href="http://www.guoxuedashi.com/so.php?sokeygx=%E7%99%BE%E8%A1%B2%E6%9C%AC%E4%BA%8C%E5%8D%81%E5%9B%9B%E5%8F%B2&submit=&kt=1"><strong>百衲本二十四史</strong></a>  <a href="http://www.guoxuedashi.com/so.php?sokeygx=%E5%8F%A4%E4%BB%8A%E5%9B%BE%E4%B9%A6%E9%9B%86%E6%88%90&submit=&kt=1"><strong>古今图书集成</strong></a>
<br>

<a href="http://www.guoxuedashi.com/so.php?sokeygx=%E4%B8%9B%E4%B9%A6%E9%9B%86%E6%88%90&submit=&kt=1">丛书集成</a> 
<a href="http://www.guoxuedashi.com/so.php?sokeygx=%E5%9B%9B%E9%83%A8%E4%B8%9B%E5%88%8A&submit=&kt=1"><strong>四部丛刊</strong></a>  
<a href="http://www.guoxuedashi.com/so.php?sokeygx=%E8%AF%B4%E6%96%87%E8%A7%A3%E5%AD%97&submit=&kt=1">說文解字</a> <a href="http://www.guoxuedashi.com/so.php?sokeygx=%E5%85%A8%E4%B8%8A%E5%8F%A4&submit=&kt=1">三国六朝文</a>
<br><a href="http://www.guoxuedashi.com/so.php?sokeytm=%E6%97%A5%E6%9C%AC%E5%86%85%E9%98%81%E6%96%87%E5%BA%93&submit=&kt=1"><strong>日本内阁文库</strong></a> <a href="http://www.guoxuedashi.com/so.php?sokeytm=%E5%9B%BD%E5%9B%BE%E6%96%B9%E5%BF%97%E5%90%88%E9%9B%86&ka=100&submit=">国图方志合集</a> <a href="http://www.guoxuedashi.com/so.php?sokeytm=%E5%90%84%E5%9C%B0%E6%96%B9%E5%BF%97&submit=&kt=1"><strong>各地方志</strong></a>

</div>
</div>


<div class="sidebar2">
<center>

</center>
</div>
<div class="sidebar greenbar">
<div class="sidebar_title green">四库全书</div>
<div class="sidebar_info">

《四库全书》是中国古代最大的丛书,编撰于乾隆年间,由纪昀等360多位高官、学者编撰,3800多人抄写,费时十三年编成。丛书分经、史、子、集四部,故名四库。共有3500多种书,7.9万卷,3.6万册,约8亿字,基本上囊括了古代所有图书,故称“全书”。<a href="http://www.guoxuedashi.com/SiKuQuanShu/">详细>>
</a>

</div> 
</div>

</div>  <!--end r-->

</div>
<!-- 内容区END --> 

<!-- 页脚开始 -->
<div class="shh">

</div>

<div class="w1180" style="margin-top:8px;">
<center><script src="http://www.guoxuedashi.com/img/plus.php?id=3"></script></center>
</div>
<div class="w1180 foot">
<a href="/b/thanks.php">特别致谢</a> | <a href="javascript:window.external.AddFavorite(document.location.href,document.title);">收藏本站</a> | <a href="#">欢迎投稿</a> | <a href="http://www.guoxuedashi.com/forum/">意见建议</a> | <a href="http://www.guoxuemi.com/">国学迷</a> | <a href="http://www.shuowen.net/">说文网</a><script language="javascript" type="text/javascript" src="https://js.users.51.la/17753172.js"></script><br />
  Copyright &copy; 国学大师 古典图书集成 All Rights Reserved.<br>
  
  <span style="font-size:14px">免责声明:本站非营利性站点,以方便网友为主,仅供学习研究。<br>内容由热心网友提供和网上收集,不保留版权。若侵犯了您的权益,来信即刪。scp168@qq.com</span>
  <br />
ICP证:<a href="http://www.beian.miit.gov.cn/" target="_blank">鲁ICP备19060063号</a></div>
<!-- 页脚END --> 
<script src="http://www.guoxuedashi.com/img/plus.php?id=22"></script>
<script src="http://www.guoxuedashi.com/img/tongji.js"></script>

</body>
</html>
