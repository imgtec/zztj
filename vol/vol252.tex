\section{資治通鑑卷二百五十二}
宋 司馬光 撰

胡三省 音註

唐紀六十八|{
	起上章攝提格盡柔兆涒灘凡七年}


懿宗昭聖恭惠孝皇帝下

咸通十一年春正月甲寅朔羣臣上尊號曰睿文英武明德至仁大聖廣孝皇帝赦天下 西川之民聞蠻寇將至爭走入成都時成都但有子城亦無壕人所占地|{
	占之贍翻}
各不過一席許雨則戴箕盎以自庇又乏水取摩訶池泥汁澄而飲之|{
	成都記摩訶池在張儀子城内隋蜀王秀取土築廣子城因為池有胡僧見之曰摩訶宫毘羅盖胡僧謂摩訶為大宫毘羅為龍謂此池廣大有龍耳因名摩訶池或曰蕭摩訶所開非也池今在成都縣東南十二里}
將士不習武備節度使盧耽召彭州刺史吳行魯使攝參謀與前瀘州刺史楊慶復 |{
	考異曰新傳云瀘州刺史楊慶錦里耆舊傳云嘉州誤也今從解圍録}
共修守備選將校分職事|{
	將即亮翻下同校戶教翻}
立戰棚具礟檑|{
	棚蒲庚翻礟普教翻檑盧對翻檑木也自城上下之以壓敵}
造器備嚴警邏先是西川將士多虚職名亦無禀給|{
	先悉薦翻}
至是掲牓募驍勇之士|{
	邏郎佐翻掲丘傑翻}
補以實職厚給糧賜應募者雲集慶復乃諭之曰汝曹皆軍中子弟年少材勇|{
	少始照翻}
平居無由自進今蠻寇憑陵乃汝曹取富貴之秋也可不免乎皆歡呼踊躍於是列兵械于庭使之各試所能兩兩角勝察其勇怯而進退之得選兵三千人號曰突將行魯彭州人也戊午蠻至眉州耽遣同節度副使王偃等齎書見其用事之臣杜元忠與之約和蠻報曰我輩行止只繫雅懷 路巖韋保衡上言康承訓討龎勛時逗撓不進|{
	上時掌翻逗音豆撓奴教翻}
又不能盡其餘黨又貪虜獲不時上功|{
	上時掌翻}
辛酉貶蜀王傅分司|{
	蜀王佶皇子也 考異曰新傳云宰相路巖韋保衡劾承訓討賊逗撓貪虜獲不時上功貶蜀王傅分司東都按此時保衡未為相盖以尚主之故上用其言故得擠承訓也}
尋再貶恩州司馬 南詔進軍新津|{
	新津漢武陽縣後周改為新津唐屬蜀州九域志在州東南七十里}
定邊之北境也盧耽遣同節度副使譚奉祀致書于杜元忠問其所以來之意蠻留之不還耽遣使告急于朝|{
	朝直遥翻}
且請遣使與和以紓一時之患朝廷命知四方館事太僕卿支詳為宣諭通和使|{
	晏公類要曰舊儀於通事舍人中以宿長一人總知館事謂之館主凡四方貢納及章表皆受而進之唐自中世以後始以他官判四方館事}
蠻以耽待之恭亦為之盤桓|{
	為于偽翻}
而成都守備由是粗完|{
	粗坐五翻}
甲子蠻長驅而北陷雙流|{
	雙流漢廣都縣地隋置雙流縣唐屬成都府九域志在府南四十里}
庚午耽遣節度副使柳槃往見之杜元忠授槃書一通曰此通和之後驃信與軍府相見之儀也其儀以王者自處|{
	處昌呂翻}
語極驕慢又遣人負綵幕至城南云欲張陳蜀王廳以居驃信|{
	隋蜀王秀鎮蜀起聽事極為宏壯廳他經翻}
癸酉廢定邊軍復以七州歸西川|{
	七州邛眉蜀雅嘉黎嶲也}
是日蠻軍抵成都城下前一日盧耽遣先鋒遊奕使王晝至漢州詗援軍且趣之|{
	詗翾正翻又火迥翻趣讀曰促}
時興元六千人鳳翔四千人已至漢州會竇滂以忠武義成徐宿四千人自導江奔漢州就授軍以自存丁丑王晝以興元資簡兵三千餘人軍於毗橋|{
	毗橋在漢州南界}
遇蠻前鋒與戰不利退保漢州時成都日望援軍之至而竇滂自以失地|{
	謂失定邊軍也}
欲西川相繼陷沒以分其責每援軍自北至輒說之曰蠻衆多於官軍數十倍官軍遠來疲弊未易遽前|{
	說式芮翻易以豉翻}
諸將信之皆狐疑不進成都十將李自孝隂與蠻通欲焚城東倉為内應城中執而殺之後數日蠻果攻城久之城中無應而止二月癸未朔蠻合梯衝四面攻成都城上以鉤繯挽之使近|{
	梯雲梯衝衝車也繯于善翻屈轉其索如環鉤施於其端}
投火沃油焚之攻者皆死盧耽以楊慶復攝左都押牙李驤各帥突將出戰|{
	帥讀曰率}
殺傷蠻二千餘人會暮焚其攻具三千餘物而還蜀人素怯其突將新為慶復所奬拔且利於厚賞勇氣自倍其不得出者皆憤鬱求奮後數日賊取民籬重沓濕而屈之以為蓬|{
	重直龍翻蓬當作篷編竹以覆舟曰篷言濕籬而屈之狀如舟之眠篷也}
置人其下舉以抵城而斸之|{
	斸陟玉翻斫也掘也}
矢石不能入火不能然|{
	然與燃同燒也}
慶復鎔鐵汁以灌之攻者又死乙酉支詳遣使與蠻約和丁亥蠻歛兵請和戊子遣使迎支詳時顔慶復以援軍將至詳謂蠻使曰受詔詣定邊約和今雲南乃圍成都則與曏日詔旨異矣且朝廷所以和者冀其不犯成都也今矢石晝夜相交何謂和乎蠻見和使不至|{
	使並疏吏翻}
庚寅復進攻城|{
	復扶又翻}
辛卯城中出兵擊之乃退初韋臯招南詔以破吐蕃既而蠻訴以無甲弩臯使匠教之數歲蠻中甲弩皆精利又東蠻苴那時勿鄧夢衝三部助臯破吐蕃有功|{
	事見二百三十三卷德宗興元五年}
其後邊吏遇之無狀東蠻怨唐深自附於南詔每從南詔入寇為之盡力|{
	為于偽翻}
得唐人皆虐殺之朝廷貶竇滂為康州司戶以顔慶復為東川節度使凡援蜀諸軍皆受慶復節制癸巳慶復至新都|{
	九域志新都縣在成都府北四十五里}
蠻分兵往拒之甲午與慶復遇慶復大破蠻軍殺二千餘人蜀民數千人爭操芟刀白棓以助官軍|{
	操七刀翻芟刀農家所以芟草棓蒲項翻}
呼聲震野|{
	呼火故翻}
乙未蠻步騎數萬復至|{
	復扶又翻}
會右武衛上將軍宋威以忠武二千人至即與諸軍會戰蠻軍大敗死者五千餘人退保星宿山|{
	宿音秀}
威進軍沱江驛|{
	沱江驛在成都府新繁縣禺貢岷山導江别為沱沱徒河翻}
距成都三十里蠻遣其臣楊定保詣支詳請和詳曰宜先解圍退軍定保還蠻圍城如故城中不知援軍之至但見其數來請和|{
	數所角翻}
知援軍必勝矣戊戌蠻復請和使者十返城中亦依違答之蠻以援軍在近攻城尤急驃信以下親立矢石之間庚子官軍至城下與蠻戰奪其升遷橋|{
	升遷橋即升僊橋秦時李氷所起舊名七星橋}
是夕蠻自燒攻具遁去比明官軍乃覺之|{
	比必利翻及也}
初朝廷使顔慶復救成都命宋威屯綿漢為後繼|{
	綿漢二州名}
威乘勝先至城下破蠻軍功居多慶復疾之威飯士欲追蠻軍|{
	飯扶晩翻}
城中戰士亦欲與北軍合勢俱進慶復牒威奪其軍勒歸漢州蠻至雙流阻新穿水|{
	九域志蜀州新津縣有新穿鎮}
造橋未成狼狽失度|{
	失度者失其常度也}
三日橋成乃得過斷橋而去|{
	斷丁管翻}
甲兵服物遺弃於路蜀人甚恨之黎州刺吏嚴師本收散卒數千保邛州蠻圍之二日不克亦捨去顔慶復始教蜀人築壅門城|{
	城門之外别築垣牆以遮城門謂之壅門今人謂之八卦牆者是也}
穿塹引水滿之植鹿角分營舖|{
	斬木為鹿角植之城外以限衝突今人謂之排杈者是分立寨屋謂之營以居士卒城上分立小屋使守卒居之以候望謂之舖舖普故翻}
蠻知有備自是不復犯成都矣|{
	復扶又翻}
先是西川牙將有職無官|{
	先悉薦翻}
及拒却南詔四人以功授監察御史|{
	此所謂官也}
堂帖人輸堂例錢三百緡貧者苦之|{
	有功授官而徵其輸錢史言唐之紀綱大壞}
三月左僕射同平章事曹確同平章事充鎮海節度使 夏四月丙午以翰林學士承旨兵部侍郎韋保衡同平章事 徐賊餘黨猶相聚閭里為羣盜散居兖鄆青齊之間詔徐州觀察使夏侯瞳招諭之|{
	瞳徒紅翻}
五月丁丑以邛州刺史吳行魯為西川留後 光州民逐刺史李弱翁弱翁奔新息|{
	新息漢古縣唐屬蔡州九域志在州東南二百五十五里去光州九十里}
左補闕楊堪等上言刺史不道百姓負寃當訴於朝廷寘諸典刑豈得羣黨相聚擅自斥逐亂上下之分此風殆不可長|{
	分扶問翻長知兩翻}
宜加嚴誅以懲來者 上令百官議處置徐州之宜|{
	處昌呂翻}
六月丙午太子少傅李膠等狀以為徐州雖屢搆禍亂|{
	謂銀刀及桂州戍卒也}
未必比屋頑凶|{
	比毘必翻}
蓋由統御失人是致姦回乘釁今使名雖降|{
	謂降節度為觀察使疏吏翻}
兵額尚存以為支郡則糧餉不給分隸别藩則人心未服或舊惡相濟更成披猖惟泗州向因攻守結釁已深|{
	事見上卷九年十年}
宜有更張庶為兩便|{
	更工衡翻}
詔從之徐州依舊為觀察使統徐濠宿三州泗州為團練使割隸淮南 加幽州節度使張允伸兼侍中 秋八月乙未同昌公主薨上痛悼不已殺翰林醫官韓宗劭等二十餘人悉收捕其親族三百餘人繫京兆獄中書侍郎同平章事劉瞻召諫官使言之諫官莫敢言者乃自上言|{
	上時掌翻}
以為修短之期人之定分昨公主有疾深軫聖慈宗劭等診療之時|{
	診止忍翻候脉也療力照翻治疾也}
惟求疾愈備施方術非不盡心而禍福難移竟成差跌|{
	跌徒結翻}
原其情狀亦可哀矜而械繫老幼三百餘人物議沸騰道路嗟歎奈何以達理知命之君涉肆暴不明之謗蓋由安不慮危忿不思難之故也伏願少回聖慮寛釋繫者上覽疏不悦瞻又與京兆尹温璋力諫於上前上大怒叱出之 魏博節度使何全皥年少驕暴好殺|{
	少詩照翻好呼到翻}
又減將士衣糧將士作亂全皥單騎走追殺之|{
	何進滔得魏博傳三世四十二年而滅}
推大將韓君雄為留後成德節度使王景崇為之請旌節|{
	為之于偽翻}
九月庚戍以君雄為魏博留後 丙辰以劉瞻同平章事充荆南節度使貶温璋振州司馬璋歎曰生不逢時死何足惜是夕仰藥卒|{
	仰牛向翻}
勑曰苟無蠧害何至於斯惡實貫盈死有餘責宜令三日内且於城外權瘞|{
	瘞於計翻}
俟經恩宥方許歸葬使中外快心姦邪知懼己巳貶右諫議大夫高湘比部郎中知制誥楊知至禮部郎中魏簹等於嶺南|{
	比音毘簹都郎翻}
皆坐與劉瞻親善為韋保衡所逐也知至汝士之子|{
	汝士虞卿從兄也見二百四十一卷穆宗長慶元年}
簹扶之子也|{
	魏扶見二百四十八卷宣宗大中三年}
保衡又與路巖共奏劉瞻云與醫官通謀誤投毒藥|{
	譛言誤投毒藥以致同昌公主於死然既言誤矣又安可以為通謀邪}
丙子貶瞻康州刺史|{
	康州去京師五千七百五十里}
翰林學士承旨鄭畋草瞻罷相制辭曰安數畝之居仍非已有却四方之賂惟畏人知巖謂畋曰侍郎乃表薦劉相也坐貶梧州刺史|{
	梧州漢蒼梧郡所治廣信縣地唐置梧州去京師五千五百里}
御史中丞孫瑝坐為瞻所引用亦貶汀州刺史|{
	瑝戶盲翻又音皇}
路巖素與劉瞻論議多不叶瞻既貶康州巖猶不快閱十道圖以驩州去長安萬里再貶驩州司戶|{
	驩州陸路至長安一萬二千四百五十二里水路一萬七千里 考異曰實録新傳皆云巖志欲殺之賴幽州節度使張公素表論瞻寃乃止按是時張允伸鎮幽州云公素恐誤也}
冬十月癸卯以西川留後吳行魯為節度使 十一月辛亥以兵部尚書鹽鐵轉運使王鐸為禮部尚書同平章事鐸起之兄子也|{
	王起見二百四十一卷長慶元年鐸起兄炎之子}
丁卯復以徐州為感化軍節度|{
	徐州本武寧軍中有銀刀之亂罷節鎮為觀察今復為感化軍}
十二月加成德節度使王景崇同平章事以左金吾上將軍李國昌為振武節度使十二年春正月辛酉葬文懿公主|{
	同昌公主諡文懿}
韋氏之人爭取庭祭之灰汰其金銀|{
	勑祭之於韋氏之庭故曰庭祭汰淘也}
凡服玩每物皆百二十輿以錦繡珠玉為儀衛明器輝煥三十餘里|{
	記檀弓孔子謂為明器者知喪道矣備物而不可用也其曰明器神明之也何取於輝煥乎}
賜酒百斛餅餤四十槖駝以飼体夫|{
	餤于廉翻又徒甘翻飼祥吏翻体蒲本翻体夫轝柩之夫也}
上與郭淑妃思公主不已樂工李可及作歎百年曲其聲悽惋|{
	歎百年曲歷叙人自少而壯自壯而老少時姢好壯時追歡極樂老時衰颯之狀其聲悽切感動人心惋烏貫翻}
舞者數百人内庫雜寶為其首飾以絁八百匹為地衣舞罷珠璣覆地|{
	絁式支翻覆敷救翻}
以魏博留後韓君雄為節度使 門下侍郎同平章事路巖與韋保衡素相表裏勢傾天下既而爭權浸有隙保衡遂短巖於上夏四月癸卯以巖同平章事充西川節度使巖出城路人以瓦礫擲之|{
	礫郎狄翻}
權京兆尹薛能巖所擢也巖謂能曰臨行煩以瓦礫相餞能徐舉笏對曰曏來宰相出府司無例發人防衛|{
	府司謂京兆府所司}
巖甚慙能汾州人也|{
	能音裳來切}
五月上幸安國寺賜僧重謙僧澈沈檀講座二各高二丈|{
	以沈香檀香為講座也沈持林翻高古號翻}
設萬人齋 秋七月以兵部尚書盧耽同平章事充山南東道節度使 冬十月以兵部侍郎鹽鐵轉運使劉鄴為禮部尚書同平章事

十三年春正月幽州節度使張允伸得風疾請委軍政就醫許之以其子簡會知留後疾甚遣使上表納旌節丙申薨允伸鎮幽州二十三年|{
	宣宗大中四年張允伸代周綝鎮幽州}
勤儉恭謹邊鄙無警上下安之 二月丁巳以兵部侍郎同平章事于琮為山南東道節度使以刑部侍郎判戶部奉天趙隱為戶部侍郎同平章事 平州刺史張公素素有威望為幽人所服張允伸薨公素帥州兵來奔喪|{
	帥讀曰率}
張簡會懼三月奔京師以為諸衛將軍|{
	汎言諸衛將軍不言何衛史畧之也}
夏四月立皇子保為吉王傑為壽王倚為睦王 以張公素為平盧留後|{
	平盧當作盧龍}
五月國子司業韋殷裕詣閤門告郭淑妃弟内作坊使敬述隂事|{
	内作坊使内諸司使之一掌造内庫軍器}
上大怒杖殺殷裕籍沒其家 |{
	考異曰續寶運録曰内作使郭敬述與宰臣韋保衡張能順頻於内宅飲酒濳通郭妃荒穢頗甚每封進文書於金合内詐稱果子内連郭妃郭敬述外結張能順國子司業韋殷裕擬傾皇祚别立太子事泄遽加貶降五月十四日内牓子貶工部尚書嚴祈郴州刺史給事中李貺勤州刺史給事中張鐸滕州刺史左金吾大將軍李敬仲儋州司戶國子司業韋殷裕勑京兆府決痛杖一頓處死家貲妻女沒官又貶叙州刺史韋君卿愛州崇平縣尉右僕射右羽林統軍張直方康州司馬續又貶駙馬于琮並扶會與韋保衛同謀不軌事其月十七日又貶尚書左丞李當道州刺史吏部侍郎王諷建州刺史左常侍李都賀州刺史翰林承旨張裼封州司馬中書舍人封彦卿潮州司戶諫議大夫楊塾新州司戶駙馬韋保衡雷州刺史又貶儋州澄邁縣尉又貶驩州長流百姓又賜自盡家貲沒官仍三族不許朝廷録用其語雜亂無稽今從實録}
乙亥閤門使田獻銛奪紫改橋陵使|{
	銛思亷翻}
以其受殷裕狀故也殷裕妻父太府少卿崔元應妻從兄中書舍人崔沆|{
	從才用翻沆下黨翻}
季父君卿皆貶嶺南官給事中杜裔休坐與殷裕善亦貶端州司戶沆鉉之子也|{
	崔鉉見二百四十七卷武宗會昌三年}
裔休悰之子也 丙子貶山南東道節度使于琮為普王傅分司|{
	普王儼皇子也後踐阼是為僖宗}
韋保衡譛之也辛巳貶尚書左丞李當吏部侍郎王渢|{
	渢房戎翻}
左散騎常侍李都翰林學士承旨兵部侍郎張裼|{
	裼他計翻又先擊翻}
前中書舍人封彦卿左諫議大夫楊塾癸未貶工部尚書嚴祁給事中李貺給事中張鐸左金吾大將軍李敬仲起居舍人蕭遘李瀆鄭彦特李藻皆處之湖嶺之南|{
	處昌呂翻不詳言各人所貶之地以其無罪故畧之也}
坐與琮厚善故也貺漢之子遘寘之子也|{
	李漢見二百四十五卷文宗太和九年蕭寘見二百五十卷五年}
甲申貶前平盧節度使于琄為涼王府長史分司|{
	琄胡大翻涼王侹皇子也}
前湖南觀察使于瓌為袁州刺史瓌琄皆琮之兄也尋再貶琮韶州刺史|{
	隋於曲江縣置韶州以縣北八十里韶石為名至京師四千九百三十二里}
琮妻廣德公主上之妹也|{
	廣德縣屬宣州}
與琮偕之韶州行則肩輿門相對坐則執琮之帶琮由是獲全時諸公主多驕縱惟廣德動遵法度事于氏宗親尊卑無不如禮内外稱之 六月以盧龍留後張公素為節度使 韋保衡欲以其黨裴條為郎官憚左丞李璋方巖恐其不放上|{
	尚書左右丞分總六曹二十四司郎官凡除授非其人左右丞得以糾劾之不令赴省供職上時掌翻}
先遣人達意璋曰朝廷遷除不應見問 秋七月乙未以璋為宣歙觀察使|{
	歙書涉翻}
八月歸義節度使張義潮薨沙州長史曹義金代領軍府制以義金為歸義節度使是後中原多故朝命不及回鶻陷甘州自餘諸州隸歸義者多為羌胡所據|{
	自唐末迄于宋朝河湟之地遂悉為戎中國不能復取朝直遥翻}
冬十二月追上宣宗諡曰元聖至明成武獻文睿智章仁神聰懿道大孝皇帝|{
	上時掌翻}
振武節度使李國昌恃功恣横|{
	横戶孟翻}
專殺長吏|{
	長知丈翻}
朝廷不能平徙國昌為大同軍防禦使國昌稱疾不赴|{
	史言沙陀跋扈不待殺段文楚而後動於惡}


十四年春三月癸巳上遣勑使詣法門寺迎佛骨羣臣諫者甚衆至有言憲宗迎佛骨尋晏駕者|{
	事見憲宗紀元和十四年死者人所甚諱也况言之於人主之前乎言之至此人所難也}
上曰朕生得見之死亦無恨廣造浮圖寶帳香轝幡花幢蓋以迎之|{
	幡孚袁翻幢傳江翻史炤曰幢童也其狀童童然}
皆飾以金玉錦繡珠翠自京城至寺三百里間道路車馬晝夜不絶夏四月壬寅佛骨至京師導以禁軍兵仗公私音樂沸天燭地綿亘數十里儀衛之盛過於郊祀元和之時不及遠矣富室夾道為綵樓及無遮會競為侈靡上御安福門降樓膜拜|{
	膜拜胡禮拜也膜莫乎翻}
流涕霑臆賜僧及京城耆老嘗見元和事者金帛迎佛骨入禁中三日出置安國崇化寺宰相已下競施金帛不可勝紀|{
	施式䜴翻勝音升}
因下德音降中外繫囚 五月丁亥以西川節度使路巖兼中書令 |{
	考異曰錦里耆舊傳十二年八月路公用邊咸郭籌策奏於邛州置定邊軍節度使復制扼大渡河脩邛崍關南路米點檀丁子弟教之斫刺刀補義軍將主管教練兵士新傳巖至西川承蠻盗邊後巖力拊循置定邊軍於邛州扼大渡治故關取檀丁子弟教擊刺補屯籍由是西山八國來朝以勞遷兼中書令按置定邊軍乃李師望耆舊傳新傳皆誤也}
南詔寇西川又寇黔南黔中經畧使秦匡謀兵少不敵弃城奔荆南|{
	黔渠今翻少詩沼翻}
荆南節度使杜悰囚而奏之六月乙未勑斬匡謀籍沒其家貲親族應緣坐者令有司搜捕以聞匡謀鳳翔人也 以中書侍郎同平章事王鐸同平章事充宣武節度使時韋保衡挾恩弄權以劉瞻于琮先在相位不禮於己譛而逐之王鐸保衡及第時主文也|{
	唐禮部校文主司謂之主文}
蕭遘同年進士也二人素薄保衡之為人保衡皆擯斥之 |{
	考異曰舊傳云保衡以楊收路巖在中書不加禮接媒蘖逐之按收獲罪時保衡未為相盖保衡雖為學士懿宗寵任之故能譛收也又曰公主薨自後恩禮漸薄按路巖于琮王鐸蕭遘被擯皆在公主薨後今從實録}
秋七月戊寅上疾大漸左軍中尉劉行深右軍中尉韓文約立少子普王儼|{
	考異曰范質五代通録梁李振謂陜州護軍韓彛範曰懿皇初升遐韓中尉殺長立少以利其權遂亂天下今將軍復欲爾邪彛範即文約孫也按懿宗八子僖宗第五餘子新舊書不載長幼又不言所終不言所殺者果何王也}
庚辰制立儼為皇太子 |{
	考異曰續寶運録云其日宰臣蕭鄴等直至寢幄問疾上微道朕三字而止羣臣不覺號哭失聲中外悉皆垂泣按是時宰相韋保衡最在上蕭鄴不為相今不取}
權句當軍國政事|{
	句古侯翻當丁浪翻}
辛巳上崩于咸寧殿|{
	年四十一}
遺詔以韋保衡攝冢宰僖宗即位八月丁未追尊母王貴妃為皇太后劉行深韓文約皆封國公 關東河南大水 九月有司上先太后諡曰惠安|{
	先太后謂上母王貴妃也上時掌翻}
司徒門下侍郎同平章事韋保衡怨家告其陰事貶保衡賀州刺史樂工李可及流嶺南可及有寵於懿宗嘗為子娶婦|{
	為于偽翻}
懿宗賜之酒二銀壺啓之無酒而中實右軍中尉西門季玄屢以為言懿宗不聽可及嘗大受賜物載以官車季玄謂曰汝它日破家此物復應以官車載還非為受賜徒煩牛足耳及流嶺南籍沒其家果如季玄言|{
	史言小人寵過而禍及}
以西川節度使路巖兼侍中加成德節度使王景崇中書令魏博節度使韓君雄盧龍節度使張公素天平節度使高駢並同平章事君雄仍賜名允中 |{
	考異曰舊傳作允忠實録新傳皆作允中今從之}
冬十月乙未以左僕射蕭倣為門下侍郎同平章事 韋保衡再貶崖州澄邁令|{
	澄邁隋縣唐屬瓊州九域志在州西五十五里}
尋賜自盡又貶其弟翰林學士兵部侍郎保乂為賓州司戶所親翰林學士戶部侍郎劉承雍為涪州司馬|{
	涪音浮}
承雍禹錫之子也|{
	劉禹錫見二百三十六卷順宗永貞元年}
癸卯赦天下 西川節度使路巖喜聲色遊宴|{
	喜許記翻}
委軍府政事於親吏邊咸郭籌皆先行後申上下畏之嘗大閱二人議事默書紙相示而焚之軍中以為有異圖驚懼不安朝廷聞之十一月戊辰徙巖荆南節度使咸籌濳知其故遂亡命以右僕射蕭鄴同平章事充河東節度使 十二月

己亥詔送佛骨還法門寺 再貶路巖為新州刺史僖宗惠聖恭定孝皇帝上之上|{
	初名儼改名儇懿宗第五子}


乾符元年|{
	是年十一月方改元}
春正月丁亥翰林學士盧攜上言|{
	上時掌翻}
以為陛下初臨大寶宜深念黎元國家之有百姓如草木之有根柢|{
	抵典禮翻又下計翻}
若秋冬培溉則春夏滋榮臣竊見關東去年旱災自虢至海|{
	自虢州東至于海也}
麥纔半收秋稼幾無冬菜至少貧者磑蓬實為麵蓄槐葉為韲或更衰羸亦難收拾|{
	幾居依翻少詩沼翻磑五對翻䃺也麵眠見翻韲牋西翻羸倫為翻}
常年不稔則散之鄰境|{
	之往也}
今所在皆饑無所依投坐守鄉閭待盡溝壑其蠲免餘稅實無可徵而州縣以有上供及三司錢|{
	戶部轉運鹽鐵為三司}
督趣甚急|{
	趣讀曰促}
動加捶撻雖撤屋伐木雇妻鬻子止可供所由酒食之費|{
	所由謂催督租稅之吏卒}
未得至於府庫也或租稅之外更有他徭朝廷儻不撫存百姓實無生計乞勑州縣應所欠殘稅並一切停徵以俟蠶麥仍所在義倉亟加賑給|{
	太宗置義倉及常平倉以備凶荒高宗以後稍假義倉以給他費至神龍中畧盡玄宗即位復置之安史之亂復廢至文宗太和九年以天下回殘錢置常平義倉本錢歲增市之以備賑給}
至深春之後有菜葉木牙繼以桑椹漸有可食在今數月之間尤為窘急行之不可稽緩勑從其言而有司竟不能行徒為空文而已 路巖行至江陵勑削官爵長流儋州|{
	儋都甘翻}
巖美姿儀囚於江陵獄再宿須髪皆白尋賜自盡籍沒其家巖之為相也密奏三品以上賜死皆令使者剔取結喉三寸以進|{
	結㗋㗋嚨上下相接之處}
驗其必死至是自罹其禍所死之處乃楊收賜死之榻也|{
	史言天之報應不爽楊收賜死見上卷咸通十年}
邊咸郭籌捕得皆伏誅初巖佐崔鉉於淮南為支使|{
	唐制節度使幕屬有掌書記觀察有支使以掌表牋書翰亦書記之任也}
鉉知其必貴曰路十終須作彼一官|{
	巖第十作彼一官謂作相也}
既而入為監察御史不出長安城十年至宰相其自監察入翰林也鉉猶在淮南聞之曰路十今已入翰林如何得老皆如鉉言 以太子少傅于琮同平章事充山南東道節度使 二月甲午葬昭聖恭惠孝皇帝于簡陵|{
	簡陵在京兆富平縣西北四十五里}
廟號懿宗 以中書侍郎同平章事趙隱同平章事充鎮海節度使以華州刺史裴坦為中書侍郎同平章事|{
	華戶化翻}
以虢州刺史劉瞻為刑部尚書瞻之貶也|{
	懿宗咸通十二年瞻貶}
人無賢愚莫不痛惜及其還也長安兩市人率錢雇百戲迎之|{
	長安城中分東西兩市}
瞻聞之改期由它道而入 |{
	考異曰玉泉子見聞録曰初瞻南遷無問賢不肖一口皆為之痛惜殆將至京東西市豪俠共率泉帛募集百戲將逆於城外瞻知之差期而易路焉瞻為相亦無它才能徒以路巖遭時嫉怒瞻為所排而人心歸向耳其實未足譚也按瞻以清慎著聞及懿宗暴怒瞻獨能不顧其身救數百人之死而玉泉子以為未足談不亦誣乎}
夏五月乙未裴坦薨以劉瞻為中書侍郎同平章事初瞻南遷劉鄴附於韋路共短之|{
	韋路謂韋保衡路巖}
及瞻還為相鄴内懼秋八月丁巳朔鄴延瞻置酒於鹽鐵院|{
	劉鄴以鹽鐵轉運使為相故延劉瞻宴於鹽鐵院}
瞻歸而遇疾辛未薨時人皆以為鄴鴆之也 以兵部侍郎判度支崔彦昭為中書侍郎同平章事彦昭羣之從子也|{
	崔羣相憲穆從才用翻}
兵部侍郎王凝正雅之從孫也|{
	王正雅見二百四十四卷文宗太和五年}
其母彦昭之從母|{
	母之姊妹謂之從母}
凝彦昭同舉進士凝先及第嘗衩衣見彦昭|{
	衩差賣翻衩衣便服不具禮也}
且戲之曰君不若舉明經彦昭怒遂為深仇|{
	唐世重進士而輕明經故當時有焚香禮進士設幕試明經之語崔彦昭之仇怒王凝盖以此也}
及彦昭為相其母謂侍婢曰為我多作韈履|{
	為我于偽翻}
王侍郎母子必將竄逐吾當與妹偕行彦昭拜且泣謝曰必不敢凝由是獲免|{
	考異曰此出中朝故事曰彦昭代凝判鹽鐵半載而入相按實録彦昭不代凝為鹽鐵其餘則取之}
冬十月以門下侍郎同平章事劉鄴同平章事充淮南節度使以吏部侍郎鄭畋為兵部侍郎翰林學士承旨戶部侍郎盧攜守本官並同平章事 |{
	考異曰舊畋傳云乾符四年遷吏部侍郎尋降制可本官同平章事今從實録此年為相}
十一月庚寅日南至|{
	冬至日南至夏至日北至}
羣臣上尊號曰聖神聰睿仁哲孝皇帝改元|{
	改元乾符上時掌翻}
魏博節度使韓允中薨軍中立其子節度副使簡為留後 南詔寇西川作浮梁濟大度河防河都知兵馬使黎州刺史黄景復俟其半濟擊之蠻敗走斷其浮梁|{
	斷丁管翻}
蠻以中軍多張旗幟當其前而分兵濳出上下流各二十里夜作浮梁詰朝俱濟|{
	詰其吉翻朝陟遥翻旦也}
襲破諸城栅夾攻景復力戰三日景復陽敗走蠻盡銳追之景復設三伏以待之蠻過三分之二乃伏擊之蠻兵大敗殺二千餘人追至大度河南而還|{
	還從宣翻又如字}
復修完城栅而守之蠻歸至之羅谷遇國中兵繼至新舊相合|{
	新者繼至之兵舊者敗歸之兵}
鉦鼓聲聞數十里|{
	聞音問}
復寇大度河|{
	復扶又翻}
與唐夾水而軍詐云求和又自上下流濳濟與景復戰連日西川援軍不至而蠻衆日益景復不能支軍遂潰 十二月党項回鶻寇天德軍 感化軍奏羣盜寇掠|{
	感化軍治徐州羣盜龎勛餘黨也}
州縣不能禁勑兖鄆等道出兵討之 南詔乘勝陷黎州入邛崍關攻雅州大度河潰兵奔入邛州|{
	九域志雅州東北至邛州一百六十里大度河潰兵黄景復之軍也}
成都驚擾民爭入城或北奔它州城中大為守備而塹壘比曏時嚴固驃信使其坦綽遺節度使牛叢書云|{
	坦綽南詔清平官之首也遺唯季翻}
非敢為寇也欲入見天子面訴數十年為讒人離間寃抑之事|{
	見賢遍翻間古莧翻}
儻蒙聖恩矜恤當還與尚書永敦鄰好今假道貴府欲借蜀王廳留止數日即東上|{
	好呼到翻上時掌翻詐言將自成都而東上長安}
叢素懦怯欲許之楊慶復以為不可斬其使者留二人授以書遣還書辭極數其罪詈辱之|{
	數所具翻}
蠻兵及新津而還|{
	宋白曰新津縣本漢犍為郡武陽縣地李膺益州記云皂里江津之所曰新津市周北圖記云閔帝元年於此立新建縣九域志縣在蜀州東南七十里}
叢恐蠻至豫焚城外民居蕩盡 |{
	考異曰錦里耆舊傳咸通十四年十一月五日雲南蠻寇再犯大度河黄景復擊敗之十二月二十五日復攻大度河三十日蠻乘勝進攻黎州十二月二十八日蠻來只到新津前後蜀州界左右便退竟不到城下按咸通十四年南詔寇西川事舊紀南詔傳唐年補録唐録備闕讀寶運録皆無之獨耆舊傳載之甚詳新書取之作南詔傳而實録但云十二月西川奏南蠻入寇黎州刺史黄景復擊退之新紀但云十二月雲南蠻寇黎州蓋亦出於耆舊傳耳舊紀乾符元年冬南詔蠻寇西蜀詔河西河東山南西道東川徵兵赴援實録乾符元年十月西川奏雲南蠻入寇十二月雲南蠻寇西川坦綽致書於牛叢欲求入覲河東山南西道及東川兵援之月末又云南蠻侵犯黎州而成都守禦無備殊不拒敵踰河越嶺洞無籬障賴積雪丈餘遂阻隔奔衝之勢又邛雅二州刺史望風奔遁蠻燒劫一空牛叢不曉兵失於探候而奏報差戾詔切責之蠻劫畧黎雅間破黎州入邛崍關成都閉三日蠻乃去新紀乾符元年十二月雲南蠻寇黎雅二州河西河東山南東道東川兵伐雲南按實録咸通十四年十一月七日路巖始移荆南八日牛叢始除西川而耆舊傳蠻入寇皆叢任内事恐誤先一年也實録新紀因此於十四年十二月添雲南寇黎州事實皆在乾符元年冬也}
蜀人尤之詔河東山南西道東川兵援之仍命天平節度使高駢詣西川制置蠻事 以韓簡為魏博留後 商州刺史王樞以軍州空窘減折糴錢|{
	窘巨隕翻德宗時度支以稅物頒諸司皆增本價為虚估給之而繆以濫惡督州縣剥價謂之折納其後又以稅物折錢使輸米粟謂之折糴折音之舌翻}
民相帥以白梃之|{
	帥讀曰率梃徒鼎翻白棓也敺烏口翻}
又殺官吏二人朝廷更除刺史李誥到官收捕民李叔汶等三十餘人斬之|{
	汶音問}
初回鶻屢求册命詔遣册立使郗宗莒詣其國會回鶻為吐谷渾嗢末所破|{
	嗢末者吐蕃奴部也虜法出師必豪室皆以奴從平居散處田牧及論恐熱亂無所歸共相嘯合數千人以嗢末自號居甘肅河沙瓜渭岷廓疊宕間其近蕃牙者最勇而馬尤良嗢烏沒翻}
逃遁不知所之詔宗莒以玉册國信授靈鹽節度使唐弘夫掌之還京師 上年少|{
	少詩沼翻}
政在臣下南牙北司互相矛盾自懿宗以來奢侈日甚用兵不息賦歛愈急|{
	歛九贍翻}
關東連年水旱州縣不以實聞上下相蒙百姓流殍|{
	流散也殍餓殍殍音被表翻}
無所控訴相聚為盜所在蠭起州縣兵少加以承平日久人不習戰每與盜遇官軍多敗|{
	是從王仙芝黄巢遂為大盜史先言唐末所以致盜之由}
是歲濮州人王仙芝始聚衆數千起於長垣|{
	滑州匡城縣本後齊之長垣縣開皇十六年改為匡城是年又分韋城縣置長垣縣新志匡城有長垣縣宋朝以長垣縣屬開封府九域志在府東北一百五里 考異曰實録二年五月仙芝反於長垣按續寶運録濮州賊王仙芝自稱天補平均大將軍兼海内諸豪都統傳檄諸道檄末稱乾符一年正月三日則仙芝起必在二年前今置於歲末}


二年春正月丙戌以高駢為西川節度使 辛巳上祀圓丘赦天下 高駢至劒州先遣使走馬開成都門|{
	開成都城諸門也 考異曰錦里耆舊傳鄆州節度使高相公駢乘急詔除劒南西州節度副大使乾符元年正月二十一日行李到劒州先遣使走馬開城門並令放出百姓二月十六日至府豁開城門並放人出今從實録置今年又劒州至成都止十二程駢正月二十一日自劒州遣使走馬開城門二月十六日始至府下又云駢三十日到上按長歷二月小無三十日蓋二十六日誤為二月十六日也}
或曰蠻寇逼近成都相公尚遠萬一豨突奈何|{
	近其靳翻豨香衣翻又許豈翻豨豕也豕健於突}
駢曰吾在交趾破蠻二十萬衆|{
	事見二百五十卷懿宗咸通七年}
蠻聞我來逃竄不暇何敢輒犯成都今春氣向暖數十萬人藴積城中生死共處汚穢鬱蒸將成癘疫不可緩也使者至成都開城縱民出各復常業乘城者皆下城解甲民大悦蠻方攻雅州聞之遣使請和引兵去駢又奏南蠻小醜易以枝梧|{
	易以䜴翻}
今西川新舊兵已多所長武鄜坊河東兵徒有勞費並乞勒還勑止河東兵而已|{
	考異曰舊紀此奏在元年十二月今因駢開成都門言之}
上之為普王也小馬坊使田令孜有寵|{
	小馬坊使亦内諸司使之一後梁改為天驥使後唐復舊長興元年改飛龍院為左飛龍院小馬坊為右飛龍院宋太平興國三年改左右天廐坊至雍熙二年又改左右騏驥院使}
及即位使知樞密遂擢為中尉 |{
	考異曰舊本紀此年正月令孜為右軍中尉新傳云帝即位擢為左神策中尉舊傳但云神策中尉今從之}
上時年十四專事遊戲|{
	考異曰續寶運録曰上是年十五歲中朝故事曰僖宗皇帝以咸通三年降誕十四年七月十九日即位年十二按舊紀亦云咸通三年五月八日生於東内即位年十二今從之 㨿考異四當作二}
政事一委令孜呼為阿父|{
	阿烏葛翻一讀如字阿保也}
令孜頗讀書多巧數招權納賄除官及賜緋紫皆不關白於上每見常自備果食兩盤與上相對飲啗從容良久而退|{
	見賢遍翻從千容翻}
上與内園小兒狎昵賞賜樂工伎兒所費動以萬計府藏空竭|{
	昵尼質翻伎渠綺翻藏徂浪翻}
令孜說上籍兩市商旅寶貨悉輸内庫|{
	說式芮翻}
有陳訴者付京兆杖殺之宰相以下鉗口莫敢言|{
	鉗其亷翻}
高駢至成都明日步騎五千追南詔至大度河殺獲甚衆擒其酋長數十人至成都斬之|{
	酋慈由翻長知丈翻}
修復邛崍關大度河諸城柵又築城于戎州馬湖鎮號平夷軍|{
	馬湖鎮當馬湖江之要}
又築城于沐源川皆蠻入蜀之要路也各置兵數千戍之自是蠻不復入寇|{
	復扶又翻}
駢召黄景復責以大度河失守腰斬之 |{
	考異曰耆舊傳云乾符元年三月十五日處置前黎州刺史充大度河把截制置土軍都知兵馬使黄景復實録乾符二年三月駢奏斬景復今事從耆舊傳年從實録}
駢又奏請自將本管|{
	本管謂西川兵}
及天平昭義義成等軍共六萬人擊南詔詔不許先是南詔督爽屢牒中書|{
	南詔清平官坦綽布爕久贊之下有幕爽主兵琮爽主戶籍慈爽主禮罰爽主刑勸爽主官人厥爽主工作萬爽主財用引爽主客禾爽主商賈亦皆清平官爽猶言省也督爽摠三省也先悉薦翻}
辭語怨望中書不答盧攜奏稱如此則蠻益驕謂唐無以答宜數其十代受恩以責之|{
	南詔之先曰細奴邏高宗朝遣使入朝生邏盛炎邏盛炎生炎閤炎閤死弟盛邏皮立盛邏皮生皮邏閤玄宗賜名歸義於開元間合六詔為一而國始強歸義子曰閤羅鳳閤羅鳳子曰鳳迦異鳳迦異子曰異牟尋異牟尋子曰尋閤勸尋閤勸子曰勸龍晟勸龍晟弟曰勸利勸利弟曰豐祐豐祐死而酋龍立自細奴邏至酋龍十三代中間鳳迦異未立而死而豐祐酋龍與唐為敵是受恩十代也數所具翻}
然自中書牒則嫌於體敵請賜高駢及嶺南節度使辛讜詔使録詔白牒與之|{
	録詔白今謂之録白是也}
從之三月以魏博留後韓簡為節度使 去歲感化軍

兵詣靈武防秋會南詔寇西川勑往救援蠻退遣還至鳳翔不肯詣靈武欲擅歸徐州内養王裕本都將劉逢搜擒唱帥者胡雄等八人斬之|{
	内養亦宦者也帥讀曰率下同}
衆然後定 初南詔圍成都楊慶復以右職優給募突將以禦之|{
	事見上懿宗咸通十一年將即亮翻下同}
成都由是獲全及高駢至悉令納牒|{
	牒職牒也}
又託以蜀中屢遭蠻寇人未復業停其禀給|{
	既奪其職牒又停其優給}
突將皆忿怨駢好妖術每兵追蠻皆夜張旗立隊對將士焚紙畫人馬|{
	好呼到翻妖於遥翻畫讀曰畫}
散小豆曰蜀兵懦怯今遣玄女神兵前行軍中壯士皆恥之|{
	高駢之好妖術終以此敗}
又索闔境官有出於胥吏者皆停之|{
	索山客翻下索之同}
令民間皆用足陌錢陌不足者皆執之劾以行賂取與皆死|{
	劾舊音戶槩翻今紇得翻}
刑罰嚴酷由是蜀人皆不悅夏四月突將作亂大譟突入府廷駢走匿於厠間|{
	厠初刺翻圊也溷也}
突將索之不獲|{
	索山客翻}
天平都將張傑帥所部數百人被甲入府擊突將|{
	高駢自天平從西川張傑盖元從部曲將被皮義翻}
突將撤牙前儀注兵仗|{
	節度使牙前列兵仗以壯威容}
無者奮梃揮拳乘怒氣力鬭天平軍不能敵走歸營突將追之營門閉不得入監軍使人招諭許以復職名禀給久之乃肯還營天平軍復開門出為追逐之勢至城北時方修毬塲役者數百人天平軍悉取其首還詣府云已誅亂者駢出見之厚以金帛賞之明日牓謝突將悉還其職名衣糧自是日令諸道將士從已來者更直府中嚴兵自衛|{
	備突將復為亂也更工衡翻}
加成德節度使王景崇兼侍中 浙西狼山鎮遏使王郢等六十九人有戰功|{
	今通州静海縣南有狼山五山相連上接大江下達巨海絶江南渡抵蘇州常熟縣福山鎮順江東至崇明沙揚帆乘順南抵明州定海縣陶隱居所謂狼五山對句章岸者也}
節度使趙隱賞以職名而不給衣糧郢等論訴不獲|{
	論盧昆翻}
遂劫庫兵作亂行收黨衆近萬人攻陷蘇常|{
	蘇常二州名相去一百八十里}
乘舟往來泛江入海轉掠二浙南及福建大為人患|{
	考異曰新紀云浙西突陳將王郢反五月遣右龍武大將軍宋皓討之按四年郢執魯寔始命皓討之置此誤也程匡柔唐補記曰浙西突將王郢反聚黨萬衆燒劫蘇常三年正月貶蘇州刺史李繪以郢亂弃城故也舊紀二年海賊王郢攻剽浙西郡邑寔録乾符三年二月浙西奏突陳將王郢等六十九人劫庫兵為亂三月浙西奏王郢聚衆萬人攻陷州縣續寶運録元年玉郢於兩浙叛敕差山北兵士討之不逾月而尅乃組頸于闕下今從舊紀}
五月以太傅分司令狐綯同平章事充鳳翔節度使 司空同平章事蕭倣薨 |{
	考異曰舊傳曰俄而盜起河南内官握兵王室濁亂倣氣勁論直同列忌之罷知政事出為廣州刺史嶺南節度使遇亂不至京師而卒舊紀三年春正月己卯朔倣以病免罷為太子太傅新紀此月蕭倣薨新傳亦云卒于位為嶺南節度在前舊紀傳皆誤今從實録}
六月以御史大夫李蔚為中書侍郎同平章事|{
	蔚紆勿翻}
辛未高駢隂籍突將之名使人夜掩捕之圍其家挑牆壞戶而入|{
	挑它凋翻蜀本作排讀如字壞音怪}
老幼孕病悉驅去殺之嬰兒或撲於階|{
	撲弼角翻}
或擊於柱流血成渠號哭震天|{
	號戶高翻}
死者數千人夜以車載尸投之於江有一婦人臨刑戟手大罵曰高駢汝無故奪有功將士職名衣糧激成衆怒幸而得免不省已自咎|{
	省悉景翻}
乃更以詐殺無辜近萬人|{
	近其靳翻}
天地鬼神豈容汝如此我必訴汝於上帝使汝它日舉家屠滅如我今日寃抑汙辱如我今日驚憂惴恐如我今日言畢拜天怫然就戮|{
	惴之睡翻怫扶弗翻觀異日高駢之禍則信如婦人之言矣天地鬼神臨之在上質之在旁豈可多殺無辜以逞私忿}
久之突將有自戍役歸者駢復欲盡族之|{
	復扶又翻下同}
有元從親吏王殷諫曰相公奉道宜好生惡殺|{
	元從言從高駢歲久非新隸也從才用翻好呼到翻惡烏路翻}
此屬在外初不同謀若復誅之則自危者多矣駢乃止 王仙芝及其黨尚君長攻陷州曹州|{
	博木翻}
衆至數萬天平節度使薛崇出兵擊之為仙芝所敗|{
	敗補邁翻}
寃句人黄巢亦聚衆數千人應仙芝|{
	寃句漢縣唐屬曹州九域志在州西四十五里顔師古曰句音胊黄巢始此}
巢少與仙芝皆以販私鹽為事|{
	少詩照翻}
巢善騎射喜任俠粗涉書傳|{
	喜許紀翻粗坐五翻傳柱戀翻}
屢舉進士不第遂為盜與仙芝攻剽州縣|{
	剽匹妙翻}
横行山東民之困於重歛者爭歸之|{
	歛力贍翻}
數月之間衆至數萬 盧龍節度使張公素性暴戾不為軍士所附大將李茂勲本回鶻阿布思之族回鶻敗降於張仲武|{
	李茂勲之降盖在會昌間也}
仲武使戍邊屢有功賜姓名納降軍使陳貢言者幽之宿將為軍士所信服|{
	納降軍在幽州丁零川}
茂勲濳殺貢言聲云貢言舉兵向薊公素出戰而敗奔京師茂勲入城衆乃知非貢言也不得已推而立之朝廷因以為留後 秋七月蝗自東而西蔽日所過赤地|{
	言蝗之多所過食草木葉及五穀皆盡}
京兆尹楊知至奏蝗入京畿不食稼皆抱荆棘而死宰相皆賀|{
	楊國忠以霖雨不害稼韓滉以霖雨不敗鹽今楊知至以蝗不食稼抱荆棘而死唐之臣以蒙蔽人主而成習其來久矣}
八月李茂勲為盧龍節度使|{
	八月之下當有以字}
九月右補闕董禹諫上遊畋乘驢擊毬上賜金帛以褒之邠寧節度使李侃奏為假父華清宫使道雅求贈官|{
	李侃為宦者假子為于偽翻}
禹上疏論之語頗侵宦官樞密使楊復㳟等列訴於上冬十月禹坐貶郴州司馬|{
	谷永專攻上身不失為九卿王章斥言王鳳則死于牢獄嗚呼有以也哉}
復㳟欽義之養孫也|{
	楊欽義見二百四十六卷文宗開成五年}
昭義軍亂大將劉廣逐節度使高湜|{
	湜承職翻}
自為留後以左金吾大將軍曹翔為昭義節度使 回鶻還至羅川|{
	唐寧州真寧縣隋羅川縣也其地即漢上郡陽周縣地宣宗大中二年回鶻西奔至是方還}
十一月遣使者同羅榆禄入貢賜拯接絹萬匹 羣盜侵淫|{
	侵當作浸}
剽掠十餘州至于淮南多者千餘人少者數百人詔淮南忠武宣武義成天平五軍節度使監軍亟加討捕及招懷十二月王仙芝寇沂州平盧節度使宋威表請以步騎五千别為一使兼帥本道兵所在討賊|{
	帥讀曰率}
仍以威為諸道行營招討草賊使仍給禁兵三千甲騎五百|{
	騎奇寄翻}
因詔河南方鎮所遣討賊都頭並取威處分|{
	處昌呂翻分扶問翻}


三年春正月天平軍奏遣將士張晏等救沂州還至義橋聞北境復有盜起|{
	復扶又翻}
留使扞禦晏等不從喧譟趣鄆州|{
	趣七喻翻}
都將張思泰李承祐走馬出城裂袖與盟以俸錢備酒殽慰諭然後定詔本軍宣慰一切無得窮詰|{
	詰區吉翻唐自中世以來姑息藩鎮至其末也姑息亂軍遂陵夷以至於亡}
勑福建江西湖南諸道觀察刺史皆訓練士卒又令天下鄉村各置弓刀皷板以備羣盜 賜兖海節度號泰寧軍 三月盧龍節度使李茂勲請以其子幽州左司馬可舉知留後自求致仕詔茂勲以左僕射致仕以可舉為盧龍留後 門下侍郎同平章事崔彦昭罷為太子太傅以左僕射王鐸兼門下侍郎同平章事 南詔遣使者詣高駢求和而盜邊不息駢斬其使者蠻之陷交趾也|{
	事見二百五十卷懿宗咸通六年}
虜安南經畧判官杜驤妻李瑶瑶宗室之疎屬也蠻遣瑶還遞木夾以遺駢|{
	遞牒以木夾之故云木夾范成大桂海虞衡志曰紹興元年安南與廣西帥司及邕通信問用兩漆板夾繫文書刻字其上謂之木夾按宋白續通典諸道州府巡院傳逓敕書皆有木夾是中國亦用木夾也遺唯季翻}
稱督爽牒西川節度使辭極驕慢駢送瑶京師甲辰復牒南詔數其負累聖恩德暴犯邊境殘賊欺詐之罪安南大度覆敗之狀折辱之|{
	數所角翻懿宗咸通七年高駢破蠻於安南上乾符二年駢破蠻於大度河折之舌翻}
原州刺史史懷操貪暴夏四月軍亂逐之 賜宣武感化節度泗州防禦使密詔選精兵數百人於巡内遊奕防衛綱船五日一具上供錢米平安狀聞奏|{
	汴徐泗三鎮汴水所經東南綱運輸上都者皆由此道羣盜從横恐為所掠故密詔選兵遊奕防衛}
五月昭王汭薨|{
	汭宣宗子}
以盧龍留後李可舉為節度使 六月撫王紘薨|{
	紘順宗子}
雄州地震裂水涌壞州城及公私廬舍俱盡|{
	雄州在靈州西南百八十里壞音怪}
秋七月以前巖州刺史高傑為左驍衛將軍充沿海水軍都知兵馬使|{
	新志調露二年析横貴二州置巖州因巖岡之北以為名}
以討王郢 鄂王潤薨|{
	潤宣宗子}
加魏博節度使韓簡同平章事 宋威擊王仙芝於沂州城下大破之 |{
	考異曰實録去年十二月宋威自青州與副使曹全晸進軍擊王仙芝仙芝敗走按仙芝若以去年十二月敗走中間半年豈能静處盖實因威除招討使連言之其實仙芝敗在此月不在十二月也}
仙芝亡去威奏仙芝已死縱遣諸道兵身還青州百官皆入賀居三日州縣奏仙芝尚在攻剽如故|{
	剽匹妙翻}
時兵始休詔復之|{
	復扶又翻}
士皆忿怨思亂八月仙芝陷陽翟郟城|{
	郟訖洽翻}
詔忠武節度使崔安濳兵擊之安濳慎由之弟也|{
	崔慎由相宣宗}
又昭義節度使曹翔將步騎五千及義成兵衛東都宫以左散騎常侍曾元裕為招討副使守東都又詔山南東道節度使李福選步騎二千守汝鄧要路仙芝進逼汝州詔邠寧節度使李侃鳳翔節度使令狐綯選步兵一千騎兵五百守陜州潼關|{
	陜失冉翻}
加成德節度使王景崇兼中書令 九月乙亥朔日有食之丙子王仙芝陷汝州執刺史王鐐鐐鐸之從父兄弟也|{
	鐐力彫翻又力弔翻從才用翻}
東都大震|{
	九域志汝州北至東都一百六十里}
士民挈家逃出城乙酉勑赦王仙芝尚君長罪除官以招諭之仙芝陷陽武攻鄭州昭義監軍判官雷殷符屯中牟|{
	中牟漢古縣隋曰郟城大業元年改曰圃田唐武德三年改曰中牟屬鄭州九域志在汴州西七十里}
擊仙芝破走之冬十月仙芝南攻唐鄧 西川節度使高駢築成都羅城使僧景仙規度|{
	度徒洛翻}
周二十五里悉召縣令庀徒賦役|{
	成都府領成都華陽新都犀浦新繁雙流廣都郫温江靈池十縣庀匹婢翻具也賦布也分布使之就役也}
吏受百錢以上皆死蜀土疏惡以甓甃之環城十里内取土皆剗丘姪平之|{
	甓蒲力翻甎也甃則救翻環音宦剗初限翻削也垤徒結翻}
無得為坎埳以害耕種|{
	埳徒感翻坎旁入也}
役者不過十日而代衆樂其均|{
	樂音洛}
不費扑撻而功辦|{
	扑普卜翻}
自八月癸丑築之至十一月戊子畢功役之始作也駢恐南詔揚聲入寇雖不敢決來役者必驚擾乃奏遣景仙託遊行入南詔說諭驃信使歸附中國仍許妻以公主|{
	新書曰浮屠景仙如此則文意明說輸芮翻妻七細翻}
因與議二國禮儀久之不決駢又聲言欲巡邊朝夕通烽火至大度河而實不行蠻中惴恐|{
	惴之睡翻}
由是訖於城成邊候無風塵之警先是西川將吏入南詔|{
	先悉薦翻將即亮翻}
驃信皆坐受其拜駢以其俗尚浮屠故遣景仙往驃信果帥其大臣迎拜|{
	帥讀曰率}
信用其言 王仙芝攻郢復二州陷之 王郢因温州刺史魯寔請降寔屢為之論奏|{
	為于為翻}
勑郢詣闕郢擁兵遷延半年不至固求望海鎮使朝廷不許以郢為右率府率|{
	唐有十率府率右率府率其一也}
仍令左神策軍補以重職其先所掠之財並令給與 十二月王仙芝攻申光廬壽舒通等州|{
	按唐書地理志通州屬山南東道宋之達州是也周世宗以南唐静海軍置通州今淮東之通州是也其地在唐則為揚州海陵縣之東境唐時淮南道未有通州此必誤參考下文通當作蘄}
淮南節度使劉鄴奏求益兵勑感化節度使薛能選精兵數千助之鄭畋以言計不行稱疾遜位不許乃上言自沂州奏捷之後|{
	謂宋威奏破王仙芝於沂州城下上時掌翻}
仙芝愈肆猖狂屠陷五六州瘡痍數千里宋威衰老多病自妄奏以來諸道尤所不服|{
	妄奏謂奏仙芝已死}
今淹留亳州殊無進討之意曾元裕擁兵蘄黄專欲望風退縮若使賊陷揚州則江南亦非國有崔安濳威望過人張自勉驍雄良將宫苑使李瑑西平王晟之孫|{
	瑑柱兖翻言瑑奕世將家}
嚴而有勇請以安濳為行營都統瑑為招討使代威自勉為副使代元裕 |{
	考異曰實録雖於此月載畋所上書亦不言行與不行新紀遂於此言安濳為諸道行營都統李瑑為招討草賊使張自勉副之按明年威元裕使副猶如故實録誤也}
上頗采其言 青滄軍士戍安南|{
	青州平盧軍滄州義昌軍}
還至桂州逐觀察使李瓚|{
	瓚才但翻 考異曰新紀在四年十二月今從實録}
瓚宗閔之子也|{
	李宗閔太和中為相}
以右諫議大夫張禹謨為桂州觀察使桂管監軍李維周驕横|{
	横戶孟翻}
瓚曲奉之浸不能制桂管有兵八百人防禦使纔得百人餘皆屬監軍又預於逐帥之謀強取兩使印|{
	兩使印謂觀察使及防禦使印也帥所類翻使疏吏翻下同}
擅補知州官奪昭州送使錢|{
	唐制諸州之稅分為三一曰上供以輸京師二曰送使以輸本道三曰留州留充本州經費}
詔禹謨并按之禹謨徹之子也|{
	張徹見二百四十二卷穆宗長慶元年}
招討副使都監楊復光奏尚君長弟讓據查牙山|{
	查鋤加翻}
官軍退保鄧州復光玄价之養子也|{
	楊玄价見二百五十卷懿宗咸通四年}
王仙芝攻蘄州蘄州刺史裴偓王鐸知舉時所擢進士也王鐐在賊中為仙芝以書說偓|{
	為于偽翻下同}
偓與仙芝約歛兵不戰許為之奏官鐐亦說仙芝許以如約偓乃開城延仙芝及黄巢輩三十餘人入城置酒大陳貨賄以贈之表陳其狀諸宰相多言先帝不赦龎勛朞年卒誅之|{
	事見上卷咸通九年十年卒子恤翻}
今仙芝小賊非龎勛之比赦罪除官益長姦宄|{
	長知兩翻}
王鐸固請許之乃以仙芝為左神策軍押牙兼監察御史遣中使以告身即蘄州授之仙芝得之甚喜鐐偓皆賀未退黄巢以官不及已大怒曰始者共立大誓横行天下今獨取官赴左軍使此五千餘衆安所歸乎 |{
	考異曰仙芝巢初起時云數月間衆至數萬至此纔有五千者盖烏合之衆聚散無常耳}
因毆仙芝傷其首|{
	毆烏口翻}
其衆諠譟不已仙芝畏衆怒遂不受命大掠蘄州城中之人半驅半殺焚其廬舍偓奔鄂州勑使奔襄州|{
	勑使授告身之中使也}
鐐為賊所拘賊乃分其軍三千餘人從仙芝及尚君長二千餘人從巢各分道而去|{
	考異曰王坤驚聽録曰乾符四年丁酉仲夏天示彗星草寇黄巢尚君長奔突即五年戊戍之歲狂寇王仙}


|{
	芝起自鄆封而侵汝鄭即大寇黄巢尚君長並賊帥之徒黨僅一千餘人攻陷汝州云云又曰黄巢望閩廣而去仙芝指鄆州南行尚君長劫陳蔡間取羣凶之願三千餘寇屬仙芝君長二千餘人屬黄巢所管明年二月仙芝陷鄂州巢陷鄆州則非巢趣閩廣仙芝趣鄆也王坤此書年月事迹差舛尤多但擇其可信者取之}


資治通鑑卷二百五十二
