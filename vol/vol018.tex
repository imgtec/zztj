<!DOCTYPE html PUBLIC "-//W3C//DTD XHTML 1.0 Transitional//EN" "http://www.w3.org/TR/xhtml1/DTD/xhtml1-transitional.dtd">
<html xmlns="http://www.w3.org/1999/xhtml">
<head>
<meta http-equiv="Content-Type" content="text/html; charset=utf-8" />
<meta http-equiv="X-UA-Compatible" content="IE=Edge,chrome=1">
<title>資治通鑒_19-資治通鑑卷十八_19-資治通鑑卷十八</title>
<meta name="Keywords" content="資治通鑒_19-資治通鑑卷十八_19-資治通鑑卷十八">
<meta name="Description" content="資治通鑒_19-資治通鑑卷十八_19-資治通鑑卷十八">
<meta http-equiv="Cache-Control" content="no-transform" />
<meta http-equiv="Cache-Control" content="no-siteapp" />
<link href="/img/style.css" rel="stylesheet" type="text/css" />
<script src="/img/m.js?2020"></script> 
</head>
<body>
 <div class="ClassNavi">
<a  href="/24shi/">二十四史</a> | <a href="/SiKuQuanShu/">四库全书</a> | <a href="http://www.guoxuedashi.com/gjtsjc/"><font  color="#FF0000">古今图书集成</font></a> | <a href="/renwu/">历史人物</a> | <a href="/ShuoWenJieZi/"><font  color="#FF0000">说文解字</a></font> | <a href="/chengyu/">成语词典</a> | <a  target="_blank"  href="http://www.guoxuedashi.com/jgwhj/"><font  color="#FF0000">甲骨文合集</font></a> | <a href="/yzjwjc/"><font  color="#FF0000">殷周金文集成</font></a> | <a href="/xiangxingzi/"><font color="#0000FF">象形字典</font></a> | <a href="/13jing/"><font  color="#FF0000">十三经索引</font></a> | <a href="/zixing/"><font  color="#FF0000">字体转换器</font></a> | <a href="/zidian/xz/"><font color="#0000FF">篆书识别</font></a> | <a href="/jinfanyi/">近义反义词</a> | <a href="/duilian/">对联大全</a> | <a href="/jiapu/"><font  color="#0000FF">家谱族谱查询</font></a> | <a href="http://www.guoxuemi.com/hafo/" target="_blank" ><font color="#FF0000">哈佛古籍</font></a> 
</div>

 <!-- 头部导航开始 -->
<div class="w1180 head clearfix">
  <div class="head_logo l"><a title="国学大师官网" href="http://www.guoxuedashi.com" target="_blank"></a></div>
  <div class="head_sr l">
  <div id="head1">
  
  <a href="http://www.guoxuedashi.com/zidian/bujian/" target="_blank" ><img src="http://www.guoxuedashi.com/img/top1.gif" width="88" height="60" border="0" title="部件查字,支持20万汉字"></a>


<a href="http://www.guoxuedashi.com/help/yingpan.php" target="_blank"><img src="http://www.guoxuedashi.com/img/top230.gif" width="600" height="62" border="0" ></a>


  </div>
  <div id="head3"><a href="javascript:" onClick="javascript:window.external.AddFavorite(window.location.href,document.title);">添加收藏</a>
  <br><a href="/help/setie.php">搜索引擎</a>
  <br><a href="/help/zanzhu.php">赞助本站</a></div>
  <div id="head2">
 <a href="http://www.guoxuemi.com/" target="_blank"><img src="http://www.guoxuedashi.com/img/guoxuemi.gif" width="95" height="62" border="0" style="margin-left:2px;" title="国学迷"></a>
  

  </div>
</div>
  <div class="clear"></div>
  <div class="head_nav">
  <p><a href="/">首页</a> | <a href="/ShuKu/">国学书库</a> | <a href="/guji/">影印古籍</a> | <a href="/shici/">诗词宝典</a> | <a   href="/SiKuQuanShu/gxjx.php">精选</a> <b>|</b> <a href="/zidian/">汉语字典</a> | <a href="/hydcd/">汉语词典</a> | <a href="http://www.guoxuedashi.com/zidian/bujian/"><font  color="#CC0066">部件查字</font></a> | <a href="http://www.sfds.cn/"><font  color="#CC0066">书法大师</font></a> | <a href="/jgwhj/">甲骨文</a> <b>|</b> <a href="/b/4/"><font  color="#CC0066">解密</font></a> | <a href="/renwu/">历史人物</a> | <a href="/diangu/">历史典故</a> | <a href="/xingshi/">姓氏</a> | <a href="/minzu/">民族</a> <b>|</b> <a href="/mz/"><font  color="#CC0066">世界名著</font></a> | <a href="/download/">软件下载</a>
</p>
<p><a href="/b/"><font  color="#CC0066">历史</font></a> | <a href="http://skqs.guoxuedashi.com/" target="_blank">四库全书</a> |  <a href="http://www.guoxuedashi.com/search/" target="_blank"><font  color="#CC0066">全文检索</font></a> | <a href="http://www.guoxuedashi.com/shumu/">古籍书目</a> | <a   href="/24shi/">正史</a> <b>|</b> <a href="/chengyu/">成语词典</a> | <a href="/kangxi/" title="康熙字典">康熙字典</a> | <a href="/ShuoWenJieZi/">说文解字</a> | <a href="/zixing/yanbian/">字形演变</a> | <a href="/yzjwjc/">金 文</a> <b>|</b>  <a href="/shijian/nian-hao/">年号</a> | <a href="/diming/">历史地名</a> | <a href="/shijian/">历史事件</a> | <a href="/guanzhi/">官职</a> | <a href="/lishi/">知识</a> <b>|</b> <a href="/zhongyi/">中医中药</a> | <a href="http://www.guoxuedashi.com/forum/">留言反馈</a>
</p>
  </div>
</div>
<!-- 头部导航END --> 
<!-- 内容区开始 --> 
<div class="w1180 clearfix">
  <div class="info l">
   
<div class="clearfix" style="background:#f5faff;">
<script src='http://www.guoxuedashi.com/img/headersou.js'></script>

</div>
  <div class="info_tree"><a href="http://www.guoxuedashi.com">首页</a> > <a href="/SiKuQuanShu/fanti/">四库全书</a>
 > <h1>资治通鉴</h1> <!--         下载:【右键另存为】即可 --></div>
  <div class="info_content zj clearfix">
  
<div class="info_txt clearfix" id="show">
<center style="font-size:24px;">19-資治通鑑卷十八</center>
    資治通鑑卷十八    宋 司馬光 撰<br />
<br />
  胡三省 音註<br />
<br />
  漢紀十【起著雍涒灘盡柔兆執徐凡九年】<br />
<br />
  世宗孝武皇帝上之下<br />
<br />
  元光二年冬十月上行幸雍祠五畤【雍於用翻畤音止】 李少君以祠竈却老方見上【祠竈者祭竈以致鬼物化丹砂以為黄金以為飲食器可以延年方士之言云爾少詩照翻】上尊之少君者故深澤侯舍人【高祖功臣有深澤侯趙將夕景帝三年孫修嗣侯七年有罪耐為司寇少君當是為修舍人班志涿郡有南深澤縣】匿其年及其生長【謂其生時及長時所居止處也長知兩翻】其游以方徧諸侯無妻子人聞其能使物及不死【如淳曰物謂鬼物也】更饋遺之【更工衡翻遺于季翻】常餘金錢衣食人皆以為不治生業而饒給又不知其何所人愈信争事之【治直之翻】少君善為巧發奇中【如淳曰時時發言有所中中竹仲翻】嘗從武安侯飲【田蚡封武安侯】坐中有九十餘老人【坐徂卧翻下同】少君乃言與其大父游射處老人為兒時從其大父識其處【師古曰識記也式志翻】一坐盡驚少君言上曰祠竈則致物致物而丹砂可化為黄金夀可益蓬萊仙者可見見之以封禪則不死黄帝是也臣嘗游海上見安期生【列仙傳安期生琅邪人賣藥東海邊時人皆言千歲】食臣棗大如瓜【食祥吏翻】安期生仙者通蓬萊中合則見人不合則隱於是天子始親祠竈遣方士入海求蓬萊安期生之屬而事化丹砂諸藥齊為黄金矣【藥之分齊齊才計翻】居久之李少君病死天子以為化去不死而海上燕齊怪迂之方士多更來言神事矣【更工衡翻】 亳人謬忌奏祠太一【如淳曰亳亦薄也晉灼曰亳縣屬濟隂郡予據班志亳屬山陽郡亳作薄謬姓也音靡幼翻與繆同戰國時趙有宦者令繆賢太一者天之尊神天文志中宮天極星其一明者太一常居也淮南子太微者太一之庭紫宫者太一之居索隱曰樂汁微圖云天宫紫微北極天一太一宋均曰天一太一北極神之别名春秋佐助期云紫宫天皇耀魄寶之所理也石氏云天一太一各一星在紫宫門外立承事天皇大帝】方曰天神貴者太一太一佐曰五帝【五帝謂東方青帝靈威仲南方赤帝赤熛怒西方白帝白招矩北方黑帝叶光紀中央黄帝舍樞紐也一說蒼帝名靈符赤帝名文祖白帝名記黑帝名玄矩黄帝名神斗】於是天子立其祠長安東南郊 㕍門馬邑豪聶壹【馬邑縣屬㕍門郡豪謂以貲財武力雄於鄉曲者聶姓也姓譜曰楚大夫食采于聶因以為氏壹其名聶尼輒翻】因大行王恢言匈奴初和親親信邊可誘以利致之伏兵襲擊必破之道也上召問公卿王恢曰臣聞全代之時【戰國之初代自為一國故曰全代其後為趙襄子所滅代始屬趙服䖍曰代未分之時也李奇曰六國之時代為一國尚能以擊匈奴况今加以漢之大乎】北有強胡之敵内連中國之兵然尚得養老長幼【長知兩翻】種樹以時倉廪常實匈奴不輕侵也今以陛下之威海内為一然匈奴侵盗不已者無它以不恐之故耳【言不示以威故匈奴不知懼也】臣竊以為擊之便韓安國曰臣聞高皇帝嘗圍於平城【事見十一卷高祖七年】七日不食及解圍反位而無忿怒之心夫聖人以天下為度者也【師古曰言當隨天下人心而寛大其度量也】不以己私怒傷天下之公故遣劉敬結和親至今為五世利臣竊以為勿擊便恢曰不然高帝身被堅執鋭行幾十年【被皮義翻幾居衣翻】所以不報平城之怨者非力不能所以休天下之心也今邊境數驚【數所角翻】士卒傷死中國槥車相望【應劭曰槥小棺也今謂之櫝金布令曰不幸死所為櫝傳歸所居縣師古曰從軍死者以槥送致其喪載槥之車相望於道言其多也槥音衛】此仁人之所隱也【隱惻也張晏曰痛也】故曰擊之便安國曰不然臣聞用兵者以飽待飢正治以待其亂定舍以待其勞故接兵覆衆伐國墮城【師古曰覆敗也墮毁也言兵與敵接則敗其衆所伐之國則墮其城也墮讀曰隳】常坐而役敵國此聖人之兵也今將卷甲輕舉【卷讀曰捲】深入長敺難以為功【敺與驅同】從行則廹脅衡行則中絶【從子容翻衡讀曰横】疾則糧乏徐則後利【師古曰後利謂不及於利後戶遘翻】不至千里人馬乏食兵法曰遺人獲也【言以軍遺敵人令其禽獲也遺于季翻】臣故曰勿擊便恢曰不然臣今言擊之者固非發而深入也將順因單于之欲誘而致之邊【誘音酉】吾選梟騎壮士隂伏而處以為之備【梟堅堯翻騎奇寄翻】審遮險阻以為其戒吾埶已定或營其左或營其右或當其前或絶其後單于可禽百全必取上從恢議 【考異曰史記韓長孺傳元光元年聶壹畫馬邑事而漢書武紀在二年盖元年壹始言之二年議乃决也】夏六月以御史大夫韓安國為護軍將軍衛尉李廣為驍騎將軍太僕公孫賀為輕車將軍大行王恢為將屯將軍【司馬彪曰輕車古之戰車李奇曰將屯主監諸屯】太中大夫李息為材官將軍將車騎材官三十餘萬匿馬邑旁谷中約單于入馬邑縱兵隂使聶壹為間【間古莧翻】亡入匈奴謂單于曰吾能斬馬邑令丞以城降【縣有令有丞長吏也】財物可盡得單于愛信以為然而許之聶壹乃詐斬死罪囚縣其頭馬邑城下【縣古懸字通】示單于使者為信曰馬邑長吏已死【長知兩翻】可急來于是單于穿塞將十萬騎入武州塞【班志武州縣屬鴈門郡崔浩曰今平城首西百里有武州城是也杜佑曰武州塞在朔州善陽縣界】未至馬邑百餘里見畜布野【畜許救翻】而無人牧者怪之乃攻亭得㕍門尉史欲殺之【師古曰漢律近塞皆置尉百里一人士史尉史各二人時㕍門尉史行徼見寇因保此亭】尉史乃告單于漢兵所居單于大驚曰吾固疑之乃引兵還出曰吾得尉史天也以尉史為天王塞下傳言單于已去漢兵追至塞度弗及乃皆罷兵【度徒洛翻】王恢主别從代出擊胡輜重【重直用翻】聞單于還兵多亦不敢出上怒恢恢曰始約為入馬邑城兵與單于接而臣擊其輜重可得利今單于不至而還臣以三萬人衆不敵祗取辱固知還而斬然完陛下士三萬人於是下恢廷尉【下遐嫁翻】廷尉當恢逗橈當斬【應劭曰逗曲行避敵也橈顧望也如淳曰軍行而逗留畏懦者要斬師古曰應說非也逗留止也橈謂屈弱也逗音豆又音住橈奴教翻】恢行千金丞相蚡蚡不敢言上而言於太后曰王恢首為馬邑事今不成而誅恢是為匈奴報仇也【蚡房吻翻是為于偽翻】上朝太后【朝直遥翻】太后以蚡言告上上曰首為馬邑事者恢故發天下兵數十萬從其言為此且縱單于不可得恢所部擊其輜重猶頗可得以尉士大夫心【尉與慰同】今不誅恢無以謝天下於是恢聞乃自殺自是之後匈奴絶和親攻當路塞【師古曰塞之當行道處者】往往入盗於漢邊不可勝數然尚貪樂關市【匈奴與漢人於邊為互市如今之回易場也勝音升樂音洛】嗜漢財物漢亦關市不絶以中其意【中竹仲翻】<br />
<br />
  三年春河水徙從頓丘東南流【師古曰頓丘丘名因以為縣本衛地也地理志屬東郡今則在魏州界 考異曰漢書武紀云東南流入渤海按頓丘屬東郡勃海乃在頓丘東此恐誤今不取】夏五月丙子復決濮陽瓠子【濮陽縣屬東郡服䖍曰瓠子隄名在東郡蘇林曰甄城以南濮陽以北為瓠子河廣百步深五丈水經瓠子河出濮陽縣北十里即瓠河口復扶又翻瓠戶故翻 考異曰史記河渠書元光中河決瓠子東注鉅野服䖍注漢書武紀曰瓠子隄名在東郡白馬蘇林曰在甄城以南濮陽以北將相名臣表曰五月丙子河決瓠子然則瓠子即濮陽縣境隄名也】注鉅野【班志鉅野縣屬山陽郡大野澤在其北師古曰即今鄆州鉅野縣】通淮泗【決河之水由鉅野而通泗水由泗水而通淮也】汎郡十六【汎敷劍翻】天子使汲黯鄭當時發卒十萬塞之輒復壞【塞悉則翻復扶又翻下同】是時田蚡奉邑食鄃【奉扶用翻鄃音輸鄃縣屬清河郡】鄃居河北河決而南則鄃無水災邑收多蚡言於上曰江河之決皆天事未易以人力彊塞【易以鼓翻彊其兩翻】塞之未必應天而望氣用數者亦以為然于是天子久之不復事塞也 初孝景時魏其侯竇嬰為大將軍武安侯田蚡乃為諸郎【諸郎諸曹郎也】侍酒跪起如子姪已而蚡日益貴幸為丞相魏其失埶賓客益衰【師古曰言素為嬰之賓客者漸以衰退不復往也】獨故燕相潁隂灌夫不去【燕王定國王澤之孫也夫自太僕出相之班志潁隂縣屬潁川郡相息亮翻】嬰乃厚遇夫相為引重【張晏曰相薦逹為聲勢也師古曰相牽引以致於尊重也為于偽翻】其游如父子然夫為人剛直使酒諸有埶在己之右者必陵之數因酒忤丞相【數所角翻忤五故翻】丞相乃奏案灌夫家屬横潁川民苦之【夫宗族賓客為權利横于潁川小兒歌之曰潁水清灌氏寧潁水濁灌氏族横戶孟翻】收繫夫及支屬皆得棄市罪【刑人與市與衆弃之故殺之于市者謂之弃市景帝中元年改磔曰弃市應劭曰先諸死刑皆磔于市今改曰弃市自非妖逆不復磔也師古曰磔謂張其尸也弃市殺之于市也】魏其上書論救灌夫上令與武安東朝廷辨之【東朝謂太后居長樂宮在未央宫之東也令于長樂宫見太后廷辨其是非也朝直遥翻下同】魏其武安因互相詆訐【訐居謁翻】上問朝臣兩人孰是唯汲黯是魏其韓安國兩以為是鄭當時是魏其後不敢堅上怒當時曰吾并斬若屬矣【若屬猶言汝輩也】即罷起入上食太后【上時掌翻】太后怒不食曰今我在也而人皆藉吾弟【晉灼曰藉蹈也藉慈夜翻】令我百歲後皆魚肉之乎【師古曰以比魚肉而食啖也】上不得已遂族灌夫使有司案治魏其得棄市罪<br />
<br />
  四年冬十二月晦論殺魏其於渭城【漢法以冬月行重刑遇春則赦若贖故以十二月晦論殺魏其侯此武安侯蚡之意也渭城縣屬扶風秦之咸陽也 考異曰班固漢武故事曰上召大臣議之羣臣多是竇嬰上亦不復窮問兩罷之田蚡大恨自殺先與太后訣兄弟共號哭訴太后太后亦哭弗食上不得已遂乃殺嬰按漢武故事語多誕妄非班固書蓋後人為之託固名耳】春三月乙卯武安侯蚡亦薨 【考異曰武安侯傳云元光四年春丞相按灌夫事其夏取夫人五年十月論灌夫及家屬十二月晦魏其弃市徐廣引武帝本紀侯表以為蚡薨在嬰死後分明四年當是三年五年當是四年今從之廣又疑十二月為二月按漢制常以立春下寛大詔書蚡恐魏其得釋故以十二月晦殺之何必改為二月也】及淮南王安敗【見後十九卷元狩元年】上聞蚡受安金有不順語【見上卷建元二年】曰使武安侯在者族矣 夏四月隕霜殺艸 御史大夫安國行丞相事引墯車蹇【如淳曰為天子導引而墯車蹇跛也予據漢制大駕則公卿奉引安國蓋因奉引而墯車也墯杜火翻】五月丁巳以平棘侯薛澤為丞相【薛澤高祖功臣廣平侯薛歐之孫廣平侯國景帝中二年罪絶中五年復封澤平棘侯班志平棘縣屬常山郡】安國病免 地震赦天下九月以中尉張歐為御史大夫韓安國疾愈復為中尉河間王德修學好古實事求是【德景帝子帝之兄也景帝前二年受封師】<br />
<br />
  【古曰實事求是務得其實每求真是也好呼到翻】以金帛招求四方善書得書多與漢朝等【朝直遥翻】是時淮南王安亦好書所招致率多浮辨獻王所得書皆古文先秦舊書【師古曰先秦猶言秦先謂未焚書之前予據獻王傳舊書即謂周官尚書禮記孟子老子之書】采禮樂古事稍稍增輯至五百餘篇被服造次【師古曰被服言常居處其中也造次謂所向必行也予謂被服者言以儒術衣被其身也被皮義翻造千到翻】必于儒者山東諸儒多從之游五年冬十月河間王來朝獻雅樂對三雍宫【應劭曰辟雍明堂靈臺也雍和也言天地君臣人民皆和也予謂對三雍宫者對三雍之制度非召對于三雍宫】及詔策所問三十餘事其對推道術而言得事之中文約指明【師古曰中竹仲翻約少也指謂義之所趨若人以手指物也】天子下太樂官常存肄河間王所獻雅聲【班表太樂官屬太常肄以至翻習也下遐嫁翻】歲時以備數然不常御也春正月河間王薨中尉常麗以聞【姓譜常姓黄帝相常先之後】曰王身端行治【師古曰端直也治理也行下孟翻】温仁恭儉篤敬愛下明知深察惠于鰥寡大行令奏諡法聰明睿知曰獻諡曰獻王【知讀曰智】<br />
<br />
  班固贊曰昔魯哀公有言寡人生于深宫之中長於婦人之手未嘗知憂未嘗知懼【師古曰哀公與孔子之言事見孫卿子長知兩翻】信哉斯言也雖欲不危亡不可得已【師古曰己語終辭】是故古人以晏安為鴆毒【師古曰左氏傳管敬仲曰晏安鴆毒不可懷也】無德而富貴謂之不幸漢興至於孝平諸侯王以百數率多驕淫失道何則沈溺放恣之中【沈持林翻】居埶使然也自凡人猶繫于習俗而况哀公之倫乎夫惟大雅卓爾不羣河間獻王近之矣<br />
<br />
  初王恢之討東越也【見上卷建元六年】使番陽令唐蒙風曉南越南越食蒙以蜀枸醤【班志番陽縣屬豫章郡番蒲何翻風讀曰諷劉德曰枸樹如桑其椹長二三寸味酢取其實以為醤美師古曰枸者緣木而生非樹也子形如桑椹又不長二三寸味尤辛不酢劉說非也崔駰曰按漢書音義枸木似榖樹其葉似桑葉用其葉作醤酢美蜀人以為珍味廣志曰枸黑色味辛下氣消穀晉灼曰枸音矩索隱從徐廣音求羽翻唐本本草注曰蒟蔓生葉似王瓜而厚大味辛香實似桑椹皮黑肉白劉淵林曰蒟醤緣木而生其子如桑椹熟時正青長二三寸以蜜藏而食辛香調五藏李心傳曰蒟醤廣蜀皆有之實類也蜀中者緣木而生如桑椹熟時正青長二三寸以蜜藏而食之廣中者蔓生葉似王瓜而厚大味辛香實似桑椹皮黑肉白其苗如浮留藤取葉合檳榔食之西戎亦時時持來細而辛烈唐蒙所見謂來自牂牁則廣生殆蜀本也蒟醤之味全類蓽撥而蓽撥辛烈尤甚世人唯用蓽撥不用蒟醤故鮮有知者】蒙問所從來曰道西北牂柯江牂柯江廣數里出番禺城下【南越志曰番禺之西有江浦焉師古曰牂柯繫船代華陽國志云楚遣莊蹻伐夜郎軍至且蘭椓船于岸而步戰既滅夜郎以且蘭有椓船牂柯處乃改為牂柯又後漢志注牂柯江中名山或曰牂柯江東通四會至番禺入海水經牂柯水東至欝林廣欝縣為欝水南流入交趾劉昫曰唐邕州治宣化縣漢欝林郡之領方縣地也驩水在縣北本牂柯河俗呼為欝狀江即駱越水也蓋廣欝縣漢亦屬欝林郡水經所謂交趾界者漢交趾州界也牂音臧柯音歌班志番禺縣屬南海郡時為南越王都廣古曠翻番音潘禺音愚】蒙歸至長安問蜀賈人賈人曰獨蜀出枸醤多持竊出市夜郎【華陽國志夜郎王竹王三郎之後武帝開為縣屬牂柯郡史記正義曰今瀘州南大江南岸協州曲州本夜郎國賈音古】夜郎者臨牂柯江江廣百餘步足以行船南越以財物役屬夜郎西至桐師【桐師西南夷種其地在夜郎之西葉榆之西南】然亦不能臣使也蒙乃上書說上曰【乃上時掌翻說式芮翻】南越王黄屋左纛地東西萬餘里名為外臣實一州主也今以長沙豫章往水道多絶難行竊聞夜郎所有精兵可得十餘萬浮船牂柯江出其不意此制越一奇也誠以漢之彊巴蜀之饒通夜郎道為置吏甚易【為于偽翻易以䜴翻】上許之乃拜蒙為中郎將將千人食重萬餘人【師古曰食根及輜重重直用翻】從巴蜀筰關入【李文子曰筰關在沈黎郡又云在犍為郡界宋白曰眉州青神縣臨青衣江郡國志漢武帝使唐蒙開西南夷路始此眉州漢犍為郡地筰才各翻】遂見夜郎侯多同【多同夜郎侯之名也】蒙厚賜喻以威德約為置吏使其子為令夜郎旁小邑皆貪漢繒帛以為漢道險終不能有也乃且聽蒙約還報上以為犍為郡【李文子曰犍為郡治鄨元光五年又冶南廣水經注曰鄨水出符縣南不狼山縣有犍山後漢志鄨水過牂柯郡入延江水水經注沅水出且蘭東至鐔城為沅水寰宇記唐播州夷州費州莊州即秦且蘭夜郎之西北隅今珍州亦其地又西高州有夜郎縣牂州建安縣有古夜郎城西近施黔東近辰沅皆其境也犍居言翻章懷大子賢曰犍為故城在眉州隆山縣西北】發巴蜀卒治道自僰道指牂柯江【班志僰道屬犍為郡宋白曰古僰國縣有蠻夷曰道故為僰道今戎州治所康曰僰國在馬湖江唐蒙鑿石開道以通之治直之翻僰蒲北翻】作者數萬人士卒多物故有逃亡者用軍興法誅其渠率【鄭玄曰縣官徵聚曰興今云軍興是也率所類翻】巴蜀民大驚恐上聞之使司馬相如責唐蒙等因諭告巴蜀民以非上意相如還報是時卭筰之君長【華陽國志雅州卭崍山本名卭筰山故卭人筰人界韋昭曰筰縣在越嶲文潁曰卭者今為卭都縣筰者今為定筰縣史記正義曰卭都西有卭僰山在睢州榮經縣界山岩峭峻曲回九折乃至上下有凝氷即王尊叱馭處康曰卭都夷其地陷為汙澤因名卭池南人呼為卭河師古曰卭都今之卭州本其地卭渠容翻筰才各翻】聞南夷與漢通得賞賜多多欲願為内臣妾請吏比南夷天子問相如相如曰卭筰冉駹者近蜀道亦易通【師古曰今開州夔州等首領多姓冉者本皆冉種也後漢書冉駹其山有六夷七羌九蠻各有部落括地志蜀西徼外羌茂州冉州本冉駹國康曰其人依山居土累石為室至十餘丈駹音厖易以䜴翻】秦時嘗通為郡縣至漢興而罷今誠復通為置郡縣愈於南夷【張楫曰愈差也又云愈猶勝也晉灼曰南夷謂牂柯犍為西夷謂越嶲益州也為置之為于偽翻】天子以為然乃拜相如為中郎將建節往使及副使王然于等乘傳因巴蜀吏幣物以賂西夷卭筰冉駹斯榆之君【康曰本葉榆澤其君長因以立號後隨畜移于徙師古曰徙音斯故又號徙榆使疏吏翻傳張戀翻】皆請為内臣除邊關關益斥西至沬若水【斥開廣也張揖曰沬水出蜀廣平徼外與青衣水合若水出旄牛徼外至僰道入江華陽國志漢嘉縣有沬水李文子曰若水南至大筰入繩水師古曰沬音妹】南至牂柯為徼通零關道【班志零關屬越嶲郡張揖曰鑿靈山為道寰宇記靈關山在雅州盧山縣北二十里靈關鎮在盧山縣北八十二里零靈通用徼吉弔翻】橋孫水【張楫曰孫水出臺登縣南至會無入若水康曰一名白沙江李文子曰孫水本名長河水】以通卬都為置一都尉十餘縣屬蜀【為于偽翻】天子大說【說讀曰悦】 詔發卒萬人治鴈門阻險【師古曰阻險所以為固用止匈奴之寇貢父曰治險阻者通道令平易以便伐匈奴治直之翻】 秋七月大風拔木 女巫楚服等教陳皇后祠祭厭勝挾婦人媚道事覺【厭一涉翻賈公彦曰按漢書婦人蠱惑媚道更相祝詛作木偶人埋之于地漢法又有官禁敢行媚道者】上使御史張湯窮治之湯深竟黨與相連及誅者三百餘人楚服梟首于市【梟堅堯翻】乙巳賜皇后册收其璽綬罷退居長門宫【長門宫如淳曰長門在長安城東南東方朔傳竇太主獻長門園上以為宫】竇太主慙愳稽顙謝上【竇太主陳皇后母也稽音啟】上曰皇后所為不軌於大義不得不廢主當信道以自慰勿受妄言以生嫌懼后雖廢供奉如法長門無異上宫也 初上嘗置酒竇太主家主見所幸賣珠兒董偃上賜之衣冠尊而不名稱為主人翁使之侍飲由是董君貴寵天下莫不聞 【考異曰漢武故事曰陳皇后廢處長門宫竇太主以宿恩猶自親近後置酒主家主見所幸董偃按東方朔傳爰叔為偃畫計令主獻長門園更名曰長門宫則偃見上在陳后廢前明矣】常從游戲北宫馳逐平樂觀【平樂觀在未央宫北周回十五里高祖時制度草創至帝增修之三輔黄圖曰上林苑中有平樂觀樂音洛觀古玩翻】鷄鞠之會【鬭鷄及蹴踘也鞠毬也以皮為之鞠音居六翻】角狗馬之足【師古曰角猶校也】上大歡樂之上為竇太主置酒宣室【蘇林曰宣室未央前殿正室也如淳曰宣室布政教之室也樂音洛為于偽翻】使謁者引内董君是時中郎東方朔陛戟殿下【師古曰持戟立列陛側也】辟戟而前曰【辟頻亦翻】董偃有斬罪三安得入乎上曰何謂也朔曰偃以人臣私侍公主其罪一也敗男女之化而亂婚姻之禮傷王制其罪二也【敗補邁翻】陛下富于春秋方積思于六經偃不遵經勸學反以靡麗為右【師古曰右尊之也思相吏翻】奢侈為務盡狗馬之樂極耳目之欲是乃國家之大賊人主之大蜮【師古曰蜮魅也音或說者以為短狐非也短狐射工耳於此不當其義今俗猶云魅蜮也貢父曰劉向說春秋蜮南方淫氣所生以應哀姜然則朔正用指偃耳何必遷就魅蜮也予按洪範五行傳曰蜮如鱉三足生於南越南越婦人多淫故其地多蜮淫女惑亂之氣所生也陸璣草木疏曰一名射影江淮水皆有之人在岸上影見水中投水影則殺之故曰射影南人將入水先以瓦石投水中令水濁然後入或曰含沙射人皮肌其瘡如疥陸佃埤雅曰蜮一名射工有長角横在口前如弩檐臨其角端曲如上弩以氣為矢因水勢以射人故俗呼水弩】其罪三也上默然不應良久曰吾業已設飲後而自改朔曰夫宣室者先帝之正處也非法度之政不得入焉故淫亂之漸其變為簒是以豎貂為淫而易牙作患慶父死而魯國全【豎貂易牙皆齊桓公嬖臣也管仲有疾桓公問之曰將何以教寡人仲曰願君之遠豎貂易牙公曰易牙烹其子以快寡人尚可疑邪對曰人之情非不愛其子其子之忍又將何有於君公曰豎貂自宫以近寡人尚可疑邪對曰人之情非不愛其身其身之忍又將何有於君公曰諾管仲死盡逐之而公食不甘宫不治居三年公曰仲父不亦過乎于是復皆召而反之明年公病豎貂易牙相與作亂塞門築墻不通人有一婦人踰墻至公所公曰我欲食婦人曰吾無所得又曰我欲飲婦人曰我無所得公曰何故曰豎貂易牙作亂故無所得公慨然歎曰若死者有知吾何面目見仲父乎蒙衣袂而絶乎夀宫蟲流出于戶蓋以揚門之扉三月不葬慶父魯桓公庶子莊公之兄通於哀姜莊公薨慶父弑其子般及閔公欲為亂而不克奔莒莒人歸之縊于密魯乃定父音甫】上曰善有詔止更置酒北宫引董君從東司馬門入【未央宫有東闕北闕東闕曰蒼龍東司馬門蒼龍闕内之司馬門也更工衡翻】賜朔黄金三十斤董君之寵由是日衰是後公主貴人多踰禮制矣 上以張湯為太中大夫與趙禹共定諸律令務在深文拘守職之吏作見知法吏傳相監司用法益刻自此始【蘇林曰拘刻於守職之吏師古曰見知人犯法而不舉告謂之故縱晉志曰見知而不舉劾各與同罪失不舉劾以贖論其不見不知不坐也傳張戀翻監古銜翻】八月螟【食心曰螟】 是歲徵吏民有明當世之務習先聖之術者縣次續食令與計偕【師古曰計者上計簿使也郡國每歲遣詣京師上之偕者俱也令所徵之人與上計者俱來而縣次給其食後世訛誤因乘此語遂謂上計為計偕闞駰不詳妄為解說云秦漢謂諸侯朝使曰計偕偕次也晉有計偕簿又改偕為階失之彌遠致誤後學】甾川人公孫弘對策曰臣聞上古堯舜之時不貴爵賞而民勸善不重刑罰而民不犯躬率以正而遇民信也末世貴爵厚賞而民不勸深刑重罰而姦不止其上不正遇民不信也夫厚賞重刑未足以勸善而禁非必信而已矣是故因能任官則分職治去無用之言則事情得不作無用之器則賦歛省【治直吏翻去羌呂翻斂力贍翻】不奪民時不妨民力則百姓富有德者進無德者退則朝廷尊有功者上無功者下則羣臣逡【李奇曰言有次第師古曰逡七旬翻】罰當罪則姦邪止賞當賢則臣下勸凡此八者治之本也故民者業之則不争理得則不怨有禮則不暴愛之則親上【師古曰各得其業則無争心各申其理則無所怨使之有禮則無暴慢子而受之則知親上也】此有天下之急者也禮義者民之所服也而賞罸順之則民不犯禁矣臣聞之氣同則從聲比則應【比頻寐翻又音毗和也】今人主和德于上百姓和合于下故心和則氣和氣和則形和形和則聲和聲和則天地之和應矣故隂陽和風雨時甘露降五穀登六畜蕃【畜許救翻蕃扶元翻】嘉禾興朱草生山不童澤不涸此和之至也時對者百餘人太常奏弘第居下策奏天子擢弘對為第一拜為博士待詔金馬門【如淳曰武帝時相馬者東方京作銅馬法獻之立馬于魯班門外更名魯班門為金馬門三輔黄圖曰金馬門宦者署武帝得大宛馬以銅鑄像立于署門因以為像】齊人轅固年九十餘亦以賢良徵公孫弘仄目而事固固曰公孫子務正學以言無曲學以阿世諸儒多疾毁固者固遂以老罷歸是時巴蜀四郡【四郡蜀郡廣漢郡犍為郡巴郡也】鑿山通西南夷千餘里戍轉相餉數歲道不通士罷餓離暑濕死者甚衆【罷讀曰疲】西南夷又數反【數所角翻】發兵興擊費以巨萬計而無功上患之詔使公孫弘視焉還奏事盛毁西南夷無所用上不聽弘每朝會【朝直遥翻】開陳其端使人主自擇不肯面折廷争于是上察其行慎厚辯論有餘習文法吏事緣飾以儒術【師古曰譬之於衣加純緣也折之舌翻争讀曰諍行下孟翻】大說之【說讀曰悦】一歲中遷至左内史【考異曰漢書武紀云元光元年五月詔策賢良於是董仲舒公孫弘等出焉按弘傳元光五年復徵賢良文學甾川國推上弘其策文頗與武紀元年策文相類又云一歲中至左内史百官表元光五年弘為左内史然則弘之舉賢良不在元光元年明矣荀紀著于此年徵吏民明當世之務下葛洪西京雜記亦云弘以元光五年為國士所推上為賢良若此續食之詔在八月則弘不容於今年已為左内史蓋此詔在今年不知何月故班氏繋之於年末耳其策文相類蓋出偶然或者此策乃弘先舉賢良時所對班氏誤以為此年之策疑未能明今從漢紀】弘奏事有不可不廷辨常與汲黯請間【師古曰求空隙之暇】黯先發之弘推其後天子常說【說讀曰悅】所言皆聽以此日益親貴弘常與公卿約議至上前皆倍其約以順上旨【倍蒲妹翻】汲黯廷詰弘曰齊人多詐而無情實始與臣等建此議今皆倍之不忠上問弘弘謝曰夫知臣者以臣為忠不知臣者以臣為不忠上然弘言左右幸臣每毁弘上益厚遇之<br />
<br />
  六年冬初筭商車【李奇曰始税商賈車船令出算】 大司農鄭當時言穿渭為渠下至河【渠起長安旁南山下至河三百餘里】漕關東粟徑易【易以䜴翻】又可以溉渠下民田萬餘頃春詔發卒數萬人穿渠如當時策三歲而通人以為便 匈奴入上谷殺略吏民遣車騎將軍衛青出上谷騎將軍公孫敖出代輕車將軍公孫賀出雲中驍騎將軍李廣出雁門各萬騎撃胡關市下衛青至龍城【龍城匈奴祭天大會諸部處】得胡首虜七百人公孫賀無所得公孫敖為胡所敗亡七千騎李廣亦為胡所敗胡生得廣置兩馬間絡而盛臥【敗補邁翻盛時征翻】行十餘里廣佯死暫騰而上胡兒馬上【師古曰騰跳躍也上時掌翻】奪其弓鞭馬南馳遂得脱歸漢下敖廣吏【下遐嫁翻】當斬贖為庶人唯青賜爵關内侯青雖出於奴虜【青本平陽公主家騎奴】然善騎射材力絶人遇士大夫以禮與士卒有恩衆樂為用有將帥材【騎奇寄翻樂音洛將即亮翻帥所類翻】故每出輒有功天下由此服上之知人 夏大旱蝗 六月上行幸雍 秋匈奴數盗邊【數所角翻】漁陽尤甚以衛尉韓安國為材官將軍屯漁陽<br />
<br />
  元朔元年【應劭曰朔蘇也孟軻曰后來其蘇蘇息也言萬民品物大繁息也師古曰朔猶始也言更為初始也蘇息之息非息生義應說失之】冬十一月詔曰朕深詔執事興亷舉孝庶幾成風紹休聖緒【師古曰休美也緒業也言紹先聖之休緒也幾居衣翻】夫十室之邑必有忠信三人並行厥有我師【論語曰十室之邑必有忠信如丘者焉又曰三人行必有我師焉】今或至闔郡而不薦一人【師古曰闔閉也摠一郡之中故曰闔】是化不下究而積行之君子壅于上聞也【師古曰究竟也言見壅遏不得聞達于天子也行下孟翻】且進賢受上賞蔽賢蒙顯戮古之道也其議二千石不舉者罪有司奏不舉孝不奉詔當以不敬論【張晏曰謂其不勤求士以報國也】不察亷不勝任也當免【張晏曰二千石當率身化下今親宰牧而無賢人為不勝任也勝音升】奏可 十二月江都易王非薨【非景帝子前二年封汝南二年徒江都】 皇子據生衛夫人之子也【是為戾太子 考異曰漢書武五子傳贊曰建元六年春戻太子生外戚傳衛皇后元朔元年生男據按枚臯傳云武帝春秋二十九乃有皇子與外戚傳合蓋贊語因蚩尤之旗致此誤亦猶五星聚在秦二世末年誤為漢元年也】三月甲子立衛夫人為皇后赦天下秋匈奴二萬騎入漢殺遼西太守略二千餘人圍韓安<br />
<br />
  國壁又入漁陽雁門各殺略千餘人安國益東徙屯北平數月病死 【考異曰安國死在明年於此終言之】天子乃復召李廣拜為右北平太守匈奴號曰漢之飛將軍避之數歲不敢入右北平車騎將軍衛青將三萬騎出雁門將軍李息出代青斬首虜數千人 東夷薉君南閭等共二十八萬人降為蒼海郡【服䖍曰薉貊在辰韓之北高麗沃沮之南東窮大海師古曰南閭薉君名食貨志彭吳開道通薉貊朝鮮置滄海郡陳夀夫餘傳魏時夫餘庫有王璧珪瓚傳世以為實耆老言先代所賜其印文言濊王之卬國有故城名濊城蓋本濊貊之地又濊傳云武帝滅朝鮮置樂浪郡自單單大嶺以西属樂浪自嶺以東七縣都尉主之皆以濊為民今不耐濊皆其種也班志樂浪束部都尉治不耐縣薉音濊降戶江翻考異曰史記平凖書曰彭吳賈滅朝鮮置蒼海之郡按滅朝鮮置蒼海兩事也不知何者出賈之謀】人徒之費擬於南夷燕齊之間靡然騷動 是歲魯共王餘長沙定王發皆薨【二王皆景帝子餘以前二年受封淮陽三年徙魯發亦以前二年受封長沙】 臨菑人主父偃【趙武靈王自號主父攴庶因以為氏】嚴安無終人徐樂【班志無終縣屬右北平郡春秋無終子之國】皆上書言事始偃游齊燕趙皆莫能厚遇諸生相與排擯不容家貧假貸無所得乃西入關上書闕下朝奏暮召入所言九事其八事為律令一事諫伐匈奴其辭曰司馬法曰國雖大好戰必亡天下雖平忘戰必危【師古曰司馬穰苴善用兵著書言兵法謂之司馬法一說司馬古主兵之官有軍陳用師之法予據史記齊威王使大夫追論古者司馬兵法而附穰苴於其中因號司馬穰苴兵法好呼到翻】夫怒者逆德也兵者凶器也争者末節也夫務戰勝窮武事者未有不悔者也昔秦皇帝并吞戰國務勝不休欲攻匈奴李斯諫曰不可夫匈奴無城郭之居委積之守【委于偽翻積子智翻委積者倉廪之藏也鄭氏曰少曰委多曰積】遷徙鳥舉難得而制也輕兵深入糧食必絶踵糧以行重不及事得其地不足以為利也得其民不可調而守也【李奇曰不可和調也】勝必殺之非民父母也靡敝中國【師古曰靡散也音縻】快心匈奴非長策也秦皇帝不聽遂使蒙恬將兵攻胡辟地千里【辟讀曰闢】以河為境地固沮澤鹹鹵不生五穀【沮將預翻五穀黍稷菽麥稻或曰黍稷秫稻粱】然後發天下丁男以守北河【河水逕安定北地朔方界皆北流至高闕始屈而東流過雲中楨陵縣又屈而南流故朔方雲中之北謂之北河杜佑曰衛青渡西河至高闕破匈奴河自今靈武郡之西南便北流千餘里過九原郡乃東流時帝都在秦所謂西河疑是此處其高闕當在河之西也史記趙武靈王築長城自代並隂山下至高闕則與漢書符矣其河自九原東流千里在京師直北漢史即云北河斯則西河之側者】暴兵露師十有餘年死者不可勝數【勝音升】終不能踰河而北是豈人衆不足兵革不備哉其勢不可也又使天下蜚芻輓粟【師古曰軍載芻藁令其疾至故曰飛芻輓謂引車船也】起於東陲琅邪負海之郡轉輸北河率三十鍾而致一后【東陲漢書作黄腄師古曰黄腄二縣並在東萊言自東萊及琅邪緣海諸郡皆令轉輸至北河六斛四斗為鍾計其道路所費凡用一百九十二斛乃得一石至杜佑曰腄即今文登縣腄直睡翻又音誰】男子疾耕不足於糧餉女子紡績不足於帷幕百姓靡敝【靡美為翻】孤寡老弱不能相養道路死者相望盖天下始畔秦也及至高皇帝定天下略地於邊聞匈奴聚于代谷之外而欲擊之御史成進諫曰不可夫匈奴之性獸聚而鳥散從之如影【師古曰擊也人之隂景言不可得子謂影物而生者也存滅不常難得而之】隨今以陛下盛德攻匈奴臣竊危之高帝不聽遂北至于代谷果有平城之圍高皇帝盖悔之甚乃使劉敬往結和親之約【事見高帝紀】然後天下忘干戈之事夫匈奴難得而制非一世也行盗侵驅【師古曰來侵邊境而驅掠人畜也】所以為業也天性固然上及虞夏殷周固弗程督【師古曰程課也督視責也】禽獸畜之【畜許六翻】不屬為人夫上不觀虞夏殷周之統而下循近世之失此臣之所大憂百姓之所疾苦也嚴安上書曰今天下人民用財侈靡車馬衣裘宫室皆競修飾調五聲使有節族【蘇林曰族音奏師古曰節止也奏凖也】雜五色使有文章重五味方丈於前以觀欲天下【孟康曰觀猶顯也師古曰顯示之使其慕欲也重直龍翻觀古玩翻】彼民之情見美則願之是教民以侈也侈而無節則不可贍民離本而儌末矣【師古曰贍足也離力智翻儌要求也一堯翻】末不可徒得故搢紳者不憚為詐帶劒者夸殺人以矯奪【師古曰夸大也競也矯偽也】而世不知愧是以犯法者衆臣願為民制度以防其淫使貧富不相燿以和其心心志定則盗賊消刑罰少隂陽和萬物蕃也【師古曰蕃扶元翻多也】㫺秦王意廣心逸欲威海外使蒙恬將兵以北攻胡又使尉屠雎將樓船之士以攻越【睢音雖】當是時秦禍北構於胡南挂於越【師古曰挂縣也】宿兵於無用之地進而不得行十餘年丁男被甲丁女轉輸苦不聊生自經於道樹【自經縊也】死者相望及秦皇帝崩天下大畔滅世絶祀窮兵之禍也故周失之弱秦失之彊不變之患也今徇西夷朝夜郎降羌僰畧薉州【朝直遥翻降戶江翻僰蒲北翻薉音穢】建城邑深入匈奴燔其龍城議者美之此人臣之利非天下之長策也徐樂上書曰臣聞天下之患在於土崩不在瓦解古今一也何謂土崩秦之末世是也陳涉無千乘之尊尺土之地身非王公大人名族之後郷曲之譽非有孔曾墨子之賢陶朱猗頓之富也【范蠡居于陶自號為陶朱公治產至鉅萬猗頓魯人用鹽鹽起與王者埒富】然起窮巷奮棘矜【棘與戟同師古曰矜者戟之把也矜讀曰其巾翻】偏袒大呼【呼火故翻】天下從風此其故何也由民困而主不恤下怨而上不知俗已亂而政不修此三者陳涉之所以為資也此之謂土崩故曰天下之患在乎土崩何謂瓦解吳楚齊趙之兵是也七國謀為大逆號皆稱萬乘之君帶甲數十萬威足以嚴其境内財足以勸其士民然不能西攘尺寸之地【師古曰攘謂侵取漢也】而身為禽於中原者此其故何也非權輕於匹夫而兵弱於陳涉也當是之時先帝之德未衰而安土樂俗之民衆【樂音洛】故諸侯無竟外之助【師古曰竟讀曰境】此之謂瓦解故曰天下之患不在瓦解此二體者安危之明要賢主之所宜留意而深察也間者關東五穀數不登年歲未復民多窮困重之以邊境之事【數所角翻師古曰復扶目翻重直用翻】推數循理而觀之民宜有不安其處者矣不安故易動易動者土崩之埶也【易以豉翻】故賢主獨觀萬化之原明於安危之機修之廟堂之上而銷未形之患也其要期使天下無土崩之埶而已矣書奏天子召見三人謂曰公等皆安在何相見之晚也皆拜為郎中 【考異曰漢書主父偃傳云元光元年三人上書按嚴安書云狥南夷朝夜郎降羌僰畧薉州此等事皆在元光元年後盖誤以朔字為光字耳】主父偃尤親幸一歲中凡四遷為中大夫大臣畏其口賂遺累千金或謂偃曰太横矣【遺于季翻横戶孟翻】偃曰吾生不五鼎食死即五鼎烹耳【張晏曰五鼎牛羊豕魚麋也諸侯王卿大夫也孔穎達曰少牢陳五鼎羊一豕二膚三魚四腊五師古曰五鼎烹謂被鑊烹之誅為主父偃被誅張本】<br />
<br />
  二年冬賜淮南王几杖毋朝【朝直遥翻 考異曰漢書武紀曰賜淮南菑川王几杖毋朝顔師古曰淮南王安菑川王志皆武帝諸父列也故賜几杖按諸侯表菑川王志在位三十五年以元光五年薨齊悼惠王世家高五王傳皆同此云菑川王志誤也】主父偃說上曰古者諸侯不過百里彊弱之形易制今諸侯或連城數十地方千里緩則驕奢易為淫亂急則阻其彊而合從以逆京師【說式芮翻易以豉翻從子容翻】以法割削之則逆節萌起【師古曰萌謂事之始生如艸木之萌芽也】前日鼂錯是也【事見十六卷景帝前二年鼂直遥翻錯干故翻】今諸侯子弟或十數而適嗣代立【適讀曰嫡】餘雖骨肉無尺地之封則仁孝之道不宣願陛下令諸侯得推恩分子弟以地侯之彼人人喜得所願上以德施實分其國不削而稍弱矣上從之春正月詔曰諸侯王或欲推私恩分子弟邑者令各條上【上時掌翻】朕且臨定其號名於是藩國始分而子弟畢侯矣 匈奴入上谷漁陽殺略吏民千餘人遣衛青李息出雲中以西至隴西擊胡之樓煩白羊王於河南得胡首虜數千牛羊百餘萬走白羊樓煩王遂取河南地詔封青為長平侯【班志長平侯國属汝南郡】青校尉蘇建張次公皆有功封建為平陵侯次公為岸頭侯【據功臣表平陵侯食邑於南陽郡武當縣界晉灼曰河東皮氏縣有岸頭亭校戶教翻】主父偃言河南地肥饒外阻河蒙恬城之以逐匈奴内省轉輸戍漕廣中國滅胡之本也上下公卿議皆言不便【下遐嫁翻】上竟用偃計立朔方郡使蘇建興十餘萬人築朔方城【括地志夏州朔方縣北什賁故城按是蘇建築什賁之號盖出蕃語也宋白曰漢朔方郡治三封縣今長澤縣冇三封故城什賁故城今為德静縣治】復繕故秦時蒙恬所為塞因河為固轉漕甚遠自山東咸被其勞【被皮義翻】費數十百鉅萬府庫並虚漢亦棄上谷之斗辟縣造陽地以予胡【孟康曰縣斗僻曲近胡師古曰斗絶也縣之斗曲入匈奴界者其中造陽地也杜佑曰造陽在今媯川郡之北辟讀口僻予讀曰與】 三月乙亥晦日有食之 夏募民徙朔方十萬口 主父偃說上曰茂陵初立【初立 建元二年】天下豪傑并兼之家亂衆之民皆可徙茂陵内實京師外銷姦猾此所謂不誅而害除上從之徙郡國豪傑及訾三百萬以上于茂陵【訾與貲同】軹人郭解【班志軹縣屬河南郡音止】關東大俠也亦在徙中衛將軍為言郭解家貧不中徙【為于偽翻言其貧不當在見徙之數中音竹仲翻】上曰解布衣權至使將軍為言【師古曰將軍為之言是為其所使也】此其家不貧卒徙解家【卒子恤翻】解平生睚眦殺人甚衆【師古曰睚音厓舉眼也眦即皆字謂目匡也言舉眼相忤者即殺之也一說睚五懈翻眦士懈翻睚眦瞋目貌二說並通】上聞之下吏捕治解【下遐嫁翻】所殺皆在赦前軹有儒生侍使者坐客譽郭解【譽音余】生曰解專以姧犯公法何謂賢解客聞殺此生斷其舌【斷丁管翻】吏以此責解解實不知殺者殺者亦絶莫知為誰吏奏解無罪公孫弘議曰解布衣為任俠行權以睚眦殺人解雖弗知此罪甚於解殺之當大逆無道【當謂處斷其罪盖以大逆無道之罪坐郭解也】遂族郭解 【考異曰荀紀以郭解事著于建元二年按武紀建元二年初置茂陵邑三年賜徙茂陵者錢當是時衛青公孫弘皆未貴又元朔二年徙郡國豪傑于茂陵此乃徙解之時也】<br />
<br />
  班固曰古者天子建國諸侯立家自卿大夫以至于庶人各有等差是以民服事其上而下無覬覦【師古曰覬幸也覦欲也幸得其所欲也覬音冀覦音俞又音喻】周室既微禮樂征伐自諸侯出桓文之後大夫世權陪臣執命【師古曰陪重也大夫世權晉六卿魯三桓齊田氏是也陪臣執命陽虎之類是也諸侯之臣於天子為陪臣大夫之家臣於諸侯為陪臣】陵夷至于戰國合從連衡【從子容翻衡讀曰横】繇是列國公子魏有信陵趙有平原齊有孟嘗楚有春申皆藉王公之埶競為游俠雞鳴狗盜【事見三卷赧王十七年】無不賓禮而趙相虞卿弃國捐君以周窮交魏齊之厄【事見五卷周赧王五十六年】信陵無忌竊符矯命戮將專師以赴平原之急【事見五卷赧王五十七年將即亮翻】皆以取重諸侯顯名天下搤腕而游談者以四豪為稱首【師古曰搤捉持也音戹腕烏貫翻四豪即魏信陵以下也】于是背公死黨之議成守職奉上之義廢矣【背蒲妹翻】及至漢興禁網疏濶未知匡改也是故代相陳豨從車千乘而吳濞淮南皆招賓客以千數【從才用翻濞普懿翻】外戚大臣魏其武安之屬競逐於京師布衣游俠劇孟郭解之徒馳騖於閭閻權行州域力折公侯衆庶榮其名迹覬而慕之雖其陷于刑辟【辟毗亦翻】自與殺身成名若季路仇牧死而不悔【季路死于衛侯輒之難仇牧死于宋閔公之難事並見左傳】故曾子曰上失其道民散久矣【見論語】非明主在上示之以好惡【好呼到翻惡烏路翻】齊之以禮法民曷由知禁而反正乎古之正法五伯三王之罪人也而六國五伯之罪人也【伯讀曰覇】夫四豪者又六國之罪人也况於郭解之倫以匹夫之細竊殺生之權其罪已不容於誅矣觀其温良泛愛振窮周急謙不伐亦皆有絶異之姿惜乎不入於道德苟放縱於末流殺身亡宗非不幸也<br />
<br />
  荀悦論曰世有三游德之賊也一曰游俠二曰游說三曰游行【說式芮翻行下孟翻】立氣埶作威福結私交以立強于世者謂之游俠飾辯辭設詐謀馳逐於天下以要時埶者謂之游說【要一遥翻】色取仁以合時好【好呼到翻】連黨類立虚譽以為權利者謂之游行此三者亂之所由生也傷道害德敗法惑世【敗補邁翻】先王之所慎也國有四民各修其業不由四民之業者謂之姧民【四民士農工商也】姧民不生王道乃成凡此三游之作生於季世周秦之末尤甚焉上不明下不正制度不立綱紀弛廢以毁譽為榮辱不核其真【譽音余下同一】以愛憎為利害不論其實以喜怒為賞罸不察其理上下相冒萬事乖錯是以言論者計薄厚而吐辭選舉者度親疎而舉筆【度徒洛翻】善惡謬於衆聲功罪亂於王法然則利不可以義求害不可以道避也是以君子犯禮小人犯法犇走馳騁越職僭度飾華廢實競趣時利【趣七喻翻】簡父兄之尊而崇賓客之禮薄骨肉之恩而篤朋友之愛忘修身之道而求衆人之譽割衣食之業以供饗宴之好【好呼到翻】苞苴盈于門庭聘問交于道路【裹曰苞藉曰苴詩箋以果實相遺者苞苴之又曰苞苴裹魚肉或以葦或以茅在傳注云聘執玉帛以相存問】書記繁於公文私務衆於官事于是流俗成而正道壞矣是以聖王在上經國序民正其制度善惡要於功罪而不淫於毁譽【要一遥翻】聽其言而責其事舉其名而指其實故實不應其聲者謂之虚情不覆其貌者謂之偽【覆敷又翻】毁譽失其真者謂之誣言事失其類者謂之罔虚偽之行不得設【行下孟翻】誣罔之辭不得行有罪惡者無僥倖無罪過者不憂懼請謁無所行【請求也謁告也】貨賂無所用息華文去浮辭【去羌呂翻】禁偽辯絶淫智放百家之紛亂壹聖人之至道養之以仁惠文之以禮樂則風俗定而大化成矣<br />
<br />
  燕王定國與父康王姬姧奪弟妻為姬殺肥如令郢人【肥如燕之屬縣燕國除入漢屬遼西郡應劭曰肥子奔燕燕封於此】郢人兄弟上書告之主父偃從中發其事公卿請誅定國上許之定國自殺國除【文帝初王澤始封于燕傳子康王嘉文帝九年嘉薨定國嗣盖立四十二年矣】齊厲王次昌亦與其姊紀翁主通【齊孝王將閭文帝十六年受封傳子懿王夀夀傳次昌】主父偃欲納其女於齊王齊紀太后不許偃因言于上曰齊臨菑十萬戶市租千金人衆殷富鉅于長安非天子親弟愛子不得王此【王于况翻】今齊王於親屬益疏【疏與疎同】又聞與其姊亂請治之于是帝拜偃為齊相且正其事偃至齊急治王後宫宦者辭及王王懼飲藥自殺偃少時游齊及燕趙【少詩照翻】及貴連敗燕齊【敗補邁翻】趙王彭祖懼【彭祖景帝子前二年封廣川五年徙趙】上書告主父偃受諸侯金以故諸侯子弟多以得封者及齊王自殺上聞大怒以為偃刼其王令自殺乃徵下吏【下遐嫁翻】偃服受諸侯金實不刼王令自殺上欲勿誅公孫弘曰齊王自殺無後國除為郡入漢主父偃本首惡陛下不誅偃無以謝天下乃遂族主父偃 張歐免上欲以蓼侯孔臧為御史大夫【班志蓼縣属衡山國春秋之蓼國也音了康曰音六未知其何據蓼侯孔聚高祖功臣臧其子也臧自言世修經學盖謂孔子後也安國為從弟安國孔子十三世孫】臧辭曰臣世以經學為業乞為太常典臣家業與從弟侍中安國【百官表侍中加官得出入禁中應劭曰入侍天子故曰侍中續漢書曰侍中比二千石無員漢官儀曰侍中左蟬右貂本秦丞相史往來殿内故謂之侍中分掌乘輿服物下至亵器虎子之属武帝時孔安國為侍中以其儒者特聽掌御座唾壺朝廷榮之從才用翻】綱紀古訓使永垂來嗣上乃以臧為太常其禮賜如三公<br />
<br />
  三年冬匈奴軍臣單于死其弟左谷蠡王伊稚斜自立為單于【匈奴左右谷蠡王在左右賢王之下谷蠡音鹿黎索隱曰稚持利翻斜士嗟翻鄒誕生音直牙翻盖稚斜胡人語近得其實】攻破軍臣單于太子於單於單亡降漢【於單音丹降戶江翻】 以公孫弘為御史大夫是時方通西南夷東置蒼海北築朔方之郡公孫弘數諫以為罷敝中國以奉無用之地願罷之【數所角翻為罷讀曰疲】天子使朱買臣等難以置朔方之便發十策弘不得一【師古曰言其利害十條弘無以應之難乃旦翻】弘乃謝曰山東鄙人不知其便若是願罷西南夷蒼海而專奉朔方上乃許之春罷蒼海郡弘為布被食不重肉【言不重肉味也重音直龍翻】汲黯曰弘位在三公奉禄甚多【奉扶用翻】然為布被此詐也上問弘弘謝曰有之夫九卿與臣善者無過黯然今日廷詰弘誠中弘之病【中竹仲翻】夫以三公為布被與小吏無差誠飾詐欲以釣名【師古曰釣取也言若釣魚之謂也】如汲黯言且無汲黯忠陛下安得聞此言天子以為謙讓愈益尊之 三月赦天下 夏四月丙子封匈奴太子於單為涉安侯數月而卒 初匈奴降者言月氏故居敦煌祁連間為彊國【降戶江翻氏音支敦煌張掖匈奴破月氏使昆邪王居之漢開置郡祁連山名即天山也匈奴呼天為祁連在張掖西北敦徒門翻】匈奴冒頓攻破之老上單于殺月氏王以其頭為飲器餘衆遁逃遠去怨匈奴無與共擊之上募能通使月氏者【使疏吏翻】漢中張騫以郎應募出隴西徑匈奴中單于得之留騫十餘歲騫得間亡鄉月氏【間古莧翻鄉讀曰嚮】西走數十日至大宛【西域傳大宛國治貴山城去長安萬二千五百七十里西南至大月氏所居六百九十里宛於元翻】大宛聞漢之饒財欲通不得見騫喜為發導譯抵康居【導者引路之人譯者傳言之人也康居國治樂越匿地到卑闐城去長安萬二千三百里為于偽翻】傳致大月氏【傳張戀翻】大月氏太子為王既擊大夏分其地而居之【大夏國在大宛西南都媯水南月氏居媯水北】地肥饒少寇【少詩沼翻】殊無報胡之心騫留歲餘竟不能得月氏要領【李奇曰要領要契也師古曰要衣要也領衣領也凡持衣者執要與領言騫不能得月氏意趣無以持歸於漢故以要領為喻要一遥翻】乃還並南山【史記曰南山即連終南山從京南東至華山東北連延至海即中條山也從京南而西連接至葱嶺萬餘里故云並南山也西域傳云其南山東出金城與漢南山屬還從宣翻又如字下同並步浪翻】欲從羌中歸復為匈奴所得【復扶又翻】留歲餘會伊稚斜逐於單匈奴國内亂騫乃與堂邑氏奴甘父逃歸【服䖍曰堂邑姓也漢人其奴名甘父父音甫】上拜騫為太中大夫甘父為奉使君騫初行時百餘人去十三歲唯二人得還 【考異曰史記西南夷傳曰元狩元年張騫使大夏來言通身毒國之利按年表騫以元朔六年二月甲辰封博望侯必非元狩元年始歸也或者元狩元年天子始令騫通身毒國疑不能明故因是歲伊稚斜立終言之】 匈奴數萬騎入塞殺代郡太守恭【代郡唐蔚州武州界】及略千餘人 六月庚午皇太后崩【武帝母王太后也】 秋罷西夷獨置南夷夜郎兩縣一都尉稍令犍為自葆就【師古曰葆與保同令自保守且成其郡縣】專力城朔方匈奴又入鴈門殺畧千餘人 是歲中大夫張湯為廷尉湯為人多詐舞智以御人時上方鄉文學【郷讀曰嚮】湯陽浮慕事董仲舒公孫弘等以千乘兒寛為奏讞掾【兒本郳姓以國為氏其後去邑以為廷尉掾專主奏讞也兒五奚翻讞魚蹇翻掾俞絹翻】以古法義决疑獄所治即上意所欲罪與監史深禍者【班表廷尉冇左右監秩千石漢官曰廷尉獄史二十七人深禍謂持文深刻欲致人於禍者】即上意所欲釋與監史輕平者上由是悦之湯於故人子弟調護之尤厚【師古曰調和適之令得其所護謂保佑之也】其造請諸公【師古曰造詣至也請謁問也造七到翻】不避寒暑是以湯雖文深意忌不專平【文深謂持文深意忌謂其意忌前也不專平謂不專於持平也】然得此聲譽汲黯數質責湯於上前【質對也對面責之也或曰質正也以正義責之數所角翻】曰公為正卿【漢官九卿之外又有列于九卿者故謂九卿為正卿】上不能褒先帝之功業下不能抑天下之邪心安國富民使囹圄空虚何空取高皇帝約束紛更之為【師古曰言何為乃紛亂而改更也更工衡翻】而公以此無種矣【言當誅及子孫種章勇翻】黯時與湯論議湯辯常在文深小苛黯伉厲守高【伉口浪翻健也高也厲嚴也】不能屈忿發罵曰天下謂刀筆吏不可為公卿果然必湯也令天下重足而立【累足而立懼之甚也重直龍翻】側目而視矣<br />
<br />
  四年冬上行幸甘泉 夏匈奴入代郡定襄上郡【上郡唐延綏銀之地高祖置定襄郡括地志定襄故城在朔州善陽縣北三百八十里】各三萬騎殺略數千人<br />
<br />
  資治通鑑卷十八<br />
<br />
<史部,編年類,資治通鑑>  <br>
   </div> 

<script src="/search/ajaxskft.js"> </script>
 <div class="clear"></div>
<br>
<br>
 <!-- a.d-->

 <!--
<div class="info_share">
</div> 
-->
 <!--info_share--></div>   <!-- end info_content-->
  </div> <!-- end l-->

<div class="r">   <!--r-->



<div class="sidebar"  style="margin-bottom:2px;">

 
<div class="sidebar_title">工具类大全</div>
<div class="sidebar_info">
<strong><a href="http://www.guoxuedashi.com/lsditu/" target="_blank">历史地图</a></strong>  
<a href="http://www.880114.com/" target="_blank">英语宝典</a>  
<a href="http://www.guoxuedashi.com/13jing/" target="_blank">十三经检索</a> 
<br><strong><a href="http://www.guoxuedashi.com/gjtsjc/" target="_blank">古今图书集成</a></strong> 
<a href="http://www.guoxuedashi.com/duilian/" target="_blank">对联大全</a> <strong><a href="http://www.guoxuedashi.com/xiangxingzi/" target="_blank">象形文字典</a></strong> 

<br><a href="http://www.guoxuedashi.com/zixing/yanbian/">字形演变</a>  <strong><a href="http://www.guoxuemi.com/hafo/" target="_blank">哈佛燕京中文善本特藏</a></strong>
<br><strong><a href="http://www.guoxuedashi.com/csfz/" target="_blank">丛书&方志检索器</a></strong> <a href="http://www.guoxuedashi.com/yqjyy/" target="_blank">一切经音义</a>  

<br><strong><a href="http://www.guoxuedashi.com/jiapu/" target="_blank">家谱族谱查询</a></strong>  <strong><a href="http://shufa.guoxuedashi.com/sfzitie/" target="_blank">书法字帖欣赏</a></strong> 
<br>

</div>
</div>


<div class="sidebar" style="margin-bottom:0px;">

<font style="font-size:22px;line-height:32px">QQ交流群9:489193090</font>


<div class="sidebar_title">手机APP 扫描或点击</div>
<div class="sidebar_info">
<table>
<tr>
	<td width=160><a href="http://m.guoxuedashi.com/app/" target="_blank"><img src="/img/gxds-sj.png" width="140"  border="0" alt="国学大师手机版"></a></td>
	<td>
<a href="http://www.guoxuedashi.com/download/" target="_blank">app软件下载专区</a><br>
<a href="http://www.guoxuedashi.com/download/gxds.php" target="_blank">《国学大师》下载</a><br>
<a href="http://www.guoxuedashi.com/download/kxzd.php" target="_blank">《汉字宝典》下载</a><br>
<a href="http://www.guoxuedashi.com/download/scqbd.php" target="_blank">《诗词曲宝典》下载</a><br>
<a href="http://www.guoxuedashi.com/SiKuQuanShu/skqs.php" target="_blank">《四库全书》下载</a><br>
</td>
</tr>
</table>

</div>
</div>


<div class="sidebar2">
<center>


</center>
</div>

<div class="sidebar"  style="margin-bottom:2px;">
<div class="sidebar_title">网站使用教程</div>
<div class="sidebar_info">
<a href="http://www.guoxuedashi.com/help/gjsearch.php" target="_blank">如何在国学大师网下载古籍?</a><br>
<a href="http://www.guoxuedashi.com/zidian/bujian/bjjc.php" target="_blank">如何使用部件查字法快速查字?</a><br>
<a href="http://www.guoxuedashi.com/search/sjc.php" target="_blank">如何在指定的书籍中全文检索?</a><br>
<a href="http://www.guoxuedashi.com/search/skjc.php" target="_blank">如何找到一句话在《四库全书》哪一页?</a><br>
</div>
</div>


<div class="sidebar">
<div class="sidebar_title">热门书籍</div>
<div class="sidebar_info">
<a href="/so.php?sokey=%E8%B5%84%E6%B2%BB%E9%80%9A%E9%89%B4&kt=1">资治通鉴</a> <a href="/24shi/"><strong>二十四史</strong></a>&nbsp; <a href="/a2694/">野史</a>&nbsp; <a href="/SiKuQuanShu/"><strong>四库全书</strong></a>&nbsp;<a href="http://www.guoxuedashi.com/SiKuQuanShu/fanti/">繁体</a>
<br><a href="/so.php?sokey=%E7%BA%A2%E6%A5%BC%E6%A2%A6&kt=1">红楼梦</a> <a href="/a/1858x/">三国演义</a> <a href="/a/1038k/">水浒传</a> <a href="/a/1046t/">西游记</a> <a href="/a/1914o/">封神演义</a>
<br>
<a href="http://www.guoxuedashi.com/so.php?sokeygx=%E4%B8%87%E6%9C%89%E6%96%87%E5%BA%93&submit=&kt=1">万有文库</a> <a href="/a/780t/">古文观止</a> <a href="/a/1024l/">文心雕龙</a> <a href="/a/1704n/">全唐诗</a> <a href="/a/1705h/">全宋词</a>
<br><a href="http://www.guoxuedashi.com/so.php?sokeygx=%E7%99%BE%E8%A1%B2%E6%9C%AC%E4%BA%8C%E5%8D%81%E5%9B%9B%E5%8F%B2&submit=&kt=1"><strong>百衲本二十四史</strong></a>  <a href="http://www.guoxuedashi.com/so.php?sokeygx=%E5%8F%A4%E4%BB%8A%E5%9B%BE%E4%B9%A6%E9%9B%86%E6%88%90&submit=&kt=1"><strong>古今图书集成</strong></a>
<br>

<a href="http://www.guoxuedashi.com/so.php?sokeygx=%E4%B8%9B%E4%B9%A6%E9%9B%86%E6%88%90&submit=&kt=1">丛书集成</a> 
<a href="http://www.guoxuedashi.com/so.php?sokeygx=%E5%9B%9B%E9%83%A8%E4%B8%9B%E5%88%8A&submit=&kt=1"><strong>四部丛刊</strong></a>  
<a href="http://www.guoxuedashi.com/so.php?sokeygx=%E8%AF%B4%E6%96%87%E8%A7%A3%E5%AD%97&submit=&kt=1">說文解字</a> <a href="http://www.guoxuedashi.com/so.php?sokeygx=%E5%85%A8%E4%B8%8A%E5%8F%A4&submit=&kt=1">三国六朝文</a>
<br><a href="http://www.guoxuedashi.com/so.php?sokeytm=%E6%97%A5%E6%9C%AC%E5%86%85%E9%98%81%E6%96%87%E5%BA%93&submit=&kt=1"><strong>日本内阁文库</strong></a> <a href="http://www.guoxuedashi.com/so.php?sokeytm=%E5%9B%BD%E5%9B%BE%E6%96%B9%E5%BF%97%E5%90%88%E9%9B%86&ka=100&submit=">国图方志合集</a> <a href="http://www.guoxuedashi.com/so.php?sokeytm=%E5%90%84%E5%9C%B0%E6%96%B9%E5%BF%97&submit=&kt=1"><strong>各地方志</strong></a>

</div>
</div>


<div class="sidebar2">
<center>

</center>
</div>
<div class="sidebar greenbar">
<div class="sidebar_title green">四库全书</div>
<div class="sidebar_info">

《四库全书》是中国古代最大的丛书,编撰于乾隆年间,由纪昀等360多位高官、学者编撰,3800多人抄写,费时十三年编成。丛书分经、史、子、集四部,故名四库。共有3500多种书,7.9万卷,3.6万册,约8亿字,基本上囊括了古代所有图书,故称“全书”。<a href="http://www.guoxuedashi.com/SiKuQuanShu/">详细>>
</a>

</div> 
</div>

</div>  <!--end r-->

</div>
<!-- 内容区END --> 

<!-- 页脚开始 -->
<div class="shh">

</div>

<div class="w1180" style="margin-top:8px;">
<center><script src="http://www.guoxuedashi.com/img/plus.php?id=3"></script></center>
</div>
<div class="w1180 foot">
<a href="/b/thanks.php">特别致谢</a> | <a href="javascript:window.external.AddFavorite(document.location.href,document.title);">收藏本站</a> | <a href="#">欢迎投稿</a> | <a href="http://www.guoxuedashi.com/forum/">意见建议</a> | <a href="http://www.guoxuemi.com/">国学迷</a> | <a href="http://www.shuowen.net/">说文网</a><script language="javascript" type="text/javascript" src="https://js.users.51.la/17753172.js"></script><br />
  Copyright &copy; 国学大师 古典图书集成 All Rights Reserved.<br>
  
  <span style="font-size:14px">免责声明:本站非营利性站点,以方便网友为主,仅供学习研究。<br>内容由热心网友提供和网上收集,不保留版权。若侵犯了您的权益,来信即刪。scp168@qq.com</span>
  <br />
ICP证:<a href="http://www.beian.miit.gov.cn/" target="_blank">鲁ICP备19060063号</a></div>
<!-- 页脚END --> 
<script src="http://www.guoxuedashi.com/img/plus.php?id=22"></script>
<script src="http://www.guoxuedashi.com/img/tongji.js"></script>

</body>
</html>
