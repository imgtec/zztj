<!DOCTYPE html PUBLIC "-//W3C//DTD XHTML 1.0 Transitional//EN" "http://www.w3.org/TR/xhtml1/DTD/xhtml1-transitional.dtd">
<html xmlns="http://www.w3.org/1999/xhtml">
<head>
<meta http-equiv="Content-Type" content="text/html; charset=utf-8" />
<meta http-equiv="X-UA-Compatible" content="IE=Edge,chrome=1">
<title>資治通鑒_135-資治通鑑卷一百三十四_135-資治通鑑卷一百三十四</title>
<meta name="Keywords" content="資治通鑒_135-資治通鑑卷一百三十四_135-資治通鑑卷一百三十四">
<meta name="Description" content="資治通鑒_135-資治通鑑卷一百三十四_135-資治通鑑卷一百三十四">
<meta http-equiv="Cache-Control" content="no-transform" />
<meta http-equiv="Cache-Control" content="no-siteapp" />
<link href="/img/style.css" rel="stylesheet" type="text/css" />
<script src="/img/m.js?2020"></script> 
</head>
<body>
 <div class="ClassNavi">
<a  href="/24shi/">二十四史</a> | <a href="/SiKuQuanShu/">四库全书</a> | <a href="http://www.guoxuedashi.com/gjtsjc/"><font  color="#FF0000">古今图书集成</font></a> | <a href="/renwu/">历史人物</a> | <a href="/ShuoWenJieZi/"><font  color="#FF0000">说文解字</a></font> | <a href="/chengyu/">成语词典</a> | <a  target="_blank"  href="http://www.guoxuedashi.com/jgwhj/"><font  color="#FF0000">甲骨文合集</font></a> | <a href="/yzjwjc/"><font  color="#FF0000">殷周金文集成</font></a> | <a href="/xiangxingzi/"><font color="#0000FF">象形字典</font></a> | <a href="/13jing/"><font  color="#FF0000">十三经索引</font></a> | <a href="/zixing/"><font  color="#FF0000">字体转换器</font></a> | <a href="/zidian/xz/"><font color="#0000FF">篆书识别</font></a> | <a href="/jinfanyi/">近义反义词</a> | <a href="/duilian/">对联大全</a> | <a href="/jiapu/"><font  color="#0000FF">家谱族谱查询</font></a> | <a href="http://www.guoxuemi.com/hafo/" target="_blank" ><font color="#FF0000">哈佛古籍</font></a> 
</div>

 <!-- 头部导航开始 -->
<div class="w1180 head clearfix">
  <div class="head_logo l"><a title="国学大师官网" href="http://www.guoxuedashi.com" target="_blank"></a></div>
  <div class="head_sr l">
  <div id="head1">
  
  <a href="http://www.guoxuedashi.com/zidian/bujian/" target="_blank" ><img src="http://www.guoxuedashi.com/img/top1.gif" width="88" height="60" border="0" title="部件查字,支持20万汉字"></a>


<a href="http://www.guoxuedashi.com/help/yingpan.php" target="_blank"><img src="http://www.guoxuedashi.com/img/top230.gif" width="600" height="62" border="0" ></a>


  </div>
  <div id="head3"><a href="javascript:" onClick="javascript:window.external.AddFavorite(window.location.href,document.title);">添加收藏</a>
  <br><a href="/help/setie.php">搜索引擎</a>
  <br><a href="/help/zanzhu.php">赞助本站</a></div>
  <div id="head2">
 <a href="http://www.guoxuemi.com/" target="_blank"><img src="http://www.guoxuedashi.com/img/guoxuemi.gif" width="95" height="62" border="0" style="margin-left:2px;" title="国学迷"></a>
  

  </div>
</div>
  <div class="clear"></div>
  <div class="head_nav">
  <p><a href="/">首页</a> | <a href="/ShuKu/">国学书库</a> | <a href="/guji/">影印古籍</a> | <a href="/shici/">诗词宝典</a> | <a   href="/SiKuQuanShu/gxjx.php">精选</a> <b>|</b> <a href="/zidian/">汉语字典</a> | <a href="/hydcd/">汉语词典</a> | <a href="http://www.guoxuedashi.com/zidian/bujian/"><font  color="#CC0066">部件查字</font></a> | <a href="http://www.sfds.cn/"><font  color="#CC0066">书法大师</font></a> | <a href="/jgwhj/">甲骨文</a> <b>|</b> <a href="/b/4/"><font  color="#CC0066">解密</font></a> | <a href="/renwu/">历史人物</a> | <a href="/diangu/">历史典故</a> | <a href="/xingshi/">姓氏</a> | <a href="/minzu/">民族</a> <b>|</b> <a href="/mz/"><font  color="#CC0066">世界名著</font></a> | <a href="/download/">软件下载</a>
</p>
<p><a href="/b/"><font  color="#CC0066">历史</font></a> | <a href="http://skqs.guoxuedashi.com/" target="_blank">四库全书</a> |  <a href="http://www.guoxuedashi.com/search/" target="_blank"><font  color="#CC0066">全文检索</font></a> | <a href="http://www.guoxuedashi.com/shumu/">古籍书目</a> | <a   href="/24shi/">正史</a> <b>|</b> <a href="/chengyu/">成语词典</a> | <a href="/kangxi/" title="康熙字典">康熙字典</a> | <a href="/ShuoWenJieZi/">说文解字</a> | <a href="/zixing/yanbian/">字形演变</a> | <a href="/yzjwjc/">金 文</a> <b>|</b>  <a href="/shijian/nian-hao/">年号</a> | <a href="/diming/">历史地名</a> | <a href="/shijian/">历史事件</a> | <a href="/guanzhi/">官职</a> | <a href="/lishi/">知识</a> <b>|</b> <a href="/zhongyi/">中医中药</a> | <a href="http://www.guoxuedashi.com/forum/">留言反馈</a>
</p>
  </div>
</div>
<!-- 头部导航END --> 
<!-- 内容区开始 --> 
<div class="w1180 clearfix">
  <div class="info l">
   
<div class="clearfix" style="background:#f5faff;">
<script src='http://www.guoxuedashi.com/img/headersou.js'></script>

</div>
  <div class="info_tree"><a href="http://www.guoxuedashi.com">首页</a> > <a href="/SiKuQuanShu/fanti/">四库全书</a>
 > <h1>资治通鉴</h1> <!--         下载:【右键另存为】即可 --></div>
  <div class="info_content zj clearfix">
  
<div class="info_txt clearfix" id="show">
<center style="font-size:24px;">135-資治通鑑卷一百三十四</center>
    資治通鑑卷一百三十四 宋 司馬光 撰<br />
<br />
  胡三省 音註<br />
<br />
  宋紀十六【起柔兆執徐盡著雍敦牂凡三年】<br />
<br />
  蒼梧王下<br />
<br />
  元徽四年春正月己亥帝耕籍田大赦 二月魏司空東郡王陸定國坐恃恩不法免官爵爲兵 魏馮太后内行不正【行下孟翻】以李弈之死怨顯祖【事見一百三十二卷明帝泰始之六年】密行鴆毒夏六月辛未顯祖殂【年二十三考異曰元行冲後魏國典云太后伏壯士於禁中太上入謁遂崩按事若如此安得不彰而中外恬然不以為怪又孝文終不之知按後魏書及北史皆無殺事而天象志云顯文暴崩蓋實有鴆毒之禍今從之】壬申大赦改元承明葬顯祖於金陵【金陵在雲中】諡曰獻文皇帝 魏大司馬大將軍代人萬安國坐矯詔殺神部長奚買奴賜死【神部八部之一也長知兩翻】 戊寅魏以征西大將軍安樂王長樂爲太尉【樂音洛】尚書左僕射宜都王目辰爲司徒南部尚書李訢爲司空【訢許斤翻】尊皇太后曰太皇太后復臨朝稱制【魏高宗之殂顯祖方年十二馮太后臨朝稱制時宋太宗泰始二年也至次年太后歸政今既鴆顯祖而高祖尚幼故復臨朝復扶又翻朝直遥翻】以馮熙爲侍中太師中書監熙自以外戚固辭内任乃除都督洛州刺史【魏太宗取洛陽以晉司州爲洛州】侍中太師如故顯祖神主袝太廟有司奏廟中執事之官請依故事皆賜爵祕書令廣平程駿上言建侯裂地帝王所重或以親賢或因功伐【以勞定國曰功積功曰伐】未聞神主祔廟而百司受封者也皇家故事蓋一時之恩豈可爲長世之法乎太后善而從之謂羣臣曰凡議事當依古典正言豈得但修【修當作循】故事乃賜駿衣一襲帛二百匹太后性聰察知書計曉政事被服儉素【被皮義翻】膳羞減於故事什七八而猜忍多權數高祖性至孝能承顔順志事無大小皆仰成於太后【經典釋文仰如字又五亮翻】太后往往專决不復關白於帝【復扶又翻】所幸宦者高平王琚安定張祐杞嶷【康曰杞姓也出自夏后氏之後嶷魚力翻】馮翊王遇略陽苻承祖高陽王質皆依勢用事祐官至尚書左僕射爵新平王琚官至征南將軍爵高平王嶷等官亦至侍中吏部尚書刺史爵爲公侯賞賜巨萬賜鐵劵許以不死【康曰說文劵契也釋名曰劵綣也相約束繾綣爲劵也】又太卜令姑臧王叡得幸於太后超遷至侍中吏部尚書爵太原公祕書令李冲雖以才進亦由私寵賞賜皆不可勝紀【太和末年高菩薩之禍后啓之也后雖獲終其世禍及門戶矣勝音升】又外禮人望東陽王丕游明根等皆極其優厚每褒賞叡等輒以丕等參之以示不私丕烈帝之玄孫【柘跋翳槐謚烈帝】冲寶之子也【魏世祖太平眞君五年李寶入朝其子冲遂貴顯於魏】太后自以失行畏人議已羣下語言小涉疑忌輒殺之然所寵幸左右苟有小過必加笞箠或至百餘【箠止橤翻】而無宿憾尋復待之如初或因此更富貴故左右雖被罰終無離心【此史之所謂權數也吁行下孟翻復扶又翻被皮義翻】 乙亥加蕭道成尚書左僕射劉秉中書令 楊運長阮佃夫等忌建平王景素益甚【通鑑書禍始於上卷上年佃音田】景素乃與錄事參軍陳郡殷濔【濔莫比翻】中兵參軍略陽垣慶延參軍沈顒左暄等謀爲自全之計【顒魚容翻】遣人往來建康要結才力之士【要讀曰邀】冠軍將軍黃回游擊將軍高道慶輔國將軍曹欣之前軍將軍韓道清長水校尉郭蘭之羽林監垣祇祖【漢東都之制羽林左右監主羽林騎屬光祿勲至晉以羽林屬二衛而監不見於志當是江左復置冠古玩翻校戶教翻】皆隂與通謀武人不得志者無不歸之時帝好獨出游走郊野欣之謀據石頭城伺帝出作亂道清蘭之欲說蕭道成因帝夜出執帝迎景素道成不從者即圖之景素每禁使緩之楊阮微聞其事遣傖人周天賜僞投景素【好呼到翻說輸芮翻下譬說同伺相吏翻傖助庚翻江東人謂楚人别種為傖亦謂西北人為傖】勸令舉兵景素知之斬天賜首送臺秋七月祇祖率數百人自建康奔京口云京師已潰亂勸令速入景素信之戊子據京口起兵士民赴之者以千數楊阮聞祇祖叛走即命纂嚴己丑遣驃騎將軍任農夫領軍將軍黃回左軍將軍蘭陵李安民將步軍右軍將軍張保將水軍以討之【驍堅堯翻騎奇寄翻任音壬民將保將即亮翻】辛卯又命南豫州刺史段佛榮爲都統【都統之名始此】蕭道成知黃回有異志故使安民佛榮與之偕行【道成知黃回不附已既使之討景素又使之討沈攸之二難既平然後殺之則足以知回於當時有幹略而道成智數又一時所不及者】回私戒其士卒道逢京口兵勿得戰道成屯玄武湖冠軍將軍蕭賾鎭東府【冠古玩翻賾士革翻】始安王伯融都鄉侯伯猷皆建安王休仁之子也楊阮忌其年長悉稱詔賜死景素欲斷竹里以拒臺軍【長知兩翻斷丁管翻下斷峽同】垣慶延垣祇祖沈顒皆曰今天時旱熱臺軍遠來疲困引之使至以逸待勞可一戰而克殷濔等固爭不能得農夫等既至縱火燒市邑慶延等各相顧望莫有鬬志景素本乏威畧恇擾不知所爲【恇去王翻】黃回迫於段佛榮且見京口軍弱遂不發張保泊西渚【西渚在京口城西今西津渡口是也】景素左右勇士數十人自相要結【要一遥翻】進擊水軍甲午張保敗死而諸將不相應赴復爲臺軍所破【復扶又翻】臺軍既薄城下顒先帥衆走【帥讀曰率】祇祖次之其餘諸軍相繼奔退獨左暄與臺軍力戰於萬歲樓下而所配兵力甚弱不能敵而散乙未拔京口黃回軍先入自以有誓不殺諸王乃以景素讓殿中將軍張倪奴倪奴擒景素斬之并其三子同黨垣祇祖等數十人皆伏誅蕭道成釋黄回高道慶不問撫之如舊【撫之所以安反側事定之後决不能容之】是日解嚴丙申大赦初巴東建平蠻反沈攸之遣軍討之及景素反攸之急追峽中軍以赴建康巴東太守劉攘兵建平太守劉道欣疑攸之有異謀勒兵斷峽不聽軍下攘兵子天賜爲荆州西曹【西曹者漢晉公府之西曹掾】攸之遣天賜往諭之攘兵知景素實反乃釋甲謝愆攸之待之如故劉道欣堅守建平攘兵譬說不回乃與伐蠻軍攻斬之【沈攸之用劉攘兵卒為攘兵所禍蕭道成用黃回而權以濟事非用人之難用勢之難也說輸芮翻】 甲辰魏主追尊其母李貴人曰思皇后【李氏李惠之女高祖之母也為後李惠貴張本】 八月丁卯立皇弟翽爲南陽王【翽呼會翻】嵩爲新興王禧爲始建王 庚午以給事黃門侍郎阮佃夫爲南豫州刺史留鎭京師 九月戊子賜驍騎將軍高道慶死【驍堅堯翻騎奇寄翻】 冬十月辛酉以吏部尚書王僧䖍爲尚書左僕射 十一月戊子魏以太尉安樂王長樂爲定州刺史司空李訢爲徐州刺史【樂音洛訢許斤翻】<br />
<br />
  順皇帝【諱準字仲謀明帝第三子也小字知觀實桂陽王休範之子諡法慈和徧服曰順蕭氏所以謚之曰順者以其順天命人心而禪代也】<br />
<br />
  昇明元年【是年七月帝即位始改元昇明】春正月乙酉朔魏改元太和己酉略陽民王元壽聚衆五千餘家自稱衝天王二<br />
<br />
  月辛未魏秦益二州刺史尉洛侯擊破之【秦益二州此魏所謂南秦東益也尉紆勿翻】 三月庚子魏以東陽王丕爲司徒 夏四月丁卯魏主如白登壬申如崞山【崞音郭】 初蒼梧王在東宫好緣漆帳竿【好呼到翻】去地丈餘喜怒乖節主帥不能禁【主帥謂東宮齋内主帥也帥所類翻】太宗屢敕陳太妃痛捶之【陳氏蒼梧王之母也即位尊為太妃捶止橤翻】及即帝位内畏太后太妃外憚諸大臣未敢縱逸自加元服内外稍無以制數出遊行【數所角翻】始出宫猶整儀衛俄而弃車騎帥左右數人或出郊野或入市㕓【騎奇寄翻下同帥讀曰率】太妃每乘青犢車【青犢車青蓋犢車也晉制諸王青蓋車時有司奏皇太妃輿服一同晉孝武李太妃故事】隨相檢攝既而輕騎遠走一二十里太妃不復能追【復扶又翻下已復同】儀衛亦懼禍不敢追尋唯整部伍别往一處瞻望而已初太宗嘗以陳太妃賜嬖人李道兒已復迎還生帝【已既也既而復迎之還也嬖卑義翻又博計翻】故帝每微行自稱劉統【劉統自言統天下也猶苻堅稱苻詔桓玄稱桓詔】或稱李將軍常著小袴衫營署巷陌無不貫穿【著陟畧翻穿如字又樞絹翻】或夜宿客舍或晝卧道傍排突厮養【厮息移翻養餘亮翻韋昭曰析薪為厮炊烹為養又厮給養馬者】與之交易或遭慢辱悅而受之凡諸鄙事裁衣作帽過目則能未嘗吹箎【箎音池以竹爲之長八尺四寸圍三寸周禮賈疏云箎八孔】執管便韻【韻諧也和也】及京口既平驕恣尤甚無日不出夕去晨返晨出暮歸從者並執鋋矛【從才用翻鋋音蟬又以前翻小矛也】行人男女及犬馬牛驢逢無免者民間擾懼商販皆息門戶晝閉行人殆絶鍼椎鑿鋸不離左右【鍼與鉗同其淹翻離力智翻】小有忤意即加屠剖【忤五故翻】一日不殺則慘然不樂殿省憂惶食息不保阮佃夫與直閣將軍申伯宗等謀因帝出江乘射雉稱太后令喚隊仗還【樂音洛射而亦翻隊有隊主副仗有仗主副】閉城門遣人執帝廢之立安成王凖事覺甲戍帝收佃夫等殺之太后數訓戒帝【數所角翻】帝不悅會端午太后賜帝毛扇【毛扇蓋羽扇也 考異曰宋略作太妃賜今從宋書】帝嫌其不華令太醫煮藥欲鴆太后左右止之曰若行此事官便應作孝子豈復得出入狡獪【復扶又翻下無復誰復同狡古巧翻獪古外翻江南人謂小兒戲為狡獪】帝曰汝語大有理乃止六月甲戌有告散騎常侍杜幼文司徒左長史沈勃游擊將軍孫超之與阮佃夫同謀者帝登帥衛士自掩三家悉誅之【登登時也登時猶言即時也散悉亶翻騎奇寄翻帥讀曰率考異曰南史曰孝武二十八子太宗殺其十六餘皆帝殺之按宋書孝武諸子十人早卒二人為景和所殺餘皆太宗殺之無及蒼梧時者南史誤也】刳解臠割嬰孩不免沈勃時居喪在廬【廬倚廬也禮居喪者居倚廬寢苫枕塊孟康註曰倚廬倚墻至地而為之無楣柱孔頴達曰居倚廬者謂於中門之外東墻下倚木為廬又說曰凡非適子自未葬倚於隱者為廬鄭注曰不欲人屬目蓋於東南角】左右未至帝揮刀獨前勃知不免手搏帝耳唾罵之曰【唾湯卧翻】汝罪踰桀紂屠戮無日遂死是日大赦帝嘗直入領軍府時盛熱蕭道成晝卧裸袒【裸郎果翻】帝立道成於室内畫腹爲的【畫讀與畫同】自引滿將射之【射而亦翻下無復射箭射同】道成歛板曰老臣無罪左右王天恩曰領軍腹大是佳射堋【堋補登翻射堋今言射垜也】一箭便死後無復射不如以骲箭射之帝乃更以骲箭射正中其齊【更工衡翻骲蒲交翻又蒲剝翻集韻云骨鏃也余謂骨鏃亦能害人况以之射人腹乎蓋當時所謂骲箭者必非骨鏃中竹仲翻齊與臍同】投弓大笑曰此手何如帝忌道成威名嘗自磨鋋曰明日殺蕭道成陳太妃罵之曰蕭道成有功於國若害之誰復爲汝盡力邪【爲于僞翻】帝乃止道成憂懼密與袁粲褚淵謀廢立粲曰主上幼年微過易改【易以䜴翻】伊霍之事非季世所行縱使功成亦終無全地淵默然【淵於此時已心歸道成矣】領軍功曹丹陽紀僧眞言於道成曰今朝廷猖狂人不自保天下之望不在袁褚明公豈得坐受夷滅存亡之機仰希熟慮道成然之【道成時爲中領軍以僧真爲功曹希望也】或勸道成奔廣陵起兵道成世子賾時爲晉熙王長史行郢州事欲使賾將郢州兵東下會京口【將即亮翻】道成密遣所親劉僧副告其從兄行青冀二州刺史劉善明曰【從才用翻】人多見勸北固廣陵恐未爲長算今秋風行起鄉若能與垣東海微共動虜則我諸計可立亦告東海太守垣榮祖善明曰宋氏將亡愚智共知北虜若動反爲公患公神武高世唯當靜以待之因機奮發功業自定不可遠去根本自貽猖蹷榮祖亦曰領府去臺百步【領府謂領軍府也】公走人豈不知若單騎輕行【騎奇寄翻】廣陵人閉門不受公欲何之公今動足下牀恐即有叩臺門者【言將有告之者】公事去矣紀僧眞曰主上雖無道國家累世之基猶爲安固公百口北度必不得俱縱得廣陵城天子居深宫施號令目公爲逆何以避之此非萬全策也道成族弟鎭軍長史順之【順之蕭衍之父也 考異曰齊高帝紀姚思亷梁書武帝紀自相國何至皇考一十餘世皆有名及官位蓋史官附會今所不取】及次子驃騎從事中郎嶷【驃匹妙翻騎奇寄翻嶷魚力翻】皆以爲帝好單行道路於此立計易以成功外州起兵鮮有克捷【好呼到翻易以䜴翻鮮息淺翻鮮少也】徒先人受禍耳【先悉薦翻】道成乃止東中郎司馬行會稽郡事李安民欲奉江夏王躋起兵於東方【明帝泰始七年以皇子躋繼江夏王義恭時蓋爲東中郎將以安民爲司馬行郡事也會工外翻夏戶雅翻躋牋西翻】道成止之越騎校尉王敬則潛自結於道成夜著青衣扶匐道路【著則略翻扶讀曰蒲說文曰手行也匐蒲北翻】爲道成聽察帝之往來道成命敬則隂結帝左右楊玉夫楊萬年陳奉伯等二十五人於殿中詗伺機便【爲干僞翻詗喧正翻又古迥翻伺候也伺相吏翻】秋七月丁亥夜帝微行至領軍府門左右曰一府皆眠何不緣墻入帝曰我今夕欲於一處作適【適意作戲謂之作適】宜待明夕員外郎桓康等於道成門間聽聞之【此員外郎蓋員外散騎郎也】戊子帝乘露車與左右於臺岡賭跳【臺岡意即臺城之來岡也賭跳者賭跳躑以高者爲勝也跳音他弔翻考異曰南史作蠻岡今從宋書】仍往青園尼寺【尼女夷翻】晩至新安寺【孝武寵姬殷貴妃死爲之立寺貴妃子子鸞封新安王故以新安爲寺名】偷狗就曇度道人煮之【曇徒含翻】飲酒醉還仁夀殿寢楊玉夫常得帝意至是忽憎之見輒切齒曰明日當殺小子取肝肺是夜令玉夫伺織女渡河【續齊諧記曰桂陽成武丁有仙道謂其弟曰七月七日織女當度河弟問曰織女何事度河答曰織女暫詣牽牛人至今云織女嫁牽牛也崔寔四時民令曰或云見天漢中弈弈有正白氣光耀五色以此爲徵應】曰見當報我不見將殺汝時帝出入無常省内諸閤夜皆不閉廂下畏相逢値無敢出者宿衛並逃避内外莫相禁攝是夕王敬則出外玉夫伺帝熟寢與楊萬年取帝防身刀刎之【御左右防身刀郎所謂千牛刀也刎扶粉翻】敕廂下奏伎【伎渠綺翻】陳奉伯袖其首依常行法稱敕開承明門出以首與敬則敬則馳詣領軍府叩門大呼【呼火故翻】蕭道成慮蒼梧王誑之不敢開門【誑居况翻】敬則於墻上投其首道成洗視乃戎服乘馬而出敬則桓康等皆從入宫至承明門詐爲行還敬則恐内人覘見以刀環塞窐孔【覘丑廉翻又丑艶翻塞悉則翻窐孔即古之所謂圭竇也窐古携翻又音携】呼門甚急門開而入佗夕蒼梧王每開門門者震懾不敢仰視至是弗之疑【懾之涉翻 考異曰齊高帝紀云衛尉丞顔靈寶窺見太祖乘馬在外竊謂親人曰今若不開内領軍入天下會是亂耳按靈寶若語所親則須有知者豈得宿衛晏然不動今從宋後廢帝紀】道成入殿殿中驚怖【怖普布翻】旣而聞蒼梧王死咸稱萬歲己丑旦道成戎服出殿庭槐樹下以太后令召袁粲褚淵劉秉入會議道成謂秉曰此使君家事何以斷之【使疏吏翻斷丁亂翻】秉未答道成須髯盡張目光如電秉曰尚書衆事可以見付軍旅處分【須與鬚同處昌呂翻分扶問翻】一委領軍道成次讓袁粲粲亦不敢當王敬則拔白刃在牀側跳躍曰天下事皆應關蕭公敢有開一言者血染敬則刀仍手取白紗帽加道成首【五代志帽自天子下及士人通冠之以白紗者名高頂帽皇太子在上省則烏紗在永福則白紗蓋貴白紗也杜佑曰宋制黑帽綴紫褾褾以繒為之長四寸廣一寸後制高屋白紗帽】令即位曰今日誰敢復動事須及熱道成正色呵之曰卿都自不解【復扶又翻呵虎何翻解戶買翻曉也】粲欲有言敬則叱之乃止褚淵曰非蕭公無以了此手取事授道成【褚淵手取其事以授道成自此天下之事一歸之矣】道成曰相與不肯我安得辭乃下議備法駕詣東城迎立安成王【東城即東府城也】於是長刀遮粲秉等各失色而去【觀史所書會議之際道成目光如電須髯盡張王敬則拔白刃跳躍繼又以長刀遮粲秉等事勢可知矣粲秉於此時聲其弑君之罪以身死之猶不愧於仇牧何待至石頭邪】秉出於路逢從弟韞【從才用翻】韞開車迎問曰今日之事當歸兄邪秉曰吾等已讓領軍矣韞拊膺曰兄肉中詎有血邪今年族矣【宋事去矣自中人以下皆知之】是日以太后令數蒼梧王罪惡【數所具翻】曰吾密令蕭領軍濳運明略安成王凖宜臨萬國追封昱爲蒼梧王儀衛至東府門安成王令門者勿開以待袁司徒粲至王乃入居朝堂【史言袁粲爲一時倚重朝直遥翻】壬辰王即皇帝位時年十一改元大赦【改元昇明】葬蒼梧王於郊壇西【南郊壇在臺城南已地世祖大明三年移南郊壇於牛頭山以正陽位】魏京兆康王子推卒 甲午蕭道成出鎭東府丙申<br />
<br />
  以道成爲司空錄尚書事驃騎大將軍【驃匹妙翻騎奇寄翻】袁粲遷中書監褚淵加開府儀同三司劉秉遷尚書令加中領軍以晉熙王爕爲揚州刺史劉秉始謂尚書萬機本以宗室居之則天下無變既而蕭道成兼總軍國布置心膂與奪自專褚淵素相憑附秉與袁粲閣手仰成矣【閣手者高拱充位而無所爲兩手若有所閣也仰牛向翻】辛丑以尚書右僕射王僧䖍爲僕射丙午以武陵王贊爲郢州刺史蕭道成改領南徐州刺史 八月壬子魏大赦 癸亥詔袁粲鎭石頭粲性冲靜每有朝命常固辭【朝直遥翻】逼切不得已乃就職至是知蕭道成有不臣之志隂欲圖之即時順命【爲袁粲以石頭死節張本】 初太宗使陳昭華母養順帝戊辰尊昭華爲皇太妃【魏明帝置昭華晉武帝制淑妃淑媛淑儀修華修容修儀婕妤充華容華爲九嬪而昭華之號不復見至宋孝武制以昭儀昭容昭華代修華修儀修容也】 丙子魏詔曰工商皂隸各有厥分【分扶問翻】而有司縱濫或染流俗自今戶内有工役者唯止本部丞【流俗北史作清流此蓋以當時授官不分流品故詔凡工役之戶官止本部丞】若有勲勞者不從此制 蕭道成固讓司空庚辰以為驃騎大將軍開府儀同三司 九月乙酉魏更定律令【更工衡翻】 戊申封楊玉夫等二十五人爲侯伯子男 冬十月氐帥楊文度遣其弟文弘襲魏仇池䧟之【帥所類翻 考異曰魏書本紀作揚黽氐傳作鼠皆避顯祖諱也】 初魏徐州刺史李訢事顯祖爲倉部尚書【晉武帝始置倉部郎屬度支尚書倉部尚書後魏所置也即前太倉尚書訢許斤翻】信用盧奴令范訢弟左將軍瑛諫曰 【考異曰魏典作摽瑛作璞今從魏書余按與摽同卑遥翻瑛音英】范能降人以色假人以財輕德義而重勢利聽其言也甘察其行也賊不早絶之後悔無及訢不從腹心之事皆以語【行下孟翻語牛倨翻】尚書趙黑與訢皆有寵於顯祖對掌選部【選須絹翻】訢以其私用人爲方州【古者八州八伯謂之方伯後世遂以州刺史爲方州】黑對顯祖發之由是有隙頃之訢發黑前爲監藏【監古銜翻藏徂浪翻】盗用官物黑坐黜爲門士黑恨之寢食爲之衰少踰年復入爲侍中尚書左僕射領選【爲于僞翻復扶又翻下我復黑復同】及顯祖殂黑白馮太后稱訢專恣出爲徐州范知太后怨訢【以其告李敷也事見一百三十二卷明帝泰始六年訢爲太倉尚書亦在是年也】乃告訢謀外叛太后徵訢至平城問狀訢對無之太后引使證之訢謂曰汝今誣我我復何言然汝受我恩如此之厚乃忍爲爾乎曰受公恩何如公受李敷恩公忍爲之於敷何爲不忍於公訢慨然嘆曰吾不用瑛言悔之何及趙黑復於中構成其罪丙子誅訢及其子令和令度黑然後寢食如故 十一月癸未魏征西將軍皮歡喜等三將軍率衆四萬擊楊文弘 丁亥魏懷州民伊祁苟自稱堯後【堯伊祁氏故云然】聚衆於重山作亂【重山即河内重門之山在共縣北重直龍翻】洛州刺史馮熙討滅之馮太后欲盡誅闔城之民雍州刺史張白澤諫曰凶渠逆黨盡已梟夷【魏雍州統京兆扶風馮翊咸陽北地平秦武都等郡凶渠謂渠魁也孔安國曰渠大也雍於用翻梟堅堯翻】城中豈無忠良仁信之士柰何不問白黑一切誅之乃止十二月魏皮歡喜軍至建安【水經注楊定自隴右徙治歷城去仇池百二十里歷城後改為建安城 考異曰是年魏置閏在十一月宋之十二月也】楊文弘棄城走 初沈攸之與蕭道成於大明景和之間同直殿省深相親善道成女爲攸之子中書侍郎文和婦攸之在荆州直閤將軍高道慶家在華容【華容縣自漢以來屬南郡按九域志今江陵府石首縣建寧鎭即其地宋白曰江陵府監利縣本漢華容縣地】假還過江陵【假居訝翻休假也過工禾翻】與攸之爭戲槊【槊色角翻】馳還建康言攸之反狀已成請以三千人襲之執政皆以爲不可道成仍保證其不然楊運長等惡攸之【惡烏路翻】密與道慶謀遣刺客殺攸之不克會蒼梧王遇弑主簿宗儼之功曹臧寅勸攸之因此起兵攸之以其長子元琰在建康爲司徒左長史故未發【長知兩翻】寅凝之之子也【臧凝之見一百二十七卷宋文帝元嘉三十年】時楊運長等已不在内【已出爲外官不在省内也】蕭道成遣元琰以蒼梧王刳斮之具示攸之攸之以道成名位素出已下一旦專制朝權心不平【斮則略翻朝直遥翻】謂元琰曰吾寜為王凌死不為賈充生【王凌賈充事並見魏紀】然亦未暇舉兵乃上表稱慶因留元琰雍州刺史張敬兒素與攸之司馬劉攘兵善【雍於用翻】疑攸之將起事密以問攘兵攘兵無所言寄敬兒馬鐙一隻【鐙都鄧翻】敬兒乃為之備攸之有素書十數行【行戶剛翻】常韜在襠角【博雅曰裲襠謂之袹腹裲里養翻襠都郎翻】云是明帝與已約誓攸之將舉兵其妾崔氏諫曰官年已老那不爲百口計【宋齊之間義從私屬以至婢僕率呼其主為官】攸之指裲襠角示之且稱太后使至賜攸之【使疏吏翻下同】燭割之得太后手令云社稷之事一以委公於是勒兵移檄遣使邀張敬兒及豫州刺史劉懷珍梁州刺史梓潼范栢年司州刺史姚道和湘州行事庾佩玉【南陽王翽未至故庾佩玉行府州事】巴陵内史王文和同舉兵敬兒懷珍文和並斬其使馳表以聞文和尋棄州奔夏口【巴陵非州也州當作郡夏戶雅翻】栢年道和佩玉皆懷兩端道和後秦高祖之孫也【後秦主姚興廟號高祖】辛酉攸之遣輔國將軍孫同等相繼東下攸之遺道成書【遺于季翻】以爲少帝昏狂【少詩照翻】宜與諸公密議共白太后下令廢之柰何交結左右親行弑逆乃至不殯流蟲在戶凡在臣下誰不惋駭又移易朝舊【朝舊謂朝廷舊臣也惋烏貫翻朝直遥翻下同】布置親黨宫閤管籥悉關家人吾不知子孟孔明遺訓固如此乎【霍光字子孟諸葛亮字孔明】足下既有賊宋之心吾寜敢捐包胥之節邪【申包胥乞秦師以存楚事見左傳】朝廷聞之忷懼【忷許拱翻】丁卯道成入守朝堂命侍中蕭嶷代鎭東府【嶷魚力翻】撫軍行參軍蕭映鎭京口映嶷之弟也戊辰内外纂嚴己巳以郢州刺史武陵王贊爲荆州刺史庚午以右衛將軍黄回爲郢州刺史督前鋒諸軍以討攸之初道成以世子賾爲晉熙王爕長史行郢州事修治器械以備攸之【道成平桂陽之難進爵縣公以賾爲世子賾士革翻治直之翻】及徵爕爲揚州以賾爲左衛將軍與爕俱下劉懷珍言於道成曰夏口衝要宜得其人道成與賾書曰汝既入朝當須文武兼資與汝意合者委以後事賾乃薦爕司馬柳世隆自代道成以世隆爲武陵王贊長史行郢州事賾將行謂世隆曰攸之一旦爲變焚夏口舟艦【夏戶雅翻艦戶黯翻】沿流而東不可制也若得攸之留攻郢城必未能猝拔君爲其内我爲其外破之必矣及攸之起兵賾行至尋陽未得朝廷處分【處昌呂翻分扶問翻】衆欲倍道趨建康【趨七喻翻】賾曰尋陽地居中流密邇畿甸若留屯湓口内藩朝廷外援夏首【夏首即夏口】保據形勝控制西南今日會此天所置也或以爲湓口城小難固左中郎將周山圖曰今據中流爲四方勢援不可以小事難之苟衆心齊一江山皆城隍也庚午賾奉爕鎭湓口賾悉以事委山圖山圖斷取行旅船板以造樓櫓立水柵【斷丁管翻立柵於水中曰水柵】旬日皆辦道成聞之喜曰賾眞我子也以賾爲西討都督賾啓山圖爲軍副時江州刺史邵陵王友鎭尋陽賾以爲尋陽城不足固表移友同鎭湓口【尋陽時治柴桑今江州德化西南九十里有柴桑山湓口在德化縣西一里江州治德化蓋近湓口古城處】留江州别駕豫章胡諧之守尋陽湘州刺史王藴遭母喪罷歸至巴陵與沈攸之深相結【巴陵距江陵四百餘里蓋使命往來深相結也】時攸之未舉兵藴過郢州欲因蕭賾出弔作難據郢城賾知之不出還至東府又欲因蕭道成出弔作難道成又不出【作難者欲殺之也難乃旦翻】藴乃與袁粲劉秉密謀誅道成將帥黃回任候伯孫曇瓘王宜興卜伯興等皆與通謀【將即亮翻帥所類翻任音壬曇徒含翻】伯興天與之子也【卜天與死於元凶劭之難】道成初聞攸之事起自往詣粲粲辭不見通直郎袁逹謂粲不宜示異同【通直郎通直散騎侍郎也晉武帝置員外散騎侍郎元帝秦興二年使二人與散騎侍郎同員直故謂之通直散騎侍郎也】粲曰彼若以主幼時艱與桂陽時不異【謂桂陽王休範反時也】劫我入臺我何辭以拒之一朝同止欲異得乎道成乃召禇淵與之連席每事必引淵共之【果如袁粲所料】時劉韞爲領軍將軍入直門下省 【考異曰南齊書韞作韜今從宋書南史】卜伯興爲直閤黃回等諸將皆出屯新亭【將即亮翻】初禇淵爲衛將軍遭母憂去職朝廷敦迫不起粲素有重名自往譬說【譬說猶說喻也說輸芮翻】淵乃從之及粲為尚書令遭母憂淵譬說懇至粲遂不起淵由是恨之【淵之恨粲以其奪己志而使之失爲子之道也而殺粲以傾宋又失爲臣之節曰忠與孝二者淵胥失焉】及沈攸之事起道成與淵議之淵曰西夏釁難事必無成【夏戶雅翻釁許覲翻難乃旦翻】公當先備其内耳【謂備袁粲等也】粲謀既定將以告淵衆謂淵與道成素善不可告粲曰淵與彼雖善豈容大作同異今若不告事定便應除之乃以謀告淵【袁粲猶以故意待禇淵也】淵即以告道成道成亦先聞其謀遣軍主蘇烈薛淵太原王天生將兵助粲守石頭薛淵固辭道成彊之【將即亮翻彊其兩翻】淵不得已涕泣拜辭道成曰卿近在石頭日夕去來何悲如是且又何辭淵曰不審公能保袁公共爲一家否今淵往與之同則負公不同則立受禍何得不悲道成曰所以遣卿正爲能盡臨事之宜使我無西顧之憂耳【石頭在臺城西故云然爲于僞翻】但當努力無所多言淵安都之從子也【從才用翻】道成又以驍騎將軍王敬則爲直閣【驍堅堯翻騎奇寄翻】與伯興共總禁兵粲謀矯太后令使韞伯興帥宿衛兵攻道成於朝堂【帥讀曰率下同朝直遥翻】回等帥所領爲應劉秉任候伯等並赴石頭本期壬申夜發秉恇擾不知所爲晡後即束裝臨去啜羮寫胷上手振不自禁【恇去王翻振當作震戰也動也禁音居吟翻勝也】未暗載婦女盡室奔石頭部曲數百赫奕滿道既至見粲粲驚曰何事遽來今敗矣【秉奔石頭則事大露故云必敗】秉曰得見公萬死何恨孫曇瓘聞之亦奔石頭丹陽丞王遜等走告道成事乃大露遜僧綽之子也【王僧綽柄用於元嘉之季】道成密使人告王敬則時閤已閉敬則欲開閤出卜伯興嚴兵爲備敬則乃鋸所止屋壁得出至中書省收韞韞已成嚴【嚴裝也成嚴謂裝束已成俟期而發也】列燭自照見敬則猝至驚起迎之曰兄何能夜顧敬則呵之曰小子那敢作賊韞抱敬則敬則拳毆其頰仆地而殺之【呵虎何翻毆烏口翻】又殺伯興【卜伯興父子俱死於劉氏之難】蘇烈等據倉城拒粲【倉城石頭倉城也】王藴聞秉已走歎曰事不成矣狼狽帥部曲數百向石頭【帥讀曰率下同 考異曰宋書云齊王使藴募人已得數百宋畧云是夕徵其私衆倏忽之間被甲數百莫知所從出按道成素已疑藴必不使之募兵宋畧近是也】本期開南門時暗夜薛淵據門射之【射而亦翻】藴謂粲已敗即散走道成遣軍主會稽戴僧靜帥數百人向石頭助烈等【會工外翻】自倉門得入與之并力攻粲孫曇瓘驍勇善戰臺軍死者百餘人王天生殊死戰故得相持自亥至丑戴僧靜分兵攻府西門焚之粲與秉在城東門見火起欲還赴府秉與二子俁陔踰城走【俁牛矩翻陔柯開翻】粲下城列燭自照謂其子最曰本知一木不能止大厦之崩但以名義至此耳僧靜乘暗踰城獨進最覺有異人以身衛粲僧靜直前斫之粲謂最曰我不失忠臣汝不失孝子遂父子俱死 【考異曰南史云僧靜奮刀直前欲斬之子最叫抱父乞先死兵士人人莫不隕涕粲曰我不失忠臣汝不失孝子仍求筆作啓云臣義奉大宋策名兩畢今便歸䰟墳隴永就山邱僧靜乃并斬之按時僧靜掩粲不備挺身直往安肯容粲作啓從容如此宋書皆無此等事今不取】百姓哀之謡曰可憐石頭城寜爲袁粲死不作禇淵生劉秉父子走至額檐湖【蕭子顯齊書作雒檐湖檐余廉翻】追執斬之任候伯等並乘船赴石頭既至臺軍已集不得入乃馳還黃回嚴兵期詰旦帥所領從御道直向臺門攻道成【詰去吉翻帥讀曰率】聞事泄不敢發道成撫之如舊王藴孫曇瓘皆逃竄先捕得藴斬之其餘粲黨皆無所問粲典籖莫嗣祖爲粲秉宣通密謀道成召詰之曰袁粲謀反何不啓聞嗣祖曰小人無識但知報恩何敢泄其大事今袁公已死義不求生藴嬖人張承伯藏匿藴道成並赦而用之【史言蕭道成能弃怨錄才嬖卑義翻又博計翻】粲簡淡平素而無經世之才好飲酒喜吟諷【好呼到翻喜許記翻】身居劇任不肯當事主事每往諮决【主事尚書省主事也尚書諸曹各有主事】或高詠對之閒居高卧門無雜賓物情不接故及於敗<br />
<br />
  裴子野論曰袁景倩民望國華【袁粲字景倩】受付託之重智不足以除姦權不足以處變【處昌呂翻】蕭條散落危而不扶及九鼎既傾三才將換區區斗城之裏【斗城言城如斗大也】出萬死而不辭蓋蹈匹夫之節而無棟梁之具矣【裴子野之論有春秋責備賢者之意故通鑑取之】<br />
<br />
  甲戌大赦 乙亥以尚書僕射王僧䖍爲左僕射新除中書令王延之爲右僕射度支尚書張岱爲吏部尚書【度徒洛翻】吏部尚書王奐爲丹陽尹延之裕之孫也劉秉弟遐爲吳郡太守司徒右長史張瓌永之子也【張永歷事文武明三帝瓌古回翻】遭父喪在吳家素豪盛蕭道成使瓌伺間取遐【間古莧翻】會遐召瓌詣府瓌帥部曲十餘人直入齋中執遐斬之【帥讀曰率】郡中莫敢動道成聞之以告瓌從父領軍冲冲曰瓌以百口一擲出手得盧矣【從才用翻樗蒲得盧者勝】道成即以瓌爲吳郡太守道成移屯閱武堂猶以重兵付黃回使西上而配以腹心【配以腹心所以防回也上時掌翻】回素與王宜興不恊恐宜興反告其謀閏月辛巳因事收宜興斬之諸將皆言回握彊兵必反【將即亮翻下同】寜朔將軍桓康請獨往刺之【刺七亦翻】道成曰卿等何疑彼無能爲也【史言道成才識雄於一時】沈攸之遣中兵參軍孫同等五將以三萬人爲前驅司馬劉攘兵等五將以二萬人次之又遣中兵參軍王靈秀等四將分兵出夏口據魯山癸巳攸之至夏口 【考異曰沈約齊紀十一月攸之遂謀爲亂張敬兒遣使詣攸之慶冬攸之呼使人於密室謂之曰奉皇太后令得袁司徒劉丹陽諸人書呼我速下可令雍州知此意答敬兒書曰信口一二而封雞毛桃耳數物置函中敬兒賀冬使即乘驛白公十二日壬辰攸之遣孫同等先發十七日丁酉張敬兒使至十八日戊戌公率衆入鎭朝堂閏月十四日癸巳攸之至夏口按是歲宋歷閏十二月庚辰朔魏歷閏十一月庚戌朔然則冬至必在十一月晦攸之對敬兒賀冬使者猶隱祕豈可十二日已發兵東下乎又攸之若十二日已舉兵於江陵豈可六十餘日始至夏口又宋順帝紀十二月攸之反丁卯齊王入守朝堂丁卯乃十二月十八日也閏月癸巳攸之圍郢城攸之傳十一月反十二月十二日遣孫同等東下閏月十四日至夏口宋畧十二月沈攸之作亂丁卯蕭道成入屯朝堂閏月癸巳攸之師及郢州南齊高帝紀十二月攸之舉兵乙卯太祖入居朝堂諸書大抵畧相符合惟齊紀不同蓋齊紀之誤今不取】自恃兵彊有驕色以郢城弱小不足攻云欲問訊安西暫泊黃金浦【時武陵王贊蓋以安西將軍鎭郢黃金浦在鸚鵡洲上相傳以爲吳將黃蓋屯兵於此得名】遣人告柳世隆曰被太后令當暫還都【被皮義翻】卿既相與奉國想得此意世隆曰東下之師久承聲問郢城小鎭自守而已宗儼之勸攸之攻郢城臧寅以爲郢城兵雖少而地險【少詩沼翻】攻守勢異非旬日可拔若不時舉【孟子曰以萬乘之國伐萬乘之國五旬而舉之戰國策白起一戰而舉鄢郢舉拔也】挫鋭損威今順流長驅計日可捷既傾根本郢城豈能自固攸之從其計欲留偏師守郢城自將大衆東下【將即亮翻】乙未將發柳世隆遣人於西渚挑戰【鸚鵡洲之西渚挑徒了翻】前軍中兵參軍焦度於城樓上肆言罵攸之且穢辱之攸之怒改計攻城令諸軍登岸燒郭邑築長圍晝夜攻戰世隆隨宜拒應攸之不能克道成命吳興太守沈文季督吳錢唐軍事【五代史志曰餘杭郡錢唐縣舊置錢唐郡蓋此時置也】文季收攸之弟新安太守登之誅其宗族【沈攸之殺沈慶之文季因事以報父仇】 乙未以後軍將軍楊運長爲宣城太守於是太宗嬖臣無在禁省者矣【嬖卑義翻又博計翻】<br />
<br />
  沈約論曰夫人君南面九重奥絶【重直龍翻】陪奉朝夕義隔卿士堦闥之任宜有司存既而恩以狎生信由恩固無可憚之姿有易親之色【易以䜴翻】孝建泰始主威獨運而刑政糾雜理難遍通耳目所寄事歸近習及覘歡愠候慘舒動中主情舉無謬旨【覘丑亷翻又丑艶翻中竹仲翻】人主謂其身卑位薄以爲權不得重曾不知鼠憑社貴狐藉虎威【漢中山靖王勝曰社鼠不熏所託者然也楚江乙曰虎求百獸而食之得狐狐曰子無食我天帝使我長百獸子以我爲不信吾爲子先行子隨我後百獸見我而敢不走乎虎以爲然遂與之行獸見之皆走虎不知百獸畏已而皆走也以爲畏狐也】外無逼主之嫌内有專用之效勢傾天下未之或悟及太宗晩運慮經盛衰權倖之徒慴憚宗戚【慴之涉翻】欲使幼主孤立永竊國權構造同異興樹禍隙帝弟宗王相繼屠勦【謂殺建安諸王也勦子小翻】寶祚夙傾實由於此矣<br />
<br />
  辛丑尚書左丞濟陽江謐建議假蕭道成黃鉞【濟子禮翻】從之 加北秦州刺史武都王楊文度都督北秦雍二州諸軍事以龍驤將軍楊文弘爲畧陽太守【雍於用翻驤思將翻】壬寅魏皮歡喜拔葭蘆斬文度魏以楊難當族弟廣香爲隂平公葭蘆戍主仍詔歡喜築駱谷城文弘奉表謝罪於魏遣子苟奴入侍魏以文弘爲南秦州刺史武都王乙巳蕭道成出頓新亭謂驃騎參軍江淹曰【道成爲驃騎大】<br />
<br />
  【將軍以淹爲參軍驃匹妙翻騎奇寄翻下同】天下紛紛君謂何如淹曰成敗在德不在衆寡公雄武有奇畧一勝也寛容而仁恕二勝也賢能畢力三勝也民望所歸四勝也奉天子以伐叛逆五勝也彼志鋭而器小一敗也有威而無恩二敗也士卒解體三敗也縉紳不懷四敗也懸兵數千里而無同惡相濟五敗也雖豺狼十萬終爲我獲道成笑曰君談過矣南徐州行事劉善明言於道成曰攸之收衆聚騎造舟治械包藏禍心於今十年【明帝泰始五年沈攸之刺郢州已懷異志至是適十年治直之翻】性既險躁才非持重【躁則到翻】而起逆累旬遲迴不進一則暗於兵機二則人情離怨三則有掣肘之患【掣昌列翻】四則天奪其魄本慮其剽勇輕速【剽匹妙翻】掩襲未備决於一戰今六師齊奮諸侯同舉此籠中之鳥耳蕭賾問攸之於周山圖山圖曰攸之相與鄰鄉【攸之吳興人而山圖義興人故曰鄰鄉】數共征伐【數所角翻】頗悉其人性度險刻士心不附今頓兵堅城之下適所以爲離散之漸耳<br />
<br />
  二年春正月己酉朔百官戎服入朝【朝直遥翻】沈攸之盡鋭攻郢城柳世隆乘閒屢破之【閒古莧翻】蕭賾遣軍主桓敬等八軍據西塞【西塞山在今武昌縣東百三十里界於兩山之間土俗編曰吳楚舊境分界於此】爲世隆聲援攸之獲郢府法曹南鄉范雲使送書入城餉武陵王贊犢一羫【羫苦江翻】柳世隆魚三十尾皆去其首【去羌呂翻】城中欲殺之雲曰老母弱弟懸命沈氏若違其命禍必及親今日就戮甘心如薺【詩谷風誰謂荼苦其甘如薺此謂甘心就死如茹薺也薺齊禮翻】乃赦之攸之遣其將皇甫仲賢向武昌中兵參軍公孫方平向西陽武昌太守臧渙降於攸之西陽太守王毓奔湓城【將即亮翻降戶江翻湓蒲奔翻】方平據西陽豫州刺史劉懷珍遣建寜太守張謨等將萬人擊之【豫州有建寜左郡孝武大明八年省郡爲建寜左縣屬西陽郡尋復爲郡蓋皆蠻左所居地也五代志永安郡麻城縣有梁北西陽縣又有建寜郡將即亮翻】辛酉方平敗走平西將軍黃回等軍至西陽泝流而進【泝蘇故翻】攸之素失人情但刼以威力初發江陵已有逃者及攻郢城三十餘日不拔逃者稍多攸之日夕乘馬歷營撫慰而去者不息攸之大怒召諸軍主曰我被太后令【被皮義翻】建義下都大事若克白紗㡌共著耳【著則畧翻】如其不振朝廷自誅我百口不關餘人比軍人叛散【比毗至翻】皆鄉等不以爲意我亦不能問叛身自今軍中有叛者軍主任其罪【任音壬】於是一人叛遣人追之亦去不返莫敢發覺咸有異計劉攘兵射書入城請降【射而亦翻】柳世隆開門納之丁卯夜攘兵燒營而去軍中見火起爭棄甲走將帥不能禁【將即亮翻帥所類翻】攸之聞之怒銜須咀之【自咀其須怒之甚也須與鬚同咀音在呂翻嚼也】收攘兵兄子天賜女壻張平虜斬之向旦攸之帥衆過江至魯山【大别山一名魯山在今漢陽軍沔陽縣東一里江水逕其南漢水從西北來注之帥讀曰率】軍遂大散 【考異曰宋略云甲辰攸之衆潰乙巳華容民斬其首按是月乙酉朔無甲辰乙巳】諸將皆走臧寅曰幸其成而弃其敗吾不忍爲也乃投水死攸之猶有數十騎自隨宣令軍中曰荆州城中大有錢可相與還取以爲資糧郢城未有追軍而散軍畏蠻抄【此蠻即緣沔而居者騎奇寄翻抄楚交翻】更相聚結可二萬人隨攸之還江陵張敬兒既斬攸之使者即勒兵偵攸之下遂襲江陵【偵丑正翻候也】攸之使子元琰與兼長史江乂别駕傳宣共守江陵城敬兒至沙橋觀望未進城中夜聞鶴唳謂爲軍來乂宣開門出走吏民崩潰元琰奔寵洲【寵洲近樂鄉楊正衡晉書音義曰寵力董翻】爲人所殺敬兒至江陵 【考異曰宋畧云辛未敬兒克江陵按己巳攸之以敬兒據城走死不容敬兒至辛未乃入城也】誅攸之二子四孫攸之將至江陵百餘里聞城已爲敬兒所據士卒隨之者皆散攸之無所歸與其子文和走至華容界皆縊於櫟林【櫟郎狄翻木名柞屬】己巳村民斬首送江陵敬兒擎之以楯覆以青繖【楯食尹翻覆敷又翻繖蘇旰翻蓋也】徇諸市郭乃送建康敬兒誅攸之親黨收其財物數十萬皆以入私初倉曹參軍金城邊榮爲府録事所辱攸之爲榮鞭殺録事【爲于僞翻】及敬兒將至榮爲留府司馬或說之使詣敬兒降【說輸芮翻】榮曰受沈公厚恩共如此大事一朝緩急便易本心吾不能也城潰軍士執以見敬兒敬兒曰邊公何不早來榮曰沈公見留守城不忍委去本不祈生【祈求也告也】何須見問敬兒曰死何難得命斬之榮歡笑而去榮客太山程邕之抱榮曰與邊公周遊不忍見邊公死乞先見殺兵人不得行戮以白敬兒敬兒曰求死甚易何爲不許先殺邕之然後及榮軍人莫不垂泣【易以䜴翻士爲知已死邊榮程邕之俱有焉】孫同宗儼之等皆伏誅【宗儼之與臧寅勸攸之舉兵孫同爲軍鋒】丙子解嚴以侍中柳世隆爲尚書右僕射蕭道成還鎮東府丁丑以右衛將軍蕭賾爲江州刺史侍中蕭嶷爲中領軍【賾士革翻嶷魚力翻】二月庚辰以尚書左僕射王僧䖍爲尚書令右僕射王延之爲左僕射癸未加蕭道成太尉都督南徐等十六州諸軍事【蕭子顯齊書都督南徐南兖徐兖青冀司豫荆雍襄郢梁益廣越十六州】以衛將軍禇淵爲中書監司空道成表送黃鉞【上流已定故表還黃鉞】吏部郎王儉僧綽之子也神彩淵曠好學博聞少有宰相之志時論亦推許之道成以儉爲太尉右長史【太尉府時置左右長史好呼到翻少詩照翻相息亮翻】待遇隆密事無大小專委之 丁亥魏主如代湯泉【此魏代都之湯泉也言代湯泉者以别下洛縣橋山之湯泉魏土地記曰代城北九十里有桑乾城城西渡桑乾水去城十里有温湯療疾有驗又下洛縣西南四十里有橋山下有温泉】癸卯還 宕昌王彌機初立三月丙子魏遣使拜彌機征南大將軍梁益二州牧河南公宕昌王【宕徒浪翻使疏吏翻】 黃回不樂在郢州【樂音洛】固求南兖遂帥部曲輒還【帥讀曰率】辛卯改都督南兖等五州諸軍事南兖州刺史【黃回刃在其頸乃輒東還此送死也】初王藴去湘州湘州刺史南陽王翽未之鎭【翽明帝子也音呼會翻】長沙内史庾佩玉行府事翽先遣中兵參軍韓幼宗將兵戍湘州與佩玉不相能及沈攸之反兩人互相疑佩玉襲殺幼宗黃回至郢州遣輔國將軍任候伯行湘州事候伯輒殺佩玉冀以自免【任候伯黃回皆與袁劉同謀任音壬】湘州刺史呂安國之鎭蕭道成使安國誅候伯 夏四月甲申魏主如崞山丁亥還【還從宣翻又如字】 蕭道成以黃回終爲禍亂回有部曲數千人欲遣收恐爲亂辛卯召回入東府至停外齋使桓康將數十人數回罪而殺之并其子竟陵相僧念【道成翦除異已至是盡矣數所具翻相息亮翻】 甲午以淮南宣城二郡太守蕭映行南兖州事仍以其弟晃代之【淮南宣城逼近京邑故道成不以授他人】 五月魏禁皇族貴戚及士民之家不顧氏族下與非類昏偶犯者以違制論 魏主與太后臨虎圈【圈求遠翻】有虎逸登閣道幾至御座侍衛皆驚靡【幾居希翻靡披靡也】吏部尚書王叡執戟禦之太后稱以爲忠親任愈重【史言馮后假公義以成其私】 六月丁酉以輔國將軍楊文弘爲北秦州刺史武都王 庚子魏皇叔若卒 蕭道成以大明以來公私奢侈秋八月奏罷御府【御府令自漢以來有之漢屬少府晉屬光禄勲據宋紀世祖大明四年改細作署令爲左右御府令】省二尚方彫飾器玩辛卯又奏禁民間華僞雜物凡十七條【按蕭子顯齊書表禁不得以金銀爲箔馬乘具不得金銀度不得織成繡裙道路不得著錦履不得用紅色爲幡蓋衣服不得剪綵帛爲雜花不得以綾作雜服飾不得打鹿行錦及局脚檉栢牀牙箱籠雜物綵帛作屏障錦緣薦席不得私作器仗不得以七實飾樂器又諸雜飾物不得以金銀爲花獸不得輒鑄金銅爲像皆頒墨敕凡十七條】乙未以蕭賾爲領軍將軍蕭嶷爲江州刺史 九月乙巳朔日有食之 蕭道成欲引時賢參贊大業夜召驃騎長史謝朏屏人與語久之朏無言唯二小兒捉燭道成慮朏難之仍取燭遣兒朏又無言道成乃呼左右朏莊之子也【朏敷尾翻屏必郢翻捉執也持也謝莊見一百三十卷明帝泰始元年道成爲驃騎大將軍長史亦其府官也】太尉右長史王儉知其指它日請間言於道成曰功高不賞古今非一以公今日位地欲終北面可乎道成正色裁之而神采内和儉因曰儉蒙公殊盻所以吐所難吐何賜拒之深宋氏失德非公豈復寜濟但人情澆薄不能持久公若小復推遷則人望去矣【復扶又翻】豈唯大業永淪七尺亦不可得保【七尺謂七尺之軀也】道成曰卿言不無理儉曰公今名位故是經常宰相宜禮絶羣后微示變革當先令禇公知之儉請銜命道成曰我當自往經少日道成自造禇淵欵言移晷【少詩沼翻造七到翻晷居洧翻日影也】乃謂曰我夢應得官淵曰今授始爾【謂方加太尉都督也】恐一二年間未容便移且吉夢未必應在旦夕道成還以告儉儉曰禇是未逹理耳儉乃唱議加道成太傳假黃鉞使中書舍人虞整作詔道成所親任遐曰【任音壬】此大事應報禇公道成曰禇公不從柰何遐曰彦回惜身保妻子非有奇才異節遐能制之淵果無違異【禇淵字彦回史言禇淵之爲人人皆得而侮薄之】丙午詔進道成假黃鉞大都督中外諸軍事太傳領揚州牧劒履上殿入朝不趨贊拜不名使持節太尉驃騎大將軍録尚書南徐州刺史如故【使疏吏翻】道成固辭殊禮【劒履上殿入朝不趨贊拜不名皆殊禮也】 以揚州刺史晉熙王爕爲司徒 戊申太傅道成以蕭映爲南兖州刺史冬十月丁丑以蕭晃爲豫州刺史 己卯獲孫曇瓘殺之【石頭之禍曇瓘逃去曇徒含翻】 魏員外散騎常侍鄭羲來聘 壬寅立皇后謝氏后莊之孫也 十一月癸亥臨澧侯劉晃坐謀反與其黨皆伏誅晃秉之從子也【沈約志臨澧縣晉武帝太康四年立屬天門郡澧音禮】 甲子徙南陽王翽爲隨郡王【翽呼會翻】魏馮太后忌青州刺史南郡王李惠【高祖之母惠女也故忌之】誣云惠將南叛十二月癸巳誅惠及妻并其子弟太后以猜嫌所夷滅者十餘家而惠所歷皆有善政魏人尤寃惜之 尚書令王僧䖍奏以朝廷禮樂多違正典大明中即以宫縣合和鞞拂【晉志曰鼙舞未詳所起然漢代已施於宴享矣傅毅張衡所賦皆其事也舊曲有五篇一關東有賢女二章和二年中三樂久長四四方皇五殿前生桂樹其辭盡亡魏作新歌五篇泰始中又别製新歌皆易其曲名拂舞出自江左舊云吳舞檢其歌非吳辭也亦陳於殿庭楊泓序曰自到江南見白符舞或言白鳬鳩舞云有此來數十年矣察其辭旨乃是吳人患孫皓虐政思屬晉也其曲有白鳩濟濟獨禄碣石淮南王五篇余觀其辭過魏晉諸公所作歌辭遠甚但失之悽楚非治世之音耳縣讀曰懸鞞與鼙同】節數雖會慮乖雅體又今之清商實由銅爵三祖風流遺音盈耳京洛相高江左彌貴【魏太祖起銅爵臺於鄴自作樂府被於管絃後遂置清商令以掌之屬光禄勲三祖謂魏太祖高祖烈祖也唐會要曰自晉播遷古樂遂分散不存苻堅滅凉始得漢魏清商之樂傳於前後二秦及宋武定關中收之入於江南隋平陳獲之隋又曰此華夏正聲也乃置清商署揔謂之清樂】中庸和雅莫近於斯而情變聽移稍復銷落十數年間亡者將半民間競造新聲雜曲煩淫無極宜命有司悉加補綴朝廷從之 是歲魏懷州刺史高允以老疾告歸鄉里尋復以安車徵至平城【復扶又翻】拜鎭軍大將軍中書監固辭不許乘車入殿朝賀不拜【朝直遥翻】<br />
<br />
  資治通鑑卷一百三十四<br />
<br />
<史部,編年類,資治通鑑>  <br>
   </div> 

<script src="/search/ajaxskft.js"> </script>
 <div class="clear"></div>
<br>
<br>
 <!-- a.d-->

 <!--
<div class="info_share">
</div> 
-->
 <!--info_share--></div>   <!-- end info_content-->
  </div> <!-- end l-->

<div class="r">   <!--r-->



<div class="sidebar"  style="margin-bottom:2px;">

 
<div class="sidebar_title">工具类大全</div>
<div class="sidebar_info">
<strong><a href="http://www.guoxuedashi.com/lsditu/" target="_blank">历史地图</a></strong>  
<a href="http://www.880114.com/" target="_blank">英语宝典</a>  
<a href="http://www.guoxuedashi.com/13jing/" target="_blank">十三经检索</a> 
<br><strong><a href="http://www.guoxuedashi.com/gjtsjc/" target="_blank">古今图书集成</a></strong> 
<a href="http://www.guoxuedashi.com/duilian/" target="_blank">对联大全</a> <strong><a href="http://www.guoxuedashi.com/xiangxingzi/" target="_blank">象形文字典</a></strong> 

<br><a href="http://www.guoxuedashi.com/zixing/yanbian/">字形演变</a>  <strong><a href="http://www.guoxuemi.com/hafo/" target="_blank">哈佛燕京中文善本特藏</a></strong>
<br><strong><a href="http://www.guoxuedashi.com/csfz/" target="_blank">丛书&方志检索器</a></strong> <a href="http://www.guoxuedashi.com/yqjyy/" target="_blank">一切经音义</a>  

<br><strong><a href="http://www.guoxuedashi.com/jiapu/" target="_blank">家谱族谱查询</a></strong>  <strong><a href="http://shufa.guoxuedashi.com/sfzitie/" target="_blank">书法字帖欣赏</a></strong> 
<br>

</div>
</div>


<div class="sidebar" style="margin-bottom:0px;">

<font style="font-size:22px;line-height:32px">QQ交流群9:489193090</font>


<div class="sidebar_title">手机APP 扫描或点击</div>
<div class="sidebar_info">
<table>
<tr>
	<td width=160><a href="http://m.guoxuedashi.com/app/" target="_blank"><img src="/img/gxds-sj.png" width="140"  border="0" alt="国学大师手机版"></a></td>
	<td>
<a href="http://www.guoxuedashi.com/download/" target="_blank">app软件下载专区</a><br>
<a href="http://www.guoxuedashi.com/download/gxds.php" target="_blank">《国学大师》下载</a><br>
<a href="http://www.guoxuedashi.com/download/kxzd.php" target="_blank">《汉字宝典》下载</a><br>
<a href="http://www.guoxuedashi.com/download/scqbd.php" target="_blank">《诗词曲宝典》下载</a><br>
<a href="http://www.guoxuedashi.com/SiKuQuanShu/skqs.php" target="_blank">《四库全书》下载</a><br>
</td>
</tr>
</table>

</div>
</div>


<div class="sidebar2">
<center>


</center>
</div>

<div class="sidebar"  style="margin-bottom:2px;">
<div class="sidebar_title">网站使用教程</div>
<div class="sidebar_info">
<a href="http://www.guoxuedashi.com/help/gjsearch.php" target="_blank">如何在国学大师网下载古籍?</a><br>
<a href="http://www.guoxuedashi.com/zidian/bujian/bjjc.php" target="_blank">如何使用部件查字法快速查字?</a><br>
<a href="http://www.guoxuedashi.com/search/sjc.php" target="_blank">如何在指定的书籍中全文检索?</a><br>
<a href="http://www.guoxuedashi.com/search/skjc.php" target="_blank">如何找到一句话在《四库全书》哪一页?</a><br>
</div>
</div>


<div class="sidebar">
<div class="sidebar_title">热门书籍</div>
<div class="sidebar_info">
<a href="/so.php?sokey=%E8%B5%84%E6%B2%BB%E9%80%9A%E9%89%B4&kt=1">资治通鉴</a> <a href="/24shi/"><strong>二十四史</strong></a>&nbsp; <a href="/a2694/">野史</a>&nbsp; <a href="/SiKuQuanShu/"><strong>四库全书</strong></a>&nbsp;<a href="http://www.guoxuedashi.com/SiKuQuanShu/fanti/">繁体</a>
<br><a href="/so.php?sokey=%E7%BA%A2%E6%A5%BC%E6%A2%A6&kt=1">红楼梦</a> <a href="/a/1858x/">三国演义</a> <a href="/a/1038k/">水浒传</a> <a href="/a/1046t/">西游记</a> <a href="/a/1914o/">封神演义</a>
<br>
<a href="http://www.guoxuedashi.com/so.php?sokeygx=%E4%B8%87%E6%9C%89%E6%96%87%E5%BA%93&submit=&kt=1">万有文库</a> <a href="/a/780t/">古文观止</a> <a href="/a/1024l/">文心雕龙</a> <a href="/a/1704n/">全唐诗</a> <a href="/a/1705h/">全宋词</a>
<br><a href="http://www.guoxuedashi.com/so.php?sokeygx=%E7%99%BE%E8%A1%B2%E6%9C%AC%E4%BA%8C%E5%8D%81%E5%9B%9B%E5%8F%B2&submit=&kt=1"><strong>百衲本二十四史</strong></a>  <a href="http://www.guoxuedashi.com/so.php?sokeygx=%E5%8F%A4%E4%BB%8A%E5%9B%BE%E4%B9%A6%E9%9B%86%E6%88%90&submit=&kt=1"><strong>古今图书集成</strong></a>
<br>

<a href="http://www.guoxuedashi.com/so.php?sokeygx=%E4%B8%9B%E4%B9%A6%E9%9B%86%E6%88%90&submit=&kt=1">丛书集成</a> 
<a href="http://www.guoxuedashi.com/so.php?sokeygx=%E5%9B%9B%E9%83%A8%E4%B8%9B%E5%88%8A&submit=&kt=1"><strong>四部丛刊</strong></a>  
<a href="http://www.guoxuedashi.com/so.php?sokeygx=%E8%AF%B4%E6%96%87%E8%A7%A3%E5%AD%97&submit=&kt=1">說文解字</a> <a href="http://www.guoxuedashi.com/so.php?sokeygx=%E5%85%A8%E4%B8%8A%E5%8F%A4&submit=&kt=1">三国六朝文</a>
<br><a href="http://www.guoxuedashi.com/so.php?sokeytm=%E6%97%A5%E6%9C%AC%E5%86%85%E9%98%81%E6%96%87%E5%BA%93&submit=&kt=1"><strong>日本内阁文库</strong></a> <a href="http://www.guoxuedashi.com/so.php?sokeytm=%E5%9B%BD%E5%9B%BE%E6%96%B9%E5%BF%97%E5%90%88%E9%9B%86&ka=100&submit=">国图方志合集</a> <a href="http://www.guoxuedashi.com/so.php?sokeytm=%E5%90%84%E5%9C%B0%E6%96%B9%E5%BF%97&submit=&kt=1"><strong>各地方志</strong></a>

</div>
</div>


<div class="sidebar2">
<center>

</center>
</div>
<div class="sidebar greenbar">
<div class="sidebar_title green">四库全书</div>
<div class="sidebar_info">

《四库全书》是中国古代最大的丛书,编撰于乾隆年间,由纪昀等360多位高官、学者编撰,3800多人抄写,费时十三年编成。丛书分经、史、子、集四部,故名四库。共有3500多种书,7.9万卷,3.6万册,约8亿字,基本上囊括了古代所有图书,故称“全书”。<a href="http://www.guoxuedashi.com/SiKuQuanShu/">详细>>
</a>

</div> 
</div>

</div>  <!--end r-->

</div>
<!-- 内容区END --> 

<!-- 页脚开始 -->
<div class="shh">

</div>

<div class="w1180" style="margin-top:8px;">
<center><script src="http://www.guoxuedashi.com/img/plus.php?id=3"></script></center>
</div>
<div class="w1180 foot">
<a href="/b/thanks.php">特别致谢</a> | <a href="javascript:window.external.AddFavorite(document.location.href,document.title);">收藏本站</a> | <a href="#">欢迎投稿</a> | <a href="http://www.guoxuedashi.com/forum/">意见建议</a> | <a href="http://www.guoxuemi.com/">国学迷</a> | <a href="http://www.shuowen.net/">说文网</a><script language="javascript" type="text/javascript" src="https://js.users.51.la/17753172.js"></script><br />
  Copyright &copy; 国学大师 古典图书集成 All Rights Reserved.<br>
  
  <span style="font-size:14px">免责声明:本站非营利性站点,以方便网友为主,仅供学习研究。<br>内容由热心网友提供和网上收集,不保留版权。若侵犯了您的权益,来信即刪。scp168@qq.com</span>
  <br />
ICP证:<a href="http://www.beian.miit.gov.cn/" target="_blank">鲁ICP备19060063号</a></div>
<!-- 页脚END --> 
<script src="http://www.guoxuedashi.com/img/plus.php?id=22"></script>
<script src="http://www.guoxuedashi.com/img/tongji.js"></script>

</body>
</html>
