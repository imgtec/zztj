<!DOCTYPE html PUBLIC "-//W3C//DTD XHTML 1.0 Transitional//EN" "http://www.w3.org/TR/xhtml1/DTD/xhtml1-transitional.dtd">
<html xmlns="http://www.w3.org/1999/xhtml">
<head>
<meta http-equiv="Content-Type" content="text/html; charset=utf-8" />
<meta http-equiv="X-UA-Compatible" content="IE=Edge,chrome=1">
<title>資治通鑒_157-資治通鑑卷一百五十六_157-資治通鑑卷一百五十六</title>
<meta name="Keywords" content="資治通鑒_157-資治通鑑卷一百五十六_157-資治通鑑卷一百五十六">
<meta name="Description" content="資治通鑒_157-資治通鑑卷一百五十六_157-資治通鑑卷一百五十六">
<meta http-equiv="Cache-Control" content="no-transform" />
<meta http-equiv="Cache-Control" content="no-siteapp" />
<link href="/img/style.css" rel="stylesheet" type="text/css" />
<script src="/img/m.js?2020"></script> 
</head>
<body>
 <div class="ClassNavi">
<a  href="/24shi/">二十四史</a> | <a href="/SiKuQuanShu/">四库全书</a> | <a href="http://www.guoxuedashi.com/gjtsjc/"><font  color="#FF0000">古今图书集成</font></a> | <a href="/renwu/">历史人物</a> | <a href="/ShuoWenJieZi/"><font  color="#FF0000">说文解字</a></font> | <a href="/chengyu/">成语词典</a> | <a  target="_blank"  href="http://www.guoxuedashi.com/jgwhj/"><font  color="#FF0000">甲骨文合集</font></a> | <a href="/yzjwjc/"><font  color="#FF0000">殷周金文集成</font></a> | <a href="/xiangxingzi/"><font color="#0000FF">象形字典</font></a> | <a href="/13jing/"><font  color="#FF0000">十三经索引</font></a> | <a href="/zixing/"><font  color="#FF0000">字体转换器</font></a> | <a href="/zidian/xz/"><font color="#0000FF">篆书识别</font></a> | <a href="/jinfanyi/">近义反义词</a> | <a href="/duilian/">对联大全</a> | <a href="/jiapu/"><font  color="#0000FF">家谱族谱查询</font></a> | <a href="http://www.guoxuemi.com/hafo/" target="_blank" ><font color="#FF0000">哈佛古籍</font></a> 
</div>

 <!-- 头部导航开始 -->
<div class="w1180 head clearfix">
  <div class="head_logo l"><a title="国学大师官网" href="http://www.guoxuedashi.com" target="_blank"></a></div>
  <div class="head_sr l">
  <div id="head1">
  
  <a href="http://www.guoxuedashi.com/zidian/bujian/" target="_blank" ><img src="http://www.guoxuedashi.com/img/top1.gif" width="88" height="60" border="0" title="部件查字,支持20万汉字"></a>


<a href="http://www.guoxuedashi.com/help/yingpan.php" target="_blank"><img src="http://www.guoxuedashi.com/img/top230.gif" width="600" height="62" border="0" ></a>


  </div>
  <div id="head3"><a href="javascript:" onClick="javascript:window.external.AddFavorite(window.location.href,document.title);">添加收藏</a>
  <br><a href="/help/setie.php">搜索引擎</a>
  <br><a href="/help/zanzhu.php">赞助本站</a></div>
  <div id="head2">
 <a href="http://www.guoxuemi.com/" target="_blank"><img src="http://www.guoxuedashi.com/img/guoxuemi.gif" width="95" height="62" border="0" style="margin-left:2px;" title="国学迷"></a>
  

  </div>
</div>
  <div class="clear"></div>
  <div class="head_nav">
  <p><a href="/">首页</a> | <a href="/ShuKu/">国学书库</a> | <a href="/guji/">影印古籍</a> | <a href="/shici/">诗词宝典</a> | <a   href="/SiKuQuanShu/gxjx.php">精选</a> <b>|</b> <a href="/zidian/">汉语字典</a> | <a href="/hydcd/">汉语词典</a> | <a href="http://www.guoxuedashi.com/zidian/bujian/"><font  color="#CC0066">部件查字</font></a> | <a href="http://www.sfds.cn/"><font  color="#CC0066">书法大师</font></a> | <a href="/jgwhj/">甲骨文</a> <b>|</b> <a href="/b/4/"><font  color="#CC0066">解密</font></a> | <a href="/renwu/">历史人物</a> | <a href="/diangu/">历史典故</a> | <a href="/xingshi/">姓氏</a> | <a href="/minzu/">民族</a> <b>|</b> <a href="/mz/"><font  color="#CC0066">世界名著</font></a> | <a href="/download/">软件下载</a>
</p>
<p><a href="/b/"><font  color="#CC0066">历史</font></a> | <a href="http://skqs.guoxuedashi.com/" target="_blank">四库全书</a> |  <a href="http://www.guoxuedashi.com/search/" target="_blank"><font  color="#CC0066">全文检索</font></a> | <a href="http://www.guoxuedashi.com/shumu/">古籍书目</a> | <a   href="/24shi/">正史</a> <b>|</b> <a href="/chengyu/">成语词典</a> | <a href="/kangxi/" title="康熙字典">康熙字典</a> | <a href="/ShuoWenJieZi/">说文解字</a> | <a href="/zixing/yanbian/">字形演变</a> | <a href="/yzjwjc/">金 文</a> <b>|</b>  <a href="/shijian/nian-hao/">年号</a> | <a href="/diming/">历史地名</a> | <a href="/shijian/">历史事件</a> | <a href="/guanzhi/">官职</a> | <a href="/lishi/">知识</a> <b>|</b> <a href="/zhongyi/">中医中药</a> | <a href="http://www.guoxuedashi.com/forum/">留言反馈</a>
</p>
  </div>
</div>
<!-- 头部导航END --> 
<!-- 内容区开始 --> 
<div class="w1180 clearfix">
  <div class="info l">
   
<div class="clearfix" style="background:#f5faff;">
<script src='http://www.guoxuedashi.com/img/headersou.js'></script>

</div>
  <div class="info_tree"><a href="http://www.guoxuedashi.com">首页</a> > <a href="/SiKuQuanShu/fanti/">四库全书</a>
 > <h1>资治通鉴</h1> <!--         下载:【右键另存为】即可 --></div>
  <div class="info_content zj clearfix">
  
<div class="info_txt clearfix" id="show">
<center style="font-size:24px;">157-資治通鑑卷一百五十六</center>
    資治通鑑卷一百五十六 宋 司馬光 撰<br />
<br />
  吳三省 音註<br />
<br />
  梁紀十二【起昭陽赤奮若盡閼逢攝提格凡二年】<br />
<br />
  高祖武皇帝十二<br />
<br />
  中大通丑年春正月辛卯上祀南郊大赦 魏竇泰奄至爾朱兆庭軍人因宴休惰忽見泰軍驚走追破之於赤谼嶺【杜佑曰石州離石縣有赤洪水即離石水赤洪其别名也高歡破爾朱兆于赤洪嶺蓋近此又曰赤洪水源出方山縣東流入離石谼戶工翻 考異曰魏帝紀正月庚寅朔甲午齊獻武王自晉陽出討兆丁酉大破之于赤洪嶺北齊帝紀出兵在去年破兆在今年按歲首宴會不應直至八日今從齊書】衆竝降散【降戶江翻下同】兆逃於窮山命左右西河張亮及蒼頭陳山提斬已首以降皆不忍兆乃殺所乘白馬自縊於樹歡親臨厚葬之【縊於賜翻又於計翻臨力鴆翻哭也】慕容紹宗攜爾朱榮妻子及兆餘衆詣歡降歡以義故待之甚厚【義故猶言義舊也】兆之在秀容左右皆密通款於歡唯張亮無啟疏【疏所故翻】歡嘉之以為丞相府參軍 魏罷諸行臺【天監十五年魏以李平為行臺節度統攻硤石諸軍踵魏初之制而置之也正光之末盜起始復置諸道行臺】 辛亥上祀明堂 丁巳魏主追尊其父為武穆帝太妃馮氏為武穆后母李氏為皇太妃 勞州刺史曹鳳東荆州刺史雷能勝等舉城降魏【曹鳳雷能勝皆蠻左也因其地授以州刺史降戶江翻】 魏侍中斛斯椿聞喬寧張子期之死【寧子期死見上卷上年】内不自安與南陽王寶炬武衛將軍元毗王思政密勸魏主圖丞相歡【椿本有圖歡之心因喬張之死懼禍將及决計為之】毗遵之玄孫也【道武建國之初常山王遵有佐命之功】舍人元士弼又言歡受詔不敬帝由是不悦椿勸帝置閤内都督部曲又增武直人數自直閤已下員别數百【武直謂武士之入直殿閤者據五代志紀北齊之制領軍府將軍掌宿衛禁掖朱華閤外凡禁宿衛官皆主之又左右衛府將軍各一人掌左右廂所主朱華閤以外各武衛將軍二人貳之其御仗属官有御仗正副都督御仗五職御仗等員直盪属官有直盪正副都督直入正副都督勳武前鋒正副都督勳武前鋒五職等員直衛属官有直衛正副都督翊衛正副都督等員直突属官有直突都督前鋒散都督等員直閤属官有朱衣直閤將軍直寢直齎直後之属又冇雲騎武騎驍騎遊擊前左右後等將軍左右虎賁等中郎將步兵越騎射聲屯騎長水等校尉奉車騎二都尉羽林監冗從僕射積弩積射強弩殿中等將軍及員外將軍武騎常侍殿中司馬督員外司馬督等蓋其制昉于晉代有損益觀北齊之制則當時之增置亦可概見矣】皆選四方驍勇者充之【驍堅堯翻】帝數出遊幸椿自部勒别為行陳【數所角翻行戶剛翻陳讀曰陣】由是朝政軍謀帝專與椿决之帝以關中大行臺賀拔岳擁重兵密與相結又出侍中賀拔勝為都督三荆等七州諸軍事【七州三荆及襄南襄郢南郢也】欲倚勝兄弟以敵歡【倚勝及岳也】歡益不悦侍中司空高乾之在信都也遭父喪不暇終服及孝武帝即位表請解職行喪詔聽解侍中【句絶】司空如故乾雖求退不謂遽見許既去内侍朝政多不關預居常怏怏【怏於兩翻】帝既貳於歡冀乾為己用嘗於華林園宴罷獨留乾謂之曰<br />
<br />
  司空弈世忠良【謂自高允以來】今              【日】復建殊效【復扶又翻】相與雖則君臣義同兄弟宜共立盟約以敦情契殷勤逼之乾對曰臣以身許國何敢有貳時事出倉猝且不謂帝有異圖遂不固辭亦不以啟歡及帝置部曲乾乃私謂所親曰主上不親勳賢而招集羣小數遣元士弼王思政往來關西與賀拔岳計議又出賀拔勝為荆州外示疎忌内實樹黨令其兄弟相近冀據有西方禍難將作必及於我乃密啟歡歡召乾詣并州面論時事乾因勸歡受魏禪歡以䄂掩其口曰勿妄言今令司空復為侍中門下之事一以相委歡屢啟請帝不許乾知變難將起密啟歡求為徐州【數所角翻近其靳翻難乃旦翻復扶又翻羅復詔復下同】二月辛酉以乾為驃騎大將軍開府儀同三司徐州刺史【驃匹妙翻騎奇計翻】以咸陽王坦為司空【為魏主殺高乾討高歡張本】 癸未上幸同泰寺講般若經【般音鉢若人者翻】七日而罷會者數萬人 魏正光以前阿至羅常附於魏【阿至羅高車種也魏書孝静帝興和三年阿至羅國主副伏羅越君子去賓來降封之為高車王】及中原多事阿至羅亦叛丞相歡招撫之阿至羅復降凡十萬戶三月辛卯詔復以歡為大行臺【魏方罷諸行臺今復命歡以此職以招撫阿至羅】使隨宜裁處【處昌呂翻】歡與之粟帛議者以為徒費無益歡不從及經畧河西大收其用【謂救曹泥及取万俟受洛干時也】 高乾將之徐州魏主聞其漏泄機事乃詔丞相歡曰乾邕與朕私有盟約今乃反復兩端歡聞其與帝盟亦惡之【惡烏故翻】即取乾前後數啟論時事者遣使封上【使疏吏翻下同上時掌翻】帝召乾對歡使責之乾曰陛下自立異圖乃謂臣為反覆人主加罪其可辭乎遂賜死帝又密勅東徐州刺史潘紹業殺其弟敖曹【按李延夀齊紀魏主遣東徐州刺史潘紹業密勅長樂太守龎蒼鷹殺敖曹則是高敖曹此時在信都也】敖曹先聞乾死伏壯士於路執紹業得勅書於袍領遂將十餘騎奔晉陽【將即亮翻騎奇計翻】歡抱其首哭曰天子枉害司空敖曹兄仲密為光州刺史帝勅青州斷其歸路【斷音短仲密由東莱歸勃海道出青州】仲密亦間行奔晉陽【間古莧翻】仲密名慎以字行 魏太師魯郡王肅卒【卒子恤翻】 丙辰南平元襄王偉卒 丁巳魏以趙郡王諶為太尉【諶氏壬翻】南陽王寶炬為太保 魏爾朱兆之入洛也【兆入洛見一百五十四卷二年】焚太常樂庫鐘磬俱盡節閔帝詔録尚書事長孫稚太常卿祖瑩等更造之至是始成命曰大成樂 魏青州民耿翔聚衆寇掠三齊【三齊因秦漢舊名言之】膠州刺史裴粲專事高談不為防禦夏四月翔掩襲州城【魏永安二年置膠州治東武城領東武高密平昌郡東武城今密州諸城縣是也】左右白賊至粲曰豈有此理左右又言已入州門粲乃徐曰耿王來可引之聽事自餘部衆且付城民翔斬之送首來降【降戶江翻下同】 五月魏東徐州民王早等殺刺史崔庠以下邳來降 【考異曰梁帝紀六月己卯魏建義城主蘭寶以下邳城降今從魏書】 六月壬申魏以驃騎大將軍樊子鵠為青膠大使督濟州刺史蔡儁等討耿翔【濟子禮翻】秋七月魏師至青州翔棄城來奔詔以為兖州刺史 壬辰魏以廣陵王欣為大司馬趙郡王諶為太師庚戌以前司徒賀拔允為太尉 【考異曰魏帝紀作賀拔渥按允字阿鞠渥蓋渥字誤為渥耳】初賀拔岳遣行臺郎馮景詣晉陽丞相歡聞岳使至甚喜【使疏吏翻】曰賀拔公詎憶吾邪與景歃血約與岳為兄弟【歃色甲翻】景還言於岳曰歡姦詐有餘不可信也府司馬宇文泰自請使晉陽【使疏吏翻】以觀歡之為人歡奇其狀貌曰此兒視瞻非常將留之泰固求復命歡既遣而悔之驛急追至關不及而返【項羽不殺沛公曹操之遣劉備桓玄之容劉裕類如此耳有天命者固非人之所能圖也】泰至長安謂岳曰高歡所以未簒者止憚公兄弟耳侯莫陳悦之徒非所忌也公但潜為之備圖歡不難今費也頭控弦之騎不下一萬夏州刺史斛拔彌俄突勝兵三千餘人【勝音升】靈州刺史曹泥河西流民紇豆陵伊利等各擁部衆未知所屬公若引軍近隴【近其靳翻隴坂也】扼其要害震之以威懷之以惠可收其士馬以資吾軍西輯氐羌北撫沙塞【靈夏塞外北臨沙漠】還軍長安匡輔魏室此桓文之舉也【舉一作功】岳大悦復遣泰詣洛陽請事密陳其狀【復扶又翻】魏主喜加泰武衛將軍使還報八月帝以岳為都督雍華等二十州諸軍事雍州刺史【魏泰和十一年分雍州置華州領華山澄城白水郡二十州雍華東華岐南岐豳原河渭涇夏東夏秦南秦梁南梁東梁巴益東益也雍於用翻華戶化翻】又割心前血遣使者齎以賜之【使疏吏翻】岳遂引兵西屯平凉以牧馬為名【所謂近隴也】斛拔彌俄突紇豆陵伊利及費也頭万俟受洛干鐵勒斛律沙門等皆附於岳【紇下沒翻万莫北翻俟渠之翻】唯曹泥附於歡秦南秦河渭四州刺史同會平凉受岳節度岳以夏州被邊要重【夏戶雅翻被皮義翻】欲求良刺史以鎮之衆舉宇文泰岳曰宇文左丞吾左右手何可廢也沈吟累日卒表用之【沈持林翻卒子恤翻】 九月癸酉魏丞相歡表讓王爵不許請分封邑十萬戶頒授勳義【勳義謂自信都從起義討爾朱有功勳者也】從之 冬十月庚申以尚書右僕射何敬容為左僕射吏部尚書謝舉為右僕射十一月癸巳魏以殷州刺史中山邸珍為徐州大都<br />
<br />
  督東道行臺僕射以討下邳【邸姓也風俗通漢有上郡太守邸杜討王早也】十二月丁巳魏主狩於嵩高己巳幸温湯【歷嵩高而南唯汝州梁縣有温湯耳】丁丑還宫 魏荆州刺史賀拔勝寇雍州【此梁之雍州治襄陽】拔下迮戍【迮側百翻亦作笮】扇動諸蠻雍州刺史廬陵王續遣軍擊之屢為所敗【敗蒲賣翻】漢南震駭勝又遣軍攻馮翊安定沔陽鄼城皆拔之【五代志竟陵郡藍水縣僑立馮翊郡沔陽郡後為復州襄陽郡隂城縣舊置鄼城郡蕭子顯齊志寧蠻府所領郡有安定郡領新安等縣五代志新安縣并入襄陽郡南漳縣當是置安定僑郡於南漳界也藍水唐并入郢州長夀縣隂城并入穀城縣沈約志馮翊郡治襄陽郡鄀縣沔彌兗翻鄼音贊】續遣電威將軍柳仲禮屯穀城以拒之【五代志穀城縣屬襄陽郡舊曰義城置義城郡】勝攻之不克乃還於是沔北盪為丘墟矣【盪徒朗翻】仲禮慶遠之孫也【柳慶遠見一百四十三卷齊東昏侯永元二年】 魏丞相歡患賀拔岳侯莫陳悦之彊右丞翟嵩曰嵩能間之【間古莧翻】使其自相屠滅歡遣之歡又使長史侯景招撫紇豆陵伊利伊利不從<br />
<br />
  六年春正月壬辰魏丞相歡擊伊利於河西擒之遷其部落於河東【河西五原河之西也河南亦五原河之東】魏主讓之曰伊利不侵不叛為國純臣【左傳戎子駒支曰為先君不侵不叛之臣】王忽伐之詎有一介行人先請之乎 魏東梁州民夷作亂【魏收志東梁州領金城直城安康魏明郡五代志西城郡舊置東梁州金城即今金州城也東梁州治焉】二月詔以行東雍州事豐陽泉企討平之【魏世祖置東雍州於平陽太和中罷孝昌中於平陽置唐州以唐堯都平陽因以名州建義初改為晉州未嘗復置東雍州也五代志曰雍州鄭縣後魏置東雍州參考魏收志鄭縣時已屬華州界未知此東雍州置於何地也魏收志豐陽縣屬上庸郡太安二年置姓譜曰國語潞泉余滿皆赤狄隗姓又吳全琮孫憚隆魏封南陽食邑白水因為泉氏企去智翻 考異曰北史作泉仚今從周書】企世為商洛豪族【商洛指漢古縣商縣上洛縣而言也隋志上洛郡有商洛縣】魏世祖以其曾祖景言為本縣令封丹水侯使其子孫襲之 壬戍魏大赦 癸亥上耕籍田大赦 魏永寧浮圖災觀者皆哭聲振城闕【魏起永寧浮圖見一百四十八卷天監十五年史言末俗深信浮圖以至於此振之印翻】 魏賀拔岳將討曹泥使都督武川趙貴至夏州與宇文泰謀之泰曰曹泥孤城阻遠未足為憂侯莫陳悦貪而無信宜先圖之岳不聽【曹泥附高歡岳不從宇文泰之言急於致討蓋欲報高歡禽伊利之役耳亦忿兵也】召悦會於高平與共討泥悦既得翟嵩之言乃謀取岳岳數與悦宴語【數所角翻】長史武川雷紹諫不聽岳使悦前行至河曲【河曲在靈州西河千里一曲河水自澆河至漢眴卷古縣率東北流至富平始曲而北流所謂河曲也富平唐靈州也】悦誘岳入營坐論軍事【誘音酉】悦陽稱腹痛而起其壻元洪景拔刀斬岳岳左右皆散走悦遣人諭之云我别受旨止取一人諸君勿怖【怖普布翻】衆以為然皆不敢動而悦心猶豫不即撫納乃還入隴屯水洛城【我朝以渭州籠竿城置德順軍水洛城在軍西一百里】岳衆散還平凉趙貴詣悦請岳尸葬之悦許之岳既死悦軍中皆相賀行臺郎中薛憕私謂所親曰悦才畧素寡輒害良將吾屬今為人虜矣何賀之有憕真度之從孫也【憕直陵翻將即亮翻薛真度見一百三十九卷齊明帝建武元年】岳衆未有所屬諸將以都督武川寇洛年最長【長知兩翻】推使總諸軍洛素無威畧不能齊衆乃自請避位趙貴曰宇文夏州英畧冠世【冠古玩翻】遠近歸心賞罰嚴明士卒用命若迎而奉之大事濟矣諸將或欲南召賀拔勝或欲東告魏朝【朝直遙翻】猶豫不决都督盛樂杜朔周曰【盛樂前漢之成樂縣也屬定襄郡後漢屬雲中郡魏晉省後魏先世園陵在焉魏收志永熙中置盛樂郡雲中治所魏土地記雲中城東八十里冇盛樂城宋白曰後魏所都盛樂唐為振武軍】遠水不救近火今日之事非宇文夏州無能濟者趙將軍議是也朔周請輕騎告哀且迎之【騎奇計翻】衆乃使朔周馳至夏州召泰泰與將佐賓客共議去留前大中大夫潁川韓褒曰此天授也又何疑乎侯莫陳悦井中蛙耳使君往必擒之衆以為悦在水洛去平凉不遠若已有賀拔公之衆則圖之實難願且留以觀變泰曰悦既害元帥【帥所類翻】自應乘勢直據平凉而退據水洛吾知其無能為也夫難得易失者時也【用漢蒯徹語意易以豉翻】若不早赴衆心將離夏州首望都督彌姐元進隂謀應悦【姐音紫又子也翻彌姐元進之族為州之首望官又為都督彌姐羌複姓】泰知之與帳下都督高平蔡祐謀執之祐曰元進會當反噬不如殺之泰曰汝有大决【言能决大計也】乃召元進等入計事泰曰隴賊逆亂當與諸人戮力討之諸人似有不同者何也祐即被甲持刀直入【被皮義翻】瞋目謂諸將曰【瞋七人翻】朝謀夕異何以為人今日必斷姦人首舉坐皆叩頭曰【斷丁管翻坐徂臥翻】願有所擇祐乃叱元進斬之并誅其黨因與諸將同盟討悦泰謂祐曰吾今以爾為子爾其以我為父乎泰與帳下輕騎馳赴平凉令杜朔周帥衆先據彈箏峽【杜祐曰彈箏峽在今原州之百泉縣即漢朝那縣地九域志渭州都盧峽即彈箏峽也水經云都盧山峽之内常有彈箏之聲又云弦歌之山峽口水流風吹摧響有似音韻也騎奇計翻帥讀曰率】時民間惶懼逃散者多軍士争欲掠之朔周曰宇文公方伐罪弔民奈何助賊為虐乎撫而遣之遠近悦附泰聞而嘉之朔周本姓赫連曾祖庫多汗避難改焉【汗音寒難乃旦翻】泰命復其舊姓名之曰達丞相歡使侯景招撫岳衆泰至安定遇之謂曰賀拔公雖死宇文泰尚存卿何為者景失色曰我猶箭耳唯人所射【射食亦翻英雄之姿表與其舉措必有異乎人者以侯景之凶狡宇文泰一語折之辭氣俱下良有以哉李密見唐太宗不覺驚服事亦類此】遂還【侯景不敢前至平凉】泰至平凉哭岳甚慟將士皆悲喜歡復使侯景與散騎常侍代郡張華原義寧太守太安王基勞泰【魏收志建義元年置義寧郡治孤遠城屬晉州五代志上黨郡沁源縣後魏置義寧郡又延和二年置太安郡於漢五原界屬朔州復扶又翻勞力到翻下慰勞同】泰不受欲刧留之曰留則共享富貴不然命在今日華原曰明公欲脅使者以死亡此非華原所懼也泰乃遣之基還言泰雄傑請及其未定擊滅之歡曰卿不見賀拔侯莫陳乎吾當以計拱手取之魏主聞岳死遣武衛將軍元毗慰勞岳軍召還洛陽并召侯莫陳悦毗至平凉軍中已奉宇文泰為主悦既附丞相歡不肯應召泰因元毗上表稱臣岳忽罹非命都督寇恪等令臣權掌軍事奉詔召岳軍入京今高歡之衆已至河東【亦謂五原河之東】侯莫陳悦猶在水洛士卒多是西人顧戀鄉邑若逼令赴闕悦躡其後歡邀其前恐敗國殄民所損更甚【此雖泰不就徵而為之辭而亦事勢所必致也敗補賣翻】乞少賜停緩【少詩沼翻】徐事誘導漸就東引【誘音酉】魏主乃以泰為大都督即統岳軍初岳以東雍州刺史李虎為左廂大都督【雍於用翻】岳死虎奔荆州說賀拔勝使收岳衆【說式芮翻】勝不從虎聞宇文泰代岳統衆乃自荆州還赴之至閿鄉【閿鄉在漢湖縣界隋改湖城縣為閿鄉縣閿音旻以李虎自荆州往返之地里考之則魏東雍州時置於鄭縣】為丞相歡别將所獲【將即亮翻】送洛陽魏主方謀取關中得虎甚喜拜衛將軍厚賜之使就泰虎歆之玄孫也【凉王李歆為沮渠蒙遜所滅】泰與悦書責以賀拔公有大功於朝廷君名微行薄【行下孟翻】賀拔公薦君為隴右行臺又高氏專權君與賀拔公同受密旨屢結盟約而君黨附國賊共危宗廟口血未乾【乾音干】匕首已發今吾與君皆受詔還闕今日進退唯君是視君若下隴東邁吾亦自北道同歸【平凉在隴山之北取道涇州東赴洛】若首鼠兩端吾則指日相見【言進兵討悦也左傳曰詰朝相見】魏主問泰以安秦隴之策泰表言宜召悦授以内官或處以瓜凉一藩【魏以燉煌郡為瓜州武威郡為凉州處昌呂翻】不然終為後患原州刺史史歸素為賀拔岳所親任河曲之變反為悦守【反為於偽翻】悦遣其黨王伯和成次安將兵二千助歸鎮原州【魏太延二年置高平鎮正光五年改置原州治高平城領高平長城二郡】泰遣都督侯莫陳崇帥輕騎一千襲之【帥讀曰率騎奇計翻】崇乘夜將十騎直扺城下餘衆皆伏於近路歸見騎少不設備【少詩沼翻】崇即入據城門高平令隴西李賢及弟遠穆在城中為崇内應於是中外鼓譟伏兵悉起遂擒歸及次安伯和等歸於平凉泰表崇行原州事三月泰引兵擊悦至原州衆軍畢集 夏四月癸丑朔日有食之 魏南秦州刺史隴西李弼說侯莫陳悦曰賀拔公無罪而公害之又不撫納其衆今奉宇文夏州以來聲言為主報讎此其勢不可敵也宜解兵以謝之不然必及禍悦不從宇文泰引兵上隴【說式芮翻為於偽翻上時掌翻】留兄子導為都督鎮原州泰軍令嚴肅秋毫無犯百姓大悦軍出木狹關【狹當作峽唐志原州平高縣西南有木峽關】雪深二尺【深式禁翻】泰倍道兼行出其不意悦聞之退保畧陽【晉武帝分天水置畧陽郡至隋廢郡為隴城縣】留萬人守水洛泰至水洛即降泰遣輕騎數百趣畧陽【降戶江翻趣七喻翻】悦退保上邽召李弼與之拒泰弼知悦必敗隂遣使詣泰請為内應悦棄州城南保山險【秦州治上邽城使疏吏翻】弼謂所部曰侯莫陳公欲還秦州汝輩何不裝束弼妻悦之姨也衆咸信之爭趣上邽弼先據城門以安集之遂舉城降泰泰即以弼為秦州刺史其夜悦出軍將戰軍自驚潰悦性猜忌既敗不聽左右近已【近其靳翻下於近同】與其二弟并子及謀殺岳者七八人棄軍迸走【迸北諍翻】數日之中盤桓往來不知所趣【趣向也七喻翻】左右勸向靈州依曹泥悦從之自乘騾【騾雷戈翻】令左右皆步從【從才用翻】欲自山中趣靈州宇文泰使原州都督賀拔頴追之悦望見追騎縊死於野【騎奇計翻縊於賜翻又於計翻】泰入上邽引薛憕為記室參軍收悦府庫財物山積泰秋毫不取皆以賞士卒左右竊一銀甕以歸泰知而罪之即剖賜將士悦黨豳州刺史孫定兒據州不下有衆數萬泰遣都督中山劉亮襲之定兒以大軍遠不為備亮先竪一纛於近城高嶺【竪而主翻立也纛從到翻又徒沃翻今軍中大皁旗名曰皁纛】自將二十騎馳入城定兒方置酒猝見亮至駭愕不知所為【愕五各翻】亮麾兵斬定兒遙指城外纛命二騎曰出召大軍城中皆懾服莫敢動【懾之涉翻】先是故氐王楊紹先乘魏亂逃歸武興復稱王【魏執楊紹先見一百四十六卷天監五年先悉薦翻復扶又翻】凉州刺史李叔仁為其民所執氐羌吐谷渾所在蜂起自南岐至瓜鄯【吐從暾入聲谷音浴鄯上扇翻又音善】跨州據郡者不可勝數宇文泰令李弼鎮原州夏州刺史拔也惡蚝鎮南秦州【拔也惡蚝自夏州徙鎮南秦勝音升拔也虜複姓蚝七吏翻】渭州刺史可朱渾道元鎮渭州【為可朱渾道元奔高歡張本可朱渾虜三字姓】衛將軍趙貴行秦州事徵豳涇東秦岐四州之粟以給軍楊紹先懼稱藩送妻子為質【質音致】夏州長史于謹言於泰曰明公據關中險固之地將士驍勇【驍堅堯翻】土地膏腴今天子在洛迫於羣兇若陳明公之懇誠算時事之利害請都關右挾天子以令諸侯奉王命以討叛亂此桓文之業千載一時也泰善之【于謹間關兵中有年矣今乃遇宇文氏卒以功名自見豈所謂知己者邪抑際遇自有時也然謹事廣陽王深所陳策畫不過隨時設變今事宇文泰則勉之以迎天子而成興王之業蓋知宇文泰之才足以有為所謂量而後入也載子亥翻】丞相歡聞泰定秦隴遣使甘言厚禮以結之【使疏吏翻】泰不受封其書使都督濟北張軌獻於魏主【濟子禮翻】斛斯椿問軌曰高歡逆謀行路皆知之人情所恃唯在西方未知宇文何如賀拔【言泰之才視賀拔岳為何如也】軌曰宇文公文足經國武能定亂椿曰誠如君言真可恃也魏主命泰二千騎鎮東雍州助為勢援【時置東雍州於華州鄭縣】仍命泰稍引軍而東泰以大都督武川梁禦為雍州刺史使將步騎五千前行先是丞相歡遣其都督太安韓軌將兵一萬據蒲坂以救侯莫陳悦【先悉薦翻將即亮翻】雍州刺史賈顯度以舟迎之梁禦見顯度說使從泰【說式芮翻】顯度即出迎禦禦入據長安魏主以泰為侍中驃騎大將軍開府儀同三司關西大都督畧陽縣公【驃匹妙翻騎奇計翻】承制封拜泰乃以寇洛為涇州刺史李弼為秦州刺史前畧陽太守張獻為南岐州刺史南岐州刺史盧待伯不受代泰遣輕騎襲而擒之【史言宇文泰所以能定覇】侍中封隆之言於丞相歡曰斛斯椿等今在京師必搆禍亂隆之與僕射孫騰爭尚魏主妺平原公主公主歸隆之騰泄其言於椿椿以白帝隆之懼逃還鄉里歡召隆之詣晉陽會騰帶仗入省擅殺御史懼罪亦逃就歡領軍婁昭辭疾歸晉陽【高歡所親無在洛者矣】帝以斛斯椿兼領軍改置都督及河内關西諸刺史華山王鷙在徐州歡使大都督邸珍奪其管鑰【去年歡使邸珍督徐州討下邳因奪其城華戶化翻】建州刺史韓賢濟州刺史蔡儁皆歡黨也【濟子禮翻】帝省建州以去賢【建州當太行路自晉陽入洛之要道也省州去賢不特銷歡黨亦去歡南道主人也去音羌呂翻】使御史舉儁罪以汝陽王叔昭代之歡上言儁勳重不可解奪汝陽懿德當受大藩臣弟永寶猥任定州【北史歡弟琛字元寶永恐當作元】宜避賢路帝不聽五月丙子魏主增置勳府庶子廂别六百人又增騎官廂别二百人【勳府庶子及騎官皆宿衛者也騎奇計翻】魏主欲伐晉陽【高歡時居晉陽】辛卯下詔戒嚴云欲自將伐梁【將即亮翻下同】河南諸州兵大閲於洛陽南臨洛水北際邙山帝戎服與斛斯椿臨觀之六月丁巳魏主密詔丞相歡稱宇文黑獺賀拔勝頗有異志【宇文泰字黑獺】故假稱南伐潛為之備王亦宜共為形援讀訖燔之歡表以為荆雍將有逆謀【荆謂賀拔勝雍謂宇文泰雍於用翻】臣今潛勒兵馬三萬自河東渡又遣恒州刺史庫狄干等將兵四萬自來違津渡【恒戶登翻自恒州渡來違津其地當在平城之西河津之要也自此渡河至夏州 考異曰丘悦三國典畧作朱違津今從北齊書及北史】領軍將軍婁昭等將兵五萬以討荆州冀州刺史尉景等將山東兵七萬突騎五萬以討江左皆勒所部伏聽處分【處昌呂翻分扶問翻】帝知歡覺其變乃出歡表令羣臣議之欲止歡軍歡亦集并州僚佐共議【高歡建大丞相府於并州僚佐皆從居之】還以表聞仍云臣為嬖佞所間【嬖卑義翻又博計翻間古莧翻】陛下一旦賜疑臣若敢負陛下使身受天殃子孫殄絶陛下若垂信赤心使干戈不動佞臣一二人願斟量廢出【斟酌也量度也斟量猶今人言酌量也量音良出當作黜】丁卯帝使大都督源子恭守陽胡【陽胡即陽壺城在邵郡白水縣白水漢河東之垣縣也水經注曰白水逕垣縣故城北又東南逕陽壺城東城即垣縣之壺丘亭白水又東南流注于河按陽壺即崤谷之北岸魏主欲入關故先使子恭守之以防歡邀截】汝陽王暹守石濟又以儀同三司賈顯智為濟州刺史帥豫州刺史斛斯元夀東趣濟州【濟子禮翻帥讀曰率趣七喻翻】元夀椿之弟也蔡儁不受代帝愈怒辛未帝復録洛中文武議意以答歡【復扶又翻】且使舍人温子升為勅賜歡曰朕不勞尺刃坐為天子所謂生我者父母貴我者高王今若無事背王規相攻討【背蒲妹翻規圖也】則使身及子孫還如王誓近慮宇文為亂賀拔應之故戒嚴欲與王俱為聲援今觀其所為更無異迹東南不賓為日已久今天下戶口減半未宜窮兵極武朕既闇昧不知佞人為誰頃高乾之死豈獨朕意【言歡亦惡乾封上其所論時事故因殺之】王忽對昂言兄枉死人之耳目何易可輕如聞庫狄干語王云本欲取懦弱者為主無事立此長君使其不可駕御今但作十五日行自可廢之更立餘者【易弋豉翻語牛倨翻長知兩翻更工行翻】如此議論自是王間勳人豈出佞臣之口去歲封隆之叛今年孫騰逃去不罪不送誰不怪王【言歡既不加二人以罪又不械送洛陽也】王若事君盡誠何不斬送二首王雖啟云西去【西去言將西攻宇文泰也】而四道俱進或欲南度洛陽或欲東臨江左【四道俱進謂河東來違津及婁昭尉景之兵也婁昭討荆州尉景臨江左皆南指洛陽河東來違津之兵則牽制宇文泰使不得東下高歡之計實出於此魏主窺見其心術而言之】言之者猶應自怪聞之者寧能不疑王若晏然居北在此雖有百萬之衆終無圖彼之心王若舉旗南指縱無匹馬隻輪猶欲奮空拳而爭死朕本寡德王已立之百姓無知或謂實可若為他人所圖則彰朕之惡假令還為王殺幽辱虀粉了無遺恨本望君臣一體若合符契不圖今日分疎至此【今人猶謂辨析為分疎】中軍將軍王思政言於魏主曰高歡之心昭然可知洛陽非用武之地宇文泰乃心王室今往就之還復舊京何慮不克帝深然之遣散騎侍郎河東柳慶見泰於高平共論時事泰請奉迎輿駕慶復命帝復私謂慶曰【復扶又翻】朕欲向荆州何如慶曰關中形勝宇文泰才畧可依荆州地非要害南迫梁寇臣愚未見其可帝又問閤内都督宇文顯和【時南北朝皆有直閣將軍魏又置閣内都督用斛斯椿之言也】顯和亦勸帝西幸時帝廣徵州郡兵東郡太守河東裴俠帥所部詣洛陽【俠戶頰翻帥讀曰率】王思政問曰今權臣擅命王室日卑奈何俠曰宇文泰為三軍所推居百二之地【漢書田肯曰秦形勝之國也帶河阻山懸隔千里持戟百萬秦得百二焉蘇林注曰得二得百萬中之二萬人也秦地險固二萬人足當諸侯百萬人也】所謂已操戈矛寧肯授人以柄雖欲投之恐無異避湯入火也思政曰然則如何而可俠曰圖歡有立至之憂西巡有將來之慮且至關右徐思其宜耳思政然之乃進俠於帝授左中郎將【將即亮翻】初丞相歡以洛陽久經喪亂欲遷都於鄴帝曰高祖定鼎河洛為萬世之基王既功存社稷宜遵太和舊事歡乃止至是復謀遷都【復扶又翻】遣三千騎鎮建興【慕容永分上黨置建興郡魏為建州騎奇計翻】益河東及濟州兵擁諸州和糴粟悉運入鄴城【和糴以充軍食蓋始於此焉歷唐至宋而民始不勝其病矣濟子禮翻】帝又勅歡曰王若厭伏人情【厭於協翻又如字】杜絶物議唯有歸河東之兵罷建興之戍送相州之粟【相州治鄴城相悉亮翻】追濟州之軍使蔡儁受代邸珍出徐止戈散馬各事家業脱須糧廩别遣轉輸則讒人結舌疑悔不生王高枕太原【枕職任翻】朕垂拱京洛矣王若馬首南向問鼎輕重朕雖不武為社稷宗廟之計欲止不能决在於王非朕能定為山止簣相為惜之【書旅曰為山九仞功虧一蕢孔安國注云喻向成也未成一簣猶不為山論語孔子曰譬如為山未成一簣止吾止也相為音於偽翻】歡上表極言宇文泰斛斯椿罪惡帝以廣寧太守廣寧任祥兼尚書左僕射加開府儀同三司祥棄官走渡河據郡待歡【魏收志廣寧郡屬朔州領石門中川二縣五代志馬邑郡善陽縣後齊置廣寧郡孝昌以來寄治并州界時歡在并州祥當直走就歡不必據郡以待歡之南也又按五代志建州沁水縣舊置廣寧郡祥所據者蓋沁水之廣寧也若其鄉里則當在朔州之廣寧】帝乃勅文武官北來者任其去留遂下制書數歡咎惡【數所具翻】召賀拔勝赴行在所勝以問太保掾范陽盧柔柔曰高歡悖逆公席卷赴都與决勝負死生以之上策也【掾於絹翻悖蒲沒翻又蒲内翻卷讀曰捲】北阻魯陽南并舊楚【江陵舊楚之郢都在其界内】東連兗豫西引關中帶甲百萬觀舋而動中策也【舋許靳翻】舉三荆之地庇身於梁功名皆去下策也勝笑而不應【賀拔勝既不能勤王又不能保境挺身奔梁卒如盧柔所料原勝之心以柔書生故易其言殊不知博歡往迹默察時變以坐論勝敗則書生之見固非武夫健將之所能及也】帝以宇文泰兼尚書僕射為關西大行臺許妻以馮翊長公主【妻七細翻長知兩翻】謂泰帳内都督秦郡楊荐曰【考魏收地形志魏無秦郡五代志曰扶風雍縣後魏置秦平郡又雍州醴泉縣後魏曰寧夷西魏置寧夷郡後周改曰秦郡】卿歸語行臺【語作倨翻】遣騎迎我以荐為直閤將軍泰以前秦州刺史駱超為大都督將輕騎一千赴洛又遣荐與長史宇文側出關候接【候接魏主也】丞相歡召其弟定州刺史琛使守晉陽【琛丑林翻】命長史崔暹佐之暹挺之子也【通鑑以此别為破六韓拔陵所敗之崔暹】歡勒兵南出告其衆曰孤以爾朱擅命建大義於海内奉戴主上【事見上卷四年】誠貫幽明橫為斛斯椿讒搆【橫戶孟翻】以忠為逆今者南邁誅椿而已以高敖曹為前鋒宇文泰亦移檄州郡數歡罪惡【數所具翻】自將大軍高平前軍屯弘農【將即亮翻】賀拔勝軍於汝水【賀拔勝蓋出魯陽屯襄城界僅越境而止耳】秋七月己丑魏主親勒兵十餘萬屯河橋以斛斯椿為前驅陳於邙山之北椿請帥精騎二千夜度河掩其勞弊【陳讀曰陣帥讀曰率騎奇計翻】帝始然之黄門侍郎楊寛說帝曰【說式芮翻】高歡以臣伐君何所不至今假兵於人恐生他變椿若度河萬一有功是滅一高歡生一高歡矣帝遂勅椿停行椿歎曰頃熒惑入南斗【晉天文志曰南斗六星天廟也將有天子之事占於斗熒惑罰入之天子不安其位後所謂天子下殿走是也】星今上信左右間搆【間古莧翻】不用吾計豈天道乎宇文泰聞之謂左右曰高歡數日行八九百里此兵家所忌當乘便擊之而主上以萬乘之重【乘成正翻】不能度河决戰方緣津據守且長河萬里扞禦為難若一處得度大事去矣即以大都督趙貴為别道行臺自蒲坂濟趣并州【别道而進示將擬高歡之後趣七喻翻】遣大都督李賢將精騎一千赴洛陽【以迎魏主也將即亮翻】帝使斛斯椿與行臺長孫稚大都督潁川王斌之鎮虎牢行臺長孫子彦鎮陜【斌音彬陜式冉翻】賈顯智斛斯元夀鎮滑臺斌之鑒之弟【安樂王鑒見一百五十卷普通五年】子彦稚之子也歡使相州刺史竇泰趣滑臺【相息亮翻】建州刺史韓賢趣石濟竇泰與顯智遇於長夀津【水經河水右逕滑臺城又東北逕凉城縣又東北為長夀津述征記曰凉城到長夀津六十里】顯智隂約降於歡引軍退【降戶江翻下同】軍司元玄覺之馳還請益師帝遣大都督侯幾紹赴之【魏書官氏志内入諸姓有侯幾氏】戰於滑臺東顯智以軍降紹戰死北中郎將田怙為歡内應歡潛軍至野王帝知之斬怙【五代志河内郡治河内縣舊曰野王】歡至河北十餘里【自野王進兵距河纔十餘里】再遣使口申誠欵帝不報【使疏吏翻】丙午歡引軍渡河魏主問計於羣臣或欲奔梁或云南依賀拔勝或云西就關中或云守洛口死戰【洛水過鞏縣東而北入於河謂之洛口】計未决元斌之與斛斯椿爭權棄椿還紿帝云高歡兵已至 【考異曰魏書斛斯椿傳云椿懼已不免復啟出帝假說遊聲以刧脅帝帝信之遂入關按齊高祖舉兵向洛而云椿刼脅帝不亦誣乎此乃魏收欲媚齊人重椿之罪耳今從齊書高祖紀及北史椿傳】丁未帝遣使召椿還遂帥南陽王寶炬清河王亶廣陽王湛以五千騎宿於瀍西南陽王别舍沙門惠臻負璽持千牛刀以從【帥讀曰率璽斯氏翻從才用翻】衆知帝將西出其夜亡者過半亶湛亦逃歸湛深之子也【廣陽王深為葛榮所殺】武衛將軍雲中獨孤信單騎追帝【令孤德棻曰獨孤部與魏俱起三十六大姓之一也】帝歎曰將軍辭父母捐妻子而來世亂識忠臣豈虛言也戊申帝西奔長安李賢遇帝於崤中【陜有三崤之山魏太和十一年置崤縣屬恒農郡】己酉歡入洛陽舍於永寧寺遣領軍婁昭等追帝請帝東還長孫子彦不能守陜棄城走高敖曹帥勁騎追帝至陜西不及【陜西陜城之西也】帝鞭馬長騖糗漿乏絶【騖音務糗去久翻米麥為之鄭玄曰漿酢酨周官漿人掌之】三二日間從官唯飲澗水【從才用翻】至湖城有王思村民以麥飯壺漿獻帝帝悦復一村十年至稠桑【湖城即漢湖縣城湖城西有稠桑驛復方目翻】潼關大都督毛鴻賓迎獻酒食從官始解饑渴八月甲寅丞相歡集百官謂曰為臣奉主匡救危亂若處不諫爭出不陪從【處昌呂翻爭讀曰諍從才用翻】緩則耽寵爭榮急則委之逃竄臣節安在衆莫能對兼尚書左僕射辛雄曰主上與近習圖事雄等不得預聞及乘輿西幸【乘繩工翻】若即追隨恐跡同佞黨留待大王又以不從蒙責雄等進退無所逃罪歡曰卿等備位大臣當以身報國羣佞用事卿等嘗有一言諫爭乎使國家之事一朝至此罪欲何歸乃收雄及開府儀同三司叱列延慶兼吏部尚書崔孝芬都官尚書劉廞兼度支尚書天水楊機散騎常侍元士弼皆殺之【歡責辛雄等以罪而殺之亦以去魏朝之望將以樹其私黨耳廞許金翻度徒洛翻】孝芬子司徒從事中郎猷間行入關魏主使以本官奏門下事【凡事經門下者使之聞奏也間古莧翻】歡推司徒清河王亶為大司馬承制决事居尚書省宇文泰使趙貴梁禦帥甲騎二千奉迎帝循河西行謂禦曰此水東流而朕西上若得復見洛陽親詣陵廟卿等功也帝及左右皆流涕泰備儀衛迎帝謁見於東陽驛【水經志渭水過長安城北又東過新豐東合西陽水又東合東陽水二水竝南出廣鄉原上時掌翻復扶又翻謁見賢遍翻】免冠流涕曰臣不能式遏寇虐【詩曰式遏寇虐無俾作慝】使乘輿播遷臣之罪也【乘成正翻】帝曰公之忠節著於遐邇朕以不德負乘致寇【易曰負且乘致寇至負也者小人之事也乘也者君子之器也小人而乘君子之器盜思奪之矣】今日相見深用厚顔【鄭玄曰顔之厚者不知慚於人】方以社稷委公公其勉之將士皆呼萬歲遂入長安以雍州廨舍為宫【廨居隘翻公宇也】大赦以泰為大將軍雍州刺史兼尚書令【雍於用翻】軍國之政咸取决焉别置二尚書分掌機事以行臺尚書毛遐周惠達為之時軍國草創二人積糧儲治器械簡士馬魏朝賴之【治直之翻朝直遥翻】泰尚馮翊長公主拜駙馬都尉【漢武帝置奉車駙馬騎三都尉魏晉以來尚主者例拜駙馬都尉長知兩翻】先是熒惑入南斗去而復還留止六旬上以諺云熒惑入南斗天子下殿走乃跣而下殿以禳之【鄭玄曰却變曰禳先悉見翻】及聞魏主西奔慚曰虜亦應天象邪 己未武興王楊紹先為秦南秦二州刺史【己未之下當有以字梁書亦然】 辛酉魏丞相歡自追迎魏主戊辰清河王亶下制大赦歡至弘農九月癸巳使行臺僕射元子思帥侍官迎帝【帥讀曰率下同】己酉攻潼關克之擒毛鴻賓進屯華隂長城【此城戰國時魏築長城自鄭濱洛者也華戶化翻】龍門都督薛崇禮以城降歡【魏收志華山郡夏陽縣有龍門山水經注河水出龍門口蓋兩山夾河故謂之龍門大禹所鑿也後魏置龍門郡龍門縣屬南汾州隋廢龍門郡以龍門縣屬河東郡此即河東之龍門也西對夏陽之龍門山降戶江翻】賀拔勝使長史元頴行荆州事守南陽自帥所部西赴關中至淅陽【漢析縣屬弘農郡宋永初志屬順陽郡魏收志折陽郡屬析州五代志析州内鄉縣舊置淅陽郡淅思歷翻】聞歡已屯華隂欲還行臺左丞崔謙曰今帝室顛覆主上蒙塵公宜倍道兼行朝於行在【朝直遙翻】然後與宇文行臺同心戮力唱舉大義天下孰不望風響應今捨此而退恐人人解體一失事機後悔何及勝不能用遂還歡退屯河東使行臺長史薛瑜守潼關 【考異曰北史作薛瑾典畧作薛長瑜北齊帝紀作薛瑜今從北齊書】大都督厙狄温守封陵築城於蒲津西㟁【水經注潼關直北隔河有層阜巍然獨秀孤峙河陽世謂之風陵蒲津即河東郡蒲阪津也唐志蒲州河西縣有蒲津關河東縣南有風陵關】以薛紹宗為華州刺史使守之【華戶化翻】以高敖曹行豫州事歡自發晉陽至是凡四十啟魏主皆不報歡乃東還遣行臺侯景等引兵向荆州荆州民鄧誕等執元頴以應景賀拔勝至景逆擊之勝兵敗帥數百騎來奔【奔梁也通鑑以梁繫年故書來奔帥讀曰率騎音奇計翻下同】魏主之在洛陽也密遣閤内都督河南趙剛召東荆州刺史馮景昭帥兵入援兵未及發魏主西入關景昭集府中文武議所從司馬馮道和請據州待北方處分【北方謂高歡也處昌呂翻分扶問翻】剛曰公宜勒兵赴行在所【天子所至為行在所】久之更無言者剛抽刀投地曰公若欲為忠臣請斬道和如欲從賊可速見殺景昭感悟即帥衆赴關中侯景引兵逼穰城【謂攻元穎時也】東荆州民楊祖歡等起兵以其衆邀景昭於路景昭戰敗剛沒蠻中【魏東荆州本蠻左所據之地】冬十月丞相歡至洛陽 【考異曰齊書北史皆云九月庚寅還至洛陽按歡九月己酉克潼關己酉九月二十九日也不容庚寅已還至洛陽庚寅乃九月十日也】又遣僧道榮奉表於孝武帝曰陛下若遠賜一制許還京洛臣當帥勒文武式清宫禁若返正無日則七廟不可無主萬國須有所歸臣寧負陛下不負社稷帝亦不答歡乃集百官耆老議所立時清河王亶出入已稱警蹕歡醜之乃託以孝昌以來昭穆失序【昭讀為韶時昭翻】永安以孝文為伯考永熙遷孝明於夾室【敬宗尊其父彭城王勰為皇帝列於七廟以孝文為伯考高歡之立魏孝武改元永熙孝武自以於孝明帝兄弟也禮兄弟不相入廟遂遷孝明帝主於夾室凡宗廟之制有東西夾室】業喪祚短職此之由【喪息亮翻】遂立清河世子善見為帝謂亶曰欲立王不如立王之子亶不自安輕騎南走歡追還之丙寅孝静帝即位於城東北【歡以善見者清河王懌之孫於孝明帝猶子也入繼大宗則昭穆順遂立之城東北者洛陽城東北也】時年十一大赦改元天平【魏自此分為東西】魏宇文泰進軍攻潼關斬薛瑜虜其卒七千人還長安進位大丞相東魏行臺薛脩義等度河據楊氏璧【據薛端傳楊氏壁在龍門西㟁當在華隂夏陽之間蓋華隂諸楊遇亂築壁以自守因以為名】魏司空參軍河東薛端糾帥村民擊却東魏復取楊氏【帥讀曰率下同復扶又翻】丞相泰遣南汾州刺史蘇景恕鎮之【魏汾州本治蒲子城孝昌中陷移治西河時西河已屬東魏故西魏僑置南汾州於楊氏壁】 丁卯以信武將軍元慶和為鎮北將軍帥衆伐東魏 初魏孝武既與丞相歡有隙齊州刺史侯淵兖州刺史樊子鵠青州刺史東莱王貴平【元貴平封東莱王】隂相連結以觀時變淵亦遣使通於歡所【使疏吏翻】及孝武帝入關清河王亶承制以汝陽王暹為齊州刺史暹至城西淵不時納城民劉桃符等潛引暹入城淵帥騎出走妻子部曲悉為暹所虜行及廣里【司馬彪續漢志濟北郡盧縣有光里光廣聲相近也】會承制以淵行青州事【承制言命出於清河王亶】歡遺淵書曰【遺於季翻】卿勿以部曲單少憚於東行齊人澆薄唯利是從【少詩沼翻澆堅堯翻】齊州尚能迎汝陽王青州豈不能開門待卿也淵乃復東【復扶又翻下同】暹歸其妻子部曲貴平亦不受代淵襲高陽郡克之【魏收志高陽郡故樂安地宋文帝置高陽郡屬冀州後入魏屬青州五代志青州北海縣舊曰下密置高陽郡】置累重於城中【累力瑞翻重直用翻】自帥輕騎遊掠於外貴平使其世子帥衆攻高陽淵夜趣東陽【青州治東陽城】見州民餽糧者紿之曰臺軍已至殺戮殆盡我世子之人也脱走還城汝何為復往聞者皆棄糧走比曉【紿待亥翻比必利翻及也】復謂行人曰臺軍昨夜已至高陽我是前鋒今至此不知侯公竟在何所城民恟懼遂執貴平出降【侯淵取韓樓亦用此術技止此耳恟許勇翻】戊辰淵斬貴平傳首洛陽 庚午東魏以趙郡王諶為大司馬【諶世壬翻】咸陽王坦為太尉開府儀同三司高盛為司徒高敖曹為司空坦樹之弟也【元樹奔梁中大通四年為魏所擒】丞相歡以洛陽西逼西魏南近梁境乃議遷鄴書下三日即行【書謂歡所下書也近其靳翻下遐稼翻】丙子東魏主發洛陽四十萬戶狼狽就道收百官馬尚書丞郎已上非陪從者盡令乘驢歡留後部分事畢還晉陽【從才用翻分扶問翻】改司州為洛州以尚書令元弼為洛州刺史鎮洛陽【魏明元帝取洛陽置洛州孝文帝徙都洛太和十七年改為司州高歡既逼東魏主遷鄴改相州為司州復以洛陽改為洛州】以行臺尚書司馬子如為尚書左僕射與右僕射高隆之侍中高岳孫騰留鄴共知朝政【朝直遙翻】詔以遷民貲產未立出粟一百三十萬石以賑之【賑之忍翻】十一月兖州刺史樊子鵠據瑕丘以拒東魏南青州刺史大野拔帥衆就之【大野虜複姓】庚寅東魏主至鄴居北城相州之廨改相州刺史為司州牧【東魏司州領魏陽平廣平汲廣宗東郡北廣平林慮頓丘濮陽黎陽清河郡相息亮翻廨音居隘翻】魏郡太守為魏尹【後北齊改魏尹為清都尹】是時六坊之衆從孝武帝西行者不及萬人【魏蓋以宿衛之士分為六坊】餘皆北徙竝給常廩春秋賜帛以供衣服【養兵之害始此】乃於常調之外【調徒弔翻】隨豐稔之處折絹糴粟以供國用 十二月魏丞相泰遣儀同李虎李弼趙貴擊曹泥於靈州 閏月元慶和克瀨鄉而據之【司馬彪續漢志陳國苦縣有賴鄉老子所居也晉苦縣屬梁國後魏并苦縣入陳留谷陽縣】 魏孝武帝閨門無禮從妹不嫁者三人【從才用翻】皆封公主平原公主明月南陽王寶炬之同產也從帝入關丞相泰使元氏諸王取明月殺之帝不悦或時彎弓或時椎案由是復與泰有隙【椎直追翻復扶又翻】癸巳帝飲酒遇酖而殂【年二十五】泰與羣臣議所立多舉廣平王贊贊孝武之兄子也侍中濮陽王順於别室埀涕謂泰曰高歡逼逐先帝立幼主以專權明公宜反其所為廣平冲幼不如立長君而奉之泰乃奉太宰南陽王寶炬而立之【寶炬孝文帝之孫京兆王愉之子長知兩翻】順素之曾孫也【按魏宗室名順者前後凡三人道武伐中山順欲於平城自立此時猶以拓拔為姓又任城王澄之子順叱高肇門者指元乂妻諫靈后粧飾斥徐紇以抗直著後聞河隂之難奔走而死此元順則常山王素之孫二人皆已改姓元氏】殯孝武帝於草堂佛寺諫議大夫宋球慟哭嘔血漿粒不入口者數日泰以其名儒不之罪也 魏賀拔勝之在荆州也表武衛將軍獨孤信為大都督東魏既取荆州魏以信為都督三荆州諸軍事尚書右僕射東南道行臺大都督荆州刺史以招懷之蠻酋樊五能攻破淅陽郡以應魏【酋慈尤翻 考異曰北史作樊大能今從魏書】東魏西荆州刺史辛纂欲討之【據隋紀辛纂時鎮穰城則西荆州即荆州以穰城在東荆州之西故云】行臺郎中李廣諫曰淅陽四面無民唯一城之地山路深險表裏羣蠻今少遣兵則不能制賊【少詩沼翻】多遣則根本虛弱脱不如意大挫威名人情一去州城難保纂曰豈可縱賊不討廣曰今所憂在心腹何暇治疥癬【治直之翻】聞臺軍不久應至【臺軍謂東魏所遣軍也】公但約勒屬城使完壘撫民以討之雖失淅陽不足惜也纂不從遣兵攻之兵敗諸將因亡不返城民密召獨孤信信至武陶【武陶疑當作武關】東魏遣恒農太守田八能【恒農即弘農後魏避顯祖諱改弘曰恒戶登翻】帥羣蠻拒信於淅陽【帥讀曰率】又遣都督張齊民以步騎三千出信之後信謂其衆曰今士卒不滿千人首尾受敵【謂田八能拒其前張齊民出其後也】若還擊齊民則土民必謂我退走【土民謂淅陽之民】必争來邀我不如進擊八能破之齊民自潰矣遂擊破八能乘勝襲穰城辛纂勒兵出戰大敗還趣城門未及闔【戰敗奔還門者惶遽未及下關也趣七喻翻】信令都督武川楊忠為前驅【楊忠隋文帝之父也隋氏自以為出於華隂楊震而忠則世居武川隋氏序其世曰本弘農華隂之楊漢太尉震十四世至文帝震八世孫北平太守鉉鉉子元夀魏初為武川鎮司馬因家於神武樹頹縣元夀生惠嘏嘏生烈烈生禎禎生忠】忠叱門者曰大軍已至城中有應爾等求生何不避走門者皆散忠帥衆入城斬纂以狥城中懾服【懾之涉翻】信分兵定三荆居半歲東魏高敖曹侯景將兵奄至城下信兵少不敵與楊忠皆來奔【為賀拔勝與信忠還魏張本將即亮翻少詩沼翻】<br />
<br />
  資治通鑑卷一百五十六  <br>
   </div> 

<script src="/search/ajaxskft.js"> </script>
 <div class="clear"></div>
<br>
<br>
 <!-- a.d-->

 <!--
<div class="info_share">
</div> 
-->
 <!--info_share--></div>   <!-- end info_content-->
  </div> <!-- end l-->

<div class="r">   <!--r-->



<div class="sidebar"  style="margin-bottom:2px;">

 
<div class="sidebar_title">工具类大全</div>
<div class="sidebar_info">
<strong><a href="http://www.guoxuedashi.com/lsditu/" target="_blank">历史地图</a></strong>  
<a href="http://www.880114.com/" target="_blank">英语宝典</a>  
<a href="http://www.guoxuedashi.com/13jing/" target="_blank">十三经检索</a> 
<br><strong><a href="http://www.guoxuedashi.com/gjtsjc/" target="_blank">古今图书集成</a></strong> 
<a href="http://www.guoxuedashi.com/duilian/" target="_blank">对联大全</a> <strong><a href="http://www.guoxuedashi.com/xiangxingzi/" target="_blank">象形文字典</a></strong> 

<br><a href="http://www.guoxuedashi.com/zixing/yanbian/">字形演变</a>  <strong><a href="http://www.guoxuemi.com/hafo/" target="_blank">哈佛燕京中文善本特藏</a></strong>
<br><strong><a href="http://www.guoxuedashi.com/csfz/" target="_blank">丛书&方志检索器</a></strong> <a href="http://www.guoxuedashi.com/yqjyy/" target="_blank">一切经音义</a>  

<br><strong><a href="http://www.guoxuedashi.com/jiapu/" target="_blank">家谱族谱查询</a></strong>  <strong><a href="http://shufa.guoxuedashi.com/sfzitie/" target="_blank">书法字帖欣赏</a></strong> 
<br>

</div>
</div>


<div class="sidebar" style="margin-bottom:0px;">

<font style="font-size:22px;line-height:32px">QQ交流群9:489193090</font>


<div class="sidebar_title">手机APP 扫描或点击</div>
<div class="sidebar_info">
<table>
<tr>
	<td width=160><a href="http://m.guoxuedashi.com/app/" target="_blank"><img src="/img/gxds-sj.png" width="140"  border="0" alt="国学大师手机版"></a></td>
	<td>
<a href="http://www.guoxuedashi.com/download/" target="_blank">app软件下载专区</a><br>
<a href="http://www.guoxuedashi.com/download/gxds.php" target="_blank">《国学大师》下载</a><br>
<a href="http://www.guoxuedashi.com/download/kxzd.php" target="_blank">《汉字宝典》下载</a><br>
<a href="http://www.guoxuedashi.com/download/scqbd.php" target="_blank">《诗词曲宝典》下载</a><br>
<a href="http://www.guoxuedashi.com/SiKuQuanShu/skqs.php" target="_blank">《四库全书》下载</a><br>
</td>
</tr>
</table>

</div>
</div>


<div class="sidebar2">
<center>


</center>
</div>

<div class="sidebar"  style="margin-bottom:2px;">
<div class="sidebar_title">网站使用教程</div>
<div class="sidebar_info">
<a href="http://www.guoxuedashi.com/help/gjsearch.php" target="_blank">如何在国学大师网下载古籍?</a><br>
<a href="http://www.guoxuedashi.com/zidian/bujian/bjjc.php" target="_blank">如何使用部件查字法快速查字?</a><br>
<a href="http://www.guoxuedashi.com/search/sjc.php" target="_blank">如何在指定的书籍中全文检索?</a><br>
<a href="http://www.guoxuedashi.com/search/skjc.php" target="_blank">如何找到一句话在《四库全书》哪一页?</a><br>
</div>
</div>


<div class="sidebar">
<div class="sidebar_title">热门书籍</div>
<div class="sidebar_info">
<a href="/so.php?sokey=%E8%B5%84%E6%B2%BB%E9%80%9A%E9%89%B4&kt=1">资治通鉴</a> <a href="/24shi/"><strong>二十四史</strong></a>&nbsp; <a href="/a2694/">野史</a>&nbsp; <a href="/SiKuQuanShu/"><strong>四库全书</strong></a>&nbsp;<a href="http://www.guoxuedashi.com/SiKuQuanShu/fanti/">繁体</a>
<br><a href="/so.php?sokey=%E7%BA%A2%E6%A5%BC%E6%A2%A6&kt=1">红楼梦</a> <a href="/a/1858x/">三国演义</a> <a href="/a/1038k/">水浒传</a> <a href="/a/1046t/">西游记</a> <a href="/a/1914o/">封神演义</a>
<br>
<a href="http://www.guoxuedashi.com/so.php?sokeygx=%E4%B8%87%E6%9C%89%E6%96%87%E5%BA%93&submit=&kt=1">万有文库</a> <a href="/a/780t/">古文观止</a> <a href="/a/1024l/">文心雕龙</a> <a href="/a/1704n/">全唐诗</a> <a href="/a/1705h/">全宋词</a>
<br><a href="http://www.guoxuedashi.com/so.php?sokeygx=%E7%99%BE%E8%A1%B2%E6%9C%AC%E4%BA%8C%E5%8D%81%E5%9B%9B%E5%8F%B2&submit=&kt=1"><strong>百衲本二十四史</strong></a>  <a href="http://www.guoxuedashi.com/so.php?sokeygx=%E5%8F%A4%E4%BB%8A%E5%9B%BE%E4%B9%A6%E9%9B%86%E6%88%90&submit=&kt=1"><strong>古今图书集成</strong></a>
<br>

<a href="http://www.guoxuedashi.com/so.php?sokeygx=%E4%B8%9B%E4%B9%A6%E9%9B%86%E6%88%90&submit=&kt=1">丛书集成</a> 
<a href="http://www.guoxuedashi.com/so.php?sokeygx=%E5%9B%9B%E9%83%A8%E4%B8%9B%E5%88%8A&submit=&kt=1"><strong>四部丛刊</strong></a>  
<a href="http://www.guoxuedashi.com/so.php?sokeygx=%E8%AF%B4%E6%96%87%E8%A7%A3%E5%AD%97&submit=&kt=1">說文解字</a> <a href="http://www.guoxuedashi.com/so.php?sokeygx=%E5%85%A8%E4%B8%8A%E5%8F%A4&submit=&kt=1">三国六朝文</a>
<br><a href="http://www.guoxuedashi.com/so.php?sokeytm=%E6%97%A5%E6%9C%AC%E5%86%85%E9%98%81%E6%96%87%E5%BA%93&submit=&kt=1"><strong>日本内阁文库</strong></a> <a href="http://www.guoxuedashi.com/so.php?sokeytm=%E5%9B%BD%E5%9B%BE%E6%96%B9%E5%BF%97%E5%90%88%E9%9B%86&ka=100&submit=">国图方志合集</a> <a href="http://www.guoxuedashi.com/so.php?sokeytm=%E5%90%84%E5%9C%B0%E6%96%B9%E5%BF%97&submit=&kt=1"><strong>各地方志</strong></a>

</div>
</div>


<div class="sidebar2">
<center>

</center>
</div>
<div class="sidebar greenbar">
<div class="sidebar_title green">四库全书</div>
<div class="sidebar_info">

《四库全书》是中国古代最大的丛书,编撰于乾隆年间,由纪昀等360多位高官、学者编撰,3800多人抄写,费时十三年编成。丛书分经、史、子、集四部,故名四库。共有3500多种书,7.9万卷,3.6万册,约8亿字,基本上囊括了古代所有图书,故称“全书”。<a href="http://www.guoxuedashi.com/SiKuQuanShu/">详细>>
</a>

</div> 
</div>

</div>  <!--end r-->

</div>
<!-- 内容区END --> 

<!-- 页脚开始 -->
<div class="shh">

</div>

<div class="w1180" style="margin-top:8px;">
<center><script src="http://www.guoxuedashi.com/img/plus.php?id=3"></script></center>
</div>
<div class="w1180 foot">
<a href="/b/thanks.php">特别致谢</a> | <a href="javascript:window.external.AddFavorite(document.location.href,document.title);">收藏本站</a> | <a href="#">欢迎投稿</a> | <a href="http://www.guoxuedashi.com/forum/">意见建议</a> | <a href="http://www.guoxuemi.com/">国学迷</a> | <a href="http://www.shuowen.net/">说文网</a><script language="javascript" type="text/javascript" src="https://js.users.51.la/17753172.js"></script><br />
  Copyright &copy; 国学大师 古典图书集成 All Rights Reserved.<br>
  
  <span style="font-size:14px">免责声明:本站非营利性站点,以方便网友为主,仅供学习研究。<br>内容由热心网友提供和网上收集,不保留版权。若侵犯了您的权益,来信即刪。scp168@qq.com</span>
  <br />
ICP证:<a href="http://www.beian.miit.gov.cn/" target="_blank">鲁ICP备19060063号</a></div>
<!-- 页脚END --> 
<script src="http://www.guoxuedashi.com/img/plus.php?id=22"></script>
<script src="http://www.guoxuedashi.com/img/tongji.js"></script>

</body>
</html>
