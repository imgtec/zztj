<!DOCTYPE html PUBLIC "-//W3C//DTD XHTML 1.0 Transitional//EN" "http://www.w3.org/TR/xhtml1/DTD/xhtml1-transitional.dtd">
<html xmlns="http://www.w3.org/1999/xhtml">
<head>
<meta http-equiv="Content-Type" content="text/html; charset=utf-8" />
<meta http-equiv="X-UA-Compatible" content="IE=Edge,chrome=1">
<title>資治通鑒_203-資治通鑑卷二百二_203-資治通鑑卷二百二</title>
<meta name="Keywords" content="資治通鑒_203-資治通鑑卷二百二_203-資治通鑑卷二百二">
<meta name="Description" content="資治通鑒_203-資治通鑑卷二百二_203-資治通鑑卷二百二">
<meta http-equiv="Cache-Control" content="no-transform" />
<meta http-equiv="Cache-Control" content="no-siteapp" />
<link href="/img/style.css" rel="stylesheet" type="text/css" />
<script src="/img/m.js?2020"></script> 
</head>
<body>
 <div class="ClassNavi">
<a  href="/24shi/">二十四史</a> | <a href="/SiKuQuanShu/">四库全书</a> | <a href="http://www.guoxuedashi.com/gjtsjc/"><font  color="#FF0000">古今图书集成</font></a> | <a href="/renwu/">历史人物</a> | <a href="/ShuoWenJieZi/"><font  color="#FF0000">说文解字</a></font> | <a href="/chengyu/">成语词典</a> | <a  target="_blank"  href="http://www.guoxuedashi.com/jgwhj/"><font  color="#FF0000">甲骨文合集</font></a> | <a href="/yzjwjc/"><font  color="#FF0000">殷周金文集成</font></a> | <a href="/xiangxingzi/"><font color="#0000FF">象形字典</font></a> | <a href="/13jing/"><font  color="#FF0000">十三经索引</font></a> | <a href="/zixing/"><font  color="#FF0000">字体转换器</font></a> | <a href="/zidian/xz/"><font color="#0000FF">篆书识别</font></a> | <a href="/jinfanyi/">近义反义词</a> | <a href="/duilian/">对联大全</a> | <a href="/jiapu/"><font  color="#0000FF">家谱族谱查询</font></a> | <a href="http://www.guoxuemi.com/hafo/" target="_blank" ><font color="#FF0000">哈佛古籍</font></a> 
</div>

 <!-- 头部导航开始 -->
<div class="w1180 head clearfix">
  <div class="head_logo l"><a title="国学大师官网" href="http://www.guoxuedashi.com" target="_blank"></a></div>
  <div class="head_sr l">
  <div id="head1">
  
  <a href="http://www.guoxuedashi.com/zidian/bujian/" target="_blank" ><img src="http://www.guoxuedashi.com/img/top1.gif" width="88" height="60" border="0" title="部件查字,支持20万汉字"></a>


<a href="http://www.guoxuedashi.com/help/yingpan.php" target="_blank"><img src="http://www.guoxuedashi.com/img/top230.gif" width="600" height="62" border="0" ></a>


  </div>
  <div id="head3"><a href="javascript:" onClick="javascript:window.external.AddFavorite(window.location.href,document.title);">添加收藏</a>
  <br><a href="/help/setie.php">搜索引擎</a>
  <br><a href="/help/zanzhu.php">赞助本站</a></div>
  <div id="head2">
 <a href="http://www.guoxuemi.com/" target="_blank"><img src="http://www.guoxuedashi.com/img/guoxuemi.gif" width="95" height="62" border="0" style="margin-left:2px;" title="国学迷"></a>
  

  </div>
</div>
  <div class="clear"></div>
  <div class="head_nav">
  <p><a href="/">首页</a> | <a href="/ShuKu/">国学书库</a> | <a href="/guji/">影印古籍</a> | <a href="/shici/">诗词宝典</a> | <a   href="/SiKuQuanShu/gxjx.php">精选</a> <b>|</b> <a href="/zidian/">汉语字典</a> | <a href="/hydcd/">汉语词典</a> | <a href="http://www.guoxuedashi.com/zidian/bujian/"><font  color="#CC0066">部件查字</font></a> | <a href="http://www.sfds.cn/"><font  color="#CC0066">书法大师</font></a> | <a href="/jgwhj/">甲骨文</a> <b>|</b> <a href="/b/4/"><font  color="#CC0066">解密</font></a> | <a href="/renwu/">历史人物</a> | <a href="/diangu/">历史典故</a> | <a href="/xingshi/">姓氏</a> | <a href="/minzu/">民族</a> <b>|</b> <a href="/mz/"><font  color="#CC0066">世界名著</font></a> | <a href="/download/">软件下载</a>
</p>
<p><a href="/b/"><font  color="#CC0066">历史</font></a> | <a href="http://skqs.guoxuedashi.com/" target="_blank">四库全书</a> |  <a href="http://www.guoxuedashi.com/search/" target="_blank"><font  color="#CC0066">全文检索</font></a> | <a href="http://www.guoxuedashi.com/shumu/">古籍书目</a> | <a   href="/24shi/">正史</a> <b>|</b> <a href="/chengyu/">成语词典</a> | <a href="/kangxi/" title="康熙字典">康熙字典</a> | <a href="/ShuoWenJieZi/">说文解字</a> | <a href="/zixing/yanbian/">字形演变</a> | <a href="/yzjwjc/">金 文</a> <b>|</b>  <a href="/shijian/nian-hao/">年号</a> | <a href="/diming/">历史地名</a> | <a href="/shijian/">历史事件</a> | <a href="/guanzhi/">官职</a> | <a href="/lishi/">知识</a> <b>|</b> <a href="/zhongyi/">中医中药</a> | <a href="http://www.guoxuedashi.com/forum/">留言反馈</a>
</p>
  </div>
</div>
<!-- 头部导航END --> 
<!-- 内容区开始 --> 
<div class="w1180 clearfix">
  <div class="info l">
   
<div class="clearfix" style="background:#f5faff;">
<script src='http://www.guoxuedashi.com/img/headersou.js'></script>

</div>
  <div class="info_tree"><a href="http://www.guoxuedashi.com">首页</a> > <a href="/SiKuQuanShu/fanti/">四库全书</a>
 > <h1>资治通鉴</h1> <!--         下载:【右键另存为】即可 --></div>
  <div class="info_content zj clearfix">
  
<div class="info_txt clearfix" id="show">
<center style="font-size:24px;">203-資治通鑑卷二百二</center>
    資治通鑑卷二百二   宋 司馬光 撰<br />
<br />
  胡三省 音註<br />
<br />
  唐紀十八【起重光協洽盡重光大荒落几十一年】<br />
<br />
  高宗天皇大聖大弘孝皇帝中之下<br />
<br />
  咸亨二年春正月甲子上幸東都 【考異曰舊本紀及太子弘傳正月乙巳幸東都留太子弘於京師監國明年十月己未又云皇太子監國新本紀唐歷統紀皆連歲言太子監國按離長安時已留太子監國及自東都將還豈得又令監國按實録此月無監國事唯明年十月有之今從之】夏四月甲申以西突厥阿史那都支為左驍衛大將軍兼匐延都督【顯慶二年平賀魯以處本昆部為匐延都督府厥九勿翻驍堅堯翻匐蒲北翻】以安集五咄陸之衆【咄當没翻】 初武元慶等既死【事見上卷乾封元年】皇后奏以其姊子賀蘭敏之為士彠之嗣【彠一虢翻】襲爵周公改姓武氏累遷弘文館學士左散騎常侍【大宗在藩於秦府置文學館學士其後弘文崇文二館皆有學士散悉亶翻騎奇寄翻】魏國夫人之死也【亦見乾封元年】上見敏之悲泣曰曏吾出視朝猶無恙朝已不救何倉猝如此【朝直遥翻】敏之號哭不對后聞之曰此兒疑我由是惡之敏之貌美蒸於太原王妃及居妃喪釋衰絰奏妓【號戶高翻惡烏路翻衰倉冋翻妓渠綺翻】司衛少卿楊思儉女有殊色上及后自選以為太子妃昏有日矣敏之逼而淫之后於是表言敏之前後辠惡請加竄逐六月丙子敕流雷州復其本姓至韶州以馬韁絞死【雷州漢徐聞縣地梁置南合州隋曰合州仍置海康縣大業廢州唐武德五年復置貞觀八年改曰雷州韶州漢南野縣地吳孫皓甘露元年分立始興郡唐武德初置番州貞觀元年改韶州舊志雷州至京師六十五百四十七里至東都五千八百三十六里韶州至京師四千九百三十二里至東都四千一百四十二里韁居良翻】朝士坐與敏之交遊流嶺南者甚衆【朝直遥翻】 秋七月乙未朔高侃破高麗餘衆於安市城【麗力知翻】 九月丙申潞州長史徐王元禮薨 冬十一月甲午朔日有食之 車駕自東都幸許汝十二月癸酉校獵於葉縣【舊志東都至許州四百里至汝州百八十里葉縣舊屬南陽郡後并省後齊置襄州後周廢州置南襄城郡隋廢郡為葉縣屬許州葉式涉翻】丙戌還東都<br />
<br />
  三年春正月辛丑以太子左衛副率梁積夀為姚州道行軍摠管【太子十率府各有副率位四品率所律翻】將兵討叛蠻【將即亮翻】 庚戌昆明蠻十四姓二萬三千戶内附置殷敦總三州【㸑蠻西有昆明蠻一曰昆彌蠻以西洱河為境即葉榆河也去長安九千里殷州居戎州西北總州居西南敦州居南遠不過五百餘里近三百里】 二月庚午徙吐谷渾於鄯州浩亹水南【漢書地理志浩亹水東至允吾入湟允吾唐為鄯州龍支縣水經注浩亹河出允吾西北塞外東逕浩亹縣故城南又東流注于湟水俗呼為閤門河吐從暾入聲谷音浴鄯時戰翻浩音誥亹音門】吐谷渾畏吐蕃之彊不安其居又鄯州地狹尋徙靈州以其部落置安樂州【時以靈州鳴沙縣地置安樂州樂音洛】以可汗諾曷鉢為刺史【可從刋入聲汗音寒】吐谷渾故地皆入於吐蕃 己卯侍中永安郡公姜恪薨 夏四月庚午上幸合璧宫 吐蕃遣其大臣仲琮入貢上問以吐蕃風俗對曰吐蕃地薄氣寒風俗朴魯然法令嚴整上下一心議事常自下而起因人所利而行之斯所以能持久也上詰以吞滅吐谷渾【見上卷龍朔三年誥去吉翻】敗薛仁貴【見上卷咸亨元年敗蒲邁翻】寇逼凉州事【吐蕃既滅吐谷渾又破西域則寇逼凉州矣】對曰臣受命貢獻而已軍旅之事非所聞也上厚賜而遣之癸未遣都水使者黄仁素使于吐蕃【使疏吏翻】 秋八月壬午特進高陽郡公許敬宗卒【卒子恤翻】太常博士袁思古議敬宗弃長子於荒徼【徼吉弔翻】嫁少女於夷貊【少詩照翻貊莫白翻】案諡法名與實爽曰繆請諡為繆【繆靡幼翻】敬宗孫太子舍人彦伯訟思古與許氏有怨請改諡太常博士王福畤議【畤音止】以為得失一朝榮辱千載【大常博士擬謚皆跡其功行為之褒貶大行大名之小行小名之載于亥翻】若嫌隙有實當據法推繩如其不然義不可奪戶部尚書戴至德謂福畤曰高陽公任遇如是何以謚之為繆對曰㫺晉司空何曾既忠且孝徒以日食萬錢秦秀謚之為繆【見八十卷晉武帝咸寧四年】許敬宗忠孝不逮於曾而飲食男女之累過之【累力瑞翻】諡之曰繆無負許氏矣詔集五品已上更議禮部尚書陽思敬議按諡法既過能改曰恭請諡曰恭詔從之敬宗嘗奏流其子昂于嶺南又以女嫁蠻酋馮盎之子多納其貨【酋慈由翻】故思古議及之福畤勃之父也【王勃見二百卷龍朔元年】 九月癸卯徙沛王賢為雍王【雍於用翻】冬十月己未詔太子監國【監古衘翻】 壬戌車駕發東都十一月戊子朔日有食之 甲辰車駕至京師 十二月高侃與高麗餘衆戰于白水山破之新羅遣兵救高麗侃擊破之 癸卯以左庶子劉仁軌同中書門下三品 太子罕接宫臣典膳丞全椒邢文偉輒減所供膳【東宫典膳局郎正六品上丞正八品上掌進膳嘗食全椒縣時屬滁州】并上書諫太子太子復書謝以多疾及入侍少暇嘉納其意【上時掌翻】頃之右史缺上曰邢文偉事吾子能撤膳進諫此直士也擢為右史【起居舍人從六品上屬中書省掌修記言之史録天子之制誥德音如記言之制以紀時政之損益季終則授之於國史龍朔改曰右史】太子因宴集命宫臣擲倒【唐散樂有舞伎舞輪伎長蹻伎跳鈴伎擲倒伎跳劒伎吞劒伎皆梁之遺伎也】次至左奉裕率王及善【隋置太子内率府擬上臺千牛衛掌東宫千牛備身侍奉之事龍朔改為左右奉裕率率所律翻】及善曰擲倒自有伶官【伶盧經翻】臣若奉令恐非所以羽翼殿下也太子謝之上聞之賜及善縑百匹尋遷左千牛衛將軍【千牛刀即人主防身刀也取莊子庖丁言解千牛而芒刃不減之義後魏有千牛備身掌宿衛侍從隋煬帝置備身府唐改千牛府】<br />
<br />
  四年春正月丙辰絳州刺史鄭惠王元懿薨 三月丙申詔劉仁軌等改修國史以許敬宗等所記多不實故也 夏四月丙子車駕幸九成宫 閏五月燕山道摠管右領軍大將軍李謹行大破高麗叛者於瓠蘆河之西【胡嶠曰黑車子之北有牛蹄突厥人身牛足其地有寒水曰瓠河夏秋水厚二尺秋冬氷徹底常燒器銷氷乃得飲余按唐書劉仁軌傳此瓠蘆河當在高麗南界新羅七里城之北燕因肩翻下同】俘獲數千人餘衆皆犇新羅時謹行妻劉氏留伐奴城高麗引靺鞨攻之劉氏檈甲帥衆守城久之虜退【靺鞨音末曷檈音宦帥讀曰率】上嘉其功封燕國夫人謹行靺鞨人突地稽之子也【突地稽見一百八十九卷高祖武德四年】武力絶人為衆夷所憚 秋七月婺州大水溺死者五千人【婺亡遇翻溺奴狄翻】 八月辛丑上以瘧疾令太子於延福殿受諸司啓事【瘧迸約翻】 冬十月壬午中書令閻立本薨 乙巳車駕還京師 十二月丙午弓月疎勒二王來降【降戶江翻下同】西突厥興㫺亡可汗之世諸部離散弓月及阿悉吉皆叛【阿悉吉即阿悉結弩失畢五俟斤之一也】蘇定方之西討也【見二百卷顯慶二年】擒阿悉吉以歸弓月南結吐蕃北招咽麪【咽麫亦鐵勒種居得嶷海咽於甸翻麫眡見翻】共攻疎勒降之上遣鴻臚卿蕭嗣業發兵討之嗣業兵未至弓月懼與踈勒皆入朝上赦其辠遣歸國【臚陵如翻朝直遥翻】<br />
<br />
  上元元年【是年八月方改元】春正月壬午以左庶子同中書門下三品劉仁軌為雞林道大摠管【帝以新羅國為雞林州】衛尉卿李弼右領軍大將軍李謹行副之發兵討新羅時新羅王法敏既納高麗叛衆又據百濟故地使人守之上大怒詔削法敏官爵其弟右驍衛員外大將軍臨海郡公仁問在京師【驍堅堯翻】立以為新羅王使歸國 三月辛亥朔日有食之 賀蘭敏之既得辠皇后奏召武元爽之子承嗣於嶺南【乾封元年元爽流振州】襲爵周公拜尚衣奉御【尚衣奉御屬殿中省掌衣服詳其制度辯其名數】夏四月辛卯遷宗正卿 秋八月壬辰追尊宣簡公為宣皇帝妣張氏為宣莊皇后懿王為光皇帝妣賈氏為光懿皇后【後魏金門鎮將熙太祖虎之祖也謚宣簡公魏憧主天賜太祖虎之父也謚懿王】太武皇帝為神堯皇帝太穆皇后為太穆神皇后文皇帝為太宗文武聖皇帝文德皇后為文德聖皇后皇帝稱天皇皇后稱天后以避先帝先后之稱【實欲自尊而以避先帝先后之稱為言武后之意也之稱尺證翻】改元赦天下戊戌敕文武官三品以上服紫金玉帶四品服深緋<br />
<br />
  金帶五品服淺緋金帶六品服深緑七品服淺緑並銀帶八品服深青九品服淺青並䃋石帶【鍮石似金而非金鍮託侯翻】庶人服黄銅鐵帶自非庶人不聽服黄【非庶人謂工商雜戶】 九月癸丑詔追復長孫晟長孫无忌官爵以无忌曾孫翼襲爵趙公聽无忌喪歸陪葬昭陵【无忌削官爵而死見二百卷顯慶四年】甲寅上御翔鸞閣【據舊書郝處俊傳翔鸞閣在含元殿東唐六典含元殿翼以二閣左】<br />
<br />
  【曰翔鸞右曰棲鳳二閣之下為東西朝堂】觀大酺【酺薄胡翻】分音樂為東西朋使雍王賢主東朋周王顯主西朋角勝為樂郝處俊諫曰二王春秋尚少志趣未定當推棃讓棗【梁元帝遺武陵王書有是言樂音洛少詩沼翻推吐雷翻】相親如一今分二朋遞相誇競俳優小人言辭無度恐其交爭勝負譏誚失禮非所以崇禮義勸敦睦也上瞿然曰【誚才笑翻瞿九遇翻驚視貌】卿遠識非衆人所及也遽止之是日衛尉卿李弼暴卒于宴所為之廢酺一日【為于偽翻】 冬十一月丙午朔車駕發京師己酉校獵華山之曲武原【華山在華州華隂縣南曲武原在華山下華戶化翻】戊辰至東都箕州録事參軍張君澈等誣告刺史蔣王惲及其子汝南郡王煒謀反【惲於粉翻煒於鬼翻】敕通事舍人薛思貞馳傳往按之【傳張戀翻】十二月癸未惲惶懼自縊死【縊於計翻】上知其非罪深痛惜之斬君澈等四人 戊子于闐王伏闍雄來朝【闐徒賢翻闍視遮翻朝直遥翻下同】 辛卯波斯王卑路斯來朝 壬寅天后上表以為國家聖緒出自玄元皇帝【老子姓李名耳唐祖之乾封元年尊為玄元皇帝上時掌翻下同】請令王公以下皆習老子每歲明經凖孝經論語策試又請自今父在為母服齊衰三年【古禮父在為母服朞為于偽翻齊音咨衰倉回翻】又京官八品以上宜量加俸禄【京官在京官也量音良】及其餘便宜合十二條詔書褒美皆行之 是歲有劉曉者上疏論選【選須絹翻下同考異曰會要作劉嶢今從統紀】以為今選曹以檢勘為公道【檢勘者謂考其功過察其假名承偽隱冒升降】書判為得人殊不知考其德行才能况書判借人者衆矣又禮部取士專用文章為甲乙故天下之士皆捨德行而趨文藝有朝登甲科而夕陷刑辟者雖日誦萬言何關理體【理體即治道行下孟翻趨七喻翻下同辟毗亦翻】文成七步未足化人况盡心卉木之間極筆烟霞之際以斯成俗豈非大謬夫人之慕名如水趨下上有所好下必甚焉陛下若取士以德行為先文藝為末則多士雷奔四方風動矣【趨七喻翻好呼到翻下隣好同】<br />
<br />
  二年春正月丙寅以于闐國為毗沙都督府分其境内為十州以于闐王尉遲伏闍雄為毗沙都督【闐徒賢翻又徒見翻尉紆勿翻闇視遮翻】 辛未吐蕃遣其大臣論吐渾彌來請和且請與吐谷渾復修鄰好【吐從瞰入聲下同復扶又翻谷音浴】上不許 二月劉仁軌大破新羅之衆於七重城【重音直龍翻】又使靺鞨浮海略新羅之南境斬獲甚衆仁軌引兵還【靺音末鞨音曷還從宣翻又音如字】詔以李謹行為安東鎮撫大使【使疏吏翻下同】屯新羅之買肖城以經略之三戰皆捷新羅乃遣使入貢且謝罪上赦之復新羅王法敏官爵金仁問中道而還【使疏吏翻還音旋又如字】改封臨海郡公 三月丁巳天后祀先蠶於邙山之陽【漢儀三月桑始生皇后親桑於苑中蠶室養蠶千薄以上祠以中牢羊豕續漢志三月皇后帥公卿諸侯夫人蠶祠先蠶禮以少牢注云今蠶神曰苑疏婦人寓氏公主唐制皇后歲祀一季春吉已享先蠶遂以親桑蜀注曰通典先蠶天駟也】百官及朝集使皆陪位【朝直遥翻下同】 上苦風眩甚議使天后攝知國政中書侍郎同三品郝處俊曰天子理外后理内天之道也【記昏義曰天子聽男教后聽女順天子理陽道后治隂德天子聽外治后聽内職教順成俗外内和順國家理治此之謂盛德】昔魏文著令雖有幼主不許皇后臨朝所以杜禍亂之萌也【事見六十九卷魏文帝黄初三年】陛下奈何以高祖太宗之天下不傳之子孫而委之天后乎中書侍郎昌樂李義琰曰處俊之言至忠陛下宜聽之【樂音洛】上乃止 天后多引文學之士著作郎元萬頃【唐著作郎從五品上掌修撰碑誌祝文祭文屬袐書省】左史劉禕之等【龍朔改起居郎為左史】使之撰列女傳臣軌百僚新戒樂書凡千餘卷【撰士免翻】朝廷奏議及百司表疏時密令參决以分宰相之權【疏所去翻】時人謂之北門學士【不從南衙於北門出入故云然】禕之子翼之子也【劉子翼仕隋以學行著】 夏四月庚辰以司農少卿韋弘機為司農卿弘機兼知東都營田受詔完葺宫苑有宦者於苑中犯法弘機杖之然後奏聞上以為能賜絹數十匹曰更有犯者卿即杖之不必奏也 初左千牛將軍長安趙瓌尚高祖女常樂公主生女為周王顯妃公主頗為上所厚天后惡之【樂音洛惡烏路翻】辛巳妃坐廢幽閉於内侍省食料給生者防人候其突烟而已數日烟不出開視死腐矣瓌自定州刺史貶栝州刺史令公主隨之官仍絶其朝謁【朝直遥翻】 太子弘仁孝謙謹上甚愛之禮接士大夫中外屬心【屬之欲翻】天后方逞其志太子奏請數迕旨【迕逆也數所角翻迕五故翻】由是失愛於天后義陽宣城二公主蕭淑妃之女也坐母得罪幽于掖庭年踰三十不嫁太子見之驚惻遽奏請出降上許之天后怒即日以公主配當上翊衛權毅王遂古【親勲翊三衛皆當上上時掌翻】己亥太子薨于合璧宫時人以為天后酖之也 【考異曰新書本紀云己亥天后殺皇太子新傳云后將逞志弘奏請數怫旨從幸合璧宫遇酖薨唐歷云弘仁孝英果深為上所鍾愛自升為太子敬禮大臣鴻儒之士未嘗居有過之地以請嫁二公主失愛於天后不以夀終實錄舊傳皆不言弘遇酖按李泌對肅宗云高宗有八子睿宗最幼天后所生四子自為行第故睿宗第四長曰孝敬皇帝為太子監國仁明孝悌天后方圖臨朝乃酖殺孝敬立雍王賢為太子新書蓋據此及唐歷也按弘之死其事不明今但云時人以為天后酖之也疑以傳疑】壬寅車駕還洛陽宫五月戊申下詔朕方欲禪位皇太子而疾遽不起宜申往命加以尊名可諡為孝敬皇帝【帝子諡皇帝始此諡神至翻】六月戊寅立雍王賢為皇太子赦天下【雍於用翻】 天后惡慈州刺史杞王上金【宋白曰慈州春秋廧咎如之國漢為北屈縣隋為汾州唐武德元年為西汾州五年改南汾州貞觀八年改南汾州為慈州以州近慈烏戍故名孫愐曰因慈氏縣名之上金帝後宫劉氏所生惡烏路翻】有司希旨奏其辠秋七月上金坐解官澧州安置 八月庚寅葬孝敬皇帝于恭陵【恭陵在洛州緱氏縣懊來山改名太平山】 戊戍以戴至德為右僕射庚子以劉仁軌為左僕射並同中書門下三品如故張文瓘為侍中郝處俊為中書令李敬玄為吏部尚書兼左庶子同中書門下三品如故劉仁軌戴至德更日受牒訴【尚辰羊翻更工衡翻】仁軌常以美言許之至德必據理難詰【難乃旦翻詰去吉翻】未嘗與奪實有寃結者密為奏辯【為于偽翻】由是時譽皆歸仁軌或問其故至德曰威福者人主之柄人臣安得盗取之上聞深重之有老嫗欲詣仁軌陳牒誤詣至德至德覽之未終嫗曰本謂是解事僕射乃不解事僕射邪【嫗威遇翻解戶買翻】歸我牒至德笑而授之時人稱其長者【長知兩翻】文瓘時兼大理卿囚聞改官皆慟哭文瓘性嚴正諸司奏議多所糾駁上甚委之【駁比角翻】<br />
<br />
  儀鳳元年【是年十一月方改元】春正月壬戌徙冀王輪為相王納州獠反【唐制儀鳳二年開山洞置納州屬盧州都督府獠魯皓翻】敕黔州都督發兵討之【黔音琴】 二月甲戌徙安東都護府於遼東故城【考異曰實録咸亨元年楊昉高侃討安舜始拔安東都護府自平壤城移於遼東州儀鳳元年二月甲戌以】<br />
<br />
  【高麗餘衆反叛移安東都護府於遼東城蓋咸亨元年言移府者終言之也儀鳳元年言高麗反者本其所以移也會要無咸亨元年移府事此年云移於遼東故城今從之】先是有華人任東官者悉罷之【先悉薦翻】徙熊津都督府於建安故城其百濟戶口先徙於徐兖等州者皆置於建安天后勸上封中嶽癸未詔以今冬有事于嵩山【中嶽嵩山在河南陽城縣】 丁亥上幸汝州之温湯【汝州梁縣西南五十里有温湯可以熟米又有黄女煬帝置温泉頓】 三月癸卯黄門侍郎來中書侍郎薛元超並同中書門下三品濟之兄元超收之子也【來濟盡忠而又死於封疆薛收事太宗於濳躍戶登翻】 甲辰上還東都【還從宣翻又音如字】 閏月吐蕃寇鄯廓河芳等州【宋白曰疊州常芬縣舊為吐谷渾所據周武成三年逐諸羌始有其地乃於三交築城置甘松防又為三州縣以隸常香郡建德三年改三川為常芬縣仍立芳州以邑隸焉取地多芳草以名州隋廢州唐復置吐從暾入聲鄯時戰翻】敕左監門衛中郎將令狐智通發興鳳等州兵以禦之【興州漢武都沮縣地後魏改為略陽縣江左為武興藩王國後魏以為武興郡置興州改略陽縣為順政縣鳳州漢故道河池縣地晉為仇池氐所據後魏置梁泉縣西魏廢帝置鳳州鄯時戰翻監古衘翻將即亮翻】己卯詔以吐蕃犯塞停封中嶽乙酉以洛州牧周王顯為洮州道行軍元帥將工部尚書劉審禮等十二摠管并州大都督相王輪為凉州道行軍元帥將左衛大將軍契苾何力等以討吐蕃【洮土刀翻帥所類翻將即亮翻并卑經翻契欺訖翻苾毗必翻】二王皆不行 庚寅車駕西還 甲寅中書侍郎李義琰同中書門下三品 戊午車駕至九成宫 六月癸亥黄門侍郎晉陵高智周同中書門下三品 秋八月乙未吐蕃寇疊州 壬寅敕桂廣交黔等都督府比來注擬上人簡擇未精自今每四年遣五品已上清正官充使仍令御史同往注擬時人謂之南選【黔音琴比毗至翻使疏吏翻令力丁翻選須絹翻】 九月壬申大理奏左威衛大將軍權善才左監門中郎將范懷義誤斫昭陵柏辠當除名【監古衘翻將即亮翻】上特命殺之大理丞太原狄仁傑奏二人辠不當死【太原漢晉陽縣隋改為太原縣而以後齊所置龍山縣為晉陽縣並帶并】州上曰善才等斫陵柏我不殺則為不孝仁傑固執不已上作色令出仁傑曰犯顏直諫自古以為難臣以為遇桀紂則難遇堯舜則易【易以豉翻】今法不至死而陛下特殺之是法不信於人也人何所措其手足且張釋之有言設有盗長陵一抔土陛下何以處之【釋之言見十四卷漢文帝三年抔薄侯翻處昌呂翻】今以一株柏殺二將軍後代謂陛下為何如矣臣不敢奉詔者恐陷陛下於不道且羞見釋之於地下故也上怒稍解二人除名流嶺南後數日擢仁傑為侍御史初仁傑為并州法曹同僚鄭崇質當使絶域【使疏吏翻】崇質母老且病仁傑曰彼母如此豈可使之有萬里之憂詣長史藺仁基請代之行【長知兩翻藺良刃翻】仁基素與司馬李孝廉不叶因相謂曰吾輩豈可不自愧乎遂相與輯睦 冬十月車駕還京師【還從宣翻又音如字】 丁酉祫享太廟用太學博士史璨議禘後三年而祫祫後二年而禘【歐陽脩曰禘祫大祭也祫以昭穆合食於太祖而禘以審諦其尊卑此禘祫之義而為禮者失之至於年數不同祖宗失位議者莫知所從禮曰三年一祫五年一禘傳曰五年再殷祭高宗上元三年十月當祫而有司議其年數史璨等議以為新君喪畢而祫明年而禘自是之後五年而再祭蓋後禘去前禘五年而祫常在禘後三年禘常在祫後二年魯宣公八年禘僖公蓋三年喪畢而祫明年而禘至八年而再禘昭公二十年禘至二十五年又禘此可知也時以其言有經據遂從之唐制國子博士正五品上掌教文武官二品已上國公子孫二品已上曾孫為生者太學博士正六品上掌教文武官五品已上郡縣公子孫從三品曾孫之為生者祫胡夾翻璨倉按翻禘大計翻】 郇王素節蕭淑妃之子也【郇音荀】警敏好學天后惡之【好呼到翻惡烏路翻】自岐州刺史左遷申州刺史【申州漢平氏鄳縣地晉分置義陽郡南齊置司州後魏改郢州後周改申州隋改義州唐復曰申州岐州在京師西三百十五里至東都一千一百七十里申州至京師一千七百九十六里東都九百四十三里】乾封初敕曰素節既有舊疾不須入朝【朝直遥翻】而素節實無疾自以久不得入覲乃著忠孝論王府倉曹參軍張柬之因使濳封其論以進后見之誣以贓賄丙午降封鄱陽王袁州安置【東之封論以進欲以感動帝心豈知適所以速素節之罪乎袁州在京師東南三千五百八十里至東都二千一百六十一里】 十一月壬申改元赦天下 庚寅以李敬玄為中書令 十二月戊午以來為河南道大使薛元超為河北道大使尚書左丞鄢陵崔知悌國子司業鄭祖玄為江南道大使【隋大業三年始置國子司業唐從四品下所職與祭酒同使疏吏翻鄢謁晩翻又於建翻又音偃】分道廵撫<br />
<br />
  二年春正月乙亥上耕籍田 初劉仁軌引兵自熊津還【見上麟德二年】扶餘隆畏新羅之逼不敢留尋亦還朝【朝直遥翻】二月丁巳以工部尚書高藏為遼東州都督封朝鮮王【朝音潮鮮音仙】遣歸遼東安輯高麗餘衆高麗先在諸州者皆遣與藏俱歸又以司農卿扶餘隆為熊津都督封帶方王亦遣歸安輯百濟餘衆仍移安東都護府於新城以統之【去年春移安東都護府於遼東故城今又移於新城統他綜翻】時百濟荒殘命隆寓居高麗之境藏至遼東謀叛潜與靺鞨通召還徙卭州而死【麗力知翻還從宣翻又音如字靺鞨音末曷卭渠容翻】散徙其人於河南隴右諸州貧者留安東城傍高麗舊城没於新羅餘衆散入靺鞨及突厥【厥九勿翻】隆亦竟不敢還故地高氏扶餘氏遂亡 三月癸亥朔以郝處俊高智周並為左庶子李義琰為右庶子【郝呼各翻處昌呂翻唐制東宫左右庶子各二員】夏四月左庶子張大安同中書門下三品大安公謹之子也【張公謹太宗朝功臣】 詔以河南北旱遣御史中丞崔謐等分道存問賑給侍御史寧陵劉思立上疏【寧陵縣漢屬陳留郡曹魏至元魏屬譙郡後齊廢隋開皇六年復置屬宋州賑津忍翻上時掌翻疏所據翻】以為今麥秀蠶老農事方殷敕使撫廵人皆竦抃忘其家業冀此天恩聚集參迎妨廢不少【使疏吏翻少詩沼翻】既緣賑給須立簿書本欲安存更成煩擾望且委州縣賑給待秋務閑出使褒貶疏奏謐等遂不行【使疏吏翻】 五月吐蕃寇扶州之臨河鎮擒鎮將杜孝昇令齎書說松州都督武居寂使降孝昇固執不從吐蕃軍還捨孝昇而去孝昇復帥餘衆拒守【吐從瞰入聲將即亮翻說輸芮翻降戶江翻復扶又翻帥讀曰率】詔以孝昇為遊擊將軍【晉官品令游擊將軍第四品唐從五品下】 秋八月徙周王顯為英王更名哲【更工衡翻】 命劉仁軌鎮洮河軍【鄯州城内有臨洮軍洮土力翻】冬十二月乙卯詔大發兵討吐蕃 詔以顯慶新禮多不師古【顯慶三年行新禮見二百卷】其五禮並依周禮行事自是禮官益無憑守每有大禮臨時撰定【撰士免翻】<br />
<br />
  三年春正月辛酉百官及蠻夷酋長朝天后于光順門【酋慈由翻長知兩翻朝直遥翻】 劉仁軌鎮洮河每有奏請多為李敬玄所抑由是怨之仁軌知敬玄非將帥才欲中傷之【仁軌以私怨奏用敬玄以至敗國殄民矯情以容袁異式挾怨以䧟李敬玄得為賢乎將即亮翻帥所類翻中竹仲翻】奏言西邊鎮守非敬玄不可敬玄固辭上曰仁軌須朕朕亦自往卿安得辭丙子以敬玄代仁軌為洮河道大摠管兼安撫大使仍檢校鄯州都督【使疏吏翻鄯時戰翻 考異曰實錄云與仁軌相知鎮守而敬玄之敗仁軌不預新舊傳皆云以代仁軌今從之】又命益州大都督府長史李孝逸等發劒南山南兵以赴之孝逸神通之子也【淮安王神通】癸未遣金吾將軍曹懷舜等分往河南北募猛士不問布衣及仕宦 夏四月戊申赦天下改來年元為通乾 五月壬戌上幸九成宫丙寅山中雨大寒從兵有凍死者【從音才用翻】 秋七月李敬玄奏破吐蕃於龍支【龍支縣屬鄯州漢允吾縣地後漢改為龍耆縣後魏改為金城縣又改為龍支積石山在今縣南】 上初即位不忍觀破陳樂命撤之辛酉太常少卿韋萬石奏久寢不作愳成廢缺請自今大宴會復奏之上從之【陳讀曰陣少詩照翻復扶又翻又如字】 九月辛酉車駕還京師 上將發兵討新羅侍中張文瓘臥疾在家自輿入見【見賢遍翻】諫曰今吐蕃為寇方發兵西討新羅雖云不順未嘗犯邊若又東征臣恐公私不勝其弊上乃止【勝音升】癸亥文瓘薨 丙寅李敬玄將兵十八萬與吐蕃將論欽陵戰於青海之上【將即亮翻下同】兵敗工部尚書右衛大將軍彭城僖公劉審禮為吐蕃所虜【諡法剛克曰僖又小心畏忌曰僖又戴記有伐有還曰僖】時審禮將前軍深入頓于濠所為虜所攻敬玄懦怯按兵不救聞審禮戰没狼狽還走頓于承風嶺【杜佑曰承風嶺在廓州廣威縣西南東北去鄯州三百一十三里故吐谷渾界】阻泥溝以自固虜屯兵高崗以壓之左領軍員外將軍黑齒常之夜帥敢死之士五百人襲擊虜營虜衆潰亂其將跋地設引兵遁去【帥讀曰率將即亮翻】敬玄乃收餘衆還鄯州 【考異曰朝野僉載曰中書令李敬玄為元帥吐蕃至樹敦城聞劉尚書没蕃著鞾不得狼狽而走遺却麥飯首尾千里地上尺餘言之大過今不取】審禮諸子自縳詣闕請入吐蕃贖其父敕聽次子易從詣吐蕃省之比至審禮已病卒【易以豉翻省悉景翻比必利翻卒子恤翻 考異曰新本紀審禮死之按舊傳永隆二年審禮卒于蕃中新紀誤也今終言之】易從晝夜號哭不絶聲【號戶高翻】吐蕃哀之還其尸易從徒跣負之以歸上嘉黑齒常之之功擢拜左武衛將軍充河源軍副使【杜佑曰河源軍在鄯州西一百二十里使疏吏翻下同】李敬玄之西征也監察御史原武婁師德應猛士詔從軍【時詔募猛士討吐蕃師德應募從軍監古衘翻】及敗敕師德收集散亡軍乃復振【復扶又翻又如字】因命使于吐蕃吐蕃將論贊婆迎之赤嶺【宋白曰石堡城西三十里有山土石皆赤北接大山南連小雪山號曰赤嶺去長安三千五百里自鄯州鄯城縣西行二百里至赤嶺將即亮翻】師德宣導上意諭以禍福贊婆甚悦為之數年不犯邊【為于偽翻】師德遷殿中侍御史充河源軍司馬【考異曰御史臺記充河源軍使今從舊傳】兼知營田事上以吐蕃為憂悉召侍臣謀之或欲和親以息民或欲嚴設守備俟公私富實而討之或欲亟發兵擊之議竟不决賜食而遣之大學生宋城魏元忠上封事【宋城縣帶宋州舊睢陽縣也隋開皇十八年更名上時掌翻】言禦吐蕃之策以為理國之要在文與武今言文者則以辭華為首而不及經綸言武者則以騎射為先而不及方略【騎奇寄翻】是皆何益於理亂哉故陸機著辨亡之論無救河橋之敗【陸機痛吳之亡著辯亡論迹吳之所以興及其所以亡其論甚悉河橋之敗見八十五卷晉惠帝太安二年】養由基射穿七札不濟鄢陵之師【左傳晉楚遇於鄢陵楚大夫養由基潘黨蹲甲而射之徹七札焉以示楚共王曰君有二臣何憂於戰王怒曰大辱國詰朝爾射死藝及戰楚師敗績杜預曰濟益也】此己然之明效也古語有之人無常俗政有理亂兵無彊弱將有巧拙故選將當以智略為本勇力為末今朝廷用人類取將門子弟及死事之家【將即亮翻下同】彼皆庸人豈足當閫外之任李左車陳湯呂蒙孟觀皆出貧賤而立殊功【李左車見十卷漢高帝三年陳湯見二十九卷元帝建昭三年呂蒙見獻帝紀六十五卷至六十八卷孟觀見八十三卷晉惠帝元康九年】未聞其家代為將也夫賞罸者軍國之切務苟有功不賞有辠不誅雖堯舜不能以致理議者皆云近日征伐虚有賞格而無事實蓋由小才之吏不知大體徒惜勲庸恐虛倉庫不知士不用命所損幾何黔首雖微不可欺罔豈得懸不信之令設虛賞之科而望其立功乎自蘇定方征遼東【見二百卷龍朔元年二年】李勣破平壤【見上卷總章元年】賞絶不行勲仍淹滯不聞斬一臺郎戮一令史以謝勲人【尚書謂曹郎皆謂之臺郎勲轉淹滯則司勲之責耳司勲令史三十三人】大非川之敗薛仁貴郭待封等不即重誅【見上卷咸亨元年即就也】曏使早誅仁貴等則自餘諸將豈敢失利於後哉臣恐吐蕃之平非旦夕可冀也又出師之要全資馬力臣請開畜馬之禁使百姓皆得畜馬【畜吁玉翻】若官軍大舉委州縣長吏以官錢增價市之則皆為官有彼胡虜恃馬力以為彊若聽人間市而畜之乃是損彼之彊為中國之利也先是禁百姓畜馬【長知兩翻畜吁玉翻先悉薦翻】故元忠言之上善其言召見令直中書省仗内供奉【仗内供奉朝會得隨百官入見】 冬十月丙午徐州刺史密貞王元曉薨 十一月壬子黄門侍郎同中書門下三品來薨【戶登翻】 十二月詔停來年通乾之號以反語不善故也【通乾反語為天窮】<br />
<br />
  調露元年【按會要是年六月十三日方改元調露】春正月己酉上幸東都司農卿韋弘機作宿羽高山上陽等宫【按六典宿羽高山二宫皆在東都禁苑中】制度壯麗上陽宫臨洛水為長廊亘一里宫成上徙御之侍御史狄仁傑劾奏弘機導上為奢泰弘機坐免官【劾戶槩翻又戶得翻 考異曰舊傳云儀鳳中機坐家人犯盗為憲司所劾免官狄仁傑傳云時司農卿韋機兼領將作少府造宿羽高山上陽等宫莫不壯麗仁傑奏其太過機竟坐免官統紀云駕幸東都上遊韋弘機所造宿羽高山等宫乘高臨深有登眺之矣乃即敕弘機造高館及成臨幸即上陽宫也今據實録營宫在前】左司郎中王本立【隋煬帝大業三年尚書都司始置左右司郎各一人掌都省之職唐置左右司郎中各掌付十有二司之事以舉正稽違省置符目】恃恩用事朝廷畏之仁傑奏其姧請付法司上特原之仁傑曰國家雖乏英才豈少本立輩【少詩沼翻】陛下何惜罪人以虧王法必欲曲赦本立請弃臣於無人之境為忠貞將來之戒本立竟得罪 【考異曰御史臺記曰狄仁傑以司農發太原運句會欠米萬餘斛高宗怒曰仁傑偷我米命殺之吏部侍郎魏玄同曰仁傑健而踈只是句當失所臣委知不偷請以官爵保明高宗意解仁傑不坐案仁傑傳未嘗為司農今不取】由是朝廷肅然 庚戌右僕射太子賓客道恭公戴至德薨【道古國名左傳之江黄道栢是也顯慶元年始置太子賓客四員正三品掌侍從規諫贊相禮儀諡法尊賢讓善曰恭】 二月壬戌吐蕃贊普卒子器弩悉弄立生八年矣時器弩悉弄與其舅麴薩若詣羊同發兵【羊同西戎國名宋祈載劉元鼎之言曰黄河上流由洪濟橋西南行二千里水益狹春可涉秋夏乃勝舟其南三百里三山中高而四下曰紫山直大羊同國古所謂崑崙者也虜曰悶摩黎山東距長安五千里河源出其間唐會要曰大羊同國東接吐蕃西接小羊同北直于闐東西千里薩桑割翻】有弟生六年在論欽陵軍中國人畏欽陵之彊欲立之欽陵不可與薩若共立器弩悉弄上聞贊普卒命裴行儉乘閒圖之【卒子恤翻間古莧翻】行儉曰欽陵為政大臣輯睦未可圖也乃止 夏四月辛酉郝處俊為侍中 偃師人明崇儼以符呪幻術為上及天后所重【呪職救翻幻戶辨翻】官至正諫大夫五月壬午崇儼為盗所殺求賊竟不得 【考異曰御史臺記鄭仁恭本榮陽人也自監察累遷刑部郎中儀鳳中明崇儼以奇術承恩寵夜遇刺客敕三司亟推鞫妄承引連坐者甚衆高宗怒促有司行刑仁恭奏曰此輩必死之囚願假其數日之命高宗曰卿以為枉邪仁恭曰臣識慮淺短非的以為枉恐萬 非實則怨氣生遂緩之旬餘果獲賊矣朝廷稱之今從實録】贈崇儼侍中丙戌命太子監國【監古衘翻】太子處事明審【處昌呂翻】時人稱<br />
<br />
  之 戊戌作紫桂宫於澠池之西【紫桂宫在澠池之西五里澠彌兗翻】六月卒亥赦天下改元 初西突厥十姓可汗阿史那都支及其别帥李遮匐與吐蕃連和侵逼安西【帥所類翻匐蒲北翻】朝議欲發兵討之吏部侍郎裴行儉曰吐蕃為寇審禮覆没干戈未息豈可復出師西方今波斯王卒其子泥洹師為質在京師【波斯為大食所滅其王卑路斯入朝授武衛將軍而死其子為質在京師朝直遥翻復扶又翻卒子恤翻洹戶官翻質音致考異曰實録作泥浬師按舊傳作泥湟師唐歷作泥洹師今從統紀】宜遣使者送歸國【使疏吏翻】道過二虜以便宜取之可不血刃而擒也上從之命行儉冊立波斯王【過工禾翻 考異曰波斯王卑路斯入朝未還請遣使送歸今從實録唐歷統紀舊傳】仍為安撫大食使行儉奏肅州刺史王方翼以為已副仍令檢校安西都護【使疏吏翻令力丁翻】 秋七月己卯朔詔以今年冬至有事于嵩山 初裴行儉嘗為西州長史【見一百九十九卷永徽五年長知兩翻】及奉使過西州吏人郊迎行儉悉召其豪傑子弟千餘人自隨且揚言天時方熱未可涉遠須稍凉乃西上【上時掌翻】阿史那都支覘知之遂不設備【覘丑亷翻又丑艶翻】行儉徐召四鎮諸胡酋長【四鎮龜兹毗沙焉耆踈勒四都督府也酋慈由翻】謂曰昔在西州縱獵甚樂【樂音洛】今欲尋舊賞誰能從吾獵者諸胡子弟爭請從行近得萬人行儉陽為畋獵校勒部伍【校古効翻】數日遂倍道西進去都支部落十餘里先遣都支所親問其安否外示閒暇似非討襲續使促召相見都支先與李遮匐約秋中拒漢使【漢家威加四夷故夷人率謂中國人為漢人猶漢時匈奴謂漢人為秦人也使疏吏翻下同】猝聞軍至計無所出帥其子弟迎謁遂擒之【帥讀曰率】因傳其契箭悉召諸部酋長【夷狄無符信以箭為契信西突厥沙鉢咥利失可汗分其國為十部部以一人統之人授一箭號十設亦曰十箭左五咄陸部置五大啜居碎葉東右五弩失畢部置五大俟斤居碎葉西】執送碎葉城簡其精騎輕齎晝夜進掩遮匐途中獲都支還使與遮匐使者同來行儉釋遮匐使者使先往諭遮匐以都支已就擒遮匐亦降【騎奇寄翻齎則兮翻匐蒲北翻降戶江翻】於是囚都支遮匐以歸遣波斯王自還其國留王方翼於安西使築碎葉城【碎葉城焉耆都督府治所也方翼築四面十二門為屈曲隱出伏没之狀】 冬十月單于大都護府突厥阿史德温傳奉職二部俱反【阿史德姓也温傳其名奉職亦一部酋長之名單音蟬厥九勿翻】立阿史那泥熟匐為可汗二十四州酋長皆叛應之衆數十萬【置二十四州見一百九十九卷永徽元年】遣鴻臚卿單于大都護府長史蕭嗣業右領軍衛將軍花大智【何承天姓苑有花姓臚陵如翻單音蟬長知兩翻嗣祥吏翻】右千牛衛將軍李景嘉等將兵討之【將兵即亮翻又音如字】嗣業等先戰屢捷因不設備會大雪突厥夜襲其營嗣業狼狽拔營走衆遂大亂為虜所敗死者不可勝數【厥九勿翻敗蒲邁翻勝音升】大智景嘉引步兵且行且戰得入單于都護府嗣業減死流桂州大智景嘉並免官突厥寇定州刺史霍王元軌命開門偃旗虜疑有伏愳而宵遁州人李嘉運與虜通謀事洩【洩息列翻】上令元軌窮其黨與元軌曰彊寇在境人心不安若多所逮繫是驅之使叛也乃獨殺嘉運餘無所問因自劾違制上覽表大喜謂使者曰朕亦悔之向無王失定州矣【劾戶槩翻又戶得翻使疏吏翻】自是朝廷有大事上多密敕問之【朝直遥翻】 壬子遣左金吾衛將軍曹懷舜屯井陘【井陘縣漢晉後魏皆属常山郡唐属恒州陘音刑】右武衛將軍崔獻屯龍門以備突厥突厥扇誘奚契丹侵掠營州【誘羊久翻契欺訖翻又音喫】都督周道務遣戶曹始平唐休璟將兵擊破之【曹魏置始平縣晉武帝置始平郡後魏廢郡以縣屬扶風隋唐屬雍州】 庚申詔以突厥背誕罷封嵩山【背蒲妹翻】 癸亥吐蕃文成公主遣其大臣論塞調傍來告喪并請和親上遣郎將宋令文詣吐蕃會贊普之葬【將即亮翻下管將同】 十一月戊寅朔以太子左庶子同中書門下三品高智周為御史大夫罷知政事 癸未上宴裴行儉謂之曰卿有文武兼資今授卿二職乃除禮部尚書兼檢校右衛大將軍甲辰以行儉為定襄道行軍大摠管將兵十八萬并西軍檢校豐州都督程務挺東軍幽州都督李文暕【暕古限翻】摠三十餘萬以討突厥並受行儉節度務挺名振之子也【程名振為將著功名於貞觀永徽之間】<br />
<br />
  永隆元年【按會要是年八月二十三日改元永隆】春二月癸丑上幸汝州之温湯戊午幸嵩山處士三原田遊巖所居【處昌呂翻】己未幸道士宗城潘師正所居 【考異曰舊傳師正趙州贊皇人今從實録】上及天后太子皆拜之【史言帝崇信異端】乙丑還東都 三月裴行儉大破突厥于黑山【黑山一名殺胡山在豐州中受降城正北如東八十里亦謂之呼延谷】禽其酋長奉軄【句絶】可汗泥熟匐為其下所殺以其首來降【降戶江翻】初行儉行至朔川 【考異曰舊傳作朔州今依實録及統紀 余按唐朔州治善陽縣漢定襄縣地單于府治金河縣漢雲中郡城也自朔州至單于府三百五十七里以裴行儉軍行次舍考之先至朔州而後至單于府北則舊傳朔州為是】謂其下曰用兵之道撫士貴誠制敵貴詐前日蕭嗣業粮運為突厥所掠士卒凍餒故敗今突厥必復為此謀【復扶又翻下復為可復同】宜有以詐之乃詐為粮車三百乘每車伏壯士五人各持陌刀勁弩【陌刀大刀也一舉刀可殺數人唐六典曰陌刀長刀也步兵所持蓋古之斬馬劒釋名曰弩怒也有怒勢也其柄曰臂似人臂也鈎弦者曰牙似牙齒也牙外曰郭為牙之規郭也合名之曰機言如機之巧也亦言如門戶樞機開闔有節也乘繩證翻】以羸兵數百為之援【羸倫為翻】且伏精兵於險要以待之虜果至羸兵弃車散走虜驅車就水屮解鞍牧馬欲取糧壯士自車中躍出擊之虜驚走復為伏兵所邀殺獲殆盡自是糧運行者虜莫敢近【近其靳翻】軍至單于府比抵暮下營掘壍己周【掘其月翻塹七艷翻】行儉遽命移就高岡諸將皆言士卒已安堵不可復動行儉不從趣使移【趣讀曰促】是夜風雨暴至前所營地水深丈餘【深式禁翻】諸將驚服問其故行儉笑曰自今但從我命不必問其所由知也奉軄既就禽餘黨走保狼山【狼山歌邏禄右廂部落所居也永徽元年置狼山州属雲中都護府】詔戶部尚書崔知悌馳傳詣定襄宣慰將士且區處餘寇【傳知戀翻處昌呂翻】行儉引軍還 夏四月乙丑上幸紫桂宫 戊辰黄門侍郎聞喜裴炎崔知温【聞喜縣漢屬河東郡隋以漢聞喜縣為絳縣以漢絳縣為曲沃縣以桐鄉置聞喜縣尋改為桐鄉縣武德元年復曰聞喜屬絳州】中書侍郎京兆王德真並同中書門下三品知温知悌之弟也 秋七月吐蕃寇河源左武衛將軍黑齒常之擊却之 【考異曰實録吐蕃大將贊婆及素和貴等帥衆三萬進寇河源屯兵于良非川辛巳河西鎮撫大使李敬玄統衆與賊戰于湟川官軍敗績副使左武衛將軍黑齒常之帥精騎三千夜襲賊營殺獲二千餘級贊婆等遂退擢常之為河源軍經略大使詔敬玄留鎮鄯州以為之援按儀鳳三年九月敬玄已與吐蕃戰敗于青海常之夜襲虜營賊乃退與此事頗相類舊書敬玄傳止一敗無再敗常之傳儀鳳中從敬玄擊吐蕃走跋地設充河源軍副使時贊婆等屯良非川常之夜襲賊營走之擢為大使事似同時新書敬玄傳戰青海又戰湟川凡再敗常之傳儀鳳三年襲跋地設調露中襲贊婆唐歷統紀皆無今年敬玄敗事又實録今年八月丁巳敬玄貶衡州刺史辛巳至丁巳纔三十七日賈耽皇華四逹記自長安至鄯州約一千七百餘里時高宗又在東都若敬玄敗後累表稱疾得報乃來至東都必數日乃貶非三十七日之内所能容也今略去敬玄湟川敗事但云吐蕃寇河源常之擊却之而已今按劉㫬唐書地理志鄯州在京師西一千九百一十三里】擢常之為河源軍經略大使常之以河源衝要欲加兵戌之而轉輸險遠乃廣置烽戍七十餘所開屯田五千餘頃歲收五百餘萬石由是戰守有備焉先是劒南募兵於茂州西南築安戎城以斷吐蕃通蠻之路【宋白曰茂州本冉駹之國漢開為汶山郡華陽國志云宣帝地節三年武都白馬羌反使駱武平定之汶山吏民詣武自訟一歲再度更賦至重邊人貧苦無以供給求省郡遂省汶山郡復置都尉今州即漢蜀郡汶江縣梁普通三年置繩州取桃關之路以繩作橋以名州後周為汶州置汶山縣唐改茂州取界内茂滋山而名】吐蕃以生羌為鄉導【先悉薦翻斷丁管翻鄉讀曰嚮】攻陷其城以兵據之由是西洱諸蠻皆降于吐蕃【洱而志翻降戶江翻】吐蕃盡據羊同党項及諸羌之地東接凉松茂嶲等州南鄰天笁西陷龜兹疎勒等四鎮【党底朗翻嶲音髓龜兹音丘慈】北抵突厥地方萬餘里諸胡之盛莫與為比 丙申鄭州刺史江王元祥薨【鄭州漢滎陽縣地漢滎陽属河南郡晉分為滎陽郡後魏屬北豫州後置鄭州隋開皇十六年改曰管州大業初復曰鄭州】 突厥餘衆圍雲州【雲州漢平城縣地後魏為代都北齊及後周為恒安鎮貞觀七年置雲州及定襄縣】代州都督竇懷悊【悊與哲同】右領軍中郎將程務挺將兵擊破之【將即亮翻】 八月丁未上還東都中書令檢校鄯州都督李敬玄軍既敗屢稱疾請還<br />
<br />
  上許之既至無疾詣中書視事上怒丁巳貶衡州刺史【衡州漢酃縣蒸陽耒陽茶陵縣地吳置湘東郡梁陳置衡山郡隋平陳置衡州京師東南三千四百三里至東都二千七百六十里】 太子賢聞宫中竊議以賢為天后姊韓國夫人所生内自疑懼明崇儼以厭勝之術為天后所信【厭於葉翻】常密稱太子不堪承繼英王貌類太宗又言相王相最貴【相悉亮翻】天后嘗命北門學士撰少陽正範【顔延之曲水詩序曰正體毓德於少陽注云東宫少陽位也少詩照翻】及孝子傳以賜太子【傳直戀翻】又數作書誚讓之【數所角翻誚才笑翻】太子愈不自安及崇儼死賊不得天后疑太子所為太子頗好聲色與戶奴趙道生等狎昵【好呼到翻昵尼質翻】多賜之金帛司議郎韋承慶上書諫不聽【上時掌翻】天后使人告其事詔薛元超裴炎與御史大夫高智周等襍鞫之於東宫馬坊搜得皂甲數百領以為反具道生又欵稱太子使道生殺崇儼上素愛太子遲回欲宥之天后曰為人子懷逆謀天地所不容大義滅親何可赦也甲子廢太子賢為庶人遣右監門中郎將令狐智通等送賢詣京師【監古銜翻將即亮翻】幽于别所黨與皆伏誅仍焚其甲於天津橋南以示士民【劉昫曰東都周之王城平王東遷所都也故城在今苑内東北隅自赧王以後及東漢魏文晉武皆都於今故洛城隋大業元年自故洛城西移十八里置新都今都城是也北據邙山南對伊闕洛水貫都有河漢之象跨洛為橋曰天津橋唐世人主往來東都西京而實都長安以長安為京師】承慶思謙之子也【韋思謙見一百九十九卷永徽元年】乙丑立左衛大將軍雍州牧英王哲為皇太子【雍於用翻】改元赦天下太子洗馬劉訥言嘗撰俳諧集以獻賢賢敗搜得之上怒曰以六經教人猶恐不化乃進俳諧鄙說豈輔導之義邪流訥言於振州【舊志振州與崖州同在大海洲中至京師八千六百六里至東都七千七百九十七里洗悉薦翻撰士免翻俳蒲皆翻 考異曰新傳云除名為民復坐事流死振州今從實録】左衛將軍高真行之子政為太子典膳丞事與賢連上以付其父使自訓責政入門真行以佩刀刺其喉真行兄戶部侍郎審行又刺其腹真行兄子琁斷其首弃之道中【刺七亦翻琁從宣翻斷丁管翻】上聞之不悦貶真行為睦州刺史審行為渝州刺史真行士亷之子也【舊志睦州京師東南三千六百五十九里至東都二千八百二十一里渝州漢末之巴東郡隋置渝州京師西南二千七百四十八里至東都三千四百三十里高士廉長孫无忌之舅】左庶子中書門下三品張大安坐阿附太子左遷普州刺史【舊志普州至京師二千三百六十里至東都三千二百三里】其餘宫僚上皆釋其罪使復位左庶子薛元超等皆舞蹈拜恩右庶子李義琰獨引咎涕泣時論美之 九月甲申以中書侍郎同中書門下三品王德真為相王府長史罷政事【唐制王府置長史司馬長史從四品上司馬從四品下長史司馬統領府僚紀綱職務相悉亮翻】 冬十月壬寅蘇州刺史曹王明【蘇州古吳國東漢為吳郡隋置蘇州因姑蘇山而名】沂州刺史嗣蔣王煒【沂州漢琅邪國後魏置北徐州後周改沂州以沂水名隋為琅邪郡唐復以琅邪郡置沂州煒于鬼翻】皆坐故太子賢之黨明降封零陵郡王黔州安置煒除名道州安置【道州漢零陵郡冷道馮乘之地隋以零陵郡置永州武德四年分置營州貞觀八年改曰道州舊志黔州京師南三千一百九十三里至東都二千二百七十七里】 丙午文成公主薨于吐蕃 己酉車駕西還 十一月壬申朔日有食之開耀元年【是年十月方改元新書作九月】春正月突厥寇原慶等州乙亥遣右衛將軍李知十等屯涇慶二州以備突厥 庚辰以初立太子敕宴百官及命婦於宣政殿【西京東内正殿曰含元後殿曰宣政】引九部伎及散樂自宣政門入【杜佑曰散樂即百戱也伎渠綺翻散悉亶翻】太常博士袁利貞上疏以為正寢非命婦宴會之地路門非倡優進御之所【上時掌翻倡音昌】請命婦會於别殿九部伎自東西門入其散樂伏望停省上乃更命置宴於麟德殿【麟德殿麟德中所作也閣本大明宫圖翰林院密邇麟德殿韋執誼曰翰林院在右銀臺門内麟德殿在西重廊之後更工衡翻】宴日賜利貞帛百段利貞昂之曾孫也利貞族孫誼為蘇州刺史自以其先自宋太尉淑以來盡忠帝室【袁淑死於宋元凶之難袁顗以死奉子勛袁昂盡節於齊室袁憲盡忠於陳後主】謂琅邪王氏雖奕世台鼎而為歷代佐命【琅邪王氏股肱晉室而王弘為宋室佐命王儉為齊室佐命梁室之興侯景之簒王亮王克為勸進之首】耻與為比嘗曰所貴於名家者為其世篤忠貞才行相繼故也【為于偽翻行下孟翻】彼鬻婚姻求禄利者又烏足貴乎時人是其言【因袁利貞併著袁誼之言以其有益於名教也】 裴行儉軍既還突厥阿史那伏念復自立為可汗【復扶又翻杜佑曰伏念頡利從兄之子】與阿史德温傅連兵為寇癸巳以行儉為定襄道大摠管以右武衛將軍曹懷舜幽州都督李文暕為副將兵討之【暕古限翻將即亮翻】 二月天后表請赦杞王上金鄱陽王素節之罪以上金為沔州刺史【沔州漢安陸之地晉置沔陽縣江左為魯山鎮隋開皇十七年置漢陽縣属復州復州治沔陽大業初改復州曰沔州唐復以沔州為復州分漢陽置沔州】素節為岳州刺史【岳州漢下雋縣地吳置巴陵縣晉置建昌郡梁置巴州隋改曰岳州因天岳山以名州大業初改羅州唐復曰岳州舊志岳州京師東南二千二百三十七里至東都一千八百一十六里】仍不聽朝集【朝直遥翻】 三月辛卯以劉仁軌兼太子少傅餘如故【仁軌先為尚書左僕射同中書門下三品】以侍中郝處俊為太子少保罷政事少府監裴匪舒善營利【少詩照翻】奏賣苑中馬糞歲得錢二千萬緡上以問劉仁軌對曰利則厚矣恐後代稱唐家賣馬糞非嘉名也乃止匪舒又為上造鏡殿成【為于偽翻】上與仁軌觀之仁軌驚趨下殿上問其故對曰天無二日土無二王【孟子識孔子之言】適視四壁有數天子不祥孰甚焉上遽令剔去【去羌呂翻】 曹懷舜與禆將竇義昭將前軍擊突厥【將即亮翻下同又音如字】或告阿史那伏念與阿史德温傅在黑沙【黑沙城後突厥點啜以為南庭】左右纔二十騎以下可逕往取也【騎奇寄翻】懷舜等信之留老弱於瓠蘆泊【水白為泊】帥輕鋭倍道進至黑沙無所見人馬疲頓乃引兵還會薛延陁部落欲西詣伏念遇懷舜軍因請降【帥讀曰率降戶江翻】懷舜等引兵徐還至長城北遇温傅小戰各引去至横水【横水去金河百四十許里】遇伏念懷舜義昭與李文暕及禆將劉敬同四軍合為方陳【陳讀曰陣】且戰且行經一日伏念乘便風擊之軍中擾亂懷舜等弃軍走軍遂大敗死者不可勝數【勝音升】懷舜等收散卒歛金帛以賂伏念與之約和殺牛為盟伏念北去懷舜等乃得還【還從宣翻又音如字】夏五月丙戌懷舜免死流嶺南 己丑河源道經略大使黑齒常之將兵擊吐蕃論贊婆於良非川破之收其糧畜而還【畜許救翻】常之在軍七年吐蕃深畏之不敢犯邊 初太原王妃之薨也【武士彠封太原王妃從其爵咸亨元年薨】天后請以太平公主為女官以追福【公主天后女也】及吐蕃求和親請尚太平公主上乃為立太平觀以公主為觀主以拒之【為于偽翻觀古玩翻】至是始選光禄卿汾隂薛曜之子紹尚焉紹母太宗女城陽公主也【據會要城陽公主先降杜荷荷誅降薛瓘新書亦然】秋七月公主適薛氏自興安門南至宣陽坊西燎炬相屬【自興安門而南歷三坊至宣陽坊萬年縣治在焉属之欲翻】夾路槐木多死紹兄顗以公主寵盛深憂之【顗魚豈翻】以問族祖戶部郎中克構【唐戶部郎掌分理戶口井田之事凡天下十道任土所出為貢賦之差】克構曰帝甥尚主國家故事苟以恭慎行之亦何傷然諺曰娶婦得公主無事取官府不得不為之懼也天后以顗妻蕭氏及顗弟緒妻成氏非貴族欲出之曰我女豈可使與田舍女為妯娌邪【為于偽翻妯直六翻娌兩耳翻妯娌娣姒婦也】或曰蕭氏瑀之姪孫國家舊姻【蕭瑀子鋭尚太宗女襄城公主】乃止 夏州羣牧使安元夀奏自調露元年九月以來喪馬一十八萬餘匹監牧吏卒為虜所殺掠者八百餘人【夏戶雅翻使疏吏翻喪息浪翻唐諸牧監掌群牧孳課之事凡諸羣牧立南北東西四使以分統之其馬皆印每歲終監牧使廵案孳數以功過相除為之考課此止夏州所喪失之數】 薛延陁逹渾等五州四萬餘帳來降【逹渾都督領姑衍州步訖若州嵠彈州鶻州低粟州降戶江翻】甲午左僕射兼太子少傅同中書門下三品劉仁軌固請解僕射許之 閏七月丁未裴炎為侍中崔知温薛元超並守中書令 上徵田遊巖為太子洗馬在東宫無所規益右衛副率蔣儼【右衛副率從四品上洗悉薦翻率所律翻】以書責之曰足下負巢由之俊節傲唐虞之聖主聲出區宇名流海内主上屈萬乘之重申三顧之榮【三顧用諸葛亮事上幸嵩山嘗至遊巖所居故云然乘繩證翻】遇子以商山之客【漢四皓隱於商山】待子以不臣之禮將以輔導儲貳漸染芝蘭耳【漸子廉翻】皇太子春秋鼎盛聖道未周僕以不才猶參庭諍足下受調護之寄【漢高帝謂四皓曰煩公卒調護太子】是可言之秋唯唯而無一談悠悠以卒年歲【唯于癸翻卒子恤翻】向使不飡周粟【夷齊採薇西山不食周粟飡千安翻】僕何敢言禄及親矣以何酬塞【遊巖有母塞悉則翻】想為不逹謹書起予【孔子謂子夏曰起予者商也為于偽翻】遊巖竟不能荅 庚申上以服餌令太子監國【監古衘翻】 裴行儉軍于代州之陘口【即鴈門之陘嶺關口陘音刑】多縱反間由是阿史那伏念與阿史德温傅浸相猜貳伏念留妻子輜重於金牙山【突厥之初建牙於金山其後分為東西突厥凡建牙之地率謂之金牙山蘇定方直抵金牙山擒賀魯此西突厥可汗所居之金牙山也裴行儉遣程務挺等掩金牙山取伏念妻子此東突厥可汗所居之金牙山也可汗所居謂之金帳故亦以金牙言之間古莧翻下同重直龍翻下同】以輕騎襲曹懷舜行儉遣禆將何迦密自通漠道程務挺自石地道掩取之【騎奇寄翻將即亮翻下同迦居牙翻又居伽翻】伏念與曹懷舜約和而還比至金牙山【還從宣翻比必利翻】失其妻子輜重士卒多疾疫乃引兵北走細沙行儉又使副摠管劉敬同程務挺等將單于府兵追躡之伏念請執温傅以自效然尚猶豫又自恃道遠唐兵必不能至不復設備【走音奏躡泥輒翻復扶又翻單音蟬】敬同等軍到伏念狼狽不能整其衆遂執温傅從間道詣行儉降候騎告以塵埃漲天而至將士皆震恐行儉曰此乃伏念執温傳來降非它盗也然受降如受敵不可無備乃命嚴備遣單使迎前勞之少選【間古莧翻降戶江翻下同騎奇寄翻使疏吏翻勞力到翻少選猶言少頃也】伏念果帥酋長縛温傅詣軍門請罪【帥讀曰率酋慈由翻長知兩翻】行儉盡平突厥餘黨以伏念温傅歸京師【厥九勿翻】冬十月丙寅朔日有食之 壬戌裴行儉等獻定襄之俘乙丑改元【改元開耀】丙寅斬阿史那伏念阿史德温傅等五十四人于都市【既書十月丙寅朔日食方書壬戍裴行儉獻俘乙丑改元又書丙寅斬阿史那伏念等是十月一月内有二丙寅矣此舊史之誤通鑑因之失於檢點也新書是年九月乙丑改元蓋壬戌獻俘亦九月事前年命行儉為定襄道行軍大總管以討突厥故曰獻定襄之俘】初行儉許伏念以不死故降裴炎疾行儉之功奏言伏念為副將張䖍朂程務挺所逼又迴紇等自磧北南向逼之窮窘而降耳【紇下没翻磧七迹翻】遂誅之行儉歎曰渾濬爭功【事見八十一卷晉武帝太康元年言若爭伏念之死則是與張䖍勗程務挺爭功】古今所耻但恐殺降無復來者【復扶又翻】因稱疾不出 丁亥新羅王法敏卒【卒子恤翻】遣使立其子政明 十一月癸卯徙故太子賢於巴州【舊志巴州至京師二千三百六十里東都二千五百八十二里】<br />
<br />
  資治通鑑卷二百二  <br>
   </div> 

<script src="/search/ajaxskft.js"> </script>
 <div class="clear"></div>
<br>
<br>
 <!-- a.d-->

 <!--
<div class="info_share">
</div> 
-->
 <!--info_share--></div>   <!-- end info_content-->
  </div> <!-- end l-->

<div class="r">   <!--r-->



<div class="sidebar"  style="margin-bottom:2px;">

 
<div class="sidebar_title">工具类大全</div>
<div class="sidebar_info">
<strong><a href="http://www.guoxuedashi.com/lsditu/" target="_blank">历史地图</a></strong>  
<a href="http://www.880114.com/" target="_blank">英语宝典</a>  
<a href="http://www.guoxuedashi.com/13jing/" target="_blank">十三经检索</a> 
<br><strong><a href="http://www.guoxuedashi.com/gjtsjc/" target="_blank">古今图书集成</a></strong> 
<a href="http://www.guoxuedashi.com/duilian/" target="_blank">对联大全</a> <strong><a href="http://www.guoxuedashi.com/xiangxingzi/" target="_blank">象形文字典</a></strong> 

<br><a href="http://www.guoxuedashi.com/zixing/yanbian/">字形演变</a>  <strong><a href="http://www.guoxuemi.com/hafo/" target="_blank">哈佛燕京中文善本特藏</a></strong>
<br><strong><a href="http://www.guoxuedashi.com/csfz/" target="_blank">丛书&方志检索器</a></strong> <a href="http://www.guoxuedashi.com/yqjyy/" target="_blank">一切经音义</a>  

<br><strong><a href="http://www.guoxuedashi.com/jiapu/" target="_blank">家谱族谱查询</a></strong>  <strong><a href="http://shufa.guoxuedashi.com/sfzitie/" target="_blank">书法字帖欣赏</a></strong> 
<br>

</div>
</div>


<div class="sidebar" style="margin-bottom:0px;">

<font style="font-size:22px;line-height:32px">QQ交流群9:489193090</font>


<div class="sidebar_title">手机APP 扫描或点击</div>
<div class="sidebar_info">
<table>
<tr>
	<td width=160><a href="http://m.guoxuedashi.com/app/" target="_blank"><img src="/img/gxds-sj.png" width="140"  border="0" alt="国学大师手机版"></a></td>
	<td>
<a href="http://www.guoxuedashi.com/download/" target="_blank">app软件下载专区</a><br>
<a href="http://www.guoxuedashi.com/download/gxds.php" target="_blank">《国学大师》下载</a><br>
<a href="http://www.guoxuedashi.com/download/kxzd.php" target="_blank">《汉字宝典》下载</a><br>
<a href="http://www.guoxuedashi.com/download/scqbd.php" target="_blank">《诗词曲宝典》下载</a><br>
<a href="http://www.guoxuedashi.com/SiKuQuanShu/skqs.php" target="_blank">《四库全书》下载</a><br>
</td>
</tr>
</table>

</div>
</div>


<div class="sidebar2">
<center>


</center>
</div>

<div class="sidebar"  style="margin-bottom:2px;">
<div class="sidebar_title">网站使用教程</div>
<div class="sidebar_info">
<a href="http://www.guoxuedashi.com/help/gjsearch.php" target="_blank">如何在国学大师网下载古籍?</a><br>
<a href="http://www.guoxuedashi.com/zidian/bujian/bjjc.php" target="_blank">如何使用部件查字法快速查字?</a><br>
<a href="http://www.guoxuedashi.com/search/sjc.php" target="_blank">如何在指定的书籍中全文检索?</a><br>
<a href="http://www.guoxuedashi.com/search/skjc.php" target="_blank">如何找到一句话在《四库全书》哪一页?</a><br>
</div>
</div>


<div class="sidebar">
<div class="sidebar_title">热门书籍</div>
<div class="sidebar_info">
<a href="/so.php?sokey=%E8%B5%84%E6%B2%BB%E9%80%9A%E9%89%B4&kt=1">资治通鉴</a> <a href="/24shi/"><strong>二十四史</strong></a>&nbsp; <a href="/a2694/">野史</a>&nbsp; <a href="/SiKuQuanShu/"><strong>四库全书</strong></a>&nbsp;<a href="http://www.guoxuedashi.com/SiKuQuanShu/fanti/">繁体</a>
<br><a href="/so.php?sokey=%E7%BA%A2%E6%A5%BC%E6%A2%A6&kt=1">红楼梦</a> <a href="/a/1858x/">三国演义</a> <a href="/a/1038k/">水浒传</a> <a href="/a/1046t/">西游记</a> <a href="/a/1914o/">封神演义</a>
<br>
<a href="http://www.guoxuedashi.com/so.php?sokeygx=%E4%B8%87%E6%9C%89%E6%96%87%E5%BA%93&submit=&kt=1">万有文库</a> <a href="/a/780t/">古文观止</a> <a href="/a/1024l/">文心雕龙</a> <a href="/a/1704n/">全唐诗</a> <a href="/a/1705h/">全宋词</a>
<br><a href="http://www.guoxuedashi.com/so.php?sokeygx=%E7%99%BE%E8%A1%B2%E6%9C%AC%E4%BA%8C%E5%8D%81%E5%9B%9B%E5%8F%B2&submit=&kt=1"><strong>百衲本二十四史</strong></a>  <a href="http://www.guoxuedashi.com/so.php?sokeygx=%E5%8F%A4%E4%BB%8A%E5%9B%BE%E4%B9%A6%E9%9B%86%E6%88%90&submit=&kt=1"><strong>古今图书集成</strong></a>
<br>

<a href="http://www.guoxuedashi.com/so.php?sokeygx=%E4%B8%9B%E4%B9%A6%E9%9B%86%E6%88%90&submit=&kt=1">丛书集成</a> 
<a href="http://www.guoxuedashi.com/so.php?sokeygx=%E5%9B%9B%E9%83%A8%E4%B8%9B%E5%88%8A&submit=&kt=1"><strong>四部丛刊</strong></a>  
<a href="http://www.guoxuedashi.com/so.php?sokeygx=%E8%AF%B4%E6%96%87%E8%A7%A3%E5%AD%97&submit=&kt=1">說文解字</a> <a href="http://www.guoxuedashi.com/so.php?sokeygx=%E5%85%A8%E4%B8%8A%E5%8F%A4&submit=&kt=1">三国六朝文</a>
<br><a href="http://www.guoxuedashi.com/so.php?sokeytm=%E6%97%A5%E6%9C%AC%E5%86%85%E9%98%81%E6%96%87%E5%BA%93&submit=&kt=1"><strong>日本内阁文库</strong></a> <a href="http://www.guoxuedashi.com/so.php?sokeytm=%E5%9B%BD%E5%9B%BE%E6%96%B9%E5%BF%97%E5%90%88%E9%9B%86&ka=100&submit=">国图方志合集</a> <a href="http://www.guoxuedashi.com/so.php?sokeytm=%E5%90%84%E5%9C%B0%E6%96%B9%E5%BF%97&submit=&kt=1"><strong>各地方志</strong></a>

</div>
</div>


<div class="sidebar2">
<center>

</center>
</div>
<div class="sidebar greenbar">
<div class="sidebar_title green">四库全书</div>
<div class="sidebar_info">

《四库全书》是中国古代最大的丛书,编撰于乾隆年间,由纪昀等360多位高官、学者编撰,3800多人抄写,费时十三年编成。丛书分经、史、子、集四部,故名四库。共有3500多种书,7.9万卷,3.6万册,约8亿字,基本上囊括了古代所有图书,故称“全书”。<a href="http://www.guoxuedashi.com/SiKuQuanShu/">详细>>
</a>

</div> 
</div>

</div>  <!--end r-->

</div>
<!-- 内容区END --> 

<!-- 页脚开始 -->
<div class="shh">

</div>

<div class="w1180" style="margin-top:8px;">
<center><script src="http://www.guoxuedashi.com/img/plus.php?id=3"></script></center>
</div>
<div class="w1180 foot">
<a href="/b/thanks.php">特别致谢</a> | <a href="javascript:window.external.AddFavorite(document.location.href,document.title);">收藏本站</a> | <a href="#">欢迎投稿</a> | <a href="http://www.guoxuedashi.com/forum/">意见建议</a> | <a href="http://www.guoxuemi.com/">国学迷</a> | <a href="http://www.shuowen.net/">说文网</a><script language="javascript" type="text/javascript" src="https://js.users.51.la/17753172.js"></script><br />
  Copyright &copy; 国学大师 古典图书集成 All Rights Reserved.<br>
  
  <span style="font-size:14px">免责声明:本站非营利性站点,以方便网友为主,仅供学习研究。<br>内容由热心网友提供和网上收集,不保留版权。若侵犯了您的权益,来信即刪。scp168@qq.com</span>
  <br />
ICP证:<a href="http://www.beian.miit.gov.cn/" target="_blank">鲁ICP备19060063号</a></div>
<!-- 页脚END --> 
<script src="http://www.guoxuedashi.com/img/plus.php?id=22"></script>
<script src="http://www.guoxuedashi.com/img/tongji.js"></script>

</body>
</html>
