\section{資治通鑑卷七十二}
宋 司馬光 撰

胡三省 音註

魏紀四|{
	起重光大淵獻盡逢攝提格凡四年}


烈祖明皇帝中之上

太和五年春二月吳主假太常潘濬節使與呂岱督軍五萬人討五溪蠻濬姨兄蔣琬為諸葛亮長史|{
	同出為姨母之姊妹曰姨妻之姊妹亦曰姨若母之兄弟則當呼為舅此蓋妻之兄弟也長知兩翻}
武陵太守衛旍奏濬遣密使與琬相聞|{
	旍與旌同使疏吏翻}
欲有自託之計吳主曰承明不為此也|{
	潘濬字承明}
即封旍表以示濬而召旍還免官衛温諸葛直軍行經歲士卒疾疫死者什八九亶洲絶遠卒不可得至|{
	卒子恤翻}
得夷洲數千人還温直坐無功誅|{
	吳遣温直見上卷上年}
漢丞相亮命李嚴以中都護署府事|{
	蜀置左右中三都護署府事署漢中留府事也}
嚴更名平|{
	更工衡翻}
亮帥諸軍入寇圍祁山以木牛運|{
	亮集曰木牛者方腹曲頭一脚四足頭入領中舌著於腹載多而行少宜可大用不可少使特行者數十里羣行者二十里也曲者為牛頭雙者為牛脚横者為牛領轉者為牛足覆者為牛背方者為牛腹垂者為牛舌曲者為牛肋刻者為牛齒立者為牛角細者為牛鞅攝者為牛鞦䩜牛仰雙轅人行六尺牛行四步載一歲糧日行二十里而又不大勞帥讀曰率}
於是大司馬曹真有疾帝命司馬懿西屯長安督將軍張郃費曜戴陵郭淮等以禦之|{
	郃古合翻又曷閣翻費父沸翻}
三月邵陵元侯曹真卒自十月不雨至于是月 司馬懿使費曜戴陵留精兵四千守上邽|{
	上邽縣前漢屬隴西郡後漢以來屬漢陽郡}
餘衆悉出西救祁山張郃欲分兵駐雍郿|{
	雍郿二縣皆屬扶風郡雍於用翻郿音媚又音眉}
懿曰料前軍能獨當之者將軍言是也若不能當而分為前後此楚之三軍所以為黥布禽也|{
	事見十二卷漢高帝十一年觀懿此言蓋自知其才不足以敵亮矣}
遂進亮分兵留攻祁山自逆懿于上邽郭淮費曜等徼亮|{
	徼讀曰邀}
亮破之因大芟刈其麥|{
	芟所衘翻}
與懿遇於上邽之東懿歛軍依險兵不得交亮引還懿等尋亮後至于鹵城張郃曰彼遠來逆我請戰不得謂我利在不戰欲以長計制之也且祁山知大軍已在近人情自固可止屯於此分為奇兵示出其後不宜進前而不敢偪坐失民望也今亮孤軍食少|{
	少詩沼翻}
亦行去矣懿不從故尋亮|{
	有意為之曰故尋者隨而躡其後}
既至又登山掘營不肯戰賈栩魏平數請戰|{
	數所角翻}
因曰公畏蜀如虎奈天下笑何懿病之|{
	懿實畏亮又以張郃嘗再拒亮名著關右不欲從其計及進而不敢戰情見勢屈為諸將所笑栩况羽翻}
諸將咸請戰夏五月辛巳懿乃使張郃攻無當監何平於南圍|{
	無當蓋蜀軍部之號言其軍精勇敵人無能當者使平監護之故名官曰無當監南圍蜀兵圍祁山之南屯監古暫翻}
自案中道向亮|{
	案據也懿分道進兵欲以解祁山之圍自據中道與亮旗鼓相向也}
亮使魏延高翔吳班逆戰魏兵大敗漢人獲甲首三千懿還保營六月亮以糧盡退軍司馬懿遣張郃追之郃進至木門|{
	木門去今天水軍天水縣十里水經注籍水出上邽當亭西山東歷當亭川又東入上邽縣左佩五水右帶五水木門谷之水其一也導源南山北流入籍水}
與亮戰蜀人乘高布伏弓弩亂發飛矢中郃右䣛而卒|{
	中竹仲翻䣛與膝同卒子恤翻}
秋七月乙酉皇子殷生大赦 黃初以來諸侯王法禁嚴切至于親姻皆不敢相通問東阿王植上疏曰堯之為教先親後疏自近及遠|{
	堯親九族九族既睦平章百姓百姓昭明協和萬邦}
周文王刑于寡妻至于兄弟以御于家邦|{
	詩大雅思齊之辭毛氏注曰刑法也寡妻嫡妻也御迎也鄭氏曰寡妻寡有之妻言賢也御治也文王以禮法接待其妻至于宗族以此又能為政治于家邦也}
伏惟陛下資帝唐欽明之德體文王翼翼之仁惠洽椒房恩昭九族羣后百寮番休遞上|{
	上時掌翻李周翰曰遞迭也言百寮宿衛以次休息更遞上直}
執政不廢於公朝|{
	朝直遥翻下同}
下情得展於私室親姻之路通慶弔之情展誠可謂恕己治人推惠施恩者矣|{
	治直之翻}
至于臣者人道絶緒禁錮明時臣竊自傷也不敢乃望交氣類|{
	易曰同聲相應同氣相求此言志同道合者謂疇昔文會之友也}
修人事叙人倫近且婚媾不通兄弟乖絶吉凶之問塞|{
	塞悉則翻}
慶弔之禮廢恩紀之違甚於路人隔閡之異殊於胡越|{
	殊絶也閡五慨翻}
今臣以一切之制|{
	一切謂權宜也一說一切謂不問可否一切整齊之也}
永無朝覲之望至於注心皇極|{
	皇極宅中之位人君居之}
結情紫闥神明知之矣然天實為之謂之何哉|{
	詩風北門之詩也鄭氏曰詩人事君無二志故歸之於天余謂植之意蓋謂君者天也天可違乎}
退惟諸王常有戚戚具爾之心|{
	詩曰戚戚兄弟莫遠具爾爾義與邇同}
願陛下沛然垂詔使諸國慶問四節得展|{
	四節謂四時之節展舒也}
以叙骨肉之歡恩全怡怡之篤義|{
	論語孔子曰兄弟怡怡}
妃妾之家膏沐之遺歲得再通|{
	呂延濟曰膏脂也沐甘漿之屬也遺于季翻}
齊義於貴宗等惠於百司|{
	貴宗謂貴戚及公卿之族也百司謂百官也}
如此則古人之所歎風雅之所詠復存於聖世矣臣伏自惟省無錐刀之用|{
	思惟也省悉景翻}
及觀陛下之所拔授若以臣為異姓竊自料度不後於朝士矣|{
	度徒洛翻}
若得辭遠游戴武弁解朱組佩青紱|{
	諸王冠遠游冠佩朱紱三都尉諸侍中常侍皆戴武弁佩青紱}
駙馬奉車趣得一號安宅京室|{
	駙馬奉車都尉及騎都尉為三都尉皆漢武帝置魏晉以下多以宗室及外戚為之}
執鞭珥筆出從華蓋入侍輦轂承荅聖問拾遺左右|{
	珥仍吏翻珥筆挿筆也古者侍臣持槖簪筆華蓋乘輿車上施之魏晉之制侍中與散騎常侍或乘輿御殿及出游幸祭祀治兵侍中居左常侍居右備切問近對拾遺補闕}
乃臣丹誠之至願不離於夢想者也|{
	離力智翻}
遠慕鹿鳴君臣之宴中詠常棣匪他之誡|{
	詩鹿鳴宴羣臣嘉賓常棣燕兄弟也其詩曰凡今之人莫如兄弟所謂匪他也又頍弁詩豈伊異人兄弟匪他}
下思伐木友生之義終懷蓼莪罔極之哀|{
	伐木燕朋友故舊其詩曰相彼鳥矣猶求友聲矧伊人矣不求友生蓼莪之詩曰哀哀父母生我劬勞欲報之德昊天罔極知念其父母必念其同氣矣蓼音六}
每四節之會塊然獨處|{
	處昌呂翻}
左右惟僕隸所對惟妻子高談無所與陳精義無所與展未嘗不聞樂而拊心臨觴而歎息也臣伏以犬馬之誠不能動人譬人之誠不能動天崩城隕霜|{
	齊大夫杞梁戰死于莒城其妻向城而哭城為之崩鄒衍盡忠於君燕惠王信讒而繫之鄒子仰天而哭正夏而天降霜}
臣初信之以臣心况徒虚語耳|{
	况譬也}
若葵藿之傾太陽雖不為回光然向之者誠也|{
	言葵藿草也傾葉於日日雖不為回光終是誠心向日也為于偽翻}
竊自比葵藿若降天地之施垂三光之明者實在陛下|{
	施式智翻下同}
臣聞文子曰不為福始不為禍先|{
	文子九篇班固曰文子老子弟子李周翰曰福始禍先謂諸王皆不表植獨先表也}
今之否隔友于同憂|{
	否隔不通也友于兄弟也否皮鄙翻}
而臣獨倡言者實不願於聖世有不蒙施之物欲陛下崇光被時雍之美宣緝熙章明之德也|{
	光被時雍言帝堯睦族之效詩周頌曰維清緝熙文王之典鄭氏箋曰緝熙光明也故植以言文王之治被皮義翻}
詔報曰蓋教化所由各有隆敝非皆善始而惡終也事使之然|{
	隆崇也謂立教之始各有所崇其流之敝則事勢使之然也惡如字}
今令諸國兄弟情禮簡怠妃妾之家膏沐疏略本無禁錮諸國通問之詔也矯枉過正下吏懼譴以至於此耳已勑有司如王所訴植復上疏曰昔漢文發代疑朝有變|{
	復扶又翻朝直遥翻}
宋昌曰内有朱虚東牟之親外有齊楚淮南琅邪此則磐石之宗願王勿疑|{
	事見十三卷漢高后八年}
臣伏惟陛下遠覧姬文二虢之援|{
	虢仲虢叔文王之母弟文王咨于二虢以成王業}
中慮周成召畢之輔|{
	召公畢公周同姓也二伯分治輔成王以成太平之功召讀曰邵下同}
下存宋昌磐石之固臣聞羊質虎皮見草則悦見豺則戰忘其皮之虎也|{
	揚子之言}
今置將不良有似於此|{
	將即亮翻}
故語曰患為之者不知知之者不得為也昔管蔡放誅周召作弼|{
	成王幼管叔蔡叔以武庚畔成王誅管叔放蔡叔以周公為師召公為保而相左右}
叔魚陷刑叔向贊國|{
	左傳晉邢侯與雍子爭田久而無成韓宣子使叔魚斷舊獄罪在雍子雍子納其女於叔魚叔魚蔽罪於邢侯邢侯怒殺叔魚及雍子于朝宣子問其罪於叔向不以叔向為私其親而從之决平也}
三監之釁臣自當之二南之輔求必不遠華宗貴族藩王之中必有應斯舉者夫能使天下傾耳注目者當權者是也故謀能移主威能懾下|{
	懾之涉翻}
豪右執政不在親戚權之所在雖疏必重埶之所去雖親必輕蓋取齊者田族非呂宗也分晉者趙魏非姬姓也|{
	齊太公姓呂其後為田成子所取非呂族也晉唐叔姬姓其後為趙籍魏斯韓䖍所分此不言韓以韓亦姬姓}
惟陛下察之苟吉專其位凶離其患者異姓之臣也|{
	離力智翻下得離同}
欲國之安祈家之貴存共其榮歿同其禍者公族之臣也今反公族疏而異姓親臣竊惑焉今臣與陛下踐氷履炭登山浮澗寒温燥濕高下共之豈得離陛下哉不勝憤懣|{
	勝音升懣音悶}
拜表陳情若有不合乞且藏之書府不便滅棄臣死之後事或可思若有毫釐少挂聖意者乞出之朝堂|{
	朝直遥翻下同}
使夫博古之士糾臣表之不合義者如是則臣願足矣帝但以優文答報而已|{
	植求自試而但以優詔荅之終疑之也}
八月詔曰先帝著令不欲使諸王在京都者謂幼主在位母后攝政防微以漸關諸盛衰也朕惟不見諸王十有二載|{
	自文帝黄初元年遣植等就國至是十二年惟思也載子亥翻}
悠悠之懷能不興思其令諸王及宗室公侯各將適子一人朝明年正月|{
	適讀曰嫡}
後有少主母后在宫者自如先帝令 漢丞相亮之攻祁山也李平留後主督運事|{
	李平即李嚴改名曰平}
會天霖雨平恐運糧不繼遣參軍狐忠|{
	狐忠即馬忠也少養外家姓狐名篤後復姓馬改名忠此姓從先名從後姓譜狐周王子狐之後又晉有狐突}
督軍成藩喻指呼亮來還|{
	喻以後主指言運糧不繼}
亮承以退軍平聞軍退乃更陽驚說軍糧饒足何以便歸又欲殺督運岑述以解己不辦之責又表漢主說軍偽退欲以誘賊|{
	此又欲解以上指喻亮之罪也誘音酉}
亮具出其前後手筆書疏本末違錯平辭窮情竭首謝罪負|{
	首式救翻}
於是亮表平前後過惡免官削爵土徙梓潼郡|{
	平蓋嘗封侯也}
復以平子豐為中郎將參軍事出教敕之曰|{
	敕戒也}
吾與君父子勠力以奬漢室表都護典漢中委君於東關|{
	東關謂江州}
謂至心感動終始可保何圖中乖乎若都護思負一意|{
	思負謂思其罪負也一意謂一意於為國無復詭變以自營也}
君與公琰推心從事否可復通|{
	否皮鄙翻}
逝可復還也詳思斯戒明吾用心亮又與蔣琬董允書曰孝起前為吾說正方腹中有鱗甲|{
	李嚴字正方為于偽翻下同}
鄉黨以為不可近|{
	近其靳翻}
吾以為鱗甲但不當犯之耳不圖復有蘇張之事出於不意|{
	謂蘇秦張儀押闔其說以反覆諸侯之間今李平復為之復扶又翻}
可使孝起知之孝起者衛尉南陽陳震也 冬十月吳主使中郎將孫布詐降以誘揚州刺史王凌吳主伏兵於阜陵以俟之|{
	阜陵縣漢屬九江郡魏改九江為淮南郡晉志曰阜陵縣漢明帝時淪為麻湖麻湖在今和州歷陽縣西三十里杜佑曰漢阜陵縣在滁州全椒縣南}
布遣人告凌云道遠不能自致乞兵見迎凌騰布書|{
	騰傳也上也}
請兵馬迎之征東將軍滿寵以為必詐不與兵而為凌作報書曰知識邪正欲避禍就順去暴歸道甚相嘉尚今欲遣兵相迎然計兵少則不足相衛多則事必遠聞|{
	聞音問}
且先密計以成本志臨時節度其宜會寵被書入朝|{
	被皮義翻朝直遥翻}
敕留府長史若凌欲往迎勿與兵也凌於後索兵不得|{
	索山客翻}
乃單遣一督將步騎七百人往迎之布夜掩擊督將迸走死傷過半|{
	迸北孟翻孫權自量其國之力不足以斃魏不過時於疆場之間設詐用奇以誘敵人之來而陷之耳非如孔明真有用蜀以爭天下之心也}
凌允之兄子也|{
	王允獻帝時誅董卓}
先是凌表寵年過耽酒不可居方任|{
	方任方面之任也先悉薦翻}
帝將召寵給事中郭謀曰寵為汝南太守豫州刺史|{
	漢建安中武王操以寵為汝南太守太和三年刺豫州是年都督揚州}
二十餘年有勲方岳|{
	自魏以下以督州為方岳之任謂其職猶古之方伯岳牧也}
及鎮淮南吳人憚之若不如所表將為所闚可令還朝|{
	朝直遥翻}
問以東方事以察之帝從之既至體氣康強帝慰勞遣還|{
	勞力到翻}
十一月戊戌晦日有食之十二月戊午博平敬侯華歆卒|{
	諡法夙夜警戒曰敬合善典法曰敬華戶化翻}
丁卯吳大赦改明年元曰嘉禾|{
	會稽南始平言嘉禾生故以改元}


六年春正月吳主少子建昌侯慮卒太子登自武昌入省吳主因自陳久離定省子道有闕|{
	記曲禮曰凡為人子之禮冬温而夏凊昏定而晨省省悉景翻離力智翻}
又陳陸遜忠勤無所顧憂乃留建業二月詔改封諸侯王皆以郡為國 帝愛女淑卒帝

痛之甚追諡平原懿公主立廟洛陽葬於南陵取甄后從孫黃與之合葬|{
	甄之人翻從才用翻}
追封黃為列侯為之置後襲爵|{
	為于偽翻下同}
帝欲自臨送葬又欲幸許司空陳羣諫曰八歲下殤禮所不備|{
	記檀弓曰周人以殷人之棺椁葬長殤以夏后氏之堲周葬中殤下殤以有虞氏之瓦棺葬無服之殤鄭玄注云略未成人陸德明曰十六至十九為長殤十二至十五為中殤八歲至十一為下殤七歲以下為無服之殤生未三月不為殤}
况未期月而以成人禮送之加為制服舉朝素衣朝夕哭臨自古以來未有此比|{
	朝直遥翻下同臨力鴆翻比毘寐翻}
而乃復自往視陵|{
	復扶又翻}
親臨祖載願陛下抑割無益有損之事此萬國之至望也又聞車駕欲幸許昌二宫上下皆悉居東舉朝大小莫不驚怪或言欲以避衰或言欲以便移殿舍|{
	避衰謂五行之氣有王有衰徙舍以避之也今人謂之避災便移殿舍謂欲營繕宫室故出幸許以便移殿舍也}
或不知何故臣以為吉凶有命禍福由人移走求安則亦無益若必當移避繕治金墉城西宫|{
	水經注金墉城在洛陽城西北角治直之翻}
及孟津别宫皆可權時分止何為舉宫暴露野次公私煩費不可計量|{
	量音良}
且吉士賢人猶不妄徙其家以寧鄉邑使無恐懼之心|{
	子思居於衛有齊寇或曰寇至盍去諸子思曰如伋去君誰與守}
况乃帝王萬國之主行止動静豈可輕脫哉少府楊阜曰文皇帝武宣皇后崩陛下皆不送葬所以重社稷備不虞也何至孩抱之赤子而送葬也哉帝皆不聽三月癸酉行東巡 吳主遣將軍周賀校尉裴潛乘海之遼東從公孫淵求馬初虞翻性疎直數有酒失又好抵忤人|{
	抵觸也數所角翻好呼到翻忤五故翻}
多見謗毁吳主嘗與張昭論及神仙翻指昭曰彼皆死人而語神仙世豈有仙人也吳主積怒非一遂徙翻交州及周賀等之遼東翻聞之以為五谿宜討遼東絶遠聽使來屬尚不足取今去人財以求馬|{
	去猶棄也去羌呂翻}
既非國利又恐無獲欲諫不敢作表以示呂岱岱不報為愛憎所白|{
	讒佞之人有愛有憎而無公是非故謂之愛憎白陳奏也}
復徙蒼梧猛陵|{
	猛陵縣屬蒼梧郡劉昫曰唐梧州孟陵縣藤州鐔津縣龔州南平武林隋建三縣皆漢猛陵縣地復扶又翻}
夏四月壬寅帝如許昌 五月皇子殷卒 秋七月以衛尉董昭為司徒 九月帝行如摩陂治許昌宫起景福承光殿|{
	治直之翻}
公孫淵隂懷貳心數與吳通|{
	數所角翻}
帝使汝南太守田豫督青州諸軍自海道幽州刺史王雄自陸道討之|{
	海道自東萊浮海陸道自遼西度遼水}
散騎常侍蔣濟諫曰凡非相吞之國不侵叛之臣|{
	光武報竇融書曰吾與爾非相吞之國左傳戎子駒支對范宣子曰為不侵不叛之臣}
不宜輕伐伐之而不能制是驅使為賊也故曰虎狼當路不治狐狸|{
	治直之翻}
先除大害小害自己今海表之地累世委質|{
	質如字}
歲選計孝|{
	計孝謂每歲上計及舉孝亷也}
不乏職貢議者先之|{
	先悉薦翻}
正使一舉便克得其民不足益國得其財不足為富儻不如意是為結怨失信也帝不聽豫等往皆無功詔令罷軍豫以吳使周賀等垂還歲晚風急必畏漂浪東道無㟁當赴成山成山無藏船之處遂輒以兵屯據成山賀等還至成山|{
	班志成山在東萊郡不夜縣後漢省不夜縣括地志成山在萊州文登縣西北百九十里}
遇風豫勒兵擊賀等斬之吳主聞之始思虞翻之言乃召翻於交州會翻已卒以其喪還|{
	還從宣翻又如字}
十一月庚寅陳思王植卒|{
	諡法追悔前過曰思}
十二月帝還許昌宫 侍中劉曄為帝所親重帝將伐蜀朝臣内外皆曰不可|{
	朝直遥翻}
曄入與帝議則曰可伐出與朝臣言則曰不可曄有膽智言之皆有形|{
	謂言蜀之可伐與不可伐皆有勝負之形可以動人之聽}
中領軍楊暨|{
	中領軍主中壘五校武衛等三營漢建安四年魏武丞相府自置中領軍文帝踐祚始置領軍將軍其後以資重者為領軍將軍資輕者則為中領軍}
帝之親臣又重曄執不可伐之議最堅每從内出輒過曄|{
	過工禾翻}
曄講不可之意後暨與帝論伐蜀事暨切諫帝曰卿書生焉知兵事|{
	焉於䖍翻下同}
暨謝曰臣言誠不足采侍中劉曄先帝謀臣常曰蜀不可伐帝曰曄與吾言蜀可伐暨曰曄可召質也|{
	質證也驗也對問也}
詔召曄至帝問曄終不言後獨見|{
	見賢遍翻下同}
曄責帝曰伐國大謀也臣得與聞大謀|{
	與讀曰預}
常恐眯夢漏泄以益臣罪|{
	眯毋禮翻一作寐說文曰寐而眯厭厭讀曰魘}
焉敢向人言之夫兵詭道也軍事未發不厭其密陛下顯然露之臣恐敵國已聞之矣於是帝謝之曄見出責暨曰夫釣者中大魚|{
	見賢遍翻中竹仲翻}
則縱而隨之須可制而後牽則無不得也人主之威豈徒大魚而已子誠直臣然計不足采不可不精思也暨亦謝之或謂帝曰曄不盡忠善伺上意所趨而合之|{
	伺相吏翻趨七喻翻}
陛下試與曄言皆反意而問之若皆與所問反者是曄常與聖意合也每問皆同者曄之情必無所逃矣|{
	言者謂曄善迎合上意上若有所問謂反上意而問之曄之對必與上所問者反而與上意所向者合每問皆然則可以見曄迎合之情矣}
帝如言以驗之果得其情從此疏焉|{
	疏與疎同}
曄遂發狂出為大鴻臚以憂死|{
	侍中在天子左右大鴻臚外朝官也臚陵如翻}


傅子曰巧詐不如拙誠信矣|{
	晉傅玄著書號傅子}
以曄之明智權計若居之以德義行之以忠信古之上賢何以加諸獨任才智不敦誠慤|{
	敦厚也崇尚也}
内失君心外困於俗卒以自危|{
	卒子恤翻}
豈不惜哉

曄嘗譖尚書令陳矯專權矯懼以告其子騫騫曰主上明聖大人大臣今若不合不過不作公耳後數日帝意果解尚書郎樂安亷昭以才能得幸好抉擿羣臣細過以求媚於上|{
	好呼到翻抉一决翻挑也擿他歷翻發動也}
黃門侍郎杜恕上疏曰伏見亷昭奏左丞曹璠以罰當關不依詔坐判問|{
	續漢志尚書左右丞各一人掌録文書期會左丞主吏民章報及騶伯史右丞主假署印綬及紙筆墨詣財用庫藏蔡質漢儀曰左丞總典臺中綱紀無所不統魏晉之制左丞主臺内禁令宗廟祠祀朝儀禮制選用署吏急假右丞掌臺内庫藏廬舍凡諸器用之物及廩振人租布刑獄兵器督録遠道文書章表奏事罰罪罰也關白也言有罪罰當關白而不依詔書故坐以判問判剖也析也問責問也剖析其事而責問之也璠孚袁翻}
又云諸當坐者别奏|{
	亷昭又云諸當坐者别奏意欲并奏今僕坐之}
尚書令陳矯自奏不敢辭罰亦不敢陳理志意懇惻臣竊愍然為朝廷惜之|{
	為于偽翻}
古之帝王所以能輔世長民者|{
	長知兩翻}
莫不遠得百姓之懽心近盡羣臣之智力今陛下憂勞萬幾或親燈火而庶事不康刑禁日弛原其所由非獨臣不盡忠亦其主不能使也百里奚愚於虞而智於秦|{
	韓信之言見十卷漢高帝三年}
豫讓苟容中行而著節智伯|{
	豫讓事范中行氏智伯伐而滅之移事智伯後趙襄子滅智伯豫讓漆身吞炭必報襄子五起而不中人問豫讓豫讓曰范中行衆人遇我我故衆人報之智伯國士遇我我故國士報之行戶剛翻}
斯則古人之明驗矣若陛下以為今世無良才朝廷乏賢佐豈可追望稷契之遐蹤|{
	契息列翻}
坐待來世之俊乂乎今之所謂賢者盡有大官而享厚禄矣然而奉上之節未立向公之心不一者委任之責不專而俗多忌諱故也臣以為忠臣不必親親臣不必忠今有疏者毁人而陛下疑其私報所憎譽人而陛下疑其私愛所親左右或因之以進憎愛之說遂使疏者不敢毁譽|{
	此言帝信其所親而疑其所疏遂使在遠之臣不敢言以至是非失其真也疏與疎同譽音余}
以致政事損益亦皆有嫌陛下當思所以闡廣朝臣之心篤厲有道之節|{
	有道謂有道之士也}
使之自同古人垂名竹帛反使如亷昭者擾亂其間臣懼大臣將遂容身保位坐觀得失為來世戒也昔周公戒魯侯曰無使大臣怨乎不以|{
	以用也見論語}
言不賢則不可為大臣為大臣則不可不用也書數舜之功稱去四凶|{
	共工驩兜鯀三苖世濟其惡然後去之數所具翻去羌呂翻}
不言有罪無問大小則去也|{
	言小過當略而不問}
今者朝臣不自以為不能以陛下為不任也不自以為不知以陛下為不問也|{
	知讀曰智}
陛下何不遵周公之所以用大舜之所以去使侍中尚書坐則侍帷幄行則從華輦親對詔問各陳所有則羣臣之行皆可得而知|{
	行下孟翻}
忠能者進闇劣者退誰敢依違而不自盡以陛下之聖明親與羣臣論議政事使羣臣人得自盡賢愚能否在陛下之所用以此治事何事不辦|{
	治直之翻下同}
以此建功何功不成每有軍事|{
	謂二邊有警急之時也}
詔書常曰誰當憂此者邪吾當自憂耳近詔又曰憂公忘私者必不然但先公後私即自辦也|{
	近詔謂近日所下詔也先悉薦翻後戶遘翻}
伏讀明詔乃知聖思究盡下情然亦怪陛下不治其本而憂其末也|{
	為治之本在於任賢事之治不治乃其末也}
人之能否實有本性雖臣亦以為朝臣不盡稱職也|{
	稱尺證翻}
明主之用人也使能者不敢遺其力而不能者不得處非其任|{
	處昌呂翻}
選舉非其人未必為有罪也舉朝共容非其人乃為怪耳|{
	朝直遥翻}
陛下知其不盡力而代之憂其職知其不能也而教之治其事豈徒主勞而臣逸哉雖聖賢並世終不能以此為治也|{
	為治直吏翻}
陛下又患臺閣禁令之不密人事請屬之不絶|{
	屬之欲翻下同}
定迎客出入之制以惡吏守寺門|{
	寺門官寺之門也}
斯實未得為禁之本也昔漢安帝時少府竇嘉辟廷尉郭躬無罪之兄子猶見舉奏章劾紛紛|{
	按范書郭躬章帝元和三年拜廷尉和帝永元六年卒不及安帝時蓋躬死後竇嘉方辟其兄子也劾戶槩翻又戶得翻}
近司隸校尉孔羨辟大將軍狂悖之弟|{
	裴松之曰按大將軍司馬宣王也晉書云宣帝第五弟名通為司隸從事疑恕所云狂悖者悖蒲内翻又蒲沒翻}
而有司嘿爾望風希指甚於受屬|{
	屬之欲翻}
選舉不以實者也嘉有親戚之寵躬非社稷重臣猶尚如此以今况古陛下自不督必行之罰以絶阿黨之原耳出入之制與惡吏守門非治世之具也|{
	治直之翻}
使臣之言少蒙察納|{
	少詩沼翻}
何患於姦不削滅而養若亷昭等乎夫糾擿姦宄忠事也|{
	擿他狄翻}
然而世憎小人行之者以其不顧道理而苟求容進也若陛下不復考其終始|{
	復扶又翻}
必以違衆迕世為奉公|{
	迕五故翻}
密行白人為盡節|{
	謂潛伺人之過失以白上乃以為盡節也}
焉有通人大才而更不能為此邪|{
	焉於䖍翻}
誠顧道理而弗為耳使天下皆背道而趨利則人主之所最病者也陛下將何樂焉|{
	背蒲妹翻趨七喻翻下同樂音洛}
恕畿之子也|{
	建安中畿守河東有能名}
帝嘗卒至尚書門|{
	卒讀曰猝尚書門尚書臺門也}
陳矯跪問帝曰陛下欲何之帝曰欲案行文書耳矯曰此自臣職分非陛下所宜臨也若臣不稱其職則請就黜退|{
	行下孟翻分扶問翻稱尺證翻}
陛下宜還帝慙回車而反帝嘗問矯司馬公忠貞可謂社稷之臣乎矯曰朝廷之望也社稷則未知也|{
	陳矯賈逵皆忠於魏而二人之子皆為晉初佐命豈但利禄之移人哉非故家喬木而教忠不先也}
吳陸遜引兵向廬江論者以為宜速救之滿寵曰廬江雖小將勁兵精|{
	將即亮翻}
守則經時|{
	謂陸遜若以兵圍守必經時而不能拔}
又賊舍船二百里來|{
	句絶舍讀曰捨}
後尾空絶不來尚欲誘致今宜聽其遂進但恐走不可及耳乃整軍趨楊宜口|{
	魏廬江郡治陽泉縣續漢志陽泉縣有陽泉湖故陽泉鄉也漢靈帝封黃琬為侯國水經注陽泉水受决水東北流逕陽泉縣故城東又西北入决水謂之陽泉口趨七喻翻}
吳人聞之夜遁是時吳人歲有來計滿寵上疏曰合肥城南臨江湖北遠夀春|{
	魏揚州治夀春距合肥二百餘里遠于願翻下同}
賊攻圍之得據水為埶官兵救之當先破賊大輩然後圍乃得解賊往甚易|{
	易以䜴翻}
而兵往救之甚難宜移城内之兵其西三十里有奇險可依更立城以固守此為引賊平地而掎其歸路|{
	掎居蟻翻}
於計為便護軍將軍蔣濟議以為既示天下以弱且望賊烟火而壞城|{
	壞音怪}
此為未攻而自拔一至於此劫略無限必淮北為守|{
	濟言望風移戍吳必劫掠無限將限淮以自守也}
帝未許寵重表曰|{
	重直用翻}
孫子言兵者詭道也故能而示之不能驕之以利示之以懾|{
	懾懼也懾之涉翻}
此為形實不必相應也又曰善動敵者形之今賊未至而移城却内所謂形而誘之也引賊遠水|{
	遠于願翻}
擇利而動舉得於外則福生於内矣尚書趙咨以寵策為長|{
	趙咨蓋必黃初初自吳使于魏者也文帝重其辯給遂臣於魏}
詔遂報聽

青龍元年春正月甲申青龍見摩陂井中二月帝如摩陂觀龍改元|{
	自是改摩陂曰龍陂}
公孫淵遣校尉宿舒|{
	姓譜宿本風姓伏羲之後封於宿風俗通漢有雁門太守宿詳}
郎中令孫綜|{
	晉志王國置郎中令淵未封王潛置之也}
奉表稱臣於吳吳主大悦為之大赦|{
	為于偽翻}
三月吳主遣太常張彌執金吾許晏將軍賀達將兵萬人金寶珍貨九錫備物乘海授淵封淵為燕王舉朝大臣自顧雍以下皆諫以為淵未可信而寵待太厚但可遣吏兵護送舒綜而已吳主不聽張昭曰淵背魏懼討|{
	背蒲妹翻}
遠來求援非本志也若淵改圖欲自明於魏兩使不反|{
	使疏吏翻}
不亦取笑於天下乎吳主反覆難昭|{
	難乃旦翻}
昭意彌切吳主不能堪案劍而怒曰吳國士人入宫則拜孤出宫則拜君孤之敬君亦為至矣而數於衆中折孤|{
	數所角翻折之舌翻}
孤常恐失計|{
	失計謂不能容昭而殺之也}
昭孰視吳主|{
	古孰熟字通}
曰臣雖知言不用每竭愚忠者誠以太后臨崩呼老臣於牀下遺詔顧命之言故在耳|{
	事見六十五卷漢獻帝建安十二年}
因涕泣橫流吳主擲刀於地與之對泣然卒遣彌晏往|{
	卒子恤翻}
昭忿言之不用稱疾不朝|{
	朝直遥翻}
吳主恨之土塞其門|{
	塞悉則翻}
昭又於内以土封之|{
	張昭事吳有古大臣之節}
夏五月戊寅北海王蕤卒 閏月庚寅朔日有食之 六月洛陽宫鞠室災|{
	鞠室者畫地為域以蹵鞠因以名室}
鮮卑軻比能誘保塞鮮卑步度根與深結和親|{
	步度根保塞見七十卷文帝黃初五年誘音酉}
自勒萬騎迎其累重於陘北|{
	累力瑞翻重直用翻陘音刑陘北陘嶺之北也唐代州雁門縣有東陘關西陘山}
并州刺史畢軌表輒出軍以外威比能内鎮步度根帝省表曰|{
	省悉景翻}
步度根已為比能所誘有自疑心今軌出軍慎勿越塞過句注也|{
	漢靈帝末羌胡大擾定襄雲中五原朔方上郡並流徙分散建安二十年集塞下荒地置新興郡自陘嶺以北並棄之故以句注為塞}
比詔書到|{
	比必寐翻}
軌已進軍屯隂館|{
	應劭曰句注山名在雁門隂館縣杜佑曰句注山即雁門縣西陘嶺句伏儼音俱包愷音鉤}
遣將軍蘇尚董弼追鮮卑軻比能遣子將千餘騎迎步度根部落與尚弼相遇戰於樓煩|{
	隂館樓煩二縣漢皆屬雁門郡而晉志無之蓋已棄之荒外矣}
二將沒步度根與泄歸泥部落皆叛出塞|{
	泄歸泥扶羅韓之子}
與軻比能合寇邊帝遣驍騎將軍秦朗將中軍討之|{
	晉職官志驍騎將軍游擊將軍並漢雜號將軍也魏置為中軍}
軻比能乃走幕北泄歸泥將其部衆來降步度根尋為軻比能所殺公孫淵知吳遠難恃乃斬張彌許晏等首傳送京師

悉沒其兵資珍寶|{
	卒如張昭之言傳直戀翻}
冬十二月詔拜淵大司馬封樂浪公|{
	樂浪音洛埌}
吳主聞之大怒曰朕年六十世事難易靡所不嘗|{
	嘗試也易以䜴翻}
近為鼠子所前却|{
	謂稱臣以誘吳使使前既又斬其使以却之也}
令人氣踊如山不自截鼠子頭以擲于海無顔復臨萬國|{
	復扶又翻}
就令顛沛不以為恨|{
	知其不可而欲興忿兵也}
陸遜上疏曰陛下以神武之資誕膺期運破操烏林|{
	事見六十五卷漢獻帝建安十三年}
敗備西陵|{
	事見六十九卷文帝黃初三年敗補邁翻}
禽羽荆州|{
	事見六十八卷建安二十四年}
斯三虜者當世雄傑皆摧其鋒聖化所綏萬里草偃|{
	言如風行而草偃也}
方蕩平華夏總一大猷|{
	猷道也謀也夏戶雅翻}
今不忍小忿而發雷霆之怒違垂堂之戒|{
	千金之子坐不垂堂以喻權不當自越海而加兵於遼東}
輕萬乘之重|{
	乘繩證翻}
此臣之所惑也臣聞之行萬里者不中道而輟足圖四海者不懷細而害大強寇在境荒服未庭陛下乘桴遠征|{
	桴芳無翻編竹木渡水大者曰栰小者曰桴}
必致闚?慼至而憂悔之無及若使大事時捷則淵不討自服今乃遠惜遼東衆之與馬|{
	謂權所以遠惜遼東而不忍棄絶之者以其民衆與其地產馬也}
奈何獨欲捐江東萬安之本業而不惜乎尚書僕射薛綜上疏曰昔漢元帝欲御樓船薛廣德請刎頸以血染車|{
	事見二十八卷永光元年刎武粉翻}
何則水火之險至危非帝王所宜涉也今遼東戎貊小國|{
	貊莫百翻}
無城隍之固備禦之術器械銖鈍|{
	銖者十分黍之重言其輕也}
犬羊無政往必禽克誠如明詔然其方土寒埆|{
	埆克角翻墝瘠也}
穀稼不殖民習鞍馬轉徙無常卒聞大軍之至自度不敵|{
	卒讀曰猝度徒洛翻}
鳥驚獸駭長驅奔竄一人匹馬不可得見雖獲空地守之無益此不可一也加又洪流滉瀁|{
	滉瀁水深廣貌滉戶廣翻瀁以兩翻又余亮翻}
有成山之難海行無常風波難免倏忽之間人船異埶雖有堯舜之德智無所施賁育之勇力不得設此不可二也|{
	賁音奔}
加以鬱霧冥其上鹹水蒸其下善生流腫轉相洿染|{
	洿烏故翻流腫者謂毒氣下流足為之腫古人謂之重膇今人謂之脚氣}
凡行海者稀無此患此不可三也天生神聖當乘時平亂康此民物今逆虜將滅海内垂定乃違必然之圖尋至危之阻忽九州之固肆一朝之忿既非社稷之重計又開闢以來所未嘗有斯誠羣僚所以傾身側息|{
	謂傾身而卧側鼻而息不得展布四體安於偃仰也}
食不甘味寢不安席者也選曹尚書陸瑁上疏曰|{
	吳選曹尚書即魏選部尚書瑁音冒}
北寇與國壤地連接苟有閒隙|{
	閒古莧翻下同}
應機而至夫所以為越海求馬曲意於淵者為赴目前之急除腹心之疾也|{
	為于偽翻}
而更棄本追末捐近治遠|{
	治直之翻}
忿以改規激以動衆斯乃猾虜所願聞非大吳之至計也|{
	北寇猾虜皆謂魏也}
又兵家之術以功役相疲勞逸相待得失之間所覺輒多|{
	兵法以逸待勞又曰逸則能勞之言敵人用智以疲我苦不自覺比我覺知則得失之間相去多矣}
且沓渚去淵道里尚遠|{
	遼東郡有沓氏縣西南臨海渚應劭曰沓長荅翻又據陳夀志景初三年以遼東東沓縣吏民渡海居齊郡界為新沓縣即沓渚之民也}
今到其岸兵勢三分使強者進取次當守船又次運糧行人雖多難得悉用加以單步負糧經遠深入賊地多馬邀截無常若淵狙詐與北未絶動衆之日唇齒相濟|{
	此慮魏乘吳伐遼之間而南侵也狙千余翻}
若實了然無所憑賴|{
	了然猶言曉然也蜀本作孑然文義尤長孑孤孑也謂淵孤立孑然無援也}
其畏怖遠迸或難卒滅|{
	怖普布翻迸北孟翻卒讀曰猝}
使天誅稽於朔野山虜乘間而起|{
	山虜謂丹陽豫章鄱陽廬陵新都等郡山越也乘蜀本作承間古莧翻}
恐非萬安之長慮也吳主未許瑁重上疏曰|{
	重直龍翻}
夫兵革者固前代所以誅暴亂威四夷也然其役皆在姦雄已除天下無事從容廟堂之上|{
	從千容翻}
以餘議議之耳至於中夏鼎沸九域盤互之時|{
	盤互謂各盤據而互為敵也夏戶雅翻}
率須深根固本愛力惜費未有正於此時舍近治遠以疲軍旅者也|{
	舍讀曰捨治直之翻}
昔尉佗叛逆僭號稱帝于時天下乂安百姓康阜然漢文猶以遠征不易告喻而已|{
	佗徒河翻事見十三卷漢文帝元年易以䜴翻}
今凶桀未殄疆場猶警|{
	場音亦}
未宜以淵為先願陛下抑威任計暫寧六師潛神嘿規以為後圖天下幸甚吳主乃止吳主數遣人慰謝張昭|{
	數所角翻}
昭固不起吳主因出過其門呼昭|{
	過工禾翻}
昭辭疾篤吳主燒其門欲以恐之|{
	恐丘共翻}
昭亦不出吳主使人滅火住門良久昭諸子共扶昭起吳主載以還宫深自克責昭不得已然後朝會|{
	朝直遥翻}
初張彌許晏等至襄平|{
	襄平縣遼東郡治所淵所都也}
公孫淵欲圖之乃先分散其吏兵中使秦旦張羣杜德黃彊等及吏兵六十人置玄莬|{
	中使中節人使也使疏吏翻陳夀曰漢武帝開玄莬郡治沃沮城後為夷貊所侵徙郡句驪西北莬同都翻}
玄莬在遼東北二百里|{
	此非玄莬郡舊治也}
太守王贊領戶二百旦等皆舍於民家仰其飲食|{
	仰牛向翻}
積四十許日旦與羣等議曰吾人遠辱國命自棄於此與死無異今觀此郡形勢甚弱若一旦同心焚燒城郭殺其長史為國報恥|{
	長知兩翻為于偽翻}
然後伏死足以無恨孰與偷生苟活長為囚虜乎羣等然之於是隂相結約當用八月十九日夜發其日中時為郡中張松所告贊便會士衆閉城門旦羣德彊皆踰城得走時羣病疽瘡著䣛|{
	疽千余翻着直略翻䣛與膝同}
不及輩旅德常扶接與俱崎嶇山谷|{
	崎丘宜翻嶇音區}
行六七百里創益困不復能前|{
	創初良翻下同}
卧草中相守悲泣羣曰吾不幸創甚死亡無日卿諸人宜速進道冀有所達空相守俱死於窮谷之中何益也德曰萬里流離死生共之不忍相委|{
	委棄也}
於是推旦彊使前|{
	推吐雷翻}
德獨留守羣採菜果食之|{
	食讀曰飤}
旦彊别數日得達句麗因宣吳主詔於句麗王位宫及其主簿|{
	高句麗國在遼東之東千里位宫漢高句麗王宫之曾孫也宫生而開目能視及長勇壯數犯漢邊位宫生墮地亦能開目視人句麗呼相似為位以似其祖故名曰位宫句麗有相加對盧沛者古鄒大加主簿擾台使者帛衣先人帛衣三國志作皁衣句音如字又音駒驪力知翻}
紿言有賜為遼東所劫奪|{
	紿徒變翻}
位宫等大喜即受詔命使人隨旦還迎羣遣皁衣二十五人送旦等還吳奉表稱臣貢貂皮千枚鶡雞皮十具|{
	郭璞注山海經曰鶡雞似雉而大青色有毛角鬭敵死乃止鶡何葛翻}
旦等見吳主悲喜不能自勝|{
	勝音升}
吳主壯之皆拜校尉 是歲吳主出兵欲圍新城|{
	合肥新城也}
以其遠水積二十餘日不敢下船|{
	大船向㟁船高㟁卑故謂舍船就㟁曰下船以自船而下也遠于願翻}
滿寵謂諸將曰孫權得吾移城必於其衆中有自大之言今大舉來欲要一切之功|{
	要一遥翻}
雖不敢至必當上㟁耀兵以示有餘|{
	上時掌翻}
乃潛遣步騎六千伏肥水隱處以待之吳主果上㟁耀兵寵伏兵卒起擊之|{
	卒讀曰猝}
斬首數百或有赴水死者吳主又使全琮攻六安亦不克 蜀庲降都督張翼|{
	水經注寧州建寧縣故庲降都督屯蜀後主建興三年分益州郡置之}
用法嚴峻南夷豪帥劉胄叛|{
	帥所類翻}
丞相亮以參軍巴西馬忠代翼召翼令還其人謂翼宜速歸即罪|{
	其人謂召翼者也即就也}
翼曰不然吾以蠻夷蠢動不稱職故還耳|{
	稱尺證翻}
然代人未至吾方臨戰場當運糧積穀為滅賊之資豈可以黜退之故而廢公家之務乎於是統攝不懈|{
	懈古隘翻}
代到乃發馬忠因其成基破胄斬之 諸葛亮勸農講武作木牛流馬|{
	亮集曰流馬尺寸之數肋長三尺五寸廣三寸厚二寸二分左右同前軸孔分墨去頭四寸徑中二寸前脚孔分墨二寸去前軸孔四寸五分廣一寸前杠孔分墨去前脚孔分墨三寸七分孔長二寸廣一寸後軸孔去前杠孔分墨一尺五分大小與前同後脚孔分墨去後軸孔三寸五分大小與前同後杠孔去後脚孔分墨二寸七分後載尅去後杠孔分墨四寸五分前杠長一尺八寸廣二寸厚一寸五分後杠與等板方囊一板厚八分長二尺七寸高一尺六寸五分廣一尺六寸每枚受米二斛三斗從上杠孔去肋下七寸前後同上杠孔去下杠孔分墨一尺三寸孔長一寸五分廣七分八孔同前後四脚廣二寸厚一寸五分形制如象靬長四寸徑面四寸三分孔徑中二脚杠長二尺一寸廣二寸五分厚一寸四分同杠耳}
運米集斜谷口治斜谷邸閣息民休士三年而後用之|{
	按明年亮即出斜谷所謂息民休士三年而後用之通自再攻祁山之後至是凡三年也斜昌遮翻谷音浴又古禄翻}


二年 |{
	考異曰唐太宗晉書景懷夏侯后傳后以此年死云宣帝居上將之重諸子並有雄才大略后知帝非魏之純臣而后既魏氏之甥帝深忌之遂以鴆崩按是時司馬懿方信任於明帝未有不臣之迹况其諸子乎徒以魏甥之故猥鴆其妻俱非事實蓋甚之之辭不然師自以他故鴆之也今不取}
春二月亮悉大衆十萬由斜谷入寇遣使約吳同時大舉 三月庚寅山陽公卒|{
	獻帝自禪位至卒十有四年年五十四}
帝素服發喪己酉大赦 夏四月大疫 崇華殿災|{
	是歲復修改崇華曰九龍殿引穀水過九龍前為玉井綺欄蟾蜍含受神龍吐出}
諸葛亮至郿|{
	郿音媚又音眉}
軍於渭水之南司馬懿引軍渡渭背水為壘以拒之|{
	背蒲妹翻}
謂諸將曰亮若出武功依山而東誠為可憂若西上五丈原|{
	水經注五丈原在郿縣西渭水逕其北又亮與步隲書曰原在武功西十里上時掌翻}
諸將無事矣亮果屯五丈原雍州刺史郭淮言於懿曰|{
	雍於用翻}
亮必爭北原宜先據之議者多謂不然淮曰若亮跨渭登原連兵北山隔絶隴道揺盪民夷|{
	盪徒朗翻}
此非國之利也懿乃使淮屯北原塹壘未成|{
	塹七艶翻}
漢兵大至淮逆擊却之亮以前者數出|{
	數所角翻}
皆以運糧不繼使己志不伸乃分兵屯田為久駐之基耕者雜於渭濱居民之間而百姓安堵軍無私焉 五月吳主入居巢湖口|{
	巢湖口即今柵江口也在和州歷陽縣西南百五十里水導源巢湖裴松之曰巢祖了翻今巢湖與焦湖通焦勦音近故有勦音今讀如字}
向合肥新城|{
	即太和六年滿寵所築新城也華夷對境圖魏合肥新城今為廬州謝步鎮}
衆號十萬又遣陸遜諸葛瑾將萬餘人入江夏沔口向襄陽|{
	瑾渠吝翻沔彌兖翻}
將軍孫韶張承入淮向廣陵淮隂六月滿寵欲率諸軍救新城殄夷將軍田豫曰|{
	殄夷將軍蓋魏所置然不在沈約志所謂四十號將軍之數}
賊悉衆大舉非圖小利欲質新城以致大軍耳|{
	質音致}
宜聽使攻城挫其銳氣不當與争鋒也城不可拔衆必罷怠罷怠然後擊之可大克也|{
	罷讀曰疲}
若賊見計|{
	言窺見吾所以待敵之計也}
必不攻城埶將自走若便進兵適入其計矣時東方吏士皆分休寵表請召中軍兵并召所休將士|{
	分休猶番休也}
須集擊之散騎常侍廣平劉劭議以為賊衆新至心專氣銳寵以少人自戰其地|{
	少詩沼翻}
若便進擊必不能制寵請待兵未有所失也以為可先遣步兵五千精騎三千先軍前發|{
	先悉薦翻}
揚聲進道震曜形埶騎到合肥疏其行隊|{
	疏讀曰疎行戶剛翻}
多其旌鼔曜兵城下引出賊後擬其歸路要其糧道賊聞大軍來騎斷其後必震怖遁走|{
	要一遥翻斷丁管翻怖普布翻}
不戰自破矣帝從之寵欲拔新城守致賊夀春帝不聽曰昔漢光武遣兵據略陽終以破隗囂|{
	事見四十二卷建武八年}
先帝東置合肥南守襄陽西固祁山賊來輒破於三城之下者地有所必爭也|{
	合肥襄陽以備吳祁山以備蜀也}
縱權攻新城必不能拔敕諸將堅守吾將自往征之比至恐權走也|{
	比必寐翻}
乃使征蜀護軍秦朗督步騎二萬助司馬懿禦諸葛亮敕懿但堅壁拒守以挫其鋒彼進不得志退無與戰久停則糧盡虜略無所獲則必走走而追之全勝之道也秋七月帝御龍舟東征滿寵募壯士焚吳攻具射殺吳主之弟子泰|{
	射而亦翻}
又吳吏士多疾病帝未至數百里疑兵先至吳主始謂帝不能出聞大軍至遂遁孫韶亦退陸遜遣親人韓扁奉表詣吳主邏者得之|{
	扁補典翻又音篇邏郎佐翻}
諸葛瑾聞之甚懼書與遜云大駕已還賊得韓扁具知吾濶狹且水乾宜當急去|{
	乾音干}
遜未荅方催人種葑豆|{
	葑菜也謂之蔓菁豆菽也}
與諸將奕棊射戲如常瑾曰伯言多智略|{
	陸遜一名議字伯言}
其必當有以乃自來見遜遜曰賊知大駕已還無所復憂得專力於吾又已守要害之處兵將意動|{
	謂敵既知權還料遜兵當退已分守要害之處欲以遮截遜所部兵既無進取之氣而有遮截之慮則其意恐動將至於或降或潰也復扶又翻}
且當自定以安之施設變術然後出耳今便示退賊當謂吾怖|{
	怖普布翻}
仍來相蹙必敗之勢也乃密與瑾立計令瑾督舟船遜悉上兵馬以向襄陽城|{
	上時掌翻}
魏人素憚遜名遽還赴城瑾便引船出遜徐整部伍張拓聲埶步趣船|{
	趣七喻翻}
魏人不敢逼行到白圍|{
	蓋立圍屯於白河口因以為名}
託言住獵潛遣將軍周峻張梁等擊江夏新市安陸石陽|{
	新市安陸二縣皆屬江夏郡魏初以文聘為江夏太守屯石陽舟車湊焉頗為繁富沈約曰江夏曲陵縣本名石陽晉武帝太康元年改曰曲陵宋明帝泰始六年併曲陵入安陸縣}
斬獲千餘人而還羣臣以為司馬懿方與諸葛亮相守未解車駕可西幸長安帝曰權走亮膽破大軍足以制之吾無憂矣遂進軍至夀春録諸將功封賞各有差 八月壬申葬漢孝獻皇帝于禪陵|{
	帝王紀曰禪陵在濁鹿城西北十里賢曰在今懷州修武縣北二十五里劉澄之地記曰以漢禪魏因以名焉}
辛巳帝還許昌 司馬懿與諸葛亮相守百餘日亮數挑戰|{
	數所角翻挑徒了翻}
懿不出亮乃遺懿巾幗婦人之服|{
	字書幗古獲翻婦人喪冠也又古對翻據劉昭注補輿服志公卿列侯夫人紺繒幗蓋婦人首飾之稱不特喪冠也遺于季翻}
懿怒上表請戰帝使衛尉辛毗杖節為軍師以制之護軍姜維謂亮曰辛佐治杖節而到賊不復出矣|{
	治直吏翻復扶又翻}
亮曰彼本無戰情所以固請戰者以示武於其衆耳將在軍君命有所不受|{
	孫武子及司馬穰苴之言也將即亮翻}
苟能制吾豈千里而請戰邪亮遣使者至懿軍懿問其寢食及事之煩簡不問戎事|{
	懿所憚者亮也問其寢食及事之煩簡以覘夀命之久近耳戎事何必問邪}
使者對曰諸葛公夙興夜寐罰二十以上皆親覽焉所噉食不至數升懿告人曰諸葛孔明食少事煩其能久乎亮病篤漢使尚書僕射李福省侍|{
	噉徒濫翻少詩沼翻省悉景翻}
因諮以國家大計福至與亮語已别去|{
	已竟也語竟而别也}
數日復還|{
	復扶又翻下同}
亮曰孤知君還意近日言語雖彌日有所不盡更來求决耳公所問者公琰其宜也福謝前實失不諮請如公百年後誰可任大事者故輒還耳乞復請蔣琬之後誰可任者亮曰文偉可以繼之又問其次亮不荅|{
	費禕字文偉亮不荅繼禕之人非高帝此後亦非乃所知之意蓋亦見蜀之人士無足以繼禕者矣嗚呼}
是月亮卒于軍中長史楊儀整軍而出百姓奔告司馬懿懿追之姜維令儀反旗鳴鼓若將向懿者懿歛軍退不敢偪|{
	猶恐亮未死也}
於是儀結陳而去|{
	陳讀曰陣}
入谷然後發喪|{
	入斜谷也}
百姓為之諺曰死諸葛走生仲達|{
	司馬懿字仲達以當時百姓之諺觀之時人之於孔明何如也}
懿聞之笑曰吾能料生不能料死故也懿案行亮之營壘處所歎曰天下奇才也|{
	方亮之出也懿以為若西上五丈原諸將無事矣及亮既死退軍懿案行其營壘處所以為天下奇才觀此則知懿已料亮之必屯五丈原而力不能制姑為此言以安諸將之心耳行下孟翻}
追至赤㟁不及而還|{
	還從宣翻又如字}
初漢前軍師魏延|{
	蜀置中軍師前軍師後軍師}
勇猛過人善養士卒每隨亮出輒欲請兵萬人與亮異道會于潼關如韓信故事|{
	韓信請兵故事見九卷漢高帝二年}
亮制而不許延常謂亮為怯歎恨己才用之不盡楊儀為人幹敏亮每出軍儀常規畫分部籌度糧穀不稽思慮斯須便了|{
	斯此也須待也言即此待之便可辦事分扶問翻度徒洛翻}
軍戎節度取辦於儀延性矜高當時皆避下之|{
	下遐稼翻}
唯儀不假借延延以為至忿有如水火|{
	言不可同處也}
亮深惜二人之才不忍有所偏廢也費禕使吳|{
	費父沸翻使疏吏翻}
吳主醉問禕曰楊儀魏延牧豎小人也雖嘗有鳴吠之益於時務然既己任之勢不得輕若一朝無諸葛亮必為禍亂矣諸君憒憒|{
	憒古對翻釋云心亂也}
不知防慮於此豈所謂貽厥孫謀乎禕對曰儀延之不協起於私忿耳而無黥韓難御之心也|{
	黥布韓信也}
今方埽除強賊混一函夏|{
	夏戶雅翻}
功以才成業由才廣若捨此不任防其後患是猶備有風波而逆廢舟檝非長計也|{
	檝與楫同}
亮病困與儀及司馬費禕等作身歿之後退軍節度令延斷後|{
	斷讀曰短}
姜維次之若延不從命軍便自發|{
	亮固知延非儀所能令矣}
亮卒儀祕不發喪令禕往揣延意指|{
	揣初委翻}
延曰丞相雖亡吾自見在|{
	此魏延矜高之語也見賢遍翻}
府親官屬便可將喪還葬|{
	府親官屬謂長史以下也}
吾當自率諸軍擊賊云何以一人死廢天下之事邪且魏延何人當為楊儀所部勒作斷後將乎|{
	將即亮翻}
自與禕共作行留部分|{
	分扶問翻}
令禕手書與己連名告下諸將|{
	時禕為亮司馬延知儀必不已從故因禕來劫與共作行留處分行謂當從亮喪還者留謂當留拒敵者延欲令禕手書處分之語告其下諸將也}
禕紿延曰當為君還解楊長史長史文吏稀更軍事|{
	紿徒亥翻為于偽翻更工衡翻}
必不違命也禕出奔馬而去延尋悔之已不及矣|{
	尋繼也言繼時而悔也}
延遣人覘儀等欲案亮成規諸營相次引軍還|{
	覘丑亷翻還從宣翻又如字下同}
延大怒攙儀未發|{
	攙初衘翻自後爭前曰攙今人猶言攙先}
率所領徑先南歸所過燒絶閣道延儀各相表叛逆一日之中羽檄交至漢主以問侍中董允留府長史蔣琬琬允咸保儀而疑延儀等令槎山通道|{
	槎仕下翻邪斫木也}
晝夜兼行亦繼延後延先至據南谷口|{
	南谷即褒谷也南谷曰褒北谷曰斜長四百七十里同為一谷}
遣兵逆擊儀等儀等令將軍何平於前禦延|{
	何平即王平也本養外家何氏後復姓王此從其初姓}
平叱先登曰公亡身尚未寒汝輩何敢乃爾延士衆知曲在延莫為用命|{
	為于偽翻}
皆散延獨與其子數人逃亡奔漢中儀遣將馬岱追斬之遂夷延三族蔣琬率宿衛諸營北行赴難|{
	難乃旦翻}
行數十里延死問至乃還|{
	問音訊也}
始延欲殺儀等冀時論以己代諸葛輔政故不降魏而南還擊儀實無反意也|{
	延雖無反意使其輔政是速蜀之亡也降戶江翻}
諸軍還成都大赦諡諸葛亮曰忠武侯初亮表於漢主曰成都有桑八百株薄田十五頃子弟衣食自有餘饒臣不别治生以長尺寸|{
	治直之翻長知兩翻}
若臣死之日不使内有餘帛外有贏財以負陛下卒如其所言|{
	卒子恤翻}
丞相長史張裔常稱亮曰公賞不遺遠罰不阿近爵不可以無功取刑不可以貴勢免此賢愚所以僉忘其身者也

陳夀評曰諸葛亮之為相國也撫百姓示儀軌|{
	儀度也軌法也}
約官職從權制開誠心布公道盡忠益時者雖讐必賞犯法怠慢者雖親必罰服罪輸情者雖重必釋游辭巧飾者雖輕必戮善無微而不賞惡無纎而不貶庶事精練物理其本|{
	言事事物物必從其本而治之}
循名責實虚偽不齒終於邦域之内咸畏而愛之刑政雖峻而無怨者以其用心平而勸戒明也可謂識治之良才管蕭之亞匹矣|{
	治直吏翻亞次也匹遇也}


初長水校尉廖立|{
	廖力弔翻姓也裴松之音理救翻姓譜廖姓周文王子伯廖之後後漢有廖湛風浴通曰古有廖叔安左傳作飂蓋其後也}
自謂才名宜為諸葛亮之副常以職位游散|{
	散悉亶翻}
怏怏怨謗無己亮廢立為民徙之汶山|{
	據立傳廢徙汶山後主初立之時也汶山漢武帝開為郡宣帝地節三年合於蜀郡蜀又分置汶山郡唐為茂州汶山縣汶音問}
及亮卒立垂泣曰吾終為左衽矣李平聞之亦發病死|{
	平廢徙見上太和五年}
平常冀亮復收已得自補復策後人不能故也|{
	復扶又翻}


習鑿齒論曰昔管仲奪伯氏駢邑三百沒齒而無怨言聖人以為難|{
	見論語鄭氏曰小國之下大夫采地方一成其定稅三百家故三百戶也其實大國下大夫亦三百戶故論語云管仲奪伯氏駢邑三百一成所以三百家者一成九百夫宫室塗巷山澤三分去一餘有六百夫又不易再易通率一家受二夫之田是定稅三百家也}
諸葛亮之使廖立垂泣李嚴致死豈徒無怨言而已哉夫水至平而邪者取法鑑至明而醜者忘怒水鑑之所以能窮物而無怨者以其無私也水鑑無私猶以免謗况大人君子懷樂生之心|{
	樂音洛}
流矜恕之德法行於不可不用刑加乎自犯之罪爵之而非私誅之而不怒天下有不服者乎

蜀人所在求為諸葛亮立廟漢主不聽|{
	為于偽翻}
百姓遂因時節私祭之於道陌上步兵校尉習隆等|{
	姓譜習國名後以為姓風俗通漢有習響為陳相}
上言請近其墓立一廟于沔陽|{
	近其靳翻}
斷其私祀|{
	斷音短}
漢主從之漢主以左將軍吳懿為車騎將軍假節督漢中|{
	代魏延也}
以丞相長史蔣琬為尚書令總統國事尋加琬行郡護假節領益州刺史時新喪元帥|{
	喪息浪翻}
遠近危悚琬出類拔萃|{
	類倫也萃聚也}
處羣僚之右|{
	處昌呂翻}
既無戚容又無喜色神守舉止有如平日由是衆望漸服吳人聞諸葛亮卒恐魏承衰取蜀增巴丘守兵萬人|{
	此巴丘即巴陵也今岳州巴陵縣有天岳山臨大江一名幕阜前有培塿謂之巴蛇冢相傳以為羿屠巴蛇於洞庭其骨若陵因謂之巴陵}
一欲以為救援二欲以事分割漢人聞之亦增永安之守以防非常漢主使右中郎將宗預使吳|{
	使疏吏翻}
吳主問曰東之與西譬猶一家而聞西更增白帝之守何也對曰臣以為東益巴丘之戍西增白帝之守皆事埶宜然俱不足以相問也吳主大笑嘉其抗盡|{
	謂抗言不為吳屈又盡情無所隱也}
禮之亞於鄧芝|{
	蜀先主殂諸葛亮當國始遣鄧芝使吳}
吳諸葛恪以丹陽山險民多果勁雖前發兵徒得外縣平民而已|{
	陸遜先嘗部伍山越為兵事見六十八卷漢獻帝建安二十四年}
其餘深遠莫能禽盡屢自求為官出之|{
	為于偽翻}
三年可得甲士四萬衆議咸以為丹陽地埶險阻與吳郡會稽新都番陽四郡隣接|{
	會工外翻番蒲何翻}
周旋數千里山谷萬重|{
	重直龍翻}
其幽邃人民未嘗入城邑對長吏|{
	長知兩翻}
皆仗兵野逸白首於林莽|{
	莽莫補翻又母黨翻草深曰莽}
逋亡宿惡咸共逃竄山出銅鐵自鑄甲兵俗好武習戰|{
	好呼到翻}
高尚氣力其升山越險抵突叢棘若魚之走淵猿狖之騰木也|{
	走音奏狖余救翻說文曰狖鼠屬善旋}
時觀間隙|{
	間古莧翻}
出為寇盜每致兵征伐尋其窟藏其戰則蠭至敗則鳥竄自前世以來不能羈也皆以為難恪父瑾聞之亦以事終不逮|{
	逮及也謂恪所出山民終不能及四萬之數也}
歎曰恪不大興吾家將赤吾族也恪盛陳其必捷吳主乃拜恪撫越將軍|{
	以招撫山越為將軍號}
領丹陽太守使行其策 冬十一月洛陽地震 吳潘濬討武陵蠻數年斬獲數萬自是羣蠻衰弱一方寧靜十一月濬還武昌|{
	太和五年吳遣潘濬討武陵蠻}


資治通鑑卷七十二
