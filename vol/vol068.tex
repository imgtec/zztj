






























































資治通鑑卷六十八   宋 司馬光 撰

胡三省 音註

漢紀六十|{
	起強圉作噩盡屠維大淵獻凡三年}


孝獻皇帝癸

建安二十二年春正月魏王操軍居巢|{
	居巢縣屬廬江郡春秋之巢國宋白曰今無為軍本巢縣之無為鎮曹操攻吳築城于此無功而退因號無為城臨濡須水上壖地秦漢為居巢春秋但各巢辭有詳畧耳 考異曰孫權傳曹公次居巢攻濡須並在去冬今從魏武紀}
孫權保濡須二月操進攻之|{
	孫權所保者十七年所築濡須塢也}
初右護軍蔣欽屯宣城|{
	宣城縣屬丹陽郡賢曰故城在今宣州南陵縣東}
蕪湖令徐盛收欽屯吏表斬之|{
	蕪湖縣屬丹陽郡春秋吳鳩兹之地宋白曰以其地卑畜水非深而生蕪藻故曰蕪湖}
及權在濡須欽與呂蒙持諸軍節度欽每稱徐盛之善權問之欽曰盛忠而勤彊有膽略器用好萬人督也今大事未定臣當助國求才豈敢挾私恨以蔽賢乎權善之三月操引軍還留伏波將軍夏侯惇都督曹仁張遼等二十六軍屯居巢|{
	晉志曰光武建武初征伐四方始置督軍御史事竟罷建安中魏武為相始遣大將軍督之二十一年命夏侯惇督二十六軍是也蕭子顯曰漢順帝時御史中丞馮赦討九江賊督揚徐二州軍事何徐宋志云起魏武王珪之職儀云起光武並非也}
權令都尉徐詳詣操請降操報使修好誓重結婚|{
	降戶江翻使疏吏翻好呼到翻重直龍翻}
權留平虜將軍周泰督濡須|{
	平虜將軍蓋孫氏創置}
朱然徐盛等皆在所部以泰寒門不服|{
	寒門言所出微也}
權會諸將大為酣樂命泰解衣權手自指其創痕|{
	樂音洛創初良翻}
問以所起泰輒記昔戰鬭處以對畢使復服權把其臂流涕曰幼平|{
	周泰字幼平}
卿為孤兄弟|{
	為於偽翻}
戰如熊虎不惜軀命被創數十|{
	被皮義翻}
膚如刻畫孤亦何心不待卿以骨肉之恩委卿以兵馬之重乎|{
	周泰傳權住宣城忽畧不治圍落山賊卒至權始上馬賊鋒刃已交泰投身衛權身被十二創是日微泰權幾危又從討黄祖拒曹公攻曹仁皆有功故委之}
坐罷住駕使泰以兵馬道從|{
	坐才卧翻道讀曰導從才用翻}
鳴鼓角作鼔吹而出|{
	樂纂曰司馬法軍中之樂鼓笛為上使聞之者壯勇而樂和細絲高竹不可用也慮悲聲感人士卒思歸之故也唐紹曰鼔吹之樂以為軍容昔黄帝涿鹿有功以為警衛劉昫曰鼔吹木軍旅之音馬上奏之自漢以來北狄之樂摠歸鼔吹署余按漢制萬人將軍給鼔吹吹昌瑞翻}
於是盛等乃服 夏四月詔魏王操設天子旌旗出入稱警蹕 六月魏以軍師華歆為御史大夫|{
	華戶化翻}
冬十月命魏王操冕十有二旒乘金根車駕六馬設五時副車|{
	董巴輿服志曰金根車輪皆朱班重牙貳轂兩轄金薄繆龍為輿倚較文虎伏軾龍首銜軛左右吉陽筩鸞雀立衡文畫輈羽蓋華蚤建大旗十二斿畫日月升龍駕六馬象鑣鏤錫金鑁方釳插翟尾朱兼樊纓赤罽易茸金就十有二左纛以氂牛尾為之在左騑馬軛上大如斗是為德車五時車安立亦皆如之各如方色白馬者朱其髦尾為朱鬛云所御駕六餘皆駕四後從為副車晉志五時安立車亦建旗十二各隨車色立車則正豎其旗安車則邪注鑁亡范翻釳許乙翻鐵孔也鑁馬首飾}
魏以五官中郎將丕為太子初魏王操娶丁夫人無子妾劉氏生子昂卞氏生四子丕彰植熊王使丁夫人母養昂昂死於穰|{
	事見六十二卷建安二年}
丁夫人哭泣無節操怒而出之以卞氏為繼室植性機警多藝能才藻敏贍操愛之操欲以女妻丁儀|{
	妻七細翻}
丕以儀目眇|{
	眇者一目小}
諫止之儀由是怨丕與弟黄門侍郎廙|{
	晉百官志給事黄門侍郎秦官也漢以後並因之與侍中俱管門下衆事無員及晉置員四人廙逸職翻又羊至翻}
及丞相主簿楊修數稱臨菑侯植之才|{
	數所角翻}
勸操立以為嗣修彪之子也操以函密訪於外尚書崔琰露板答曰|{
	露板不封也}
春秋之義立子以長|{
	春秋公羊傳曰立嫡以長不以賢立子以貴不以長長知兩翻}
加五官將仁孝聰明宜承正統|{
	將即亮翻}
琰以死守之植琰之兄女婿也尚書僕射毛玠曰近者袁紹以嫡庶不分覆宗滅國廢立大事非所宜聞東曹掾邢顒曰以庶代宗先世之戒也願殿下深察之|{
	掾俞絹翻顒魚容翻}
丕使人問太中大夫賈詡以自固之術詡曰願將軍恢崇德度躬素士之業朝夕孜孜不違子道如此而已丕從之深自砥礪它日操屛人問詡|{
	屏必郢翻}
詡嘿然不對操曰與卿言而不答何也詡曰屬有所思|{
	屬之欲翻下右屬同}
故不即對耳操曰何思詡曰思袁本初劉景升父子也|{
	袁紹父子事見六十四卷六年七年劉表父子事見六十五卷十三年}
操大笑操嘗出征丕植並送路側植稱述功德發言有章左右屬目操亦悅焉丕悵然自失濟隂吳質耳語曰王當行流涕可也及辭丕涕泣而拜操及左右咸歔欷|{
	濟子禮翻歔音虛欷音希又許既翻}
於是皆以植多華辭而誠心不及也植既任性而行不自雕飾五官將御之以術矯情自飾宫人左右並為之稱說|{
	為于偽翻}
故遂定為太子左右長御賀卞夫人曰|{
	漢皇后宫有旁側長御}
將軍拜太子|{
	丕為五官將故稱之為將軍}
天下莫不喜夫人當傾府藏以賞賜|{
	藏徂浪翻}
夫人曰王自以丕年大故用為嗣我但當以免無教導之過為幸耳亦何為當重賜遺乎|{
	遺于季翻}
長御還具以語操|{
	語牛倨翻}
操悅曰怒不變容喜不失節故最為難太子抱議郎辛毗頸而言曰辛君知我喜不|{
	不讀曰否}
毗以告其女憲英憲英歎曰太子代君主宗廟社稷者也代君不可以不戚主國不可以不懼宜戚而懼而反以為喜何以能久魏其不昌乎|{
	女子之智識有男子不能及者}
久之臨菑侯植乘車行馳道中開司馬門出|{
	漢令乙騎乘車馬行馳道中已論者没入車馬改具又宫衛令出入司馬門者皆下是司馬門猶可得而出入也若魏制則司馬門惟車駕出乃開耳}
操大怒公車令坐死由是重諸侯科禁而植寵日衰植妻衣繡操登臺見之以違制命還家賜死|{
	以違制命罪植妻則當時蓋禁衣錦繡也衣于既翻}
法正說劉備曰|{
	說輸芮翻}
曹操一舉而降張魯定漢中|{
	降戶江翻}
不因此勢以圖巴蜀而留夏侯淵張郃屯守|{
	郃古合翻又曷閤翻}
身遽北還此非其智不逮而力不足也必將内有憂偪故耳今策淵郃才畧不勝國之將帥舉衆往討必可克之克之之日廣農積穀觀釁伺隙上可以傾覆寇敵尊奬王室中可以蠶食雍涼廣拓境土|{
	晉志曰漢改周之雍州為涼州以地處西方常寒涼也地勢西北邪出在南山之間南隔西羌西通西域于時號為斷匈奴右臂獻帝時涼州數亂河西五郡去州隔遠乃别立雍州末又依古典為九州乃令關右盡為雍州魏時復分以為涼州雍于用翻}
下可以固守要害為持久之計此蓋天以與我時不可失也備善其策乃率諸將進兵漢中遣張飛馬超吳蘭等屯下辨|{
	下辨縣屬武都郡賢曰今成州同谷縣師古曰辨音步見翻又步莧翻}
魏王操遣都護將軍曹洪拒之 魯肅卒孫權以從事中郎彭城嚴畯代肅|{
	畯音悛}
督兵萬人鎮陸口衆人皆為畯喜|{
	為於偽翻}
畯固辭以樸素書生不閑軍事|{
	閑習也}
發言懇惻至於流涕權乃以左護軍虎威將軍呂蒙兼漢昌太守以代之|{
	虎威將軍蓋孫權置沈約志曹魏置四十號將軍虎威第三十四}
衆嘉嚴畯能以實讓 定威校尉吳郡陸遜|{
	定威校尉亦權創置}
言於孫權曰方今克敵寧亂非衆不濟而山寇舊惡依阻深地|{
	舊惡謂自舊為惡者}
夫腹心未平難以圖遠可大部伍取其精銳|{
	言可大為部伍擇取精銳也}
權從之以為帳下右部督會丹陽賊帥費棧作亂|{
	費父沸翻姓也棧士限翻}
扇動山越權命遜討棧破之遂部伍東三郡|{
	東三郡丹陽新都會稽也}
彊者為兵羸者補戶|{
	羸倫為翻}
得精卒數萬人宿惡盪除|{
	盪徒朗翻}
所過肅清還屯蕪湖會稽太守淳于式表遜枉取民人愁擾所在|{
	言遜之所在民人皆愁擾也會工外翻}
遜後詣都言次稱式佳吏|{
	孫權時都秣陵言次謂言論之次猶今云語次}
權曰式白君而君薦之何也遜對曰式意欲養民是以白遜若遜復毁式以亂聖聽不可長也權曰此誠長者之事顧人不能為耳|{
	復扶又翻長知兩翻}
魏王操使丞相長史王必典兵督許中事|{
	魏王操猶領漢丞相而居鄴故以必為長史典兵督許}
時關羽彊盛京兆金禕覩漢祚將移乃與少府耿紀司直韋晃|{
	司直即丞相司直禕吁韋翻}
太醫令吉本|{
	風俗通吉周尹吉甫之後漢有漢中太守吉恪}
本子邈邈弟穆等謀殺必挾天子以攻魏南引關羽為援二十三年春正月吉邈等率其黨千餘人夜攻王必燒其門射必中肩|{
	射食亦翻中竹仲翻}
帳下督扶必犇南城|{
	許昌之南城也}
會天明邈等衆潰必與潁川典農中郎將嚴匡共討斬之|{
	潁川典農中郎將屯田許下}
三月有星孛於東方|{
	孛蒲内翻}
曹洪將擊吳蘭張飛屯固山聲言欲斷軍後|{
	斷丁管翻下同}
衆議狐疑騎都尉曹休曰|{
	漢武帝置三都尉騎都尉其一也}
賊實斷道者當伏兵潛行今乃先張聲勢此其不能明矣宜及其未集促擊蘭蘭破飛自走矣洪從之進擊破蘭斬之三月張飛馬超走|{
	情見勢屈宜其走也}
休魏王族子也 夏四月代郡上谷烏桓無臣氐等反先是魏王操召代郡太守裴潜為丞相理曹掾|{
	先悉薦翻掾于絹翻}
操美潜治代之功|{
	治直之翻}
潜曰潜於百姓雖寛於諸胡為峻今繼者必以潜為治過嚴而事加寛惠|{
	治直吏翻}
彼素驕恣過寛必弛既弛將攝之以法|{
	攝持也整也}
此怨叛所由生也以埶料之代必復叛|{
	後魏陸侯治高車與潜異世而同轍復扶又翻}
於是操深悔還潜之速後數十日三單于反問果至操以其子鄢陵侯彰行驍騎將軍|{
	鄢陵縣屬潁川郡驍騎將軍始於漢武帝以命李廣陸德明曰鄢謁晩翻又于建翻漢書作傿師古曰音偃}
使討之彰少善射御膂力過人|{
	少詩照翻}
操戒彰曰居家為父子受事為君臣動以王灋從事爾其戒之 劉備屯陽平關夏侯淵張郃徐晃等與之相拒備遣其將陳式等絶馬鳴閣道|{
	馬鳴閣在今利州昭化縣}
徐晃擊破之張郃屯廣石|{
	廣石當在巴漢之間}
備攻之不能克急書發益州兵諸葛亮以問從事犍為楊洪洪曰漢中益州咽喉|{
	犍居言翻咽音烟}
存亡之機會若無漢中則無蜀矣此家門之禍也發兵何疑時法正從備北行亮於是表洪領蜀郡太守衆事皆辦遂使即真|{
	遂使之代法正}
初犍為太守李嚴辟洪為功曹嚴未去犍為而洪已為蜀郡洪舉門下書佐何祗有才策|{
	漢制郡閣下及諸曹各有書佐幹主文書靈帝光和二年樊毅復華下民租口算碑載其上尚書奏牘前書年月朔日弘農太守臣毅頓首死罪上尚書後書臣毅誠惶誠恐頓首頓首死罪死罪上尚書後繫掾臣條屬臣淮書佐臣謀}
洪尚在蜀郡而祗已為廣漢太守是以西土咸服諸葛亮能盡時人之器用也秋七月魏王操自將擊劉備九月至長安 曹彰擊代郡烏身自搏戰鎧中數箭|{
	鎧可亥翻中竹仲翻}
意氣益厲乘勝逐北至桑乾之北|{
	桑乾縣屬代郡宋白曰今雲州東至桑乾督帳一百五十里孟康曰乾音干}
大破之斬首獲生以千數時鮮卑大人軻比能|{
	軻比能本小種鮮卑以勇健不貪斷法平端衆推之為大人}
將數萬騎觀望彊弱見彰力戰所向皆破乃請服北方悉平 南陽吏民苦繇役|{
	繇讀曰徭苦於供給曹仁之軍也}
冬十月宛守將侯音反|{
	宛於元翻}
南陽太守東里衮|{
	鄭子產居東里支子以為氏}
與功曹應余迸竄得出音遣騎追之飛矢交流余以身蔽衮被七創而死|{
	被皮義翻創初良翻}
音騎執衮以歸時征南將軍曹仁屯樊以鎮荆州魏王操命仁還討音功曹宗子卿說音曰|{
	說輸芮翻}
足下順民心舉大事遠近莫不望風然執郡將|{
	將即亮翻}
逆而無益何不遣之音從之子卿因夜踰城從太守收餘民圍音會曹仁軍至共攻之

二十四年春正月曹仁屠宛斬侯音復屯樊|{
	復扶又翻}
初夏侯淵戰雖數勝|{
	數所角翻}
魏王操常戒之曰為將當有怯弱時不可但恃勇也將當以勇為本行之以智計但知任勇一匹夫敵耳及淵與劉備相拒踰年備自陽平南渡沔水緣山稍前營於定軍山|{
	華陽國志曰漢中沔陽縣有定軍山北臨沔水據灋正傳於定軍興勢作營則定軍山正在興勢也今按興勢山在洋州興道縣西北二十里去沔陽地里相遠當從華陽國志 考異曰備傳云於定軍山勢作營法正傳作定軍興勢今從黄忠傳}
淵引兵爭之灋正曰可擊矣備使討虜將軍黄忠乘高鼓譟攻之淵軍大敗斬淵 |{
	考異曰淵傳曰備夜燒圍鹿角淵使張郃護東圍自將輕兵護南圍備挑郃戰郃軍不利淵分兵半助郃為備所襲戰死張郃傳曰備于走馬谷燒都圍淵救火從他道與備相遇交戰短兵接刃淵遂没今從劉備黄忠灋正傳}
及益州刺史趙顒|{
	顒刺益州操所命也淵軍既敗顒亦死顒魚容翻}
張郃引兵還陽平|{
	自廣石還陽平}
是時新失元帥軍中擾擾不知所為督軍杜襲|{
	初操東還留襲督漢中軍事帥所類翻}
與淵司馬太原郭淮收歛散卒號令諸軍曰張將軍國家名將劉備所憚今日事急非張將軍不能安也遂權宜推郃為軍主郃出勒兵按陳|{
	陳讀曰陣下同}
諸將皆受郃節度衆心乃定明日備欲渡漢水來攻諸將以衆寡不敵欲依水為陳以拒之郭淮曰此示弱而不足挫敵非算也不如遠水為陳引而致之半濟而後擊之備可破也既陳備疑不渡淮遂堅守示無還心以狀聞於魏王操操善之遣使假郃節復以淮為司馬 二月壬子晦日有食之 三月魏王操自長安出斜谷軍遮要以臨漢中|{
	斜谷道險操恐為備所邀截先以軍遮要害之處乃進臨漢中或云遮要地名}
劉備曰曹公雖來無能為也我必有漢川矣乃歛衆拒險終不交鋒操運米北山下黄忠引兵欲取之過期不還翊軍將軍趙雲將數十騎出營視之|{
	翊軍將軍備所創置也}
值操揚兵大出雲猝與相遇遂前突其陳且鬭且却魏兵散而復合追至營下雲入營更大開門偃旗息鼔魏兵疑雲有伏引去雲雷鼔震天|{
	雷盧對翻}
惟以勁弩於後射魏兵|{
	射而亦翻}
魏兵驚駭自相蹂踐墯漢水中死者甚多|{
	蹂人九翻}
備明旦自來至雲營視昨戰處曰子龍一身都是膽也|{
	言其膽大能以孤軍亢操大兵}
操與備相守積月魏軍士多亡|{
	亡逃亡也}
夏五月操悉引出漢中諸軍還長安劉備遂有漢中操恐劉備北取武都氐以逼關中|{
	武都本白馬氐地}
問雍州刺史張既既曰可勸使北出就穀以避賊前至者厚其寵賞則先者知利後必慕之操從之使既之武都徙氐五萬餘落出居扶風天水界|{
	操蓋已棄武都而不有矣諸氐散居秦川符氏亂華自此始}
武威顔俊張掖和鸞酒泉黄華西平麴演等各據其郡自號將軍更相攻擊俊遣使送母及子詣魏王操為質以求助|{
	更工衡翻質音致}
操問張既既曰俊等外假國威内生傲悖|{
	悖蒲内翻又蒲設翻}
計定勢足後即反耳今方事定蜀且宜兩存而鬭之猶卞莊子之刺虎坐收其敝也|{
	戰國策曰卞莊子刺虎管豎子止之曰兩虎方食牛牛甘必爭鬭則大者傷小者亡從傷刺之一舉必有兩獲莊子然之果獲二虎刺七亦翻}
王曰善歲餘鸞遂殺俊武威王祕又殺鸞 劉備遣宜都太守扶風孟達從秭歸北攻房陵殺房陵太守蒯祺|{
	張勃吳録曰劉備分南郡立宜都郡領夷道狼山夷陵三縣房陵縣本屬漢中郡此郡疑劉表所置使蒯祺守之否則祺自立也蒯苦怪翻}
又遣養子副軍中郎將劉封自漢中乘沔水下統達軍|{
	劉封本羅侯寇氏之子長沙劉氏之甥備至荆州以未有繼嗣養之為子}
與達會攻上庸上庸太守申耽舉郡降|{
	上庸縣屬漢中郡賢曰故城在今房州清水縣西魏畧曰申耽初在西城上庸間聚衆數千家與張魯通又遣使詣曹公公加其號為將軍使領上庸都尉降戶江翻}
備加耽征北將軍領上庸太守以耽弟儀為建信將軍西城太守|{
	西城縣屬漢中郡備亦分為郡以授儀唐為金州}
秋七月劉備自稱漢中王設壇場於沔陽|{
	沔陽縣屬漢中郡}
陳兵列衆羣臣陪位讀奏訖乃拜受璽綬御王冠|{
	璽斯氏翻綬音受王冠遠游冠也}
因驛拜章上還所假左將軍宜城亭侯印綬|{
	左將軍及宜城亭侯皆操所表授也上時掌翻}
立子禪為王太子拔牙門將軍義陽魏延為鎮遠將軍|{
	牙門鎮遠皆劉備創置將軍號}
領漢中太守以鎮漢川|{
	魏文帝分南陽郡立義陽郡又立義陽縣屬焉此在延入蜀之後史追書也鎮遠將軍蓋備所創置宋白曰義陽唐為申州宋為信陽軍}
備還治成都以許靖為太傅灋正為尚書令關羽為前將軍張飛為右將軍馬超為左將軍黄忠為後將軍|{
	前後左右將軍皆漢官}
餘皆進位有差遣益州前部司馬犍為費詩即授關羽印綬|{
	犍居言翻費父沸翻}
羽聞黄忠位與已並怒曰大丈夫終不與老兵同列不肯受拜詩謂羽曰夫立王業者所用非一昔蕭曹與高祖少小親舊|{
	少詩照翻}
而陳韓亡命後至論其班列韓最居上|{
	謂陳平韓信自楚而來韓信王而蕭曹侯故曰韓最居上}
未聞蕭曹以此為怨今漢中王以一時之功隆崇漢室然意之輕重寧當與君侯齊乎|{
	言備以一時使忠與羽班而意之輕重則不在此曹操嘗表羽為漢壽亭侯故稱之為君侯}
且王與君侯譬猶一體同休等戚禍福共之愚謂君侯不宜計官號之高下爵禄之多少為意也僕一介之使|{
	使疏吏翻}
銜命之人君侯不受拜如是便還但相為惜此舉動|{
	為于偽翻}
恐有後悔耳羽大感悟遽即受拜 詔以魏王操夫人卞氏為王后 孫權攻合肥時諸州兵戍淮南|{
	魏改漢九江郡為淮南郡}
揚州刺史温恢謂兖州刺史裴潜曰此間雖有賊然不足憂今水潦方生而子孝縣軍無有遠備|{
	曹仁字子孝時為征南將軍縣讀曰懸}
關羽驍猾政恐征南有變耳|{
	驍堅堯翻}
已而關羽果使南郡太守糜芳守江陵將軍傅士仁守公安羽自率衆攻曹仁於樊仁使左將軍于禁立義將軍龎德等屯樊北|{
	操以龎德自漢中來歸故進號立義將軍}
八月大霖雨漢水溢平地數丈于禁等七軍皆没禁與諸將登高避水羽乘大船就攻之禁等窮迫遂降|{
	降戶江翻下同}
龎德在隄上被甲持弓箭不虛發|{
	射必中也龎皮江翻被皮義翻}
自平旦力戰至日過中羽攻益急矢盡短兵接德戰益怒氣愈壯而水浸盛吏士盡降|{
	降戶江翻下同}
德乘小船欲還仁營水盛船覆失弓矢獨抱船覆水中為羽所得立而不跪|{
	示不屈伏}
羽謂曰卿兄在漢中|{
	魏畧曰德從兄柔在蜀}
我欲以卿為將不早降何為德罵羽曰豎子何謂降也魏王帶甲百萬威振天下汝劉備庸才耳豈能敵邪我寧為國家鬼不為賊將也羽殺之|{
	將即亮翻}
魏王操聞之曰吾知于禁三十年|{
	操牧兵兖州禁即為將}
何意臨危處難|{
	處昌呂翻難乃旦翻}
反不及龎德邪封德二子為列侯羽急攻樊城城得水往往崩壞衆皆恟懼|{
	恟許勇翻}
或謂曹仁曰今日之危非力所支可及羽圍未合乘輕船夜走汝南太守滿寵曰山水速疾冀其不久聞羽遣别將已在郟下|{
	寵為汝南太守操令助曹仁屯樊城郟縣屬潁川郡師古曰郟音夾晉地理志襄城郡復有郟縣蓋東漢省而魏晉復置縣也}
自許以南百姓擾擾羽所以不敢遂進者恐吾軍掎其後耳|{
	掎居蟻翻}
今若遁去洪河以南非復國家有也|{
	洪河大河也}
君宜待之仁曰善乃沈白馬與軍人盟誓|{
	沈持林翻}
同心固守城中人馬纔數千人城不没者數板|{
	城高二尺為一板}
羽乘船臨城立圍數重|{
	重直龍翻}
外内斷絶羽又遣别將圍將軍呂常於襄陽荆州刺史胡修南鄉太守傅方皆降於羽|{
	水經注漢建安中割南陽右壤為南鄉郡屬荆州}
初沛國魏諷有惑衆才傾動鄴都魏相國鍾繇辟以為西曹掾|{
	此魏相國府之西曹掾也}
滎陽任覽與諷友善同郡鄭袤|{
	袤音茂}
泰之子也每謂覽曰諷姦雄終必為亂九月諷潜結徒黨與長樂衛尉陳禕謀襲鄴|{
	樂音洛禕吁韋翻}
未及期禕懼而告之太子丕誅諷連坐死者數千人鍾繇坐免官 初丞相主簿楊修與丁儀兄弟謀立曹植為魏嗣|{
	修為漢丞相主簿操官屬也}
五官將丕患之以車載廢簏内朝歌長吳質與之謀|{
	長知兩翻}
修以白魏王操操未及推驗丕懼告質質曰無害也明日復以簏載絹以入修復白之推驗無人|{
	推案也復扶又翻}
操由是疑焉其後植以驕縱見疏|{
	植乘車行馳道中私開司馬門出既得罪矣曹仁為關羽所圍操欲遣植救仁而植醉不能受命於是益見疏}
而植故連綴修不止修亦不敢自絶每當就植慮事有闕忖度操意|{
	忖寸本翻度徒洛翻}
豫作答教十餘條敕門下教出隨所問答之於是教裁出答已入操怪其捷推問始泄操亦以修袁術之甥惡之|{
	惡烏路翻}
乃發脩前後漏泄言教交關諸侯|{
	以修豫作答教謂之漏泄與植往來謂之交關諸侯}
收殺之 魏王操以杜襲為留府長史駐關中|{
	置留府于關中者以備蜀也}
關中營帥許攸|{
	帥所類翻此又一許攸非自袁紹來奔之許攸也}
擁部曲不歸附而有慢言操大怒先欲伐之羣臣多諫宜招懷攸共討彊敵操横刀於䣛|{
	䣛與膝同}
作色不聽襲入欲諫操逆謂之曰吾計已定卿勿復言|{
	復扶又翻}
襲曰若殿下計是邪臣方助殿下成之若殿下計非邪雖成宜改之殿下逆臣令勿言何待下之不闡乎|{
	闡開也大也明也}
操曰許攸慢吾如何可置|{
	置捨也}
襲曰殿下謂許攸何如人邪操曰凡人也襲曰夫惟賢知賢惟聖知聖凡人安能知非凡人邪方今豺狼當路而狐狸是先人將謂殿下避彊攻弱進不為勇退不為仁臣聞千鈞之弩不為鼷鼠發機萬石之鐘不以莛撞起音|{
	三十斤為鈞千鈞之弩言其重也鼷鼠小鼠也說文曰有螫毒者或謂之甘鼠陸佃埤雅曰鼷鼠者甘口齧人及鳥獸皆不痛博物志云鼠之最小者本草說鼷鼠極細不可卒見四斤為石石百二十斤也莛草莖也東方朔曰以莛撞鐘是皆言力勢重者不以輕觸而發動也鼷音奚莛音廷撞直江翻}
今區區之許攸何足以勞神武哉操曰善遂厚撫攸攸即歸服 冬十月魏王操至洛陽 陸渾民孫狼等作亂|{
	陸渾縣屬弘農郡秦晉遷陸渾之戎于此宋白曰陸渾河南府伊陽縣地師古曰渾音胡昆翻}
殺縣主簿南附關羽羽授狼印給兵還為寇賊自許以南往往遙應羽羽威震華夏|{
	夏戶雅翻}
魏王操議徙許都以避其銳丞相軍司馬司馬懿西曹屬蔣濟言於操曰于禁等為水所没非戰攻之失於國家大計未足有損劉備孫權外親内疎關羽得志權必不願也可遣人勸權躡其後許割江南以封權則樊圍自解操從之初魯肅嘗勸孫權以曹操尚存宜且撫輯關羽與之同仇不可失也及呂蒙代肅屯陸口以為羽素驍雄有兼并之心|{
	驍堅堯翻}
且居國上流其勢難久密言於權曰今令征虜守南郡|{
	孫皎時為征虜將軍}
潘璋住白帝|{
	北即甘寧據楚關之計也}
蔣欽將游兵萬人循江上下應敵所在蒙為國家前據襄陽|{
	為于偽翻}
如此何憂於操何賴于羽且羽君臣矜其詐力所在反覆不可以腹心待也今羽所以未便東向者以至尊聖明蒙等尚在也今不於彊壯時圖之一旦僵仆欲復陳力其可得邪|{
	僵仆謂死也復扶又翻}
權曰今欲先取徐州|{
	自廣陵以北皆徐州之地}
然後取羽何如對曰今操遠在河北撫集幽冀未暇東顧徐土守兵聞不足言|{
	曹操審知天下之勢慮此熟矣此兵法所謂城有所不守也}
往自可克然地勢陸通驍騎所騁|{
	騁丑郢翻}
至尊今日取徐州操後旬必來争雖以七八萬人守之猶當懷憂|{
	呂蒙自量吳國之兵力不足北向以爭中原者知車騎之地非南兵之所便也}
不如取羽全據長江形勢益張易為守也權善之|{
	易以䜴翻}
權嘗為其子求昏於羽|{
	為于偽翻}
羽罵其使不許昏|{
	使疏吏翻}
權由是怒及羽攻樊呂蒙上疏曰羽討樊而多留備兵必恐蒙圖其後故也蒙嘗有病乞分士衆還建業以治疾為名|{
	治直之翻}
羽聞之必撤備兵盡赴襄陽大軍浮江晝夜馳上|{
	上時掌翻}
襲其空虚則南郡可下而羽可禽也|{
	此南郡謂江陵}
遂稱病篤權乃露檄召蒙還|{
	露檄欲使羽知之}
隂與圖計蒙下至蕪湖定威校尉陸遜謂蒙曰關羽接境如何遠下後不當可憂也蒙曰誠如來言然我病篤遜曰羽矜其驍氣陵轢于人|{
	轢郎狄翻}
始有大功意驕志逸但務北進未嫌於我有相聞病必益無備今出其不意自可禽制下見至尊宜好為計|{
	英雄之士所見畧同蒙所以知其意思深長也}
蒙曰羽素勇猛既難為敵且已據荆州恩信大行兼始有功膽勢益盛未易圖也|{
	兵事尚密遜之言雖當蒙之心蒙未敢容易為遜言之易以䜴翻}
蒙至都權問誰可代卿者蒙對曰陸遜意思深長|{
	思相吏翻}
才堪負重觀其規慮終可大任而未有遠名非羽所忌無復是過也|{
	復扶又翻下同}
若用之當令外自韜隱内察形便然後可克權乃召遜拜偏將軍右部督以代蒙遜至陸口為書與羽稱其功美深自謙抑為盡忠自託之意羽意大安無復所嫌稍撤兵以赴樊|{
	果墮蒙計}
遜具啓形狀陳其可禽之要羽得于禁等人馬數萬糧食乏絶擅取權湘關米|{
	吳與蜀分荆州以湘水為界故置關}
權聞之遂發兵襲羽權欲令征虜將軍孫皎與呂蒙為左右部大督|{
	征虜將軍始於光武以命祭遵}
蒙曰若至尊以征虜能宜用之以蒙能宜用蒙昔周瑜程普為左右部督督兵攻江陵雖事決于瑜普自恃久將|{
	將即亮翻}
且俱是督遂共不睦幾敗國事此目前之戒也|{
	事見六十六卷建安十五年幾居希翻敗補邁翻}
權寤謝蒙曰以卿為大督命皎為後繼可也魏王操之出漢中也使平寇將軍徐晃屯宛以助曹仁|{
	平寇將軍蓋亦曹操所置考沈約志不在四十號之數}
及于禁陷没晃前至陽陵陂關羽遣兵屯偃城|{
	括地志偃城在襄州安養縣北三里古郾子之國}
晃既到詭道作都塹示欲截其後羽兵燒屯走|{
	詭道出偃城之後通為長塹故曰都塹}
晃得偃城連營稍前操使趙儼以議郎參曹仁軍事與徐晃俱前餘救兵未到晃所督不足解圍而諸將呼責晃促救仁儼謂諸將曰今賊圍素固水潦猶盛我徒卒單少|{
	少詩沼翻}
而仁隔絶不得同力此舉適所以敝内外耳當今不若前軍偪圍遣諜通仁使知外救以勵將士計北軍不過十日尚足堅守然後表裏俱發破賊必矣如有緩救之戮余為諸君當之|{
	為于偽翻}
諸將皆喜晃營距羽圍三丈所作地道及箭飛書與仁消息數通|{
	消者浸微浸滅之意息者漸生漸長之意消息數通則城内城外各知安否也晃營迫羽圍如此而不能制使呂蒙不襲取江陵羽亦必為操所破而操假手於蒙者欲使兩寇自敝而坐收漁人田父之功也數所角翻}
孫權為牋與魏王操請以討羽自效及乞不漏令羽有備操問羣臣羣臣咸言宜密之董昭曰軍事尚權期於合宜宜應權以密而内露之羽聞權上若還自護圍則速解便獲其利可使兩賊相對銜持|{
	以馬為喻也兩馬欲相鞮齧既加之銜勒兩不能動矣而欲鬭之氣未衰相對銜持則兩雖跳梁力必自敝上時掌翻}
坐待其敝祕而不露使權得志非計之上又圍中將吏不知有救計糧怖懼|{
	計城中之糧不足以持久則心懷怖懼也怖普布翻}
儻有他意為難不小|{
	難乃旦翻}
露之為便且羽為人彊梁自恃二城守固必不速退操曰善即敕徐晃以權書射著圍裏及羽屯中|{
	射而亦翻著直畧翻}
圍裏聞之志氣百倍羽果猶豫不能去|{
	羽雖見權書自恃江陵公安守固非權旦夕可拔又因水勢結圍以臨樊城有必破之勢釋之而去必喪前功此其所以猶豫也}
魏王操自雒陽南救曹仁羣下皆謂王不亟行今敗矣侍中桓階獨曰大王以仁等為足以料事勢不也|{
	不讀曰否}
曰能大王恐二人遺力邪|{
	二人謂曹仁呂常也}
曰不然然則何為自往曰吾恐虜衆多而徐晃等勢不便耳階曰今仁等處重圍之中|{
	重直龍翻下同}
而守死無貳者誠以大王遠為之勢也夫居萬死之地必有死爭之心内懷死爭外有彊救大王案六軍以示餘力何憂於敗而欲自往操善其言乃駐軍摩陂|{
	據水經摩陂在潁川郟縣縱廣可一十五里魏青龍元年有龍見于陂于是改曰龍陂}
前後遣殷署朱蓋等凡十二營詣晃關羽圍頭有屯又别屯四冢晃乃揚聲當攻圍頭屯而密攻四冢欲壞自將步騎五千出戰|{
	自將之上有羽字文意乃明}
晃擊之退走羽圍塹鹿角十重|{
	重直龍翻}
晃追羽與俱入圍中破之傳方胡修皆死羽遂撤圍退然舟船猶據沔水襄陽隔絶不通呂蒙至尋陽盡伏其精兵中|{
	居侯翻盧谷翻博雅曰舟也}
使白衣搖櫓作商賈人服|{
	賈音古}
晝夜兼行羽所置江邊屯候盡收縛之是故羽不聞知|{
	屯侯雖被收縛使麋傅無叛心羽猶可得聞知也}
糜芳士仁素皆嫌羽輕已羽之出軍芳仁供給軍資不悉相及羽言還當治之|{
	治直之翻}
芳仁咸懼於是蒙令故騎都尉虞翻|{
	權以翻為騎都尉以謗徙丹陽蒙請以自隨時無官爵故稱故官}
為書說仁為陳成敗|{
	說輸芮翻為于偽翻}
仁得書即降|{
	降戶江翻下同}
翻謂蒙曰此譎兵也|{
	謂蒙以詭計行兵也譎古穴翻}
當將仁行留兵備城遂將仁至南郡|{
	將如字}
糜芳城守蒙以仁示之芳遂開門出降蒙入江陵釋于禁之囚得關羽及將士家屬皆撫慰之約令軍中不得干歷人家有所求取蒙麾下士與蒙同郡人取民家一笠以覆官鎧|{
	覆敷救翻}
官鎧雖公蒙猶以為犯軍令不可以鄉里故而廢灋遂垂涕斬之於是軍中震悚道不拾遺蒙旦暮使親近存恤耆老問所不足疾病者給醫藥饑寒者賜衣糧羽府藏財寶皆封閉以待權至|{
	藏徂浪翻}
關羽聞南郡破即走南還|{
	還從宣翻又如字}
曹仁會諸將議咸曰今因羽危懼可追禽也趙儼曰權邀羽連兵之難|{
	邀當作徼徼幸也難乃旦翻謂與曹仁連兵}
欲掩制其後顧羽還救恐我乘其兩疲故順辭求效|{
	求效猶言求自效也或曰巽順其辭以求成效}
乘釁因變以觀利鈍耳今羽已孤迸|{
	言羽失根本而埶孤犇迸也}
更宜存之以為權害若深入追北權則改虞於彼將生患于我矣|{
	虞度也防也謂度羽不能為害則改其防羽之心而防操則必為操之患矣}
王必以此為深慮仁乃解嚴|{
	趙儼之計此戰國策士所謂兩利而俱存之之計也解嚴解所嚴兵不復追羽也是後陸遜敗劉備于峽中收兵而還不復追備計亦出此}
魏王操聞羽走恐諸將追之果疾敕仁如儼所策關羽數使人與呂蒙相聞|{
	數所角翻}
蒙輒厚遇其使|{
	使疏吏翻}
周遊城中家家致問或手書示信羽人還私相參訊|{
	訊問也}
咸知家門無恙見待過于平時故羽吏士無鬭心|{
	呂蒙所以禽關羽者摧之而已恙金亮翻}
會權至江陵荆州將吏悉皆歸附獨治中從事武陵潘濬稱疾不見權遣人以牀就家輿致之濬伏面著牀席不起涕泣交横哀哽不能自勝|{
	著直畧翻勝音升}
權呼其字與語|{
	潘濬字承明}
慰諭懇惻使親近以手巾拭其面濬起下地拜謝即以為治中荆州軍事一以諮之|{
	郝普糜芳傅士仁之在吳未有所聞也而潘濬所以自見者與陸遜諸葛瑾班識者當于此而觀人}
武陵部從事樊伷誘導諸夷圖以武陵附漢中王備|{
	漢制州牧刺史部諸郡各郡置部從事伷與胄同誘音酉}
外白差督督萬人往討之|{
	差初佳翻擇也督將也}
權不聽特召問濬濬答以五千兵往足以擒伷權曰卿何以輕之濬曰伷是南陽舊姓|{
	南陽之樊光武之母黨故謂之舊姓}
頗能弄唇吻而實無才畧|{
	今人以辨給觀人才何其謬也吻武粉翻口邊曰吻}
臣所以知之者伷昔嘗為州人設饌|{
	為于偽翻饌雛戀翻又雛皖翻}
比至日中|{
	比必寐翻}
食不可得而十餘自起此亦侏儒觀一節之驗也|{
	侏儒優人以能諧笑取寵觀其一節足以驗其技}
權大笑即遣濬將五千人往果斬平之權以呂蒙為南郡太守封孱陵侯|{
	孱士連翻}
賜錢一億黄金五百斤以陸遜領宜都太守|{
	吳録曰蜀昭烈帝立宜都郡于西陵即夷陵也唐為峽州夷陵郡}
十一月漢中王備所置宜都太守樊友委郡走諸城長吏及蠻夷君長皆降於遜|{
	長知兩翻}
遜請金銀銅印以假授初附擊蜀將詹晏等|{
	詹姓也周有詹父楚有詹尹}
及秭歸大姓擁兵者皆破降之前後斬獲招納凡數萬計權以遜為右護軍鎮西將軍進封婁侯屯夷陵守峽口|{
	婁縣前漢屬會稽郡後漢屬吳郡范成大吳郡志婁縣今謂之崑山縣東北三里有村落名婁縣蓋古婁縣治所也峽口西陵峽口也宜都記曰自黄牛灘東入西陵界至峽口一百許里山水紆曲兩岸高山重嶂非日中夜半不見日月}
關羽自知孤窮乃西保麥城|{
	荆州記曰南郡當陽縣東南有麥城}
孫權使誘之羽偽降|{
	誘音酉降戶江翻}
立幡旗為象人於城上因遁走兵皆解散纔十餘騎權先使朱然潘璋斷其逕路|{
	斷丁管翻}
十二月璋司馬馬忠獲羽及其子平於章鄉|{
	水經注漳水出臨沮縣東荆山南逕臨沮縣之漳鄉南潘璋禽關羽於此漳水又南逕當陽縣又南逕麥城東}
斬之遂定荆州初偏將軍吳郡全琮|{
	全姓琮名}
上疏陳關羽可取之計權恐事泄寢而不答及已禽羽權置酒公安顧謂琮曰君前陳此孤雖不相答今日之捷抑亦君之功也於是封琮陽華亭侯權復以劉璋為益州牧駐秭歸未幾璋卒|{
	劉備入益州遷璋于公安今為權所得幾居豈翻}
呂蒙未及受封而疾發權迎置於所館之側所以治護者萬方時有加鍼權為之慘慼|{
	治直之翻為于偽翻}
欲數見其顔色|{
	數所角翻}
又恐勞動常穿壁瞻之見小能下食則喜顧左右不然則咄唶|{
	咄當没翻咨也唶子夜翻嘆也}
夜不能寐病中瘳為下赦令|{
	為于偽翻下同}
羣臣畢賀已而竟卒年四十二權哀痛殊甚為置守冢三百家權後與陸遜論周瑜魯肅及蒙曰公瑾雄烈膽畧兼人遂破孟德開拓荆州邈焉寡儔子敬因公瑾致達於孤孤與晏語便及大畧帝王之業此一快也|{
	事見六十三卷五年}
後孟德因獲劉琮之勢張言方率數十萬衆水步俱下|{
	張言者張大而言之}
孤普請諸將咨問所宜無適先對|{
	無適先對猶言莫適先對也適音的}
至張子布秦文表|{
	秦松字文表}
俱言宜遣使修檄迎之子敬即駮言不可|{
	駮異也立異議以糾駮衆議之非駮北角翻}
勸孤急呼公瑾付任以衆逆而擊之此二快也|{
	事見六十五卷十三年}
後雖勸吾借玄德地|{
	事見六十六卷十五年}
是其一短不足以損其二長也周公不求備於一人|{
	論語載周公語魯公之言}
故孤忘其短而貴其長常以比方鄧禹也|{
	鄧禹建策以開光武中興之業而其後不能定赤眉故以肅比之}
子明少時|{
	呂蒙字子明少詩照翻}
孤謂不辭劇易|{
	劇艱也易以䜴翻}
果敢有膽而已及身長大|{
	長知兩翻}
學問開益籌畧奇至可以次于公瑾但言議英發不及之耳圖取關羽勝於子敬子敬答孤書云帝王之起皆有驅除羽不足忌|{
	謂關羽之強適足為吳之驅除也}
此子敬内不能辦外為大言耳孤亦恕之不苟責也然其作軍屯營不失令行禁止部界無廢負|{
	謂部界之内無有廢職以為罪負也}
路無拾遺其灋亦美矣孫權與于禁乘馬併行|{
	併讀曰並}
虞翻呵禁曰汝降虜|{
	降戶江翻}
何敢與吾君齊馬首乎抗鞭欲擊禁|{
	抗舉也}
權呵止之 孫權之稱藩也魏王操召張遼等諸軍悉還救樊未至而圍解徐晃振旅還摩陂操迎晃七里置酒大會王舉酒謂晃曰全樊襄陽將軍之功也亦厚賜桓階以為尚書操嫌荆州殘民及其屯田在漢川者|{
	此漢川謂襄樊上下漢水左右之地也}
皆欲徙之司馬懿曰荆楚輕脆易動|{
	易以䜴翻}
關羽新破諸為惡者藏竄觀望徙其善者既傷其意將令去者不敢復還操曰是也是後諸亡者悉還出 魏王操表孫權為票騎將軍假節領荆州牧封南昌侯|{
	南昌縣屬豫章郡票匹妙翻}
權遣校尉梁寓入貢又遣朱光等歸|{
	朱光為權所獲見上卷十九年}
上書稱臣於操稱說天命操以權書示外曰是兒欲踞吾著爐火上邪|{
	著直畧翻蓋言漢以火德王權欲使操加其上也然操必以權書示外者正欲以觀衆心耳}
侍中陳羣等皆曰漢祚已終非適今日殿下功德巍巍羣生注望|{
	注猶屬望}
故孫權在遠稱臣此天人之應異氣齊聲殿下宜正大位復何疑哉|{
	復扶又翻}
操曰若天命在吾吾為周文王矣|{
	文王三分天下有其二以服事殷}


臣光曰教化國家之急務也而俗吏慢之風俗天下之大事也而庸君忽之夫惟明智君子深識長慮然後知其為益之大而收功之遠也光武遭漢中衰羣雄糜沸奮起布衣紹恢前緒征伐四方日不暇給乃能敦尚經術賓延儒雅開廣學校|{
	校戶教翻}
修明禮樂武功既成文德亦洽繼以孝明孝章遹追先志|{
	遹述也遵也遹音聿}
臨雍拜老横經問道自公卿大夫至於郡縣之吏咸選用經明行修之人|{
	行下孟翻}
虎賁衛士皆習孝經|{
	賁音奔}
匈奴子弟亦遊太學是以教立於上俗成於下其忠厚清修之士豈惟取重於搢紳|{
	搢紳謂搢笏垂紳在朝公卿大夫也}
亦見慕於衆庶愚鄙汚穢之人豈惟不容於朝廷亦見棄于鄉里自三代既亡風化之美未有若東漢之盛者也及孝和以降貴戚擅權嬖倖用事|{
	嬖卑義翻又必計翻}
賞罰無章賄賂公行賢愚渾殽是非顛倒可謂亂矣然猶緜緜不至於亡者上則有公卿大夫袁安楊震李固杜喬陳蕃李膺之徒面引廷爭|{
	爭讀曰諍}
用公義以扶其危下則有布衣之士符融郭泰范滂許劭之流立私論以救其敗|{
	私論者謂其不得預議于朝而私立論于下以矯朝議之失也}
是以政治雖濁而風俗不衰|{
	治直吏翻}
至有觸冒斧鉞僵仆於前而忠義奮發繼起於後隨踵就戮視死如歸夫豈特數子之賢哉亦光武明章之遺化也當是之時苟有明君作而振之則漢氏之祚猶未可量也|{
	量音良}
不幸承陵夷頹敝之餘重以桓靈之昏虐保養姦囘|{
	孔安國曰囘邪也重直用翻}
過於骨肉殄滅忠良甚於寇讐積多士之憤蓄四海之怒於是何進召戎董卓乘釁袁紹之徒從而構難|{
	難乃旦翻}
遂使乘輿播越|{
	乘繩證翻}
宗廟丘墟王室蕩覆烝民塗炭大命隕絶不可復救|{
	復扶又翻}
然州郡擁兵專地者雖互相吞噬猶未嘗不以尊漢為辭以魏武之暴戾彊伉|{
	伉口浪翻}
加有大功於天下其蓄無君之心久矣乃至没身不敢廢漢而自立豈其志之不欲哉猶畏名義而自抑也由是觀之教化安可慢風俗安可忽哉

資治通鑑卷六十八














































































































































