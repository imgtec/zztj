<!DOCTYPE html PUBLIC "-//W3C//DTD XHTML 1.0 Transitional//EN" "http://www.w3.org/TR/xhtml1/DTD/xhtml1-transitional.dtd">
<html xmlns="http://www.w3.org/1999/xhtml">
<head>
<meta http-equiv="Content-Type" content="text/html; charset=utf-8" />
<meta http-equiv="X-UA-Compatible" content="IE=Edge,chrome=1">
<title>資治通鑒_78-資治通鑑卷七十七_78-資治通鑑卷七十七</title>
<meta name="Keywords" content="資治通鑒_78-資治通鑑卷七十七_78-資治通鑑卷七十七">
<meta name="Description" content="資治通鑒_78-資治通鑑卷七十七_78-資治通鑑卷七十七">
<meta http-equiv="Cache-Control" content="no-transform" />
<meta http-equiv="Cache-Control" content="no-siteapp" />
<link href="/img/style.css" rel="stylesheet" type="text/css" />
<script src="/img/m.js?2020"></script> 
</head>
<body>
 <div class="ClassNavi">
<a  href="/24shi/">二十四史</a> | <a href="/SiKuQuanShu/">四库全书</a> | <a href="http://www.guoxuedashi.com/gjtsjc/"><font  color="#FF0000">古今图书集成</font></a> | <a href="/renwu/">历史人物</a> | <a href="/ShuoWenJieZi/"><font  color="#FF0000">说文解字</a></font> | <a href="/chengyu/">成语词典</a> | <a  target="_blank"  href="http://www.guoxuedashi.com/jgwhj/"><font  color="#FF0000">甲骨文合集</font></a> | <a href="/yzjwjc/"><font  color="#FF0000">殷周金文集成</font></a> | <a href="/xiangxingzi/"><font color="#0000FF">象形字典</font></a> | <a href="/13jing/"><font  color="#FF0000">十三经索引</font></a> | <a href="/zixing/"><font  color="#FF0000">字体转换器</font></a> | <a href="/zidian/xz/"><font color="#0000FF">篆书识别</font></a> | <a href="/jinfanyi/">近义反义词</a> | <a href="/duilian/">对联大全</a> | <a href="/jiapu/"><font  color="#0000FF">家谱族谱查询</font></a> | <a href="http://www.guoxuemi.com/hafo/" target="_blank" ><font color="#FF0000">哈佛古籍</font></a> 
</div>

 <!-- 头部导航开始 -->
<div class="w1180 head clearfix">
  <div class="head_logo l"><a title="国学大师官网" href="http://www.guoxuedashi.com" target="_blank"></a></div>
  <div class="head_sr l">
  <div id="head1">
  
  <a href="http://www.guoxuedashi.com/zidian/bujian/" target="_blank" ><img src="http://www.guoxuedashi.com/img/top1.gif" width="88" height="60" border="0" title="部件查字,支持20万汉字"></a>


<a href="http://www.guoxuedashi.com/help/yingpan.php" target="_blank"><img src="http://www.guoxuedashi.com/img/top230.gif" width="600" height="62" border="0" ></a>


  </div>
  <div id="head3"><a href="javascript:" onClick="javascript:window.external.AddFavorite(window.location.href,document.title);">添加收藏</a>
  <br><a href="/help/setie.php">搜索引擎</a>
  <br><a href="/help/zanzhu.php">赞助本站</a></div>
  <div id="head2">
 <a href="http://www.guoxuemi.com/" target="_blank"><img src="http://www.guoxuedashi.com/img/guoxuemi.gif" width="95" height="62" border="0" style="margin-left:2px;" title="国学迷"></a>
  

  </div>
</div>
  <div class="clear"></div>
  <div class="head_nav">
  <p><a href="/">首页</a> | <a href="/ShuKu/">国学书库</a> | <a href="/guji/">影印古籍</a> | <a href="/shici/">诗词宝典</a> | <a   href="/SiKuQuanShu/gxjx.php">精选</a> <b>|</b> <a href="/zidian/">汉语字典</a> | <a href="/hydcd/">汉语词典</a> | <a href="http://www.guoxuedashi.com/zidian/bujian/"><font  color="#CC0066">部件查字</font></a> | <a href="http://www.sfds.cn/"><font  color="#CC0066">书法大师</font></a> | <a href="/jgwhj/">甲骨文</a> <b>|</b> <a href="/b/4/"><font  color="#CC0066">解密</font></a> | <a href="/renwu/">历史人物</a> | <a href="/diangu/">历史典故</a> | <a href="/xingshi/">姓氏</a> | <a href="/minzu/">民族</a> <b>|</b> <a href="/mz/"><font  color="#CC0066">世界名著</font></a> | <a href="/download/">软件下载</a>
</p>
<p><a href="/b/"><font  color="#CC0066">历史</font></a> | <a href="http://skqs.guoxuedashi.com/" target="_blank">四库全书</a> |  <a href="http://www.guoxuedashi.com/search/" target="_blank"><font  color="#CC0066">全文检索</font></a> | <a href="http://www.guoxuedashi.com/shumu/">古籍书目</a> | <a   href="/24shi/">正史</a> <b>|</b> <a href="/chengyu/">成语词典</a> | <a href="/kangxi/" title="康熙字典">康熙字典</a> | <a href="/ShuoWenJieZi/">说文解字</a> | <a href="/zixing/yanbian/">字形演变</a> | <a href="/yzjwjc/">金 文</a> <b>|</b>  <a href="/shijian/nian-hao/">年号</a> | <a href="/diming/">历史地名</a> | <a href="/shijian/">历史事件</a> | <a href="/guanzhi/">官职</a> | <a href="/lishi/">知识</a> <b>|</b> <a href="/zhongyi/">中医中药</a> | <a href="http://www.guoxuedashi.com/forum/">留言反馈</a>
</p>
  </div>
</div>
<!-- 头部导航END --> 
<!-- 内容区开始 --> 
<div class="w1180 clearfix">
  <div class="info l">
   
<div class="clearfix" style="background:#f5faff;">
<script src='http://www.guoxuedashi.com/img/headersou.js'></script>

</div>
  <div class="info_tree"><a href="http://www.guoxuedashi.com">首页</a> > <a href="/SiKuQuanShu/fanti/">四库全书</a>
 > <h1>资治通鉴</h1> <!--         下载:【右键另存为】即可 --></div>
  <div class="info_content zj clearfix">
  
<div class="info_txt clearfix" id="show">
<center style="font-size:24px;">78-資治通鑑卷七十七</center>
    資治通鑑卷七十七   宋 司馬光 撰<br />
<br />
  胡三省 音註<br />
<br />
  魏紀九【起柔兆困敦盡重光大荒落凡六年】<br />
<br />
  高貴鄉公下<br />
<br />
  甘露元年【是年六月改元】春正月漢姜維進位大將軍 二月丙辰帝宴羣臣於太極東堂與諸儒論夏少康漢高祖優劣以少康為優【帝謂少康生於滅亡之後降為諸侯之隸能布其德而兆其謀卒滅過戈克復禹績祀夏配天不失舊物非至德弘仁豈濟斯勛漢祖因土崩之埶杖一時之權專任智力以成功業行事動靜多違聖檢為人子則數危其親為人君則囚繫賢相為人父則不能衛子身没之後社稷幾傾若與少康易時而處未必能復大禹之績嗚呼帝固有志於少康矣然而不能殱澆豷而身死人手者不能布其德而兆其謀也予觀帝之所以論二君優劣書生之譚耳未能如石勒辭氣之雄爽也夏戶雅翻少詩沼翻】夏四月賜大將軍昭衮冕之服赤舄副焉【九錫之漸也】 丙辰帝幸太學與諸儒論書易及禮諸儒莫能及【時帝與博士淳于俊論易庾峻論書馬照論禮記考其難疑答問不過擿抉經義及王鄭之異同耳非人君之學也】帝嘗與中護軍司馬望侍中王沈散騎常侍裴秀黄門侍郎鍾會等講宴於東堂并屬文論【沈持林翻散悉亶翻騎奇寄翻屬之欲翻】特加禮異謂秀為儒林丈人沈為文籍先生帝性急請召欲速以望職在外特給追鋒車虎賁五人【望為中護軍其職在外傅子曰追鋒車施通幰遽則乘之令虎賁五人舁之也晉志曰追鋒車去小平蓋加通幰如軺車駕二馬追鋒之名取其迅速也施於戎陳之間是為傳乘賁音奔】每有集會輒犇馳而至秀濳之子也【裴潜事武帝守代郡著名】 六月丙午改元【盖以甘露降而改元也】姜維在鍾提議者多以為維力已竭未能更出安西將軍鄧艾曰洮西之敗【見上卷上年】非小失也士卒彫殘倉廪空虛百姓流離今以策言之彼有乘勝之埶我有虚弱之實一也彼上下相習五兵犀利【管子曰蚩尤受盧山之金而作五兵孔頴逹曰步卒之五兵謂弓矢一殳二矛三戈四戟五也鄭司農所謂戈矛戟酋矛夷矛車之五兵也犀堅也古以犀兕為甲故謂堅為犀】我將易兵新器仗未復二也【將易艾自謂初代王經也兵新謂遣還洮西敗卒更差軍守也將即亮翻】彼以船行吾以陸軍勞逸不同三也【言蜀船自涪戍白水可以上沮水由沮水入武都下辨自此而西北水路漸峻陿小舟猶可入也魏軍度隴而西皆陸行】狄道隴西南安祁山各當有守彼專為一我分為四四也從南安隴西因食羌穀若趣祁山【趣七喻翻下同】熟麥千頃為之外倉賊有黠計其來必矣【黠下八翻桀黠也】秋七月姜維復率衆出祁山【復扶又翻】聞鄧艾已有備乃回從董亭趣南安【水經注董亭在南安郡西南谷水歷其下東北注于渭】艾據武城山以拒之【水經注渭水過獂道南獂道南安郡治也又東逕武城縣西武城川水入焉盖以山名縣也酈道元後魏人武城縣必後魏所立而魏收地形志無之蓋廢省也】維與艾爭險不克其夜渡渭東行緣山趣上邽艾與戰於段谷【水經注上邽之南有段溪水水出西南馬門溪東北流合籍水杜佑曰秦州上邽縣有段谷水趣七喻翻】大破之以艾為鎭西將軍都督隴右諸軍事維與其鎭西大將軍胡濟期會上邽濟失期不至故敗士卒星散死者甚衆【言士卒迸散如星不能收拾成隊伍】蜀人由是怨維維上書謝求自貶黜乃以衛將軍行大將軍事 八月庚午詔司馬昭加號大都督奏事不名假黄鉞癸酉以太尉司馬孚為太傅 九月以司徒高柔為太尉 文欽說吳人以伐魏之利【說輸芮翻】孫峻使欽與驃騎將軍呂據【驃匹妙翻】及車騎將軍劉纂鎮南將軍朱異前將軍唐咨自江都入淮泗【江都縣屬廣陵郡此自邗溝入淮自淮入泗也】以圖青徐【魏青州統齊濟南樂安城陽東萊徐州統下邳彭城東海琅邪東莞東安廣陵臨淮晉志曰周禮曰正東曰青州蓋取土居少陽其色為青徐州取舒緩之義或云因徐丘以立名】峻餞之於石頭遇暴疾以後事付從父弟偏將軍綝【從才用翻綝丑林翻】丁亥峻卒吳人以綝為侍中武衛將軍都督中外諸軍事召呂據等還【還從宣翻又如字】 己丑吳大司馬呂岱卒年九十六始岱親近吳郡徐原慷慨有才志岱知其可成賜巾褠【釋名巾謹也二十成人士冠庶人巾言當自謹修於四教褠單衣漢魏以來士庶以為禮服褠古侯翻】與共言論後遂薦拔官至侍御史原性忠壮好直言【好呼到翻】岱時有得失原輒諫爭【爭讀曰諍】又公論之【公然於衆中論其得失】人或以告岱岱歎曰是我所以貴德淵者也【徐原字德淵】及原死岱哭之甚哀曰徐德淵呂岱之益友【論語孔子曰益者三友友直友諒友多聞】今不幸【論語曰不幸短命死矣】岱復於何聞過【復扶又翻】談者美之 呂據聞孫綝代孫峻輔政大怒與諸督將連名共表薦滕胤為丞相【將即亮翻】綝更以胤為大司馬代呂岱駐武昌據引兵還使人報胤欲共廢綝冬十月綝遣從兄憲將兵逆據於江都使中使敕文欽劉纂唐咨等共擊取據又遣侍中左將軍華融中書丞丁晏【魏晉之制中書無丞此吳所置華戶化翻】告喻胤宜速去意【言宜速往武昌否則且有誅罰】胤自以禍及因留融晏勒兵自衛召典軍楊崇將軍孫咨【楊崇蓋胤帳下典軍】告以綝為亂迫融等使有書難綝【有者對無之稱於此則文義不為通通鑑既因三國志舊文今亦不欲輕改難乃旦翻】綝不聼表言胤反許將軍劉丞以封爵使率兵騎攻圍胤胤又刼融等使詐為詔發兵融等不從皆殺之或勸胤引兵至蒼龍門【蒼龍門吳建業宫之東門也】將士見公出必委綝就公【委弃也】時夜已半胤恃與據期又難舉兵向宫乃約令部曲【約勒而號令之】說呂侯兵已在近道故皆為胤盡死無離散者【為于偽翻】胤顔色不變談笑如常時大風比曉據不至【比必寐翻】綝兵大會遂殺胤及將士數十人夷胤三族己酉大赦改元太平或勸呂據犇魏者據曰吾耻為叛臣遂自殺【據父範佐孫策以造吳故耻為叛臣自殺以明節】 以司空鄭冲為司徒左僕射盧毓為司空【晉志曰尚書僕射漢本置一人至漢獻帝建安四年以執金吾榮郃為尚書左僕射僕射分置左右蓋自此始經魏至晉迄于江左省置無常置二則為左右僕射或不兩置但曰尚書僕射今闕則左為省主若左右並闕則置尚書僕射以主省事毓余六翻】毓固讓驃騎將軍王昶光禄大夫王觀司隸校尉琅邪王祥詔不許祥性至孝繼母朱氏遇之無道祥愈恭謹朱氏子覽年數歲每見祥被楚撻【楚荆也撻擊也被皮義翻】輒涕泣抱持母母以非理使祥覽輒與祥俱往及長娶妻【長知兩翻】母虐使祥妻覽妻亦趨而共之母患之為之少止【為于偽翻】祥漸有時譽母深疾之密使酖祥覽知之逕起取酒祥爭而不與母遽奪反之【漢書齊悼惠王傳奪反孝惠巵師古曰反音幡】自後母賜祥饌【饌雛戀翻又雛睆翻】覽輒先嘗母懼覽致斃遂止漢末遭亂祥隱居三十餘年不應州郡之命母終毁瘁【瘁秦醉翻病勞也】杖而後起徐州刺史呂䖍檄為别駕委以州事州界清靜政化大行時人歌之曰海沂之康實賴王祥【徐州之地東際海西北距泗沂故曰海沂】邦國不空别駕之功 十一月吳孫綝遷大將軍綝負貴倨傲多行無禮峻從弟憲嘗與誅諸葛恪【與讀曰預】峻厚遇之官至右將軍無難督平九官事【九官即九卿也魏明帝太和二年吳主還建業留尚書九官于武昌】綝遇憲薄於峻時憲怒與將軍王惇謀殺綝事泄綝殺惇憲服藥死<br />
<br />
  二年春三月大梁成侯盧毓卒 夏四月吳主臨正殿大赦始親政事孫綝表奏多見難問【難乃旦翻】又科兵子弟十八已下十五以上三千餘人【科程也程其長短小大也或曰科當作料音聊量度也】選大將子弟年少有勇力者使將之【少詩照翻將即亮翻】日於苑中教習曰吾立此軍欲與之俱長【長丁丈翻今知兩翻】又數出中書視大帝時舊事問左右侍臣曰先帝數有特制【特制謂特出上意以手詔宣行也數所角翻】今大將軍問事【問事猶言奏事不言奏者自卑挹之意】但令我書可邪【書可畫可也】嘗食生梅使黄門至中藏取蜜【中藏中藏府也掌幣帛金銀諸貨物蜜蜂糖也藏徂浪翻下同】蜜中有鼠矢召問藏吏藏吏叩頭吳主曰黄門從爾求蜜邪吏曰向求【謂向者嘗求蜜也】實不敢與黄門不服吳主令破鼠矢矢中燥因大笑謂左右曰若矢先在蜜中中外當俱濕今外濕裏燥此必黄門所為也詰之果服【詰去吉翻】左右莫不驚悚征東大將軍諸葛誕素與夏侯玄鄧颺等友善玄等死【玄死見上卷正元元年颺死見七十五卷邵陵厲公嘉平元年颺余章翻又余亮翻】王凌毋丘儉相繼誅滅【王凌死見七十五卷嘉平三年毋丘儉死見上卷正元二年】誕内不自安乃傾帑藏振施【帑它郎翻施式智翻】曲赦有罪以收衆心畜養揚州輕俠數千人以為死士【畜許六翻】因吳人欲向徐堨【徐堨即徐塘在東關之東堨烏葛翻】請十萬衆以守壽春又求臨淮築城以備吳寇司馬昭初秉政長史賈充請遣參佐慰勞四征【魏置征東將軍屯淮南征南將軍屯襄沔以備吳征西將軍屯關隴以備蜀征北將軍屯幽并以備鮮卑皆授以重兵司馬昭初當國故充請慰勞以觀其志趣勞力到翻】且觀其志昭遣充至淮南充見誕論說時事因曰洛中諸賢皆願禪代君以為如何誕厲聲曰卿非賈豫州子乎【充父逵先為豫州而卒故稱之】世受魏恩豈可欲以社稷輸人乎若洛中有難【難乃旦翻】吾當死之充默然還言於昭曰諸葛誕再在揚州【誕先督揚州東關之敗改督豫州毋丘儉既死復督揚州】得士衆心今召之必不來然反疾而禍小不召則反遲而禍大不如召之昭從之甲子詔以誕為司空召赴京師誕得詔書愈恐疑揚州刺史樂綝間已遂殺綝【征東將軍與揚州刺史同治壽春魏四征之任率以其州刺史為儲帥故誕疑綝間已間古莧翻】斂淮南及淮北郡縣屯田口十餘萬官兵【魏郡縣皆置屯田凡屯田口悉官兵也】揚州新附勝兵者四五萬人【勝音升】聚穀足一年食為閉門自守之計遣長史吳綱將少子靚至吳【將如字少詩沼翻靚疾郢翻又疾正翻】稱臣請救并請以牙門子弟為質【牙門諸將之子弟也質音致】 吳滕胤呂據之妻皆夏口督孫壹之妹也【壹孫奐庶子也夏戶雅翻】六月孫綝使鎭南將軍朱異自虎林將兵襲壹異至武昌壹將部曲來犇乙已詔拜壹車騎將軍交州牧封吳侯開府辟召儀同三司衮冕赤舄事從豐厚【崇異孫壹者以招攜貳也】 司馬昭奉帝及太后討諸葛誕【昭若自行恐後有挾兩宫為變者故奉之以討誕】吳綱至吳吳人大喜使將軍全懌全端唐咨王祚將三萬衆與文欽同救誕以誕為左都護假節大司徒驃騎將軍青州牧封壽春侯懌琮之子端其從子也六月甲子車駕次項司馬昭督諸軍二十六萬進屯丘頭【是役也司馬昭改丘頭曰武丘以旌武功武丘唐為沈丘縣】以鎭南將軍王基行鎭東將軍都督揚豫諸軍事與安東將軍陳騫等圍壽春基始至圍城未合文欽全懌等從城東北因山乘險得將其衆突入城【壽春城外他無山唯城北有八公山耳】昭敕基歛軍堅壁基累求進討會吳朱異率三萬人進屯安豐為文欽外埶【安豐縣漢屬廬江郡魏分屬安豐郡今安豐縣在壽春南八十里】詔基引諸軍轉據北山基謂諸將曰今圍壘轉固兵馬向集但當精修守備以待越逸而更移兵守險使得放縱雖有智者不能善其後矣遂守便宜上疏曰今與賊家對敵當不動如山若遷移依險人心揺蕩於埶大損諸軍並據深溝高壘衆心皆定不可傾動此御兵之要也書奏報聼【報基聼行其策時帝在軍故諸軍節度皆禀詔指而裁其可否者實司馬昭也】於是基等四面合圍表裏再重【重直龍翻】塹壘甚峻文欽等數出犯圍【數所角翻】逆擊走之司馬昭又使奮武將軍監青州諸軍事石苞【監古銜翻】督兖州刺史州泰徐州刺史胡質簡鋭卒為游軍以備外宼泰擊破朱異於陽淵【水經注決水出廬江雩婁縣北過安豐縣東又北右會陽泉水水西有陽泉縣故城故陽泉鄉也漢靈帝封黄琬為侯國决水又北入于淮】異走泰追之殺傷二千人秋七月吳大將軍綝大發兵出屯鑊里【後吳主責孫綝以留湖中不上岸一步則鑊里當在巢縣界】復遣朱異帥將軍丁奉黎斐等五人前解壽春之圍【復扶又翻帥讀曰率】異留輜重於都陸【水經注博鄉縣王莽改曰楊陸泄水出焉北過芍陂又西北入于淮意者都陸即楊陸歟又據晉紀都陸在黎漿南重直用翻】進屯黎漿【水經注芍陂瀆水東注黎漿水水東逕黎漿亭南又東注肥水謂之黎漿水口】石苞州泰又擊破之太山太守胡烈以奇兵五千襲都陸盡焚異資粮異將餘兵食葛葉走歸孫綝綝使異更死戰異以士卒乏食不從綝命綝怒九月己巳綝斬異於鑊里辛未引兵還建業【壽春之圍已固雖使周瑜呂蒙陸遜復生不能解也若綝能舉荆揚之衆出襄陽以向宛洛壽春城下之兵必分歸以自救諸葛誕文欽等於此時决圍力戰猶庶幾焉】綝既不能拔出諸葛誕而喪敗士衆【喪息浪翻敗補邁翻】自戮名將由是吳人莫不怨之【為後吳誅孫綝張本】司馬昭曰異不得至壽春而吳人殺之非其罪也欲以謝壽春而堅誕意使其猶望救耳今當堅圍備其越逸而多方以誤之乃縱反閒【閒古莧翻】揚言吳救方至大軍乏食分遣羸就穀淮北埶不能久誕等益寛恣食俄而城中乏糧外救不至將軍蒋班焦彞皆誕腹心謀主也言於誕曰朱異等以大衆來而不能進孫綝殺異而歸江東外以發兵為名内實坐須成敗【須待也】今宜及衆心尚固士卒思用并力决死攻其一面雖不能盡克猶有可全者空坐守死無為也【言不若决死而求生無為坐守而待斃】文欽曰公今舉十餘萬之衆歸命於吳而欽與全端等皆同居死地父兄子弟盡在江表就孫綝不欲來主上及其親戚豈肯聼乎且中國無歲無事軍民並疲今守我一年内變將起奈何舍此【舍讀曰捨】欲乘危徼倖乎【徼堅堯翻】班彝固勸之欽怒誕欲殺班彞二人懼十一月弃誕踰城來降全懌兄子輝儀在建業【輝儀懌兄全緒之二子輝一作禕】與其家内爭訟攜其母將部曲數十家來奔于是懌與兄子靖及全端弟翩緝皆將兵在壽春城中司馬昭用黄門侍郎鍾會策密為輝儀作書【為于偽翻】使輝儀所親信齎入城告懌等說吳中怒懌等不能拔壽春【言不能拔壽春之衆於重圍也】欲盡誅諸將家故逃來歸命十二月懌等帥其衆數千人開門出降【帥讀曰率降戶江翻】城中震懼不知所為詔拜懌平東將軍封臨湘侯端等封拜各有差 漢姜維聞魏分關中兵以赴淮南欲乘虚向秦川【秦地四塞以為固渭水貫其中渭川左右沃壤千里世謂之秦川】率數萬人出駱谷至沈嶺時長城積穀甚多而守兵少征西將軍都督雍凉諸軍事司馬望【雍於用翻】及安西將軍鄧艾進兵據之以拒維維壁於芒水【水經注駱谷水出郿塢東南山駱谷北流逕長城西又北流注于渭渭水又東芒水從南來注之水出南山芒谷北逕盩厔縣竹圃中又北流注于渭予按駱谷在今洋州眞符縣屈回八十里凡八十四盤】數挑戰【數所角翻挑徒了翻】望艾不應是時維數出兵蜀人愁苦中散大夫譙周作仇國論以諷之【續漢志曰中散大夫秩六百石漢官曰秩比二千石胡廣曰光禄大夫本為中大夫武帝元狩五年置諫大夫為光禄大夫世祖中興以為諫議大夫又有太中中散大夫此四等於古者為天子之下大夫視列國之上卿】曰或問往古能以弱勝彊者其術如何曰吾聞之處大無患者常多慢處小有憂者常思善【處昌呂翻】多慢則生亂思善則生治理之常也故周文養民以少取多句踐卹衆以弱斃彊此其術也【文王治岐由方百里起三分天下冇其二所謂以少取多也句踐歸越弔死問十年生聚十年教訓以弱越斃強吳】或曰曩者項彊漢弱相與戰爭項羽與漢約分鴻溝各歸息民張良以為民志已定則難動也率兵追羽終斃項氏【事見十卷漢高帝四年】豈必由文王之事乎曰當商周之際王侯世尊【言世世居尊位】君臣久固民習所專【民習見君臣之分明故專于戴上】深根者難拔據固者難遷當此之時雖漢祖安能杖劔鞭馬而取天下乎及秦罷侯置守之後【謂罷列國諸侯分置三十六郡郡置守也】民疲秦役天下土崩或歲易主或月易公鳥驚獸駭莫知所從於是豪彊並爭虎裂狼分疾者獲多遲後者見吞今我與彼皆傳國易世矣既非秦末鼎沸之時實有六國並據之埶故可為文王難為漢祖夫民之疲勞則騷擾之兆生上慢下暴則瓦解之形起諺曰射幸數跌不如審發【跌差也射數差而不中不如審而後發也書曰若虞機張往省括于度則釋】是故智者不為小利移目不為意似改步【孔頴逹曰舉足謂之步為于偽翻】時可而後動數合而後舉故湯武之師不再戰而克【湯伐桀鳴條一戰而革夏命武王伐紂一戎衣而天下大定】誠重民勞而度時審也【度徒洛翻】如遂極武黷征【征伐不欲數數則黷】土崩埶生不幸遇難【難乃旦翻】雖有智者將不能謀之矣【姜維以數戰亡蜀卒如譙周之言】<br />
<br />
  三年春正月文欽謂諸葛誕曰蒋班焦彛謂我不能出而走全端全懌又率衆逆降【逆迎也降戶江翻】此敵無備之時也可以戰矣誕及唐咨等皆以為然遂大為攻具晝夜五六日攻南圍欲决圍而出圍上諸軍臨高發石車火箭【石車即砲車也車昌遮翻】逆燒破其攻具矢石雨下死傷蔽地血流盈壍【壍七豔翻】復還城城内食轉竭出降者數萬口欽欲盡出北方人省食與吳人堅守誕不聼由是爭恨欽素與誕有隙徒以計合事急愈相疑【言誕欽初以詭計苟合事急愈相猜疑】欽見誕計事誕遂殺欽欽子鴦虎將兵在小城中【鴦虎欽二子也時壽春蓋别有小城】聞欽死勒兵赴之衆不為用遂單走踰城出自歸於司馬昭軍吏請誅之昭曰欽之罪不容誅其子固應就戮然鴦虎以窮歸命且城未拔殺之是堅其心也乃赦鴦虎使將數百騎廵城呼曰【呼火故翻】文欽之子猶不見殺其餘何懼又表鴦虎皆為將軍賜爵關内侯城内皆喜且日益饑困司馬昭身自臨圍見城上持弓者不發曰可攻矣【知其衆無拒守之心也】乃四面進軍同時鼓譟登城二月乙酉克之誕窘急單馬將其麾下突小城欲出司馬胡奮部兵擊斬之夷其三族誕麾下數百人皆拱手為列不降每斬一人輒降之【降戶江翻下同】卒不變以至於盡【史言諸葛誕得人心人蒙其恩而為之死卒子恤翻】吳將于詮曰【詮且緣翻】大丈夫受命其主以兵救人既不能克又束手於敵吾弗取也乃免胄冒陳而死【陳讀曰陣】唐咨王祚等皆降【唐咨本魏人降吳見七十卷文帝黄初六年】吳兵萬衆器仗山積司馬昭初圍壽春王基石苞等皆欲急攻之昭以為壽春城固而衆多攻之必力屈若有外寇表裏受敵此危道也今三叛相聚於孤城之中【三叛謂諸葛誕文欽唐咨也】天其或者使同就戮吾當以全策縻之但堅守三面若吳賊陸道而來軍糧必少吾以游兵輕騎絕其轉輸可不戰而破也吳賊破欽等必成禽矣乃命諸軍案甲而守之卒不煩攻而破【卒子恤翻】議者又以為淮南仍為叛逆【仍相因也】吳兵室家在江南不可縱宜悉坑之昭曰古之用兵全國為上戮其元惡而已【言全其國之人民止戮其君所謂誅其君而弔其民也】吳兵就得亡還適可以示中國之大度耳一無所殺分布三河近郡以安處之【河南都也河東河内皆近京師處昌呂翻】拜唐咨安遠將軍其餘禆將咸假位號衆皆悦服其淮南將士吏民為誕所脅略者皆赦之聼文鴦兄弟收歛父喪給其車牛致葬舊墓【文欽譙人也舊墓在焉歛力贍翻】昭遺王基書曰【遺于季翻】初議者云云求移者甚衆【謂前詔諸軍轉據北山】時未臨履亦謂宜然【臨履謂親臨其地而履行營壘處所也】將軍深筭利害獨秉固志上違詔命下拒衆議終至制敵禽賊雖古人所述不是過也昭欲遣諸軍輕兵深入招迎唐咨等子弟因釁有滅吳之埶王基諫曰昔諸葛恪乘東關之勝竭江表之兵以圍新城城既不拔而衆死者大半【事見上卷邵陵厲公嘉平五年】姜維因洮西之利輕兵深入糧餉不繼軍復上邽【謂段谷之敗也】夫大捷之後上下輕敵輕敵則慮難不深【難乃旦翻】今賊新敗於外又内患未弭【謂孫綝君臣相猜】是其修備設慮之時也且兵出踰年人有歸志今俘馘十萬罪人斯得【謂禽諸葛誕也書曰周公居東二年則罪人斯得】自歷代征伐未有全兵獨克如今之盛者也武皇帝克袁紹於官渡自以所獲已多不復追犇【復扶又翻】懼挫威也【事見六十三卷漢獻帝建安五年】昭乃止以基為征東將軍都督揚州諸軍事進封東武侯<br />
<br />
  習鑿齒曰君子謂司馬大將軍於是役也可謂能以德攻矣【左傳晉文公城濮之勝君子謂晉於是役也能以德攻】夫建業者異道各有所尚而不能兼并也故窮武之雄斃于不仁【如夫差智伯也】存義之國喪於懦退【如宋襄公是也喪息浪翻】今一征而禽三叛大虜吳衆席卷淮浦俘馘十萬【生虜為俘截耳為馘古者戰勝馘所格之左耳而獻之】可謂壮矣而未及安坐賞王基之功種惠吳人結異類之情【書曰臯陶邁種德孔安國注曰種布也夫種則有穫種惠於吳人使歸心中國以成他日混一之功如種藝之有秋也】寵鴦葬欽忘疇㫺之隙不咎誕衆使揚土懷愧功高而人樂其成業廣而敵懷其德【樂音洛】武昭既敷文算又洽推是道也天下其孰能當之哉【鑿齒晉人其辭盖有溢美者】<br />
<br />
  司馬昭之克壽春鍾會謀畫居多昭親待日隆委以腹心之任時人比之子房【比之張良也】 漢姜維聞諸葛誕死復還成都復拜大將軍【維以段谷之敗貶行大將軍事】 夏五月詔以司馬昭為相國【漢書百官表曰相國丞相皆秦官又按蕭何傳何自丞相拜相國則相國尊於丞相】封晉公食邑八郡【晉書帝記曰以并州之太原上黨西河樂平新興雁門司州之河東平陽凡八郡封為晉公】加九錫昭前後九讓乃止 秋七月吳主封故齊王奮為章安侯【奮徙章安見上卷邵陵厲公嘉平五年】 八月以驃騎將軍王昶為司空【昶音丑兩翻】 詔以關内侯王祥為三老鄭小同為五更帝率羣臣詣太學行養老乞言之禮【記曰凡養老五帝憲三王又乞言五帝憲養氣體而不乞言有善則記之以為惇史三王亦憲既養老而後乞言亦微其禮皆有惇史鄭玄注曰憲法也養之為法其德行三王又從之求善言可施行也惇史惇厚者也微其禮者依違言之更音工衡翻】小同玄之孫也【鄭玄别傳曰玄有子為孔融吏舉孝廉融之被圍往赴為賊所害有遺腹子以丁卯日生而玄以丁卯歲生故名曰小同】 吳孫綝以吳主親覽政事多所難問【難音乃旦翻】甚懼返自鑊里遂稱疾不朝【朝直遥翻】使弟威遠將軍據入倉龍門宿衛【古倉蒼字通用】武衛將軍恩偏將軍幹長水校尉闓【闓音開又苦亥翻】分屯諸營欲以自固吳主惡之【惡音烏路翻】乃推朱公主死意【朱公主死見上卷正元二年推尋也尋問公主所以見殺之意】全公主懼曰我實不知皆朱據二子熊損所白是時熊為虎林督損為外部督【吳外部督建業外營兵】吳主皆殺之損妻即孫峻妹也綝諫不從由是益懼吳主隂與全公主及將軍劉丞謀誅綝全后父尚為太常衛將軍吳主謂尚子黄門侍郎紀曰孫綝專埶輕小於孤【謂輕視之以為幼小也】孤前勅之使速上岸為唐咨等作援而留湖中不上岸一步【上時掌翻】又委罪於朱異擅殺功臣不先表聞築第橋南【綝蓋築第於朱雀橋南】不復朝見此為自在無所復畏【自在謂居處自如不復知有君上復扶又翻見賢遍翻】不可久忍今規取之【規圖也】卿父作中軍都督【衛將軍督中軍】使密嚴整士馬孤當自出臨橋率宿衛虎騎左右無難一時圍之【吳有左右無難督督無難營兵】作版詔勅綝所領皆解散不得舉手正爾自當得之【正爾猶言正如此也】卿去但當使密耳卿宣詔卿父勿令卿母知之女人既不曉大事且綝同堂姊邂逅漏洩誤孤非小也【邂戶廨翻逅戶茂翻】紀承詔以告尚尚無遠慮以語紀母母使人密語綝【語牛倨翻】九月戊午綝夜以兵襲尚執之遣弟恩殺劉承於蒼龍門外【劉承即劉丞】比明遂圍宫【比必寐翻】吳主大怒上馬帶鞬執弓欲出【鞬居言翻戢弓矢器】曰孤大皇帝適子【適讀曰嫡】在位已五年誰敢不從者侍中近臣及乳母共牽攀止之不得出歎咤不食【咤陟駕翻】罵全后曰爾父憒憒【憒烏外翻類篇曰悶也】敗我大事【敗補邁翻】又遣呼紀紀曰臣父奉詔不謹負上無面目復見【復扶又翻下同】因自殺綝使光禄勲孟宗告太廟廢吳主為會稽王【吳主亮時年十六會工外翻】召羣臣議曰少帝荒病昏亂不可以處大位承宗廟【少詩沼翻處昌呂翻下同】已告先帝廢之諸君若有不同者下異議皆震怖【怖普布翻】曰唯將軍令綝遣中書郎李崇奪吳主璽綬【璽斯氏翻綬音受】以吳主罪班告遠近尚書桓彛不肯署名綝怒殺之典軍施正勸綝迎立琅邪王休綝從之【吳制中營置左右典軍】己未綝使宗正楷與中書郎董朝【楷以吳同姓為宗正中書郎即晉中書侍郎之職】迎琅邪王於會稽【吳建興元年休徙丹陽既又徙會稽會工外翻】遣將軍孫耽送會稽王亮之國亮時年十六徙全尚於零陵尋追殺之遷全公主於豫章冬十月戊午琅邪王行至曲阿【杜祐曰曲阿今丹陽郡丹陽縣界】有老公遮王叩頭曰事久變生天下喁喁【喁魚容翻師古曰喁喁衆口向上也又相應和聲】是日進及布塞亭孫綝以琅邪王未至欲入居宫中召百官會議皆惶怖失色徒唯唯而已【唯以水翻諾也】選曹郎虞汜曰明公為國伊周處將相之任【汜音祀處昌呂翻】擅廢立之威將上安宗廟下惠百姓大小踴躍自以伊霍復見今迎王未至而欲入宮如是羣下揺蕩衆聽疑惑非所以永終忠孝揚名後世也綝不懌而止汜翻之子也綝命弟恩行丞相事率百僚以乘輿法駕迎琅邪王於永昌亭孫恩奉上璽符【乘繩證翻上時掌翻】王三讓乃受羣臣以次奉引【引讀曰靷】王就乘輿百官陪位綝以兵千人迎於半野拜於道側王下車荅拜即日御正殿大赦改元永安【吳主休字子烈吳主權第六子】孫綝稱艸莽臣詣闕上書上印綬節钺求避賢路【謂他有賢者進用恐妨其路求引身避之】吳主引見慰諭【見賢遍翻】下詔以綝為丞相荆州牧增邑五縣【綝遷大將軍封永寧侯今休以援立之功增其封邑】以恩為御史大夫衛將軍中軍督封縣侯孫據幹闓皆拜將軍封侯又以長水校尉張布為輔義將軍封永康侯【初休為王時布為左右督素見信愛及即位遂寵任之為布擅吳立孫皓以亡國喪身張本宋白曰吳赤烏八年分烏傷之上浦立永康縣屬東陽郡】先是丹陽太守李衡數以事侵琅邪王【休徙丹陽見七十五卷邵陵厲公嘉平四年先悉薦翻數所角翻下同】其妻習氏諫之【習姓按風俗通漢有外黄令習一】衡不聽琅邪王上書乞徙它郡詔徙會稽及琅邪王即位李衡憂懼謂妻曰不用卿言以至於此吾欲奔魏何如妻曰不可君本庶民耳先帝相拔過重既數作無禮而復逆自猜嫌【復扶又翻】逃叛求活以此北歸何面目見中國人乎琅邪王素好善慕名【好呼到翻】方欲自顯於天下終不以私嫌殺君明矣可自囚詣獄表列前失顯求受罪如此乃當逆見優饒【逆迎也言將優加其官以饒益之】非但直活而已衡從之吳主詔曰丹陽太守李衡以往事之嫌自拘司敗【左傳楚箴尹克黄自拘於司敗司敗即司寇也】夫射鉤斬祛在君為君【齊桓公與公子糾爭國管仲射桓公中帶鉤子糾死桓公以管仲為相遂覇諸侯晉獻公使寺人披伐蒲公子重耳踰垣而走披斬其袪及重耳反國與披謀國事發呂郤之謀薦趙衰守原為于偽翻】其遣衡還郡勿令自疑又加威遠將軍授以棨戟【果如習氏所料】己丑吳主封故南陽王和子皓為烏程侯【和死皓全見上卷邵陵厲公嘉平五年】 羣臣奏立皇后太子吳主曰朕以寡德奉承洪業涖事日淺恩澤未敷后妃之號嗣子之位非所急也有司固請吳主不許孫綝奉牛酒詣吳主吳主不受齎詣左將軍張布酒酣出怨言曰初廢少主時多勸吾自為之者吾以陛下賢明故迎之帝非我不立今上禮見拒是與凡臣無異當復改圖耳【上時掌翻復扶又翻】布以告吳主【綝以布為吳主所信倚故詣之酒酣失言遂以賈禍綝之凶愚其赤族宜矣】吳主銜之恐其有變數加賞賜【數所角翻】戊戍吳主詔曰大將軍掌中外諸軍事事統煩多其加衛將軍御史大夫恩侍中與大將軍分省諸事【分綝之權也】或有告綝懷怨侮上欲圖反者吳主執以付綝綝殺之由是益懼因孟宗求出屯武昌吳主許之綝盡敕所督中營精兵萬餘人皆令装載【中營兵即中軍也吳人謂装船為装載綝欲以此兵自隨上武昌載才再翻車船装物皆曰載詩云載輸爾載】又取武庫兵器吳主咸令給與綝求中書兩郎典知荆州諸軍事主者奏中書不應外出吳主特聽之其所請求一無違者將軍魏邈說吳主曰綝居外必有變【說輸芮翻】武衛士施朔又告綝謀反【武衛士武衛之士也】吳主將討綝密問輔義將軍張布布曰左將軍丁奉雖不能吏書而計畧過人能斷大事【斷丁亂翻】吳主召奉告之且問以計畫奉曰丞相兄弟支黨甚盛恐人心不同不可卒制【卒讀曰猝】可因臘會有陛兵以誅之【陛兵宿衛之兵夾殿陛者所謂陛戟之士】吳主從之十二月丁卯建業中謡言明會有變【明會明日臘會也吳以土德王用辰臘】綝聞之不悦夜大風發屋揚沙綝益懼戊辰臘會稱疾不至吳主彊起之【彊其兩翻】使者十餘輩綝不得已將入衆止焉綝曰國家屢有命不可辭可豫整兵令府内起火因是可得速還遂入尋而火起【尋繼時也】綝求出吳主曰外兵自多不足煩丞相也綝起離席【離力智翻】奉布目左右縛之綝叩頭曰願徙交州吳主曰卿何不徙滕胤呂據於交州乎綝復曰【復扶又翻】願没為官奴吳主曰卿何不以胤據為奴乎【胤據死見上甘露元年】遂斬之以綝首令其衆曰諸與綝同謀者皆赦之放仗者五千人孫闓乘船欲降此追殺之【闓音開又可亥翻綝之諸弟據恩幹蓋已就誅獨闓走欲投北降戶江翻】夷綝三族發孫峻棺取其印綬斵其木而埋之【古者棺椁厚薄皆有度斵而薄之以示貶】己巳吳主以張布為中軍督改葬諸葛恪滕胤呂據等其罹恪等事遠徙者一切召還朝臣有乞為諸葛恪立碑者【為于偽翻】吳主詔曰盛夏出軍士卒傷損無尺寸之功不可謂能受託孤之任死於豎子之手不可謂智遂寑【恪死見上卷嘉平五年】初漢昭烈留魏延鎮漢中【事見六十八卷漢獻帝建安二十四年】皆實兵諸圍以禦外敵敵若來攻使不得入及興埶之役王平捍拒曹爽【事見七十四卷邵陵厲公正始五年】皆承此制及姜維用事建議以為錯守諸圍【錯倉故翻】適可禦敵不獲大利不若使敵至諸圍皆歛兵聚穀退就漢樂二城【諸葛亮築漢樂二城見七十一卷明帝太和三年】聽敵入平【謂縱敵使入平地也】重關頭鎮守以捍之令游軍旁出以伺其虚敵攻關不克野無散穀千里運糧自然疲乏引退之日然後諸城並出與游軍并力傳之此殄敵之術也於是漢主令督漢中胡濟却住漢夀監軍王含守樂城【樂城在沔陽東山上周三十里甚險固諸葛亮所築沔水逕其北又北逕西樂城東而北流注于漢】護軍蒋斌守漢城【姜維自弃險要以開狡焉啓疆之心書此為亡蜀張本斌音彬】<br />
<br />
  四年春正月黄龍二見寧陵井中【見賢遍翻下同】先是頓丘冠軍陽夏井中屢有龍見【陳夀志曰去年青龍仍見頓丘冠軍陽夏縣界井中寧陵縣前漢屬陳留郡後漢魏屬梁國頓丘縣漢屬東郡魏屬魏郡冠軍縣屬南陽郡陽夏縣漢屬陳國魏屬梁國先悉薦翻夏音賈】羣臣以為吉祥帝曰龍者君德也上不在天下不在田而數屈於井【數所角翻】非嘉兆也作潛龍詩以自諷司馬昭見而惡之【帝有誅昭之志不務善晦而憤鬰之氣見於辭而不能自揜盖亦淺矣此其所以死於權臣之手乎惡烏路翻】 夏六月京陵穆侯王昶卒 漢主封其子諶為北地王【諶時壬翻】詢為新興王䖍為上黨王尚書令陳祗以巧佞有寵於漢主姜維雖位在祗上而多率衆在外希親朝政權任不及祗秋八月丙子祗卒漢主以僕射義陽董厥為尚書令尚書諸葛瞻為僕射冬十一月車騎將軍孫壹為婢所殺【二年孫壹來降】 是歲以王基為征南將軍都督荆州諸軍事【據晉書文帝紀時分荆州為二都督基鎮新野州泰鎮襄陽】<br />
<br />
  元皇帝上【諱奐字景明武帝之孫燕王宇之子也甘露二年封安次縣常道鄉公諡法行義說民曰元帝本名璜即位改名奐】<br />
<br />
  景元元年【是年六月方改元】春正月朔日有食之 夏四月詔有司率遵前命復進大將軍昭位相國封晉公加九錫【遵前年之命也復扶又翻】 帝見威權日去不勝其忿【勝音升】五月己丑召侍中王沈【沈持林翻下同】尚書王經散騎常侍王業謂曰司馬昭之心路人所知也【言路人亦知其將簒】吾不能坐受廢辱今日當與卿自出討之【卿下當有等字】王經曰昔魯昭公不忍季氏敗走失國為天下笑【魯季氏世執魯國之政至昭公時伐之不勝公孫于齊次于陽州死于乾侯事見左傳】今權在其門為日久矣朝廷四方皆為之致死【為于偽翻】不顧逆順之理非一日也且宿衛空闕兵甲寡弱陛下何所資用而一旦如此毋乃欲除疾而更深之邪禍殆不測宜見重詳【重直用翻重再也詳審也】帝乃出懷中黄素詔投地【說文曰素白緻繒也此黄素詔者盖以白緻繒染為黄色以書詔】曰行之决矣正使死何懼况不必死邪於是入白太后沈業犇走告昭呼經欲與俱經不從【帝禮遇王沈呼為文籍先生而臨變乃爾吁 考異曰世語曰經因沈業申意今從晉諸公贊】帝遂拔劔升輦率殿中宿衛蒼頭官僮鼓譟而出昭弟屯騎校尉伷遇帝於東止車門左右呵之伷衆犇走【伷讀曰胄】中護軍賈充自外入逆與帝戰於南闕下帝自用劔衆欲退騎督成倅弟太子舍人濟問充曰事急矣當云何充曰司馬公畜養汝等【畜許六翻騎督督騎兵晉志太子舍人職比散騎中書等侍郎時未立太子不應置東宫官屬濟本昭之私人授以是官耳騎奇寄翻倅七内翻】正為今日【為于偽翻】今日之事無所問也濟即抽戈前刺帝殞于車下【時年二十刺七亦翻】昭聞之大驚自投於地太傅孚犇往枕帝股而哭甚哀【枕帝於股也左傳齊崔杼弑其君光晏子枕尸股而哭之三踊而出枕職任翻】曰殺陛下者臣之罪也昭入殿中召羣臣會議尚書左僕射陳泰不至昭使其舅尚書荀顗召之泰曰世之論者以泰方於舅【方比也】今舅不如泰也【言顗阿附司馬氏而已忠於魏室】子弟内外咸共逼之乃入見昭悲慟昭亦對之泣曰玄伯【陳泰字玄伯】卿何以處我【處昌呂翻】泰曰獨有斬賈充少可以謝天下耳【少詩沼翻 考異曰魏氏春秋曰帝之崩也太傅司馬孚右僕射陳泰枕帝尸於股號哭盡哀大將軍入禁中泰見之悲慟大將軍亦對之泣謂曰玄伯其如我何泰曰獨有斬賈充少可以謝天下耳大將軍久之曰卿更思其他泰曰豈可使泰復發後言遂嘔血薨裴松之以為違實今從干寶晉紀】昭久之曰卿更思其次泰曰泰言惟有進於此【言當以弑君之罪罪昭】不知其次昭乃不復更言【復扶又翻】顗彧之子也【彧於六翻】太后下令罪狀高貴鄉公廢為庶人葬以民禮收王經及其家屬付廷尉經謝其母母顔色不變笑而應曰人誰不死正恐不得其所以此并命何恨之有【非此母不生此子】及就誅故吏向雄哭之哀動一市【向姓也音式亮翻】王沈以功封安平侯庚寅太傅孚等上言請以王禮葬高貴鄉公太后許之使中護軍司馬炎迎燕王宇之子常道鄉公璜於鄴【水經注曰白祀溝水出廣陽縣之婁城東東南逕常道城西故鄉亭也西去良鄉城四十里魏少帝璜所封也廣陽故燕國】以為明帝嗣炎昭之子也辛卯羣公奏太后自今令書皆稱詔制【羣公自上公三公至諸從公也】癸卯司馬昭固讓相國晉公九錫之命太后詔許之戊申昭上言成濟兄弟大逆不道夷其族六月癸丑<br />
<br />
  太后詔常道鄉公更名奐【更工衡翻】甲寅常道鄉公入洛陽是日即皇帝位年十五大赦改元 丙辰詔進司馬昭爵位九錫如前昭固讓乃止 癸亥以尚書右僕射王觀為司空 吳都尉嚴密建議作浦里塘【據范書方術傳浦里塘在丹陽郡宛陵縣界陳志濮陽興傳亦云嚴密建丹陽湖田作浦里塘】羣臣皆以為難唯衛將軍陳留濮陽興以為可成【濮陽以邑為姓陳留風俗傳漢有長沙太守濮陽逸吳主休居會稽時興為太守深與相結及即位遂與張布並見信用】遂會諸軍民就作功費不可勝數【數音升】士卒多死亡民大愁怨 會稽郡謡言王亮當還為天子而亮宫人告亮使巫禱祠有惡言有司以聞吳主黜亮為候官侯遣之國【晉志曰建安郡故秦閩中郡漢高祖以封閩越王及武帝滅之徙其人名為東治後漢改為候官都尉吳置建安郡以候官為縣屬焉宋白曰漢武帝元鼎六年立都尉居候官以禦兩越所謂南北一候也】亮自殺衛送者皆伏罪冬十月陽鄉肅侯王觀卒【謚法剛德克就曰肅】 十一月詔尊<br />
<br />
  燕王待以殊禮 十二月甲午以司隸校尉王祥為司空 尚書王沈為豫州刺史初到下敎敕屬城及士民曰若有能陳長吏可否【長知兩翻】說百姓所患者給穀五百斛若說刺史得失朝政寛猛者給穀千斛主簿陳廞禇䂮【廞許今翻䂮力灼翻】入白曰敎旨思聞苦言示以勸賞竊恐拘介之士或憚賞而不言貪昧之人將慕利而妄舉苟不合宜賞不虛行則遠聽者未知當否之所在【當丁浪翻】徒見言之不用因謂設而不行愚以為告下之事可少須後【須待也】沈又敎曰夫興益於上受分於下【興益謂進言有益於上也受分謂受賞也】斯乃君子之操何不言之有禇䂮復白曰【復扶又翻下同】堯舜周公所以能致忠諫者以其欵誠之心著也氷炭不言而冷熱之質自明者以其有實也若好忠直【好呼到翻】如氷炭之自然則諤諤之言將不求而自至若德不足以配唐虞【配合也】明不足以並周公實不可以同氷炭雖懸重賞忠諫之言未可致也沈乃止<br />
<br />
  二年春三月襄陽太守胡烈【襄陽縣漢屬南郡沈約曰魏武平荆州分南郡編以北及南陽之山都立襄陽郡】表言吳將鄧由李光等十八屯同謀歸化遣使質任【質音致】欲令郡兵臨江迎拔詔王基部分諸軍徑造沮水以迎之【造七到翻應劭曰沮水出漢中房陵東入江師古曰沮千余翻南郡臨沮縣以臨沮水得名水經注曰自臨沮界東南過枝江縣又東南入于江】若由等如期到者便當因此震蕩江表基馳驛遺司馬昭書【遺于季翻下同】說由等可疑之狀且當清澄【謂事之虚實未定如水之混濁莫測其淺深且當清澄以俟之盖亦當時常語也】未宜便舉重兵深入應之又曰夷陵東西皆險陿【陿與狹同】竹木叢蔚卒有要害弩馬不陳【蔚音尉又紆勿翻卒讀曰猝謂猝然敵人於要害之地設伏邀擊弩馬不得陳其力也】今者筋角濡弱【考工記弓人為弓冬析幹春液角夏治筋秋合三材春液角夏治筋以陽煦而筋角濡滑也冬析幹秋合三材以隂凝而堅緻也春夏之交陽氣蒸潤筋角濡弱則弓弩之力不勁】水潦方降廢盛農之務要難必之利此事之危者也【要一遥翻】姜維之趣上邽【趣七喻翻】文欽之據夀春皆深入求利以取覆没此近事之鑒戒也嘉平已來累有内難【謂曹爽足弟既死累有廢立之事毌丘儉諸葛誕相繼而舉兵也難乃旦翻】當今之宜當務鎮安社稷撫寧上下力農務本懷柔百姓未宜動衆以求外利也昭累得基書意狐疑敕諸軍已上道者且權停住所在【令各就其所至之地而住軍也】須候節度【須待也】基復遺昭書曰昔漢祖納酈生之說欲封六國寤張良之謀而趣銷印【事見十卷漢高帝三年復扶又翻下同趣讀曰促】基謀慮淺短誠不及留侯亦懼襄陽有食其之謬【食其音異基】昭於是罷兵報基書曰凡處事者多曲相從順鮮能確然共盡理實【處昌呂翻鮮息淺翻】誠感忠愛每見規示輒依來旨已罷軍嚴既而由等果不降【降戶江翻】烈奮之弟也 秋八月甲寅復命司馬昭進爵位如前不受 冬十月漢主以董厥為輔國大將軍諸葛瞻為都護衛將軍共平尚書事以侍中樊建為尚書令時中常侍黄皓用事厥瞻皆不能矯正【揉曲使直曰矯】士大夫多附之唯建不與皓往來祕書令郤正久在内職與皓比屋【比毗至翻近也並也聯也又簿必翻相次也】周旋三十餘年澹然自守以書自娛既不為皓所愛亦不為皓所憎故官不過六百石【秘書令秩六百石】而亦不罹其禍漢主弟甘陵王永憎皓皓譖之使十年不得朝見【朝直遥翻見賢遍翻】吳主使五官中郎將薛珝聘于漢【珝况羽翻】及還吳主問漢政得失對曰主闇而不知其過臣下容身以求免罪入其朝不聞直言經其野民皆菜色臣聞燕雀處堂子母相樂以為至安也突决棟焚而燕雀怡然不知禍之將及其是之謂乎【魏相子順引先人之言也嗚呼蜀之亡形成矣薛珝見而知之濮陽興張布用事浦里塘之役吳民愁怨韋昭盛冲以切直而不得居王所珝亦知之否邪知而不言無亦容身而求免罪邪處昌呂翻樂音洛】珝綜之子也【薛綜見七十二卷明帝青龍元年】 是歲鮮卑索頭部大人拓跋力微始遣其子沙漠汗入貢因留為質【索音昔名翻汗音寒質音致】力微之先世居北荒不交南夏【魏收曰魏之先出自黄帝黄帝子曰昌意昌意少子受封北國有大鮮卑山因以為號黄帝以土德王北人謂土為托謂后為拔故以為氏或曰自謂托天而生抜地而長故為托拔氏蕭子顯曰匈奴女名托跋妻李陵胡俗以母為姓故為李陵之後而甚諱之有言其是陵後者輒見殺夏戶雅翻】至可汗毛始彊大【可汗北方之尊稱猶漢時之單于也宋白曰虜俗呼天為可汗可讀如渴汗何干翻】統國三十六大姓九十九後五世至可汗推寅【魏書曰漢桓帝時鮮卑檀石槐分其地為東西三部其大人曰置鞬落羅曰律推演宴荔游等皆為大帥推演蓋即推寅也按魏收魏書帝紀毛死貸立貸死觀立觀死樓立樓死越立越死推寅立推寅盖俗云鑽研之義】南遷大澤又七世至可汗鄰【推寅死利立利死俟立俟死肆立肆死機立機死蓋立蓋死儈立儈死隣立】使其兄弟七人及族人乙旃氏車惃氏【車昌遮翻惃胡昆翻又公渾翻又古本翻】分統部衆為十族【按魏書官氏志毛統國有九十九姓至鄰七分國人使諸兄弟各攝領之乃分其氏以兄為紇骨氏後改為胡氏次兄為普氏後改為周氏次兄為抜抜氏後改為長孫氏弟為逹奚氏後改為奚氏次弟為伊婁氏後改為伊氏次弟為丘敦氏後改為丘氏次弟為侯氏後改為亥氏七族之興自此始也又命叔父之胤曰乙旃氏後改為叔孫氏又命疏族為車惃氏後改為車氏凡與托抜氏為十姓百世不通婚】鄰老以位授其子詰汾使南遷遂居匈奴故地詰汾卒力微立復徙居定襄之盛樂【漢定襄郡有成樂縣後漢屬雲中郡建安二十年併雲中定襄五原朔方為新興郡郡止置一縣以屬新興而盛樂故縣弃之荒外故力微得居之後魏既盛南都平城置盛樂宫於其地永熙中又置盛樂郡復扶又翻】部衆浸盛諸部皆畏服之【拓抜氏始見于此鮮卑軻比能與魏為敵者也軻比能死北邉差安而拓抜氏盛矣為後魏張本】<br />
<br />
  資治通鑑卷七十七  <br>
   </div> 

<script src="/search/ajaxskft.js"> </script>
 <div class="clear"></div>
<br>
<br>
 <!-- a.d-->

 <!--
<div class="info_share">
</div> 
-->
 <!--info_share--></div>   <!-- end info_content-->
  </div> <!-- end l-->

<div class="r">   <!--r-->



<div class="sidebar"  style="margin-bottom:2px;">

 
<div class="sidebar_title">工具类大全</div>
<div class="sidebar_info">
<strong><a href="http://www.guoxuedashi.com/lsditu/" target="_blank">历史地图</a></strong>  
<a href="http://www.880114.com/" target="_blank">英语宝典</a>  
<a href="http://www.guoxuedashi.com/13jing/" target="_blank">十三经检索</a> 
<br><strong><a href="http://www.guoxuedashi.com/gjtsjc/" target="_blank">古今图书集成</a></strong> 
<a href="http://www.guoxuedashi.com/duilian/" target="_blank">对联大全</a> <strong><a href="http://www.guoxuedashi.com/xiangxingzi/" target="_blank">象形文字典</a></strong> 

<br><a href="http://www.guoxuedashi.com/zixing/yanbian/">字形演变</a>  <strong><a href="http://www.guoxuemi.com/hafo/" target="_blank">哈佛燕京中文善本特藏</a></strong>
<br><strong><a href="http://www.guoxuedashi.com/csfz/" target="_blank">丛书&方志检索器</a></strong> <a href="http://www.guoxuedashi.com/yqjyy/" target="_blank">一切经音义</a>  

<br><strong><a href="http://www.guoxuedashi.com/jiapu/" target="_blank">家谱族谱查询</a></strong>  <strong><a href="http://shufa.guoxuedashi.com/sfzitie/" target="_blank">书法字帖欣赏</a></strong> 
<br>

</div>
</div>


<div class="sidebar" style="margin-bottom:0px;">

<font style="font-size:22px;line-height:32px">QQ交流群9:489193090</font>


<div class="sidebar_title">手机APP 扫描或点击</div>
<div class="sidebar_info">
<table>
<tr>
	<td width=160><a href="http://m.guoxuedashi.com/app/" target="_blank"><img src="/img/gxds-sj.png" width="140"  border="0" alt="国学大师手机版"></a></td>
	<td>
<a href="http://www.guoxuedashi.com/download/" target="_blank">app软件下载专区</a><br>
<a href="http://www.guoxuedashi.com/download/gxds.php" target="_blank">《国学大师》下载</a><br>
<a href="http://www.guoxuedashi.com/download/kxzd.php" target="_blank">《汉字宝典》下载</a><br>
<a href="http://www.guoxuedashi.com/download/scqbd.php" target="_blank">《诗词曲宝典》下载</a><br>
<a href="http://www.guoxuedashi.com/SiKuQuanShu/skqs.php" target="_blank">《四库全书》下载</a><br>
</td>
</tr>
</table>

</div>
</div>


<div class="sidebar2">
<center>


</center>
</div>

<div class="sidebar"  style="margin-bottom:2px;">
<div class="sidebar_title">网站使用教程</div>
<div class="sidebar_info">
<a href="http://www.guoxuedashi.com/help/gjsearch.php" target="_blank">如何在国学大师网下载古籍?</a><br>
<a href="http://www.guoxuedashi.com/zidian/bujian/bjjc.php" target="_blank">如何使用部件查字法快速查字?</a><br>
<a href="http://www.guoxuedashi.com/search/sjc.php" target="_blank">如何在指定的书籍中全文检索?</a><br>
<a href="http://www.guoxuedashi.com/search/skjc.php" target="_blank">如何找到一句话在《四库全书》哪一页?</a><br>
</div>
</div>


<div class="sidebar">
<div class="sidebar_title">热门书籍</div>
<div class="sidebar_info">
<a href="/so.php?sokey=%E8%B5%84%E6%B2%BB%E9%80%9A%E9%89%B4&kt=1">资治通鉴</a> <a href="/24shi/"><strong>二十四史</strong></a>&nbsp; <a href="/a2694/">野史</a>&nbsp; <a href="/SiKuQuanShu/"><strong>四库全书</strong></a>&nbsp;<a href="http://www.guoxuedashi.com/SiKuQuanShu/fanti/">繁体</a>
<br><a href="/so.php?sokey=%E7%BA%A2%E6%A5%BC%E6%A2%A6&kt=1">红楼梦</a> <a href="/a/1858x/">三国演义</a> <a href="/a/1038k/">水浒传</a> <a href="/a/1046t/">西游记</a> <a href="/a/1914o/">封神演义</a>
<br>
<a href="http://www.guoxuedashi.com/so.php?sokeygx=%E4%B8%87%E6%9C%89%E6%96%87%E5%BA%93&submit=&kt=1">万有文库</a> <a href="/a/780t/">古文观止</a> <a href="/a/1024l/">文心雕龙</a> <a href="/a/1704n/">全唐诗</a> <a href="/a/1705h/">全宋词</a>
<br><a href="http://www.guoxuedashi.com/so.php?sokeygx=%E7%99%BE%E8%A1%B2%E6%9C%AC%E4%BA%8C%E5%8D%81%E5%9B%9B%E5%8F%B2&submit=&kt=1"><strong>百衲本二十四史</strong></a>  <a href="http://www.guoxuedashi.com/so.php?sokeygx=%E5%8F%A4%E4%BB%8A%E5%9B%BE%E4%B9%A6%E9%9B%86%E6%88%90&submit=&kt=1"><strong>古今图书集成</strong></a>
<br>

<a href="http://www.guoxuedashi.com/so.php?sokeygx=%E4%B8%9B%E4%B9%A6%E9%9B%86%E6%88%90&submit=&kt=1">丛书集成</a> 
<a href="http://www.guoxuedashi.com/so.php?sokeygx=%E5%9B%9B%E9%83%A8%E4%B8%9B%E5%88%8A&submit=&kt=1"><strong>四部丛刊</strong></a>  
<a href="http://www.guoxuedashi.com/so.php?sokeygx=%E8%AF%B4%E6%96%87%E8%A7%A3%E5%AD%97&submit=&kt=1">說文解字</a> <a href="http://www.guoxuedashi.com/so.php?sokeygx=%E5%85%A8%E4%B8%8A%E5%8F%A4&submit=&kt=1">三国六朝文</a>
<br><a href="http://www.guoxuedashi.com/so.php?sokeytm=%E6%97%A5%E6%9C%AC%E5%86%85%E9%98%81%E6%96%87%E5%BA%93&submit=&kt=1"><strong>日本内阁文库</strong></a> <a href="http://www.guoxuedashi.com/so.php?sokeytm=%E5%9B%BD%E5%9B%BE%E6%96%B9%E5%BF%97%E5%90%88%E9%9B%86&ka=100&submit=">国图方志合集</a> <a href="http://www.guoxuedashi.com/so.php?sokeytm=%E5%90%84%E5%9C%B0%E6%96%B9%E5%BF%97&submit=&kt=1"><strong>各地方志</strong></a>

</div>
</div>


<div class="sidebar2">
<center>

</center>
</div>
<div class="sidebar greenbar">
<div class="sidebar_title green">四库全书</div>
<div class="sidebar_info">

《四库全书》是中国古代最大的丛书,编撰于乾隆年间,由纪昀等360多位高官、学者编撰,3800多人抄写,费时十三年编成。丛书分经、史、子、集四部,故名四库。共有3500多种书,7.9万卷,3.6万册,约8亿字,基本上囊括了古代所有图书,故称“全书”。<a href="http://www.guoxuedashi.com/SiKuQuanShu/">详细>>
</a>

</div> 
</div>

</div>  <!--end r-->

</div>
<!-- 内容区END --> 

<!-- 页脚开始 -->
<div class="shh">

</div>

<div class="w1180" style="margin-top:8px;">
<center><script src="http://www.guoxuedashi.com/img/plus.php?id=3"></script></center>
</div>
<div class="w1180 foot">
<a href="/b/thanks.php">特别致谢</a> | <a href="javascript:window.external.AddFavorite(document.location.href,document.title);">收藏本站</a> | <a href="#">欢迎投稿</a> | <a href="http://www.guoxuedashi.com/forum/">意见建议</a> | <a href="http://www.guoxuemi.com/">国学迷</a> | <a href="http://www.shuowen.net/">说文网</a><script language="javascript" type="text/javascript" src="https://js.users.51.la/17753172.js"></script><br />
  Copyright &copy; 国学大师 古典图书集成 All Rights Reserved.<br>
  
  <span style="font-size:14px">免责声明:本站非营利性站点,以方便网友为主,仅供学习研究。<br>内容由热心网友提供和网上收集,不保留版权。若侵犯了您的权益,来信即刪。scp168@qq.com</span>
  <br />
ICP证:<a href="http://www.beian.miit.gov.cn/" target="_blank">鲁ICP备19060063号</a></div>
<!-- 页脚END --> 
<script src="http://www.guoxuedashi.com/img/plus.php?id=22"></script>
<script src="http://www.guoxuedashi.com/img/tongji.js"></script>

</body>
</html>
