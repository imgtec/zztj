<!DOCTYPE html PUBLIC "-//W3C//DTD XHTML 1.0 Transitional//EN" "http://www.w3.org/TR/xhtml1/DTD/xhtml1-transitional.dtd">
<html xmlns="http://www.w3.org/1999/xhtml">
<head>
<meta http-equiv="Content-Type" content="text/html; charset=utf-8" />
<meta http-equiv="X-UA-Compatible" content="IE=Edge,chrome=1">
<title>資治通鑒_194-資治通鑑卷一百九十三_194-資治通鑑卷一百九十三</title>
<meta name="Keywords" content="資治通鑒_194-資治通鑑卷一百九十三_194-資治通鑑卷一百九十三">
<meta name="Description" content="資治通鑒_194-資治通鑑卷一百九十三_194-資治通鑑卷一百九十三">
<meta http-equiv="Cache-Control" content="no-transform" />
<meta http-equiv="Cache-Control" content="no-siteapp" />
<link href="/img/style.css" rel="stylesheet" type="text/css" />
<script src="/img/m.js?2020"></script> 
</head>
<body>
 <div class="ClassNavi">
<a  href="/24shi/">二十四史</a> | <a href="/SiKuQuanShu/">四库全书</a> | <a href="http://www.guoxuedashi.com/gjtsjc/"><font  color="#FF0000">古今图书集成</font></a> | <a href="/renwu/">历史人物</a> | <a href="/ShuoWenJieZi/"><font  color="#FF0000">说文解字</a></font> | <a href="/chengyu/">成语词典</a> | <a  target="_blank"  href="http://www.guoxuedashi.com/jgwhj/"><font  color="#FF0000">甲骨文合集</font></a> | <a href="/yzjwjc/"><font  color="#FF0000">殷周金文集成</font></a> | <a href="/xiangxingzi/"><font color="#0000FF">象形字典</font></a> | <a href="/13jing/"><font  color="#FF0000">十三经索引</font></a> | <a href="/zixing/"><font  color="#FF0000">字体转换器</font></a> | <a href="/zidian/xz/"><font color="#0000FF">篆书识别</font></a> | <a href="/jinfanyi/">近义反义词</a> | <a href="/duilian/">对联大全</a> | <a href="/jiapu/"><font  color="#0000FF">家谱族谱查询</font></a> | <a href="http://www.guoxuemi.com/hafo/" target="_blank" ><font color="#FF0000">哈佛古籍</font></a> 
</div>

 <!-- 头部导航开始 -->
<div class="w1180 head clearfix">
  <div class="head_logo l"><a title="国学大师官网" href="http://www.guoxuedashi.com" target="_blank"></a></div>
  <div class="head_sr l">
  <div id="head1">
  
  <a href="http://www.guoxuedashi.com/zidian/bujian/" target="_blank" ><img src="http://www.guoxuedashi.com/img/top1.gif" width="88" height="60" border="0" title="部件查字,支持20万汉字"></a>


<a href="http://www.guoxuedashi.com/help/yingpan.php" target="_blank"><img src="http://www.guoxuedashi.com/img/top230.gif" width="600" height="62" border="0" ></a>


  </div>
  <div id="head3"><a href="javascript:" onClick="javascript:window.external.AddFavorite(window.location.href,document.title);">添加收藏</a>
  <br><a href="/help/setie.php">搜索引擎</a>
  <br><a href="/help/zanzhu.php">赞助本站</a></div>
  <div id="head2">
 <a href="http://www.guoxuemi.com/" target="_blank"><img src="http://www.guoxuedashi.com/img/guoxuemi.gif" width="95" height="62" border="0" style="margin-left:2px;" title="国学迷"></a>
  

  </div>
</div>
  <div class="clear"></div>
  <div class="head_nav">
  <p><a href="/">首页</a> | <a href="/ShuKu/">国学书库</a> | <a href="/guji/">影印古籍</a> | <a href="/shici/">诗词宝典</a> | <a   href="/SiKuQuanShu/gxjx.php">精选</a> <b>|</b> <a href="/zidian/">汉语字典</a> | <a href="/hydcd/">汉语词典</a> | <a href="http://www.guoxuedashi.com/zidian/bujian/"><font  color="#CC0066">部件查字</font></a> | <a href="http://www.sfds.cn/"><font  color="#CC0066">书法大师</font></a> | <a href="/jgwhj/">甲骨文</a> <b>|</b> <a href="/b/4/"><font  color="#CC0066">解密</font></a> | <a href="/renwu/">历史人物</a> | <a href="/diangu/">历史典故</a> | <a href="/xingshi/">姓氏</a> | <a href="/minzu/">民族</a> <b>|</b> <a href="/mz/"><font  color="#CC0066">世界名著</font></a> | <a href="/download/">软件下载</a>
</p>
<p><a href="/b/"><font  color="#CC0066">历史</font></a> | <a href="http://skqs.guoxuedashi.com/" target="_blank">四库全书</a> |  <a href="http://www.guoxuedashi.com/search/" target="_blank"><font  color="#CC0066">全文检索</font></a> | <a href="http://www.guoxuedashi.com/shumu/">古籍书目</a> | <a   href="/24shi/">正史</a> <b>|</b> <a href="/chengyu/">成语词典</a> | <a href="/kangxi/" title="康熙字典">康熙字典</a> | <a href="/ShuoWenJieZi/">说文解字</a> | <a href="/zixing/yanbian/">字形演变</a> | <a href="/yzjwjc/">金 文</a> <b>|</b>  <a href="/shijian/nian-hao/">年号</a> | <a href="/diming/">历史地名</a> | <a href="/shijian/">历史事件</a> | <a href="/guanzhi/">官职</a> | <a href="/lishi/">知识</a> <b>|</b> <a href="/zhongyi/">中医中药</a> | <a href="http://www.guoxuedashi.com/forum/">留言反馈</a>
</p>
  </div>
</div>
<!-- 头部导航END --> 
<!-- 内容区开始 --> 
<div class="w1180 clearfix">
  <div class="info l">
   
<div class="clearfix" style="background:#f5faff;">
<script src='http://www.guoxuedashi.com/img/headersou.js'></script>

</div>
  <div class="info_tree"><a href="http://www.guoxuedashi.com">首页</a> > <a href="/SiKuQuanShu/fanti/">四库全书</a>
 > <h1>资治通鉴</h1> <!--         下载:【右键另存为】即可 --></div>
  <div class="info_content zj clearfix">
  
<div class="info_txt clearfix" id="show">
<center style="font-size:24px;">194-資治通鑑卷一百九十三</center>
    資治通鑑卷一百九十三 宋 司馬光 撰<br />
<br />
  胡三省 音注<br />
<br />
  唐紀九【起著雍困敦九月盡重灮單閼凡三年有奇】<br />
<br />
  太宗文武大聖大廣孝皇帝上之中<br />
<br />
  貞觀二年九月丙午初令致仕官在本品之上【按唐會要是時詔内外文武官年老致仕者參朝之班宜在本品見任之上觀古玩翻】 上曰比見羣臣屢上表賀祥瑞【比毗至翻上時掌翻】夫家給人足而無瑞不害為堯舜百姓愁怨而多瑞不害為桀紂【夫音扶】後魏之世吏焚連理木煮白雉而食之豈足為至治乎【治直吏翻】丁未詔自今大瑞聽表聞【按儀制令凡景星慶雲為大瑞其名物六十有四白狼赤兎為上瑞其名物三十有八蒼烏朱鴈為中瑞其名物三十有二嘉禾芝草木連理為下瑞其名物十四】自外諸瑞申所司而已【唐六典禮部郎中凡祥瑞應見皆辯其物名】嘗有白鵲構巢於寢殿槐上合歡如腰鼓【合音閤】左右稱賀上曰我常笑隋煬帝好祥瑞【好呼到翻】瑞在得賢此何足賀命毁其巢縱鵲於野外 天少雨【少詩沼翻】中書舍人李百藥上言【上時掌翻】往年雖出宫人竊聞太上皇宫及掖庭宫人無用者尚多豈惟虚費衣食且隂氣鬱積亦足致旱上曰婦人幽閉深宫誠為可愍灑掃之餘亦何所用【灑所賣翻掃素報翻又並如字】宜皆出之任求伉儷【伉苦浪翻儷郎計翻】於是遣尚書左丞戴胄給事中洹水杜正倫【洹水縣周建德六年分臨漳東北界置屬魏州洹于元翻】於掖庭西門簡出之【掖音亦】前後所出三千餘人 己未突厥寇邊【厥九勿翻】朝臣或請修古長城【古長城秦蒙恬所築者也自漢至隋沿邊所築城障非一處而長城之延袤未有如秦者也朝直遥翻】發民乘堡障上曰突厥災異相仍頡利不懼而修德暴虐滋甚骨肉相攻亡在朝夕朕方為公掃清沙漠【為于偽翻】安用勞民遠修障塞乎 壬申以前司農卿竇靜為夏州都督【夏戶雅翻】靜在司農少卿趙元楷善聚斂【少始照翻斂力贍翻下重斂同】靜鄙之對官屬大言曰【司農官屬有丞主簿上林太倉鉤盾導官四署令丞太倉永豐龍門等倉司竹慶善石門温泉湯等監京都諸宫苑總監諸園苑監苑四面監九成宫監諸鹽池監諸屯監各有監副監丞苑總監又有主簿諸鹽池諸屯監無副監】隋煬帝奢侈重斂司農非公不可今天子節儉愛民公何所用哉元楷大慙 上問王珪曰近世為國者益不及前古何也對曰漢世尚儒術宰相多用經術士故風俗淳厚近世重文輕儒參以法律此治化之所以益衰也【治直吏翻】上然之 冬十月御史大夫參預朝政安吉襄公杜淹薨【朝直遥翻下同】 交州都督遂安公壽以貪得罪【遂安公壽宗室也】上以瀛州刺史盧祖尚才兼文武廉平公直徵入朝諭以交趾久不得人須卿鎮撫祖尚拜謝而出既而悔之辭以舊疾上遣杜如晦等諭旨曰匹夫猶敦然諾【敦然諾猶重然諾也言既許人則必踐言】柰何既許朕而復悔之祖尚固辭戊子上復引見諭之【復扶又翻見賢遍翻】祖尚固執不可上大怒曰我使人不行何以為政命斬於朝堂【閣本太極宫圖東西朝堂在承天門左右承天門外朝也東朝堂之前有肺石西朝堂之前有登聞鼔】尋悔之他日與侍臣論齊文宣帝何如人魏徵對曰文宣狂暴然人與之爭事理屈則從之有前青州長史魏愷使於梁還除光州長史不肯行【長知兩翻使疏吏翻下同】楊遵彦奏之文宣怒召而責之愷曰臣先任大州使還有勞無過更得小州此臣所以不行也文宣顧謂遵彦曰其言有理卿赦之此其所長也【楊愔字遵彦相齊文宣帝大見親任】上曰然曏者盧祖尚雖失人臣之義朕殺之亦為太暴由此言之不如文宣矣命復其官蔭【復其官則得蔭其子若孫唐制凡用蔭一品子正七品上二品子正七品下三品子從七品上從三品子從七品下正四品子正八品上從四品子正八品下正五品子從八品上從五品及國公子從八品下三品以上蔭曾孫五品以上蔭孫孫降子一等曾孫降孫一等贈官降正官一等死事者與正官同郡縣公子視從五品孫縣男以上子降一等勲官二品子又降一等二王後孫視正三品】徵狀貌不逾中人而有膽畧善囘人主意每犯顔苦諫或逢上怒甚徵神色不移上亦為霽威【人主之威重於雷霆霽威言猶雨霽則雷霆亦收威為于偽翻】嘗謁告上冢【上時掌翻】還言於上曰人言陛下欲幸南山外皆嚴裝已畢而竟不行何也上笑曰初實有此心畏卿嗔故中輟耳【嗔昌真翻】上嘗得佳鷂自臂之望見徵來匿懷中徵奏事固久不已鷂竟死懷中 十一月辛酉上祀圜丘【武德元年制每歲冬至祀昊天上帝於圜丘以景皇帝配】 十二月壬午以黄門侍郎王珪為守侍中上嘗閒居【閒讀曰閑】與珪語有美人侍側上指示珪曰此廬江王瑗之姬也【廬江王瑗反死見一百九十一卷武德九年瑗于眷翻】瑗殺其夫而納之珪避席曰陛下以廬江納之為是邪非邪【邪音耶】上曰殺人而取其妻卿何問是非對曰昔齊桓公知郭公之所以亡由善善而不能用然棄其所言之人管仲以為無異於郭公【齊桓公遇郭氏之墟問父老曰郭何故亡對曰善善惡惡公曰若子之言何至於亡對曰善善而不能用惡惡而不能去此其所以亡也】今此美人尚在左右臣以為聖心是之也上悅即出之還其親族 【考異曰實錄新舊書皆云帝雖不出此美人而甚重其言按太宗賢主既重珪言何得反棄而不用乎且是人汎侍左右又非嬖寵著名之人太宗何愛而留之今從貞觀政要】上使太常少卿祖孝孫教宫人音樂不稱旨【少始照翻稱尺證翻】上責之温彦博王珪諫曰孝孫雅士今乃使之教宫人又從而譴之臣竊以為不可上怒曰朕寘卿等於腹心當竭忠直以事我乃附下罔上為孝孫遊說邪【為于偽翻說輸芮翻】彦博拜謝珪不拜曰陛下責臣以忠直今臣所言豈私曲邪【邪音耶】此乃陛下負臣非臣負陛下上默然而罷明日上謂房玄齡曰自古帝王納諫誠難朕昨責温彦博王珪至今悔之 【考異曰魏文貞公故事太宗曰人皆以祖孝孫為知音令教聲曲多不諧韻此其未至精妙為不存意乎乃勅所司令舉其罪公進諫曰陛下生平不愛音聲今忽為教女樂責孝孫臣恐天下怪愕太宗曰汝等並是我心腹應須中正向乃附下罔上為孝孫辭温彦博等拜謝公及王珪進曰陛下不以臣等不肖置之樞近今臣所言豈是為私不意陛下責臣至此常奉明旨勿以臨時嗔怒即便曲從成我大過臣等不敢失墜所以每觸龍鱗今以為責只是陛下負臣臣終不負陛下太宗怒未已懍然作色公又曰祖孝孫學問立身何如白明達陛下平生禮遇孝孫復何如白明達今過聽一言便謂孝孫可疑明達可信臣恐羣臣衆庶有以窺陛下太宗怒乃解今從舊傳】公等勿為此不盡言也【為于偽翻下為朕同】 上曰為朕養民者唯在都督刺史朕常疏其名於屏風坐卧觀之得其在官善惡之跡皆注於名下以備黜陟縣令尤為親民不可不擇乃命内外五品已上各舉堪為縣令者以名聞 上曰比有奴告其主反者【比毗至翻】此弊事夫謀反不能獨為必與人共之何患不發何必使奴告邪【邪音耶】自今有奴告主者皆勿受仍斬之 西突厥統葉護可汗為其伯父所殺伯父自立是為莫賀咄侯屈利俟毗可汗【厥九勿翻可從刋入聲汗音寒咄當没翻俟渠之翻】國人不服弩失畢部推泥孰莫賀設為可汗【西突厥有五弩失畢部泥孰亦一啜之部帥】泥孰不可統葉護之子咥力特勒【咥徒結翻又丑栗翻】避莫賀咄之禍亡在康居泥孰迎而立之是為乙毗鉢羅肆葉護可汗與莫賀咄相攻連兵不息俱遣使來請昏【使疏吏翻】上不許曰汝國方亂君臣未定何得言昏且諭以各守部分勿復相攻【分扶問翻復扶又翻】於是西域諸國及敕勒先役屬西突厥者皆叛之【史言天方福華東西突厥皆亂厥九勿翻】突厥北邊諸姓多叛頡利可汗歸薛延陁共推其俟斤夷男為可汗【頡奚結翻俟渠之翻】夷男不敢當上方圖頡利遣遊擊將軍喬師望間道齎册書拜夷男為真珠毗伽可汗【間古莧翻伽求迦翻】賜以鼓纛【纛徒到翻】夷男大喜遣使入貢【使疏吏翻】建牙於大漠之鬱督軍山下東至靺鞨西至西突厥南接沙磧【靺音末鞨音曷磧七迹翻】北至俱倫水迴紇拔野古阿跌同羅僕骨霫諸部皆屬焉【史言突厥衰而薛延陁彊於漠北霫而立翻】<br />
<br />
  三年春正月戊午上祀太廟癸亥耕藉於東郊【初議藉田方面所在給事中孔頴達曰禮天子藉田於南郊諸侯於東郊晉武帝猶於東南今於城東不合古禮帝曰禮緣人情亦何常之有且虞書云平秩東作則是堯舜敬授人時已在東矣又乘青輅載黛耜者所以順於春氣故知合於東方且朕見居少陽之地田於東郊蓋其宜也於是遂定按帝自謂居少陽之地蓋以即位以來居東宫也藉秦昔翻】 沙門法雅坐妖言誅【妖一遥翻】司空裴寂嘗聞其言辛未寂坐免官遣還鄉里寂請留京師上數之曰【數所具翻又所主翻】計公勲庸安得至此直以恩澤為羣臣第一武德之際貨賂公行紀綱紊亂皆公之由也【上皇聞帝此言其心為如何紊音問】但以故舊不忍盡法得歸守墳墓幸已多矣寂遂歸蒲州【裴寂本蒲州桑泉人】未幾又坐狂人信行言寂有天命寂不以聞當死流靜州【武德四年以始安郡之龍平豪靜蒼梧郡之蒼梧置靜州靜平郡幾居豈翻】會山羌作亂【以為山羌則當是劒南之靜州然劒南之靜州武后時方置若以為嶺南之靜州則羌當作蠻】或言刼寂為主上曰寂當死我生之必不然也俄聞寂率家僮破賊上思其佐命之功徵入朝會卒【帥讀曰率朝直遥翻下同卒子恤翻】 二月戊寅以房玄齡為左僕射杜如晦為右僕射以尚書右丞魏徵守祕書監參預朝政三月己酉上錄繫囚有劉恭者頸有勝文自云當勝<br />
<br />
  天下坐是繫獄上曰若天將興之非朕所能除若無天命勝文何為乃釋之 丁巳上謂房玄齡杜如晦曰公為僕射當廣求賢人隨才授任此宰相之職也【唐六典左右僕射左右丞相之職也掌總領六官紀綱百揆】比聞聽受辭訟【比毗至翻下比見比來同】日不暇給安能助朕求賢乎因敕尚書細務屬左右丞【屬之欲翻付也】唯大事應奏者乃關僕射玄齡明達政事輔以文學夙夜盡心惟恐一物失所用法寛平聞人有善若己有之不以求備取人不以己長格物與杜如晦引拔士類常如不及至於臺閣規模皆二人所定上每與玄齡謀事必曰非如晦不能决及如晦至卒用玄齡之策蓋玄齡善謀如晦能斷故也【卒子恤翻斷丁亂翻】二人深相得同心徇國故唐世稱賢相推房杜焉玄齡雖蒙寵待或以事被譴輒累日詣朝堂稽顙請罪恐懼若無所容【史言房玄齡忠謹被皮義翻朝直遥翻稽音啟】玄齡監修國史上語之曰【唐以宰相監修國史至今因之監工銜翻語牛倨翻】比見漢書載子虚上林賦浮華無用其上書論事詞理切直者朕從與不從皆當載之【太宗之存心如此安有有獻而不納者乎上時掌翻】 夏四月乙亥上皇徙居弘義宫更名大安宫【唐會要武德五年營弘義宫以帝有尅定天下之功别建此宫以居之既禪位高祖以弘義宫有山林勝景雅好之遂徙居焉改名大安宫馬周所謂大安宫在城之西者也更工衡翻】上始御太極殿【高祖之傳位也帝即位於東宫之顯德殿高祖徙居大安宫帝始御太極殿】謂羣臣曰中書門下機要之司詔敕有不便者皆應論執比來唯睹順從不聞違異若但行文書則誰不可為何必擇才也房玄齡等皆頓首謝故事凡軍國大事則中書舍人各執所見雜署其名謂之五花判事中書侍郎中書令省審之給事中黄門侍郎駮正之上始申明舊制由是鮮有敗事【省悉景翻駮北角翻鮮息善翻】 茌平人馬周【茌平縣漢屬東郡應劭曰在茌山之平地者也後魏屬東平原郡後齊廢隋開皇初復置屬貝州唐屬博州賢曰漢茌平故城在博州之聊城縣東北茌仕疑翻】客遊長安舍於中郎將常何之家【唐諸衛中郎將正四品下將即亮翻】六月壬午以旱詔文武官極言得失何武人不學不知所言周代之陳便宜二十餘條 【考異曰舊傳云貞觀五年據實錄詔在此年五年不見有詔令百官上封事今從唐歷附此】上怪其能以問何對曰此非臣所能家客馬周為臣具草耳【為于偽翻】上即召之未至遣使督促者數輩及謁見與語甚悦令直門下省尋除監察御史奉使稱旨【鄭樵曰秦以御史監郡謂之監察御史漢罷其名晉孝武太元中始置檢校御史掌行馬外事隋改檢校御史為監察御史使疏吏翻見賢遍翻監工銜翻使疏吏翻下同稱尺證翻】上以常何為知人賜絹三百匹 秋八月己巳朔日有食之 丙子薛延陁毗伽可汗【伽求迦翻可從刋入聲汗音寒】遣其弟統特勒入貢上賜以寶刀及寶鞭謂曰卿所部有大罪者斬之小罪者鞭之夷男甚喜突厥頡利可汗大懼始遣使稱臣請尚公主修壻禮代州都督張公謹上言突厥可取之狀以為頡利縱欲逞暴誅忠良暱姦佞一也【頡奚結翻上時掌翻暱尼質翻】薛延陁等諸部皆叛二也【薛延陁諸部叛突厥事始上卷二年】突利拓設欲谷設皆得罪無所自容三也【突利得罪見上卷二年拓設即阿史那社爾與欲谷設分統敕勒諸部欲谷設即為囘紇所破者也按舊書李大亮傳頡利既亡之後拓設諸種散在伊吾】塞北霜旱糇糧乏絶四也【糇音侯】頡利疏其族類親委諸胡胡人反覆大軍一臨必生内變五也華人入北其衆甚多【華人因隋末之亂避而入北】比聞所在嘯聚保據山險大軍出塞自然響應六也【比毗至翻】上以頡利可汗既請和親復援梁師都【事見上卷上年復扶又翻】丁亥命兵部尚書李靖為行軍總管討之以張公謹為副九月丙午突厥俟斤九人帥三千騎來降戊午拔野古僕骨同羅奚酋長並帥衆來降【厥九勿翻俟渠之翻帥讀曰率騎奇寄翻降戶江翻酋慈由翻長知兩翻】 冬十一月辛丑突厥寇河西肅州刺史公孫武達【武德二年分甘州之福祿瓜州之玉門置肅州酒泉郡】甘州刺史成仁重與戰破之【甘肅二州相去四百二十里】捕虜千餘口 上遣使至涼州都督李大亮有佳鷹使者諷大亮使獻之大亮密表曰陛下久絶畋遊而使者求鷹【使疏吏翻下同】若陛下之意深乖昔旨【昔旨謂絶畋遊之旨】如其自擅乃是使非其人癸卯上謂侍臣曰李大亮可謂忠直手詔褒美賜以胡缾及荀悅漢紀【按舊書李大亮傳帝詔曰今賜卿胡缾一枚雖無千鎰之重乃朕自用之物荀悅漢紀叙致既明論議深博極為治之體盡君臣之義今以賜卿宜加尋閱】 庚申以行并州都督李世勣為通漢道行軍總管【舊書李勣傳作通漠道當從之後高宗朝裴行儉遣兵由通漠道掩取阿史那伏念輜重】兵部尚書李靖為定襄道行軍總管華州刺史柴紹為金河道行軍總管【華戶化翻】靈州大都督薛萬徹為暢武道行軍總管【暢武非地名也營州邊於東胡故命萬徹為總管使之宣暢威武以美名寵之耳新書帝紀作營州都薛萬徹】衆合十餘萬皆受世勣節度分道出擊突厥乙丑任城王道宗擊突厥於靈州破之【任音壬】十二月戊辰突利可汗入朝【朝直遥翻】上謂侍臣曰往者太上皇以百姓之故稱臣於突厥【事見一百八十四卷隋恭帝義寧元年六月一本此下有考異】朕常痛心今單于稽顙庶幾可雪前恥【稽音啟幾居希翻】壬午靺鞨遣使入貢【靺音末鞨音曷使疏吏翻】上曰靺鞨遠來蓋突厥已服之故也昔人謂禦戎無上策【嚴尤諫王莽曰匈奴為害所從來久周秦漢征之皆未有得上策者也周得中策漢得下策秦無策焉】朕今治安中國而四夷自服豈非上策乎 癸未右僕射杜如晦以疾遜位上許之 乙酉上問給事中孔頴達曰論語以能問於不能以多問於寡有若無實若虚【曾子之言】何謂也頴達具釋其義以對且曰非獨匹夫如是帝王亦然帝王内藴神明外當玄默故易稱以蒙養正以明夷莅衆【易曰蒙以養正聖功也明夷君子以莅衆用晦而明】若位居尊極炫耀聰明【炫熒絹翻】以才陵人飾非拒諫則下情不通取亡之道也上深善其言庚寅突厥郁射設帥所部來降【厥九勿翻降戶江翻】 閏月丁<br />
<br />
  未東謝酋長謝元深南謝酋長謝強來朝諸謝皆南蠻别種在黔州之西【東謝蠻在西爨之南居黔州之西三百里南謝蠻在隋牂柯郡地南百里有桂領關酋慈由翻長知兩翻朝直遥翻下同種章勇翻黔音琴】詔以東謝為應州南謝為莊州隸黔州都督【宋白曰黔州黔中郡秦置漢通謂五溪之地又為武陵郡之酉陽縣地武帝於此置涪陵縣蜀先主立黔安郡後周建德三年置黔州貞觀四年移州治於涪陵江東彭水之東】是時遠方諸國來朝貢者甚衆服裝詭異中書侍郎顔師古請圖寫以示後作王會圖從之 【考異曰實錄新舊傳皆云正會圖按汲冢周書有王會篇柳宗元鐃鼔歌呂述黠戛斯朝貢圖皆作王會今從之】乙丑牂柯酋長謝能羽及充州蠻入貢【牂音臧柯音哥】詔以牂柯為牂州【昆明東九百里即牂柯蠻國其王號鬼主其别帥曰羅殿王東距辰州二千四百里其南一千五百里即交州也牂州之北一百五十里有别部曰充州蠻牂柯音臧哥】党項酋長細封步賴來降以其地為軌州各以其酋長為刺史党項地亘三千里姓别為部不相統壹細封氏費聽氏往利氏頗超氏野辭氏旁當氏米擒氏拓跋氏皆大姓也步賴既為唐所禮餘部相繼來降以其地為崌奉巖遠四州【党項漢西羌别種魏晉後微甚周滅宕昌鄧至而党項始強其地古析支也東距松州西葉護南春桑述桑等羌北吐谷渾山谷崎嶇大抵三千里拓跋氏之後為西夏李繼遷党底朗翻酋慈由翻長知兩翻降戶江翻】 是歲戶部奏中國人自塞外歸及四夷前後降附者男女一百二十餘萬口 房玄齡王珪掌内外官考【唐考法凡百司之長歲校其屬功過差以九等流内之官叙以四善一曰德義有聞二曰清慎明著三曰公平可稱四曰恪勤匪懈善狀之外有二十七最一曰獻可替否拾遺補闕為近侍之最二曰銓衡人物擢盡才良為選司之最三曰揚清激濁褒貶必當為考校之最四曰禮制儀式動合經典為禮官之最五曰音律克諧不失節奏為樂官之最六曰決斷不滯與奪合理為判事之最七曰部統有方警守無失為宿衛之最八曰兵士調習戎裝充備為督領之最九曰推鞫得情處斷平允為法官之最十曰讐校精審明於刋定為校正之最十一曰承旨敷奏吐納明敏為宣納之最十二曰訓導有方生徒充業為學官之最十三曰賞罰嚴明攻戰必勝為軍將之最十四曰禮義興行肅清所部為政教之最十五曰詳錄典正詞理兼舉為文史之最十六曰訪察精審彈舉必當為糾正之最十七曰明於勘覆稽失無隱為句檢之最十八曰職事修理供承彊濟為監察之最十九曰功課皆充丁匠無怨為役使之最二十曰耕耨以時收穫成課為屯官之最二十一曰謹於蓋藏明於出納為倉庫之最二十二曰推步盈虚究理精密為歷官之最二十三曰占候醫卜效驗多著為方術之最二十四曰檢察有方行旅無壅為關津之最二十五曰市㕓弗擾姦濫不行為市司之最二十六曰牧養肥殖蕃息滋多為牧官之最二十七曰邊境清肅城隍修理為鎮防之最一最四善為上上一最三善為上中一最二善為上下無最而有二善為中上無最而有一善為中中職事粗理善最不聞為中下愛憎任情處斷乖理為下上背公向私職事廢闕為下中居官諂詐貪濁有狀為下下凡定考皆集於尚書省唱第然後奏】治書侍御史萬年權萬紀奏其不平【治直之翻】上命侯君集推之魏徵諫曰玄齡珪皆朝廷舊臣素以忠直為陛下所委所考既多其間能無一二人不當【當丁浪翻】察其情終非阿私若推得其事則皆不可信豈得復當重任且萬紀比來恒在考堂曾無駮正【復扶又翻比毗至翻恒戶登翻】及身不得考乃始陳論此正欲激陛下之怒非竭誠徇國也使推之得實未足裨益朝廷若其本虚徒失陛下委任大臣之意臣所愛者治體【治直吏翻】非敢苟私二臣上乃釋不問 濮州刺史龎相壽坐貪汚解任【濮博木翻龎薄江翻】自陳嘗在秦王幕府上憐之欲聽還舊任魏徵諫曰秦王左右中外甚多恐人人皆恃恩私足使為善者懼上欣然納之謂相壽曰我昔為秦王乃一府之主今居大位乃四海之主不得獨私故人大臣所執如是朕何敢違賜帛遣之相壽流涕而去四年春正月李靖帥驍騎三千自馬邑進屯惡陽嶺【惡陽嶺在定襄古城南善陽嶺在白道川南帥讀曰率驍堅堯翻騎奇寄翻】夜襲定襄破之【舊志朔州馬邑郡治善陽縣漢定襄縣地有秦時馬邑城武周塞後魏置桑乾郡隋置善陽縣又隋志雲州定襄郡治大利城即文帝所築以處突厥啟民可汗者也李靖所襲破者當是此城唐謂之北定襄城又舊志曰雲州隋馬邑郡之雲内縣恒安鎮也貞觀十四年自朔州北定襄城移雲州及定襄縣置於此即後魏所都平城也開元二十年改定襄為雲中縣而武德四年已分忻州之秀容為定襄縣今見於九域志者忻州之定襄而北定襄自石晉割地入於北國其名晦矣宋祁曰古定襄城其地南大河北白道畜牧廣衍龍荒之最壤宋白曰朔州北三百餘里定襄故城後魏初之雲中也】突厥頡利可汗不意靖猝至【厥九勿翻頡奚結翻可從刋入聲汗音寒】大驚曰唐不傾國而來靖何敢孤軍至此其衆一日數驚乃徙牙於磧石【大磧之口也磧七迹翻】靖復遣諜離其心腹【復扶又翻諜達協翻】頡利所親康蘇密以隋蕭后及煬帝之孫政道來降【蕭后入突厥見一百八十八卷高祖武德二年降戶江翻下同】乙亥至京師先是有降胡言中國人或潛通書啟於蕭后者【先悉薦翻】至是中書舍人楊文瓘請鞫之上曰天下未定突厥方彊愚民無知或有斯事今天下已安既往之罪何須問也李世勣出雲中與突厥戰於白道大破之【漢地理志雲中郡治雲中縣酈道元曰雲中城東八十里有成樂城今雲中郡治一名石盧城又有後魏雲中宫在雲中故城東四十里虞氏記云趙武侯自五原河曲築長城東至陰山又於河西造一大城其一箱崩不就乃改卜隂山河曲而禱焉畫見羣鵠遊於雲中徘徊經日見大光在其下武侯曰此為我乎乃即其處築城今雲中故城是也又有芒于水出塞外南逕陰山東西千餘里芒于水又西南逕白道南谷口有城在右策帶長城背山面澤謂之白道自北出有高阪謂之白道嶺芒于水又南西逕雲中城北新志雲州雲中縣有隂山道青坡道皆出兵路宋白曰漢雲中郡在唐勝州東北四十里榆林縣界雲中故城是也趙武侯所築漢五原故城亦在今勝州榆林縣界】 二月己亥上幸驪山温湯【驪力知翻】 甲辰李靖破突厥頡利可汗於陰山【厥九勿翻頡奚結翻可從刋入聲汗音寒】先是頡利既敗竄于鐵山【鐵山蓋在陰山北先悉薦翻】餘衆尚數萬遣執失思力入見謝罪請舉國内附身自入朝【見賢遍翻朝直遥翻】上遣鴻臚卿唐儉等慰撫之又詔李靖將兵迎頡利【臚陵如翻將即亮翻】頡利外為卑辭内實猶豫欲俟草青馬肥亡入漠北靖引兵與李世勣會白道相與謀曰 【考異曰舊書靖傳以為謀出於靖勣傳以為謀出於勣蓋相與謀耳】頡利雖敗其衆猶盛若走度磧北保依九姓【新書囘鶻傳有九姓曰藥羅葛曰胡咄葛曰啒羅勿曰貃歌息訖曰阿勿嘀曰葛薩曰斛嗢素曰藥勿葛曰奚邪勿此囘紇後來彊盛所服九姓是時所謂九姓即謂拔野古延陁囘紇之屬】道阻且遠追之難及今詔使至彼【使疏吏翻】虜必自寛若選精騎一萬齎二十日糧往襲之不戰可擒矣【騎奇寄翻】以其謀告張公謹公謹曰詔書已許其降【降戶江翻】使者在彼柰何擊之靖曰此韓信所以破齊也【謂漢遣酈食其說下齊韓信乘其無備襲破之使疏吏翻】唐儉輩何足惜遂勒兵夜發世勣繼之軍至陰山遇突厥千餘帳俘以隨軍頡利見使者大喜意自安靖使武邑蘇定方帥二百騎為前鋒【武邑縣前漢屬信都後漢屬安平晉屬武邑郡後齊廢隋開皇六年復置屬冀州帥讀曰率下同】乘霧而行去牙帳七里虜乃覺之頡利乘千里馬先走靖軍至虜衆遂潰 【考異曰舊書靖傳曰靖軍逼其牙帳十五里虜始覺定方傳曰靖使定方為前鋒乘霧而行去賊一里許忽然霧歇望見其牙帳掩擊殺數十百人頡利畏威先走今從唐歷】唐儉脫身得歸靖斬首萬餘級俘男女十餘萬獲雜畜數十萬【畜許救翻】殺隋義成公主擒其子疊羅施頡利帥萬餘人欲度磧李世勣軍於磧口頡利至不得度其大酋長皆帥衆降【頡奚結翻帥讀曰率磧七迹翻】世勣虜五萬餘口而還【還從宣翻又如字】斥地自陰山北至大漠【此後方盡有隋恒安定襄之地】露布以聞 丙午上還宫 甲寅以克突厥赦天下【厥九勿翻】以御史大夫温彦博為中書令守侍中王珪為侍中守戶部尚書戴胄為戶部尚書參預朝政太常少卿蕭瑀為御史大夫與宰臣參議朝政【朝直遥翻少詩照翻瑀音禹】 三月戊辰以突厥夾畢特勒阿史那思摩為右武候大將軍四夷君長詣闕請上為天可汗【長知兩翻下同可從刋入聲汗音寒】上曰我為大唐天子又下行可汗事乎羣臣及四夷皆稱萬歲是後以璽書賜西北君長皆稱天可汗【璽斯氏翻】庚午突厥思結俟斤帥衆四萬來降【俟渠之翻】丙子以突利可汗為右衛大將軍北平郡王初始畢可汗以啟民母弟蘇尼失為沙鉢羅設督部落五萬家牙直靈州西北及頡利政亂蘇尼失所部獨不攜貳【尼女夷翻】突利之來奔也【見去年十二月】頡利立之為小可汗及頡利敗走往依之將奔吐谷渾【吐從暾入聲谷音浴】大同道行軍總管任城王道宗引兵逼之【新志曰黄河東濡有古大同城今大同城永濟栅也比逕大泊十七里至金河任音壬】使蘇尼失執送頡利頡利以數騎夜走匿於荒谷【頡奚結翻騎奇寄翻】蘇尼失懼馳追獲之庚辰行軍副總管張寶相帥衆奄至沙鉢羅營俘頡利送京師蘇尼失舉衆來降【帥讀曰率 考異曰太宗實錄云蘇尼失舉衆歸國因以頡利屬於軍吏舊傳云蘇尼失令子忠擒頡利以獻蓋寶相逼之而蘇尼失使忠獻之也】漠南之地遂空 蔡成公杜如晦疾篤【杜如晦先封蔡國公薨後徙封萊國公賀琛諡法佐相克終曰成民和臣福曰成】上遣太子問疾又自臨視之甲申薨上每得佳物輒思如晦遣使賜其家【使疏吏翻】久之語及如晦必流涕謂房玄齡曰公與如晦同佐朕今獨見公不見如晦矣 突厥頡利可汗至長安【厥九勿翻】夏四月戊戌上御順天樓【舊書帝紀曰御順天門唐六典皇城南門中曰承天門隋開皇二年作初曰廣陽門仁壽元年改曰昭陽門武德元年改曰順天門神龍元年改曰承天門若元正冬至大陳設燕會赦過宥罪除舊布新受萬國之朝貢四夷之賓客則御承天門以聽政蓋古之外朝也順天樓即順天門樓】盛陳文物引見頡利數之曰汝藉父兄之業縱淫虐以取亡罪一也數與我盟而背之二也恃彊好戰暴骨如莽三也蹂我稼穡掠我子女四也我宥汝罪存汝社稷而遷延不來五也然自便橋以來不復大入為寇【便橋事見一百九十一卷高祖武德九年見賢遍翻數所具翻又所主翻數與所角翻背蒲妹翻好呼報翻暴步卜翻復扶又翻下復何同蹂人九翻】以是得不死耳頡利哭謝而退詔館於太僕厚廩食之【館古換翻食讀曰飤】上皇聞擒頡利歎曰漢高祖困白登不能報今我子能滅突厥吾託付得人復何憂哉【復扶又翻】上皇召上與貴臣十餘人及諸王妃主置酒凌烟閣【閣本太極宫圖兩儀殿之北為延嘉殿延嘉殿之東為功臣閣功臣閣之東為凌烟閣】酒酣上皇自彈琵琶上起舞公卿迭起為夀逮夜而罷突厥既亡【厥九勿翻】其部落或北附薛延陁或西奔西域其降唐者尚十萬口詔羣臣議區處之宜【降戶江翻下同處昌呂翻】朝士多言北狄自古為中國患今幸而破亡宜悉徙之河南兖豫之間【此兖豫言禹迹九州大界也朝直遥翻】分其種落【種章勇翻下種類同】散居州縣教之耕織可以化胡虜為農民永空塞北之地中書侍郎顔師古以為突厥鐵勒皆上古所不能臣陛下既得而臣之請皆寘之河北【河北謂北河之北】分立酋長領其部落則永永無患矣【酋慈由翻長知兩翻下同】禮部侍郎李百藥以為突厥雖云一國然其種類區分各有酋帥【帥所類翻】今宜因其離散各即本部署為君長【長知兩翻】不相臣屬縱欲存立阿史那氏唯可使存其本族而已國分則弱而易制勢敵則難相吞滅各自保全必不能抗衡中國仍請於定襄置都護府為其節度此安邊之長策也【酋慈秋翻長知兩翻易以豉翻下未易易為同】夏州都督竇靜【夏戶雅翻】以為戎狄之性有如禽獸不可以刑法威不可以仁義教况彼首丘之情未易忘也【首式又翻記曰狐死正丘首】置之中國有損無益恐一旦變生犯我王畧莫若因其破亡之餘施以望外之恩假之王侯之號妻以宗室之女【妻七細翻】分其土地析其部落使其權弱勢分易為羈制可使常為藩臣永保邊塞【易以䜴翻】温彦博以為徙於兖豫之間則乖違物性非所以存養之也請準漢建武故事置降匈奴於塞下全其部落順其土俗以實空虚之地使為中國扞蔽策之善者也魏徵以為突厥世為寇盜百姓之讎也【厥九勿翻】今幸而破亡陛下以其降附不忍盡殺宜縱之使還故土不可留之中國夫戎狄人面獸心弱則請服彊則叛亂固其常性【降戶江翻】今降者衆近十萬數年之後蕃息倍多【近其靳翻蕃扶元翻】必為腹心之疾不可悔也晉初諸胡與民雜居中國郭欽江統皆勸武帝驅出塞外以絶亂階【郭欽論見八十一卷晉武帝太康元年江統論見八十三卷惠帝永熙九年】武帝不從後二十餘年伊洛之間遂為氊裘之域此前事之明鑑也彦博曰王者之於萬物天覆地載靡有所遺【覆敷救翻】今突厥窮來歸我柰何棄之而不受乎孔子曰有教無類若救其死亡授以生業教之禮義數年之後悉為吾民選其酋長使入宿衛【酋慈由翻長知兩翻】畏威懷德何後患之有上卒用彦博策處突厥降衆【卒子恤翻處昌呂翻】東自幽州西至靈州分突利故所統之地置順祐化長四州都督府又分頡利之地為六州左置定襄都督府右置雲中都督府以統其衆【定襄都督府僑治寧朔雲中都督府僑治朔方之境按寧朔縣亦屬朔方郡舊書温彦博傳曰帝從彦博議處降人於朔方之地則二都督府僑治朔方明矣】五月辛未以突利為順州都督使帥部落之官【順州僑治營州南之五柳戍帥讀曰率】上戒之曰爾祖啟民挺身奔隋隋立以為大可汗奄有北荒【事見一百七十八卷隋文帝開皇十九年可從刋入聲汗音寒】爾父始畢反為隋患【事見一百八十二卷煬帝大業十一年】天道不容故使爾今日亂亡如此我所以不立爾為可汗者懲啟民前事故也今命爾為都督爾宜善守國法勿相侵掠非徒欲中國久安亦使爾宗族永全也壬申以阿史那蘇尼失為懷德郡王阿史那思摩為懷化郡王頡利之亡也【頡奚結翻】諸部落酋長皆棄頡利來降【酋慈由翻長知兩翻降戶江翻】獨思摩隨之竟與頡利俱擒上嘉其忠拜右武候大將軍尋以為北開州都督使統頡利舊衆 【考異曰舊傳云為化州都督按化州乃突利故地安得云統頡利部落也】丁丑以右武衛大將軍史大柰為豐州都督【隋以五原郡置豐州大業初廢唐初張長遜降復置豐州尋廢是年復以突厥降戶置豐州九原郡】其餘酋長至者皆拜將軍中郎將布列朝廷【將即亮翻朝直遥翻】五品已上百餘人殆與朝士相半因而入居長安者近萬家【近其靳翻】辛巳詔自今訟者有經尚書省判不服聽於東宫上啟委太子裁決【上時掌翻】若仍不伏然後聞奏 丁亥御史大夫蕭瑀劾奏李靖破頡利牙帳御軍無法突厥珍物虜掠俱盡請付法司推科【瑀音禹劾戶槩翻又戶得翻下同頡奚結翻厥九勿翻 考異曰舊傳御史大夫温彦博害其功譖靖軍無綱紀致令虜中奇寶散於亂兵之手據實錄彦博二月已為中書令三月始禽頡利今從實錄】上特敕勿劾及靖入見【見賢遍翻】上大加責讓靖頓首謝久之上乃曰隋史萬歲破達頭可汗有功不賞以罪致戮【可從刋入聲汗音寒事見一百七十九卷隋文帝開皇二十年】朕則不然錄公之功赦公之罪加靖左光祿大夫賜絹千匹加真食邑通前五百戶未幾上謂靖曰前有人讒公今朕意已寤公勿以為懷復賜絹二千匹【幾居豈翻復扶又翻】 林邑獻火珠【唐書婆利東有羅刹國其人極陋朱髮黑身獸牙鷹爪與林邑人作市以夜而來自掩其面其國出火珠狀如水精日午時以珠承日影以艾承之則火出】有司以其表辭不順請討之上曰好戰者亡【好呼到翻】隋煬帝頡利可汗皆耳目所親見也小國勝之不武况未可必乎語言之間何足介意 六月丁酉以阿史那蘇尼失為北寧州都督以中郎將史善應為北撫州都督【尼女夷翻將即亮翻】壬寅以右驍衛將軍康蘇為北安州都督【此三州與祐化長北開四州後皆省史善應亦阿史那種史單書其姓耳驍堅堯翻】 乙卯發卒修洛陽宫以備巡幸給事中張玄素上書諫【上時掌翻】以為洛陽未有巡幸之期而預修宫室非今日之急務昔漢高祖納婁敬之說自洛陽遷長安【事見十一卷漢高帝五年】豈非洛陽之地不及關中之形勝邪【邪音耶】景帝用晁錯之言而七國構禍【事見十六卷漢景帝三年晁直遥翻錯七故翻】陛下今處突厥於中國【處昌呂翻厥九勿翻】突厥之親何如七國豈得不先為憂而宫室可遽興乘輿可輕動哉【乘繩證翻】臣見隋氏初營宫室近山無大木皆致之遠方二千人曳一柱以木為輪則戛摩火出乃鑄鐵為轂行一二里鐵轂輒破别使數百人齎鐵轂隨而易之【轂古祿翻】盡日不過行二三十里計一柱之費已用數十萬功則其餘可知矣陛下初平洛陽凡隋氏宫室之宏侈者皆令毁之【見一百八十九卷高祖武德四年令力丁翻】曾未十年復加營繕何前日惡之而今日效之也【復扶又翻惡烏路翻】且以今日財力何如隋世陛下役瘡痍之人襲亡隋之弊恐又甚於煬帝矣上謂玄素曰卿謂我不如煬帝何如桀紂對曰若此役不息亦同歸於亂耳上歎曰吾思之不熟乃至於是顧謂房玄齡曰朕以洛陽土中朝貢道均意欲便民故使營之今玄素所言誠有理宜即為之罷役【為于偽翻】後日或以事至洛陽雖露居亦無傷也仍賜玄素綵二百匹 秋七月甲子朔日有食之 乙丑上問房玄齡蕭瑀曰隋文帝何如主也對曰文帝勤於為治每臨朝或至日昃五品已上引坐論事衛士傳飱而食【侍衛未得下牙不皇坐食故立駐傳餐而食也治直吏翻下同朝直遥翻飱千安翻】雖性非仁厚亦勵精之主也上曰公得其一未知其二文帝不明而喜察不明則照有不通喜察則多疑於物事皆自決不任羣臣天下至廣一日萬機雖復勞神苦形豈能一一中理羣臣既知主意唯取決受成雖有愆違莫敢諫爭此所以二世而亡也【喜許記翻復扶又翻中竹仲翻爭讀曰諍】朕則不然擇天下賢才寘之百官使思天下之事關由宰相審熟便安然後奏聞有功則賞有罪則刑誰敢不竭心力以修職業何憂天下之不治乎因敕百司自今詔勑行下有未便者皆應執奏毋得阿從不盡己意 癸酉以前太子少保李綱為太子少師以兼御史大夫蕭瑀為太子少傅【唐東宫三少並正二品掌教諭太子少始照翻瑀音禹】李綱有足疾上賜以步輿【步輿即步挽輿也】使之乘至閣下數引入禁中問以政事【數所角翻】每至東宫太子親拜之太子每視事上令綱與房玄齡侍坐【坐徂卧翻】先是蕭瑀與宰相參議朝政【先悉薦翻朝直遥翻】瑀氣剛而辭辯房玄齡等皆不能抗上多不用其言 【考異曰舊傳云玄齡等心知其是不用其言按玄齡若用心如此安得為賢相且事之用捨在太宗非由玄齡今不取】玄齡魏徵温彦博嘗有微過瑀劾奏之【劾戶槩翻又戶得翻】上竟不問瑀由此怏怏自失【瑀音禹怏於兩翻】遂罷御史大夫為太子少傅不復預聞朝政【復扶又翻朝直遥翻】西突厥種落散在伊吾【伊吾即漢伊吾盧之地在大磧外東至陽關二千七百三十里是年置伊吾縣及伊州伊吾郡於其地厥九勿翻種章勇翻】詔以涼州都督李大亮為西北道安撫大使於磧口貯糧【此磧即伊吾東之磧使疏吏翻磧七迹翻貯丁呂翻】來者賑給使者招慰相望於道【賑津忍翻使疏吏翻】大亮上言【上時掌翻】欲懷遠者必先安近中國如本根四夷如枝葉疲中國以奉四夷猶拔本根以益枝葉也臣遠考秦漢近觀隋室外事戎狄皆致疲弊今招致西突厥但見勞費未見其益况河西州縣蕭條【北涼瓜沙肅等州皆河西也】突厥微弱以來始得耕穫今又供億此役民將不堪不若且罷招慰為便伊吾之地率皆沙磧其人或自立君長求稱臣内屬者羈縻受之使居塞外為中國藩蔽此乃施虛惠而收實利也上從之 八月丙午詔以常服未有差等自今三品以上服紫四品五品服緋六品七品服綠八品服青婦人從其夫色【自四品以下緋綠青有深淺之異九品則服淺青】 甲寅詔以兵部尚書李靖為右僕射靖性沈厚【沈持林翻】每與時宰參議恂恂如不能言 突厥既亡營州都督薛萬淑遣契丹酋長貪没折說諭東北諸夷奚霫室韋等十餘部皆内附【說輸芮翻下同霫而立翻】萬淑萬均之兄也 戊午突厥欲谷設來降【厥九勿翻降戶江翻】欲谷設突利之弟也頡利敗欲谷設奔高昌聞突利為唐所禮遂來降 九月戊辰伊吾城主入朝【朝直遥翻】隋末伊吾内屬置伊吾郡隋亂臣於突厥頡利既滅舉其屬七城來降【頡奚結翻降戶江翻】因以其地置西伊州【西伊州六年改曰伊州】 思結部落饑貧朔州刺史新豐張儉招集之其不來者仍居磧北【磧七迹翻】親屬私相往還儉亦不禁及儉徙勝州都督州司奏思結將叛詔儉往察之儉單騎入其部落說諭徙之代州即以儉檢校代州都督思結卒無叛者【騎奇寄翻說式芮翻卒子恤翻】儉因勸之營田歲大稔儉恐虜蓄積多有異志奏請和糴以充邊儲部落喜營田轉力而邊備實焉丙子開南蠻地置費州夷州【二州皆漢牂柯郡之地武德四年以思州寧夷】<br />
<br />
  【縣置夷州貞觀元年廢是年復以思州之都上縣開南蠻置夷州義泉郡隋之明陽郡地也費州涪州郡隋黔安郡之涪川縣地是年分思州之涪川扶陽并開南蠻置宋白曰費州因州界費水為名】 己卯上幸隴州【後魏分涇岐二州之地置東秦州大統十七年改隴州治汧源縣在長安西四百九十六里】冬十一月壬辰以右衛大將軍侯君集為兵部尚書<br />
<br />
  參議朝政 甲子車駕還京師 上讀明堂鍼灸書云人五藏之系咸附於背【唐藝文志有黄帝明堂經明堂偃側人圖明堂人形圖明堂孔穴圖皆鍼灸之書也藏徂浪翻鍼諸深翻灸居又翻】戊寅詔自今毋得笞囚背十二月甲辰上獵於鹿苑【武德元年分京兆之高陵置鹿苑縣】乙巳還宫甲寅高昌王麴文泰入朝西域諸國咸欲因文泰遣<br />
<br />
  使入貢【朝直遥翻使疏吏翻】上遣文泰之臣厭怛紇干往迎之【厭於葉翻怛當割翻紇下没翻】魏徵諫曰昔光武不聽西域送侍子置都護以為不以蠻夷勞中國【事見四十三卷漢光武建武二十三年】今天下初定前者文泰之來勞費已甚【此即謂文泰入唐境之時】今借使十國入貢其徒旅不減千人邊民荒耗將不勝其弊若聽其商賈往來與邊民交市則可矣【勝音升賈音古】儻以賓客遇之非中國之利也時厭怛紇干已行上遽令止之 諸宰相侍宴上謂王珪曰卿識鑒精通復善談論【復扶又翻】玄齡以下卿宜悉加品藻且自謂與數子何如對曰孜孜奉國知無不為臣不如玄齡才兼文武出將入相臣不如李靖敷奏詳明出納惟允臣不如温彦博處繁治劇衆務畢舉臣不如戴胄【處昌呂翻治直之翻】恥君不及堯舜以諫爭為己任臣不如魏徵至於激濁揚清嫉惡好善【好呼到翻】臣於數子亦有微長上深以為然衆亦服其確論【確克角翻】上之初即位也嘗與羣臣語及教化上曰今承大亂<br />
<br />
  之後恐斯民未易化也魏徵對曰不然久安之民驕佚驕佚則難教經亂之民愁苦愁苦則易化譬猶飢者易為食渴者易為飲也【孟子之言易以豉翻】上深然之封德彝非之曰三代以還人漸澆訛故秦任法律漢雜霸道蓋欲化而不能豈能之而不欲邪魏徵書生未識時務若信其虚論必敗國家【敗補邁翻】徵曰五帝三王不易民而化昔黄帝征蚩尤顓頊誅九黎湯放桀武王伐紂皆能身致太平【神農氏世衰蚩尤為暴虐黄帝征之禽殺蚩尤少皥氏衰九黎亂德顓頊誅之成湯放桀於南巢武王殺紂於牧野】豈非承大亂之後邪若謂古人淳朴漸至澆訛則至於今日當悉化為鬼魅矣【邪音耶澆堅堯翻魅音媚】人主安得而治之上卒從徵言【治直之翻卒子恤翻】元年關中饑米斗直絹一匹二年天下蝗三年大水上勤而撫之民雖東西就食未嘗嗟怨是歲天下大稔流散者咸歸鄉里米斗不過三四錢終歲斷死刑纔三十九人東至於海南極五嶺皆外戶不閉【斷丁亂翻孔頴達曰外戶而不閉者扉從外闔也不閉者不用關閉之也重門擊柝本禦暴客既無盜竊亂賊則戶無俟於閉也但為風塵入寢故設扉耳無所捍拒故從外而掩也】行旅不齎糧取給於道路焉上謂長孫無忌曰貞觀之初上書者皆云人主當獨運威權不可委之臣下又云宜震耀威武征討四夷唯魏徵勸朕偃武修文中國既安四夷自服朕用其言今頡利成擒其酋長並帶刀宿衛部落皆襲衣冠徵之力也但恨不使封德彝見之耳【封德彝薨於元年】徵再拜謝曰突厥破滅海内康寧皆陛下威德臣何力焉上曰朕能任公公能稱所任【稱尺證翻】則其功豈獨在朕乎 房玄齡奏閱府庫甲兵遠勝隋世上曰甲兵武備誠不可闕然煬帝甲兵豈不足邪卒亡天下【卒子恤翻】若公等盡力使百姓乂安此乃朕之甲兵也 上謂秘書監蕭璟曰卿在隋世數見皇后乎【隋煬帝蕭后璟同產也故帝問及之數所角翻】對曰彼兒女且不得見臣何人得見之魏徵曰臣聞煬帝不信齊王恒有中使察之【煬帝猜防齊王暕事畧見隋紀恒戶登翻使疏吏翻】聞其宴飲則曰彼營何事得遂而喜聞其憂悴【悴秦醉翻】則曰彼有他念故爾父子之間且猶如是况他人乎上笑曰朕今視楊政道勝煬帝之於齊王遠矣璟瑀之兄也【瑀音禹】 西突厥肆葉護可汗既先可汗之子為衆所附莫賀咄可汗所部酋長多歸之【厥九勿翻可從刋入聲汗音寒咄當没翻酋慈由翻長知兩翻】肆葉護引兵擊莫賀咄莫賀咄兵敗逃於金山為泥熟設所殺諸部共推肆葉護為大可汗【肆葉護與莫賀咄相攻事始上二年】<br />
<br />
  五年春正月詔僧尼道士致拜父母【尼女夷翻】 癸酉上大獵於昆明池四夷君長咸從【長知兩翻從才用翻】甲戌宴高昌王文泰及羣臣丙子還宫親獻禽于大安宫 癸未朝集使趙郡王孝恭等上表以四夷咸服請封禪【朝直遥翻上時掌翻】上手詔不許【此元正朝集既畢將歸者唐制凡天下朝集使皆以十月二十五日至京師十一月一日戶部引見訖於尚書省與羣官禮見然後集於考堂應考績之事元日陳其貢篚於殿庭朝直遥翻使疏吏翻】 有司上言皇太子當冠用二月吉請造兵備儀仗【上時掌翻冠古玩翻唐皇太子冠禮詳見新書禮樂志】上曰東作方興宜改用十月少傅蕭瑀奏據陰陽不若二月【少始照翻】上曰吉凶在人若動依陰陽不顧禮義吉可得乎循正而行自與吉會農時最急不可失也 二月甲辰詔諸州有京觀處【觀古玩翻】無問新舊宜悉剗削加土為墳掩蔽枯朽勿令暴露己酉封皇弟元裕為鄶王【鄶古外翻】元名為譙王靈蘷為<br />
<br />
  魏王元祥為許王元曉為密王庚戌封皇子愔為梁王惲為郯王【愔於今翻惲於粉翻郯音談】貞為漢王治為晉王慎為申王囂為江王簡為代王 夏四月壬辰代王簡薨 壬寅靈州斛薛叛【斛薛部内附處之靈州今叛】任城王道宗追擊破之【任音壬】 隋末中國人多没於突厥【厥九勿翻】及突厥降上遣使以金帛贖之【降戶江翻使疏吏翻下同】五月乙丑有司奏凡得男女八萬口 六月甲寅太子少師新昌貞公李綱薨初周齊王憲女孀居無子綱贍恤甚厚綱薨其女以父禮喪之【李綱先為齊王憲參軍事見一百七十三卷陳宣帝大建十年】 秋八月甲辰遣使詣高麗【麗力知翻】收隋氏戰亡骸骨葬而祭之 河内人李好德得心疾妄為妖言【好呼到翻妖於驕翻】詔按其事大理丞張藴古奏好德被疾有徵【徵明也證也驗也被皮義翻下同】法不當坐治書侍御史權萬紀劾奏藴古貫在相州【貫鄉籍也治直之翻劾戶槩翻又戶得翻相息亮翻】好德之兄厚德為其刺史情在阿縱按事不實上怒命斬之於市既而悔之因詔自今有死罪雖令即決仍三覆奏乃行刑權萬紀與侍御史李仁發俱以告訐有寵於上【訐居謁翻】由是諸大臣數被譴怒【數所角翻】魏徵諫曰萬紀等小人不識大體以訐為直以讒為忠陛下非不知其無堪蓋取其無所避忌欲以警策羣臣耳而萬紀等挾恩依勢逞其姦謀凡所彈射【射而亦翻】皆非有罪陛下縱未能舉善以厲俗柰何昵姦以自損乎【昵尼質翻】上默然賜絹五百匹久之萬紀等姦狀自露皆得罪【為帝疏權萬紀張本】 九月上修仁夀宫更命曰九成宫又將修洛陽宫民部尚書戴胄表諫以亂離甫定百姓彫弊帑藏空虚若營造不已公私勞費殆不能堪【更工衡翻藏徂浪翻】上嘉之曰戴胄於我非親但以忠直體國知無不言故以官爵酬之耳久之竟命將作大匠竇璡修洛陽宫璡鑿池築山彫飾華靡上遽命毁之免璡官【璡將鄰翻又則刃翻】 冬十月丙午上逐兎於後苑【唐長安苑城袤遠包漢長安故城在其中程大昌曰唐太極宫之北有内苑有禁苑太極宫居都城之北内苑又居宫北禁苑又居内苑之北禁苑廣矣西面全包漢之都城東抵霸水其西南兩面攙出太極宫前與承天門相齊承天門之西排立三門皆禁苑之門也曰光化曰芳林曰景耀六典曰禁苑在大内宫城之北北臨渭水東距滻川西盡都城其周一百二十里】左領軍將軍執失思力諫曰天命陛下為華夷父母柰何自輕上又將逐鹿思力脫巾解帶跪而固諫上為之止【為于偽翻】初上令羣臣議封建魏徵議以為若封建諸侯則卿大夫咸資俸祿必致厚斂【斂力贍翻】又京畿賦税不多所資畿外若盡以封國邑經費頓闕又燕秦趙代俱帶外夷【燕因肩翻】若有警急追兵内地難以奔赴禮部侍郎李百藥以為運祚脩短定命自天堯舜大聖守之而不能固漢魏微賤拒之而不能却今使勲戚子孫皆有民有社易世之後將驕淫自恣攻戰相殘害民尤深不若守令之迭居也【守式又翻】中書侍郎顔帥古以為不若分王諸子勿令過大間以州縣【王于况翻間古莧翻】雜錯而居互相維持使各守其境協力同心足扶京室為置官寮皆省司選用【為于偽翻省司謂尚書省主者】法令之外不得擅作威刑朝貢禮儀具為條式一定此制萬世無虞【朝直遥翻】十一月詔皇家宗室及勲賢之臣宜令作鎮藩部貽厥子孫非有大故毋或黜免所司明為條例定等級以聞 丁巳林邑獻五色鸚鵡【鸚鵡能言鳥也萬震南州志曰鸚鵡有三種一種白一種青一種五色交州以南諸國盡有之白及五色者性尤慧解陸佃埤雅鸚鵡人舌能言青羽赤喙蓋青者又凡種也舊說衆鳥足趾前三後一其目下瞼眨上惟鸚鵡四趾齊分兩瞼俱動如人目瞼九儉翻眼瞼也貶則洽翻目動也】丁卯新羅獻美女二人魏徵以為不宜受上喜曰林邑鸚鵡猶能自言苦寒思歸其國况二女遠别親戚乎并鸚鵡各付使者而歸之【使疏吏翻】 倭國遣使入貢【倭烏禾翻】上遣新州刺史高表仁持節往撫之表仁與其王爭禮不宣命而還【還從宣翻又如字】丙子上祀圓丘 十二月太僕寺丞李世南開党項<br />
<br />
  之地十六州四十七縣【党底朗翻】 上謂侍臣曰朕以死刑至重故令三覆奏蓋欲思之詳熟故也而有司須臾之間三覆已訖又古刑人君為之徹樂減膳朕庭無常設之樂然常為之不啖酒肉但未有著令又百司斷獄唯據律文雖情在可矜而不敢違法其間豈能盡無寃乎丁亥制決死囚者二日中五覆奏下諸州者三覆奏行刑之日尚食勿進酒肉【爲于偽翻斷丁亂翻唐尚食局屬殿中監有奉御直長掌御膳】内教坊及太常不舉樂【武德中置内教坊於禁中有内教博士太常寺有太樂署鼓吹署】皆令門下覆視有據法當死而情可矜者錄狀以聞由是全活甚衆其五覆奏者以決前一二日至決日又三覆奏唯犯惡逆者一覆奏而已【隋立十惡之科四曰惡逆謂毆及謀殺祖父母父母殺伯叔父母姑兄子外祖父母夫夫之祖父母父母者唐遵用之】 己亥朝集使利州都督武士彠等復上表請封禪【朝直遥翻彠一虢翻復扶又翻】不許 壬寅上幸驪山温湯戊申還宫 上謂執政曰朕常恐因喜怒妄行賞罰故欲公等極諫公等亦宜受人諫不可以己之所欲惡人違之【惡烏路翻】苟自不能受諫安能諫人 康國求内附【康國即漢康居國一曰薩末鞬亦曰颯末鞬元魏謂之悉萬斤其王姓温本月氐人始居祁連北昭武城為突厥所破稍南依葱嶺即有其地以昭武為姓示不忘本也】上曰前代帝王好招來絶域以求服遠之名無益於用而糜弊百姓今康國内附儻有急難於義不得不救師行萬里豈不疲勞勞百姓以取虚名朕不為也遂不受謂侍臣曰治國如治病病雖愈猶宜將護儻遽自放縱病復作則不可救矣今中國幸安四夷俱服誠自古所希然朕日慎一日唯懼不終故欲數聞卿輩諫爭也【好呼到翻難乃旦翻治直之翻復扶又翻數所角翻爭讀曰諍】魏徵曰内外治安臣不以為喜唯喜陛下居安思危耳【治直吏翻】 上嘗與侍臣論獄魏徵曰煬帝時嘗有盜發帝令於士澄捕之【於如字姓也出何承天姓苑】少涉疑似皆栲訊取服【少始沼翻栲音考】凡二千餘人帝悉令斬之【令力丁翻】大理丞張元濟怪其多試尋其狀内五人嘗為盜餘皆平民竟不敢執奏盡殺之上曰此豈唯煬帝無道其臣亦不盡忠君臣如此何得不亡公等宜戒之 是歲高州總管馮盎入朝未幾羅竇諸洞獠反【竇州漢端溪縣地隋為瀧州懷德縣武德四年置南扶州貞觀六年更名竇州取州界有羅竇洞為名朝直遥翻幾居豈翻獠魯皓翻】敕盎帥部落二萬為諸軍前鋒【帥讀曰率】獠數萬人屯據險要諸軍不得進盎持弩謂左右口盡吾此矢足知勝負矣連發七矢中七人【中竹仲翻】獠皆走因縱兵乘之斬首千餘級上美其功前後賞賜不可勝數【勝音升】盎所居地方二千里奴婢萬餘人珍貨充積然為治勤明所部愛之【治直吏翻】 新羅王真平卒【卒子恤翻】無嗣國人立其女善德為王<br />
<br />
  資治通鑑卷一百九十三<br />
<br />
<史部,編年類,資治通鑑>  <br>
   </div> 

<script src="/search/ajaxskft.js"> </script>
 <div class="clear"></div>
<br>
<br>
 <!-- a.d-->

 <!--
<div class="info_share">
</div> 
-->
 <!--info_share--></div>   <!-- end info_content-->
  </div> <!-- end l-->

<div class="r">   <!--r-->



<div class="sidebar"  style="margin-bottom:2px;">

 
<div class="sidebar_title">工具类大全</div>
<div class="sidebar_info">
<strong><a href="http://www.guoxuedashi.com/lsditu/" target="_blank">历史地图</a></strong>  
<a href="http://www.880114.com/" target="_blank">英语宝典</a>  
<a href="http://www.guoxuedashi.com/13jing/" target="_blank">十三经检索</a> 
<br><strong><a href="http://www.guoxuedashi.com/gjtsjc/" target="_blank">古今图书集成</a></strong> 
<a href="http://www.guoxuedashi.com/duilian/" target="_blank">对联大全</a> <strong><a href="http://www.guoxuedashi.com/xiangxingzi/" target="_blank">象形文字典</a></strong> 

<br><a href="http://www.guoxuedashi.com/zixing/yanbian/">字形演变</a>  <strong><a href="http://www.guoxuemi.com/hafo/" target="_blank">哈佛燕京中文善本特藏</a></strong>
<br><strong><a href="http://www.guoxuedashi.com/csfz/" target="_blank">丛书&方志检索器</a></strong> <a href="http://www.guoxuedashi.com/yqjyy/" target="_blank">一切经音义</a>  

<br><strong><a href="http://www.guoxuedashi.com/jiapu/" target="_blank">家谱族谱查询</a></strong>  <strong><a href="http://shufa.guoxuedashi.com/sfzitie/" target="_blank">书法字帖欣赏</a></strong> 
<br>

</div>
</div>


<div class="sidebar" style="margin-bottom:0px;">

<font style="font-size:22px;line-height:32px">QQ交流群9:489193090</font>


<div class="sidebar_title">手机APP 扫描或点击</div>
<div class="sidebar_info">
<table>
<tr>
	<td width=160><a href="http://m.guoxuedashi.com/app/" target="_blank"><img src="/img/gxds-sj.png" width="140"  border="0" alt="国学大师手机版"></a></td>
	<td>
<a href="http://www.guoxuedashi.com/download/" target="_blank">app软件下载专区</a><br>
<a href="http://www.guoxuedashi.com/download/gxds.php" target="_blank">《国学大师》下载</a><br>
<a href="http://www.guoxuedashi.com/download/kxzd.php" target="_blank">《汉字宝典》下载</a><br>
<a href="http://www.guoxuedashi.com/download/scqbd.php" target="_blank">《诗词曲宝典》下载</a><br>
<a href="http://www.guoxuedashi.com/SiKuQuanShu/skqs.php" target="_blank">《四库全书》下载</a><br>
</td>
</tr>
</table>

</div>
</div>


<div class="sidebar2">
<center>


</center>
</div>

<div class="sidebar"  style="margin-bottom:2px;">
<div class="sidebar_title">网站使用教程</div>
<div class="sidebar_info">
<a href="http://www.guoxuedashi.com/help/gjsearch.php" target="_blank">如何在国学大师网下载古籍?</a><br>
<a href="http://www.guoxuedashi.com/zidian/bujian/bjjc.php" target="_blank">如何使用部件查字法快速查字?</a><br>
<a href="http://www.guoxuedashi.com/search/sjc.php" target="_blank">如何在指定的书籍中全文检索?</a><br>
<a href="http://www.guoxuedashi.com/search/skjc.php" target="_blank">如何找到一句话在《四库全书》哪一页?</a><br>
</div>
</div>


<div class="sidebar">
<div class="sidebar_title">热门书籍</div>
<div class="sidebar_info">
<a href="/so.php?sokey=%E8%B5%84%E6%B2%BB%E9%80%9A%E9%89%B4&kt=1">资治通鉴</a> <a href="/24shi/"><strong>二十四史</strong></a>&nbsp; <a href="/a2694/">野史</a>&nbsp; <a href="/SiKuQuanShu/"><strong>四库全书</strong></a>&nbsp;<a href="http://www.guoxuedashi.com/SiKuQuanShu/fanti/">繁体</a>
<br><a href="/so.php?sokey=%E7%BA%A2%E6%A5%BC%E6%A2%A6&kt=1">红楼梦</a> <a href="/a/1858x/">三国演义</a> <a href="/a/1038k/">水浒传</a> <a href="/a/1046t/">西游记</a> <a href="/a/1914o/">封神演义</a>
<br>
<a href="http://www.guoxuedashi.com/so.php?sokeygx=%E4%B8%87%E6%9C%89%E6%96%87%E5%BA%93&submit=&kt=1">万有文库</a> <a href="/a/780t/">古文观止</a> <a href="/a/1024l/">文心雕龙</a> <a href="/a/1704n/">全唐诗</a> <a href="/a/1705h/">全宋词</a>
<br><a href="http://www.guoxuedashi.com/so.php?sokeygx=%E7%99%BE%E8%A1%B2%E6%9C%AC%E4%BA%8C%E5%8D%81%E5%9B%9B%E5%8F%B2&submit=&kt=1"><strong>百衲本二十四史</strong></a>  <a href="http://www.guoxuedashi.com/so.php?sokeygx=%E5%8F%A4%E4%BB%8A%E5%9B%BE%E4%B9%A6%E9%9B%86%E6%88%90&submit=&kt=1"><strong>古今图书集成</strong></a>
<br>

<a href="http://www.guoxuedashi.com/so.php?sokeygx=%E4%B8%9B%E4%B9%A6%E9%9B%86%E6%88%90&submit=&kt=1">丛书集成</a> 
<a href="http://www.guoxuedashi.com/so.php?sokeygx=%E5%9B%9B%E9%83%A8%E4%B8%9B%E5%88%8A&submit=&kt=1"><strong>四部丛刊</strong></a>  
<a href="http://www.guoxuedashi.com/so.php?sokeygx=%E8%AF%B4%E6%96%87%E8%A7%A3%E5%AD%97&submit=&kt=1">說文解字</a> <a href="http://www.guoxuedashi.com/so.php?sokeygx=%E5%85%A8%E4%B8%8A%E5%8F%A4&submit=&kt=1">三国六朝文</a>
<br><a href="http://www.guoxuedashi.com/so.php?sokeytm=%E6%97%A5%E6%9C%AC%E5%86%85%E9%98%81%E6%96%87%E5%BA%93&submit=&kt=1"><strong>日本内阁文库</strong></a> <a href="http://www.guoxuedashi.com/so.php?sokeytm=%E5%9B%BD%E5%9B%BE%E6%96%B9%E5%BF%97%E5%90%88%E9%9B%86&ka=100&submit=">国图方志合集</a> <a href="http://www.guoxuedashi.com/so.php?sokeytm=%E5%90%84%E5%9C%B0%E6%96%B9%E5%BF%97&submit=&kt=1"><strong>各地方志</strong></a>

</div>
</div>


<div class="sidebar2">
<center>

</center>
</div>
<div class="sidebar greenbar">
<div class="sidebar_title green">四库全书</div>
<div class="sidebar_info">

《四库全书》是中国古代最大的丛书,编撰于乾隆年间,由纪昀等360多位高官、学者编撰,3800多人抄写,费时十三年编成。丛书分经、史、子、集四部,故名四库。共有3500多种书,7.9万卷,3.6万册,约8亿字,基本上囊括了古代所有图书,故称“全书”。<a href="http://www.guoxuedashi.com/SiKuQuanShu/">详细>>
</a>

</div> 
</div>

</div>  <!--end r-->

</div>
<!-- 内容区END --> 

<!-- 页脚开始 -->
<div class="shh">

</div>

<div class="w1180" style="margin-top:8px;">
<center><script src="http://www.guoxuedashi.com/img/plus.php?id=3"></script></center>
</div>
<div class="w1180 foot">
<a href="/b/thanks.php">特别致谢</a> | <a href="javascript:window.external.AddFavorite(document.location.href,document.title);">收藏本站</a> | <a href="#">欢迎投稿</a> | <a href="http://www.guoxuedashi.com/forum/">意见建议</a> | <a href="http://www.guoxuemi.com/">国学迷</a> | <a href="http://www.shuowen.net/">说文网</a><script language="javascript" type="text/javascript" src="https://js.users.51.la/17753172.js"></script><br />
  Copyright &copy; 国学大师 古典图书集成 All Rights Reserved.<br>
  
  <span style="font-size:14px">免责声明:本站非营利性站点,以方便网友为主,仅供学习研究。<br>内容由热心网友提供和网上收集,不保留版权。若侵犯了您的权益,来信即刪。scp168@qq.com</span>
  <br />
ICP证:<a href="http://www.beian.miit.gov.cn/" target="_blank">鲁ICP备19060063号</a></div>
<!-- 页脚END --> 
<script src="http://www.guoxuedashi.com/img/plus.php?id=22"></script>
<script src="http://www.guoxuedashi.com/img/tongji.js"></script>

</body>
</html>
