










 


 
 


 

  
  
  
  
  





  
  
  
  
  
 
  

  

  
  
  



  

 
 

  
   




  

  
  


    資治通鑑卷二百五十七 宋 司馬光 撰

  胡三省 音註

  唐紀七十三【起彊圉恊洽四月盡著雍涒灘凡一年有奇】

  僖宗惠聖恭定孝皇帝下之下

  光啓三年夏四月甲辰朔約逐蘇州刺史張雄 【考異曰吳越備史四月六合鎮將徐約攻陷蘇州約曹州人也初從黄巢攻天長遂歸高駢駢用為六合鎮將浙西周寶子壻楊茂實為蘇州刺史約攻破之遂有其地据實録寶以其壻為蘇州刺史朝廷已除趙載代之張雄據蘇州必在載後備史恐誤今從新紀傳】帥其衆逃入海【此句上更有一雄字文意乃足張雄據蘇州見上卷上年帥讀曰率】 高駢聞秦宗權將寇淮南遣左廂都知兵馬使畢師鐸將百騎屯高郵時呂用之用事宿將多為所誅師鐸自以黄巢降將常自危【畢師鐸降高駢見二百五十三卷乾符六年鐸將以下皆即亮翻】師鐸有美妾用之欲見之師鐸不許用之因師鐸出竊往見之師鐸慙怒出其妾由是有隙師鐸將如高郵用之待之加厚師鐸益疑懼謂禍在旦夕師鐸子娶高郵鎮遏使張神劒女師鐸密與之謀神劒以為無是事神劒名雄人以其善用劒故謂之神劒【考異曰十國紀年張雄淮南人善劒號張神劒今欲别於前蘇州刺史張雄故從妖亂志但稱神劒】時府中籍籍亦以為師鐸且受誅【漢書事籍籍如此顔師古注云籍籍紛紛也】其母使人語之曰設有是事汝自努力前去勿以老母弱子為累【語牛倨翻累良瑞翻】師鐸疑未決會駢子四十三郎者素惡用之【惡鳥路翻】欲使師鐸帥外鎮將吏疏用之罪惡聞於其父【帥讀曰率】密使人紿之曰用之比來頻啓令公【比毗至翻近也襄王煴加駢中書令故稱令公紿徒亥翻】欲因此相圖已有委曲在張尚書所【當時機密文書謂之委曲張尚書謂神劒】宜備之師鐸問神劒曰昨夜使司有文書【使司謂淮南節度使司】翁胡不言【以婚姻呼之為翁】神劒不寤曰無之師鐸不自安歸營謀於腹心皆勸師鐸起兵誅用之師鐸曰用之數年以來人怨鬼怒安知天不假手於我誅之邪淮寜軍使鄭漢章我鄉人【按新書高駢傳駢置淮寜軍於淮口畢師鐸鄭漢章皆寃句人】昔歸順時副將也【謂去黄巢歸高駢時也】素切齒於用之聞吾謀必喜乃夜與百騎潛詣漢章漢章大喜悉發鎮兵及驅居民合千餘人從師鐸至高郵師鐸詰張神劒以所得委曲【詰極吉翻】神劒驚曰無有師鐸聲色寖厲神劒奮曰公何見事之暗用之姦惡天地所不容况近者重賂權貴得嶺南節度復不行【事見上卷上年復扶又翻】或云謀竊據此土使其得志吾輩豈能握刀頭事此妖物邪要冎此數賊以謝淮海何必多言【冎古瓦翻禹貢曰淮海惟楊州】漢章喜遂命取酒割臂血瀝酒共飲之乙巳衆推師鐸為行營使為文告天地移書淮南境内言誅用之及張守一諸葛殷之意以漢章為行營副使神劒為都指揮使神劒以師鐸成敗未可知請以所部留高郵曰一則為公聲援二則供給糧餉師鐸不悅漢章曰張尚書謀亦善苟終始同心事捷之日子女玉帛相與共之今日豈可復相違【復扶又翻】師鐸乃許之戊申師鐸漢章發高郵庚戌詗騎以白高駢【自高郵東南至楊州一百里詗翾正翻又火迥翻】呂用之匿之朱珍至淄青旬日應募者萬餘人又襲青州獲馬千

  匹【時王敬武鎮淄青朱珍以他鎮之將來募兵既不能制又為所襲蓋羣盜縱横力強者勝莫適為主故也】辛亥還至大梁朱全忠喜曰吾事濟矣時蔡人方寇汴州其將張晊屯北郊秦賢屯板橋【北郊謂汴州城北郊原之地即赤岡也據舊史板橋在汴州城西】各有衆數萬列三十六寨連延二十餘里全忠謂諸將曰彼蓄銳休兵方來擊我未知朱珍之至謂吾兵少畏怯自守而已宜出其不意先擊之乃自引兵攻秦賢寨士卒踊躍爭先賢不為備連拔四寨斬萬餘級蔡人大驚以為神全忠又使牙將新野郭言募兵於河陽陜虢得萬餘人而還【陜失冉翻還從宣翻又如字】 畢師鐸兵奄至廣陵城下城中驚擾壬子呂用之引麾下勁兵誘以重賞出城力戰【誘音酉】師鐸兵少却用之始得斷橋塞門為守備是日駢登延和閣【斷丁管翻塞悉則翻延和閣駢所起見二百五十四卷中和二年】閣諠譟聲左右以師鐸之變告駢驚急召用之詰之用之徐對曰師鐸之衆思歸為門衛所遏適已隨宜區處【昌處呂翻】計尋退散儻或不已止煩玄女一力士耳願令公勿憂駢曰近者覺君之妄多矣君善為之勿使吾為周侍中【周侍中謂周寶也事見上卷本年】言畢慘沮久之用之慙懅而退【懅亦慙也音遽】師鐸退屯山光寺【山光寺在廣陵城北】以廣陵城堅兵多甚有悔色癸丑遣其屬孫約與其子詣宣州乞師於觀察使秦彦且許以克城之日迎彦為帥【帥所類翻】會師鐸館客畢慕顔自城中逃出言衆心離散用之憂窘若堅守之不日當潰師鐸乃悅是日未明駢召用之問以事本末用之始以實對駢曰吾不欲復出兵相攻【復扶又翻】君可選一温信大將【温柔和也信誠實不妄言者也】以我手札諭之若其未從當别處分【處昌呂翻按書及春秋分器記曲禮分母求多漢書分職分部並音扶問翻則處分之分亦當同音今人讀為分判之分誤也】用之退念諸將皆仇敵必不利於已甲寅遣所部討擊副使許戡齎駢委曲【委曲即駢手扎也】及用之誓狀并酒殽出勞師鐸【勞力到翻】師鐸始亦望駢舊將勞問得以具陳用之姦惡披泄積憤【披開也分也決壅為泄】見戡至大罵曰梁纘韓問何在乃使此穢物來戡未及發言已牽出斬之乙卯師鐸射書入城【射而亦翻】用之不發即焚之丁巳用之以甲士百人入見駢於延和閣下駢大驚匿于寢室久而後出曰節度使所居無故以兵入欲反邪命左右驅出用之大懼出子城南門舉策指之曰吾不可復入此【復扶又翻】自是高呂始判矣是夜駢召其從子前左金吾衛將軍傑密議軍事戊午署傑都牢城使泣而勉之以親信五百人給之用之命諸將大索城中丁壯【索山客翻】無問朝士書生悉以白刃驅縳登城令分立城上自旦至暮不得休息又恐其與外寇通數易其地【數所角翻】家人餉之莫知所在由是城中人亦恨師鐸入城之晚也駢遣大將石鍔【鍔逆各翻】以師鐸幼子及其母書并駢委曲至揚子諭師鐸師鐸遽遣其子還曰令公但斬呂張以示師鐸師鐸不敢負恩願以妻子為質【質音致】駢恐用之屠其家收師鐸母妻子置使院【使院節度使司官屬治事之所】辛酉秦彦遣其將秦稠將兵三千至揚子助師鐸壬戍宣州軍攻南門不克癸亥又攻羅城東南隅城幾陷者數四【幾居依翻】甲子羅城西南隅守者焚戰格以應師鐸【戰格列木為之漢人謂之笓格今謂之排杈】師鐸毁其城以内其衆用之帥其衆千人力戰於三橋北【帥讀曰率】師鐸垂敗會高傑以牢城兵自子城出欲擒用之以授師鐸用之乃開參佐門北走駢召梁纘以昭義軍百餘人保子城乙丑師鐸縱兵大掠駢不得已命徹備與師鐸相見於延和閣下交拜如賓主之儀署師鐸節度副使行軍司馬仍承制加左僕射鄭漢章等各遷官有差左莫邪都虞候申及本徐州健將【高駢置左右莫邪都見二百五十四卷中和二年】入見駢說之曰【說式芮翻】師鐸逆黨不多請令公及此選元從三十人【及此言及此時也從才用翻】夜自教場門出比師鐸覺之追不及矣【比必利翻及也】然後發諸鎮兵還取府城此轉禍為福也若一二日事定寖恐艱難及亦不得在左右矣言之且泣駢猶豫不聽【楚靈王有言大福不再祇取辱耳高駢蓋知行留皆禍故猶豫不聽】及恐語泄遂竄匿會張雄至東塘【張雄弃蘇州逃入海又自海泝江而上至楊州東塘】及往歸之丙寅師鐸果分兵守諸門搜捕用之親黨悉誅之師鐸入居使院秦稠以宣軍千人分守使宅及諸倉庫【使疏吏翻】丁卯駢牒請解所任以師鐸兼判府事師鐸遣孫約至宣城趣秦彦過江【趣讀曰促】或說師鐸曰【說式芮翻】僕射曏者舉兵蓋以用之輩姦邪暴横【横戶孟翻】高令公坐自聾瞽不能區理【區分别也理調治也】故順衆心為一方去害【去羌呂翻】今用之既敗軍府廓然僕射宜復奉高公而佐之但總其兵權以號令誰敢不服用之乃淮南一叛將耳移書所在立可梟擒如此外有推奉之名内得兼并之實雖朝廷聞之亦無虧臣節使高公聰明必知内愧如其不悛【悛丑緣翻改也】乃机上肉耳奈何以此功業付之它人豈惟受制於人終恐自相魚肉前日秦稠先守倉庫其相疑已可見且秦司空為節度使廬州夀州其肯為之下乎【廬州楊行密夀州張翺】僕見戰攻之端未有窮已豈惟淮南之人肝腦塗地竊恐僕射功名成敗未可知也不若及今亟止秦司空【亟紀力翻急也】勿使過江彼若粗識安危必不敢輕進【粗坐五翻】就使它日責我以負約猶不失為高氏忠臣也師鐸大以為不然明日以告鄭漢章漢章曰此智士也散求之其人畏禍竟不復出【復扶又翻】戊辰駢遷家出居南第師鐸以甲士百人為衛其實囚之也是日宣軍以所求未獲焚進奉兩樓數十間寶貨悉為煨燼【新書高駢傳駢自乾符以來貢獻不入天子寶貨山積於進奉樓按駢乾符末始自浙西徙淮南中和二年罷兵權利權貢獻始絶矣煨烏回翻燼徐刃翻】己巳師鐸於府廳視事凡官吏非有兵權者皆如故復遷駢於東第【復扶又翻下同】自城陷諸軍大掠不已至是師鐸始以先鋒使唐宏為静街使禁止之駢先為鹽鐵使【乾符六年駢為鹽鐵轉運使中和二年解使職】積年不貢奉貨財在揚州者填委如山駢作郊天御樓六軍立仗儀服【郊天及御樓肆赦六軍皆立仗】及大殿元會内署行幸供張器用皆刻鏤金玉蟠龍蹙鳳數十萬事悉為亂兵所掠歸於閭閻張陳寢處其中【供居用翻張知亮翻張陳同鏤郎豆翻處昌呂翻】庚午獲諸葛殷杖殺之弃尸道旁怨家抉其目斷其舌【抉於決翻斷都管翻下斷手同】衆以瓦石投之須臾成冢呂用之之敗也其黨鄭首歸師鐸師鐸署杞知海陵監事【海陵監筦榷鬻鹽】至海陵陰記高霸得失聞於師鐸【高霸時為海陵鎮遏使】霸獲其書杖杞背斷手足刳目截舌然後斬之 蔡將盧塘屯於萬勝【萬勝鎮在中牟縣】夾汴水而軍以絶汴州運路【薛史梁紀曰盧塘於圃田北夾汴水為梁以扼運路宋白亦曰萬勝寨在圃田北】朱全忠乘霧襲之掩殺殆盡 【考異曰薛居正五代史云四月庚午按長歷四月甲辰朔無庚午薛史誤】於是蔡兵皆徙就張晊屯於赤岡【赤岡在汴城北】全忠復就撃之殺二萬餘人蔡人大懼或軍中自相驚全忠乃還大梁養兵休士 辛未高駢密以金遺守者【駢冀守者恩之因以求出遺惟季翻】畢師鐸聞之壬午復迎駢入道院【道院高駢所起以迎神仙】收高氏子弟甥姪十餘人同幽之 前蘇州刺史張雄帥其衆自海泝江屯於東塘遣其將趙暉入據上元【張雄馮弘鐸由此得據昇州帥讀日率】 畢師鐸之攻廣陵也呂用之詐為高駢牒署廬州刺史楊行密行軍司馬追兵入援廬江人袁襲說行密曰高公昏惑用之姦邪師鐸悖逆凶德參會【三者合集為參會說式芮翻】而求兵於我此天以淮南授明公也趣赴之【趣讀曰促】行密乃悉發廬州兵復借兵於和州刺史孫端【復扶又翻 考異曰妖亂志中和三年高駢差梁纘知和州纘以孫端窺伺和州已久不如因而與之以責其効駢強之既行果為端所敗及歸和州尋陷於端蓋端自是遂據和州也】合數千人赴之五月至天長鄭漢章之從師鐸也留其妻守淮口用之帥衆攻之【帥讀曰率】旬日不克漢章引兵救之用之聞行密至天長引兵歸之【為用之為行密所誅張本】 丙子朱全忠出撃張晊大破之秦宗權聞之自鄭州引精兵會之【宗權引兵會晊以撃全忠】 張神劒求貨於畢師鐸師鐸報以俟秦司空之命神劒怒亦以其衆歸楊行密及海陵鎮遏使高霸曲溪人劉金盱眙人賈令威悉以其衆屬焉【楊州盱眙縣西南十里有曲溪劉金曲溪屯將也】行密衆至萬七千人張神劒運高郵糧以給之 朱全忠求救於兖鄆朱瑄朱瑾皆引兵赴之義成軍亦至【二年朱全忠并義成軍徵其兵以撃蔡人】辛巳全忠以四鎮兵攻秦宗權於邊孝村大破之【邊孝村在汴州北郊】斬首二萬餘級宗權宵遁全忠追之至陽武橋而還【陽武橋在鄭州陽武縣縣在汴州西北九十里還從宣翻又如字】全忠深德朱瑄兄事之蔡人之守東都河陽許汝懷鄭陜虢者聞宗權敗皆弃去宗權發鄭州孫儒發河陽皆屠滅其人焚其廬舍而去宗權之勢自是稍衰朝廷以扈駕都頭楊守宗知許州事朱全忠以其將孫從益知鄭州事 錢鏐遣東安都將杜稜浙江都將阮結静江都將成及將兵討薛朗【九域志杭州新城縣有東安鎮浙江静江二都蓋分屯杭州城外沿江一帶自定山下至海門討薛朗以其遂周寶也】 甲午秦彦將宣歙兵三萬餘人乘竹筏沿江而下趙暉邀撃於上元殺溺殆半【歙書涉翻】丙申彦入廣陵自稱權知淮南節度使仍以畢師鐸為行軍司馬補池州刺史趙鍠為宣歙觀察使【鍠戶肓翻】戊戌楊行密帥諸軍抵廣陵城下為八寨以守之【帥讀曰率下同】秦彦閉城自守 【考異曰妖亂志六月癸卯朔秦彦命鄭漢章等守諸門按寇至城下即應城守豈有戊戌行密至癸卯始守城乎今不取】 六月戊申天威都頭楊守立【天威亦神策五十四都之一】與鳳翔節度使李昌符争道麾下相歐【歐烏口翻擊也】帝命中使諭之不止是夕宿衛皆嚴兵為備己酉昌符擁兵燒行宫庚戌復攻大安門【復扶又翻】守立與昌符戰於通衢昌符兵敗帥麾下走保隴州【九域志鳳翔西至隴州一百五十里】杜讓能聞難挺身步入侍韋昭度質其家於軍中【難乃旦翻質音致】誓誅反賊故軍士力戰而勝之守立復恭之假子也壬子以扈駕都將武定節度使李茂貞【李茂貞時領武定節宿衛】為隴州招討使以討昌符 甲寅河中牙將常行儒殺節度使王重榮重榮用法嚴末年尤甚行儒嘗被罸恥之【被皮義翻】遂作亂夜攻府舍重榮逃於别墅【墅承與翻】明旦行儒得而殺之制以陜虢節度使王重盈為護國節度使又以重盈子珙權知陜虢留後【珙居勇翻】重盈至河中執行儒殺之【舊書帝紀云常行儒殺王重榮推重榮兄重盈為兵馬留後】 戊午秦彦遣畢師鐸秦稠將兵八千出城西撃楊行密稠敗死士卒死者什七八城中乏食樵採路絶宣州軍始食人【宣州軍秦彦兵也】壬戌亳州將謝殷逐其刺史宋衮 孫儒既去河陽李罕之召張全義於澤州【去年孫儒陷河陽張全義據懷州李罕之據澤州以拒之蓋懷州逼近河陽全義尋退屯澤州也舊書帝紀云李罕之自澤州收河陽懷州刺史張全義收洛陽】與之收合餘衆罕之據河陽全義據東都共求援於河東李克用以其將安金俊為澤州刺史將騎助之 【考異曰大祖紀年録七月癸巳澤州刺史張全義弃城而遁太祖以安金俊為澤州刺史薛居正五代史亦云七月武皇以金俊為澤州刺史按實録六月全義已除河南尹薛史罕之傳罕之求援克用遣澤州刺史安金俊助之蓋二人先以澤州賂克用非七月也】表罕之為河陽節度使全義為河南尹【考異曰薛居正五代史克用表張言為河南尹東都留守實録以澤州刺史李罕之為河陽節度使懷州刺史張全義為河南尹按諸葛爽表全義為澤州刺史及仲方敗罕之據澤州全義據懷州耳非刺史也】初東都經黄巢之亂遺民聚為三城以相保繼以秦宗權孫儒殘暴僅存壞垣而已全義初至白骨蔽地荆棘彌望居民不滿百戶全義麾下纔百餘人相與保中州城【城在三城之中間故謂之中州城】四野俱無耕者全義乃於麾下選十八人材器可任者人給一旗一牓謂之屯將【將即亮翻】使詣十八縣故墟落中植旗張牓招懷流散勸之樹藝【河南二十縣河南洛陽二縣在城中其外偃師鞏緱氏陽城登封陸渾伊闕新安澠池福昌長水永寜夀安密河清潁陽伊陽王屋凡十八縣】惟殺人者死餘但笞杖而已無嚴刑無租稅民歸之者如市又選壯者教之戰陳以禦寇盜【陳讀曰陣】數年之後都城坊曲漸復舊制諸縣戶口率皆歸復桑麻蔚然野無曠土【蔚音鬱不耕之土曰曠土曠空也】其勝兵者【勝音升】大縣至七千人小縣不減二千人乃奏置令佐以治之【治直之翻】全義明察人不能欺而為政寛簡出見田疇美者輒下馬與僚佐共觀之召田主勞以酒食【勞力到翻】有蠶麥善收者【蠶四伏無病而成繭麥就實黄熟而豐厚為善收】或親至其家悉呼出老幼賜以茶綵衣物民間言張公不喜聲伎【喜許記翻伎渠綺翻】見之未嘗笑獨見佳麥良繭則笑耳有田荒穢者則集衆杖之或訴以乏人牛乃召其鄰里責之曰彼誠乏人牛何不助之衆皆謝乃釋之由是鄰里有無相助故比屋皆有蓄積【比毗至翻又毗必翻】凶年不饑遂成富庶焉【史究言張全義治河南之績効】杜稜等敗薛朗將李君於陽羨【補邁翻於放翻陽羨漢古縣晉】

  【立義興郡隋廢郡改陽羨為義興縣唐武德七年分義興置陽羨縣尋省併入義興九域志義興縣在常州西南百二十里】 秋七月癸未淮南將吳苖帥其徒八千人踰城降楊行密【帥讀曰率】 八月壬寅朔李茂貞奏隴州刺史薛知籌以城降斬李昌符滅其族【中和元年李昌言逐鄭畋據岐兄弟七年而滅】 朱全忠引兵過亳州遣其將霍存襲謝殷斬之【是年六月謝殷殺刺史據亳州】 丙子以李茂貞同平章事充鳳翔節度使【為李茂貞以岐兵跋扈張本】 以韋昭度守太保兼侍中 朱全忠欲兼兖鄆而以朱瑄兄弟有功於已【朱瑄兄弟救汴州破蔡兵】攻之無名乃誣瑄招誘宣武軍士移書誚讓瑄復書不遜【考異曰編遺録八月丙午都指揮使朱珍以諸都將士日有逃逸者初未曉其端今乃知為鄆帥朱瑄因前】

  【年與我師會合討伐蔡寇睹將士驍勇潛有窺覬之心密於境上懸金帛招誘如至者皆厚而納焉積亡既多上察之且不平是事因移文追索亡者朱瑄來言不遜上益怒其欺罔乃議舉兵伐之新傳全忠與朱瑄情好篤密而内忌其雄且所據皆勁兵地欲造怨乃圖之即聲言瑄納汴亡命移書讓瑄以新有恩於全忠故答檄恚望全忠由是顯結其隙高若拙後史補曰梁太祖皇帝到梁園深有大志然兵力不足常欲外掠又虞四境之難每有鬱然之狀時有薦敬秀才於門下乃白梁祖曰明公方欲圖大事輕重必為四境所侵但令麾下將士詐為叛者而逃即明公奏於主上及告四鄰以自襲叛徒為名梁祖曰天降奇人以佐於吾初從其謀一出而致聚十倍蓋翔為温畫策詐令軍士叛歸瑄以為舋端也】全忠遣其將朱珍葛從周襲曹州壬子拔之殺刺史丘弘禮又攻濮州與兖鄆兵戰於劉橋【劉橋在曹州乘氏縣東北濮州范縣西南按薛史戰於臨濮之劉橋】殺數萬人朱瑄朱瑾僅以身免全忠與兖鄆始有隙 秦彦以張雄兵彊冀得其用以僕射告身授雄以尚書告身三通授禆將馮弘鐸等【此等告身蓋高駢為諸道都統時朝廷所給空名告身也】廣陵人競以珠玉金繒詣雄軍貿食【貿音茂以物易物曰貿】通犀帶一得米五升【通犀帶通天犀帶也陸佃埤雅曰犀形似水牛大腹庳脚脚有三蹄黑色三角一在頂上一在額上一在鼻上鼻上即食角也小而不橢亦有一角者舊說犀之通天者惡影常飲濁水重霧厚露之夜不濡其裏白星徹端世云犀望星而徹角即此也可以破水駭雞又犀之美者有光故雞見影而驚其次角理復有正挿倒挿正挿者角腰以上通倒挿者角腰以下通亦曰尖花小而根花大謂之倒挿犀亦絶愛其角墮角即自埋之王粲遊海賦曰羣犀代角巨象解齒是也交州記曰犀有二角鼻上角長額上角短或曰三角者水犀也二角者山犀也在頂者謂之頂犀在鼻者謂之鼻犀犀有四輩其文或如桑椹或如狗鼻者上黔犀無文螺犀文旋㹀犀文細牯犀文大而匀】錦衾一得糠五升雄軍既富不復肯戰未幾復助楊行密【幾居豈翻復扶下翻下同】丁卯彦悉出城中兵萬二千人遣畢師鐸鄭漢章將之陳於城西延袤數里【楊行密軍於楊子蓋並廣陵之西山以逼廣陵城陳讀曰陣下同袤音茂】軍勢甚盛行密安卧帳中曰賊近告我牙將李宗禮曰衆寡不敵宜堅壁自守徐圖還師李濤怒曰吾以順討逆何論衆寡大軍至此去將安歸濤願將所部為前鋒保為公破之【保為于偽翻】濤趙州人也行密乃積金帛麰米於一寨【麰音牟小麥也】使羸弱守之多伏精兵於其旁自將千餘人衝其陳兵始交行密陽不勝而走廣陵兵追之入空寨爭取金帛麰米伏兵四起廣陵衆亂行密縱兵撃之俘斬殆盡積尸十里溝瀆皆滿師鐸漢章單騎僅免自是秦彦不復言出師矣 九月以戶部侍郎判度支張濬為兵部侍郎同平章事 高駢在道院秦彦供給甚薄左右無食至然木像煮革帶食之有相㗖者彦與畢師鐸出師屢敗疑駢為厭勝【厭於涉翻又於琰翻】外圍益急恐駢黨有為内應者有妖尼王奉仙言於彦曰楊州分野極災【分扶問翻】必有一大人死自此喜矣甲戌命其將劉匡時殺駢并其子弟甥姪無少長皆死同坎瘞之【瘞於計翻】乙亥楊行密聞之帥士卒縞素向城大哭三日【帥讀曰率】 朱珍攻濮州朱瑄遣弟罕將步騎萬人救之辛卯朱全忠逆擊罕於范【范漢縣唐屬濮州九域志在州東六十里】擒斬之 冬十月秦彦遣鄭漢章將步騎五千出擊張神劒高霸寨破之神劒奔高郵霸奔海陵【張神劒高霸各奔歸舊屯之地】 丁未朱珍拔濮州刺史朱裕奔鄆珍進兵攻鄆【九域志濮州東至鄆州一百八十里】瑄使裕詐遺珍書【遺惟季翻】約為内應珍夜引兵赴之瑄開門納汴軍閉而殺之死者數千人汴軍乃退瑄乘勝復取曹州【復扶又翻】以其屬郭詞為刺史 甲寅立皇子陞為益王 杜稜等拔常州丁從實奔海陵【光啓二年六月丁從實取常州至是而敗 考異曰實録五月鏐攻常州丁從實投高霸吳越備史在十月新紀十月甲寅陷常州今從之】錢鏐奉周寶歸杭州屬櫜鞬具部將禮郊迎之【杭州本鎮海巡屬故鏐以部將禮迎寶屬音之欲翻櫜音羔鞬其言翻】 楊行密圍廣陵且半年秦彦畢師鐸大小數十戰多不利城中無食米斗直錢五十緡草根木實皆盡以堇泥為餅食之【堇居隱翻堇泥黏土也】餓死者大半宣軍掠人詣肆賣之驅縳屠割如羊豕訖無一聲積骸流血滿於坊市彦師鐸無如之何嚬蹙而已【攢眉為嚬皺頞為蹙】外圍益急彦師鐸憂懣殆無生意【懣音悶】相對抱膝終日悄然【悄七小反詩曰憂心悄悄】行密亦以城久不下欲引還【欲引還廬州】己巳夜大風雨呂用之部將張審威帥麾下士三百晨伏於西壕【帥讀曰率下同】俟守者易代潛登城啓關納其衆守者皆不鬬而潰先是彦師鐸信重尼奉仙雖戰陳日時賞罰輕重皆取決焉【先悉薦翻陳讀曰陣】至是復咨於奉仙曰何以取濟【復扶又翻】奉仙曰走為上策乃自開化門出奔東塘行密帥諸軍合萬五千人入城以梁纘不盡節於高氏為秦畢用斬於戟門之外【唐設戟之制廟社宫殿之門二十有四東宫之門一十有八一品之門十六二品及京兆河南太原尹大都督大都護之門十四三品及上都督中都督上都護上州之門十二下都督下督護中州下州之門各十設戟於門故謂之戟門】韓問聞之赴井死【梁纘韓問一體之人纘既誅問知不免於罪故赴井而死】以高駢從孫愈攝副使使改殯駢及其族城中遺民纔數百家饑羸非復人狀行密輦西寨米以賑之【楊行密寨在廣陵城西此餉軍之米也】行密自稱淮南留後秦宗權遣其弟宗衡將兵萬人度淮與楊行密爭楊州以孫儒為副張佶劉建鋒馬殷及宗權族弟彦暉皆從【從才用翻】十一月辛未抵廣陵城西據行密故寨【攻守之勢地有所必争楊行密之攻廣陵也寨於城西蔡人之攻行密又據其故寨蓋爭形勝者難以他圖也】行密輜重之未入城者為蔡人所得【重直用翻】秦彦畢師鐸至東塘張雄不納將度江趣宣州【秦彦欲還趣舊治趣七喻翻】宗衡召之乃引兵還與宗衡合未幾宗權召宗衡還蔡拒朱全忠孫儒知宗權勢不能久稱疾不行宗衡屢促之儒怒甲戌與宗衡飲酒坐中手刃之傳首於全忠【坐徂卧翻】宗衡將安仁義降於行密仁義本沙陀將也【路振九國志安仁義初事李國昌於塞上以過奔河陽因入秦宗權軍中】行密悉以騎兵委之列於田頵之上【楊行密起於合肥一時諸將田頵為冠一旦得安仁義列於頵上卒收其力用史言其知人善任】儒分兵掠鄰州未幾衆至數萬【孫儒未即攻廣陵先掠鄰州以益其衆幾居豈翻】以城下乏食與彦師鐸襲高郵 初宣武都指揮使朱珍與排陳斬斫使李唐賓勇略功名略相當【陳讀曰陣】全忠每戰使二人偕往無不捷然二人素不相下珍使人迎其妻於大梁不白全忠全忠怒追還其妻殺守門者使親吏蔣玄暉召珍以漢賓代揔其衆【漢賓當作唐賓】館驛巡官馮翊敬翔諫曰【唐制節度使屬官有行軍司馬副使判官支使掌書記巡官衙推各一人同節度副使十人館驛巡官四人】朱珍未易輕取【易以䜴翻】恐其猜懼生變全忠悔使人追止之珍果自疑丙子夜珍置酒召諸將唐賓疑其有異圖斬關奔大梁珍亦弃軍單騎繼至全忠兩惜其才皆不罪遣還濮州【為珍殺唐賓張本】因引兵歸全忠多權數將佐莫測其所為惟敬翔能逆知之往往助其所不及全忠大悅自恨得翔晚凡軍機民政悉以咨之【全忠之移唐祚敬翔之力也李振之徒何關成敗之數哉薛史翔傳曰太祖初鎮大梁有觀察支使王發者翔里人也往依之發無由薦逹翔久之計窘乃代人為牋刺往往有警句傳於軍中太祖不知書喜淺近語聞翔所作愛之召署館驛巡官太祖與蔡賊相拒機略之間翔頗預之太祖大悅恨得翔之晚 考異曰薛居正五代史翔傳曰翔每有所禆贊亦未嘗顯諫上俛仰顧步間微示持疑爾而太祖已察必改行之故禆佐之跡人莫得知按張昭遠莊宗列傳曰温狡譎多謀人不測其際惟翔視彼舉錯即揣知其心或有所不備因為之助温大悅自以為得翔之晚故軍謀政術一切諮之薛史誤】 辛巳高郵鎮遏使張神劒帥麾下二百人逃歸揚州【帥讀曰率】丙戍孫儒屠高郵戊子高郵殘兵七百人潰圍而至楊行密慮其為變分隸諸將一夕盡阬之明日殺神劒於其第【張神劒反覆於呂畢之間而死於楊行密之手挾狡用數者有時而窮也】楊行密恐孫儒乘勝取海陵壬寅命鎮遏使高霸帥其兵民悉歸府城【揚州府城】曰有違命者族之於是數萬戶弃資產焚廬舍挈老幼遷於廣陵戊戌霸與弟【于於翻】部將余繞山【史炤曰風俗通余姓秦由余之後】前常州刺史丁從實至廣陵行密出郭迎之與霸約為兄弟【甘言以安其心】置其將卒於法雲寺【今揚州城中江都縣廨之西有法雲寺然非其舊也】 己亥秦宗權陷鄭州【宗權既弃鄭州今復攻陷之】 朝廷以淮南久亂閏月以朱全忠兼淮南節度使東南面招討使【為朱全忠與揚行密爭淮南張本 考異曰舊紀十一月秦彦引孫儒之兵攻廣陵行密遣使求援於朱全忠制授全忠兼淮南節度使行營兵馬都統薛居正五代史梁太祖紀朝廷就加帝兼領淮南節度在八月十國紀年曰初僖宗聞淮南亂以朱全忠兼淮南節度使至是行密遣使以破賊告朱全忠在十月初入揚州時今從實録】陳敬瑄惡顧彦朗與王建相親【惡烏路翻】恐其合兵圖已謀於田令孜令孜曰建吾子也【令孜養建為子見上卷中和四年】不為楊興元所容故作賊耳【楊興元謂楊守亮事見上卷三年】今折簡召之可致麾下乃遣使以書召之建大喜詣梓州見彦朗曰十軍阿父見召【令孜先為神策十軍觀軍容使待建同父子故稱之】當往省之【省悉景翻】因見陳太師【帝之自成都東還也陳敬瑄進檢校太師故稱之】求一大州若得之私願足矣乃留其家於梓州【顧彦朗治梓州】帥麾下精兵二千【帥讀曰率】與從子宗鐬【從才用翻鐬火外翻】假子宗瑤宗弼宗侃宗弁俱西宗瑤燕人姜郅【燕於賢翻】宗弼許人魏弘夫宗侃許人田師侃宗弁鹿弁也建至鹿頭關西川參謀李乂謂敬瑄曰王建虎也奈何延之入室彼安肯為公下乎敬瑄悔亟遣人止之且增修守備建怒破關而進敗漢州刺史張頊於綿竹【綿竹漢縣江左置晉熙郡隋廢郡為李水縣大業三年改曰綿竹唐屬漢州九域志在州東北九十三里敗補邁翻下同】遂拔漢州進軍學射山又敗西川將句惟立於蠶此【九域志成都府成都縣有蠶此鎮句古侯翻又古侯翻】又拔德陽敬瑄遣使讓之對曰十軍阿父召我來及門而拒之重為顧公所疑【重直用翻】進退無歸矣田令孜登樓慰諭之建與諸將於清遠橋上髠髪羅拜【成都南門樓即大玄樓也樓前有清遠橋】曰今既無歸且辭阿父作賊矣顧彦朗以其弟彦暉為漢州刺史發兵助建急攻成都 【考異曰始建宿衛之時嘗領壁州刺史光啓三年四月已出為利州刺史而舊紀薛居正五代史實録新紀皆云以壁州刺史攻成都誤也張耆舊傳曰光啓四年戊申十月十日田軍容除西川監軍使此月到十一月一日僖宗皇帝晏駕昭宗即位改文德元年文德二年己酉太師有除未下聞朝廷降使三軍百姓僧道詣驛就使車訴論二十年鐵劵有一人驛亭截耳時有微雨卧蹍於泥天使視之無言良久曰不必不必索馬揮鞭便發太師軍容專差親信於人衆中探使有何言既聞二人神色俱喪乃理兵講武更創置三都黄頭都以親密者管之諸軍頻閲隊十月探知朝廷除韋相公授西川節度使已宣麻軍容甚有懼色乃以書召閬州王司徒計其過綿州即出兵拒之令其怒怒必攻諸州所在出兵交戰此是軍容計恐韋相公來交代以兵隔之言王司徒來侵我我所舉兵蓋與王氏相敵欲遮其反名十二月二十日驅人上城一更出兵數千人排於城外北向堤上二十一日王司徒大軍已至城下於城北街去來鬭數合已時川軍被一時築過橋堤上排者大走並收入城至暮王司徒收軍宿七里亭二十二日早又進軍逼城至午又退止七里亭二十三日早引軍入新繁濛陽諸縣界城内出軍日有相持此年十一月改元龍紀元年己酉二月二十五日大戰三郊郊當作交乃各下數寨相守所至縣邑大遭焚燒戶口逃竄十國紀年曰王建起兵攻成都諸書歲月不同蓋建事成之後其徒以擅舉兵為恥為之隱惡襲據閬州多言除移尤諱光啓末寇西川攻陳敬瑄事或移在文德年韋昭度鎮蜀敬瑄不受代後或云朝廷削奪敬瑄官爵建始會昭度討伐皆若受命勤王之師故李昊蜀書毛文錫紀事張彭錦里耆舊傳楊堪平蜀德政碑吳融生祠堂碑馮涓大廳壁記收復卭州壁記皆當時撰錄而自相抵牾吳融云歲在作噩之年相國韋公奉命代蜀又云聖上即位之明年詔大丞相韋公鎮蜀起兵屬丞相以討不庭尋拜公永平節度兼都指揮使今按舊僖宗紀光啟三年十二月東川顧彦朗壁州刺史王建連兵五萬攻成都陳敬瑄告難於朝詔中使諭之唐年補録光啟三年十二月以西川陳敬瑄東川顧彦朗相持詔李茂貞移書和解與唐莊宗功臣列傳唐烈祖實録五代史王建傳莊宗實録范質五代通録王衍傳所載略同韋昭度以文德元年六月始除西川節度使十月至成都陳敬瑄不受代昭度表敬瑄叛十二月丁亥除昭度招討使王建永平節度使據長歷是年十二月甲子朔丁亥二十四日也龍紀元年丁酉歲正月詔命始至成都吳融据昭度受招討使歲月故云作噩之年伐蜀是歲乃昭宗即位之明年韋公鎮蜀在前一年蓋融誤以伐蜀為鎮蜀耳舊紀云文德元年六月以韋昭度為西川節度兩川招撫制置使新書昭宗本紀文德元年十月陳敬瑄反十二月丁亥韋昭度為招討使皆是也而舊紀誤云龍紀元年正月除昭度東都留守五月王建陷成都自稱留後新書陳敬瑄傳全用張耆舊傳云先除昭度節度使然後田令孜召建以限朝廷與本紀及韋昭度傳自相違戾最為差繆張自言年僅八十追記為兒童以來平生見聞為耆舊傳故其叙事鄙俚倒錯與舊史年月不相符合今從五代史王建傳又新紀文德元年六月王建陷漢州執刺史張頊實録龍紀元年正月建破鹿頭關張頊來拒戰敗之按光啟三年十二月韋昭度討陳敬瑄以漢州刺史顧彦暉為軍前指揮使蓋其年冬建破漢州顧彦朗即以彦暉為刺史新紀實録皆誤今從十國紀年】三日不克而退還屯漢州敬瑄告難於朝【難乃旦翻朝直遙翻】詔遣中使和解之又令李茂貞以書諭之皆不從 楊行密欲遣高霸屯天長以拒孫儒袁襲曰霸高氏舊將常挾兩端我勝則來不勝則叛今處之天長【處昌呂翻】是自絶其歸路也不如殺之己酉行密伏甲執霸及丁從實余繞山皆殺之【高霸之死猶張神劒之死也】又遣千騎掩殺其黨於法雲寺死者數千人是日大雪寺外數坊地皆赤高出走明日獲而殺之呂用之之在天長也【是年五月用之歸行密於天長】紿楊行密曰用之有銀五萬鋌【鋌徒鼎翻】埋於所居克城之日願備麾下一醉之資庚戌行密閲士卒顧用之曰僕射許此曹銀何食言邪因牽下械繋命田頵鞫之云與鄭董瑾謀因中元夜邀高駢至其第建黄籙齋【道書以正月十五為上元七月十五為中元十月十五為下元黄籙大齋者普召天神地祗人鬼而設醮焉追懴罪根冀升仙界以為功德不可思議皆誕說也】乘其入静【道家所謂入静即禪家入定而稍異入静者静處一室屏去左右澄神静慮無思無營冀以接天神】縊殺之聲言上升因令莫邪都帥諸軍推用之為節度使【帥讀曰率】是日腰斬用之怨家刳割立盡并誅其族黨軍士發其中堂得桐人書駢姓名於胷桎梏而釘之【釘丁定翻】袁襲言於行密曰廣陵饑敝已甚蔡賊復來民必重困【蔡賊謂孫儒也復扶又翻重直用翻下輜重同】不如避之甲寅行密遣和州將延陵宗以其衆二千人歸和州【孫端所遣助楊行密者今遣還】乙卯又命指揮使蔡儔將兵千人輜重數千兩歸於廬州【為蔡儔背楊行密張本】 趙暉據上元會周寶敗浙西潰卒多歸之【周寶敗見上卷本年上元縣近京口故浙西潰卒多歸之】衆至數萬暉遂自驕大治南朝臺城而居之【隋之平陳也悉毁建康臺城平蕩耕墾更於石頭城置蔣州唐廢蔣州以其地隸潤州光啟二年復置昇州治上元縣蓋臺城之堙廢久矣治直之翻】服用奢僭張雄在東塘暉不與通問雄泝江而上【上時掌翻】暉以兵塞其中流【塞悉則翻】雄怒戊午攻上元拔之暉奔當塗未至為其下所殺餘衆降雄悉阬之【是年夏張雄遣趙暉入據上元今忿其拒已而阬其降者降戶江翻】 朱全忠遣内客將張廷範致朝命於楊行密【致閏月之朝命也】以行密為淮南節度副使又以宣武行軍司馬李璠為淮南留後遣牙將郭言將兵千人送之感化節度使時溥自以於全忠為先進官為都統顧不得領淮南而全忠得之意甚恨望全忠以書假道於溥溥不許璠至泗州溥以兵襲之郭言力戰得免而還徐汴始構怨【自此以後豈特徐汴構怨哉朱全忠以得朝命遂與楊行密争淮南再交兵而再不得志然後息心耳璠孚袁翻】 十二月 【考異曰長歷閏十一月庚子朔十二月己巳朔新舊紀閏月無事不見新紀十二月癸巳在此月是亦以十一月為閏妖亂志有後十一月十國紀年亦閏十一月惟薛居正五代史梁紀十二月後有閏月實録閏十二月庚子朔今不取】癸巳秦宗權所署山南東道留後趙德諲陷荆南節度使張瓌留其將王建肇守城而去【光啟元年張瓌據荆南至是而敗新書城陷瓌死人無識者并投於井復州長史陳璠從瓌至江陵密斷首置囊中走京師獻之授安州刺史與此異】遺民纔數百家 饒州刺史陳儒陷衢州【按路振九國志陳儒同安賊也九域志饒州東南至衢州七百二十九里宋白曰衢州春秋越西鄙之地晉為東陽之境輿地志云漢獻帝初平三年分太末立新安縣晉太康元年以弘農有新安改名信安唐武德四年析婺州西境於信安縣置衢州先有洪水?山為三道因曰三衢州以是名】 上蔡賊帥馮敬章陷蘄州【帥所類翻地名解蘄州以水隈多蘄菜因名州北有蘄水南入于江蘄渠希翻】乙未周寶卒於杭州 【考異曰吳越備史寶病卒實録鏐迎至郡氣卒於樟亭驛新】

  【紀十月丁卯鏐殺周寶十國紀年此月乙未寶卒或曰鏐殺之新傳云鏐迎寶舍樟亭未幾殺之今從吳越備史】 錢鏐以杜稜為常州制置使命阮結等進攻潤州丙申克之劉浩走擒薛朗以歸【光啟三年劉浩逐周寶而奉薛朗至是而敗又自是而後楊行密孫儒之兵迭争常潤二州之民死於兵荒其存者什無一二矣 考異曰吳越備史明年正月丙寅克潤州斬薛朗按朗斬於杭州必不同在一日今從十國紀年】文德元年【是年二月改元】春正月甲寅孫儒殺秦彦畢師鐸鄭漢章彦等之歸宗衡也其衆猶二千餘人其後稍稍為儒所奪禆將唐宏知其必及禍恐并死乃誣告彦等潛召汴軍儒殺彦等以宏為馬軍使 張守一與呂用之同歸楊行密復為諸將合仙丹【復扶又翻為于偽翻合音閤】又欲千軍府之政行密怒而殺之【張守一之殺宜哉嗜利而招權弗可改也已】 蔡將石璠將萬餘人寇陳亳【陳亳二州】朱全忠遣朱珍葛從周將數千騎撃擒之癸亥以全忠為蔡州四面行營都統代時溥 【考異曰新紀正月癸亥全忠為蔡州都統編遺録二月癸未上以時溥阻我兼鎮具事奏聞丙戍上奉唐帝正月二十五日制命授蔡州四面行營都統則丙戌乃全忠受詔之日實録薛居正五代史皆云二月丙戌因此而誤也舊紀五月丁酉朔制以全忠為蔡州都統月日尤誤今從編遺録新紀】諸鎮兵皆受全忠節度張廷範至廣陵揚行密厚禮之及聞李璠來為留後

  怒有不受之色廷範密使人白全忠宜自以大軍赴鎮全忠從之至宋州廷範自廣陵逃來曰行密未可圖也甲子李璠至言徐軍遮道【徐軍謂時漙軍】全忠乃止 丙寅錢鏐斬薛朗 【考異曰新紀丙寅薛朗伏誅鏐陷潤州十國紀年丁巳斬朗今從吳越備史】剖其心以祭周寶【薛朗逐周寶見上卷上年】以阮結為潤州制置使 二月朱全忠奏以楊行密為淮南留後 乙亥上不豫壬午發鳳翔已丑至長安庚寅赦天下改元以韋昭度兼中書令 魏博節度使樂彦禎驕泰不法發六州民【六州魏博貝相澶衛】築羅城方八十里【羅城魏州羅城也】人苦其役其子從訓尤凶險既殺王鐸【事見上卷中和四年】魏人皆惡之【惡烏路翻】從訓聚亡命五百餘人為親兵謂之子將牙兵疑之籍籍不安【魏博牙兵始於田承嗣廢置主帥率由之今樂從訓復置親兵牙兵疑其見圖故不安將即亮翻】從訓懼易服逃出止於近縣彦禎因以為相州刺史從訓遣人至魏運甲兵金帛交錯於路牙兵益疑彦禎懼請避位居龍興寺為僧【中和三年樂彦禎得魏博至是而敗 考異曰舊傳彦禎危懼而卒實録彦禎懼自求避位退居龍興寺軍衆迫令為僧舊紀魏博兵亂逐彦禎若卒不應云逐今從實録】衆推都將趙文㺹知留後事【㺹皮變翻】從訓引兵三萬至城下文㺹不出戰衆復殺之【復扶又翻】推牙將貴鄉羅弘信知留後事先是人有言見白須翁言弘信當為地主者【先悉荐翻】文㺹既死衆羣聚呼曰【呼火故翻】誰欲為節度使者弘信出應曰白須翁已命我矣衆環視曰可也遂立之弘信引兵出與從訓戰敗之【舊書帝紀書是年魏博軍亂逐其帥樂彦禎彦禎子相州刺史從訓帥衆攻魏州牙軍立其小校羅宗弁為留後出兵拒之蓋并趙文㺹羅弘信姓名為一人敗補邁翻】從訓收餘衆保内黄【内黄漢縣時屬魏州九域志縣在州西南一百二十四里宋白曰魏以河北為内河南為外以陳留有外黄此為内黄故縣城在今縣西北十九里】魏人圍之先是朱全忠將討蔡州遣押牙雷鄴以銀萬兩請糴於魏【先悉荐翻】牙兵既逐彦禎殺鄴於館從訓既敗乃求救於全忠 初河陽節度使李罕之與張全義刻臂為盟相得歡甚罕之勇而無謀性復貪暴【復扶又翻】意輕全義聞其勤儉力穡笑曰此田舍一夫耳全義聞之不以為忤【忤五故翻】罕之屢求穀帛全義皆與之而罕之徵求無厭【厭於鹽翻】河南不能給小不如所欲輒械河南主吏至河陽杖之【九域志河南東北至河陽八十五里】河南將佐皆憤怒全義曰李太尉所求奈何不與竭力奉之狀若畏之者罕之益驕罕之所部不耕稼專以剽掠為資啗人為糧【剽匹妙翻啗徒濫翻】至是悉其衆攻絳州絳州刺史王友遇降之進攻晉州護國節度使王重盈密結全義以圖之全義潛發屯兵【張全義尹河南十八縣各置屯將以領屯兵屯兵即民兵也】夜乘虚襲河陽黎明入三城【河陽有南城北城中潬城】罕之踰垣步走全義悉俘其家遂兼領河陽節度使罕之奔澤州【九域志河陽北至澤州九十里】求救於李克用 三月戊戌朔日有食之既 【考異曰舊紀僖宗百僚上徽號曰聖文睿德光武弘孝皇帝三月戊戌朔御正殿受册昭宗紀大順元年正月戊子朔百僚上徽號曰聖文睿德光武弘孝皇帝豈有二帝徽號正同今從新紀止是昭宗尊號】 己亥上疾復作【復扶又翻】壬寅大漸皇弟吉王保長而賢羣臣屬望【屬之欲翻】十軍觀軍容使楊復恭請立其弟夀王傑是日下詔立傑為皇太弟監軍國事 【考異曰唐年補録僖宗御樓後疾復暴崩楊復恭等祕喪不發時十六宅諸王從行乃於六宅中推帝為監國帝之上有盛王儀王皆懿宗之子帝居六宅之第三人舊紀羣臣以吉王最賢又在夀王之上將立之惟楊復恭請以夀王監國按昭宗懿宗第七子吉王保第六新舊傳懿宗八子無盛王儀王今從舊紀】右軍中尉劉季述遣兵迎傑於六王宅【帝兄弟八人侹早薨見王六人居六王宅】入居少陽院【少詩照翻】宰相以下就見之癸卯上崩於靈符殿【年二十七】遺制太弟傑更名敏【更工衡翻】以韋昭度攝冢宰昭宗即位體貌明粹有英氣喜文學【喜許記翻】以僖宗威令不振朝廷日卑有恢復前烈之志尊禮大臣夢想賢豪踐阼之始中外忻忻焉【人心厭亂思治承僖宗之後見昭宗之初政意其足以有為也】 朱全忠裹糧於宋州將攻秦宗權會樂從訓來告急乃移軍屯滑州遣都押牙李唐賓等將步騎三萬攻蔡州遣都指揮使朱珍等分兵救樂從訓 【考異曰薛居正五代史珍傳曰珍軍於内黄敗樂從訓萬餘人按珍往救從訓而云敗從訓誤也葛從周傳曰從太祖度河拔黎陽李固臨河等鎮至内黄破魏軍萬餘衆据薛史紀傳皆云太祖遣朱珍等救從訓獨從周傳云從太祖恐誤也】自白馬濟河下黎陽臨河李固三鎮【元豐九域志澶州有臨河縣在州西六十里魏州魏縣有李固鎮薛史晉紀鄴西有棚曰李固清淇合流在其側】進至内黄敗魏軍萬餘人獲其將周儒等十人【敗補邁翻】 李克用以其將康君立為南面招討使督李存孝薛阿檀史儼安金俊安休休五將騎七千助李罕之攻河陽張全義嬰城自守城中食盡求救於朱全忠以妻子為質【質音致】 王建攻彭城陳敬瑄救之乃去建大掠西川十二州皆被其患【西川統益彭蜀漢嘉眉邛簡資雅黎茂十二州被皮義翻】 夏四月庚午追尊上母王氏曰恭憲皇后 壬午孫儒襲揚州克之 【考異曰實録儒陷揚州在五月恐是約奏到日今据舊紀云四月壬午朔新紀云戊辰妖亂志云四月癸未朔甲申儒陷揚州吳録十國紀年無日但云四月今從舊紀紀年】楊行密出走儒自稱淮南節度使行密將奔海陵袁襲勸歸廬州再為進取之計從之 朱全忠遣其將丁會葛從周牛存節將兵數萬救河陽李存孝令李罕之以步兵攻城自帥騎兵逆戰於温【温縣屬孟州孟州治河陽九域志温在河陽東七十里帥讀曰率】河東軍敗安休休懼罪奔蔡州汴人分兵欲斷太行路【斷都管翻行戶剛翻太行路在河陽北河東兵之歸路也】康君立等懼引兵還全忠表丁會為河陽留後復以張全義為河南尹會夀春人存節博昌人也全義德全忠出已由是盡心附之【朱全忠至此又併有洛孟矣】全忠每出戰全義主給其糧仗無乏李罕之為澤州刺史領河陽節度使罕之留其子頎事克用【頎渠希翻】身還澤州專以寇鈔為事【鈔楚交翻】自懷孟晉絳數百里間州無刺史縣無令長田無麥禾邑無煙火者殆將十年【令力正翻長知兩翻】河中絳州之間有摩雲山絶高民保聚其上寇盜莫能近【近其靳翻】罕之攻拔之時人謂之李摩雲 樂從訓移軍洹水羅弘信遣其將程公信撃從訓斬之與父彦禎皆梟首軍門癸巳遣使以厚幣犒全忠軍請修好【好呼到翻】全忠乃召軍還詔以羅弘信權知魏博留後 歸州刺史郭禹撃荆南逐王建肇【王建肇去年據荆南】建肇奔黔州詔以禹為荆南留後荆南兵荒之餘止有一十七家禹勵精為治撫集彫殘通商務農晚年殆及萬戶【昭宗天復三年成汭為淮南將李神福所敗而死所謂晚年殆此時也治直吏翻】時藩鎮各務兵力相殘莫以養民為事獨華州刺史韓建招撫流散勸課農桑數年之間民富軍贍時人謂之北韓南郭秦宗權别將常厚據夔州禹與其將汝陽許存攻奪之久之朝廷以禹為荆南節度使建肇為武泰節度使【黔州武泰軍】禹奏復姓名為成汭【禹更姓名事見上卷光啟元年】 加李克用兼侍中 五月己亥加朱全忠兼侍中 趙德諲既失荆南【荆南時為成汭所奪】且度秦宗權必敗【度徒洛翻】壬寅舉山南東道來降【中和四年秦宗權遣趙德諲據襄陽至是來降降戶江翻】且自託於朱全忠全忠表請以德諲自副制以山南東道為忠義軍以德諲為節度使充蔡州四面行營副都統 朱全忠既得洛孟無西顧之憂乃大發兵擊秦宗權大破宗權於蔡州之南【舊書帝紀云蔡州行營奏大破賊於隴陂遂進兵以逼賊城】克北關門宗權屯守中州【中州蔡州中城也】全忠分諸將為二十八寨以環之【環音宦】 加鳳翔節度使李茂貞檢校侍中 陳敬瑄方與王建相攻貢賦中絶【言敬瑄前此常輸貢賦中困於兵以致斷絶王建因以為敬瑄罪而問之】建以成都尚彊退無所掠欲罷兵周庠綦毋諫以為不可庠曰邛州城塹完固食支數年可據之以為根本【邛渠容翻】建曰吾在軍中久觀用兵者不倚天子之重則衆心易離【易以䜴翻】不若疏敬瑄之罪表請朝廷令大臣為帥而佐之則功庶可成【帥所類翻】乃使庠草表請討敬瑄以贖罪因求邛州顧彦朗亦表請赦建罪移敬瑄它鎮以靖兩川【王建於東川巡内起兵以攻西川連兵不決兩川皆為之不安】初黄巢之亂上為夀王從僖宗幸蜀【事見二百五十四卷僖宗廣明元年】時事出倉猝諸王多徒行至山谷中夀王疲乏不能前卧磻石上田令孜自後至趣之行【磻蒲官翻趣讀曰促】王曰足痛幸軍容給一馬令孜曰此深山安得馬以鞭抶王使前【抶丑栗翻擊也】王顧而不言心銜之及即位遣人監西川軍令孜不奉詔【令孜倚陳敬瑄不肯離西川】上方憤藩鎮跋扈欲以威制之會得彦朗建表以令孜所恃者敬瑄耳六月以韋昭度兼中書令充西川節度使兼兩川招撫制置等使徵敬瑄為龍武統軍王建軍薪都時綿竹土豪何義陽安仁費師懃等【武德二年分臨邛依政置安仁縣屬邛州九域志在州東北三十八里費父沸翻懃巨斤翻】所在擁兵自保衆或萬人少者千人建遣王宗瑤說之【說式芮翻】皆帥衆附於建【帥讀曰率】給其資糧建軍復振【復扶又翻】 置佑國軍於河南府以張全義為節度使秋七月李罕之引河東兵寇河陽丁會擊却之 升鳳州為節度府割興利州隸之以鳳州防禦使滿存為節度使同平章事【僖宗中和二年以興鳳二州置感義軍楊晟為節度使以守散關未及立軍府晟既敗走不再除帥今始立軍府於鳳州就除滿存為節度使】 以權知魏博留後羅弘信為節度使 八月戊辰朱全忠拔蔡州南城 楊行密畏孫儒之逼欲輕兵襲洪州袁襲曰鍾傳定江西已久【中和二年鍾傳據洪州】兵強食足未易圖也趙鍠新得宣州【去年趙鍠得宣州鍠戶翻】怙亂殘暴衆心不附公宜卑辭厚幣說和州孫端上元張雄【說式芮翻】使自採石濟江侵其境彼必來逆戰公自銅官濟江會之【今池州東北一百四十里銅陵縣有銅官渚】破鍠必矣行密從之使蔡儔守廬州帥諸將濟自糝潭【九域志無為軍無為縣有糝澤鎮今江行自糝潭口東過泥汉口又東過柵江口帥讀曰率下同糝桑感翻】孫端張雄為趙鍠所敗【敗補邁翻】鍠將蘇塘漆朗將兵二萬屯曷山【宣州當塗縣西南有曷山其東則東梁山】袁襲曰公引兵急趨曷山【趨七喻翻】堅壁自守彼求戰不得謂我畏怯因其怠可破也行密從之塘等大敗遂圍宣州鍠兄乾之自池州帥衆救宣州【武德四年以宣州之秋浦南陵二縣置池州貞觀元年州廢永泰元年復分宣州之秋浦青陽饒州之至德置池州九域志池州東至宣州三百二十五里】行密使其將陶雅擊乾之於九華破之【九華山在池州青陽縣界舊名九子山李白以峰有如蓮華改曰九華】乾之奔江西以雅為池州制置使 九月朱全忠以饋運不繼且秦宗權殘破不足憂引兵還丙申遣朱珍將兵五千送楚州刺史劉瓚之官【朱全忠自以兼領淮南楚州其巡屬也故自除刺史】錢鏐遣其從弟銶將兵攻徐約於蘇州【銶音求】 冬十月徐兵邀朱珍劉瓚不聽前【徐兵時漙之兵也】珍等撃之取沛滕二縣斬獲萬計 孟方立遣其將奚忠信將兵三萬襲遼州【遼州本漢上艾沾二縣之地晉置樂平郡武德三年置遼州八年改曰箕州先天元年避玄宗名改曰儀州中和三年復曰遼州】李克修邀擊大破之擒忠信送晉陽辛卯葬惠聖恭定孝皇帝於靖陵【靖陵在京兆奉天縣東北十里】廟號僖宗 陳敬瑄田令孜聞韋昭度將至治兵完城以拒之【治直之翻】 十一月時溥自將步騎七萬屯吳康鎮【薛居正五代史朱珍攻豐下之時漙以全師會戰豐南吳康里】朱珍與戰大破之朱全忠又遣别將攻宿州刺史張友降之【降戶江翻下同】 丙申秦宗權别將攻陷許州執忠武留後王藴復取許州【去年宗權為全忠所敗弃許州王藴蓋全忠所命也】 十二月蔡將申叢執宗權折其足而囚之【折而設翻】降於全忠全忠表叢為蔡州留後 初感義節度使楊晟既失興鳳【見上卷光啟二年】走據文龍成茂四州王建攻西川田令孜以晟已之故將假威戎軍節度使使守彭州【楊晟故神策指揮使】王建攻彭州陳敬瑄眉州刺史山行章將兵五萬壁新繁以救之【新繁漢繁縣蜀後主加新字唐屬成都府九域志在府西北二十五里宋白曰新繁本漢繁縣蜀後主延熙十年涼州胡率衆降禪居之繁縣移戶於此俗謂之新繁縣名因俗而改】 丁亥以韋昭度為行營招討使山南西道節度使楊守亮副之東川節度使顧彦朗為行軍司馬割邛蜀黎雅置永平軍以王建為節度使治邛州充行營諸軍都指揮使 戊子削陳敬瑄官爵 山南西道節度使楊守厚陷夔州【按新書楊守亮時帥山南西道守厚為綿州刺史無亦楊守亮遣守厚陷夔州歟】

  資治通鑑卷二百五十七


    


 


 



 

 
  







 


  
  
 
 
 


  

 















	
	









































 
  



















 





 












  
  
  

 





