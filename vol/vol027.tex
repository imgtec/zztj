<!DOCTYPE html PUBLIC "-//W3C//DTD XHTML 1.0 Transitional//EN" "http://www.w3.org/TR/xhtml1/DTD/xhtml1-transitional.dtd">
<html xmlns="http://www.w3.org/1999/xhtml">
<head>
<meta http-equiv="Content-Type" content="text/html; charset=utf-8" />
<meta http-equiv="X-UA-Compatible" content="IE=Edge,chrome=1">
<title>資治通鑒_28-資治通鑑卷二十七_28-資治通鑑卷二十七</title>
<meta name="Keywords" content="資治通鑒_28-資治通鑑卷二十七_28-資治通鑑卷二十七">
<meta name="Description" content="資治通鑒_28-資治通鑑卷二十七_28-資治通鑑卷二十七">
<meta http-equiv="Cache-Control" content="no-transform" />
<meta http-equiv="Cache-Control" content="no-siteapp" />
<link href="/img/style.css" rel="stylesheet" type="text/css" />
<script src="/img/m.js?2020"></script> 
</head>
<body>
 <div class="ClassNavi">
<a  href="/24shi/">二十四史</a> | <a href="/SiKuQuanShu/">四库全书</a> | <a href="http://www.guoxuedashi.com/gjtsjc/"><font  color="#FF0000">古今图书集成</font></a> | <a href="/renwu/">历史人物</a> | <a href="/ShuoWenJieZi/"><font  color="#FF0000">说文解字</a></font> | <a href="/chengyu/">成语词典</a> | <a  target="_blank"  href="http://www.guoxuedashi.com/jgwhj/"><font  color="#FF0000">甲骨文合集</font></a> | <a href="/yzjwjc/"><font  color="#FF0000">殷周金文集成</font></a> | <a href="/xiangxingzi/"><font color="#0000FF">象形字典</font></a> | <a href="/13jing/"><font  color="#FF0000">十三经索引</font></a> | <a href="/zixing/"><font  color="#FF0000">字体转换器</font></a> | <a href="/zidian/xz/"><font color="#0000FF">篆书识别</font></a> | <a href="/jinfanyi/">近义反义词</a> | <a href="/duilian/">对联大全</a> | <a href="/jiapu/"><font  color="#0000FF">家谱族谱查询</font></a> | <a href="http://www.guoxuemi.com/hafo/" target="_blank" ><font color="#FF0000">哈佛古籍</font></a> 
</div>

 <!-- 头部导航开始 -->
<div class="w1180 head clearfix">
  <div class="head_logo l"><a title="国学大师官网" href="http://www.guoxuedashi.com" target="_blank"></a></div>
  <div class="head_sr l">
  <div id="head1">
  
  <a href="http://www.guoxuedashi.com/zidian/bujian/" target="_blank" ><img src="http://www.guoxuedashi.com/img/top1.gif" width="88" height="60" border="0" title="部件查字,支持20万汉字"></a>


<a href="http://www.guoxuedashi.com/help/yingpan.php" target="_blank"><img src="http://www.guoxuedashi.com/img/top230.gif" width="600" height="62" border="0" ></a>


  </div>
  <div id="head3"><a href="javascript:" onClick="javascript:window.external.AddFavorite(window.location.href,document.title);">添加收藏</a>
  <br><a href="/help/setie.php">搜索引擎</a>
  <br><a href="/help/zanzhu.php">赞助本站</a></div>
  <div id="head2">
 <a href="http://www.guoxuemi.com/" target="_blank"><img src="http://www.guoxuedashi.com/img/guoxuemi.gif" width="95" height="62" border="0" style="margin-left:2px;" title="国学迷"></a>
  

  </div>
</div>
  <div class="clear"></div>
  <div class="head_nav">
  <p><a href="/">首页</a> | <a href="/ShuKu/">国学书库</a> | <a href="/guji/">影印古籍</a> | <a href="/shici/">诗词宝典</a> | <a   href="/SiKuQuanShu/gxjx.php">精选</a> <b>|</b> <a href="/zidian/">汉语字典</a> | <a href="/hydcd/">汉语词典</a> | <a href="http://www.guoxuedashi.com/zidian/bujian/"><font  color="#CC0066">部件查字</font></a> | <a href="http://www.sfds.cn/"><font  color="#CC0066">书法大师</font></a> | <a href="/jgwhj/">甲骨文</a> <b>|</b> <a href="/b/4/"><font  color="#CC0066">解密</font></a> | <a href="/renwu/">历史人物</a> | <a href="/diangu/">历史典故</a> | <a href="/xingshi/">姓氏</a> | <a href="/minzu/">民族</a> <b>|</b> <a href="/mz/"><font  color="#CC0066">世界名著</font></a> | <a href="/download/">软件下载</a>
</p>
<p><a href="/b/"><font  color="#CC0066">历史</font></a> | <a href="http://skqs.guoxuedashi.com/" target="_blank">四库全书</a> |  <a href="http://www.guoxuedashi.com/search/" target="_blank"><font  color="#CC0066">全文检索</font></a> | <a href="http://www.guoxuedashi.com/shumu/">古籍书目</a> | <a   href="/24shi/">正史</a> <b>|</b> <a href="/chengyu/">成语词典</a> | <a href="/kangxi/" title="康熙字典">康熙字典</a> | <a href="/ShuoWenJieZi/">说文解字</a> | <a href="/zixing/yanbian/">字形演变</a> | <a href="/yzjwjc/">金 文</a> <b>|</b>  <a href="/shijian/nian-hao/">年号</a> | <a href="/diming/">历史地名</a> | <a href="/shijian/">历史事件</a> | <a href="/guanzhi/">官职</a> | <a href="/lishi/">知识</a> <b>|</b> <a href="/zhongyi/">中医中药</a> | <a href="http://www.guoxuedashi.com/forum/">留言反馈</a>
</p>
  </div>
</div>
<!-- 头部导航END --> 
<!-- 内容区开始 --> 
<div class="w1180 clearfix">
  <div class="info l">
   
<div class="clearfix" style="background:#f5faff;">
<script src='http://www.guoxuedashi.com/img/headersou.js'></script>

</div>
  <div class="info_tree"><a href="http://www.guoxuedashi.com">首页</a> > <a href="/SiKuQuanShu/fanti/">四库全书</a>
 > <h1>资治通鉴</h1> <!--         下载:【右键另存为】即可 --></div>
  <div class="info_content zj clearfix">
  
<div class="info_txt clearfix" id="show">
<center style="font-size:24px;">28-資治通鑑卷二十七</center>
    資治通鑑卷二十七  宋 司馬光 撰<br />
<br />
  胡三省 音註<br />
<br />
  漢紀十九【起昭陽大淵獻盡玄黓涒灘凡十年】<br />
<br />
  中宗孝宣皇帝下<br />
<br />
  神爵四年春二月以鳳皇甘露降集京師赦天下 潁川太守黄霸在郡前後八年【地節四年潁川太守讓入為左馮翊以霸為潁川太守至元康三年霸入守京兆尹數月還故官至是適九年中間入尹京是在潁川前後八年】政事愈治【治直吏翻下為治政治同】是時鳳皇神爵數集郡國【數所角翻】潁川尤多夏四月詔曰潁川太守霸宣明詔令百姓鄉化孝子弟弟【郷讀曰嚮弟弟上讀曰悌】貞婦順孫日以衆多田者讓畔【師古曰畔田界也】道不拾遺養視鰥寡贍助貧窮獄或八年無重罪囚其賜爵關内侯黄金百斤秩中二千石而潁川孝弟有行義民【行下孟翻】三老力田皆以差賜爵及帛後數月徵霸為太子太傅 五月匈奴單于遣弟呼留若王勝之來朝【師古曰呼留若者王之號也勝之其人名考異曰匈奴傳握衍朐鞮單于立復修和親遣弟伊酋若王勝之入漢獻見蓋即謂此也】冬十月鳳皇十一集杜陵 河南太守嚴延年為治隂鷙酷烈衆人所謂當死者一朝出之所謂當生者詭殺之【師古曰詭違正理而殺也】吏民莫能測其意深淺戰栗不敢犯禁冬月傳屬縣囚會論府上【傳知戀翻又直戀翻師古曰總集郡府而論殺】流血數里河内號曰屠伯【鄧展曰言延年殺人如屠兒之殺六畜也伯長也】延年素輕黄霸為人及比郡為守【師古曰比接近也音頻二翻】褒賞反在已前心内不服河南界中又有蝗蟲府丞義出行蝗【行下孟翻】還見延年延年曰此蝗豈鳳皇食邪義年老頗悖【師古曰悖心惡惑也音布内翻】素畏延年恐見中傷延年本嘗與義俱為丞相史實親厚之饋遺之甚厚【中竹仲翻遺于季翻】義愈益恐自筮得死卦忽忽不樂【樂音洛】取告至長安【師古曰取告取休假也】上書言延年罪名十事已拜奏因飲藥自殺以明不欺事下御史丞按驗【百官表御史大夫有兩丞秩千石一曰中丞下遐稼翻】得其語言怨望誹謗政治數事十一月延年坐不道棄市初延年母從東海來欲從延年臘【風俗通引禮傳曰夏曰嘉平殷曰清祀周曰大蜡漢改曰臘臘者獵也因獵取獸以祭先祖或曰新故交接大祭以報功也蔡邕獨斷曰臘者歲終大祭縱吏民飲宴高堂隆曰王者各以其行之盛祖以其終臘水始於申盛於子終於辰故水行之君以子祖辰臘火始於寅盛於午終於戌故大行之君以午祖戌臘木始於亥盛於卯終於未故木行之君以卯祖未臘金始於己盛於酉終於丑故金行之君以酉祖丑臘土始於未盛於戌終於辰故土行之君以戌祖辰臘師古曰建丑之月為臘祭因會飲若今之蜡節也】到洛陽適見報囚【師古曰奏報行決也原父曰檢尋前後直謂斷决囚為報耳非奏得報也如今有司書囚罪長吏判凖斷是所謂報也】母大驚便止都亭【凡郡縣皆有都亭秦法十里一亭郡縣治所則置都亭】不肯入府延年出至都亭謁母母閉閤不見延年免冠頓首閤下良久母乃見之因數責延年【數所具翻】幸得備郡守專治千里【守式又翻治直之翻】不聞仁愛教化有以全安愚民顧乘刑罰多刑殺人【師古曰顧反也乘因也】欲以立威豈為民父母意哉延年服罪重頓首謝【師古曰重音直用翻】因為母御歸府舍【為于偽翻下同】母畢正臘【師古曰臘及正歲禮畢也正音之盈翻】謂延年曰天道神明人不可獨殺【師古曰言多殺人者已亦當死也】我不意當老見壮子被刑戮也【師古曰言素意不謂如此也】行矣去汝東歸埽除墓地耳【師古曰言待其喪至也】遂去歸郡見昆弟宗人復為言之後歲餘果敗東海莫不賢智其母【師古曰稱其賢智也】 匈奴握衍朐鞮單于暴虐好殺伐【好呼到翻】國中不附及太子左賢王數讒左地貴人【左地貴人謂左谷蠡王以下至左大當戶統兵者也數所角翻】左地貴人皆怨會烏桓擊匈奴東邊姑夕王頗得人民單于怒姑夕王恐即與烏禪幕及左地貴人共立稽侯㹪為呼韓邪單于【㹪先安翻又所姦翻】發左地兵四五萬人西擊握衍朐鞮單于至姑且水北【師古曰且音子余翻】未戰握衍朐鞮單于兵敗走使人報其弟右賢王曰匈奴共攻我若肯發兵助我乎【師古曰若汝也此下亦同】右賢王曰若不愛人殺昆弟諸貴人各自死若處無來汙我【師古曰言於汝所居處自死汙烏故翻】握衍朐鞮單于恚自殺【恚於避翻】左大且渠都隆奇亡之右賢王所【都隆奇本立握衍朐鞮單于故亡且子余翻】其民盡降呼韓邪單于【降戶江翻】呼韓邪單于歸庭數月罷兵使各歸故地乃收其兄呼屠吾斯在民間者立為左谷蠡王【谷音鹿蠡盧奚翻下同】使人告右賢貴人欲令殺右賢王其冬都隆奇與右賢王共立日逐王薄胥堂為屠耆單于發兵數萬人東襲呼韓邪單于呼韓邪單于兵敗走屠耆單于還以其長子都塗吾西為左谷蠡王少子姑瞀樓頭為右谷蠡王留居單于庭【屠耆使二子守單于庭而身西還也師古曰瞀音莫構翻】<br />
<br />
  五鳳元年春正月上幸甘泉郊泰畤【畤音止】皇太子冠【冠古玩翻考異曰按宣紀太子冠在此年而荀紀於元康三年疑二疏去位事已云皇太子冠至是又重複言之蓋誤也】 秋七月匈奴屠耆單于使先賢撣兄右奥鞬王與烏藉都尉各二萬騎屯東方以備呼韓邪單于【撣音纒又音田】是時西方呼揭王來與唯犂當戶謀【師古曰揭音丘例翻唯音弋癸翻】共讒右賢王言欲自立為單于屠耆單于殺右賢王父子後知其寃復殺唯犂當戶【復扶又翻】於是呼揭王恐遂畔去自立為呼揭單于右奧鞬王聞之即自立為車犂單于【奧音郁鞬居言翻】烏藉都尉亦自立為烏藉單于凡五單于屠耆單于自將兵東擊車犂單于使都隆奇擊烏藉烏藉車犂皆敗西北走與呼揭單于兵合為四萬人烏藉呼揭皆去單于號【去羌呂翻】共并力尊輔車犂單于屠耆單于聞之使左大將都尉將四萬騎分屯東方以備呼韓邪單于自將四萬騎西擊車犂單于車犂單于敗西北走屠耆單于即引兵西南留闒敦地【師古曰闒音蹋敦音頓又音對】漢議者多曰匈奴為害日久可因其壞亂舉兵滅之詔問御史大夫蕭望之對曰春秋晉士匄帥師侵齊聞齊侯卒引師而還君子大其不伐喪【師古曰士匄晉大夫范宣子也公羊傳襄十九年齊侯環卒晉士匄帥師侵齊至穀聞齊侯卒乃還還者何善辭也大其不伐喪也卒子恤翻】以為恩足以服孝子誼足以動諸侯前單于慕化鄉善稱弟【蘇林曰弟順也師古曰郷讀曰嚮弟音悌仲馮曰漢與匈奴嘗約為兄弟此弟直自為弟耳】遣使請求和親海内欣然夷狄莫不聞未終奉約不幸為賊臣所殺今而伐之是乘亂而幸災也彼必犇走遠遁不以義動兵恐勞而無功宜遣使者弔問輔其微弱救其災患四夷聞之咸貴中國之仁義如遂蒙恩得復其位必稱臣服從此德之盛也上從其議 冬十有二月乙酉朔日有食之 韓延夀代蕭望之為左馮翊望之聞延夀在東郡時放散官錢千餘萬使御史案之【師古曰望之以延壽代己為馮翊而有能名出已之上故忌害之欲䧟以罪法】延夀聞知即部吏案校望之在馮翊時廪犧官錢放散百餘萬【左馮翊屬官有廪犧令丞尉師古曰廪主藏穀犧主養牲皆所以供祭祀也校居孝翻】望之自奏職在總領天下聞事不敢不問而為延夀所拘持上由是不直延夀各令窮竟所考望之卒無事實【卒子恤翻】而望之遣御史案東郡者得其試騎士日奢僭踰制【師古曰試騎士每歲大試也余謂即都試也據延夀傳治飾兵車畫龍虎朱爵延壽衣黄紈方領駕四馬傳總建幢棨植羽葆鼔車歌車功曹引車皆駕四馬建棨戟五騎為伍分左右部軍假司馬千人持幢旁轂歌者先居射室望見延夀車噭咷楚歌延夀坐射室騎吏持戟夾陛列立騎士從者帶弓鞬羅後令騎士兵車四面營陳被甲鞮鍪居馬上抱弩負蘭又使騎士戲車弄馬盗驂所謂奢僭踰制者也噭音叫咷音它釣翻】又取官銅物候月食鑄刀劒效尚方事【據劉向傳上令典尚方鑄作事師古注曰尚方鑄巧作金銀之所若今之中尚署又漢制尚方主作御刀劒】及取官錢私假傜使吏【師古曰假謂顧賃也】及治飾車甲三百萬以上【治直之翻】延夀竟坐狡猾不道棄市吏民數千人送至渭城老小扶持車轂争奏酒炙【師古曰奏進也炙之夜翻燔肉也】延夀不忍距逆人人為飲【為于偽翻】計飲酒石餘使掾史分謝送者遠苦吏民延夀死無所恨百姓莫不流涕<br />
<br />
  二年春正月上幸甘泉郊泰畤【考異曰宣紀云三月行幸甘泉荀紀作正月按漢制常以正月郊祀蓋荀悦作紀之時本猶未誤也又楊惲傳曰行必不至河東矣蓋時亦幸河東祠后土史脱之也】 車騎將軍韓增薨五月將軍許延夀為大司馬車騎大將軍 丞相丙吉年老上重之蕭望之意常輕吉上由是不悦丞相司直奏望之遇丞相禮節倨慢【時緐延夀為丞相司直師古曰緐音婆】又使吏買賣私所附益凡十萬三千【師古曰使其吏為望之家有所買賣而吏以其私錢增益之用潤望之也】請逮捕繫治秋八月壬午詔左遷望之為太子太傅以太子太傅黄霸為御史大夫 匈奴呼韓邪單于遣其弟右谷蠡王等西襲屠耆單于屯兵殺略萬餘人【谷蠡音鹿黎】屠耆單于聞之即自將六萬騎擊呼韓邪單于屠耆單于兵敗自殺都隆奇乃與屠耆少子右谷蠡王姑瞀樓頭亡歸漢車犂單于東降呼韓邪單于【降戶江翻下同】冬十一月呼韓邪單于左大將烏厲屈與父呼遫累烏厲溫敦皆見匈奴亂率其衆數萬人降漢封烏厲屈為新城侯烏厲溫敦為義陽侯【師古曰呼遫累其官號也遫古速字累音力追翻功臣侯表新城侯食邑於汝南之細陽義陽侯食邑於南陽之平氏 考異曰宣紀呼遫累單于帥衆來降功臣表信成侯王定以匈奴烏桓屠驀單于子左大將軍率衆降侯義陽侯厲温敦以匈奴謼連累單于率衆降侯此即屈與敦也未嘗為單于或降時自稱單于或紀表二者誤也】是時李陵子復立烏籍都尉為單于【復扶又翻】呼韓邪單于捕斬之遂復都單于庭然衆裁數萬人屠耆單于從弟休旬王自立為閏振單于在西邊【從才用翻】呼韓邪單于兄左賢王呼屠吾斯亦自立為郅支骨都侯單于在東邊光禄勲平通侯楊惲【功臣侯表平通侯食邑於汝南之博陽】廉潔無私然伐其行能【伐矜也行身所行也能才所堪也行下孟翻】又性刻害好發人隂伏【好呼到翻】由是多怨於朝廷與太僕戴長樂相失【樂音洛】人有上書告長樂罪長樂疑惲敎人告之亦上書告惲罪曰惲上書訟韓延夀郎中丘常謂惲曰聞君侯訟韓馮翊當得活乎惲曰事何容易【易以䜴翻】脛脛者未必全也【師古曰脛脛直貌也脛下頂翻】我不能自保【師古曰言我尚不能自保訟人何以得活】真人所謂鼠不容穴銜窶數者也【李奇曰真人正人也如淳曰所以不容穴正坐銜窶數自妨故不得入穴也師古曰窶數戴盆器也以盆盛物戴於頭者則以窶數薦之今賣白團餅人所用者是也窶音其羽翻數音山羽翻】又語長樂曰【語牛倨翻】正月以來天隂不雨此春秋所記夏侯君所言【張晏曰夏侯勝諫昌邑王曰天久隂不雨臣下必有謀上者春秋無久隂不雨之異也漢史記勝所言故曰春秋所記謂說春秋災異耳師古曰春秋有不雨事說者因論久隂附著之也張晏謂漢史為春秋失之矣】事下廷尉【下遐稼翻】廷尉定國奏惲怨望為訞惡言【于定國也訞與妖同】大逆不道上不忍加誅有詔皆免惲長樂為庶人【考異曰宣紀十二月楊惲坐前為光禄勲有罪免為庶人不悔過怨望大逆不道要斬荀紀因而用之惲傳惲與孫會宗書曰臣之得罪己三年矣又因日食之變騶馬猥佐成上書告惲罪下獄死又楊譚稱杜延年為御史大夫按百官表惲以神爵元年為光禄勲五年免戴長樂亦以其年為太僕五年免杜延年以五鳳三年六月辛酉為御史大夫又按蕭望之傳使光禄勲惲策免望之其事在今年八月惲猶為光禄勲至四年四月乃有日蝕之變蓋惲以今年十二月免為庶人至四年乃死宣紀誤也】<br />
<br />
  三年春正月癸卯博陽定侯丙吉薨<br />
<br />
  班固贊曰古之制名必由象類遠取諸物近取諸身【易大傳有是言】故經謂君為元首臣為股肱【師古曰謂虞書益稷云元首明哉股肱良哉也】明其一體相待而成也是故君臣相配古今常道自然之埶也近觀漢相高祖開基蕭曹為冠【師古曰名位在衆人之上也余謂此言其相業冠羣后耳冠古玩翻】孝宣中興丙魏有聲是時黜陟有序【黜降也陟陞也】衆職修理公卿多稱其位【稱尺證翻】海内興於禮讓覽其行事豈虚虖哉【師古曰言君明臣賢所以致治非徒然】<br />
<br />
  二月壬辰黄霸為丞相霸材長於治民及為丞相功名損於治郡【治直之翻】時京兆尹張敞舍鶡雀飛集丞相府【蘇林曰今虎賁所著鶡也師古曰蘇說非也此鶡音芬本作鳻此通用耳鳻雀大而色青出羌中非武賁所著也武賁鶡者色黑出上黨以其鬪死不止故用其羽飾武臣首云今時俗所謂鶡雞者也音曷非此鳻雀也】霸以為神雀議欲以聞敞奏霸曰竊見丞相謂與中二千石博士雜問郡國上計長史守丞為民興利除害成大化【為于偽翻】條其對有耕者讓畊男女異路道不拾遺及舉孝子貞婦者為一輩先上殿【師古曰丞相所坐屋也古者屋之高嚴通呼為殿不必宫中也余據鄭玄周禮注漢司徒府有天子以下大會殿後漢之司徒府則前漢之丞相府也】舉而不知其人數者次之不為條教者在後叩頭謝丞相口雖不言而心欲其為之也長史守丞對時臣敞舍有鶡雀飛止丞相府屋上丞相以下見者數百人邊吏多知鶡雀者問之皆陽不知丞相圖議上奏【師古曰圖謀也】曰臣問上計長史守丞以興化條【師古曰凡言條者一一而疏舉之若木條然也】皇天報下神爵後知從臣敞舍來乃止郡國吏竊笑丞相仁厚有知畧【知與智同】微信奇怪也臣敞非敢毁丞相也誠恐羣臣莫白【恐羣臣莫敢白其事也】而長史守丞畏丞相指歸舍法令各為私教【師古曰舍廢也讀曰捨】務相增加澆淳散樸【師古曰不雜為淳以水澆之則味漓薄樸大質也割之散也】並行偽貌有名亡實【亡古無字通】傾揺解怠【師古曰解讀曰懈】甚者為妖【妖於驕翻】假令京師先行讓畔異路道不拾遺其實亡益廉貪貞淫之行【亡古無字通行下孟翻】而以偽先天下【先悉薦翻】固未可也即諸侯先行之偽聲軼於京師【師古曰軼過也音逸】非細事也漢家承敝通變造起律令所以勸善禁姦條貫詳備不可復加【復扶又翻】宜令貴臣明飭長史守丞【師古曰飭讀與敇同】歸告二千石舉三老孝弟力田孝廉廉吏務得其人郡事皆以法令為檢式【師古曰檢局也音居儉翻】毋得擅為條教敢挾詐偽以奸名譽者必先受戮【師古曰奸求也音干】以正明好惡【好呼到翻惡烏路翻】天子嘉納敞言召上計吏使侍中臨飭如敞指意霸甚慙又樂陵侯史高以外屬舊恩侍中貴重【樂陵縣屬平原郡師古曰樂音來各翻史高者帝祖母史良娣兄恭之長子】霸薦高可太尉天子使尚書召問霸太尉官罷久矣【尚書屬少府成帝建始四年增置為五員自文帝罷太尉官至景帝以周亞夫為太尉尋罷至武帝以田蚡為太尉罷後不復除授】夫宣明教化通逹幽隱使獄無寃刑邑無盗賊君之職也將相之官朕之任焉【師古曰言欲拜將相自在朕也】侍中樂陵侯高帷幄近臣朕之所自親【師古曰言具知其材質】君何越職而舉之【丞相職總百官進賢退不肖霸薦史高以為所薦非其人可也以為越職則非也蓋自武帝以來丞相之失其職也久矣】尚書令受丞相對【後漢志尚書令承秦所置武帝用宦者更為尚書謁者令成帝用士人復故掌凡選署及奏下尚書曹文書衆事帝既使尚書召問霸故使尚書令受其對也尚書令中書令沈約以為兩官注已見前】霸免冠謝罪數日乃决【師古曰乃得免罪也】自是後不敢復有所請【復扶又翻】然自漢興言治民吏以霸為首【治直之翻】 三月上幸河東祠后土減天下口錢【如淳曰漢儀注民年七歲至十四出口賦錢人二十三二十錢以食天子其三錢者武帝加口錢以補車騎馬】赦天下殊死以下 六月辛酉以西河太守杜延年為御史大夫【考異曰荀紀作辛巳百官表作辛酉按長歷此月丙午朔無辛巳】置西河北地屬國以處匈奴降者【處昌呂翻】 廣陵厲王胥使巫李女須祝詛上求為天子【師古曰女須者巫之名也祝職救翻詛芷助翻】事覺藥殺巫及宫人二十餘人以絕口公卿請誅胥<br />
<br />
  四年春胥自殺 匈奴單于稱臣遣弟右谷蠡王入侍【考異曰按匈奴傳呼韓邪稱臣即遣銖婁渠堂入侍事在明年時匈奴有三單于不知此單于為誰也余按通鑑據班紀而書此事又參考匈奴傳以明其異】以邊塞亡寇【亡古無字通】減戍卒什二 大司農中丞耿夀昌奏言歲數豐穰穀賤農人少利【時穀石五錢所謂穀賤傷農者也數所角翻少詩沼翻】故事歲漕關東穀四百萬斛以給京師用卒六萬人宜糴三輔弘農河東上黨太原郡穀足供京師可以省關東漕卒過半上從其計夀昌又白令邊郡皆築倉以穀賤增其賈而糴穀貴時減賈而糶【賈讀曰價】名曰常平倉【常平倉始此】民便之上乃下詔賜夀昌爵關内侯 夏四月辛丑朔日有食之 楊惲既失爵位家居治產業以財自娯其友人安定太守西河孫會宗與惲書諫戒之為言大臣廢退當闔門惶懼為可憐之意【師古曰闔閉也為于偽翻】不當治產業通賓客有稱譽【治直之翻】惲宰相子【惲宰相楊敞之子也】有材能少顯朝廷【少詩沼翻】一朝以晻昧語言見廢【晻與暗同】内懷不服報會宗書曰竊自思念過已大矣行已虧矣【行下孟翻】常為農夫以沒世矣是故身率妻子戮力耕桑不意當復用此為譏議也【復扶又翻】夫人情所不能止者聖人弗禁故君父至尊親送其終也有時而既【師古曰君至尊父至親張晏曰喪不過三年臣見放逐降居三月復初師古曰既已也原父曰惲但云送終三年本不及放逐三月也余謂惲之此言實因廢棄而有怨望之意】臣之得罪已三年矣田家作苦歲時伏臘【作苦謂耕作勞苦也史記秦作伏祠改蜡曰臘釋名曰伏者金氣伏藏之日也金畏火故三伏皆庚日歷忌曰四時代謝皆以相生至於立秋以金代火金畏火故庚日必伏毛晃曰夏有三伏冬有臘故稱歲時伏臘】烹羊炰羔【師古曰炰羔炙肉也即今所謂爊也余按羊子曰羔未離乳者也其肉嫩美炰音步交翻】斗酒自勞【勞音來到翻】酒後耳熱仰天拊缶而呼烏烏【應劭曰缶瓦器也秦人擊之以節歌師古曰缶即今之盆類也李斯上秦王書云擊甕叩缶彈筝搏髀而歌呼烏烏快耳者真秦聲也是關中舊有此曲】其詩曰田彼南山蕪穢不治種一頃豆落而為萁【治田曰田音堂練翻詩云無田甫田張晏曰山高而在陽人召之象也蕪穢不治言朝廷之荒亂也一頃百畝以喻百官也言豆貞實之物當在囷倉零落在野喻已見放棄也萁曲而不直言朝臣皆謟諛也師古曰萁豆莖也音基治直之翻】人生行樂耳【樂音洛】須富貴何時【師古曰須待也】誠荒淫無度不知其不可也【師古曰自謂為可也】又惲兄子安平侯譚【惲兄忠襲父敞爵安平侯忠死譚嗣】謂惲曰侯罪薄又有功【謂惲有發霍氏謀反之功也】且復用【復扶又翻】惲曰有功何益縣官不足為盡力【為于偽翻】譚曰縣官實然蓋司隸韓馮翊皆盡力吏也俱坐事誅【蓋司隸事見上卷神爵二年韓馮翊事見上元年蓋古盍翻】會有日食之變騶馬猥佐成上書告惲驕奢不悔過【如淳曰騶馬以給騶使乘之佐主猥馬吏也有史有佐名成也】日食之咎此人所致章下廷尉按驗得所予會宗書帝見而惡之【下遐稼翻予讀曰與惡烏路翻】廷尉當惲大逆無道要斬【師古曰當謂處斷其罪要與腰同】妻子徙酒泉郡譚坐免為庶人諸在位與惲厚善者未央衛尉韋玄成及孫會宗等皆免官<br />
<br />
  臣光曰以孝宣之明魏相丙吉為丞相于定國為廷尉而趙蓋韓楊之死皆不厭衆心【厭於贍翻滿也】其為善政之累大矣【累力瑞翻】周官司寇之法有議賢議能【周官小司寇之職以八辟麗邦法附刑罰三曰議賢之辟四曰議能之辟鄭玄注曰賢謂有德行者能謂有道藝者鄭衆曰若今時亷吏有罪先請是也】若廣漢延夀之治民可不謂能乎【治直之翻】寛饒惲之剛直可不謂賢乎然則雖有死罪猶將宥之况罪不足以死乎楊子以韓馮翊之愬蕭為臣之自失【楊子或問臣之自失曰韓馮翊之愬蕭趙京兆之犯魏】夫所以使延夀犯上者望之激之也上不之察而延夀獨蒙其辜不亦甚哉<br />
<br />
  匈奴閏振單于率其衆東擊郅支單于郅支與戰殺之并其兵遂進攻呼韓邪呼韓邪兵敗走郅支都單于庭【郅支忘呼韓邪樹立之恩以兄弟而尋干戈為漢所誅宜矣】<br />
<br />
  甘露元年【以甘露降紀元說文露潤澤也五經通義和氣津疑為露也蔡邕月令曰露者隂之液也】春正月行幸甘泉郊泰畤 楊惲之誅也公卿奏京兆尹張敞惲之黨友不宜處位【處昌呂翻】上惜敞材獨寑其奏不下【師古曰天子惜敞故留所奏事不出下遐稼翻】敞使掾絮舜有所案驗【李奇曰絮音挐師古曰絮姓也音女居翻又音人餘翻舜其名】舜私歸其家曰五日京兆耳【舜以敞被奏當免在位不久也】安能復案事【復扶又翻】敞聞舜語即部吏收舜繫獄晝夜驗治竟致其死事【罪不至死而以事致之所謂文致也治直之翻】舜當出死敞使主簿持教告舜曰五日京兆竟何如冬月已盡延命乎【主簿處郡閤下主文簿因以名官師古曰言汝不欲望延汝命乎】乃棄舜市會立春行寃獄使者出【行下孟翻】舜家載尸并編敞教【師古曰編聯也聯之於章前也】自言使者使者奏敞賊殺不辜上欲令敞得自便【師古曰從輕法以免也】即先下敞前坐楊惲奏免為庶人【下遐稼翻】敞詣闕上印綬【上時掌翻】便從闕下亡命【此即令之得自便也師古曰亡命不還其本縣邑也賢曰命名也謂脱其名籍而逃亡】數月京師吏民解弛【師古曰弛放也解讀曰懈或如字】枹鼓數起【盗賊多也枹音膚數所角翻下同】而冀州部中有大賊天子思敞功効使使者即家在所召敞【師古曰就其所居處而召之】敞身被重劾【師古曰謂前有賊殺不辜之事劾戶槩翻下同】及使者至妻子家室皆泣而敞獨笑曰吾身亡命為民郡吏當就捕今使者來此天子欲用我也裝隨使者【治行裝而隨使者也】詣公車上書曰臣前幸得備位列卿待罪京兆【西都之制為三輔者列於九卿待罪者謙言也謂身居其官而不稱職則將有瘝曠之罪故謂居職為待罪西都之臣率有是言】坐殺掾絮舜舜本臣敞素所厚吏數蒙恩貸【師古曰貸音土帶翻宥罪曰貸】以臣有章劾當免受記考事【師古曰記書也若今之州縣為符敎也】便歸臥家謂臣五日京兆背恩忘義【背蒲妹翻】傷薄俗化臣竊以舜無狀枉法以誅之臣敞賊殺不辜鞠獄故不直雖伏明法死無所恨天子引見敞【見賢遍翻】拜為冀州刺史【冀州部魏郡鉅鹿常山清河等郡廣平真定中山信都河間等國 考異曰荀紀載於五鳳二年因楊惲事并致此誤也百官表敞以神爵元年為京兆尹八年免敞傳云為京兆九歲免】敞到部盗賊屏迹【屏必郢翻】 皇太子柔仁好儒見上所用多文法吏以刑繩下常侍燕從容言陛下持刑太深宜用儒生【好呼到翻下同從千容翻】帝作色曰【師古曰作動也意怒故動色】漢家自有制度本以霸王道雜之奈何純任德教用周政乎【師古曰姬周之政】且俗儒不逹時宜【風俗通曰儒者區也言其區别古今居則翫聖哲之辭動則行典籍之道稽先王之制立當時之事此通儒也若能納而不能出能言而不能行講誦而已無能往來此俗儒也】好是古非今使人眩於名實【師古曰眩亂視也音胡眄翻】不知所守何足委任乃歎曰亂我家者太子也<br />
<br />
  臣光曰王霸無異道昔三代之隆禮樂征伐自天子出則謂之王天子微弱不能治諸侯諸侯有能率其與國同討不庭以尊王室者則謂之霸【庭直也不庭不直也一說以諸侯不朝為不庭治直之翻】其所以行之也皆本仁祖義任賢使能賞善罰惡禁暴誅亂顧名位有尊卑德澤有深淺功業有鉅細政令有廣狹耳非若白黑甘苦之相反也漢之所以不能復三代之治者由人主之不為非先王之道不可復行於後世也【復扶又翻】夫儒有君子有小人【論語孔子謂子夏曰汝為君子儒母為小人儒謝顯道為之說曰志於義則大是以謂之君子志於利則小是以謂之小人】彼俗儒者誠不足與為治也【治直吏翻下同】獨不可求真儒而用之乎稷契臯陶伯益伊尹周公孔子皆大儒也【契息列翻陶音遥】使漢得而用之功烈豈若是而止邪孝宣謂太子懦而不立闇於治體必亂我家則可矣乃曰王道不可行儒者不可用豈不過哉非所以訓示子孫垂法將來者也<br />
<br />
  淮陽憲王好法律【淮陽王欽上次子也好呼到翻】聰逹有材王母張倢伃尤幸【倢伃音接予】上由是疏太子而愛淮陽憲王數嗟歎憲王曰真我子也常有意欲立憲王然用太子起於微細上少依倚許氏【疏讀曰疎數所角翻少詩照翻依倚許氏事見二十四卷昭帝元平元年】及即位而許后以殺死【事見二十四卷本始三年】故弗忍也久之上拜韋玄成為淮陽巾尉以玄成嘗讓爵於兄【事見二十五卷元康四年】欲以感諭憲王由是太子遂安匈奴呼韓邪單于之敗也左伊秩訾王為呼韓邪計【師古曰訾音子移翻】勸令稱臣入朝事漢從漢求助如此匈奴乃定呼韓邪問諸大臣皆曰不可匈奴之俗本上氣力而下服役【師古曰以服役於人為下】以馬上戰鬭為國故有威名於百蠻戰死壯士所有也【師古曰言人皆有此事耳余謂壯士健闘則戰死乃本分必有之事】今兄弟争國不在兄則在弟【郢支兄也呼韓邪弟也】雖死猶有威名子孫常長諸國【師古曰為諸國之長帥也長知兩翻下同】漢雖彊猶不能兼并匈奴奈何亂先古之制臣事於漢卑辱先單于【師古曰言忝辱之更令卑下也余謂此言先單于與漢争為長雄而今單于臣事之是卑辱先單于於地下也】為諸國所笑雖如是而安何以復長百蠻【復扶又翻】左伊秩訾曰不然彊弱有時今漢方盛烏孫城郭諸國皆為臣妾自且鞮侯單于以來匈奴日削不能取復【且鞮侯單于呼韓邪之曾祖也復報也且子余翻】雖屈彊於此【師古曰屈音其勿翻】未嘗一日安也今事漢則安存不事則危亡計何以過此諸大人相難久之【難乃旦翻】呼韓邪從其計【從左伊秩訾王之計也】引衆南近塞【近其靳翻】遣子右賢王銖婁渠堂入侍【師古曰銖音殊婁音力于翻】郅支單于亦遣子右大將駒于利受入侍 二月丁巳樂成敬侯許延夀薨【恩澤侯表樂成侯食邑於南陽之平氏】 夏四月黄龍見新豐【見賢遍翻】 丙申太上皇廟火甲辰孝文廟火上素服五日烏孫狂王復尚楚主解憂【復扶又翻】生一男鴟靡不與主和又暴惡失衆漢使衛司馬魏和意副任昌至烏孫【衛也為和意之副任音壬】公主言狂王為烏孫所患苦易誅也【易以䜴翻】遂謀置酒使士拔劒擊之劒旁下【師古曰不正下也】狂王傷上馬馳去其子細沈瘦會兵圍和意昌及公主於赤谷城【師古曰瘦音搜赤谷城烏孫國都去長安八千九百里】數月都護鄭吉發諸國兵救之乃解去漢遣中郎將張遵持醫藥治狂王賜金帛【治直之翻】因收和意昌係瑣【係瑣即今云鎻索也】從尉犁檻車至長安斬之初肥王翁歸靡胡婦子烏就屠狂王傷時驚與諸翖侯俱去居北山中【其山在烏孫之北也翖與翕同音許及翻】揚言母家匈奴兵來故衆歸之後遂襲殺狂王自立為昆彌是歲漢遣破羌將軍辛武賢將兵萬五千人至燉煌通渠積穀欲以討之【時立表穿渠於卑鞮侯井以西孟康曰大井六通渠也下流湧出在白龍堆東上山下燉徒門翻】初楚主侍者馮嫽【師古曰嫽音了嫽者慧也故以為名】能史書【史吏也史書猶言吏書也】習事【内習漢事外習西域諸國事也】嘗持漢節為公主使【使疏吏翻】城郭諸國敬信之號曰馮夫人為烏孫右大將妻【烏孫國官相大禄之下有左右大將二人蓋貴人也】右大將與烏就屠相愛都護鄭吉使馮夫人說烏就屠【說輪芮翻】以漢兵方出必見滅不如降烏就屠恐曰願得小號以自處【處昌呂翻】帝徵馮夫人自問狀【即此事與數詔問趙充國事參而觀之通鑑所紀一千三百餘年間明審之君一人而已】遣謁者竺次期門甘延夀為副送馮夫人馮夫人錦車持節【應劭曰錦車以錦衣車也】詔烏就屠詣長羅侯赤谷城立元貴靡為大昆彌【元貴靡肥王翁歸靡嫡長男楚主解憂所生也事見上卷神爵二年】烏就屠為小昆彌皆賜印綬破羌將軍不出塞還後烏就屠不盡歸翖侯人衆漢復遣長羅侯將三校屯赤谷因為分别人民地界【復扶又翻校戶教翻别彼列翻】大昆彌戶六萬餘小昆彌戶四萬餘然衆心皆附小昆彌【為漢以兩昆彌憂勞張本】<br />
<br />
  二年春正月立皇子嚻為定陶王【考異曰諸侯王表十月乙亥立今據宣紀】 詔赦天下减民筭三十【師古曰一筭減錢三十也漢律人出一筭筭百二十錢】 珠厓郡反夏四月遣護軍都尉張禄將兵擊之【百官表護軍都尉秦官武帝元狩四年屬大司馬】 杜延年以老病免五月己丑廷尉于定國為御史大夫 秋七月立皇子宇為東平王 冬十二月上行幸萯陽宫屬玉觀【應劭曰萯陽宮在鄠秦文王所起伏儼曰在扶風李斐曰萯音倍師古曰應說李音是也服䖍曰屬玉觀以玉飾因名焉在扶風李奇曰屬玉音鸑鷟其上有此烏因以為名晉灼曰屬玉水鳥似鵁鶄以名觀也師古曰晉說是也屬音之欲翻觀工玩翻】 是歲營平壮武侯趙充國薨【恩澤侯表營平侯食邑於濟南夫以趙充國之賢之功而班史列之恩澤侯者以其初封以定策功也如衛青霍去病本以破匈奴功封而班史亦列於恩澤侯以其由衛思后戚屬得進也班史書法猶有古史官典刑後之為史者不復知此矣】先是充國以老乞骸骨賜安車駟馬黄金罷就弟【先悉薦翻弟與第同】朝廷每有四夷大議常與參兵謀【師古曰與讀曰豫】問籌策焉匈奴呼韓邪單于欵五原塞【師古曰欵叩也按班志漢五原郡即秦九原郡】<br />
<br />
  【治稒陽别有五原縣宋白曰漢五原故城在今勝州榆林縣界】願奉國珍朝三年正月【師古曰欲於甘露三年正月行朝禮朝直遥翻】詔有司議其儀丞相御史曰聖王之制先京師而後諸夏先諸夏而後夷狄匈奴單于朝賀其禮儀宜如諸侯王位次在下【此議猶依傍成周盛時朝諸侯之制先後皆去聲】太子太傅蕭望之以為單于非正朔所加【言班歷所不及也】故稱敵國宜待以不臣之禮位在諸侯王上外夷稽首稱藩中國讓而不臣此則羈縻之誼謙亨之福也【望之此議取春秋傳王者不治夷狄之意馬絡曰羈牛靷曰縻言其在荒服待之若馬牛然取羈縻不絶而已師古曰易謙卦之辭曰謙亨天道下濟而光明地道卑而上行言謙之為德無所不通也亨火庚翻】書曰戎狄荒服【師古曰逸書也余謂此語或者伏生之書有之今國語猶載此言】言其來服荒忽亡常【亡古無字通】如使匈奴後嗣卒有鳥竄鼠伏闕於朝享【朝朝見也享供時享也享獻也古者諸侯見於天子必以所貢助祭於廟孝經所謂四海之内各以其職來祭者也卒讀曰粹師古曰子恤翻】不為畔臣【師古曰卒終也謂本以客禮待之若後不來非叛臣】萬世之長策也天子采之下詔曰匈奴單于稱北蕃朝正朔【謂朝明年正月朔也】朕之不德不能弘覆【覆敷救翻】其以客禮待之令單于位在諸侯王上贊謁稱臣而不名<br />
<br />
  荀悦論曰春秋之義王者無外欲一於天下也【春秋之義王者無外故天王有入無出大夫出不言奔欲一乎天下也】戎狄道路遼遠人迹介絶故正朔不及禮教不加非尊之也其埶然也詩云自彼氐羌莫敢不來王【商頌殷武之詩也】故要荒之君必奉王貢若不供職則有辭讓號令加焉【國語祭公謀父曰蠻夷要服戎狄荒服要服者貢荒服者王有不貢則修名有不王則修德於是讓不貢告不王於是有威讓之令有文告之辭要一遥翻】非敵國之謂也望之欲待以不臣之禮加之王公之上僭度失序以亂天常非禮也若以權時之宜則異論矣<br />
<br />
  詔遣車騎都尉韓昌迎單于發所過七郡二千騎為陳道上【按漢書郡下又有郡字師古注曰所過之郡每為發兵陳列於道以為寵衛也七郡謂過五原朔方西河上郡北地馮翊而後至長安也為于偽翻】<br />
<br />
  三年春正月上行幸甘泉郊泰畤 匈奴呼韓邪單于來朝贊謁稱藩臣而不名賜以冠帶衣裳黄金璽盭綬【白虎通衣者隱也裳者障也所以隱形自障蔽也璽斯氏翻綬音受師古曰盭古戾字戾草名也以戾染綬亦諸侯王之制也】玉具劔佩刀弓一張矢四發【孟康曰玉具劔摽首鐔衛盡用玉為之也師古曰鐔劒口旁横出者也衛劒鼻也鐔音淫衛字本作彘其音同耳服䖍曰發十二矢也韋昭曰射禮三而止每射四矢故以十二為一發也師古曰發猶今言箭一放兩放也今則以一矢為一放也】棨戟十【師古曰棨有衣之戟也棨音啟】安車一乘鞍勒一具【師古曰勒馬轡也】馬十五匹黄金二十斤錢二十萬衣被七十七襲【師古曰一稱為一襲猶今人之言一副衣服也】錦繡綺縠雜帛八千匹絮六千斤禮畢使使者道單于先行宿長平【師古曰道讀曰導導引也如淳曰長平阪名也在也陽南上原之阪有長平觀去長安五十里師古曰涇水之南原即今所謂睚城阪也】上自甘泉宿池陽宫【池陽縣屬左馮翊有離宫在馬賢曰池陽縣故城在今涇陽縣西北】上登長平阪詔單于毋謁【師古曰不令拜也】其左右當戶皆得列觀及諸蠻夷君長王侯數萬咸迎於渭橋下夾道陳【陳如字陳列也又塗也】上登渭橋咸稱萬歲單于就邸長安置酒建章宫饗賜單于觀以珍寶【師古曰觀示也觀古玩翻】二月遣單于歸國單于自請願留居幕南光禄塞下【師古曰徐自為所築者也余按武帝遣光禄徐自為出五原塞築亭障列城後人因謂之光禄塞】有急保漢受降城【恐郅支來攻故請有急入城自保】漢遣長樂衛尉高昌侯董忠【功臣表高昌侯食邑於千乘樂音洛】車騎都尉韓昌將騎萬六千又發邊郡士馬以千數送單于出朔方鷄鹿塞【師古曰雞鹿塞在朔方窳渾縣之西北】詔忠等留衛單于助誅不服又轉邊穀米糒【師古曰糒乾飯也音備】前後三萬四千斛給贍其食先是自烏孫以西至安息諸國近匈奴者皆畏匈奴而輕漢【先悉薦翻近其靳翻】及呼韓邪朝漢後咸尊漢矣上以戎狄賓服思股肱之美乃圖畫其人於麒麟閣【麒麟閣在未央宮中張晏曰武帝獲麒麟時作此閣圖盡其像於閣遂以為名師古曰漢宮閣疏云蕭何造畫古畫字通】法其容貌署其官爵姓名【師古曰署表也題也】唯霍光不名曰大司馬大將軍博陸侯姓霍氏其次張安世韓增趙充國魏相丙吉杜延年劉德梁丘賀蕭望之蘇武凡十一人【圖畫功臣自此始觀麟閣股肱之次魏丙列於霍張韓趙之下則知漢之丞相在中朝諸將軍之後矣梁丘姓也左傳齊有梁丘據】皆有功德知名當世是以表而揚之明著中興輔佐列於方叔召虎仲山甫焉【師古曰三人皆周宣王之臣有文武之功佐宣王中興者也言宣帝亦重興漢室而霍光等並為名臣皆比於方叔之屬召讀曰邵】 鳳皇集新蔡【新蔡縣屬汝南郡春秋蔡平侯自蔡徙此因名】 三月己巳建成安侯黄霸薨【恩澤侯表建成侯食邑於沛】 五月甲午于定國為丞相封西平侯【恩澤侯表西平侯食邑於臨淮】太僕沛郡陳萬年為御史大夫 詔諸儒講五經同異蕭望之等平奏其議上親稱制臨决焉乃立梁丘易大小夏侯尚書穀梁春秋博士【梁丘賀大夏侯勝小夏侯建穀梁赤】 烏孫大昆彌元貴靡及鴟靡皆病死公主上書言年老土思【土思者懷故鄉也】願得歸骸骨葬漢地天子閔而迎之冬至京師待之一如公主之制【楚主本以宗室女嫁烏孫今待之如公主之制儀比皇女】後二歲卒元貴靡子星靡代為大昆彌弱【師古曰言其尚幼小】馮夫人上書願使烏孫鎮撫星靡漢遣之【使疏吏翻】都護奏烏孫大吏大禄大監皆可賜以金印紫綬以尊輔大昆彌【初烏孫王昆莫中子大禄彊善將摠萬餘騎後遂以為官名又其國官有大監二人漢列侯金印紫綬今特賜之】漢許之其後段會宗為都護乃招還亡叛安定之星靡死子雌栗靡代立 皇太子所幸司馬良娣病且死謂太子曰妾死非天命乃諸娣妾良人更祝詛殺我【漢嬪御之秩良人視八百石爵比左庶長師古曰更音工衡翻祝職救翻詛莊助翻】太子以為然及死太子悲恚發病【恚於避翻】忽忽不樂【樂音洛】帝乃令皇后擇後宫家人子可以娯侍太子者得元城王政君【元城縣屬魏郡應劭曰魏武侯公子元食邑於此因而遂氏焉】送太子宮政君故繡衣御史賀之孫女也【王賀事見二十一卷武帝天漢二年】見於丙殿【殿蓋以甲乙丙丁為次因名見賢遍翻】壹幸有身是歲生成帝於甲館畫堂【應劭曰甲觀在太子宮甲地主用乳生也畫堂畫九子母如淳曰畫堂堂名甲觀觀名三輔黄圖云太子宮有甲觀師古曰甲者甲乙丙丁之次也元后傳見於丙殿此其例也而應氏以為在宫之甲地謬矣畫堂但畫飾耳豈必九子母乎霍光止畫 室中是則宫殿中通冇畫彩之飾】為世適皇孫帝愛之自名曰驁字大孫常置左右【適讀曰嫡嫡正出也曰世適者謂正統繼世之重也政君之入太子宮亦姬侍耳以子貴遂為正妃鷔五到翻大讀曰太為王氏竊漢張本】<br />
<br />
  四年夏廣川王海陽坐禽獸行賊殺不辜廢徙房陵【地節四年立廣川王文海陽文之子也内亂為禽獸行行下孟翻 考異曰諸侯表作汝陽宣紀景十三王傳作海陽今從之】 冬十月未央宮宣室閣火 是歲徙定陶王嚻為楚王 匈奴呼韓邪郅支兩單于俱遣使朝獻漢待呼韓邪使有加焉<br />
<br />
  黄龍元年春正月上行幸甘泉郊泰畤 匈奴呼韓邪單于來朝二月歸國始郅支單于以為呼韓邪兵弱降漢不能復自還【復扶又翻】即引其衆西欲攻定右地又屠耆單于小弟本侍呼韓邪亦亡之右地收兩兄餘兵【兩兄屠耆閏振也】得數千人自立為伊利目單于【目漢書作自】道逢郅支合戰郅支殺之并其兵五萬餘人郅支聞漢出兵穀助呼韓邪即遂留居右地自度力不能定匈奴【度徒洛翻】乃益西近烏孫【近其靳翻】欲其并力遣使見小昆彌烏就屠烏就屠殺其使發八千騎迎郅支郅支覺其謀勒兵逢擊烏孫破之【師古曰以兵逆之相逢即擊故云逢擊】因北擊烏掲堅昆丁令并三國【掲音丘例翻】數遣兵擊烏孫常勝之【數所角翻】堅昆東去單于庭七千里南至車師五千里郅支留都之 三月有星孛于王良閣道【營室曰離宮閣道漢中四星曰天駟旁一星曰王良又紫宮後十七星絶漢抵營室曰閣道孛蒲内翻】入紫微 帝寑疾選大臣可屬者【屬之欲翻】引外屬侍中樂陵侯史高【屬讀如本字外屬猶言外戚也】太子太傅蕭望之少傅周堪至禁中拜高為大司馬車騎將軍望之為前將軍光禄勲堪為光禄大夫皆受遺詔輔政領尚書事【漢尚書職典樞機凡諸曹文書衆事皆由之自是之後凡受遺輔政皆領尚書事至東都曰録尚書事】冬十二月甲戌帝崩于未央宫【臣瓚曰帝年十八即位即位二十五年夀四十三】<br />
<br />
  班固贊曰孝宣之治信賞必罰【師古曰有功必賞有罪必罰治直吏翻】綜核名實政事文學法理之士咸精其能至於技巧工匠器械自元成閒鮮能及之【師古曰械者器之摠名也一曰有盛為械無盛為器鮮少也言少有能及之者枝渠綺翻鮮息淺翻】亦足以知吏稱其職民安其業也【稱尺證翻】遭值匈奴乖亂推亡固存【李奇曰推亡者若紂為無道天下苦之有滅亡之形周武遂推而弊之固存譬如鄰國以道蒞民上下一心勢必能存因就而堅固之今匈奴内自紛争宣帝能朝呼韓邪而固存之走郅支使遠遁是謂推亡也師古曰尚書仲虺之誥曰推亡固存邦乃其昌言有亡道者則推而滅之有存道者則輔而固之王者如此國乃昌盛故此贊引之推吐雷翻】信威北夷【師古曰信讀曰申古字通用一說恩信及威並著北夷余謂前意是】單于慕義稽首稱藩功光祖宗業垂後嗣可謂中興侔德殷宗周宣矣【師古曰侔等殷之高宗及周之宣王也】<br />
<br />
  癸巳太子即皇帝位謁高廟尊皇太后曰太皇太后【蘇林曰上官后】皇后曰皇太后<br />
<br />
  資治通鑑卷二十七<br />
<br />
<史部,編年類,資治通鑑>  <br>
   </div> 

<script src="/search/ajaxskft.js"> </script>
 <div class="clear"></div>
<br>
<br>
 <!-- a.d-->

 <!--
<div class="info_share">
</div> 
-->
 <!--info_share--></div>   <!-- end info_content-->
  </div> <!-- end l-->

<div class="r">   <!--r-->



<div class="sidebar"  style="margin-bottom:2px;">

 
<div class="sidebar_title">工具类大全</div>
<div class="sidebar_info">
<strong><a href="http://www.guoxuedashi.com/lsditu/" target="_blank">历史地图</a></strong>  
<a href="http://www.880114.com/" target="_blank">英语宝典</a>  
<a href="http://www.guoxuedashi.com/13jing/" target="_blank">十三经检索</a> 
<br><strong><a href="http://www.guoxuedashi.com/gjtsjc/" target="_blank">古今图书集成</a></strong> 
<a href="http://www.guoxuedashi.com/duilian/" target="_blank">对联大全</a> <strong><a href="http://www.guoxuedashi.com/xiangxingzi/" target="_blank">象形文字典</a></strong> 

<br><a href="http://www.guoxuedashi.com/zixing/yanbian/">字形演变</a>  <strong><a href="http://www.guoxuemi.com/hafo/" target="_blank">哈佛燕京中文善本特藏</a></strong>
<br><strong><a href="http://www.guoxuedashi.com/csfz/" target="_blank">丛书&方志检索器</a></strong> <a href="http://www.guoxuedashi.com/yqjyy/" target="_blank">一切经音义</a>  

<br><strong><a href="http://www.guoxuedashi.com/jiapu/" target="_blank">家谱族谱查询</a></strong>  <strong><a href="http://shufa.guoxuedashi.com/sfzitie/" target="_blank">书法字帖欣赏</a></strong> 
<br>

</div>
</div>


<div class="sidebar" style="margin-bottom:0px;">

<font style="font-size:22px;line-height:32px">QQ交流群9:489193090</font>


<div class="sidebar_title">手机APP 扫描或点击</div>
<div class="sidebar_info">
<table>
<tr>
	<td width=160><a href="http://m.guoxuedashi.com/app/" target="_blank"><img src="/img/gxds-sj.png" width="140"  border="0" alt="国学大师手机版"></a></td>
	<td>
<a href="http://www.guoxuedashi.com/download/" target="_blank">app软件下载专区</a><br>
<a href="http://www.guoxuedashi.com/download/gxds.php" target="_blank">《国学大师》下载</a><br>
<a href="http://www.guoxuedashi.com/download/kxzd.php" target="_blank">《汉字宝典》下载</a><br>
<a href="http://www.guoxuedashi.com/download/scqbd.php" target="_blank">《诗词曲宝典》下载</a><br>
<a href="http://www.guoxuedashi.com/SiKuQuanShu/skqs.php" target="_blank">《四库全书》下载</a><br>
</td>
</tr>
</table>

</div>
</div>


<div class="sidebar2">
<center>


</center>
</div>

<div class="sidebar"  style="margin-bottom:2px;">
<div class="sidebar_title">网站使用教程</div>
<div class="sidebar_info">
<a href="http://www.guoxuedashi.com/help/gjsearch.php" target="_blank">如何在国学大师网下载古籍?</a><br>
<a href="http://www.guoxuedashi.com/zidian/bujian/bjjc.php" target="_blank">如何使用部件查字法快速查字?</a><br>
<a href="http://www.guoxuedashi.com/search/sjc.php" target="_blank">如何在指定的书籍中全文检索?</a><br>
<a href="http://www.guoxuedashi.com/search/skjc.php" target="_blank">如何找到一句话在《四库全书》哪一页?</a><br>
</div>
</div>


<div class="sidebar">
<div class="sidebar_title">热门书籍</div>
<div class="sidebar_info">
<a href="/so.php?sokey=%E8%B5%84%E6%B2%BB%E9%80%9A%E9%89%B4&kt=1">资治通鉴</a> <a href="/24shi/"><strong>二十四史</strong></a>&nbsp; <a href="/a2694/">野史</a>&nbsp; <a href="/SiKuQuanShu/"><strong>四库全书</strong></a>&nbsp;<a href="http://www.guoxuedashi.com/SiKuQuanShu/fanti/">繁体</a>
<br><a href="/so.php?sokey=%E7%BA%A2%E6%A5%BC%E6%A2%A6&kt=1">红楼梦</a> <a href="/a/1858x/">三国演义</a> <a href="/a/1038k/">水浒传</a> <a href="/a/1046t/">西游记</a> <a href="/a/1914o/">封神演义</a>
<br>
<a href="http://www.guoxuedashi.com/so.php?sokeygx=%E4%B8%87%E6%9C%89%E6%96%87%E5%BA%93&submit=&kt=1">万有文库</a> <a href="/a/780t/">古文观止</a> <a href="/a/1024l/">文心雕龙</a> <a href="/a/1704n/">全唐诗</a> <a href="/a/1705h/">全宋词</a>
<br><a href="http://www.guoxuedashi.com/so.php?sokeygx=%E7%99%BE%E8%A1%B2%E6%9C%AC%E4%BA%8C%E5%8D%81%E5%9B%9B%E5%8F%B2&submit=&kt=1"><strong>百衲本二十四史</strong></a>  <a href="http://www.guoxuedashi.com/so.php?sokeygx=%E5%8F%A4%E4%BB%8A%E5%9B%BE%E4%B9%A6%E9%9B%86%E6%88%90&submit=&kt=1"><strong>古今图书集成</strong></a>
<br>

<a href="http://www.guoxuedashi.com/so.php?sokeygx=%E4%B8%9B%E4%B9%A6%E9%9B%86%E6%88%90&submit=&kt=1">丛书集成</a> 
<a href="http://www.guoxuedashi.com/so.php?sokeygx=%E5%9B%9B%E9%83%A8%E4%B8%9B%E5%88%8A&submit=&kt=1"><strong>四部丛刊</strong></a>  
<a href="http://www.guoxuedashi.com/so.php?sokeygx=%E8%AF%B4%E6%96%87%E8%A7%A3%E5%AD%97&submit=&kt=1">說文解字</a> <a href="http://www.guoxuedashi.com/so.php?sokeygx=%E5%85%A8%E4%B8%8A%E5%8F%A4&submit=&kt=1">三国六朝文</a>
<br><a href="http://www.guoxuedashi.com/so.php?sokeytm=%E6%97%A5%E6%9C%AC%E5%86%85%E9%98%81%E6%96%87%E5%BA%93&submit=&kt=1"><strong>日本内阁文库</strong></a> <a href="http://www.guoxuedashi.com/so.php?sokeytm=%E5%9B%BD%E5%9B%BE%E6%96%B9%E5%BF%97%E5%90%88%E9%9B%86&ka=100&submit=">国图方志合集</a> <a href="http://www.guoxuedashi.com/so.php?sokeytm=%E5%90%84%E5%9C%B0%E6%96%B9%E5%BF%97&submit=&kt=1"><strong>各地方志</strong></a>

</div>
</div>


<div class="sidebar2">
<center>

</center>
</div>
<div class="sidebar greenbar">
<div class="sidebar_title green">四库全书</div>
<div class="sidebar_info">

《四库全书》是中国古代最大的丛书,编撰于乾隆年间,由纪昀等360多位高官、学者编撰,3800多人抄写,费时十三年编成。丛书分经、史、子、集四部,故名四库。共有3500多种书,7.9万卷,3.6万册,约8亿字,基本上囊括了古代所有图书,故称“全书”。<a href="http://www.guoxuedashi.com/SiKuQuanShu/">详细>>
</a>

</div> 
</div>

</div>  <!--end r-->

</div>
<!-- 内容区END --> 

<!-- 页脚开始 -->
<div class="shh">

</div>

<div class="w1180" style="margin-top:8px;">
<center><script src="http://www.guoxuedashi.com/img/plus.php?id=3"></script></center>
</div>
<div class="w1180 foot">
<a href="/b/thanks.php">特别致谢</a> | <a href="javascript:window.external.AddFavorite(document.location.href,document.title);">收藏本站</a> | <a href="#">欢迎投稿</a> | <a href="http://www.guoxuedashi.com/forum/">意见建议</a> | <a href="http://www.guoxuemi.com/">国学迷</a> | <a href="http://www.shuowen.net/">说文网</a><script language="javascript" type="text/javascript" src="https://js.users.51.la/17753172.js"></script><br />
  Copyright &copy; 国学大师 古典图书集成 All Rights Reserved.<br>
  
  <span style="font-size:14px">免责声明:本站非营利性站点,以方便网友为主,仅供学习研究。<br>内容由热心网友提供和网上收集,不保留版权。若侵犯了您的权益,来信即刪。scp168@qq.com</span>
  <br />
ICP证:<a href="http://www.beian.miit.gov.cn/" target="_blank">鲁ICP备19060063号</a></div>
<!-- 页脚END --> 
<script src="http://www.guoxuedashi.com/img/plus.php?id=22"></script>
<script src="http://www.guoxuedashi.com/img/tongji.js"></script>

</body>
</html>
