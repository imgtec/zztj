










 


 
 


 

  
  
  
  
  





  
  
  
  
  
 
  

  

  
  
  



  

 
 

  
   




  

  
  


    資治通鑑卷一百四十七 宋 司馬光 撰

  胡三省 音註

  梁紀三【起著雍困敦盡閼逢敦牂凡七年}


  高祖武皇帝三

  天監七年春正月魏潁川太守王神念來犇【為後神念子僧辯有功於興復張本守式又翻}
 壬子以衛尉吳平侯昺兼領軍將軍詔吏部尚書徐勉定百官九品為十八班以班多者為貴二月乙丑增置鎮衛將軍以下為十品凡二十四班不登十品别有八班又置施外國將軍二十四班凡一百九號【丞相太宰太傳太保大司馬大將軍太尉司徒司空為十八班諸將軍開府儀同三司左右光禄開府儀同三司為十七班尚書令太子太傅左右光祿大夫為十六班尚書左僕射太子少傅尚書僕射右僕射中書監特進領護軍將軍為十五班中領護軍吏部尚書太子詹事金紫光禄大夫太常卿為十四班中書令列曹尚書國子祭酒宗正太府卿光禄大夫為十三班侍中散騎常侍左右衛將軍司徒左長史衛尉卿為十二班御史中丞尚書吏部郎秘書監通直散騎常侍太子左右二衛率左右驍游大中大夫皇弟皇子師司農少府廷尉卿太子中庶子光禄卿為十一班給事黄門侍郎員外散騎常侍皇弟皇子府長史太僕大匠卿太子家令率更令僕揚州别駕中散大夫司徒右長史雲騎游騎皇弟皇子府司馬朱衣直閤將軍為十班尚書左丞鴻臚卿中書侍郎國子博士太子庶子揚州中從事皇弟皇子公府從事中郎太舟卿大長秋皇弟皇子府諮議嗣王府長史前左右後四軍嗣王府司馬庶姓公府長史司馬為九班秘書丞太子中舍人司徒左西掾司徒屬皇弟皇子友散騎侍郎尚書右丞南徐州别駕皇弟皇子公府掾屬皇弟皇子單為二衛司馬嗣王庶姓公府從事中郎左右中郎將嗣王庶姓公府諮議皇弟皇子之庶子府長史司馬蕃王府長史司馬庶姓持節府長史司馬為八班五校東宫三校皇弟皇子之庶子府中録事中記室中直兵參軍南徐州中從事皇弟皇子之庶子府蕃王府諮議為七班太子洗馬通直散騎侍郎司徒主簿尚書侍郎著作郎皇弟皇子府功曹史五經博士皇弟皇子府録事記室中兵參軍皇弟皇子荆江雍郢南徐五州别駕領護軍長史司馬嗣王庶姓公府掾屬南臺治書侍御史廷尉三官謁者僕射太子門大夫嗣王庶姓公府中録事中記室中直兵參軍庶姓府諮議為六班尚書郎中皇弟皇子文學及府主簿太子太傅少傅丞皇弟皇子湘豫司益廣青衡七州别駕皇弟皇子荆江雍郢南兖五州中從事嗣王庶姓荆江等五州别駕太常丞皇弟皇子國郎中令三將東宫二將嗣王府功曹史庶姓公府録事記室中兵參軍皇弟皇子之庶子府蕃王府中録事中記室中直兵參軍為五班給事中皇弟皇子府正參軍中書舍人建康三官皇弟皇子北徐北兖梁交南梁五州别駕皇弟皇子湘豫司益廣青衡七州别駕中從事嗣王庶姓湘豫等七州别駕嗣王庶姓荆江等五州中從事宗正太府衛尉司農少府廷尉太子詹事等丞積射彊弩將軍太子左右積弩將軍皇弟皇子國大農嗣王國郎中令嗣王庶姓公府主簿皇弟皇子之庶子府蕃王府功曹史録事記室中兵參軍為四班太子舍人司徒祭酒皇弟皇子公府祭酒員外散騎侍郎皇弟皇子府行參軍太子太傅少傅五官功曹主簿二衛司馬公車令胄子律博士皇弟皇子越桂寧霍四州别駕皇弟皇子北徐北兖梁交南梁五州中從事庶姓北徐等五州别駕湘豫司益廣青衡七州中從事嗣王庶姓公府正參軍皇弟皇子之庶子府蕃王府曹主簿武衛將軍光禄丞皇弟皇子國中尉太僕大匠丞嗣王國大農蕃王國郎中令庶姓持節府中錄事中記室中直兵參軍北館令為三班秘書郎著作佐郎楊南徐州主簿嗣王庶姓公府祭酒皇弟皇子單為領護詹事二衛等功曹五官主簿太學博士皇弟皇子國常侍奉朝請國子助教皇弟皇子越桂寧霍四州中從事皇弟皇子荆江等五州主簿嗣王庶姓越桂等四州别駕嗣王庶姓北徐等五州中從事鴻臚丞尚書五都令史武騎常侍材官將軍明堂二廟帝陵令嗣王庶姓公府行參軍皇弟皇子之庶子府正參軍蕃王國大農庶姓持節府録事記室中直兵參軍庶姓持節府功曹史為二班揚南徐州西曹祭酒從事皇弟皇子國侍郎嗣王國常侍南徐州議曹從事東宫通事舍人南臺侍御史大舟丞二衛殿中將軍皇弟皇子之庶子府蕃王府行參軍蕃王國中尉皇弟皇子湘豫等七州主簿皇弟皇子荆雍等州西曹祭酒議曹從事皇弟皇子西曹從事祭酒議曹祭酒部傳從事嗣王庶姓越桂等四州中從事嗣王庶姓荆江等五州主簿庶姓持節府主簿汝隂巴陵二國郎中令太官太樂太市太史太醫太祝東西治左右尚方南北武庫車府等令為一班又詔以將軍之名高卑舛雜更加釐定於是有司奏置一百二十五號將軍以鎮衛驃騎車騎為二十四班内外通用征東征西征南征北施外中軍中衛中撫中護施内為二十三班鎮東鎮西鎮南鎮北施外鎮左鎮右鎮前鎮後施内為二十二班安東安西安南安北施外安左安右安前安後施内為二十一班平東平西平南平北施外翊左翊右翊前翊後施内為二十班是為重號將軍忠武軍師為十九班武臣爪牙龍騎雲麾為十八班代舊前後左右四將軍鎮兵翊師宣惠宣毅為十七班代舊四中郎將十號為一品智威仁威勇威信威嚴威為十六班代舊征虜智武仁武勇武信武嚴武為十五班代舊冠軍十號為一品所謂五德將軍者也輕車征遠鎮朔武旅貞毅為十四班代舊輔國凡將軍加大者唯至貞毅而已通進一階優者方得比位從公寧遠明威振遠電耀威耀為十三班代舊寧朔十號為一品武威武騎武猛壯武飊武為十二班電威馳鋭追鋒羽騎突騎為十一班十號為一品折衝冠武和戎安壘猛烈為十班掃狄雄信掃虜武鋭椎鋒為九班十號為一品畧遠貞威决勝開遠光野為八班厲鋒輕鋭討狄蕩虜蕩夷為七班十號為一品武毅鐵騎樓船宣猛樹功為六班克狄平虜討夷平狄威戎為五班十號為一品伏波雄戟長劍衝冠雕騎為四班佽飛安夷克戎綏狄威虜為三班十號為一品前鋒武義開邊招遠全威為二班綏虜蕩寇横野馳射為一班十號為一品凡十品二十四班亦以班多為貴其制品十取其盈數以法十日班二十四以法氣序制簿悉以大號居後以為選法自小遷大也其不登十品應須軍號者有牙門代舊建威期門代舊建武為八班侯騎代舊振威熊渠代舊振武為七班中堅代舊奮威典戎代舊奮武為六班戈船代舊揚威繡衣代舊揚武為五班執訊代舊廣威行陣代舊廣武為四班鷹揚為三班陵江為二班偏將軍禆將軍為一班凡十四號别為八班以象八風所施甚輕又有武安鎮遠雄義擬車騎為二十四班撫東撫西撫南撫北擬四征為二十三班寧東寧西寧南寧北擬四鎮為二十二班威東威西威南威北擬四安為二十一班綏東綏西綏南綏北擬四平為二十班凡十九號為一品安遠安邊擬忠武軍師為十九班輔義安沙衛海撫河擬武臣等四號為十八班平遠撫朔寜沙航海擬鎮兵等四號為十七班凡十號為一品翊海朔野拓遠威河龍幕擬智威等五號為十六班威隴安漠綏邊寜寇梯山擬智武等五號為十五班凡十號為一品寧境綏河明信明義威漠擬輕車等五號為十四班安隴向義宣節振朔候律擬寧遠等五號為十三班凡十號為一品平寇定遠寧隴陵海振漠擬武威等五號為十二班馳義横朔明節執信懷德擬電威等五號為十一班凡十號為一品撫邊定隴綏關立信奉義擬折衝等五號為十班綏隴寧邊定朔立節懷威擬掃狄等五號為九班凡十號為一品懷關静朔掃寇寧河安朔擬畧遠等五號為八班揚化超隴執義來化度嶂擬厲鋒等五號為七班凡十號為一品平河振隴雄邊横沙寧關擬武毅等五號為六班懷信宣義弘節浮遼鑿空擬克狄等五號為五班凡十號為一品扞海欵塞歸義陵河明信擬伏波等五號為四班奉忠守義弘信仰化立義擬佽飛等五號為三班凡十號為一品綏方奉正承化浮海渡河擬先鋒等五號為二班懷義奉信歸誠懷澤伏義擬綏虜等五號為一品凡十號為一班大凡一百九號亦為二十四班施於外國}
庚午詔置州望郡宗鄉豪各一人專掌搜薦【搜求也搜才能}


  【而薦之於上}
 乙亥以南兖州刺史呂僧珍為領軍將軍領軍掌内外兵要宋孝建以來制局用事與領軍分兵權典事以上皆得呈奏領軍拱手而已及吳平侯昺在職峻切官曹肅然制局監皆近倖頗不堪命以是不得久留中丙子出為雍州刺史【先用僧珍次日出昺昺音丙雍於用翻}
 三月戊子魏皇子昌卒侍御師王顯失於療治【醫師侍御左右因以名官後魏之制太醫令屬太常掌醫藥而門下省别有尚藥局侍御師蓋今之御醫也此又一王顯非御史中尉之王顯也治直之翻}
時人皆以為承高肇之意也 夏四月乙卯皇太子納妃大赦 五月己亥詔復置宗正太僕大匠鴻臚又增太府太舟仍先為十二卿【復扶又翻五代史志曰是年以太常為太常卿加置宗正卿以大司農為司農卿是為春卿加置太府卿以少府為少府卿加置太僕卿是為夏卿以衛尉為衛尉卿廷尉為廷尉卿將作大匠為大匠卿是為秋卿以光禄勲為光禄卿大鴻臚為鴻臚卿都水使者為大舟卿是為冬卿凡十二卿皆置丞及功曹主簿}
 癸卯以安成王秀為荆州刺史先是巴陵馬營蠻緣江為寇州郡不能討【先悉薦翻}
秀遣防閤文熾帥衆燔其林木【梁制上宫東宫置直閤王公置防閤文姓也帥讀曰率}
蠻失其險州境無寇【蠻無所依阻故不敢為寇}
 秋七月甲午魏立高貴嬪為皇后【嬪毗賓翻}
尚書令高肇益貴重用事肇多變更先朝舊制減削封秩抑黜勲人由是怨聲盈路羣臣宗室皆卑下之【更工衡翻朝直遥翻下遐嫁翻}
唯度支尚書元匡與肇抗衡先是造棺置聽事欲輿棺詣闕論肇罪惡自殺以切諫肇聞而惡之【度徒洛翻惡烏路翻}
會匡與太常劉芳議權量事【魏因議樂併議定權量量力讓翻}
肇主芳議匡遂與肇喧競【競爭也}
表肇指鹿為馬【以肇比趙高}
御史中尉王顯奏彈匡誣毁宰相有司處匡死刑【處昌呂翻}
詔恕死降為光禄大夫 八月癸丑竟陵壯公曹景宗卒 初魏主為京兆王愉納于后之妹為妃【為于偽翻}
愉不愛愛妾李氏生子寶月于后召李氏入宫捶之【捶止橤翻}
愉驕奢貪縱所為多不法帝召愉入禁中推案杖愉五十出為冀州刺史愉自以年長【長知兩翻}
而勢位不及二弟【二弟清河王懌廣平王懷}
潛懷愧恨又身與妾屢被頓辱高肇數譖愉兄弟愉不勝忿【被皮義翻數所角翻勝音升}
癸亥殺長史羊靈引司馬李遵詐稱得清河王懌密疏云高肇弑逆遂為壇於信都之南【魏冀州刺史治信都}
即皇帝位大赦改元建平立李氏為皇后法曹參軍崔伯驥不從愉殺之在北州鎮皆疑魏朝有變【謂州鎮在冀州之北者朝直遥翻}
定州刺史安樂王詮具以狀告之【樂音洛詮且緣翻}
州鎮乃安乙丑魏以尚書李平為都督北討諸軍行冀州事以討愉平崇之從父弟也【從才用翻}
 丁卯魏大赦改元永平魏京兆王愉遣使說平原太守清河房亮亮斬其使愉遣其將張靈和擊之為亮所敗【說式芮翻將即亮翻敗補邁翻}
李平軍至經縣【經縣漢晉屬安平國魏收志屬鉅鹿郡}
諸軍大集夜有蠻兵數千斫平營矢及平帳平堅卧不動俄而自定【蠻兵蓋亦李平所統欲為内變而平不動故自定}
九月辛巳朔愉逆戰於城南草橋平奮擊大破之愉脱身走入城平進圍之壬辰安樂王詮破愉兵於城北 癸巳立皇子績為南康王 魏高后之立也彭城武宣王勰固諫【勰音協}
魏主不聽高肇由是怨之數譖勰於魏主【數所角翻}
魏主不之信勰薦其舅潘僧固為長樂太守京兆王愉之反脅僧固與之同【冀州與長樂郡同治信都故僧固為愉所脅樂音洛}
肇因誣勰北與愉通南招蠻賊【伊闕以南接于淮汝江沔皆有蠻左}
彭城郎中令魏偃前防閤高祖珍【高祖珍前嘗為勰防閤時已去官故曰前防閤}
希肇提擢構成其事肇令侍中元暉以聞暉不從又令左衛元珍言之帝以問暉暉明勰不然又以問肇肇引魏偃高祖珍為證帝乃信之戊戌召勰及高陽王雍廣陽王嘉清河王懌廣平王懷高肇俱入宴勰妃李氏方產固辭不赴中使相繼召之【使疏吏翻}
不得已與妃訣而登車入東掖門度小橋牛不肯進擊之良久【良久稍久也或曰甚久也}
更有使者責勰來遲乃去牛【去羌呂翻}
人挽而進宴於禁中至夜皆醉各就别所消息【令各就便安之處消酒毒而息真氣}
俄而元珍引武士齎毒酒而至勰曰吾無罪願一見至尊死無恨元珍曰至尊何可復見【復扶又翻}
勰曰至尊聖明不應無事殺我乞與告者一對曲直武士以刀鐶築之勰大言曰寃哉皇天忠而見殺武士又築之勰乃飲毒酒武士就殺之向晨以褥裹尸載歸其第云王因醉而薨李氏號哭大言曰【號戶刀翻}
高肇枉理殺人天道有靈汝安得良死魏主舉哀於東堂贈官葬禮皆優厚加等在朝貴賤莫不喪氣【朝直遥翻喪息浪翻}
行路士女皆流涕曰高令公枉殺賢王【肇為尚書令故稱為令公}
由是中外惡之益甚【為高肇被誅張本惡烏路翻}
京兆王愉不能守信都癸卯燒門攜李氏及其四子從百餘騎突走【騎奇寄翻}
李平入信都斬愉所置冀州牧韋超等遣統軍叔孫頭追執愉置信都以聞羣臣請誅愉魏主不許命鎖送洛陽申以家人之訓【愉魏主弟也故欲訓責之}
行至野王高肇密使人殺之 【考異曰魏書及北史愉傳皆云愉每宿止亭傳必攜李手盡其私情雖鏁縶之中飲賞自若畧無愧懼之色至野王愉語人曰雖主上慈深不忍殺我吾亦何面見至尊於是歔欷流涕絶氣而死或云高肇令人殺之按愉既敗被執猶畧無愧懼安能慙見魏主遽感激絶氣而死蓋肇潛使人殺愉因以此言紿魏主耳}
諸子至洛魏主皆赦之魏主將屠李氏中書令崔光諫曰李氏方姙刑至刳胎乃桀紂所為【武王數紂之罪曰刳剔孕婦}
酷而非法請俟產畢然後行刑從之李平捕愉餘黨千餘人將盡殺之録事參軍高顥曰此皆脅從前既許之原免矣宜為表陳【為于偽翻下為國同}
平從之皆得免死顥祐之孫也【高祐允之從祖弟以文學事魏孝文}
濟州刺史高植帥州軍擊愉有功當封【濟子禮翻帥讀曰率}
植不受曰家荷重恩為國致效【致效言致身而效死也荷下可翻}
乃其常節何敢求賞植肇之子也加李平散騎常侍【敢悉亶翻騎奇寄翻}
高肇及中尉王顯素惡平顯彈平在冀州隱截官口【此謂叛黨男女合沒為官口者惡烏路翻彈徒丹翻}
肇奏除平名【除名不得通籍禁門}
初顯祖之世柔然萬餘口降魏置之高平薄骨律二鎮【魏世祖太延二年置高平鎮是後肅宗正光五年改置原州又大延二年置薄骨律鎮肅宗孝昌中改置靈州宋白曰太和十年改薄骨律鎮為沃野鎮降戶江翻}
及太和之末叛走略盡唯千餘戶在太中大夫王通請徙置淮北以絶其叛詔太僕卿楊椿持節往徙之椿上言先朝處之邊徼所以招附殊俗且别異華戎也【朝直遥翻處昌呂翻下河處同徼吉弔翻别彼列翻}
今新附之戶甚衆若舊者見徙新者必不自安是驅之使叛也且此屬衣毛食肉樂冬便寒【衣於既翻樂音洛}
南土濕熱往必殱盡進失歸附之心退無藩衛之益置之中夏【殱息廉翻夏戶雅翻}
或生後患非良策也不從遂徙於濟州緣河處之及京兆王愉之亂皆浮河赴愉所在抄掠如椿之言【濟子禮翻抄楚交翻}
 庚子魏郢州司馬彭珍等叛魏潛引梁兵趨義陽三關戍主侯登等以城來降郢州刺史婁悦嬰城自守魏以中山王英都督南征諸軍事將步騎三萬出汝南以救之【趨七喻翻降戶江翻將即亮翻騎奇寄翻 考異曰田益宗傳詔曰英統馬步七萬邢巒統精騎三萬蓋虛聲耳今從魏帝紀}
冬十月魏懸瓠軍主白早生殺豫州刺史司馬悦【瓠戶}


  【故翻 考異曰梁帝紀作白早生馬仙琕傳作琅邪王司馬慶曾今皆從魏書}
自號平北將軍求救於司州馬仙琕【琕部田翻}
時荆州刺史安成王秀為都督【秀以荆州刺史督諸州司州其所統也}
仙琕籖求應赴【籖前叙求應赴之事注見一百二十卷宋文帝元嘉元年}
參佐咸謂宜待臺報【謂宜奏上天臺而待報江左率謂朝廷為臺亦謂之天臺}
秀曰彼待我以自存【彼謂白早生}
援之宜速待敕雖舊【謂舊制須待臺敕}
非應急也即遣兵赴之上亦詔仙琕救早生仙琕進頓楚王城【楚王城即楚王戌}
遣副將齊苟兒【將即亮翻 考異曰魏書作苟仁今從梁書南北史}
以兵二千助守懸瓠詔以早生為司州刺史 【考異曰梁帝紀十月丙子魏陽關主許敬珍以城内附詔大舉北伐以始興王憺帥衆入清王茂帥衆向宿豫丁丑白早生與豫州刺史胡遜以城内屬以早生為司州胡遜為豫州刺史明年正月壬辰魏鎮東參軍成景雋斬宿豫城主嚴仲寶以城内屬二月丁卯魏楚王城主李國興以城内附姓名年月事迹既與魏書參差又徧檢諸列傳皆無其事今並從魏書}
 丙寅以吳興太守張稷為尚書左僕射【守式又翻}
 魏以尚書邢巒行豫州事將兵擊白早生魏主問之曰卿言早生走也守也何時可平對曰早生非有深謀大智正以司馬悦暴虐乘衆怒而作亂民迫於凶威不得已而從之縱使梁兵入城水路不通糧運不繼亦成禽耳早生得梁之援溺於利欲必守而不走若臨以王師士民必翻然歸順不出今年當傳首京師魏主悦命巒先使中山王英繼之巒帥騎八百倍道兼行五日至鮑口丙子早生遣其大將胡孝智將兵七千離城二百里逆戰【帥讀曰率騎奇寄翻下同將即亮翻下椿將同離力智翻}
巒奮擊大破之乘勝長驅至懸瓠早生出城逆戰又破之因渡汝水圍其城詔加巒都督南討諸軍事丁丑魏鎮東參軍成景雋殺宿豫戍主嚴仲賢以城來降【降戶江翻}
時魏郢豫二州自懸瓠以南至于安陸諸城皆沒唯義陽一城為魏堅守【為于偽翻}
蠻帥田益宗帥羣蠻以附魏【蠻帥所類翻宗帥讀曰率}
魏以為東豫州刺史【魏東豫州治新息廣陵城領汝南東新蔡新蔡弋陽長陵郡}
上以車騎大將軍開府儀同三司五千戶郡公招之益宗不從十一月庚寅魏遣安東將軍楊椿將兵四萬攻宿豫魏主聞邢巒屢捷命中山王英趣義陽英以衆少累表請兵弗許【趣與趨同七喻翻少詩沼翻}
英至懸瓠輒與巒共攻之十二月己未齊苟兒等開門出降斬白早生及其黨數十人英乃引兵前趨義陽【趨七喻翻}
寧朔將軍張道凝先屯楚王城癸亥棄城走英追擊斬之魏義陽太守狄道辛祥與婁悦共守義陽將軍胡武城陶平虜攻之祥夜出襲其營擒平虜斬武城由是州境獲全論功當賞婁悦恥功出其下間之於執政賞遂不行【間古莧翻史言高肇專魏賞罰無章}
 壬申魏東荆州表桓暉之弟叔興前後招撫太陽蠻歸附者萬餘戶請置郡十六縣五十【自是之後蠻左郡縣不可勝紀矣}
詔前鎮東府長史酈道元案行置之【行下孟翻}
道元範之子也【酈範見一百三十二卷宋明帝泰始三年}
 是歲柔然佗汗可汗復遣紇奚勿六跋獻貂裘於魏【佗徒河翻汗音寒可從刋入聲復扶又翻}
魏主弗受報之如前【前事見上卷五年}
初高車侯倍窮奇為嚈噠所殺【嚈噠國大月氐之種類也亦曰高車之别種其原出於塞北自金山而南在于闐之西去長安一萬一百里其王都拔底城蓋王舍城也嚈益涉翻噠當割翻又陁葛翻又宅軋翻}
執其子彌俄突而去其衆分散或犇魏或犇柔然魏主遣羽林監河南孟威撫納降戶置於高平鎮【降戶江翻}
高車王阿伏至羅殘暴國人殺之立其宗人跋利延嚈噠奉彌俄突以伐高車國人殺跋利延迎彌俄突而立之彌俄突與佗汗可汗戰于蒲類海不勝西走三百餘里佗汗軍于伊吾北山會高昌王麴嘉求内徙于魏時孟威為龍驤將軍【驤思將翻}
魏主遣威發凉州兵三千人迎之至伊吾佗汗見威軍怖而遁去【怖普布翻}
彌俄突聞其離駭追擊大破之殺佗汗于蒲類海北割其髮送于威且遣使入貢于魏【使疏吏翻}
魏主使東城子于亮報之賜遺甚厚【遺于季翻}
高昌王嘉失期不至威引兵還沱汗可汗子醜奴立號豆羅伏跋豆伐可汗【魏收曰魏言彰制也}
改元建昌 宋齊舊儀祀天皆服衮冕兼著作郎高陽許懋請造大裘從之【周禮天官司裘掌為大裘以供王祀天之服鄭衆注云大裘黑羔裘服以祀天示質時有司尋大裘之制唯鄭玄注司服云大裘羔裘也既無所出未可為據按六冕之服皆玄上纁下今宜以玄繒為之其制式如裘其裳以纁皆無文繡冕則無旒制曰可}
上將有事太廟詔以齋日不樂自今輿駕始出鼓吹從而不作還宫如常儀【還宫則鼓吹振作吹昌瑞翻}


  八年春正月辛巳上祀南郊大赦時有請封會稽禪國山者【國山在義興國山縣隋廢義興郡為義興縣并國山入焉我朝太平興國元年以太宗藩邸舊諱改義興為宜興會工外翻}
上命諸儒草封禪儀欲行之許懋建議以為舜柴岱宗是為巡狩而鄭引孝經鉤命决云封于太山考績柴燎禪乎梁甫刻石紀號此緯書之曲說非正經之通義也【緯于貴翻}
舜五載一巡狩春夏秋冬周徧四嶽【書舜典歲二月東巡狩至于岱宗柴望秩于山川五月南巡狩至于南岳八月西巡狩至于西岳十有一月北巡狩至于北岳載子亥翻}
若為封禪何其數也【數所角翻}
又如管夷吾所說七十二君燧人之前世質民淳安得泥金檢玉結繩而治安得鐫文告成夷吾又云唯受命之君然後得封禪周成王非受命之君云何得封太山禪社首神農即炎帝也而夷吾分為二人妄亦甚矣若聖主不須封禪若凡主不應封禪蓋齊桓公欲行此事夷吾知其不可故舉怪物以屈之【班志曰齊桓公既霸會諸侯于葵丘而欲封禪管仲曰古者封太山禪梁父者七十二家夷吾所記者十有二焉昔無懷氏封太山禪云云伏羲封太山禪云云神農氏封太山禪云云炎帝封太山禪云云黄帝封太山禪云云顓頊封太山禪云云帝嚳封太山禪云云堯封太山禪云云舜封太山禪云云禹封太山禪會稽湯封太山禪云云成王封太山禪于社首皆受命乃得封禪桓公曰寡人九合諸侯一匡天下昔三代受命何以異此管仲睹桓公不可窮以辭因設之以事曰古之封禪鄗上黍北里禾所以為盛江淮以一茅三脊所以為藉東海致比目之魚西海致比翼之烏然後物有不召而自至者十有五焉今鳳凰麒麟不至嘉禾不生而蓬蒿藜莠茂鴟鶚羣翔而欲封禪無乃不可乎桓公乃止}
秦始皇嘗封太山孫皓嘗遣兼司空董朝至陽羨封禪國山【五代志曰義興舊曰陽羨}
皆非盛德之事不足為法然則封禪之禮皆道聽所說失其本文由主好名於上【好呼到翻}
而臣阿旨於下也古者祀天祭地禮有常數誠敬之道盡此而備至於封禪非所敢聞上嘉納之因推演懋議稱制旨以答請者由是遂止 魏中山王英至義陽將取三關先策之曰三關相須如左右手若克一關兩關不待攻而破攻難不如攻易宜先攻東關【東關即武陽關易以豉翻下勢易同}
又恐其并力於東乃使長史李華帥五統向西關【五統五統軍之衆西關即平靖關帥讀曰率}
以分其兵勢自督諸軍向東關先是馬仙琕使雲騎將軍馬廣屯長薄軍主胡文超屯松峴【先悉薦翻騎奇寄翻峴戶典翻 考異曰梁馬仙琕傳云遣馬廣會超守三關今從魏中山王英傳}
丙申英至長薄戊戌長薄潰馬廣遁入武陽英進圍之上遣冠軍將軍彭甕生驃騎將軍徐元季將兵援武陽【冠古玩翻驃匹妙翻騎奇寄翻季將即亮翻下同 考異曰英傳作徐超秀今從魏帝紀}
英故縱之使入城曰吾觀此城形勢易取【易以䜴翻}
甕生等既入英促兵攻之六日而拔虜三將及士卒七千餘人進攻廣峴【廣峴蓋即黄峴關}
太子左衛率李元履棄城走又攻西關馬仙琕亦棄城走上使南郡太守韋叡將兵救仙琕【邵陽之捷叡遷左衛將軍尋為安西長史南郡太守}
叡至安陸增築城二丈餘更開大塹起高樓【塹七艷翻}
衆頗譏其示怯叡曰不然為將當有怯時不可專勇中山王英急追馬仙琕將復邵陽之恥聞叡至乃退上亦有詔罷兵初魏主遣中書舍人鮦陽董紹慰勞叛城【鮦陽縣漢屬汝南郡晉屬汝隂郡魏屬新蔡郡孟康曰鮦音紂紅翻隋廢新蔡郡為縣屬豫州鮦陽之地當在新蔡縣界勞力到翻}
白早生襲而囚之送於建康魏主既克懸瓠命於齊苟兒等四將之中分遣二人敕楊州為移【魏楊州治夀陽移移文也}
以易紹及司馬悦首 【考異曰紹傳云歸苟兒等十人今從司馬悦傳}
移書未至領軍將軍呂僧珍與紹言愛其文義言於上上遣主書霍靈超謂紹曰今聽卿還令卿通兩家之好【好呼到翻下通好同}
彼此息民豈不善也因召見賜衣物【見賢遍翻}
令舍人周捨慰勞之【舍人中書通事舍人勞力到翻}
且曰戰争多年民物塗炭吾是以不恥先言與魏朝通好比亦有書全無報者【朝直遥翻好呼到翻下同比毗至翻近也}
卿宜備申此意今遣傳詔霍靈秀送卿至國遲有嘉問【遲直利翻待也}
又謂紹曰卿知所以得不死不【死不讀曰否}
今者獲卿乃天意也夫立君以為民也【為于偽翻}
凡在民上豈可以不思此乎若欲通好今以宿豫還彼彼當以漢中見歸紹還魏言之魏主不從 三月魏荆州刺史元志將兵七萬寇潺溝【潺溝在漢北據梁書吳平侯昺傳破志於潺溝流尸蓋漢水則潺溝之水南注于漢潺仕山翻}
驅廹羣蠻羣蠻悉渡漢水來降雍州刺史吳平侯昺納之【降戶江翻雍于用翻昺音丙}
綱紀皆以蠻累為邊患不如因此除之【州郡上佐謂之綱紀言其綱紀州郡之事也}
昺曰窮來歸我誅之不祥且魏人來侵吾得蠻以為屏蔽不亦善乎【屏必郢翻}
乃開樊城受其降命司馬朱思遠等擊志於潺溝大破之斬首萬餘級志齊之孫也【拓跋齊見一百二十卷宋文帝元嘉四年}
 夏四月戊申以臨川王宏為司空加車騎將軍王茂開府儀同三司【騎奇寄翻}
 丁卯魏楚王城主李國興以城降 秋七月癸巳巴陵王蕭寶義卒 九月辛巳魏封故北海王詳子顥為北海王【詳得罪死事見一百四十五卷天監元年}
 魏公孫崇造樂尺以十二黍為寸劉芳非之更以十黍為寸尚書令高肇等奏崇所造八音之器及度量皆與經傳不同詰其所以然云必依經文聲則不協請更令芳依周禮造樂器俟成集議並呈從其善者詔從之【夫作樂者先定律律起于黄鍾黄鍾之長以黍審其度黄鍾之龠以黍審其容周禮典同雖曰掌六律六同之和以辯天地四方隂陽之聲以為樂器然度之長短容之多少未嘗詳言之也冬官考工既出于漢而鳬氏為鍾但言其廣長圓徑深厚而絫黍之法無聞焉肇請令芳依周禮造樂器未知其何所依也魏收曰太和中詔中書監高閭修正音律久未能定閭表太樂祭酒公孫崇參知律呂鐘磬之事景明四年并州獲古銅權詔付崇以為鍾律之準永平中崇更造新尺以一黍之長絫為寸法尋太常卿劉芳受詔修樂以秬黍中者一黍之廣即為一分而中尉元匡以一黍之廣度黍二縫以取一分三家分競久不能决太和十九年高祖詔以一黍之廣用成分體九十黍之長以定銅尺有司奏從前詔而芳尺同高祖所制故遂典修金石更工衡翻傳直戀翻詰去吉翻}
 冬十月癸丑魏以司空廣陽王嘉為司徒 十一月己丑魏主於式乾殿為諸僧及朝臣講維摩詰經【為于偽翻朝直遥翻}
時魏主專尚釋氏不事經籍中書侍郎河東裴延雋上疏以為漢光武魏武帝雖在戎馬之間未嘗廢書先帝遷都行師手不釋卷良以學問多益不可暫輟故也陛下升法座親講大覺凡在瞻聽塵蔽俱開然五經治世之模楷應務之所先【治直之翻}
伏願經書互覽孔釋兼存則内外俱周真俗斯暢矣時佛教盛于洛陽沙門之外自西域來者三千餘人魏主别為之立永明寺千餘間以處之處士南陽馮亮有巧思【為于偽翻處昌呂翻思相吏翻}
魏主使與河南尹甄琛沙門統僧暹擇嵩山形勝之地立閒居寺極巖壑土木之美由是遠近承風無不事佛比及延昌【甄之人翻琛丑林翻比必利翻}
州郡共有一萬三千餘寺 是歲魏宗正卿元樹來犇賜爵鄴王樹翼之弟也時翼為青冀二州刺史鎮郁州【水經注朐山東北海中有大洲謂之郁洲}
久之翼謀舉州降魏事泄而死【元翼來降見上卷五年降戶江翻}
九年春正月乙亥以尚書令沈約為左光禄大夫右光禄大夫王瑩為尚書令約文學高一時而貪冒榮利用事十餘年政之得失唯唯而已【冒莫北翻下同唯于癸翻}
自以久居端揆有志台司論者亦以為宜而上終不用乃求外出又不許徐勉為之請三司之儀【梁官制有開府同三司之儀在開府儀同三司下為于偽翻}
上不許 庚寅新作緣淮塘北岸起石頭迄東冶南岸起後渚籬門迄三橋 三月丙戌魏皇子詡生詡母胡充華臨涇人【充華晉武帝制宋明帝時以婕妤充華等五職位亞九嬪蕭齊之世位列九殯臨涇縣自漢以來屬安定郡詡况羽翻}
父國珍襲武始伯【隋志金城郡狄道縣後魏置武始郡}
充華初選入掖庭同列以故事祝之願生諸王公主勿生太子充華曰妾之志異於諸人奈何畏一身之死而使國家無嗣乎及有娠同列勸去之【娠音身去羌呂翻}
充華不可私自誓曰若幸而生男次第當長【長丁丈翻今知兩翻下漸長同}
男生身死所不憾也既而生詡先是魏主頻喪皇子【先悉薦翻喪直浪翻}
年漸長深加慎護擇良家宜子者以為乳保【乳母保母也}
養於别宫皇后充華皆不得近【近其靳翻}
 己丑上幸國子學親臨講肄乙未詔皇太子以下及王侯之子年可從師者皆入學 舊制尚書五都令史皆用寒流夏四月丁巳詔曰尚書五都職參政要非但總領衆局亦乃方軌二丞【方軌謂並駕也二丞謂左右丞}
可革用士流秉此羣目於是以都令史視奉朝請【朝直遥翻}
用太學博士劉納兼殿中都司空法曹參軍劉顯兼吏部都太學博士孔䖍孫兼金部都司空法曹參軍蕭軌兼左右戶都宣毅墨曹參軍王顒兼中兵都【宣毅將軍府之墨曹參軍顒魚容翻}
並以才地兼美首膺其選 六月宣城郡吏吳承伯挾妖術聚衆【妖於驕翻}
癸丑攻郡殺太守朱僧勇轉屠旁縣閏月己丑承伯踰山奄至吳興東土人素不習兵吏民恇擾奔散或勸太守蔡撙避之撙不可【恇去王翻撙慈損翻}
募勇敢閉門拒守承伯盡鋭攻之撙帥衆出戰大破之臨陳斬承伯【帥讀曰率陳讀曰陣}
撙興宗之子也【蔡興宗仕宋大明泰始之間以方正自持}
承伯餘黨入新安攻陷黟歙諸縣【黟音伊歙書涉翻}
太守謝覽遣兵拒之不勝逃奔會稽臺軍討賊平之覽瀹之子也【謝瀹仕宋齊之間位要近有清望會外翻}
工 冬十月魏中山獻武王英卒 上即位之三年詔定新歷員外散騎侍郎祖暅奏其父冲之考古法為正歷不可改【散悉亶翻騎奇寄翻暅古鄧翻又况晩翻}
至八年詔太史課新舊二歷新歷密舊歷疎是歲始行冲之大明歷【舊歷何承天歷也新歷祖冲之歷也冲之上已見一百二十九卷宋孝武帝大明六年}
 魏劉芳奏所造樂器及教文武二舞登歌鼓吹曲等已成【吹昌瑞翻}
乞如前敕集公卿羣儒議定與舊樂參呈若臣等所造形制合古擊拊會節請於來年元會用之詔舞可用新餘且仍舊

  十年春正月辛丑上祀南郊大赦 尚書左僕射張稷自謂功大賞薄【稷以殺齊東昏侯為功}
嘗侍宴樂夀殿【樂音洛}
酒酣怨望形於辭色上曰卿兄殺郡守【稷兄瓌殺劉遐事見一百三十四卷宋順帝昇明元年守式又翻}
弟殺其君有何名稱稷曰臣乃無名稱【稱尺證翻}
至於陛下不得言無勲東昏暴虐義師亦來伐之豈在臣而已上捋其須【捋盧括翻須古鬚字通}
曰張公可畏人稷既懼且恨乃求出外癸卯以稷為青冀二州刺史王珍國亦怨望【王珍國與稷同殺東昏侯其怨望之心與稷同}
罷梁秦二州刺史還【還從宣翻又音如字 考異曰梁書珍國未嘗為梁秦刺史今從南史}
酒後於坐啓云【坐徂卧翻}
臣近入梁山便哭上大驚曰卿若哭東昏則已晩若哭我我復未死珍國起拜謝竟不答坐即散【復扶又翻坐徂卧翻}
因此疎退久之除都官尚書 丁巳魏汾州山胡劉龍駒聚衆反侵擾夏州詔諫議大夫薛和東秦汾華夏四州之衆以討之【魏高祖太和十一年分秦州置華州治華隂領華山登城白水郡又置夏州治統萬領化政闡熙金明代各郡夏戶雅翻華戶化翻}
 辛酉上祀明堂 三月琅邪民王萬夀殺東莞琅邪二郡太守劉晣【邪音耶莞音管晣之舌翻考異曰梁帝紀云三月辛丑按長歷是月丁酉朔而盧昶傳云三月十四夜萬夀等攻掩朐城蓋辛酉也今}


  【不日以闕疑又梁馬仙琕傳及魏帝紀盧昶傳皆云劉晣而梁帝紀云鄧晣蓋字誤也}
據朐山召魏軍【朐音劬}
 壬戌魏廣陽懿烈王嘉卒 魏徐州刺史盧昶遣郯城戍副張天惠【秦置郯郡漢改為東海郡魏復置郯郡屬東徐州郯音談}
琅邪戍主傅文驥相繼赴朐山青冀二州刺史張稷遣兵拒之不勝【梁青冀二州治鬰洲}
夏四月文驥等據朐山詔振遠將軍馬仙琕擊之【琕部田翻}
魏又遣假安南將軍蕭寶寅假平東將軍天水趙遐將兵據朐山受盧昶節度【將即亮翻}
甲戌魏薛和破劉龍駒悉平其黨表置東夏州【東夏州領偏城朔方定陽上郡唐之延州魏之東夏州也夏戶雅翻}
 五月丙辰魏禁天文學以國子祭酒張充為尚書左僕射充緒之子也【張緒岱之兄子善談名理}
 馬仙琕圍朐山張稷權頓六里以督饋運上數兵助之【數所角翻}
秋魏盧昶上表請益兵六千米十萬石魏主以兵四千給之冬十一月己亥魏主詔揚州刺史李崇等治兵夀陽以分朐山之勢【治直之翻}
盧昶本儒生不習軍旅朐山城中糧樵俱竭傳文驥以城降【降戶江翻}
十二月庚辰昶引兵先遁諸軍相繼皆潰會大雪軍士凍死及墮手足者三分之二仙琕追擊大破之二百里間僵尸相屬魏兵免者什一二收其糧畜器械不可勝數【屬之欲翻勝音升 考異曰魏帝紀盧昶敗在十一月今從梁帝紀梁紀曰斬馘十餘萬按盧昶表云此兵九千賊衆四萬求益兵六千魏主以四千給之安得十餘萬衆蓋梁史為夸大耳}
昶單騎而走棄其節傳儀衛俱盡【傳張戀翻}
至郯城借趙遐節以為軍威魏主命黄門侍郎甄琛馳驛鎖昶窮其敗狀【甄之人翻琛丑林翻驛人質翻驛傳也}
及趙遐皆免官唯蕭寶寅全軍而歸盧昶之在朐山也御史中尉游肇言於魏主曰朐山蕞爾僻在海濱卑濕難居【蕞徂外翻}
於我非急於賊為利為利故必致死以争之非急故不得已而戰以不得已之衆擊必死之師恐稽延歲月所費甚大假令得朐山徒致交争終難全守所謂無用之田也【左傳吳將伐齊子胥諫曰得志於齊猶獲石田也無所用之}
聞賊屢以宿豫求易朐山若必如此持此無用之地復彼舊有之疆兵役時解其利為大魏主將從之會昶敗遷肇侍中肇明根之子也【游明根事魏太武及孝文以耆宿見重}
馬仙琕為將能與士卒同勞逸所衣不過布帛所居無幃幕衾屏飲食與厮養最下者同【將即亮翻衣於既翻厮息移翻養養馬者音余亮翻韋昭曰析薪為厮炊烹為養}
其在邊境常單身潛入敵境伺知壁壘村落險要處【伺相吏翻}
所攻戰多捷士卒亦樂為之用【樂音洛為于偽翻}
魏以甄琛為河南尹琛表曰國家居代患多盜竊世

  祖憤廣置主司里宰皆以下代令長及五等散男有經畧者乃得為之【長知兩翻五等散男謂爵為五等男而居散官者魏書曰魏公侯伯子男有開國有散凡散各降開國一品非以其居散官而謂之散男也散悉亶翻}
又多置吏士為其羽翼崇而重之始得禁止今遷都已來天下轉廣四遠赴會事過代都五方雜沓寇盜公行里正職輕任碎多是下材人懷苟且不能督察請取武官八品將軍已下幹用貞濟者【貞濟謂堅貞而濟事也}
以本官俸恤領里尉之任高者領六部尉中者領經途尉下者領里正【魏官既給俸又給恤親之禄故謂之俸恤魏分洛陽城中為六部置六部尉因張平子東京賦經途九軌置經途尉經途城中之大途也其餘處各置里正}
不爾請少高里尉之品【少詩沼翻}
選下品中應遷者進而為之督責有所輦轂可清【自漢以來京師謂之輦轂下}
詔曰里正可進至勲品【勲品勲官初品也}
經途從九品六部尉正九品【洛陽六部尉並置於東漢之時曹操為洛陽北部尉此其證也從才用翻}
諸職中簡取不必武人琛又奏以羽林為游軍於諸坊巷司察盜賊於是洛城清静後常踵焉 是歲梁之境内有州二十三【此據五代史志按蕭子顯齊志齊有揚南徐豫兖南兖北徐青冀江廣交越荆巴郢司雍梁秦益寧湘南豫二十三州時已廢巴州當以王茂所立宛州足之}
郡三百五十縣千二十二是後州名浸多廢置離合不可勝記【勝音升}
魏朝亦然【朝直遥翻下同}
 上敦睦九族優惜朝士有犯罪者皆屈法申之百姓有罪則案之如法其緣坐則老幼不免一人逃亡舉家質作【質音致又如字質作質其家屬而罰作之}
民既窮窘姦宄益深嘗因郊祀有秣陵老人【江南以建康秣陵為赤縣隋廢秣陵建康併為江寜縣窘渠隕翻宄音軌}
遮車駕言曰陛下為法急於庶民緩於權貴非長久之道誠能反是天下幸甚上於是思有以寛之

  十一年春正月壬辰詔自今逋讁之家及罪應質作若年有老小可停將送【所謂寛庶民者如此而已而不能繩權貴以法君子是以知梁政之亂也}
 以臨川王宏為太尉驃騎將軍王茂為司空尚書令【驃匹妙翻騎奇寄翻下同}
 丙辰魏以車騎大將軍尚書令高肇為司徒清河王懌為司空廣平王懷進號驃騎大將軍加儀同三司肇雖登三司猶自以去要任怏怏形於言色【要任謂尚書令怏於兩翻}
見者嗤之【嗤丑之翻}
尚書右丞高綽國子博士封軌素以方直自業【業事也以方直為事所謂彊作之也作之不已乃成君子}
及肇為司徒綽送迎往來軌竟不詣肇綽顧不見軌乃遽歸嘆曰吾平生自謂不失規矩今日舉措不如封生遠矣綽允之孫軌懿之族孫也【高允事魏世祖以下四朝封懿去燕歸魏以疎慢見黜}
清河王懌有才學聞望懲彭城之禍【謂彭城王勰無罪見殺也聞音問}
因侍宴謂肇曰天子兄弟詎有幾人而翦之幾盡【謂又殺京兆王愉也之幾居依翻}
昔王莽頭秃籍渭陽之資遂簒漢室【事見漢紀}
今君身曲亦恐終成亂階會大旱肇擅録囚徒欲以收衆心懌言於魏主曰昔季氏旅於泰山孔子疾之【朱元晦曰旅祭名也禮諸侯祭封内山川季氏祭之僭也}
誠以君臣之分【分扶問翻}
宜防微杜漸不可瀆也減膳録囚乃陛下之事今司徒行之豈人臣之義乎明君失之於上姦臣竊之於下禍亂之基於此在矣帝笑而不應夏四月魏詔尚書與羣司鞫理獄訟令饑民就穀燕恒二州及六鎮【燕因肩翻恒戶登翻}
 乙酉魏大赦改元延昌 冬十月乙亥魏立皇子詡為太子始不殺其母【為後胡后亂魏張本}
以尚書右僕射郭祚領太子少師祚嘗從魏主幸東宫懷黄㼐以奉太子【㼐扶田翻博雅白㼐瓜屬此黄㼐又一種也}
時應詔左右趙桃弓深為帝所信任祚私事之時人謂之桃弓僕射黄㼐少師 十一月乙未以吳郡太守袁昂兼尚書右僕射 初齊太子步兵校尉平昌伏曼容表求制一代禮樂世祖詔選學士十人修五禮【五禮吉凶軍賓嘉}
丹楊尹王儉總之儉卒【卒子恤翻}
以事付國子祭酒何胤胤還東山【胤隱會稽東山}
齊明帝敕尚書令徐孝嗣掌之孝嗣誅率多散逸詔驃騎將軍何佟之掌之【佟徒冬翻}
經齊末兵火僅有在者帝即位佟之啟審省置之宜【啟之於上審禮局之宜省宜置也}
敕使外詳【使外詳議以聞也}
時尚書以為庶務權輿【毛萇曰權輿始也此言王業創始也}
宜俟隆平欲且省禮局併還尚書儀曹詔曰禮壞樂缺實宜以時修定但頃之修撰不得其人所以歷年不就有名無實此既經國所先可即撰次【左傳曰禮經國家定社稷序人民為後嗣者也撰具也述也}
於是尚書僕射沈約等奏請五禮各置【舊學士十人共修五禮今請分五禮各置學士}
舊學士一人令自舉學古一人相助抄撰【抄楚交翻録也}
其中疑者依石渠白虎故事請制旨斷决【石渠事見二十七卷漢宣帝甘露三年白虎事見四十六卷章帝建初四年斷丁亂翻}
乃以右軍記室明山賓等分掌五禮佟之總其事佟之卒以鎮北諮議參軍伏暅代之暅曼容之子也【暅古鄧翻}
至是五禮成列上之合八千一十九條詔有司遵行 己酉臨川王宏以公事左遷驃騎大將軍是歲魏以桓叔興為南荆州刺史治安昌【漢南陽郡有安昌侯}


  【國晉泰始中割南陽東鄙之安昌平林平氏義陽四縣置義陽郡治安昌城後義陽移治石城山上因梁希侵逼徙治仁順城而安昌則俗謂之白茅城}
隸東荆州

  十二年春正月辛卯上祀南郊大赦 二月辛酉以兼尚書右僕射袁昂為右僕射 己卯魏高陽王雍進位太保 鬱洲迫近魏境【近其靳翻}
其民多私與魏人交市朐山之亂或隂與魏通朐山平心不自安青冀二州刺史張稷不得志政令寛弛僚吏頗多侵漁庚辰鬱洲民徐道角等夜襲州城殺稷送其首降魏【胊音劬降戶江翻考異曰魏帝紀作郁州人徐玄明今從梁康絢傳又絢傳稷死在朐山叛之明年今從魏帝紀 按鬱洲即郁洲}
魏遣前南兖州刺史樊魯將兵赴之【將即亮翻}
於是魏饑民餓死者數萬侍中游肇諫以為朐山濱海卑濕難居鬱洲又在海中得之尤為無用其地於賊要近【要謂海道之要近謂南近江淮}
去此閒遠【魏圖東南其用兵必於淮漢之間鬱洲介在海中又非兵衝故曰閒遠}
以閒遠之兵攻要近之衆不可敵也方今年饑民困唯宜安静而復勞以軍旅費以饋運臣見其損未見其益魏主不從復遣平西將軍奚康生將兵逆之【復扶又翻}
未北兖州刺史康絢遣司馬霍奉伯討平之【梁北兖州當治淮隂絢許縣翻}
 辛巳新作太極殿 上嘗與侍中太子少傅建昌侯沈約各疏策事約少上三事出謂人曰此公護前不則羞死【帝每集文學之士策經史事羣臣多引短推長帝乃悦故約退有是言護前者自護其所短不使人在己前忌前者忌人在己前也約少詩沼翻不讀曰否}
上聞之怒欲治其罪徐勉固諫而止【治直之翻}
上有憾於張稷【以其怨望故憾之}
從容與約語及之【從千容翻}
約曰左僕射出作邊州【謂為青冀二州刺史}
已往之事何足復論【復扶又翻}
上以約與稷昏家相為【為于偽翻}
怒曰卿言如此是忠臣邪乃輦歸内殿約懼不覺上起猶坐如初及還未至牀而憑空頓於戶下【踣而首先至地為頓}
因病夢齊和帝以劍斷其舌【斷丁管翻}
乃呼道士奏赤章於天稱禪代之事不由己出上遣主書黄穆之視疾夕還增損不即啟聞懼罪乃白赤章事上大怒中使譴責者數四【帝本信釋氏報應之說謂天可欺也故因赤章之事而怒責約古人不肯移腹心之疾而寘諸股肱雅異於是使疏吏翻}
約益懼閏月乙丑卒【卒子恤翻}
有司諡曰文上曰情懷不盡曰隱改諡隱侯 夏五月夀陽久雨大水入城廬舍皆沒魏揚州刺史李崇勒兵泊於城上水增未已乃乘船附於女牆【城上短牆曰女牆所謂陴也今人謂之女頭}
城不沒者二板將佐勸崇棄夀陽保北山【夀陽北山即八公山}
崇曰吾忝守藩岳德薄致災淮南萬里繫于吾身一旦動足百姓瓦解揚州之地恐非國物吾豈愛一身取愧王尊【漢王尊為東郡太守河水盛溢泛浸瓠子金隄老弱奔走尊止宿隄上吏民争叩頭救止尊不肯去及水盛隄壞吏民皆奔走唯一主簿泣在尊旁立不動而水波稍却回還吏民咸壯尊之勇節}
但憐此士民無辜同死可結筏隨高人規自脱【規圖也筏音伐}
吾必與此城俱沒幸諸君勿言揚州治中裴絢帥城南民數千家汎舟南走避水高原【絢許縣翻帥讀曰率}
謂崇還北因自稱豫州刺史【自宋以來置豫州於夀陽絢乘水聚民自稱豫州刺史以求梁應援}
與别駕鄭祖起等送任子來請降馬仙琕遣兵赴之崇聞絢叛未測虚實遣國侍郎韓方興單舸召之【崇爵陳留公故有國侍郎降戶江翻琕部田翻舸古我翻}
絢聞崇在悵然驚恨報曰比因大水顛狽為衆所推【比毗至翻}
今大計已爾勢不可追恐民非公民吏非公吏願公早行無犯將士崇遣從弟寧朔將軍神等將水軍討之【將即亮翻從才用翻}
絢戰敗神追拔其營絢走為村民所執還至尉升湖曰吾何面見李公乎乃投水死絢叔業之兄孫也【裴叔業降魏見一百四十三卷齊東昏侯之永元二年}
鄭祖起等皆伏誅崇上表以水災求解州任魏主不許崇沈深寛厚【沈持林翻}
有方畧得士衆心在夀春十年【天監六年魏主命李崇鎮夀春至是年財七年耳至十五年乃徵拜尚書左僕射適十年史終言之}
常養壯士數千人寇來無不摧破鄰敵謂之卧虎上屢設反間以疑之【間古莧翻}
又授崇車騎大將軍開府儀同三司萬戶郡公諸子皆為縣侯而魏主素知其忠篤委信不疑 六月癸巳新作太廟 秋八月戊午以臨川王宏為司空 魏恒肆二州地震山鳴【魏世祖真君七年置肆州領新興秀容鴈門郡治九原恒戶登翻}
踰年不已民覆壓死傷甚衆【是後破六韓拔陵等作亂恒肆以北悉為盜區此其祥歟}
 魏主幸東宫以中書監崔光為太子少傅命太子拜之光辭不敢當帝不許太子南面再拜詹事王顯啟請從太子拜於是宫臣皆拜光北面立不敢答唯西面拜謝而出

  十三年春二月丁亥上耕藉田大赦宋齊藉田皆用正月至是始用二月及致齋祀先農【漢儀正月始耕耕日以太牢祀先農臣瓚注曰先農即神農炎帝也}
 魏東豫州刺史田益宗衰老與諸子孫聚斂無厭【斂力贍翻厭於鹽翻}
部内苦之咸言欲叛魏主遣中書舍人劉桃符慰勞益宗【勞力到翻}
桃符還啟益宗侵擾之狀魏主賜詔曰桃符聞卿息魯生在淮南貪暴【此淮南大槩謂淮水之南}
為爾不已損卿誠効可令魯生赴闕當加任使魯生久未至詔徙益宗為鎮東將軍濟州刺史又慮其不受代遣後將軍李世哲與桃符帥衆襲之奄入廣陵【此新息之廣陵也濟子禮翻帥讀曰率}
魯生與其弟魯賢超秀皆犇關南招引梁兵攻取光城已南諸戍【宋文帝元嘉十五年以豫部蠻民立光城等七縣明帝大明中立光城左郡五代志弋陽郡光山縣舊置光城郡}
上以魯生為北司州刺史魯賢為北豫州刺史超秀為定州刺史【北司北豫因各人所統之地而授以刺史魏收志定州治蒙籠城領弋陽汝隂安定新蔡北建寜郡皆蠻郡也水經注舉水出龜頭山西北流逕蒙籠戍南梁定州治}
三月魏李世哲擊魯生等破之復置郡戍【復扶又翻}
以益宗還洛陽授征南將軍金紫光禄大夫益宗上表稱為桃符所讒及言魯生等為桃符逼逐使叛乞攝桃符與臣對辯虚實詔不許曰既經大宥【謂己宥其謀叛之罪}
不容方更為獄 秋七月乙亥立皇子綸為邵陵王繹為湘東王紀為武陵王 冬十月庚辰魏主遣驍騎將軍馬義舒慰諭柔然【驍堅堯翻騎奇寄翻下同}
 魏王足之入寇也【事見一百四十六卷五年}
上命寧州刺史涪人李略禦之【涪音浮}
許事平用為益州足退上不用略怨望有異謀上殺之其兄子苗犇魏步兵校尉泰山淳于誕嘗為益州主簿自漢中入魏二人共說魏主以取蜀之策【說式芮翻}
魏主信之辛亥以司徒高肇為大將軍平蜀大都督將步騎十五萬寇益州命益州刺史傅豎眼出巴北【巴北巴郡以北也巴西郡梁置北巴州閬中縣梁置北巴郡將即亮翻豎而庾翻}
梁州刺史羊祉出涪城安西將軍奚康生出綿竹撫軍將軍甄琛出劍閣【甄之人翻琛丑林翻}
乙卯以中護軍元遥為征南將軍都督鎮遏梁楚【此梁楚謂古梁楚大界汴汝之間也}
游肇諫以為今頻年水旱百姓不宜勞役往昔開拓皆因城主歸欵故有征無戰【不因薛安都常珍奇沈文秀魏不得淮汝青徐不因裴叔業魏不得夀陽游肇之言可謂深知當時疆事者}
今之陳計者真偽難分或有怨於彼不可全信蜀地險隘鎮戍無隙豈得虚承浮說而動大軍舉不慎始悔將何及不從以淳于誕為驍騎將軍假李苗龍驤將軍皆領鄉導統軍【以統軍鄉導因以名官驍堅堯翻騎奇寄翻驤思將翻鄉讀曰嚮}
 魏降人王足陳計【王足來奔見上卷六年降戶江翻}
求堰淮水以灌夀陽上以為然使水工陳承伯材官將軍祖暅視地形【暅居鄧翻}
咸謂淮内沙土漂輕不堅實功不可就上弗聽發徐揚民率二十戶取五丁以築之假太子右衛率康絢都督淮上諸軍事并護堰作於鍾離【康絢護堰作而置司於鍾離率所律翻絢許縣翻}
役人及戰士合二十萬南起浮山北抵巉石【水經注淮水自鍾離縣又東逕浮山山北對巉石山巉助銜翻杜佑曰浮山堰在濠州城西一百一十二里}
依岸築土合脊於中流 魏以前定州刺史楊津為華州刺史【華戶化翻}
津椿之弟也先是官受調絹尺度特長任事因緣共相進退【先悉薦翻調徒弔翻下同任事謂任受絹之事者也任音壬因緣謂因緣為姦進退謂有賂者則進而為長無賂者則退而為短}
百姓苦之津令悉依公尺其輸物尤善者賜以杯酒所輸少劣亦為受之【少詩沼翻為于偽翻}
但無酒以示恥於是人競相勸官調更勝舊日 魏太子尚幼每出入東宫左右乳母而已宫臣皆不知之 詹事楊昱上言乞自今召太子必降手敕令臣等翼從【從才用翻}
魏主從之命宫臣在直者從至萬歲門【萬歲門蓋洛陽宫城之東門}
 魏御史中尉王顯問治書侍御史陽固曰【治直之翻}
吾作太府卿府庫充實卿以為何如固曰公收百官之禄四分之一州郡贓贖悉輸京師以此充府未足為多且有聚斂之臣寧有盜臣【記大學之言斂力贍翻}
可不戒哉顯不悦因事奏免固官

  資治通鑑卷一百四十七


    


 


 



 

 
  







 


  
  
 
 
 


  

 















	
	









































 
  



















 





 












  
  
  

 





