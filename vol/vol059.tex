<!DOCTYPE html PUBLIC "-//W3C//DTD XHTML 1.0 Transitional//EN" "http://www.w3.org/TR/xhtml1/DTD/xhtml1-transitional.dtd">
<html xmlns="http://www.w3.org/1999/xhtml">
<head>
<meta http-equiv="Content-Type" content="text/html; charset=utf-8" />
<meta http-equiv="X-UA-Compatible" content="IE=Edge,chrome=1">
<title>資治通鑒_60-資治通鑑卷五十九_60-資治通鑑卷五十九</title>
<meta name="Keywords" content="資治通鑒_60-資治通鑑卷五十九_60-資治通鑑卷五十九">
<meta name="Description" content="資治通鑒_60-資治通鑑卷五十九_60-資治通鑑卷五十九">
<meta http-equiv="Cache-Control" content="no-transform" />
<meta http-equiv="Cache-Control" content="no-siteapp" />
<link href="/img/style.css" rel="stylesheet" type="text/css" />
<script src="/img/m.js?2020"></script> 
</head>
<body>
 <div class="ClassNavi">
<a  href="/24shi/">二十四史</a> | <a href="/SiKuQuanShu/">四库全书</a> | <a href="http://www.guoxuedashi.com/gjtsjc/"><font  color="#FF0000">古今图书集成</font></a> | <a href="/renwu/">历史人物</a> | <a href="/ShuoWenJieZi/"><font  color="#FF0000">说文解字</a></font> | <a href="/chengyu/">成语词典</a> | <a  target="_blank"  href="http://www.guoxuedashi.com/jgwhj/"><font  color="#FF0000">甲骨文合集</font></a> | <a href="/yzjwjc/"><font  color="#FF0000">殷周金文集成</font></a> | <a href="/xiangxingzi/"><font color="#0000FF">象形字典</font></a> | <a href="/13jing/"><font  color="#FF0000">十三经索引</font></a> | <a href="/zixing/"><font  color="#FF0000">字体转换器</font></a> | <a href="/zidian/xz/"><font color="#0000FF">篆书识别</font></a> | <a href="/jinfanyi/">近义反义词</a> | <a href="/duilian/">对联大全</a> | <a href="/jiapu/"><font  color="#0000FF">家谱族谱查询</font></a> | <a href="http://www.guoxuemi.com/hafo/" target="_blank" ><font color="#FF0000">哈佛古籍</font></a> 
</div>

 <!-- 头部导航开始 -->
<div class="w1180 head clearfix">
  <div class="head_logo l"><a title="国学大师官网" href="http://www.guoxuedashi.com" target="_blank"></a></div>
  <div class="head_sr l">
  <div id="head1">
  
  <a href="http://www.guoxuedashi.com/zidian/bujian/" target="_blank" ><img src="http://www.guoxuedashi.com/img/top1.gif" width="88" height="60" border="0" title="部件查字,支持20万汉字"></a>


<a href="http://www.guoxuedashi.com/help/yingpan.php" target="_blank"><img src="http://www.guoxuedashi.com/img/top230.gif" width="600" height="62" border="0" ></a>


  </div>
  <div id="head3"><a href="javascript:" onClick="javascript:window.external.AddFavorite(window.location.href,document.title);">添加收藏</a>
  <br><a href="/help/setie.php">搜索引擎</a>
  <br><a href="/help/zanzhu.php">赞助本站</a></div>
  <div id="head2">
 <a href="http://www.guoxuemi.com/" target="_blank"><img src="http://www.guoxuedashi.com/img/guoxuemi.gif" width="95" height="62" border="0" style="margin-left:2px;" title="国学迷"></a>
  

  </div>
</div>
  <div class="clear"></div>
  <div class="head_nav">
  <p><a href="/">首页</a> | <a href="/ShuKu/">国学书库</a> | <a href="/guji/">影印古籍</a> | <a href="/shici/">诗词宝典</a> | <a   href="/SiKuQuanShu/gxjx.php">精选</a> <b>|</b> <a href="/zidian/">汉语字典</a> | <a href="/hydcd/">汉语词典</a> | <a href="http://www.guoxuedashi.com/zidian/bujian/"><font  color="#CC0066">部件查字</font></a> | <a href="http://www.sfds.cn/"><font  color="#CC0066">书法大师</font></a> | <a href="/jgwhj/">甲骨文</a> <b>|</b> <a href="/b/4/"><font  color="#CC0066">解密</font></a> | <a href="/renwu/">历史人物</a> | <a href="/diangu/">历史典故</a> | <a href="/xingshi/">姓氏</a> | <a href="/minzu/">民族</a> <b>|</b> <a href="/mz/"><font  color="#CC0066">世界名著</font></a> | <a href="/download/">软件下载</a>
</p>
<p><a href="/b/"><font  color="#CC0066">历史</font></a> | <a href="http://skqs.guoxuedashi.com/" target="_blank">四库全书</a> |  <a href="http://www.guoxuedashi.com/search/" target="_blank"><font  color="#CC0066">全文检索</font></a> | <a href="http://www.guoxuedashi.com/shumu/">古籍书目</a> | <a   href="/24shi/">正史</a> <b>|</b> <a href="/chengyu/">成语词典</a> | <a href="/kangxi/" title="康熙字典">康熙字典</a> | <a href="/ShuoWenJieZi/">说文解字</a> | <a href="/zixing/yanbian/">字形演变</a> | <a href="/yzjwjc/">金 文</a> <b>|</b>  <a href="/shijian/nian-hao/">年号</a> | <a href="/diming/">历史地名</a> | <a href="/shijian/">历史事件</a> | <a href="/guanzhi/">官职</a> | <a href="/lishi/">知识</a> <b>|</b> <a href="/zhongyi/">中医中药</a> | <a href="http://www.guoxuedashi.com/forum/">留言反馈</a>
</p>
  </div>
</div>
<!-- 头部导航END --> 
<!-- 内容区开始 --> 
<div class="w1180 clearfix">
  <div class="info l">
   
<div class="clearfix" style="background:#f5faff;">
<script src='http://www.guoxuedashi.com/img/headersou.js'></script>

</div>
  <div class="info_tree"><a href="http://www.guoxuedashi.com">首页</a> > <a href="/SiKuQuanShu/fanti/">四库全书</a>
 > <h1>资治通鉴</h1> <!--         下载:【右键另存为】即可 --></div>
  <div class="info_content zj clearfix">
  
<div class="info_txt clearfix" id="show">
<center style="font-size:24px;">60-資治通鑑卷五十九</center>
    資治通鑑卷五十九   宋 司馬光 撰<br />
<br />
  胡三省 音註<br />
<br />
  漢紀五十一【起著雍執徐盡上章敦牂凡三年】<br />
<br />
  孝靈皇帝下<br />
<br />
  中平五年春正月丁酉赦天下 二月有星孛于紫宫【紫宫即太微也匡衛十二星之内皆曰紫宫天子之宫也孛蒲内翻】 黄巾餘賊郭大等起於河西白波谷【帝紀作西河當從之又按宋白續通典河南府河清縣今理白波鎮無以此谷於孟津為河西歟】寇太原河東 三月屠各胡攻殺并州刺史張懿【屠直於翻】 太常江夏劉焉見王室多故建議以為四方兵寇由刺史威輕既不能禁且用非其人以致離叛宜改置牧伯選清名重臣以居其任焉内欲求交阯牧【以交阯僻遠可以避禍也】侍中廣漢董扶【扶學圖䜟何進薦之徵拜侍中】私謂焉曰京師將亂益州分野有天子氣【蔡邕月令章句自危十度至壁八度謂之豕韋之次衛之分野自壁八度至胃一度謂之降婁之次魯之分野自胃一度至畢六度謂之大梁之次趙之分野自畢六度至井十度謂之實沈之次晉之分野自井十度至柳三度謂之鶉首之次秦之分野自柳三度至張十二度謂之鶉火之次周之分野自張十二度至軫六度謂之鶉尾之次楚之分野自軫六度至亢八度謂之壽星之次鄭之分野自亢八度至尾四度謂之大火之次宋之分野自尾四度至斗六度謂之析木之次燕之分野自斗六度至須女二度謂之星紀之次越之分野自須女二度至危十度謂之玄枵之次齊之分野晉書天文志用後魏太史令陳卓所言郡國所入宿受今亦載之自軫十二度至氐四度為壽星於辰在辰鄭分屬兖州自氐五度至尾九度為大火於辰在卯宋分屬豫州自尾十度至南斗十一度為析木於辰在寅燕分屬幽州自南斗十二度至須女七度為星紀於辰在丑吳越分屬揚州自須女八度至危十五度為玄枵於辰在子齊分屬青州自危十六度至奎四度為諏訾於辰在亥衛分屬并州自奎五度至胃六度為降婁於辰在戌魯分屬徐州自胃七度至畢十一度為大梁於辰在酉趙分屬冀州自畢十二度至東井十五度為實沈於辰在申魏分屬益州自東井十六度至柳八度為鶉首於辰在未秦分屬雍州自柳九度至張十六度為鶉火於辰在午周分屬三河自張十七度至軫十一度為鶉尾於辰在已楚分屬荆州分扶問翻】焉乃更求益州會益州刺史郤儉賦斂煩擾謡言遠聞【郤乞逆翻春秋晉大夫郤氏 考異曰范書作郗儉今從陳壽蜀志斂力贍翻聞音問】而耿鄙張懿皆為盜所殺朝廷遂從焉議選列卿尚書為州牧各以本秩居任【列卿秩中二千石尚書秩六百石耳東都以後尚書職任重於列卿】以焉為益州牧太僕黄琬為豫州牧宗正東海劉虞為幽州牧州任之重自此而始焉魯恭王之後虞東海恭王之五世孫也虞嘗為幽州刺史民夷懷其恩信故用之董扶及太倉令趙韙【百官志大倉令秩六百石主受郡國傳漕穀屬大司農韙羽鬼翻】皆棄官隨焉入蜀 詔發南匈奴兵配劉虞討張純單于羌渠遣左賢王將騎詣幽州國人恐發兵無已於是右部䤈落反【建武中右部薁犍曰逐王比來降立為䤈落尸逐鞮單于右部䤈落者蓋其支庶分居右部因以為種落之號䤈馨兮翻】與屠各胡合【屠直於翻】凡十餘萬人攻殺羌渠 【考異曰帝紀屠各胡攻殺并州刺史張懿遂與南匈奴左部胡合殺其單于今從匈奴傳】國人立其子右賢王於扶羅為持至尸逐侯單于【賢曰於扶羅卽前趙劉淵之祖也是為亂晉之首】 夏四月太尉曹嵩罷 五月以永樂少府南陽樊陵為太尉【樂音洛】六月罷 益州賊馬相趙祇等起兵緜竹【緜竹縣屬廣漢郡賢曰故城在今益州緜竹縣東】自號黄巾殺刺史郤儉進擊巴郡犍為旬月之間破壞三郡【犍居言翻壞音怪】有衆數萬自稱天子州從事賈龍率吏民攻相等數日破走州界清静龍乃選吏卒迎劉焉焉徙治緜竹撫納離叛務行寛惠以收人心【為劉焉專制益州張本】 郡國七大水 故太傳陳蕃子逸與術士襄楷會於冀州刺史王芬坐【坐才卧翻】楷曰天文不利宦者黄門常侍眞族滅矣逸喜芬曰若然者芬願驅除因與豪傑轉相招合上書言黑山賊攻劫郡縣欲因以起兵會帝欲北廵河間舊宅【帝先為解瀆亭侯有舊宅在河間】芬等謀以兵徼劫【徼讀曰邀】誅諸常侍黄門因廢帝立合肥侯以其謀告議郎曹操【以此謀告操蓋亦知操之為時雄矣】操曰夫廢立之事天下之至不祥也古人有權成敗計輕重而行之者伊霍是也【此等語豈常人所能及哉】伊霍皆懷至忠之誠據宰輔之埶因秉政之重同衆人之欲故能計從事立今諸君徒見曩者之易【易以䜴翻】未覩當今之難而造作非常欲望必克不以危乎芬又呼平原華歆陶丘洪共定計【華戶化翻姓譜堯子丹朱居陶丘其後氏焉】洪欲行歆止之曰夫廢立大事伊霍之所難芬性疎而不武此必無成洪乃止會北方夜半有赤氣東西竟天太史上言北方有陰謀【上時掌翻】不宜北行帝乃止敕芬罷兵俄而徵之芬懼解印綬亡走至平原自殺【綬音受】 秋七月以射聲校尉馬日磾為太尉日磾融之族孫也【磾丁奚翻】 八月初置西園八校尉【校戶教翻】以小黄門蹇碩為上軍校尉【姓譜蹇姓也左傳有秦大夫蹇叔】虎賁中郎將袁紹為中軍校尉屯騎校尉鮑鴻為下軍校尉議郎曹操為典軍校尉趙融為助軍左校尉馮芳為助軍右校尉諫議大夫夏牟為左校尉淳于瓊為右校尉皆統於蹇碩 【考異曰范書袁紹傳紹為佐軍校尉何進傳淳于瓊為佐軍校尉今從樂資山陽公載記】帝自黄巾之起留心戎事碩壯健有武畧帝親任之雖大將軍亦領屬焉 九月司徒許相罷以司空丁宫為司徒光禄勲南陽劉弘為司空 以衛尉條侯董重為票騎將軍重永樂太后兄子也【票匹妙翻樂音洛】 冬十月青徐黄巾復起【復扶又翻】寇郡縣 望氣者以為京師當有大兵兩宮流血帝欲厭之【厭一葉翻】乃大發四方兵講武於平樂觀下【水經註穀水自白馬寺東南逕平樂觀在上西門外樂音洛觀古玩翻】起大壇上建十二重華蓋蓋高十丈壇東北為小壇復建九重華蓋高九丈列步騎數萬人結營為陳【重直龍翻高居傲翻陳讀曰陣下同】甲子帝親出臨軍駐大華蓋下大將軍進駐小華蓋下帝躬擐甲介馬【賢曰擐貫也音官介亦甲也】稱無上將軍行陳三币而還【行下孟翻帀作答翻】以兵授進帝問討虜校尉蓋勲曰【蓋古盍翻】吾講武如是何如對曰臣聞先王耀德不觀兵【國語載祭公謀父之言】今寇在遠而設近陳不足以昭果毅【左傳曰戎昭果毅以聽之謂武殺敵為果致果為毅】祇黷武耳帝曰善恨見君晩羣臣初無是言也勲謂袁紹曰上甚聰明但蔽於左右耳與紹謀共誅嬖倖【嬖卑義翻又必計翻 考異曰勲傳云勲時與宗正劉虞佐軍校尉袁紹同典禁兵勲謂虞紹云云按虞於匈奴未叛之前已為幽州牧又宗正非典兵之官今除之】蹇碩懼出勲為京兆尹 十一月王國圍陳倉詔復拜皇甫嵩為左將軍【復扶又翻】督前將軍董卓合兵四萬人以拒之 張純與丘力居鈔略青徐幽冀四州【鈔楚交翻】詔騎都尉公孫瓚討之瓚與戰於屬國石門【屬國遼東屬國也賢曰石門山谷在今營州柳城縣西南瓚藏早翻】純等大敗棄妻子踰塞走悉得所略男女瓚深入無繼反為丘力居等所圍於遼西管子城二百餘日糧盡衆潰士卒死者什五六 董卓謂皇甫嵩曰陳倉危急請速救之嵩曰不然百戰百勝不如不戰而屈人兵陳倉雖小城守固備未易可拔【易以豉翻】王國雖強攻陳倉不下其衆必疲疲而擊之全勝之道也將何救焉國攻陳倉八十餘日不拔<br />
<br />
  六年春二月國衆疲敝解圍去皇甫嵩進兵擊之董卓曰不可兵法窮寇勿廹歸衆勿追【賢曰司馬兵法之言】嵩曰不然前吾不擊避其鋭也今而擊之待其衰也所擊疲師非歸衆也國衆且走莫有鬭志以整擊亂非窮寇也遂獨進擊之使卓為後拒連戰大破之斬首萬餘級卓大慙恨由是與嵩有隙【為後獻帝初平二年卓怖嵩張本】韓遂等共廢王國而劫故信都令漢陽閻忠使督統諸部忠病死遂等稍爭權利更相殺害【更工衡翻】由是寖衰 幽州牧劉虞到部遣使至鮮卑中告以利害責使送張舉張純首厚加購賞丘力居等聞虞至喜各遣譯自歸舉純走出塞餘皆降散虞上罷諸屯兵【上時掌翻奏也】但留降虜校尉公孫瓚將步騎萬人屯右北平【瓚以石門之捷自騎都尉拜降虜校尉降戶江翻校戶教翻】三月張純客王政殺純送首詣虞公孫瓚志欲掃滅烏桓而虞欲以恩信招降由是與瓚有隙【為後初平四年瓚殺虞張本】夏四月丙子朔日有食之 太尉馬日磾免遣使即拜幽州牧劉虞為太尉封容丘侯【容丘縣屬東海郡 考異曰袁紀三月己丑光禄劉虞為司馬領幽州牧今從范書】 蹇碩忌大將軍進與諸常侍共說帝遣進西擊韓遂【說輸芮翻】帝從之進陰知其謀奏遣袁紹收徐兗二州兵須紹還而西以稽行期 初帝數失皇子【數所角翻】何皇后生子辯養於道人史子眇家號曰史侯【賢曰道人謂有道術之人】王美人生子協董太后自養之號曰董侯羣臣請立太子帝以辯輕佻無威儀【佻初彫翻輕薄也】欲立協猶豫未决會疾篤屬協於蹇碩【屬之欲翻託也】丙辰帝崩于嘉德殿【年三十四嘉德殿在南宮九龍門内】碩時在内欲先誅何進而立協使人迎進欲與計事進即駕往碩司馬潘隱與進早舊迎而目之進驚馳從儳道歸營【廣雅曰儳疾也仕鍳翻】引兵入屯百郡邸【天下郡國百餘皆置邸京師謂之百郡邸者百郡總為一邸也】因稱疾不入戊午皇子辯即皇帝位年十四 【考異曰帝紀云年十七張璠漢紀曰帝年十四今從之】尊皇后曰皇太后太后臨朝【朝直遥翻下同】赦天下改元為光熹封皇弟協為勃海王協年九歲以後將軍袁隗為太傳與大將軍何進參錄尚書事進既秉朝政忿蹇碩圖已陰規誅之袁紹因進親客張津勸進悉誅諸宦官進以袁氏累世貴寵【袁安為司空司徒子敞為司空孫湯為司空司徒太尉湯子逢為司空少子隗亦為三公是累世貴寵也】而紹與從弟虎賁中郎將術皆為豪桀所歸信而用之【從才用翻下同】復博徵智謀之士【復扶又翻】何顒荀攸及河南鄭泰等二十餘人以顒為北軍中候攸為黄門侍郎【百官志給事黄門侍郎六百石掌侍從左右給事中關通内外獻帝起居注曰帝初即位令侍中給事黄門侍郎員各六人出入禁中近侍帷幄省尚書事蓋前無定員至帝始定員數也顒魚容翻】泰為尚書與同腹心攸爽之從孫也蹇碩疑不自安與中常侍趙忠宋典等書曰大將軍兄弟秉國專朝今與天下黨人謀誅先帝左右掃滅我曹但以碩典禁兵故且沈吟今宜共閉上閤【上閤省閤也沈持林翻】急捕誅之中常侍郭勝進同郡人也太后及進之貴幸勝有力焉 【考異曰袁紀作郭脉九州春秋作郎勝今從何進傳】故親信何氏與趙忠等議下從碩計而以其書示進庚午進使黄門令收碩誅之因悉領其屯兵票騎將軍董重【票匹妙翻】與何進權勢相害中官挾重以為黨助董太后每欲參干政事何太后輒相禁塞【塞猶遏也塞悉則翻】董后忿恚詈曰汝今輈張怙汝兄耶【恚於避翻賢曰輈張猶彊梁也兄謂進也輈音舟】吾敕票騎斷何進頭如反手耳【斷丁管翻】何太后聞之以告進五月進與三公共奏孝仁皇后使故中常侍夏惲等交通州郡辜較財利悉入西省【夏戶雅翻惲於粉翻較讀曰榷西省即謂永樂宫司】故事蕃后不得留京師【賢曰蕃后謂平帝母衛姬王莽攝政恐其專權后不得留在京師故以為故事也】請遷宫本國奏可辛巳進舉兵圍票騎府收董重免官自殺六月辛亥董太后憂怖暴崩【怖普布翻 考異曰九州春秋曰太后憂懼自殺今從皇后紀】民間由是不附何氏 辛酉葬孝靈皇帝于文陵【賢曰在雒陽西北二十里】何進懲蹇碩之謀稱疾不入陪喪又不送山陵 大水 秋七月徙勃海王協為陳留王 司徒丁宫罷 袁紹復說何進曰【復扶又翻說輸芮翻】前竇武欲誅内寵而反為所害者但坐言語漏泄五營兵士皆畏服中人而竇氏反用之自取禍滅【事見五十六卷建寧元年】今將軍兄弟並領勁兵【謂進及弟苖也】部曲將吏皆英俊名士樂盡力命【欒音洛】事在掌握此天贊之時也將軍宜一為天下除患以垂名後世不可失也【為于偽翻下同】進乃白太后請盡罷中常侍以下以三署郎補其處太后不聽曰中官統領禁省自古及今漢家故事不可廢也且先帝新棄天下我奈何楚楚與士人共對事乎【楚詞註曰楚楚鮮明貌詩曰衣裳楚楚】進難違太后意且欲誅其放縱者紹以為中官親近至尊【近其靳翻】出納號令今不悉廢後必為患而太后母舞陽君及何苖數受諸宦官賂遺【數所角翻下同遺于季翻】知進欲誅之數白太后為其障蔽又言大將軍專殺左右擅權以弱社稷太后疑以為然進新貴素敬憚中官雖外慕大名而内不能斷【斷丁亂翻下同】故事久不决紹等又為畫策多召四方猛將及諸豪傑使並引兵向京城以脅太后進然之主簿廣陵陳琳諫曰諺稱掩目捕雀夫微物尚不可欺以得志况國之大事其可以詐立乎今將軍總皇威握兵要龍驤虎步高下在心此猶鼓洪爐燎毛髪耳但當速發雷霆行權立斷則天人順之而反委釋利器【利器謂兵柄也】更徵外助大兵聚會彊者為雄所謂倒持干戈授人以柄功必不成秖為亂階耳進不聽典軍校尉曹操聞而笑曰宦者之官古今宜有但世主不當假之權寵使至於此既治其罪【治直之翻】當誅元惡一獄吏足矣何至紛紛召外兵乎欲盡誅之事必宣露吾見其敗也初靈帝徵董卓為少府【據卓傳中平六年徵卓為少府蓋即是年也】卓上書言所將湟中義從及秦胡兵【將即亮翻從才用翻】皆詣臣言牢直不畢禀賜斷絶妻子饑凍牽挽臣車使不得行羌胡憋腸狗態【賢曰言羌胡心腸憋惡情態如狗也方言云憋惡也郭璞云憋怤急性也憋音芳列翻怤音芳于翻】臣不能禁止輒將順安慰增異復上【賢曰如其更增異志當復聞上洪氏隸釋曰漢靈帝建寧二年魯相史晨祠孔廟奏後云增異輒上光和二年樊毅復華下民租口筭奏後云增異復上此蓋當時奏文結末之常語蓋言繼今事有增於此者異於此者將復上奏也復扶又翻上時掌翻】朝廷不能制及帝寢疾璽書拜卓并州牧【璽斯氏翻】令以兵屬皇甫嵩卓復上書言臣誤蒙天恩掌戎十年士卒大小相狎彌久戀臣畜養之思為臣奮一旦之命【畜許六翻為于偽翻】乞將之北州效力邊垂【將如字又即亮翻之往也】嵩從子酈說嵩曰【從才用翻酈音歷 考異曰袁紀作從子邐今從范書】 天下兵柄在大人與董卓耳今怨隙已結勢不俱存卓被詔委兵而上書自請此逆命也彼度京師政亂【被皮義翻度徒洛翻】故敢躊躇不進此懷姦也二者刑所不赦且其凶戾無親將士不附大人今為元帥【嵩討王國時為督故曰元帥】杖國威以討之上顯忠義下除凶害無不濟也嵩曰違命雖罪專誅亦有責也【卓不釋兵為違命嵩擅討卓為專誅】不如顯奏其事使朝廷裁之乃上書以聞帝以讓卓卓亦不奉詔駐兵河東以觀時變何進召卓使將兵詣京師 【考異曰進傳曰召卓屯關中上林苑按時卓已駐河東若屯上林則更為西去非所以脅太后也今從卓傳】侍御史鄭泰諫曰董卓彊忍寡義志欲無厭【厭於鹽翻】若借之朝政【借子夜翻】授以大事將恣凶欲以危朝廷明公以親德之重據阿衡之權秉意獨斷【斷丁亂翻】誅除有罪誠不宜假卓以為資援也且事留變生殷鑒不遠【謂竇武之事可為殷鹽也】宜在速决尚書盧植亦言不宜召卓進皆不從泰乃棄官去謂荀攸曰何公未易輔也【易以豉翻】進府掾王匡騎都尉鮑信皆泰山人進使還郷里募兵并召東郡太守橋瑁屯成臯【瑁音冒】使武猛都尉丁原將數千人寇河内燒孟津火照城中【賢曰武猛謂其有武藝而勇猛取其嘉名因以名官】皆以誅宦官為言董卓聞召即時就道并上書曰中常侍張讓等竊倖承寵濁亂海内臣聞揚湯止沸莫若去薪【去羌呂翻前書枚乘諫吳王曰欲湯之凔一人炊之百人揚之無益也不如絶薪止火而已凔音則亮翻寒也】潰癰雖痛勝於内食【言癰疽藴結破之雖痛勝於内食肌肉浸淫滋大也】昔趙鞅興晉陽之甲以逐君側之惡【公羊傳曰晉趙鞅取晉陽之甲以逐荀寅與士吉射者謁為君側之惡人也此逐君側之惡人曷為以叛言之無君命也】今臣輒鳴鐘鼓如雒陽【賢曰鳴鐘鼓者聲其罪也】請收讓等以清姦穢太后猶不從何苖謂進曰始共從南陽來俱以貧賤依省内以致富貴【言何后因宦官得進進兄弟以此致富貴也】國家之事亦何容易【易以豉翻】覆水不收宜深思之【水覆於地不可復收言事發則不可收拾】且與省内和也卓至澠池【澠彌兖翻】而進更狐疑使諫議大夫种邵宣詔止之卓不受詔遂前至河南【河南周之王城去雒陽不遠种音冲】邵迎勞之【勞力到翻】因譬令還軍卓疑有變使其軍士以兵脅邵邵怒稱詔叱之軍士皆披【披芳靡翻】遂前質責卓卓辭屈乃還軍夕陽亭【賢曰夕陽亭在河南城西】邵暠之孫也袁紹懼進變計因脅之曰交構已成形埶已露將軍復欲何待而不早决之乎事久變生復為竇氏矣【復扶又翻】進於是以紹為司隸校尉假節專命擊斷【漢司隸校尉持節至元帝時諸葛豐為司隸始去節今假紹節重其權也斷丁亂翻】從事中郎王允為河南尹紹使雒陽方畧武吏司察宦者而促董卓等使馳驛上奏欲進兵平樂觀【上時掌翻樂音洛觀古玩翻】太后乃恐悉罷中常侍小黄門使還里舍唯留進所私人以守省中諸常侍小黄門皆詣進謝罪唯所措置進謂曰天下匈匈正患諸君耳今董卓垂至諸君何不早各就國袁紹勸進便於此决之【勸進於此時悉誅之也】至于再三進不許紹又為書告諸州郡詐宣進意使捕案中官親屬進謀積日頗泄中官懼而思變張讓子婦太后之妺也讓向子婦叩頭曰老臣得罪當與新婦俱歸私門唯受恩累世【賢曰唯思念也】今當遠離宫殿【離力智翻】情懷戀戀願復一入直【復扶又翻下同】得暫奉望太后陛下顔色然後退就溝壑死不恨矣子婦言於舞陽君入白太后乃詔諸常侍皆復入直八月戊辰進入長樂宫【樂音洛】白太后請盡誅諸常侍中常侍張讓段珪相謂曰大將軍稱疾不臨喪不送葬今欻入省【賢曰欻音許勿翻】此意何為竇氏事竟復起邪使潛聽具聞其語乃率其黨數十人持兵竊自側闥入伏省戶下進出因詐以太后詔召進入坐省閣讓等詰進曰天下憒憒【詰去吉翻說文曰憒憒亂也古對翻】亦非獨我曹罪也先帝嘗與太后不快幾至成敗【事見上卷光和四年幾居希翻】我曹涕泣救解各出家財千萬為禮和悦上意但欲託卿門戶耳今乃欲滅我曹種族不亦太甚乎【種章勇翻】於是尚方監渠穆拔劍斬進於嘉德殿前【案百官志尚方有令丞而無監桓靈之世諸署令悉以宦者為之尚方監必亦置於是時也渠姓也左傳天王使宰渠伯糾來聘又衛有渠孔御戎】讓珪等為詔以故太尉樊陵為司隸校尉少府許相為河南尹尚書得詔板疑之曰請大將軍出共議中黄門以進頭擲與尚書曰何進謀反已伏誅矣進部曲將吳匡張璋在外聞進被害【被皮義翻】欲引兵入宫宫門閉虎賁中郎將袁術與匡共斫攻之中黄門持兵守閣會日暮術因燒南宫青瑣門【衛瓘曰青瑣門邊青鏤也一曰天子門内有眉格再重裏青畫曰瑣 考異曰何進傳作九龍門今從袁紀】欲以脅出讓等讓等入白太后言大將軍兵反燒宫攻尚書闥【尚書闥即尚書門】因將太后少帝及陳留王劫省内官屬從複道走北宫【將如字攜也挾也】尚書盧植執戈於閣道牕下仰數段珪【數所具翻】珪懼乃釋太后太后投閣乃免袁紹與叔父隗矯詔召樊陵許相斬之紹及何苖引兵屯朱雀闕下捕得趙忠等斬之吳匡等素怨苗不與進同心而又疑其與宦官通謀乃令軍中曰殺大將軍即車騎也【時苖為車騎將軍】吏士能為報讎乎【為于偽翻】皆流涕曰願致死匡遂引兵與董卓弟奉車都尉旻攻殺苗棄其屍於苑中紹遂閉北宫門勒兵捕諸宦者無少長皆殺之【少詩照翻長知兩翻】几二千餘人或有無須而誤死者【須古鬚字通】紹因進兵排宫或上端門屋以攻省内【宫之正南門曰端門省禁也】庚午張讓段珪等困廹遂將帝與陳留王數十人步出穀門【穀門位在子雒城正北門也】夜至小平津【賢曰小平津在今鞏縣西北杜佑曰鞏縣西北有小平縣故城又北有津曰小平津】六璽不自隨公卿無得從者【從才用翻】唯尚書盧植河南中部掾閔貢夜至河上【漢官儀諸郡置五郡督郵以監屬縣河南尹置四部督郵中部為掾掾俞絹翻】貢厲聲質責讓等且曰今不速死吾將殺汝因手劍斬數人【手式又翻】讓等惶怖【怖普布翻下同】叉手再拜叩頭向帝辭曰臣等死陛下自愛遂投河而死貢扶帝與陳留王夜步逐螢光南行欲還宫行數里得民家露車【露車者上無巾蓋四旁無帷裳蓋民家以載物者耳】共乘之至雒舍止【雒舍地名在北芒之北】辛未帝獨乘一馬陳留王與貢共乘一馬從雒舍南行公卿稍有至者董卓至顯陽苑【顯陽苑桓帝延熹二年所造作雒陽西】遠見火起知有變引兵急進未明到城西聞帝在北因與公卿往奉迎於北芒阪下帝見卓將兵卒至【將即亮翻卒讀曰猝】恐怖涕泣羣公謂卓曰有詔却兵卓曰公諸人為國大臣不能匡正王室至使國家播蕩【東都羣臣謂天子為國家】何却兵之有卓與帝語語不可了【了曉解也】乃更與陳留王語問禍亂由起王答自初至終無所遺失卓大喜以王為賢且為董太后所養卓自以與太后同族遂有廢立之意是日帝還宫赦天下改光熹為昭寧失傳國璽【為下獻帝初平二年孫堅得璽張本璽斯氏翻】餘璽皆得之以丁原為執金吾騎都尉鮑信自泰山募兵適至說袁紹曰【說輸芮翻】董卓擁強兵將有異志今不早圖必為所制及其新至疲勞襲之可禽也紹畏卓不敢發信乃引兵還泰山董卓之入也步騎不過三千自嫌兵少恐不為遠近所服率四五日輒夜潛出軍近營明且乃大陳旌鼓而還以為西兵復至【復扶又翻】雒中無知者俄而進及弟苖部曲皆歸於卓卓又陰使丁原部曲司馬五原呂布殺原而并其衆卓兵於是大盛乃諷朝廷以久雨策免司空劉弘而代之初蔡邕徙朔方【事見五十七卷光和元年】會赦得還五原太守王智甫之弟也奏邕謗訕朝廷邕遂亡命江海積十二年董卓聞其名而辟之稱疾不就卓怒詈曰我能族人邕懼而應命到署祭酒甚見敬重舉高第三日之間周歷三臺【邕舉高第補侍御史又轉治書御史遷尚書三日之間周歷三臺】遷為侍中 董卓謂袁紹曰天下之主宜得賢明每念靈帝令人憤毒【賢曰毒恨也】董侯似可今欲立之為能勝史侯否人有小智大癡亦知復何如為當且爾劉氏種不足復遺【且爾猶言且如此也卓意欲廢漢自立】紹曰漢家君天下四百許年恩澤深渥兆民戴之今上富於春秋未有不善宣於天下公欲廢嫡立庶恐衆不從公議也卓按劍叱紹曰豎子敢然【敢然猶言敢如此也】天下之事豈不在我我欲為之誰敢不從爾謂董卓刀為不利乎紹勃然曰天下健者豈惟董公引佩刀横揖徑出卓以新至見紹大家故不敢害紹縣節於上東門【縣所假司隸節也上東門位在寅賢曰雒陽城東面北頭門也縣讀曰懸】逃犇冀州九月癸酉卓大會百僚奮首而言曰皇帝闇弱不可以奉宗廟為天下主今欲依伊尹霍光故事更立陳留王何如【更工衡翻】公卿以下皆惶恐莫敢對卓又抗言曰【賢曰抗高也】昔霍光定策延年按劍【事見二十四卷昭帝元平元年】有敢沮大議皆以軍法從事【沮在呂翻】坐者震動尚書盧植獨曰昔太甲既立不明昌邑罪過千餘故有廢立之事今上富於春秋行無失德非前事之比也卓大怒罷坐將殺植蔡邕為之請【坐徂卧翻為于偽翻】議郎彭伯亦諫卓曰盧尚書海内大儒人之望也今先害之天下震怖【怖普布翻】卓乃止但免植官植遂逃隱於上谷卓以廢立議示太傅袁隗隗報如議甲戌卓復會羣僚於崇德前殿【復扶又翻】遂脅太后策廢少帝曰皇帝在喪無人子之心威儀不類人君今廢為弘農王立陳留王恊為帝袁隗解帝璽綬以奉陳留王扶弘農王下殿北面稱臣太后鯁涕【言不敢出聲但鯁咽而流涕也】羣臣含悲莫敢言者卓又議太后踧廹永樂宫【踧子六翻】至令憂死逆婦姑之禮【左傳曰婦養姑者也虧姑以成婦逆莫大焉】乃遷太后於永安宫赦天下改昭寧為永漢丙子卓酖殺何太后公卿以下不布服會葬素衣而已卓又發何苖棺出其尸支解節斷棄於道邊殺苖母舞陽君棄尸於苑枳落中【落籬落也枳似棘多刺江南為橘江北為枳人以栫籬】 詔除公卿以下子弟為郎以補宦官之職侍於殿上 乙酉以太尉劉虞為大司馬封襄賁侯【襄賁縣屬東海郡應劭曰賁音肥】董卓自為太尉領前將軍事加節傳斧鉞虎賁更封郿侯【傳知戀翻郿縣屬扶風賢曰今岐州縣師古曰郿音媚】 丙戌以太中大夫楊彪為司空 甲午以豫州牧黄琬為司徒 董卓率諸公上書追理陳蕃竇武及諸黨人悉復其爵位遣使弔祠擢用其子孫 自六月雨至於是月 冬十月乙巳葬靈思皇后 白波賊寇河東 【考異曰帝紀五年九月南單于叛與白波賊寇河東案匈奴傳帝崩之後於扶羅乃與白波賊為寇紀誤今從傳】董卓遣其將牛輔擊之初南單于於扶羅既立國人殺其父者遂叛【單于羌渠被殺事見上卷中平五年】共立須卜骨都侯為單于於扶羅詣闕自訟會靈帝崩天下大亂於扶羅將數千騎與白波賊合兵寇郡縣時民皆保聚鈔掠無利【鈔楚交翻】而兵遂挫傷復欲歸國國人不受乃止河東平陽須卜骨都侯為單于一年而死南庭遂虚其位以老王行國事 十一月以董卓為相國【漢自蕭何為相國後不復除拜】贊拜不名入朝不趨劍履上殿 十二月戊戌以司徒黄琬為太尉司空楊彪為司徒光禄勲荀爽為司空初尚書武威周毖城門校尉汝南伍瓊說董卓矯桓靈之政擢用天下名士以收衆望卓從之【毖兵媚翻說輸芮翻 考異曰范書云吏部尚書漢陽周珌侍中汝南伍瓊袁紀作侍中周毖今從魏志及英雄記】命毖瓊與尚書鄭泰長史何顒等沙汰穢惡顯拔幽滯於是徵處士荀爽陳紀韓融申屠蟠【處昌呂翻】復就拜爽平原相【復扶又翻】行至宛陵【宛陵縣屬河南尹在雒陽東】遷光禄勲視事三日進拜司空自被徵命及登台司凡九十三日又以紀為五官中郎將融為大鴻臚紀寔之子融韶之子也爽等皆畏卓之暴無敢不至獨申屠蟠得徵書人勸之行蟠笑而不答卓終不能屈年七十餘以壽終卓又以尚書韓馥為冀州牧侍中劉岱為兖州刺史陳留孔伷為豫州刺史【伷音胄 考異曰九州春秋作孔胄今從董卓傳】東平張邈為陳留太守潁川張咨為南陽太守卓所親愛並不處顯職但將校而已【將校謂中郎將校尉處昌呂翻】 詔除光熹昭寧永漢三號【除三號復稱中平六年】 董卓性殘忍一旦專政據有國家甲兵珍寶威震天下所願無極語賓客曰我相貴無上也【自言非人臣之相其悖逆如此語牛倨翻相息亮翻】侍御史擾龍宗詣卓白事不解劍【擾龍姓也蓋古擾龍氏之後】立檛殺之【檛側瓜翻】是時雒中貴戚室第相望金帛財產家家充積卓縱放兵士突其廬舍剽虜資物【剽匹妙翻】妻畧婦女不避貴戚人情崩恐不保朝夕卓購求袁紹急周毖伍瓊說卓曰夫廢立大事非常人所及袁紹不逹大體恐懼出奔非有它志今急購之勢必為變袁氏樹恩四世【袁安四世至紹】門生故吏徧於天下若收豪桀以聚徒衆英雄因之而起則山東非公之有也不如赦之拜一郡守紹喜於免罪必無患矣卓以為然乃即拜紹勃海太守封邟鄉侯【邟苦浪翻】又以袁術為後將軍曹操為驍騎校尉術畏卓出犇南陽操變易姓名間行東歸過中牟【中牟縣屬河南尹間古莧翻】為亭長所疑執詣縣時縣已被卓書【被皮義翻】唯功曹心知是操以世方亂不宜拘天下雄雋因白令釋之【白中牟令也】操至陳留散家財合兵得五千人是時豪傑多欲起兵討卓者袁紹在勃海冀州牧韓馥遣數部從事守之不得動揺【部從事部郡國從事也勃海一郡遣部從事數人守之恐紹起兵也】東郡太守橋瑁【瑁莫報翻】詐作京師三公移書與州郡陳卓罪惡云見逼廹無以自救企望義兵解國患難【企欺冀翻難乃旦翻】馥得移請諸從事問曰今當助袁氏邪助董氏邪治中從事劉子惠曰今興兵為國【為于偽翻】何謂袁董馥冇慙色子惠復言兵者凶事不可為首今宜往視他州有發動者然後和之【復扶又翻和戶卧翻】冀州於他州不為弱也他人功未有在冀州之右者也馥然之馥乃作書與紹道卓之惡聽其舉兵 【考異曰范書魏志俱有此事范書在舉兵之後魏志在舉兵之前若在舉兵後時紹已為盟主馥何敢禁其發兵若在舉兵前則近是也今從魏志】孝獻皇帝甲【諱協謚法聰明睿智曰獻古今注協之字曰合張璠記曰靈帝以帝似已故名曰協帝王紀曰協字伯和蜀謚帝曰愍魏謚帝曰獻此從魏謚者以魏受漢禪為正也】<br />
<br />
  初平元年春正月關東州郡皆起兵以討董卓推勃海太守袁紹為盟主紹自號車騎將軍諸將皆板授官號【時卓挾天子紹等罔攸禀命故權宜板授官號】紹與河内太守王匡屯河内冀州牧韓馥留鄴給其軍糧豫州刺史孔伷屯潁川兖州刺史劉岱陳留太守張邈邈弟廣陵大守超東郡太守橋瑁山陽太守袁遺濟北相鮑信與曹操俱屯酸棗【酸棗縣屬陳留郡瑁音冒】後將軍袁術屯魯陽【魯陽縣屬南陽郡】衆各數萬豪傑多歸心袁紹者鮑信獨謂曹操曰夫畧不世出能撥亂反正者君也苟非其人雖彊必斃君殆天之所啟乎辛亥赦天下 癸酉董卓使郎令李儒酖殺弘農王<br />
<br />
  辯 卓議大發兵以討山東尚書鄭泰曰夫政在德不在衆也卓不悦曰如卿此言兵為無用邪泰曰非謂其然也以為山東不足加大兵耳明公出自西州少為將帥閑習軍事【少詩照翻】袁本初公卿子弟生處京師張孟卓東平長者坐不闚堂【處昌呂翻長知兩翻張邈字孟卓賢曰坐不闚堂言不妄視也】孔公緒清談高論噓枯吹生【孔伷字公緒賢曰枯者嘘之使生生者吹之使枯言談論有所抑揚也】並無軍旅之才臨鋒决敵非公之儔也【謂臨兵鋒而與敵人决勝負也】况王爵不加尊卑無序若恃衆怙力將各棊峙以觀成敗不肯同心共膽與齊進退也【此數語公業雖以以釋言於卓然關東諸將情態實不過如此】且山東承平日久民不習戰關西頃遭羌寇婦女皆能挾弓而鬭天下所畏者無若并凉之人與羌胡義從【從才用翻】而明公擁之以為爪牙譬猶驅虎兕以赴犬羊【兕序姊翻似牛一角而青色身重千斤角重百斤】鼓烈風以掃枯葉誰敢禦之無事徵兵以驚天下使患役之民相聚為非棄德恃衆自虧威重也卓乃悦 董卓以山東兵盛欲遷都以避之公卿皆不欲而莫敢言【畏其暴也】卓表河南尹朱儁為太僕以為已副使者召拜儁辭不肯受因曰國家西遷必孤天下之望【孤負也】以成山東之釁臣不知其可也使者曰召君受拜而君拒之不問徙事而君陳之何也儁曰副相國非臣所堪也遷都非計事所急也辭所不堪言其所急臣之宜也由是止不為副卓大會公卿議曰高祖都關中十有一世光武宫雒陽於今亦十一世矣案石包䜟【當時緯書之外又有石包室䜟盖時人附益為之如孔子閉房記之類】宜徙都長安以應天人之意百官皆默然司徒楊彪曰移都改制天下大事故盤庚遷亳殷民胥怨【書序曰盤庚五遷將治亳殷民咨胥怨】昔關中遭王莽殘破故光武更都雒邑【更工衡翻】歷年已久百姓安樂【樂音洛下同】今無故捐宗廟棄園陵恐百姓驚動必有糜沸之亂【賢曰如糜粥之沸也詩云如沸如羮】石包䜟妖邪之書豈可信用卓曰關中肥饒故秦得并吞六國且隴右材木自出杜陵有武帝陶竈并功營之可使一朝而辦百姓何足與議若有前却我以大兵驅之可令詣滄海【賢曰言不敢避險難也】彪曰天下動之至易【易以䜴翻】安之甚難惟明公慮焉卓作色曰公欲沮國計邪太尉黄琬曰此國之大事楊公之言得無可思卓不答司空荀爽見卓意壯恐害彪等因從容言曰【從才容翻】相國豈樂此邪【樂音洛】山東兵起非一日可禁故當遷以圖之此秦漢之勢也【謂秦漢都關中因山河形埶以制天下】卓意小解琬退又為駁議【駁北角翻】二月乙亥卓以災異奏免琬彪等以光禄勲趙謙為太尉太僕王允為司徒城門校尉伍瓊督軍校尉周毖固諫遷都卓大怒曰卓初入朝二君勸用善士故卓相從而諸君到官舉兵相圖此二君賣卓卓何用相負庚辰收瓊毖斬之楊彪黄琬恐懼詣卓謝卓亦悔殺瓊毖乃復表彪琬為光禄大夫【復扶又翻】 卓徵京兆尹蓋勲為議郎【蓋古盍翻】時左將軍皇甫嵩將兵三萬屯扶風【潘岳關中記曰三輔舊治長安城中長吏各在其縣治民光武東都之後扶風出治槐里馮翊出治高陵】勲密與嵩謀討卓會卓亦徵嵩為城門校尉嵩長史梁衍說嵩曰董卓寇掠京邑廢立從意今徵將軍大則危禍小則困辱今及卓在雒陽天子來西以將軍之衆迎接至尊奉令討逆徵兵羣帥【說輸芮翻帥所類翻】袁氏逼其東將軍廹其西此成禽也嵩不從遂就徵【嵩前不能從兄子酈之言今又不從衍之策自揣其才不足以制卓故也】勲以衆弱不能獨立亦還京師卓以勲為越騎校尉河南尹朱儁為卓陳軍事卓折儁曰我百戰百勝决之於心卿勿妄說且汙我刀【為于偽翻折之舌翻汙烏故翻】蓋勲曰昔武丁之明猶求箴諫【賢曰武丁殷王高宗也謂傅說曰啟乃心沃朕心說復于王曰惟木從繩則正后從諫則聖余謂蓋勲忠直之士時卓方謀僭逆不應以武丁之事為言據國語楚左史倚相曰昔衛武公年數九十有五矣猶箴儆於國曰毋謂我老耄而捨我必恭恪於朝朝夕以交戒我聞一二之言必誦志而納之以訓道我及其沒也謂之叡聖武公勲蓋以衛武公之事責卓也史書傳寫誤以公為丁耳】况如卿者而欲杜人之口乎卓乃謝之 卓遣軍至陽城值民會於社下【此二月事也陽城縣屬潁川郡】悉就斬之駕其車重【重直用翻】載其婦女以頭繫車轅歌呼還雒云攻賊大獲卓焚燒其頭以婦女與甲兵為婢妾【甲兵謂甲兵之士】 丁亥車駕西遷董卓收諸富室以罪惡誅之没入其財物死者不可勝計【勝音升】悉驅徙其餘民數百萬口於長安步騎驅蹙更相蹈藉【藉慈夜翻】饑餓寇掠積尸盈路卓自留屯畢圭苑中悉燒宫廟官府居家二百里内室屋蕩盡無復雞犬又使呂布發諸帝陵及公卿以下冢墓收其珍寶卓獲山東兵以猪膏塗布十餘匹用纒其身然後燒之先從足起 三月乙巳車駕入長安 【考異曰袁紀作己巳今從范書】居京兆府舍【師古曰三輔黄圖曰京兆府在尚冠前街東入故中尉府】後乃稍葺宫室而居之時董卓未至朝政大小皆委之王允允外相彌縫内謀王室甚有大臣之度自天子及朝中皆倚允允屈意承卓卓亦雅信焉【朝直遥翻】 董卓以袁紹之故戊午殺太傳袁隗太僕袁基及其家尺口以上五十餘人【尺口謂嬰孩也】初荆州刺史王叡【裴松之曰叡晉太保祥伯父也】與長沙太守孫堅共<br />
<br />
  擊零桂賊【零桂零陵桂陽也】以堅武官言頗輕之及州郡舉兵討董卓叡與堅亦皆起兵叡素與武陵太守曹寅不相能揚言當先殺寅寅懼詐作按行使者檄移堅說叡罪過令收行刑訖以狀上【上時掌翻】堅承檄即勒兵襲叡叡聞兵至登樓望之遣問欲何為堅前部答曰兵久戰勞苦欲詣使君求資直耳【據吳錄資直者衣資之直也】叡見堅驚曰兵自求賞孫府君何以在其中堅曰被使者檄誅君【被皮義翻】叡曰我何罪堅曰坐無所知叡窮廹刮金飲之而死【陶弘景曰生金有毒不鍊服之殺人】堅前至南陽衆已數萬人南陽太守張咨不肯給軍糧誘而斬之【陳壽志曰堅以牛酒誘之吳歷曰堅詐疾以誘之】郡中震慄無求不獲前到魯陽【魯陽縣屬南陽郡】與袁術合兵術由是得據南陽 【考異曰范書術傳云劉表上術為南陽太守表傳云術阻兵屯魯陽表不得至荆州魏志術傳孫堅殺張咨術得據南陽魏武帝紀此年二月已云術屯南陽蓋術初奔魯陽此春孫堅取南陽術乃據之猶以魯陽為治所也】表堅行破虜將軍領豫州刺史詔以北軍中候劉表為荆州刺史時寇賊縱横道路梗塞【縱子容翻塞悉則翻】表單馬入宜城【賢曰宜城縣屬南郡本鄢惠帝三年改名宜城】請南郡名士蒯良蒯越與之謀曰今江南宗賊甚盛【賢曰宗黨共為賊】各擁衆不附若袁術因之禍必至矣吾欲徵兵恐不能集其策焉出【焉於䖍翻】蒯良曰衆不附者仁不足也附而不治者義不足也苟仁義之道行百姓歸之如水之趣下【趣七喻翻】何患徵兵之不集乎蒯越曰袁術驕而無謀宗賊帥多貪暴為下所患【帥所類翻下同】若使人示之以利必以衆來使君誅其無道撫而用之一州之人有樂存之心【樂音洛】聞君威德必襁負而至矣【襁居兩翻】兵集衆附南據江陵北守襄陽荆州八郡可傳檄而定【郡國志荆州郡南陽南郡江夏零陵桂陽長沙武陵七郡漢官儀以章陵足為八郡】公路雖至無能為也【袁術字公路】表曰善乃使越誘宗賊帥至者五十五人皆斬之而取其衆【誘音酉帥所類翻】遂徙治襄陽【荆州刺史本治武陵漢壽襄陽縣屬南郡】鎮撫郡縣江南悉平【荆部在江南者長沙武陵零陵桂陽四郡也為劉表專制荆州張本】 董卓在雒陽袁紹等諸軍皆畏其彊莫敢先進曹操曰舉義兵以誅暴亂大衆已合諸君何疑向使董卓倚王室據舊京東向以臨天下雖以無道行之猶足為患今焚燒宫室劫遷天子海内震動不知所歸此天亡之時也一戰而天下定矣遂引兵西將據成臯張邈遣將衛兹分兵隨之進至滎陽汴水【班志汴水在滎陽西南】遇卓將玄菟徐榮【莬同都翻】與戰操兵敗為流矢所中所乘馬被創【中竹仲翻被皮義翻創初良翻】從弟洪以馬與操操不受【從才用翻】洪曰天下可無洪不可無君遂步從操夜遁去榮見操所將兵少力戰盡日謂酸棗未易攻也【易以豉翻】亦引兵還操到酸棗諸軍十餘萬日置酒高會不圖進取操責讓之因為謀曰【為于偽翻下同】諸君能聽吾計使勃海引河内之衆臨孟津【勃海謂袁紹也】酸棗諸將守成臯據敖倉塞轘轅太谷全制其險【塞悉則翻轘音環】使袁將軍率南陽之軍軍丹析入武關【此謂袁術也丹水及析縣皆屬弘農郡】以震三輔皆高壘深壁勿與戰益為疑兵示天下形勢以順誅逆可立定也【觀操之計但欲形格勢禁待其變起於下耳非主于戰也】今兵以義動持疑不進失天下望竊為諸君恥之邈等不能用操乃與司馬沛國夏侯惇等詣揚州募兵得千餘人還屯河内【從袁紹也】頃之酸棗諸軍食盡衆散劉岱與橋瑁相惡岱殺瑁以王肱領東郡太守青州刺史焦和亦起兵討董卓【姓譜周武王封神農之後於焦後以國為氏】務及諸將西行【務進兵與酸棗諸將相及也】不為民人保障兵始濟河黄巾已入其境青州素殷實甲兵甚盛和每望寇犇北未嘗接風塵交旗鼓也性好卜筮信鬼神【好呼到翻】入見其人清談干雲出觀其政賞罸淆亂州遂蕭條悉為丘墟頃之和病卒袁紹使廣陵臧洪領青州以撫之 夏四月以幽州牧劉虞為大傳道路壅塞【塞悉則翻】信命竟不得通先是幽部應接荒外【荒外言荒服之外也先悉薦翻】資費甚廣歲常割青冀賦調二億有餘以足之【調徒弔翻】時處處斷絶委輸不至【委於偽翻輸舂遇翻】而虞敝衣繩屨食無兼肉務存寛政勸督農桑開上谷胡市之利通漁陽鹽鐵之饒【上谷舊有關市與胡人貿易漁陽舊有鹽官鐵官】民悦年登穀石三十青徐士庶避難歸虞者百餘萬口虞皆收視温卹為安立生業【難乃旦翻為于偽翻】流民皆忘其遷徙焉 五月司空荀爽薨 六月辛丑以光禄大夫种拂為司空拂邵之父也 董卓遣大鴻臚韓融少府陰脩執金吾胡母班將作大匠吳脩越騎校尉王瓌安集關東解譬袁紹等胡毋班吳脩王瓌至河内袁紹使王匡悉收擊殺之【瓌工回翻 考異曰謝承後傳漢書曰班王匡之妹夫班與匡書云僕與太傅馬公太僕趙岐少府陰脩俱受詔命關東諸郡雖實嫉卓猶以銜奉王命不敢玷辱而足下獨囚僕於獄欲以釁鼓此悖暴無道之甚者也按范書此年六月遣韓融等安集關東袁術王匡各執而殺之三年八月遣馬日磾及趙岐慰撫天下袁紀遣馬趙亦在三年八月時董卓已死而此書云與馬趙俱受詔又云恚卓遷怒自相乖迕疑非班書今不取】袁術亦殺陰脩惟韓融以名德免董卓壞五銖錢【賢曰光武中興除王莽貨泉更用五銖錢孔頴逹曰五銖者其重五銖凡十黍為一參十參為一銖二十四銖為一兩錢邊作五銖字壞音怪】更鑄小錢【更工衡翻】悉取雒陽及長安銅人鐘虡飛亷銅馬之屬以鑄之【銅人秦始皇所鑄也賢曰鐘虡以銅為之前書音義曰虡鹿頭龍身神獸也說文鐘鼓之跗以猛獸為飾也武帝置飛亷館音義曰飛亷神禽身似鹿頭如爵有角蛇尾文如豹文明帝永平五年迎取長安飛亷銅馬置上西門外名平樂館銅馬則東京門所作置於金馬門外者也余據馬援亦進銅馬虡音巨】由是貨賤物貴穀石至數萬錢 冬孫堅與官屬會飲於魯陽城東董卓步騎數萬猝至堅方行酒談笑整頓部曲無得妄動後騎漸益堅徐罷坐【坐才卧翻】導引入城乃曰向堅所以不即起者恐兵相蹈藉諸君不得入耳卓兵見其整不敢攻而還 王匡屯河陽津【河陽津即孟津】董卓襲擊大破之 左中郎將蔡邕議孝和以下廟號稱宗者皆宜省去以遵先典從之【禮祖冇功而宗有德和帝以下無德可宗故去之去羌呂翻 考異曰袁紀在明年今從范書】中郎將徐榮薦同郡故冀州刺史公孫度於董卓卓<br />
<br />
  以為遼東太守度到官以法誅滅郡中名豪大姓百餘家郡中震慄乃東伐高句驪【句如字又音駒驪力知翻】西擊烏桓語所親吏柳毅陽儀等曰【語牛倨翻姓譜柳本自魯孝公子子展之孫以王父字為氏至展禽食采於柳下因為氏】漢祚將絶當與諸卿圖王耳於是分遼東為遼西中遼郡各置太守越海收東萊諸縣置營州刺史自立為遼東侯平州牧立漢二祖廟承制郊祀天地藉田【杜佑曰藉借也謂借人力以理之勸率天下使務農也春秋傳曰郊而後耕遂藉人力以成歲功故謂之帝藉臣瓚曰親耕以躬親為義不得以假借為稱藉謂蹈藉也師古曰瓚說是說文帝藉千畝藉秦昔翻】乘鸞路設旄頭羽騎【羽騎羽林騎也】<br />
<br />
  資治通鑑卷五十九  <br>
   </div> 

<script src="/search/ajaxskft.js"> </script>
 <div class="clear"></div>
<br>
<br>
 <!-- a.d-->

 <!--
<div class="info_share">
</div> 
-->
 <!--info_share--></div>   <!-- end info_content-->
  </div> <!-- end l-->

<div class="r">   <!--r-->



<div class="sidebar"  style="margin-bottom:2px;">

 
<div class="sidebar_title">工具类大全</div>
<div class="sidebar_info">
<strong><a href="http://www.guoxuedashi.com/lsditu/" target="_blank">历史地图</a></strong>  
<a href="http://www.880114.com/" target="_blank">英语宝典</a>  
<a href="http://www.guoxuedashi.com/13jing/" target="_blank">十三经检索</a> 
<br><strong><a href="http://www.guoxuedashi.com/gjtsjc/" target="_blank">古今图书集成</a></strong> 
<a href="http://www.guoxuedashi.com/duilian/" target="_blank">对联大全</a> <strong><a href="http://www.guoxuedashi.com/xiangxingzi/" target="_blank">象形文字典</a></strong> 

<br><a href="http://www.guoxuedashi.com/zixing/yanbian/">字形演变</a>  <strong><a href="http://www.guoxuemi.com/hafo/" target="_blank">哈佛燕京中文善本特藏</a></strong>
<br><strong><a href="http://www.guoxuedashi.com/csfz/" target="_blank">丛书&方志检索器</a></strong> <a href="http://www.guoxuedashi.com/yqjyy/" target="_blank">一切经音义</a>  

<br><strong><a href="http://www.guoxuedashi.com/jiapu/" target="_blank">家谱族谱查询</a></strong>  <strong><a href="http://shufa.guoxuedashi.com/sfzitie/" target="_blank">书法字帖欣赏</a></strong> 
<br>

</div>
</div>


<div class="sidebar" style="margin-bottom:0px;">

<font style="font-size:22px;line-height:32px">QQ交流群9:489193090</font>


<div class="sidebar_title">手机APP 扫描或点击</div>
<div class="sidebar_info">
<table>
<tr>
	<td width=160><a href="http://m.guoxuedashi.com/app/" target="_blank"><img src="/img/gxds-sj.png" width="140"  border="0" alt="国学大师手机版"></a></td>
	<td>
<a href="http://www.guoxuedashi.com/download/" target="_blank">app软件下载专区</a><br>
<a href="http://www.guoxuedashi.com/download/gxds.php" target="_blank">《国学大师》下载</a><br>
<a href="http://www.guoxuedashi.com/download/kxzd.php" target="_blank">《汉字宝典》下载</a><br>
<a href="http://www.guoxuedashi.com/download/scqbd.php" target="_blank">《诗词曲宝典》下载</a><br>
<a href="http://www.guoxuedashi.com/SiKuQuanShu/skqs.php" target="_blank">《四库全书》下载</a><br>
</td>
</tr>
</table>

</div>
</div>


<div class="sidebar2">
<center>


</center>
</div>

<div class="sidebar"  style="margin-bottom:2px;">
<div class="sidebar_title">网站使用教程</div>
<div class="sidebar_info">
<a href="http://www.guoxuedashi.com/help/gjsearch.php" target="_blank">如何在国学大师网下载古籍?</a><br>
<a href="http://www.guoxuedashi.com/zidian/bujian/bjjc.php" target="_blank">如何使用部件查字法快速查字?</a><br>
<a href="http://www.guoxuedashi.com/search/sjc.php" target="_blank">如何在指定的书籍中全文检索?</a><br>
<a href="http://www.guoxuedashi.com/search/skjc.php" target="_blank">如何找到一句话在《四库全书》哪一页?</a><br>
</div>
</div>


<div class="sidebar">
<div class="sidebar_title">热门书籍</div>
<div class="sidebar_info">
<a href="/so.php?sokey=%E8%B5%84%E6%B2%BB%E9%80%9A%E9%89%B4&kt=1">资治通鉴</a> <a href="/24shi/"><strong>二十四史</strong></a>&nbsp; <a href="/a2694/">野史</a>&nbsp; <a href="/SiKuQuanShu/"><strong>四库全书</strong></a>&nbsp;<a href="http://www.guoxuedashi.com/SiKuQuanShu/fanti/">繁体</a>
<br><a href="/so.php?sokey=%E7%BA%A2%E6%A5%BC%E6%A2%A6&kt=1">红楼梦</a> <a href="/a/1858x/">三国演义</a> <a href="/a/1038k/">水浒传</a> <a href="/a/1046t/">西游记</a> <a href="/a/1914o/">封神演义</a>
<br>
<a href="http://www.guoxuedashi.com/so.php?sokeygx=%E4%B8%87%E6%9C%89%E6%96%87%E5%BA%93&submit=&kt=1">万有文库</a> <a href="/a/780t/">古文观止</a> <a href="/a/1024l/">文心雕龙</a> <a href="/a/1704n/">全唐诗</a> <a href="/a/1705h/">全宋词</a>
<br><a href="http://www.guoxuedashi.com/so.php?sokeygx=%E7%99%BE%E8%A1%B2%E6%9C%AC%E4%BA%8C%E5%8D%81%E5%9B%9B%E5%8F%B2&submit=&kt=1"><strong>百衲本二十四史</strong></a>  <a href="http://www.guoxuedashi.com/so.php?sokeygx=%E5%8F%A4%E4%BB%8A%E5%9B%BE%E4%B9%A6%E9%9B%86%E6%88%90&submit=&kt=1"><strong>古今图书集成</strong></a>
<br>

<a href="http://www.guoxuedashi.com/so.php?sokeygx=%E4%B8%9B%E4%B9%A6%E9%9B%86%E6%88%90&submit=&kt=1">丛书集成</a> 
<a href="http://www.guoxuedashi.com/so.php?sokeygx=%E5%9B%9B%E9%83%A8%E4%B8%9B%E5%88%8A&submit=&kt=1"><strong>四部丛刊</strong></a>  
<a href="http://www.guoxuedashi.com/so.php?sokeygx=%E8%AF%B4%E6%96%87%E8%A7%A3%E5%AD%97&submit=&kt=1">說文解字</a> <a href="http://www.guoxuedashi.com/so.php?sokeygx=%E5%85%A8%E4%B8%8A%E5%8F%A4&submit=&kt=1">三国六朝文</a>
<br><a href="http://www.guoxuedashi.com/so.php?sokeytm=%E6%97%A5%E6%9C%AC%E5%86%85%E9%98%81%E6%96%87%E5%BA%93&submit=&kt=1"><strong>日本内阁文库</strong></a> <a href="http://www.guoxuedashi.com/so.php?sokeytm=%E5%9B%BD%E5%9B%BE%E6%96%B9%E5%BF%97%E5%90%88%E9%9B%86&ka=100&submit=">国图方志合集</a> <a href="http://www.guoxuedashi.com/so.php?sokeytm=%E5%90%84%E5%9C%B0%E6%96%B9%E5%BF%97&submit=&kt=1"><strong>各地方志</strong></a>

</div>
</div>


<div class="sidebar2">
<center>

</center>
</div>
<div class="sidebar greenbar">
<div class="sidebar_title green">四库全书</div>
<div class="sidebar_info">

《四库全书》是中国古代最大的丛书,编撰于乾隆年间,由纪昀等360多位高官、学者编撰,3800多人抄写,费时十三年编成。丛书分经、史、子、集四部,故名四库。共有3500多种书,7.9万卷,3.6万册,约8亿字,基本上囊括了古代所有图书,故称“全书”。<a href="http://www.guoxuedashi.com/SiKuQuanShu/">详细>>
</a>

</div> 
</div>

</div>  <!--end r-->

</div>
<!-- 内容区END --> 

<!-- 页脚开始 -->
<div class="shh">

</div>

<div class="w1180" style="margin-top:8px;">
<center><script src="http://www.guoxuedashi.com/img/plus.php?id=3"></script></center>
</div>
<div class="w1180 foot">
<a href="/b/thanks.php">特别致谢</a> | <a href="javascript:window.external.AddFavorite(document.location.href,document.title);">收藏本站</a> | <a href="#">欢迎投稿</a> | <a href="http://www.guoxuedashi.com/forum/">意见建议</a> | <a href="http://www.guoxuemi.com/">国学迷</a> | <a href="http://www.shuowen.net/">说文网</a><script language="javascript" type="text/javascript" src="https://js.users.51.la/17753172.js"></script><br />
  Copyright &copy; 国学大师 古典图书集成 All Rights Reserved.<br>
  
  <span style="font-size:14px">免责声明:本站非营利性站点,以方便网友为主,仅供学习研究。<br>内容由热心网友提供和网上收集,不保留版权。若侵犯了您的权益,来信即刪。scp168@qq.com</span>
  <br />
ICP证:<a href="http://www.beian.miit.gov.cn/" target="_blank">鲁ICP备19060063号</a></div>
<!-- 页脚END --> 
<script src="http://www.guoxuedashi.com/img/plus.php?id=22"></script>
<script src="http://www.guoxuedashi.com/img/tongji.js"></script>

</body>
</html>
