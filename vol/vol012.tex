\chapter{資治通鑑卷十二}
宋 司馬光 撰

胡三省 音註

漢紀四|{
	起玄黓攝提格盡昭陽赤奮若凡十二年}


太祖高皇帝下

八年冬上擊韓王信餘寇於東垣|{
	班志高帝十一年更名東垣曰真定武帝元鼎四年置真定國垣音轅}
過柏人|{
	班志柏人縣屬趙國括地志柏人故城在邢州柏人縣西北十二里至唐天寶元年更柏人曰堯山}
貫高等壁人於厠中欲以要上|{
	文穎曰置人厠壁中以伺高祖也要一遥翻}
上欲宿心動問曰縣名為何曰柏人上曰柏人者迫於人也遂不宿而去十二月帝行自東垣至 春三月行如洛陽 令賈人毋得衣錦繡綺縠絺紵罽操兵乘騎馬|{
	師古曰賈人坐販賣者也綺文繒也即今之細綾也絺細葛也紵織紵為布及疏也罽織毛若今毼及氍毹之類也操持也兵凡兵器也乘駕車也騎單騎也賈音古衣於既翻絺允知翻紵音佇罽居例翻操千高翻余據錦織文也繡刺文而五采備者也縠縐紗也騎奇寄翻}
秋九月行自洛陽至淮南王梁王趙王楚王皆從|{
	從才用翻}
匈奴冒頓數苦北邊|{
	數所角翻下同}
上患之問劉敬劉敬曰天下初定士卒罷於兵|{
	罷讀曰疲}
未可以武服也冒頓殺父代立妻羣母以力為威未可以仁義說也|{
	說式芮翻}
獨可以計久遠子孫為臣耳然恐陛下不能為上曰奈何對曰陛下誠能以適長公主妻之|{
	適讀曰嫡謂皇后所生也長知兩翻}
厚奉遺之|{
	遺于季翻下同}
彼必慕以為閼氏|{
	閼氏音煙支}
生子必為太子陛下以歲時漢所餘彼所鮮數問遺|{
	鮮息善翻少也}
因使辯士風諭以禮節|{
	風與諷同}
冒頓在固為子壻死則外孫為單于豈嘗聞外孫敢與大父抗禮者哉可無戰以漸臣也若陛下不能遣長公主而令宗室及後宫詐稱公主彼知不肯貴近無益也帝曰善|{
	近其靳翻}
欲遣長公主呂后日夜泣曰妾唯太子一女奈何棄之匈奴上竟不能遣

九年冬上取家人子名為長公主|{
	師古曰於外庶人家取女而名之為公主}
以妻單于|{
	妻千細翻}
使劉敬往結和親約

臣光曰建信侯謂冒頓殘賊不可以仁義說而欲與為婚姻何前後之相違也夫骨肉之恩尊卑之敘唯仁義之人為能知之奈何欲以此服冒頓哉盖上世帝王之御夷狄也服則懷之以德叛則震之以威未聞與為婚姻也且冒頓視其父如禽獸而獵之奚有於婦翁建信侯之術固已疎矣况魯元已為趙后又可奪乎

劉敬從匈奴來因言匈奴河南白羊樓煩王去長安近者七百里輕騎一日一夜可以至秦中秦中新破|{
	秦中謂關中故秦地也新破謂經兵革之後未殷實}
少民地肥饒可益實|{
	少詩沼翻下同}
夫諸侯初起時非齊諸田楚昭屈景莫能興|{
	齊之王族諸田也楚之王族昭屈景也皆二國之彊家師古曰今高陵櫟陽諸田華隂好畤諸景及三輔諸屈諸懷尚多皆此時之所徙也屈九勿翻}
今陛下雖都關中實少民東有六國之彊族一日有變陛下亦未得高枕而臥也|{
	枕之鴆翻}
臣願陛下徙六國後及豪桀名家居關中無事可以備胡諸侯有變亦足率以東伐此彊本弱末之術也上曰善十一月徙齊楚大族昭氏屈氏景氏懷氏田氏五族及豪桀於關中與利田宅|{
	謂便利田宅也}
凡十餘萬口 十二月上行如洛陽貫高怨家知其謀上變告之|{
	謀謂謀弑上事始上卷七年怨於元翻又如字變非常也謂上告非常之事}
於是上逮捕趙王及諸反者|{
	師古曰逮捕謂事相連及者皆捕之一曰在道守禁相屬不絶若今之傳送囚耳貢父曰逮者其人存在直追取之捕者其人亡當討捕也故有或但言逮或但言捕知異義也一曰逮易辭捕加力也逮徒呼召之捕則加束縛矣}
趙午等十餘人皆爭自剄貫高獨怒罵曰誰令公為之今王實無謀而并捕王公等皆死誰白王不反者|{
	白明白也}
乃轞車膠致|{
	師古曰轞車者車而為檻形以版四周之無所通見史記正義曰膠致者膠密不得開送致京師也}
與王詣長安高對獄曰獨吾屬為之王實不知吏治搒笞數千刺剟|{
	搒音彭剟丁劣翻索隱曰剟亦刺也應劭曰以鐵刺之也}
身無可擊者終不復言呂后數言張王以公主故不宜有此|{
	數所角翻}
上怒曰使張敖據天下豈少而女乎|{
	少詩沼翻而汝也}
不聽廷尉以貫高事辭聞上曰壯士誰知者以私問之|{
	盖欲求貫高平日相知昵者以其私問之}
中大夫泄公曰|{
	班表郎中令之屬有太中大夫中大夫皆掌論議泄音薛泄姓也秦時衛有泄姬}
臣之邑子素知之此固趙國立義不侵為然諾者也|{
	言以義自立不受侵辱重於然諾也}
上使泄公持節往問之箯輿前|{
	韋昭曰如今輿床人輿以行師古曰箯輿者編竹木以為輿形如今之食輿高時榜笞刺剟委困故以箯輿處之索隱曰服䖍云編竹木如今峻可以糞除也何休注公羊筍音峻筍者竹箯一名編齊魯以北名為筍郭璞三蒼注云箯轝土器音鞭}
泄公與相勞苦如生平驩|{
	勞力到翻相勞且問其所苦也}
因問張王果有計謀不|{
	不讀曰否}
高曰人情寧不各愛其父母妻子乎今吾三族皆以論死|{
	謂以罪論抵死}
豈愛王過於吾親哉顧為王實不反|{
	為于偽翻}
獨吾等為之具道本指所以為者王不知狀於是泄公入具以報上春正月上赦趙王敖廢為宣平侯徙代王如意為趙王上賢貫高為人使泄公具告之曰張王已出因赦貫高貫高喜曰吾王審出乎泄公曰然泄公曰上多足下故赦足下貫高曰所以不死一身無餘者白張王不反也今王已出吾責已塞|{
	塞悉則翻}
死不恨矣且人臣有篡弑之名何面目復事上哉|{
	復扶又翻}
縱上不殺我我不愧於心乎乃仰絶亢遂死|{
	蘇林曰亢頸大脉也俗所謂胡脉也師古曰亢者總謂頸耳爾雅云亢鳥龍即㗋嚨也亢音岡又下郎翻}


荀悦論曰貫高首為亂謀殺主之賊|{
	殺讀曰弑}
誰能證明其王小亮不塞大逆私行不贖公罪|{
	塞悉則翻行下孟翻}
春秋之義大居正|{
	大居正者以居正為大也}
罪無赦可也

臣光曰高祖驕以失臣貫高狠以亡君使貫高謀逆者高祖之過也使張敖亡國者貫高之罪也

詔丙寅前有罪殊死已下皆赦之 二月行自洛陽至初上詔趙羣臣賓客敢從張王者皆族郎中田叔孟

舒皆自髠鉗為王家奴以從|{
	田叔孟舒皆趙國郎中也從才用翻}
及張敖既免上賢田叔孟舒等召見與語漢廷臣無能出其右者|{
	師古曰古者以右為尊言材用無有過之者故云無出其右也貢父曰古者居則貴左兵則貴右貴右似戰國時俗也}
上盡拜為郡守諸侯相|{
	班表郡守秦官掌治其郡秩二千石漢初諸侯王國亦置丞相統衆官羣卿大夫都官如漢朝景帝中五年令諸侯王不得復治國天子為置吏改丞相曰相秩二千石}
夏六月晦日有食之 更以丞相何為相國|{
	自丞相進相國則相國之位尊於丞相矣}


十年夏五月太上皇崩于櫟陽宫秋七月癸卯葬太上皇於萬年|{
	師古曰三輔黄圖云高祖初居櫟陽太上皇因居櫟陽既崩葬其北原起萬年邑置長丞焉 考異曰漢書五月太上皇后崩七月癸卯太上皇崩葬萬年荀紀五月無后字七月無崩字盖荀悦之時漢書本尚未訛謬故也今從之}
楚王梁王皆來送葬赦櫟陽囚|{
	臣瓚曰萬年陵在櫟陽縣故特赦之}
定陶戚姬有寵於上|{
	如淳曰姬音怡衆妾之總稱也漢官曰姬妾數百臣瓚曰漢秩禄令及茂陵書姬内官也秩比二千石位次倢伃下在七子八子之上索隱曰如淳音怡非也茂陵書姬是内官是矣然官號及婦人通稱姬者姬周之姓所以左傳稱伯姬叔姬以言天子之宗女貴於他姓故遂以姬為婦人美號}
生趙王如意上以太子仁弱謂如意類已雖封為趙王常留之長安上之關東戚姬常從|{
	從才用翻}
日夜啼泣欲立其子呂后年長常留守益疏|{
	長知兩翻守式又翻疏與疎同}
上欲廢太子而立趙王大臣爭之皆莫能得御史大夫周昌廷爭之彊上問其說昌為人吃|{
	吃音訖言之難也}
又盛怒曰臣口不能言然臣期期知其不可陛下欲廢太子臣期期不奉詔|{
	師古曰以口吃故重言期期貢父曰期讀如荀子目欲綦色之綦楚人謂極為綦孔頴達曰釋詁曰汔也杜預曰汔期也然則期字雖别皆是近義言其近當如此史記稱高祖欲廢太子周昌曰臣期知其不可周昌又曰臣期不奉詔言期者意亦與汔同}
上欣然而笑呂后側耳於東廂聽|{
	韋昭曰東廂殿東堂也師古曰正寢之東西室皆曰廂言似箱箧之形}
既罷見昌為跪謝曰微君太子幾廢|{
	為于偽翻幾居依翻}
時趙王年十歲上憂萬歲之後不全也符璽御史趙堯|{
	符璽御史御史之掌符璽者也屬御史大夫璽斯氏翻}
請為趙王置貴彊相|{
	為于偽翻相息亮翻}
及呂后太子羣臣素所敬憚者上曰誰可者堯曰御史大夫昌其人也上乃以昌相趙|{
	為呂后殺戚夫人及如意張本}
而以堯代昌為御史大夫 |{
	考異曰史記漢書張良傳皆云十二年上擊黥布還愈欲易太子按百官表十年趙堯為御史大夫則是時太子位已定今從之}
初上以陽夏侯陳豨為相國監趙代邊兵|{
	夏音賈豨許豈翻又音希徐廣曰為趙相國將兵守代監古衘翻}
豨過辭淮隂侯淮隂侯挈其手辟左右|{
	辟音闢除也屛除左右也}
與之步於庭仰天歎曰子可與言乎豨曰唯將軍令之淮隂侯曰公之所居天下精兵處也而公陛下之信幸臣也人言公之畔陛下必不信再至陛下乃疑矣三至必怒而自將|{
	將即亮翻}
吾為公從中起天下可圖也|{
	為于偽翻}
陳豨素知其能也信之曰謹奉教豨常慕魏無忌之養士|{
	魏無忌信陵君也}
及為相守邊告歸|{
	漢律二千石有予告有賜告予告者在官有功㝡法所當得也賜告者病滿三月當免天子優賜其告使得帶印綬將官屬歸家治病至成帝時郡國二千石賜告不得歸家至和帝時賜予皆絶師古曰告者請謁之言謂請休耳或謂之謝謝亦告也左傳曰韓獻子告老禮記曰若不得謝漢書諸云謝病皆同義}
過趙賓客隨之千餘乘|{
	乘繩證翻}
邯鄲官舍皆滿趙相周昌求入見上|{
	見賢遍翻}
具言豨賓客甚盛擅兵於外數歲恐有變上令人覆案豨客居代者諸不法事多連引豨豨恐韓王信因使王黄曼丘臣等說誘之|{
	說式芮翻誘音酉}
太上皇崩上使人召豨豨稱病不至九月遂與王黄等反自立為代王刼略趙代上自東擊之至邯鄲喜曰豨不據邯鄲而阻漳水吾知其無能為矣周昌奏常山二十五城亡其二十城請誅守尉|{
	秦滅趙置鉅鹿邯鄲郡漢始置常山郡杜佑通典曰漢常山郡故城在趙州元氐縣西守者郡守尉者都尉守式又翻}
上曰守尉反乎對曰不|{
	不讀曰否}
上曰是力不足亡罪上令周昌選趙壯士可令將者白見四人|{
	將即亮翻下同見賢遍翻}
上嫚罵曰豎子能為將乎四人慙皆伏地上封各千戶以為將左右諫曰從入蜀漢伐楚賞未徧行今封此何功上曰非汝所知陳豨反趙代地皆豨有吾以羽檄徵天下兵未有至者今計唯獨邯鄲中兵耳吾何愛四千戶不以慰趙子弟皆曰善又聞豨將皆故賈人上曰吾知所以與之矣乃多以金購豨將豨將多降|{
	師古曰與如也言能如之何也貢父曰與猶待也原父曰知與之者知所以與之之術也豨將皆故賈人賈人嗜利乃多以金購之賈音古}


十一年冬上在邯鄲陳豨將侯敞將萬餘人游行王黄將騎千餘軍曲逆張春將卒萬餘人度河攻聊城|{
	班志聊城縣屬東郡括地志聊城故城在博州聊城縣西二十里春秋時齊之西界聊攝也戰國時亦為齊地}
漢將軍郭蒙與齊將擊大破之太尉周勃道太原入定代地至馬邑不下攻殘之|{
	殘謂多所殺戮}
趙利守東垣帝攻拔之更命曰真定|{
	更工衡翻}
帝購王黄曼丘臣以千金其麾下皆生致之於是陳豨軍遂敗淮隂侯信稱病不從擊豨隂使人至豨所與通謀信謀與家臣夜詐詔赦諸官徒奴|{
	有罪而居作者為徒有罪而没入官者為奴}
欲發以襲呂后太子部署已定待豨報其舍人得罪於信信囚欲殺之春正月舍人弟上變告信欲反狀於呂后|{
	按班書功臣表告信反者舍人樂說也封慎陽侯}
呂后欲召恐其儻不就|{
	儻或然之辭}
乃與蕭相國謀詐令人從上所來言豨已得死列侯羣臣皆賀相國紿信曰雖疾彊入賀|{
	彊其兩翻}
信入呂后使武士縛信斬之長樂鐘室|{
	師古曰懸鐘之室}
信方斬曰吾悔不用蒯徹之計|{
	不用蒯徹見十卷四年}
乃為兒女子所詐豈非天哉遂夷信三族

臣光曰世或以韓信首建大策與高祖起漢中定三秦遂分兵以北禽魏取代仆趙脅燕東擊齊而有之南滅楚垓下漢之所以得天下者大抵皆信之功也觀其距蒯徹之說迎高祖于陳|{
	見上卷六年}
豈有反心哉良由失職怏怏遂陷悖逆夫以盧綰里閈舊恩猶南面王燕信乃以列侯奉朝請|{
	閈侯旰翻王于況翻朝直遥翻請才性翻又如字}
豈非高祖亦有負於信哉臣以為高祖用詐謀禽信於陳言負則有之雖然信亦有以取之也始漢與楚相距滎陽信滅齊不還報而自王|{
	見十卷四年}
其後漢追楚至固陵與信期共攻楚而信不至|{
	見十卷五年}
當是之時高祖固有取信之心矣顧力不能耳及天下已定信復何恃哉|{
	復扶又翻}
夫乘時以徼利者市井之志也|{
	徼一遥翻}
醻功而報德者士君子之心也|{
	醻時流翻}
信以市井之志利其身而以士君子之心望於人不亦難哉是故太史公論之曰假令韓信學道謙讓不伐已功不矜其能則庶幾哉|{
	幾居衣翻}
於漢家勲可以比周召太公之徒後世血食矣不務出此而天下已集乃謀畔逆夷滅宗族不亦宜乎

將軍柴武斬韓王信于參合|{
	姓譜柴姓高柴之後班志參合縣屬代郡括地志參合故城在朔州定襄縣北}
上還洛陽聞淮隂侯之死且喜且憐之|{
	喜者喜除其偪憐者憐其功大}
問呂后曰信死亦何言呂后曰信言恨不用蒯徹計上曰是齊辯士蒯徹也乃詔齊捕蒯徹蒯徹至上曰若教淮隂侯反乎對曰然臣固教之豎子不用臣之策故令自夷於此如用臣之計陛下安得而夷之乎上怒曰烹之徹曰嗟乎寃哉烹也上曰若教韓信反何寃對曰秦失其鹿天下共逐之高材疾足者先得焉跖之狗吠堯堯非不仁狗固吠非其主當是時臣唯獨知韓信非知陛下也且天下鋭精持鋒|{
	鋭精言磨淬精鐵而鋭之也}
欲為陛下所為者甚衆顧力不能耳|{
	師古曰顧念也余謂顧反視也反已而自視其力有所不能也}
又可盡烹之邪上曰置之|{
	置猶舍也又赦也}
立子恒為代王都晉陽|{
	晉陽漢為太原郡治所如淳曰文紀言都中都又文帝過太原復晉陽中都二歲似遷都于中都也恒戶登翻}
大赦天下 上之擊陳豨也徵兵於梁梁王稱病使將將兵詣邯鄲上怒使人讓之梁王恐欲自往謝其將扈輒曰王始不往見讓而往往則為禽矣不如遂發兵反梁王不聽梁太僕得罪亡走漢告梁王與扈輒謀反於是上使使掩梁王梁王不覺遂囚之洛陽有司治反形已具|{
	臣瓚曰扈輒勸越反而越不誅是反形已具也}
請論如法上赦以為庶人傳處蜀青衣|{
	青衣道屬蜀郡臣瓚曰今漢嘉是也章懷太子賢曰青衣道在大江青衣二水之會今嘉州龍遊縣也傳張戀翻處昌呂翻}
西至鄭逢呂后從長安來彭王為呂后泣涕自言無罪願處故昌邑|{
	二世二年彭越起於昌邑為于偽翻}
呂后許諾與俱東至洛陽呂后白上曰彭王壯士今徙之蜀此自遺患不如遂誅之妾謹與俱來於是呂后乃令其舍人告彭越復謀反|{
	復扶又翻}
廷尉王恬開奏請族之上可其奏三月夷越三族|{
	此以漢書本紀為据史記高祖紀作夏夷彭越三族年表書越反誅又在十年夏誅彭越盖以盧綰言為据}
梟越首洛陽下詔有收視者輒捕之梁大夫欒布使於齊|{
	姓譜欒晉卿欒氏之後}
還奏事越頭下祠而哭之吏捕以聞上召布罵欲烹之方提趨湯|{
	提挈也挈而趨鼎欲投之於湯趨七俞翻}
布顧曰願一言而死上曰何言布曰方上之困於彭城敗滎陽成皋間項王所以遂不能西者徒以彭王居梁地與漢合從苦楚也|{
	從子容翻}
當是之時王一顧與楚則漢破與漢則楚破且垓下之會微彭王項氏不亡天下已定彭王剖符受封亦欲傳之萬世今陛下一徵兵於梁彭王病不行而陛下疑以為反反形未具以苛小案誅滅之臣恐功臣人人自危也今彭王已死臣生不如死請就烹於是上乃釋布罪拜為都尉 丙午立皇子恢為梁王 |{
	考異曰漢書諸侯王表作三月丙午按劉羲叟長歷三月丙辰朔無丙午今從史記年表今按史記年表作二月丙午但通鑑先書三月夷彭越三族方於此書立子恢為梁王則又是三月丙午}
丙寅立皇子友為淮陽王罷東郡頗益梁罷潁川郡頗益淮陽 夏四月行自洛陽至 五月詔立秦南海尉趙佗為南粤王|{
	晉志秦使任嚻趙佗攻粤畧取陸梁地遂定南粤以為桂林南海象三郡非三十六郡之限乃置南海尉以典之所謂東南一尉也余謂始皇二十六年分天下為三十六郡郡置守尉監三十三年取南粤置南海桂林象郡此南海尉止典南海一郡兵猶三十六郡之尉也安得兼典桂林象郡任嚻既死秦已破滅趙佗始擊并桂林象郡以此知非兼典也佗徒河翻}
使陸賈即授璽綬|{
	姓譜陸古天子陸終之後}
與剖符通使使和集百越無為南邊患害初秦二世時南海尉任囂病且死|{
	任音壬囂音敖}
召龍川令趙佗|{
	班志龍川縣屬南海郡裴氏廣州記龍川本博羅縣之東鄉有龍穿地而出即宂流泉因以為號師古曰今循州}
語曰|{
	語牛倨翻}
秦為無道天下苦之聞陳勝等作亂天下未知所安南海僻遠吾恐盜兵侵地至此欲興兵絶新道自備|{
	蘇林曰新道秦所新通越道}
待諸侯變會病甚且番禺負山險阻南海|{
	班志番禺縣屬南海郡尉佗所都今為廣州治所番音潘禺音愚又魚容翻}
東西數千里頗有中國人相輔此亦一州之主也可以立國郡中長吏無足與言者|{
	長知兩翻}
故召公告之即被佗書|{
	韋昭曰被之以書音光被之被皮義翻}
行南海尉事囂死佗即移檄告横浦陽山湟谿關曰|{
	武帝伐南越遣楊僕出豫章下横浦則横浦通豫章之路也杜佑曰横浦關在䖍州大庾縣西南南康記曰南野大庾嶺三十里至横浦有秦時關其下謂為塞上班志陽山侯國屬桂陽郡姚氏曰連州陽山縣上流百餘里有騎田嶺當是陽山關新唐書地理志連州陽山縣有故秦湟谿關郡國志陽山縣理洭水之南即其故墟本南越置關之邑故關在縣西北四十里茂溪口湟音皇}
盜兵且至急絶道聚兵自守因稍以法誅秦所置長吏以其黨為假守秦已破滅佗即擊并桂林象郡|{
	桂林後武帝改為鬱林郡象郡武帝改為日南郡}
自立為南越武王|{
	韋昭曰生以武為號不稽於古也}
陸生至尉佗魋結|{
	服䖍曰今兵士椎頭䯻也師古曰椎髻者一撮之髻其形如椎魋音椎結讀曰髻}
箕倨見陸生陸生說佗曰足下中國人|{
	尉佗本真定人故賈云然}
親戚昆弟墳墓在真定今足下反天性棄冠帶|{
	背父母之國不念墳墓宗族是反天性也椎髻以從蠻夷之俗是棄冠帶也}
欲以區區之越與天子抗衡為敵國禍且及身矣且夫秦失其政諸侯豪傑並起唯漢王先入關據咸陽項羽倍約自立為西楚霸王|{
	倍蒲妹翻}
諸侯皆屬可謂至彊然漢王起巴蜀鞭笞天下遂誅項羽滅之五年之間海内平定此非人力天之所建也天子聞君王王南越|{
	王王下于況翻下故王同}
不助天下誅暴逆將相欲移兵而誅王天子憐百姓新勞苦故且休之遣臣授君王印剖符通使君王宜郊迎北面稱臣乃欲以新造未集之越|{
	師古曰未集言未成也}
屈彊於此|{
	師古曰屈彊謂不柔服也屈其勿翻}
漢誠聞之掘燒王先人冢夷滅宗族使一偏將將十萬衆臨越則越殺王降漢如反覆手耳於是尉佗乃蹶然起坐|{
	師古曰蹶然驚起之貌也蹶音厥}
謝陸生曰居蠻夷中久殊失禮義因問陸生曰我孰與蕭何曹參韓信賢陸生曰王似賢也復曰我孰與皇帝賢|{
	復扶又翻}
陸生曰皇帝繼五帝三皇之業統理中國中國之人以億計地方萬里萬物殷富政由一家自天地剖判未始有也今王衆不過數十萬皆蠻夷崎嶇山海間|{
	崎丘宜翻嶇音區}
譬若漢一郡耳何乃比於漢尉佗大笑曰吾不起中國故王此使我居中國何遽不若漢|{
	師古曰言有何迫促而不如漢也余謂遽者急促也今江南人謂之便何至便不如漢也遽其庶翻}
乃留陸生與飲數月曰越中無足與語至生來令我日聞所不聞賜陸生槖中裝直千金|{
	張晏曰槖中裝珠玉之寶也裝裹也如淳曰明月珠之屬也師古曰有底曰囊無底曰槖言其寶物質輕而價重可入囊槖以齎行故曰槖中裝}
佗送亦千金|{
	蘇林曰非槖中物故曰佗送師古曰佗猶餘也}
陸生卒拜尉佗為南越王|{
	卒子恤翻}
令稱臣奉漢約歸報帝大悦拜賈為太中大夫陸生時時前說稱詩書帝罵之曰乃公居馬上而得之安事詩書陸生曰居馬上得之寧可以馬上治之乎|{
	治直之翻}
且湯武逆取而以順守之文武並用長久之術也昔者吳王夫差智伯秦始皇皆以極武而亡鄉使秦已并天下行仁義法先聖陛下安得而有之|{
	鄉讀曰嚮}
帝有慙色曰試為我著秦所以失天下吾所以得之者及古成敗之國陸生乃粗述存亡之徵|{
	為于偽翻粗坐五翻畧也}
凡著十二篇每奏一篇帝未嘗不稱善左右呼萬歲號其書曰新語 帝有疾惡見人|{
	惡烏路翻}
卧禁中詔戶者無得入羣臣|{
	戶者謂守門戶者也}
羣臣絳灌等莫敢入十餘日舞陽侯樊噲排闥直入|{
	班志舞陽縣屬潁川郡應劭曰舞水出其縣之南史記正義在許州葉縣東十里師古曰闥宫中小門也一曰門屏也音土曷翻}
大臣隨之上獨枕一宦者卧|{
	枕之鴆翻}
噲等見上流涕曰始陛下與臣等起豐沛定天下何其壯也今天下已定又何憊也|{
	憊蒲拜翻疲極也}
且陛下病甚大臣震恐不見臣等計事顧獨與一宦者絶乎且陛下獨不見趙高之事乎|{
	謂與李斯謀殺扶蘇立胡亥也}
帝笑而起 秋七月淮南王布反初淮隂侯死布已心恐及彭越誅醢其肉以賜諸侯|{
	師古曰反者被誅皆以為醢即刑法志所謂菹其骨肉是也賈公彦曰有骨為臡無骨為醢菜肉通全物若䐑為菹細切為韲作臡醢者必先膊乾其肉及漬剉之雜以粱麯及鹽漬以美酒塗置甄中百日則成矣菹醯醬所和}
使者至淮南淮南王方獵見醢因大恐隂令人部聚兵候伺旁郡警急布所幸姬病就醫醫家與中大夫賁赫對門|{
	賁音肥姓也赫其名也姓譜有賁姓以為縣賁父之後風俗通魯有賁浦皆音奔}
赫乃厚餽遺從姬飲醫家|{
	遺于季翻}
王疑其與亂欲捕赫赫乘傳詣長安上變|{
	傳柱戀翻}
言布謀反有端可先未發誅也上讀其書語蕭相國相國曰布不宜有此恐仇怨妄誣之|{
	語牛倨翻怨於元翻}
請繫赫使人微驗淮南王|{
	師古曰微驗者不顯言其事}
淮南王見赫以罪亡上變固已疑其言國隂事漢使又來頗有所驗遂族赫家發兵反反書聞上乃赦賁赫以為將軍上召諸將問計皆曰發兵擊之坑豎子耳何能為乎汝隂侯滕公|{
	班志汝隂縣屬汝南郡春秋胡子之國史記正義曰汝隂即今陽城余據唐陽城縣屬河南郡與漢汝南之汝隂相去頗遠又據史記滕公傳平城圍解增食細陽千戶細陽縣屬汝南郡盖與汝隂鄰境索隱曰汝隂屬汝南亦據班志也}
召故楚令尹薛公問之令尹曰是固當反膝公曰上裂地而封之疏爵而王之|{
	疏分也}
其反何也令尹曰往年殺彭越前年殺韓信此三人者同功一體之人也自疑禍及身故反耳滕公言之上上乃召見問薛公薛公對曰布反不足怪也使布出於上計山東非漢之有也出於中計勝敗之數未可知也出於下計陛下安枕而卧矣上曰何謂上計對曰東取吳西取楚并齊取魯傳檄燕趙固守其所山東非漢之有也何謂中計東取吳西取楚并韓取魏據敖倉之粟塞成臯之口勝敗之數未可知也何謂下計東取吳西取下蔡歸重於越身歸長沙|{
	吳謂荆王劉賈所封之地楚謂楚王交所封之地齊謂齊王肥所封之地魯亦入楚境韓地時以益淮陽國魏地梁王友所封也下蔡縣屬沛郡春秋時之州來也越會稽地故越王句踐之墟也長沙吳芮所封國時其子臣嗣封黥布都六阻淮為固故策其西取下蔡東取劉賈以據全淮越在東南故策其歸輜重於越以自厚為深固不可取之計布娶於長沙王故策其身歸長沙料其出於麗山之徒慮不及遠也重直用翻}
陛下安枕而卧漢無事矣上曰是計將安出對曰出下計上曰何為廢上中計而出下計對曰布故麗山之徒也|{
	麗與驪同事見八卷秦二世二年}
自致萬乘之主此皆為身不顧後為百姓萬世慮者也|{
	皆為于偽翻下間為為妻為上同}
故曰出下計上曰善封薛公千戶乃立皇子長為淮南王 |{
	考異曰史記諸侯年表云十二月庚子厲王長元年漢書諸侯王表十月庚午立今從漢書帝紀}
是時上有疾欲使太子往擊黥布太子客東園公綺里季夏黄公甪里先生|{
	此所謂四皓也避秦之亂隱於商山索隱曰按陳留志云園公姓唐字宣明居園中因以為號夏黄公姓崔名廣字少通齊人隱居夏里修道故號曰夏黄公甪里先生河内軹人太伯之後姓周名術字元道京師號曰霸上先生一曰甪里先生甪盧谷翻}
說建成侯呂釋之曰|{
	班志建成侯國屬沛郡}
太子將兵有功則位不益|{
	師古曰太子嗣君位已至矣雖更立功位無加益}
無功則從此受禍矣君何不急請呂后承間為上泣言黥布天下猛將也善用兵今諸將皆陛下故等夷|{
	間古莧翻師古曰夷平也言故時皆齊等}
乃令太子將此屬無異使羊將狼莫肯為用且使布聞之則鼓行而西耳上雖病彊載輜車|{
	彊其兩翻師古曰輜車衣車也}
卧而護之諸將不敢不盡力上雖苦為妻子自彊於是呂釋之立夜見呂后呂后承間為上泣涕而言如四人意 |{
	考異曰史記漢書皆云呂澤夜見呂后按恩澤侯表有周呂侯澤建成侯釋之今此上云建成侯而下云呂澤恐誤當為釋之是又留侯世家上欲廢太子立戚夫人子趙王如意大臣多諫爭未能得堅決者也呂后恐不知所為人或謂呂后曰留侯善畫計策上信用之呂后乃使建成侯呂澤刼留侯曰君常為上謀臣今上易太子君安得高枕而卧乎留侯曰始上數在困急之中幸用臣筴今天下安定以愛欲易太子骨肉之間雖臣等百餘人何益呂澤強要曰為我畫計留侯曰此難以口舌爭也顧上有不能致者天下有四人四人者年老矣皆以為上嫚侮人故逃匿山中義不為漢臣然上高此四人今公誠能無愛金玉璧帛令太子為書卑辭安車因使辯士固請宜來來以為客時時從入朝令上見之則必異而問之問之上知此四人賢則一助也於是呂后令呂澤使人奉太子書卑辭厚禮迎此四人四人至客建成侯所上欲使太子擊黥布四人相謂曰凡來者將以存太子太子將兵事危矣乃說建成侯云云上遂自行上破布歸置酒太子侍四人從太子年皆八十有餘鬚眉皓白衣冠甚偉上怪問之曰彼何為者四人前對各言名姓曰東園公甪里先生綺里季夏黄公上乃大驚曰吾求公數歲公辟逃我今公何自從吾兒游乎四人皆曰陛下輕士善罵臣等義不受辱故恐而亡匿竊聞太子為人仁孝恭敬愛士天下莫不延頸欲為太子死故臣等來耳上曰煩公幸卒調護太子四人為夀已畢起去上目送之召戚夫人指示四人者曰我欲易之彼四人輔之羽翼已成難動矣呂氏真而主矣戚夫人泣上曰為我楚舞吾為若楚歌歌曰鴻鵠高飛一舉千里羽翮已就横絶四海横絶四海當可奈何雖有繒繳尚安所施歌數闋戚夫人嘘唏流涕上起去罷酒竟不易太子者留侯本招此四人之功也按高祖剛猛伉厲非畏搢紳譏議者也但以大臣皆不肯從恐身後趙王不能獨立故不為耳若決意欲廢太子立如意不顧義理以留侯之久故親信猶云非口舌所能爭豈山林四叟片言遽能柅其事哉借使四叟實能柅其事不過汚高祖數寸之刃耳何至悲歌云羽翮已成繒繳安施乎若四叟實能制高祖使不敢廢太子是留侯為子立黨以制其父也留侯豈為此哉此特辯士欲夸大四叟之事故云然亦猶蘇秦約六國從秦兵不敢闚函谷關十五年魯仲連折新垣衍秦將聞之却軍五十里耳凡此之類皆非事實司馬遷好奇多愛而采之今皆不取}
上曰吾惟豎子固不足遣|{
	惟思也}
而公自行耳於是上自將兵而東羣臣居守|{
	守式又翻}
皆送至霸上留侯病自彊起至曲郵|{
	司馬彪曰長安縣東有曲郵聚索隱曰今在新豐西俗謂之郵頭漢書舊儀云五里一郵郵人居間相去二里半按郵乃今之也}
見上曰臣宜從|{
	從才用翻}
病甚楚人剽疾|{
	剽匹妙翻}
願上無與爭鋒因說上令太子為將軍監關中兵上曰子房雖病彊卧而傅太子|{
	監古衘翻彊其兩翻}
是時叔孫通為太傅留侯行少傅事|{
	班志太子太傅少傅古官予據古世子有三師三少至漢惟太傅少傅耳少詩照翻}
發上郡北地隴西車騎巴蜀材官及中尉卒三萬人為皇太子衛軍霸上|{
	應劭曰材官有材力者漢官儀曰民年二十三為正一歲為衛士二歲為材官騎士習射御騎馳戰陳常以八月太守都尉令長丞尉會都試課殿最水處則習船邉郡將萬騎行障塞烽火追虜師古曰車常擬軍興者若近代之戎車也騎常所養馬并其人使行充騎若今武馬及所養者主也至光武罷省班表中尉秦官掌徼循京師武帝太初元年更名執金吾}
布之初反謂其將曰上老矣厭兵必不能來使諸將諸將獨患淮隂彭越今皆已死餘不足畏也故遂反果如薛公之言東擊荆荆王賈走死富陵|{
	班志富陵縣屬臨淮郡括地志富陵故城在楚州盱眙縣東北六十里}
盡劫其兵渡淮擊楚楚發兵與戰徐僮間|{
	班志臨淮郡有徐縣僮縣楚盖發兵與布戰于二縣之間杜預曰徐在下邳僮縣東括地志大徐城在泗州徐城縣北四十里古徐國也}
為三軍欲以相救為奇|{
	師古曰不聚一處而分為三欲互相救出奇譎}
或說楚將曰布善用兵民素畏之且兵法諸侯自戰其地為散地今别為三彼敗吾一軍|{
	散如字敗補邁翻}
餘皆走安能相救不聽布果破其一軍其二軍散走布遂引兵而西

十二年冬十月上與布軍遇於鄿西|{
	班志鄿縣屬沛郡}
布兵精甚上壁庸城|{
	以布軍鋭甚故堅壁以挫之庸城地名必亦在鄿縣西}
望布軍置陳如項籍軍上惡之|{
	陳讀曰陣惡烏路翻}
與布相望見遥謂布曰何苦而反布曰欲為帝耳上怒罵之遂大戰布軍敗走渡淮數止戰不利|{
	數所角翻}
與百餘人走江南上令别將追之上還過沛留置酒沛宫|{
	括地志沛宫故地在徐州沛縣東南二十里一十步}
悉

召故人父老諸母子弟佐酒道舊故為笑樂酒酣|{
	樂音洛下同應劭曰不醒不醉曰酣一曰酣洽也音戶甘翻}
上自為歌起舞忼慨傷懷泣數行下|{
	行戶剛翻}
謂沛父兄曰游子悲故鄉|{
	師古曰游子行客也悲謂顧念也}
朕自沛公以誅暴逆遂有天下其以沛為朕湯沐邑復其民世世無有所與|{
	復除其民不豫賦役復方目翻與讀曰預}
樂飲十餘日乃去 漢别將擊英布軍洮水南北皆大破之|{
	蘇林曰洮音兆徐廣曰洮音道在江淮間余據布軍既敗走江南則洮水當在江南羅含湘中記零陵有洮水水經注洮水出洮陽縣西南東流注于湘水如淳註洮陽之洮音韜盖布舊與長沙王婚其敗也往從之而洮水又在長沙境内疑近是也杜佑曰漢洮陽縣城在永州湘源縣西北按今全州漢洮陽縣地有洮水在清湘縣北}
布故與番君婚以故長沙成王臣使人誘布偽欲與亡走越布信而隨之番陽人殺布兹鄉民田舍|{
	番音婆師古曰兹鄉鄡陽縣之鄉也班志鄡陽縣屬豫章郡鄡古么翻余據史記及漢書高紀皆言追斬布番陽竊意兹鄉當在番陽界非鄡陽}
周勃悉定代郡鴈門雲中地斬陳豨於當城|{
	班志當城縣屬代郡闞駰十三州記當城在高柳東八十里縣當桓都山作城故曰當城史記正義曰當城在朔州定襄縣界 考異曰盧綰傳云漢使樊噲擊斬豨按斬豨者周勃非樊噲也}
上以荆王賈無後更以荆為吳國辛丑立兄仲之子濞為吳王|{
	服䖍曰濞音帔普懿翻}
王三郡五十三城|{
	為後濞以吳反張本}
十一月上過魯以太牢祠孔子 上從破黥布歸疾益甚愈欲易太子張良諫不聽因疾不視事|{
	良先行太子少傅事以諫不聽因稱疾不肯視事}
叔孫通諫曰昔者晉獻公以驪姬之故廢太子立奚齊晉國亂者數十年為天下笑|{
	晉獻公嬖驪姬欲立其子故廢太子申生而以驪姬之子奚齊屬荀息而立之公薨里克殺奚齊荀息立其弟卓子里克殺卓子迎立惠公惠公為秦所執既歸而薨子懷公立秦納文公而殺懷公晉乃定}
秦以不蚤定扶蘇令趙高得以詐立胡亥自使滅祀|{
	事見秦紀}
此陛下所親見今太子仁孝天下皆聞之呂后與陛下攻苦食啖|{
	徐廣曰攻猶今人言擊也啖一作淡如淳曰食無菜茹為啖師古曰啖當作淡淡謂無味之食也言共攻擊勤苦之事食無味之食也孔文祥曰與帝俱攻冒苦難俱食淡也或曰攻治也余按周禮艸人注物地占其形色知鹹啖也釋文啖直覽翻疏作鹹淡則知啖淡古字通用}
其可背哉|{
	背蒲妹翻}
陛下必欲廢適而立少|{
	適謂太子少謂趙王適讀曰嫡}
臣願先伏誅以頸血汙地|{
	汙烏故翻}
帝曰公罷矣吾直戲耳叔孫通曰太子天下本本一搖天下振動奈何以天下為戲乎時大臣固爭者多上知羣臣心皆不附趙王乃止不立 相國何以長安地陿|{
	陿與狭同}
上林中多空地棄願令民得入田毋收槀為禽獸食|{
	師古曰槀禾稈也言恣人田之不收其槀税也索隱曰苗子還種田人收槀入官槀工老翻}
上大怒曰相國多受賈人財物乃為請吾苑下相國廷尉械繫之|{
	賈音古為于偽翻下遐嫁翻}
數日王衛尉侍前問曰|{
	師古曰前問謂進而請也}
相國何大罪陛下繫之暴也上曰吾聞李斯相秦皇帝有善歸主有惡自與今相國多受賈豎金而為之請吾苑以自媚於民|{
	師古曰媚愛也求愛于民}
故繫治之王衛尉曰夫職事苟有便於民而請之真宰相事陛下奈何乃疑相國受賈人錢乎且陛下距楚數歲陳豨黥布反陛下自將而往當是時相國守關中關中搖足則關以西非陛下有也相國不以此時為利今乃利賈人之金乎且秦以不聞其過亡天下李斯之分過又何足法哉陛下何疑宰相之淺也帝不懌|{
	師古曰懌悦也感衛尉之言故慙悔而不悦也}
是日使使持節赦出相國相國年老素恭謹入徒跣謝帝曰相國休矣相國為民請苑吾不許我不過為桀紂主而相國為賢相吾故繫相國欲令百姓聞吾過也 陳豨之反也燕王綰發兵擊其東北|{
	陳豨反於代代在燕之西南故綰擊其東北}
當是時陳豨使王黄求救匈奴燕王綰亦使其臣張勝於匈奴言豨等軍破張勝至胡故燕王臧荼子衍出亡在胡見張勝曰公所以重於燕者以習胡事也燕所以久存者以諸侯數反|{
	數所角翻}
兵連不決也今公為燕欲急滅豨等|{
	為于偽翻}
豨等已盡次亦至燕公等亦且為虜矣公何不令燕且緩陳豨而與胡和事寛得長王燕|{
	王于况翻}
即有漢急可以安國張勝以為然乃私令匈奴助豨等擊燕燕王綰疑張勝與胡反上書請族張勝勝還具道所以為者燕王乃詐論他人脱勝家屬使得為匈奴間|{
	間古莧翻}
而隂使范齊之陳豨所欲令久亡連兵勿決|{
	欲使之連兵相持勝負久而不决也}
漢擊黥布豨常將兵居代漢擊斬豨其裨將降言燕王綰使范齊通計謀於豨所帝使使召盧綰綰稱病又使辟陽侯審食其|{
	班志辟陽縣屬信都國辟必 亦翻姓譜有審姓食其音異基}
御史大夫趙堯往迎燕王因驗問左右綰愈恐閉匿|{
	謂閉其蹤跡藏匿其人也}
謂其幸臣曰非劉氏而王獨我與長沙耳往年春漢族淮隂夏誅彭越皆呂氏計今上病屬任呂后|{
	屬之欲翻}
呂后婦人專欲以事誅異姓王者及大功臣乃遂稱病不行其左右皆亡匿語頗泄辟陽侯聞之歸具報上上益怒又得匈奴降者言張勝亡在匈奴為燕使於是上曰盧綰果反矣春二月使樊噲以相國將兵擊綰立皇子建為燕王詔曰南武侯織亦粤之世也立以為南海王|{
	文頴曰高祖五年以象郡桂林南海長沙立吳芮為長沙王象郡桂林南海屬尉佗佗未降遥奪以封芮耳後佗降漢十一年更立佗為南越王自此王三郡芮惟得長沙桂陽耳今封織南海王復遥奪佗一郡織未得王之}
上擊布時為流矢所中行道疾甚呂后迎良醫醫入見曰疾可治|{
	治直之翻下同}
上嫚罵之曰吾以布衣提三尺取天下|{
	師古曰三尺謂劔也中竹仲翻見賢遍翻}
此非天命乎命乃在天雖扁鵲何益|{
	扁鵲古之良醫扁補辨翻}
遂不使治疾賜黄金五十斤罷之呂后問曰陛下百歲後蕭相國既死誰令代之上曰曹參可問其次曰王陵可然少戇|{
	少者多少之少師古曰戇愚也古者下紺翻今則竹巷翻}
陳平可以助之陳平知有餘然難獨任周勃厚重少文|{
	知讀曰智少詩沼翻}
然安劉氏者必勃也可令為太尉呂后復問其次|{
	復扶又翻}
上曰此後亦非乃所知也|{
	師古曰乃汝也言自此之後汝亦終矣不復知之}
夏四月甲辰帝崩于長樂宫|{
	夀五十三 考異曰漢書云呂后與審食其謀盡誅諸將酈商見審食其說以如此大臣内畔諸將外反亡可蹻足待也審食其入言之乃以丁未發喪按呂后雖暴戾亦安敢一旦盡誅大臣又時陳平不在滎陽樊噲不在代此說恐妄今不取}
丁未發喪大赦天下 盧綰與數千人居塞下候伺幸上疾愈自入謝|{
	師古曰冀得上疾愈自入謝以為己身之幸也}
聞帝崩遂亡入匈奴五月丙寅葬高帝於長陵|{
	班志長陵縣高帝置屬左馮翊皇甫謐曰長陵在渭水北去長安城三十五里臣瓚曰在長安北四十里括地志在雍州咸陽縣東三十里漢官儀曰古不墓祭秦始皇起寢於墓側漢因而不改諸陵寢皆以晦朔二十四氣三伏社臘及四時上飯其親陵所宫人隨鼓漏理被枕具盥水陳粧具陵旁起邑置令丞尉奉守}
初高祖不脩文學而性明達好謀能聽自監門戍卒見之如舊初順民心作三章之約|{
	見九卷元年}
天下既定命蕭何次律令|{
	帝既滅項羽四夷未附兵革未息三章之法不足以禦奸蕭何攗摭秦法取其宜于時者作律九章}
韓信申軍法|{
	帝命張良韓信序次兵法凡百八十二家刪取要用定著三十五家諸呂用事而盜取之}
張蒼定章程|{
	如淳曰章歷數之章術也程者權衡尺斗斛之平法也師古曰程法式也}
叔孫通制禮儀|{
	見上卷六年七年}
又與功臣剖符作誓丹書鐵契金匱石室藏之宗廟|{
	剖符作誓謂剖符封功臣刑白馬與為山河帶礪之盟也丹書鐵契者以鐵為契以丹書之如淳曰金匱猶金縢也師古曰以金為匱以石為室重緘封之重慎之義盖謂以丹書盟誓之言于鐵券盛之以金匱石室而藏之宗廟也}
雖日不暇給規摹弘遠矣|{
	鄧展曰若畫工規模物之摹韋昭曰正員之器曰規摹者如畫工未施朱土摹之矣師古曰取喻規摹謂立制立範也給足也日不暇給言衆事繁多常汲汲也余謂日不暇給盖言項羽既平諸侯又叛也}
己巳太子即皇帝位尊皇后曰皇太后 初高帝病甚人有惡樊噲云黨於呂氏即一日上晏駕|{
	師古曰惡謂毁譛言其罪惡也音如字晏駕者天子當晨起早作而忽崩隕不出臨朝凡臣子之心猶謂宫車晚出也}
欲以兵誅趙王如意之屬帝大怒用意平謀召絳侯周勃受詔床下曰陳平亟馳傳載勃代噲將平至軍中即斬噲頭二人既受詔馳傳未至軍|{
	如淳曰駟馬高足為置傳中足為馳傳律諸當乘傳及發駕置傳皆持尺五寸木傳信封以御史大夫印章其乘傳參封之參三也有期會累封兩端端各兩封凡四封也傳株戀翻}
行計之曰樊噲帝之故人也功多且又呂后弟呂之夫|{
	音須師古曰行計謂于道中行且計也}
有親且貴帝以忿怒故欲斬之則恐後悔寧囚而致上自誅之未至軍為壇以節召樊噲噲受詔即反接|{
	師古曰反縛兩手也}
載檻車傳詣長安|{
	傳柱戀翻遞也}
而令絳侯勃代將將兵定燕反縣平行聞帝崩|{
	師古曰未至京師於道中聞高帝崩}
畏呂讒之於太后乃馳傳先去逢使者詔平與灌嬰屯滎陽平受詔立復馳至宫哭殊悲因固請得宿衛中|{
	請得宿衛禁中也復扶又翻下同}
太后乃以為郎中令|{
	班表郎中令秦官掌宫殿掖門戶武帝太初元年更名光禄勲}
使傅教惠帝是後呂讒乃不得行樊噲至則赦復爵邑太后令永巷囚戚夫人髠鉗衣赭衣令舂|{
	赭衣囚服也以赤土}


|{
	染之赭止也翻}
遣使召趙王如意使者三反趙相周昌謂使者曰高帝屬臣趙王|{
	屬之欲翻}
王年少|{
	少詩照翻下同}
竊聞太后怨戚夫人欲召趙王并誅之臣不敢遣王王且亦病不能奉詔太后怒先使人召昌昌至長安乃使人復召趙王王來未到帝知太后怒自迎趙王霸上與入宫自挟與起居飲食太后欲殺之不得間|{
	間古莧翻隙也}


孝惠皇帝|{
	荀悦曰諱盈之字曰滿師古曰臣下以滿字代盈者則知帝諱盈也他皆類此高帝嫡長子應劭曰禮諡法柔質慈民曰惠師古曰孝子善述人之志故漢家之諡自惠帝以下皆稱孝也}


元年冬十二月帝晨出射趙王年少不能蚤起太后使人持酖飲之|{
	廣志鴆鳥大如鴞毛紫緑色有毒頸長七八寸食蝮蛇雄名運日雌名隂諧以其毛歷飲食則殺人范咸大曰鴆聞邕州朝天鋪及山深處有之形如鵶差大黑身赤目音如羯鼓唯食毒蛇遇蛇則鳴聲邦邦然蛇入石穴則于穴外禹步作法有頃石碎啄蛇吞之山有鴆草木不生秋冬之間脱羽往時人以銀作爪拾取著銀瓶中否則手爛墮鴆矢著人立死集於石石亦裂此禽至兇極毒所謂酖即鴆酒也陸佃埤雅曰鴆似鷹而紫黑喙長七八寸作銅色食蛇蛇入口輒爛屎溺著石石亦為之爛羽翮有毒以櫟酒飲殺人惟犀角可以解故有鴆處必有犀飲於禁翻}
犂明|{
	徐廣曰犂猶比也比至天明也諸言犂明者將明時也呂靜曰犂結也力奚翻程大昌曰徐說非也犂黎古字通黎黑也黑與明相雜欲曉未曉之交也猶曰昧爽也昧暗也爽明也亦明暗相雜也遲明即未及乎明也厥明質明則已曉也康云力追切未知何據}
帝還趙王已死太后遂斷戚夫人手足去眼煇耳飲瘖藥|{
	斷丁管翻去羌呂翻師古曰去其眼睛以藥薰耳令聾也瘖不能言也以瘖藥飲之瘖於今翻}
使居厠中命曰人彘居數日乃召帝觀人彘帝見問知其戚夫人乃大哭因病歲餘不能起使人請太后曰此非人所為臣為太后子終不能治天下|{
	師古曰令太后治事已自如太子然余謂惠帝之意盖以謂身為太后子而不能容父之寵姬是終不能治天下也治直之翻}
帝以此日飲為淫樂不聽政|{
	樂音洛}


臣光曰為人子者父母有過則諫諫而不聽則號泣而隨之|{
	見記曲禮號戶高翻}
安有守高祖之業為天下之主不忍母之殘酷遂棄國家而不恤縱酒色以傷生若孝惠者可謂篤於小仁而未知大誼也

徙淮陽王友為趙王|{
	高祖十一年封友於淮陽}
春正月始作長安城西北方|{
	漢都長安蕭何雖治宫室未暇築城帝始築之至五年乃畢故書以始事杜佑曰惠帝所築長安城在今大興城西北苑中}


二年冬十月齊悼惠王來朝|{
	高祖庶長子肥也朝直遥翻}
飲於太后前帝以齊王兄也置之上坐|{
	盖於宫中以兄弟齒列為序非外朝君臣之禮坐徂卧翻}
太后怒酌酖酒置前賜齊王為夀齊王起帝亦起取巵太后恐自起泛帝巵|{
	漢書音義泛音幡索隱音捧余據泛駕之泛其義為覆則音覂亦通}
齊王怪之因不敢飲佯醉去問知其酖大恐齊内史士說王|{
	師古曰内史王國官士其名也班表王國有内史掌治民}
使獻城陽郡為魯元公主湯沐邑太后喜乃罷歸齊王 春正月癸酉有兩龍見蘭陵家人井中|{
	班志蘭陵縣屬東海郡師古曰家人言庶人之家五行志曰温陵之家見賢遍翻}
隴西地震 夏旱 郃陽侯仲薨|{
	仲即代王喜封郃陽事見上卷高祖七年}
酇文終侯蕭何病|{
	諡法有始有卒曰終蒙曰克成令名曰終}
上親自臨視因問曰君即百歲後誰可代君者對曰

知臣莫如主帝曰曹參何如何頓首曰帝得之矣臣死不恨 秋七月辛未何薨何置田宅必居窮僻處為家不治垣屋|{
	師古曰僻隱也垣墻也治直之翻}
曰後世賢師吾儉不賢毋為埶家所奪癸巳以曹參為相國參聞何薨吿舍人趣治行|{
	師古曰舍人猶言家人也一曰私屬官主家事者也余據戰國時蘇秦使舍人資送張儀入秦李斯為呂不韋舍人謂為私屬官可也以為主家事則拘矣趣讀曰促速也冶行謂飭治行裝也}
吾將入相居無何|{
	居無何謂居無幾時也相息亮翻下同}
使者果召參始參微時與蕭何善及為將相有隙至何且死所推賢惟參|{
	言推舉以為賢也}
參代何為相舉事無所變更|{
	師古曰舉皆也言凡事無更改更工衡翻}
一遵何約束擇郡國吏木訥於文辭|{
	木質朴也訥謇於言也}
重厚長者即召除為丞相史|{
	漢制丞相官屬長史之下有掾史令史等}
吏之言文刻深欲務聲名者輒斥去之日夜飲醇酒|{
	斥卻也逐也師古曰醇酒不澆謂厚酒也去羌呂翻}
卿大夫以下吏及賓客見參不事事|{
	言不事丞相之事}
來者皆欲有言參輒飲以醇酒間欲有所言復飲之醉而後去終莫得開說以為常|{
	開啟也謂有所啟白以為常者飲之以酒也飲於禁翻復扶又翻}
見人有細過專掩匿覆盖之|{
	覆敷救翻}
府中無事參子窋為中大夫|{
	窋張律翻}
帝怪相國不治事以為豈少朕與|{
	師古曰言豈以我為年少故也治直之翻與讀曰歟}
使窋歸以其私問參參怒笞窋二百曰趣入侍天下事非若所當言也至朝時帝讓參曰乃者我使諫君也|{
	師古曰乃者猶言曩者朝直遥翻}
參免冠謝曰陛下自察聖武孰與高帝上曰朕乃安敢望先帝又曰陛下觀臣能孰與蕭何賢上曰君似不及也參曰陛下言之是也高帝與蕭何定天下法令既明今陛下垂拱參等守職遵而勿失不亦可乎帝曰善參為相國出入三年百姓歌之曰蕭何為法較若畫一|{
	較若猶今言較然也畫一言其整齊也}
曹參代之守而勿失載其清淨|{
	師古曰載猶乘也}
民以寧壹三年春發長安六百里内男女十四萬六千人城長安三十日罷 以宗室女為公主嫁匈奴冒頓單于是時冒頓方彊為書使使遺高后辭極䙝嫚|{
	遺于季翻下同䙝息列翻汚也嫚傲也}
高后大怒召將相大臣議斬其使者發兵擊之樊噲曰臣願得十萬衆横行匈奴中中郎將季布曰噲可斬也|{
	漢有五官左右中郎三將秩二千石典領中郎屬郎中令}
前匈奴圍高帝於平城|{
	見上卷高祖七年 考異曰季布傳云前陳豨反於代相匈奴圍高帝于平城按平城之圍乃韓王信反非陳豨反也}
漢兵三十二萬噲為上將軍不能解圍今歌吟之聲未絶傷夷者甫起而噲欲搖動天下妄言以十萬衆横行是面謾也|{
	謾莫連翻又莫官切又音慢欺誑也}
且夷狄譬如禽獸得其善言不足喜惡言不足怒也高后曰善令大謁者張釋報書|{
	謁者秦官掌賓贊受事員七十人大謁者盖其長也 考異曰史記文帝本紀及惠景間侯者表漢書匈奴傳皆作澤史記呂后本紀八年中大謁者張釋漢書紀作釋卿恩澤侯表及周勃傳皆云張釋顔師古注曰荆燕吳傳云張擇今從史記呂后本紀漢書恩澤侯表周勃傳}
深自謙愻以謝之|{
	愻與遜同順也}
并遺以車二乘馬二駟|{
	乘繩證翻}
冒頓復使使來謝|{
	復扶又翻}
曰未嘗聞中國禮義陛下幸而赦之因獻馬遂和親 夏五月立閩越君揺為東海王揺與無諸皆越王句踐之後也|{
	句音鉤}
從諸侯滅秦功多其民便附故立之都東甌世號東甌王|{
	閩越王無諸高祖五年受封都冶今福州官是也帝又封揺於東海東海即東甌今温州永嘉是也應劭曰揺封東海在吳郡東南濱海此閩越所由分也}
六月發諸侯王列侯徒隸二萬人城長安|{
	自元年始作長安城}


|{
	西北方今年春又發長安六百里内男女就役不欲復勞之故發王侯徒隸}
秋七月都廏災|{
	都廏大廏也屬太僕}
是歲蜀湔氐反|{
	班志湔氐道屬蜀郡㟭山在西徼外江水所出又百官表有蠻夷曰道則其地盖湔氐居之故曰道也湔則前翻裴松之音翦氐丁奚翻}
擊平之四年冬十月立皇后張氏后帝姊魯元公主女也太后欲為重親故以配帝|{
	后張敖女也魯元公主降敖而生后因下文重親故直書帝姊魯元公主女既以紀人倫之變且著外戚固寵也重直龍翻}
春正月舉民孝弟力田者復其身|{
	善事父母為孝善事兄長為弟力田者取其竭力服勤于田事孝弟人倫之大力田人生之本故令郡國舉之復其身以風厲天下也弟讀曰悌復方目翻}
三月甲子皇帝冠赦天下|{
	帝年十七即位至是始冠孔頴達曰案畧說周公對成王云古者冒而句領注云古人謂三皇時以冒覆頭句領繞頸至黄帝時則有冕也世本謂黄帝造火食旃冕是冕起于黄帝也但黄帝以前則以羽皮為之冠黄帝以後乃用布帛其冠之年則天子諸侯十二而冠故襄九年左傳云古者國君十五而生子冠而生子禮也其士則二十而冠古者行冠禮於廟初加緇布冠次加皮弁冠三加爵弁冠所謂三加彌尊加有成也諸侯則四加而有玄冕故大戴禮云公冠四加也諸侯尚加四則天子當五加衮冕也鄭樵曰漢改皇帝冠為加元服初加緇布進賢次爵弁次武弁次通天冠冠訖皆於高祖廟如禮謁見}
省法令妨吏民者除挾書律|{
	應劭曰挟藏也張晏曰秦律挟書者族今始除之}
帝以朝太后於長樂宫及間往數蹕煩民|{
	師古曰非大朝見中間小謁見曰間往天子出入警蹕辟止行人數蹕則人以為煩鄭氏周禮注曰國有事王當出則禁絶行者若今時衛士填街蹕也賈公彦疏曰漢儀大駕行幸使衛士填塞街巷備非常也蹕壁吉翻}
乃築複道於武庫南|{
	武庫在長樂未央之間故築複道始於武庫南}
奉常叔孫通諫曰此高帝月出遊衣冠之道也|{
	服䖍曰持高廟中衣冠月旦以遊於衆廟已而復之應劭曰月旦出高帝衣冠備法駕名曰遊衣冠如淳曰高祖之衣冠藏在宫中之寢三月出遊其道正直今之所作複道下故言乘宗廟道上行也晉灼曰黄圖高廟在長安城門街東寢在桂宫北服言衣冠藏於廟中如言宫中皆非也師古曰諸家之說皆未允也謂從高帝陵寢出衣冠遊於高廟每月一為之漢制則然而後之學者不曉其意謂以月出之時夜遊衣冠皆非也}
子孫奈何乘宗廟道上行哉帝懼曰急壞之|{
	壞音怪}
通曰人主無過舉今已作百姓皆知之矣願陛下為原廟渭北|{
	師古曰原重也先已有廟今更立之故云重也}
月出遊之益廣宗廟大孝之本上乃詔有司立原廟|{
	鄭氏曰廟之言貌也死者精神不可得而見但以生時之居立宫室象貌為之耳孝經注宗尊也廟貌也}


臣光曰過者人之所必不免也惟聖賢為能知而改之古之聖王患其有過而不自知也故設誹謗之木置敢諫之鼓|{
	後漢書曰堯置敢諫之鼓賈誼曰三代之君則有進善之旌誹謗之木敢諫之鼓}
豈畏百姓之聞其過哉是以仲虺美成湯曰改過不吝傅說戒高宗曰無恥過作非由是觀之則為人君者固不以無過為賢而以改過為美也今叔孫通諫孝惠乃云人主無過舉是教人君以文過遂非也豈不繆哉

長樂宫鴻臺灾|{
	三輔黄圖鴻臺在長樂宫中秦始皇二十七年築高四十丈上起觀宇帝嘗射飛鴻于臺上故曰鴻臺}
秋七月乙亥未央宫凌室灾丙子織室灾|{
	凌室藏冰之室織室掌織作繒帛之處班表少府有東織西織凌力證翻又音陵}


五年冬雷|{
	洪範論曰陽用事百八十三日而終隂用事百八十三日而終雷出地百八十三日而入地入地百八十三日而復出地是其常經也冬雷為失常}
桃李華棗實 春正月復發長安六百里内男女十四萬五千人城長安三十日罷 夏大旱江河水少谿谷水絶 秋八月平陽懿侯曹參薨|{
	諡法温柔賢善曰懿}


六年冬十月以王陵為右丞相陳平為左丞相 齊悼惠王肥薨 夏留文成侯張良薨|{
	周公諡法安民立政曰成賀琛臣諡佐相克終曰成}
以周勃為太尉

七年冬發車騎材官詣滎陽太尉灌嬰將|{
	將即亮翻}
春正月辛丑朔日有食之 夏五月丁卯日有食之既 秋八月戊寅帝崩于未央宫大赦天下九月辛丑葬安陵|{
	臣瓚曰夀二十四安陵在長安北三十里師古曰去長陵一十里}
初呂太后命張皇后取他人子養之而殺其母以為太子既葬太子即皇帝位年幼太后臨朝稱制|{
	師古曰天子之言一曰制書二曰詔書制書者謂制度之命也非皇后所得稱今太后臨朝行天子事故稱制}


資治通鑑卷十二
