\section{資治通鑑卷八十七}
宋 司馬光 撰

胡三省 音註

晉紀九|{
	起屠維大荒落盡重光協洽凡三年}


孝懷皇帝中

永嘉三年春正月辛丑朔熒惑犯紫微|{
	紫微即紫宫也}
漢太史令宣于修之 |{
	考異曰晉春秋作鮮于修之今從載記十六國春秋余按姓氏諸書有鮮于而無宣于}
言于漢主淵曰不出三年必克洛陽蒲子崎嶇難以久安|{
	崎丘奇翻嶇丘于翻}
平陽氣象方昌請徙都之淵從之大赦改元河瑞|{
	時汾水得玉璽淵因改元河瑞}
三月戊申高密孝王略薨以尚書左僕射山簡為征南將軍都督荆湘交廣四州諸軍事鎮襄陽|{
	代略也}
簡濤之子也嗜酒不恤政事表順陽内史劉璠得衆心恐百姓劫璠為主詔徵璠為越騎校尉|{
	璠扶元翻}
南州由是遂亂父老莫不追思劉宏|{
	史言劉宏父子得江漢間民心}
丁巳太傅越自滎陽入京師|{
	越自去年徙屯滎陽}
中書監王敦謂所親曰太傅專執威權而選用表請尚書猶以舊制裁之今日之來必有所誅帝之為太弟也與中庶子繆播親善及即位以播為中書監繆胤為太僕卿|{
	太僕九卿也但晉官未有卿字卿字衍}
委以心膂帝舅散騎常侍王延尚書何綏太史令高堂冲並參機密越疑朝臣貳於己|{
	散悉亶翻騎奇寄翻朝直遥翻}
劉輿潘滔勸越悉誅播等越乃誣播等欲為亂乙丑遣平東將軍王秉帥甲士三千入宫執播等十餘人於帝側付廷尉殺之|{
	越因繆播兄弟以克河間今又殺之權勢之爭可畏哉帥讀曰率}
帝歎息流涕而已綏曾之孫也初何曾侍武帝宴退謂諸子曰主上開創大業吾每宴見|{
	見賢遍翻}
未嘗聞經國遠圖惟說平生常事非貽厥孫謀之道也及身而已後嗣其殆乎|{
	嗣祥吏翻}
汝輩猶可以免指諸孫曰此屬必及於難|{
	難乃旦翻}
及綏死兄嵩哭之曰我祖其殆聖乎曾日食萬錢猶云無下箸處|{
	箸遟據翻梜也}
子劭日食二萬綏及弟機羨汰侈尤甚與人書疏詞禮簡傲河内王尼見綏書謂人曰伯蔚居亂世而矜豪乃爾其能免乎人曰伯蔚聞卿言必相危害尼曰伯蔚比聞我言自已死矣|{
	何綏字伯蔚比必寐翻及也蔚紆勿翻}
及永嘉之末何氏無遺種|{
	種章勇翻}


臣光曰何曾議武帝偷惰取過目前不為遠慮知天下將亂子孫必與其憂|{
	與讀曰預}
何其明也然身為僭侈使子孫承流卒以驕奢亡族|{
	卒子恤翻}
其明安在哉且身為宰相知其君之過不以告而私語於家非忠臣也

太傅越以王敦為揚州刺史|{
	為敦亂東晉張本}
劉寔連年請老朝廷不許尚書左丞劉坦上言古之養老以不事為優|{
	不事謂不使任事也}
不以吏之為重謂宜聽寔所守丁卯詔寔以侯就第以王衍為太尉太傅越解兖州牧領司徒越以頃來興事多由殿省|{
	謂誅楊駿廢賈后誅趙王倫齊王冏及討成都王頴及羊后太子覃屢廢屢立皆殿中人為之}
乃奏宿衛有侯爵者皆罷之時殿中武官並封侯由是出者略盡皆泣涕而去更使右衛將軍何倫左衛將軍王秉領東海國兵數百人宿衛|{
	自是帝左右皆越私人}
左積弩將軍朱誕奔漢|{
	武帝泰始四年罷振威揚威護軍置左右積弩將軍}
具陳洛陽孤弱勸漢主淵攻之淵以誕為前鋒都督以滅晉大將軍劉景為大都督將兵攻黎陽克之乂敗王堪於延澤沈男女三萬餘人於河|{
	敗補邁翻沈持林翻}
淵聞之怒曰景何面復見朕|{
	復扶又翻}
且天道豈能容之吾所欲除者司馬氏耳細民何罪黜景為平虜將軍|{
	劉淵之識略非聰曜所能及也}
夏大旱江漢河洛皆竭可涉|{
	川竭亡國之徵}
漢安東大將軍石勒寇鉅鹿常山衆至十餘萬集衣冠人物别為君子營|{
	石勒起於胡羯餓隸而能如此此其所以能跨有中原也}
以趙郡張賓為謀主刁膺為股肱夔安孔萇支雄桃豹逯明為爪牙|{
	姓譜夔子之後以國為姓後趙支雄傳云其先月支胡人也桃春秋魯邑以邑為姓一曰古高士左伯桃之後逯盧谷翻}
并州諸胡羯多從之|{
	羯居謁翻}
初張賓好讀書|{
	好呼到翻}
濶達有大志常自比張子房及石勒狥山東賓謂所親曰吾歷觀諸將無如此胡將軍者|{
	勒本胡也故謂之胡將軍}
可與共成大業乃提劒詣軍門大呼請見勒亦未之奇也賓數以策干勒|{
	呼火故翻數所角翻}
已而皆如所言勒由是奇之署為軍功曹動靜咨之 漢主淵以王彌為侍中都督青徐兖豫荆揚六州諸軍事征東大將軍青州牧與楚王聰共攻壺關以石勒為前鋒都督劉琨遣護軍黄肅韓述救之聰敗述於西澗勒敗肅於封田皆殺之|{
	西澗封田皆當在壺關東南敗補邁翻 考異曰石勒載記肅作秀封作白今從十六國春秋及劉琨集}
太傅越遣淮南内史王曠 |{
	考異曰十六國春秋作王廣今從帝紀}
將軍施融曹超將兵拒聰等曠濟河欲長驅而前融曰彼乘險閒出|{
	閒古莧翻}
我雖有數萬之衆猶是一軍獨受敵也且當阻水為固以量形埶|{
	量音良}
然後圖之曠怒曰君欲沮衆邪|{
	沮在呂翻}
融退曰彼善用兵曠闇於事勢吾屬今必死矣曠等於太行|{
	行戶剛翻}
與聰遇戰於長平之間曠兵大敗融超皆死聰遂破屯留長子|{
	屯音純}
凡斬獲萬九千級上黨太守龎淳以壺關降漢|{
	降戶江翻 考異曰十六國春秋作劉惇劉琨傳作襲醇今從帝紀}
劉琨以都尉張倚領上黨太守據襄垣|{
	襄垣縣屬上黨郡宋白曰襄垣趙襄子所築因以為名}
初匈奴劉猛死|{
	見七十九卷武帝太始八年}
右賢王去卑之子誥升爰代領其衆誥升爰卒子虎立居新興號鐵弗氏|{
	鐵弗氏之後為赫連勃勃}
與白部鮮卑皆附於漢 |{
	考異曰劉琨集作百部今從後魏書晉書}
劉琨自將擊虎|{
	將即亮翻 考異曰帝紀七月劉聰及王彌圍壺關琨使兵救之為聰所敗王廣等及聰戰又敗龎淳以郡降賊十六國春秋淵五月遣聰攻壺關敗韓述黄肅六月晉遣王廣等來討七月戰于長平晉師敗劉惇以壺關降按劉琨集載六月癸巳琨答太傅府書曰聰彌入上黨龎惇不能禦又曰安居失利韓述授首封田之敗黄肅不還浹辰之間名將仍殄又曰即重遣江陶都尉張倚領上黨太守疾據襄垣續遣鷹揚將軍趙擬梁徐都尉李茂與倚併力輕行夜襲賊捐弃輜車宵遁而退追尋討截獲三分之二當聰彌之未走烏九劉虎搆為變逆西招白部遣使致任稱臣於淵殘州困弱内外受敵輒背聰而討虎自四月八日攻圍然則琨討虎以上事皆在四月以前也蓋晉漢二史皆據奏報事畢而言之今依琨集為定}
劉聰遣兵襲晉陽不克 五月漢主淵封子裕為齊王隆為魯王 秋八月漢主淵命楚王聰等進攻洛陽詔平北將軍曹武等拒之皆為聰所敗|{
	敗補邁翻}
聰長驅至宜陽自恃驟勝怠不設備九月弘農太守垣延詐降|{
	垣姓延名降戶江翻}
夜襲聰軍聰大敗而還王浚遣祁宏與鮮卑段務勿塵擊石勒于飛龍山|{
	隋地理志恒山郡石邑縣有飛龍山括地志封龍山一名飛龍山在恒山鹿泉縣南四十五里}
大破之勒退屯黎陽 冬十月漢主淵復遣楚王聰王彌始安王曜汝隂王景帥精騎五萬寇洛陽|{
	復扶又翻帥讀曰率騎奇寄翻}
大司空鴈門剛穆公呼延翼帥步卒繼之|{
	北狄傳匈奴四姓有呼延氏卜氏蘭氏喬氏而呼延氏最貴}
丙辰聰等至宜陽朝廷以漢兵新敗不意其復至大懼辛酉聰屯西明門|{
	西明門洛城西面南頭第二門也}
北宫純等夜帥勇士千餘人出攻漢壁斬其征虜將軍呼延顥壬戌聰南屯洛水|{
	洛水過洛城南}
乙丑呼延翼為其下所殺其衆自大陽潰歸淵勑聰等還師聰表稱晉兵微弱不可以翼顥死故還師固請留攻洛陽淵許之太傅越嬰城自守戊寅聰親祈嵩山|{
	嵩山在河南陽城縣}
留平晉將軍安陽哀王厲冠軍將軍呼延朗督攝留軍|{
	冠古玩翻}
太傅參軍孫詢說越乘虛出擊朗斬之|{
	說輸芮翻}
厲赴水死王彌謂聰曰今軍既失利洛陽守備猶固運車在陜糧食不支數日|{
	聰自宜陽而東又南進屯于洛水既為晉所敗運車在陜糧道隔絶陜失冉翻}
殿下不如與龍驤還平陽|{
	淵以族子曜為龍驤大將軍驤思將翻}
裹糧發卒更為後舉下官亦收兵穀待命於兖豫不亦可乎聰自以請留未敢還宣于修之言於淵曰歲在辛未乃得洛陽今晉氣猶盛大軍不歸必敗淵乃召聰等還天水人訇琦等|{
	訇呼宏翻}
殺成太尉李離尚書令閻式以梓潼降羅尚|{
	降戶江翻}
成主雄遣太傅驤司徒雲司空璜攻之不克雲璜戰死初譙周有子居巴西成巴西太守馬脫殺之其子登詣劉宏請兵以復讐宏表登為梓潼内史使自募巴蜀流民得二千人西上|{
	上時掌翻}
至巴郡從羅尚求益兵不得登進攻宕渠|{
	宕渠縣漢屬巴郡自蜀以來屬巴西郡賢曰宕渠故城在今渠州流江縣東北宕徒浪翻}
斬馬脫食其肝會梓潼降登進據涪城|{
	涪音浮}
雄自攻之為登所敗|{
	敗補邁翻}
十一月甲申漢楚王聰始安王曜歸于平陽王彌南出轘轅|{
	轘音環}
流民之在潁川襄城汝南南陽河南者數萬家|{
	襄陽縣漢屬潁川郡武帝泰始二年分立襄城郡}
素為居民所苦皆燒城邑殺二千石長吏以應彌|{
	長知兩翻}
石勒寇信都|{
	信都縣漢屬信都國後漢屬安平國晉同}
殺冀州刺史王斌|{
	斌音彬}
王浚自領冀州詔車騎將軍王堪北中郎將裴憲將兵討勒勒引兵還拒之魏郡太守劉矩以郡降勒|{
	降戶江翻}
勒至黎陽裴憲棄軍奔淮南王堪退保倉垣|{
	倉垣城在陳留浚儀縣水經汴水出浚儀縣北東逕倉垣城南即入梁縣之倉垣亭也城臨汴水}
十二月漢主淵以陳留王歡樂為太傅|{
	樂音洛}
楚王聰為大司徒江都王延年為大司空遣都護大將軍曲陽王賢與征北大將軍劉靈安北將軍趙固平北將軍王桑東屯内黄|{
	内黄縣屬魏郡應劭曰陳留有外黄故加内云}
王彌表左長史曹嶷行安東將軍東徇青州且迎其家淵許之|{
	嶷魚力翻為曹嶷據青州張本王彌家在東萊}
初東夷校尉勃海李臻與王浚約共輔晉室浚内有異志臻恨之和演之死也|{
	見八十五卷惠帝永興元年}
别駕昌黎王誕亡歸李臻說臻舉兵討浚臻遣其子成將兵擊浚 |{
	考異曰燕書王誕傳成作咸今從李洪傳說輸芮翻}
遼東太守龎本素與臻有隙乘虚襲殺臻遣人殺成於無慮|{
	無慮縣前漢屬遼東後漢屬遼東屬國晉省應劭曰慮音閭周禮所謂其山醫巫閭是也}
誕亡歸慕容廆詔以勃海封釋代臻為東夷校尉龎本復謀殺之|{
	廆乎罪翻復扶又翻}
釋子悛勸釋伏兵請本收斬之悉誅其家|{
	悛七倫翻又且緣翻}


四年春正月乙丑朔大赦 漢主淵立單徵女為皇后|{
	單徵氐酋也歸漢見上卷二年單音善}
梁王和為皇太子大赦封子义為北海王以長樂王洋為大司馬|{
	樂音洛}
漢鎮東大將軍石勒濟河抜白馬王彌以三萬衆會之共寇徐豫兖州二月勒襲鄄城|{
	鄄音絹}
殺兖州刺史袁孚遂抜倉垣殺王堪復北濟河攻冀州諸郡|{
	復扶乂翻}
民從之者九萬餘口成太尉李國鎮巴西帳下文石殺國以巴西降羅尚|{
	降戶江翻}
太傅越徵建威將軍吳興錢璯|{
	吳分吳郡丹陽置吳興郡以自烏程興故也璯黄外翻}
及揚州刺史王敦璯謀殺敦以反敦奔建業告琅邪王睿璯遂反進寇陽羨|{
	陽羨縣前漢屬會稽郡後漢屬吳郡自吴以來分屬吴興郡賢曰陽羨故城在今常州義興縣南}
睿遣將軍郭逸等討之周玘糾合鄉里與逸等共討璯斬之玘三定江南|{
	惠帝永興元年討石氷永嘉元年討陳敏今又誅璯是三定江南}
睿以玘為吳興太守於其鄉里置義興郡以旌之|{
	時分吴興之陽羨及長城縣之西鄉丹陽之永世為義興郡}
曹嶷自大梁引兵而東所至皆下遂克東平進攻琅邪夏四月王浚將祁宏敗漢冀州刺史劉靈於廣宗殺之|{
	將即亮翻下同廣宗縣漢屬鉅鹿郡晉屬安平國敗補邁翻}
成主雄謂其將張寶曰汝能得梓潼吾以李離之官賞汝寶乃先殺人而亡奔梓潼訇琦等信之委以心腹會羅尚遣使至梓潼|{
	使疏吏翻}
琦等出送之寶從後閉門琦等奔巴西雄以寶為太尉 幽并司冀秦雍六州大蝗食草木牛馬毛皆盡|{
	雍於用翻}
秋七月漢楚王聰始安王曜石勒及安北大將軍趙國圍河内太守裴整于懷詔征虜將軍宋抽救懷勒與平北大將軍王桑逆擊抽殺之河内人執整以降|{
	降戶江翻}
漢主淵以整為尚書左丞河内督將郭默收整餘衆自為塢主|{
	城之小者曰塢天下兵爭聚衆築塢以自守未有朝命故自為塢主將即亮翻}
劉琨以默為河内太守 羅尚卒於巴郡詔以長沙太守下邳皮素代之|{
	姓譜皮姓樊仲皮之後}
庚午漢主淵寢疾辛未以陳留王歡樂為太宰長樂王洋為太傅|{
	樂音洛}
江都王延年為太保楚王聰為大司馬大單于並録尚書事置單于臺於平陽西|{
	單音蟬}
以齊王裕為大司徒魯王隆為尚書令北海王义為撫軍大將軍領司隸校尉始安王曜為征討大都督領單于左輔廷尉喬智明為冠軍大將軍領單于右輔光禄大夫劉殷為左僕射王育為右僕射任顗為吏部尚書|{
	任音壬顗魚豈翻}
朱紀為中書監護軍馬景領左衛將軍永安王安國領右衛將軍安昌王盛安邑王欽西陽王璿皆領武衛將軍分典禁兵|{
	璿旬緣翻}
初盛少時不好讀書|{
	少詩照翻好呼到翻}
唯讀孝經論語曰誦此能行足矣安用多誦而不行乎李熹見之歎曰望之如可易|{
	易弋䜴翻慢易也}
及至肅如嚴君可謂君子矣淵以其忠篤故臨終委以要任丁丑淵召太宰歡樂等入禁中受遺詔輔政己卯淵卒 |{
	考異曰十六國春秋八月丁丑淵召太宰歡樂等受遺詔己卯卒辛未葬按長歷七月壬戌朔十六日丁丑十八日己卯八月辛卯朔無丁丑己卯及辛未辛未乃九月十一日蓋淵以七月卒九月葬十六國春秋誤也}
太子和即位|{
	和字玄泰淵之嫡子}
和性猜忌無恩宗正呼延攸翼之子也淵以其無才行終身不遷官侍中劉乘素惡楚王聰衛尉西昌王銳恥不預顧命乃相與謀說和曰先帝不惟輕重之勢使三王總彊兵於内|{
	惟思也三王謂安昌王盛安邑王欽西陽王璿也或曰三王謂齊王裕魯王隆北海王义行下孟翻惡烏路翻說輸芮翻}
大司馬擁十萬衆屯於近郊|{
	謂聰屯平陽西也}
陛下便為寄坐耳|{
	言大權非已出託位於臣民之上勢同寄寓也坐徂卧翻}
宜早為之計和攸之甥也深信之辛巳夜召安昌王盛安邑王欽等告之盛曰先帝梓宫在殯四王未有逆節|{
	聰淵之第四子故曰四王或曰謂聰裕隆义也}
一旦自相魚肉天下謂陛下何且大業甫爾陛下勿信讒夫之言以疑兄弟兄弟尚不可信他人誰足信哉攸銳怒之曰今日之議理無有二領軍是何言乎命左右刃之盛既死欽懼曰惟陛下命壬午銳帥馬景攻楚王聰于單于臺|{
	單音蟬}
攸帥永安王安國攻齊王裕于司徒府乘帥安邑王欽攻魯王隆使尚書田密武衛將軍劉璿攻北海王义密璿挾义斬關歸于聰聰命貫甲以待之|{
	貫甲擐甲也帥讀曰率}
銳知聰有備馳還與攸乘共攻隆裕攸乘疑安國欽有異志殺之是日斬裕癸未斬隆甲申聰攻西明門克之|{
	劉淵都平陽諸城門皆用洛陽諸城門名}
鋭等走入南宫前鋒隨之乙酉殺和於光極西室|{
	劉淵起光極殿於平陽}
收鋭攸乘梟首通衢|{
	梟堅堯翻}
羣臣請聰即帝位聰以北海王义單后之子也以位讓之 |{
	考異曰載記作义十六國春秋作义今從之}
义涕泣固請聰久而許之曰义及羣公正以禍難尚殷貪孤年長故耳|{
	難乃旦翻長知兩翻}
此家國之事孤何敢辭俟义年長當以大業歸之遂即位|{
	聰字玄明淵第四子}
大赦改元光興尊單氏曰皇太后其母張氏曰帝太后以义為皇太弟領大單于大司徒其妻呼延氏為皇后呼延氏淵后之從父妹也|{
	從才用翻}
封其子粲為河内王易為河間王翼為彭城王悝為高平王|{
	悝苦回翻}
仍以粲為撫軍大將軍都督中外諸軍事以石勒為并州刺史封汲郡公 略陽臨渭氐酋蒲洪驍勇多權略羣氐畏服之|{
	晉志略陽郡在臨渭縣蓋魏所置也載記曰氐之先蓋有扈氏之苗裔世為西戎酋長洪家池中生蒲長五丈五節如竹形時咸謂之蒲家因以為氏其後洪以䜟文有草付應王又其孫堅背文有艸付字遂改姓苻氏}
漢主聰遣使拜洪平遠將軍洪不受自稱護氐校尉秦州刺史略陽公|{
	蒲洪事始此}
九月辛未葬漢主淵于永光陵謚曰光文皇帝廟號

高祖 雍州流民多在南陽詔書遣還鄉里流民以關中荒殘皆不願歸征南將軍山簡南中郎將杜蕤各遣兵送之促期令發京兆王如遂潛結壯士夜襲二軍破之|{
	二軍山簡及杜蕤所遣之軍也}
於是馮翊嚴嶷|{
	嶷魚力翻}
京兆侯脫各聚衆攻城鎮殺令長以應之|{
	長知兩翻}
未幾衆至四五萬|{
	幾居豈翻}
自號大將軍領司雍二州牧|{
	雍於用翻}
稱藩于漢 冬十月漢河内王粲始安王曜及王彌帥衆四萬寇洛陽石勒帥騎二萬會粲于大陽敗監軍裴邈于澠池|{
	帥讀曰率敗補邁翻下同澠彌兖翻}
遂長驅入洛川粲出轘轅掠梁陳汝潁間|{
	轘音還}
勒出成臯關|{
	晉志河南成臯縣有關}
壬寅圍陳留太守王讚於倉垣為讚所敗退屯文石津|{
	據帝紀文石津在河北又據永嘉六年勒自葛陂北行至東燕使孔萇自文石津潛度枋頭取向氷船則文石津在東燕之東北枋頭之東南}
劉琨自將討劉虎及白部|{
	白部鮮卑也琨以劉虎及白部皆附漢故討之}
遣使卑辭厚禮說鮮卑拓抜猗盧以請兵|{
	說輸芮翻}
猗盧使其弟弗之子欝律帥騎二萬助之遂破劉虎白部屠其營琨與猗盧結為兄弟表猗盧為大單于以代郡封之為代公時代郡屬幽州王浚不許遣兵擊猗盧猗盧拒破之浚與琨由是有隙猗盧以封邑去國懸遠民不相接乃帥部落萬餘家自雲中入鴈門從琨求陘北之地|{
	陘北石陘關之北也陘音刑}
琨不能制且欲倚之為援乃徙樓煩馬邑隂館繁畤崞五縣民於陘南|{
	樓煩匈奴之所居其地在北河之南今嵐州樓煩郡非古樓煩也漢馬邑則唐之大同軍是其地漢隂館縣在句注西北繁畤縣在武州川崞縣為北齊北顯州平寇縣今五縣雖存皆非古縣地矣陘謂陘嶺陘音刑}
以其地與猗盧|{
	考異曰懷帝紀永嘉五年十一月猗盧寇太原劉琨徙五縣居之六年八月辛亥劉琨乞師于猗盧表盧為代公宋書索虜傳在永嘉三年晉春秋在永嘉四年且云猗盧率萬餘家避難自雲中入鴈門後魏序紀在穆帝三年即永嘉四年也琨集永嘉四年六月癸巳上太傳府牋云盧感封代之恩故知在四年六月之前又琨與丞相牋曰昔車騎感猗㐌救州之勲表以代郡封㐌為代公見聽時大駕在長安會值戎事道路不通竟未施行盧以封事見託琨實為表上追述車騎前意即蒙聽許遣兼謁者僕射拜盧賜印及符冊浚以此見責戎狄封華郡誠為失禮然蓋以救弊耳亦猶浚先以遼西封務勿塵此禮之失浚實啟之浚遂與盧爭代郡舉兵擊盧為所破紛錯之由始結於此鴈門郡有五縣在陘北盧新并塵官國甚彊盛從琨求陘北地以並遣三萬餘家散在五縣間既非所制又於琨殘弱之計得相聚集未為失宜即徙陘北五縣著陘南盧因移頗侵逼浚西陲圍塞諸軍營浚不復見恕危弱而見罪責以此觀之盧非避難而來也}
由是猗盧益盛琨遣使言於太傅越請出兵共討劉聰石勒越忌苟晞及豫州刺史馮嵩|{
	越晞有隙事見上卷二年嵩蓋亦不心附越者}
恐為後患不許琨乃謝猗盧之兵遣歸國|{
	考異曰後魏序紀曰劉琨乞師救洛穆帝遣步騎二萬助之東海王越以洛陽饑荒不許按琨與丞相牋曰琨傾身竭辭北和猗盧遂引大衆躬啟戎行即具白太傅切陳愚見取賊之計聰宜時討勒不可縱而宰相意異所慮不同更憂苟晞馮嵩之徒而稽二寇之誅遣使折抑挫臣鋭氣臣即解甲遣盧衆歸國若猗盧果遣衆赴洛琨牋安得不言也}
劉虎收餘衆西度河居朔方肆盧川|{
	肆盧川在朔方塞内後拓抜氏於其地置肆盧郡真君七年併入秀容郡魏收地形志秀容郡秀容縣有肆盧城}
漢主聰以虎宗室封樓煩公 壬子以劉琨為平北大將軍王浚為司空進鮮卑段務勿塵為大單于|{
	單音蟬}
京師饑困日甚太傅越遣使以羽檄徵天下兵使入援京師帝謂使者曰為我語諸征鎮今日尚可救後則無及矣既而卒無至者|{
	為于偽翻語牛倨翻卒子恤翻}
征南將軍山簡遣督護王萬將兵入援軍于涅陽|{
	涅陽縣屬南陽郡應劭曰在涅水之陽師古曰涅音乃結翻}
為王如所敗|{
	敗補邁翻}
如遂大掠沔漢進逼襄陽簡嬰城自守荆州刺史王澄|{
	澄時治江陵}
自將欲援京師至沶口|{
	水經注零水上通梁州沔陽縣東逕新城郡之沶鄉縣謂之沶水又東歷宜城西山謂之沶溪東流合於夷水謂之沶口楊正衡曰沶音怡}
聞簡敗衆散而還|{
	還從宣翻又如字}
朝議多欲遷都以避難|{
	朝直遥翻難乃旦翻}
王衍以為不可賣車牛以安衆心山簡為嚴嶷所逼自襄陽徙屯夏口|{
	夏戶雅翻}
石勒引兵濟河將趣南陽|{
	趣七喻翻}
王如侯脫嚴嶷等聞之遣衆一萬屯襄城以拒勒勒擊之盡俘其衆進屯宛北是時侯脫據宛|{
	宛於元翻}
王如據穰|{
	穰縣漢屬南陽郡晉屬義陽郡}
如素與脫不協遣使重賂勒結為兄弟說勒使攻脫|{
	說輸芮翻}
勒攻宛克之嚴嶷引兵救宛不及而降|{
	降戶江翻}
勒斬脫囚嶷送于平陽盡并其衆遂南寇襄陽攻抜江西壘壁三十餘所|{
	勒既南寇襄陽循漢而下攻掠江西}
還趣襄城王如遣弟璃襲勒勒迎擊滅之復屯江西|{
	復扶又翻江西大江之西也趣七喻翻}
太傅越既殺王延等|{
	見上永嘉三年}
大失衆望又以胡寇益盛内不自安乃戎服入見|{
	見賢遍翻}
請討石勒且鎮集兖豫帝曰今胡虜侵逼郊畿人無固志朝廷社稷倚賴於公豈可遠出以孤根本對曰臣出幸而破賊則國威可振猶愈於坐待困窮也十一月甲戌越帥甲士四萬向許昌留妃裴氏世子毗及龍驤將軍李惲右衛將軍何倫守衛京師|{
	帥讀曰率惲於粉翻}
防察宫省以潘滔為河南尹摠留事越表以行臺自隨用太尉衍為軍司朝賢素望悉為佐吏名將勁卒咸入其府|{
	朝直遥翻將即亮翻}
於是宫省無復守衛荒饉日甚殿内死人交横盜賊公行府寺營署並掘塹自守|{
	塹七艶翻}
越東屯項以馮嵩為左司馬自領豫州牧竟陵王楙白帝遣兵襲何倫不克帝委罪於楙楙逃竄得免|{
	楙即東平王楙帝踐祚改封竟陵王}
揚州都督周馥以洛陽孤危上書請遷都壽春太傅越以馥不先白已而直上書大怒召馥及淮南太守裴頠馥不肯行令頠帥兵先進|{
	帥讀曰率}
頠詐稱受越密旨襲馥為馥所敗|{
	敗補邁翻}
退保東城|{
	東城縣漢屬九江郡後漢屬下邳國晉屬淮南郡宋白曰濠州定遠縣漢東城縣地}
詔加張軌鎮西將軍都督隴右諸軍事|{
	考異曰帝紀云安西按惠帝永興二年已加軌安西將軍今從本傳}
光禄大夫傅祗太

常摯虞遺軌書|{
	遺于季翻}
告以京師饑匱軌遣參軍杜勲獻馬五百匹㲜布三萬匹|{
	㲜布織毳為布也㲜吐敢翻}
成太傅驤攻譙登於涪城羅尚子宇及參佐素惡登不給其糧|{
	涪音浮惡烏路翻}
益州刺史皮素怒欲治其罪|{
	治直之翻}
十二月素至巴郡羅宇使人夜殺素建平都尉暴重殺宇巴郡亂驤知登食盡援絶攻涪愈急士民皆熏鼠食之餓死甚衆無一人離叛者驤子壽先在登所登乃歸之|{
	永興元年羅尚掠得驤妻及其子壽囚在登所}
三府官屬表巴東監軍南陽韓松為益州刺史|{
	三府平西將軍府益州刺史府西戎校尉府皆羅尚兼領者也監古銜翻}
治巴東 初帝以王彌石勒侵逼京畿詔苟晞督帥州郡討之|{
	帥讀曰率}
會曹嶷破琅邪|{
	嶷魚力翻}
北收齊地兵勢甚盛苟純閉城自守晞還救青州與嶷連戰破之|{
	永嘉元年苟晞討魏植留弟純守青州}
是歲寧州刺史王遜到官表李釗為朱提太守|{
	朱提音銖時}
時寧州外逼於成内有夷寇城邑丘墟遜惡衣菜食招集離散勞來不倦|{
	勞力到翻來力代翻}
數年之間州境復安誅豪右不奉灋者十餘家以五苓夷昔為亂首|{
	見八十五卷惠帝太安二年苓力丁翻}
擊滅之内外震服 漢主聰自以越次而立忌其嫡兄恭因恭寢穴其壁間刺而殺之|{
	刺七亦翻}
漢太后單氏卒|{
	單音善}
漢主聰尊母張氏為皇太后單氏年少美色聰烝焉|{
	下淫上曰烝上淫下曰報少詩照翻}
太弟义屢以為言單氏慙恚而死|{
	恚於避翻}
义寵由是漸衰然以單氏故尚未之廢也呼延后言於聰曰父死子繼古今常道陛下承高祖之業|{
	劉淵廟號高祖}
太弟何為者哉陛下百年後粲兄弟必無種矣|{
	種章勇翻}
聰曰然吾當徐思之呼延氏曰事留變生太弟見粲兄弟浸長|{
	長知兩翻下齒長同}
必有不安之志萬一有小人交構其間未必不禍發于今日也|{
	言义將殺聰}
聰心然之|{
	為元帝建武元年聰殺义張本}
义舅光禄大夫單冲泣謂义曰踈不間親主上有意於河内王矣殿下何不避之义曰河瑞之末主上自惟嫡庶之分|{
	間古莧翻分扶問翻惟思也}
以大位讓义义以主上齒長故相推奉天下者高祖之天下兄終弟及何為不可粲兄弟既壯猶今日也且子弟之間親疎距幾主上寧可有此意乎|{
	聰讓义事見上义此言必不發於是年通鑑因呼延氏之言遂連書之幾居豈翻}


五年春正月壬申苟晞為曹嶷所敗|{
	敗補邁翻}
棄城奔高平|{
	高平縣舊屬梁國晉為高平國泗水逕其西有高平山山東西十里南北五里高四里其山最高頂上方平故謂之高平山縣亦取名焉}
石勒謀保據江漢參軍都尉張賓以為不可會軍中饑疫死者太半乃渡沔寇江夏|{
	夏戶雅翻}
癸酉抜之 乙亥成太傅驤抜涪城|{
	涪音浮}
獲譙登太保始抜巴西殺文石於是成主雄大赦改元玉衡譙登至成都雄欲宥之登詞氣不屈雄殺之 巴蜀流民布在荆湘間數為土民所侵苦|{
	數所角翻}
蜀人李驤聚衆據樂鄉反|{
	此又一李驤也非成太傅之李驤}
南平太守應詹與醴陵令杜弢共擊破之|{
	醴陵縣屬長沙郡弢土刀翻}
王澄使成都内史王機討驤|{
	惠帝時蜀亂割南郡之華容州陵監利三縣别立豐都一縣置成都郡為成都王穎國}
驤請降|{
	降戶江翻}
澄偽許而襲殺之以其妻子為賞沈八千餘人於江流民益怨忿蜀人杜疇等復反|{
	沈持林翻復扶又翻}
湘州參軍馮素與蜀人汝班有隙|{
	汝姓也商有汝鳩汝方晉有汝寛汝齊}
言於刺史荀眺曰巴蜀流民皆欲反眺信之|{
	眺他弔翻}
欲盡誅流民流民大懼四五萬家一時俱反以杜弢州里重望|{
	弢蜀郡人以才學著稱於西州弢他刀翻}
共推為主弢自稱梁益二州牧領湘州刺史 裴頠求救於琅邪王睿睿使揚威將軍甘卓等攻周馥於壽春馥衆潰奔項 |{
	考異曰帝紀戊寅睿使卓攻馥於壽春馥衆潰未知其為命卓之日與攻日潰日故闕之}
豫州都督新蔡王確執之馥憂憤而卒確騰之子也 揚州刺史劉陶卒琅邪王睿復以安東軍諮祭酒王敦為揚州刺史尋加都督征討諸軍事|{
	去年敦奔建業}
庚辰平原王幹薨 二月石勒攻新蔡殺新蔡莊王確於南頓進抜許昌殺平東將軍王康 氐苻成隗文復叛|{
	苻成等歸羅尚見八十五卷惠帝太安二年復扶又翻}
自宜都趣巴東|{
	趣七喻翻}
建平都尉暴重討之重因殺韓松自領三府事 東海孝獻王越既與苟晞有隙|{
	事始上卷永嘉二年}
河南尹潘滔尚書劉望等復從而譖之|{
	復扶又翻}
晞怒表求滔等首揚言司馬元超為宰相不平|{
	東海王越字元超}
使天下淆亂苟道將豈可以不義使之乃移檄諸州自稱功伐陳越罪狀帝亦惡越專權多違詔命所留將士何倫等抄掠公卿逼辱公主|{
	抄楚交翻}
密賜晞手詔使討之晞數與帝文書往來|{
	惡烏路翻數所角翻}
越疑之使遊騎於成臯間伺之|{
	騎奇寄翻下同伺相吏翻}
果獲晞使及詔書|{
	使疏吏翻}
乃下檄罪狀晞以從事中郎楊瑁為兖州刺史|{
	瑁莫報翻}
使與徐州刺史裴盾共討晞|{
	盾徒損翻}
晞遣騎收潘滔滔夜遁得免執尚書劉曾侍中程延斬之越憂憤成疾以後事付王衍三月丙子薨于項 |{
	考異曰帝紀五年正月帝密詔苟晞討越乙未越遣楊瑁裴盾共擊晞三月戊午詔下越罪狀告方鎮討之以晞為大將軍丙子越薨晞傳晞移告諸州陳越罪狀帝惡越專權乃詔晞施檄六州協同大舉晞移諸征鎮帝又密詔晞討越晞復上表稱李初至奉被手詔卷甲長驅次于倉垣五年帝復詔晞陳越罪惡詔至之日宣告天下率齊大舉晞表稱輒遣王讚將兵詣項越使騎於成臯間獲晞使遂大構嫌隙晉春秋五年正月上遣李初詔晞討越按越若已得晞使則帝亦不能自安潘滔何倫等不容晏然在洛且滔等未去帝亦不敢明言使晞討越年月事迹既前後參差如此今並置於越薨之時庶為不失}
秘不發喪衆共推衍為元帥|{
	帥所類翻}
衍不敢當以讓襄陽王範範亦不受範瑋之子也於是衍等相與奉越喪還葬東海何倫李惲等聞越薨奉裴妃及世子毗自洛陽東走城中士民爭隨之帝追貶越為縣主以苟晞為大將軍大都督督青徐兖豫荆揚六州諸軍事 益州將吏共殺暴重表巴郡太守張羅行三府事羅與隗文等戰死文等驅掠吏民西降於成|{
	降戶江翻}
三府文武共表平西司馬蜀郡王異行三府事|{
	益州殘兵自不足以進取未及朞年而五易帥適為秦氏兼并之資耳}
領巴郡太守 初梁州刺史張光會諸郡守於魏興共謀進取張燕唱言漢中荒敗迫近大賊|{
	近其靳翻}
克復之事當俟英雄光以燕受鄧定賂致失漢中|{
	事見上卷永嘉元年}
今復沮衆|{
	復扶又翻沮雍呂翻}
呵出斬之治兵進戰|{
	呵虎何翻治直之翻}
累年乃得至漢中綏撫荒殘百姓悦服|{
	光為梁州刺史見上卷二年}
夏四月石勒率輕騎追太傅越之喪及於苦縣甯平城|{
	苦縣屬陳郡水經註甯平城在沙水北本前漢淮陽國之寧平縣也後漢改淮陽為陳國晉省寧平縣而故城猶在賢曰寧平故城在今亳州谷陽縣西南騎奇寄翻下同}
大敗晉兵縱騎圍而射之|{
	敗補邁翻射而亦翻}
將士十餘萬人相踐如山無一人得免者執太尉衍襄陽王範任城王濟武陵莊王澹|{
	任音壬澹徒覽翻又徒濫翻}
西河王喜梁懷王禧齊王超|{
	西河王喜宣帝弟西河繆王斌之後超齊王冏之子}
吏部尚書劉望廷尉諸葛銓豫州刺史劉喬太傳長史庾敳等|{
	敳魚開翻}
坐之幕下問以晉故衍具陳禍敗之由云計不在己且自言少無宦情不豫世事因勸勒稱尊號冀以自免勒曰君少壯登朝名蓋四海身居重任何得言無宦情邪破壞天下非君而誰|{
	少詩照翻壞音怪}
命左右扶出衆人畏死多自陳述獨襄陽王範神色儼然顧呵之曰今日之事何復紛紜|{
	復扶又翻}
勒謂孔萇曰吾行天下多矣未嘗見此輩人當可存乎|{
	勒欲存之以諸人儀觀之清楚耳}
萇曰彼皆晉之王公終不為吾用勒曰雖然要不可加以鋒刃夜使人排牆殺之濟宣帝弟子景王陵之子禧澹之子也剖越柩焚其尸曰亂天下者此人也吾為天下報之故焚其骨以告天地何倫等至洧倉|{
	水經洧水東南過潁川長社縣分一支東流過許昌縣又東入汶倉城内俗以是水為汶水故有汶倉之名蓋洧水之邸閣耳}
遇勒戰敗東海世子及宗室四十八王皆没於勒 |{
	考異曰東海王越傳云三十六王今從帝紀}
何倫犇下邳李惲犇廣宗|{
	惲於粉翻}
裴妃為人所掠賣久之渡江初琅邪王睿之鎮建業裴妃意也故睿德之厚加存撫以其子冲繼越後 漢趙固王桑攻裴盾殺之|{
	盾時在彭城盾徒損翻}
杜弢攻長沙五月荀眺棄城奔廣州弢追擒之於是弢南破零桂東掠武昌殺二千石長吏甚衆|{
	弢土刀翻}
以太子太傅傳祗為司徒尚書令荀藩為司空加王浚大司馬侍中大都督督幽冀諸軍事南陽王模為太尉大都督張軌為車騎大將軍琅邪王睿為鎮東大將軍兼督揚江湘交廣五州諸軍事初太傅越以南陽王模不能綏撫關中|{
	時關中饑荒疾癘盜賊公行模不能制}
表徵為司空將軍淳于定說模使不就徵|{
	說輸芮翻}
模從之表遣世子保為平西中郎將鎮上邽秦州刺史裴苞拒之模使帳下都尉陳安攻苞苞犇安定太守賈疋納之|{
	疋音雅}
苟晞表請遷都倉垣使從事中郎劉會將船數十艘|{
	艘蘇刀翻}
宿衛五百人穀千斛迎帝帝將從之公卿猶豫左右戀資財遂不果行既而洛陽饑困人相食百官流亡者什八九帝召公卿議將行而衛從不備|{
	從才用翻下導從同}
帝撫手歎曰如何會無車輿乃使傳祗出詣河隂治舟楫|{
	河隂本漢平隂縣魏文帝改曰河隂在洛陽東北屬河南郡治直之翻}
朝士數十人導從|{
	從才用翻}
帝步出西掖門至銅駝街|{
	水經註洛陽城中太尉司徒兩坊間謂之銅駝街魏明帝置銅駞於閶闔南街即此陸機洛陽記曰洛陽有銅駞街漢鑄銅駞二枚在宫南四會道相對俗語曰金馬門外集衆賢銅駞陌上集少年}
為盜所掠不得進而還|{
	藉使帝得至倉垣亦遺石勒禽矣}
度支校尉東郡魏浚率流民數百家保河隂之峽石|{
	水經註河南新安縣東有千秋亭亭東有雍谷溪囘岫縈紆石路阻峽故亦有峽石之稱}
時刧掠得穀麥獻之帝以為揚威將軍平陽太守度支如故|{
	度徒洛翻}
漢主聰使前軍大將軍呼延晏將兵二萬七千寇洛陽比及河南|{
	河南縣屬河南尹周東都王城郟鄏也比必寐翻}
晉兵前後十二敗死者三萬餘人始安王曜王彌石勒皆引兵會之未至晏留輜重於張方故壘|{
	張方故壘在洛陽西七里重直用翻}
癸未先至洛陽甲申攻平昌門|{
	平昌門洛城南面東頭第一門}
丙戌克之遂焚東陽門及諸府寺六月丁未朔晏以外繼不至俘掠而去帝具舟於洛水將東走晏盡焚之庚寅荀藩及弟光禄大夫組奔轘轅|{
	轘音環}
辛卯王彌至宣陽門|{
	宣陽門洛城南面東來第四門亦謂之謻門}
壬辰始安王曜至西明門丁酉王彌呼延晏克宣陽門入南宫升太極前殿縱兵大掠悉收宫人珍寶帝出華林園門欲奔長安漢兵追執之幽於端門曜自西明門入屯武庫戊戌曜殺太子詮吳孝王晏竟陵王楙右僕射曹馥尚書閭丘冲河南尹劉默等士民死者三萬餘人遂發掘諸陵焚宫廟官府皆盡曜納惠帝羊皇后遷帝及六璽於平陽|{
	璽斯氏翻}
石勒引兵出轘轅屯許昌光禄大夫劉蕃尚書盧志犇并州|{
	從劉琨也蕃琨之父也}
丁未漢主聰大赦改元嘉平以帝為特進左光禄大夫封平阿公 |{
	考異曰帝紀聰以帝為會稽公載記三十國春秋云平阿公晉春秋云平河公河字蓋誤十六國三十國晉春秋明年二月乃封帝會稽公蓋先封平阿後進會稽帝紀闕畧今從諸書}
以侍中庾珉王儁為光禄大夫珉敳之兄也|{
	敳魚開翻}
初始安王曜以王彌不待已至先入洛陽怨之彌說曜曰|{
	說輸芮翻下暾說同}
洛陽天下之中山河四塞城池宫室不假修營宜白主上自平陽徙都之曜以天下未定洛陽四面受敵不可守不用彌策而焚之彌罵曰屠各子豈有帝王之意邪|{
	晉書曰北狄以部落為類其入居塞内者有屠各等十九種皆有部落不相雜錯屠各最豪貴故得為單于統理諸種屠直於翻杜佑曰頭曼冒頓即屠各種也}
遂與曜有隙引兵東屯項關|{
	陳郡項縣有項關}
前司隸校尉劉暾說彌曰|{
	暾他昆翻}
今九州糜沸羣雄競逐將軍於漢建不世之功又與始安王相失將何以自容不如東據本州|{
	彌青州東萊人}
徐觀天下之勢上可以混壹四海下不失鼎峙之業策之上者也彌心然之 司徒傅祇建行臺於河隂司空荀藩在陽城|{
	陽城縣漢屬潁川郡晉屬河南郡}
河南尹華薈在城臯汝隂太守平陽李矩為之立屋輸穀以給之薈歆之曾孫也|{
	華戶化翻汝隂縣漢屬汝南郡魏分置汝隂郡後廢武帝泰始二年復為郡薈烏外翻為于偽翻}
藩與弟組族子中護軍崧薈與弟中領軍恒建行臺於密|{
	密縣漢屬河南郡晉屬滎陽郡}
傳檄四方推琅邪王睿為盟主藩承制以崧為襄城太守矩為滎陽太守前冠軍將軍河南褚翜為梁國内史|{
	冠古玩翻翜山立翻又所甲翻}
揚威將軍魏浚屯洛北石梁塢劉琨承制假浚河南尹浚詣荀藩諮謀軍事藩邀李矩同會矩夜赴之矩官屬皆曰浚不可信不宜夜往矩曰忠臣同心何所疑乎遂往相與結歡而去浚族子該聚衆據一泉塢|{
	水經註洛水過盧氏縣南又東逕一合塢南城在川北原上高二十丈南北東三箱天險峭絶惟築西面即為全固一合之名起於是矣劉曜之攻河南也晉將軍魏該奔於此該傳曰一泉塢在宜陽}
藩以為武威將軍|{
	沈約志魏置將軍四十號威武第十無武威}
豫章王端太子詮之弟也|{
	詮且緣翻}
東奔倉垣苟晞率羣官奉以為皇太子置行臺端承制以晞領太子太傅都督中外諸軍録尚書事自倉垣徙屯蒙城|{
	蒙縣屬梁國}
撫軍將軍秦王業吳孝王之子荀藩之甥也年十二南犇密 |{
	考異曰晉書愍帝諱鄴又改建鄴為建康按三十國晉春秋愍帝名子業或作業又吳志孫權改秣陵為建業取興建基業為名皆不為鄴字今從之}
藩等奉之南越許昌|{
	趣七喻翻}
前豫州刺史天水閻鼎聚西州流民數千人於密欲還鄉里荀藩以鼎有才而擁衆用鼎為豫州刺史以中書令李絙|{
	絙居登翻 考異曰閻鼎傳作李恒今從王浚傳}
司徒左長史彭城劉疇鎮軍長史周顗|{
	東海王越子毗為鎮軍將軍以顗為長史顗魚豈翻}
司馬李述等為之參佐顗浚之子也時海内大亂獨江東差安中國士民避亂者多南渡江鎮東司馬王導說琅邪王睿收其賢俊與之共事睿從之辟掾屬百餘人|{
	自漢以來公府有掾有屬職官分紀曰掾屬當敦明信義肅清風俗非禮不言非法不行以訓羣吏說輸芮翻掾以絹翻下同}
時人謂之百六掾以前潁川太守勃海刁協為軍諮祭酒前東海太守王承廣陵相卞壼為從事中郎江寧令諸葛恢|{
	壼苦本翻吳孫權改秣陵為建業晉平吳復曰秣陵武帝太康二年分秣陵立江寧縣}
歷陽參軍陳國陳頵為行參軍|{
	頵居筠翻}
前太傅掾庾亮為西曹掾承渾之弟子恢靚之子|{
	靚疾正翻}
亮兖之弟子也江州刺史華軼|{
	華戶化翻軼音逸}
歆之曾孫也自以受朝廷

之命|{
	軼永嘉中除江州}
而為琅邪王睿所督多不受其教令郡縣多諫之軼曰吾欲見詔書耳及睿承荀藩檄承制署置官司改易長吏|{
	長知兩翻}
軼與豫州刺史裴憲皆不從命睿遣揚州刺史王敦歷陽内史甘卓與揚烈將軍廬江周訪合兵擊軼軼兵敗奔安成|{
	吳孫皓寶鼎二年分豫章廬陵長沙立安成郡宋白曰吉州安福縣本漢安成縣今縣西六十里有安成故城}
訪追斬之及其五子裴憲犇幽州睿以甘卓為湘州刺史周訪為尋陽太守又以揚武將軍陶侃為武昌太守|{
	吳孫權改鄂曰武昌晉武帝太康元年復立鄂縣而武昌如故改吴之江夏曰武昌郡漢尋陽縣屬廬江郡其地在江北惠帝永興元年分廬江武昌立尋陽郡治豫章之柴桑尋陽遂在江南}
秋七月王浚設壇告類|{
	告類祭也以事類告天及五帝也}
立皇太子|{
	考異曰晉書初無其名劉琨與丞相牋曰浚設壇場有所建立稱皇太子不知為誰}
布告天下稱受中詔承制封拜備置百官列署征鎮以荀藩為太尉琅邪王睿為大將軍浚自領尚書令以裴憲及其壻棗嵩為尚書以田徽為兖州刺史李惲為青州刺史|{
	惲於粉翻}
南陽王模使牙門趙染戍蒲坂|{
	劉聰在平陽欲窺關中蒲坂兵衝也坂音反}
染求馮翊太守不得而怒帥衆降漢|{
	帥讀曰率降戶江翻}
漢主聰以染為平西將軍八月聰遣染與安西將軍劉雅帥騎二萬攻模于長安河内王粲始安王曜帥大衆繼之染敗模兵於潼關長驅至下邽|{
	敗補邁翻下邽縣前漢屬京兆後漢省併入鄭縣桓帝復置晉屬馮翊郡應劭曰有上邽故稱下秦武公伐邽戎置宋白曰四夷縣道記下邽縣東南二十五里有下邽故城在渭水北師古曰邽音圭}
涼州將北宫純自長安帥其衆降漢|{
	將即亮翻}
漢兵圍長安模遣淳于定出戰而敗模倉庫虚竭士卒離散遂降於漢趙染送模於河内王粲九月粲殺模 |{
	考異曰帝紀八月模遇害按劉琨上丞相牋曰平昌以九月遇禍世子時鎮隴右故得無恙今以為據}
關西饑饉白骨蔽野士民存者百無一二聰以始安王曜為車騎大將軍雍州牧更封中山王鎮長安|{
	雍於用翻}
以王彌為大將軍封齊公 苟晞驕奢苛暴前遼西太守閻亨纘之子也數諫晞晞殺之|{
	數所角翻}
從事中郎明預有疾|{
	姓譜明秦大夫孟明之後為平原望姓}
自轝入諫晞怒曰我殺閻亨何關人事而輿病罵我|{
	轝羊茹翻}
預曰明公以禮待預故預以禮自盡今明公怒預其如遠近怒明公何桀為天子猶以驕暴而亡况人臣乎願明公且置是怒思預之言晞不從由是衆心離怨加以疾疫饑饉石勒攻王纘於陽夏擒之|{
	陽夏縣屬陳郡夏音賈}
遂襲蒙城執晞及豫章王端鎖晞頸以為左司馬漢主聰拜勒幽州牧王彌與勒外相親而内相忌劉暾說彌使召曹嶷之兵以圖勒|{
	說輸芮翻}
彌為書使暾召嶷且邀勒共向青州暾至東阿|{
	東阿縣漢屬東郡晉屬濟北國}
勒游騎獲之勒潛殺暾而彌不知會彌將徐邈高梁輒引所部兵去彌兵漸衰|{
	將即亮翻}
彌聞勒擒苟晞心惡之|{
	惡烏路翻}
以書賀勒曰公獲苟晞而用之何其神也使晞為公左彌為公右天下不足定也勒謂張賓曰王公位重而言卑其圖我必矣賓因勸勒乘彌小衰誘而取之|{
	誘音酉}
時勒方與乞活陳午相攻於關|{
	關在陳留浚儀縣班志曰澤在河南開封縣東北臣瓚曰今浚儀有陂是也}
彌亦與劉瑞相持甚急彌請救於勒勒未之許張賓曰公常恐不得王公之便今天以王公授我矣陳午小豎不足憂王公人傑當早除之勒乃引兵擊瑞斬之彌大喜謂勒實親已不復疑也冬十月勒請彌燕于已吾|{
	已吾縣後漢屬陳留郡魏晉省陳留風俗傳曰縣故宋地雜以陳楚之地故梁國寧陵縣之徙種龍鄉也以成哀之世戶至八九千冠帶之士求置縣永元十一年陳王削地以大棘鄉直陽鄉闕  自隸之命以嘉名曰已吾猶有陳楚之俗焉}
彌將往長史張嵩諫不聽酒酣勒手斬彌而并其衆表漢主聰稱彌叛逆聰大怒遣使讓勒專害公輔有無君之心然猶加勒鎮東大將軍督并幽二州諸軍事領并州刺史以尉其心苟晞王讚潛謀叛勒勒殺之并晞弟純勒引兵掠豫州諸郡臨江而還|{
	還從宣翻又如字}
屯于葛陂|{
	續漢書郡國志汝南郡鮦陽縣有葛陂賢曰葛陂在今豫州新蔡縣西北}
初勒之為人所掠賣也|{
	事見上卷惠帝永興二年}
與其母王氏相失劉琨得之并其從子虎送於勒因遺勒書曰|{
	從才用翻遺于季翻下同}
將軍用兵如神所向無敵所以周流天下而無容足之地百戰百勝而無尺寸之功者蓋得主則為義兵附逆則為賊衆故也成敗之數有似呼吸吹之則寒嘘之則温今相授侍中車騎大將軍領護匈奴中郎將襄城郡公將軍其受之勒報書曰事功殊途非腐儒所知君當逞節本朝吾自夷難為効|{
	勒書意度雄爽此必張賓為之難乃旦翻}
遺琨名馬珍寶厚禮其使謝而絶之時虎年十七殘忍無度為軍中患勒白母曰此兒凶暴無賴|{
	應劭曰賴者恃也晉灼曰許慎曰賴利也無利入於家也或曰江准之間謂小兒多詐狡獪為無賴師古曰晉說是也}
使軍人殺之聲名可惜不若自除之母曰快牛為犢多能破車汝小忍之及長便弓馬勇冠當時勒以為征虜將軍每屠城邑鮮有遺類|{
	長知兩翻冠古玩翻鮮息淺翻}
然御衆嚴而不煩莫敢犯者指授攻討所向無前勒遂寵任之|{
	石虎始此為虎誅夷勒諸子張本}
勒攻滎陽太守李矩矩擊却之 初南陽王模以從事中郎索綝為馮翊太守綝靖之子也模死綝與安夷護軍金城麴允頻陽令梁肅俱奔安定|{
	索昔各翻姓也綝丑林翻頻陽縣屬馮翊郡秦厲公置應劭曰在頻水之陽杜佑曰京兆同官縣漢祋祤縣晉為頻陽縣時綝等自京兆南山犇安定安夷護軍蓋亦置司於長安}
時安定太守賈疋與諸氐羌皆送任子於漢綝等遇之於隂密擁還臨涇|{
	隂密縣屬安定郡商之密國詩所謂密人不恭敢距大邦者也臨涇縣時為安定郡治所}
與疋謀興復晉室疋從之乃共推疋為平西將軍率衆五萬向長安雍州刺史麴特新平太守竺恢皆不降於漢|{
	特與恢同守新平雍於用翻}
聞疋起兵與扶風太守梁綜帥衆十萬會之|{
	帥讀曰率}
綜肅之兄也漢河内王粲在新豐使其將劉雅趙染攻新平不克索綝救新平大小百戰雅等敗退中山王曜與疋等戰於黄丘|{
	黄丘在馮翊雲陽縣黄嶔山下}
曜衆大敗疋遂襲漢梁州刺史彭蕩仲殺之|{
	蕩仲安定盧水胡也據後蕩仲子天護漢以為涼州刺史此梁當作涼}
麴特等擊破粲於新豐粲還平陽於是疋等兵勢大振關西胡晉翕然響應閻鼎欲奉秦王業入關據長安以號令四方河隂令傅暢祇之子也亦以書勸之鼎遂行荀藩劉疇周顗李述等皆山東人不欲西行中塗逃散鼎遣兵追之不及殺李絙等|{
	絙居登翻}
鼎與業自宛趣武關遇盜於上洛|{
	上洛縣漢屬宏農郡漢元鼎四年置居洛水上因以為名晉初改為京兆南部武帝泰始二年分京兆南部置上洛郡杜佑曰上洛漢長利縣宛於元翻趣七喻翻}
士卒敗散收其餘衆進至藍田使人告賈疋疋遣兵迎之十二月入于雍城|{
	雍於用翻}
使梁綜將兵衛之周顗奔琅邪王睿睿以顗為軍諮祭酒前騎都尉譙國桓彞亦避亂過江見睿微弱謂顗曰我以中州多故來此求全而單弱如此將何以濟既而見王導共論世事退謂顗曰向見管夷吾無復憂矣|{
	以王導比管仲也}
諸名士相與登新亭遊宴|{
	金陵覽古曰新亭在江寧縣十里近臨江渚按新亭蓋近勞勞亭}
周顗中坐歎曰風景不殊舉目有江河之異|{
	言洛都遊宴多在河濱而新亭臨江渚也坐徂卧翻}
因相視流涕王導愀然變色曰|{
	愀七小翻}
當共戮力王室克復神州|{
	戰國時騶衍以為中國者於天下乃八十一分居其一分耳中國名曰赤縣神州赤縣神州内自有九州禹之所序九州是也孔頴達曰按地統書括地象云地中央曰崑崙又云其東南方五十里曰神州以此言之崑崙在西北别統四方九州其神州者是崑崙東南一州耳於一州中更分九州則禹貢之九州是也又隋祭北郊有神州迎州冀州戎州拾州柱州宫州咸州揚州從祀其崑崙所統之四方九州歟}
何至作楚囚對泣邪衆皆收淚謝之陳頵遺王導書曰|{
	頵紆倫翻又居筠翻}
中華所以傾弊者正以取才失所先白望而後實事|{
	白望猶虚名也}
浮競驅馳互相貢薦言重者先顯言輕者後叙遂相波扇|{
	以水為譬也波者水之動也風起則波生相扇而動}
乃至陵遲加有莊老之俗傾惑朝廷養望者為弘雅政事者為俗人王職不卹灋物墜喪|{
	喪息浪翻}
夫欲制遠先由近始今宜改張|{
	漢董仲舒論政曰譬猶琴瑟必改而更張之乃可鼓也}
明賞信罰抜卓茂於密縣|{
	事見四十卷漢光武建武元年}
顯朱邑於桐鄉|{
	朱邑為舒桐鄉嗇夫亷平不苛以愛利為行漢宣帝舉而用之官至大司農}
然後大業可舉中興可冀耳導不能從 劉琨長於招懷而短於撫御一日之中雖歸者數千而去者亦相繼琨遣子遵請兵於代公猗盧又遣族人高陽内史希合衆於中山幽州所統代郡上谷廣甯之民多歸之|{
	廣甯縣漢屬上谷郡晉武帝太康中分立廣甯郡唐屬媯州界}
衆至三萬王浚怒遣燕相胡矩督諸軍|{
	燕於賢翻相息亮翻}
與遼西公段疾陸眷共攻希殺之驅略三郡士女而去疾六眷務勿塵之子也猗盧遣其子六修將兵助琨戌新興 |{
	考異曰晉春秋作利孫按利孫即六修也胡語訛轉焉耳余按孔頴達曰聲相近者聲轉字異}
琨牙門將邢延以碧石獻琨琨以與六修六修復就延求之|{
	復扶又翻}
不得執延妻子延怒以所部兵襲六修六修走延遂以新興附漢請兵以攻并州 李臻之死也|{
	事見上永嘉三年}
遼東附塞鮮卑素喜連木丸津託為臻報仇|{
	素喜連木丸津二部也為于偽翻}
攻陷諸縣殺掠士民屢敗郡兵|{
	敗補邁翻}
連年為寇東夷校尉封釋不能討請與連和連津不從民失業歸慕容廆者甚衆廆稟給遣還願留者即撫存之廆少子鷹揚將軍翰|{
	廆戶罪翻據載記翰於皝為庶兄皝廆第三子則翰非少子也少詩照翻}
言於廆曰自古有為之君莫不尊天子以從民望成大業今連津外以龎本為名内實幸災為亂封使君已誅本請和|{
	誅龎本見上永嘉三年龎皮江翻}
而寇暴不已中原離亂州師不振|{
	州師謂平州之兵東夷校尉所統者是也}
遼東荒散莫之救恤單于不若數其罪而討之|{
	廆自稱鮮卑大單于}
上則興復遼東下則并吞二部|{
	二部謂素喜連及木丸津也}
忠義彰於本朝|{
	朝直遥翻}
私利歸於我國此霸王之基也廆笑曰孺子乃能及此乎遂帥衆東擊連津|{
	帥讀曰率}
以翰為前鋒破斬之盡併二部之衆得所掠民三千餘家及前歸廆者悉以付郡遼東賴以復存|{
	復扶又翻}
封釋疾病屬其孫奕於廆|{
	屬之欲翻}
釋卒廆召奕與語說之曰奇士也補小都督釋子冀州主簿悛幽州參軍抽來奔喪|{
	說與悦同悛丑緣翻又七倫翻}
廆見之曰此家抎抎千斤犍也|{
	抎羽敏翻說文從高而下也犍居言翻犗牛也言千斤之犍人間不可多得若從天而下也}
以道不通喪不得還皆留仕廆廆以抽為長史悛為參軍|{
	史言封氏諸子遂為慕容佐命之臣}
王浚以妻舅崔毖為東夷校尉毖琰之曾孫也|{
	為毖與慕容氏構怨張本崔琰事曹公毖音祕}


資治通鑑卷八十七
