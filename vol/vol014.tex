<!DOCTYPE html PUBLIC "-//W3C//DTD XHTML 1.0 Transitional//EN" "http://www.w3.org/TR/xhtml1/DTD/xhtml1-transitional.dtd">
<html xmlns="http://www.w3.org/1999/xhtml">
<head>
<meta http-equiv="Content-Type" content="text/html; charset=utf-8" />
<meta http-equiv="X-UA-Compatible" content="IE=Edge,chrome=1">
<title>資治通鑒_15-資治通鑑卷十四_15-資治通鑑卷十四</title>
<meta name="Keywords" content="資治通鑒_15-資治通鑑卷十四_15-資治通鑑卷十四">
<meta name="Description" content="資治通鑒_15-資治通鑑卷十四_15-資治通鑑卷十四">
<meta http-equiv="Cache-Control" content="no-transform" />
<meta http-equiv="Cache-Control" content="no-siteapp" />
<link href="/img/style.css" rel="stylesheet" type="text/css" />
<script src="/img/m.js?2020"></script> 
</head>
<body>
 <div class="ClassNavi">
<a  href="/24shi/">二十四史</a> | <a href="/SiKuQuanShu/">四库全书</a> | <a href="http://www.guoxuedashi.com/gjtsjc/"><font  color="#FF0000">古今图书集成</font></a> | <a href="/renwu/">历史人物</a> | <a href="/ShuoWenJieZi/"><font  color="#FF0000">说文解字</a></font> | <a href="/chengyu/">成语词典</a> | <a  target="_blank"  href="http://www.guoxuedashi.com/jgwhj/"><font  color="#FF0000">甲骨文合集</font></a> | <a href="/yzjwjc/"><font  color="#FF0000">殷周金文集成</font></a> | <a href="/xiangxingzi/"><font color="#0000FF">象形字典</font></a> | <a href="/13jing/"><font  color="#FF0000">十三经索引</font></a> | <a href="/zixing/"><font  color="#FF0000">字体转换器</font></a> | <a href="/zidian/xz/"><font color="#0000FF">篆书识别</font></a> | <a href="/jinfanyi/">近义反义词</a> | <a href="/duilian/">对联大全</a> | <a href="/jiapu/"><font  color="#0000FF">家谱族谱查询</font></a> | <a href="http://www.guoxuemi.com/hafo/" target="_blank" ><font color="#FF0000">哈佛古籍</font></a> 
</div>

 <!-- 头部导航开始 -->
<div class="w1180 head clearfix">
  <div class="head_logo l"><a title="国学大师官网" href="http://www.guoxuedashi.com" target="_blank"></a></div>
  <div class="head_sr l">
  <div id="head1">
  
  <a href="http://www.guoxuedashi.com/zidian/bujian/" target="_blank" ><img src="http://www.guoxuedashi.com/img/top1.gif" width="88" height="60" border="0" title="部件查字,支持20万汉字"></a>


<a href="http://www.guoxuedashi.com/help/yingpan.php" target="_blank"><img src="http://www.guoxuedashi.com/img/top230.gif" width="600" height="62" border="0" ></a>


  </div>
  <div id="head3"><a href="javascript:" onClick="javascript:window.external.AddFavorite(window.location.href,document.title);">添加收藏</a>
  <br><a href="/help/setie.php">搜索引擎</a>
  <br><a href="/help/zanzhu.php">赞助本站</a></div>
  <div id="head2">
 <a href="http://www.guoxuemi.com/" target="_blank"><img src="http://www.guoxuedashi.com/img/guoxuemi.gif" width="95" height="62" border="0" style="margin-left:2px;" title="国学迷"></a>
  

  </div>
</div>
  <div class="clear"></div>
  <div class="head_nav">
  <p><a href="/">首页</a> | <a href="/ShuKu/">国学书库</a> | <a href="/guji/">影印古籍</a> | <a href="/shici/">诗词宝典</a> | <a   href="/SiKuQuanShu/gxjx.php">精选</a> <b>|</b> <a href="/zidian/">汉语字典</a> | <a href="/hydcd/">汉语词典</a> | <a href="http://www.guoxuedashi.com/zidian/bujian/"><font  color="#CC0066">部件查字</font></a> | <a href="http://www.sfds.cn/"><font  color="#CC0066">书法大师</font></a> | <a href="/jgwhj/">甲骨文</a> <b>|</b> <a href="/b/4/"><font  color="#CC0066">解密</font></a> | <a href="/renwu/">历史人物</a> | <a href="/diangu/">历史典故</a> | <a href="/xingshi/">姓氏</a> | <a href="/minzu/">民族</a> <b>|</b> <a href="/mz/"><font  color="#CC0066">世界名著</font></a> | <a href="/download/">软件下载</a>
</p>
<p><a href="/b/"><font  color="#CC0066">历史</font></a> | <a href="http://skqs.guoxuedashi.com/" target="_blank">四库全书</a> |  <a href="http://www.guoxuedashi.com/search/" target="_blank"><font  color="#CC0066">全文检索</font></a> | <a href="http://www.guoxuedashi.com/shumu/">古籍书目</a> | <a   href="/24shi/">正史</a> <b>|</b> <a href="/chengyu/">成语词典</a> | <a href="/kangxi/" title="康熙字典">康熙字典</a> | <a href="/ShuoWenJieZi/">说文解字</a> | <a href="/zixing/yanbian/">字形演变</a> | <a href="/yzjwjc/">金 文</a> <b>|</b>  <a href="/shijian/nian-hao/">年号</a> | <a href="/diming/">历史地名</a> | <a href="/shijian/">历史事件</a> | <a href="/guanzhi/">官职</a> | <a href="/lishi/">知识</a> <b>|</b> <a href="/zhongyi/">中医中药</a> | <a href="http://www.guoxuedashi.com/forum/">留言反馈</a>
</p>
  </div>
</div>
<!-- 头部导航END --> 
<!-- 内容区开始 --> 
<div class="w1180 clearfix">
  <div class="info l">
   
<div class="clearfix" style="background:#f5faff;">
<script src='http://www.guoxuedashi.com/img/headersou.js'></script>

</div>
  <div class="info_tree"><a href="http://www.guoxuedashi.com">首页</a> > <a href="/SiKuQuanShu/fanti/">四库全书</a>
 > <h1>资治通鉴</h1> <!--         下载:【右键另存为】即可 --></div>
  <div class="info_content zj clearfix">
  
<div class="info_txt clearfix" id="show">
<center style="font-size:24px;">15-資治通鑑卷十四</center>
    資治通鑑卷十四    宋 司馬光 撰<br />
<br />
  胡三省 音註<br />
<br />
  漢紀六【起閼逢困敦盡重光協洽凡八年】<br />
<br />
  太宗孝文皇帝中<br />
<br />
  前三年冬十月丁酉晦日有食之十一月丁卯晦日有食之 詔曰前遣列侯之國【事見上卷上年】或辭未行丞相朕之所重其為朕率列侯之國【為于偽翻】十二月免丞相勃遣就國乙亥以太尉灌嬰為丞相罷太尉官屬丞相【漢承秦制以丞相太尉御史大夫為三公今周勃自丞相罷就國灌嬰自太尉為丞相因罷太尉官盖三公不必備之意且兵柄難以輕屬也】 夏四月城陽景王章薨【諡法由義而濟曰景耆意大慮曰景布義行剛曰景】 初趙王敖獻美人於高祖得幸有娠【娠音身】及貫高事發【見十二卷高祖九年】美人亦坐繫河内美人母弟趙兼因辟陽侯審食其言呂后【食其音異基】呂后妬弗肯白美人已生子恚即自殺【恚於避翻】吏奉其子詣上上悔名之曰長令呂后母之而葬其母真定後封長為淮南王【見十二卷高祖十一年】淮南王蚤失母常附呂后故孝惠呂后時得無患而常心怨辟陽侯以為不彊爭之於呂后使其母恨而死也及帝即位淮南王自以最親【時高祖諸子惟帝及長在故自以為最親】驕蹇數不奉法【驕蹇謂不順也數所角翻】上常寛假之是歲入朝【朝直遥翻】從上入苑囿獵與上同車常謂上大兄王有材力能扛鼎【扛音江舉也】乃往見辟陽侯自袖鐵椎椎辟陽侯令從者魏敬剄之【從才用翻剄古頂翻】馳走闕下肉袒謝罪帝傷其志為親故赦弗治【為于偽翻治直之翻】當是時薄太后及太子諸大臣皆憚淮南王淮南王以此歸國益驕恣出入稱警蹕稱制擬於天子袁盎諫曰諸侯太驕必生患上不聽【為淮南王謀反廢張本】五月匈奴右賢王入居河南地【右賢王匈奴貴王也居西方直上郡以西接氐羌師古曰北地郡之北黄河之南即白羊王所居余謂其地在北河之南蒙恬所收衛青所奪皆是地也】侵盜上郡保塞蠻夷殺掠人民上幸甘泉【蔡邕曰天子車駕所至臣民以為僥倖故曰幸見令長三老官屬親臨軒作樂賜以酒食帛葛越巾佩帶之屬民爵有級數或賜田租之半故因謂之幸也師古曰甘泉宫在雲陽本秦林光宫括地志在雍州雲陽縣西北三十八里元和郡國志雲陽縣西北三十八里有車箱阪縈紆曲折財通單軌上阪即平原宏敞甘泉宫之地亦曰車盤嶺沈敏求長安志雲陽磨石嶺山有甘泉】遣丞相灌嬰發車騎八萬五千詣高奴擊右賢王發中尉材官屬衛將軍軍長安【此中尉所掌材官士也觀此益足以明二年罷 衛將軍軍衛將軍之官本不罷也】右賢王走出塞上自甘泉之高奴因幸太原見故羣臣皆賜之復晉陽中都民三歲租【班志晉陽中都二縣皆屬太原郡高帝十一年立帝為代王都晉陽如淳註曰文紀言都中都又帝復晉陽中都二歲似遷都於中都也括地志中都故城在汾州平遥縣西南十三里宋白曰漢文帝為代王都中都故介休縣東南中都城也史記諸侯年表高帝十年封子恒為代王都中都復方目翻】留游太原十餘日初大臣之誅諸呂也朱虚侯功尤大大臣許盡以趙地王朱虚侯盡以梁地王東牟侯【王于况翻下以義推】及帝立聞朱虚東牟之初欲立齊王【事見上卷呂后八年】故絀其功【絀敕律翻貶下也】及王諸子乃割齊二郡以王之興居自以失職奪功頗怏怏聞帝幸太原以為天子且自擊胡遂發兵反帝聞之罷丞相及行兵皆歸長安【行兵行擊匈奴之兵也】以棘蒲侯柴武為大將軍將四將軍十萬衆擊之祁侯繒賀為將軍軍滎陽【應劭曰棘蒲即常山平棘縣師古非之余據靳歙傳則棘蒲趙地也在安陽以東宋白曰棘蒲春秋時晉邑漢初為棘蒲後改為平棘盖亦本應說也班志祁縣屬太原郡晉大夫賈辛邑括地志并州祁縣城是也柴武繒賀皆高帝功臣姓譜柴姓高柴之後繒亦姓也以國為氏國語云申繒方強韋昭注繒出於姒姓】秋七月上自太原至長安詔濟北吏民兵未至先自定及以軍城邑降者皆赦之復官爵與王興居去來者赦之【師古曰雖始與興居共反今棄之去而來降者亦赦之貢父曰高帝詔曰與綰居去來歸者赦之今此文當云與王興居居去來者赦之盖脱一居字也余謂貢父說是濟子禮翻降戶江翻】八月濟北王興居兵敗自殺 初南陽張釋之為騎郎【秦置南陽郡漢因之郎屬郎中令掌守門戶出充車騎郎中有車騎戶三將主車曰車郎主騎曰騎郎主戶衛曰戶郎皆以中郎將主之騎奇寄翻】十年不得調【調徒釣翻選也】欲免歸袁盎知其賢而薦之為謁者僕射【班表謁者掌賓讃受事秩比六百石有僕射秩比千石應劭曰謁請也白也僕主也漢官儀曰僕射秦官也僕主也古者主武事每官必有主射者以督課之】釋之從行登虎圈上問上林尉諸禽獸簿【虎圈養虎之所在上林圈求遠翻班表有令有八丞十二尉武帝以後屬水衡都尉禽獸簿謂簿録禽獸之大數也】十餘問尉左右視盡不能對【盖帝問之而不能對故倉皇失措而左右視也師古曰視其屬官盡不能對非也】虎圈嗇夫從旁代尉對上所問禽獸簿甚悉欲以觀其能【師古曰能謂材也能本獸名形似羆足似鹿為物堅中而強力故人之有賢材者皆謂之能】口對響應無窮者【虎圈嗇夫掌虎圈之吏也悉詳盡也響應者如響應聲言其捷也】帝曰吏不當若是邪尉無賴【言其才無足恃賴也援神契曰蝟多賴故不使超揚賴才也孟子富歲子弟多賴朱子曰賴藉也】乃詔釋之拜嗇夫為上林令釋之久之前曰陛下以絳侯周勃何如人也上曰長者也【長知兩翻】又復問東陽侯張相如何如人也【班志東陽縣屬臨淮郡】上復曰長者【復扶又翻】釋之曰夫絳侯東陽侯稱為長者此兩人言事曾不能出口豈效此嗇夫喋喋利口捷給哉【晉灼曰喋音牒】且秦以任刀筆之吏【師古曰刀所以削書也古者用簡牒故吏皆以刀筆自隨也揚子曰刀不利筆不銛說文楚謂之聿吳謂之不律燕謂之弗秦謂之筆釋名筆述也述事而書之也】爭以亟疾苛察相高【亟居力翻急也】其敝徒文具而無實不聞其過陵遲至於土崩【師古曰陵丘陵也陵遲言如丘陵之逶遲稍卑下也又曰陵夷夷平也言其頹替若丘陵之漸平也】今陛下以嗇夫口辨而超遷之臣恐天下隨風而靡爭為口辯而無其實夫下之化上疾於景響舉錯不可不審也【錯七故翻後以義推】帝曰善乃不拜嗇夫上就車召釋之參乘【乘䋲證翻】徐行問釋之秦之敝具以質言【如淳曰質誠也】至宫上拜釋之為公車令頃之太子與梁王共車入朝不下司馬門於是釋之追止太子梁王無得入殿門遂劾不下公門不敬奏之【班表公車令屬衛尉漢官儀公車司馬令掌殿司馬門如淳曰宫衛令諸出入殿門公車司馬門者皆下不如令者罰金四兩程大昌曰通典衛尉公車令曰胡廣云諸門各陳屯夾道其旁設兵以示威武交節立戟以遮呵出入劾戶槩翻又戶得翻】薄太后聞之帝免冠謝教兒子不謹薄太后乃使使承詔赦太子梁王然後得入帝由是奇釋之拜為中大夫【中大夫掌論議屬郎中令其位在太中大夫之下諫大夫之上武帝太初元年更名中大夫曰光禄大夫秩比二千石太中大夫秩比千石如故至後漢志有光禄大夫太中大夫中散大夫諫議大夫胡廣曰光禄大夫本為中大夫武帝元狩五年置為光禄大夫諫大夫世祖中興以為諫議大夫又有太中中散大夫此四等於古皆為天子之下大夫視列國之上卿】頃之至中郎將從行至覇陵上謂羣臣曰嗟乎以北山石為椁用紵絮斮陳漆其間【師古曰美石出京師北山今宜州石是斮絮以漆著其間也紵竹呂翻康曰紵檾屬細者為絟麤者為紵陸璣草木疏曰紵以麻也科生數十莖宿根在地中至春自生不歲種也荆揚之間一歲三收今官園種之歲再刈刈便生剝之以鐵若竹挟之表厚皮自脱但得其裏韌如筋者謂之徽紵今南越紵布皆用此麻檾口穎翻斮側畧翻】豈可動哉左右皆曰善釋之曰使其中有可欲者雖錮南山猶有隙使其中無可欲者雖無石槨又何戚焉【錮音固冶銅鑄塞以為固也師古曰有可欲謂多藏金玉而厚葬之人皆欲發取之也是有間隙也無可欲謂不寘器備而薄葬人無欲攻掘取之者故無憂戚也】帝稱善是歲釋之為廷尉上行出中渭橋【張晏曰中渭橋在渭橋中路臣瓚曰中渭橋兩岸之中索隐曰張晏臣瓚之說皆非也案今渭橋有三所一所在城西北咸陽路曰西渭橋一所在城東北高陵路曰東渭橋其中渭橋在長安故城之北】有一人從橋下走乘輿馬驚【乘繩證翻】於是使騎捕之屬廷尉【屬之欲翻下同】釋之奏當此人犯蹕當罰金【崔浩曰奏當謂處其罪也索隐曰案百官志云廷尉掌平刑罰奏當一應郡國讞疑罪皆處當以報之也如淳曰蹕止行人乙令蹕先至而犯者罰金四兩】上怒曰此人親驚吾馬馬賴和柔令他馬固不敗傷我乎而廷尉乃當之罸金釋之曰法者天下公共也今法如是更重之是法不信於民也且方其時上使使誅之則已今已下廷尉【下遐嫁翻】廷尉天下之平也壹傾天下用法皆為之輕重民安所錯其手足【錯七故翻】唯陛下察之上良久曰廷尉當是也其後人有盜高廟坐前玉環得【得言捕得也坐徂卧翻】帝怒下廷尉治釋之按盜宗廟服御物者為奏當棄市上大怒曰人無道乃盜先帝器吾屬廷尉者欲致之族而君以法奏之【索隐曰謂依律而斷也屬之欲翻】非吾所以共承宗廟意也【共讀曰恭】釋之免冠頓首謝曰法如是足也且罪等然以逆順為差【如淳曰罪等俱應死也盜玉環不若長陵土之逆仲馮曰此等讀如等級之等言凡罪之等差】今盜宗廟器而族之有如萬分一假令愚民取長陵一抔土【長陵高祖陵也張晏曰不欲指言故以取土喻之也師古曰抔謂以手掬之也抔步侯翻】陛下且何以加其法乎帝乃白太后許之<br />
<br />
  四年冬十二月潁隂懿侯灌嬰薨 春正月甲午以御史大夫陽武張蒼為丞相【班志陽武縣屬河南郡】蒼好書博聞尤邃律歷【好呼到翻】 上召河東守季布【河東本韓魏之地秦置郡】欲以為御史大夫有言其勇使酒難近者【應劭曰使酒酗酒也師古曰言因酒霑洽而使氣也近謂附近天子而為大臣近其靳翻】至留邸一月見罷【師古曰既引見而罷令還郡也貢父曰見罷猶言見逐見棄耳非引見也】季布因進曰臣無功竊寵待罪河東陛下無故召臣此人必有以臣欺陛下者【師古曰謂妄言其賢故云欺也】今臣至無所受事罷去此人必有毁臣者夫陛下以一人之譽而召臣以一人之毁而去臣【譽音余去羌呂翻】臣恐天下有識聞之有以闚陛下之淺深也上默然慙良久曰河東吾股肱郡故特召君耳 上議以賈誼任公卿之位大臣多短之曰洛陽之人年少初學【少詩照翻】專欲擅權紛亂諸事於是天子後亦疏之不用其議以為長沙王太傅【長沙王吳差也漢制諸侯王國有太傅輔王疏與疎同】 絳侯周勃既就國每河東守尉行縣至絳【漢承秦制郡有守有尉守掌治其郡尉掌佐守典武職甲卒行縣循行屬縣也行下孟翻】勃自畏恐誅常被甲令家人持兵以見之【被皮義翻】其後人有上書告勃欲反下廷尉【上時掌翻下遐嫁翻】廷尉逮捕勃治之勃恐不知置辭【師古曰置立也辭對獄之辭】吏稍侵辱之勃以千金與獄吏吏乃書牘背示之曰【牘木簡也以書獄辭李奇曰牘吏所執簿韋昭曰牘版也索隱曰簿即牘也故魏志秦宓以簿擊頬則亦簡牘之類也】以公主為證公主者帝女也勃太子勝之尚之【韋昭曰尚奉也不敢言娶也】薄太后亦以為勃無反事帝朝太后太后以冒絮提帝曰【應劭曰冒絮陌頟絮也如淳曰太后恚怒遭得左右物提之也晉灼曰巴蜀異物志謂頭上巾為冒絮師古曰冒覆也老人所以覆其頭提擊之也提徒計翻索隱音抵擲也】絳侯始誅諸呂綰皇帝璽【綰烏版翻】將兵於北軍不以此時反今居一小縣顧欲反邪帝既見絳侯獄辭乃謝曰吏方驗而出之於是使使持節赦絳侯復爵邑絳侯既出曰吾嘗將百萬軍然安知獄吏之貴乎 作顧成廟【服䖍曰顧成廟在長安城南還顧見城故名之應劭曰帝自為廟制度卑狭若顧望而成猶文王靈臺不日成之故曰顧成也如淳曰身存而為廟若周之顧命也景帝廟號德陽武帝廟號龍淵昭帝廟號徘徊宣帝廟號樂遊元帝廟號長夀成帝廟號陽池師古曰以還顧見城於義無取义書本不作城郭字應說近之】<br />
<br />
  五年春二月地震 初秦用半兩錢【秦半兩錢重如其文】高祖嫌其重難用更鑄莢錢【更工衡翻下同如淳曰如榆莢也莢音頬杜佑曰莢錢如榆莢重一銖半徑五分文曰漢興即應劭所謂五分錢】於是物價騰踊米至石萬錢夏四月更造四銖錢【應劭曰文帝以五分錢太輕小更作四銖錢文亦曰半兩今民間半兩錢最輕小者是也】除盜鑄錢令使民得自鑄賈誼諫曰法使天下公得雇租鑄銅錫為錢【師古曰雇租謂雇傭之直或租其本】敢雜以鉛鐵為它巧者其罪黥然鑄錢之情非殽雜為巧則不可得贏【師古曰殽謂亂雜也不得贏謂無餘利也言不雜鉛鐵則無利也殽音爻】而殽之甚微為利甚厚【師古曰微謂精妙也言殽雜鉛鐵其術精妙不可覺知而得利甚厚故令人輕犯姦而不可止也余謂微細也言姦民殽雜鉛鐵其所費甚微而得利甚厚也】夫事有召禍而法有起姦今令細民人操造幣之埶【師古曰操持也人人皆得鑄錢也操千高翻】各隱屛而鑄作【屛必郢翻蔽也言各自隱蔽而鑄錢也】因欲禁其厚利微姦雖黥罪日報其埶不止【蘇林曰報論余據張湯傳有訊鞫論報嚴延年傳有報囚師古註皆以為論奏穫報原父註則謂報者為斷决囚若今有司書囚罪長吏判凖斷是也】乃者民人抵罪多者一縣百數及吏之所疑搒笞犇走者甚衆【搒音彭】夫縣法以誘民【縣讀曰懸師古曰懸謂開立之】使入陷阱孰多於此【師古曰阱穿地以陷獸也阱才性翻】又民用錢郡縣不同或用輕錢百加若干【應劭曰時錢重四銖法錢百文當重一斤十六銖輕則以錢足之若干枚令滿平也師古曰若干且設數之言也干猶箇也謂當如此箇數也而胡廣云若順也干求也當順所求而與之矣】或用重錢平稱不受【應劭曰用重錢則平稱有餘不能受也臣瓚曰秦錢重半兩漢初鑄莢錢文帝更鑄四銖錢秦錢與莢錢皆當廢而故與四銖並行民以其見廢故用輕錢則百加若干用重錢則雖一當一猶復不受是以郡縣不同也師古曰應說是也稱尺孕翻】法錢不立【師古曰依法之錢也】吏急而壹之乎則大為煩苛而力不能勝縱而弗呵乎則市肆異用錢文大亂苟非其術何鄉而可哉今農事棄捐而采銅者日蕃【勝音升鄉讀曰嚮蕃扶元翻】釋其耒耨冶鎔炊炭姦錢日多五穀不為多【言民棄其農而冶銅炊炭故五穀不為多為于偽翻】善人怵而為姦邪愿民陷而之刑戮刑戮將甚不詳奈何而忽【怵先律翻又音黜誘也言動心於為姦邪也愿謹也師古曰詳平也忽忽忘也】國知患此吏議必曰禁之禁之不得其術其傷必大令禁鑄錢則錢必重【師古曰令謂法令也】重則其利深盜鑄如雲而起棄市之罪又不足以禁矣姦數不勝而法禁數潰銅使之然也【數所角翻】銅布於天下其為禍博矣故不如收之賈山亦上書諫以為錢者亡用器也【亡與無通】而可以易富貴富貴者人主之操柄也令民為之是與人主共操柄不可長也上不聽【操千高翻長知兩翻】是時太中大夫鄧通方寵幸上欲其富賜之蜀嚴道銅山使鑄錢【班志嚴道屬蜀郡括地志雅州榮經縣北三里有銅山即鄧通得賜銅山鑄錢者也唐榮經即漢嚴道也】吳王濞有豫章銅山【豫章秦鄣郡地高帝分置豫章郡】招致天下亡命者以鑄錢東煮海水為鹽以故無賦而國用饒足【史言吳以彊富致叛】於是吳鄧錢布天下 初帝分代為二國【事見上卷二年】立皇子武為代王參為太原王是歲徙代王武為淮陽王以太原王參為代王盡得故地【故代國之地】<br />
<br />
  六年冬十月桃李華【華讀如花】淮南厲王長自作法令行於其國逐漢所置吏請自置相二千石【王國自相至内史中尉皆吏二千石漢為置之餘得自置今長驕橫逐漢所置吏而請自置之】帝曲意從之又擅刑殺不辜及爵人至關内侯【關内侯爵第十九爵自上出非侯王所擅】數上書不遜順【數所角翻】帝重自切責之【師古曰重難也】乃令薄昭與書風諭之引管蔡及代頃王濟北王興居以為儆戒【周公誅管叔蔡叔代頃王高祖兄仲也諡法甄心動懼曰頃敏以敬慎曰頃廢為侯事見十一卷高祖七年興居事見上三年風讀曰諷頃音傾】王不說【說讀曰悦】令大夫但士伍開章等七十人【開姓也姓譜衛公子開方之後】與棘蒲侯柴武太子奇謀以輦車四十乘反谷口【師古曰輦車古人輓行以載兵器也谷口在長安北處多險阻班志谷口縣屬左馮翊括地志谷口故城在雍州醴泉縣東北四十里乘繩證翻】令人使閩越匈奴事覺有司治之【使疏吏翻下以義推】使使召淮南王王至長安丞相張蒼典客馮敬行御史大夫事與宗正廷尉奏長罪當棄市制曰其赦長死罪廢勿王徙處蜀郡嚴道卭郵【卭郵置名師古曰郵行書之舍余據班志嚴道有卭來山卭水所出盖於其地置郵驛也杜佑曰卭州臨卭縣南有卭來山在雅州百丈縣嚴道今雅州宋白曰秦㓕楚徙嚴王之族以實此地故曰嚴道勿王于况翻處昌呂翻卭渠容翻郵音尤】盡誅所與謀者載長以輜車令縣以次傳之【傳直戀翻下同】袁盎諫曰上素驕淮南王弗為置嚴傳相【為于偽翻相息亮翻】以故致此淮南王為人剛今暴摧折之臣恐卒逢霧露病死【卒讀曰猝又音子恤翻終也】陛下有殺弟之名奈何上曰吾特苦之耳今復之【師古曰暫困苦之令其自悔即追還也】淮南王果憤恚不食死縣傳至雍【班志雍縣屬扶風雍於用翻】雍令發封以死聞【輜車有封前此所經縣傳莫敢發至雍令乃發之】上哭甚悲謂袁盎曰吾不聽公言卒亡淮南王【卒子恤翻】今為奈何盎曰獨斬丞相御史以謝天下乃可上即令丞相御史逮考諸縣傳送淮南王不發封餽侍者皆棄市以列侯葬淮南王於雍置守冢三十戶匈奴單于遺漢書曰前時皇帝言和親事稱書意合<br />
<br />
  歡【遺于季翻下同師古曰稱副也言與所遺書意相副而共結歡親稱尺證翻下同】漢邊吏侵侮右賢王右賢王不請聽後義盧侯難支等計【索隱曰難支匈奴將名也】與漢吏相距絶二主之約離兄弟之親故罰右賢王使之西求月氏擊之以天之福吏卒良馬力強以夷滅月氏盡斬殺降下定之【氏音支降戶江翻】樓蘭烏孫呼揭【樓蘭國在西域之東垂後曰鄯善自武帝開河西之後地最近漢當白龍堆之道烏孫國治赤谷城師古曰烏孫於西域諸戎其形最異今之胡人青眼赤須狀類獼猴是其種也史記正義呼揭國在瓜州西北余據班史匈奴北服丁零呼揭之國宣帝時匈奴乖亂其西方呼揭王自立為呼揭單于西域傳呼掲不在三十六國之數而烏孫國東與匈奴接則呼揭盖在烏孫之東匈奴西北也師古曰揭丘例翻索隱其列翻正義音犂】及其旁二十六國皆已為匈奴諸引弓之民【釋名曰弓穹也張之穹穹然也】并為一家北州以定願寢兵休士卒養馬除前事復故約以安邊民皇帝即不欲匈奴近塞則且詔吏民遠舍【近其靳翻】帝報書曰單于欲除前事復故約朕甚嘉之此古聖王之志也漢與匈奴約為兄弟所以遺單于甚厚倍約離兄弟之親者常在匈奴【倍蒲妹翻】然右賢王事已在赦前單于勿深誅單于若稱書意明吿諸吏使無負約有信敬如單于書後頃之冒頓死子稽粥立【稽音雞粥音育】號曰老上單于老上單于初立帝復遣宗室女翁主為單于閼氏【復扶又翻閼氏音煙支】使宦者燕人中行說傳翁主【中行姓說名中行本出荀氏晉荀林父將中行因以為氏行戶江翻說讀曰悦】說不欲行漢強使之【強其兩翻】說曰必我也為漢患者【言為漢患者必我也史倒其文因當時語】中行說既至因降單于單于甚親幸之初匈奴好漢繒絮食物【繒帛也絮綿也好呼到翻下同】中行說曰匈奴人衆不能當漢之一郡然所以強者以衣食異無仰於漢也今單于變俗好漢物漢物不過什二則匈奴盡歸於漢矣【師古曰言漢費物十分之二則匈奴之衆將盡歸於漢矣】其得漢繒絮以馳草棘中衣袴皆裂敝以示不如旃裘之完善也得漢食物皆去之【去丘呂翻棄也】以示不如湩酪之便美也【湩竹用翻又都奉翻乳汁也酪盧各翻以乳為之】於是說教單于左右疏記以計課其人衆畜牧其遺漢書牘及印封皆令長大倨傲其辭【遺于季翻】自稱天地所生日月所置匈奴大單于漢使或訾笑匈奴俗無禮義者【訾將此翻毁也】中行說輒窮漢使曰匈奴約束徑易行【易以豉翻】君臣簡可久一國之政猶一體也故匈奴雖亂必立宗種【種章勇翻】今中國雖云有禮義及親屬益疎則相殺奪以至易姓皆從此類也嗟土室之人【匈奴之人逐水草居廬帳非如中國有室屋故謂中國人為土室之人師古曰嗟者歎愍之言】顧無多辭喋喋佔佔【師古曰顧思念也喋喋利口也佔佔衣裳貌也言漢人且當思念無為喋喋佔佔佔冒占翻】顧漢所輸匈奴繒絮米糵令其量中必善美而已矣【師古曰顧念也中猶滿也量中者滿其數也中竹仲翻】何以言為乎且所給備善則已不備苦惡則候秋熟以騎馳蹂而稼穡耳【師古曰苦猶麤也蹂踐也而汝也韋昭曰苦音靡塩之塩蹂人九翻】梁太傅賈誼【誼自長沙徵為梁懷王太傅】上疏曰臣竊惟今之事勢可為痛哭者一可為流涕者二可為長太息者六若其它背理而傷道者難徧以疏舉【背蒲妹翻】進言者皆曰天下已安已治矣臣獨以為未也曰安且治者非愚則諛皆非事實知治亂之體者也【治直吏翻下同】夫抱火厝之積薪之下而寢其上【厝千故翻置也】火未及然因謂之安方今之勢何以異此陛下何不壹令臣得孰數之於前【孰古熟字通用】因陳治安之策試詳擇焉使為治勞志慮苦身體乏鐘鼓之樂勿為可也樂與今同而加之諸侯軌道【師古曰軌道言遵法制也樂音洛】兵革不動匈奴賓服百姓素朴生為明帝没為明神名譽之美垂於無窮使顧成之廟稱為太宗上配太祖與漢亡極【亡古無字通】立經陳紀為萬世法雖有愚幼不肖之嗣猶得蒙業而安以陛下之明達因使少知治體者得佐下風致此非難也夫樹國固必相疑之勢【鄭氏曰今建立國泰大其埶固必相疑也臣瓚曰樹國於險固諸侯彊大則必與天子有相疑之埶也師古曰鄭說是】下數被其殃上數爽其憂【如淳曰爽忒也數所角翻】甚非所以安上而全下也今或親弟謀為東帝親兄之子西鄉而擊【親弟謂淮南厲王長謀反親兄之子謂齊悼惠王子濟北王興居欲西擊滎陽鄉讀曰嚮】今吳又見告矣【如淳曰時吳王濞不循漢法有告之者】天子春秋鼎盛【應劭曰鼎方也】行義未過德澤有加焉【行下孟翻】猶尚如是况莫大諸侯權力且十此者乎【師古莫大謂無有大於其國者言最大也十此謂十倍於此余謂誼之大意盖謂淮南濟北當文帝之時尚敢以一國為變使諸侯相合襲是跡而動則其權力十倍於此為患莫大焉】然而天下少安何也大國之王幼弱未壯漢之所置傅相方握其事數年之後諸侯之王大抵皆冠【師古曰大抵猶言大略也冠古玩翻】血氣方剛漢之傅相稱病而賜罷彼自丞尉以上徧置私人如此有異淮南濟北之為邪此時而欲為治安雖堯舜不治黄帝曰日中必熭操刀必割【孟康曰熭音衛日中盛者必暴熭也臣瓚曰太公曰日中不熭是謂失時操刀不割失利之期言當及時也師古曰熭謂暴曬之也】今令此道順而全安甚易【易以豉翻】不肯蚤為已乃墮骨肉之屬而抗剄之【應劭曰抗其頭而剄之也師古曰墮毁也抗舉也剄割頸也墮許規翻剄工頂翻】豈有異秦之季世虖【虖古乎字】其異姓負彊而動者漢已幸而勝之矣又不易其所以然同姓襲是跡而動既有徵矣【徵證驗也】其勢盡又復然殃旤之變未知所移明帝處之尚不能以安【處昌呂翻】後世將如之何臣竊跡前事【師古曰尋前事之蹤跡】大抵彊者先反長沙乃二萬五千戶耳功少而最完勢疏而最忠【漢初功臣封王者獨長沙王吳芮傳國至文帝時】非獨性異人也亦形勢然也曩令樊酈絳灌據數十城而王今雖以殘亡可也令信越之倫列為徹侯而居雖至今存可也然則天下之大計可知已欲諸王之皆忠附則莫若令如長沙王欲臣子勿菹醢則莫若令如樊酈等【菹臻魚翻虀也醢呼改翻肉醬也】欲天下之治安莫若衆建諸侯而少其力力少則易使以義【師古曰使以義使之遵禮義也少詩沼翻】國小則亡邪心令海内之勢如身之使臂臂之使指莫不制從諸侯之君不敢有異心輻湊並進而歸命天子割地定制令齊趙楚各為若干國使悼惠王幽王元王之子孫畢以次各受祖之分地【分扶問翻】地盡而止其分地衆而子孫少者建以為國空而置之須其子孫生者舉使君之【須待也】一寸之地一人之衆天子亡所利焉【亡古無字通下同】誠以定治而已如此則卧赤子天下之上而安植遺腹朝委裘而天下不亂【服䖍曰言天下安雖赤子遺腹在位猶不危也應劭曰植遺腹朝委裘皆未有所知也孟康曰委裘若容衣天子未坐事先帝裘衣也植音值朝直遥翻】當時大治後世誦聖【師古曰稱其聖明】陛下誰憚而久不為此天下之勢方病大瘇【如淳曰腫足曰瘇師古曰瘇上勇翻】一脛之大幾如要【脛戶定翻脚脛釋名曰脛莖也直而長似物莖也幾居依翻下同】一指之大幾如股平居不可屈伸一二指慉身慮無聊【師古曰慉謂動而痛也聊賴也慉丑六翻】失今不治必為錮疾【師古曰錮疾堅久之疾】後雖有扁鵲不能為己【師古曰扁鵲良醫也為治也已語終辭】病非徒瘇也又苦?盭【師古曰?古蹠字之石翻足下曰蹠今所呼脚掌是也盭古戾字言足蹠反戾不可行也】元王之子帝之從弟也今之王者從弟之子也惠王之子親兄子也今之王者兄子之子也【楚元王交高帝之弟其子於文帝為從弟齊悼惠王肥高帝之庶長子其子於文帝為親兄子從才用翻】親者或亡分地以安天下疏者或制大權以偪天子【師古曰廣立藩屏則天下安故曰以安天下偪古逼字】臣故曰非徒病瘇也又苦?盭可痛哭者此病是也天下之勢方倒縣【縣古懸字通下同】凡天子者天下之首何也上也蠻夷者天下之足何也下也今匈奴嫚侮侵掠至不敬也而漢歲致金絮采繒以奉之足反居上首顧居下【師古曰顧亦反也】倒懸如此莫之能解猶為國有人乎【師古曰顛倒如此而不能解救豈謂國有明智之人乎】可為流涕者此也今不獵猛敵而獵田彘不搏反寇而搏畜菟翫細娛而不圖大患德可遠加而直數百里外威令不勝可為流涕者此也今庶人屋壁得為帝服倡優下賤得為后飾且帝之身自衣皁綈【綈徒奚翻厚繒也衣於既翻下能衣同】而富民牆屋被文繡【被皮義翻】天子之后以緣其領庶人孽妾以緣其履【師古曰緣熒絹翻孽庶賤者】此臣所謂舛也夫百人作之不能衣一人欲天下亡寒胡可得也一人耕之十人聚而食之欲天下亡飢不可得也飢寒切於民之肌膚欲其亡為姦邪不可得也可為長太息者此也商君遺禮義棄仁恩并心於進取行之二歲秦俗日敗故秦人家富子壯則出分【分扶問翻】家貧子壯則出贅借父耰鉏慮有德色【師古曰耰摩田器言以耰及鉏借與其父而容色自矜以為恩德也耰音憂】母取箕箒立而誶語【服䖍曰誶猶罵也張晏曰誶語讓也誶音碎】抱哺其子與公併倨【師古曰哺飤也言婦抱其子而哺之乃與其舅併倨無禮之甚也哺音步併步鼎翻】婦姑不相說【說讀曰悦】則反脣而相稽【應劭曰稽計也相與計校也稽工奚翻】其慈子耆利不同禽獸者亡幾耳【師古曰惟有慈愛其子而貪耆財利不異於禽獸也無幾言不多也幾居豈翻仲馮曰誼謂秦人不知孝義但知愛子貪利而已此其去禽獸無幾也耆古嗜字通用】今其遺風餘俗猶尚未改棄禮義捐亷恥日甚可謂月異而歲不同矣逐利不耳慮非顧行也【師古曰言其所追赴惟計利與不耳念慮之中非顧所行之善惡貢父曰慮大率也不讀曰否】今其甚者殺父兄矣而大臣特以簿書不報期會之間以為大故至於俗流失世壞敗因恬而不知怪【師古曰恬安也徒兼翻】慮不動於耳目以為是適然耳【師古曰適當也謂事理當然】夫移風易俗使天下囘心而鄉道【鄉讀曰嚮】類非俗吏之所能為也俗吏之所務在於刀筆筐箧【師古曰刀所以削書札筐箧所以盛書也箧音古頰翻】而不知大體陛下又不自憂竊為陛下惜之【為于偽翻】豈如今定經制令君君臣臣上下有差父子六親各得其宜【賢曰六親謂父子兄弟夫婦也】此業壹定世世常安而後有所持循矣【師古曰執持而順行之】若夫經制不定是猶度江河亡維楫【師古曰維所以繋船楫所以刺船也詩曰紼縭維之楫音集又音接】中流而遇風波船必覆矣可為長太息者此也夏殷周為天子皆數十世秦為天子二世而亡人性不甚相遠也【遠于萬翻】何三代之君有道之長而秦無道之暴也其故可知也古之王者太子乃生【師古曰乃始也】固舉以禮有司齊肅端冕見之南郊【齊讀曰齋見戶電翻】過闕則下過廟則趨故自為赤子【仲馮曰嬰兒體色赤故曰赤子】而教固已行矣孩提有識【師古曰孩小兒也提謂提撕之】三公三少明孝仁禮義以道習之【太師太傅太保為三公少師少傅少保為三少少詩照翻】逐去邪人不使見惡行【去羌呂翻行下孟翻】於是皆選天下之端士孝弟博聞有道術者以衛翼之使與太子居處出入【處昌呂翻】故太子乃生而見正事聞正言行正道左右前後皆正人也夫習與正人居之不能毋正猶生長於齊不能不齊言也習與不正人居之不能毋不正猶生長於楚之地不能不楚言也孔子曰少成若天性習貫如自然【師古曰貫亦習也工宦翻下積貫同】習與智長故切而不媿【師古曰每被切磋故無大過可愧恥之事長知兩翻】化與心成故中道若性夫三代之所以長久者以其輔翼太子有此具也及秦而不然使趙高傳胡亥而教之獄所習者非斬劓人則夷人之三族也【劓魚器翻割鼻也】胡亥今日即位而明日射人【射而亦翻】忠諫者謂之誹謗深計者謂之妖言其視殺人若艾草菅然【艾與刈同師古曰菅茅也音姦】豈惟胡亥之性惡哉彼其所以道之者非其理故也【道讀曰導】鄙諺曰前車覆後車誡秦世之所以亟絶者其轍跡可見也【師古曰亟急也車跡曰轍】然而不避是後車又將覆也天下之命縣於太子【縣讀曰懸】太子之善在於早諭教與選左右【師古曰諭曉告也與猶及也】夫心未濫而先諭教則化易成也【易以豉翻】開於道術智誼之指則教之力也若其服習積貫則左右而已夫胡粤之人生而同聲嗜欲不異及其長而成俗累數譯而不能相通【譯傳言也夷狄與中國言語不同故欲通夷狄之言者譯之周禮象胥是也長知兩翻】有雖死而不相為者【蘇林曰言其人不能易事相為處】則教習然也臣故曰選左右早諭教最急夫教得而左右正則太子正矣太子正而天下定矣書曰一人有慶兆民賴之【師古曰周書呂刑之辭也一人天子也言天子有善則兆庶獲其利】此時務也凡人之智能見已然不能見將然夫禮者禁於將然之前【師古曰將然謂欲有其事】而灋者禁於已然之後是故灋之所為用易見而禮之所為生難知也【易以豉翻】若夫慶賞以勸善刑罰以懲惡先王執此之政堅如金石行此之令信如四時據此之公無私如天地豈顧不用哉然而曰禮云禮云者貴絶惡於未萌而起教於微眇【師古曰眇細小也】使民日遷善遠辠而不自知也【遠于願翻】孔子曰聽訟吾猶人也必也使毋訟乎【師古曰論語載孔子之言也言使吾聽訟與衆人齊等然能先以德義化之使無訟】為人主計者莫如先審取舍【師古曰取所擇用也舍所棄置也舍讀曰捨下同】取舍之極定於内而安危之萌應於外矣【師古曰極中也萌始生也】秦王之欲尊宗廟而安子孫與湯武同然而湯武廣大其德行六七百歲而弗失秦王治天下十餘歲則大敗【治直之翻】此亡他故矣【亡古無字通下同】湯武之定取舍審而秦王之定取舍不審矣夫天下大器也今人之置器置諸安處則安置諸危處則危天下之情與器無以異在天子之所置之湯武置天下於仁義禮樂累子孫數十世此天下所共聞也秦王置天下於灋令刑罰旤幾及身【幾居依翻】子孫誅絶此天下之所共見也是非其明效大驗邪人之言曰聽言之道必以其事觀之則言者莫敢妄言今或言禮誼之不如灋令教化之不如刑罰人主胡不引殷周秦事以觀之也【胡何也】人主之尊譬如堂羣臣如陛衆庶如地故陛九級上廉遠地則堂高【遠于願翻下同】陛無級廉近地則堂卑高者難攀卑者易陵【師古曰級等也廉側隅也陵乘也】理勢然也故古者聖王制為等列内有公卿大夫士外有公侯伯子男然後有官師小吏【師古曰官師一官之長】延及庶人等級分明而天子加焉故其尊不可及也里諺曰欲投鼠而忌器此善諭也鼠近於器尚憚不投恐傷其器况於貴臣之近主乎【近其靳翻】廉恥節禮以治君子故有賜死而無戮辱是以黥劓之辠不及大夫【杜佑曰刑不上大夫者古之大夫有坐不廉汙穢者則曰簠簋不飭淫亂男女無别者則曰帷薄不脩罔上不忠者則曰臣節未著罷軟不勝任者則曰下官不職干國之紀則曰行事不請此五者大夫定罪之名矣不忍斥然正以呼之其在五刑之域者云云如後誼所云】以其離主上不遠也【離力智翻】禮不敢齒君之路馬蹵其芻者有罰【齒謂審其齒歲也蹵蹋也芻馬所食草記曲禮以足蹵路馬芻有誅齒路馬有誅蹵千六翻】所以為主上豫遠不敬也今自王侯三公之貴皆天子之所改容而禮之也古天子之所謂伯父伯舅也【師古曰天子呼諸侯長者同姓則曰伯父異姓則曰伯舅伯長也】而令與衆庶同黥劓髠刖笞傌棄市之灋【髠苦昆翻刖音月斷足也笞丑之翻傌音罵毛晃曰戮辱也】然則堂不無陛虖被戮辱者不泰迫虖【師古曰迫天子也】廉恥不行大臣無乃握重權大官而有徒隸無恥之心虖夫望夷之事二世見當以重灋者【如淳曰決罪曰當閻樂殺二世於望夷宫本由秦制無忌上之風也仲馮曰趙高殺二世盖又以法定其罪】投鼠而不忌器之習也臣聞之履雖鮮不加於枕冠雖敝不以苴履【師古曰苴者履中之藉苴子余翻】夫嘗已在貴寵之位天子改容而禮貌之矣【師古曰禮貌謂加禮容而敬之也】吏民嘗俯伏以敬畏之矣今而有過帝令廢之可也退之可也賜之死可也滅之可也若夫束縳之係緤之【師古曰緤謂以長繩係之也緤先列翻】輸之司寇編之徒官【師古曰司寇主刑罰之官編次列也徒官謂刑徒輸作于官者】司寇小吏詈罵而榜笞之【榜音彭】殆非所以令衆庶見也夫卑賤者習知尊貴者之一旦吾亦乃可以加此也【蘇林曰知有一旦之刑】非所以尊尊貴貴之化也古者大臣有坐不廉而廢者不謂不廉曰簠簋不飭【師古曰簠簋所以盛飯也方曰簠圓曰簋埤雅曰龜有靈德伏匿而噎善潛而不志於養故古者簠簋皆為龜形於其上而大臣以貪墨坐廢者曰簠簋不飭賈公彦曰簠内圓外方簋内方外圓皆受斗二升簠音甫又音扶簋音軌】坐汙穢淫亂男女無别者不曰汙穢曰帷薄不修坐罷軟不勝任者不謂罷軟曰下官不職【師古曰罷廢於事也軟弱也罷讀曰疲軟人兖翻勝音升】故貴大臣定有其辠矣猶未斥然正以呼之也尚遷就而為之諱也故其在大譴大何之域者【師古曰譴責也何問也域界局也】聞譴何則白冠氂纓【鄭氏曰以毛作纓白冠喪服也】盤水加劒造請室而請辠耳【應劭曰請室請罪之室蘇林曰音潔清之清胡公漢官車駕出有清室令在前先驅此官有别獄也如淳曰水性平若已有正罪君以平法治之也加劒當以自刎也或曰殺牲以盤水取頸血故示若此也造七到翻】上不執縛係引而行也其有中辠者聞命而自弛上不使人頸盭而加也【師古曰中辠非大非小也弛廢也自廢而死蘇林曰不戾其頸而親加刀鋸弛式爾翻盭古戾字音盧計翻】其有大罪者聞命則北面再拜跪而自裁【師古曰裁謂自刑殺也】上不使人捽抑而刑之也【師古曰捽持頭髮也抑按也捽才兀翻】曰子大夫自有過耳吾遇子有禮矣【服䖍曰子者男子美號】遇之有禮故羣臣自憙【師古曰憙讀曰喜許吏翻喜好也好為志氣也】嬰以廉恥故人矜節行【師古曰嬰加也矜尚也行下孟翻下同】上設廉恥禮義以遇其臣而臣不以節行報其上者則非人類也故化成俗定則為人臣者皆顧行而忘利守節而伏義故可以託不御之權可以寄六尺之孤【言臣下矜尚節行故可託以權柄不須復加制御應劭曰六尺之孤未能自立者也】此厲廉恥行禮義之所致也主上何喪焉【師古曰喪失也言如此則於主上無所失喪息浪翻】此之不為而顧彼之久行【此謂以禮義廉恥遇其臣彼謂戮辱貴臣言不為此而反久行彼也】故曰可為長太息者此也誼以絳侯前逮繫獄卒無事【卒子恤翻】故以此譏上上深納其言養臣下有節是後大臣有罪皆自殺不受刑【漢人相傳以大臣不對理陳寃為故事多有聞命而引決者然詣獄受刑者亦多有之史特大槩言之耳】<br />
<br />
  七年冬十月令列侯太夫人【如淳曰列侯之妻稱夫人列侯死子復為列侯乃得稱太夫人子不為列侯不得稱也】夫人諸侯王子及吏二千石無得擅徵捕 夏四月赦天下六月癸酉未央宫東闕罘罳災【如淳曰東闕與其兩旁罘罳皆災也晉灼曰東闕之罘罳獨災也師古曰罘罳謂連闕曲閣也以覆重刻垣墉處其形罘罳然一曰屏也崔豹古今注曰罘罳屏也又云罘者復也罳者思也臣朝君至屏外復思所奏之事於其下孔穎達曰屏謂之樹今浮思也釋宫文漢時謂屛為浮思解者以為天子外屏人臣至屏俯伏思念其事按匠人城隅謂角浮思也漢時東闕浮思災以此諸文參之則浮思小樓也故城隅闕上皆有之然則屛上亦為屋以覆屛墻故稱屛曰浮思蘇鶚演義曰罘者浮也罳者思也謂織絲之文輕疎虚浮之貌盖宫殿門闕有此物也余謂蘇鶚之說有見於唐禁中之罘罳唐太和甘露之變宦者奉乘輿决罘罳北出者也此罘罳當以舊註為正】 民有歌淮南王者曰一尺布尚可縫一斗粟尚可舂兄弟二人不相容帝聞而病之【臣瓚曰一尺布可縫而共衣一斗粟可舂而共食况以天下之廣而兄弟不相容乎】<br />
<br />
  八年夏封淮南厲王子安等四人為列侯【淮南厲王長子安封阜陵侯勃封安陽侯賜封陽周侯良封東城侯】賈誼知上必將復王之也上疏諫曰淮南王之悖逆無道【悖蒲内翻】天下孰不知其罪陛下幸而赦遷之自疾而死天下孰以王死之不當【當丁浪翻】今奉尊罪人之子適足以負謗於天下耳【師古曰言若尊王其子則是淮南王無罪漢枉殺之也】此人少壯【師古曰少壯猶言稍長大少詩沼翻】豈能忘其父哉白公勝所為父報仇者大父與叔父也【為于偽翻】白公為亂非欲取國代主發忿快志剡手以衝仇人之匈固為俱靡而已【白公勝楚平王之孫太子建之子建得罪於平王出奔而死於鄭勝又奔吳子胥以吳師入郢勝盖預焉是讎其大父也及其還楚殺子西子期是讎其叔父也剡式冉翻利也靡武彼翻師古曰言與仇人俱斃康曰武皮切碎也】淮南雖小黥布嘗用之矣【事見十二卷高祖十一年】漢存特幸耳【師古曰言漢之勝布得存此直天幸耳】夫擅仇人足以危漢之資於策不便【師古曰言假四子以資權則當危漢】予之衆積之財【予讀曰與】此非有子胥白公報於廣都之中即疑有剸諸荆軻起於兩柱之間【剸諸吳人為闔閭刺殺王僚荆軻事見七卷始皇二十年兩柱之間南面鄉明人君聽政正坐之處剸音專】所謂假賊兵為虎翼者也【應劭曰周書云無為虎傳翼將飛入邑擇人而食之】願陛下少留計【少詩沼翻】上弗聽 有長星出于東方【文穎曰孛彗長三星其占略同然其形象少異孛星光芒短其光四出蓬蓬孛孛也彗星光芒參參如掃彗長星有一直指或竟天或三丈二丈無常也大法彗孛星多為除舊布新長星多為兵革事】<br />
<br />
  九年春大旱<br />
<br />
  十年冬上行幸甘泉 將軍薄昭殺漢使者帝不忍加誅使公卿從之飲酒欲令自引分【引分猶言引決也】昭不肯使羣臣喪服往哭之乃自殺<br />
<br />
  臣光曰李德裕以為漢文帝誅薄昭斷則明矣【斷丁亂翻下同】於義則未安也秦康送晉文興如存之感【詩小序曰秦康公之母晉獻公之女文公遭驪姬之難未反而秦姬卒穆公納文公康公時為太子贈送文公于渭之陽念母之不見也我見舅氏如母存焉】况太后尚存唯一弟薄昭斷之不疑非所以慰母氏之心也臣愚以為法者天下之公器惟善持法者親疎如一無所不行則人莫敢有所恃而犯之也夫薄昭雖素稱長者文帝不為置賢師傅而用之典兵驕而犯上至於殺漢使者非有恃而然乎若又從而赦之則與成哀之世何異哉魏文帝嘗稱漢文帝之美而不取其殺薄昭曰舅后之家但當養育以恩而不當假借以權既觸罪法又不得不害譏文帝之始不防閑昭也斯言得之矣然則欲慰母心者將慎之於始乎<br />
<br />
  資治通鑑卷十四  <br>
   </div> 

<script src="/search/ajaxskft.js"> </script>
 <div class="clear"></div>
<br>
<br>
 <!-- a.d-->

 <!--
<div class="info_share">
</div> 
-->
 <!--info_share--></div>   <!-- end info_content-->
  </div> <!-- end l-->

<div class="r">   <!--r-->



<div class="sidebar"  style="margin-bottom:2px;">

 
<div class="sidebar_title">工具类大全</div>
<div class="sidebar_info">
<strong><a href="http://www.guoxuedashi.com/lsditu/" target="_blank">历史地图</a></strong>  
<a href="http://www.880114.com/" target="_blank">英语宝典</a>  
<a href="http://www.guoxuedashi.com/13jing/" target="_blank">十三经检索</a> 
<br><strong><a href="http://www.guoxuedashi.com/gjtsjc/" target="_blank">古今图书集成</a></strong> 
<a href="http://www.guoxuedashi.com/duilian/" target="_blank">对联大全</a> <strong><a href="http://www.guoxuedashi.com/xiangxingzi/" target="_blank">象形文字典</a></strong> 

<br><a href="http://www.guoxuedashi.com/zixing/yanbian/">字形演变</a>  <strong><a href="http://www.guoxuemi.com/hafo/" target="_blank">哈佛燕京中文善本特藏</a></strong>
<br><strong><a href="http://www.guoxuedashi.com/csfz/" target="_blank">丛书&方志检索器</a></strong> <a href="http://www.guoxuedashi.com/yqjyy/" target="_blank">一切经音义</a>  

<br><strong><a href="http://www.guoxuedashi.com/jiapu/" target="_blank">家谱族谱查询</a></strong>  <strong><a href="http://shufa.guoxuedashi.com/sfzitie/" target="_blank">书法字帖欣赏</a></strong> 
<br>

</div>
</div>


<div class="sidebar" style="margin-bottom:0px;">

<font style="font-size:22px;line-height:32px">QQ交流群9:489193090</font>


<div class="sidebar_title">手机APP 扫描或点击</div>
<div class="sidebar_info">
<table>
<tr>
	<td width=160><a href="http://m.guoxuedashi.com/app/" target="_blank"><img src="/img/gxds-sj.png" width="140"  border="0" alt="国学大师手机版"></a></td>
	<td>
<a href="http://www.guoxuedashi.com/download/" target="_blank">app软件下载专区</a><br>
<a href="http://www.guoxuedashi.com/download/gxds.php" target="_blank">《国学大师》下载</a><br>
<a href="http://www.guoxuedashi.com/download/kxzd.php" target="_blank">《汉字宝典》下载</a><br>
<a href="http://www.guoxuedashi.com/download/scqbd.php" target="_blank">《诗词曲宝典》下载</a><br>
<a href="http://www.guoxuedashi.com/SiKuQuanShu/skqs.php" target="_blank">《四库全书》下载</a><br>
</td>
</tr>
</table>

</div>
</div>


<div class="sidebar2">
<center>


</center>
</div>

<div class="sidebar"  style="margin-bottom:2px;">
<div class="sidebar_title">网站使用教程</div>
<div class="sidebar_info">
<a href="http://www.guoxuedashi.com/help/gjsearch.php" target="_blank">如何在国学大师网下载古籍?</a><br>
<a href="http://www.guoxuedashi.com/zidian/bujian/bjjc.php" target="_blank">如何使用部件查字法快速查字?</a><br>
<a href="http://www.guoxuedashi.com/search/sjc.php" target="_blank">如何在指定的书籍中全文检索?</a><br>
<a href="http://www.guoxuedashi.com/search/skjc.php" target="_blank">如何找到一句话在《四库全书》哪一页?</a><br>
</div>
</div>


<div class="sidebar">
<div class="sidebar_title">热门书籍</div>
<div class="sidebar_info">
<a href="/so.php?sokey=%E8%B5%84%E6%B2%BB%E9%80%9A%E9%89%B4&kt=1">资治通鉴</a> <a href="/24shi/"><strong>二十四史</strong></a>&nbsp; <a href="/a2694/">野史</a>&nbsp; <a href="/SiKuQuanShu/"><strong>四库全书</strong></a>&nbsp;<a href="http://www.guoxuedashi.com/SiKuQuanShu/fanti/">繁体</a>
<br><a href="/so.php?sokey=%E7%BA%A2%E6%A5%BC%E6%A2%A6&kt=1">红楼梦</a> <a href="/a/1858x/">三国演义</a> <a href="/a/1038k/">水浒传</a> <a href="/a/1046t/">西游记</a> <a href="/a/1914o/">封神演义</a>
<br>
<a href="http://www.guoxuedashi.com/so.php?sokeygx=%E4%B8%87%E6%9C%89%E6%96%87%E5%BA%93&submit=&kt=1">万有文库</a> <a href="/a/780t/">古文观止</a> <a href="/a/1024l/">文心雕龙</a> <a href="/a/1704n/">全唐诗</a> <a href="/a/1705h/">全宋词</a>
<br><a href="http://www.guoxuedashi.com/so.php?sokeygx=%E7%99%BE%E8%A1%B2%E6%9C%AC%E4%BA%8C%E5%8D%81%E5%9B%9B%E5%8F%B2&submit=&kt=1"><strong>百衲本二十四史</strong></a>  <a href="http://www.guoxuedashi.com/so.php?sokeygx=%E5%8F%A4%E4%BB%8A%E5%9B%BE%E4%B9%A6%E9%9B%86%E6%88%90&submit=&kt=1"><strong>古今图书集成</strong></a>
<br>

<a href="http://www.guoxuedashi.com/so.php?sokeygx=%E4%B8%9B%E4%B9%A6%E9%9B%86%E6%88%90&submit=&kt=1">丛书集成</a> 
<a href="http://www.guoxuedashi.com/so.php?sokeygx=%E5%9B%9B%E9%83%A8%E4%B8%9B%E5%88%8A&submit=&kt=1"><strong>四部丛刊</strong></a>  
<a href="http://www.guoxuedashi.com/so.php?sokeygx=%E8%AF%B4%E6%96%87%E8%A7%A3%E5%AD%97&submit=&kt=1">說文解字</a> <a href="http://www.guoxuedashi.com/so.php?sokeygx=%E5%85%A8%E4%B8%8A%E5%8F%A4&submit=&kt=1">三国六朝文</a>
<br><a href="http://www.guoxuedashi.com/so.php?sokeytm=%E6%97%A5%E6%9C%AC%E5%86%85%E9%98%81%E6%96%87%E5%BA%93&submit=&kt=1"><strong>日本内阁文库</strong></a> <a href="http://www.guoxuedashi.com/so.php?sokeytm=%E5%9B%BD%E5%9B%BE%E6%96%B9%E5%BF%97%E5%90%88%E9%9B%86&ka=100&submit=">国图方志合集</a> <a href="http://www.guoxuedashi.com/so.php?sokeytm=%E5%90%84%E5%9C%B0%E6%96%B9%E5%BF%97&submit=&kt=1"><strong>各地方志</strong></a>

</div>
</div>


<div class="sidebar2">
<center>

</center>
</div>
<div class="sidebar greenbar">
<div class="sidebar_title green">四库全书</div>
<div class="sidebar_info">

《四库全书》是中国古代最大的丛书,编撰于乾隆年间,由纪昀等360多位高官、学者编撰,3800多人抄写,费时十三年编成。丛书分经、史、子、集四部,故名四库。共有3500多种书,7.9万卷,3.6万册,约8亿字,基本上囊括了古代所有图书,故称“全书”。<a href="http://www.guoxuedashi.com/SiKuQuanShu/">详细>>
</a>

</div> 
</div>

</div>  <!--end r-->

</div>
<!-- 内容区END --> 

<!-- 页脚开始 -->
<div class="shh">

</div>

<div class="w1180" style="margin-top:8px;">
<center><script src="http://www.guoxuedashi.com/img/plus.php?id=3"></script></center>
</div>
<div class="w1180 foot">
<a href="/b/thanks.php">特别致谢</a> | <a href="javascript:window.external.AddFavorite(document.location.href,document.title);">收藏本站</a> | <a href="#">欢迎投稿</a> | <a href="http://www.guoxuedashi.com/forum/">意见建议</a> | <a href="http://www.guoxuemi.com/">国学迷</a> | <a href="http://www.shuowen.net/">说文网</a><script language="javascript" type="text/javascript" src="https://js.users.51.la/17753172.js"></script><br />
  Copyright &copy; 国学大师 古典图书集成 All Rights Reserved.<br>
  
  <span style="font-size:14px">免责声明:本站非营利性站点,以方便网友为主,仅供学习研究。<br>内容由热心网友提供和网上收集,不保留版权。若侵犯了您的权益,来信即刪。scp168@qq.com</span>
  <br />
ICP证:<a href="http://www.beian.miit.gov.cn/" target="_blank">鲁ICP备19060063号</a></div>
<!-- 页脚END --> 
<script src="http://www.guoxuedashi.com/img/plus.php?id=22"></script>
<script src="http://www.guoxuedashi.com/img/tongji.js"></script>

</body>
</html>
