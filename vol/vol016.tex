<!DOCTYPE html PUBLIC "-//W3C//DTD XHTML 1.0 Transitional//EN" "http://www.w3.org/TR/xhtml1/DTD/xhtml1-transitional.dtd">
<html xmlns="http://www.w3.org/1999/xhtml">
<head>
<meta http-equiv="Content-Type" content="text/html; charset=utf-8" />
<meta http-equiv="X-UA-Compatible" content="IE=Edge,chrome=1">
<title>資治通鑒_17-資治通鑑卷十六_17-資治通鑑卷十六</title>
<meta name="Keywords" content="資治通鑒_17-資治通鑑卷十六_17-資治通鑑卷十六">
<meta name="Description" content="資治通鑒_17-資治通鑑卷十六_17-資治通鑑卷十六">
<meta http-equiv="Cache-Control" content="no-transform" />
<meta http-equiv="Cache-Control" content="no-siteapp" />
<link href="/img/style.css" rel="stylesheet" type="text/css" />
<script src="/img/m.js?2020"></script> 
</head>
<body>
 <div class="ClassNavi">
<a  href="/24shi/">二十四史</a> | <a href="/SiKuQuanShu/">四库全书</a> | <a href="http://www.guoxuedashi.com/gjtsjc/"><font  color="#FF0000">古今图书集成</font></a> | <a href="/renwu/">历史人物</a> | <a href="/ShuoWenJieZi/"><font  color="#FF0000">说文解字</a></font> | <a href="/chengyu/">成语词典</a> | <a  target="_blank"  href="http://www.guoxuedashi.com/jgwhj/"><font  color="#FF0000">甲骨文合集</font></a> | <a href="/yzjwjc/"><font  color="#FF0000">殷周金文集成</font></a> | <a href="/xiangxingzi/"><font color="#0000FF">象形字典</font></a> | <a href="/13jing/"><font  color="#FF0000">十三经索引</font></a> | <a href="/zixing/"><font  color="#FF0000">字体转换器</font></a> | <a href="/zidian/xz/"><font color="#0000FF">篆书识别</font></a> | <a href="/jinfanyi/">近义反义词</a> | <a href="/duilian/">对联大全</a> | <a href="/jiapu/"><font  color="#0000FF">家谱族谱查询</font></a> | <a href="http://www.guoxuemi.com/hafo/" target="_blank" ><font color="#FF0000">哈佛古籍</font></a> 
</div>

 <!-- 头部导航开始 -->
<div class="w1180 head clearfix">
  <div class="head_logo l"><a title="国学大师官网" href="http://www.guoxuedashi.com" target="_blank"></a></div>
  <div class="head_sr l">
  <div id="head1">
  
  <a href="http://www.guoxuedashi.com/zidian/bujian/" target="_blank" ><img src="http://www.guoxuedashi.com/img/top1.gif" width="88" height="60" border="0" title="部件查字,支持20万汉字"></a>


<a href="http://www.guoxuedashi.com/help/yingpan.php" target="_blank"><img src="http://www.guoxuedashi.com/img/top230.gif" width="600" height="62" border="0" ></a>


  </div>
  <div id="head3"><a href="javascript:" onClick="javascript:window.external.AddFavorite(window.location.href,document.title);">添加收藏</a>
  <br><a href="/help/setie.php">搜索引擎</a>
  <br><a href="/help/zanzhu.php">赞助本站</a></div>
  <div id="head2">
 <a href="http://www.guoxuemi.com/" target="_blank"><img src="http://www.guoxuedashi.com/img/guoxuemi.gif" width="95" height="62" border="0" style="margin-left:2px;" title="国学迷"></a>
  

  </div>
</div>
  <div class="clear"></div>
  <div class="head_nav">
  <p><a href="/">首页</a> | <a href="/ShuKu/">国学书库</a> | <a href="/guji/">影印古籍</a> | <a href="/shici/">诗词宝典</a> | <a   href="/SiKuQuanShu/gxjx.php">精选</a> <b>|</b> <a href="/zidian/">汉语字典</a> | <a href="/hydcd/">汉语词典</a> | <a href="http://www.guoxuedashi.com/zidian/bujian/"><font  color="#CC0066">部件查字</font></a> | <a href="http://www.sfds.cn/"><font  color="#CC0066">书法大师</font></a> | <a href="/jgwhj/">甲骨文</a> <b>|</b> <a href="/b/4/"><font  color="#CC0066">解密</font></a> | <a href="/renwu/">历史人物</a> | <a href="/diangu/">历史典故</a> | <a href="/xingshi/">姓氏</a> | <a href="/minzu/">民族</a> <b>|</b> <a href="/mz/"><font  color="#CC0066">世界名著</font></a> | <a href="/download/">软件下载</a>
</p>
<p><a href="/b/"><font  color="#CC0066">历史</font></a> | <a href="http://skqs.guoxuedashi.com/" target="_blank">四库全书</a> |  <a href="http://www.guoxuedashi.com/search/" target="_blank"><font  color="#CC0066">全文检索</font></a> | <a href="http://www.guoxuedashi.com/shumu/">古籍书目</a> | <a   href="/24shi/">正史</a> <b>|</b> <a href="/chengyu/">成语词典</a> | <a href="/kangxi/" title="康熙字典">康熙字典</a> | <a href="/ShuoWenJieZi/">说文解字</a> | <a href="/zixing/yanbian/">字形演变</a> | <a href="/yzjwjc/">金 文</a> <b>|</b>  <a href="/shijian/nian-hao/">年号</a> | <a href="/diming/">历史地名</a> | <a href="/shijian/">历史事件</a> | <a href="/guanzhi/">官职</a> | <a href="/lishi/">知识</a> <b>|</b> <a href="/zhongyi/">中医中药</a> | <a href="http://www.guoxuedashi.com/forum/">留言反馈</a>
</p>
  </div>
</div>
<!-- 头部导航END --> 
<!-- 内容区开始 --> 
<div class="w1180 clearfix">
  <div class="info l">
   
<div class="clearfix" style="background:#f5faff;">
<script src='http://www.guoxuedashi.com/img/headersou.js'></script>

</div>
  <div class="info_tree"><a href="http://www.guoxuedashi.com">首页</a> > <a href="/SiKuQuanShu/fanti/">四库全书</a>
 > <h1>资治通鉴</h1> <!--         下载:【右键另存为】即可 --></div>
  <div class="info_content zj clearfix">
  
<div class="info_txt clearfix" id="show">
<center style="font-size:24px;">17-資治通鑑卷十六</center>
    資治通鑑卷十六    宋 司馬光 撰<br />
<br />
  胡三省 音註<br />
<br />
  漢紀八【起強圉大淵獻盡上章困敦凡十四年】<br />
<br />
  孝景皇帝下<br />
<br />
  前三年冬十月梁王來朝【朝直遥翻】時上未置太子與梁王宴飲從容言曰【從千容翻】千秋萬歲後傳於王王辭謝雖知非至言然心内喜【孔穎達曰喜者外境會心之謂】太后亦然詹事竇嬰【班表詹事秦官掌皇后太子家應劭曰詹省也給也臣瓚曰茂陵書詹事秩真二千石師古曰皇后太子各置詹事隨其所在以名官】引巵酒進上曰天下者高祖之天下父子相傳漢之約也上何以得傳梁王太后由此憎嬰【引酒進之盖罰爵也】嬰因病免太后除嬰門籍不得朝請【門籍出入宫殿門之籍也請材性翻又如字】梁王以此益驕 春正月乙巳赦 長星出西方 洛陽東宫災【洛陽縣河南郡治所高祖先居洛陽因築宫室有南宫北宫東宫】初孝文時吳太子入見【楚漢春秋曰吳太子名賢字德明見賢遍翻】得侍<br />
<br />
  皇太子飲博吳太子博爭道不恭皇太子引博局提吳太子殺之【提徒計翻】遣其喪歸葬至吳吳王愠曰【愠於問翻師古曰怒也孔穎達曰愠者外境違心之謂事與心違所以怒生】天下同宗【師古曰猶言同姓共為一家】死長安即葬長安何必來葬為復遣喪之長安葬吳王由此稍失藩臣之禮稱疾不朝【朝直遥翻】京師知其以子故繫治驗問吳使者吳王恐始有反謀後使人為秋請【應劭曰冬當斷獄秋先請擇其輕重也孟康曰律春曰朝秋曰請如淳曰濞不自行使人代己致請禮索隱曰音淨孟說是】文帝復問之【復扶又翻】使者對曰王實不病漢繫治使者數輩吳王恐以故遂稱病夫察見淵中魚不祥【服䖍曰言天子察見下之私則不祥也索隱曰案此語見韓子及文子韋昭曰知臣下隂私使憂患生變為不祥故當赦宥使自新也】唯上棄前過與之更始【師古曰言赦其已往之事使得自新也更工衡翻】於是文帝乃赦吳使者歸之而賜吳王几杖老不朝吳得釋其罪謀亦益解然其居國以銅鹽故百姓無賦【索隱曰吳國有鑄錢煮鹽之利故百姓不别徭賦也】卒踐更輒予平賈【服䖍曰以當為更卒出錢三百謂之過更自行為卒謂之踐更吳王欲得民心以為卒者雇其庸隨時月予平賈晉灼曰謂借人自代為卒者官為出錢雇其時庸平賈也師古曰晉說是索隱曰案漢律卒更有三踐更居更過更也此言踐更輒與平賈者謂為踐更合自出錢今吳王欲得人心乃予平賈官讎之也予讀曰與下同賈讀曰價】歲時存問茂材賞賜閭里他郡國吏欲來捕亡人者公共禁弗予如此者四十餘年鼂錯數上書言吳過可削文帝寛不忍罰以此吳日益横【鼂直遥翻錯千故翻數所角翻横戶孟翻】及帝即位錯說上曰昔高帝初定天下昆弟少諸子弱【說式芮翻少詩沼翻】大封同姓齊七十餘城楚四十餘城吳五十餘城封三庶孽分天下半今吳王前有太子之郤【郤與隙同下有郤同】詐稱病不朝於古灋當誅文帝弗忍因賜几杖德至厚當改過自新反益驕溢即山鑄錢【師古曰即就也】煮海水為鹽誘天下亡人謀作亂今削之亦反不削亦反削之其反亟禍小不削反遲禍大上令公卿列侯宗室雜議莫敢難獨竇嬰爭之由此與錯有郤【難乃旦翻郤與隙同】及楚王戊來朝錯因言戊往年為薄太后服私姦服舍【師古曰服舍居喪之次若堊室之屬也】請誅之詔赦削東海郡【東海郡即秦郯郡高帝更名】及前年趙王有罪削其常山郡膠西王卬以賣爵事有姦【膠西後改為高密】削其六縣廷臣方議削吳吳王恐削地無己因發謀舉事念諸侯無足與計者聞膠西王勇好兵【好呼到翻】諸侯皆畏憚之於是使中大夫應高口說膠西王曰【應本自周武王後左傳曰䢴晉應韓武之穆也】今者主上任用邪臣聽信讒賊侵削諸侯誅罰良重【師古曰良實也信也】日以益甚語有之曰狧穅及米【師古曰狧古字食爾翻狧用舌食也盖以犬為諭言初狧穅遂至食米也索隱曰言狧穅盡則至米謂削土盡則至滅國也】吳與膠西知名諸侯也一時見察不得安肆矣【師古曰肆縱也】吳王身有内疾【師古曰謂疾在身中不顯於外也】不能朝請二十餘年常患見疑無以自白脅肩累足猶懼不見釋【師古曰脅翕也謂斂之也累足重足也並謂懼耳釋解也放也累與絫同】竊聞大王以爵事有過所聞諸侯削地罪不至此【師古曰言其罪皆不至于削地】此恐不止削地而已王曰有之子將奈何高曰吳王自以為與大王同憂願因時循理棄軀以除患於天下意亦可乎膠西王瞿然駭曰【瞿居具翻說文瞿遠視貌師古曰瞿然無守之貌】寡人何敢如是主上雖急固有死耳安得不事高曰御史大夫鼂錯營惑天子【師古曰營謂囘繞之也】侵奪諸侯諸侯皆有背叛之意人事極矣彗星出【背蒲妹翻彗祥歲翻又徐醉翻又旋芮翻】蝗蟲起此萬世一時而愁勞聖人所以起也【索隱曰所謂殷憂以啟明聖也】吳王内以鼂錯為誅外從大王後車方洋天下【方音房又音旁洋音羊師古曰方洋猶翺翔也】所向者降【降戶江翻】所指者下莫敢不服大王誠幸而許之一言則吳王率楚王略函谷關守滎陽敖倉之粟距漢兵治次舍須大王【師古曰次舍息止之處須待也治直之翻】大王幸而臨之則天下可併兩主分割不亦可乎王曰善歸報吳王吳王猶恐其不果乃身自為使者至膠西面約之膠西羣臣或聞王謀諫曰諸侯地不能當漢十二為叛逆以憂太后非計也【文頴曰謂王之太后也】今承一帝尚云不易【易以䜴翻】假令事成兩主分爭患乃益生王不聽遂發使約齊菑川膠東濟南皆許諾【齊王將閭菑川王賢膠東王雄渠濟南王辟光皆文帝封濟子禮翻】初楚元王好書【好呼到翻】與魯申公穆生白生俱受詩於浮丘伯及王楚以三人為中大夫【及王于况翻】穆生不耆酒元王每置酒常為穆生設醴及子夷王孫王戊即位【楚元王交高祖異母弟楚子重子辛皆出於穆王楚人謂之二穆故楚有穆姓秦有白乙丙白圭楚有白公浮丘複姓夷王名郢客元王子戊元王孫師古曰醴甘酒少麯多米二宿而熟不耆之耆讀曰嗜為于偽翻下同】常設後乃忘設焉【忘巫放翻】穆生退曰可以逝矣醴酒不設王之意怠不去楚人將鉗我於市遂稱疾卧申公白生彊起之【彊其兩翻】曰獨不念先王之德與【與讀曰歟】今王一旦失小禮何足至此穆生曰易稱知幾其神乎幾者動之微吉凶之先見者也【幾居依翻師古曰易下繫之辭見戶電翻】君子見幾而作不俟終日先王之所以禮吾三人者為道存也今而忽之是忘道也忘道之人胡可與久處豈為區區之禮哉【區區謂小也處昌呂翻為于偽翻】遂謝病去申公白生獨留王戊稍淫暴太傅韋孟作詩諷諫不聽亦去居於鄒【姓譜韋姓出顓頊大彭豕韋之後】戊因坐削地事遂與吳通謀申公白生諫戊戊胥靡之衣之赭衣使雅舂於市【晉灼曰高肱舉杵正身而舂之師古曰為木杵而手舂即今所謂步臼者耳衣之於既翻】休侯富使人諫王【孟子去齊居休趙岐註曰休地名盖即富所封之地富楚元王之子夷王之弟也】王曰季父不吾與我起先取季父矣休侯懼乃與母太夫人犇京師【臣瓚曰侯母號太夫人】及削吳會稽豫章郡書至吳王遂先起兵誅漢吏二千石以下膠西膠東菑川濟南楚趙亦皆反楚相張尚太傅趙夷吾諫王戊戊殺尚夷吾趙相建德内史王悍諫王遂遂燒殺建德悍【悍下罕翻又侯旰翻】齊王後悔背約城守【背蒲妹翻守式又翻】濟北王城壞未完其郎中令刼守王不得發兵膠西王膠東王為渠率【師古曰渠大也率所類翻】與菑川濟南共攻齊圍臨菑【臨菑齊都】趙王遂發兵住其西界欲待吳楚俱進北使匈奴與連兵【使疏吏翻下同】吳王悉其士卒下令國中曰寡人年六十二身自將【將即亮翻】少子年十四亦為士卒先諸年上與寡人同下與少子等皆發凡二十餘萬人南使閩東越【使疏吏翻】閩東越亦發兵從【從才用翻】吳王起兵于廣陵【廣陵吳都】西涉淮因并楚兵發使遺諸侯書罪狀鼂錯【遺于季翻】欲合兵誅之吳楚共攻梁破棘壁【索隱曰按左氏傳宣公二年宋華元戰於大棘杜預曰在襄邑東南盖即棘壁是也括地志大棘故城在宋州寧陵縣西南七十里】殺數萬人乘勝而前鋭甚梁孝王遣將軍擊之又敗梁兩軍【敗補邁翻】士卒皆還走梁王城守睢陽【睢陽梁都睢音雖】初文帝且崩戒太子曰即有緩急周亞夫真可任將兵及七國反書聞上乃拜中尉周亞夫為太尉將三十六將軍往擊吳楚遣曲周侯酈寄擊趙【班志曲周縣屬廣平國】將軍欒布擊齊復召竇嬰拜為大將軍使屯滎陽監齊趙兵【班志滎陽縣屬河南郡監古銜翻】初鼂錯所更令三十章【更工衡翻】諸侯讙譁【讙許元翻】錯父聞之從潁川來【錯潁川人】謂錯曰上初即位公為政用事侵削諸侯疏人骨肉【疏與疎同】口語多怨公何為也錯曰固也不如此天子不尊宗廟不安父曰劉氏安矣而鼂氏危吾去公歸矣遂飲藥死曰吾不忍見禍逮身後十餘日吳楚七國俱反以誅錯為名上與錯議出軍事錯欲令上自將兵而身居守【守式又翻】又言徐僮之旁吳所未下者可以予吳【徐僮二縣皆屬臨淮郡錯初議削諸侯地以彊漢及七國反乃欲以徐僮之旁予吳是自畔其說惡得無死乎予讀曰與】錯素與吳相袁盎不善【相息亮翻】錯所居坐盎輒避盎所居坐錯亦避【坐徂卧翻】兩人未嘗同堂語及錯為御史大夫使吏按盎受吳王財物抵辠詔赦以為庶人吳楚反錯謂丞史曰【班表御史大夫有兩丞秩千石侍御史十五人】袁盎多受吳王金錢專為蔽匿言不反今果反欲請治盎宜知其計謀丞史曰事未發治之有絶【如淳曰事未發之時治之乃有所絶也治直之翻】今兵西向治之何益且盎不宜有謀錯猶與未決【猶與即猶豫也與去聲】人有告盎盎恐夜見竇嬰為言吳所以反願至前口對狀嬰入言上乃召盎盎入見【為于偽翻入見賢遍翻】上方與錯調兵食【師古曰調計也計發兵食也調徒釣翻】上問盎今吳楚反於公意何如對曰不足憂也上曰吳王即山鑄錢煮海為鹽誘天下豪傑白頭舉事此其計不百全豈發乎何以言其無能為也對曰吳銅鹽之利則有之安得豪傑而誘之【誘音酉】誠令吳得豪傑亦且輔而為誼不反矣吳所誘皆無賴子弟亡命鑄錢姦人【章懷太子賢曰命名也謂脱其名籍而逃亡】故相誘以亂錯曰盎策之善上曰計安出盎對曰願屏左右上屏人獨錯在盎曰臣所言人臣不得知乃屏錯【屛必郢翻】錯趨避東廂甚恨上卒問盎【卒子恤翻下卒受同】對曰吳楚相遺書言高皇帝王子弟各有分地【遺于季翻分扶問翻】今賊臣鼂錯擅適諸侯【適讀曰讁】削奪之地以故反欲西共誅錯復故地而罷方今計獨有斬錯發使赦吳楚七國【使疏吏翻下使吳同】復其故地則兵可毋血刃而俱罷於是上默然良久曰顧誠何如吾不愛一人以謝天下盎曰愚計出此唯上孰計之【孰與熟同】乃拜盎為太常【中六年始改奉常為太常時盎猶為奉常也】密裝治行【治直之翻】後十餘日上令丞相青中尉嘉廷尉歐【丞相陶青中尉嘉失其姓廷尉張歐】劾奏錯不稱主上德信欲疏羣臣百姓又欲以城邑予吳無臣子禮大逆無道錯當要斬【劾戶槩翻疏與疎同予讀曰與要與腰同】父母妻子同產無少長皆棄市【少詩照翻長知兩翻】制曰可錯殊不知壬子上使中尉召錯紿載行市【師古曰誑云乘車案行市中也行下孟翻】錯衣朝衣斬東市【衣朝上於既翻下直遥翻】上乃使袁盎與吳王弟子宗正德侯通使吳【高祖兄仲之子廣封德侯生通德侯國在泰山界】謁者僕射鄧公為校尉上書言軍事見上【校戶教翻上書之上時掌翻】上問曰道軍所來【如淳曰道路從吳軍所來也臣瓚曰道由也】聞鼂錯死吳楚罷不【不讀曰否】鄧公曰吳為反數十歲矣發怒削地以誅錯為名其意不在錯也且臣恐天下之士拑口不敢復言矣【拑其炎翻復扶又翻】上曰何哉鄧公曰夫鼂錯患諸侯彊大不可制故請削之以尊京師萬世之利也計畫始行卒受大戮【卒子恤翻或讀為猝】内杜忠臣之口外為諸侯報仇臣竊為陛下不取也【為于偽翻】於是帝喟然長息曰公言善吾亦恨之袁盎劉通至吳吳楚兵已攻梁壁矣宗正以親故先入見諭吳王令拜受詔【宗正於濞猶子之親也】吳王聞袁盎來知其欲說【說式芮翻下同】笑而應曰我已為東帝尚誰拜不肯見盎而留軍中欲刼使將【將即亮翻】盎不肯使人圍守且殺之盎得間脱亡歸報【間古莧翻】太尉亞夫言於上曰楚兵剽輕難與爭鋒【剽匹妙翻輕虚勁翻】願以梁委之絶其食道乃可制也上許之亞夫乘六乘傳【張晏曰傳車六乘也乘繩證翻傳張戀翻余據漢有乘傳馳傳文帝之自代入立也張武等乘六乘傳今亞夫乘六乘傳六乘傳之見於史者二盖又與乘傳不同也】將會兵滎陽【師古曰會兵謂集大兵】發至霸上趙涉遮說亞夫曰吳王素富懷輯死士久矣此知將軍且行必置間人於殽澠阸陿之間【澠彌兗翻殽山澠池之間其道阸陿阸於懈翻陿與狹同】且兵事尚神密將軍何不從此右去走藍田出武關抵洛陽間不過差一二日【自霸上左趨殽澠至洛陽其道便近若自霸上右趨藍田出武關至洛陽其道迂曲故差一二日走音奏間如字】直入武庫【洛陽有武庫】擊鳴鼓諸侯聞之以為將軍從天而下也太尉如其計至洛陽喜曰七國反吾乘傳至此不自意全【師古曰言不自意得安全至洛陽也】今吾據滎陽滎陽以東無足憂者 【考異曰史記漢書皆云太尉得劇孟喜如得一敵國曰吳楚無足憂者按孟一游俠之士耳亞夫得之何足為輕重盖其徒欲為孟重名妄撰此言不足信也】使吏搜殽澠間果得吳伏兵乃請趙涉為護軍太尉引兵東北走昌邑【昌邑梁地後為山陽郡治所走音奏下同】吳攻梁急梁數使使條侯求救條侯不許【班志勃海郡有脩縣音條數所角翻使使上如字下疏吏翻】又使使愬條侯於上上使告條侯救梁亞夫不奉詔堅壁不出而使弓高侯等將輕騎兵出淮泗口【韓王信之子頹當自匈奴中來歸封為弓高侯功臣表弓高屬營陵地理志弓高屬河間國盖頹當受封於文帝之初而河間國則二年所置故志與表異泗水南入淮故謂之淮泗口騎奇寄翻】絶吳楚兵後塞其饟道【塞悉則翻饟古餉字】梁使中大夫韓安國及楚相張尚弟羽為將軍羽力戰安國持重乃得頗敗吳兵吳兵欲西梁城守不敢西【敗補邁翻守式又翻】即走條侯軍會下邑【下邑縣屬梁國】欲戰條侯堅壁不肯戰吳糧絶卒飢數挑戰終不出【數所角翻挑徙了翻】條侯軍中夜驚内相攻擊擾亂至帳下亞夫堅卧不起頃之復定吳犇壁東南陬【陬子侯翻隅也】亞夫使備西北已而其精兵果犇西北不得入吳楚士卒多飢死叛散乃引而去二月亞夫出精兵追擊大破之吳王濞棄其軍與壯士數千人夜亡走楚王戊自殺吳王之初發也吳臣田禄伯為大將軍田禄伯曰兵屯聚而西無它奇道難以立功臣願得五萬人别循江淮而上【上時掌翻】收淮南長沙入武關與大王會此亦一奇也吳王太子諫曰王以反為名此兵難以借人人亦且反王奈何且擅兵而别多它利害【蘇林曰禄伯儻將兵降漢自為己利於吳生患也】徒自損耳吳王即不許田禄伯吳少將桓將軍說王曰吳多步兵步兵利險漢多車騎車騎利平地願大王所過城不下直去疾西據洛陽武庫食敖倉粟阻山河之險以令諸侯雖無入關天下固已定矣大王徐行留下城邑漢軍車騎至馳入梁楚之郊事敗矣吳王問諸老將老將曰此年少椎鋒可耳安知大慮【老將即亮翻下并將為將同】於是王不用桓將軍計王專并將兵兵未度淮諸賓客皆得為將校尉司馬【凡軍行有大將禆將領軍皆有部曲部有校尉曲有軍軍司馬又有假假司馬皆有副其别營領屬為别部司馬】獨周丘不用丘者下邳人【班志下邳屬東海郡】亡命吳酤酒無行【行下孟翻】王薄之不任周丘乃上謁說王曰臣以無能不得待罪行間【上時掌翻說式芮翻行戶剛翻】臣非敢求有所將也【將即亮翻】願請王一漢節必有以報王乃予之【予讀曰與】周丘得節夜馳入下邳下邳時聞吳反皆城守至傳舍召令入戶使從者以罪斬令【傳張戀翻令力正翻從才用翻】遂召昆弟所善豪吏吿曰吳反兵且至屠下邳不過食頃今先下家室必完能者封侯矣出乃相吿下邳皆下周丘一夜得三萬人使人報吳王遂將其兵北畧城邑比至陽城兵十餘萬破陽城中尉軍【陽城漢書作城陽城陽國都莒其地南接下邳之境班表王國有中尉掌武職比必寐翻及也】聞吳王敗走自度無與共成功【度徒洛翻】即引兵歸下邳未至疽發背死【史言吳王有才不能用以至于敗】 壬午晦日有食之 吳王之棄軍亡也軍遂潰往往稍降太尉條侯及梁軍【降戶江翻】吳王度淮走丹徒【班志丹徒縣屬會稽郡即春秋之朱方括地志丹徒故城在潤州丹徒縣東南十八里南徐州記秦使赭衣鑿其處因謂之丹徒鑿處今在故縣西北六里丹徒峴東南】保東越【欲依東越以自保也】兵可萬餘人收聚亡卒漢使人以利㗖東越【㗖徒覽翻餌之也又徒濫翻譙也食也】東越即紿吳王出勞軍【勞力到翻】使人鏦殺吳王【孟康曰方言戟謂之鏦蘇林曰鏦音從容之從師古曰鏦謂以矛戟撞殺之鏦楚江翻】盛其頭馳傳以聞【盛時征翻傳張戀翻】吳太子駒亡走閩越吳楚反凡三月皆破滅於是諸將乃以太尉謀為是然梁王由此與太尉有隙【為梁王毁短亞夫張本】三王之圍臨菑也齊王使路中大夫吿於天子【張晏曰姓路官為中大夫姓譜路本自帝摯之後】天子復令路中大夫還報告齊王堅守漢兵今破吳楚矣路中大夫至三國兵圍臨菑數重無從入三國將與路中大夫盟曰若反言漢已破矣【重直龍翻將即亮翻師古曰若汝也反謂反易其辭也】齊趣下三國【趣讀曰促】不且見屠路中大夫既許至城下望見齊王曰漢已發兵百萬使太尉亞夫擊破吳楚方引兵救齊齊必堅守無下三國將誅路中大夫齊初圍急隂與三國通謀約未定會路中大夫從漢來其大臣乃復勸王無下三國會漢將欒布平陽侯等兵至齊【據班史齊王傳作平陽侯曹襄史記索隱曰平陽侯按表是簡侯曹奇】擊破三國兵解圍已【句斷】後聞齊初與三國有謀將欲移兵伐齊齊孝王懼飲藥自殺膠西膠東菑川王各引兵歸國膠西王徒跣席藁飲水謝太后王太子德曰漢兵還臣觀之已罷【罷與疲同】可襲願收王餘兵擊之不勝而逃入海未晚也王曰吾士卒皆已壞不可用弓高侯韓頹當遺膠西王書曰奉詔誅不義降者赦除其罪復故不降者滅之【遺于季翻降戶江翻】王何處須以從事【謂膠西王於降與不降之間欲以何自處吾待以行事處昌呂翻】王肉袒叩頭詣漢軍壁謁曰臣卬奉法不謹驚駭百姓乃苦將軍遠道至于窮國敢請菹醢之罪弓高侯執金鼓見之曰王苦軍事願聞王發兵狀王頓首䣛行【䣛與膝同】對曰今者鼂錯天子用事臣變更高皇帝法令侵奪諸侯地卬等以為不義恐其敗亂天下【更工衡翻敗補邁翻】七國發兵且誅錯今聞錯己誅卬等謹已罷兵歸將軍曰王苟以錯為不善何不以聞及未有詔虎符擅發兵擊義國以此觀之意非徒欲誅錯也乃出詔書為王讀之【為于偽翻下同】曰王其自圖王曰如卬等死有餘罪遂自殺太后太子皆死膠東王菑川王濟南王皆伏誅酈將軍兵至趙趙王引兵還邯鄲城守【邯鄲趙都】酈寄攻之七月不能下匈奴聞吳楚敗亦不肯入邊欒布破齊還并兵引水灌趙城城壞王遂自殺帝以齊首善【師古曰言其初首無逆亂之心】以迫刼有謀非其辠也召立齊孝王太子夀是為懿王濟北王亦欲自殺【濟北王志齊悼惠王子文帝十六年受封】幸全其妻子齊人公孫玃謂濟北王曰【玃俱碧翻康俱縛切】臣請試為大王明說梁王通意天子說而不用死未晚也【為于偽翻說式芮翻】公孫玃遂見梁王曰夫濟北之地東接彊齊南牽吳越北脅燕趙此四分五裂之國【張晏曰四方受敵濟北居中央為五晉灼曰四分即交午而裂如田字也】權不足以自守勁不足以捍寇又非有奇怪云以待難也雖墜言於吳非其正計也【如淳曰非有奇材異計欲為亂逆也但假權許吳以避禍耳晉灼曰非有以怪異之心而城守須待變難以應吳也師古曰二說皆非也此言權謀勁力旣不能扞守又無奇怪神靈可以禦難恐不能自全故墜言於吳也墜猶失也難乃旦翻】鄉使濟北見情實示不從之端【鄉讀曰向見賢遍翻】則吳必先歷齊畢濟北【歷過也畢了也】招燕趙而總之如此則山東之從結而無隙矣【從子容翻】今吳王連諸侯之兵白徒之衆【師古曰與驅同白徒素非習軍旅之人猶言白丁也】西與天子爭衡濟北獨底節不下使吳失與而無助跬步獨進【師古曰半步曰跬跬空累翻】瓦解土崩破敗而不救者未必非濟北之力也夫以區區之濟北而與諸侯爭彊是以羔犢之弱而扞虎狼之敵也【小羊曰羔小牛曰犢】守職不橈【橈奴教翻】可謂誠一矣功義如此尚見疑於上脅肩低首累足撫衿使有自悔不前之心【自悔者悔不與吳同也不前不敢前進以自歸於漢也】非社稷之利也臣恐藩臣守職者疑之臣竊料之能歷西山徑長樂抵未央攘袂而正議者【師古曰西山謂殽及華山也抵至也攘郤也袂衣袖也攘袂猶今人言捋臂耳余謂長樂太后居之未央天子居之徑長樂抵未央猶言自太后所至帝所也樂音洛】獨大王耳上有全亡之功下有安百姓之名德淪於骨髓恩加於無窮願大王留意詳惟之【惟思也】孝王大說【說讀曰悦】使人馳以聞濟北王得不坐徙封於菑川 河間王太傅衛綰擊吳楚有功拜為中尉綰以中郎將事文帝醇謹無他上為太子時召文帝左右飲而綰稱病不行文帝且崩屬上曰綰長者善遇之故上亦寵任焉【屬之欲翻】 夏六月乙亥詔吏民為吳王濞等所詿誤當坐及逋逃亡軍者皆赦之【詿戶卦翻亡軍從軍而逃者也】帝欲以吳王弟德哀侯廣之子續吳以楚元王子禮續楚【德哀侯廣之子即德侯通也禮時封平陸侯為宗正】竇太后曰吳王老人也宜為宗室順善今乃首率七國紛亂天下奈何續其後不許吳許立楚後乙亥徙淮陽王餘為魯王汝南王非為江都王王故吳地立宗正禮為楚王立皇子端為膠西王勝為中山王【中山王都盧奴】<br />
<br />
  四年春復置關用傳出入【應劭曰文帝十三年除關無用傳至此復用傳以七國新反備非常傳張戀翻】 夏四月己巳立子榮為皇太子徹為膠東王 六月赦天下 秋七月臨江王閼薨 冬十月戊戌晦【月末為晦】日有食之【李心傳曰漢景帝四年中四年皆以冬十月日食今通鑑書於夏秋之後盖編輯者自志中摘出不思漢初以十月為歲首故誤係之歲末耳余按此誤劉貢父已言之通鑑盖承用漢書本紀也】 初吳楚七國反吳使者至淮南淮南王欲發兵應之其相曰王必欲應吳臣願為將王乃屬之【將即亮翻下同屬之欲翻委也言以兵事委之】相已將兵因城守不聽王而為漢【守式又翻為于偽翻】漢亦使曲城侯將兵救淮南【晉灼曰曲城侯功臣表蟲達也師古曰晉說非此蟲達之子耳名捷達已先薨也班志曲城縣屬東萊郡】以故得完吳使者至廬江廬江王不應而往來使越【使疏吏翻】至衡山衡山王堅守無二心及吳楚已破衡山王入朝上以為貞信勞苦之曰【勞來到翻】南方卑濕徙王王於濟北以褒之【王於之王于况翻】廬江王以邊越數使使相交【師古曰邊越者邊界與越相接據班志廬江故淮南文帝别為國廬江水出陵陽東南而北入於江陵陽縣屬丹陽郡文帝初分淮南為廬江國在江南若班志之廬江郡則其地盡在江北矣數所角翻】徙為衡山王王江北【衡山王都六其地在江北】五年春正月作陽陵邑【班志陽陵縣屬馮翊本弋陽縣索隱曰帝豫作夀陵於此因更縣名在長安東北四十五里】夏募民徙陽陵賜錢二十萬 遣公主嫁匈奴單于 徙廣川王彭祖為趙王 濟北貞王勃薨【諡法清白守節曰貞】<br />
<br />
  六年冬十二月雷霖雨【雨三日以往為霖】 初上為太子薄太后以薄氏女為妃及即位為皇后無寵秋九月皇后薄氏廢 楚文王禮薨 初燕王臧荼有孫女曰臧兒嫁為槐里王仲妻生男信與兩女而仲死【班志槐里縣屬扶風秦之廢丘也高祖二年更名】更嫁長陵田氏【更工衡翻】生男蚡勝【蚡扶粉翻】文帝時臧兒長女為金王孫婦生女俗【長知兩翻下同】臧兒卜筮之曰兩女皆當貴臧兒乃奪金氏婦金氏怒不肯予決【予讀曰與決别也言不肯與别師古曰決絶也】内之太子宫生男徹徹方在身【身與娠同師古曰漢史多以娠為任身字】時王夫人夢日入其懷及帝即位長男榮為太子其母栗姬齊人也長公主嫖欲以女嫁太子【長知兩翻嫖文帝女景帝之姊師古曰年最長故謂之長公主余謂帝女稱公主帝之姊妹稱長公主嫖降堂邑侯陳午生女是為武帝陳皇后嫖匹昭翻】栗姬以後宫諸美人皆因長公主見帝故怒而不許長公主欲予王夫人男徹【予讀曰與】王夫人許之由是長公主日讒栗姬而譽王夫人之美【譽音余】帝亦自賢之又有曩者所夢日符【王夫人之震武帝也夢日入其懷所謂符也】計未有所定王夫人知帝嗛栗姬【嗛乎監翻口有所銜也康曰恨也史記曰帝嘗體不安屬諸子為王者於栗姬曰善視之栗姬怒不肯應言不遜帝恚心嗛之而未發也】因怒未解隂使人趣大行【晉灼曰禮有大行人小行人主諡官臣瓚曰大行是官名掌九儀之制以賓諸侯者師古曰大行令本名行人典客屬官也後改曰大行令余按班表帝中六年改典客曰大行令武帝太初元年改大行令為大鴻臚更名行人為大行令意其有誤不然則追書也原父曰史記文景事最畧漢書則頗有所録盖班氏博採他書成之故於景帝世謂典客為鴻臚行人為大行由他書即武帝時官記景帝世事班氏失於改革耳非表誤也趣讀曰促】請立栗姬為皇后帝怒曰是而所宜言邪【而汝也】遂按誅大行<br />
<br />
  七年冬十一月己酉廢太子榮為臨江王太子太傅竇嬰力爭不能得乃謝病免栗姬恚恨而死 庚寅晦日有食之 二月丞相陶青免乙巳太尉周亞夫為丞相罷太尉官 夏四月乙巳立皇后王氏 丁巳立膠東王徹為皇太子 是歲以太僕劉舍為御史大夫【劉舍高祖功臣桃安侯劉襄之子襄本項氏親賜姓】濟南太守郅都為中尉【濟南王辟光反國除為郡郅之日翻風俗通郅商時侯國後以為氏】始都為中郎將敢直諫嘗從入上林賈姬如厠【賈姬即賈夫人生趙王彭祖中山王勝】野彘卒來入厠【卒讀曰猝】上目都都不行上欲自持兵救賈姬都伏上前曰亡一姬復一姬進【復扶又翻】天下所少寧賈姬等乎陛下縱自輕奈宗廟太后何上乃還彘亦去太后聞之賜都金百斤由此重都都為人勇悍公廉不發私書問遺無所受【悍下罕翻遺于季翻】請謁無所聽及為中尉先嚴酷【先悉薦翻】行法不避貴戚列侯宗室見都側目而視號曰蒼鷹【師古曰言其鷙擊之甚】<br />
<br />
  中元年夏四月乙巳赦天下 地震衡山原都雨雹大者尺八寸【原都地名盖屬衡山國雨王遇翻】<br />
<br />
  二年春二月匈奴入燕【燕因肩翻】 三月臨江王榮坐侵太宗廟壖垣為宫徵詣中尉府對簿【帝即位之初令天下郡國各立太祖太宗之廟故臨江王國亦有之壖與堧同而緣翻師古曰簿者獄辭之文書簿步戶翻】臨江王欲得刀筆為書謝上【師古曰刀所以削治書也古者著書於簡牘故必用刀焉】而中尉郅都禁吏不予魏其侯使人間與臨江王【伺間隙而與之也魏其侯竇嬰班志魏其侯國屬琅邪郡予讀曰與間古莧翻】臨江王既為書謝上因自殺竇太后聞之怒後竟以危法中都而殺之【師古曰謂構成其罪中竹仲翻考異曰史記本紀後二年正月郅將軍擊匈奴酷吏傳郅都死後宗室多犯法上乃召甯成為中尉成為中】<br />
<br />
  【尉在中六年則後二年所謂郅將軍者非都也疑别一人漢書紀無郅將軍事】 夏四月有星孛于西北【孛蒲内翻】 立皇子越為廣川王寄為膠東王【廣川王彭祖王趙故立越為王膠東王徹為太子故立寄為王】 秋九月甲戌晦日有食之 初梁孝王以至親有功【梁王以母弟之親又有破吳楚之功】得賜天子旌旗從千乘萬騎出蹕入警王寵信羊勝公孫詭以詭為中尉勝詭多奇邪計欲使王求為漢嗣栗太子之廢也【太子榮栗姬之子故號栗太子】太后意欲以梁王為嗣嘗因置酒謂帝曰安車大駕用梁王為寄帝跪席舉身曰諾罷酒帝以訪諸大臣大臣袁盎等曰不可昔宋宣公不立子而立弟以生禍亂五世不絶【宋宣公舍其子與夷而立穆公穆公又舍其子馮而立與夷其後馮卒與與夷爭國見春秋傳】小不忍害大義故春秋大居正【公羊傳之言】由是太后議格遂不復言【格音閣止也】王又嘗上書願賜容車之地徑至長樂宫自使梁國士衆築作甬道朝太后【甬余拱翻朝直遥翻】袁盎等皆建以為不可【建建議也】梁王由此怨袁盎及議臣乃與羊勝公孫詭謀隂使人刺殺袁盎及他議臣十餘人【刺七亦翻】賊未得也於是天子意梁【意梁者以意測度知其為梁所為也】逐賊果梁所為上遣田叔呂季主往按梁事捕公孫詭羊勝詭勝匿王後宫使者十餘輩至梁責二千石急梁相軒丘豹及内史韓安國以下舉國大索【姓譜楚文王庶子食采於軒丘其後為氏索山客翻】月餘弗得安國聞詭勝匿王所乃入見王而泣曰主辱者臣死大王無良臣故紛紛至此今勝詭不得請辭賜死王曰何至此安國泣數行下曰大王自度於皇帝孰與臨江王親王曰弗如也安國曰臨江王適長太子【行戶剛翻度徒洛翻適讀曰嫡長知兩翻】以一言過【師古曰景帝常屬諸姬子栗姬言不遜由是廢太子】廢王臨江用宫垣事卒自殺中尉府【王于況翻卒子恤翻下同】何者治天下終不用私亂公【治直之翻】今大王列在諸侯訹邪臣浮說【訹音戌誘也】犯上禁橈明法【橈奴敎翻】天子以太后故不忍致法於大王太后日夜涕泣幸大王自改大王終不覺寤有如太后宫車即晏駕大王尚誰攀乎語未卒王泣數行而下【卒子恤翻行戶剛翻】謝安國曰吾今出勝詭王乃令勝詭皆自殺出之上由此怨望梁王梁王恐使鄒陽入長安見皇后兄王信說曰【說式芮翻下同】長君弟得幸於上後宫莫及而長君行迹多不循道理者【長知兩翻行下孟翻】今袁盎事即窮竟梁王伏誅太后無所發怒切齒側目於貴臣竊為足下憂之【為于偽翻下精為同】長君曰為之奈何陽曰長君誠能請為上言之得毋竟梁事長君必固自結於太后太后厚德長君入於骨髓而長君之弟幸於兩宫【長君之弟謂皇后也如淳曰兩宫太后宫及帝宫也】金城之固也【師古曰言其榮寵無極而不可壞故取喻於金城】昔者舜之弟象日以殺舜為事及舜立為天子封之於有卑【卑音鼻師古及柳宗元皆以為零陵之鼻亭即象所封】夫仁人之於兄弟無藏怒無宿怨厚親愛而已【用孟子語意】是以後世稱之以是說天子徼幸梁事不奏長君曰諾乘間入言之【徼一遥翻間古莧翻】帝怒稍解是時太后憂梁事不食日夜泣不止帝亦患之會田叔等按梁事來還至霸昌廐【霸昌廐在長安東括地志在雍州萬年縣東北三十八里】取火悉燒梁之獄辭空手來見帝【見賢遍翻】帝曰梁有之乎叔對曰死罪有之上曰其事安在田叔曰上毋以梁事為問也上曰何也曰今梁王不伏誅是漢法不行也伏法而太后食不甘味卧不安席此憂在陛下也上大然之使叔等謁太后且曰梁王不知也造為之者獨在幸臣羊勝公孫詭之屬為之耳謹已伏誅死梁王無恙也【恙余亮翻】太后聞之立起坐餐氣平復梁王因上書請朝【朝直遥翻】既至關茅蘭說王使乘布車從兩騎入匿於長公主園【服䖍曰茅蘭孝王大夫張晏曰布車降服自比喪人也長公主即館陶長公主嫖】漢使使迎王王已入關車騎盡居外不知王處太后泣曰帝果殺吾子帝憂恐於是梁王伏斧質於闕下謝罪太后帝大喜相泣復如故悉召王從官入關【從才用翻】然帝益疏王不與同車輦矣【疏與疎同下同】帝以田叔為賢拜為魯相【相魯王餘也】<br />
<br />
  三年冬十一月罷諸侯御史大夫官 夏四月地震旱禁酤酒【酤工護翻謂賣酒也】 三月丁巳立皇子乘為清河王【高帝置清河郡於齊趙之間今以為王國】 秋九月蝗 有星孛于西北【孛蒲内翻】 戊戌晦日有食之 初上廢栗太子周亞夫固爭之不得上由此疏之而梁孝王每朝常與太后言條侯之短【梁王與條侯有隙見前三年】竇太后曰皇后兄王信可侯也帝讓曰始南皮章武先帝不侯【南皮侯竇彭祖太后弟長君之子章武侯竇廣國太后弟也班志南皮章武皆屬勃海郡】及臣即位乃侯之信未得封也竇太后曰人生各以時行耳自竇長君在時竟不得侯死後其子彭祖顧得侯吾甚恨之帝趣侯信也【趣讀曰促】帝曰請得與丞相議之上與丞相議亞夫曰高皇帝約非劉氏不得王非有功不得侯今信雖皇后兄無功侯之非約也帝默然而止其後匈奴王徐盧等六人降【降戶江翻】帝欲侯之以勸後丞相亞夫曰彼背主降陛下【背蒲内翻】陛下侯之則何以責人臣不守節者乎帝曰丞相議不可用乃悉封徐盧等為列侯【徐盧容城侯賜栢侯陸疆遒侯僕䵣易侯范代范陽侯邯鄲翕侯䵣師古音怛】亞夫因謝病九月戊戌亞夫免以御史大夫桃侯劉舍為丞相【索隱曰桃縣屬信都郡】<br />
<br />
  四年夏蝗 冬十月戊午日有食之<br />
<br />
  五年夏立皇子舜為常山王【高帝置常山郡屬趙國呂后分為王國文帝併為趙國今復以王舜】 六月丁巳赦天下 大水 秋八月己酉未央宫東闕災 九月詔諸獄疑若雖文致於法【謂原情定罪本不至於死而以律文傅致之】而於人心不厭者輒讞之【厭服也師古曰一涉翻又於涉翻讞魚列翻又魚蹇翻平議也】 地震<br />
<br />
  六年冬十月梁王來朝上疏欲留上弗許【禇少孫曰諸侯王朝見天子漢法凡當四見耳始到入小見到正月朔旦奉皮薦璧玉賀正月法見後三日為王置酒賜金錢財物後二日復入小見辭去凡留長安不過二十日小見者燕見於禁門内飲於省中】王歸國意忽忽不樂【樂音洛】 十一月改諸廷尉將作等官名【時改廷尉為大理將作少府為大匠奉常為太常典客為大行令長信詹事為長信少府將行為大長秋主爵中尉為都尉】春二月乙卯上行幸雍郊五畤【畤音止】 三月雨雪 夏四月梁孝王薨竇太后聞之哭極哀不食曰帝果殺吾子帝哀懼不知所為與長公主計之乃分梁為五國盡立孝王男五人為王買為梁王明為濟川王彭離為濟東王定為山陽王不識為濟隂王【梁仍都睢陽濟川國在陳留東郡之間濟東國後入漢為大河郡後又為東平國山陽國即山陽郡濟隂國即濟隂郡濟子禮翻】女五人皆食湯沐邑奏之太后太后乃說為帝加一餐【說讀曰悦為于偽翻】孝王未死時財以巨萬計及死藏府餘黄金尚四十餘萬斤【藏徂浪翻】他物稱是【稱尺證翻】 上既減笞法【見上卷元年】笞者猶不全乃更減笞三百曰二百笞二百曰一百又定箠令【師古曰箠策也所以擊者也箠止蕊翻】箠長五尺【長直亮翻】其本大一寸竹也末薄半寸皆平其節當笞者笞臋【如淳曰然則先時笞背也臋徒門翻】畢一罪乃更人【更工衡翻】自是笞者得全然死刑既重而生刑又輕民易犯之【易以豉翻】 六月匈奴入雁門至武泉入上郡取苑馬【雁門有句注之險如淳曰漢儀注太僕牧師諸苑三十六所分布北邊西邊以郎為苑監官奴婢三萬人養馬三十萬疋師古曰武泉雲中縣也養鳥獸通名曰苑故謂牧馬處曰苑食貨志景帝始造苑馬以廣用】吏卒戰死者二千人隴西李廣為上郡太守嘗從百騎出遇匈奴數千騎見廣以為誘騎【誘騎者見少以誘敵誘音酉下同】皆驚上山陳【師古曰為陳以待廣也陳讀曰陣下同】廣之百騎皆大恐欲馳還走廣曰吾去大軍數十里今如此以百騎走匈奴追射我立盡【射而亦翻下同】今我留匈奴必以我為大軍之誘必不敢擊我廣令諸騎曰前未到匈奴陣二里所止令曰皆下馬解鞍其騎曰虜多且近即有急奈何廣曰彼虜以我為走今皆解鞍以示不走用堅其意【師古曰示以堅牢令敵意知之】於是胡騎遂不敢擊有白馬將出護其兵【師古曰將之乘白馬者也護謂監視之將即亮翻】李廣上馬與十餘騎犇射殺白馬將而復還至其騎中解鞍令士皆縱馬卧是時會暮胡兵終怪之不敢擊夜半時胡兵亦以為漢有伏軍於旁欲夜取之胡皆引兵而去平旦李廣乃歸其大軍秋七月辛亥晦日有食之 自郅都之死長安左右<br />
<br />
  宗室多暴犯法上乃召濟南都尉南陽甯成為中尉其治效郅都其廉弗如然宗室豪傑皆人人惴恐【惴之瑞翻】城陽共王喜薨【共王喜文帝前四年嗣父章爵為王八年徙王淮陽後四年復還城陽至是而薨共讀曰恭】<br />
<br />
  後元年春正月詔曰獄重事也人有智愚官有上下獄疑者讞有司有司所不能決移廷尉讞而後不當讞者不為失【師古曰假令讞訖其理不當所讞之人不為罪失讞魚列翻又魚蹇翻】欲令治獄者務先寛【治直之翻】 三月赦天下 夏大酺五日民得酤酒【中三年禁民酤酒今弛此禁酺音蒲】 五月丙戌地震上庸地震二十二日【班志上庸縣屬漢中郡】壞城垣【壞音怪】 秋七月丙午丞相舍免 乙巳晦日有食之 八月壬辰以御史大夫衛綰為丞相衛尉南陽直不疑為御史大夫【姓譜楚人直弓之後】初不疑為郎同舍有告歸悞持其同舍郎金去已而同舍郎覺亡意不疑【師古曰疑其盜取】不疑謝有之【師古曰告云實取】買金償後告歸者至而歸金亡金郎大慙以此稱為長者稍遷至中大夫人或廷毁不疑【師古曰當廷見之時而毁之】以為盜嫂不疑聞曰我乃無兄然終不自明也 帝居禁中召周亞夫賜食獨置大胾【師古曰胾大臠孔穎達曰熟肉帶骨而臠曰殽純肉而臠曰胾胾側吏翻】無切肉又不置箸亞夫心不平顧謂尚席取箸【應劭曰尚席主席者也】上視而笑曰此非不足君所乎【孟康曰設胾無箸者此非不足滿於君所乎嫌恨之也如淳曰非故不足君之食具偶失之也師古曰孟說近之帝言賜君食而不設箸此由我意於君有不足乎】亞夫免冠謝上上曰起亞夫因趨出上目送之曰此鞅鞅非少主臣也居無何亞夫子為父買工官尚方甲楯五百被可以葬者【少詩沼翻為于偽翻楯食尹翻如淳曰工官官名張晏曰被具也五百具甲楯也師古曰被皮義翻】取庸苦之不與錢【師古曰庸謂賃也苦謂極苦使也余謂亞夫之子無識苦使其人而不與賃錢致其懷怨而禍及其父亞夫之死雖由景帝之少恩其子亦深可罪也】庸知其盜買縣官器怨而上變告子【上時掌翻】事連汙亞夫書既聞上下吏吏簿責亞夫【如淳曰簿問其辭情師古曰簿責者書之於簿一一責問之也汙烏故翻下戶嫁翻】亞夫不對上罵之曰吾不用也【孟康曰言不用汝對欲殺之也如淳曰恐獄吏畏其復用事不敢折辱也師古曰孟說是也一云帝責吏云不勝其任吾不用汝故召亞夫令詣廷尉也】召詣廷尉廷尉責問曰君侯欲反何亞夫曰臣所買器乃葬器也何謂反乎吏曰君縱不欲反地上即欲反地下耳吏侵之益急初吏捕亞夫亞夫欲自殺其夫人止之以故不得死遂入廷尉因不食五日歐血而死 是歲濟隂哀王不識薨【濟子禮翻】<br />
<br />
  二年春正月地一日三動 三月匈奴入鴈門太守馮敬與戰死發車騎材官屯鴈門 春以歲不登禁内郡食馬粟没入之【師古曰食讀曰飤以粟食馬者没其馬入官】 夏四月詔曰雕文刻鏤傷農事者也【鏤力豆翻】錦繡纂組害女工者也【應劭曰纂今五采屬綷是也組今綬紛絛是也臣瓚曰許慎云纂赤組也師古曰瓚說是也綷會也會五采者今謂之錯綵非纂也綷子内翻絛他牢翻】農事傷則飢之本女工害則寒之原也夫飢寒並至而能亡為非者寡矣【亡古無字通】朕親耕后親桑以奉宗廟粢盛祭服為天下先【盛時征翻】不受獻減太官省繇賦【師古曰省所領翻繇讀曰徭】欲天下務農蠶素有蓄積以備災害彊毋攘弱衆毋暴寡老耆以夀終幼孤得遂長【師古曰遂成也長知兩翻】今歲或不登民食頗寡其咎安在或詐偽為吏【張晏曰以詐偽人為吏也臣瓚曰律所謂矯枉以為吏者也師古曰二說並非也直謂詐自稱吏耳】以貨賂為市漁奪百姓侵牟萬民【師古曰漁言若漁獵之為也李奇曰牟食苗根蟲也侵牟食民比之蛑賊也杜佑曰牟取也】縣丞長吏也姦法與盜盜甚無謂也【李斐曰姦法因法作姦也文頴曰與盜謂盜者當治而知情反佐與之是則共盜無異也師古曰與盜盜者共盗為盜耳】其令二千石各脩其職不事官職耗亂者【師古曰耗不明也讀與眊同音莫報翻】丞相以聞請其罪布告天下使明知朕意 五月詔算貲四得官【服䖍曰貲萬錢算百二十七也應劭曰古者疾吏之貪衣食足知榮辱限貲十算乃得為吏十算十萬也賈人有財不得為吏廉士無貲又不得官故減貲四算得官矣】 秋大旱<br />
<br />
  三年冬十月日月皆食赤五日 十二月晦雷日如紫五星逆行守太微【晉天文志太微天子廷也五帝座也十二諸侯府也其外蕃九卿也南蕃中二星間曰端門東曰左執法廷尉象也西曰右執法御史大夫象也左執法之東左掖門也右執法之西右掖門也東蕃四星南第一星曰上相其北東太陽門也第二星曰次相其北中華東門也第三星曰次將其北東太隂門也第四星曰上將所謂四輔也西蕃四星第一星曰上將其北西太陽門也第二星曰次將其北中華西門也第三星曰次相其北西太隂門也第四星曰上相次亦四輔也】月貫天廷中 春正月詔曰農天下之本也黄金珠玉飢不可食寒不可衣以為幣用【師古曰幣者所以通有無易貴賤也】不識其終始間歲或不登意為末者衆農民寡也其令郡國務勸農桑益種樹可得衣食物吏發民若取庸【韋昭曰發民用其民也取庸取其資以顧庸也】采黄金珠玉者坐贓為盜二千石聽者與同罪 甲寅皇太子冠【冠古玩翻】 甲子帝崩于未央宫【臣瓚曰夀四十八】太子即皇帝位年十六尊皇太后為太皇太后皇后為皇太后二月癸酉葬孝景皇帝于陽陵【臣瓚曰自崩及葬凡十日】 三月<br />
<br />
  封皇太后同母弟田蚡為武安侯【班志武安縣屬魏郡又據溝洫志蚡封武安而奉邑食清河之鄃蚡房吻翻】勝為周陽侯【史記正義絳州聞喜縣東二十九里有周陽故城】<br />
<br />
  班固贊曰孔子稱斯民也三代之所以直道而行也【師古曰此論語載孔子之辭也言今此時之人亦夏殷周之所馭以政化淳壹故能直道而行傷今不然】信哉周秦之敝罔密文峻而姦軌不勝【師古曰不可勝】漢興掃除煩苛與民休息至于孝文加之以恭儉孝景遵業五六十載之間至於移風易俗黎民醇厚【師古曰黎衆也醇不澆雜】周云成康漢言文景美矣<br />
<br />
  漢興接秦之弊作業劇而財匱自天子不能具鈞駟【四馬一色謂之鈞駟】而將相或乘牛車【師古曰以牛駕車也余據漢時以牛車為賤魏晉以後王公始多乘牛車】齊民無藏盖【蘇林曰無物可盖藏】天下已平高祖乃令賈人不得衣絲乘車重租税以困辱之【賈音古衣於既翻】孝惠高后時為天下初定復弛商賈之律然市井之子孫亦不得仕宦為吏量吏禄度官用以賦於民【師古曰纔取足量音良度徒洛翻】而山川園池市井租税之入自天子以至於封君湯沐邑皆各為私奉養焉不領于天子之經費【師古曰言各收其所賦税以自供不入於國朝之倉廪府庫也經常也】漕轉山東粟以給中都官【師古曰中都官京師諸官府也】歲不過數十萬石繼以孝文孝景清淨恭儉安養天下七十餘年之間國家無事非遇水旱之災民則人給家足都鄙廩庾皆滿而府庫餘貨財京師之錢累鉅萬貫朽而不可校【師古曰累鉅萬謂數百萬萬也校謂計數也】太倉之粟陳陳相因【師古曰陳謂久舊也】充溢露積於外至腐敗不可食衆庶街巷有馬而阡陌之間成羣【師古曰謂田中之阡陌也】乘字牝者擯而不得聚會【孟康曰皆乘父馬有牝馬間其間則踶齧故斥出不得會同師古曰言時富饒恥乘字牝不必以其踶齧也】守閭閻者食粱肉為吏者長子孫【如淳曰時無事吏不數轉至于生長子孫而不轉職也長知兩翻】居官者以為姓號【如淳曰貨殖傳倉氏庫氏是也】故人人自愛而重犯法先行義而後詘辱焉【師古曰以行義為先以愧辱相絀也行下孟翻】當此之時罔疏而民富役財驕溢或至兼并豪黨之徒以武斷於鄉曲【師古曰恃其豪富則擅行威罰也斷丁亂翻】宗室有土【師古曰謂國之宗姓受封邑土地者也】公卿大夫以下爭於奢侈室廬輿服僭于上無限度物盛而衰固其變也自是之後孝武内窮侈靡外攘夷狄天下蕭然財力耗矣<br />
<br />
  資治通鑑卷十六<br />
<br />
<史部,編年類,資治通鑑>  <br>
   </div> 

<script src="/search/ajaxskft.js"> </script>
 <div class="clear"></div>
<br>
<br>
 <!-- a.d-->

 <!--
<div class="info_share">
</div> 
-->
 <!--info_share--></div>   <!-- end info_content-->
  </div> <!-- end l-->

<div class="r">   <!--r-->



<div class="sidebar"  style="margin-bottom:2px;">

 
<div class="sidebar_title">工具类大全</div>
<div class="sidebar_info">
<strong><a href="http://www.guoxuedashi.com/lsditu/" target="_blank">历史地图</a></strong>  
<a href="http://www.880114.com/" target="_blank">英语宝典</a>  
<a href="http://www.guoxuedashi.com/13jing/" target="_blank">十三经检索</a> 
<br><strong><a href="http://www.guoxuedashi.com/gjtsjc/" target="_blank">古今图书集成</a></strong> 
<a href="http://www.guoxuedashi.com/duilian/" target="_blank">对联大全</a> <strong><a href="http://www.guoxuedashi.com/xiangxingzi/" target="_blank">象形文字典</a></strong> 

<br><a href="http://www.guoxuedashi.com/zixing/yanbian/">字形演变</a>  <strong><a href="http://www.guoxuemi.com/hafo/" target="_blank">哈佛燕京中文善本特藏</a></strong>
<br><strong><a href="http://www.guoxuedashi.com/csfz/" target="_blank">丛书&方志检索器</a></strong> <a href="http://www.guoxuedashi.com/yqjyy/" target="_blank">一切经音义</a>  

<br><strong><a href="http://www.guoxuedashi.com/jiapu/" target="_blank">家谱族谱查询</a></strong>  <strong><a href="http://shufa.guoxuedashi.com/sfzitie/" target="_blank">书法字帖欣赏</a></strong> 
<br>

</div>
</div>


<div class="sidebar" style="margin-bottom:0px;">

<font style="font-size:22px;line-height:32px">QQ交流群9:489193090</font>


<div class="sidebar_title">手机APP 扫描或点击</div>
<div class="sidebar_info">
<table>
<tr>
	<td width=160><a href="http://m.guoxuedashi.com/app/" target="_blank"><img src="/img/gxds-sj.png" width="140"  border="0" alt="国学大师手机版"></a></td>
	<td>
<a href="http://www.guoxuedashi.com/download/" target="_blank">app软件下载专区</a><br>
<a href="http://www.guoxuedashi.com/download/gxds.php" target="_blank">《国学大师》下载</a><br>
<a href="http://www.guoxuedashi.com/download/kxzd.php" target="_blank">《汉字宝典》下载</a><br>
<a href="http://www.guoxuedashi.com/download/scqbd.php" target="_blank">《诗词曲宝典》下载</a><br>
<a href="http://www.guoxuedashi.com/SiKuQuanShu/skqs.php" target="_blank">《四库全书》下载</a><br>
</td>
</tr>
</table>

</div>
</div>


<div class="sidebar2">
<center>


</center>
</div>

<div class="sidebar"  style="margin-bottom:2px;">
<div class="sidebar_title">网站使用教程</div>
<div class="sidebar_info">
<a href="http://www.guoxuedashi.com/help/gjsearch.php" target="_blank">如何在国学大师网下载古籍?</a><br>
<a href="http://www.guoxuedashi.com/zidian/bujian/bjjc.php" target="_blank">如何使用部件查字法快速查字?</a><br>
<a href="http://www.guoxuedashi.com/search/sjc.php" target="_blank">如何在指定的书籍中全文检索?</a><br>
<a href="http://www.guoxuedashi.com/search/skjc.php" target="_blank">如何找到一句话在《四库全书》哪一页?</a><br>
</div>
</div>


<div class="sidebar">
<div class="sidebar_title">热门书籍</div>
<div class="sidebar_info">
<a href="/so.php?sokey=%E8%B5%84%E6%B2%BB%E9%80%9A%E9%89%B4&kt=1">资治通鉴</a> <a href="/24shi/"><strong>二十四史</strong></a>&nbsp; <a href="/a2694/">野史</a>&nbsp; <a href="/SiKuQuanShu/"><strong>四库全书</strong></a>&nbsp;<a href="http://www.guoxuedashi.com/SiKuQuanShu/fanti/">繁体</a>
<br><a href="/so.php?sokey=%E7%BA%A2%E6%A5%BC%E6%A2%A6&kt=1">红楼梦</a> <a href="/a/1858x/">三国演义</a> <a href="/a/1038k/">水浒传</a> <a href="/a/1046t/">西游记</a> <a href="/a/1914o/">封神演义</a>
<br>
<a href="http://www.guoxuedashi.com/so.php?sokeygx=%E4%B8%87%E6%9C%89%E6%96%87%E5%BA%93&submit=&kt=1">万有文库</a> <a href="/a/780t/">古文观止</a> <a href="/a/1024l/">文心雕龙</a> <a href="/a/1704n/">全唐诗</a> <a href="/a/1705h/">全宋词</a>
<br><a href="http://www.guoxuedashi.com/so.php?sokeygx=%E7%99%BE%E8%A1%B2%E6%9C%AC%E4%BA%8C%E5%8D%81%E5%9B%9B%E5%8F%B2&submit=&kt=1"><strong>百衲本二十四史</strong></a>  <a href="http://www.guoxuedashi.com/so.php?sokeygx=%E5%8F%A4%E4%BB%8A%E5%9B%BE%E4%B9%A6%E9%9B%86%E6%88%90&submit=&kt=1"><strong>古今图书集成</strong></a>
<br>

<a href="http://www.guoxuedashi.com/so.php?sokeygx=%E4%B8%9B%E4%B9%A6%E9%9B%86%E6%88%90&submit=&kt=1">丛书集成</a> 
<a href="http://www.guoxuedashi.com/so.php?sokeygx=%E5%9B%9B%E9%83%A8%E4%B8%9B%E5%88%8A&submit=&kt=1"><strong>四部丛刊</strong></a>  
<a href="http://www.guoxuedashi.com/so.php?sokeygx=%E8%AF%B4%E6%96%87%E8%A7%A3%E5%AD%97&submit=&kt=1">說文解字</a> <a href="http://www.guoxuedashi.com/so.php?sokeygx=%E5%85%A8%E4%B8%8A%E5%8F%A4&submit=&kt=1">三国六朝文</a>
<br><a href="http://www.guoxuedashi.com/so.php?sokeytm=%E6%97%A5%E6%9C%AC%E5%86%85%E9%98%81%E6%96%87%E5%BA%93&submit=&kt=1"><strong>日本内阁文库</strong></a> <a href="http://www.guoxuedashi.com/so.php?sokeytm=%E5%9B%BD%E5%9B%BE%E6%96%B9%E5%BF%97%E5%90%88%E9%9B%86&ka=100&submit=">国图方志合集</a> <a href="http://www.guoxuedashi.com/so.php?sokeytm=%E5%90%84%E5%9C%B0%E6%96%B9%E5%BF%97&submit=&kt=1"><strong>各地方志</strong></a>

</div>
</div>


<div class="sidebar2">
<center>

</center>
</div>
<div class="sidebar greenbar">
<div class="sidebar_title green">四库全书</div>
<div class="sidebar_info">

《四库全书》是中国古代最大的丛书,编撰于乾隆年间,由纪昀等360多位高官、学者编撰,3800多人抄写,费时十三年编成。丛书分经、史、子、集四部,故名四库。共有3500多种书,7.9万卷,3.6万册,约8亿字,基本上囊括了古代所有图书,故称“全书”。<a href="http://www.guoxuedashi.com/SiKuQuanShu/">详细>>
</a>

</div> 
</div>

</div>  <!--end r-->

</div>
<!-- 内容区END --> 

<!-- 页脚开始 -->
<div class="shh">

</div>

<div class="w1180" style="margin-top:8px;">
<center><script src="http://www.guoxuedashi.com/img/plus.php?id=3"></script></center>
</div>
<div class="w1180 foot">
<a href="/b/thanks.php">特别致谢</a> | <a href="javascript:window.external.AddFavorite(document.location.href,document.title);">收藏本站</a> | <a href="#">欢迎投稿</a> | <a href="http://www.guoxuedashi.com/forum/">意见建议</a> | <a href="http://www.guoxuemi.com/">国学迷</a> | <a href="http://www.shuowen.net/">说文网</a><script language="javascript" type="text/javascript" src="https://js.users.51.la/17753172.js"></script><br />
  Copyright &copy; 国学大师 古典图书集成 All Rights Reserved.<br>
  
  <span style="font-size:14px">免责声明:本站非营利性站点,以方便网友为主,仅供学习研究。<br>内容由热心网友提供和网上收集,不保留版权。若侵犯了您的权益,来信即刪。scp168@qq.com</span>
  <br />
ICP证:<a href="http://www.beian.miit.gov.cn/" target="_blank">鲁ICP备19060063号</a></div>
<!-- 页脚END --> 
<script src="http://www.guoxuedashi.com/img/plus.php?id=22"></script>
<script src="http://www.guoxuedashi.com/img/tongji.js"></script>

</body>
</html>
