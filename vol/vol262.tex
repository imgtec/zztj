<!DOCTYPE html PUBLIC "-//W3C//DTD XHTML 1.0 Transitional//EN" "http://www.w3.org/TR/xhtml1/DTD/xhtml1-transitional.dtd">
<html xmlns="http://www.w3.org/1999/xhtml">
<head>
<meta http-equiv="Content-Type" content="text/html; charset=utf-8" />
<meta http-equiv="X-UA-Compatible" content="IE=Edge,chrome=1">
<title>資治通鑒_263-資治通鑑卷二百六十二_263-資治通鑑卷二百六十二</title>
<meta name="Keywords" content="資治通鑒_263-資治通鑑卷二百六十二_263-資治通鑑卷二百六十二">
<meta name="Description" content="資治通鑒_263-資治通鑑卷二百六十二_263-資治通鑑卷二百六十二">
<meta http-equiv="Cache-Control" content="no-transform" />
<meta http-equiv="Cache-Control" content="no-siteapp" />
<link href="/img/style.css" rel="stylesheet" type="text/css" />
<script src="/img/m.js?2020"></script> 
</head>
<body>
 <div class="ClassNavi">
<a  href="/24shi/">二十四史</a> | <a href="/SiKuQuanShu/">四库全书</a> | <a href="http://www.guoxuedashi.com/gjtsjc/"><font  color="#FF0000">古今图书集成</font></a> | <a href="/renwu/">历史人物</a> | <a href="/ShuoWenJieZi/"><font  color="#FF0000">说文解字</a></font> | <a href="/chengyu/">成语词典</a> | <a  target="_blank"  href="http://www.guoxuedashi.com/jgwhj/"><font  color="#FF0000">甲骨文合集</font></a> | <a href="/yzjwjc/"><font  color="#FF0000">殷周金文集成</font></a> | <a href="/xiangxingzi/"><font color="#0000FF">象形字典</font></a> | <a href="/13jing/"><font  color="#FF0000">十三经索引</font></a> | <a href="/zixing/"><font  color="#FF0000">字体转换器</font></a> | <a href="/zidian/xz/"><font color="#0000FF">篆书识别</font></a> | <a href="/jinfanyi/">近义反义词</a> | <a href="/duilian/">对联大全</a> | <a href="/jiapu/"><font  color="#0000FF">家谱族谱查询</font></a> | <a href="http://www.guoxuemi.com/hafo/" target="_blank" ><font color="#FF0000">哈佛古籍</font></a> 
</div>

 <!-- 头部导航开始 -->
<div class="w1180 head clearfix">
  <div class="head_logo l"><a title="国学大师官网" href="http://www.guoxuedashi.com" target="_blank"></a></div>
  <div class="head_sr l">
  <div id="head1">
  
  <a href="http://www.guoxuedashi.com/zidian/bujian/" target="_blank" ><img src="http://www.guoxuedashi.com/img/top1.gif" width="88" height="60" border="0" title="部件查字,支持20万汉字"></a>


<a href="http://www.guoxuedashi.com/help/yingpan.php" target="_blank"><img src="http://www.guoxuedashi.com/img/top230.gif" width="600" height="62" border="0" ></a>


  </div>
  <div id="head3"><a href="javascript:" onClick="javascript:window.external.AddFavorite(window.location.href,document.title);">添加收藏</a>
  <br><a href="/help/setie.php">搜索引擎</a>
  <br><a href="/help/zanzhu.php">赞助本站</a></div>
  <div id="head2">
 <a href="http://www.guoxuemi.com/" target="_blank"><img src="http://www.guoxuedashi.com/img/guoxuemi.gif" width="95" height="62" border="0" style="margin-left:2px;" title="国学迷"></a>
  

  </div>
</div>
  <div class="clear"></div>
  <div class="head_nav">
  <p><a href="/">首页</a> | <a href="/ShuKu/">国学书库</a> | <a href="/guji/">影印古籍</a> | <a href="/shici/">诗词宝典</a> | <a   href="/SiKuQuanShu/gxjx.php">精选</a> <b>|</b> <a href="/zidian/">汉语字典</a> | <a href="/hydcd/">汉语词典</a> | <a href="http://www.guoxuedashi.com/zidian/bujian/"><font  color="#CC0066">部件查字</font></a> | <a href="http://www.sfds.cn/"><font  color="#CC0066">书法大师</font></a> | <a href="/jgwhj/">甲骨文</a> <b>|</b> <a href="/b/4/"><font  color="#CC0066">解密</font></a> | <a href="/renwu/">历史人物</a> | <a href="/diangu/">历史典故</a> | <a href="/xingshi/">姓氏</a> | <a href="/minzu/">民族</a> <b>|</b> <a href="/mz/"><font  color="#CC0066">世界名著</font></a> | <a href="/download/">软件下载</a>
</p>
<p><a href="/b/"><font  color="#CC0066">历史</font></a> | <a href="http://skqs.guoxuedashi.com/" target="_blank">四库全书</a> |  <a href="http://www.guoxuedashi.com/search/" target="_blank"><font  color="#CC0066">全文检索</font></a> | <a href="http://www.guoxuedashi.com/shumu/">古籍书目</a> | <a   href="/24shi/">正史</a> <b>|</b> <a href="/chengyu/">成语词典</a> | <a href="/kangxi/" title="康熙字典">康熙字典</a> | <a href="/ShuoWenJieZi/">说文解字</a> | <a href="/zixing/yanbian/">字形演变</a> | <a href="/yzjwjc/">金 文</a> <b>|</b>  <a href="/shijian/nian-hao/">年号</a> | <a href="/diming/">历史地名</a> | <a href="/shijian/">历史事件</a> | <a href="/guanzhi/">官职</a> | <a href="/lishi/">知识</a> <b>|</b> <a href="/zhongyi/">中医中药</a> | <a href="http://www.guoxuedashi.com/forum/">留言反馈</a>
</p>
  </div>
</div>
<!-- 头部导航END --> 
<!-- 内容区开始 --> 
<div class="w1180 clearfix">
  <div class="info l">
   
<div class="clearfix" style="background:#f5faff;">
<script src='http://www.guoxuedashi.com/img/headersou.js'></script>

</div>
  <div class="info_tree"><a href="http://www.guoxuedashi.com">首页</a> > <a href="/SiKuQuanShu/fanti/">四库全书</a>
 > <h1>资治通鉴</h1> <!--         下载:【右键另存为】即可 --></div>
  <div class="info_content zj clearfix">
  
<div class="info_txt clearfix" id="show">
<center style="font-size:24px;">263-資治通鑑卷二百六十二</center>
    資治通鑑卷二百六十二 宋 司馬光 撰<br />
<br />
  胡三省 音註<br />
<br />
  唐紀七十八【起上章涒灘盡重光作噩凡二年】<br />
<br />
  昭宗聖穆景文孝皇帝中之中<br />
<br />
  光化三年春正月宣州將康儒攻睦州【宣州將田頵所遣將也】錢鏐使其從弟銶拒之【從才用翻銶音求】 二月庚申以西川節度使王建兼中書令 壬申加威武節度使王審知同平章事 壬午以吏部尚書崔胤同平章事充清海節度使 李克用大發軍民治晉陽城塹【愳朱全忠之攻逼也治直之翻】押牙劉延業諫曰大王聲振華夷宜揚兵以嚴四境不宜近治城塹損威望而啓寇心克用謝之賞以金帛夏四月加定難軍節度使李承慶同平章事【難乃旦翻】 朱全忠遣葛從周帥兖鄆滑魏四鎮兵十萬擊劉仁恭【帥讀曰率】五月庚寅拔德州斬刺史傅公和己亥圍劉守文於滄州仁恭復遣使卑辭厚禮求援於河東【復扶又翻】李克用遣周德威將五千騎出黄澤攻邢洺以救之【黄澤關在遼州遼山縣黄澤嶺】 邕州軍亂逐節度使李鐬【懿宗咸通三年升邕管經畧使為嶺南西道節度使鐬呼會翻】鐬借兵鄰道討平之 六月癸亥加東川節度使王宗滌同平章事司空門下侍郎同平章事王摶明達有度量時稱良相【以其時言之稱為良相所謂彼善於此也】上素疾宦官樞密使宋道弼景務修專横【横戶孟翻】崔胤日與上謀去宦官【去羌呂翻】宦官知之由是南北司益相憎嫉各結藩鎮為援以相傾奪摶恐其致亂從容言於上曰【從千容翻】人君當務明大體無所偏私宦官擅權之弊誰不知之顧其埶未可猝除宜俟多難漸平以道消息【難乃旦翻以道消息者言惡者以漸殺其埶則久而自消善者以漸培其根則久而自長】願陛下言勿輕泄以速姦變胤聞之譛摶於上曰王摶姦邪已為道弼輩外應上疑之及胤罷相【去年胤罷相見上卷】意摶排已愈恨之及出鎮廣州遺朱全忠書具道摶語【是年二月出胤廣州摶語即從容言於上者遺唯季翻】令全忠表論之全忠上言胤不可離輔弼之地【上時掌翻下連上同離力智翻】摶與敕使相表裏同危社稷表連上不已上雖察其情迫於全忠不得已胤至湖南復召還【復扶又翻】丁卯以胤為司空門下侍郎同平章事摶罷為工部侍郎以道弼監荆南軍務修監青州軍【監古衘翻】戊辰貶摶溪州刺史己巳又貶崖州司戶道弼長流驩州務修長流愛州是日皆賜自盡摶死於藍田驛道弼務修死於霸橋驛【藍田驛在藍田縣霸橋驛在長安城南近霸橋】於是胤專制朝政埶震中外【朝直遥翻】宦官皆側目不勝其憤【為劉季述韓全誨之亂張本勝音升】 劉仁恭將幽州兵五萬救滄州營於乾寧軍【乾寜軍在滄州西一百里蓋乾寜間始置此軍也宋白曰乾寜軍本古盧臺軍地後為馮橋鎮臨御河之岸接滄幽二州之界周顯德六年收復關南始建為乾寧軍九域志云太平興國七年始置軍】葛從周留張存敬氏叔琮守滄州寨自將精兵逆戰於老鵶堤【老鵶堤在乾寧軍東南】大破仁恭斬首三萬級仁恭走保瓦橋【瓦橋在涿州歸義縣南至莫州三十里宋白曰瓦橋亦謂之瓦子濟橋在涿州南易州東周顯德收復二關以其地控幽薊建為雄州】秋七月李克用復遣都指揮使李嗣昭將兵五萬攻邢洺以救仁恭敗汴軍於内丘【復扶又翻敗補邁翻下同范成大北使録内丘縣至邢州三十五里 考異曰唐太祖紀年録七月嗣昭攻堯山至内丘遇汴軍三千戰敗之擒其將李瓌薛居正五代史後唐紀與紀年録同惟唐末見聞録八月二十五日嗣昭領馬步五萬取馬嶺進軍下山東某日山東告捷收得洺州九月二日嗣昭兵士失利却回新紀八月庚辰陷洺州薛史唐紀九月嗣昭弃城歸蓋據此也按編遺録八月中云前月二十五日上於毬場饗士忽有大風占者云賊風果於是時李進通領蕃寇出攻洺州然則嗣昭出兵乃七月二十五日也編遺録又曰八月乙丑出兵救洺州乙丑九日也又進通敗奔歸太原在八月見聞録誤今從編遺録紀年録梁紀】王鎔遣使和解幽汴會久雨朱全忠召從周還【滄州下濕雨水難以駐軍且欲救邢洺故召還】 庚戌以昭義留後孟遷為節度使 甲寅以西川節度使王建兼東川信武軍兩道都指揮制置等使【時置武信軍於遂州信武當作武信王建兼指揮制置兩道則可以制宗滌宗佶蓋諷朝廷以此命之】 八月李嗣昭又敗汴軍于沙門河【沙門河疑當作沙河即邢州沙河縣也 考異曰編遺録七月二十五日李進通領蕃寇出并州來攻洺州八月乙丑發大軍救應之上尋亦自領衙軍相繼北征翌日達滑臺軍前馳報洺州已陷刺史朱紹宗因踰堞墮而傷足為賊所擒唐太祖紀年録八月李嗣昭又遇汴軍於沙門河擊而敗之進攻洺州刺史朱紹宗挈其族夜遁我師追及擒之唐末見聞録八月二十五日嗣昭進軍下山東某日山東告捷收得洺州捉得刺史朱温姪男舊紀八月庚辰嗣昭攻洺州下之薛史梁紀八月河東遣李進通襲陷洺州新紀亦在庚辰乃二十五日也實録在九月約奏到今從編遺録】進攻洺州乙丑朱全忠引兵救之未至嗣昭拔洺州擒刺史朱紹宗全忠命葛從周將兵擊嗣昭 宣州將康儒食盡自清溪遁歸【康儒是年正月攻睦州清溪漢歙縣地後分置新安縣隋改為雉山文明元年復為新安開元二十年改為還淳永貞元年避憲宗名改曰清溪屬睦州九域志縣在州西一百六十六里】 九月葛從周自鄴縣度漳水營於黄龍鎮朱全忠自將中軍三萬涉洺水置營李嗣昭弃城走【弃洺州城而走】從周設伏於青山口邀擊大破之【考異曰唐太祖紀年録葛從周攻洺州嗣昭弃城而歸是役也王郃郎楊師悦陷賊洺州復為汴有唐末見】<br />
<br />
  【聞録九月二日嗣昭兵士失利却回被汴州捉到王郃郎編遺録薛居正五代史梁紀八月帝遣葛從周屯黄龍鎮親領中軍涉洺而寨晉人愳而宵遁洺州復平唐紀九月汴帥自將兵三萬圍洺州嗣昭弃城而歸葛從周伏青山口嗣昭軍不利實録九月嗣昭奔洺州敗於青山口今從唐末見聞録唐紀實録又按考異所録唐紀蓋後唐紀】 崔胤以太保門下侍郎同平章事徐彦若位在已上惡之【惡烏路翻】彦若亦自求引去【徐彦若可謂知遥增擊而去之之意者】時藩鎮皆為彊臣所據惟嗣薛王知柔在廣州【知柔鎮廣州見二百六十卷乾寧元年】乃求代之乙巳以彦若同平章事充清海節度使初荆南節度成汭以澧朗本其巡屬為雷滿所據【肅宗至德二載置荆南節度領荆澧朗郢復夔峽忠萬十州其後增領分隸不一自雷滿據澧朗又分置武貞軍節度】屢求割隸荆南朝廷不許汭頗怨望【薛史曰汭奏請割隸彦若為相執不行汭由是衘之】及彦若過荆南汭置酒從容以為言【從千容翻】彦若曰令公位尊方面自比桓文【成汭進中書令故稱之為令公】雷滿小盜不能取乃怨朝廷乎汭甚慙 丙午中書侍郎兼吏部尚書同平章事崔遠罷守本官以刑部尚書裴贄為中書侍郎同平章事贄坦之弟子也【裴坦見二百五十一卷懿宗咸通十年】升桂管為靜江軍以經略使劉士政為節度使 朱全忠以王鎔與李克用交通移兵伐之【自洺州移兵伐趙】下臨城踰滹沱攻鎮州南門焚其關城全忠自至元氏鎔懼遣判官周式詣全忠請和全忠盛怒謂式曰僕屢以書諭王公竟不之聽今兵已至此期於無捨式曰鎮州密邇太原【鎮州與太原僅隔山耳九域志鎮州西距太原四百三十里】困於侵暴【李克用自得河東以來屢攻趙】四鄰各自保莫相救恤王公與之連和乃為百姓故也【為于偽翻下為人為之同】今明公果能為人除害則天下誰不聽命豈惟鎮州明公為唐桓文當崇禮義以成霸業若但窮威武則鎮州雖小城堅食足明公雖有十萬之衆未易攻也况王氏秉旄五代【庭湊元逵紹鼎紹懿景崇及鎔為五世蓋紹鼎紹懿兄弟也共為一世】時推忠孝人欲為之死庸可冀乎全忠笑攬式袂延之帳中曰與公戲耳【周式之說朱全忠猶屈完之說齊桓公也而當時汴鎮攻守之勢誠亦如此全忠易怒為笑而延之以其言中其要害也】乃遣客將開封劉捍入見鎔【客將主賓客掌通名贊謁】鎔以其子節度副使昭祚及大將子弟為質【質音致】以文繒二十萬犒軍【文繒絹之有文者今謂之花絹】全忠引還以女妻昭祚【還從宣翻又如字妻七細翻】成德判官張澤言於王鎔曰河東勍敵也【勍車京翻】今雖有朱氏之援譬如火發於家安能俟遠水乎彼幽滄易定猶附河東不若說朱公乘勝兼服之【幽劉仁恭滄劉守文易定王郜說式芮翻下同】使河北諸鎮合而為一則可以制河東矣鎔復遣周式往說全忠【復扶又翻】全忠喜遣張存敬會魏博兵擊劉仁恭甲寅拔瀛州冬十月丙辰拔景州執刺史劉仁霸辛酉拔莫州 靜江節度使劉士政聞馬殷悉平嶺北【湖南之地在五嶺之北】大懼遣副使陳可璠屯全義嶺以備之【璠孚袁翻武德四年分始安置臨源縣大歷三年更名全義屬桂州國朝改全義為興安縣在桂州東北一百五十里】殷遣使修好於士政【好呼到翻】可璠拒之殷遣其將秦彦暉李瓊等將兵七千擊士政湖南軍至全義士政又遣指揮使王建武屯秦城【范成大桂海虞衡志曰秦城在桂林城北八十里相傳以為始皇發戍五嶺之地城在湘水之南瀜灕二水之間遺址尚存石甃亦無恙城北二十里有嚴關羣山環之鳥道不可方軌秦取百粤以其地為桂林象郡而戍兵乃止湘南蓋嶺有喉衿在是稍南又不可以宿兵也】可璠掠縣民耕牛以犒軍縣民怨之請為湖南鄉導【犒苦到翻鄉讀曰嚮】曰此西南有小徑距秦城纔五十里僅通單騎彦暉遣李瓊將騎六十步兵三百襲秦城中宵踰垣而入擒王建武比明復還之以練造可璠壁下示之【比必利翻及也充夜翻縶縛也造七到翻】可璠猶未之信斬其首投壁中桂人震恐瓊因勒兵擊之擒可璠降其將士二千皆殺之引兵趣桂州【趣七喻翻下同】自秦城以南二十餘壁皆望風奔潰遂圍桂州數日士政出降【乾寜二年劉士政襲據桂州至是而敗】桂宜巖柳象五州【宜州之地秦屬象郡漢屬交趾日南二郡界後没于蠻唐初開置粤州乾封中更曰宜州】皆降於湖南【馬殷又兼有桂管 考異曰唐烈祖實録新唐書本紀路振九國志楚世家皆云光化二年殷克桂州馬氏行年記及王舉大定録云天復元年惟曹衍湖湘馬氐故事云天復甲子宣晟自安州入桂州天祐四年丁卯十二月收嶺北七州明年十月平桂州差繆極甚新唐書方鎮表光化三年升桂管經畧使為靜江節度使而本紀乾寧二年安州防禦使宣晟陷桂州靜江軍節度使周元靜部將劉士政死之歲月既已倒錯又以士政為元靜部將同死尤為乖誤今据武安節度掌書記林崇禧撰武威王廟碑云我王臨位五歲而桂林歸款自乾寧三年至光化三年五年矣又與實録合故從之】馬殷以李瓊為桂州刺史未幾【幾居豈翻】表為靜江節度使 張存敬攻劉仁恭下二十城將自瓦橋趣幽州道濘不能進【濘乃定翻泥淖也】乃引兵西攻易定辛巳拔祁州【景福二年王處存表以定州無極深澤二縣置祁州】殺刺史楊約 癸未以保義留後朱友謙為節度使【朱全忠請之也】 張存敬攻定州義武節度使王郜【郜居號翻】遣後院都知兵馬使王處直將兵數萬拒之【唐中世以來方鎮多置後院兵處昌呂翻】處直請依城為柵俟其師老而擊之孔目官梁汶曰昔幽鎮兵三十萬攻我【汶音問薛史作問僖宗光啟元年幽州李可舉鎮州王鎔攻王處存事見二百五十六卷】于時我軍不滿五千一戰敗之【敗補邁翻】今存敬兵不過三萬我軍十倍於昔奈何示怯欲依城自固乎郜乃遣處直逆戰于沙河【沙河在新城北望都縣南】易定兵大敗死者過半餘衆擁處直奔還甲申王郜弃城奔晉陽【王處存素睦於晉又昏姻也故郜奔之】軍中推處直為留後存敬進圍定州丙申朱全忠至城下處直登城呼曰【呼火故翻】本道事朝廷甚忠【義武自張存忠以來事朝廷最為忠順】於公未嘗相犯何為見攻全忠曰何故附河東對曰吾兄與晉王同時立勲【謂王處存與李克用同平黄巢立功】封疆密邇【自定州出飛狐即河東之境】且昏姻也修好往來乃常理耳【好呼到翻】請從此改圖全忠許之【定州城池高深朱全忠知不可猝攻而拔故許其和】乃歸辠於梁汶而族之以謝全忠以繒帛十萬犒師全忠乃還仍為處直表求節钺【為于偽翻】處直處存之母弟也劉仁恭遣其子守光將兵救定州軍於易水之上【易水在易州遂城縣界遂城縣於宋為安肅軍昔燕太子丹送荆軻於易水之上即此地】全忠遣張存敬襲之殺六萬餘人由是河北諸鎮皆服於全忠【史言河北諸鎮皆覊服於全忠全忠不能并有其地也】先是王郜告急於河東【先悉薦翻】李克用遣李嗣昭將步騎三萬下太行攻懷州拔之【行戶剛翻】進攻河陽河陽留後侯言不意其至狼狽失據嗣昭壞其羊馬城【壞音怪城外别立短垣以屏蔽謂之羊馬城】會佑國軍將閻寶引兵救之【河南府佑國軍東北至河陽八十五里】力戰於壕外河東兵乃退寶鄆州人也 初崔胤與帝密謀盡誅宦官及宋道弼景務修死【事見上六月】宦官益懼上自華州還【光化元年上還自華州事見上卷還從宣翻又如字】忽忽不樂【樂音洛】多縱酒喜怒不常左右尤自危於是左軍中尉劉季述右軍中尉王仲先樞密使王彦範薛齊偓等隂相與謀曰主上輕佻多變詐難奉事【佻土彫翻】專聽任南司【時宦官謂之北司謂南牙百官為南司】吾輩終罹其禍不若奉太子立之尊主上為太上皇引岐華兵為援【岐李茂貞華韓建華戶化翻】控制諸藩誰能害我哉十一月上獵苑中【禁苑在宫城北】因置酒夜醉歸手殺黄門侍女數人明旦日加辰巳宫門不開季述詣中書白崔胤曰宫中必有變我内臣也得以便宜從事請入視之乃帥禁兵千人破門而入【帥讀曰率】訪問具得其狀出謂胤曰主上所為如是豈可理天下廢昏立明自古有之為社稷大計非不順也胤畏死不敢違庚寅季述召百官陳兵殿庭【陳兵以脅百官也】作胤等連名狀請太子監國以示之使署名胤及百官不得已皆署之【監古衘翻】上在乞巧樓【按劉季述傳乞巧樓在思玄門内近思政殿】季述仲先伏甲士千人於門外【即宣化門外】與宣武進奏官程巖等十餘人入請對季述仲先甫登殿將士大呼【呼火故翻】突入宣化門至思政殿前逢宫人輒殺之上見兵入驚墮床下起將走季述仲先掖之令坐宫人走白皇后后趨至拜請曰軍容勿驚宅家有事取軍容商量【量音良今人謂議事為商量】季述等乃出百官狀白上曰陛下厭倦大寶中外羣情願太子監國請陛下保頤東宫【頤養也言於少陽院自保養也】上曰昨與卿曹樂飲不覺太過【樂音洛】何至於是對曰此非臣等所為皆南司衆情不可遏也願陛下且之東宫【之往也】待事小定復迎歸大内耳后曰宅家趣依軍容語【趣讀曰促】即取傳國寶以授季述宦官扶上與后同輦嬪御侍從者纔十餘人【從才用翻】適少陽院季述以銀檛畫地數上曰【檛側加翻數所具翻俗從上聲】某時某事汝不從我言其罪一也如此數十不止【歷數之至數十不止】乃手鎖其門鎔鐵錮之【錮音固】遣左軍副使李師䖍將兵圍之上動靜輒白季述穴牆以通飲食凡兵器針刀皆不得入上求錢帛俱不得求紙筆亦不與時大寒嬪御公主無衣衾號哭聞於外【號戶刀翻聞音問】季述等矯詔令太子監國迎太子入宫 【考異曰按此月乙酉朔己丑五日庚寅六日也廢立之日舊紀云庚寅舊宦者傳唐年補紀皆云六日無云五日者而實録新紀云己丑誤也唐太祖紀年録先云六日後云七日尤誤也崔胤所恃者昭宗耳季述議廢立安肯即從之補録紀年録言脅之以兵是也唐補紀云皇后穴牆取太子又云令旨宣告大臣與社稷為主又云后白軍容令聖上養疾皆程匡柔為宦者諱耳不可信也】辛卯矯詔令太子嗣位更名縝【更工衡翻下同縝止忍翻】以上為太上皇皇后為太上皇后甲午太子即皇帝位更名少陽院曰問安宫季述加百官爵秩與將士皆受優賞欲以求媚於衆殺睦王倚【倚上弟也】凡宫人左右方士僧道為上所寵信者皆榜殺之【榜音彭】每夜殺人晝以十車載尸出一車或止一兩尸欲以立威將殺司天監胡秀林【武德四年改太史監曰太史局有令有丞高宗龍朔二年改太史局曰祕書閣局令曰祕書閻郎中武后光宅元年改太史局曰渾天監俄改曰渾儀監長安二年復曰太史局中宗景龍二年改太史局曰太史監乾元元年改曰司天臺置監一人正三品掌察天文稽歷數】秀林曰軍容幽囚君父更欲多殺無辜乎季述憚其言正而止季述欲殺崔胤而憚朱全忠但解其度支鹽鐵轉運使而已 【考異曰舊傳劉季述畏朱全忠之強不敢殺崔胤但罷知政事落使務守本官而已胤復致書於全忠請出師返正故全忠令張存敬急攻晉絳河中按舊紀新紀新宰相表此際皆無胤罷相事全忠攻晉絳河中乃在明年返正後今不取】左僕射致仕張濬在長水【乾寧三年上復欲相張濬以李克用言而止濬遂致仕居長水宋白曰長水本漢盧氏縣地後魏延昌二年分盧氏東境庫谷已西沙渠谷已東為南陜縣北有陜縣故名南陜廢帝元年改為長淵以縣東洛水長淵為名唐以犯唐祖諱改名長水九域志在河南府西二百四十里】見張全義於洛陽勸之匡復又與諸藩鎮書勸之進士無棣李愚客華州上韓建書畧曰僕每讀書見父子君臣之際有傷教害義者恨不得肆之市朝【上時掌翻朝直遥翻下並同】明公居近關重鎮【蓋謂華州控扼潼關距關為近】君父幽辱月餘坐視凶逆而忘勤王之舉僕所未諭也僕竊計中朝輔弼雖有志而無權外鎮諸侯雖有權而無志惟明公忠義社禝是依往年車輅播遷號泣奉迎累歲供饋再復廟朝【謂乾寧三年迎上駐蹕華州光化元年歸長安也廟朝謂宗廟朝廷也號戶刀翻】義感人心至今謌詠此時事勢尤異前日明公地處要衝【處昌呂翻】位兼將相自宫闈變故已涉旬時【旬時即旬日也】若不號令率先以圖反正遲疑未決一朝山東侯伯唱義連衡【衡讀曰横】鼓行而西明公求欲自安其可得乎【言山東勤王之師若至華州韓建亦不得安其位矣其後朱全忠攻岐遂徙建許州卒如李愚之言】此必然之勢也不如馳檄四方諭以逆順軍聲一振則元凶破膽旬浹之間二豎之首傳於天下【旬浹謂一日二日至于十日浹即恊翻二竪謂劉季述王仲先】計無便於此者建雖不能用厚待之愚堅辭而去朱全忠在定州行營聞亂丁未南還十二月戊辰至大梁季述遣養子希度詣全忠許以唐社稷輸之又遣供奉官李奉本以太上皇誥示全忠【劉季述矯為之誥也】全忠猶豫未決會僚佐議之或曰朝廷大事非藩鎮所宜預知天平節度副使李振獨曰王室有難【難乃旦翻】此霸者之資也今公為唐桓文安危所屬【李振以齊桓晉文諂朱全忠屬之欲翻】季述一宦竪耳乃敢囚廢天子公不能討何以復令諸侯【復扶又翻】且幼主位定則天下之權盡歸宦官矣是以太阿之柄授人也全忠大悟即囚希度奉本遣振如京師詗事【詗火迥翻又翾正翻】既還又遣親吏蔣玄暉如京師與崔胤謀之又召程巖赴大梁 【考異曰薛居正五代史李振十一月太祖遣振入奏於長安邸吏程巖白振曰劉中尉命其姪希貞來計大事既至巖乃先啓曰主上嚴急内官憂恐左中尉欲行廢黜敢以事告振顧希貞曰百歲奴事三歲主亂國不義廢君不祥非敢聞也况梁王以百萬之師匡輔天子幸熟計之希貞大沮而去振復命劉季述果作亂程巖率諸道邸吏牽帝下殿以立幼主振至陜陜已賀矣護軍韓彛範言其事振曰懿皇初昇遐韓中尉殺長立幼以利其權遂亂天下今將軍復欲爾邪彛範即文約孫也由是不敢言編遺録上雖聞其事未知摭實但懷憤激丁未上離定州軍前十二月戊辰達大梁欲潜謀返正乃遣李振偵視其事振迴益詳其宜也尋馳蔣玄暉與崔胤密圖大義薛史梁紀季述幽昭宗立德王裕為帝仍遣其養子希度來言為以唐之神器輸於帝時帝方在河朔聞之遽還于汴大計未決會李振自長安使回因言於帝云云帝悟因請振復使于長安與時宰潜謀返正按季述廢立之前李振若已嘗立異今豈敢復入長安與崔胤謀返正乎今從編遺録注曰貞明中史臣李琪張衮郗殷象馮錫嘉修撰太祖實録事多漏略敬翔别纂成三十卷補其闕號曰大梁編遺録又按唐太祖紀年録及舊張濬傳皆云濬勸諸藩匡復而梁實録及李振傳皆云濬勸全忠附中官與紀年録及舊傳相違恐梁實録誤振傳据實録也唐補紀曰自監國居位將及五旬牋表不來朝野驚虞亢旱時多虹蜺背璚崔胤覩其不祥便謀内變潜行書檄於關外播楊辭舌於街衢朱全忠封崔胤檄書併手扎等與季述云彼已翻覆早宜别圖無何季述以此書示于崔胤曰比來同匡社稷却為鬭亂藩方不審相公何至於此胤唯云無此事遭人反圖刻蠟偽名自古乃有軍容若行怪怒則乞俯存家族季述乃與言誓相保始終胤其夜便致書謝全忠云昨以丹誠諮撓尊聽却蒙封示左軍劉公其人已知意旨今日與胤設盟不相損害然遠託令公為主方應保全兼送女僕二人細馬兩匹全忠覧書大詬曰劉季述我與伊同王事十二三年兄弟之故特令報渠不能自謀却示崔相道我兩頭三面直是難容我若不殺此公不姓朱也乃擲於地囚其使者走一健步直申崔公從兹與大梁同謀大事按崔胤曏來内倚昭宗外挾全忠與宦官為敵今昭宗既廢胤所以得未死者以與全忠親密故也全忠安肯以其書示季述季述恨胤深入骨髓若得此書立當殺胤豈肯復以示胤而與之盟誓也此殊不近人情皆由程匡柔黨宦官疾胤之亂耳】 清海節度使薛王知柔薨 是歲加楊行密兼侍中 睦州刺史陳晟卒弟詢自稱刺史 太子即位累旬藩鎮牋表多不至王仲先性苛察素知左右軍多積弊及為中尉鉤校軍中錢穀得隱沒為姦者痛捶之【捶止橤翻】急徵所負將士頗不安有鹽州雄毅軍使孫德昭為左神策指揮使自劉季述廢立常憤惋不平【惋烏貫翻】崔胤聞之遣判官石戩與之遊【判官度支鹽鐵判官也戩即棧翻】德昭每酒酣必泣戩知其誠乃密以胤意說之曰自上皇幽閉中外大臣至于行間士卒孰不切齒【說式芮翻行戶剛翻】今反者獨季述仲先耳公誠能誅此二人迎上皇復位則富貴窮一時忠義流千古苟狐疑不決則功落它人之手矣德昭謝曰德昭小校【校戶教翻】國家大事安敢專之苟相公有命不敢愛死戩以白胤胤割衣帶手書以授之德昭復結右軍清遠都將董彦弼周承誨【清遠都亦神策五十四都之一復扶又翻】謀以除夜伏兵安福門外以俟之<br />
<br />
  天復元年【是年四月方改元】春正月乙酉朔王仲先入朝至安福門孫德昭擒斬之馳詣少陽院叩門呼曰【呼火故翻】逆賊已誅請陛下出勞將士【勞力到翻】何后不信曰果爾以其首來德昭獻其首上乃與后毁扉而出【扉門扇也】崔胤迎上御長樂門樓【新書儀衛志太極宫端門曰承天門承天門分為東西廊下門自東廊下入長樂門自西廊下入永安門凡朝會之仗門内各有挾門隊樂音洛】帥百官稱賀【帥讀曰率】周承誨擒劉季述王彦範繼至方詰責已為亂梃所斃【詰去吉翻梃徒鼎翻】薛齊偓赴井死出而斬之滅四人之族并誅其黨二十餘人宦官奉太子匿于左軍獻傳國寶上曰裕幼弱為凶豎所立非其罪也命還東宫黜為德王復名裕【裕之為宦官所立也更名縝今復其舊名】丙戌以孫德昭同平章事充靜海節度使【靜海軍安南孫德昭遥領也】賜姓名李繼昭丁亥崔胤進位司徒胤固辭上寵待胤益厚己丑朱全忠聞劉季述等誅折程巖足【折而設翻薛史梁紀曰昭宗之廢也汴之邸吏程巖牽昭宗衣下殿帝召巖至汴折其足至長安殺之】械送京師并劉希度李奉本等皆斬於都市由是益重李振【李振請誅劉季述等見上】庚寅以周承誨為嶺南西道節度使賜姓名李繼誨董彦弼為寧遠節度賜姓李並同平章事與李繼昭俱留宿衛十日乃出還家【即句休之制也】賞賜傾府庫時人謂之三使相【未幾周承誨董彦弼復朋比宦官獨孫德昭不肯爾】癸巳進朱全忠爵東平王 【考異曰舊紀二月以全忠守中書令進封梁王薛居正五代史梁紀正月癸巳進封帝為梁王酬返正之功也實録癸巳沛郡王朱全忠加定謀宣力功臣進封東平王新紀二月辛未封全忠為梁王按編遺録此年二月辛未表讓梁王三年二月制云兔苑名邦睢陽奥壤光膺簡冊大啓封疆可守太尉中書令進封梁王或者今年已曾封梁王全忠讓不受改封東平王至三年乃進封梁王而三年制辭前官爵已稱梁王蓋誤也今從實録】 丙午敕近年宰臣延英奏事樞密使侍側爭論紛然既出又稱上旨未允復有改易橈權亂政【復扶又翻橈奴教翻或奴巧翻】自今並依大中舊制俟宰臣奏事畢方得升殿承受公事【大中故事凡宰相對延英兩中尉先降樞密使侯旨殿西宰相奏事已畢樞密使案前受事】賜兩軍副使李師度徐彦孫自盡皆劉季述之黨也 鳳翔彰義節度使李茂貞來朝加茂貞守尚書令【唐自太宗以尚書令即阼不復授人郭子儀有大功雖授之而不敢受王行瑜怙強力雖求之而終不獲蓋君臣上下猶知守先朝之法也今以授李茂貞唐法蕩然於此極矣】兼侍中進爵岐王劉季述王仲先既死崔胤陸扆上言【上時掌翻】禍亂之興皆由中官典兵乞令胤主左軍扆主右軍則諸侯不敢侵陵王室尊矣上猶豫兩日未決李茂貞聞之怒曰崔胤奪軍權未得已欲翦滅諸侯上召李繼昭李繼誨李彦弼謀之皆曰臣等累世在軍中未聞書生為軍主若屬南司必多所變更【更工衡翻】不若歸之北司為便上乃謂胤扆曰將士意不欲屬文臣卿曹勿堅求於是以樞密使韓全誨鳳翔監軍使張彦弘為左右中尉全誨亦前鳳翔監軍也【為韓全誨刧上幸鳳翔張本】又徵前樞密使致仕嚴遵美為兩軍中尉觀軍容處置使遵美曰一軍猶不可為况兩軍乎【按新書宦者傳嚴遵美嘗歷左神策觀軍容使故云然處昌呂翻】固辭不起以袁易簡周敬容為樞密使李茂貞辭還鎮崔胤以宦官典兵終為肘腋之患欲以外兵制之諷茂貞留兵三千于京師充宿衛以茂貞假子繼筠將之左諫議大夫萬年韓偓以為不可胤曰兵自不肯去非留之也偓曰始者何為召之邪胤無以應【新書韓偓傳胤召李茂貞入朝使留族子繼筠宿衛故斥言之而胤無以應偓於角翻】偓曰留此兵則家國兩危不留則家國兩安胤不從【李繼筠卒與宦官劫帝幸鳳翔 考異曰唐補紀曰其月八日李茂貞朝覲留二千人在右街侍衛而回崔胤申朱全忠請三千人在南坊宅側安下鳳翔劫駕西去朱全忠又闇以車子載器仗稱是紬絹進奉推車子人皆是官健入崔胤宅中人心驚惶不同前後崔胤累差人喚召朱全忠不到新傳韓全誨等知崔胤必除已乃已因諷茂貞留選士四千宿衛以李繼徽總之胤亦諷朱全忠納兵二千居南司以婁敬思領之盖取唐補紀耳按韓偓金鑾密記偓對昭宗云當留兵之時臣五六度與崔胤力爭胤曰某實不留兵是兵不肯去臣曰其初何用召來又胤云且喜岐兵只留三千人據此則是胤召茂貞入朝仍留其兵也又舊紀梁實録編遺録薛居正五代史梁紀等諸書皆不言全忠嘗遣兵宿衛京師若如唐補紀所言岐汴各遣兵數千人戍京師則昭宗欲西幸時兩道兵必先鬭於闕下不則汴兵皆為宦官所誅不則先遁去今皆無此事蓋程匡柔得於傳聞又黨於宦官深疾崔胤未足信也然胤所以欲留茂貞兵為已援者蓋以茂頁自以誅劉季述為已功必能與已同心讐疾宦官以利誘之遂復與宦官為一耳今從金鑾記】 朱全忠既服河北欲先取河中以制河東己亥召諸將謂曰王珂駑材恃太原自驕汰【駑音奴王珂恃李克用翁壻之親而不事朱全忠故云然】吾今斷長蛇之腰諸君為我以一繩縛之【言河東河中兩鎮連衡以通長安今若取河中是斷李克用之腰也斷丁管翻為于偽翻】庚子遣張存敬將兵三萬自汜水度河出含山路以襲之【含山在絳州東張濬之敗也出含口至河陽度河西歸即此路】全忠以中軍繼其後戊申存敬至絳州晉絳不意其至皆無守備庚戍絳州刺史陶建釗降之【釗音昭降戶江翻】壬子晉州刺史張漢瑜降之全忠遣其將侯言守晉州何絪守絳州【絪音因】屯兵二萬以扼河東援兵之路朝廷恐全忠西入關急賜詔和解之全忠不從珂遣間使告急于李克用道路相繼【間古莧翻】克用以汴兵先據晉絳兵不得進【九域志太原西南二百六十里至汾州汾州南三百五十里至晉州晉州南百二十五里至絳州絳州西南六十五里至河中府援兵擇便利投間隙而行固不盡由驛道但汴兵已屯晉絳以塞其衝并兵縱由捷徑得進汴兵遮前險守後要進不得援河中退不得歸太原也】珂妻遺李克用書曰【遺唯季翻下又遺同】兒旦暮為俘虜大人何忍不救克用報曰今賊兵塞晉絳【塞悉則翻】衆寡不敵進則與汝兩亡不若與王郎舉族歸朝【自晉以來婦翁皆呼壻為郎迨今猶然】珂又遺李茂貞書言天子新返正詔藩鎮無得相攻同奬王室今朱公不顧詔命首興兵相加其心可見河中若亡則同華邠岐俱不自保【同華韓建邠李茂貞養子繼徽岐茂貞所鎮也】天子神器拱手授人其埶必然矣公宜亟帥關中諸鎮兵固守潼關赴救河中【帥讀曰率】僕自知不武願於公西偏授一小鎮此地請公有之關中安危國祚修短繫公此舉願審思之茂貞素無遠圖不報【此時李茂貞若能救河中以連河東異時鳳翔必無受圍之困】 二月甲寅朔河東將李嗣昭攻澤州拔之乙卯張存敬引兵發晉州己未至河中遂圍之王珂埶窮將奔京師而人心離貳會浮梁壞流澌塞河舟行甚難【浮梁謂蒲津之浮梁也河中府治河東縣架浮梁以通河西縣自此路西入長安塞悉則翻】珂挈其族數百欲夜登舟親諭守城者皆不應牙將劉訓曰今人情擾擾若夜出涉河必爭舟紛亂一夫作難事不可知不若且送款存敬徐圖向背珂從之壬戌珂植白幡於城隅【難乃旦翻背蒲妹翻植直吏翻又如字】遣使以牌印請降於存敬存敬請開城珂曰吾於朱公有家世事分【珂父重榮朱全忠以舅事之分扶問翻】請公退舍俟朱公至吾自以城授之存敬從之且使走白全忠乙丑全忠至洛陽聞之喜【凡用兵者擁強大之衆以臨弱小必曰以此衆戰誰能禦之以此攻城何城不克此以聲形臨敵者也而弱小者能堅力一心而守之以大衆困於堅城之下者亦多矣故善用兵者不以大衆為可恃而以攻城為最下王珂之迎降朱全忠之所以喜也】馳往赴之戊辰至虞鄉【九域志虞鄉在河中府東六十里】先哭於重榮之墓盡哀河中人皆悦珂欲面縛牽羊出迎全忠遽使止之曰太師舅之恩何可忘【全忠由重榮歸國故云然】若郎君如此使僕異日何以見舅於九泉乃以常禮出迎握手歔欷【歔音虚欷音希又許既翻】聯轡入城全忠表張存敬為護國軍留後王珂舉族遷于大梁【僖宗廣明元年王重榮據河中傳兄重盈以及子珂凡二十二年而亡朱全忠自此有河中晉絳】其後全忠遣珂入朝遣人殺之于華州全忠聞張夫人疾亟遽自河中東歸【張夫人全忠之妻也】李克用遣使以重幣請修好於全忠【好呼到翻】全忠雖遣使報而忿其書辭蹇傲決欲攻之 【考異曰唐末見聞録乾寧四年六月差軍將發往汴州為使其書云云汴州回書云云據全忠書有前年洹水曾獲賢郎去歲青山又擒列將又云鎮定歸款蒲晉求和則非乾寧四年明矣唐年補録天復元年五月壬午制以朱全忠兼領河中仍詔與太原通和初朝廷以全忠吞併河朔又收下蒲津必恐兵起相侵乃下詔太原夷門使務和好時太原意亦以全忠漸強先以書聘全忠書辭與見聞錄同全忠答太原書又進表云臣與太原曾於頃歲首締歡盟及其偶掇猜嫌止為各爭言氣又云但以來書意旨未息披攘又云臣詳兹來意益切憤懷不敢遂與通和必恐有孤朝寄已遣諸軍進討訖續寶運録載全忠表云臣當道先自河府抽軍便赴太原進討已累具狀分析聞奏訖臣今月二十三日部領牙隊到東都李克用差到專使張特與臣書一封并駞馬弓箭銀器匹段等與臣通和其張特臣且與回書放歸訖當月河府抽回兵士即勒權於河陽屯駐見排比收復潞州使邐迤赴太原進討次日李克用與臣書一封謹隨狀封進天復四年二月奏其年三月二日表到駕前奉襄宗三月八日敕云云云天復四年尤誤也編遣録天復元年二月李克用遣軍將張特執檄厚幣而來釋憾亦差軍將持函以為報又曰辛巳上欲北回軍便征北虜近者李克用以甘言重幣請通和好遂具事奏聞語與補録同唐太祖紀年録天復元年六月太祖以梁寇方強難以兵伏陽降心以緩其謀乃遣押牙張特持幣馬書檄以諭之請復舊好書詞大陳北邊五部壬馬之盛皆吾外援朱温視之不懌令敬翔修報詞旨踈拙人士嗤之薛居正五代史梁紀天復元年二月李克用遣牙將張特來聘帝亦遣使報命李襲吉傳天復中武皇議欲修好於梁命襲吉以貽梁祖書辭與見聞録同其年月日各參差不同據全忠答太原書云今月二十二日使至又上表云先自河府抽軍赴太原又云二十三日到東都則克用書達全忠必在天復元年二月下旬今從編遺録梁紀】 以翰林學士戶部侍郎王溥為中書侍郎同平章事以吏部侍郎裴樞為戶部侍郎同平章事溥正雅之從孫也【王正雅見二百四十四卷文宗大和五年從才用翻】常在崔胤幕府故胤引之 贈諡故睦王倚曰恭哀太子【倚為宦官所殺見上年】 加幽州節度使劉仁恭魏博節度使羅紹威並兼侍中 三月癸未朔朱全忠至大梁【自河中歸至大梁】癸卯遣氏叔琮等將兵五萬攻李克用入自太行魏博都將張文恭入自磁州新口【武宗之討劉稹也自遼州開新路達于磁州武安縣故謂之新口】葛從周以兖鄆兵會成德兵入自土門洺州刺史張歸厚入自馬嶺義武節度使王處直入自飛狐【沈括曰北岳常山之岑謂之大茂山自石晉割燕雲與契丹以大茂山分脊為界飛狐路在大茂山西自銀冶寨北出倒馬關度北界却自石門子令水鋪入缾形梅回兩寨之間至代州今大茂祠中多唐人古碑殿前一亭有李克用題名云太原河東節度使李克用親領步騎五十萬問罪幽陵回師自飛狐路即歸鴈門】權知晉州侯言以慈隰晉絳兵入自隂地叔琮入天井關進軍昂車【昂車即昂車關在澤州昂車嶺】辛亥沁州刺史蔡訓以城降河東都將蓋瑋詣侯言降即令權知沁州【蓋古盍切姓也】壬子叔琮拔澤州李存璋弃城走叔琮進攻潞州昭義節度使孟遷降之河東屯將李審建王周將步軍一萬騎二千詣叔琮降叔琮進趣晉陽【趣七喻翻】夏四月乙卯叔琮出石會關營于洞渦驛【洞渦驛臨洞渦水】張歸厚引兵至遼州丁巳遼州刺史張鄂降别將白奉國會成德兵自井陘入【陘音刑】己未拔承天軍與叔琮烽火相應 甲戌上謁太廟丁丑赦天下改元雪王涯等十七家【王涯等誅夷見二百四十五卷文宗太和九年崔胤將誅宦官故先雪王涯等】 初楊復恭為中尉借度支賣麴一年之利以贍兩軍自是不復肯歸【度徒洛翻復扶又翻下同】至是崔胤草赦【草赦文及諸條件】欲抑宦官聽酤者自造麴但月輸榷酤錢兩軍先所造麴趣令減價賣之過七月無得復賣【榷訖岳翻酤音故復扶又翻會要會昌六年九月勑揚州等入道州府置榷酤并置官店酤酒代百姓納榷酒錢并充資助軍用如有人私酤酒及置私麴者罪止一身不得没入家產蓋榷酤賣麯本皆屬度支】 東川節度使王宗滌以疾求代王建表馬步使王宗裕為留後 氏叔琮等引兵抵晉陽城下數挑戰【數所角翻挑徒了翻】城中大恐李克用登城備禦不遑飲食時大雨積旬城多頹壞隨加完補河東將李嗣昭李嗣源鑿暗門夜出攻汴壘屢有殺獲李存進敗汴軍於洞渦【敗補邁翻】時汴軍既衆芻糧不給久雨士卒瘧利【瘧逆約翻寒熱迭作為瘧泄下為利】全忠乃召兵還五月叔琮等自石會關歸 【考異曰編遺録四月壬戌李克用遣張特齎書請尋懽盟乃指揮諸軍所在且駐留見差發專人之太原許通懽好兼并州地寒節侯甚晚戊馬既多野草不足於芻牧尋令氏叔琮迴戈後唐太祖紀五月氏叔琮及四面賊軍皆退薛史梁紀班師在四月後唐紀汴軍退在五月蓋全忠以四月命班師而叔琮等以五月離晉陽故國史記之各異也】諸道軍亦退河東將周德威李嗣昭以精騎五千躡之殺獲甚衆先是汾州刺史李瑭舉州附於汴軍【先悉薦翻】克用遣其將李存審攻之三日而拔執瑭斬之氏叔琮過上黨孟遷挈族隨之南徙朱全忠遣丁會代守潞州【為丁會歸李克用張本】 朱全忠奏乞除河中節度使而諷吏民請已為帥【帥讀曰率】癸卯以全忠為宣武宣義天平護國四鎮節度使【當是時自蒲陜以東至于海南距淮北距河諸鎮皆為朱全忠所有使全忠以鄰道自廣則當兼領佑國河陽陜虢不應越此三鎮而領河中全忠所以領河中者上以制朝廷下以制李克用也】 己酉加鎮海鎮東節度使錢鏐守侍中崔胤之罷兩軍賣麴也并近鎮亦禁之李茂貞惜其<br />
<br />
  利表乞入朝論奏【李茂貞在鳳翔近鎮也故爭賣麴之利】韓全誨請許之茂貞至京師全誨深與相結崔胤始愳隂厚朱全忠益甚與茂貞為仇敵矣 以佑國節度使張全義兼中書令六月癸亥朱全忠如河中 【考異曰薛居正五代史梁紀庚申帝發自大梁今】<br />
<br />
  【從編遺録】 上之返正也中書舍人令狐渙給事中韓偓皆預其謀故擢為翰林學士數召對訪以機密渙綯之子也【數所角翻令狐綯相宣宗】時上悉以軍國事委崔胤每奏事上與之從容【從千容翻】或至然燭宦官畏之側目皆咨胤而後行胤志欲盡除之韓偓屢諫曰事禁太甚此輩亦不可全無恐其黨迫切更生它變胤不從丁卯上獨召偓問曰敕使中為惡者如林何以處之【處昌呂翻下同】對曰東内之變敕使誰非同惡處之當在正旦【謂誅劉季述等時也】今已失其時矣上曰當是時卿何不為崔胤言之【為于偽翻】對曰臣見陛下詔書云自劉季述等四家之外其餘一無所問夫人主所重莫大於信既下此詔則守之宜堅若復戮一人則人人懼死矣【復扶又翻】然後來所去者已為不少【去羌呂翻少詩沼翻】此其所以忷忷不安也陛下不若擇其尤無良者數人明示其罪寘之於法然後撫諭其餘曰吾恐爾曹謂吾心有所貯【貯丁呂翻藏蓄也】自今可無疑矣乃擇其忠厚者使為之長【長知兩翻】其徒有善則奬之有罪則懲之咸自安矣今此曹在公私者以萬數【公謂有職名於官者私謂乞丐攜養於宦者私家未有名籍在於官者】豈可盡誅邪夫帝王之道當以重厚鎮之公正御之至于瑣細機巧此機生則彼機應矣終不能成大功所謂理絲而棼之者也【治絲而棼左傳魯衆仲之言杜預注云絲見棼緼益所以亂】况今朝廷之權散在四方苟能先收此權則事無不可為者矣上深以為然曰此事終以屬卿【嗚呼世固有能知之言之而不能究于行者韓偓其人也屬之欲翻】 李克用遣其將李嗣昭周德威將兵出隂地關攻隰州刺史唐禮降之進攻慈州刺史張瓌降之 閏月以河陽節度使丁會為昭義節度使 【考異曰薛居正五代史會傳自河陽以疾致政于洛陽梁祖季年猜忌故將功大者多遭族滅會隂有避禍之志稱疾者累年天復元年梁祖并有河中晉絳乃起會為昭義節度使按光化二年六月會自河陽為昭義節度使九月李克用取潞州表孟遷為節度使時罕之已卒必是會却領河陽至此纔二年則非致政稱疾累年也又是時全忠未嘗誅戮大將疑會降河東後作傳者誤以天祐中事在前言之耳】孟遷為河陽節度使從朱全忠之請也 道士杜從法以妖妄誘昌普合三州民作亂【妖一遥翻誘音酉昌州乾元中割瀘普渝資等州界置普州漢牛鞞資中墊江德陽四縣之境梁置普慈郡後周置普州合州漢墊江地宋置東宕渠郡西魏置合州九域志普州東至昌州一百七十五里昌州東至合州一百八十里】王建遣行營兵馬使王宗黯將兵三萬會東川武信兵討之宗黯即吉諫也崔胤請上盡誅宦官但以宫人堂内諸司事【時宦官牙領内諸司使】宦官屬耳頗聞之【屬之欲翻】韓全誨等涕泣求衷于上上乃令胤有事封疏以聞勿口奏宦官求美女知書者數人内之宫中隂令詗密其事【詗古永翻又翾正翻】盡得胤密謀上不之覺也全誨等大懼每宴聚流涕相訣别日夜謀所以去胤之術胤時領三司使【去羌呂翻三司戶部度支鹽鐵】全誨等教禁軍對上諠譟訴胤減損冬衣上不得已解胤鹽銕使時朱全忠李茂貞各有挾天子令諸侯之意全忠欲上幸東都茂貞欲上幸鳳翔胤知謀泄事急遺朱全忠書【遺唯季翻】稱被密詔【被皮義翻】令全忠以兵迎車駕且言昨者返正皆令公良圖【胤言返正之謀皆出於全忠按舊書帝紀全忠并河中進檢校太師兼中書令故稱令公】而鳳翔先入朝抄取其功【李茂貞入朝見上正月抄楚交翻】今不速來必成罪人豈惟功為它人所有且見征討矣全忠得書秋七月甲寅遽歸大梁發兵 【考異曰唐太祖紀年録會汴入寇同華宦者知崔胤之謀時胤專掌三司泉貨韓全誨教禁兵伺胤出聚而呼譟訴以冬衣減損軍人又上前披訴天子徇衆情罷崔胤知政事崔胤怒急召朱温請以兵師入輔唐補紀時朱全忠在河中胤潜作急令全忠入朝又修書云云全忠得此書詔便發河中還汴按是時全忠未寇同華胤亦未罷紀年録誤今從唐補紀】西川龍臺鎮使王宗侃等討杜從法平之【九域志普州安岳縣有龍臺鎮】 八月甲申上問韓偓曰聞陸扆不樂吾返正【樂音洛下同】正旦易服乘小馬出啓夏門有諸【啓夏門京城南面東來第一門夏戶雅翻】對曰返正之謀獨臣與崔胤輩數人知之扆不知也一旦忽聞宫中有變人情能不驚駭易服逃避何妨有之陛下責其為宰相無死難之志則可也【難乃旦翻】至於不樂返正恐出讒人之口願陛下察之上乃止韓全誨等懼誅謀以兵制上乃與李繼昭李繼誨李彦弼李繼筠深相結繼昭獨不肯從它日上問韓偓外間何所聞對曰惟聞敕使憂愳與功臣及繼筠交結【功臣謂李繼昭李繼誨李彦弼也】將致不安亦未知其果然不耳【然不讀曰否】上曰是不虛矣比日繼誨彦弼輩語漸倔強【比毗至翻倔其勿翻強其兩翻】令人難耐令狐渙欲令朕召崔胤及全誨等於内殿置酒和解之何如對曰如此則彼凶悖益甚【悖蒲昧翻又蒲没翻】上曰為之奈何對曰獨有顯罪數人速加竄逐餘者許其自新庶幾可息【幾居依翻】若一無所問彼必知陛下心有所貯益不自安事終未了耳【貯下呂翻】上曰善既而宦官自恃黨援已成稍不遵敕旨上或出之使監軍或黜守諸陵【黜守諸陵者剥色配役諸陵也】皆不行上無如之何 或告楊行密云錢鏐為盜所殺行密遣步軍都指揮使李神福等將兵取杭州兩浙將顧全武等列八寨以拒之 九月癸丑上急召韓偓謂曰聞全忠欲來除君側之惡大是盡忠然須令與茂貞共其功若兩帥交争則事危矣【帥所類翻】卿為我語崔胤速飛書兩鎮【為于偽翻語牛倨翻兩鎮謂汴岐】使相與合謀則善矣壬戌上又謂偓曰繼誨彦弼輩驕横益甚【横戶孟翻】累日前與繼筠同入輒于殿東令小兒歌以侑酒【侑佑也】令人驚駭對曰臣必知其然兹事失之於初當正旦立功之時【謂誅劉王迎上反正時】但應以官爵田宅金帛酧之不應聽其出入禁中此輩素無知識數求入對或僭易薦人【數所角翻易以䜴翻】稍有不從則生怨望况惟知嗜利為敕使以厚利雇之【言韓全誨等以利㗖繼誨彦弼惟其所指使而為之用若受傭雇然】令其如此耳崔胤本留衛兵欲以制敕使也【言留岐兵以制宦官事見是年正月】今敕使衛兵相與為一將若之何汴兵若來必與岐兵鬭于闕下臣竊寒心上但愀然憂沮而已【秋子小翻】冬十月戊戌朱全忠大舉兵發大梁 【考異曰薛居正五代史十月戊戌奉密詔赴長安是時朝廷軍國大政專委崔胤崔每事裁抑宦官宦官側目崔一日於便殿奏欲盡去之全誨等屬垣聞之中官視崔眥裂以重賂甘言誘藩臣以為城社時因讌聚則相向流涕時崔專掌三司貨泉全誨等教禁兵於昭宗前訴之昭宗不得已罷崔知政事崔急召太祖請以兵入輔故有是行按帝幸鳳翔前崔胤未罷相此與太祖紀年録畧同亦誤】李神福與顧全武相拒久之神福杭俘使出入卧<br />
<br />
  内神福謂諸將曰杭兵尚彊我師且當夜還杭俘走告全武神福命勿追【逸杭俘使之告全武以誘之】暮遣羸兵先行神福為殿【羸倫為翻殿丁練翻】使行營都尉呂師造伏兵青山下【沈括曰臨安縣有青山鎮路振九國志作設伏青山路】全武素輕神福出兵追之神福師造夾擊大破之斬首五千級生擒全武錢鏐聞之驚泣曰喪我良將【喪息浪翻】神福進攻臨安【臨安縣錢鏐所起之地衣錦軍在焉九域志臨安縣在杭州西一百二十里】兩浙將秦昶帥衆三千降之【帥讀曰率】 韓全誨聞朱全忠將至丁酉令李繼筠李彦弼等勒兵劫上請幸鳳翔宫禁諸門皆增兵防守 【考異曰按金鑾記二十日入直隔夜崔公傳語明日請相看侵早到門崔出御札相示然則添人把門及降御札皆十九日事實録己亥差人把門己亥乃二十一日實録誤也】人及文書出入搜閲甚嚴上遣人密賜崔胤御札言皆悽愴【愴楚亮翻】末云我為宗社大計埶須西行卿等但東行也【西行謂將幸鳳翔使胤等東行趣朱全忠進兵】惆悵惆悵【惆丑留翻悵丑亮翻】戊戌上遣趙國夫人出語韓偓【命宫人出至學士院語之也新舊書帝紀曰趙國夫人寵顔語牛倨翻】朝來彦弼輩無禮極甚欲召卿對其埶未可且言上與皇后但涕泣相向自是學士不復得對矣癸卯韓全誨等令上入閤召百官【百官自閤門入見於内殿謂之入閤】追寢正月丙午敕書【丙午敕書依大中舊制見上】悉如咸通以來近例是日開延英全誨等即侍側同議政事丁未神策都指揮使李繼筠遣部兵掠内庫寶貨帷帳法物韓全誨遣人密送諸王宫人先之鳳翔【之往也】戊申朱全忠至河中表請車駕幸東都京城大駭士民亡竄山谷是日百官皆不入朝闕前寂無人十一月己酉朔李繼筠等勒兵闕下禁人出入諸軍大掠士民衣紙及布襦者滿街極目【衣於既翻襦汝朱翻】韓建以幕僚司馬鄴知匡國留後朱全忠引四鎮兵七萬趣同州【四鎮兵宣武宣義天平護國兵也趣七喻翻】鄴迎降 韓全誨等以李繼昭不與之同遏絶不令見上時崔胤居第在開化坊【按五代史孫德昭傳開化坊在長安東街】繼昭帥所部六十餘人【六十當作六千帥讀曰率】及關東諸道兵在京師者共守衛之【史言崔胤所以不死於羣閹之手】百官及士民避亂者皆往依之【依李繼昭之兵以避禁兵及岐兵暴掠】庚戌上遣供奉官張紹孫召百官崔胤等皆表辭不至壬子韓全誨等陳兵殿前言於上曰全忠以大兵逼京師欲刼天子幸洛陽求傳禪臣等請奉陛下幸鳳翔收兵拒之上不許杖劒登乞巧樓全誨等逼上下樓上行纔及夀春殿李彦弼已於御院縱火【御院天子及后妃所居之地】是日冬至上獨坐思政殿翹一足一足蹋闌干【蹋與踏同闌干殿檻也】庭無羣臣旁無侍者頃之不得已與皇后妃嬪諸王百餘人皆上馬慟哭聲不絶出門回顧禁中火已赫然是夕宿鄠縣【九域志鄠縣在長安南六十里考異曰續寶運録其年十月朱全忠發士馬十二月入長安聖上幸鳳翔宰臣裴諗翰林學士令狐渙等扈】<br />
<br />
  【從其皇后王氏及千官太子玉印龍服並是汴州迎在華州相次修東都宫室旋迎赴東都其年十一月初鳳翔士馬入京劫掠街西諸坊寶貨士女至甚及七日汴州士馬入京赴救長安士庶並走攅在開化坊其說妄謬今不取】朱全忠遣司馬鄴入華州謂韓建曰公不早知過自歸又煩此軍少留城下矣【司馬鄴本韓建幕僚以同州降因使之諭建少詩沼翻】是日全忠自故市引兵南渡渭韓建遣節度副使李巨川請降獻銀三萬兩助軍全忠乃西南趣赤水【趣七喻翻】癸丑李茂貞迎車駕於田家磑【磑五對翻】上下馬慰接之【史言昭宗屈體以接李茂貞】甲寅車駕至盩厔乙卯留一日朱全忠至零口西【宋白曰昭應縣界有零口天授二年於此置鴻州於郭下置鴻門縣蓋古鴻門之地也昭應漢新豐縣地宋又改昭應為臨潼九域志臨潼縣有零口鎮】聞車駕西幸與僚佐議復引兵還赤水左僕射致仕張濬說全忠曰【張濬時居長水說式芮翻】韓建茂貞之黨不先取之必為後患全忠聞建有表勸天子幸鳳翔乃引兵逼其城建單騎迎謁全忠責之對曰建目不知書凡表章書檄皆李巨川所為全忠以巨川常為建畫策斬之軍門【李巨川之誅晚矣常為于偽翻】謂建曰公許人可即往衣錦【漢人曰富貴不歸故鄉如衣錦夜行韓建許州長社人也衣於既翻】丁巳以建為忠武節度使理陳州【唐置忠武軍於許州黄巢之自長安東出也趙犨陳人也守陳州有功朝廷以忠武節授之奏徙忠武軍治陳州按是時天子已西幸韓建自華徙陳皆朱全忠為之未經表授即以為忠武節度使何所禀命乎】以兵援送之【慮韓建中路逸而歸岐又慮其在華久其將士有刼奪之者 考異曰編遺録上引兵逼華州韓建輕騎出牆歸投上於西溪亭子與建飲膳畢却歸赤水營旬日乃請建充忠武節度使梁太祖實録乙卯大軍及華州建來降甲辰署建權知華州事仍以宣武牙推龔麟佐之唐太祖紀年録丙辰汴軍攻華州九日建以城降唐補紀同州刺史王行約閉城登壘全忠斫開城門屠之不留噍類華州韓建聞此出城三十里迎之只於迎處云令公本貫許州便仰衣錦乃差人押出關東舊傳建今李巨川至河中送款敬翔疾其文筆勸全忠害之薛居正五代史梁祖紀丙辰帝表建權知忠武軍事徙令赴任實録乙卯全忠取華州丙辰次武功徙建為忠武節度使按此月無甲辰蓋丙辰字誤也全忠乙卯取華州丙辰豈能遽至武功唐補紀又云昭宗不知崔胤偽行詔命聞朱全忠平陷兩州十一月三日亥時奔波西去按行約乃克用取同州時節度使也程匡柔妄謬多此類今取華州日從梁太祖實録李巨川死從昭宗實録】以前商州刺史李存權知華州徙忠武節度使趙珝為匡國節度使【趙珝徙節同州亦非天子出命】車駕之在華州也【乾寧三年四年車駕在華州】商賈輻湊【賈音古天子行在所從兵及百司供億浩繁故商賈輻湊以牟利輻湊者蓋以車輻皆内湊于轂為諭夫三十輻共一轂轂者衆輻聚湊之所四方之商賈内嚮而聚湊焉故曰輻湊】韓建重征之二年得錢九百萬緡至是全忠盡取之【史言自古聚財者率為他人積】是時京師無天子行在無宰相崔胤使太子太師盧渥等二百餘人列狀請朱全忠西迎車駕又使王溥至赤水見全忠計事 【考異曰編遺録于時長安無人主朝廷無敕畫帝在岐下無輔臣自漢魏以來喪亂未若今日胤請王溥自西京至赤水請上進軍迎駕戊午離赤水薛居正五代史梁紀己未發赤水按唐太祖紀年録朱温至長樂崔胤帥百官班迎編遺録胤請王溥自西京至赤水軍前商議實録胤東寓華州又云胤召溥至赤水皆誤也舊紀亦云胤令溥至赤水促全忠迎駕今從之發赤水日從編遺録】全忠復書曰進則懼脇君之謗退則懷負國之慚然不敢不勉戊午全忠發赤水 辛酉以兵部侍郎盧光啓權句當中書事【時無宰相權使之句當句占侯翻當丁浪翻】車駕留岐山三日壬戌至鳳翔 朱全忠至長安宰相帥百官班迎於長樂坡明日行復班辭於臨臯驛【班迎班辭非藩臣所得當崔胤之奉朱全忠至此為一身脱死計非為唐社稷計也宦官既誅胤亦死於全忠之手宜矣帥讀曰率樂音洛復扶又翻】全忠賞李繼昭之功【以其能保衛崔胤及百官也】初令權知匡國留後復留為兩街制置使賜與甚厚繼昭盡獻其兵八千人【孫德昭畏朱温之雄猜也】全忠使判官李擇裴鑄入奏事稱奉密詔及得崔胤書令臣將兵入朝韓全誨等矯詔答以朕避災至此非宦官所刼密詔皆崔胤詐為之卿宜斂兵歸保土宇茂貞遣其將符道昭屯武功以拒全忠【九域志武功縣在長安西北一百五十里】癸亥全忠將康懷貞擊破之 丁卯以盧光啓為右諫議大夫參知機務【參知機務唐久不除授盧光啟自權句當中書為之】 戊辰朱全忠至鳳翔軍於城東【考異曰實録乙丑全忠駐軍岐城之東丙寅全忠軍至城下按全忠癸亥離長安乙丑丙寅至岐太速今從】<br />
<br />
  【編遺録新紀】李茂貞登城謂曰天子避災非臣下無禮讒人誤公至此全忠報曰韓全誨刼遷天子今來問罪迎扈還宫岐王苟不預謀何煩陳諭上屢詔全忠還鎮全忠乃拜表奉辭【屢詔全忠歸鎮韓全誨李茂貞挾天子以令之也全忠拜表奉辭若不敢逆詔指者然其意則有在矣】辛未移兵北趣邠州【全忠之意在此茂貞養子繼徽鎮邠邠岐輔車之援也若先得邠則岐孤九域志鳳翔東北至邠州二百二十二里趣七喻翻下同 考異曰金鑾記曰十七日早聞岐師昨夜二更却迴云軍大衂汴令有表迎駕并述行止汴軍在岐東下寨十八日十九日白麻盧光啟可御史大夫參知機務二十日翰林學士姚洎兼知外制誥二十四日汴令有表奉辭東去二十五日汴軍離發延英門舊紀癸酉全忠辭去今從編遺録】甲戌制守司空兼門下侍郎同平章事崔胤責授工部尚書 【考異曰實録載制辭曰四居極位一無可稱又曰無功及人為國生事按舊傳前為罷知政事落使務後云同平章事鹽鐵轉運使實録前云罷胤鹽鐵使至此制官位中復帶鹽鐵使皆誤】戶部侍郎同平章事裴樞罷守本官【皆宦官之意也時宰相皆不扈從】乙亥朱全忠攻邠州丁丑靜難節度使李繼徽請降復姓名楊崇本全忠質其妻於河中令崇本仍鎮邠州【難乃旦翻質音致為朱全忠漁色邠岐復連兵張本】全忠之西入關也韓全誨李茂貞以詔命徵兵河東茂貞仍以書求援於李克用克用遣李嗣昭將五千騎自沁州趣晉州與汴兵戰于平陽北破之【漢平陽縣隋改為臨汾晉州治焉唐府兵未廢時有平陽府】乙亥全忠發邠州戊寅次三原【自邠州東南至三原一百五十餘里】十二月癸未崔胤至三原見全忠趣之迎駕【趣讀曰促】己丑全忠遣朱友寧攻盩厔不下戊戌全忠自往督戰盩厔降屠之【九域志盩厔縣在鳳翔府東南二百里盩音厔音窒】全忠令崔胤帥百官及京城居民悉遷于華州【帥讀曰率】詔以裴贄充大明宫留守清海節度使徐彦若薨遺表薦行軍司馬劉隱權留<br />
<br />
  後【劉隱始得廣州】 李神福知錢鏐定不死【或言錢鏐為盜所殺見上文八月】而臨安城堅久攻不拔欲歸恐為鏐所邀【自臨安退還宣州有千秋嶺之險】乃遣人守衛鏐祖考丘壟禁樵采【錢鏐臨安人其祖父丘壟在焉】又使顧全武通家信鏐遣使謝之神福於要路多張旗幟為虚寨鏐以為淮南兵大至遂請和神福受其犒賂而還【還音旋又如字】 朱全忠之入關也【是年十一月朱全忠入關】戎昭節度使馮行襲遣副使魯崇矩聽命於全忠【按光化元年以馮行襲為昭信軍節度使天祐二年始改昭信軍為戎昭軍】韓全誨遣中使二十餘人分道徵江淮兵屯金州以脅全忠行襲盡殺中使收其詔敕送全忠【馮行襲以昭信節度使治金州故得盡殺中使】又遣使徵兵於王建朱全忠亦遣使乞師于建建外修好於全忠罪狀李茂貞【好呼到翻】而隂勸茂貞堅守許之救援以武信節度使王宗佶前東川節度使王宗滌等為扈駕指揮使將兵五萬聲言迎車駕其實襲茂貞山南諸州【為王建取山南西道張本】 江西節度使鍾傳將兵圍撫州刺史危全諷天火燒其城士民讙驚【讙與諠同】諸將請急攻之傳曰乘人之危非仁也乃祝曰全諷之罪無為害民火尋止全諷聞之謝罪聽命以女妻傳子匡時【妻七細翻】傳少時嘗獵【少詩照翻】醉遇虎與鬭虎搏其肩而傳亦持虎腰不置旁人共殺虎乃得免既貴悔之常戒諸子曰士處世貴智謀勿效吾暴虎也【詩曰袒裼暴虎注云暴虎空手以搏之也處昌呂翻】 武貞節度使雷滿薨子彦威自稱留後<br />
<br />
  資治通鑑卷二百六十二<br />
<br />
<史部,編年類,資治通鑑>  <br>
   </div> 

<script src="/search/ajaxskft.js"> </script>
 <div class="clear"></div>
<br>
<br>
 <!-- a.d-->

 <!--
<div class="info_share">
</div> 
-->
 <!--info_share--></div>   <!-- end info_content-->
  </div> <!-- end l-->

<div class="r">   <!--r-->



<div class="sidebar"  style="margin-bottom:2px;">

 
<div class="sidebar_title">工具类大全</div>
<div class="sidebar_info">
<strong><a href="http://www.guoxuedashi.com/lsditu/" target="_blank">历史地图</a></strong>  
<a href="http://www.880114.com/" target="_blank">英语宝典</a>  
<a href="http://www.guoxuedashi.com/13jing/" target="_blank">十三经检索</a> 
<br><strong><a href="http://www.guoxuedashi.com/gjtsjc/" target="_blank">古今图书集成</a></strong> 
<a href="http://www.guoxuedashi.com/duilian/" target="_blank">对联大全</a> <strong><a href="http://www.guoxuedashi.com/xiangxingzi/" target="_blank">象形文字典</a></strong> 

<br><a href="http://www.guoxuedashi.com/zixing/yanbian/">字形演变</a>  <strong><a href="http://www.guoxuemi.com/hafo/" target="_blank">哈佛燕京中文善本特藏</a></strong>
<br><strong><a href="http://www.guoxuedashi.com/csfz/" target="_blank">丛书&方志检索器</a></strong> <a href="http://www.guoxuedashi.com/yqjyy/" target="_blank">一切经音义</a>  

<br><strong><a href="http://www.guoxuedashi.com/jiapu/" target="_blank">家谱族谱查询</a></strong>  <strong><a href="http://shufa.guoxuedashi.com/sfzitie/" target="_blank">书法字帖欣赏</a></strong> 
<br>

</div>
</div>


<div class="sidebar" style="margin-bottom:0px;">

<font style="font-size:22px;line-height:32px">QQ交流群9:489193090</font>


<div class="sidebar_title">手机APP 扫描或点击</div>
<div class="sidebar_info">
<table>
<tr>
	<td width=160><a href="http://m.guoxuedashi.com/app/" target="_blank"><img src="/img/gxds-sj.png" width="140"  border="0" alt="国学大师手机版"></a></td>
	<td>
<a href="http://www.guoxuedashi.com/download/" target="_blank">app软件下载专区</a><br>
<a href="http://www.guoxuedashi.com/download/gxds.php" target="_blank">《国学大师》下载</a><br>
<a href="http://www.guoxuedashi.com/download/kxzd.php" target="_blank">《汉字宝典》下载</a><br>
<a href="http://www.guoxuedashi.com/download/scqbd.php" target="_blank">《诗词曲宝典》下载</a><br>
<a href="http://www.guoxuedashi.com/SiKuQuanShu/skqs.php" target="_blank">《四库全书》下载</a><br>
</td>
</tr>
</table>

</div>
</div>


<div class="sidebar2">
<center>


</center>
</div>

<div class="sidebar"  style="margin-bottom:2px;">
<div class="sidebar_title">网站使用教程</div>
<div class="sidebar_info">
<a href="http://www.guoxuedashi.com/help/gjsearch.php" target="_blank">如何在国学大师网下载古籍?</a><br>
<a href="http://www.guoxuedashi.com/zidian/bujian/bjjc.php" target="_blank">如何使用部件查字法快速查字?</a><br>
<a href="http://www.guoxuedashi.com/search/sjc.php" target="_blank">如何在指定的书籍中全文检索?</a><br>
<a href="http://www.guoxuedashi.com/search/skjc.php" target="_blank">如何找到一句话在《四库全书》哪一页?</a><br>
</div>
</div>


<div class="sidebar">
<div class="sidebar_title">热门书籍</div>
<div class="sidebar_info">
<a href="/so.php?sokey=%E8%B5%84%E6%B2%BB%E9%80%9A%E9%89%B4&kt=1">资治通鉴</a> <a href="/24shi/"><strong>二十四史</strong></a>&nbsp; <a href="/a2694/">野史</a>&nbsp; <a href="/SiKuQuanShu/"><strong>四库全书</strong></a>&nbsp;<a href="http://www.guoxuedashi.com/SiKuQuanShu/fanti/">繁体</a>
<br><a href="/so.php?sokey=%E7%BA%A2%E6%A5%BC%E6%A2%A6&kt=1">红楼梦</a> <a href="/a/1858x/">三国演义</a> <a href="/a/1038k/">水浒传</a> <a href="/a/1046t/">西游记</a> <a href="/a/1914o/">封神演义</a>
<br>
<a href="http://www.guoxuedashi.com/so.php?sokeygx=%E4%B8%87%E6%9C%89%E6%96%87%E5%BA%93&submit=&kt=1">万有文库</a> <a href="/a/780t/">古文观止</a> <a href="/a/1024l/">文心雕龙</a> <a href="/a/1704n/">全唐诗</a> <a href="/a/1705h/">全宋词</a>
<br><a href="http://www.guoxuedashi.com/so.php?sokeygx=%E7%99%BE%E8%A1%B2%E6%9C%AC%E4%BA%8C%E5%8D%81%E5%9B%9B%E5%8F%B2&submit=&kt=1"><strong>百衲本二十四史</strong></a>  <a href="http://www.guoxuedashi.com/so.php?sokeygx=%E5%8F%A4%E4%BB%8A%E5%9B%BE%E4%B9%A6%E9%9B%86%E6%88%90&submit=&kt=1"><strong>古今图书集成</strong></a>
<br>

<a href="http://www.guoxuedashi.com/so.php?sokeygx=%E4%B8%9B%E4%B9%A6%E9%9B%86%E6%88%90&submit=&kt=1">丛书集成</a> 
<a href="http://www.guoxuedashi.com/so.php?sokeygx=%E5%9B%9B%E9%83%A8%E4%B8%9B%E5%88%8A&submit=&kt=1"><strong>四部丛刊</strong></a>  
<a href="http://www.guoxuedashi.com/so.php?sokeygx=%E8%AF%B4%E6%96%87%E8%A7%A3%E5%AD%97&submit=&kt=1">說文解字</a> <a href="http://www.guoxuedashi.com/so.php?sokeygx=%E5%85%A8%E4%B8%8A%E5%8F%A4&submit=&kt=1">三国六朝文</a>
<br><a href="http://www.guoxuedashi.com/so.php?sokeytm=%E6%97%A5%E6%9C%AC%E5%86%85%E9%98%81%E6%96%87%E5%BA%93&submit=&kt=1"><strong>日本内阁文库</strong></a> <a href="http://www.guoxuedashi.com/so.php?sokeytm=%E5%9B%BD%E5%9B%BE%E6%96%B9%E5%BF%97%E5%90%88%E9%9B%86&ka=100&submit=">国图方志合集</a> <a href="http://www.guoxuedashi.com/so.php?sokeytm=%E5%90%84%E5%9C%B0%E6%96%B9%E5%BF%97&submit=&kt=1"><strong>各地方志</strong></a>

</div>
</div>


<div class="sidebar2">
<center>

</center>
</div>
<div class="sidebar greenbar">
<div class="sidebar_title green">四库全书</div>
<div class="sidebar_info">

《四库全书》是中国古代最大的丛书,编撰于乾隆年间,由纪昀等360多位高官、学者编撰,3800多人抄写,费时十三年编成。丛书分经、史、子、集四部,故名四库。共有3500多种书,7.9万卷,3.6万册,约8亿字,基本上囊括了古代所有图书,故称“全书”。<a href="http://www.guoxuedashi.com/SiKuQuanShu/">详细>>
</a>

</div> 
</div>

</div>  <!--end r-->

</div>
<!-- 内容区END --> 

<!-- 页脚开始 -->
<div class="shh">

</div>

<div class="w1180" style="margin-top:8px;">
<center><script src="http://www.guoxuedashi.com/img/plus.php?id=3"></script></center>
</div>
<div class="w1180 foot">
<a href="/b/thanks.php">特别致谢</a> | <a href="javascript:window.external.AddFavorite(document.location.href,document.title);">收藏本站</a> | <a href="#">欢迎投稿</a> | <a href="http://www.guoxuedashi.com/forum/">意见建议</a> | <a href="http://www.guoxuemi.com/">国学迷</a> | <a href="http://www.shuowen.net/">说文网</a><script language="javascript" type="text/javascript" src="https://js.users.51.la/17753172.js"></script><br />
  Copyright &copy; 国学大师 古典图书集成 All Rights Reserved.<br>
  
  <span style="font-size:14px">免责声明:本站非营利性站点,以方便网友为主,仅供学习研究。<br>内容由热心网友提供和网上收集,不保留版权。若侵犯了您的权益,来信即刪。scp168@qq.com</span>
  <br />
ICP证:<a href="http://www.beian.miit.gov.cn/" target="_blank">鲁ICP备19060063号</a></div>
<!-- 页脚END --> 
<script src="http://www.guoxuedashi.com/img/plus.php?id=22"></script>
<script src="http://www.guoxuedashi.com/img/tongji.js"></script>

</body>
</html>
