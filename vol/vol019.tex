<!DOCTYPE html PUBLIC "-//W3C//DTD XHTML 1.0 Transitional//EN" "http://www.w3.org/TR/xhtml1/DTD/xhtml1-transitional.dtd">
<html xmlns="http://www.w3.org/1999/xhtml">
<head>
<meta http-equiv="Content-Type" content="text/html; charset=utf-8" />
<meta http-equiv="X-UA-Compatible" content="IE=Edge,chrome=1">
<title>資治通鑒_20-資治通鑑卷十九_20-資治通鑑卷十九</title>
<meta name="Keywords" content="資治通鑒_20-資治通鑑卷十九_20-資治通鑑卷十九">
<meta name="Description" content="資治通鑒_20-資治通鑑卷十九_20-資治通鑑卷十九">
<meta http-equiv="Cache-Control" content="no-transform" />
<meta http-equiv="Cache-Control" content="no-siteapp" />
<link href="/img/style.css" rel="stylesheet" type="text/css" />
<script src="/img/m.js?2020"></script> 
</head>
<body>
 <div class="ClassNavi">
<a  href="/24shi/">二十四史</a> | <a href="/SiKuQuanShu/">四库全书</a> | <a href="http://www.guoxuedashi.com/gjtsjc/"><font  color="#FF0000">古今图书集成</font></a> | <a href="/renwu/">历史人物</a> | <a href="/ShuoWenJieZi/"><font  color="#FF0000">说文解字</a></font> | <a href="/chengyu/">成语词典</a> | <a  target="_blank"  href="http://www.guoxuedashi.com/jgwhj/"><font  color="#FF0000">甲骨文合集</font></a> | <a href="/yzjwjc/"><font  color="#FF0000">殷周金文集成</font></a> | <a href="/xiangxingzi/"><font color="#0000FF">象形字典</font></a> | <a href="/13jing/"><font  color="#FF0000">十三经索引</font></a> | <a href="/zixing/"><font  color="#FF0000">字体转换器</font></a> | <a href="/zidian/xz/"><font color="#0000FF">篆书识别</font></a> | <a href="/jinfanyi/">近义反义词</a> | <a href="/duilian/">对联大全</a> | <a href="/jiapu/"><font  color="#0000FF">家谱族谱查询</font></a> | <a href="http://www.guoxuemi.com/hafo/" target="_blank" ><font color="#FF0000">哈佛古籍</font></a> 
</div>

 <!-- 头部导航开始 -->
<div class="w1180 head clearfix">
  <div class="head_logo l"><a title="国学大师官网" href="http://www.guoxuedashi.com" target="_blank"></a></div>
  <div class="head_sr l">
  <div id="head1">
  
  <a href="http://www.guoxuedashi.com/zidian/bujian/" target="_blank" ><img src="http://www.guoxuedashi.com/img/top1.gif" width="88" height="60" border="0" title="部件查字,支持20万汉字"></a>


<a href="http://www.guoxuedashi.com/help/yingpan.php" target="_blank"><img src="http://www.guoxuedashi.com/img/top230.gif" width="600" height="62" border="0" ></a>


  </div>
  <div id="head3"><a href="javascript:" onClick="javascript:window.external.AddFavorite(window.location.href,document.title);">添加收藏</a>
  <br><a href="/help/setie.php">搜索引擎</a>
  <br><a href="/help/zanzhu.php">赞助本站</a></div>
  <div id="head2">
 <a href="http://www.guoxuemi.com/" target="_blank"><img src="http://www.guoxuedashi.com/img/guoxuemi.gif" width="95" height="62" border="0" style="margin-left:2px;" title="国学迷"></a>
  

  </div>
</div>
  <div class="clear"></div>
  <div class="head_nav">
  <p><a href="/">首页</a> | <a href="/ShuKu/">国学书库</a> | <a href="/guji/">影印古籍</a> | <a href="/shici/">诗词宝典</a> | <a   href="/SiKuQuanShu/gxjx.php">精选</a> <b>|</b> <a href="/zidian/">汉语字典</a> | <a href="/hydcd/">汉语词典</a> | <a href="http://www.guoxuedashi.com/zidian/bujian/"><font  color="#CC0066">部件查字</font></a> | <a href="http://www.sfds.cn/"><font  color="#CC0066">书法大师</font></a> | <a href="/jgwhj/">甲骨文</a> <b>|</b> <a href="/b/4/"><font  color="#CC0066">解密</font></a> | <a href="/renwu/">历史人物</a> | <a href="/diangu/">历史典故</a> | <a href="/xingshi/">姓氏</a> | <a href="/minzu/">民族</a> <b>|</b> <a href="/mz/"><font  color="#CC0066">世界名著</font></a> | <a href="/download/">软件下载</a>
</p>
<p><a href="/b/"><font  color="#CC0066">历史</font></a> | <a href="http://skqs.guoxuedashi.com/" target="_blank">四库全书</a> |  <a href="http://www.guoxuedashi.com/search/" target="_blank"><font  color="#CC0066">全文检索</font></a> | <a href="http://www.guoxuedashi.com/shumu/">古籍书目</a> | <a   href="/24shi/">正史</a> <b>|</b> <a href="/chengyu/">成语词典</a> | <a href="/kangxi/" title="康熙字典">康熙字典</a> | <a href="/ShuoWenJieZi/">说文解字</a> | <a href="/zixing/yanbian/">字形演变</a> | <a href="/yzjwjc/">金 文</a> <b>|</b>  <a href="/shijian/nian-hao/">年号</a> | <a href="/diming/">历史地名</a> | <a href="/shijian/">历史事件</a> | <a href="/guanzhi/">官职</a> | <a href="/lishi/">知识</a> <b>|</b> <a href="/zhongyi/">中医中药</a> | <a href="http://www.guoxuedashi.com/forum/">留言反馈</a>
</p>
  </div>
</div>
<!-- 头部导航END --> 
<!-- 内容区开始 --> 
<div class="w1180 clearfix">
  <div class="info l">
   
<div class="clearfix" style="background:#f5faff;">
<script src='http://www.guoxuedashi.com/img/headersou.js'></script>

</div>
  <div class="info_tree"><a href="http://www.guoxuedashi.com">首页</a> > <a href="/SiKuQuanShu/fanti/">四库全书</a>
 > <h1>资治通鉴</h1> <!--         下载:【右键另存为】即可 --></div>
  <div class="info_content zj clearfix">
  
<div class="info_txt clearfix" id="show">
<center style="font-size:24px;">20-資治通鑑卷十九</center>
    資治通鑑卷十九    宋 司馬光 撰<br />
<br />
  胡三省 音註<br />
<br />
  漢紀十一【起彊圉大荒落盡玄黓閹茂凡六年】<br />
<br />
  世宗孝武皇帝中之上<br />
<br />
  元朔五年冬十一月乙丑薛澤免以公孫弘為丞相封平津侯【勃海郡高成縣有平津鄉宋白曰滄州鹽山縣勃海高成縣也冇平津郷 考異曰史記將相名臣表漢書公卿百官表弘為相皆在今年建元以來侯者表恩澤侯表皆云元朔三年封侯按三年弘始為御史大夫盖誤書五為三因置于三年耳】丞相封侯自弘始【漢初常以列侯為丞相弘則既相而後封侯故丞相封侯自弘始】時上方興功業弘于是開東閣以延賢人【師古曰閣小門也東向開之避當庭門而引客别於掾史官屬也】與參謀議每朝覲奏事因言國家便宜上亦使左右文學之臣與之論難【難乃旦翻】弘嘗奏言十賊彍弩【張晏曰彍音郭師古曰引滿曰彍】百吏不敢前請禁民毋得挾弓弩便上下其議【下遐嫁翻】侍中吾丘夀王對曰臣聞古者作五兵【師古曰五兵謂矛戟弓劒戈吾讀曰虞】非以相害以禁暴討邪也秦兼天下銷甲兵折鋒刃其後民以耰鉏箠梃相撻擊【師古曰耰摩田之器也箠馬撾也梃大杖也折而設翻耰音憂梃大鼎翻撻音闥】犯法滋衆盗賊不勝【師古曰滋益也不勝言不可勝也】卒以亂亡【卒子恤翻】故聖王務教化而省禁防知其不足恃也禮曰男子生桑弧蓬矢以舉之明示有事也【記内則國君世子生三日射人以桑弧蓬矢六射天地四方注云天地四方男子之所有事也】大射之禮自天子降及庶人三代之道也【古者天子射豹侯諸侯射熊侯卿大夫射麋侯士射鹿侯豕侯周官又以郷射之禮詢衆庶】愚聞聖王合射以明教矣未聞弓矢之為禁也且所為禁者為盗賊之以攻奪也【為盗之為于偽翻】攻奪之罪死然而不止者大姧之於重誅固不避也臣恐邪人挾之而吏不能止良民以自備而抵法禁【師古曰抵觸也】是擅賊威而奪民救也竊以為大不便書奏上以難弘弘詘服焉【難乃旦翻詘與屈同】弘性意忌外寛内深諸嘗與弘有隙無近遠雖陽與善後竟報其過董仲舒為人廉直以弘為從諛弘嫉之膠西王端驕恣數犯汝【端景帝子前三年受封數所角翻下同】所殺傷二千石甚衆弘乃薦仲舒為膠西相仲舒以病免汲黯常毁儒面觸弘弘欲誅之以事【以事致其罪而誅之】乃言上曰右内史界部中多貴臣宗室難治非素重臣不能任請徙黯為右内史【右内史後為右扶風治直之翻任音壬】上從之春大旱 匈奴右賢王數侵擾朔方天子令車騎將軍青將三萬騎出高闕衛尉蘇建為游擊將軍左内史李沮為彊弩將軍【沮音俎】太僕公孫賀為衛騎將軍代相李蔡為輕車將軍皆領屬車騎將軍俱出朔方大行李息岸頭侯張次公為將軍俱出右北平凡十餘萬人擊匈奴右賢王以為漢兵遠不能至飲酒醉衛青等兵出塞六七百里夜至圍右賢王右賢王驚夜逃獨與壮騎數百馳潰圍北去得右賢禆王十餘人【師古曰禆王小王也猶禆將也禆頻移翻】衆男女萬五千餘人畜數十百萬【師古曰數十萬以至百萬畜許敕翻】于是引兵而還至塞天子使使者持大將軍印即軍中拜衛青為大將軍諸將皆屬焉夏四月乙未復益封青八千七百 【復扶戶又翻】封青三子伉不疑登皆為列侯【師古曰伉音抗又工郎翻伉為宜春侯不疑為隂安侯登為發干侯】青固謝曰【師古曰固謂再三也】臣幸得待罪行間【行戶剛翻】賴陛下神靈軍大捷皆諸校尉力戰之功也陛下幸已益封臣青臣青子在襁褓中未有勤勞上列地封為三侯【列漢書作裂】非臣待罪行間所以勸士力戰之意也天子曰我非忘諸校尉功也乃封護軍都尉公孫敖為合騎侯【晉灼曰合騎侯猶冠軍從票之名也予據功臣表合騎侯食邑于勃海高成】都尉韓說為龍頟侯【班志龍頟侯國属平原郡頟音洛】公孫賀為南窌侯【窌匹孝翻又普孝翻】李蔡為樂安侯【樂安功臣表作安樂食邑於琅邪之昌縣】校尉李朔為涉軹侯【涉軹班史衛青傳作陟軹功臣表作軹食邑于齊郡之西安】趙不虞為隨成侯【隨成侯功臣表食邑于干乘縣】公孫戎奴為從平侯【從平侯食邑於東郡樂昌】李沮李息及校尉豆如意【班史豆作竇】皆賜爵關内侯於是青尊寵於羣臣無二公卿以下皆卑奉之獨汲黯與亢禮【亢音抗】人或說黯曰自天子欲羣臣下大將軍【說式芮翻師古曰下遐嫁翻】大將軍尊重君不可以不拜黯曰夫以大將軍有揖客反不重邪【師古曰言能降貴以禮士最為重也】大將軍聞愈賢黯數請問國家朝廷所疑【數所角翻】遇黯加於平日大將軍青雖貴有時侍中上踞厠而視之【如淳曰厠溷也孟康曰厠牀邊側也師古曰如說是也仲馮曰厠當從孟說占者見大臣則御坐為起然則踞厠者輕之也】丞相弘燕見上或時不冠至如汲黯見【見賢遍翻】上不冠不見也上嘗坐武帳中【應劭曰武帳織成帳為武士象也孟康曰今御武帳置兵闌五兵于帳中也師古曰孟說是韋昭曰以武名之示威】黯前奏事上不冠望見黯避帳中使人可其奏其見敬禮如此 夏六月詔曰盖聞導民以禮風之以樂【師古曰風教也詩序曰上以風化下】今禮壞樂崩朕甚閔焉其令禮官勸學興禮以為天下先于是丞相弘等奏請為博士官置弟子五十人復其身【為于偽翻復方目翻】第其高下以補郎中文學掌故【兒寛以射策為掌故功次補廷尉文學卒史蘇林曰卒史秩六百石臣瓚曰漢注卒史秩百石師古曰瓚說是予謂掌故掌故府之典籍者也以兒寛自掌故補卒史推之則掌故之品秩從可知也】即有秀才異等輒以名聞【秀才異等謂冇俊秀之才異於常等者】其不事學若下材輒罷之又吏通一藝以上者請皆選擇以補右職【吏謂百石已上及比百石以下也右職謂中二千石二千石之卒史也】上從之自此公卿大夫士吏彬彬多文學之士矣 秋匈奴萬騎入代殺都尉朱英略千餘人 初淮南王安好讀書屬文喜立名譽【好呼到翻屬之欲翻喜許記翻】招致賓客方術之士數千人其羣臣賓客多江淮間輕薄士常以厲王遷死感激安【遷死見十四卷文帝前六年】建元六年彗星見【彗祥歲翻又徐醉翻又旋芮翻見賢遍翻】或說王曰先吳軍時彗星出長數尺然尚流血千里【說式芮翻先悉薦翻長直亮翻謂吳王濞起兵時也】今彗星竟天天下兵當大起王心以為然乃益治攻戰具積金錢【治直之翻下同】郎中雷被獲罪于太子遷【雷被善用劒與太子戲誤中太子故得罪師古曰被皮義翻姓譜雷古方雷氏後】時有詔欲從軍者輒詣長安被即願奮擊匈奴太子惡被於王【惡毁惡也如字】斥免之欲以禁後【師古曰令後人更不敢效之也】是歲被亡之長安上書自明事下廷尉治【下遐嫁翻】蹤跡連王公卿請逮捕治王太子遷謀令人衣衛士衣持戟居王旁漢使有非是者即刺殺之【人衣於既翻刺七亦翻】因發兵反天子使中尉宏即訊王【師古曰即就也就問也】王視中尉顔色和遂不發公卿奏安壅閼奮擊匈奴者格明詔當弃市【閼音遏師古曰格音閣謂閣止不行之】詔削二縣既而安自傷曰吾行仁義反見削地耻之于是為反謀益甚安與衡山王賜相責望禮節間不相能【賜即安之弟也孝文十六年與安同受封師古曰兄弟相責故有嫌】衡山王聞淮南王有反謀恐為所并亦結賓客為反具以為淮南已西欲發兵定江淮之間而有之衡山王后徐來譛太子爽於王欲廢之而立其弟孝王囚太子而佩孝以王印令招致賓客賓客來者微知淮南衡山有逆計日夜從容勸之【從千容翻】王乃使孝客江都人枚赫陳喜作輣車鍜矢【輣薄庚翻兵車也樓車也鍛都元翻冶鐵也】刻天子璽將相軍吏印秋衡山王當入朝過淮南淮南王乃昆弟語【師古曰為相親愛之言】除前隙約束反具【師古曰共契約為反具】衡山王即上書謝病上賜書不朝<br />
<br />
  六年春二月大將軍青出定襄擊匈奴【杜佔曰漢定襄郡在今馬邑北三百餘里後魏置雲中郡】以合騎侯公孫敖為中將軍太僕公孫賀為左將軍翕侯趙信為前將軍【功臣表翕侯國在魏郡内黄界】衛尉蘇建為右將軍郎中令李廣為後將軍左内史李沮為彊弩將軍【師古曰沮音俎】咸屬大將軍斬首數千級而還【賢曰秦法斬首一賜爵一級故因謂斬首為級】休士馬于定襄雲中雁門 赦天下夏四月衛青復將六將軍出定襄擊匈奴【復扶又翻】斬首虜萬餘人右將軍建前將軍信并軍三千餘騎獨逢單于兵與戰一日餘漢兵且盡信故胡小王降漢漢封為翕侯【信元光四年十月壬午受封】及敗匈奴誘之遂將其餘騎可八百降匈奴【誘音酉將即亮翻降戶江翻】建盡亡其軍脱身亡自歸大將軍議郎周覇曰【班表議郎屬郎中令秩比六百石】自大將軍出未嘗斬裨將今建棄軍可斬以明將軍之威軍正閎長史安曰不然【凡軍行置軍正掌舉軍法以正軍中軍法曰正無屬將軍將軍有罪以聞劉昭志大將軍長史秩千石如淳曰律都軍官長史一人】兵法小敵之堅大敵之禽也【孫子之言言大小不敵小雖堅於戰終必為大所禽】今建以數千當單于數萬力戰一日餘士盡不敢有二心自歸而斬之是示後無反意也不當斬大將軍曰青幸得以肺腑待罪行間不患無威而覇說我以明威甚失臣意【言失為臣之意也行戶剛翻說式芮翻】且使臣職雖當斬將【將即亮翻】以臣之尊寵而不敢擅誅於境外而具歸天子天子自裁之於以見為人臣不敢專權不亦可乎軍吏皆曰善遂囚建詣行在所【蔡邕獨斷曰天子以四海為家故謂所居為行在所】初平陽縣吏霍仲孺給事平陽侯家與青姊衛少兒私通生霍去病【霍姓以國為氏】去病年十八為侍中善騎射再從大將軍擊匈奴為票姚校尉【服䖍曰票姚音飄揺師古曰票匹妙翻姚羊召翻票姚勁疾之貌苟悦漢紀作票鷂字去病後為票騎將軍尚取票姚之字耳今讀者音飄揺則不當其義也】與輕勇騎八百直棄大軍數百里赴利斬捕首虜過當【師古曰計其所將人數則捕斬首為多過於所當一曰漢軍失亡者少而殺獲匈奴數多故曰過當也】於是天子曰票姚校尉去病斬首虜二千餘級得相國當戶斬單于大父行藉若侯產生捕季父羅姑【匈奴左右大當戶在左右大都尉之下左右骨都侯之上大父行單于祖行也張晏曰藉若胡侯也產其名也師古曰此人單于祖父之行也季父亦單于季父也羅姑其名行戶郎翻】比再冠軍【師古曰比頻也比毗至翻冠古玩翻】封去病為冠軍侯【帝以去病功冠諸軍以南陽穰縣盧陽郷宛縣臨駣聚為冠軍侯國駣音桃】上谷太守郝賢四從大將軍捕斬首虜二千餘級封賢為衆利侯【姓譜殷帝乙有子期封太原郝郷後因氏焉功臣表衆利侯食邑于琅邪郡姑幕縣】是歲失兩將軍亡翕侯軍功不多故大將軍不益封止賜千金右將軍建至天子不誅贖為庶人單于既得翕侯以為自次王【師古曰自次者尊重次於單于】用其姊妻之【妻七細翻】與謀漢信教單于益北絶幕【師古曰直度曰絶幕與漠同隂山以北皆大漠不生草木】以誘罷漢兵徼極而取之【師古曰罷讀曰疲徼要也誘令疲徼其困極然後取之徼一遥翻】無近塞單于從其計【近其靳翻】是時漢比歲發十餘萬衆擊胡【比毗至翻】斬捕首虜之士受賜黄金二十餘萬斤而漢軍士馬死者十餘萬兵甲轉漕之費不與焉【與讀曰預】于是大司農經用竭不足以奉戰士六月詔令民得買爵及贖禁錮免臧罪置賞官名曰武功爵級十七萬凡直三十餘萬金諸買武功爵至千夫者得先除為吏【禁錮重擊也臣瓚曰茂陵中書有武功爵一級曰造士二級曰閑輿衛三級曰良士四級曰元戎士五級曰官首六級曰秉鐸七級曰千夫八級曰樂卿九級曰執戎十級曰政戾庶長十一級曰軍衛此武帝所制以寵軍功師古曰下云級十七萬凡直三十餘萬金今瓚引茂陵中書說之不盡也貢父曰直三十餘萬金其價之差殊不可詳也或說七當作一與茂陵書合矣予謂賣爵當級級稍增其價豈可例云級十七萬若每級十七萬比至三十餘萬金當一萬七千餘級又非也然則誤衍此萬字盖武功爵其級十七參考顔劉注皆因求其說而不得遂疑茂陵書所謂十一級為不足又疑史之正文萬字為衍皆未為允也盖級十七萬者賣爵一級為錢十七萬至二級則三十四萬矣自此以上烏得不每級而增乎王莽時黄金一斤直錢萬以此推之則三十萬金為錢三十餘萬萬矣此當時鬻武功爵所直之數也夫民入錢買爵隨其錢之多少為爵級之高下爵之高下有定直而民錢之多少無定數若比而同之其失彌遠矣史記作直八十萬金索隱曰一金萬錢初一級十七萬自此以上每級加二萬至十七級合成三十四萬也】吏道雜而多端官職耗廢矣【師古曰耗亂也音莫報翻】<br />
<br />
  元狩元年冬十月上行幸雍祠五畤【雍於用翻畤音止】獲獸一角而足有五蹄有司言陛下肅祗郊祀上帝報享錫一角獸盖麟云【麟麋身牛尾馬足五色圜蹄一角角端有肉音中鍾呂行中規矩游必擇地詳而後處不履生蟲不踐生草不羣居不侶行不入陷穽不罹羅網王者至仁則出今并州界有麟大小如鹿非瑞應麟也京房易傳曰麟麕身牛尾馬蹄有五采腹下黄高丈二爾雅麟麕身牛尾一角盖麟似麕圓頂一角曰盖云者意其為麟而未知其果為麟也】於是以慶五畤畤加一牛以燎【畤音止】久之有司又言元宜以天瑞命不宜以一二數一元曰建二元以長星曰光今元以郊得一角獸曰狩云於是濟北王【濟北王勃淮南厲王子孝文十六年封衡山王孝景四年徙封濟北今王勃子成王胡也濟北王都盧後天漢四年國除入漢為太山郡濟子禮翻】以為天子且封禪上書獻太山及其旁邑天子以他縣償之 淮南王安與賓客左吳等日夜為反謀【姓譜齊公族冇左右公子後因氏馬予按衛亦有左右公子姓譜之說非是魯冇左丘明】案輿地圖【蘇林曰輿猶盡載之意索隱曰志林云輿地圖漢家所畫非出遠也】部署兵所從入諸使者道長安來為妄言言上無男漢不治即喜即言漢廷治有男王怒以為妄言非也【治直吏翻】王召中郎伍被【被皮義翻姓譜伍姓出于楚伍舉】與謀反事被曰王安得此亡國之言乎臣見宫中生荆棘露霑衣也王怒繫伍被父母囚之三月復召問之【復扶又翻】被曰㫺秦為無道窮奢極虐百姓思亂者十家而六七高皇帝起於行陳之中【行戶剛翻陳讀曰陣】立為天子此所謂蹈瑕候間【間古莧翻】因秦之亡而動者也今大王見高皇帝得天下之易也【易以䜴翻】獨不觀近世之吳楚乎【事見十五卷景帝三年】夫吳王王四郡【四郡束陽郡鄣郡吳郡豫章郡王王下于况翻】國富民衆計定謀成舉兵而西然破於大梁【謂為梁孝王所破也】奔走而東身死祀絶者何誠逆天道而不知時也方今大王之兵衆不能十分吳楚之一天下安寜萬倍吳楚之時大王不從臣之計今見大王棄千乘之君賜絶命之書為羣臣先死於東宫也【如淳曰東宫淮南王所居也】王涕泣而起王有孽子不害最長【庶生曰孽長知兩翻】王弗愛王后太子皆不以為子兄數【言后不以為子大子不以為兄數秩數也】不害有子建材高有氣常怨望太子隂使人告太子謀殺漢中尉事【事見上元朔五年】下廷尉治【下遐嫁翻】王患之欲發復問伍被曰【復扶又翻】公以為吳興兵是邪非邪被曰非也臣聞吳王悔之甚願王無為吳王之所悔王曰吳何知反漢將一日過成臯者四十餘人今我絶成臯之口據三川之險【漢河南秦三川郡也其地當伊洛河三川之會】招山東之兵舉事如此左吳趙賢朱驕如皆以為什事九成公獨以為有禍無福何也必如公言不可儌幸邪【師古曰徼要也幸非妄之福也徼一堯翻】被曰必不得已被有愚計當今諸侯無異心百姓無怨氣可偽為丞相御史請書【言偽為丞相御史奏請于天子之書】徙郡國豪傑高貲於朔方益發甲卒急其會日又偽為詔獄書【漢時左右都司空上林中都官皆有詔獄盖奉詔以鞫囚因以為名】逮諸侯太子幸臣【逮追對獄也】如此則民怨諸侯懼即使辨士隨而說之【說式芮翻下同】儻可儌幸什得一乎王曰此可也雖然吾以為不至若此【言不須為此也】于是王乃作皇帝璽丞相御史大夫將軍軍吏中二千石及旁近郡太守都尉印漢使節【使疏吏翻】欲使人偽得罪而西【言使人詐為得罪而逃去西如京師】事大將軍一日發兵【一日猶言一旦】即刺殺大將軍【刺七亦翻】且曰漢廷大臣獨汲黯好直諫【好呼到翻】守節死義難惑以非至如說丞相弘等如發蒙振落耳【發蒙謂物所蒙覆發而去之振落謂木葉將落振而墜之皆言其易說式芮翻】王欲發國中兵恐其相二千石不聽王乃與伍被謀先殺相二千石又欲令人衣求盗衣【求盗卒也掌逐捕盗賊漢書本紀高帝時為亭長令求盗之薛治竹皮冠人衣於既翻】持羽檄從東方來呼曰南越兵入界【呼火故翻】欲因以發兵會廷尉逮捕淮南太子淮南王聞之與太子謀召相二千石欲殺而發兵召相相至内史中尉皆不至王念獨殺相無益也即罷相【罷遣出去也相息亮翻】王猶豫計未决太子即自剄不殊【晉灼曰不殊不死也師古曰言雖自刑而身首不能絶也剄古頂翻下同】伍被自詣吏告與淮南王謀反蹤跡如此吏因捕太子王后圍王宫盡求捕王所與謀反賓客在國中者索得反具以聞上下公卿治其黨與【索山客翻求也搜也上時掌翻下遐嫁翻聞上句斷】使宗正以符節治王未至淮南王安自剄殺王后荼太子遷諸所與謀反者皆族天子以伍被雅辭多引漢之美欲勿誅【雅素也雅辭素來言語也】廷尉湯曰被首為王畫反計【為于偽翻】罪不可赦乃誅被侍中莊助素與淮南王相結交私論議王厚賂遺助【遺于季翻】上薄其罪欲勿誅張湯争以為助出入禁門腹心之臣而外與諸侯交私如此不誅後不可治助竟棄市衡山王上書請廢太子爽立其弟孝為太子爽聞即遣所善白嬴之長安上書言孝作輣車鍜矢與王御者姦欲以敗孝【敗蒲邁翻】會有司捕所與淮南王謀反者得陳喜于衡山王子孝家吏劾孝首匿喜【師古曰為頭首而藏匿之】孝聞律先自告除其罪即先自告所與謀反者枚赫陳喜等公卿請逮捕衡山王治之王自剄死王后徐來太子爽及孝皆棄市所與謀反者皆族凡淮南衡山二獄所連引列侯二千石豪傑等死者數萬人 夏四月赦天下 丁卯立皇子據為太子年七歲 五月乙巳晦日有食之匈奴萬人入上谷殺數百人 初張騫自月氏還【事見上卷元朔四年氏音支】具為天子言西域諸國風俗【為于偽翻】大宛在漢正西可萬里其俗土著耕田【土著謂有城郭常居不隨水草移徙也宛於元翻著直畧翻】多善馬馬汗血【孟康曰大宛國有高山其上有馬不可得因取五色母馬置其下與集生駒皆汗血因號天馬子云一說汗血者汗從肩膊出如血號能一日千里】有城郭室屋如中國其東北則烏孫東則于窴【于窴國在南山下居西城窴徒賢翻又徒見翻】于窴之西則水皆西流注西海【水經註崑崙山西有大水名新頭河度葱嶺入北天竺境又西南流屈而東南流逕中天竺國又西逕安息南注於雷翥海雷翥海即西海在安息之西犂軒之東東南連交州海】其東水東流注鹽澤【水經註河水一源出于窴國南山北流與葱嶺河合東注蒲昌海西域傳鹽澤一名蒲昌海去玉門陽關三百餘里廣袤三百里其水停居冬夏不增减皆以為潜行地下南出於積石為中國河云玉門陽關皆在敦煌西界括地志蒲昌海一名泑澤亦名鹽澤亦名輔日海亦名穿蘭亦名臨海在沙州西南玉門關在沙州夀昌縣西六里】鹽澤濳行地下其南則河源出焉【索隱曰按漢書西南夷傳云河有兩源其一出葱嶺一出于窴山海經云河出崑崙東北隅郭璞云河出崑崙潜行地下至蒽嶺山于窴國復分流岐出合而東注泑澤已而復行積石為中國河泑澤即鹽澤也西域傳云于窴在南山下與郭璞註山海經不同廣志云蒲昌海在蒲類海東唐長慶中劉元鼎為盟會使言河之上流由洪濟西南行二千里水益狹冬春可涉夏秋乃勝舟其南三百里三山中高四下曰歷山直大羊同國古所謂昆侖者也虜曰問摩黎山東距長安五千里河源其間流澄緩下稍合衆流色赤行益遠它水并注則濁河源東北直莫賀延磧尾隱測其地盖劔南之西】鹽澤去長安可五千里匈奴右方居鹽澤以東至隴西長城【即秦所築長城也秦築長城起臨洮臨洮縣漢屬隴西郡】南接羌鬲漢道焉【鬲與隔同】烏孫康居奄蔡大月氏皆行國隨畜牧【奄蔡國在康居西北臨大澤無涯盖北海云隨畜牧逐水草而居無城郭常處故曰行國】與匈奴同俗大夏在大宛西南與大宛同俗臣在大夏時見卭竹杖蜀布【臣瓚曰卭山名生竹高節可作杖服䖍曰蜀布細布也史記正義曰卭都卭山出此竹因名卭竹節高實中或奇生可為杖布土蘆布卭渠容翻】問曰安得此大夏國人曰吾賈人往市之身毒【孟康曰身毒即天竺也所謂浮屠胡也鄧展曰毒音篤李奇曰一名天篤師古曰亦曰捐毒賈音古索隱曰身音乾】身毒在大夏東南可數千里其俗土著與大夏同以騫度之【著直畧翻度徒洛翻】大夏去漢萬二千里居漢西南今身毒國又居大夏東南數千里有蜀物此其去蜀不遠矣今使大夏從羌中險羌人惡之【使疏吏翻惡烏路翻】少北則為匈奴所得【少詩沼翻】從蜀宜徑又無寇【師古曰宜當也逕直也從蜀向大夏其道當直】天子既聞大宛及大夏安息之屬【安息治番兜城臨媯水去長安萬一千六百里其俗亦土著】皆大國多奇物土著頗與中國同業而兵弱貴漢財物其北有大月氏康居之屬兵強可以賂遺設利朝也【師古曰設施也施之以利誘令入朝遺于季翻朝直遥翻】誠得而以義屬之【師古曰謂不以兵革】則廣地萬里重九譯【譯傳言之人周官象胥之職也遠方之人言語不同更歷九譯乃能通於中國重直龍翻】致殊俗威德徧於四海欣然以騫言為然乃令騫因蜀犍為發間使王然于等四道並出【師古曰間使者求問隙而行間古莧翻使疏吏翻】出駹出冉出徙出卭僰指求身毒國【徙斯榆也以手點物為指使之出求路指身毒而行徙讀與斯同僰蒲墨翻】各行一二千里其北方閉氐莋南方閉嶲昆明【服䖍曰漢使見閉於夷也師古曰嶲即今嶲州也昆明又在其西南即今南寜州諸㸑所居是其地莋音昨又音作嶲先蕊翻】昆明之屬無君長善寇盗輒殺畧漢使終莫得通於是漢以求身毒道始通滇國【滇國地有滇池因以名國楚使莊蹻以兵定夜郎諸國至滇池因留王其地華陽國志滇池周回三百里所出深廣下流淺狹如倒流故謂之滇池漢為益州郡後改為永昌郡魏晉之間為晉寜郡唐為昆州括地志滇池澤在昆州晉寜縣西南三十里長知兩翻滇音顛】滇王當羌謂漢使者曰漢孰與我大及夜郎侯亦然以道不通故各自以為一州主不知漢廣大使者還因盛言滇大國足事親附天子注意焉乃復事西南夷【元朔四年罷西夷至是復通師古曰事謂經畧通之專以為事也復扶又翻】<br />
<br />
  二年冬十月上幸雍祠五畤【雍於用翻畤音止】 三月戊寅平津獻侯公孫弘薨壬辰以御史大夫樂安侯李蔡為丞相廷尉張湯為御史大夫【考異曰漢書百官公卿表元狩三年三月壬辰廷尉張湯為御史大夫六年有罪自殺史記將相名臣表元狩二年御史大夫湯按李蔡既遷湯即應補其缺豈可留之朞年復與李蔡為丞相月日正同乎又按長歷三年三月無壬辰又以得罪之年推之在今年明矣今從史記表】霍去病為票騎將軍【票騎將軍始此票頻妙翻】將萬騎出隴西擊匈奴歷五王國轉戰六日過焉支山千餘里【括地志焉支山一名刪丹山在甘州刪丹縣東南五十里焉音烟】殺折蘭王斬盧侯王【張晏曰折蘭盧侯胡國名也殺者殺之而已斬者獲其首也師古曰折蘭匈奴中姓也今鮮卑中有是蘭姓者即其種也折上列翻】執渾邪王子【師古曰渾下昆翻】及相國都尉獲首虜八千九百餘級收休屠王祭天金人【孟康曰匈奴祭天處本在雲陽甘泉山下秦擊奪其地後徙之休屠王右地故休屠王有祭天金人像也如淳曰祭天以金人為主也張晏曰佛徒祠金人也師古曰作金人以為天神之像而祭之今之佛像是其遺法屠音儲】詔益封去病二千戶夏去病復與合騎侯公孫敖將數萬騎俱出北地異道【復扶又翻】衛尉張騫郎中令李廣俱出右北平異道廣將四千騎先行可數百里騫將萬騎在後匈奴左賢王將四萬騎圍廣廣軍士皆恐廣乃使其子敢獨與數十騎馳貫胡騎【貫穿也】出其左右而還告廣曰胡虜易與耳【易以䜴翻】軍士乃安廣為圜陳外嚮【陳讀曰陣】胡急擊之矢下如雨漢兵死者過半漢矢且盡廣乃令士持滿毋發【師古曰注矢於弓弩而引滿之不發矢也】而廣身自以大黄射其裨將殺數人【徐廣曰南都賦黄間機張善弩之名裴駰曰案鄭德曰黄肩弩淵中黄朱之孟康曰太公六韜云陷堅敗強敵用大黄連弩韋昭曰角弩色黄而體大也射而亦翻】胡虜益解會日暮吏士皆無人色【師古曰言懼甚】而廣意氣自如【師古曰自如猶云如舊】益治軍【師古曰廵部曲整行陳也治直之翻】軍中皆服其勇明日復力戰【復扶又翻】死者過半所殺亦過當會博望侯軍亦至【張騫從大將軍撃匈奴知水艸處軍得以不乏封博望侯師古曰取其能廣博瞻望班志博望侯國属南陽郡括地志博望故城在鄧州向城縣東南四十五里】匈奴軍乃解去漢軍罷【罷讀曰疲】弗能追罷歸漢法博望侯留遲後期當死贖為庶人廣軍功自如無賞【自如言功過正相當也廣軍失亡多而殺虜亦過當故曰自如】而票騎將軍去病深入二千餘里與合騎侯失不相得票騎將軍踰居延【居延澤古文以為流沙帝開置居延縣屬張掖郡使路博德築遮虜障于其北】過小月氐【匈奴破大月氐月氐西擊大夏而臣之其餘小衆不能去者保南山羌號小月氐】至祁連山得單桓酋涂王【張晏曰單桓酋涂皆胡王也師古曰酋才猶翻涂音塗】及相國都尉以衆降者二千五百人【降戶江翻】斬首虜三萬二百級獲裨小王七十餘人天子益封去病五千戶封其裨將有功者鷹擊司馬趙破奴為從票侯【以從票騎有功因以為號功臣侯表不書食邑之地】校尉高不識為宜冠侯【功臣表宜冠侯食邑於琅邪之昌縣】校尉僕多為煇渠侯【僕多本匈奴種來降漢功臣表僕多作僕朋煇渠侯食邑于南陽之魯陽縣】合騎侯敖坐行留不與票騎會當斬贖為庶人是時諸宿將所將士馬兵皆不如票騎票騎所將常選【師古曰選取驍鋭索隱曰選宣變翻】然亦敢深入常與壮騎先其大軍【先悉薦翻】軍亦有天幸未嘗困絶也而諸宿將常留落不偶【師古曰留謂遲留落謂墜落故不諧耦而無功也】由此票騎日以親貴比大將軍矣匈奴入代雁門殺畧數百人 江都王建【建易王非之子景帝之孫】與其父易王所幸淖姬等及女弟徵臣姦【淖鄭氏音卓師古音奴教翻淖姓也戰國時楚有淖齒】建游雷陂【雷陂即廣陵之雷塘在今揚州堡城之北平岡之上】天大風建使郎二人乘小船入陂中船覆兩郎溺攀船乍見乍没【見賢徧翻】建臨觀大笑令勿救皆死凡殺不辜三十五人專為淫虐自知罪多恐誅與其后成光共使越婢下神祝詛上【祝織救翻詛莊助翻】又聞淮南衡山隂謀建亦作兵器刻皇帝璽為反具事發覺有司請捕誅建自殺后成光等皆棄市國除 膠東康王寄薨【寄景帝子中二年受封】 秋匈奴渾邪王降是時單于怒渾邪王休屠王居西方為漢所殺虜數萬人欲召誅之渾邪王與休屠王恐謀降漢先遣使向邊境要遮漢人【要一遥翻】令報天子是時大行李息將城河上得渾邪王使【使疏吏翻】馳傳以聞【傳張戀翻下同】天子聞之恐其以詐降而襲邊乃令票騎將軍將兵往迎之休屠王後悔渾邪王殺之并其衆票騎既渡河與渾邪王衆相望渾邪王禆將見漢軍而多不欲降者【師古曰恐被掩覆也】頗遁去票騎乃馳入得與渾邪王相見斬其欲亡者八千人遂獨遣渾邪王乘傳詣至行在所【傳音張戀翻】盡將其衆渡河降者四萬餘人號稱十萬既至長安天子所以賞賜者數十巨萬封渾邪王萬戶為漯隂侯【班志漯隂縣屬平原郡漯他合翻】封其禆王呼毒尼等四人皆為列侯【呼毒尼為下摩侯雁疪為煇渠侯禽黎為河綦侯文當戶調雖為常樂侯文穎曰雁音鷹疪音庇廕之庇師古曰疪匹履翻】益封票騎千七百戶渾邪之降也漢發車二萬乘以迎之【考異曰漢書食貨志云三萬兩今侯史記平凖書汲黯傳】縣官無錢從民貰馬【貰始制翻貸也師古曰賖買也】民或匿馬馬不具上怒欲斬長安令右内史汲黯曰長安令無罪獨斬臣黯民乃肯出馬且匈奴畔其主而降漢漢徐以縣次傳之何至令天下騷動罷敝中國【罷讀曰疲】而以事夷狄之人乎上默然及渾邪至賈人與市者坐當死五百餘人黯請間見高門【晉灼曰三輔黄圖未央宫中有高門殿賈音古見賢遍翻】曰夫匈奴攻當路塞【言塞障當匈奴所入之路也】絶和親中國興兵誅之死傷者不可勝計【勝音升】而費以巨萬百數【師古曰即數百鉅萬也】臣愚以為陛下得胡人皆以為奴婢以賜從軍死事者家所鹵獲因予之【鹵與虜同予讀曰與】以謝天下之苦塞百姓之心【師古曰塞滿也塞悉則翻】今縱不能渾邪率數萬之衆來降虚府庫賞賜發良民侍養譬若奉驕子愚民安知市買長安中物而文吏繩以為闌出財物于邊關乎【應劭曰闌妄也律胡市吏民不得持兵器及錢出關雖於京師市買其法一也臣瓚曰無符傳出入為闌也】陛下縱不能得匈奴之資以謝天下又以微文殺無知者五百餘人是所謂庇其葉而傷其枝者也臣竊為陛下不取也【為于偽翻】上默然不許曰吾久不聞汲黯之言今又復妄發矣居頃之乃分徙降者邊五郡故塞外而皆在河南因其故俗為五屬國【五郡謂隴西北地丄郡朔方雲中也故塞秦之先與匈奴所關之塞自秦使蒙恬奪匈奴地而邊關益斥秦項之亂冒頓南侵與中國關於故塞及衛青收河南而邊關復蒙恬之舊所謂故塞外其地在北河之南也師古曰凡言屬國存其國號而屬漢朝故曰屬國史記正義曰以來降之民徙置五郡各依本國之俗而屬於漢故曰屬國】而金城河西【河水出金城河關縣西南塞外積石山東流逕金城郡界自允吾以西通謂之金城河渡河而西則武威等四郡之地然金城郡昭帝於元始六年方置史追書也】西並南山至鹽澤空無匈奴【並步浪翻】匈奴時有候者到而希矣休屠王太子日磾與母閼氏弟倫俱没入官輸黄門養馬久之【磾丁奚翻班表黄門屬少府師古曰黄門之署職任親近以供天子百物在焉閼氏音烟支】帝游宴見馬【師古曰方於游宴之時而召閲諸馬】後宫滿側日磾等數十人牽馬過殿下莫不竊視【師古曰視宫人】至日磾獨不敢日磾長八尺二寸【長直亮翻】容貌甚嚴馬又肥好上異而問之具以本狀對上奇焉即日賜湯沐衣冠拜為馬監【黄門有馬監狗監】遷侍中駙馬都尉光禄大夫【侍中得出入禁中駙馬都尉帝所置秩比二千石師古曰駙副馬也非正駕車皆為副馬一曰駙近也疾也光禄大夫本中大夫帝改其名】日磾既親近【近其靳翻】未嘗有過失上甚信愛之賞賜累千金出則驂乘【乘繩正翻】入侍左右貴戚多竊怨曰陛下妄得一胡兒反貴重之上聞愈厚焉以休屠作金人為祭天主故賜日磾姓金氏【為金氏貴顯張本】<br />
<br />
  三年春有星孛于東方【孛蒲内翻】 夏五月赦天下 淮南王之謀反也膠東康王寄微聞其事私作戰守備及吏治淮南事辭出之【師古曰獄辭所連發出其事】寄母王夫人即皇太后之女弟也於上㝡親意自傷發病而死不敢置後上聞而憐之立其長子賢為膠東王【康王寄去年薨今年方置後】又封其所愛少子慶為六安王王故衡山王地【衡山國都六故改為六安】秋匈奴入右北平定襄各數萬騎殺畧千餘人 山<br />
<br />
  東大水民多飢乏天子遣使者虛郡國倉廥以振貧民【廥工外翻芻藁之藏也一曰庫廐名】猶不足又募豪富吏民能假貸貧民者以名聞尚不能相救乃徙貧民於關以西及充朔方以南新秦中【應劭曰秦遣蒙恬却匈奴得其河南造陽之地千里地甚好于是為築城郭徙民充之名曰新秦四方錯雜奢險不同今俗名新富貴者為新秦由是名也】七十餘萬口衣食皆仰給縣官數歲假予產業使者分部護之【仰牛向翻予讀曰與分扶問翻】冠盖相望其費以億計不可勝數【勝音升】 漢既得渾邪王地隴西北地上郡益少胡寇詔减三郡戍卒之半以寛天下之繇【繇讀曰徭】 上將討昆明【以其閉漢使故也】以昆明有滇池方三百里乃作昆明池以習水戰【昆明池在長安西南周回四十里三輔舊事昆明池盖地三百二十頃】是時法既益嚴吏多廢免兵革數動【數所角翻】民多買復【師古曰入財于官以取優復復方目翻】及五大夫【五大夫舊爵二十等之第九級也漢法至此始免徭役】徵發之士益鮮【鮮少也先淺翻】於是除千夫五大夫為吏不欲者出馬【師古曰千夫五大夫不欲為吏者使之出馬也干大武功爵第七級】以故吏弄法皆謫令伐棘上林穿昆明池 是歲得神馬於渥洼水中【李斐曰南陽新野有暴利長當武帝時遭刑屯田敦煌界數於此水旁見羣野馬中冇奇馬與凡馬異來飲此水利長先作土人持勒絆於水傍後馬玩習久之代土人持勒絆收得其馬獻之欲神異此馬云從水中出渥音握洼於佳翻】上方立樂府【樂府冇安世房中歌十七章郊祀歌十九章使童男女七十人歌之師古曰始置之也樂府之名盖起於此哀帝時罷之】使司馬相如等造為詩賦以宦者李延年為協律都尉【協律都尉先無此官武帝始置於此】佩二千石印絃次初詩以合八音之調詩多爾雅之文【初詩新造之詩也八音金石竹匏土革木也調徒釣翻爾惟三卷二千篇文帝時列於學官張晏曰爾近也雅正也】通一經之士不能獨知其辭必集會五經家相與共講習讀之乃能通知其意【漢時五經之學各專門名家故通一經者不能盡通歌詩之辭意必集五經家相與講讀乃得通也】及得神馬次以為歌汲黯曰凡王者作樂上以承祖宗下以化兆民今陛下得馬詩以為歌協於宗廟先帝百姓豈能知其音邪【詩大序曰聲成文謂之音注云聲謂宫商角徵羽也成文謂五聲上下相應鄭康成曰五聲雜比曰音單出曰聲】上默然不說【說讀曰悦 考異曰史記樂書武帝作十九章歌常以正月上辛祠太乙甘泉使僮男僮女七十人俱歌又常得神馬渥洼水中復次以為太一之歌後伐大宛得千里馬次以為歌中尉汲黯進曰陛下得馬詩以為歌云云丞相公孫弘曰黯毁謗聖制當族漢書禮樂志武帝定郊祀之禮祠太一於甘泉祭后土於汾隂乃立樂府作十九章之歌以正月上辛用事甘泉圜丘按天馬歌本志云元狩三年馬生渥洼水中作武紀云元鼎四年秋馬生渥洼水中五年十一月立泰畤於甘泉太初四年貳師獲汗血馬作西極天馬之歌公孫弘以元狩二年甍汲黯以元狩三年免右内史五年為淮陽太守元鼎五年卒又黯未嘗為中尉或者馬生渥洼水作歌在元狩三年汲黯為右内史而譏之言當族者非公孫弘也雖未立泰畤或以歌之於郊廟其十九章之歌當時未能盡備也】 上招延士大夫常如不足然性嚴峻羣臣雖素所愛信者或小有犯法或欺罔輒按誅之無所寛假汲黯諫曰陛下求賢甚勞未盡其用輒已殺之以有限之士恣無已之誅臣恐天下賢才將盡陛下誰與共為治乎黯言之甚怒上笑而諭之【黯言之甚怒上乃笑而諭之即其笑怒之間而觀其君臣相與之意則帝之於黯非但能容其直而從容不迫方喻之以其所見使他人處此固將順之不暇矣而黯自言其心猶以為非此豈面從退有後言者哉黯之事君固人所難能而帝之容黯亦非後世之君所可及矣治直吏翻】曰何世無才患人不能識之耳苟能識之何患無人夫所謂才者猶有用之器也有才而不肯盡用與無才同不殺何施黯曰臣雖不能以言屈陛下而心猶以為非願陛下自今改之無以臣為愚而不知理也上顧羣臣曰黯自言為便辟則不可【朱熹曰便者便人之所好辟者避人之所惡便毗連翻辟讀曰僻】自言為愚豈不信然乎<br />
<br />
  四年冬有司言縣官用度太空而富商大賈冶鑄煮鹽財或絫萬金不佐國家之急【賈音古絫古累字】請更錢造幣以贍用而摧浮淫并兼之徒是時禁苑有白鹿而少府多銀錫乃以白鹿皮方尺緣以藻繢【緣以絹翻師古曰繢繡也繢五采而為之繢黄外翻】為皮幣直四十萬王侯宗室朝覲聘享必以皮幣薦璧然后得行【后與後同】又造銀錫為白金三品【如淳曰雜銀錫為白金】大者圜之其文龍直三千次方之其文馬直五百小者橢之其文龜直三百【時議以為天用莫如龍地用莫如馬人用莫如龜故以為白金三品之文師古曰橢圜而長也音他果翻】令縣官銷半兩錢更鑄三銖錢【建元五年廢三銖錢行半兩錢更工衡翻】盗鑄諸金錢罪皆死而吏民之盗鑄白金者不可勝數【勝音升】於是以東郭咸陽孔僅為大農丞領鹽鐵事【師古曰二人也姓東郭名咸陽姓孔名僅班表大農令有兩丞齊有大夫東郭氏】桑弘羊以計筭用事【姓譜桑秦大夫子桑之後】咸陽齊之大煮鹽僅南陽大冶皆致生絫千金弘羊洛陽賈人子以心計【心計者不必用籌筭而知其數也賈音古下同】年十三侍中三人言利事析秋毫矣【毫至秋而鋭小言其剖析微細雖秋毫之小亦可分而為二也】詔禁民敢私鑄鐵器煮鹽者左趾【韋昭曰釱以鐵為之著左足以代刖也索隱曰三蒼云踏脚鉗也張斐漢晉律序狀如跟衣著足下重六斤以代刖至魏武改以滅代也晉律鉗重二斤長翹一尺五寸師古曰徒計翻】没入其器物公卿又請令諸賈人末作各以其物自占【師古曰占隱度也各隱度其財物之多少而為名簿送之於官也占之贍翻下同】率緡錢二千而一算【李斐曰緡也以貫錢一貫千錢出算二十也瓚曰此緡錢為是儲緡錢也故隨其用所施而出筭予謂率計緡錢二千而出一筭筭百二十錢緡眉巾翻】及民有軺車若船五丈以上者皆有算【軺小車也弋招翻】匿不自占占不悉戍邊一歲没入緡錢【匿藏也悉盡也藏匿而不自占占而不盡者罸戌邊一歲没其官入錢】有能告者以其半畀之其法大抵出張湯湯每朝奏事語國家用日晏【師古曰論事既多至于日晚朝直遥翻】天子忘食丞相充位【但充其位無所建明】天下事皆决于湯百姓騷動不安其生咸指怨湯 初河南人卜式數請輸財縣官以助邊【數所角翻】天子使使問式欲官乎式曰臣少田牧不習仕宦不願也【少詩照翻】使者問曰家豈有寃欲言事乎式曰臣生與人無分争邑人貧者貸之不善者教之所居人皆從式式何故見寃於人無所欲言也使者曰苟如此子何欲而然式曰天子誅匈奴愚以為賢者宜死節于邊有財者宜輸委【委于偽翻蓄也宜輸其所蓄也】如此而匈奴可滅也上由是賢之欲尊顯以風百姓【師古曰風讀曰諷又如字】乃召拜式為中郎爵左庶長賜田十頃布告天下使明知之未幾又擢式為齊太傅【齊王次昌元朔三年薨無後國除元狩六年始封皇子閎為齊王式盖傅閎也史因其輸財得官而終書之幾居豈翻】 春有星孛于東北【孛蒲内翻】夏有長星出于西北上與諸將議曰翕侯趙信為單于畫計【為于偽翻】常以為<br />
<br />
  漢兵不能度幕輕留【幕沙漠也師古曰言輕易漢軍留而不去也一曰謂漢軍不能輕入而久留也予謂後說是】今大發士卒其埶必得所欲乃粟馬十萬【師古曰以粟秣馬也】令大將軍青票騎將軍去病各將五萬騎私負從馬復四萬匹【師古曰私負衣装及私將馬自從者皆非公家所發之限從才用翻】步兵轉者踵軍後又數十萬人【師古曰轉者謂運輜重也踵接也】而敢力戰深入之士皆屬票騎票騎始為出定襄當單于捕虜言單于東乃更令票騎出代郡令大將軍出定襄郎中令李廣數自請行【數所角翻】天子以為老弗許良久乃許之以為前將軍太僕公孫賀為左將軍主爵都尉趙食其為右將軍【食其音異箕】平陽侯曹襄為後將軍皆屬大將軍趙信為單于謀曰漢兵既度幕人馬罷匈奴可坐收虜耳【師古曰言收虜漢軍人馬可不費力故言坐罷讀曰疲】乃悉遠北其輜重【師古曰送輜重遠去令處北也】以精兵待幕比大將軍既出塞捕虜知單于所居乃自以精兵走之【走音奏】而令前將軍廣并於右將軍軍出東道【師古曰并合也合軍而同道】東道回遠而水草少【師古曰回繞也曲也戶悔翻】廣自請曰臣部為前將軍今大將軍乃徙令臣出東道且臣結髮而與匈奴戰今乃一得當單于【結髪者言始勝冠即在戰陣及今得當單于也】臣願居前先死單于【師古曰致死而取單于】大將軍亦隂受上誡以為李廣老數奇【孟康曰奇隻不偶也如淳曰數為匈奴所敗為奇不耦師古曰言廣命隻不耦合也孟說是矣數所角翻奇居宜翻】毋令當單于恐不得所欲【師古曰謂不勝敵也余謂指欲禽單于脱有邂逅失之為不得所欲】而公孫敖新失侯大將軍亦欲使敖與俱當單于【敖失侯見上二年青本與敖友又脱青于阸故青欲使當單于而立功】故徙前將軍廣廣知之固自辭於大將軍大將軍不聽廣不謝而起行意甚愠怒【愠于運翻】大將軍出塞千餘里度幕見單于兵陳而待【言結陳以待敵也陳與陣同】于是大將軍令武剛車自環為營【張晏曰武剛車兵車也師古曰環繞也續漢志諸軍有矛戟其飾幡斿旗幟有巾有盖謂之武剛車環音宦】而縱五千騎往當匈奴匈奴亦縱可萬騎會日且入【言日欲没也】大風起砂礫擊面【師古曰礫小石也音歷】兩軍不相見漢益縱左右翼繞單于【師古曰翼謂左右舒引其兵如烏之張翼】單于視漢兵多而士馬尚彊自度戰不能如漢兵【度徒洛翻】單于遂乘六騾壮騎可數百直冒漢圍西北馳去【師古曰騾者驢種馬子堅忍單于自乘善走騾而壮騎隨之也冒犯也騾來戈翻冒莫克翻】時已昏漢匈奴相紛拏【師古曰紛拏亂相持也拏女居翻】殺傷大當【殺傷各大相當】漢軍左校捕虜言單于未昏而去漢軍發輕騎夜追之大將軍軍因隨其後匈奴兵亦散走遲明【遲直二翻】行二百餘里不得單于捕斬首虜萬九千級遂至窴顔山趙信城【窴徒賢翻如淳曰趙信降匈奴築城居之】得匈奴積粟食軍【師古曰食讀曰飤】留一日悉燒其城餘粟而歸前將軍廣與右將軍食其軍無導惑失道後大將軍【師古曰惑迷也在後不及期也】不及單于戰大將軍引還過幕南乃遇二將軍大將軍使長史責問廣食其失道狀急責廣之幕府對簿【師古曰簿謂文狀也】廣曰諸校尉無罪乃我自失道吾今自上簿至幕府【上時掌翻】廣謂其麾下曰廣結髪與匈奴大小七十餘戰今幸從大將軍出接單于兵而大將軍徙廣部行回遠而又迷失道豈非天哉且廣年六十餘矣終不能復對刀筆之吏【復扶又翻】遂引刀自剄【剄古頂翻】廣為人廉得賞賜輒分其麾下飲食與士共之為二千石四十餘年家無餘財猨臂善射【如淳曰臂如猨臂通肩也】度不中不發【度徒洛翻中竹仲翻】將兵乏絶之處【孔穎達曰暫無曰乏不續曰絶】見水士卒不盡飲廣不近水【近其靳翻】士卒不盡食廣不嘗食士以此愛樂為用【樂音洛】及死一軍皆哭百姓聞之知與不知無老壮皆為垂涕【師古曰知謂素相䜟知也為于偽翻】而右將軍獨下吏【下遐嫁翻】當死贖為庶人單于之遁走其兵往往與漢兵相亂而隨單于單于久不與其大衆相得其右谷蠡王以為單于死乃自立為單于【谷蠡音鹿黎】十餘日真單于復得其衆而右谷蠡王乃去其單于號【師古曰去除也羌呂翻】票騎將軍騎兵車重與大將軍軍等【重直用翻】而無裨將悉以李敢等為大校當裨將【校戶教翻】出代右北平二千餘里絶大幕直左方兵【師古曰直當也匈奴分其國為左右諸左王將居東方直上谷以東接濊貃朝鮮右王將居西方直上郡以西接氐羌故謂之左右方亦謂之左右地】獲屯頭王韓王等三人將軍相國當戶都尉八十三人封狼居胥山禪于姑衍登臨翰海【張晏曰登海邊山以望海也有大功故增山而廣地也如淳曰翰海北海名崔浩曰羣鳥之所解羽故曰翰海廣志瀚海在沙漠北師古曰積土增高曰封為墠祭地曰禪】鹵獲七萬四百四十三級天子以五千八百戶益封票騎將軍又封其所部右北平太守路博德等四人為列侯【路博德為邳離侯衛山為義陽侯復陸支為杜侯伊即靬為衆利侯】從票侯破奴等二人益封校尉敢為關内侯食邑軍吏卒為官賞賜甚多而大將軍不得益封軍吏卒皆無封侯者兩軍之出塞塞閲官及私馬凡十四萬匹而復入塞者不滿三萬匹乃益置大司馬位大將軍票騎將軍皆為大司馬定令令票騎將軍秩禄與大將軍等【應劭曰司馬主武事諸武官亦以為號漢官儀曰時議者以為軍中有侯司馬故加大為大司馬以别異之自此票騎將軍同大將軍品秩位亞丞相】自是之後大將軍青日退而票騎日益貴大將軍故人門下士多去事票騎輒得官爵唯任安不肯票騎將軍為人少言不泄【孔文祥曰謂質重少言膽氣在中也】有氣敢往天子嘗欲教之孫吳兵法【孫孫武吳吳起也】對曰顧方略何如耳不至學古兵法天子為治第令票騎視之對曰匈奴未滅無以家為也【治直之翻】由此上益重愛之然少貴不省士【師古曰省視也言不恤視軍士也少詩沼翻】其從軍天子為遣太官齎數十乘【班表太官有令有丞主膳食師古曰齎與資同予謂音則兮翻亦通裝也為于偽翻乘繩證翻】既還重車餘棄粱肉【師古曰重直用翻粱粟類也米之善者】而士有飢者其在塞外卒乏糧或不能自振而票騎尚穿域蹋鞠【服䖍曰穿域作鞠室也師古曰鞫以皮為之實以毛蹴蹋為戲也劉向别録曰蹴鞫相傳以為黄帝所作或曰起戰國之時所以講武知有材也蹋徒臘翻鞠居六翻】事多此類大將軍為人仁喜士讓【師古曰喜許吏翻】以和柔自媚於上兩人志操如此【操七到翻】是時漢所殺虜匈奴合八九萬而漢士卒物故亦數萬【魏臺訪議高堂隆曰聞之先師物無也故事也言無復所能於事也索隱曰漢以來謂死為物故就朽故也師古曰物故謂死也言其同於鬼物而故也盖漢軍死者亦數萬】是後匈奴遠遁而幕南無王庭【冒頓之強盡取蒙恬所奪匈奴地而王庭列置於幕南今匈奴為漢所攻遠遁幕北故幕南無王庭也】漢渡河自朔方以西至令居【班志令居縣屬金城郡令音零】往往通渠置田官【置官以主屯田】吏卒五六萬人稍蠶食匈奴以北【蠶食言如蠶之食葉以漸而侵其地也】然亦以馬少不復大出擊匈奴矣【少詩沼翻復扶又翻】匈奴用趙信計遣使於漢好辭請和親天子下其議【下遐嫁翻】或言和親或言遂臣之丞相長史任敞曰【班表丞相冇二長史秩二千石任音壬】匈奴新破困宜可使為外臣朝請於邊漢使任敞於單于單于大怒留之不遣【朝直遥翻請才性翻使疏吏翻】是時博士狄山議以為和親便【姓譜狄春秋狄國之後又曰周文王封少子於狄城】上以問張湯湯曰此愚儒無知狄山曰臣固愚愚忠若御史大夫湯乃詐忠于是上作色曰吾使生居一郡【師古曰博士儒官也故呼為生】能無使虜入盗乎曰不能曰居一縣對曰不能復曰居一障間【師古曰障謂塞上要險之處别築為城因置吏士而為蔽障以禦寇也障之尚翻又漢制每塞要處别築為城置人鎮守謂之候城此即障也】山自度辯窮且下吏【師古曰度計也見詰辯而辭窮當下吏也下遐嫁翻】曰能於是上遣山乘障【師古曰乘登也登而守之】至月餘匈奴斬山頭而去自是之後羣臣震慴【師古曰震動也慴失氣也慴之涉翻】無敢忤湯者【忤五故翻】是歲汲黯坐法免以定襄太守義縱為右内史河内太守王温舒為中尉【守式又翻】先是甯成為關都尉【函谷關都尉也先悉薦翻】吏民出入關者號曰寜見乳虎無值甯成之怒【師古曰猛虎產乳護養其子則博噬過當故以為喻乳人喻翻】及義縱為南陽太守【義姓也縱其名】至關甯成側行送迎【側行不敢正行言恭甚】至郡遂按甯氏破碎其家南陽吏民重足一迹【言絫足也畏懼之甚重直龍翻】後徙定襄太守初至掩定襄獄中重罪輕繫二百餘人及賓客昆弟私入視亦二百餘人一捕鞫日為死罪解脱【一切皆捕而鞫問之也服䖍曰律諸囚徒私解脱桎梏鉗赭加罪一等為人解脱與同罪縱鞫相賂餉者二百人以為解脱死罪盡殺之師古曰鞫窮也謂窮治也】是日皆報殺四百餘人【師古曰奏請得報而論殺原父曰縱掩定襄獄一切捕鞫而云是日皆報殺則非奏請報可之報矣然則以論次為報也】其後郡中不寒而栗是時趙禹張湯以深刻為九卿然其治尚輔法而行縱專以鷹擊為治【師古曰言如鷹隼之擊也治直吏翻】王温舒始為廣平都尉【廣平本屬趙國景武之間分為廣平郡征和元年立為平王國】擇郡中豪敢往吏十餘人【師古曰豪桀而性果敢一往無所顧者以為吏也】以為爪牙皆把其隂重罪而縱使督盗賊【師古曰縱放也督察視也】快其意所欲得此人雖有百罪弗法【師古曰言所捕盗賊得其人而快温舒意則不問其先所犯罪也弗法謂弗行法也】即有避因其事夷之亦滅宗【師古曰避謂不盡意捕擊也】以其故齊趙之郊盗賊不敢近廣平【近其靳翻】廣平聲為道不拾遺遷河内太守以九月至令郡具馬五十匹為驛【師古曰以私馬於道上往往置驛自河内至長安】捕郡中豪猾相連坐千餘家上書請大者至族小者乃死家盡没入償臧【師古曰以臧獲罪者既没入之又令出倍臧或收入官或還其主也予謂没入共家以償所受之臧其義似逕臧讀曰贓】奏行不過二三日得可【奏而天子可之謂之得可】事論報至流血十餘里河内皆怪其奏以為神速盡十二月郡中毋聲毋敢夜行【古毋無通】野無犬吠之盗其頗不得失之旁郡國追求會春温舒頓足歎曰嗟乎令冬月益展一月足吾事矣【師古曰立春之後不復行刑故云然展伸也】天子聞之皆以為能故擢為中二千石【郡守二千石正卿及列卿皆中二千石】 齊人少翁以鬼神方見上上有所幸王夫人卒【上夫人齊王閎之母】少翁以方夜致鬼如王夫人之貌天子自帷中望見焉【考異曰漢書以此事置李夫人傳中古今相承皆以為李夫人事史記封禪書少翁見上上有所幸王夫人卒少翁以方夜致王夫人及竈鬼之貌云按李夫人卒時少翁死已久漢書誤也今從史記】于是乃拜少翁為文成將軍賞賜甚多以客禮禮之文成又勸上作甘泉宫中為臺室畫天地太一諸鬼神而置祭具以致天神居歲餘其方益衰神不至乃為帛書以飯牛【師古曰謂雜艸以飯牛也飯扶晚翻】佯不知言曰此牛腹中有奇殺視得書書言甚怪天子識其手書【謂識其親書手蹟也】問其人果是偽書于是誅文成將軍而隱之【隱謂秘誅文成之事不令人知之也】<br />
<br />
  資治通鑑卷十九  <br>
   </div> 

<script src="/search/ajaxskft.js"> </script>
 <div class="clear"></div>
<br>
<br>
 <!-- a.d-->

 <!--
<div class="info_share">
</div> 
-->
 <!--info_share--></div>   <!-- end info_content-->
  </div> <!-- end l-->

<div class="r">   <!--r-->



<div class="sidebar"  style="margin-bottom:2px;">

 
<div class="sidebar_title">工具类大全</div>
<div class="sidebar_info">
<strong><a href="http://www.guoxuedashi.com/lsditu/" target="_blank">历史地图</a></strong>  
<a href="http://www.880114.com/" target="_blank">英语宝典</a>  
<a href="http://www.guoxuedashi.com/13jing/" target="_blank">十三经检索</a> 
<br><strong><a href="http://www.guoxuedashi.com/gjtsjc/" target="_blank">古今图书集成</a></strong> 
<a href="http://www.guoxuedashi.com/duilian/" target="_blank">对联大全</a> <strong><a href="http://www.guoxuedashi.com/xiangxingzi/" target="_blank">象形文字典</a></strong> 

<br><a href="http://www.guoxuedashi.com/zixing/yanbian/">字形演变</a>  <strong><a href="http://www.guoxuemi.com/hafo/" target="_blank">哈佛燕京中文善本特藏</a></strong>
<br><strong><a href="http://www.guoxuedashi.com/csfz/" target="_blank">丛书&方志检索器</a></strong> <a href="http://www.guoxuedashi.com/yqjyy/" target="_blank">一切经音义</a>  

<br><strong><a href="http://www.guoxuedashi.com/jiapu/" target="_blank">家谱族谱查询</a></strong>  <strong><a href="http://shufa.guoxuedashi.com/sfzitie/" target="_blank">书法字帖欣赏</a></strong> 
<br>

</div>
</div>


<div class="sidebar" style="margin-bottom:0px;">

<font style="font-size:22px;line-height:32px">QQ交流群9:489193090</font>


<div class="sidebar_title">手机APP 扫描或点击</div>
<div class="sidebar_info">
<table>
<tr>
	<td width=160><a href="http://m.guoxuedashi.com/app/" target="_blank"><img src="/img/gxds-sj.png" width="140"  border="0" alt="国学大师手机版"></a></td>
	<td>
<a href="http://www.guoxuedashi.com/download/" target="_blank">app软件下载专区</a><br>
<a href="http://www.guoxuedashi.com/download/gxds.php" target="_blank">《国学大师》下载</a><br>
<a href="http://www.guoxuedashi.com/download/kxzd.php" target="_blank">《汉字宝典》下载</a><br>
<a href="http://www.guoxuedashi.com/download/scqbd.php" target="_blank">《诗词曲宝典》下载</a><br>
<a href="http://www.guoxuedashi.com/SiKuQuanShu/skqs.php" target="_blank">《四库全书》下载</a><br>
</td>
</tr>
</table>

</div>
</div>


<div class="sidebar2">
<center>


</center>
</div>

<div class="sidebar"  style="margin-bottom:2px;">
<div class="sidebar_title">网站使用教程</div>
<div class="sidebar_info">
<a href="http://www.guoxuedashi.com/help/gjsearch.php" target="_blank">如何在国学大师网下载古籍?</a><br>
<a href="http://www.guoxuedashi.com/zidian/bujian/bjjc.php" target="_blank">如何使用部件查字法快速查字?</a><br>
<a href="http://www.guoxuedashi.com/search/sjc.php" target="_blank">如何在指定的书籍中全文检索?</a><br>
<a href="http://www.guoxuedashi.com/search/skjc.php" target="_blank">如何找到一句话在《四库全书》哪一页?</a><br>
</div>
</div>


<div class="sidebar">
<div class="sidebar_title">热门书籍</div>
<div class="sidebar_info">
<a href="/so.php?sokey=%E8%B5%84%E6%B2%BB%E9%80%9A%E9%89%B4&kt=1">资治通鉴</a> <a href="/24shi/"><strong>二十四史</strong></a>&nbsp; <a href="/a2694/">野史</a>&nbsp; <a href="/SiKuQuanShu/"><strong>四库全书</strong></a>&nbsp;<a href="http://www.guoxuedashi.com/SiKuQuanShu/fanti/">繁体</a>
<br><a href="/so.php?sokey=%E7%BA%A2%E6%A5%BC%E6%A2%A6&kt=1">红楼梦</a> <a href="/a/1858x/">三国演义</a> <a href="/a/1038k/">水浒传</a> <a href="/a/1046t/">西游记</a> <a href="/a/1914o/">封神演义</a>
<br>
<a href="http://www.guoxuedashi.com/so.php?sokeygx=%E4%B8%87%E6%9C%89%E6%96%87%E5%BA%93&submit=&kt=1">万有文库</a> <a href="/a/780t/">古文观止</a> <a href="/a/1024l/">文心雕龙</a> <a href="/a/1704n/">全唐诗</a> <a href="/a/1705h/">全宋词</a>
<br><a href="http://www.guoxuedashi.com/so.php?sokeygx=%E7%99%BE%E8%A1%B2%E6%9C%AC%E4%BA%8C%E5%8D%81%E5%9B%9B%E5%8F%B2&submit=&kt=1"><strong>百衲本二十四史</strong></a>  <a href="http://www.guoxuedashi.com/so.php?sokeygx=%E5%8F%A4%E4%BB%8A%E5%9B%BE%E4%B9%A6%E9%9B%86%E6%88%90&submit=&kt=1"><strong>古今图书集成</strong></a>
<br>

<a href="http://www.guoxuedashi.com/so.php?sokeygx=%E4%B8%9B%E4%B9%A6%E9%9B%86%E6%88%90&submit=&kt=1">丛书集成</a> 
<a href="http://www.guoxuedashi.com/so.php?sokeygx=%E5%9B%9B%E9%83%A8%E4%B8%9B%E5%88%8A&submit=&kt=1"><strong>四部丛刊</strong></a>  
<a href="http://www.guoxuedashi.com/so.php?sokeygx=%E8%AF%B4%E6%96%87%E8%A7%A3%E5%AD%97&submit=&kt=1">說文解字</a> <a href="http://www.guoxuedashi.com/so.php?sokeygx=%E5%85%A8%E4%B8%8A%E5%8F%A4&submit=&kt=1">三国六朝文</a>
<br><a href="http://www.guoxuedashi.com/so.php?sokeytm=%E6%97%A5%E6%9C%AC%E5%86%85%E9%98%81%E6%96%87%E5%BA%93&submit=&kt=1"><strong>日本内阁文库</strong></a> <a href="http://www.guoxuedashi.com/so.php?sokeytm=%E5%9B%BD%E5%9B%BE%E6%96%B9%E5%BF%97%E5%90%88%E9%9B%86&ka=100&submit=">国图方志合集</a> <a href="http://www.guoxuedashi.com/so.php?sokeytm=%E5%90%84%E5%9C%B0%E6%96%B9%E5%BF%97&submit=&kt=1"><strong>各地方志</strong></a>

</div>
</div>


<div class="sidebar2">
<center>

</center>
</div>
<div class="sidebar greenbar">
<div class="sidebar_title green">四库全书</div>
<div class="sidebar_info">

《四库全书》是中国古代最大的丛书,编撰于乾隆年间,由纪昀等360多位高官、学者编撰,3800多人抄写,费时十三年编成。丛书分经、史、子、集四部,故名四库。共有3500多种书,7.9万卷,3.6万册,约8亿字,基本上囊括了古代所有图书,故称“全书”。<a href="http://www.guoxuedashi.com/SiKuQuanShu/">详细>>
</a>

</div> 
</div>

</div>  <!--end r-->

</div>
<!-- 内容区END --> 

<!-- 页脚开始 -->
<div class="shh">

</div>

<div class="w1180" style="margin-top:8px;">
<center><script src="http://www.guoxuedashi.com/img/plus.php?id=3"></script></center>
</div>
<div class="w1180 foot">
<a href="/b/thanks.php">特别致谢</a> | <a href="javascript:window.external.AddFavorite(document.location.href,document.title);">收藏本站</a> | <a href="#">欢迎投稿</a> | <a href="http://www.guoxuedashi.com/forum/">意见建议</a> | <a href="http://www.guoxuemi.com/">国学迷</a> | <a href="http://www.shuowen.net/">说文网</a><script language="javascript" type="text/javascript" src="https://js.users.51.la/17753172.js"></script><br />
  Copyright &copy; 国学大师 古典图书集成 All Rights Reserved.<br>
  
  <span style="font-size:14px">免责声明:本站非营利性站点,以方便网友为主,仅供学习研究。<br>内容由热心网友提供和网上收集,不保留版权。若侵犯了您的权益,来信即刪。scp168@qq.com</span>
  <br />
ICP证:<a href="http://www.beian.miit.gov.cn/" target="_blank">鲁ICP备19060063号</a></div>
<!-- 页脚END --> 
<script src="http://www.guoxuedashi.com/img/plus.php?id=22"></script>
<script src="http://www.guoxuedashi.com/img/tongji.js"></script>

</body>
</html>
