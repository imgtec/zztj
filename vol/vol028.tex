<!DOCTYPE html PUBLIC "-//W3C//DTD XHTML 1.0 Transitional//EN" "http://www.w3.org/TR/xhtml1/DTD/xhtml1-transitional.dtd">
<html xmlns="http://www.w3.org/1999/xhtml">
<head>
<meta http-equiv="Content-Type" content="text/html; charset=utf-8" />
<meta http-equiv="X-UA-Compatible" content="IE=Edge,chrome=1">
<title>資治通鑒_29-資治通鑑卷二十八_29-資治通鑑卷二十八</title>
<meta name="Keywords" content="資治通鑒_29-資治通鑑卷二十八_29-資治通鑑卷二十八">
<meta name="Description" content="資治通鑒_29-資治通鑑卷二十八_29-資治通鑑卷二十八">
<meta http-equiv="Cache-Control" content="no-transform" />
<meta http-equiv="Cache-Control" content="no-siteapp" />
<link href="/img/style.css" rel="stylesheet" type="text/css" />
<script src="/img/m.js?2020"></script> 
</head>
<body>
 <div class="ClassNavi">
<a  href="/24shi/">二十四史</a> | <a href="/SiKuQuanShu/">四库全书</a> | <a href="http://www.guoxuedashi.com/gjtsjc/"><font  color="#FF0000">古今图书集成</font></a> | <a href="/renwu/">历史人物</a> | <a href="/ShuoWenJieZi/"><font  color="#FF0000">说文解字</a></font> | <a href="/chengyu/">成语词典</a> | <a  target="_blank"  href="http://www.guoxuedashi.com/jgwhj/"><font  color="#FF0000">甲骨文合集</font></a> | <a href="/yzjwjc/"><font  color="#FF0000">殷周金文集成</font></a> | <a href="/xiangxingzi/"><font color="#0000FF">象形字典</font></a> | <a href="/13jing/"><font  color="#FF0000">十三经索引</font></a> | <a href="/zixing/"><font  color="#FF0000">字体转换器</font></a> | <a href="/zidian/xz/"><font color="#0000FF">篆书识别</font></a> | <a href="/jinfanyi/">近义反义词</a> | <a href="/duilian/">对联大全</a> | <a href="/jiapu/"><font  color="#0000FF">家谱族谱查询</font></a> | <a href="http://www.guoxuemi.com/hafo/" target="_blank" ><font color="#FF0000">哈佛古籍</font></a> 
</div>

 <!-- 头部导航开始 -->
<div class="w1180 head clearfix">
  <div class="head_logo l"><a title="国学大师官网" href="http://www.guoxuedashi.com" target="_blank"></a></div>
  <div class="head_sr l">
  <div id="head1">
  
  <a href="http://www.guoxuedashi.com/zidian/bujian/" target="_blank" ><img src="http://www.guoxuedashi.com/img/top1.gif" width="88" height="60" border="0" title="部件查字,支持20万汉字"></a>


<a href="http://www.guoxuedashi.com/help/yingpan.php" target="_blank"><img src="http://www.guoxuedashi.com/img/top230.gif" width="600" height="62" border="0" ></a>


  </div>
  <div id="head3"><a href="javascript:" onClick="javascript:window.external.AddFavorite(window.location.href,document.title);">添加收藏</a>
  <br><a href="/help/setie.php">搜索引擎</a>
  <br><a href="/help/zanzhu.php">赞助本站</a></div>
  <div id="head2">
 <a href="http://www.guoxuemi.com/" target="_blank"><img src="http://www.guoxuedashi.com/img/guoxuemi.gif" width="95" height="62" border="0" style="margin-left:2px;" title="国学迷"></a>
  

  </div>
</div>
  <div class="clear"></div>
  <div class="head_nav">
  <p><a href="/">首页</a> | <a href="/ShuKu/">国学书库</a> | <a href="/guji/">影印古籍</a> | <a href="/shici/">诗词宝典</a> | <a   href="/SiKuQuanShu/gxjx.php">精选</a> <b>|</b> <a href="/zidian/">汉语字典</a> | <a href="/hydcd/">汉语词典</a> | <a href="http://www.guoxuedashi.com/zidian/bujian/"><font  color="#CC0066">部件查字</font></a> | <a href="http://www.sfds.cn/"><font  color="#CC0066">书法大师</font></a> | <a href="/jgwhj/">甲骨文</a> <b>|</b> <a href="/b/4/"><font  color="#CC0066">解密</font></a> | <a href="/renwu/">历史人物</a> | <a href="/diangu/">历史典故</a> | <a href="/xingshi/">姓氏</a> | <a href="/minzu/">民族</a> <b>|</b> <a href="/mz/"><font  color="#CC0066">世界名著</font></a> | <a href="/download/">软件下载</a>
</p>
<p><a href="/b/"><font  color="#CC0066">历史</font></a> | <a href="http://skqs.guoxuedashi.com/" target="_blank">四库全书</a> |  <a href="http://www.guoxuedashi.com/search/" target="_blank"><font  color="#CC0066">全文检索</font></a> | <a href="http://www.guoxuedashi.com/shumu/">古籍书目</a> | <a   href="/24shi/">正史</a> <b>|</b> <a href="/chengyu/">成语词典</a> | <a href="/kangxi/" title="康熙字典">康熙字典</a> | <a href="/ShuoWenJieZi/">说文解字</a> | <a href="/zixing/yanbian/">字形演变</a> | <a href="/yzjwjc/">金 文</a> <b>|</b>  <a href="/shijian/nian-hao/">年号</a> | <a href="/diming/">历史地名</a> | <a href="/shijian/">历史事件</a> | <a href="/guanzhi/">官职</a> | <a href="/lishi/">知识</a> <b>|</b> <a href="/zhongyi/">中医中药</a> | <a href="http://www.guoxuedashi.com/forum/">留言反馈</a>
</p>
  </div>
</div>
<!-- 头部导航END --> 
<!-- 内容区开始 --> 
<div class="w1180 clearfix">
  <div class="info l">
   
<div class="clearfix" style="background:#f5faff;">
<script src='http://www.guoxuedashi.com/img/headersou.js'></script>

</div>
  <div class="info_tree"><a href="http://www.guoxuedashi.com">首页</a> > <a href="/SiKuQuanShu/fanti/">四库全书</a>
 > <h1>资治通鉴</h1> <!--         下载:【右键另存为】即可 --></div>
  <div class="info_content zj clearfix">
  
<div class="info_txt clearfix" id="show">
<center style="font-size:24px;">29-資治通鑑卷二十八</center>
    資治通鑑卷二十八   宋 司馬光 撰<br />
<br />
  胡三省 音註<br />
<br />
  漢紀二十【起昭陽作噩盡屠維單閼凡七年】<br />
<br />
  孝元皇帝上【荀悦曰諱奭之字曰盛應劭曰諡法行義悦民曰元】<br />
<br />
  初元元年春正月辛丑葬孝宣皇帝于杜陵【臣瓚曰自崩至葬凡二十八日杜陵在長安南五十里】赦天下 三月丙午立皇后王氏封后父禁為陽平侯【恩澤侯表陽平侯食邑於東郡】以三輔太常郡國公田及苑可省者振業貧民【太常掌諸陵邑故亦有公田苑師古曰振業振起之令有作業】貲不滿千錢者賦貸種食【師古曰賦給與之也貸假也種音之勇翻賈公彦曰種食者或為種子或為食用】 封外祖平恩戴侯同產弟子中常侍許嘉為平恩侯【文頴曰戴侯許廣漢諡法典禮不愆曰戴余按廣漢先坐腐刑及薨無後今以嘉紹封百官表侍中中常侍皆加官西都參用士人東都始以宦者為中常侍】 夏六月以民疾疫令太官省膳減樂府員省苑馬以振困乏【樂府員大凡八百二十九人武帝所立漢官儀牧師諸苑三十六所分置北邊西邊養馬三十萬匹】 關東郡國十一大水饑或人相食轉旁郡錢穀以相救 上素聞琅邪王吉貢禹皆明經潔行【姓譜貢姓子貢之後行下孟翻】遣使者徵之吉道病卒禹至拜為諫大夫上數虛己問以政【易咸卦君子以虚受人師古曰虛已謂聽受其言也數所角翻】禹奏言古者人君節儉什一而税無他賦役故家給人足高祖孝文孝景皇帝宫女不過十餘人廐馬百餘匹後世争為奢侈轉轉益甚臣下亦相放效【師古曰放音甫往翻下同】臣愚以為如太古難宜少放古以自節焉【少詩沼翻】方今宫室已定無可奈何矣其餘盡可減損故時齊三服官輸物不過十笥【李斐曰齊國舊有三服之官春獻冠幘縰為首服紈素為冬服輕綃為夏服凡三如淳曰地理志曰齊冠帶天下胡公曰服官主作文繡以給衮龍之服地理志襄邑亦有服官師古曰齊三服官李說是也縰與纚同音山爾翻即今之方目也紈素今之絹也輕綃今之輕也襄邑自出文繡非齊三服也】方今齊三服官作工各數千人一歲費數鉅萬【萬萬為鉅萬】廐馬食粟將萬匹武帝時又多取好女至數千人以填後宫及棄天下多藏金錢財物鳥獸魚鼈凡百九十物又皆以後宫女置於園陵至孝宣皇帝時陛下惡有所言【師古曰不能自言減省之事惡烏路翻惡有所言者惡以天下儉其親此語承上園陵事】羣臣亦隨故事甚可痛也故使天下承化取女皆大過度【師古曰取讀曰娶】諸侯妻妾或至數百人豪富吏民畜歌者至數十人【此所謂取女過度也】是以内多怨女外多曠夫【師古曰曠空也室家空也】及衆庶葬埋皆虛地上以實地下其過自上生【師古曰自從也上謂天子也】皆在大臣循故事之辠也唯陛下深察古道從其儉者大減損乘輿服御器物三分去二【乘䋲證翻去羌呂翻】擇後宫賢者留二十人餘悉歸之及諸陵園女無子者宜悉遣【漢制天子晏駕後宫送葬因留奉陵寑】廐馬可無過數十匹獨舍長安城南苑地以為田獵之囿【師古曰舍置也獨留置之其餘皆廢去舍讀曰捨】以方今天下饑饉可無大自損減以救之稱天意乎天生聖人盖為萬民非獨使自娛樂而已也【稱只證翻為于偽翻樂音洛】天子納善其言下詔令諸宫館希御幸者勿繕治【治直之翻】太僕減穀食馬水衡減肉食獸【太僕掌輿馬漢舊儀云天子大廐未央丞華輅軨陪馬騶駼大廐也馬皆萬匹水衡都尉掌上林苑禽獸屬焉師古曰繕補也減謂損其數省者全去之】<br />
<br />
  臣光曰忠臣之事君也責其所難則其易者不勞而正【易以豉翻】補其所短則其長者不勸而遂孝元踐位之初虚心以問禹禹宜先其所急後其所緩然則優游不斷【先後皆去聲斷丁亂翻】讒佞用權當時之大患也而禹不以為言恭謹節儉孝元之素志也而禹孜孜言之何哉使禹之智不足以知烏得為賢知而不言為罪愈大矣<br />
<br />
  匈奴呼韓邪單于復上書言民衆困乏【復扶又翻】詔雲中五原郡轉穀二萬斛以給之 是歲初置戊巳校尉使屯田車師故地【師古曰戊巳校尉者鎮安西域無常治處亦猶甲乙等各冇方位而戊與巳四季寄王故以名官也時有戊校尉又冇巳校尉一說戊與巳位在中央今所置校尉在三十六國之中故曰戊巳也余謂車師之地不在三十六國之中當從師古前說為是宣帝元康二年以車師地與匈奴今匈奴欵附故復屯田故地】<br />
<br />
  二年春正月上行幸甘泉郊泰畤【畤音止】 樂陵侯史高以外屬領尚書事前將軍蕭望之光禄大夫周堪為之副望之名儒與堪皆以師傅舊恩天子任之數宴見言治亂陳王事【數所角翻見賢遍翻治直吏翻陳王者之事也】望之選白宗室明經有行【行下孟翻】散騎諫大夫劉更生給事中【明經有行言其通於經術且行修飭也百官表曰散騎加官騎並乘輿車師古曰並音步浪翻騎而散從無常職也給事中給事禁中也散悉亶翻】與侍中金敞並拾遺左右四人同心謀議勸導上以古制多所欲匡正上甚鄕納之【師古曰郷讀曰嚮意信嚮之而納用其言】史高充位而已由此與望之有隙中書令弘恭【弘姓也衛冇大夫弘演】僕射石顯自宣帝時久典樞機明習文法【續漢志尚書令承秦所置武帝用宦者更為中書謁者令成帝用士人復故令掌凡選署及奏下尚書曹文書衆事僕射署尚書事令不在則奏下衆事辯已見前】帝即位多疾以顯久典事中人無外黨【師古曰少骨肉之親無婚姻之家也】精專可信任遂委以政事無大小因顯白决【白奏也决斷也】貴幸傾朝【朝直遥翻】百僚皆敬事顯顯為人巧慧習事能深得人主微指内深賊持詭辯以中傷人【師古曰詭違也違道之辯中竹仲翻】忤恨睚眦輒被以危法【忤五故翻睚五懈翻眦仕懈翻師古曰被加也音皮義翻危法謂以法危殺之】亦與車騎將軍高為表裏議論常獨持故事不從望之等望之等患苦許史放縱又疾恭顯擅權建白以為中書政本國家樞機【師古曰建白者立此議而白之】宜以通明公正處之【處昌呂翻】武帝游宴後庭故用宦者非古制也宜罷中書宦官應古不近刑人之義【師古曰禮刑人不在君側故曰應古近其靳翻】由是大與高恭顯忤【師古曰忤謂相違逆也忤五故翻】上初即位謙讓重改作【師古曰重難也未欲更置士人於中書也】議久不定出劉更生為宗正【散騎給事中中朝官也宗正外朝官也故云出】望之堪數薦名儒茂材以備諫官【數所角翻】會稽鄭朋隂欲附望之【會土外翻】上書言車騎將軍高遣客為姦利郡國及言許史子弟罪過章視周堪【師古曰視讀曰示以朋所奏之章示堪也】堪白令朋待詔金馬門朋奏記望之曰今將軍規橅云若管晏而休遂行日昃至周召乃留乎【師古曰問望之立意當趣如管晏而止為欲恢廓其道日昃不食追周召之蹟然後已乎橅讀曰模其字從木】若管晏而休則下走將歸延陵之臯没齒而已矣【應劭曰下走僕也張晏曰吳公子札食邑延陵薄吳王之行棄國而耕於皐澤朋云望之所為若但如管晏則不處漢朝將歸會稽尋延陵之軌隱耕皐澤之中也師古曰下走自謙言趨走之使也没齒終身也】如將軍興周召之遺業親日昃之兼聽則下走其庶幾願竭區區奉萬分之一【召讀曰邵庶幾居希翻】望之始見朋接待以意【師古曰與之相見納用其說也余謂接待以意者推誠待之接以殷勤】後知其傾邪絶不與通朋楚士怨恨【張晏曰朋會稽人會稽并屬楚蘇林曰楚人脆急也】更求入許史推所言許史事【推吐審翻】曰皆周堪劉更生教我我關東人何以知此於是侍中許章白見朋【見賢遍翻下同】朋出揚言曰我見言前將軍小過五大罪一【前將軍謂望之也】待詔華龍行汙穢【師古曰華音胡化翻姓也行下孟翻】欲入堪等堪等不納亦與朋相結恭顯令二人告望之等謀欲罷車騎將軍疏退許史狀【車騎將軍謂史高疏與疎同】候望之出休日【漢制自三署郎以上入直禁中者十日一出休沐】令朋龍上之事下弘恭問狀【上時掌翻下遐稼翻下既下同】望之對曰外戚在位多奢淫欲以匡正國家非為邪也恭顯奏望之堪更生朋黨相稱舉數譛訴大臣【數所角翻】毀離親戚欲以專擅權勢為臣不忠誣上不道請謁者召致廷尉時上初即位不省召致廷尉為下獄也【省悉井翻察也悟也】可其奏後上召堪更生曰繫獄上大驚曰非但廷尉問邪以責恭顯皆叩頭謝上曰令出視事恭顯因使史高言上新即位未以德化聞天下而先驗師傅既下九卿大夫獄【劉更生為宗正九卿也周堪為光禄大夫聞音問下遐稼翻】宜因決免於是制詔丞相御史前將軍望之傅朕八年【宣帝五鳳二年蕭望之為太子太傅至黄龍元年為八年】無它罪過今事久遠識忘難明【師古曰言不能盡記有遺忘者故難明忘巫放翻】其赦望之罪收前將軍光禄勲印綬及堪更生皆免為庶人 二月丁巳立弟竟為清河王【考異曰荀紀竟作寛今從漢書】 戊午隴西地震敗城郭屋室壓殺人衆【敗補邁翻考異曰劉向傳云三月地大震今從元紀】 三月立廣陵厲王子霸為王【宣帝五鳳四年廣陵厲王胥以罪自殺國除今復立其子】 詔罷黄門乘輿狗馬【師古曰黄門近署也故親幸之物屬焉百官表黄門寺屬少府乘繩證翻】水衡禁囿【百官表水衡都尉屬官有禁囿等九官令丞】宜春下苑【孟康曰宜春宫名也在杜縣東晉灼曰史記云葬二世杜南宜春苑中師古曰宜春下苑即今京城東南隅曲江池是】少府佽飛外池【百官表少府屬官有左弋十二官令丞武帝大初元年更名左弋為佽飛佽飛掌弋射有九丞兩尉如淳曰佽飛其矰繳以射鳬鴈給祭祀是故有池也佽飛荆人入水斬蛟勇士也故以名官佽音次】嚴籞池田【蘇林曰嚴飾池上之屋及其池也晉灼曰嚴籞射苑也許慎曰嚴弋射所蔽也池田苑中田也師古曰晉說是也】假與貧民又詔赦天下舉茂材異等直言極諫之士 夏四月立子驁為皇太子【鷔五到翻】待詔鄭朋薦太原太守張敞先帝名臣宜傅輔皇太子上以問蕭望之望之以為敞能吏任治煩亂材輕非師傅之器【敞傳云敝無威儀罷朝會過走馬章臺街使御吏驅自以便面拊馬又為婦畫眉所謂材輕也任音壬治直之翻】天子使使者徵敞欲以為左馮翊會病卒 詔賜蕭望之爵關内侯給事中朝朔望【朝直遥翻考異曰元紀此詔在今冬按劉向傳云前弘恭石顯秦望之等獄決三月地大震然則望之等黜免在今春地震前也又曰夏客星見昴卷舌間上感悟下詔賜望之爵關内侯望之傳曰後數月賜望之爵關内侯盖紀見望之死在十二月因置此詔於彼上耳】關東饑齊地人相食 秋七月己酉地復震【復扶又翻下同】<br />
<br />
  【考異曰劉向傳曰冬地復震元紀此月詔曰一年中地再動漢紀在七月己酉今從之】 上復徵周堪劉更生欲以為諫大夫弘恭石顯白皆以為中郎【百官表諫大夫秩比八百石中郎秩比六百石並屬光禄勲】上器重蕭望之不已欲倚以為相【相息亮翻】恭顯及許史兄弟侍中諸曹皆側目於望之等更生乃使其外親上變事【外親謂母黨也上時掌翻下同】言地震殆為恭等不為三獨夫動【應劭曰三獨夫謂蕭望之周堪及向師古曰獨夫猶言匹夫也殆近也為于偽翻】臣愚以為宜退恭顯以章蔽善之罰【師古曰章明也】進望之等以通賢者之路如此則太平之門開災異之原塞矣【塞悉則翻下同】書奏恭顯疑其更生所為白請考姦詐辭果服遂逮更生繫獄免為庶人會望之子散騎中郎伋亦上書訟望之前事【散騎中郎者本為中郎而加散騎官也】事下有司復奏望之前所坐明白無譖訴者【師古曰言望之自有罪非人讒譛而訴之也下遐稼翻復扶又翻下同】而教子上書稱引亡辜之詩【史不載伋書不知其所稱引者何詩詩變雅云無罪無辜讒口嗸嗸豈伋所引者即此詩乎亡古無字通】失大臣體不敬請逮捕弘恭石顯等知望之素高節不詘辱【詘與屈同】建白望之前幸得不坐復賜爵邑不悔過服罪深懷怨望教子上書歸非於上【師古曰言歸惡於天子也】自以託師傅終必不坐【師古曰言恃舊恩自謂終無罪坐懷此心】非頗屈望之於牢獄塞其怏怏心則聖朝無以施恩厚上曰蕭太傅素剛安肯就吏顯等曰人命至重【言人所重者性命也】望之所坐語言薄罪【既以語言為薄罪則不當下吏孝元於此不能破恭顯之姦可謂不明矣】必無所憂上乃可其奏冬十二月顯等封詔以付謁者敕令召望之手付因令太常急發執金吾車騎馳圍其第【太常掌諸陵縣執金吾掌徼循京師蕭望之時居杜陵故令太常發執金吾車騎往圍其第以恐脅之速其自盡也】使者至召望之望之以問門下生魯國朱雲雲者好節士【好呼到翻】勸望之自裁【自裁猶自殺也】於是望之仰天歎曰吾嘗備位將相年踰六十矣老入牢獄苟求生活不亦鄙乎字謂雲曰游【師古曰朱雲字游呼其字】趣和藥來【趣讀曰促和戶卧翻】無久留我死遂飲鴆自殺【果墮恭顯計中】天子聞之驚拊手曰曩固疑其不就牢獄果然殺吾賢傅是時太官方上晝食【上時掌翻】上乃卻食為之涕泣哀動左右【詩云啜其泣矣何嗟及矣為于偽翻】於是召顯等責問以議不詳【師古曰詳審也】皆免冠謝良久然後已上追念望之不忘每歲時遣使者祠祭望之冢終帝之世【平曰墓封曰冢高曰墳】臣光曰甚矣孝元之為君易欺而難悟也【易以䜴翻】夫恭顯之譖訴望之其邪說詭計誠有所不能辯也至于始疑望之不肯就獄恭顯以為必無憂已而果自殺則恭顯之欺亦明矣在中智之君孰不感動奮發以厎邪臣之罰【厎致也】孝元則不然雖涕泣不食以傷望之而終不能誅恭顯纔得其免冠謝而已如此則姦臣安所懲乎是使恭顯得肆其邪心而無復忌憚者也【復扶又翻】<br />
<br />
  是歲弘恭病死石顯為中書令 初武帝滅南越開置珠厓儋耳郡【事見二十卷武帝元鼎六年儋丁甘翻】在海中洲上【師古曰居海中之洲也水中可居者曰洲】吏卒皆中國人多侵陵之其民亦暴惡自以阻絶數犯吏禁【數所角翻】率數年壹反殺吏漢輒發兵擊定之二十餘年間凡六反【據賈捐之傳自初為郡至昭帝始元元年二十餘年間凡六反】至宣帝時又再反【始元五年罷儋耳郡并屬珠厓至宣帝神爵三年珠厓三縣反後七年甘露元年九縣復反】上即位之明年珠厓山南縣反發兵擊之諸縣更叛連年不定【海中洲上以黎母山為王環山列置諸縣山南縣盖置於黎母山之南也師古曰吏音工衡翻】上博謀於羣臣欲大發軍待詔賈捐之曰【捐之時待詔金馬門】臣聞堯舜禹之聖德地方不過數千里西被流沙東漸于海朔南暨聲教【師古曰此引禹貢之辭漸入也一曰浸也朔北方也暨及也被皮義翻漸子廉翻】言欲與聲教則治之不欲與者不彊治也【與讀曰豫治直之翻彊其兩翻】故君臣歌德【師古曰言皆有德可歌頌】含氣之物各得其宜武丁成王殷周之大仁也然地東不過江黄【杜預曰江國在汝南安陽縣黄國今弋陽縣】西不過氐羌南不過蠻荆北不過朔方是以頌聲並作視聽之物咸樂其生【樂音洛】越裳氏重九譯而獻此非兵革之所能致也【晉灼曰遠國使來因九譯言語乃通也張晏曰越不著衣裳慕中國化遣譯來著衣裳故曰越裳也師古曰越裳自是國名非以襲衣裳始為稱號也王充論衡作越嘗此則不作衣裳之字明矣晉志曰吳孫皓置九德郡即周時越裳氏地】以至于秦興兵遠攻貪外虚内而天下潰畔孝文皇帝偃武行文當此之時斷獄數百賦役輕簡【斷丁亂翻下同】孝武皇帝厲兵馬以攘四夷天下斷獄萬數賦煩役重寇賊並起軍旅數發【數所角翻】父戰死於前子鬭傷於後女子乘亭障孤兒號於道老母寡婦飲泣巷哭【師古曰涙流被面以入于口故言飲泣也巷哭者哭于路也號戶刀翻】是皆廓地泰大征伐不休之故也今關東民衆久困流離道路人情莫親父母莫樂夫婦【樂音洛】至嫁妻賣子法不能禁義不能止此社稷之憂也今陛下不忍悁悁之忿【悁縈年翻又吉掾翻忿也憂也詩中心悁悁又急躁貌】欲驅士衆擠之大海之中【師古曰擠墜也音子詣翻又子奚翻余謂擠排也推也】快心幽冥之地非所以救助饑饉保全元元也詩云蠢爾蠻荆大邦為讐【師古曰詩小雅采芑之詩也蠢動貌也蠻荆荆州之蠻也言敢與大國為讐敵也】言聖人起則後服中國衰則先畔自古而患之何况乃復其南方萬里之蠻乎【言珠厓又在蠻荆之南去京師萬里復扶又翻】駱越之人【南越王尉佗以兵威役屬西甌駱師古曰西甌即駱越也言西者以别束甌也余謂今安南之地古之駱越也珠厓盖亦駱越地宋白曰高貴二州亦古駱越地】父子同川而浴相習以鼻飲【范成大曰今邕管溪洞及沿海喜鼻飲隨貧富以銀錫陶器或大瓢盛水入塩并山薑汁數滴器側有竅施管如瓶觜内鼻中吸水升腦下入喉吸水時含魚肉鮓一臠故水得安流入鼻不與氣相激既飲必噫氣謂涼腦快膈莫此若但可飲水或傳為飲酒非是】與禽獸無異本不足郡縣置也顓顓獨居一海之中【師古曰顓與專同專專猶區區也一曰圜貌也】霧露氣濕多毒草蟲蛇水土之害人未見虜戰士自死又非獨珠厓有珠犀瑇瑁也【海中有珠池珠母者蚌也採珠必蜑丁皆居海艇中以大舶環池採珠以石懸大絙别以小繩繫蜑腰没水取珠氣迫則撼繩繩動舶人覺乃絞取人緣大絙上然而死於採珠者亦多矣此我太祖皇帝所以罷劉氏媚川都也師古曰犀狀如牛頭如猪而四足類象黑色一角當額前鼻上又有小角劉欣明交州記曰犀其毛如豕蹄冇三甲頭如馬有三角鼻上角短額上頂上角長異物志曰角中特有光耀白理如綫自本達末則為通天犀抱朴于曰通天犀有白理如綫者以盛米雞即駭矣其真者刻為魚衘入水水開三尺本草圖經曰犀出永昌山谷及益州今出南海者為上郭璞爾雅注曰犀三角一在頂上一在額上一在鼻上鼻上者即食角小而不橢瑇瑁如龜其甲相覆而生若甲然甲上有斑文瑇音代瑁音妹】棄之不足惜不擊不損威其民譬猶魚鼈何足貪也臣竊以往者羌軍言之【此盖指宣帝神爵元年羌反時】暴師曾未一年兵出不踰千里費四十餘萬萬大司農錢盡乃以少府禁錢續之【續漢志大司農掌諸錢穀金帛諸貨幣邊郡諸官請調度者皆為報給損多益寡取相給足百官表少府掌山林池澤之税以給共養應劭註曰名曰禁錢以給私養自别為藏少者小也故稱少府師古曰大司農供軍國之用少府以養天子也】夫一隅為不善費尚如此况於勞師遠攻亡士毋功乎【毋與無同】求之往古則不合施之當今又不便臣愚以為非冠帶之國禹貢所及春秋所治皆可且無以為【師古曰為猶用也治直之翻】願遂棄珠厓專用恤關東為憂上以問丞相御史御史大夫陳萬年以為當擊丞相于定國以為前日興兵擊之連年護軍都尉校尉及丞凡十一人還者二人卒士及轉輸死者萬人以上費用三萬萬餘尚未能盡降【降戶江翻】今關東困乏民難揺動捐之議是上從之捐之賈誼曾孫也<br />
<br />
  三年春詔曰珠厓虜殺吏民背畔為逆【背蒲妹翻】今廷議者或言可擊或言可守或欲棄之其指各殊朕日夜惟思議者之言羞威不行則欲誅之狐疑辟難則守屯田【師古曰辟讀曰避下同欲屯田與之相守以待其敝】通乎時變則憂萬民夫萬民之饑餓與遠蠻之不討危孰大焉且宗廟之祭凶年不備【王制冢宰制國用視年之豐耗祭用數之仂鄭氏曰筭今年一歲經用之數用其什一夫以凶年之入制經用之什一以供祭則宗廟之禮宜有不備者矣】况乎辟不嫌之辱哉【嫌當讀作慊慊之為言厭也意自足也】今關東大困倉庫空虛無以相贍又以動兵非特勞民凶年隨之其罷珠厓郡民有慕義欲内屬便處之【師古曰欲有來入内郡者所至之處即安置之余謂便處者各隨其所便而處之也處昌呂翻】不欲勿彊【彊其兩翻】 夏四月乙未晦茂陵白鶴舘災赦天下夏旱 立長沙煬王弟宗為王【長沙煬王旦定王發之玄孫初元元年薨】<br />
<br />
  【無後今立其弟紹封鄭氏曰煬音供養之養謚法好内遠禮曰煬去禮遠衆曰煬】 長信少府貢禹上言諸離宫及長樂宫衛可減其大半以寛繇役【繇讀曰徭】六月詔曰朕惟烝庶之饑寒【烝衆也】遠離父母妻子【離力智翻】勞於非業之作【師古曰不急之事故云非業也】衛於不居之宫恐非所以佐隂陽之道也其罷甘泉建章宫衛令就農百官各省費【師古曰費用之物務減損】條奏毋有所諱 是歲上復擢周堪為光禄勲【復扶又翻】堪弟子張猛為光禄大夫給事中大見信任【猛張騫孫也】<br />
<br />
  四年春正月上行幸甘泉郊泰畤【畤音止下同】三月行幸河東祠后土赦汾隂徒【徒有罪居作者】<br />
<br />
  五年春正月以周子南君為周承休侯【文穎曰姓姬名延其祖父姫嘉本周後武帝元鼎四年封為周子南君奉周祀師古曰承休侯國在潁川郡】 上行幸雍【雍於用翻】祠五畤【畤音止】 夏四月有星孛于參【孛蒲内翻天文志參為白虎三星直者是為衡石下有三星鋭曰罰為斬艾事其外四星左右肩股也參所今翻】 上用諸儒貢禹等之言詔太官毋日殺【師古曰不得日日宰殺】所具各減半【師古曰食具也】乘輿秣馬無乏正事而已【師古曰秣養馬以粟秣食之也正事謂駕供祭祀蒐狩之事非游田者也乘繩證翻秣音末】罷角抵上林宫館希御幸者【角抵見二十一卷武帝元封三年】齊三服官北假田官【李斐曰主假賃見官田與民收其假税也故置田農之官晉灼曰匈奴傳秦始皇渡河據陽山北假中王莽傳五原北假膏壤殖穀北假地名師古曰晉說是也酈道元曰自高闕以束夾山帶河陽山以西皆北假也】鹽鐵官常平倉【武帝置鹽鐵官宣帝置常平倉】博士弟子毋置員以廣學者【武帝為博士官置弟子五十人昭帝增弟子員滿百人宣帝末增倍之今不限員數以廣學者後數年以用度不足更為設員千人】令民有能通一經者皆復【復方目翻】省刑罰七十餘事 陳萬年卒六月辛酉長信少府貢禹為御史大夫禹前後言得失書數十上【上時掌翻】上嘉其質直多采用之 匈奴郅支單于自以道遠又怨漢擁護呼韓邪而不助已困辱漢使者江乃始等遣使奉獻因求侍子【郅支遣子入侍見上卷宣帝甘露元年】漢議遣衛司馬谷吉送之【谷姓也】御史大夫貢禹博士東海匡衡以為郅支單于郷化未醇【師古曰不雜曰醇醇壹也厚也郷讀曰嚮下同】所在絶遠宜令使者送其子至塞而還【還從宣翻又如字】吉上書言中國與夷狄有羈縻不絶之義今既養全其子十年德澤甚厚空絶而不送近從塞還示棄捐不畜【畜許六翻】使無郷從之心【師古曰郷從謂嚮化而從命也】棄前恩立後怨不便議者見前江乃始無應敵之數智勇俱困以致恥辱即豫為臣憂臣幸得建彊漢之節承明聖之詔宣諭厚恩不宜敢桀【師古曰言郅支畏威當不敢桀猾也】若懷禽獸心加無道於臣則單于長嬰大罪【師古曰嬰猶帶也】必逃遁遠舍【師古曰舍止也】不敢近邊【近其靳翻】没一使以安百姓國之計臣之願也願送至庭【郅支單于庭也】上許焉既至郅支單于怒竟殺吉等【考異曰陳湯傳初元四年郅支求侍子元帝紀五年谷吉使匈奴不還湯傳又云御史大夫貢禹議吉不可遣按禹今年六月始為御史大夫或者郅支以四年求侍子而吉以五年使匈奴也】自知負漢又聞呼韓邪益彊恐見襲擊欲遠去會康居王數為烏孫所困【數所角翻下同】與諸翕侯計以為匈奴大國烏孫素服屬之今郅支單于困阨在外可迎置東邊使合兵取烏孫而立之【師古曰言與郅支并力共滅烏孫以其地立郅支令居之也】長無匈奴憂矣即使使至堅昆通語郅支【宣帝黄龍元年郅支都堅昆】郅支素恐又怨烏孫【怨烏孫事亦見上卷黄龍元年】聞康居計大說【說讀曰悦】遂與相結引兵而西郅支人衆中寒道死餘財三千人【師古曰中寒傷於寒也道死死於道上也中竹仲翻財與纔同】到康居康居王以女妻郅支郅支亦以女予康居王【妻七細翻予讀曰與】康居甚尊敬郅支欲倚其威以脅諸國郅支數借兵擊烏孫深入至赤谷城殺略民人敺畜產去【師古曰敺與驅同】烏孫不敢追西邊空虚不居者五千里【西域傳烏孫國治赤谷城西至康居蕃地五千里若云空虚者五千里則自赤谷以西皆不居矣此已扺其國都不得云西邊也陳湯傳作且千里當從之】 冬十二月丁未貢禹卒丁巳長信少府薛廣德為御史大夫<br />
<br />
  永光元年春正月上行幸甘泉郊泰畤禮畢因留射獵薛廣德上書曰竊見關東困極人民流離陛下日撞亡秦之鍾【師古曰撞音丈江翻】聽鄭衛之樂臣誠悼之今士卒暴露從官勞倦【應劭曰從官謂宦者及虎賁羽林太醫太官是也師古曰從官親近天子常侍從者皆是也從才用翻下同】願陛下亟反宫思與百姓同憂樂【樂音洛】天下幸甚上即日還【還從宣翻又如字】 二月詔丞相御史舉質樸敦厚遜讓有行者【行下孟翻】光禄歲以此科第郎從官【師古曰始令丞相御史舉此四科人而擢用之而見在郎及從官又令光禄每歲依此科考校定其第高下用知其人賢否也】 三月赦天下 雨雪隕霜殺桑【雨于具翻】 秋上酎祭宗廟出便門【師古曰便門長安城南面西頭第一門酎直又翻】欲御樓舡薛廣德當乘輿車免冠頓首曰宜從橋詔曰大夫冠【乘繩證翻說文冠絭也所以絭髪弁冕之總名也】廣德曰陛下不聽臣臣自刎【刎扶粉翻】以血汙車輪陛下不得入廟矣【師古曰言不以理終不得立廟也一曰以見死傷犯於齋潔不得入廟祠也原父曰一說是也時上欲入廟汙烏故翻】上不說先敺光禄大夫張猛進曰【師古曰先驅導乘輿也說說曰悦讀曰驅】臣聞主聖臣直乘舡危就橋安聖主不乘危御史大夫言可聽上曰曉人不當如是邪【師古曰謂諫諍之言當如猛之詳善也】乃從橋 九月隕霜殺稼天下大饑丞相于定國大司馬車騎將軍史高御史大夫薛廣德俱以災異乞骸骨賜安車駟馬黄金六十斤罷太子太傅韋玄成為御史大夫【考異曰百官表七月癸未大司馬高免辛亥韋玄成為御史大夫十一月戊寅丞相定國免荀紀七月己未高免薛廣德傳酎祭後月餘以歲惡民流乞骸骨罷廣德為御史大夫凡十月免月日參差未知孰是故皆没不書】廣德歸縣其安車以傳示子孫為榮【師古曰縣其所賜安車以示榮幸也致仕縣車盖亦古法韋孟詩縣車之義以洎小臣是也貢父曰致仕縣車言休息不出也故韋孟及薛廣德自縣其安車也縣讀曰懸】 帝之為太子也從太中大夫孔霸受尚書及即位賜霸爵關内侯號褒成君【如淳曰為帝師教令成就故曰褒成君】給事中上欲致霸相位霸為人謙退不好權埶【好呼到翻】常稱爵位泰過何德以堪之御史大夫屢缺上輒欲用霸霸讓位自陳至于再三上深知其至誠乃弗用以是敬之賞賜甚厚 戊子侍中衛尉王接為大司馬車騎將軍【接平昌侯王無故之子】 石顯憚周堪張猛等數譛毀之【數所角翻】劉更生懼其傾危上書曰臣聞舜命九官【師古曰尚書禹作司空棄后稷契司徒皐陶作士垂共工益朕虞伯夷秩宗夔典樂龍納言凡九官也】濟濟相讓和之至也【濟子禮翻】衆臣和於朝則萬物和於野故簫韶九成鳳皇來儀【師古曰韶舜樂名舉簫管之屬示其備也於韶樂九奏則鳳皇見其容儀言感至和也】至周幽厲之際【師古曰厲王夷王之子厲王生宣王宣王生幽王】朝廷不和轉相非怨則日月薄食水泉沸騰山谷易處【師古曰薄迫也謂被掩迫也沸涌出也騰乘也言百川沸涌而相乘陵山頂隆高而盡崩壞陵谷易處】霜降失節【謂正月繁霜也正月夏之四月正陽之月也】由此觀之和氣致祥乖氣致異祥多者其國安異衆者其國危天地之常經古今之通義也今陛下開三代之業招文學之士優游寛容使得並進今賢不肖渾殽【師古曰言雜亂也渾音胡本翻】白黑不分邪正雜糅【師古曰糅和也音汝救翻】忠讒並進章交公車人滿北軍【如淳曰漢儀注中壘校尉主北軍壘門内尉一人主上書者獄上章於公車冇不如法者以付北軍尉北軍尉以法治之楊惲上書遂幽北闕北闕公車所在】朝臣舛午膠戾乖剌【師古曰言志意不和各相違背午音五故翻剌音來曷翻】更相讒愬轉相是非【更工衡翻】所以營惑耳目感移心意不可勝載【師古曰言其誣罔天子也營謂回繞之勝音升下同】分曹為黨【師古曰曹輩也】往往羣朋將同心以䧟正臣正臣進者治之表也正臣䧟者亂之機也乘治亂之機未知孰任而災異數見【治直吏翻數所角翻見賢遍翻】此臣所以寒心者也初元以來六年矣按春秋六年之中災異未有稠如今者也【師古曰稠多也音直流翻】原其所以然者由讒邪並進也讒邪之所以並進者由上多疑心既已用賢人而行善政如或譛之則賢人退而善政還矣【師古曰還謂收還也】夫執狐疑之心者來讒賊之口持不斷之意者開羣枉之門【斷丁亂翻】讒邪進則衆賢退羣枉盛則正士消故易有否泰【師古曰否音皮鄙翻】小人道長君子道消則政日亂君子道長小人道消則政日治【長知兩翻治直吏翻下同】昔者鯀共工驩兜與舜禹雜處堯朝【師古曰鯀崇伯之名即檮杌也共工少皥氏之後即窮奇也驩兜帝鴻氏之後即渾敦也鯀音工本翻共音恭驩音火官翻處昌呂翻】周公與管蔡並居周位當是時迭進相毀【師古曰迭互也音大結翻】流言相謗豈可勝道哉【勝音升】帝堯成王能賢舜禹周公而消共工管蔡故以大治榮華至今孔子與季孟偕仕于魯【師古曰季孟謂季孫孟孫皆桓公之後代執國權而卑公室余謂季孫孟孫季孟之通稱與孔子偕仕者季孫斯孟孫何忌也】李斯與叔孫俱宦於秦【師古曰叔孫者叔孫通也】定公始皇賢季孟李斯而消孔子叔孫故以大亂汙辱至今故冶亂榮辱之端在所信任信任既賢在於堅固而不移詩云我心匪石不可轉也【師古曰此柏舟之詩也言石性雖堅尚可移轉已志須確執德不傾過於石也】言守善篤也易曰渙汗其大號【師古曰此易渙卦九五爻辭也言王者渙然大發號令如汗之出也】言號令如汗汗出而不反者也今出善令未能踰時而反【師古曰踰時三月也】是反汗也用賢未能三旬而退是轉石也論語曰見不善如探湯【師古曰論語載孔子之言探湯言其除難無所避也探吐南翻】今二府奏佞讇不當在位【師古曰讇古謟字】歷年而不去故出令則如反汗用賢則如轉石去佞則如抜山【去羌呂翻】如此望隂陽之調不亦難乎是以羣小窺見間隙【間古莧翻】緣飾文字巧言醜詆【師古曰詆毀也辱也音丁禮翻】流言飛文譁于民間【放言於外以誣人曰流言孔穎逹曰流謂水流造作虛語使人傳之如水之流然故謂之流言為飛書以詆毀若今之匿名書曰飛文師古曰譁讙也音火瓜翻】故詩云憂心悄悄愠于羣小【師古曰柏舟言仁人不遇之詩悄悄憂貌愠怒也悄音千小翻愠於問翻】小人成羣誠足愠也昔孔子與顔淵子貢更相稱譽不為朋黨【師古曰事具見論語更工衡翻譽音余】禹稷與皐陶傳相汲引不為比周【師古曰事見尚書傳柱戀翻逓也比頻寐翻】何則忠於為國無邪心也今佞邪與賢臣並交戟之内【師古曰交戟謂宿衛者】合黨共謀違善依惡歙歙訿訿【詩小旻歙歙訿訿毛氏注曰潝潝然患其上訿訿然思不稱於上爾雅云潝潝訿訿莫供職也韓詩云不善之貌歙與潝同許急翻訿音紫】數設危險之言【數所角翻】欲以傾移主上如忽然用之此天地之所以先戒災異之所以重至者也【師古曰重音直用翻】自古明聖未有無誅而治者也故舜有四放之罰【師古曰謂舜流共工于幽州放驩兜于崇山竄三苖于三危殛鯀于羽山】孔子有兩觀之誅【應劭曰少正卯姦人之雄故孔子為司寇七日誅之於兩觀之下師古曰兩觀謂闕也觀古玩翻】然後聖化可得而行也今以陛下明知【知讀曰智】誠深思天地之心覽否泰之卦歷周唐之所進以為法原秦魯之所消以為戒【師古曰歷謂歷觀之原謂思其本也周成王唐唐堯】考祥應之福災異之禍以揆當世之變放遠佞邪之黨壞散險詖之聚【遠于願翻壞音怪師古曰揆度也險言曰詖詖彼義翻】杜閉羣枉之門廣開衆正之路決斷狐疑分别猶豫【斷丁亂翻别彼列翻】使是非炳然可知則百異消滅而衆祥並至太平之基萬世之利也顯見其書愈與許史比而怨更生等【比毗至翻下同】是歲夏寒日青無光顯及許史皆言堪猛用事之咎上内重堪又患衆口之浸潤【鄭氏曰譛人之言如水之浸潤漸以成之孔子曰浸潤之譛不行焉可謂明也已矣】無所取信時長安令楊興以材能幸常稱譽堪【譽音余下同】上欲以為助乃見問興朝臣斷斷不可光禄勲何邪【師古曰斷斷忿疾之意也斷音牛斤翻】興者傾巧士謂上疑堪因順指曰堪非獨不可于朝廷自州里亦不可也【周禮五黨為州五家為鄰五鄰為里漢人謂同州郷而居者為州里】臣見衆人聞堪與劉更生等謀毀骨肉以為當誅故臣前書言堪不可誅傷為國養恩也【為于偽翻下同】上曰然此何罪而誅今宜奈何興曰臣愚以為可賜爵關内侯食邑三百戶勿令典事明主不失師傅之恩此最策之得者也上於是疑之司隸校尉琅邪諸葛豐【姓譜葛氏先本琅邪諸縣人徙陽都時人本其先之所居謂之諸葛氏風俗通云葛嬰為陳涉將冇功而誅孝文録其後封諸縣侯因并氏焉】始以剛直特立著名於朝數侵犯貴戚【數所角翻下同】在位者多言其短後坐春夏繫治人【春夏生長之時故仲春省囹圉去桎梏毋肆掠止獄訟仲夏挺重囚益其食春夏而繫治人為不順天時】徙城門校尉豐於是上書告堪猛罪上不直豐乃制詔御史城門校尉豐前與光禄勲堪光禄大夫猛在朝之時數稱言堪猛之美豐前為司隸校尉不順四時修法度專作苛暴以獲恩威朕不忍下吏【下遐稼翻】以為城門校尉【百官表城門校尉掌京師十二城門屯兵】不内省諸已【省悉景翻】而反怨堪猛以求報舉【師古曰言舉其事以報怨】告按無證之辭暴揚難驗之罪毀譽恣意不顧前言【師古曰前言謂譽堪猛之美今乃吏言其短是不顧也譽音余下同】不信之大也朕憐豐之耆老不忍加刑其免為庶人又曰豐言堪猛貞信不立朕閔而不治又惜其材能未有所効其左遷堪為河東太守猛槐里令【槐里周之大丘秦曰廢丘高帝二年改為槐里屬右扶風】<br />
<br />
  臣光曰諸葛豐之於堪猛前譽而後毀其志非為朝廷進善而去姦也【去羌呂翻】欲比周求進而已矣斯亦鄭朋楊興之流烏在其為剛直哉人君者察美惡辯是非賞以勸善罰以懲姦所以為治也【治直吏翻】使豐言得實則豐不當黜若其誣罔則堪猛何辜焉今兩責而俱棄之則美惡是非果安在哉<br />
<br />
  賈捐之與楊興善捐之數短石顯【師古曰談說其長短余謂此言數陳其短耳數所角翻】以故不得官稀復進見【復扶又翻下同見賢遍翻】興新以材能得幸捐之謂興曰京兆尹缺【按百官表初元四年京兆尹成永光四年光禄大夫琅邪張譚為京兆尹四年不騰任免盖是時成已去而譚未除是以缺官也】使我得見言君蘭【張晏曰楊興字君蘭 考異曰荀紀作君簡今從漢書】京兆尹可立得興曰君房下筆言語妙天下【賈捐之字君房師古曰於天下最為精妙耳】使君房為尚書令勝五鹿充宗遠甚【續漢志曰尚書令承秦所置武帝用宦者更為中書謁者令是時石顯為中書令五鹿充宗為尚書令疑兩官並置也百官表成帝建始元年尚書令五鹿充宗為少府五年貶為玄菟太守逆而數之則知充宗是年猶為尚書令也姓譜趙大夫食采于五鹿以邑為氏】捐之曰令我得代充宗君蘭為京兆京兆郡國首尚書百官本天下真大冶士則不隔矣【治直吏翻】捐之復短石顯興曰顯方貴上信用之今欲進第從我計【師古曰第但也】且與合意即得入矣捐之即與興共為薦顯奏稱譽其美【譽音余下同】以為宜賜爵關内侯引其兄弟以為諸曹又共為薦興奏以為可試守京兆尹石顯聞知白之上乃下興捐之獄【下遐稼翻】令顯治之奏興捐之懷詐偽更相薦譽【更工衡翻】欲得大位罔上不道捐之竟坐棄市興髠鉗為城旦臣光曰君子以正攻邪猶懼不克况捐之以邪攻邪其能免乎<br />
<br />
  徙清河王竟為中山王 匈奴呼韓邪單于民衆益盛塞下禽獸盡單于足以自衛不畏郅支其大臣多勸單于北歸者【師古曰塞下無禽獸則射獵無所得又不畏郅支故欲北歸舊處】久之單于竟北歸庭民衆稍稍歸之其國遂定<br />
<br />
  二年春二月赦天下 丁酉御史大夫韋玄成為丞相右扶風鄭弘為御史大夫 三月壬戌朔日有食之夏六月赦天下 上問給事中匡衡以地震日食之變【匡衡時以博士給事中風俗通云匡魯邑句須為之宰其後氏焉】衡上疏曰陛下躬聖德開太平之路閔愚吏民觸法抵禁比年大赦使百姓得改行自新【比毗至翻行下孟翻】天下幸甚臣竊見大赦之後姦邪不為衰止【為于偽翻】今日大赦明日犯法相隨入獄此殆導之未得其務也今天下俗貪財賤義好聲色上侈靡親戚之恩薄婚姻之黨隆苟合徼幸【好呼到翻下同徼工堯翻】以身設利不改其原【師古曰設施也原本也】雖歲赦之刑猶難使錯而不用也【師古曰歲赦謂每歲一赦也錯置也錯音千故翻】臣愚以為宜壹曠然大變其俗夫朝廷者天下之楨幹也【楨幹版築之具題曰楨旁曰幹以築垣墻喻治天下也楨音貞】朝有變色之言則下有争鬭之患上有自專之士則下有不讓之人上有克勝之佐則下有傷害之心上有好利之臣則下有盗竊之民此其本也【師古曰言下之所行皆取化於上也】治天下者審所上而已【師古曰上謂崇尚也治直之翻】教化之流非家至而人說之也【師古曰非家家皆到人人勸說之也】賢者在位能者布職朝廷崇禮百僚敬讓道德之行由内及外自近者始然後民知所法遷善日進而不自知也詩曰商邑翼翼四方之極【師古曰商頌殷武之詩也商邑京師也極中也言商邑之禮俗翼翼然可則倣是乃四方之中正也】今長安天子之都親承聖化然其習俗無以異於遠方郡國來者無所法則或見侈靡而放效之【師古曰放依也音甫往翻】此教化之原本風俗之樞機宜先正者也臣聞天人之際精祲有以相盪【李奇曰祲氣也言天人精氣相動也師古曰祲謂隂陽之氣相浸漸以成炎祥者也音子鴆翻】善惡有以相推事作乎下者象動乎上隂變則静者動陽蔽則明者晻【鄧展曰静者動謂地震也明者晻謂日食也師古曰晻與暗同】水旱之災隨類而至陛下祗畏天戒哀閔元元宜省靡麗考制度近忠正遠巧佞【近其靳翻遠于願翻】以崇至仁匡失俗道德弘于京師淑問揚乎疆外【師古曰淑善也問名也】然後大化可成禮讓可興也上說其言【說讀曰悦】遷衡為光禄大夫<br />
<br />
  荀悦論曰夫赦者權時之宜非常典也漢興承秦兵革之後大愚之世比屋可刑【比毗至翻又毗必翻】故設三章之法大赦之令【約法三章事見九卷高帝元年赦自古有之至於大赦則始于秦高祖既并天下即皇帝位大赦天下後世因之為永制】蕩滌穢流與民更始【更工衡翻】時埶然也後世承業襲而不革失時宜矣若惠文之世無所赦之若孝景之時七國皆亂異心並起姦詐非一及武帝末年賦役繁興羣盜並起加以太子之事巫蠱之禍天下紛然百姓無聊及光武之際撥亂之後如此之比宜為赦矣<br />
<br />
  秋七月隴西羌彡姐旁種反【師古曰彡音所廉翻又音先廉翻姐音紫今西羌尚有此姓而彡音先冉翻】詔召丞相韋玄成等入議是時歲比不登【比毗至翻】朝廷方以為憂而遭羌變玄成等漠然莫有對者【師古曰漠無聲也音莫】右將軍馮奉世曰羌虜近在竟内背畔【竟古境字通用背蒲妹翻】不以時誅無以威制遠蠻臣願帥師討之【帥讀曰率】上問用兵之數對曰臣聞善用兵者役不再興糧不三載故師不久暴而天誅亟決【師古曰暴露也亟急也載子亥翻】往者數不料敵【師古曰料量也音聊】而師至於折傷再三發調則曠日煩費威武虧矣【折而設翻調徒弔翻】今反虜無慮三萬人【師古曰無慮舉凡之言無小思慮而大計也】法當倍用六萬人然羌戎弓矛之兵耳器不犀利【如淳曰今俗以刀兵利為犀晉灼曰犀堅也師古曰晉說是】可用四萬人一月足以決丞相御史兩將軍【兩將軍車騎將軍王接左將軍許嘉也】皆以為民方收斂時未可多發發萬人屯守之且足【且足猶言且可也歛力贍翻】奉世曰不可天下被饑饉【被皮義翻】士馬羸耗守戰之備久廢不簡【師古曰簡謂選揀】夷狄皆有輕邊吏之心而羌首難【師古曰言創首為寇難也難乃旦翻】今以萬人分屯數處虜見兵少必不畏懼戰則挫兵病師守則百姓不救如此怯弱之形見【見賢遍翻】羌人乘利諸種並和【種章勇翻師古曰和應也音胡卧翻】相扇而起臣恐中國之役不得止于四萬非財幣之所能解也故少發師而曠日【師古曰曠空也空費其日而無功也】與一舉而疾決利害相萬也【師古曰相比為萬倍也】固争之不能得【言奉世不能得請也】有詔益二千人於是遣奉世將萬二千人騎以將屯為名【師古曰且云領兵屯田不言討賊將即亮翻】典屬國任立護軍都尉韓昌為偏裨到隴西分屯三處【任立為右軍屯白石韓昌為前軍屯臨洮奉世為中軍屯首陽西極上任音壬】昌先遣兩校尉與羌戰羌衆盛多皆為所破殺兩校尉【校戶教翻】奉世具上地形部衆多少之計願益三萬六千人乃足以決事書奏天子大為發兵六萬餘人【上時掌翻為于偽翻】八月拜太常弋陽侯任千秋為奮武將軍以助之【昭帝時宫以捕上官桀功封弋陽侯千秋其子也弋陽侯國屬汝南郡應劭曰弋山在西北】冬十月兵畢至隴西十一月並進羌虜大破斬首數千級餘皆走出塞兵未決間漢復發募士萬人【復扶又翻】拜定襄太守韓安國為建威將軍【師古曰自别有此韓安國非武帝時人也】未進聞羌破而還詔罷吏士【吏軍吏士卒也】頗留屯田備要害處【師古曰要害者在我為要于敵為害也】<br />
<br />
  資治通鑑卷二十八  <br>
   </div> 

<script src="/search/ajaxskft.js"> </script>
 <div class="clear"></div>
<br>
<br>
 <!-- a.d-->

 <!--
<div class="info_share">
</div> 
-->
 <!--info_share--></div>   <!-- end info_content-->
  </div> <!-- end l-->

<div class="r">   <!--r-->



<div class="sidebar"  style="margin-bottom:2px;">

 
<div class="sidebar_title">工具类大全</div>
<div class="sidebar_info">
<strong><a href="http://www.guoxuedashi.com/lsditu/" target="_blank">历史地图</a></strong>  
<a href="http://www.880114.com/" target="_blank">英语宝典</a>  
<a href="http://www.guoxuedashi.com/13jing/" target="_blank">十三经检索</a> 
<br><strong><a href="http://www.guoxuedashi.com/gjtsjc/" target="_blank">古今图书集成</a></strong> 
<a href="http://www.guoxuedashi.com/duilian/" target="_blank">对联大全</a> <strong><a href="http://www.guoxuedashi.com/xiangxingzi/" target="_blank">象形文字典</a></strong> 

<br><a href="http://www.guoxuedashi.com/zixing/yanbian/">字形演变</a>  <strong><a href="http://www.guoxuemi.com/hafo/" target="_blank">哈佛燕京中文善本特藏</a></strong>
<br><strong><a href="http://www.guoxuedashi.com/csfz/" target="_blank">丛书&方志检索器</a></strong> <a href="http://www.guoxuedashi.com/yqjyy/" target="_blank">一切经音义</a>  

<br><strong><a href="http://www.guoxuedashi.com/jiapu/" target="_blank">家谱族谱查询</a></strong>  <strong><a href="http://shufa.guoxuedashi.com/sfzitie/" target="_blank">书法字帖欣赏</a></strong> 
<br>

</div>
</div>


<div class="sidebar" style="margin-bottom:0px;">

<font style="font-size:22px;line-height:32px">QQ交流群9:489193090</font>


<div class="sidebar_title">手机APP 扫描或点击</div>
<div class="sidebar_info">
<table>
<tr>
	<td width=160><a href="http://m.guoxuedashi.com/app/" target="_blank"><img src="/img/gxds-sj.png" width="140"  border="0" alt="国学大师手机版"></a></td>
	<td>
<a href="http://www.guoxuedashi.com/download/" target="_blank">app软件下载专区</a><br>
<a href="http://www.guoxuedashi.com/download/gxds.php" target="_blank">《国学大师》下载</a><br>
<a href="http://www.guoxuedashi.com/download/kxzd.php" target="_blank">《汉字宝典》下载</a><br>
<a href="http://www.guoxuedashi.com/download/scqbd.php" target="_blank">《诗词曲宝典》下载</a><br>
<a href="http://www.guoxuedashi.com/SiKuQuanShu/skqs.php" target="_blank">《四库全书》下载</a><br>
</td>
</tr>
</table>

</div>
</div>


<div class="sidebar2">
<center>


</center>
</div>

<div class="sidebar"  style="margin-bottom:2px;">
<div class="sidebar_title">网站使用教程</div>
<div class="sidebar_info">
<a href="http://www.guoxuedashi.com/help/gjsearch.php" target="_blank">如何在国学大师网下载古籍?</a><br>
<a href="http://www.guoxuedashi.com/zidian/bujian/bjjc.php" target="_blank">如何使用部件查字法快速查字?</a><br>
<a href="http://www.guoxuedashi.com/search/sjc.php" target="_blank">如何在指定的书籍中全文检索?</a><br>
<a href="http://www.guoxuedashi.com/search/skjc.php" target="_blank">如何找到一句话在《四库全书》哪一页?</a><br>
</div>
</div>


<div class="sidebar">
<div class="sidebar_title">热门书籍</div>
<div class="sidebar_info">
<a href="/so.php?sokey=%E8%B5%84%E6%B2%BB%E9%80%9A%E9%89%B4&kt=1">资治通鉴</a> <a href="/24shi/"><strong>二十四史</strong></a>&nbsp; <a href="/a2694/">野史</a>&nbsp; <a href="/SiKuQuanShu/"><strong>四库全书</strong></a>&nbsp;<a href="http://www.guoxuedashi.com/SiKuQuanShu/fanti/">繁体</a>
<br><a href="/so.php?sokey=%E7%BA%A2%E6%A5%BC%E6%A2%A6&kt=1">红楼梦</a> <a href="/a/1858x/">三国演义</a> <a href="/a/1038k/">水浒传</a> <a href="/a/1046t/">西游记</a> <a href="/a/1914o/">封神演义</a>
<br>
<a href="http://www.guoxuedashi.com/so.php?sokeygx=%E4%B8%87%E6%9C%89%E6%96%87%E5%BA%93&submit=&kt=1">万有文库</a> <a href="/a/780t/">古文观止</a> <a href="/a/1024l/">文心雕龙</a> <a href="/a/1704n/">全唐诗</a> <a href="/a/1705h/">全宋词</a>
<br><a href="http://www.guoxuedashi.com/so.php?sokeygx=%E7%99%BE%E8%A1%B2%E6%9C%AC%E4%BA%8C%E5%8D%81%E5%9B%9B%E5%8F%B2&submit=&kt=1"><strong>百衲本二十四史</strong></a>  <a href="http://www.guoxuedashi.com/so.php?sokeygx=%E5%8F%A4%E4%BB%8A%E5%9B%BE%E4%B9%A6%E9%9B%86%E6%88%90&submit=&kt=1"><strong>古今图书集成</strong></a>
<br>

<a href="http://www.guoxuedashi.com/so.php?sokeygx=%E4%B8%9B%E4%B9%A6%E9%9B%86%E6%88%90&submit=&kt=1">丛书集成</a> 
<a href="http://www.guoxuedashi.com/so.php?sokeygx=%E5%9B%9B%E9%83%A8%E4%B8%9B%E5%88%8A&submit=&kt=1"><strong>四部丛刊</strong></a>  
<a href="http://www.guoxuedashi.com/so.php?sokeygx=%E8%AF%B4%E6%96%87%E8%A7%A3%E5%AD%97&submit=&kt=1">說文解字</a> <a href="http://www.guoxuedashi.com/so.php?sokeygx=%E5%85%A8%E4%B8%8A%E5%8F%A4&submit=&kt=1">三国六朝文</a>
<br><a href="http://www.guoxuedashi.com/so.php?sokeytm=%E6%97%A5%E6%9C%AC%E5%86%85%E9%98%81%E6%96%87%E5%BA%93&submit=&kt=1"><strong>日本内阁文库</strong></a> <a href="http://www.guoxuedashi.com/so.php?sokeytm=%E5%9B%BD%E5%9B%BE%E6%96%B9%E5%BF%97%E5%90%88%E9%9B%86&ka=100&submit=">国图方志合集</a> <a href="http://www.guoxuedashi.com/so.php?sokeytm=%E5%90%84%E5%9C%B0%E6%96%B9%E5%BF%97&submit=&kt=1"><strong>各地方志</strong></a>

</div>
</div>


<div class="sidebar2">
<center>

</center>
</div>
<div class="sidebar greenbar">
<div class="sidebar_title green">四库全书</div>
<div class="sidebar_info">

《四库全书》是中国古代最大的丛书,编撰于乾隆年间,由纪昀等360多位高官、学者编撰,3800多人抄写,费时十三年编成。丛书分经、史、子、集四部,故名四库。共有3500多种书,7.9万卷,3.6万册,约8亿字,基本上囊括了古代所有图书,故称“全书”。<a href="http://www.guoxuedashi.com/SiKuQuanShu/">详细>>
</a>

</div> 
</div>

</div>  <!--end r-->

</div>
<!-- 内容区END --> 

<!-- 页脚开始 -->
<div class="shh">

</div>

<div class="w1180" style="margin-top:8px;">
<center><script src="http://www.guoxuedashi.com/img/plus.php?id=3"></script></center>
</div>
<div class="w1180 foot">
<a href="/b/thanks.php">特别致谢</a> | <a href="javascript:window.external.AddFavorite(document.location.href,document.title);">收藏本站</a> | <a href="#">欢迎投稿</a> | <a href="http://www.guoxuedashi.com/forum/">意见建议</a> | <a href="http://www.guoxuemi.com/">国学迷</a> | <a href="http://www.shuowen.net/">说文网</a><script language="javascript" type="text/javascript" src="https://js.users.51.la/17753172.js"></script><br />
  Copyright &copy; 国学大师 古典图书集成 All Rights Reserved.<br>
  
  <span style="font-size:14px">免责声明:本站非营利性站点,以方便网友为主,仅供学习研究。<br>内容由热心网友提供和网上收集,不保留版权。若侵犯了您的权益,来信即刪。scp168@qq.com</span>
  <br />
ICP证:<a href="http://www.beian.miit.gov.cn/" target="_blank">鲁ICP备19060063号</a></div>
<!-- 页脚END --> 
<script src="http://www.guoxuedashi.com/img/plus.php?id=22"></script>
<script src="http://www.guoxuedashi.com/img/tongji.js"></script>

</body>
</html>
