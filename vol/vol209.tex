<!DOCTYPE html PUBLIC "-//W3C//DTD XHTML 1.0 Transitional//EN" "http://www.w3.org/TR/xhtml1/DTD/xhtml1-transitional.dtd">
<html xmlns="http://www.w3.org/1999/xhtml">
<head>
<meta http-equiv="Content-Type" content="text/html; charset=utf-8" />
<meta http-equiv="X-UA-Compatible" content="IE=Edge,chrome=1">
<title>資治通鑒_210-資治通鑑卷二百九_210-資治通鑑卷二百九</title>
<meta name="Keywords" content="資治通鑒_210-資治通鑑卷二百九_210-資治通鑑卷二百九">
<meta name="Description" content="資治通鑒_210-資治通鑑卷二百九_210-資治通鑑卷二百九">
<meta http-equiv="Cache-Control" content="no-transform" />
<meta http-equiv="Cache-Control" content="no-siteapp" />
<link href="/img/style.css" rel="stylesheet" type="text/css" />
<script src="/img/m.js?2020"></script> 
</head>
<body>
 <div class="ClassNavi">
<a  href="/24shi/">二十四史</a> | <a href="/SiKuQuanShu/">四库全书</a> | <a href="http://www.guoxuedashi.com/gjtsjc/"><font  color="#FF0000">古今图书集成</font></a> | <a href="/renwu/">历史人物</a> | <a href="/ShuoWenJieZi/"><font  color="#FF0000">说文解字</a></font> | <a href="/chengyu/">成语词典</a> | <a  target="_blank"  href="http://www.guoxuedashi.com/jgwhj/"><font  color="#FF0000">甲骨文合集</font></a> | <a href="/yzjwjc/"><font  color="#FF0000">殷周金文集成</font></a> | <a href="/xiangxingzi/"><font color="#0000FF">象形字典</font></a> | <a href="/13jing/"><font  color="#FF0000">十三经索引</font></a> | <a href="/zixing/"><font  color="#FF0000">字体转换器</font></a> | <a href="/zidian/xz/"><font color="#0000FF">篆书识别</font></a> | <a href="/jinfanyi/">近义反义词</a> | <a href="/duilian/">对联大全</a> | <a href="/jiapu/"><font  color="#0000FF">家谱族谱查询</font></a> | <a href="http://www.guoxuemi.com/hafo/" target="_blank" ><font color="#FF0000">哈佛古籍</font></a> 
</div>

 <!-- 头部导航开始 -->
<div class="w1180 head clearfix">
  <div class="head_logo l"><a title="国学大师官网" href="http://www.guoxuedashi.com" target="_blank"></a></div>
  <div class="head_sr l">
  <div id="head1">
  
  <a href="http://www.guoxuedashi.com/zidian/bujian/" target="_blank" ><img src="http://www.guoxuedashi.com/img/top1.gif" width="88" height="60" border="0" title="部件查字,支持20万汉字"></a>


<a href="http://www.guoxuedashi.com/help/yingpan.php" target="_blank"><img src="http://www.guoxuedashi.com/img/top230.gif" width="600" height="62" border="0" ></a>


  </div>
  <div id="head3"><a href="javascript:" onClick="javascript:window.external.AddFavorite(window.location.href,document.title);">添加收藏</a>
  <br><a href="/help/setie.php">搜索引擎</a>
  <br><a href="/help/zanzhu.php">赞助本站</a></div>
  <div id="head2">
 <a href="http://www.guoxuemi.com/" target="_blank"><img src="http://www.guoxuedashi.com/img/guoxuemi.gif" width="95" height="62" border="0" style="margin-left:2px;" title="国学迷"></a>
  

  </div>
</div>
  <div class="clear"></div>
  <div class="head_nav">
  <p><a href="/">首页</a> | <a href="/ShuKu/">国学书库</a> | <a href="/guji/">影印古籍</a> | <a href="/shici/">诗词宝典</a> | <a   href="/SiKuQuanShu/gxjx.php">精选</a> <b>|</b> <a href="/zidian/">汉语字典</a> | <a href="/hydcd/">汉语词典</a> | <a href="http://www.guoxuedashi.com/zidian/bujian/"><font  color="#CC0066">部件查字</font></a> | <a href="http://www.sfds.cn/"><font  color="#CC0066">书法大师</font></a> | <a href="/jgwhj/">甲骨文</a> <b>|</b> <a href="/b/4/"><font  color="#CC0066">解密</font></a> | <a href="/renwu/">历史人物</a> | <a href="/diangu/">历史典故</a> | <a href="/xingshi/">姓氏</a> | <a href="/minzu/">民族</a> <b>|</b> <a href="/mz/"><font  color="#CC0066">世界名著</font></a> | <a href="/download/">软件下载</a>
</p>
<p><a href="/b/"><font  color="#CC0066">历史</font></a> | <a href="http://skqs.guoxuedashi.com/" target="_blank">四库全书</a> |  <a href="http://www.guoxuedashi.com/search/" target="_blank"><font  color="#CC0066">全文检索</font></a> | <a href="http://www.guoxuedashi.com/shumu/">古籍书目</a> | <a   href="/24shi/">正史</a> <b>|</b> <a href="/chengyu/">成语词典</a> | <a href="/kangxi/" title="康熙字典">康熙字典</a> | <a href="/ShuoWenJieZi/">说文解字</a> | <a href="/zixing/yanbian/">字形演变</a> | <a href="/yzjwjc/">金 文</a> <b>|</b>  <a href="/shijian/nian-hao/">年号</a> | <a href="/diming/">历史地名</a> | <a href="/shijian/">历史事件</a> | <a href="/guanzhi/">官职</a> | <a href="/lishi/">知识</a> <b>|</b> <a href="/zhongyi/">中医中药</a> | <a href="http://www.guoxuedashi.com/forum/">留言反馈</a>
</p>
  </div>
</div>
<!-- 头部导航END --> 
<!-- 内容区开始 --> 
<div class="w1180 clearfix">
  <div class="info l">
   
<div class="clearfix" style="background:#f5faff;">
<script src='http://www.guoxuedashi.com/img/headersou.js'></script>

</div>
  <div class="info_tree"><a href="http://www.guoxuedashi.com">首页</a> > <a href="/SiKuQuanShu/fanti/">四库全书</a>
 > <h1>资治通鉴</h1> <!--         下载:【右键另存为】即可 --></div>
  <div class="info_content zj clearfix">
  
<div class="info_txt clearfix" id="show">
<center style="font-size:24px;">210-資治通鑑卷二百九</center>
    資治通鑑卷二百九   宋 司馬光 撰<br />
<br />
  胡三省 音注<br />
<br />
  唐紀二十五【起著雍涒灘盡上章閹茂七月凡二年有奇】<br />
<br />
  中宗大和大聖大昭孝皇帝下<br />
<br />
  景龍二年春二月庚寅宫中言皇后衣笥裙上有五色雲起上令圖以示百官韋巨源請布之天下從之仍赦天下迦葉志忠奏昔神堯皇帝未受命天下歌桃李子【桃李子見一百八十卷隋煬帝大業十三年迦居加翻】文武皇帝未受命天下歌秦王破陣樂【破陣樂見一百九十二卷太宗貞觀二年】天皇大帝未受命天下歌堂堂【調露初京城民謡有側堂堂撓堂堂之言太常丞李嗣貞曰側者不正撓者不安自隋以來樂府有堂堂曲再言堂者唐再受命之象鄭樵曰堂堂陳後主所作唐高宗常歌之】則天皇后未受命天下歌娬媚娘【永徽後民歌娬媚娘曲盖隋時已有此曲矣娬音武】應天皇帝未受命天下歌英王石州【其歌不見於史志忠以上初封英王遂傅會以為受命之符】順天皇后未受命天下歌桑條韋【永徽末里歌有桑條韋也安時韋也樂志忠遂傅會以為后妃之德專蠶桑供宗廟事上桑韋歌十二篇】盖天意以為順天皇后宜為國母主蠶桑之事謹上桑韋歌十二篇【上時掌翻下同】請編之樂府皇后祀先蠶則奏之太常卿鄭愔又引而申之【愔於今翻】上悦皆受厚賞右補闕趙延禧上言周唐一統符命同歸故高宗封陛下為周王【顕慶二年帝封周王儀鳳二年徙封英王】則天時唐同泰獻洛水圖【見二百四卷武后垂拱三年】孔子曰其或繼周者雖百代可知也陛下繼則天子孫當百代王天下【王于况翻】上悦擢延禧為諫議大夫 丁亥蕭至忠上疏以為恩倖者止可富之金帛食以粱肉【上時掌翻疏所去翻食讀曰飤祥吏翻】不可以公器為私用今列位已廣冗員倍之干求未厭日月增數陛下降不貲之澤近戚有無涯之請賣官利己鬻法狥私臺寺之内朱紫盈滿忽事則不存軄務恃勢則公違憲章徒沗官曹無益時政上雖嘉其意竟不能用 三月丙辰朔方道大總管張仁愿築三受降城於河上【中受降城在黄河北岸南去朔方千三百餘里安北都護府治焉東受降城在勝州東北二百里西南去朔方千六百餘里西受降城在豐州北黄河外八十里東南去朔方千餘里宋祁曰中城南直朔方西城南直靈武東城南直榆關宋白曰東受降城東北至單于都護府百二十里東南至朔州四百里西南度河至勝州八里西至中受降城三百里本漢雲中郡地中受降城西北至天德軍二百里南至麟州四百里北至磧口五百里本秦九原郡地在榆林漢更名五原開元十年於此置安北大都護府西受降城東南度河至豐州八十里西南至定遠城七百里東北至磧口三百里降戶江翻】初朔方軍與突厥以河為境河北有拂雲祠【祠在拂雲堆因以為名厥九勿翻】突厥將入寇必先詣祠祈禱牧馬料兵而後度河時默啜悉衆西擊突騎施【騎奇寄翻】仁愿請乘虛奪取漠南地於河北築三受降城首尾相應以絶其南寇之路太子少師唐休暻以為兩漢以來皆北阻大河今築城寇境恐勞人費功終為虜有【璟俱永翻】仁愿固請不已上竟從之仁愿表留歲滿鎮兵以助其功【戍邉歲滿當歸者留以助築城之功】咸陽兵二百餘人逃歸仁愿悉擒之斬於城下軍中股慄六旬而成以拂雲祠為中城距東西兩城各四百餘里皆據津要【宋白曰東受降城本漢雲中郡地中受降城本秦九原郡地西受降城盖漢臨河縣舊理處】拓地三百餘里於牛頭朝那山北【朝那山注見二百三卷高宗弘道元年】置烽候千八百所以左玉鈐衛將軍論弓仁為朔方軍前鋒遊奕使戍諾真水為邏衛【遊奕使領遊兵以廵奕者也中受降城西二百里至大同川北行二百四十餘里至步越多山又東北三百餘里至帝割逹城又東北至諾真水杜佑曰遊奕於軍中選驍勇諳山川泉井者充日夕邏侯於亭障之外捉生問事其副使子將並久軍行人取善騎射人使疏吏翻】自是突厥不敢度山畋牧朔方無復寇掠【復扶又翻】減鎮兵數萬人仁愿建三城不置壅門及備守之具【壅門即古之懸或曰門外築地以遮壅城門今之甕城是也壅城之外又有八卦墻萬人敵皆以遮壅城門范祖禹曰張仁愿築三受降城不置甕門曲敵戰格】或問之仁愿曰兵貴進取不利退守寇至當併力出戰回首望城者猶應斬之安用守備生其恧之心也【恧女六翻】其後常元楷為朔方軍總管始築壅門人是以重仁愿而輕元楷 夏四月癸未置修文館大學士四員直學士八員學士十二員選公卿以下善為文者李嶠等為之【武德四年置修文館於門下省九年改曰弘文館五品以上曰學士六品已上曰直學士又有文學直舘皆它官領之武后垂拱後以宰相兼領館事號曰館主神龍元年避孝敬皇帝諱改曰昭文館二年改曰修文館上官昭容勸帝置大學士四人以象四時直學士八人以象八節學士十二人以象十二時】每遊幸禁苑或宗戚宴集學士無不畢從賦詩屬和【從才用翻屬之欲翻和戶卧翻】使上官昭容第其甲乙【北齊河清新令有昭容八十一御女之一也唐昭容位亞昭儀於九品之次第二是年冬方以上官倢伃為昭容】優者賜金帛同預宴者惟中書門下及長參王公親貴數人而已至大宴方召八座九列諸司五品以上預焉於是天下靡然爭以文華相尚儒學忠讜之士莫得進矣【讜音】黨 秋七月癸巳以左屯衛大將軍朔方道大總管張仁愿同中書門下三品 甲午清源尉呂元泰上疏【上時掌翻疏所去翻下同】以為邊境未寧鎮戍不息士卒困苦轉輸疲弊而營建佛寺日廣月滋勞人費財無有窮極昔黄帝堯舜禹湯文武惟以儉約仁義立德垂名晉宋以降塔廟競起而喪亂相繼由是好尚失所奢靡相高人不堪命故也伏願回營造之資充疆塲之費使烽燧永息羣生富庶則如來慈悲之施【喪息浪翻好呼到翻施式豉翻】平等之心孰過於此疏奏不省【省悉景翻】 安樂長寧公主及皇后妹郕國夫人上官婕妤婕妤母沛國夫人鄭氏尚宫柴氏賀婁氏【唐宫官有六尚職掌如六尚書尚宫二人正五品掌導引中宫總司記司言司簿司闈四司之官賀婁氏後為臨淄王所誅樂音洛婕妤音接予】女巫第五英兒隴西夫人趙氏皆依埶用事請謁受賕雖屠沽臧獲【臧獲奴婢也方言曰陑岱之間罵奴曰臧罵婢曰獲燕之北郊民而壻婢謂之臧女而婦奴謂之獲】用錢三十萬則别降墨敕除官斜封付中書時人謂之斜封官錢三萬則度為僧尼其員外同正試攝檢校判知官凡數千人【時有員外置之官有員外同正之官有試官有攝官有檢校官判謂判某官事知謂知某官事也】西京東都各置兩吏部侍郎為四銓選者歲數萬人【選須絹翻】上官婕妤及後宫多立外第出入無節朝士往往從之遊處以求進逹安樂公主尤驕横【朝直遥翻處昌呂翻横下孟翻】宰相以下多出其門與長寧公主競起第舍【長寧公主上女也下嫁楊慎交】以侈麗相高擬於宫掖而精巧過之安樂公主請昆明池上以百姓蒲魚所資不許公主不悦乃更奪民田作定昆池延袤數里【新書曰定言可抗訂之也朝野僉載定昆池方四十九里直抵南山考異曰新傳云四十九里直抵南山盖併池上田言之今從舊傳】累石象華山【華戶化翻】引水象天津【天津謂天河也河圖括地象曰河精上為天漢鄭玄曰天河水氣也精光運轉於天楊泉物理論曰星者元氣之英也漢水之精也氣發而著精華浮上宛轉隨流名曰天河一曰雲漢】欲以勝昆明故名定昆安樂有織成裙直錢一億花卉鳥獸皆如粟粒正視旁視日中影中各為一色上好擊毬【好呼到翻】由是風俗相尚駙馬武崇訓楊慎交洒油以築毬塲慎交恭仁曾孫也【恭仁楊師道之兄也】上及皇后公主多營佛寺左拾遺京兆辛替否上疏諫畧曰臣聞古之建官員不必備士有完行【行下孟翻】家有廉節朝廷有餘俸百姓有餘食伏惟陛下百倍行賞十倍增官金銀不供其印束帛不充於錫【錫賜也予也】遂使富商豪賈盡居纓冕之流鬻伎行巫或涉膏腴之地【賈音古伎渠綺翻】又曰公主陛下之愛女然而用不合於古義行不根於人心將恐變愛成憎翻福為禍何者竭人之力費人之財奪人之家愛數子而取三怨使邊疆之士不盡力朝廷之士不盡忠人之散矣獨持所愛何所恃乎君以人為本本固則邦寧【書五子之歌曰民惟邦本本固邦寧】邦寧則陛下之夫婦母子長相保也又曰若以造寺必為理體【理體猶言治體也避高宗諱以治為理】養人不足經邦則殷周以往皆暗亂漢魏已降皆聖明殷周已往為不長漢魏已降為不短矣陛下緩其所急急其所緩親未來而疎見在【見賢遍翻】失真實而冀虚無重俗人之為輕天子之業雖以隂陽為炭萬物為銅役不食之人使不衣之士猶尚不給【用漢劉陶語意】况資於天生地養風動雨潤而後得之乎一旦風塵再擾霜雹荐臻沙彌不可操干戈寺塔不足禳饑饉臣竊惜之疏奏不省【操于高翻省悉景翻】時斜封官皆不由兩省而授兩省莫敢執奏即宣示所司吏部員外郎李朝隱前後執破一千四百餘人怨謗紛然朝隱一無所顧【朝直遥翻】 冬十月己酉修文館直學士起居舍人武平一上表請抑損外戚權寵不敢斥言韋氏但請抑損己家上優制不許平一名甄以字行載德之子【武氏之盛載德封頴川郡王】 十一月庚申突騎施酋長娑葛自立為可汗殺唐使者御史中丞馮嘉賓遣其弟遮努等帥衆犯塞【騎奇寄翻酋慈由翻長知兩翻娑素何翻可從刋入聲汗音寒使疏吏翻帥讀曰率】初娑葛既代烏質勒統衆【見上卷神龍二年】父時故將闕啜忠節不服【將即亮翻啜陟劣翻 考異曰郭元振傳作阿史那闕啜忠節尖厥傳止謂之闕啜忠節文館記謂之阿史那忠節元振疏皆云忠節乃其名也突厥有五啜其一曰胡禄居闕啜或者忠節官為闕啜歟今從突厥傳 今按西突厥亦姓阿史那氏闕部落之名啜官名也忠節人名也諸家冇書阿史那闕啜忠節者詳書之也或書官以綴其名或書姓以綴其名者約文也】數相攻擊忠節衆弱不能支金山道行軍總管郭元振奏追忠節入朝宿衛忠節行至播仙城經畧使右威衛將軍周以悌說之曰【唐置四鎮經畧使於安西府數所角翻朝直遥翻下同使疏吏翻下間使同說輸芮翻】國家不愛高官顯爵以待君者以君有部落之衆故也今脱身入朝一老胡耳豈惟不保寵祿死生亦制於人手方今宰相宗楚客紀處訥用事不若厚賂二公請留不行發安西兵及引吐蕃以擊娑葛【相息亮翻處昌呂翻訥内骨翻吐從暾入聲】求阿史那獻為可汗以招十姓【獻阿史那彌射之孫元慶之子】使郭䖍瓘發拔汗那兵以自助【杜環經行記拔汗那國在怛邏斯南千里東隔山去踈勒二千餘里西去右國千餘里】既不失部落又得報仇比於入朝豈可同日語哉郭䖍瓘者歷城人【歷城縣漢晉屬濟南郡後魏以來帶齊州】時為西邊將忠節然其言遣間使賂楚客處訥請如以悌之策【將即亮翻間古莧翻】元振聞其謀上疏以為往歲吐蕃所以犯邊正為求十姓四鎮之地不獲故耳【求十姓四鎮事始二百五卷武后萬歲通天元年為于偽翻下能為同】比者息兵請和【謂入貢而金城公主下嫁也比毗至翻】非能慕悦中國之禮義也直以國多内難【謂贊普南征而死國中大亂嫡庶競立將相爭權自相屠滅難乃旦翻】人畜疫癘恐中國乘其弊故且屈志求自昵【昵尼質翻】使其國小安豈能忘取十姓四鎮之地哉今忠節不論國家大計直欲為吐蕃鄉導【畜許救翻鄉讀曰嚮】恐四鎮危機將從此始頃緣默啜憑陵所應者多兼四鎮兵疲敝埶未能為忠節經畧非憐突騎施也忠節不體國家中外之意而更求吐蕃吐蕃得志則忠節在其掌握豈得復事唐也【復扶又翻】往年吐蕃無恩於中國猶欲求十姓四鎮之地【即謂萬歲通天元年事】今若破娑葛有功請分于闐疎勒不知以何理抑之又其所部諸蠻及婆羅門等方不服若借唐兵助討之亦不知以何詞拒之是以古之智者皆不願受夷狄之惠盖豫憂其求請無厭【厭於鹽翻】終為後患故也又彼請阿史那獻者豈非以獻為可汗子孫欲依之以招懷十姓乎按獻父元慶叔父僕羅兄俀子及斛瑟羅懷道等皆可汗子孫也往者唐及吐蕃徧曾立之以為可汗欲以招撫十姓【武后垂拱元年冊元慶為可汗見二百三卷冊斛瑟羅按舊書亦在是卷二年俀子見二百五卷延載元年長安四年冊懷道為可汗見二百七卷僕羅俀子盖皆吐蕃所立俀吐猥翻】皆不能致尋自破滅何則此屬非有過人之才恩威不足以動衆雖復可汗舊種【復扶又翻種章勇翻】衆心終不親附况獻又疏遠於其父兄乎若使忠節兵力自能誘脅十姓【誘音酉】則不必求立可汗子孫也又欲令郭䖍瓘入拔汗那發其兵䖍瓘前此已嘗與忠節擅入拔汗那發兵不能得其片甲匹馬而拔汗那不勝侵擾【勝音升】南引吐蕃奉俀子還侵四鎮時拔汗那四旁無彊寇為援䖍瓘等恣為侵掠如獨行無人之境猶引俀子為患今北有娑葛急則與之并力内則諸胡堅壁拒守外則突厥伺隙邀遮【伺相吏翻】臣料䖍瓘等此行必不能如往年之得志内外受敵自䧟危亡徒與虜結隙令四鎮不安以臣愚揣之實為非計【揣初委翻】楚客等不從建議遣馮嘉賓持節安撫忠節侍御史呂守素處置四鎮【處昌呂翻】以將軍牛師奨為安西副都護發甘凉以西兵兼徵吐蕃以討娑葛娑葛遣使娑臘獻馬在京師聞其謀馳還報娑葛於是娑葛發五千騎出西安五千騎出撥換五千騎出焉耆五千騎出踈勒入寇【騎奇寄翻】元振在疎勒柵于河口不敢出忠節逆嘉賓於計舒河口娑葛遣兵襲之生禽忠節殺嘉賓禽呂守素於僻城縳於驛柱冎而殺之【冎古瓦翻 考異曰御史臺記云嘉賓為中丞神龍中起復持節甘凉時郭元振都督凉州奏中書令宗楚客受娑葛金兩石請紹封可汗楚客憾之既用事時議云委嘉賓與侍御史呂守素按元振元振竊知之乃諷番落害嘉賓於驛中獲函中敕云元振父亡匿不發喪至是為發之仍按其不臣之狀便誅之元振以為偽敕具以聞今從舊傳】 上以安樂公主將適左衛中郎將武延秀遣使召太子賓客武攸緒於嵩山【郎將即亮翻使疏吏翻】攸緒將至上敕禮官於兩儀殿設别位欲行問道之禮聽以山服葛巾入見不名不拜【見賢遍翻下辭見同】仗入【自太極殿前喚仗從東西上閤門入立於兩儀殿前】通事舍人引攸緒就位【引就問道之位】攸緒趨立辭見班中再拜如常儀【凡百官自中朝出為外官赴朝辭自外官入朝覲者引入見其辭見者不與百官序班自為班立謂之辭見班杜佑曰唐制供奉官左右散騎常侍門下中書侍郎諫議大夫給事中中書舍人左右遺補通事舍人在横班辭見者各從兼官班在正官之次品式令前官被召見及赴朝參致仕者在本品見任上以理解官者同在品下】上愕然竟不成所擬之禮上屢延之内殿頻煩寵錫皆謝不受親貴謁候寒温之外不交一言初武崇訓之尚公主也【帝盖自房陵還始以公主適崇訓】延秀數得侍宴【數所角翻】延秀美姿儀善歌舞公主悦之及崇訓死【見上卷元年】遂以延秀尚焉己卯成禮假皇后仗【唐六典宫官六尚尚服局有司仗典仗掌仗之官掌羽儀仗衛之事又按唐制皇后乘重翟厭翟翟車安車四望車金根車而公主乘厭翟車則下皇后一等此時盖以重翟及皇后儀衛假之也】分禁兵以盛其儀衛命安國相王障車【相息亮翻】庚辰赦天下【考異曰實録新舊紀皆云己卯大赦今從景龍文舘記成禮之明日】以延秀為太常卿兼右衛將軍辛巳宴羣臣於兩儀殿命公主出拜公卿公卿皆伏地稽首【稽音啟】 癸未牛師奨與突騎施娑葛戰於火燒城師奨兵敗沒娑葛遂陷安西【安西都護府時在龜兹】斷四鎮路【斷音短】遣使上表求宗楚客頭【使疏吏翻上時掌翻】楚客又奏以周以悌代郭元振統衆徵元振入朝【朝直遥翻】以阿史那獻為十姓可汗置軍焉耆以討娑葛娑葛遺元振書【遺于季翻】稱我與唐初無惡但讐闕啜宗尚書受闕啜金欲枉破奴部落馮中丞牛都護相繼而來【宗尚書謂楚客馮中丞謂嘉賓牛都護謂師奨各稱其官也】 豈得坐而待死又聞史獻欲來【史獻即阿史那獻約言之】徒擾軍州恐未有寧日乞大使商量處置元振奏娑葛書楚客怒奏言元振有異圖召將罪之元振使其子鴻間道具奏其狀乞留定西土不敢歸周以悌竟坐流白州復以元振代以悌【處昌呂翻間古莧翻復扶又翻 考異曰元載玄宗實録舊傳皆云復以元振代以悌元振奏稱西土未寧逗留不敢歸京師按既代以悌則復留居西邊矣何所逗留今從新傳】赦娑葛罪冊為十四姓可汗【西突厥先有十姓今併咽麫葛邏禄莫賀逹干都摩支為十四姓】 以婕妤上官氏為昭容 十二月御史中丞姚廷筠奏稱比見諸司不遵律令格式事無大小皆悉聞奏臣聞為君者任臣為臣者奉法萬機樷委不可徧覽豈有修一水竇伐一枯木皆取斷宸衷【比毗至翻斷丁亂翻】自今若軍國大事及條式無文者聽奏取進止自餘各準法處分【處昌呂翻分扶分翻】其有故生疑滯致有稽失望令御史糾弹從之 丁巳晦敕中書門下與學士諸王駙馬入閤守歲設庭燎置酒奏樂【閤内殿也守歲之宴古無之梁庾肩吾除夕詩聊傾栢葉酒試奠五辛盤盖江左已有此矣然未至君臣相與酣適也隋焬帝淫奢每除夜殿前諸院設火山數十盡沉香木根每一山皆焚沉香數車火光暗則以甲煎沃之焰起數丈香聞數十里一夜之間用沉香二百餘乘甲煎過二百餘石歐陽修詩隋宫守夜沉香火謂此也帝之為此亡隋之續耳】酒酣上謂御史大夫竇從一曰聞卿久無伉儷【酣戶甘翻伉苦浪翻儷力計翻】朕甚憂之今夕歲除為卿成禮從一但唯唯拜謝【為于偽翻唯于癸翻】俄而内侍引燭籠步障金縷羅扇自西廊而上【内侍之官唐從四品上掌在内侍奉出入宮掖宣傳之事後魏曰長秋卿北齊曰中侍中後周曰司内上士隋曰内侍唐因之中官之貴極於此矣若有殊勲懋績則有拜大將軍者仍兼内侍之官上時掌翻】扇後有人衣禮衣花釵【唐制命婦之服有翟衣内命婦受冊從蠶朝會外命婦嫁及受冊從蠶大朝會之服也青質繡翟編次於衣及裳重為九等一品翟九等花釵九樹二品翟八等花釵八樹三品至五品皆降殺以一禮衣者内命婦常參外命婦朝參辭見禮會之服也制同翟衣加雙佩小綬去舃加履人衣於既翻】令與從一對坐上命從一誦却扇詩數首【唐人成昏之夕有催粧詩却扇詩李商隱代董秀才却扇詩云莫將畫扇出帷來遮掩春山滯上才若道團圓是明月此中須放桂花開】扇却去花易服而出【去羌呂翻】徐視之乃皇后老乳母王氏本蠻婢也上與侍臣大笑詔封莒國夫人嫁為從一妻俗謂乳母之壻曰阿㸙從一每謁見及進表狀自稱翊聖皇后阿㸙時人謂之國㸙【阿烏葛翻㸙正奢翻見賢遍翻】從一欣然有自負之色<br />
<br />
  三年春正月丁卯制廣東都聖善寺【按西京已有聖善寺東都亦有聖善寺皆帝所建為武后追福】居民失業者數十家 長寧安樂諸公主多縱僮奴掠百姓子女為奴婢侍御史袁從之收繫獄治之【樂音洛治直之翻】公主訴於上上手制釋之從之奏稱陛下縱奴掠良人何以理天下上竟釋之 二月己丑上幸玄武門與近臣觀宫女拔河【以麻絙巨竹分朋而挽水謂之拔河以定勝負考異曰唐紀云觀宫女大酺今從實録】又命宫女為市肆公卿為商旅與<br />
<br />
  之交易因為忿爭言辭䙝慢上與后臨觀為樂【䙝息列翻樂音洛】丙申監察御史崔琬對仗彈宗楚客紀處訥潜通戎狄受其貨賂致生邊患【謂受闕啜忠節賂以致娑葛畔換也 考異曰景龍文舘記曰監察御史崔琬具衣冠對仗彈大學士兵部尚書郢國公宗楚客及侍中紀處訥時楚客在列奏言臣以庸妄叨居樞密中外朋結謀臣臣先奏聞計垂天鍳上頷之謂琬曰楚客事朕知且去侍仗下來至仗下後琬方讀奏敕令於西省對問中書門下奏無狀有進止即令復位初娑葛父子與阿史那忠節代為仇讐娑葛頻乞國家為除忠節安西都護郭元振表請如其奏宗楚客固執言忠節竭誠於國作扞玉闕若許娑葛除之恐非威彊拯弱之義上由是不許無何娑葛擅殺御史中丞馮嘉賓殿中侍御史呂守素破滅忠節侵擾四鎮時碎葉鎮守使中郎周以悌率鎮兵數百人大破之奪其所侵忠節及于闐部衆數萬口奏到上大悦拜以悌左屯衛將軍仍以元振四鎮經畧使授之敕書簿責元振宗議發勁卒令以悌同郭䖍瓘比討仍邀吐蕃及西域諸部計會同擊娑葛右臺御史解琬議稱不可後竟與之和娑葛聞前事大怨乃付元振狀稱宗先取忠節金上以問之宗具以前事奏時太平安樂二公主以親貴權寵各立黨與隂相傾奪爰自要官宰臣皆分為兩時太平尤與宗不善故諷琬以弹之外傳取娑葛金非也今從實録記】故事大臣被彈【被皮義翻】俯僂趨出【俛首為俯傴背為僂僂力主翻】立於朝堂待罪【朝直遥翻】至是楚客更憤怒作色自陳忠鯁為琬所誣上竟不窮問命琬與楚客結為兄弟以和解之時人謂之和事天子 壬寅以韋巨源為左僕射楊再思為右僕射並同中書門下三品 上數與近臣學士宴集令各效伎藝以為樂【數所角翻伎渠綺翻樂音洛】工部尚書張錫舞談容娘將作大匠宗晋卿舞渾脱【長孫無忌以烏羊毛為渾脱氈㡌人多効之謂之趙公渾脱因演以為舞】左衛將軍張洽舞黄麞【如意初里歌曰黄麞黄麞草裏藏彎弓射爾傷亦演以為舞】左金吾將軍杜元談誦婆羅門呪【今所謂天竺神呪也】中書舍人盧藏用效道士上章國子司業河東郭山惲獨曰臣無所解【上時掌翻惲於粉翻解戶買翻暁也】請歌古詩上許之山惲乃歌鹿鳴蟋蟀【鹿鳴宴群臣嘉賓蟋蟀取好樂無荒之義然山惲欲以所業自見以附於儒學而已非能納君於善】明日上賜山惲敕嘉美其意賜時服一襲上又嘗宴侍臣使各為迴波辭【時内宴酒酣侍臣率起為迴波舞故使為迴波辭】衆皆為謟語或自求榮祿諫議大夫李景伯曰迴波爾持酒巵微臣職在箴規侍宴既過三爵【左傳曰臣侍君宴不過三爵過三爵非禮也】諠譁竊恐非儀上不悦蕭至忠曰此真諫官也 三月戊午以宗楚客為中書令蕭至忠為侍中太府卿韋嗣立為中書侍郎同中書門下三品 【考異曰新表云嗣立守兵部尚書今從實録】中書侍郎崔湜趙彦昭並同平章事崔湜通於上官昭容故昭容引以為相【湜常職翻相息亮翻】彦昭張掖人也【張掖故匈奴渾邪王地漢武帝開置張掖郡及得縣應邵曰張國臂掖故曰張掖得郡所治匈奴王號也晉改得為永平後魏置張掖軍隋開皇十七年改永平為酒泉大業初改為張掖縣其地自西魏以來為甘州治所取州甘峻山為名音禄】時政出多門濫官充溢人以為三無坐處謂宰相御史及員外官也韋嗣立上疏以為比者造寺極多【比毗至翻】務取崇麗大則用錢百數十萬小則三五萬無慮所費千萬以上人力勞弊怨嗟盈路佛之為教要在降伏身心【降戶江翻】豈彫畫土木相誇壯麗萬一水旱為災戎狄構患雖龍象如雲將何救哉又食封之家其數甚衆昨問戶部云用六十餘萬丁一丁絹兩匹凡百二十餘萬匹【唐初之制一丁歲輸絹二匹】臣頃在太府每歲庸絹多不過百萬少則六七十萬匹【少詩沼翻下同】比之封家所入殊少夫有佐命之勲始可分茅胙土國初功臣食封者不過三二十家今以恩澤食封者乃踰百數國家租賦大半私門私門有餘徒益奢侈公家不足坐致憂危制國之方豈謂為得封戶之物諸家自徵僮僕依埶陵轢州縣多索裹頭【轢郎狄翻裹頭謂行槖□裏以自資者今謂荅頭朱古臥翻】轉行貿易煩擾驅廹不勝其苦不若悉計丁輸之太府使封家於左藏受之【勝音升藏徂浪翻】於事為愈【謂猶勝於封家自徵也】又員外置官數倍正闕曹署典吏困於秪承府庫倉儲竭於資奉又刺史縣令近年以來不存簡擇京官有犯及聲望下者方遣刺州吏部選人衰耄無手筆者方補縣令【選須絹翻下選法同】以此理人何望率化望自今應除三省兩臺及五品以上清望官【兩臺謂左右御史臺】皆先於刺史縣令中選用則天下理矣上弗聽 戊寅以禮部尚書韋温為太子少保同中書門下三品太常卿鄭愔為吏部尚書同平章事【案下書吏部侍郎同平章事鄭愔又考新書本紀是年是月是日書太常少卿鄭愔守吏部侍郎同中書門下平章事則知傳寫通鑑者誤以侍郎為尚書也】温皇后之兄也 太常博士唐紹以武氏昊陵順陵置守戶五百與昭陵數同梁宣王魯忠王墓守戶多於親王五倍【梁宣王武三思魯忠王武崇訓】韋氏褒德廟衛兵多於太廟【立褒德廟見上卷元年】上疏請量裁减不聽【量音良】紹臨之孫也【唐臨歷事高祖太宗高宗】 中書侍郎兼知吏部侍郎同平章事崔湜吏部侍郎同平章事鄭愔俱掌銓衡傾附勢要贓賄狼籍數外留人授擬不足逆用三年闕【選法之壞至於我宋極矣吏部注擬率一官而三人共之居之者一人未至者一人伺之者又一人稍有美闕伺之者又不特一人也豈止逆用三年闕哉】選法大壞湜父挹為司業受選人錢湜不之知長名放之【高宗總章二年裴行儉始設長名牓凡選人之集於吏部者得者留不得者放宋白曰長名牓定留放留者入選放者不得入選】其人訴曰公所親受某賂柰何不與官湜怒曰所親為誰當擒取杖殺之其人曰公勿杖殺將使公遭憂湜大慙侍御史靳恒與監察御史李尚隱對仗弹之【靳居焮翻恒戶登翻監古銜翻弹徒丹翻】上下湜等獄命監察御史裴漼按之【漼七罪翻】安樂公主諷漼寛其獄漼復對仗彈之夏五月丙寅愔免死流吉州湜貶江州司馬【舊志江州京師東南二千九百四十八里至東都二千一百九十七里】上官昭容密與安樂公主武延秀曲為申理【復扶又翻為于偽翻】明日以湜為襄州刺史【舊志襄州京師一千一百八十二里至東都八百五十三里】愔為江州司馬 六月右僕射同中書門下三品楊再思薨秋七月突騎施娑葛遣使請降【騎奇寄翻娑素何翻使疏吏翻降戶江翻】庚辰拜欽化可汗賜名守忠 八月己酉以李嶠同中書門下三品韋安石為侍中蕭至忠為中書令至忠女適皇后舅子崔無詖【詖彼義翻】成昏日上主蕭氏后主崔氏時人謂之天子嫁女皇后娶婦 上將祀南郊丁酉國子祭酒祝欽明國子司業郭山惲建言古者大祭祀后祼獻以瑶爵皇后當助祭天地太常博士唐紹蒋欽緒駁之以為鄭玄注周禮内司服惟有助祭先王先公無助祭天地之文皇后不當助祭南郊【周禮内宰大祭祀后祼獻則贊瑶爵亦如之注云謂祭宗廟王既祼而出迎牲后乃從後祼也獻謂薦腥薦熟后亦從後獻也瑶爵謂尸卒食王既酳尸后亞獻之其爵以瑶為飾又内司服掌王后之六服禕衣揄狄闕狄鞠衣展衣禒衣素沙注云禕衣榆狄闕狄三者皆祭服從王祭先王則服禕衣祭先公則服揄狄祭群小祀則服闕狄今世有圭衣者盖三狄之遺俗據周禮則内宰所謂大祭祀指言祭宗廟也祝欽明等因唐制以天地宗廟並為大祀遂以周禮大祭祀傅會其說以謟韋后而周禮鄭義所謂祼也獻也瑶爵也乃祭時行禮之三節今欽明言后祼獻以瑶爵亦皆鄭義自為之說也祼古玩翻駮北角翻】國子司業鹽官禇無量議【鹽官漢海鹽地舊有鹽官吴因立為縣名唐屬杭州】以為祭天惟以始祖為主不配以祖妣故皇后不應預祭韋巨源定儀注請依欽明議上從之以皇后為亞獻仍以宰相女為齋娘助執豆籩欽明又欲以安樂公主為終獻紹欽緒固爭乃止以巨源攝太尉為終獻欽緒膠水人也【膠水漢膠束國地晉武帝置長廣郡後魏為光州治所隋仁夀元年改長廣為膠水縣屬萊州】 己巳上幸定昆池命從官賦詩黄門侍郎李日知詩曰所願蹔思居者逸勿使時稱作者勞【從才用翻蹔與暫同】及睿宗即位謂日知曰當是時朕亦不敢言之【睿宗之言盖謂當時畏安樂公主之勢也】 九月戊辰以蘇瓌為右僕射同中書門下三品【瓌古回翻】 太平安樂公主各樹朋黨更相黨毁【更工衡翻】上患之冬十一月癸亥上謂修文館直學士武平一曰比聞内外親貴多不輯睦以何法和之平一以為此由讒謟之人隂為離間【比毗至翻間古莧翻】宜深加誨諭斥逐姦險若猶未巳伏願捨近圖遠抑慈存嚴示以知禁無令積惡上賜平一帛而不能用其言 上召前修文舘學士崔湜鄭愔入陪大禮乙丑上祀南郊赦天下并十惡咸赦除之【十惡恩赦之所不原】流人並放還齋娘有壻者皆改官 甲戌開府儀同三司平章軍國重事豆盧欽望薨【平章軍國重事盖自豆盧欽望始】 乙亥吐蕃贊普遣其大臣尚贊咄等千餘人逆金城公主【咄當沒翻 考異曰實錄乙亥吐蕃大臣尚贊吐等來迎女文舘記云吐蕃使其大首領瑟瑟告身贊咄金告身尚欽藏以下來迎金城公主譯者云贊咄猶此左僕射欽藏猶此侍中盖贊咄即贊吐也今從文舘記】 河南道廵察使監察御史宋務光【使疏吏翻下同】以於時食實封者凡一百四十餘家【唐制食實封者得真戶戶皆三丁以上一分入國開元定制以三丁為限租賦全入於封家】應出封戶者凡五十四州皆割上腴之田或一封分食數州而太平安樂公主又取高貲多丁者刻剝過苦應充封戶者甚於征役滑州地出綾縑【唐六典州貢方紋綾】人多趨射【趨七喻翻射而亦翻】尤受其弊人多流亡請稍分封戶散配餘州又徵封使者煩擾公私請附租庸每年送納上弗聽 時流人皆放還均州刺史譙王重福獨不得歸【重福徙均州見上卷神龍元年重直龍翻】乃上表自陳曰陛下焚柴展禮郊祀上玄蒼生並得赦除赤子偏加擯棄【赤子重福自謂也】皇天平分之道固若此乎天下之人聞者為臣流涕【為于偽翻】况陛下慈念豈不愍臣栖遑【栖遑者離索憂廹之意】表奏不報前右僕射致仕唐休璟年八十餘進取彌鋭娶賀婁<br />
<br />
  尚宫養女為其子婦十二月壬辰以休璟為太子少師同中書門下三品【璟俱永翻 考異曰舊紀誤作壬戌今從實錄】 甲午上幸驪山温湯庚子幸韋嗣立莊舍【别業為莊】以嗣立與周高士韋夐同族賜爵逍遥公【韋夐事見一百六十七卷陳高祖永定三年夐休正翻】嗣立皇后之踈屬也由是顧賞尤重乙巳還宫 是歲關中饑米斗百錢運山東江淮穀輸京師牛死什八九羣臣多請車駕復幸東都韋后家本杜陵不樂東遷乃使巫覡彭君卿等說上云今歲不利東行後復有言者【復扶又翻樂音洛覡刑狄翻說輸芮翻】上怒曰豈有逐糧天子邪乃止<br />
<br />
  睿宗玄真大聖大興孝皇帝上<br />
<br />
  【諱旦高宗第八子也初名旭輪後去旭名輪後改名旦初諡大聖真皇帝廟號睿宗天寶八載追尊玄真大聖皇帝十三載加尊玄真大聖大興孝皇帝】<br />
<br />
  景雲元年【是年六月改元唐隆七月始改元景雲】春正月丙寅夜中宗與韋后微行觀燈於市里又縱宫女數千人出遊多不歸者 上命紀處訥送金城公主適吐蕃處訥辭又命趙彦昭彦昭亦辭丁丑命左驍衛大將軍楊矩送之【驍堅堯翻】己卯上自送公主至始平二月癸未還宫公主至吐蕃贊普為之别築城以居之 庚戍上御棃園毬塲【程大昌曰棃園在光化門北光化門者禁苑南面西頭第一門在芳林景曜門之西也中宗令學士自芳林門入集於棃園分朋拔河則棃園在太極宫西禁苑之内矣開元二年玄宗置教坊於蓬萊宫上自教法曲謂之棃園弟子至天寶中即東宫置宜春北苑命宫女數百人為棃園弟子即是棃園者按樂之地而預教者名為弟子耳凡蓬萊宫宜春院皆不在棃園之内也】命文武三品以上抛毬及分朋拔河韋巨源唐休璟衰老隨絙踣地【絙古登翻踣蒲北翻】久之不能興上及皇后妃主臨觀大笑 夏四月丙戌上遊芳林園【按唐禁苑廣矣漢長安都城盡入唐苑之内而漕渠首受豐水北流矩折入於禁苑而東流又矩折北流而入於渭苑地自漕渠之東大安宫垣之西南出與宫城齊南列三門中曰芳林自芳林門而入禁苑其地以芳林園為稱】命公卿馬上摘櫻桃【櫻桃按爾雅名楔荆桃樹多隂先百果熟大如拇指圓而色朱味甜每一朶率一二十顆核如豆大以鶯所含亦名含桃】 初則天之世長安城東隅民王純家井溢浸成大池數十頃號隆慶池【池在隆慶坊南程大昌曰帝王之興若符瑞理固有之然而傅會者多六典所記隆慶坊有井忽湧為小池周袤十數丈常有雲氣或黄龍出其中至景雲間潜復出水其沼浸廣里人悉移居遂鴻洞為龍池然予詳而考之長安志曰龍池在躍龍門南本是平地自垂拱初載後因雨水流為小流後又引龍首渠水分溉之日以滋廣至景龍中彌亘數頃深至數丈常有雲龍之祥後因謂之龍池志又曰隋城外東南角有龍首堰自此堰分滻水北流至長樂坡分為二渠其西渠自永嘉坊西南流經興慶宫則是興慶之能變平地為龍池者實引滻之力也至六典所紀則全沒導滻之實乃言初時井溢已乃泉生合二水以成此池專以歸諸變化也】相王子五王列第於其北【壽春王成器臨淄王隆基衡陽王成義巴陵王隆範彭城王隆業五王皆相王子】望氣者言常鬰鬰有帝王氣比日尤盛【比毗至翻】乙未上幸隆慶池 【考異曰景龍文舘記以為其月十二日按長歷是月壬午朔今從實錄本紀】結綵為樓宴侍臣泛舟戲象以厭之【厭於葉翻時人以為玄宗受命之祥】定州人郎岌上言韋后宗楚客將為逆亂【岌魚及翻上時掌翻】韋后白上杖殺之五月丁卯許州司兵參軍偃師燕欽融復上言皇后淫亂干預國政【唐諸州兵曹司兵參軍事掌武官選兵甲器仗門禁管籥軍防烽堠傳驛畋獵燕因肩翻復扶又翻上時掌翻】宗族彊盛安樂公主武延秀宗楚客圖危宗社上召欽融面詰之欽融頓首抗言神色不撓上默然宗楚客矯制令飛騎撲殺之【詰去吉翻撓奴教翻騎奇寄翻撲弼角翻】投於殿庭石上折頸而死楚客大呼稱快【折而設翻呼火故翻】上雖不窮問意頗怏怏不悦【怏於兩翻】由是韋后及其黨始憂懼【為韋后弑逆張本】 己卯上宴近臣國子祭酒祝欽明自請作八風舞揺頭轉目備諸醜態【祝欽明所謂八風舞非春秋魯大夫衆仲所謂舞者所以節八音行八風者也借八風之名而備諸淫醜之態耳今人謂淫放不返為風此則欽明所謂八風也】上笑欽明素以儒學著名吏部侍郎盧藏用私謂諸學士曰祝公五經掃地盡矣【諸學士者修文舘學士及直學士也】 散騎常侍馬秦客以醫術光祿少卿楊均以善烹調皆出入宫掖得幸於韋后恐事泄被誅【散悉亶翻騎奇寄翻被皮義翻】安樂公主欲韋后臨朝自為皇太女乃相與合謀於餅餤中進毒六月壬午中宗崩於神龍殿【年五十五神龍殿以年號名自兩儀殿東入神龍門至神龍殿六典兩儀殿之北曰甘露門其内甘露殿左曰神龍門其内則神龍殿樂音洛朝直遥翻餤弋廉翻又徒甘翻】韋后祕不發喪自總庶政癸未召諸宰相入禁中徵諸府兵五萬人屯京城使駙馬都尉韋捷韋灌【韋捷尚中宗女成安公主韋灌尚定安公主】衛尉卿韋璿左千牛中郎將韋錡長安令韋播郎將高嵩分領之【璿似宣翻錡渠宜翻 考異曰景龍文舘記徵諸兵士二千人屯皇城左右衛令韋捷韋濯押當又令韋錡押翊林軍韋播高嵩分押左右營萬騎韋元廵六街實録兵五萬人韋濯作韋灌今從之】璿温之族弟播從子嵩其甥也【從才用翻下同】中書舍人韋元徼廵六街【長安城中左右六街金吾街使主之左右金吾將軍掌晝夜廵警之法以執禦非違徼吉弔翻】又命左監門大將軍兼内侍薛思簡等將兵五百人馳驛戍均州以備譙王重福【等將即亮翻重直龍翻下同】以刑部尚書裴談工部尚書張錫並同中書門下三品仍充東都留守【守式又翻】吏部尚書張嘉福中書侍郎岑羲吏部侍郎崔湜並同平章事羲長倩之從子也太平公主與上官昭容謀草遺制立温王重茂為皇太子皇后知政事相王旦參謀政事宗楚客密謂韋温曰相王輔政於理非宜且於皇后嫂叔不通問【引記曲禮之言相息亮翻】聽朝之際何以為禮遂帥諸宰相表請皇后臨朝罷相王政事【朝直遥翻帥讀曰率】蘇瓌曰遺詔豈可改邪温楚客怒瓌懼而從之乃以相王為太子太師甲申梓宫遷御太極殿【西内正殿曰太極殿】集百官發喪皇后臨朝攝政赦天下改元唐隆進相王旦太尉雍王守禮為豳王【雍於用翻】夀春王成器為宋王以從人望命韋温總知内外守捉兵馬事丁亥殤帝即位時年十六尊皇后為皇太后立妃陸氏為皇后壬辰命紀處訥持節廵撫關内道岑羲河南道張嘉福河北道宗楚客與太常卿武延秀司農卿趙履温國子祭酒葉静能及諸韋共勸韋后遵武后故事【欲遵武后易姓事也】南北衛軍【南軍十六衛軍北軍羽林及萬騎也】臺閣要司【臺閣尚書諸司也】皆以韋氏子弟領之廣聚黨衆中外連結楚客又密上書稱引圖讖謂韋氏宜革唐命【識楚譛翻 考異曰舊傳安樂府曹符鳳說武延秀曰天下之心未忘武氏䜟云黑衣神孫被天裳公神皇之孫也大周之業可以興勸延秀裳衣皂袍以應之中宗實錄云宗楚客與弟將作大匠晉卿太常少卿李■〈忄曳〉將作少監李守貞日夜潜圖令延秀速起事太上實録云楚客神龍初為太僕卿與武三思潜謀簒逆累遷同三品及三思誅附安樂而韋氏尤信任之楚客嘗謂所親曰始吾在卑位尤愛宰相及居之又思太極南面一日足矣雖附韋氏志窺宸極此所謂天下之惡皆歸焉者也今所不取】謀害殤帝深忌相王及太平公主密與韋温安樂公主謀去之【去羌呂翻】相王子臨淄王隆基先罷潞州别駕【唐制上州别駕從四品下中州正五品下下州從五品上】在京師隂聚才勇之士謀匡復社稷初太宗選官戶及蕃口驍勇者著虎文衣跨豹文韀【驍堅堯翻著則畧翻韀則前翻馬被具也】從遊獵於馬前射禽獸謂之百騎【射而亦翻騎奇寄翻下同】則天時稍增為千騎隸左右羽林中宗謂之萬騎置使以領之【使疏吏翻】隆基皆厚結其豪傑兵部侍郎崔日用素附韋武與宗楚客善知楚客謀恐禍及己遣寶昌寺僧普潤密詣隆基告之勸其速發隆基乃與太平公主及公主子衛尉卿薛崇暕【暕古限翻】苑總監灨人鍾紹京【鍾紹京西京苑總監也唐京都苑各冇總監一人從五品下掌宫苑内舘園池之事凡禽魚果木皆總而司之贑縣漢屬豫章郡吴晉屬廬陵郡宋以下為南康郡治所唐帶䖍州贑師古古暗翻劉晌古濫翻】尚衣奉御王崇曄前朝邑尉劉幽求【朝直遥翻】利仁府折衝麻嗣宗【唐雍州有府百三十一其逸者百二十利仁府必屬雍州】謀先事誅之韋播高嵩數榜捶萬騎欲以立威【先悉薦翻數所角翻榜音彭捶止橤翻】萬騎皆怨果毅葛福順陳玄禮見隆基訴之隆基諷以誅諸韋皆踴躍請以死自效萬騎果毅李仙鳬亦預其謀或謂隆基當啓相王隆基曰我曹為此以狥社稷事成福歸於王不成以身死之不以累王也【累力瑞翻】今啟而見從則王預危事不從將敗大計遂不啟【史言隆基有大畧所以能平内難敗補邁翻】庚子晡時隆基微服與幽求等入苑中【唐禁苑在皇城之北苑城東西二十七里南北三十里東抵覇水西連故長安城南連京城比枕渭水苑内離宫亭觀二十四所漢長安故城東西十二里皆隸入苑中】會鍾紹京廨舍【廨古隘翻】紹京悔欲拒之其妻許氏曰忘身狥國神必助之且同謀素定今雖不行庸得免乎紹京乃趨出拜謁隆基執其手與坐【紹京趨出拜謁者示尊奉隆基也隆基執手與坐示不敢當且以結其心也】時羽林將士皆屯玄武門逮夜葛福順李仙鳬皆至隆基所請號而行【凡用兵下營及攻襲就主帥取號以備緩急相照應】向二鼓天星散落如雪劉幽求曰天意如此時不可失福順拔劔直入羽林營斬韋璿韋播高嵩以狥曰韋后酖殺先帝謀危社稷今夕當共誅諸韋馬鞭以上皆斬之【言諸韋男女長及馬鞭以上者皆斬】立相王以安天下敢有懷兩端助逆黨者罪及三族羽林之士皆欣然聽命乃送璿等首於隆基隆基取火視之遂與幽求等出苑南門【禁苑南門直宫城之玄武門】紹京帥丁匠二百餘人執斧鋸以從【帥讀曰率下同從才用翻】使福順將左萬騎攻玄德門仙鳬將右萬騎攻白獸門【白獸門即白獸闥即杜甫北征詩所謂寂寞白獸闥者是也與玄德門皆通内諸門之數將即亮翻下同】約會於凌烟閣前即大譟【譟蘇到翻】福順等共殺守門將斬關而入隆基勒兵玄武門外三鼓聞譟聲帥總監及羽林兵而入諸衛兵在太極殿宿衛梓宫者【此南牙諸衛兵也】聞譟聲皆被甲應之【被皮義翻】韋后惶惑走入飛騎營有飛騎斬其首獻於隆基安樂公主方照鏡畫眉軍士斬之斬武延秀於肅章門外斬内將軍賀婁氏於太極殿西【時韋氏以婦人為内將軍盖即賀婁尚宫為之也】初上官昭容引其從母之子王昱為左拾遺【母之姊妹謂之從母從才用翻】昱說昭容母鄭氏曰【說輸苪翻】武氏天之所廢不可興也今媫妤附於三思此滅族之道也願姨思之鄭氏以戒昭容昭容弗聽及太子重俊起兵討三思索昭容【事見上卷景龍元年索山客翻下同】昭容始懼思昱言自是心附帝室與安樂公主各樹朋黨及中宗崩昭容草遺制立温王以相王輔政宗韋改之及隆基入宫昭容執燭帥宫人迎之以制草示劉幽求幽求為之言【為于偽翻】隆基不許斬於旗下時少帝在太極殿【少詩照翻】劉幽求曰衆約今夕共立相王何不早定隆基遽止之捕索諸韋在宫中及守諸門并素為韋后所親信者皆斬之比暁内外皆定辛巳隆基出見相王【比必利翻見賢遍翻】叩頭謝不先啟之罪相王抱之泣曰社稷宗廟不墜於地汝之力也遂迎相王入輔少帝閉宫門及京城門分遣萬騎收捕諸韋親黨斬太子少保同中書門下三品韋温於東市之北中書令宗楚客衣斬衰乘青驢逃出至通化門【衣於既翻衰倉回翻通化門京城東面北來第一門】門者曰公宗尚書也去布㡌執而斬之并斬其弟晉卿 【考異曰太上實録云斬楚客於春明門外今從僉載太上録殺晉卿於定陵按定陵中宗陵也於時未有今不取去羌呂翻】相王奉少帝御安福門慰諭百姓【唐六典曰皇城西面二門北曰安福南曰順義安福門西直開遠門】初趙履温傾國資以奉安樂公主為之起第舍築臺穿池無休已擫紫衫以項挽公主犢車【為于偽翻擫益涉翻】公主死履温馳詣安福樓下舞蹈稱萬歲聲未絶相王令萬騎斬之百姓怨其勞役爭割其肉立盡祕書監汴王邕娶韋后妹崇國夫人【崇古國翻】與御史大夫竇從一各手斬其妻首以獻邕鳳之孫也【鳳高祖之子】左僕射同中書門下三品韋巨源聞亂家人勸之逃匿巨源曰吾位大臣豈可聞難不赴【射寅謝翻難乃旦翻】出至都街為亂兵所殺時年八十於是梟馬秦客楊均葉静能等首尸韋后於市崔日用將兵誅諸韋於杜曲【唐京城南韋杜二族居之謂之韋曲杜曲語云城南韋杜去天尺五時諸韋門宗彊盛侵杜曲而居之梟堅堯翻將知亮翻又音如字】襁褓兒無免者【襁居兩翻褓音保】諸杜濫死非一是日赦天下云逆賊魁首已誅自餘支黨一無所問以臨淄王隆基為平王兼知内外閑廏【平王固以平州為國名實以平内難褒以此名六典尚乘奉御掌内外閑廐之馬一曰左右飛黄閑二曰左右吉良閑三曰左右龍媒閑四曰左右騶駼閑五日左右駃騠閑六曰左右天苑閑開元時仗内六閑曰飛龍翔麟鳳苑鵷鸞吉良六羣等六廐奔星内駒等兩閑仗外有左飛右飛左萬右萬等四閑東南内西南内等兩廐】押左右廂萬騎【左右廂即前所謂左萬騎右萬騎也】薛崇暕賜爵立節王以鍾紹京守中書侍郎劉幽求守中書舍人並參知機務麻嗣宗行右金吾衛中郎將武氏宗屬誅死流竄殆盡【武氏宗屬至是時誅竄宜盡矣而史曰殆盡者攸緒平一能避權遠勢而武惠妃者猶足以成殺三子之祸也】侍中紀處訥行至華州吏部尚書同平章事張嘉福行至懷州皆收斬之【舊志華州京師東百八十里懷州京師東九百六十九里華戶化翻】壬寅劉幽求在太極殿有宫人與宦官令幽求作制書立太后幽求曰國有大難【難乃旦翻】人情不安山陵未畢遽立太后不可平王隆基曰此勿輕言遣十道使齎璽書宣撫及詣均州宣慰譙王重福【使疏吏翻璽斯氏翻重直龍翻】貶竇從一為濠州司馬【舊志濠州京師東南二千一百五十里】罷諸公主府官【中宗時太平安樂等七公主皆開府置官屬】癸卯太平公主傳少帝命請讓位於相王相王固辭以平王隆基為殿中監同中書門下三品以宋王成器為左衛大將軍衡陽王成義為右衛大將軍巴陵王隆範為左羽林大將軍彭城王隆業為右羽林大將軍光祿少卿嗣道王微檢校右金吾衛大將軍微元慶之孫也【道王元慶高祖之子】以黄門侍郎李日知中書侍郎鍾紹京並同中書門下三品太平公主之子薛崇訓為右千牛衛將軍隆基有二奴王毛仲李守德皆趫勇善騎射【趫巨嬌翻善走也】常侍衛左右隆基之入苑中也毛仲避匿不從【從才用翻】事定數日方歸隆基不之責仍超拜將軍毛仲本高麗也【為王毛仲貴寵致祸張本麗力知翻】汴王邕貶沁州刺史【舊志沁州京師東北一千二十五里沁七鴆翻】左散騎常侍駙馬都尉楊慎交貶巴州刺史中書令蕭至忠貶許州刺史【舊志許州京師東一千二百里】兵部尚書同中書門下三品韋嗣立貶宋州刺史中書侍郎同平章事趙彦昭貶絳州刺史吏部侍郎同平章事崔湜貶華州刺史劉幽求言於宋王成器平王隆基曰相王疇昔已居宸極羣望所屬【嗣聖元年則天廢中宗而立相王及革命以王為皇嗣屬之欲翻】今人心未安家國事重相王豈得尚守小節不早即位以鎮天下乎隆基曰王性恬淡不以代事嬰懷【代事即世事避太宗諱云爾】雖有天下猶讓於人【謂既讓武后又讓中宗也】况親兄之子安肯代之乎幽求曰衆心不可違王雖欲高居獨善其如社稷何成器隆基入見相王【見賢遍翻】極言其事相王乃許之甲辰少帝在太極殿東隅西向相王立於梓宫旁太平公主曰皇帝欲以此位讓叔父可乎幽求跪曰國家多難皇帝仁孝追蹤堯舜誠合至公相王代之任重慈愛尤厚矣【難乃旦翻任音壬】乃以少帝制傳位相王時少帝猶在御座太平公主進曰天下之心已歸相王此非兒座遂提下之【下遐嫁翻】睿宗即位御承天門赦天下【京城西内正門曰承天門】復以少帝為温王以鍾紹京為中書令鍾紹京少為司農録事【唐九寺皆有録事官九品盖流外也少詩照翻】既典朝政【朝直遥翻下同】縱情賞罰衆皆惡之【惡烏路翻】太常少卿薛稷勸其上表禮讓【上時掌翻】紹京從之稷入言於上曰紹京雖有勲勞素無才德出自胥徒一旦超居元宰恐失聖朝具聸之美【詩云赫赫師尹民具爾聸】上以為然丙午改除戶部尚書尋出為蜀州刺史【舊志蜀州去京師三千三百三十二里】 上將立太子以宋王成器嫡長而平王隆基有大功疑不能决成器辭曰國家安則先嫡長國家危則先有功苟違其宜【長知兩翻先悉薦翻】四海失望臣死不敢居平王之上涕泣固請者累日大臣亦多言平王功大宜立劉幽求曰臣聞除天下之禍者當享天下之福平王拯社稷之危救君親之難【難乃旦翻】論功莫大語德最賢無可疑者上從之丁未立平王隆基為太子 【考異曰劉子玄先撰太上皇實録盡傳位後又撰睿宗實録終橋陵文字頗不同睿宗録及舊紀皆云丙午立太子今從太上皇録】隆基復表讓成器不許【復扶又翻】 則天大聖皇后復舊號為天后追謚雍王賢曰章懷太子【賢廢見二百二卷高宗永隆元年雍於用翻下同】 戊申以宋王成器為雍州牧揚州大都督太子太師 置温王重茂於内宅【恐群不逞挾之以為變也】 以太常少卿薛稷為黄門侍郎參知機務稷以工書事上於藩邸其子伯陽尚仙源公主【仙源公主帝女也後封荆山公主】故為相 追削武三思武崇訓爵謚斵棺暴尸平其墳墓 以許州刺史姚元之為兵部尚書同中書門下三品宋州刺史韋嗣立許州刺史蕭至忠為中書令絳州刺史趙彦昭為中書侍郎華州刺史崔湜為吏部侍郎並同平章事越州長史宋之問饒州刺史冉祖雍坐諂附韋武皆流嶺表 己酉立衡陽王成義為申王巴陵王隆範為岐王彭城王隆業為薛王加太平公主實封滿萬戶太平公主沈敏多權畧【沈持林翻】武后以為類己故於諸子中獨愛幸頗得預密謀然尚畏武后之嚴未敢招權埶及誅張易之公主有力焉【誅張易之見二百七卷中宗神龍元年】中宗之世韋后安樂公主皆畏之又與太子共誅韋氏旣屢立大功益尊重上常與之圖議大政每入奏事坐語移時或時不朝謁【朝直遥翻】則宰相就第咨之每宰相奏事上輒問嘗與太平議否又問與三郎議否然後可之三郎謂太子也公主所欲上無不聽自宰相以下進退繫其一言其餘薦士驟歷清顕者不可勝數權傾人主趨附其門者如市【勝音升趨七喻翻】子薛崇行崇敏崇簡皆封王田園遍於近甸收市營造諸器玩遠至嶺蜀輸送者相屬於路【屬之欲翻】居處奉養擬於宫掖【處昌呂翻】 追贈郎岌燕欽融諫議大夫 秋七月庚戍朔贈韋月將宣州刺史【韋月將死見上卷中宗神龍二年】 癸丑以兵部侍郎崔日用為黄門侍郎參知機務 追復故太子重俊位號【太子重俊死見上卷中宗景龍元年】雪敬暉桓彦範崔玄暐張柬之袁恕己成王千里李多祚等罪復其官爵【五王事見上卷神龍二年千里多祚與重俊同死見景龍元年】 丁巳以洛州長史宋璟檢校吏部尚書同中書門下三品岑羲罷為右散騎常侍兼刑部尚書璟與姚元之恊心革中宗弊政進忠良退不肖賞罰盡公請託不行綱紀修舉當時翕然以為復有貞觀永徽之風【復扶又翻又如字】 壬戌崔湜罷為尚書左丞張鍚為絳州刺史蕭至忠為晉州刺史【舊志晉州京師東比七百二十五里】韋嗣立為許州刺史趙彦昭為宋州刺史丙寅姚元之兼中書令兵部尚書同中書門下三品李嶠貶懷州刺史丁卯太子少師同中書門下三品唐休璟致仕右武衛大將軍同中書門下三品張仁愿罷為左衛大將軍 黄門侍郎參知機務崔日用與中書侍郎參知機務薛稷爭於上前稷曰日用傾側曏附武三思非忠臣賣友邀功非義士日用曰臣往雖有過今立大功【立大功謂誅韋氏之謀日用發之】稷外託國婣【謂稷子伯陽尚主】内附張易之宗楚客非傾側而何上由是兩罷之戊辰以日用為雍州長史稷為左散騎常侍 己巳赦天下改元【改元景雲】凡韋氏餘黨未施行者咸赦之 乙亥廢武氏崇恩廟及昊陵順陵【中宗景龍元年復武氏陵廟】追廢韋后為庶人安樂公主為悖逆庶人【悖蒲内翻又蒲沒翻】 韋后之臨朝也吏部侍郎鄭愔貶江州司馬【朝直遥翻愔於今翻】潜過均州與刺史譙王重福及洛陽人張靈均謀舉兵誅韋氏未發而韋氏敗重福遷集州刺史未行靈均說重福曰大王地居嫡長當為天子【長知兩翻】相王雖有功不當繼統東都士庶皆願王來若潜入洛陽發左右屯營兵【東都置左右屯營兵以衛宫城】襲殺留守據東都如從天而下也然後西取陜州東取河南北天下指麾而定【守式又翻陜式冉翻】重福從之靈均乃密與愔結謀聚徒數十人時愔自祕書少監左遷沅州刺史【武后天授二年改巫州為沅州舊志沅州京師南四千一百九十七里至東都三千九百里】遲留洛陽以俟重福草制立重福為帝改元為中元克復 【考異曰太上皇實録云改元為中宗克復元年今從新書】尊上為皇季叔以温王為皇太弟愔為左丞相知内外文事靈均為右丞相天柱大將軍知武事右散騎常侍嚴善思為禮部尚書知吏部事重福與靈均詐乘驛詣東都愔先供張駙馬都尉裴巽第以待重福【供居用翻張知亮翻】洛陽縣官微聞其謀<br />
<br />
  資治通鑑卷二百九  <br>
   </div> 

<script src="/search/ajaxskft.js"> </script>
 <div class="clear"></div>
<br>
<br>
 <!-- a.d-->

 <!--
<div class="info_share">
</div> 
-->
 <!--info_share--></div>   <!-- end info_content-->
  </div> <!-- end l-->

<div class="r">   <!--r-->



<div class="sidebar"  style="margin-bottom:2px;">

 
<div class="sidebar_title">工具类大全</div>
<div class="sidebar_info">
<strong><a href="http://www.guoxuedashi.com/lsditu/" target="_blank">历史地图</a></strong>  
<a href="http://www.880114.com/" target="_blank">英语宝典</a>  
<a href="http://www.guoxuedashi.com/13jing/" target="_blank">十三经检索</a> 
<br><strong><a href="http://www.guoxuedashi.com/gjtsjc/" target="_blank">古今图书集成</a></strong> 
<a href="http://www.guoxuedashi.com/duilian/" target="_blank">对联大全</a> <strong><a href="http://www.guoxuedashi.com/xiangxingzi/" target="_blank">象形文字典</a></strong> 

<br><a href="http://www.guoxuedashi.com/zixing/yanbian/">字形演变</a>  <strong><a href="http://www.guoxuemi.com/hafo/" target="_blank">哈佛燕京中文善本特藏</a></strong>
<br><strong><a href="http://www.guoxuedashi.com/csfz/" target="_blank">丛书&方志检索器</a></strong> <a href="http://www.guoxuedashi.com/yqjyy/" target="_blank">一切经音义</a>  

<br><strong><a href="http://www.guoxuedashi.com/jiapu/" target="_blank">家谱族谱查询</a></strong>  <strong><a href="http://shufa.guoxuedashi.com/sfzitie/" target="_blank">书法字帖欣赏</a></strong> 
<br>

</div>
</div>


<div class="sidebar" style="margin-bottom:0px;">

<font style="font-size:22px;line-height:32px">QQ交流群9:489193090</font>


<div class="sidebar_title">手机APP 扫描或点击</div>
<div class="sidebar_info">
<table>
<tr>
	<td width=160><a href="http://m.guoxuedashi.com/app/" target="_blank"><img src="/img/gxds-sj.png" width="140"  border="0" alt="国学大师手机版"></a></td>
	<td>
<a href="http://www.guoxuedashi.com/download/" target="_blank">app软件下载专区</a><br>
<a href="http://www.guoxuedashi.com/download/gxds.php" target="_blank">《国学大师》下载</a><br>
<a href="http://www.guoxuedashi.com/download/kxzd.php" target="_blank">《汉字宝典》下载</a><br>
<a href="http://www.guoxuedashi.com/download/scqbd.php" target="_blank">《诗词曲宝典》下载</a><br>
<a href="http://www.guoxuedashi.com/SiKuQuanShu/skqs.php" target="_blank">《四库全书》下载</a><br>
</td>
</tr>
</table>

</div>
</div>


<div class="sidebar2">
<center>


</center>
</div>

<div class="sidebar"  style="margin-bottom:2px;">
<div class="sidebar_title">网站使用教程</div>
<div class="sidebar_info">
<a href="http://www.guoxuedashi.com/help/gjsearch.php" target="_blank">如何在国学大师网下载古籍?</a><br>
<a href="http://www.guoxuedashi.com/zidian/bujian/bjjc.php" target="_blank">如何使用部件查字法快速查字?</a><br>
<a href="http://www.guoxuedashi.com/search/sjc.php" target="_blank">如何在指定的书籍中全文检索?</a><br>
<a href="http://www.guoxuedashi.com/search/skjc.php" target="_blank">如何找到一句话在《四库全书》哪一页?</a><br>
</div>
</div>


<div class="sidebar">
<div class="sidebar_title">热门书籍</div>
<div class="sidebar_info">
<a href="/so.php?sokey=%E8%B5%84%E6%B2%BB%E9%80%9A%E9%89%B4&kt=1">资治通鉴</a> <a href="/24shi/"><strong>二十四史</strong></a>&nbsp; <a href="/a2694/">野史</a>&nbsp; <a href="/SiKuQuanShu/"><strong>四库全书</strong></a>&nbsp;<a href="http://www.guoxuedashi.com/SiKuQuanShu/fanti/">繁体</a>
<br><a href="/so.php?sokey=%E7%BA%A2%E6%A5%BC%E6%A2%A6&kt=1">红楼梦</a> <a href="/a/1858x/">三国演义</a> <a href="/a/1038k/">水浒传</a> <a href="/a/1046t/">西游记</a> <a href="/a/1914o/">封神演义</a>
<br>
<a href="http://www.guoxuedashi.com/so.php?sokeygx=%E4%B8%87%E6%9C%89%E6%96%87%E5%BA%93&submit=&kt=1">万有文库</a> <a href="/a/780t/">古文观止</a> <a href="/a/1024l/">文心雕龙</a> <a href="/a/1704n/">全唐诗</a> <a href="/a/1705h/">全宋词</a>
<br><a href="http://www.guoxuedashi.com/so.php?sokeygx=%E7%99%BE%E8%A1%B2%E6%9C%AC%E4%BA%8C%E5%8D%81%E5%9B%9B%E5%8F%B2&submit=&kt=1"><strong>百衲本二十四史</strong></a>  <a href="http://www.guoxuedashi.com/so.php?sokeygx=%E5%8F%A4%E4%BB%8A%E5%9B%BE%E4%B9%A6%E9%9B%86%E6%88%90&submit=&kt=1"><strong>古今图书集成</strong></a>
<br>

<a href="http://www.guoxuedashi.com/so.php?sokeygx=%E4%B8%9B%E4%B9%A6%E9%9B%86%E6%88%90&submit=&kt=1">丛书集成</a> 
<a href="http://www.guoxuedashi.com/so.php?sokeygx=%E5%9B%9B%E9%83%A8%E4%B8%9B%E5%88%8A&submit=&kt=1"><strong>四部丛刊</strong></a>  
<a href="http://www.guoxuedashi.com/so.php?sokeygx=%E8%AF%B4%E6%96%87%E8%A7%A3%E5%AD%97&submit=&kt=1">說文解字</a> <a href="http://www.guoxuedashi.com/so.php?sokeygx=%E5%85%A8%E4%B8%8A%E5%8F%A4&submit=&kt=1">三国六朝文</a>
<br><a href="http://www.guoxuedashi.com/so.php?sokeytm=%E6%97%A5%E6%9C%AC%E5%86%85%E9%98%81%E6%96%87%E5%BA%93&submit=&kt=1"><strong>日本内阁文库</strong></a> <a href="http://www.guoxuedashi.com/so.php?sokeytm=%E5%9B%BD%E5%9B%BE%E6%96%B9%E5%BF%97%E5%90%88%E9%9B%86&ka=100&submit=">国图方志合集</a> <a href="http://www.guoxuedashi.com/so.php?sokeytm=%E5%90%84%E5%9C%B0%E6%96%B9%E5%BF%97&submit=&kt=1"><strong>各地方志</strong></a>

</div>
</div>


<div class="sidebar2">
<center>

</center>
</div>
<div class="sidebar greenbar">
<div class="sidebar_title green">四库全书</div>
<div class="sidebar_info">

《四库全书》是中国古代最大的丛书,编撰于乾隆年间,由纪昀等360多位高官、学者编撰,3800多人抄写,费时十三年编成。丛书分经、史、子、集四部,故名四库。共有3500多种书,7.9万卷,3.6万册,约8亿字,基本上囊括了古代所有图书,故称“全书”。<a href="http://www.guoxuedashi.com/SiKuQuanShu/">详细>>
</a>

</div> 
</div>

</div>  <!--end r-->

</div>
<!-- 内容区END --> 

<!-- 页脚开始 -->
<div class="shh">

</div>

<div class="w1180" style="margin-top:8px;">
<center><script src="http://www.guoxuedashi.com/img/plus.php?id=3"></script></center>
</div>
<div class="w1180 foot">
<a href="/b/thanks.php">特别致谢</a> | <a href="javascript:window.external.AddFavorite(document.location.href,document.title);">收藏本站</a> | <a href="#">欢迎投稿</a> | <a href="http://www.guoxuedashi.com/forum/">意见建议</a> | <a href="http://www.guoxuemi.com/">国学迷</a> | <a href="http://www.shuowen.net/">说文网</a><script language="javascript" type="text/javascript" src="https://js.users.51.la/17753172.js"></script><br />
  Copyright &copy; 国学大师 古典图书集成 All Rights Reserved.<br>
  
  <span style="font-size:14px">免责声明:本站非营利性站点,以方便网友为主,仅供学习研究。<br>内容由热心网友提供和网上收集,不保留版权。若侵犯了您的权益,来信即刪。scp168@qq.com</span>
  <br />
ICP证:<a href="http://www.beian.miit.gov.cn/" target="_blank">鲁ICP备19060063号</a></div>
<!-- 页脚END --> 
<script src="http://www.guoxuedashi.com/img/plus.php?id=22"></script>
<script src="http://www.guoxuedashi.com/img/tongji.js"></script>

</body>
</html>
