






























































資治通鑑卷一百四十二 宋 司馬光 撰

胡三省 音註

齊紀八【屠維單閼一年}


東昏侯上【諱寶卷字智藏明帝第二子也本名明賢明帝輔政後改焉明帝長子寶義有廢疾故立帝為太子其後蕭衍蕭頴胄以荆雍起兵輔南康王寶融以攻帝廢帝為東昏侯荆雍在西謂帝以昏虐居東故廢為東昏侯}


永元元年春正月戊寅朔大赦改元 太尉陳顯達督平北將軍崔慧景軍四萬擊魏欲復雍州諸郡【去年魏克雍州五郡雍於用翻}
癸未魏遣前將軍元英拒之【元英即拓跋英魏既改姓元氏史因而書之}
乙酉魏主鄴【去年十二月甲寅魏主自鄴班師今車駕始自鄴}
辛卯帝祀南郊 戊戌魏主至洛陽過李冲冢【冲死見上卷上年魏主令葬冲於洛陽覆舟山近杜預冢今自鄴還過其冢按魏主詔代人遷洛者葬洛餘州從便冲隴西人也以其貴寵亦令葬洛}
時臥疾望之而泣見留守官語及冲輒流涕【李冲與任城王澄等同守留臺魏主還洛見留守官而冲已死故語及輒流涕念之之甚也守式又翻}
魏主謂任城王澄曰朕離京以來舊俗少變不【任音壬離力智翻少詩沼翻下同不讀曰否}
對曰聖化日新帝曰朕入城見車上婦人猶戴㡌著小襖【此代北婦人之服也乘車婦人皆貴臣之家也著陟略翻襖烏浩翻裌衣也}
何謂日新對曰著者少不著者多帝曰任城此何言也必欲使滿城盡著邪澄與留守官皆免冠謝【史言魏主汲汲於用夏變夷}
甲辰魏大赦魏主之幸鄴也李彪迎拜於鄴南且謝罪【彪既得罪歸鄉里故迎魏主於鄴南}
帝曰朕欲用卿思李僕射而止慰而遣之會御史臺令史龍文觀告太子恂被收之日【恂被收見一百四十卷明帝建武三年被皮義翻}
有手書自理彪不以聞尚書表收彪赴洛陽帝以為彪必不然以牛車散載詣洛陽【散悉但翻散載者不加縶縳}
會赦得免 魏太保齊郡靈王簡卒二月辛亥魏以咸陽王禧為太尉 魏主連年在外【魏主自明帝建武元年南伐至是首尾四年}
馮后私於宦者高菩薩【菩蓬晡翻薩桑葛翻}
及帝在懸瓠病篤【事見上卷上年}
后益肆意無所憚中常侍雙蒙等為之心腹【雙姓蒙名姓譜顓帝後封於雙蒙城其後以為氏}
彭城公主為宋王劉昶子婦寡居【昶丑兩翻}
后為其母弟北平公馮夙求昏帝許之公主不願后強之【后為于偽翻強其兩翻}
公主密與家僮冒雨詣懸瓠訴於帝且具道后所為帝疑而秘之后聞之始懼隂與母常氏使女巫猒禱【猒於葉翻又於琰翻}
曰帝疾若不起一旦得如文明太后輔少主稱制者【文明太后后之姑也其包藏祸心若此豈非姑之教也少詩照翻}
當賞報不貲【貲即移翻貲之為言量也不貲言無量之可比也}
帝還洛收高菩薩雙蒙等案問具伏帝在含温室夜引后入賜坐東楹去御榻二丈餘命菩薩等陳狀【陳后淫佚之狀}
既而召彭城王勰北海王詳入坐【勰音恊}
曰昔為汝嫂今是路人但入勿避又曰此嫗欲手刃吾脅【嫗威遇翻老婦曰嫗}
吾以文明太后家女不能廢但虚置宫中有心庶能自死【言若有人心必當自取盡也}
汝等勿謂吾猶有情也二王出賜后辭訣后再拜稽首涕泣【稽音啟}
入居後宫諸嬪御奉之猶如后禮【嬪毗賓翻}
唯命太子不復朝謁而已【太子儲君也命不復朝謁絶之不使以母禮事之復扶又翻朝直遥翻}
初馮熙以文明太后之兄尚恭宗女博陵長公主【景穆太子廟號恭宗長知兩翻}
熙有三女二為皇后一為左昭儀【二后廢后及幽后也昭儀早卒瑶光寺之練行尼魏主為之廢后非得罪於宗廟也幽后所為彰灼如此乃不能正其罪廢后獨非文明家女□}
由是馮氏貴寵冠羣臣賞賜累巨萬【漢書音義曰巨萬萬萬也冠古玩翻}
公主生二子誕脩熙為太保誕為司徒脩為侍中尚書庶子聿為黄門郎黄門侍郎崔光與聿同直【庶子妾御所生以此觀之魏以黄門郎與黄門侍郎為兩官同直同直禁中也}
謂聿曰君家富貴太盛終必衰敗聿曰我家何所負而君無故詛我【詛莊助翻呪也}
光曰不然物盛必衰此天地之常理若以古事推之不可不慎後歲餘而脩敗脩性浮競誕屢戒之不悛【悛丑緣翻}
乃白於太后及帝而杖之脩由是恨誕求藥使誕左右毒之事覺帝欲誅之誕自引咎懇乞其生帝亦以其父老杖脩百餘黜為平城民及誕熙繼卒【太和十九年馮誕卒是年二月也四月馮熙又卒}
幽后尋廢【太和二十年幽后廢}
聿亦擯棄馮氏遂衰【史言外戚罕有能全保其福禄者}
魏以彭城王勰為司徒【勰音協}
陳顯逹與魏元英戰屢破之攻馬圈城四十日【按陳顯逹傳馬圈在南郷界杜佑曰馬圈城去襄陽三百里在今南陽郡界穰縣北杜佑曰後魏馬圈鎮漢涅陽縣地圈渠篆翻}
城中食盡噉死人肉及樹皮【噉徒濫翻又徒覧翻}
癸酉魏人突圍走斬獲千計顯逹入城將士競取城中絹遂不窮追【史言齊師貪鹵掠以縱敵將即亮翻}
顯逹又遣軍主莊丘黑進擊南鄉拔之【莊丘複姓也蕭子顯曰南鄉城順陽舊治也}
魏主謂任城王澄曰顯逹侵擾朕不親行無以制之【任音王}
三月庚辰魏主發洛陽命于烈居守【守音狩凡留守大守之守皆同}
以右衛將軍宋弁兼祠部尚書攝七兵事以佐之【攝七兵事者攝尚書七兵曹事也杜佑曰魏始置五兵尚書謂中兵外兵别兵都兵騎兵也晉又分中外兵各為左右後魏遂為七兵尚書}
弁精勤吏治【治直吏翻}
恩遇亞於李冲癸未魏主至梁城【魏收志北荆州汝北郡有梁縣汝源縣五代志襄城郡承休縣舊曰汝源置汝北郡唐志汝州臨汝郡本襄城郡治梁縣又有梁縣故城在西南四十五里}
崔慧景攻魏順陽順陽太守清河張烈固守【五代志鄧州順陽縣舊置順陽郡唐武德六年省順陽入冠軍貞觀元年省冠軍入新城其地在今鄧州菊潭臨湍二縣之間也杜佑曰漢順陽故城在鄧州穰縣西亦後穰縣地}
甲申魏主遣振威將軍慕容平城將騎五千救之【將即亮翻騎奇寄翻}
自魏主有疾彭城王勰常居中侍醫藥晝夜不離左右【離力智翻}
飲食必先嘗而後進蓬首垢面衣不解帶帝久疾多忿近侍失指動欲誅斬勰承顏伺間多所匡救【伺相吏翻間古莧翻下間道同}
丙戌以勰為使持節都督中外諸軍事【使疏吏翻}
勰辭曰臣侍疾無暇安能治軍願更請一王使摠軍要【軍要猶言軍權也左傳曰握兵之要杜預注云威權在己治直之翻}
臣得專心醫藥帝曰侍疾治軍皆憑於汝吾病如此深慮不濟安六軍保社禝者捨汝而誰何容方更請人以違心寄乎【心寄謂推心以託之也}
丁酉魏主至馬圈命荆州刺史廣陽王嘉斷均口【水經曰均水出淅縣北山南流過其縣之東又南當涉都縣邑北南入于沔注云即郡國志筑陽縣之涉都郷均水於此入沔謂之均口斷丁管翻}
邀齊兵歸路嘉建之子也【楚王建見一百二十五卷宋文帝元嘉二十七年}
陳顯逹引兵渡水西【均水之西也}
據鷹子山築城人情沮恐【沮在呂翻}
與魏戰屢敗魏武衛將軍元嵩免胄䧟陳【陳讀曰陣}
將士隨之齊兵大敗嵩澄之弟也戊戌軍主崔恭祖胡松以烏布幔盛顯逹數人擔之【幔莫半翻盛時征翻擔都甘翻負也}
間道自分磧山出均水口南走【磧七迹翻}
己亥魏收顯逹軍資億計班賜將士追奔至漢水而還【還從宣翻又如字下同}
左軍將軍張千戰死 【考異曰魏書作張千逹今從齊書}
士卒死者三萬餘人顯逹之北伐軍入汋均口【水經注順陽縣西有石山南臨汋水汋水又南流注于沔水謂之汋口詳考經及注汋水均水實一水也故謂之汋均口汋實若翻}
廣平馮道根【沈約宋志廣平太守江左僑立治襄陽宋為實土以漢朝陽縣地立廣平郡及廣平縣領鄼隂北陽等縣按水經注朝陽在新野西白水又出其西}
說顯逹曰汋均水迅急易進難退【說輸芮翻易以豉翻}
魏若守隘則首尾俱急不如悉棄船於鄼城陸道步進【鄼縣即漢蕭何所封之邑屬南陽郡晉屬順陽郡江左僑立廣平郡鄼縣屬焉馮道根廣平鄼人也水經沔水自均口東南過鄼縣之西南五代志襄州隂城縣西魏置鄼城郡隘烏懈翻鄼音贊}
列營相次鼓行而前破之必矣顯逹不從道根以私屬從軍【私屬者家之奴客及其親黨非官之所調者}
及顯逹夜走軍人不知山路道根每及險要輒停馬指示之衆賴以全詔以道根為汋均口戍副【凡邊戍有戍主戍副}
顯逹素有威名至是大損【陳顯逹之敗固是弱不可以敵彊亦天為之也齊師潰於戊戌魏主殂於丙午儻顯逹更能支持數日安知不能轉敗為功邪}
御史中丞范【由山}
奏免顯逹官顯逹亦自表解職皆不許更以顯逹為江州刺史 【考異曰齊明帝紀永泰元年七月癸卯以顯逹為江州本傳顯逹敗於馬圈求降號不許乃除江州又云東昏立顯逹彌不樂京師得此授甚喜按明帝末顯逹方以三公將兵擊魏不容無故除江州今從本傳}
崔慧景亦棄順陽走還 庚子魏主疾甚北還至穀塘原謂司徒勰曰後宫久乖隂德【記曰天子理陽道后冶隂德鄭注云隂德謂主隂事隂令也}
吾死之後可賜自盡葬以后禮庶免馮門之醜又曰吾病益惡殆必不起雖摧破顯逹而天下未平嗣子幼弱社禝所倚唯在於汝霍子孟諸葛孔明以異姓受顧託【漢武帝託昭帝於霍光昭烈帝託後主於諸葛亮事並見前}
况汝親賢可不勉之勰泣曰布衣之士猶為知己畢命【古語有之士為知己者死為于偽翻}
况臣託靈先帝依陛下之末光乎【託靈託體皆兄弟同氣之謂也}
但臣以至親久參機要寵靈輝赫海内莫及所以敢受而不辭正恃陛下日月之明恕臣忘退之過耳今復任以元宰【復扶又翻}
摠握機要震主之聲取罪必矣昔周公大聖成王至明猶不免疑而况臣乎如此則陛下愛臣更為未盡始終之美【彭城王勰慮禍避權如此猶終不免於高肇之手况咸陽王禧北海王詳等邪}
帝默然久之曰詳思汝言理實難奪乃手詔太子曰汝叔父勰清規懋賞【懋美也}
與白雲俱潔厭榮捨紱以松竹為心吾少與綢繆【紱音弗少詩照翻鄭康成曰綢繆猶纒綿也綢直留翻繆莫侯翻}
未忍暌離百年之後其聽勰辭蟬捨冕遂其冲挹之性以侍中護軍將軍北海王詳為司空鎮南將軍王肅為尚書令鎮南大將軍廣陽王嘉為左僕射尚書宋弁為吏部尚書與侍中大尉禧尚書右僕射澄等六人輔政夏四月丙午朔殂于穀塘原【年三十三謚孝文皇帝廟號高祖}
高祖友愛諸弟終始無間【間古莧翻}
嘗從容謂咸陽王禧等曰【從千容翻}
我後子孫邂逅不肖【不期而會曰邂逅肖似也不似其先曰不肖邂戶懈翻逅胡豆翻}
汝等觀望可輔則輔之不可輔則取之勿為它人有也【以禧之驕貪如此孝文以此語之是啟其姦心也景明之禍帝實胎之}
親任賢能從善如流精勤庶務朝夕不倦常曰人主患不能處心公平推誠于物【處昌呂翻}
能是二者則胡越之人皆可使如兄弟矣用法雖嚴於大臣無所容貸然人有小過常多闊略嘗于食中得蟲又左右進羮誤傷帝手皆笑而赦之天地五郊宗廟二分之祭【五郊謂迎氣五郊也按鄭康成說古者天子春分朝日秋分夕月故曰二分之祭魏則朝日以朔夕月以朏猶仍古謂之二分之祭}
未嘗不身親其禮每出廵遊及用兵有司奏脩道路帝輒曰粗脩橋梁通車馬而已勿去草剗令平也【粗坐五翻去羌呂翻剗楚限翻}
在淮南行兵如在境内禁士卒無得踐傷粟稻【踐息淺翻}
或伐民樹以供軍用皆留絹償之宫室非不得已不修衣弊浣濯而服之鞌勒用鐵木而已幼多力善射能以指彈碎羊骨【魏紀云能以指彈辟羊髆骨羊骨唯髆骨頗脆他骨未易彈碎也彈徒丹翻}
射禽獸無不命中【先命其處而後射中之謂之命中射而亦翻}
及年十五遂不復畋獵【復扶又翻下同}
帝謂内官曰時事不可以不直書人主威福在已無能制之者若史策復不書其惡將何所畏忌邪【自此以上史言魏孝文德美}
彭城王勰與任城王澄謀以陳顯逹去尚未遠恐其覆相掩逼【覆反也恐凶問外露陳顯逹知之反兵追掩以相逼}
乃秘不發喪徙御卧輿唯二王與左右數人知之勰出入神色無異奉膳進藥可决外奏一如平日數日至宛城【宛於元翻}
夜進卧輿於郡聽事得加棺歛【魏書禮志卧輦飾如乾象輦丹漆駕六馬聽他經翻聽受也中庭曰聽事言受事察訟於是也漢晉皆作聽事六朝以後乃始加广作廳棺古玩翻歛力贍翻}
還載卧輿内外莫有知者遣中書舍人張儒奉詔徵太子密以凶問告留守于烈烈處分行留舉止無變【史言魏孝文之殂執羈絏守社禝者皆能以常處變不動聲色盖其善用人之效也處昌呂翻分扶問翻}
太子至魯陽【魯陽縣漢晉屬南陽郡魏太和十一年置魯陽鎮十八年改為荆州二十二年罷州置魯陽郡唐汝州魯山縣本魯陽縣也}
遇梓宮乃喪丁巳即位【帝諱恪孝文皇帝第二子也}
大赦彭城王勰跪授遺敕數紙東宫官屬多疑勰有異志密防之而勰推誠盡禮卒無間隙【推誠謂推誠於東宮官屬也盡禮謂事嗣君盡禮也卒子恤翻間古莧翻}
咸陽王禧至魯陽留城外以察其變久之乃入【亦疑勰有異志也}
謂勰曰汝此行不唯勤勞亦實危險勰曰兄年長識高故知有夷險【長知兩翻}
彥和握蛇騎虎不覺艱難【勰字彥和蛇螫虎噬握之騎之罕有能免於螫噬者故以為喻}
禧曰汝恨吾後至耳勰等以高祖遺詔賜馮后死北海王詳使長秋卿白整入授后藥【長秋卿皇后宮卿也即漢之大長秋}
后走呼不肯飲【走且呼也呼火故翻}
曰官豈有此是諸王輩殺我耳整執持彊之乃飲藥而卒【彊其兩翻 考異曰元嵩傳曰將遣使者賜馮后死而難其人顧任城王澄曰任城不負我嵩亦當不負任城可使嵩也乃引高平侯嵩入内親詔遣之高祖紀曰詔司徒勰徵太子與喪會魯陽踐祚按馮后傳梓宫至魯陽乃行遺詔賜后死安有高祖遣嵩之事又勰傳高祖崩勰遏秘喪事遣張儒徵世宗亦無高祖詔勰徵太子事}
喪至洛城南咸陽王禧等知后審死相視曰設無遺詔我兄弟亦當决策去之豈可令失行婦人宰制天下殺我輩也【去羌呂翻行下孟翻}
謚曰幽皇后【謚法壅遏不通曰幽}
五月癸亥加撫軍大將軍始安王遥光開府儀同三司 丙申魏葬孝文帝于長陵【長陵在瀍西}
廟號高祖魏世宗欲以彭城王勰為相勰屢陳遺旨請遂素懷帝對之悲慟勰懇請不已乃以勰為使持節侍中都督冀定等七州軍事驃騎大將軍開府儀同三司定州刺史【使疏吏翻驃匹妙翻騎奇寄翻七州冀定相瀛幽平營也}
勰猶固辭帝不許乃之官魏任城王澄以王肅羈旅位加已上【王肅本江南人而奔魏故以為羇旅肅為尚書令而澄為右僕射故以為位加已上}
意頗不平會齊人降者嚴叔懋告肅謀逃還江南【降戶江翻}
澄輒禁止肅【禁止不令入宫省}
表稱謀叛案驗無實咸陽王禧等奏澄擅禁宰輔免官還第尋出為雍州刺史【任城王澄之才畧魏宗室中之巨擘也太和之間朝廷有大議澄每出辭氣加萬乘而軼其上孝文外雖容之内實憚之况咸陽王禧等乎因王肅而斥逐之耳主少國疑之時澄之能全其身者幸也雍於用翻}
六月戊辰魏追尊皇妣高氏為文昭皇后【高氏卒見上卷明帝建武四年}
配饗高祖增修舊冢號終寧陵【據后傳陵在長陵東南}
追賜后父颺爵勃海公謚曰敬【颺余章翻}
以其嫡孫猛襲爵封后兄肇為平原公肇弟顯為澄城公【澄城漢馮翊之徵縣左傳之北徵也魏真君七年置澄城郡}
三人同日受封魏主素未識諸舅始賜衣幘引見【見賢遍翻}
皆惶懼失措數日之間富貴赫奕【赫明也奕盛也為高肇以擅權致禍張本}
秋八月戊申魏用高祖遺詔三夫人以下皆遣還家【魏高祖始定内官左右昭儀位視大司馬三夫人位視三公}
帝自在東宮不好學【好呼到翻}
唯嬉遊無度性重澀少言【澀色入聲}
及即位不與朝士相接【朝直遥翻下同}
專親信宦官及左右御刀應敕等【御刀捉御刀在左右者應敕在左右祗應敕命者應於證翻}
是時揚州刺史始安王遥光尚書令徐孝嗣右僕射江袥【袥音石}
右將軍蕭坦之侍中江祀衛尉劉暄更直内省分日帖敕【内省在禁中以别華林省及下省帖敕者於敕後聯紙書行所謂畫敕也更工衡翻}
雍州刺史蕭衍聞之謂從舅録事參軍范陽張弘策曰【張弘策范陽方城人衍母張氏弘策之從父弟雍於用翻從才用翻}
一國三公猶不堪【左傳晉士蒍曰狐裘蒙茸一國三公吾誰適從}
况六貴同朝勢必相圖亂將作矣避禍圖福無如此州但諸弟在都恐罹世患當更與益州圖之耳【衍兄懿時為益州刺史}
乃密與弘策脩武備它人皆不得預謀招聚驍勇以萬數多伐材竹沈之檀溪【水經注檀溪水出襄陽縣西柳子山下溪去城里餘北流注于沔即劉備乘的盧墮處也驍堅堯翻沈直禁翻又持林翻}
積茅如岡阜【大脊曰岡大陵曰阜}
皆不之用中兵參軍東平呂僧珍覺其意亦私具櫓數百張先是僧珍為羽林監【羽林監漢官監羽林兵先悉薦翻}
徐孝嗣欲引置其府僧珍知孝嗣不能久固求從衍是時衍兄懿罷益州刺史還仍行郢州事衍使弘策說懿曰【說輸芮翻下又自說同}
今六貴比肩人自畫敕爭權睚眦理相圖滅【圖謀也謀相滅也或曰圖當作屠睚五懈翻眦士懈翻}
主上自東宮素無令譽媟近左右慓輕忍虐【媟私列翻近其靳翻慓匹妙翻急疾也輕區竟翻}
安肯委政諸公虛坐主諾【言必不肯付朝政以聽於六貴但擁虛位有可無否惟主作諾而已}
嫌忌積久必大行誅戮始安欲為趙王倫形迹已見【趙王倫事見八十四卷晉惠帝永寧元年見賢遍翻}
然性猜量狹徒為禍階蕭坦之忌克陵人徐孝嗣聽人穿鼻【言如牛然聽人穿鼻而受制於人}
江祏無斷【斷丁管翻}
劉暄闇弱一朝禍中外土崩吾兄弟幸守外藩宜為身計及今猜防未生當悉召諸弟恐異時拔足無路矣【後卒如衍所料史言朝政不綱則姦雄生心}
郢州控帶荆湘【郢州當荆湘下流二州之所赴集也}
雍州士馬精彊世治則竭誠本朝【治直吏翻朝直遥翻}
世亂則足以匡濟與時進退此萬全之策也若不早圖後悔無及弘策又自說懿曰以卿兄弟英武天下無敵據郢雍二州為百姓請命廢昏立明易於反掌【為于偽翻易以豉翻}
此桓文之業也勿為豎子所欺取笑身後雍州揣之已熟【揣初委翻}
願善圖之懿不從衍乃迎其弟驃騎外兵參軍偉及西中郎外兵參軍憺至襄陽【憺徙濫翻}
初高宗雖顧命羣公而多寄腹心在江祏兄弟【顧命見上卷上年江祏江祀兄弟高宗母景皇后之姪也故寄以腹心}
二江更直殿内【更工衡翻更迭也}
動止關之帝稍欲行意徐孝嗣不能奪蕭坦之時有異同而祏執制堅確帝深忿之帝左右會稽茹法珍吳興梅蟲兒等為帝所委任【會工外翻茹音如}
祏常裁折之法珍等切齒徐孝嗣謂祏曰主上稍有異同詎可盡相乖反【立異為乖不順指為反}
祏曰但以見付必無所憂帝失德寖彰祏議廢帝立江夏王寶玄【夏戶雅翻}
劉暄嘗為寶玄郢州行事執事過刻有人獻馬寶玄欲觀之暄曰馬何用觀妃索煮肫【肫之春翻鳥藏曰肫又徒渾翻豕也}
帳下諮暄暄曰旦已煮鵝不煩復此【復扶又翻又也}
寶玄恚曰舅殊無渭陽情【詩渭陽序曰秦康公之母晉獻公之女文公遭驪姬之難未反而秦姬卒穆公納文公康公時為太子贈送文公於渭之陽念母之不見也我見舅氏如母存焉劉暄明帝劉皇后之弟故寶玄呼之為舅今按詩小序渭陽之事乃甥用情於舅後世率以舅不能用情於甥者為無渭陽情誤矣恚於避翻}
暄由是忌寶玄不同祏議更欲立建安王寶寅祏密謀于始安王遥光遥光自以年長欲自取以微旨動祏祏弟祀亦以少主難保【長知兩翻少詩照翻下同}
勸祏立遥光祏意回惑以問蕭坦之坦之時居母喪起復為領軍將軍【起復者起之於苫塊之中使復其位也}
謂祏曰明帝立已非次天下至今不服若復為此【復扶又翻下可復復能不復生復同}
恐四方瓦解我期不敢言耳遂還宅行喪【蕭坦之冒于榮勢豈能終喪者直以廢立大事不欲預其禍託此以引避耳}
祏祀密謂吏部郎謝朓曰江夏王少脱不堪負荷【朓土了翻荷下可翻又如字}
豈可復行廢立始安年長入纂不乖物望非以此要富貴【要讀如邀}
政是求安國家耳【政與正同}
遥光又遣所親丹楊丞南陽劉渢密致意於朓【渢房戎翻}
欲引以為黨朓不荅頃之遥光以脁兼知衛尉事脁懼【以郎兼卿事本無足懼其所懼者以己為遥光所引用將罹其難也}
即以祏謀告太子右衛率左興盛【率所律翻}
興盛不敢發朓又說劉暄曰始安一旦南面則劉渢劉晏居卿今地但以卿為反覆人耳晏者遥光城局參軍也暄陽驚馳告遥光及祏遥光欲出脁為東陽郡脁常輕祏【謝朓以人門輕江祏}
祏固請除之遥光乃收朓付廷尉與孝嗣祏暄等連名啟朓扇動内外妄貶乘輿竊論宫禁間謗親賢【乘繩證翻間古苑翻}
輕議朝宰朓遂死獄中【謝朓以告王敬則超擢而死於遥光之手行險以徼幸一之謂甚其可再乎朝直遥翻}
暄以遥光若立己失元舅之尊不肯同祏議故祏遲疑久不决遥光大怒遣左右黄曇慶刺暄於青溪橋【曇徒含翻刺七亦翻}
曇慶見暄部伍多不敢暄覺之遂祏謀帝命收祏兄弟時祀直内殿疑有異遣信報祏曰劉暄似有異謀今作何計祏曰政當靜以鎮之俄有詔召祏入見停中書省【見賢遍翻}
初袁文曠以斬王敬則功當封【斬敬則見上卷明帝永泰元年}
祏執不與【時崔恭祖以刺仆敬則與文曠爭功祏執不與當為此也}
帝使文曠取祏【取謂殺之也}
文曠以刀環築其心曰復能奪我封不【不讀曰否}
并弟祀皆死劉暄聞祏等死眠中大驚投出戶外問左右收至未良久意定還坐大悲曰不念江行自痛也【暄自知禍將及己}
帝自是無所忌憚益得自恣日夜與近習於後堂鼓叫戲馬常以五更就寢至晡乃起羣臣節朔朝見【朔謂每月朔旦朔旦朝參之外一月之内又自有朝參日分因謂之節}
晡後方前或際闇遣出【晡後造朝帝復不出故際闇而遣退}
臺閣案奏閲數十日乃報或不知所在宦者以裹魚肉還家並是五省黄案【魏晉以來有六曹尚書江左有吏部祠部五兵左民度支五尚書各為一省謂之尚書五省案文案也藏之以為案據尚書用黄札故曰黄案}
帝常習騎致適【致極也適歡適也}
顧謂左右曰江祏常禁吾乘馬小子若在吾豈能得此因問祏親戚餘誰對曰江祥今在冶【帝誅祏兄弟獨祥免死配東冶}
帝於馬上作敕賜祥死始安王遥光素有異志與其弟荆州刺史遥欣密謀舉兵據東府使遥欣引兵自江陵急下刻期將發而遥欣病卒江祏被誅【被皮義翻}
帝召遥光入殿告以祏罪遥光懼【懼禍及也}
還省【省謂中書省也遥光時為中書令}
即陽狂號哭遂稱疾不復入臺【還東府遂稱疾不復入臺城號戶高翻}
先是遥光弟豫州刺史遥昌卒【先悉薦翻卒子恤翻}
其部曲皆歸遥光及遥欣喪還停東府前渚荆州衆力送者甚盛【前渚秦淮渚也東府前臨秦淮}
帝既誅二江慮遥光不自安欲遷為司徒使還第【遷司徒以崇其位望而使還第養疾}
召入諭旨遥光恐見殺乙卯晡時收集二州部曲於東府東門【二州部曲自荆州豫州來者}
召劉渢劉晏等謀舉兵以討劉暄為名夜遣數百人破東冶出囚於尚方取仗【仗兵仗也}
又召驍騎將軍垣歷生【驍堅堯翻}
歷生隨信而至蕭坦之宅在東府城東遥光遣人掩取之坦之露袒踰墻走【露者露髻袒者肉袒}
向臺【向臺而走欲入言其事}
道逢遊邏主顏端【遊邏主將兵在臺城外廵邏者也邏郎佐翻}
執之【見坦之露袒挺身走疑其得罪逃竄故執之}
告以遥光反不信自往詗問知實【詗火迥翻又翾正翻有所伺謂之詗}
乃以馬與坦之相隨入臺遥光又掩取尚書左僕射沈文季於其宅欲以為都督會文季已入臺垣歷生說遥光帥城内兵夜攻臺輦荻燒城門【荻亭歷翻萑也說輸芮翻帥讀曰率下同}
曰公但乘轝隨後【轝與輿同}
反掌可克遥光狐疑不敢出天稍曉遥光戎服出聽事命上仗登城行賞賜【上時掌翻}
歷生復勸出軍遥光不肯冀臺中自有變及日出臺軍稍至臺中始聞亂衆情惶惑向曉有詔召徐孝嗣孝嗣入人心乃安左將軍沈約聞變【據梁書沈約傳約時為左衛將軍此逸衛字}
馳入西掖門【掖音亦}
或勸戎服約曰臺中方擾攘見我戎服或者謂同遥光乃朱衣而入丙辰詔曲赦建康中外戒嚴徐孝嗣以下屯衛宫城蕭坦之帥臺軍討遥光孝嗣内自疑懼與沈文季戎服共坐南掖門上欲與之共論世事文季輒引以他辭終不得及蕭坦之屯湘宫寺【湘宫寺宋明帝所起}
左興盛屯東籬門【臺城外城六門皆設籬門而已無郛郭東府在臺城東故命興盛屯東籬門以討遥光}
鎮軍司馬曹虎屯青溪大橋【按曹虎傳大橋青溪中橋也}
衆軍圍東城三面燒司徒府【宋元嘉中彭城王義康為司徒徙居東府於東府之側起司徒府}
遥光遣垣歷生從西門出戰臺軍屢敗殺軍主桑天愛遥光之起兵也問諮議參軍蕭暢暢正色不從戊午暢與撫軍長史沈昭略潜自南門出詣臺自歸衆情大沮【東府之衆情也沮在呂翻}
暢衍之弟昭略文季之兄子也己未垣歷生從南門出戰因棄矟降曹虎虎命斬之【矟色角翻降戶江翻 考異曰歷生出戰為曹虎所禽謂虎曰卿以主上為聖明梅茹為賢相我當死且我今死卿明日亦死遂殺之按歷生若見獲遥光不當殺其子今從齊書}
遥光大怒於牀上自踊使殺歷生子其晩臺軍以火箭燒東北角樓至夜城潰遥光還小齋帳中著衣帢坐秉燭自照令人反拒齋閤皆重關【著陟略翻帢苦洽翻重直龍翻}
左右並踰屋散出臺軍主劉國寶等先入遥光聞外兵至滅燭扶匐牀下【扶音蒲匐蒲北翻}
軍人排閤入於闇中牽出斬之臺軍入城焚燒室屋且盡劉渢走還家為人所殺荆州將潘紹聞遥光作亂謀欲應之【欲以江陵應之也將即亮翻}
西中郎司馬夏侯詳【時南康王寶融以西中郎將鎮江陵以夏侯詳為司馬夏戶雅翻}
呼紹議事因斬之州府以安【州荆州府西中郎府也}
己巳以徐孝嗣為司空加沈文季鎮軍將軍侍中僕射如故【沈文季加鎮軍將軍號本職如故}
蕭坦之為尚書右僕射丹楊尹右將軍如故【帝即位之初坦之為右將軍遥光既平使為右僕射丹楊尹而右將軍軍號如故}
劉暄為領軍將軍曹虎為散騎常侍右衛將軍【散悉亶翻騎奇寄翻}
皆賞平始安之功也 魏南徐州刺史沈陵來降【魏高祖置南徐州於宿豫降戶江翻}
陵文季之族子也【沈文秀為宋守東陽明帝泰始五年没於魏文秀文季羣從也陵之入魏當在是時}
時魏徐州刺史京兆王愉年少【少詩照翻}
府事皆决於長史盧淵淵知陵將叛敕諸城潜為之備【敕戒也}
屢以聞於魏朝魏朝不聽陵遂殺將佐帥宿豫之衆來奔【朝直遥翻將即亮翻帥讀曰率}
濱淮諸戍以有備得全陵在邉歷年隂結邉州豪傑陵既叛郡縣多捕送陵黨淵皆撫而赦之唯歸罪於陵衆心乃安【根連株逮則沿邊豪傑懼罪必相帥南奔故悉赦之以安反側}
閠月丙子立江陵公寶覽為始安王奉靖王後【遥光既誅靖王無後故也始安貞王道生長子鳳卒于宋世明帝建武元年贈始安靖王遥光靖王子也}
以沈陵為北徐州刺史【齊南徐州治京口北徐州治鍾離今沈陵自魏南徐州來降因其位任改曰北徐}
江祏等既敗帝左右捉刀應敕之徒皆恣横用事【横戶孟翻}
時人謂之刀敕蕭坦之剛狠而專嬖倖畏而憎之遥光死二十餘日帝遣延明主帥黄文濟將兵圍坦之宅殺之【延明主帥蓋延明殿主帥也狠戶懇翻嬖卑義翻又博計翻帥所類翻將即亮翻下同}
并其子袐書郎賞坦之從兄翼宗為海陵太守【沈約志晉安帝分廣陵立海陵郡今泰州即其地從才用翻守式又翻}
未【受海陵之命而未行也}
坦之謂文濟曰從兄海陵宅故應無它【無它言無它變猶今人言無事也}
文濟曰海陵宅在何處坦之以告文濟白帝帝仍遣收之檢其家至貧唯有質錢帖數百【質錢帖者以物質錢錢主給帖與之以為照驗他日出子本錢收贖}
還以啟帝原其死繫尚方茹法珍等譛劉暄有異志【茹音如}
帝曰暄是我舅豈應有此直閣新蔡徐世標曰明帝乃武帝同堂【明帝高帝兄子於武帝同堂兄弟也}
恩遇如此猶滅武帝之後【恩遇事見一百三十八卷武帝永明十一年滅武帝後見明帝紀}
舅焉可信邪【焉於虔翻何也}
遂殺之曹虎善於誘納日食荒客常數百人【誘音酉食讀曰飤荒客自蠻中及化外來者}
晩節吝嗇罷雍州有錢五千萬它物稱是【雍於用翻稱尺證翻}
帝疑虎舊將且利其財遂殺之坦之暄虎所新除官【坦之虎新除官見上}
皆未及拜而死初高宗殂以隆昌事戒帝曰作事不可在人後【謂欎林王欲殺高宗持疑不以及禍高宗以是而戒帝自謂密矣而非所以貽謀燕翼子也}
故帝數與近習謀誅大臣【數所角翻}
皆於倉猝决意無疑於是大臣人人莫能自保【史言帝昏暴果於誅殺上下揺心}
九月丁未以豫州刺史裴叔業為南兖州刺史征虜長史張冲為豫州刺史 壬戌以頻誅大臣大赦 丙戌魏主謁長陵欲引白衣左右吳人茹皓同車【雖引在左右未命以官故曰白衣左右茹音如}
皓奮衣將登給事黄門侍郎元匡進諫帝推之使下【推吐雷翻}
皓失色而退匡新城之子也【陽平王新城魏高祖之弟}
益州刺史劉季連聞帝失德遂自驕恣用刑嚴酷蜀

人怨之是月遣兵襲中水不克【沈約宋書資江為中水涪江為内水今謂之中江在資州資陽縣西資州漢犍為郡之資中縣地}
於是蜀人趙續伯等皆起兵作亂季連不能制 枝江文忠公徐孝嗣以文士不顯同異【言依違取容於昏暴之朝}
故名位雖重猶得久存虎賁中郎將許凖為孝嗣陳說事機【賁音奔將即亮翻為于偽翻}
勸行廢立孝嗣持疑久之謂必無用干戈之理須帝出遊【須待也}
閉城門召百官集議廢之雖有此懷終不能决諸嬖倖亦稍憎之西豐忠憲侯沈文季自託老病不預朝權【嬖卑義翻又博計翻朝直遥翻}
侍中沈昭略謂文季曰叔父行年六十為員外僕射【文季雖為僕射而不預事故昭略謂之員外僕射}
欲求自免豈可得乎文季笑而不應冬十月乙未帝召孝嗣文季昭略入華林省文季登車顧曰此行恐往而不反帝使外監茹法珍賜以藥酒昭略怒罵孝嗣曰廢昏立明古今令典宰相無才致有今日以甌擲其面【甌小器也所以盛酒}
曰使作破面鬼孝嗣飲藥酒至斗餘乃卒【卒子恤翻}
孝嗣子演尚武康公主况尚山隂公主【武康公主武帝女山隂公主明帝女}
皆坐誅昭略弟昭光聞收至家人勸之逃昭光不忍捨其母入執母手悲泣收者殺之昭光兄子曇亮逃已得免聞昭光死嘆曰家門屠滅何以生為絶吭而死【曇徒含翻吭戶郎翻又戶浪翻沈慶之沈文季皆託老疾不預朝權而終不免於死國無道而富貴則進退皆䧟危機也}
初太尉陳顯逹自以高武舊將【將即亮翻下同}
當高宗之世内懷危懼深自貶損常乘朽弊車道從鹵簿止用羸小者十數人【道讀曰導從才用翻羸倫為翻}
嘗侍宴酒酣啟高宗借枕高宗令與之顯逹撫枕曰臣年衰老富貴已足唯欠枕枕死【酣戶江翻枕枕上如字下之任翻}
特就陛下乞之高宗失色曰公醉矣顯逹以年禮告退【禮大夫七十而致事時顯逹年已七十矣}
高宗不許及王敬則反時顯逹將兵拒魏【事見上卷高宗永泰元年}
始安王遥光疑之啟高宗欲追軍還會敬則平乃止及帝即位顯逹彌不樂在建康得江州甚喜【樂音洛顯逹自馬圈敗還除江州刺史}
嘗有疾不令治既而自愈意甚不悦【蓋求死不得死以至於反也悲夫治直之翻}
聞帝屢誅大臣傳云當遣兵襲江州十一月丙辰顯逹舉兵於尋陽令長史庾弘遠等與朝貴書數帝罪惡【朝直遥翻數所具翻}
云欲奉建安王為主【帝弟寶寅封建安王時為郢州刺史}
須京塵一靜西迎大駕【郢州治夏口在尋陽西}
乙丑以護軍將軍崔慧景為平南將軍督衆軍擊顯逹後軍將軍胡松驍騎將軍李叔獻帥水軍據梁山【驍堅堯翻騎奇寄翻帥讀曰率下同}
左衛將軍左興盛督前鋒軍屯杜姥宅【姥莫補翻}
十二月癸未以前輔國將軍楊集始為秦州刺史【楊集始請降見上卷明帝建武四年}
陳顯逹發尋陽敗胡松於采石【采石山在今太平州當塗縣北八十里山下有采石磯敗補邁翻}
建康震恐甲申軍于新林左興盛帥諸軍拒之顯逹多置屯火於岸側潜軍夜渡襲宫城乙酉顯逹以數千人登落星岡【石頭城西有横壠謂之落星岡}
新亭諸軍聞之奔還宫城大駭閉門設守【守舒救翻}
顯逹執馬矟從步兵數百於西州前與臺軍戰再合顯逹大勝手殺數人矟折【矟色角翻折而設翻}
臺軍繼至顯逹不能抗走至西州後【據蕭子顯齊書顯逹走至西州後烏榜村}
騎官趙潭注刺顯逹墜馬斬之【顯逹傳云潭注矟刺顯逹落馬蓋盡力注矟而刺之也騎官蓋在馬隊主副之下猶今傔官也騎奇寄翻刺七亦翻}
諸子皆伏誅長史庾弘遠炳之之子也【庾炳之柄用於宋元嘉之季}
斬於朱雀航將刑索帽著之曰子路結纓【索山客翻著陟略翻左傳衛侯輒既立其父蒯聵入爭國刼衛卿孔悝與之登臺子路曰太子無勇若燔臺半必舍孔叔太子懼下石乞孟黶以敵子路以戈擊之斷纓子路曰君子死冠不免結纓而死}
吾不可以不冠而死謂觀者曰吾非賊乃是義兵為諸軍請命耳【為于偽翻軍當作君}
陳公太輕事若用吾言天下將免塗炭弘遠子子曜抱父乞代命并殺之帝既誅顯逹益自驕恣漸出遊走又不欲人見之每出先驅斥所過人家唯置空宅【所謂屏除也}
尉司擊鼓蹋圍【晉初洛陽置六部尉江左建康亦置六部尉}
鼓聲所聞【聞音問}
便應奔走不暇衣履犯禁者應手格殺【格擊也}
一月凡二十餘出出輒不言定所東西南北無處不驅常以三四更中【更工衡翻}
鼓聲四出火光照天幡戟横路士民喧走相隨老小震驚啼號塞路處處禁斷【號戶高翻塞悉則翻斷音短}
不知所過【言雖奔走而路斷不知何所可過}
四民廢業樵蘇路斷吉凶失時【吉謂冠婚凶謂喪葬皆不得以時而行事}
乳母寄產【乳儒遇翻育也}
或輿病棄尸不得殯葬巷陌懸幔為高鄣置仗人防守謂之屏除【幔莫半翻仗人謂執仗之人屏必郢翻}
亦謂之長圍嘗至沈公城有一婦人臨產不去因剖腹觀其男女又嘗至定林寺【定林寺舊基在蔣山應潮井後}
有沙門老病不能去藏革間命左右射之百箭俱【射七亦翻}
帝有膂力牽弓至三斛五斗又好擔幢白虎幢高七丈五尺於齒上擔之折齒不倦【好呼到翻擔都甘翻幢傅江翻旛也高居號翻}
自制擔幢校具【校具猶言器械也}
伎衣飾以金玉【伎渠綺翻}
侍衛滿側逞諸變態曾無愧色學乘馬於東冶營兵俞靈韻常著織成袴褶金薄㡌【著則略翻褶音習}
執七寶矟急裝縛袴凌冒雨雪不避阬穽馳騁渴乏輒下馬解取腰邊蠡器酌水飲之【冒莫北翻又如字穽疾正翻騁丑郢翻蠡憐題翻瓠瓢也今謂之馬杓爾雅翼曰蠃古字通於蠡蠃之為量小傳曰以蠡測海言不能極其量也}
復上馬馳去【復扶又翻上時掌翻}
又選無賴小人善走者為逐馬左右五百人常以自隨或於市側過親幸家環回宛轉周徧城邑或出郊射雉置射雉場二百九十六處奔走往來略不暇息【史言帝之昏狂甚于宋欎林王射而亦翻}
王肅為魏制官品百司皆如江南之制凡九品品各有二【九品每品各有正從二品歷隋唐至今猶然}
侍中郭祚兼吏部尚書祚清謹重惜官位每有銓授雖得其人必徘徊久之然後下筆曰此人便已貴矣人以是多怨之然所用者無不稱職【稱尺證翻}


資治通鑑卷一百四十二














































































































































