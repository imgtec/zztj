<!DOCTYPE html PUBLIC "-//W3C//DTD XHTML 1.0 Transitional//EN" "http://www.w3.org/TR/xhtml1/DTD/xhtml1-transitional.dtd">
<html xmlns="http://www.w3.org/1999/xhtml">
<head>
<meta http-equiv="Content-Type" content="text/html; charset=utf-8" />
<meta http-equiv="X-UA-Compatible" content="IE=Edge,chrome=1">
<title>資治通鑒_46-資治通鑑卷四十五_46-資治通鑑卷四十五</title>
<meta name="Keywords" content="資治通鑒_46-資治通鑑卷四十五_46-資治通鑑卷四十五">
<meta name="Description" content="資治通鑒_46-資治通鑑卷四十五_46-資治通鑑卷四十五">
<meta http-equiv="Cache-Control" content="no-transform" />
<meta http-equiv="Cache-Control" content="no-siteapp" />
<link href="/img/style.css" rel="stylesheet" type="text/css" />
<script src="/img/m.js?2020"></script> 
</head>
<body>
 <div class="ClassNavi">
<a  href="/24shi/">二十四史</a> | <a href="/SiKuQuanShu/">四库全书</a> | <a href="http://www.guoxuedashi.com/gjtsjc/"><font  color="#FF0000">古今图书集成</font></a> | <a href="/renwu/">历史人物</a> | <a href="/ShuoWenJieZi/"><font  color="#FF0000">说文解字</a></font> | <a href="/chengyu/">成语词典</a> | <a  target="_blank"  href="http://www.guoxuedashi.com/jgwhj/"><font  color="#FF0000">甲骨文合集</font></a> | <a href="/yzjwjc/"><font  color="#FF0000">殷周金文集成</font></a> | <a href="/xiangxingzi/"><font color="#0000FF">象形字典</font></a> | <a href="/13jing/"><font  color="#FF0000">十三经索引</font></a> | <a href="/zixing/"><font  color="#FF0000">字体转换器</font></a> | <a href="/zidian/xz/"><font color="#0000FF">篆书识别</font></a> | <a href="/jinfanyi/">近义反义词</a> | <a href="/duilian/">对联大全</a> | <a href="/jiapu/"><font  color="#0000FF">家谱族谱查询</font></a> | <a href="http://www.guoxuemi.com/hafo/" target="_blank" ><font color="#FF0000">哈佛古籍</font></a> 
</div>

 <!-- 头部导航开始 -->
<div class="w1180 head clearfix">
  <div class="head_logo l"><a title="国学大师官网" href="http://www.guoxuedashi.com" target="_blank"></a></div>
  <div class="head_sr l">
  <div id="head1">
  
  <a href="http://www.guoxuedashi.com/zidian/bujian/" target="_blank" ><img src="http://www.guoxuedashi.com/img/top1.gif" width="88" height="60" border="0" title="部件查字,支持20万汉字"></a>


<a href="http://www.guoxuedashi.com/help/yingpan.php" target="_blank"><img src="http://www.guoxuedashi.com/img/top230.gif" width="600" height="62" border="0" ></a>


  </div>
  <div id="head3"><a href="javascript:" onClick="javascript:window.external.AddFavorite(window.location.href,document.title);">添加收藏</a>
  <br><a href="/help/setie.php">搜索引擎</a>
  <br><a href="/help/zanzhu.php">赞助本站</a></div>
  <div id="head2">
 <a href="http://www.guoxuemi.com/" target="_blank"><img src="http://www.guoxuedashi.com/img/guoxuemi.gif" width="95" height="62" border="0" style="margin-left:2px;" title="国学迷"></a>
  

  </div>
</div>
  <div class="clear"></div>
  <div class="head_nav">
  <p><a href="/">首页</a> | <a href="/ShuKu/">国学书库</a> | <a href="/guji/">影印古籍</a> | <a href="/shici/">诗词宝典</a> | <a   href="/SiKuQuanShu/gxjx.php">精选</a> <b>|</b> <a href="/zidian/">汉语字典</a> | <a href="/hydcd/">汉语词典</a> | <a href="http://www.guoxuedashi.com/zidian/bujian/"><font  color="#CC0066">部件查字</font></a> | <a href="http://www.sfds.cn/"><font  color="#CC0066">书法大师</font></a> | <a href="/jgwhj/">甲骨文</a> <b>|</b> <a href="/b/4/"><font  color="#CC0066">解密</font></a> | <a href="/renwu/">历史人物</a> | <a href="/diangu/">历史典故</a> | <a href="/xingshi/">姓氏</a> | <a href="/minzu/">民族</a> <b>|</b> <a href="/mz/"><font  color="#CC0066">世界名著</font></a> | <a href="/download/">软件下载</a>
</p>
<p><a href="/b/"><font  color="#CC0066">历史</font></a> | <a href="http://skqs.guoxuedashi.com/" target="_blank">四库全书</a> |  <a href="http://www.guoxuedashi.com/search/" target="_blank"><font  color="#CC0066">全文检索</font></a> | <a href="http://www.guoxuedashi.com/shumu/">古籍书目</a> | <a   href="/24shi/">正史</a> <b>|</b> <a href="/chengyu/">成语词典</a> | <a href="/kangxi/" title="康熙字典">康熙字典</a> | <a href="/ShuoWenJieZi/">说文解字</a> | <a href="/zixing/yanbian/">字形演变</a> | <a href="/yzjwjc/">金 文</a> <b>|</b>  <a href="/shijian/nian-hao/">年号</a> | <a href="/diming/">历史地名</a> | <a href="/shijian/">历史事件</a> | <a href="/guanzhi/">官职</a> | <a href="/lishi/">知识</a> <b>|</b> <a href="/zhongyi/">中医中药</a> | <a href="http://www.guoxuedashi.com/forum/">留言反馈</a>
</p>
  </div>
</div>
<!-- 头部导航END --> 
<!-- 内容区开始 --> 
<div class="w1180 clearfix">
  <div class="info l">
   
<div class="clearfix" style="background:#f5faff;">
<script src='http://www.guoxuedashi.com/img/headersou.js'></script>

</div>
  <div class="info_tree"><a href="http://www.guoxuedashi.com">首页</a> > <a href="/SiKuQuanShu/fanti/">四库全书</a>
 > <h1>资治通鉴</h1> <!--         下载:【右键另存为】即可 --></div>
  <div class="info_content zj clearfix">
  
<div class="info_txt clearfix" id="show">
<center style="font-size:24px;">46-資治通鑑卷四十五</center>
    資治通鑑卷四十五   宋 司馬光 撰<br />
<br />
  胡三省 音註<br />
<br />
  漢紀三十七【起重光作噩盡旃蒙大淵獻凡十五年】<br />
<br />
  顯宗孝明皇帝下<br />
<br />
  永平四年春帝近出觀覽城第【城雒陽城第宅也賢曰有甲乙之次故曰第】欲遂校獵河内【河内郡在雒陽北百二十里】東平王蒼上書諫帝覽奏即還宫 秋九月戊寅千乘哀王建薨無子國除【乘䋲證翻】 冬十月乙卯司徒郭丹司空馮魴免【魴音房】以河南尹沛國范遷為司徒太僕伏恭為司空恭湛之兄子也陵鄉侯梁松坐怨望縣飛書誹謗下獄死【松嗣父統爵為陵鄉】<br />
<br />
  【侯縣讀曰懸下遐稼翻】初上為太子太中大夫鄭興子衆以通經知名【知名者有名於時人皆知之也】太子及山陽王荆因梁松以縑帛請之衆曰太子儲君無外交之義【儲副也】漢有舊防蕃王不宜私通賓客松曰長者意不可逆衆曰犯禁觸罪不如守正而死遂不往及松敗賓客多坐之唯衆不染於辭 于寘王廣德將諸國兵三萬人攻莎車誘莎車王賢殺之【寘徒賢翻莎素禾翻】并其國匈奴發諸國兵圍于寘廣德請降匈奴立賢質子不居徵為莎車王【質音致】廣德又攻殺之更立其弟齊黎為莎車王【更工衡翻】 東平王蒼自以至親輔政【蒼輔政始上卷中元二年】聲望日重意不自安前後累上疏稱自漢興以來宗室子弟無得在公卿位者乞上驃騎將軍印綬退就藩國辭甚懇切帝乃許蒼還國而不聽上將軍印綬【上時掌翻】<br />
<br />
  五年春二月蒼罷歸藩【東平國在雒陽東六百七十二里】帝以驃騎長史為東平太傅掾為中大夫令史為王家郎【百官志將軍長史一人秩千石掾屬二十九人秩比四百石至比二百石令史及御屬三十一人百石帝特為蒼置掾史員四十人王國太傅秩二千石中大夫比六百石郎二百石掾俞絹翻】加賜錢五千萬布十萬匹 冬十月上行幸鄴是月還宮 十一月北匈奴寇五原十二月寇雲中南單于擊却之 是歲發遣邊民在内郡者賜裝錢人二萬【賜錢為辨裝也】 安豐戴侯竇融年老子孫縱誕多不灋長子穆尚内黄公主【内黄縣屬魏郡】矯稱隂太后詔令六安侯劉盱去婦以女妻之【六安國屬廬江郡賢曰今之廬州按前漢以六安為王國後漢以六安為侯國屬廬江郡賢以唐之廬州為漢之廬江郡可也若漢之六安侯國實在唐夀州界劉昫地理志夀州安豐縣漢六國故城在縣南此為可据此後章帝元和二年徙江陵王恭為六安王以廬江郡為國却可以用賢注妻七細翻】盱婦家上書言狀帝大怒盡免穆等官諸竇為郎吏者皆將家屬歸故郡【竇氏故扶風平陵人】獨留融京師融尋薨後數歲穆等復坐事與子勲宣皆下獄死【復扶又翻下遐稼翻】久之詔還融夫人與小孫一人居雒陽<br />
<br />
  六年春二月王雒山出寶鼎獻之【據本紀王雒山在廬江郡】夏四月甲子詔曰祥瑞之降以應有德方今政化多僻何以致兹易曰鼎象三公【三公鼎足承君故云然此盖易緯之辭】豈公卿奉職得其理邪其賜三公帛五十匹九卿二千石半之先帝詔書禁人上事言聖【見四十二卷光武建武七年上時掌翻】而間者章奏頗多浮詞自今若有過稱虛譽尚書皆宜抑而不省【省悉景翻】示不為諂子蚩也【蚩笑也】 冬十月上行幸魯十二月還幸陽城【陽城縣屬頴川】壬午還宫 是歲南單于適死單于莫之子蘇立為丘除車林鞮單于【鞮丁奚翻下同】數月復死【復扶又翻下同】單于適之弟長立為湖邪尸逐侯鞮單于<br />
<br />
  七年春正月癸卯皇太后隂氏崩二月庚申葬光烈皇后【西京諸后皆從帝謚惟衛思后許恭哀后不以夀終而别追諡之從帝謚而又加一字自隂后始范曄曰漢世皇后皆因帝謚為稱明帝始建光烈之稱其後並以德為配至於賢愚優劣混同一貫賢曰謚法執德遵業曰烈】 北匈奴猶盛數寇邊【數所角翻】遣使求合市上冀其交通不復為寇許之 以東海相宋均為尚書令初均為九江太守【九江郡在雒陽東南一千五百里】五日一聽事悉省掾史閉督郵府内屬縣無事【郡有五部督郵監屬縣閉之府内者恐以伺察為功能侵擾屬縣適以多事故也】百姓安業九江舊多虎暴常募設檻穽【賢曰檻為機以捕獸穽謂穿地陷之】而猶多傷害均下記屬縣曰夫江淮之有猛獸猶北土之有雞豚也今為民害咎在殘吏而勞勤張捕【張設也設為機穽以伺鳥獸曰張裴炎猩猩銘所謂奴欲張我是也】非憂恤之本也其務退姦貪思進忠善可一去檻穽【去羌呂翻】除削課制其後無復虎患【復扶又翻】帝聞均名故任以樞機均謂人曰國家喜文法亷吏以為足以止姦也【喜許記翻】然文吏習為欺謾而亷吏清在一已【謾音慢又莫連翻】無益百姓流亡盜賊為害也均欲叩頭爭之時未可改也久將自苦之乃可言耳未及言會遷司隸校尉後上聞其言追善之<br />
<br />
  八年春正月己卯司徒范遷薨 三月辛卯以太尉虞延為司徒衛尉趙憙行太尉事 越騎司馬鄭衆使北匈奴【越騎校尉司馬一人秩千石】單于欲令衆拜衆不為屈單于圍守閉之不與水火衆拔刀自誓【自誓以死不為單于屈也】單于恐而止乃更發使隨衆還京師初大司農耿國上言宜置度遼將軍屯五原以防南匈奴逃亡朝廷不從南匈奴須卜骨都侯等知漢與北虜交使内懷嫌怨欲畔【匈奴異姓大臣左右骨都侯也又異姓有呼衍氏須卜氏立林氏蘭氏皆匈奴國中名族常與單于婚姻】密使人詣北虜令遣兵迎之鄭衆出塞疑有異伺候果得須卜使人【伺相吏翻使疏吏翻】乃上言宜更置大將以防二虜交通由是始置度遼營以中郎將吳棠行度遼將軍事將黎陽虎牙營士屯五原曼栢【漢官儀曰光武以幽冀兵克定天下故於黎陽立營以謁者監領兵騎千人賢曰昭帝拜范明友為度遼將軍至此復置焉曼栢縣在今勝州銀城縣界】 秋郡國十四大水 冬十月北宫成 丙子募死罪繫囚詣度遼營有罪亡命者令贖罪各有差楚王英奉黄縑白紈詣國相曰【漢成帝王國省内史令相治民職如太守秩二千石紈今之絹也師古曰紈素也縑并絲絹也相息亮翻】託在藩輔過惡果積歡喜大恩奉送縑帛以贖愆罪國相以聞詔報曰楚王誦黄老之微言尚浮屠之仁慈潔齊三月【齊讀曰齋】與神為誓何嫌何疑當有悔吝其還贖以助伊蒲塞桑門之盛饌【塞悉則翻饌雛戀翻又雛皖翻】初帝聞西域有神其名曰佛因遣使之天竺求其道得其書及沙門以來其書大抵以虚無為宗貴慈悲不殺以為人死精神不滅隨復受形生時所行善惡皆有報應故所貴修煉精神以至為佛善為宏闊勝大之言以勸誘愚俗精於其道者號曰沙門於是中國始傳其術圖其形像而王公貴人獨楚王英最先好之【袁宏漢紀浮屠佛也西域天竺國有佛道焉佛者漢言覺也將以覺悟羣生也其教以修善心為主不殺生專務清静其精者為沙門沙門漢言息也盖息意去欲以歸於無為長丈六尺黄金色初明帝夢見金人長大以問羣臣或曰西方有神其名曰佛陛下所夢得無是乎於是遣使天竺問其道術而圖其形像焉賢曰伊蒲塞即優婆塞也中國翻為近住言受戒行堪近僧住也桑門即沙門梵云沙門那或曰桑門唐言勤息秦譯云勤行又云善覺魏收曰漢武帝遣霍去病討匈奴獲休屠王金人以為大神列於甘泉宫不祭祀但燒香禮拜而已此則佛道流通之漸也張騫使大夏傳其旁有身毒國一名天竺始聞有浮屠之教哀帝元夀元年博士弟子秦景憲受大月氏王使伊存口授浮屠經中國聞之未信了也後明帝夜夢金人頂有白光飛行殿庭乃訪羣臣傅毅始以佛對帝遣郎中蔡愔等使天竺寫浮屠遺範仍與沙門攝摩騰竺法蘭東還洛陽中國有沙門跪拜之法自此始愔之還以白馬負經而至漢因立白馬寺於洛城雍關西好呼到翻】 壬寅晦日有食之旣【既盡也】詔羣司勉修職事極言無諱於是在位者皆上封事各言得失帝覽章深自引咎以所上班示百官【上時掌翻】詔曰羣僚所言皆朕之過民寃不能理吏黠不能禁【黠下八翻】而輕用民力繕修宫宇出入無節喜怒過差永覽前戒竦然兢懼徒恐薄德久而致怠耳【人主能切已省察然後能有是言】 北匈奴雖遣使入貢而寇鈔不息【鈔楚交翻】邊城書閉帝議遣使報其使者鄭衆上疏諫曰臣聞北單于所以要致漢使者【要一遥翻】欲以離南單于之衆堅三十六國之心也【賢曰武帝間通西域本三十六國余謂堅其心者欲使之專附匈奴】又當揚漢和親誇示鄰敵令西域欲歸化者局足狐疑懷土之人絶望中國耳漢使旣到便偃蹇自信【信音申】若復遣之虜必自謂得謀【得謀猶言得計復扶又翻下同】其羣臣駁議者不敢復言【賢曰駁議謂勸單于歸漢駁北角翻】如是南庭動揺烏桓有離心矣【南單于庭在西河美稷動揺謂欲出塞北去烏桓本附匈奴漢置校尉領護使不得與匈奴交通離心謂其心不親附漢而貳於匈奴也】南單于久居漢地具知形埶萬分離析旋為邊害今幸有度遼之衆揚威北垂雖勿報答不敢為患帝不從復遣衆往衆因上言臣前奉使不為匈奴拜【為于偽翻下同】單于恚恨遣兵圍臣【恚於避翻】今復銜命必見陵折臣誠不忍持大漢節對氈裘獨拜【前書匈奴傳曰自君王以下皆食畜肉衣其皮革被旃裘旃與氈同】如令匈奴遂能服臣將有損大漢之彊帝不聽衆不得已旣行在路連上書固争之詔切責衆追還繋廷尉會赦歸家其後帝見匈奴來者聞衆與單于爭禮之狀乃復召衆為軍司馬【漢制大將軍營五部部校尉一人比二千石軍司馬一人比千石其不置校尉部但軍司馬一人帝召衆為軍司馬使與馬廖擊車師】<br />
<br />
  九年夏四月甲辰詔司隸校尉部刺史歲上墨綬長吏視事三歲已上治狀尤異者各一人與計偕上及尤不治者亦以聞【杜佑曰後漢十三州部司隸治河南今府豫治譙今鄼縣兖治昌邑今魯郡金鄉縣徐治郯今臨淮郡下邳縣青治臨淄今北海郡縣涼治隴今天水郡隴城縣并治晉陽今太原府冀治鄗今趙郡鄗縣幽治薊今范陽郡揚治歷陽今郡縣荆治漢夀今武陵郡武陵縣交治廣信今蒼梧郡蒼梧縣漢制千石六百石墨綬三采青赤紺長丈六尺八十首四百石三百石長同此墨綬長吏謂大縣令以下上時掌翻治直吏翻】 是歲大有年【穀梁傳曰五穀皆熟書大有年】 賜皇子恭號曰靈夀王黨號曰重熹王【賢曰取其美名也】未有國邑 帝崇尚儒學自皇太子諸王侯及大臣子弟功臣子孫莫不受經又為外戚樊氏郭氏隂氏馬氏諸子立學於南宫號四姓小侯【賢曰以非列侯故曰小侯禮記曰庶方小侯亦其義也余據東平王蒼傳送列侯印十九枚諸王子年五歲以上能趨拜者皆令帶之意四姓小侯亦猶是也】置五經師搜選高能以授其業自期門羽林之士悉令通孝經章句匈奴亦遣子入學 廣陵王荆復呼相士謂曰我貌類先帝先帝三十得天下我今亦三十可起兵未相者詣吏告之【相息亮翻】荆惶恐自繫獄帝加恩不考極其事詔不得臣屬吏民唯食租如故【恐其復謀不軌故不得臣屬吏民唯食國之租税】使相中尉謹宿衛之荆又使巫祭祀祝詛【祝職又翻詛莊助翻】詔長水校尉樊鯈等雜治其獄【鯈直留翻治直之翻】事竟奏請誅荆帝怒曰諸卿以我弟故欲誅之即我子卿等敢爾邪鯈對曰天下者高帝天下非陛下之天下也春秋之義君親無將將而必誅【賢曰春秋公羊傳之文也將者將為弑逆之事也】臣等以荆屬託母弟【帝與荆皆出於隂后】陛下留聖心加惻隱故敢請耳如令陛下子臣等專誅而已【賢曰專謂不請也】帝歎息善之鯈宏之子也十年春二月廣陵思王荆自殺【諡法追悔前過曰思】國除 夏四月戊子赦天下 閏月甲午上幸南陽召校官弟子作雅樂【賢曰校學也戶教翻雅樂註見上】奏鹿鳴帝自奏塤箎和之以娛嘉賓【鄭玄註周禮云塤燒土為之大如鴈子鄭衆云有六孔世本曰暴辛公作箎以竹為之長尺四寸有八孔孔頴達曰土曰壎竹曰箎周禮小師職作塤古今字異耳釋樂云大塤謂之嘂音叫孫炎曰音大如叫呼也郭璞曰塤燒土為之大如鵞子鋭上平底形似稱鎚六孔小者如雞子釋樂又云大箎謂之沂李廵曰大箎其聲非一也郭璞曰箎以竹為之長尺四寸圍三寸一孔上出逕三分横吹之小者尺二寸廣雅云八孔鄭司農小師注云箎七孔盖不數其上出者故七也世本云暴辛公作塤蘇成公作箎譙周古史考云古有塤箎尚矣周幽王時暴辛公善塤蘇成公善箎記者因以為作繆矣釋名塤喧也聲濁喧然塤況袁翻箎音池和戶卧翻】還幸南頓冬十二月甲午還宫 初陵陽侯丁綝卒【陵陽縣屬丹陽郡綝丑林翻】子鴻當襲封上書稱病讓國於弟盛不報旣葬乃挂衰絰於冢廬而逃去【衰倉回翻】友人九江鮑駿遇鴻於東海【東海郡在雒陽東一千五百里】讓之曰昔伯夷吳札亂世權行故得申其志耳【賢曰伯夷孤竹君之子讓其弟叔齊季札吳王夀夢之季子也諸兄欲讓以國季子乃舍其室而耕皆是權時所行非常道也伯夷當紂時季札當周末故言亂世也】春秋之義不以家事廢王事【春秋衛靈公卒孫輒立父蒯聵與輒爭國公羊傳曰輒者蒯聵之子然則曷為不立蒯聵而立輒蒯聵無道靈公逐之而立輒然則輒之義可以立乎曰可不以父命辭於王命不以家事辭於王事故駿引以為言】今子以兄弟私恩而絶父不滅之基可乎鴻感悟垂涕乃還就國鮑駿因上書薦鴻經學至行【行下孟翻】上徵鴻為侍中<br />
<br />
  十一年春正月東平王蒼與諸王俱來朝月餘還國帝臨送歸宫悽然懷思乃遣使手詔賜東平國中傳曰辭别之後獨坐不樂【樂音洛下同】因就車歸伏軾而吟瞻望永懷實勞我心誦及采菽以增歎息【采菽詩小雅之章也其詩曰采菽采菽筐之筥之君子來朝何錫予之毛詩註云菽所以芼太牢而待君子羊則苦豕則薇箋云菽大豆也采其葉以為藿三牲牛羊豕芼以藿正義曰傳既言羊則苦豕則薇則菽不揔芼三牲而言菽所以芼太牢者舉牛之芼則羊豕之苦薇從可知矣】日者問東平王處家何等最樂王言為善最樂【處昌呂翻樂音洛】其言甚大副是要腹矣【要讀曰腰蒼腰帶十圍】今送列侯印十九枚諸王子年五歲已上能趨拜者皆令帶之<br />
<br />
  十二年春哀牢王柳貌率其民五萬餘戶内附以其地置哀牢博南二縣【哀牢夷者九隆種也居牢山絶域荒外山川阻深未嘗通中國西南去雒陽七千里賢曰在今匡州匡川縣西張柬之曰姚州哀牢國地】始通博南山度蘭倉水【華陽國志曰博南縣西山高三十里越之得蘭倉水有金沙洗取融為金】行者苦之歌曰漢德廣開不賓度蘭倉為它人【為于偽翻】 初平帝時河汴決壞久而不修建武十年光武欲修之浚儀令樂俊上言民新被兵革未宜興役乃止【浚儀縣屬陳留郡被皮義翻】其後汴渠東侵日月彌廣兖豫百姓怨歎以為縣官恒興他役不先民急會有薦樂浪王景能治水者【樂浪在雒陽東北五千里恒戶登翻先悉薦翻樂浪音洛琅】夏四月詔發卒數十萬遣景與將作謁者王吳脩汴渠隄自滎陽東至千乘海口千餘里【謁者屬光祿勲王吳以謁者而將作故謂之將作謁者賢曰汴渠即莨蕩渠也汴自滎陽首受河所謂石門在滎陽山北一里過汴以東積石為隄亦號金隄成帝陽嘉中所作也】十里立一水門令更相洄注【爾雅曰逆流而上曰洄郭璞註云旋流也更工衡翻】無復潰漏之患【復扶又翻】景雖減省役費然猶以百億計焉【十萬曰億】 秋七月乙亥司空伏恭罷乙未以大司農牟融為司空【風俗通牟子國祝融之後後因氏焉】是時天下安平人無徭役歲比登稔百姓殷富粟斛三十牛羊被野【比毗至翻被皮義翻】<br />
<br />
  十三年夏四月汴渠成河汴分流復其舊迹【河汴之隄決壞則汴水東侵而與河合今隄成則河東北入海而汴東南入泗是分流復其舊迹也】辛巳帝行幸滎陽廵行河渠【行下孟翻】遂度河登太行幸上黨【行戶剛翻】壬寅還宫 冬十月壬辰晦日有食之 楚王英與方士作金龜玉鶴刻文字為符瑞男子燕廣【姓譜燕召公之後為秦所滅子孫以國為氏燕於賢翻】告英與漁陽王平顔忠等造作圖書有逆謀事下案驗【下遐稼翻】有司奏英大逆不道請誅之帝以親親不忍十一月廢英徙丹陽涇縣【賢曰今宣州縣】賜湯沐邑五百戶【賢曰湯沐者取其賦税以供湯沐之具也】男女為侯主者食邑如故許太后勿上璽綬留住楚宫【許太后者英母許氏上時掌翻】先是有私以英謀告司徒虞延者【先悉薦翻】延以英藩戚至親不然其言及英事覺詔書切讓延<br />
<br />
  十四年春三月甲戍延自殺以太常周澤行司徒事頃之復為太常 【考異曰澤傳云十二年按十二年不闕司徒當是虞延免後邢穆未至閒澤行司徒事爾故云數月】夏四月丁巳以鉅鹿太守南陽邢穆為司徒【鉅鹿郡在雒陽北一千一百里邢本周公之胤為衛所滅子孫以國為氏】 楚王英至丹陽自殺詔以諸侯禮葬於涇封燕廣為折姦侯是時窮治楚獄遂至累年【治直之翻下同】其辭語相連自京師親戚諸侯州郡豪桀及考案吏阿附坐死徙者以千數而繫獄者尚數千人初樊鯈弟鮪【鯈直留翻鮪于軌翻】為其子賞求楚王英女【為于偽翻】鯈聞而止之曰建武中吾家並受榮寵一宗五侯【謂宏封長羅侯弟丹射陽侯兄子尋玄鄉侯族兄忠更父侯宏又封夀張侯也】時特進一言女可以配王男可以尚主【賢曰宏為特進】但以貴寵過盛即為禍患故不為也且爾一子柰何棄之於楚乎鮪不從及楚事覺鯈已卒上追念鯈謹恪故其諸子皆得不坐英隂疏天下名士上得其錄有吳郡太守尹興名【吳郡在雒陽東三千三百里】乃徵興及掾史五百餘人詣廷尉就考【掾俞絹翻】諸吏不勝掠治【勝音升掠音亮治直之翻】死者太半唯門下掾陸續主簿梁宏功曹史駟勲備受五毒【門下掾在郡門下緫錄衆事功曹史主選署功勞五毒四肢及身備受楚毒也或云鞭箠及灼及徽經為五毒】肌肉消爛終無異辭續母自吳來雒陽作食以饋續續雖見考辭色未嘗變而對食悲泣不自勝治獄使者問其故續曰母來不得見故悲耳問何以知之續曰母截肉未嘗不方斷葱以寸為度【斷丁管翻】故知之使者以狀聞上乃赦興等禁錮終身顔忠王平辭引隧鄉侯耿建朗陵侯臧信濩澤侯鄧鯉曲成侯劉建【耿純弟宿封隧鄉侯建盖紹封者也朗陵侯臧信宮之子也鄧鯉劉建皆無可考濩澤侯國屬河東郡曲成侯國屬東萊郡賢曰故城在今萊州掖縣西北師古曰濩音烏虢翻】建等辭未嘗與忠平相見是時上怒甚吏皆惶恐諸所連及率一切䧟入無敢以情恕者侍御史寒朗心傷其寃 【考異曰范書作寒陸龜蒙離合詩云初寒朗詠徘徊立袁紀作寋按今有寋姓音件與袁紀合今從之余按姓譜有寒姓以為夏諸侯后寒之後又曰周武王子寒侯之後】試以建等物色獨問忠平【賢曰物色謂形狀也】而二人錯㦍不能對【賢曰錯㦍猶倉卒也錯音七故翻㦍音五故翻】朗知其詐乃上言建等無姦專為忠平所誣疑天下無辜類多如此帝曰即如是忠平何故引之對曰忠平自知所犯不道【漢法有大逆不道】故多有虚引冀以自明帝曰即如是何不早奏對曰臣恐海内别有發其姦者帝怒曰吏持兩端促提下捶之【捶止蕊翻】左右方引去朗曰願一言而死帝曰誰與共為章對曰臣獨作之上曰何以不與三府議【三府太尉司徒司空府也】對曰臣自知當必族滅不敢多汙染人【汙烏故翻】上曰何故族滅對曰臣考事一年不能窮盡姦狀反為罪人訟寃【為于偽翻下同】故知當族滅然臣所以言者誠冀陛下一覺悟而已臣見考囚在事者咸共言妖惡大故【故事也囚也妖於驕翻】臣子所宜同疾今出之不如入之【言出其罪不如入其罪也】可無後責是以考一連十考十連百又公卿朝會陛下問以得失皆長跪言舊制大罪禍及九族陛下大恩裁止於身天下幸甚【裁與纔同】及其歸舍口雖不言而仰屋竊歎莫不知其多寃無敢牾陛下言者【牾五故翻逆也】臣今所陳誠死無悔帝意解詔遣朗出後二日車駕自幸洛陽獄錄囚徒【師古曰省錄之知其情狀為寃滯為不也今之慮囚本錄聲之去者耳音力具翻而近俗不曉其意訛其文遂為思慮之慮失其源矣】理出千餘人時天旱即大雨馬后亦以楚獄多濫乘間為帝言之【間古莧翻】帝惻然感悟夜起彷徨【彷徨釋徘徊也莊子註猶翺翔也余謂彷徨不自安之貌】由是多所降宥任城令汝南袁安遷楚郡太守【任城縣屬東平國任音壬】到郡不入府先往按楚王英獄事理其無明驗者條上出之【上時掌翻】府丞掾史皆叩頭爭以為阿附反虜法與同罪不可安曰如有不合太守自當坐之不以相及也遂分别具奏【别彼列翻】帝感悟即報許得出者四百餘家 夏五月封故廣陵王荆子元夀為廣陵侯食六縣【篤兄弟之恩也】又封竇融孫嘉為安豐侯【念功臣之世也】 初作夀陵制令流水而已無得起墳萬年之後埽地而祭杅水脯糒而已【說文曰杅飲器音于方言曰盌謂之盂】過百日唯四時設奠置吏卒數人供給灑掃【灑所賣翻掃悉報翻又並如字】敢有所興作者以擅議宗廟法從事【前書曰擅議宗廟者棄市】<br />
<br />
  十五年春二月庚子上東廵癸亥耕于下邳【下邳縣本屬東海郡是年以臨淮郡為下邳國下邳縣屬焉在雒陽東一千四百里】三月至魯幸孔子宅親御講堂【孔子宅在闕里講堂講授之堂魯共王升孔子堂聞金石絲竹之音即此】命皇太子諸王說經又幸東平大梁【浚儀縣本大梁】夏四月庚子還宫封皇子恭為鉅鹿王黨為樂成王【樂成國本信都郡帝更名在雒陽北二千里】衍為下邳王暢為汝南王昞為常山王長為濟隂王【濟子禮翻】帝親定其封域裁令半楚淮陽馬后曰諸子數縣於制不亦儉乎帝曰我子豈宜與先帝子等歲給二千萬足矣 乙巳赦天下 謁者僕射耿秉數上言請擊匈奴【百官志謁者僕射秩比千石為謁者臺率主謁者古重習武有主射以督錄之故曰僕射數所角翻】上以顯親侯竇固嘗從其世父融在河西【爾雅曰父之昆弟先生為世父後生為叔父】明習邊事乃使秉固與太僕祭肜虎賁中郎將馬廖【廖音聊】下博侯劉張【張齊王縯之孫】好畤侯耿忠等共議之【畤音止】耿秉曰昔者匈奴援引弓之類【援于元翻】并左袵之屬故不可得而制孝武旣得河西四郡及居延朔方【居延武帝置縣屬張掖郡賢曰故城在今甘州張掖縣北】虜失其肥饒畜兵之地羌胡分離唯有西域俄復内屬【復扶又翻】故呼韓邪單于請事欵塞其勢易乘也【易以䜴翻】今有南單于形勢相似然西域尚未内屬北虜未有釁隙臣愚以為當先撃白山【西河舊事曰白山冬夏有雪故曰白山匈奴謂之天山過之皆下馬拜焉去蒲類海百里之内】得伊吾【賢曰伊吾即伊吾盧地本屬匈奴後取其地置宜禾都尉以為屯田今伊州細職縣伊吾故城是也又曰伊吾故城在今瓜州晉昌縣北】破車師通使烏孫諸國以斷其右臂【使疏吏翻斷丁管翻】伊吾亦有匈奴南呼衍一部破此復為折其左角【復扶又翻折而設翻】然後匈奴可撃也上善其言議者或以為今兵出白山匈奴必并兵相助又當分其東以離其衆上從之十二月以秉為駙馬都尉固為奉車都尉以騎都尉秦彭為秉副【三都尉皆武帝置奉車都尉掌乘輿駙馬都尉掌天子之副馬師古曰駙副也一回近也疾也】耿忠為固副皆置從事司馬出屯涼州秉國之子忠弇之子廖援之子也<br />
<br />
  十六年春二月遣肜與度遼將軍吳棠將河東西河羌胡及南單于兵萬一千騎出高闕塞【高闕在朔方北】竇固耿忠率酒泉敦煌張掖甲卒及盧水羌胡萬二千騎出酒泉塞【賢曰案湟水東經臨羌縣故城北又東盧溪水注之水出西南盧川郡其地也余據西南夷傳冉駹夷北有黄石北地盧水胡敦徒門翻】耿秉秦彭率武威隴西天水募士及羌胡萬騎出張掖居延塞騎都尉來苖護烏桓校尉文穆將太原鴈門代郡上谷漁陽右北平定襄郡兵及烏桓鮮卑萬一千騎出平城塞伐北匈奴竇固耿忠至天山【賢曰天山即祁連山一名雪山今名折羅漢山在伊州北漢一作漫】撃呼衍王斬首千餘級追至蒲類海【賢曰蒲類海今名婆悉海在今庭州蒲昌縣東南】取伊吾盧地置宜禾都尉留吏士屯田伊吾盧城耿秉秦彭擊匈林王【匈林恐當作句林建武時匈奴嘗遣句林王迎盧芳句音古侯翻】絶幕六百餘里至三木樓山而還來苖文穆至匈河水上【據前書匈河水去令居數千里臣瓚曰去令居千里】虜皆犇走無所獲祭肜與南匈奴左賢王信不相得出高闕塞九百餘里得小山信妄言以為涿邪山【北史曰循弱水西行得涿邪山】不見虜而還肜與吳棠坐逗留畏懦下獄免【下遐稼翻 考異曰袁紀棠皆作常今從范書】肜自恨無功出獄數日歐血死臨終謂其子曰吾蒙國厚恩奉使不稱【稱尺證翻】身死誠慚恨義不可以無功受賞死後若悉簿上所得物【若汝也皆為文簿而上之上時掌翻】身自詣兵屯效死前行以副吾心旣卒【行戶剛翻卒子恤翻】其子逢上疏具陳遺言帝雅重肜方更任用聞之大驚嗟嘆良久烏桓鮮卑每朝賀京師常過肜冢拜謁仰天號泣遼東吏民為立祠四時奉祭焉【肜先為遼東太守威信行於烏桓鮮卑號戶刀翻為于偽翻】竇固獨有功加位特進固使假司馬班超與從事郭恂俱使西域【百官志大將軍營五部部有校尉一人軍司馬一人又有軍假司馬為副貳使疏吏翻下同】超行到鄯善【鄯上扇翻】鄯善王廣奉超禮敬甚備後忽更踈懈【懈古隘翻】超謂其官屬曰寧覺廣禮意薄乎官屬曰胡人不能常久無它故也超曰此必有北虜使來狐疑未知所從故也明者睹未萌況已著邪乃召侍胡詐之曰匈奴使來數日今安在乎侍胡惶恐曰到已三日去此三十里超乃閉侍胡【侍胡鄯善所遣侍超者使疏吏翻】悉會其吏士三十六人與共飲酒酣因激怒之曰卿曹與我俱在絶域今虜使到裁數日而王廣禮敬即廢如令鄯善收吾屬送匈奴骸骨長為豺狼食矣為之奈何官屬皆曰今在危亡之地死生從司馬超曰不入虎穴不得虎子當今之計獨有因夜以火攻虜使彼不知我多少必大震怖可殄盡也【怖普布翻】滅此虜則鄯善破膽功成事立矣衆曰當與從事議之超怒曰吉凶決於今日從事文俗吏聞此必恐而謀泄死無所名非壯士也衆曰善初夜超遂將吏士往犇虜營【初夜甲夜也】會天大風超令十人持鼔藏虜舍後約曰見火然皆當鳴鼔大呼【呼火故翻】餘人悉持兵弩夾門而伏超乃順風縱火前後鼔噪虜衆驚亂超手格殺三人吏兵斬其使及從士三十餘級【從才用翻】餘衆百許人悉燒死明日乃還告郭恂【還從宣翻又如字】恂大驚旣而色動【意欲分超功而不能自揜於外故色動】超知其意舉手曰掾雖不行班超何心獨擅之乎【從事掾也掾俞絹翻】恂乃悦超於是召鄯善王廣以虜使首示之一國震怖超告以漢威德自今以後勿復與北虜通【復扶又翻】廣叩頭願屬漢無二心遂納子為質還白竇固固大喜具上超功効【質音致上時掌翻】并求更選使使西域帝曰吏如班超何故不遣而更選乎今以超為軍司馬令遂前功固復使超使于寘【復扶又翻下同寘徒賢翻】欲益其兵超願但將本所從三十六人曰于寘國大而遠今將數百人無益於彊如有不虞多益為累耳【累力瑞翻】是時于寘王廣德雄張南道【賢曰雄張猶熾盛也張竹亮翻予謂張者自大之意】而匈奴遣使監護其國【監古銜翻】超旣至于寘廣德禮意甚踈且其俗信巫巫言神怒何故欲向漢漢使有騧馬急求取以祠我【賢曰續漢及華蹻書並作騩說文馬淺黑色也音京媚翻予謂騧音瓜黄馬黑喙曰騧讀如本字】廣德遣國相私來比就超請馬【相息亮翻】超密知其狀報許之而令巫自來取馬有頃巫至超即斬其首收私來比鞭笞數百以巫首送廣德因責讓之廣德素聞超在鄯善誅滅虜使大惶恐即殺匈奴使者而降【降戶江翻】超重賜其王以下加鎮撫焉於是諸國皆遣子入侍西域與漢絶六十五載至是乃復通焉【王莽天鳳三年焉耆撃教王駿西域遂絶至此五十八載耳此言與漢絶六十五載盖自始建國元年數之謂莽簒漢而西域遂與漢絶也復扶又翻載子亥翻】超彪之子也 淮陽王延性驕奢而遇下嚴烈有上書告延與姬兄謝弇及姊婿韓光招姦猾作圖䜟祠祭祝詛事下案驗【䜟楚譖翻祝職救翻詛莊助翻下遐稼翻】五月癸丑弇光及司徒邢穆皆坐死所連及死徙者甚衆 戊午晦日有食之 六月丙寅以大司農西河王敏為司徒 有司奏請誅淮陽王延上以延罪薄於楚王英秋七月徙延為阜陵王食二縣【賢曰阜陵縣名屬九江郡故城在今滁州全椒縣南】 是歲北匈奴大入雲中雲中太守亷范拒之吏以衆少欲移書傍郡求救范不許會日暮范令軍士各交縛兩炬三頭爇火營中星列【賢曰用兩炬交縛如十字爇其三頭手持一端使敵人望之疑兵士之多爇懦劣翻】虜謂漢兵救至大驚待旦將退范令軍中蓐食晨往赴之【賢曰蓐食早起食於寢蓐中也】斬首數百級虜自相軫藉死者千餘人【賢曰轔轢也藉相蹈藉也轢良力翻】由此不敢復向雲中【復扶又翻】范丹之孫也【亷丹為王莽將】<br />
<br />
  十七年春正月上當謁原陵夜夢先帝太后如平生歡旣寤悲不能寐即案歷明旦日吉遂率百官上陵【上時掌翻】其日降甘露於陵樹 【考異曰帝紀云甘露降甘陵皇后紀云謂原陵甘露降於樹然則實降原陵也帝紀誤以原為甘】帝令百官采取以薦會畢帝從席前伏御床視太后鏡匳中物【匳鏡匣也音亷】感動悲涕令易脂澤裝具左右皆泣莫能仰視【沈約曰三代以前無墓祭至秦始出寢起於墓側漢因秦上陵皆有園寢故稱寢殿起居衣服象生人之具古寢之意也】 北海敬王睦薨【睦北海靖王興之子】睦少好學【少詩照翻好呼到翻下同】光武及上皆愛之嘗遣中大夫詣京師朝賀【賢曰中大夫王國官也掌奉玉使京師奉璧賀正月朝直遥翻】召而謂之曰朝廷設問寡人【賢曰朝廷謂天子也】大夫將何辭以對使者曰大王忠孝慈仁敬賢樂士【樂音洛】臣敢不以實對睦曰吁子危我哉【賢曰吁音于孔安國註尚書曰吁者疑怪之聲余按吁匈于翻】此乃孤幼時進趣之行也【趣讀曰趨又七喻翻行下孟翻】大夫其對以孤襲爵以來志意衰惰聲色是娛犬馬是好乃為相愛耳其智慮畏慎如此【時禁切藩王法憲頗峻故睦慮及此】 二月乙巳司徒王敏薨三月癸丑以汝南太守鮑昱為司徒昱永之子也<br />
<br />
  益州刺史梁國朱輔【益州部漢中巴郡廣漢蜀郡犍為牂柯越嶲益州永昌等郡益州刺史治廣漢郡雒縣】宣示漢德威懷遠夷自汶山以西【汶山在蜀郡湔氐道西徼外江水所出杜佑曰茂州漢汶山縣汶晉書音讀曰岷湔裴松之音剪】前世所不至正朔所未加白狼槃木等百餘國皆舉種稱臣奉貢【種章勇翻】白狼王唐菆作詩三章歌頌漢德【菆側鳩翻又徂丸翻】輔使犍為郡掾由恭譯而獻之【犍為郡在雒陽西三千二百七十里夷言不與中國通故譯而後獻犍居言翻掾俞絹翻由姓也秦有由余或曰楚王孫由子之後】 初龜兹王建為匈奴所立【龜兹音見前】倚恃虜威據有北道攻殺疏勒王立其臣兜題為疏勒王班超從間道至疏勒【間古莧翻范史疏勒國去雒陽萬三百里】去兜題所居槃槖城九十里逆遣吏田慮先往降之【降戶江翻下同】敕慮曰兜題本非疏勒種【種章勇翻】國人必不用命若不即降便可執之慮旣到兜題見慮輕弱殊無降意【降戶江翻】慮因其無備遂前劫縛兜題左右出其不意皆驚懼奔走慮馳報超超即赴之悉召疏勒將吏說以龜兹無道之狀因立其故王兄子忠為王 【考異曰袁紀云求索故王近屬得兄榆勒立之更名忠續漢書云求得故王兄子榆勒立之更名忠今從超傳】國人大悦超問忠及官屬當殺兜題邪生遣之邪咸曰當殺之超曰殺之無益於事當令龜兹知漢威德遂解遣之 夏五月戊子公卿百官以帝威德懷遠祥物顯應並集朝堂奉觴上夀【班固西都賦左右廷中朝堂百僚之位蕭曹丙魏謀謨乎其上盖在殿庭左右也賢曰夀者人之所欲故卑下奉觴進酒皆言上夀朝直遥翻】制曰天生神物以應王者遠人慕化實由有德朕以虚薄何以享斯唯高祖光武聖德所被【被皮義翻】不敢有辭其敬舉觴太常擇吉日策告宗廟仍推恩賜民爵及粟有差【時賜天下男子爵人二級三老孝悌力田人三級流人無名數欲占者人一級鰥寡孤獨篤癃貧不能自存者粟人三斛】 冬十一月遣奉車都尉竇固駙馬都尉耿秉騎都尉劉張出敦煌昆侖塞撃西域【賢曰昆侖山名因以為塞在今肅州酒泉縣西南山有昆侖之體故名之周穆王見西王母于此山有石室王母臺又曰前書敦煌郡廣至縣有昆侖障宜禾都尉居也廣至故城在今瓜州常樂縣東敦徒門翻侖盧昆翻】秉張皆去符傳以屬固【符傳皆合之以為信符兵符也張晏曰傳若今過所也如淳曰兩首書繒帛分持其一出入關合之乃得過謂之傳此傳盖亦行兵所用以為信非度關所用之傳也專將則有符傳今以兵屬固故去之去羌呂翻傳株戀翻】合兵萬四千騎撃破白山虜於蒲類海上遂進撃車師車師前王即後王之子也其廷相去五百餘里【車師前王居交河城後王居務塗谷】固以後王道遠山谷深士卒寒苦欲攻前王秉以為先赴後王并力根本則前王自服固計未決秉奮身而起曰請行前乃上馬引兵北入衆軍不得已並進斬首數千級後王安得震怖走出門迎秉脱帽抱馬足降【降戶江翻】秉將以詣固其前王亦歸命遂定車師而還【還從宣翻又如字】於是固奏復置西域都護及戊巳校尉【宣帝置都護元帝置戊巳校尉自王莽之亂西域與中國絶不復置今通西域復置之】以陳睦為都護 【考異曰袁紀睦作穆今從范書】司馬耿恭為戊校尉屯後王部金蒲城【賢曰金蒲城車師後王城廷也今庭州蒲昌縣城是也杜佑曰金蒲城即車師後王所治務塗谷今北庭府蒲類縣也】謁者關寵為己校尉屯前王部柳中城【賢曰柳中今西州縣 考異曰袁紀作折中今從范書】屯各置數百人恭況之孫也【耿況以上谷歸光武子孫多著功名】<br />
<br />
  十八年春二月詔竇固等寵兵還京師 北單于遣左鹿蠡王率二萬騎擊車師【蠡慮奚翻】耿恭遣司馬將兵三百人救之皆為所沒匈奴遂破殺車師後王安得而攻金蒲城恭以毒藥傅矢語匈奴曰漢家箭神其中瘡者必有異虜中矢者視瘡皆沸【傅音附語牛倨翻中竹仲翻】大驚會天暴風雨隨雨擊之殺傷甚衆匈奴震怖【怖普布翻】相謂曰漢兵神真可畏也遂解去 夏六月己未有星孛於太微【晉天文志太微天子廷也十二諸侯府也孛蒲内翻】 耿恭以疏勒城傍有澗水可固引兵據之【此疏勒城在車師後部非疏勒國城也據西域傳疏勒國去長史所居五千里後部去長史所居五百里耿恭自後部金蒲城移據疏勒城其後范羌又自前部交河城從山北至疏勒迎恭審觀本末則非疏勒國城明矣】秋七月匈奴復來攻【復扶又翻】擁絶澗水恭於城中穿井十五丈不得水吏士渇乏至笮馬糞汁而飲之【賢曰笮謂壓笮也音側駕翻】恭身自率士輓籠【輓音晚師古曰籠所以盛土也音盧紅翻鄭氏周禮注竁土之器曰籠陸德明音力董翻朱熹曰蕢土籠也】有頃水泉奔出衆皆稱萬歲乃令吏士揚水以示虜虜出不意以為神明遂引去 八月壬子帝崩於東宫前殿年四十八遺詔無起寢廟藏主於光烈皇后更衣别室【賢曰禮藏主於廟既不起寢廟故藏於后之易衣别室更易也更工衡翻下同】帝遵奉建武制度無所變更后妃之家不得封侯與政館陶公主為子求郎【館陶公主光武女紅夫也適駙馬都尉韓光與讀曰預為于偽翻】不許而賜錢千萬謂羣臣曰郎官上應列宿【史記曰太微宮後二十五星郎位也宿音秀】出宰百里苟非其人則民受其殃是以難之公車以反支日不受章奏【隂陽書曰凡反支日用月朔為正戍亥朔一日反支申酉朔二日反支午未朔三日反支辰巳朔四日反支寅卯朔五日反支子丑朔六日反支】帝聞而怪曰民廢農桑遠來詣闕而復拘以禁忌【復扶又翻】豈為政之意乎於是遂蠲其制尚書閻章二妹為貴人章精力曉舊典久次當遷重職帝為後宫親屬竟不用是以吏得其人民樂其業【樂音洛】遠近畏服戶口滋殖焉 太子即位年十八尊皇后曰皇太后明帝初崩馬氏兄弟爭欲入宫北宮衛士令楊仁被甲持戟嚴勒門衛人莫敢輕進者【東都南北宫皆有衛士令一人秩六百石各掌其宫衛士漢官曰北宮員吏七十二人衛士四百七十一人朱爵司馬主南掖門員吏四人衛士百二十四人東明司馬主東門員吏十三人衛士百八十人朔平司馬主北門員吏五人衛士百一十七人凡員吏皆隊長佐凡居宮中者皆有口籍於門之所屬宫名兩字為鐵印文符案省符乃内之若外人以事當入本宫長吏為封棨傳其有官位者令御者言其官胡廣曰符用木長可二寸銕印以符之被皮義翻】諸馬乃共譛仁於章帝言其峻刻帝知其忠愈善之拜為什邡令【什邡縣屬廣漢郡此即高帝封雍齒之什方也邡讀曰方】 壬戌葬孝明皇帝于顯節陵【帝王紀曰顯節陵故富夀亭也西北去雒陽三十七里】 冬十月丁未赦天下 詔以行太尉事節鄉侯熹為太傅司空融為太尉並錄尚書事【光武不任三公事歸臺閣惟錄尚書事者權任稍重自是迄于齊梁謂之錄公賢曰武帝初以張子孺領尚書事錄尚書事由此始晉百官志曰漢武時左右曹諸吏分平尚書奏事知樞要者始領尚書事張安世以車騎將軍霍光以大將軍王鳳以大司馬師丹以左將軍並領尚書事後漢章帝以太傅趙熹太尉牟融並錄尚書事尚書有錄名自此始亦西京領尚書之任猶唐虞大麓之職也沈約曰漢東京每帝即位輒置太傅錄尚書事薨輒省】 十一月戊戌以蜀郡太守第五倫為司空倫在郡公清所舉吏多得其人故帝自遠郡用之【續漢志蜀郡在雒陽西三千一百里守式又翻】 焉耆龜兹攻沒都護陳睦北匈奴圍關寵於柳中城會中國有大喪救兵不至車師復叛【復扶又翻下同】與匈奴共攻耿恭恭率厲士衆禦之數月食盡窮困乃煮鎧弩食其筋革【鎧可亥翻】恭與士卒推誠同死生故皆無二心而稍稍死亡餘數十人單于知恭已困欲必降之遣使招恭曰若降者當封為白屋王【按李廵注爾雅五狄有白屋一種降戶江翻】妻以女子【妻七細翻】恭誘其使上城手擊殺之炙諸城上單于大怒更益兵圍恭不能下關寵上書求救詔公卿會議司空倫以為不宜救司徒鮑昱曰今使人於危難之地急而棄之外則縱蠻夷之暴内則傷死難之臣【難乃旦翻】誠令權時後無邊事可也匈奴如復犯塞為寇陛下將何以使將【將即亮翻】又二部兵人裁各數十【賢曰二部謂關寵及恭也】匈奴圍之歷旬不下是其寡弱力盡之效也【力盡猶言盡力也】可令敦煌酒泉太守各將精騎二千多其幡幟倍道兼行以赴其急【幟昌志翻】匈奴疲極之兵必不敢當四十日間足還入塞帝然之乃遣征西將軍耿秉屯酒泉行太守事遣酒泉太守段彭 【考異曰耿恭傳云秦彭今從帝紀】與謁者王蒙皇甫援【姓譜宋有皇甫充石宋之公族也漢初有皇甫鸞自魯徙居茂陵改父為甫余按詩周亦有皇父卿士】發張掖酒泉敦煌三郡及鄯善兵合七千餘人以救之【鄯上扇翻】 甲辰晦日有食之 太后兄弟虎賁中郎廖及黄門郎防光【百官志給事黄門侍郎六百石掌侍從左右漢舊儀曰黄門郎屬黄門令日暮入對青瑣門拜名曰夕郎】終明帝世未嘗改官帝以廖為衛尉防為中郎將光為越騎校尉廖等傾身交結冠蓋之士爭赴趣之【趣七喻翻】第五倫上疏曰臣聞書曰臣無作威作福其害于而家凶于而國【尚書洪範之言】近世光烈皇后雖友愛天至而抑損隂氏不假以權勢【謂隂后不為宗親求位也】其後梁竇之家互有非法明帝即位竟多誅之【謂梁松竇穆等也】自是洛中無復權戚書記請託一皆斷絶【復扶又翻斷丁管翻】又諭諸外戚曰苦身待士不如為國【為于偽翻】戴盆望天事不兩施【司馬遷書曰戴盆何以望天】今之議者復以馬氏為言竊聞衛尉廖以布三千匹城門校尉防以錢三百萬私贍三輔衣冠知與不知莫不畢給又聞臘日亦遺其在雒中者錢各五千【遺于季翻】越騎校尉光臘用羊三百頭米四百斛肉五千斤臣愚以為不應經義惶恐不敢不以聞陛下情欲厚之亦宜所以安之臣今言此誠欲上忠陛下下全后家也 是歲京師及兗豫徐州大旱【兖州部陳留東郡東平泰山濟北山陽濟隂等郡國豫州部汝南潁川二郡梁沛陳魯等國徐州部東海琅邪彭城廣陵下邳等郡國杜佑曰兖州盖以沇水為名又兖之為言端也信也端言陽氣端端故其氣纎殺也徐州盖取舒緩之義或云因徐丘以為名一】<br />
<br />
  資治通鑑卷四十五<br />
<br />
<史部,編年類,資治通鑑>  <br>
   </div> 

<script src="/search/ajaxskft.js"> </script>
 <div class="clear"></div>
<br>
<br>
 <!-- a.d-->

 <!--
<div class="info_share">
</div> 
-->
 <!--info_share--></div>   <!-- end info_content-->
  </div> <!-- end l-->

<div class="r">   <!--r-->



<div class="sidebar"  style="margin-bottom:2px;">

 
<div class="sidebar_title">工具类大全</div>
<div class="sidebar_info">
<strong><a href="http://www.guoxuedashi.com/lsditu/" target="_blank">历史地图</a></strong>  
<a href="http://www.880114.com/" target="_blank">英语宝典</a>  
<a href="http://www.guoxuedashi.com/13jing/" target="_blank">十三经检索</a> 
<br><strong><a href="http://www.guoxuedashi.com/gjtsjc/" target="_blank">古今图书集成</a></strong> 
<a href="http://www.guoxuedashi.com/duilian/" target="_blank">对联大全</a> <strong><a href="http://www.guoxuedashi.com/xiangxingzi/" target="_blank">象形文字典</a></strong> 

<br><a href="http://www.guoxuedashi.com/zixing/yanbian/">字形演变</a>  <strong><a href="http://www.guoxuemi.com/hafo/" target="_blank">哈佛燕京中文善本特藏</a></strong>
<br><strong><a href="http://www.guoxuedashi.com/csfz/" target="_blank">丛书&方志检索器</a></strong> <a href="http://www.guoxuedashi.com/yqjyy/" target="_blank">一切经音义</a>  

<br><strong><a href="http://www.guoxuedashi.com/jiapu/" target="_blank">家谱族谱查询</a></strong>  <strong><a href="http://shufa.guoxuedashi.com/sfzitie/" target="_blank">书法字帖欣赏</a></strong> 
<br>

</div>
</div>


<div class="sidebar" style="margin-bottom:0px;">

<font style="font-size:22px;line-height:32px">QQ交流群9:489193090</font>


<div class="sidebar_title">手机APP 扫描或点击</div>
<div class="sidebar_info">
<table>
<tr>
	<td width=160><a href="http://m.guoxuedashi.com/app/" target="_blank"><img src="/img/gxds-sj.png" width="140"  border="0" alt="国学大师手机版"></a></td>
	<td>
<a href="http://www.guoxuedashi.com/download/" target="_blank">app软件下载专区</a><br>
<a href="http://www.guoxuedashi.com/download/gxds.php" target="_blank">《国学大师》下载</a><br>
<a href="http://www.guoxuedashi.com/download/kxzd.php" target="_blank">《汉字宝典》下载</a><br>
<a href="http://www.guoxuedashi.com/download/scqbd.php" target="_blank">《诗词曲宝典》下载</a><br>
<a href="http://www.guoxuedashi.com/SiKuQuanShu/skqs.php" target="_blank">《四库全书》下载</a><br>
</td>
</tr>
</table>

</div>
</div>


<div class="sidebar2">
<center>


</center>
</div>

<div class="sidebar"  style="margin-bottom:2px;">
<div class="sidebar_title">网站使用教程</div>
<div class="sidebar_info">
<a href="http://www.guoxuedashi.com/help/gjsearch.php" target="_blank">如何在国学大师网下载古籍?</a><br>
<a href="http://www.guoxuedashi.com/zidian/bujian/bjjc.php" target="_blank">如何使用部件查字法快速查字?</a><br>
<a href="http://www.guoxuedashi.com/search/sjc.php" target="_blank">如何在指定的书籍中全文检索?</a><br>
<a href="http://www.guoxuedashi.com/search/skjc.php" target="_blank">如何找到一句话在《四库全书》哪一页?</a><br>
</div>
</div>


<div class="sidebar">
<div class="sidebar_title">热门书籍</div>
<div class="sidebar_info">
<a href="/so.php?sokey=%E8%B5%84%E6%B2%BB%E9%80%9A%E9%89%B4&kt=1">资治通鉴</a> <a href="/24shi/"><strong>二十四史</strong></a>&nbsp; <a href="/a2694/">野史</a>&nbsp; <a href="/SiKuQuanShu/"><strong>四库全书</strong></a>&nbsp;<a href="http://www.guoxuedashi.com/SiKuQuanShu/fanti/">繁体</a>
<br><a href="/so.php?sokey=%E7%BA%A2%E6%A5%BC%E6%A2%A6&kt=1">红楼梦</a> <a href="/a/1858x/">三国演义</a> <a href="/a/1038k/">水浒传</a> <a href="/a/1046t/">西游记</a> <a href="/a/1914o/">封神演义</a>
<br>
<a href="http://www.guoxuedashi.com/so.php?sokeygx=%E4%B8%87%E6%9C%89%E6%96%87%E5%BA%93&submit=&kt=1">万有文库</a> <a href="/a/780t/">古文观止</a> <a href="/a/1024l/">文心雕龙</a> <a href="/a/1704n/">全唐诗</a> <a href="/a/1705h/">全宋词</a>
<br><a href="http://www.guoxuedashi.com/so.php?sokeygx=%E7%99%BE%E8%A1%B2%E6%9C%AC%E4%BA%8C%E5%8D%81%E5%9B%9B%E5%8F%B2&submit=&kt=1"><strong>百衲本二十四史</strong></a>  <a href="http://www.guoxuedashi.com/so.php?sokeygx=%E5%8F%A4%E4%BB%8A%E5%9B%BE%E4%B9%A6%E9%9B%86%E6%88%90&submit=&kt=1"><strong>古今图书集成</strong></a>
<br>

<a href="http://www.guoxuedashi.com/so.php?sokeygx=%E4%B8%9B%E4%B9%A6%E9%9B%86%E6%88%90&submit=&kt=1">丛书集成</a> 
<a href="http://www.guoxuedashi.com/so.php?sokeygx=%E5%9B%9B%E9%83%A8%E4%B8%9B%E5%88%8A&submit=&kt=1"><strong>四部丛刊</strong></a>  
<a href="http://www.guoxuedashi.com/so.php?sokeygx=%E8%AF%B4%E6%96%87%E8%A7%A3%E5%AD%97&submit=&kt=1">說文解字</a> <a href="http://www.guoxuedashi.com/so.php?sokeygx=%E5%85%A8%E4%B8%8A%E5%8F%A4&submit=&kt=1">三国六朝文</a>
<br><a href="http://www.guoxuedashi.com/so.php?sokeytm=%E6%97%A5%E6%9C%AC%E5%86%85%E9%98%81%E6%96%87%E5%BA%93&submit=&kt=1"><strong>日本内阁文库</strong></a> <a href="http://www.guoxuedashi.com/so.php?sokeytm=%E5%9B%BD%E5%9B%BE%E6%96%B9%E5%BF%97%E5%90%88%E9%9B%86&ka=100&submit=">国图方志合集</a> <a href="http://www.guoxuedashi.com/so.php?sokeytm=%E5%90%84%E5%9C%B0%E6%96%B9%E5%BF%97&submit=&kt=1"><strong>各地方志</strong></a>

</div>
</div>


<div class="sidebar2">
<center>

</center>
</div>
<div class="sidebar greenbar">
<div class="sidebar_title green">四库全书</div>
<div class="sidebar_info">

《四库全书》是中国古代最大的丛书,编撰于乾隆年间,由纪昀等360多位高官、学者编撰,3800多人抄写,费时十三年编成。丛书分经、史、子、集四部,故名四库。共有3500多种书,7.9万卷,3.6万册,约8亿字,基本上囊括了古代所有图书,故称“全书”。<a href="http://www.guoxuedashi.com/SiKuQuanShu/">详细>>
</a>

</div> 
</div>

</div>  <!--end r-->

</div>
<!-- 内容区END --> 

<!-- 页脚开始 -->
<div class="shh">

</div>

<div class="w1180" style="margin-top:8px;">
<center><script src="http://www.guoxuedashi.com/img/plus.php?id=3"></script></center>
</div>
<div class="w1180 foot">
<a href="/b/thanks.php">特别致谢</a> | <a href="javascript:window.external.AddFavorite(document.location.href,document.title);">收藏本站</a> | <a href="#">欢迎投稿</a> | <a href="http://www.guoxuedashi.com/forum/">意见建议</a> | <a href="http://www.guoxuemi.com/">国学迷</a> | <a href="http://www.shuowen.net/">说文网</a><script language="javascript" type="text/javascript" src="https://js.users.51.la/17753172.js"></script><br />
  Copyright &copy; 国学大师 古典图书集成 All Rights Reserved.<br>
  
  <span style="font-size:14px">免责声明:本站非营利性站点,以方便网友为主,仅供学习研究。<br>内容由热心网友提供和网上收集,不保留版权。若侵犯了您的权益,来信即刪。scp168@qq.com</span>
  <br />
ICP证:<a href="http://www.beian.miit.gov.cn/" target="_blank">鲁ICP备19060063号</a></div>
<!-- 页脚END --> 
<script src="http://www.guoxuedashi.com/img/plus.php?id=22"></script>
<script src="http://www.guoxuedashi.com/img/tongji.js"></script>

</body>
</html>
