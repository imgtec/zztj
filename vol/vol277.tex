






























































資治通鑑卷二百七十七 宋 司馬光 撰

胡三省 音註

後唐紀【起上章攝提格盡玄黓執徐六月凡二年有奇}


明宗聖德和武欽孝皇帝中之下

長興元年【是年二月方改元}
春正月董璋遣兵築七寨於劍門辛巳孟知祥遣趙季良如梓州修好【先是董璋在東川與孟知祥鄰鎮而未嘗通問天成三年兩鎮因爭鹽利而有違言去年璋遣使求昏於知祥今知祥遣報使以修好兩釋嫌怨以從講解懼朝廷加兵也同舟遇風則胡越相應如左右手斯之謂矣安重誨患兩川之難制不能因其構隙而鬬之反從而合之可以為善謀國乎兵法曰合則能離之安重誨反是好呼到翻下同}
鴻臚少卿郭在徽奏請鑄當五千三千一千大錢朝廷以其指虛為實無識妄言左遷衛尉少卿同正【此唐官所謂員外置同正員者也}
吳徙平原王澈為德化王【江州德化縣本漢尋陽縣宋白曰南唐所改}
二月乙未朔趙季良還成都謂孟知祥曰董公貪殘

好勝志大謀短終為西川之患【史紀趙季良之言為董璋攻孟知祥張本}
都指揮使李仁罕張業欲置宴召知祥先二日有尼告二將謀以宴日害知祥【先悉薦翻}
知祥詰之無狀【無謀害之狀也詰去吉翻}
丁酉推始言者軍校都延昌王行本腰斬之【校戶教翻都姓也春秋時鄭大夫公孫閼字子都子孫以為氏}
戊戌就宴盡去左右【去羌呂翻}
獨詣仁罕第仁罕叩頭流涕曰老兵惟盡死以報德由是諸將皆親附而服之【史言孟知祥能推心以得人死力}
壬子孟知祥董璋同上表言兩川聞朝廷於閬中建節綿遂益兵無不憂恐【閬中建節謂置保寧軍於閬州綿遂益兵謂武䖍裕刺綿州夏魯奇帥遂州皆益兵戍之事並見上卷上年}
上以詔書慰諭之 乙卯上祀圓丘大赦改元【改元長興}
鳳翔節度使兼中書令李從曮入朝陪祀三月壬申制徙從曮為宣武節度使【天成元年李從曮再鎮鳳翔至是徙鎮}
癸酉吳主立江都王璉為太子【璉立展翻}
丙子以宣徽使朱弘昭為鳳翔節度使 康福奏克保靜鎮斬李匡賓【李匡賓據保靜鎮見上卷上年}
復以安義為昭義軍【梁均王龍德二年晉王改昭義軍曰安義軍見二百七十一卷}
帝將立曹淑妃為后淑妃謂王德妃曰吾素病中煩【中煩謂胷中煩熱}
倦於接對妹代我為之德妃曰中宫敵偶至尊誰敢干之庚寅立淑妃為皇后德妃事后恭謹后亦憐之初王德妃因安重誨得進常德之【歐史曰德妃王氏邠州餅家女也有美色號花見羞少賣為梁將劉鄩侍兒鄩卒王氏無所歸是時帝正室夏夫人已卒方求别室有言王氏於安重誨者以告於帝而納之}
帝性儉約及在位久宫中用度稍侈重誨每規諫妃取外庫錦造地衣重誨切諫引劉后為戒【謂莊宗劉皇后也}
妃由是怨之 高從誨遣使奉表詣吳告以墳墓在中國【高季興陜州硤石人也故云然}
恐為唐所討吳兵援之不及謝絶之【高季興請附於吳見二百七十五卷天成二年}
吳遣兵擊之不克 董璋恐綿州刺史武䖍裕窺其所為【按九域志綿州東南至梓州一百三十七里以其逼近故恐為所窺}
夏四月甲午朔表兼行軍司馬囚之府廷【以兼行軍司馬誘之至梓州而囚之府廷東川府廷也}
宣武節度使苻習自恃宿將【苻習本成德將從莊宗戰於河上故自恃為耆宿}
論議多抗安重誨重誨求其過失奏之丁酉詔習以太子太師致仕戊戌加孟知祥兼中書令夏魯奇同平章事 初帝

在真定【莊宗同光二年帝鎮真定}
李從珂與安重誨飲酒爭言從琦毆重誨【毆烏口翻}
重誨走免既醒悔謝重誨終銜之至是重誨用事自皇子從榮從厚皆敬事不暇【不暇謂不敢自暇也}
時從琦為河中節度使同平章事重誨屢短之於帝帝不聽重誨乃矯以帝命諭河東牙内指揮使楊彦温使逐之【河東當作河中}
是日【承上戊戌故曰是日}
從珂出城閲馬彦温勒兵閉門拒之從珂使人扣門詰之曰【詰去吉翻}
吾待汝厚何為如是對曰彦温非敢負恩受樞密院宣耳【樞密院用宣三省用堂帖今堂帖謂之省劄宣謂之密劄}
請公入朝從珂止于虞鄉【九域志虞鄉縣在河中府東六十里}
遣使以狀聞使者至壬寅帝問重誨曰彦温安得此言【謂言受樞密院宣也}
對曰此姦人妄言耳宜速討之帝疑之欲誘致彦温訊其事【訊問也誘音酉}
除彦温絳州刺史重誨固請發兵擊之乃命西都留守索自通【索蘇各翻姓也}
步軍都指揮使藥彦稠將兵討之【藥姓也漢有藥崧按薛史藥彦稠沙陀三部落人必非崧後}
帝令彦稠必生致彦温吾欲面訊之召從珂詣洛陽從珂知為重誨所構馳入自明 加安重誨兼中書令 李從珂至洛陽上責之使歸第絶朝請【薛史曰歸清化里第}
辛亥索自通等拔河中斬楊彦温【承安重誨指斬楊彦温以滅口為潞王殺藥彦稠索自通自投於水張本}
癸丑傳首來獻上怒藥彦稠不生致【不生致楊彦温也}
深責之安重誨諷馮道趙鳳奏從珂失守宜加罪上曰吾兒為姦黨所傾未明曲直公輩何為發此言意不欲置之人間邪此皆非公輩之意也【言二人為安重誨所使}
二人惶恐而退它日趙鳳又言之上不應明日重誨自言之上曰朕昔為小校【校戶教翻}
家貧賴此小兒拾馬糞自贍以至今日為天子曾不能庇之邪卿欲如何處之於卿為便【上亦以此語激安重誨處昌呂翻}
重誨曰陛下父子之間臣何敢言惟陛下裁之上曰使閒居私第亦可矣何用復言【復扶又翻}
丙辰以索自通為河中節度使自通至鎮承重誨指籍軍府甲仗數上之【上時掌翻}
以為從珂私造賴王德妃居中保護從珂由是得免士大夫不敢與從珂往來惟禮部郎中史館脩撰呂琦居相近時往見之從珂每有奏請皆咨琦而後行【從珂居閒奏請咨呂琦而後行及其在位能厚琦而不能用琦何也}
戊午帝加尊號曰聖明神武文德恭孝皇帝 安重誨言昭義節度使王建立過魏州有揺衆之語五月丙寅制以太傅致仕【安重誨王建立交惡見上卷天成三年}
董璋閲集民兵皆剪髪黥面復於劍門北置永定關布列烽火【復扶又翻}
孟知祥累表請割雲安等十三鹽監隸西川【雲安縣漢巴郡胊䏰縣地周武帝改為雲安縣屬巴東郡唐屬夔州後改為雲安監又夔州大昌縣萬州南浦縣漁陽監皆有鹽官隸寧江軍巡屬而所謂十三監未知盡在何所}
以鹽直贍寧江屯兵辛卯許之 六月癸巳朔日有食之 辛亥敕防禦團練使刺史行軍司馬節度副使自今皆朝廷除之諸道無得奏薦 董璋遣兵掠遂閬鎮戍秋七月戊辰兩川以朝廷繼遣兵屯遂閬復有論奏【復扶又翻}
自是東北商旅少敢入蜀【少詩沼翻}
八月乙未捧聖軍使李行德【按五代會要周應順元年改龍武神武四十指揮為捧聖左右軍據此則是時先已有捧聖軍矣宋白曰長興三年改在京龍武神武四十指揮為捧聖左右軍}
十將張儉引告密人邊彦温告安重誨發兵云欲自討淮南【因天成二年安重誨常有伐吳之議遂以是誣告之}
又引占相者問命【相息亮翻}
帝以問侍衛都指揮使安從進藥彦稠二人曰此姦人欲離間陛下勲舊耳【間古莧翻}
重誨事陛下三十年【梁均王貞明二年帝始為安國節度以安重誨為中門使至是纔十六年蓋帝與重誨皆應州人其相從久矣}
幸而富貴何苦謀反臣等請以宗族保之帝乃斬彦温召重誨慰撫之君臣相泣【蓋是時安重誨之跡已危矣}
以前忠武節度使張延朗行工部尚書充三司使三司使之名自此始【自宋熙寧以前三司使位亞執政專制國計權任重矣}
吳徐知誥以海州都指揮使王傳拯有威名得士心值團練使陳宣罷歸知誥許以傳拯代之既而復遣宣還海州徵傳拯還江都傳拯怒以為宣毁之己亥帥麾下入辭宣【帥讀曰率}
因斬宣焚掠城郭帥其衆五千來奔知誥曰是吾過也免其妻子漣水制置使王巖將兵入海州【漣水至海州一百八十里}
以巖為威衛大將軍知海州傳拯綰之子也【吳先以王綰知海州楊隆演之建國也加鎮東大將軍}
其季父輿為光州刺史傳拯遣間使持書至光州【間古莧翻使疏吏翻}
輿執之以聞因求罷歸【以兄子外叛身居邊郡心迹危疑故求罷歸}
知誥以輿為控鶴都虞候時政在徐氏典兵宿衛者尤難其人知誥以輿重厚慎密故用之 壬寅趙鳳奏竊聞近有姦人誣陷大臣揺國柱石行之未盡【言未盡行誅也}
帝乃收李行德張儉皆族之 立皇子從榮為秦王丙辰立從厚為宋王 董璋之子光業為宫苑使在洛陽璋與書曰朝廷割吾支郡為節鎮【謂夏魯奇鎮遂州李仁矩鎮閬州又傳割綿龍也}
屯兵三千是殺我必矣汝見樞要為吾言【樞要謂兩樞密董璋意專指安重誨為于偽翻}
如朝廷更發一騎入斜谷吾必反與汝訣矣【騎奇寄翻斜余遮翻谷音浴}
光業以書示樞密承旨李䖍徽未幾朝廷又遣别將荀咸乂將兵戍閬州【幾居豈翻}
光業謂䖍徽曰此兵未至吾父必反吾不敢自愛【言不敢愛其死也}
恐煩朝廷調發【言恐須用兵調徒釣翻}
願止此兵吾父保無它䖍徽以告安重誨重誨不從璋聞之遂反利閬遂三鎮以聞【利帥李彦琦閬帥李仁矩遂州夏魯奇}
且言已聚兵將攻三鎮重誨曰臣久知其如此陛下含容不討耳帝曰我不負人人負我則討之 九月癸亥西川進奏官蘇愿白孟知祥云朝廷欲大發兵討兩川【進奏官在京師故以其事白其主帥}
知祥謀於副使趙季良季良請以東川兵先取遂閬然後併兵守劍門則大軍雖來吾無内顧之憂矣【兩川同心協力守險則西川無内顧之憂}
知祥從之遣使約董璋同舉兵璋移檄利閬遂三鎮數其離間朝廷【數所具翻間古莧翻下無間同}
引兵擊閬州【九域志梓州東北至閬州三百九里}
庚午知祥以都指揮使李仁罕為行營都部署漢州刺史趙廷隱副之簡州刺史張業為先鋒指揮使將兵三萬攻遂州【九域志遂州北至梓州二百五里}
别將牙内都指揮使侯弘實先登指揮使孟思恭將兵四千會璋攻閬州 安重誨久專大權中外惡之者衆【惡烏路翻}
王德妃及武德使孟漢瓊浸用事數短重誨於上【數所角翻}
重誨内憂懼表解機務上曰朕無間於卿誣罔者朕既誅之矣【謂李行德張儉也}
卿何為爾甲戌重誨復面奏曰【復扶又翻}
臣以寒賤致位至此忽為人誣以反非陛下至明臣無種矣【種章勇翻}
由臣才薄任重恐終不能鎮浮言願賜一鎮以全餘生上不許重誨求之不已上怒曰聽卿去朕不患無人前成德節度使范延光勸上留重誨且曰重誨去誰能代之上曰卿豈不可延光曰臣受驅策日淺且才不逮重誨何敢當此上遣孟漢瓊詣中書議重誨事馮道曰諸公果愛安令【時安重誨兼中書令故稱之}
宜解其樞務為便【馮道肯發此言蓋知之矣}
趙鳳曰公失言乃奏大臣不可輕動 東川兵至閬州諸將皆曰董璋久蓄反謀以金帛㗖其士卒【㗖徒濫翻}
鋭氣不可當宜深溝高壘以挫之不過旬日大軍至賊自走矣李仁矩曰蜀兵懦弱安能當我精卒遂出戰兵未交而潰歸董璋晝夜攻之庚辰城陷殺仁矩滅其族初璋為梁將指揮使姚洪嘗隸麾下至是將兵千人戍閬州璋密以書誘之洪投諸厠【誘音酉}
城陷璋執洪而讓之曰吾自行間奬拔汝【行戶剛翻}
今日何相負洪曰老賊汝昔為李氏奴【董璋先為汴富人李讓家僮}
掃馬糞得臠炙感恩無窮【臠力兖翻肉作片也炙之夜翻燔肉也}
今天子用汝為節度使何負於汝而反邪汝猶負天子吾受汝何恩而云相負哉汝奴才固無恥吾義士豈忍為汝所為乎吾寧為天子死【寧為于偽翻}
不能與人奴並生璋怒然鑊於前【鑊戶郭翻鼎大無足曰鑊然燒也}
令壯士十人刲其肉自啗之【刲涓畦翻割也}
洪至死罵不絶聲帝置洪二子於近衛厚給其家 甲申以范延光為樞密使安重誨如故【言雖進用范延光而安重誨職任如故}
丙戌下制削董璋官爵興兵討之丁亥以孟知祥兼西南供饋使【孟知祥之兵已攻遂州朝廷豈不知之邪猶欲懷輯之以離董璋之交耳唇亡齒寒已了了於知祥胸中此策安所施哉}
以天雄節度石敬瑭為東川行營都招討使【節度之下當有使字蜀本有使字}
以夏魯奇為之副璋使孟思恭分兵攻集州【集州本漢宕渠縣宇文周置集州隋廢為難江縣唐復置集州宋熙寧五年復廢州為難江縣屬巴州九域志縣在州北一百六十里}
思恭輕進敗歸璋怒遣還成都知祥免其官戊子以石敬瑭權知東川事庚寅以右武衛上將軍王思同為西都留守兼行營馬步都虞候為伐蜀前鋒 漢主遣其將梁克貞李守鄘攻交州拔之執靜海節度使曲承美以歸【唐末曲顥據交州至承美而敗}
以其將李進守交州 冬十月癸巳李仁罕圍遂州夏魯奇嬰城固守孟知祥命都押牙高敬柔帥資州義軍二萬人築長城環之【帥讀曰率環音宦}
魯奇遣馬軍都指揮使康文通出戰文通聞閬州陷遂以其衆降於仁罕戊戌董璋引兵趣利州【九域志閬州西北至利州二百四十里趣七喻翻下同}
遇雨糧運不繼還閬州知祥聞之驚曰比破閬中【比毗至翻}
正欲徑取利州其帥不武必望風自遁【利帥李彦琦帥所類翻}
吾獲其倉廩據漫天之險【漫天寨在利州北有小漫天大漫天二寨}
北軍終不能西救武信【武信軍遂州}
今董公僻處閬州遠棄劍閣非計也【處昌召翻}
欲遣兵三千助守劍門璋固辭曰此已有備【為劍門失守張本}
錢鏐因朝廷冊閩王使者裴羽還【裴羽蓋冊閩王延鈞者也還從宣翻又如字}
附表引咎其子傳瓘及將佐屢為鏐上表自訢【為于偽翻}
癸卯敕聽兩浙綱使自便【繫治兩浙綱使見上卷上年}
以宣徽北院使馮贇為左衛上將軍北都留守 丁未族誅董光業【以其父璋反也}
楚王殷寢疾遣使詣闕請傳位於其子希聲朝廷疑殷已死辛亥以希聲為起復武安節度使兼侍中 孟知祥以故蜀鎮江節度使張武為峽路行營招收討伐使將水軍趣夔州【前蜀置鎮江軍於夔州張武其舊帥也趣七喻翻}
以左飛棹指揮使袁彦超副之【天成元年孟知祥置左右飛棹六營}
癸丑東川兵陷徵合巴蓬果五州【徧考隋唐地理志五代職方考元豐九域志皆無徵州按東川之兵時自遂閬東畧九域志合州在遂州東二百二十里果州在遂州東南一百八十里巴州在閬州東二百四十五里蓬州在果州東北一百八十五里徵州必在遂合果三州之間}
丙辰吳左僕射同平章事嚴可求卒【嚴可求忠於徐氏者也徐温既卒可求相吳坐視徐知詢之廢不能出一計權不在焉故也}
徐知誥以其長子大將軍景通為兵部尚書參政事知誥將出鎮金陵故也 漢將梁克貞入占城取其寶貨以歸【占城國在西南海上其地方千里東至海西至雲南南鄰真臘北抵驩州其人俗與大食同其乘象馬其食稻米}
十一月戊辰張武至渝州刺史張環降之遂取瀘州【九域志渝瀘二州相去七百餘里降戶江翻瀘音盧}
遣先鋒將朱偓分兵趣黔涪【九域志涪州西至渝州三百四十里東南至黔州四百九十里將即亮翻趣七喻翻黔其今翻涪音浮}
己巳楚王殷卒【年七十九}
遺命諸子兄弟相繼置劍於祠堂曰違吾命者戮之【為殷諸子爭國以至于亡張本}
諸將議遣兵守四境然後發喪兵部侍郎黄損曰吾喪君有君【用左傳語吾喪息浪翻}
何備之有宜遣使詣鄰道告終稱嗣而已 石敬瑭入散關階州刺史王弘贄瀘州刺史馮暉與前鋒馬步都虞候王思同步軍都指揮使趙在禮引兵出人頭山後過劍門之南還襲劍門克之殺東川兵三千人獲都指揮使齊彦温據而守之暉魏州人也甲戍弘贄等破劍州而大軍不繼乃焚其廬舍取其資糧還保劍門【今利州昭化縣南有白衛嶺與劍門相接九域志劍州東北至劍門五十五里 考異曰實録軍前奏今月十三日王弘贄馮暉自利州入山路出劍門關外倒下殺董璋把關兵士約三千人獲都指揮使齊彦温大軍進攻入劍門次又丙戌奏今月十七日收下劍州破賊千餘人獲指揮使劉太李昊蜀高祖實録己卯東川告急今月十八日北軍自白衛嶺人頭山後過從小劍路至漢源驛出頭倒入劍門打破關寨掩捉彦温及將士五百餘人遂相次構喚大軍據關下營又龐福誠謝鍠相謂曰北軍昨來既得關寨之後隔一日大軍曾下至劍州而乃搬運糧食燒舍自驚還奔關寨十國紀年後蜀史壬申弘贄暉襲陷劍門癸酉攻焚劍州取糧還屯劍門己卯東川告急使至成都知祥命牙内都指揮使李肇帥兵五千赴援董璋自閬州帥兩川兵屯木馬寨先是龎福誠謝鍠屯閬州北來蘇寨聞劍門陷懼北軍據劍州帥部兵千餘人由間道先董璋至劍州壁于衙城後士卒方食北軍萬餘人自北山馳下福誠等趨河橋迎擊之北軍小却福誠帥數百人夜升北山顛轉至北軍壁外大呼譟鍠命將士以弓弩短兵急擊之北軍驚擾棄戈甲而遁鍠追襲之北軍退保劍門十餘日不窺劍州按劍門至成都尚十許程若十八日劍門失守何得二十日孟知祥已聞之邪今從實録十三日壬申為定若隔一日下至劍州則十五日甲戌非十七日也蓋思同等以大軍未至故收糧燒舍還保劍門故福誠等得復入劍州李昊叙事甚詳無執劉太事今刪之晉高祖實録云甲申平劍州破賊千餘人尤誤也}
乙亥詔削孟知祥官爵己卯董璋遣使至成都告急知祥聞劍門失守大懼曰董公果誤我庚辰遣牙内都指揮使李肇將兵五千赴之戒之曰爾倍道兼行先據劍州北軍無能為也又遣使詣遂州令趙廷隱將萬人會屯劍州【時趙廷隱與李仁罕圍遂州孟知祥知夏魯奇無能為而劍閣之險不可不爭故使趙廷隱赴之}
又遣故蜀永平節度使李筠將兵四千趣龍州守要害【防唐兵由鄧艾故道而入蜀也史言孟知祥慮患之周}
時天寒士卒恐懼觀望不進廷隱流涕諭之曰今北軍勢盛汝曹不力戰却敵則妻子皆為人有矣衆心乃奮【蜀兵皆亡國之餘王衍之亡也蜀人妻子係虜者多矣趙廷隱以其所經見實利害告之夫安得而不奮}
董璋自閬州將兩川兵屯木馬寨【木馬寨在閬州西北劍州東南宋白曰梁大同中於巴嶺側近立東巴州治木馬按木馬地名在今洋州界無復遺址}
先是西川牙内指揮使太谷龎福誠昭信指揮使謝鍠屯來蘇村【益昌江東越大山數重有狹徑名來蘇蜀人于江西置柵守之度江出劍門南二十里至青強店與官路合九域志蓬州儀隴縣有來蘇鎮即其地鍠戶盲翻}
聞劍門失守相謂曰使北軍更得劍州則二蜀勢危矣遽引部兵千餘人間道趣劍州始至【間古莧翻趣七喻翻下同}
官軍萬餘人自北山大下會日暮二人謀曰衆寡不敵逮明則吾屬無遺矣福誠夜引兵數百升北山大譟於官軍營後鍠帥餘衆操短兵自其前急擊之【帥讀曰率操七刀翻}
官軍大驚空營遁去復保劍門十餘日不出孟知祥聞之喜曰吾始謂弘贄等克劍門徑據劍州堅守其城或引兵直趣梓州董公必棄閬州奔還我軍失援亦須解遂州之圍如此則内外受敵兩川震動勢可憂危今乃焚毁劍州運糧東歸劍門頓兵不進吾事濟矣【孟知祥喜兵勢之小寛自言其料敵方畧此如棊工之說棊耳}
官軍分道趣文州將襲龍州【自文州界青塘嶺至龍州一百五十里郡志云自北至南者右肩不得易所負謂之左擔路鄧艾伐蜀所由之路也}
為西川定遠指揮使潘福超義勝都頭太原沙延祚所敗【姓苑云沙姓神農夙沙氏之後此傅會之說耳}
甲申張武卒於渝州知祥命袁彦超代將其兵朱偓將至涪州武泰節度使楊漢賓棄黔南奔忠州【九域志黔州北至忠州三百七十九里}
偓追至豐都【舊唐書地理志曰豐都漢巴郡枳縣地後漢置平都縣隋義寧二年分臨江置豐都縣九域志豐都縣在忠州西九十二里}
還取涪州【九域志忠州豐都縣西至涪州百許里涪音浮}
知祥以成都支使崔善權武泰留後董璋遣前陵州刺史王暉將兵三千會李肇等分屯劍州南山丙戌馬希聲襲位稱遺命去建國之制【楚王建國見上卷天成二}


【年去羌呂翻}
復藩鎮之舊 契丹東丹王托雲自以失職【突欲不得立見二百七十五卷天成元年}
帥部曲四十人越海自登州來奔【九域志登州東北至海五里新唐志登州東北海行過大謝島龜歆島淤島烏湖島三百里北度烏湖海至馬石山東之都里鎮二百里東傍海壖過青泥浦桃花浦杏花浦石人汪槖馳彎烏骨江八百里乃南傍海壖過烏牧島貝江口椒島得新羅西北之長口鎮又過秦王石橋麻田島古寺島得物島千里至鴨渌江唐恩浦口乃東南陸行七百里至新羅王城自鴨渌江口舟行百餘里乃小舫沂流東北三十里至泊灼口得渤海之境又泝流五百里至丸都縣城故高麗王都又東北沂流五百里至神州又陸行四百里至顯州天寶中王所都又正北如東六百里至勃海王城按契丹東丹王居扶餘城在唐高麗扶餘川中 考異曰實録按巴堅妻令元帥太子往勃海代慕華歸西樓欲立為契丹王而元帥太子既典兵柄不欲之勃海遂自立為契丹王謀害慕華其母不能止慕華懼遂航海内附按天皇王入汴猶求害東丹者誅之豈有在國欲殺之理今不取}
十二月壬辰石敬瑭至劍門乙未進屯劍州北山趙廷隱陳于牙城後山【郭忱劍州靜照堂記曰前瞰巨澗後倚層巒又春風樓記曰邊山而立是州一逕坡陁中貫大溪太守之居已在平山内外居民悉在山上則劍州之山川可知矣陳讀曰陣下同}
李肇王暉陳于河橋【按劍州無所謂河路振九國志曰王師陷劍門趙廷隱帥兵據石橋恐當作石橋}
敬瑭引步兵進擊廷隱廷隱擇善射者五百人伏敬瑭歸路按甲待之矛矟欲相及乃揚旗鼓譟擊之北軍退走顛墜下山俘斬百餘人敬瑭又使騎兵衝河橋李肇以強弩射之【射而亦翻}
騎兵不能進薄暮敬瑭引去廷隱引兵躡之與伏兵合擊敗之【敗補邁翻}
敬瑭還屯劍門 癸卯夔州奏復取開州【舊唐書地理志曰開州亦漢巴郡胊䏰縣地梁置永豐縣西魏改曰永寧隋開皇末改曰盛山縣唐武德初置開州時蓋為蜀兵所陷而復取之也}
庚戌以武安節度使馬希聲為武安靜江節度使加兼中書令 石敬瑭征蜀未有功使者自軍前來多言道險狹進兵甚難關右之人疲於轉餉往往竄匿山谷聚為盜賊上憂之壬子謂近臣曰誰能辦吾事者吾當自行耳安重誨曰臣職忝機密軍威不振臣之罪也臣請自往督戰上許之重誨即拜辭癸丑遂行日馳數百里西方藩鎮聞之無不惶駭【陜州保義軍華州鎮國軍同州匡國軍耀州順義軍鳳翔山南西道皆西方藩鎮也}
錢帛芻糧晝夜輦運赴利州人畜斃踣於山谷者不可勝紀【踣蒲北翻勝音升}
時上已疎重誨石敬瑭本不欲西征及重誨離上側【離力智翻}
乃敢累表奏論以為蜀不可伐上頗然之 西川兵先戍夔州者千五百人上悉縱歸二年春正月壬戌孟知祥奉表謝【表謝遣還戍兵而已遂劍之兵未嘗解也}
庚午李仁罕陷遂州夏魯奇自殺 癸酉石敬瑭復

引兵至劍州【復扶又翻下同}
屯于北山孟知祥梟夏魯奇首以示之【梟堅堯翻}
魯奇二子從敬瑭在軍中泣請往取其首葬之敬瑭曰知祥長者必葬而父【長知兩翻而汝也}
豈不愈於身首異處乎【言知祥若收葬之則身首猶合於一處若取葬其首而身在敵中必異處也}
既而知祥果收葬之敬瑭與趙廷隱戰不利復還劍門 丙戌加高從誨兼中書令 東川歸合州于武信軍【合州本武信巡屬東川先取合州今西川取遂州故歸之武信}
初鳳翔節度使朱弘昭諂事安重誨連得大鎮重誨過鳳翔弘昭迎拜馬首館於府舍【館古玩翻}
延入寢室妻子羅拜奉進酒食禮甚謹重誨為弘昭泣言讒人交構幾不免賴主上明察得保宗族【為于偽翻讒人謂李行德張儉等事見上年}
重誨既去弘昭即奏重誨怨望有惡言不可令至行營恐奪石敬瑭兵柄又遺敬瑭書言重誨舉措孟浪【遺唯季翻孟浪猶言張大而無拘束也}
若至軍前恐將士疑駭不戰自潰宜逆止之敬瑭大懼即上言重誨至恐人情有變宜急徵還宣徽使孟漢瓊自西方還亦言重誨過惡有詔召重誨還二月己丑朔石敬瑭以遂閬既陷糧運不繼燒營北歸軍前以告孟知祥【軍前謂趙廷隱李肇之軍}
知祥匿其書謂趙季良曰北軍漸進奈何季良曰不過綿州必遁知祥問其故曰我逸彼勞彼懸軍千里糧盡能無遁乎【史言懸軍涉險糧道不繼為敵人所窺}
知祥大笑以書示之安重誨至三泉得詔亟歸過鳳翔朱弘昭不内重誨

懼馳騎而東 兩川兵追石敬瑭至利州【劍州北至利州二百三十里}
壬辰昭武節度使李彦琦棄城走甲午兩川兵入利州孟知祥以趙廷隱為昭武留後【孟知祥遂得據漫天之險如其宿規矣}
廷隱遣使密言於知祥曰董璋多詐可與同憂不可與共樂它日必為公患因其至劍州勞軍請圖之【樂音洛勞力到翻}
并兩川之衆可以得志於天下知祥不許【趙廷隱所以能拒石敬瑭者依險而戰也平原易地烏能當北兵就使殺董璋并兩川之衆亦不能得志於天下孟知祥之不許蓋審已量彼也}
璋入廷隱營留宿而去廷隱歎曰不從吾謀禍難未已【難乃旦翻}
庚子孟知祥以武信留後李仁罕【孟知祥得遂閬二鎮就以與其將故李仁罕趙廷隱各竭其力}
為峽路行營招討使使將水軍東畧地 辛丑以樞密使兼中書令安重誨為護國節度使【安重誨還未至京師而除河中不容其入朝也}
趙鳳言於上曰重誨陛下家臣其心終不叛主但以不能周防為人所讒陛下不察其心死無日矣上以為朋黨不悦【考趙鳳前後所言誠有黨安重誨之心明宗已累見其情而趙鳳言之不已乃所以速其死也}
乙巳趙廷隱李肇自劍州引還【引還成都}
留兵五千戍利州丙午董璋亦還東川留兵三千戍果閬【果閬二州名}
丁巳李仁罕陷忠州 吳徐知誥欲以中書侍郎内樞使宋齊丘為 【内樞使即内樞密使之職}
齊丘自以資望素淺欲以退讓為高謁歸洪州葬父【宋齊丘本洪州進士}
因入九華山【九華山在池州青陽縣界本名九子山李白以九峯如蓮花改為九華}
止于應天寺啟求隱居吳主下詔徵之知誥亦以書招之皆不至知誥遣其子景通自入山敦諭齊丘始還朝【究觀宋齊丘晩年之心迹則始焉之所為者皆偽也朝直遥翻}
除右僕射致仕更命應天寺曰徵賢寺【更工衡翻}
三月己未朔李仁罕陷萬州庚申陷雲安監【九域志萬州在忠州東北二百八十六里雲安軍又在萬州東北二百五十七里監又在軍東北三十里其地產鹽故置監}
辛酉賜契丹東丹王托雲姓東丹名慕華以為懷化節度使瑞慎等州觀察使【時置懷化軍於慎州瑞州領遠來一縣慎州領逢龍一縣蓋皆後唐所置薛史瑞慎二州本遼東之地唐末為懷化節度余按唐貞觀十年以烏突汗達于部落置威州於營州之境後更名瑞州僑治良鄉之廣陽城武德初以速末烏素固部落置慎州僑治良鄉之故都鄉城}
其部曲及先所俘契丹將特哩衮賜姓名特哩衮姓狄名懷忠【擒特哩衮見上天成三年}
李仁罕至夔州寧江節度使安崇阮棄鎮與楊漢賓自均房逃歸壬戌仁罕陷夔州【孟知祥遂并有夔忠萬三州}
帝既解安重誨樞務乃召李從珂泣謂曰如重誨意汝安得復見吾【安重誨欲殺從珂事見上元年}
丙寅以從珂為左衛大將軍 壬申横海節度使同平章事孔循卒 乙酉復以錢鏐為天下兵馬都元帥尚父吳越國王遣監門上將軍張籛往諭旨以曏日致仕安重誨矯制也【籛則前翻錢鏐致仕事見上卷天成四年}
丁亥以太常卿李愚為中書侍郎同平章事 夏四月辛卯以王德妃為淑妃【唐制因隋之舊貴妃淑妃賢妃各一人正一品時曹后自淑妃正位中宫故陞德妃為淑妃}
閩奉國節度使兼中書令王延稟聞閩王延鈞有疾以次子繼昇知建州留後帥建州刺史繼雄將水軍襲福州【帥讀曰率}
癸卯延稟攻西門繼雄攻東門延鈞遣樓船指揮使王仁達將水軍拒之仁達伏甲舟中偽立白幟請降繼雄喜屏左右【幟昌志翻降戶江翻屏必郢翻又卑正翻}
登仁達舟慰撫之仁達斬繼雄梟首於西門延稟方縱火攻城見之慟哭仁達因縱兵擊之衆潰左右以斛舁延稟而走【斛槩量之器十斗為斛舁音余又羊茹翻}
甲辰追擒之延鈞見之曰果煩老兄再下【王延稟此語見二百七十五卷天成二年}
延稟慙不能對延鈞囚于别室遣使者如建州招撫其黨其黨殺使者奉繼昇及弟繼倫奔吳越仁達延鈞從子也【為延鈞忌仁達而殺之張本從才用翻}
以宣徽北院使趙延壽為樞密使 己酉天雄節度使同平章事石敬瑭兼六軍諸衛副使辛亥以朱弘昭為宣徽南院使 五月閩王延鈞斬王延稟於市復其姓名曰周彦琛遣其弟都教練使延政如建州撫慰吏民【為王延政以建州與福州相攻張本}
丁卯罷畝税麴錢【計畝税麴錢見上卷天成三年}
城中官造麴減舊半價鄉村聽百姓自造民甚便之 己卯以孟漢瓊知内侍省事充宣徽北院使漢瓊本趙王鎔奴也時范延光趙延壽雖為樞密使懲安重誨以剛愎得罪【愎蒲逼翻}
每於政事不敢可否獨漢瓊與王淑妃居中用事人皆憚之先是宫中須索稍踰常度重誨輒執奏由是非分之求殆絶【先悉薦翻須求也索亦求也索山客翻分扶問翻}
至是漢瓊直以中宫之命取府庫物不復關由樞密院及三司亦無文書所取不可勝紀【勝音升}
辛巳以相州刺史孟鵠為左驍衛大將軍充三司使昭武留後趙廷隱自成都赴利州踰月請兵進取興

元及秦鳳孟知祥以兵疲民困不許【孟知祥量力而後動所以能跨有三蜀也}
護國節度使兼中書令安重誨内不自安表請致仕閏月庚寅制以太子太師致仕是日其子崇贊崇緒逃奔河中壬辰以保義節度使李從璋為護國節度使甲午遣步軍指揮使藥彦稠將兵趣河中【揺于讒口遣藥彦稠以兵討安重誨}
安崇贊等至河中重誨驚曰汝安得來既而曰吾知之矣此非渠意為人所使耳【渠猶言其也}
吾以死徇國夫復何言【夫音扶復扶又翻}
乃執二子表送詣闕明日有中使至見重誨慟哭久之重誨問其故中使曰人言令公有異志朝廷已遣藥彦稠將兵至矣重誨曰吾受國恩死不足報敢有異志更煩國家發兵貽主上之憂罪益重矣崇贊等至陜有詔繫獄皇城使翟光鄴素惡重誨【惡烏路翻}
帝遣詣河中察之曰重誨果有異志則誅之【史言帝無决然殺重誨之旨郭崇韜之死亦猶是也上無道揆下無法守無怪乎爾}
光鄴至河中李從璋以甲士圍其第自入見重誨拜于庭下重誨驚降堦答拜從璋奮檛擊其首妻張氏驚救亦檛殺之 【考異曰五代史闕文李從璋奮檛擊重誨于地重誨曰重誨死無恨但不與官家誅得潞王他日必為朝廷之患言終而絶按重誨自以私憾欲殺從珂當是時從珂未有跋扈之跡重誨何以知其為朝廷之患此恐是清泰篡立之後人譽重誨者造此語未可信也}
奏至己亥下詔以重誨離間孟知祥董璋錢鏐為重誨罪【間古莧翻離間事並見上}
又誣其欲自擊淮南以圖兵柄【因邊彦温所告而誣之}
遣元隨竊二子歸本道并二子誅之丙午帝遣西川進奏官蘇愿東川軍將劉澄各還本鎮諭以安重誨專命興兵致討今已伏辜 六月乙丑復以李從珂同平章事充西都留守【安重誨既死復用李從珂守長安}
丙子命諸道均民田税 閩王延鈞好神仙之術道士陳守元巫者徐彦林與盛韜共誘之作寶皇宫極土木之盛【薛史福州城中有王霸壇鍊丹井壇旁有皁莢木久枯一旦忽生枝葉井中有白龜浮出掘地得石銘有王霸裔孫之文昶以為已應之于壇側建寶皇宫好呼到翻}
以守元為宫主【陳守元盛韜等見信而薛文傑得行其姦妄矣史言閩政自是愈亂}
秋九月己亥更賜東丹慕華姓名曰李贊華【是年三月慕華賜名今更賜姓}
吳鎮南節度使同平章事徐知諫卒以諸道副都統鎮海節度使守中書令徐知詢代之賜爵東海郡王徐知誥之召知詢入朝也【事見上卷天成四年}
知諫豫其謀知詢遇其喪於塗【知諫之喪自洪州還而知詢往赴洪州故相遇于塗}
撫棺泣曰弟用心如此我亦無憾然何面見先王於地下乎【先王謂徐温也}
辛丑加樞密使范延光同平章事 辛亥敕解縱五坊鷹隼【隼聳尹翻}
内外無得更進馮道曰陛下可謂仁及禽獸上曰不然朕昔嘗從武皇獵【武皇晉王克用諡}
時秋稼方熟有獸逸入田中遣騎取之比及得獸餘稼無幾【比必利翻幾居豈翻}
以是思之獵有損無益故不為耳 冬十月丁卯洋州指揮使李進唐攻通州拔之【洋州東南至通州七百三十九里先是蜀人蓋嘗取通州故復攻抜之宋乾德二年改通州為達州以淮南有通州也}
壬午以王延政為建州刺史 十一月甲申朔日有食之 癸巳蘇愿至成都孟知祥聞甥妷在朝廷者皆無恙【恙余亮翻}
遣使告董璋欲與之俱上表謝罪璋怒曰孟公親戚皆完固宜歸附璋已族滅【謂朝廷族誅其子董光業也}
尚何謝為詔書皆在蘇愿腹中劉澄安得豫聞璋豈不知邪由是復為怨敵【為董璋攻西川敗死張本復扶又翻}
乙未李仁罕自夔州引兵還成都【孟知祥既盡得前蜀夔黔之土不復東畧}
吳中書令徐知誥表稱輔政歲久請歸老金陵乃以知誥為鎮海寧國節度使鎮金陵餘官如故總録朝政如徐温故事【徐温先鎮京口總録吳朝之政後徙金陵朝直遥翻}
以其子兵部尚書參政事景通為司徒同平章事知中外左右諸軍事留江都輔政【徐知誥襲徐温之跡而景通襲知誥之跡吳祚因此移於李氏}
以内樞使同平章事王令謀為左僕射兼門下侍郎以宋齊丘為右僕射兼中書侍郎並同平章事兼内樞使以佐景通賜德勝節度使張崇爵清河王【吳置德勝軍於廬州}
崇在廬州貪暴州人苦之屢嘗入朝厚以貨結權要由是常得還鎮為廬州患者二十餘年 十二月甲寅朔初聽百姓自鑄農器并雜鐵器【按五代會要雜鐵器謂燒器動使諸物熟鐵亦任百姓自煉徐無黨曰税農具錢至今因之}
每田二畝夏秋輸農具三錢 武安靜江節度使馬希聲聞梁太祖嗜食雞慕之既襲位日殺五十雞為膳居喪無戚容庚申葬武穆王于衡陽【馬殷諡武穆王衡陽本漢蒸陽縣吳分置臨蒸縣隋改臨蒸縣為衡陽縣唐屬衡州為治所}
將發引頓食雞數盤【引讀曰靷黑角翻羮也}
前吏部侍郎潘起譏之曰昔阮籍居喪食蒸豚【晉阮籍任情不羈而性至孝母終將葬食一蒸豚飲二斗酒然後臨决直言窮矣舉聲一號吐血數升毁瘠骨立殆至滅性然不可以訓也}
何代無賢 癸亥徐知誥至金陵昭武留後趙廷隱白孟知祥以利州城塹已完頃在

劍州與牙内都指揮使李肇同功【事見上年十一月}
願以昭武讓肇知祥褒諭不許廷隱三讓癸酉知祥召廷隱還成都以肇代之 閩陳守元等稱寶皇之命謂閩王延鈞曰苟能避位受道當為天子六十年延鈞信之丙子命其子節度副使繼鵬權軍府事延鈞避位受籙道名玄錫 愛州將楊廷藝養假子三千人圖復交州漢交州守將李進知之受其賂不以聞是歲廷藝舉兵圍交州【舊唐書地理志交州西至愛州界小黄江口水路四百一十六里}
漢主遣承旨程寶救之未至城陷進逃歸漢主殺之寶圍交州廷藝出戰寶敗死【去年漢取交州今復失之}


三年春正月樞密使范延光言自靈州至邠州方渠鎮【宋白曰通遠軍本靈州方渠鎮晉天福四年改為威州仍割木波馬嶺二鎮隸之周廣順二年避諱改為環州顯德四年降為通遠軍}
使臣及外國入貢者多為党項所掠請發兵擊之己丑遣靜難節度使藥彦稠前朔方節度使康福將步騎七千討党項【党底朗翻}
乙未孟知祥妻福慶長公主卒【歐史長興元年秋改封知祥妻瓊華長公主為福慶長公主長知兩翻}
孟知祥以朝廷恩意優厚而董璋塞綿州路不聽遣使入謝【由成都趣劍利路由綿州塞悉則翻}
與節度副使趙季良等謀欲發使自峽江上表【上時掌翻}
掌書記李昊曰公不與東川謀而獨遣使則異日負約之責在我矣乃復遣使語之【復扶又翻語牛倨翻}
璋不從二月趙季良與諸將議遣昭武都監太原高彦儔將兵攻取壁州【舊唐書地理志壁州諾水縣後漢之宣漢縣梁分宣漢置始寧縣元魏分始寧置諾水縣唐武德初分巴東之始寧置壁州治諾水宋廢壁州以其地入巴州之曾口通江二縣}
以絶山南兵轉入山後諸州者【山後諸州謂巴蓬果等州}
孟知祥謀於僚佐李昊曰朝廷遣蘇愿等西歸未嘗報謝今遣兵侵軼【軼徒結翻}
公若不顧墳墓甥妷【孟知祥之先墳墓在邢州龍岡縣其甥妷時皆仕于朝}
則不若傳檄舉兵直取梁洋安用壁州乎知祥乃止季良由是惡昊【惡烏路翻}
辛未初令國子監校定九經雕印賣之【印賣九經始此}
藥彦稠等奏破党項十九族俘二千七百人 賜高從誨爵勃海王 吳徐知誥作禮賢院於府舍【作之於金陵府舍}
聚圖書延士大夫與孫晟及海陵陳覺談議時事 孟知祥三遣使說董璋【說式芮翻}
以主上加禮於兩川苟不奉表謝罪恐復致討【復扶又翻}
璋不從三月辛丑遣李昊詣梓州極論利害璋見昊詬怒不許【詬古候翻又許候翻}
昊還【還從宣翻又如字}
言於知祥曰璋不通謀議【不通謀議猶今人言不容商量也}
且有窺西川之志公宜備之 甲辰閩王延鈞復位【王延鈞避位受籙見上年}
吳越武肅王錢鏐疾謂將吏曰吾疾必不起諸兒皆愚懦誰可為帥者衆泣曰兩鎮令公仁孝有功孰不愛戴【天成三年錢鏐以兩鎮授傳瓘事見上卷按是年秋朝廷始加傳瓘中書令其下過呼之耳}
鏐乃悉出印鑰授傳瓘【印吳越國印及鎮海鎮東印也鑰内外城諸門及宫門契鑰也}
曰將吏推爾宜善守之又曰子孫善事中國勿以易姓廢事大之禮【時中國率數年一易姓錢鏐之意蓋謂偏據一隅知以小事大而已苟中國有主則臣事之其自興自仆吾不問也}
庚戌卒年八十一傳瓘與兄弟同幄行喪内牙指揮使陸仁章曰令公嗣先王霸業將吏旦暮趨謁當與諸公子異處【處昌呂翻}
乃命主者更設一幄扶傳瓘居之告將吏曰自今惟謁令公禁諸公子從者無得妄入【從才用翻}
晝夜警衛未嘗休息【陸仁章雖不學而其所為闇與趙熹合}
鏐末年左右皆附傳瓘獨仁章數以事犯之至是傳瓘勞之【數所角翻勞力到翻}
仁章曰先王在位仁章不知事令公今日盡節猶事先王也【先王謂鏐}
傳瓘嘉歎久之傳瓘既襲位更名元瓘兄弟名傳者皆更為元【更工衡翻}
以遺命去國儀【吳越建國見二百七十一卷梁均王龍德三年}
用藩鎮法除民田荒絶者租税【荒者有主而不耕絶者戶絶而無主}
命處州刺史曹仲達權知政事置擇能院掌選舉殿最【殿丁練翻}
以浙西營田副使沈崧領之内牙指揮使富陽劉仁【富陽縣本漢富春縣晉避鄭太后諱改名富陽後世遂因之九域志富陽縣屬杭州在州西南七十三里}
及陸仁章久用事仁章性剛仁好毁短人皆為衆所惡【惡烏路翻}
一日諸將共詣府門請誅之元瓘使從子仁俊諭之【從才用翻}
曰二將事先王久吾方圖其功汝曹乃欲逞私憾而殺之可乎吾為汝王汝當禀吾命不然吾當歸臨安以避賢路【錢氏本居臨安}
衆懼而退乃以仁章為衢州刺史仁為湖州刺史中外有上書告訐者【訐居謁翻}
元瓘皆置不問由是將吏輯睦初契丹錫里策稜特哩衮皆為趙德鈞所擒【錫里}


【特哩衮契丹管軍頭目之稱被擒見上卷天成三年}
契丹屢遣使請之上謀於羣臣德鈞等皆曰契丹所以數年不犯邊數求和者以此輩在南故也縱之則邊患復生【數所角翻復扶又翻}
上以問冀州刺史楊檀對曰策稜契丹之驍將曏助王都謀危社稷幸而擒之陛下免其死為賜已多契丹失之如喪手足【喪息浪翻}
彼在朝廷數年知中國虛實若得歸為患必深彼纔出塞則南向發矢矣恐悔之無及上乃止檀沙陀人也【楊檀後改名光遠}
上欲授李贊華以河南藩鎮羣臣皆以為不可上曰吾與其父約為昆弟故贊華歸我吾老矣後世繼體之君雖欲招之其可致乎夏四月癸亥以贊華為義成節度使為選朝士為僚屬輔之【為選于偽翻}
贊華但優游自奉不豫政事上嘉之雖時有不法亦不問以莊宗後宫夏氏妻之【五代會要莊宗昭容夏氏封虢國夫人薛史曰明宗入洛莊宗宫人數百悉令歸其骨肉惟夏氏無所歸明宗以夏魯奇是其同宗因命歸之今以妻贊華妻七細翻}
贊華好飲人血姬妾多刺臂以吮之婢僕小過或抉目或刀刲火灼【好呼到翻刺七亦翻吮士兖翻抉於穴翻刲頃畦翻}
夏氏不忍其殘奏離昏為尼 乙丑加宋王從厚兼中書令 東川節度使董璋會諸將謀襲成都皆曰必克前陵州刺史王暉曰劍南萬里成都為大時方盛夏師出無名必無成功孟知祥聞之遣馬軍都指揮使潘仁嗣將三千人詣漢州詗之【詗古永翻又休正翻}
璋入境破白楊林鎮【白楊林鎮當在漢州界上}
執戍將武弘禮聲勢甚盛知祥憂之趙季良曰璋為人勇而無恩士卒不附城守則難克野戰則成擒矣今不守巢穴公之利也璋用兵精鋭皆在前鋒公宜以羸兵誘之以勁兵待之始雖小衂後必大捷【此孫臏三駟之說也自古以來以此取勝者多矣楚以之破吳師而滅舒鳩周訪以之破杜曾而清襄沔王茂章以之斬朱友寧其策畧皆不出此羸倫為翻衂女六翻}
璋素有威名今舉兵暴至人心危懼公當自出禦之以彊衆心趙廷隱以季良言為然曰璋輕而無謀【輕墟正翻}
舉兵必敗當為公擒之【為于偽翻}
辛巳以廷隱為行營馬步軍都部署將三萬人拒之五月壬午朔廷隱入辭董璋檄書至又有遺季良廷隱及李肇書【董璋書獨不及李仁罕者以趙季良者孟知祥之謀主趙廷隱李肇嘗與璋同禦石敬瑭干劍州故皆先以書誘之李仁罕未嘗共事故不及遺唯季翻}
誣之云季良廷隱與己通謀召己令來知祥以書授廷隱廷隱不視投之於地曰不過為反間【令力呈翻下同間古莧翻}
欲令公殺副使與廷隱耳【趙季良為節度副使故廷隱稱之}
再拜而行知祥曰事必濟矣肇素不知書視之曰璋教我反耳囚其使者然亦擁衆為自全計【李肇時鎮利州}
璋兵至漢州潘仁嗣與戰于赤水大敗為璋所擒【赤水在漢州東南}
璋遂克漢州癸未知祥留趙季良高敬柔守成都自將兵八千趣漢州至彌牟鎮【九域志成都府新都縣有彌牟鎮趣七喻翻}
趙廷隱陳於鎮北【陳讀曰陣下同}
甲申遲明【遲直二翻}
廷隱陳於雞蹤橋【薛史孟知祥傳云知祥親帥其衆與趙廷隱等逆戰於金雁橋璋軍大敗按金雁橋在漢州雒縣南璋兵既敗知祥追之夕宿雒縣豈金雁橋即雞蹤橋邪}
義勝定遠都知兵馬使張公鐸陳於其後俄而璋望西川兵盛退陳於武侯廟下【諸葛武侯有功於蜀蜀人所在為立廟}
璋帳下驍卒大譟曰日中曝我輩何為【曝步木翻}
璋乃上馬前鋒始交東川右廂馬步都指揮使張守進降於知祥言璋兵盡此無復後繼【復扶又翻}
當急擊之知祥登高冢督戰左明義指揮使毛重威左衝山指揮使李瑭守雞蹤橋【孟知祥置左右衝山六營見二百七十五卷天成元年}
皆為東川兵所殺趙廷隱三戰不利牙内都指揮副使侯弘實兵亦却知祥懼以馬箠指後陳【箠止橤翻}
張公鐸帥衆大呼而進【帥讀曰率呼火故翻}
東川兵大敗死者數千人擒東川中都指揮使元璝牙内副指揮使董光演等八十餘人【中都指揮使中軍都指揮使也一本有軍字璝公回翻}
璋拊膺曰親兵皆盡吾何依乎與數騎遁去餘衆七千人降【降戶江翻下同}
復得潘仁嗣【復扶又翻}
知祥引兵追璋至五侯津【五侯津在漢州西南}
東川馬步都指揮使元瓌降【元瓌疑即前元璝通鑑集衆書以成書以其官有中興馬步之異其字有瓌與璝之異因再書之耳}
西川兵入漢州府第求璋不得士卒爭璋軍資故璋走得免趙廷隱追至赤水又降其卒三千人是夕知祥宿雒縣【自唐以來漢州治雒縣知祥入漢州不居州宅而宿雒縣蓋漢州州宅為追兵剽掠不可居故宿雒縣廨舍也}
命李昊草牓諭東川吏民及草書勞問璋【勞力到翻}
且言將如梓州【如往也}
詢負約之由請見伐之罪乙酉知祥會廷隱於赤水遂西還【還從宣翻又如字}
命廷隱將兵攻梓州璋至梓州肩輿而入王暉迎問曰太尉全軍出征【洪邁曰唐節度帶檢校官其初只檢校散騎常侍如李愬在唐鄧時所稱者也後乃轉尚書及僕射司空司徒能至此者蓋少僖昭以降藩鎮盛強武夫得志纔建節鉞其資級已高於是復升太保太傅太尉其上惟有太師故將帥悉稱太尉余按唐制太師太傅太保為三師太尉司徒司空為三公太尉古以主兵故呼將帥為太尉耳若唐末藩鎮固亦有加太師者唐自睿宗之末邊鎮置節度使如薛訥等已是後來使相之職其帶御史大夫中丞六曹尚書者僕射侍中中書令者往往有之李愬之帥唐鄧隨以資淺帶散騎常侍耳洪說未為精當}
今還者無十人何也璋涕泣不能對至府第方食暉與璋從子牙内都虞侯延浩帥兵三百大譟而入【帥讀曰率下同}
璋引妻子登城子光嗣自殺璋至北門樓呼指揮使潘稠使討亂兵稠引十卒登城斬璋首及取光嗣首以授王暉暉舉城迎降趙廷隱入梓州封府庫以待知祥李肇聞璋敗始斬其使以聞【李肇持兩端聞璋敗始斬其使}
丙戌知祥入成都丁亥復將兵八千如梓州至新都【九域志新都縣在成都府北四十五里復扶又翻}
趙廷隱獻董璋首己丑發玄武【舊唐書地理志玄武漢氐道縣晉改曰玄武九域志宋大中祥符五年改為中江縣在梓州西九十里}
趙廷隱帥東川將吏來迎 康福奏党項鈔盜者已伏誅餘皆降附【鈔楚交翻}
壬辰孟知祥有疾癸巳疾甚中門副使王處回侍左右庖人進食必空器而出以安衆心李仁罕自遂州來趙廷隱迎于板橋【板橋在梓州東南}
仁罕不稱東川之功侵侮廷隱廷隱大怒乙未知祥疾瘳【瘳丑留翻}
丁酉入梓州戊戌犒賞將士既罷【犒苦到翻}
知祥謂李仁罕趙廷隱曰二將誰當鎮此仁罕曰令公再與蜀州亦行耳【先是朝廷加孟知祥中書令故李仁罕稱之仁罕蓋先嘗領蜀州}
廷隱不對知祥愕然退命李昊草牒俟二將有所推則命一人為留後昊曰昔梁祖莊宗皆兼領四鎮【梁太祖領宣武宣義天平護國四鎮莊宗領河東魏博盧龍成德四鎮}
今二將不讓惟公自領之為便耳公宜亟還府【府謂成都西川帥府所在}
更與趙僕射議之【趙僕射謂趙季良}
己亥契丹使者多囉卿辭歸國上曰朕志在安邊不可不少副其求【少詩沼翻}
乃遣策古舍利與之俱歸契丹以不得策稜自是數寇雲州及振武【數所角翻}
孟知祥命李仁罕歸遂州留趙廷隱東川巡檢以李昊行梓州軍府事昊曰二虎方爭僕不敢受命願從公還乃以都押牙王彦銖為東川監押【監古銜翻}
癸卯知祥至成都趙廷隱尋亦引兵西還【還從宣翻又如字}
知祥謂李昊曰吾得東川為患益深昊請其故知祥曰自吾發梓州得仁罕七狀皆云公宜自領東川不然諸將不服廷隱言本不敢當東川因仁罕不讓遂有爭心耳君為我曉廷隱【為干偽翻}
復以閬州為保寧軍【董璋取閬州廢保寧軍今孟知祥復以為節鎮以賞趙廷隱}
益以果蓬渠開四州往鎮之吾自領東川以絶仁罕之望廷隱猶不平請與仁罕鬬勝者為東川昊深解之乃受命六月以廷隱為保寧留後戊午趙季良帥將吏請知祥兼鎮東川許之【帥讀曰率}
季良等又請知祥稱王權行制書賞功臣不許董璋之攻知祥也山南西道節度使王思同以聞范延光言於上曰若兩川併於一賊撫衆守險則取之益難宜及其交爭早圖之上命思同以興元之兵密規進取【興元之兵既不足以進取王思同之才又不足以進取劉曄料魏之羣臣無能敵劉備孫權者所以重于用兵也}
未幾【幾居豈翻}
聞璋敗死延光曰知祥雖據全蜀然士卒皆東方人知祥恐其思歸為變亦欲倚朝廷之重以威其衆陛下不屈意撫之彼則無從自新上曰知祥吾故人為人離間至此何屈意之有【離間蓋指安重誨也孟知祥本有據蜀之志使重誨不相猜阻亦必别求釁端而動明宗蓋未能察見知祥之心術也間古莧翻}
乃遣供奉官李存瓌【此供奉官乃殿頭供奉官非禁中供奉官也}
賜知祥詔曰董璋狐狼【曰狐者言其依憑窟穴曰狼者言其抗厲犯上}
自貽族滅卿丘園親戚皆保安全【言董光業族滅而孟知祥墳墓甥姪皆無他}
所宜成家世之美名守君臣之大節存瓌克寧之子知祥之甥也【李克寧妻孟氏見二百六十六卷梁太祖開平二年}
閩王延鈞謂陳守元曰為我問寶皇【為于偽翻}
既為六十年天子【陳守元此語見上年}
後當如何明日守元入白昨夕奏章得寶皇旨當為大羅仙主徐彦林等亦曰北廟崇順王嘗見寶皇其言與守元同延鈞益自負【史言王延鈞之昏愚}
始謀稱帝表朝廷云錢鏐卒請以臣為吳越王馬殷卒請以臣為尚書令【錢鏐卒于是年三月馬殷卒于去年十一月}
朝廷不報自是職貢遂絶

資治通鑑卷二百七十七
















































































































































