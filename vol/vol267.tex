<!DOCTYPE html PUBLIC "-//W3C//DTD XHTML 1.0 Transitional//EN" "http://www.w3.org/TR/xhtml1/DTD/xhtml1-transitional.dtd">
<html xmlns="http://www.w3.org/1999/xhtml">
<head>
<meta http-equiv="Content-Type" content="text/html; charset=utf-8" />
<meta http-equiv="X-UA-Compatible" content="IE=Edge,chrome=1">
<title>資治通鑒_268-資治通鑑卷二百六十七_268-資治通鑑卷二百六十七</title>
<meta name="Keywords" content="資治通鑒_268-資治通鑑卷二百六十七_268-資治通鑑卷二百六十七">
<meta name="Description" content="資治通鑒_268-資治通鑑卷二百六十七_268-資治通鑑卷二百六十七">
<meta http-equiv="Cache-Control" content="no-transform" />
<meta http-equiv="Cache-Control" content="no-siteapp" />
<link href="/img/style.css" rel="stylesheet" type="text/css" />
<script src="/img/m.js?2020"></script> 
</head>
<body>
 <div class="ClassNavi">
<a  href="/24shi/">二十四史</a> | <a href="/SiKuQuanShu/">四库全书</a> | <a href="http://www.guoxuedashi.com/gjtsjc/"><font  color="#FF0000">古今图书集成</font></a> | <a href="/renwu/">历史人物</a> | <a href="/ShuoWenJieZi/"><font  color="#FF0000">说文解字</a></font> | <a href="/chengyu/">成语词典</a> | <a  target="_blank"  href="http://www.guoxuedashi.com/jgwhj/"><font  color="#FF0000">甲骨文合集</font></a> | <a href="/yzjwjc/"><font  color="#FF0000">殷周金文集成</font></a> | <a href="/xiangxingzi/"><font color="#0000FF">象形字典</font></a> | <a href="/13jing/"><font  color="#FF0000">十三经索引</font></a> | <a href="/zixing/"><font  color="#FF0000">字体转换器</font></a> | <a href="/zidian/xz/"><font color="#0000FF">篆书识别</font></a> | <a href="/jinfanyi/">近义反义词</a> | <a href="/duilian/">对联大全</a> | <a href="/jiapu/"><font  color="#0000FF">家谱族谱查询</font></a> | <a href="http://www.guoxuemi.com/hafo/" target="_blank" ><font color="#FF0000">哈佛古籍</font></a> 
</div>

 <!-- 头部导航开始 -->
<div class="w1180 head clearfix">
  <div class="head_logo l"><a title="国学大师官网" href="http://www.guoxuedashi.com" target="_blank"></a></div>
  <div class="head_sr l">
  <div id="head1">
  
  <a href="http://www.guoxuedashi.com/zidian/bujian/" target="_blank" ><img src="http://www.guoxuedashi.com/img/top1.gif" width="88" height="60" border="0" title="部件查字,支持20万汉字"></a>


<a href="http://www.guoxuedashi.com/help/yingpan.php" target="_blank"><img src="http://www.guoxuedashi.com/img/top230.gif" width="600" height="62" border="0" ></a>


  </div>
  <div id="head3"><a href="javascript:" onClick="javascript:window.external.AddFavorite(window.location.href,document.title);">添加收藏</a>
  <br><a href="/help/setie.php">搜索引擎</a>
  <br><a href="/help/zanzhu.php">赞助本站</a></div>
  <div id="head2">
 <a href="http://www.guoxuemi.com/" target="_blank"><img src="http://www.guoxuedashi.com/img/guoxuemi.gif" width="95" height="62" border="0" style="margin-left:2px;" title="国学迷"></a>
  

  </div>
</div>
  <div class="clear"></div>
  <div class="head_nav">
  <p><a href="/">首页</a> | <a href="/ShuKu/">国学书库</a> | <a href="/guji/">影印古籍</a> | <a href="/shici/">诗词宝典</a> | <a   href="/SiKuQuanShu/gxjx.php">精选</a> <b>|</b> <a href="/zidian/">汉语字典</a> | <a href="/hydcd/">汉语词典</a> | <a href="http://www.guoxuedashi.com/zidian/bujian/"><font  color="#CC0066">部件查字</font></a> | <a href="http://www.sfds.cn/"><font  color="#CC0066">书法大师</font></a> | <a href="/jgwhj/">甲骨文</a> <b>|</b> <a href="/b/4/"><font  color="#CC0066">解密</font></a> | <a href="/renwu/">历史人物</a> | <a href="/diangu/">历史典故</a> | <a href="/xingshi/">姓氏</a> | <a href="/minzu/">民族</a> <b>|</b> <a href="/mz/"><font  color="#CC0066">世界名著</font></a> | <a href="/download/">软件下载</a>
</p>
<p><a href="/b/"><font  color="#CC0066">历史</font></a> | <a href="http://skqs.guoxuedashi.com/" target="_blank">四库全书</a> |  <a href="http://www.guoxuedashi.com/search/" target="_blank"><font  color="#CC0066">全文检索</font></a> | <a href="http://www.guoxuedashi.com/shumu/">古籍书目</a> | <a   href="/24shi/">正史</a> <b>|</b> <a href="/chengyu/">成语词典</a> | <a href="/kangxi/" title="康熙字典">康熙字典</a> | <a href="/ShuoWenJieZi/">说文解字</a> | <a href="/zixing/yanbian/">字形演变</a> | <a href="/yzjwjc/">金 文</a> <b>|</b>  <a href="/shijian/nian-hao/">年号</a> | <a href="/diming/">历史地名</a> | <a href="/shijian/">历史事件</a> | <a href="/guanzhi/">官职</a> | <a href="/lishi/">知识</a> <b>|</b> <a href="/zhongyi/">中医中药</a> | <a href="http://www.guoxuedashi.com/forum/">留言反馈</a>
</p>
  </div>
</div>
<!-- 头部导航END --> 
<!-- 内容区开始 --> 
<div class="w1180 clearfix">
  <div class="info l">
   
<div class="clearfix" style="background:#f5faff;">
<script src='http://www.guoxuedashi.com/img/headersou.js'></script>

</div>
  <div class="info_tree"><a href="http://www.guoxuedashi.com">首页</a> > <a href="/SiKuQuanShu/fanti/">四库全书</a>
 > <h1>资治通鉴</h1> <!--         下载:【右键另存为】即可 --></div>
  <div class="info_content zj clearfix">
  
<div class="info_txt clearfix" id="show">
<center style="font-size:24px;">268-資治通鑑卷二百六十七</center>
    資治通鑑卷二百六十七 宋 司馬光 撰<br />
<br />
  胡三省 音註<br />
<br />
  後梁紀二【起著雍執徐八月盡重光協洽二月凡二年有奇】<br />
<br />
  太祖神武元聖孝皇帝中<br />
<br />
  開平二年八月吳越王鏐遣寧國節度使王景仁奉表詣大梁【王茂章奔兩浙見二百六十五卷唐昭宣帝天祐三年】陳取淮南之策景仁即茂章也避梁諱改焉【帝曾祖諱茂琳按薛史梁紀元年六月司天監上言請改月辰内戊字為武避諱也】淮南遣步軍都指揮使周本南面統軍使呂師造擊吳越九月圍蘇州吳越將張仁保攻常州之東洲拔之【宋白曰通州海門縣東南隔水二百餘里本東洲鎮】淮南兵死者萬餘人淮南以池州團練使陳璋為水陸行營都招討使帥柴再用等諸將救東洲【帥讀曰率】大破仁保於魚蕩復取東洲【復扶又翻】柴再用方戰舟壞長矟浮之僅而得濟【矟所角翻】家人為之飯僧千人再用悉取其食以犒部兵【為于偽翻飯扶晚翻犒苦到翻】曰士卒濟我僧何力焉【史言柴再用善養士卒而不惑於異端】 丙子蜀立皇后周氏后許州人也【周氏蓋蜀主建糟糠之妻也】 晉周德威李嗣昭將兵三萬出隂地關攻晉州刺史徐懷玉拒守帝自將救之丁丑發大梁乙酉至陜州【將即亮翻陜式冉翻】戊子岐王所署延州節度使胡敬璋寇上平關【金人疆域圖隰州石樓縣有上平關按延州東至隰州百三十里耳胡敬璋蓋度河來寇也】劉知俊擊破之周德威等聞帝將至乙未退保隰州【九域志晉州西北至隰州二百五十五里】 荆南節度使高季昌遣兵屯漢口【漢口漢水入江之口其地在鄂州漢陽縣東大别山下】絶楚朝貢之路楚王殷遣其將許德勲將水軍擊之至沙頭【沙頭即今江陵城南沙頭市】季昌懼而請和殷又遣步軍都指揮使呂師周將兵擊嶺南【呂師周降馬殷見上卷元年】與清海節度使劉隱十餘戰取昭賀梧蒙龔富六州【蒙州隋始安郡之隋化縣唐武德四年置南恭州貞觀二年更名蒙州龔州本漢猛陵縣地隋為永平郡武林縣唐貞觀三年置鷰州七年移鷰州於今州東仍於鷰州舊所置龔州又武德四年以始安郡之龍平豪靜及蒼梧郡之蒼梧置富州九域志昭州東至賀州三百二十五里南至梧州四百九十里南稍斜至龔州五百五十里宋開寶廢富州以龍平縣隸昭州在州東南百六十二里熙寧五年廢蒙州以立山縣隸昭州在州南二百十二里】殷土字旣廣乃養士息民湖南遂安 冬十月蜀主立後宫張氏為貴妃徐氏為賢妃其妹為德妃張氏郪人宗懿之母也【郪漢縣唐帶梓州】二徐耕之女也【徐耕見二百五十八卷唐昭宗大順二年為徐妃亡蜀張本】 華原賊帥溫韜聚衆嵯峨山【帥所類翻嵯才何翻峨音俄】㬥掠雍州諸縣唐帝諸陵發之殆徧【温韜傳韜在華原七年唐諸陵在其境内者悉發掘之取其所藏金寶而昭陵最固韜從埏道下見宫室制度閎麗不異人間中為正寢東西廂列石牀牀上石函中為鐵匣悉藏前代圖書鍾王筆迹紙墨如新韜悉取之遂傳人間惟乾陵風雨不可發雍於用翻】 庚戌蜀主講武於星宿山步騎三十萬【宿音秀】丁巳帝還大梁 【考異曰編遺錄在乙卯今從實錄薛史】 辛酉以劉隱為清海靜海節度使【兼交廣二鎮也然劉氏終不能有安南】以膳部郎中趙光裔右補闕李殷衡充官告使隱皆留之【史言羣雄割據各收拾衣冠之胄以為用】光裔光逢之弟殷衡德裕之孫也 依政進士梁震【依政泰蒲陽縣漢臨卭縣後魏置蒲陽郡及依政縣唐屬卭州九域志在州東南五十里】唐末登第至是歸蜀過江陵高季昌愛其才識留之欲奏為判官震耻之【高季昌出於奴僕故梁震耻為之僚屬】欲去恐及禍乃曰震素不慕榮官明公不以震為愚必欲使之參謀議但以白衣侍樽俎可也何必在幕府季昌許之震終身止稱前進士不受高氏辟署季昌甚重之以為謀主呼曰先輩【唐人呼進士為先輩至今猶然】 帝從吳越王鏐之請以亳州團練使寇彥卿為東南面行營都指揮使擊淮南十一月彥卿帥衆二千襲霍丘【帥讀曰率】為土豪朱景所敗【敗補邁翻下同】又攻廬壽二州皆不勝淮南遣滁州刺史史儼拒之彥卿引歸【寇彥卿兵勢已挫而史儼河東健將汴兵所畏也故聞其至而退】 定難節度使李思諫卒【難乃旦翻】甲戌其子彞昌自為留後 劉守文舉滄德兵攻幽州劉守光求救於晉晉王遣兵五千助之丁亥守文兵至盧臺軍【盧臺軍宋為乾寧軍地九域志乾寧軍在滄州西北九十里】為守光所敗又戰玉田亦敗【玉田漢無終縣唐萬歲通天元年更名玉田屬薊州在薊州東南八十里又東北至平州二百里西至幽州三百里亦敗讀如字】守文乃還【還從宣翻又如字】 癸巳中書侍郎同平章事張策以刑部尚書致仕以左僕射楊涉同平章事 保塞節度使胡敬璋卒靜難節度使李繼徽以其將劉萬子代鎮延州【保塞靜難二鎮時皆屬岐】 是歲弘農王遣軍將萬全感齎書間道詣晉及岐告以嗣位【間古莧翻岐晉淮南之與國】 帝將遷都洛陽三年春正月己巳遷太廟神主於洛陽甲戌帝發大梁壬申以博王友文為東都留守【梁以大梁為東都】己卯帝至洛陽庚寅饗太廟辛巳祀圓丘大赦 丙申以用度稍充初給百官全俸【唐自廣明喪亂以來百官俸料額存而已至是復全給】 二月丁酉朔日有食之 保塞節度使劉萬子㬥虐失衆心且謀貳於梁李繼徽使延州牙將李延實圖之延實因萬子葬胡敬璋攻而殺之遂據延州馬軍都指揮使河西高萬興與其弟萬金聞變以其衆數千人詣劉知俊降【為高萬興兄弟取鄜延張本】岐王置翟州於鄜城【後魏置敷城郡及敷城縣隋改曰鄜城唐屬坊州九域志縣在鄜州東一百二十里翟徒歷翻】其守將亦降 三月甲戌帝發洛陽以山南東道節度使楊師厚兼潞州四面行營招討使 庚辰帝至河中發步騎會高萬興兵取丹延【宋白曰丹州秦上郡地符姚時為三堡鎮後魏大統三年割鄜延二州地置汾州理三堡鎮廢帝以河東汾州同名改為丹州因丹楊川以為名延州項羽以董翳為翟王都高奴即其地魏滅赫連以為統萬鎮後為東夏州後改延州】 丙戌以朔方節度使兼中書令韓遜為潁川王遜本靈州牙校【校戶教翻】唐末據本鎮朝廷因而授以節鉞 辛卯丹州刺史崔公實請降【丹州堡塞軍巡屬】 徐温以金陵形勝戰艦所聚【艦戶黯翻】乃自以淮南行軍副使領昇州刺史留廣陵以其假子元從指揮使知誥為昇州防遏兼樓船副使往治之【從才用翻治直之翻為徐知誥完理昇州徐温遂居之張本】 夏四月丙申朔劉知俊移軍攻延州李延實嬰城自守知俊遣白水鎮使劉儒分兵圍坊州【後魏太和二年分澄城置白水郡及縣隋廢郡以縣屬馮翊唐屬同州九域志在州西北一百二十里】 庚子以王審知為閩王劉隱為南平王 劉知俊克延州李延實降【降戶江翻】 淮南兵圍蘇州推洞屋攻城【推吐雷翻洞屋以木撑柱為之冒以牛皮其狀如洞】吳越將臨海孫琰置輪於竿首垂絙投錐以揭之攻者盡露礮至則張網以拒之【絙居登翻揭丘傑翻礮與砲同匹貌翻】淮南人不能克吳越王鏐遣牙内指揮使錢鏢【鏢甫招翻】行軍副使杜建徽等將兵救之蘇州有水通城中淮南張網綴鈴懸水中魚鼈過皆知之吳越遊奕都虞候司馬福欲潛行入城故以竿觸網敵聞鈴聲舉網福因得過凡居水中三日乃得入城由是城中號令與援兵相應敵以為神吳越王鏐嘗遊府園見園卒陸仁章樹藝有智而志之【志者記之於心】及蘇州被圍【被皮義翻】使仁章通信入城果得報而返鏐以諸孫畜之【畜吁玉翻養也】累遷兩府軍糧都監使【兩府鎮海鎮東兩節度府】卒獲其用【卒子恤翻】仁章睦州人也辛亥吳越兵内外合擊淮南兵大破之擒其將何朗等三十餘人奪戰艦二百艘【艘蘇遭翻】周本夜遁又追敗之於皇天蕩【此皇天蕩非真州大江中之皇天蕩按宋熙寧三年平江府崑山縣人郟亶上奏言水利長洲縣界有長蕩皇天蕩此則是也敗補邁翻】鍾泰章將精兵二百為殿【殿丁練翻】多樹旗幟於菰蔣中【菰即蔣也幟昌志翻】追兵不敢進而還【還從宣翻又如字】岐王所署保大節度使李彥博 【考異曰編遺録五代史作彥容今從劉恕廣本】坊州刺史李彥昱皆棄城奔鳳翔鄜州都將嚴弘倚舉城降【鄜音夫】己未以高萬興為保塞節度使以絳州刺史牛存節為保大節度使【梁遂取鄜坊丹延兩鎮】 淮南初置選舉以駱知祥掌之【喪亂以來選舉之法廢楊氏能復置之故書】 五月丁卯帝命劉知俊乘勝取邠州知俊難之【李繼徽據邠州有鳳翔之援故劉知俊以取之為難】辭以闕食乃召還 佑國節度使王重師鎮長安數年帝在河中怒其貢奉不時己巳召重師入朝以左龍虎統軍劉捍為佑國留後 癸酉帝發河中己卯至洛陽劉捍至長安王重師不為禮捍譛之於帝云重師潛與邠岐通甲申貶重師溪州刺史尋賜自盡夷其族【為劉知俊殺劉桿以叛張本】 劉守文頻年攻劉守光不克【劉守文自元年攻守光事始見上卷】乃大發兵以重賂招契丹吐谷渾之衆四萬屯薊州守光逆戰於雞蘇【按薛史梁紀是年劉守光上言於薊州西與兄守文戰生禽守文蓋即雞蘇也】為守文所敗【敗補邁翻】守文單馬立於陳前【陳讀曰陣】泣謂其衆曰勿殺吾弟守光將元行欽識之直前擒之【劉守光以子囚父天下之賊也劉守文旣聲其罪而討之有誅無赦小不忍以敗大事身為俘囚自取之也】滄德兵皆潰守光囚之别室栫以藂棘【栫才甸翻藂與叢同】乘勝進攻滄州滄州節度判官呂兖孫鶴推守文子延祚為帥乘城拒守兖安次人也【安次漢縣唐屬幽州在州東南一百三十里帥所類翻】 忠武節度使兼侍中劉知俊【去年更同州匡國軍為忠武軍事見上卷】功名浸盛以帝猜忍日甚内不自安及王重師誅知俊益懼帝將伐河東【河東謂晉】急徵知俊入朝欲以為河東西面行營都統且以知俊有丹延之功厚賜之知俊弟右保勝指揮使知浣從帝在洛陽密使人語知俊云【語牛倨翻】入必死又白帝請帥弟姪往迎知俊【帥讀曰率】帝許之六月乙未朔知俊奏為軍民所留遂以同州附於岐 【考異曰實錄六月庚戌知俊據本郡反削奪官爵興師討伐編遺錄六月乙未初奏本道軍民遮留尋聞擒使臣及將送鳳翔蓋編遺據奏到之日實錄据削奪之日也】執監軍及將佐之不從者皆械送於岐遣兵襲華州逐刺史蔡敬思【九域志同州南至華州七十里華戶化翻】以兵守潼關濳遣人以重利㗖長安諸將【㗖徒濫翻】執劉捍送於岐殺之知俊遣使請兵於岐亦遣使請晉人出兵攻晉絳遺晉王書曰不過旬日可取兩京復唐社稷【遺唯季翻長安可以言取梁都洛陽未易取也】 丁未朔方節度使韓遜奏克鹽州斬岐所署刺史李繼直【唐末鹽州奏事專達朝廷不隸靈夏至是靈鹽遂復合為一鎮】帝遣近臣諭劉知俊曰朕待卿甚厚何忽相負對曰臣不背德【背蒲妹翻】但畏族滅如王重師耳帝復使謂之曰【復扶又翻】劉捍言重師隂結邠岐朕今悔之無及捍死不足塞責【塞悉則翻】知俊不報庚戌詔削知俊官爵以山南東道節度使楊師厚為西路行營招討使帥侍衛馬步軍都指揮使劉鄩等討之【帥讀曰率】辛亥帝發洛陽劉鄩至潼關東獲劉知俊伏路兵藺如海等三十人釋之使為前導【劉知俊旣得潼閟於關外沿路伏兵以候望劉鄩反得而用之以為嚮導史炤曰藺姓也其先韓獻子玄孫曰康食采於藺因氏焉】劉知浣迷失道盤桓數日乃至關下關吏納之如海等繼至關吏不知其已被擒亦納之【被皮義翻】鄩兵乘門開直進遂克潼關追及知浣擒之 癸丑帝至陜【陜式冉翻】 丹州馬軍都頭王行思等作亂刺史宋知誨逃歸 帝遣劉知俊姪嗣業持詔詣同州招諭知俊知俊欲輕騎詣行在謝罪弟知偃止之楊師厚等至華州知俊將聶賞開門降【聶尼輒翻姓也史炤曰楚大夫食采於聶因以為氏】知俊聞潼關不守官軍繼至蒼黃失圖乙卯舉族奔岐楊師厚至長安岐兵已據城師厚以奇兵並南山急趨自西門入遂克之【並步浪翻按唐長安城十門西南三門惟延平門近南山耳長安旣丘墟之餘且城大難守使楊師厚不以奇兵入西門岐兵亦不能久也】庚申以劉鄩權佑國留後岐王厚禮劉知俊以為中書令地狹無藩鎮處之【處昌呂翻】但厚給俸祿而已【蹄涔不容尺鯉為劉知俊奔蜀張本】 劉守光遣使上表告捷且言俟滄德事畢為陛下掃平并寇【為于偽翻河東并州之地時與梁為敵故言并寇】亦致書晉王云欲與之同破偽梁【劉守光反覆梁晉之間自以為得計不知乃所以速亡也】 撫州刺史危全諷自稱鎮南節度使帥撫信袁吉之兵號十萬攻洪州【唐置鎮南軍於洪州撫信袁吉皆巡屬也危全諷自稱節度舉兵以攻洪州欲兼而有之九域志撫州西北至洪州二百九里帥讀曰率】淮南守兵纔千人將吏皆懼節度使劉威密遣使告急於廣陵日召僚佐宴飲全諷聞之屯象牙潭不敢進【象牙潭在撫州金溪縣東北】請兵於楚楚王殷遣指揮使苑玫【苑姓也左傳齊有大夫苑何忌玫莫杯翻】會袁州刺史彭彥章圍高安以助全諷玫蔡州人【玫莫杯翻】彥章玕之兄也【彭玕見二百六十有五卷唐昭宣帝天祐三年】徐温問將於嚴可求可求薦周本乃以本為西南面行營招討應援使將兵七千救高安本以前攻蘇州無功【事見上四月】稱疾不出可求即其臥内強起之【強其兩翻】本曰蘇州之役敵不能勝我但主將權輕耳今必見用願毋置副貳乃可可求許之本曰楚人為全諷聲援耳非欲取高安也吾敗全諷【敗補邁翻】援兵必還【援兵謂圍高安之兵還從宣翻】乃疾趣象牙潭【趣七喻翻】過洪州劉威欲犒軍【犒苦到翻】本不肯留或曰全諷兵強君宜觀形勢然後進本曰賊衆十倍於我我軍聞之必懼不若乘其鋭而用之 秋七月甲子以劉守光為燕王 梁兵克丹州擒王行思 商州刺史李稠驅士民西走【將奔蜀也】將吏追斬之 【考異曰薛史稠弃郡西奔本州將吏以都牙校李玫權知州事歐陽史商州軍亂逐其刺史李稠稠奔于岐實錄丙寅陜州奏商州刺史李稠弃郡逃山谷又曰商州將吏以稠驅虜士庶西遁追斬無遺暫令都押牙李玫主州事今從之】推都押牙李玫主州事 庚午改佑國軍曰永平【開平元年徙佑國軍於長安今改曰永平】河東兵寇晉州抄掠至堯祠而去【堯都平陽有祠在汾城東十里東原】<br />
<br />
  【上平陽唐為臨汾縣晉州所治也】 癸酉帝發陜州乙亥至洛陽寢疾初帝召山南東道節度使楊師厚欲使督諸將攻潞<br />
<br />
  州以前兖海留後王班為留後鎮襄州 【考異曰薛史作王珏今從實錄】師厚屢為班言牙兵王求等凶悍宜備之【為于偽翻】班自恃左右有壯士不以為意每衆辱之戊寅讁求戍西境是夕作亂殺班推都指揮使雍丘劉玘為留後【雍於用翻玘區里翻】玘偽從之明日與指揮使王延順逃詣帝所 【考異曰姚顗明宗實錄薛史玘傳皆云翌日受賀衙庭享士伏甲幕下中筵盡斬其亂將以聞以功為復州刺史按梁祖實錄八月丁酉賜玘王延順物以其違逆將之難來歸編遺錄斬李洪等敕云始扶劉玘旣奔竄以歸朝若使玘翌日便斬亂將襄州何由至九月始收復蓋玘脱身歸朝及梁亡入唐妄云斬亂將自誇大史官不能考察從而書之耳】亂兵奉平淮指揮使李洪為留後附於蜀未幾房州刺史楊虔亦叛附於蜀【幾居豈翻】 危全諷在象牙潭營柵臨溪亘數十里【亘居鄧翻】庚辰周本隔溪布陳【陳讀曰陣】先使羸兵嘗敵【羸倫為翻嘗試也】全諷兵涉溪追之本乘其半濟縱兵擊之全諷兵大潰自相蹂藉【蹂忍久翻又如又翻藉慈夜翻】溺水死者甚衆本分兵斷其歸路【斷音短】擒全諷及將士五千人乘勝克袁州執刺史彭彥章進攻吉州【九域志袁州南至吉州三百一十五里】歙州刺史陶雅使其子敬昭及都指揮使徐章將兵襲饒信信州刺史危仔倡請降【唐僖宗中和二年危全諷據撫州仔倡據信州至是皆亡】饒州刺史唐寶棄城走行營都指揮使米志誠都尉呂師造等敗苑玫於上高【敗補邁翻】吉州刺史彭玕帥衆數千人奔楚【唐昭宗天祐三年彭玕附楚】楚王殷表玕為郴州刺史【郴丑林翻】為子希範娶其女【為于偽翻】淮南以左先鋒指揮使張景思知信州遣行營都虞候骨言將兵五千送之【骨姓也唐初有骨儀】危仔倡聞兵至奔吳越吳越王鏐以仔倡為淮南節度副使更其姓曰元氏【歐史十國世家曰錢鏐惡危姓更之曰元更工衡翻】危全諷至廣陵弘農王以其嘗有德於武忠王釋之資給甚厚【楊行密諡武忠時淮南諸將議曰昔先王攻趙鍠全諷屢饟給吾軍乃釋之】八月虔州刺史盧光稠以州附於淮南於是江西之地盡入於楊氏光稠亦遣使附於梁 甲寅上疾小瘳始復視朝【自七月乙亥寢疾至是凡四十日朝直遙翻下同】 以鎮國節度使康懷貞為西路行營副招討使 蜀主命太子宗懿判六軍開永和府妙選朝士為僚屬 辛酉均州刺史張敬方奏克房州【楊虔以房州附蜀見上九域志均州南至房州二百一十五里】 岐王欲遣劉知俊將兵攻靈夏【夏戶雅翻】且約晉王使攻晉絳晉王引兵南下先遣周德威等將兵出隂地關攻晉州刺史邊繼威悉力固守晉兵穿地道陷城二十餘步城中血戰拒之一夕城復成詔楊師厚將兵救晉州周德威以騎扼蒙阬之險【蒙阬在汾水東東西三百餘里蹊徑不通】師厚擊破之進抵晉州晉兵解圍遁去 【考異曰實錄云殺戮生禽賊將蕭萬通等賊由是弃寨而遁莊宗實錄云汴軍至蒙阬周德威逆戰敗之斬首二百級師厚退絳州是役也小將蕭萬通戰沒師厚進營平陽德威收軍而退二軍各言勝捷然旣殺蕭萬通師厚何背退保絳州旣敗而退豈得復進營平陽德威旣戰勝安肯便收軍蓋晉軍實敗走莊宗實録妄言耳】 李洪寇荆南高季昌遣其將倪可福擊敗之【敗補邁翻】詔馬步都指揮使陳暉將兵會荆南兵討洪 蜀主以御史中丞王鍇為中書侍郎同平章事【鍇苦駭翻】 陳暉軍至襄州李洪逆戰大敗王求死九月丁酉拔其城斬叛兵千人執李洪楊虔等送洛陽斬之 丁未以保義節度使王檀為潞州東面行營招討使 劉守光奏遣其子中軍兵馬使繼威安撫滄州吏民戊申以繼威為義昌留後 辛亥侍中韓建罷守太保左僕射同平章事楊涉罷守本官以太常卿趙光逢為中書侍郎翰林奉旨工部侍郎杜曉為戶部侍郎並同平章事【梁改翰林承旨為翰林奉旨以廟諱誠避嫌諱也然誠字與承字各自翻切不同】曉讓能之子也【杜讓能死國難見二百五十九卷唐昭宗景福二年】 淮南遣使者張知遠脩好於福建【好呼到翻】知遠倨慢閩王審知斬之表上其書【上時掌翻】始與淮南絶審知性儉約常躡麻屨【躡尼輒翻】府舍卑陋未嘗營葺寛刑薄賦公私富實境内以安歲自海道登萊入貢没溺者什四五【自福建入貢大梁陸行當由衢信取饒池界度江取舒廬夀度淮而後入梁境然自信饒至廬夀皆屬楊氏而朱楊為世仇不可得而假道故航海入貢今自福州洋過温州洋取台州洋過天門山入明州象山洋過涔江掠洌港直東北度大洋抵登萊岸風濤至險故没溺者衆】冬十月甲子蜀司天監胡秀林獻永昌歷行之【歐陽脩曰】<br />
<br />
  【永昌歷止行於其國今亡不復見】 湖州刺史高灃性凶忍嘗召州吏議曰吾欲盡殺百姓可乎吏曰如此租賦何從出當擇可殺者殺之耳時灃糾民為兵有言其咨怨者灃悉集民兵於開元寺【開元寺今諸州間亦有之蓋唐開元中所置也】紿云犒享入則殺之死者踰半在外者覺之縱火作亂灃閉門大索【索山客翻】凡殺三千人吳越王鏐欲誅之戊辰灃以州叛附於淮南【高灃父子以一州之地介居錢楊之間率兩附以自存為日久矣今專附淮南錢氏之兵至矣】舉兵焚義和臨平鎮【九域志杭州仁和縣有臨平鎮按仁和縣本錢塘縣宋朝太平興國初改錢塘縣曰仁和蓋亦先有義和地名又避太宗藩邸舊名遂改曰仁和也】鏐命指揮使錢鏢討之 十一月甲午帝告謝於圓丘【告謝者告天而謝得天下也案歐史是日日南至徐無黨注曰不曰有事於南郊蓋比南郊禮差簡】戊戌大赦 鄴王羅紹威得風痺病【痺必至翻】上表稱魏故大鎮多外兵願得有功重臣鎮之臣乞骸骨歸第帝聞之撫案動容【撫案動容非矜羅紹威之病也魏博大鎮世襲者百五十年一旦委鎮請代出於意料之表喜溢於中不知手之撫容之動也】己亥以其子周翰為天雄節度副使知府事謂使者曰亟歸語而主【亟紀力翻急也語牛倨翻而汝也】為我彊飯【為于偽翻彊其兩翻飯扶晚翻】如有不可諱【謂死也】當世世貴爾子孫以相報也今使周翰領軍府尚冀爾復愈耳 【考異曰梁功臣列傳朝廷自開創有大事皆降使咨訪紹威有謀慮亦馳簡獻替或中途相遇意互合者十得五六太祖嘆曰竭忠力一人而已又曰子三人長廷規司農卿尚安陽公主又尚金華公主早卒次周翰起復雲麾將軍充天雄節度留後尋檢校司徒正授魏博節度使亦早卒次曰周敬薛史亦同實錄己亥以司門郎中羅廷規充魏博節度副使知府事仍改名周翰時鄴王紹威病日甚慮以後事故奏請焉莊宗列傳紹威卒温以其子周翰嗣政莊宗實錄紹威厚率重歛傾府藏以奉温小有違忤温即遣人詬辱紹威方懷愧耻悔自弱之謀乃潛收兵市馬隂有覆温之志而賂温益厚温怪其曲事慮蓄奸謀而莫之察乃賜紹威妓妾數人皆氶嬖愛未半歲温却召還以此得其隂事内相矛楯薛史又云開平四年夏詔金華公主出家為尼居於宋州玄靜寺蓋太祖推恩於羅氏令終其婦節也唐餘錄歐陽史皆同惟唐莊宗實錄獨異按均帝時趙巖等言羅紹威前恭後倨太祖每深含怒似與此言合然梁祖若聞紹威有隂謀必不使周翰更居魏疑後唐史以紹威與梁最親疾之而載此傳聞之語今從衆書廷規更名周翰亦恐實錄之誤】 岐王欲取靈州以處劉知俊【處昌呂翻】且以為牧馬之地使知俊自將兵攻之朔方節度使韓遜告急詔鎮國節度使康懷貞感化節度使寇彥卿將兵攻邠寧以救之懷貞等所向皆捷克寧衍二州拔慶州南城刺史李彥廣出降【寧慶衍三州皆靜難軍巡屬岐地也周顯德五年廢衍州為定平鎮隸汾州九域志熙寧五年以汾州定平縣隸涇州在州南六十里】遊兵侵掠至涇州之境劉知俊聞之十二月己丑解靈州圍引兵還帝急召懷貞等還遣兵迎援於三原青谷懷貞等還至三水【三水漢古縣唐屬邠州九域志在州東北六十里】知俊遣兵據險邀之【薛史曰知俊邀擊懷貞等於邠州長城嶺】左龍驤軍使壽張王彥章力戰【五代會要曰開平元年改左右親隨軍將馬軍為左右龍驤軍】懷貞等乃得過懷貞與禆將李德遇許從實王審權分道而行皆與援兵不相値至昇平【唐天寶十二載分宜君置昇平縣屬坊州】劉知俊伏兵山口懷貞大敗僅以身免德遇等軍皆沒岐王以知俊為彰義節度使鎮涇州王彥章驍勇絶倫【驍堅堯翻】每戰用二鐵槍皆重百斤一置鞍中一在手所向無前時人謂之王鐵槍 蜀蜀州刺史王宗弁稱疾罷歸成都杜門不出【王宗弁鹿弁也蜀主養以為子賜姓名】蜀主疑其矜功怨望加檢校太保固辭不受謂人曰廉者足而不憂貪者憂而不足吾小人致位至此足矣豈可求進不已乎蜀主嘉其志而許之賜與有加【宗弁之祈閑以蜀主之雄猜也】 劉守光圍滄州久不下【劉守光自五月攻滄州】執劉守文至城下示之猶固守城中食盡民食堇泥軍士食人驢馬相噉騣尾【噉徒濫翻騣子紅翻】呂兖選男女羸弱者飼以麴麫而烹之【羸倫為翻飼祥吏翻麴丘六翻酒母麫眠見翻麥粉】以給軍食謂之宰殺務<br />
<br />
  四年春正月乙未劉延祚力盡出降時劉繼威尚幼【劉繼威守光之子也】守光使大將張萬進周知裕輔之鎮滄州【為張萬進殺劉繼威張本】以延祚及其將佐歸幽州族呂兖而釋孫鶴兖子琦年十五門下客趙玉紿監刑者曰此吾弟也勿妄殺監刑者信之遂挈以逃琦足痛不能行玉負之變姓名乞食於路僅而得免琦感家門殄滅力學自立晉王聞其名授代州判官【孫鶴終不免於誅呂琦能自樹立天乎人也】 辛丑以盧光稠為鎮南留後【盧光稠以䖍州附梁鎮南軍置於洪州時已為淮南所有】 劉守光為其父仁恭請致仕【為于偽翻】丙午以仁恭為太師致仕守光尋使人濳殺其兄守文歸罪於殺者而誅之二月萬全感自岐歸廣陵【前年淮南使萬全感使晉及岐】岐王承制加弘農王兼中書令嗣吳王【唐昭宗天復二年封楊行密吳王今岐王承制加隆演嗣王】於是吳王赦其境内 高灃求救於吳吳常州刺史李蕳等將兵應之湖州將盛師友沈行思閉城不内灃帥麾下五千人奔吳【按唐昭宗乾寧四年李彥徽奔淮南錢鏐取湖州天復二年徐許亂杭州湖州刺史高彥遣子渭入援唐昭宣帝天祐三年彥卒子灃代立至是而敗帥讀曰率】三月癸巳吳越王鏐巡湖州以錢鏢為刺史 蜀太子宗懿驕㬥好陵㬥舊臣【好呼到翻】内樞密使唐道襲蜀主之嬖臣也太子屢謔之於朝【嬖匹計翻又卑義翻謔迄却翻戲也】由是有隙互相訴於蜀主蜀主恐其交惡以道襲為山南西道節度使同平章事道襲薦宣徽北院使鄭頊為内樞密使頊受命之日即欲按道襲昆弟盜用内庫金帛道襲懼奏頊褊急不可大任【褊補辨翻】丙午出頊為果州刺史以宣徽南院使潘炕為内樞密使【為宗懿殺道襲張本炕口盎翻】 夏州都指揮使高宗益作亂殺節度使李彞昌將吏共誅宗益推彞昌族父蕃漢都指揮使李仁福為帥 【考異曰薛史仁福本党項拓拔氏唐末拓拔思恭以破黄巢功賜姓故仁福之族亦賜李歐陽史云不知其於思諫為親疎也按仁福諸子皆連彛字則於彞昌必父行也 按李仁福子孫彊盛遂為宋朝西邊之禍所謂西夏也】癸丑仁福以聞夏四月甲子以仁福為定難節度使【難乃旦翻】 丁卯宋州節度使衡王友諒獻瑞麥一莖三穗帝曰豐年為上瑞今宋州大水安用此為詔除本縣令名【本縣指產瑞麥之縣令力正翻】遣使詰責友諒【詰去吉翻】以兖海留後惠王友能代為宋州留後【歐陽史職方考梁都大梁徙宣武節度使於宋州薛史開平三年五月升宋州為宣武軍節鎮仍以亳輝潁為屬郡】友諒友能皆全昱子也【廣王全昱帝兄也】 帝以晉州刺史下邑華温琪拒晉兵有功欲賞之【華戶化翻姓也】會護國節度使冀王友謙上言晉絳邊河東乞别建節鎮壬申以晉絳沁三州為定昌軍以温琪為節度使【沁七鴆翻】 左金吾大將軍寇彥卿入朝至天津橋有民不避道投諸欄外而死【據歐史寇彥卿傳民姓梁名現】彥卿自首於帝【首式又翻】帝以彥卿才幹有功久在左右命以私財遺死者家以贖罪【遺唯季翻】御史司憲崔沂【唐高宗以御史大夫為大司憲蓋以御史執法之官故名之梁置御史司憲旣曰御史復曰司憲蓋不考名官之義也】劾奏彥卿殺人闕下請論如灋帝命彥卿分析【崔沂請依法論彥卿之罪帝欲寛之故使分析分析者使彥卿置對分疎辯析梁現致死之由劾戶㮣翻又戶得翻】彥卿對令從者舉置欄外【從才用翻】不意誤死帝欲以過失論沂奏在灋以勢力使令為首下手為從不得歸罪從者不鬬而故毆傷人【毆烏口翻】加傷罪一等不得為過失辛巳責授彥卿遊擊將軍左衛中郎將彥卿揚言有得崔沂首者賞錢萬緍沂以白帝帝使人謂彥卿【謂者告之語也】崔沂有毫髪傷我當族汝時功臣驕横【横戶孟翻】由是稍肅沂沆之弟也【崔沆見二百五十四卷唐僖宗廣明元年】 五月吳徐温母周氏卒將吏致祭為偶人高數尺衣以羅錦温曰此皆出民力奈何施於此而焚之宜解以衣貧者【偶人起於古之芻靈中世謂之俑則機械發動其手足耳目真有類於生人孔子曰始作俑者其無後乎正謂此也高居號翻衣於旣翻】未幾起復為内外馬步軍都軍使領潤州觀察使【幾居豈翻起復之制通古今疑之禮記子夏問曰三年之喪卒哭金革之事無避也者禮與其非禮與孔子曰吾聞諸老昔者魯公伯禽有為為之也今以三年之喪從其利者吾弗知也注云伯禽封於魯有徐戎作難卒哭而征之急王事也自漢以後不許二千石以上行三年喪魏晉聽行三年喪而大臣率有以奪情起復者習俗聞見以為當然莫之非也嗚呼此豈非孔子所謂以三年之喪從其利者乎若王莽之志不在喪徐温之起復所謂從其利者又難言也】 岐王屢求貨於蜀蜀主皆與之又求巴劒二州蜀主曰吾奉茂貞勤亦至矣若與之地是棄民也寧多與之貨乃復以絲茶布帛七萬遺之【復扶又翻遺唯季翻】 己亥以劉繼威為義昌節度使【劉守光之請也】癸丑天雄節度使兼中書令鄴貞莊王羅紹威卒【自開】<br />
<br />
  【平以後國主皆書殂其後書卒】詔以其子周翰為天雄留後 匡國節度使長樂忠敬王馮行襲疾篤【二年改許州忠武軍為匡國軍見上卷樂音洛】表請代者許州牙兵二千皆秦宗權餘黨帝深以為憂六月庚戌命崇政院直學士李珽馳往視行襲病【崇政院直學士即宋朝樞密直學士之職五代會要開平二年十一月置崇政院直學士二員選有政術文學者為之後又改為直崇政院李珽即諫成汭造大船者汭敗歸趙匡凝匡凝敗歸梁珽徒鼎翻】曰善諭朕意勿使亂我近鎮珽至許州謂將吏曰天子握百萬兵去此數舍【三十里為一舍九域志許州至洛陽三百一十五里】馮公忠純勿使上有所疑汝曹赤心奉國何憂不富貴由是衆莫敢異議行襲欲使人代受詔珽曰東首加朝服禮也【論語曰疾君視之東首加朝服拖紳受詔如見君天威不違顔咫尺之意首式又翻朝直遙翻】乃即臥内宣詔謂行襲曰公善自輔養勿視事此子孫之福也行襲泣謝遂解兩使印授珽【兩使印節度使觀察使印使疏吏翻】使代掌軍府帝聞之曰予固知珽能辦事馮族亦不亡矣庚辰行襲卒甲申以李珽權知匡國留後悉以行襲兵分隸諸校冒馮姓者皆還宗【冒馮姓者皆行襲之養子也使之歸宗所以消散其黨校戶教翻】 楚王殷求為天策上將詔加天策上將軍殷始開天策府以弟賨為左相存為右相【賨藏宗翻】殷遣將侵荆南軍于油口【油口在江陵府公安縣】高季昌擊破之斬首五千級逐北至白田而還【還從宣翻又如字】 吳水軍指揮使敖駢圍吉州刺史彭玕弟瑊於赤石【瑊古咸翻即吉州之赤石洞彭氏巢穴也】楚兵救瑊虜駢以歸秋七月蜀門下侍郎兼吏部尚書同平章事韋莊卒吳越王鏐表宦者周延誥等二十五人唐末避禍至此非劉韓之黨乞原之【劉韓謂劉季述韓全誨也】上曰此屬吾知其無罪但今革弊之初不欲置之禁掖可且留於彼諭以此意 岐王與邠涇二帥【邠帥李繼徽涇帥劉知俊帥所類翻】各遣使告晉請合兵攻定難節度使李仁福【難乃旦翻】晉王遣振武節度使周德威將兵會之合五萬衆圍夏州仁福嬰城拒守【夏戶雅翻】 八月以劉守光兼義昌節度使 【考異曰實錄是歲五月以義昌留後劉繼威為義昌節度使八月又云以守光兼義昌節度使不言置繼威於何處或者復為留後不然守光兼幽滄節度使繼威但為滄州節度使皆不可知今兩存之 余謂先是以劉守光子繼威為義昌節度使繼威童騃故復命守光兼領之蓋亦守光之志也】 鎮定自帝踐祚以來【祚當作阼】雖不輸常賦而貢獻甚勤會趙王鎔母何氏卒庚申遣使弔之且授起復官時鄰道弔客皆在館使者見晉使歸言於帝曰鎔濳與晉通鎮定勢彊恐終難制帝深然之【為遣兵圖鎮定二鎮附晉張本】 壬戌李仁福來告急甲子以河南尹兼中書令張全義為西京留守帝恐晉兵襲西京【晉兵自潞州下懷孟則西京震動矣】以宣化留後李思安為東北面行營都指揮使【據歐史職方考梁以鄧州為宣化軍】將兵萬人屯河陽【所以衛洛陽也】丙寅帝發洛陽己巳至陜【陜式冉翻】辛未以鎮國節度使楊師厚為西路行營招討使會感化節度使康懷貞將兵三萬屯三原【唐末以徐州數經叛亂廢武寧軍尋復以為感化軍歐史職方考徐州直注武寧軍華州注感化軍蓋梁改華州鎮國軍為感化軍也一曰感化軍陜州梁初改同州為忠武軍蓋劉知俊之叛又改同州為鎮國軍】帝憂晉兵出澤州逼懷州旣而聞其在綏銀磧中【晉兵趨夏州率自麟府濟河西至夏州按九域志麟州西至夏州三百五十里西南至銀州一百八十里綏州西至夏州四百里所謂磧中皆旱海及無定河川之地磧七迹翻】曰無足慮也甲申遣夾馬指揮使李遇劉綰自鄜延趨銀夏邀其歸路【梁置左右堅鋭夾馬突將趨七喻翻】 吳越王鏐築捍海石塘【今杭州城外瀕浙江皆有石塘上起六和塔下抵艮山門外皆錢氏所築】廣杭州城大脩臺館由是錢塘富庶盛於東南 九月己丑上發陜甲午至洛陽疾復作【復扶又翻】 李遇等至夏州岐晉兵皆解去 冬十月遣鎮國節度使楊師厚相州刺史李思安將兵屯澤州以圖上黨 吳越王鏐之巡湖州也留沈行思為巡檢使與盛師友俱歸行思謂同列陳瓌曰王若以師友為刺史何以處我【是年三月鏐巡湖州處昌呂翻】時瓌已得鏐密旨遣行思詣府【詣鎮海軍府】乃紿之曰何不自詣王所論之行思從之旣至數日瓌送其家亦至行思恨瓌賣己璆自衣錦軍歸【錢鏐生於臨安石鏡鎮里中有大木鏐幼與羣兒戲木下鏐坐大石指揮羣兒為隊伍號令有法羣兒憚之及貴唐昭宗改鏐所居鄉為廣義鄉里為勲貴里所居營曰衣錦營石鏡山為衣錦山鏐每遊衣錦軍宴故老山林皆覆以錦號其幼所常戲大木曰衣錦將軍作歌曰三郎還鄉兮衣錦衣父老遠來相追隨斗牛無孛又無欺吳越一王駟馬歸紿徒亥翻衣於旣翻】將吏迎謁行思取鍛槌擊瓌殺之【鍛都玩翻小冶也槌傳追翻】因詣鏐與師友論功【論逐高灃之功】奪左右槊欲刺師友【槊色角翻刺七亦翻】衆執之鏐斬行思以師友為婺州刺史 十一月己丑以寧國節度使同平章事王景仁充北面行營都指揮招討使潞州副招討使韓勍副之【勍渠京翻】以李思安為先鋒將趣上黨【趣七喻翻】尋遣景仁等屯魏州【意在圖鎮定不在上黨也】楊師厚還陜 蜀主更太子宗懿名曰元坦【更工衡翻】庚戌立假子宗裕為通王宗範為夔王宗鐬為昌王【鐬火外翻】宗壽為嘉王宗翰為集王立其子宗仁為普王宗輅為雅王宗紀為褒王宗智為榮王宗澤為興王宗鼎為彭王宗傑為信王宗衍為鄭王初唐末宦官典兵者多養軍中壯士為子以自彊【如田令孜楊復恭之類】由是諸將亦傚之而蜀主尤多惟宗懿等九人及宗特宗平真其子宗裕宗鐬宗壽皆其族人宗翰姓孟蜀主之姊子宗範姓張其母周氏為蜀主妾自餘假子百二十人皆功臣雖冒姓連名而不禁昏姻【史言假父假子皆以利合非人倫之正】 上疾小愈辛亥校獵於伊洛之間【伊洛二水之間也】 上疑趙王鎔貳於晉【先有疑心因晉使在館愈疑之】且欲因鄴王紹威卒除移鎮定會燕王守光發兵屯淶水欲侵定州上遣供奉官杜廷隱丁廷徽監魏博兵三千分屯深冀【唐末置東頭供奉官西頭供奉官後皆為西班寄祿】聲言恐燕兵南寇助趙守禦又云分兵就食趙將石公立戍深州白趙王鎔請拒之鎔遽命開門移公立於外以避之公立出門【出深州城門】指城而泣曰朱氏滅唐社稷三尺童子知其為人而我王猶恃姻好以長者期之【鎔子昭祚娶梁女見二百六十二卷唐昭宗光化三年好呼到翻長知兩翻】此所謂開門揖盜者也惜乎此城之人今為虜矣梁人有亡奔真定以其謀告鎔者鎔大懼又不敢先自絶但遣使詣洛陽訴稱燕兵已還與定州講和如故【定州謂義武節度使王處直也】深冀民見魏博兵入奔走驚駭乞召兵還上遣使詣真定慰諭之未幾廷隱等閉門盡殺趙戍兵乘城拒守【幾居豈翻】鎔始命石公立攻之不克乃遣使求援於燕晉鎔使者至晉陽義武節度使王處直使者亦至欲共推晉王為盟主合兵攻梁晉王會將佐謀之皆曰鎔久臣朱温【唐昭宗光化三年王鎔服於朱全忠及其受禪遂臣事之】歲輸重賂【輸舂遇翻】結以昏姻其交深矣此必詐也宜徐觀之王曰彼亦擇利害而為之耳王氏在唐世猶或臣或叛【謂王武俊承宗及王廷湊也】况肯終為朱氏之臣乎彼朱温之女何如壽安公主【王鎔曾祖元逵尚唐絳王悟女壽安公主】今救死不贍何顧昏姻我若疑而不救正墮朱氏計中宜趣發兵赴之【趣讀曰促】晉趙叶力破梁必矣乃發兵遣周德威將之出井陘屯趙州【史言晉王識虛實見兵勢陘音刑】鎔使者至幽州燕王守光方獵幕僚孫鶴馳詣野謂守光曰趙人來乞師此天欲成王之功業也守光曰何故對曰比常患其與朱温膠固【比毗至翻近也】温之志非盡吞河朔不已今彼自為讎敵王若與之并力破梁則鎮定皆斂衽而朝燕矣【鎮王鎔定王處直朝直遙翻】王不出師但恐晉人先我矣守光曰王鎔數負約【先悉薦翻數所角翻】今使之與梁自相弊吾可以坐承其利【自戰國以來卞莊刺虎鷸蚌相持犬兎俱斃皆此說也苟不能審勢見機則此說誤人多矣】又何救焉趙使者交錯于路守光竟不為出兵自是鎮定復稱唐天祐年號復以武順為成德軍【鎮定臣梁稱開平年號避梁廟諱改成德軍為武順軍旣與梁猜阻故年號軍號皆復唐之舊】司天言來月太隂虧不利宿兵於外上召王景仁等<br />
<br />
  還洛陽十二月己未上聞趙與晉合晉兵已屯趙州乃命王景仁等將兵擊之庚申景仁等自河陽度河會羅周翰兵合四萬軍於邢洺 虔州刺史盧光稠疾病欲以位授譚全播【疾革曰病譚全播與盧光稠同起兵者也】全播不受光稠卒其子韶州刺史延昌來奔喪全播立而事之吳遣使拜延昌虔州刺史延昌受之亦因楚王殷密通表於梁曰我受淮南官以緩其謀耳必為朝廷經略江西【為于偽翻盧延昌此言欲得鎮南旌節耳】丙寅以延昌為鎮南留後延昌表其將廖爽為韶州刺史【廖力救翻今讀如料姓苑云周王子伯廖之後後漢有廖湛】爽贛人也【贛音紺】吳淮南節度判官嚴可求請置制置使於新淦縣【新淦漢古縣唐屬吉州九域志在虔州北六百里宋白曰縣南有子淦山因名淦音紺又音甘】遣兵戍之以圖虔州每更代輒濳益其兵虔人不之覺也【更工衡翻為淮南併虔州張本】 庚午蜀主以御史中丞周庠戶部侍郎判度支庾傳素並為中書侍郎同平章事 太常卿李燕等刋定梁律令格式癸酉行之【按五代會要新刪定令三十卷式二十卷格一十卷律并目錄一十三卷律疏三十卷共一百三卷目為大梁新定格式律令頒下施行】 丁丑王景仁等進軍柏鄉 辛巳蜀大赦改明年元曰永平趙王鎔復告急於晉【復扶又翻王景仁等之軍侵逼故復告急】晉王以蕃<br />
<br />
  漢副總管李存審守晉陽自將兵自贊皇東下【贊皇縣以山得名宋白曰贊皇本漢鄗縣地隋開皇六年置贊皇縣縣南有贊皇山因名按九域志宋廢贊皇縣為鎮屬高邑縣高邑縣在趙州西南四十二里】王處直遣將將兵以從【從才用翻】辛巳晉王至趙州與周德威合獲梁芻蕘者二百人【刈草曰芻采薪曰蕘蕘如招翻】問之曰初發洛陽梁主有何號令對曰梁主戒上將云鎮州反覆終為子孫之患今悉以精兵付汝鎮州雖以鐵為城必為我取之【必為于偽翻】晉王命送於趙【使趙人聞此言以堅其附晉之心】壬午晉王進軍距柏鄉三十里遣周德威等以胡騎迫梁營挑戰【挑徒了翻】梁兵不出癸未復進距柏鄉五里營於野河之北又遣胡騎廹梁營馳射且詬之【復扶又翻詬古候翻又許候翻】梁將韓勍等將步騎三萬分三道追之鎧胄皆被繒綺鏤金銀光彩炫耀【被皮義翻繒慈陵翻鏤郎豆翻炫熒絹翻】晉人望之奪氣周德威謂李存璋曰梁人志不在戰徒欲曜兵耳不挫其鋭則吾軍不振乃徇於軍曰彼皆汴州天武軍【五代會要曰開平元年四月改左右長直為左右龍虎軍左右内衙為左右羽林軍左右堅銳夾馬突將為左右神武軍左右親隨軍將馬軍為左右龍驤軍其年九月置左右天興左右廣勝軍仍以親王為軍使二年十月置左右神捷軍十二月改左右天武為左右龍虎軍左右龍虎為左右天武軍左右天威為左右羽林軍左右羽林為左右天威軍左右英武為左右神武軍左右神武為左右英武軍前朝置神虎等六軍謂之衛士至是以天武天威英武等六軍易其軍號而任勲舊焉】屠酤傭販之徒耳衣鎧雖鮮十不能當汝一擒獲一夫足以自富此乃奇貨不可失也德威自引千餘精騎擊其兩端【陳有厚薄中軍堅厚不可衝擊擊其兩端以其薄也】左右馳突出入數四俘獲百餘人且戰且却距野河而止梁兵亦退德威言於晉王曰賊勢甚盛宜按兵以待其衰王曰吾孤軍遠來救人之急三鎮烏合利於速戰【鎮定河東是為三鎮言三鎮之兵合而為一當乘初至之鋭以破敵曠日持久情見勢屈敵人聞之其心必離】公乃欲按兵持重何也德威日鎮定之兵長於守城短於野戰且吾所恃者騎兵利於平原廣野可以馳突今壓賊壘門騎無所展其足且衆寡不敵使彼知吾虛實則事危矣王不悦退卧帳中諸將莫敢言德威往見張承業曰大王驟勝而輕敵【謂夾寨之勝也】不量力而務速戰【量音良】今去賊咫尺所限者一水耳【謂野河之水也】彼若造浮橋以薄我我衆立盡矣不若退軍高邑【高邑漢鄗縣光武更名高邑唐屬趙州九域志在州西南四十二里在柏鄉北三十餘里】誘賊離營【誘音酉離力智翻】彼出則歸彼歸則出别以輕騎掠其饋餉不過踰月破之必矣承業入褰帳撫王曰【褰起虔翻】此豈王安寢時耶周德威老將知兵其言不可忽也王蹶然興曰予方思之時梁兵閉壘不出有降者詰之【降戶江翻詰去吉翻】曰景仁方多造浮橋王謂德威曰果如公言是日拔營退保高邑 辰州蠻酋宋鄴溆州蠻酋潘金盛恃其所居深險數擾楚邊至是鄴寇湘鄉【酋慈由翻溆音叙數所角翻宋白曰秦置黔中郡於今沅陵縣西二十里漢改黔中郡為武陵郡建武二十五年宗均受羣蠻降置辰陽縣隋為辰州因辰溪為名唐貞觀八年分辰州龍標縣置巫州天授三年改沅州大歷五年改溆州唐武德四年分衡山置湘鄉縣屬潭州九域志在州西南一百五十五里楚書作潘全盛】金盛寇武岡【宋白曰晋武帝分都梁立武岡縣今崗東五十里有漢都梁故城是也後漢武陵蠻為漢所伐來保此崗故謂之武崗郡國志云武岡接武陵因以為名】楚王殷遣昭州刺史呂師周將衡山兵五千討之【考異曰湖湘故事呂師周斬潘金晟於武岡其年十月十一日辰州宋鄴溆州昌師益一時歸投於馬氏今】<br />
<br />
  【從十國紀年】 寧遠節度使龐巨昭高州防禦使劉昌魯皆唐官也黃巢之寇嶺南也巨昭為容管觀察使昌魯為高州刺史帥羣蠻據險以拒之【帥讀曰率】巢衆不敢入境唐嘉其功置寧遠軍於容州以巨昭為節度使【按通鑑唐昭宗乾寧四年置寧遠軍於容州以李克用大將蓋寓領節度使考之新書方鎮表容州置節鎮亦在是年龎巨昭建節當在是年之後】以昌魯為高州防禦使及劉隱據嶺南二州不從隱遣弟嚴攻高州昌魯大破之又攻容州亦不克昌魯自度終非隱敵【度徒洛翻】是歲致書請自歸於楚楚王殷大喜遣横州刺史姚彥章將兵迎之彥章至容州禆將莫彥昭說巨昭曰湖南兵遠來疲乏宜撤儲偫【說式芮翻撤直列翻偫直里翻】弃城濳於山谷以待之彼必入城我以全軍掩之彼外無繼援可擒也巨昭曰馬氏方興今雖勝之後將何如不若具牛酒迎之彦昭不從巨昭殺之舉州迎降 【考異曰湖湘故事龐巨曦本唐末邕容等州防禦使聞馬氏令公以征南步軍指揮使李瓊知桂州軍事領兵士收服嶺外昭梧象柳宜蒙賀桂等州巨曦聞此雄勢謂諸首領曰李瓊有破竹之勢若長駈兵馬此來侵吞吾境其將奈何時容南指揮使莫彥昭對曰李瓊兵馬其勢已雄必然輕敵今欲燒燬城内軍儲且各入山峒抛州城與李瓊候纔入州却依前出諸山峒兵士復攻之堅守旬月之間城内必無軍糧外無救應方可制造攻具再攻擊之必取勝也龎巨曦曰吾每至中宵獨占氣象馬氏合當五十餘年興霸湖外苟五十餘年對壘安知孰非是以憂疑不暇遂至深夜斬莫彥昭於其第明日以其故密走事宜於湖南又曰天復末甲子十有二月容南龐巨曦深慮廣南劉巖不道加害於已遂差小吏間路密持書款歸於馬氏是時湖南遣澧州刺史姚彥章領馬步軍八千徑往容南巨曦遂帥萬餘衆歸於馬氏又曰高州防禦使劉昌魯以廣南先主劉巖欲并吞嶺外數召昌魯欲籍没其家族昌魯知之乃刺血寫書投馬氏具述懸急湖南遂遣捉生指揮使張可求部轄兵馬於界首應接一行三千餘口歸于馬氏今從十國紀年】彥章進至高州【九域志容州東南至高州二百八十二里】以兵援送巨昭昌魯之族及士卒千餘人歸長沙楚王殷以彥章知容州事【為姚彥章不能守容州張本】以昌魯為永順節度副使【馬殷并朗州奏改武貞軍為永順軍】昌魯鄴人也<br />
<br />
  乾化元年【按歐史是年五月甲申朔大赦改元】春正月丙戌朔日有食之【考異曰李昊蜀書丁亥朔日食今從實錄等諸書】 柏鄉比不儲芻【比毗至翻近也言近】<br />
<br />
  【時趙人不儲芻於柏鄉蓋亦虞梁兵之至以資敵也】梁兵刈芻自給晉人日以遊軍抄之【抄楚交翻】梁兵不出周德威使胡騎環營馳射而詬之【環音宦詬古翻】梁兵疑有伏愈不敢出剉屋茅坐席以飼馬【剉寸臥翻飼祥吏翻】馬多死丁亥周德威與别將史建瑭李嗣源將精騎三千壓梁壘門而詬之王景仁韓勍怒悉衆而出德威等轉戰至高邑南李存璋以步兵陳於野河之上【陳讀曰陣下同】梁軍横亘數里競前奪橋鎮定步兵禦之勢不能支晉王謂匡衛都指揮使李建及曰賊過橋則不可復制矣建及選卒二百援鎗大譟【復扶又翻援于元翻】力戰却之建及許州人姓王李罕之之假子也【薛史李建及本姓王少事李罕之光啓中罕之選部下驍勇百人以獻李克用建及在籍中後以功賜姓名】晉王登高丘以望曰梁兵爭進而囂【囂虛驕翻又牛刀翻】我兵整而靜我必勝戰自己至午勝負未決晉王謂周德威曰兩軍已合勢不可離我之興亡在此一舉我為公先登公可繼之德威叩馬而諫曰觀梁兵之勢可以勞逸制之未易以力勝也【為于偽翻易以豉翻】彼去營三十餘里雖挾糗糧亦不暇食日昳之後飢渴内迫矢刃外交【糗去久翻昳徒結翻日昃也】士卒勞倦必有退志當是時我以精騎乘之必大捷於今未可也王乃止【梁晉爭天下周德威以勇聞是難能也然觀其制勝以計不以勇是又難能矣】時魏滑之兵陳於東宋汴之兵陳於西至晡梁軍未食士無鬬志景仁等引兵稍却周德威疾呼曰【呼火故翻】梁兵走矣晉兵大譟爭進魏滑兵先退李嗣源帥衆譟於西陳之前曰東陳已走爾何久留梁兵互相驚怖遂大潰【置陳延亘東西不相知為敵所譟故驚怖而潰帥讀曰率怖普布翻】李存璋引步兵乘之呼曰梁人亦吾人也父兄子弟餉軍者勿殺於是戰士悉解甲投兵而弃之囂聲動天地趙人以深冀之憾不顧剽掠【憾梁遣杜廷隱等殺深冀戍兵也剽匹妙翻】但奮白刃追之梁之龍驤神捷精兵殆盡【薛史本紀開成二年以尹皓部下五百人為神捷軍】自野河至柏鄉僵尸蔽地【僵居良翻】王景仁韓勍李思安以數十騎走【王景仁嘗以勞逸制梁兵而不知又為周德威以勞逸制之也】晉兵夜至柏鄉梁兵已去弃糧食資財器械不可勝計【勝音升】凡斬首二萬級李嗣源等奔至邢州【九域志自柏鄉西南至邢州一百五十餘里】河朔大震保義節度使王檀嚴備然後開城納敗卒給以資糧散遣歸本道晉王收兵屯趙州杜廷隱等聞梁兵敗弃深冀而去悉驅二州丁壯為奴婢老弱者阬之城中存者壞垣而已癸巳復以楊師厚為北面都招討使將兵屯河陽收集散兵旬餘得萬人己亥晉王遣周德威史建瑭將三千騎趣澶魏【趣七喻翻澶魏二州名澶市連翻】張承業李存璋以步兵攻邢州自以大軍繼之移檄河北州縣諭以利害帝遣别將徐仁溥將兵千人自西山夜入邢州【西山即太行連延至上黨諸山】助王檀城守己酉罷王景仁招討使落平章事【以其敗也】蜀主之女普慈公主嫁岐王從子秦州節度使繼崇【蜀主】<br />
<br />
  【以蕭梁郡名封其女宋白曰梁武帝置普慈郡於普州安岳縣從才用翻】公主遣宦者宋光嗣以絹書遺蜀主【遺唯季翻】言繼崇驕矜嗜酒求歸成都蜀主召公主歸寧【已嫁之女父母在則有時而歸寧寧安也】辛亥公主至成都蜀主留之以宋光嗣為閤門南院使岐王怒始與蜀絶【為蜀伐岐張本】光嗣福州人也 呂師周攀藤緣崖入飛山洞襲潘金盛擒送武岡斬之【飛山在今靖州北十五里比諸山為最高峻四面絶壁千仭環山有壕塹其遺趾尚存】 二月己未晉王至魏州攻之不克上以羅周翰年少且忌其舊將佐【謂羅紹威之元從將佐也少詩照翻】庚申以戶部尚書李振為天雄節度副使命杜廷隱將兵千人衛之【先是帝以羅紹威之請撫案動容至此心術露矣】自楊劉濟河間道夜入魏州助周翰城守【間古莧翻守手又翻】癸亥晉王觀河於黎陽梁兵萬餘將度河聞晉王至皆弃舟而去【史言梁兵懼晉王之甚】帝召蔡州刺史張慎思至洛陽久未除代蔡州右廂<br />
<br />
  指揮使劉行琮作亂縱兵焚掠將奔淮南順化指揮使王存儼誅行琮撫遏其衆自領州事以衆情馳奏時東京留守博王友文不先請遽發兵討之兵至鄢陵【九域志鄢陵縣在大梁東南一百六十里】帝曰存儼方懼若臨之以兵則飛去矣馳使召還甲子授存儼權知蔡州事 乙丑周德威自臨清攻貝州拔夏津高唐【夏津本古鄃縣高天寶元年更名夏津屬貝州九域志在魏州東北二百五十里】攻博州拔東武朝城【漢東郡東武陽縣後魏曰武陽唐開元七年更名朝城屬魏州故朝城縣管内猶有地名東武九域志朝城在魏州東南八十里】攻澶州刺史張可臻弃城走帝斬之德威進攻黎陽拔臨河淇門逼衛州掠新鄉共城【隋分汲獲嘉二縣地於古新樂城置新鄉縣共城漢共縣地唐志屬衛州九域志共城在州西北五十五里共音恭】庚午帝親帥軍屯白司馬阪以備之【史言晉兵乘勝聲勢之盛梁祖選富家子有材力者置帳下號廳子都薛居正曰大祖置廳子都最為親軍白司馬阪在洛陽城北帥讀曰率】 盧龍義昌節度使兼中書令燕王守光旣克滄州【去年正月克滄州】自謂得天助淫虐滋甚每刑人必置諸鐵籠以火逼之又為鐵刷刷人面聞梁兵敗於柏鄉使人謂趙王鎔及王處直曰聞二鎮與晉王破梁兵舉軍南下僕亦有精騎三萬欲自將之為諸公啓行【詩元戎十乘以先啓行註云啓突敵陣之前行行戶剛翻】然四鎮連兵必有盟主僕若至彼何以處之【四鎮謂并幽鎮定處昌呂翻】鎔患之遣使告于晉王晉王笑曰趙人告急守光不能出一卒以救之及吾成功乃復欲以兵威離間二鎮【復扶又翻間古莧翻】愚莫甚焉諸將曰雲代與燕接境彼若擾我城戍動搖人情吾千里出征緩急難應此亦腹心之患也不若先取守光然後可以專意南討王曰善【為晉攻燕滅之張本】會楊師厚自磁相引兵救邢魏壬申晉解圍去師厚追之逾漳水而還邢州圍亦解【先解魏州圍又解邢州圍磁祥之翻相息亮翻還音旋又如字】師厚留屯魏州趙王鎔自來謁晉王於趙州【九域志鎮州南至趙州九十五里】大犒將士【犒苦到翻】自是遣其養子德明將三十七都常從晉王征討德明本姓張名文禮燕人也【張文禮後遂殺王鎔而亂鎮州】壬午晉王發趙州歸晉陽留周德威等將三千人戍趙州<br />
<br />
  資治通鑑卷二百六十七  <br>
   </div> 

<script src="/search/ajaxskft.js"> </script>
 <div class="clear"></div>
<br>
<br>
 <!-- a.d-->

 <!--
<div class="info_share">
</div> 
-->
 <!--info_share--></div>   <!-- end info_content-->
  </div> <!-- end l-->

<div class="r">   <!--r-->



<div class="sidebar"  style="margin-bottom:2px;">

 
<div class="sidebar_title">工具类大全</div>
<div class="sidebar_info">
<strong><a href="http://www.guoxuedashi.com/lsditu/" target="_blank">历史地图</a></strong>  
<a href="http://www.880114.com/" target="_blank">英语宝典</a>  
<a href="http://www.guoxuedashi.com/13jing/" target="_blank">十三经检索</a> 
<br><strong><a href="http://www.guoxuedashi.com/gjtsjc/" target="_blank">古今图书集成</a></strong> 
<a href="http://www.guoxuedashi.com/duilian/" target="_blank">对联大全</a> <strong><a href="http://www.guoxuedashi.com/xiangxingzi/" target="_blank">象形文字典</a></strong> 

<br><a href="http://www.guoxuedashi.com/zixing/yanbian/">字形演变</a>  <strong><a href="http://www.guoxuemi.com/hafo/" target="_blank">哈佛燕京中文善本特藏</a></strong>
<br><strong><a href="http://www.guoxuedashi.com/csfz/" target="_blank">丛书&方志检索器</a></strong> <a href="http://www.guoxuedashi.com/yqjyy/" target="_blank">一切经音义</a>  

<br><strong><a href="http://www.guoxuedashi.com/jiapu/" target="_blank">家谱族谱查询</a></strong>  <strong><a href="http://shufa.guoxuedashi.com/sfzitie/" target="_blank">书法字帖欣赏</a></strong> 
<br>

</div>
</div>


<div class="sidebar" style="margin-bottom:0px;">

<font style="font-size:22px;line-height:32px">QQ交流群9:489193090</font>


<div class="sidebar_title">手机APP 扫描或点击</div>
<div class="sidebar_info">
<table>
<tr>
	<td width=160><a href="http://m.guoxuedashi.com/app/" target="_blank"><img src="/img/gxds-sj.png" width="140"  border="0" alt="国学大师手机版"></a></td>
	<td>
<a href="http://www.guoxuedashi.com/download/" target="_blank">app软件下载专区</a><br>
<a href="http://www.guoxuedashi.com/download/gxds.php" target="_blank">《国学大师》下载</a><br>
<a href="http://www.guoxuedashi.com/download/kxzd.php" target="_blank">《汉字宝典》下载</a><br>
<a href="http://www.guoxuedashi.com/download/scqbd.php" target="_blank">《诗词曲宝典》下载</a><br>
<a href="http://www.guoxuedashi.com/SiKuQuanShu/skqs.php" target="_blank">《四库全书》下载</a><br>
</td>
</tr>
</table>

</div>
</div>


<div class="sidebar2">
<center>


</center>
</div>

<div class="sidebar"  style="margin-bottom:2px;">
<div class="sidebar_title">网站使用教程</div>
<div class="sidebar_info">
<a href="http://www.guoxuedashi.com/help/gjsearch.php" target="_blank">如何在国学大师网下载古籍?</a><br>
<a href="http://www.guoxuedashi.com/zidian/bujian/bjjc.php" target="_blank">如何使用部件查字法快速查字?</a><br>
<a href="http://www.guoxuedashi.com/search/sjc.php" target="_blank">如何在指定的书籍中全文检索?</a><br>
<a href="http://www.guoxuedashi.com/search/skjc.php" target="_blank">如何找到一句话在《四库全书》哪一页?</a><br>
</div>
</div>


<div class="sidebar">
<div class="sidebar_title">热门书籍</div>
<div class="sidebar_info">
<a href="/so.php?sokey=%E8%B5%84%E6%B2%BB%E9%80%9A%E9%89%B4&kt=1">资治通鉴</a> <a href="/24shi/"><strong>二十四史</strong></a>&nbsp; <a href="/a2694/">野史</a>&nbsp; <a href="/SiKuQuanShu/"><strong>四库全书</strong></a>&nbsp;<a href="http://www.guoxuedashi.com/SiKuQuanShu/fanti/">繁体</a>
<br><a href="/so.php?sokey=%E7%BA%A2%E6%A5%BC%E6%A2%A6&kt=1">红楼梦</a> <a href="/a/1858x/">三国演义</a> <a href="/a/1038k/">水浒传</a> <a href="/a/1046t/">西游记</a> <a href="/a/1914o/">封神演义</a>
<br>
<a href="http://www.guoxuedashi.com/so.php?sokeygx=%E4%B8%87%E6%9C%89%E6%96%87%E5%BA%93&submit=&kt=1">万有文库</a> <a href="/a/780t/">古文观止</a> <a href="/a/1024l/">文心雕龙</a> <a href="/a/1704n/">全唐诗</a> <a href="/a/1705h/">全宋词</a>
<br><a href="http://www.guoxuedashi.com/so.php?sokeygx=%E7%99%BE%E8%A1%B2%E6%9C%AC%E4%BA%8C%E5%8D%81%E5%9B%9B%E5%8F%B2&submit=&kt=1"><strong>百衲本二十四史</strong></a>  <a href="http://www.guoxuedashi.com/so.php?sokeygx=%E5%8F%A4%E4%BB%8A%E5%9B%BE%E4%B9%A6%E9%9B%86%E6%88%90&submit=&kt=1"><strong>古今图书集成</strong></a>
<br>

<a href="http://www.guoxuedashi.com/so.php?sokeygx=%E4%B8%9B%E4%B9%A6%E9%9B%86%E6%88%90&submit=&kt=1">丛书集成</a> 
<a href="http://www.guoxuedashi.com/so.php?sokeygx=%E5%9B%9B%E9%83%A8%E4%B8%9B%E5%88%8A&submit=&kt=1"><strong>四部丛刊</strong></a>  
<a href="http://www.guoxuedashi.com/so.php?sokeygx=%E8%AF%B4%E6%96%87%E8%A7%A3%E5%AD%97&submit=&kt=1">說文解字</a> <a href="http://www.guoxuedashi.com/so.php?sokeygx=%E5%85%A8%E4%B8%8A%E5%8F%A4&submit=&kt=1">三国六朝文</a>
<br><a href="http://www.guoxuedashi.com/so.php?sokeytm=%E6%97%A5%E6%9C%AC%E5%86%85%E9%98%81%E6%96%87%E5%BA%93&submit=&kt=1"><strong>日本内阁文库</strong></a> <a href="http://www.guoxuedashi.com/so.php?sokeytm=%E5%9B%BD%E5%9B%BE%E6%96%B9%E5%BF%97%E5%90%88%E9%9B%86&ka=100&submit=">国图方志合集</a> <a href="http://www.guoxuedashi.com/so.php?sokeytm=%E5%90%84%E5%9C%B0%E6%96%B9%E5%BF%97&submit=&kt=1"><strong>各地方志</strong></a>

</div>
</div>


<div class="sidebar2">
<center>

</center>
</div>
<div class="sidebar greenbar">
<div class="sidebar_title green">四库全书</div>
<div class="sidebar_info">

《四库全书》是中国古代最大的丛书,编撰于乾隆年间,由纪昀等360多位高官、学者编撰,3800多人抄写,费时十三年编成。丛书分经、史、子、集四部,故名四库。共有3500多种书,7.9万卷,3.6万册,约8亿字,基本上囊括了古代所有图书,故称“全书”。<a href="http://www.guoxuedashi.com/SiKuQuanShu/">详细>>
</a>

</div> 
</div>

</div>  <!--end r-->

</div>
<!-- 内容区END --> 

<!-- 页脚开始 -->
<div class="shh">

</div>

<div class="w1180" style="margin-top:8px;">
<center><script src="http://www.guoxuedashi.com/img/plus.php?id=3"></script></center>
</div>
<div class="w1180 foot">
<a href="/b/thanks.php">特别致谢</a> | <a href="javascript:window.external.AddFavorite(document.location.href,document.title);">收藏本站</a> | <a href="#">欢迎投稿</a> | <a href="http://www.guoxuedashi.com/forum/">意见建议</a> | <a href="http://www.guoxuemi.com/">国学迷</a> | <a href="http://www.shuowen.net/">说文网</a><script language="javascript" type="text/javascript" src="https://js.users.51.la/17753172.js"></script><br />
  Copyright &copy; 国学大师 古典图书集成 All Rights Reserved.<br>
  
  <span style="font-size:14px">免责声明:本站非营利性站点,以方便网友为主,仅供学习研究。<br>内容由热心网友提供和网上收集,不保留版权。若侵犯了您的权益,来信即刪。scp168@qq.com</span>
  <br />
ICP证:<a href="http://www.beian.miit.gov.cn/" target="_blank">鲁ICP备19060063号</a></div>
<!-- 页脚END --> 
<script src="http://www.guoxuedashi.com/img/plus.php?id=22"></script>
<script src="http://www.guoxuedashi.com/img/tongji.js"></script>

</body>
</html>
