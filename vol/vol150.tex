<!DOCTYPE html PUBLIC "-//W3C//DTD XHTML 1.0 Transitional//EN" "http://www.w3.org/TR/xhtml1/DTD/xhtml1-transitional.dtd">
<html xmlns="http://www.w3.org/1999/xhtml">
<head>
<meta http-equiv="Content-Type" content="text/html; charset=utf-8" />
<meta http-equiv="X-UA-Compatible" content="IE=Edge,chrome=1">
<title>資治通鑒_151-資治通鑑卷一百五十_151-資治通鑑卷一百五十</title>
<meta name="Keywords" content="資治通鑒_151-資治通鑑卷一百五十_151-資治通鑑卷一百五十">
<meta name="Description" content="資治通鑒_151-資治通鑑卷一百五十_151-資治通鑑卷一百五十">
<meta http-equiv="Cache-Control" content="no-transform" />
<meta http-equiv="Cache-Control" content="no-siteapp" />
<link href="/img/style.css" rel="stylesheet" type="text/css" />
<script src="/img/m.js?2020"></script> 
</head>
<body>
 <div class="ClassNavi">
<a  href="/24shi/">二十四史</a> | <a href="/SiKuQuanShu/">四库全书</a> | <a href="http://www.guoxuedashi.com/gjtsjc/"><font  color="#FF0000">古今图书集成</font></a> | <a href="/renwu/">历史人物</a> | <a href="/ShuoWenJieZi/"><font  color="#FF0000">说文解字</a></font> | <a href="/chengyu/">成语词典</a> | <a  target="_blank"  href="http://www.guoxuedashi.com/jgwhj/"><font  color="#FF0000">甲骨文合集</font></a> | <a href="/yzjwjc/"><font  color="#FF0000">殷周金文集成</font></a> | <a href="/xiangxingzi/"><font color="#0000FF">象形字典</font></a> | <a href="/13jing/"><font  color="#FF0000">十三经索引</font></a> | <a href="/zixing/"><font  color="#FF0000">字体转换器</font></a> | <a href="/zidian/xz/"><font color="#0000FF">篆书识别</font></a> | <a href="/jinfanyi/">近义反义词</a> | <a href="/duilian/">对联大全</a> | <a href="/jiapu/"><font  color="#0000FF">家谱族谱查询</font></a> | <a href="http://www.guoxuemi.com/hafo/" target="_blank" ><font color="#FF0000">哈佛古籍</font></a> 
</div>

 <!-- 头部导航开始 -->
<div class="w1180 head clearfix">
  <div class="head_logo l"><a title="国学大师官网" href="http://www.guoxuedashi.com" target="_blank"></a></div>
  <div class="head_sr l">
  <div id="head1">
  
  <a href="http://www.guoxuedashi.com/zidian/bujian/" target="_blank" ><img src="http://www.guoxuedashi.com/img/top1.gif" width="88" height="60" border="0" title="部件查字,支持20万汉字"></a>


<a href="http://www.guoxuedashi.com/help/yingpan.php" target="_blank"><img src="http://www.guoxuedashi.com/img/top230.gif" width="600" height="62" border="0" ></a>


  </div>
  <div id="head3"><a href="javascript:" onClick="javascript:window.external.AddFavorite(window.location.href,document.title);">添加收藏</a>
  <br><a href="/help/setie.php">搜索引擎</a>
  <br><a href="/help/zanzhu.php">赞助本站</a></div>
  <div id="head2">
 <a href="http://www.guoxuemi.com/" target="_blank"><img src="http://www.guoxuedashi.com/img/guoxuemi.gif" width="95" height="62" border="0" style="margin-left:2px;" title="国学迷"></a>
  

  </div>
</div>
  <div class="clear"></div>
  <div class="head_nav">
  <p><a href="/">首页</a> | <a href="/ShuKu/">国学书库</a> | <a href="/guji/">影印古籍</a> | <a href="/shici/">诗词宝典</a> | <a   href="/SiKuQuanShu/gxjx.php">精选</a> <b>|</b> <a href="/zidian/">汉语字典</a> | <a href="/hydcd/">汉语词典</a> | <a href="http://www.guoxuedashi.com/zidian/bujian/"><font  color="#CC0066">部件查字</font></a> | <a href="http://www.sfds.cn/"><font  color="#CC0066">书法大师</font></a> | <a href="/jgwhj/">甲骨文</a> <b>|</b> <a href="/b/4/"><font  color="#CC0066">解密</font></a> | <a href="/renwu/">历史人物</a> | <a href="/diangu/">历史典故</a> | <a href="/xingshi/">姓氏</a> | <a href="/minzu/">民族</a> <b>|</b> <a href="/mz/"><font  color="#CC0066">世界名著</font></a> | <a href="/download/">软件下载</a>
</p>
<p><a href="/b/"><font  color="#CC0066">历史</font></a> | <a href="http://skqs.guoxuedashi.com/" target="_blank">四库全书</a> |  <a href="http://www.guoxuedashi.com/search/" target="_blank"><font  color="#CC0066">全文检索</font></a> | <a href="http://www.guoxuedashi.com/shumu/">古籍书目</a> | <a   href="/24shi/">正史</a> <b>|</b> <a href="/chengyu/">成语词典</a> | <a href="/kangxi/" title="康熙字典">康熙字典</a> | <a href="/ShuoWenJieZi/">说文解字</a> | <a href="/zixing/yanbian/">字形演变</a> | <a href="/yzjwjc/">金 文</a> <b>|</b>  <a href="/shijian/nian-hao/">年号</a> | <a href="/diming/">历史地名</a> | <a href="/shijian/">历史事件</a> | <a href="/guanzhi/">官职</a> | <a href="/lishi/">知识</a> <b>|</b> <a href="/zhongyi/">中医中药</a> | <a href="http://www.guoxuedashi.com/forum/">留言反馈</a>
</p>
  </div>
</div>
<!-- 头部导航END --> 
<!-- 内容区开始 --> 
<div class="w1180 clearfix">
  <div class="info l">
   
<div class="clearfix" style="background:#f5faff;">
<script src='http://www.guoxuedashi.com/img/headersou.js'></script>

</div>
  <div class="info_tree"><a href="http://www.guoxuedashi.com">首页</a> > <a href="/SiKuQuanShu/fanti/">四库全书</a>
 > <h1>资治通鉴</h1> <!--         下载:【右键另存为】即可 --></div>
  <div class="info_content zj clearfix">
  
<div class="info_txt clearfix" id="show">
<center style="font-size:24px;">151-資治通鑑卷一百五十</center>
    資治通鑑卷一百五十  宋 司馬光 撰<br />
<br />
  胡三省 音註<br />
<br />
  梁紀六【起閼逢執徐盡旃蒙大荒落凡二年】<br />
<br />
  高祖武皇帝六<br />
<br />
  普通五年春正月辛丑魏主祀南郊 三月魏以臨淮王彧都督北討諸軍事討破六韓拔陵【拔陵反見上卷上年】夏四月高平鎮民赫連恩等反推敕勒酋長胡琛為高平王【酋慈秋翻長知兩翻】攻高平鎮以應拔陵魏將盧祖遷擊破之琛北走【將即亮翻】衛可孤攻懷朔鎮經年外援不至楊鈞使賀拔勝詣臨淮王彧告急勝募敢死少年十餘騎夜伺隙潰圍出賊騎追及之勝曰我賀拔破胡也【少詩照翻騎奇寄翻伺相吏翻賀拔勝字破胡】賊不敢逼勝見彧於雲中 【考異曰勝傳云至朔州見彧按後魏地理志雲中舊名朔州及改懷朔鎮為朔州不容更以雲中為朔州今但云雲中 按魏氏初都平城北邊列置諸鎮孝昌以後改鎮為州尋即荒廢其地漫不可考杜佑以為魏都平城於郡北三百餘里置懷朔鎮又云遷洛之後於郡北三百餘里置朔州又云後魏初雲中在郡北三百餘里定襄故城北夫其曰皆在郡北三百餘里將是一處邪將是三處邪宋白曰朔州馬邑郡東北至故雲中三百六十里後魏為畿内之地亦曾為懷朔鎮孝文遷洛之後於州北三百八十里定襄故城置朔州又曰後魏初雲中定襄故城是則是朔州與後魏初雲中共一處通鑑此後書改懷朔鎮為朔州更命朔州為雲州此即魏志所謂雲中舊名朔州之證也是則懷朔鎮與雲中是兩處矣是後李崇自崔暹白道之敗引還雲中後又自雲中引還平城其退師道里先後可見而唐之雲中郡乃魏之平城詳而考之歷代建置州郡其名淆雜難指一處為定也】說之曰懷朔被圍旦夕淪陷大王今頓兵不進懷朔若陷則武川亦危賊之鋭氣百倍雖有良平不能為大王計矣彧許為出師勝還復突圍而入鈞復遣勝出覘武川【說式芮翻許為于偽翻復扶又翻下聊復同覘丑廉翻又丑艷翻】武川已陷勝馳還懷朔亦潰勝父子俱為可孤所虜五月臨淮王彧與破六韓拔陵戰於五原【五原即漢五原郡地魏收志朔州治五原杜佑曰魏置朔州於懷朔鎮在唐朔州馬邑郡北三百餘里今榆林九原即漢之五原郡地蓋漢之五原壤地甚廣唐之豐勝朔三州皆漢之五原郡地魏收志朔州附化郡有五原縣彧與拔陵當戰於此】兵敗彧坐削除官爵安北將軍隴西李叔仁又敗於自道【武川鎮北有白道谷谷口有白道城自城北出有高阪謂之白道嶺】賊勢日盛魏主引丞相令僕尚書侍中黄門於顯陽殿問之曰今寇連恒朔逼近金陵【魏未遷洛以前諸帝皆葬雲中之金陵恒戶登翻近其靳翻】計將安出吏部尚書元修義請遣重臣督軍鎮恒朔以捍寇帝曰去歲阿那瓌叛亂遣李崇北征崇上表求改鎮為州朕以舊章難革不從其請尋崇此表開鎮戶非冀之心致有今日之患但既往難追聊復略論耳然崇貴戚重望【李崇文成皇后兄誕之子歷方面有時望】器識英敏意欲遣崇行何如僕射蕭寶寅等皆曰如此實合羣望崇曰臣以六鎮遐僻密邇寇戎【杜佑曰六鎮並在今馬邑雲中單于界】欲以慰悦彼心豈敢導之為亂臣罪當就死陛下赦之今更遣臣北行正是報恩改過之秋但臣年七十加之疲病不堪軍旅願更擇賢材帝不許修義天賜之子也【天賜見一百三十三卷宋明帝泰始七年】<br />
<br />
  臣光曰李崇之表乃所銷禍於未萌制勝於無形魏肅宗既不能用及亂生之後曾無愧謝之言乃更以為崇罪不明之君烏可與謀哉詩云聽言則對誦言如醉匪用其良覆俾我悖【詩桑柔之辭也注云見道聽之言則應答之見誦詩書之言則冥卧如醉不能用善反使我為悖逆之行】其是之謂矣<br />
<br />
  壬申加崇使持節開府儀同三司北討大都督【使疏吏翻】命撫軍將軍崔暹鎮軍將軍廣陽王深皆受崇節度深嘉之子也【按魏收魏書作廣陽王淵李延夀北史作廣陽王深蓋避唐諱通鑑承用之廣陽王嘉見一百四十三卷齊東昏侯永元元年 考異曰魏帝紀深作淵今從列傳及北史】 六月以豫州刺史裴邃督征討諸軍事以伐魏 魏自破六韓拔陵之反二夏豳凉寇盜蜂起【二夏夏州及東夏州也魏收地形志夏州治統萬領化政闡熙金明代名郡東夏州領偏城朔方定陽上郡宋白曰魏改統萬鎮為東夏州後改延州按魏克統萬以為鎮太和十一年改夏州延昌二年置東夏州治廣武唐始改為延州治膚施後魏太和元年置廣武縣後周改豐林縣隋分豐林金明置膚施縣唐延州治焉則魏東夏州治廣武非統萬也然魏收地形志以廣武為太原鴈門之廣武亦誤皇興二年置華州於北地太和十一年改為班州十四年為豳州領北地趙興襄樂郡凉州領武安臨松建昌番和泉城武興武威昌松東徑凉濘郡夏戶雅翻】秦州刺史李彦政刑殘虐在下皆怨是月城内薛珍等聚黨突入州門擒彦殺之推其黨莫折大提為帥【莫折虜複姓帥所類翻】大提自稱秦王魏遣雍州刺史元志討之【雍於用翻】初南秦州豪右楊松柏兄弟數為寇盜【數所角翻】刺史博陵崔遊誘之使降引為主簿接以辭色使說下羣氏【誘音酉說式芮翻】既而因宴會盡收斬之由是所部莫不猜懼遊聞李彦死自知不安欲逃去未果城民張長命韓祖香孫掩等攻遊殺之以城應大提大提遣其黨卜胡襲高平克之殺鎮將赫連略行臺高元榮大提尋卒【卒子恤翻】子念生自稱天子置百官改元天建 丁酉魏大赦 秋七月甲寅魏遣吏部尚書元修義兼尚書僕射為西道行臺帥諸將討莫折念生【帥讀曰率】 崔暹違李崇節度與破六韓拔陵戰于白道大敗單騎走還【騎奇寄翻】拔陵并力攻崇崇力戰不能禦引還雲中與之相持廣陽王深上言先朝都平城【上時掌翻朝直遥翻】以北邊為重盛簡親賢擁麾作鎮【謂鎮將也】配以高門子弟以死防遏非唯不廢仕宦乃更獨得復除【高門子弟謂其先世與魏同起于代北者所謂大姓九十九復方目翻】當時人物忻慕為之太和中僕射李冲用事凉州土人悉免厮役【李寶自敦煌入朝于魏至子冲親貴厚其鄉人故凉土之人悉免厮役】帝鄉舊門仍防邊戍自非得罪當世莫肯與之為伍本鎮驅使但為虞侯白直【杜佑曰白直無月給】一生推遷不過軍主然其同族留京師者得上品通官在鎮者即為清塗所隔或多逃逸乃峻邊兵之格鎮人不聽浮遊在外於是少年不得從師長者不得遊宦【少詩照翻長知兩翻】獨為匪人言之流涕自定鼎伊洛邊任益輕唯底滯凡才乃出為鎮將轉相模習專事聚斂或諸方姦吏犯罪配邊為之指蹤政以賄立邊人無不切齒及阿那瓌背恩縱掠【斂力贍翻背蒲妹翻】奔命追之十五萬衆度沙漠不日而還【事見上卷上年還從宣翻又如字】邊人見此援師遂自意輕中國【師速而疾邊人見其不能盡敵而反意遂輕之】尚書令臣崇求改鎮為州抑亦先覺朝廷未許而高闕戍主御下失和【酈道元曰趙武靈王既襲胡服自代並隂山下至高闕為塞山下有長城長城之際連山刺天其山中斷兩岸雙闕雲舉望若闕焉故有高闕之名自闕北出荒中闕口有城跨山結局謂之高闕戍】拔陵殺之遂相帥為亂【帥讀曰率】攻城掠地所過夷滅王帥屢北賊黨日盛此段之舉指望銷平而崔暹隻輪不返臣崇與臣逡巡復路【復路者還即舊路也】相與還次雲中將士之情莫不解體今日所慮非止西北將恐諸鎮尋亦如此天下之事何易可量書奏不省【為魏主思崇深之言張本易以䜴翻量音良省悉景翻】詔徵崔暹繫廷尉暹以女妓田園賂元乂卒得不坐【妓渠綺翻卒子恤翻下卒無同】丁丑莫折念生遣其都督楊伯年攻仇鳩河池二戍【河池即今鳳州河池縣有河池水仇鳩亦當與河池相近】東益州刺史魏子建遣將軍伊祥等擊破之斬首千餘級東益州本氐王楊紹先之國【天監五年魏克武興滅楊紹先之國置東益州】將佐皆以城民勁勇二秦反者皆其族類請先收其器械子建曰城民數經行陣【數所角翻行戶剛翻】撫之足以為用急之則腹背為患乃悉召城民慰諭之既而漸分其父兄子弟外戍諸郡内外相顧卒無叛者【卒子恤翻】子建蘭根之族兄也 魏凉州幢帥於菩提等執刺史宋穎據州反【幢傳江翻帥所類翻菩薄乎翻】 八月庚寅徐州刺史成景儁拔魏童城【童城即下邳僮縣城也】 魏員外散騎侍郎李苗上書曰凡食少兵精利於速戰【散悉亶翻騎奇寄翻上時掌翻少詩沼翻】糧多卒衆事宜持久今隴賊猖狂非有素蓄雖據兩城【兩城謂天水及高平】本無德義其勢在於疾攻日有降納【降戶江翻】遲則人情離沮坐待奔潰夫飊至風舉逆者求萬一之功高壁深壘王師有全制之策但天下久泰人不曉兵奔利不相待逃難不相顧【難乃旦翻】將無法令士非教習不思長久之計各有輕敵之心如令隴東不守汧軍敗散【汧軍謂元志之軍也汧在隴阪之東將即亮翻汧口堅翻】則兩秦遂彊【兩秦謂莫折念生及張長命等】三輔危弱國之右臂於斯廢矣宜勒大將堅壁勿戰别命偏裨帥精兵數千出麥積崖以襲其後【麥積崖在今秦州天水縣東百里狀如麥積故名裨賓彌翻帥讀曰率】則汧隴之下羣妖自散【妖於驕翻】魏以苗為統軍與别將淳于誕俱出梁益未至莫折念生遣其弟高陽王天生將兵下隴甲午都督元志與戰於隴口【隴口隴坻之口也】志兵敗棄衆東保岐州【魏岐州治雍城】 東西部敕勒皆叛魏附於破六韓拔陵魏主始思李崇及高陽王深之言丙申下詔諸州鎮軍貫【貫籍也】非有罪配隸者皆免為民改鎮為州以懷朔鎮為朔州更命朔州曰雲州【魏先置朔州於雲中之盛樂以漢五原郡地為懷朔鎮今以懷朔為朔州改舊朔州為雲州因雲中郡而得名也按後廣陽王深自五原拔軍向朔州則懷朔鎮雖置於漢五原郡地與五原别為兩城宋白曰漢五原故城在唐勝州榆林縣界後魏孝文於唐朔州北三百八十里定襄故城置朔州更工衡翻】遣兼黄門侍郎酈道元為大使撫慰六鎮【使疏吏翻】時六鎮已盡叛道元不果行先是代人遷洛者多為選部所抑不得仕進【先悉薦翻選須絹翻】及六鎮叛元乂乃用代來人為傳詔以慰悦之廷尉評代人山偉奏記稱乂德美乂擢偉為尚書二千石郎【廷尉評即漢之廷尉平魏晉以來平旁加言今大理評事即其職也後漢尚書有二千石曹魏置二千石郎魏收官氏志内入諸姓土難氏後改為山氏】 秀容人乞伏莫于聚衆攻郡殺太守【水經注魏立秀容護軍以統胡人其治所去汾水六十里地形志永興二年置秀容郡屬肆州】丁酉南秀容牧子萬于乞真殺太僕卿陸延秀容酋長爾朱榮討平之榮羽健之玄孫也【羽健見一百一十卷晉安帝隆安二年酋慈秋翻長知兩翻】其祖代勤嘗出獵部民射虎誤中其髀代勤拔箭不復推問【射兩亦翻中竹仲翻復扶又翻】所部莫不感悦官至肆州刺史賜爵梁郡公年九十餘而卒【卒子恤翻】子新興立新興時畜牧尤蕃息【蕃扶元翻】牛羊駝馬色别為羣彌漫川谷不可勝數【勝音升】魏每出師新興輒獻馬及資糧以助軍高祖嘉之新興老請傳爵於子榮魏朝許之【朝直遥翻】榮神機明决御衆嚴整時四方兵起榮隂有大志散其畜牧資財招合驍勇結納豪桀【爾朱榮事始此驍堅堯翻】於是侯景司馬子如賈顯度及五原段榮太安竇泰【時魏于懷朔鎮置朔州并置太安郡】皆往依之顯度顯智之兄也 戊戌莫折念生遣都督竇雙攻魏盤頭郡【盤頭郡屬東益州五代志興州長舉縣魏置盤頭郡】東益州刺史魏子建遣將軍竇念祖擊破之九月戊申成景儁拔魏睢陵戊午北兖州刺史趙景<br />
<br />
  悦圍荆山【梁北兖州治淮隂水經注曰地理志平阿縣有當塗山淮出于荆山之左當塗之右魏收志梁北徐州沛郡已吾縣有當塗山荆山今之懷遠軍正據荆山以沈約志言之皆屬馬頭郡界五代志鍾離郡塗山縣古當塗也後齊置荆山郡】裴邃帥騎三千襲夀陽【帥讀曰率騎奇寄翻】壬戌夜斬關而入克其外郭魏揚州刺史長孫稚禦之一日九戰後軍蔡秀成失道不至邃引兵還别將擊魏淮陽【此梁所遣别將也非裴邃所部將即亮翻下同】魏使行臺酈道元都督河間王琛救夀陽【琛丑林翻】安樂王鑒救淮陽鑒詮之子也【樂音洛詮丑緣翻安樂王詮事見一百四十六卷天監五年】 魏西道行臺元修義得風疾不能治軍【治直之翻】壬申魏以尚書左僕射齊王蕭寶寅為西道行臺大都督帥諸將討莫折念生【為蕭寶寅以關中叛魏張本】宋穎密求救於吐谷渾王伏連籌【吐從暾入聲谷音浴】伏連籌自將救凉州於菩提棄城走追斬之城民趙天安等復推宋穎為刺史【復扶又翻】 河間王琛軍至西硤石解渦陽圍復荆山戍青冀二州刺史王神念與戰為琛所敗【敗補邁翻】冬十月戊寅裴邃元樹攻魏建陵城克之辛巳拔曲木【曲木當作曲沭水經注沭水過建陵縣故城東又南逕陵山西魏立大堰遏水西流兩瀆之會置城防之曰曲沭戍沭食聿翻】掃虜將軍彭寶孫拔琅邪 魏營州城民劉安定就德興【魏書官氏志菟賴氏改為就氏西方諸姓也】執刺史李仲遵據城反城民王惡兒斬安定以降德興東走自稱燕王【降戶江翻燕因肩翻】 胡琛遣其將宿勤明達寇豳夏北華三州【魏高祖太和十五年置東秦州于杏城後改為北華州領中部敷城凡三郡宿勤虜複姓夏戶雅翻華戶化翻】魏遣都督北海王顥帥諸將討之顥詳之子也【詳得罪見一百四十五卷天監三年帥讀曰率】 甲申彭寶孫拔亶丘辛卯裴邃拔狄城【水經注肥水自荻丘過漢九江成德縣故城西王莽更曰平阿又北入芍陂】丙申又拔甕城進屯黎漿壬寅魏東海太守韋敬欣以司吾城降【漢東海郡司吾縣之故城也】定遠將軍曹世宗拔曲陽甲辰又拔秦墟魏守將多棄城走【水經注洛水逕漢淮南郡曲陽故城東應劭曰縣在淮曲之陽洛水又北歷秦墟下注淮謂之洛口魏收志曲陽縣屬霍州北沛郡五代志曲陽縣後廢入鍾離定遠縣】 魏使黄門侍郎盧同持節詣營州慰勞就德興降而復反詔以同為幽州刺史兼尚書行臺同屢為德興所敗而還【勞力到翻復扶又翻敗補邁翻還從宣翻又如字】 魏朔方胡反圍夏州刺史源子雍【夏戶雅翻夏州治統萬城】城中食盡煮馬皮而食之衆無貳心子雍欲自出求糧留其子延伯守統萬將佐皆曰今四方離叛糧盡援絶不若父子俱去子雍泣曰吾世荷國恩當畢命此城但無食可守故欲往東州【荷下可翻東州謂東夏州也】為諸君營數月之食【為于偽翻】若幸而得之保全必矣乃帥羸弱詣東夏州運糧【帥讀曰率羸倫為翻】延伯與將佐哭而送之子雍行數日胡帥曹阿各拔邀擊擒之【帥所類翻】子雍潛遣人齎書敕城中努力固守闔城憂懼延伯諭之曰吾父吉凶未可知方寸焦爛但奉命守城所為者重【為于偽翻下為陳同】不敢以私害公諸君幸得此心於是衆感其義莫不奮勵子雍雖被擒胡人常以民禮事之子雍為陳禍福勸阿各拔降會阿各拔卒其弟桑生竟帥其衆随子雍降【降戶江翻】子雍見行臺北海王顥具陳諸賊可滅之狀顥給子雍兵令其先驅時東夏州闔境皆反所在屯結子雍轉鬬而前九旬之中凡數十戰遂平東夏州徵税粟以饋統萬二夏由是獲全子雍懷之子也【史言源氏諸子皆有才具而天降喪亂終無救魏氏之衰也】 魏廣陽王深上言今六鎮盡叛高車二部亦與之同【高車自阿伏至羅興窮奇分為二部所謂東西部敕勒也】以此疲兵擊之必無勝理不若選練精兵守恒州諸要【諸要謂要衝之地恒戶登翻】更為後圖遂與李崇引兵還平城崇謂諸將曰雲中者白道之衝【以此觀之則魏之雲中漢之盛樂縣唐之振武軍節度使治所皆雲山之陽】賊之咽喉若此地不全則并肆危矣當留一人鎮之誰可者衆舉費穆崇乃請穆為朔州刺史【請奏請也時雲中已改為雲州朔當作雲】 賀拔度拔父子及武川宇文肱糾合鄉里豪傑共襲衛可孤殺之度拔尋與鐵勒戰死肱逸豆歸之玄孫也【肱宇文泰之父也逸豆歸晉康帝建元二年為慕容皝所滅】李崇引國子博士祖為長史廣陽王深奏瑩詐增首級盜沒軍資坐除名崇亦免官削爵徵還深專總軍政【為深内困於讒外困于賊張本】 莫折天生進攻魏岐州十一月戊申陷之執都督元志及刺史裴芬之送莫折念生殺之念生又使卜胡等寇涇州敗光禄大夫薛巒於平凉東【魏置平凉郡治鶉隂縣有平凉城敗補邁翻】巒安都之孫也【宋泰始初薛安都降魏】 丙辰彭寶孫拔魏東莞【莞音官】壬戌裴邃攻夀陽之安城【魏收志梁置新興郡治安城縣】丙寅馬頭安城皆降 高平人攻殺卜胡共迎胡琛魏以黄門侍郎楊昱兼侍中持節監北海王顥軍以<br />
<br />
  救豳州豳州圍解【監工銜翻】蜀賊張映龍姜神達攻雍州【蜀賊者蜀人之徙關中者也乘魏亂起而為盜因謂之蜀賊後爾朱天光西討蜀賊斷路皆其黨也雍於用翻】雍州刺史元修義請援一日一夜書移九通都督李叔仁遲疑不赴昱曰長安關中基本若長安不守大軍自然瓦散留此何益遂與叔仁進擊之斬神達餘黨散走十二月戊寅魏荆山降 壬辰魏以京兆王繼為太師大將軍都督西道諸軍以討莫折念生 乙巳武勇將軍李國興攻魏平靖關卒丑信威長史楊乾攻武陽關壬寅攻峴關【此義陽之三關也峴戶典翻】皆克之國興進圍郢州魏郢州刺史裴詢與蠻酋西郢州刺史田朴特相表裏以拒之【魏郢州治義陽西郢州又當在義陽之西蠻中也酋慈秋翻】圍城近百日【近其靳翻】魏援軍至國興引還詢駿之子也【裴駿見一百二十四卷宋文帝元嘉二十二年】 魏汾州諸胡反【魏孝文帝太和十二年置汾州治蒲子領西河吐京五城定陽凡四郡】以章武王融為大都督將兵討之【將即亮翻】 魏魏子建招諭南秦諸氐稍稍降附遂復六郡十二戍斬韓祖香魏以子建兼尚書為行臺刺史如故【刺史謂子建本為東益州刺史】梁巴二益工秦諸州皆受節度【魏置梁州于南鄭置巴州于漢巴西郡置益州于晉夀郡東益州于武興郡秦州于上邽南秦州于仇池】 莫折念生遣兵攻凉州城民趙天安復執刺史以應之【復扶又翻】 是歲侍中太子詹事周捨坐事免散騎常侍錢唐朱异代掌機密軍旅謀議方鎮改易朝儀詔敕皆典之异好文義多藝能精力敏贍上以是任之【為朱异亂梁張本散悉亶翻騎奇寄翻异羊吏翻好呼到翻朝直遥翻】六年春正月丙午雍州刺史晉安王綱遣安北長史柳渾破魏南鄉郡司馬董當門破魏晉城庚戌又破馬圈彫陽二城【雍於用翻圈求遠翻】 辛亥上祀南郊大赦 魏徐州刺史元法僧素附元乂見乂驕恣恐禍及己遂謀反魏遣中書舍人張文伯至彭城法僧謂曰吾欲與法去危就安能從我乎文伯曰我寧死見文陵松柏【文陵謂孝文帝陵】安能去忠義而從叛逆乎法僧殺之庚申法僧殺行臺高諒稱帝 【考異曰法僧傳作高謨今從魏帝紀及魏紀云自稱宋王法僧傳及北史皆云稱尊號梁書法僧傳云稱帝按法僧立諸子為王必稱帝也今從梁書】改元大啓立諸子為王魏兵擊之法僧乃遣其子景仲來降【降戶江翻 考異曰法僧傳云魏室大亂法僧據鎮議欲匡復既而魏亂稍定將討法僧法僧懼歸欵按時魏亂未定今從北史】安東長史元顯和麗之子也【元麗見一百四十六卷天監五年】舉兵與法僧戰法僧擒之執其手命使共坐顯和不肯曰與翁皆出皇家【元法僧陽平王熙之曾孫熙道武子也元麗小新城之孫小新城景穆之子顯和麗之子也以族屬長幼之次呼法僧為翁】一朝以地外叛獨不畏良史乎法僧猶欲慰諭之顯和曰我寧死為忠鬼不能生為叛臣乃殺之上使散騎常侍朱异使於法僧以宣城太守元略為大都督【元略來奔見上卷四年使疏吏翻】與將軍義興陳慶之胡龍牙成景雋等將兵應接【將即亮翻】 莫折天生軍於黑水【水經注就水出南山就谷北流與黑水合黑水上合三泉於流水之右三泉奇言歸一瀆北流會於就水就水又北流注于渭】兵勢甚盛魏以岐州刺史崔延伯為征西將軍西道都督帥衆五萬討之延伯與行臺蕭寶寅軍于馬嵬延伯素驍勇寶寅趣之使戰【帥讀曰率嵬五回翻驍堅堯翻趣讀曰促】延伯曰明晨為公參賊勇怯乃選精兵數千西度黑水整陳向天生營【為于偽翻陳讀曰陣】寶寅軍於水東遥為繼援延伯直抵天生營下揚威脅之徐引兵還天生見延伯衆少【少詩沼翻】争開營逐之其衆多於延伯十倍蹙延伯於水次寶寅望之失色延伯自為後殿【殿丁練翻】不與之戰使其衆先渡部伍嚴整天生兵不敢擊須臾渡畢延伯徐渡天生之衆亦引還寶寅喜曰崔君之勇關張不如【關張謂關羽張飛也】延伯曰此賊非老奴敵也明公但安坐觀老奴破之癸亥延伯勒兵出寶寅舉軍繼其後天生悉衆逆戰延伯身先士卒【先悉薦翻】陷其前鋒將士盡鋭競進大破之俘斬十餘萬追奔至小隴【隴山有大隴山小隴山大隴山在清水縣東北小隴山在岐州武都郡南田縣西北五代志南田作南由南由唐隴州之吳山縣即其地】岐雍及隴東皆平將士稽留採掠天生遂塞隴道【塞悉則翻】由是諸軍不能進寶寅破宛川【五代志扶風郡陳倉縣後魏曰宛川】俘其民以為奴婢以美女十人賞岐州刺史魏蘭根蘭根辭曰此縣介於彊寇不能自立故附從以救死官軍之至宜矜而撫之奈何助賊為虐翦以為賤役乎悉求其父兄而歸之 乙巳裴邃拔魏新蔡郡【魏收志新蔡郡治石母臺隋廢為縣唐以後屬蔡州】詔侍中領軍將軍西昌侯淵藻將衆前驅南兖州刺史豫章王綜與諸將繼進【將即亮翻】癸酉裴邃拔鄭城【水經注潁水過慎縣故城南而東南流逕蜩蟟郭東俗謂之鄭城又東南入淮】汝潁之間所在響應魏河間王琛等憚邃威名軍於城父【城父縣漢屬沛郡魏晉以來屬譙郡宋併城父為浚儀縣屬陳留郡郡寄治譙郡長垣縣界魏收志陳留郡浚儀縣注有城父城父音甫】累月不進魏朝遣廷尉少卿崔孝芬持節齎齋庫刀以趣之【齋庫刀千牛刀也齋刀以趣其進言若復逼留將斬之也朝直遥翻】孝芬挺之子也琛至夀陽欲出兵决戰長孫稚以為久雨未可出琛不聽引兵五萬出城擊邃邃為四甄以待之使直閣將軍李祖憐先挑戰而偽退【甄稽延翻挑徒了翻】稚琛悉衆追之四甄競魏師大敗斬首萬餘級琛走入城稚勒兵而殿遂閉門自固不敢復出【殿丁練翻復扶又翻】魏安樂王鑒將兵討元法僧擊元略於彭城南畧大敗【樂音洛 考異曰魏帝紀敘元略等事便在庚申法僧叛下不應如此之速今移之於月末】與數十騎走入城鑒不設備法僧出擊大破之鑒單騎奔歸【騎奇寄翻】將軍王希拔魏南陽平【宋僑置陽平郡於沛郡南界後入於魏為南陽平郡以别相州之古陽平郡也後又徙郡寄治彭城他甘翻】執太守薛曇尚曇尚虎子之子也【薛虎子事魏孝文帝歷州鎮有聲績曇徒含翻】甲戌以法僧為司空封始安郡公魏以安豐王延明為東道行臺臨淮王彧為都督以擊彭城 魏以京兆王繼為太尉 二月乙未趙景悦拔魏龍亢【龍亢縣漢屬沛郡晉屬譙國魏太和十九年置下蔡郡龍亢屬焉五代志潁州潁上縣舊置下蔡郡晉勺曰亢音剛】 初魏劉騰既卒【騰卒見上卷四年】胡太后及魏主左右防衛微緩元乂亦自寛時出遊於外留連不返其所親諫乂不納太后察知之去秋太后對帝謂羣臣曰今隔絶我母子不聽往來復何用我為【復扶又翻下無復同】我當出家修道於嵩山閒居寺耳【魏作間居寺見一百四十七卷天監八年】因自欲下髮帝及羣臣叩頭泣涕殷勤苦請太后聲色愈厲帝乃宿於嘉福殿積數日乃與太后密謀黜乂然帝深匿形迹太后有忿恚欲得往來顯陽之言皆以告乂【魏主常居顯陽殿故太后欲往來恚於避翻】又對乂流涕敘太后欲出家憂怖之心日有數四【怖普布翻】乂殊不以為疑乃勸帝從太后所欲於是太后數御顯陽殿二宫無復禁礙乂舉元法僧為徐州法僧反太后數以為言【數所角翻】乂深愧悔丞相高陽王雍雖位居乂上而深畏憚之會太后與帝遊洛水雍邀二宫幸其第日晏帝與太后至雍内室從官皆不得入【從才用翻】遂相與定圖乂之計于是太后謂乂曰元郎若忠於朝廷無反心何故不去領軍以餘官輔政【去羌呂翻】乂甚懼免冠求解領軍乃以乂為驃騎大將軍開府儀同三司尚書令侍中領左右【驃匹妙翻騎奇寄翻】 戊戌魏大赦 壬辰莫折念生遣都督楊鮓等攻仇池郡【以上戊戌下三月己酉推之壬辰當作壬寅魏收志仇池郡屬東益州五代志漢陽郡上禄縣魏置仇池郡】行臺魏子建擊破之 三月己酉上幸白下城履行六軍頓所【行下孟翻】乙丑命豫章王綜權頓彭城總督衆軍并攝徐州府事己巳以元法僧之子景隆為衡州刺史【吳孫亮太平二年分長沙西部都尉立衡陽郡梁置衡州按五代志梁置衡州於南海含洭縣】景仲為廣州刺史上召法僧及元略還建康法僧驅彭城吏民萬餘人南渡 【考異曰南史云武官戍彭城者三千餘人法僧皆印額為奴逼將南渡魏書梁書皆無此事】法僧至建康上寵待甚厚元略惡其為人【惡烏路翻】與之言未嘗笑 魏詔京兆王繼班師【去年魏使繼西討今將誅其子乂故詔使班師】 北凉州刺史錫休儒等自魏興侵魏梁州攻直城【梁置北凉州于魏興凉當作梁魏收志東梁州金城郡領直城縣五代志金州安康縣蕭詧改直州蓋因直城以名州魏以其地出金故郡曰金城州曰金州錫姓也漢有錫光】魏梁州刺史傅豎眼遣其子敬紹擊之【豎而庾翻】休儒等敗還 柔然王阿那瓌為魏討破六韓拔陵魏遣牒云具仁齎雜物勞賜之【為于偽翻勞力到翻下同】阿那瓌勒衆十萬自武川西向沃野屢破拔陵兵【稽古録是年書蠕蠕殺破六韓拔陵在誅元乂之下】夏四月魏主復遣中書舍人馮儁勞賜阿那瓌【復扶又翻】阿那瓌部落浸彊自稱敕連頭兵豆伐可汗【魏收曰魏言摠攬也】 魏元乂雖解兵權猶總任内外殊不自意有廢黜之理胡太后意猶豫未决侍中穆紹勸太后速去之紹亮之子也【穆氏從魏起於代北崇夀亮奕世貴顯去羌呂翻】潘嬪有寵於魏主【嬪毘賓翻】宦官張景嵩說之云【說式芮翻】乂欲害嬪嬪泣訴於帝曰乂非獨欲害妾將不利於陛下帝信之因乂出宿解乂侍中明旦乂將入宫門者不納辛卯太后復臨朝攝政下詔追削劉騰官爵除乂名為民清河國郎中令韓子熙上書為清河王懌訟寃【懌死見上卷元年復扶又翻書為于偽翻】乞誅元乂等曰昔趙高柄秦令關東鼎沸【事見秦紀】今元乂專魏使四方雲擾開逆之端起於宋維成禍之末良由劉騰宜梟首洿宫斬骸沈族以明其罪【梟堅堯翻洿哀都翻沈持林翻】太后命發劉騰之墓露散其骨籍沒家貲盡殺其養子以子熙為中書舍人子熙麒麟之孫也【韓麒麟見一百三十五卷齊武帝永明元年】初宋維父弁常曰維性疎險必敗吾家李崇郭祚游肇亦曰伯緒凶疎【宋維字伯緒敗補邁翻】終傾宋氏若得殺身幸矣維阿附元乂超遷至洛州刺史【魏初置洛州於洛陽荆州於上洛太和遷洛以洛州為司州又置荆州於穰城以上洛之荆州為洛州領上洛上庸魏興始平萇和郡】至是除名尋賜死乂之解領軍也太后以乂黨與尚彊未可猝制乃以侯剛代乂為領軍以安其意尋出剛為冀州刺史加儀同三司未至州黜為征虜將軍卒于家【卒子恤翻】太后欲殺賈粲以乂黨多恐驚動内外乃出粲為濟州刺史【濟子禮翻】尋追殺之籍沒其家唯乂以妹夫未忍行誅先是給事黄門侍郎元順以剛直忤乂意【先悉薦翻忤五故翻】出為齊州刺史太后徵還為侍中侍坐於太后【坐徂卧翻】乂妻在太后側順指之曰陛下奈何以一妹之故不正元乂之罪使天下不得伸其寃憤太后嘿然順澄之子也【任城王雲及澄魏宗室之賢王也】它日太后從容謂侍臣曰【從千容翻】劉騰元乂昔嘗邀朕求鐵劵冀得不死朕賴不與韓子熙曰事關生殺豈繫鐵劵且陛下昔雖不與何解今日不殺太后憮然【憮罔甫翻朱元晦曰憮然猶悵然】未幾有告乂及弟瓜謀誘六鎮降戶反於定州又招魯陽諸蠻侵擾伊闕【伊闕在河南新城縣界隋開皇初改新城縣為伊闕縣幾居豈翻降戶江翻】欲為内應得其手書太后猶未忍殺之羣臣固執不已魏主亦以為言太后乃從之賜乂及弟瓜死于家猶贈乂驃騎大將軍儀同三司尚書令【驃匹妙翻騎奇寄翻】江陽王繼廢於家病卒前幽州刺史盧同坐乂黨除名【去年同為幽州刺史】太后頗事粧飾數出遊幸元順面諫曰禮婦人夫沒自稱未亡人首去珠玉【數所角翻去羌呂翻】衣不文采陛下母臨天下年垂不惑【四十而不惑】修飾過甚何以儀刑後世太后慙而還宫召順責之曰千里相徵豈欲衆中見辱耶順曰陛下不畏天下之笑而恥臣之一言乎順與穆紹同直順因醉入其寢所紹擁被而起正色讓順曰身二十年侍中與卿先君亟連職事【亟去連翻數數也】縱卿方進用何宜相排突也遂謝事還家詔諭久之乃起初鄭羲之兄孫儼為司徒胡國珍行參軍私得幸於太后人未之知蕭寶寅之西討以儼為開府屬【開府有掾有屬】太后再攝政儼請奉使還朝【使疏吏翻朝直遥翻】太后留之拜諫議大夫中書舍人領嘗食典御晝夜禁中每休沐太后常遣宦者随之儼見其妻唯得言家事而已中書舍人樂安徐紇粗有文學【粗坐五翻】先以諂事趙修坐徙枹罕【趙修得罪見一百四十五卷天監二年枹音膚】後還復除中書舍人又諂事清河王懌懌死出為鴈門太守還洛復諂事元乂乂敗太后以紇為懌所厚復召為中書舍人【復扶又翻】紇又諂事鄭儼儼以紇有智數仗為謀主【仗直兩翻憑也】紇以儼有内寵傾身承接共相表裏勢傾内外號為徐鄭儼累遷至中書令車騎將軍【騎奇寄翻】紇累遷至給事黄門侍郎仍令舍人總攝中書門下之事軍國詔令莫不由之紇有機辯彊力終日治事畧無休息不以為勞【治直之翻】時有急詔令數吏執筆或行或卧人别占之【占口占也】造次俱成【造七到翻】不失事理【人必小有才也然後能迎世取寵以竊一時之權朱异徐紇是也】然無經國大體專好小數見人矯為恭謹遠近輻湊附之【為爾朱榮討徐鄭張本好呼到翻】給事黄門侍郎袁飜李神軌皆領中書舍人為太后所信任時人云神軌亦得幸於太后衆莫能明也神軌求婚於散騎常侍盧義僖義僖不許黄門侍郎王誦謂義僖曰昔人不以一女易衆男【引樂廣事事見八十五卷晉惠帝太安二年】卿豈易之邪義僖曰所以不從者正為此耳【為于偽翻】從之恐禍大而速誦乃堅握義僖手曰我聞有命不敢以告人【詩唐國風揚之水之辭也】女遂適他族臨婚之夕太后遣中使宣勅停之内外惶怖義僖夷然自若【使疏吏翻怖普布翻】神軌崇之子義僖度世之孫也【盧度世見宋紀】 胡琛據高平遣其大將萬俟醜奴宿勤明達等寇魏涇州【琛丑林翻將即亮翻萬當作万音莫北翻俟渠之翻万俟虜複姓北史曰萬俟其先匈奴之别也】將軍盧祖遷伊甕生討之不克蕭寶寅崔延伯既破莫折天生引兵會祖遷等於安定甲卒十二萬鐵馬八千軍威甚盛醜奴軍於安定西北七里時以輕騎挑戰【騎奇寄翻挑徒了翻】大兵未交輟委走延伯恃其勇且新有功遂唱議為先驅擊之别造大盾内為鎖柱使壯士負以趨謂之排城置輜重於中戰士在外自安定北緣原北上【盾食尹翻重直用翻上時掌翻】將戰有賊數百騎詐持文書云是降簿【言是降人之名籍也降戶江翻下同】且乞緩師寶寅延伯未及閲視宿勤明達引兵自東北至降賊自西競下覆背擊之【覆或作腹】延伯上馬奮擊逐北徑抵其營賊皆輕騎延伯軍雜步卒戰久疲乏賊乘間得入排城延伯遂大敗死傷近二萬人【間古莧翻近其靳翻】寶寅收衆退保安定延伯自恥其敗乃繕甲兵募驍勇【驍堅堯翻下同】復自安定西進去賊七里結營【時賊屯安定西彭阬復扶又翻下復還復失同】壬辰不告寶寅獨出襲賊大破之俄頃平其數柵賊見軍士採掠散亂復還擊之魏兵大敗延伯中流矢卒士卒死者萬餘人時大寇未平復失驍將朝野為之憂恐【中竹仲翻卒子恤翻將即亮翻朝直遥翻為于偽翻】於是賊勢愈盛而羣臣自外來者太后問之皆言賊弱以求悦媚由是將帥求益兵者往往不與【帥所類翻】五月夷陵烈侯裴邃卒【時邃卒于軍中諡法有功安民曰烈秉德尊業曰烈】邃深沈有思略【沈持林翻思相吏翻】為政寛明將吏愛而憚之壬子以中護軍夏侯亶督夀陽諸軍事馳驛代邃 益州刺史臨汝侯淵猷遣其將樊文熾蕭世澄等將兵圍魏益州長史和安於小劍魏益州刺史邴蚪遣統軍河南胡小虎崔珍寶將兵救之文熾襲破其柵皆擒之使小虎於城下說和安令早降【將即亮翻說式芮翻降戶江翻】小虎遥謂安曰我柵失備為賊所擒觀其兵力殊不足言努力堅守魏行臺傅梁州援兵已至【魏行臺子建傅梁州豎眼】語未終軍士以刀敺殺之【敺烏口翻】西南道軍司淳于誕引兵救小劍文熾置柵於龍鬚山上以防歸路戊辰誕密募壯士夜登山燒其柵梁軍望見歸路絶皆忷懼【忷許拱翻】誕乘而擊之文熾大敗僅以身免虜世澄等將吏十一人斬獲萬計魏子建以世澄購胡小虎之尸得而葬之 魏魏昌武康伯李崇卒【魏收地形志魏昌縣屬中山郡諡法克定禍亂曰武温柔好樂曰康 考異曰魏帝紀在五月戊子按長歷是月乙亥朔無戊子今不書日】 初帝納東昏侯寵姬吳淑媛【魏文帝置淑媛宋明帝以淑媛為九嬪之首齊梁因之媛于眷翻】七月而生豫章王綜宫中多疑之及淑媛寵衰怨望密謂綜曰汝七月生兒安得比諸皇子然汝太子次第幸保富貴勿泄也與綜相抱而泣綜由是自疑晝則談謔如常【謔近卻翻】夜則於静室閉戶披髮席藁私於别室祭齊氏七廟又微服至曲阿拜齊太宗陵【齊無太宗當是高宗】聞俗說割血瀝骨滲則為父子【滲所䕃翻】遂潛發東昏侯冢并自殺一男試之皆驗由是常懷異志專伺時變【伺相吏翻】綜有勇力能手制奔馬輕財好士唯留附身故衣餘皆分施【好呼到翻施式豉翻】恒致罄乏屢上便宜求為邊任【恒戶登翻上時掌翻】上未之許常於内齋布沙於地終日跣行足下生胝【胝丁尼翻皮厚也】日能行三百里王侯妃主及外人皆知其志而上性嚴重人莫敢言又使通問於蕭寶寅謂之叔父為南兖州刺史不見賓客辭訟隔簾聽之出則垂帷於輿惡人識其面【惡烏路翻】及在彭城魏安豐王延明臨淮王彧將兵二萬逼彭城【將即亮翻考異曰南史陳慶之傳云衆十萬今從梁書】勝負久未决上慮綜敗沒敕綜引軍還綜恐南歸不復得至北邊【復扶又翻下路復復取同】乃密遣人送降欵於彧魏人皆不之信彧募人入綜軍驗其虛實無敢行者殿中侍御史濟隂鹿悆為彧監軍【鹿姓也風俗通漢有巴郡太守鹿旗濟子禮翻悆羊茹翻】請行曰若綜有誠心與之盟約如其詐也何惜一夫時兩敵相對内外嚴固悆單騎間出【騎奇寄翻間古莧翻】徑趣彭城【趣七喻翻】為綜軍所執問其來狀悆曰臨淮王使我來欲有交易耳時元略已南還綜聞之謂成景儁等曰我常疑元略規欲反城【規圖也反孚袁翻】將驗其虚實故遣左右為略使【使疏吏翻下同】入魏軍中呼彼一人今其人果來可遣人詐為略有疾在深室呼至戶外令人傳言謝之綜又遣腹心安定梁話迎悆密以意狀語之【意者傳綜欲降之意狀者告以詭與成景儁設謀之狀語牛倨翻】悆薄暮入城先引見胡龍牙龍牙曰元中山甚欲相見【元略之南奔也梁封為中山王故稱之】故遣呼卿又曰安豐臨淮將少弱卒【將即亮翻少詩沼翻】規復此城容可得乎悆曰彭城魏之東鄙勢在必争得否在天非人所測龍牙曰當如卿言又引見成景儁景儁與坐謂曰卿不為刺客邪悆曰今者奉使欲返命本朝【朝直遥翻】相刺之事更卜後圖景儁為設飲食【為于偽翻】乃引至一所詐令一人自室中出為元略致意曰我昔有以南向【有以言有所為也】且遣相呼欲聞鄉事晚來疾作不獲相見悆曰早奉音旨冒險祇赴不得瞻見内懷反側遂辭退諸將競問魏士馬多少悆盛陳有勁兵數十萬諸將相謂曰此華辭耳【史記商君曰貌言華也將即亮翻少詩沼翻】悆曰崇朝可驗何華之有【鄭玄曰崇朝終朝也】乃遣悆還成景儁送之戲馬臺【蘇軾曰彭城三面阻水樓堞之下以汴泗為池獨其南可通軍馬而戲馬臺在焉其廣百步其高十仞】北望城壍【壍七艷翻】謂曰險固如此豈魏所能取悆曰攻守在人何論險固悆還於路復與梁話申固盟約【鹿悆既得綜之誠欵知魏城之必可取與梁將語率多大言蓋其中心喜躍不能自揜於言語之間使梁將有如臾駢絺疵之流必能察知其情矣】六月庚辰綜與梁話及淮隂苗文寵夜出步投魏軍 【考異曰南史綜傳綜夜潛與梁話苗寵三騎開北門涉汴河遂奔蕭城自稱隊主見延明而拜延明坐之問其名是不答曰殿下問人有見識者延明召使視之曰豫章王也延明喜下地執其手答其拜送于洛陽按魏書及北史鹿悆傳皆豫有盟約豈得不知又魏書蕭贊傳作濟隂芮文寵北史作濟隂苗文寵今從南史】及旦齋内諸閤猶閉不開衆莫知所以唯見城外魏軍呼曰【呼火故翻】汝豫章王昨夜已來在我軍中汝尚何為城中求王不獲軍遂大潰魏人入彭城乘勝追擊復取諸城至宿預而還【還從宣翻又如字下同】將佐士卒死沒者什七八唯陳慶之帥所部得還【史言陳慶之臨亂能整異於諸將帥讀曰率】上聞之驚駭有司奏削綜爵土絶屬籍更其子直姓悖氏【更工衡翻下更名同】未旬日詔復屬籍封直為永新侯【吳立永新縣宋屬安成郡】西豐侯正德自魏還【事見上卷四年】志行無悛【行下孟翻悛丑緣翻】多聚亡命夜剽掠於道【剽匹妙翻】以輕車將軍從綜北伐弃軍輒還上積其前後罪惡免官削爵徙臨海未至追赦之綜至洛陽見魏主還就館為齊東昏侯舉哀服斬衰三年【見賢遍翻為于偽翻衰倉雷翻】太后以下並就館弔之賞賜禮遇甚厚拜司空封高平郡公丹楊王更名贊 【考異曰梁書南史皆云改名纘今從魏書北史】以苗文寵梁話皆為光禄大夫封鹿悆為定陶縣子除員外散騎常侍綜長史濟陽江革司馬范陽祖暅之皆為魏所虜【散悉亶翻騎奇寄翻濟子禮翻暅居鄧翻】安豐王延明聞其才名厚遇之革稱足疾不拜延明使暅之作欹器漏刻銘革唾罵暅之曰卿荷國厚恩乃為虜立銘【荷下可翻為于偽翻下不為同】孤負朝廷延明聞之令革作大小寺碑 【考異曰南史作丈八寺碑今從梁書】祭彭祖文【彭城大彭氏之墟也故祭之】革辭不為延明將箠之【箠止蘂翻】革厲色曰江革行年六十今日得死為幸誓不為人執筆延明知不可屈乃止日給脱粟飯三升僅全其生而已上密召夏侯亶還使休兵合肥俟淮堰成復進【夏戶雅翻復扶又翻】 癸未魏大赦改元孝昌 破六韓拔陵圍魏廣陽王深于五原軍主賀拔勝募二百人開東門出戰斬首百餘級賊稍退深拔軍向朔州勝常為殿【殿丁練翻】雲州刺史費穆招撫離散四面拒敵時北境州鎮皆沒唯雲中一城獨存【去年李崇使費穆守雲中】道路阻絶援軍不至糧仗俱盡穆弃城南奔爾朱榮於秀容既而詣闕請罪詔原之長流參軍于謹【長流參軍主禁防從公府置長流參軍小府無長流置禁防參軍】言於廣陽王深曰今寇盜蠭起未易專用武力勝也謹請奉大王之威命諭以禍福庶幾稍可離也【易以豉翻幾居依翻】深許之謹兼通諸國語乃單騎詣叛胡營見其酋長開示恩信於是西部鐵勒酋長乜列河等將三萬餘戶南詣深降【騎奇寄翻酋慈秋翻長知兩翻乜母野翻虜姓也將即亮翻降戶江翻】深欲引兵至折敷嶺迎之【通典作折敦嶺】謹曰破六韓拔陵兵勢甚盛聞乜列河等來降必引兵邀之若先據險要未易敵也不若以乜列河餌之而伏兵以待之必可破也深從之拔陵果引兵邀擊乜列河盡俘其衆伏兵拔陵大敗復得乜列河之衆而還【復扶又翻下輩復同】柔然頭兵可汗大破破六韓拔陵【可從刋入聲汗音寒】斬其將孔雀等拔陵避柔然南徙渡河【此河謂北河也】將軍李叔仁以拔陵稍偪求援于廣陽王深深帥衆赴之【帥讀曰率】賊前後降附者二十萬人深與行臺元纂表乞於恒州北别立郡縣安置降戶随宜賑貸息其亂心魏朝不從【朝直遥翻】詔黄門侍郎楊昱分處之冀定瀛三州就食【處昌呂翻】深謂纂曰此輩復為乞活矣【乞活事見八十六卷晉惠帝光熙元年按繫年圖書是年蠕蠕殺破六韓拔陵通鑑明年書拔陵遣費律誘斬胡琛】 秋七月壬戌大赦 八月魏柔玄鎮民杜洛周聚衆反于上谷改元真王攻沒郡縣高歡蔡儁尉景及段榮安定彭樂皆從之洛周圍魏燕州刺史博陵崔秉【燕因肩翻】九月丙辰魏以幽州刺史常景兼尚書為行臺與幽州都督元譚討之景爽之孫也【魏之儒風振於常爽】自盧龍塞至軍都關皆置兵守險譚屯居庸關【盧龍關在遼西肥如縣唐改肥如為盧龍縣杜佑曰盧龍塞在平州盧龍縣城西北二百里軍都關在燕郡軍都縣唐志曰幽州昌平縣北十五里有軍都陘西北三十五里有納疑關即居庸故關亦謂之軍都關考之漢志上谷郡有軍都居庸兩縣蓋各縣有關凡此屯守皆以防杜洛周水經注居庸關在上谷沮陽城東南六十里軍都關在居庸山南】 冬十月吐谷渾遣兵擊趙天安天安降凉州復為魏【去年趙天安以凉州應莫折念生復扶又翻降戶江翻】平西將軍高徽奉使嚈噠還至枹罕【嚈益涉翻噠當割翻又宅軋翻枹音膚】會河州刺史元祚卒【卒子恤翻】前刺史梁釗之子景進引莫折念生兵圍其城長史元永等推徽行州事勒兵固守景進亦自行州事徽請兵于吐谷渾吐谷渾救之【吐從暾入聲谷音浴】景進敗走徽湖之孫也【史述高徽與高歡同其所自出】 魏方有事於西北二荆西郢羣蠻皆反斷三鵶路【西荆治上洛北荆治襄城西郢治汝南真陽縣杜佑曰北荆州今即伊陽縣三鵶鎮在今汝州魯陽縣西南十九里名高平城古遶角城在縣東南又云百里山在鄧州向城縣北是三鵶之第一鵶又北分嶺山北即三鵶之第二鵶其第三鵶入汝州魯山縣界斷丁管翻】殺都督寇掠北至襄城汝水有冉氏向氏田氏種落最盛【三姓盖皆居汝源種章勇翻】其餘大者萬家小者千室各稱王侯屯據險要道路不通十二月壬午魏主下詔曰朕將親御六師掃蕩逋穢令先討荆蠻疆理南服時羣蠻引梁將曹義宗等圍魏荆州【此荆州治穰城將即亮翻】魏都督崔暹將兵數萬救之至魯陽不敢進魏更以臨淮王彧為征南大將軍將兵討魯陽蠻【更工衡翻】司空長史辛雄為行臺左丞東趣葉城【葉城時為襄州治所此即漢南陽郡之葉縣城也趣七喻翻葉式涉翻】别遣征虜將軍裴衍恒農太守京兆王羆【恒戶登翻 考異曰周書羆傳羆未嘗為恒農太守今從魏書】將兵一萬自武關出通三鵶路以救荆州衍等未至彧軍已屯汝上【汝上汝水之上也】州郡被蠻寇者争來請救彧以處分道别不欲應之【被皮義翻處昌呂翻分扶問翻】辛雄曰今裴衍未至王士衆已集蠻左唐突【自宋以來豫部諸蠻率謂之蠻左所置蠻郡謂之左郡】撓亂近畿【撓呼高翻擾也】王秉麾閫外見可而進何論别道彧恐後有得失之責邀雄符下【符尚書行臺符也下遐稼翻】雄以羣蠻聞魏主將自出心必震動可乘勢破也遂符彧軍令速赴擊羣蠻聞之果散走魏主欲自出討賊中書令袁飜諫而止辛雄自軍中上疏曰凡人所以臨陳忘身觸白刃而不憚者一求榮名二貪重賞三畏刑罰四避禍難【陳讀曰陣難乃旦翻】非此數者雖聖王不能使其臣慈父不能厲其子矣明主深知其情故賞必行罰必信使親疎貴賤勇怯賢愚聞鍾鼓之聲見旌旗之列莫不奮激競赴敵場豈懕久生而樂速死哉【懕與厭同樂音洛】利害懸於前欲罷不能耳自秦隴逆節蠻左亂常己歷數載【載子亥翻】三方之師敗多勝少跡其所由不明賞罰之故也【三方之師謂西討秦隴北禦邊鎮南擊蠻左也】陛下雖降明詔賞不移時然將士之勲歷稔不决【歷稔猶言歷年一年五穀一稔故以年為稔】亡軍之卒晏然在家是使節士無所勸慕庸人無所畏懾【懾之涉翻】進而擊賊死交而賞賒【賒遠也】退而逃散身全而無罪此其所以望敵奔沮不肯盡力者也陛下誠能號令必信賞罰必行則軍威必張盜賊必息矣疏奏不省【省悉景翻】曹義宗等取順陽馬圈【圈求遠翻】與裴衍等戰於淅陽【漢弘農郡有析縣晉分屬順陽郡元魏置淅陽郡以其地在淅水之陽也即隋唐南鄉内鄉二縣之地】義宗等敗退衍等復取順陽進圍馬圈洛州刺史董紹以馬圈城堅衍等糧少上書言其必敗未幾義宗擊衍等破之復取順陽【復扶又翻少詩沼翻上時掌翻幾居豈翻】魏以王羆為荆州刺史 邵陵王綸攝南徐州事在州喜怒不恒肆行非法遨遊市里問賣䱇者曰刺史何如對言躁虐綸怒令吞䱇而死【䱇與鱓同音市演翻鱓魚似蛇今江東溝港皆有之躁則到翻】百姓惶駭道路以目【道路相逢者但以目相視而不敢言】嘗逢喪車奪孝子服而著之【著則略翻】匍匐號叫【號戶刀翻下悲號同】籖帥懼罪密以聞上始嚴責綸而不能改於是遣代綸悖慢逾甚【帥所類翻悖蒲内翻又蒲沒翻】乃取一老翁短瘦類上者加以衮冕置之高坐朝以為君自陳無罪使就坐剥褫【坐徂卧翻朝直遥翻禠敕豸翻】捶之於庭又作新棺貯司馬崔會意以轜車挽歌為送葬之法【捶止藥翻貯丁呂翻轜音而喪車也莊子曰紼謳所生必於斥苦紼謳挽歌也左傳公孫夏使其徒歌虞殯杜預注曰虞殯送葬歌曲田横之死其徒有蒿里薤露之歌搜神記曰挽歌喪家之樂執紼者相和之聲】使嫗乘車悲號【嫗威遇翻】會意不能堪輕騎還都以聞上恐其奔逸以禁兵取之將于獄賜盡太子統流涕固諫得免戊子免綸官削爵土 魏山胡劉蠡升反自稱天子置百官【山胡即汾州之稽胡】 初敕勒酋長斛律金事懷朔鎮將楊鈞為軍主行兵用匈奴法望塵知馬步多少嗅地知軍遠近【酋慈秋翻長知兩翻將即亮翻少詩沼翻嗅許敕翻】及破六韓拔陵反金擁衆歸之拔陵署金為王既而知拔陵終無所成乃詣雲州降仍稍引其衆南出黄瓜堆【水經注桑乾水與武周水合而東南流屈逕黄瓜堆南又東南流逕桑乾郡北】為杜洛周所破脱身歸爾朱榮榮以為别將<br />
<br />
  資治通鑑卷一百五十  <br>
   </div> 

<script src="/search/ajaxskft.js"> </script>
 <div class="clear"></div>
<br>
<br>
 <!-- a.d-->

 <!--
<div class="info_share">
</div> 
-->
 <!--info_share--></div>   <!-- end info_content-->
  </div> <!-- end l-->

<div class="r">   <!--r-->



<div class="sidebar"  style="margin-bottom:2px;">

 
<div class="sidebar_title">工具类大全</div>
<div class="sidebar_info">
<strong><a href="http://www.guoxuedashi.com/lsditu/" target="_blank">历史地图</a></strong>  
<a href="http://www.880114.com/" target="_blank">英语宝典</a>  
<a href="http://www.guoxuedashi.com/13jing/" target="_blank">十三经检索</a> 
<br><strong><a href="http://www.guoxuedashi.com/gjtsjc/" target="_blank">古今图书集成</a></strong> 
<a href="http://www.guoxuedashi.com/duilian/" target="_blank">对联大全</a> <strong><a href="http://www.guoxuedashi.com/xiangxingzi/" target="_blank">象形文字典</a></strong> 

<br><a href="http://www.guoxuedashi.com/zixing/yanbian/">字形演变</a>  <strong><a href="http://www.guoxuemi.com/hafo/" target="_blank">哈佛燕京中文善本特藏</a></strong>
<br><strong><a href="http://www.guoxuedashi.com/csfz/" target="_blank">丛书&方志检索器</a></strong> <a href="http://www.guoxuedashi.com/yqjyy/" target="_blank">一切经音义</a>  

<br><strong><a href="http://www.guoxuedashi.com/jiapu/" target="_blank">家谱族谱查询</a></strong>  <strong><a href="http://shufa.guoxuedashi.com/sfzitie/" target="_blank">书法字帖欣赏</a></strong> 
<br>

</div>
</div>


<div class="sidebar" style="margin-bottom:0px;">

<font style="font-size:22px;line-height:32px">QQ交流群9:489193090</font>


<div class="sidebar_title">手机APP 扫描或点击</div>
<div class="sidebar_info">
<table>
<tr>
	<td width=160><a href="http://m.guoxuedashi.com/app/" target="_blank"><img src="/img/gxds-sj.png" width="140"  border="0" alt="国学大师手机版"></a></td>
	<td>
<a href="http://www.guoxuedashi.com/download/" target="_blank">app软件下载专区</a><br>
<a href="http://www.guoxuedashi.com/download/gxds.php" target="_blank">《国学大师》下载</a><br>
<a href="http://www.guoxuedashi.com/download/kxzd.php" target="_blank">《汉字宝典》下载</a><br>
<a href="http://www.guoxuedashi.com/download/scqbd.php" target="_blank">《诗词曲宝典》下载</a><br>
<a href="http://www.guoxuedashi.com/SiKuQuanShu/skqs.php" target="_blank">《四库全书》下载</a><br>
</td>
</tr>
</table>

</div>
</div>


<div class="sidebar2">
<center>


</center>
</div>

<div class="sidebar"  style="margin-bottom:2px;">
<div class="sidebar_title">网站使用教程</div>
<div class="sidebar_info">
<a href="http://www.guoxuedashi.com/help/gjsearch.php" target="_blank">如何在国学大师网下载古籍?</a><br>
<a href="http://www.guoxuedashi.com/zidian/bujian/bjjc.php" target="_blank">如何使用部件查字法快速查字?</a><br>
<a href="http://www.guoxuedashi.com/search/sjc.php" target="_blank">如何在指定的书籍中全文检索?</a><br>
<a href="http://www.guoxuedashi.com/search/skjc.php" target="_blank">如何找到一句话在《四库全书》哪一页?</a><br>
</div>
</div>


<div class="sidebar">
<div class="sidebar_title">热门书籍</div>
<div class="sidebar_info">
<a href="/so.php?sokey=%E8%B5%84%E6%B2%BB%E9%80%9A%E9%89%B4&kt=1">资治通鉴</a> <a href="/24shi/"><strong>二十四史</strong></a>&nbsp; <a href="/a2694/">野史</a>&nbsp; <a href="/SiKuQuanShu/"><strong>四库全书</strong></a>&nbsp;<a href="http://www.guoxuedashi.com/SiKuQuanShu/fanti/">繁体</a>
<br><a href="/so.php?sokey=%E7%BA%A2%E6%A5%BC%E6%A2%A6&kt=1">红楼梦</a> <a href="/a/1858x/">三国演义</a> <a href="/a/1038k/">水浒传</a> <a href="/a/1046t/">西游记</a> <a href="/a/1914o/">封神演义</a>
<br>
<a href="http://www.guoxuedashi.com/so.php?sokeygx=%E4%B8%87%E6%9C%89%E6%96%87%E5%BA%93&submit=&kt=1">万有文库</a> <a href="/a/780t/">古文观止</a> <a href="/a/1024l/">文心雕龙</a> <a href="/a/1704n/">全唐诗</a> <a href="/a/1705h/">全宋词</a>
<br><a href="http://www.guoxuedashi.com/so.php?sokeygx=%E7%99%BE%E8%A1%B2%E6%9C%AC%E4%BA%8C%E5%8D%81%E5%9B%9B%E5%8F%B2&submit=&kt=1"><strong>百衲本二十四史</strong></a>  <a href="http://www.guoxuedashi.com/so.php?sokeygx=%E5%8F%A4%E4%BB%8A%E5%9B%BE%E4%B9%A6%E9%9B%86%E6%88%90&submit=&kt=1"><strong>古今图书集成</strong></a>
<br>

<a href="http://www.guoxuedashi.com/so.php?sokeygx=%E4%B8%9B%E4%B9%A6%E9%9B%86%E6%88%90&submit=&kt=1">丛书集成</a> 
<a href="http://www.guoxuedashi.com/so.php?sokeygx=%E5%9B%9B%E9%83%A8%E4%B8%9B%E5%88%8A&submit=&kt=1"><strong>四部丛刊</strong></a>  
<a href="http://www.guoxuedashi.com/so.php?sokeygx=%E8%AF%B4%E6%96%87%E8%A7%A3%E5%AD%97&submit=&kt=1">說文解字</a> <a href="http://www.guoxuedashi.com/so.php?sokeygx=%E5%85%A8%E4%B8%8A%E5%8F%A4&submit=&kt=1">三国六朝文</a>
<br><a href="http://www.guoxuedashi.com/so.php?sokeytm=%E6%97%A5%E6%9C%AC%E5%86%85%E9%98%81%E6%96%87%E5%BA%93&submit=&kt=1"><strong>日本内阁文库</strong></a> <a href="http://www.guoxuedashi.com/so.php?sokeytm=%E5%9B%BD%E5%9B%BE%E6%96%B9%E5%BF%97%E5%90%88%E9%9B%86&ka=100&submit=">国图方志合集</a> <a href="http://www.guoxuedashi.com/so.php?sokeytm=%E5%90%84%E5%9C%B0%E6%96%B9%E5%BF%97&submit=&kt=1"><strong>各地方志</strong></a>

</div>
</div>


<div class="sidebar2">
<center>

</center>
</div>
<div class="sidebar greenbar">
<div class="sidebar_title green">四库全书</div>
<div class="sidebar_info">

《四库全书》是中国古代最大的丛书,编撰于乾隆年间,由纪昀等360多位高官、学者编撰,3800多人抄写,费时十三年编成。丛书分经、史、子、集四部,故名四库。共有3500多种书,7.9万卷,3.6万册,约8亿字,基本上囊括了古代所有图书,故称“全书”。<a href="http://www.guoxuedashi.com/SiKuQuanShu/">详细>>
</a>

</div> 
</div>

</div>  <!--end r-->

</div>
<!-- 内容区END --> 

<!-- 页脚开始 -->
<div class="shh">

</div>

<div class="w1180" style="margin-top:8px;">
<center><script src="http://www.guoxuedashi.com/img/plus.php?id=3"></script></center>
</div>
<div class="w1180 foot">
<a href="/b/thanks.php">特别致谢</a> | <a href="javascript:window.external.AddFavorite(document.location.href,document.title);">收藏本站</a> | <a href="#">欢迎投稿</a> | <a href="http://www.guoxuedashi.com/forum/">意见建议</a> | <a href="http://www.guoxuemi.com/">国学迷</a> | <a href="http://www.shuowen.net/">说文网</a><script language="javascript" type="text/javascript" src="https://js.users.51.la/17753172.js"></script><br />
  Copyright &copy; 国学大师 古典图书集成 All Rights Reserved.<br>
  
  <span style="font-size:14px">免责声明:本站非营利性站点,以方便网友为主,仅供学习研究。<br>内容由热心网友提供和网上收集,不保留版权。若侵犯了您的权益,来信即刪。scp168@qq.com</span>
  <br />
ICP证:<a href="http://www.beian.miit.gov.cn/" target="_blank">鲁ICP备19060063号</a></div>
<!-- 页脚END --> 
<script src="http://www.guoxuedashi.com/img/plus.php?id=22"></script>
<script src="http://www.guoxuedashi.com/img/tongji.js"></script>

</body>
</html>
