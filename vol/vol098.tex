資治通鑑卷九十八   宋 司馬光 撰

胡三省 音註

晉紀二十|{
	起著雍涒灘盡上章閹茂凡三年}


孝宗穆皇帝上之下

永和四年夏四月林邑寇九眞|{
	既再破日南故進寇九眞九眞郡唐愛州}
殺士民什八九 趙秦公韜有寵於趙王虎欲立之以太子宣長|{
	長知兩翻}
猶豫未决宣嘗忤旨|{
	忤五故翻}
虎怒曰悔不立韜也韜由是益驕造堂於太尉府號曰宣光殿梁長九丈|{
	長直亮翻}
宣見之大怒斬匠截梁而去|{
	以犯其名也}
韜怒增之至十丈宣聞之謂所幸楊杯牟成趙生曰凶豎傲愎敢爾|{
	愎弼力翻}
汝能殺之吾入西宮|{
	虎居西宮}
當盡以韜之國邑分封汝等韜死主上必臨喪吾因行大事蔑不濟矣|{
	左傳潘崇謂楚商臣曰能行大事乎杜預注曰大事謂弑君}
杯等許諾秋八月韜夜與僚屬宴於東明觀|{
	水經注石氏立東明觀於鄴東城上}
因宿於佛精舍|{
	佛精舍佛寺也僧徒專精修行之地故謂之精舍事物紀原曰漢明帝於東都門外立精舍以處攝摩騰竺法蘭}
宣使楊杯等緣獼猴梯而入|{
	梯小而長人如獮猴攀緣而上故曰獮猴梯}
殺韜置其刀箭而去旦日宣奏之虎哀驚氣絶久之方蘇將出臨其喪司空李農諫曰害秦公者未知何人賊在京師鑾輿不宜輕出虎乃止嚴兵發哀於太武殿宣往臨韜喪不哭直言呵呵|{
	呵虎何翻呵呵笑聲}
使舉衾觀尸大笑而去|{
	大被曰衾}
收大將軍記事參軍鄭靖尹武等將委之以罪虎疑宣殺韜欲召之恐其不入乃詐言其母杜后哀過危惙|{
	惙陟劣翻類篇丑例翻困劣也言其氣息惙然僅相屬也}
宣不謂見疑入朝中宮因留之|{
	朝直遥翻}
建興人史科知其謀告之|{
	魏土地記曰建興郡治陽阿縣陽阿縣漢屬上黨郡魏收志曰慕容永分上黨置建興郡則其地非石趙所置建興郡也水經注曰田融言趙立建興郡於廣宗城内斯其是矣}
虎使收楊杯牟成皆亡去獲趙生詰之具服|{
	詰去吉翻}
虎悲怒彌甚囚宣於席庫|{
	席庫藏席之所}
以鐵環穿其頷而鏁之取殺韜刀箭舐其血哀號震動宮殿|{
	鏁蘇果翻舐直氏翻號戶高翻}
佛圖澄曰宣韜皆陛下之子今為韜殺宣是重禍也|{
	為于偽翻}
陛下若加慈恕福祚猶長若必誅之宣當為彗星下掃鄴宮|{
	彗祥歲翻又旋芮翻乂徐醉翻}
虎不從積柴於鄴北樹標其上標末置鹿盧穿之以䋲|{
	鹿盧即轆轤}
倚梯柴積送宣其下使韜所幸宦者郝稚劉霸拔其髪抽其舌牽之登梯郝稚以繩貫其頷鹿盧絞上|{
	上時掌翻}
劉霸斷其手足|{
	斷丁管翻}
斫眼潰腸如韜之傷四面縱火煙炎際天|{
	炎讀曰燄}
虎從昭儀已下數千人登中臺以觀之|{
	中臺者三臺之中臺即銅爵臺也}
火滅取灰分置諸門交道中|{
	交道午道也一縱一横為午道}
殺其妻子九人宣少子纔數歲|{
	少詩照翻}
虎素愛之抱之而泣欲赦之其大臣不聽就抱中取而殺之兒挽虎衣大叫至於絶帶虎因此發病又廢其后杜氏為庶人誅其四率已下三百人宦者五十人皆車裂節解弃之漳水|{
	東宫有左右前後四率率所律翻支解者解其四支節解者凡骨節節節解之也}
洿其東宮以養猪牛|{
	洿汪乎翻}
東宮衛士十餘萬人皆謫戍凉州|{
	趙未得凉州置凉州於金城謫使戍凉州之邉也為下高力叛張本}
先是趙攬言於虎曰宮中將有變宜備之及宣殺韜虎疑其知而不告亦誅之|{
	史言趙攬談天於猜暴之朝以自禍先悉薦翻}
朝廷論平蜀之功欲以豫章郡封桓温尚書左丞荀蕤曰温若復平河洛|{
	復扶又翻}
將何以賞之乃加温征西大將軍開府儀同三司封臨賀郡公加譙王無忌前將軍袁喬龍驤將軍封湘西伯|{
	驤思將翻}
蕤崧之子也|{
	荀崧荀藩之弟永嘉之禍相與建行臺於密建興之初又嘗鎮宛}
温旣滅蜀威名大振朝廷憚之會稽王昱以揚州刺史殷浩有盛名朝野推服引為心膂與參綜朝權欲以抗温由是與温寖相疑貳|{
	為温廢浩脅制朝廷張本朝直遥翻會工外翻}
浩以征北長史荀羨前江州刺史王羲之夙有令名擢羨為吳國内史|{
	江左郡國以吳為甲}
羲之為護軍將軍以為羽翼羨蕤之弟羲之導之從子也|{
	從才用翻}
羲之以為内外恊和然後國家可安勸浩不宜與温搆隙浩不從 燕王皝有疾召世子雋屬之曰|{
	屬之欲翻}
今中原未平方資賢傑以經世務恪知勇兼濟才堪任重汝其委之以成吾志又曰陽士秋士行高潔忠幹貞固|{
	陽騖字士秋行卜孟翻}
可託大事汝善待之九月丙申薨|{
	年五十二}
趙王虎議立太子太尉張舉曰燕公斌有武畧|{
	斌音彬}


彭城公遵有文德惟陛下所擇虎曰卿言正起吾意戎昭將軍張犲曰|{
	載記曰虎置左右戎昭曜武將軍位在左右衛上}
燕公母賤又嘗有過|{
	謂欲殺張賀度也事見九十六卷成帝咸康六年}
彭城公母前以太子事廢|{
	遵與邃同母鄭氏廢見九十五卷咸康三年}
今立之臣恐不能無微恨陛下宜審思之初虎之拔上邽也|{
	見九十四卷成帝咸和四年}
張犲獲前趙主曜幼女安定公主有殊色納於虎虎嬖之|{
	嬖卑義翻又博計翻}
生齊公世犲以虎老病欲立世爲嗣冀劉氏為太后已得輔政乃說虎曰陛下再立太子其母皆出於倡賤|{
	說輸芮翻倡音昌}
故禍亂相尋今宜擇母貴子孝者立之虎曰卿勿言吾知太子處矣虎再與羣臣議於東堂虎曰吾欲以純灰三斛自滌其腸何為專生惡子年踰二十輒欲殺父今世方十歲比其二十吾已老矣|{
	比必寐翻}
乃與張舉李農定議令公卿上書請立世為太子大司農曹莫不肯署名虎使張犲問其故莫頓首曰天下重器不宜立少|{
	少詩照翻}
故不敢署虎曰莫忠臣也然未達朕意張舉李農知朕意矣可令諭之遂立世為太子以劉昭儀為后|{
	虎父子相殘廢長立少天將假手於冉閔以夷其種類也}
冬十一月甲辰葬燕文明王|{
	皝諡曰文明}
世子雋即位|{
	雋字宣英皝之第二子}
赦境内遣使詣建康告喪|{
	使疏吏翻}
以弟交為左賢王左長史陽騖為郎中令 十二月以左光禄大夫領司徒錄尚書事蔡謨為侍中司徒謨上疏固讓謂所親曰我若為司徒將為後代所哂|{
	哂式忍翻}
義不敢拜也

五年春正月辛未朔大赦 趙王虎即皇帝位|{
	虎以成帝咸康三年即天王位今即皇帝位}
大赦改元太寧諸子皆進爵為王故東宮高力等萬餘人謫戍凉州|{
	石宣簡多力之士以衛東宮號曰高力置督將以領之}
行達雍城|{
	扶風雍縣城也雍於用翻下同}
旣不在赦例又敕雍州刺史張茂送之茂皆奪其馬使之步推鹿車致糧戍所|{
	推吐雷翻風俗通曰鹿車窄小裁容一鹿}
高力督定陽梁犢|{
	定陽縣漢屬上郡晉省後魏太安中置定陽郡唐為延州臨真縣}
因衆心之怨謀作亂東歸衆聞之皆踊抃大呼|{
	踊跳躍也抃拊手也呼火故翻}
犢乃自稱晉征東大將軍帥衆攻拔下辨|{
	辨步莧翻}
安西將軍劉寧自安定擊之為犢所敗|{
	敗補邁翻}
高力皆多力善射一當十餘人雖無兵甲掠民斧施一丈柯攻戰若神|{
	柯斧柄也}
所向崩潰戍卒皆隨之攻陷郡縣殺長吏二千石|{
	長知兩翻}
長驅而東比至長安|{
	比必寐翻}
衆已十萬樂平王苞盡銳拒之一戰而敗犢遂東出潼關進趣洛陽|{
	趣七喻翻}
趙主虎以李農為大都督行大將軍事統衛軍將軍張賀度等步騎十萬討之戰于新安|{
	新安縣漢屬弘農郡自晉以後屬河南郡騎奇寄翻下同}
農等大敗戰于洛陽又敗退壁成臯犢遂東掠滎陽陳留諸郡|{
	武帝泰始二年分河南置滎陽郡}
虎大懼以燕王斌為大都督督中外諸軍事統冠軍大將軍姚弋仲車騎將軍蒲洪等討之|{
	斌音彬冠古玩翻}
弋仲將其衆八千餘人至鄴|{
	自灄頭至鄴將即亮翻}
求見虎虎病未之見引入領軍省|{
	領軍省領軍將軍視事之所}
賜以已所御食弋仲怒不食曰主上召我來擊賊當面見授方略我豈為食來邪|{
	為于偽翻下羌為同}
且主上不見我我何以知其存亡邪虎力疾見之弋仲讓虎曰兒死愁邪何為而病兒幼時不擇善人敎之使至於為逆旣為逆而誅之又何愁焉且汝久病所立兒幼|{
	謂太子世也}
汝若不愈天下必亂當先憂此勿憂賊也犢等窮困思歸相聚為盗所過殘暴何所能至老羌為汝一舉了之|{
	了决也}
弋仲性狷直人無貴賤皆汝之虎亦不之責|{
	狷吉掾翻以石虎之猜暴而能容姚弋仲之狷直者以其當理而切於事情也苟徒狷直而不切當殆難以免矣}
於坐授使持節征西大將軍賜以鎧馬|{
	坐狙臥翻使疏吏翻鎧可亥翻}
弋仲曰汝看老羌堪破賊否乃被鎧跨馬于庭中|{
	被皮義翻}
因策馬南馳不辭而出遂與斌等擊犢於滎陽大破之斬犢首而還討其餘黨盡滅之虎命弋仲劒履上殿入朝不趨|{
	上時掌翻}
進封西平郡公蒲洪為車騎大將軍開府儀同三司都督雍秦州諸軍事雍州刺史|{
	雍於用翻}
進封略陽郡公 始平人馬朂聚兵自稱將軍|{
	杜佑曰漢平陵晉改為始平有馬嵬故城}
趙樂平王苞討滅之誅三千餘家夏四月益州刺史周撫龍驤將軍朱燾擊范賁斬之益州平|{
	范賁亂始上卷三年驤始將翻}
詔遣謁者陳沈如燕|{
	沈持林翻}
拜慕容雋為使持節侍中大都督督河北諸軍事幽平二州牧|{
	使疏吏翻 考異曰雋載記云幽冀并平四州牧今從帝記}
大將軍大單于燕王|{
	單音蟬}
桓温遣督護滕畯帥交廣之兵擊林邑王文於盧容|{
	畯祖峻翻盧容縣自漢以來屬日南郡有盧容浦去郡二百里帥讀曰率}
為文所敗|{
	敗補邁翻}
退屯九真 乙卯趙王虎病甚以彭城王遵為大將軍鎮關右燕王斌為丞相錄尚書事張犲為鎮衛大將軍領軍將軍吏部尚書|{
	劉聰置十六大將軍鎮衛其一也石虎置鎮衛將軍在車騎將軍上今以張犲為鎮衛大將軍崇其號也領軍將軍則掌兵柄吏部尚書則典選事是文武二柄悉以付犲}
並受遺詔輔政劉后惡斌輔政恐不利於太子與張犲謀去之|{
	惡烏路翻去羌呂翻}
斌時在襄國遣使詐謂斌曰主上疾已漸愈王須獵者可少停也斌素好獵嗜酒|{
	少詩沼翻好呼到翻}
遂留獵且縱酒劉氏與犲因矯詔稱斌無忠孝之心免官歸第使犲弟雄帥龍騰五百人守之|{
	此石虎幽石宏之故智也張犲踵而用之非虎敎之邪石斌亦庸人耳君父疾篤徵之受遺輔政就使虎疾少愈亦當夜衣而行乃酣酒縱獵其無智識如此就使得權諸弟亦將紾其臂而奪之張舉謂有武略妄矣}
乙丑遵自幽州至鄴敕朝堂受拜|{
	朝直遥翻}
配禁兵三萬遣之遵涕泣而去是日虎疾小瘳問遵至未左右對曰去已久矣虎曰恨不見之虎臨西閤|{
	太武殿之西閤也}
龍騰中郎二百餘人列拜於前虎問何求皆曰聖體不安宜令燕王入宿衛典兵馬或言乞以為皇太子虎曰燕王不在内邪召以來左右言王酒病不能入虎曰促持輦迎之當付璽綬亦竟無行者|{
	左右皆為劉后母子故竟無行者璽斯氏翻綬音受}
尋惛眩而入|{
	惛迷忘也眩目視亂也}
張犲使張雄矯詔殺斌|{
	殺斌欲以一衆心豈知已有從臾石遵入立者}
戊辰劉氏復矯詔以犲為太保都督中外諸軍錄尚書事如霍光故事|{
	復扶又翻}
侍中徐統歎曰亂將作矣吾無為預之仰藥而死己巳虎卒太子世即位尊劉氏為皇太后劉氏臨朝稱制|{
	朝直遥翻}
以張犲為丞相犲辭不受請以彭城王遵義陽王鑒為左右丞相以慰其心劉氏從之犲與太尉張舉謀誅司空李農舉素與農善密告之農奔廣宗帥乞活數萬家保上白|{
	乞活李惲田徽之餘衆也自永嘉以來屯聚於上白帥讀曰率}
劉氏使張舉統宿衛諸軍圍之犲以張離為鎮軍大將軍監中外諸軍事以為已副|{
	監工銜翻}
彭城王遵至河内聞喪姚弋仲蒲洪劉寧及征虜將軍石閔武衛將軍王鸞等討梁犢還遇遵於李城|{
	續漢志河内平睪縣有李城史記邯鄲李同却秦兵趙封其父李侯即此城}
共說遵曰殿下長且賢先帝亦有意以殿下為嗣|{
	謂虎欲從張舉之言也說輸芮翻長知兩翻}
正以末年惛惑為張犲所誤今女主臨朝姧臣用事上白相持未下京師宿衛空虛殿下若聲張豺之罪鼓行而討之其誰不開門倒戈而迎殿下者遵從之遵自李城舉兵還趣鄴|{
	趣七喻翻}
洛州刺史劉國帥洛陽之衆往會之檄至鄴張豺大懼馳召上白之軍丙戌遵軍於蕩陰|{
	蕩音湯}
戎卒九萬石閔為前鋒耆雋羯士皆曰彭城王來奔喪吾當出迎之不能為張豺守城也|{
	方言曰耆長也說文曰老也左傳注曰強也禮記音義曰至也言至老境也羯士石氏之種類也為于偽翻}
踰城而出豺斬之不能止張離亦帥龍騰二千斬關迎遵劉氏懼召張豺入對之悲哭曰先帝梓宫未殯而禍難至此|{
	難乃旦翻}
今嗣子冲幼託之將軍將軍將若之何欲加遵重位能弭之乎豺惶怖不知所出|{
	怖普布翻}
但云唯唯|{
	唯于癸翻}
乃下詔以遵為丞相領大司馬大都督督中外諸軍錄尚書事加黄鉞九錫己丑遵至安陽亭|{
	安陽縣屬魏郡此蓋安陽縣都亭也}
張豺懼而出迎遵命執之庚寅遵擐甲曜兵入自鳳陽門升太武前殿擗踊盡哀|{
	擐音宦擗毗亦翻拊心也踊跳也}
退如東閤|{
	太武殿之東閤也}
斬張豺于平樂市|{
	鄴都有平樂市樂音洛}
夷其三族假劉氏令曰嗣子幼冲先帝私恩所授皇業至重非所克堪其以遵嗣位於是遵即位大赦罷上白之圍辛卯封世為譙王|{
	載記曰世凡立三十三日}
廢劉氏為太妃 |{
	考異曰晉春秋及十六國春秋鈔皆云廢太后為昭儀今從載記十六國春秋及載記又云世立三十三日按四月乙巳至五月庚寅凡二十一日}
尋皆殺之李農來歸罪使復其位尊母鄭氏為皇太后|{
	鄭氏即邃母鄭后被廢見九十五卷成帝咸康三年}
立妃張氏為皇后故燕王斌子衍為皇太子以義陽王鍳為侍中太傅沛王冲為太保樂平公苞為大司馬汝陰王琨為大將軍武興公閔為都督中外諸軍事輔國大將軍|{
	為石閔得兵柄以夷胡羯張本}
甲午鄴中暴風拔樹震電雨雹大如盂升|{
	如盂及升也雨于具翻}
太武暉華殿災及諸門觀閤蕩然無餘乘輿服御燒者大半|{
	觀古玩翻乘繩證翻}
金石皆盡火月餘乃滅時沛王冲鎮薊|{
	冲蓋代遵鎮薊薊音計}
聞遵殺世自立謂其僚佐曰世受先帝之命遵輒廢而殺之罪莫大焉其敕内外戒嚴孤將親討之於是留寧北將軍沭堅戍幽州|{
	沭食律翻姓也}
帥衆五萬自薊南下傳檄燕趙所在雲集比至常山|{
	帥讀曰率比必寐翻}
衆十餘萬軍于苑鄉遇遵赦書冲曰皆吾弟也死者不可復追何為復相殘乎|{
	復扶又翻下同}
吾將歸矣|{
	欲歸薊也}
其將陳暹曰彭城簒弑自尊為罪大矣王雖北斾臣將南轅俟平京師擒彭城然後奉迎大駕冲乃復進遵馳遣王擢以書喻冲冲弗聼遵使武興公閔及李農帥精卒十萬討之戰于平棘|{
	平棘縣漢屬常山郡晉屬趙國唐為趙州治所}
冲兵大敗獲冲于元氏|{
	元氏縣漢屬常山郡晉屬趙國闞駰曰趙公子元之封邑故曰元氏唐元氏縣屬趙州劉昫曰元氏漢常山郡所治故城在今趙州元氏縣西南}
賜死阬其士卒三萬餘人武興公閔言於遵曰蒲洪人傑也今以洪鎮關中臣恐秦雍之地非國家之有|{
	雍於用翻}
此雖先帝臨終之命然陛下踐祚自宜改圖遵從之罷洪都督餘如前制洪怒歸枋頭遣使來降|{
	洪屯枋頭見九十五卷成帝咸和八年降戶江翻}
燕平狄將軍慕容霸上書於燕王雋曰石虎窮凶極暴天之所弃餘燼僅存自相魚肉今中國倒懸企望仁恤若大軍一振勢必投戈北平太守孫興亦表言石氏大亂宜以時進取中原雋以新遭大喪弗許|{
	以去年皝薨也}
霸馳詣龍城言於雋曰難得而易失者時也萬一石氏衰而復興|{
	易以䜴翻復扶又翻}
或有英雄據其成資|{
	謂中原或有英雄乘亂而取趙據有其已成之資也}
豈惟失此大利亦恐更為後患雋曰鄴中雖亂鄧恒據安樂|{
	以前參考安樂當作樂安}
兵彊糧足今若伐趙東道不可由也當由盧龍盧龍山徑險狹|{
	水經注濡水東南逕盧龍塞塞道自無終縣東出渡濡水向林蘭陘東至清陘盧龍之險峻阪縈折故有九崢之名}
虜乘高斷要|{
	斷丁管翻}
首尾為患將若之何霸曰恒雖欲為石氏拒守其將士顧家人懷歸志若大軍臨之自然瓦解臣請為殿下前驅東出徒河潛趨令支出其不意|{
	為于偽翻趨七喻翻令音鈴又即定翻支音祁}
彼聞之勢必震駭上不過閉門自守下不免弃城逃潰何暇禦我哉然則殿下可以安步而前無復留難矣雋猶豫未决以問五材將軍封奕|{
	燕置五材將軍蓋取宋子罕所謂天生五材誰能去兵之義}
對曰用兵之道敵彊則用智敵弱則用勢是故以大呑小猶狼之食豚也以治易亂猶日之消雪也大王自上世以來積德累仁兵彊士練石虎極其殘暴死未瞑目|{
	瞑莫定翻}
子孫爭國上下乖亂中國之民墜於塗炭延頸企踵以待振拔大王若揚兵南邁先取薊城次指鄴都宣耀威德懷撫遺民彼孰不扶老提幼以迎大王凶黨將望旗氷碎安能為害乎從事中郎黄泓曰|{
	泓烏宏翻}
今太白經天歲集畢北陰國受命此必然之驗也|{
	漢書天文志太白經天天下革民更王孟康注曰謂出東入西出西入東也太白陰星出東當伏東出西當㐲西過午為經天晉灼曰日陽也日出則星亡晝見午上為經天歲星所在國不可伐可以伐人昴畢間為天街其陰陰國歲集畢北明陰國當受命而王}
宜速出師以承天意折衝將軍慕輿根曰中國之民困於石氏之亂咸思易主以救湯火之急此千載一時不可失也|{
	載子亥翻}
自武宣王以來|{
	慕容廆謚武宣王}
招賢養民務農訓兵正俟今日今時至不取更復顧慮|{
	復扶又翻}
豈天意未欲使海内平定邪將大王不欲取天下也雋笑而從之以慕容恪為輔國將軍慕容評為輔弼將軍左長史陽騖為輔義將軍謂之三輔|{
	輔弼輔義二將軍號亦一時創置}
慕容霸爲前鋒都督建鋒將軍|{
	建鋒將軍亦創置也}
選精兵二十餘萬講武戒嚴為進取之計 |{
	考異曰燕景昭紀集兵在四月時石虎方死諸子未爭十六國春秋在五月故從之而燕書載封奕慕輿根言俱指冉閔按是時閔未篡趙蓋撰史者附會耳故削去}
六月葬趙王虎於顯原陵廟號太祖桓温聞趙亂出屯安陸|{
	安陸縣自漢以來屬江夏郡唐為安州治所温自江陵出屯安陸}
遣諸將經營北方|{
	將即亮翻}
趙揚州刺史王浹舉壽春降|{
	浹即恊翻降戶江翻下同}
西中郎將陳逵進據壽春征北大將軍褚裒上表請伐趙|{
	褚裒時鎮京口裒蒲侯翻}
即日戒嚴直指泗口朝議以裒事任貴重宜先遣偏帥|{
	裒太后之父又當方面故云事任貴重}
裒奏言前已遣督護王頤之等徑造彭城|{
	造七到翻}
後遣督護麋嶷進據下邳|{
	嶷魚力翻}
今宜速發以成聲勢秋七月加裒征討大都督督徐兖青揚豫五州諸軍事裒帥衆三萬徑赴彭城|{
	帥讀曰率}
北方士民降附者日以千計朝野皆以為中原指期可復光祿大夫蔡謨獨謂所親曰胡滅誠為大慶然恐更貽朝廷之憂其人曰何謂也謨曰夫能順天乘時濟羣生於艱難者非上聖與英雄不能為也自餘則莫若度德量力|{
	度徒洛翻量音良}
觀今日之事殆非時賢所及必將經營分表疲民以逞|{
	言必不能長驅以定中原勢須隨所得之地分列屯戍畫境而守疲民以逞其志也或說分音扶問翻言人之才具各有分量收復中原非當時人才所能辦也經之營之過於其分量之外則不能成功丁壯苦征戰老弱困轉輸疲民以逞而不能濟也}
旣而才畧踈短不能副心財殫力竭智勇俱困安得不憂及朝廷乎|{
	其後殷浩之敗卒如蔡謨所料}
魯郡民五百餘家相與起兵附晉求援於褚裒裒遣部將王龕李邁將銳卒三千迎之|{
	龕苦含翻將即亮翻}
趙南討大都督李農帥騎二萬與龕等戰于代陂|{
	騎奇寄翻}
龕等大敗皆没於趙八月裒退屯廣陵陳逵聞之焚壽春積聚毁城遁還|{
	積子賜翻聚慈諭翻}
裒上疏乞自貶詔不許命裒還鎮京口解征討都督時河北大亂遺民二十餘萬口渡河欲來歸附會裒已還威勢不接皆不能自拔死亡略盡 |{
	考異曰裒傳云為慕容皝及苻健所掠死亡咸盡按是時慕容皝卒已踰年矣永和六年慕容雋始率衆南征石鍳即位後蒲洪始有衆十萬永和六年洪死健始嗣位皆與裒不相接今不取}
趙樂平王苞謀帥關右之衆攻鄴|{
	帥讀曰率下同}
左長史石光司馬曹曜等固諫苞怒殺光等百餘人苞性貪而無謀雍州豪傑知其無成並遣使告晉梁州刺史司馬勲帥衆赴之|{
	勲宣帝弟子濟南王遂之曾孫雍於用翻使疏吏翻}
楊初襲趙西城破之|{
	楊初武都氐王也此西城蓋即漢隗囂奔處}
九月凉州官屬共上張重華為丞相凉王雍秦凉三州牧|{
	上時掌翻}
重華屢以錢帛賜左右寵臣又喜博奕|{
	博樗蒲奕奕棋喜許記翻}
頗廢政事徵事索振諫曰先王夙夜勤儉以實府庫正以讐耻未雪志平海内故也殿下嗣位之初彊寇侵逼|{
	徵事凉所置官謂趙來攻也事見上卷二年三年索惜各翻}
賴重餌之故得戰士死力|{
	謂以錢帛厚賞戰士得其出力致死}
僅保社稷今蓄積已虛而寇讐尚在豈可輕有耗散以與無功之人乎昔漢光武躬親萬機章奏詣闕報不終日故能隆中興之業今章奏停滯動經時月|{
	三月為一時三旬為一月}
下情不得上通沈寃困於囹圄|{
	沈持林翻囹盧經翻獄也圄偶許翻守也}
殆非明主之事也重華謝之 司馬勲出駱谷破趙長城戍|{
	長城戍即魏司馬望鄧艾據之以拒姜維之地}
壁于懸鉤去長安二百里使治中劉煥攻長安斬京兆太守劉秀離|{
	劉秀離蓋迎戰而敗死渙未能至長安城下也}
又拔賀城三輔豪傑多殺守令以應勲凡三十餘壁衆五萬人趙樂平王苞乃輟攻鄴之謀使其將麻秋姚國等將兵拒勲趙主遵遣車騎將軍王朗帥精騎二萬以討勲為名因刼苞送鄴勲兵少畏朗不敢進|{
	使桓温於是時攻關中關中可取也少詩沼翻}
冬十月釋懸鉤拔宛城殺趙南陽太守袁景復還梁州|{
	宛於元翻復扶又翻}
初趙主遵之發李城也謂武興公閔曰努力事成以爾為太子旣而立太子衍閔恃功欲專朝政|{
	朝直遥翻}
遵不聼閔素驍勇屢立戰功夷夏宿將皆憚之|{
	驍堅堯翻}
旣為都督總内外兵權乃撫循殿中將士皆奏為殿中員外將軍爵關外侯|{
	殿中將軍晉初置殿中員外將軍又後來所置也關外侯漢獻帝建安二十年魏武王置所謂名號侯也余按秦漢列侯則有國邑關内侯無國邑列位於朝無官位者居京師故謂之關内侯列侯就國者多出關外後曹操置關外侯於關内侯之下非秦漢列爵意也}
遵弗之疑而更題名善惡以挫抑之衆咸怨怒中書令孟準左衛將軍王鸞勸遵稍奪閔兵權閔益恨望|{
	恨望猶怨望也}
準等咸勸誅之十一月遵召義陽王鑒樂平王苞汝陰王琨淮南王昭等入議於鄭太后前曰閔不臣之迹漸著今欲誅之如何鑒等皆曰宜然鄭氏曰李城還兵無棘奴豈有今日|{
	冉閔小字棘奴}
小驕縱之|{
	謂閔恃功頗驕宜寛縱之}
何可遽殺鑒出遣宦者楊環馳以告閔閔遂劫李農及右衛將軍王基密謀廢遵使將軍蘇彦周成帥甲士三千人執遵於南臺|{
	三臺之南臺也水經注銅雀臺之南則金雀臺高八丈有屋百九十間帥讀曰率}
遵方與婦人彈碁|{
	藝經曰彈碁兩人對局白黑碁各六枚先列碁相當更先彈也其局以石為之局形四隤而中高魏文帝善彈碁能用手巾角時一書生又能低頭以所冠葛巾撇碁劉貢父詩云漢皇初厭蹵鞠勞侍臣始作彈碁戲彈碁蓋始於漢也世說曰彈碁始自魏内宮粧奩之戲此說誤也按西京雜記漢武帝好蹵鞠言事者以為勞體非至尊所宜帝命擇似而不勞者家君作彈碁奏之帝大悦}
問成曰反者誰也成曰義陽王鑒當立遵曰我尚如是鑒能幾時遂殺之於琨華殿|{
	載記曰遵在位凡一百八十三日}
并殺鄭太后張后太子衍孟準王鸞及上光祿張裴|{
	載記曰石虎置上中光祿大夫位在左右光祿大夫上}
鑒即位大赦以武興公閔為大將軍封武德王司空李農為大司馬並錄尚書事郎闓為司空|{
	闓苦亥翻又音開}
秦州刺史劉羣為尚書左僕射侍中盧諶為中書監|{
	諶是壬翻}
秦雍流民相率西歸|{
	成帝咸和四年石虎破殺劉徙氐羌十五萬落于司冀州八年破石生徙秦雍民及氐羌十餘萬戶于關東今因趙亂故相帥西歸雍於用翻帥讀曰率}
路由枋頭共推蒲洪為主衆至十餘萬|{
	蒲洪勸石虎徙秦雍民夷以實關東而身委質於趙及趙之亂得因以為資奸雄俟時而動也}
洪子健在鄴斬關出奔枋頭鑒懼洪之逼欲以計遣之乃以洪為都督關中諸軍事征西大將軍雍州牧領秦州刺史洪會官屬議應受與不|{
	不讀曰否}
主簿程朴請且與趙連和如列國分境而治洪怒曰吾不堪為天子邪而云列國乎引朴斬之|{
	姚弋仲猶知盡忠於石氏蒲洪則直欲奪取之而後已}
都鄉元侯禇裒|{
	漢志常山郡有都鄉侯國晉志不見此特以漢舊侯國名封禇裒耳非必有實土也}
還至京口聞哭聲甚多以問左右對曰皆代陂死者之家也裒慙憤發疾十二月己酉卒以吳國内史荀羨為使持節監徐兖二州揚州之晉陵諸軍事徐州刺史時年二十八中興方伯未有如羨之少者|{
	少詩照翻}
趙主鑒使樂平王苞中書令李松殿中將軍張才夜攻石閔李農於琨華殿不克禁中擾亂鑒懼偽若不知者夜斬松才於西中華門并殺苞新興王祗虎之子也時鎮襄國與姚弋仲蒲洪等連兵|{
	祗北連姚弋仲南連蒲洪以討閔農}
移檄中外欲共誅閔農閔農以汝陰王琨為大都督與張舉及侍中呼延盛帥步騎七萬分討祗等|{
	帥讀曰率騎奇寄翻}
中領軍石成侍中石啓前河東太守石暉謀誅閔農閔農皆殺之龍驤將軍孫伏都劉銖等帥羯士三千伏於胡天|{
	胡天蓋石氏禁中署舍之名驤思將翻}
亦欲誅閔農鑒在中臺伏都帥三十餘人將升臺挾鑒以攻之鑒見伏都毁閣道臨問其故伏都曰李農等反已在東掖門臣欲帥衛士討之謹先啓知鑒曰卿是功臣好為官陳力|{
	魏晉以下率謂天子為官天子亦時自言之陳展也為于偽翻}
朕從臺上觀卿勿慮無報也|{
	言若能誅閔農將厚賞以報之}
於是伏都銖帥衆攻閔農不克屯於鳳陽門閔農帥衆數千毁金明門而入鑒懼閔之殺已馳招閔農開門内之謂曰孫伏都反卿宜速討之閔農攻斬伏都等自鳳陽至琨華横尸相枕|{
	枕職任翻}
流血成渠宣令内外六夷敢稱兵仗者斬|{
	稱舉也}
胡人或斬關或踰城而出者不可勝數|{
	閔旣誅孫伏都等又禁胡人稱兵仗胡人知禍之將及故去勝音升}
閔使尚書王簡少府王欝帥衆數千守鑒於御龍觀|{
	觀古玩翻}
懸食以給之下令城中曰近日孫劉搆逆支黨伏誅良善一無預也今日已後與官同心者留不同者各任所之敕城門不復相禁|{
	復扶又翻}
於是趙人百里内悉入城|{
	趙人謂中國人也}
胡羯去者填門閔知胡之不為己用班令内外趙人斬一胡首送鳳陽門者文官進位三等武官悉拜牙門一日之中斬首數萬閔親帥趙人以誅胡羯無貴賤男女少長皆斬之|{
	少詩照翻長知兩翻}
死者二十餘萬尸諸城外悉為野犬豺狼所食其屯戍四方者閔皆以書命趙人為將帥者誅之或高鼻多須濫死者半|{
	高鼻多鬚其狀似羯胡故亦見殺將卽亮翻帥所類翻}
燕王雋遣使至凉州|{
	使疏吏翻}
約張重華共擊趙 高句麗王釗送前東夷護軍宋晃于燕燕王雋赦之更名曰活|{
	更工衡翻下同}
拜為中尉|{
	晃奔高麗見九十六卷成帝咸康四年}


六年春正月趙大將軍閔欲滅去石氏之迹|{
	去羌呂翻}
託以䜟文有繼趙李更國號曰衛易姓李氏大赦改元青龍太宰趙庶太尉張舉中軍將軍張春光祿大夫石岳撫軍石寧|{
	撫軍之下當有將軍字}
武衛將軍張季及公侯卿校龍騰等萬餘人出奔襄國|{
	從石祗也校戶教翻}
汝陰王琨奔冀州|{
	趙之冀州治信都}
撫軍將軍張沈據滏口|{
	滏口滏水之口也唐代宗永泰元年薛嵩奏於滏口之右故臨水縣城置昭義縣以屬磁州沈持林翻滏音釜}
張賀度據石瀆|{
	魏收地形志鄴縣有石竇堰}
建義將軍段勤據黎陽|{
	建義將軍蓋亦後趙所置}
寧南將軍楊羣據桑壁|{
	後趙蓋於征鎮安平之外又置四寧括地志易州遂成縣界有桑丘城又水經注常山蒲吾縣東南有桑中縣故城俗謂之石勒城}
劉國據陽城|{
	續漢志中山蒲隂縣有陽城據後劉國自繁陽引兵會石琨擊冉閔則此陽城乃繁陽城也}
段龕據陳留姚弋仲據灄頭|{
	龕苦含翻灄書涉翻}
蒲洪據枋頭衆各數萬皆不附於閔勤末柸之子龕蘭之子也|{
	段末柸先據令支段蘭自宇文入趙}
王朗麻秋自長安赴洛陽秋承閔書誅朗部胡千餘人|{
	朗所部有胡兵千餘人閔命秋誅之}
朗奔襄國秋帥衆歸鄴|{
	帥讀曰率下同}
蒲洪使其子龍驤將軍雄迎擊獲之|{
	驤思將翻}
以為軍師將軍汝陰王琨及張舉王朗帥衆七萬伐鄴大將軍閔帥騎千餘與戰於城北閔操兩刃矛馳騎擊之|{
	兩刃矛者鋏之兩㫄皆利其刃騎奇寄翻操千高翻}
所向摧陷斬首三千級琨等大敗而去閔與李農帥騎三萬討張賀度于石瀆閏月 |{
	考異曰帝紀後云閏月三十國晉春秋皆云閏正月按長 閏二月帝紀閏月有丁丑己丑按是歲正月癸酉朔若閏正月即無丁丑己丑今以長歷為据}
衛主鑒密遣宦者齎書召張沈等使乘虛襲鄴宦者以告閔農閔農馳還廢鑒殺之|{
	載記曰鑒立一百三日}
并殺趙主虎二十八孫盡滅石氏|{
	載記曰始勒以成帝咸和三年僭立二主四子凡二十三年}
姚弋仲子曜武將軍益|{
	曜武曜威蓋皆石氏所置}
武衛將軍若帥禁兵數千斬關奔灄頭弋仲帥衆討閔軍于混橋司徒申鍾等上尊號於閔閔以讓李農農固辭閔曰吾屬故晉人也今晉室猶存請與諸君分割州郡各稱牧守公侯|{
	守式又翻}
奉表迎晉天子還都洛陽尚書胡睦進曰陛下聖德應天宜登大位晉氏衰微遠竄江表豈能總馭英雄混一四海乎閔曰胡尚書之言可謂識機知命矣乃即皇帝位|{
	冉閔字永曾少字棘奴虎之養孫也父瞻本姓冉名良魏郡内黄人勒破陳午獲瞻時年十二命虎子之}
大赦改元永興國號大魏 朝廷聞中原大亂復謀進取|{
	復扶又翻}
己丑以揚州刺史殷浩為中軍將軍假節都督揚豫徐兖青五州諸軍事|{
	為殷浩喪師張本}
以蒲洪為氐王使持節征北大將軍都督河北諸軍事冀州刺史廣川郡公蒲健為假節右將軍監河北征討前鋒諸軍事襄國公|{
	去年蒲洪遣使來降今經畧中原故授任以懷來之使疏吏翻監工銜翻}
姚弋仲蒲洪各有據關右之志弋仲遣其子襄帥衆五萬擊洪洪迎擊破之斬獲三萬餘級洪自稱大都督大將軍大單于三秦王改姓苻氏|{
	洪以䜟文草付應王又其孫堅背有艸付字遂改姓苻氏苻上從竹者非單音蟬}
以南安雷弱兒為輔國將軍安定梁楞為前將軍領左長史|{
	楞盧登翻}
馮翊魚遵為右將軍領右長史|{
	風俗通宋公子魚之後以王父字為氏}
京兆段陵為左將軍領左司馬天水趙俱隴西牛夷北地辛牢皆為從事中郎互酋毛貴為單于輔相|{
	互即氐字}
二月燕王雋使慕容霸將兵二萬自東道出徒河慕輿于自西道出蠮螉塞雋自中道出盧龍塞以伐趙|{
	杜佑曰盧龍塞在今平州城西北二百里}
以慕容恪鮮于亮為前驅命慕輿埿槎山通道|{
	槎仕下翻邪斫木曰槎}
留世子曄守龍城以内史劉斌為大司農|{
	斌音彬}
與典書令皇甫真留統後事霸軍至三陘|{
	樂安城在遼西遼陽縣東魏收地形志海陽縣有横山蓋即三陘之地陘音形}
趙征東將軍鄧恒惶怖焚倉庫弃安樂遁去|{
	安樂當作樂安果如慕容霸所料怖普布翻}
與幽州刺史王午共保薊|{
	薊音計}
徒河南部都尉孫泳急入安樂撲滅餘火籍其穀帛霸收安樂北平兵糧|{
	安樂並當作樂安}
與雋會臨渠|{
	臨渠城臨泃渠泃水出右北平無終縣西山東南至雍奴縣入鮑丘水魏武征蹋頓從泃口鑿渠逕雍奴縣州以通河海者也泃古侯翻}
三月燕兵至無終王午留其將王佗以數千人守薊|{
	佗徒河翻}
與鄧恒走保魯口|{
	魏收地形志博陵郡饒陽縣有魯口城博陵郡唐為定州}
乙巳雋拔薊執王佗斬之雋欲悉阬其士卒千餘人慕容霸諫曰趙為暴虐王興師伐之將以拯民於塗炭而撫有中州也今始得薊而阬其士卒恐不可以為王師之先聲也雋入都于薊中州士女降者相繼|{
	降戶江翻}
燕兵至范陽范陽太守李產欲為石氏拒燕|{
	為于偽翻}
衆莫為用乃帥八城令長出降|{
	范陽郡統涿良鄉方城長鄉道故安范陽容城八縣帥讀曰率}
雋復以產為太守產子績為幽州别駕弃其家從王午在魯口鄧恒謂午曰績鄉里在北|{
	績范陽人范陽在魯口之北}
父已降燕今雖在此恐終難相保徒為人累不如去之|{
	累力瑞翻去羌呂翻謂殺之也}
午曰此何言也夫以當今喪亂|{
	喪息浪翻}
而績乃能立義捐家情節之重雖古烈士無以過乃欲以猜嫌害之燕趙之士聞之謂我直相聚為賊了無意識衆情一散不可復集|{
	復扶又翻}
此為坐自屠潰也恒乃止午猶慮諸將不與已同心或致非意|{
	謂諸將殺之非午之意}
乃遣績歸績始辭午往見燕王雋雋讓之曰卿不識天命弃職邀名今日乃始來邪對曰臣眷戀舊主志存徵節官身所在何事非君|{
	績謂其身為官身言委質事君身非我有也}
殿下方以義取天下臣未謂得見之晩也雋悦善待之雋以弟宜為代郡城郎|{
	此秦漢以來之代郡非後魏之代都此代郡治代後魏代都乃秦漢之平城也城郎城大皆鮮卑所置付以城郭之任郎主也}
孫泳為廣甯太守悉置幽州郡縣守宰甲子雋使中部俟釐慕輿句督薊中留事|{
	俟釐蓋亦鮮卑部帥之稱俟渠之翻}
自將擊鄧恒於魯口軍至清梁|{
	魏收地形志高陽蠡吾縣有清凉城水經注中山蒲陰縣東南有清梁亭}
恒將鹿勃早將數千人夜襲燕營|{
	鹿姓也風俗通後漢有巴郡太守鹿旗}
半已得入先犯前鋒都督慕容霸突入幕下霸起奮擊手殺十餘人早不能進由是燕軍得嚴|{
	謂得以嚴備也}
雋謂慕輿根曰賊鋒甚銳宜且避之根正色曰我衆彼寡力不相敵故乘夜來戰冀萬一獲利今求賊得賊正當擊之復何所疑|{
	復扶又翻}
王但安臥臣等自為王破之|{
	為于偽翻}
雋不能自安内史李洪從雋出營外屯高冢上根帥左右精勇數百人從中牙直前擊早|{
	中牙雋所居也}
李洪徐整騎隊還助之|{
	騎奇寄翻}
早乃退走衆軍追擊四十餘里早僅以身免所從士卒死亡略盡雋引兵還薊|{
	雋之還薊亦鹿勃早有以挫其銳否則進攻魯口矣}
魏主閔復姓冉氏尊母王氏為皇太后立妻董氏為皇后子智為皇太子明裕皆為王|{
	明裕閔之三子}
以李農為太宰領太尉錄尚書事封齊王其子皆封縣公遣使者持節赦諸軍屯皆不從|{
	諸軍屯張沈及蒲洪等也}
麻秋說苻洪曰|{
	說輸芮翻}
冉閔石祗方相持中原之亂未可平也不如先取關中基業已固然後東爭天下誰敢敵之洪深然之旣而秋因宴鴆洪欲并其衆世子健收秋斬之洪謂健曰吾所以未入關者以為中州可定今不幸為豎子所困中州非汝兄弟所能辦我死汝急入關言終而卒健代統其衆乃去大都督大將軍三秦王之號|{
	去羌呂翻}
稱晉官爵遣其叔父安來告喪且請朝命|{
	朝直遥翻}
趙新興王祗即皇帝位于襄國 |{
	考異曰晉帝紀祗即位在閏月三十國晉春秋皆在三月按十六國春秋祗稱帝拜姚弋仲苻健官而不言苻洪洪三月死故疑祗以三月即位}
改元永寧以汝隂王琨為相國六夷據州郡者皆應之|{
	六夷胡羯氐羌段氏及已蠻也}
祗以姚弋仲為右丞相親趙王待以殊禮弋仲子襄雄勇多才略士民多愛之請弋仲以為嗣弋仲以襄非長子不許|{
	襄弋仲之第五子長知兩翻}
請者日以千數弋仲乃使之將兵|{
	將即亮翻}
祗以襄為驃騎將軍豫州刺史新昌公又以苻健為都督河南諸軍事鎮南大將軍開府儀同三司兖州牧略陽郡公 夏四月趙主祗遣汝陰王琨將兵十萬伐魏魏主閔殺李農及其三子并尚書令王謨侍中王衍中常侍嚴震趙昇閔遣使臨江告晉曰|{
	使疏吏翻}
逆胡亂中原今已誅之能共討者可遣軍來也朝廷不應 五月廬江太守袁真攻魏合肥克之虜其居民而還|{
	還從宣翻又如字}
六月趙汝陰王琨進據邯鄲|{
	邯鄲音寒丹}
鎮南將軍劉國

自繁陽會之|{
	繁陽縣漢屬魏郡晉屬頓丘郡隋廢繁陽入相州内黄縣}
魏衛將軍王泰擊琨大破之死者萬餘人劉國還繁陽 初段蘭卒於令支|{
	段蘭屯令支見上卷康帝建元元年令音鈴又郎定翻支音祁}
段龕代領其衆因石氏之亂擁部落南徙秋七月龕引兵東據廣固|{
	龕自陳留而東據廣固}
自稱齊王 八月代郡人趙榼帥三百餘家叛燕歸趙并州刺史張平|{
	榼苦合翻帥讀曰率}
燕王雋徙廣甯上谷二郡民於徐無|{
	徐無縣漢晉屬右北平郡後周廢入無終縣唐改無終為玉田縣屬薊州}
代郡民於凡城|{
	恐其復叛歸趙故徙之}
王朗之去長安也朗司馬杜洪據長安自稱晉征北將軍雍州刺史以馮翊張琚為司馬關西夷夏皆應之|{
	雍於用翻夏戶雅翻}
苻健欲取之恐洪知之乃受趙官爵|{
	趙主祗所授者也}
以趙俱為河内太守戍温牛夷為安集將軍戍懷|{
	安集將軍苻氏置以安集民夷為號温縣懷縣並屬河内郡温縣唐屬孟州懷縣故城在懷州武陟縣西}
治宮室於枋頭|{
	治直之翻}
課民種麥示無西意有知而不種者健殺之以狥旣而自稱晉征西大將軍都督關中諸軍事雍州刺史以武威賈玄碩為左長史略陽梁安為右長史段純為左司馬辛牢為右司馬京兆王魚安定程肱胡文等為軍諮祭酒悉衆而西以魚遵為前鋒行至盟津|{
	盟讀曰孟}
為浮梁以濟遣弟輔國將軍雄帥衆五千自潼關入兄子揚武將軍菁帥衆七千自軹關入|{
	從河南入潼關至華陰從河北入軹關自蒲津西渡河至渭北合兵以攻長安帥讀曰率下同}
臨别執菁手曰若事不捷汝死河北我死河南不復相見|{
	復扶又翻}
旣濟焚橋自帥大衆隨雄而進杜洪聞之與健書侮嫚之以張琚弟先為征虜將軍帥衆萬三千逆戰于潼關之北先兵大敗走還長安洪悉召關中之衆以拒健洪弟郁勸洪迎健洪不從郁帥所部降於健|{
	降戶江翻下同}
健遣苻雄狥渭北氐酋毛受屯高陵|{
	高陵縣漢屬馮翊晉改曰高陵屬京兆}
徐磋屯好畤|{
	磋倉何翻好畤縣前漢屬右扶風後漢晉省畤音止}
羌酋白犢屯黄白|{
	即黄白城酋慈由翻}
衆各數萬皆斬洪使|{
	使疏吏翻}
遣子降於健苻菁魚遵所過城邑無不降附洪懼固守長安 張賀度段勤劉國靳豚會于昌城|{
	魏收地形志魏郡昌樂縣有昌城昌樂縣後魏太和二十一年分魏縣置靳居焮翻}
將攻鄴魏主閔自將擊之戰于蒼亭|{
	蒼亭在河上西南至東阿六十里自將即亮翻}
賀度等大敗死者二萬八千人追斬靳豚於陰安|{
	陰安縣漢屬魏郡晉屬頓丘郡劉昫曰陰安城在澶州頓丘縣北}
盡俘其衆而歸閔戎卒三十餘萬旌旗鉦鼓綿亘百餘里雖石氏之盛無以過也故晉散騎常侍隴西辛謐有高名|{
	散悉亶翻騎奇寄翻}
歷劉石之世徵辟皆不就閔備禮徵為太常謐遺閔書|{
	遺于季翻}
以為物極則反致至則危|{
	戰國策曰物至而反冬夏是也致至則危累碁是也高誘注曰冬至生夏至殺故曰反致極也}
君王功已成矣宜因茲大捷歸身晉朝|{
	朝直遥翻}
必有由夷之廉享松喬之壽矣|{
	由夷許由伯夷也松喬赤松子王子喬也}
因不食而卒 九月燕王雋南狥冀州取章武河間|{
	晉武帝泰始元年分渤海置章武國五代志後魏以河間置瀛州統内有平舒縣舊置章武郡}
初勃海賈堅少尚氣節|{
	少詩照翻下同}
仕趙為殿中督趙亡堅弃魏主閔還鄉里擁部曲數千家燕慕容評狥勃海遣使招之堅終不降|{
	使疏吏翻降戶江翻}
評與戰擒之雋以評為章武太守封裕為河間太守雋與慕容恪皆愛賈堅之材堅時年六十餘恪聞其善射置牛百步上以試之堅曰少之時能令不中今老矣往往中之乃射再發|{
	少詩照翻中竹仲翻射而亦翻}
一矢拂脊一矢磨腹皆附膚落毛上下如一觀者咸服其妙雋以堅為樂陵太守治高城|{
	高城縣自漢以來屬勃海郡賢曰高城故城在今滄州鹽山縣南}
苻菁與張先戰于渭北擒之三輔郡縣堡壁皆降冬十月苻健長驅至長安杜洪張琚奔司竹|{
	扶風盩厔縣有司竹園宋白曰盩在鄠盩厔之間漢官有竹丞魏置司守之官後魏有司竹都尉}
燕王雋還薊留諸將守之雋還至龍城謁陵廟 十一月魏主閔帥步騎十萬攻襄國|{
	帥讀曰率騎奇寄翻下同}
署其子太原王胤為大單于驃騎大將軍以降胡一千配之為麾下|{
	單音蟬驃匹妙翻降戶江翻}
光祿大夫韋謏諫曰胡羯皆我之仇敵|{
	謏蘇島翻閔先誅胡羯故謏云然}
今來歸附苟存性命耳萬一為變悔之何及請誅屏降胡去單于之號|{
	屏必郢翻去羌呂翻}
以防微杜漸閔方欲撫納羣胡大怒誅謏及其子伯陽|{
	為下降胡執胤降趙張本}
甲午苻健入長安以民心思晉乃遣參軍杜山伯詣建康獻捷并修好於桓温於是秦雍夷夏皆附之|{
	夷夏皆附健以其歸晉也好呼到翻雍於用翻夏戶雅翻}
趙凉州刺史石寧獨據上邽不下十二月苻雄擊斬之 蔡謨除司徒三年不就職|{
	四年謨除司徒}
詔書屢下太后遣使諭意|{
	使疏吏翻}
謨終不受於是帝臨軒遣侍中紀據黄門郎丁纂徵謨謨陳疾篤使主簿謝攸陳讓自旦至申使者十餘返而謨不至時帝方八歲甚倦問左右曰所召人何以至今不來臨軒何時當竟太后以君臣俱疲乃詔必不來者宜罷朝|{
	朝直遥翻下同}
中軍將軍殷浩奏免吏部尚書江虨官|{
	虨逋閑翻}
會稽王昱令曹曰|{
	下令於尚書曹也昱時錄尚書六條事會工外翻}
蔡公傲違上命無人臣之禮若人主卑屈於上大義不行於下亦不知所以為政矣公卿乃奏謨悖慢傲上罪同不臣|{
	悖蒲内翻}
請送廷尉以正刑書謨懼帥子弟詣闕稽顙自到廷尉待罪|{
	帥讀曰率稽音啟}
殷浩欲加謨大辟|{
	辟毗亦翻}
會徐州刺史荀羨入朝|{
	羨自京口朝建康}
浩以問羨羨曰蔡公今日事危|{
	謂謨死也}
明日必有桓文之舉|{
	言將舉兵以問其罪}
浩乃止下詔免謨為庶人

資治通鑑卷九十八














































































































































