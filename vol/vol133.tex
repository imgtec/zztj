<!DOCTYPE html PUBLIC "-//W3C//DTD XHTML 1.0 Transitional//EN" "http://www.w3.org/TR/xhtml1/DTD/xhtml1-transitional.dtd">
<html xmlns="http://www.w3.org/1999/xhtml">
<head>
<meta http-equiv="Content-Type" content="text/html; charset=utf-8" />
<meta http-equiv="X-UA-Compatible" content="IE=Edge,chrome=1">
<title>資治通鑒_134-資治通鑑卷一百三十三_134-資治通鑑卷一百三十三</title>
<meta name="Keywords" content="資治通鑒_134-資治通鑑卷一百三十三_134-資治通鑑卷一百三十三">
<meta name="Description" content="資治通鑒_134-資治通鑑卷一百三十三_134-資治通鑑卷一百三十三">
<meta http-equiv="Cache-Control" content="no-transform" />
<meta http-equiv="Cache-Control" content="no-siteapp" />
<link href="/img/style.css" rel="stylesheet" type="text/css" />
<script src="/img/m.js?2020"></script> 
</head>
<body>
 <div class="ClassNavi">
<a  href="/24shi/">二十四史</a> | <a href="/SiKuQuanShu/">四库全书</a> | <a href="http://www.guoxuedashi.com/gjtsjc/"><font  color="#FF0000">古今图书集成</font></a> | <a href="/renwu/">历史人物</a> | <a href="/ShuoWenJieZi/"><font  color="#FF0000">说文解字</a></font> | <a href="/chengyu/">成语词典</a> | <a  target="_blank"  href="http://www.guoxuedashi.com/jgwhj/"><font  color="#FF0000">甲骨文合集</font></a> | <a href="/yzjwjc/"><font  color="#FF0000">殷周金文集成</font></a> | <a href="/xiangxingzi/"><font color="#0000FF">象形字典</font></a> | <a href="/13jing/"><font  color="#FF0000">十三经索引</font></a> | <a href="/zixing/"><font  color="#FF0000">字体转换器</font></a> | <a href="/zidian/xz/"><font color="#0000FF">篆书识别</font></a> | <a href="/jinfanyi/">近义反义词</a> | <a href="/duilian/">对联大全</a> | <a href="/jiapu/"><font  color="#0000FF">家谱族谱查询</font></a> | <a href="http://www.guoxuemi.com/hafo/" target="_blank" ><font color="#FF0000">哈佛古籍</font></a> 
</div>

 <!-- 头部导航开始 -->
<div class="w1180 head clearfix">
  <div class="head_logo l"><a title="国学大师官网" href="http://www.guoxuedashi.com" target="_blank"></a></div>
  <div class="head_sr l">
  <div id="head1">
  
  <a href="http://www.guoxuedashi.com/zidian/bujian/" target="_blank" ><img src="http://www.guoxuedashi.com/img/top1.gif" width="88" height="60" border="0" title="部件查字,支持20万汉字"></a>


<a href="http://www.guoxuedashi.com/help/yingpan.php" target="_blank"><img src="http://www.guoxuedashi.com/img/top230.gif" width="600" height="62" border="0" ></a>


  </div>
  <div id="head3"><a href="javascript:" onClick="javascript:window.external.AddFavorite(window.location.href,document.title);">添加收藏</a>
  <br><a href="/help/setie.php">搜索引擎</a>
  <br><a href="/help/zanzhu.php">赞助本站</a></div>
  <div id="head2">
 <a href="http://www.guoxuemi.com/" target="_blank"><img src="http://www.guoxuedashi.com/img/guoxuemi.gif" width="95" height="62" border="0" style="margin-left:2px;" title="国学迷"></a>
  

  </div>
</div>
  <div class="clear"></div>
  <div class="head_nav">
  <p><a href="/">首页</a> | <a href="/ShuKu/">国学书库</a> | <a href="/guji/">影印古籍</a> | <a href="/shici/">诗词宝典</a> | <a   href="/SiKuQuanShu/gxjx.php">精选</a> <b>|</b> <a href="/zidian/">汉语字典</a> | <a href="/hydcd/">汉语词典</a> | <a href="http://www.guoxuedashi.com/zidian/bujian/"><font  color="#CC0066">部件查字</font></a> | <a href="http://www.sfds.cn/"><font  color="#CC0066">书法大师</font></a> | <a href="/jgwhj/">甲骨文</a> <b>|</b> <a href="/b/4/"><font  color="#CC0066">解密</font></a> | <a href="/renwu/">历史人物</a> | <a href="/diangu/">历史典故</a> | <a href="/xingshi/">姓氏</a> | <a href="/minzu/">民族</a> <b>|</b> <a href="/mz/"><font  color="#CC0066">世界名著</font></a> | <a href="/download/">软件下载</a>
</p>
<p><a href="/b/"><font  color="#CC0066">历史</font></a> | <a href="http://skqs.guoxuedashi.com/" target="_blank">四库全书</a> |  <a href="http://www.guoxuedashi.com/search/" target="_blank"><font  color="#CC0066">全文检索</font></a> | <a href="http://www.guoxuedashi.com/shumu/">古籍书目</a> | <a   href="/24shi/">正史</a> <b>|</b> <a href="/chengyu/">成语词典</a> | <a href="/kangxi/" title="康熙字典">康熙字典</a> | <a href="/ShuoWenJieZi/">说文解字</a> | <a href="/zixing/yanbian/">字形演变</a> | <a href="/yzjwjc/">金 文</a> <b>|</b>  <a href="/shijian/nian-hao/">年号</a> | <a href="/diming/">历史地名</a> | <a href="/shijian/">历史事件</a> | <a href="/guanzhi/">官职</a> | <a href="/lishi/">知识</a> <b>|</b> <a href="/zhongyi/">中医中药</a> | <a href="http://www.guoxuedashi.com/forum/">留言反馈</a>
</p>
  </div>
</div>
<!-- 头部导航END --> 
<!-- 内容区开始 --> 
<div class="w1180 clearfix">
  <div class="info l">
   
<div class="clearfix" style="background:#f5faff;">
<script src='http://www.guoxuedashi.com/img/headersou.js'></script>

</div>
  <div class="info_tree"><a href="http://www.guoxuedashi.com">首页</a> > <a href="/SiKuQuanShu/fanti/">四库全书</a>
 > <h1>资治通鉴</h1> <!--         下载:【右键另存为】即可 --></div>
  <div class="info_content zj clearfix">
  
<div class="info_txt clearfix" id="show">
<center style="font-size:24px;">134-資治通鑑卷一百三十三</center>
    資治通鑑卷一百三十三 宋 司馬光 撰<br />
<br />
  胡三省 音註<br />
<br />
  宋紀十五【起重光大淵獻盡旃蒙單閼凡五年】<br />
<br />
  太宗明皇帝下<br />
<br />
  泰始七年春二月戊戌分交廣置越州治臨漳【劉昫曰廉州治合浦縣秦象郡地吳改為珠官郡宋分置臨漳郡及越州領郡三治於此又據沈約志越州領百梁龍蘇永寧永昌富昌南流臨漳合浦宋夀九郡蕭子顯曰臨漳郡本合浦郡之北界也按沈約宋志作臨障宋白續通典作臨瘴以臨界内瘴江為名瘴江一名合浦江】 初上為諸王寛和有令譽獨為世祖所親即位之初義嘉之黨多蒙全宥【晉安王子勛改元義嘉】隨才引用有如舊臣及晩年更猜忌忍虐好鬼神【好呼到翻】多忌諱言語文書有禍敗凶喪及疑似之言應囘避者數百千品有犯必加罪戮改騧字為【騧古花翻】以其似禍字故也左右忤意往往有刳斮者【忤五故翻斮側畧翻斬也】時淮泗用兵府藏空竭内外百官並斷俸祿【藏徂浪翻斷丁管翻 考異曰宋本紀云曰給料祿俸今從南史自失青徐之後宋魏交兵於淮泗之間】而奢費過度每所造器用必為正御副御次副各三十枚嬖倖用事貨賂公行【嬖卑義翻又傳計翻後同】上素無子密取諸王姬有孕者内宫中生男則殺其母【孕以證翻 考異曰宋書云閉其母於幽房今從宋畧】使寵姬子之至是寢疾以太子幼弱深忌諸弟南徐州刺史晉平刺王休祐前鎭江陵貪虐無度【休祐鎭江陵事始上百三十一卷二年刺來葛翻】上不使之鎭留之建康遣上佐行府州事【上佐謂長史司馬也】休祐性剛狠前後忤上非一【狠戶墾翻忤五故翻】上積不能平且慮將來難制欲方便除之【施方畧乘便利而殺之也】甲寅休祐從上於巖山射雉【據休祐傳巖山在建康城南又據宋紀巖山在秣陵縣界世祖景寧陵在焉射而亦翻】左右從者並在仗後【從才用翻】日欲闇上遣左右壽寂之等數人逼休祐令墜馬因共敺拉殺之傳呼驃騎落馬【敺烏口翻拉盧合翻驃匹妙翻騎奇寄翻】上陽驚遣御醫絡驛就視【絡驛猶絡繹也】比其左右至休祐已絶【比必寐翻及也絶氣絶也】去車輪輿還第【去羌呂翻下悉去同】追贈司空葬之如禮建康民間訛言荆州刺史巴陵王休若有至貴之相上以此言報之休若憂懼戊午以休若代休祐為南徐州刺史休若腹心將佐皆謂休若還朝必不免禍中兵參軍京兆王敬先說休若曰【相息亮翻將即亮翻朝直遥翻下同訖輸芮翻】今主上彌留【書顧命曰疾大漸惟幾病日臻既彌留呂祖謙曰疾大進而頻於死病日加則愈留夏撰曰重疾謂之病言重病日至而又久留於體曾不减去將必死也】政成省閤羣豎恟恟欲悉去宗支以便其私【恟許拱翻去起呂翻】殿下聲著海内受詔入朝必往而不返荆州帶甲十餘萬地方數千里上可以匡天子除姦臣下可以保境土全一身孰與賜劒邸第使臣妾飲泣而不敢葬乎【呂向曰泣淚也淚入口曰飲】休若素謹畏偽許之敬先出使人執之以白於上而誅之 三月辛酉魏假員外散騎常侍邢祐來聘【散悉亶翻騎奇寄翻】 魏主使殿中尚書胡莫寒簡西部敕勒為殿中武士【魏書官氏志拓拔鄰以兄為紇骨氏後改為胡氏自魏世祖破柔然高車敕勒皆來降其部落附塞下而居自武周塞外以西謂之西部以東謂之東部依漠南而居者謂之北部】莫寒大納貨賂衆怒殺莫寒及高平假鎭將奚陵【假鎭將者未得為眞將即亮翻下同】夏四月諸部敕勒皆叛魏主使汝隂王天賜將兵討之以給事中羅雲為前鋒敕勒詐降襲雲殺之【降戶江翻】天賜僅以身免 晉平刺王旣死【刺來逹翻】建安王休仁益不自安上與嬖臣楊運長等為身後之計運長等亦慮上晏駕後休仁秉政已輩不得專權彌贊成之上疾嘗暴甚内外莫不屬意於休仁【屬之欲翻】主書以下皆往東府訪休仁所親信豫自結納其或在直不得出者皆恐懼上聞愈惡之【惡烏路翻】五月戊午召休仁入見【見賢遍翻】既而謂曰今夕停尚書下省宿明可早來其夜遣人齎藥賜死休仁罵曰上得天下誰之力邪【事見一百三十一卷元年二年】孝武以誅鉏兄弟子孫㓕絶【孝武誅鉏兄弟謂殺南平王鑠竟陵王誕海陵王休茂也子孫㓕絶於泰始之世事並見前】今復為爾【復扶又翻下已復同】宋祚其能久乎上慮有變力疾乘輿出端門休仁死乃入下詔稱休仁規結禁兵謀為亂逆朕未忍明法申詔詰厲【詰起吉翻】休仁慙恩懼罪遽自引决可宥其二子降為始安縣王聽其子伯融襲封上慮人情不悦乃與諸大臣及方鎭詔稱休仁與休祐深相親結語休祐云汝但作佞此法自足安身我從來頗得此力休祐之隕木欲為民除患而休仁從此日生嬈懼【語牛倨翻為于偽翻嬈集韻爾紹翻擾也】吾每呼令入省便入辤楊太妃【楊太妃休仁所生母也】吾春中多與之射雉【射而亦翻】或隂雨不出休仁輒語左右云我已復得今一日休仁既經南討【謂南拒尋陽之兵故也】與宿衛將帥經習狎共事【將即亮翻帥所類翻】吾前者積日失適【失適謂體中不安和也】休仁出入殿省無不和顔厚相撫勞如其意趣人莫能測事不獲已反覆思惟不得不有近日處分恐當不必即解【勞力到翻處昌呂翻分扶問翻解戶買翻曉也】故相報知上與休仁素厚雖殺之每謂人曰我與建安年時相鄰【謂年齒不相遠也】少便欵狎景和泰始之間勲誠實重事計交切不得不相除痛念之至不能自已因流涕不自勝【少詩照翻勝音升史言帝殘害骨肉不能自揜其天性之傷】初上在藩與禇淵以風素相善【風素相善者以其風標雅素而與之善也蕭子顯齊書風作夙】及即位深相委仗上寢疾淵為吳郡太守【蕭子顯齊書禇淵傳云為吳興太守按吳郡近畿大郡也吳興次郡也淵以大尚書出守當得大郡吳郡為是】急召之既至入見【見賢遍翻】上流涕曰吾近危篤【近其蘄翻】故召卿欲使著黃耳黄者乳母服也【著則畧翻力賀翻女人上衣也言託孤於淵】上與淵謀誅建安王休仁淵以為不可上怒曰卿癡人不足與計事淵懼而從命復以淵為吏部尚書【泰始之初淵為吏部尚書今去郡還朝復為之復扶又翻】庚午以尚書右僕射袁粲為尚書令禇淵為左僕射 上惡太子屯騎校尉壽寂之勇健會有司奏寂之擅殺邏尉徙越州【惡烏路翻邏郎左翻徙合浦也】於道殺之 丙戌追廢晉平王休祐為庶人 巴陵王休若至京口聞建安王死益懼上以休若和厚能諧輯物情恐將來傾奪幼主欲追使殺之慮不奉詔欲徵入朝又恐猜駭【使疏吏翻朝直遇翻】六月丁酉以江州刺史桂陽王休範為南徐州刺史以休若為江州刺史手書殷勤召休若使赴七月七日宴 丁未魏主如河西 秋七月巴陵哀王休若至建康乙丑賜死於第贈侍中司空復以桂陽王休範為江州刺史【復扶又翻】時上諸弟俱盡唯休範以人才凡劣不為上所忌故得全【為後休範稱兵張本】<br />
<br />
  沈約論曰聖人立法垂制所以必稱先王盖由遺訓餘風足以貽之來世也太祖經國之義雖弘隆家之道不足彭城王照不窺古徒見昆弟之義未識君臣之禮冀以家情行之國道主猜而猶犯恩薄而未悟致以呵訓之微行遂成㓕親之大禍【謂文帝殺彭城王義康也沈約言義康之罪文帝當呼而訓之不當遂殺之也行下孟翻】開端樹隙垂之後人太宗因易隙之情據已行之典翦落洪枝【謂據文帝已行之典而翦除兄弟也洪大也枝兄弟也嫡統為本枝庶為枝易以豉翻】不待顧慮既而本根無庇幼主孤立神器以勢弱傾移靈命隨樂推囘改【樂音洛】斯盖履霜有漸堅氷自至所由來遠矣<br />
<br />
  裴子野論曰夫噬虎之獸知愛己子摶狸之鳥非護異巢太宗保字螟蛉剿拉同氣既迷在原之天屬未識父子之自然【詩曰螟蛉之子蜾羸負之教誨爾子式穀似之故世俗謂抱養者為螟蛉又曰眷令在原兄弟急難剿子小翻絶也拉盧合翻】宋德告終非天廢也夫危亡之君未嘗不先弃本枝嫗煦旁孽【鄭康成曰體曰嫗氣曰煦陸德明曰嫗於具翻徐於甫翻煦許具翻徐况甫翻孽魚列翻說文庶子為孽旁孽旁枝之庶子也】推誠嬖狎【嬖卑義翻又博計翻】疾惡父兄【惡烏路翻】前乘覆車後來并轡借使仲叔有國猶不失配天而他人入室將七廟絶祀曾是莫懷甘心揃落【揃子踐翻】晉武背文明之託而覆中州者賈后【事見晉武帝紀背蒲妹翻】太祖棄初寧之誓而登合殿者元凶【事見文帝紀】禍福無門奚其豫擇友于兄弟不亦安乎<br />
<br />
  丙寅魏主至隂山 初吳喜之討會稽也言於上曰得尋陽王子房及諸賊帥皆即於東戮之【會工外翻帥所類翻】既而生送子房釋顧琛等【事見一百三十一卷之二年】上以其新立大功不問而心銜之及克荆州剽掠贓以萬計【尋陽既平建安王休仁遣喜進克荆州剽匹妙翻】壽寂之死喜為淮陵太守督豫州諸軍事【淮陵漢縣屬臨淮郡後屬下邳國晉復屬臨淮惠帝永寧元年以為淮臨國宋為郡屬南徐州宋白曰泗州招信縣本漢淮陵縣】聞之内懼啓乞中散大夫【漢大夫掌論議中散大夫後漢志始有之魏晉以來以為冗散散悉亶翻】上尤疑駭或譖蕭道成在淮隂有貳心於魏 【考異曰南齊書太祖紀云帝常嫌太祖非人臣相而民間流言蕭諱當為天子帝愈以為疑今從宋畧】上封銀壺酒使喜自持賜道成道成懼欲逃喜以情告道成且先為之飲道成即飲之【為于偽翻 考異曰南齊妃云太祖戎服出門迎即酌飲之喜還帝意乃解宋畧云道成懼弗肯飲將出奔喜語以情先為之酌於是喜得罪而道成被徵盖南齊書欲成太祖之美故云爾今從宋畧】喜還朝保證道成【朝直遥翻】或密以啓上上以喜多計數素得人情恐其不能事幼主乃召喜入内殿與共言謔甚欵既出賜以名饌【謔迄却翻饌雛戀翻又雛皖翻】尋賜死然猶詔賻賜【賻音附】又與劉勔等詔曰吳喜輕狡萬端苟取物情昔大明中黟歙有亡命數千人攻縣邑殺官長劉子尚遣三千精甲討之再往失利【孝武第二子曰豫章王子尚與景和同母也黟音伊歙音攝長知兩翻】孝武以喜將數十人至縣說誘羣賊【將即亮翻說輸芮翻誘音酉】賊即歸降詭數幻惑【詭過委翻幻戶辦翻】乃能如此及泰始初東討止有三百人直造三吳【造七到翻】凡再經薄戰【薄伯各翻】而自破岡以東至海十郡無不清蕩【十郡謂晉陵義興吳郡吳興南東海會稽東陽臨海永嘉新安等郡也】百姓聞吳河東來便望風自退若非積取三吳人情何以得弭伏如此【弭緜婢翻】尋喜心迹豈可奉守文之主遭國家可乘之會邪譬如餌藥當人羸冷資散石以全身【散如寒食散之類石謂丹石也羸倫為翻散悉但翻】及熱勢發動去堅積以止患非忘其功勢不獲已耳【用人如此人不自保其肯終為之用乎】 戊寅以淮隂為北兖州【淮隂為南兖州事見上卷上年】徵蕭道成入朝【朝直遥翻下同】道成所親以朝廷方誅大臣勸勿就徵道成曰諸卿殊不見事主上自以太子稚弱【稚持利翻】翦除諸弟何預它人今唯應速發淹留顧望必將見疑且骨肉相殘自非靈長之祚禍難將興方與卿等戮力耳【史言骨肉相殘則姦雄生心因之而起為蕭氏取宋張本難乃旦翻】既至拜散騎常侍太子左衛率【散悉但翻騎奇寄翻率所律翻】 八月丁亥魏主還平城 戊子以皇子躋繼江夏文獻王義恭【景和之初義恭父子皆死事見一百三十一卷夏戶雅翻】 庚寅上疾有間【間讀如字】大赦戊戌立皇子凖為安成王實桂陽王休範之子也 魏顯祖聰睿夙成剛毅有斷而好黃老浮屠之學每引朝士及沙門共談玄理雅薄富貴常有遺世之心以叔父中都大官京兆王子推沈雅仁厚素有時譽欲禪以帝位【中都大官即闕   穆帝之子斷丁亂翻沈持林翻】時太尉源賀督諸軍屯漠南馳傳召之【傳株戀翻】既至會公卿大議皆莫敢先言任城王雲子推之弟也【任音壬】對曰陛下方隆太平臨覆四海【覆敷又翻】豈得上違宗廟下棄兆民且父子相傳其來久矣陛下必欲委棄塵務則皇太子宜承正統夫天下者祖宗之天下陛下若更授旁支恐非先聖之意啓姦亂之心斯乃禍福之原不可不愼也源賀曰陛下今欲禪位皇叔臣恐紊亂昭穆後世必有逆祀之譏【春秋魯莊公薨子般弑季友立閔公閔公復弑立僖公閔公弟也僖公兄也及僖公薨魯人以先大後小為順遂躋僖公于閔公之上仲尼以臧文仲不知者三縱逆祀其一也言宗廟之祀姪為昭而叔為穆亂也後世必以逆祀貽譏更工衡翻紊亡運翻昭時招翻】願深思任城之言東陽公丕等曰皇太子雖聖德早彰然實冲幼陛下富於春秋始覽萬機柰何欲隆獨善不以天下為心其若宗廟何其若億兆何尚書陸馛曰陛下若捨太子更議諸王臣請刎頸殿庭不敢奉詔【馛蒲撥翻刎扶粉翻】帝怒變色以問宦者選部尚書酒泉趙黑黑曰臣以死奉戴皇太子不知其它帝默然【陸馛之言則怒而變色趙黑之言則默然心服者以衆屬於正嫡也但朝廷大議作色於陸馛而默爾於宦官臣庶何觀魏之朝綱可想而見矣選須絹翻】時太子宏生五年矣帝以其幼故欲傳位子推中書令高允曰臣不敢多言願陛下上思宗廟託付之重追念周公抱成王之事帝乃曰然則立太子羣公輔之有何不可【高允之言婉而當且於衆言交進之後故轉移上意為力差易】又曰陸馛直臣也必能保吾子乃以馛為太保與源賀持節奉皇帝璽紱傳位於太子【璽斯氏翻考異曰後魏天象志云上廹於太后傳位太子按馮太后若迫顯祖傳位當奪其大政安得猶縂萬機今從帝紀】丙午高祖即皇帝位【諱宏顯祖獻文皇帝之長子也】大赦改元延興高祖幼有至性前年顯祖病癕高祖親吮【吮徂兖翻】及受禪悲泣不自勝【勝音升】顯祖問其故對曰代親之感内切於心丁未顯祖下詔曰朕希心玄古志存澹泊爰命儲宫踐升大位【澹徒覽翻踐慈衍翻】朕得優游恭已栖心浩然羣臣奏曰昔漢高祖稱皇帝尊其父為太上皇明不統天下也【見十一卷漢高帝六年】今皇帝幼冲萬機大政猶宜陛下總之謹上尊號曰太上皇帝【太上皇帝之號始此上時掌翻】顯祖從之巳酉上皇徙居崇光宫采椽不斷【徐廣曰采一名櫟一作柞索隱曰采木名即今之櫟木也余謂采椽者盖自山采來之椽因而用之不施斧斤示樸也椽重緣翻】土階而已國之大事咸以聞崇光宫在北苑中又建鹿野浮圖於苑中之西山【釋子相傳以為戶迦國波羅奈城東北十里許有鹿野苑本辟支佛住此常有野鹿故以名苑今倣西國而建浮圖也又據魏書道武帝天興二年破高車以其衆起鹿苑於南臺隂北距長城東苞白登屬之西山廣輪數百里盖因代都鹿苑之舊名附合西國鹿野之事而建此浮圖也】與禪僧居之【禪時連翻】 冬十月魏沃野統萬二鎭勑勒叛【沃野即漢朔方郡沃野縣也統萬即赫連故都魏以為鎮置鎮將陸恭之風土記朔方故城後魏改為沃野鎮去統萬八百餘里】遣太尉源賀帥衆討之降二千餘落【帥讀曰率降戶江翻】追擊餘黨至枹罕金城大破之【抱音膚】斬首八千餘級虜男女萬餘口雜畜三萬餘頭詔賀都督三道諸軍屯于漠南先是魏每歲秋冬軍三道並出以備柔然春中乃還【還從宣翻又如字下同】賀以為往來疲勞不可支久請募諸州鎭武健者三萬餘人築三城以處之【先悉薦翻處昌呂翻】使冬則講武春則耕種【此即古屯田之說也】不從 庚寅魏以南安王楨為都督凉州及西戎諸軍事領護西域校尉鎭凉州【校戶教翻】 上命北琅邪蘭陵二郡太守垣崇祖經畧淮北【守式又翻】崇祖自郁洲將數百人入魏境七百里據蒙山【此指言舊琅邪蘭陵郡也本屬徐州彭城既沒崇祖率部曲據郁洲使領二郡太守未能有其地也魏收志蒙山在東安郡新泰縣東南水經注朐山縣東北海中有大洲謂之郁洲山海經所謂郁山在海中者是也杜佑曰鬱洲在海州東海縣亦曰郁洲】十一月魏東兖州刺史于洛侯擊之崇祖引還 上以故第為湘宫寺【始封湘東王故以故第為湘宫寺】備極壯麗欲造十級浮圖而不能乃分為二新安大守巢尚之罷郡入見【見賢遍翻】上謂曰卿至湘宫寺未此是我大功德用錢不少【少詩沼翻】通直散騎侍郎會稽虞愿侍側【散騎侍郎曹魏初與散騎常侍同置至晉武帝置員外散騎侍郎及元帝太興元年使員外二人與散騎侍郎同員直故謂之通直散騎侍郎後增為四人散悉亶翻騎奇寄翻會工外翻】曰此皆百姓賣兒貼婦錢所為【貼婦謂夫先有婦苦於上之征求而不能贍縱之外求淫夫貼以贍之又貼亦賣也通典北齊武平以後聽人貼賣園田】佛若有知當慈悲嗟愍罪高浮圖何功德之有侍坐者失色【坐徂卧翻】上怒使人驅下殿愿徐去無異容上好圍棊【好呼到翻下同】棊甚拙與第一品彭城丞王抗圍棊【當時圍棊之品王抗為第一】抗每假借之曰皇帝飛棊臣抗不能斷【圍棊之勢聯屬不斷然後可以勝人若為人斷之則為所勝斷如字】上終不悟好之愈篤愿又曰堯以此教丹朱【博物志堯造圍棊以教子丹朱或云舜以子商均愚故作圍棊以教之其法非智者不能也胡旦曰以棊為易則聰明者而或不能以為難則愚下小人往往精絶】非人主所宜好也上雖怒甚以愿王國舊臣【上為湘東王愿為國常侍】每優容之 王景文常以盛滿為憂屢辭位任上不許然中心以景文外戚貴盛張永累經軍旅疑其將來難信乃自為謡言曰一士不可親弓長射殺人【射而亦翻】景文彌懼自表解揚州情甚切至詔報曰人居貴要但問心若為耳【言但問其存心何如耳】大明之世巢徐二戴位不過執戟權亢人主【巢謂巢尚之徐謂徐爰二戴謂法興明寶亢口浪翻高也】今袁粲作僕射領選【選須絹翻】而人往往不知有粲粲遷為令居之不疑人情向粲淡然亦復不改常日以此居貴位要任當有致憂競不【袁粲之簡淡雅素目足以鎮雅俗而明帝謂其可以託孤則眞違才易務矣然粲才雖不足以死繼之無愧於爲臣之大節其視禇淵相去豈不遠哉復扶又翻競當作兢不讀曰否】夫貴高有危殆之懼卑賤有塡壑之憂有心於避禍不如無心於任運存亡之要巨細一揆耳<br />
<br />
  泰豫元年春正月甲寅朔上以疾久不平改元戊午皇太子會四方朝賀者於東宮并受貢計【朝直遥翻】 大陽蠻酋桓誕擁沔水以北滍葉以南八萬餘落降於魏【此即五水蠻也宋置大陽成於蘄陽縣西此縣即漢江夏郡蘄春縣也沔水以北滍葉以南皆羣蠻所居誕擁以降魏而誕實大陽蠻酋也酋慈由翻沔彌兖翻滍直里翻葉式涉翻降戶江翻】自云桓玄之子亡匿蠻中以智略為羣蠻所宗魏以誕為征南將軍東荆州刺史襄陽王【東荆州治北陽縣】聽自選郡縣吏使起部郎京兆韋珍與誕安集新民區置諸事皆得其所【據晉志武帝置起部郎杜佑通典曰晉宋有起部而不常置起部工部也取虞書百工起哉為義自是之後諸蠻皆倚魏以侵擾南國】 二月柔然侵魏上皇遣將擊之【將即亮翻下同】柔然走東部敕勒叛奔柔然上皇自將追之至石磧【磧七迹翻石磧即石漠】不及而還 上疾篤慮晏駕之後皇后臨朝江安懿侯王景文以元舅之勢必為宰相【景文皇后兄也朝直遥翻相息亮翻】門族彊盛或有異圖己未遣使齎藥賜景文死【使疏吏翻】手敕曰與卿周旋欲全卿門戶故有此處分【處昌呂翻分扶問翻】敕至景文正與客棊叩函看己【巳畢也】復置局下神色不變方與客思行爭劫【棊有行有劫行者欲擊東而聲出於西也劫者先有彼我兩急之勢彼欲出此則我劫彼以制之也行下孟翻】局竟歛子内奩畢【子棊子也奕戲既畢則歛而納諸奩中奩力鹽翻】徐曰奉敕見賜以死方以敕示客【句斷】中直兵焦度趙智畧憤怒【中直兵典親兵將官也洪适曰宋有中直兵外兵騎兵參軍】曰大丈夫安能坐受死州中文武數百足以一奮【王景文時為楊州刺史】景文曰知卿至心若見念者為我百口計乃作墨啓荅敕致謝欽藥而卒 【考異曰南史云帝使謂景文曰朕不謂卿有罪然吾不能獨死請子先之若使者有此語則坐客不容不知更終棊局又曰景文酌酒謂客曰此酒不可相勸自仰而欽之按焦度勸拒命必不對坐客言之何得死時客猶在坐也今從宋書】贈開府儀同三司上夢有人告曰豫章太守劉愔反既寤遣人就郡殺之【愔於含翻】 魏顯祖還平城【書顯祖以别魏主】 庚午魏主耕籍田 夏四月以垣崇祖行徐州事徙戍龍沮【魏收地形志東彭城郡有龍沮縣縣有即丘城五代志琅邪郡治臨沂縣舊曰即丘沮子余翻】 己亥上大漸【書顧命云疾大漸呂祖謙注曰疾大進而瀕於死也】以江州刺史桂陽王休範為司空又以尚書右僕射禇淵為護軍將軍加中領軍劉勔右僕射【勔彌兖翻】詔淵勔與尚書令袁粲荆州刺史蔡興宗郢州刺史沈攸之並受顧命禇淵素與蕭道成善引薦於上詔又以道成為右衛將軍領衛尉【左右衛晉官衛尉漢官也史言禁衛兵柄皆歸道成】與袁粲等共掌機事是夕上殂【年三十四】庚子太子即皇帝位大赦時蒼梧王方十歲袁粲禇淵秉政承太宗奢侈之後務弘節儉欲救其弊而阮佃夫王道隆等用事貨賂公行不能禁也【佃音田】 乙巳以安成王凖為揚州刺史五月戊寅葬明皇帝于高寧陵【據南史陵在臨沂縣莫府山】廟號太宗六月乙巳尊皇后曰皇太后 【考異曰宋畧本紀作癸木今從宋本紀】立妃江氏為皇后 秋七月柔然部帥無盧眞將三萬騎寇魏敦煌鎭將尉多侯擊走之多侯眷之子也【尉眷事魏太武有平赫連之功帥所類翻將即亮翻下同騎奇寄翻敦徒門翻尉紆勿翻】又寇晉昌守將薛奴擊走之【魏書官氏志西方諸姓叱干氏改為薛氏】 戊午魏主如隂山戊辰尊帝母陳貴妃為皇太妃更以諸國太妃為太姬【更工衡翻姬音怡】 右軍將軍王道隆以蔡興宗彊直不欲使居上流閠月甲辰以興宗為中書監更以沈攸之為都督荆襄等八州諸軍事荆州刺史興宗辭中書監不拜王道隆每詣興宗躡履到前不敢就席良久去竟不呼坐沈攸之自以材畧過人自至夏口以來隂蓄異志【夏口郢州也攸之鎭郢州見上卷五年夏戶雅翻】及徙荆州擇郢州士馬器仗精者多以自隨到官以討蠻為名大兵力招聚才勇部勒嚴整常如敵至重賦歛以繕器甲【斂力贍翻】舊應供臺者皆割留之養馬至二千餘匹治戰艦近千艘【治直之翻艦戶黯翻近其靳翻艘蘇遭翻】倉廪府庫莫不充積士子商旅過荆州者多為所羈留四方亡命歸之者皆蔽匿擁護所部或有逃亡無遠近窮追必得而止舉錯專恣不復承用符敕【錯千故翻復扶又翻】朝廷疑而憚之【為後攸之稱兵張本臺省所下者為符出命經中書門下者為敕】為政刻暴或鞭撻士大夫上佐以下面加詈辱【詈力智翻】然吏事精明人不敢欺境内盗賊屏息【屏必郢翻】夜戶不閉攸之賧罰羣蠻太甚【何承天纂文曰賧蠻夷贖罪貨也音徒濫翻】又禁五溪魚鹽蠻怨叛酉溪蠻王田頭擬死【水經酉水導源巴郡臨江縣東逕遷陵縣故城北又東逕酉陽故縣南又東逕沅陵縣北又南注沅水】弟婁侯簒立其子田都走入獠中【獠魯皓翻】於是羣蠻大亂掠抄至武陵城下【抄楚交翻】武陵内史蕭嶷遣隊主張英兒擊破之誅婁侯立田都羣蠻乃定嶷賾之弟也【史言蕭道成二子皆有幹時之用嶷魚力翻賾士革翻】八月戊午樂安宣穆公蔡興宗卒九月辛巳魏主還平城冬十月柔然侵魏及五原十一月上皇自將討之【將即亮翻】將度漠柔然北走數千里上皇乃還【還從宣翻又如字】 丁亥魏封上皇之弟略為廣川王 己亥以郢州刺史劉秉為尚書左僕射秉道憐之孫也和弱無幹能以宗室清令【令善也】故袁禇引之【史言袁禇尚虚名而無實用所以受制於姦雄也】 中書通事舍人阮佃夫【晉初初置中書舍人通事各一人江左令舍人通事謂之中書通事舍人佃音田】加給事中輔國將軍權任轉重欲用其所親吳郡張澹為武陵郡袁粲等皆不同佃夫稱敕施行粲等不敢執【史言袁粲等撓於權倖不能裁之以正澹徒覽翻】 魏有司奏諸祠祀合一千七十五所歲用牲七萬五千五百上皇惡其多殺【惡烏路翻】詔自今非天地宗廟社稷皆勿用牲薦以酒脯而已<br />
<br />
  蒼梧王上【諱昱字德融明帝長子也小字慧震】<br />
<br />
  元徽元年春正月戊寅朔改元大赦 庚辰魏員外散騎常侍崔演來聘【散悉亶翻騎奇寄翻】 戊戌魏上皇還至雲中【還自討柔然】 癸丑魏詔守令勸課農事同部之内貧富相通家有兼牛通借無者若不從詔一門終身不仕 戊午魏上皇至平城【自雲中至平城】 甲戌魏詔縣令能靜一縣刼盗者兼治二縣即食其祿能靜二縣者兼治三縣【治直之翻下同】三年遷為郡守【守式又翻】二千石能靜二郡上至三郡亦如之三年遷為刺史 桂陽王休範素凡訥少知解【凡庸常也訥言難也少詩沼翻解戶買翻解曉也】不為諸兄所齒遇【齒列也遇待也不以諸弟之列待之也】物情亦不向之故太宗之末得免於禍及帝即位年在冲幼素族秉政近習用權【素族謂袁禇也近習謂阮佃夫王道隆楊運長也】休範自謂尊親莫二【帝諸父皆誅死唯休範在故謂尊親莫二】應入為宰輔既不如志怨憤頗甚典籖新蔡許公輿為之謀主令休範折節下士【折而設翻下戶嫁翻】厚相資給於是遠近赴之歲中萬計收養勇士繕治器械朝廷知其有異志亦隂為之備會夏口闕鎭【夏戶雅翻】朝廷以其地居尋陽上流欲使腹心居之二月乙亥以晉熙王爕為郢州刺史爕始四歲以黃門郎王奐為長史行府州事【黃門郎即黃門侍郎】配以資力使鎭夏口【夏戶雅翻】復恐其過尋陽為休範所劫留【復扶又翻】使自太洑徑去【南史休範傳作大子洑此盖即劉胡自江外趣沔口之路】休範聞之大怒密與許公輿謀襲建康表治城隍多解材板而蓄之奐景文之兄子也 吐谷渾王拾寅寇魏澆河夏四月戊申魏以司空長孫觀為大都督兵討之【吐從暾入聲谷音浴澆堅堯翻長如兩翻】 魏以孔子二十八世孫乘為崇聖大夫【崇聖大夫以尊崇先聖名官】給十戶以供洒掃【洒所賣翻掃素報翻又並上聲】秋七月魏詔河南六州之民【河南六州青徐兖豫齊東徐也】戶收絹<br />
<br />
  一匹綿一斤租三十石 乙亥魏主如隂山 八月庚申魏上皇如河西長孫觀入吐谷渾境芻其秋稼吐谷渾王拾寅窘急請降【窘巨隕翻降戶江翻】遣子斤入侍自是歲修職貢九月辛巳上皇還平城 遣使如魏【報聘也使疏吏翻】冬十月癸酉割南豫兖州之境置徐州治鍾離【鍾離禹塗山氏之國春秋鍾離子之國漢為縣屬九江郡晉屬淮南郡晉安帝分立鍾離郡宋立徐州於此宋白曰今鍾離縣東四十里有鍾離城】 魏上皇將入寇詔州郡之民十丁取一以充行【以充征行也】戶收租五十石以備軍糧 魏武都氐反攻仇池詔長孫觀回師討之 武都王楊僧嗣卒於葭蘆從弟文度自立為武興王【從才用翻】遣使降魏魏以文度為武興鎭將【將即亮翻下同】 十一月丁丑尚書令袁粲以母憂去職 癸巳魏上皇南廵至懷州【魏天安二年以河内郡置懷州因古地名以名州也】枋頭鎭將代人薛虎子先為馮太后所黜為門士【魏有宰士門士宰士掌酒食門士守門戶枋音方先悉薦翻】時山東饑盗賊競起相州民孫誨等五百人稱虎子在鎭境内清晏乞還虎子上皇復以虎子為枋頭鎭將【相息亮翻下同復扶又翻又如字】即日之官數州盗賊皆息【數州謂冀相懷等州】 十二月癸卯朔日有食之 乙巳江州刺史桂陽王休範進位太尉 詔起袁粲以衛軍將軍攝職【所謂起復也】粲固辭 壬子柔然侵魏柔玄鎭二部敕勒應之【據水經注柔玄鎭在長川城東城南小山於延水所出也此即六鎭之一】 魏州鎭十一水旱相州民餓死者二千八百餘人 是歲魏妖人劉舉聚衆自稱天子【妖如遥翻】齊州刺史武昌王平原討斬之平原提之子也【武昌王提見一百二十五卷宋文帝元嘉二十四年】<br />
<br />
  二年春正月丁丑魏太尉源賀以疾罷 二月甲辰魏上皇還平城 三月丁亥魏員外散騎常侍許赤虎來聘 夏五月壬午桂陽王休範反 【考異曰宋書作壬子按長歷此月辛未朔無壬子今從宋畧】掠民船使軍隊稱力請受【軍有軍主副隊有隊主副稱力請受者稱其衆力之多少而請船也稱尺證翻】付以材板合手裝治【合音閤治直之翻】數日即辦丙戌休範率衆二萬騎五百發尋陽【騎奇寄翻】晝夜取道以書與諸執政稱楊運長王道隆蠱惑先帝使建安巴陵二王無罪被戮【事見上明帝泰始七年蠱音古被皮義翻】望執錄二豎以謝寃魂【豎常句翻】庚寅大雷戍主杜道欣馳下告變朝廷惶駭護軍褚淵征北將軍張永領軍劉勔僕射劉秉右衛將軍蕭道成游擊將軍戴明寶驍騎將軍阮佃夫右軍將軍王道隆中書舍人孫千齡員外郎楊運長【員外郎即員外散騎侍郎驍堅堯翻騎奇寄翻】集中書省計事莫有言者道成曰昔上流謀逆皆因淹緩致敗【謂南郡王義宣晉安王子勛等也】休範必遠懲前失輕兵急下乘我無備今應變之術不宜遠出若偏師失律則大沮衆心【沮在呂翻】宜頓新亭白下堅守宫城東府石頭以待賊至千里孤軍後無委積【委於偽翻積子智翻周禮遺人凡賓客會同師役掌其道路之委積注云少曰委多曰積又宰夫掌牢禮委積注云委積謂牢米薪芻左傳居則具一日之積杜預注云芻米薪】求戰不得自然瓦解我請頓新亭以當其鋒征北守白下領軍屯宣陽門為諸軍節度諸貴安坐殿中不須競出我自破賊必矣因索筆下議衆並注同【並注名同道成議也索山客翻】孫千齡隂與休範通謀獨曰宜依舊遣軍據梁山道成正色曰賊今已近梁山豈可得至新亭既是兵衝所欲以死報國耳常時乃可屈曲相從今不得也坐起【坐徂卧翻】道成顧謂劉勔曰領軍已同鄙議不可改易袁粲聞難扶曳入殿【袁粲居喪毀瘠故扶曳而入難乃旦翻】即日内外戒嚴道成將前鋒兵出屯新亭【將即亮翻】張永屯白下前南兖州刺史沈懷明戍石頭袁粲褚淵入衛殿省時倉猝不暇授甲開南北二武庫隨將士意所取蕭道成至新亭治城壘未畢【將即亮翻下同治直之翻】辛卯休範前軍已至新林【新林浦去今建康城二十里】道成方解衣高卧以安衆心徐索白虎幡【索山客翻】登西垣使寧朔將軍高道慶羽林監陳顯逹員外郎王敬則帥舟師與休範戰頗有殺獲【帥讀曰率】壬辰休範自新林捨舟步上【上時掌翻】其將丁文豪請休範直攻臺城休範遣文豪别將兵趣臺城【趣七喻翻】自以大衆攻新亭壘道成率將士悉力拒戰自己至午外勢愈盛衆皆失色道成曰賊雖多而亂尋當破矣休範白服乘肩輿自登城南臨滄觀【臨滄觀在勞山上江寧縣南十五里亦曰勞勞亭觀古玩翻】以數十人自衛 【考異曰張敬兒傳云左右數百人皆散走按休範左右若有數百人黄回敬兒雖勇何敢徑往取之今從休範傳】屯騎校尉黃回與越騎校尉張敬兒謀詐降以取之【騎奇寄翻校戶教翻降戶江翻下同】囘謂敬兒曰卿可取之我誓不殺諸王敬兒以白道成道成曰卿能辦事當以本州相賞【敬兒南陽冠軍人本州謂雍州也為後敬兒求雍州張本】乃與囘出城南放仗走大呼稱降休範喜召至輿側囘陽致道成密意休範信之以二子德宣德嗣付道成為質【呼火故翻質音致】二子至道成即斬之休範置囘敬兒於左右所親李恒鍾爽諫不聽【恒戶登翻】時休範日飲醇酒囘見休範無備目敬兒敬兒奪休範防身刀斬休範首左右皆散走敬兒騎馬持首歸新亭道成遣隊主陳靈寶送休範首還臺靈寶道逢休範兵弃首於水【考異曰南齊書云埋首道側宋畧云弃諸溝中今從宋書】挺身得逹唱云已平而無以為驗衆莫之信休範將士亦不之知其將杜黑騾攻新亭甚急【將即亮翻騾盧戈翻考異曰宋書南齊書作黑蠡而今從宋畧】蕭道成在射堂司空主簿蕭惠<br />
<br />
  朗帥敢死士數十人突入東門【休範先為司空以惠朗為主簿帥讀曰率下同】至射堂下道成上馬帥麾下摶戰惠朗乃退道成復得保城【復扶又翻】惠朗惠開之弟也【蕭惠開見一百三十一卷明帝泰始之元年二年】其姊為休範妃惠朗兄黃門郎惠明時為道成軍副在城内了不自疑【左傳宋元公與華氏戰于新里翟僂新居于新里既戰說甲于公而歸華姓居于公里亦如之杜預注曰古之為軍不呰小忿僂力主翻姓他口翻呰于斯翻又音紫】道成與黑騾拒戰自晡逹旦矢石不息其夜大雨鼓叫不復相聞【復扶又翻】將士積日不得寢食軍中馬夜驚城内亂走道成秉燭正坐厲聲呵之如是者數四丁文豪破臺軍於皂莢橋【皂莢橋當在新亭之北】直至朱雀桁南【朱雀桁即大航也在秦淮水上以其在朱雀門外故名桁與航同音戶剛翻】杜黑騾亦捨新亭北趣朱雀桁【趣七喻翻】右軍將軍王道隆將羽林精兵在朱雀門内急召鄱陽忠昭公劉勔於石頭勔至命撤桁以折南軍之勢道隆怒曰賊至但當急擊寧可開桁自弱邪勔不敢復言道隆趣勔進戰【折之舌翻復扶又翻趣讀曰促】勔度桁南戰敗而死黑騾等乘勝度淮道隆弃衆走還臺黑騾兵追殺之【蕭道成所謂諸貴不須競出者正慮此也】黃門侍郎王藴重傷踣於御溝之側【踣蒲北翻】或扶之以免藴景文之兄子也於是中外大震道路皆云臺城已陷白下石頭之衆皆潰張永沈懷明逃還宫中傳新亭亦䧟太后執帝手泣曰天下敗矣先是月犯右執法太白犯上將【先悉薦翻太微南蕃中二星曰端門東曰左執法西曰右執法東蕃四星其北星曰上將西蕃四星南第一星亦曰上將】或勸劉勔解職勔曰吾執心行已無愧幽明若災眚必至避豈得免勔晩年頗慕高尚立園宅名為東山遺落世務罷遣部曲蕭道成謂勔曰將軍受顧命輔幼主當此艱難之日而深尚從容【從千容翻】廢省羽翼一朝事至悔可追乎勔不從而敗甲午撫軍長史禇澄開東府門納南軍 【考異曰宋書作撫軍興籖茅恬開東府納賊南齊書作車騎典韯茅恬盖皆為禇澄諱耳今從宋畧】擁安成王凖據東府稱桂陽王教曰安成王吾子也勿得侵犯澄淵之弟也杜黑騾徑進至杜姥宅【晉成帝杜皇后母裴氏立第南掖門外世謂之杜姥宅姥莫補翻】中書舍人孫千齡開承明門出降【文帝元嘉二十五年新作閭闔廣莫二門改廣莫門曰承明門降戶江翻】宫省恇擾【恇去王翻】時府藏已竭【藏徂浪翻】皇太后太妃剔取宫中金銀器物以充賞衆莫有鬭志俄而丁文豪之衆知休範已死稍欲退散文豪厲聲曰我獨不能定天下邪許公輿詐稱桂陽王在新亭士民惶惑詣蕭道成壘投刺者以千數【凡求見之禮先投刺以自逹毛晃曰書姓名以自白故曰刺】道成得皆焚之登北城謂曰劉休範父子昨已就戮尸在南岡下【南岡即勞山之岡也以在新亭城南故謂之南岡】身是蕭平南【道成之出屯新亭也加平南將軍】諸君諦視之【諦音帝審也】名刺皆已焚勿憂懼也道成遣陳顯逹張敬兒及輔師將軍任農夫馬軍主東平周盤龍等將兵自石頭濟淮從承明門入衛宫省袁粲慷慨謂諸將曰今寇賊已逼而衆情離沮孤子受先帝付託不能綏靜國家【粲時居喪故自稱孤子任音壬將即亮翻沮在呂翻】請與諸君同死社稷被甲上馬將驅之【被皮義翻】於是陳顯逹等引兵出戰大破杜黑騾於杜姥宅飛矢貫顯逹目丙申張敬兒等又破黑騾等於宣陽門斬黑騾及丁文豪進克東府餘黨悉平蕭道成振旅還建康【還從宣翻又如字】百姓緣道聚觀曰全國家者此公也道成與袁粲褚淵劉秉皆上表引咎解職不許丁酉解嚴大赦 柔然遣使來聘【使疏吏翻】 六月庚子以平南將軍蕭道成為中領軍南兖州刺史留衛建康【道成自此得政矣】與袁粲褚淵劉秉更日入直决事號為四貴 桂陽王休範之反也使道士陳公昭作天公書題云沈丞相付荆州刺史沈攸之門者攸之不開視推得公昭送之朝廷 【考異曰宋畧云桂陽遺攸之書署曰沈丞相攸之斬其使今從宋畧】及休範反攸之謂僚佐曰桂陽必聲言我與之同若不顛沛勤王【顛沛勤王者危難之際奔走顛沛以從王事也】必增朝野之惑乃與南徐州刺史建平王景素郢州刺史晉熙王爕湘州刺史王僧虔雍州刺史張興世【雍於用翻】同舉兵討休範休範留中兵參軍毛惠連等守尋陽爕遣中兵參軍馮景祖襲之癸卯惠連等開門請降【降戶江翻 考異曰宋畧作癸亥按下有戊申今從宋書】殺休範二子諸鎮皆罷兵景素宏之子也【建平王宏文帝之子也】 乙卯魏詔曰下民兇戾不顧親戚一人為惡殃及闔門朕為民父母深所愍悼自今非謀反大逆外叛罪止其身於是始罷門房之誅【門誅者誅其一門房誅者誅其一房時河北大族如崔如李子孫分沠各自為房】魏顯祖勤於為治【冶直吏翻】賞罰嚴明愼擇牧守【守式又翻】進亷退貪諸曹疑事舊多奏决又口傳詔敕或致矯擅上皇命事無大小皆據律正名不得為疑奏合則制可違則彈詰盡用墨詔【制可者手詔可其所奏彈詰者劾問之合謂與律合也違謂悖於律也詰去吉翻】由是事皆精審尤重刑罰大刑多令覆鞫【大刑謂死罪】或囚繫積年羣臣頗以為言上皇曰滯獄誠非善治【治直吏翻】不猶愈於倉猝而濫乎夫人幽苦則思善故智者以囹圄為福堂【囹盧丁翻圄音語】朕特苦之欲其改悔而加矜恕爾由是囚繫雖滯而所刑多得其宜又以赦令長姦故自延興以後不復有赦【泰始七年魏改元延興長知兩翻復扶又翻】 秋七月庚辰立皇弟友為邵陵王 乙酉加荆州刺史沈攸之開府儀同三司攸之固辭執政欲徵攸之而憚於發命乃以太后令遣中使謂曰【使疏吏翻】公久勞於外宜還京師任寄實重未欲輕之進退可否在公所擇攸之曰臣無廊廟之資居中實非其才至於撲討蠻蜑【蜑亦蠻屬音蕩旱翻毛晃曰蜑南夷海種也范成大桂海漁衡志曰蜑海上水居蠻也以舟為家沿海蜑有三種漁蜑取魚蠔蛋取蠔木蜑伐山取材大率皆取海物為糧生食之入水能視陳師道曰二廣居山谷間不隸州縣謂之傜人舟居謂之蜑人島居謂之黎人余謂巴黔亦自有蜑人】克清江漢不敢有辭雖自上如此【上時掌翻】去留伏聽朝旨【朝直遥翻】乃止 癸巳柔然寇魏敦煌尉多侯擊破之尚書奏敦煌僻遠介居西北強寇之間【西謂吐谷渾北柔然也敦徒門翻尉紆勿翻】恐不能自固請内徙就凉州羣臣集議皆以為然給事中昌黎韓秀獨以為敦煌之置為日已久雖逼彊寇人習戰鬬縱有草竊不為大害循常置戍足以自全而能隔閡西北二虜使不得相通【閡五慨翻】今徙就凉州不唯有蹙國之名且姑臧去敦煌千有餘里防邏甚難【邏力佐翻】二虜必有交通闚?之志【漢武帝開河西四郡以隔絶西羌月氏不得與匈奴通其規畫正如此也?音俞】若騷動凉州則關中不得安枕又士民或安土重遷招引外寇為國深患不可不慮也乃止九月 【考異曰後魏帝紀使將軍元蘭五將三萬騎及假東陽王丕為後繼伐蜀漢不言勝負列傳及宋書皆無之今不取】丁酉以尚書令袁粲為中書監領司徒加禇淵尚書令劉秉丹陽尹粲固辭求反居墓所不許【粲受遺輔政適罹艱棘國有大難釋衰經以從金革之事大難既平求歸終喪禮也】淵以禇澄為吳郡太守司徒左長史蕭惠明言於朝【朝直遥翻 考異曰宋畧作惠朗按惠朗不為司徒長史今從南史】曰禇澄開門納賊更為股肱大郡王藴力戰幾死棄而不收【幾居希翻又音祁】賞罰如此何憂不亂淵甚慙冬十月庚申以侍中王藴為湘州刺史 十一月丙戍帝加元服大赦 十二月癸亥立皇弟躋為江夏王贊為武陵王【躋牋西翻夏戶雅翻】 是歲魏建安貞王陸馛卒【馛蒲撥翻】<br />
<br />
  三年春正月辛巳帝祀南郊明堂【二祀並舉也】 蕭道成以襄陽重鎭張敬兒人位俱輕不欲使居之而敬兒求之不已【盖道成先許之也】謂道成曰沈攸之在荆州公知其欲何所作不出敬兒以表裏制之恐非公之利道成笑而無言【道成居内敬兒居外以制其後故曰表裏制之也】三月己巳以驍騎將軍張敬兒為都督雍梁二州諸軍事雍州刺史【驍堅堯翻騎奇寄翻雍於用翻】沈攸之聞敬兒上【上時掌翻】恐其見襲隂為之備敬兒既至奉事攸之親敬甚至動輒咨禀信饋不絶攸之以為誠然酬報欵厚累書欲因遊獵會境上敬兒報以為心期有在影迹不宜過敦【謂動則有影行則有迹人將窺見之也敦厚也】攸之益信之敬兒得其事迹皆密白道成道成與攸之書問張雍州遷代之日將欲誰擬攸之即以示敬兒欲以間之【史言攸之墮敬兒術中而不悟且為襲江陵張本問古莧翻】 夏五月丙午魏主使員外散騎常侍許赤虎來聘 丁未魏主如武州山辛酉如車輪山【地形志秀容郡敷城縣有車輪泉車尺遮翻】六月庚午魏初禁殺牛馬【牛者農之所資馬者兵之所資禁殺當也槐興於北荒畜收蕃庶殺之者不禁今始禁之】袁粲禇淵皆固讓新官秋七月庚戌復以粲為尚書令【復扶又翻】八月庚子加護軍將軍褚淵中書監 冬十二月丙寅魏徙建昌王長樂為安樂王【樂音洛】 己丑魏城陽王長壽卒 南徐州刺史建平王景素孝友清令服用儉素又好文學【好呼到翻】禮接士大夫由是有美譽太宗特愛之異其禮秩時太祖諸子俱盡諸孫唯景素為長帝凶狂失德朝野皆屬意於景素帝外家陳氏深惡之【長知两翻屬之欲翻惡烏路翻】楊運長阮佃夫等欲專權勢不利立長君亦欲除之其腹心將佐多勸景素舉兵鎭軍參軍濟陽江淹獨諫之【景素時以鎮北將軍鎮京口以淹為主簿鎭軍當作鎭北齊子禮翻】景素不悦是歲防閤將軍王季符得罪於景素【江左之制禁衛有直閤將軍王國有防閤將軍】單騎亡奔建康告景素謀反【騎奇寄翻】運長等即欲兵討之袁粲蕭道成以為不可景素亦遣世子延齡詣闕自陳乃徙季符於梁州奪景素征北將軍開府儀同三司【征北亦當作鎭北】<br />
<br />
  資治通鑑卷一百三十三  <br>
   </div> 

<script src="/search/ajaxskft.js"> </script>
 <div class="clear"></div>
<br>
<br>
 <!-- a.d-->

 <!--
<div class="info_share">
</div> 
-->
 <!--info_share--></div>   <!-- end info_content-->
  </div> <!-- end l-->

<div class="r">   <!--r-->



<div class="sidebar"  style="margin-bottom:2px;">

 
<div class="sidebar_title">工具类大全</div>
<div class="sidebar_info">
<strong><a href="http://www.guoxuedashi.com/lsditu/" target="_blank">历史地图</a></strong>  
<a href="http://www.880114.com/" target="_blank">英语宝典</a>  
<a href="http://www.guoxuedashi.com/13jing/" target="_blank">十三经检索</a> 
<br><strong><a href="http://www.guoxuedashi.com/gjtsjc/" target="_blank">古今图书集成</a></strong> 
<a href="http://www.guoxuedashi.com/duilian/" target="_blank">对联大全</a> <strong><a href="http://www.guoxuedashi.com/xiangxingzi/" target="_blank">象形文字典</a></strong> 

<br><a href="http://www.guoxuedashi.com/zixing/yanbian/">字形演变</a>  <strong><a href="http://www.guoxuemi.com/hafo/" target="_blank">哈佛燕京中文善本特藏</a></strong>
<br><strong><a href="http://www.guoxuedashi.com/csfz/" target="_blank">丛书&方志检索器</a></strong> <a href="http://www.guoxuedashi.com/yqjyy/" target="_blank">一切经音义</a>  

<br><strong><a href="http://www.guoxuedashi.com/jiapu/" target="_blank">家谱族谱查询</a></strong>  <strong><a href="http://shufa.guoxuedashi.com/sfzitie/" target="_blank">书法字帖欣赏</a></strong> 
<br>

</div>
</div>


<div class="sidebar" style="margin-bottom:0px;">

<font style="font-size:22px;line-height:32px">QQ交流群9:489193090</font>


<div class="sidebar_title">手机APP 扫描或点击</div>
<div class="sidebar_info">
<table>
<tr>
	<td width=160><a href="http://m.guoxuedashi.com/app/" target="_blank"><img src="/img/gxds-sj.png" width="140"  border="0" alt="国学大师手机版"></a></td>
	<td>
<a href="http://www.guoxuedashi.com/download/" target="_blank">app软件下载专区</a><br>
<a href="http://www.guoxuedashi.com/download/gxds.php" target="_blank">《国学大师》下载</a><br>
<a href="http://www.guoxuedashi.com/download/kxzd.php" target="_blank">《汉字宝典》下载</a><br>
<a href="http://www.guoxuedashi.com/download/scqbd.php" target="_blank">《诗词曲宝典》下载</a><br>
<a href="http://www.guoxuedashi.com/SiKuQuanShu/skqs.php" target="_blank">《四库全书》下载</a><br>
</td>
</tr>
</table>

</div>
</div>


<div class="sidebar2">
<center>


</center>
</div>

<div class="sidebar"  style="margin-bottom:2px;">
<div class="sidebar_title">网站使用教程</div>
<div class="sidebar_info">
<a href="http://www.guoxuedashi.com/help/gjsearch.php" target="_blank">如何在国学大师网下载古籍?</a><br>
<a href="http://www.guoxuedashi.com/zidian/bujian/bjjc.php" target="_blank">如何使用部件查字法快速查字?</a><br>
<a href="http://www.guoxuedashi.com/search/sjc.php" target="_blank">如何在指定的书籍中全文检索?</a><br>
<a href="http://www.guoxuedashi.com/search/skjc.php" target="_blank">如何找到一句话在《四库全书》哪一页?</a><br>
</div>
</div>


<div class="sidebar">
<div class="sidebar_title">热门书籍</div>
<div class="sidebar_info">
<a href="/so.php?sokey=%E8%B5%84%E6%B2%BB%E9%80%9A%E9%89%B4&kt=1">资治通鉴</a> <a href="/24shi/"><strong>二十四史</strong></a>&nbsp; <a href="/a2694/">野史</a>&nbsp; <a href="/SiKuQuanShu/"><strong>四库全书</strong></a>&nbsp;<a href="http://www.guoxuedashi.com/SiKuQuanShu/fanti/">繁体</a>
<br><a href="/so.php?sokey=%E7%BA%A2%E6%A5%BC%E6%A2%A6&kt=1">红楼梦</a> <a href="/a/1858x/">三国演义</a> <a href="/a/1038k/">水浒传</a> <a href="/a/1046t/">西游记</a> <a href="/a/1914o/">封神演义</a>
<br>
<a href="http://www.guoxuedashi.com/so.php?sokeygx=%E4%B8%87%E6%9C%89%E6%96%87%E5%BA%93&submit=&kt=1">万有文库</a> <a href="/a/780t/">古文观止</a> <a href="/a/1024l/">文心雕龙</a> <a href="/a/1704n/">全唐诗</a> <a href="/a/1705h/">全宋词</a>
<br><a href="http://www.guoxuedashi.com/so.php?sokeygx=%E7%99%BE%E8%A1%B2%E6%9C%AC%E4%BA%8C%E5%8D%81%E5%9B%9B%E5%8F%B2&submit=&kt=1"><strong>百衲本二十四史</strong></a>  <a href="http://www.guoxuedashi.com/so.php?sokeygx=%E5%8F%A4%E4%BB%8A%E5%9B%BE%E4%B9%A6%E9%9B%86%E6%88%90&submit=&kt=1"><strong>古今图书集成</strong></a>
<br>

<a href="http://www.guoxuedashi.com/so.php?sokeygx=%E4%B8%9B%E4%B9%A6%E9%9B%86%E6%88%90&submit=&kt=1">丛书集成</a> 
<a href="http://www.guoxuedashi.com/so.php?sokeygx=%E5%9B%9B%E9%83%A8%E4%B8%9B%E5%88%8A&submit=&kt=1"><strong>四部丛刊</strong></a>  
<a href="http://www.guoxuedashi.com/so.php?sokeygx=%E8%AF%B4%E6%96%87%E8%A7%A3%E5%AD%97&submit=&kt=1">說文解字</a> <a href="http://www.guoxuedashi.com/so.php?sokeygx=%E5%85%A8%E4%B8%8A%E5%8F%A4&submit=&kt=1">三国六朝文</a>
<br><a href="http://www.guoxuedashi.com/so.php?sokeytm=%E6%97%A5%E6%9C%AC%E5%86%85%E9%98%81%E6%96%87%E5%BA%93&submit=&kt=1"><strong>日本内阁文库</strong></a> <a href="http://www.guoxuedashi.com/so.php?sokeytm=%E5%9B%BD%E5%9B%BE%E6%96%B9%E5%BF%97%E5%90%88%E9%9B%86&ka=100&submit=">国图方志合集</a> <a href="http://www.guoxuedashi.com/so.php?sokeytm=%E5%90%84%E5%9C%B0%E6%96%B9%E5%BF%97&submit=&kt=1"><strong>各地方志</strong></a>

</div>
</div>


<div class="sidebar2">
<center>

</center>
</div>
<div class="sidebar greenbar">
<div class="sidebar_title green">四库全书</div>
<div class="sidebar_info">

《四库全书》是中国古代最大的丛书,编撰于乾隆年间,由纪昀等360多位高官、学者编撰,3800多人抄写,费时十三年编成。丛书分经、史、子、集四部,故名四库。共有3500多种书,7.9万卷,3.6万册,约8亿字,基本上囊括了古代所有图书,故称“全书”。<a href="http://www.guoxuedashi.com/SiKuQuanShu/">详细>>
</a>

</div> 
</div>

</div>  <!--end r-->

</div>
<!-- 内容区END --> 

<!-- 页脚开始 -->
<div class="shh">

</div>

<div class="w1180" style="margin-top:8px;">
<center><script src="http://www.guoxuedashi.com/img/plus.php?id=3"></script></center>
</div>
<div class="w1180 foot">
<a href="/b/thanks.php">特别致谢</a> | <a href="javascript:window.external.AddFavorite(document.location.href,document.title);">收藏本站</a> | <a href="#">欢迎投稿</a> | <a href="http://www.guoxuedashi.com/forum/">意见建议</a> | <a href="http://www.guoxuemi.com/">国学迷</a> | <a href="http://www.shuowen.net/">说文网</a><script language="javascript" type="text/javascript" src="https://js.users.51.la/17753172.js"></script><br />
  Copyright &copy; 国学大师 古典图书集成 All Rights Reserved.<br>
  
  <span style="font-size:14px">免责声明:本站非营利性站点,以方便网友为主,仅供学习研究。<br>内容由热心网友提供和网上收集,不保留版权。若侵犯了您的权益,来信即刪。scp168@qq.com</span>
  <br />
ICP证:<a href="http://www.beian.miit.gov.cn/" target="_blank">鲁ICP备19060063号</a></div>
<!-- 页脚END --> 
<script src="http://www.guoxuedashi.com/img/plus.php?id=22"></script>
<script src="http://www.guoxuedashi.com/img/tongji.js"></script>

</body>
</html>
