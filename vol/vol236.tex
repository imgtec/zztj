










 


 
 


 

  
  
  
  
  





  
  
  
  
  
 
  

  

  
  
  



  

 
 

  
   




  

  
  


    資治通鑑卷二百三十六 宋 司馬光 撰

  胡三省 音註

  唐紀五十二【起重光大荒落盡旃蒙作噩凡五年}


  德宗神武聖文皇帝十一

  貞元十七年春正月甲寅韓全義至長安竇文場為掩其敗迹【為于偽翻下同}
上禮遇甚厚全義稱足疾不任朝謁【任音壬朝直遥翻 考異曰舊全義傳令中使就第賜宴自還至辭都不謁見而去議者以隳敗法制從古以還未有如貞元之甚按實錄壬戌宴全義於麟德殿又云自還及歸不見不辭于正朝蓋非不謁也但不於正朝耳}
遣司馬崔放入對放為全義引咎謝無功【為于偽翻}
上曰全義為招討使能招來少誠其功大矣何必殺人然後為功邪【德宗之耳目為宦官所聾瞽率類此}
閏月甲戍歸夏州【夏戶雅翻}
 韋士宗既入黔州【去年士宗復入黔州事見上卷黔渠今翻又其亷翻}
妄殺長吏人心大擾士宗懼三月脱身亡走夏四月辛亥以右諫議大夫裴佶為黔州觀察使【佶其吉翻}
 五月壬戌朔日有食之朔方邠寧慶節度使楊朝晟【朔方兵分居邠故仍以朔方軍號冠之其實只}


  【節度邠寧慶三州}
防秋于寧州乙酉薨初渾瑊遣兵馬使李朝寀將兵戍定平【武德二年改寧州定安縣置定平縣仍屬寧州九域志在州南六十里朝直遥翻宷倉宰翻將即亮翻}
瑊薨朝寀請以其衆隸神策軍詔許之楊朝晟疾亟【亟汜力翻}
召僚佐謂曰朝晟必不起朔方命帥多自本軍雖徇衆情殊非國體【帥所類翻下同}
寧州刺史劉南金練習軍旅宜使攝行軍且知軍事比朝廷擇帥【比必利翻及也}
必無虞矣又以手書授監軍劉英倩英倩以聞軍士私議曰朝廷命帥吾納之即命劉君吾事之若命帥于它軍彼必以其麾下來吾屬被斥矣必拒之己丑上遣中使往察軍情軍中多與南金辛卯上復遣高品薛盈珍齎詔詣寧州【唐内侍省有高品一千九百六十六人復扶又翻}
六月甲午盈珍至軍宣詔曰朝寀所將本朔方軍今將并之以壯軍勢威戎狄以李朝宷為使南金副之軍中以為何如諸將皆奉詔丙申都虞史經言於衆曰李公命收弓刀而送甲胄二千軍士皆曰李公欲内麾下二千為腹心吾輩妻子其可保乎夜造劉南金【造七到翻}
欲奉以為帥南金曰節度使固我所欲然非天子之命則不可軍中豈無它將乎【將即亮翻}
衆曰弓刀皆為官所收惟軍事府尚有甲兵【軍事所知軍事所居也}
欲因以集事南金曰諸君不願朝寀為帥宜以情告敕使若操甲兵【操七刀翻}
乃拒詔也命閉門不内軍士去詣兵馬使高固固逃匿搜得之固曰諸君能用吾言則可衆曰惟命固曰毋殺人毋掠金帛衆曰諾乃共詣監軍請奏之衆曰劉君既得朝旨為副帥必撓吾事【撓奴巧翻}
詐稱監軍命召計事至而殺之戊戌制以李朝寀為邠寧節度使是日寧州告變者至上追還制書復遣薛盈珍往詗軍情【復扶又翻下同詗火迴翻又翾正翻}
壬寅至軍軍中以高固為請盈珍即以上旨命固知軍事或傳戊戌制書至邠州邠軍惑不知所從【薛盈珍已命高固知寧州軍事而又有傳李朝寀制書至邠者故留邠之軍惑而不知所適從}
姦人乘之且為變留後孟子周悉内精甲於府廷日饗士卒内以悦衆心外以威姦黨邠軍無變子周之謀也 李錡既執天下利權【十五年李錡為諸道鹽鐵轉運使事見上卷}
以貢獻固主恩以饋遺結權貴【遺唯季翻}
恃此驕縱無所忌憚盗取縣官財所部官屬無罪受戮者相繼浙西布衣崔善貞詣闕上封事言宫市進奉及鹽鐵之弊因言錡不法事上覽之不悦命械送錡錡聞其將至先鑿阬于道旁己亥善貞至并鎻械内阬中生瘞之【瘞於計翻}
遠近聞之不寒而慄錡復欲為自全計增廣兵衆選有材力善射者謂之挽彊【言其力能挽彊弓也杜甫詩挽弓當挽彊}
胡奚雜類謂之蕃落【胡奚之俘配隸江南者錡收養之}
給賜十倍它卒轉運判官盧坦屢諫不悛【悛丑緣翻}
與幕僚李約等皆去之約勉之子也【李勉歷事肅代德三朝貞元中為相}
 己酉以高固為邠寧節度使固宿將以寛厚得衆節度使忌之置于散地【散悉但翻}
同列多輕侮之及起為帥一無所報復軍中遂安 丁巳成德節度使王武俊薨 秋七月戊寅吐蕃寇鹽州辛巳以成德節度副使王士真為節度使 己丑吐

  蕃陷麟州殺刺使郭鋒夷其城郭掠居人及党項部落而去鋒曜之子也【曜郭子儀之子也}
僧延素為虜所得虜將有徐舍人者謂延素曰我英公五代孫也【李勣封英國公}
武后時吾高祖建義不成【謂敬業也事見二百二卷武后光宅元年}
子孫流播異域雖代居禄位典兵然思本之心不忘顧宗族大無由自抜耳今聽汝歸遂縱之上遣使敕韋臯出兵深入吐蕃以分其勢紓北邊患【紓緩也}
臯遣將將兵二萬分出九道攻吐蕃維保松州及棲雞老翁城【宋白曰保州本維州之定廉縣南按吐蕃為夷落之極塞開元二十八年羌夷内附置奉州天寶改雲山郡八載移治天保軍改為天保郡尋沒乾元元年復歸附乃改為保州按王涯傳曰綿州威蕃柵西抵棲雞城蓋在茂州界}
 河東節度使鄭儋暴薨不及命後事軍中喧譁將有它變中夜十餘騎執兵召掌書記令狐楚至軍門諸將環之【環音宦}
使草遺表楚在白刃之中操筆立成楚德棻之族也【令狐德棻事太宗疑族字下有孫及曾玄等字棻撫文翻}
八月戊午以河東行軍司馬嚴綬為節度使 九月韋臯奏大破吐蕃于雅州【宋白曰雅州即秦嚴道縣地後魏立蒙山郡唐立雅州按郡國志漢源縣有離山蜀守李冰所鑿離即古雅字也州以此為名舊志雅州京師西南二千七百二十三里}
 左神策中尉竇文場致仕以副使楊志亷代之 韋臯屢破吐蕃轉戰千里凡抜城七軍鎮五焚堡百五十斬首萬餘級捕虜六千降戶三千遂圍維州及昆明城冬十月庚子加臯檢校司徒兼中書令賜爵南康郡王南詔王異牟尋虜獲尤多上遣中使慰撫之 戊午鹽州刺史杜彦先弃城奔慶州【為吐蕃所逼也鹽州修築距是年纔八年}


  十八年春正月驃王摩羅思那遣其子悉利移入貢驃國在南詔西南六千八百里【新書驃古朱波也在永昌南二千里去京師萬四千里驃毗召翻}
聞南詔内附而慕之因南詔入見【見賢遍翻}
仍獻其樂 吐蕃遣其大相兼東鄙五道節度使論莽熱將兵十萬解維州之圍西川兵據險設伏以待之吐蕃至出千人挑戰【挑徒了翻}
虜悉衆追之伏發虜衆大敗擒論莽熱士卒死者太半維州昆明竟不下引兵還【還從宣翻又如字}
乙亥臯遣使獻論莽熱 【考異曰舊韋臯傳云十月遣使獻論莽熱今從實録}
上赦之 浙東觀察使裴肅既以進奉得進【裴肅以進奉得亷車事見上卷十二年}
判官齊總代掌後務【據新唐書肅卒于官齊總代掌後務}
刻剥以求媚又過之三月癸酉詔擢總為衢州刺史給事中長安許孟容封還詔書【封還詔書不肯書讀所謂糾駮也亦謂之塗歸唐人語也}
曰衢州無它虞齊總無殊績忽此超奬深駭羣情若總必有可録願明書勞課然後超資改官以解衆疑詔遂留中己亥上召孟容慰奬之 秋七月辛未嘉王府諮議高弘本正牙奏事【嘉王運代宗之子諮議参軍正五品上掌計謀議事唐東内以含元殿為正牙西内以太極殿為正牙唐制天子居曰衙行曰駕牙與衙同}
自理逋債【逋欠也}
乙亥詔公卿庶僚自今勿令正牙奏事如有陳奏宜延英門請對議者以為正牙奏事自武德以來未之或改所以逹羣情講政事弘本無知黜之可也不當因人而廢事 淮南節度使杜佑累表求代冬十月丁亥以刑部尚書王鍔為淮南副節度使兼行軍司馬【鍔五各翻副節度使恐當作節度副使}
己酉鄜坊節度使王栖曜薨中軍將何朝宗謀作亂

  夜縱火都虞候裴玢潜匿不救火【朝直遙翻玢府巾翻}
旦擒朝宗斬之以同州刺史劉公濟為鄜坊節度使以玢為行軍司馬

  十九年春二月丁亥名安黄軍曰奉義【以寵伊慎也}
 己亥安南牙將王季元逐其觀察使裴泰泰奔朱鳶【劉昫曰朱鳶漢縣名今縣吳軍平縣地晉武帝更名海平江左置武平郡隋廢郡為朱鳶縣唐屬交州}
明日左兵馬使趙匀斬季元及其黨迎泰而復之 甲辰杜佑入朝【自淮南入朝}
三月壬子朔以佑檢校司空同平章事以王鍔為淮南節度使 鴻臚卿王權請遷獻懿二祖于德明興聖廟【玄宗天寶二年尊咎繇為德明皇帝凉武昭王為興聖皇帝立廟京師臚陵如翻}
每禘祫正太祖東向之位從之【建中二年奉獻祖正東向之位事見二百二十七卷}
乙亥以司農卿李實兼京兆尹實為政暴戾上愛信之實恃恩驕傲許人薦引不次拜官及誣譖斥逐皆如期而效士大夫畏之側目 夏四月涇原節度使劉昌奏請徙原州治平凉從之【七年劉昌築平凉事見二百三十三卷原州本治高平唐為平高縣為吐蕃所䧟}
乙亥吐蕃遣其臣論頰熱入貢 六月辛卯以右神策中尉副使孫榮義為中尉與楊志亷皆驕縱招權【楊志亷時為左軍中尉 考異曰實録十七年六月以中官楊志亷充左神策護軍中尉七月丙戍以内給事楊志亷為左右神策護軍中尉副使九月戊寅以志亷為左神策中尉十九年六月辛卯以榮義為右神策中尉二十年十月戊申以志亷為特進右監軍將軍左軍中尉其重複差互如此蓋十七年六月攝領耳七年始為副使九月及十九年六月始正為中尉二十年十月但進階加官耳舊傳又云先是竇文場致仕十五年以後志亷榮義為左右軍中尉亦踵竇之事此蓋言其大畧耳未必為中尉適在十五年也 余按右監軍將軍當作右監門將軍}
依附者衆宦官之勢益盛 壬辰遣右龍武大將軍薛伾使于吐蕃 陳許節度使上官涚薨其壻田偁欲脅其子使襲軍政【偁齒繩翻}
牙將王沛亦涚之壻也知其謀以告監軍范日用討擒之乙未以陳許行軍司馬劉昌裔為節度使沛許州人也 自正月不雨至于秋七月 己未中書侍郎同平章事齊抗以疾罷為太子賓客 初翰林待詔王伾善書山隂王叔文善碁【山隂漢古縣隋廢山隂入會稽縣唐初復分會稽置山隂縣二縣俱在越州郭下}
俱出入東宫娯侍太子伾杭州人也叔文譎詭多計自言讀書知治道乘間常為太子言民間疾苦【譎古宂翻治直吏翻乘間古莧翻為于偽翻}
太子嘗與諸侍讀及叔文等論及宫市事【太宗時晉王府有侍讀及為太子亦置焉其後或置或否無常員掌講導經學}
太子曰寡人方欲極言之衆皆稱贊獨叔文無言既退太子自留叔文謂曰向者君獨無言豈有意邪叔文曰叔文蒙幸太子有所見敢不以聞太子職當視膳問安【世子之記曰朝夕至于大寢之門外問内䜿曰今日安否何如内䜿曰安世子乃有喜色其有不安節則内䜿以告世子世子色憂不滿容内䜿言復初然後亦復初朝夕之食上世子必在視寒煖之節食下問所膳羞必知所進以命膳宰然後退若内䜿言疾則世子親齊玄而養膳宰之饌必敬視之疾之藥必親嘗之嘗饌善則世子亦能食嘗饌寡則世子亦不能飽以至于復初然後亦復初}
不宜言外事陛下在位久如疑太子收人心何以自解太子大驚因泣曰非先生寡人無以知此遂大愛幸與王伾相依附叔文因為太子言【為于偽翻}
某可為相某可為將幸異日用之密結翰林學士韋執誼及當時朝士有名而求速進者陸淳呂温李景儉韓曄韓泰陳諫柳宗元劉禹錫等定為死友而凌凖程异等又因其黨以進日與遊處【處昌呂翻}
蹤跡詭秘莫有知其端者藩鎮或隂進資幣與之相結淳吳人嘗為左司郎中温渭之子時為左拾遺【呂渭見上卷十六年}
景儉瑀之孫進士及第【瑀寧王憲之子封漢中王}
曄滉之族子【韓滉休之子貞元中為相}
諫嘗為侍御史宗元禹錫時為監察御史左補闕張正一上書得召見 【考異曰順宗實録作張正買今從德宗實録}
正一與吏部員外郎王仲舒主客員外郎劉伯芻等相親善 【考異曰韓愈集有仲舒神道碑云諱弘中字某按實録新舊傳皆名仲舒字弘中愈又作燕喜亭記稱為王弘中然則弘中必字也碑文誤耳順宗實録云正買與王仲舒劉伯芻裴□常仲孺呂洞相善數遊止今從德宗實録}
叔文之黨疑正一言已隂事令執誼反譖正一等于上云其朋黨遊宴無度九月甲寅正一等皆坐遠貶人莫知其由【為伾叔文等亂順宗初政張本}
伯芻迺之子也【劉迺見二百三十卷興元元年}
 鹽夏節度判官崔文先權知鹽州為政苛刻冬閏十月庚戌部將李庭俊作亂殺而臠食之左神策兵馬使李興幹戍鹽州殺庭俊以聞 丁巳門下侍郎同平章事崔損薨 十一月戊寅朔以李興幹為鹽州刺史得專奏事【李興幹出于神策軍宦官因其定亂之功而崇奬之}
自是鹽州不隸夏州【貞元三年置夏州節度使領綏鹽二州今鹽州得專逹於朝廷其後鹽州屬朔方節度夏州節度又增銀宥威三州隸之}
 十二月庚申以太常卿高郢為中書侍郎吏部侍郎鄭珣瑜為門下侍郎並同平章事珣瑜餘慶之從父兄弟也【鄭餘慶貞元十四年為相十六年坐于䪹貶從才用翻}
 建中初敕京城諸使及府縣繫囚每季終委御史廵按有寃濫者以聞【寃枉屈也濫淫刑也}
近歲北軍移牒而已【宦官勢横御史不敢復入北軍按囚但移文北司牒取繫囚姓名及事囚應故事而已不問其有無寃濫}
監察御史崔薳遇下嚴察下吏欲陷之引以入右神策軍軍使以下駭懼具奏其狀上怒杖薳四十流崖州【薳韋委翻}
 京兆尹嗣道王實務徵求以給進奉言于上曰今歲雖旱而禾苖甚美由是租税皆不免人窮至壞屋賣瓦木麥苖以輸官【壞音怪}
優人成輔端為謡嘲之【徒歌曰謡}
實奏輔端誹謗朝政杖殺之【朝直遥翻}
監察御史韓愈上疏以京畿百姓窮困應今年税錢及草粟等徵未得者請俟來年蠶麥愈坐貶陽山令【陽山漢縣屬桂陽郡後漢省晉平吳分浛洭縣復置唐屬連州神龍元年移縣治于浛水之北 考異曰韓愈河南令張署墓誌曰自京兆武功尉拜監察御史為幸臣所讒與同輩韓愈李方叔三人俱為縣令南方又祭署文曰貞元十九君為御史余以無能同詔並峙又曰我落陽山以尹鼯猱君飄臨武山林之牢歲弊寒兇雪虐風饕與署同貶當在此年冬}


  二十年春正月丙戍天德軍都防禦團練使豐州刺史李景畧卒初景畧嘗宴僚佐行酒者誤以醯進【醯呼西翻醋也}
判官京兆任迪簡以景畧性嚴恐行酒者得罪強飲之【任音壬強其兩翻}
歸而嘔血軍士聞之泣下及景畧卒軍士皆曰判官仁者欲奉以為帥【帥所類翻}
監軍抱置别室軍士發扃取之監軍以聞詔以代景畧 吐蕃贊普死其弟嗣立 【考異曰實録及舊傳皆云贊普以貞元十三年四月卒長子立一歲又卒次子嗣立韓愈順宗實録張薦傳云二十年贊普死遣薦弔贈新傳云十三年贊普死其子足之煎立二十年贊普死遣工部侍郎張薦弔祠其弟嗣立疑實録舊傳誤以是字為一字今從順宗録及新傳按字當作事}
 夏四月丙寅名陳許軍曰忠武 左金吾大將軍李昇雲將禁兵鎮咸陽疾病其子政諲【諲音因}
與虞侯上官望等謀效山東藩鎮使將士奏攝父事六月壬子昇雲卒甲寅詔追削昇雲官爵籍沒其家 昭義節度使李長榮薨上使中使以手詔授本軍大將但軍士所附者即授時大將來希皓為衆所服中使將以手詔付之希皓言于衆曰此軍取人合是希皓但作節度使不得【唐人多讀作如佐音}
若朝廷以一束草來希皓亦必敬事【言若束草為節度使亦必敬而事之來希皓之忠純如此而其後不復見於史必盧從史畏偪而去之也}
中使言面奉進止只令此軍取大將抜與節鉞朝廷不别除人希皓固辭兵馬使盧從史 【考異曰杜牧上李司徒書作押衙盧從史今從實録}
其位居四潛與監軍相結起出伍曰【出儔伍之中而言}
若來大夫不肯受詔從史請且句當此軍【句古翻當丁浪翻}
監軍曰盧中丞若如此此亦固合聖旨中使因探懷取詔以授之【探吐南翻}
從史捧詔再拜舞蹈希皓亟迴揮同列北面稱賀軍士畢集更無一言秋八月己未詔以從史為節度使 九月太子始得風疾不能言

  順宗至德弘道大聖大安孝皇帝

  【諱誦德宗長子按此宣宗大中三年追崇諡號也考之會要葬陵諡冊與此追崇諡號一同蓋會要所載初諡誤也}


  永貞元年【是年八月始改元永貞}
春正月辛未朔諸王親戚入賀德宗太子獨以疾不能來德宗涕泣悲歎由是得疾日益甚凡二十餘日中外不通莫知兩宫安否癸巳德宗崩【年六十四}
蒼猝召翰林學士鄭絪衛次公等至金鑾殿【絪音因程大昌雍録曰金鑾坡者龍首山之支隴隱起平地而坡陁靡迤者也其上有殿名曰金鑾殿旁有坡名曰金鑾坡又曰金鑾殿者在蓬萊山正西微南龍首山坡隴之北殿西有坡德宗即之以造東學士院以其在開元學士院之東也}
草遺詔宦官或曰禁中議所立尚未定衆莫敢對次公遽言曰太子雖有疾地居冢嫡中外屬心【屬之欲翻}
必不得已猶應立廣陵王【廣陵王純太子長子}
不然必大亂絪等從而和之【和胡卧翻}
議始定次公河東人也太子知人情憂疑紫衣麻鞋 【考異曰按祕喪則不應麻鞋發喪則不應紫衣蓋當時蒼猝偶著此服非祕喪也以未成服故不衣縗絰耳}
力疾出九仙門【雍錄曰九仙門在内西苑之東北角右神策軍右羽林軍右龍武軍列營于九仙門之西按閣本大明宫圖宫城西面右銀臺門又北為九仙門}
召見諸軍使人心粗安【粗坐五翻}
甲午宣遺詔于宣政殿 【考異曰德宗實録癸巳宣遺詔今從順宗實録}
太子縗服見百官【縗倉回翻}
丙申即皇帝位于太極殿【即位于西内前殿}
衛士尚疑之企足引領而望之【企去智翻}
曰真太子也乃喜而泣時順宗失音不能决事常居宫中施簾帷獨宦者李忠言昭容牛氏侍左右百官奏事自帷中可其奏自德宗大漸王伾先入稱詔召王叔文坐翰林中使决事伾以叔文意入言于忠言稱詔行下【下戶嫁翻}
外初無知者以杜佑攝冢宰二月癸卯上始朝百官于紫宸門【紫宸門紫宸殿門也長安志宣政殿北曰紫宸門門内有紫宸殿即内衙之正殿}
 己酉加義武節度使張茂昭同平章事 辛亥以吏部郎中韋執誼為尚書左丞同平章事王叔文欲掌國政首引執誼為相已用事于中與相唱和【和戶卧翻}
壬子李師古發兵屯西境以脅滑州時告哀使未至諸道義成牙將有自長安還得遺詔者節度使李元素以師古鄰道欲示無外【春秋公羊傳曰王者無外此唐人以化外待藩鎮故有此語}
遣使密以遺詔示之師古欲乘國喪侵噬鄰境乃集將士謂曰聖上萬福而元素忽傳遺詔是反也宜擊之遂杖元素使者發兵屯曹州 【考異曰舊韓愈傳云撰順宗實録繁簡不當穆宗文宗嘗詔史臣添改時愈壻李漢蔣係在顯位諸公難之而韋處厚竟别撰順宗實録三卷景祐中詔編次崇文總目順宗實録有七本皆五卷題曰韓愈等撰五本畧而二本詳編次者兩存之其中多異同今以詳畧為别此李師古脅滑州事詳本有而畧本無詳錄又云使衡密以其本示之師古不受杖衡幾死衡蓋使者之名而無姓又云遂以師至濮州伺候為變按韓愈撰韓弘碑云屯兵于曹今從之}
且告假道于汴【九域志曹州西北至滑州一百二十里汴州北至滑州界一百里東北至曹州界一百三里三州之界蓋犬牙相入}
宣武節度使韓弘使謂曰汝能越吾界而為盗邪有以相待無為空言元素告急弘使謂曰吾在此公安無恐或告翦棘夷道【翦芟截也夷平也}
兵且至矣請備之弘曰兵來不除道也不為之應師古詐窮變索【索蘇各翻索散也盡也言韓弘逆得師古之情其所設詭變索然散盡也}
且聞上即位乃罷兵元素表請自貶朝廷兩慰解之元素泌之族弟也【李泌歷事肅代德貞元中為相}
吳少誠以牛皮鞵材遺師古【鞵與鞋同遺唯季翻}
師古以鹽資少誠潛過宣武界事覺弘皆留輸之庫曰此于法不得以私相餽師古等皆憚之 辛酉詔數京兆尹道王實殘暴掊斂之罪【數所具翻掊蒲侯翻斂力贍翻}
貶通州長史【宋白曰通州漢宕渠縣地後漢分置宣漢縣}
市井讙呼皆䄂瓦礫遮道伺之實由間道獲免【讙許元翻礫郎擊翻間古莧翻}
 壬戍以殿中丞王伾為左散騎常侍依前翰林待詔蘇州司功王叔文為起居舍人翰林學士伾寢陋吳語【狀貌寢陋常操鄉音不能學華言}
上所䙝狎而叔文頗任事自許微知文義好言事【好呼到翻}
上以故稍敬之不得如伾出入無阻叔文入至翰林而伾入至柿林院【柿鉏里翻}
見李忠言牛昭容計事大抵叔文依伾伾依忠言忠言依牛昭容轉相交結每事先下翰林【下遐稼翻}
使叔文可否然後宣于中書韋執誼承而行之外黨則韓泰柳宗元等主采聼外事謀議唱和【和戶卧翻}
日夜汲汲如狂互相推奬曰伊曰周曰管曰葛【以伊尹周公管仲諸葛孔明互相比况}
僴然自得【僴下赧翻僴然勁忿貌}
謂天下無人榮辱進退生于造次【朱氏曰造次急遽苟且之時造七到翻}
惟其所欲不拘程式士大夫畏之道路以目【國語周厲王監謗國人莫敢言道路以目韋昭注曰不敢發言以目相眄而已}
素與往還者相次拔擢至一日除數人【除者除官也}
其黨或言曰某可為某官不過一二日輒己得之于是叔文及其黨十餘家之門晝夜車馬如市客候見叔文伾者至宿其坊中餅肆酒壚下【長安城中分為左右街畫為百有餘坊餅肆賣餅之家酒壚賣酒之處顔師古曰賣酒之處累土為壚以居酒瓮四邊隆起其一面高形如鍜壚故名壚耳}
一人得千錢乃容之伾尤闒茸【闒吐盍翻茸而隴翻闒茸獰劣也史炤曰顔師古曰闒茸猥賤也闒下也茸細毛貌謂非豪傑也}
專以納賄為事作大匱貯金帛【貯工呂翻}
夫婦寢其上【恐人盗之}
甲子上御丹鳳門赦天下諸色逋負一切蠲免【蠲除也}


  常貢之外悉罷進奉貞元之末政事為人患者如宫市五坊小兒之類悉罷之【宫市事見上卷貞元十三年五坊一曰鵰坊二曰鶻坊三曰鷂坊四曰鷹坊五曰狗坊小兒者給役五坊者也唐時給役者多呼為小兒如苑監小兒飛龍小兒五坊小兒是也五坊屬宣微院}
先是五坊小兒張捕鳥雀于閭里者皆為暴横【先悉薦翻横戶孟翻}
以取人錢物至有張羅網于門不許人出入者或張井上使不得汲者【汲汲水也}
近之輒曰汝驚供奉鳥雀即痛毆之【近其靳翻毆烏口翻擊也}
出錢物求謝乃去或相聚飲食于酒食之肆醉飽而去賣者或不知就索其直多被敺詈或時留蛇一囊為質【索山客翻被皮義翻質音致}
此蛇所以致鳥雀而捕之者今留付汝幸善飼之【飼與飤同祥吏翻}
勿令饑渇賣者愧謝求哀乃攜挈而去上在東宫皆知其弊故即位首禁之 乙丑罷鹽鐵使月進錢先是鹽鐵月進羨餘【羨弋線翻}
而經入益少【少詩沼翻}
至是罷之 三月辛未以王伾為翰林學士 德宗之末十年無赦羣臣以微過譴逐者皆不復叙用至是始得量移【復扶又翻量音良}
壬申追忠州别駕陸贄郴州别駕鄭餘慶杭州刺史韓臯道州刺史陽城赴京師【陸贄貶見上卷貞元十一年陽城貶見十四年鄭餘慶貶見十六年韓臯為京兆尹十四年貶撫州員外司馬未幾徙杭州刺史追猶召也}
贄之秉政也貶駕部員外郎李吉甫為明州長史【贄疑吉甫黨竇參故貶之}
既而徙忠州刺史贄昆弟門人咸以為憂至而吉甫忻然以宰相禮事之贄初猶慙懼後遂為深交吉甫栖筠之子【李栖筠事代宗以直聞}
韋臯在成都屢上表請以贄自代贄與陽城皆未聞追詔而卒【卒子恤翻}
 丙戌加杜佑度支及諸道鹽鐵轉運使以浙西觀察使李錡為鎮海節度使解其鹽鐵轉運使 【考異曰舊錡傳云德宗于潤州置鎮海軍新書方鎮表元和二年升浙西觀察使為鎮海軍節度使按實録八月辛酉詔曰頃年江淮租賦爰及榷税委在藩服使其平均太上君臨之初務從省便令使府歸在中朝然則云德宗元和者皆誤也}
錡雖失利權而得節旄故反謀亦未發 戊子名徐州軍曰武寧以張愔為節度使 加彰義節度使吳少誠同平章事 以王叔文為度支鹽鐵轉運副使先是叔文與其黨謀【先悉薦翻}
得國賦在手則可以結諸用事人取軍士心以固其權又懼驟使重權【度支鹽鐵轉運利權所在權莫重焉王叔文起于卑渫遽領使職自知其驟其心不安而懼使踈吏翻}
人心不服藉杜佑雅有會計之名【雅素也會古外翻}
位重而務自全易可制【易以豉翻}
故先令佑主其名而自除為副以專之叔文雖判兩使【度支一使鹽鐵轉運一使}
不以簿書為意日夜與其黨屏人竊語【屏必郢翻又卑正翻}
人莫測其所為以御史中丞武元衡為左庶子德宗之末叔文之黨多為御史元衡薄其為人待之莽鹵【莽莫補翻鹵即古翻莽鹵言不以為意也}
元衡為山陵儀仗使劉禹錫求為判官不許叔文以元衡在風憲欲使附已使其黨誘以權利【誘音酉}
元衡不從由是左遷元衡平一之孫也【武平一武載德之子武后時避事隱嵩山}
侍御史竇羣奏屯田員外郎劉禹錫挾邪亂政不宜在朝【唐屯田郎掌天下屯田及京文武職田諸司公廨錢以品給之朝直遥翻}
又嘗謁叔文揖之曰事固有不可知者叔文曰何謂也羣曰去歲李實怙恩挾貴氣蓋一時公當此時逡廵路旁乃江南一吏耳【叔文本蘇州司功故云然}
今公一旦復據其地【復扶又翻}
安知路旁無如公者乎其黨欲逐之韋執誼以羣素有彊直名止之 【考異曰舊劉禹錫傳曰羣即日罷官羣傳曰其黨議欲貶羣官韋執誼止之又曰叔文雖異其言竟不之用按順宗實録凡為伾文所排擯者無不載未嘗言羣罷官今從之}
 上疾久不愈時扶御殿羣臣瞻望而已莫有親奏對者中外危懼思早立太子而王叔文之黨欲專大權惡聞之【惡烏路翻下同}
宦官俱文珍劉光錡薛盈珍皆先朝任使舊人【朝直遥翻}
疾叔文忠言等朋黨專恣乃啓上召翰林學士鄭絪衛次公李程王涯入金鑾殿草立太子制時牛昭容輩以廣陵王淳英睿惡之絪不復請書紙為立嫡以長字呈上【復扶又翻下同長知丈翻}
上頷之癸巳立淳為太子更名純【更工衡翻}
程神符五世孫也【神符淮安王神通之弟}
賈耽以王叔文黨用事心惡之稱疾不出屢乞骸骨

  丁酉諸宰相會食中書故事宰相方食百寮無敢謁見者叔文至中書欲與執誼計事令直省通之【直省吏職也以直中書省故名}
直省以舊事告叔文怒叱直省直省懼入白執誼逡廵慙赧【赧奴版翻慙而面赤也}
竟起迎叔文就其閤語良久杜佑高郢鄭珣瑜皆停筯以待有報者云叔文索飯【索山客翻}
韋相公已與之同食閤中矣佑郢心知不可畏叔文執誼莫敢出言珣瑜獨歎曰吾豈可復居此位顧左右取馬徑歸遂不起二相皆天下重望【二相謂賈耽鄭珣瑜}
相次歸卧叔文執誼益無所顧忌遠近大懼【史甚言其事}
 夏四月壬寅立皇弟諤為欽王誠為珍王子經為郯王緯為均王縱為漵王紓為莒王綢為密王總為郇王約為邵王結為宋王緗為集王絿為冀王綺為和王絢為衡王纁為會王綰為福王紘為撫王緄為岳王紳為袁王綸為桂王繟為翼王【紓式居翻綢直留翻緗思良翻絿音求絢許縣翻纁許云翻緄古本翻繟充善翻自經以下皆皇子也史提子字以别二弟此所封諸王或以古國名然多以當時州名}
 乙巳上御宣政殿冊太子百官睹太子儀表退皆相賀至有感泣者中外大喜而王叔文獨有憂色口不敢言但吟杜甫題諸葛亮祠堂詩曰出師未捷身先死長使英雄淚滿襟聞者哂之【哂矢忍翻笑不壞顔為哂}
先是太常卿杜黄裳為裴延齡所惡留滯臺閣十年不遷【杜黄裳自佐朔方軍入為侍御史十年不遷先悉薦翻惡烏路翻}
及其壻韋執誼為相始遷太常卿黄裳勸執誼帥羣臣請太子監國【帥讀曰率}
執誼驚曰丈人甫得一官奈何啓口議禁中事黄裳勃然曰黄裳受恩三朝【三朝謂肅代德也}
豈得以一官相買乎拂衣起出戊申以給事中陸淳為太子侍讀仍更名質【避太子名也}
韋執誼自以專權恐太子不悦故以質為侍讀使潛伺太子意且解之【伺相吏翻}
及質發言太子怒曰陛下令先生為寡人講經義耳【為于偽翻}
何為預它事質惶懼而出 五月辛未以右金吾大將軍范希朝為左右神策京西諸城鎮行營節度使甲戍以度支郎中韓泰為其行軍司馬王叔文自知為内外所憎疾欲奪取宦官兵權以自固藉希朝老將使主其名而實以泰專其事【此與用杜佑掌利權同一計數也}
人情不測其所為益疑懼 辛卯以王叔文為戶部侍郎依前充度支鹽鐵轉運副使俱文珍等惡其專權削去翰林之職【惡烏路翻去羌呂翻}
叔文見制書大驚謂人曰叔文日時至此商量公事【日時猶云日日時時也約言之耳}
若不得此院職事則無因而至矣【此院謂翰林學士院也}
王伾即為疏請【為于偽翻}
不從再疏乃許三五日一入翰林去學士名叔文始懼 六月己亥貶宣歙廵官羊士諤為汀州寧化尉【唐制節度觀察其屬皆有廵官開元二十六年開山洞置黄連縣天寶元年更名寧化九域志在州東北一百八十里}
士諤以公事至長安遇叔文用事公言其非叔文聞之怒欲下詔斬之執誼不可則令杖煞之【煞與殺同}
執誼又以為不可遂貶焉由是叔文始大惡執誼【惡烏路翻}
往來二人門下者皆懼先時劉闢以劍南支度副使將韋臯之意于叔文【唐六典凡天下邊軍皆有支度之使以計軍資糧仗之用將奉也行也先悉薦翻}
求都領劍南三川【劍南東川西川及山南西道為三川}
謂叔文曰太尉使闢致微誠于公【太尉謂韋臯}
若與某三川當以死相助若不與亦當有以相酬叔文怒【以闢以言脅之故怒}
亦將斬之執誼固執不可闢尚遊長安未去聞貶士諤遂逃歸執誼初為叔文所引用深附之既得位欲掩其迹且廹于公議故時時為異同輒使人謝叔文曰非敢負約乃欲曲成兄事耳叔文詬怒不之信【詬呼漏翻又古候翻}
遂成仇怨 癸丑韋臯上表以為陛下哀毁成疾重勞萬機【重直用翻}
故久而未安請權令皇太子親監庶政【監古衘翻}
候皇躬痊愈復歸春宫【東宫謂之春宫}
臣位兼將相今之所陳乃其職分【分扶問翻}
又上太子牋以為聖上遠法高宗亮隂不言委政臣下而所附非人王叔文王伾李忠言之徒輒當重任賞罰任情墮紀紊綱【墮讀曰隳紊亡運翻}
散府庫之積以賂權門樹置心腹徧于貴位潛結左右憂在蕭墻竊恐傾太宗盛業危殿下家邦願殿下即日奏聞斥逐羣小使政出人主則四方獲安臯自恃重臣遠處西蜀度王叔文不能動揺遂極言其姦【處昌呂翻度徒洛翻}
俄而荆南節度使裴均河東節度使嚴綬牋表繼至意與臯同 【考異曰實録畧本云尋而裴垍嚴綬表繼至悉與臯同又云外有臯裴垍嚴綬等牋表詳本裴垍皆作裴均按裴垍時為考功員外郎裴均為荆南節度使今從詳本}
中外皆倚以為援而邪黨震懼均光庭之曾孫也【裴光庭相玄宗}
 王叔文既以范希朝韓泰主京西神策軍諸宦者尚未寤會邊上諸將各以狀辭中尉且言方屬希朝宦者始寤兵柄為叔文等所奪乃大怒曰從其謀吾屬必死其手密令其使歸告諸將曰無以兵屬人希朝至奉天諸將無至者韓泰馳歸白之叔文計無所出唯曰奈何奈何無幾【幾居豈翻無幾言無多時也}
其母病甚丙辰叔文盛具酒饌與諸學士及李忠言俱文珍劉光琦等飲于翰林【饌雛戀翻又雛晥翻}
叔文言曰叔文母病以身任國事之故不得親醫藥今將求假歸侍【假古暇翻求假請告也}
叔文比竭心力不避危難皆為朝廷之恩【比毗至翻難乃旦翻為于偽翻}
一旦去歸百謗交至誰肯見察以一言相助乎文珍隨其語輒折之【折之舌翻}
叔文不能對但引滿相勸酒數行而罷丁巳叔文以母喪去位 【考異曰實録詳本曰叔文母將死前一日叔文以五十人擔酒饌入翰林讌李忠言劉光琦俱文珍及諸學士等中飲叔文執盞云云又曰羊士諤毁叔文叔文將杖殺之而韋執誼懦不敢劉闢以韋臯廹脅叔文求三川叔文平生不識闢叔文今日名位何如而闢欲前執叔文手豈非凶人邪叔文時已令掃木場將集衆斬之執誼又執不可每念失此兩賊令人不快又自陳判度支已來所為國家興利除害出若干錢以為功能俱文珍隨語折之叔文無以對命滿酌雙巵對飲酒數行而罷方飲時有暫起至廳側者聞叔文從人相謂曰母死已臰不欲棺歛方與人飲酒不知欲何所為歸之明日而其母死或傳母死數日乃發喪國史補曰王叔文以度支使設饌于翰林大宴諸閹䄂金以贈明日又至揚言聖人適于苑中射兎上下馬如飛敢有異議者腰斬其日丁母憂今從二本實録}
 秋七月丙子加李師古檢校侍中 王叔文既有母喪韋執誼益不用其語叔文怒與其黨日夜謀起復必先斬執誼而盡誅不附己者聞者忷懼自叔文歸第王伾失據日詣宦官及杜佑請起叔文為相【杜佑時為首相故請之}
且總北軍既不獲則請以為威遠軍使平章事【據舊郭子儀傳肅宗上元元年以子儀為諸道兵馬都統令帥英武威遠等禁軍及諸鎮之師取范陽既而為魚朝恩所沮不行則威遠軍肅宗置也至德宗時以左右威遠營隸鴻臚賈耽以鴻臚卿兼威遠軍使至元和二年勅左右威遠營置來已久著在國章其英武軍並合并入左右威遠營其後遂以宦官為使不復隸鴻臚宋白曰左右威遠營本屬鴻臚寺建中元年七月隸金吾}
又不得其黨皆憂悸不自保【悸其季翻}
是日伾坐翰林中疏三上不報【上時掌翻}
知事不濟行且卧至夜忽叫曰伾中風矣【中竹仲翻}
明日遂輿歸不出己丑以倉部郎中判度支案陳諫為河中少尹【唐諸都各置尹一人少尹二人從四品下掌貳府州之事歲終則更次入計}
伾叔文之黨至是始去癸巳横海軍節度使程懷信薨以其子副使執㳟為留後 【考異曰舊傳曰程懷信死懷直子執㳟知留後事乃遣懷直歸滄州十六年卒執㳟代襲父位朝廷因而授之按懷信逐懷直而奪其位安肯以懷直之子知留後又德宗實録俱無是事順宗實録畧本亦無蓋舊傳誤也惟詳本永貞元年七月癸巳横海軍節度使程懷信卒以其子副使執㳟為横海軍節度使路隋憲宗實録元和元年五月丙子以横海留後程執㳟為節度使蓋順録留後字誤為使字耳}
 乙未制以積疹未復【疹丑刃翻病也}
其軍國政事權令皇太子純句當【句古翻當丁浪翻}
時内外共疾王叔文黨與專恣上亦惡之【惡烏路翻}
俱文珍屢啓上請令太子監國【監古御翻}
上固厭倦萬幾遂許之又以太常卿杜黄裳為門下侍郎左金吾大將軍袁滋為中書侍郎並同平章事俱文珍等以其舊臣故引用之【杜黄裳代宗時已佐朔方軍袁滋建中初已位于朝故以為舊臣}
又以鄭珣瑜為吏部尚書高郢為刑部尚書並罷政事太子見百官于東朝堂【唐六典大明宫含元殿夾殿有兩閣左曰翔鸞翔鸞閣下為東朝堂右曰棲鳳棲鳳閣下為西朝堂朝直遥翻}
百官拜賀太子涕泣不荅拜八月庚子制令太子即皇帝位朕稱太上皇制敕稱誥辛丑太上皇徙居興慶宫誥改元永貞立良娣王氏為太上皇后后憲宗之母也壬寅貶王伾開州司馬王叔文渝州司戶【舊志開州京師南一千四百六十里渝州京師西南二千七百四十八里}
伾尋病死貶所明年賜叔文死乙巳憲宗即位於宣政殿【德宗大行在殯上皇在興慶宫不敢於前殿即位}
 丙午昇平公主獻女口五十【公主郭妃母也}
上曰上皇不受獻朕何敢違遂却之庚戍荆南獻毛龜二上曰朕所寶惟賢嘉禾神芝皆虛美耳所以春秋不書祥瑞自今凡有嘉瑞但凖令申有司【禮部掌祥瑞}
勿復以聞【復扶又翻}
及珍禽奇獸皆毋得獻 癸丑西川節度使南康忠武王韋臯薨臯在蜀二十一年【德宗貞元元年韋臯代張延賞鎮蜀}
重加賦歛【歛力贍翻}
豐貢獻以結主恩厚給賜以撫士卒士卒婚嫁死喪皆供其資費以是得久安其位而士卒樂為之用【樂音洛}
服南詔摧吐蕃幕僚歲久官崇者則為刺史已復還幕府【復扶又翻}
終不使還朝恐泄其所為故也【朝直遥翻下同}
府庫既實時寛其民三年一復租賦【復方目翻除也}
蜀人服其智謀而畏其威至今畫像以為土神家家祀之支度副使劉闢自為留後 朗州武陵龍陽江漲流萬餘家【武陵漢臨沅縣地隋省臨沅置武陵縣唐帶朗州龍陽縣吳置九域志在州東南八十里}
 壬午奉義節度使伊慎入朝【自安州入朝}
 辛卯夏綏節度使韓全義入朝全義敗於溵水而還不朝覲而去【事見上卷貞元十六年及上十七年}
上在藩邸聞其事而惡之【惡烏路翻}
全義懼乃請入朝劉闢使諸將表求節鉞朝廷不許己未以袁滋為劍南東西川山南西道安撫大使 度支奏裴延齡所置别庫皆減正庫之物别貯之【貯丁呂翻裴延齡事見上卷貞元十年}
請併歸正庫從之 辛酉遣度支鹽鐵轉運副使潘孟陽宣慰江淮行視租賦榷税利害因察官吏否臧百姓疾苦【行下孟翻否音鄙}
 癸亥以尚書左丞鄭餘慶同平章事 九月戊辰禮儀使奏曾太皇太后沈氏歲月滋深迎訪理絶【迎訪事始見二百二十六卷德宗建中元年}
按晉庾蔚之議尋求三年之外俟中壽而服之【晉荀組云二親䧟沒萬無一冀者宜使依王法隨例行喪庾蔚之云二親為戎狄所破存亡未可知者宜盡尋求之理尋求之理絶三年之外便宜婚宦胤嗣不可絶王政不可廢故也猶宜以哀素自居不豫吉慶之事俟中夀而服之也若境内賊亂清平肆眚之後尋覔無蹤跡者便宜制服莊子曰人生上壽一百中壽八十下壽六十蔚紆勿翻}
伏請以大行皇帝啓攅宫日【記檀弓曰天子之殯也菆塗龍輴以椁加斧於椁上畢塗屋鄭玄注曰天子之殯居棺以龍輴攅木題湊象椁四注如屋以覆之盡塗之及葬而啓之攅才官翻}
皇帝帥百官舉哀【帥讀曰率}
即以其日為忌從之 壬申監修國史韋執誼奏始令史官撰日歷【葉伯益曰唐永貞初韋執誼奏修撰私家紀緑非是望令各撰日歷月終館中撰定從之此日歷之所從起也}
 己卯貶神策行軍司馬韓泰為撫州刺史司封郎中韓曄為池州刺史禮部員外郎柳宗元為邵州刺史屯田員外郎劉禹錫為連州刺史【皆王伾王叔文之黨也舊志撫州京師東南三千三百一十二里連州京師南三千六百六十五里}
 冬十月丁酉右僕射同平章事賈耽薨 戊戍以中書侍郎同平章事袁滋同平章事充西川節度使徵劉闢為給事中 舒王誼薨 太常議曾太皇太后諡曰睿真皇后 山人羅令則自長安如普潤矯稱太上皇誥徵兵於秦州刺史劉澭且說澭以廢立【說式芮翻}
澭執送長安并其黨杖殺之己酉葬神武孝文皇帝于崇陵【新舊帝紀作神武聖文皇帝當從之崇陵在京兆雲陽縣北十五里嵯峨山}
廟號德宗 十一月己巳祔睿真皇后德宗皇帝主于太廟禮儀使杜黄裳等議以為國家法周制太祖猶后稷高祖猶文王太宗猶武王皆不遷高宗在三昭三穆之外請遷主于西夾室從之 壬申貶中書侍郎同平章事韋執誼為崖州司馬執誼以嘗與王叔文異同且杜黄裳壻故獨後貶然叔文敗執誼亦自失形勢知禍將至雖尚為相常不自得奄奄無氣【奄衣亷翻奄奄言氣息微也}
聞人行聲輒惶悸失色以至於貶【悸其季翻}
 戊寅以韓全義為太子少保致仕 劉闢不受徵阻兵自守袁滋畏其彊不敢進上怒貶滋為吉州刺史 復以右庶子武元衡為御史中丞【是年三月武元衡自御史中丞左遷右庶子王叔文等惡之也}
 朝議謂王叔文之黨或自員外郎出為刺史貶之太輕【朝直遥翻}
己卯再貶韓泰為䖍州司馬韓曄為饒州司馬柳宗元為永州司馬劉禹錫為朗州司馬【舊志䖍州京師東南四千十七里饒州三千二百六十三里 永州京師南三千二百七十四里朗州二千一百五十九里}
又貶河中少尹陳諫為台州司馬和州刺史凌凖為連州司馬岳州刺史程异為郴州司馬【台州京師東南四千一百七十七里和州二千六百八十三里岳州二千二百三十七里}
 回鶻懷信可汗卒遣鴻臚少卿孫杲臨弔冊其嗣為騰里野合俱録毗伽可汗【自懷信立回鶻藥葛羅氏絶矣此後史皆書冊其嗣以表懷信子孫也}
 十二月甲辰加山南東道節度使于頔同平章事 以奉義節度使伊慎為右僕射 己酉以給事中劉闢為西川節度副使知節度事【西川節度使領益彭蜀漢眉嘉資簡維茂黎雅松抉文龍戎翼卭嶲姚柘㳟當悉奉疊靜等州治成都然西邊諸州多淪於異域矣}
上以初嗣位力未能討故也右諫議大夫韋丹上疏以為今釋闢不誅則朝廷可以指臂而使者惟兩京耳此外誰不為叛上善其言壬子以丹為東川節度使丹津之五世孫也【津韋孝寛之子也}
 辛酉百官請上上皇尊號曰應乾聖壽太上皇上尊號曰文武大聖孝德皇帝上許上上皇尊號而自辭不受 壬戌以翰林學士鄭絪為中書侍郎同平章事 以刑部郎中杜兼為蘇州刺史兼辭行上書稱李錡且反必奏族臣上然之留為吏部郎中

  資治通鑑卷二百三十六


    


 


 



 

 
  







 


  
  
 
 
 


  

 















	
	









































 
  



















 





 












  
  
  

 





