<!DOCTYPE html PUBLIC "-//W3C//DTD XHTML 1.0 Transitional//EN" "http://www.w3.org/TR/xhtml1/DTD/xhtml1-transitional.dtd">
<html xmlns="http://www.w3.org/1999/xhtml">
<head>
<meta http-equiv="Content-Type" content="text/html; charset=utf-8" />
<meta http-equiv="X-UA-Compatible" content="IE=Edge,chrome=1">
<title>資治通鑒_181-資治通鑑卷一百八十_181-資治通鑑卷一百八十</title>
<meta name="Keywords" content="資治通鑒_181-資治通鑑卷一百八十_181-資治通鑑卷一百八十">
<meta name="Description" content="資治通鑒_181-資治通鑑卷一百八十_181-資治通鑑卷一百八十">
<meta http-equiv="Cache-Control" content="no-transform" />
<meta http-equiv="Cache-Control" content="no-siteapp" />
<link href="/img/style.css" rel="stylesheet" type="text/css" />
<script src="/img/m.js?2020"></script> 
</head>
<body>
 <div class="ClassNavi">
<a  href="/24shi/">二十四史</a> | <a href="/SiKuQuanShu/">四库全书</a> | <a href="http://www.guoxuedashi.com/gjtsjc/"><font  color="#FF0000">古今图书集成</font></a> | <a href="/renwu/">历史人物</a> | <a href="/ShuoWenJieZi/"><font  color="#FF0000">说文解字</a></font> | <a href="/chengyu/">成语词典</a> | <a  target="_blank"  href="http://www.guoxuedashi.com/jgwhj/"><font  color="#FF0000">甲骨文合集</font></a> | <a href="/yzjwjc/"><font  color="#FF0000">殷周金文集成</font></a> | <a href="/xiangxingzi/"><font color="#0000FF">象形字典</font></a> | <a href="/13jing/"><font  color="#FF0000">十三经索引</font></a> | <a href="/zixing/"><font  color="#FF0000">字体转换器</font></a> | <a href="/zidian/xz/"><font color="#0000FF">篆书识别</font></a> | <a href="/jinfanyi/">近义反义词</a> | <a href="/duilian/">对联大全</a> | <a href="/jiapu/"><font  color="#0000FF">家谱族谱查询</font></a> | <a href="http://www.guoxuemi.com/hafo/" target="_blank" ><font color="#FF0000">哈佛古籍</font></a> 
</div>

 <!-- 头部导航开始 -->
<div class="w1180 head clearfix">
  <div class="head_logo l"><a title="国学大师官网" href="http://www.guoxuedashi.com" target="_blank"></a></div>
  <div class="head_sr l">
  <div id="head1">
  
  <a href="http://www.guoxuedashi.com/zidian/bujian/" target="_blank" ><img src="http://www.guoxuedashi.com/img/top1.gif" width="88" height="60" border="0" title="部件查字,支持20万汉字"></a>


<a href="http://www.guoxuedashi.com/help/yingpan.php" target="_blank"><img src="http://www.guoxuedashi.com/img/top230.gif" width="600" height="62" border="0" ></a>


  </div>
  <div id="head3"><a href="javascript:" onClick="javascript:window.external.AddFavorite(window.location.href,document.title);">添加收藏</a>
  <br><a href="/help/setie.php">搜索引擎</a>
  <br><a href="/help/zanzhu.php">赞助本站</a></div>
  <div id="head2">
 <a href="http://www.guoxuemi.com/" target="_blank"><img src="http://www.guoxuedashi.com/img/guoxuemi.gif" width="95" height="62" border="0" style="margin-left:2px;" title="国学迷"></a>
  

  </div>
</div>
  <div class="clear"></div>
  <div class="head_nav">
  <p><a href="/">首页</a> | <a href="/ShuKu/">国学书库</a> | <a href="/guji/">影印古籍</a> | <a href="/shici/">诗词宝典</a> | <a   href="/SiKuQuanShu/gxjx.php">精选</a> <b>|</b> <a href="/zidian/">汉语字典</a> | <a href="/hydcd/">汉语词典</a> | <a href="http://www.guoxuedashi.com/zidian/bujian/"><font  color="#CC0066">部件查字</font></a> | <a href="http://www.sfds.cn/"><font  color="#CC0066">书法大师</font></a> | <a href="/jgwhj/">甲骨文</a> <b>|</b> <a href="/b/4/"><font  color="#CC0066">解密</font></a> | <a href="/renwu/">历史人物</a> | <a href="/diangu/">历史典故</a> | <a href="/xingshi/">姓氏</a> | <a href="/minzu/">民族</a> <b>|</b> <a href="/mz/"><font  color="#CC0066">世界名著</font></a> | <a href="/download/">软件下载</a>
</p>
<p><a href="/b/"><font  color="#CC0066">历史</font></a> | <a href="http://skqs.guoxuedashi.com/" target="_blank">四库全书</a> |  <a href="http://www.guoxuedashi.com/search/" target="_blank"><font  color="#CC0066">全文检索</font></a> | <a href="http://www.guoxuedashi.com/shumu/">古籍书目</a> | <a   href="/24shi/">正史</a> <b>|</b> <a href="/chengyu/">成语词典</a> | <a href="/kangxi/" title="康熙字典">康熙字典</a> | <a href="/ShuoWenJieZi/">说文解字</a> | <a href="/zixing/yanbian/">字形演变</a> | <a href="/yzjwjc/">金 文</a> <b>|</b>  <a href="/shijian/nian-hao/">年号</a> | <a href="/diming/">历史地名</a> | <a href="/shijian/">历史事件</a> | <a href="/guanzhi/">官职</a> | <a href="/lishi/">知识</a> <b>|</b> <a href="/zhongyi/">中医中药</a> | <a href="http://www.guoxuedashi.com/forum/">留言反馈</a>
</p>
  </div>
</div>
<!-- 头部导航END --> 
<!-- 内容区开始 --> 
<div class="w1180 clearfix">
  <div class="info l">
   
<div class="clearfix" style="background:#f5faff;">
<script src='http://www.guoxuedashi.com/img/headersou.js'></script>

</div>
  <div class="info_tree"><a href="http://www.guoxuedashi.com">首页</a> > <a href="/SiKuQuanShu/fanti/">四库全书</a>
 > <h1>资治通鉴</h1> <!--         下载:【右键另存为】即可 --></div>
  <div class="info_content zj clearfix">
  
<div class="info_txt clearfix" id="show">
<center style="font-size:24px;">181-資治通鑑卷一百八十</center>
    資治通鑑卷一百八十  宋 司馬光 撰<br />
<br />
  胡三省音註<br />
<br />
  隋紀四【起閼逢困敦盡彊圉單閼凡四年】<br />
<br />
  高祖文皇帝下<br />
<br />
  仁壽四年春正月丙午赦天下 帝將避暑於仁壽宫術士章仇太翼固諫不聽太翼曰是行恐鑾輿不返帝大怒繫之長安獄期還而斬之甲子幸仁壽宫乙丑詔賞賜支度事無巨細並付皇太子夏四月乙卯帝不豫六月庚申赦天下秋七月甲辰上疾甚卧與百僚辭訣並握手歔欷【歔音虛欷音希又許既翻】命太子赦章仇太翼丁未崩於大寶殿【年六十四】高祖性嚴重令行禁止每旦聽朝日昃忘倦【朝直遥翻昃阻力翻日中則昃】雖嗇於財至於賞賜有功即無所愛將士戰没必加優賞仍遣使者勞問其家【將即亮翻使疏吏翻勞力到翻】愛養百姓勸課農桑輕徭薄賦其自奉養務為儉素乘輿御物故弊者隨宜補用【乘繩證翻】自非享宴所食不過一肉後宫皆服澣濯之衣天下化之開皇仁壽之間丈夫率衣絹布【衣於既翻】不服綾綺装帶不過銅鐵骨角無金玉之飾故衣食滋殖倉庫盈溢受禪之初民戶不滿四百萬末年踰八百九十萬【此以開皇初元戶口之數比較仁夀末年大業初之數而言之也按周之平齊得戶三百三萬而隋受周禪戶不滿四百萬則周氏初有關中西并巴蜀南兼江漢見戶不滿百萬也陳氏之亡戶六十萬大約隋氏混壹天下見戶未及五百萬及其盛也蓋幾倍之滋音兹殖音植禪音墠】獨冀州已一百萬戶【隋以信都郡為冀州此以古冀州之域言之也然禹之冀州兼有幽并營三州地其界比它州為最大其後以天文畫埜分州自胃七度至畢十一度為大梁冀州分隋志以信都清河魏汲河内長平上黨河東絳文城臨汾龍泉西河離石鴈門馬邑定襄樓煩太原襄國武安趙恒山博陵河間涿上谷漁陽北平安樂遼西等郡為冀州則其地亦兼有幽并營三州地故其戶最多】然猜忌苛察信受讒言功臣故舊無始終保全者乃至子弟皆如仇敵此其所短也【此上總論文帝平生】初文獻皇后既崩【獨孤后崩謚文獻見上卷二年】宣華夫人陳氏容華夫人蔡氏皆有寵陳氏陳高宗之女【陳宣帝廟號高宗】蔡氏丹陽人也【丹陽郡時置蔣州】上寢疾於仁壽宫尚書左僕射楊素兵部尚書柳述黃門侍郎元巖【隋制門下省納言二人給事黃門侍郎四人其位任重矣此又一元巖前蜀王秀長史之元巖封平昌郡公此元巖封龍涸縣公見隋書列女華陽王楷妃傳】皆入閤侍疾召皇太子入居大寶殿太子慮上有不諱須預防擬【防禦也隄備也擬凖也凖擬揣度以待之也】手自為書封出問素素條錄事狀以報太子宫人誤送上所上覽而大恚【恚於避翻】陳夫人平旦出更衣【更工衡翻】為太子所逼拒之得免歸於上所上怪其神色有異問其故夫人泫然曰太子無禮上恚抵床曰畜生何足付大事【抵觸也今人詈人猶曰畜生言其無識無禮若馬牛犬豕然待畜養而生者也泫戶畎翻】獨孤誤我【事見上卷開皇二十年】乃呼柳述元巖曰召我兒述等將呼太子上曰勇也述巖出閤為勑書【儲嗣之重廢置之間輕易如此烏得不君臣皆敗乎】楊素聞之以白太子矯詔執述巖繫大理獄追東宫兵士帖上臺宿衛【帖裨也帝之猜防太子勇也屏去東宫宿衛之勇健者知出蘇孝慈而不知備張衡之入寢殿也悕矣】門禁出入並取宇文述郭衍節度令右庶子張衡入寢殿侍疾盡遣後宫出就别室俄而上崩故中外頗有異論【此上叙帝所以見弑考異曰趙毅大業略記曰高祖在仁夀宫病甚追帝侍疾而高祖美人尤嬖幸者唯陳蔡二人而已帝乃召蔡於别室既還面傷而髪亂高祖問之蔡泣曰皇太子為非禮高祖大怒齧指出血召兵部尚書柳述黃門侍郎元巖等令發詔追庶人勇即令廢立帝事迫召左僕射楊素左庶子張衡進毒藥帝簡驍健官奴三十人皆服婦人之服衣下置仗立於門巷之間以為之衛素等既入而高祖暴崩馬總通歷曰上有疾於仁壽殿與百僚辭訣並握手歔欷是時唯太子及陳宣華夫人侍疾太子無禮宣華訴之帝怒曰死狗那可付後事遽令召勇楊素秘不宣乃屏左右令張衡入拉帝血濺御屏寃痛之聲聞于外今從隋書】陳夫人與後宫聞變相顧戰栗失色晡後太子遣使者齎小金合帖紙於際親署封字以賜夫人夫人見之惶懼以為鴆毒不敢發使者促之乃發【使疏吏翻下同】合中有同心結數枚宫人咸悦相謂曰得免死矣陳氏恚而却坐不肯致謝諸宫人共逼之乃拜使者其夜太子烝焉【杜預曰上淫曰烝】乙卯發喪 【考異曰大業略記曰十八日發喪杜寶大業雜記曰甲戌文帝崩辛巳發喪壬午煬帝即位案長歷是月乙未朔乙卯二十一日也無甲戌辛巳壬午日今從隋書】太子即皇帝位會伊州刺史楊約來朝【楊約出刺伊州見上卷二年朝直遥翻】太子遣約入長安易留守者矯稱高祖之詔賜故太子勇死縊殺之【縊於賜翻】然後陳兵集衆發高祖凶問煬帝聞之【書煬帝以别大行】曰令兄之弟果堪大任追封勇為房陵王不為置嗣【房陵郡王隋志房陵郡光遷縣舊曰房陵置新城郡梁末置岐州後周郡縣並改為光遷大業初置房陵郡 考異曰大業略記云庶人勇八男亦隂加酖害恐其為厲皆倒埋之按隋書北史皆云煬帝踐極儼常從行卒于道實酖之也諸弟分徙嶺表仍勑在所皆殺焉今從之按通鑑下文大業三年殺儼及七弟】八月丁卯梓宫至自仁壽宫丙子殯于大興前殿【大興前殿大興宫正殿也】柳述元巖並除名述徙龍川巖徙南海【隋志龍川郡平陳置循州南海郡舊置廣州】帝令蘭陵公主與述離絶欲改嫁之公主以死自誓不復朝謁【復扶又翻朝直遥翻】上表請與述同徙帝大怒公主憂憤而卒臨終上表請葬於柳氏帝愈怒竟不哭葬送甚薄【上時掌翻卒子恤翻】 太史令袁充奏言皇帝即位與堯受命年合諷百官表賀禮部侍郎許善心議以為國哀甫爾不宜稱賀左衛大將軍宇文述素惡善心【宇文述自左衛率遷左衛大將軍豈特以舊恩哉既以醻功且親之以自衛也惡烏路翻】諷御史劾之【劾戶槩翻又戶得翻】左遷給事郎降品二等 漢王諒有寵於高祖為并州總管【開皇十七年漢王諒代秦王俊為并州總管】自山以東至于滄海南距黃河五十二州皆隸焉特許以便宜從事不拘律令諒自以所居天下精兵處見太子勇以讒廢【事見上卷開皇二十年】居常怏怏【怏於兩翻】及蜀王秀得罪【見上卷二年】尤不自安隂蓄異圖言於高祖以突厥方彊【厥九勿翻】宜修武備於是大發工役繕治器械【治直之翻】招集亡命左右私人殆將數萬突厥嘗寇邊高祖使諒禦之為突厥所敗【敗補邁翻】其所領將帥坐除解者八十餘人【將即亮翻帥所類翻除除名也解解官也】皆配防嶺表諒以其宿舊奏請留之高祖怒曰爾為藩王惟當敬依朝命【朝直遥翻】何得私論宿舊廢國家憲法邪嗟乎小子爾一旦無我或欲妄動彼取爾如籠内雞雛耳何用腹心為王頍者僧辯之子【王僧辯事梁有平侯景之功為陳覇先所殺頍丘弭翻】倜儻好奇略【倜它狄翻好呼到翻】為諒諮議參軍【隋制諸王府諮議參軍在長史司馬之下掾屬之上也】蕭摩訶陳氏舊將【將即亮翻】二人俱不得志每鬱鬱思亂皆為諒所親善贊成其隂謀會熒惑守東井【熒惑罰星東井秦分】儀曹鄴人傅奕曉星歷【按隋制王府諸曹無儀曹蓋不在諸參軍之數鄴縣屬魏郡】諒問之曰是何祥也對曰天上東井黃道所經【晉志東井八星天之南門黃道所經】熒惑過之乃其常理若入地上井則可怪耳【奕知諒有異圖詭對以自免於禍】諒不悦及高祖崩煬帝遣車騎將軍屈突通以高祖璽書徵之【騎奇寄翻屈區勿翻璽斯氏翻】先是高祖與諒密約若璽書召汝勑字傍别加一點【高歡與侯景亦有此約而皆以階亂先悉薦翻】又與玉麟符合者【開皇七年頒青龍符於東方總管刺史西方以騶虞南方以朱雀北方以玄武是後三子分居方面并楊益三總管統屬甚廣故為玉麟符漢王諒既敗惟留守東西兩都用玉麟符至唐猶然】當就徵及發書無驗諒知有變詰通【詰去吉翻】通占對不屈乃遣歸長安諒遂發兵反總管司馬安定皇甫誕切諫【安定郡涇州】諒不納誕流涕曰竊料大王兵資非京師之敵加以君臣位定逆順勢殊士馬雖精難以取勝一旦陷身叛逆絓於刑書【絓戶掛翻】雖欲為布衣不可得也諒怒囚之嵐州刺史喬鍾葵將赴諒【嵐州樓煩之地也按隋志大業四年方置樓煩郡管下秀容縣舊置肆州開皇十八年置忻州大業初廢又按唐志樓煩郡平劉武周置東會州武德六年改嵐州而義寧元年復分秀容置忻州喬鍾葵者既為嵐州刺史而隋志不載嵐州建置當考嵐盧含翻宋白曰後魏置嵐州因岢嵐山為名】其司馬京兆陶模拒之曰漢王所圖不軌公荷國厚恩【荷下可翻】當竭誠効命豈得身為厲階乎鍾葵失色曰司馬反邪臨之以兵辭氣不橈【邪音耶撓奴敎翻屈也】鍾葵義而釋之軍吏曰若不斬模無以壓衆心乃囚之於是從諒反者凡十九州王頍說諒曰王所部將吏家屬盡在關西【說輪芮翻將即亮翻此關西謂蒲津關以西】若用此等則宜長驅深入直據京都所謂疾雷不及掩耳【淮南子之言】若但欲割據舊齊之地【南距大河北盡燕代皆高齊之地也】宜任東人諒不能决乃兼用二策唱言楊素反將誅之【諒若如宋武陵王聲元凶之罪而舉兵天下其誰能敵之】總管府兵曹聞喜裴文安【兵曹兵曹參軍也聞喜縣屬絳州】說諒曰井陘以西在王掌握之内山東士馬亦為我有宜悉發之分遣羸兵屯守要害仍命隨方略地帥其精鋭直入蒲津【同州朝邑縣有蒲津關度河東即蒲州城陘音刑羸倫為翻帥讀曰率】文安請為前鋒王以大軍繼後風行雷擊頓於霸上【自武關入則頓於霸上自蒲津入豈須頓於霸上蓋欲乘高以臨長安耳】咸陽以東可指麾而定京師震擾兵不暇集上下相疑羣情離駭我陳兵號令誰敢不從旬日之問事可定矣 【考異曰大業略記云司兵參軍裴文安說諒曰今梓宫尚在仁壽宫比其徵兵動移旬月今若簡驍勇萬騎令文安督領不淹十五日徑據長安其在京被黜停私之徒並擢授高位付以心膂共守京城則咸陽以東府縣非彼之有然後大王總兵鼓行而西聲勢一接天下可指麾而定也諒不從大業雜記云文安又說曰先人有奪人之心殿下選精騎一萬徑往京師奔喪曉夜兼行誰敢止約至京徑掩仁壽宫彼縱徵召未暇禦我大軍絡繹隨王而至此則次計王直資河北彼率天下之兵百道攻我則難為主人此下計也今從隋書】諒大悦於是遣所署大將軍余公理出大谷趣河陽【姓苑余姓由余之後隋志大谷縣屬太原郡舊曰陽邑開皇十八年改焉水經注大谷谷名在祁縣東南河陽縣屬懷州欲由此渡孟津趣七喻翻下同】大將軍綦良出滏口趣黎陽【綦姓也此二軍皆欲使渡河略河南滏音釜】大將軍劉建出井陘略燕趙【陘音刑】柱國喬鍾葵出鴈門【鴈門郡代州也時李景以代州拒諒使鍾葵自嵐州攻之】署文安為柱國與柱國紇單貴王聃等直指京師【紇單虜複姓紇下没翻單可寒翻又達演翻聃他酣翻】帝以右武衛將軍洛陽丘和【隋制左右武衛將軍領外軍宿衛風俗通丘姓左丘明之後又云太公封於營丘支孫以地為氏又魏書官氏志後魏獻帝次弟丘敦氏後改為丘氏按拓拔南都洛陽凡北人從之南遷者三字姓複姓皆改從單字姓為河南洛陽人丘和既洛陽人蓋即丘敦氏之後】為蒲州刺史鎮蒲津諒選精鋭數百騎戴羃䍦【羃莫狄翻䍦音離新唐志曰婦人施羃䍦以蔽身永徽中始用帷冒施羣及頸武后時帷冒益盛中宗後無復羃䍦矣按帷冒起於隋騎奇寄翻下同】詐稱諒宫人還長安門司弗覺徑入蒲州【門司蒲州之掌城門者】城中豪傑亦有應之者丘和覺其變踰城逃歸長安蒲州長史勃海高義明司馬北平榮毗【勃海郡開皇六年置棣州大業二年改滄州北平郡舊置平州榮姓周榮公之後】皆為反者所執裴文安等未至蒲津百餘里諒忽改圖令紇單貴斷河橋守蒲州【此蒲津之橋也諒欲斷河謂可坐有舊齊之地耳斷丁管翻】而召文安還文安至謂諒曰兵機詭速本欲出其不意王既不行文安又返使彼計成大事去矣諒不對以王聃為蒲州刺史裴文安為晉州刺史薛粹為絳州刺史梁菩薩為潞州刺史韋道正為韓州刺史張伯英為澤州刺史【隋志臨汾郡晉州絳郡後魏置東雍州後周改絳州上黨郡後周置潞州上黨郡襄垣縣後周置韓州大業初州廢長平郡舊曰建州開皇初改澤州菩薄乎翻薩桑葛翻聃他甘翻】代州總管天水李景發兵拒諒諒遣其將劉暠襲景【將即亮翻暠古老翻】景擊斬之諒復遣喬鍾葵帥勁勇三萬攻之【復扶又翻帥讀曰率下同】景戰士不過數千加以城池不固為鍾葵所攻崩毁相繼景且戰且築士卒皆殊死鬬鍾葵屢敗司馬馮孝慈司法呂玉【司法即法曹行參軍】並驍勇善戰【驍堅堯翻】儀同三司侯莫陳乂多謀畫工拒守之術景知三人可用推誠任之已無所關預唯在閤持重時撫循而已楊素將輕騎五千襲王聃紇單貴於蒲州夜至河際收商賈船得數百艘【賈音古艘蘇遭翻】船内多置草踐之無聲遂衘枚而濟遲明擊之【遲直二翻】紇單貴敗走聃懼以城降【降戶江翻】有詔徵素還初素將行計日破賊皆如所量【量音良】於是以素為并州道行軍總管河北道安撫大使帥衆數萬以討諒【使疏吏翻】諒之初起兵也妃兄豆盧毓為府主簿苦諫不從私謂其弟懿曰吾匹馬歸朝自得免禍此乃身計非為國也不若且偽從之徐伺其便【朝直遥翻為于偽翻】毓勣之子也【豆盧勣見一百七十四卷陳宣帝太建十二年勣則歷翻】毓兄顯州刺史賢【隋志淮安郡後魏置東荆州西魏改淮州開皇五年又改顯州】言於帝曰臣弟毓素懷志節必不從亂但逼兇威不能自遂臣請從軍與毓為表裏諒不足圖也帝許之賢密遣家人齎勑書至毓所與之計議諒出城將往介州【隋志西河郡後魏置汾州後齊置南朔州後周改曰介州】令毓與總管屬朱濤留守【屬在掾下守手又翻】毓謂濤曰漢王構逆敗不旋踵吾屬豈可坐受夷滅孤負國家邪【邪音耶】當與卿出兵拒之濤驚曰王以大事相付何得有是語因拂衣而去毓追斬之出皇甫誕於獄與之恊計及開府儀同三司宿勤武等【宿勤虜複姓後魏末有宿勤明達叛亂】閉城拒諒部分未定【分扶問翻】有人告諒諒襲擊之 【考異曰皇甫誕傳云楊素將至諒屯清源以拒之按諒屯清源時素軍已迫何暇自還襲毓今從毓傳】毓見諒至紿其衆曰此賊軍也【紿徒亥翻】諒攻城南門稽胡守南城【稽胡步落稽也散居介石二州】不識諒射之【射而亦翻】矢下如雨諒移攻西門守兵識諒即開門納之毓誕皆死綦良攻慈州刺史上官政不克【隋志魏郡滏陽縣後周置開皇十年置慈州大業初州廢】引兵攻行相州事薛胄又不克【魏郡置相州治安陽相息亮翻下同】遂自滏口攻黎州【隋志汲郡黎陽縣舊置黎州】塞白馬津【白馬津在東郡白馬縣北對黎陽岸塞之使不得渡塞悉則翻】余公理自太行下河内【行戶剛翻】帝以右衛將軍史祥為行軍總管軍於河隂【河隂縣東魏置屬洛陽郡北對河陽岸】祥謂軍吏曰余公理輕而無謀【輕墟政翻】恃衆而驕不足破也公理屯河陽祥具舟南岸公理聚兵當之祥簡精鋭於下流濳濟公理聞之引兵拒之戰於須水【按九域志鄭州滎陽縣有須水鎮然其地在河南史祥既濟河擊余公理當遇戰於河陽界水經志湨水出原城西北原山勲掌公東南流過河陽無辟城又南入于河疑須水當作湨水湨古閴翻杜佑通典作湨水音同則須字誤明矣】公理未成列祥擊之公理大敗祥東趣黎陽綦良軍不戰而潰祥寧之子也【史寧從宇文氏於兵間屢有戰功】帝將發幽州兵疑幽州總管竇抗有貳心問可使取抗者於素素薦前江州刺史勃海李子雄【隋志九江郡舊置江州】授上大將軍拜廣州刺史【拜廣州而使之往幽州未得之廣州】又以左領軍將軍長孫晟為相州刺史【隋志左右領軍府各掌十二軍籍帳差科辭訟之事】發山東兵與李子雄共經略之晟辭以男行布在諒所部帝曰公體國之深終不以兒害義朕今相委公其勿辭李子雄馳至幽州止傳舍【傳直戀翻】召募得千餘人抗來詣子雄子雄伏甲擒之抗榮定之子也【竇榮定見一百七十五卷陳長城公至德二年】子雄遂發幽州兵步騎三萬自井陘西擊諒時劉建圍戍將京兆張祥於井陘子雄破建於抱犢山下【隋志恒州石邑縣有抱犢山】建遁去李景被圍月餘【被皮義翻】詔朔州刺史代人楊義臣救之【馬邑郡朔州與代州接境楊義臣本姓尉遲尉遲迥之亂義臣父崇以宗族之故自囚於獄高祖慰釋之後崇與突厥戰死義臣尚幼養於宫中以其父誠節賜姓楊氏】義臣帥馬步二萬夜出西陘【新唐志代州鴈門縣有東陘關西陘關帥讀曰率】喬鍾葵悉衆拒之義臣自以兵少【少詩沼翻】悉取軍中牛驢得數千頭復令兵數百人人持一鼓濳驅之匿於澗谷間晡後義臣復與鍾葵戰【復扶又翻】兵初合命驅牛驢者疾進一時鳴鼓塵埃漲天鍾葵軍不知以為伏兵發因而奔潰義臣縱擊大破之晉絳呂三州皆為諒城守【隋志臨汾郡霍邑縣後魏置東安郡開皇十六年置汾州十八年改呂州為于偽翻】楊素各以二千人縻之而去諒遣其將趙子開擁衆十餘萬栅絶徑路屯據高壁【高壁嶺名將即亮翻下同】布陳五十里【陳讀曰陣】素令諸將以兵臨之自引奇兵濳入霍山【霍山在霍邑東北亦曰太岳山禹貢所謂岳陽指是山之陽也史記謂之霍太山】緣崖谷而進素營於谷口自坐營外使軍司入營簡留三百人守營【漢晉謂軍司馬為軍司今軍吏亦謂之軍司】軍士憚北兵之彊不欲出戰多願守營因爾致遲素責所由軍司具對素即召所留三百人出營悉斬之更令簡留人皆無願留者素乃引軍馳進出北軍之北直指其營鳴鼓縱火北軍不知所為自相蹂踐殺傷數萬【蹂人九翻】諒所署介州刺史梁脩羅屯介休【隋介州治隰城縣而介休縣屬焉】聞素至棄城走諒聞趙子開敗大懼自將衆且十萬拒素於蒿澤會大雨諒欲引軍還王頍諫曰楊素懸軍深入士馬疲弊王以鋭卒自將擊之其勢必克今望敵而退示人以怯沮戰士之心益西軍之氣【楊素軍自長安來故謂之西軍】願王勿還諒不從退守清源【開皇十六年分晋陽置清源縣在晉陽西南宋白曰地理志榆次縣梗陽鄉魏戌邑今梗陽故城在清源縣南一百二十步此縣自漢至晉皆為榆次縣地隋置清源縣因縣西清源水為名】王頍謂其子曰氣候殊不佳兵必敗汝可隨我楊素進擊諒大破之擒蕭摩訶諒退保晉陽素進兵圍之諒窮蹙請降【降戶江翻】餘黨悉平帝遣楊約齎手詔勞素【勞力到翻】王頍將奔突厥至山中徑路斷絶知必不免謂其子曰吾之計數不減楊素但坐言不見從遂至於此不能坐受擒獲以成竪子名吾死之後汝慎勿過親故於是自殺瘞之石窟中其子數日不得食遂過其故人【過古禾翻瘞於計翻】竟為所擒并獲頍尸梟於晉陽【梟其首也梟工堯翻】羣臣奏漢王諒當死帝不許除名為民絶其屬籍竟以幽死諒所部吏民坐諒死徙者二十餘萬家初高祖與獨孤后甚相愛重誓無異生之子嘗謂羣臣曰前世天子溺於嬖幸【溺奴狄翻嬖卑義翻又博計翻下同】嫡庶分爭遂有廢立或至亡國朕旁無姬侍五子同母可謂真兄弟也豈有此憂邪【邪音耶】帝又懲周室諸王微弱故使諸子分據大鎮專制方面權侔帝室及其晩節父子兄弟迭相猜忌五子皆不以壽終<br />
<br />
  臣光曰昔辛伯諗周桓公曰内寵並后外寵貳政嬖子配嫡大都偶國亂之本也【左傳辛伯有是言而狐突引之諗式甚翻告也深諫也】人主誠能慎此四者亂何自生哉隋高祖徒知嫡庶之多爭孤弱之易摇【易以䜴翻】曾不知勢鈞位逼雖同產至親不能無相傾奪考諸辛伯之言得其一而失其三乎<br />
<br />
  冬十月己卯葬文皇帝於太陵廟號高祖與文獻皇后同墳異穴 詔除婦人及奴婢部曲之課男子二十二成丁【隋因周齊之制婦人及奴婢部曲課役各隨給田為差軍人以二十一成丁至是以戶口益多府庫盈溢故有是詔是後兵役䌓興盗賊羣起而是詔為具文矣】 章仇太翼言於帝曰陛下酉命雍州為破木之衝【木旺在卯雍州在西酉位也故為破木之衝雍於用翻】不可久居又䜟云脩治洛陽還晉家【治直之翻】帝深以為然十一月乙未幸洛陽留晉王昭守長安楊素以功拜其子萬石仁行姪玄挺為儀同三司賚物五萬段綺羅千匹諒妓妾二十人【妓渠綺翻】 丙申發丁男數十萬掘塹自龍門東接長平汲郡【龍門縣屬蒲州長平郡澤州汲郡衛州塹七艶翻】抵臨清關【唐志衛州新鄉縣東北有臨清關】度河至浚儀襄城【浚儀汴州襄城汝州】達於上洛【上洛商州】以置關防 壬子陳叔寶卒贈大將軍長城縣公【長城縣屬吳郡今長興縣是也卒子恤翻】諡曰煬【諡法好内怠政曰煬帝諡陳叔寶曰煬豈知已不令終亦諡曰煬乎】 蜀王秀之得罪也【見上卷二年】右衛大將軍元胄坐與交通除名久不得調【調徒釣翻】時慈州刺史上官政坐事徙嶺南將軍丘和以蒲州失守除名【守式又翻】胄與和有舊酒酣謂和曰上官政壯士也今徙嶺表得無大事乎因自拊腹曰若是公者不徒然矣和奏之胄竟坐死【元胄乘危而擠元旻於死豈知丘和在其後乎】於是徵政為驍衛將軍【唐六典曰漢武帝以李廣為驍騎將軍後省光武改屯騎為驍騎晉文王立晉臺以為宿衛之官歷宋齊梁陳後魏北齊並有驍騎將軍之職後周有左右驍騎率上士二人至隋煬帝改左右備身為左右驍衛尋以其所領名豹騎而又别置備身驍堅堯翻】以和為代州刺史<br />
<br />
  煬皇帝上之上【諱廣一名英小字阿?高祖第二子也謚法好内怠政曰煬去禮遠衆曰煬逆天虐民曰煬】<br />
<br />
  大業元年春正月壬辰朔赦天下改元 立妃蕭氏為皇后 廢諸州總管府【後周置諸州總管隋因之又有增置今廢之】 丙辰立晉王昭為皇太子 高祖之末羣臣有言林邑多奇寶者時天下無事劉方新平交州【劉方平交州見上卷仁壽三年】乃授方驩州道行軍總管【隋志日南郡梁置德州開皇十八年改曰驩州】經略林邑方遣欽州刺史甯長真等以步騎萬餘出越裳【隋志寧越郡置欽州越裳縣屬日南郡】方親帥大將軍張愻等以舟師出比景【比景漢縣屬日南郡隋置比景郡帥讀曰率愻蘇困翻】是月軍至海口【林邑出海之口】 二月戊辰勑有司大陳金寶器物錦綵車馬引楊素及諸將討漢王諒有功者立於前【將即亮翻】使奇章公牛弘宣詔稱揚功伐【隋志巴州其章縣梁置又符陽縣舊置其章郡其一作奇牛弘傳封奇章郡公積功曰伐左傳大夫稱伐漢紀非有功伐】賜賚各有差【賚來代翻】素等再拜舞蹈而出己卯以素為尚書令【唐六典秦變周法天下之事皆决丞相府置尚書於禁中有令丞掌通章奏而已漢初因之武宣之後稍以委任及光武親總吏職天下事皆上尚書與人主參决乃下三府尚書令為端揆之官魏晉已來其任尤重】 詔天下公除惟帝服淺色黃衫鐵裝帶 三月丁未詔楊素與納言楊達將作大匠宇文愷營建東京【後周并齊以洛陽為東京】每月役丁二百萬人徙洛州郭内居民及諸州富商大賈數萬戶以實之廢二崤道開葼册道【左傳晉禦秦師於殽殽有二陵焉南陵夏后臯之墓也北陵文王所以避風雨也酈道元曰言山徑委湥峯阜交隂故可以避風雨水經有盤崤石崤千崤之山盤崤之山崤水所出也石崤之山石崤水所出也所謂崤有二陵則石崤之山也千崤之山千崤之水出焉其水北流瀍洛二道漢建安中曹公西討惡南路之嶮更開北道自後行旅率多從之山側附路有石銘云晉太康三年弘農太守梁柳修復故道太崤以東西崤以西明非一崤也魏書地形志恒農郡有崤縣太和十一年置縣有三崤山志又有西恒農郡治恒農縣有桃林隋志河南郡桃林縣開皇十六年置有上陽宫陜縣後魏置陜州恒農郡後周又置崤郡開皇初郡廢大業初州廢置恒農宫又熊耳縣後周置有後魏崤縣大業初廢有二崤及峽石山新唐志陜州峽石縣本崤移治峽石塢有繡嶺宫靈寶縣本桃林古函谷關在縣西有桃源宫洛州永寧縣本熊耳西五里有崎岫宫南三十三里有蘭峯宫此皆東西二京往來緣道離宫雜出於隋唐所置不載所謂葼冊道不知此道起於何所入於何所山海經曰夸父之山在湖縣西九里其山多椶柟其北曰桃林或者椶柟字後訛為葼冊遂為葼冊道歟無徵不信又當博考杜佑曰隋大業七年移潼關道於南北鎮城間坍獸檻谷置去舊關四里餘賈音古坍音闕葼子紅翻】 戊申詔曰聽採輿頌謀及庶民故能審刑政之得失今將巡歷淮海觀省風俗【省悉景翻】 勑宇文愷與内史舍人封德彞等營顯仁宫南接皁澗北跨洛濱【隋志河南郡壽安縣有顯仁宫水經注洛水徑宜陽縣故城南又東與黑澗水合水出陸渾西山歷黑澗西北入洛皁才早翻】發大江之南五嶺以北奇材異石輸之洛陽又求海内嘉木異草珍禽奇獸以實園苑辛亥命尚書右丞皇甫議發河南淮北諸郡民前後百餘萬開通濟渠【杜佑曰陳留郡城西有通濟渠煬帝開以通江淮漕運兼引汴水即莨蕩渠也 考異曰雜記作皇甫公儀又云發兵夫五十餘萬今從略記】自西苑引穀洛水達于河【是歲營建東京東去故都十八里南直伊闕之口北倚邙山之塞東出瀍水之東西出澗水之西其城西面連苑距上陽宫七里苑牆周迴一百二十六里北拒北邙西至孝水南帶洛水支渠穀洛二水會于其間故自苑引之為渠以達于河】復自板渚引河歷滎澤入汴【板渚在虎牢之東水經河水東合汜水又東過板城北有津謂之板城渚口又東過滎陽縣蒗蕩渠出焉是渠南出為汴水漢之滎陽石門即其地也隋志滎陽郡滎澤縣開皇四年置曰廣武仁壽元年改焉】又自大梁之東引汴水入泗達子淮【大梁即浚儀也引河入汴汴入泗蓋皆故道】又發淮南民十餘萬開䢴溝自山陽至揚子入江【春秋吳城䢴溝通江淮此亦因故道也䢴溝貫今揚州城中山陽今淮安州揚子今真州䢴音寒】渠廣四十步【廣古曠翻】渠旁皆築御道樹以柳自長安至江都【江都郡揚州】置離宫四十餘所庚申遣黃門侍郎王弘等往江南造龍舟及雜船數萬艘【艘蘇遭翻】東京官吏督役嚴急役丁死者什四五所司以車載死丁東至城臯【隋志鄭州滎陽縣舊置城臯郡】北至河陽相望於道又作天經宫於東京四時祭高祖【經曰夫孝天之經也故以名宫】 林邑王梵志【梵扶泛翻】遣兵守險劉方擊走之師度闍黎江【闍視遮翻】林邑兵乘巨象四面而至方戰不利乃多掘小坑草覆其上【覆敷又翻】以兵挑之【挑徒了翻】既戰偽北林邑逐之象多陷地顛躓【躓音致】轉相驚駭軍遂亂方以弩射象象却走蹂其陳【射而亦翻蹂人九翻陳讀曰陣】因以鋭師繼之林邑大敗俘馘萬計方引兵追之屢戰皆捷過馬援銅柱南【新唐書林邑奔浪陀州其南大浦有五銅柱山形若倚蓋西重巖東涯海漢馬援所植也杜佑曰林邑南水步二千餘里有西屠夷馬援所樹兩銅柱表界處也銅柱山周十里形如倚蓋西跨重巖東臨大海宋白曰馬援討交趾自日南南行四百餘里至林邑又南行二千餘里有西屠夷國援至其國鑄二銅柱於象林南界與西屠夷分境計交州至銅柱五千里宋杜之說銅柱在林邑南今此所記則林邑在銅柱南】八日至其國都夏四月梵志棄城走入海方入城獲其廟主十八皆鑄金為之刻石紀功而還士卒腫足死者什四五方亦得疾卒於道【卒子恤翻】初尚書右丞李綱數以異議忤楊素及蘇威【數色角翻忤五故翻】素薦綱於高祖以為方行軍司馬方承素意屈辱之幾死【幾居希翻】軍還久不得調【調徒釣翻】威復遣綱詣南海應接林邑久而不召綱自歸奏事威劾奏綱擅離所職下吏按問【劾戶槩翻又戶得翻離力智翻下遐嫁翻】會赦免官屏居於鄠【鄠縣屬京兆郡為李綱為何潘仁所逼致張本屛必郢翻鄠音戶】 五月築西苑周二百里【與六典所紀小異】其内為海周十餘里為蓬萊方丈瀛洲諸山【象海中三神山】高出水百餘尺臺觀殿閣羅絡山上向背如神【觀古玩翻背蒲妹翻】北有龍鱗渠縈紆注海内緣渠作十六院門皆臨渠每院以四品夫人主之【内命婦之品胝百官】堂殿樓觀窮極華麗宫樹秋冬彫落則剪綵為華葉綴於枝條色渝則易以新者常如陽春沼内亦剪綵為荷芰菱芡乘輿遊幸則去冰而布之【芰奇寄翻芡巨險翻乘繩證翻去羌呂翻】十六院競以殽羞精麗相高求市恩寵上好以月夜從宫女數千騎遊西苑作清夜遊曲於馬上奏之【用曹植清夜遊西園之詩以名曲好呼到翻騎奇計翻】 帝待諸王恩薄多所猜忌滕王綸衛王集内自憂懼呼術者問吉凶及章醮求福或告其怨望呪詛【呪職救翻詛莊助翻】有司奏請誅之秋七月丙午詔除名為民徙邊郡綸瓚之子集爽之子也【瓚高祖之母弟爽異母弟瓚藏旱翻】 八月壬寅上行幸江都 【考異曰雜記作九月今從隋帝紀及略記】發顯仁宫王弘遣龍舟奉迎乙巳上御小朱航自漕渠出洛口【洛水入河之口】御龍舟 【考異曰略記云甲子進龍舟按長歷是月戊子朔無甲子】龍舟四重高四十五尺【重直龍翻高工號翻 考異曰略記云高五丈雜記言其制度尤詳今從之】長二百丈【長尺亮翻】上重有正殿内殿東西朝堂中二重有百二十房皆飾以金玉下重内侍處之【朝直遥翻處直呂翻】皇后乘翔螭舟制度差小而裝飾無異别有浮景九艘三重皆水殿也【螭丑知翻艘蘇遭翻下同】又有漾彩朱鳥蒼螭白虎玄武飛羽青鳬陵波五樓道場玄壇板黃篾等數千艘【託盍翻大船曰篾音蔑】後宫諸王公主百官僧尼道士蕃客乘之【尼女夷翻】及載内外百司供奉之物共用挽船士八萬餘人其挽漾彩以上者九千餘人謂之殿脚皆以錦綵為袍又有平乘青龍艨艟艚八櫂艇舸等數千艘【艨莫公翻艟尺庸翻艚昨遭翻字書闕擢讀曰棹艇徒頂翻舸賈我翻】並十二衛兵乘之并載兵器帳幕兵士自引不給夫舳艫相接二百餘里照耀川陸騎兵翊兩㟁而行【舳艫音逐盧騎奇寄翻下同】旌旗蔽野所過州縣五百里内皆令獻食多者一州至百轝【轝音余】極水陸珍奇後宫厭飫將發之際多棄埋之【飫於據翻】 契丹寇營州【遼西郡置營州契欺訖翻又音喫】詔通事謁者韋雲起【隋志帝即位增置謁者臺改内史省通事舍人為謁者臺職通事謁者員二十人從六品】護突厥兵討之啟民可汗發騎二萬受其處分【厥九勿翻可從刋入聲汗音寒處昌呂翻分扶問翻】雲起分為二十營四道俱引營相去一里不得交雜聞鼓聲而行聞角聲而止自非公使勿得走馬【公使謂公事使之】三令五申擊鼓而發有紇干犯約斬之【紇干突厥小官紇下没翻】持首以徇於是突厥將帥入謁皆膝行股栗莫敢仰視【將即亮翻帥所類翻】契丹本事突厥情無猜忌雲起既入其境使突厥詐云向柳城【此古柳城也隋志遼西郡營州並治柳城縣乃龍城縣龍城本和龍城自後魏以來營州治焉開皇元年改為龍山縣十八年改為柳城】與高麗交易敢漏泄事實者斬契丹不為備去其營五十里馳進襲之盡獲其男女四萬口殺其男子以女子及畜產之半賜突厥餘皆收之以歸帝大喜集百官曰雲起用突厥平契丹才兼文武朕今自舉之擢為治書侍御史【治直之翻】 初西突厥阿波可汗為葉護可汗所虜【見一百七十六卷陳長城公禎明元年】國人立鞅素特勒之子是為泥利可汗泥利卒子達漫立號處羅可汗其母向氏【向式亮翻】本中國人更嫁泥利之弟婆實特勒【更工衡翻】開皇末婆實與向氏入朝【朝直遥翻】遇達頭之亂遂留長安舍於鴻臚寺【鴻臚寺主蕃客臚音閭】處羅多居烏孫故地撫御失道國人多叛復為鐵勒所困【復扶又翻】鐵勒者匈奴之遺種【種章勇翻】族類最多有僕骨同羅契苾薛延陁等部其酋長皆號俟斤【酋才由翻長知兩翻俟渠之翻】族姓雖殊通謂之鐵勒大抵與突厥同俗以寇抄為生無大君長分屬東西兩突厥是歲處羅引兵擊鐵勒諸部厚税其物又猜忌薛延陁恐其為變集其酋長數百人盡殺之於是鐵勒皆叛立俟利發俟斤契苾歌楞為莫何可汗【苾毗必翻楞盧登翻】又立薛延陁俟斤字也咥為小可汗【咥昌栗翻又徒結翻】與處羅戰屢破之莫何勇毅絶倫甚得衆心為鄰國所憚伊吾高昌焉耆皆附之<br />
<br />
  二年春正月辛酉東京成進將作大匠宇文愷位開府儀同三司 丁卯遣十使併省州縣【使疏吏翻】 二月丙戌詔吏部尚書牛弘等議定輿服儀衛制度 【考異曰帝紀云尚書令牛弘禮部侍郎許善心按弘未嘗為尚書令善心於帝即位之初已左遷蓋紀誤也】以開府儀同三司何稠為太府少卿使之營造送江都稠智思精巧【思相吏翻】博覽圖籍參會古今多所損益衮冕畫日月星辰皮弁用漆紗為之【書日月星辰山龍華蟲作會周升日月於旌旗而闕三辰今復古制五經通義弁高五寸前後玉飾詩云會弁如星董巴曰以鹿皮為之何稠用漆紗施象牙簪導弁加簪導自稠始也】又作黃麾三萬六千人仗【黃麾仗汔唐遵而用之大朝會大駕】及輅輦車輿皇后鹵簿百官儀服務為華盛以稱上意【稱尺證翻】課州縣送羽毛民求捕之網羅被水陸【被皮義翻】禽獸有堪氅毦之用者殆無遺類【氅昌兩翻毦乃吏翻羽毛飾也】烏程有高樹【烏程屬湖州郡國志曰古烏氏程氏居此能醖酒故以名縣】踰百尺旁無附枝上有鶴巢民欲取之不可上【上時掌翻】乃伐其根鶴恐殺其子自拔氅毛投於地時人或稱以為瑞曰天子造羽儀鳥獸自獻羽毛所役工十萬餘人用金銀錢帛鉅億計帝每出遊幸羽儀填街溢路亘二十餘里三月庚午上發江都夏四月庚戌自伊闕陳法駕備千乘萬騎入東京【隋志伊闕縣舊曰新城開皇十八年更名屬河南郡北至東京二百餘里乘繩證翻騎奇寄翻】辛亥御端門【唐六典東京皇城南面三門中曰端門】大赦免天下今年租賦制五品已上文官乘車在朝弁服佩玉【隋志五品已上服紫自公已下佩水蒼玉朝直遥翻】武官馬加珂戴幘服袴褶【珂螺屬生海中潔白如雪通俗文曰馬勒飾曰珂温公類篇曰鵰入海為珂爾雅翼曰珂黃黑色其骨白可以飾馬此等飾非特取其容兼取其聲故說文蘇切貝聲也董巴曰幘起於秦人施於武將初為絳袹以表貴賤珂丘何翻褶音習】文物之盛近世莫及也六月壬子以楊素為司徒進封豫章王為齊王【古】<br />
<br />
  【限翻】 秋七月庚申制百官不得計考增級必有德行功能灼然顯著者進擢之【行下孟翻】帝頗惜名位羣臣當進職者多令兼假而已雖有闕員留而不補時牛弘為吏部尚書不得專行其職别勑納言蘇威左翊衛大將軍宇文述【帝改左右衛為左右翊衛】左驍衛大將軍張瑾【驍堅堯翻】内史侍郎虞世基御史大夫裴藴黃門侍郎裴矩參掌選事時人謂之選曹七貴【選宣戀翻】雖七人同在坐【坐徂卧翻】然與奪之筆虞世基獨專之受納賄賂多者超越等倫無者注色而已【注其入仕所歷之色也宋末參選者具脚色狀今謂之根脚】藴邃之從曾孫也【裴邃為梁將著功名從才用翻】 元德太子昭自長安來朝【帝令昭留守長安朝直遥翻】數月將還欲乞少留【少詩沼翻】帝不許拜請無數體素肥因致勞疾甲戌薨 【考異曰雜記云初太子之遘疾也時與楊素同在侍宴帝既深忌於素並起二巵同至傳酒者不悟是藥酒錯進太子既飲二日而毒發下血二斗餘宫人聞素平常始知毒酒誤飲太子祕不敢言太子知之歎曰豈意代楊素死乎數日而薨後素亦竟以毒斃按它書皆無此說蓋時人見太子與素相繼死妄有此論耳】帝哭之數聲而止尋奏聲伎無異平日【伎渠綺翻】 楚景武公楊素雖有大功特為帝所猜忌外示殊禮内情甚薄太史言隋分野有大喪乃徙素為楚公意言楚與隋同分欲以厭之【分扶問翻厭於葉翻】素寢疾帝每令名醫診候賜以上藥然密問醫者恒恐不死【診章忍翻恒戶登翻】素亦自知名位已極不肯餌藥亦不將慎謂其弟約曰我豈須更活邪乙亥素薨贈太尉公弘農等十郡太守葬送甚盛【邪音耶守式又翻】 八月辛卯封皇孫倓為燕王侗為越王【倓徒甘翻燕因肩翻侗他紅翻】侑為代王皆昭之子也 九月乙丑立秦孝王子浩為秦王【帝弟秦王俊諡秦孝王】 帝以高祖末年法令峻刻冬十月詔改修律令 置洛口倉於鞏東南原上【鞏縣屬河南郡洛水至鞏縣入河謂之洛口】築倉城周回二十餘里穿三千窖【窖工孝翻】窖容八千石以還置監官并鎮兵千人【監古衘翻】十二月置回洛倉於洛陽北七里倉城周回十里穿三百窖 初齊温公之世【齊主緯周封為温公】有魚龍山車等戲謂之散樂【散悉亶翻】周宣帝時鄭譯奏徵之【見一百七十四卷陳高宗太建十一年散悉亶翻】高祖受禪命牛弘定樂非正聲清商及九部四舞之色悉放遣之【正聲謂鄭譯等所定之樂也開皇九年平陳置清商署管宋齊舊樂即清樂也杜佑曰清樂者其始即清商三調是也並漢氏以來舊典樂器形制并歌章古調與魏三祖所作者皆備於史籍屬晉朝遷播夷羯竊據其音分散苻堅平張氏於凉州得之宋武平關中因而入南及隋平陳後文帝聽而善其節奏曰此華夏正聲也因置清商署總謂之清樂帝定清樂西凉龜兹天竺康國疎勒安國高麗禮畢為九部又開皇定令牛弘請存鞞鐸巾拂四舞與諸伎並陳因謂之四舞】帝以啟民可汗將入朝欲以富樂誇之太常少卿裴藴希旨奏括天下周齊梁陳樂家子弟皆為樂戶【可從刋入聲汗音寒朝直遥翻富樂音洛少始照翻】其六品以下至庶人有善音樂者皆直太常帝從之於是四方散樂大集東京閲之於芳華苑積翠池側【芳華苑蓋即西苑】有舍利獸先來跳躍激水滿衢黿鼉龜鼈水人蟲魚偏覆于地【覆敷又翻】又有鯨魚噴霧翳日倏忽化成黃龍長七八丈【長直亮翻】又二人戴竿上有舞者歘然騰過左右易處【歘許勿翻】又有神鼇負山幻人吐火千變萬化伎人皆衣錦繡繒綵【伎渠綺翻衣於既翻】舞者鳴環佩綴花毦【毦乃吏翻】課京兆河南製其衣兩京錦綵為之空竭【為于偽翻】帝多製艶篇令樂正白明達造新聲播之音極哀怨【隋置太樂署清商署各有樂師員帝改樂師為樂正置員十人】帝甚悦謂明達曰齊氏偏隅樂工曹妙達猶封王【隋志齊後主賞胡戎樂耽愛無已於是繁手淫聲爭新哀怨故曹妙達安馬駒之徒至有封王開府煬帝溺於淫聲以亡國自况淫昏甚矣】我今天下大同方且貴汝宜自脩謹<br />
<br />
  三年春正月朔旦大陳文物時突厥啓民可汗入朝見而慕之請襲冠帶帝不許明日又率其屬上表固請【厥九勿翻可從刋入聲汗音寒帥讀曰率上時掌翻】帝大悦謂牛弘等曰今衣冠大備致單于解辮卿等功也【單音蟬】各賜帛甚厚 三月辛亥帝還長安 癸丑帝使羽騎尉朱寛入海【開皇六年置武騎屯騎驍騎游騎飛騎旅騎雲騎羽騎八尉羽騎從九品騎奇寄翻】求訪異俗至流求國而還【隋書流求國居海島之中當建安郡東水行五日而至是後陳稜自義安擊流求至高華嶼又東行二日至鼊嶼又一日便至流求還從宣翻音如字】 初雲定興閻毗坐媚事太子勇與妻子皆没官為奴婢【事見上卷開皇二十年】上即位多所營造聞其有巧思【思相吏翻】召之使典其事以毗為朝請郎【開皇置八郎朝請第三朝直遥翻】時宇文述用事定興以明珠絡帳賂述并以奇服新聲求媚於述述大喜兄事之上將有事四夷大作兵器述薦定興可使監造上從之【監古衘翻】述謂定興曰兄所作器仗並合上心而不得官者為長寧兄弟猶未死耳定興曰此無用物何不勸上殺之述因奏房陵諸子【廢太子勇追封房陵王】年並成立今欲興兵誅討若使之從駕則守掌為難若留於一處又恐不可進退無用請早處分【處昌呂翻分扶問翻】帝然之乃鴆殺長寧王儼分徙其七弟於嶺表仍遣間使於路盡殺之【間古莧翻使疏吏翻】襄城王恪之妃柳氏自殺以從恪 夏四月庚辰下詔欲安輯河北巡省趙魏【省悉景翻】 牛弘等造新律成凡十八篇謂之大業律甲申始頒行之民久厭嚴刻喜於寛政其後征役繁興民不堪命有司臨時迫脅以求濟事不復用律令矣【復扶又翻又音如字】旅騎尉劉炫預修律令弘嘗從容問炫曰【騎奇寄翻炫熒絹翻從千容翻】周禮士多而府史少【少詩沼翻】今令史百倍於前減則不濟其故何也炫曰古人委任責成歲終考其殿最【殿丁甸翻】案不重校【重直龍翻】文不繁悉府史之任掌要目而已今之文簿恒慮覆治【治直之翻】若鍜鍊不密則萬里追證百年舊案故諺云老吏抱案死事繁政弊職此之由也弘曰魏齊之時令史從容而已今則不遑寧處何故【處昌呂翻】炫曰往者州唯置綱紀【此綱紀謂長史司馬】郡置守丞縣置令而已其餘具僚則長官自辟【長知兩翻】受詔赴任每州不過數十今則不然大小之官悉由吏部纎介之迹皆屬考功【考功侍郎掌内外文武官吏之功課皆具錄當年功過行能而考校之】省官不如省事官事不省而望從容其可得乎弘善其言而不能用 壬辰改州為郡改度量權衡並依古式改上柱國以下官為大夫【舊上柱國下至都督凡十一等今改為光祿左右光祿金紫銀青光祿正議通議朝請朝散九大夫】置殿内省【殿内省掌諸供奉】與尚書門下内史祕書為五省增謁者司隸臺【謁者臺掌受詔勞問出使慰撫持節察按及受寃枉而申奏之司隸臺掌諸巡察】與御史為三臺分太府寺置少府監【太府寺止掌左右藏黃藏其尚方司織司染鎧甲弓弩掌冶皆屬少府監少始照翻】與長秋國子將作都水為五監【改内侍省為長秋監】又增改左右翊衛等為十六府【改左右衛為左右翊衛左右備身為左右驍衛左右武衛依舊名改領軍為左右屯衛加置左右禦衛改左右武候為左右候衛是為十二衛改領左右府為左右備身府左右監門依舊名凡十六府】廢伯子男爵唯留王公侯三等 丙寅車駕北巡己亥頓赤岸澤五月丁巳突厥啟民可汗遣其子拓特勒來朝【厥九勿翻可從刋入聲汗音寒朝直遥翻】戊午發河北十餘郡丁男鑿太行山達于并州以通馳道【行戶剛翻】丙寅啟民遣其兄子毗黎伽特勒來朝【伽求加翻】辛未啟民遣使請自入塞奉迎輿駕【使疏吏翻】上不許 初高祖受禪唯立四親廟同殿異室而已【四親廟一曰皇高祖太原府君廟二曰皇曾祖康王廟三曰皇祖獻王廟四曰皇考太祖武元皇帝廟】帝即位命有司議七廟之制禮部侍郎攝太常少卿許善心等【時定制尚書省六部各侍郎一人以貳尚書之職諸曹侍郎並改為郎】奏請為太祖高祖各立一殿【為于偽翻】凖周文武二祧與始祖而三【祧土彫翻】餘並分室而祭從迭毁之法至是有司請如前議於東京建宗廟帝謂祕書監柳䛒曰【䛒與辯同】今始祖及二祧已具後世子孫處朕何所【處昌呂翻】六月丁亥詔為高祖建别廟仍修月祭禮既而方事巡幸竟不果立 帝過鴈門【帝改代州為鴈門郡】鴈門太守丘和【守手又翻】獻食甚精至馬邑【帝改朔州為馬邑郡】馬邑太守楊廓獨無所獻帝不悦以和為博陵太守【改定州為博陵郡丘和自邊郡遷内郡以示賞也】仍使廓至博陵觀和為式由是所至獻食競為豐侈戊子車駕頓榆林郡【時改勝州為榆林郡】帝欲出塞耀兵徑突厥中指于涿郡【厥九勿翻時改幽州為涿郡】恐啟民驚懼先遣武衛將軍長孫晟諭旨【長知兩翻晟承正翻】啟民奉詔因召所部諸國奚霫室韋等酋長數十人咸集【霫居鮮卑故地保冷陘山南奥支水室韋契丹之類也其南者為契丹其北者為室韋新唐書室韋蓋丁零苖裔也地據黃龍北傍□越河霫而立翻酋才由翻長知兩翻】晟見牙帳中草穢欲令啟民親除之示諸部落以明威重乃指前草曰此根大香啟民遽嗅之【嗅許救翻】曰殊不香也晟曰天子行幸所在諸侯躬自灑掃耕除御路以表至敬之心今牙内蕪穢謂是留香草耳啟民乃悟曰奴之罪也奴之骨肉皆天子所賜得効筋力豈敢有辭特以邊人不知法耳賴將軍教之將軍之惠奴之幸也遂拔所佩刀自芟庭草【芟所衘翻】其貴人及諸部爭效之於是發榆林北境至其牙東達於薊【涿郡治薊】長三千里【長直亮翻】廣百步【廣古曠翻】舉國就役開為御道帝聞晟策益嘉之丁酉啟民及義成公主來朝行宫【朝直遥翻】己亥吐谷渾高昌並遣使入貢【吐從暾入聲谷音浴使疏吏翻】甲辰上御北樓觀漁於河以宴百僚定襄太守周法尚朝于行宫【改雲州為定襄郡守式又翻】太守卿元壽言於帝曰漢武出關旌旗千里【事見二十卷漢武帝元封元年】今御營之外請分為二十四軍日别遣一軍發相去三十里旗幟相望鉦鼓相聞首尾相屬【幟昌志翻鉦音征屬之欲翻】千里不絶此亦出師之盛者也法尚曰不然兵亘千里動間山川【間古莧翻】猝有不虞四分五裂腹心有事首尾未知道路阻長難以相救雖有故事乃取敗之道也帝不懌曰卿意如何法尚曰結為方陳【陳讀曰陣】四面外拒六宫及百官家屬並在其内若有變起所當之面即令抗拒内引奇兵出外奮擊車為壁壘重設鈎陳【鉤陳曲陳如鉤象天之鉤陳星重直龍翻陳如字】此與據城理亦何異若戰而捷抽騎追奔【騎奇寄翻】萬一不捷屯營自守臣謂此萬全之策也帝曰善因拜法尚左武衛將軍啟民可汗復上表以為先帝可汗憐臣賜臣安義公主種種無乏【可從刋入聲汗音寒復扶又翻上時掌翻種章勇翻】臣兄弟嫉妬共欲殺臣臣當是時走無所適仰視唯天俯視唯地奉身委命依歸先帝先帝憐臣且死養而生之以臣為大可汗還撫突厥之民【事見一百七十八卷開皇十九年厥九勿翻】至尊今御天下還如先帝養生臣及突厥之民種種無乏臣荷戴聖恩【荷下可翻】言不能盡臣今非昔日突厥可汗乃是至尊臣民願率部落變改衣服一如華夏【帥讀曰率夏戶雅翻】帝以為不可秋七月辛亥賜啟民璽書諭以磧北未靜猶須征戰【璽斯氏翻磧七亦翻】但存心恭順何必變服帝欲誇示突厥令宇文愷為大帳其下可坐數千人甲寅帝於城東御大帳備儀衛宴啟民及其部落作散樂【散昔亶翻】諸胡駭悦爭獻牛羊駝馬數千萬頭帝賜啟民帛二千萬段其下各有差又賜啟民路車乘馬鼓吹幡旗【乘繩證翻吹昌瑞翻】贊拜不名位在諸侯王上又詔發丁男百餘萬築長城西拒榆林東至紫河【隋志定襄郡大利縣有隂山有紫河通典紫河發源朔州善陽縣金河上承紫河】尚書左僕射蘇威諫上不聽築之二旬而畢帝之徵散樂也太常卿高熲諫不聽熲退謂太常丞李懿曰周天元以好樂而亡殷鑒不遠安可復爾【好呼到翻復扶又翻下同】熲又以帝遇啟民過厚謂太府卿何稠曰此虜頗知中國虚實山川險易【易以䜴翻】恐為後患又謂觀王雄曰【雄自安德郡王改封觀王觀古玩翻】近來朝廷殊無綱紀禮部尚書宇文㢸私謂熲曰天元之侈以今方之不亦甚乎又言長城之役幸非急務光祿大夫賀若弼亦私議宴可汗太侈並為人所奏帝以為誹謗朝政丙子高熲宇文㢸賀若弼皆坐誅【㢸古弼字若人者翻朝直遥翻下同】熲諸子徙邊弼妻子没官為奴婢事連蘇威亦坐免官熲有文武大略明達世務自蒙寄任竭誠盡節進引貞良以天下為己任蘇威楊素賀若弼韓擒虎皆熲所推薦自餘立功立事者不可勝數【勝音升】當朝執政將二十年朝野推服物無異議海内富庶熲之力也及死天下莫不傷之先是蕭琮以皇后故甚見親重【先悉薦翻】為内史令改封梁公宗族緦麻以上皆隨才擢用諸蕭昆弟布列朝廷琮性澹雅不以職務為意身雖覊旅見北間豪貴無所降下【澹徒覽翻下遐嫁翻】與賀若弼善弼既誅又有童謡曰蕭蕭亦復起帝由是忌之遂廢於家未幾而卒【幾居豈翻卒子恤翻】 八月壬午車駕發榆林歷雲中泝金河【隋志榆林郡有金河縣杜佑曰單于都護府秦漢雲中郡地也治金河縣縣有金河上承紫河宋白曰金河縣即漢盛樂縣地】時天下承平百物豐實甲士五十餘萬馬十萬匹旌旗輜重千里不絶【重直用翻】令宇文愷等造觀風行殿上容侍衛者數百人離合為之下施輪軸倏忽推移又作行城周二千步以板為榦衣之以布【衣於既翻】飾以丹青樓櫓悉備胡人驚以為神每望御營十里之外屈膝稽顙無敢乘馬【稽音啟】啟民奉廬帳以俟車駕乙酉帝幸其帳啟民奉觴上壽跪伏恭甚【上時掌翻下同】王侯以下袒割於帳前【袒而割肉】莫敢仰視帝大悦賦詩曰呼韓頓顙至屠耆接踵來何如漢天子空上單于臺【上時掌翻單音蟬】皇后亦幸義成公主帳帝賜啟民及公主金甕各一并衣服被褥錦綵特勒以下受賜各有差帝還啟民從入塞己丑遣歸國癸巳入樓煩關【樓煩郡治靜樂縣縣有關官】壬寅至太原詔營晉陽宫帝謂御史大夫張衡曰朕欲過公宅可為朕作主人【過工禾翻為于偽翻】衡乃先馳至河内具牛酒【張衡河内人帝改懷州為河内郡】帝上太行開直道九十里【開直道抵張衡所居行戶剛翻】九月己未至濟源【開皇十六年置濟源縣屬河内郡濟子禮翻】幸衡宅帝悦其山泉留宴三日賜賚甚厚衡復獻食【賚來代翻復扶又翻】帝令頒賜公卿下至衛士無不霑洽己巳至東都 壬申以齊王為河南尹癸酉以民部尚書楊文思為納言 冬十月勑河南諸郡送一藝戶陪東都三千餘家置十二坊於洛水南以處之【藝戶謂其家以技藝多者陪助也處昌呂翻】西域諸胡多至張掖交市【帝改甘州為張掖郡交市為互市也】帝使吏部侍郎裴矩掌之矩知帝好遠略商胡至者矩誘訪諸國山川風俗王及庶人儀形服飾撰西域圖記三卷合四十四國入朝奏之【好呼報翻誘音酉撰士免翻朝直遥翻下同】仍别造地圖窮其要害從西傾以去【西傾山在隴西臨洮縣西南洮水之所出也】縱橫所亘將二萬里發自敦煌【帝改瓜州為敦煌郡敦徒門翻】至于西海【此西海在條支西】凡為三道北道從伊吾中道從高昌南道從鄯善【伊吾唐為伊州高昌唐為西州鄯善唐為納縳波地鄯時戰翻】總湊敦煌且云以國家威德將士驍雄【將即亮翻驍堅堯翻】汎濛汜而越崑崙易如反掌【蒙汜蒙谷之水也日所入處史記曰禹本紀言河出崑崙汜洋里翻崙盧昆翻易以䜴翻】但突厥吐渾分領羌胡之國為其壅遏故朝貢不通【吐從入聲朝直遥翻】今並因商人密送誠欵引領翹首願為臣妾若服而撫之務存安輯皇華遣使弗動兵車諸蕃既從渾厥可滅混壹戎夏其在兹乎【渾厥謂吐谷渾突厥也使疏吏翻厥九勿翻夏戶雅翻】帝大悦賜帛五百段日引矩至御坐親問西域事矩盛言胡中多諸珍寶吐谷渾易可并吞【谷音浴易弋豉翻】帝於是慨然慕秦皇漢武之功甘心將通西域四夷經略咸以委之以矩為黃門侍郎復使至張掖引致諸胡啗之以利勸令入朝【復扶又翻啗徒濫翻又徒覽翻】自是西域胡往來相繼所經郡縣疲於送迎糜費以萬萬計卒令中國疲弊以至於亡【卒子恤翻】皆矩之唱導也 鐵勒寇邊帝遣將軍馮孝慈出敦煌擊之不利鐵勒尋遣使謝罪請降【降戶剛翻】帝使裴矩慰撫之<br />
<br />
  資治通鑑卷一百八十  <br>
   </div> 

<script src="/search/ajaxskft.js"> </script>
 <div class="clear"></div>
<br>
<br>
 <!-- a.d-->

 <!--
<div class="info_share">
</div> 
-->
 <!--info_share--></div>   <!-- end info_content-->
  </div> <!-- end l-->

<div class="r">   <!--r-->



<div class="sidebar"  style="margin-bottom:2px;">

 
<div class="sidebar_title">工具类大全</div>
<div class="sidebar_info">
<strong><a href="http://www.guoxuedashi.com/lsditu/" target="_blank">历史地图</a></strong>  
<a href="http://www.880114.com/" target="_blank">英语宝典</a>  
<a href="http://www.guoxuedashi.com/13jing/" target="_blank">十三经检索</a> 
<br><strong><a href="http://www.guoxuedashi.com/gjtsjc/" target="_blank">古今图书集成</a></strong> 
<a href="http://www.guoxuedashi.com/duilian/" target="_blank">对联大全</a> <strong><a href="http://www.guoxuedashi.com/xiangxingzi/" target="_blank">象形文字典</a></strong> 

<br><a href="http://www.guoxuedashi.com/zixing/yanbian/">字形演变</a>  <strong><a href="http://www.guoxuemi.com/hafo/" target="_blank">哈佛燕京中文善本特藏</a></strong>
<br><strong><a href="http://www.guoxuedashi.com/csfz/" target="_blank">丛书&方志检索器</a></strong> <a href="http://www.guoxuedashi.com/yqjyy/" target="_blank">一切经音义</a>  

<br><strong><a href="http://www.guoxuedashi.com/jiapu/" target="_blank">家谱族谱查询</a></strong>  <strong><a href="http://shufa.guoxuedashi.com/sfzitie/" target="_blank">书法字帖欣赏</a></strong> 
<br>

</div>
</div>


<div class="sidebar" style="margin-bottom:0px;">

<font style="font-size:22px;line-height:32px">QQ交流群9:489193090</font>


<div class="sidebar_title">手机APP 扫描或点击</div>
<div class="sidebar_info">
<table>
<tr>
	<td width=160><a href="http://m.guoxuedashi.com/app/" target="_blank"><img src="/img/gxds-sj.png" width="140"  border="0" alt="国学大师手机版"></a></td>
	<td>
<a href="http://www.guoxuedashi.com/download/" target="_blank">app软件下载专区</a><br>
<a href="http://www.guoxuedashi.com/download/gxds.php" target="_blank">《国学大师》下载</a><br>
<a href="http://www.guoxuedashi.com/download/kxzd.php" target="_blank">《汉字宝典》下载</a><br>
<a href="http://www.guoxuedashi.com/download/scqbd.php" target="_blank">《诗词曲宝典》下载</a><br>
<a href="http://www.guoxuedashi.com/SiKuQuanShu/skqs.php" target="_blank">《四库全书》下载</a><br>
</td>
</tr>
</table>

</div>
</div>


<div class="sidebar2">
<center>


</center>
</div>

<div class="sidebar"  style="margin-bottom:2px;">
<div class="sidebar_title">网站使用教程</div>
<div class="sidebar_info">
<a href="http://www.guoxuedashi.com/help/gjsearch.php" target="_blank">如何在国学大师网下载古籍?</a><br>
<a href="http://www.guoxuedashi.com/zidian/bujian/bjjc.php" target="_blank">如何使用部件查字法快速查字?</a><br>
<a href="http://www.guoxuedashi.com/search/sjc.php" target="_blank">如何在指定的书籍中全文检索?</a><br>
<a href="http://www.guoxuedashi.com/search/skjc.php" target="_blank">如何找到一句话在《四库全书》哪一页?</a><br>
</div>
</div>


<div class="sidebar">
<div class="sidebar_title">热门书籍</div>
<div class="sidebar_info">
<a href="/so.php?sokey=%E8%B5%84%E6%B2%BB%E9%80%9A%E9%89%B4&kt=1">资治通鉴</a> <a href="/24shi/"><strong>二十四史</strong></a>&nbsp; <a href="/a2694/">野史</a>&nbsp; <a href="/SiKuQuanShu/"><strong>四库全书</strong></a>&nbsp;<a href="http://www.guoxuedashi.com/SiKuQuanShu/fanti/">繁体</a>
<br><a href="/so.php?sokey=%E7%BA%A2%E6%A5%BC%E6%A2%A6&kt=1">红楼梦</a> <a href="/a/1858x/">三国演义</a> <a href="/a/1038k/">水浒传</a> <a href="/a/1046t/">西游记</a> <a href="/a/1914o/">封神演义</a>
<br>
<a href="http://www.guoxuedashi.com/so.php?sokeygx=%E4%B8%87%E6%9C%89%E6%96%87%E5%BA%93&submit=&kt=1">万有文库</a> <a href="/a/780t/">古文观止</a> <a href="/a/1024l/">文心雕龙</a> <a href="/a/1704n/">全唐诗</a> <a href="/a/1705h/">全宋词</a>
<br><a href="http://www.guoxuedashi.com/so.php?sokeygx=%E7%99%BE%E8%A1%B2%E6%9C%AC%E4%BA%8C%E5%8D%81%E5%9B%9B%E5%8F%B2&submit=&kt=1"><strong>百衲本二十四史</strong></a>  <a href="http://www.guoxuedashi.com/so.php?sokeygx=%E5%8F%A4%E4%BB%8A%E5%9B%BE%E4%B9%A6%E9%9B%86%E6%88%90&submit=&kt=1"><strong>古今图书集成</strong></a>
<br>

<a href="http://www.guoxuedashi.com/so.php?sokeygx=%E4%B8%9B%E4%B9%A6%E9%9B%86%E6%88%90&submit=&kt=1">丛书集成</a> 
<a href="http://www.guoxuedashi.com/so.php?sokeygx=%E5%9B%9B%E9%83%A8%E4%B8%9B%E5%88%8A&submit=&kt=1"><strong>四部丛刊</strong></a>  
<a href="http://www.guoxuedashi.com/so.php?sokeygx=%E8%AF%B4%E6%96%87%E8%A7%A3%E5%AD%97&submit=&kt=1">說文解字</a> <a href="http://www.guoxuedashi.com/so.php?sokeygx=%E5%85%A8%E4%B8%8A%E5%8F%A4&submit=&kt=1">三国六朝文</a>
<br><a href="http://www.guoxuedashi.com/so.php?sokeytm=%E6%97%A5%E6%9C%AC%E5%86%85%E9%98%81%E6%96%87%E5%BA%93&submit=&kt=1"><strong>日本内阁文库</strong></a> <a href="http://www.guoxuedashi.com/so.php?sokeytm=%E5%9B%BD%E5%9B%BE%E6%96%B9%E5%BF%97%E5%90%88%E9%9B%86&ka=100&submit=">国图方志合集</a> <a href="http://www.guoxuedashi.com/so.php?sokeytm=%E5%90%84%E5%9C%B0%E6%96%B9%E5%BF%97&submit=&kt=1"><strong>各地方志</strong></a>

</div>
</div>


<div class="sidebar2">
<center>

</center>
</div>
<div class="sidebar greenbar">
<div class="sidebar_title green">四库全书</div>
<div class="sidebar_info">

《四库全书》是中国古代最大的丛书,编撰于乾隆年间,由纪昀等360多位高官、学者编撰,3800多人抄写,费时十三年编成。丛书分经、史、子、集四部,故名四库。共有3500多种书,7.9万卷,3.6万册,约8亿字,基本上囊括了古代所有图书,故称“全书”。<a href="http://www.guoxuedashi.com/SiKuQuanShu/">详细>>
</a>

</div> 
</div>

</div>  <!--end r-->

</div>
<!-- 内容区END --> 

<!-- 页脚开始 -->
<div class="shh">

</div>

<div class="w1180" style="margin-top:8px;">
<center><script src="http://www.guoxuedashi.com/img/plus.php?id=3"></script></center>
</div>
<div class="w1180 foot">
<a href="/b/thanks.php">特别致谢</a> | <a href="javascript:window.external.AddFavorite(document.location.href,document.title);">收藏本站</a> | <a href="#">欢迎投稿</a> | <a href="http://www.guoxuedashi.com/forum/">意见建议</a> | <a href="http://www.guoxuemi.com/">国学迷</a> | <a href="http://www.shuowen.net/">说文网</a><script language="javascript" type="text/javascript" src="https://js.users.51.la/17753172.js"></script><br />
  Copyright &copy; 国学大师 古典图书集成 All Rights Reserved.<br>
  
  <span style="font-size:14px">免责声明:本站非营利性站点,以方便网友为主,仅供学习研究。<br>内容由热心网友提供和网上收集,不保留版权。若侵犯了您的权益,来信即刪。scp168@qq.com</span>
  <br />
ICP证:<a href="http://www.beian.miit.gov.cn/" target="_blank">鲁ICP备19060063号</a></div>
<!-- 页脚END --> 
<script src="http://www.guoxuedashi.com/img/plus.php?id=22"></script>
<script src="http://www.guoxuedashi.com/img/tongji.js"></script>

</body>
</html>
