<!DOCTYPE html PUBLIC "-//W3C//DTD XHTML 1.0 Transitional//EN" "http://www.w3.org/TR/xhtml1/DTD/xhtml1-transitional.dtd">
<html xmlns="http://www.w3.org/1999/xhtml">
<head>
<meta http-equiv="Content-Type" content="text/html; charset=utf-8" />
<meta http-equiv="X-UA-Compatible" content="IE=Edge,chrome=1">
<title>資治通鑒_9-資治通鑑卷八_9-資治通鑑卷八</title>
<meta name="Keywords" content="資治通鑒_9-資治通鑑卷八_9-資治通鑑卷八">
<meta name="Description" content="資治通鑒_9-資治通鑑卷八_9-資治通鑑卷八">
<meta http-equiv="Cache-Control" content="no-transform" />
<meta http-equiv="Cache-Control" content="no-siteapp" />
<link href="/img/style.css" rel="stylesheet" type="text/css" />
<script src="/img/m.js?2020"></script> 
</head>
<body>
 <div class="ClassNavi">
<a  href="/24shi/">二十四史</a> | <a href="/SiKuQuanShu/">四库全书</a> | <a href="http://www.guoxuedashi.com/gjtsjc/"><font  color="#FF0000">古今图书集成</font></a> | <a href="/renwu/">历史人物</a> | <a href="/ShuoWenJieZi/"><font  color="#FF0000">说文解字</a></font> | <a href="/chengyu/">成语词典</a> | <a  target="_blank"  href="http://www.guoxuedashi.com/jgwhj/"><font  color="#FF0000">甲骨文合集</font></a> | <a href="/yzjwjc/"><font  color="#FF0000">殷周金文集成</font></a> | <a href="/xiangxingzi/"><font color="#0000FF">象形字典</font></a> | <a href="/13jing/"><font  color="#FF0000">十三经索引</font></a> | <a href="/zixing/"><font  color="#FF0000">字体转换器</font></a> | <a href="/zidian/xz/"><font color="#0000FF">篆书识别</font></a> | <a href="/jinfanyi/">近义反义词</a> | <a href="/duilian/">对联大全</a> | <a href="/jiapu/"><font  color="#0000FF">家谱族谱查询</font></a> | <a href="http://www.guoxuemi.com/hafo/" target="_blank" ><font color="#FF0000">哈佛古籍</font></a> 
</div>

 <!-- 头部导航开始 -->
<div class="w1180 head clearfix">
  <div class="head_logo l"><a title="国学大师官网" href="http://www.guoxuedashi.com" target="_blank"></a></div>
  <div class="head_sr l">
  <div id="head1">
  
  <a href="http://www.guoxuedashi.com/zidian/bujian/" target="_blank" ><img src="http://www.guoxuedashi.com/img/top1.gif" width="88" height="60" border="0" title="部件查字,支持20万汉字"></a>


<a href="http://www.guoxuedashi.com/help/yingpan.php" target="_blank"><img src="http://www.guoxuedashi.com/img/top230.gif" width="600" height="62" border="0" ></a>


  </div>
  <div id="head3"><a href="javascript:" onClick="javascript:window.external.AddFavorite(window.location.href,document.title);">添加收藏</a>
  <br><a href="/help/setie.php">搜索引擎</a>
  <br><a href="/help/zanzhu.php">赞助本站</a></div>
  <div id="head2">
 <a href="http://www.guoxuemi.com/" target="_blank"><img src="http://www.guoxuedashi.com/img/guoxuemi.gif" width="95" height="62" border="0" style="margin-left:2px;" title="国学迷"></a>
  

  </div>
</div>
  <div class="clear"></div>
  <div class="head_nav">
  <p><a href="/">首页</a> | <a href="/ShuKu/">国学书库</a> | <a href="/guji/">影印古籍</a> | <a href="/shici/">诗词宝典</a> | <a   href="/SiKuQuanShu/gxjx.php">精选</a> <b>|</b> <a href="/zidian/">汉语字典</a> | <a href="/hydcd/">汉语词典</a> | <a href="http://www.guoxuedashi.com/zidian/bujian/"><font  color="#CC0066">部件查字</font></a> | <a href="http://www.sfds.cn/"><font  color="#CC0066">书法大师</font></a> | <a href="/jgwhj/">甲骨文</a> <b>|</b> <a href="/b/4/"><font  color="#CC0066">解密</font></a> | <a href="/renwu/">历史人物</a> | <a href="/diangu/">历史典故</a> | <a href="/xingshi/">姓氏</a> | <a href="/minzu/">民族</a> <b>|</b> <a href="/mz/"><font  color="#CC0066">世界名著</font></a> | <a href="/download/">软件下载</a>
</p>
<p><a href="/b/"><font  color="#CC0066">历史</font></a> | <a href="http://skqs.guoxuedashi.com/" target="_blank">四库全书</a> |  <a href="http://www.guoxuedashi.com/search/" target="_blank"><font  color="#CC0066">全文检索</font></a> | <a href="http://www.guoxuedashi.com/shumu/">古籍书目</a> | <a   href="/24shi/">正史</a> <b>|</b> <a href="/chengyu/">成语词典</a> | <a href="/kangxi/" title="康熙字典">康熙字典</a> | <a href="/ShuoWenJieZi/">说文解字</a> | <a href="/zixing/yanbian/">字形演变</a> | <a href="/yzjwjc/">金 文</a> <b>|</b>  <a href="/shijian/nian-hao/">年号</a> | <a href="/diming/">历史地名</a> | <a href="/shijian/">历史事件</a> | <a href="/guanzhi/">官职</a> | <a href="/lishi/">知识</a> <b>|</b> <a href="/zhongyi/">中医中药</a> | <a href="http://www.guoxuedashi.com/forum/">留言反馈</a>
</p>
  </div>
</div>
<!-- 头部导航END --> 
<!-- 内容区开始 --> 
<div class="w1180 clearfix">
  <div class="info l">
   
<div class="clearfix" style="background:#f5faff;">
<script src='http://www.guoxuedashi.com/img/headersou.js'></script>

</div>
  <div class="info_tree"><a href="http://www.guoxuedashi.com">首页</a> > <a href="/SiKuQuanShu/fanti/">四库全书</a>
 > <h1>资治通鉴</h1> <!--         下载:【右键另存为】即可 --></div>
  <div class="info_content zj clearfix">
  
<div class="info_txt clearfix" id="show">
<center style="font-size:24px;">9-資治通鑑卷八</center>
    資治通鑑卷八     宋 司馬光 撰<br />
<br />
  胡三省音註<br />
<br />
  秦紀三【起昭陽大荒落盡閼逢敦牂凡二年】<br />
<br />
  二世皇帝下<br />
<br />
  二年冬十月泗川監平將兵圍沛公於豐【泗川郡即泗水郡秦郡置守尉監文穎曰秦時御史監郡若今刺史平人名】沛公出與戰破之令雍齒守豐【雍於用翻姓也風俗通雍姓周文王子雍伯之後】十一月沛公引兵之薛泗川守壯兵敗於薛走至戚沛公左司馬得殺之【壯者泗川守之名班志戚縣屬東海郡括地志沂州臨沂縣冇戚縣故城戚如字如淳將毒翻余以地理考之沛郡之與東海相去頗遠壯兵敗而走未必能至東海之戚班志沛郡有廣戚縣章懷太子賢曰廣戚故城在今徐川沛縣東恐是走至廣戚之戚也師古曰得者司馬之名貢父曰得殺之者得而殺之漢書多以獲為得司馬掌兵周之夏卿春秋之時晉置三軍及新軍各有卿佐復置司馬以掌軍中刑戮之事後復分為左右又其後也軍行有軍司馬假司馬下至部曲有候有司馬】 周章出關止屯曹陽【晉灼曰曹陽亭在弘農東十三里魏武改曰好陽師古曰曹水之陽也其水出陜縣西峴頭山而北流入河今謂之好陽澗在陜縣西四十五里括地志在陜州桃林縣東十四里】二月餘章邯追敗之復走澠池【敗補邁翻復扶又翻】十餘日章邯擊大破之周文自刎軍遂不戰【刎扶粉翻】吳叔圍滎陽李由為三川守守滎陽【秦滅周置三川郡其治所當在洛陽由蓋守滎陽以扞叔宋白曰秦立三川郡初理洛陽後徙滎陽】叔弗能下楚將軍田臧等相與謀曰周章軍已破矣【周章即周文】秦兵旦暮至我圍滎陽城弗能下秦兵至必大敗不如少遺兵守滎陽【遺兵留兵也少詩沼翻】悉精兵迎秦軍今假王驕【陳涉之自王也以吳叔為假王】不知兵權不足與計事恐敗因相與矯王令以誅吳叔【師古曰矯托也託言受王令也】獻其首於陳王陳王使使賜田臧楚令尹印以為上將田臧乃使諸將李歸等守滎陽自以精兵西迎秦軍於敖倉【周宣王狩于敖左傳晉師在敖鄗之間後漢志滎陽有敖亭秦立敖倉孟康曰敖地名在滎陽西北山上臨河有大倉】與戰田臧死軍破章邯進兵擊李歸等滎陽下破之李歸等死陽城人鄧說將兵居郯【師古曰郯東海縣音談索隱曰非也此時章邯軍未至東海此郯别是地名或恐郯當作郟郟是郟鄏之地史記正義曰郟是春秋時郟地楚郟敖葬之今汝州郟縣城是鄧說陽城人陽城河南府縣與郟縣相近又近陳余按索隱以為河南之郟鄏正義以為汝州之郟時章邯兵至滎陽則已過郟鄏而東矣正義之說近之】章邯别將擊破之銍人伍逢將兵居許【伍姓也春秋時楚有伍舉伍奢許春秋許子之國班志屬潁川魏文帝改曰許昌唐為許州】章邯擊破之兩軍皆散走陳陳王誅鄧說 二世數誚讓李斯居三公位如何令盜如此【數所角翻誚七笑翻責也秦以丞相太尉御史大夫為三公漢因之】李斯恐懼重爵禄不知所出乃阿二世意以書對曰夫賢主者必能行督責之術者也【索隱曰督者察也察其罪責之以刑罰也】故申子曰有天下而不恣睢【恣資二翻睢香萃翻謂肆情放縱也】命之曰以天下為桎梏者無佗焉不能督責而顧以其身勞於天下之民若堯禹然故謂之桎梏也【桎梏械也在足曰桎在手曰梏桎職日翻梏姑沃翻】夫不能修申韓之明術行督責之道專以天下自適也而徒務苦形勞神以身徇百姓則是黔首之役非畜天下者也何足貴哉故明主能行督責之術以獨斷於上【斷丁亂翻】則權不在臣下然後能滅仁義之塗絶諫說之辯犖然行恣睢之心【犖呂角翻】而莫之敢逆如此羣臣百姓救過不給何變之敢圖二世說【說讀曰悦】於是行督責益嚴税民深者為明吏殺人衆者為忠臣刑者相半於道而死人日成積於市秦民益駭懼思亂 趙李良已定常山【去年趙王使李良畧常山】還報趙王趙王復使良略太原至石邑秦兵塞井陘未能前【班志石邑縣屬常山郡井陘山在西塞悉則翻陘音刑】秦將詐為二世書以招良良得書未信還之邯鄲益請兵未至道逢趙王姊出飲良望見以為王伏謁道旁王姊醉不知其將【將即亮翻】使騎謝李良李良素貴起慚其從官【拜謁而起顧從官而慙也從才用翻】從官有一人曰天下畔秦能者先立且趙王素出將軍下今女兒乃不為將軍下車請追殺之李良已得秦書固欲反趙未決因此怒遣人追殺王姊因將其兵襲邯鄲邯鄲不知竟殺趙王邵騷趙人多為張耳陳餘耳目者以故二人獨得脱【以故者以此故也】 陳人秦嘉符離人朱雞石等起兵圍東海守於郯【陳當作凌陳勝傳作凌人秦嘉秦姓也春秋時魯有秦堇父班志曰東海郡漢高帝置應劭注曰即秦郯郡余按裴駰所云三十六郡本亦無郯郡漢東海郡則治郯耳】陳王聞之使武平君畔為將軍監郯下軍秦嘉不受命自立為大司馬惡屬武平君【惡烏路翻】告軍吏曰武平君年少不知兵事勿聽因矯以王命殺武平君畔二世益遣長史司馬欣董翳佐章邯擊盜【時章邯為上將將兵東討故使欣為長史以佐之據項籍傳翳為都尉姓譜飂叔安裔子董父好龍帝舜嘉焉因賜姓董】章邯已破伍逢擊陳柱國房君殺之又進擊陳西張賀軍陳王出監戰張賀死【監古衘翻】臘月【張晏曰秦之臘月夏之九月臣瓚曰建丑之月也師古曰史記云胡亥二年十月誅葛嬰十一月周文死十二月陳涉死瓚說是也】陳王之汝隂【之往也】還至下城父【師古曰下城父地名在沛郡城父縣東劉昭曰汝南山桑縣故屬沛有下城父聚父音甫】其御莊賈殺陳王以降【降戶江翻】初陳涉旣為王其故人皆往依之妻之父亦往焉陳王以衆賓待之長揖不拜妻之父怒曰怙亂僭號而傲長者不能久矣不辭而去陳王跪謝遂不為顧客出入愈益發舒言陳王故情或說陳王曰客愚無知顓妄言輕威陳王斬之諸故人皆自引去由是無親陳王者陳王以朱防為中正胡武為司過主司羣臣諸將徇地至令之不是輒繫而罪之以苛察為忠其所不善者弗下吏輒自治之諸將以其故不親附此其所以敗也【史言陳王棄其親故遂死於莊賈之手故先以故人二字發其端乃及慢其妻父事次及客事客先與陳王傭耕及其據陳而王遮道求見陳王載與俱歸後以客言其故情遂殺之輕威者言輕其為君之威重也顓與專同】陳王故涓人將軍呂臣為蒼頭軍【魏有蒼頭二十萬蓋前乎此時已有蒼頭軍矣應劭曰時軍皆著青巾故曰蒼頭服䖍曰蒼頭謂士卒青帛巾若赤眉之號以相識别也】起新陽【班志新陽縣屬汝南郡應劭曰在新水之陽括地志新陽故城在豫州真陽縣西南四十二里】攻陳下之殺莊賈復以陳為楚葬陳王於碭諡曰隱王初陳王令銍人宋留將兵定南陽入武關留已徇南陽聞陳王死南陽復為秦宋留以軍降二世車裂留以徇 魏周市將兵略豐沛使人招雍齒雍齒雅不欲屬沛公即以豐降魏【雅素也】沛公攻之不克 趙張耳陳餘收其散兵得數萬人擊李良良敗走歸章邯客有說耳餘曰兩君覉旅而欲附趙難可獨立立趙後輔以誼可就功乃求得趙歇春正月耳餘立歇為趙王居信都【項羽改信都曰襄國漢復為信都縣屬信都國後漢復曰襄國】 東陽甯君秦嘉【文穎曰秦嘉東陽郡人為甯縣君臣瓚曰陳勝傳凌人秦嘉然則嘉非東陽人嘉起于郯號大司馬又不為甯縣君東陽甯君自一人秦嘉又一人師古曰瓚說是東陽者為所屬縣甯君者姓甯時號為君姓譜衛卿甯氏之後又晉有甯嬴以邑為姓】聞陳王軍敗迺立景駒為楚王引兵之方與欲擊秦軍定陶下【班志方與縣屬山陽郡定陶縣屬濟隂郡史記正義曰方與今濟州縣定陶今曹州縣方與音房預】使公孫慶使齊欲與之并力俱進【并必正翻】齊王曰陳王戰敗不知其死生楚安得不請而立王公孫慶曰齊不請楚而立王楚何故請齊而立王且楚首事當令於天下【首事謂最先起兵伐秦】田儋殺公孫慶秦左右校復攻陳下之【索隱曰左右校即左右校尉校戶教翻】呂將軍走徼兵復聚【如淳曰徼要也徼散卒復相聚也師古曰徼工堯翻余謂從如氏之說當音於堯翻】與番盜黥布相遇【番即番陽縣漢屬豫章郡英布為盜于江中番陽今吳芮妻之以女故謂番盗番蒲何翻】攻擊秦左右校破之青波復以陳為楚黥布者六人也【六春秋之六國也秦為縣屬九江郡漢屬六安國括地志六故城在夀州安豐縣南百三十里宋白曰今蘄州東廣濟縣即秦漢之六縣英布都六古城猶存】姓英氏【姓譜英出自偃姓皋陶之後封於英因以為氏】坐灋黥以刑徒論輸驪山驪山之徒數十萬人布皆與其徒長豪桀交通【長知兩翻】乃率其曹耦【曹輩也】亡之江中為羣盜番陽令吳芮甚得江湖間民心號曰番君布往見之其衆已數千人番君迺以女妻之【妻七細翻】使將其兵擊秦 楚王景駒在留【班志留縣屬楚國括地志留城在徐州沛縣東南五十里即張良封處】沛公往從之張良亦聚少年百餘人欲往從景駒道遇沛公遂屬焉沛公拜良為廏將【廐將蓋掌馬】良數以太公兵灋說沛公沛公善之常用其策良為佗人言皆不省【說輸芮翻為于偽翻下平為同省悉井翻察也後以義推】良曰沛公殆天授故遂留不去【張良從沛公始此】沛公與良俱見景駒欲請兵以攻豐時章邯司馬將兵北定楚地【師古曰古夷字類編曰古仁字又延知翻】屠相至碭【班志相縣為沛郡治所括地志故相城在徐州符離縣西九十里相息亮翻碭徒郎翻】東陽甯君沛公引兵西與戰蕭西【班志蕭縣屬沛郡唐屬徐州蕭西謂在蕭縣之西】不利還收兵聚留二月攻碭三日拔之收碭兵得六千人與故合九千人三月攻下邑拔之【班志下邑縣屬梁國】還擊豐不下 廣陵人召平為陳王徇廣陵未下【廣陵縣屬九江郡班志為廣陵國都唐為揚州姓譜召姓周文王子召公奭之後召實照翻】聞陳王敗走章邯且至迺渡江矯陳王令拜項梁為楚上柱國曰江東已定急引兵西擊秦梁迺以八千人渡江而西聞陳嬰已下東陽【班志東陽縣屬臨淮郡明帝分屬下邳後復分屬廣陵括地志東陽故城在楚川旴眙縣東七十里水經注曰淮隂縣楚漢之間為東陽郡】遣使欲與連和俱西陳嬰者故東陽令史【蘇林曰令史曹史也晉灼曰漢儀注令吏曰令史丞吏曰丞史師古曰晉說是】居縣中素信謹稱為長者東陽少年殺其令相聚得二萬人欲立嬰為王嬰母謂嬰曰自我為汝家婦未嘗聞汝先世之有貴者今㬥得大名不祥不如有所屬事成猶得封侯事敗易以亡非世所指名也【易以豉翻】嬰乃不敢為王謂其軍吏曰項氏世世將家有名於楚今欲舉大事將非其人不可【將即亮翻】我倚名族亡秦必矣其衆從之乃以兵屬梁英布旣破秦軍引兵而東聞項梁西渡淮布與蒲將軍皆以其兵屬焉項梁衆凡六七萬人軍下邳【班志下邳縣屬東海郡應劭曰邳在薛其後徙此故曰下邳臣瓚曰有上邳故曰下邳史記正義曰下邳泗水縣也】景駒秦嘉軍彭城東欲以距梁梁謂軍吏曰陳王先首事戰不利未聞所在今秦嘉倍陳王而立景駒大逆無道【倍蒲妹翻】乃進兵擊秦嘉秦嘉軍敗走追之至胡陵【胡陵即湖陸班志屬山陽郡漢章帝改曰湖陵】嘉還戰一日嘉死軍降景駒走死梁地【梁地故魏地也】梁已并秦嘉軍軍胡陵將引軍而西章邯軍至栗【班志栗縣屬沛郡】項梁使别將朱雞石餘樊君與戰餘樊君死朱雞石軍敗亡走胡陵梁乃引兵入薛誅朱雞石【括地志曰故薛城古薛侯國也在今徐州滕縣界】沛公從騎百餘往見梁梁與沛公卒五千人五大夫將十人沛公還引兵攻豐拔之雍齒犇魏項梁使項羽别攻襄城【班志襄城縣屬潁川郡史記正義曰今許州縣】襄城堅守不下已拔皆阬之還報梁聞陳王定死召諸别將會薛計事沛公亦往焉居鄛人范增年七十【班志居巢縣屬廬江郡春秋楚人圍巢巢國也史記正義曰即夏桀所奔地晉灼曰鄛音勦絶之勦師古音巢】素居家好奇計往說項梁曰陳勝敗固當夫秦滅六國楚最無罪自懷王入秦不反楚人憐之【事見四卷周赧王十九年】至今【當屬上句】故楚南公曰楚雖三戶亡秦必楚【服䖍曰南公南方之老人虞喜志林曰南公者道士識廢興之數知亡秦者必楚漢書藝文志南公十三篇六國時人在隂陽家流臣贊曰楚人怨秦雖三戶足以亡秦】今陳勝首事不立楚後而自立其勢不長今君起江東楚蠭起之將皆争附君者【師古曰蠭古蜂字蠭起如蠭之起言其衆也一說蠭與鋒同言鋒鋭而起者爾雅翼曰蠭近其房輒羣起攻人故曰蠭起之將】以君世世楚將為能復立楚之後也於是項梁然其言乃求得楚懷王孫心於民間為人牧羊夏六月立以為楚懷王從民望也【徐廣曰順民望以其祖諡為號】陳嬰為上柱國封五縣與懷王都盱眙【班志盱眙縣屬臨淮郡史記正義曰今楚州縣阮勝之南兖州記盱眙本春秋善道地宋屬泗州音吁怡】項梁自號為武信君張良說項梁曰君已立楚後而韓諸公子横陽君成最賢可立為王益樹黨項梁使良求韓成立以為韓王以良為司徒與韓王將千餘人西略韓地得數城秦輒復取之往來為游兵潁川【潁川故韓地秦置郡】 章邯已破陳王乃進兵擊魏王於臨濟【後漢志陳留郡平丘縣有臨濟亭水經注曰田儋死處史記正義曰今齊州臨濟縣又曰故城在淄州高苑縣北二里余按正義所云臨濟乃田儋所起狄縣地也非魏王咎所居臨濟也後漢志及水經注為是】魏王使周市出請救於齊楚齊王儋及楚將項它皆將兵隨市救魏【它徒河翻】章邯夜衘枚擊大破齊楚軍於臨濟下【師古曰御枚者止言語讙囂欲令敵人不知其來也周官有衘枚氏枚狀如箸横衘之繣結於項繣結礙也絜繞也蓋為結紐而繞項也衘戶緘翻繣音獲絜音頡】殺齊王及周市魏王咎為其民約降約定自燒殺其弟豹亡走楚楚懷王予魏豹數千人復徇魏地【為于偽翻予讀曰與】齊田榮收其兄儋餘兵東走東阿【班志東阿縣屬東郡括地志東阿故城在濟州東阿縣西南二十五里】章邯追圍之齊人聞田儋死乃立故齊王建之弟假為王田角為相角弟間為將以距諸侯秋七月大霖雨【雨三日以往為霖】武信君引兵攻亢父【亢父音抗甫】聞田榮之急迺引兵擊破章邯軍東阿下章邯走而西田榮引兵東歸齊武信君獨追北使項羽沛公别攻城陽屠之【括地志濮州雷澤縣本漢城陽在州東九十一里余按班志濟隂成陽縣有雷澤此成陽與定陶濮陽皆相近非城陽國之城陽】楚軍軍濮陽東【班志濮陽縣屬東郡括地志濮陽縣在濮州西八十六里濮音卜】復與章邯戰又破之章邯復振【李奇曰振整也如淳曰振起也收散卒自振迅而起】守濮陽環水【文潁曰决水以自環守為固環音宦】沛公項羽去攻定陶八月田榮擊逐齊王假假亡走楚田角亡走趙田間前救趙因留不敢歸田榮迺立儋子市為齊王榮相之田横為將平齊地章邯兵益盛項梁數使使告齊趙發兵共擊章邯田榮曰楚殺田假趙殺角間乃出兵楚趙不許田榮怒終不肯出兵郎中令趙高【班表郎中令秦官掌宫殿掖門戶臣瓚曰掌郎内諸臣故曰郎中令武帝改光】<br />
<br />
  【祿勲】恃恩專恣以私怨誅殺人衆多恐大臣入朝奏事言之乃說二世曰天子之所以貴者但以聞聲羣臣莫得見其面故也且陛下富於春秋【謂少年此去春秋多也】未必盡通諸事今坐朝廷譴舉有不當者【譴去戰翻責也當丁浪翻】則見短於大臣非所以示神明于天下也【見賢遍翻】陛下不如深拱禁中【蔡邕曰本為禁中門閣有禁非侍御之臣不得妄入行道豹尾中亦為禁中】與臣及侍中習灋者待事事來有以揆之如此則大臣不敢奏疑事天下稱聖主矣二世用其計乃不坐朝廷見大臣常居禁中趙高侍中用事【班表秦制侍中左右曹諸吏散騎中常侍皆加官所加或列侯卿大夫將軍將都尉尚書太醫太官令至郎中亡員多至數十人侍中中常侍得入禁中應劭曰入侍天子故曰侍中後漢志侍中比二千石掌侍左右贊導衆事顧問應對】事皆決於趙高高聞李斯以為言乃見丞相曰關東羣盜多今上急益發繇【繇讀曰徭役也古字借用】治阿房宫【冶直之翻】聚狗馬無用之物臣欲諫為位賤【為于偽翻下同】此真君侯之事君何不諫李斯曰固也吾欲言之久矣今時上不坐朝廷常居深宫吾所言者不可傳也欲見無閒【閒古莧翻隙也又讀曰閑餘暇也】趙高曰君誠能諫請為君候上閒語君于是趙高侍二世方燕樂婦女居前使人告丞相上方閒可奏事丞相至宫門上謁如此者三二世怒曰吾常多閒日丞相不來吾方燕私丞相輒來請事丞相豈少我哉且固我哉【少我謂輕我以為幼少也固我謂輕我以為固陋也】趙高因曰夫沙丘之謀丞相與焉【事見上卷始皇三十七年與讀曰預】今陛下已立為帝而丞相貴不益此其意亦望裂地而王矣且陛下不問臣臣不敢言丞相長男李由為三川守楚盜陳勝等皆丞相傍縣之子【傍縣近縣也李斯汝南上蔡人陳勝潁川陽城人汝南潁川相近也】以故楚盜公行過三川城守不肯擊高聞其文書相往來未得其審故未敢以聞且丞相居外權重於陛下二世以為然欲案丞相恐其不審乃先使人按驗三川守與盜通狀李斯聞之因上書言趙高之短曰高擅利擅害與陛下無異昔田常相齊簡公竊其恩威下得百姓上得羣臣卒弑簡公而取齊國【事見左傳卒子恤翻】此天下所明知也今高有邪佚之志危反之行【行下孟翻】私家之富若田氏之於齊矣而又貪欲無厭【厭于鹽翻後以義推】求利不止列勢次主【言趙高居中用事其位列權勢次于人主也】其欲無窮劫陛下之威信其志若韓玘為韓安相也【索隱曰玘一作起並音怡韓大夫殺其君悼公者然韓無悼公或鄭之嗣君案表韓玘事昭侯昭侯以下四世至王安斯說非也余觀李斯書意正以胡亥亡國之禍近在旦夕故指韓安以其用韓玘而亡韓之事警動之韓安之時其臣必有韓玘者特史逸其事耳李斯與韓安同時而韓安亡國之事接乎胡亥之耳目所謂殷鑒不遠也索隱于數百載之下議其說為非可乎信讀曰伸】陛下不圖臣恐其必為變也二世曰何哉夫高故宦人也然不為安肆志不以危易心潔行脩善自使至此以忠得進以信守位朕實賢之【所謂臨亂之君各賢其臣也行下孟翻】而君疑之何也且朕非屬趙君當誰任哉【屬之欲翻】且趙君為人精亷彊力下知人情上能適朕君其勿疑二世雅愛趙高恐李斯殺之乃私告趙高高曰丞相所患者獨高高已死丞相即欲為田常所為是時盜賊益多而關中卒發東擊盜者無已右丞相馮去疾左丞相李斯將軍馮劫進諫曰關東羣盜並起秦發兵誅擊所殺亡甚衆然猶不止盜多皆以戍漕轉作事苦【戍征戍也漕水運也轉陸運也作役作也事苦言其事勞苦也】賦税大也請且止阿房宫作者減省四邊戍轉二世曰凡所為貴有天下者得肆意極欲主重明灋【謂君臣之勢上之所主者重則下之勢輕主重猶言居重也重如字康直龍切非也】下不敢為非以制御四海矣夫虞夏之主貴為天子親處窮苦之實以徇百姓尚何於灋【言尚何事於灋也處昌呂翻】且先帝起諸侯兼天下天下已定外攘四夷以安邊境作宫室以章得意而君觀先帝功業有緒今朕即位二年之間羣盜並起君不能禁又欲罷先帝之所為是上無以報先帝次不為朕盡忠力何以在位下去疾斯劫吏【下遐嫁翻】案責佗罪去疾劫自殺獨李斯就獄二世以屬趙高治之【屬之欲翻】責斯與子由謀反狀皆收捕宗族賓客趙高治斯榜掠千餘【榜音彭笞擊也掠音亮考箠也】不勝痛自誣服【自誣以反而服其罪也勝音升】斯所以不死者自負其辯有功實無反心欲上書自陳幸二世寤而赦之乃從獄中上書曰臣為丞相治民三十餘年矣【治直之翻】逮秦地之陿隘不過千里兵數十萬臣盡薄材隂行謀臣資之金玉使游說諸侯隂脩甲兵飭政教官鬬士尊功臣故終以脅韓弱魏破燕趙夷齊楚卒兼六國虜其王立秦為天子又北逐胡貉【卒子恤翻貉莫客翻北方國豸種】南定百越以見秦之彊【見賢遍翻】更剋畫平斗斛度量文章布之天下以樹秦之名此皆臣之罪也臣當死久矣上幸盡其能力乃得至今願陛下察之書上趙高使吏棄去不奏曰囚安得上書趙高使其客十餘輩詐為御史謁者侍中更往覆訊斯【御史之名周官有之戰國亦有御史秦趙澠池之會各命書其事則皆記事之職至秦漢為糾察之任更迭也覆審也訊問也更工衡翻】斯更以其實對輒使人復榜之後二世使人驗斯斯以為如前終不敢更言辭服奏當上【奏當者獄具而奏當處其罪也漢路温舒曰奏當之成雖咎繇聽之猶以為死有餘辜上時掌翻】二世喜曰微趙君幾為丞相所賣【幾居依翻】及二世所使案三川守由者至則楚兵已擊破之使者來會丞相下吏高皆妄為反辭以相傅會【傅讀曰附凡傅會之傅皆同音】遂具斯五刑論【班志秦法當三族者皆先黥劓斬左右趾笞殺之梟其首俎其骨肉于市其誹謗詈詛者又先斷舌謂之具五刑】腰斬咸陽市斯出獄與其中子俱執【中讀曰仲】顧謂其中子曰吾欲與若復牽黄犬俱出上蔡東門逐狡兎豈可得乎遂父子相哭而夷三族二世乃以趙高為丞相事無大小皆決焉 項梁已破章邯於東阿引兵西北至定陶再破秦軍項羽沛公又與秦軍戰於雍丘【班志雍丘縣屬陳留郡故杞國也史記正義曰雍丘今汴州縣】大破之斬李由項梁益輕秦有驕色宋義諫曰戰勝而將驕卒惰者敗今卒少惰矣秦兵日益臣為君畏之項梁弗聽乃使宋義使於齊道遇齊使者高陵君顯【晉灼曰高陵縣屬琅琊郡】曰公將見武信君乎曰然曰臣論武信君必敗公徐行即免死疾行則及禍二世悉起兵益章邯擊楚軍大破之定陶項梁死時連雨自七月至九月項羽沛公攻外黄未下【班志外黄縣屬陳留郡張晏曰魏郡有内黄故曰外括地志曰故周城即外黄之地在雍丘縣之東】去攻陳留【班志陳留縣屬陳留郡孟康曰留鄭邑也後為陳所并故曰陳留臣瓚曰宋亦有留彭城留是也留屬陳者稱陳留括地志陳留汴州縣在州東五十里】聞武信君死士卒恐乃與將軍呂臣引兵而東徙懷王自盱眙都彭城【班志彭城縣屬楚國彭門記彭祖顓頊之玄孫至商末夀及七百六十七歲今墓猶存故邑號彭城】呂臣軍彭城東項羽軍彭城西沛公軍碭 魏豹下魏二十餘城楚懷王立豹為魏王 後九月【文穎曰即閏九月時律歷廢不知閏故謂之後九月如淳曰時因秦以十月為歲首至九月則歲終後九月即閏月師古曰文說非也若以律歷廢不知閏者則當徑謂之十月不應有後九月蓋秦之歷法應置閏者總置于歲末此意當取左傳歸餘于終耳何以明之據漢表及史記漢未改秦歷之前迄至高后文帝屢書後九月是知固然非歷廢也貢父曰予謂顔說後九月亦為未盡秦知置歷有閏何故皆以為九月乎蓋司馬氏為史記既以秦正月稱十月遂以閏月轉為後九月是司馬氏如此叙之非秦法也】楚懷王并呂臣項羽軍自將之以沛公為碭郡長【蘇林曰長如郡守也】封武安侯將碭郡兵封項羽為長安侯號為魯公呂臣為司徒其父呂青為令尹章邯已破項梁以為楚地兵不足憂乃度河北擊趙大破之引兵至邯鄲皆徙其民河内夷其城郭張耳與趙王歇走入鉅鹿城【班志鉅鹿縣屬鉅鹿郡應劭曰鹿林之大者臣瓚曰山足曰鹿括地志曰今邢州平郷城本鉅鹿宋白曰十三州志鉅鹿堯時大麓之地禹為大陸之野秦滅趙置鉅鹿郡鉅亦大稱也】王離圍之陳餘北收常山兵得數萬人軍鉅鹿北章邯軍鉅鹿南棘原趙數請救於楚【數所角翻下同】高陵君顯在楚見楚王曰宋義論武信君之軍必敗居數日軍果敗兵未戰而先見敗徵【徵讀曰證】此可謂知兵矣王召宋義與計事而大說之【說讀曰悦】因置以為上將軍項羽為次將范增為末將以救趙諸别將皆屬宋義號為卿子冠軍【如淳曰卿者大夫之號子者子男之爵冠軍人之首也文頴曰卿子人相褒尊之稱猶言公子也上將故言冠軍劉伯莊曰公之子為公子卿子謂卿之子也師古曰冠軍言其在諸軍之上】初楚懷王與諸將約先入定關中者王之【秦地西有隴關東有函谷關南有武關北有臨晉關西南有散關秦地居其中故謂之關中注已見前】當是時秦兵彊常乘勝逐北諸將莫利先入關【言莫有以入關為利者蓋畏秦也】獨項羽怨秦之殺項梁奮願與沛公西入關懷王諸老將皆曰項羽為人慓悍猾賊【慓疾也悍勇也猾狡也賊殘害也慓頻妙翻又匹妙翻悍下旦翻又下罕翻】嘗攻襄城襄城無遺類皆阬之諸所過無不殘滅且楚數進取前陳王項梁皆敗不如更遣長者扶義而西【師古曰扶助也以義自助也余謂扶義猶言仗義也】告諭秦父兄秦父兄苦其主久矣今誠得長者往無侵㬥宜可下項羽不可遣獨沛公素寛大長者可遣懷王乃不許項羽而遣沛公西略地收陳王項梁散卒以伐秦沛公道碭至陽城與杠里【道碭自碭取道而西也此據班書書之陽城史記作成陽韋昭注曰在潁川則是謂陽城也索隱曰在濟隂則是謂成陽也杠里孟康服䖍皆以為縣名而班志無之余按沛公之兵自碭而攻秦道成陽與杠里而後破東郡尉于成武成陽縣屬濟隂成武縣屬山陽濟隂唐為曹州成武屬焉若取道潁川之陽城當自此西趨洛陜安得復至成武耶書成陽為是杠里之地蓋在成陽成武之間杠音江】攻秦壁破其二軍<br />
<br />
  三年冬十月齊將田都畔田榮助楚救趙【為項羽封田都張本】沛公攻破東郡尉於成武【秦㓕衛置東郡尉郡尉也班志成武即衛楚丘也括】<br />
<br />
  【地志今曹川縣】 宋義行至安陽【師古曰今相州安陽縣索隱曰傅寛傳云從攻安陽杠里則當俱在河南師古以為相州縣按此兵猶未度河不應即至相州安陽後魏書地形志已氏有安陽城後改已氏為楚丘今宋州楚丘西北四十里有安陽故城是也】留四十六日不進項羽曰秦圍趙急宜疾引兵渡河楚擊其外趙應其内破秦軍必矣宋義曰不然夫搏牛之蝱不可以破蟣蝨【蘇林曰蝱喻秦蝨喻章邯等言小大不同勢欲滅秦當先寛邯等也如淳曰言本欲以大力伐秦而不可以救趙也師古曰搏擊也言以手擊牛之背可以殺其上蝱而不能破蝨今將兵力欲滅秦不可盡力與邯戰即未能禽徒費力也如說近之搏音博蝱音盲蟣居喜翻蝨音瑟】今秦攻趙戰勝則兵疲我承其敝不勝則我引兵皷行而西必舉秦矣【鼓行者擊鼓而行堂堂之陳也】故不如先鬬秦趙夫被堅執鋭義不如公坐運籌策公不如義因下令軍中曰有猛如虎狠如羊【狠何墾翻此併下三語指項羽也】貪如狼彊不可使者皆斬之乃遣其子宋襄相齊身送之至無鹽【班志東平國冇無鹽縣索隱曰在今鄆州之東】飲酒高會【師古曰高會者大會也】天寒大雨士卒凍饑項羽曰將戮力而攻秦久留不行今歲饑民貧士卒食半菽【菽豆也臣瓚曰士卒食蔬菜以菽雜半之】軍無見糧【言軍無見在之糧見賢遍翻】乃飲酒高會不引兵渡河因趙食與趙并力攻秦乃曰承其敝夫以秦之彊攻新造之趙其勢必舉趙舉秦彊何敝之承且國兵新破王坐不安席埽境内而專屬於將軍【屬之欲翻下道屬同】國家安危在此一舉今不恤士卒而徇其私非社稷之臣也【徇其私謂身送其子相齊也】十一月項羽晨朝上將軍宋義【朝直遙翻】即其帳中斬宋義頭出令軍中曰宋義與齊謀反楚楚王隂令籍誅之當是時諸將皆慴服莫敢枝梧【如淳曰枝梧猶枝扞也臣瓚曰小柱為枝邪柱為梧今屋極邪柱也】皆曰首立楚者將軍家也今將軍誅亂乃相與共立羽為假上將軍【以未得懷王之命故目為假】使人追宋義子及之齊殺之使桓楚報命於懷王懷王因使羽為上將軍十二月沛公引兵至栗遇剛武侯【應劭曰剛武侯楚懷王將功臣表棘蒲剛侯陳武武一姓柴宜為剛侯武魏將也孟康曰功臣表以將軍起薛至霸上入漢中非懷王將又非魏將例未有稱諡者師古曰史失其姓名惟識其爵號不知誰也不當改剛武侯為剛侯武應說非也】奪其軍四千餘人并之與魏將皇欣武滿軍合攻秦軍破之【皇姓也左傳鄭有大夫皇頡】 故齊王建孫安下濟北【濟水以北之地聊城博陽諸城是也】從項羽救趙【為項羽王田安張本】 章邯築甬道屬河餉王離【恐敵抄其糧運故夾築垣牆以通餉道屬之欲翻餉式亮翻】王離兵食多急攻鉅鹿鉅鹿城中食盡兵少張耳數使人召前陳餘【召前者召陳餘使前救鉅鹿也】陳餘度兵少不敵秦不敢前【度徒洛翻下同】數月張耳大怒怨陳餘使張黶陳澤往讓陳餘曰【史記正義澤音釋】始吾與公為刎頸交今王與耳旦暮且死而公擁兵數萬不肯相救安在其相為死【相為於偽翻下欲為同】苟必信胡不赴秦軍俱死且有十一二相全【言十分之中冀有一二分得以勝秦而相保全也】陳餘曰吾度前終不能救趙徒盡亡軍【度徒洛翻】且餘所以不俱死欲為趙王張君報秦今必俱死如以肉委餓虎何益張黶陳澤要以俱死餘乃使黶澤將五千人先嘗秦軍【嘗試也】至皆沒當是時齊師燕師皆來救趙張敖亦北收代兵得萬餘人來【張敖耳之子也】皆壁餘旁未敢擊秦項羽已殺卿子冠軍威震楚國乃遣當陽君蒲將軍將卒二萬渡河救鉅鹿戰少利【言其戰略有利也】絶章邯甬道王離軍乏食陳餘復請兵【復扶又翻】項羽乃悉引兵渡河皆沈船破釡甑燒廬舍持三日糧以示士卒必死無一還心於是至則圍王離與秦軍遇九戰大破之章邯引兵郤諸侯兵乃敢進擊秦軍遂殺蘇角虜王離涉閒不降自燒殺【涉姓也閒名也】當是時楚兵冠諸侯【冠古玩翻】軍救鉅鹿者十餘壁莫敢縱兵及楚擊秦諸侯將從壁上觀楚戰士無不一當十呼聲動天地【將即亮翻呼火故翻】諸侯軍無不人人惴恐【惴之睡翻】於是已破秦軍項羽召見諸侯將諸侯將入轅門【張晏曰軍行以車為陳轅相向為門師古曰周禮掌舍王行則設東宮轅門杜估曰昂車以其轅表門】無不膝行而前莫敢仰視項羽由是始為諸侯上將軍諸侯皆屬焉於是趙王歇及張耳乃得出鉅鹿城謝諸侯張耳與陳餘相見責讓陳餘以不肯救趙及問張黶陳澤所在疑陳餘殺之數以問餘【數所角翻】餘怒曰不意君之望臣深也【望怨望也又責望也爾雅翼曰怨者必望故以望為怨不意君之望臣深是也】豈以臣為重去將印哉【重難也言豈以去將印為難也豈疑辭重如字】乃脱解印綬推與張耳【推通囘翻】張耳亦愕不受陳餘起如厠客有說張耳曰臣聞天與不取反受其咎【索隱曰此辭出國語】今陳將軍與君印君不受反天不祥急取之張耳乃佩其印收其麾下而陳餘還亦望張耳不讓遂趨出獨與麾下所善數百人之河上澤中漁獵【為張耳陳餘相攻殺張本】趙王歇還信都春二月沛公北擊昌邑【班志昌邑縣屬山陽郡括地志曰曹州成武縣東北三十二里有梁丘故城是也賢曰昌邑故城在兖州金鄉縣西北】遇彭越彭越以其兵從沛公【姓譜彭姓大彭之後】越昌邑人常漁鉅野澤中為羣盜【班志山陽郡鉅野縣有大野澤鉅野縣唐屬鄆州】陳勝項梁之起澤間少年相聚百餘人往從彭越曰請仲為長【彭越字仲長知兩翻下同】越謝曰臣不願也少年彊請乃許【彊其兩翻】與期旦日日出會【索隱曰旦日謂明日之朝日出時也】後期者斬旦日日出十餘人後後者至日中於是越謝曰臣老諸君彊以為長今期而多後不可盡誅誅最後者一人令校長斬之【校長一校之長】皆笑曰何至於是請後不敢於是越引一人斬之設壇祭令徒屬皆大驚莫敢仰視乃略地收諸侯散卒得千餘人遂助沛公攻昌邑昌邑未下沛公引兵西過高陽【文穎曰高陽聚邑名屬陳留圉縣臣瓚曰陳留傳高陽在雍丘西南水經注睢水首受陳留浚儀蒗蕩水東逕高陽故亭北】高陽人酈食其家貧落魄【酈音歷姓譜黄帝之支孫封於酈後以為氏食其音異基鄭氏曰魄音薄應邵曰落魄志行衰薄之貌師古曰落魄失業無次也】為里監門沛公麾下騎士適食其里中人食其見謂曰諸侯將過高陽者數十人吾問其將皆握齪【握齪急促貌苛細也齪初角翻】好苛禮自用不能聽大度之言吾聞沛公慢而易人多大略【易以豉翻】此真吾所願從游莫為我先【索隱曰先謂先容言無人為我作紹介也】若見沛公【若汝也】謂曰臣里中有酈生年六十餘長八尺人皆謂之狂生生自謂我非狂生騎士曰沛公不好儒諸客冠儒冠來者【客冠古玩翻】沛公輒解其冠溲溺其中【溲所由翻溺乃弔翻溲即溺也】與人言常大罵未可以儒生說也酈生曰第言之【第但也】騎士從容言如酈生所誡者【從千容翻】沛公至高陽傳舍【師古曰傳置之舍人所止息前人已去後人復來轉相傳也傳張戀翻】使人召酈生酈生至入謁沛公方倨牀使兩女子洗足而見酈生【倨與踞同洗先典翻樂彦曰牀邊曰倨】酈生入則長揖不拜曰足下欲助秦攻諸侯乎且欲率諸侯破秦也沛公罵曰豎儒天下同共苦秦久矣故諸侯相率而攻秦何謂助秦攻諸侯乎酈生曰必聚徒合義兵誅無道秦不宜倨見長者於是沛公輟洗起攝衣【史記正義曰攝斂著也余謂攝衣起而持其衣也】延酈生上坐謝之酈生因言六國縱横時沛公喜賜酈生食問曰計將安出酈生曰足下起糾合之衆收散亂之兵不滿萬人欲以徑入彊秦此所謂探虎口者也【探吐南翻】夫陳留天下之衝四通五達之郊也【如淳曰四面往來通之并數中央為五達也臣瓚曰四通五達言無險阻】今其城中又多積粟臣善其令請得使之令下足下【令下之令力丁翻使也下降也】即不聽足下引兵攻之臣為内應於是遣酈生行沛公引兵隨之遂下陳留號酈食其為廣野君酈生言其弟商時商聚少年得四千人來屬沛公沛公以為將將陳留兵以從酈生常為說客使諸侯 三月沛公攻開封未拔【班志開封縣屬河南郡宋白曰今縣南五十里開封古城是漢理所】西與秦將楊熊會戰白馬又戰曲遇東【後漢志河南中牟縣有曲遇聚蘇林曰曲音齲遇音顒師古曰齲音丘羽翻】大破之楊熊走之滎陽二世使使者斬之以徇夏四月沛公南攻潁川屠之【潁川郡治陽翟】因張良遂略韓地【文穎曰河南新鄭南至潁川皆韓地也張良家世相韓故因之】時趙别將司馬卬方欲度河入關沛公乃北攻平隂【班志平隂縣屬河南郡史記正義曰今河隂是】絶河津南戰洛陽東軍不利南出轘轅【後漢志河南緱氏縣有轘轅關臣瓚曰險道名也在緱氏縣東南索隱曰轘轅為九十二曲是險道也轘音環】張良引兵從沛公沛公令韓王成留守陽翟與良俱南六月與南陽守齮戰犨東破之【齮魚豈翻班志犨縣屬南陽郡水經注滍水出魯陽縣西逕犨縣故城北犨昌牛翻】略南陽郡南陽守走保城守宛【宛南陽郡治所括地志曰宛故城在宛大城之南隅其西南有二面是師古曰宛於元翻】沛公引兵過宛西張良諫曰沛公雖欲急入關秦兵尚衆距險【依險以距敵也】今不下宛宛從後擊彊秦在前此危道也於是沛公乃夜引軍從他道還偃旗幟【旗旂之屬幟即幖也或曰旗幟總稱幟昌志翻】遲明圍宛城三匝【文穎曰遲未也天未明之頃已圍其城矣師古曰文說得其大意耳此言圍城事畢然後天明明遲於事故曰遲明變為去聲音文二翻】南陽守欲自剄其舍人陳恢曰死未晚也乃踰城見沛公曰臣聞足下約先入咸陽者王之今足下留守宛宛郡縣連城數十其吏民自以為降必死故皆堅守乘城今足下盡日止攻士死傷者必多引兵去宛宛必隨足下後足下前則失咸陽之約後有彊宛之患為足下計莫若約降封其守因使止守引其甲卒與之西諸城未下者聞聲争開門而待足下足下通行無所累【累力瑞翻】沛公曰善秋七月南陽守齮降封為殷侯封陳恢千戶引兵西無不下者至丹水【班志丹水縣屬弘農郡括地志曰故丹城在酆州内鄉縣西南百二十里南去丹水二百步汲冢紀年曰后稷放帝子丹朱於丹水輿地志云即秦時丹水縣】高武侯鰓襄侯王陵降【鰓音魚鰓之鰓先才翻人名也史失其姓韋昭曰漢封王陵為安國侯陵初起兵在南陽南陽有穰縣疑襄當為穰而魚禾字省耳臣瓚曰時韓成封穰侯江夏有襄是陵所封也師古曰王陵亦非安國者韋昭改襄為穰蓋亦穿鑿索隱曰王陵封安國侯是定天下為丞相時封耳此言襄侯當如臣瓚解蓋初封江夏之襄也】還攻胡陽遇番君别將梅鋗【姓譜梅本自子姓殷有梅伯為紂所醢】與偕攻析酈皆降【班志南陽郡有湖陽縣故廖國析縣屬弘農郡本楚之白羽也酈縣屬南陽郡師古曰析今内鄉縣酈今菊潭縣鋗呼玄翻析先歷翻酈直益翻又郎益翻廖力救翻】所過亡得鹵掠【亡古毋無二字通鹵與虜同】秦民皆喜 王離軍既沒章邯軍棘原項羽軍漳南【括地志濁漳水一名漳水今俗名柳河在邢州平鄉縣南】相持未戰秦軍數却【數所角翻】二世使人讓章邯章邯恐使長史欣請事至咸陽留司馬門【師古曰凡言司馬門者宫垣之内兵衛所在四面皆有司馬主武事總言之外門為司馬門】三日趙高不見有不信之心長史欣恐還走其軍【走音奏】不敢出故道趙高果使人追之不及欣至軍報曰趙高用事於中下無可為者今戰能勝高必嫉妬吾功不能勝不免於死願將軍孰計之【孰古熟字通後以義推】陳餘亦遺章邯書曰白起為秦將南征鄢郢北阬馬服攻城略地不可勝計而竟賜死【馬服謂趙括也白起事並見五卷赧王紀遺於季翻勝音升】蒙恬為秦將北逐戎人開榆中地數千里竟斬陽周【事見上卷始皇紀】何者功多秦不能盡封因以灋誅之今將軍為秦將三歲矣所亡失以十萬數而諸侯並起滋益多彼趙高素諛日久今事急亦恐二世誅之故欲以灋誅將軍以塞責使人更代將軍以脱其禍夫將軍居外久多内郤【塞悉責翻更工衡翻郤讀曰隙】有功亦誅無功亦誅且天之亡秦無愚智皆知之今將軍内不能直諫外為亡國將孤特獨立而欲常存豈不哀哉將軍何不還兵與諸侯為從【從子容翻】約共攻秦分王其地南面稱孤此孰與身伏鈇質妻子為戮乎【何休曰伏鈇質要斬之罪崔浩曰質斬人椹也師古曰質謂鍖也古者斬人加於鍖上而斫之鈇音夫又匪父翻】章邯狐疑【狐性多疑每度河聽水且聽且度故以喻人之懷疑不決者】隂使侯始成使項羽【鄭氏曰侯軍侯也始姓也成名也】欲約約未成項羽使蒲將軍日夜引兵度三戶【服䖍曰三戶漳水津也孟康曰在鄴西三十里水經注曰漳水東逕三戶峽為三戶津括地志三戶津在相州陽縣界】軍漳南與秦軍戰再破之項羽悉引兵擊秦軍汗水上【水經注汙水出武安山東南退汙城北入漳郡國志鄴縣有汙城師古曰汙水在鄴西南史記正義曰汙水源出懷州河内縣太行山又云故邗城在河内縣西北二十七里古邗國地也余據此時章邯與項羽相持於邢相之間正義以為河内汙水非也汙音于】大破之章邯使人見項羽欲約項羽召軍吏謀曰糧少欲聽其約軍吏皆曰善項羽乃與期洹水南殷虛上【應劭曰洹水在湯隂界殷虚故殷都也臣瓚曰洹水在今安陽縣北去朝歌殷都一百五十里然則此殷虛非朝歌也汲冢古文曰昔殷盤庚遷於北冢曰殷虚南去鄴三十里是舊殷乎然則朝歌非盤庚所遷者索隱曰按釋例洹水出汲郡林慮縣東北至長樂入清水是也汲冢古文曰盤庚自奄遷於北冢曰殷虛南去鄴三十里是殷虛舊地名號北冢也宋白曰相州安陽縣其地即紂之都戰國策云紂聚兵百萬左飲淇水竭右飲洹水不流按邑地在淇洹二水之間本殷墟所謂北冢即此地此國時為魏寧新中邑史記秦昭襄王拔魏寧新中邑更名安陽虛讀與墟字同】已盟章邯見項羽而流涕為言趙高項羽乃立章邯為雍王【為言之為于偽翻雍于用翻】置楚軍中使長史欣為上將軍將秦軍為前行【行戶剛翻】 瑕丘申陽下河南引兵從項羽【服䖍曰瑕丘縣名申姓陽名也班志山陽郡有瑕丘縣瑕音遐河南即漢河南郡地姓譜四岳之後封於申周有申伯左傳齊有申鮮虞楚有申叔】 初中丞相趙高【史記李斯既死二世拜趙高為中丞相蓋以其宦人得入禁中】欲專秦權恐羣臣不聽乃先設驗持鹿獻於二世曰馬也二世笑曰丞相誤邪謂鹿為馬問左右或默或言馬以阿順趙高或言鹿者高因陰中諸言鹿者以灋【中竹仲翻】後羣臣皆畏高莫敢言其過高前數言關東盜無能為也及項羽虜王離等而章邯等軍數敗【數所角翻】上書請益助自關以東大抵盡畔秦吏應諸侯諸侯咸率其衆西鄉【鄉讀曰嚮】八月沛公將數萬攻武關屠之高恐二世怒誅及其身乃謝病不朝見二世夢白虎齧其左驂馬殺之【以馬駕車夾轅曰服兩旁曰驂驂七含翻】心不樂【樂音洛】怪問占夢【周禮春官之屬有占夢掌其歲時觀天地之會辨陰陽之氣以日月星辰占六夢之吉凶一曰正夢二曰噩夢三曰思夢四曰寤夢五曰喜夢六曰懼夢云】卜曰涇水為祟【祟雖遂翻人禍也鬼厲也】二世乃齋於望夷宮【張晏曰望夷宮在長陵西北長平觀道東故亭處是也臨涇水作之以望北夷括地志曰在雍州咸陽縣東南八里】欲祠涇水沈四白馬【沈持林翻】使使責讓高以盜賊事高懼乃陰與其壻咸陽令閻樂【姓譜太伯曾孫仲弈封於閻鄉又曰唐叔虞之後晉成公子懿食采於閻左傳齊有閻職晋有閻嘉】及弟趙成謀曰上不聽諫今事急欲歸禍於吾欲易置上更立子嬰子嬰仁儉百姓皆載其言乃使郎中令為内應詐為有大賊令樂召吏發卒追劫樂毋置高舍遣樂將吏卒千餘人至望夷宮殿門縛衛令僕射曰【衛尉掌宮門屯兵其屬有衛士令秦官自侍中尚書博士郎及軍屯吏騶永巷皆有僕射取其領事之號】賊入此何不止衛令曰周廬設卒甚謹【胡廣曰周廬者衛士於周垣内為區廬師古曰區廬者今之仗宿屋薛綜曰士傅宮外向為盧舍晝則巡行非常夜則警備不虞】安得賊敢入宮樂遂斬衛令直將吏入行射郎宦者【射而亦翻郎屬郎中令宦者屬少府】郎宦者大驚或走或格格者輒死死者數十人郎中令與樂俱入射上幄坐幃【三禮圖曰上下四旁悉周曰幄幄乙角翻幃羽非翻單帳也】二世怒召左右左右皆惶擾不鬬旁有宦者一人侍不敢去二世入内謂曰公何不早告我乃至於此宦者曰臣不敢言故得全使臣早言皆已誅安得至今閻樂前即二世【即就也】數曰足下驕恣誅殺無道天下共畔足下足下其自為計二世曰丞相可得見否樂曰不可二世曰吾願得一郡為王弗許又曰願為萬戶侯弗許曰願與妻子為黔首比諸公子閻樂曰臣受命於丞相為天下誅足下【為于偽翻】足下雖多言臣不敢報麾其兵進二世自殺閻樂歸報趙高趙高乃悉召諸大臣公子告以誅二世之狀曰秦故王國始皇君天下故稱帝今六國復自立秦地益小乃以空名為帝不可宜如故便乃立子嬰為秦王以黔首葬二世杜南宜春苑中【宜春苑在杜縣南漢起宜春觀於此地】九月趙高令子嬰齋戒當廟見受玉璽【玉璽即以卞和玉所刻傳國璽見賢遍翻】齋五日子嬰與其子二人謀曰丞相高殺二世望夷宮恐羣臣誅之乃詐以義立我我聞趙高乃與楚約滅秦宗室而分王關中今使我齋見廟此欲因廟中殺我我稱病不行丞相必自來來則殺之高使人請子嬰數輩子嬰不行高果自往曰宗廟重事王奈何不行子嬰遂刺殺高於齋宮【刺七亦翻】三族高家以狥遣將將兵距嶢關【應劭曰嶢山之關李奇曰在上洛北藍田南武門之西土地記嶢關地名嶢柳道通荆州晉地道記曰嶢關當上洛西北嶢音堯】沛公欲擊之張良曰秦兵尚彊未可輕願先遣人益張旗幟於山上為疑兵使酈食其陸賈往說秦將㗖以利【師古曰㗖者本為食㗖耳音徒敢翻以食餧人令其㗖食音則改變為徒濫翻今言以利誘之取食為譬】秦將果欲連和沛公欲許之張良曰此獨其將欲叛恐其士卒不從不如因其懈怠擊之沛公引兵繞嶢關踰蕢山【宋敏求長安志曰嶢關即藍田關在縣東南九十里蕢山在縣東南二十五里蕢鄭氏音匱師古從蘇林音削】擊秦軍大破之藍田南遂至藍田又戰其北秦兵大敗<br />
<br />
  資治通鑑卷八<br />
<br />
<史部,編年類,資治通鑑>  <br>
   </div> 

<script src="/search/ajaxskft.js"> </script>
 <div class="clear"></div>
<br>
<br>
 <!-- a.d-->

 <!--
<div class="info_share">
</div> 
-->
 <!--info_share--></div>   <!-- end info_content-->
  </div> <!-- end l-->

<div class="r">   <!--r-->



<div class="sidebar"  style="margin-bottom:2px;">

 
<div class="sidebar_title">工具类大全</div>
<div class="sidebar_info">
<strong><a href="http://www.guoxuedashi.com/lsditu/" target="_blank">历史地图</a></strong>  
<a href="http://www.880114.com/" target="_blank">英语宝典</a>  
<a href="http://www.guoxuedashi.com/13jing/" target="_blank">十三经检索</a> 
<br><strong><a href="http://www.guoxuedashi.com/gjtsjc/" target="_blank">古今图书集成</a></strong> 
<a href="http://www.guoxuedashi.com/duilian/" target="_blank">对联大全</a> <strong><a href="http://www.guoxuedashi.com/xiangxingzi/" target="_blank">象形文字典</a></strong> 

<br><a href="http://www.guoxuedashi.com/zixing/yanbian/">字形演变</a>  <strong><a href="http://www.guoxuemi.com/hafo/" target="_blank">哈佛燕京中文善本特藏</a></strong>
<br><strong><a href="http://www.guoxuedashi.com/csfz/" target="_blank">丛书&方志检索器</a></strong> <a href="http://www.guoxuedashi.com/yqjyy/" target="_blank">一切经音义</a>  

<br><strong><a href="http://www.guoxuedashi.com/jiapu/" target="_blank">家谱族谱查询</a></strong>  <strong><a href="http://shufa.guoxuedashi.com/sfzitie/" target="_blank">书法字帖欣赏</a></strong> 
<br>

</div>
</div>


<div class="sidebar" style="margin-bottom:0px;">

<font style="font-size:22px;line-height:32px">QQ交流群9:489193090</font>


<div class="sidebar_title">手机APP 扫描或点击</div>
<div class="sidebar_info">
<table>
<tr>
	<td width=160><a href="http://m.guoxuedashi.com/app/" target="_blank"><img src="/img/gxds-sj.png" width="140"  border="0" alt="国学大师手机版"></a></td>
	<td>
<a href="http://www.guoxuedashi.com/download/" target="_blank">app软件下载专区</a><br>
<a href="http://www.guoxuedashi.com/download/gxds.php" target="_blank">《国学大师》下载</a><br>
<a href="http://www.guoxuedashi.com/download/kxzd.php" target="_blank">《汉字宝典》下载</a><br>
<a href="http://www.guoxuedashi.com/download/scqbd.php" target="_blank">《诗词曲宝典》下载</a><br>
<a href="http://www.guoxuedashi.com/SiKuQuanShu/skqs.php" target="_blank">《四库全书》下载</a><br>
</td>
</tr>
</table>

</div>
</div>


<div class="sidebar2">
<center>


</center>
</div>

<div class="sidebar"  style="margin-bottom:2px;">
<div class="sidebar_title">网站使用教程</div>
<div class="sidebar_info">
<a href="http://www.guoxuedashi.com/help/gjsearch.php" target="_blank">如何在国学大师网下载古籍?</a><br>
<a href="http://www.guoxuedashi.com/zidian/bujian/bjjc.php" target="_blank">如何使用部件查字法快速查字?</a><br>
<a href="http://www.guoxuedashi.com/search/sjc.php" target="_blank">如何在指定的书籍中全文检索?</a><br>
<a href="http://www.guoxuedashi.com/search/skjc.php" target="_blank">如何找到一句话在《四库全书》哪一页?</a><br>
</div>
</div>


<div class="sidebar">
<div class="sidebar_title">热门书籍</div>
<div class="sidebar_info">
<a href="/so.php?sokey=%E8%B5%84%E6%B2%BB%E9%80%9A%E9%89%B4&kt=1">资治通鉴</a> <a href="/24shi/"><strong>二十四史</strong></a>&nbsp; <a href="/a2694/">野史</a>&nbsp; <a href="/SiKuQuanShu/"><strong>四库全书</strong></a>&nbsp;<a href="http://www.guoxuedashi.com/SiKuQuanShu/fanti/">繁体</a>
<br><a href="/so.php?sokey=%E7%BA%A2%E6%A5%BC%E6%A2%A6&kt=1">红楼梦</a> <a href="/a/1858x/">三国演义</a> <a href="/a/1038k/">水浒传</a> <a href="/a/1046t/">西游记</a> <a href="/a/1914o/">封神演义</a>
<br>
<a href="http://www.guoxuedashi.com/so.php?sokeygx=%E4%B8%87%E6%9C%89%E6%96%87%E5%BA%93&submit=&kt=1">万有文库</a> <a href="/a/780t/">古文观止</a> <a href="/a/1024l/">文心雕龙</a> <a href="/a/1704n/">全唐诗</a> <a href="/a/1705h/">全宋词</a>
<br><a href="http://www.guoxuedashi.com/so.php?sokeygx=%E7%99%BE%E8%A1%B2%E6%9C%AC%E4%BA%8C%E5%8D%81%E5%9B%9B%E5%8F%B2&submit=&kt=1"><strong>百衲本二十四史</strong></a>  <a href="http://www.guoxuedashi.com/so.php?sokeygx=%E5%8F%A4%E4%BB%8A%E5%9B%BE%E4%B9%A6%E9%9B%86%E6%88%90&submit=&kt=1"><strong>古今图书集成</strong></a>
<br>

<a href="http://www.guoxuedashi.com/so.php?sokeygx=%E4%B8%9B%E4%B9%A6%E9%9B%86%E6%88%90&submit=&kt=1">丛书集成</a> 
<a href="http://www.guoxuedashi.com/so.php?sokeygx=%E5%9B%9B%E9%83%A8%E4%B8%9B%E5%88%8A&submit=&kt=1"><strong>四部丛刊</strong></a>  
<a href="http://www.guoxuedashi.com/so.php?sokeygx=%E8%AF%B4%E6%96%87%E8%A7%A3%E5%AD%97&submit=&kt=1">說文解字</a> <a href="http://www.guoxuedashi.com/so.php?sokeygx=%E5%85%A8%E4%B8%8A%E5%8F%A4&submit=&kt=1">三国六朝文</a>
<br><a href="http://www.guoxuedashi.com/so.php?sokeytm=%E6%97%A5%E6%9C%AC%E5%86%85%E9%98%81%E6%96%87%E5%BA%93&submit=&kt=1"><strong>日本内阁文库</strong></a> <a href="http://www.guoxuedashi.com/so.php?sokeytm=%E5%9B%BD%E5%9B%BE%E6%96%B9%E5%BF%97%E5%90%88%E9%9B%86&ka=100&submit=">国图方志合集</a> <a href="http://www.guoxuedashi.com/so.php?sokeytm=%E5%90%84%E5%9C%B0%E6%96%B9%E5%BF%97&submit=&kt=1"><strong>各地方志</strong></a>

</div>
</div>


<div class="sidebar2">
<center>

</center>
</div>
<div class="sidebar greenbar">
<div class="sidebar_title green">四库全书</div>
<div class="sidebar_info">

《四库全书》是中国古代最大的丛书,编撰于乾隆年间,由纪昀等360多位高官、学者编撰,3800多人抄写,费时十三年编成。丛书分经、史、子、集四部,故名四库。共有3500多种书,7.9万卷,3.6万册,约8亿字,基本上囊括了古代所有图书,故称“全书”。<a href="http://www.guoxuedashi.com/SiKuQuanShu/">详细>>
</a>

</div> 
</div>

</div>  <!--end r-->

</div>
<!-- 内容区END --> 

<!-- 页脚开始 -->
<div class="shh">

</div>

<div class="w1180" style="margin-top:8px;">
<center><script src="http://www.guoxuedashi.com/img/plus.php?id=3"></script></center>
</div>
<div class="w1180 foot">
<a href="/b/thanks.php">特别致谢</a> | <a href="javascript:window.external.AddFavorite(document.location.href,document.title);">收藏本站</a> | <a href="#">欢迎投稿</a> | <a href="http://www.guoxuedashi.com/forum/">意见建议</a> | <a href="http://www.guoxuemi.com/">国学迷</a> | <a href="http://www.shuowen.net/">说文网</a><script language="javascript" type="text/javascript" src="https://js.users.51.la/17753172.js"></script><br />
  Copyright &copy; 国学大师 古典图书集成 All Rights Reserved.<br>
  
  <span style="font-size:14px">免责声明:本站非营利性站点,以方便网友为主,仅供学习研究。<br>内容由热心网友提供和网上收集,不保留版权。若侵犯了您的权益,来信即刪。scp168@qq.com</span>
  <br />
ICP证:<a href="http://www.beian.miit.gov.cn/" target="_blank">鲁ICP备19060063号</a></div>
<!-- 页脚END --> 
<script src="http://www.guoxuedashi.com/img/plus.php?id=22"></script>
<script src="http://www.guoxuedashi.com/img/tongji.js"></script>

</body>
</html>
