<!DOCTYPE html PUBLIC "-//W3C//DTD XHTML 1.0 Transitional//EN" "http://www.w3.org/TR/xhtml1/DTD/xhtml1-transitional.dtd">
<html xmlns="http://www.w3.org/1999/xhtml">
<head>
<meta http-equiv="Content-Type" content="text/html; charset=utf-8" />
<meta http-equiv="X-UA-Compatible" content="IE=Edge,chrome=1">
<title>資治通鑒_218-資治通鑑卷二百十七_218-資治通鑑卷二百十七</title>
<meta name="Keywords" content="資治通鑒_218-資治通鑑卷二百十七_218-資治通鑑卷二百十七">
<meta name="Description" content="資治通鑒_218-資治通鑑卷二百十七_218-資治通鑑卷二百十七">
<meta http-equiv="Cache-Control" content="no-transform" />
<meta http-equiv="Cache-Control" content="no-siteapp" />
<link href="/img/style.css" rel="stylesheet" type="text/css" />
<script src="/img/m.js?2020"></script> 
</head>
<body>
 <div class="ClassNavi">
<a  href="/24shi/">二十四史</a> | <a href="/SiKuQuanShu/">四库全书</a> | <a href="http://www.guoxuedashi.com/gjtsjc/"><font  color="#FF0000">古今图书集成</font></a> | <a href="/renwu/">历史人物</a> | <a href="/ShuoWenJieZi/"><font  color="#FF0000">说文解字</a></font> | <a href="/chengyu/">成语词典</a> | <a  target="_blank"  href="http://www.guoxuedashi.com/jgwhj/"><font  color="#FF0000">甲骨文合集</font></a> | <a href="/yzjwjc/"><font  color="#FF0000">殷周金文集成</font></a> | <a href="/xiangxingzi/"><font color="#0000FF">象形字典</font></a> | <a href="/13jing/"><font  color="#FF0000">十三经索引</font></a> | <a href="/zixing/"><font  color="#FF0000">字体转换器</font></a> | <a href="/zidian/xz/"><font color="#0000FF">篆书识别</font></a> | <a href="/jinfanyi/">近义反义词</a> | <a href="/duilian/">对联大全</a> | <a href="/jiapu/"><font  color="#0000FF">家谱族谱查询</font></a> | <a href="http://www.guoxuemi.com/hafo/" target="_blank" ><font color="#FF0000">哈佛古籍</font></a> 
</div>

 <!-- 头部导航开始 -->
<div class="w1180 head clearfix">
  <div class="head_logo l"><a title="国学大师官网" href="http://www.guoxuedashi.com" target="_blank"></a></div>
  <div class="head_sr l">
  <div id="head1">
  
  <a href="http://www.guoxuedashi.com/zidian/bujian/" target="_blank" ><img src="http://www.guoxuedashi.com/img/top1.gif" width="88" height="60" border="0" title="部件查字,支持20万汉字"></a>


<a href="http://www.guoxuedashi.com/help/yingpan.php" target="_blank"><img src="http://www.guoxuedashi.com/img/top230.gif" width="600" height="62" border="0" ></a>


  </div>
  <div id="head3"><a href="javascript:" onClick="javascript:window.external.AddFavorite(window.location.href,document.title);">添加收藏</a>
  <br><a href="/help/setie.php">搜索引擎</a>
  <br><a href="/help/zanzhu.php">赞助本站</a></div>
  <div id="head2">
 <a href="http://www.guoxuemi.com/" target="_blank"><img src="http://www.guoxuedashi.com/img/guoxuemi.gif" width="95" height="62" border="0" style="margin-left:2px;" title="国学迷"></a>
  

  </div>
</div>
  <div class="clear"></div>
  <div class="head_nav">
  <p><a href="/">首页</a> | <a href="/ShuKu/">国学书库</a> | <a href="/guji/">影印古籍</a> | <a href="/shici/">诗词宝典</a> | <a   href="/SiKuQuanShu/gxjx.php">精选</a> <b>|</b> <a href="/zidian/">汉语字典</a> | <a href="/hydcd/">汉语词典</a> | <a href="http://www.guoxuedashi.com/zidian/bujian/"><font  color="#CC0066">部件查字</font></a> | <a href="http://www.sfds.cn/"><font  color="#CC0066">书法大师</font></a> | <a href="/jgwhj/">甲骨文</a> <b>|</b> <a href="/b/4/"><font  color="#CC0066">解密</font></a> | <a href="/renwu/">历史人物</a> | <a href="/diangu/">历史典故</a> | <a href="/xingshi/">姓氏</a> | <a href="/minzu/">民族</a> <b>|</b> <a href="/mz/"><font  color="#CC0066">世界名著</font></a> | <a href="/download/">软件下载</a>
</p>
<p><a href="/b/"><font  color="#CC0066">历史</font></a> | <a href="http://skqs.guoxuedashi.com/" target="_blank">四库全书</a> |  <a href="http://www.guoxuedashi.com/search/" target="_blank"><font  color="#CC0066">全文检索</font></a> | <a href="http://www.guoxuedashi.com/shumu/">古籍书目</a> | <a   href="/24shi/">正史</a> <b>|</b> <a href="/chengyu/">成语词典</a> | <a href="/kangxi/" title="康熙字典">康熙字典</a> | <a href="/ShuoWenJieZi/">说文解字</a> | <a href="/zixing/yanbian/">字形演变</a> | <a href="/yzjwjc/">金 文</a> <b>|</b>  <a href="/shijian/nian-hao/">年号</a> | <a href="/diming/">历史地名</a> | <a href="/shijian/">历史事件</a> | <a href="/guanzhi/">官职</a> | <a href="/lishi/">知识</a> <b>|</b> <a href="/zhongyi/">中医中药</a> | <a href="http://www.guoxuedashi.com/forum/">留言反馈</a>
</p>
  </div>
</div>
<!-- 头部导航END --> 
<!-- 内容区开始 --> 
<div class="w1180 clearfix">
  <div class="info l">
   
<div class="clearfix" style="background:#f5faff;">
<script src='http://www.guoxuedashi.com/img/headersou.js'></script>

</div>
  <div class="info_tree"><a href="http://www.guoxuedashi.com">首页</a> > <a href="/SiKuQuanShu/fanti/">四库全书</a>
 > <h1>资治通鉴</h1> <!--         下载:【右键另存为】即可 --></div>
  <div class="info_content zj clearfix">
  
<div class="info_txt clearfix" id="show">
<center style="font-size:24px;">218-資治通鑑卷二百十七</center>
    資治通鑑卷二百十七  宋 司馬光 撰<br />
<br />
  胡三省 音註<br />
<br />
  唐紀三十三【起閼逢敦牂盡柔兆涒難四月凡二年有奇】<br />
<br />
  玄宗至道大聖大明孝皇帝下之下<br />
<br />
  十三載【卷冒當書天寶年號載子亥翻】春正月己亥安祿山入朝【朝直遥翻<br />
<br />
  國忠屢言祿山濳圖悖山厚賂璆琳盛言祿山】<br />
<br />
  【忠於國國忠又言祿山自此不復見矣玄宗手詔追禄山祿山來朝舊傳亦同按玄宗實錄并祿山事迹遣璆琳送甘子于范陽覘禄山反狀在十四載五月而肅宗實錄及舊傳云十二載誤也今從唐歷】是時楊國忠言祿山必反且曰陛下試召之必不來上使召之祿山聞命即至庚子見上於華清宫【見賢遍翻】泣曰臣本胡人陛下寵擢至此為國忠所疾臣死無日矣上憐之賞賜巨萬由是益親信祿山國忠之言不能入矣太子亦知祿山必反言於上上不聽 甲辰太清宫奏學士李琪【此崇玄館學士也】見玄元皇帝乘紫雲告以國祚延昌 唐初詔勑皆中書門下官有文者為之乾封以後始召文士元萬頃范履冰等草諸文辭常於北門候進止時人謂之北門學士中宗之世上官昭容專其事上即位始置翰林院密邇禁庭延文章之士下至僧道書畫琴棊數術之工皆處之謂之待詔【處昌呂翻】刑部尚書張均及弟太常卿垍皆翰林院供奉【唐天子在大明宫翰林院在右銀臺門内在興慶宫院在金明門内若在西内院在顯福門内若在東都及華清宫皆有待詔之所其待詔者有詞學經術合練僧道卜祝術藝書奕各别院以廪之日晚而退其所重者詞學帝即位以來張說陸堅張九齡徐安貞張垍等召入禁中謂之翰林待詔王者尊極一日萬禨四方進奏中外表疏批答或詔從中出宸翰所揮亦資其檢討謂之視草故常簡當局四人以備顧問至德以後天下用兵多務深謀密詔皆從中出名曰翰林學士得充選者文士為榮亦如中書舍人例置學士六人内擇年深德重者一人為承旨所以獨當密命故也德宗好文尤難其選貞元以後為學士承旨者多至宰相尚辰羊翻垍巨冀翻】上欲加安祿山同平章事已令張垍草制楊國忠諫曰祿山雖有軍功目不知書豈可為宰相制書若下【令力丁翻相悉亮翻下遐稼翻】恐四夷輕唐上乃止乙巳加祿山左僕射【射寅謝翻】賜一子三品一子四品官 丙午上還宫【還自華清宫還從宣翻又音如字】 安祿山求兼領閑廐羣牧庚申以祿山為閑廐隴右羣牧等使【使疏吏翻下同】祿山又求兼總監【此郡牧總監也唐有四十八監以牧馬或曰此總監即苑總監】壬戌兼知總監事禄山奏以御史中丞吉温為武部侍郎【武部即兵部】充閑廐副使楊國忠由是惡温【惡烏路翻】祿山密遣親信選健馬堪戰者數千匹别飼之【飼祥吏翻】 二月壬申上朝獻太清宫上聖祖尊號曰大聖祖高上大道金闕玄元大皇太帝【朝直遥翻上聖之上時掌翻下以義推】癸酉享太廟上高祖諡曰神堯大聖光孝皇帝太宗諡曰文武大聖大廣孝皇帝高宗諡曰天皇大聖大弘孝皇帝中宗諡曰天和大聖大昭孝皇帝睿宗諡曰玄真大聖大興孝皇帝以漢家諸帝皆諡孝故也甲戌羣臣上尊號曰開元天地大寶聖文神武證道孝德皇帝【凡上尊號上諡之上皆時掌翻】赦天下丁丑楊國忠進位司空甲申臨軒冊命 己丑安祿<br />
<br />
  山奏臣所部將士討奚契丹九姓同羅等勲効甚多【將即亮翻契欺訖翻又音喫】乞不拘常格超資加賞仍好寫告身付臣軍授之於是除將軍者五百餘人中郎將者二千餘人祿山欲反故先以此收衆心也三月丁酉朔祿山辭歸范陽【舊志范陽在京師東北二千五百二十里】上解御衣以賜之祿山受之驚喜恐楊國忠奏留之疾驅出關【出潼關】乘船沿河而下令船夫執䋲板立於岸側【凡挽船夫用板長二尺許斜搭冑前一端至肩一端至脇䋲貫板之兩端以接船繂而挽之】十五里一更【更工衡翻易也】晝夜兼行日數百里過郡縣不下船自是有言祿山反者上皆縳送由是人皆知其將反無敢言者祿山之長安也上令高力士餞之長樂坡【長樂坡即產坡在長安城東樂音洛】及還上問祿山慰意乎對曰觀其意怏怏必知欲命為相而中止故也【怏於兩翻相息亮翻】上以告國忠曰此議它人不知必張垍兄弟告之也【國忠之下更有國忠二字文意乃明】上怒貶張均為建安太守垍為盧溪司馬垍弟給事中埱為宜春司馬【建安郡隋為泉州唐改曰閩州别置泉州帝改閩州為福州長樂郡以建州為建安郡盧溪郡辰州舊志建安郡京師東南四千九百三十五里盧溪郡京師南三千四百五里埱昌六翻 考異曰唐歷云垍嘗贊相禮儀雍容有度上心悦之翌日謂垍曰朕罷希烈相以卿代之垍曰不敢貴妃在坐告國忠斥之舊垍傳天寶中玄宗嘗幸垍内宅謂垍曰希烈累辭機務朕擇其代者孰可垍錯愕未對帝即曰無踰吾愛壻矣垍降階陳謝楊國忠聞而惡之及希烈罷相舉韋見素代垍垍深觖望按本紀三月丁酉垍貶官韋見素八月乃知政事而云垍深觖望舊傳誤也明皇雜錄云上幸張垍宅謂垍曰中外大臣才堪宰輔者與我悉數吾當舉而用之垍逡巡不對上曰固無如愛子壻垍降階拜舞上曰即舉成命既逾月垍頗懷怏怏意其為李林甫所排會祿山自范陽入覲祿山濳賂貴妃求帶平章事上不許垍因私第備言上前時行幸内第面許相垍與明公同制入輔今既中變當必為姧臣所排祿山大懷恚怒明日謁見因流涕請罪上慰勉久之因問其故祿山俱以垍所陳對上命高力士送歸焉亦以怏怏聞由是上怒按李林甫時已死亦誤也】哥舒翰亦為其部將論功【為于偽翻將即亮翻】勑以隴右十將特進火拔州都督燕山郡王火拔歸仁為驃騎大將軍【十將亦唐中世以來軍中將領之職名火拔突厥别部也開元中置火拔州唐制特進文散階正二品驃騎大將軍武散階從二品燕因肩翻驃匹妙翻騎奇寄翻】河源軍使王思禮加特進臨洮太守成如璆討擊副使范陽魯炅臯蘭府都督渾惟明並加雲麾將軍【貞觀中鐵勒來降以渾部置臯蘭都督府雲麾將軍武散階從三品上洮土刀翻守式又翻璆音求炅火迥翻】隴右討擊副使郭英乂為左羽林將軍英乂知運之子也翰又奏嚴挺之之子武為節度判官河東呂諲為支度判官【諲伊真翻】前封丘尉高適為掌書記安邑曲環為别將【河東郡蒲州唐制邉軍有支度使以計軍資糧仗之用其屬有判官巡官封丘縣漢晉以來屬陳留唐屬汴州安邑縣屬蒲州姓譜晉穆侯子成師封于曲沃其後氏焉漢有代郡太守曲謙貨殖傳冇曲叔諲伊真翻】 程千里執阿布思獻於闕下斬之甲子以千里為金吾大將軍以封常清權北庭都護伊西節度使 夏四月癸巳安祿山奏擊奚破之虜其王李日越 六月乙丑朔日有食之不盡如鈎 侍御史劍南留後李宓【楊國忠領劒南節度使以宓為留後宓音密又音伏】將兵七萬擊南詔閤羅鳳誘之深入至太和城【新書作大和城夷語山陀為和故謂大和閤羅鳳所居也將即亮翻誘音酉】閉壁不戰宓糧盡士卒罹瘴疫及飢死什七八乃引還【還從宣翻又如字】蠻追擊之宓被擒【被皮義翻】全軍皆没楊國忠隱其敗更以捷聞益中國兵討之前後死者幾二十萬人【并鮮于仲通之敗死者有此數幾居依翻】無敢言者上嘗謂高力士曰朕今老矣朝事付之宰相邊事付之諸將夫復何憂【朝直遥翻相息亮翻將即亮翻夫音扶復扶又翻】力士對曰臣聞雲南數喪師又邊將擁兵太盛陛下將何以制之臣恐一旦禍不可復救【數所角翻喪息浪翻復扶又翻】何得謂無憂也上曰卿勿言朕徐思之【高力士之言明皇豈無所動於其心哉禍機將直付之無可奈何僥幸其身之不及見而已】 秋七月癸丑哥舒翰奏於所開九曲之地置洮陽澆河二郡及神策軍以臨洮太守成如璆兼洮陽太守充神策軍使【洮陽澆河二郡皆置於洮廓二州西南廓州本澆河郡天寶元年更名寜塞郡洮州西八十里磨環州置神策軍新書曰澆河郡置於積石之西澆堅堯翻】 楊國忠忌陳希烈希烈累表辭位上欲以武部侍郎吉温代之國忠以温附安禄山奏言不可以文部侍郎韋見素和雅易制【易音以䜴翻】薦之八月丙戌以希烈為太子太師罷政事【陳希烈遂以此怨望降賊】以見素為武部尚書同平章事 【考異曰舊見素傳曰時楊國忠用事左相陳希烈畏其權寵凡事唯諾無敢明玄宗知之不悦天寶十三年秋霖雨六十餘日天子以宰相或未稱職見此咎議命楊國忠精求端士時兵部侍郎吉温方承寵遇上意欲用之國忠以温祿山賓佐懼其威權奏寢其事國忠訪于中書舍人竇華宋昱等華昱言見素方雅柔而易制上亦以經事相王府有舊恩可之希烈傳曰國忠用事素忌疾之乃引韋見素同列罷希烈知政事按明皇若惡希烈阿狥國忠當更自擇剛直之士豈得尚卜相于國忠今從希烈傳】自去歲水旱相繼關中大飢楊國忠惡京兆尹李峴不附己以災沴歸咎於峴九月貶長沙太守【惡烏路翻沴音戾長沙郡潭州舊志長沙郡京師南二千四百四十五里】峴禕之子也【信安王禕開元初以軍功有寵於上禕吁韋翻】上憂雨傷稼國忠取禾之善者獻之曰雨雖多不害稼也上以為然扶風太守房琯言所部水灾【扶風郡岐州】國忠使御史推之【宋白曰唐故事侍御史各二人知東西推又各分京城諸司及諸道州府為東西之限隻日則臺院受事雙日則殿院受事又有監察御史出使推按謂之推事御史】是歲天下無敢言灾者高力士侍側上曰淫雨不已【賈公彦曰雨三日已上曰淫】卿可盡言對曰自陛下以權假宰相賞罰無章陰陽失度臣何敢言上默然 冬十月乙酉上幸華清宫十一月己未置内侍監二員正三品【唐制宫官不得過三品置内侍】<br />
<br />
  【四人從四品上中官之貴極於此矣至帝始隳其制楊思勗以軍功高力士以恩寵皆拜大將軍階至從一品猶曰勲官也今置内侍監正三品則職事官矣】 河東太守兼本道采訪使韋陟斌之兄也【使疏吏翻斌音彬】文雅有盛名楊國忠恐其入相【相息亮翻】使人告陟汚事下御史按問陟賂中丞吉温使求救於安祿山復為國忠所【下遐稼翻復扶又翻】閏月壬寅貶陟桂嶺尉温澧陽長史【桂嶺漢臨賀縣地隋置桂嶺縣唐屬賀州澧陽郡澧州舊志澧陽郡京師東南一千八百九十三里】安祿山為温訟寃【為于偽翻】且言國忠讒疾上兩無所問 戊午上還宫 是歲戶部奏天下郡三百二十一縣千五百三十八鄉萬六千八百二十九戶九百六萬九千一百五十四口五千二百八十八萬四百八十八【有唐戶口之盛極于此】<br />
<br />
  十四載春正月蘇毗王子悉諾邏去吐蕃來降【新書曰蘇毗吐蕃疆部也邏郎佐翻】 二月辛亥安祿山使副將何千年入奏請以蕃將三十二人代漢將上命立進畫【進畫者命中書為日勑進請御畫而行之唐六典中書掌王言其制有七其四曰日勑正謂御畫日勑也增減官員廢置州縣除免官爵授六品以下官則用之將即亮翻】給告身韋見素謂楊國忠曰祿山久有異志今又有此請其反明矣明日見素當極言上未允公其繼之國忠許諾壬子國忠見素入見【入見賢遍翻】上迎謂曰卿等有疑祿山之意邪見素因極言禄山反已有迹所請不可許上不悦國忠逡巡不敢言【逡七倫翻】上竟從祿山之請它日國忠見素言於上曰臣有策可坐消祿山之謀今若除祿山平章事召詣闕以賈循為范陽節度使呂知誨為平盧節度使楊光翽為河東節度使【使疏吏翻翽呼會翻】則勢自分矣上從之已草制上留不更遣中使輔璆琳以珍果賜禄山濳察其變【輔姓也左傳晉有大夫輔躒又智果别族為輔氏即考異前所引以甘子賜祿山事璆音求】璆琳受祿山厚賂還盛言祿山竭忠奉國無有二心【還從宣翻又音如字】上謂國忠等曰祿山朕推心待之必無異志東北二虜藉其鎮遏朕自保之卿等勿憂也事遂寢 【考異曰實録正月辛巳祿山表請以蕃將三十人代漢將上遣中使袁思藝宣付中書令即日進畫便寫告身楊國忠韋見素相謂曰流言傳祿山有不臣之心今又請代漢將其反明矣乃請陳事既見上先曰卿等有疑祿山之意邪國忠等遽走下階垂涕具陳祿山反狀因以祿山表留上前而出俄頃上又令袁思藝宣曰此之一奏姑容之朕徐為圖之國忠奉詔自後國忠每對未嘗不懇請其事國忠曰臣有一策可銷其難伏望下制以祿山帶左僕射平章事追赴朝廷以賈循等分帥三道上許之草制訖留之未行上濳令輔璆琳送甘子私候其狀還固稱無事其制遂寢先是上引宰相對見常置白麻於座前及璆琳還上乃謂宰臣曰祿山必無二心其制朕已焚矣後璆琳受祿山賄事泄上因祭龍堂遣備儲具責以不䖍乃命左右撲殺之始有疑祿山意祿山事迹云請不以蕃將代漢將論祿山反狀及請追祿山赴闕並是韋見素之意旨國忠曾無預焉仍語見素曰祿山出自寒微位居衆上時所忌嫉成疑似耳見素曰公若實為此見社稷危矣將至上前懇論見素約以事如未諧公繼之國忠都無一言俯僂而退見素却到中書嗚咽流涕此非他也國忠要禄山速反以明己之先見耳宋巨玄宗幸蜀記云是年春二月二十二日辛亥祿山使何千年表請以蕃將三十二人代漢將掌兵其日宰相韋見素楊國忠在省見素慘然國忠問曰堂老何色之戚也見素曰祿山逆狀行路共知今以蕃酋代漢將是亂將作矣與公位當此地能無戚乎國忠於是亦惘然久之乃曰與奪之間在于宸斷豈我輩所能是非邪見素曰知禍之萌而不能防亦將焉用彼相矣明日對見僕必懇論冀其萬一若不允子必繼之國忠曰事則不諧恐虚犯龍顔自貽伊戚見素曰如正其言而獲死猶愈於阿從而偷生翌日壬子二相入對見素言祿山濳貯異圖迹已昭彰因叩頭流涕久之國忠但俯僂逡巡更無所補上不悦遂以他事議之既退還省見素謂國忠曰聖意未囘計將安出國忠曰祿山未必有反意但時所誹疾便成疑似耳見素曰公若為此見社稷危矣遂憫然不言二十四日癸丑上又使思藝宣旨令且依此遣卿等所議後别籌之自是見素數奏其凶狀三月己未朔見素請以禄山同中書門下平章事追赴闕庭及輔璆琳送甘子禄山紿璆琳曰主上耄年信任非次國忠之輩苟狥榮班今若進逆耳之言苦口之藥以吾之心事將無益今欲耀兵彊諫以迹鬻拳此意决矣祿山以物贈璆琳璆琳既受金帛及還奏曰祿山盡忠奉國必無二心特望官家不以東北為慮上然之謂宰臣曰祿山朕自得之卿勿憂也見素起曰臣忤拂聖旨僭黷大臣罪合萬死然愚者千慮或有一中願陛下審察之自餘與實錄及事迹所述略同按祿山方賂璆琳泯其反迹安肯對之遽出悖語又國忠平日數言祿山欲反此際安得不與見素同心蓋所謂天下之惡皆歸焉者也今取其可信者】循華原人也時為節度副使 隴右河西節度使哥舒翰入朝道得風疾遂留京師家居不出 三月辛巳命給事中裴士淹宣慰河北 夏四月安祿山奏破奚契丹【契欺訖翻】 癸巳以蘇毗王子悉諾邏為懷義王賜姓名李忠信 安祿山歸至范陽朝廷每遣使者至皆稱疾不出迎【朝直遥翻使疏吏翻】盛陳武備然後見之裴士淹至范陽二十餘日乃得見無復人臣禮【復扶又翻又如字】楊國忠日夜求祿山反狀使京兆尹圍其第 【考異曰肅宗實錄國忠日夜伺求禄山反狀或矯詔以兵圍其宅或令府縣捕其門客李起安岱李方來等皆令侍御史鄭昂之陰推劾濳槌殺之慶宗尚郡主又供奉在京密報其父祿山轉懼唐歷是夏京兆尹李峴貶零陵太守先是楊國忠使門客蹇昂何盈求祿山隂事命京兆尹圍捕其宅得安岱李方來等與祿山反狀使侍御史鄭昂之縊殺之祿山怒使嚴莊上表自理具陳國忠罪狀二十餘事上懼其生變遂歸過於峴以安之安祿山事迹與唐歷同外有命京兆尹李峴於其宅得李起安岱李方來等又貶吉温為澧陽長史以激怒祿山幸其速反上竟不之悟玄宗幸蜀記與事迹同按李峴傳十三載連雨六十餘日國忠歸咎京兆尹貶長沙太守新宗室宰相傳楊國忠使客宴昂何盈摘安祿山隂事諷京兆捕其第得安岱李方來等與祿山反狀縊殺之祿山怒上書自言帝懼變出峴為零陵太守今從實錄】捕祿山客李超等送御史臺獄濳殺之祿山子慶宗尚宗女榮義郡主供奉在京師【在京師為太僕卿得隨供奉官班見】密報禄山禄山愈懼六月上以其子成昏手詔祿山觀禮禄山辭疾不至秋七月祿山表獻馬三千匹每匹執控夫二人遣蕃將二十二人部送【欲以襲京師也】河南尹逹奚珣疑有變奏請諭祿山以進車馬宜俟至冬官自給夫無煩本軍於是上稍寤始有疑禄山之意會輔璆琳受賂事亦泄上托以他事撲殺之上遣中使馮神威齎手詔諭祿山如珣策【撲弼角翻使疏吏翻 考異曰祿山事迹作承威今從玄宗幸蜀記】且曰朕新為卿作一湯【自天寶六載以來華清宫中益治湯井池臺觀環列山谷御湯曰九龍殿亦曰蓮花湯明皇雜錄曰明皇幸華清宫新廣湯制作宏麗安祿山於范陽以白玉石為魚龍鳬雁仍以石梁及蓮花同獻雕鐫巧妙殆非人功上大悦命陳于湯中仍以石梁横亘湯上而蓮花纔出於水際上至其所解衣欲入而魚龍鳬雁皆若奮鱗舉翼狀欲飛動上恐遽命撒去而蓮花至今猶存又當於宫中置長湯數十間屋皆周囘甃以文石為銀鏤漆船及白木香船實於其中至于楫棹皆飾以珠玉又于湯中累瑟瑟及沉香為山以狀瀛洲方文津陽門詩注曰宫中除供奉兩湯外内更有湯十六所長湯每賜諸嬪御其修廣與諸湯不侔甃以文瑶密石中央有玉蓮花捧湯實以成池又縫綴錦繡為鳬雁置於水中上時於其間泛鈒鏤小舟以嬉遊焉次西曰太子湯又次西宜春湯又次西長湯十六所今唯太子少陽二湯存焉又有玉女殿湯今石星痕湯玉名甕湯所出也為于偽翻】十月於華清宫待卿神威至范陽宣旨禄山踞牀微起亦不拜曰聖人安隱【聖人謂上唐隱讀曰穩也帖多有寫穩字為隱字者】又曰馬不獻亦可十月灼然詣京師即令左右引神威置館舍不復見數日遣還亦無表神威還見上泣曰臣幾不得見大家【復扶又翻幾居依翻】 八月辛卯免今載百姓租庸 冬十月庚寅上幸華清宫 【考異曰舊紀壬辰今從實錄新紀】 安禄山專制三道陰蓄異志殆將十年以上待之厚欲俟上晏駕然後作亂會楊國忠與禄山不相悦屢言禄山且反上不聽國忠數以事激之【數所角翻】欲其速反以取信於上祿山由是決意遽反獨與孔目官太僕丞嚴莊掌書記屯田員外郎高尚將軍阿史那承慶密謀自餘將佐皆莫之知但怪其自八月以來屢饗士卒秣馬厲兵而已會有奏事官自京師還禄山詐為勑書悉召諸將示之曰有密旨令祿山將兵入朝討楊國忠【將即亮翻朝直遥翻】諸君宜即從軍衆愕然相顧莫敢異言十一月甲子禄山所部兵及同羅奚契丹室韋凡十五萬衆號二十萬反於范陽 【考異曰平致美薊門紀亂曰是其年八月後慰諭兵士磨厲戈矛頗異於常識者竊怪矣至是禄山勒兵夜將出命屬官等謂曰奏事官胡逸自京囘奉密旨遣祿山將隨身兵馬入朝來莫令那人知羣公勿怪便請隨軍那人意楊國忠也】命范陽節度副使賈循守范陽平盧節度副使呂知誨守平盧别將高秀巖守大同【中受降城西二百里有大同州又代州北有大同軍去太原八百餘里新志大同軍在朔州馬邑縣按宋白續通典中受降城西之大同州乃隋大同城之舊墟開元五年分善陽縣東三十里置大同軍以戍遣復於軍内置馬邑縣直代州北】諸將皆引兵夜詰朝祿山出薊城南【詰去吉翻薊音計】大閱誓衆以討楊國忠為名牓軍中曰有異議扇動軍人者斬及三族於是引兵而南祿山乘鐵轝步騎精鋭煙塵千里鼓譟震地【轝與輿同騎奇寄翻譟蘇到翻】時海内久承平百姓累世不識兵革猝聞范陽兵起遠近震駭河北皆禄山統内【祿山兼河北道采訪使】所過州縣望風瓦解守令或開門出迎【守式又翻】或乘城竄匿或為所擒戮無敢拒之者禄山先遣將軍何千年高邈將奚騎二十聲言獻射生手乘驛詣太原乙丑北京副留守楊光翽出迎因刼之以去 【考異曰肅宗實錄云先令千年領壯士數千人詐稱獻俘以車千乘包旌旗戈甲器械先俟于河陽橋不見後來所用又千年時方詣太原執楊光翽未暇向河陽也今不取薊門紀亂云是月甲午縳光翽按是月有甲子安得甲午今不取】太原具言其狀東受降城亦奏祿山反上猶以為惡祿山者詐為之【降戶江翻惡烏路翻】未之信也庚午上聞祿山定反乃召宰相謀之楊國忠揚揚有德色【蜀本作得色當從之】曰今反者獨禄山耳將士皆不欲也不過旬日必傳首詣行在上以為然大臣相顧失色上遣特進畢思琛詣東京【琛丑林翻】金吾將軍程千里詣河東各簡募數萬人隨便團結以拒之辛未安西節度使封常清入朝【朝直遥翻】上問以討賊方略常清大言曰今太平積久故人望風憚賊然事有逆順勢有奇變臣請走馬詣東京開府庫募驍勇挑馬箠度河【驍堅堯翻挑徒刁翻箠止橤翻】計日取逆胡之首獻闕下上悦壬申以常清為范陽平盧節度使常清即日乘驛詣東京募兵旬日得六萬人乃斷河陽橋為守禦之備【斷音短】甲戌祿山至博陵南【博陵郡本定州高陽郡天寶元年更郡名舊志博陵郡京師東北二千九百六里】何千年等執楊光翽見祿山責光翽以附楊國忠斬之以狥 【考異曰幸蜀記云十九甲戌至真定南逢楊光翽按唐歷禄山遣驍騎何千年等劫光翽歸遇于博陵郡殺之蓋幸蜀記誤以定州為真定耳祿山事迹曰其年九月傳太原尹楊光翽首至按祿山十一月始反而事迹云九月取光翽誤也】祿山使其將安忠志將精兵軍土門【將即亮翻下同】忠志奚人祿山養為假子又以張獻誠攝博陵太守獻誠守珪之子也【張守珪卵翼禄山實為厲階】祿山至藁城常山太守顔杲卿力不能拒與長史袁履謙往迎之祿山輒賜杲卿金紫質其子弟使仍守常山【常山郡本恒州恒山郡天寶元年更郡名劉昫曰常山郡舊治元氏魏道武登常山郡北望安樂壘美之遂移郡治於安樂城今州城是也魏收志九門縣有安樂壘質音致】又使其將李欽湊將兵數千人守井陘口以備西來諸軍【西來之軍謂河東路兵東出井陘口者陘音刑】杲卿歸途中指其衣謂履謙曰何為著此【著陟略翻】履謙悟其意乃陰與杲卿謀起兵討祿山杲卿思魯之玄孫也【顔思魯之推之子師古之父也】 丙子上還宫斬太僕卿安慶宗賜榮義郡主自盡以朔方節度使安思順為戶部尚書思順弟元貞為太僕卿以朔方右廂兵馬使九原太守郭子儀為朔方節度使【九原郡豐州】右羽林大將軍王承業為太原尹【太原為北都故置尹】置河南節度使領陳留等十三郡以衛尉卿猗氏張介然為之【陳留郡汴州 考異曰實錄以介然為汴州刺史舊紀以介然為陳留太守按是時無刺史郭納見為太守介然直為節度使耳】以程千里為潞州長史諸郡當賊衝者始置防禦使丁丑以榮王琬為元帥右金吾大將軍高仙芝副之統諸軍東征【帥所類翻】出内府錢帛於京師募兵十一萬號曰天武軍旬日而集皆市井子弟也十二月丙戌高仙芝將飛騎彍騎及新募兵邊兵在<br />
<br />
  京師者合五萬人長安上遣宦者監門將軍邊令誠監其軍屯於陜【將即亮翻騎奇寄翻彍盧郭翻又古博翻監古衘翻陜失冉翻舊志陜郡在京師東四百九十里至東都三百三十里】 丁亥安禄山自靈昌度河【靈昌郡本滑州東郡天寶元年更郡名】以絙約敗船及草木横絶河流一夕冰合如浮梁遂䧟靈昌郡【僖志靈昌郡去京師一千四百四十里至東都五百三十里】禄山步騎散漫人莫知其數所過殘滅張介然至陳留纔數日祿山至授兵登城衆忷懼不能守【忷許拱翻】庚寅太守郭納以城降禄山入北郭聞安慶宗死慟哭曰我何罪而殺我子時陳留將士降者夾道近萬人【降戶江翻近其靳翻】禄山皆殺之以快其忿斬張介然於軍門 【考異曰舊紀辛卯陷陳留郡禄山事迹庚午陷陳留郡傳張介然荔非元瑜等首至今從實錄】以其將李庭望為節度使守陳留【舊志陳留郡京師東一千三百五十里東都四百一里】壬辰上下制欲親征其朔方河西隴右兵留守城堡之外皆赴行營令節度使自將之期二十日畢集 初平原太守顔真卿【漢置平原郡唐為德州天寶元年復改為郡】知禄山且反因霖雨完城浚壕料丁壯實倉廩祿山以其書生易之【料連條翻量度也又力弔翻易以䜴翻】及禄山反牒真卿以平原博平兵七千人防河津【博平郡博州】真卿遣平原司兵李平間道奏之【間古莧翻】上始聞禄山反河北郡縣皆風靡嘆曰二十四郡曾無一人義士邪及平至【舊志平原郡至京師一千九百八十二里】大喜曰朕不識顔真卿作何狀乃能如是真卿遣親客密懷購賊牒詣諸郡由是諸郡多應者真卿杲卿之從弟也【從才用翻】安禄山引兵向滎陽太守崔無詖拒之士卒乘城者聞鼓角聲自隊如雨癸巳祿山陷滎陽【滎陽郡鄭州西至洛陽二百六十里舊志滎陽郡至京師一千一百五里東都二百七十里考異曰唐歷舊紀作甲午今從實錄】殺無詖以其將武令珣守之祿山聲勢益張【張知亮翻】以其將田承嗣安忠志張孝忠為前鋒封常清所募兵皆白徒未更訓練【更工衡翻】屯武牢以拒賊賊以鐵騎蹂之【蹂人九翻】官軍大敗常清收餘衆戰于葵園又戰敗上東門内又敗【葵園在甖子谷南上東門即洛陽上春門也唐六與東都城東面三門北曰上東】丁酉祿山陷東京賊鼔譟自四門入縱兵殺掠常清戰于都亭驛又敗退守宣仁門又敗乃自苑西壞牆西走【壞音怪 考異曰常清表云自今月七日交兵至十三日不已按七日祿山猶未至滎陽蓋與賊前鋒戰耳】河南尹奚達珣降於祿山【降戶江翻】留守李憕謂御史中丞盧奕曰吾曹荷國重任【守式又翻橙直陵翻荷下可翻】雖知力不敵必死之奕許諾憕收殘兵數百欲戰皆棄憕潰去憕獨坐府中奕先遣妻子懷印間道走長安【走音奏】朝服坐臺中【朝直遥翻】左右皆散祿山屯於閑廐使人執憕奕及采訪判官蔣清皆殺之奕罵祿山數其罪【數所具翻】顧賊黨曰凡為人當知順逆我死不失節夫復何恨【夫音扶復扶又翻】憕文水人【文水縣屬并州本漢大陵縣魏置受陽縣隋為文水縣】奕懷慎之子清欽緒之子也【盧懷慎開元初賢相蔣欽緒見二百九卷中宗景龍三年】禄山以其黨張萬頃為河南尹封常清帥餘衆至陜【帥讀曰率】陜郡太守竇廷芝已奔河東吏民皆散常清謂高仙芝曰常清連日血戰賊鋒不可當且潼關無兵若賊豕突入關則長安危矣陜不可守不如引兵先據潼關以拒之仙芝乃帥見兵西趣潼關【見賢遍翻趣七喻翻 考異曰肅宗實錄云仙芝領大軍初至陜方欲進師會常清軍敗至欲廣其賊勢以雪己罪勸仙芝班師仙芝素信常清言即日夜走保潼關朝野大駭今從本傳】賊尋至官軍狼狽走無復部伍士馬相騰踐死者甚衆至潼關修完守備賊至不得入而去祿山使其將崔乾祐屯陜【復扶又翻踐悉銑翻潼音同將即亮翻陜失冉翻】臨汝弘農濟陰濮陽雲中郡皆降於祿山【弘農郡本虢州虢郡天寶元年更名郡濮陽郡濮州雲中郡雲州濟子禮翻濮博木翻降戶江翻】是時朝廷徵兵諸道皆未至關中忷懼會祿山方謀稱帝留東京不進故朝廷得為之備兵亦稍集祿山以張通儒之弟通晤為睢陽太守與陳留長史楊朝宗將胡騎千餘東略地【朝直遥翻忷許拱翻睢音雖守式又翻將即亮翻騎奇寄翻】郡縣官多望風降走惟東平太守嗣吳王祗濟南太守李隨起兵拒之【東平郡鄆州濟南郡本齊州齊郡天寶元年更名臨淄郡五載更今郡名嗣祥吏翻】祗禕之弟也【禕吁韋翻】郡縣之不從賊者皆倚吳王為名單父尉賈賁帥吏民南擊睢陽斬張通晤【單父古縣時屬睢陽郡單音善父音甫】李庭望引兵欲東狥地聞之不敢進而還【還從宣翻又音如字】 庚子以永王璘為山南節度使江陵長史源洧為之副【江陵郡本荆州南郡天寶元年更郡名璘力珍翻洧于軌翻】穎王璬為劒南節度使蜀郡長史崔圓為之副【蜀郡益州璬公了翻長知兩翻】二王皆不出閤洧光裕之子也【源光裕見二百一十二卷開元十三年】上議親征辛丑制太子監國【監古銜翻考異曰唐歷幸蜀記皆云十六日辛丑按長歷辛丑十七日也實錄又作己丑尤誤肅宗實錄云詔以上監】<br />
<br />
  【國仍令總統六軍親征寇逆按制書云今親總六師率衆百萬鋪敕元惡巡撫洛陽則是上親征使太子留守也今從玄宗實錄】謂宰相曰朕在位垂五十載倦于憂勤去秋已欲傳位太子值水旱相仍不欲以餘災遺子孫【遺唯季翻】淹留俟稍豐不意逆胡横【横戶孟翻下同】朕當親征且使之監國事平之日朕將高枕無為矣【枕之任翻】楊國忠大懼退謂韓虢秦三夫人曰太子素惡吾家專横久矣若一旦得天下吾與姊妹併命在旦暮矣相與聚哭使三夫人說貴妃【惡烏路翻横下孟翻說式芮翻】銜土請命於上事遂寢 顔真卿召募勇士旬日至萬餘人諭以舉兵討安禄山繼以涕泣士皆感憤祿山使其黨段子光齎李憕盧奕蔣清首狥河北諸郡至平原壬寅真卿執子光腰斬以狥取三人首續以蒲身棺歛葬之祭哭受弔【棺音貫歛力贍翻】祿山以海運使劉道玄攝景城太守清池尉賈載鹽山尉河内穆寜共斬道玄【自帝事邊功運青萊之粟浮海以給幽平之兵故置海運使景州本滄州勃海郡天寶更郡名清池漢浮陽縣地開皇十八年更名鹽山漢高城縣地隋開皇十八年以縣有鹽山更名清池帶郡鹽山屬邑也】得其甲仗五十餘船攜道玄首謁長史李暐暐收嚴莊宗族悉誅之是日送道玄首至平原真卿召載寜及清河尉張澹詣平原計事【澹徒覽翻 考異曰舊穆寜傳祿山偽署劉道玄為景城守寜倡義起兵斬道玄首傳檄郡邑多有應者賊將史思明來寇郡寜以攝東光令將兵禦之思明遣使說誘寜立斬之郡懼賊怨深後大兵至奪寜兵及攝縣初寜佐采訪使巡按嘗過平原與太守顔真卿密揣祿山必反至是真卿亦唱義舉郡兵以拒祿山會間使持書遺真卿曰夫子為衛君乎更無他詞真卿得書大喜因奏署大理評事河北采訪支使按寜以道玄首謁李暐暐即族嚴莊家豈有懼賊怨深而奪寜兵乎真卿既殺段子光帥諸郡以討禄山寜書中安得尚為隱語道玄首至平原真卿已召寜計事豈待得此書然後用之況真卿領采訪使乃在明年常山陷後今皆不取】饒陽太守盧全誠據城不受代 【考異曰包諝河洛春秋作盧皓今從殷仲容顔氏行狀】河間司法李奐殺禄山所署長史王懷忠李隨遣遊奕將訾嗣賢濟河【將即亮翻訾即移翻姓也漢有訾順】殺祿山所署博平太守馬冀各有衆數千或萬人共推真卿為盟主軍事皆禀焉禄山使張獻誠將上谷博陵常山趙郡文安五郡團結兵萬人圍饒陽【饒陽郡深州河間郡瀛州上谷郡易州趙郡趙州文安郡莫州將即亮翻】 高仙芝之東征也監軍邊令誠數以事干之仙芝多不從令誠入奏事具言仙芝常清撓敗之狀【數所角翻橈奴教翻】且云常清以賊揺衆而仙芝棄陜地數百里又盜減軍士糧賜上大怒癸卯遣令誠齎勑即軍中斬仙芝及常清初常清既敗三遣使奉表陳賊形勢【使疏吏翻】上皆不之見常清乃自馳詣闕至渭南勑削其官爵令還仙芝軍白衣自效常清草遺表曰臣死之後望陛下不輕此賊無忘臣言時朝議皆以為祿山狂悖不日受首故常清云然【云然者猶曰言如此也朝直遥翻悖蒲内翻又蒲没翻】令誠至潼關先引常清宣勑示之常清以表附令誠上之【上時掌翻 考異曰明皇幸蜀記安祿山事迹皆曰常清配隸仙芝軍感憤頗深遂作遺表飲藥而死令誠至常清已死而舊傳以為勑令却赴潼關自草表待罪是日臨刑託令誠上之蓋二書見常清表有仰天飲鴆向日封章即為尸諫之臣死作聖朝之鬼故云然今從舊傳】常清既死陳尸蘧蒢【蘧蒢蘆䕠也】仙芝還至聽事令誠索陌刀手百餘人自隨【索山客翻】乃謂仙芝曰大夫亦有恩命仙芝遽下令誠宣勑仙芝曰我遇敵而退死則宜矣今上戴天下履地謂我盜減糧賜則誣也時士卒在前皆大呼稱枉其聲震地遂斬之【呼火故翻史言高仙芝由邉令誠而得節亦由邉令誠而喪元】以將軍李承光攝領其衆河西隴右節度使哥舒翰病廢在家 【考異曰舊金梁鳳傳云天寶十三載哥舒翰入京師裴冕為河西留後在武威是翰雖病在京師猶領河西隴右兩鎮也】上藉其威名且素與禄山不協召見【見賢遍翻】拜兵馬副元帥將兵八萬以討祿山【帥所類翻】仍勑天下四面進兵會攻洛陽翰以病固辭上不許以田良丘為御史中丞充行軍司馬起居郎蕭昕為判官蕃將火拔歸仁等各將部落以從【將即亮翻從才用翻】并仙芝舊卒號二十萬軍於潼關 【考異曰肅宗實錄云以翰為皇太子先鋒兵馬使元帥領河隴朔方募兵十萬并仙芝舊卒號二十萬拒戰于潼關十二月十七日大軍唐歷亦云先鋒兵馬使元帥舊傳云先鋒兵馬元帥祿山事迹云翰為副元帥領河隴諸蕃部落奴刺頡跌朱邪契苾渾蹛林奚結沙陀蓬子處密吐谷渾思結等十三部落督蕃漢兵二十一萬八千人鎮于潼關舊紀云丙午命翰守潼關按玄宗實錄癸卯斬常清仙芝命翰為兵馬副元帥統兵八萬鎮潼關時榮王為元帥故以翰副之蓋誅仙芝之日即命翰代仙芝舊紀丙午肅宗實錄十七日軍皆太早也玄宗實錄所云八萬者蓋止謂漢兵隨翰東征者耳并諸蕃部落及仙芝舊兵則及十餘萬因號二十萬也】翰病不能治事【治直之翻】悉以軍政委田良丘良丘復不敢專決使王思禮主騎李承光主步二人爭長無所統壹【復扶又翻長知兩翻】翰用法嚴而不恤士卒皆懈弛無鬬志【史言哥舒翰所以敗】 安禄山大同軍使高秀巖寇振武軍【杜佑曰振武軍在單于都護府城内西去朔方千七百餘里】朔方節度使郭子儀擊敗之【敗補邁翻】子儀乘勝拔靜邉軍【據舊史靜邉軍當在單于府東北王忠嗣鎮河東所築也宋白曰雲中郡西至靜邊軍一百八十里】大同兵馬使薛忠義寇靜邊軍子儀使左兵馬使李光弼右兵馬使高濬左武鋒使僕固懷恩右武鋒使渾釋之等逆擊大破之亢其騎七千【騎奇寄翻下同 考異曰陳雄汾陽王家傳此戰在十二月十二日嫌其與祿山陷東都相亂故并置此】進圍雲中使别將公孫瓊巖將二千騎擊馬邑拔之開東陘關【馬邑郡朔州雁門縣有東陘關西陘關時東河太原閉關以拒秀巖子儀既破秀巖始開關杜祐曰代州雁門郡郡南三十里有東陘關甚險固西陘山即句注山陘音刑下同】甲辰加子儀御史大夫懷恩哥濫拔延之曾孫也世為金微都督【哥濫拔延見一百九十八卷太宗貞觀二十年金微都督府亦置於是年舊史曰僕固即鐵勒僕骨部語訛為僕固】釋之渾部酋長世為臯蘭都督【酋慈由翻長知兩翻】 顔杲卿將起兵參軍馮䖍前真定令賈深藁城尉崔安石郡人翟萬德内丘丞張通幽皆預其謀【真定縣帶常山郡内丘漢中丘縣也隋諱忠改曰内丘屬鉅鹿郡翟萇伯翻】又遣人語太原尹王承業密與相應【語牛倨翻】會顔真卿自平原遣杲卿甥盧逖濳告杲卿欲連兵斷禄山歸路以緩其西入之謀【斷丁管翻下乃斷同】時祿山遣其金吾將軍高邈詣幽州徵兵未還杲卿以禄山命召李欽湊使帥衆詣郡受犒賚【帥讀曰率犒苦到翻賚來代翻】丙午薄暮欽湊至杲卿使袁履謙馮䖍等攜酒食妓樂往勞之【妓渠綺翻勞力到翻下慰勞同】并其黨皆大醉乃斷欽湊首收其甲兵盡縳其黨明日斬之悉散井陘之衆有頃高邈自幽州還且至藁城杲卿使馮䖍往擒之【斷丁管翻是年十一月安祿山使李欽湊屯井陘口今斬之而散其衆陘音刑還從宣翻又音如字】南境又白何千年自東京來崔安石與翟萬德馳詣醴泉驛迎千年又擒之【醴泉驛在常山郡界南直趙郡】同日致於郡下千年謂杲卿曰今太守欲輸力王室既善其始當慎其終此郡應募烏合難以臨敵宜深溝高壘勿與爭鋒俟朔方軍至併力齊進傳檄趙魏斷燕薊要膂【斷音短燕因肩翻薊音計要讀曰腰】今且宜聲云李光弼引步騎一萬出井陘因使人說張獻誠云足下所將多團練之人無堅甲利兵難以當山西勁兵【常山饒陽以并代為山西合天下言之則河南河北通謂之山東函關以西為山西說式芮翻將即亮翻又音如字】獻誠必解圍遁去此亦一奇也杲卿悦用其策獻誠果遁去其團練兵皆潰杲卿乃使人入饒陽城慰勞將士【勞力到翻將即亮翻】命崔安石等狥諸郡云大軍已下井陘朝夕當至先平河北諸郡先下者賞後至者誅於是河北諸郡響應凡十七郡皆歸朝廷兵合二十餘萬 【考異曰河洛春秋曰祿山至藁城杲卿上書陳國忠罪惡宜誅之狀且曰鉞下才不世出天實縱之所向輒平無思不服昔漢高仗赤帝之運猶納食其之言魏武應黄星之符亦用荀彧之策又曰今河北殷實百姓富饒衣冠禮樂天下莫敵孔子曰十室之邑必有忠信萬家之邦非無豪傑如或結聚豈非後患者乎伏惟精彼前軍嚴其後殿所過持重且詳觀地圖凡有隘狹必加防遏慎擇良吏委之腹心自洛以東且為己有廣輓芻粟繕理甲兵傳檄西都望風自振若唐祚未改王命尚行君相恊謀士庶奔命則盛兵鞏洛東據敖倉南臨白馬之津北守飛狐之塞自當抗衡上國割據一方若景命已移謳歌所繫即當長驅岐雍臨馬渭河黔首歸命孰有出鉞下之古者禄山大悦加杲卿章服仍舊常山太守并五軍團練使鎮井陘口留同羅及曳落河一百人首領各一人其趙邢洺相衛等州並皆替換及滄瀛深不從祿山張獻誠圍深州月餘不下前趙州司戶包處遂前原氏尉張通幽藁城縣尉崔安晟恒州長史袁履謙等同上書說杲卿曰明公身荷寵光位居守牧乃棄萬全之良計履必死之畏途取適於目前忘累于身後竊為明公不取今若拒祿山之命招十萬之兵峙乃芻茭積其食粟分守要害大振威聲通井陘之路與東都合勢如此則洪勲盛烈何可勝言者哉輕進瞽言萬無一用䰟銷東岱先懷屠烈之憂心拱北辰願立忠貞之節杲卿覽書大悦於是僉議偽以祿山命追井陘鎮兵就恒州宴設酋長各賜帛三百段馬一疋金銀器物各一牀美人各一其餘通賜物一萬段設於州南焦同驛自曉至暮并以歌妓數百人悦其意密於酒中致毒與飲令盡醉悉無所覺乃盡收其器械一一縳之明日盡斬弃尸於滹沱河中殷亮顔杲卿傳曰祿山起杲卿計無所出乃與長史袁履謙謁于藁城縣禄山以杲卿嘗為己判官矯詔賜紫金魚袋使自守常山郡以其孫誕弟子詢為質俾崇郡刺史蔣欽湊以趙郡甲卒七千人守土門約杲卿將見欽湊以私號召之杲卿罷歸途中指其衣服而謂履謙曰此害身之物也祿山雖以誅君側為名其寶反矣我與公世為唐臣忝居藩翰寜可從之作逆邪履謙愀然變色感歎良久曰為之奈何唯公所命不敢違杲卿乃使人告太原尹王承業以殺欽湊俟其緩急相應承業亦使報命杲卿恐漏泄示己不事事多委政於履謙終日不相謁唯使男泉明往來通其言召前真定令賈深處士權渙郭仲邕就履謙以謀之適會杲卿從父弟真卿據平原殺段子光使杲卿妹子盧逖并以購祿山所行勑牒濳告杲卿大悦匿逖于家逖之未至杲卿先使人以私號召欽湊至杲卿辭之曰日暮夜恐有他盜城門閉矣請俟詰朝相見因遣參軍馮䖍宗室李峻靈夀尉李栖默郡人翟萬德等即于驛亭偶欽湊夜久醉熟以斧斫殺之悉散土門兵先是禄山使其腹心偽金吾將軍高邈徵兵於范陽路出常山杲卿候知之其日邈至於滿城驛杲卿令崔安石馮䖍殺之邈前驅數人先至遽殺之遂生擒邈送于郡遇何千年狎至安石于路絶行人之南者馳至醴泉驛候千年亦斬其人而檎之如邈日未午二凶偕致肅宗實錄杲卿初聞祿山起兵于范陽杲卿召長史袁履謙前真定令賈深内丘丞張通幽謂之曰今祿山一朝以幽并騎過常山趨洛陽有問鼎之志天子在長安方欲徵天下兵東向問罪事不及矣如賊軍暴至吾屬為虜必矣不若因其未萌招義徒西據土門北通河朔待海内之救上以安國家下以全臣節此策之上者遂即日購士得千餘人命履謙將兵鎮土門命賈深防東路通幽守郡城賊將李歸仁令弟欽湊領步騎五千人先鎮土門仍令以兵隸于杲卿又使麾下騎將高邈馳報祿山令促其行覘者知其謀而白杲卿杲卿召履謙告之履謙曰事將亟矣若不早誅欽湊謀不集也遂詐追欽湊令赴郡計事命履謙署人吏以待之欽湊夜至郡杲卿命憇于驛乃使參軍李循馮䖍縣尉李栖默等享欽湊于驛醉而夜殺之履謙持欽湊首謁于杲卿杲卿與履謙且喜事之捷又懼賊之來相對泣杲卿收涙勵履謙曰大丈夫名不掛青史安用生為吾與公累世事唐豈偷安于胡羯但使死而不朽亦何恨也有頃藁城尉崔安石報高邈自禄山所至己宿上谷郡界又使馮䖍縣尉翟萬德并命安石共方略詰朝邈騎數人先至驛䖍盡阬之邈繼至䖍紿之曰太守將音樂迎候邈無疑至廳下馬䖍安石等指揮人吏以棒亂擊邈仆生縳之無何南界又報何千年自東京宿趙郡安石萬德先於郡南醴泉驛候之千年至知邈被擒令麾下騎與安石戰敗又生擒千年並送于郡舊傳曰祿山陷東都杲卿忠誠感懼賊寇潼關即危宗社時從弟真卿為平原太守遣信告杲卿相與起義兵犄角斷賊歸路以紓西寇之勢杲卿乃與長史袁履謙前真定令賈深前内丘丞張通幽等謀閉土門以背之祿山遣蔣欽湊高邈帥衆五千守土門杲卿欲誅欽湊開土門之路時欽湊軍隸常山郡屬欽湊遣高邈往幽州未還杲卿遣吏召欽湊至郡計事是月二十二日夜欽湊至舍之於傳舍會飲既醉令袁履謙與參軍馮䖍縣尉李栖默手力翟萬德等殺欽湊中夜履謙攜欽湊首見杲卿相與垂泣喜事之濟也是夜藁城尉崔安石報高邈還至滿城即令馮䖍翟萬德與安石往圖之詰朝邈之騎從數人至藁城驛安石皆殺之俄而邈至安石紿之曰太守備酒樂於傳舍邈方據廳下馬馮䖍等擒而縶之是日賊將何千年自東都來趙郡馮䖍翟萬德伏兵於醴泉驛千年至又擒之即日縳二賊將還郡按祿山初自范陽擁數十萬衆南下常山當其所出之塗若杲卿不從命遽以千餘人拒之則應時韲粉安得復守故郡乎况時祿山猶以誅楊國忠為名未僭位號杲卿迎于藁城受其金紫殆不能免矣肅宗實錄所云者蓋欲全忠臣之節耳然杲卿忠直剛烈糜軀狥國舍生取義自古罕儔豈肯更上書媚悦祿山比之漢高魏武為之畫割據并吞之策此則粗有知識者必知其不然也蓋包諝乃處遂之子欲言杲卿初無討賊立節之意由己父上書勸成之以大其父功耳觀所載杲卿上祿山書處遂等上杲卿書田承嗣上史朝義疏其文體如一足知皆諝所撰也又張通幽兄為逆黨又教王承業奪杲卿之功終以反覆被誅其行事如此而包諝云初與處遂同上書勸杲卿為忠義尤難信也舊傳云欽湊高邈同守土門欽湊遣邈往幽州二將既握兵同鎮土門欽湊豈得擅遣邈往幽州今從殷亮杲卿傳祿山自遣邈徵兵是也河洛春秋云留同羅曳落河百人彼鎮井陘遏山西之軍重任也豈百人所能守乎殷傳云七千人守土門此七千人又非履謙一夕所能縳也蓋祿山留精兵百人以為欽湊腹心爪牙其餘皆團練民兵脅從者耳故履謙得醉之以酒誅欽湊及百人而散其餘耳河洛春秋云酒中置毒按時履謙等與欽湊同飲豈得偏置毒于客酒中乎今不取舊傳及殷傳皆云欽湊姓蔣今從玄宗肅宗實錄唐歷姓李玄宗實錄十二月己亥杲卿殺賊將李欽湊執何千年高邈送京師按己亥十五日也而真卿以壬寅斬段子光壬寅十八日也真卿既殺子光乃報杲卿同舉義兵今從舊傳為二十二日丙午殺欽湊肅宗實録又云杲卿之斬欽湊等因使狥諸郡曰今上使榮王為元帥哥舒翰為副徵天下兵四十萬東向討逆按實錄癸卯始命翰為副元帥計丙午常山亦未知今不取河洛春秋云十三郡悉舉義兵歸朝廷殷亮顔氏行狀舊顔真卿傳唐歷皆云十七郡歸順蓋河洛春秋不數平原景城河間饒陽先定者耳顔氏行狀曰不欵者六郡而已時魏郡亦未下蓋舉其終數耳】其附祿山者唯范陽盧龍密雲漁陽汲鄴六郡而已【攷唐志無盧龍郡當是改平州北平郡為盧龍郡也密雲郡本檀州安樂郡天寶元年更郡名漁陽郡薊州汲郡衛州】杲卿又密使人入范陽招賈循郟城人馬燧說循曰【郟城漢穎川郟縣之地後魏置龍山縣及南陽縣隋開皇初改龍山曰汝南十八年改汝南曰輔城南陽曰期城大業初改輔城曰郟城廢期城入焉郟音夾說式芮翻】禄山負恩悖逆【悖蒲内翻又蒲没翻】雖得洛陽終歸夷滅公若誅諸將之不從命者以范陽歸國傾其根柢此不世之功也循然之猶豫不時别將牛潤容知之以告禄山禄山使其黨韓朝陽召循朝陽至范陽引循屛語【將即亮翻下同屏必郢翻】使壯士縊殺之滅其族【縊於計翻】以别將牛廷玠知范陽軍事史思明李立節將蕃漢步騎萬人擊博陵常山馬燧亡入西山【范陽郡之西山南連上谷中山之諸山】隱者徐遇匿之得免 初禄山欲自將攻潼關至新安間河北有變而還 【考異曰玄宗實錄十五年正月壬戌祿山將犯潼關次于新安間有備而還按祿山以此月丁酉陷東都至壬戌凡二十六日非乘虛掩襲也豈得至新安然後知其有備乎蓋常山有變則幽薊路絶故懼而歸耳今從肅宗木紀】蔡希德將兵萬人自河内北擊常山【河内郡懷州】 戊申榮王琬薨贈謚靖恭太子 是歲吐蕃贊普乞棃蘇籠獵贊卒子娑悉籠獵贊立【吐從暾入聲卒子恤翻娑素禾翻】<br />
<br />
  肅宗文明武德大聖大宣孝皇帝上之上<br />
<br />
  【諱亨玄宗第三子也初名嗣昇開元十五年更名浚二十三年更名璵二十八年更名紹天寶三載更名亨】<br />
<br />
  至德元載【是年七月太子即位於靈武始改元至德】春正月乙卯朔祿山自稱大燕皇帝改元聖武以達奚珣為侍中張通儒為中書令 【考異曰幸蜀記云以珣為左相通儒為右相今從實錄】高尚嚴莊為中書侍郎 李隨至睢陽有衆數萬丙辰以隨為河南節度使【是載始置河南節度使治汴州領陳留睢陽靈昌淮陽汝陰譙濟陰濮陽淄川琅邪彭城臨淮東海十三郡睢音雖使疏吏翻下同】以前高要尉許遠為睢陽太守兼防禦使【許遠先仕於蜀忤章兼瓊貶高要尉史為許遠堅守睢陽張本】濮陽客尚衡起兵討祿山以郡人王栖曜為衙前總管攻拔濟陰殺祿山將邢超然【濮博木翻濟子禮翻將即亮翻】 顔杲卿使其子泉明賈深翟萬德獻李欽湊首及何千年高邈于京師【翟萇伯翻】張通幽泣請曰通幽兄陷賊【謂通儒也】乞與泉明偕行以救宗族杲卿哀而許之至太原通幽欲自託于王承業乃教之留泉明等更其表【更工衡翻】多自為功毁短杲卿别遣使獻之杲卿起兵纔八日守備未完史思明蔡希德引兵皆至城下 【考異曰河洛春秋云十二月乙未思明希德齊至城下杲卿丙午始殺李欽湊云乙未誤也今從諸書】杲卿告急於承業承業既竊其功利於城陷遂擁兵不救杲卿晝夜拒戰糧盡矢竭壬戌城陷 【考異曰實錄癸亥城陷河洛春秋正月一日城陷舊思明傳正月六日圍常山九日拔之今從玄宗實錄唐歷舊紀杲卿傳】賊縱兵殺萬餘人執杲卿及袁履謙等送洛陽【舊志常山郡京師東北一千七百六十里至東都一千一百三十六里】王承業使者至京師玄宗大喜拜承業羽林大將軍麾下受官爵者以百數徵顔杲卿為衛尉卿朝命未至常山已陷【朝直遥翻】杲卿至洛陽祿山數之曰【數所具翻】汝自范陽戶曹我奏汝為判官不數年超至太守【杲卿為范陽戶曹祿山表為營田判官假常山太守】何負於汝而反邪杲卿瞋目罵曰汝本營州牧羊羯奴【瞋昌真翻羯居謁翻】天子擢汝為三道節度使恩幸無比何負于汝而反我世為唐臣祿位皆唐有雖為汝所奏豈從汝反邪我為國討賊【我為于偽翻】恨不斬汝何謂反也臊羯狗何不速殺我祿山大怒并袁履謙等縳於中橋之柱而冎之【臊蘇遭翻中橋天津中橋也冎古瓦翻】杲卿履謙比死【比必利翻】罵不虛口顔氏一門死於刀鋸者三十餘人史思明李立節蔡希德既克常山引兵撃諸郡之不從者所過殘滅於是鄴廣平鉅鹿趙上谷博陵文安魏信都等郡復為賊守【鉅鹿郡邢州信都郡冀州文安郡莫州復扶又翻】饒陽太守盧全誠獨不從思明等圍之河間司法李奐將七千人景城長史李暐遣其子祀將八千人救之皆為思明所敗【敗補邁翻】 上命郭子儀罷圍雲中還朔方益兵進取東京選良將一人分兵先出井陘定河北子儀薦李光弼癸亥以光弼為河東節度使分朔方兵萬人與之 【考異曰杜牧張保臯傳曰安祿山亂朔方節度使安思順以祿山從弟賜死詔郭汾陽代之後旬日復詔李臨淮持節分朔方半兵東出趙魏當思順時汾陽臨淮俱為牙門都將二人不相能雖周盤飲食常睇相視不交一言及汾陽代思順臨淮欲亡去計未决詔至分汾陽兵東討臨淮入請曰一死固甘乞免妻子汾陽趨下持手上堂偶坐曰今國亂主遷非公不能東伐豈懷私忿時邪悉召軍吏出詔書讀之如詔約束及别執手泣涕相勉以忠義按於時玄宗未幸蜀唐之號令猶行于天下若制書除光弼為節度使子儀安敢擅殺之杜或得於傳聞之誤也今從汾陽家傳及舊傳】 甲子加哥舒翰左僕射同平章事餘如故 置南陽節度使以南陽太守魯炅為之將嶺南黔中襄陽子弟五萬人屯葉北以備安祿山炅表薛愿為潁川太守兼防禦使【南陽郡鄧州襄陽郡襄州葉縣時屬汝州潁川郡許州炅火迥翻將即亮翻黔音琹葉式涉翻】龎堅為副使愿故太子瑛之妃兄堅玉之曾孫也【龎玉去隋歸唐為將龎皮江翻】 乙丑安祿山遣其子慶緒寇潼關哥舒翰擊却之 己巳加顔真卿戶部侍郎兼本郡防禦使真卿以李暐為副 二月丙戌加李光弼魏郡太守河北道采訪使 史思明等圍饒陽二十九日不下李光弼將蕃漢步騎萬餘人太原弩手三千人出井陘【騎奇寄翻陘音刑 考異曰玄宗實錄己亥光弼以朔方馬步五千東出土門收常山郡河洛春秋云光弼從大同城下領蕃漢兵馬步一萬餘人并太原弩手三千人救真定蓋實錄言朔方元領之兵河洛言到真定之數耳】己亥至常山常山團練兵三千人殺胡兵執安思義出降【降戶江翻】光弼謂思義曰汝自知當死否思義不應光弼曰汝久更陳行【更工衡翻陳讀曰陣行胡剛翻】視吾此衆可敵思明否今為我計當如何汝策可取當不殺汝思義曰大夫士馬遠來疲弊猝遇大敵恐未易當【易以䜴翻】不如移軍入城早為備禦先料勝負然後出兵胡騎雖銳不能持重【騎奇寄翻下同】苟不獲利氣沮心離於時乃可圖矣思明今在饒陽此去不二百里【九域志真定至饒陽二百三十五里忠義蓋指思明下營處言之】昨暮羽書已去計其先鋒來晨必至而大軍繼之不可不留意也光弼悦釋其縳即移軍入城史思明聞常山不守立解饒陽之圍明日未旦先鋒已至思明等繼之合二萬餘騎直抵城下光弼遣步卒五千自東門出戰賊守門不退光弼命五百弩於城上齊射之【射而亦翻下兵射同】賊稍却乃出弩手千人分為四隊使其矢相繼賊不能當歛軍道北光弼出兵五千為槍城於道南夾呼沱水而陳賊數以騎兵摶戰光弼之兵射之人馬中矢者大半【陳讀曰陣數所角翻中竹仲翻】乃退小憇以俟步兵有村民告賊步兵五千自饒陽來晝夜行百七十里至九門南逢壁度憇息【九門縣屬常山郡在郡東宋白曰戰國策云本有九室而居趙武靈王改為九門縣憩去例翻】光弼遣步騎各二千匿旗鼔並水濳行【並步浪翻】至逢壁賊方飯縱兵掩擊殺之無遺思明聞之失勢退入九門時常山九縣【真定藁城石邑九門行唐井陘平山獲鹿靈夀凡九縣】七附官軍惟九門藁城為賊所據光弼遣禆將張奉璋以兵五百戍石邑【石邑縣自漢以來屬常山郡在郡西南戍兵多於餘縣者所以通太原之路也宋白曰隋改漢上曲陽縣為石邑尋移石邑於井陘縣於舊石邑縣置恒陽縣以在恒山之陽為名則此石邑在井陘也】餘皆三百人戍之 上以吳王祗為靈昌太守河南都知兵馬使【上謂玄宗使疏吏翻】賈賁前至雍丘有衆二千先是譙郡太守楊萬石以郡降安祿山【先悉薦翻降戶江翻】逼真源令河東張巡使為長史西迎賊巡至真源帥吏民哭于玄元皇帝廟【雍丘縣漢晉屬陳留郡後魏屬湯夏郡隋屬梁郡唐屬汴州譙郡亳州老子苦縣人有祠在焉唐祖之故改縣曰真源九域志縣在譙郡西七十里帥讀曰率】起兵討賊吏民樂從者數千人【樂音洛】巡選精兵千人西至雍丘與賈賁合初雍丘令令狐潮以縣降賊賊以為將使東擊淮陽救兵于襄邑破之【淮陽郡陳州宋白曰襄邑縣春秋宋襄牛地也宋襄公葬焉故曰襄陵今墓在縣西北隅秦始皇以承匡縣卑濕遂徙于襄陵又以陵字犯諱改為襄邑】俘百餘人拘於雍丘將殺之往見李庭望淮陽兵遂殺守者潮棄妻子走故賈賁得以其間入雍丘【間古莧翻考異曰肅宗實録曰雍丘令令狐潮據城以應祿山百姓有違令者百餘人將殺之覘者報官軍至潮不及行刑遂反縛仆於地令人守之遽出軍以禦官軍縛者忽一人幸脱殺守者互解其縛閉城門以拒潮相持累日賁聞之入其城領衆殺潮母妻及子以堅人志舊張巡傳潮欲以城降賊民吏百餘人不從命潮皆反接仆之於地將斬之會賊來攻城潮遽出鬭而反接者自解其縛閉城門拒潮召賁賁與巡引衆入雍丘新傳潮舉縣附賊遂自將東敗淮陽兵虜其衆反接在庭將殺之暫出行部淮陽兵更解縛起殺守者迎賁等入潮不得歸巡乃屠其妻子磔城上按潮既以城降賊賊來即當出迎豈有更出鬭者今從李翰張中丞傳及新傳】庚子潮引賊精兵攻雍丘賁出戰敗死張巡力戰却賊因兼領賁衆自稱吳王先鋒使三月乙卯潮復與賊將李懷仙楊朝宗謝元同等四萬餘衆奄至城下衆懼莫有固志【復扶又翻下日復瘡復同朝直遥翻】巡曰賊兵精銳有輕我心今出其不意擊之彼必驚潰賊勢小折然後城可守也乃使千人乘城自率千人分數隊開門突出巡身先士卒直衝賊陳人馬辟易【先悉薦翻陳讀曰陣辟讀曰闢易讀如字】賊遂退明日復進攻城設百礮環城【復扶又翻礮與砲同環音宦】樓堞皆盡【堞達協翻】巡于城上立木柵以拒之賊蟻附而登巡束蒿灌脂焚而投之賊不得上【上時掌翻】時伺賊隙出兵擊之或夜縋斫營【伺相吏翻縋馳偽翻】積六十餘日大小三百餘戰帶甲而食裹瘡復戰賊遂敗走巡乘勝追之獲胡兵二千人而還軍聲大振【還從宣翻又音如字】 初戶部尚書安思順知禄山反謀因入朝奏之【尚辰羊翻朝直遥翻】及祿山反上以思順先奏不之罪也哥舒翰素與之有隙【事見上卷天寶十載】使人詐為禄山遺思順書於關門擒之以獻且數思順七罪請誅之【遺于季翻數所角翻又所主翻】丙辰思順及弟太僕卿元貞皆坐死家屬徙嶺外楊國忠不能救由是始畏翰 郭子儀至朔方益選精兵戊午進軍于代【此代謂代州】 戊辰吳王祗擊謝元同走之拜陳留太守河南節度使【守式又翻使疏吏翻】 壬午以河東節度使李光弼為范陽長史河北節度使【長知兩翻 考異曰實錄云乙丑光弼收趙郡按壬午三月二十九日乙丑十二日也河洛春秋收趙郡在四月今從之】加顔真卿河北采訪使真卿以張澹為支使先是清河客李萼【先悉薦翻考異曰顔氏行狀作李華今從舊傳】年二十餘為郡人乞師於真卿【為于】<br />
<br />
  【偽翻】曰公首唱大義河北諸郡恃公以為長城今清河公之西鄰【清河郡貝州九域志德州西南至貝州二百三十里】國家平日聚江淮河南錢帛於彼以贍北軍【贍時艶翻】謂之天下北庫今有布三百餘萬匹帛八十餘萬匹錢三十餘萬緡糧三十餘萬斛㫺討默啜甲兵皆貯清河庫【謂武后時也啜陟劣翻貯丁呂翻】今有五十餘萬事【一物可以給一事因謂之事】戶七萬口十餘萬竊計財足以三平原之富兵足以倍平原之彊公誠資以士卒撫而有之以二郡為腹心則餘郡如四支無不隨所使矣真卿曰平原兵新集尚未訓練自保恐不足何暇及鄰雖然借若諾子之請則將何為乎萼曰清河遣僕銜命於公者非力不足而借公之師以嘗寇也亦欲觀大賢之明義耳今仰瞻高意未有決辭定色僕何敢遽言所為哉真卿奇之欲與之兵衆以為萼年少輕慮【少詩照翻】徒分兵力必無所成真卿不得已辭之萼就館復為書說真卿【說扶又翻說式芮翻】以為清河去逆效順奉粟帛器械以資軍公乃不納而疑之僕囘轅之後清河不能孤立必有所繋託將為公西面之彊敵公能無悔乎真卿大驚遽詣其館以兵六千借之送至境執手别真卿問曰兵已行矣可以言子之所為乎萼曰聞朝廷遣程千里將精兵十萬出崞口討賊【崞口在洛州邯鄲縣西蓋即壺關之險也又按舊唐書崞口在湘州西山崞音郭】賊據險拒之不得前今當引兵先擊魏郡執祿山所署太守袁知泰納舊太守司馬垂使為西南主人分兵開崞口出千里之師因討汲鄴以北至于幽陵郡縣之未下者【幽陵即謂幽州】平原清河帥諸同盟【帥讀曰率】合兵十萬南臨孟津分兵循河據守要害制其北走之路計官軍東討者不下二十萬河南義兵西向者亦不減十萬公但當表朝廷堅壁勿戰不過月餘賊必有内潰相圖之變矣真卿曰善命錄事參軍李擇交及平原令范冬馥將其兵【平原縣屬平原郡古平原郡治焉故城在今縣西南二十五里今縣治北齊所築城時平原郡治安德縣】會清河兵四千及博平兵千人軍于堂邑西南【宋白曰堂邑縣屬博平郡本漢清縣發于二縣地隋開皇十六年於此置堂邑縣因縣西北有堂邑故城為名】袁知泰遣其將白嗣恭等將二萬餘人來逆戰三郡兵力戰盡日魏兵大敗斬首萬餘級捕虜千餘人得馬千匹軍資甚衆知泰奔汲郡遂克魏郡軍聲大振時北海太守賀蘭進明亦起兵真卿以書召之并力【北海郡青州】進明將步騎五千度河【將即亮翻騎奇寄翻下同】真卿陳兵逆之相揖哭於馬上哀動行伍【行戶江翻】進明屯平原城南休養士馬真卿每事咨之由是軍權稍移於進明矣真卿不以為嫌真卿以堂邑之功讓進明進明奏其狀取捨任意勑加進明河北招討使擇交冬馥微進資級清河博平有功者皆不錄進明攻信都郡久之不克錄事參軍長安第五琦勸進明厚以金帛募勇士遂克之 【考異曰顔氏行狀云進明失律於信都城下有詔抵罪公縱之使赴行在進明之全乃公之護也今從舊傳又唐歷三月四日乙酉真卿充河北采訪使時進明起義兵北度河與真卿同經畧六月真卿破袁知泰於堂邑進明再拔信都統紀皆在三月舊紀破知泰于信都皆在六月按三月無乙酉乙酉四月二日也今從統紀】 李光弼與史思明相守四十餘日思明絶常山糧道城中乏草馬食薦藉【藉慈夜翻】光弼以車五百乘之石邑取草【之往也乘䋲證翻】將車者皆衣甲弩手千人衛之為方陳而行【衣於既翻陳讀曰陣】賊不能奪蔡希德引兵攻石邑張奉璋拒却之光弼遣使告急于郭子儀子儀引兵自井陘出【使疏吏翻陘音刑】夏四月壬辰至常山與光弼合蕃漢步騎共十餘萬甲午子儀光弼與史思明等戰于九門城南【宋白曰九門縣戰國趙邑戰國策云本有九室而居趙武靈王改為九門縣】思明大敗中郎將渾瑊射李立節殺之【將即亮翻渾胡昆翻又戶本翻瑊古咸翻射而亦翻】瑊釋之之子也思明收餘衆奔趙郡蔡希德奔鉅鹿思明自趙郡如博陵時博陵已降官軍【降戶江翻下同】思明盡殺郡官河朔之民苦賊殘暴所至屯結多至二萬人少者萬人各為營以拒賊及郭李軍至爭出自效【少詩沼翻】庚子攻趙郡一日城降士卒多虜掠光弼坐城門收所獲悉歸之民大悦子儀生擒四千人皆捨之斬祿山太守郭獻璆【璆音求】光弼進圍博陵十日不拔引兵還恒陽就食【恒陽即恒山郡以其地在恒山之陽也唐置恒陽軍於郡北又博陵郡有恒陽縣漢之上曲陽縣也隋改為恒陽縣在博陵西十里恒戶登翻還從宣翻又音如字】 楊國忠問士之可為將者於左拾遺博平張鎬及蕭昕【鎬下老翻昕許斤翻將即亮翻下同】鎬昕薦左贊善大夫永夀來瑱【武德二年分新平置永夀縣屬邠州瑱他見翻】丙午以瑱為頴川太守賊屢攻之瑱前後破賊甚衆加本郡防禦使【守式又翻使疏吏翻】人謂之來嚼鐵 安祿山使平盧節度使呂知誨誘安東副大都護馬靈詧殺之【馬靈詧即夫蒙靈詧也開元二年徙安東都護府于平州天寶二年徙於遼西故郡城誘羊久翻】平盧遊奕使武陟劉客奴【武陟漢陽縣地隋開皇十六年分置武陟縣時屬河内郡】先鋒使董秦及安東將王玄志同謀討誅知誨遣使踰海與顔真卿相聞請取范陽以自效真卿遣判官賈載齎糧及戰士衣助之真卿時惟一子頗纔十餘歲使詣客奴為質【質音致】朝廷聞之以客奴為平盧節度使【朝直遥翻】賜名正臣玄志為安東副大都護董秦為平盧兵馬使 南陽節度使魯炅立柵於滍水之南安禄山將武令珣畢思琛攻之【滍直里翻】<br />
<br />
  資治通鑑卷二百十七<br />
<br />
<史部,編年類,資治通鑑>  <br>
   </div> 

<script src="/search/ajaxskft.js"> </script>
 <div class="clear"></div>
<br>
<br>
 <!-- a.d-->

 <!--
<div class="info_share">
</div> 
-->
 <!--info_share--></div>   <!-- end info_content-->
  </div> <!-- end l-->

<div class="r">   <!--r-->



<div class="sidebar"  style="margin-bottom:2px;">

 
<div class="sidebar_title">工具类大全</div>
<div class="sidebar_info">
<strong><a href="http://www.guoxuedashi.com/lsditu/" target="_blank">历史地图</a></strong>  
<a href="http://www.880114.com/" target="_blank">英语宝典</a>  
<a href="http://www.guoxuedashi.com/13jing/" target="_blank">十三经检索</a> 
<br><strong><a href="http://www.guoxuedashi.com/gjtsjc/" target="_blank">古今图书集成</a></strong> 
<a href="http://www.guoxuedashi.com/duilian/" target="_blank">对联大全</a> <strong><a href="http://www.guoxuedashi.com/xiangxingzi/" target="_blank">象形文字典</a></strong> 

<br><a href="http://www.guoxuedashi.com/zixing/yanbian/">字形演变</a>  <strong><a href="http://www.guoxuemi.com/hafo/" target="_blank">哈佛燕京中文善本特藏</a></strong>
<br><strong><a href="http://www.guoxuedashi.com/csfz/" target="_blank">丛书&方志检索器</a></strong> <a href="http://www.guoxuedashi.com/yqjyy/" target="_blank">一切经音义</a>  

<br><strong><a href="http://www.guoxuedashi.com/jiapu/" target="_blank">家谱族谱查询</a></strong>  <strong><a href="http://shufa.guoxuedashi.com/sfzitie/" target="_blank">书法字帖欣赏</a></strong> 
<br>

</div>
</div>


<div class="sidebar" style="margin-bottom:0px;">

<font style="font-size:22px;line-height:32px">QQ交流群9:489193090</font>


<div class="sidebar_title">手机APP 扫描或点击</div>
<div class="sidebar_info">
<table>
<tr>
	<td width=160><a href="http://m.guoxuedashi.com/app/" target="_blank"><img src="/img/gxds-sj.png" width="140"  border="0" alt="国学大师手机版"></a></td>
	<td>
<a href="http://www.guoxuedashi.com/download/" target="_blank">app软件下载专区</a><br>
<a href="http://www.guoxuedashi.com/download/gxds.php" target="_blank">《国学大师》下载</a><br>
<a href="http://www.guoxuedashi.com/download/kxzd.php" target="_blank">《汉字宝典》下载</a><br>
<a href="http://www.guoxuedashi.com/download/scqbd.php" target="_blank">《诗词曲宝典》下载</a><br>
<a href="http://www.guoxuedashi.com/SiKuQuanShu/skqs.php" target="_blank">《四库全书》下载</a><br>
</td>
</tr>
</table>

</div>
</div>


<div class="sidebar2">
<center>


</center>
</div>

<div class="sidebar"  style="margin-bottom:2px;">
<div class="sidebar_title">网站使用教程</div>
<div class="sidebar_info">
<a href="http://www.guoxuedashi.com/help/gjsearch.php" target="_blank">如何在国学大师网下载古籍?</a><br>
<a href="http://www.guoxuedashi.com/zidian/bujian/bjjc.php" target="_blank">如何使用部件查字法快速查字?</a><br>
<a href="http://www.guoxuedashi.com/search/sjc.php" target="_blank">如何在指定的书籍中全文检索?</a><br>
<a href="http://www.guoxuedashi.com/search/skjc.php" target="_blank">如何找到一句话在《四库全书》哪一页?</a><br>
</div>
</div>


<div class="sidebar">
<div class="sidebar_title">热门书籍</div>
<div class="sidebar_info">
<a href="/so.php?sokey=%E8%B5%84%E6%B2%BB%E9%80%9A%E9%89%B4&kt=1">资治通鉴</a> <a href="/24shi/"><strong>二十四史</strong></a>&nbsp; <a href="/a2694/">野史</a>&nbsp; <a href="/SiKuQuanShu/"><strong>四库全书</strong></a>&nbsp;<a href="http://www.guoxuedashi.com/SiKuQuanShu/fanti/">繁体</a>
<br><a href="/so.php?sokey=%E7%BA%A2%E6%A5%BC%E6%A2%A6&kt=1">红楼梦</a> <a href="/a/1858x/">三国演义</a> <a href="/a/1038k/">水浒传</a> <a href="/a/1046t/">西游记</a> <a href="/a/1914o/">封神演义</a>
<br>
<a href="http://www.guoxuedashi.com/so.php?sokeygx=%E4%B8%87%E6%9C%89%E6%96%87%E5%BA%93&submit=&kt=1">万有文库</a> <a href="/a/780t/">古文观止</a> <a href="/a/1024l/">文心雕龙</a> <a href="/a/1704n/">全唐诗</a> <a href="/a/1705h/">全宋词</a>
<br><a href="http://www.guoxuedashi.com/so.php?sokeygx=%E7%99%BE%E8%A1%B2%E6%9C%AC%E4%BA%8C%E5%8D%81%E5%9B%9B%E5%8F%B2&submit=&kt=1"><strong>百衲本二十四史</strong></a>  <a href="http://www.guoxuedashi.com/so.php?sokeygx=%E5%8F%A4%E4%BB%8A%E5%9B%BE%E4%B9%A6%E9%9B%86%E6%88%90&submit=&kt=1"><strong>古今图书集成</strong></a>
<br>

<a href="http://www.guoxuedashi.com/so.php?sokeygx=%E4%B8%9B%E4%B9%A6%E9%9B%86%E6%88%90&submit=&kt=1">丛书集成</a> 
<a href="http://www.guoxuedashi.com/so.php?sokeygx=%E5%9B%9B%E9%83%A8%E4%B8%9B%E5%88%8A&submit=&kt=1"><strong>四部丛刊</strong></a>  
<a href="http://www.guoxuedashi.com/so.php?sokeygx=%E8%AF%B4%E6%96%87%E8%A7%A3%E5%AD%97&submit=&kt=1">說文解字</a> <a href="http://www.guoxuedashi.com/so.php?sokeygx=%E5%85%A8%E4%B8%8A%E5%8F%A4&submit=&kt=1">三国六朝文</a>
<br><a href="http://www.guoxuedashi.com/so.php?sokeytm=%E6%97%A5%E6%9C%AC%E5%86%85%E9%98%81%E6%96%87%E5%BA%93&submit=&kt=1"><strong>日本内阁文库</strong></a> <a href="http://www.guoxuedashi.com/so.php?sokeytm=%E5%9B%BD%E5%9B%BE%E6%96%B9%E5%BF%97%E5%90%88%E9%9B%86&ka=100&submit=">国图方志合集</a> <a href="http://www.guoxuedashi.com/so.php?sokeytm=%E5%90%84%E5%9C%B0%E6%96%B9%E5%BF%97&submit=&kt=1"><strong>各地方志</strong></a>

</div>
</div>


<div class="sidebar2">
<center>

</center>
</div>
<div class="sidebar greenbar">
<div class="sidebar_title green">四库全书</div>
<div class="sidebar_info">

《四库全书》是中国古代最大的丛书,编撰于乾隆年间,由纪昀等360多位高官、学者编撰,3800多人抄写,费时十三年编成。丛书分经、史、子、集四部,故名四库。共有3500多种书,7.9万卷,3.6万册,约8亿字,基本上囊括了古代所有图书,故称“全书”。<a href="http://www.guoxuedashi.com/SiKuQuanShu/">详细>>
</a>

</div> 
</div>

</div>  <!--end r-->

</div>
<!-- 内容区END --> 

<!-- 页脚开始 -->
<div class="shh">

</div>

<div class="w1180" style="margin-top:8px;">
<center><script src="http://www.guoxuedashi.com/img/plus.php?id=3"></script></center>
</div>
<div class="w1180 foot">
<a href="/b/thanks.php">特别致谢</a> | <a href="javascript:window.external.AddFavorite(document.location.href,document.title);">收藏本站</a> | <a href="#">欢迎投稿</a> | <a href="http://www.guoxuedashi.com/forum/">意见建议</a> | <a href="http://www.guoxuemi.com/">国学迷</a> | <a href="http://www.shuowen.net/">说文网</a><script language="javascript" type="text/javascript" src="https://js.users.51.la/17753172.js"></script><br />
  Copyright &copy; 国学大师 古典图书集成 All Rights Reserved.<br>
  
  <span style="font-size:14px">免责声明:本站非营利性站点,以方便网友为主,仅供学习研究。<br>内容由热心网友提供和网上收集,不保留版权。若侵犯了您的权益,来信即刪。scp168@qq.com</span>
  <br />
ICP证:<a href="http://www.beian.miit.gov.cn/" target="_blank">鲁ICP备19060063号</a></div>
<!-- 页脚END --> 
<script src="http://www.guoxuedashi.com/img/plus.php?id=22"></script>
<script src="http://www.guoxuedashi.com/img/tongji.js"></script>

</body>
</html>
