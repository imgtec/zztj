\chapter{資治通鑑卷一百八十一}
宋 司馬光 撰

胡三省 音註

隋紀五|{
	起著雍執徐盡玄黓涒灘凡五年}


煬皇帝上之下

大業四年春正月乙巳詔發河北諸軍百餘萬穿永濟渠引沁水南達于河北通涿郡|{
	班志沁水出上黨穀遠縣羊頭山世靡谷師古曰今至懷州武陟縣界入河穀遠隋為沁源縣沁七鴆翻 考異曰雜記三年六月敕開永濟渠引汾水入河於汾水東北開渠合渠水至於涿郡二千餘里通龍舟按永濟渠即今御河未嘗通汾水雜記誤也}
丁男不供始役婦人 壬申以太府卿元壽為内史令 裴矩聞西突厥處羅可汗思其母請遣使招懷之二月己卯帝遣司朝謁者崔君肅齎詔書慰諭之|{
	帝置謁者臺大夫一人置司朝謁者二人以貳之處昌呂翻厥九勿翻可從刋入聲汗音寒使疏吏翻朝直遙翻 考異曰隋帝紀作崔毅今從西突厥傳}
處羅見君肅甚踞受詔不肯起君肅謂之曰突厥本一國中分為二每歲交兵積數十歲而莫能相滅者明知其勢敵耳然啓民舉其部落百萬之衆卑躬折節入臣天子者其故何也|{
	折而列翻}
正以切恨可汗不能獨制欲借兵於大國共滅可汗耳羣臣咸欲從啓民之請天子既許之師出有日矣顧可汗母向夫人|{
	可從刋入聲汗音寒向式亮翻}
懼西國之滅旦夕守闕哭泣哀祈匍匐謝罪請發使召可汗令入内屬天子憐之故復遣使至此|{
	匍音蒲匐蒲北翻使疏吏翻下同令力丁翻復扶又翻}
今可汗乃踞慢如此則向夫人為誑天子|{
	誑居况翻}
必伏尸都市傳首虜庭|{
	虜庭謂啓民庭}
發大隋之兵資東國之衆左提右挈以擊可汗亡無日矣奈何愛兩拜之禮絶慈母之命惜一語稱臣使社稷為墟乎處羅矍然而起|{
	處昌呂翻矍居縳翻}
流涕再拜跪受詔書因遣使者隨君肅貢汗血馬 三月壬戍倭王多利思比孤|{
	隋書倭國在百濟新羅東南水陸三千里於大海之中依山島而居都於邪靡堆則魏志所謂邪馬臺者也在會稽之東與儋耳相近杜佑曰倭在帶方東南大海中去遼東萬二千里大較在閩川會稽之東亦與朱崖儋耳相近自謂太伯之後一名日本自云國在日邊因以為稱倭烏禾翻}
入貢遺帝書曰|{
	遺于季翻}
日出處天子致書日没處天子無恙|{
	恙余亮翻}
帝覽之不悦謂鴻臚卿曰|{
	臚凌如翻}
蠻夷書無禮者勿復以聞|{
	復扶又翻}
乙丑車駕幸五原|{
	帝改豐州為五原部}
因出塞廵長城|{
	去年所築者}
行宫設六合板城|{
	隋志帝北廵出塞行宫設六合城方一百二十步高四丈二尺六合以木為之方一尺外面一方有板離合為之塗以青色壘六板為城高三丈六尺上加女牆板高六尺開南北門又於城西角起樓敵二門觀門樓檻皆丹青綺畫又造六合殿千人帳載以槍車車載六合三板其車軨解合交义即為馬槍皆車上張幕幕下張平一弩傳矢五人更守兩車之間施車軨馬槍皆外其轅以為外圍次内布鐵菱次内施蟄鞬中施弩牀長六尺濶三尺牀桄陛挿鋼錐皆長五寸謂之蝦鬚皆施機關張則錐皆外向其牀上施旋機弩以繩連弩機人從外來觸繩則弩機旋轉向所觸而發其外又以矰周圍行宫二丈一鈴一柱柱舉矰去地二尺五寸當行宫南北門施槌磬連矰以機發之有人觸矰則衆鈴發響槌擊兩磬以知所警名為擊磬考異曰雜記云帝幸啓民帳造行城周二千步高二十餘丈今從隋禮儀志}
載以槍車每頓舍則外其轅以為外圍内布鐵菱|{
	爾雅翼曰今軍旅以鐵作茨以布敵路謂之鐵蒺藜或云鐵蒺藜菱角起於煬帝征遼為之然六韜中已有此物晁錯傳謂之渠答}
次施弩牀皆挿鋼錐|{
	鋼音剛精鐵也}
外向上施旋機弩以繩連機人來觸繩則弩機旋轉向所觸而發其外又以矰周圍施鈴柱槌磬以知所警|{
	矰作滕翻槌直追翻}
帝募能通絶域者屯田主事常駿等請使赤土|{
	屯田曹屬工部尚書尚書諸曹各有主事流外吏職也隋書赤土國扶南之别種在南海中水行百餘日而達所都土色多赤因以為號杜佑曰崖州直南水行便風十餘日到赤土國其國到五月日亭午物影都在南一日三食飯皆旋炊不然逡廵過時即便臭敗熱氣特甚使疏利翻}
帝大悦丙寅命駿齎物五千段以賜其王赤土者南海中遠國也 帝無日不治宫室|{
	治直之翻}
兩京及江都苑囿亭殿雖多久而益厭每遊幸左右顧矚|{
	矚之欲翻}
無可意者不知所適乃備責天下山川之圖躬自歷覽以求勝地可置宫苑者夏四月詔於汾州之北汾水之源營汾陽宫|{
	隋志樓煩郡汾源縣舊岢嵐也大業四年改為靜樂有汾陽宫管涔山天池汾水十三州志汾水出武州之燕京山管涔之異名也水經注燕京山上有大池世謂之天池按煬帝起汾陽宫環天池詳見後五臺註}
初元德太子薨|{
	見上卷二年}
河南尹齊王暕次當為嗣元德吏兵二萬餘人悉隸於暕|{
	暕古限翻}
帝為之妙選僚屬|{
	為于偽翻}
以光祿少卿柳謇之為齊王長史|{
	少始照翻謇九輦翻長知兩翻}
且戒之曰齊王德業修備富貴自鍾卿門|{
	鍾聚也}
若有不善罪亦相及謇之慶之從子也|{
	柳慶事宇文泰從才用翻}
暕寵遇日隆百官趨謁闐咽道路暕以是驕恣昵近小人|{
	闐田年翻昵尼質翻近其靳翻}
所為多不法遣左右喬令則庫狄仲錡|{
	庫狄複姓錡魚豈翻}
陳智偉求聲色令則等因此放縱訪人家有美女輒矯暕命呼之載入暕第淫而遣之仲錡智偉詣隴西撾炙諸胡責其名馬|{
	帝改渭州為隴西郡撾側瓜翻}
得數匹以進暕暕令還主仲錡等詐言王賜取歸其家暕不知也樂平公主嘗奏帝言柳氏女美|{
	樂平公主周天元后也樂音洛}
帝未有所答久之主復以柳氏進暕|{
	復扶又翻}
暕納之其後帝問主柳氏女安在主曰在齊王所帝不悦暕從帝幸汾陽宫大獵詔暕以千騎入圍|{
	騎奇寄翻}
暕大獲麋鹿以獻而帝未有得也乃怒從官皆言為暕左右所遏獸不得前|{
	從才用翻}
帝於是發怒求暕罪失時制縣令無故不得出境有伊闕令皇甫詡得幸於暕違禁攜之至汾陽宫御史韋德裕希旨劾奏暕|{
	劾戶槩翻又戶得翻}
帝令甲士千餘人大索暕第|{
	令力丁翻索山客翻}
因窮治其事|{
	治直之翻}
暕妃韋氏早卒|{
	卒子恤翻}
暕與妃姊元氏婦通產一女暕召相工|{
	相息亮翻}
令徧視後庭相工指妃姊曰此產子者當為皇后暕以元德太子有三子|{
	三子侑倓侗}
恐不得立隂挾左道為猒勝|{
	猒於葉翻}
至是皆發帝大怒斬令則等數人賜妃姊死暕府僚皆斥之邊遠柳謇之坐不能匡正除名|{
	謇九輦翻}
時趙王杲尚幼帝謂侍臣曰朕唯有暕一子不然者當肆諸市朝以明國憲暕自是恩寵日衰雖為京尹不復關預時政帝恒令虎賁郎將一人監其府事|{
	帝制十二衛每衛置護軍四人掌副貳將軍無則一人攝尋改護軍為虎賁郎將正四品朝直遙翻復扶又翻恒戶登翻令力丁翻賁音奔將即亮翻監古銜翻}
暕有微失虎賁輒奏之帝亦常慮暕生變所給左右皆以老弱備員而已太史令庾質季才之子也其子為齊王屬|{
	隋王府有掾有屬}
帝謂質曰汝不能一心事我乃使兒事齊王何向背如此|{
	背蒲妹翻}
對曰臣事陛下子事齊王實是一心不敢有二帝猶怒出為合水令|{
	開皇十六年置合水縣為慶州治所帝改慶州為弘化郡唐改合水縣為安化}
乙卯詔以突厥啓民可汗|{
	厥九勿翻可從刋入聲汗音寒}
遵奉朝化思改戎俗宜於萬壽戌置城造屋其帷帳牀褥以上務從優厚 秋七月辛巳發丁男二十餘萬築長城自榆谷而東|{
	此榆谷當在榆林西}
裴矩說鐵勒|{
	說式芮翻}
使擊吐谷渾大破之吐谷渾可汗伏允東走入西平境内|{
	帝改鄯州為西平郡吐讀暾入聲谷音浴可從刋入聲汗音寒}
遣使請降求救帝遣安德王雄出澆河|{
	前已書觀王雄此復書安德王雄何也按雄傳雄從帝征吐谷渾還方徙封觀王高熲誅之時雄尚為安德王通鑑因舊史成文而書之耳帝改廓州為澆河郡使疏吏翻降戶江翻澆古堯翻}
許公宇文述出西平迎之|{
	宇文述封許國公}
述至臨羌城|{
	漢臨羌縣城也}
吐谷渾畏述兵盛不敢降帥衆西遁|{
	帥讀曰率}
述引兵追之拔曼頭赤水二城|{
	隋志帝平吐谷渾置河源郡於古赤水城管下有曼頭城曼音萬}
斬三千餘級獲其王公以下二百人虜男女四千口而還|{
	還從宣翻又如字下同}
伏允南奔雪山|{
	此即蜀西山之西雪山也}
其故地皆空東西四千里南北二千里皆為隋有置州縣鎮戍|{
	置鄯善且末西海河源四郡顯武濟遠肅寧伏戎宣德威定遠化赤水等縣志云置於五年}
天下輕罪徙居之 八月辛酉上親祠恒岳|{
	恒岳北岳恒山恒戶登翻}
赦天下河北道郡守畢集|{
	守式又翻}
裴矩所致西域十餘國皆來助祭 |{
	考異曰裴矩傳云三年誤也今從帝紀}
九月辛未徵天下鷹師悉集東京|{
	鷹師善調習鷹隼者也}
至者萬餘人 冬十月乙卯頒新式|{
	去年四月壬辰改度量權衡並依古式今頒於天下}
常駿等至赤土境赤土王利富多塞遣使以三十舶迎之進金鏁以纜駿船|{
	使疏吏翻下同舶莫百翻鏁蘇果翻}
凡汎海百餘日入境月餘乃至其都|{
	赤土所都名僧祗城}
其王居處|{
	處昌呂翻}
器用窮極珍麗待使者禮亦厚遣其子那邪迦隨駿入貢|{
	迦音加}
帝以右翊衛將軍河東薛世雄為玉門道行軍大將|{
	帝改蒲州為河東郡隋志玉門縣屬敦煌郡改行軍總管為行軍大將將即亮翻}
與突厥啓民可汗連兵擊伊吾|{
	厥九勿翻可從刋入聲汗音寒 考異曰世雄擊伊吾帝紀無之本傳前有從帝征吐谷渾後云歲餘以世雄為玉門大將與突厥啓民可汗擊伊吾然則似在大業六七年也按是時啓民已卒伐吐谷渾之歲伊吾吐屯設獻地數千里恩寵甚厚隋何故伐之今移在獻地之前}
師出玉門啓民不至世雄孤軍度磧伊吾初謂隋軍不能至皆不設備聞世雄軍已度磧大懼請降|{
	流沙亦謂之磧磧七迹翻降戶江翻}
世雄乃於漢故伊吾城東築城留銀青光祿大夫王威以甲卒千餘人戍之而還|{
	還從宣翻又音如字}
五年春正月丙子改東京為東都 突厥啓民可汗來朝禮賜益厚|{
	厥九勿翻可從刋入聲汗音寒朝直遙翻}
癸未詔天下均田戊子上自東都西還 己丑制民間鐵义撘鉤刃

之類皆禁之|{
	撘多臘翻作管翻}
二月戊申車駕至西京 三月己巳西廵河右乙亥幸扶風舊宅|{
	河右河西武威諸郡地帝改岐州為扶風郡}
夏四月癸亥出臨津關|{
	臨津關當在枹罕界臨河津水經注河水自澆河東流逕邯川城南又東逕臨津城北白土城南為緣河濟渡之地}
度黄河至西平陳兵講武將擊吐谷渾五月乙亥上大獵於拔延山|{
	隋志西平郡化隆縣有拔延山杜祐曰拔延山在廓州廣威縣隋煬帝征吐渾經此山吐從入聲谷音浴}
長圍亘二十里 |{
	考異曰隋唐紀作二千里疑二十里字誤}
庚辰入長寧谷|{
	長寧谷在古晉昌郡界水經注湟水逕臨羌縣故城南又東長寧川水注之長寧水東南流逕晉昌川又有長寧亭亭北有養女嶺即浩舋西平之北山}
度星嶺丙戍至浩舋川|{
	水經注浩舋河出塞外逕西平之鮮谷塞又東逕養女北山東南隋志西平郡湟水縣有舊浩舋縣浩舋音告門浩又音閤}
以橋未成斬都水使者黄亘及督役者九人|{
	帝改都水監為都水使者 考異曰隋帝紀云梁浩舋御馬度而橋壞今從略記}
數日橋成乃行吐谷渾可汗伏允帥衆保覆袁川|{
	可從刋入聲汗音寒帥讀曰率}
帝分命内史元壽南屯金山兵部尚書段文振北屯雪山太僕卿楊義臣東屯琵琶峽將軍張壽西屯泥嶺四面圍之伏允以數十騎遁出遣其名王詐稱伏允保車我真山|{
	騎奇寄翻車昌遮翻}
壬辰詔右屯衛大將軍張定和往捕之定和輕其衆少不被甲挺身登山吐谷渾伏兵射殺之|{
	少詩沼翻被皮義翻射而亦翻}
其亞將柳武建擊吐谷渾破之|{
	將即亮翻}
甲午吐谷渾仙頭王窮䠞|{
	䠞與蹙翻}
帥男女十餘萬口來降|{
	帥讀曰率降戶江翻}
六月丁酉遣左光祿大夫梁默等追討伏允兵敗為伏允所殺衛尉卿劉權出伊吾道擊吐谷渾至青海|{
	隋志西海郡有青海吐谷渾傳青海在伏俟城東周囘千餘里}
虜獲千餘口乘勝追奔至伏俟城|{
	吐谷渾都伏俟城在青海西十五里}
辛丑帝謂給事郎蔡徵曰自古天子有廵狩之禮而江東諸帝多傅脂粉坐深宫不與百姓相見此何理也對曰此其所以不能長世丙午至張掖|{
	帝改甘州為張掖郡}
帝之將西廵也命裴矩說高昌王麴伯雅|{
	麴姓也漢末有西平麴演說輸芮翻}
及伊吾吐屯設等|{
	吐屯設意突厥所置以守伊吾}
啗以厚利召使入朝壬子帝至燕支山|{
	隋志武威郡番禾縣有燕支山啗徒濫翻又徒覽翻朝直遙翻燕因肩翻}
伯雅吐屯設等及西域二十七國謁於道左皆令佩金玉被錦罽|{
	令力丁翻被皮義翻罽音計}
焚香奏樂歌舞諠譟帝復令武威張掖士女盛飾縱觀衣服車馬不鮮者郡縣督課之騎乘填咽|{
	騎奇寄翻乘繩證翻}
周亘數十里以示中國之盛吐屯設獻西域數千里之地上大悦癸丑置西海河源鄯善且末等郡|{
	西海郡置於伏俟城河源郡置於赤水城鄯善郡置於古樓蘭城且末郡置於古且末城酈道元曰且末城東去鄯善郡七百二十里鄯時戰翻且子閭翻}
讁天下罪人為戍卒以守之命劉權鎮河源郡積石鎮|{
	志云河源郡有積石山河所出也杜佑曰積石山在西平郡龍支縣南}
大開屯田扞禦吐谷渾以通西域之路是時天下凡有郡一百九十縣一千二百五十五戶八百九十萬有奇|{
	奇居宜翻}
東西九千三百里南北萬四千八百一十五里隋氏之盛極於此矣帝謂裴矩有綏懷之畧進位銀青光祿大夫自西京諸縣及西北諸郡皆轉輸塞外每歲鉅億萬計經途險遠及遇寇鈔|{
	鈔楚交翻}
人畜死亡不達者郡縣皆徵破其家由是百姓失業西方先困矣初吐谷渾伏允使其子順來朝|{
	吐從暾入聲谷音浴朝直遙翻}
帝留順不遣伏允敗走無以自資帥數千騎客於党項|{
	隋書党項羌者三苗之後也其種有宕昌白狼皆自稱獮猴種東接臨洮西平西拒葉護南北數千里處山谷間每姓别為部落帥讀曰率党他朗翻}
帝立順為可汗|{
	可從刋入聲汗音寒}
送至玉門令統其餘衆以其大寶王尼洛周為輔|{
	統他綜翻尼女夷翻}
至西平其部下殺洛周順不果入而還|{
	還從宣翻又音如字}
丙辰上御觀風殿|{
	即觀風行殿也}
大備文物引高昌王麴伯雅及伊吾吐屯設升殿宴飲 |{
	考異曰略記在六月壬寅今從隋帝紀}
其餘蠻夷使者陪階庭者二十餘國奏九部樂|{
	杜佑曰煬帝立清樂龜兹西凉天竺康國疏勒安國高麗禮畢為九部使疏吏翻}
及魚龍戲以娛之賜賚有差戊午赦天下吐谷渾有青海俗傳置牝馬於其上得龍種|{
	吐谷渾傳青海中有小山其俗至冬輒放牝馬於其上言得龍種吐谷渾嘗得波斯草馬放入青海因生驄駒能日行千里時稱青海驄種章勇翻}
秋七月置馬牧於青海縱牝馬二千匹於川谷以求龍種無効而止車駕東還經大斗拔谷|{
	還從宣翻又如字新唐志凉州西二百里有大斗軍本赤水守捉開元十六年為軍因大斗拔谷為名}
山路隘險魚貫而出|{
	單行相次若貫魚然}
風雪晦冥文武饑餒沾濕夜久不逮前營|{
	逮及也}
士卒凍死者大半 |{
	考異曰帝紀在六月癸卯按西邊地雖寒不容六月大雪凍死人畜今從畧記略記作達十拔谷今從帝紀}
馬驢什八九後宫妃主或狼狽相失與軍士雜宿山間九月乙未車駕入西京冬十一月丙子復幸東都|{
	復扶又翻}
民部侍郎裴藴以民間版籍脱漏戶口及詐注老小尚多奏令貌閲|{
	令力丁翻閲其貌以驗老小}
若一人不實則官司解職又許民糾得一丁者令被糾之家代輸賦役|{
	被皮義翻}
是歲諸郡計帳進丁二十萬三千新附口六十四萬一千五百帝臨朝覽狀|{
	朝直遙翻}
謂百官曰前代無賢才致此罔冒今戶口皆實全由裴藴由是漸見親委未幾擢授御史大夫與裴矩虞世基參掌機密藴善候伺人主微意|{
	幾居豈翻伺相吏翻}
所欲罪者則曲法鍜成其罪所欲宥者則附從輕典因而釋之是後大小之獄皆以付藴刑部大理莫敢與爭必稟承進止然後决斷|{
	斷丁亂翻}
藴有機辯言若懸河或重或輕皆由其口剖析明敏時人不能致詰|{
	書曰知人則哲能官人史言知人善任之難}
突厥啓民可汗卒|{
	厥九勿翻可從刋入聲汗音寒卒子恤翻}
上為之廢朝三日|{
	為于偽翻朝直遙翻下同}
立其子咄吉|{
	咄當没翻}
是為始畢可汗表請尚公主詔從其俗 初内史侍郎薛道衡以才學有盛名久當樞要高祖末出為襄州總管|{
	帝改襄州為襄陽郡}
帝即位自番州刺史召之|{
	隋志廣州仁壽元年改番州蓋因番禺以名州帝改為南海郡番依漢書音義音潘}
欲用為秘書監道衡既至上高祖文皇帝頌|{
	上時掌翻}
帝覽之不悦顧謂蘇威曰道衡致美先朝|{
	致極也}
此魚藻之義也|{
	詩小序曰魚藻刺幽王也言萬物失其性王居鎬京將不能以自樂故君子思古之武王焉}
拜司隸大夫將置之罪司隸刺史房彦謙勸道衡杜絶賓客卑辭下氣|{
	帝置司隸大夫一人為司隸臺率又置司隸刺史十四人正六品廵察畿外諸郡}
道衡不能用會議新令久不决道衡謂朝士曰向使高熲不死令决當久行有人奏之帝怒曰汝憶高熲邪|{
	朝直遙翻熲居永翻邪音耶}
付執法者推之|{
	推尋繹也推考而尋繹其事也}
裴藴奏道衡負才恃舊有無君之心推惡於國|{
	推吐雷翻}
妄造禍端論其罪名似如隱昧原其情意深為悖逆|{
	悖蒲内翻又蒲没翻}
帝曰然我少時與之行役|{
	謂伐陳時少詩照翻}
輕我童稚|{
	稚遲二翻}
與高熲賀若弼等外擅威權|{
	若人者翻}
及我即位懷不自安賴天下無事未得反耳公論其逆妙體本心道衡自以所坐非大過促憲司早斷|{
	斷丁亂翻}
冀奏日帝必赦之敕家人具饌以備賓客來候者|{
	饌雛戀翻又雛皖翻}
及奏帝令自盡道衡殊不意未能引决憲司重奏縊而殺之妻子徙且末|{
	令力丁翻重直龍翻縊於賜翻且子閭翻}
天下寃之 帝大閲軍實稱器甲之美宇文述因進言此皆雲定興之功帝即擢定興為太府丞

六年春正月癸亥朔未明三刻有盗數十人素冠練衣焚香持華自稱彌勒佛入自建國門|{
	釋氏之說以為釋迦佛衰謝彌勒佛出世故盗稱之以為姧建國門蓋東都皇城端門也唐六典云武德五年平王世充惡其壯麗焚乾陽殿及建國門華讀曰花 考異曰雜記在五年正月又云三百入今從隋書}
監門者皆稽首|{
	監古銜翻稽音啓}
既而奪衛士仗將為亂齊王暕遇而斬之於是都下大索|{
	暕古限翻索山客翻}
連坐者千餘家 帝以諸蕃酋長畢集洛陽|{
	酋才由翻長知兩翻}
丁丑於端門街|{
	洛陽皇城端門外之街}
盛陳百戲戲場周圍五千步執絲竹者萬八千人聲聞數十里|{
	聞音問}
自昏至旦燈火光燭天地終月而罷所費巨萬自是歲以為常|{
	丁丑正月十五日今人元宵行樂蓋始盛於此}
諸蕃請入豐都市交易|{
	東都東市曰豐都南市曰大同北市曰通遠}
帝許之先命整飾店肆簷宇如一盛設帷帳珍貨充積人物華盛賣菜者亦藉以龍須席|{
	龍須席以龍須草織成今淮上安慶府居人多能織龍須席}
胡客或過酒食店|{
	過工禾翻}
悉令邀延就坐|{
	坐徂卧翻}
醉飽而散不取其直紿之曰中國豐饒酒食例不取直胡客皆驚歎其黠者頗覺之|{
	紿徒亥翻黠戶八翻慧也}
見以繒帛纒樹曰中國亦有貧者衣不蓋形何如以此物與之纒樹何為市人慚不能答帝稱裴矩之能謂羣臣曰裴矩大識朕意凡所陳奏皆朕之成算未發之頃矩輒以聞自非奉國盡心孰能若是是時矩與右翊衛大將軍宇文述内史侍郎虞世基御史大夫裴藴光祿大夫郭衍皆以諂諛有寵述善於供奉容止便辟|{
	便毗連翻辟讀曰僻}
侍衛者咸取則焉郭衍嘗勸帝五日一視朝|{
	朝直遙翻翻下同}
曰無効高祖空自勤苦帝益以為忠曰唯有郭衍心與朕同帝臨朝凝重|{
	朝直遙翻}
發言降詔辭義可觀而内存聲色其在兩都及廵遊常以僧尼道士女官自隨|{
	女官即女道士}
謂之四道場梁公蕭鉅琮之弟子千牛左右宇文皛慶之孫也|{
	隋制千牛備身左右十二人掌供御弓箭宇文慶見一百七十三卷陳高宗太建十一年皛戶了翻}
皆有寵於帝帝每日於苑中林亭間盛陳酒饌|{
	饌雛晥翻又雛戀翻}
敕燕王倓與鉅皛及高祖嬪御為一席|{
	倓徒甘翻嬪毗賓翻}
僧尼道士女官為一席帝與諸寵姬為一席略相連接罷朝即從之|{
	朝直遙翻}
宴飲更相勸侑|{
	更工衡翻}
酒酣殽亂靡所不至以是為常楊氏婦女之美者往往進御皛出入宫掖不限門禁至於妃嬪公主皆有醜聲帝亦不之罪也 帝復遣朱寛招撫流求|{
	復扶又翻}
流求不從帝遣虎賁郎將廬江陳稜朝請大夫同安張鎮周發東陽兵萬餘人自義安汎海擊之|{
	賁音奔將即亮翻帝改廬州為廬江郡熙州為同安郡婺州為東陽郡潮州為義安郡}
行月餘至其國以鎮周為先鋒流求王渴刺兜|{
	刺盧達翻}
遣兵逆戰屢破之遂至其都|{
	流求國王所居曰婆羅檀洞塹柵三重環以流水樹棘為藩}
渴刺兜自將出戰|{
	將即亮翻}
又敗退入栅稜等乘勝攻拔之斬渇刺兜虜其民萬餘口而還|{
	還從宣翻又如字}
二月乙巳稜等獻流求俘頒賜百官進稜位右光祿大夫鎮周金紫光祿大夫 乙卯詔以近世茅土妄假名實相乖自今唯有功勲乃得賜封仍令子孫承襲於是舊賜五等爵非有功者皆除之 庚申以所徵周齊梁陳散樂|{
	散悉但翻}
悉配太常皆置慱士弟子以相傳授樂工至三萬餘人 三月癸亥帝幸江都宫 初帝欲大營汾陽宫令御史大夫張衡具圖奏之|{
	考異曰張衡傳云帝幸衡宅之明年幸汾陽宫又云明年復幸汾陽宫按今紀皆無其事恐傳誤}
衡乘

間進諫曰比年勞役繁多|{
	間古莧翻比毗至翻}
百姓疲弊伏願留神稍加抑損帝意甚不平後目衡謂侍臣曰張衡自謂由其計畫令我有天下也|{
	令力丁翻}
乃錄齊王暕携皇甫翊從駕及前幸涿郡祠恒岳時父老謁見者衣冠多不整|{
	暕古限翻從才用翻恒戶登翻見賢遍翻}
譴衡以憲司不能舉正|{
	張衡為御史大夫故譴之以憲司職分}
出為榆林太守久之衡督役築樓煩城|{
	大業四年置樓煩郡後魏之嵐州也本漢之汾陽縣地時置汾陽宫故築城守式入翻}
因帝廵幸得謁帝帝惡衡不損瘦|{
	惡烏路翻}
以為不念咎謂衡曰公甚肥澤宜且還郡復遣之榆林|{
	還從宣翻又音如字復扶又翻又音如字}
未幾敕衡督役江都宫禮部尚書楊玄感使至江都|{
	幾居豈翻使疏吏翻}
衡謂玄感曰薛道衡真為枉死玄感奏之江都郡丞王世充又奏衡頻减頓具帝於是發怒鎻詣江都市將斬之久乃得釋除名為民放還田里以王世充領江都宫監世充本西域胡人姓支氏父没幼從其母嫁王氏因冒其姓世充性譎詐有口辯頗涉書傳好兵法習律令|{
	譎古穴翻傳直戀翻好呼到翻下同}
帝數幸江都|{
	數所角翻}
世充能伺候顔色為阿諛雕飾池臺奏獻珍物由是有寵|{
	為王世充乘時僭竊張本伺相吏翻}
夏六月甲寅制江都太守秩同京尹|{
	隋制京尹正三品}
冬十二月己未文安憲侯牛弘卒|{
	按牛弘傳弘爵奇章郡公卒贈文安縣侯諡曰憲此書其贈諡也隋志文安縣屬河間郡卒子恤翻}
弘寛厚恭儉學術精博隋室舊臣始終信任悔吝不及者唯弘一人而已弟弼好酒而䣱|{
	䣱香句翻醉怒也}
嘗因醉射殺弘駕車牛|{
	射而亦翻}
弘來還宅|{
	還從宣翻又音如字}
其妻迎謂之曰叔射殺牛弘無所怪問直答云作脯坐定其妻又曰叔忽射殺牛大是異事弘曰已知之矣顔色自若讀書不輟 敕穿江南河自京口至餘杭八百餘里|{
	今浙西運河自杭州達鎮江府入大江是也鎮江古京口也帝改杭州為餘杭郡}
廣十餘丈|{
	廣古曠翻}
使可通龍舟并置驛宫草頓欲東廵會稽|{
	帝改越州復曰會稽郡會古外翻}
上以百官從駕皆服袴褶|{
	從才用翻褶音習}
於軍旅間不便是歲始詔從駕涉遠者文武官皆戎衣五品以上通着紫袍六品以下兼用緋綠|{
	自此文武官常服遂以為品色著則略翻緋音非}
胥史以青庶人以白屠商以皁|{
	皁才早翻}
士卒以黄 帝之幸啓民帳也|{
	見上卷三年}
高麗使者在啓民所|{
	麗力知翻使疏吏翻}
啓民不敢隱與之見帝|{
	見賢遍翻}
黄門侍郎裴矩說帝曰高麗本箕子所封之地漢晉皆為郡縣|{
	周武王封箕子於朝鮮秦末衛滿據之傳國至孫右渠漢武帝滅之開為四郡漢末公孫度據之傳國至孫淵魏滅之至晉皆為郡縣高麗之先出於夫餘朱蒙建國號高句驪以高為氏魏晉以來中國兵亂高麗内侵併有遼東地說輸芮翻朝漢書音義音潮}
今乃不臣别為異域先帝欲征之久矣但楊諒不肖師出無功|{
	事見一百七十八卷開皇十八年}
當陛下之時安可不取使冠帶之境遂為蠻貊之鄉乎今其使者親見啓民舉國從化可因其恐懼脇使入朝|{
	朝直遙翻下同}
帝從之敕牛弘宣旨曰朕以啓民誠心奉國故親至其帳明年當往涿郡爾還日語高麗王|{
	還從宣翻又音如字語牛倨翻麗力知翻}
勿自疑懼存育之禮當如啓民苟或不朝將帥啓民往廵彼土|{
	帥讀曰率}
高麗王元懼藩禮頗闕帝將討之課天下富人買武馬匹至十萬錢簡閲器仗務令精新或有濫惡則使者立斬|{
	令力丁翻}


七年春正月壬寅真定襄侯郭衍卒|{
	真定縣侯也隋志恒山郡治真定縣卒子恤翻}
二月己未上升釣臺臨楊子津大宴百僚乙亥帝自江都行幸涿郡御龍舟度河入永濟渠仍勑選部門下内史御史四司之官於船前選補|{
	選部之選宣戀翻選補之選如字}
其受選者三千餘人或徒步隨船三千餘里不得處分|{
	處昌呂翻分扶問翻}
凍餒疲頓因而致死者什一二 壬午下詔討高麗敕幽州摠管元弘嗣|{
	大業初已廢諸州總管府此書元弘嗣前官}
往東萊海口|{
	帝改萊州為東萊郡}
造船三百艘|{
	艘蘇遭翻}
官吏督役晝夜立水中略不敢息自腰以下皆生蛆死者什三四|{
	蛆子余翻}
夏四月庚午車駕至涿郡之臨朔宫|{
	唐志幽州薊縣有故隋臨朔宮考異曰略記曰丙午幸涿郡之新宫按長歷是月丙辰朔無丙午今從帝紀}
文武從官九品

以上並令給宅安置|{
	從才用翻}
先是詔總徵天下兵無間遠近俱會於涿|{
	涿即涿郡先悉薦翻}
又發江淮以南水手一萬人弩手三萬人嶺南排鑹手三萬人|{
	鑹七亂翻小矟也}
於是四遠奔赴如流五月勑河南淮南江南造戎車五萬乘送高陽|{
	隋志高陽縣屬河間郡乘繩證翻}
供載衣甲幔幕|{
	幔莫半翻}
令兵士自挽之發河南北民夫以供軍須秋七月發江淮以南民夫及船運黎陽及洛口諸倉米至涿郡|{
	令力丁翻黎陽縣屬汲郡有黎陽倉洛口倉初置見上卷二年}
舳艫相次千餘里|{
	舳艫音逐盧}
載兵甲及攻取之具往還在道常數十萬人填咽於道晝夜不絶死者相枕|{
	枕職任翻}
臭穢盈路天下騷動 山東河南大水漂没三十餘郡冬十月乙卯底柱崩偃河逆流數十里|{
	砥柱在河南郡陜縣北河中底與砥同}
初帝西廵|{
	見五年}
遣侍御史韋節|{
	隋制御史臺侍御史八人}
召西突厥處羅可汗|{
	厥九勿翻處昌呂翻可從刋入聲汗音寒}
令與車駕會大斗拔谷國人不從處羅謝使者辭以佗故帝大怒無如之何會其酋長射匱遣使來求婚|{
	令力丁翻使疏吏翻酋才由翻長知兩翻}
裴矩因奏曰處羅不朝恃彊大耳臣請以計弱之分裂其國即易制也|{
	朝直遙翻下同易以豉翻}
射匱者都六之子達頭之孫|{
	杜佑曰都六者突厥始建號者也今矩言都六為達頭之子則非始建號者也}
世為可汗君臨西面今聞其失職附屬處羅故遣使來以結援耳願厚禮其使拜為大可汗則突厥勢分兩從我矣|{
	言射匱處羅將兩皆從隋也}
帝曰公言是也因遣矩朝夕至舘微諷諭之帝於仁風殿召其使者言處羅不順之狀稱射匱向善吾將立為大可汗令發兵誅處羅然後為婚取桃竹白羽箭一枝|{
	桃竹桃枝竹也今江南有之}
以賜射匱因謂之曰此事宜速使疾如箭也使者返路經處羅處羅愛箭將留之使者譎而得免|{
	譎古穴翻}
射匱聞而大喜興兵襲處羅處羅大敗棄妻子將數千騎東走緣道被刼寓於高昌東保時羅漫山|{
	新唐志伊州伊吾縣有折羅漫山亦曰天山將即亮翻又音如字領也騎奇寄翻被皮義翻}
高昌王麴伯雅上狀|{
	上言其狀}
帝遣裴矩與向氏親要左右馳至玉門關晉昌城|{
	新唐志玉門關在沙州夀昌縣西北}
曉諭處羅使入朝十二月己未處羅來朝於臨朔宫|{
	朝直遙翻}
帝大悦接以殊禮帝與處羅宴處羅稽首謝入見之晩帝以温言慰勞之|{
	稽音啓見賢遍翻勞力到翻}
備設天下珍膳盛陳女樂羅綺絲竹眩曜耳目然處羅終有怏怏之色|{
	怏於兩翻}
帝自去歲謀討高麗詔山東置府令養馬以供軍役又發民夫運米積於瀘河懷遠二鎮|{
	新唐志曰隋於營州之境汝羅故城置遼西郡領遼西瀘河懷遠三縣瀘音盧}
車牛往者皆不返士卒死亡過半耕稼失時田疇多荒加之饑饉穀價踊貴東北邊尤甚斗米直數百錢所運米或粗惡|{
	粗倉乎翻}
令民糴而償之又發鹿車夫六十餘萬|{
	鹿車小車也言其小止容一鹿}
二人共推米三石|{
	推吐雷翻}
道途險遠不足充餱糧|{
	餱戶鉤翻乾食也}
至鎮無可輸皆懼罪亡命重以官吏貪殘|{
	重直用翻}
因緣侵漁百姓困窮財力俱竭安居則不勝凍餒|{
	勝音升}
死期交急剽掠則猶得延生|{
	剽匹妙翻}
於是始相聚為羣盗鄒平民王薄擁衆據長白山|{
	鄒平縣宋所僑置平原郡縣之地隋開皇十八年改名鄒平時屬齊郡唐併入齊州臨濟縣長白山在章丘縣界亦屬齊郡宋白曰淄州長山縣宋於此僑立廣川郡及武彊縣隋改武彊為長山以縣西南三十五里長白山為名}
剽掠齊濟之郊|{
	帝改齊州為齊郡濟州為濟北郡剽匹妙翻濟子禮翻}
自稱知世郎言事可知矣又作無向遼東浪死歌以相感勸|{
	浪死猶言徒死也}
避征役者多往歸之平原東有豆子䴚|{
	帝改德州為平原郡䴚舉朗翻鹽澤也}
負海帶河地形深阻自高齊以來羣盗多匿其中有劉霸道者家於其旁累世仕宦貲產富厚霸道喜遊俠|{
	喜許記翻}
食客常數百人及羣盗起遠近多往依之有衆十餘萬號阿舅賊|{
	阿烏葛翻}
漳南人竇建德|{
	漳南本漢東陽縣地久廢開皇六年復置十八年改為漳南屬清河郡宋白曰取地居漳水之南為名}
少尚氣俠膽力過人為鄉黨所歸附會募人征高麗建德以勇敢選為二百人長|{
	少詩照翻長知兩翻}
同縣孫安祖亦以驍勇選為征士|{
	驍堅堯翻}
安祖辭以家為水所漂妻子餒死縣令怒笞之安祖刺殺令|{
	刺七亦翻}
亡抵建德 |{
	考異曰杜儒童隋季革命記云安祖以盜羊為縣令所考今從舊唐書建德傳}
建德匿之官司逐捕蹤跡至建德家建德謂安祖曰文皇帝時天下殷盛發百萬之衆以伐高麗尚為所敗|{
	即謂開皇十八年事敗補邁翻}
今水潦為災百姓困窮加之往歲西征|{
	謂西征吐谷渾}
行者不歸瘡痍未復主上不恤乃更發兵親擊高麗天下必大亂丈夫不死當立大功豈可但為亡虜邪|{
	邪音耶}
乃集無賴少年得數百人使安祖將之|{
	少詩照翻將即亮翻}
入高雞泊中為羣盗|{
	新唐書曰高雞泊廣袤數百里葭薍阻奥可以違難}
安祖自號將軍時鄃人張金稱聚衆河曲|{
	鄃漢縣舊廢開皇十六年復置屬清河郡河曲清河之曲新唐書作河渚鄃音輸}
蓚人高士達聚衆於清河境内為盗|{
	蓚舊曰修開皇五年改屬信都郡蓚音條}
郡縣疑建德與賊通悉收其家屬殺之|{
	新書曰時羣盗往來漳南剽殺人焚鄉聚獨不入建德閭由是郡縣疑其與賊通}
建德帥麾下二百人亡歸士達|{
	帥讀曰率}
士達自稱東海公以建德為司兵頃之孫安祖為張金稱所殺其衆盡歸建德兵至萬餘人建德能傾身接物與士卒均勞逸由是人爭附之為之致死|{
	為于偽翻竇建德始此}
自是所在羣盗蜂起不可勝數|{
	勝音升數所具翻}
徒衆多者至萬餘人攻陷城邑甲子勑都尉鷹揚與郡縣相知追捕|{
	隋志奉車駙馬都尉屬三衛帝並廢之此蓋置都尉以討羣盗帝又改驃騎為鷹揚郎將}
隨獲斬决|{
	隨所獲而斬决之}
然莫能禁止

八年春正月 |{
	考異曰略記云癸丑帝御前殿按長歷是月辛巳朔無癸丑略記甲子多差誤今不取皆從隋書}
帝分西突厥處羅可汗之衆為三|{
	厥九勿翻處昌呂翻可從刋入聲汗音寒}
使其弟闕度設將羸弱萬餘口居于會寧|{
	突厥之官典兵者謂之設靈州鳴沙縣後周置會州會寧郡尋廢唐復置將即亮翻下同羸倫為翻 考異曰隋西突厥傳作達度闕設今從裴矩傳}
又使特勒大奈别將餘衆居于樓煩|{
	突厥之官子弟為特勒}
命處羅將五百騎常從車駕廵幸賜號曷婆那可汗|{
	騎奇寄翻 考異曰唐李軌傳作曷婆那可汗今從隋書按今隋書作曷薩那}
賞賜甚厚 初嵩高道士潘誕|{
	隋忘河南郡嵩陽縣有蒿高山}
自言三百歲為帝合煉金丹帝為之作嵩陽觀|{
	為于偽翻合音閤觀古玩翻}
華屋數百間以童男童女各一百二十人充給使位視三品常役數千人所費巨萬云金丹應用石膽石髓|{
	髓息委翻}
發石工鑿嵩高大石深百尺者數十處|{
	深式浸翻}
凡六年丹不成帝詰之|{
	詰去吉翻}
誕對以無石膽石髓若得童男女膽髓各三斛六斗可以代之帝怒鎻詣涿郡斬之且死語人曰|{
	語牛倨翻}
此乃天子無福值我兵解時至|{
	解佳買翻一作假音學仙者謂蜕骨登仙為尸解故其徒謂死為解化今誕謂兵死為兵解}
我應生梵摩天云|{
	梵扶泛翻}
四方兵皆集涿郡帝徵合水令庾質|{
	質出合水見上四年}
問曰高麗之衆不能當我一郡今朕以此衆伐之卿以為克不|{
	麗力知翻不讀曰否}
對曰伐之可克然臣竊有愚見不願陛下親行帝作色曰朕今總兵至此豈可未見賊而先自退邪|{
	邪音耶}
對曰戰而未克懼損威靈若車駕留此命猛將勁卒|{
	將即亮翻}
指授方略倍道兼行出其不意克之必矣事機在速緩則無功帝不悦曰汝既憚行自可留此右尚方署監事耿詢上書切諫|{
	監事監作者也秩九品監古銜翻上時掌翻}
帝大怒命左右斬之何稠苦救得免壬午詔左十二軍出鏤方長岑溟海蓋馬建安南蘇遼東玄菟扶餘朝鮮沃沮樂浪等道|{
	帝指授諸軍所出之道多用漢縣舊名漢志鏤方長岑朝鮮屬樂浪郡蓋馬屬玄菟郡有蓋馬大山遼東漢郡名溟海蓋即漢樂浪郡之海冥縣建安南蘇扶餘皆高麗國城守之處沃沮亦古地名是時其地已入新羅界鏤郎豆翻菟音塗朝音潮鮮音仙沮子餘翻樂音洛浪音郎}
右十二軍出黏蟬含資渾彌臨屯候城提奚蹋頓肅慎碣石東暆帶方襄平等道|{
	漢志黏蟬含資渾彌提奚東暆帶方等縣屬樂浪郡候城襄平屬遼東郡臨屯亦漢武帝所置郡名蹋頓即漢末遼西烏丸蹋頓所居肅慎古肅慎氏之國其地時為靺鞨所居碣石禹貢之碣石也杜佑以為此碣石在高麗中佑曰碣石山在漢樂浪郡遂城縣秦長城起於此山今驗長城東截遼水而入高麗遺址猶存黏女亷翻蟬服䖍音提蹋徒盍翻碣其列翻暆應劭曰音稜}
駱驛引途|{
	駱驛相繼不絶也}
摠集平壤|{
	平壤城高麗國都也亦曰長安城東西六里隨山屈曲南臨水杜佑曰高麗王自東晉以後居平壤城即漢樂浪郡王險城}
凡一百一十三萬三千八百人號二百萬其餽運者倍之宜社於南桑乾水上類上帝於臨朔宫南|{
	記王制天子將出類乎上帝宜乎社鄭氏注類宜皆祭名孔穎達曰天道遠以事類而祭告之也社主殺戮故求便宜社主隂萬物於此斷殺故曰宜桑乾河逕薊城南水經濕水出雁門隂舘縣東北過代郡桑乾縣謂之桑乾水東過廣陽薊縣北今在薊城南城邑有變遷也乾音干}
祭馬祖於薊城北|{
	周禮祭馬祖鄭氏注曰馬祖天駟也}
帝親授節度每軍大將亞將各一人騎兵四十隊隊百人十隊為團步卒八十隊分為四團團各有偏將一人其鎧胄纓拂旗旛每團異色|{
	將即亮翻騎奇寄翻鎧可亥翻}
受降使者一人承詔慰撫不受大將節制其輜重散兵等亦為四團|{
	降戶江翻使疏吏翻重直用翻散悉但翻}
使步卒挾之而行進止立營皆有次叙儀法癸未第一軍發日遣一軍相去四十里連營漸進終四十日發乃盡首尾相繼鼓角相聞旌旗亘九百六十里御營内合十二衛三臺五省九寺分隸内外前後左右六軍次後發又亘八十里近古出師之盛未之有也 甲辰内史令元壽薨 二月壬戍觀德王雄薨|{
	觀王雄謚曰德觀古玩翻}
北平襄侯段文振為兵部尚書上表以為帝寵待突厥太厚處之塞内|{
	上時掌翻處昌呂翻}
資以兵食戎狄之性無親而貪異日必為國患宜以時諭遣令出塞外|{
	令力丁翻下同}
然後明設烽候緣邊鎮防務令嚴重此萬歲之長策也兵曹郎斛斯政椿之孫也|{
	帝改尚書諸曹侍郎為郎兵曹郎開皇之兵部侍郎也斛斯椿構間後魏孝武高歡者也}
以器幹明悟為帝所寵任使專掌兵事文振知政險薄不可委以機要屢言於帝帝不從及征高麗以文振為左候衛大將軍出南蘇道文振於道中疾篤上表曰竊見遼東小醜未服嚴刑遠降六師親勞萬乘但夷狄多詐深須防擬口陳降欵母宜遽受|{
	乘繩證翻降戶江翻}
水潦方降不可淹遲唯願嚴勒諸軍星馳速發水陸俱前出其不意則平壤孤城勢可拔也若傾其本根餘城自克如不時定脱遇秋霖深為艱阻兵糧既竭彊敵在前靺鞨出後|{
	靺音末鞨音曷}
遲疑不决非上策也三月辛卯文振卒|{
	卒子恤翻}
帝甚惜之 癸巳上始御師進至遼水衆軍總會臨水為大陳|{
	陣讀曰陣}
高麗兵阻水拒守隋兵不得濟左屯衛大將軍麥鐵杖謂人曰丈夫性命自有所在豈能然艾灸頞瓜蔕歕鼻治黄不差而臥死兒女手中乎|{
	黄熱病也熱則頭痛故燃艾以灸頞熱則上壅瓜蔕味苦寒故噴鼻以通關鬲差愈也然與燃同灸居又翻頞烏葛翻鼻頞說文曰頞鼻莖蔕音帝歕蒲悶翻差楚懈翻治直之翻}
乃自請為前鋒謂其三子曰吾荷國恩|{
	荷下可翻}
今為死日我得良殺汝當富貴帝命工部尚書宇文愷造浮橋三道於遼水西岸既成引橋趨東岸|{
	趨七喻翻}
橋短不及岸丈餘高麗兵大至隋兵驍勇者爭赴水接戰|{
	驍堅堯翻}
高麗兵乘高擊之隋兵不得登岸死者甚衆麥鐵杖躍登岸與虎賁郎將錢士雄孟义等皆戰死 |{
	考異曰雜記作錢英孟金釵今從隋帝紀}
乃歛兵引橋復就西岸詔贈鐵杖宿公|{
	宿古國翻}
使其子孟才襲爵次子仲才季才並拜官正議大夫更命少府監何稠接橋|{
	更工衡翻少始照翻}
二日而成諸軍相次繼進大戰于東岸高麗兵大敗死者萬計諸軍乘勝進圍遼東城即漢之襄平城也車駕度遼|{
	考異曰隋帝紀癸巳上御師甲子臨遼水橋戊戍麥鐵杖死甲午車駕度遼乙未大頓丙申大赦按長歷是月庚辰朔不容有甲子又戊戍之下不容有甲午乙未丙申此必誤也今並除之}
引曷薩那可汗及高昌王伯雅觀戰處以懾憚之|{
	懾之涉翻}
因下詔赦天下命刑部尚書衛文昇尚書右丞劉士龍撫遼左之民給復十年|{
	復方日翻}
建置郡縣以相統攝 夏五月壬午納言楊達薨 諸將之東下也帝親戒之曰今者弔民伐罪非為功名諸將或不識朕意欲輕兵掩襲孤軍獨鬪立一身之名以邀勲賞非大軍行法|{
	言非大軍征行之法將即亮翻}
公等進軍當分為三道有所攻擊必三道相知毋得輕軍獨進以致失亡又凡軍事進止皆須奏聞待報毋得專擅遼東數出戰不利|{
	數所角翻}
乃嬰城固守帝命諸軍攻之又勑諸將高麗若降即宜撫納不得縱兵遼東城將陷城中人輒言請降|{
	降戶江翻}
諸將奉旨不敢赴機先令馳奏比報至|{
	比必寐翻}
城中守禦亦備隨出拒戰如此再三帝終不寤既而城久不下六月己未帝幸遼東城南觀其城池形勢因召諸將詰責之曰公等自以官高又恃家世欲以暗懦待我邪|{
	詰去吉翻懦乃卧翻又乃亂翻邪音耶下同}
在都之日公等皆不願我來恐見病敗耳我今來此正欲觀公等所為斬公輩耳公今畏死莫肯盡力謂我不能殺公邪諸將咸戰懼失色帝因留城西數里御六合城|{
	此六合城畧如三年行城之制周囘八里城及女垣高十仭}
高麗諸城各堅守不下右翊衛大將軍來護兒帥江淮水軍舳艫數百里浮海先進入自水|{
	班志水西至增地縣入海皆在樂浪界師讀曰率舳艫音逐盧普大翻}
去平壤六十里與高麗相遇進擊大破之護兒欲乘勝趣其城|{
	趣七喻翻}
副總管周法尚止之請俟諸軍至俱進護兒不聼簡精甲四萬直造城下|{
	造七到翻}
高麗伏兵於羅郭内空寺中出兵與護兒戰而偽敗護兒逐之入城縱兵俘掠無復部伍伏兵發護兒大敗僅而獲免士卒還者不過數千人高麗追至船所周法尚整陳待之高麗乃退護兒引兵還屯海浦不敢復留應接諸軍|{
	陳讀曰陣復扶又翻考異曰北史云護破高麗斬高元弟建武因破其郛營於城外以待諸軍今從隋書及革命記}
左翊衛大將軍宇文述出扶餘道右翊衛大將軍于仲文出樂浪道|{
	隋制十三衛各置大將軍一人來護兒于仲文並書右翊衛大將軍何也攷二人本傳于仲文帝即位之初為右翊衛大將軍征吐渾時來護兒已為右翊衛大將軍通鑑蓋追書仲文官也}
左驍衛大將軍荆元恒出遼東道|{
	驍堅堯翻恒戶登翻}
右翊衛將軍薛世雄出沃沮道左屯衛將軍辛世雄出玄莬道|{
	沮子余翻菟音塗}
右禦衛將軍張瑾出襄平道右武候將軍趙孝才出碣石道涿郡太守檢校左武衛將軍崔弘昇出遂城道檢校右禦衛虎賁郎將衛文昇出增地道|{
	守式又翻將即亮翻}
皆會於鴨綠水西|{
	班志玄菟郡西蓋馬縣有馬訾水新唐書馬訾水出靺鞨之白山色若鴨頭號鴨綠水平壤城在鴨綠東南金人謂鴨綠水為混同江杜佑曰鴨淥水闊三百步在平壤西北四百五十里遼水東南四百八十里}
述等兵自瀘河懷遠二鎮人馬皆給百日糧又給排甲槍矟|{
	矟色角翻}
并衣資戎具火幕人别三石已上重莫能勝致|{
	勝音升}
下令軍中士卒有遺弃米粟者斬軍士皆於幕下掘坑埋之纔行及中路糧已將盡高麗遣大臣乙支文德詣其營詐降|{
	乙支東夷複姓麗力知翻降戶江翻下同考異曰革命記作尉支文德今從隋書及北史}
實欲觀虛實于仲文先奉密旨若遇高元及文德來者必擒之仲文將執之尚書右丞劉士龍為慰撫使|{
	使疏吏翻下同}
固止之仲文遂聽文德還|{
	還從宣翻}
既而悔之遣人紿文德曰更欲有言可復來|{
	紿待亥翻復扶又翻}
文德不顧濟鴨綠水而去仲文與述等既失文德内不自安述以糧盡欲還仲文議以精鋭追文德可以有功述固止仲文怒曰將軍仗十萬之衆不能破小賊何顔以見帝且仲文此行固知無功何則古之良將能成功者軍中之事决在一人|{
	將即亮翻下同}
今人各有心何以勝敵時帝以仲文有計畫令諸軍諮稟節度故有此言由是述等不得已而從之與諸將度水追文德文德見述軍士有饑色故欲疲之每戰趣走述一日之中七戰皆捷既恃驟勝又逼羣議於是遂進東濟薩水|{
	薩桑葛翻}
去平壤城三十里因山為營文德復遣使詐降|{
	復扶又翻}
請於述曰若旋師者當奉高元朝行在所|{
	朝直遙翻}
述見士卒疲弊不可復戰|{
	復扶又翻}
又平壤城險固度難猝拔|{
	度徒洛翻}
遂因其詐而還|{
	使來護兒之師不敗而先退則營於平壤城外與宇文述諸軍猶聲援相接不致有薩水之狼狽也還從宣翻又如字下同 考異曰革命記云許公即至平壤城頭即樹降幡約至五日檢錄簿籍圖書開門待命期過五日無一言許公頻催竟無報答又十數日乃云船糧敗却迴公今更欲何待然始抗旌拒守分兵以據險要許公知被欺即卷甲歸每日常設方陳而行四面俱時受敵傷殺既衆糧食又盡過遼水者什無二三按煬帝驕暴高麗若明言不降述等必不敢還今從隋書}
述等為方陳而行高麗四面鈔擊|{
	陳讀曰陣鈔楚交翻麗力知翻}
述等且戰且行秋七月壬寅至薩水軍半濟高麗自後擊其後軍右屯衛將軍辛世雄戰死於是諸軍俱潰不可禁止將士奔還一日一夜至鴨綠水行四百五十里將軍天水王仁恭為殿|{
	殿丁練翻}
擊高麗却之來護兒聞述等敗亦引還唯衛文昇一軍獨全|{
	還從宣翻又如字}
初九軍度遼凡三十萬五千及還至遼東城唯二千七百人資儲器械巨萬計|{
	巨萬萬萬也}
失亡蕩盡帝大怒鎻繫述等癸卯引還 |{
	考異曰雜記七月帝自涿郡還東都十一月宇文述等糧盡遁歸高麗出兵邀截亡失蕩盡帝怒敕所司鎻將隨行無幾斬劉士龍等於軍市特赦述今從隋書}
初百濟王璋遣使請討高麗帝使之覘高麗動靜|{
	麗力知翻使疏吏翻覘丑亷翻又丑艶翻}
璋内與高麗濳通隋軍將出璋使其臣國智牟來請師期帝大悦厚加賞賜遣尚書起部郎席律詣百濟|{
	隋志起部郎屬工部尚書姓苑席姓其先姓藉避項羽諱改姓席氏}
告以期會|{
	告以起師之期及會師之日也}
及隋軍度遼百濟亦嚴兵境上聲言助隋實持兩端是行也唯於遼水西拔高麗武厲邏|{
	高麗置邏於遼水之西以警察度遼者邏郎佐翻}
置遼東郡及通定鎮而已八月勑運黎陽洛口太原等倉穀|{
	開皇三年於衛州置黎陽倉其汾晉之粟漕運以給京師汾晉以北諸州輸之太原倉}
向望海頓|{
	望海頓當在遼西界}
使民部尚書樊子蓋留守涿郡九月庚寅車駕至東都 |{
	考異曰雜記十月車駕幸涿郡徵召兵馬將遂度遼之功蓋誤今不取}
冬十月甲寅工部尚書宇文愷卒|{
	卒子恤翻}
十一月己卯以宗女為華容公主嫁高昌 宇文述素有寵於帝且其子士及尚帝女南陽公主故帝不忍誅甲申與于仲文等皆除名為民斬劉士龍以謝天下|{
	以士龍縱乙支文德也}
薩水之敗高麗追圍薛世雄於白石山世雄奮擊破之由是獨得免官以衛文昇為金紫光祿大夫諸將皆委罪於于仲文帝既釋諸將獨繫仲文憂恚發病因篤乃出之卒于家|{
	恚於避翻卒子恤翻 考異曰略記于仲文以下斬於市今從隋書}
是歲大旱疫山東尤甚 張衡既放廢|{
	衡放還田里見上六年}
帝每令親人覘衡所為|{
	覘丑亷翻又丑艶翻}
帝還自遼東衡妾告衡怨望謗訕朝政|{
	朝直遙翻}
詔賜盡于家衡臨死大言我為人作何等事|{
	謂仁壽四年事也為于偽翻}
而望久活監刑者塞耳|{
	監古銜翻塞悉則翻}
促令殺之|{
	令力丁翻}


資治通鑑卷一百八十一
