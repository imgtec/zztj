資治通鑑卷一百九十五
宋 司馬光 撰

胡三省 音註

唐紀十一|{
	起彊圉作噩五月盡上章困敦凡三年有奇始丁酉終庚子}


太宗文武大聖大廣孝皇帝中之上

貞觀十一年|{
	觀古玩翻}
五月壬申魏徵上疏以為陛下欲善之志不及於昔時聞過必改少虧於曩日|{
	上時掌翻少詩沼翻}
譴罰積多威怒微厲乃知貴不期驕富不期侈非虚言也|{
	書周官曰位不期驕禄不期侈孔安國注曰貴不與驕期而驕自至富不與侈期而侈自至魏徵引之}
且以隋之府庫倉廪戶口甲兵之盛考之今日安得擬倫然隋以富疆動之而危我以寡弱静之而安安危之理皎然在目昔隋之未亂也自謂必無亂其未亡也自謂必無亡故賦役無窮征伐不息以至禍將及身而尚未之寤也夫鑒形莫如止水鑒敗莫如亡國|{
	夫音扶}
伏願取鑒於隋去奢從約親忠遠佞|{
	去羌呂翻遠于願翻}
以當今之無事行疇昔之恭儉則盡善盡美固無得而稱焉夫取之實難守之甚易陛下能得其所難豈不能保其所易乎|{
	去羌呂翻遠干願翻易以豉翻}
六月右僕射虞恭公温彦博薨|{
	射寅謝翻諡法尊賢敬讓曰恭執事堅固曰恭執禮御賓曰恭}
彦博久掌機務知無不為上謂侍臣曰彦博以憂國之故精神耗竭我見其不逮已二年矣恨不縱其安逸竟夭天年|{
	天於紹翻}
丁巳上幸明德宫|{
	顯慶二年改明德宫監為東都苑南面監}
己未詔荆州都督荆王元景等二十一王所任刺史咸令子孫世襲|{
	令力丁翻}
戊辰又以功臣長孫無忌等十四人為刺史|{
	長知两翻}
亦令世襲非有大故無得黜免 己巳徙許王元祥為江王 秋七月癸未大雨穀洛溢入洛陽宫|{
	按唐六典洛陽都城隋大業二年詔楊素宇丈愷移故都創造南直伊闕之口北倚邙山之塞東出瀍水之東西踰澗水之西洛水貫都有河漢之象焉東去故都十八里都城西連禁苑穀二水會于禁苑之間至玄宗開元二十四年以穀洛二水或泛溢疲費人功遂出内庫和雇修三陂以禦之一曰積翠二曰月陂三曰上陽爾後二水無勞役之患}
壞官寺民居|{
	壞音怪}
溺死者六千餘人|{
	溺奴狄翻}
魏徵上疏以為文子曰同言而信信在言前同令而行誠在令外|{
	漢書藝文志曰太子老子弟子與孔子並時上時掌翻}
自王道休明十有餘年然而德化未洽者由待下之情未盡誠信故也今立政致治必委之君子|{
	治直吏翻}
事有得失或訪之小人其待君子也敬而疎遇小人也輕而狎狎則言無不盡疎則情不上通夫中智之人豈無小慧|{
	夫音扶}
然才非經國慮不及遠雖竭力盡誠猶未免有敗况内懷姦宄其禍豈不深乎|{
	宄音軌}
夫雖君子不能無小過苟不害於正道斯可略矣既謂之君子而復疑其不信|{
	復扶又翻}
何異立直木而疑其影之曲乎陛下誠能慎選君子以禮信用之何憂不治|{
	治直吏翻}
不然危亡之期未可保也上賜手詔褒美曰昔晉武帝平吳之後志意驕怠何曾位極台司不能直諫乃私語子孫自矜明智|{
	事見八十七卷晉懷帝永嘉三年語牛倨翻}
此不忠之大者也得公之諫朕知過矣當置之几案以比弦韋|{
	用堇安于西門豹事}
乙未車駕還洛陽|{
	自明德宫還洛陽宫還從宣翻又音如字}
詔洛陽宫為水所毁者少加修繕纔令可居|{
	少詩沼翻下同令九丁翻}
自外衆材給城中壞廬舍者令百官各上封事極言朕過|{
	上時掌翻}
壬寅廢明德宫及飛山宫之玄圃院給遭水者 八月甲子上謂侍臣曰上封事者皆言朕游獵太頻今天下無事武備不可忘朕時與左右獵于後苑無一事煩民夫亦何傷|{
	夫音扶}
魏徵曰先王惟恐不聞其過陛下既使之上封事正得恣其陳述苟其言可取固有益於國若其無取亦無所損上曰公言是也皆勞而遣之|{
	勞力到翻}
侍御史馬周上疏|{
	上時掌翻}
以為三代及漢歷年多者八百少者不減四百良以恩結人心人不能忘故也自是以降多者六十年少者纔二十餘年皆無恩於人本根不固故也陛下當隆禹湯文武之業為子孫立萬代之基豈得但持當年而已今之戶口不及隋之什一而給役者兄去弟還道路相繼陛下雖加恩詔使之裁損然營繕不休民安得息故有司徒行文書曾無事實昔漢之文景恭儉養民武帝承其豐富之資故能窮奢極欲而不至於亂曏使高祖之後即傳武帝漢室安得久存乎|{
	斯確論也}
又京師及四方所造乘輿器用及諸王妃主服飾議者皆不以為儉|{
	乘繩證翻}
夫昧爽丕顯後世猶怠|{
	左傳晉叔向引讒鼎之銘以為言杜預注曰昧旦早起也丕大也言夙興以務大顯後世猶懈怠夫音扶}
陛下少居民間知民疾苦尚復如此况皇太子生長深宫不更外事|{
	少詩照翻復扶又翻長竹两翻更工衡翻}
萬歲之後固聖慮所當憂也臣觀自古以來百姓愁怨聚為盗賊其國未有不亡者人主雖欲追改不能復全|{
	復扶又翻下不復同又音如字}
故當修於可修之時不可悔之於己失之後也盖幽厲嘗笑桀紂矣煬帝亦笑周齊矣不可使後之笑今如今之笑煬帝也貞觀之初天下饑歉|{
	觀古玩翻歉苦簟翻穀梁傳曰一穀不升曰歉}
斗米直匹絹而百姓不怨者知陛下憂念不忘故也今比年豐穰|{
	比毗至翻}
匹絹得粟十餘斛而百姓怨咨者知陛下不復念之多營不急之務故也|{
	復扶又翻}
自古以來國之興亡不以蓄積多少在於百姓苦樂|{
	少詩沼翻樂音洛}
且以近事驗之隋貯洛口倉而李密因之|{
	貯丁呂翻}
東都積布帛而世充資之西京府庫亦為國家之用至今未盡夫蓄積固不可無要當人有餘力然後收之不可強歛以資寇敵也|{
	夫音扶強其两翻歛力贍翻}
夫儉以息人陛下已於貞觀之初親所履行在於今日為之固不難也陛下必欲為久長之謀不必遠求上古但如貞觀之初則天下幸甚|{
	觀古玩翻}
陛下寵遇諸王頗有過厚者|{
	時魏王泰有寵於帝故周言及之}
萬代之後不可不深思也且魏武帝愛陳思王及文帝即位囚禁諸王但無縲絏耳|{
	事見漢獻帝紀及魏文帝紀縲力追翻絏息列翻朱元晦曰縲黑索也絏攣也古者獄以黑索拘攣罪人}
然則武帝愛之適所以苦之也又百姓所以治安|{
	治直吏翻}
唯在刺史縣令苟選用得人則陛下可以端拱無為今朝廷唯重内官而輕州縣之選刺史多用武人或京官不稱職治補外任|{
	朝直遥翻稱尺證翻}
邊遠之處用人更輕所以百姓未安殆由於此疏奏上稱善久之謂侍臣曰刺史朕當自選縣令宜詔京官已上各舉一人 冬十月癸丑詔勲戚亡者皆陪葬山陵|{
	唐制凡功臣密戚請陪陵葬者聼之以文武分為左右兩列若宫人陪葬則陵戶為之成墳唐會要載昭陵陪葬者宫嬪公主主壻勲貴及祖父陪陵而子孫從葬者及四夷君長入宿衛而陪葬者名氏最多用此詔也}
上獵于洛陽苑|{
	唐六典洛陽苑在都城之西北距北邙西至孝水南帶洛水支渠穀洛二水會于其間東而十七里南面三十九里西面五十里北面二十周迴一百二十六里}
有羣豕突出林中上引弓四發殪四豕有豕突前及馬鐙|{
	殪壹計翻鐙都鄧翻鞍鐙}
民部尚書唐儉投馬摶之上拔劎斬豕顧笑曰天策長史不見上將撃賊邪|{
	武德中帝開天策上將府以唐儉為長史長知两翻將即亮翻邪音耶}
何懼之甚對曰漢高祖以馬上得之不以馬上治之|{
	漢陸賈諫高祖之言治直之翻}
陛下以神武定四方豈復逞雄心於一獸上悦為之罷獵|{
	復扶乂翻為于偽翻}
尋加光禄大夫 安州都督吳王恪數出畋獵|{
	數所角翻}
頗損居人侍御史柳範奏彈之丁丑恪坐免官削戶三百上曰長史權萬紀事吾兒不能匡正罪當死柳範曰房玄齡事陛下猶不能止畋獵豈得獨罪萬紀上大怒拂衣而入久之獨引範謂曰何面折我|{
	折之舌翻}
對曰陛下仁明臣不敢不盡愚直|{
	古語有之君仁則臣直又曰君明則臣直故柳範云然}
上悦 十一月辛卯上幸懷州丙午還洛陽宫 故荆州都督武士彠女年十四上聞其美召入後宫為才人|{
	為武氏亂唐張本彠一虢 考異曰舊則天木紀崩時年八十三唐歷焦璐唐朝年代記統紀馬總唐年小録聖運圖會要皆云八十一唐録政要貞觀十三年入宫據武氏入宫年十四今從吳兢實録為八十二故置此年}


十二年春正月乙未禮部尚書王珪奏三品已上遇親王於路皆降乘非禮|{
	乘繩證翻}
上曰卿輩苟自崇貴輕我諸子特進魏徵曰諸王位次三公今三品皆九卿八座為王降乘誠非所宜當|{
	為于偽翻}
上曰人生壽夭難期萬一太子不幸安知諸王它日不為公輩之主何得輕之|{
	時太子承乾有足疾魏王泰有寵太宗此言固有以泰代承乾之心矣夭于紹翻}
對曰自周以來皆子孫相繼不立兄弟所以絶庶孽之窺窬塞禍亂之源本|{
	孽魚列翻塞悉則翻}
此為國者所深戒也上乃從珪奏 吏部尚書高士亷黄門侍郎韋挺禮部侍郎令狐德棻中書侍郎岑文本撰氏族志成上之|{
	令音鈴棻符分翻撰士免翻上時掌翻}
先是山東人士崔盧李鄭諸族好自矜地望|{
	先悉薦翻好呼到翻}
雖累葉陵夷苟他族欲與為昏姻|{
	白虎通曰昏者昏時行禮故曰昏姻者婦人因夫故曰姻賢曰妻父曰婚婿父曰姻}
必多責財幣或捨其鄉里而妄稱名族或兄弟齊列而更以妻族相陵上惡之|{
	惡烏路翻}
命士亷等徧責天下譜諜質諸史籍考其真偽辦其昭穆|{
	譜博古翻諜逹協翻昭時召翻}
第其甲乙褒進忠賢貶退姦逆分為九等士亷等以黄門侍郎崔民幹為第一上曰漢高祖與蕭曹樊灌皆起閭閻布衣卿輩至今推仰以為英賢豈在世禄乎高氏偏據山東梁陳僻在江南雖有人物盖何足言况其子孫才行衰薄|{
	行下孟翻下德行同}
官爵陵替而猶卬然以門地自負販鬻松檟依託富貴棄亷忘恥不知世人何為貴之今三品以上或以德行或以勲勞或以文學致位貴顯|{
	行下孟翻}
彼衰世舊門誠何足慕而求與為昏雖多輸金帛猶為彼所偃蹇我不知其解何也|{
	解猶說也}
今欲釐正訛謬捨名取實而卿曹猶以崔民幹為第一是輕我官爵而徇流俗之情也乃更命刋定專以今朝品秩為高下|{
	更工衡翻朝直遥翻}
於是以皇族為首外戚次之降崔民幹為第三|{
	九等之次皇族為上之上外戚為上之中崔民幹為上之下}
凡二百九十三姓千六百五十一家頒於天下 二月乙卯車駕西還|{
	自洛陽西還長安還從宣翻又音如字}
癸亥幸河北觀砥柱|{
	自西還便道幸河北縣河北縣漢晉属河東郡後魏置河北郡隋廢郡復為縣属蒲州縣南河中有砥柱山貞觀元年以河北縣度属陜州括地志曰陜州河北縣本漢大陽縣}
甲子巫州獠反|{
	貞觀元年分辰州之龍標縣置巫州獠魯皓翻}
夔州都督齊善行敗之|{
	敗補邁翻}
俘男女三千餘口 乙丑上祀禹廟丁卯至柳谷觀鹽池|{
	禹都安邑後人立廟於其地安邑有鹽池則柳谷亦當在安邑}
庚午至蒲州刺史趙元楷課父老服黄紗單衣迎車駕盛飾廨舍樓觀|{
	廨古隘翻觀古玩翻}
又飼羊百餘頭魚數百頭以饋貴戚|{
	飼祥吏翻}
上數之曰朕廵省河洛|{
	數所具翻又所主翻省悉景翻}
凡有所須皆資庫物卿所為乃亡隋之弊俗也甲戌幸長春宫 戊寅詔曰隋故鷹撃郎將堯君素雖桀犬吠堯有乖倒戈之志而疾風勁草實表歲寒之心可贈蒲州刺史仍訪其子孫以聞|{
	將即亮翻堯君素事始一百八十四卷隋恭帝義寧元年終一百八十六卷高祖武德二年漢鄒陽曰桀之犬可使吠堯武王伐紂前徒倒戈攻其後以北吠扶廢翻}
閏月庚辰朔日有食之 丁未車駕至京師 三月辛亥著作佐郎鄧世隆表請集上文章上曰朕之辭令有益于民者史皆書之足為不朽若為無益集之何用梁武帝父子陳後主隋煬帝皆有文集行於世何救於亡為人主患無德政文章何為遂不許丙子以皇孫生宴五品以上於東宫上曰貞觀之前

從朕經營天下玄齡之功也貞觀以來繩愆糾繆魏徵之功也|{
	觀古玩翻}
皆賜之佩刀上謂徵曰朕政事何如往年對曰威德所加比貞觀之初則遠矣人悦服則不逮也上曰遠方畏威慕德故來服若其不逮何以致之對曰陛下往以未治為憂故德義日新今以既治為安故不逮|{
	治直吏翻}
上曰今所為猶往年也何以異對曰陛下貞觀之初恐人不諫常導之使言中間悦而從之今則不然雖勉從之猶有難色所以異也上曰其事可聞歟對曰陛下昔欲殺元律師孫伏伽以為法不當死陛下賜以蘭陵公主園直百萬或云賞太厚|{
	蘭陵公主上女也下嫁竇懷悊上以其園賞孫代伽}
陛下云朕即位以來未有諫者故賞之此導之使言也司戶柳雄妄訴隋資|{
	隋資隋朝所授官資也}
陛下欲誅之納戴胄之諫而止是悦而從之也近皇甫德參上書諫修洛陽宫陛下恚之雖以臣言而罷勉從之也|{
	皇甫德麥事見上卷八年上時掌翻恚于避翻}
上曰非公不能及此人苦不自知耳 夏五月壬申弘文館學士永興文懿公虞世南卒|{
	唐六典弘文館學士無員數後漢有東觀魏有崇文館宋元嘉有玄史两館宋泰始至齊永明有總文館梁有士林館北齊有文林館後周有崇文館或典校理或司撰著或兼訓生徒若今弘文館之任也武德初置修文館武德末改為弘文館永興縣屬鄂州謚法温柔賢善曰懿卒子恤翻}
上哭之慟世南外和柔而内忠直上嘗稱世南有五絶一德行|{
	行下孟翻}
二忠直三博學四文辭五書翰 秋七月癸酉以吏部尚書高士亷為右僕射 乙亥吐蕃寇弘州|{
	弘恐當作松吐從暾入聲}
八月霸州山獠反|{
	按天寶元年招附生羌置静戎郡乾元元年方置霸州又松州都督府所管党項羈縻州有霸州然當以其酋豪為刺史而此霸州又是儀鳳二年松州加督三十八州之數獠魯皓翻}
燒殺刺史向邵陵及吏民百餘家 初上遣使者馮德遐撫慰吐蕃|{
	吐從暾入聲}
吐蕃聞突厥吐谷渾皆尚公主|{
	厥九勿翻谷音浴}
遣使隨德遐入朝|{
	使疏吏翻朝直遥翻}
多齎金寶奉表求婚上未之許使者還言於贊普棄宗弄讚曰臣初至唐唐待我甚厚許尚公主會吐谷渾王入朝相離間|{
	間古莫翻}
唐禮遂衰亦不許昏弄讚遂發兵撃吐谷渾吐谷渾不能支遁千青海之北民畜多為吐蕃所掠吐蕃進破党項白蘭諸羌帥衆二十餘萬屯松州西境|{
	党底朗翻帥讀曰率}
遣使貢金帛云來迎公主尋進攻松州敗都督韓威|{
	敗補邁翻下敗吐同}
羌酋閻州刺史别叢臥施諾州刺史把利步利並以州叛歸之|{
	貞觀五年以党項降羌置羈縻州有□州諾州皆屬松州都督府無閻州酋慈由翻}
連兵不息其大臣諫不聼而自縊者凡八輩|{
	縊于計翻又于賜翻}
壬寅以吏部尚書侯君集為當彌道行軍大總管甲辰以右領軍大將軍執失思力為白蘭道左武衛將軍牛進逹為闊水道左領軍將軍劉簡為洮河道行軍總管督步騎五萬撃之|{
	洮土刀翻騎奇寄翻}
吐蕃攻城十餘日進逹為先鋒九月辛亥掩其不備敗吐蕃于松州城下|{
	宋白曰松州之地漢魏諸羌居之及晉内附以其地属汶山郡後魏時鄧至王像舒據之遣使朝貢始置甘松縣後周置龍涸防唐置松州去長安二千二百五十里}
斬首千餘級弄讚懼引兵退遣使謝罪因復請婚|{
	使疏吏翻復扶又翻}
上許之 甲寅上問侍臣創業與守成孰難房玄齡曰草昧之初|{
	易曰天造草昧王弼注云造物之始始于冥昧故曰草昧也廣雅草造也董云草昧微物}
與羣雄並起角力而後臣之創業難矣魏徵曰自古帝王莫不得之於艱難失之於安逸守成難矣上曰玄齡與吾共取天下出百死得一生故知創業之難徵與吾共安天下常恐驕奢生於富貴禍亂生於所忽故知守成之難然創業之難既已往矣守成之難方當與諸公慎之玄齡等拜曰陛下及此言四海之褔也 初突厥頡利既亡北方空虛|{
	厥九勿翻頡奚結翻}
薛延陀真珠可汗帥其部落建庭于都尉揵山北獨邏水南|{
	按薛延陀建庭之地在鬱督軍山東南距京師纔三千里而贏新書曰烏德揵山左右嗢昆河獨邏河皆屈曲東北流嗢昆在南獨邏在北過回紇牙帳東北五百里而合流可從刋入聲汗音寒帥讀曰率揵居言翻邏郎佐翻}
勝兵二十萬|{
	勝音升}
立其二子拔酌頡利苾主南北部|{
	苾 必翻}
上以其彊盛恐後難制癸亥拜其二子皆為小可汗各賜鼓纛|{
	纛徒到翻}
外示優崇實分其勢 冬十月乙亥巴州獠反|{
	後漢於宕渠北界置漢昌縣後魏於縣置大谷郡又於郡北置巴州随改為清化郡唐復為巴州獠魯皓翻下同}
己卯畋于始平|{
	曹魏置始平縣屬扶風晉分立始平郡後魏復為縣屬扶風隋屬京兆九域志在府西八十里}
乙未還京師 鈞州獠反遣桂州都督張寶德討平之 十一月丁未初置左右屯營飛騎於玄武門以諸將軍領之又簡飛騎才力驍健善騎射者號百騎衣五色袍乘駿馬以虎皮為韀|{
	騎奇寄翻驍堅堯翻衣於既翻韀則前翻}
凡遊幸則從焉 己巳明州獠反|{
	吳置越裳縣屬九德郡以古越裳之地也隋屬驩州日南郡武德五年以越裳地置明州}
遣交州都督李道彦討平之 十二月辛巳左武候將軍上官懷仁擊反獠于壁州|{
	後漢和帝分岩縣之東置宣漢縣梁分宣漢置始寧縣元魏分始寧縣置諾水縣武德八年分巴州之始寧縣置壁州始寧郡}
大破之虜男女萬餘口 是歲以給事中馬周為中書舍人周有機辯中書侍郎岑文本常稱馬君論事援引事類掦榷古今|{
	毛晃曰掦榷大舉又掎也舉而引之也榷訖岳翻}
舉要刪煩會文切理一字不可增亦不可減聼之靡靡令人忘倦 霍王元軌好讀書恭謹自守舉措不妄為徐州刺史與處士劉玄平為布衣交|{
	好呼到翻處昌呂翻}
人問玄平王所長玄平曰無長問者怪之玄平曰夫人有所短乃見所長|{
	夫音扶}
至於霍王無所短吾何以稱其長哉 初西突厥咥利失可汗分其國為十部每部有酋長一人|{
	酋慈由翻長知兩翻可從刋入聲汗音寒}
仍各賜一箭謂之十箭又分左右廂左廂號五咄陸置五大啜居碎葉以東右廂號五弩失畢置五大俟斤居碎葉以西通謂之十姓|{
	咄陸五啜號處木昆律啜胡禄屋闕啜攝舍提敦啜突騎施賀邏施啜鼠尼施處半啜弩失畢五俟斤號阿悉結闕俟斤哥舒闕俟斤拔寒幹暾沙鉢俟斤阿悉結泥孰俟斤阿舒虚半俟斤碎葉城在焉耆碎葉川出安西西北千里至碎葉杜佑曰碎葉川長千餘里東頭有熱海西頭有怛邏期城咄當沒翻啜陟劣翻康曰俟渠之切}
咥利失失衆心為其臣統吐屯所襲咥利失兵敗與其弟步利設走保焉耆|{
	新書曰焉耆國直京師西七千里而嬴横六百里縱四百里其國東高昌西龜兹南尉黎北烏孫漢舊國也}
統吐屯等將立欲谷設為大可汗會統吐屯為人所殺欲谷設兵亦敗咥利失復得故地|{
	復扶又翻又音如字}
至是西部竟立欲谷設為乙毗咄陸可汗乙毗咄陸既立與咥利失大戰殺傷甚衆因中分其地自伊列水以西屬乙咄陸以東屬咥利失|{
	伊列水亦名伊麗水注詳見後}
處月處密與高昌共攻抜焉耆五城掠男女一千五

百人焚其廬舍而去|{
	為伐高昌張本}


十三年春正月乙巳車駕謁獻陵|{
	唐謁陵之制設行宫距陵十里設坐於齋室設小次於陵所道西南大次於寢西南侍臣次於大次西南陪位者次又於西南皆東向文官於北武官於南朝集使又於其南皆相地之宜皇帝至行宫即齋室陵令以玉册進署設御位於陵東南隅西向有岡麓之陔則隨地之宜又設位於寢宫之殿東陛之東南西向尊坫陳於堂戶東南百官行從宗室客使位神道左右寢宫則分方序立大次前其日未明五刻陳黄麾仗于陵寢三刻行事官及宗室親五等諸親三等以上及客使之當陪者就位皇帝素服乘馬華盖繖扇侍臣騎從詣小次出次至位再拜又再拜在位者皆再拜又再拜少選太常卿請辭皇帝再拜又再拜奉禮曰奉辭在位者再拜皇帝還小次乘馬詣大次仗衛列立以俟行百官宗室諸親客使序立次前皇帝步至寢宮南門仗衛止乃入由東序進殿陛東南位再拜升自東階北向再拜又再拜入省服玩抆拭帳簀進太牢之饌加珍羞皇帝出尊所酌酒入三奠爵北向立太祝二人持玉册立于戶外東向跪讀皇帝再拜又再拜乃出戶當前北向立太常卿請辭皇帝再拜出東門還大次宿行宫}
丁未還宫 戊午加左僕射房玄齡太子少師玄齡自以居端揆十五年|{
	左右僕射尚書省長官故曰端揆按武德九年房玄齡為中書令貞觀二年為左僕射至是財十一年未及十五年也少始照翻}
男遺愛尚上女高陽公主女為韓王妃|{
	韓王元嘉高袓之子}
深畏滿盈上表請解機務|{
	上時掌翻}
上不許玄齡固請不已詔斷表乃就職|{
	斷音短丁管翻今之讓官者奉表三讓不許勑斷來章則閤門不復受其表即唐制之斷表也}
太子欲拜玄齡設儀衛待之玄齡不敢謁見而歸時人善其有讓玄齡以度支繫天下利害嘗有闕求其人未得乃自領之|{
	唐制度支郎中掌天下租賦物產豐約之宜水陸道塗之利歲計所出而支調之以近及遠與中書門下議定乃奏國之大計所關也玄齡審官求賢未得其人故自領之唐中世以後宰相多判度支盖昉于此度徒洛翻}
禮部尚書永寧懿公王珪薨|{
	永寧縣屬洛州}
珪性寛裕自奉養甚薄於令三品已上皆立家廟|{
	唐制三品已上得立廟祭三代}
珪通貴已久獨祭於寢為法司所劾|{
	劾戶槩翻又戶得翻}
上不問命有司為之立廟以愧之|{
	司為于偽翻下上為同}
二月庚辰以光禄大夫尉遲敬德為鄜州都督|{
	尉紆勿翻鄜芳無翻}
上嘗謂敬德曰人或言卿反何也對曰臣反是實臣從陛下征伐四方身經百戰今之存者皆鋒鏑之餘也天下已定乃更疑臣反乎因解衣投地出其瘢痍|{
	瘢薄官翻痍音夷}
上為之流涕曰卿復服朕不疑卿故語卿何更恨邪上又嘗謂敬德曰朕欲以女妻卿何如|{
	邪音耶語牛倨翻妻七細翻}
敬德叩頭謝曰臣妻雖鄙陋相與共貧賤久矣臣雖不學聞古人富不易妻此非臣所願也上乃止 戊戍尚書奏近世掖庭之選|{
	掖音亦}
或微賤之族禮訓蔑聞|{
	謂由侍兒及歌舞得進者}
或刑戮之家憂怨所積|{
	謂緣坐沒入掖庭者}
請自今後宫及東宫内職有闕皆選良家有才行者充|{
	行下孟翻}
以禮聘納其沒宫口及素微賤之人皆不得補用上從之 上既詔宗室羣臣襲封刺史左庶子于志寧以為古今事殊恐非久安之道上疏争之|{
	上時掌翻下同}
侍御史馬周亦上疏以為堯舜之父猶有朱均之子|{
	朱均謂丹朱商均也}
儻有孩童嗣職萬一驕愚兆庶被其殃而國家受其敗|{
	孩何開翻被皮義翻}
正欲絶之也則子文之治猶在|{
	左傳楚闘椒作亂莊王滅若敖氏既而思子文之治楚國也曰子文無後何以勸善使其孫箴尹克黄復其所治直吏翻}
正欲留之也而欒黶之惡已彰|{
	右傳秦伯問於士鞅曰晉大夫其誰先亡對曰其欒氏乎欒黶汰虐已甚猶可以免其在盈乎秦伯曰何故對曰武子之德在民如周人之思召公焉愛其甘棠况其子乎欒黶死盈之善未及民武子所施沒矣而黶之怨實彰將於是乎在}
與其毒害於見存之百姓|{
	見賢遍翻}
則寧使割恩於已亡之一臣明矣然則向所謂愛之者乃適所以傷之也臣謂宜賦以茅土疇其戶邑必有材行隨器授官|{
	行下孟翻}
使其人得奉大恩而子孫終其褔禄會司空趙州刺史長孫無忌等皆不願之國上表固讓|{
	長知两翻}
稱承恩以來形影相弔若履春冰|{
	春來冰薄履之則有陷溺之懼}
宗族憂虞如寘湯火緬惟三代封建盖由力不能制因而利之禮樂節文多非已出兩漢罷候置守蠲除曩弊深恊事宜|{
	守式又翻}
今因臣等復有變更|{
	復扶又翻更工衡翻}
恐紊聖朝綱紀|{
	紊音問朝直遥翻}
且後世愚幼不肖之嗣或抵冒邦憲自取誅夷|{
	冒莫北翻}
更因延世之賞致成勦絶之禍良可哀愍|{
	勦子小翻}
願停渙汗之旨賜其性命之恩無忌又因子婦長樂公主固請於上|{
	主嫁無忌子冲樂音洛}
且言臣披荆棘事陛下今海内寧一奈何棄之外州與遷徙何異上曰割地以封功臣古今通義意欲公之後嗣輔朕子孫共傳永久而公等乃復發言怨望朕豈強公等以茅土邪|{
	復扶又翻強其两翻邪音耶}
庚子詔停世封刺史 高昌王麴文泰多遏絶西域朝貢|{
	朝直遥翻下同}
伊吾先臣西突厥既而内屬|{
	事見一百九十三卷四年厥九勿翻}
文泰與西突厥共擊之上下書切責|{
	下遐嫁翻}
徵其大臣阿史那矩欲與議事文泰不遣遣其長史麴雍來謝罪|{
	長知兩翻}
頡利之亡也|{
	見一百九十三卷四年}
中國人在突厥者或奔高昌詔文泰歸之文泰蔽匿不遣又與西突厥共擊破焉耆焉耆訴之|{
	掠焉耆見上卷六年又見上年}
上遣虞部郎中李道裕往問狀|{
	虞部郎掌京城街巷種植山澤苑囿草木薪炭供頓田獵之事屬工部}
且謂其使者曰高昌數年以來朝貢脱略無藩臣禮所置官號皆凖天朝築城掘溝預備攻討我使者至彼文泰語之云鷹飛于天雉伏于蒿猫遊于堂鼠噍于宂|{
	使疏吏翻語牛倨翻噍而笑翻}
各得其所豈不能自生邪又遣使謂薛延陀曰既為可汗則與天子匹敵何為拜其使者事人無禮又聞鄰國為惡|{
	使疏吏翻下同間古莧翻}
不誅善何以勸明年當發兵擊汝三月薛延陀可汗遣使上言奴受恩思報請發所部為軍導以擊高昌|{
	可從刋入聲汗音寒上時掌翻}
上遣民部尚書唐儉右領軍大將軍執失思力齎繒帛賜薛延陀與謀進取|{
	繒慈陵翻}
夏四月戊寅上幸九成宫初突厥突利可汗之弟結社率從突利入朝|{
	厥九勿翻可從刋入聲汗音寒朝直遥翻}
歷位中郎將|{
	將即亮翻}
居家無賴怨突利斥之乃誣告其謀反上由是薄之久不進秩結社率隂結故部落得四十餘人謀因晉王治四鼓出官開門辟仗|{
	辟毗亦翻}
馳入宫門直指御帳可有大功甲申擁突利之子賀邏鶻夜伏于宫外|{
	邏郎佐翻鶻戶骨翻}
會大風晉王未出結社率恐曉遂犯行宫踰四重幕弓矢亂發衛士死者數十人折衝孫武開等帥衆奮擊|{
	重直龍翻折之舌翻折衝折衝都尉也帥讀曰率}
久之乃退馳入御廐盗馬二十餘匹北走度渭欲奔其部落追獲斬之原賀邏鶻投于嶺表 庚寅遣武候將軍上官懷仁擊巴壁洋集四州反獠平之|{
	洋音祥獠魯皓翻}
虜男女六千餘口 五月旱甲寅詔五品以上上封事|{
	上封時掌翻下同}
魏徵上疏以為陛下志業比貞觀之初漸不克終者凡十條|{
	觀古玩翻}
其間一條以為頃年以來輕用民力乃云百姓無事則驕逸勞役則易使|{
	易以豉翻}
自古未有因百姓逸而敗勞而安者也此恐非興邦之至言上深加奬歎云已列諸屏障朝夕聸仰并録付史官仍賜徵黄金十斤廐馬二匹 六月渝州人侯弘仁自牂柯開道經西趙出邕州以通交桂|{
	東謝蠻西接牂柯蠻南接西趙蠻牂柯之别帥曰羅殿今廣西買馬路自桂州至邕州横山寨二十餘程自横山至杞國二十二程又至羅殿十程此即候弘仁所通者也邕州漢欝林郡領方縣地晉分欝林置晉興郡隋廢晉興為宣化縣屬欝林郡唐武德四年置南晉州貞觀六年改邕州朗寧郡牂柯音臧哥}
蠻俚降者二萬八千餘戶|{
	俚音里降戶江翻}
丙申立皇弟元嬰為滕王 自結社率之反言事者多云突厥留河南不便|{
	河南謂北河之南漢衛青擊匈奴所收河南地是也厥九勿翻}
秋七月庚戍詔右武候大將軍化州都督懷化郡王李思摩為乙彌泥孰俟利苾可汗賜之鼓纛|{
	俟渠之翻苾毗必翻纛徒到翻}
突厥及胡在諸州安置者並令度河還其舊部俾世作藩屏|{
	屏必郢翻}
長保邊塞突厥咸憚薛延陀不肯出塞上遣司農卿郭嗣本賜薛延陀璽書|{
	璽斯氏翻}
言頡利既敗|{
	頡奚結翻}
其部落咸來歸化我略其舊過嘉其後善待其逹官皆如吾百寮部落皆如吾百姓中國貴尚禮義不滅人國前破突厥止為頡利一人為百姓害|{
	止為于偽翻}
實不貪其土地利其人畜恒欲更立可汗|{
	恒戶登翻下同}
故置所降部落於河南任其畜牧今戶口蕃滋|{
	蕃扶元翻}
吾心甚喜既許立之不可失信秋中將遣突厥度河復其故國爾薛延陀受册在前|{
	延陀受册見一百九十三卷二年}
突厥受册在後後者為小前者為大爾在磧北突厥在磧南各守土疆鎮撫部落其踰分故相抄掠|{
	磧七迹翻分扶問翻抄楚交翻}
我則發兵各問其罪薛延陀奉詔於是遣思摩帥所部建牙於河北|{
	河北則大磧之南帥讀曰率}
上御齊政殿餞之思摩涕泣奉觴上壽曰奴等破亡之餘分為灰壤|{
	上壽時掌翻分扶問翻}
陛下存其骸骨復立為可汗|{
	復扶又翻下復下同可從刋入聲汗音寒}
願萬世子孫恒事陛下|{
	恒戶登翻}
又遣禮部尚書趙郡王孝恭等齎册書就其種落築壇於河上而立之|{
	種章勇翻}
上謂侍臣曰中國根榦也四夷枝葉也割根榦以奉枝葉木安得滋榮朕不用魏徵言幾致狼狽|{
	謂結社率之變也魏徵言見上卷四年幾居希翻}
又以左屯衛將軍阿史那忠為左賢王左武衛將軍阿失那泥孰為右賢王忠蘇尼失之子也|{
	蘇尼失見一百九十三卷四年}
上遇之甚厚妻以宗女|{
	妻七細翻}
及出塞懷慕中國見使者必泣涕請入待詔許之 八月辛未朔日有食之 詔以身體髮膚不敢毁傷|{
	引孝經孔子之言}
比來訴訟者或自毁耳目|{
	比毗至翻}
自今有犯先笞四十然後依法|{
	依法處斷其所訴之事也}
冬十月甲申車駕還京師|{
	自九成宫還也}
十一月辛亥以侍中楊師道為中書令 戊辰尚書左丞劉洎為黄門侍郎參知政事|{
	洎其冀翻}
上猶冀高昌王文泰悔過復下璽書示以禍褔徵之入朝|{
	下遐嫁翻璽所氏翻朝直遥翻下同}
文泰竟稱疾不至十二月壬申遣交河行軍大總管吏部尚書侯君集副總管兼左屯衛大將軍薛萬均等將兵擊之|{
	將即亮翻}
乙亥立皇子褔為趙王 己丑吐谷渾王諾曷鉢來朝以宗女為弘化公主妻之|{
	妻七細翻}
壬辰上畋于咸陽|{
	咸陽秦都漢為渭城縣屬右扶風晉廢縣後魏置咸陽郡隋廢武德元年分涇陽始平置咸陽縣屬京兆九域志在府西四十里}
癸巳還宫 太子承乾頗以遊畋廢學右庶子張玄素諫不聼 是歲天下州府凡三百五十八縣一千五百一十一 太史令傳奕精究術數之書而終不之信遇病不呼醫餌藥有僧自西域來善呪術能令人立死復呪之使蘇上擇飛騎中壯者試之皆如其言|{
	呪職救翻復扶又翻騎奇寄翻}
以告奕奕曰此邪術也臣聞邪不干正請使呪臣必不能行上命僧呪奕奕初無所覺須臾僧忽僵仆若為物所擊遂不復蘇|{
	僵居良翻復扶又翻}
又有婆羅門僧|{
	天竺漢身毒國也或曰摩伽陀曰婆羅門}
言得佛齒所擊前無堅物長安士女輻湊如市奕時臥疾謂其子曰吾聞有金剛石性至堅物莫能傷唯羚羊角能破之|{
	杜佑曰扶南國出金剛石可以刻玉狀如紫石英其所生乃在百丈水底盤石上始如鍾乳人取之竟日乃出以鐵鎚之而不傷鐵乃自損以羚羊角扣之漼然冰泮陶弘景曰羚羊今出建平宜都蠻中及西域多两角一角者為勝角甚多節感蹙員繞陳藏器餘曰羚羊有神夜宿以角掛樹不著地羚音零}
汝往試焉其子往見佛齒出角叩之應手而碎觀者乃止奕臨終戒其子無得學佛書時年八十五又集魏晉以來駁佛教者為高識傳十卷行於世|{
	駁北角翻傳直戀翻}
西突厥咥利失可汗之臣俟利發與乙毗咄陸可汗通謀作亂咥利失窮蹙逃奔鏺汗而死|{
	新書曰寧遠者本拔汗那或曰鏺汗元魏所謂破洛那居西鞬城在真珠河之北去京師八千里厥九勿翻咥徒結翻又丑栗翻可從刋入聲汗音寒俟渠之翻咄當沒翻鏺普活翻}
弩失畢部落迎其弟子薄布特勒立之是為乙毗沙鉢羅葉護可汗沙鉢羅葉護既立建庭於雖合水北謂之南庭自龜茲鄯善且末吐火羅焉耆石史何穆康等國皆附之|{
	龜茲一曰丘茲一曰屈茲東距京師七千里而贏自于闐東關東行入大流沙行千里至故折摩馱那古且末也又千里至故納縳波古樓蘭也吐火羅或曰吐豁羅曰覩貨羅元魏謂之吐呼羅居葱嶺烏滸河之南古大夏也石國或曰柘支曰柘折曰赭時漢大宛北鄙也去京師九千里東北距西突厥王姓石治柘折城故康居小王窳匿城也史或曰佉沙曰羯霜那居獨莫水南康居小王蘇□城故地南四百里抵吐火羅何或曰屈霜彌伽曰貴霜匿即康居小王附墨城故地新書康漢康居也枝庶分王曰安曰曹曰米曰何曰火尋曰戊地曰史世謂九姓意者穆亦康國枝庶歟龜茲音丘慈鄯時戰翻且子余翻}
咄陸建牙於鏃曷山西謂之北庭|{
	舊書自焉耆西北七日行至其南庭又正北八日行至其北庭鏃作木翻}
自厥越失拔悉彌駁馬結骨火燖觸水昆等國皆附之|{
	拔悉彌蓋即拔悉密在葛邏禄之西駁馬或曰弊刺曰遏邏支直突厥之北距京師萬四千里北極于海以馬耕田雖畜馬而不乘資湩酪以食馬色皆駁故以名國結骨古堅昆國也當伊吾西焉耆北白山之旁堅昆後語訛為結骨稍號紇骨亦曰紇扢斯又曰黠戞斯火燖或為貨利習彌曰過利居烏滸水之陽西南與波斯接西北抵突厥駁北角翻燖徐塩翻}
以伊列水為境|{
	伊列漢時西域故國在康居北陳湯與甘延壽謀郅支曰北擊伊列西取安息此其證也 考異曰沙鉢羅葉護傳云東以伊列河為界按乙毗咄陸傳云自伊列河以西屬咄陸以東屬咥利失沙鉢羅葉護既因咥利失之地應云西以伊列河為界今未知二傳孰誤故但云伊列水為境}
十四年春正月甲寅上幸魏王泰第赦雍州長安繫囚大辟以下免延康里今年租賦賜泰府僚屬及同里老人有差|{
	魏王泰第在長安城中延康里按雍州二赤縣長安萬年皆治長安城中今止赦長安囚盖延康里屬長安縣管雍於用翻辟毗亦翻}
二月丁丑上幸國子監觀釋奠|{
	按唐國子監在安上門西唐制仲春仲秋釋奠于文宣王皆以上丁上戊以祭酒司業博士三獻}
命祭酒孔頴逹講孝經賜祭酒以下至諸生高第帛有差|{
	周官有師氏保氏漢始置祭酒博士晉始立國子學唐國子祭酒從三品掌那國儒學訓導之政令}
是時上大徵天下名儒為學官數幸國子監使之講論學生能明一大經已上皆得補官|{
	唐取士以禮記春秋左氏傳為大經詩儀禮周禮為中經易尚書春秋公羊傳穀梁傳為小經數所角翻已上時掌翻}
增築學舍千二百間增學生滿二千二百六十員自屯營飛騎亦給博士使授以經有能通經者聼得貢舉於是四方學者雲集京師乃至高麗百濟新羅高昌吐蕃諸酋長亦遣子弟請入國學升講筵者至八千餘人|{
	騎奇寄翻麗力知翻酋慈由翻長知兩翻吐從暾入聲 考異曰舊傳云八十餘人今從新書}
上以師說多門章句繁雜命孔頴逹與諸儒撰定五經疏謂之正義令學者習之|{
	五經正義今行於世撰士免翻疏所去翻令力丁翻}
壬午上行幸驪山温湯|{
	驪力知翻}
辛卯還宫 乙未詔求近世名儒梁皇甫侃禇仲都周熊安生沈重陳沈文阿周弘正張譏隋何妥劉炫等子孫以聞當加引擢|{
	妥吐火翻炫熒絹翻}
三月竇州道行軍總管党仁弘擊羅竇反獠破之俘七千餘口|{
	獠魯皓翻}
辛丑流鬼國遣使入貢去京師萬五千里濱於北海南鄰靺鞨|{
	流鬼國直黑水靺鞨東北少海之北三面阻海南與莫曳靺鞨鄰東南航海十五日行乃至人依島嶼散居多沮澤初附百濟後附新羅東夷也杜佑曰流鬼國在北海之北使疏吏翻靺音末鞨音曷}
未嘗通中國重三譯而來|{
	重直龍翻}
上以其使者佘志為騎都尉|{
	孫愐曰佘視遮翻姓也}
丙辰置寧朔大使以護突厥|{
	厥九勿翻}
夏五月壬寅徙燕王靈夔為魯王|{
	燕因肩翻}
上將幸洛陽命將作大匠閻立德行清暑之地|{
	行下孟翻}
秋八月庚午作襄城宫於汝州西山|{
	秦置將作掌營繕宫室歷代不改漢景帝置將作大匠唐從三品掌供邦國修造土木工匠之政令新志貞觀中置清暑宫于汝州臨汝縣鳴臯山南按汝水睨廣成澤}
立德立本之兄也|{
	閻立本高宗朝為相}
高昌王文泰聞唐兵起謂其國人曰唐去我七千里沙磧居其二千里地無水草寒風如刀熱風如燒安能致大軍乎往吾入朝|{
	入朝見一百九十三卷四年磧七迹翻朝直遥翻}
見秦隴之北城邑蕭條非復有隋之北|{
	復扶又翻}
今來伐我發兵多則糧運不給三萬已下吾力能制之當以逸待勞坐收其弊若頓兵城下不過二十日食盡必走然後從而虜之何足憂也及聞唐兵臨磧口憂懼不知所為發疾卒|{
	卒子恤翻}
子智盛立軍至柳谷|{
	新志西州交河縣北行二百一十里至柳谷渡}
詗者言文泰刻日將葬|{
	詗休正翻又古迥翻}
國人咸集於彼諸將請襲之|{
	將即亮翻下同}
候君集曰不可天子以高昌無禮故使吾討之今襲人於墟墓之間非問罪之師也於是鼓行而進至田城 |{
	考異曰實録作田地城今從舊傳按田城即田地城也麴嘉之王高昌也置田地太守封其二子一為交河公一為田地公新書曰田地城即漢戊已校尉所治地宋白曰西州高昌縣本晉田地縣之地輿地志云晉咸和二年置高昌郡立田地縣唐改高昌縣}
諭之不下詰朝攻之|{
	詰去吉翻}
及午而克虜男女七千餘口以中郎將辛獠兒為前鋒夜趨其都城|{
	將即亮翻獠魯皓翻趨七喻翻}
高昌逆戰而敗大軍繼至抵其城下智盛致書於君集曰得罪於天子者先王也天罰所加身已物故智盛襲位未幾惟尚書憐察|{
	幾居豈翻}
君集報曰苟能悔過當束手軍門智盛猶不出君集命填塹攻之飛石雨下城中人皆室處|{
	塹七艷翻處昌呂翻}
又為巢車高十丈俯瞰城中|{
	左傳楚子登巢車以望晉軍釋文云兵車高如巢以望敵也杜預曰車上施櫓杜佑說見前巢居傲翻瞰苦濫翻}
有行人及飛石所中皆唱言之先是文泰與西突厥可汗相結|{
	中竹仲翻先悉薦翻厥九勿翻可從刋入聲汗音寒 考異曰舊傳云與欲谷設約按欲谷設去歲已敗死今不取}
約有急相助可汗遣其葉護屯可汗浮圖城|{
	葉護突厥逹官也為大臣之首自交河城至浮圖城三百七十里}
為文泰聲援及君集至可汗懼而西走千餘里葉護以城降智盛窮蹙癸酉開門出降|{
	高昌自麴嘉有國傳九世一百三十四年而亡降戶江翻}
君集分兵略地下其二十二城戶八千四十六口一萬七千七百 |{
	考異曰舊傳戶八千口三萬七千七百今從實録}
地東西八百里南北五百里上欲以高昌為州縣魏徵諫曰陛下初即位文泰夫婦首來朝|{
	文泰入朝見四年}
其後稍驕倨故王誅加之罪止文泰可矣宜撫其百姓存其社稷復立其子則威德被於遐荒四夷皆悦服矣|{
	復扶又翻被皮義翻}
今若利其土地以為州縣則常須千餘人鎮守數年一易往來死者什有三四供辦衣資違離親戚|{
	離力智翻}
十年之後隴右虚耗矣陛下終不得高昌撮粟尺帛以佐中國所謂散有用以事無用臣未見其可上不從九月以其地為西州以可汗浮圖城為庭州|{
	西州治高昌縣漢車師前王庭也庭州治金滿縣漢車師後王庭也宋白曰二州相去四百五十里}
各置屬縣乙卯置安西都護府於交河城留兵鎮之君集虜高昌王智盛及其羣臣豪傑而還|{
	還從宣翻又如字}
於是唐地東極于海西至焉耆南盡林邑北抵大漠皆為州縣凡東西九千五百一十里南北一萬九百一十八里侯君集之討高昌也遣使約焉耆與之合勢|{
	使疏吏翻}
焉耆喜聼命及高昌破焉耆王詣軍門謁見君集且言焉耆三城先為高昌所奪君集奏并高昌所掠焉耆民悉歸之|{
	高昌掠焉耆見六年}
冬十月甲戌荆王元景等復表請封禪|{
	復扶又翻}
上不許 初陳倉折衝都尉魯寧坐事繫獄自恃高班慢罵陳倉尉尉氏劉仁軌|{
	陳倉縣屬岐州唐制畿縣尉正九品下上縣尉從九品上中下縣從九品下}
仁軌杖殺之州司以聞上怒命斬之猶不解曰何物縣尉敢殺吾折衝命追至長安面詰之仁軌曰魯寧對臣百姓辱臣如此臣實忿而殺之辭色自若魏徵侍側曰陛下知隋之所以亡乎上曰何也徵曰隋末百姓彊而陵官吏如魯寧之比是也|{
	魯寧官為折衝本陳倉百姓}
上悦擢仁軌為櫟陽丞|{
	漢高皇帝葬太上皇于櫟陽北原號萬年陵改櫟陽為萬年縣至隋猶因之唐都長安改隋大興縣曰萬年而舊萬年縣復曰櫟陽屬京兆唐畿縣丞正八品下}
上將幸同州校獵仁軌上言|{
	上時掌翻}
今秋大稔民收穫者什纔一二使之供承獵事治道葺橋動費一二萬功實妨農事願少留鑾輿旬日俟其畢務則公私俱濟上賜璽書嘉納之|{
	治直之翻少詩沼翻璽斯氏翻}
尋遷新安令|{
	唐初置新安郡貞觀元年廢郡為縣屬洛州唐制畿縣令正六品下上縣從六品上中縣正七品上下縣從七品下}
閏月乙未行幸同州庚戌還宫丙辰吐蕃贊普遣其相禄東贊獻金五千兩及珍玩數百以請昏|{
	相息亮翻}
上許以文成公主妻之|{
	文成公主宗女也妻七細翻}
十一月甲子朔冬至上祀南郊時戊寅歷以癸亥為

朔|{
	行戊寅歷見一百八十七卷武德三年}
宣義郎李淳風表稱古歷分日起於子半今歲甲子朔旦冬至而故太史令傅仁均減餘稍多子初為朔遂差三刻用乖天正請更加考定衆議以仁均定朔微差淳風推校精密請如淳風議從之丁卯禮官奏請加高袓父母服齊衰五月嫡子婦服

朞嫂叔弟妻夫兄舅皆服小功從之|{
	按新志高袓作曾祖舊服齊衰三月嫡子婦舊服大功衆子婦舊服小功今加衆子婦服大功而嫂叔弟妻夫兄舅舊服緦者皆加服小功齊音咨衰土回翻}
丙子百官復表請封禪|{
	復扶又翻}
詔許之更命諸儒詳定儀注以太常卿韋挺等為封禪使|{
	使疏吏翻}
司門員外郎韋元方給給使過所稽緩|{
	唐司門郎掌天下諸門諸關出入往來之籍凡天下之關二十有六所以限内外隔華夷設險作固閑邪正禁者也凡度關者先經刑部司門請過所給使禁中給使令者宦官也唐内給使無常員凡無官品者號内給使屬宫闈署令}
給使奏之上怒出元方為華隂令|{
	華隂縣屬華州華戶化翻}
魏徵諫曰帝王震怒不可妄發前為給使遂夜出敕書|{
	為于偽翻}
事如軍機誰不驚駭况宦者之徒古來難養輕為言語易生患害獨行遠使深非事宜漸不可長|{
	易以豉翻使疏吏翻長知兩翻}
所宜深慎上納其言 尚書左丞韋悰句司農木橦價貴於民間|{
	悰藏宗翻句古候翻橦諸容翻木一截也唐式柴方三尺五寸為一橦按通典韋棕句司農木橦七十價百姓四十價奏其隐沒}
奏其隐沒上召大理卿孫伏伽書司農罪|{
	伽求迦翻}
伏伽曰司農無罪上怪問其故對曰只為官橦貴所以私橦賤|{
	為于偽翻}
向使官橦賤私橦無由賤矣但見司農識大體不知其過也上悟屢稱其善顧謂韋悰曰卿識用不逮伏伽遠矣 十二月丁酉侯君集獻俘于觀德殿|{
	觀德殿射殿也閣本太極宫圖射殿在宜春門北}
行飲至禮大酺三日|{
	酺薄乎翻}
尋以智盛為左武衛將軍金城郡公上得高昌樂工以付太常增九部樂為十部|{
	唐六典曰凡大宴會則設十部之伎於庭以備華夷一曰宴樂伎有景雲樂之舞慶善樂之舞破陳樂之舞承天樂之舞二曰清樂伎三曰西涼伎四曰天竺伎五曰高麗伎六曰龜茲伎七曰安國伎八曰疏勒伎九曰高昌伎十曰康國伎}
君集之破高昌也私取其珍寶將士知之|{
	將即亮翻}
競為盗竊君集不能禁為有司所劾詔下君集等獄|{
	劾戶槩翻又戶得翻下遐嫁翻}
中書侍郎岑文本上疏|{
	上時掌翻}
以為高昌昏迷陛下命君集等討而克之不踰旬日並付大理雖君集等自掛網羅恐海内之人疑陛下唯録其過而遺其功也臣聞命將出師主於克敵|{
	將即亮翻}
苟能克敵雖貪可賞若其敗績雖亷可誅是以漢之李廣利陳湯晉之王濬隋之韓擒虎皆負罪譴人主以其有功咸受封賞|{
	李廣利事見二十卷漢武帝太初四年陳湯事見二十九卷漢元帝竟寧元年王濬事見八十一卷晉武帝太康元年韓擒虎事見一百七十七卷隋文帝開皇九年}
由是觀之將帥之臣亷慎者寡貪求者衆|{
	帥所類翻}
是以黄石公軍勢曰使智使勇使貪使愚故智者樂立其功勇者好行其志貪者急趍其利愚者不計其死伏願録其微勞忘其大過使君集重升朝列|{
	樂音洛好呼到翻趍七喻翻重直龍翻朝直遥翻}
復備驅馳|{
	復扶又翻又音如字}
雖非清貞之臣猶得貪愚之將|{
	將即亮翻}
斯則陛下雖屈法而德彌顯君集等雖蒙宥而過更彰矣上乃釋之又有告薛萬均私通高昌婦女者萬均不服内出高昌婦女付大理與萬均對辯魏徵諫曰臣聞君使臣以禮臣事君以忠|{
	論語載孔子答魯定公之言}
今遣大將軍與亡國婦女對辯帷箔之私實則所得者輕虚則所失者重昔秦穆飲盗馬之士|{
	秦穆公亡馬岐下野人得而共食之者三百人吏逐得欲法之公曰君子不以畜害人吾聞食馬肉不飲酒者傷人乃飲之酒其後穆公伐晉三百人者聞穆公為晉所圍椎鋒爭死以報食馬之德於是穆公獲晉侯以歸飲於禁翻}
楚莊赦絶纓之罪|{
	說苑楚莊王賜羣臣酒日暮酒酣燭滅有引美人之衣者美人援絶其冠纓告王趣火來上視絶纓者王曰賜人酒使醉失禮奈何欲顯婦人之節而辱士乎乃命左右曰今日與寡人飲不絶冠纓者不歡羣臣皆絶去其纓而上火盡歡而罷後晉與楚戰有一臣常在前五合五獲首却敵卒勝之莊王怪問乃夜絶纓者報王也}
况陛下道高堯舜而曾二君之不逮乎上遽釋之侯君集馬病蚛顙行軍總管趙元楷親以指霑其膿而齅之|{
	蚛直衆翻蟲食曰蚛齅許救翻}
御史劾奏其諂左遷栝州刺史|{
	永嘉郡隋開皇九年置處州十三年改曰栝州劾戶槩翻又戶得翻}
高昌之平也諸將皆即受賞行軍總管阿史那社爾以無勑旨獨不受及别勑既下|{
	下遐嫁翻}
乃受之所取唯老弱故弊而已上嘉其亷慎以高昌所得寶刀及雜綵千段賜之 癸卯上獵於樊川|{
	酈道元曰樊川在漢杜縣亦曰樊鄉漢高袓至櫟陽以樊噲灌廢丘功最賜食邑於此鄉因名樊川程大昌曰樊川一名御宿川在萬年縣南三十五里}
乙巳還宫 魏徵上疏以為在朝羣臣當樞機之寄者任之雖重信之未篤|{
	上時掌翻朝直遥翻}
是以人或自疑心懷苟且陛下寛於大事急於小罪臨時責怒未免愛憎夫委大臣以大體責小臣以小事|{
	夫音扶}
為治之道也今委之以職則重大臣而輕小臣至於有事則信小臣而疑大臣信其所輕疑其所重將求致治其可得乎|{
	治直吏翻}
若任以大官求其細過刀筆之吏順旨成風舞文弄法曲成其罪自陳也則以為心不伏辜不言也則以為所犯皆實進退維谷莫能自明|{
	詩桑柔曰進退維谷註谷窮也}
則苟求免禍矯偽成俗矣上納之 上謂侍臣曰朕雖平定天下其守之甚難魏徵對曰臣聞戰勝易守勝難|{
	易以豉翻}
陛下之及此言宗廟社稷之褔也 上聞右庶子張玄素在東宫數諫爭|{
	數所角翻爭讀曰諍}
擢為銀青光禄大夫行左庶子太子嘗於宫中擊鼓玄素叩閤切諫太子出其鼓對玄素毁之太子久不出見官屬玄素諫曰朝廷選俊賢以輔至德今動經時月不見宫臣將何以裨益萬一且宫中唯有婦人不知有能如樊姬者乎|{
	樊姬楚莊王姬也莊王好畋樊姬為不食禽獸之肉鄙笑虞丘子虞丘子愧之進孫叔敖為相莊王以覇}
太子不聼玄素少為刑部令史|{
	少詩照翻}
上嘗對朝臣問之曰|{
	朝直遥翻}
卿在隋何官對曰縣尉又問未為尉時何官對曰流外|{
	按隋之視品即唐之流外銓也宋白曰唐制吏部郎中一人掌考天下之文吏班秩階品一人掌小銓亦分九品通謂之行署以其在九流之外故謂之流外銓亦謂之小選杜佑曰宋齊流外自諸衛録事及五省令史始}
又問何曹玄素耻之出閣殆不能步色如死灰諫議大夫褚遂良上疏|{
	上時掌翻}
以為君能禮其臣乃能盡其力玄素雖出寒微陛下重其才擢至三品翼贊皇儲豈可復對羣臣窮其門戶|{
	復扶又翻}
棄宿昔之恩成一朝之耻使之鬰結于懷何以責其伏節死義乎上曰朕亦悔此問卿疏深會我心遂良亮之子也|{
	褚亮始事薛舉武德中為文學館學士}
孫伏伽與玄素在隋皆為令史伏伽或於廣坐自陳往事一無所隐|{
	史言孫伏伽識度過於張玄素伽求迦翻坐徂臥翻}
戴州刺史賈崇以所部有犯十惡者|{
	十惡之條一曰謀反二曰謀大逆三曰謀叛四曰謀惡逆五曰不道六曰大不敬七曰不孝八曰不睦九曰不義十曰内亂}
御史劾之|{
	劾戶槩翻又戶得翻下同}
上曰昔唐虞大聖貴為天子不能化其子况崇為刺史獨能使其民比屋為善乎|{
	比毗必翻又毗至翻}
若坐是貶黜則州縣互相掩蔽縱捨罪人自今諸州有犯十惡者勿劾刺史但令明加糾察如法施罪庶以肅清姦惡耳 上自臨治兵|{
	治直之翻}
以部陳不整命大將軍張士貴杖中郎將等怒其杖輕下士貴吏|{
	陳讀曰陣下遐嫁翻下同郎將即亮翻}
魏徵諫曰將軍之職為國爪牙使之執杖已非後法况以杖輕下吏乎上亟釋之 言事者多請上親覽表奏以防壅蔽上以問魏徵對曰斯人不知大體必使陛下一一親之豈惟朝堂|{
	朝直遥翻}
州縣之事亦當親之矣

資治通鑑卷一百九十五
