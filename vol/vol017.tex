<!DOCTYPE html PUBLIC "-//W3C//DTD XHTML 1.0 Transitional//EN" "http://www.w3.org/TR/xhtml1/DTD/xhtml1-transitional.dtd">
<html xmlns="http://www.w3.org/1999/xhtml">
<head>
<meta http-equiv="Content-Type" content="text/html; charset=utf-8" />
<meta http-equiv="X-UA-Compatible" content="IE=Edge,chrome=1">
<title>資治通鑒_18-資治通鑑卷十七_18-資治通鑑卷十七</title>
<meta name="Keywords" content="資治通鑒_18-資治通鑑卷十七_18-資治通鑑卷十七">
<meta name="Description" content="資治通鑒_18-資治通鑑卷十七_18-資治通鑑卷十七">
<meta http-equiv="Cache-Control" content="no-transform" />
<meta http-equiv="Cache-Control" content="no-siteapp" />
<link href="/img/style.css" rel="stylesheet" type="text/css" />
<script src="/img/m.js?2020"></script> 
</head>
<body>
 <div class="ClassNavi">
<a  href="/24shi/">二十四史</a> | <a href="/SiKuQuanShu/">四库全书</a> | <a href="http://www.guoxuedashi.com/gjtsjc/"><font  color="#FF0000">古今图书集成</font></a> | <a href="/renwu/">历史人物</a> | <a href="/ShuoWenJieZi/"><font  color="#FF0000">说文解字</a></font> | <a href="/chengyu/">成语词典</a> | <a  target="_blank"  href="http://www.guoxuedashi.com/jgwhj/"><font  color="#FF0000">甲骨文合集</font></a> | <a href="/yzjwjc/"><font  color="#FF0000">殷周金文集成</font></a> | <a href="/xiangxingzi/"><font color="#0000FF">象形字典</font></a> | <a href="/13jing/"><font  color="#FF0000">十三经索引</font></a> | <a href="/zixing/"><font  color="#FF0000">字体转换器</font></a> | <a href="/zidian/xz/"><font color="#0000FF">篆书识别</font></a> | <a href="/jinfanyi/">近义反义词</a> | <a href="/duilian/">对联大全</a> | <a href="/jiapu/"><font  color="#0000FF">家谱族谱查询</font></a> | <a href="http://www.guoxuemi.com/hafo/" target="_blank" ><font color="#FF0000">哈佛古籍</font></a> 
</div>

 <!-- 头部导航开始 -->
<div class="w1180 head clearfix">
  <div class="head_logo l"><a title="国学大师官网" href="http://www.guoxuedashi.com" target="_blank"></a></div>
  <div class="head_sr l">
  <div id="head1">
  
  <a href="http://www.guoxuedashi.com/zidian/bujian/" target="_blank" ><img src="http://www.guoxuedashi.com/img/top1.gif" width="88" height="60" border="0" title="部件查字,支持20万汉字"></a>


<a href="http://www.guoxuedashi.com/help/yingpan.php" target="_blank"><img src="http://www.guoxuedashi.com/img/top230.gif" width="600" height="62" border="0" ></a>


  </div>
  <div id="head3"><a href="javascript:" onClick="javascript:window.external.AddFavorite(window.location.href,document.title);">添加收藏</a>
  <br><a href="/help/setie.php">搜索引擎</a>
  <br><a href="/help/zanzhu.php">赞助本站</a></div>
  <div id="head2">
 <a href="http://www.guoxuemi.com/" target="_blank"><img src="http://www.guoxuedashi.com/img/guoxuemi.gif" width="95" height="62" border="0" style="margin-left:2px;" title="国学迷"></a>
  

  </div>
</div>
  <div class="clear"></div>
  <div class="head_nav">
  <p><a href="/">首页</a> | <a href="/ShuKu/">国学书库</a> | <a href="/guji/">影印古籍</a> | <a href="/shici/">诗词宝典</a> | <a   href="/SiKuQuanShu/gxjx.php">精选</a> <b>|</b> <a href="/zidian/">汉语字典</a> | <a href="/hydcd/">汉语词典</a> | <a href="http://www.guoxuedashi.com/zidian/bujian/"><font  color="#CC0066">部件查字</font></a> | <a href="http://www.sfds.cn/"><font  color="#CC0066">书法大师</font></a> | <a href="/jgwhj/">甲骨文</a> <b>|</b> <a href="/b/4/"><font  color="#CC0066">解密</font></a> | <a href="/renwu/">历史人物</a> | <a href="/diangu/">历史典故</a> | <a href="/xingshi/">姓氏</a> | <a href="/minzu/">民族</a> <b>|</b> <a href="/mz/"><font  color="#CC0066">世界名著</font></a> | <a href="/download/">软件下载</a>
</p>
<p><a href="/b/"><font  color="#CC0066">历史</font></a> | <a href="http://skqs.guoxuedashi.com/" target="_blank">四库全书</a> |  <a href="http://www.guoxuedashi.com/search/" target="_blank"><font  color="#CC0066">全文检索</font></a> | <a href="http://www.guoxuedashi.com/shumu/">古籍书目</a> | <a   href="/24shi/">正史</a> <b>|</b> <a href="/chengyu/">成语词典</a> | <a href="/kangxi/" title="康熙字典">康熙字典</a> | <a href="/ShuoWenJieZi/">说文解字</a> | <a href="/zixing/yanbian/">字形演变</a> | <a href="/yzjwjc/">金 文</a> <b>|</b>  <a href="/shijian/nian-hao/">年号</a> | <a href="/diming/">历史地名</a> | <a href="/shijian/">历史事件</a> | <a href="/guanzhi/">官职</a> | <a href="/lishi/">知识</a> <b>|</b> <a href="/zhongyi/">中医中药</a> | <a href="http://www.guoxuedashi.com/forum/">留言反馈</a>
</p>
  </div>
</div>
<!-- 头部导航END --> 
<!-- 内容区开始 --> 
<div class="w1180 clearfix">
  <div class="info l">
   
<div class="clearfix" style="background:#f5faff;">
<script src='http://www.guoxuedashi.com/img/headersou.js'></script>

</div>
  <div class="info_tree"><a href="http://www.guoxuedashi.com">首页</a> > <a href="/SiKuQuanShu/fanti/">四库全书</a>
 > <h1>资治通鉴</h1> <!--         下载:【右键另存为】即可 --></div>
  <div class="info_content zj clearfix">
  
<div class="info_txt clearfix" id="show">
<center style="font-size:24px;">18-資治通鑑卷十七</center>
    資治通鑑卷十七    宋 司馬光 撰<br />
<br />
  胡三省 音註<br />
<br />
  漢紀九【起重光赤奮若盡強圉恊洽凡七年】<br />
<br />
  世宗孝武皇帝上之上【荀悦曰諱徹之字曰通景帝中子也應劭曰禮諡法威強叡德曰武】<br />
<br />
  建元元年【自古帝王未有年號始起於此貢父曰封禪書云其後三年有司言元宜以天瑞命不宜以一二推所謂其後三年者盖盡元狩六年至元鼎三年也然元鼎四年方得寶鼎又無緣先三年稱之以此而言自元鼎以前之年皆有司所追命其實年號之起在元鼎故元封改元則始有詔書也】冬十月詔舉賢良方正直言極諫之士上親策問以古今治道對者百餘人廣川董仲舒對曰道者所繇適於治之路也【師古曰繇從也適往也治直吏翻繇古由字】仁義禮樂皆其具也故聖王已没而子孫長久安寧數百歲此皆禮樂教化之功也夫人君莫不欲安存而政亂國危者甚衆所任者非其人而所繇者非其道是以政日以仆滅也夫周道衰於幽厲非道亡也幽厲不繇也至於宣王思昔先王之德興滯補敝明文武之功業周道粲然復興【復扶又翻】此夙夜不懈行善之所致也孔子曰人能弘道非道弘人【師古曰論語載孔子之言也言明智之人則能行道内無其質非道所化】故治亂廢興在於己非天降命不可得反其所操持誖謬失其統也【操千高翻下同】為人君者正心以正朝廷正朝廷以正百官正百官以正萬民正萬民以正四方四方正遠近莫敢不壹於正而亡有邪氣奸其間者【奸音干犯也】是以隂陽調而風雨時羣生和而萬民殖諸福之物可致之祥莫不畢至而王道終矣孔子曰鳳鳥不至河不出圖吾己矣夫【論語載孔子之言】自悲可致此物而身卑賤不得致也【師古曰鳳鳥河圖皆王者之瑞仲尼自歎有德無位故不至也】今陛下貴為天子富有四海居得致之位操可致之勢又有能致之資行高而恩厚知明而意美愛民而好士可謂誼主矣【行下孟翻知讀曰智好呼到翻】然而天地未應而美祥莫至者何也凡以教化不立而萬民不正也夫萬民之從利也如水之走下【走音奏】不以教化隄防之不能止也古之王者明於此故南面而治天下【治直之翻】莫不以教化為大務立太學以教於國設庠序以化於邑【學記曰古之教者家有塾黨有庠遂有序國有學也】漸民以仁摩民以義【漸音沾謂浸潤之也摩謂砥厲之也】節民以禮故其刑罰甚輕而禁不犯者教化行而習俗美也聖王之繼亂世也掃除其迹而悉去之【去羌呂翻】復脩教化而崇起之【復扶又翻】教化己明習俗己成子孫循之【師古曰循順也順而行之】行五六百歲尚未敗也秦滅先聖之道為苟且之治故立十四年而亡【自始皇初并天下數之至亡十四年】其遺毒餘烈至今未滅使習俗薄惡人民嚚頑抵冒殊扞熟爛如此之甚者也【文頴曰扞突也師古曰口不道忠信之言為嚚心不則德義之經為頑抵觸也冒犯也殊絶也扞拒也嚚魚巾翻冒如字又莫克翻】竊譬之琴瑟不調甚者必解而更張之乃可鼓也為政而不行甚者必變而更化之乃可理也故漢得天下以來常欲治而至今不可善治者失之於當更化而不更化也【更工衡翻】臣聞聖王之治天下也【自此以下係第二策】少則習之學長則材諸位【謂授之位以試其材少詩沼翻長知兩翻】爵禄以養其德刑罰以威其惡故民曉於禮誼而恥犯其上武王行大誼平殘賊周公作禮樂以文之至於成康之隆囹圄空虚四十餘年【爾雅釋名囹領也圄禦也領録囚徒禁禦也禮記正義崇精問曰獄周曰園土殷曰羑里夏曰均臺囹圄何代之獄焦氏答曰月令秦書則獄名也漢曰若盧魏曰司空是也】此亦教化之漸而仁誼之流非獨傷肌膚之效也【漸子廉翻】至秦則不然師申商之法【申不害商鞅也】行韓非之說憎帝王之道以貪狼為俗【師古曰狼性皆貪故謂貪者為貪狼也】誅名而不察實【師古曰誅責也】為善者不必免而犯惡者未必刑也是以百官皆飾虚辭而不顧實外有事君之禮内有背上之心造偽飾詐趨利無恥【背蒲妹翻趨七喻翻】是以刑者甚衆死者相望而姦不息俗化使然也今陛下并有天下莫不率服而功不加於百姓者殆王心未加焉曾子曰尊其所聞則高明矣行其所知則光大矣高明光大不在於他在乎加之意而已【師古曰曾子之書也曾子曾參】願陛下因用所聞設誠於内而致行之則三王何異哉夫不素養士而欲求賢譬猶不琢玉而求文采也故養士之大者莫大虖太學太學者賢士之所關也【師古曰關由也】教化之本原也今以一郡一國之衆對亡應書者【師古曰書謂舉賢良文學之詔書亡古無字通下同】是王道往往而絶也臣願陛下興太學置明師以養天下之士數考問以盡其材【數所角翻】則英俊宜可得矣今之郡守縣令民之師帥所使承流而宣化也故師帥不賢則主德不宣恩澤不流【帥所類翻】今吏既亡教訓於下或不承用主上之灋暴虐百姓與姦為市【師古曰言小吏有為姦欺者守令不舉乃反與交易求利也】貧窮孤弱寃苦失職甚不稱陛下之意是以隂陽錯繆氛氣充塞【稱尺證翻塞悉則翻】羣生寡遂黎民未濟皆長吏不明使至於此也夫長吏多出於郎中中郎吏二千石子弟選郎吏又以富訾未必賢也【長知兩翻訾讀曰貲】且古所謂功者以任官稱職為差非謂積日累久也故小材雖累日不離於小官賢材雖未久不害為輔佐【師古曰害猶妨也離力智翻】是以有司竭力盡知務治其業而以赴功【知讀曰智治直之翻】今則不然累日以取貴積久以致官是以廉恥貿亂賢不肖渾殽未得其真【貿音茂渾戶本翻】臣愚以為使諸列侯郡守二千石各擇其吏民之賢者歲貢各二人以給宿衛且以觀大臣之能所貢賢者有賞所貢不肖者有罰夫如是諸吏二千石皆盡心於求賢天下之士可得而官使也【授之以官而任使之】徧得天下之賢人則三王之盛易為【易以豉翻】而堯舜之名可及也毋以日月為功實試賢能為上量材而授官録德而定位【量音良師古曰録謂存視也】則廉恥殊路賢不肖異處矣臣聞衆少成多積小致鉅【自此以下係第三策師古曰鉅大也】故聖人莫不以晻致明【晻古暗字】以微致顯是以堯發於諸侯舜興虖深山【師古曰堯謂從唐侯升天子之位孟康曰舜耕於歷山】非一日而顯也盖有漸以致之矣言出於己不可塞也行發於身不可掩也言行治之大者君子之所以動天地也【塞悉則翻行下孟翻】故盡小者大慎微者著【師古曰能盡衆小則致高大能謹於微則其善著明也】積善在身猶長日加益而人不知也【師古曰長言身形之脩短自幼及壯也】積惡在身猶火銷膏而人不見也此唐虞之所以得令名而桀紂之可為悼懼者也夫樂而不亂復而不厭者謂之道【樂音洛師古曰復謂反覆行之也音扶目翻】道者萬世亡敝敝者道之失也【師古曰言有敝非道由失道故有敝亡古無字通下同】先王之道必有偏而不起之處故政有眊而不行【眊莫報翻不明也】舉其偏者以補其敝而已矣三王之道所祖不同非其相反將以捄溢扶衰所遭之變然也【捄與救同】故孔子曰無為而治者其舜乎改正朔易服色以順天命而已其餘盡循堯道何更為哉【更工衡翻】故王者有改制之名亡變道之實然夏尚忠殷尚敬周尚文者所繼之捄當用此也【師古曰繼謂所受先代之次也捄謂救其敝也】孔子曰殷因於夏禮所損益可知也周因於殷禮所損益可知也其或繼周者雖百世可知也【師古曰論語載孔子之言謂忠敬與文因循為教立政垂則不遠此也】此言百王之用以此三者矣夏因於虞而獨不言所損益者其道一而所上同也道之大原出于天天不變道亦不變是以禹繼舜舜繼堯三聖相受而守一道亡捄敝之政也【師古曰言政和平不須救弊也】故不言其所損益也繇是觀之繼治世者其道同繼亂世者其道變今漢繼大亂之後若宜少損周之文致【師古曰致至極也貢父曰致當屬下句少詩沼翻】用夏之忠者夫古之天下亦今之天下共是天下以古凖今壹何不相逮之遠也安所繆盭而陵夷若是【盭古戾字師古曰安焉也】意者有所失於古之道與有所詭於天之理與【詭違也異也與歟同】夫天亦有所分予予之齒者去其角傅其翼者兩其足【師古曰謂牛無上齒則有角其餘無角者則有上齒傅著也言鳥不四足分扶問翻予讀曰與去羌呂翻傅讀曰附】是所受大者不得取小也古之所予禄者不食於力不動於末【師古曰末謂工商之業】是亦受大者不得取小與天同意者也夫己受大又取小天不能足而况人虖此民之所以囂囂苦不足也【囂音敖囂囂衆怨愁聲也】身寵而載高位【載乘也】家温而食厚禄因乘富貴之資力以與民爭利於下民安能如之哉民日削月朘【孟康曰朘音揎謂轉踧也蘇林曰朘音攜石俗語謂朒為朘縮師古曰孟說是也揎音宣踧音子六翻】寖以大窮富者奢侈羨溢【羨饒也讀與衍同音弋戰翻】貧者窮急愁苦民不樂生安能避罪此刑罰之所以蕃【樂音洛師古曰蕃多也音扶元翻】而姦邪不可勝者也天子大夫者下民之所視效遠方之所四面而内望也近者視而放之【師古曰放依也音甫往翻】遠者望而效之豈可以居賢人之位而為庶人行哉【行下孟翻下同】夫皇皇求財利常恐乏匱者庶人之意也皇皇求仁義常恐不能化民者大夫之意也【皇皇急速也】易曰負且乘致寇至【此易解卦六三之辭也】乘車者君子之位也負擔者小人之事也此言居君子之位而為庶人之行者患禍必至也若居君子之位當君子之行則舍公儀休之相魯無可為者矣【公儀休相魯之其家見織帛怒而出其妻食於舍而茹葵愠而拔其葵曰吾巳食祿而奪園夫紅女利乎舍讀曰捨言為君子者當如公儀休若廢而不遵則無可為者矣】春秋大一統者天地之常經古今之通誼也【師古曰一統者萬物之統皆歸於一也春秋公羊傳隱公元年春王正月何言乎王正月大一統也此言諸侯皆繫統天子不得自專也】今師異道人異論百家殊方指意不同是以上無以持一統法制數變下不知所守【數所角翻】臣愚以為諸不在六藝之科孔子之術者皆絶其道勿使並進邪辟之說滅息【辟讀曰僻】然後統紀可一而法度可明民知所從矣天子善其對以仲舒為江都相會稽莊助亦以賢良對策【漢書作嚴助盖明帝諱莊避之也會工外翻】天子擢為中大夫【按考異曰漢書武紀元光元年五月詔舉賢良董仲舒公孫弘出焉仲舒傳曰仲舒對册推明孔氏抑黜百家立學校之官州縣舉茂材孝廉皆自仲舒發之今舉孝亷在元光元年十一月若對策在下五月則不得云自仲舒發之盖武紀誤也然仲舒對策不知果在何時元光元年以前唯今年舉賢良見於紀三年閩越東甌相攻莊助已為中大夫故皆著之於此仲舒傳又云遼東高廟長陵高園災仲舒推說其意主父偃竊其書奏之仲舒由是得辠按二災在建元六年主父偃傳上書召見在元光元年盖仲舒追述二災而作書或作書不上而偃後來方見其草藁也】丞相衛綰奏所舉賢良或治申韓蘇張之言亂國政者請皆罷奏可董仲舒少治春秋【冶直之翻少詩照翻】孝景時為博士進退容止非禮不行學者皆師尊之及為江都相事易王【江都易王非景帝子帝之兄也諡法好更故舊曰易音亦】易王帝兄素驕好勇【好呼到翻下同】仲舒以禮匡正王敬重焉 春二月赦 行三銖錢【師古曰新壞四銖錢造此錢也重如其文】夏六月丞相衛綰免丙寅以魏其侯竇嬰為丞相武<br />
<br />
  安侯田蚡為太尉上雅嚮儒術嬰蚡俱好儒推轂代趙綰為御史大夫蘭陵王臧為郎中令【謂薦進賢者若推車轂然主於進也推吐雷翻轂古禄翻班志代縣屬代郡蘭陵縣屬東海郡】綰請立明堂以朝諸侯【王者之堂所以正四時出教化自秦滅先王之禮其制不存朝直遥翻下同】且薦其師申公秋天子使使束帛加璧安車駟馬以迎申公【古者高車立乘安車坐乘據申公傳安車以蒲裹輪孔頴達曰安車若今小車者古者乘四馬之車立乘既老故乘一馬小車坐乘也余按孔氏所謂小車乃古之大夫致事者適四方所乘私車也今加禮申公迎以駟馬安車非小車也】既至見天子天子問治亂之事申公年八十餘對曰為治者不在多言顧力行何如耳【治直吏翻】是時天子方好文詞見申公對默然然己招致則以為太中大夫舍魯邸議明堂巡狩改歷服色事【漢制郡國皆立邸於京師申公魯人故舍魯邸】 是歲内史甯成抵罪髠鉗<br />
<br />
  二年冬十月淮南王安來朝上以安屬為諸父而材高甚尊重之【安淮南王長之子長於文帝為弟安於景帝為從弟於帝為諸父行】每宴見談語昏暮然後罷【見賢遍翻】安雅善武安侯田蚡【雅素也】其入朝武安侯迎之覇上與語曰上無太子王親高皇帝孫行仁義天下莫不聞宫車一日晏駕非王尚誰立者安大喜厚遺蚡金錢財物【遺于季翻】 太皇竇太后好黄老言不悦儒術趙綰請毋奏事東宫【漢長樂宫在東太后居之故謂之東宫亦謂之東朝】竇太后大怒曰此欲復為新垣平邪【事見十五卷文帝十六年復扶又翻】隂求得趙綰王臧姦利事以讓上上因廢明堂事諸所興為皆廢下綰臧吏皆自殺【下遐嫁翻】丞相嬰太尉蚡免申公亦以疾免歸初景帝以太子太傅石奮及四子皆二千石乃集其門號奮為萬石君【石姓衛大夫石碏之後師古曰集合也凡最計也摠合其一門之計五人為二千石故號萬石君】萬石君無文學而恭謹無與比子孫為小吏來歸謁萬石君必朝服見之不名【朝直遥翻】子孫有過失不責讓為便坐【師古曰便坐於便側之處非正室也坐徂臥翻】對案不食然後諸子相責因長老肉袒謝罪改之乃許子孫勝冠者在側【勝音升】雖燕居必冠其執喪哀戚甚悼子孫遵教皆以孝謹聞乎郡國【聞音問】及趙綰王臧以文學獲罪竇太后以為儒者文多質少【少詩沼翻】今萬石君家不言而躬行乃以其長子建為郎中令少子慶為内史建在上側事有可言屏人恣言極切至廷見如不能言者【謂事有當諫正者廷見謂於百官正朝畢集之時屛必郢翻見賢遍翻】上以是親之慶嘗為太僕御出【為上御車而出 考異曰按百官公卿表慶不為太僕盖嘗攝職也】上問車中幾馬慶以策數馬畢舉手曰六馬慶於諸子中最為簡易矣【易以豉翻】竇嬰田蚡既免以侯家居蚡雖不任職以王太后故親幸數言事多效【謂言事多見聽用數所角翻】士吏趨勢利者【趨七喻翻】皆去嬰而歸蚡蚡日益横【為嬰蚡交愬張本横戶孟翻】春二月丙戌朔日有食之 三月乙未以太常栢至侯許昌為丞相【昌高祖功臣許盎之孫栢至地闕】 初堂邑侯陳午尚帝姑館陶公主嫖帝之為太子公主有力焉【班志堂邑縣屬臨淮郡陳午高祖功臣陳嬰之孫館陶縣屬魏郡公主援上為太子事見上卷景帝前七年】以其女為太子妃及即位妃為皇后竇太主恃功求請無厭【厭於鹽翻】上患之皇后驕妬擅寵而無子與醫錢凡九千萬欲以求子然卒無之【卒子恤翻】后寵浸衰皇太后謂上曰汝新即位大臣未服先為明堂太皇太后已怒今又忤長主【忤五故翻長知兩翻】必重得罪【重直用翻】婦人性易悦耳【易以豉翻】宜深慎之上乃於長主皇后復稍加恩禮【復扶又翻】上袚覇上【孟康曰袚除也於霸水上自袚除今之上已袚禊也袚音廢又音拂】還過上姊平陽公主【班志平陽縣屬河東郡公主景帝女降平陽侯曹夀】悦謳者衛子夫【師古曰齊歌曰謳一侯翻】子夫母衛媪平陽公主家僮也【師古曰僮者婢妾之總稱媪者年老之號非當時所呼也衛者舉夫家姓媪烏皓翻】主因奉送子夫乃入宫恩寵日隆陳皇后聞之恚幾死者數矣【恚於避翻愠怒也幾居衣翻數所角翻】上愈怒子夫同母弟衛青其父鄭季本平陽縣吏給事侯家【師古曰縣遣於侯家供事也】與衛媪私通而生青冒姓衛氏【冒姓者青本鄭氏子而冒衛姓也】青長為侯家騎奴大長公主執囚青【大長公主即館陶公主也長知兩翻騎奇寄翻下同】欲殺之其友騎郎公孫敖與壯士簒取之【郎之騎從者郎中有車戶騎三將逆取曰簒】上聞乃召青為建章監侍中【建章宫監據史太初元年起建章宫盖因舊宫而大起也青時為建章監而兼侍中】賞賜數日間累千金既而以子夫為夫人青為太中大夫 夏四月有星如日夜出 初置茂陵邑【班志茂陵邑屬扶風黄圖曰本槐里之茂鄉武帝起陵邑在長安西北八十里】 時大臣議者多寃鼂錯之策【鼂錯事見上卷景帝前三年】務摧抑諸侯王數奏暴其過惡吹毛求疵【謂暴露其過惡數所角翻疵才斯翻病也瑕也】笞服其臣使證其君諸侯王莫不悲怨<br />
<br />
  三年冬十月代王登長沙王發中山王勝濟川王明來朝【代王登王參之子文帝之孫長沙中山王皆景帝子濟川王梁孝王之子濟子禮翻】上置酒勝聞樂聲而泣上問其故對曰悲者不可為累欷思者不可為歎息【累重也欷歔欷也悲思之積於心聞欷歎之聲則其悲思益甚累力癸翻欷許既翻】今臣心結日久每聞幼眇之聲【幼一笑翻眇音妙精微也】不知涕泣之横集也臣得蒙肺附為東藩屬又稱兄【肺附一作肺腑史記正義曰顔師古曰舊解云肺附如肝肺之相附著也一說肺碎木札也喻其輕薄附著大材按顔此說並是疎繆又改腑為附就其義重疎繆矣八十一難云寸口者脉之大會手太隂之動脉也呂廣云太隂肺之脉也肺為諸藏之主通隂陽故十二經脉皆會于太隂所以決吉凶者十二經有病皆于寸口知其何經之動浮沉濇滑春秋逆順知其死生顧野王曰肺腑腹心也余謂史若從肺附則顔說為是若從肺腑則依正義勝王中山在關東故曰東藩以親屬言則勝於帝兄也泣亦淚也】今羣臣非有葭莩之親鴻毛之重【張晏曰葭蘆葉也莩葉裏白皮皆取喻於輕薄也師古曰葭蘆也莩者其筩中白皮至薄者也葭莩喻著鴻毛喻輕薄甚莩音孚】羣居黨議朋友相為使夫宗室擯卻【擯郤斥退也擯必刃翻郤丘畧翻】骨肉氷釋臣竊傷之具以吏所侵聞於是上乃厚諸侯之禮省有司所奏諸侯事加親親之恩焉【省悉井翻】 河水溢于平原【平原本齊地高祖置郡禹疏九河皆在平原勃海郡界】大饑人相食 秋七月有星孛于西北【孛蒲内翻】 濟川王明坐殺中傅【濟川王明梁孝王子應劭曰中傅宦者也漢諸王國有太傅秩二千石掌傅王以德義中傅出入王宫在王左右亦主傅教導王梁王傳作中尉此從帝紀】廢遷房陵【班志房陵縣屬漢中郡】 七國之敗也【事見上卷景帝前三年】吳王子駒亡走閩越怨東甌殺其父常勸閩越擊東甌閩粤從之發兵圍東甌東甌使人告急天子天子問田蚡蚡對曰越人相攻擊固其常又數反覆【數所角翻下同】自秦時棄不屬【不屬不臣屬也】不足以煩中國往救也莊助曰【莊姓也戰國時楚有莊周趙有莊豹】特患力不能救德不能覆【覆敷又翻】誠能何故棄之且秦舉咸陽而棄之【師古曰舉揔也言總天下乃至京師皆棄之】何但越也今小國以窮困來吿急天子不救尚安所愬又何以子萬國乎上曰太尉不足與計 【考異曰史記東越漢書嚴助傳皆云建元三年閩越圍東甌天子問太尉田蚡按是時蚡不為太尉云太尉誤也下云太尉不足與計盖追呼其官或亦誤耳】吾新即位不欲出虎符發兵郡國乃遣助以節發兵會稽【會稽東南邊越】會稽守欲距灋不為發【以法距之為無漢虎符驗會工外翻守式又翻為于偽翻】助乃斬一司馬諭意指【謂曉喻以天子不欲出虎符之意】遂發兵浮海救東甌未至閩越引兵罷東甌請舉國内徙乃悉舉其衆來處於江淮之間【處昌呂翻】 九月丙子晦日有食之 上自初即位招選天下文學材智之士待以不次之位【師古曰不拘常次言超擢之】四方士多上書言得失自眩鬻者以千數【漢書作衒行賣也鬻亦賣也衒與眩同音州縣之縣又工縣翻鬻音育】上簡拔其俊異者寵用之莊助最先進後又得吳人朱買臣趙人吾丘夀王【姓譜吾音虞即虞丘氏史記有楚相虞丘子】蜀人司馬相如平原東方朔【風俗通曰伏羲之後帝出乎震主東方子孫為東方氏】吳人枚臯濟南終軍等【姓譜枚姓出於周官銜枚氏其後以官為姓風俗通六國有賢人枚被終姓出於顓頊裔孫陸終濟子禮翻】並在左右每令與大臣辨論中外相應以義理之文大臣數屈焉然相如特以辭賦得幸朔臯不根持論好詼諧【言其議論無所根據好呼到翻詼古囘翻李奇曰詼嘲也】上以俳優畜之【師古曰俳雜戲也優調戲也左傳曰少相狎長相優俳優即今伶人調戲者】雖數賞賜終不任以事也【數所角翻】朔亦觀上顔色時時直諫有所補益是歲上始為微行北至池陽西至黄山【班志池陽縣屬馮翊黄山宫名在扶風槐里縣】南獵長楊東游宜春【長陽宫名水經注云槐里縣東有漏水出南山赤谷東北逕長楊宫宫有長楊因名其地在盭厔界師古曰宜春宫也在長安東南說者乃以為在鄠非也在鄠者自是宜春觀在長安城西非東游也】與左右能騎射者期諸殿門【期門之號始此】常以夜出自稱平陽侯【平陽侯曹夀尚帝姊見尊寵故稱之】旦明入南山下射鹿豕狐兔【終南山横亘關中南面西起秦隴東徹藍田凡雍岐郿鄠長安萬年相去且八百里而連綿峙據其南者皆此一山也射而亦翻】馳騖禾稼之地民皆號呼罵詈【號戶高翻】鄠杜令欲執之【班志鄠縣屬扶風杜縣屬京兆宣帝更為杜陵鄠音戶】示以乘輿物乃得免【乘繩證翻】又嘗夜至栢谷【水經河水逕湖縣故城北又東合栢谷水經注云水出弘農縣西石隄山北逕栢谷亭下即帝微行處】投逆旅宿就逆旅主人求漿主人翁曰無漿正有溺耳【溺奴吊翻】且疑上為姦盜聚少年欲攻之主人嫗睹上狀貌而異之【嫗威遇翻】止其翁曰客非常人也且又有備不可圖也翁不聽嫗飲翁以酒醉而縛之【飲於禁翻】少年皆散走嫗乃殺雞為食以謝客明日上歸召嫗賜金千斤拜其夫為羽林郎【羽林郎屬郎中令師古曰羽林宿衛之官言如羽之疾如林之多也一說曰羽所以為王者羽翼】後乃私置更衣從宣曲以南十二所夜投宿長楊五柞等諸宫【師古曰為休息更衣之處宣曲宫名在昆明池西五柞宫名水經注在盩庢縣長楊宫東北更工衡翻柞昨作二音】上以道遠勞苦又為百姓所患乃使太中大夫吾丘夀王舉籍阿城以南盩厔以東宜春以西提封頃畮及其賈直【師古曰舉計其數以為簿籍也阿城本秦阿房宫以其墻壁崇廣故俗呼為阿城盩厔屬扶風山曲曰盩水曲曰厔杜佑曰盩厔唐為宜夀縣提封亦謂提舉四封之内總計其大數也盩音輈厔音窒賈讀曰價】欲除以為上林苑屬之南山又詔中尉左右内史【師古曰時未為京兆扶風馮翊故云中尉及左右内史也予據班表帝後改右内史為京兆尹左内史為左馮翊主爵都尉為右扶風是為三輔屬之欲翻】表屬縣草田【草田荒田之未耕墾者】欲以償鄠杜之民夀王奏事上大說稱善【說讀曰悦】時東方朔在傍進諫曰夫南山天下之阻也漢興去三河之地【河南河内河東為三河漢高帝始居洛陽後西都關中是去三河之地也】止覇滻以西都涇渭之南此所謂天下陸海之地【覇水出藍田縣藍田谷滻水亦出藍田谷逕藍田川北出霸陵入霸水霸又北入于渭涇水注見六卷渭水出隴西首陽縣西南鳥鼠同宂山東流與霸水涇水合又東至船司空入河陸海師古曰高平曰陸關中地高故稱之耳海者萬物所出言關中陸產饒富是以謂之陸海也】秦之所以虜西戎兼山東者也其山出玉石金銀銅鐵良材百工所取給萬民所卬足也【卬古仰字通用音牛向翻】又有秔稻棃栗桑麻竹箭之饒土宜薑芋水多䵷魚【芋即蹲䲭也其葉似藕荷而長不圓其根大者為芋魁其小者附麗甚衆白膩可食䵷與蛙同師古曰似蝦蟆而小長脚】貧者得以人給家足無飢寒之憂故酆鎬之間號為土膏【周文王都酆武王都鎬水經渭水東過槐里縣故城南東合甘水又東豐水從南來注之又東北與鎬水合班志豐水出鄠縣東南鎬水上承鎬池水於昆明池北皆在上林苑中】其賈畮一金【賈與價同】今規以為苑絶陂池水澤之利而取民膏腴之地上乏國家之用下奪農桑之業是其不可一也盛荆棘之林廣狐菟之苑【菟古兔字通用】大虎狼之虚壞人塜墓【虚讀曰墟壞音怪】發人室廬令幼弱懷土而思耆老泣涕而悲是其不可二也【賀瑒曰耆至也至老之境也】斥而營之垣而囿之騎馳東西車騖南北【師古曰亂馳曰騖】有深溝大渠夫一日之樂不足以危無隄之輿【蘇林曰隄限也輿乘輿也無限若言不訾也不敢斥天子故曰輿也張晏曰一日之樂謂田獵也無隄之輿謂天子富貴無隄限貢父曰不足以危不字當作亦隄亦防也言車輿馳騁不為防慮必有顛蹷之變樂音洛】是其不可三也夫殷作九市之宫而諸侯畔【應劭曰紂於宫中設九市】靈王起章華之臺而楚民散【師古曰楚靈王作章華之臺納亡人以實之卒有乾谿之禍也章華臺在華容城也】秦興阿房之殿而天下亂糞土愚臣逆盛意罪當萬死上乃拜朔為太中大夫給事中【百官表給事中加官師古曰漢官解詁云掌侍從左右無員常侍中續漢志給事中關通内外盖以給事禁中名官也】賜黄金百斤然遂起上林苑如夀王所奏上又好自擊熊豕【說文熊似豕山居冬蟄春出詩疏熊能攀緣上樹見人則顛倒投地而下豕謂野豕也生一歲為豵二歲為豜二獸皆能突人】馳逐野獸司馬相如上疏諫曰臣聞物有同類而殊能者故力稱烏獲捷言慶忌勇期賁育【烏獲秦武王力士也慶忌吳王僚之子射能捷矢也孟賁古之勇士水行不避蛟龍陸行不避豺狼發怒吐氣聲響動天夏育亦猛士也賁音奔】臣之愚竊以為人誠有之獸亦宜然今陛下好陵阻險射猛獸卒然遇逸材之獸駭不存之地【師古曰不存不可得安存也貢父曰不存猶言不虞下文云存變之意射而亦翻卒讀曰猝】犯屬車之清塵【屬車註見十三卷師古曰屬者言聨屬不絶也塵謂行而起塵也言清者尊貴之意也說者乃以清塵為清道灑塵非也】輿不及還轅人不暇施巧雖有烏獲逢蒙之技【逢蒙古之善射者也孟子曰逢蒙學射於羿逢皮江翻】不得用【宜承上文為句】枯木朽株盡為難矣是胡越起於轂下而羌夷接軫也【軫後車横木也】豈不殆哉雖萬全而無患然本非天子之所宜近也【近其靳翻】且夫清道而後行中路而馳猶時有銜橜之變【張揖曰銜馬勒銜也橜騑馬口長銜也師古曰橜謂車之鈎心也銜橜之變言馬銜或斷鈎心或出則致傾敗以傷人也橜距月翻】况乎涉豐草【豐草茂草也】騁丘墟前有利獸之樂【虚讀曰墟樂音洛下同】而内無存變之意其為害也不難矣夫輕萬乘之重不以為安樂出萬有一危之塗以為娛臣竊為陛下不取【為于偽翻】盖明者遠見於未萌【師古曰萌謂事始若草木初生者也】而知者避危於無形【知讀曰智】旤固多藏於隱微而發於人之所忽者也故鄙諺曰家累千金坐不垂堂【張揖曰畏櫩瓦墮中人也師古曰垂堂者近堂邊外自恐墜墮耳非畏櫩瓦也言富人之子則自愛深矣】此言雖小可以諭大上善之 【考異曰此多非今年事因莊助救東甌及微行始出終言之】<br />
<br />
  四年夏有風赤如血 六月旱 秋九月有星孛于東北【孛蒲内翻】 是歲南越王佗死【佗徒河翻】其孫文王胡立五年春罷三銖錢行半兩錢【建元元年行三銖錢至是而罷又新鑄半兩錢】置五經博士 夏五月大蝗 秋八月廣川惠王越清河哀王乘皆薨無後國除【二王皆景帝子越中二年四月受封乘中三年三月受封至是國除】<br />
<br />
  六年春二月乙未遼東高廟災【景帝令郡國各立高祖廟故遼東有高廟】夏四月壬子高園便殿火上素服五日【師古曰凡言便殿便室便坐者皆非正大之處所以就便安也園者於陵上作之既有正寢以象平生又立便殿為休息閒宴之處耳便如字沈約曰漢氏諸陵皆有園寢承秦所為也說者以為古前廟後寢以象人主前有朝後有寢也廟以藏主四時祭祀寢有衣冠象生之具以薦新秦始出寢起於墓側漢因不改及魏武帝葬高陵有司依漢立陵上祭殿文帝以為古不墓祭皆設於廟高陵上殿屋皆毁壞車馬還廐衣服藏府文帝自作終制又曰夀陵無立寢殿造園邑自是至今陵寢遂絶】 五月丁亥太皇太后崩【孝文皇后竇氏也】 六月癸巳丞相昌免【許昌也】武安侯田蚡為丞相蚡驕侈治宅甲諸第田園極膏腴【師古曰甲諸第者言為諸第之最也以甲乙之次言甲則為上矣膏腴謂肥厚之處治直之翻】市買郡縣物相屬於道多受四方賂遺其家金玉婦女狗馬聲樂玩好不可勝數【屬之欲翻遺于季翻好呼到翻勝音升】每入奏事坐語移日所言皆聽薦人或起家至二千石權移主上上乃曰君除吏已盡未吾亦欲除吏嘗請考工地益宅【考工少府之屬官也主作器械】上怒曰君何不遂取武庫是後乃稍退 秋八月有星孛于東方長竟天【孛蒲内翻】 閩越王郢興兵擊南越邊邑南越王守天子約不敢擅興兵使人上書吿天子於是天子多南越義大為發兵【為于偽翻下同】遣大行王恢出豫章大農令韓安國出會稽【大農令本秦之治粟内史也漢初因之景帝中六年更名大農令帝太初元年更名大司農】擊閩越淮南王安上書諫曰陛下臨天下布德施惠天下攝然【孟康曰攝安也奴恊翻】人安其生自以没身不見兵革今聞有司舉兵將以誅越臣安竊為陛下重之【師古曰重難也】越方外之地剪髮文身之民也【晉灼曰淮南之越又髮張揖以為古剪字師古曰與剪同張說是也應劭曰越人常在水中故斷其髮文其身以象龍子故不見傷害】不可以冠帶之國法度理也自三代之盛胡越不與受正朔【與讀曰預】非彊勿能服威弗能制也以為不居之地不牧之民不足以煩中國也【師古曰地不可居而民不可牧養也】自漢初定以來七十二年越人相攻擊者不可勝數【勝音升】然天子未嘗舉兵而入其地也臣聞越非有城郭邑里也處谿谷之間篁竹之中【處昌呂翻服䖍曰竹叢曰篁師古曰竹田曰篁音皇】習於水鬭便於用舟地深昧而多水險【昧暗也言多草木也】中國之人不知其埶阻而入其地雖百不當其一得其地不可郡縣也攻之不可暴取也以地圖察其山川要塞相去不過寸數而間獨數百千里【師古曰間中間也或八九百里或千里也】險阻林叢弗能盡著【師古曰不能盡載于地圖也著竹助翻】視之若易行之甚難天下賴宗廟之靈方内大寧戴白之老【師古曰言白髪在首】不見兵革民得夫婦相守父子相保陛下之德也越人名為藩臣貢酎之奉不輸大内【貢者以土產之物來貢也酎者三重釀醇酒也味厚故以薦宗廟也漢制於正月旦作酒八月成曰酎酎之言純也八月嘗酎於太廟諸侯王各出金助祭所謂酎金也大内都内也國之寶藏班表治粟屬官有都内令丞言越國僻遠既不輸土貢又不輸酎金於中國得其地無益也酎直又翻】一卒之奉不給上事【給供也】自相攻擊而陛下發兵救之是反以中國而勞蠻夷也【師古曰疲勞中國之人于蠻夷之地】且越人愚戇輕薄【戇陟降翻】負約反覆其不用天子之法度非一日之積也【師古曰積久也】壹不奉詔舉兵誅之臣恐後兵革無時得息也間者數年歲比不登民待賣爵贅子以接衣食【比毗至翻如淳曰淮南俗賣子與人作奴婢名曰贅子三年不能贖遂為奴婢師古曰贅質也一說云贅子者謂令子出就婦家為贅壻】賴陛下德澤振救之得毋轉死溝壑四年不登五年復蝗民生未復【年復扶又翻未復如字】今發兵行數千里資衣糧【師古曰資猶齎也】入越地輿轎而隃領拕舟而入水【轎竹輿車江南人又謂之籃輿領山嶺也不通舟車故用肩輿以行轎旗妙翻拕音它曳也】行數百千里夾以深林叢竹水道上下擊石【謂水道多巨石船行上下皆與石相撃觸也】林中多蝮蛇猛獸【應劭曰蝮蛇一名虺蠚螫也螫人手足則割去其肉不然則死師古曰爾雅及說文皆以為蝮即虺也博三寸首大如臂而郭璞曰各自一種蛇其蝮蛇大頭細頸焦尾色如綬文文間有毛似猪鬛鼻上有針大者七八尺一名反鼻非虺之類也以今俗名證之郭說得矣虺若土色所在有之俗呼土虺其蝮惟出南方蝮敷福翻】夏月暑時歐泄霍亂之病相隨屬也【歐吐也泄利也師古曰泄以制翻屬之欲翻】曾未施兵接刃死傷者必衆矣前時南海王反陛下先臣使將軍間忌將兵擊之【文頴曰先臣淮南厲王長也間忌人姓名也師古曰淮南王傳作簡忌此本作閭傳寫字誤省耳左傳有魯大夫簡叔】以其軍降處之上淦【班志豫章郡有新淦縣應劭註云淦水所出上淦盖淦水之上流也降戶江翻處昌呂翻淦音紺又工含翻】後復反會天暑多雨樓船卒水居擊棹【師古曰言常居舟中水上而又有擊棹行舟之役故多死也復扶又翻】未戰而疾死者過半親老涕泣孤子啼號【號戶高翻】破家散業迎尸千里之外裹骸骨而歸悲哀之氣數年不息長老至今以為記曾未入其地而禍已至此矣【曾才登翻】陛下德配天地明象日月恩至禽獸澤及草木一人有飢寒不終其天年而死者為之悽愴於心今方内無狗吠之警而使陛下甲卒死亡暴露中原霑漬山谷邊境之民為之早閉晏開【師古曰晏晚也言有兵難故邊城早閉而晚開也為于偽翻下同】朝不及夕【師古曰言憂危亡不自保也】臣安竊為陛下重之不習南方地形者多以越為人衆兵彊能難邊城【服䖍曰為邊城作難也難乃旦翻】淮南全國之時多為邊吏【師古曰全國謂未分為三之時也淮南人於邊為吏與越接境故知其地形也】臣竊聞之與中國異【師古曰言其風土不同】限以高山人迹絶車道不通天地所以隔外内也其入中國必下領水領水之山峭峻漂石破舟【領水即贛水也班志所謂彭水出豫章南壄縣東入湖漢水庾仲初所謂大庾墧水北入豫章注於江者是也漂石破舟言三百里贛石】不可以大船載食糧下也越人欲為變必先田餘干界中【班志豫章郡有餘汗縣應劭曰汗音干今饒州餘干縣漢古縣名也】積食糧乃入伐材治船【治直之翻】邊城守候誠謹越人有入伐材者輒收捕焚其積聚【積子賜翻聚慈諭翻】雖百越柰邊城何且越人緜力薄材【師古曰緜弱也言其柔弱如緜】不能陸戰又無車騎弓弩之用然而不可入者以保地險而中國之人不耐其水土也臣聞越甲卒不下數十萬所以入之五倍乃足【師古曰不下言不減也漢軍多之五倍然後可入其地也】輓車奉餉者不在其中【師古曰輓音晚引車也】南方暑濕近夏癉熱【近其靳翻師古曰癉黄病也丁幹翻】暴露水居蝮蛇蠚生【師古曰蠚毒也音壑】疢疾多作【疢丑刃翻病也】兵未血刃而病死者什二三雖舉越國而虜之不足以償所亡【師古曰舉謂摠取也】臣聞道路言閩越王弟甲弑而殺之甲以誅死【甲者閩越王弟之名】其民未有所屬陛下若欲來内處之中國【處昌呂翻】使重臣臨存【師古曰存謂省問之】施德垂賞以招致之此必攜幼扶老以歸聖德若陛下無所用之則繼其絶世存其亡國建其王侯以為畜越【李奇曰如人畜養六畜也師古曰直謂之畜養之耳非六畜也畜許六翻】此必委質為藩臣世共貢職【共讀曰供】陛下以方寸之印丈二之組填撫方外不勞一卒不頓一戟【師古曰頓壞也讀曰鈍】而威德並行今以兵入其地此必震恐以有司為欲屠滅之也必雉兔逃入山林險阻【師古曰如雉兔之逃竄而入山林險阻之中】背而去之則復相羣聚留而守之歷歲經年則士卒罷勌【背蒲妹翻罷讀曰疲勌即倦字】食糧乏絶民苦兵事盜賊必起臣聞長老言秦之時嘗使尉屠睢擊越【張晏曰郡都尉姓屠名睢晉有屠岸賈屠蒯睢音雖】又使監禄鑿渠通道【張晏曰監郡御史也名禄案秦郡置守尉監】越人逃入深山林叢不可得攻留軍屯守空地曠日引久士卒勞勌越出擊之秦兵大敗乃發適戍以備之【適讀曰讁】當此之時外内騷動皆不聊生亡逃相從羣為盜賊於是山東之難始興【難乃旦翻】兵者凶事一方有急四面皆聳臣恐變故之生姦邪之作由此始也臣聞天子之兵有征而無戰言莫敢校也【師古曰校計也不敢與計強弱曲直】如使越人蒙徼幸以逆執事之顔行【徼工堯翻文穎曰顔行猶鴈行在前行故曰顔也行戶剛翻】厮輿之卒有一不備而歸者【張晏曰厮微輿衆也師古曰厮析薪者輿主駕車者皆言賤役之人也】雖得越王之首臣猶竊為大漢羞之【為于偽翻】陛下以四海為境生民之屬皆為臣妾垂德惠以覆露之【覆謂盖幬也露謂使之霑潤澤也覆敷又翻】使安生樂業則澤被萬世【樂音洛被皮義翻】傳之子孫施之無窮天下之安猶泰山而四維之也【師古曰維謂聨繫之】夷狄之地何足以為一日之閒【如淳曰得其地不足為一日閒暇之娛也】而煩汗馬之勞乎詩云王猶允塞徐方既來【師古曰大雅常武之詩也允信也塞滿也言王道信充滿於天下則徐方淮夷盡來服也塞悉則翻】言王道甚大而遠方懷之也臣安竊恐將吏之以十萬之師為一使之任也【師古曰言漢發一使鎮撫之則越人賓服不煩兵往使疏吏翻】是時漢兵遂出未隃領【隃與踰同領與嶺同】閩越王郢發兵距險其弟餘善乃與相宗族謀曰【相閩越國相也音息亮翻】王以擅發兵擊南越不請故天子兵來誅漢兵衆彊即幸勝之後來益多【師古曰言漢地廣大兵衆盛彊今雖勝之後必復來也】終滅國而止今殺王以謝天子天子聽罷兵固國完不聽乃力戰不勝即亡入海皆曰善即鏦殺郢王【鏦初江翻短矛也】使使奉其頭致大行大行曰所為來者誅王【為于偽翻下同】今王頭至謝罪不戰而殞利莫大焉乃以便宜案兵告大農軍而使使奉王頭馳報天子詔罷兩將兵【將即亮翻】曰郢等首惡獨無諸孫繇君丑不與謀焉【張晏曰繇邑號也師古曰繇音揺與讀曰預】乃使中郎將立丑為越繇王奉閩越先祭祀餘善已殺郢威行於國國民多屬竊自立為王繇王不能制上聞之為餘善不足復興師曰餘善數與郢謀亂【復扶又翻數所角翻】而後首誅郢師得不勞因立餘善為東越王與繇王並處【處昌呂翻】上使莊助諭意南越南越王胡頓首曰天子乃為臣興兵討閩越死無以報德遣太子嬰齊入宿衛謂助曰國新被寇【被皮義翻】使者行矣胡方日夜裝入見天子【見賢遍翻下同】助還過淮南上又使助諭淮南王安以討越事嘉答其意安謝不及助既去南越南越大臣皆諫其王曰漢興兵誅郢亦行以驚動南越且先王昔言事天子期無失禮要之不可以說好語入見【言不可喜漢使好語而入朝也說讀曰悦】則不得復歸亡國之勢也於是胡稱病竟不入見 是歲韓安國為御史大夫 東海太守濮陽汲黯為主爵都尉【按汲黯傳其先有寵於衛君至黯十世世為卿大夫盖食采於汲因以為氏班表主爵中尉秦官掌列侯景帝中六年更名都尉武帝太初元年更名右扶風治内史右地濮博本翻】始黯為謁者以嚴見憚東越相攻上使黯往視之不至至吳而還報曰越人相攻固其俗然不足以辱天子之使河内失火延燒千餘家上使黯往視之還報曰家人失火屋比延燒【師古曰家人猶言庶人家也比近也言屋相近故連延而燒也比頻寐翻】不足憂也臣過河南河南貧人傷水旱萬餘家或父子相食臣謹以便宜持節發河南倉粟以振貧民臣請歸節伏矯制之罪【師古曰矯託也託言奉制詔而行之也漢律矯制者論棄市罪】上賢而釋之其在東海治官理民好清靜擇丞史任之【如淳曰擇丞及史任之也漢律太守都尉諸侯内史史各一人卒史書佐各十人余據漢制郡守之屬有丞有諸曹史好呼到翻】責大指而已不苛小黯多病臥閨閣内不出歲餘東海大治稱之【治直吏翻下同】上聞召為主爵都尉列於九卿【漢太常郎中令中大夫令太僕大理大行令宗正大司農少府為正九卿中尉主爵都尉内史列于九卿】其治務在無為引大體不拘文法黯為人性倨少禮【師古曰倨簡傲也少詩沼翻】面折不能容人之過【折之舌翻】時天子方招文學儒者上曰吾欲云云【張晏曰所言欲施仁義也師古曰云云猶言如此如此也史畧其辭耳】黯對曰陛下内多欲而外施仁義柰何欲效唐虞之治乎上默然怒變色而罷朝公卿皆為黯懼上退謂左右曰甚矣汲黯之戇也羣臣或數黯【師古曰數責也音所具翻】黯曰天子置公卿輔弼之臣寧令從諛承意陷主於不義乎且己在其位縱愛身柰辱朝廷何黯多病病且滿三月上常賜告者數【數所角翻】終不愈最後病莊助為請告【為于偽翻】上曰汲黯何如人哉助曰使黯任職居官無以踰人然至其輔少主守城深堅招之不來麾之不去雖自謂賁育亦不能奪之矣【賁音奔】上曰然古有社稷之臣至如黯近之矣【近其靳翻】 匈奴來請和親天子下其議【下遐嫁翻】大行王恢燕人也習胡事議曰漢與匈奴和親率不過數歲即復倍約【倍蒲妹翻】不如勿許興兵擊之韓安國曰匈奴遷徙鳥舉難得而制【言其輕疾逐水草遷徙若鳥之舉也】自上古不屬為人【不以人類待之】今漢行數千里與之爭利則人馬罷乏【罷讀曰疲】虜以全制其敝此危道也不如和親羣臣議者多附安國於是上許和親<br />
<br />
  元光元年冬十一月初令郡國舉孝廉各一人【師古曰孝謂善事父母者廉謂清廉有廉隅者也】從董仲舒之言也 衛尉李廣為驍騎將軍屯雲中【周末置左右前後將軍秦漢因之位上卿至武帝置驍騎車騎等將軍後來名號浸多不可勝紀謂之雜號將軍盤洲洪氏曰西漢雜號將軍掌征伐背叛事訖則罷不常置也驍堅堯翻】中尉程不識為車騎將軍【姓譜程本自顓頊重黎之後周宣王時程伯休父入為大司馬封於程者以為氏與司馬氏同出】屯鴈門六月罷廣與程不識俱以邉太守將兵有名當時廣行無部伍行陳【部者軍行各有分部伍者五人為伍也部有校尉伍有伍長行戶剛翻陳讀曰陣】就善水草舍止人人自便不擊刁斗以自衛【孟康曰刁斗以銅作鐎受一斗晝炊飲食夜撃持行故名曰刁斗蘇林曰形如鋗無緣荀悦曰刁斗小鈴如宫中傳夜鈴也索隐曰鋗即鈴也埤蒼云鐎温器有柄斗似銚無緣師古曰鐎音譙鋗火玄翻鋗即銚也銚音姚緣去聲】莫府省約文書然亦遠斥候【淮南子曰斥度也候視也望也】未嘗遇害程不識正部曲行伍營陳撃刁斗士吏治軍簿【師古曰簿文簿治直之翻下言治】至明軍不得休息然亦未嘗遇害不識曰李廣軍極簡易【易以豉翻】然虜卒犯之無以禁也【卒讀曰猝】而其士卒亦佚樂咸樂為之死【樂音洛下同】我軍雖煩擾然虜亦不得犯我然匈奴畏李廣之略士卒亦多樂從李廣而苦程不識【師古曰苦謂厭苦之也】<br />
<br />
  臣光曰易曰師出以律否臧凶【否音鄙易師卦初六爻辭王弼注曰齊衆以律失律則散故師出以律律不可失失律而臧何異於否失令有功法所不赦故師出不以律否臧皆凶】言治衆而不用法無不凶也李廣之將使人人自便以廣之材如此焉可也然不可以為法何則其繼者難也况與之並時而為將乎夫小人之情樂於安肆而昧於近禍彼既以程不識為煩擾而樂於從廣且將仇其上而不服然則簡易之害非徒廣軍無以禁虜之倉卒而已也故曰兵事以嚴終為將者亦嚴而已矣然則傚程不識雖無功猶不敗傚李廣鮮不覆亡哉【鮮息淺翻】<br />
<br />
  夏四月赦天下 五月詔舉賢良文學上親策之 秋七月癸未日有食之<br />
<br />
  資治通鑑卷十七  <br>
   </div> 

<script src="/search/ajaxskft.js"> </script>
 <div class="clear"></div>
<br>
<br>
 <!-- a.d-->

 <!--
<div class="info_share">
</div> 
-->
 <!--info_share--></div>   <!-- end info_content-->
  </div> <!-- end l-->

<div class="r">   <!--r-->



<div class="sidebar"  style="margin-bottom:2px;">

 
<div class="sidebar_title">工具类大全</div>
<div class="sidebar_info">
<strong><a href="http://www.guoxuedashi.com/lsditu/" target="_blank">历史地图</a></strong>  
<a href="http://www.880114.com/" target="_blank">英语宝典</a>  
<a href="http://www.guoxuedashi.com/13jing/" target="_blank">十三经检索</a> 
<br><strong><a href="http://www.guoxuedashi.com/gjtsjc/" target="_blank">古今图书集成</a></strong> 
<a href="http://www.guoxuedashi.com/duilian/" target="_blank">对联大全</a> <strong><a href="http://www.guoxuedashi.com/xiangxingzi/" target="_blank">象形文字典</a></strong> 

<br><a href="http://www.guoxuedashi.com/zixing/yanbian/">字形演变</a>  <strong><a href="http://www.guoxuemi.com/hafo/" target="_blank">哈佛燕京中文善本特藏</a></strong>
<br><strong><a href="http://www.guoxuedashi.com/csfz/" target="_blank">丛书&方志检索器</a></strong> <a href="http://www.guoxuedashi.com/yqjyy/" target="_blank">一切经音义</a>  

<br><strong><a href="http://www.guoxuedashi.com/jiapu/" target="_blank">家谱族谱查询</a></strong>  <strong><a href="http://shufa.guoxuedashi.com/sfzitie/" target="_blank">书法字帖欣赏</a></strong> 
<br>

</div>
</div>


<div class="sidebar" style="margin-bottom:0px;">

<font style="font-size:22px;line-height:32px">QQ交流群9:489193090</font>


<div class="sidebar_title">手机APP 扫描或点击</div>
<div class="sidebar_info">
<table>
<tr>
	<td width=160><a href="http://m.guoxuedashi.com/app/" target="_blank"><img src="/img/gxds-sj.png" width="140"  border="0" alt="国学大师手机版"></a></td>
	<td>
<a href="http://www.guoxuedashi.com/download/" target="_blank">app软件下载专区</a><br>
<a href="http://www.guoxuedashi.com/download/gxds.php" target="_blank">《国学大师》下载</a><br>
<a href="http://www.guoxuedashi.com/download/kxzd.php" target="_blank">《汉字宝典》下载</a><br>
<a href="http://www.guoxuedashi.com/download/scqbd.php" target="_blank">《诗词曲宝典》下载</a><br>
<a href="http://www.guoxuedashi.com/SiKuQuanShu/skqs.php" target="_blank">《四库全书》下载</a><br>
</td>
</tr>
</table>

</div>
</div>


<div class="sidebar2">
<center>


</center>
</div>

<div class="sidebar"  style="margin-bottom:2px;">
<div class="sidebar_title">网站使用教程</div>
<div class="sidebar_info">
<a href="http://www.guoxuedashi.com/help/gjsearch.php" target="_blank">如何在国学大师网下载古籍?</a><br>
<a href="http://www.guoxuedashi.com/zidian/bujian/bjjc.php" target="_blank">如何使用部件查字法快速查字?</a><br>
<a href="http://www.guoxuedashi.com/search/sjc.php" target="_blank">如何在指定的书籍中全文检索?</a><br>
<a href="http://www.guoxuedashi.com/search/skjc.php" target="_blank">如何找到一句话在《四库全书》哪一页?</a><br>
</div>
</div>


<div class="sidebar">
<div class="sidebar_title">热门书籍</div>
<div class="sidebar_info">
<a href="/so.php?sokey=%E8%B5%84%E6%B2%BB%E9%80%9A%E9%89%B4&kt=1">资治通鉴</a> <a href="/24shi/"><strong>二十四史</strong></a>&nbsp; <a href="/a2694/">野史</a>&nbsp; <a href="/SiKuQuanShu/"><strong>四库全书</strong></a>&nbsp;<a href="http://www.guoxuedashi.com/SiKuQuanShu/fanti/">繁体</a>
<br><a href="/so.php?sokey=%E7%BA%A2%E6%A5%BC%E6%A2%A6&kt=1">红楼梦</a> <a href="/a/1858x/">三国演义</a> <a href="/a/1038k/">水浒传</a> <a href="/a/1046t/">西游记</a> <a href="/a/1914o/">封神演义</a>
<br>
<a href="http://www.guoxuedashi.com/so.php?sokeygx=%E4%B8%87%E6%9C%89%E6%96%87%E5%BA%93&submit=&kt=1">万有文库</a> <a href="/a/780t/">古文观止</a> <a href="/a/1024l/">文心雕龙</a> <a href="/a/1704n/">全唐诗</a> <a href="/a/1705h/">全宋词</a>
<br><a href="http://www.guoxuedashi.com/so.php?sokeygx=%E7%99%BE%E8%A1%B2%E6%9C%AC%E4%BA%8C%E5%8D%81%E5%9B%9B%E5%8F%B2&submit=&kt=1"><strong>百衲本二十四史</strong></a>  <a href="http://www.guoxuedashi.com/so.php?sokeygx=%E5%8F%A4%E4%BB%8A%E5%9B%BE%E4%B9%A6%E9%9B%86%E6%88%90&submit=&kt=1"><strong>古今图书集成</strong></a>
<br>

<a href="http://www.guoxuedashi.com/so.php?sokeygx=%E4%B8%9B%E4%B9%A6%E9%9B%86%E6%88%90&submit=&kt=1">丛书集成</a> 
<a href="http://www.guoxuedashi.com/so.php?sokeygx=%E5%9B%9B%E9%83%A8%E4%B8%9B%E5%88%8A&submit=&kt=1"><strong>四部丛刊</strong></a>  
<a href="http://www.guoxuedashi.com/so.php?sokeygx=%E8%AF%B4%E6%96%87%E8%A7%A3%E5%AD%97&submit=&kt=1">說文解字</a> <a href="http://www.guoxuedashi.com/so.php?sokeygx=%E5%85%A8%E4%B8%8A%E5%8F%A4&submit=&kt=1">三国六朝文</a>
<br><a href="http://www.guoxuedashi.com/so.php?sokeytm=%E6%97%A5%E6%9C%AC%E5%86%85%E9%98%81%E6%96%87%E5%BA%93&submit=&kt=1"><strong>日本内阁文库</strong></a> <a href="http://www.guoxuedashi.com/so.php?sokeytm=%E5%9B%BD%E5%9B%BE%E6%96%B9%E5%BF%97%E5%90%88%E9%9B%86&ka=100&submit=">国图方志合集</a> <a href="http://www.guoxuedashi.com/so.php?sokeytm=%E5%90%84%E5%9C%B0%E6%96%B9%E5%BF%97&submit=&kt=1"><strong>各地方志</strong></a>

</div>
</div>


<div class="sidebar2">
<center>

</center>
</div>
<div class="sidebar greenbar">
<div class="sidebar_title green">四库全书</div>
<div class="sidebar_info">

《四库全书》是中国古代最大的丛书,编撰于乾隆年间,由纪昀等360多位高官、学者编撰,3800多人抄写,费时十三年编成。丛书分经、史、子、集四部,故名四库。共有3500多种书,7.9万卷,3.6万册,约8亿字,基本上囊括了古代所有图书,故称“全书”。<a href="http://www.guoxuedashi.com/SiKuQuanShu/">详细>>
</a>

</div> 
</div>

</div>  <!--end r-->

</div>
<!-- 内容区END --> 

<!-- 页脚开始 -->
<div class="shh">

</div>

<div class="w1180" style="margin-top:8px;">
<center><script src="http://www.guoxuedashi.com/img/plus.php?id=3"></script></center>
</div>
<div class="w1180 foot">
<a href="/b/thanks.php">特别致谢</a> | <a href="javascript:window.external.AddFavorite(document.location.href,document.title);">收藏本站</a> | <a href="#">欢迎投稿</a> | <a href="http://www.guoxuedashi.com/forum/">意见建议</a> | <a href="http://www.guoxuemi.com/">国学迷</a> | <a href="http://www.shuowen.net/">说文网</a><script language="javascript" type="text/javascript" src="https://js.users.51.la/17753172.js"></script><br />
  Copyright &copy; 国学大师 古典图书集成 All Rights Reserved.<br>
  
  <span style="font-size:14px">免责声明:本站非营利性站点,以方便网友为主,仅供学习研究。<br>内容由热心网友提供和网上收集,不保留版权。若侵犯了您的权益,来信即刪。scp168@qq.com</span>
  <br />
ICP证:<a href="http://www.beian.miit.gov.cn/" target="_blank">鲁ICP备19060063号</a></div>
<!-- 页脚END --> 
<script src="http://www.guoxuedashi.com/img/plus.php?id=22"></script>
<script src="http://www.guoxuedashi.com/img/tongji.js"></script>

</body>
</html>
