<!DOCTYPE html PUBLIC "-//W3C//DTD XHTML 1.0 Transitional//EN" "http://www.w3.org/TR/xhtml1/DTD/xhtml1-transitional.dtd">
<html xmlns="http://www.w3.org/1999/xhtml">
<head>
<meta http-equiv="Content-Type" content="text/html; charset=utf-8" />
<meta http-equiv="X-UA-Compatible" content="IE=Edge,chrome=1">
<title>資治通鑒_232-資治通鑑卷二百三十一_232-資治通鑑卷二百三十一</title>
<meta name="Keywords" content="資治通鑒_232-資治通鑑卷二百三十一_232-資治通鑑卷二百三十一">
<meta name="Description" content="資治通鑒_232-資治通鑑卷二百三十一_232-資治通鑑卷二百三十一">
<meta http-equiv="Cache-Control" content="no-transform" />
<meta http-equiv="Cache-Control" content="no-siteapp" />
<link href="/img/style.css" rel="stylesheet" type="text/css" />
<script src="/img/m.js?2020"></script> 
</head>
<body>
 <div class="ClassNavi">
<a  href="/24shi/">二十四史</a> | <a href="/SiKuQuanShu/">四库全书</a> | <a href="http://www.guoxuedashi.com/gjtsjc/"><font  color="#FF0000">古今图书集成</font></a> | <a href="/renwu/">历史人物</a> | <a href="/ShuoWenJieZi/"><font  color="#FF0000">说文解字</a></font> | <a href="/chengyu/">成语词典</a> | <a  target="_blank"  href="http://www.guoxuedashi.com/jgwhj/"><font  color="#FF0000">甲骨文合集</font></a> | <a href="/yzjwjc/"><font  color="#FF0000">殷周金文集成</font></a> | <a href="/xiangxingzi/"><font color="#0000FF">象形字典</font></a> | <a href="/13jing/"><font  color="#FF0000">十三经索引</font></a> | <a href="/zixing/"><font  color="#FF0000">字体转换器</font></a> | <a href="/zidian/xz/"><font color="#0000FF">篆书识别</font></a> | <a href="/jinfanyi/">近义反义词</a> | <a href="/duilian/">对联大全</a> | <a href="/jiapu/"><font  color="#0000FF">家谱族谱查询</font></a> | <a href="http://www.guoxuemi.com/hafo/" target="_blank" ><font color="#FF0000">哈佛古籍</font></a> 
</div>

 <!-- 头部导航开始 -->
<div class="w1180 head clearfix">
  <div class="head_logo l"><a title="国学大师官网" href="http://www.guoxuedashi.com" target="_blank"></a></div>
  <div class="head_sr l">
  <div id="head1">
  
  <a href="http://www.guoxuedashi.com/zidian/bujian/" target="_blank" ><img src="http://www.guoxuedashi.com/img/top1.gif" width="88" height="60" border="0" title="部件查字,支持20万汉字"></a>


<a href="http://www.guoxuedashi.com/help/yingpan.php" target="_blank"><img src="http://www.guoxuedashi.com/img/top230.gif" width="600" height="62" border="0" ></a>


  </div>
  <div id="head3"><a href="javascript:" onClick="javascript:window.external.AddFavorite(window.location.href,document.title);">添加收藏</a>
  <br><a href="/help/setie.php">搜索引擎</a>
  <br><a href="/help/zanzhu.php">赞助本站</a></div>
  <div id="head2">
 <a href="http://www.guoxuemi.com/" target="_blank"><img src="http://www.guoxuedashi.com/img/guoxuemi.gif" width="95" height="62" border="0" style="margin-left:2px;" title="国学迷"></a>
  

  </div>
</div>
  <div class="clear"></div>
  <div class="head_nav">
  <p><a href="/">首页</a> | <a href="/ShuKu/">国学书库</a> | <a href="/guji/">影印古籍</a> | <a href="/shici/">诗词宝典</a> | <a   href="/SiKuQuanShu/gxjx.php">精选</a> <b>|</b> <a href="/zidian/">汉语字典</a> | <a href="/hydcd/">汉语词典</a> | <a href="http://www.guoxuedashi.com/zidian/bujian/"><font  color="#CC0066">部件查字</font></a> | <a href="http://www.sfds.cn/"><font  color="#CC0066">书法大师</font></a> | <a href="/jgwhj/">甲骨文</a> <b>|</b> <a href="/b/4/"><font  color="#CC0066">解密</font></a> | <a href="/renwu/">历史人物</a> | <a href="/diangu/">历史典故</a> | <a href="/xingshi/">姓氏</a> | <a href="/minzu/">民族</a> <b>|</b> <a href="/mz/"><font  color="#CC0066">世界名著</font></a> | <a href="/download/">软件下载</a>
</p>
<p><a href="/b/"><font  color="#CC0066">历史</font></a> | <a href="http://skqs.guoxuedashi.com/" target="_blank">四库全书</a> |  <a href="http://www.guoxuedashi.com/search/" target="_blank"><font  color="#CC0066">全文检索</font></a> | <a href="http://www.guoxuedashi.com/shumu/">古籍书目</a> | <a   href="/24shi/">正史</a> <b>|</b> <a href="/chengyu/">成语词典</a> | <a href="/kangxi/" title="康熙字典">康熙字典</a> | <a href="/ShuoWenJieZi/">说文解字</a> | <a href="/zixing/yanbian/">字形演变</a> | <a href="/yzjwjc/">金 文</a> <b>|</b>  <a href="/shijian/nian-hao/">年号</a> | <a href="/diming/">历史地名</a> | <a href="/shijian/">历史事件</a> | <a href="/guanzhi/">官职</a> | <a href="/lishi/">知识</a> <b>|</b> <a href="/zhongyi/">中医中药</a> | <a href="http://www.guoxuedashi.com/forum/">留言反馈</a>
</p>
  </div>
</div>
<!-- 头部导航END --> 
<!-- 内容区开始 --> 
<div class="w1180 clearfix">
  <div class="info l">
   
<div class="clearfix" style="background:#f5faff;">
<script src='http://www.guoxuedashi.com/img/headersou.js'></script>

</div>
  <div class="info_tree"><a href="http://www.guoxuedashi.com">首页</a> > <a href="/SiKuQuanShu/fanti/">四库全书</a>
 > <h1>资治通鉴</h1> <!--         下载:【右键另存为】即可 --></div>
  <div class="info_content zj clearfix">
  
<div class="info_txt clearfix" id="show">
<center style="font-size:24px;">232-資治通鑑卷二百三十一</center>
    資治通鑑卷二百三十一 宋 司馬光 撰<br />
<br />
  胡三省 音註<br />
<br />
  唐紀四十七【起閼逢困敦五月盡旃蒙赤奮若七月凡一年有奇始甲子五月終乙丑凡一年零三月】<br />
<br />
  德宗神武聖文皇帝六<br />
<br />
  興元元年五月鹽鐵判官萬年王紹以江淮繒帛來至【萬年京縣屬京兆繒慈陵翻】上命先給將士然後御衫【始改御裌而御衫衫單衣也將即亮翻】韓滉欲遣使獻綾羅四十擔詣行在【滉呼廣翻使疏吏翻羅綺也綾文繒丁度曰古者芒氏初作羅一曰帛之美者今人以絲縷織而交眼者爲羅擔都濫翻肩負爲擔天子所至爲行在所】幕僚何士幹請行滉喜曰君能相爲行【爲于僞翻】請今日過江士幹許諾歸别家則家之薪米儲偫已羅門庭矣【偫直里翻】登舟則資裝器用已充舟中矣下至厨籌【厨籌當作厠籌】滉皆手筆記列無不周備每擔夫與白金一版置腰間【史言韓滉彊敏精密】又運米百艘以餉李晟【艘蘇遭翻下同晟成正翻 考異曰柳玭叙訓曰上初至梁省奏甚悅又知西平聚兵必乏糧糗命運米百艘按五月初梁州尚未春服月末已克長安梁潤相去數千里詔命豈能遽逹乎今不取】自負囊米置舟中將佐爭舉之須臾而畢艘置五弩手以爲防援有寇則叩舷相警【將即亮翻扣擊也船邊曰舷音胡田翻】五百弩已彀矣比至渭橋【彀居候翻引滿比必利翻及也】盗不敢近【近其靳翻】時關中兵荒米斗直錢五百及滉米至減五之四滉爲人彊力嚴毅自奉儉素夫人常衣絹裙【衣於既翻絹與掾翻縑帛織成而無紋其精善者曰繒俗亦謂之絹】破然後易吐蕃既破韓旻等【吐從暾入聲破韓旻見上卷是年四月】大掠而去朱<br />
<br />
  泚使田希鑒厚以金帛賂之吐蕃受之韓遊瓌以聞渾瑊又奏尚結贊屢遣人約刻日共取長安既而不至聞其衆今春大疫近已引兵去【泚且禮翻又音此瓌工回翻渾戶昆翻又戶本翻瑊古衘翻 考異曰實錄舊本紀皆云乙丑渾瑊與蕃將論莽羅衣衆大破朱泚將韓旻等于武功武亭川吐蕃傳亦同邠志曰李懷光竟不署敇結贊亦不進軍又曰渾公出斜谷曹子逹赴渾公吐蕃以二萬騎從之既勝泚軍大掠而去泚使田希鑒以金帛賂之盖尚結贊雖引兵入塞止屯邠南但遣論莾羅衣將偏軍助瑊破泚于武功大掠而去既受泚賂遂引兵歸國瑊於吐蕃歸國之時有此奏耳】上以李晟渾瑊兵少【少詩沼翻】欲倚吐蕃以復京城聞其去甚憂之以問陸贄贄以爲吐蕃貪狡有害無益得其引去實可欣賀乃上奏其略曰吐蕃遷延顧望反覆多端深入郊畿隂受賊使【使疏吏翻】致令羣帥進退憂虞【帥所類翻】欲捨之獨前則慮其懷怨乘躡【乘其虛躡其後也躡尼輒翻】欲待之合勢則苦其失信稽延戎若未歸寇終不滅又曰將帥意陛下不見信任且患蕃戎之奪其功士卒恐陛下不恤舊勞而畏蕃戎之專其利賊黨懼蕃戎之勝不死則悉遺人禽【遺唯季翻】百姓畏蕃戎之來有財必盡爲所掠是以順於王化者其心不得不怠䧟於寇境者其勢不得不堅又曰今懷光别保蒲絳吐蕃遠避封疆形勢既分腹背無患瑊晟諸帥才力得伸又曰但願陛下愼於撫接勤於砥礪中興大業旬月可期不宜尚眷眷於犬羊之羣以失將士之情也上復使謂贄曰卿言吐蕃形勢甚善然瑊晟諸軍當議規畫令其進取朕欲遣使宣慰卿宜審細條流以聞【條分也流派也】贄以爲賢君選將委任責成故能有功况今秦梁千里【秦謂咸陽長安古秦中之地梁謂梁州】兵勢無常遥爲規畫未必合宜彼違命則失君威從命則害軍事進退羈礙難以成功【史炤曰覊馬絡頭也礙謂覊所掛礙也余謂贄言羈礙者蓋謂欲進則有所羈而不得進欲退則有所礙而不得退也】不若假以便宜之權待以殊常之賞則將帥感悦智勇得伸乃上奏其略曰鋒鏑交於原野而决策於九重之中機會變於斯須而定計於千里之外用捨相礙否臧皆凶【易曰師出以律否臧凶王弼注曰齊衆以律失律則散律不可失失律而臧何異於否失令有功法所不赦故師出不以律否臧皆凶陸德明釋文曰否音鄙惡也臧作郎翻善也】上有掣肘之譏【宓子賤爲單父宰請吏於魯侯魯侯使二吏與之俱至單父子賤使吏書而掣其肘書惡則從而怒之二吏歸以告魯侯魯侯曰此謂吾橈其政也】下無死綏之志【兵志曰將軍死綏有前無却】又曰傳聞與指實不同懸算與臨事有異又曰設使其中有肆情干命者陛下能於此時戮其違詔之罪乎是則違命者既不果行罰從命者又未必合宜徒費空言秖勞睿慮匪惟無益其損實多又曰君上之權特異臣下惟不自用乃能用人 癸酉涇王侹薨【侹肅宗子音他頂翻】徐海沂密觀察使高承宗卒【建中二年李洧以徐州歸國明年以爲徐沂密】<br />
<br />
  【觀察使洧卒高承宗代之】甲戍使其子明應知軍事 乙亥李抱真王武俊距貝州三十里而軍朱滔聞兩軍將至急召馬寔寔晝夜兼行赴之或謂滔曰武俊善野戰不可當其鋒宜徙營稍前逼之使回紇絶其糧道我坐食德棣之餫【餫音運糧運曰餫】依營而陳【陳讀曰陣】利則進攻否則入保待其饑疲然後可制也滔疑未决會馬寔軍至滔命明日出戰寔言軍士冒暑困憊【憊音蒲拜翻】請休息數日乃戰常侍楊布【滔倣天朝置常侍】將軍蔡雄引回紇逹干見滔逹干曰回紇在國與鄰國戰常以五百騎破鄰國數千騎如掃葉耳今受大王金帛牛酒前後無筭思爲大王立效【爲于僞翻下同】此其時矣明日願大王駐馬高丘觀回紇爲大王剪武俊之騎使匹馬不返【爲于僞翻】布雄曰大王英略盖世舉燕薊全軍將掃河南清關中今見小敵冘豫不擊【冘讀與猶同按後漢書馬援傳計冘與未决章懷太子賢注曰冘行貌也義見說文豫亦未定也冘音以林翻毛晃曰冘豫不定後漢馬援傳計冘豫未决字從犬曲其足與古尤同與侵韻冘韻不同唐史冘豫音淫誤今從晃】失遠近之望將何以成霸業乎逹干請戰是也滔喜遂决意出戰丙子旦武俊遣其兵馬使趙琳將五百騎伏于桑林【桑林之地在經城西南】抱真列方陳於後【陳讀曰陣下同】武俊引騎兵居前自當回紇回紇縱兵衝之武俊使其騎控馬避之回紇突出其後將還武俊乃縱兵擊之趙琳自林中出横擊之回紇敗走武俊急追之滔騎兵亦走自踐其步陳步騎皆東奔滔不能制遂走趣其營【趣七喻翻下同】抱真武俊合兵追擊之時滔引三萬人出戰死者萬餘人逃潰者亦萬餘人滔纔與數千人入營堅守會日暮昏霧兩軍不能進抱真軍其營之西北武俊軍其東北滔夜焚營引兵出南門趣德州遁去委弃所掠資財山積兩軍以霧不能追也滔殺楊布蔡雄而歸幽州心既内慙又恐范陽留守劉怦因敗圖己【怦普耕翻】怦悉發留守兵夾道二十里具儀仗迎之入府相對悲喜時人多之初張孝忠以易州歸國詔以孝忠爲義武節度使以易定滄三州隷之【事見二百二十七卷建中三年】滄州刺史李固烈李惟岳之妻兄也【李惟岳本姓張故娶李氏】請歸恒州孝忠遣押牙安喜程華交其州事【安喜縣漢之盧奴縣屬中山國燕主慕容亟改爲不連北齊改爲安喜隋改爲鮮虞唐武德四年復爲安喜帶定州】固烈悉取軍府綾縑珍貨數十車將行軍士大譟曰刺史掃府庫之實以行將士於後饑寒奈何遂殺固烈屠其家程華聞亂自竇逃出亂兵求得之請知州事華不得已從之孝忠聞之即版華攝滄州刺史 【考異曰舊張孝忠傳曰遣華往滄州交檢府藏程日華傳曰孝忠今華詣固烈交郡固烈死孝忠版華知滄州事燕南記曰孝忠差牙官程華與固烈交割固烈死孝忠聞之當日差人送文牒令攝刺史按固烈既去則滄州無主孝忠豈得但今華交檢府藏今從華傳及燕南記】華素寛厚推心以待將士將士安之會朱滔王武俊叛更遣人招華【更工衡翻迭也】華皆不從時孝忠在定州自滄如定必過瀛州瀛隸朱滔道路阻澀【澀色立翻史炤曰阻隔也澀不通滑也】滄州錄事參軍李宇說華表陳利害請别爲一軍華從之【說輸芮翻下同】遣宇奉表詣行在上即以華爲滄州刺史橫海軍副大使知節度事賜名日華令日華歲供義武租錢十二萬緡王武俊又使人說誘之時軍中乏馬日華紿使者曰王大夫必欲相屬當以二百騎相助武俊給之日華悉留其馬遣其士歸武俊怒而方與馬燧等相拒不能攻取日華由是獲全及武俊歸國日華乃遣人謝過償其馬價且賂之武俊喜復與交好【騎奇寄翻好呼到翻】 庚寅李晟大陳兵諭以收復京城先是姚令言等屢遣諜人覘晟進軍之期【先悉薦翻諜徒恊翻覘丑廉翻】皆爲邏騎所獲【邏郎佐翻廵察者也】晟引示以所陳兵謂曰歸語諸賊【語牛倨翻】努力固守勿不忠於賊也皆飲之酒【飲於禁翻】給錢而縱之遂引兵至通化門外曜武而還【還從宣翻又如字】賊不敢出晟召諸將問兵所從入皆請先取外城據坊市然後北攻宫闕晟曰坊市狹隘賊若伏兵格闘居人驚亂非官軍之利也今賊重兵皆聚苑中不若自苑北攻之潰其腹心賊必奔亡如此則宫闕不殘坊市無擾策之上者也諸將皆曰善乃牒渾瑊及鎮國節度使駱元光商州節度使尚可孤刻期集于城下【京城之下也】壬辰尚可孤敗泚將仇敬忠於藍田西斬之【敗補邁翻】乙未李晟移軍於光泰門外米倉村【光泰門苑城東北門程大昌曰光泰門在通化門北小城之東門門東七里有長樂城呂大防長安圖光泰門者京城東門大明宫東苑之東】丙申晟方自臨築壘泚驍將張庭芝李希倩引兵大至晟謂諸將曰始吾憂賊濳匿不出今來送死此天贊我不可失也命副元帥兵馬使吳詵等縱兵擊之時華州營在北兵少【華州兵駱元光之兵華戶化翻少詩沼翻】賊并力攻之晟命牙前將李演等帥精兵救之演等力戰賊敗走演等追之乘勝入光泰門再戰又破之會夜晟斂兵還賊餘衆走入白華門【白華殿門也】夜聞慟哭希倩希烈之弟也丁酉晟復出兵【復扶又翻】諸將請待西師至夾攻之【西師謂渾瑊之師也】晟曰賊數敗已破膽【數所角翻】不乘勝取之使其成備非計也賊又出戰官軍屢捷駱元光敗泚衆於滻西【敗補邁翻】戊戍晟陳兵於光泰門外使李演及牙前兵馬使王佖將騎兵【佖蒲必翻】牙前將史萬頃將步兵直抵苑墻神䴥村【按新書李晟傳神䴥村在苑北䴥古牙翻】晟先使人夜開苑墻二百餘步比演等至【比必利翻及也】賊已樹栅塞之自栅中刺射官軍【塞悉則翻刺七亦翻射而亦翻】官軍不得進晟怒叱諸將曰縱賊如此吾先斬公輩矣萬頃懼帥衆先進拔栅而入【帥讀曰率下同】佖演引騎兵繼之賊衆大潰諸軍分道並入姚令言等猶力戰晟命决勝軍使唐良臣等步騎蹙之且戰且前凡十餘合賊不能支至白華門有賊數千騎出官軍之背晟帥百餘騎回禦之左右呼曰【呼火故翻】相公來賊皆驚潰【涇原將士素畏服李晟故聞其來而驚潰】先是泚遣張光晟將兵五千屯九曲【先悉薦翻】去東渭橋十餘里光晟密輸欵于晟及泚敗光晟勸泚出亡泚乃與姚令言帥餘衆西走猶近萬人【帥讀曰率近其靳翻】光晟送泚出城還降於晟【降戶江翻】晟遣兵馬使田子奇以騎兵追泚晟屯含元殿前舍於右金吾仗【含元殿唐東内之前殿也左金吾仗在殿之東右金吾仗在殿之西】令諸軍曰晟賴將士之力克清宫禁長安士庶久䧟賊庭若小有震驚非弔民伐罪之意晟與公等室家相見非晚五日内無得通家信命京兆尹李齊運等安慰居人晟大將高明曜取賊妓【妓渠綺翻女樂也】尚可孤軍士擅取賊馬晟皆斬之軍中股栗公私安堵秋毫無犯遠坊有經宿乃知官軍入城者【史言李晟御軍嚴整】是日渾瑊戴休顔韓遊瓌亦克咸陽敗賊三千餘衆【敗捕邁翻】聞泚西走分兵邀之己亥晟使京西兵馬使孟涉屯白華門尚可孤屯望仙門【唐大明宫南面五門其中曰丹鳳門丹鳳之東為望仙門又東爲延政門丹鳳之西爲建福門又西爲興安門也】駱元光屯章敬寺晟以牙前三千人屯安國寺【程太曰章敬寺在東城之外安國寺在大明宫東南】以鎮京城斬泚黨李希倩敬釭彭偃等八人於市 王武俊既破朱滔還恒州表讓幽州盧龍節度使上許之【王武俊兼幽州盧龍節度使見上卷是年二月恒戶登翻】 六月癸卯李晟遣掌書記吳人于公異作露布上行在【上時掌翻】曰臣已肅清宫禁祗謁寢園鍾簴不移【簴其呂翻說文曰簴鍾鼔之拊也飾爲猛獸釋名曰横曰栒縱曰簴又云簴天上神獸也鹿頭龍身象之爲簴以架鍾鼔】廟貌如故【孔頴逹曰廟之言貌也死者精神不可得而見但以生時之宫室象貌爲之耳孝經注云宗尊也廟貌也】上泣下曰天生李晟以爲社稷非爲朕也【爲于偽翻史言于公異爲李晟作露布得體】晟在渭橋熒惑守歲【歲星所在其國有福熒惑守之是爲罰星】久之乃退賓佐皆賀曰熒惑退舍皇家之福也宜速進兵晟曰天子野次臣下知死敵而已天象高遠誰得知之既克長安乃謂之曰曏非相拒也吾聞五星贏縮無常【前漢書天文志曰凡五星早出爲贏贏爲客晩出爲縮縮爲主又晉書天文志曰失次而上爲贏失次而下爲縮】萬一復來守歲【復扶又翻】吾軍不戰自潰矣皆謝曰非所及也朱泚將奔吐蕃其衆隨道散亡比至涇州【比必利翻及也】纔百餘騎田希鑒閉城拒之泚謂之曰汝之節吾所授也【朱泚以田希鑒爲涇原節度使見上卷是年四月】柰何臨危相負使焚其門希鑒取節投火中曰還汝節泚衆皆哭涇卒遂殺姚令言詣希鑒降泚獨與范陽親兵及宗族賓客北趣驛馬關【趣七喻翻】寧州刺史夏侯英拒之至彭原西城屯【彭原本彭陽縣隋開皇十八年更名唐屬寧州】其將梁庭芬射泚墜阬中【射而亦翻】韓旻等斬之詣涇州降源休李子平奔鳳翔李楚琳斬之皆傳首行在 上命陸贄草詔賜渾瑊使訪求奉天所失裹頭内人【裹頭内人在宫中給使令者也内人給使令者皆冠巾故謂之裹頭内人】贄上奏以爲巨盜始平疲瘵之民瘡痍之卒尚未循拊【瘵仄介翻】而首訪婦人非所以副維新之望也謀始盡善克終已稀始而不謀終則何有所賜瑊詔未敢承旨上遂不降詔竟遣中使求之乙巳詔吏部侍郎斑宏充宣慰使勞問將士撫慰蒸黎【按詩傳箋蒸衆也黎亦衆也勞力到翻】丙午李晟斬文武官受朱泚寵任者崔宣洪經綸等十餘人【考異曰袁皓興元聖功錄載李晟奏宥郭晞狀曰晞頃因鑾輿順動山谷濳藏逆賊所知舁致城邑迫脅授】<br />
<br />
  【任前後極多蒼黄之中僞令仍及堅臥當節即懼嚴刑隨俗從官又傷素業然晞已染汙俗尚可昭明子儀勲勞書在王府父為中興之佐子有疑謗之名非止在于一身實恐玷於先烈况臣揔領士馬孤立渭橋頻有帛書累陳誠效按晞舊傳泚欲令掌兵晞陽瘖泚以兵脅之終不語賊知其不可用乃止晞濳奔奉天從駕還京不云終臣事泚而皓載晟此狀恐非其實今不取】又表守節不屈者劉廼蔣沇等【劉廼事見上卷是年二月蔣沇事見二百二十八年建中四年】己酉以李晟爲司徒中書令駱元光尚可孤各遷官有差【賞收復京城之功也】以檢校御史中丞田希鑒爲涇原節度使 詔改梁州爲興元府【以紀元爲府號始此】 甲寅以渾瑊爲侍中韓遊瓌戴休顔各遷官有差【賞扈衛之功也】 朱泚之敗也李忠臣奔樊川【酈道元水經注曰樊川即杜縣之樊鄉漢高祖還定三秦以樊噲灌廢丘最賜邑於此鄉也桉其地在唐長安城南程大昌曰樊川在萬年縣南三十五里】擒獲丙辰斬之 上問陸贄今至鳳翔有迎駕諸軍形勢甚盛欲因此遣人代李楚琳何如贄上奏以爲如此則事同脅執以言乎除亂則不武以言乎務理則不誠用是時廵後將安入【書周官六年王乃時廵考制度于四岳諸侯各朝于方岳孔安國曰春東夏南秋西冬北故謂之時廵】議者或謂之權臣竊未諭其理夫權之爲義取類權衡【衡以平物權則權物之輕重揆之以衡平】今輦路所經首行脅奪易一帥而虧萬乘之義得一方而結四海之疑【帥讀曰率乘繩證翻】乃是重其所輕而輕其所重謂之權也不亦反乎以反道爲權以任數爲智君上行之必失衆臣下用之必䧟身歷代之所以多喪亂而長姦邪由此誤也【陸䞇此論所以正漢儒反經合道爲權之失程氏曰漢儒以反經合道爲權故有權變權術之說皆非也權只是經字自漢以下無人識權字喪悉浪翻長知兩翻】不如奠枕京邑【史炤曰奠枕安枕也楊子曰奠枕于京】徵授一官彼喜於恩宥將奔走不暇安敢輒有旅拒【史炤曰旅衆也拒捍也謂率衆以相捍也】復勞誅鉏哉【復扶又翻】戊午車駕發漢中 李晟綜理長安以備百司【史炤曰綜機縷也理治也謂整治其事使皆有紀若機之綜縷也】自請至鳳翔迎扈上不許内常侍尹元貞奉使同華輒詣河中招諭李懷光【此唐之中世閹宦之常態也華戶化翻】晟奏元貞矯制擅赦元惡請理其罪【理治也避高宗諱以治爲理】 秋七月丙子車駕至鳳翔斬喬琳蔣鎮張光晟等李晟以光晟雖臣賊而滅賊亦頗有力欲全之上不許 副元帥判官高郢數勸李懷光歸欵【數所角翻高郢判李懷光幕府懷光此時已罷副元帥而不肯釋兵史仍書郢元官】懷光遣其子璀詣行在謝罪【璀七罪翻】請束身歸朝【朝直遥翻】庚辰詔遣給事中孔巢父齎先除懷光太子太保敕【懷光除見上卷本年三月】詣河中宣慰朔方將士悉復官爵如故【朔方將士懷光所部也】 壬午車駕至長安渾瑊韓遊瓌戴休顔以其衆扈從【從才用翻】李晟駱元光尚可孤以其衆奉迎步騎十餘萬旌旗數十里晟謁見上於三橋【見賢遍翻】先賀平賊後謝收復之晩伏路左請罪上駐馬慰撫爲之掩涕【爲于僞翻掩面垂涕謂之掩涕】命左右扶上馬【上時掌翻】至宫每閒日【閒讀曰閑唐世天子以隻日視朝雙日謂之閑日】輒宴勲臣賞賜豐渥李晟爲之首渾瑊次之諸將相又次之 曹王臯遣其將伊愼王鍔圍安州李希烈遣其甥劉戒虛將步騎八千救之臯遣其别將李伯濳逆之於應山【劉昫曰應山本漢南陽郡隨縣地梁分隨縣置永陽縣隋改爲應山以縣北山爲名唐屬隨州九域志應山縣在隨州北一百八里】斬首千餘級生擒戒虛徇於城下安州遂降以伊慎爲安州刺史又擊希烈將康叔夜於厲鄉走之【記祭法厲山氏之有天下也其子農能植百穀注云厲山氏炎帝也起於厲山西漢書地理志注云隨故厲國皇甫諡曰今隨之厲鄉九域志隨州厲郷村有厲山今自棗陽至厲鄉道路交錯號九十九岡】丁亥孔巢父至河中李懷光素服待罪巢父不之止<br />
<br />
  懷光左右多胡人皆嘆曰太尉無官矣【胡人不習朝章見懷光素服待罪故以爲無官】巢父又宣言於衆曰軍中誰可代太尉領軍者於是懷光左右發怒諠譟宣詔未畢衆殺巢父及中使談守盈懷光亦不之止 【考異曰邠志曰七月十二日駕還長安上使諫議大夫孔巢父中官談懷仙特詔赦懷光曰奉天之時非卿不能救朕今日之事非朕不能容卿宜委軍赴闕以保官爵使者將至懷光隂導其卒使留已卒之蕃渾希懷光意輒害二使欲食其肉懷光異而覆之全尸以聞今從實錄】復治兵爲拒守之備【復扶又翻治直之翻】 辛卯赦天下 初肅宗在靈武【見二百十九卷至德元載】上爲奉節王學文於李泌代宗之世泌居蓬萊書院【見二百二十四卷永泰元年史炤曰泌兵媚翻】上爲太子亦與之遊及上在興元泌爲杭州刺史上急詔徵之與睦州刺史杜亞俱詣行在乙未以泌爲左散騎常侍亞爲刑部侍郎命泌日直西省以候對【唐門下省謂之東省中書省謂之西省】朝野皆屬目附之【屬之欲翻】上問泌河中密邇京城朔方兵素稱精鋭如逹奚小俊等皆萬人敵朕晝夕憂之柰何對曰天下事甚有可憂者若惟河中不足憂也夫料敵者料將不料兵今懷光將也【將郎亮翻下同】小俊之徒乃兵耳何足爲意懷光既解奉天之圍視朱泚垂亡之虜不能取乃與之連和使李晟得取以爲功今陛下已還宫闕懷光不束身歸罪乃虐殺使臣【謂殺孔巢父談守盈也使疏吏翻】鼠伏河中如夢魘之人耳【魘於琰翻】但恐不日爲帳下所梟【梟古堯翻】使諸將無以藉手也初上發吐蕃以討朱泚【事見二百二十九卷本年正月】許成功以伊西北庭之地與之及泚誅吐蕃來求地上欲召兩鎮節度使郭昕李元忠還朝【昕元忠見二百二十七卷建中二年】以其地與之李泌曰安西北庭人性驍悍控制西域五十七國【西域漢時有三十六國其後稍分至唐冇五十七國】及十姓突厥【西突厥冇五弩矢畢五咄陸凡十姓】又分吐蕃之勢使不能併兵東侵【謂東侵涇邠岐隴諸州】柰何拱手與之且兩鎮之人勢孤地遠盡忠竭力爲國家固守近二十年【代宗初吐蕃陷河隴獨安西北庭爲唐固守爲于僞翻近其靳翻】誠可哀憐一旦棄之以與戎狄彼其心必深怨中國他日從吐蕃入寇如報私讎矣况日者吐蕃觀望不進隂持兩端大掠武功受賂而去【事見上卷本年四月】何功之有衆議亦以爲然上遂不與 李希烈聞李希倩伏誅忿怒八月壬寅遣中使至蔡州殺顔眞卿 【考異曰顔氏行狀其年八月二十四日又使辛景臻等害公于龍興寺又曰初遭難後嗣曹王臯上表曰臣見蔡州歸順脚力張希璨王仕顆等說去年八月二十四日蔡州城中見封有鄰兒不得名字云希烈令僞皇城使辛景臻右軍安華於龍興寺殺顔真卿實錄及舊傳云三日今從之】中使曰有敕真卿再拜中使曰今賜卿死真卿曰老臣無狀罪當死不知使者幾日長安使者曰自大梁來非長安也真卿曰然則賊耳何謂敕邪遂縊殺之 李晟以涇州倚邊屢害軍帥常為亂根【帝初即位涇州有劉文喜之亂既而又有姚令言之亂既而田希鑒又殺馮河清帥所類翻下同】奏請往理不用命者【理即治也】力田積粟以攘吐蕃癸卯以晟兼鳳翔隴右節度等使及四鎮北庭涇原行營副元帥進爵西平王時李楚琳入朝晟請與俱至鳳翔而斬之以懲逆亂上以新復京師務安反仄不許 先是上命渾瑊駱元光討李懷光軍于同州【九域志同州至河中七十五里先悉薦翻】懷光遣其將徐庭光以精卒六千軍于長春宫以拒之瑊等數爲所敗不能進【數所角翻敗補邁翻】時度支用度不給【度支之度徒洛翻】議者多請赦懷光上不許李懷光遣其妹壻要廷珍守晉州【要於消翻姓也姓苑吳人要離之後後漢有河南令要兢】牙將毛朝守隰州【朝直遥翻音揚】鄭抗守慈州馬燧皆遣人說下之【晉隰慈三州皆與馬燧廵屬接壤故得說下之宋白曰慈州文城郡赤狄廧咎如之國郡西南有采桑津晉里克敗赤狄之地漢爲北屈縣隋爲汾州大業爲丈城郡唐貞觀爲慈州以州城内舊冇慈烏戍因名治吉鄉縣漢北屈縣也說式芮翻】上乃加渾瑊河中絳州節度使充河中同華陜虢行營副元帥加馬燧奉誠軍晉慈隰節度使充管内諸軍行營副元帥【渾戶混翻又戶本翻瑊古咸翻華戶化翻是年正月置奉誠軍于同州以授康日知事見二百二十九卷帥所類翻陜失冉翻使疏吏翻】與鎭國節度使駱元光【肅宗上元二年置鎭國節度於華州廣德元年罷今復置】鄜坊節度使唐朝臣合兵討懷光【鄜音膚】初王武俊急攻康日知於趙州馬燧奏請詔武俊與李抱眞同擊朱滔以深趙隸武俊改日知爲晉慈隰節度使上從之日知未至而三州降燧【降戶江翻下同】故上使燧兼領之燧表讓三州於日知且言因降而授恐後有功者踵以爲常上嘉而許之燧遣使迎日知既至籍府庫而歸之 甲辰以鳳翔節度使李楚琳為左金吾大將軍 丙子加渾瑊朔方行營元帥 李晟至鳳翔治殺張鎰之罪【殺張鎰見二百二十八卷建中四年治直之翻鎰弋質翻】斬禆將王斌等十餘人【斌音彬】朱滔爲王武俊所攻殆不能軍上表待罪【上時掌翻】 癸未馬燧將步騎三萬攻絳州【絳州時屬李懷光將即亮翻又音如字騎奇寄翻】 度支以李懷光所部將士數萬與懷光同反不給冬衣上曰朔方軍累代忠義【度徒洛翻自肅代以來朔方軍輸力王室功高天下】今爲懷光所制耳將士何罪冬十月詔朔方及諸軍在懷光所者冬衣及賞錢皆當别貯【貯丁呂翻】俟道路稍通即時給之 李勉累表乞自貶【以討李希烈喪師失守也】辛丑罷勉都統節度使【建中閒勉以永平節度使都統討李希烈之兵】其檢校司徒同平章事如故 丙辰李懷光將閻晏寇同州官軍敗於沙苑詔徵邠州之軍韓遊瓌將甲士六千赴之 乙丑馬燧拔絳州分兵取聞喜萬泉虞郷永樂猗氏【武德元年分芮縣置永樂縣屬芮州州廢屬鼎州又廢鼎州以縣屬河中府燧既取永樂則兵逼河中矣樂音洛】 初魚朝恩既誅代宗不復使宦官典兵【事見二百二十四卷大歷五年復扶又翻】上即位悉以禁兵委白志貞【白志貞初名白琇珪典禁兵事始見二百二十五卷大歷十四年】志貞得罪【見二百二十九卷建中四年】上復以宦官竇文塲代之從幸山南兩軍稍集【兩軍謂左右神策軍】上還長安頗忌宿將握兵多者稍稍罷之戊辰以文塲監神策軍左廂兵馬使王希遷監右廂兵馬使始令宦官分典禁旅【宦官握兵柄自此不可奪矣將即亮翻監古衘翻 考異曰舊竇文塲傳云文塲與霍仙鳴分統禁旅盖希遷尋罷而仙鳴代之也今從實錄】 閠月丙子以涇原節度使田希鑒爲衛尉卿李晟初至鳳翔希鑒使參候晟謂使者曰涇州逼近吐蕃【近其靳翻】萬一入寇州兵能獨禦之乎欲遣兵防援又未知田尚書意使者歸以告希鑒希鑒果請援兵晟遣心腹將彭令英等戍涇州晟尋託廵邉詣涇州希鑒出迎晟與之並轡而入道舊結歡希鑒妻李氏以叔父事晟晟謂之田郎晟命具三日食曰廵撫畢即還鳳翔希鑒不復疑【復扶又翻】晟置宴希鑒與將佐俱至晟營晟伏甲於外廡既食而飲彭令英引涇州諸將下堂晟曰我與汝曹久别各宜自言姓名於是得爲亂者石奇等三十餘人讓之曰汝曹屢爲逆亂殘害忠良固天地所不容悉引出斬之希鑒尚在座晟顧曰田郎亦不得無過以親知之故當使身首得完希鑒曰唯【唯于癸翻】遂引出縊殺之并其子萼 【考異曰舊晟傳曰晟至涇州希鑒迎謁于座執而誅之還鎮表李觀爲涇原節度使幸泰天録十月丁丑李晟誅田希鑒于涇州實錄閩月癸酉除李觀涇原節度使丙子以希鑒爲衛尉卿丁丑晟誅希鑒今從之】晟入其營諭以誅希鑒之意衆股栗無敢動者 李希烈遣其將翟崇暉悉衆圍陳州久之不克【翟萇伯翻】李澄知大梁兵少不能制滑州遂焚希烈所授旌節誓衆歸國【李澄請降事始上卷上年】甲午以澄爲汴滑節度使 【考異曰二月已云上以澄爲滑州節度使蓋于時但許之耳】 宋亳節度使劉洽遣馬步都虞劉昌與隴右幽州行營節度使曲環等將兵三萬救陳州十一月癸卯敗翟崇暉于州西【敗補邁翻】斬首三萬五千級擒崇暉以獻乘勝進攻汴州李希烈懼奔歸蔡州李澄引兵趣汴州【趣七喻翻】至城北恇怯不敢進【恇去王翻】劉洽兵至城東戊午李希烈守將田懷珍開門納之明日澄入舍於浚儀【浚儀帶汴州劉澄盖舍於縣治輿地志夷門之下新里之東浚水之北象而儀之以爲邑名漢武元年廢新里而立浚儀縣】兩軍之士日有忿鬩【鬩許激翻闘也狠也戾也又相怨也】會希烈鄭州守將孫液降於澄澄引兵屯鄭州詔以都統司馬寶鼎薛珏爲汴州刺史【都統司馬宋滑河陽都統司馬也寶鼎縣屬河中府本汾隂縣開元十年獲寶鼎更名珏古岳翻】李勉至長安素服待罪議者多以勉失守大梁【勉失守事見二百二十九卷建中四年】不應尚爲相【相息亮翻】李泌言於上曰李勉公忠雅正而用兵非其所長及大梁不守將士棄妻子而從之者殆二萬人足以見其得衆心矣且劉洽出勉麾下勉至睢陽【睢陽宋州】悉舉其衆以授之卒平大梁【卒子恤翻】亦勉之功也上乃命勉復其位議者又言韓滉聞鑾輿在外聚兵修石頭城【事見二百二十九卷建中四年】隂蓄異志上疑之以問李泌對曰滉公忠清儉自車駕在外滉貢獻不絶【事見上卷】且鎮江東十五州盜賊不起皆滉之力也【唐時浙江東西道所統惟潤昇常湖蘇杭睦越明台温衢處婺十四州前此滉遣宣潤弩手援寧陵盖兼統宣州爲十五州也】所以修石頭城者滉見中原板蕩謂陛下將有永嘉之行【引□永嘉之亂元帝南度以爲言】爲迎扈之備耳此乃人臣忠篤之慮柰何更以爲罪乎滉性剛嚴不附權貴故多謗毁願陛下察之臣敢保其無他上曰外議洶洶章奏如麻【如麻言其多如麻可束也】卿弗聞乎對曰臣固聞之其子臯為考功員外郎今不敢歸省其親【省悉景翻覲省也】正以謗語沸騰故也上曰其子猶懼如此卿柰何保之對曰滉之用心臣知之至熟願上章明其無它【願上時掌翻】乞宣示中書使朝衆皆知之【朝衆謂在朝百官之衆也朝直遥翻下同】上曰朕方欲用卿人亦何易可保愼勿違衆恐并為卿累也【易以豉翻累良瑞翻】泌退遂上章請以百口保滉它日上謂泌曰卿竟上章已為卿留中【爲于偽翻】雖知卿與滉親舊豈得不自愛其身乎對曰臣豈肯私於親舊以負陛下顧滉實無異心臣之上章以為朝廷非為身也上曰如何其爲朝廷【爲于僞翻下同】對曰今天下旱蝗關中米斗千錢倉廩耗竭而江東豐稔願陛下早下臣章【下戶嫁翻下同】以解朝衆之惑面諭韓臯使之歸覲【歸覲者歸覲省父母也】令滉感激無自疑之心速運粮儲豈非為朝廷邪上曰善朕深諭之矣即下泌章令韓臯謁告歸覲面賜緋衣諭以卿父比以謗言【比畀至翻】朕今知其所以釋然不復信矣【復扶又翻】因言關中乏糧歸語卿父【語牛倨翻】宜速致之臯至潤州滉感悦流涕即日自臨水濱發米百萬斛聽臯留五日即還朝臯别其母啼聲聞於外【聞音問】滉怒召出撻之自送至江上冒風濤而遣之既而陳少遊聞滉貢米亦貢二十萬斛【陳少遊時鎮淮南】上謂李泌曰韓滉乃能化陳少遊貢米矣對曰豈惟少遊諸道將爭入貢矣 吏部尚書同平章事蕭復奉使自江淮還【蕭復出使見二百二十九卷興元元年四月還從宣翻又如字】與李勉盧翰劉從一俱見上【見賢遍翻】勉等退復獨留言於上曰陳少遊任兼將相首敗臣節【敗補邁翻陳少遊事見二百二十九卷建中四年】韋臯幕府下僚獨建忠義【韋臯事見二百二十八卷建中四年】請以臯代少遊鎮淮南上然之尋遣中使馬欽緒揖劉從一附耳語而去諸相還閤【諸相在省中坐政事堂既退各居閤子】從一詣復曰欽緒宣旨令從一與公議朝來所言事即奏行之【朝如字下朝來同】勿令李盧知敢問何事也復曰唐虞黜陟岳牧僉諧【事見堯典舜典】爵人於朝與士共之【記王制之言】使李盧不堪為相則罷之既在相位朝廷政事安得不與之同議而獨隱此事乎此最當今之大弊朝來主上已有斯言【朝早也陟遥翻】復已面陳其不可不謂聖意尚爾復不惜與公奏行之但恐浸以成俗未敢以告竟不以語從一從一奏之【語牛倨翻】上愈不悅復乃上表辭位乙丑罷為左庶子劉洽克汴州得李希烈起居注云某月日陳少遊上表歸順【史究言陳少遊敗臣節之事】少遊聞之慙懼發疾十二月乙亥薨贈太尉賻祭如常儀【賻符過翻】淮南大將王韶欲自為留後令將士推已知軍事且欲大掠韓滉遣使謂之曰汝敢爲亂吾即日全軍渡江誅汝矣韶等懼而止上聞之喜謂李泌曰滉不惟安江東又能安淮南真大臣之器卿可謂知人庚辰加滉平章事江淮轉運使滉運江淮粟帛入貢府【謂朝廷受貢藏財物之府】無虚月朝廷賴之使者勞問相繼【使疏吏翻勞力到翻】恩遇始深矣 是歲蝗徧遠近草木無遺惟不食稻大饑道殣相望【詩云行有死人尚或殣之殣渠吝翻瘞尸也又餓殍爲殣道殣相望本左傳之言】<br />
<br />
  貞元元年春正月丁酉朔赦天下改元 癸丑贈顔真卿司徒諡曰文忠 新州司馬盧【盧貶新州見二百二十九卷建中四年】遇赦移吉州長史謂人曰吾必再入未幾上果用為饒州刺史【幾居豈翻】給事中袁高應草制執以白盧翰劉從一曰盧作相致鑾輿播遷海内瘡痍柰何遽遷大郡願相公執奏翰等不從更命它舍人草制【更工衡翻下更舍同】乙卯制出高執之不下【執之不肯書讀下戶嫁翻】且奏杞極惡窮凶百辟疾之若讎六軍思食其肉何可復用【復扶又翻】上不聽補闕陳京趙需等上疏曰杞三年擅權【建中二年盧為相四年貶】百揆失叙【書舜典納于百揆百揆時叙孔安國注曰舜舉八凱使揆度百事百事時叙無廢事業令云失叙謂事業廢也】天地神祗所知華夏蠻貊同弃儻加巨姦之寵必失萬姓之心丁巳袁高復于正牙論奏【唐謂大明官含光殿爲正牙亦謂之南牙】上曰已再更赦高曰赦者止原其罪不可為刺史陳京等亦争之不已曰之執政百官常如兵在其頸今復用之則姦黨皆唾掌而起上大怒左右辟易【辟讀曰闢易如字辟易言開遠而易其故處】諫者稍引却京顧曰趙需等勿退此國大事當以死争之上怒稍解戊午上謂宰相與小州刺史可乎李勉曰陛下欲與之雖大州亦可其如天下失望何壬戍以杞為灃州别駕使謂袁高曰朕徐思卿言誠為至當【當丁浪翻】又謂李泌曰朕已可袁高所奏泌曰累日外人竊議比陛下於桓靈今承德音乃堯舜之不逮也上悦杞竟卒於灃州高恕己之孫也【袁恕己與張束之等誅二張中宗復辟】 三月李希烈陷鄧州 戊午以汴滑節度使李澄為鄭滑節度使【汴州歸劉洽李澄得鄭州故以鄭滑節度授之也】以代宗女嘉城公主妻田緒【嘉城縣名隋置唐爲松州治所妻七細翻】李懷光都虞呂鳴岳密通欵於馬燧事泄懷光殺之屠其家事連幕僚高郢李鄘懷光集將士而責之郢鄘抗言逆順無所慙隱懷光囚之鄘邕之姪孫也【李邕以讒死于天寶之末】馬燧軍寶鼎敗懷光兵於陶城【敗補邁翻唐書地理志河中有陶城府酈道元曰陶城在蒲坂城西北即舜所都也舜陶河濱盖即此地與歷山相近按唐河中府治河東縣河東古蒲坂也】斬首萬餘級分兵會渾瑊逼河中 夏四月丁丑以曹王臯為荆南節度【節度之下當有使字】李希烈將李思登以隨州降之 壬午馬燧渾瑊破李懷光兵於長春宫南遂掘塹圍宫城懷光諸將相繼來降詔以燧瑊為招撫使 五月丙申劉洽更名玄佐【更工衡翻】 韓遊瓌請兵於渾瑊共取朝邑【朝直遥翻】李懷光將閻晏欲爭之士卒指邠軍曰彼非吾父兄則吾子弟【朔方軍分屯河中邠州故云然時韓遊瓌將邠軍以討李懷光】柰何以白刃相向乎語甚【囂喧也】晏遽引兵去懷光知衆心不從乃詐稱欲歸國聚貨財飾車馬云俟路通入貢由是得復踰旬月【史言李懷光偷延視息復扶又翻】 六月辛巳以劉玄佐兼汴州刺史 辛卯以金吾大將軍韋臯為西川節度使【為韋臯以功烈著于西南張本】 朱滔病死將士奉前涿州刺史劉怦知軍事【自朱滔得幽州滔每出兵皆以劉怦知留後事素得衆心故滔死而衆奉之怦普耕翻】時連年旱蝗【老子有言師之所聚荆棘生焉大兵之後必有凶年】度支資糧匱竭【度徒洛翻】言事者多請赦李懷光李晟上言赦懷光有五不可【晟成正翻上時掌翻】河中距長安纔三百里同州當其衝多兵則朱為示信少兵則不足隄防【少詩沼翻】忽驚東偏【同州在長安東北】何以制之一也今赦懷光必以晉絳慈隰還之渾瑊既無所詣康日知又應遷移【先已命渾瑊為蒲絳節度使康日知為晉慈隰節度使故云然渾戶昆翻又戶本翻瑊古咸翻】土宇不安何以奬勵二也陛下連兵一年討除小醜兵力未窮遽赦其反逆之罪今西有吐蕃北有回紇南有淮西【吐從暾入聲紇下沒翻李希烈時據淮西僭號故以之與二虜並言】皆觀我彊弱不謂施德澤愛黎元乃謂兵屈於人而自罷耳必競起窺覦之心三也【覦音俞】懷光既赦則朔方將士皆應叙勲行賞【謂解奉天圍勲賞也將即亮翻】今府庫方虛賞不滿望是愈激之使叛四也既解河中罷諸道兵賞典不舉怨言必起五也今河中斗米五百芻藁且盡牆壁之間餓殍甚衆【殍彼表翻】且軍中大將殺戮畧盡陛下但敇諸道圍守旬時彼必有内潰之變何必養腹心之疾為它日之悔哉又請發兵二萬自備資糧獨討懷光秋七月甲午朔馬燧自行營入朝奏稱懷光凶逆尤甚赦之無以令天下願更得一月糧必為陛下平之【燧音遂朝直遥翻必爲于偽翻 考異曰鄴侯家傳稱李泌語曰臣但恐梟于帳下太速何足憂也臣能為陛下取之上曰未諭卿意何故以太速為憂而卿能取也對曰臣為陛下憂不在河中乃在太原今馬燧亦蹭蹬矣領河東十萬之師遣王權領五千赴難及再幸梁洋遂抽歸本道男暢在奉天亦使北歸陛下更收復後宣慰云王權擅抽兵馬暢不扈從並宜釋放此則尤不安矣臣比年曾與之言甚有心路今之雄傑也若使之有異志則不比希烈朱泚之徒或能旰食伏望陛下聽臣之言援鞚遠馭以羈之上曰卿所欲何也對曰馬燧保有河東十餘州以待陛下還官此亦功也臣為常侍與燧兄炫同列然其兄弟素不相能其語無益臣重表兄鄭叔規為賓佐臣令以炫意請至京城欲與相見即至臣激燧令其取懷光自効必可致也因令燧為忠臣矣又曰貞元元年上因郊天改元時馬燧在太原遣其行軍司馬鄭叔規奏事請因鴻恩以雪懷光并致書與先公先公不與之報留其信物且令叔規語之曰比年展奉得接語言心期以爲大夫且河東節度以破靈曜之功上所與也奉天之難握十萬彊兵而令懷光解圍及懷光圖危社稷車駕幸梁洋逢此際會又令它人立盖代之功今聖主已歸宮闕懷光蹭蹬在于近畿旦夕為帳下所梟乃尸居也不速出軍收取以自解而快上心者即不及矣若河中既平公即如懷光之蹭蹬矣欲於滔俊之下作倔彊之臣亦必不成不言公才畧不及也緣腹中有三二百卷書蹭蹬至此必自内慙是進不立忠勲退不能効夷狄既而持疑則舟中帳下皆敵國矣可惜八尺之軀聲氣如鍾而心不果决乃婦人也著裙可矣欲奉答以裙衫而家累在冮東未至今聖上收復之後含垢匿瑕與人更始某又特蒙聽信已于上前保薦可使司徒以取懷光今弟來又請雪之大失所望且望弟速去為說若河中既平司徒何面目更來朝而與士人相見今雖請雪昨赦書亦許束身入朝矣若以建中冋征之故當一使諭之準赦歸朝必為保全如不奉詔當領全師問罪因速上表求自征之至河中輕騎入朝親禀廟畧乃天與之便也能如是當與司徒為中朝應接有須陳奏必聞聖聽若不能何敢有書也叔規既去具奏于上上每憂河中驍將達奚小俊等突犯官闕居常不安會東面苑牆忽有崩倒者上大驚以為有應之者將啟賊上顧問泌對曰此賊不足憂也乃猶杋上肉耳但恐梟懸太速不得與馬燧藉手為憂上曰古人云輕敵者亡今卿心輕敵如是朕甚憂之對曰陛下初經難危憂慮太過輕敵者亡誠如聖旨至如懷光豈可謂之敵乎陛下比在梁洋元惡據宮闕渠以朔方全軍在河中李晟在東渭橋此時可以傍助逆順之勢不然苟欲偷安脅為遲棊亦可而竟如醉如魘都不能動令陛下復歸宮闕又安足慮之哉臣伏計馬燧請討之章即至若以宗社之靈此賊且未為帳下所圖得河東軍有以藉手陛下無憂矣不喜于平懷光喜于得馬燧也既而馬燧表至請全軍南收河中仍自糧上大悦召先公對曰馬燧果請全軍討懷光來矣兼請至行營已來自備軍糧何其畏伏卿如此也對曰此乃畏伏天威而然于臣何有而能使其畏伏臣曾與之言諳其為人頗見機識勢今之雄傑也臣非故令叔規傳詞以激怒之且曰欲寄婦人之服當艱虞之時握十萬彊兵收復功在它人今聖主已還宮闕惟有懷光不速收取以立功自解它時復何面目至朝廷與公卿相見則蹭蹬之勢又不及懷光猶有解重圍之功料以此告之燧必能覺悟果得如此既以師至河中旬月當平而燧因此有功便為忠臣矣上曰當盡用卿言初叔規至太原具以先公言告燧燧搏䏶驚曰有是哉賴子之至京也不然燧幾為懷光矣非賢表兄豈有告燧者乎即日上表請行叔規又請如泌言先寫表本示懷光勸其束身歸朝彼必不從然後表請全軍往討則聖上信司徒誠心又可以忠義告四鄰不然朝救而夕請誅恐中外尤疑燧曰誠然乃令叔規草書寫表本馳驛以告懷光果不從於是乃請全軍南討尋發太原使者相繼奏事及與先公書言征討之謀及須上聞者先公因對皆為奏之又諷令下營訖輕騎由臨晉度朝謁燧皆然之七月乃自臨晉度夏陽來朝上大悦遂具告以先公言卿才畧必可使圖懷光初見卿請雪朕所未諭今乃果然比亦有人毀卿言詞百端聞於遠近惟先公保卿于朕朕信其言今見卿益知先公忠讜豁然體至誠奉國矣燧謝恩出而請先公至中書具說上言泣下拜謝後對上曰馬燧昨對其器質意趣固不易有且甚有心路感而用之必有成算皆如卿言信雄傑也按泌到長安數日即除常侍興元元年七月乙未也八月癸卯加燧晉慈隰節度使然則癸卯之前燧已取晉慈隰三州矣故朝廷命爲副元帥以討懷光十月已拔絳州及猗氏等諸縣矣貞元元年正月改元赦於時燧豈得猶在太原雪懷光邪自乙未至癸卯纔九日自長安至晉陽千餘里若因泌諷諭鄭叔規始來京師又令叔規還激勸燧又使燧以書諭懷光懷光不從然後上表興師伐之事多如此豈九日之内所能容也此直李繁欲取馬燧平河中之功皆歸于其父耳今從舊燧傳李肇國史補曰馬司徒面雪李懷光上曰惟卿不合雪人惶恐而退李公聞之請全軍自備資糧以討兇逆由此李馬不平邠志曰七月馬公朝于京師請赦懷光隴右節度李公聞之上表請 兵二萬獨討懷光芻糧之費軍中自備上以李公表示馬燧因曰朱泚之反不得已也懷光悖逆使朕再遷此而可赦何者爲罪馬公雨泣曰十日之内請獻其首上遣之按是時懷光垂亡燧功已成八九故自入朝爭之豈肯面雪懷光邪今從舊傳】上許之 陜虢都兵馬使達奚抱暉鴆殺節度使張勸代總軍務邀求旌節且隂召李懷光將達奚小俊為援上謂李泌曰若蒲陜連衡則猝不可制【蒲李懷光陜謂抱暉】且抱暉據陜則水陸之運皆絶矣【江淮水陸之運皆經陜州而後至長安】不得不煩卿一往辛丑以泌為陜虢都防禦水陸運使上欲以神策軍送泌之官問須幾何人對曰陜城三面懸絶攻之未可以歲月下也臣請以單騎入之上曰單騎如何可入對曰陜城之人不貫逆命【貫讀與慣同】此特抱暉為惡耳若以大兵臨之彼閉壁定矣臣今單騎抵其近郊彼舉大兵則非敵若遣小校來殺臣未必不更為臣用也【校戶敎翻】且今河東全軍屯安邑馬燧入朝願敕燧與臣同辭皆行使陜人欲加害於臣則畏河東移軍討之此亦一勢也【以形臨之謂之勢】上曰雖然朕方大用卿寜失陜州不可失卿當更使它人往耳對曰它人必不能入今事變之初衆心未定故可出其不意奪其姦謀它人猶豫遷延彼既成謀則不得前矣上許之泌見陜州進奏官及將吏在長安者【唐諸鎮皆置進奏院在長安以進奏官主之】語之曰主上以陜虢饑故不授泌節而領運使欲令督江淮米以賑之耳【語牛倨翻】陜州行營在夏縣【行營在夏縣亦以討河中也夏縣唐初屬虞州貞觀十七年屬絳州時屬陜州其地跨河之南北九域志夏縣在陜州北九十八里夏戶雅翻】若抱暉可用當使將之有功則賜旌節矣抱暉覘者馳告之【將即亮翻覘丑廉翻】抱暉稍自安泌具以語白上曰欲使其士卒思米抱暉思節必不害臣矣上曰善戊申泌與馬燧俱辭行庚戍加泌陜虢觀察使泌出潼關鄜坊節度使唐朝臣以步騎三千布于關外【朝臣時帶鄜坊節守潼關】曰奉密詔送公至陜泌曰辭日奉進止【自唐以來率以奉聖旨為奉進止盖言聖旨使之進則進使之止則止也程大昌曰今奏劄言取進止猶言此劄之或留或合稟承可否也唐中葉遂以處分為進止而不曉文義者習而不察槩謂有旨為進止如玉堂宣底所載凡宣旨皆云有進止者相承之誤也】以便宜從事此一人不可相躡而來來則吾不得入陜矣唐臣以受詔不敢去【唐臣當作朝臣】泌寫宣以却之【沈存中曰唐故事中書舍人職掌詔誥皆寫四本一本為底一本為宣此宣謂行出耳未以名書也晚唐樞密使自禁中受旨出付中書即謂之宣中書承受錄之於籍謂之宣底如今之聖語簿也余謂宣者因奉宣上旨而得名或以口傳為宣或以行文書為宣口傳為宣多命中臣而宰相亦有之劉栖楚之叩墀也牛僧孺宣曰所奏知門外俟進止此宰相之口宣也李泌寫宣以却還唐朝臣之兵此宰相行文書為宣也】因疾驅而前抱暉不使將佐出迎惟偵者相繼【偵丑鄭翻】泌宿曲沃將佐不俟抱暉之命來迎泌笑曰吾事濟矣去城十五里抱暉亦出謁泌稱其攝事保完城隍之功曰軍中煩言不足介意公等職事皆按堵如故抱暉出而喜泌既入城視事賓佐有請屏人白事者【屏必郢翻又卑正翻】泌曰易帥之際軍中煩言乃其常理【杜預注左傳曰煩言忿爭也余謂煩雜碎也此煩言謂雜碎之言帥所類翻】泌到自妥貼矣【史炤曰妥安也貼伏也亦作帖】不願聞也由是反仄者皆自安泌但索簿書治糧儲【索山客翻治直之翻】明日召抱暉至宅語之曰【宅者觀察所居也唐諸鎮將吏謂節度觀察所居者為使宅語牛倨翻】吾非愛汝而不誅恐自今有危疑之地朝廷所命將帥皆不能入故匄汝餘生汝為我齎版幣祭前使【為于偽翻前使謂張勸版以祝幣以燎】愼無入關自擇安處濳來取家保無它也泌之辭行也上籍陜將預于亂者七十五人授泌使誅之泌既遣抱暉日中宣慰使至泌奏已遣抱暉餘不足問上復遣中使至陜必使誅之【復扶又翻】泌不得已械兵馬使林滔等五人送京師懇請赦之詔謫戍天德【天德軍在振武東北宋白曰天寶八年張齊丘於可敦城置横塞軍十二年安思順奏廢横塞軍請于大同城西築城置軍玄宗賜名天安軍乾元後改為天德軍緣居人校少遂南移四里權居永清柵其城則隋大同城之故墟在牟郍山鉗耳胔之北】歲餘竟殺之而抱暉遂亡命不知所之達奚小俊引兵至境聞泌已入陜而還 壬辰以劉怦為幽州盧龍節度使 大旱灞滻將竭長安井皆無水度支奏中外經費纔支七旬<br />
<br />
  資治通鑑卷二百三十一  <br>
   </div> 

<script src="/search/ajaxskft.js"> </script>
 <div class="clear"></div>
<br>
<br>
 <!-- a.d-->

 <!--
<div class="info_share">
</div> 
-->
 <!--info_share--></div>   <!-- end info_content-->
  </div> <!-- end l-->

<div class="r">   <!--r-->



<div class="sidebar"  style="margin-bottom:2px;">

 
<div class="sidebar_title">工具类大全</div>
<div class="sidebar_info">
<strong><a href="http://www.guoxuedashi.com/lsditu/" target="_blank">历史地图</a></strong>  
<a href="http://www.880114.com/" target="_blank">英语宝典</a>  
<a href="http://www.guoxuedashi.com/13jing/" target="_blank">十三经检索</a> 
<br><strong><a href="http://www.guoxuedashi.com/gjtsjc/" target="_blank">古今图书集成</a></strong> 
<a href="http://www.guoxuedashi.com/duilian/" target="_blank">对联大全</a> <strong><a href="http://www.guoxuedashi.com/xiangxingzi/" target="_blank">象形文字典</a></strong> 

<br><a href="http://www.guoxuedashi.com/zixing/yanbian/">字形演变</a>  <strong><a href="http://www.guoxuemi.com/hafo/" target="_blank">哈佛燕京中文善本特藏</a></strong>
<br><strong><a href="http://www.guoxuedashi.com/csfz/" target="_blank">丛书&方志检索器</a></strong> <a href="http://www.guoxuedashi.com/yqjyy/" target="_blank">一切经音义</a>  

<br><strong><a href="http://www.guoxuedashi.com/jiapu/" target="_blank">家谱族谱查询</a></strong>  <strong><a href="http://shufa.guoxuedashi.com/sfzitie/" target="_blank">书法字帖欣赏</a></strong> 
<br>

</div>
</div>


<div class="sidebar" style="margin-bottom:0px;">

<font style="font-size:22px;line-height:32px">QQ交流群9:489193090</font>


<div class="sidebar_title">手机APP 扫描或点击</div>
<div class="sidebar_info">
<table>
<tr>
	<td width=160><a href="http://m.guoxuedashi.com/app/" target="_blank"><img src="/img/gxds-sj.png" width="140"  border="0" alt="国学大师手机版"></a></td>
	<td>
<a href="http://www.guoxuedashi.com/download/" target="_blank">app软件下载专区</a><br>
<a href="http://www.guoxuedashi.com/download/gxds.php" target="_blank">《国学大师》下载</a><br>
<a href="http://www.guoxuedashi.com/download/kxzd.php" target="_blank">《汉字宝典》下载</a><br>
<a href="http://www.guoxuedashi.com/download/scqbd.php" target="_blank">《诗词曲宝典》下载</a><br>
<a href="http://www.guoxuedashi.com/SiKuQuanShu/skqs.php" target="_blank">《四库全书》下载</a><br>
</td>
</tr>
</table>

</div>
</div>


<div class="sidebar2">
<center>


</center>
</div>

<div class="sidebar"  style="margin-bottom:2px;">
<div class="sidebar_title">网站使用教程</div>
<div class="sidebar_info">
<a href="http://www.guoxuedashi.com/help/gjsearch.php" target="_blank">如何在国学大师网下载古籍?</a><br>
<a href="http://www.guoxuedashi.com/zidian/bujian/bjjc.php" target="_blank">如何使用部件查字法快速查字?</a><br>
<a href="http://www.guoxuedashi.com/search/sjc.php" target="_blank">如何在指定的书籍中全文检索?</a><br>
<a href="http://www.guoxuedashi.com/search/skjc.php" target="_blank">如何找到一句话在《四库全书》哪一页?</a><br>
</div>
</div>


<div class="sidebar">
<div class="sidebar_title">热门书籍</div>
<div class="sidebar_info">
<a href="/so.php?sokey=%E8%B5%84%E6%B2%BB%E9%80%9A%E9%89%B4&kt=1">资治通鉴</a> <a href="/24shi/"><strong>二十四史</strong></a>&nbsp; <a href="/a2694/">野史</a>&nbsp; <a href="/SiKuQuanShu/"><strong>四库全书</strong></a>&nbsp;<a href="http://www.guoxuedashi.com/SiKuQuanShu/fanti/">繁体</a>
<br><a href="/so.php?sokey=%E7%BA%A2%E6%A5%BC%E6%A2%A6&kt=1">红楼梦</a> <a href="/a/1858x/">三国演义</a> <a href="/a/1038k/">水浒传</a> <a href="/a/1046t/">西游记</a> <a href="/a/1914o/">封神演义</a>
<br>
<a href="http://www.guoxuedashi.com/so.php?sokeygx=%E4%B8%87%E6%9C%89%E6%96%87%E5%BA%93&submit=&kt=1">万有文库</a> <a href="/a/780t/">古文观止</a> <a href="/a/1024l/">文心雕龙</a> <a href="/a/1704n/">全唐诗</a> <a href="/a/1705h/">全宋词</a>
<br><a href="http://www.guoxuedashi.com/so.php?sokeygx=%E7%99%BE%E8%A1%B2%E6%9C%AC%E4%BA%8C%E5%8D%81%E5%9B%9B%E5%8F%B2&submit=&kt=1"><strong>百衲本二十四史</strong></a>  <a href="http://www.guoxuedashi.com/so.php?sokeygx=%E5%8F%A4%E4%BB%8A%E5%9B%BE%E4%B9%A6%E9%9B%86%E6%88%90&submit=&kt=1"><strong>古今图书集成</strong></a>
<br>

<a href="http://www.guoxuedashi.com/so.php?sokeygx=%E4%B8%9B%E4%B9%A6%E9%9B%86%E6%88%90&submit=&kt=1">丛书集成</a> 
<a href="http://www.guoxuedashi.com/so.php?sokeygx=%E5%9B%9B%E9%83%A8%E4%B8%9B%E5%88%8A&submit=&kt=1"><strong>四部丛刊</strong></a>  
<a href="http://www.guoxuedashi.com/so.php?sokeygx=%E8%AF%B4%E6%96%87%E8%A7%A3%E5%AD%97&submit=&kt=1">說文解字</a> <a href="http://www.guoxuedashi.com/so.php?sokeygx=%E5%85%A8%E4%B8%8A%E5%8F%A4&submit=&kt=1">三国六朝文</a>
<br><a href="http://www.guoxuedashi.com/so.php?sokeytm=%E6%97%A5%E6%9C%AC%E5%86%85%E9%98%81%E6%96%87%E5%BA%93&submit=&kt=1"><strong>日本内阁文库</strong></a> <a href="http://www.guoxuedashi.com/so.php?sokeytm=%E5%9B%BD%E5%9B%BE%E6%96%B9%E5%BF%97%E5%90%88%E9%9B%86&ka=100&submit=">国图方志合集</a> <a href="http://www.guoxuedashi.com/so.php?sokeytm=%E5%90%84%E5%9C%B0%E6%96%B9%E5%BF%97&submit=&kt=1"><strong>各地方志</strong></a>

</div>
</div>


<div class="sidebar2">
<center>

</center>
</div>
<div class="sidebar greenbar">
<div class="sidebar_title green">四库全书</div>
<div class="sidebar_info">

《四库全书》是中国古代最大的丛书,编撰于乾隆年间,由纪昀等360多位高官、学者编撰,3800多人抄写,费时十三年编成。丛书分经、史、子、集四部,故名四库。共有3500多种书,7.9万卷,3.6万册,约8亿字,基本上囊括了古代所有图书,故称“全书”。<a href="http://www.guoxuedashi.com/SiKuQuanShu/">详细>>
</a>

</div> 
</div>

</div>  <!--end r-->

</div>
<!-- 内容区END --> 

<!-- 页脚开始 -->
<div class="shh">

</div>

<div class="w1180" style="margin-top:8px;">
<center><script src="http://www.guoxuedashi.com/img/plus.php?id=3"></script></center>
</div>
<div class="w1180 foot">
<a href="/b/thanks.php">特别致谢</a> | <a href="javascript:window.external.AddFavorite(document.location.href,document.title);">收藏本站</a> | <a href="#">欢迎投稿</a> | <a href="http://www.guoxuedashi.com/forum/">意见建议</a> | <a href="http://www.guoxuemi.com/">国学迷</a> | <a href="http://www.shuowen.net/">说文网</a><script language="javascript" type="text/javascript" src="https://js.users.51.la/17753172.js"></script><br />
  Copyright &copy; 国学大师 古典图书集成 All Rights Reserved.<br>
  
  <span style="font-size:14px">免责声明:本站非营利性站点,以方便网友为主,仅供学习研究。<br>内容由热心网友提供和网上收集,不保留版权。若侵犯了您的权益,来信即刪。scp168@qq.com</span>
  <br />
ICP证:<a href="http://www.beian.miit.gov.cn/" target="_blank">鲁ICP备19060063号</a></div>
<!-- 页脚END --> 
<script src="http://www.guoxuedashi.com/img/plus.php?id=22"></script>
<script src="http://www.guoxuedashi.com/img/tongji.js"></script>

</body>
</html>
