周纪一(起着雍摄提格(戊寅),尽玄黓困敦(壬子),凡三十五年。)

  〔尔雅:太岁在甲曰阏逢,在乙曰旃蒙,在丙曰柔兆,在丁曰强圉,在戊曰着雍,在己曰屠维,在庚曰上章,在辛曰重光,在壬曰玄黓,在癸曰昭阳,是为岁阳。在寅曰摄提格,在卯曰单阏,在辰曰执徐,在已曰大荒落,在午曰敦牂,在未曰协洽,在申曰涒滩,在酉曰作噩,在戌曰掩茂,在亥曰大渊献,在子曰困敦,在丑曰赤奋若,是为岁阴。周纪分注"起着雍摄提格",起戊寅也。"尽玄黓困敦",尽壬子也。阏,读如字;史记作"焉",于干翻。着,陈如翻。雍,于容翻。黓,逸职翻。单阏,上音丹,又特连翻;下乌葛翻,又于连翻。牂,作郎翻。涒,吐魂翻。滩,吐丹翻。困敦,音顿。杜预世族谱曰:周,黄帝之苗裔,姬姓。后稷之后,封于邰;及夏衰,稷子不窋窜于西戎。至十二代孙太王,避狄迁岐;至孙文王受命,武王克商而有天下。自武王至平王凡十三世,自平王至威烈王又十八世,自威烈王至赧王又五世。张守节曰:因太王居周原,国号曰周。地理志云:右扶风美阳县岐山西北中水乡,周太王所邑。括地志云:故周城一名美阳城,在雍州武功县西北二十五里。纪,理也,统理众事而系之年月。温公系年用春秋之法,因史、汉本纪而谓之纪。邰,汤来翻。夏,户雅翻。窋,竹律翻。在雍,于用翻。〕

  威烈王〔名午,考王之子。谥法:猛以刚果曰威;有功安民曰烈。沈约曰:诸复谥,有谥人,无谥法。〕

  二十三年(戊寅、前四○三)上距春秋获麟七十八年,距左传赵襄子惎智伯事七十一年。〔惎,毒也,音其冀翻。〕

  ①初命晋大夫魏斯、赵籍、韩虔为诸侯。〔此温公书法所由始也。魏之先,毕公高后,与周同姓;其苗裔曰毕万,始封于魏,至魏舒,始为晋正卿;三世至斯。赵之先,造父后;至叔带,始自周适晋;至赵夙,始封于耿。至赵盾,始为晋正卿,六世至籍。韩之先,出于周武王,至韩武子事晋,封于韩原。至韩厥,为晋正卿;六世至虔。三家者,世为晋大夫,于周则陪臣也。周室既衰,晋主夏盟,以尊王室,故命之为伯。三卿窃晋之权,暴蔑其君,剖分其国,此王法所必诛也。威烈王不惟不能诛之,又命之为诸侯,是崇奖奸名犯分之臣也,通鉴始于此,其所以谨名分欤!〕

  臣光曰:臣闻天子之职莫大于礼,礼莫大于分,分莫大于名。〔分,扶问翻;下同。〕何谓礼?纪纲是也。何谓分?君、臣是也。何谓名?公、侯、卿、大夫是也。

  夫以四海之广,〔夫以,音扶。〕兆民之众,受制于一人,虽有绝伦之力,高世之智,莫【章:十二行本"莫"下有"敢"字;乙十一行本同;孔本同。】不奔走而服役者,岂非以礼为之纪纲【章:十二行本,二字互乙;乙十一行本同;孔本同。】哉!是故天子统三公,〔统,他综翻。〕三公率诸侯,诸侯制卿大夫,卿大夫治士庶人。〔治,直之翻。〕贵以临贱,贱以承贵。上之使下犹心腹之运手足,根本之制支叶,下之事上犹手足之卫心腹,支叶之庇本根,然后能上下相保而国家治安。〔治,直吏翻。〕故曰天子之职莫大于礼也。

  文王序易,以乾、坤为首。孔子系之曰:"天尊地卑,乾坤定矣。卑高以陈,贵贱位矣。"〔系,户计翻。〕言君臣之位犹天地之不可易也。春秋抑诸侯,尊王【章:十二行本"王"作"周";乙十一行本同;孔本同;退齐校同。】室,王人虽微,序于诸侯之上,以是见圣人于君臣之际未尝不惓惓也。〔惓,逵员翻。汉刘向传:忠臣畎亩,犹不忘君惓惓之义也。惓惓,犹言勤勤也。〕非有桀、纣之暴,汤、武之仁,人归之,天命之,君臣之分当守节伏死而已矣。是故以微子而代纣则成汤配天矣,〔史记:商帝乙生三子:长曰微子启,次曰中衍,季曰纣。纣之母为后。帝乙欲立启为太子,太史据法争之曰:"有妻之子,不可立妾之子。"乃立纣。纣卒以暴虐亡殷国。孔〔郑〕玄义曰:物之大者莫若于天;推父比天,与之相配,行孝之大,莫大于此;所谓"严父莫大于配天"也。又孔氏曰:礼记称万物本乎天,人本乎祖。俱为其本,可以相配,故王者皆以祖配天。谥法:除残去虐曰汤。然谥法起于周;盖殷人先有此号,周人遂引以为谥法。分,扶问翻。长,知两翻。卒,子恤翻。〕以季札而君吴则太伯血食矣,〔吴王寿梦有子四人:长曰诸樊,次曰余祭,次曰余昧,次曰季札。季札贤,寿梦欲立之,季札让不可,于是立诸樊。诸樊卒,以授余祭,欲兄弟以次相传,必致国于季札;季札终让而逃之。其后诸樊之子光与余昧之子僚争国,至于夫差,吴遂以亡。宗庙之祭用牲,故曰血食。太伯,吴立国之君。范宁曰:太者,善大之称;伯者,长也。周太王之元子,故曰太伯。陆德明曰:寿梦,莫公翻。余祭,侧介翻。余昧,音末。〕然二子宁亡国而不为者,诚以礼之大节不可乱也。故曰礼莫大于分也。

  夫礼,辨贵贱,序亲疏,裁群物,制庶事,非名不着,非器不形;名以命之,器以别之,〔夫,音扶。别,彼列翻。〕然后上下粲然有伦,此礼之大经也。名器既亡,则礼安得独在哉!昔仲叔于奚有功于卫,辞邑而请繁缨,孔子以为不如多与之邑。惟名与器,不可以假人,君之所司也;政亡则国家从之。〔左传:卫孙桓子帅师与齐师战于新筑,卫师败绩。新筑人仲叔于奚救孙桓子,桓子是以免。既而卫人赏之邑,辞;请曲县、繁缨以朝,许之。孔子闻之曰:"不如多与之邑,惟名与器不可以假人。"繁缨,马饰也。繁,马鬣上饰;缨,马膺前饰。晋志注曰:缨在马膺如索帬。繁,音蒲官翻。缨,伊盈翻。索,昔各翻。〕卫君待孔子而为政,孔子欲先正名,以为名不正则民无所措手足。〔见论语。〕夫繁缨,小物也,而孔子惜之;正名,细务也,而孔子先之:〔先,悉荐翻。〕诚以名器既乱则上下无以相保故也。夫事未有不生于微而成于著,圣人之虑远,故能谨其微而治之。〔治,直之翻;下同。〕众人之识近,故必待其著而后救之;治其微则用力寡而功多,救其著则竭力而不能及也。易曰:"履霜坚冰至,"〔坤初六爻辞。象曰:"履霜坚冰,阴始凝也。驯致其道,至坚冰也。"〕书曰:"一日二日万几,"〔皋陶谟之辞。孔安国注曰:几,微也。言当戒惧万事之微。几,居依翻。〕谓此类也。故曰分莫大于名也。〔分,扶问翻。〕

  呜呼!幽、厉失德,周道日衰,纲纪散坏,下陵上替,诸侯专征,〔谓齐桓公,晋文公至悼公以及楚庄王、吴夫差之类。〕大夫擅政,〔谓晋六卿、鲁三家、齐田氏之类。〕礼之大体什丧七八矣,〔丧,息浪翻。〕然文、武之祀犹绵绵相属者,〔属,联属也,音之欲翻。凡联属之属皆同音。〕盖以周之子孙尚能守其名分故也。何以言之?昔晋文公有大功于王室,请隧于襄王,襄王不许,曰:"王章也。未有代德而有二王,亦叔父之所恶也。不然,叔父有地而隧,又何请焉!"文公于是惧而不敢违。〔太叔带之难,襄王出居于氾。晋文公帅师纳王,杀太叔带。既定襄王于郏,王劳之以地,辞;请隧焉,王弗许云云。杜预曰:阙地通路曰隧,此乃王者葬礼也。诸侯皆县柩而下。王章者,章显王者异于诸侯。古者天子谓同姓诸侯为伯父、叔父。隧,音遂。恶,乌路翻。难,乃旦翻。泛,音泛。劳,力到翻。阙,其月翻。县,音玄。柩,其久翻。〕是故以周之地则不大于曹、滕,以周之民则不众于邾、莒,〔曹、滕、邾、莒,春秋时小国。莒,居许翻。〕然历数百年,宗主天下,虽以晋、楚、齐、秦之强不敢加者,何哉?徒以名分尚存故也。至于季氏之于鲁,田常之于齐,白公之于楚,智伯之于晋,〔鲁大夫季氏,自季友以来,世执鲁国之政。季平子逐昭公,季康子逐哀公,然终身北面,不敢篡国。田常,即陈恒。田氏本陈氏;温公避国讳,改"恒"曰"常"。陈成子得齐国之政,杀阚止,弑简公,而亦不敢自立。史记世家以陈敬仲完为田敬仲完,陈成子恒为田常,故通鉴因以为据。白公胜杀楚令尹子西、司马子期,石乞曰:"焚库弑王,不然不济!"白公曰:"弑王不祥,焚库无聚。"智伯当晋之衰,专其国政,侵伐邻国,于晋大夫为最强;攻晋出公,出公道死。智伯欲并晋而不敢,乃奉哀公骄立之。〕其势皆足以逐君而自为,然而卒不敢者,〔卒,子恤翻,终也。〕岂其力不足而心不忍哉,乃畏奸名犯分而天下共诛之也。〔奸,居寒翻,亦犯也。分,扶问翻。〕今晋大夫暴蔑其君,剖分晋国,〔史记六国年表:定王十六年,赵、魏、韩灭智伯,遂三分晋国。〕天子既不能讨,又宠秩之,使列于诸侯,是区区之名分复不能守而并弃之也。〔陆德明经典释文:凡复字,其义训又者,并音扶又翻。〕先王之礼于斯尽矣!

  乌呼!君臣之礼既坏矣,〔此坏,其义为成坏之坏,读如字。〕则天下以智力相雄长,〔长,知两翻。〕遂使圣贤之后为诸侯者,社稷无不泯绝,〔谓齐、宋亡于田氏,鲁、陈、越亡于楚,郑亡于韩也。泯,弥忍翻,尽也,又弥邻翻。毛晃曰:没也,灭也。〕生民之类糜灭几尽,〔说文曰:糜,糁也;取糜烂之义,音忙皮翻。几,居依翻,又渠希翻,近也。〕岂不哀哉!

  或者以为当是之时,周室微弱,三晋强盛,〔三家分晋国,时因谓之"三晋",犹后之三秦、三齐也。〕虽欲勿许,其可得乎!是大不然。夫三晋虽强,苟不顾天下之诛而犯义侵礼,则不请于天子而自立矣。不请于天子而自立,则为悖逆之臣,〔夫,音扶。悖,蒲内翻,又蒲没翻。〕天下苟有桓、文之君,必奉礼义而征之。今请于天子而天子许之,是受天子之命而为诸侯也,谁得而讨之!故三晋之列于诸侯,非三晋之坏礼,乃天子自坏之也。〔坏,音怪,人毁之也。〕

  ②初,智宣子将以瑶为后,智果曰:"不如宵也。〔韦昭曰:智宣子,晋卿荀跞之子申也,瑶,宣子之子智伯也,谥曰襄子。智果,智氏之族也。宵,宣子之庶子也。按谥法:圣善周闻曰宣。智氏溢美也。〕瑶之贤于人者五,其不逮者一也。〔韦昭曰:不仁也。〕美鬓长大则贤,〔通鉴俗传写者多作"美须",非也。国语作"美鬓",今从之。〕【章:十二行本正作"鬓";孔本同。乙十一行本作"须"。】射御足力则贤,,伎艺毕给则贤,巧文辩惠则贤,〔韦昭曰:给,足也。巧文,行巧于文辞。伎,渠绮翻。〕强毅果敢则贤;如是而甚不仁。夫以其五贤陵人而以不仁之,其谁能待之?〔韦昭曰:待,犹假也。〕若果立瑶也,智宗必灭。"弗听。智果别族于太史,为辅氏。〔此事见国语。按左传哀公二十三年,晋荀瑶伐齐,始见于传。哀二十三年,史记元王五年也。荀跞,智文子也。定十四年,智文子犹见于传。智宣子之事,传无所考。立瑶之议,当在元王五年之前。韦昭曰:太史掌氏姓,周礼春官之属;小史掌定世系,辨昭穆。郑司农注云:史官主书,故韩宣子聘鲁,观书于太史。世系谓帝系、世本之属是也;小史主定之。贾公彦疏曰,注引太史证之者,太史史官之长,共其事故也。盖周之制,小史定姓氏,其书则太史掌之。智果欲避智氏之祸,故于太史别族。宋祁国语补音:别,彼列翻;又如字。〕

  赵简子之子,长曰伯鲁,幼曰无恤。〔赵简子,文子之孙鞅也。谥法:一德不懈曰简。白虎通曰:子,孳也,孳孳无已也。赵岐曰:子者,男子之通称也。长,知两翻。〕将置后,不知所立,乃书训戒之辞于二简,〔孔颖达曰:书者,舒也。书纬璇玑钤云:书者,如也。则书者,写其言如其意,得展舒也。世本曰:沮诵、苍颉作书。释文〔名〕曰:书,庶也,纪庶物也;亦言着也,着之简纸,求不灭也。简,竹策也。〕以授二子曰:"谨识之!"〔识,职吏翻,记也。〕三年而问之,伯鲁不能举其辞;求其简,已失之矣。问无恤,诵其辞甚习;〔习,熟也。〕求其简,出诸袖中而奏之。〔毛晃曰:奏,进上也。〕于是简子以无恤为贤,立以为后。

  简子使尹铎为晋阳,〔姓谱:尹,少昊之子,封于尹城,子孙因为氏。韦昭曰:晋阳,赵氏邑。为,治也。班志曰:晋阳,故诗唐国。周成王灭唐,封弟叔虞。龙山在西,晋水所出,东入汾。臣瓒曰:所谓唐,今河东永安县是也,去晋四百里。括地志曰:晋阳故城,今名晋城,在蒲州虞乡县西。今按水经注:晋水出晋阳县西龙山。昔智伯遏晋水以灌晋阳,其水分为二流,北渎即智氏故渠也。同过水出沾县北山,西过榆次县南,又西到晋阳县南。榆次县南水侧有凿台,战国策所谓"智伯死于凿台之下",即此处也。参而考之,晋阳故城恐不在蒲州。水经注又云:叔虞封于唐,县有晋水,故改名为晋。子夏序诗,"此晋也而谓之唐",是也,与班志合。瓒说及括地志未知何据。〕请曰:"以为茧丝乎?抑为保障乎?"简子曰:"保障哉!"〔茧丝,谓浚民之膏泽,如抽茧之绪,不尽则不止。保障,谓厚民之生,如筑堡以自障,愈培则愈厚。宋祁曰:障,之亮翻,又音章。〕尹铎损其户数。〔韦昭曰:损其户,则民优而税少。〕简子谓无恤曰:"晋国有难,而无以尹铎为少,〔而,汝也。难,乃旦翻,患也,厄也。少,音多少之少。重之为多,轻之为少。〕无以晋阳为远,必以为归。"

  及智宣子卒,〔卒,子恤翻。〕智襄子为政,〔谥法:有劳定国曰襄。为政,为晋国之政。〕与韩康子、魏桓子宴于蓝台。〔韩康子,韩宣子之曾孙庄子之子虔〔虎〕也。魏桓子,魏献子之子曼多之孙驹也。谥法:温柔好乐曰康;辟土服远曰桓。尔雅:四方而高曰台。〕智伯戏康子而侮段规。〔姓谱:段,郑共叔段之后。〕智国闻之,谏曰:"主不备难,【章:十二行本无"难"字;乙十一行本同。】难必至矣!"〔春秋以来,大夫之家臣谓大夫曰主。难,乃旦翻;下同。〕智伯曰:"难将由我。我不为难,谁敢兴之!"对曰:"不然。夏书有之:『一人三失,怨岂在明,不见是图。』〔书五子之歌之辞。夏,户雅翻。见,贤遍翻,发见也,着也,形也。〕夫君子能勤小物,故无大患。今主一宴而耻人之君相,〔夫,音扶。段规,韩康子之相也。相,息酱翻;下同。〕又弗备:曰『不敢兴难』,无乃不可乎!蚋、蚁、蜂、虿,皆能害人,〔宋祁曰:蚋,如锐翻;又字林:人劣翻。秦人谓蚊为蚋。今按:蚋,小虫,日中群集人之肌肤而嘬其血,蚊之类也。蜂,细腰而能螫人。虿亦毒虫,长尾,音丑迈翻。〕况君相乎!"弗听。

  智伯请地于韩康子,康子欲弗与。段规曰:"智伯好利而愎,不与,将伐我;不如与之。彼狃于得地,〔好,呼到翻。愎,弼力翻,狠也。狃,女九翻,骄,忲也,又相狎也。〕必请于他人;他人不与,必向之以兵,然后【章:十二行本,"后"作"则";乙十一行本同。】我得免于患而待事之变矣。"康子曰:"善"使使者致万家之邑于智伯。〔毛晃曰:邑,都邑。四井为邑,四邑为丘;邑方二里,丘方四里。载师以公邑之田任甸地,以家邑之田任稍地。注:公邑,谓六遂余地。家邑,大夫之采地。此又与四井之邑不同。又都,国都;邑,县也。左传:凡邑有先君宗庙之主曰都,无曰邑。邑曰筑,都曰城。此谓大县邑也。杜预引周礼"四县为都,四井为邑",恐误。四井之邑方二里,岂能容宗庙城郭!如论语"十室之邑",西都赋"都都相望,邑邑相属",则是四县四井之都邑也。若千室之邑、万家之邑,则非井邑矣。项安世曰:小司徒井牧田野,以四井为邑,凡三十六家;除公田四夫,凡三十二家;遂大夫会为邑者之政,以里为邑,凡二十五家。遂大夫盖论里井之制,二十五家共一里门,即六乡之二十五家为一闾也;小司徒盖论沟洫之制,四井为邑,共享一沟,即匠人所谓"井间广四尺深四尺谓之沟"也。居则度人之众寡,沟则度水之众寡,此其所以异欤!毛、项二说皆明周制,参而考之,战国之所谓邑非周制矣。致,送至也。〕智伯悦。又求地于魏桓子,桓子欲弗与。任章曰:"何故弗与?"〔任章,魏桓子之相也。姓谱:黄帝二十五子,十二人各以德为姓,第一曰任氏。又任为风姓之国,实太昊之后,主济祀,今济州任城即其地。任,市林翻。〕桓子曰:"无故索地,故弗与。"任章曰:"无故索地,诸大夫必惧;〔索,山客翻,求也〕,吾与之地,智伯必骄。彼骄而轻敌,此惧而相亲。以相亲之兵待轻敌之人,智氏之命必不长矣。周书曰:『将欲败之,必姑辅之。将欲取之,必姑与之。』〔逸书也。败,补迈翻。〕主不如与之,以骄智伯,然后可以择交而图智氏矣,奈何独以吾为智氏质乎!"〔质,脂利翻,物相缀当也。又质读如字,亦通。质,谓椹质也,质的也。椹质受斧,质的受矢。言智伯怒魏桓子,必加兵于魏,如椹质之受斧,质的之受矢也。〕桓子曰:"善。"复与之万家之邑一。〔复,扶又翻。〕

  智伯又求蔡、皋狼之地于赵襄子,〔康曰:皋,姑劳切;狼,卢当切;春秋蔡地,后为赵邑。余据春秋之时,晋、楚争盟,晋不能越郑而服蔡。三家分晋,韩得成皋,因以并郑,时蔡已为楚所灭,郑之南境亦入于楚,就使皋狼为蔡地,赵襄子安得而有之!汉书地理志西河郡有皋狼县,又有蔺县。汉之西河,春秋以来皆为晋境,而古文"蔺"字与"蔡"字近,或者"蔡"字其"蔺"字之讹也。〕襄子弗与。智伯怒,帅韩、魏之甲以攻赵氏。〔帅,读曰率。〕襄子将出,曰:"吾何走乎?"〔走,则豆翻,疾趋之也。趋,七喻翻。〕从者曰:"长子近,且城厚完。"〔从,才用翻。长子县,周史辛伯所封邑。班志属上党郡。陆德明曰:长子之长,丁丈翻。颜师古曰:长,读为短长之长;今读为长幼之长,非也。崔豹古今注曰:城,盛也,所以盛受民物也。淮南子曰:鲧作城。盛,时征翻。〕襄子曰:"民罢力以完之,〔罢,读曰疲。〕又毙死以守之,其谁与我!"〔韦昭曰:谓谁与我同力也。〕从者曰:"邯郸之仓库实。"〔邯郸,即春秋邯郸午之邑也。班志,邯郸县属赵国。张晏曰:邯郸山在东城下。单,尽也。城郭从邑,故旁加邑。宋白曰:邯郸本卫地,后属晋;七国时为赵都,赵敬侯自晋阳始都邯郸。余按史记六国年表,周安王之十六年,赵敬侯之元年;烈王之二年,赵成侯之元年。成侯二十二年,魏克邯郸,是年显王之十六年也。二十四年,魏归邯郸。若敬侯已都邯郸,魏克其国都而赵不亡,何也?至显王二十二年,公子范袭邯郸,不胜而死,是年肃侯之三年也。意此时赵方都邯郸,盖肃侯徙都,非敬侯也。邯,音寒。郸,音丹,康多寒切。〕襄子曰:"浚民之膏泽以实之,〔韦昭曰:浚,煎也,读曰醮。宋祁曰:浚,苏俊翻;醮,子召翻;余谓浚读当如宋音。浚者,疏瀹也,淘也,深也。〕又因而杀之,其谁与我!其晋阳乎,先主之所属也,〔古者诸侯之大夫,其家之臣子皆称之曰主,死则曰先主,考左传可见已。属,陟玉翻。〕尹铎之所宽也,民必和矣。"乃走晋阳。

  三家以国人围而灌之,城不浸者三版;〔高二尺为一版;三版,六尺。〕沈灶产鼁,民无叛意。〔沈,持林翻。颜师古汉书音义曰,鼃,黾也,似虾蟆而长脚,其色青。史游急就章曰:蛙,虾蟆。陆佃埤雅曰;鼁,似虾蟆而长踦,瞋目如怒。鼁,与蛙同,音下娲翻。〕智伯行水,〔据经典释文,凡巡行之行,音下孟翻;后仿此。〕魏桓子御,韩康子骖乘。〔兵车,尊者居左,执弓矢;御者居中;有力者居右,持矛以备倾侧,所谓车右是也。韩、魏畏智氏之强,一为之御,一为之右。骖,与参同,参者,三也。三人同车则曰骖乘,四人同车则曰驷乘。左传:齐伐晋,烛庸之越驷乘。杜预注曰:四人共乘者殿车。乘,石证翻。〕智伯曰:"吾乃今知水可以亡人国也。"桓子肘康子:康子履桓子之跗,以汾水可以灌安邑,绛水可以灌平阳也。〔跗,音夫,足趾也。班志:汾水出汾阳北山。汾阳县属太原郡,安邑县属河东郡。史记正义曰:安邑故城在绛州夏县东北十五里。应劭曰:绛水出河东绛县西南。平阳县亦属河东郡。安邑,魏绛始居邑。平阳,韩武子玄孙贞子始居之。桓、康二子之肘足接,盖各为都邑虑也。水经注曰:绛水出绛县西南,盖以故绛为言,其水出绛山东,西北流而合于浍,犹在绛县界中。智伯所谓"汾水可以灌安邑",或亦有之;"绛水可以灌平阳",未识所由。余谓自春秋之季至于元魏,历年滋多,郡县之离合,川谷之迁改,有不可以一时所睹为据者。史记正义曰:韩初都平阳,今晋州也。括地志曰:绛水一名白,今名沸泉,源出绛山,飞泉奋涌,扬波注县,积壑三十余丈,望之极为奇观,可接引北灌平阳城。郦道元父范,历仕三齐,少长齐地,熟其山川,后入关死于道,未尝至河东也。此盖因耳学而致疑。括地志成于唐之魏王泰,泰者,太宗之爱子,罗致天下一时名儒以作此书,其考据宜详,当取以为据。〕絺疵谓智伯曰:"韩、魏必反矣。"智伯曰:"子何以知之?"絺疵曰:"以人事知之。夫从韩、魏之兵以攻赵,赵亡,难必及韩、魏矣。〔夫,音扶。难,乃旦翻。〕今约胜赵而三分其地,城不没者三版,人马相食,城降有日,而二子无喜志,有忧色,是非反而何?"明日,智伯以絺疵之言告二子,二子曰:"此夫谗人欲为赵氏游说,使主疑于二家而懈于攻赵氏也。不然,夫二家岂不利朝夕分赵氏之田,而欲为危难不可成之事乎!"二子出,絺疵入曰:"主何以臣之言告二子也?"智伯曰:"子何以知之?"对曰:"臣见其视臣端而趋疾,知臣得其情故也。"智伯不悛,絺疵请使于齐。〔夫,音扶;余并同。难,乃旦翻。降,户江翻,下也,服也。说,输芮翻。懈,居隘翻,怠也。危难,如字。悛,丑缘翻,改也,止也。絺,抽迟翻,姓也。康曰:"絺"当作"郗",姓谱诸书未有从丝者,疑借字。余按姓谱:絺姓,周苏忿生支子,封于絺,因氏焉。为赵之为,音于伪翻。使,疏吏翻。疵请出使以避祸也。〕

  赵襄子使张孟谈潜出见二子,曰:"臣闻唇亡则齿寒。今智伯帅韩、魏以攻赵,赵亡则韩、魏为之次矣。"〔帅,读曰率。〕二子曰:"我心知其然也;恐事未遂而谋泄,则祸立至矣。"张孟谈曰:"谋出二主之口,入臣之耳,何伤也!"二子乃潜与张孟谈约,为之期日而遣之。〔姓谱:张氏本自轩辕第五子挥,始造弦,寔张网罗,世掌其职,后因氏焉。风俗传云:张、王、李、赵,黄帝所赐姓也。又晋有解张,字张侯,自此晋国有张氏。唐姓氏谱:张氏出自姬姓,黄帝子少昊青阳氏第五子挥正始制弓矢,子孙赐姓张。周宣王卿士张仲,其后裔事晋为大夫。〕襄子夜使人杀守堤之吏,而决水灌智伯军。智伯军救水而乱,韩、魏翼而击之,襄子将卒犯其前,〔将,即亮翻,又音如字。将,领也。卒,臧没翻。说文:吏人给事者衣为卒,卒衣有题识;其字从"衣"从"十"。〕大败智伯之众,〔以此败彼曰败。败,比迈翻。〕遂杀智伯,尽灭智氏之族。〔史记六国年表,三晋灭智氏在周定王十六年,上距获麟二十七年。皇甫谧曰:元王十一年癸未,三晋灭智伯。〕唯辅果在。〔以别族也。〕

  臣光曰:智伯之亡也,才胜德也。夫才与德异,而世俗莫之能辨,〔夫,音扶。〕通谓之贤,此其所以失人也。夫聪察强毅之谓才,正直中和之谓德。才者,德之资也,德者,才之帅也。〔夫,音扶。帅,所类翻。〕云梦之竹,天下之劲也;〔书禹贡:云土梦作乂。孔安国注云:云梦之泽在江南。左传:楚王以郑伯田江南之梦。杜预注云:楚之云梦跨江南北。班志:云梦泽在南郡华容县南。祝穆曰:据左传郧夫人弃子文于梦中,言梦而不言云,楚子避吴入于云中,言云而不言梦,则知云、梦二泽也。汉阳志:云在江之北,梦在江之南。又安陆有云梦泽,枝江有云梦城。盖古之云梦泽甚广,而后世悉为邑居聚落,故地之以云梦得名者非一处。竹箭之产,荆楚为良;云梦,楚之地也。梦,如字,又莫公翻。〕然而不矫揉,不羽括,则不能以入坚。〔矫,举夭翻。揉,如久翻。康曰:揉曲为矫,揉所以桡曲而使之直也。羽者,箭翎。括者,箭窟受弦处。括,音聒,通作"筈"。〕棠溪之金,天下之利也;〔左传:楚封吴夫概王于棠溪。战国之时,其地属韩,出金甚精利。刘昭郡国志:汝南郡吴房县有棠溪亭。杜佑通典曰:棠溪在今汝州郾城县界。九域志:蔡州有冶炉城,韩国铸剑之地。〕然而不镕范,不砥砺,则不能以击强。〔毛晃曰:镕,销也,铸也;说文:铸器法也。董仲舒传:犹金在镕。注:镕,谓铸器之模范。范,法也,式也。礼运:范金合土。土。砥,轸氏翻,柔石也。砺,力制翻,也。〕是故才德全尽谓之"圣人",才德兼亡谓之"愚人";德胜才谓之"君子",才胜德谓之"小人"。凡取人之术,苟不得圣人、君子而与之,与其得小人,不若得愚人。何则?君子挟才以为善,小人挟才以为恶。挟才以为善者,善无不至矣;挟才以为恶者,恶亦无不至矣。〔挟,檄颊翻。〕愚者虽欲为不善,智不能周,力不能胜,譬如乳狗搏人,人得而制之。〔挟,户颊翻。朱元晦曰:挟者,兼有而恃之之称。胜,音升。乳,儒遇翻,乳育也。乳狗,育子之狗也。搏,伯各翻。〕小人智足以遂其奸,勇足以决其暴,是虎而翼者也,其为害岂不多哉!〔虎而傅翼,其为害也愈甚。〕夫德者人之所严,〔严,敬也。〕而才者人之所爱;爱者易亲,严者易疏,〔易,以豉翻。〕是以察者多蔽于才而遗于德。自古昔以来,国之乱臣,家之败子,才有余而德不足,以至于颠覆者多矣,岂特智伯哉!故为国为家者苟能审于才德之分而知所先后,〔先,悉荐翻。后,户遘翻。〕又何失人之足患哉!

  ③三家分智氏之田。赵襄子漆智伯之头,以为饮器。〔说文:桼,木汁可以鬖物,下从水,象桼如水滴而下也。汉书张骞传:匈奴破月氏王,以其头为饮器。韦昭注曰:饮器,椑榼也。晋灼曰:饮器,虎子属也。或曰,饮酒之器也。师古曰:匈奴尝以月氏王头与汉使歃血盟,然则饮酒之器是也。韦云椑榼,晋云虎子,皆非也。椑榼,即今之偏榼,所以盛酒耳,非用饮者也。虎子,亵器,所以溲便者。椑,音鼙。榼,克合翻。氏,音支。使,疏吏翻。歃,色甲翻。盛,时征翻。亵,息列翻。溲,疏鸠翻。便,毗连翻。〕智伯之臣豫让欲为之报仇,〔豫,姓也。让,名也。战国之时又有豫且,不知其同时否也。为,音于伪翻;下同。〕乃诈为刑人,挟匕首,入襄子宫中涂厕。〔挟,持也。刘向曰:匕首,短剑。盐铁论曰:匕首长尺八寸;头类匕,故云匕首。匕,音比。厕,初吏翻,圊也。长,直亮翻。〕襄子如厕心动,索之,获豫让。〔索,山客翻。〕左右欲杀之,襄子曰:"智伯死无后,而此人欲为报仇,真义士也,吾谨避之耳。"乃舍之。〔舍,读曰舍。〕

  豫让又漆身为癞,吞炭为哑。〔癞,落盖翻,恶疾也。哑倚下翻,瘖也。〕行乞于市,〔神农日中为市,致天下之民,聚天下之货,交易而退,此立市之始也,郑氏周礼注曰:市,杂聚之处。〕其妻不识也。行见其友,其友识之,为之泣曰:"以子之才,臣事赵孟,必得近幸,〔自春秋之时,赵宜子谓之宣孟,赵文子谓之赵孟,其后遂袭而呼为赵孟。孟,长也。〕子乃为所欲为,顾不易邪?〔易,以豉翻。〕何乃自苦如此?求以报仇,不亦难乎!"豫让曰:【章:十二行本"曰"下有"不可"二字;乙十一行本同;孔本同;张校同;退斋校同。】"既已委质为臣,〔经典释文曰:质,职日翻。委质,委其体以事君也。后汉书注:委质,屈膝。〕而又求杀之,是二心也。凡吾所为者,极难耳。然所以为此者,将以愧天下后世之为人臣怀二心者也。"襄子出,豫让伏于桥下。襄子至桥,马惊;索之,得豫让,遂杀之。〔自智宣子立瑶,至豫让报仇,其事皆在威烈王二十三年之前,故先以「初」字发之。温公之意,盖以天下莫大于名分,观命三大夫为诸侯之事,则知周之所以益微,七雄之所以益盛;莫重于宗社,观智、赵立后之事,则知智宣子之所以失,赵简子之所以得;君臣之义当守节伏死而已,观豫让之事,则知策名委质者必有霣而无贰。其为后世之鉴,岂不昭昭也哉〕!

  襄子为伯鲁之不立也,有子五人,不肯置后。封伯鲁之子于代,〔代国在夏屋句注之北,赵襄子灭之。班志有代郡代县。为,于伪翻。夏,户雅翻。〕曰代成君,早卒;〔成,谥也。谥法:安民立政曰成。〕立其子浣为赵氏后。〔浣,户管翻。〕襄子卒,弟桓子逐浣而自立;〔史记六国表,威烈王元年,襄子卒;二年,赵桓子元年,卒;明年,国人立献侯浣。"浣",索隐作"晚"。卒,子恤翻;下同。〕一年卒。赵氏之人曰:"桓子立非襄主意。"乃共杀其子,复迎浣而立之,是为献子。〔复,扶又翻;又音如字。献子,即献侯。六国表:威烈王三年,献侯之元年。盖分晋之后,三晋僭侯久矣。谥法:知质有圣曰献。〕献子生籍,是为烈侯。〔谥法:有功安民曰烈,秉德尊业曰烈。〕

  ④魏斯者,魏桓子之孙也,是为文侯。〔谥法:学勤好问曰文;慈惠安民曰文。〕韩康子生武子;武子生虔,是为景侯。〔谥法;克定祸乱曰武;布义行刚曰景。六国表:威烈王二年,魏文侯斯元年;十八年,韩景侯虔元年。盖其在国僭爵已久,不敢以通王室;威烈王遂因而命之,识者重为周惜。通鉴于此序三家之世也。〕

  ⑤魏文侯以卜子夏、田子方为师。〔卜,以官为氏。田本出于陈,陈敬仲以陈为田氏。徐广曰:始食采地,由是改姓田氏。索隐曰:陈、田二声相近,遂为田氏。夏,户雅翻。〕每过段干木之庐必式。〔过,工禾翻。唐人志氏族曰:李耳,字伯阳,一字聃;其后有李宗,魏封于段,为干木大夫,是以段为氏也。余按:通鉴赧王四十二年,魏有段干子,则段干,复姓也。书:武王式商容闾。注云:式其闾巷,以礼贤。记曲礼:国君抚式,士下之。注云:升车必正立,据式小俛,崇敬也。师古曰:式,车前横木。古者立乘;凡言式车者,谓俛首抚式,以礼敬人。孔颖达曰:式,谓俯下头也。古者车箱长四尺四寸而三分,前一后二,横一木,下去车床三尺三寸,谓之为式;又于式上二尺二寸横一木,谓之较,较去车床凡五尺五寸。于时立乘,若平常则凭较,故诗云"倚重较兮"是也。若应为敬,则落隐下式,而头得俯俛,故记云"式视马尾"是也。较,讫岳翻。〕四方贤士多归之。

  文侯与群臣饮酒,乐,而天雨,命驾将适野。左右曰:"今日饮酒乐,天又雨,君将安之?"文侯曰:"吾与虞人期猎,虽乐,岂可无一会期哉?"乃往,身自罢之。〔周礼有山虞、泽虞,以掌山泽。注云:虞,度也,度佑山林之大小及其所生。身自罢之者,身往告之,以雨而罢猎也。乐,音洛。〕

  韩借师于魏以伐赵,文侯曰:"寡人与赵,兄弟也,不敢闻命。"赵借师于魏以伐韩,文侯应之亦然。二国皆怒而去。已而知文侯以讲于己也,〔讲,和也。〕皆朝于魏。〔朝,直遥翻。〕魏于是始大于三晋,诸侯莫能与之争。

  使乐羊伐中山,克之;〔乐,姓也。本自有殷微子之后。宋戴公四世孙乐吕为大司寇。中山,春秋之鲜虞也,汉为中山郡。宋白曰:唐定州,春秋白狄鲜虞之地。隋图经曰:中山城在今唐昌县东北三十一里,中山故城是也。杜佑曰:城中有山,故曰中山。〕以封其子击。文侯问于群臣曰:"我何如主?"皆曰:"仁君。"任座曰:"君得中山,不以封君之弟而以封君之子,何谓仁君!"文侯怒,任座趋出。〔任座亦习见当时邻国之事而为是言耳。任音壬,"座"一作"痤",音才戈翻。〕次问翟璜,〔翟,姓也,音直格翻,又音狄。姓谱:翟为晋所灭,子孙以国为氏。今人多读从上音。璜,户光翻。〕对曰:"仁君。"文侯曰:"何以知之?"对曰:"臣闻君仁则臣直。向者任座之言直,臣是以知之。"文侯悦,使翟璜召任座而反之,亲下堂迎之,以为上客。

  文侯与田子方饮,文侯曰:"钟声不比乎?〔比,音毗。不比,言不和也。〕左高。"〔此盖编钟之悬,左高,故其声不和。〕田子方笑。文侯曰:"何笑?"子方曰:"臣闻之,君明乐官,不明乐音,今君审于音,臣恐其聋于官也。"〔明乐官,知其才不才;明乐音,知其和不和。五声合和,然后成音。诗大序曰:声成文,谓之音。〕文侯曰:"善。"

  子击出,遭田子方于道,下车伏谒。〔古文〔史〕考曰:黄帝作车,引重致远;少昊氏加牛;禹时奚仲加马。释名曰:车,居也。韦昭曰:古唯尺遮翻,自汉以来,始有"居"音。萧子显曰:三皇氏乘祗车出谷口,车之始也。祗,翘移翻。〕子方不为礼。子击怒,谓子方曰:"富贵者骄人乎?贫贱者骄人乎?"子方曰:"亦贫贱者骄人耳,富贵者安敢骄人!国君而骄人则失其国,大夫而骄人则失其家。失其国者未闻有以国待之者也,失其家者未闻有以家待之者也。夫士贫贱者,言不用,行不合则纳履而去耳,安往而不得贫贱哉!"子击乃谢之。〔夫,音扶。行,下孟翻。〕

  文侯谓李克曰:"先生尝有言曰:『家贫思良妻,国乱思良相。』今所置非成则璜,二子何如?"〔李氏出自颛顼曾孙皋陶,为尧大理,以官命族为理氏。商纣时,裔孙利贞逃难,食木子得全,改为李氏。置,言置相也。相,息亮翻。难,乃旦翻。〕对曰:"卑不谋尊,疏不谋戚。臣在阙门之外,不敢当命。"〔在阙门之外,谓疏远也。〕文侯曰:"先生临事勿让!"克曰:"君弗察故也。居视其所亲,富视其所与,达视其所举,穷视其所不为,贫视其所不取,五者足以定之矣,何待克哉!"文侯曰:"先生就舍,吾之相定矣。"〔相,息亮翻。〕李克出,见翟璜。翟璜曰:"今者闻君召先生而卜相,果谁为之?"克曰:"魏成。"翟璜忿然作色曰:"西河守吴起,臣所进也。〔班志;魏地,其界自高陵以东,尽河东、河内。高陵县,汉属冯翊,其地在河西,所谓"西河之外"者也。魏初使吴起守之,秦兵不敢东向。至惠王时,秦使卫鞅击虏其将公子昂,遂献西河之外于秦。吴,以国为姓。相,息亮翻,守,式又翻。〕君内以邺为忧,臣进西门豹。〔班志,邺县属魏郡。西门豹为邺令,凿渠以利民。王符潜夫论姓氏篇曰:如有东门、西郭、南宫、北郭,皆因居以为姓。西门盖亦此类。邺,鱼怯翻。〕君欲伐中山,臣进乐羊。中山已拔,无使守之,臣进先生。君之子无傅,臣进屈侯鲋。〔傅者,傅之以德义,因以为官名。傅,芳遇翻。屈,九勿翻,姓也。余按屈,晋地,时属魏;鲋盖魏封屈侯也。鲋,音符遇翻。〕以耳目之所睹记,臣何负于魏成!"〔不胜为负。〕李克曰:"子【章:十二行本"子"下有"之"字;乙十一行本同;孔本同。】言克于子之君者,岂将比周以求大官哉?〔比,毗至翻。阿党为比。〕君问相于克,克之对如是。〔李克自叙其答魏文侯之言也。〕所以知君之必相魏成者,魏成食禄千钟,〔孔颖达曰:禄者,谷也。故郑注司禄云:禄也言榖,年榖丰然后制禄。授神契云:禄者,录也。白虎通曰:上以收录接下,下以名录谨以事上是也。六斛四斗为一钟。〕什九在外,什一在内;是以东得卜子夏、田子方、段干木。〔夏,户雅翻。〕此三人者,君皆师之:子所进五人者,君皆臣之。子恶得与魏成比也!"〔恶,读曰乌,何也。〕翟璜逡巡再拜曰:"璜,鄙人也,失对。愿卒为弟子,"愿卒为子弟!〔逡,七伦翻。逡巡,却退貌。卒,子恤翻,终也。孔颖达曰:先生,师也。谓师为先生者,言彼先己而生,其德多厚也。自称为弟子者,言己自处如弟子,则尊其师如父兄也。〕

  吴起者,卫人,仕于鲁。齐人伐鲁,鲁人欲以为将,起取齐女为妻,〔将,即亮翻;下同。取,读曰娶。孔颖达曰:妻之为言齐也;以礼见问,得与夫敌体也。〕鲁人疑之,起杀妻以求将,大破齐师。或谮之鲁侯曰:"起始事曾参,〔世本曰:曾姓出自郐国。陆德明曰:参,所金翻,一音七南翻。〕母死不奔丧,曾参绝之;今又杀妻以求为君将。起,残忍薄行人也!〔行,下孟翻。〕且以鲁国区区而有胜敌之名,则诸侯图鲁矣。"起恐得罪,闻魏文侯贤,乃往归之。文侯问诸李克,李克曰:"起贪而好色;〔死好,呼到翻。〕然用兵,司马穰苴弗能过也。"〔司马,官名。穰苴本齐田姓,仕齐为是官,故以称之;齐景公之贤将也。穰,如羊翻。苴,子余翻。〕于是文侯以为将,击秦,拔五城。

  起之为将,与士卒最下者同衣食,卧不设席,行不骑乘,〔骑马为骑,乘车为乘,言起与士卒同其劳苦,行不用车马也。〕亲裹赢粮,〔师古曰:赢,担也。此言起亲裹士卒所赍担之粮。赢,怡成翻。〕与士卒分劳苦。卒有病疽者,起为吮之。〔疽,七余翻,痈也。吮,徐兖翻;说文:嗽也,康所角切。〕卒母闻而哭之。人曰:"子,卒也,而将军自吮其疽,何哭为?"母曰:"非然也。往年吴公吮其父疽,【章:十二行本无"疽"字;乙十一行本同;孔本同。】其父战不旋踵,遂死于敌。吴公今又吮其子,妾不知其死所矣,是以哭之。"

  ⑥燕愍公薨,子僖公立。〔燕自召公奭受封于北燕,其地则唐幽州蓟县故城是也。自召公至愍公三十二世。燕,因肩翻。愍,读与闵同。谥法:使民悲伤曰闵;小心畏忌曰僖。〕

  二十四年(己卯、前四○二)

  ①王崩,子安王骄立。

  ②盗杀楚声王,国人立其子悼王。〔周成王封熊绎于楚,姓芈氏,居丹阳,今枝江县故丹阳城是也。括地志曰:归州秭归县丹阳城,熊绎之始国。其后强大,北封畛于汝,南并吴、越,地方五千里。自熊绎至声王三十世。索隐曰:声王,名当。悼王,名疑。諡法:不生其国曰声。注云:生于外家。年中早夭曰悼。注云:年不称志。又云:恐惧从处曰悼。注云:从处,言险圮也。〕

  安王〔諡法,好和不争曰安。〕

  元年(庚辰、前四○一)

  ①秦伐魏,至阳孤。〔周孝王邑非子于秦。徐广曰:今陇西县秦亭是也。括地志曰:秦州清水县本名秦。十三州志曰:秦亭,秦谷是也。至襄公取周地,穆公霸西戎,日以强大。是年,秦简公之十四年也。自非子至简公二十八世。"阳孤",史记作"阳狐"。【章:乙十一行本正作"狐"。】正义引括地志曰:阳狐郭在魏州元城县东北三十里。余按此时西河之外皆为魏境,若秦兵至元城,则是越魏都安邑而东矣。水经注:河东垣县有阳壶城。九域志:绛州有阳壶城。九域志:绛州有阳壶城识之以广异闻,且俟知者。〕

  二年(辛巳,公元前四零零年)

  ①魏、韩、赵伐楚,至桑丘。〔水经注:澺水自葛陂东南迳新蔡县故城东,而东南流注于汝水;又东南迳下桑里,左迤为横塘陂。史记作"乘丘"。正义:地理志,乘丘故城在兖州瑕丘县西北三十五里。当从之。〕

  ②郑围韩阳翟。〔周宣王封其弟友于郑。杜预世族谱曰:封于咸 林,今京兆郑邑是也。幽王无道,友徙其人于虢、郐间,遂有其地,今河南新郑是也。友,諡桓公。是年,郑繻公骀之二十三年。自桓公至繻公二十二世。班志,阳翟县属颍川郡。索隐曰:翟,音狄,温公类篇音苌伯切。繻,询趋翻。骀,堂来翻。〕

  ③韩景侯薨,子烈侯取立。

  ④赵烈侯薨,国人立其弟武侯。

  ⑤秦简公薨,子惠公立。〔諡法:爱民好与曰惠。〕

  三年(壬午,公元前三九九年)

  ①王子定奔晋。

  ②虢山崩,壅河。〔徐广曰:虢山在陕。裴駰曰:弘农陕县,故虢国。北虢在大阳,东虢在荥阳。括地志曰:虢山在陜州陕县,西临黄河;今临河有冈阜,似是颓山之余。水经注曰:陜城西北带河,水涌起方数十丈。父老云:石虎载铜翁仲至此沈没,水所以涌。洪河巨渎,宜不为金狄梗流,盖魏文侯时虢山崩壅河所致耳。陕,失冉翻。〕

  四年(癸未,公元前三九八年)

  ①楚围郑。郑人杀其相驷子阳。〔郑穆公之子騑,字子驷;古者以王父之字为氏,子阳其后也。相,息亮翻。騑,芳菲翻。〕

  五年(甲申,公元前三九七年)

  ①日有食之。〔杜预曰:日行迟,一岁一周天。月行速,一月一周天;一岁凡十二交会。然日、月,动物,虽行度有大量,不能不小有赢缩,故有虽交会而不食者,或有频交而食者。孔颖达曰:日月交会,谓朔也。周天三百六十五度四分度之一。日月皆右行于天,一昼一夜,日行一度,月行十三度十九分度之七,二十九日日有余,而月行天一周,追及于日而与之会。交会而日月同道则食;月或在日道表,或在日道里,则不食矣。又历家为交食之法,大率以一百七十有三日有奇为限。然月先在里,则依限而食者多;若月在表,则依限而食者少。杜预见其参差,乃云"虽行度有大量,不能不小有赢缩,故有虽交会而不食者,或有频交而食者",此得之矣。苏氏曰:交当朔则日食,然亦有交而不食者。交而食,阳微而阴乘之也;交而不食,阳盛而阴不能揜也。朱元晦曰:此则系乎人事之感。盖臣子背君父,妾妇乘其夫,小人陵君子,夷狄侵中国,所感如是,则阴盛阳微而日为之食矣。是以圣人于春秋,每食必书,而诗人亦以为丑也。今此书年而不书月与晦、朔,史失之也。释名曰:日、月亏曰食;稍小侵亏,如虫食草木之叶也。亦作"蚀"。〕

  ②三月,盗杀韩相侠累。侠累与濮阳严仲子有恶。仲子闻轵人聂政之勇,以黄金百溢为政母寿,欲因以报仇。〔相,息亮翻。侠,户颊翻。累,力追翻。濮阳,春秋之帝丘,汉为濮阳县,属东郡。应劭曰:濮水南入巨野。水北为 阳。濮,博木翻。恶,如字,不善也;康乌故切,非。轵,春秋原邑,晋文公所围者;汉为轵县,属河内郡;音只。姓谱曰:楚大夫食采于聂,因以为氏。聂,尼辄 翻。溢,夷质翻。二十四两为溢。〕政不受,曰:"老母在,政身未敢以许人也!"及母卒,仲子乃使政刺侠累。〔卒,子恤翻。刺,七亦翻,又如字。〕侠累方坐 府上,兵卫甚众,聂政直入上阶,〔上,时掌翻。〕刺杀侠累,因自皮面决【章:乙十一行本作扶】。眼,自屠出肠。韩人暴其尸于市,〔暴,步木翻,又音如字,露也。〕购问,莫能识。其姊嫈闻而往,哭之曰:"是轵深井里聂政也!〔史记正义曰:深井里在怀州济源县南三十里。〕以妾尚在之故,重自刑以绝从。妾奈何畏殁身之诛,终灭贤弟之名!"遂死于政尸之旁。〔皮面,以刀剺面而去其皮。悬赏以募告者曰购。购,古侯翻。嫈,乌茎翻。绝从之从,读曰踪,谓自绝其踪迹。或曰:从,读如字,谓绝其从坐之罪也。〕

  六年(乙酉,公元前三九六年)

  ①郑驷子阳之党弑繻公,〔繻者,諡法所不载。史记注:"繻",或作"缭"。繻,询趋翻。〕而立其弟乙,〔白虎通曰:弟,悌也,心顺、行笃也。行,下孟翻。〕是为康公。

  ②宋悼公薨,子休公田立。〔武王封微子启于宋,唐宋州之脽阳县是也。自微子二十七世至悼公,名购由。休,亦諡法所不载。〕

  八年(丁亥,公元前三九四年)

  ①齐伐鲁,取最。【章:十二行本"最"下有"韩救鲁"三字;乙十一行本同;孔本同;张校同;退斋校同。】〔武王封太公于齐,唐青州之临淄是也。括地志曰:天齐水在临淄东南十五里。封禅书曰:齐之所以为齐者,以天齐。是年,康公贷之十一年。自太公至康公二十九世。成王封伯禽于鲁,唐兖州之曲阜是也。是年,穆公之十六年。自伯禽至穆公凡二十八世。〕

  ②郑负黍叛,复归韩。〔据史记,繻公之十六年,败韩于负黍,盖以此时取之,而今复叛归韩也。刘昭郡国志:颖川郡阳城县有负黍聚。古今地名云;负黍山在阳城县西南二十七里,或云在西南三十五里。〕

  九年(戊子,公元前三九三年)

  ①魏伐郑。

  ②晋烈公薨,子孝公倾立。〔周成王封弟叔虞于唐。括地志曰:故唐城在并州晋阳县北二里,尧所筑也。都城记曰:唐叔 虞之子燮父徙居晋水旁,今并州理故唐城,即燮父初徙之处;其城南半入州城中。毛诗谱曰:燮父以尧墟南有晋水,改曰晋侯。自唐叔至烈公三十七世。烈公,名 止。諡法:慈惠爱亲曰孝。〕

  十一年(庚寅,公元前三九一年)

  ①秦伐韩宜阳,取六邑。〔班志,宜阳县属弘农郡。史记正义曰:宜阳县故城,在河南府福昌县东十四里,故韩城是也。此邑即周礼"四井为邑"之邑。〕

  ②初,田常生襄子盘,盘生庄子白,白生太公和。〔此序齐田氏之世也。田常,即左传陈成子恒也。温公避仁庙讳,改"恒"曰"常"。自陈公子完奔齐,五世至常得政。諡法:胜敌志强曰庄。〕是岁,齐田和迁齐康公于海上,使食一城,以奉其先祀。

  十二年(辛卯,公元前三九零年)

  ①秦、晋战于武城。〔此非鲁之武城。左传:晋阴饴甥会秦伯,盟于王城。杜预曰:冯翊临晋县东有王城,今名武乡。括地志:故武城,一名武平城,在华州郑县东北十三里。〕

  ②齐伐魏,取襄阳。〔"阳",当作"陵"。徐广曰:今之南平阳也。余据晋志,南平阳县属山阳郡。班志,陈留郡有襄邑县。师古曰:圈称云:襄邑,宋地,本承匡襄陵乡也,宋襄公所葬,故曰襄陵。秦始皇以承匡卑湿,徙县襄陵,因曰襄邑。〕

  ③鲁败齐师于平陆。〔班志,东平国有东平陆县,战国时之平陆也。史记正义曰:平陆,兖州县,即古厥国。宋白曰:郓州中都县,汉为平陆县,史记"鲁败齐师于平陆"是也。败,补迈翻。〕

  十三年(壬辰,公元前三八九年)

  ①秦侵晋。

  ②齐田和会魏文侯、楚人、卫人于浊泽,〔康曰,浊,水名;汉志:浊水出齐郡广县妫山。余谓康说误矣。徐广史记注曰:长社有浊泽。水经注曰:皇陂水出胡城西北。胡城,颍阴之狐人亭也。皇陂,古长社之浊泽也。记:诸侯相见于郤地曰会。孔颖达曰:诸侯未及期而相见曰遇。会者,谓及期之礼,既及期,又至所期之地。〕求为诸侯。魏文侯为之请于王及诸侯,王许之。〔为之之为,于伪翻。〕

  十五年(甲午,前三八七年)

  ①秦伐蜀,取南郑。〔谱记普〔疑衍〕云:蜀之先,肇自人皇之际。黄帝子昌意娶蜀山氏女,生帝俈。既立,封其支庶于蜀,历虞、夏、商、周。周衰,先称王者蚕丛。余据武王伐纣,庸、蜀诸国皆会于牧野。孔安国曰:蜀,叟也,春秋之时不与中国通。班志,南郑县属汉中郡,唐为梁州治所。"俈",通作"喾",音括沃翻。〕

  ②魏文侯薨,太子击立,〔王者以嫡长子为太子,谓之国储副君。诸侯曰世子。周衰,率上僭。孔颖达曰:太者,大中之大也。上,时掌翻。长,知两翻。〕是为武侯。

  武侯浮西河而下,〔西河,即禹贡之"龙门西河"。〕中流顾谓吴起曰: "美哉山河之固,此魏国之宝也!"对曰:"在德不在险。昔三苗氏,左洞庭,右彭蠡,德义不修,禹灭之。〔武陵、长沙、零、桂之水,汇为洞庭,周七百里。彭蠡泽在汉豫章郡彭泽县西。书:有苗弗率,汝徂征。三苗所居,盖今江南西道之地。蠡,里弟翻。〕夏桀之居,左河济,右泰华,伊阙在其南,羊肠在其北;修政不仁,汤放之。〔济水出河东垣县王屋山,南流贯河而南,合于荥渎。禹贡所谓"导沇水,东流为济,溢为荥"者也。自汉筑荥阳石门,而济与河合流而注于海,不入荥渎。禹贡所谓"导沇水,东流为济,入于河"。桀都安邑,盖恃以为险。泰华山在京兆华阴县南。水经:伊水出南阳县西荀渠山,东北流至河南新城县,又东南过伊阙中,大禹所凿也。两山相对,望之若阙。左传"女宽守阙塞",即其地。括地志:伊阙山在洛州南十九里。班志,上党壶关县有羊肠阪。此安邑四履所凭,山河之固也。书曰:成汤放桀于南巢。济,子礼翻。华,户化翻。〕商纣之国,左孟门,右太行,常山在其北,大河经其南;修政不德,武王杀之。〔水经注:孟门在河东北屈县西,即龙门上口也。淮南子曰:龙门未辟,吕梁未凿,河出孟门之上,溢而逆流,无有丘陵,名曰洪水。太行山在河内野王县西北。常山在常山郡上曲阳县西北。河水自孟门南抵华阴,屈而东流;纣都朝歌,河经其南。北屈之孟门在朝歌西北,恐不可言"左"。索隐曰:孟门别一山,在朝歌东边。此特左、右二字之差而误耳。春秋说题辞:河之为言荷也;荷精分布,怀阴引度也。释名:河,下也,随地下处而通流也。书曰:武王胜殷,杀纣。太行之行,户刚翻。北屈,陆求忽翻,颜居勿翻。〕由此观之,在德不在险。若君不修德,舟中之人皆敌国也!"武侯曰:"善。"

  魏置相,相田文。〔相,息亮翻。此田文非齐之田文。〕吴起不悦,谓田文曰:"请与子论功可乎?"田文曰:"可。"起曰:"将三军,使士卒乐死,敌国不敢谋,子孰与起?"文曰:"不如子。"〔将,即亮翻。乐,音洛。〕起曰:"治百官,亲万民,实府库,子孰与起?"文曰:"不如子。"〔治,直之翻。〕起曰:"守西河,秦兵不敢东乡,韩、赵宾从,子孰与起?"文曰: "不如子。"〔乡,读曰向。宾从,犹言宾服也。〕起曰:"此三者子皆出吾下,而位居吾上,何也?"文曰:"主少国疑,大臣未附,百姓不信,方是之时,属之子乎,属之我乎?"〔少,诗照翻。属,子〔之〕欲翻。〕起默然良久曰:"属之子矣!"

  久之,魏相公叔尚【章:十二行本"尚"下有"魏公"二字;乙十一行本同;孔本同;张校同:退斋校同。】主而害吴起。〔如淳曰:天子嫁女于诸侯,必使诸侯同姓者主之,故谓之公主。帝姊妹曰长公主,诸王女曰翁主。师古曰:如说得之。天子不亲主婚,故谓之公主;诸王则自主婚,故其女曰翁主。翁者,父也,言父主其婚也;亦曰王主,言王自主其婚也。扬雄方言云:周、晋、秦、陇谓父曰翁。而臣瓒,王茂,或云"公者比于上爵",或云"主者妇人尊称",皆失之。刘贡父曰:予谓公主之称本出秦旧,男为公子,女为公主。古者大夫妻称主,故以公配之。若谓同姓主之,故谓之公主,则周之事,秦不知用也。古之嫁女,礼当如周使大夫主之,何不谓之夫主乎?然则谓之王主者,犹言王子也;谓之公主者,缘公而生耳。毛晃曰:尚,崇也,高也,贵也,饰也,加也,尊也。娶公主谓之尚,言帝王之女尊而尚之,不敢言娶也。相,息亮翻。〕公叔之仆曰:"起易去也。起为人刚劲自喜。〔易,以豉翻。去,起吕翻。师古曰:喜,许吏翻。〕子先言于君曰:『吴起,贤人也,而君之国小,臣恐起之无留心也。君盍试延以女,起无留心,则必辞矣。』子因与起归而使公主辱子,起见公主之贱子也,必辞,则子之计中矣。"〔中,竹仲翻。〕公叔从之,吴起果辞公主。魏武侯疑之而未信,起惧诛,遂奔楚。

  楚悼王素闻其贤,至则任之为相。起明灋审令,〔相,息亮翻。灋,古法字。〕捐不急之官,癈公族疏远者,以抚养战斗之士,要在强兵,破游说之言从横者。〔捐,余专翻,弃也,除去也。汉书音义曰:以利合曰从,以威力相胁曰横。或曰:南北曰从,从者,连南北为一,西乡以摈秦。东西曰横,横者,离山东之交,使之西乡以事秦。说,式芮翻。从,即容翻。"横",亦作"衡",音同〕。于是南平百越,〔韦昭曰:越有百邑。〕北却三晋,西伐秦,诸侯皆患楚之强;而楚之贵戚大臣多怨吴起者。

  ③秦惠公薨,子出公立。〔出,非諡也;以其失国出死,故曰出公。〕

  ④赵武侯薨,国人复立烈侯之太子章,是为敬侯。〔諡法:夙夜警戒曰敬。〕

  ⑤韩烈侯薨,子文侯立。

  十六年(乙未,公元前三八六年)

  ①初命齐大夫田和为诸侯。〔田氏自此遂有齐国。田和是为太公。〕

  ②赵公子朝作乱,【章:乙十一行本"乱"下有"出"字;孔本同;退斋校同;此处百衲本缺。】奔魏;与魏袭邯郸,不克。〔邯,音寒。郸,音丹。〕

  十七年(丙申,公元前三八五年)

  ①秦庶长改逆献公于河西而立之;杀出子及其母,沈之渊旁。〔后秦制爵,一级曰公士,二上造,三簪袅,四不更,五大夫,六官大夫,七公大夫,八公乘,九五大夫,十左庶长,十一右庶长,十二左更,十三中更,十四右更,十五少上造,十六大上造,十七驷车庶长,十八大庶长,十九关内侯,二十彻侯。师古曰:庶长,言众列之长。注又详见下卷显王十年前。据史记:威烈王十一年秦灵公卒,子献公师隰不得立,立灵公季父悼子,是为简公。出子,简公之孙也。今庶长改迎献公而杀出子。正义曰:西者,秦州西县,秦之旧地。时献公在西县,故迎立之。余谓此言河西,非西县也。灵公之卒,献公不得立,出居河西;河西者,黄河之西,盖汉凉州之地。"袅",当作"褭",乃了翻。更,工衡翻。乘,绳证翻。长,知丈翻。〕

  ②韩伐郑,取阳城;〔汉阳城县属颍川郡;是为地中,成周于此以土圭测日景。〕伐宋,执宋公。

  ③齐太公薨,子桓公午立。

  十九年(戊戌,公元前三八三年)

  ①魏败赵师于兔台。〔史记赵世家曰:魏败我兔台,筑刚平。正义曰:兔台、刚平,并在河北。败,补迈翻。〕

  二十年(己亥,公元前三八二年)

  ①日有食之,既。〔既,尽也〕

  二十一年(庚子,公元前三八一年)

  ①楚悼王薨。贵戚大臣作乱,攻吴起;起走之王尸而伏之。〔之,往也,往赴王尸而伏其侧。〕击起之徒因射刺起,并中王尸。〔射,而亦翻。刺,七亦翻。中,竹仲翻。〕既葬,肃王即位,〔諡法:刚德克就曰肃;执心决断曰肃。〕使令尹尽诛为乱者;〔令尹,楚相也。〕坐起夷宗者七十余家。〔夷,杀也;夷宗者,杀其同宗也。〕

  二十二年(辛丑,公元前三八零年)

  ①齐伐燕,取桑丘。魏、韩、赵伐齐,至桑丘。〔此桑丘,非二年所书楚之桑丘。括地志曰:桑丘故城,俗名敬城,在易州遂城县,盖燕之南界也。〕

  二十三年(壬寅,公元前三七九年)

  ①赵袭卫,不克。〔成王封康叔于卫,居河、淇之间,故殷垆也。至懿公为狄所灭,东徙度河。文公徙居楚丘,遂国于濮阳。是年,慎公颓之三十五年。自康叔至慎公凡三十二世。〕

  ②齐康公薨,无子,田氏遂并齐而有之。〔姜氏至此灭矣。〕

  二十四年(癸卯,公元前三七八年)

  ①狄败魏师于浍。〔汉之中山、上党、西河、上郡,自春秋以来,狄皆居之,此亦其种也。水经:浍水出河东绛县东浍山,西过绛县南,又西南过虒祁宫南,又西南至王桥,入汾水。括地志:浍山在绛州翼城县东北。败,补迈翻。浍,古外翻。〕

  ②魏、韩、赵伐齐,至灵丘。〔史记正义曰:灵丘,河东蔚州县。余按蔚州之灵丘,即汉代郡之灵丘,此时齐境安能至代北邪!此即孟子谓蚔鼃辞灵丘请士师之地。班志曰:齐地北有千乘、清河以南。汉清河郡有灵县,清河北接赵、魏之境,此为近之。蚳,音迟。鼃,乌花翻。〕

  ③晋孝公薨,子靖公俱酒立。〔諡法:柔众安民曰靖;又,恭己鲜言曰靖。〕

  二十五年(甲辰,公元前三七七年)

  ①蜀伐楚,取兹方。〔据史记:蜀伐楚,取兹方,楚为捍关以拒之。则兹方之地在捍关之西。刘昭志:巴郡鱼复县有捍关。〕

  ②子思言苟变于卫侯曰:"其才可将五百乘。"〔古者兵车一乘,甲士三人,步卒七十二人;五百乘,三万七千五百人。国语曰:苟本自黄帝之子。将,即亮翻;下同。乘,绳证翻。〕公曰:"吾知其可将;然变也尝为吏,赋于民而食人二鸡子,故弗用也。"子思曰:"夫圣人之官人,犹匠之用木也,〔夫,音扶。〕取其所长,弃其所短;故杞梓连抱而有数尺之朽,良工不弃。今君处战国之世,〔处,昌吕翻。〕选爪牙之士,而以二卵弃干城之将,〔诗:赳赳武夫,公侯干城。毛氏传曰:干,捍也;音户旦翻。郑氏笺曰:干也,城也,皆所以御难也。干,读如字。〕此不可使闻于邻国也。"公再拜曰:"谨受教矣!"

  卫侯言计非是,而群臣和者如出一口。〔和,户卧翻。〕子思曰:"以吾观卫,所谓『君不君,臣不臣』者也!"〔"君不君,臣不臣,"论语载齐景公之言。〕公丘懿子曰:"何乃若是?"〔公丘,复姓。諡法:温柔贤善曰懿。〕子思曰:"人主自臧,则众谋不进。〔臧,善也。〕事是而臧之,犹却众谋,况和非以长恶乎!〔和,户卧翻。长,知丈翻。〕夫不察事之是非而悦人赞己,暗莫甚焉;不度理之所在而阿谀求容,谄莫甚焉。〔度,徒洛翻。〕君暗臣谄,以居百姓之上,民不与也。若此不已,国无类矣!"

  子思言于卫侯曰:"君之国事将日非矣!"公曰:"何故?"对曰:"有由然焉。君出言自以为是,而卿大夫莫敢矫其非;卿大夫出言亦自以为是,而士庶人莫敢矫其非。君臣既自贤矣,〔白虎通曰:君,群也,群下之所归心也。臣,坚也,厉志自坚也。〕而群下同声贤之,贤之则顺而有福,矫之则逆而有祸,如此则善安从生!诗曰:『具曰予圣,谁知乌之雌雄?』〔诗正月之辞。毛氏传曰:君臣俱自谓圣也。郑氏笺曰:时君臣贤愚适同,如乌之雌雄相似,谁能别异之乎?又曰:乌〔鸟〕之雌雄不可别者,以翼〔知之〕,右掩左,雄,左掩右,雌,阴阳相下之义也。〕抑亦似君之君臣乎!"

  ③鲁穆公薨,子共公奋立。〔諡法:布德就义曰穆;中情见貌曰穆;尊贤敬让曰共;既过能改曰共;执事坚固曰共。共,读曰恭。考异曰:司马迁史记六国表:周威烈王十九年甲戌,鲁穆公元年。烈王元年丙午,共公元年。显王十七年己巳,康公元年。二十六年戊寅,景公元年。赧王元年丁未,平公元年。二十年丙寅,文公元年。四十三年己丑,顷公元年。五十九年乙巳,周亡。秦庄襄王元年壬子,楚灭鲁。按鲁世家,穆公三十三年卒,若元甲戌,终乙巳,则是三十二年也。共公二十二年卒,若元丙午,终戊辰,则是二十三年也。康公九年卒,景公二十五年卒,平公二十二年卒,若元丁未,终乙丑,则是十九年也。文公二十三年卒,顷公二十四年楚灭鲁。班固汉书律历志"文公"作"缗公";其在位之年与世家异者,惟平公二十年耳。本志自鲁僖公五年正月辛亥朔旦冬至推之,至成公十二年正月庚寅朔旦冬至,定公七年正月己巳朔旦冬至,元公四年正月戊申朔旦冬至,康公四年正月丁亥朔旦冬至,缗公二十二年正月丙寅朔旦冬至,汉高祖八年十一月乙巳朔旦冬至,武帝元朔六年十一月甲申朔旦冬至,元帝初元二年十一月癸亥朔旦冬至,其间相距皆七十六年,此最为得实,又与鲁世家注、皇甫谧所纪岁次皆合,今从之。六国表差谬,难可尽据也。余按考异"自鲁僖公五年至汉元帝初元二年六百余年间,十二月朔旦冬至,相距皆七十六年,此最为得实,又与鲁世家注、皇甫谧所纪岁次皆合",盖谓刘彝叟长历也。且言"史记六国表差谬,难可尽据"。又按通鉴目录编年用刘彝叟长历。汉武帝太初元年,初用夏正定历,史记历书是年书阏逢摄提格,目录书强圉赤奋若。阏逢摄提格,甲寅也,强圉赤奋若,丁丑也,有二十四年之差,温公用彝叟历,邵康节皇极经世书亦用彝叟历。康节少自雄其才,既学,力慕高远,一见李之才,遂从而受学,庐于共城百源,冬不炉,夏不扇,夜不就席者数年,覃思于易经也。皇极经世书不能违彝叟历。及其来居于洛,而温公亦奉祠以书局在洛,相过从稔,又夙所敬者也。余意其讲明之间必尝及此,而决于用彝叟历。读考异此一段,辞意可见。〕

  ④韩文侯薨,子哀侯立。

  二十六年(乙巳,公元前三七六年)

  ①王崩,子烈王喜立。

  ②魏、韩、赵共废晋靖公为家人而分其地。〔唐叔不祀矣。〕

  烈王〔名喜,安王之子。〕

  元年(丙午,公元前三七五年)

  ①日有食之。

  ②韩灭郑,因徙都之。〔韩本都平阳,其地属汉之河东郡;中间徙都阳翟。郑都新郑,其地属汉之河南郡。郑桓公始封于郑,其地属汉之京兆;后灭虢、郐而国于溱,洧之间,故曰新郑,左传郑庄公所谓"吾先君新邑于此"是也。今韩既灭郑,自阳翟徙都之。韩既都郑,故时人亦谓韩王为郑王,考之战国策、韩非子可见。〕

  ③赵敬侯薨,子成侯种立。〔种,章勇翻。〕

  三年(戊申,公元前三七三年)

  ①燕败齐师于林狐。〔败,补迈翻。〕

  ②鲁伐齐,入阳关。〔徐广曰:阳关在巨平。班志,巨平县属泰山郡。括地志:阳关故城在兖州博城县南二十九里,其城之西临汶水。汶,音问。〕

  ③魏伐齐,至博陵。〔史记正义曰:博陵在济州西界。宋白曰:史记,齐威王伐晋至博陵。徐广曰:东郡之博平,汉为县。〕

  ④燕僖公薨,子桓公立。

  ⑤宋休公薨,子辟公立。〔辟亦諡法之所不戴。〕

  ⑥卫慎公薨,子声公训立。〔諡法:敏以敬曰慎。载记:思虑深远曰慎。〕

  四年(己酉,公元前三七二年)

  ①赵伐卫,取都鄙七十三。〔周礼:太宰以八则治都鄙。注云:都之所居曰鄙。都鄙,卿大夫之采邑。盖周之制,四县为都,方四十里,一千六百井,积一万四千四百夫;五酇为鄙,鄙五百家也。此时卫国褊小,若都鄙七十三,以成周之制率之,其地广矣,尽卫之提封,未必能及此数也。更俟博考。〕

  ②魏败赵师于北蔺。〔班志,西河郡有蔺县。史记正义曰:在石州。其地于赵为西北,故曰北蔺。蔺,离进翻。〕

  五年(己酉,公元前三七二年)

  ①魏伐楚,取鲁阳。〔左传所谓"刘累迁于鲁县",即鲁阳也。班志,鲁阳县属南阳郡。史记正义曰:今汝州鲁山县。〕

  ②韩严遂弑哀侯,国人立其子懿侯。初,哀侯以韩廆为相而爱严遂,二人甚相害也。严遂令人刺韩廆于朝,廆走哀侯,哀侯抱之;人刺韩廆,兼及哀侯。〔战国策以聂政刺韩相事及并中哀侯为一事;此从史记。蜀本注曰:按太史公年表及韩世家,于韩烈侯三年皆书"聂政杀韩相侠累",于哀侯六年又皆书"严遂弑哀侯"。以刺客传考之,聂政杀侠累事在哀侯时;以战国策考之亦然。从传与战国策,则是年表,世家于烈侯三年书"盗杀侠累"误矣。通鉴于烈侯三年载聂政杀侠累事,又于哀侯六年载严遂杀其君哀侯,是从年表、世家所书。盖刺客传初不言并杀哀侯,止战国策言之,通鉴岂以此疑之欤!故载并刺哀侯,不书聂政,止曰"使人"。以此求之,则通鉴之意不以严仲子为严遂,亦不以侠累为韩廆,止从年表、世家而不信其传也。余按温公与刘道原书,亦疑此事。廆,户贿翻。相,息亮翻。刺,七亦翻。朝,直遥翻。走,音奏。〕

  ③魏武侯薨,不立太子,子罃与公中缓争立,国内乱。〔罃,于耕翻。中,读曰仲。〕

  六年(辛亥,公元前三七零年)

  ①齐威王来朝。是时周室微弱,诸侯莫朝,而齐独朝之,天下以此益贤威王。〔朝,直遥翻。〕

  ②赵伐齐,至鄄。〔班志,济阴郡有鄄城县。鄄,工掾翻。〕

  ③魏败赵师于怀。〔班志,河内郡有怀县。魏收地形志,怀州武德郡有怀县,县管内有怀城。败,补迈翻。〕

  ④齐威王召即墨大夫,语之曰:"自子之居即墨也,毁言日至。〔班志,即墨县属胶东国。括地志:即墨故城,在莱州胶水县南六十里。宋白曰:城临墨水,故曰即墨。语,牛倨翻,下同。〕然吾使人视即墨,田野辟,〔辟,读 曰辟;下同。〕人民给,官无事,东方以宁;是子不事吾左右以求助也!"封之万家。召阿大夫,语之曰:"自子守阿,誉言日至。〔阿,即东阿县;班志属东郡。 誉,音余,称其美也。〕吾使人视阿,田野不辟,人民贫馁。昔日赵攻鄄,子不救;〔鄄,工掾翻。〕卫取薛陵,子不知;〔薛陵,春秋薛国之墟也。班志,薛县属鲁国,而卫国在汉东郡陈留界。薛陵属齐而近于卫,故为所取。齐后封田婴于此。〕是子厚币事吾左右以求誉也!"是日,烹阿大夫及左右尝誉者。于是群臣耸惧,莫敢饰诈,务尽其情,齐国大治,强于天下。〔誉,音余。治,直吏翻。〕

  ⑤楚肃王薨,无子,立其弟良夫,是为宣王。

  ⑥宋辟公薨,子剔成立。〔剔,他历翻。〕

  七年(壬子,公元前三六九年)

  ①日有食之。

  ②王崩,弟扁立,〔据班书古今人表师古注:扁,音篇。〕是为显王。

  ③魏大夫王错出奔韩。〔姓谱:王氏之所自出非一。出太原、琅邪者,周灵王太子晋之后。北海、陈留,齐王田和之后。东海出自姬姓。高平、京兆,魏信陵君之后。天水、东平、新蔡、新野、山阳、中山、章武、东莱、河东者,殷王子比干为纣所害,子孙以王者之后,号曰王氏。余谓此皆后世以诸郡着姓言之耳。春秋之时自有王姓,莫能审其所自出。〕公孙颀谓韩懿侯曰:"魏乱,可取也。"〔公孙,姓也。黄帝,公孙氏。颀,渠希翻。〕懿侯乃与赵成侯合兵伐魏,战于浊泽,大破之,遂围魏。〔史记正义曰:徐广以为长社浊泽,非也。括地志云:浊水源出蒲州解县东北平地,尔时魏都安邑,韩、赵伐魏,岂至河南长社邪!解县浊水近于魏都,当是也。〕成侯曰:"杀罃,立公中缓,割地而退,我二国之利也。"懿侯曰:"不可。杀魏君,暴也;割地而退,贪也。不如两分之。魏分为两,不强于宋、卫,则我终无魏患矣。"赵人不听。懿侯不悦,以其兵夜去。赵成侯亦去。罃遂杀公中缓而立,〔中,读曰仲。〕是为惠王。

  太史公曰:魏惠王所以身不死、国不分者,二国之谋不和也。若从一家之谋,魏必分矣。故曰:"君终,无适子,其国可破也。"〔索隐曰:盖古人之言及俗说,故云"故曰"。适,读曰嫡。〕