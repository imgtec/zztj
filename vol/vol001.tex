% <div class="info_txt clearfix" id="show">
% <center style="font-size:24px;">2-資治通鑑卷一</center>
%     資治通鑑卷一     宋 司馬光 撰\par
% 
% 胡三省 音註
% \par

周紀一|{
	【起著雍攝提格 盡玄黓困敦 凡三十五年
	爾雅曰
	太歲
	在甲曰閼逢
	在乙曰旃蒙
	在丙曰柔兆
	在丁曰彊圉
	在戊曰著雍
	在己曰屠維
	在庚曰上章
	在辛曰重光
	在壬曰玄黓
	在癸曰昭陽
	是為歲陽
	在寅曰攝提格
	在卯曰單閼
	在辰曰執徐
	在巳曰大荒落
	在午曰敦牂
	在未曰協洽
	在申曰涒灘
	在酉曰作噩
	在戌曰掩茂
	在亥曰大淵獻
	在子曰困敦
	在丑曰赤奮
	若是為歲名
	周紀分註起著雍攝提格起戊寅也盡玄黓困敦盡壬子也閼讀如字史記作焉於乾翻著陳如翻雍於容翻黓逸職翻單閼上音丹又特連翻下
	烏葛翻
	又於連翻牂作郎翻涒吐䰟翻灘吐丹翻困敦音頓 杜預世族譜曰周黄帝之苖裔姬姓
	后稷之後封于邰及夏衰稷子不窟竄于西戎至十二代孫太王避狄遷岐至孫文王受命武王克商而有天下自武王至平王凡十三世自平王至威烈王又十八世自威烈王至赧王又五世張守節曰因太王居周原國號曰周地理志云右扶風美陽縣岐山西北中水鄉周太王所邑括地志云故周城一名美陽城在雍州武功縣西北二十五里紀理也統理衆事而繫之年月温公繫年用春秋之法因史漢本紀而謂之紀邰湯來翻夏戶雅翻窟竹律翻雍於用翻】}
\par
威烈王|{
	【名午考王之子諡法猛以剛果曰威有功安民曰烈沈約曰諸複諡有諡人無諡法】}
\par
二十三年|{
	【上距春秋獲麟七十八年距左傳趙襄子惎智伯事七十一年惎毒也音安冀翻】}
初命晉大夫魏斯趙籍韓䖍為諸侯|{
	【此温公書法所由始也魏之先畢公高後與周同姓其苖裔曰畢萬始封于魏至魏舒始為晉正卿三世至斯趙之先造父後至叔帶始自周適晉至趙夙始封於耿至趙盾始為晉正卿六世至籍韓之先出於周武王至韓武子事晉封於韓原至韓厥為晉正卿六世至䖍三家者世為晉大夫於周則陪臣也周室既衰晉主夏盟以尊王室故命之為伯三卿竊晉之權暴蔑其君剖分其國此王法所必誅也威烈王不惟不能誅之又命之為諸侯是崇奨姧名犯分之臣也通鑑始於此其所以謹名分歟】}
\par
臣光曰臣聞天子之職莫大於禮禮莫大於分分莫大於名|{
	【分扶問翻下同】}
何謂禮紀綱是也何謂分君臣是也何謂名公侯卿大夫是也夫以四海之廣|{
	【夫以音扶】}
兆民之衆受制於一人雖有絶倫之力高世之智莫不奔走而服役者豈非以禮為之紀綱哉是故天子統三公|{
	【統他綜翻】}
三公率諸侯諸侯制卿大夫卿大夫治士庶人|{
	【治直之翻】}
貴以臨賤賤以承貴上之使下猶心腹之運手足根本之制支葉下之事上猶手足之衛心腹支葉之庇本根然後能上下相保而國家治安|{
	【治直吏翻】}
故曰天子之職莫大於禮也文王序易以乾坤為首孔子繫之曰天尊地卑乾坤定矣卑高以陳貴賤位矣|{
	【繫戶計翻】}
言君臣之位猶天地之不可易也春秋抑諸侯尊王室王人雖微序於諸侯之上以是見聖人於君臣之際未嘗不惓惓也|{
	【惓逵員翻漢劉向傳忠臣畎畝猶不忘君惓惓之義也惓惓猶言勤勤也】}
非有桀紂之暴湯武之仁人歸之天命之君臣之分當守節伏死而已矣是故以微子而代紂則成湯配天矣|{
	【史記商帝乙生三子長曰微子啓次曰中衍季曰紂紂之母為后帝乙欲立啓為太子太史據法爭之曰有妻之子不可立妾之子乃立紂紂卒以暴虐亡殷國孔玄義曰物之大者莫若於天推父比天與之相配行孝之大莫大於此所謂嚴父莫大於配天也又孔氏曰禮記稱萬物本乎天人本乎祖俱為其本可以相配故王者皆以祖配天諡法除殘去虐曰湯然諡法起於周盖殷人先有此號周人遂引以為諡法分扶問翻長知兩翻卒子恤翻】}
以季札而君吳則太伯血食矣|{
	【吳王夀夢有子四人長曰諸樊次曰餘祭次曰餘昧次曰季札季札賢夀夢欲立之季札讓不可於是立諸樊諸樊卒以授餘祭欲兄弟以次相傳必致國於季札季札終讓而逃之其後諸樊之子光與餘昧之子僚爭國至於夫差吳遂以亡宗廟之祭用牲故曰血食太伯吳立國之君范甯曰太者善大之稱伯者長也周太王之元子故曰太伯陸德明曰夀夢莫公翻餘祭側介翻餘昧音末】}
然二子寜亡國而不為者誠以禮之大節不可亂也故曰禮莫大於分也夫禮辯貴賤序親疏裁羣物制庶事非名不著非器不形名以命之器以别之|{
	【夫音扶别彼列翻】}
然後上下粲然有倫此禮之大經也名器既亡則禮安得獨在哉昔仲叔于奚有功於衛辭邑而請繁纓孔子以為不如多與之邑惟名與器不可以假人君之所司也政亡則國家從之|{
	【左傳衛孫桓子帥師與齊師戰于新築衛師敗績新築人仲叔于奚救孫桓子桓子是以免既而衛人賞之邑辭請曲縣繁纓以朝許之孔子聞之曰不如多與之邑惟名與器不可以假人繁纓馬飾也繁馬鬛上飾纓馬膺前飾晉志註曰纓在馬膺如索帬繁音蒲官翻纓伊盈翻索昔各翻】}
衛君待孔子而為政孔子欲先正名以為名不正則民無所措手足|{
	【見論語】}
夫繁纓小物也而孔子惜之正名細務也而孔子先之|{
	【先悉薦翻】}
誠以名器旣亂則上下無以相保故也夫事未有不生於微而成於著聖人之慮遠故能謹其微而治之|{
	【治直之翻下同】}
衆人之識近故必待其著而後救之治其微則用力寡而功多救其著則竭力而不能及也易曰履霜堅冰至|{
	【坤初六爻辭象曰履霜堅冰隂始凝也馴致其道至堅冰也】}
書曰一日二日萬幾|{
	【臯陶謨之辭孔安國註曰幾微也言當戒懼萬事之微幾居依翻】}
謂此類也故曰分莫大於名也|{
	【分扶問翻】}
嗚呼幽厲失德周道日衰綱紀散壞下陵上替諸侯專征|{
	【謂齊桓公晉文公至悼公以及楚莊王吳夫差之類】}
大夫擅政|{
	【謂晉六卿魯三家齊田氏之類】}
禮之大體什喪七八矣|{
	【喪息浪翻】}
然文武之祀猶緜緜相屬者|{
	【屬聯屬也音之欲翻凡聯屬之屬皆同音】}
蓋以周之子孫尚能守其名分故也何以言之昔晉文公有大功於王室請隧於襄王襄王不許曰王章也未有代德而有二王亦叔父之所惡也不然叔父有地而隧又何請焉文公於是懼而不敢違|{
	【太叔帶之難襄王出居于氾晉文公帥師納王殺太叔帶既定襄王于郟王勞之以地辭請隧焉王弗許云云杜預曰闕地通路曰隧此乃王者葬禮也諸侯皆縣柩而下王章者章顯王者異於諸侯古者天子謂同姓諸侯為伯父叔父隧音遂惡烏路翻難乃旦翻氾音汎勞力到翻闕其月翻縣音玄柩其久翻】}
是故以周之地則不大於曹滕以周之民則不衆於邾莒|{
	【曹滕邾莒春秋時小國莒居許翻】}
然歷數百年宗主天下雖以晉楚齊秦之彊不敢加者何哉徒以名分尚存故也至於季氏之於魯田常之於齊白公之於楚智伯之於晉|{
	【魯大夫季氏自季友以來世執魯國之政季平子逐昭公季康子逐哀公然終身北面不敢簒國田常即陳恒田氏本陳氏温公避國諱改恒曰常陳成子得齊國之政殺闞止弑簡公而亦不敢自立史記世家以陳敬仲完為田敬仲完陳成子恒為田常故通鑑因以為據白公勝殺楚令尹子西司馬子期石乞曰焚庫弑王不然不濟白公曰弑王不祥焚庫無聚智伯當晉之衰專其國政侵伐鄰國於晉大夫為最彊攻晉出公出公道死智伯欲并晉而不敢乃奉哀公驕立之】}
其勢皆足以逐君而自為然而卒不敢者|{
	【卒子恤翻終也】}
豈其力不足而心不忍哉乃畏姧名犯分而天下共誅之也|{
	【姧居寒翻亦犯也分扶問翻】}
今晉大夫暴蔑其君剖分晉國|{
	【史記六國年表定王十六年趙魏韓滅智伯遂三分晉國】}
天子既不能討又寵秩之使列於諸侯是區區之名分復不能守而并棄之也|{
	【陸德明經典釋文凡復字其義訓又者並音扶又翻】}
先王之禮於斯盡矣或者以為當是之時周室微弱三晉彊盛|{
	【三家分晉國時因謂之三晉猶後之三秦三齊也】}
雖欲勿許其可得乎是大不然夫三晉雖彊苟不顧天下之誅而犯義侵禮則不請於天子而自立矣不請於天子而自立則為悖逆之臣|{
	【夫音扶悖蒲内翻又蒲没翻】}
天下苟有桓文之君必奉禮義而征之今請於天子而天子許之是受天子之命而為諸侯也誰得而討之故三晉之列於諸侯非三晉之壞禮乃天子自壞之也|{
	【壞音怪人毁之也】}
烏呼君臣之禮旣壞矣|{
	【此壞其義為成壞之壞讀如字】}
則天下以智力相雄長|{
	【長知兩翻】}
遂使聖賢之後為諸侯者社稷無不泯絶|{
	【謂齊宋亡於田氏魯陳越亡於楚鄭亡於韓也泯彌忍翻盡也又彌鄰翻毛晃曰没也滅也】}
生民之類糜滅幾盡|{
	【說文曰糜糝也取糜爛之義音忙皮翻幾居依翻又渠希翻近也】}
豈不哀哉
\par
初智宣子將以瑶為後智果曰不如宵也|{
	【韋昭曰智宣子晉卿荀躒之子申也瑤宣子之子智伯也謚曰襄子智果智氏之族也宵宣子之庶子也按謚法聖善周聞曰宣智氏溢美也】}
瑶之賢於人者五其不逮者一也|{
	【韋昭曰不仁也】}
美鬢長大則賢|{
	【通鑑俗傳寫者多作美鬚非也國語作美鬢今從之】}
射御足力則賢伎藝畢給則賢巧文辯慧則賢|{
	【韋昭曰給足也巧文巧於文辭也伎渠綺翻】}
彊毅果敢則賢如是而甚不仁夫以其五賢陵人而以不仁行之其誰能待之|{
	【韋昭曰待猶假也】}
若果立瑶也智宗必滅弗聽智果别族於太史為輔氏|{
	【此事見國語按左傳哀公二十三年晉荀瑶伐齊始見于傳哀二十三年史記元王五年也荀躒智文子也定十四年智文子猶見于傳智宣子之事傳無所考立瑶之議當在元王五年之前韋昭曰太史掌氏姓周禮春官之屬小史掌定世繫辯昭穆鄭司農注云史官主書故韓宣子聘魯觀書于太史世繫謂帝繫世本之屬是也小史主定之賈公彦疏曰註引太史證之者太史史官之長共其事故也蓋周之制小史定姓氏其書則太史掌之智果欲避智氏之禍故於太史别族宋祁國語補音别彼列翻又如字】}
趙簡子之子長曰伯魯幼曰無恤|{
	【趙簡子文子之孫鞅也謚法一德不懈曰簡白虎通曰子孳也孳孳無已也趙岐曰子者男子之通稱也長知兩翻】}
將置後不知所立乃書訓戒之辭於二簡|{
	【孔頴達曰書者舒也書緯璇璣鈐云書者如也則書者寫其言如其意得展舒也世本曰沮誦蒼頡作書釋文曰書庶也紀庶物也亦言著也著之簡紙求不滅也簡竹策也】}
以授二子曰謹識之|{
	【識職吏翻記也】}
三年而問之伯魯不能舉其辭求其簡已失之矣問無恤誦其辭甚習|{
	【習熟也】}
求其簡出諸袖中而奏之|{
	【毛晃曰奏進上也】}
於是簡子以無恤為賢立以為後簡子使尹鐸為晉陽|{
	【姓譜尹少昊之子封于尹城子孫因為氏韋昭曰晉陽趙氏邑為治也班志曰晉陽故詩唐國周成王滅唐封弟叔虞龍山在西晉水所出東入汾臣瓚曰所謂唐今河東永安縣是也去晉四百里括地志曰晉陽故城今名晉城在蒲州虞鄉縣西今按水經注晉水出晉陽縣西龍山昔智伯遏晉水以灌晉陽其水分為二流北瀆即智氏故渠也洞過水出沾縣北山西過榆次縣南又西到晉陽縣南榆次縣南水側有鑿臺戰國策所謂智伯死於鑿臺之下即此處也參而考之晉陽故城恐不在蒲州水經注人云叔虞封於唐縣有晉水故改名為晉子夏序詩此晉也而謂之唐是也與班志合瓚說及括地志未知何據】}
請曰以為繭絲乎抑為保障乎簡子曰保障哉|{
	【繭絲謂浚民之膏澤如抽繭之緒不盡則不止保障謂厚民之生如築堡以自障愈培則愈厚宋祁曰障之亮翻又音章】}
尹鐸損其戶數|{
	【韋昭曰損其戶則民優而税少】}
簡子謂無恤曰晉國有難而無以尹鐸為少|{
	【而汝也難乃旦翻患也阨也少音多少之少重之為多輕之為少】}
無以晉陽為遠必以為歸及智宣子卒|{
	【卒子恤翻】}
智襄子為政|{
	【謚法有勞定國曰襄為政為晉國之政】}
與韓康子魏桓子宴於藍臺|{
	【韓康子韓宣子之曾孫莊子之子䖍也魏桓子魏獻子之子曼多之孫駒也謚法温柔恕恭曰康辟上服遠曰桓爾雅四方而高曰臺】}
智伯戲康子而侮段規|{
	【姓譜段鄭共叔段之後】}
智國聞之諫曰主不備難難必至矣|{
	【春秋以來大夫之家臣謂大夫曰主難乃旦翻下同】}
智伯曰難將由我我不為難誰敢興之對曰不然夏書有之一人三失怨豈在明不見是圖|{
	【書五子之歌之辭夏戶雅翻見賢遍翻發見也著也形也】}
夫君子能勤小物故無大患今主一宴而恥人之君相|{
	【夫音扶段規韓康子之相也相息醤翻下同】}
又弗備曰不敢興難無乃不可乎蜹蟻蜂蠆皆能害人|{
	【宋祁曰蜹如蜕翻人字林乂劣翻秦人謂蚊為蜹今按蜹小蟲日中羣集人之肌膚而嘬其血蚊之類也蜂細腰而能螫人蠆亦毒蟲長尾音丑邁翻】}
况君相乎弗聽智伯請地於韓康子康子欲弗與段規曰智伯好利而愎不與將伐我不如與之彼狃於得地|{
	【好呼到翻愎弼力翻狠也狃女九翻驕忲也又相狎也】}
必請於他人他人不與必嚮之以兵然後我得免於患而待事之變矣康子曰善使使者致萬家之邑於智伯|{
	【毛晃曰邑都邑四井為邑四邑為丘邑方二里丘方四里載師以公邑之田任甸地以家邑之田任稍地註公邑謂六遂餘地家邑大夫之采地此又與四井之邑不同又都國都邑縣也左傳凡邑有先君宗廟之主曰都無曰邑邑曰築都曰城此謂大縣邑也杜預引周禮四縣為都四井為邑恐誤四井之邑方二里豈能容宗廟城郭如論語十室之邑西都賦都都相望邑邑相屬則是四縣四井之都邑也若千室之邑萬家之邑則非井邑矣項安世曰小司徒井牧田野以四井為邑凡三十六家除公田四夫凡三十二家遂大夫會為邑者之政以里為邑凡二十五家遂大夫蓋論里井之制二十五家共一里門即六鄉之二十五家為一閭也小司徒蓋論溝洫之制四井為邑共用一溝即匠人所謂井間廣四尺深四尺謂之溝也居則度人之衆寡溝則度水之衆寡此其所以異歟毛項二說皆明周制參而考之戰國之所謂邑非周制矣致送至也】}
智伯悦又求地於魏桓子桓子欲弗與任章曰何故弗與|{
	【任章魏桓子之相也姓譜黄帝二十五子十二人各以德為姓第一曰任氏又任為風姓之國實太昊之後主濟祀今濟州任城即其地任市林翻】}
桓子曰無故索地故弗與任章曰無故索地諸大夫必懼|{
	【索山客翻求也】}
吾與之地智伯必驕彼驕而輕敵此懼而相親以相親之兵待輕敵之人智氏之命必不長矣周書曰將欲敗之必姑輔之將欲取之必姑與之|{
	【逸書也敗補邁翻】}
主不如與之以驕智伯然後可以擇交而圖智氏矣奈何獨以吾為智氏質乎|{
	【質脂利翻物相綴當也又質讀如字亦通質謂椹質也質的也椹質受斧質的受矢言智伯怒魏桓子必加兵於魏如椹質之受斧質的之受矢也】}
桓子曰善復與之萬家之邑一|{
	【復扶又翻】}
智伯又求蔡臯狼之地於趙襄子|{
	【康曰臯姑勞切狼盧當切春秋蔡地後為趙邑余據春秋之時晉楚爭盟晉不能越鄭而服蔡三家分晉韓得成臯因以并鄭時蔡已為楚所滅鄭之南境亦人于楚就使臯狼為蔡地趙襄子安得而有之漢書地理志西河郡有臯狼縣又有藺縣漢之西河春秋以來皆為晉境而古文藺字與蔡字近或者蔡字其藺字之訛也】}
襄子弗與智伯怒帥韓魏之甲以攻趙氏|{
	【帥讀曰率】}
襄子將出曰吾何走乎|{
	【走則豆翻疾趨之也趨七喻翻】}
從者曰長子近且城厚完|{
	【從才用翻長子縣周史辛伯所封邑班志屬上黨郡陸德明曰長子之長丁丈翻顔師古曰長讀為短長之長今讀為長幼之長非也崔豹古今注曰城盛也所以盛受民物也淮南子曰鯀作城盛時征翻】}
襄子曰民罷力以完之|{
	【罷讀曰疲】}
又斃死以守之其誰與我|{
	【韋昭曰謂誰與我同力也】}
從者曰邯鄲之倉庫實|{
	【邯鄲即春秋邯鄲午之邑也班志邯鄲縣屬趙國張晏曰邯鄲山在東城下鄲盡也城郭從邑故旁加邑宋白曰邯鄲本衛地後屬晉七國時為趙都趙敬侯自晉陽始都邯鄲余按史記六國年表周安王之十六年趙敬侯之元年烈王之二年趙成侯之元年成侯一十二年魏克邯鄲是年顯王之十六年也二十四年魏歸邯鄲若敬侯已都邯鄲魏克其國都而趙不亡何也至顯王二十二年公子范襲邯鄲不勝而死是年肅侯之三年也意此時趙方都邯鄲蓋肅侯徙都非敬侯也邯音 寒鄲音丹康多寒翻】}
襄子曰浚民之膏澤以實之|{
	【韋昭曰浚煎也讀曰醮宋祁曰浚蘇俊翻醮子召翻余謂浚讀當如宋音浚省疏瀹也淘也深也】}
又因而殺之其誰與我其晉陽乎先主之所屬也|{
	【古者諸侯之大夫其家之臣子皆稱之曰主死則曰先主考左傳可見已屬陟玉翻】}
尹鐸之所寛也民必和矣乃走晉陽三家以國人闈而灌之城不浸者三版|{
	【高二尺為一版三版六尺】}
沈竈產鼃民無叛意|{
	【沈持林翻顔師古漢書音義曰鼃黽也似蝦蟇而長脚其色青史游急就章曰蛙蝦蟇陸佃埤雅曰鼃似蝦蟇而長踦瞋目如怒鼃與蛙同音下媧翻】}
智伯行水|{
	【據經典釋文凡廵行之行音下孟翻後倣此】}
魏桓子御韓康子驂乘|{
	【兵車尊者居左執弓矢御者居中有力者居右持矛以備傾側所謂車右是也韓魏畏智氏之彊一為之御一為之右驂與參同參者三也三人同車則曰驂乘四人同車則曰駟乘左傳齊伐晉燭庸之越駟乘杜預注曰四人共乘者殿車乘石證翻】}
智伯曰吾乃今知水可以亡人國也桓子肘康子康子履桓子之跗以汾水可以灌安邑絳水可以灌平陽也|{
	【跗音夫足趾也班志汾水出汾陽北山汾陽縣屬太原郡安邑縣屬河東郡史記正義曰安邑故城在絳州夏縣東北十五里應劭曰絳水出河東絳縣西南平陽縣亦屬河東郡安邑魏絳始居邑平陽韓武子玄孫貞子始居之桓康二子之肘足接蓋各為都邑慮也水經注曰絳水出絳縣西南蓋以故絳為言其水出絳山東西北流而合于澮猶在絳縣界中智伯所謂汾水可以灌安邑或亦有之絳水可以灌平陽未識所由余謂自春秋之季至于元魏歷年滋多郡縣之離合川谷之遷改有不可以一時所睹為據者史記正義曰韓初都平陽今晉州也括地志曰絳水一名白今名沸泉源出絳山飛泉奮湧揚波注縣積壑三十餘丈望之極為奇觀可接引北灌平陽城酈道元父範歷仕三齊少長齊地熟其山川後入關死於道未嘗至河東也此蓋因耳學而致疑括地志成於唐之魏王泰泰者太宗之愛子羅致天下一時名儒以作此書其考據宜詳當取以為據】}
絺疵謂智伯曰韓魏必反矣智伯曰子何以知之絺疵曰以人事知之夫從韓魏之兵以攻趙趙亡難必及韓魏矣|{
	【夫音扶難乃旦翻】}
今約勝趙而三分其地城不没者三版人馬相食城降有日而二子無喜志有憂色是非反而何明日智伯以絺疵之言告二子二子曰此夫讒人欲為趙氏游說使主疑於二家而懈於攻趙氏也不然夫二家豈不利朝夕分趙氏之田而欲為危難不可成之事乎二子出絺疵入曰主何以臣之言告二子也智伯曰子何以知之對曰臣見其視臣端而趨疾知臣得其情故也智伯不悛絺疵請使於齊|{
	【夫音扶餘並同難乃旦翻降戶江翻下也服也說輸芮翻懈居隘翻怠也危難如字悛丑緣翻改也止也絺抽遲翻姓也康曰絺當作郗姓譜諸書未有從絲者疑借字余按姓譜絺姓周蘇忿生支子封於絺因氏焉為趙之為音于偽翻使疏吏翻疵請出使以避禍也】}
趙襄子使張孟談潜出見二子曰臣聞脣亡則齒寒今智伯帥韓魏以攻趙趙亡則韓魏為之次矣|{
	【帥讀曰率】}
二子曰我心知其然也恐事未遂而謀泄則禍立至矣張孟談曰謀出二主之口入臣之耳何傷也二子乃潜與張孟談約為之期日而遣之|{
	【姓譜張氏本自軒轅第五子揮始造弦寔張網羅世掌其職後因氏馬風俗傳云張王李趙黄帝所賜姓也又晉有解張字張侯自此晉國有張氏唐姓氏譜張氏出自姬姓黄帝子少昊青陽氏第五子揮正始制弓矢子孫賜姓張周宣王卿士張仲其後裔事晉為大夫】}
襄子夜使人殺守隄之吏而決水灌智伯軍智伯軍救水而亂韓魏翼而擊之襄子將卒犯其前|{
	【將即亮翻又音如字將領也卒滅没翻說文吏人給事者衣為卒卒衣有題識其字從衣從十】}
大敗智伯之衆|{
	【以此敗彼曰敗敗比邁翻】}
遂殺智伯盡滅智氏之族|{
	【史記六國年表三晉滅智氏在周定王十六年上距獲麟二十七年皇甫謐曰元王十一年癸未三晉滅智伯】}
唯輔果在|{
	【以别族也】}
\par
臣光曰智伯之亡也才勝德也夫才與德異而世俗莫之能辯|{
	【夫音扶】}
通謂之賢此其所以失人也夫聰察彊毅之謂才正直中和之謂德才者德之資也德者才之帥也|{
	【夫音扶帥所類翻】}
雲夢之竹天下之勁也|{
	【書禹貢雲土夢作人孔安國注云雲夢之澤在江南左傳楚王以鄭伯旧江南之夢杜預注云楚之雲夢跨江南北班志雲夢澤在南郡華容縣南祝穆曰據左傳䢵夫人弃子文於夢中言夢而不言雲楚子避吳入于雲中言雲而不言夢則知雲夢二澤也漢陽志雲在江之北夢在江之南人安陸有雲夢澤枝江有雲夢城蓋古之雲夢澤甚廣而後世悉為邑居聚落故地之以雲夢得名者非一處竹箭之產荆楚為良雲夢楚之地也夢如字又莫公翻】}
然而不矯揉不羽括則不能以入堅|{
	【矯舉夭翻揉如久翻康曰揉曲為矯揉所以撓曲而使之直也羽者箭翎括者箭窟受弦處括音聒通作筈】}
棠谿之金天下之利也|{
	【左傳楚封吳夫概王於棠谿戰國之時其地屬韓出金甚精利劉昭郡國志汝南郡吳房縣有棠谿亭杜佑通典曰棠谿在今汝州郾城縣界九域志蔡州有冶爐城韓國鑄劒之處】}
然而不鎔範不砥礪則不能以擊彊|{
	【毛晃曰鎔銷也鑄也說文鑄器法也董仲舒傳猶金在鎔註鎔謂鑄器之模範範法也式也禮運範金合上砥軫氏翻柔石也礪力制翻䃺也】}
是故才德全盡謂之聖人才德兼亡謂之愚人德勝才謂之君子才勝德謂之小人凡取人之術苟不得聖人君子而與之與其得小人不若得愚人何則君子挾才以為善小人挾才以為惡挾才以為善者善無不至矣挾才以為惡者惡亦無不至矣|{
	【挾檄頰翻】}
愚者雖欲為不善智不能周力不能勝譬如乳狗摶人人得而制之|{
	【挾戶頰翻朱元晦曰挾者兼有而恃之之稱勝音升乳孺遇翻乳育也乳狗育子之狗也搏伯各翻】}
小人智足以遂其姧勇足以决其暴是虎而翼者也其為害豈不多哉|{
	【虎而傅翼其為害也愈甚】}
夫德者人之所嚴|{
	【嚴敬也】}
而才者人之所愛愛者易親嚴者易疎|{
	【易以䜴翻】}
是以察者多蔽於才而遺於德自古昔以來國之亂臣家之敗子才有餘而德不足以至于顛覆者多矣豈特智伯哉故為國為家者苟能審於才德之分而知所先後|{
	【先悉薦翻後戶遘翻】}
又何失人之足患哉
\par
三家分智氏之田趙襄子漆智伯之頭以為飲器|{
	【說文桼木汁可以鬖物下從水象桼如水滴而下也漢書張騫傳匈奴破月氏王以其頭為飲器韋昭註曰飲器椑榼也晉灼曰飲器虎子屬也或曰飲酒之器也師古曰匈奴嘗以月氏王頭與漢使歃血盟然則飲酒之器是也韋云椑榼晉云虎子皆非也椑榼即今之偏榼所以盛酒耳非用飲者也虎子褻器所以溲便者椑音鼙榼克合翻氏音支使疏吏翻歃色甲翻盛時征翻褻息列翻溲疎鳩翻便毘連翻】}
智伯之臣豫讓欲為之報仇|{
	【豫姓也讓名也戰國之時又有豫且不知其同時否也為音于偽翻下同】}
乃詐為刑人挾匕首入襄子宫中塗厠|{
	【挾持也劉向曰匕首短劒鹽鐵論曰匕首長尺八寸頭類匕故云匕首匕音比厠初吏翻圊也長直亮翻】}
襄子如厠心動索之獲豫讓|{
	【索山客翻】}
左右欲殺之襄子曰智伯死無後而此人欲為報仇眞義士也吾謹避之耳乃舍之|{
	【舍讀曰捨】}
豫讓又漆身為癩吞炭為啞|{
	【癩落蓋翻惡疾也啞倚下翻瘖也】}
行乞於市|{
	【神農日中為市致天下之民聚天下之貨交易而退此立市之始也鄭氏周禮注曰市雜聚之處】}
其妻不識也行見其友其友識之為之泣曰以子之才臣事趙孟必得近幸|{
	【自春秋之時 宣子謂之宣孟趙文子謂之趙孟其後 襲而呼為趙孟孟長也】}
子乃為所欲為顧不易邪|{
	【易以䜴翻】}
何乃自苦如此求以報仇不亦難乎豫讓曰旣已委質為臣|{
	【經典釋文曰質職日翻委質委其體以事君也後漢書註委質屈膝】}
而又求殺之是二心也凡吾所為者極難耳然所以為此者將以愧天下後世之為人臣懷二心者也襄子出豫讓伏於橋下襄子至橋馬驚索之得豫讓遂殺之|{
	【自智宣子立瑶至豫讓報仇其事皆在威烈王二十三年之前故先以初字發之温公之意蓋以天下莫大於名分觀命三大夫為諸侯之事則知周之所以益微七雄之所以益盛莫重於宗社觀智趙立後之事則知智宣子之所以失趙簡子之所以得君臣之義當守節伏死而已觀豫讓之事則知策名委質者必有霣而無貳其為後世之鑑豈不昭昭也哉】}
襄子為伯魯之不立也有子五人不肯置後封伯魯之子於代|{
	【代國在夏屋句注北趙襄子滅之班志有代郡代縣為于偽翻夏戶雅翻】}
曰代成君早卒|{
	【成謚也謚法安民立政曰成】}
立其子浣為趙氏後|{
	【浣戶管翻】}
襄子卒弟桓子逐浣而自立|{
	【史記六國表威烈王元年襄子卒二年趙桓子元年卒明年國人立獻侯浣浣索隱作晚卒子恤翻下同】}
一年卒趙氏之人曰桓子立非襄主意乃共殺其子復迎浣而立之是為獻子|{
	【復扶又翻又音如字獻子即獻侯六國喪威烈王三年獻侯之元年蓋分晉之後三晉僭侯久矣謚法知質有聖曰獻】}
獻子生籍是為烈侯|{
	【謚法有功安民曰烈秉德尊業曰烈】}
魏斯者魏桓子之孫也是為文侯|{
	【謚法學勤好問曰文慈惠安民曰文】}
韓康子生武子武子生䖍是為景侯|{
	【謚法克定禍亂曰武布義行剛曰景六國表威烈王二年魏文侯斯元年十八年韓景侯䖍元年蓋其在國僭爵已久不敢以通王室威烈王遂因而命之識者重為周惜通鑑于此序三家之世也】}
魏文侯以卜子夏田子方為師|{
	【卜以官為氏田本出於陳陳敬仲以陳為田氏徐廣曰始食采地由是改姓田氏索隱曰陳田二聲相近遂為田氏夏戶雅翻】}
每過段干木之廬必式|{
	【過工禾翻唐人志氏族曰李耳字伯陽一字聃其後有李宗魏封於段為干木大夫是以段為氏也余按通鑑赧王四十二年魏有段干子則段干複姓也書武王式商容閭註云式其閭巷以禮賢記曲禮國君撫式士下之註云升車必正立據式小俛崇敬也師古曰式車前横木古者立乘凡言式車者謂俛首撫式以禮敬人孔頴達曰式謂俯下頭也古者車箱長四尺四寸而三分前一後二横一木下去車牀三尺三寸謂之為式又於式上二尺二寸横一木謂之較較去車牀凡五尺五寸於時立乘若平常則憑較故詩云倚重較兮是也若應為敬則落隱下式而頭得俯俛故記云式視馬尾是也較訖岳翻】}
四方賢士多歸之文侯與羣臣飲酒樂而天雨命駕將適野左右曰今日飲酒樂天又雨君將安之文侯曰吾與虞人期獵雖樂豈可無一會期哉乃往身自罷之|{
	【周禮有山虞澤虞以掌山澤注云虞度也度知山林之大小及其所生身自罷之者身往告之以雨而罷獵也樂音洛】}
韓借師於魏以伐趙文侯曰寡人與趙兄弟也不敢聞命趙借師於魏以伐韓文侯應之亦然二國皆怒而去已而知文侯以講於己也|{
	【講和也】}
皆朝于魏|{
	【朝直遥翻】}
魏於是始大於三晉諸侯莫能與之爭使樂羊伐中山克之|{
	【樂姓也本自有殷微子之後宋戴公四世孫樂呂為大司寇中山春秋之鮮虞也漢為中山郡宋白曰唐定州春秋白狄鮮虞之地隋圖經曰中山城在今唐昌縣東北三十一里中山故城是也杜佑曰城中有山故曰中山】}
以封其子擊文侯問於羣臣曰我何如主皆曰仁君任座曰君得中山不以封君之弟而以封君之子何謂仁君文侯怒任座趨出|{
	【任座亦習見當時鄰國之事而為是言耳任音壬座一作痤音才戈翻】}
次問翟璜|{
	【翟姓也音直格翻又音狄姓譜翟為晉所滅子孫以國為氏今人多讀從上音璜戶光翻】}
對曰仁君文侯曰何以知之對曰臣聞君仁則臣直嚮者任座之言直臣是以知之文侯悦使翟璜召任座而反之親下堂迎之以為上客文侯與田子方飲文侯曰鐘聲不比乎|{
	【比音毗不比言不和也】}
左高|{
	【此蓋編鐘之懸左高故其聲不和】}
田子方笑文侯曰何笑子方曰臣聞之君明樂官不明樂音今君審於音臣恐其聾於官也|{
	【明樂官知其才不才明樂音知其和不和五聲合和然後成音詩大序曰聲成文謂之音】}
文侯曰善子擊出遭田子方於道下車伏謁|{
	【古文考曰黄帝作車引重致遠少昊氏加牛禹時奚仲加馬釋名曰車居也韋昭曰古唯尺遮翻自漢以來始有居音蕭子顯曰三皇氏乘祗車出谷口車之始也祗翹移翻】}
子方不為禮子擊怒謂子方曰富貴者驕人乎貧賤者驕人乎子方曰亦貧賤者驕人耳富貴者安敢驕人國君而驕人則失其國大夫而驕人則失其家失其國者未聞有以國待之者也失其家者未聞有以家待之者也夫士貧賤者言不用行不合則納履而去耳安往而不得貧賤哉子擊乃謝之|{
	【夫音扶行下孟翻】}
文侯謂李克曰先生嘗有言曰家貧思良妻國亂思良相今所置非成則璜二子何如|{
	【李氏出自顓頊曾孫臯陶為堯大理以官命族為理氏商紂時裔孫利貞逃難食木子得全改為李氏置言置相也相息亮翻難乃旦翻】}
對曰卑不謀尊疏不謀戚臣在闕門之外不敢當命|{
	【在闕門之外謂疏遠也】}
文侯曰先生臨事勿讓克曰君弗察故也居視其所親富視其所與達視其所舉窮視其所不為貧視其所不取五者足以定之矣何待克哉文侯曰先生就舍吾之相定矣|{
	【相息亮翻】}
李克出見翟璜翟璜曰今者聞君召先生而卜相果誰為之克曰魏成翟璜忿然作色曰西河守吳起臣所進也|{
	【班志魏地其界自高陵以東盡河東河内高陵縣漢屬馮翊其地在河西所謂西河之列者也魏初使吳起守之秦兵不敢東向至惠王時秦使衛鞅擊虜其將公子卭遂獻西河之外於秦吳以國為姓相息亮翻守式又翻】}
君内以鄴為憂臣進西門豹|{
	【班志鄴縣屬魏郡西門豹為鄴令鑿渠以利民王符潜夫論姓氏篇曰如有東門西郭南宫北郭皆因居以為姓西門蓋亦此類鄴魚怯翻】}
君欲伐中山臣進樂羊中山已拔無使守之臣進先生君之子無傅臣進屈侯鮒|{
	【傅者傅之以德義因以為官名傅芳遇翻屈九勿翻姓也余按屈晉地時屬魏鮒蓋魏封屈侯也鮒音符遇翻】}
以耳目之所睹記臣何負於魏成|{
	【不勝為負】}
李克曰子言克於子之君者豈將比周以求大官哉|{
	【比毗至翻阿黨為比】}
君問相於克克之對如是|{
	【李克自叙其答魏文侯之言也】}
所以知君之必相魏成者魏成食祿千鍾|{
	【孔頴達曰祿者穀也故鄭註司錄云祿也言穀年穀豐然後制祿援神契云祿者錄也白虎通曰上以收錄接下下以名錄謹以事上是也六斛四斗為一鍾】}
什九在外什一在内是以東得卜子夏田子方段干木|{
	【夏戶雅翻】}
此三人者君皆師之子所進五人者君皆臣之子惡得與魏成比也|{
	【惡讀曰烏何也】}
翟璜逡巡再拜曰璜鄙人也失對願卒為弟子|{
	【逡七倫翻逡巡却退貌卒子恤翻終也孔頴達曰先生師也謂師為先生者言彼先己而生其德多厚也自稱為弟子者言己自處如弟子則尊其師如父兄也】}
吳起者衛人仕於魯齊人伐魯魯人欲以為將起取齊女為妻|{
	【將即亮翻下同取讀曰娶孔頴達曰妻之為言齊也以禮見問得與夫敵體也】}
魯人疑之起殺妻以求將大破齊師或譛之魯侯曰起始事曾參|{
	【世本曰曾姓出自鄫國陸德明曰參所金翻一音七南翻】}
母死不奔喪曾參絶之今又殺妻以求為君將起殘忍薄行人也|{
	【行下孟翻】}
且以魯國區區而有勝敵之名則諸侯圖魯矣起恐得罪聞魏文侯賢乃往歸之文侯問諸李克李克曰起貪而好色|{
	【好呼到翻】}
然用兵司馬穰苴弗能過也|{
	【司馬官名穰苴本齊田姓仕齊為是官故以稱之齊景公之賢將也穰如羊翻苴子余翻】}
於是文侯以為將擊秦拔五城起之為將與士卒最下者同衣食卧不設席行不騎乘|{
	【騎馬為騎乘車為乘言起與士卒同其勞苦行不用車馬也】}
親裹贏糧|{
	【師古曰贏擔也此言起親裹士卒所齎擔之糧贏恰成翻】}
與士卒分勞苦卒有病疽者起為吮之|{
	【疽七余翻癰也吮徐兖翻說文嗽也康所角翻】}
卒母聞而哭之人曰子卒也而將軍自吮其疽何哭為母曰非然也往年吳公吮其父疽其父戰不旋踵遂死于敵吳公今又吮其子妾不知其死所矣是以哭之 燕湣公薨子僖公立|{
	【燕自召公奭受封於北燕其地則唐幽州薊縣故城是也自召公至湣公三十二世燕因肩翻湣讀與閔同謚法使民悲傷曰閔小心畏忌曰僖】}
二十四年王崩子安王驕立 盜殺楚聲王國人立其子悼王|{
	【周成王封熊繹於楚姓芈氏居丹陽今枝江縣故丹陽城是也括地志曰歸州秭歸縣丹陽城熊繹之始國其後彊大北封畛於汝南并吳越地方五千里自熊繹至聲王三十世索隱曰聲王名當悼王名疑謚法不生其國曰聲注云生於外家年中早夭曰悼注云年不稱志又云恐懼從處曰悼註云從處言險圯也】}
\par
安王|{
	【謚法好和不爭曰安】}
\par
元年秦伐魏至陽孤|{
	【周孝王邑非子於秦徐廣曰今隴西縣秦亭是也括地志曰秦州清水縣本名秦十三州志曰秦亭秦谷是也至襄公取周地穆公覇西戎日以彊大是年秦簡公之十四年也自非子至簡公二十八世陽孤史記作陽狐正義引括地志曰陽狐郭在魏州元城縣東北三十里余按此時西河之外皆為魏境若秦兵至元城則是越魏都安邑而東矣水經注河東垣縣有陽壺城九域志絳州有陽壺城姑識之以廣異聞且俟知者】}
\par
二年魏韓趙伐楚至桑丘|{
	【水經注澺水自葛陂東南逕新蔡縣故城東而東南流注于汝水又東南逕下桑里左迆為横塘陂史記作乘丘正義地理志乘丘故城在兖州瑕丘縣西北三十五里當從之】}
鄭圍韓陽翟|{
	【周宣王封其弟友于鄭杜預世族譜曰封於咸林今京兆鄭邑是也幽王無道友徙其人於虢鄶之間遂有其地今河南新鄭是也友謚桓公是年鄭繻公駘之二十三年自桓公至繻公二十二世班志陽翟縣屬潁川郡索隱曰翟音狄温公類篇音萇伯翻繻詢趨翻駘堂來翻】}
韓景侯薨子烈侯取立 趙烈侯薨國人立其弟武侯 秦簡公薨子惠公立|{
	【諡法愛民好與曰惠】}
\par
三年王子定奔晉 虢山崩壅河|{
	【徐廣曰虢山在陜裴駰曰弘農陜縣故虢國北虢在大陽東虢在滎陽括地志曰虢山在陜州陜縣西臨黄河今臨河有岡阜似是頹山之餘水經注曰陜城西北帶河水湧起方數十丈父老云石虎載銅翁仲至此沉没水所以湧洪河巨瀆宜不為金狄梗流蓋魏文侯時虢山崩壅河所致耳陜失冉翻】}
\par
四年楚圍鄭鄭人殺其相駟子陽|{
	【鄭穆公之子騑字子駟古者以王父之字為氏子陽其後也相息亮翻騑芳菲翻】}
\par
五年曰有食之|{
	【杜預曰日行遲一歲一周天月行速一月一周天一歲凡十二交會然日月動物雖行度有大量不能不小有贏縮故有雖交會而不食者或有頻交而食者孔頴達曰日月交會謂朔也周天三百六十五度四分度之一日月皆右行於天一晝一夜日行一度月行十三度十九分度之七二十九日日有餘而月行天一周追及於日而與之會交會而日月同道則食月或在日道表或在日道裏則不食矣又歷家為交食之法大率以一百七十有三日有奇為限然月先在裏則依限而食者多若月在表則依限而食者少杜預見其參差乃云雖行度有大量不能不小有贏縮故有雖交會而不食者或有頻交而食者此得之矣蘇氏曰交當朔則日食然亦冇交而不食者交而食陽微而隂乘之也交而不食陽盛而陰不能揜也朱元晦曰此則繫乎人事之感蓋臣子背君父妾婦乘其夫小人陵君子夷狄侵中國所感如是則隂盛陽微而日為之食矣是以聖人於春秋每食必書而詩人亦以為醜也今此書年而不書月與晦朔史失之也釋名曰日月虧曰食稍小侵虧如蟲食草木之葉也亦作蝕】}
\par
三月盜殺韓相俠累俠累與濮陽嚴仲子有惡仲子聞軹人聶政之勇以黄金百溢為政母夀欲因以報仇|{
	【相息亮翻俠戶頰翻累力追翻濮陽春秋之帝丘漢為濮陽縣屬東郡應劭曰濮水南入鉅野水北為陽濮博木翻惡如字不善也康烏故翻非軹春秋原邑晉文公所闈者漢為軹縣屬河内郡音只姓譜曰楚大夫食采於聶因以為氏聶尼輒翻溢夷質翻二十四兩為溢】}
政不受曰老母在政身未敢以許人也及母卒仲子乃使政刺俠累|{
	【卒子恤翻刺七亦翻又如字】}
俠累方坐府上兵衛甚衆聶政直入上階|{
	【上時掌翻】}
刺殺俠累因自皮面抉眼自屠出腸韓人暴其尸於市|{
	【暴步木翻又如字】}
購問莫能識其姊嫈聞而往哭之曰是軹深井里聶政也|{
	【史記正義曰深井里在懷州濟源縣南三十里】}
以妾尚在之故重自刑以絶從妾奈何畏歿身之誅終滅賢弟之名遂死於政尸之㫄|{
	【皮面以刀剺而而去其皮懸賞以募告者曰購購古候翻嫈烏莖翻絶從之從讀曰蹤謂自絶其蹤跡又或曰從讀如字謂絶其從坐之罪也】}
\par
六年鄭駟子陽之黨弑繻公|{
	【繻者謚法所不載史記注繻或作繚繻詢趨翻】}
而立其弟乙|{
	【白虎通曰弟悌也心順行篤也行下孟翻】}
是為康公 宋悼公薨子休公田立|{
	【武王封微子啓於宋唐宋州之睢陽縣是也自微子二十七世至悼公名購由休亦謚法所不載】}
\par
八年齊伐魯取最|{
	【武王封太公於齊唐青州之臨淄是也括地志曰天齊水在臨淄東南十五里封禪書曰齊之所以為齊者以天齊是年康公貸之十一年自太公至康公二十九世成王封伯禽於魯唐兖州之曲阜是也是年穆公之十六年自伯禽至穆公凡二十八世】}
鄭負黍叛復歸韓|{
	【據史記繻公之十六年敗韓於負黍蓋以此時取之而今復叛歸韓也劉昭郡國志潁川郡陽城縣有負黍聚古今地名云負黍山在陽城縣西南二十七里或云在西南三十五里】}
\par
九年魏伐鄭 晉烈公薨子孝公傾立|{
	【周成王封弟叔虞於唐括地志曰故唐城在并州晉陽縣北二里堯所築也都城記曰唐叔虞之子燮父徙居晉水㫄今并州理故唐城即燮父初徙之處其城南半入州城中毛詩譜曰燮父以堯墟南有晉水改曰晉侯自唐叔至烈公三十七世烈公名止謚法慈惠愛親曰孝】}
\par
十一年秦伐韓宜陽取六邑|{
	【班志宜陽縣屬弘農郡史記正義曰宜陽縣故城在河南府福昌縣東十四里故韓城是也此邑即周禮四井為邑之邑】}
初田常生襄子盤盤生莊子白白生太公和|{
	【此序齊田氏之世也田常即左傳陳成子恒也温公避仁廟諱改恒曰常自陳公子完奔齊 五世至常得政謚法勝敵志強曰莊】}
是歲齊田和遷齊康公於海上使食一城以奉其先祀
\par
十二年秦晉戰于武城|{
	【此非魯之武城左傳晉陰飴甥會秦伯盟于王城杜預曰馮翊臨晉縣東有王城今名武鄉括地志故武城一名武平城在華州鄭縣東北十三里】}
齊伐魏取襄陽|{
	【陽當作陵徐廣曰今之南平陽也余據晉志南平陽縣屬山陽郡班志陳留郡有襄邑縣師古曰圈稱云襄邑宋地木承匡襄陵鄉也宋襄公所葬故曰襄陵秦始皇以承匡卑濕徙縣襄陵因曰襄邑】}
魯敗齊師于平陸|{
	【班志東平國有東平陸縣戰國時之平陸也史記正義曰平陸兖州縣即古厥國宋白曰鄆州中都縣漢為平陸縣史記魯敗齊師于平陸是也敗補邁翻】}
\par
十三年秦侵晉 齊田和會魏文侯楚人衛人于濁澤|{
	【康曰濁水名漢志濁水出齊郡廣縣媯山余謂康說誤矣徐廣史記註曰長社有濁澤水經注曰皇陂水出胡城西北胡城潁陰之狐人亭也皇陂古長社之濁澤也記諸侯相見於郤地曰會孔頴達曰諸侯未及期而相見曰遇會者謂及期之禮既及期人至所期之地】}
求為諸侯魏文侯為之請於王及諸侯王許之|{
	【為之之為于偽翻】}
\par
十五年秦伐蜀取南鄭|{
	【譜記普云蜀之先肇自人皇之際黄帝子昌意娶蜀山氏女生帝俈既立封其支庶于蜀歷虞夏商周周衰先稱王者蠶樷余據武王伐紂庸蜀諸國皆會于牧野孔安國曰蜀叟也春秋之時不與中國通班志南鄭縣屬漢中郡唐為梁州治所俈通作嚳音括沃翻】}
魏文侯薨太子擊立|{
	【王者以嫡長子為太子謂之國儲副君諸侯曰世子周衰率上僭孔頴達曰太者大中之太也上時掌翻長知兩翻】}
是為武侯武侯浮西河而下|{
	【西河即禹貢之龍門西河】}
中流顧謂吳起曰美哉山河之固此魏國之寶也對曰在德不在險昔三苖氏左洞庭右彭蠡德義不修禹滅之|{
	【武陵長沙零桂之水匪為洞庭周七百里彭蠡澤在漢豫章郡彭澤縣西書有苖弗率汝徂征三苖所居蓋今江南西道之地蠡里弟翻】}
夏桀之居左河濟右泰華伊闕在其南羊腸在其北修政不仁湯放之|{
	【濟水出河東垣縣王屋山南流貫河而南合于滎瀆禹貢所謂導沇水東流為濟溢為滎者也自漢築滎陽石門而濟與河合流而注于海不入滎瀆禹貢所謂導沇水東流為濟入于河桀都安邑蓋恃以為險泰華山在京兆華陰縣南水經伊水出南陽縣西荀渠山東北流至河南新城縣又東南過伊闕中大禹所鑿也兩山相對望之若闕左傳女寛守闕塞即其地括地志伊闕山在洛州南十九里班志上黨壺關縣有羊腸阪此安邑四履所憑山河之固也書曰成湯放桀于南巢濟子禮翻華戶化翻】}
商紂之國左孟門右太行常山在其北大河經其南修政不德武王殺之|{
	【水經注孟門在河東北屈縣西即龍門上口也淮南子曰龍門未闢呂梁未鑿河出孟門之上溢而逆流無有丘陵名曰洪水太行山在河内野王縣西北常山在常山郡上曲陽縣西北河水自孟門南抵華陰屈而東流紂都朝歌河經其南北屈之孟門在朝歌西北恐不可言左索隱曰孟門别一山在朝歌東此特左右二字之差而誤耳春秋說題辭河之為言荷也荷精分布懷陰引度也釋名河下也随地下處而通流也書曰武王勝殷殺紂太行之行戶剛翻北屈陸求忽翻顔居勿翻】}
由此觀之在德不在險若君不修德舟中之人皆敵國也武侯曰善魏置相相田文|{
	【相息亮翻此田文非齊之田文】}
吳起不悦謂田文曰請與子論功可乎田文曰可起曰將三軍使士卒樂死敵國不敢謀子孰與起文曰不如子|{
	【將即亮翻樂音洛】}
起曰治百官親萬民實府庫子孰與起文曰不如子|{
	【治直之翻】}
起曰守西河秦兵不敢東鄉韓趙賓從子孰與起文曰不如子|{
	【鄉讀曰向賓從猶言賓服也】}
起曰此三者子皆出吾下而位居吾上何也文曰主少國疑大臣未附百姓不信方是之時屬之子乎屬之我乎|{
	【少詩照翻屬子欲翻】}
起默然良久曰屬之子矣久之魏相公叔尚主而害吳起|{
	【如淳口天子嫁女於諸侯必使諸侯同姓者主之故謂之公主帝姊妹曰長公主諸王女曰翁主師古曰如說得之天子不親主㛰故謂之公主諸王則自主婚故其女曰翁主翁者父也言父主其婚也亦曰王主言王自主其婚也揚雄方言云周晉秦隴謂父曰翁而臣瓚王楙或云公者比於上爵或云主者婦人尊稱皆失之劉貢父曰予謂公主之稱本出秦舊男為公子女為公主古者大夫妻稱主故以公配之若謂同姓主之故謂之公主則周之事秦不知用也古之嫁女禮當如周使大夫主之何不謂之夫主乎然則謂之王主者猶言王子也謂之公主者緣公而生耳毛晃曰尚崇也高也貴也飾也加也尊也娶公主謂之尚言帝王之女尊而尚之不敢言娶也相息亮翻】}
公叔之僕曰起易去也起為人剛勁自喜|{
	【易以䜴翻去起呂翻師古曰喜許吏翻】}
子先言於君曰吳起賢人也而君之國小臣恐起之無留心也君盍試延以女起無留心則必辭矣子因與起歸而使公主辱子起見公主之賤子也必辭則子之計中矣|{
	【中竹仲翻】}
公叔從之吳起果辭公主魏武侯疑之而未信起懼誅遂奔楚楚悼王素聞其賢至則任之為相起明灋審令|{
	【相息亮翻灋占法字】}
捐不急之官廢公族疏遠者以撫養戰鬭之士要在彊兵破遊說之言從横者|{
	【捐余專翻弃也除去也漢書音義曰以利合曰從以威力相脅曰横或曰南北曰從從者連南北為一西鄉以擯秦東西曰横横者離山東之交使之西鄉以事秦說式芮翻從即容翻横亦作衡音同】}
於是南平百越|{
	【韋昭曰越有百邑】}
北却三晉西伐秦諸侯皆患楚之彊而楚之貴戚大臣多怨吳起者 秦惠公薨子出公立|{
	【出非謚也以其失國出死故曰出公】}
趙武侯薨國人復立烈侯之太子章是為敬侯|{
	【謚法夙夜警戒曰敬】}
韓烈侯薨子文侯立十六年初命齊大夫田和為諸侯|{
	【田氏自此遂有齊國田和是為太公】}
趙公子朝作亂奔魏與魏襲邯鄲不克|{
	【邯音寒鄲音丹】}
\par
十七年秦庶長改逆獻公于河西而立之殺出子及其母沈之淵旁|{
	【後秦制爵一級曰公士二上造三簪裊四不更五大夫六官大夫七公大夫八公乘九五大夫十左庶長十一右庶長十二左更十三中更十四右更十五少上造十六大上造十七駟車庶長十八大庶長十九關内侯二十徹侯師古曰庶長言衆列之長註又詳見下卷顯王十年前據史記威烈王十一年秦靈公卒子獻公師隰不得立立靈公季父悼子是為簡公出子簡公之孫也今庶長改迎獻公而殺出子正義曰西者秦州西縣秦之舊地時獻公在西縣故迎立之余謂此言河西非西縣也靈公之卒獻公不得立出居河西河西者黄河之西蓋漢凉州之地裊當作褭乃了翻更工衡翻乘繩證翻長知丈翻】}
齊伐魯 韓伐鄭取陽城|{
	【漢陽城縣屬潁川郡是為地中成周於此以土圭測日景】}
伐宋執宋公齊太公薨子桓公午立
\par
十九年魏敗趙師于兔臺|{
	【史記趙世家曰魏敗我兔臺築剛平正義曰兔臺剛平並在河北敗補邁翻】}
\par
二十年日有食之既|{
	【既盡也】}
\par
二十一年楚悼王薨貴戚大臣作亂攻吳起起走之王尸而伏之|{
	【之往也往赴王尸而伏其側】}
擊起之徒因射刺起并中王尸|{
	【射而亦翻刺七亦翻中竹仲翻】}
既葬肅王即位|{
	【謚法剛德克就曰肅執心决斷曰肅】}
使令尹盡誅為亂者|{
	【令尹楚相也】}
坐起夷宗者七十餘家|{
	【夷殺也夷宗者殺其同宗也】}
\par
二十二年齊伐燕取桑丘魏韓趙伐齊至桑丘|{
	【此桑丘非二年所書楚之桑丘括地忠曰桑丘故城俗名敬城在易州遂城縣蓋燕之南界也】}
\par
二十三年趙襲衛不克|{
	【成王封康叔於衛居河淇之間故殷墟也至懿公為狄所滅東徙度河文公徙居楚丘遂國於濮陽是年愼公頹之三十五年自康叔至愼公凡三十二世】}
齊康公薨無子田氏遂并齊而有之|{
	【姜氏至此滅矣】}
是歲齊桓公亦薨子威王因齊立|{
	【謚法彊毅訅正曰威訅渠留翻齊桓公田午訅謀也】}
\par
二十四年狄敗魏師于澮|{
	【漢之中山上黨西河上郡自春秋以來狄皆居之此亦其種也水經澮水出河東絳縣東澮山西過絳縣南又西南過祁宫南又西南至王橋入汾水括地志澮山在絳州翼城縣東北敗補邁翻澮古外翻】}
魏韓趙伐齊至靈丘|{
	【史記正義曰靈丘河東蔚州縣余按蔚州之靈丘即漢代郡之靈丘此時齊境安能至代北邪此即孟子謂蚳鼃辭靈丘請士師之地班志曰齊地北有千乘清河以南漢清河郡有靈縣清河北接趙魏之境此為近之蚳音遲鼃烏花翻】}
晉孝公薨子靖公俱酒立|{
	【謚法柔衆安民曰靖又恭已鮮言曰靖】}
\par
二十五年蜀伐楚取兹方|{
	【據史記蜀伐楚取兹方楚為扞關以拒之則兹方之地在扞關之西劉昭志巴郡魚復縣有扞關】}
子思言苟變於衛侯曰其才可將五百乘|{
	【古者兵車一乘甲士三人步卒七十二人五百乘三萬七千五百人國語曰苟本自黄帝之子將即亮翻下同乘繩證翻】}
公曰吾知其可將然變也嘗為吏賦於民而食人二鷄子故弗用也子思曰夫聖人之官人猶匠之用木也|{
	【夫音扶】}
取其所長棄其所短故梓連抱而有數尺之朽良工不棄今君處戰國之世|{
	【處昌呂翻】}
選爪牙之士而以二卵棄干城之將|{
	【詩赳赳武夫公侯干城毛氏傳曰干扞也音戶旦翻鄭氏箋曰干也城也皆所以禦難也干讀如字】}
此不可使聞於鄰國也公再拜曰謹受教矣衛侯言計非是而羣臣和者如出一口|{
	【和戶卧翻】}
子思曰以吾觀衛所謂君不君臣不臣者也|{
	【君不君臣不臣論語載齊景公之言】}
公丘懿子曰何乃若是|{
	【公丘複姓謚法温柔賢善曰懿】}
子思曰人主自臧則衆謀不進|{
	【臧善也】}
事是而臧之猶却衆謀况和非以長惡乎|{
	【和戶卧翻長知丈翻】}
夫不察事之是非而悦人讚已闇莫甚焉不度理之所在而阿諛求容諂莫甚焉|{
	【度徒洛翻】}
君闇臣諂以居百姓之上民不與也若此不已國無類矣子思言於衛侯曰君之國事將日非矣公曰何故對曰有由然焉君出言自以為是而卿大夫莫敢矯其非卿大夫出言亦自以為是而士庶人莫敢矯其非君臣既自賢矣|{
	【白虎通曰君羣也羣心之所歸心也臣堅也厲志自堅也】}
而羣下同聲賢之賢之則順而有福矯之則逆而有禍如此則善安從生詩曰具曰予聖誰知烏之雌雄|{
	【詩正月之辭毛氏傳曰君臣俱自謂聖也鄭氏箋曰時君臣賢愚適同如烏之雌雄相似誰能别異之乎又曰烏之雌雄不可别者以翼右掩左雄左掩右雌陰陽相下之義也】}
抑亦似君之君臣乎 魯穆公薨子共公奮立|{
	【謚法布德就義曰穆中情見貌曰穆尊賢敬讓曰共既過能改曰共執事堅固曰共共讀曰㳟 考異曰司馬遷史記六國表周威烈王十九年甲戌魯穆公元年烈王元年丙午共公元年顯王十七年己巳康公元年二十六年戊寅景公元年赧王元年丁未平公元年二十年丙寅文公元年四十三年己丑頃公元年五十九年乙巳周亡秦莊襄元年壬子楚滅魯按魯世家穆公三十三年卒若元甲戌終乙巳則是三十二年也共公二十二年卒若元丙午終戊辰則是二十三年也康公九年卒景公二十五年卒平公二十二年卒若元丁未終乙丑則是十九年也文公二十三年卒頃公二十四年楚滅魯班固漢書律歷志文公作緡公其在位之年與世家異者惟平公二十年耳本志自魯僖公五年正月辛亥朔旦冬至推之至成公十二年正月庚寅朔旦冬至定公七年正月己巳朔旦冬至元公四年正月戊申朔旦冬至康公四年正月丁亥朔旦冬至緡公二十二年正月丙寅朔旦冬至漢高祖八年十一月乙巳朔旦冬至武帝元朔六年十一月甲申朔旦冬至元帝初元二年十一月癸亥朔旦冬至共間相距皆七十六年此最為得實又與魯世家注皇甫謐所紀歲次皆合今從之六國表差謬難可盡據也余按考異自魯僖公五年至漢元帝初元二年六百餘年間十二月朔旦冬至相距皆七十六年此最為得實又與魯世家注皇甫謐所紀歲次皆合蓋謂劉彛叟長歷也且言史記六國表差謬難可盡據又按通鑑目錄編年用劉彝叟長歷漢武帝太初元年初用夏正定歷史記歷書是年書閼逢攝提格目錄書彊圉赤奮若閼逢攝提格甲寅也彊圉赤奮若丁丑也有二十四年之差温公用彝叟歷邵康節皇極經世書亦用彝叟歷康節少自雄其才既學力慕高遠一見李之才遂從而受學廬于共城百源冬不爐夏不扇夜不就席者數年覃思於易經也皇極經世書不能違彜叟歷及其來居於洛而温公亦奉祠以書局在洛相過從稔又夙所敬者也余意其講明之間必嘗及此而决于用彜叟歷讀考異此一段辭意可見】}
韓文侯薨子哀侯立
\par
二十六年王崩子烈王喜立 魏韓趙共廢晉靖公為家人而分其地|{
	【唐叔不祀矣】}
\par
烈王|{
	【名喜安王之子】}
\par
元年日有食之 韓滅鄭因徙都之|{
	【韓本都平陽其地屬漢之河東郡中間徙都陽翟鄭都新鄭其地屬漢之河南郡鄭桓公始封于鄭其地屬漢之京兆後滅虢鄶而國於溱洧之間故曰新鄭左傳鄭莊公所謂吾先君新邑於此是也今韓既滅鄭自陽翟徙都之韓既都鄭故時人亦謂韓王為鄭王考之戰國策韓非子可見】}
趙敬侯薨子成侯種立|{
	【種章勇翻】}
三年燕敗齊師於林狐|{
	【敗補邁翻】}
魯伐齊入陽關|{
	【徐廣曰陽關在鉅平班志鉅平縣屬太山郡括地志陽關故城在兖州博城縣南二十九里其城之西臨汶水汶音問】}
魏伐齊至博陵|{
	【史記正義曰博陵在濟州西界宋白曰史記齊威王伐晉至博陵徐廣曰東郡之博平漢為縣】}
燕僖公薨子桓公立 宋休公薨子辟公立|{
	【辟亦謚法之所不載】}
衛愼公薨子聲公訓立|{
	【謚法敏以敬曰愼戴記思慮深遠曰愼】}
四年趙伐衛取都鄙七十三|{
	【周禮太宰以八則治都鄙注云都之所居曰鄙都鄙卿大夫之采邑蓋周之制四縣為都方四十里一千六百井積一萬四千四百夫五都為鄙鄙五百家也此時衛國褊小若都鄙七十三以成周之制率之其地廣矣盡衛之提封未必能及此數也更俟博考】}
魏敗趙師于北藺|{
	【班志西河郡有藺縣史記正義曰在石州其地於趙為西北故曰北藺藺離進翻】}
\par
五年魏伐楚取魯陽|{
	【左傳所謂劉累遷於魯縣即魯陽也班志魯陽縣屬南陽郡史記正義曰今汝州魯山縣】}
韓嚴遂弑哀侯國人立其子懿侯初哀侯以韓廆為相而愛嚴遂二人甚相害也嚴遂令人刺韓廆於朝廆走哀侯哀侯抱之人刺韓廆兼及哀侯|{
	【戰國策以聶政刺韓相事及并中哀侯為一事此從史記蜀本注曰按太史公年表及韓世家於韓烈侯三年皆書聶政殺韓相俠累於哀侯六年又皆書嚴遂弑哀侯以刺客傳考之聶政殺俠累事在哀侯時以戰國策考之亦然從傳與戰國策則是年表世家於烈侯三年書盗殺俠累誤矣通鑑於烈侯三年載聶政刺俠累事又於哀侯六年韓嚴遂殺其君哀侯是從年表世家所書蓋刺客傳初不言并殺哀侯止戰國策言之通鑑豈以此疑之歟故載并刺哀侯不書聶政止曰使人以此求之則通鑑之意不以嚴仲子為嚴遂亦不以俠累為韓廆止從年表世家而不信其傳也余按温公與劉道原書亦疑此事廆戶賄翻相息亮翻刺七亦翻朝直遥翻走音奏】}
魏武侯薨不立太子子罃與公中緩爭立國内亂|{
	【罃於耕翻中讀曰仲】}
\par
六年齊威王來朝是時周室微弱諸侯莫朝而齊獨朝之天下以此益賢威王|{
	【朝直遥翻】}
趙伐齊至鄄|{
	【班志濟陰郡有鄄城縣鄄工掾翻】}
魏敗趙師于懷|{
	【班志河内郡有懷縣魏收地形志懷州武德郡有懷縣縣管内有懷城敗補邁翻】}
齊威王召即墨大夫語之曰自子之居即墨也毁言日至|{
	【班志即墨縣屬膠東國括地志即墨故城在萊州膠水縣南六十里宋白曰城臨墨水故曰即墨語牛倨翻下同】}
然吾使人視即墨田野辟|{
	【辟讀曰闢下同】}
人民給官無事東方以寜是子不事吾左右以求助也封之萬家召阿大夫語之曰自子守阿譽言日至|{
	【阿即東阿縣班志屬東郡譽音余稱其美也】}
吾使人視阿田野不辟人民貧餒昔日趙攻鄄子不救|{
	【鄄工掾翻】}
衛取薛陵子不知|{
	【薛陵春秋薛國之墟也班志薛縣屬魯國而衛國在漢東郡陳留界薛陵屬齊而近於衛故為所取齊後封田嬰於此】}
是子厚幣事吾左右以求譽也是日烹阿大夫及左右嘗譽者於是羣臣聳懼莫敢飾詐務盡其情齊國大治彊於天下|{
	【譽音余治直吏翻】}
楚肅王薨無子立其弟良夫是為宣王宋辟公薨子剔成立|{
	【剔他歷翻】}
\par
七年日有食之 王崩弟扁立|{
	【據班書古今人表師古注扁音篇】}
是為顯王 魏大夫王錯出奔韓|{
	【姓譜王氏之所自出非一出太原琅邪者周靈王太子晉之後北海陳留齊王田和之後東海出自姬姓高平京兆魏信陵君之後天水東平新蔡新野山陽中山章武東萊河東者殷王子比干為紂所害子孫以王者之後號曰王氏余謂此皆後世以諸郡著姓言之耳春秋之時自有王姓莫能審其所自出】}
公孫頎謂韓懿侯曰魏亂可取也|{
	【公孫姓也黄帝公孫氏頎渠希翻】}
懿侯乃與趙成侯合兵伐魏戰于濁澤大破之遂圍魏|{
	【史記正義曰徐廣以為長社濁澤非也括地志云濁水源出蒲州解縣東北平地爾時魏都安邑韓趙伐魏豈至河南長社邪解縣濁水近於魏都當是也】}
成侯曰殺罃立公中緩割地而退我二國之利也懿侯曰不可殺魏君暴也割地而退貪也不如兩分之魏分為兩不彊於宋衛則我終無魏患矣趙人不聽懿侯不悦以其兵夜去趙成侯亦去罃遂殺公中緩而立|{
	【中讀曰仲】}
是為惠王太史公曰魏惠王所以身不死國不分者二國之謀不和也若從一家之謀魏必分矣故曰君終無適子其國可破也|{
	【索隱曰蓋古人之言及俗說故云故曰適讀曰嫡】}
\par
% 資治通鑑卷一
% \par
