\chapter{資治通鑑卷一}


%\catcode`曰\active
%\def曰{\textcolor{blue}{\string曰}}

宋 司馬光 撰\\胡三省 音註\\

周紀一|{
	起著雍攝提格,盡玄黓困敦,凡三十五年。
	爾雅:太歲
	在甲曰閼逢,
	在乙曰旃蒙,
	在丙曰柔兆,
	在丁曰強圉,
	在戊曰著雍,
	在己曰屠維,
	在庚曰上章,
	在辛曰重光,
	在壬曰玄黓,
	在癸曰昭陽,
	是爲歲陽。
	在寅曰攝提格,
	在卯曰單閼,
	在辰曰執徐,
	在巳曰大荒落,
	在午曰敦牂,
	在未曰協洽,
	在申曰涒灘,
	在酉曰作噩,
	在戌曰掩茂,
	在亥曰大淵獻,
	在子曰困敦,
	在丑曰赤奮若,
	是爲歲名。
	周紀分註 “起著雍攝提格” 起戊寅也。
	“盡玄黓困敦” 盡壬子也。
	閼,讀如字;史記作“焉”,於乾翻。
	著,陳如翻。
	雍,於容翻。
	黓,逸職翻。
	單閼,上音丹,又特連翻;
	下,烏葛翻,又於連翻。
	牂,作郎翻。
	涒,吐䰟翻。
	灘,吐丹翻。
	困敦,音頓。
	杜預世族譜曰:周,黃帝之苗裔,姬姓。
	后稷之後,封于邰;及夏衰,稷子不窋竄于西戎。
	至十二代孫太王,避狄遷岐;
	至孫文王受命,武王克商而有天下。
	自武王至平王凡十三世,
	自平王至威烈王又十八世,
	自威烈王至赧王又五世。
	張守節曰:因太王居周原,國號曰周。
	地理志云:右扶風美陽縣岐山西北中水鄉,周太王所邑。
	括地志云:故周城一名美陽城,在雍州武功縣西北二十五里。
	紀,理也,統理衆事而繫之年月。
	溫公繫年用春秋之法,因史、漢本紀而謂之紀。
	邰,湯來翻。
	夏,戶雅翻。
	窋,竹律翻。
	在雍,於用翻。
}
\par

威烈王|{
	名午,考王之子。
	諡法:猛以剛果曰威;有功安民曰烈。
	沈約曰:諸複諡,有諡人,無諡法。
	}

二十三年|{
	上距春秋獲麟七十八年,距左傳趙襄子惎智伯事七十一年。
	惎,毒也,音其冀翻。
}

1初命晉大夫魏斯、趙籍、韓虔爲諸侯。|{
	此溫公書法所由始也。
	魏之先,畢公高後,與周同姓;
	其苗裔曰畢萬,始封于魏。
	至魏舒,始爲晉正卿;
	三世至斯。
	趙之先,造父後;
	至叔帶,始自周適晉;
	至趙夙,始封於耿。
	至趙盾,始爲晉正卿;六世至籍。
	韓之先,出於周武王;至韓武子事晉,封於韓原。
	至韓厥,爲晉正卿;六世至虔。
	三家者,世爲晉大夫,於周則陪臣也。
	周室旣衰,晉主夏盟,以尊王室,故命之爲伯。
	三卿竊晉之權,暴蔑其君,剖分其國,此王法所必誅也。
	威烈王不惟不能誅之,又命之爲諸侯,是崇奨奸名犯分之臣也。
	通鑑始於此,其所以謹名分歟!
}


臣光曰:臣聞天子之職莫大於禮,禮莫大於分,分莫大於名。|{
	分,扶問翻;下同。
}
何謂禮?紀綱是也。
何謂分?君臣是也。
何謂名?公、侯、卿大夫是也。


夫以四海之廣,|{
	夫以,音扶。
}
兆民之衆,受制於一人,
雖有絕倫之力,高世之智,
莫[敢]不奔走而服役者,
豈非以禮爲之紀綱[互乙]哉!
是故天子統三公,|{
	統,他綜翻。
	}
三公率諸侯,諸侯制卿大夫,卿大夫治士庶人。|{
	治,直之翻。
	}
貴以臨賤,賤以承貴。
上之使下猶心腹之運手足,根本之制支葉,下之事上猶手足之衛心腹,支葉之庇本根,然後能上下相保而國家治安。|{
	治,直吏翻。
}
故曰天子之職莫大於禮也。


文王序易,以乾、坤爲首。
孔子繫之曰:“天尊地卑,乾坤定矣。
卑高以陳,貴賤位矣。”|{
	繫,戶計翻。}
言君臣之位猶天地之不可易也。
春秋抑諸侯,尊王[周]室,王人雖微,序於諸侯之上,以是見聖人於君臣之際未嘗不惓惓也。|{
	惓,逵員翻。
	漢劉向傳:忠臣畎畝,猶不忘君惓惓之義也。
	惓惓,猶言勤勤也。}
非有桀、紂之暴,湯、武之仁,人歸之,天命之,君臣之分當守節伏死而已矣。
是故以微子而代紂則成湯配天矣,|{
	史記:商帝乙生三子,
	長曰微子啓,
	次曰中衍,
	季曰紂。
	紂之母爲后。
	帝乙欲立啓爲太子,太史據法爭之曰:
	“有妻之子,不可立妾之子。”
	乃立紂。
	紂卒以暴虐亡殷國。
	孔[鄭]玄義曰:物之大者莫若於天;
	推父比天,與之相配,行孝之大,莫大於此,所謂 “嚴父莫大於配天” 也。
	又孔氏曰:禮記稱萬物本乎天,人本乎祖。
	俱爲其本,可以相配,故王者皆以祖配天。
	諡法:除殘去虐曰湯。
	然諡法起於周;盖殷人先有此號,周人遂引以爲諡法。
	分,扶問翻。
	長,知兩翻。
	卒,子恤翻。
}
以季札而君吳則太伯血食矣,|{
	吳王壽夢有子四人:長曰諸樊,次曰餘祭,次曰餘昧,次曰季札。
	季札賢,壽夢欲立之,季札讓不可,於是立諸樊。
	諸樊卒,以授餘祭,欲兄弟以次相傳,必致國於季札;季札終讓而逃。
	之其後諸樊之子光與餘昧之子僚爭國,至於夫差,吳遂以亡。
	宗廟之祭用牲,故曰血食。
	太伯,吳立國之君。
	范甯曰:太者,善大之稱;伯者,長也。
	周太王之元子,故曰太伯。
	陸德明曰:壽夢,莫公翻。
	餘祭,側介翻。
	餘昧,音末。
	}
然二子寧亡國而不爲者,誠以禮之大節不可亂也。
故曰禮莫大於分也。


夫禮,辯貴賤,序親疏,裁羣物,制庶事,非名不著,非器不形;
名以命之,器以別之,|{
	夫,音扶。
	別,彼列翻。
	}
然後上下粲然有倫,此禮之大經也。
名器旣亡,則禮安得獨在哉!
昔仲叔于奚有功於衛,辭邑而請繁纓,孔子以爲不如多與之邑。
惟名與器,不可以假人,君之所司也;政亡則國家從之。|{
	左傳:衛孫桓子帥師與齊師戰于新築,衛師敗績。
	新築人仲叔于奚救孫桓子,桓子是以免。
	旣而衛人賞之邑,辭;請曲縣,繁纓以朝,許之。
	孔子聞之曰:“不如多與之邑,惟名與器不可以假人。”
	繁纓,馬飾也。
	繁,馬鬣上飾;
	纓,馬膺前飾。
	晉志註曰:纓在馬膺如索帬。
	繁,音蒲官翻。
	纓,伊盈翻。
	索,昔各翻。
	}
衛君待孔子而爲政,孔子欲先正名,以爲名不正則民無所措手足。|{
	見論語。
	}
夫繁纓,小物也,而孔子惜之;
正名,細務也,而孔子先之:|{
	先,悉薦翻。
	}
誠以名器旣亂則上下無以相保故也。
夫事未有不生於微而成於著,聖人之慮遠,故能謹其微而治之,|{
	治,直之翻;下同。
	}
衆人之識近,故必待其著而後救之;
治其微則用力寡而功多,
救其著則竭力而不能及也。
易曰:“履霜堅冰至,”|{
	坤初六爻辭。
	象曰:履霜堅冰,隂始凝也。馴致其道,至堅冰也。
	}
書曰:“一日二日萬幾,”|{
	臯陶謨之辭。
	孔安國註曰:幾,微也。
	言當戒懼萬事之微。
	幾,居依翻。}
謂此類也。
故曰分莫大於名也。|{
	分,扶問翻。
	}
嗚呼!幽、厲失德,周道日衰,綱紀散壞,下陵上替,諸侯專征,|{
	謂齊桓公,晉文公至悼公以及楚莊王、吳夫差之類。
	}
大夫擅政,|{
	謂晉六卿、魯三家、齊田氏之類。
	}
禮之大體什喪七八矣,|{
	喪,息浪翻。
	}
然文、武之祀猶緜緜相屬者,|{
	屬,聯屬也,音之欲翻。
	凡聯屬之屬皆同音。
	}
蓋以周之子孫尚能守其名分故也。
何以言之?
昔晉文公有大功於王室,請隧於襄王,襄王不許,曰:
	“王章也。未有代德而有二王,亦叔父之所惡也。
	不然,叔父有地而隧,又何請焉!”
文公於是懼而不敢違。|{
	太叔帶之難,襄王出居于氾。
	晉文公帥師納王,殺太叔帶。
	旣定襄王于郟,王勞之以地,辭;請隧焉,王弗許云云。
	杜預曰:闕地通路曰隧,此乃王者葬禮也。
	諸侯皆縣柩而下。
	王章者,章顯王者異於諸侯。
	古者天子謂同姓諸侯爲伯父叔父
	隧音遂
	惡,烏路翻。
	難,乃旦翻。
	氾,音汎。
	勞,力到翻。
	闕,其月翻。
	縣,音玄。
	柩,其久翻。
	}
是故以周之地則不大於曹、滕,以周之民則不衆於邾、莒,|{
	曹、滕、邾、莒,春秋時小國。
	莒,居許翻。
	}
然歷數百年,宗主天下,雖以晉、楚、齊、秦之強不敢加者,何哉?
徒以名分尚存故也。
至於季氏之於魯,田常之於齊,白公之於楚,智伯之於晉,|{
	魯大夫季氏 ,自季友以來,世執魯國之政。
	季平子逐昭公,季康子逐哀公 ,然終身北面,不敢簒國。
	田常,卽陳恆。
	田氏本陳氏;
	溫公避國諱,改`恆'曰`常'。
	陳成子得齊國之政,殺闞止,弑簡公,而亦不敢自立。
	史記世家以陳敬仲完爲田敬仲完,陳成子恆爲田常,故通鑑因以爲據。
	白公勝殺楚令尹子、西司馬子期,石乞曰:“焚庫弑王,不然不濟!”
	白公曰:“弑王不祥,焚庫無聚。”
	智伯當晉之衰,專其國政,侵伐鄰國,於晉大夫爲最強;
	攻晉出公,出公道死。
	智伯欲幷晉而不敢,乃奉哀公驕立之。
	}
其勢皆足以逐君而自爲,然而卒不敢者,|{
	卒,子恤翻,終也。
	}
豈其力不足而心不忍哉,乃畏奸名犯分而天下共誅之也。|{
	奸,居寒翻,亦犯也。
	分,扶問翻。}
今晉大夫暴蔑其君,剖分晉國,|{
	史記六國年表:定王十六年,趙、魏、韓滅智伯,遂三分晉國。
	}
天子旣不能討,又寵秩之,使列於諸侯,是區區之名分復不能守而幷棄之也。|{
	陸德明經典釋文:凡復字,其義訓又者,並音扶又翻。
	}
\textcolor{red}{先王之禮於斯盡矣!}

或者以爲當是之時,周室微弱,三晉強盛,|{
	三家分晉國,時因謂之`三晉',猶後之三秦、三齊也。
	}
雖欲勿許,其可得乎!
是大不然。
夫三晉雖強,苟不顧天下之誅而犯義侵禮,則不請於天子而自立矣。
不請於天子而自立,則爲悖逆之臣,|{
	【夫音扶悖蒲內翻又蒲沒翻】}
天下苟有桓、文之君,必奉禮義而征之。
今請於天子而天子許之,是受天子之命而爲諸侯也誰得而討之!
故三晉之列於諸侯,非三晉之壞禮,乃天子自壞之也。|{
	【壞音怪人毁之也】
	}


烏呼君臣之禮旣壞矣,|{
	此壞,其義爲成壞之壞,讀如字。
	}
則天下以智力相雄長,|{
	長,知兩翻。
	}
遂使聖賢之後爲諸侯者,社稷無不泯絕,|{
	謂齊、宋亡於田氏,
	魯、陳、越亡於楚,
	鄭亡於韓也。
	泯,彌忍翻,盡也,又彌鄰翻。
	毛晃曰:沒也,滅也。
	}
生民之類糜滅幾盡,|{
	說文曰:糜,糝也;
	取糜爛之義,音忙皮翻。
	幾,居依翻,又渠希翻,近也。
	}
豈不哀哉!

2初,智宣子將以瑤爲後,智果曰:“不如宵也|{
	韋昭曰:智宣子,晉卿荀躒之子申也,瑤,宣子之子智伯也,諡曰襄子。
	智果,智氏之族也。
	宵,宣子之庶子也。
	按諡法:聖善周聞曰宣。
	智氏溢美也。
	}
瑤之賢於人者五,其不逮者一也。|{
	韋昭曰:不仁也。
	}
美鬢長大則賢,|{
	通鑑俗傳寫者多作‘美鬚’,非也。
	國語作‘美鬢’今從之。
	}
射御足力則賢,伎藝畢給則賢,巧文辯慧則賢,|{
	韋昭曰:給,足也。
	巧文,巧於文辭。
	伎,渠綺翻。
	}
強毅果敢則賢;如是而甚不仁。
夫以其五賢陵人而以不仁行之,其誰能待之?|{
	韋昭曰:待猶假也。
	}
若果立瑤也,智宗必滅。”
弗聽。
智果別族於太史,爲輔氏。|{
	此事見國語。
	按左傳哀公二十三年,晉荀瑤伐齊,始見于傳。
	哀二十三年,史記元王五年也。
	荀躒,智文子也。
	定十四年,智文子猶見于傳。
	智宣子之事,傳無所考。
	立瑤之議,當在元王五年之前。
	韋昭曰:太史掌氏姓,周禮春官之屬。
	小史掌定世繫,辩昭穆。
	鄭司農注云:史官主書,故韓宣子聘魯,觀書于太史。
	世繫,謂帝繫、世本之屬是也;
	小史主定之。
	賈公彥疏曰:註引太史證之者,太史史官之長,共其事故也。
	蓋周之制,小史定姓氏,其書則太史掌。
	之智果欲避智氏之禍,故於太史別族。
	宋祁國語補音:別,彼列翻;又如字。
	}

趙簡子之子,長曰伯魯,幼曰無恤。|{
	趙簡子,文子之孫鞅也。
	諡法:一德不懈曰簡。
	白虎通曰:子,孳也,孳孳無已也。
	趙岐曰:子者男子之通稱也。
	長,知兩翻。
	}
將置後,不知所立,乃書訓戒之辭於二簡,|{
	孔穎達曰:書者,舒也。
	書緯璇璣鈐云:書者,如也。
	則書者,寫其言如其意,得展舒也。
	世本曰:沮誦、蒼頡作書。
	釋文名曰:書,庶也,紀庶物也;
	亦言著也,著之簡紙,求不滅也。
	簡,竹策也。
	}
以授二子曰:“謹識之!”|{
	識,職吏翻,記也。
	}
三年而問之,伯魯不能舉其辭;求其簡,已失之矣。
問無恤,誦其辭甚習:|{
	習,熟也。}
求其簡,出諸袖中而奏之。|{
	毛晃曰:奏,進上也。
	}
於是簡子以無恤爲賢,立以爲後。


簡子使尹鐸爲晉陽,|{
	【姓譜尹少昊之子封于尹城子孫因爲氏韋昭曰晉陽趙氏邑爲治也班志曰晉陽故詩唐國周成王滅唐封弟叔虞龍山在西晉水所出東入汾臣瓚曰所謂唐今河東永安縣是也去晉四百里括地志曰晉陽故城今名晉城在蒲州虞鄉縣西今按水經注晉水出晉陽縣西龍山昔智伯遏晉水以灌晉陽其水分爲二流北瀆卽智氏故渠也洞過水出沾縣北山西過榆次縣南又西到晉陽縣南榆次縣南水側有鑿臺戰國策所謂智伯死於鑿臺之下卽此處也參而考之晉陽故城恐不在蒲州水經注人云叔虞封於唐縣有晉水故改名爲晉子夏序詩此晉也而謂之唐是也與班志合瓚說及括地志未知何據】}
請曰以爲繭絲乎
抑爲保障乎
簡子曰保障哉|{
	【繭絲謂浚民之膏澤如抽繭之緒不盡則不止保障謂厚民之生如築堡以自障愈培則愈厚宋祁曰障之亮翻又音章】}
尹鐸損其戶數|{
	【韋昭曰損其戶則民優而税少】}
簡子謂無恤曰晉國有難而無以尹鐸爲少|{
	【而汝也難乃旦翻患也阨也少音多少之少重之爲多輕之爲少】}
無以晉陽爲遠必以爲歸及智宣子卒|{
	【卒子恤翻】}
\textcolor{blue}{智襄子}爲政|{
	【諡法有勞定國曰襄爲政爲晉國之政】}
與韓康子魏桓子宴於藍臺|{
	【韓康子韓宣子之曾孫莊子之子虔也魏桓子魏獻子之子曼多之孫駒也諡法温柔恕恭曰康辟上服遠曰桓爾雅四方而高曰臺】}
智伯戲康子而侮段規|{
	【姓譜段鄭共叔段之後】}
智國聞之諫曰主不備難難必至矣|{
	【春秋以來大夫之家臣謂大夫曰主難乃旦翻下同】}
智伯曰難將由我我不爲難誰敢興之對曰不然夏書有之一人三失怨豈在明不見是圖|{
	【書五子之歌之辭夏戶雅翻見賢遍翻發見也著也形也】}
夫君子能勤小物故無大患
今主一宴而恥人之君相|{
	【夫音扶段規韓康子之相也相息醤翻下同】}
又弗備,曰‘不敢興難’,無乃不可乎!
蜹、蟻、蜂、蠆,皆能害人,|{
	【宋祁曰蜹如蜕翻人字林乂劣翻秦人謂蚊爲蜹今按蜹小蟲日中羣集人之肌膚而嘬其血蚊之類也蜂細腰而能螫人蠆亦毒蟲長尾音丑邁翻】
	}
况君相乎
弗聽


智伯請地於韓康子康子欲弗與
段規曰智伯好利而愎不與將伐我不如與之
彼狃於得地|{
	【好呼到翻愎弼力翻狠也狃女九翻驕忲也又相狎也】
	}
必請於他人他人不與必嚮之以兵然後我得免於患而待事之變矣
康子曰善
使使者致萬家之邑於智伯|{
	【毛晃曰邑都邑四井爲邑四邑爲丘邑方二里丘方四里載師以公邑之田任甸地以家邑之田任稍地註公邑謂六遂餘地家邑大夫之采地此又與四井之邑不同又都國都邑縣也左傳凡邑有先君宗廟之主曰都無曰邑邑曰築都曰城此謂大縣邑也杜預引周禮四縣爲都四井爲邑恐誤四井之邑方二里豈能容宗廟城郭如論語十室之邑西都賦都都相望邑邑相屬則是四縣四井之都邑也若千室之邑萬家之邑則非井邑矣項安世曰小司徒井牧田野以四井爲邑凡三十六家除公田四夫凡三十二家遂大夫會爲邑者之政以里爲邑凡二十五家遂大夫蓋論里井之制二十五家共一里門卽六鄉之二十五家爲一閭也小司徒蓋論溝洫之制四井爲邑共用一溝卽匠人所謂井間廣四尺深四尺謂之溝也居則度人之衆寡溝則度水之衆寡此其所以異歟毛項二說皆明周制參而考之戰國之所謂邑非周制矣致送至也】}
智伯悅又求地於魏桓子桓子欲弗與任章曰何故弗與|{
	【任章魏桓子之相也姓譜黃帝二十五子十二人各以德爲姓第一曰任氏又任爲風姓之國實太昊之後主濟祀今濟州任城卽其地任市林翻】
	}
桓子曰無故索地故弗與任章曰無故索地諸大夫必懼|{
	【索山客翻求也】
	}
吾與之地智伯必驕
彼驕而輕敵此懼而相親以相親之兵待輕敵之人智氏之命必不長矣
周書曰將欲敗之必姑輔之將欲取之必姑與之|{
	【逸書也敗補邁翻】
	}
主不如與之以驕智伯然後可以擇交而圖智氏矣奈何獨以吾爲智氏質乎|{
	【質脂利翻物相綴當也又質讀如字亦通質謂椹質也質的也椹質受斧質的受矢言智伯怒魏桓子必加兵於魏如椹質之受斧質的之受矢也】}
桓子曰善
復與之萬家之邑一。|{
	【復扶又翻】
	}
智伯又求蔡臯狼之地於趙襄子|{
	【康曰臯姑勞切狼盧當切春秋蔡地後爲趙邑余據春秋之時晉楚爭盟晉不能越鄭而服蔡三家分晉韓得成臯因以幷鄭時蔡已爲楚所滅鄭之南境亦人于楚就使臯狼爲蔡地趙襄子安得而有之漢書地理志西河郡有臯狼縣又有藺縣漢之西河春秋以來皆爲晉境而古文藺字與蔡字近或者蔡字其藺字之訛也】}
襄子弗與
智伯怒帥韓魏之甲以攻趙氏|{
	【帥讀曰率】}
襄子將出曰吾何走乎|{
	【走則豆翻疾趨之也趨七喻翻】}
從者曰長子近且城厚完|{
	【從才用翻長子縣周史辛伯所封邑班志屬上黨郡陸德明曰長子之長丁丈翻顔師古曰長讀爲短長之長今讀爲長幼之長非也崔豹古今注曰城盛也所以盛受民物也淮南子曰鯀作城盛時征翻】}
襄子曰民罷力以完之|{
	【罷讀曰疲】}
又斃死以守之其誰與我|{
	【韋昭曰謂誰與我同力也】}
從者曰邯鄲之倉庫實|{
	【邯鄲卽春秋邯鄲午之邑也班志邯鄲縣屬趙國張晏曰邯鄲山在東城下鄲盡也城郭從邑故旁加邑宋白曰邯鄲本衛地後屬晉七國時爲趙都趙敬侯自晉陽始都邯鄲余按史記六國年表周安王之十六年趙敬侯之元年烈王之二年趙成侯之元年成侯一十二年魏克邯鄲是年顯王之十六年也二十四年魏歸邯鄲若敬侯已都邯鄲魏克其國都而趙不亡何也至顯王二十二年公子范襲邯鄲不勝而死是年肅侯之三年也意此時趙方都邯鄲蓋肅侯徙都非敬侯也邯音 寒鄲音丹康多寒翻】}
襄子曰浚民之膏澤以實之|{
	【韋昭曰浚煎也讀曰醮宋祁曰浚蘇俊翻醮子召翻余謂浚讀當如宋音浚省疏瀹也淘也深也】}
又因而殺之其誰與我
其晉陽乎,先主之所屬也,|{
	【古者諸侯之大夫其家之臣子皆稱之曰主死則曰先主考左傳可見已屬陟玉翻】}
尹鐸之所寛也,民必和矣。”
乃走晉陽。
三家以國人闈而灌之城不浸者三版|{
	【高二尺爲一版三版六尺】}
沈竈產鼃民無叛意|{
	【沈持林翻顔師古漢書音義曰鼃黽也似蝦蟇而長脚其色青史游急就章曰蛙蝦蟇陸佃埤雅曰鼃似蝦蟇而長踦瞋目如怒鼃與蛙同音下媧翻】}
智伯行水|{
	【據經典釋文凡廵行之行音下孟翻後倣此】
	}
魏桓子御,韓康子驂乘。|{
	【兵車尊者居左執弓矢御者居中有力者居右持矛以備傾側所謂車右是也韓魏畏智氏之強一爲之御一爲之右驂與參同參者三也三人同車則曰驂乘四人同車則曰駟乘左傳齊伐晉燭庸之越駟乘杜預注曰四人共乘者殿車乘石證翻】
	}
智伯曰:“吾乃今知水可以亡人國也。”
桓子肘康子:康子履桓子之跗,以汾水可以灌安邑,絳水可以灌平陽也。|{
	跗,音夫,足趾也。
	班志:汾水出汾陽北山。
	汾陽縣屬太原郡,安邑縣屬河東郡。
	史記正義曰:安邑故城在絳州夏縣東北十五里。
	應劭曰:絳水出河東絳縣西南。
	平陽縣亦屬河東郡。
	安邑,魏絳始居邑。
	平陽,韓武子玄孫貞子始居之。
	桓、康二子之肘足接,蓋各爲都邑慮也。
	水經注曰:絳水出絳縣西南,蓋以故絳爲言,其水出絳山東,西北流而合于澮,猶在絳縣界中。
	智伯所謂“汾水可以灌安邑”,或亦有之:“絳水可以灌平陽”,未識所由。
	余謂自春秋之季至于元魏歷年滋多郡縣之離合川谷之遷改有不可以一時所睹爲據者史記正義曰韓初都平陽今晉州也括地志曰絳水一名白今名沸泉源出絳山飛泉奮湧揚波注縣積壑三十餘丈望之極爲奇觀可接引北灌平陽城酈道元父範歷仕三齊少長齊地熟其山川後入關死於道未嘗至河東也此蓋因耳學而致疑括地志成於唐之魏王泰泰者太宗之愛子羅致天下一時名儒以作此書其考據宜詳當取以爲據】}
絺疵謂智伯曰:“韓、魏必反矣。”
智伯曰:“子何以知之?”
絺疵曰:“以人事知之。夫從韓、魏之兵以攻趙,趙亡,難必及韓、魏矣。|{
	夫,音扶。
	難,乃旦翻。
	}
今約勝趙而三分其地,城不沒者三版,人馬相食,城降有日,而二子無喜志,有憂色,是非反而何?”
明日,智伯以絺疵之言告二子,二子曰:“此夫讒人欲爲趙氏游說,使主疑於二家而懈於攻趙氏也。
不然夫二家豈不利朝夕分趙氏之田而欲爲危難不可成之事乎
二子出絺疵入曰主何以臣之言告二子也
智伯曰子何以知之
對曰臣見其視臣端而趨疾知臣得其情故也
智伯不悛
絺疵請使於齊|{
	夫,音扶;餘並同。
	難,乃旦翻。
	降,戶江翻,下也,服也。
	說,輸芮翻。
	懈,居隘翻,怠也。
	危難,如字。
	悛,丑緣翻,改也,止也。
	絺,抽遲翻,姓也。
	康曰:“絺”當作“郗”,姓譜諸書未有從絲者,疑借字。
	余按姓譜:絺姓,周蘇忿生支子,封於絺,因氏焉。
	爲趙之爲,音于偽翻。
	使,疏吏翻。
	疵請出使以避禍也。
	}
趙襄子使張孟談潛出見二子,曰:“臣聞脣亡則齒寒。
今智伯帥韓、魏以攻趙,趙亡則韓、魏爲之次矣。”|{
	帥,讀曰率。
	}
二子曰:“我心知其然也,恐事未遂而謀泄,則禍立至矣。”
張孟談曰:“謀出二主之口,入臣之耳,何傷也!”
二子乃潛與張孟談約,爲之期日而遣之。|{
	姓譜:張氏本自軒轅第五子揮,始造弦,寔張網羅,世掌其職,後因氏焉。
	風俗傳云:張、王、李、趙、黃帝所賜姓也。
	又晉有解張,字張侯,自此晉國有張氏。
	唐姓氏譜:張氏出自姬姓,黃帝子少昊青陽氏第五子揮正始制弓矢,子孫賜姓張。
	周宣王卿士張仲,其後裔事晉爲大夫
	}
襄子夜使人殺守隄之吏,而決水灌智伯軍。
智伯軍救水而亂,韓、魏翼而擊之,襄子將卒犯其前,|{
	將,卽亮翻,又音如字。
	將,領也。
	卒,滅沒翻
	說文:吏人給事者衣爲卒,卒衣有題識;其字從“衣”從“十”。
	}
大敗智伯之衆|{
	【以此敗彼曰敗敗比邁翻】
	}
遂殺智伯盡滅智氏之族|{
	【史記六國年表三晉滅智氏在周定王十六年上距獲麟二十七年皇甫謐曰元王十一年癸未三晉滅智伯】
	}
唯輔果在|{
	【以別族也】
	}

\textcolor{green}{臣光曰}:智伯之亡也,才勝德也。
夫才與德異,而世俗莫之能辨,|{
	夫,音扶。
	}
通謂之賢,此其所以失人也。%因为才能而忽略其它方面
夫聰察強毅之謂才,正直中和之謂德。
才者,德之資也;
德者,才之帥也。|{
	夫,音扶。
	帥,所類翻。
	}
雲夢之竹,天下之勁也;|{
	書禹貢:雲土夢作乂。
	人孔安國注云:雲夢之澤在江南。
	左傳:楚王以鄭伯旧江南之夢。
	杜預注云:楚之雲夢跨江南北。
	班志:雲夢澤在南郡華容縣南。
	祝穆曰:據左傳䢵夫人弃子文於夢中,言夢而不言雲,楚子避吳入于雲中,言雲而不言夢,則知雲、夢二澤也。
	漢陽志:雲在江之北,夢在江之南。
	又安陸有雲夢,澤枝江有雲夢城。
	蓋古之雲夢澤甚廣,而後世悉爲邑居聚落,故地之以雲夢得名者非一處。
	竹箭之產,荆楚爲良;
	雲夢,楚之地也。
	夢,如字,又莫公翻。
	}
然而不矯揉,不羽括,則不能以入堅。|{
	矯,舉夭翻。
	揉,如久翻。
	康曰:揉曲爲矯,揉所以撓曲而使之直也。
	羽者,箭翎。
	括者,箭窟受弦處。
	括,音聒,通作筈。
	}
棠谿之金,天下之利也;|{
	左傳:楚封吳夫槪王於棠谿。
	戰國之時,其地屬韓,出金甚精利。
	劉昭郡國志:汝南郡吳房縣有棠谿亭。
	杜佑通典曰:棠谿在今汝州郾城縣界。
	九域志:蔡州有冶爐城,韓國鑄劒之地。
	}
然而不鎔範,不砥礪,則不能以擊強。|{
	毛晃曰:鎔,銷也,鑄也;
	說文:鑄器法也。
	董仲舒傳:猶金在鎔。
	註:鎔,謂鑄器之模範。
	範,法也,式也。
	禮運:範金合土。
	砥,軫氏翻,柔石也。
	礪,力制翻,䃺也。
	}
是故才德全盡謂之聖人,
才德兼亡謂之愚人;
德勝才謂之君子,
才勝德謂之小人。
凡取人之術,苟不得聖人、君子而與之,
與其得小人,不若得愚人。%螺丝钉岗位
何則?
君子挾才以爲善,小人挾才以爲惡。
挾才以爲善者,善無不至矣;
挾才以爲惡者,惡亦無不至矣。|{
	挾,檄頰翻。
	}
愚者雖欲爲不善,智不能周,力不能勝,譬如乳狗搏人,人得而制之。|{
	挾,戶頰翻。
	朱元晦曰:挾者,兼有而恃之之稱。
	勝,音升。
	乳,孺遇翻,乳育也。
	乳狗,育子之狗也。
	搏,伯各翻。}
小人智足以遂其姦,勇足以決其暴,是虎而翼者也,其爲害豈不多哉。|{
	虎而傅翼,其爲害也愈甚。
	}
夫德者人之所嚴,|{
	嚴,敬也。
	}
而才者人之所愛;
愛者易親,嚴者易疏,|{
	易,以䜴翻。
	}
是以\textcolor{red}{察者多蔽於才而遺於德}。
自古昔以來,國之亂臣,家之敗子,
才有餘而德不足,以至於顚覆者多矣,豈特智伯哉!
故爲國爲家者苟能審於才德之分而知所先後,|{
	先,悉薦翻。
	後,戶遘翻。
	}
又何失人之足患哉!


3三家分智氏之田趙襄子漆智伯之頭以爲飲器|{
	【說文桼木汁可以鬖物下從水象桼如水滴而下也漢書張騫傳匈奴破月氏王以其頭爲飲器韋昭註曰飲器椑榼也晉灼曰飲器虎子屬也或曰飲酒之器也師古曰匈奴嘗以月氏王頭與漢使歃血盟然則飲酒之器是也韋云椑榼晉云虎子皆非也椑榼卽今之偏榼所以盛酒耳非用飲者也虎子褻器所以溲便者椑音鼙榼克合翻氏音支使疏吏翻歃色甲翻盛時征翻褻息列翻溲疎鳩翻便毘連翻】}
智伯之臣豫讓欲爲之報仇|{
	【豫姓也讓名也戰國之時又有豫且不知其同時否也爲音于偽翻下同】}
乃詐爲刑人挾匕首入襄子宫中塗厠|{
	【挾持也劉向曰匕首短劒鹽鐵論曰匕首長尺八寸頭類匕故云匕首匕音比厠初吏翻圊也長直亮翻】}
襄子如厠心動索之獲豫讓|{
	【索山客翻】}
左右欲殺之襄子曰智伯死無後而此人欲爲報仇眞義士也吾謹避之耳乃舍之|{
	【舍讀曰捨】}
豫讓又漆身爲癩吞炭爲啞|{
	【癩落蓋翻惡疾也啞倚下翻瘖也】}
行乞於市|{
	【神農日中爲市致天下之民聚天下之貨交易而退此立市之始也鄭氏周禮注曰市雜聚之處】}
其妻不識也
行見其友其友識之爲之泣曰以子之才臣事趙孟必得近幸|{
	【自春秋之時 宣子謂之宣孟趙文子謂之趙孟其後 襲而呼爲趙孟孟長也】}
子乃爲所欲爲顧不易邪|{
	【易以䜴翻】}
何乃自苦如此
求以報仇不亦難乎豫讓曰旣已委質爲臣|{
	【經典釋文曰質職日翻委質委其體以事君也後漢書註委質屈膝】}
而又求殺之是二心也
凡吾所爲者極難耳
然所以爲此者將以愧天下後世之爲人臣懷二心者也
襄子出豫讓伏於橋下
襄子至橋馬驚索之得豫讓遂殺之|{
	【自智宣子立瑤至豫讓報仇其事皆在威烈王二十三年之前故先以初字發之溫公之意蓋以天下莫大於名分觀命三大夫爲諸侯之事則知周之所以益微七雄之所以益盛莫重於宗社觀智趙立後之事則知智宣子之所以失趙簡子之所以得君臣之義當守節伏死而已觀豫讓之事則知策名委質者必有霣而無貳其爲後世之鑑豈不昭昭也哉】
	}
襄子爲伯魯之不立也有子五人不肯置後
封伯魯之子於代|{
	【代國在夏屋句注北趙襄子滅之班志有代郡代縣爲于偽翻夏戶雅翻】}
曰代成君早卒|{
	【成謚也諡法安民立政曰成】}
立其子浣爲趙氏後|{
	【浣戶管翻】}
襄子卒弟桓子逐浣而自立|{
	【史記六國表威烈王元年襄子卒二年趙桓子元年卒明年國人立獻侯浣浣索隱作晚卒子恤翻下同】}
一年卒趙氏之人曰桓子立非襄主意
乃共殺其子復迎浣而立之是爲獻子|{
	【復扶又翻又音如字獻子卽獻侯六國喪威烈王三年獻侯之元年蓋分晉之後三晉僭侯久矣諡法知質有聖曰獻】}
獻子生籍是爲烈侯|{
	【諡法有功安民曰烈秉德尊業曰烈】}
魏斯者魏桓子之孫也是爲文侯|{
	【諡法學勤好問曰文慈惠安民曰文】}
韓康子生武子武子生虔是爲景侯|{
	【諡法克定禍亂曰武布義行剛曰景六國表威烈王二年魏文侯斯元年十八年韓景侯虔元年蓋其在國僭爵已久不敢以通王室威烈王遂因而命之識者重爲周惜通鑑于此序三家之世也】}
魏文侯以卜子夏、田子方爲師。|{
	卜,以官爲氏。
	田本出於陳,陳敬仲以陳爲田氏。
	徐廣曰:始食采地,由是改姓田氏。
	索隱曰:陳、田二聲相近,遂爲田氏。
	夏,戶雅翻。
	}
每過段干木之廬必式。|{
	過,工禾翻。
	唐人志氏族曰:李耳,字伯陽,一字耼;
	其後有李宗,魏封於段,爲干木大夫,是以段爲氏也。
	余按:通鑑赧王四十二年,魏有段干子,則段干,複姓也。
	書:武王式商容閭。
	註云:式其閭巷,以禮賢。
	記曲禮:國君撫式,士下之。
	註云:升車必正立,據式小俛,崇敬也。
	師古曰:式,車前橫木。
	古者立乘;凡言式車者,謂俛首撫式,以禮敬人。
	孔穎達曰:式,謂俯下頭也。
	古者車箱長四尺四寸而三分,前一後二,橫一木,下去車牀三尺三寸,謂之爲式;
	又於式上二尺二寸橫一木,謂之較,較去車牀凡五尺五寸。
	於時立乘,若平常則憑較,故詩云“倚重較兮”是也。
	若應爲敬,則落隱下式,而頭得俯俛,故記云“式視馬尾”是也。
	較,訖岳翻。
	}
四方賢士多歸之。
文侯與羣臣飲酒,樂,而天雨。命駕將適野。
左右曰:“今日飲酒樂,天又雨,君將安之?”
文侯曰:“吾與虞人期獵,雖樂,豈可無一會期哉?”乃往,身自罷之。|{
	周禮有山虞、澤虞,以掌山澤。
	註云:虞,度也,度知山林之大小及其所生。
	身自罷之者,身往告之,以雨而罷獵也。
	樂,音洛。
	}
韓借師於魏以伐趙,文侯曰:“寡人與趙,兄弟也,不敢聞命。”
趙借師於魏以伐韓,文侯應之亦然。
二國皆怒而去。
已而知文侯以講於己也,|{
	講,和也。
	}
皆朝于魏。|{
	朝,直遙翻。
	}
魏於是始大於三晉,諸侯莫能與之爭。
使樂羊伐中山,克之;|{
	樂,姓也。
	本自有殷微子之後。
	宋戴公四世孫樂呂爲大司寇。
	中山,春秋之鮮虞也,漢爲中山郡。
	宋白曰:唐定州,春秋白狄鮮虞之地。
	隋圖經曰:中山城在今唐昌縣東北三十一里,中山故城是也。
	杜佑曰:城中有山,故曰中山。
	}
以封其子擊。
文侯問於羣臣曰:“我何如主?”
皆曰:“仁君。”
任座曰:“君得中山,不以封君之弟而以封君之子,何謂仁君!”
文侯怒,任座趨出。|{
	任座亦習見當時鄰國之事而爲是言耳。
	任音壬,“座”一作“痤”,音才戈翻。
	}
次問翟璜,|{
	翟,姓也,音直格翻,又音狄。
	姓譜:翟爲晉所滅,子孫以國爲氏。
	今人多讀從上音。
	璜,戶光翻。
	}
對曰:“仁君。”
文侯曰:“何以知之?”
對曰:“臣聞君仁則臣直。嚮者任座之言直臣是以知之。”
文侯悅,使翟璜召任座而反之,親下堂迎之,以爲上客。
文侯與田子方飲,文侯曰:“鐘聲不比乎?|{
	比,音毗。
	不比,言不和也。
	}
左高。”|{
	此蓋編鐘之懸,左高,故其聲不和。
	}
田子方笑。
文侯曰:“何笑?”
子方曰:“臣聞之,君明樂官,不明樂音,今君審於音,臣恐其聾於官也。”|{
	明樂官,知其才不才;明樂音,知其和不和。
	五聲合和,然後成音。
	詩大序曰:聲成文,謂之音。
	}
文侯曰:“善。”
子擊出,遭田子方於道,下車伏謁。|{
	古文 史 考曰:黃帝作車,引重致遠;少昊氏加牛;禹時奚仲加馬。
	釋名曰:車,居也。
	韋昭曰:古唯尺遮翻,自漢以來,始有“居”音。
	蕭子顯曰:三皇氏乘祗車出谷口,車之始也。
	祗,翹移翻。
	}
子方不爲禮。
子擊怒,謂子方曰:“富貴者驕人乎?貧賤者驕人乎?”
子方曰:“亦貧賤者驕人耳,富貴者安敢驕人!
	國君而驕人則失其國,大夫而驕人則失其家。
	失其國者未聞有以國待之者也,失其家者未聞有以家待之者也。
	夫士貧賤者,言不用,行不合則納履而去耳,安往而不得貧賤哉!”
子擊乃謝之。|{
	夫,音扶。
	行,下孟翻。
	}
文侯謂李克曰先生嘗有言曰家貧思良妻國亂思良相
今所置非成則璜二子何如|{
	【李氏出自顓頊曾孫臯陶爲堯大理以官命族爲理氏商紂時裔孫利貞逃難食木子得全改爲李氏置言置相也相息亮翻難乃旦翻】
	}
對曰卑不謀尊疏不謀戚
臣在闕門之外不敢當命|{
	【在闕門之外謂疏遠也】
	}
文侯曰先生臨事勿讓
克曰君弗察故也
居視其所親富視其所與達視其所舉窮視其所不爲貧視其所不取五者足以定之矣何待克哉
文侯曰先生就舍吾之相定矣|{
	【相息亮翻】}
李克出見翟璜
翟璜曰今者聞君召先生而卜相果誰爲之
克曰魏成
翟璜忿然作色曰西河守吳起臣所進也|{
	【班志魏地其界自高陵以東盡河東河內高陵縣漢屬馮翊其地在河西所謂西河之列者也魏初使吳起守之秦兵不敢東向至惠王時秦使衛鞅擊虜其將公子卭遂獻西河之外於秦吳以國爲姓相息亮翻守式又翻】
	}
君內以鄴爲憂臣進西門豹|{
	【班志鄴縣屬魏郡西門豹爲鄴令鑿渠以利民王符潛夫論姓氏篇曰如有東門西郭南宫北郭皆因居以爲姓西門蓋亦此類鄴魚怯翻】
	}
君欲伐中山臣進樂羊
中山已拔無使守之臣進先生
君之子無傅臣進屈侯鮒|{
	【傅者傅之以德義因以爲官名傅芳遇翻屈九勿翻姓也余按屈晉地時屬魏鮒蓋魏封屈侯也鮒音符遇翻】
	}
以耳目之所睹記臣何負於魏成|{
	【不勝爲負】}
李克曰子言克於子之君者豈將比周以求大官哉|{
	【比毗至翻阿黨爲比】
	}
君問相於克克之對如是|{
	【李克自叙其答魏文侯之言也】}
所以知君之必相魏成者魏成食祿千鍾|{
	【孔達曰祿者穀也故鄭註司錄云祿也言穀年穀豐然後制祿援神契云祿者錄也白虎通曰上以收錄接下下以名錄謹以事上是也六斛四斗爲一鍾】
	}
什九在外什一在內是以東得卜子夏田子方段干木|{
	【夏戶雅翻】}
此三人者君皆師之子所進五人者君皆臣之
子惡得與魏成比也|{
	【惡讀曰烏何也】
	}
翟璜逡巡再拜曰璜鄙人也失對願卒爲弟子|{
	【逡七倫翻逡巡卻退貌卒子恤翻終也孔達曰先生師也謂師爲先生者言彼先己而生其德多厚也自稱爲弟子者言己自處如弟子則尊其師如父兄也】
	}
\textcolor{blue}{吳起}者,衛人,仕於魯。
齊人伐魯魯人欲以爲將起取齊女爲妻|{
	【將卽亮翻下同取讀曰娶孔達曰妻之爲言齊也以禮見問得與夫敵體也】
	}
魯人疑之起殺妻以求將大破齊師
或譛之魯侯曰起始事曾參|{
	【世本曰曾姓出自鄫國陸德明曰參所金翻一音七南翻】
	}
母死不奔喪曾參絕之今又殺妻以求爲君將
起殘忍薄行人也|{
	【行下孟翻】
	}
且以魯國區區而有勝敵之名則諸侯圖魯矣
起恐得罪聞魏文侯賢乃往歸之
文侯問諸李克李克曰起貪而好色|{
	【好呼到翻】}
然用兵司馬穰苴弗能過也|{
	【司馬官名穰苴本齊田姓仕齊爲是官故以稱之齊景公之賢將也穰如羊翻苴子余翻】
	}
於是文侯以爲將擊秦拔五城
起之爲將與士卒最下者同衣食卧不設席行不騎乘|{
	【騎馬爲騎乘車爲乘言起與士卒同其勞苦行不用車馬也】}
親裹贏糧|{
	【師古曰贏擔也此言起親裹士卒所齎擔之糧贏恰成翻】}
與士卒分勞苦
卒有病疽者起爲吮之|{
	【疽七余翻癰也吮徐兖翻說文嗽也康所角翻】}
卒母聞而哭之
人曰子,卒也而將軍自吮其疽何哭爲
母曰非然也
往年吳公吮其父疽其父戰不旋踵遂死于敵
吳公今又吮其子妾不知其死所矣是以哭之

燕湣公薨,子僖公立。|{
	燕自召公奭受封於北燕,
	其地則唐幽州薊縣故城是也。
	自召公至湣公三十二世。
	燕,因肩翻。
	湣,讀與閔同。
	諡法:使民悲傷曰閔;
	小心畏忌曰僖。
	}


二十四年

王崩,子安王驕立。
盜殺楚聲王,國人立其子悼王。|{
	【周成王封熊繹於楚姓芈氏居丹陽今枝江縣故丹陽城是也括地志曰歸州秭歸縣丹陽城熊繹之始國其後強大北封畛於汝南幷吳越地方五千里自熊繹至聲王三十世索隱曰聲王名當悼王名疑諡法不生其國曰聲注云生於外家年中早夭曰悼注云年不稱志又云恐懼從處曰悼註云從處言險圯也】}
\par
\section{安王}|{
	【諡法好和不爭曰安】
	}
\par 元年
秦伐魏至陽孤|{
	【周孝王邑非子於秦徐廣曰今隴西縣秦亭是也括地志曰秦州清水縣本名秦十三州志曰秦亭秦谷是也至襄公取周地穆公覇西戎日以強大是年秦簡公之十四年也自非子至簡公二十八世陽孤史記作陽狐正義引括地志曰陽狐郭在魏州元城縣東北三十里余按此時西河之外皆爲魏境若秦兵至元城則是越魏都安邑而東矣水經注河東垣縣有陽壺城九域志絳州有陽壺城姑識之以廣異聞且俟知者】}
\par 二年
魏、韓、趙伐楚,至桑丘。|{
	【水經注澺水自葛陂東南逕新蔡縣故城東而東南流注于汝水又東南逕下桑里左迆爲橫塘陂史記作乘丘正義地理志乘丘故城在兖州瑕丘縣西北三十五里當從之】}
鄭圍韓陽翟|{
	【周宣王封其弟友于鄭杜預世族譜曰封於咸林今京兆鄭邑是也幽王無道友徙其人於虢鄶之間遂有其地今河南新鄭是也友謚桓公是年鄭繻公駘之二十三年自桓公至繻公二十二世班志陽翟縣屬潁川郡索隱曰翟音狄溫公類篇音萇伯翻繻詢趨翻駘堂來翻】}
韓景侯薨子烈侯取立 趙烈侯薨國人立其弟武侯 秦簡公薨子惠公立|{
	【諡法愛民好與曰惠】}
\par 三年
王子定奔晉
虢山崩壅河|{
	【徐廣曰虢山在陜裴駰曰弘農陜縣故虢國北虢在大陽東虢在滎陽括地志曰虢山在陜州陜縣西臨黃河今臨河有岡阜似是頹山之餘水經注曰陜城西北帶河水湧起方數十丈父老云石虎載銅翁仲至此沉沒水所以湧洪河巨瀆宜不爲金狄梗流蓋魏文侯時虢山崩壅河所致耳陜失冉翻】}
\par 四年
楚圍鄭
鄭人殺其相駟子陽|{
	【鄭穆公之子騑字子駟古者以王父之字爲氏子陽其後也相息亮翻騑芳菲翻】}
\par 五年
曰有食之|{
	【杜預曰日行遲一歲一周天月行速一月一周天一歲凡十二交會然日月動物雖行度有大量不能不小有贏縮故有雖交會而不食者或有頻交而食者孔達曰日月交會謂朔也周天三百六十五度四分度之一日月皆右行於天一晝一夜日行一度月行十三度十九分度之七二十九日日有餘而月行天一周追及於日而與之會交會而日月同道則食月或在日道表或在日道裏則不食矣又歷家爲交食之法大率以一百七十有三日有奇爲限然月先在裏則依限而食者多若月在表則依限而食者少杜預見其參差乃云雖行度有大量不能不小有贏縮故有雖交會而不食者或有頻交而食者此得之矣蘇氏曰交當朔則日食然亦冇交而不食者交而食陽微而隂乘之也交而不食陽盛而陰不能揜也朱元晦曰此則繫乎人事之感蓋臣子背君父妾婦乘其夫小人陵君子夷狄侵中國所感如是則隂盛陽微而日爲之食矣是以聖人於春秋每食必書而詩人亦以爲醜也今此書年而不書月與晦朔史失之也釋名曰日月虧曰食稍小侵虧如蟲食草木之葉也亦作蝕】
	}
\par
三月盜殺韓相俠累
俠累與濮陽嚴仲子有惡
仲子聞軹人\textcolor{blue}{聶政}之勇以黃金百溢爲政母壽欲因以報仇|{
	【相息亮翻俠戶頰翻累力追翻濮陽春秋之帝丘漢爲濮陽縣屬東郡應劭曰濮水南入鉅野水北爲陽濮博木翻惡如字不善也康烏故翻非軹春秋原邑晉文公所闈者漢爲軹縣屬河內郡音只姓譜曰楚大夫食采於聶因以爲氏聶尼輒翻溢夷質翻二十四兩爲溢】
	}
政不受,曰:“老母在,政身未敢以許人也!”
及母卒,仲子乃使政刺俠累。|{
	【卒子恤翻刺七亦翻又如字】}
俠累方坐府上兵衛甚衆聶政直入上階|{
	【上時掌翻】}
刺殺俠累因自皮面抉眼自屠出腸
韓人暴其尸於市|{
	【暴步木翻又如字】}
購問莫能識
其姊嫈聞而往哭之曰是軹深井里聶政也|{
	【史記正義曰深井里在懷州濟源縣南三十里】}
以妾尚在之故重自刑以絕從
妾奈何畏歿身之誅終滅賢弟之名
遂死於政尸之㫄|{
	【皮面以刀剺而而去其皮懸賞以募告者曰購購古候翻嫈烏莖翻絕從之從讀曰蹤謂自絕其蹤跡又或曰從讀如字謂絕其從坐之罪也】
	}
\par 六年
鄭駟子陽之黨弑繻公|{
	【繻者諡法所不載史記注繻或作繚繻詢趨翻】}
而立其弟乙|{
	【白虎通曰弟悌也心順行篤也行下孟翻】}
是爲康公

宋悼公薨子休公田立|{
	【武王封微子啓於宋唐宋州之睢陽縣是也自微子二十七世至悼公名購由休亦諡法所不載】}
\par 八年
齊伐魯,取最。|{
	【武王封太公於齊唐青州之臨淄是也括地志曰天齊水在臨淄東南十五里封禪書曰齊之所以爲齊者以天齊是年康公貸之十一年自太公至康公二十九世成王封伯禽於魯唐兖州之曲阜是也是年穆公之十六年自伯禽至穆公凡二十八世】
	}
[韓救魯]

鄭負黍叛,復歸韓。|{
	【據史記繻公之十六年敗韓於負黍蓋以此時取之而今復叛歸韓也劉昭郡國志潁川郡陽城縣有負黍聚古今地名云負黍山在陽城縣西南二十七里或云在西南三十五里】}
\par 九年
魏伐鄭
晉烈公薨,子孝公傾立。|{
	【周成王封弟叔虞於唐括地志曰故唐城在幷州晉陽縣北二里堯所築也都城記曰唐叔虞之子燮父徙居晉水㫄今幷州理故唐城卽燮父初徙之處其城南半入州城中毛詩譜曰燮父以堯墟南有晉水改曰晉侯自唐叔至烈公三十七世烈公名止諡法慈惠愛親曰孝】}
\par 十一年
秦伐韓宜陽,取六邑。|{
	【班志宜陽縣屬弘農郡史記正義曰宜陽縣故城在河南府福昌縣東十四里故韓城是也此邑卽周禮四井爲邑之邑】}
初,田常生襄子盤,盤生莊子白,白生太公和。|{
	【此序齊田氏之世也田常卽左傳陳成子恆也溫公避仁廟諱改恆曰常自陳公子完奔齊 五世至常得政諡法勝敵志強曰莊】}
是歲,齊田和遷齊康公於海上,使食一城,以奉其先祀。
\par 十二年

秦、晉戰于武城。|{
	【此非魯之武城左傳晉陰飴甥會秦伯盟于王城杜預曰馮翊臨晉縣東有王城今名武鄉括地志故武城一名武平城在華州鄭縣東北十三里】}
齊伐魏,取襄陽。|{
	【陽當作陵徐廣曰今之南平陽也余據晉志南平陽縣屬山陽郡班志陳留郡有襄邑縣師古曰圈稱云襄邑宋地木承匡襄陵鄉也宋襄公所葬故曰襄陵秦始皇以承匡卑濕徙縣襄陵因曰襄邑】}
魯敗齊師于平陸|{
	【班志東平國有東平陸縣戰國時之平陸也史記正義曰平陸兖州縣卽古厥國宋白曰鄆州中都縣漢爲平陸縣史記魯敗齊師于平陸是也敗補邁翻】}
\par
十三年秦侵晉 齊田和會魏文侯楚人衛人于濁澤|{
	【康曰濁水名漢志濁水出齊郡廣縣媯山余謂康說誤矣徐廣史記註曰長社有濁澤水經注曰皇陂水出胡城西北胡城潁陰之狐人亭也皇陂古長社之濁澤也記諸侯相見於郤地曰會孔達曰諸侯未及期而相見曰遇會者謂及期之禮旣及期人至所期之地】}
求爲諸侯魏文侯爲之請於王及諸侯王許之|{
	【爲之之爲于偽翻】}
\par 十五年

秦伐蜀,取南鄭。|{
	【譜記普云蜀之先肇自人皇之際黃帝子昌意娶蜀山氏女生帝俈旣立封其支庶于蜀歷虞夏商周周衰先稱王者蠶樷余據武王伐紂庸蜀諸國皆會于牧野孔安國曰蜀叟也春秋之時不與中國通班志南鄭縣屬漢中郡唐爲梁州治所俈通作嚳音括沃翻】
	}
魏文侯薨,太子擊立,|{
	【王者以嫡長子爲太子謂之國儲副君諸侯曰世子周衰率上僭孔達曰太者大中之太也上時掌翻長知兩翻】}
是爲武侯。

武侯浮西河而下|{
	【西河卽禹貢之龍門西河】
	}
中流顧謂吳起曰:“美哉山河之固,此魏國之寶也!”
對曰:“在德不在險。
昔三苗氏,左洞庭,右彭蠡;德義不脩,禹滅之。|{
	【武陵長沙零桂之水匪爲洞庭周七百里彭蠡澤在漢豫章郡彭澤縣西書有苗弗率汝徂征三苗所居蓋今江南西道之地蠡里弟翻】}
夏桀之居,左河濟,右泰華,伊闕在其南,羊腸在其北;脩政不仁,湯放之。|{
	【濟水出河東垣縣王屋山南流貫河而南合于滎瀆禹貢所謂導沇水東流爲濟溢爲滎者也自漢築滎陽石門而濟與河合流而注于海不入滎瀆禹貢所謂導沇水東流爲濟入于河桀都安邑蓋恃以爲險泰華山在京兆華陰縣南水經伊水出南陽縣西荀渠山東北流至河南新城縣又東南過伊闕中大禹所鑿也兩山相對望之若闕左傳女寛守闕塞卽其地括地志伊闕山在洛州南十九里班志上黨壺關縣有羊腸阪此安邑四履所憑山河之固也書曰成湯放桀于南巢濟子禮翻華戶化翻】}
商紂之國,左孟門,右太行,常山在其北,大河經其南;脩政不德,武王殺之。|{
	【水經注孟門在河東北屈縣西卽龍門上口也淮南子曰龍門未闢呂梁未鑿河出孟門之上溢而逆流無有丘陵名曰洪水太行山在河內野王縣西北常山在常山郡上曲陽縣西北河水自孟門南抵華陰屈而東流紂都朝歌河經其南北屈之孟門在朝歌西北恐不可言左索隱曰孟門別一山在朝歌東此特左右二字之差而誤耳春秋說題辭河之爲言荷也荷精分布懷陰引度也釋名河下也随地下處而通流也書曰武王勝殷殺紂太行之行戶剛翻北屈陸求忽翻顔居勿翻】
	}
由此觀之,在德不在險。
若君不脩德,舟中之人皆敵國也!”
武侯曰:“善”
魏置相,相田文。|{
	【相息亮翻此田文非齊之田文】}
吳起不悅,謂田文曰:“請與子論功可乎?”
田文曰:“可。”
起曰:“將三軍,使士卒樂死,敵國不敢謀,子孰與起?”
文曰:“不如子。”|{
	【將卽亮翻樂音洛】}
起曰:“治百官,親萬民,實府庫,子孰與起?”
文曰:“不如子。”|{
	【治直之翻】
	}
起曰:“守西河,秦兵不敢東鄉,韓、趙賓從子孰與起?”
文曰:“不如子。”|{
	【鄉讀曰向賓從猶言賓服也】}
起曰:“此三者子皆出吾下而位居吾上何也?”
文曰:“主少國疑,大臣未附,百姓不信,方是之時,屬之子乎,屬之我乎?”|{
	【少詩照翻屬子欲翻】}
起默然良久曰:“屬之子矣!”

久之,魏相公叔尚[魏公]主而害吳起.|{
	【如淳口天子嫁女於諸侯必使諸侯同姓者主之故謂之公主帝姊妹曰長公主諸王女曰翁主師古曰如說得之天子不親主㛰故謂之公主諸王則自主婚故其女曰翁主翁者父也言父主其婚也亦曰王主言王自主其婚也揚雄方言云周晉秦隴謂父曰翁而臣瓚王楙或云公者比於上爵或云主者婦人尊稱皆失之劉貢父曰予謂公主之稱本出秦舊男爲公子女爲公主古者大夫妻稱主故以公配之若謂同姓主之故謂之公主則周之事秦不知用也古之嫁女禮當如周使大夫主之何不謂之夫主乎然則謂之王主者猶言王子也謂之公主者緣公而生耳毛晃曰尚崇也高也貴也飾也加也尊也娶公主謂之尚言帝王之女尊而尚之不敢言娶也相息亮翻】
	}
公叔之僕曰:“起易去也。


\textcolor{red}{起爲人剛勁自喜。}|{
	易,以䜴翻。
	去,起呂翻。
	師古曰:喜,許吏翻。
	}
子先言於君曰:
“吳起,賢人也,而君之國小,臣恐起之無留心也。
君盍試延以女,起無留心,則必辭矣。”
子因與起歸而使公主辱子,起見公主之賤子也,必辭,則子之計中矣。”|{
	【中竹仲翻】}
公叔從之,吳起果辭公主。
魏武侯疑之而未信,起懼誅,遂奔楚。


楚悼王素聞其賢,至則任之爲相。%吴起奔楚
起明灋審令,|{
	【相息亮翻灋占法字】}
捐不急之官,廢公族疏遠者,以撫養戰鬭之士,要在強兵,破遊說之言從橫者。|{
	【捐余專翻弃也除去也漢書音義曰以利合曰從以威力相脅曰橫或曰南北曰從從者連南北爲一西鄉以擯秦東西曰橫橫者離山東之交使之西鄉以事秦說式芮翻從卽容翻橫亦作衡音同】}
於是南平百越,|{
	【韋昭曰越有百邑】}
北卻三晉,西伐秦,諸侯皆患楚之強;
而楚之貴戚大臣多怨吳起者。

秦惠公薨,子出公立。|{
	出,非謚也;
	以其失國出死,故曰出公。
	}

趙武侯薨,國人復立烈侯之太子章,是爲敬侯。|{
	諡法:夙夜警戒曰敬。
	}

韓烈侯薨子文侯立


十六年
初命齊大夫田和爲諸侯。|{
	田氏自此遂有齊國。
	田和是爲太公。
	}
趙公子朝作亂,\xout{出,}奔魏;與魏襲邯鄲,不克。|{
	邯音寒鄲音丹】}
\par
十七年
秦庶長改逆獻公于河西而立之;
殺出子及其母,沈之淵旁。|{
	後秦制爵,
	一級曰公士,
	二上造,
	三簪裊,
	四不更,
	五大夫,
	六官大夫,
	七公大夫,
	八公乘,
	九五大夫,
	十左庶長,
	十一右庶長,
	十二左更,
	十三中更,
	十四右更,
	十五少上造,
	十六大上造,
	十七駟車庶長,
	十八大庶長,
	十九關內侯,
	二十徹侯。
	師古曰:庶長,言衆列之長。
	註又詳見下卷顯王十年前。
	據史記:威烈王十一年秦靈公卒,子獻公師隰不得立,立靈公季父悼子,是爲簡公。
	出子,簡公之孫也。
	今庶長改迎獻公而殺出子。
	正義曰:西者,秦州西縣,秦之舊地。
	時獻公在西縣,故迎立之。
	余謂此言河西,非西縣也。
	靈公之卒,獻公不得立,出居河西;
	河西者,黃河之西,蓋漢涼州之地。
	“裊”,當作“褭”,乃了翻。
	更,工衡翻。
	乘,繩證翻。
	長,知丈翻。
	}
齊伐魯。
韓伐鄭,取陽城;|{
	漢陽城縣屬潁川郡;
	是爲地中,成周於此以土圭測日景。
	}
伐宋,執宋公。
齊太公薨,子桓公午立。


十九年魏敗趙師于兔臺|{
	【史記趙世家曰魏敗我兔臺築剛平正義曰兔臺剛平並在河北敗補邁翻】}
\par
二十年日有食之旣|{
	【旣盡也】}
\par



二十一年楚悼王薨貴戚大臣作亂攻吳起 起走之王尸而伏之|{
	【之往也往赴王尸而伏其側】}
擊起之徒因射刺起幷中王尸|{
	【射而亦翻刺七亦翻中竹仲翻】}
旣葬肅王卽位|{
	【諡法剛德克就曰肅執心决斷曰肅】}
使令尹盡誅爲亂者|{
	【令尹楚相也】}
坐起夷宗者七十餘家|{
	【夷殺也夷宗者殺其同宗也】}
\par
二十二年齊伐燕取桑丘魏韓趙伐齊至桑丘|{
	【此桑丘非二年所書楚之桑丘括地忠曰桑丘故城俗名敬城在易州遂城縣蓋燕之南界也】}
\par
二十三年趙襲衛不克|{
	【成王封康叔於衛居河淇之間故殷墟也至懿公爲狄所滅東徙度河文公徙居楚丘遂國於濮陽是年愼公頹之三十五年自康叔至愼公凡三十二世】}
齊康公薨無子田氏遂幷齊而有之|{
	【姜氏至此滅矣】}
是歲齊桓公亦薨子威王因齊立|{
	【諡法強毅訅正曰威訅渠留翻齊桓公田午訅謀也】}
\par
二十四年狄敗魏師于澮|{
	【漢之中山上黨西河上郡自春秋以來狄皆居之此亦其種也水經澮水出河東絳縣東澮山西過絳縣南又西南過祁宫南又西南至王橋入汾水括地志澮山在絳州翼城縣東北敗補邁翻澮古外翻】}
魏韓趙伐齊至靈丘|{
	【史記正義曰靈丘河東蔚州縣余按蔚州之靈丘卽漢代郡之靈丘此時齊境安能至代北邪此卽孟子謂蚳鼃辭靈丘請士師之地班志曰齊地北有千乘清河以南漢清河郡有靈縣清河北接趙魏之境此爲近之蚳音遲鼃烏花翻】}
晉孝公薨子靖公俱酒立|{
	【諡法柔衆安民曰靖又恭已鮮言曰靖】}
\par
二十五年蜀伐楚取兹方|{
	【據史記蜀伐楚取兹方楚爲扞關以拒之則兹方之地在扞關之西劉昭志巴郡魚復縣有扞關】}
子思言苟變於衛侯曰其才可將五百乘|{
	【古者兵車一乘甲士三人步卒七十二人五百乘三萬七千五百人國語曰苟本自黃帝之子將卽亮翻下同乘繩證翻】}
公曰吾知其可將然變也嘗爲吏賦於民而食人二鷄子故弗用也子思曰夫聖人之官人猶匠之用木也|{
	【夫音扶】}
取其所長棄其所短故梓連抱而有數尺之朽良工不棄今君處戰國之世|{
	【處昌呂翻】}
選爪牙之士而以二卵棄干城之將|{
	【詩赳赳武夫公侯干城毛氏傳曰干扞也音戶旦翻鄭氏箋曰干也城也皆所以禦難也干讀如字】}
此不可使聞於鄰國也公再拜曰謹受教矣衛侯言計非是而羣臣和者如出一口|{
	【和戶卧翻】}
子思曰以吾觀衛所謂君不君臣不臣者也|{
	【君不君臣不臣論語載齊景公之言】}
公丘懿子曰何乃若是|{
	【公丘複姓諡法温柔賢善曰懿】}
子思曰人主自臧則衆謀不進|{
	【臧善也】}
事是而臧之猶卻衆謀况和非以長惡乎|{
	【和戶卧翻長知丈翻】}
夫不察事之是非而悅人讚已闇莫甚焉不度理之所在而阿諛求容諂莫甚焉|{
	【度徒洛翻】}
君闇臣諂以居百姓之上民不與也若此不已國無類矣子思言於衛侯曰君之國事將日非矣公曰何故對曰有由然焉君出言自以爲是而卿大夫莫敢矯其非卿大夫出言亦自以爲是而士庶人莫敢矯其非君臣旣自賢矣|{
	【白虎通曰君羣也羣心之所歸心也臣堅也厲志自堅也】}
而羣下同聲賢之賢之則順而有福矯之則逆而有禍如此則善安從生詩曰具曰予聖誰知烏之雌雄|{
	【詩正月之辭毛氏傳曰君臣俱自謂聖也鄭氏箋曰時君臣賢愚適同如烏之雌雄相似誰能別異之乎又曰烏之雌雄不可別者以翼右掩左雄左掩右雌陰陽相下之義也】}
抑亦似君之君臣乎 魯穆公薨子共公奮立|{
	【諡法布德就義曰穆中情見貌曰穆尊賢敬讓曰共旣過能改曰共執事堅固曰共共讀曰㳟 考異曰司馬遷史記六國表周威烈王十九年甲戌魯穆公元年烈王元年丙午共公元年顯王十七年己巳康公元年二十六年戊寅景公元年赧王元年丁未平公元年二十年丙寅文公元年四十三年己丑頃公元年五十九年乙巳周亡秦莊襄元年壬子楚滅魯按魯世家穆公三十三年卒若元甲戌終乙巳則是三十二年也共公二十二年卒若元丙午終戊辰則是二十三年也康公九年卒景公二十五年卒平公二十二年卒若元丁未終乙丑則是十九年也文公二十三年卒頃公二十四年楚滅魯班固漢書律歷志文公作緡公其在位之年與世家異者惟平公二十年耳本志自魯僖公五年正月辛亥朔旦冬至推之至成公十二年正月庚寅朔旦冬至定公七年正月己巳朔旦冬至元公四年正月戊申朔旦冬至康公四年正月丁亥朔旦冬至緡公二十二年正月丙寅朔旦冬至漢高祖八年十一月乙巳朔旦冬至武帝元朔六年十一月甲申朔旦冬至元帝初元二年十一月癸亥朔旦冬至共間相距皆七十六年此最爲得實又與魯世家注皇甫謐所紀歲次皆合今從之六國表差謬難可盡據也余按考異自魯僖公五年至漢元帝初元二年六百餘年間十二月朔旦冬至相距皆七十六年此最爲得實又與魯世家注皇甫謐所紀歲次皆合蓋謂劉彛叟長歷也且言史記六國表差謬難可盡據又按通鑑目錄編年用劉彝叟長歷漢武帝太初元年初用夏正定歷史記歷書是年書閼逢攝提格目錄書強圉赤奮若閼逢攝提格甲寅也強圉赤奮若丁丑也有二十四年之差溫公用彝叟歷邵康節皇極經世書亦用彝叟歷康節少自雄其才旣學力慕高遠一見李之才遂從而受學廬于共城百源冬不爐夏不扇夜不就席者數年覃思於易經也皇極經世書不能違彜叟歷及其來居於洛而溫公亦奉祠以書局在洛相過從稔又夙所敬者也余意其講明之間必嘗及此而决于用彜叟歷讀考異此一段辭意可見】}
韓文侯薨子哀侯立
\par
二十六年王崩子烈王喜立 魏韓趙共廢晉靖公爲家人而分其地|{
	【唐叔不祀矣】}
\par
\section{烈王}|{
	【名喜安王之子】}
\par
元年日有食之 韓滅鄭因徙都之|{
	【韓本都平陽其地屬漢之河東郡中間徙都陽翟鄭都新鄭其地屬漢之河南郡鄭桓公始封于鄭其地屬漢之京兆後滅虢鄶而國於溱洧之間故曰新鄭左傳鄭莊公所謂吾先君新邑於此是也今韓旣滅鄭自陽翟徙都之韓旣都鄭故時人亦謂韓王爲鄭王考之戰國策韓非子可見】}
趙敬侯薨子成侯種立|{
	【種章勇翻】}
三年燕敗齊師於林狐|{
	【敗補邁翻】}
魯伐齊入陽關|{
	【徐廣曰陽關在鉅平班志鉅平縣屬太山郡括地志陽關故城在兖州博城縣南二十九里其城之西臨汶水汶音問】}
魏伐齊至博陵|{
	【史記正義曰博陵在濟州西界宋白曰史記齊威王伐晉至博陵徐廣曰東郡之博平漢爲縣】}
燕僖公薨子桓公立 宋休公薨子辟公立|{
	【辟亦諡法之所不載】}
衛愼公薨子聲公訓立|{
	【諡法敏以敬曰愼戴記思慮深遠曰愼】}
四年趙伐衛取都鄙七十三|{
	【周禮太宰以八則治都鄙注云都之所居曰鄙都鄙卿大夫之采邑蓋周之制四縣爲都方四十里一千六百井積一萬四千四百夫五都爲鄙鄙五百家也此時衛國褊小若都鄙七十三以成周之制率之其地廣矣盡衛之提封未必能及此數也更俟博考】}
魏敗趙師于北藺|{
	【班志西河郡有藺縣史記正義曰在石州其地於趙爲西北故曰北藺藺離進翻】}
\par
五年魏伐楚取魯陽|{
	【左傳所謂劉累遷於魯縣卽魯陽也班志魯陽縣屬南陽郡史記正義曰今汝州魯山縣】}
韓嚴遂弑哀侯國人立其子懿侯初哀侯以韓廆爲相而愛嚴遂二人甚相害也嚴遂令人刺韓廆於朝廆走哀侯哀侯抱之人刺韓廆兼及哀侯|{
	【戰國策以聶政刺韓相事及幷中哀侯爲一事此從史記蜀本注曰按太史公年表及韓世家於韓烈侯三年皆書聶政殺韓相俠累於哀侯六年又皆書嚴遂弑哀侯以刺客傳考之聶政殺俠累事在哀侯時以戰國策考之亦然從傳與戰國策則是年表世家於烈侯三年書盗殺俠累誤矣通鑑於烈侯三年載聶政刺俠累事又於哀侯六年韓嚴遂殺其君哀侯是從年表世家所書蓋刺客傳初不言幷殺哀侯止戰國策言之通鑑豈以此疑之歟故載幷刺哀侯不書聶政止曰使人以此求之則通鑑之意不以嚴仲子爲嚴遂亦不以俠累爲韓廆止從年表世家而不信其傳也余按溫公與劉道原書亦疑此事廆戶賄翻相息亮翻刺七亦翻朝直遙翻走音奏】}
魏武侯薨不立太子子罃與公中緩爭立國內亂|{
	【罃於耕翻中讀曰仲】}
\par
六年

1\textcolor{blue}{齊威王}來朝。
是時周室微弱,諸侯莫朝,而齊獨朝之,天下以此益賢威王。|{
	朝,直遙翻。
	}

2趙伐齊,至鄄。|{
	班志,濟陰郡有鄄城縣。鄄,工掾翻。
	}

3魏敗趙師于懷。|{
	班志,河內郡有懷縣。
	魏收地形志,懷州武德郡有懷縣,縣管內有懷城。
	敗,補邁翻。
	}

4齊威王召卽墨大夫,語之曰:‘自子之居卽墨也,毁言日至。|{
	班志,卽墨縣屬膠東國。
	括地志:卽墨故城,在萊州膠水縣南六十里。
	宋白曰:城臨墨水,故曰卽墨。
	語,牛倨翻;下同。
	}
然吾使人視卽墨,田野辟,|{
	辟,讀曰闢;下同。
	}
人民給,官無事,東方以寧;
是子不事吾左右以求助也!’
封之萬家。
召阿大夫,語之曰:‘自子守阿,譽言日至。|{
	阿,卽東阿縣;
	班志屬東郡。
	譽,音余,稱其美也。
	}
吾使人視阿,田野不辟,人民貧餒。
昔日趙攻鄄,子不救;|{
	鄄,工掾翻。
	}
衛取薛陵,子不知;|{
	薛陵,春秋薛國之墟。
	也班,志薛縣屬魯國,而衛國在漢東郡陳留界。
	薛陵屬齊而近於衛,故爲所取。
	齊後封田嬰於此。
	}
是子厚幣事吾左右以求譽也!
是日,烹阿大夫及左右嘗譽者。
於是羣臣聳懼,莫敢飾詐,務盡其情,齊國大治,強於天下。|{
	譽,音余。
	治,直吏翻。
	}

5楚肅王薨,無子,立其弟良夫,是爲宣王。

6宋辟公薨,子剔成立。|{
	剔,他歷翻。
	}


七年日有食之 王崩弟扁立|{
	【據班書古今人表師古注扁音篇】}
是爲顯王 魏大夫王錯出奔韓|{
	【姓譜王氏之所自出非一出太原琅邪者周靈王太子晉之後北海陳留齊王田和之後東海出自姬姓高平京兆魏信陵君之後天水東平新蔡新野山陽中山章武東萊河東者殷王子比干爲紂所害子孫以王者之後號曰王氏余謂此皆後世以諸郡著姓言之耳春秋之時自有王姓莫能審其所自出】}
公孫頎謂韓懿侯曰魏亂可取也|{
	【公孫姓也黃帝公孫氏頎渠希翻】}
懿侯乃與趙成侯合兵伐魏戰于濁澤大破之遂圍魏|{
	【史記正義曰徐廣以爲長社濁澤非也括地志云濁水源出蒲州解縣東北平地爾時魏都安邑韓趙伐魏豈至河南長社邪解縣濁水近於魏都當是也】}
成侯曰殺罃立公中緩割地而退我二國之利也懿侯曰不可殺魏君暴也割地而退貪也不如兩分之魏分爲兩不強於宋衛則我終無魏患矣趙人不聽懿侯不悅以其兵夜去趙成侯亦去罃遂殺公中緩而立|{
	【中讀曰仲】}
是爲惠王太史公曰魏惠王所以身不死國不分者二國之謀不和也若從一家之謀魏必分矣故曰君終無適子其國可破也|{
	【索隱曰蓋古人之言及俗說故云故曰適讀曰嫡】}
\par
資治通鑑卷一

季劄,季札\\
寜,寧\\
别,別\\
謚,諡\\
為,爲\\
苖,苗\\
温,溫\\
鬛,鬣\\
謚,諡\\
没,沒\\
潜,潛\\
䖍,虔\\
夀,壽\\
眾,衆\\
復姓,複姓\\
横                                                                                                                                                            ,橫\\
孔頴,孔穎\\
遥,遙\\
悦,悅\\
内,內\\
并,幷\\
既,旣,旣\\
即,卽\\
恒,恆\\
姧,奸\\
絶,絕\\
彦,彥\\
修,脩\\
概,槪\\

历史人物
智伯
预让
魏文侯
