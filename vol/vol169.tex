資治通鑑卷一百六十九
宋 司馬光 撰

胡三省 音註

陳紀三|{
	起昭陽協洽盡柔兆閹茂凡四年}


世祖文皇帝下

天嘉四年春正月齊以太子少傅魏收兼尚書右僕射時齊主終日酣飲朝事專委侍中高元海|{
	酣戶甘翻朝直遙翻下同}
元海庸俗帝亦輕之以收才名素盛故用之而收畏懦避事尋坐阿縱除名 |{
	考異曰北齊書帝紀正月乙亥收為僕射己卯除名相去五日不容如此之速恐誤今去其日}
兖州刺史畢義雲作書與高元海論叙時事元海入宫不覺遺之給事中李孝貞得而奏之帝由是疎元海以孝貞兼中書舍人徵義雲還朝和士開復譖元海|{
	復扶又翻}
帝以馬鞭箠元海六十責曰汝昔敎我反|{
	事見上卷二年箠止蕊翻}
以弟反兄幾許不義以鄴城兵抗并州幾許無智|{
	幾居豈翻}
出為兖州刺史 甲申周廸衆潰脫身踰嶺奔晉安|{
	臨川郡南城縣有東興嶺通晉安}
依陳寶應官軍克臨川獲廸妻子寶應以兵資廸留異又遣子忠臣隨之虞寄與寶應書以十事諫之曰自天厭梁德英雄互起人人自以為得之然夷凶翦亂四海樂推者陳氏也|{
	樂音洛}
豈非歷數有在惟天所授乎一也以王琳之彊侯瑱之力進足以搖蕩中原争衡天下退足以屈彊江外|{
	瑱他甸翻又音鎮屈其勿翻彊其兩翻}
雄張偏隅|{
	張知亮翻}
然或命一旅之師或資一士之說琳則瓦解氷泮投身異域瑱則厥角稽顙|{
	書泰誓曰若崩厥角言如角之崩也孟子曰若崩厥角稽首文雖小異意則大同此止言厥角稽顙當以顛蹶之蹶為義說式芮翻稽音啓}
委命闕庭斯又天假其威而除其患二也今將軍以藩戚之重|{
	陳編寶應於屬籍故云然}
東南之衆盡忠奉上戮力勤王豈不勲高竇融|{
	竇融以河西歸漢累世貴盛}
寵過吳芮|{
	吳芮以長沙奉漢高祖賢之制詔御史長沙王忠其定著今至傳國五世}
析珪判野|{
	揚雄解嘲曰析人之珪師古註云析分也判亦分也判野謂畫野分土君國子民而傳之後世也}
南面稱孤乎三也聖朝棄瑕忘過寛厚得人至於余孝頃潘純陀李孝欽歐陽頠等|{
	朝直遙翻高祖永定元年歐陽頠為周文育所禽潘純陀李孝欽皆王琳將也孝欽及余孝頃二年為周廸所禽純陀盖琳敗而歸陳也頠魚委翻}
悉委以心腹任以爪牙胸中豁然曾無纎芥况將軍舋非張繡罪異畢諶|{
	張繡殺曹操之子其後歸操操厚待之事見漢獻帝紀又操為兖州以畢諶為别駕張邈以兖州叛刼諶母弟妻子操謝遣之諶頓首言無二心既出遂亡去及破呂布諶生得衆為之懼操曰夫人孝於親者豈有不忠於君乎吾所求也以為魯相舋許覲翻諶氏壬翻}
當何慮於危亡何失於富貴四也方今周齊鄰睦境外無虞并兵一向匪朝伊夕非劉項競逐之機楚趙連從之勢|{
	從子容翻}
何得雍容高拱坐論西伯哉五也|{
	范曄論隗囂曰若囂命會符運敵非天力雖坐論西伯豈為過哉注云言不遇光武為敵則不謝西伯也}
且留將軍狼顧一隅亟經摧衂|{
	亟去吏翻頻數也}
聲實虧喪膽氣衰沮其將帥首鼠兩端唯利是視孰能被堅執鋭長驅深入繫馬埋輪奮不顧命以先士卒者乎六也|{
	喪息浪翻沮在呂翻將即亮翻帥所類翻被皮義翻先悉薦翻}
將軍之彊孰如侯景將軍之衆孰如王琳武皇滅侯景於前今上摧王琳於後此乃天時非復人力|{
	復扶又翻}
且兵革已後民皆厭亂其孰能棄墳墓捐妻子出萬死不顧之計從將軍於白刃之間乎七也歷觀前古子陽季孟顛覆相尋餘善右渠危亡繼及|{
	子陽公孫述字季孟隗囂字二人事見漢光武紀餘善右渠事見漢武帝紀}
天命可畏山川難恃况將軍欲以數郡之地當天下之兵以諸侯之資拒天子之命強弱逆順可得侔乎八也且非我族類其心必異|{
	左傳季文子引史佚之言}
不愛其親豈能及物留將軍身縻國爵子尚王姬|{
	易曰我有好爵吾與爾縻之子尚王姬謂異子貞臣尚主也}
猶且棄天屬而不顧背明君而孤立危急之日豈能同憂共患不背將軍者乎|{
	背蒲妹翻}
至於師老力屈懼誅利賞必有韓智晉陽之謀張陳井陘之勢九也|{
	韓智事見一卷周威烈王二十三年張陳事始秦二世三年終漢高帝三年陘音刑}
北軍萬里遠鬬鋒不可當|{
	兵自建康來建康於晉安為北故曰北軍萬里遠鬬者無反顧之心有必死之志故其鋒不可當}
將軍自戰其地人多顧後衆寡不敵將帥不侔師以無名而出事以無機而動以此稱兵|{
	稱猶舉也}
未知其利十也為將軍計莫若絶親留氏|{
	寶應娶留異之女為妻}
釋甲偃兵一遵詔旨方今藩維尚少|{
	少詩沼翻}
皇子幼冲凡豫宗族皆蒙寵樹况以將軍之地將軍之才將軍之名將軍之勢而克脩藩服北面稱臣寧與劉澤同年而語其功業哉|{
	劉澤漢高祖疎屬事見十三卷漢高后七年}
寄感恩懷德不覺狂言斧鉞之誅其甘如薺|{
	薺齊濟翻}
寶應覽書大怒或謂寶應曰虞公病勢稍篤言多錯謬寶應意乃小釋亦以寄民望故優容之 周梁躁公侯莫陳崇從周主如原州|{
	諡法好變動民曰躁}
帝夜還長安人竊怪其故崇謂所親曰吾比聞術者言晉公今年不利車駕今忽夜還不過晉公死耳|{
	比毗至翻宇文護封晉公}
或發其事乙酉帝召諸公於大德殿面責崇崇惶恐謝罪其夜冢宰護遣使將兵就崇第逼令自殺|{
	護當恐懼脩省引咎避權不當專殺功臣使疏吏翻將即亮翻}
葬如常儀 壬辰以高州刺史王法為南徐州刺史臨川太守周敷為南豫州刺史|{
	五代志高凉郡梁置高州南豫州時治宣城巨俱翻}
周主命司憲大夫拓跋廸|{
	唐六典御史大夫秦官歷晉宋齊梁陳後魏北齊後周並不置大夫而以中丞為臺主後周秋官置司憲中大夫二人掌丞司寇之法以左右刑罰盖比御史中丞之職也}
造大律十五篇|{
	五代志周造大律凡二十五篇一刑名二法例三祀享四朝會五婚姻六戶禁七水火八興繕九衛宫十市㕓十一鬭競十二刼盗十三賊叛十四毁亡十五違制十六關津十七諸侯十八廐牧十九雜犯二十詐偽二十一請求二十二告言二十三逃亡二十四繫訊二十五斷獄當從志作二十五篇}
其制罪一曰杖刑自十至五十二曰鞭刑自六十至百三曰徒刑自一年至五年四曰流刑自二千五百里至四千五百里五曰死刑磬絞斬梟裂|{
	古者公族有罪磬於甸人鄭玄曰懸縊殺之曰磬絞者全其身首斬者殊死梟者掛其首於木上裂者車裂梟堅堯翻}
凡二十五等|{
	五刑之屬各有五合二十五等}
庚戌以司空南徐州刺史侯安都為江州刺史 辛酉周詔大冢宰晉國公親則懿昆|{
	昆兄也}
任當元輔自今詔誥及百司文書並不得稱公名護抗表固讓 三月乙丑朔日有食之 齊詔司空斛律光督步騎二萬築勲掌城於軹關|{
	五代志軹關在河内郡王屋縣騎奇寄翻軹音只}
仍築長城二百里置十二戍丙戌齊以兼尚書右僕射趙彦深為左僕射|{
	左僕射當作右}


|{
	僕射盖先是兼官今正除右僕射也}
夏四月乙未周以柱國達奚武為太保 周主將視學以太傅燕國公于謹為三老|{
	燕因肩翻}
謹上表固辭不許仍賜以延年杖戊午帝幸太學謹入門帝迎拜於門屏之間謹答拜有司設三老席於中楹南向太師護升階設几謹升席南面憑几而坐大司馬豆盧寜升階正舄帝升階立於斧扆之前西面|{
	扆屏風也斧扆畫文為斧形扆於豈翻}
有司進饌帝跪設醬豆|{
	醬食味之主古之養老執醬而饋今跪而設豆}
親為之袒割|{
	為於偽翻袒割袒而割牲也}
謹食畢帝親跪授爵以酳|{
	酳羊晉翻以酒漱口也}
有司撤訖帝北面立而訪道謹起立於席後對曰木受繩則正后從諫則聖|{
	書傅說告高宗之言}
明王虛心納諫以知得失天下乃安又曰去食去兵信不可去|{
	論語孔子答子貢之言去羌呂翻}
願陛下守信勿失又曰有功必賞有罪必罰則為善者日進為惡者日止又曰言行者立身之基|{
	行下孟翻}
願陛下三思而言九慮而行勿使有過天子之過如日月之食人莫不知願陛下慎之帝再拜受言謹答拜禮成而出|{
	三代而下視學養老乞言之禮惟漢明帝周武帝行之}
司空侯安都恃功驕横|{
	横戶孟翻}
數聚文武之士騎射賦詩|{
	數所角翻下又數同騎奇寄翻}
齋中賓客動至千人部下將帥多不遵法度檢問收攝|{
	攝録也捕也將即亮翻帥所類翻}
輒奔歸安都上性嚴整内銜之安都弗之覺每有表啓封訖有事未盡開封自書之云又啓某事及侍宴酒酣或箕踞傾倚常陪樂遊園褉飲|{
	樂音洛}
謂上曰何如作臨川王時上不應安都再三言之上曰此雖天命抑亦明公之力宴訖啓借供帳水飾欲載妻妾於御堂宴飲上雖許之意甚不懌明日安都坐於御座賓客居羣臣位稱觴上夀|{
	上時掌翻}
會重雲殿災安都帥將士帶甲入殿上甚惡之隂為之備|{
	此皆日前事史歷叙安都致敗之由重直龍翻惡烏路翻}
及周廸反朝議謂當使安都討之|{
	朝直遙翻下同}
而上更使吳明徹|{
	更工衡翻}
又數遣臺使案問安都部下檢括亡叛|{
	使疏吏翻}
安都遣其别駕周宏實自託於舍人蔡景歷|{
	蔡景歷為中書舍人自武帝以來特蒙親任盖陳朝事權皆在中書也}
并問省中事景歷錄其狀具奏之因希旨稱安都謀反上慮其不受召故用為江州五月安都自京口還建康部伍入於石頭六月帝引安都宴於嘉德殿又集其部下將帥會於尚書朝堂於坐收安都囚於嘉德西省|{
	坐徂臥翻}
又收其將帥盡奪馬仗而釋之因出蔡景歷表以示於朝乃下詔㬥其罪惡明日賜死宥其妻子資給其喪初高祖在京口|{
	高祖與王僧辯既平臺城出鎮京口}
嘗與諸將宴杜僧明周文育侯安都為夀|{
	奉觴上夀也}
各稱功伐|{
	積功曰伐}
高祖曰卿等悉良將也而並有所短杜公志大而識闇狎於下而驕於上周侯交不擇人而推心過差侯郎慠誕而無厭輕佻而肆志|{
	厭於鹽翻佻他彫翻}
並非全身之道卒皆如其言|{
	知臣莫若君誠哉是言也卒子恤翻}
乙卯齊主使兼散騎常侍崔子武來聘|{
	散悉亶翻騎奇寄翻下同}
齊侍中開府儀同三司和士開有寵於齊主齊主外朝視事及在内宴賞須臾之間不得不與士開相見或累日不歸一日數入或放還之後俄頃即追未至之間連騎督趣|{
	趣讀曰促}
姦諂百端寵愛日隆前後賞賜不可勝紀每侍左右言辭容止極諸鄙䙝以夜繼晝無復君臣之禮常謂帝曰自古帝王盡為灰土堯舜桀紂竟復何異陛下宜及少壯極意為樂縱横行之|{
	勝音升復扶又翻少詩照翻樂音洛縱子容翻}
一日取快可敵千年國事盡付大臣何慮不辦無為自勤約也帝大悅於是委趙彦深掌官爵元文遙掌財用唐邕掌外騎兵|{
	外兵及騎兵也勃海王歡相魏丞相府外兵曹騎兵曹分掌兵馬及文宣受禪諸司咸歸尚書惟此二曹不廢謂之外兵省騎兵省據和士開傳時委邕掌外兵白建掌騎兵}
信都馮子琮胡長粲掌東宫帝三四日一視朝書數字而已|{
	朝直遙翻}
畧無所言須臾罷入長粲僧敬之子也|{
	胡僧敬見一百五十八卷梁武帝大同七年}
帝使士開與胡后握槊河南康獻王孝瑜諫曰皇后天下之母豈可與臣下接手孝瑜又言趙郡王叡其父死於非命|{
	叡父琛勃海王歡之弟也亂歡後庭因杖而斃}
不可親近|{
	近其靳翻}
由是叡及士開共譖之士開言孝瑜奢僭叡言山東唯聞河南王不聞有陛下|{
	齊主多居晉陽在山西司冀定殷瀛滄之地皆在山東}
帝由是忌之孝瑜竊與爾朱御女言|{
	齊制八十一御女比正四品古之御妻也孝瑜傳云爾朱事太后孝瑜先與之通}
帝聞之大怒庚申頓飲孝瑜酒三十七盃|{
	飲於禁翻}
孝瑜體肥大腰帶十圍帝使左右婁子彥載以出酖之於車至西華門煩躁投水而絶|{
	躁則到翻}
贈太尉錄尚書事諸侯在宫中者莫敢舉聲唯河間王孝琬大哭而出|{
	孝琬孝瑜之弟也}
秋七月戊辰周主幸原州 八月辛丑齊以三臺宫為大興聖寺九月壬戌廣州刺史陽山穆公歐陽頠卒詔子紇襲

父爵位|{
	陽山郡公五代志南海郡含洭縣梁置陽山郡為歐陽紇不就徵阻兵而反張本頠魚委翻紇下没翻}
甲子周主自原州登隴|{
	登隴坂也}
周廸復越東興嶺為寇|{
	東興嶺在臨川郡南城縣界唐志撫州南城縣武德四年析置永城東興二縣七年省沈約曰東興縣吳立屬臨川郡復扶又翻}
辛未詔護軍章昭達將兵討之|{
	昭達時為護軍將軍}
丙戌周主如同州 初周人欲與突厥木杆可汗連兵伐齊|{
	厥九勿翻杆公旦翻可從刊入聲汗音寒}
許納其女為后遣御伯大夫楊荐|{
	唐六典曰後周天官府置御伯中大夫二人天子出入則侍於左右大祭祀盥洗則授中武帝改御伯為納言盖侍中之職也宣帝末又别置侍中為加官}
及左武伯太原王慶|{
	左武伯盖侍衛之官注見後}
往結之齊人聞之懼亦遣使求昏於突厥賂遺甚厚|{
	使疏吏翻遺子季翻}
木杆貪齊幣重欲執荐等送齊荐知之責木杆曰太祖昔與可汗共敦鄰好|{
	好呼到翻}
蠕蠕部落數千來降太祖悉以付可汗使者以快可汗之意|{
	事見一百六十六卷梁敬帝紹泰元年蠕人兖翻降戶江翻}
如何今日遽欲背恩忘義獨不愧鬼神乎|{
	背蒲妹翻}
木杆慘然良久曰君言是也吾意决矣當相與共平東賊然後遣女荐等復命 |{
	考異曰典畧在保定二年按王慶傳云是歲乃興入并之役故置於此}
公卿請發十萬人擊齊柱國楊忠獨以為得萬騎足矣戊子遣忠將步騎一萬與突厥自北道伐齊又遣大將軍達奚武帥步騎三萬自南道出平陽期會於晉陽|{
	忠將即亮翻又音如字領也騎奇寄翻帥讀曰率下同}
冬十一月辛酉章昭達大破周迪迪脫身潛竄山谷民相與匿之雖加誅戮無肯言者 十二月辛卯周主還長安|{
	自隴上還}
丙申大赦 章昭達進軍度嶺趣建安討陳寶應|{
	趣七喻翻}
詔益州刺史余孝頃|{
	梁元帝之世益州之地已入於周陳命余孝頃遙領益州刺史耳}
督會稽東陽臨海永嘉諸軍自東道會之|{
	會工外翻}
是歲初祭始興昭烈王於建康用天子禮|{
	帝嗣高祖以子伯茂奉始興昭烈王之祀今初以天子禮祀之非禮也}
周楊忠拔齊二十餘城齊人守陘嶺之隘|{
	唐志代州雁門縣有東陘關西陘關陘音刑隘烏懈翻}
忠擊破之突厥木杆地頭步離三可汗以十萬騎會之|{
	木杆分國為三部木杆牙帳居都斤山地頭可汗統東方步離可汗統西方厥九勿翻杆公旦翻}
己丑自恒州三道俱入|{
	恒戶登翻}
時大雪數旬南北千餘里平地數尺齊主自鄴倍道赴之戊午至晉陽斛律光將步兵三萬屯平陽|{
	拒達奚武之兵也}
己未周師及突厥逼晉陽齊主畏其彊戎服帥宫人欲東走避之趙郡王叡河間王孝琬叩馬諫孝琬請委叡部分|{
	分扶問翻}
必得嚴整帝從之命六軍進止皆取叡節度而使并州刺史段韶總之|{
	委叡部分而段韶總其事}


五年春正月庚申朔齊主登北城|{
	晉陽北城也}
軍容甚整突厥咎周人曰爾言齊亂故來伐之今齊人眼中亦有鐵何可當邪|{
	厥九勿翻邪音耶}
周人以步卒為前鋒從西山下去城二里許諸將咸欲逆擊之|{
	將即亮翻}
段韶曰步卒力勢自當有限今積雪既厚逆戰非便不如陳以待之彼勞我逸破之必矣既至齊悉其鋭師鼓譟而出突厥震駭引上西山不肯戰|{
	陳讀曰陣上時掌翻}
周師大敗而還|{
	還從宣翻又如字下同}
突厥引兵出塞縱兵大掠自晉陽以往七百餘里人畜無遺|{
	謂晉陽以北七百餘里畜許救翻}
段韶追之不敢逼突厥還至陘嶺凍滑乃鋪氈以度胡馬寒瘦膝已下皆無毛比至長城|{
	陘音刑比必利翻長城即文宣所築者}
馬死且盡截矟杖之以歸達奚武至平陽未知忠退斛律光與書曰鴻鵠已翔於寥廓羅者猶視於沮澤|{
	漢司馬相如難蜀父老曰焦明已翔乎寥廓羅者猶視乎藪澤沮將預翻}
武得書亦還光逐之入周境獲二千餘口而還光見帝於晉陽帝以新遭大寇抱光頭而哭任城王湝進曰何至於此乃止|{
	見賢遍翻湝音皆又戶皆翻}
初齊顯祖之世周人常懼齊兵西度每至冬月守河椎氷及世祖即位嬖倖用事朝政漸紊|{
	嬖卑義翻又博計翻朝直遙翻紊音問}
齊人椎氷以備周兵之逼斛律光憂之曰國家常有吞關隴之志今日至此而唯翫聲色乎 辛巳上祀南郊 二月庚寅朔日有食之 初齊顯祖命羣臣刋定魏麟趾格為齊律|{
	見一百六十三卷梁簡文帝大寶元年}
久而不成時軍國多事决獄罕依律文相承謂之變法從事世祖即位思革其弊乃督脩律令者至是而成律十二篇|{
	五代志齊律十二篇一名例二禁衛三婚戶四擅興五違制六詐偽七鬬訟八賊盗九捕斷十毁損十一廐牧十二雜律}
令四十卷其刑名有五一曰死重者轘之|{
	轘胡慣翻即車裂也}
次梟首次斬次絞二曰流投邉裔為兵三曰刑自五歲至一歲四曰鞭自百至四十五曰杖自三十至十凡十五等|{
	死四等流一等刑五等鞭五等杖三等通十八等今曰凡十五等通鑑依五代志大凡為十五等之文也梟堅堯翻}
其流外官及老小閹癡|{
	老者小者閹者癡者與流外官皆贖鄭玄曰閹精氣閉藏者癡不慧也}
并過失應贖者皆以絹代金三月辛酉班行之因大赦|{
	赦其宿罪此後有犯者皆以法令施行}
是後為吏者始守法令又敕仕門子弟常講習之|{
	仕門謂入仕之家}
故齊人多曉法又令民十八受田輸租調二十充兵六十免力役六十六還田免租調一夫受露田八十畝|{
	杜佑曰不栽樹者謂之露田調力弔翻下夫調牛調同}
婦人四十畝奴婢依良人|{
	言奴婢受田依良人畝數}
牛受六十畝|{
	按五代志丁牛一頭受田六十畝限止四年丁牛者勝耕之牛牧牛者得受其田}
大率一夫一婦調絹一匹綿八兩墾租二石義租五斗奴婢準良人之半|{
	奴婢者官常役其力故所調半於良人}
牛調二尺墾租一斗義租五升墾租送臺義租納郡以備水旱|{
	調力弔翻}
己巳齊羣盜田子禮等數十人共刼太師彭城景思王浟為主詐稱使者徑向浟第至内室稱敕牽浟上馬臨以白刃欲引向南殿浟大呼不從|{
	浟夷周翻使疏吏翻上時掌翻呼火故翻}
盜殺之 庚辰周初令百官執笏|{
	記玉藻曰史進象笏書思對命注云意所思念將以告君者也對所以對君也命所受命也書之於笏為失忘也又曰凡有指畫於君前者用笏造受命於君前用笏笏畢用也因飾焉笏度二尺有六寸其中博三寸其殺六分而去一應劭曰昔荆軻逐秦王其後謁者持匕首以備不虞從此侍宫皆執刀劒高祖偃武脩文始制手版代焉隋志曰中世以來唯八座尚書執笏笏者白筆綴其頭紫囊裹之其餘公卿但執手版謂之筆笏盖以記事受言西魏以降通用象牙五品以下通用竹木爾雅釋名笏忽也君有教命及所白則書其上以備忽忘也唐會要曰笏舊制三品已上前挫後直五品以上前挫後屈武德已來一例上圓下方開元八年諸笏三品已上前詘後直五品已上前詘後挫並用象九品已上任用竹木上挫下方男以上聽依品爵執笏假官板亦依此例}
齊以斛律光為司徒武興王普為尚書左僕射普歸彦之兄子也甲申以馮翊王潤為司空 夏四月辛卯齊主使兼散騎常侍皇甫亮來聘|{
	散悉亶翻騎奇寄翻}
庚子周主遣使來聘|{
	使疏吏翻}
癸卯周以鄧公河南竇熾為大宗伯五月壬戌封世宗之子賢為畢公 甲子齊主還鄴|{
	自晉陽還}
壬午齊以趙郡王叡為錄尚書事前司徒婁叡為太尉甲申以段韶為太師丁亥以任城王湝為大將軍|{
	湝戶皆翻又音皆}
壬辰齊主如晉陽 周以太保達奚武為同州刺史 六月齊主殺樂陵王百年時白虹暈日兩重|{
	重直龍翻}
又横貫而不達赤星見齊主欲以百年猒之|{
	見賢遍翻猒於叶翻}
會博陵人賈德胄教百年書百年嘗作數敕字德胄封以奏之帝發怒使召百年百年自知不免割帶玦留與其妃斛律氏見帝於凉風堂|{
	見賢遍翻}
使百年書敕字驗與德胄所奏相似遣左右亂捶之又令曳之遶堂行且捶所過血皆遍地氣息將盡乃斬之|{
	孝昭殺文宣之子武成又殺孝昭之子天之報應固不爽遺言諄諄竟何益哉捶止蘂翻}
棄諸池池水盡赤妃把玦哀號不食|{
	號戶高翻}
月餘亦卒玦猶在手拳不可開其父光自擘之乃開|{
	擘博厄翻分擘也}
庚寅周改御伯為納言 初周太祖之從賀拔岳在關中也遣人迎晉公護於晉陽|{
	梁武帝中大通二年宇文泰從賀拔岳入關按下護答母書云違離膝下三十五年逆而數之正在是年}
護母閻氏及周主之姑皆留晉陽|{
	周主之姑蓋宇文泰之妹也}
齊人以配中山宫|{
	中山宫慕容氏之故宫自魏以來以為别宫}
及護用事遣間使入齊求之莫知音息|{
	間古莧翻音息猶今人言信息也使疏吏翻下同}
齊遣使者至玉璧求通互市護欲訪求母姑使司馬下大夫尹公正至玉璧與之言|{
	司馬下大夫即軍司馬之職}
使者甚悅勲州刺史韋孝寛獲關東人復縱之因致書為言西朝欲通好之意|{
	復扶又翻又音如字為於偽翻好呼到翻下同周在關西故稱西朝朝直遙翻}
是時周人以前攻晉陽不得志謀與突厥再伐齊|{
	厥九勿翻}
齊主聞之大懼許遣護母西歸且求通好先遣其姑歸 秋八月丁亥朔日有食之 周遣柱國楊忠會突厥伐齊至北河而還|{
	水經河水東逕沃野故城南又北屈而為南河出焉河水又北迤西溢於窳渾縣故城東又屈而東流為北河東逕高闕南還從宣翻又音如字}
戊子周以齊公憲為雍州牧|{
	雍於用翻}
宇文貴為大司徒九月丁巳以衛公直為大司空追錄佐命元功封開府儀同三司隴西公李昞為唐公|{
	錄昞父虎佐命之功也李氏有天下國號曰唐本此昞音丙}
太馭中大夫長樂公若干鳳為徐公|{
	周官太馭中大夫掌馭玉路以祀及犯軷屬夏官樂音洛}
昞虎之子鳳惠之子也|{
	李虎始見一百五十六卷梁武帝中大通六年若千惠見一百五十八卷大同九年}
乙丑齊主封其子綽為南陽王儼為東平王儼太子之母弟也 突厥寇齊幽州衆十餘萬入長城大掠而還 周皇姑之歸也齊主遣人為晉公護母作書言護幼時數事又寄其所著錦袍以為信驗|{
	為於偽翻著陟畧翻}
且曰吾屬千載之運|{
	屬之欲翻載子亥翻}
蒙大齊之德矜老開恩許得相見禽獸草木母子相依吾有何罪與汝分離今復何福還望見汝|{
	復扶又翻又音如字}
言此悲喜死而更蘇世間所有求皆可得母子異國何處可求假汝貴極王公富過山海有一老母八十之年飄然千里死亡旦夕不得一朝蹔見|{
	蹔與暫同}
不得一日同處|{
	處昌呂翻}
寒不得汝衣饑不得汝食汝雖窮榮極盛光耀世間於吾何益吾今日之前汝既不得申其供養|{
	供居用翻養羊亮翻}
事往何論今日以後吾之殘命唯繫於汝爾戴天履地中有鬼神勿云冥昧而可欺負護得書悲不自勝|{
	勝音升}
復書曰區宇分崩遭遇災禍違離膝下三十五年|{
	離力智翻}
受生稟氣皆知母子誰同薩保|{
	護字薩保薩桑葛翻}
如此不孝子為公侯母為俘隸暑不見母暑寒不見母寒衣不知有無食不知饑飽冺如天地之外無由暫聞分懷寃酷終此一生|{
	分扶問翻}
死若有知冀奉見於泉下耳不謂齊朝解網惠以德音磨敦四姑並許矜放|{
	護兄弟呼其母為阿摩敦四姑即周主之姑也第四朝直遙翻下同}
初聞此旨魂爽飛越|{
	左傳鄭子產曰人生始化曰魄既生魄陽曰魂用物精多則魂魄強是以有精爽至於神明}
號天叩地|{
	號戶刀翻}
不能自勝|{
	勝音升下足勝同}
齊朝霈然之恩既已霑洽有家有國信義為本伏度來期已應有日|{
	朝直遙翻度徒洛翻}
一得奉見慈顔永畢生願生死肉骨|{
	生死謂使死者復生肉骨謂使枯骨再肉}
豈過今恩負山戴嶽未足勝荷|{
	勝音升荷下可翻}
齊人留護母使更與護書邀護重報往返再三時段韶拒突厥軍於塞下|{
	厥九勿翻}
齊主使黄門徐世榮乘傳齎周書問韶韶以周人反覆本無信義比晉陽之役其事可知護外託為相其實主也既為母請和|{
	傳張戀翻比毗至翻相息亮翻為於偽翻下主為同}
不遣一介之使|{
	使疏吏翻下同}
若據移書即送其母恐示之以弱不如且外許之待和親堅定然後遣之未晩齊主不聽即遣之|{
	護母至長安席未及煖而洛陽之師已出卒如段韶之言}
閻氏至周舉朝稱慶周主為之大赦凡所資奉窮極華盛每四時伏臘周主帥諸親戚行家人之禮稱觴上夀|{
	朝直遙翻帥讀曰率下同上時掌翻}
突厥自幽州還|{
	還從宣翻又如字}
留屯塞北更集諸部兵遣使告周欲與共擊齊如前約閏月乙巳突厥寇齊幽州晉公護新得其母未欲伐齊恐負突厥約更生邉患不得已徵二十四軍及左右廂散隸秦隴巴蜀之兵|{
	二十四軍六柱國及十二大將軍所統關中諸府兵也安定公泰相魏左右各十二軍並屬相府左右廂禁衛兵也兼有秦隴巴蜀之兵散隸於左右廂者散悉亶翻}
并羌胡内附者凡二十萬人冬十月甲子周主授護斧鉞於廟廷丁卯親勞軍於沙苑|{
	勞力到翻}
癸酉還宫護軍至潼關遣柱國尉遲迴帥精兵十萬為前鋒趣洛陽大將軍權景宣帥山南之兵趣懸瓠|{
	山南荆襄之兵尉紆勿翻帥讀曰率下同趣七喻翻}
少師楊檦出軹關|{
	少始照翻檦與標同}
周迪復出東興|{
	復扶又翻下衆復同}
宣城太守錢肅鎮東興以城降迪|{
	守式又翻降戶江翻}
吳州刺史陳詳將兵擊之|{
	五代志鄱陽郡梁置吳州陳廢鄱陽之吳州而於吳郡置吳州將即亮翻又音如字領也}
詳兵大敗迪衆復振南豫州刺史西豐脫侯周敷帥所部擊之|{
	諡法無脫諡盖以周敷輕脫而死故以為諡}
至定川|{
	據周敷傳定川縣名}
與迪對壘迪紿敷曰吾昔與弟戮力同心|{
	事見一百六十六卷梁敬帝太平元年紿蕩亥翻}
豈規相害今願伏罪還朝|{
	朝直遙翻}
因弟披露心腑先乞挺身共盟敷許之方登壇為迪所殺 陳寶應據晉安建安二郡水陸為柵以拒章昭達昭達與戰不利因據上流命軍士伐木為筏施拍其上會大雨江漲昭達放筏衝寶應水柵盡壞之|{
	筏音伐壞音怪}
又出兵攻其步軍方合戰上遣將軍余孝頃自海道適至|{
	去年遣孝頃督會稽諸郡兵自東道合攻陳寶應}
并力乘之十一月己丑寶應大敗逃至莆口|{
	莆口在唐泉州莆田縣界莆田今興化軍即其地虞寄傳作莆田莆音蒲}
謂其子曰早從虞公計不至今日昭達追擒之并擒留異及其族黨送建康斬之異子貞臣以尚主得免寶應賓客皆死上聞虞寄嘗諫寶應命昭達禮遣詣建康既見勞之曰管寧無恙|{
	漢末管寧客遼東不受公孫度爵命已而復得還鄉里故以寄况之勞力到翻}
以為衡陽王掌書記|{
	上子伯信封衡陽王奉獻王昌祀}
周晉公護進屯弘農尉遲迥圍洛陽雍州牧齊公憲同州刺史達奚武涇州總管王雄軍於邙山|{
	雍於用翻邙音亡}
戊戌齊主遣兼散騎常侍劉逖來聘|{
	散悉亶翻騎奇寄翻}
初周楊檦為邵州刺史|{
	五代志絳郡垣縣後魏置邵郡及白水縣後周置邵州改白水為亳城隋廢州及郡改亳城為垣縣}
鎮捍東境二十餘年數與齊戰|{
	數所角翻}
未嘗不捷由是輕之既出軹關獨引兵深入又不設備甲辰齊太尉婁叡將兵奄至大破檦軍檦遂降齊|{
	將即亮翻下同降戶江翻下同}
權景宣圍懸瓠十二月齊豫州道行臺豫州刺史太原王士良永州刺史蕭世怡並以城降之景宣使開府郭彥守豫州謝徹守永州|{
	五代志汝南郡城陽縣舊置楚州後齊曰永州}
送士良世怡及降卒千人於長安周人為土山地道以攻洛陽三旬不克晉公護命諸將塹斷河陽路|{
	塹七艷翻斷音短}
遏齊救兵然後同攻洛陽諸將以為齊兵必不敢出唯張斥候而已齊遣蘭陵王長恭大將軍斛律光救洛陽畏周兵之彊未敢進齊主召并州刺史段韶謂曰洛陽危急今欲遣王救之突厥在北復須鎮禦如何|{
	復扶又翻}
對曰北虜侵邉事等疥癬今西隣闚逼乃腹心之病請奉詔南行齊主曰朕意亦爾乃令韶督精騎一千發晉陽|{
	騎奇寄翻}
丁巳齊主亦自晉陽赴洛陽己未齊太宰平原靖翼王淹卒 段韶自晉陽行五

日濟河會連日隂霧壬戌韶至洛陽帥帳下三百騎與諸將登邙阪觀周軍形勢|{
	邙阪北邙之阪也帥讀曰率}
至太和谷與周軍遇韶即馳告諸營追集騎士結陳以待之|{
	陳讀曰陣下光陳同}
韶為左軍蘭陵王長恭為中軍斛律光為右軍周人不意其至皆忷懼|{
	忷許拱翻}
韶遙謂周人曰汝宇文護纔得其母遽來為寇何也周人曰天遣我來有何可問韶曰天道賞善罰惡當遣汝送死來耳周人以步兵在前上山逆戰|{
	上北邙逆齊兵與戰上時掌翻}
韶且戰且却以誘之|{
	誘音酉}
待其力弊然後下馬擊之周師大敗一時瓦解投墜谿谷死者甚衆蘭陵王長恭以五百騎突入周軍遂至金墉城下城上人弗識長恭免胄示之面乃下弩手救之周師在城下者亦解圍遁去委棄營幕自邙山至穀水|{
	水經穀水出弘農澠池縣墦塚林穀陽谷東北過穀城縣北又東過河南縣北東南入於洛}
三十里中軍資器械彌滿川澤唯齊公憲達奚武及庸忠公王雄在後勒兵拒戰|{
	諡法危身奉上曰忠}
王雄馳馬衝斛律光陳光退走雄追之光左右皆散唯餘一奴一矢雄按矟不及光者丈餘|{
	矟色角翻}
謂光曰吾惜爾不殺當生將爾見天子光射雄中額雄抱馬走至營而卒|{
	射而亦翻中竹仲翻卒子恤翻}
軍中益懼齊公憲拊循督勵衆心小安至夜收軍憲欲待明更戰達奚武曰洛陽軍散人情震駭若不因夜速還明日欲歸不得武在軍久備見形勢公少年未經事|{
	少詩照翻}
豈可以數營士卒委之虎口乎乃還權景宣亦棄豫州走丁卯齊主至洛陽己巳以段韶為太宰斛律光為太尉蘭陵王長恭為尚書令|{
	賞戰勝之功也}
壬申齊主如虎牢遂自滑臺如黎陽丙子至鄴楊忠引兵出沃野應接突厥軍糧不給諸軍憂之計無所出忠乃招誘稽胡酋長咸在坐|{
	此稽胡與離石稽胡同種散居銀夏之間誘音酉酋慈秋翻長知兩翻坐徂臥翻}
詐使河州刺史王傑勒兵鳴鼓而至曰大冢宰已平洛陽欲與突厥共討稽胡之不服者坐者皆懼忠慰諭而遣之於是諸胡相帥饋輸軍糧填積屬周師罷歸忠亦還|{
	帥讀曰率屬之欲翻}
晉公護本無將畧是行也又非本心故無功|{
	兵出無名事故不成其宇文護之謂乎將即亮翻}
與諸將稽首謝罪周主慰勞罷之|{
	稽音啓勞力到翻}
是歲齊山東大水饑死者不可勝計|{
	勝音升}
宕昌王梁彌定屢寇周邉周大將軍田宏討滅之以其地置宕州|{
	宕州在長安西南一千六百五十六里宕徒浪翻}


六年春正月癸卯齊以任城王湝為大司馬|{
	湝戶皆翻又音皆任音壬}
齊主如晉陽 二月辛丑周遣陳公純許公貴神武公竇毅南陽公楊荐等|{
	魏收志朔州有神武郡領尖山樹頹二縣水經注樹頹水出沃陽縣東山下西南流右合誥升爰水其水左合中陵川後魏置神武郡於神武川治尖山縣隋為神武縣屬馬邑郡荐在甸翻}
備皇后儀衛行殿并六宫百二十人詣突厥可汗牙帳逆女毅熾之兄子也|{
	熾時為柱國周主既誅宇文護以為太傅}
丙寅周以柱國安武公李穆為大司空綏德公陸通為大司寇|{
	李穆陸通皆縣公也五代志襄陽郡舊有安武縣西魏併為南漳縣雕隂郡有綏德縣西魏置}
壬申周主如岐州 夏四月甲寅以安成王頊為司

空頊以帝弟之重勢傾朝野直兵鮑僧叡恃頊勢為不法|{
	自秦以來王公府皆有直兵}
御史中丞徐陵為奏彈之從南臺官屬引奏案而入|{
	御史臺為南臺}
上見陵章服嚴肅為歛容正坐|{
	為於偽翻下上為同}
陵進讀奏版時頊在殿上侍立仰視上流汗失色陵遣殿中御史引頊下殿|{
	殿中侍御史居殿中察非法故使之引頊下殿}
上為之免頊侍中中書監朝廷肅然 丙午齊大將軍東安王婁叡坐事免齊著作郎祖珽有文學多技藝而疎率無行|{
	珽它鼎翻技渠綺翻行下孟翻}
嘗為高祖中外府功曹|{
	高歡都督中外諸軍事以珽為府功曹}
因宴失金叵羅|{
	叵羅盃之屬叵普火翻}
於珽髻上得之又坐詐盗官粟三千石鞭二百配甲坊|{
	珽與令史李雙倉督成租等作晉州啓請粟三千石珽代功曹趙彥深宣教給之事覺鞭配}
顯祖時珽為秘書丞盗華林遍畧及有它贜當絞除名為民|{
	華林遍畧梁武帝集諸學士所撰也南人持至鄴下賣之高澄集書吏一日一夜寫畢退還其本珽盗遍畧數帙質錢樗蒲重以得罪至顯祖時又盗遍畧一部及擬補令史十餘人皆有受由是除名}
顯祖雖憎其數犯法而愛其才伎令直中書省|{
	伎渠綺翻}
世祖為長廣王珽為胡桃油獻之|{
	珽善為胡桃油以塗畫}
因言殿下有非常骨法孝徵夢殿下乘龍上天|{
	孝徵祖珽字也是時人多以字行上時掌翻}
王曰若然當使兄大富貴及即位擢拜中書侍郎遷散騎常侍與和士開共為姦諂珽私說士開曰君之寵幸振古無比|{
	振古猶云自古也說式芮翻下宜說微說同}
宫車一日晚駕|{
	晚晏也義與晏駕同}
欲何以克終士開因從問計珽曰宜說主上云文襄文宣孝昭之子俱不得立今宜令皇太子早踐大位以定君臣之分|{
	踐慈演翻分扶問翻}
若事成中宫少主必皆德君此萬全計也請君微說主上令粗解|{
	少詩照翻粗坐五翻解戶買翻曉也}
珽當自外上表論之士開許諾會有彗星見|{
	彗祥歲翻見賢遍翻}
太史奏云彗除舊布新之象當有易主珽於是上書言陛下雖為天子未為極貴宜傳位東宫且以上應天道并上魏顯祖禪子故事|{
	見一百三十二卷宋明帝泰始六年是上併上時掌翻下公上同}
齊主從之丙子使太宰段韶持節奉皇帝璽綬傳位於太子緯太子即皇帝位於晉陽宫|{
	諱緯字仁綱武成帝之長子也璽斯氏翻綬音受}
大赦改元天統又詔以太子妃斛律氏為皇后於是羣公上世祖尊號為太上皇帝軍國大事咸以聞使黄門侍郎馮子琮尚書左丞胡長粲輔導少主出入禁中專典敷奏子琮胡后之妹夫也祖珽拜秘書監加儀同三司大被親寵|{
	被皮義翻}
見重二宫 丁丑齊以賀拔仁為太師侯莫陳相為太保馮翊王潤為司徒趙郡王叡為司空河南王孝琬為尚書令戊寅以瀛州刺史尉粲為太尉斛律光為大將軍東安王婁叡為太尉|{
	尉粲婁叡並為太尉此承齊紀之悮按尉粲傳粲為太傅當從之婁叡封郡王五代志郎邪郡沂水縣舊置東安郡}
尚書僕射趙彥深為左僕射|{
	按齊紀彥深自右僕射遷為左僕射}
五月突厥遣使至齊|{
	使疏吏翻}
始與齊通 六月己巳齊主使兼散騎常侍王季高來聘 秋七月辛巳朔日有食之 上遣都督程靈洗自鄱陽别道擊周廸破之廸與麾下十餘人竄於山穴中日月浸久從者亦稍苦之|{
	從才用翻}
後遣人潛出臨川市魚鮭|{
	吳人總稱魚菜為鮭音戶皆翻}
臨川太守駱牙執之令取廸自效因使腹心勇士隨之入山其人誘廸出獵|{
	守式又翻誘音酉}
勇士伏於道傍出斬之丙戌傳首至建康 庚寅周主如秦州八月丙子還長安 己卯立皇子伯固為新安王伯恭為晉安王伯仁為廬陵王伯義為江夏王|{
	夏戶雅翻}
冬十月辛亥周以函谷關城為通洛防以金州刺史賀若敦為中州刺史鎮函谷|{
	五代志河南郡新安縣後周置中州杜佑在今洛州新安縣東}
敦恃才負氣顧其流輩皆為大將軍敦獨未得兼以湘州之役全軍而返謂宜受賞翻得除名|{
	事見上卷元年二年}
對臺使出怨言|{
	使疏吏翻}
晉公護怒徵還逼令自殺臨死謂其子弼曰吾志平江南今而不果汝必成吾志吾以舌死汝不可不思因引錐刺弼舌出血以誡之|{
	刺七亦翻弼能從父之志而取江南不能守父之戒而保其身}
十一月癸未齊太上皇至鄴 齊世祖之為長廣王也數為顯祖所捶|{
	數所角翻捶止蘃翻}
心常銜之顯祖每見祖珽常呼為賊故珽亦怨之且欲求媚於世祖乃說世祖曰文宣狂㬥何得稱文既非創業何得稱祖若文宣為祖陛下萬歲後當何所稱帝從之己丑改諡太祖獻武皇帝廟號高祖獻明皇后為武明皇后令有司更議文宣諡號|{
	說式芮翻更工衡翻諡神至翻}
十二月乙卯封皇子伯禮為武陵王 壬戌齊上皇如晉陽 庚午齊改諡文宣皇帝為景烈皇帝廟號威宗|{
	諡法布義行剛曰景有功安民曰烈猛以彊果曰威有威可畏曰威以刑服遠曰威}


天康元年|{
	是年二月改元}
春正月己卯日有食之 癸未周大赦改元天和 辛卯齊主祀圜丘癸巳祫太廟|{
	五代志齊制圜丘方澤並三年一祭謂之禘祀圜丘則以蒼璧束帛正月上辛祀昊天上帝太廟則春祠夏禴秋嘗冬蒸皆以孟月并臘凡五祭三年一禘五年一祫謂之殷祭}
丙申齊以吏部尚書尉瑾為右僕射 己亥周主耕籍田|{
	籍在亦翻}
庚子齊主如晉陽 周遣小載師杜杲來聘|{
	周禮載師掌任土之法以物地事授地職而待其政令屬地官其官有上士二人中士四人而無大小之别五代志後周置載師掌任土之法辯夫家田里之數會六畜車乘之稽審賦役歛弛之節制畿疆修廣之役頒施惠之要審牧產之政}
二月庚戌齊上皇還鄴 丙子大赦改元|{
	改元天康}
三月己卯以安成王頊為尚書令 丙午周主祀南郊夏四月辛亥大雩上不豫臺閣衆事並令尚書僕射到仲舉五兵尚書

孔奐共决之奐琇之之曾孫也|{
	孔琇之見一百三十九卷齊明帝建武元年琇音秀}
疾篤奐仲舉與司空尚書令揚州刺史安成王頊吏部尚書袁樞中書舍人劉師知入侍醫藥樞君正之子也|{
	袁君正見一百六十三卷梁武帝太清三年}
太子伯宗柔弱上憂其不能守位謂頊曰吾欲遵太伯之事|{
	言以天下讓也}
頊拜伏泣涕固辭上又謂仲舉奐等曰今三方鼎峙四海事重宜須長君|{
	長知兩翻}
朕欲近則晉成遠隆殷法|{
	晉成帝立母弟為嗣事見九十七卷咸康八年殷法兄死弟及}
卿等宜遵此意孔奐流涕對曰陛下御膳違和痊復非久|{
	痊愈也復謂復初痊且緣翻}
皇太子春秋鼎盛聖德日躋|{
	毛萇曰躋升也鄭元曰言日進也}
安成介弟之尊足為周旦若有廢立之心臣等愚誠不敢聞詔上曰古之遺直復見於卿|{
	復扶又翻}
乃以奐為太子詹事

臣光曰夫人臣之事君將順其美正救其惡|{
	孝經記夫子之言}
孔奐在陳處腹心之重任|{
	處昌呂翻}
决社稷之大計苟以世祖之言為不誠則當如竇嬰面辯袁盎廷爭|{
	竇嬰事見十六卷漢景帝三年袁盎事見十二年爭讀曰諍}
防微杜漸以絶覬覦之心|{
	覬音冀覦音俞}
以為誠邪|{
	邪音耶}
則當請明下詔書宣告中外使世祖有宋宣之美高宗無楚靈之惡|{
	左傳宋宣公舍其子與夷而立其弟穆公穆公卒捨其子馮而立與夷君子曰宋宣公可謂知人矣立穆公其子饗之楚康王有疾其弟圍入問王疾縊而弑之遂殺其二子幕及平夏而自立是為靈王}
不然謂太子嫡嗣不可動揺欲保輔而安全之則當盡忠竭節如晉之荀息趙之肥義|{
	左傳晉獻公有疾屬其子奚齊於荀息息曰臣竭其股肱之力繼之以死公薨里克殺奚齊荀息將死之人曰不如立卓子而輔之荀息立卓子以葬獻公里克殺卓子荀息死之肥義事見四卷周赧王二十年}
奈何於君之存則逆探其情而求合焉|{
	探吐南翻}
及其既沒則權臣移國而不能救嗣主失位而不能死斯乃姦諛之尤者而世祖謂之遺直以託六尺之孤豈不悖哉|{
	悖蒲内翻}


癸酉上殂|{
	殂祚於翻}
上起自艱難知民疾苦性明察儉約每夜刺閨取外事分判者前後相續|{
	以錐繭物曰刺閨宫中小門也就閨中刺取外事故曰刺閨刺七賜翻}
敕傳更籖於殿中者必投籖於階石之上令鎗然有聲|{
	更工衡翻更籖更籌也鎗楚庚翻}
曰吾雖眠亦令驚覺|{
	覺古孝翻}
太子即位大赦五月己卯尊皇太后曰太皇太后皇后曰皇太后 乙酉齊以兼尚書左僕射武興王普為尚書令 吐谷渾龍涸王莫昌帥部落附於周以其地為扶州|{
	五代志同昌郡嘉誠縣後周置縣并龍涸郡及扶州總管府吐從暾入聲谷音浴帥讀曰率}
庚寅以安成王頊為驃騎大將軍司徒錄尚書都督中外諸軍事|{
	頊吁玉翻驃匹妙翻騎奇寄翻}
丁酉以中軍大將軍開府儀同三司徐度為司空以吏部尚書袁樞為左僕射吳興太守沈欽為右僕射|{
	守式又翻}
御史中丞徐陵為吏部尚書陵以梁末以來選授多濫乃為書示衆曰梁元帝承侯景之凶荒王太尉接荆州之禍敗|{
	王太尉謂僧辯也荆州禍敗謂江陵陷沒也}
故使官方窮此紛雜|{
	方法也窮極也}
永安之時|{
	南史徐陵傳作永定永定高祖受禪初元也當從之}
聖朝草創白銀難得黄札易營|{
	朝直遙翻易以豉翻}
權以官階代於錢絹致令員外常侍路上比肩諮議參軍市中無數豈是朝章固應如此|{
	朝直遙翻}
今衣冠禮樂日富年華|{
	謂一日富於一日一年華於一年也}
何可猶作舊意非理望也衆咸服之 己亥齊立上皇子宏為齊安王仁固為北平王仁英為高平王仁光為淮南王 六月齊遣兼散騎常侍韋道儒來聘|{
	散悉亶翻騎奇寄翻}
丙寅葬文皇帝於永寜陵廟號世祖 秋七月戊寅周築武功等諸城以置軍士|{
	武功即漢扶風武功縣周紀築武功都斜谷武都留谷津抗諸城}
丁酉立妃王氏為皇后|{
	臨海王立后}
八月齊上皇如晉陽周信州蠻冉令賢向五子王等據巴峽反|{
	巴峽在巴郡巴縣有明月廣德等峽亦謂之三峽}
攻陷白帝黨與連結二千餘里周遣開府儀同三司元契趙剛等前後討之終不克九月詔開府儀同三司陸騰督開府儀同三司王亮司馬裔討之騰軍於湯口|{
	水經江水自朐䏰縣東逕瞿巫灘左則湯溪水注之謂之湯口}
令賢於江南據險要置十城遠結涔陽蠻為聲援|{
	涔鉏簪翻丁度曰涔陽渚在郢中此盖荆州蠻也又水經涔水出漢中南鄭縣東南旱山東北流逕成固縣南城北北至沔陽縣南入於沔水經又曰涔水出作唐縣西北天門郡界東南流注於澧水九域志江陵府公安縣有涔陽鎮此涔陽當從九域志}
自帥精卒固守水邏城|{
	帥讀曰率邏即佐翻}
騰召諸將問計皆欲先取水邏後攻江南騰曰令賢内恃水邏金湯之固外託涔陽輔車之援資糧充實器械精新以我懸軍攻其嚴壘脫一戰不克更成其氣不如頓軍湯口先取江南翦其羽毛然後進軍水邏此制勝之術也乃遣王亮帥衆度江旬日拔其八城捕虜及納降各千計遂間募驍勇|{
	閒當作簡驍堅堯翻}
數道進攻水邏蠻帥冉伯犂冉安西素與令賢有仇騰說誘賂以金帛使為鄉導|{
	帥讀曰率說式芮翻鄉讀曰嚮}
水邏之旁有石勝城令賢使其兄子龍真據之騰密誘龍真龍真遂以城降水邏衆潰斬首萬餘級捕虜萬餘口令賢走追獲斬之騰積骸於水邏城側為京觀是後羣蠻望之輒大哭不敢復叛|{
	觀古玩翻復扶又翻}
向五子王據石墨城使其子寶勝據雙城|{
	今歸州巴東縣北臨大江有鐵鎗頭長數丈經數百年不損目曰向王鎗盖諸向所據處也}
水邏既平騰頻遣諭之猶不下進擊皆擒之盡斬諸向酋長捕虜萬餘口|{
	酋慈秋翻長知兩翻}
信州舊治白帝騰徙之於八陳灘北|{
	諸葛亮壘石為八陣於魚復平沙之上今謂之八陣磧夔州圖經云八陣磧在奉節縣西南七里又云在永安宫南一里渚下平磧上聚細石為之各高五丈皆碁布相當中間相去九尺正中開南北巷悉廣五尺凡六十四聚或為人散亂及為夏水所沒水退則依然如故又有二十四聚作兩層其後每層各十二聚陳讀曰陣}
以司馬裔為信州刺史小吏部隴西辛昂|{
	周既建六官以六部分屬六官小吏部屬天官}
奉使梁益且為騰督軍糧|{
	使疏吏翻下同為於偽翻}
時臨信楚合等州民多從亂|{
	五代志巴東郡臨江縣後周置臨州巴郡梁置楚州涪陵郡西魏置合州唐改臨州為忠州}
昂諭以禍福赴者如歸乃令老弱負糧壯夫拒戰咸樂為用|{
	樂音洛}
使還會巴州萬榮郡民反|{
	五代志清化郡梁置巴州所領永穆縣舊置萬榮郡唐志永穆縣屬通州我朝改通州為達州}
攻圍郡城遏絶山路昂謂其徒曰凶狡猖狂若待上聞孤城必陷苟利百姓專之可也遂募通開二州|{
	五代志通州郡梁置萬州西魏曰通州所領西流縣後魏之漢興縣也西魏置開州唐省西流縣入盛山縣杜佑曰通州漢宕渠之地梁於此置萬州以州内地萬餘頃故以為名西魏改通州以居四達之地}
得三千人倍道兼行出其不意直趣賊壘|{
	趣七喻翻}
賊以為大軍至望風瓦解一郡獲全周朝嘉之以為渠州刺史|{
	五代志宕渠郡梁置渠州朝直遙翻}
冬十月齊以侯莫陳相為太傅任城王湝為太保婁叡為大司馬馮翊王潤為太尉開府儀同三司韓祖念為司徒 庚申帝享太廟 十一月乙亥周遣使來弔|{
	使疏吏翻}
丙戌周主行視武功等新城|{
	行下孟翻}
十二月庚申還長安 齊河閒王孝琬怨執政|{
	怨讒殺其兄孝瑜也}
為草人而射之|{
	射而亦翻}
和士開祖珽譛之於上皇曰草人以擬聖躬也又前突厥至并州孝琬脫兜鍪抵地|{
	抵諸氏翻側擊也北齊書作抵丁禮翻}
云我豈老嫗須著此物此言屬大家也|{
	嫗威遇翻著陟略翻屬之欲翻此時已謂天子為大家言比上皇於婦人}
又魏世謠言河南種穀河北生白楊樹上金雞鳴河南北者河間也孝琬將建金雞大赦耳|{
	五代志曰後齊赦日武庫令設金雞及鼓於闕門外之右集囚於闕前撾鼔千聲釋焉爾雅翼曰海中星占曰天雞聲動為有赦故後魏北齊赦日皆設金雞掲於竿}
上皇頗惑之會孝琬得佛牙置第内夜有光上皇聞之使搜之得填庫矟幡數百|{
	填讀曰鎮矟色角翻}
上皇以為反具收訊諸姬有陳氏者無寵誣孝琬云孝琬常畫陛下像而哭之其實世宗像也|{
	孝琬父澄諡文襄皇帝廟號世宗}
上皇怒使武衛赫連輔玄倒鞭撾之|{
	倒鞭者執小頭以大頭撾之}
孝琬呼叔上皇曰何敢呼我為叔孝琬曰臣神武皇帝嫡孫文襄皇帝嫡子魏孝静皇帝之甥何為不得呼叔上皇愈怒折其兩脛而死|{
	折而設翻}
安德王延宗哭之淚赤|{
	延宗亦文襄之子幼為文宣所養問欲作何王對曰欲作衝天王文宣問楊愔愔曰天下無此郡名願使安於德乃封安德王淚赤者泣盡而繼之以血也魏中興初分樂陵置安德郡}
又為草人鞭而訊之曰何故殺我兄奴告之上皇覆延宗於地馬鞭鞭之二百幾死|{
	幾居依翻}
是歲齊賜侍中中書監元文遙姓高氏|{
	文遙孝昭帝之后黨也}
頃之遷尚書左僕射魏末以來縣令多用厮役|{
	厮音斯今相傳讀從詵入聲}
由是士流恥為之文遙以為縣令治民之本|{
	治直之翻}
遂請革選|{
	革更改也}
密擇貴遊子弟發敕用之猶恐其披訴悉召之集神武門令趙郡王叡宣旨唱名厚加慰諭而遣之齊之士人為縣自此始

資治通鑑卷一百六十九
