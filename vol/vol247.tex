\chapter{資治通鑑卷二百四十七}
宋 司馬光 撰

胡三省 音註

唐紀六十三|{
	起昭陽大淵獻盡閼逢困敦七月凡一年有奇}


武宗至道昭肅孝皇帝中

會昌三年春正月回鶻烏介可汗帥衆侵逼振武劉沔遣麟州刺史石雄都知兵馬使王逢帥沙陀朱邪赤心三部及契苾拓跋三千騎襲其牙帳|{
	拓跋即党項部落也帥讀曰率契欺訖翻 考異曰舊回鶻傳云豐州刺史石雄後唐獻祖紀年錄云石州刺史石雄按是時田牟為豐州刺史今從實録}
沔自以大軍繼之雄至振武登城望回鶻之衆寡見氊車數十乘|{
	氊車以氊為車屋乘繩證翻}
從者皆衣朱碧類華人|{
	從才用翻下侍從同衣於既翻華人謂中國人也}
使諜問之曰公主帳也雄使諜告之曰|{
	諜達協翻間也}
公主至此家也當求歸路今將出兵擊可汗請公主潛與侍從相保駐車勿動雄乃鑿城為十餘穴引兵夜出直攻可汗牙帳至其帳下虜乃覺之可汗大驚不知所為棄輜重走|{
	重直用翻}
雄追擊之庚子大破回鶻於殺胡山|{
	殺胡山即黑山}
可汗被瘡與數百騎遁去雄迎太和公主以歸 |{
	考異曰舊石雄傳曰三年回鶻大畧雲朔劉沔以太原之師屯于雲州沔謂雄曰國家以公主之故不欲急攻我輩捍邉但能除患專之可也雄受教自選勁騎得沙陀部落兼契苾拓跋雜虜夜馬邑徑趨烏介之牙時虜帳逼振武雄既入城登堞視其衆寡見氊車數十云云遂迎公主還太原回鶻傳烏介去幽州八十里下營是夜河東劉沔帥兵奄至烏介驚走東北依和解室韋下營不及將太和公主同走石雄兵遇公主帳因迎歸國後唐獻祖紀年錄曰沔表帝為前鋒回鶻可汗樹牙於殺胡山帝與石雄銜枚夜進圍其牙帳烏介可汗輕騎而遁帝於牙帳謁見太和公主奉而歸國按一品集會昌二年十月十七日狀訪聞劉沔頗練邉事唯臨機決策不免遲疑深恐過為慎重漸失事機望賜劉沔詔比緣回鶻未為侵擾且務綏懷今既殺戮邊人驅劫牛馬頻已有詔速令驅除自度便宜臨機應變不得過懷疑慮皆待朝廷指揮既假以使名令為諸軍節制邊境之事皆以責成向後或要移營進軍一切自取機便不必皆候進止實録戊寅詔劉沔云云如前據德裕此狀則沔豈敢不俟詔旨擅遣石雄襲擊可汗牙帳况已有不須聞奏之詔也舊德裕傳德裕曰杷頭烽北便是沙磧彼中野戰須用騎兵若以步卒敵之理難必勝今烏介所恃者公主如令勇將出騎奪得公主虜自敗矣上然之即令德裕草制處分伐叛記曰上問討襲之計德裕奏若以步兵與回鶻野戰必無勝理回鶻常質公主同行臣思得一計料回鶻必未知有斫營石雄驍勇無敵若令揀蕃渾及漢兵鋭卒銜枚夜進必取得公主兼可汗可擒上從之遂令石雄領蕃渾及漢兵夜進回鶻果無遊奕伏道直至帳幕方覺遂取得公主惟可汗輕騎而遁按德裕尋自請駐斫營事而石雄於城上見公主牙帳迎得之非因德裕之策今不取}
斬首萬級降其部落二萬餘人丙午劉沔捷奏至 李思忠入朝自以回鶻降將懼邊將猜忌|{
	降戶江翻將即亮翻}
乞并弟思貞等及愛弘順皆歸闕庭 庚戌以石雄為豐州都防禦使|{
	賞破回鶻之功也}
烏介可汗走保黑車子族|{
	胡嶠曰轄戛之北單于突厥又北黑車子善作車帳其人知孝義地貧無所產詳考新舊書黑車子即室韋之一種按是時賜戛斯詔云黑車子去漢界一千餘里 考異曰舊回鶻傳云烏介驚走東北約四百里外依和解室韋下營嫁妹與室韋依附之今從伐叛記實錄新傳舊張仲武傳又云烏介既敗乃依康居求活盡徙餘種寄託黑車子蓋以李德裕紀聖功碑云烏介并丁令以圖安依康居而求活盡徙餘種屈意黑車彼所謂康居用郅支故事耳致此誤也}
其潰兵多詣幽州降 二月庚申朔日有食之 詔停歸義軍|{
	置歸義軍見上卷上年}
以其士卒分隸諸道為騎兵優給糧賜 辛未戛斯遣使者注吾合索獻名馬二|{
	新書曰注吾虜姓也合言猛索者左也謂武猛善左射者索作素宋白曰索上聲}
詔太僕卿趙蕃飲勞之|{
	飲於禁翻勞力到翻}
甲戌上引對班在勃海使之上上欲令趙蕃就戛斯求安西北庭李德裕等上言安西去京師七千餘里北庭五千餘里借使得之當復置都護|{
	復扶又翻}
以唐兵萬人戍之不知此兵於何處追饋運從何道得通此乃用實費以易虛名非計也 |{
	考異曰德裕傳云三年二月趙蕃奏戛斯攻安西北庭都護府宜出師影援德裕奏辭與此同獻替記曰三年二月十一日延英德裕奏九日奉宣令臣等向趙蕃說於戛斯處邀求安西北庭深恐不可其下辭亦與此同按實録辛未注吾合索始至命趙蕃飲勞之丙子中書門下奏九日奉宣其辭亦與獻替記同不知宋據何書得此辛未及丙子日也今且沒其日繫於注吾合索入對之下以傳疑}
上乃止中書侍郎同平章事崔珙罷為右僕射 戛斯求

冊命李德裕奏宜與之結歡令自將兵求殺使者罪人|{
	戛斯遣使者送太和公主為回鶻所殺事見上卷上年}
及討黑車子上恐加可汗之名即不修臣禮踵回鶻故事求歲遺及賣馬|{
	遺唯季翻下同}
猶豫未決德裕奏戛斯已自稱可汗今欲藉其力恐不可吝此名回鶻有平安史之功故歲賜絹二萬匹且與之和市戛斯未嘗有功於中國豈敢遽求賂遺乎若慮其不臣當與之約必如回鶻稱臣乃行冊命又當敘同姓以親之使執子孫之禮上從之 庚寅太和公主至京師改封安定大長公主|{
	太和公主以長慶元年嫁回鶻至此得還安定新書作定安長知丈翻}
詔宰相帥百官迎謁於章敬寺前|{
	帥讀曰率}
公主詣光順門去盛服脫簪珥謝回鶻負恩和蕃無狀之罪|{
	唐公主入蕃者謂之和蕃公主今太和公主以回鶻犯邉故自謝和蕃無狀去羌呂翻}
上遣中使慰諭然後入宫陽安等六公主不來慰問安定公主各罰俸物及封絹|{
	陽安公主順宗之女宋白曰不至者陽安宣城真寧義寧臨真真源義昌六公主}
賜魏博節度使何重順名弘敬 三月以太僕卿趙蕃為安撫戛斯使上命李德裕草賜戛斯可汗書諭以貞觀二十一年戛斯先君身自入朝|{
	二十一年當作二十二年}
授左屯衛將軍堅昆都督迄于天寶朝貢不絶比為回鶻所隔|{
	比毗至翻}
回鶻凌虐諸蕃可汗能復讎雪怨茂功壯節近古無儔今回鶻殘兵不滿千人散投山谷可汗既與為怨須盡殱夷|{
	殱子廉翻滅也}
儻留餘燼必生後患又聞可汗受氏之源與我同族|{
	孔穎達曰天子賜姓賜氏諸侯但得賜氏不得賜姓降於天子也故隱八年左傳云無駭卒公問族於衆仲衆仲對曰天子建德因生以賜姓胙之土而命之氏諸侯以字為諡因以為族官有世功則有官族邑亦如之以此言之天子因諸侯先祖所生賜之曰姓杜預注云若舜生媯汭賜姓曰媯封舜之後於陳以所封之土命為氏舜後姓媯而氏曰陳故鄭駮異義云炎帝姓姜太皥之所賜也黄帝姓姬炎帝之所賜也故堯賜伯夷姓曰姜賜禹姓曰姒賜契姓曰子賜稷姓曰姬著在書傳如鄭此言是天子賜姓也諸侯賜卿大夫以氏若同姓公之子曰公子公子之子曰公孫公孫之子其親已遠不得上達於公故以王父字為氏若適夫人之子則以五十字伯仲為氏若魯之仲孫季孫是也若庶子妾子則以二十字為氏若臧氏展氏是也若異姓則以父祖官及所食之邑為氏以官為氏者則司馬司城是也以邑為氏者若韓趙魏是也凡賜氏族者此為卿乃賜有大功者生賜以族若叔孫得臣是也雖公子之身有大功德則以公子之字賜以為族若襄仲遂是也其無功德死後乃賜族若無駭是也若子孫若為卿其君不賜族自以王父字為族也氏族對之為别散則通也故左傳問族於衆仲下云公命以字為展氏是也其姓與氏散亦得通故春秋有姜氏子氏姜子皆姓而云氏是也}
國家承北平太守之後可汗乃都尉苖裔|{
	北平太守謂李廣都尉謂李陵}
以此合族尊卑可知今欲冊命可汗特加美號緣未知可汗之意且遣諭懷待趙蕃回日别命使展禮自回鶻至塞上及戛斯入貢每有詔敕上多命德裕草之德裕請委翰林學士上曰學士不能盡人意須卿自為之 劉沔奏歸義軍回鶻三千餘人及酋長四十三人凖詔分隸諸道皆大呼連營據滹沱河|{
	酋慈由翻長知丈翻呼火故翻章懷太子後漢書注曰山海經注云大戲之山滹沱之水出焉在今代州繁峙縣東流入定州深澤縣界九域志忻代二州注皆有滹沱水}
不肯從命已盡誅之回鶻降幽州者前後三萬餘人皆散隸諸道 李德裕追論維州悉怛謀事|{
	事見二百四十四卷文宗太和五年}
云維州據高山絶頂三面臨江在戎虜平川之衝是漢地入兵之路初河隴並沒唯此獨存吐蕃潛以婦人嫁此州門者二十年後兩男長成|{
	長知兩翻}
竊開壘門引兵夜入遂為所陷號曰無憂城從此得併力於西邉更無虞於南路|{
	并力於西邉謂吐蕃并力以攻岐隴邠涇靈夏也無虞於南路謂西川在吐蕃之南也自長安言之西川亦在劒關之南若吐蕃寇蜀則南路自維茂入北路自嶲州入}
憑陵近甸旰食累朝|{
	朝直遥翻旰古案翻}
貞元中韋臯欲經略河湟須此城為始萬旅盡銳急攻數年雖擒論莽熱而還|{
	還從宣翻又如字}
城堅卒不可克|{
	見二百三十六卷德宗貞元十七十八年卒子恤翻}
臣初到西蜀外揚國威中緝邊備其維州熟臣信令空壁來歸臣始受其降南蠻震懾山西八國皆願内屬其吐蕃合水棲雞等城|{
	翼州有合江守捉城與棲雞城本皆唐地沒于吐蕃}
既失險阨自須抽歸可減八處鎮兵坐收千餘里舊地且維州未降前一年吐蕃猶圍魯州|{
	魯州河曲六胡州之一也在宥州西界}
豈顧盟約臣受降之初指天為誓面許奏聞各加酬賞當時不與臣者望風疾臣詔臣執送悉怛謀等令彼自戮臣寧忍以三百餘人命棄信偷安累表陳論乞垂矜捨答詔嚴切竟令執還體備三木輿於竹畚|{
	畚布忖翻}
及將就路寃叫嗚嗚將吏對臣無不隕涕其部送者更為蕃帥譏誚云既已降彼|{
	此言吐蕃謂中國為彼也帥所類翻}
何用送來復以此降人戮於漢境之上|{
	復扶又翻}
恣行殘忍用固攜離|{
	謂戎蠻有攜離内向之心者畏吐蕃屠戮之慘不敢復懷反側以威虐固制之}
至乃擲其嬰孩承以槍槊絶忠款之路快兇虐之情從古已來未有此事雖時更一紀|{
	更工衡翻十二年為一紀太和五年悉怛謀死至是年適十二年}
而運屬千年|{
	謂千載一遇之運也屬之欲翻}
乞追奬忠魂各加褒贈詔贈悉怛謀右衛將軍

臣光曰論者多疑維州之取捨不能決牛李之是非臣以為昔荀吳圍鼓鼓人或請以城叛吳弗許曰或以吾城叛吾所甚惡也人以城來吾獨何好焉|{
	惡烏路翻好呼到翻下同}
吾不可以欲城而邇姦使鼓人殺叛者而繕守備|{
	見春秋左氏傳}
是時唐新與吐蕃修好而納其維州以利言之則維州小而信大以害言之則維州緩而關中急然則為唐計者宜何先乎悉怛謀在唐則為向化在吐蕃不免為叛臣其受誅也又何矜焉且德裕所言者利也僧孺所言者義也匹夫徇利而忘義猶恥之况天子乎譬如鄰人有牛逸而入于家或勸其兄歸之或勸其弟攘之勸歸者曰攘之不義也且致訟勸攘者曰彼嘗攘吾羊矣何義之拘牛大畜也|{
	畜許救翻}
鬻之可以富家以是觀之牛李之是非端可見矣|{
	元祐之初弃米脂等四寨以與西夏蓋當時國論大指如此}


夏四月辛未李德裕乞退就閒局上曰卿每辭位使我旬日不得所|{
	不得所猶言不安其所也}
今大事皆未就卿豈得求去初昭義節度使劉從諫累表言仇士良罪惡|{
	見二百四十五}


|{
	卷文宗太和八年}
士良亦言從諫窺伺朝廷|{
	伺相吏翻}
及上即位從諫有馬高九尺獻之上不受|{
	周禮馬八尺以上為龍七尺以上為騋六尺以上為馬馬高九尺蓋稀有也高古報翻}
從諫以為士良所為怒殺其馬由是與朝廷相猜恨遂招納亡命繕完兵械鄰境皆潛為之備從諫榷馬牧及商旅歲入錢五萬緡|{
	榷古岳翻}
又賣鐵煮鹽亦數萬緡大商皆假以牙職|{
	牙職牙前將校之職}
使通好諸道因為販易商人倚從諫勢所至多陵轢將吏諸道皆惡之|{
	好呼到翻轢郎狄翻惡烏路翻}
從諫疾病謂妻裴氏曰吾以忠直事朝廷而朝廷不明我志諸道皆不我與我死他人主此軍則吾家無炊火矣乃與幕客張谷陳揚庭謀效河北諸鎮以弟右驍衛將軍從素之子稹為牙内都知兵馬使從子匡周為中軍兵馬使|{
	稹止忍翻 考異曰實錄作莊周今從一品集}
孔目官王協為押牙親軍兵馬使以奴李士貴為使宅十將兵馬使劉守義劉守忠董可武崔玄度分將牙兵谷鄆州人|{
	鄆音運}
揚庭洪州人也從諫尋薨稹祕不喪王協為稹謀曰|{
	為于偽翻}
正當如寶歷年様為之|{
	敬宗寶歷元年劉悟死從諫得襲事見二百四十三卷}
不出百日旌節自至但嚴奉監軍厚遺勑使|{
	遺唯季翻}
四境勿出兵城中暗為備而已使押牙姜崟奏求國醫上遣中使解朝政以醫問疾|{
	崟魚音翻解戶買翻姓也}
稹又逼監軍崔士康奏稱從諫疾病請命其子稹為留後上遣供奉官薛士幹往諭指云恐從諫疾未平宜早就東都療之俟稍瘳别有任使仍遣稹入朝必厚加官爵|{
	供奉官亦宦者也}
上以澤潞事謀於宰相宰相多以為回鶻餘燼未滅邊境猶須警備復討澤潞|{
	復扶又翻}
國力不支請以劉稹權知軍事諫官及羣臣上言者亦然李德裕獨曰澤潞事體與河朔三鎮不同河朔習亂已久人心難化是故累朝以來置之度外澤潞近處心腹|{
	處昌呂翻}
一軍素稱忠義嘗破走朱滔擒盧從史|{
	走朱滔見二百三十一卷德宗貞元元年擒盧從史見二百三十八卷憲宗元和三年}
頃時多用儒臣為帥|{
	帥所類翻}
如李抱真成立此軍|{
	見二百二十三卷代宗永泰元年}
德宗猶不許承襲使李緘護喪歸東都|{
	見二百三十五卷貞元十年}
敬宗不恤國務宰相又無遠畧劉悟之死因循以授從諫從諫跋扈難制累上表廹脇朝廷|{
	事見文宗紀}
今垂死之際復以兵權擅付豎子朝廷若又因而授之則四方諸鎮誰不思效其所為天子威令不復行矣|{
	復扶又翻}
上曰卿以何術制之果可克否對曰稹所恃者河朔三鎮但得鎮魏不與之同則稹無能為也若遣重臣往諭王元逵何弘敬|{
	王元逵鎮帥何弘敬魏帥也}
以河朔自艱難以來列聖許其傳襲已成故事與澤潞不同今朝廷將加兵澤潞不欲更出禁軍至山東其山東三州隸昭義者委兩鎮攻之|{
	山東三州謂邢洺磁也}
兼令徧諭將士以賊平之日厚加官賞苟兩鎮聽命不從旁沮撓官軍|{
	沮在呂翻撓奴教翻又奴巧翻}
則稹必成擒矣上喜曰吾與德裕同之保無後悔遂決意討稹 |{
	考異曰按舊紀武宗實錄所載德裕之語皆出於伐叛記伐叛記繫于劉從諫始亡之時至此君臣誅討之意已決矣下百官議及宰臣再議皆備禮耳德裕之言當在事初實錄置此誤也}
羣臣言者不復入矣|{
	復扶又翻下同}
上命德裕草詔賜成德節度使王元逵魏博節度使何弘敬其畧曰澤潞一鎮與卿事體不同勿為子孫之謀欲存輔車之勢|{
	古語云輔車相依車尺遮翻}
但能顯立功效自然福及後昆丁丑上臨朝稱其語要切曰當如此直告之是也又賜張仲武詔以回鶻餘燼未滅塞上多虞專委卿禦侮|{
	以烏介可汗尚在黑車子也}
元逵弘敬得詔息聽命解朝政至上黨|{
	考異曰實錄云時從諫死二十日矣按姜崟等云自四月六日後不見本使而辛巳為從諫輟朝自六日至辛巳纔十八日耳實錄自相違今不取}
劉稹見朝政曰相公危困不任拜詔|{
	任音壬}
朝政欲突入兵馬使劉武德董可武躡簾而立朝政恐有他變遽走出稹贈賮直數千緡|{
	賮徐刃翻}
復遣牙將梁叔文入謝薛士幹入境俱不問從諫之疾直為已知其死之意都押牙郭誼等乃大出軍至龍泉驛迎候勑使請用河朔事體又見監軍言之崔士康懦怯不敢違於是將吏扶稹出見士衆喪士幹竟不得入牙門稹亦不受勑命誼兖州人也解朝政復命上怒杖之配恭陵囚姜崟梁叔文辛巳始為從諫輟朝|{
	為于偽翻}
贈太傅詔劉稹護喪歸東都又召見劉從素令以書諭稹|{
	令父以書諭其子也從素時在朝為右驍衛將軍見賢遍翻}
稹不從丁亥以忠武節度使王茂元為河陽節度使邠寧節度使王宰為忠武節度使茂元栖曜之子宰智興之子也|{
	王栖曜見二百三十卷德宗興元元年王智興始見二百二十七卷建中二年}
黄州刺史杜牧上李德裕書自言嘗問淮西將董重質以三州之衆四歲不破之由重質以為由朝廷徵兵太雜客軍數少既不能自成一軍事須帖付地主勢羸力弱心志不一多致敗亡故初戰二年戰則必勝是多殺客軍及二年已後客軍殫少止與陳許河陽全軍相搏|{
	陳許謂李光顔之兵河陽謂烏重胤之兵}
縱使唐州兵不能因虛取城|{
	唐州謂李愬之兵}
蔡州事力亦不支矣其時朝廷若使鄂州夀州唐州只保境不用進戰但用陳許鄭滑兩道全軍帖以宣潤弩手令其守隘即不出一歲無蔡州矣今者上黨之叛復與淮西不同|{
	復扶又翻}
淮西為寇僅五十歲其人味為寇之腴見為寇之利風俗益固氣燄已成自以為天下之兵莫與我敵根深源闊取之固難夫上黨則不然自安史南下不甚附隸|{
	肅宗時蔡希德攻上黨不能克}
建中之後每奮忠義是以郳公抱真能窘田悦走朱滔|{
	郳五稽翻李抱真封郳公窘田悦見二百二十七卷德宗建中二年三年}
常以孤窮寒苦之軍横折河朔彊梁之衆|{
	折之舌翻}
以此證驗人心忠赤習尚專一可以盡見劉悟卒從諫求繼與扶同者只鄆州隨來中軍二千耳|{
	扶同猶今俗言扶合也劉悟自鄆帥滑自滑徙潞鄆兵二千實從之唐末所謂元從也}
值寶歷多故因以授之今纔二十餘歲|{
	按寶歷元年以昭義節授劉從諫至是年纔十九年}
風俗未改故老尚存雖欲劫之必不用命今成德魏博雖盡節效順亦不過圍一城攻一堡係纍穉老而已|{
	纍倫追翻穉直二翻}
若使河陽萬人為壘窒天井之口|{
	天井關在澤州晉城縣南亦名太行關關南有天井泉三所故名杜牧此說欲杜潞人之南窺懹洛也}
高壁深塹勿與之戰只以忠武武寧兩軍|{
	忠武陳許兵武寧徐州兵}
帖以青州五千精甲宣潤二千弩手徑擣上黨不過數月必覆其巢穴矣時德裕制置澤潞亦頗采牧言 上雖外尊寵仇士良内實忌惡之|{
	惡烏路翻}
士良頗覺之遂以老病求散秩詔以左衛上將軍兼内侍監知省事|{
	知内侍省事}
李德裕言於上曰議者皆云劉悟有功|{
	劉悟以誅李師道為功}


稹未可亟誅宜全恩禮請下百官議|{
	下戶嫁翻}
以盡人情上曰悟亦何功當時廹於救死耳非素心徇國也藉使有功父子為將相二十餘年國家報之足矣稹何得復自立|{
	復扶又翻}
朕以為凡有功當顯賞有罪亦不可苟免也德裕曰陛下之言誠得理國之要 五月李德裕言太子賓客分司李宗閔與劉從諫交通不宜寘之東都戊戌以宗閔為湖州刺史|{
	史言李德裕修怨 考異曰獻替記云四月十九日上言東都李宗閔我聞比與從諫交通今澤潞事如何可别與一官不要令在東都德裕曰臣等續商量上又云不可與方鎮只與一遠郡德裕又奏云須與一郡此蓋德裕自以宿憾因劉稹事害宗閔畏人譏議故於獻替記載此語以隱其跡耳今從實錄}
河陽節度使王茂元以步騎三千守萬善|{
	九域志懷州河内縣有萬善鎮}
河東節度使劉沔以步騎二千守芒車關|{
	芒車關即昂車關魏收地形志上黨郡沾縣有昂車嶺其地當在唐儀州東南界石會關之西新唐志潞州武鄉縣北有昂車關}
步兵一千五百軍榆社|{
	九域志遼州遼山縣有榆社鎮唐之榆社縣也宋白曰榆社縣隋開皇十六年置今潞州襄垣縣理是也因今縣西北榆社故城為名}
成德節度使王元逵以步騎三千守臨洺掠堯山|{
	堯山本柏人縣天寶元年更名屬邢州宋白曰以唐堯大麓之地名之洺音名}
河中節度使陳夷行以步騎一千守翼城步兵五百益冀氏|{
	冀氏本漢猗氏縣地後魏於古猗氏縣城南置冀氏郡及冀氏縣隋廢郡存縣唐屬晉州九域志在州東北二百八十里}
辛丑制削奪劉從諫及子稹官爵以元逵為澤潞北面招討使何弘敬為南面招討使與夷行劉沔茂元合力攻討先是河朔諸鎮有自立者|{
	先悉薦翻}
朝廷必先有弔祭使次冊贈使宣慰使繼往商度軍情|{
	度徒洛翻}
必不可與節則别除一官俟軍中不聽出然後始用兵故常及半歲軍中得繕完為備至是宰相亦欲且遣使開諭上即命下詔討之 |{
	考異曰獻替記云五月十一日德裕疾病先請假在宅李相紳其日亦請假李相讓夷獨對上便決攻討之意李相歸中書後錄聖意四紙令德裕草制至薄晚封進明日遂降麻處分舊本紀下制討稹今從實錄}
王元逵受詔之日出師屯趙州|{
	九域志鎮州南至趙州九十五里}
壬寅以翰林學士承旨崔鉉為中書侍郎同平章事|{
	翰林學士第一廳為承旨廳以翰林學士久次者為之 考異曰實錄李讓夷引鉉為相今從實錄}
鉉元畧之子也|{
	崔元畧見二百二十三卷敬宗寶歷元年}
上夜召學士韋琮以鉉名授之令草制宰相樞密皆不之知時樞密使劉行深楊欽義皆愿慤不敢預事老宦者尤之曰此由劉楊懦怯墮敗舊風故也|{
	墮讀曰隳敗補邁翻}
琮乾度之子也|{
	韋乾度憲宗朝為吏部郎中}
以武寧節度使李彦佐為晉絳行營諸軍節度招討使 劉沔自代州還太原|{
	以回鶻已破走也}
築望仙觀於禁中|{
	會要是年修望仙樓及廊舍共五百三十九間觀古玩翻}
六月王茂元遣兵馬使馬繼等將步騎二千軍於天井關南科斗店劉稹遣衙内十將薛茂卿將親軍二千拒之 戛斯可汗遣將軍温仵合入貢|{
	仵音午}
上賜之書諭以速平回鶻黑車子乃遣使行冊命 癸酉仇士良以左衛上將軍内侍監致仕其黨送歸私第士良教以固權寵之術曰天子不可令閒常宜以奢靡娯其耳目使日新月盛無暇更及他事然後吾輩可以得志慎勿使之讀書親近儒生|{
	近其靳翻}
彼見前代興亡心知憂懼則吾輩疎斥矣其黨拜謝而去|{
	觀仇士良之教其黨則閹寺豈可親近哉}
丙子詔王元逵李彦佐劉沔王茂元何弘敬以七月中旬五道齊進劉稹求降皆不得受又詔劉沔自將兵取仰車關路以臨賊境|{
	仰車關即昂車關}
吐蕃鄯州節度使尚婢婢世為吐蕃相婢婢好讀書

不樂仕進|{
	好呼到翻樂音洛}
國人敬之年四十餘彛泰贊普彊起之使鎮鄯州|{
	彛泰達磨之兄文宗開成三年卒彊其兩翻}
婢婢寛厚沈勇有謀略|{
	沈持林翻}
訓練士卒多精勇論恐熱雖名義兵實謀簒國|{
	論恐熱起兵事始上卷二年}
忌婢婢恐襲其後欲先滅之是月大舉兵擊婢婢旌旗雜畜千里不絶至鎮西|{
	鎮西軍在河州西一百八十里畜許救翻}
大風震電天火燒殺禆將十餘人雜畜以百數恐熱惡之|{
	惡烏路翻}
盤桓不進婢婢謂其下曰恐熱之來視我如螻蟻以為不足屠也今遇天災猶豫不進吾不如迎伏以却之使其志益驕而不為備然後可圖也乃遣使以金帛牛酒犒師且致書言相公舉義兵以匡國難|{
	難乃旦翻}
闔境之内孰不向風苟遣一介賜之折簡敢不承命何必遠辱士衆親臨下藩婢婢資性愚僻惟嗜讀書先贊普授以藩維誠為非據夙夜慙惕惟求退居相公若賜以骸骨聽歸田里乃愜平生之素願也|{
	愜詰叶翻}
恐熱得書喜徧示諸將曰婢婢惟把書卷安知用兵待吾得國當位以宰相坐之於家亦無所用也乃復為書勤厚答之引兵歸婢婢聞之撫髀笑曰我國無主則歸大唐豈能事此犬鼠乎 秋七月以山南東道節度使盧鈞為昭義節度招撫使朝廷以鈞在襄陽寛厚有惠政得衆心故使領昭義以招懷之 上遣刑部侍郎兼御史中丞李回宣慰河北三鎮令幽州乘秋早平回鶻鎮魏早平澤潞回太祖之八世孫也|{
	太祖第六子禕生德良六世至回}
甲辰李德裕言於上曰臣見曏日河朔用兵諸道利於出境仰給度支|{
	仰牛向翻}
或隂與賊通借一縣一柵據之自以為功坐食轉輸|{
	輸舂遇翻}
延引歲時今請賜諸軍詔旨令王元逵取邢州何弘敬取洺州王茂元取澤州李彦佐劉沔取潞州毋得取縣上從之晉絳行營節度使李彦佐自徐州行甚緩又請休兵於絳州兼請益兵李德裕言於上曰彦佐逗遛顧望殊無討賊之意所請皆不可許宜賜詔切責令進軍翼城|{
	九域志翼城縣在絳州東北二百里宋白曰翼城本漢絳縣地後魏明帝置北絳縣於曲沃縣東隋改為翼城縣因縣東古翼城而名}
上從之德裕因請以天德防禦使石雄為彦佐之副俟至軍中令代之乙巳以雄為晉絳行營節度副使仍詔彦佐進屯翼城劉稹上表自陳亡父從諫為李訓雪寃言仇士良罪惡|{
	事見二百四十五卷文宗開成五年為于偽翻}
由此為權倖所疾謂臣父潛懷異志臣所以不敢舉族歸朝乞陛下稍垂寛察活臣一方何弘敬亦為之奏雪|{
	為于偽翻}
皆不報李回至河朔何弘敬王元逵張仲武皆具櫜鞬郊迎|{
	櫜姑勞翻鞬居言翻}
立於道左不敢令人控馬讓制使先行|{
	曰制使以别宦官之勑使}
自兵興以來未之有也|{
	兵興以來謂天寶之後}
回明辯有膽氣三鎮無不奉詔 王元逵奏拔宣務柵|{
	宣務柵當在堯山縣東北}
擊堯山劉稹遣兵救堯山元逵擊敗之|{
	敗補邁翻}
詔切責李彦佐劉沔王茂元使速進兵逼賊境且稱元逵之功以激厲之加元逵同平章事八月乙丑昭義大將李丕來降議者或謂賊故遣丕降欲以疑誤官軍李德裕言於上曰自用兵半年未有降者今安問誠之與詐且須厚賞以勸將來但不可置之要地耳 上從容言文宗好聽外議諫官言事多不著名|{
	從干容翻好呼到翻著陟略翻}
有如匿名書李德裕曰臣頃在中書文宗猶不爾|{
	德裕謂太和間已為相時文宗猶不如此}
此乃李訓鄭注教文宗以術御下遂成此風人主但當推誠任人有欺罔者威以明刑孰敢哉上善之 王元逵前鋒入邢州境已踰月|{
	九域志趙州南至邢州境七十四里}
何弘敬猶未出師元逵屢有密表稱弘敬懷兩端丁卯李德裕上言忠武累戰有功軍聲頗振王宰年力方壯謀略可稱|{
	自曲環李光顔以來忠武軍屢立戰功王宰智興之子於當時諸帥蓋少年中之翹楚者}
請賜弘敬詔以河陽河東皆閡山險未能進軍|{
	河陽閡太行之險河東閡石會昂車之險閡牛代翻}
賊屢出兵焚掠晉絳今遣王宰將忠武全軍徑魏博直抵磁州以分賊勢弘敬必懼此攻心伐謀之術也從之詔宰悉選步騎精兵自相魏趣磁州|{
	趣七喻翻下同磁疾之翻相州東至魏州百八十里北至磁州六十里}
甲戌薛茂卿破科斗寨擒河陽大將馬繼等焚掠小寨一十七距懷州纔十餘里茂卿以無劉稹之命故不敢入|{
	言不敢入懷州}
時議者鼎沸以為劉悟有功不可絶其嗣又從諫養精兵十萬糧支十年如何可取上亦疑之以問李德裕對曰小小進退兵家之常願陛下勿聽外議則成功必矣上乃謂宰相曰為我語朝士|{
	為于偽翻語牛倨翻朝直遥翻}
有上疏沮議者我必於賊境上斬之議者乃止|{
	沮在呂翻}
何弘敬聞王宰將至恐忠武兵入魏境軍中有變蒼黄出師丙子弘敬奏已自將全軍度漳水趣磁州庚辰李德裕上言河陽兵力寡弱自科斗店之敗賊勢愈熾王茂元復有疾|{
	復扶又翻}
人情危怯欲退保懷州臣竊見元和以來諸賊常視官軍寡弱之處併力攻之一軍不支然後更攻他處今魏博未與賊戰西軍閡險不進|{
	西軍謂河東晉絳兵也}
故賊得併兵南下|{
	自太行南趨懷州謂之下}
若河陽退縮不惟虧沮軍聲兼恐震驚洛師|{
	東都謂之洛師書洛誥曰朝至于洛師}
望詔王宰更不之磁州|{
	魏博既出師攻磁州故請詔王宰移軍之往也}
亟以忠武軍應援河陽不惟扞蔽東都兼可臨制魏博若令全軍供餉難給且令先鋒五千人赴河陽亦足張聲勢|{
	張知亮翻}
甲申又奏請勑王宰以全軍繼進仍急以器械繒帛助河陽窘乏上皆從之|{
	繒慈陵翻}
王茂元軍萬善劉稹遣牙將張巨劉公直等會薛茂卿共攻之期以九月朔圍萬善乙酉公直等潛師先過萬善南五里焚雍店巨引兵繼之過萬善覘知城中守備單弱|{
	覘丑廉翻}
欲專有功遂攻之日昃城且拔乃使人告公直等時義成軍適至|{
	時以河陽兵寡令王宰以忠武軍合義成兵援之義成軍滑州兵}
茂元困急欲帥衆棄城走|{
	帥讀曰率}
都虞候孟章諫曰賊衆自有前却半在雍店半在此乃亂兵耳今義成軍纔至尚未食聞僕射走則自潰矣願且強留|{
	強其兩翻}
茂元乃止會日暮公直等不至巨引兵退始登山|{
	登太行阪也}
微雨晦黑自相驚曰追兵近矣皆走人馬相踐墜崖谷死者甚衆|{
	踐慈演翻}
上以王茂元王宰兩節度使共處河陽非宜|{
	處昌呂翻}
庚寅李德裕等奏茂元習吏事而非將才|{
	將即亮翻}
請以宰為河陽行營攻討使茂元病愈止令鎮河陽病困亦免他虞九月辛卯以宰兼河陽行營攻討使 何弘敬奏拔肥鄉平恩|{
	肥鄉漢邯鄲縣地曹魏置肥鄉縣至唐與平恩皆屬洺州九域志肥鄉在州東三十五里平恩在州東九十里}
殺傷甚衆得劉稹牓帖皆謂官軍為賊云遇之即須痛殺癸巳上謂宰相何弘敬已克兩縣可釋前疑|{
	謂王元逵密奏弘敬持兩端也}
既有殺傷雖欲持兩端不可得已乃加弘敬檢校左僕射 丙午河陽奏王茂元薨李德裕奏王宰止可令以忠武節度使將萬善營兵不可使兼領河陽恐其不愛河陽州縣恣為侵擾又河陽節度先領懷州刺史常以判官攝事割河南五縣租賦隸河陽|{
	見二百二十七卷德宗建中元年}
不若遂置孟州|{
	治置孟州因孟津為名也}
其懷州别置刺史俟昭義平日仍割澤州隸河陽節度則太行之險不在昭義而河陽遂為重鎮東都無復憂矣上采其言戊申以河南尹敬昕為河陽節度懷孟觀察使王宰將行營以扞敵昕供饋餉而已|{
	昕許斤翻}
庚戌以石雄代李彦佐為晉絳行營節度使 |{
	考異曰實錄召彦佐入奉朝請俟罷兵日赴鎮按彦佐前已罷武寧今又罷晉絳復赴何鎮實錄誤也}
令自冀氏取潞州仍分兵屯翼城以備侵軼|{
	軼徒結翻突也}
是月吐蕃論恐熱屯大夏川|{
	大夏川在河州大夏縣西有大夏水漢古縣也夏戶雅翻}
尚婢婢遣其將厖結心及莽羅薛呂將精兵五萬擊之至河州南莽羅薛呂伏兵四萬於險阻厖結心伏萬人於柳林中以千騎登山飛矢繫書罵之恐熱怒將兵數萬追之厖結心陽敗走時為馬乏不進之狀恐熱追之益急不覺行數十里伏兵斷其歸路|{
	斷音短}
夾擊之會大風飛沙溪谷皆溢恐熱大敗伏尸五十里溺死者不可勝數|{
	勝音升}
恐熱單騎遁歸 石雄代李彦佐之明日即引兵踰烏嶺|{
	五代志翼城縣有烏嶺山}
破五寨殺獲千計時王宰軍萬善劉沔軍石會皆顧望未進上得雄捷書喜甚冬十月庚申臨朝謂宰相曰雄真良將 |{
	考異曰獻替伐叛記皆云十月五日上言石雄破賊而實録己巳奏到庚午對宰臣言乃是十五日恐誤}
李德裕因言比年前潞州市有男子磬折唱曰|{
	比毗至翻磬折言曲折其身如磬之形折之舌翻}
石雄七千人至矣劉從諫以為妖言斬之|{
	妖於驕翻}
破潞州者必雄也詔賜雄帛為優賞雄悉置軍門自依士卒例先取一匹餘悉分將士故士卒樂為之致死|{
	樂音洛為于偽翻}
初劉沔破回鶻得太和公主|{
	見上會昌三年}
張仲武疾之由是有隙上使李回至幽州和解之仲武意終不平朝廷恐其以私憾敗事|{
	敗補邁翻}
辛未徙沔為義成節度使以前荆南節度使李石為河東節度使党項寇鹽州以前武寧節度使李彦佐為朔方靈鹽節度使十一月邠寧奏党項入寇李德裕奏党項愈熾不可不為區處|{
	處昌呂翻}
聞党項分隸諸鎮|{
	綏銀靈鹽夏邠寧延麟勝慶等州皆有党項諸鎮分領之}
剽掠於此則亡逃歸彼|{
	剽匹妙翻}
節度使各利其駝馬不為擒送|{
	為于偽翻}
以此無由禁戢臣屢奏不若使一鎮統之陛下以為一鎮專領党項權太重臣今請以皇子兼統諸道擇中朝廉幹之臣為之副居於夏州理其辭訟庶為得宜乃以兖王岐為靈夏等六道元帥|{
	岐皇子也夏戶雅翻}
兼安撫党項大使又以御史中丞李回為安撫党項副使史館修撰鄭亞為元帥判官令齎詔往安撫党項及六鎮百姓|{
	六鎮鹽州夏州靈武涇原及振武邠寧也}
安南經略使武渾役將士治城|{
	治直之翻}
將士作亂燒城樓劫府庫渾奔廣州監軍段士則撫安亂衆 忠武軍素號精勇王宰治軍嚴整昭義人甚憚之薛茂卿以科斗寨之功意望超遷或謂劉稹曰留後所求者節耳茂卿太深入多殺官軍激怒朝廷此節所以來益遲也由是無賞茂卿愠懟|{
	愠於問翻懟直類翻}
密與王宰通謀十二月丁巳宰引兵攻天井關茂卿小戰遽引兵走宰遂克天井關守之關東西寨聞茂卿不守皆退走宰遂焚大小箕村茂卿入澤州密使諜召宰進攻澤州當為内應宰疑不敢進失期不至茂卿拊膺頓足而已稹知之誘茂卿至潞州殺之并其族|{
	誘音酉}
以兵馬使劉公直代茂卿安全慶守烏嶺李佐堯守彫黄嶺|{
	彫黄嶺在潞州長子縣西}
郭僚守石會康良佺守武鄉|{
	武鄉漢垣縣後魏改曰鄉縣移治於南亭川武后加武字屬潞州}
僚誼之姪也戊辰王宰進攻澤州 |{
	考異曰一品集十月二十三日狀緣王宰兵已深入須取澤州按此月三日宰始得天井關於十月之末豈能深入取澤州蓋十二月十三日狀二字誤在月下耳}
與劉公直戰不利公直乘勝復天井關甲戌宰進擊公直大破之遂圍陵川克之|{
	陵川漢泫氏縣地隋開皇十六年置陵川縣唐屬澤州九域志在州東北一百五里}
河東奏克石會關洺州刺史李恬石之從兄也石至太原劉稹遣軍將賈羣詣石以恬書與石云稹願舉族歸命相公奉從諫喪歸葬東都石囚羣以其書聞李德裕上言今官軍四合捷書日至賊勢窮蹙故偽輸誠款冀以緩師稍得自完復來侵軼|{
	軼徒結翻}
望詔石答恬書云前書未敢聞奏若郎君誠能悔過舉族面縛待罪境上則石當親往受降護送歸闕若虛為誠款先求解兵次望洗雪則石必不敢以百口保人 |{
	考異曰一品集正月四日狀曰臣等得李石狀報劉稹潛有款誠云云又曰今饋運之費計至春末並足如二月已來尚未殄滅然議納降亦未為晩又草詔賜石曰必不得因此遷延令其得計仍不得先受章表便與奏聞按實録上貶崔碣仍詔敢言罷兵者送賊境戮之德裕狀正月四日上然石奏必在楊弁未亂前故置於此}
仍望詔諸道乘其上下離心速進兵攻討不過旬朔必内自生變上從之右拾遺崔碣上疏請受其降|{
	碣渠列翻}
上怒貶碣鄧城令 初劉沔破回鶻留兵三千戍横水柵河東行營都知兵馬使王逢奏乞益榆社兵|{
	王逢時以河東兵屯榆社}
詔河東以兵二千赴之時河東無兵守倉庫者及工匠皆出從軍李石召横水戍卒千五百人使都將楊弁將之詣逢壬午戍卒至太原先是軍士出征人給絹二匹|{
	先悉薦翻}
劉沔之去竭府庫自隨石初至軍用乏以已絹益之人纔得一匹時已歲盡軍士求過正旦而行監軍呂義忠累牒趣之|{
	趣讀曰促}
楊弁因衆心之怒又知城中空虛遂作亂

四年春正月乙酉朔楊弁帥其衆剽掠城市殺都頭梁季叶|{
	帥讀曰率}
李石奔汾州|{
	太原府西南至汾州一百餘里}
弁據軍府釋賈羣之囚使其姪與之俱詣劉稹約為兄弟稹大喜石會關守將楊珍聞太原亂復以關降於稹戊子呂義忠遣使言狀朝議喧然或言兩地皆應罷兵|{
	兩地謂并潞也}
王宰又上言遊奕將得劉稹表|{
	將即亮翻}
臣近遣人至澤潞賊有意歸附若許招納乞降詔命李德裕上言宰擅受稹表遣人入賊中曾不聞奏觀宰意似欲擅招撫之功昔韓信破田榮|{
	榮當作横事見十卷漢高祖三年四年}
李靖擒頡利|{
	見一百九十三卷太宗貞觀四年}
皆因其請降潛兵掩襲止可令王宰失信豈得損朝廷威命建立奇功實在今日必不可以太原小擾失此事機望即遣供奉官至行營督其進兵掩其無備必須劉稹與諸將皆舉族面縛方可受納 |{
	考異曰一品集奏狀云如劉稹自來却令送入輒不得受按稹若自來豈有却送入之理恐是稹下脫不字}
兼遣供奉官至晉絳行營密諭石雄以王宰若納劉稹則雄無功可紀雄於垂成之際須自取奇功勿失此便又為相府與宰書言昔王承宗雖逆命猶遣弟承恭奉表詣張相祈哀又遣其子知感知信入朝憲宗猶未之許|{
	見二百四十卷元和十三年}
今劉稹不詣尚書面縛又不遣血屬祈哀|{
	血屬謂父子兄弟至親同出於一氣者}
置章表於衢路之間遊奕將不即毁除實恐非是况稹與楊弁通姦逆狀如此而將帥大臣容受其詐是私惠歸於臣下不赦在於朝廷事體之間交恐不可自今更有章表宜即所在焚之惟面縛而來始可容受德裕又上言太原人心從來忠順止是貧虛賞犒不足况千五百人何能為事必不可姑息寛縱且用兵未罷深慮所在動心頃張延賞為張朏所逐逃奔漢州還入成都|{
	事見德宗紀朏敷尾翻}
望詔李石義忠還赴太原行營召旁近之兵討除亂者上皆從之是時李石已至晉州詔復還太原辛卯詔王逢悉留太原兵守榆社以易定千騎宣武兖海步兵三千討楊弁又詔王元逵以步騎五千自土門入應接逢軍 |{
	考異曰實錄詔側近行營量抽兵翦撲又詔王元逵以兵五千扼土門張仲武把鴈門以為聲援今從伐叛記}
忻州刺史李丕奏楊弁遣人來為遊說|{
	說式芮翻}
臣已斬之兼斷其北出之路|{
	斷音短恐楊弁之軍北出扇動雜虜與回鶻餘衆合故斷其路}
兵討之辛丑上與宰相議太原事李德裕曰今太原兵皆在外為亂者止千餘人諸州鎮必無應者計不日誅翦惟應速詔王逢進軍至城下必自有變上曰仲武見鎮魏討澤潞有功必有慕羨之心使之討太原何如德裕對曰鎮州趣太原路最便近|{
	九域志鎮州西至太原府四百三十里武宗之意蓋欲使張仲武出兵道鎮州趣太原耳趣七喻翻}
仲武去年討回鶻與太原爭功恐其不戢士卒平人受害乃止上遣中使馬元實至太原曉諭亂兵且覘其彊弱楊弁與之酣飲三日且賂之戊申元實自太原還上遣詣宰相議之元實於衆中大言相公須早與之節李德裕曰何故元實曰自牙門至柳子列十五里曳地光明甲|{
	柳子列因其地列植柳樹而名}
若之何取之德裕曰李相正以太原無兵|{
	李石舊相也故呼為李相}
故横水兵赴榆社庫中之甲盡在行營弁何能遽致如此之衆乎元實曰太原人勁悍皆可為兵弁召募所致耳德裕曰召募須有貨財李相止以欠軍士絹一匹無從可得故致此亂弁何從得之元實辭屈德裕曰縱其有十五里光明甲必須殺此賊因奏稱楊弁微賊決不可恕|{
	以其起於卒伍而逐節帥也}
如國力不及寧捨劉稹|{
	當時君相志叶議從劉稹勢已窮蹙必不肯捨之而不討德裕此言蓋深激武宗以明楊弁之決不可恕耳}
河東兵戍榆社者聞朝廷令客軍取太原恐妻孥為所屠滅乃擁監軍呂義忠自取太原壬子克之生擒楊弁盡誅亂卒 三月甲寅朔日有食之 乙卯呂義忠奏克太原丙辰李德裕言於上曰王宰久應取澤州今已遷延兩月蓋宰與石雄素不相叶|{
	王宰父智興奏石雄罪流白州故不叶}
今得澤州距上黨猶二百里而石雄所屯距上黨纔百五十里宰恐攻澤州綴昭義大軍而雄得乘虛入上黨獨有其功耳又宰生子晏實其父智興愛而子之晏實今為磁州刺史為劉稹所質|{
	質音致}
宰之顧望不敢進或為此也|{
	為于偽翻}
上命德裕草詔賜宰督其進兵且曰朕顧兹小寇終不貸刑亦知晏實是卿愛弟將申大義在抑私懷 丁巳以李石為太子少傅分司以河中節度使崔元式為河東節度使石雄為河中節度使元式元略之弟也|{
	元略時宰崔鉉之父}
己未石雄拔良馬等三寨一堡|{
	初退渾李萬江歸李抱玉於潞州牧津梁寺地美水草馬如鴨而健世謂之津梁種良馬寨蓋置於其地}
辛酉太原獻楊弁及其黨五十四人皆斬於狗脊嶺|{
	按宋白續通典狗脊嶺在京城東市}
壬申李德裕言于上曰事固有激而成功者陛下命王宰趣磁州|{
	趣七喻翻}
而何弘敬出師遣客軍討太原而戍兵先取楊弁今王宰久不進軍請徙劉沔鎮河陽仍令以義成精兵二千直抵萬善處宰肘腋之下|{
	處昌呂翻}
若宰識朝廷此意必不敢淹留若宰進軍沔以重兵在南聲勢亦壯上曰善戊寅以義成節度使劉沔為河陽節度使 王逢擊昭義將康良佺敗之|{
	敗補邁翻}
良佺棄石會關退屯鼓腰嶺|{
	佺丑緣翻鼓腰嶺當在潞州武鄉縣北 考異曰實録王宰奏賊將康良佺敗棄石會關移軍入三十里守鼓腰嶺按石會關在潞州北與河東接宰時在澤州南何以得敗良佺蓋逢字誤為宰耳}
戛斯遣將軍諦德伊斯難珠等入貢|{
	諦音帝}
言欲徙居回鶻牙帳請兵之期集會之地上賜詔諭以今秋可汗擊回鶻黑車子之時當令幽州太原振武天德四鎮出兵要路邀其亡逸便申冊命並依回鶻故事 朝廷以回鶻衰微吐蕃内亂議復河湟四鎮十八州|{
	開元之盛隴右河西分為兩鎮而已蓋淪陷之後吐蕃分為四鎮也十八州秦源河渭蘭鄯階成洮岷臨廓疊宕甘涼瓜沙也}
乃以給事中劉濛為巡邊使 |{
	考異曰實錄以濛為巡邊使在明年二月壬寅壬寅二十五日也按一品集會昌四年二月二十二日奏狀曰緣李回等稱戛斯使云今冬必欲就黑車子收回鶻可汗餘燼切望國家兵馬應接戛斯使回日已賜敕書許令幽州太原振武天德各於要路出兵邀截又曰仍令代北諸軍摐摐排比又曰其幽州兵馬至多不必先令排比待至冬初續降中使賜詔戛斯使來在四年二月德裕奏狀所謂今冬防秋冬初者皆四年事也不容至五年二月始以濛為巡邊使濛之奉使要在今年春夏不知的何月日且附於此}
使之先備器械糗糧及詗吐蕃守兵衆寡|{
	糗去久翻詗翾正翻又火迥翻}
又令天德振武河東訓卒礪兵以俟今秋戛斯擊回鶻邀其潰敗之衆南來者皆委濛與節度團練使詳議以聞濛晏之孫也|{
	劉晏以讒死于建中之初}
以道士趙歸真為右街道門教授先生 吐蕃論恐熱之將岌藏豐贊惡恐熱殘忍降於尚婢婢|{
	惡烏路翻降戶江翻}
恐熱兵擊婢婢於鄯州婢婢分兵為五道拒之恐熱退保東谷|{
	九域志河州東南一十五里有東谷堡宋熙寧七年置}
婢婢為木柵圍之絶其水原恐熱將百餘騎突圍走保薄寒山餘衆皆降於婢婢夏四月王宰進攻澤州 上好神仙|{
	好呼到翻}
道士趙歸真得幸諫官屢以為言丙子李德裕亦諫曰歸真敬宗朝罪人|{
	見二百四十三卷寶歷二年}
不宜親近|{
	近其靳翻}
上曰朕宮中無事時與之談道滌煩耳至於政事朕必問卿等與次對官雖百歸真不能惑也德裕曰小人見勢利所在則奔趣之如夜蛾之投燭聞旬日以來歸真之門車馬輻凑願陛下深戒之戊寅以左僕射王起同平章事充山南西道節度使

起以文臣未嘗執政直除使相前無此比固辭|{
	唐中世以後節度使同平章事者則謂之使相比毗至翻例也}
上曰宰相無内外之異朕有闕失卿飛表以聞 李德裕以州縣佐官太冗奏令吏部郎中柳仲郢裁減六月仲郢奏減一千二百一十四員|{
	考異曰獻替記云減得二千二員新傳曰罷二千餘員舊柳仲郢傳曰減一千二百員今從之}
仲郢

公綽之子也|{
	柳公綽事憲穆歷方鎮京尹有聲績}
宦官有仇士良宿惡於其家得兵仗數千詔削其官爵籍沒家貲 秋七月辛卯上與李德裕議以王逢將兵屯翼城上曰聞逢用法太嚴有諸對曰臣亦嘗以此詰之逢言前有白刃法不嚴其誰肯進上曰言亦有理卿更召而戒之德裕因言劉稹不可赦上曰固然德裕曰昔李懷光未平京師蝗旱米斗千錢太倉米供天子及六宫無數旬之儲德宗集百官遣中使馬欽緒詢之左散騎常侍李泌取桐葉破以授欽緒獻之德宗召問其故對曰陛下與懷光君臣之分如此葉不可復合矣|{
	分扶問翻或讀如字復扶又翻}
由是德宗意定既破懷光遂用為相獨任數年|{
	見德宗紀}
上曰亦大是奇士|{
	李泌相業卓有可稱觀此則可以傳信唐人毁之者皆妄也}
上聞揚州倡女善為酒令|{
	倡音昌酒令者行令而行酒也唐人多好為之却掃編日皇甫松著醉郷日月載骰子令又有旗旛令閃擪令抛打令今人不復曉其法惟優伶家猶用手打令以為戲云}
勑淮南監軍選十七人獻之監軍請節度使杜悰同選且欲更擇良家美女教而獻之悰曰監軍自受勑悰不敢預聞監軍再三請之不從監軍怒具表其狀上覽表默然左右請并勑節度使同選上曰勑藩方選倡女入宫豈聖天子所為杜悰不徇監軍意得大臣體真宰相才也朕甚愧之遽勑監軍勿復選甲辰以悰同平章事 |{
	考異曰新表悰入相在閏月壬戌今從實錄}
兼度支鹽鐵轉運使及悰中謝|{
	既受命入謝謂之中謝}
上勞之曰|{
	勞力到翻}
卿不從監軍之言朕知卿有致君之心今相卿如得一魏徵矣|{
	武宗之期望杜悰者如此然悰在相位其所論諫史無稱焉}


資治通鑑卷二百四十七
