<!DOCTYPE html PUBLIC "-//W3C//DTD XHTML 1.0 Transitional//EN" "http://www.w3.org/TR/xhtml1/DTD/xhtml1-transitional.dtd">
<html xmlns="http://www.w3.org/1999/xhtml">
<head>
<meta http-equiv="Content-Type" content="text/html; charset=utf-8" />
<meta http-equiv="X-UA-Compatible" content="IE=Edge,chrome=1">
<title>資治通鑒_202-資治通鑑卷二百一_202-資治通鑑卷二百一</title>
<meta name="Keywords" content="資治通鑒_202-資治通鑑卷二百一_202-資治通鑑卷二百一">
<meta name="Description" content="資治通鑒_202-資治通鑑卷二百一_202-資治通鑑卷二百一">
<meta http-equiv="Cache-Control" content="no-transform" />
<meta http-equiv="Cache-Control" content="no-siteapp" />
<link href="/img/style.css" rel="stylesheet" type="text/css" />
<script src="/img/m.js?2020"></script> 
</head>
<body>
 <div class="ClassNavi">
<a  href="/24shi/">二十四史</a> | <a href="/SiKuQuanShu/">四库全书</a> | <a href="http://www.guoxuedashi.com/gjtsjc/"><font  color="#FF0000">古今图书集成</font></a> | <a href="/renwu/">历史人物</a> | <a href="/ShuoWenJieZi/"><font  color="#FF0000">说文解字</a></font> | <a href="/chengyu/">成语词典</a> | <a  target="_blank"  href="http://www.guoxuedashi.com/jgwhj/"><font  color="#FF0000">甲骨文合集</font></a> | <a href="/yzjwjc/"><font  color="#FF0000">殷周金文集成</font></a> | <a href="/xiangxingzi/"><font color="#0000FF">象形字典</font></a> | <a href="/13jing/"><font  color="#FF0000">十三经索引</font></a> | <a href="/zixing/"><font  color="#FF0000">字体转换器</font></a> | <a href="/zidian/xz/"><font color="#0000FF">篆书识别</font></a> | <a href="/jinfanyi/">近义反义词</a> | <a href="/duilian/">对联大全</a> | <a href="/jiapu/"><font  color="#0000FF">家谱族谱查询</font></a> | <a href="http://www.guoxuemi.com/hafo/" target="_blank" ><font color="#FF0000">哈佛古籍</font></a> 
</div>

 <!-- 头部导航开始 -->
<div class="w1180 head clearfix">
  <div class="head_logo l"><a title="国学大师官网" href="http://www.guoxuedashi.com" target="_blank"></a></div>
  <div class="head_sr l">
  <div id="head1">
  
  <a href="http://www.guoxuedashi.com/zidian/bujian/" target="_blank" ><img src="http://www.guoxuedashi.com/img/top1.gif" width="88" height="60" border="0" title="部件查字,支持20万汉字"></a>


<a href="http://www.guoxuedashi.com/help/yingpan.php" target="_blank"><img src="http://www.guoxuedashi.com/img/top230.gif" width="600" height="62" border="0" ></a>


  </div>
  <div id="head3"><a href="javascript:" onClick="javascript:window.external.AddFavorite(window.location.href,document.title);">添加收藏</a>
  <br><a href="/help/setie.php">搜索引擎</a>
  <br><a href="/help/zanzhu.php">赞助本站</a></div>
  <div id="head2">
 <a href="http://www.guoxuemi.com/" target="_blank"><img src="http://www.guoxuedashi.com/img/guoxuemi.gif" width="95" height="62" border="0" style="margin-left:2px;" title="国学迷"></a>
  

  </div>
</div>
  <div class="clear"></div>
  <div class="head_nav">
  <p><a href="/">首页</a> | <a href="/ShuKu/">国学书库</a> | <a href="/guji/">影印古籍</a> | <a href="/shici/">诗词宝典</a> | <a   href="/SiKuQuanShu/gxjx.php">精选</a> <b>|</b> <a href="/zidian/">汉语字典</a> | <a href="/hydcd/">汉语词典</a> | <a href="http://www.guoxuedashi.com/zidian/bujian/"><font  color="#CC0066">部件查字</font></a> | <a href="http://www.sfds.cn/"><font  color="#CC0066">书法大师</font></a> | <a href="/jgwhj/">甲骨文</a> <b>|</b> <a href="/b/4/"><font  color="#CC0066">解密</font></a> | <a href="/renwu/">历史人物</a> | <a href="/diangu/">历史典故</a> | <a href="/xingshi/">姓氏</a> | <a href="/minzu/">民族</a> <b>|</b> <a href="/mz/"><font  color="#CC0066">世界名著</font></a> | <a href="/download/">软件下载</a>
</p>
<p><a href="/b/"><font  color="#CC0066">历史</font></a> | <a href="http://skqs.guoxuedashi.com/" target="_blank">四库全书</a> |  <a href="http://www.guoxuedashi.com/search/" target="_blank"><font  color="#CC0066">全文检索</font></a> | <a href="http://www.guoxuedashi.com/shumu/">古籍书目</a> | <a   href="/24shi/">正史</a> <b>|</b> <a href="/chengyu/">成语词典</a> | <a href="/kangxi/" title="康熙字典">康熙字典</a> | <a href="/ShuoWenJieZi/">说文解字</a> | <a href="/zixing/yanbian/">字形演变</a> | <a href="/yzjwjc/">金 文</a> <b>|</b>  <a href="/shijian/nian-hao/">年号</a> | <a href="/diming/">历史地名</a> | <a href="/shijian/">历史事件</a> | <a href="/guanzhi/">官职</a> | <a href="/lishi/">知识</a> <b>|</b> <a href="/zhongyi/">中医中药</a> | <a href="http://www.guoxuedashi.com/forum/">留言反馈</a>
</p>
  </div>
</div>
<!-- 头部导航END --> 
<!-- 内容区开始 --> 
<div class="w1180 clearfix">
  <div class="info l">
   
<div class="clearfix" style="background:#f5faff;">
<script src='http://www.guoxuedashi.com/img/headersou.js'></script>

</div>
  <div class="info_tree"><a href="http://www.guoxuedashi.com">首页</a> > <a href="/SiKuQuanShu/fanti/">四库全书</a>
 > <h1>资治通鉴</h1> <!--         下载:【右键另存为】即可 --></div>
  <div class="info_content zj clearfix">
  
<div class="info_txt clearfix" id="show">
<center style="font-size:24px;">202-資治通鑑卷二百一</center>
    資治通鑑卷二百一   宋 司馬光 撰<br />
<br />
  胡三省 音注<br />
<br />
  唐紀十七【起玄黓閹茂八月盡上章敦牂凡八年有奇】<br />
<br />
  高宗天皇大聖大弘孝皇帝中之上<br />
<br />
  龍朔二年八月壬寅以許敬宗爲太子少師同東西臺三品知西臺事【同中書門下三品知中書事】 九月戊寅初令八品九品衣碧【衣於既翻】 冬十月丁酉上幸驪山温湯太子監國【監古衘翻】丁未還宫 庚戌西臺侍郎陜人上官儀同東西臺三品【西臺侍郎即中書侍郎陜失冉翻】 癸丑詔以四年正月有事於泰山仍以來年二月幸東都 左相許圉師之子奉輦直長自然遊獵犯人田【奉輦直長即尚輦直長殿中六局直長正七品龍朔改尚輦局爲奉輦局相息亮翻長知兩翻】田主怒自然以鳴鏑射之【射而亦翻】圉師杖自然一百而不以聞田主詣司憲訟之司憲大夫楊德裔不爲治【治直之翻】西臺舍人袁公瑜遣人易姓名上封事告之【西臺舍人即中書舍人為于偽翻上時掌翻】上曰圉師爲宰相侵陵百姓匿而不言豈非作威作福圉師謝曰臣備位樞軸以直道事陛下不能悉允衆心故爲人所攻訐【訐居謁翻】至於作威福者或手握彊兵或身居重鎮臣以文吏奉事聖明惟知閉門自守何敢作威福上怒曰汝恨無兵邪許敬宗曰人臣如此罪不容誅遽令引出詔特免官【考異曰舊本紀十一月辛未圉師下獄新本紀十一月辛未圉師貶䖍州刺史今據實錄辛未免官久之貶䖍州刺史舊紀貶䖍州刺史在三年二月新本紀誤】 癸酉立皇子旭輪爲殷王【旭輪後改名旦是為睿宗】 十二月戊申詔以方討高麗百濟【麗力知翻】河北之民勞於征役其封泰山幸東都並停 䫻海道總管蘇海政【䫻越筆翻】受詔討龜兹【龜兹音丘慈】敕興㫺亡繼往絶二可汗發兵與之俱【可從刋入聲汗音寒】至興㫺亡之境繼往絶素與興㫺亡有怨【事見上卷顯慶二年注】密謂海政曰彌射謀反請誅之【阿史那彌射是為興昔亡可汗】時海政兵纔數千集軍吏謀曰彌射若反我輩無噍類【噍才笑翻】不如先事誅之【先悉薦翻】乃矯稱敕令大總管齎帛數萬段賜可汗及諸酋長興㫺亡帥其徒受賜海政悉收斬之其鼠尼施抜塞幹兩部亡走【鼠尼施啜咄陸五部之一也抜塞幹俟斤弩失畢五部之一也酋慈由翻長知兩翻帥讀曰率尼女夷翻】海政與繼往絶追討平之軍還至踈勒南弓月部復引吐蕃之衆來欲與唐兵戰海政以師老不敢戰以軍資賂吐蕃約和而還由是諸部落皆以興昔亡爲寃各有離心繼往絶尋卒【復扶又翻卒子恤翻】十姓無主有阿史那都支及李遮匐收其餘衆附於吐蕃【為都支遮匐連兵反張本】 是歲西突厥宼庭州刺史來濟將兵拒之【厥九勿翻將即亮翻】謂其衆曰吾久當死幸蒙存全以至今日當以身報國遂不釋甲胄赴敵而死<br />
<br />
  三年春正月左武衛將軍鄭仁泰討鐵勒叛者餘種悉平之【種章勇翻】 乙酉以李義府爲右相【右相中書令也】仍知選事【選須絹翻】 二月徙燕然都護府於回紇更名瀚海都護徙故瀚海都護於雲中古城更名雲中都護【燕然都護置於貞觀二十一年見一百九十八卷瀚海都護置於永徽元年見一百九十九卷燕因肩翻更工衡翻】以磧爲境磧北州府皆隸瀚海磧南隸雲中【雲中都護府治金河即秦漢雲中舊城東北至朔州三百七十里麟德元年更名單于大都護府杜佑曰單于都護府南至榆林郡百二十里東南到馬邑郡三百五十里磧七逆翻】 三月許圉師再貶䖍州刺史【䖍州在京師東南四千一十七里至東都三千四百里】楊德裔以阿黨流庭州圉師子文思自然並免官 右相河間郡公李義府典選【選須絹翻】恃中宮之勢專以賣官爲事銓綜無次怨讟盈路上頗聞之從容謂義府曰卿子及壻頗不謹多爲非法我尚爲卿掩覆卿宜戒之【從千容翻爲于偽翻覆敷又翻】義府勃然變色頸頰俱張【張知亮翻】曰誰告陛下上曰但我言如是何必就我索其所從得邪【索山客翻】義府殊不引咎緩步而去上由是不悦望氣者杜元紀謂義府所居第有獄氣宜積錢二十萬緡以厭之【厭於協翻】義府信之聚歛尤急【歛力贍翻】義府居母喪朔望給哭假【假古訝翻】輒微服與元紀出城東登古塜候望氣色或告義府窺覘災眚隂有異圖【覘丑亷翻又丑艶翻眚所景翻】又遣其子右司議郎津召長孫無忌之孫延受其錢七百緡除延司津監【唐東宫司議郎四人正六品上掌啟奏記注龍朔改司議郎為左司議郎太子舍人為右司議郎漢官有都水長屬主爵掌諸池沼後改為使者後漢改為河隄謁者晉置都水臺有使者一人掌舟檝之事梁改為太舟卿北齊亦曰都水臺隋改為都水監唐因之貞觀改為使者從六品龍朔元年改為司津監掌川澤津梁之政令】右金吾倉曹參軍楊行頴告之夏四月乙丑下義府獄【下遐嫁翻】遣司刑太常伯劉祥道與御史詳刑共鞫之【司刑太常伯即刑部尚書詳刑大理也唐自永徽以後大獄以尚書刑部御史臺大理寺官雜按謂之三司】仍命司空李勣監焉【監古衘翻】事皆有實戊子詔義府除名流嶲州津除名流振州諸子及壻並除名流庭州朝野莫不稱慶【嶲音髓朝直遥翻】或作河間道行軍元帥劉祥道破銅山大賊李義府露布【李義府河間人故云然帥所類翻】牓之通衢義府多取人奴婢及敗各散歸其家故其露布云混奴婢而亂放各識家而競入【此姑述時人快義府之得罪而有是通鑑因采而誌之以為世鑒學者為文類有所祖漢高帝為太上皇營新豐後人誌其事其辭云混雞犬而亂放各識家而競入此語所祖有自來矣】 乙未置鷄林大都督府於新羅國以金法敏為之 丙午蓬萊宫含元殿成上始移仗居之更命故宫曰西内【故宫謂太極宫自武德以來人主居之自是以後謂之西内更工衡翻】戊申始御紫宸殿聽政【蓬萊宫正殿曰含元殿含元之後曰宣政殿宣政殿北曰紫宸門内有紫宸殿即内衙之正殿】 五月壬午柳州蠻酋吳君解反【柳州漢渾中縣地隋置馬平縣唐武德四年置南昆州貞觀八年改曰柳州】遣冀州長史劉伯英右武衛將軍馮士翽嶺南兵討之【翽呼外翻】 吐蕃與吐谷渾互相攻各遣使上表論曲直更來求援【吐從暾入聲谷音浴使疏吏翻上時掌翻更工衡翻迭也】上皆不許吐谷渾之臣素和貴有罪逃奔吐蕃具言吐谷渾虚實吐蕃發兵擊吐谷渾大破之吐谷渾可汗曷鉢與弘化公主帥數千帳弃國走依涼州請徙居内地【唐會要曰吐谷渾自永嘉之末始西度洮水建國於群羌之故地龍朔三年為吐蕃所滅凡三百五十年帥讀曰率下同】上以涼州都督鄭仁泰為青海道行軍大總管帥右武衛將軍獨孤卿雲辛文陵等分屯涼鄯二州以備吐蕃【鄯時戰翻涼鄯相去五百八十里】六月戊申又以左武衛大將軍蘇定方為安集大使節度諸軍為吐谷渾之援吐蕃禄東贊屯青海遣使者論仲琮入見【吐蕃立國之初有大論小論以統國事後因以為貴姓見賢遍翻】表陳吐谷渾之罪且請和親上不許遣左衛郎將劉文祥使於吐蕃降璽書責讓之【將即亮翻璽斯氏翻】 秋八月戊申上以海東累歲用兵百姓困於征調【調徒弔翻】士卒戰溺死者甚衆【溺奴狄翻】詔罷三十六州所造船遣司元太常伯竇德玄等【司元太常伯即戶部尚書】分詣十道問人疾苦黜陟官吏德玄毅之曾孫也【竇毅太穆皇后之父】 九月戊午熊津道行軍總管右威衛將軍孫仁師等破百濟餘衆及倭兵於白江拔其周留城【倭烏禾翻】初劉仁願劉仁軌既克眞峴城【克眞峴城見上卷二年】詔孫仁師將兵浮海助之【將即亮翻下同】百濟王豐南引倭人以拒唐兵仁師與仁願仁軌合兵勢大振諸將以加林城水陸之衝欲先攻之仁軌曰加林險固急攻則傷士卒緩之則曠日持久周留城虜之巢穴羣凶所聚除惡務本【書泰誓之言】宜先攻之若克周留諸城自下於是仁師仁願與新羅王法敏將陸軍以進仁軌與别將杜爽扶餘隆將水軍及糧船自熊津入白江以會陸軍同趣周留城【趣七喻翻】遇倭兵於白江口四戰皆捷焚其舟四百艘烟炎灼天【艘蘇遭翻炎讀曰燄】海水皆赤百濟王豐脫身奔高麗王子忠勝忠志等帥衆降【帥讀曰率下之帥皆帥降戶江翻下同】百濟盡平唯别帥遲受信據任存城不下【帥所類翻尉紆勿翻】初百濟西部人黑齒常之長七尺餘驍勇有謀略【長直亮翻驍堅堯翻】仕百濟為達率兼郡將猶中國刺史也【新羅官有十六品左平一品達率二品五方各有方領一人以達率為之方有十郡郡有將三人以德率為之德率四品百濟置官蓋與新羅略同也率所類翻】蘇定方克百濟常之帥所部隨衆降定方縶其王及太子縱兵刧掠壯者多死常之懼與左右十餘人遁歸本部收集亡散保任存山結柵以自固旬月間歸附者三萬餘人定方遣兵攻之常之拒戰唐兵不利常之復取二百餘城【復扶又翻】定方不能克而還【還從宣翻又如字】常之與别部將沙吒相如【沙吒夷人複姓吒陟加翻】各據險以應福信百濟既敗皆帥其衆降劉仁軌使常之相如自將其衆取任存城仍以糧仗助之孫仁師曰此屬獸心何可信也仁軌曰吾觀二人皆忠勇有謀敦信重義但向者所託未得其人今正是其感激立效之時不用疑也遂給其糧仗分兵隨之攻拔任存城遲受信弃妻子奔高麗詔劉仁軌將兵鎮百濟召孫仁師劉仁願還百濟兵火之餘比屋彫殘【比毗必翻又毗至翻】僵尸滿野仁軌始命瘞骸骨籍戶口理村聚署官長通道塗立橋梁補隄堰復陂塘課耕桑賑貧乏養孤老立唐社禝頒正朔及廟諱【卒如仁軌之志所謂有志者事竟成也僵居良翻瘞於計翻長知兩翻賑津忍翻】百濟大悦闔境各安其業然後修屯田儲糗粮訓士卒以圖高麗【糗去九翻】劉仁願至京師上問之曰卿在海東前後奏事皆合機宜復有文理【復扶又翻】卿本武人何能如是仁願曰此皆劉仁軌所為非臣所及也上悦加仁軌六階【勲有級官有階】正除帶方州刺史為築第長安厚賜其妻子遣使齎璽書勞勉之【為于偽翻使疏吏翻勞力到翻】上官儀曰仁軌遭黜削而能盡忠【黜削謂白衣從軍自効也】仁願秉節制而能推賢皆可謂君子矣 冬十月辛巳朔詔太子每五日於光順門内視諸司奏事【唐六典大明宫紫宸殿内朝正殿也殿之南面曰紫宸門左曰崇明門右曰光順門】其事之小者皆委太子决之 十二月庚子詔改來年元 壬寅以安西都護高賢為行軍總管將兵擊弓月以救于闐 是歲大食擊波斯拂菻破之【拂菻古大秦國也居西海上一曰海西國去京師四萬里北直突厥可薩部西瀕海東南接波斯杜佑曰大秦前漢犂靬國也菻力錦翻又力鴆翻類篇曰佛菻】南侵婆羅門吞滅諸胡勝兵四十餘萬【勝音升】<br />
<br />
  麟德元年【去年絳州麟見又含元殿前麟趾見於是改元】春正月甲子改雲中都護府為單于大都護府以殷王旭輪為單于大都護【單音蟬】初李靖破突厥【見一百九十三卷太宗貞觀四年厥九勿翻】遷三百帳於雲中城阿史德氏為之長【長知兩翻】至是部落漸衆阿史德氏詣闕請如胡法立親王為可汗以統之上召見謂曰今之可汗古之單于也故更為單于都護府【更工衡翻改也】而使殷王遥領之 二月戊子上行幸萬年宫【永徽元年改九成宫為萬年宫】 夏四月壬子衛州刺史道孝王元慶薨 丙午魏州刺史郇公孝協坐贓賜死司宗卿隴西王博义奏孝協父叔良死王事【司宗卿即宗丘卿叔良太祖之孫高祖時叔良擊突厥中流矢薨郇音荀】孝協無兄弟恐絶嗣上曰畫一之法不以親疎異制【漢書云蕭何為法講若畫一注云畫一言整齊也】苟害百姓雖皇太子亦所不赦孝協有一子何憂乏祀乎孝協竟自盡於第 五月戊申朔遂州刺史許悼王孝薨【孝上子也後宫所生】 乙卯於昆明之弄棟川置姚州都督府【劉昫曰漢益州郡之雲南縣古滇國後漢屬永昌郡蜀劉氏分永昌為建寜郡又分永昌建寜置雲南郡而治于弄棟晉改為晋寜郡又置寜州武德四年安撫大使李英以此川人多姓姚故置姚州今陞置都督府管州三十三】 秋七月丁未朔詔以三年正月有事於岱宗 八月丙子車駕還京師幸舊宅【舊宅帝為晋王時所居也】留七日壬午還蓬萊宫 丁亥以司列太常伯劉祥道兼右相【司列太常伯即吏部尚書】大司憲竇德玄為司元太常伯檢校左相【太府憲即御史大夫司元太常伯即戶部尚書左相即侍中】冬十月庚辰檢校熊津都督劉仁軌上言【上時掌翻】臣伏覩所存戌兵疲羸者多勇健者少【羸倫為翻少詩沼翻】衣服貧敝唯思西歸無心展效臣問以往在海西見百姓人人應募爭欲從軍或請自辦衣糧謂之義征何為今日士卒如此咸言今日官府與曩時不同人心亦殊曩時東西征役身没王事並蒙敕使弔祭【使疏吏翻】追贈官爵或以死者官爵囘授子弟凡度遼海者皆賜勲一轉自顯慶五年以來征人屢經度海官不記錄其死者亦無人誰何【誰何問也問其為誰緣何而死也】州縣每發百姓為兵其壯而富者行錢參逐皆亡匿得免【謂州縣官發人為兵其吏卒之參陪隨逐者富民行錢與之相為掩蔽得以亡匿按元和四年御史臺奏比來常參官入光範門及中書省所將參從人數頗多參從猶參逐也】貧者身雖老弱被發即行頃者破百濟及平壤苦戰【破百濟見上卷顯慶五年平壤苦戰見龍朔二年被皮義翻】當時將帥號令許以勲賞無所不至及達西岸惟聞枷鏁推禁奪賜破勲州縣追呼無以自存公私困弊不可悉言以是昨發海西之日已有逃亡自殘者非獨至海外而然也又本因征役授勲級以為榮寵而比年出征皆使勲官挽引【比毗至翻挽引謂挽引舟車】勞苦與白丁無殊百姓不願從軍率皆由此臣又問曩日士卒留鎮五年尚得支濟今爾等始經一年何為如此單露咸言初發家日惟令備一年資裝今已二年未有還期臣檢校軍士所留衣今冬僅可充事來秋以往全無凖擬陛下留兵海外欲殄滅高麗百濟高麗舊相黨援倭人雖遠亦共為影響若無鎮兵還成一國今既資戍守又置屯田所藉士卒同心同德而衆有此議何望成功自非有所更張厚加慰勞【董仲舒曰琴瑟不調必改而更張之更工衡翻勞力到翻】明賞重罸以起士心若止如今日以前處置恐師衆疲老立效無日逆耳之事或無人為陛下盡言【處昌呂翻為于偽翻】故臣披露肝膽昧死奏陳上深納其言遣右威衛將軍劉仁願將兵渡海以代舊鎮之兵【將即亮翻】仍敕仁軌俱還仁軌謂仁願曰國家懸軍海外欲以經略高麗其事非易【易以豉翻】今收穫未畢而軍吏與士卒一時代去軍將又歸【將即亮翻下軍將同】夷人新服衆心未安必將生變不如且留舊兵漸令收穫辦具資糧節級遣還【節級猶今人言節次也】軍將且留鎮撫未可還也仁願曰吾前還海西大遭讒謗云吾多留兵衆謀據海東幾不免禍【幾居希翻】今日唯知准敕【准與凖同本朝寇凖為相省吏避其名凡文書準字皆去十後遂因而不改】豈敢擅有所為仁軌曰人臣苟利於國知無不為豈恤其私乃上表陳便宜【上時掌翻】自請留鎮海東上從之仍以扶餘隆為熊津都尉【考異曰實録作熊津都督按時劉仁軌檢校熊津都督豈可復以隆為之明年實録稱熊津都尉扶餘隆與金法敏盟今從之】使招輯其餘衆 初武后能屈身忍辱奉順上意故上排羣議而立之及得志專作威福上欲有所為動為后所制上不勝其忿有道士郭行眞出入禁中嘗為厭勝之術【勝音升厭於協翻又於琰翻】宦者王伏勝發之上大怒密召西臺侍郎同東西臺三品上官儀議之儀因言皇后專恣海内所不與請廢之上意亦以為然即命儀草詔左右奔告於后后遽詣上自訴詔草猶在上所上羞縮不忍復待之如初【復扶又翻】猶恐后怨怒因紿之曰我初無此心皆上官儀教我儀先為陳王諮議與王伏勝俱事故太子忠【忠自陳王立為皇太子王府諮議參軍正五品上掌訏謨左右】后於是使許敬宗誣奏儀伏勝與忠謀大逆十二月丙戍儀下獄【下遐嫁翻】與其子庭芝王伏勝皆死籍没其家戊子賜忠死於流所【顯慶五年忠徙黔州】右相劉祥道坐與儀善罷政事為司禮太常伯【司禮太常伯即禮部尚書】左肅機鄭欽泰等【左肅機即尚書左丞】朝士流貶者甚衆皆坐與儀交通故也【朝直遥翻】自是上每視事則后垂簾於後政無大小皆與聞之【與讀曰預】天下大權悉歸中宫黜陟殺生决於其口天子拱手而已中外謂之二聖 【考異曰唐歷群臣朝謁萬方表奏皆呼為二聖帝坐于東間后坐于西間后隨其愛憎生殺在口按武后雖悍戾豈得高宗尚在與高宗對坐受群臣朝謁乎恐不至此今從實錄】 太子右中護檢校西臺侍郎樂彥瑋【龍朔改左右庶子為左右中護】西臺侍郎孫處約並同東西臺三品<br />
<br />
  二年春正月丁卯吐蕃遣使入見請復與吐谷渾和親【見賢遍翻復扶又翻】仍求赤水地畜牧【即河源之赤水也本吐谷渾地畜吁玉翻】上不許 二月壬午車駕發京師丁酉至合璧宫 上語及隋煬帝謂侍臣曰煬帝拒諫而亡朕常以為戒虛心求諫而竟無諫者何也李勣對曰陛下所為盡善羣臣無得而諫【禇遂良韓瑗之死不唯拒諫且殺諫者矣羣臣誰敢復諫乎李勣獻諛以苟利禄而不知凶于其家】 三月甲寅以兼司戎太常伯姜恪同東西臺三品恪寶誼之子也【司戎太常伯即兵部尚書姜寶誼從高祖起兵于太原】辛未東都乾元殿成【乾元殿洛陽宫正殿也武后垂拱四年毁為明堂】閏月壬申朔車駕至東都 踈勒弓月引吐蕃侵于闐敕西州都督崔知辯左武衛將軍曹繼叔將兵救之【將即亮翻 考異曰實錄作西川都督按於時未有西川之名必西州也】 夏四月戊辰左侍極陸敦信【龍朔改左右散騎常侍為左右侍極】檢校右相【句斷】西臺侍郎孫處約太子右中護檢校西臺侍郎樂彦瑋並罷政事【處昌呂翻】 祕閣郎中李淳風【龍朔改太史局為秘閣局令為郎中丞為郎】以傅仁均戊寅歷推步浸踈乃增損劉焯皇極歷【戊寅歷始行見一百八十七卷高祖武德二年隋時劉焯造甲子元歷謂之皇極歷為張賓所擯不得行焯之若翻】更撰麟德歷五月辛卯行之【更工衡翻】 秋七月己丑兖州都督鄧康王元裕薨 上命熊津都尉扶餘隆與新羅王法敏釋去舊怨【去羌呂翻】八月壬子同盟于熊津城劉仁軌以新羅百濟耽羅倭國使者浮海西還【耽羅國一曰儋羅居新羅武州南島上初附百濟後附新羅】會祠泰山高麗亦遣太子福男來侍祠 冬十月癸丑皇后表稱封禪舊儀祭皇地祗太后昭配而令公卿行事禮有未安至日妾請帥内外命婦奠獻【内命婦自三妃至采女以備古者三夫人九嬪二十七世婦八十一御妻又有六尚二十四司二十四典二十四掌龍朔二年又置贊德宣儀承閨承旨衛仙供奉侍櫛侍巾亦分為九品皆内官也外命婦皇姑封大長公主皇姊妹封長公主皇女封公主皇太子之女封郡主王之女封縣主王母妻為妃一品及國公母妻為國夫人三品以上母妻為郡夫人五品勲官三品封母妻為縣君散官並同職事勲官四品封母妻為鄉君其母並加太字各視其夫子之品】詔禪社首以皇后為亞獻【兖州傳城縣有社首山】越國太妃燕氏為終獻【燕氏越王貞之母蓋太宗妃嬪此時唯燕氏在也燕因肩翻】壬戍詔封禪壇所設上帝后上位先用藳秸陶匏等【秸古黠翻】並宜改用茵褥罍爵其諸郊祀亦宜凖此又詔自今郊廟享宴文舞用功成慶善之樂武舞用神功破陳之樂【陳讀曰陣】丙寅上發東都從駕文武儀仗數百里不絶【從才用翻】列營置幕彌亘原野東自高麗西至波斯烏長諸國【自吐火羅踰五種至婆羅覩邏北踰山行六百里得烏萇國長讀曰萇】朝會者各帥其屬扈從穹廬毳幕牛羊駝馬填咽道路時比歲豐稔【朝直遥翻帥讀曰率從才用翻比毗至翻】米斗至五錢麥豆不列于市 十一月戊子上至濮陽【濮陽顓頊之墟春秋衛成公自楚丘徙此漢為濮陽縣帶東郡晉分為濮陽郡隋為縣屬滑州唐屬濮州濮慱木翻】竇德玄騎從【騎奇寄翻從才用翻】上問濮陽謂之帝丘何也德玄不能對許敬宗自後躍馬而前曰昔顓頊居此故謂之帝丘上稱善敬宗謂人曰大臣不可以無學吾見德玄不能對心實羞之德玄聞之曰人各有能有不能吾不強對以所不知此吾所能也【強其兩翻】李勣曰敬宗多聞信美矣德玄之言亦善也夀張人張公藝九世同居【夀張縣前漢曰夀良屬東郡光武改夀張屬東平國隋屬濟州唐屬鄆州】齊隋唐皆旌表其門上過夀張幸其宅問所以能共居之故公藝書忍字百餘以進上善之賜以縑帛十二月丙午車駕至齊州留十日丙辰發靈巖頓至泰山下有司於山南為圓壇山上為登封壇社首山上為降禪方壇<br />
<br />
  乾封元年春正月戊辰朔上祀昊天上帝于泰山南己巳登泰山封玉牒上帝冊藏以玉匱配帝冊藏以金匱皆纒以金繩封以金泥印以玉璽藏以石䃭【璽斯氏翻䃭古禫翻】庚午降禪于社首祭皇地祗上初獻畢執事者皆趨下宦者執帷皇后升壇亞獻帷帟皆以錦繡為之【周禮注在旁曰帷在上曰帟帟幄中座上承塵也帟音亦】酌酒實俎豆登歌皆用宫人壬申上御朝覲壇受朝賀【朝直遥翻下同】赦天下改元文武官三品已上賜爵一等四品已下加一階先是階無泛加皆以勞考叙進至五品三品仍奏取進止【先悉薦翻】至是始有泛階比及末年服緋者滿朝矣【比必利翻】時大赦惟長流人不聽還李義府憂憤發病卒【龍朔三年李義府流嶲州】自義府流竄朝士日憂其復入及聞其卒衆心乃安【復扶又翻卒子恤翻】丙戍車駕發泰山辛卯至曲阜【曲阜魯侯伯禽所都應劭云曲阜在魯城中委曲長七八里隋始置曲阜縣屬兖州】贈孔子太師以少牢致祭【少詩照翻】癸未至亳州謁老君廟【亳州谷陽縣漢苦縣也有老子祠是年改為真源縣亳州至東都八百九十八里】上尊號曰太上玄元皇帝丁丑至東都留六日甲申幸合璧宫夏四月甲辰至京師【東都至京師八百五十里】謁太廟 庚戌左侍極兼檢校右相陸敦信以老疾辭職拜大司成兼左侍極罷政事【大司成即國子祭酒】 五月庚寅鑄乾封泉寶錢一當十俟期年盡廢舊錢【期讀曰朞】 高麗泉蓋蘇文卒長子男生代為莫離支【卒子恤翻長知兩翻】初知國政出廵諸城使其弟男建男產知留後事或謂二弟曰男生惡二弟之逼【惡烏路翻】意欲除之不如先為計二弟初未之信又有告男生者曰二弟恐兄還奪其權欲拒兄不納男生潜遣所親往平壤伺之【伺相吏翻】二弟收掩得之乃以王命召男生男生懼不敢歸男建自為莫離支發兵討之男生走保别城使其子獻誠詣闕求救六月壬寅以左驍衛大將軍契苾何力為遼東道安撫大使將兵救之以獻誠為右武衛將軍使為鄉導【契欺訖翻苾毗必翻大使疏吏翻將即亮翻鄉讀曰嚮】又以右金吾衛將軍龎同善營州都督高侃為行軍摠管同討高麗 秋七月乙丑朔徙殷王旭輪為豫王以大司憲兼檢校太子左中護劉仁軌為右相初仁軌為給事中按畢正義事【事見上卷顯慶元年】李義府怨之出為青州刺史會討百濟仁軌當浮海運糧時未可行【海行非遇順風不可】義府督之遭風失船丁夫溺死甚衆命監察御史袁異式往鞫之【溺奴狄翻監古衘翻】義府謂異式曰君能辦事不憂無官異式至謂仁軌曰君與朝廷何人為讐宜早自為計仁軌曰仁軌當官不職國有常刑公以法斃之無所逃命若使遽自引决以快讐人竊所未甘乃具獄以聞異式將行仍自掣其鎻【恐鎻不入簧行後得私開之也掣昌列翻】獄上【上時掌翻】義府言於上曰不斬仁軌無以謝百姓舍人源直心曰海風暴起非人力所及上乃命除名以白衣從軍自效【事見上卷顯慶五年】義府又諷劉仁願使害之仁願不忍殺及為大司憲異式懼不自安仁軌瀝觴告之曰仁軌若念疇昔之事有如此觴仁軌既知政事異式尋遷詹事丞【詹事丞正六品上】時論紛然仁軌聞之遽薦為司元大夫【司元大夫即戶部郎中】監察御史杜易簡謂人曰【易以䜴翻】斯所謂矯枉過正矣八月辛丑司元太常伯兼檢校左相竇德玄薨 初<br />
<br />
  武士彠娶相里氏【彠一虢翻相息亮翻】生男元慶元爽又娶楊氏生三女長適越王府法曹賀蘭越石【長知兩翻】次皇后次適郭孝慎士彠卒元慶元爽及士彠兄子惟良懷運皆不禮於楊氏楊氏深衘之越石孝慎及孝慎妻並早卒越石妻生敏之及一女而寡后既立楊氏號榮國夫人越石妻號韓國夫人【唐制國夫人位一品】惟良自始州長史超遷司衛少卿【司衛少卿即衛尉少卿】懷運自瀛州長史遷淄州刺史元慶自右衛郎將為宗正少卿【此時已改宗正為司宗】元爽自安州戶曹累遷少府少監【此時已改少府監為内府監】榮國夫人嘗置酒謂惟良等曰頗憶疇昔之事乎今日之榮貴復何如對曰惟良等幸以功臣子弟早登宦籍揣分量才不求貴逹豈意以皇后之故曲荷朝恩夙夜憂懼不為榮也【復扶又翻分扶問翻荷下可翻朝直遥翻】榮國不悦皇后乃上疏請出惟良等為遠州刺史【上時掌翻】外示謙抑實惡之也【惡烏路翻下后惡同】於是以惟良檢校始州刺史元慶為龍州刺史元爽為濠州刺史【龍州古江油秦漢曹魏為無人之地鄧艾伐蜀由隂平景谷行無人之地七百里始至江油晉置隂平郡於此置平武縣至梁有楊李二姓大豪分據其地後魏平蜀置龍州濠州漢鍾離縣地晉安帝分置鍾離郡梁置北徐州後齊曰西楚州隋開皇二年改曰豪州唐曰濠州始州至京師一千六百六十二里至東都二千五百六十里龍州至京師二千六百六十里東都三千一百一十五里豪州至京師二千一百五十里東都一千三百一十三里】元慶至州以憂卒【卒子恤翻下同】元爽坐事流振州而死韓國夫人及其女以后故出入禁中皆得幸於上韓國尋卒其女賜號魏國夫人上欲以魏國為内職心難后未决后惡之會惟良懷運與諸州刺史詣泰山朝覲從至京師惟良等獻食【朝直遥翻從才用翻 考異曰舊傳云后諷上幸楊氏宅惟良等獻食今從實錄】后密置毒醢中使魏國食之暴卒因歸罪於惟良懷運丁未誅之改其姓為蝮氏懷運兄懷亮早卒其妻善氏尤不禮於榮國坐惟良等沒入掖庭榮國令后以他事束棘鞭之肉盡見骨而死 九月龎同善大破高麗兵泉男生帥衆與同善合詔以男生為特進遼東大都督兼平壤道安撫大使封玄菟郡公【帥讀曰率使疏吏翻下同菟同都翻】 戊子金紫光禄大夫致仕廣平宣公劉祥道薨子齊賢嗣齊賢為人方正上甚重之為晉州司馬將軍史興宗嘗從上獵苑中因言晉州產佳鷂劉齊賢今為司馬請使捕之上曰劉齊賢豈捕鷂者邪【鷂弋照翻】卿何以此待之 冬十二月己酉以李勣為遼東道行軍大摠管以司列少常伯安陸郝處俊副之【安陸縣漢屬江夏郡宋分屬安陸郡隋唐屬安州處昌呂翻】以擊高麗龎同善契苾何力並為遼東道行軍副大摠管兼安撫大使如故其水陸諸軍摠管并運粮使竇義積獨孤卿雲郭待封等並受勣處分【處昌呂翻分扶問翻】河北諸州租賦悉詣遼東給軍用待封孝恪之子也【郭孝恪事太宗戰死於龜兹】勣欲與其壻京兆杜懷恭偕行以求勲效懷恭辭以貧勣贍之復辭以無奴馬【贍昌艶翻下同復扶又翻】又贍之懷恭辭窮乃亡匿岐陽山中謂人曰公欲以我立法耳勣聞之流涕曰杜郎疎放【今人猶呼壻為郎】此或有之乃止<br />
<br />
  二年春正月上耕藉田有司進耒耜加以彫飾上曰耒耜農夫所執豈宜如此之麗命易之【耒盧對翻】既而耕之九推乃止【耕籍之制月令及鄭玄注周禮皆云天子三推盧植注禮記曰天子耕籍一發九推耒此用盧說也推吐雷翻】 自行乾封泉寶錢穀帛踊貴商賈不行【賈音古】癸未詔罷之 二月丁酉涪陵悼王愔薨【愔上弟也涪音浮愔於今翻】 辛丑復以萬年宫為九成宫【永徽一年改九成宫為萬年宫復扶又翻又如字】 生羌十二州為吐蕃所破三月戊寅悉罷之上屢責侍臣不進賢衆莫敢對司列少常伯李安期對曰天下未嘗無賢亦非羣臣敢蔽賢也【司列少常伯即吏部侍郎少始照翻】比來公卿有所薦引為讒者已指為朋黨滯淹者未獲伸而在位者先獲辠是以各務杜口耳陛下果推至誠以待之其誰不願舉所知此在陛下非在羣臣也上深以為然安期百藥之子也【李百藥德林之子比毗至翻】 夏四月乙卯西臺侍郎楊弘武戴至德正諫大夫兼東臺侍郎李安期東臺舍人昌樂張文瓘司列少常伯兼正諫大夫河北趙仁本並同東西臺三品【龍朔改給事中為東臺舍人諫議大夫為正諫夫夫樂音洛】弘武素之弟子【楊素仕隋貴顯】至德胄之兄子也【戴胄相太宗】時造蓬萊上陽合璧等宫【上陽宫在洛陽宫城之西南隅南臨洛水西距穀水東即宫城北連禁苑宫内正門正殿皆東向正門曰提象正殿曰觀風其内别殿亭觀九所上陽之西隔穀水有西上陽宫虹梁跨穀行幸往來】頻征伐四夷廐馬萬匹倉庫漸虛張文瓘諫曰隋鑒不遠願勿使百姓生怨上納其言减廐馬數千匹 秋八月己丑朔日有食之 辛亥東臺侍郎同東西臺三品李安期出為荆州長史【荆州京師東南一千七百三十里至東都一千三百三十五里宋白曰荆州秦南郡地漢為臨江國江左置荆州以為重鎮擬周之分陜唐為大都督府】 九月庚申上以久疾命太子弘監國【監古衘翻】辛未李勣拔高麗之新城使契苾何力守之勣初度<br />
<br />
  遼謂諸將曰新城高麗西邊要害不先得之餘城未易取也【易以豉翻】遂攻之城人師夫仇等縳城主開門降【降戶江翻】勣引兵進擊一十六城皆下之龎同善高侃尚在新城泉男建遣兵襲其營左武衛將軍薛仁貴擊破之侃進至金山與高麗戰不利高麗乘勝逐北仁貴引兵横擊大破之斬首五萬餘級【新書作斬馘五千】拔南蘇木底蒼巖三城【三城後皆置為州】與泉男生軍合郭待封以水軍自别道趣平壤勣遣别將馮師本載糧仗以資之【趣七喻翻將即亮翻】師本船破失期待封軍中飢窘欲作書與勣恐為虜所得知其虚實乃作離合詩以與勣【離合詩離析字畫合之成文以見其意】勣怒曰軍事方急何以詩為必斬之行軍管記通事舍人元萬頃為釋其義【管記掌軍中書檄為于偽翻】勣乃更遣粮仗赴之萬頃作檄高麗文曰不知守鴨綠之險泉男建報曰謹聞命矣即移兵據鴨綠津唐兵不得度上聞之流萬頃於嶺南郝處俊在高麗城下未及成列高麗奄至軍中大駭處俊據胡床方食乾糒【胡床即今之交床乾音干糒音備】潜簡精鋭擊敗之【敗補邁翻】將士服其膽略 冬十二月甲午詔自今祀昊天上帝五帝皇地祗神州地祗並以高祖太宗配仍合祀昊天上帝五帝於明堂【北兼用貞觀顯慶之禮】 是歲海南獠䧟瓊州【瓊州本隋朱崖郡之瓊山縣貞觀五年置瓊州獠魯皓翻】<br />
<br />
  總章元年【以將作明堂改元是年三月方改元】春正月壬子以右相劉仁軌為遼東道副大總管 二月壬午李勣等拔高麗扶餘城【扶餘國之故墟故城存其名】薛仁貴既破高麗於金山乘勝將三千人將攻扶餘城諸將以其兵少止之仁貴曰兵不在多顧用之何如耳遂為前鋒以進與高麗戰大破之殺獲萬餘人遂拔扶餘城扶餘川中四十餘城皆望風請服侍御史洛陽賈言忠奉使自遼東還【使疏吏翻】上問以軍事言忠對曰高麗必平上曰卿何以知之對曰隋煬帝東征而不克者人心離怨故也【事見隋煬帝紀】先帝東征而不克者高麗未有舋也【事見太宗紀舋許覲翻】今高藏微弱權臣擅命蓋蘇文死男建兄弟内相攻奪男生傾心内附為我鄉導【鄉讀曰嚮】彼之情偽靡不知之以陛下明聖國家富彊將士盡力以乘高麗之亂其勢必克不俟再舉矣且高麗連年飢饉妖異屢降【妖於喬翻】人心危駭其亡可翹足待也上又問遼東諸將孰賢【謂征遼東之諸將也】對曰薛仁貴勇冠三軍龎同善雖不善鬬而持軍嚴整高侃勤儉自處忠果有謀契苾何力沈毅能斷雖頗忌前【冠古玩翻處昌呂翻沈持林翻斷丁亂翻忌前忌人在己前也】而有統御之才然夙夜小心忘身憂國皆莫及李勣也上深然其言泉男建復遣兵五萬人救扶餘城【復扶又翻】與李勣等遇於薛賀水【新書作薩賀水】合戰大破之斬獲三萬餘人進攻大行城拔之 朝廷議明堂制度略定三月庚寅赦天下改元 戊寅上幸九成宫夏四月丙辰彗星見于五車【五車五星五帝車舍也五帝坐也主天子五兵】<br />
<br />
  【一曰主五糓豐耗西北大星曰天庫主太白主秦次東北曰獄主辰星主燕趙次東星曰天倉主歲星主魯衛次東南曰司空主填星主楚次西南曰卿星主熒惑主魏五星有變皆以其所占之據舊紀五車在昴畢間見賢遍翻下同彗祥歲翻】上避正殿减常膳撤樂許敬宗等奏請復常曰彗見東北高麗將滅之兆也上曰朕之不德謫見于天豈可歸咎小夷且高麗百姓亦朕之百姓也不許戊辰彗星滅 辛巳西臺侍郎同東西臺三品楊弘武薨 八月辛酉卑列道行軍摠管右威衛將軍劉仁願坐征高麗逗留流姚州 癸酉車駕還京師 九月癸巳李勣拔平壤勣既克大行城諸軍出它道者皆與勣會進至鴨綠柵高麗發兵拒戰勣等奮擊大破之追奔二百餘里拔辱夷城諸城遁逃及降者相繼【降戶江翻下同】契苾何力先引兵至平壤城下勣軍繼之圍平壤月餘高麗王藏遣泉男產帥首領九十八人持白幡詣勣降勣以禮接之【帥讀曰率】泉男建猶閉門拒守頻遣兵出戰皆敗男建以軍事委僧信誠信誠密遣人詣勣請為内應後五日信誠開門勣縱兵登城鼓譟焚城四月【月當作角否則作周】男建自刺不死【刺七亦翻】遂擒之高麗悉平 冬十月戊午以烏荼國婆羅門盧迦逸多為懷化大將軍【烏荼國一曰烏萇直天竺南東距勃律六百里西罽賓四百里婆羅門僧也唐置懷化大將軍從三品以授蕃官】逸多自言能合不死藥【合音閤】上將餌之東臺侍郎郝處俊諫曰脩短有命非藥可延貞觀之末【觀古玩翻】先帝服那羅邇娑婆寐藥竟無效大漸之際名醫不知所為議者歸罪娑婆寐將加顯戮恐取笑戎狄而止【娑婆寐事見上卷顯慶二年】前鑒不遠願陛下深察上乃止 李勣將至上命先以高藏等獻于昭陵具軍容奏凱歌入京師獻于太廟十二月丁巳上受俘于含元殿【東内正殿曰含元殿唐六典曰含元殿即龍首山之東趾階上高於平地四十餘尺南去丹鳳門四百餘步東西廣五百步殿前玉階三級每級引出一螭頭其下為龍尾道委蛇屈曲几七轉】以高藏政非己出赦以為司平太常伯員外同正【司平太常伯即工部尚書按舊書永徽五年尚藥奉御蔣孝璋員外特置仍同正員員外同正自此始】以泉男產為司宰少卿【司宰少卿即光禄少卿】僧信誠為銀青光祿大夫泉男生為右衛大將軍李勣以下封賞有差泉男建流黔中【黔音琴】扶餘豐流嶺南分高麗五部百七十六城六十九萬餘戶為九都督府四十二州【新城州遼城州哥勿州衛樂州舍利州居素州越喜州去旦州建安州凡有九都督府四十二州存於志者南蘇蓋牟代那倉巖磨米積利黎山延津木底安市諸北識利拂湼拜漢十四州而已】百縣置安東都護府於平壤以統之擢其酋帥有功者為都督刺史縣令與華人參理【酋慈由翻帥所類翻理猶治也時避上名以治為理通鑑因唐史成文】以右威衛大將軍薛仁貴檢校安東都護摠兵二萬人以鎮撫之丁卯上祀南郊告平高麗以李勣為亞獻己巳謁太廟 渭南尉劉延祐弱冠登進士第【渭南縣屬雍州後魏之渭南郡後周廢為縣冠古玩翻】政事為畿縣最【唐雍州諸縣萬年長安為赤縣餘縣為畿縣六典曰城内為京縣城外為畿縣】李勣謂之曰足下春秋甫爾遽擅大名宜稍自貶抑無為獨出人右也【史言李勣愛人以德】 時有敇征遼軍士逃亡限内不首及首而更逃者身斬【首式又翻】妻子籍没太子上表【上時掌翻】以為如此之比其數至多或遇病不及隊伍怖懼而逃【怖普布翻】或因樵採為賊所掠或渡海漂没或深入賊庭為所傷殺軍法嚴重同隊恐并獲罪即舉以為逃軍旅之中不暇勘當【當丁浪翻】直據隊司通狀關移所屬妻子没官情實可哀書曰與其殺不辜寧失不經【書大禹謨之言注云經常也寧失不常之罪不枉不辜之善】伏願逃亡之家免其配没從之 甲戌司戎太常伯姜恪兼檢校左相司平太常伯閻立本守右相 是歲京師及山東江淮旱飢二年春二月辛酉以張文瓘為東臺侍郎以右肅機檢校太子中護譙人李敬玄為西臺侍郎【譙縣帶亳州】並同東西臺三品先是同三品不入衘至是始入衘【先悉薦翻考異曰陳紀在乾封二年文瓘始同三品衘今從舊本紀陳字必誤】 癸亥以雍州長史盧承慶為司刑太常伯【雍於用翻】承慶常考内外官有一官督運遭風失米承慶考之曰監運損粮考中下【監古衘翻】其人容色自若無言而退承慶重其雅量改注曰非力所及考中中既無喜容亦無愧詞又改曰寵辱不驚考中上三月丙戍東臺侍郎郝處俊同東西臺三品 丁亥<br />
<br />
  詔定明堂制度其基八觚【觚攻乎翻方稜也】其宇上圓覆以清陽玉葉【覆敷又翻時按淮南子清陽為天故覆明堂以清陽之色玉葉非必以玉為之蓋亦瓦之類謂之葉者尚朴之意猶茨之以茅也曰玉者示寶貴之耳】其門牆階級窗櫺楣柱枊楶枅栱【說文曰在牆曰牖在屋曰窗釋名窗聰也於内見外之聰明也櫺盧經翻楯間于窗隔也楣屋耜說文曰楣屋櫋毛晃曰棟下横木曰楣柱楹也枊魚剛翻斜桷謂之飛枊楶子結翻梁上欂櫨枅堅奚翻屋櫨所以承桁栱居竦翻大杙又栱科】皆法天地隂陽律歷之數詔下之後衆議猶未决又會飢饉竟不果立 夏四月己酉朔上幸九成宫 高麗之民多離叛者敇徙高麗戶三萬八千二百於江淮之南及山南京西諸州空曠之地留其貧弱者使守安東 六月戊申朔日有食之 秋八月丁未朔詔以十月幸凉州時隴右虛耗議者多以為未宜遊幸上聞之辛亥御延福殿【九成宫中有延福殿】召五品已上謂曰自古帝王莫不廵守【守手又翻】故朕欲廵視遠俗若果為不可何不面陳而退有後言何也自宰相以下莫敢對詳刑大夫來公敏獨進曰【詳刑大夫即大理少卿】廵守雖帝王常事然高麗新平餘寇尚多西邊經略亦未息兵隴右戶口彫弊鑾輿所至供億百端誠為未易【易以䜴翻】外閒實有竊議但明制己行故羣臣不敢陳論耳上善其言為之罷西廵【為于偽翻】未幾擢公敏為黄門侍郎【幾居豈翻】甲戍改瀚海都護府為安北都護府【瀚海府見龍朔三年】九月丁丑朔詔徙吐谷渾部落就凉州南山議者恐吐蕃侵暴使不能自存欲先發兵擊吐蕃右相閻立本以為去歲飢歉未可興師議久不决竟不果徙庚寅大風海溢漂永嘉安固六千餘家【漢順帝永建四年分章安束甌鄉立永寧縣江左改曰永豐隋平陳改曰永嘉縣又孫吳立羅陽縣孫皓改曰安陽縣晉平吳改曰安固縣並屬永嘉郡唐初屬東嘉州貞觀元年州廢二縣屬栝州】冬十月丁巳車駕還京師 十一月丁亥徙豫王旭輪為冀王更名輪【更工衡翻】 司空太子太師英貞武公李勣寢疾【英者封國名貞武其謚也】上悉召其子弟在外者使歸侍疾上及太子所賜藥勣則餌之子弟為之迎醫【為于偽翻下親為非為久為同】皆不聽進曰吾本山東田夫遭值聖明致位三公年將八十 【考異曰舊傳云勣年八十六臨終語弟弼云年將八十新傳改云年踰八十按新舊傳實録皆云大業末翟讓聚衆為盗勣年十七往從之自大業十三年至此五十二年若據舊傳年八十六則年十七當在開皇時不得云大業末也總章元年賈言忠對高宗云勣年登八十去此止一年若據新傳勣㓕高麗時年已八十五亦不得云年登八十今從實録】豈非命邪脩短有期豈能復就醫工求活【復扶又翻下不復同】一旦忽謂其弟司衛少卿弼曰吾今日少愈可共置酒為樂【日少詩沼翻樂音洛】於是子孫悉集酒闌謂弼曰吾自度必不起【度徒洛翻】故欲與汝曹為别耳汝曹勿悲泣聽我約束武見房杜平生勤苦僅能立門戶遭不肖子蕩覆無餘【謂房遺愛杜荷也】吾有此子孫今悉付汝葬畢汝即遷入我堂撫養孤幼謹察視之其有志氣不倫交遊非類者皆先撾殺然後以聞【以勣之智蓋知敬業必為變也豈知敬業乃忠於唐室邪撾則瓜翻】自是不復更言十二月戊申薨上聞之悲泣葬日幸未央宫登樓望轜車慟哭起冢象隂山鐵山烏德鞬山【轜音而轜車喪車也所以載柩烏德鞬山在回紇牙帳西南】以旌其破突厥薛延陀之功【勣破突厥見一百九十三卷貞觀四年破薛延陀見一百九十八卷二十年厥九勿翻】勣為將有謀善斷【將即亮翻斷丁亂翻】與人議事從善如流戰勝則歸功於下所得金帛悉散之將士故人思致死所向克捷臨事選將必訾相其狀貌豐厚者遣之【訾即移翻訾之為言量也相息亮翻】或問其故勣曰薄命之人不足與成功名閨門雍睦而嚴其姊嘗病勣已為僕射親為之煮粥風回爇其須髩【爇如悦翻】姊曰僕妾幸多何自苦如是勣曰非為無人使令也顧姊老勣亦老雖欲久為姊煮粥其可得乎勣常謂人我年十二三時為亡賴賊【亡讀曰無】逢人則殺十四五為難當賊有所不愜則殺人【愜苦叶翻】十七八為佳賊臨陳乃殺之【陳讀曰陣】二十為大將用兵以救人死勣長子震早卒【長知兩翻卒子恤翻】震子敬業襲爵 【考異曰劉餗小說云高宗時羣蠻為寇討之輒不利乃除徐敬業為刺史發卒郊迎敬業盡放令還單騎至府賊聞新刺史至皆繕理以待敬業一無所問處置他事已畢方曰賊安在曰在南岸乃從二佐吏而往觀之莫不駭愕賊初持兵覘望及見船中無人及兵仗更閉營藏隱敬業直入其營内告云國家知汝等為貪吏所害非有他惡可悉歸田後去者為賊唯召其帥責以不早降之意各杖數十而遣之境内肅然其祖英公壯其膽略曰吾不辦此然破我家必此兒也按敬業武后時舉兵旋踵敗亡若有智勇何至如此今不取】 時承平既久選人益多【選須絹翻下同】是歲司列少常伯裴行儉始與員外郎張仁禕【唐制尚書二十四司各司有郎中二員從五品上員外郎二員從六品上禕吁韋翻】設長名姓歷牓引銓注之法又定州縣升降官資高下其後遂為永制無能革之者大畧唐之選法取人以身言書判【唐擇人之法有四一曰身取其體貌豐偉二曰言取其言辭辯正三曰書取其楷法遒美四曰判取其文理優長】計資量勞而擬官始集而試觀其書判已試而銓察其身言已銓而注詢其便利已注而唱集衆告之然後類以為甲先簡僕射乃上門下給事中讀侍郎省侍中審之不當者駮下【上時掌翻省悉景翻當丁浪翻駮比角翻】既審然後上聞主者受旨奉行各給以符謂之告身兵部武選亦然【武選兵部主之】課試之法以騎射及翹關負米【翹關長丈七尺徑二寸半凡十舉後手持關距出處無過一尺負米者負米五斛行二十步皆為中第騎奇寄翻】人有格限未至而能試文三篇謂之宏詞試判三條謂之拔萃入等者得不限而授其黔中嶺南閩中州縣官不由吏部委都督選擇土人補授【黔音琴閩眉巾翻選如字】凡居官以年為考六品以下四考為滿<br />
<br />
  咸亨二年【是年三月始改元】春正月丁丑右相劉仁軌請致仕許之 三月甲戍朔以旱赦天下改元 丁丑改蓬萊宫為含元宫【即含元殿以為宫名】 壬辰太子少師許敬宗請致仕許之 敕突厥酋長子弟事東宫【酋慈由翻長知兩翻】西臺舍人徐齊聃上疏【聃他酣翻上時掌翻】以為皇太子當引文學端良之士寘左右豈可使戎狄醜類入侍軒闥又奏齊獻公即陛下外祖雖子孫有犯豈應上延祖禰今周忠孝公廟甚修而齊獻公廟毁廢【齊獻公文德皇后父長孫晟也周忠孝公皇后父武士彠也襧乃禮翻】不審陛下何以垂示海内彰孝理之風【孝理即孝治避上名改治為理】上皆從之齊聃充容之弟也【齊聃姊入宫為充容列於九嬪】夏四月吐蕃陷西域十八州又與于闐襲龜兹撥換城陷之罷龜兹于闐焉耆踈勒四鎮【闐徒賢翻龜兹音丘慈】辛亥以右衛大將軍薛仁貴為邏娑道行軍大摠管【邏娑川吐蕃贊普牙在焉有邏些城】左衛員外大將軍阿史那道真左衛將軍郭待封副之以討吐蕃且援送吐谷渾還故地 庚午上幸九成宫 高麗酋長劒牟岑反立高藏外孫安舜為主以左監門大將軍高侃為東州道行軍摠管【高麗在東時巳列置州府故曰東州道監古衘翻】發兵討之安舜殺劒牟岑奔新羅 六月壬寅朔日有食之 秋八月丁巳車駕還京師 郭待封先與薛仁貴並列及征吐蕃耻居其下仁貴所言待封多違之軍至大非川【自鄯州鄯城縣西行三百餘里至大非川】將趣烏海【烏海在漢哭山西隋屬河源郡界杜佑曰吐蕃國出鄯城五百里過烏海暮春之月山有積雪地有冷瘴令人氣急不甚為害趣七喻翻】仁貴曰烏海險遠軍行甚難輜重自隨難以趨利【重直用翻下同趣七喻翻】宜留二萬人為兩柵於大非嶺上輜重悉置柵内吾屬帥輕鋭倍道兼行掩其未備破之必矣仁貴帥所部前行擊吐蕃於河口大破之【河口積石河口也帥讀曰率】斬獲甚衆進屯烏海以俟待封待封不用仁貴策將輜重徐進未至烏海遇吐蕃二十餘萬待封軍大敗還走悉弃輜重仁貴退屯大非川吐蕃相論欽陵將兵四十餘萬就擊之【將即亮翻下同杜佑曰禄東贊論欽陵本姓薛氏世為大論後遂以官為氏大論吐蕃統理國事之官也相息亮翻】唐兵大敗死傷略盡仁貴待封與阿史那道真並脫身免與欽陵約和而還【還音旋又如字】敕大司憲樂彦瑋即軍按其敗狀械送京師三人皆免死除名欽陵禄東贊之子也【禄東贊事始一百九十五卷太宗貞觀十四年】與弟贊婆悉多于勃倫皆有才畧禄東贊卒【卒子恤翻下氏卒同】欽陵代之三弟將兵居外鄰國畏之 關中旱飢九月丁丑詔以明年正月幸東都 甲申皇后母魯國忠烈夫人楊氏卒敕文武九品以上及外命婦並詣宅弔哭閏月癸卯皇后以久旱請避位不許 壬子加贈司徒周忠孝公武士彠為太尉太原王夫人為王妃 甲寅以左相姜恪為凉州道行軍大摠管以禦吐蕃 冬十月乙未太子右中護同東西臺三品趙仁本為左肅機罷政事【龍朔二年改左右庶子為左右中護】 庚寅詔官名皆復舊【改官名見上卷龍朔二年按新書帝紀係十二月庚寅】<br />
<br />
  資治通鑑卷二百一<br />
<br />
<史部,編年類,資治通鑑>  <br>
   </div> 

<script src="/search/ajaxskft.js"> </script>
 <div class="clear"></div>
<br>
<br>
 <!-- a.d-->

 <!--
<div class="info_share">
</div> 
-->
 <!--info_share--></div>   <!-- end info_content-->
  </div> <!-- end l-->

<div class="r">   <!--r-->



<div class="sidebar"  style="margin-bottom:2px;">

 
<div class="sidebar_title">工具类大全</div>
<div class="sidebar_info">
<strong><a href="http://www.guoxuedashi.com/lsditu/" target="_blank">历史地图</a></strong>  
<a href="http://www.880114.com/" target="_blank">英语宝典</a>  
<a href="http://www.guoxuedashi.com/13jing/" target="_blank">十三经检索</a> 
<br><strong><a href="http://www.guoxuedashi.com/gjtsjc/" target="_blank">古今图书集成</a></strong> 
<a href="http://www.guoxuedashi.com/duilian/" target="_blank">对联大全</a> <strong><a href="http://www.guoxuedashi.com/xiangxingzi/" target="_blank">象形文字典</a></strong> 

<br><a href="http://www.guoxuedashi.com/zixing/yanbian/">字形演变</a>  <strong><a href="http://www.guoxuemi.com/hafo/" target="_blank">哈佛燕京中文善本特藏</a></strong>
<br><strong><a href="http://www.guoxuedashi.com/csfz/" target="_blank">丛书&方志检索器</a></strong> <a href="http://www.guoxuedashi.com/yqjyy/" target="_blank">一切经音义</a>  

<br><strong><a href="http://www.guoxuedashi.com/jiapu/" target="_blank">家谱族谱查询</a></strong>  <strong><a href="http://shufa.guoxuedashi.com/sfzitie/" target="_blank">书法字帖欣赏</a></strong> 
<br>

</div>
</div>


<div class="sidebar" style="margin-bottom:0px;">

<font style="font-size:22px;line-height:32px">QQ交流群9:489193090</font>


<div class="sidebar_title">手机APP 扫描或点击</div>
<div class="sidebar_info">
<table>
<tr>
	<td width=160><a href="http://m.guoxuedashi.com/app/" target="_blank"><img src="/img/gxds-sj.png" width="140"  border="0" alt="国学大师手机版"></a></td>
	<td>
<a href="http://www.guoxuedashi.com/download/" target="_blank">app软件下载专区</a><br>
<a href="http://www.guoxuedashi.com/download/gxds.php" target="_blank">《国学大师》下载</a><br>
<a href="http://www.guoxuedashi.com/download/kxzd.php" target="_blank">《汉字宝典》下载</a><br>
<a href="http://www.guoxuedashi.com/download/scqbd.php" target="_blank">《诗词曲宝典》下载</a><br>
<a href="http://www.guoxuedashi.com/SiKuQuanShu/skqs.php" target="_blank">《四库全书》下载</a><br>
</td>
</tr>
</table>

</div>
</div>


<div class="sidebar2">
<center>


</center>
</div>

<div class="sidebar"  style="margin-bottom:2px;">
<div class="sidebar_title">网站使用教程</div>
<div class="sidebar_info">
<a href="http://www.guoxuedashi.com/help/gjsearch.php" target="_blank">如何在国学大师网下载古籍?</a><br>
<a href="http://www.guoxuedashi.com/zidian/bujian/bjjc.php" target="_blank">如何使用部件查字法快速查字?</a><br>
<a href="http://www.guoxuedashi.com/search/sjc.php" target="_blank">如何在指定的书籍中全文检索?</a><br>
<a href="http://www.guoxuedashi.com/search/skjc.php" target="_blank">如何找到一句话在《四库全书》哪一页?</a><br>
</div>
</div>


<div class="sidebar">
<div class="sidebar_title">热门书籍</div>
<div class="sidebar_info">
<a href="/so.php?sokey=%E8%B5%84%E6%B2%BB%E9%80%9A%E9%89%B4&kt=1">资治通鉴</a> <a href="/24shi/"><strong>二十四史</strong></a>&nbsp; <a href="/a2694/">野史</a>&nbsp; <a href="/SiKuQuanShu/"><strong>四库全书</strong></a>&nbsp;<a href="http://www.guoxuedashi.com/SiKuQuanShu/fanti/">繁体</a>
<br><a href="/so.php?sokey=%E7%BA%A2%E6%A5%BC%E6%A2%A6&kt=1">红楼梦</a> <a href="/a/1858x/">三国演义</a> <a href="/a/1038k/">水浒传</a> <a href="/a/1046t/">西游记</a> <a href="/a/1914o/">封神演义</a>
<br>
<a href="http://www.guoxuedashi.com/so.php?sokeygx=%E4%B8%87%E6%9C%89%E6%96%87%E5%BA%93&submit=&kt=1">万有文库</a> <a href="/a/780t/">古文观止</a> <a href="/a/1024l/">文心雕龙</a> <a href="/a/1704n/">全唐诗</a> <a href="/a/1705h/">全宋词</a>
<br><a href="http://www.guoxuedashi.com/so.php?sokeygx=%E7%99%BE%E8%A1%B2%E6%9C%AC%E4%BA%8C%E5%8D%81%E5%9B%9B%E5%8F%B2&submit=&kt=1"><strong>百衲本二十四史</strong></a>  <a href="http://www.guoxuedashi.com/so.php?sokeygx=%E5%8F%A4%E4%BB%8A%E5%9B%BE%E4%B9%A6%E9%9B%86%E6%88%90&submit=&kt=1"><strong>古今图书集成</strong></a>
<br>

<a href="http://www.guoxuedashi.com/so.php?sokeygx=%E4%B8%9B%E4%B9%A6%E9%9B%86%E6%88%90&submit=&kt=1">丛书集成</a> 
<a href="http://www.guoxuedashi.com/so.php?sokeygx=%E5%9B%9B%E9%83%A8%E4%B8%9B%E5%88%8A&submit=&kt=1"><strong>四部丛刊</strong></a>  
<a href="http://www.guoxuedashi.com/so.php?sokeygx=%E8%AF%B4%E6%96%87%E8%A7%A3%E5%AD%97&submit=&kt=1">說文解字</a> <a href="http://www.guoxuedashi.com/so.php?sokeygx=%E5%85%A8%E4%B8%8A%E5%8F%A4&submit=&kt=1">三国六朝文</a>
<br><a href="http://www.guoxuedashi.com/so.php?sokeytm=%E6%97%A5%E6%9C%AC%E5%86%85%E9%98%81%E6%96%87%E5%BA%93&submit=&kt=1"><strong>日本内阁文库</strong></a> <a href="http://www.guoxuedashi.com/so.php?sokeytm=%E5%9B%BD%E5%9B%BE%E6%96%B9%E5%BF%97%E5%90%88%E9%9B%86&ka=100&submit=">国图方志合集</a> <a href="http://www.guoxuedashi.com/so.php?sokeytm=%E5%90%84%E5%9C%B0%E6%96%B9%E5%BF%97&submit=&kt=1"><strong>各地方志</strong></a>

</div>
</div>


<div class="sidebar2">
<center>

</center>
</div>
<div class="sidebar greenbar">
<div class="sidebar_title green">四库全书</div>
<div class="sidebar_info">

《四库全书》是中国古代最大的丛书,编撰于乾隆年间,由纪昀等360多位高官、学者编撰,3800多人抄写,费时十三年编成。丛书分经、史、子、集四部,故名四库。共有3500多种书,7.9万卷,3.6万册,约8亿字,基本上囊括了古代所有图书,故称“全书”。<a href="http://www.guoxuedashi.com/SiKuQuanShu/">详细>>
</a>

</div> 
</div>

</div>  <!--end r-->

</div>
<!-- 内容区END --> 

<!-- 页脚开始 -->
<div class="shh">

</div>

<div class="w1180" style="margin-top:8px;">
<center><script src="http://www.guoxuedashi.com/img/plus.php?id=3"></script></center>
</div>
<div class="w1180 foot">
<a href="/b/thanks.php">特别致谢</a> | <a href="javascript:window.external.AddFavorite(document.location.href,document.title);">收藏本站</a> | <a href="#">欢迎投稿</a> | <a href="http://www.guoxuedashi.com/forum/">意见建议</a> | <a href="http://www.guoxuemi.com/">国学迷</a> | <a href="http://www.shuowen.net/">说文网</a><script language="javascript" type="text/javascript" src="https://js.users.51.la/17753172.js"></script><br />
  Copyright &copy; 国学大师 古典图书集成 All Rights Reserved.<br>
  
  <span style="font-size:14px">免责声明:本站非营利性站点,以方便网友为主,仅供学习研究。<br>内容由热心网友提供和网上收集,不保留版权。若侵犯了您的权益,来信即刪。scp168@qq.com</span>
  <br />
ICP证:<a href="http://www.beian.miit.gov.cn/" target="_blank">鲁ICP备19060063号</a></div>
<!-- 页脚END --> 
<script src="http://www.guoxuedashi.com/img/plus.php?id=22"></script>
<script src="http://www.guoxuedashi.com/img/tongji.js"></script>

</body>
</html>
