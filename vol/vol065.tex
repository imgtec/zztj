<!DOCTYPE html PUBLIC "-//W3C//DTD XHTML 1.0 Transitional//EN" "http://www.w3.org/TR/xhtml1/DTD/xhtml1-transitional.dtd">
<html xmlns="http://www.w3.org/1999/xhtml">
<head>
<meta http-equiv="Content-Type" content="text/html; charset=utf-8" />
<meta http-equiv="X-UA-Compatible" content="IE=Edge,chrome=1">
<title>資治通鑒_66-資治通鑑卷六十五_66-資治通鑑卷六十五</title>
<meta name="Keywords" content="資治通鑒_66-資治通鑑卷六十五_66-資治通鑑卷六十五">
<meta name="Description" content="資治通鑒_66-資治通鑑卷六十五_66-資治通鑑卷六十五">
<meta http-equiv="Cache-Control" content="no-transform" />
<meta http-equiv="Cache-Control" content="no-siteapp" />
<link href="/img/style.css" rel="stylesheet" type="text/css" />
<script src="/img/m.js?2020"></script> 
</head>
<body>
 <div class="ClassNavi">
<a  href="/24shi/">二十四史</a> | <a href="/SiKuQuanShu/">四库全书</a> | <a href="http://www.guoxuedashi.com/gjtsjc/"><font  color="#FF0000">古今图书集成</font></a> | <a href="/renwu/">历史人物</a> | <a href="/ShuoWenJieZi/"><font  color="#FF0000">说文解字</a></font> | <a href="/chengyu/">成语词典</a> | <a  target="_blank"  href="http://www.guoxuedashi.com/jgwhj/"><font  color="#FF0000">甲骨文合集</font></a> | <a href="/yzjwjc/"><font  color="#FF0000">殷周金文集成</font></a> | <a href="/xiangxingzi/"><font color="#0000FF">象形字典</font></a> | <a href="/13jing/"><font  color="#FF0000">十三经索引</font></a> | <a href="/zixing/"><font  color="#FF0000">字体转换器</font></a> | <a href="/zidian/xz/"><font color="#0000FF">篆书识别</font></a> | <a href="/jinfanyi/">近义反义词</a> | <a href="/duilian/">对联大全</a> | <a href="/jiapu/"><font  color="#0000FF">家谱族谱查询</font></a> | <a href="http://www.guoxuemi.com/hafo/" target="_blank" ><font color="#FF0000">哈佛古籍</font></a> 
</div>

 <!-- 头部导航开始 -->
<div class="w1180 head clearfix">
  <div class="head_logo l"><a title="国学大师官网" href="http://www.guoxuedashi.com" target="_blank"></a></div>
  <div class="head_sr l">
  <div id="head1">
  
  <a href="http://www.guoxuedashi.com/zidian/bujian/" target="_blank" ><img src="http://www.guoxuedashi.com/img/top1.gif" width="88" height="60" border="0" title="部件查字,支持20万汉字"></a>


<a href="http://www.guoxuedashi.com/help/yingpan.php" target="_blank"><img src="http://www.guoxuedashi.com/img/top230.gif" width="600" height="62" border="0" ></a>


  </div>
  <div id="head3"><a href="javascript:" onClick="javascript:window.external.AddFavorite(window.location.href,document.title);">添加收藏</a>
  <br><a href="/help/setie.php">搜索引擎</a>
  <br><a href="/help/zanzhu.php">赞助本站</a></div>
  <div id="head2">
 <a href="http://www.guoxuemi.com/" target="_blank"><img src="http://www.guoxuedashi.com/img/guoxuemi.gif" width="95" height="62" border="0" style="margin-left:2px;" title="国学迷"></a>
  

  </div>
</div>
  <div class="clear"></div>
  <div class="head_nav">
  <p><a href="/">首页</a> | <a href="/ShuKu/">国学书库</a> | <a href="/guji/">影印古籍</a> | <a href="/shici/">诗词宝典</a> | <a   href="/SiKuQuanShu/gxjx.php">精选</a> <b>|</b> <a href="/zidian/">汉语字典</a> | <a href="/hydcd/">汉语词典</a> | <a href="http://www.guoxuedashi.com/zidian/bujian/"><font  color="#CC0066">部件查字</font></a> | <a href="http://www.sfds.cn/"><font  color="#CC0066">书法大师</font></a> | <a href="/jgwhj/">甲骨文</a> <b>|</b> <a href="/b/4/"><font  color="#CC0066">解密</font></a> | <a href="/renwu/">历史人物</a> | <a href="/diangu/">历史典故</a> | <a href="/xingshi/">姓氏</a> | <a href="/minzu/">民族</a> <b>|</b> <a href="/mz/"><font  color="#CC0066">世界名著</font></a> | <a href="/download/">软件下载</a>
</p>
<p><a href="/b/"><font  color="#CC0066">历史</font></a> | <a href="http://skqs.guoxuedashi.com/" target="_blank">四库全书</a> |  <a href="http://www.guoxuedashi.com/search/" target="_blank"><font  color="#CC0066">全文检索</font></a> | <a href="http://www.guoxuedashi.com/shumu/">古籍书目</a> | <a   href="/24shi/">正史</a> <b>|</b> <a href="/chengyu/">成语词典</a> | <a href="/kangxi/" title="康熙字典">康熙字典</a> | <a href="/ShuoWenJieZi/">说文解字</a> | <a href="/zixing/yanbian/">字形演变</a> | <a href="/yzjwjc/">金 文</a> <b>|</b>  <a href="/shijian/nian-hao/">年号</a> | <a href="/diming/">历史地名</a> | <a href="/shijian/">历史事件</a> | <a href="/guanzhi/">官职</a> | <a href="/lishi/">知识</a> <b>|</b> <a href="/zhongyi/">中医中药</a> | <a href="http://www.guoxuedashi.com/forum/">留言反馈</a>
</p>
  </div>
</div>
<!-- 头部导航END --> 
<!-- 内容区开始 --> 
<div class="w1180 clearfix">
  <div class="info l">
   
<div class="clearfix" style="background:#f5faff;">
<script src='http://www.guoxuedashi.com/img/headersou.js'></script>

</div>
  <div class="info_tree"><a href="http://www.guoxuedashi.com">首页</a> > <a href="/SiKuQuanShu/fanti/">四库全书</a>
 > <h1>资治通鉴</h1> <!--         下载:【右键另存为】即可 --></div>
  <div class="info_content zj clearfix">
  
<div class="info_txt clearfix" id="show">
<center style="font-size:24px;">66-資治通鑑卷六十五</center>
    資治通鑑卷六十五   宋 司馬光 撰<br />
<br />
  胡三省 音註<br />
<br />
  漢紀五十七【起柔兆閹茂盡著雍困敦凡三年】<br />
<br />
  孝獻皇帝庚<br />
<br />
  建安十一年春正月有星孛于北斗【晉天文志北斗七星在太微北一曰天樞二曰璇三曰璣四曰權五曰玉衡六曰開陽七曰搖光一至四為魁五至七為杓孛蒲内翻】 曹操自將擊高幹【將即亮翻】留其世子丕守鄴使别駕從事崔琰傳之操圍壺關三月壺關降【降戶江翻】高幹自入匈奴求救單于不受幹獨與數騎亡欲南犇荆州【騎奇寄翻欲奔劉表也】上洛都尉王琰捕斬之【上洛縣前漢屬弘農後漢屬京兆嶢關在縣西北故置都尉劉昫曰言縣在洛水之上故以為名】并州悉平曹操使陳郡梁習以别部司馬領并州刺史時荒亂之餘胡狄雄張【張知亮翻】吏民亡叛入其部落【南匈奴部落皆在并州界】兵家擁衆各為寇害【謂諸豪右擁衆自保者】習到官誘喻招納【誘音酉】皆禮召其豪右稍稍薦舉使詣幕府豪右已盡次發諸丁彊以為義從【言其以義從軍也從才用翻】又因大軍出征令諸將分請以為勇力吏兵已去之後稍移其家前後送鄴凡數萬口其不從命者興兵致討斬首千數降附者萬計單于恭順名王稽顙【名王即匈奴諸部王也降戶江翻稽音啟】服事供軄同於編戶【編聯次也編于民籍故曰編戶】邊境肅清百姓布野勤勸農桑令行禁止【令之則行禁之則止】長老稱詠以為自所聞識刺史未有如習者【長知兩翻】習乃貢達名士避地州界者河内常林楊俊王象荀緯及太原王凌之徒操悉以為縣長【緯于貴翻長知兩翻】後皆顯名於世初山陽仲長統遊學至并州過高幹【仲長複姓過工禾翻】幹善遇之訪以世事統謂幹曰君有雄志而無雄材好士而不能擇人【好呼到翻】所以為君深戒也幹雅自多【自以為多才也】不悦統言統遂去之幹死荀彧舉統為尚書郎【百官志尚書侍郎三十六人四百石一曹六人主作文書起草蔡質漢儀曰尚書郎初從三署詣臺試初上臺稱守尚書郎中歲滿稱尚書郎三年稱侍郎】著論曰昌言【孔安國曰昌當也當理之言】其言治亂略曰豪傑之當天命者未始有天下之分者也【治直吏翻分扶問翻】無天下之分故戰爭者競起焉角智者皆窮角力者皆負【角競也校也】形不堪復伉【復扶又翻伉口浪翻敵也】埶不足復校乃始羈首係頸就我之御紲耳【賢曰銜勒也紲羈也紲息列翻】及繼體之時豪傑之心既絶士民之志已定貴有常家尊在一人當此之時雖下愚之才居之猶能使恩同天地威侔鬼神周孔數千無所復角其聖賁育百萬無所復奮其勇矣【賁音奔】彼後嗣之愚主見天下莫敢與之違自謂若天地之不可亡也乃奔其私嗜騁其邪欲君臣宣淫【左傳泄冶曰公卿宣淫民無効焉杜預曰宣示也】上下同惡荒廢庶政弃忘人物信任親愛者盡佞諂容說之人也【說讀作悦】寵貴隆豐者盡后妃姬妾之家也遂至熬天下之脂膏斵生民之骨髓怨毒無聊禍亂並起中國擾攘四夷侵叛土崩瓦解一朝而去昔之為我哺乳之子孫者今盡是我飲血之寇讐也至于運徙埶去猶不覺悟者豈非富貴生不仁沈溺致愚疾邪【沈持林翻】存亡以之迭代治亂從此周復【左傳美惡周必復天之道也】天道常然之大數也 秋七月武威太守張猛殺雍州刺史邯鄲商【興平元年分凉州河西四郡置雍州雍於用翻】州兵討誅之猛奐之子也 八月曹操東討海賊管承至淳于【淳于縣屬北海國賢曰故城在今密州安邱縣東北】遣將樂進李典擊破之承走入海島 昌豨復叛操遣于禁討斬之【豨許豈翻又音希豨降見上卷建安六年復扶又翻】是歲立故琅邪王容子熙為琅邪王齊北海阜陵下邳常山甘陵濟隂平原八國皆除【容光武子琅邪孝王京之雲孫也容薨國絶今復立其子齊光武兄武王縯之後北海縯少子靖王興之後阜陵光武子質王延之後下邳明帝子惠王衍之後常山明帝子頃王昞之後甘陵章帝子清河孝王慶之後濟隂明帝子悼王長薨而無子國除久矣據范史當是濟北章帝子惠王壽之後亦以是年國除平原和帝子懷王勝始封薨而無子以河間王開子翼繼之翼廢為蠡吾侯子志立為桓帝復以帝兄碩為平原王奉翼後至是國亦除八國皆除而獨立熙繼琅邪者容先遣弟邈至長安貢獻操時在東郡邈盛稱其忠誠操以此德容故為容立後除八國者漸以弱漢宗室也濟子禮翻】 烏桓乘天下亂畧有漢民十餘萬戶袁紹皆立其酋豪為單于【酋慈由翻】以家人子為己女妻焉【妻七細翻】遼西烏桓蹋頓尤彊【蹋徒臘翻】為紹所厚故尚兄弟歸之數入塞為寇【數所角翻】欲助尚復故地曹操將擊之鑿平虜渠泉州渠以通運【操紀云鑿渠自呼沱入?水名平虜渠又從泃河口鑿入潞河名泉州渠以通海?音孤泃音句賢曰呼沱河舊在饒陽南至曹操因饒河故瀆決令北注新溝水所以今在饒陽縣北說文?水出雁門葰人戍夫山東北入海水經注泃水出右北平無終縣西山西北流過平谷縣而東南流又南流入於潞河又東合泉州渠口曹操所鑿也渠東至海陽縣樂安亭南與濡水合而入于海泉州平谷二縣皆屬漁陽郡賢曰泉州故城在今幽州雍奴縣南海陽縣屬遼西郡葰相維翻】 孫權擊山賊麻保二屯平之【水經注江水過陸口而東左得麻屯口南直蒲圻洲水北入百有餘里吴所屯也】十二年春二月曹操自淳于還鄴丁酉操奏封大功臣二十餘人皆為列侯因表萬歲亭侯荀彧功狀【九域志鄭州有萬歲亭彧所封也】三月增封彧千戶又欲授以三公彧使荀攸深自陳讓至于十數乃止 曹操將擊烏桓諸將皆曰袁尚亡虜耳夷狄貪而無親豈能為尚用今深入征之劉備必說劉表以襲許【說輸芮翻】萬一為變事不可悔郭嘉曰公雖威震天下胡恃其遠必不設備因其無備卒然擊之可破滅也【卒讀曰猝】且袁紹有恩於民夷而尚兄弟生存今四州之民徒以威附德施未加【施式豉翻】舍而南征【舍讀曰捨】尚因烏桓之資招其死主之臣【言欲為其主致死而留滯不得逞者】胡人一動民夷俱應以生蹋頓之心成覬覦之計【覬音冀覦音俞】恐青冀非己之有也表坐談客耳自知才不足以御備重任之則恐不能制輕任之則備不為用雖虛國遠征公無憂矣操從之行至易【易縣前漢屬涿郡後漢省宋白曰漢易縣故城在今涿州歸義縣東南十五里大易故城是】郭嘉曰兵貴神速今千里襲人輜重多難以趨利【重直用翻下同趨七喻翻】且彼聞之必為備不如留輜重輕兵兼道以出掩其不意初袁紹數遣使召田疇於無終【疇保無終見六十卷初元四年數所角翻】又即授將軍印使安輯所統疇皆拒之及曹操定冀州河間邢顒謂疇曰黄巾起來二十餘年海内鼎沸百姓流離今聞曹公法令嚴民厭亂矣亂極則平請以身先遂裝還鄉里【顒魚容翻顒從畤游積五年乃歸先悉荐翻】疇曰邢顒天民之先覺者也【伊尹曰予天民之先覺者也此以道自任者也若邢顒之先覺特幾見耳】操以顒為冀州從事疇忿烏桓多殺其本郡冠盖【謂郡中名勝之士】意欲討之而力未能操遣使辟疇疇戒其門下趣治嚴【趣讀曰促嚴即裝也自東都避明帝諱改裝曰嚴後遂因之】門人皆曰袁公慕君禮命五至君義不屈今曹公使一來而君若恐弗及者何也【使疏吏翻下同】疇笑曰此非君所識也遂隨使者到軍拜為蓚令【蓚縣前漢屬信都後漢屬勃海顔師古曰蓚音條】隨軍次無終時方夏水雨而濱海洿下【洿汪胡翻】濘滯不通虜亦遮守蹊要【蹊徑路也蹊要徑路要處也濘乃定翻】軍不得進操患之以問田疇疇曰此道秋夏每常有水淺不通車馬深不載舟船為難久矣舊北平郡治在平岡道出盧龍達于柳城【前漢右北平郡治平岡縣後漢省平岡縣改治土垠縣垠音銀賢曰土垠故城在今平州西南水經注曰自無終東出盧龍塞又東越青陘至凡城二百許里自几城東北出趣平岡可百八十里向黄龍則五百里故田疇引軍出盧龍塞塹山堙谷五百餘里逕白檀歷平岡登白狼山望柳城也】自建武以來陷壞斷絶垂二百載【載子亥翻】而尚有微逕可從今虜將以大軍當由無終不得進而退懈弛無備若嘿回軍從盧龍口越白檀之險出空虛之地路近而便掩其不備蹋頓可不戰而禽也操曰善乃引軍還而署大木表於水側路傍曰方今夏暑道路不通且俟秋冬乃復進軍【復扶又翻】虜候騎見之誠以為大軍去也【騎奇寄翻】操令疇將其衆為鄉導【將即亮翻鄉讀曰嚮】上徐無山【史記正義徐無山在右北平徐無縣西北徐無山即田疇所保聚處】塹山堙谷五百餘里經白檀歷平崗涉鮮卑庭【白檀縣屬右北平郡宋白曰白檀故城在檀州燕樂縣界此時鮮卑庭已在右北平郡界盖慕容廆之先也塹七艷翻】東指柳城未至二百里虜乃知之尚熙與蹋頓及遼西單于樓班【樓班丘力居之子也】右北平單于能臣抵之等【右北平單于曰烏延能臣抵之或者烏延之異名歟】將數萬騎迎軍八月操登白狼山【水經注白狼山在右北平石城縣西烏丸傳逆戰于凡城則白狼山盖在凡城】卒與虜遇【卒讀曰猝】衆甚盛操車重在後【車重即輜重重直用翻】被甲者少左右皆懼【被皮義翻少詩沼翻】操登高望虜陣不整乃縱兵擊之使張遼為前鋒虜衆大崩斬蹋頓及名王已下胡漢降者二十餘萬口【降戶江翻】遼東單于速僕丸【速僕丸即蘇僕延語有輕重耳】與尚熙奔遼東太守公孫康其衆尚有數千騎或勸操遂擊之操曰吾方使康斬送尚熙首不煩兵矣九月操引兵自柳城還公孫康欲取尚熙以為功乃先置精勇于廏中然後請尚熙入未及坐康叱伏兵禽之遂斬尚熙并速僕丸首送之諸將或問操公還而康斬尚熙何也操曰彼素畏尚熙吾急之則并力緩之則自相圖其埶然也操梟尚首【梟古堯翻】令三軍敢有哭之者斬牽招獨設祭悲哭【牽招先為袁氏從事故祭哭之】操義之舉為茂才時天寒且旱二百里無水軍又乏食殺馬數千匹以為糧鑿地入三十餘丈方得水既還科問前諫者【科條也問前諫者科具其姓名也】衆莫知其故人人皆懼操皆厚賞之曰孤前行乘危以徼倖【徼堅堯翻】雖得之天所佐也顧不可以為常諸君之諫萬安之計是以相賞後勿難言之 冬十月辛卯有星孛于鶉尾【蔡邕曰自張十二度至軫六度謂之鶉尾之次陳卓曰自張十七度至軫十一度謂之鶉尾於辰在巳】 乙巳黄巾殺濟南王贇【賢曰贇河間孝王開五代孫靈帝立其父康為濟南王以奉孝仁皇祀濟子理翻贇於倫翻】 十一月曹操至易水烏桓單于代郡普富盧上郡那樓皆來賀師還論功行賞【還從宣翻又如字】以五百戶封田疇為亭侯疇曰吾始為劉公報仇率衆遁逃【事見六十卷初平四年為于偽翻】志義不立反以為利非本志也固讓不受操知其至心許而不奪【不奪其志也孔子曰匹夫不可奪志】操之北伐也劉備說劉表襲許【說輸芮翻】表不能用及聞操還表謂備曰不用君言故為失此大會【猶言大機會也】備曰今天下分裂日尋干戈事會之來豈有終極乎若能應之於後者則此未足為恨也【豪傑之言故自與常人不同】 是歲孫權西擊黄祖虜其人民而還 權母吴氏疾篤引見張昭等屬以後事而卒【屬之欲翻卒子恤翻】 初琅邪諸葛亮寓居襄陽隆中【亮從父玄為豫章太守將亮之官會漢朝以朱皓代玄玄與亮往依劉表漢晉春秋曰亮家于南陽之鄧縣在襄陽城西二十里號曰隆中按東坡詩萬山西北古隆中也故其萬山詩云回頭望西北隱隱龜背起傳云古隆中萬樹桑柘美】每自比管仲樂毅時人莫之許也惟潁川徐庶與崔州平謂為信然州平烈之子也【崔烈事靈帝以入錢為公】劉備在荆州訪士於襄陽司馬徽徽曰儒生俗士豈識時務識時務者在乎俊傑此間自有伏龍鳳雛備問為誰曰諸葛孔明龎士元也【諸葛亮字孔明龎統字士元龎皮江翻】徐庶見備於新野備器之【物之有用者謂之器器之者器重之也重其才之足以用世也】庶謂備曰諸葛孔明臥龍也將軍豈願見之乎備曰君與俱來庶曰此人可就見不可屈致也將軍宜枉駕顧之備由是詣亮【備以梟雄之才聞徐庶一言三枉駕以見孔明此必庶之材器有以取重于備備遂信之也庶自辭備歸操之後寂無所聞今觀其捨舊從新之言質天地而無愧則其人從可知矣】凡三往乃見因屏人曰漢室傾頹姦臣竊命孤不度德量力欲信大義于天下【屏必郢翻度徒洛翻量音良信讀曰申】而智術淺短遂用猖蹶【猖披猖蹶顛蹶】至于今日然志猶未已君謂計將安出亮曰今曹操已擁百萬之衆挾天子而令諸侯此誠不可與爭鋒孫權據有江東已歷三世國險而民附賢能為之用此可與為援而不可圖也荆州北據漢沔利盡南海【謂自桂陽蒼梧跨有交州則利盡南海也】東連吳會【吳會者言吳為東南一都會也】西通巴蜀此用武之國而其主不能守此殆天所以資將軍也益州險塞沃野千里天府之土劉璋闇弱張魯在北民殷國富而不知存恤智能之士思得明君【張松法正之徒雖未與亮交際亮固逆知之也】將軍既帝室之胄【胄裔也】信義著於四海若跨有荆益保其巖阻撫和戎越結好孫權【好呼到翻下同】内修政治外觀時變則覇業可成漢室可興矣【所謂俊傑者量時審勢規畫定于胷中儻非其人未易與之言也治直吏翻】備曰善於是與亮情好日密【好呼到翻】關羽張飛不悦備解之曰孤之有孔明猶魚之有水也【魚有水則生無水則死】願諸君勿復言【復扶又翻】羽飛乃止司馬徽清雅有知人之鑒同縣龎德公素有重名徽兄事之諸葛亮每至德公家獨拜牀下德公初不令止【觀孔明獨拜德公於牀下孔明所以事德公者為何如邪德公初不令止德公所以自居者為何如邪德公於是不可及矣】德公從子統少時樸鈍未有識者【從才用翻少詩沼翻】惟德公與徽重之德公嘗謂孔明為臥龍士元為鳳雛德操為水鑑故德操與劉備語而稱之【司馬徽字德操】<br />
<br />
  十三年春正月司徒趙温辟曹操子丕操表温辟臣子弟選舉故不以實策免之【操以温辟其子怒而免之駕言選舉不以實耳 考異曰獻帝起居注在十五年范書帝紀在十三年按是年罷三公官温不至十五年也】 曹操還鄴作玄武池以肄舟師【鄴城有玄武苑操鑿池其中肄以四翻習也】 初巴郡甘寜將僮客八百人歸劉表【寧走荆州事見六十一卷興平元年】表儒人不習軍事寧觀表事埶終必無成恐一朝衆散并受其禍【聚而不用其禍必至】欲東入吳黄祖在夏口【應劭曰沔水自江夏别至南郡華容為夏水過江夏郡而入于江盖指夏水入江之地為夏口庾仲雍曰夏口一曰沔口或曰魯口水經注曰沔水南至江夏沙羨縣北南入于江然則曰夏口以夏水得名曰沔口以沔水得名曰魯口以魯山得名實一處也其地在江北自孫權置夏口督屯江南今鄂州治是也故何尚之云夏口在荆江之中正對沔口賢注亦謂夏口戍在今鄂州于是相承以鄂州為夏口而江北之夏口晦矣】軍不得過乃留依祖三年祖以凡人畜之【畜許六翻養也】孫權擊祖祖軍敗走權校尉凌操將兵急追之【姓譜衛康叔支子為周凌人子孫以為氏】寧善射將兵在後射殺操【射殺之射而亦翻】祖由是得免軍罷還營待寧如初祖都督蘇飛數薦寧【數所角翻】祖不用寧欲去恐不免飛乃白祖以寧為邾長【邾縣屬江夏郡地道記曰楚滅邾徙其君於此賢曰邾故城在今復州竟陵縣東飛盖開其奔吳之路也長知兩翻宋白曰黄州漢邾縣也】寧遂亡奔孫權 【考異曰吴志孫權傳建安八年十二年皆嘗討黄祖凌統傳父操死時統年十五攝父兵後擊麻保屯刺殺陳勤按周瑜孫瑜傳以十一年擊麻保屯則操死似在八年然後五年寧乃奔權似晚今無年月可據追言之】周瑜呂蒙共薦達之權禮異同於舊臣寧獻策於權曰今漢祚日微曹操終為簒盜南荆之地山川形便誠國之西埶也【謂在吳之西據上流之形埶】寧觀劉表慮既不遠兒子又劣【言又弱於表也】非能承業傳基者也至尊當早圖之不可後操【言若不先圓劉表必為操所圖也後戶遘翻】圖之之計宜先取黄祖祖今昏耄已甚財穀並乏左右貪縱吏士心怨舟船戰具頓廢不修【頓壞也左傳甲兵不頓頓讀曰鈍】怠於耕農軍無法伍至尊今往其破可必一破祖軍鼓行而西據楚關【楚關扞關也蜀伐楚楚為扞關以拒之故曰楚關】大埶彌廣即可漸規巴蜀矣權深納之張昭時在坐難曰今吳下業業【坐徂臥翻難乃旦翻業業危懼之意】若軍果行恐必致亂寧謂昭曰國家以蕭何之任付君君居首而憂亂奚以希慕古人乎【言固有攸當者張昭不得以彊辭距也守式又翻】權舉酒屬寧曰興霸【甘寧字興霸屬之欲翻】今年行討如此酒矣决以付卿卿但當勉建方畧令必克祖則卿之功何嫌張長史之言乎【昭為權長史權之此言既以奬甘寧之氣又以全張昭之體不有居者誰守社稷不有行者誰扞牧圉長知兩翻】權遂西擊黄祖祖横兩蒙衝【釋名曰船狹而長曰蒙衝以衝突敵船】挾守沔口以栟閭大紲繫石為矴【栟閭椶櫚也郭璞曰落穫也中作器索栟卑盈翻紲音薛長繩也矴丁定翻錘舟石】上有千人以弩交射飛矢雨下軍不得前偏將軍董襲與别部司馬凌統俱為前部各將敢死百人人被兩鎧乘大舸【將即亮翻被皮義翻方言南楚江湘凡船大者謂之舸小者謂之艖舸嘉我翻】突入蒙衝裏襲身以刀斷兩紲【斷丁管翻】蒙衝乃横流大兵遂進祖令都督陳就以水軍逆戰平北都尉呂蒙【蒙自别部司馬以功為平北都尉】勒前鋒親梟就首【梟堅堯翻】于是將士乘勝水陸並進傳其城【傅讀曰附】盡鋭攻之遂屠其城祖挺身走追斬之【挺拔也】虜其男女數萬口權先作兩函欲以盛祖及蘇飛首【盛時征翻】權為諸將置酒甘寧下席叩頭血涕交流為權言飛疇昔舊恩【舊恩謂薦而不用又開之使奔吳也為于偽翻下同】寧不值飛固已捐骸於溝壑不得致命於麾下今飛罪當夷戮特從將軍乞其首領權感其言謂曰今為君置之若走去何寧曰飛免分裂之禍受更生之恩逐之尚必不走豈當圖亡哉【亡謂亡走】若爾【爾猶言如此也】寧頭當代入函權乃赦之凌統怨寧殺其父操常欲殺寧權命統不得讐之令寧將兵屯於它所 夏六月罷三公官復置丞相御史大夫【漢初以丞相御史大夫太尉為三公哀帝元夀二年以大司馬大司徒大司空為三公中興以來以太尉司徒司空為三公今復置丞相御史而操自為丞相事權出于一矣】癸巳以曹操為丞相操以冀州别駕從事崔琰為丞相西曹掾司空東曹掾陳留毛玠為丞相東曹掾元城令河内司馬朗為主簿弟懿為文學掾冀州主簿盧為法曹議令史【别駕從事州牧行部則奉引錄衆事漢制公府西曹掾主府史署用東曹掾主二千石長吏遷除及軍吏黄閣主簿錄省衆事文學掾漢郡曹有之操於公府創置也法曹主郵驛科程事時公府諸曹皆置議令史元城縣屬魏郡】毓植之子也琰玠並典選舉其所舉用皆清正之士雖於時有盛名而行不由本者終莫得進拔敦實斥華偽進冲遜抑阿黨【行下孟翻冲謙虚也和也】由是天下之士莫不以亷節自勵雖貴寵之臣輿服不敢過度至乃長吏還者垢面羸衣獨乘柴車軍吏入府朝服徒行【長知兩翻朝直遙翻】吏潔於上俗移於下操聞之歎曰用人如此使天下人自治吾復何為哉【復扶又翻】司馬懿少聰達多大略【少詩沼翻】崔琰謂其兄朗曰君弟聰亮明允剛斷英特【斷丁亂翻】非子所及也操聞而辟之懿辭以風痺【痺必至翻濕病也】操怒欲收之懿懼就職【司馬懿始此】 操使張遼屯長社臨發軍中有謀反者夜驚亂起火一軍盡擾遼謂左右曰勿動是不一營盡反必有造變者欲以驚動人耳乃令軍中其不反者安坐遼將親兵數十人中陳而立【將即亮翻陳讀曰陣】有頃皆定即得首謀者殺之遼在長社于禁屯潁隂樂進屯陽翟三將任氣多共不恊【共相與也】操使司空主簿趙儼并參三軍每事訓諭遂相親睦 初前將軍馬騰與鎮西將軍韓遂結為異姓兄弟【晉職官志曰四鎮通于柔遠盖漢末始置也】後以部曲相侵更為讐敵朝廷使司隸校尉鍾繇凉州刺史韋端和解之徵騰入屯槐里曹操將征荆州使張既說騰令釋部曲還朝【說輸芮翻】騰許之已而更猶豫既恐其為變乃移諸縣促儲偫【偫直里翻】二千石郊迎騰不得已發東【發而東入朝也】操表騰為衛尉 【考異曰典畧曰建安十五年徵騰為衛尉按張既傳曹公將征荆州今既說騰入朝盖三字誤為五耳】以其子超為偏將軍統其衆悉徙其家屬詣鄴【為後十七年族騰張本】 秋七月曹操南擊劉表 八月丁未以光祿勲山陽郗慮為御史大夫【郗丑脂翻姓譜郗為高平望姓】 壬子太中大夫孔融弃市融恃其才望數戲侮曹操【數所角翻】發辭偏宕【賢曰偏邪跌宕不拘正理予謂此偏非邪之謂言其論議抑揚有所偏重也宕徒浪翻過也】多致乖忤【忤五故翻】操以融名重天下外相容忍而内甚嫌之融又上書宜準古王畿之制千里寰内不以封建諸侯操疑融所論建漸廣益憚之【周禮方千里曰國畿其外方五百里曰侯畿鄭玄曰畿限也千里寰内不以封建則操不可以居鄴矣故憚之】融與郗慮有隙慮承操風旨構成其罪令丞相軍謀祭酒路粹【軍師祭酒軍謀祭酒皆操所置】奏融昔在北海【建安初融為北海相】見王室不靜而招合徒衆欲規不軌及與孫權使語謗訕朝廷【使疏吏翻】又前與白衣禰衡跌蕩放言【賢曰跌蕩無儀檢也故縱也禰乃禮翻】更相贊揚【更工衡翻】衡謂融曰仲尼不死融答顏回復生【復扶又翻】大逆不道宜極重誅操遂收融并其妻子皆殺之初京兆脂習與融善【脂姓也魏畧脂習字元升後為中大夫】每戒融剛直太過必罹世患及融死許下莫敢收者習往撫尸曰文舉舍我死【孔融字文舉舍讀曰捨】吾何用生為操收習欲殺之既而赦之 初劉表二子琦琮【琦居宜翻琮徂宗翻】表為琮娶其後妻蔡氏之姪蔡氏遂愛琮而惡琦【為于偽翻惡烏路翻】表妻弟蔡瑁【瑁莫報翻】外甥張允並得幸于表日相與毁琦而譽琮【譽音余】琦不自寧與諸葛亮謀自安之術亮不對後乃共升高樓因令去梯【去羌呂翻】謂亮曰今日上不至天下不至地言出子口而入吾耳可以言未亮曰君不見申生在内而危重耳居外而安乎【申生晉獻公之太子為驪姬所譛自縊而死重耳申生之弟懼驪姬之讒出奔獻公卒後重耳入是為文公遂為覇主重直龍翻】琦意感悟隂規出計會黄祖死琦求代其任表乃以琦為江夏太守【夏戶雅翻】表病甚琦歸省疾【省悉景翻】瑁允恐其見表而父子相感更有託後之意乃謂琦曰將軍命君撫臨江夏其任至重今釋衆擅來必見遣怒傷親之歡重增其疾【重直用翻】非孝敬之道也遂遏於戶外使不得見琦流涕而去表卒瑁允等遂以琮為嗣琮以侯印授琦琦怒投之地將因犇喪作難【難乃旦翻】會曹操軍至琦犇江南【按劉備敗於當陽濟沔與琦會然後俱到夏口琦奔江南在劉琮降後史究其終言之】章陵太守蒯越【四親園廟在章陵時以為郡置守】及東曹掾傳巽等勸劉琮降操【降戶江翻下同】曰逆順有大體強弱有定埶以人臣而拒人主逆道也以新造之楚而禦中國必危也以劉備而敵曹公不當也【當如字言不敵也】三者皆短將何以待敵且將軍自料何如劉備若備不足禦曹公則雖全楚不能以自存也若足禦曹公則備不為將軍下也琮從之 【考異曰范書陳志表傳皆云韓嵩亦說琮降按嵩時被囚必不預謀】九月操至新野琮遂舉州降以節迎操【節漢節也琮父表受之於漢】諸將皆疑其詐婁圭曰天下擾擾各貪王命以自重今以節來是必至誠操遂進兵時劉備屯樊【樊城在襄陽東北臨漢水周大夫樊仲山甫之邑也唐為襄州安養縣】琮不敢告備備久之乃覺遣所親問琮琮令官屬宋忠詣備宣旨時曹操已在宛備乃大驚駭謂忠曰卿諸人作事如此不早相語【語牛倨翻】今禍至方告我不亦太劇乎【劇甚也】引刀向忠曰今斷卿頭【斷丁管翻】不足以解忿亦恥丈夫臨别復殺卿輩【復扶又翻】遣忠去乃呼部曲共議或勸備攻琮荆州可得備曰劉荆州臨亡託我以孤遺【無父曰孤遺弃也言父母弃之而去故曰孤遺今人謂孤獨無所依仰者為孤遺】背信自濟【背蒲妹翻】吾所不為死何面目以見劉荆州乎備將其衆去過襄陽【將即亮翻過工禾翻下同】駐馬呼琮琮懼不能起琮左右及荆州人多歸備備過辭表墓涕泣而去比到當陽【比必寐翻當陽縣屬南郡】衆十餘萬人輜重數千兩【重直用翻兩音亮】日行十餘里别遣關羽乘船數百艘【艘蘇刀翻】使會江陵或謂備曰宜速行保江陵【江陵南郡治所】今雖擁大衆被甲者少【被皮義翻少詩沼翻】若曹公兵至何以拒之備曰夫濟大事必以人為本今人歸吾吾何忍弃去<br />
<br />
  習鑿齒論曰劉玄德雖顛沛險難而信義愈明【難乃旦翻】埶偪事危而言不失道追景升之顧則情感三軍戀赴義之士則甘與同敗終濟大業不亦宜乎<br />
<br />
  劉琮將王威說琮曰【說輸芮翻】曹操聞將軍既降劉備已走必懈弛無備輕行單進若給威奇兵數千徼之於險操可獲也【徼一遙翻】獲操即威震四海非徒保守今日而已琮不納【使琮用威言操其殆哉】操以江陵有軍實【軍實糧儲器械之類】恐劉備據之乃釋輜重【重直用翻】輕軍到襄陽聞備已過操將精騎五千急追之一日一夜行三百餘里及於當陽之長坂【當陽長坂在今荆門軍當陽縣東南百二十里荆州記云當陽縣東有櫟林長坂宋白曰漢當陽舊城在今縣北春秋傳楚伐麋頴容釋例曰麋當陽也孔頴達曰陂者曰坂陂彼寄翻又普羅翻李巡曰陂者謂高峯山坡】備棄妻子與諸葛亮張飛趙雲等數十騎走操大獲其人衆輜重徐庶母為操所獲庶辭備指其心曰本欲與將軍共圖王霸之業者以此方寸之地也今已失老母方寸亂矣無益於事請從此别遂詣操張飛將二十騎拒後【拒後即古之殿也】飛據水斷橋瞋目横矛曰身是張益德也【瞋七人翻自此迄于梁陳士大夫率自謂曰身張飛字益德】可來共决死操兵無敢近者或謂備趙雲已北走備以手戟擿之曰子龍不弃我走也【擿讀與擲同趙雲字子龍】頃之雲身抱備子禪與關羽船會得濟沔遇劉琦衆萬餘人與俱到夏口曹操進軍江陵以劉琮為青州刺史封列侯并蒯越等侯者凡十五人釋韓嵩之囚【囚韓嵩事見六十三卷建安四年】待以交友之禮使條品州人優劣皆擢而用之以嵩為大鴻臚蒯越為光祿勲劉先為尚書鄧羲為侍中荆州大將南陽文聘别屯在外琮之降也呼聘欲與俱聘曰聘不能全州當待罪而已操濟漢【漢即沔也漢書地理志注曰東漢水受氐道水一名沔過江夏謂之夏水入江如淳曰漢中人謂漢水為沔水師古曰漢上曰沔祝穆曰天下之大川以漢名者二班固謂之東漢西漢而黎州之漢水源於飛越嶺者不與焉固之所謂東漢則禹貢之漾漢其源出于今興元之西縣嶓冢山逕洋金房均襄郢復至漢陽入江者是也西漢則蘇代所謂漢中之甲輕舟出于巴乘夏水下漢四日而至五渚者其源出于西和州徼外徑階沔州與嘉陵水會俗謂之西漢又徑大安軍利劒閬果合與涪水會至渝州入江】聘乃詣操操曰來何遲邪聘曰先日不能輔弼劉荆州以奉國家荆州雖沒常願據守漢川保全土境生不負於孤弱死無愧於地下而計不在己以至于此實懷悲慙無顏早見耳遂歔欷流涕【歔音虚欷許既翻又音希】操為之愴然【為于偽翻愴七亮翻】字謂之曰仲業【文聘字仲業】卿真忠臣也厚禮待之使統本兵為江夏太守初袁紹在冀州遣使迎汝南士大夫西平和洽【姓譜和本羲和之後一云卞和之後】以為冀州土平民彊英桀所利不如荆州土險民弱易依倚也【易以䜴翻】遂從劉表表以上客待之洽曰所以不從本初辟爭地也【辟讀曰避】昏世之主不可黷近【近其靳翻】久而不去讒慝將興遂南之武陵表辟南陽劉望之為從事而其友二人皆以讒毁為表所誅望之又以正諫不合投傳告歸【傳株戀翻】望之弟廙謂望之曰【廙逸職翻又羊至翻】趙殺鳴犢仲尼回輪【史記孔子將西見趙簡子至河而聞竇鳴犢舜華之死臨河而歎曰丘之不濟命也夫子貢進曰何謂也孔子曰竇鳴犢舜華晉之賢大夫也趙簡子未得志之時須此兩人而後從政丘聞之刳胎殺天則麒麟不至竭澤涸漁則蛟龍不合隂陽覆巢毀卵則鳳凰不翔何則君子諱傷其類夫鳥獸之於不義也尚知避之而况乎丘哉乃還】今兄既不能法柳下惠和光同塵於内【柳下惠為士師三黜而不去孟子曰柳下惠不羞汙君不卑小官遺佚而不怨阨窮而不憫故曰爾為爾我為我雖袒裼祼裎於我側爾焉能凂我哉故由由然與之偕而不自失焉所謂和光同塵也】則宜模范蠡遷化於外【謂范蠡去越而扁舟五湖卒居於陶隨其所遷而自為變化也】坐而自絶於時殆不可也望之不從尋復見害【復扶又翻】廙犇揚州南陽韓暨避袁術之命徙居山都山劉表又辟之遂遁居孱陵【山都山在南陽郡山都縣孱陵縣屬武陵郡後劉備改曰公安賢曰孱陵故城在荆州公安縣西南孱士顔翻】表深恨之暨懼應命除宜城長河東裴潛亦為表所禮重潛私謂王暢之子粲及河内司馬芝曰劉牧非霸王之才【王于况翻】乃欲西伯自處【處昌呂翻】其敗無日矣遂南適長沙于是操以暨為丞相士曹屬【丞相府有戶曹賊曹兵曹鎧曹士曹掾屬各一人兵鎧十三曹盖操置也】潛參丞相軍事【時方用兵故丞相府置參軍事職官分紀漢三公府有參軍事盖亦謂此時所置耳】洽廙粲皆為掾屬【漢公府並有掾屬東西曹掾比四百石餘曹比三百石其屬比二百石三公為天子之股肱掾屬則三公之喉舌魏晉置多者或數十人】芝為菅令【菅縣屬濟南郡應劭曰菅音姦考異曰粲傳曰太祖置酒漢濱粲奉觴賀云云按操恐劉備據江陵至襄陽即過日行三百里引用名士皆】<br />
<br />
  【至江陵後所為不得更置酒漢濱恐誤】從人望也 冬十月癸未朔日有食之 初魯肅聞劉表卒言於孫權曰荆州與國鄰接江山險固沃野萬里士民殷富若據而有之此帝王之資也今劉表新亡二子不協軍中諸將各有彼此【謂有附琦者有附琮者】劉備天下梟雄與操有隙【梟堅堯翻前書張良曰九江王布楚梟將師古曰梟言最勇健也有隙謂備欲殺操不遂也】寄寓於表表惡其能而不能用也【惡烏路翻】若備與彼協心上下齊同則宜撫安與結盟好【好呼到翻】如有離違【離違言人有離心互相違異也】宜别圖之以濟大事肅請得奉命弔表二子并慰勞其軍中用事者及說備使撫表衆【勞力到翻說輸芮翻下同】同心一意共治曹操【治直之翻】備必喜而從命如其克諧天下可定也今不速往恐為操所先權即遣肅行到夏口聞操已向荆州晨夜兼道比至南郡【比必寐翻】而琮已降備南走肅徑迎之與備會于當陽長坂肅宣權旨論天下事勢致殷勤之意且問備曰豫州今欲何至【備先為豫州牧故以稱之】備曰與蒼梧太守吳巨有舊欲往投之肅曰孫討虜聰明仁惠敬賢禮士江表英豪咸歸附之【曹操表權為討虜將軍故稱之】已據有六郡兵精糧多足以立事今為君計莫若遣腹心自結於東以共濟世業【荆州在西吳在東世業猶言世事也】而欲投吳巨巨是凡人偏在遠郡行將為人所併豈足託乎備甚悦肅又謂諸葛亮曰我子瑜友也即共定交子瑜者亮兄瑾也【諸葛瑾字子瑜瑾渠吝翻】避亂江東為孫權長史備用肅計進住鄂縣之樊口【住止軍也水經注江水過鄂縣北而東流右得樊口樊山下寒溪水所注也陸游曰黄州與樊口正相對郡國志鄂縣屬江夏郡孫策破黄祖於此改曰武昌今夀昌軍是也通鑑以為孫權所改】曹操自江陵將順江東下諸葛亮謂劉備曰事急矣請奉命求救於孫將軍遂與魯肅俱詣孫權亮見權於柴桑【柴桑縣屬豫章郡晉置尋陽郡於江南即此柴桑縣地也今江州德化縣西南九十里冇柴桑山】說權曰【說式芮翻】海内大亂將軍起兵江東劉豫州收衆漢南與曹操共爭天下今操芟夷大難畧已平矣【杜預曰芟刈也夷殺也芟所銜翻難乃旦翻下同】遂破荆州威震四海英雄無用武之地故豫州遁逃至此願將軍量力而處之【量音良處昌呂翻】若能以吳越之衆與中國抗衡【衡以取平上下相當無所卑屈曰抗】不如蚤與之絶若不能何不按兵束甲北面而事之今將軍外托服從之名而内懷猶豫之計事急而不斷【斷丁亂翻】禍至無日矣權曰苟如君言劉豫州何不遂事之乎亮曰田横齊之壯士耳猶守義不辱【事見十一卷漢高帝五年】况劉豫州王室之胄【冑系也】英才盖世衆士慕仰若水之歸海若事之不濟此乃天也安能復為之下乎【復扶又翻】權勃然曰【勃然作色慍怒也】吾不能舉全吳之地十萬之衆受制於人吾計决矣非劉豫州莫可以當曹操者然豫州新敗之後安能抗此難乎【難乃旦翻】亮曰豫州軍雖敗於長坂今戰士還者及關羽水軍精甲萬人劉琦合江夏戰士亦不下萬人曹操之衆遠來疲敝聞追豫州輕車一日一夜行三百餘里此所謂強弩之末勢不能穿魯縞者也【前書韓安國曰衝風之衰不能起毛羽強弩之末力不能入魯縞師古注曰縞素也曲阜之地俗善作之尤為輕細故以取喻也】故兵法忌之曰必蹶上將軍【兵法百里而趨利者蹶上將】且北方之人不習水戰又荆州之民附操者偪兵埶耳非心服也今將軍誠能命猛將統兵數萬與豫州協規同力破操軍必矣操軍破必北還如此則荆吳之埶強鼎足之形成矣【荆謂備吳謂權鼎足之形謂三分天下也】成敗之機在於今日權大悦與其羣下謀之是時曹操遺權書曰【遺于李翻】近者奉辭伐罪旌麾南指劉琮束手今治水軍八十萬衆方與將軍會獵於吳【治直之翻】權以示臣下莫不響震失色長史張昭等曰曹公豺虎也挾天子以征四方動以朝廷為辭今日拒之事更不順且將軍大埶可以拒操者長江也今操得荆州奄有其地劉表治水軍蒙衝鬬艦乃以千數【杜佑曰蒙衝以生牛皮蒙船覆背兩廂開掣棹孔左右有弩窗矛穴敵不得近矢石不能敗此不用大船務於速疾乘人之所不及非戰之船也鬭艦船上設女牆可高三尺牆下開製棹孔船内五尺又建棚與女牆齊棚上又建女牆重列戰敵上無覆背前後左右樹牙旗幟旛金鼓此戰船也艦戶黯翻】操悉浮以沿江兼有步兵水陸俱下此為長江之險已與我共之矣而埶力衆寡又不可論愚謂大計不如迎之魯肅獨不言權起更衣【更工衡翻】肅追於宇下【韓詩曰屋霤為宇陸德明曰屋四垂為宇又隤下曰宇考工記曰宇欲卑】權知其意執肅手曰卿欲何言肅曰向察衆人之議專欲誤將軍不足與圖大事今肅可迎操耳如將軍不可也何以言之今肅迎操操當以肅還付鄉黨品其名位猶不失下曹從事【下曹從事諸曹從事之最下者】乘犢車【晉志曰犢車牛車也古之貴者不乘牛車漢武帝推恩之末諸侯寡弱貧者至乘牛車其後稍貴之自靈獻以來天子至士遂為常乘】從吏卒交游士林【士林多士之林謂京邑大都四方賢士所聚也】累官故不失州郡也將軍迎操欲安所歸乎願早定大計莫用衆人之議也權歎息曰諸人持議甚失孤望今卿廓開大計正與孤同時周瑜受使至番陽肅勸權召瑜還【瑜已受命出使盖行未遠也使疏吏翻番蒲何翻】瑜至謂權曰操雖託名漢相其實漢賊也【相息亮翻】將軍以神武雄才兼仗父兄之烈割據江東地方數千里兵精足用英雄樂業【英雄之事猶樂其業言無它志也樂音洛】當横行天下為漢家除殘去穢【為于偽翻去羌呂翻】况操自送死而可迎之邪請為將軍籌之【為于偽翻下保為同】今北土未平馬超韓遂尚在關西為操後患而操舍鞍馬仗舟楫與吳越爭衡【舍讀曰捨北人便於鞍馬南人便於舟楫言操舍長就所短】今又盛寒馬無藁草【說文曰禾莖為藁音工老翻】驅中國士衆遠涉江湖之間不習水土必生疾病此數者用兵之患也而操皆冒行之將軍禽操宜在今日瑜請得精兵數萬人進住夏口【前書地理志曰夏水過江夏郡入江水水經注曰黄鵠山東北對夏口城亦沙羨縣治盖齊梁之魯山城今之漢陽軍即其地所謂漢口也祝穆曰夏口一名魯口似指漢水之口然何尚之云夏口在荆江之中正對沔口而章懷太子亦謂夏口戍在鄂州故唐史皆指鄂州為夏口盖本在江北自孫權取對岸夏口之名以名之而江北之名始晦】保為將軍破之權曰老賊欲廢漢自立久矣徒忌二袁呂布劉表與孤耳今數雄已滅惟孤尚存孤與老賊勢不兩立君言當擊甚與孤合此天以君授孤也因抜刀砍前奏案曰諸將吏敢復有言當迎操者與此案同【言欲斬之也復扶又翻】乃罷會是夜瑜復見權曰諸人徒見操書言水步八十萬而各恐懾不復料其虚實便開此議甚無謂也【謂迎操之議也懾之涉翻】今以實校之彼所將中國人不過十五六萬且已久疲【將即亮翻】所得表衆亦極七八萬耳尚懷狐疑夫以疲病之卒御狐疑之衆【言新附之人心懷狐疑未能出死命而為之力戰也】衆數雖多甚未足畏瑜得精兵五萬自足制之願將軍勿慮權撫其背曰公瑾卿言至此甚合孤心子布元表諸人【秦松字文表元恐當作文】各顧妻子挾持私慮深失所望獨卿與子敬與孤同耳【魯肅字子敬】此天以卿二人贊孤也五萬兵難卒合【卒讀曰猝】已選三萬人船糧戰具俱辦卿與子敬程公【程公程普也時江東諸將普年最長人皆呼程公】便在前發孤當續發人衆多載資糧為卿後援卿能辦之者誠决【謂能辦操則誠為能决勝也】邂逅不如意【不期而會曰邂逅謂兵之勝負或有不如本心之所期者也】便還就孤孤當與孟德决之遂以周瑜程普為左右督將兵與備并力逆操【將即亮翻】以魯肅為贊軍校尉【使之贊軍謀因以為官稱】助畫方略劉備在樊口日遣邏吏於水次候望權軍【邏即佐翻巡也】吏望見瑜船馳往白備備遣人慰勞之【勞力到翻】瑜曰有軍任不可得委署【委弃也署置也】儻能屈威【謂能自屈其威而來見】誠副其所望備乃乘單舸往見瑜曰今拒曹公深為得計戰卒有幾【舸古我翻幾居豈翻】瑜曰三萬人備曰恨少【少詩沼翻】瑜曰此自足用豫州但觀瑜破之備欲呼魯肅等共會語瑜曰受命不得妄委署若欲見子敬可别過之【過音戈詩云不我過杜甫詩吟詩許見過皆從平聲】備深愧喜【愧者自愧呼肅之非喜者喜瑜之整也】進與操遇於赤壁【水經注江水自沙羨而東右逕赤壁山北郡縣志赤壁山在蒲圻西百二十里北岸烏林與赤壁相對即周瑜用黄盖策焚曹公船處杜佑曰赤壁在鄂州蒲圻縣武昌志曰曹操自江陵追劉備至巴丘遂至赤壁遇周瑜兵大敗取華容道歸赤壁山在今嘉魚縣對江北之烏林巴丘今巴陵華容今石首也黄州赤壁非是今之華容縣則晉之安南縣也】時操軍衆已有疾疫初一交戰操軍不利引次江北瑜等在南㟁瑜部將黄盖曰今寇衆我寡難與持久操軍方連船艦首尾相接可燒而走也乃取蒙衝鬬艦十艘【艘蘇曹翻船之總名】載燥狄枯柴灌油其中裹以帷幙上建旌旗豫備走舸繫於其尾【杜佑曰走舸舷上立女牆置棹夫多戰卒少皆選勇力精鋭者往返如飛鷗乘人之所不及金鼓旗幟列之於上此戰船也】先以書遺操【遺于季翻】詐云欲降【降戶江翻下同】時東南風急盖以十艦最著前【著直畧翻】中江舉帆餘船以次俱進操軍吏士皆出營立觀指言盖降去北軍二里餘同時發火火烈風猛船往如箭燒盡北船延及岸上營落頃之烟炎張天【炎與燄同以贍翻張知亮翻】人馬燒溺死者甚衆瑜等率輕鋭繼其後靁鼓大震【靁盧對翻疾擊鼓也】北軍大壞操引軍從華容道步走【華容縣屬南郡從此道可至華容縣也杜佑曰古華容在竟陵郡監利縣】遇泥濘道不通【濘乃定翻】天又大風悉使羸兵負屮填之騎乃得過羸兵為人馬所蹈藉陷泥中死者甚衆【羸倫為翻】劉備周瑜水陸並進追操至南郡時操軍兼以飢疫死者大半操乃留征南將軍曹仁横野將軍徐晃守江陵【横野大將軍光武以命王常】折衝將軍樂進守襄陽【折衝將軍始此】引軍北還周瑜程普將數萬衆與曹仁隔江未戰甘寧請先徑進取夷陵往即得其城因入守之益州將襲肅舉軍降【先取夷陵則與益州為鄰故襲肅舉軍以降襲姓肅名】周瑜表以肅兵益横野中郎將呂蒙【横野本將軍號以資序未至故為中郎將】蒙盛稱肅有膽用且慕化遠來於義宜益不宜奪也權善其言還肅兵曹仁遣兵圍甘寧寧困急求救于周瑜諸將以為兵少不足分呂蒙謂周瑜程普曰留凌公績於江陵【凌統字公績】蒙與君行解圍釋急埶亦不久蒙保公績能十日守也瑜從之大破仁兵于夷陵獲馬三百匹而還于是將士形勢自倍瑜乃渡江屯北岸與仁相拒十二月孫權自將圍合肥【合肥曹操置揚州刺史治焉時刺史已移治夀春 考異曰魏志武紀十二月權為備攻合肥公自江陵征備至巴丘遣張喜救合肥權聞喜至乃走公至赤壁與備戰不利孫盛異同評曰按吴志備先破公軍然後權攻合肥而北紀云先攻合肥後有赤壁之事二者不同吴志為是又陳矯傳云陳登為權所圍于匡奇令矯求救於曹操而先賢行狀云登為策所圍按策始欲攻登未濟江已為許貢客所殺吴書云權征合肥命張昭别討匡奇於時陳矯已為曹仁長史又陳登年三十六而卒必已不在不知登之被圍果在何時也】使張昭攻九江之當塗不克【此古當塗縣也】劉備表劉琦為荆州刺史引兵南徇四郡武陵太守金旋長沙太守韓玄桂陽太守趙範零陵太守劉度皆降廬江營帥靁緒率部曲數萬口歸備【帥所類翻】備以諸葛亮為軍師中郎將【軍師亦古將軍號曹操初置軍師祭酒而備置軍師中郎將皆以一時軍事創置官名也然軍師祭酒止决軍謀中郎將則有兵柄亮後又進軍師將軍】使督零陵桂陽長沙三郡調其賦税以充軍實【調徒弔翻】以偏將軍趙雲領桂陽太守 益州牧劉璋聞曹操克荆州遣别駕張松致敬於操松為人短小放蕩然識達精果操時已定荆州走劉備不復存録松【復扶又翻】主簿楊修白操辟松操不納松以此怨歸勸劉璋絶操與劉備相結璋從之【為後十六年璋迎備張本】<br />
<br />
  習鑿齒論曰昔齊桓一矜其功而叛者九國【公羊傳曰葵丘之會桓公震而矜之叛者九國】曹操暫自驕伐而天下三分皆勤之於數十年之内而弃之於俯仰之頃豈不惜乎<br />
<br />
  曹操追念田疇功恨前聽其讓【事見上十二年】曰是成一人之志而虧王法大制也乃復以前爵封疇【復扶又翻下同】疇上疏陳誠以死自誓操不聽欲引拜之至於數四終不受有司劾疇狷介違道苟立小節宜免官加刑操下世子及大臣博議【劾戶槩翻又戶得翻狷吉縣翻下遐稼翻】世子丕以疇同於子文辭祿【國語鬭旦曰楚成王聞子文之朝不及夕也以令尹秩之成王每出子文之禄必逃王止而又復人謂子文曰人生求富而子逃之何也對曰夫從政者以庇民也民多曠者而我取富焉是勤民以自封也死無日矣我逃死非逃富也】申胥逃賞【左傳吴破楚入郢申包胥如秦乞師立依庭牆而哭日夜不絶聲勺飲不入口者七日秦師乃出大敗吴師楚子入于郢賞申包胥包胥曰吾為君也非為身也君既定矣又何求遂逃賞】宜勿奪以優其節尚書荀彧司隸校尉鍾繇亦以為可聽操猶欲侯之疇素與夏侯惇善操使惇自以其情喻之惇就疇宿而勸之疇揣知其指【揣初委翻】不復發言惇臨去固邀疇疇曰疇負義逃竄之人耳【謂不能為劉虞報讐自竄於徐無山也】蒙恩全活為幸多矣豈可賣盧龍之塞以易賞祿哉縱國私疇疇獨不愧於心乎將軍雅知疇者猶復如此若必不得已請願效死刎首於前言未卒【刎武粉翻卒子恤翻】涕泣横流惇具以答操操喟然知不可屈乃拜為議郎操幼子倉舒卒操傷惜之甚司空掾邴原女早亡操欲求與倉舒合葬原辭曰嫁殤非禮也【未成人而死曰殤生未為配偶而死合葬故曰非禮】原之所以自容於明公公之所以待原者以能守訓典而不易也若聽明公之命則是凡庸也明公焉以為哉【焉於䖍翻】操乃止 孫權使威武中郎將賀齊【虞預曰賀氏本姓慶氏齊伯父純安帝時為侍中避帝父孝德皇帝諱改為賀氏】討丹陽黟歙賊黟帥陳僕祖山等二萬戶屯林歷山【魏氏春秋曰丹陽郡黟縣有林歷山歙縣亦屬丹陽郡師古曰黟音伊歙音攝帥所類翻】四面壁立不可得攻軍住經月齊隂募輕捷士於隱險處夜以銕戈拓山潛上【上時掌翻下同】縣布以援下人【縣讀曰懸援于元翻引也】得上者百餘人令分布四面鳴鼓角賊大驚守路者皆逆走還依衆大軍因是得上大破之權乃分其地為新都郡【權分歙縣為徙新新定休陽黎陽并黝為六縣置新都郡晉武帝太康元年更名新安郡唐睦州是也皇宋改徽州】以齊為太守<br />
<br />
  資治通鑑卷六十五<br />
<br />
<史部,編年類,資治通鑑>  <br>
   </div> 

<script src="/search/ajaxskft.js"> </script>
 <div class="clear"></div>
<br>
<br>
 <!-- a.d-->

 <!--
<div class="info_share">
</div> 
-->
 <!--info_share--></div>   <!-- end info_content-->
  </div> <!-- end l-->

<div class="r">   <!--r-->



<div class="sidebar"  style="margin-bottom:2px;">

 
<div class="sidebar_title">工具类大全</div>
<div class="sidebar_info">
<strong><a href="http://www.guoxuedashi.com/lsditu/" target="_blank">历史地图</a></strong>  
<a href="http://www.880114.com/" target="_blank">英语宝典</a>  
<a href="http://www.guoxuedashi.com/13jing/" target="_blank">十三经检索</a> 
<br><strong><a href="http://www.guoxuedashi.com/gjtsjc/" target="_blank">古今图书集成</a></strong> 
<a href="http://www.guoxuedashi.com/duilian/" target="_blank">对联大全</a> <strong><a href="http://www.guoxuedashi.com/xiangxingzi/" target="_blank">象形文字典</a></strong> 

<br><a href="http://www.guoxuedashi.com/zixing/yanbian/">字形演变</a>  <strong><a href="http://www.guoxuemi.com/hafo/" target="_blank">哈佛燕京中文善本特藏</a></strong>
<br><strong><a href="http://www.guoxuedashi.com/csfz/" target="_blank">丛书&方志检索器</a></strong> <a href="http://www.guoxuedashi.com/yqjyy/" target="_blank">一切经音义</a>  

<br><strong><a href="http://www.guoxuedashi.com/jiapu/" target="_blank">家谱族谱查询</a></strong>  <strong><a href="http://shufa.guoxuedashi.com/sfzitie/" target="_blank">书法字帖欣赏</a></strong> 
<br>

</div>
</div>


<div class="sidebar" style="margin-bottom:0px;">

<font style="font-size:22px;line-height:32px">QQ交流群9:489193090</font>


<div class="sidebar_title">手机APP 扫描或点击</div>
<div class="sidebar_info">
<table>
<tr>
	<td width=160><a href="http://m.guoxuedashi.com/app/" target="_blank"><img src="/img/gxds-sj.png" width="140"  border="0" alt="国学大师手机版"></a></td>
	<td>
<a href="http://www.guoxuedashi.com/download/" target="_blank">app软件下载专区</a><br>
<a href="http://www.guoxuedashi.com/download/gxds.php" target="_blank">《国学大师》下载</a><br>
<a href="http://www.guoxuedashi.com/download/kxzd.php" target="_blank">《汉字宝典》下载</a><br>
<a href="http://www.guoxuedashi.com/download/scqbd.php" target="_blank">《诗词曲宝典》下载</a><br>
<a href="http://www.guoxuedashi.com/SiKuQuanShu/skqs.php" target="_blank">《四库全书》下载</a><br>
</td>
</tr>
</table>

</div>
</div>


<div class="sidebar2">
<center>


</center>
</div>

<div class="sidebar"  style="margin-bottom:2px;">
<div class="sidebar_title">网站使用教程</div>
<div class="sidebar_info">
<a href="http://www.guoxuedashi.com/help/gjsearch.php" target="_blank">如何在国学大师网下载古籍?</a><br>
<a href="http://www.guoxuedashi.com/zidian/bujian/bjjc.php" target="_blank">如何使用部件查字法快速查字?</a><br>
<a href="http://www.guoxuedashi.com/search/sjc.php" target="_blank">如何在指定的书籍中全文检索?</a><br>
<a href="http://www.guoxuedashi.com/search/skjc.php" target="_blank">如何找到一句话在《四库全书》哪一页?</a><br>
</div>
</div>


<div class="sidebar">
<div class="sidebar_title">热门书籍</div>
<div class="sidebar_info">
<a href="/so.php?sokey=%E8%B5%84%E6%B2%BB%E9%80%9A%E9%89%B4&kt=1">资治通鉴</a> <a href="/24shi/"><strong>二十四史</strong></a>&nbsp; <a href="/a2694/">野史</a>&nbsp; <a href="/SiKuQuanShu/"><strong>四库全书</strong></a>&nbsp;<a href="http://www.guoxuedashi.com/SiKuQuanShu/fanti/">繁体</a>
<br><a href="/so.php?sokey=%E7%BA%A2%E6%A5%BC%E6%A2%A6&kt=1">红楼梦</a> <a href="/a/1858x/">三国演义</a> <a href="/a/1038k/">水浒传</a> <a href="/a/1046t/">西游记</a> <a href="/a/1914o/">封神演义</a>
<br>
<a href="http://www.guoxuedashi.com/so.php?sokeygx=%E4%B8%87%E6%9C%89%E6%96%87%E5%BA%93&submit=&kt=1">万有文库</a> <a href="/a/780t/">古文观止</a> <a href="/a/1024l/">文心雕龙</a> <a href="/a/1704n/">全唐诗</a> <a href="/a/1705h/">全宋词</a>
<br><a href="http://www.guoxuedashi.com/so.php?sokeygx=%E7%99%BE%E8%A1%B2%E6%9C%AC%E4%BA%8C%E5%8D%81%E5%9B%9B%E5%8F%B2&submit=&kt=1"><strong>百衲本二十四史</strong></a>  <a href="http://www.guoxuedashi.com/so.php?sokeygx=%E5%8F%A4%E4%BB%8A%E5%9B%BE%E4%B9%A6%E9%9B%86%E6%88%90&submit=&kt=1"><strong>古今图书集成</strong></a>
<br>

<a href="http://www.guoxuedashi.com/so.php?sokeygx=%E4%B8%9B%E4%B9%A6%E9%9B%86%E6%88%90&submit=&kt=1">丛书集成</a> 
<a href="http://www.guoxuedashi.com/so.php?sokeygx=%E5%9B%9B%E9%83%A8%E4%B8%9B%E5%88%8A&submit=&kt=1"><strong>四部丛刊</strong></a>  
<a href="http://www.guoxuedashi.com/so.php?sokeygx=%E8%AF%B4%E6%96%87%E8%A7%A3%E5%AD%97&submit=&kt=1">說文解字</a> <a href="http://www.guoxuedashi.com/so.php?sokeygx=%E5%85%A8%E4%B8%8A%E5%8F%A4&submit=&kt=1">三国六朝文</a>
<br><a href="http://www.guoxuedashi.com/so.php?sokeytm=%E6%97%A5%E6%9C%AC%E5%86%85%E9%98%81%E6%96%87%E5%BA%93&submit=&kt=1"><strong>日本内阁文库</strong></a> <a href="http://www.guoxuedashi.com/so.php?sokeytm=%E5%9B%BD%E5%9B%BE%E6%96%B9%E5%BF%97%E5%90%88%E9%9B%86&ka=100&submit=">国图方志合集</a> <a href="http://www.guoxuedashi.com/so.php?sokeytm=%E5%90%84%E5%9C%B0%E6%96%B9%E5%BF%97&submit=&kt=1"><strong>各地方志</strong></a>

</div>
</div>


<div class="sidebar2">
<center>

</center>
</div>
<div class="sidebar greenbar">
<div class="sidebar_title green">四库全书</div>
<div class="sidebar_info">

《四库全书》是中国古代最大的丛书,编撰于乾隆年间,由纪昀等360多位高官、学者编撰,3800多人抄写,费时十三年编成。丛书分经、史、子、集四部,故名四库。共有3500多种书,7.9万卷,3.6万册,约8亿字,基本上囊括了古代所有图书,故称“全书”。<a href="http://www.guoxuedashi.com/SiKuQuanShu/">详细>>
</a>

</div> 
</div>

</div>  <!--end r-->

</div>
<!-- 内容区END --> 

<!-- 页脚开始 -->
<div class="shh">

</div>

<div class="w1180" style="margin-top:8px;">
<center><script src="http://www.guoxuedashi.com/img/plus.php?id=3"></script></center>
</div>
<div class="w1180 foot">
<a href="/b/thanks.php">特别致谢</a> | <a href="javascript:window.external.AddFavorite(document.location.href,document.title);">收藏本站</a> | <a href="#">欢迎投稿</a> | <a href="http://www.guoxuedashi.com/forum/">意见建议</a> | <a href="http://www.guoxuemi.com/">国学迷</a> | <a href="http://www.shuowen.net/">说文网</a><script language="javascript" type="text/javascript" src="https://js.users.51.la/17753172.js"></script><br />
  Copyright &copy; 国学大师 古典图书集成 All Rights Reserved.<br>
  
  <span style="font-size:14px">免责声明:本站非营利性站点,以方便网友为主,仅供学习研究。<br>内容由热心网友提供和网上收集,不保留版权。若侵犯了您的权益,来信即刪。scp168@qq.com</span>
  <br />
ICP证:<a href="http://www.beian.miit.gov.cn/" target="_blank">鲁ICP备19060063号</a></div>
<!-- 页脚END --> 
<script src="http://www.guoxuedashi.com/img/plus.php?id=22"></script>
<script src="http://www.guoxuedashi.com/img/tongji.js"></script>

</body>
</html>
