資治通鑑卷一百七十七 宋 司馬光 撰

胡三省 音注

隋紀一|{
	起屠維作噩盡重光大淵獻凡三年 隋即春秋隨國為楚所滅以為縣秦漢屬南陽郡晉屬義陽郡後分置隨郡梁曰隨州後入西魏楊忠從周太祖以功封隨國公子堅襲爵受周禪遂以隨為國號又以周齊不遑寧處去辶作隋以辶訓走故也辶音綽}


高祖文皇帝上之上|{
	諱堅姓楊氏隋書云弘農郡華隂人也漢太尉震八代孫鈜生子元夀後魏時為武川鎮司馬子孫因家焉元夀玄孫忠從周太祖起義關西寔生帝自陳宣帝太建十三年至開皇九年隋有西北八年矣以通鑑紀年於此九年為隋紀年之始故書上之上}


開皇九年|{
	帝以陳高宗太建十三年受周禪至是年平陳混一天下通鑑紀事乃以開皇繋年}
春正月乙丑朔陳主朝會羣臣大霧四塞|{
	朝直遥翻塞悉則翻}
入人鼻皆辛酸陳主昏睡至晡時乃寤|{
	日加申為晡晡奔謨翻}
是日賀若弼自廣陵引兵濟江|{
	若人者翻}
先是弼以老馬多買陳船而匿之買弊船五六十艘置於瀆内|{
	先悉薦翻艘蘇漕翻水注澮曰凟}
陳人覘之以為内國無船|{
	覘丑亷翻又丑艶翻内國即中國隋避諱改曰内}
弼又請緣江防人每交代之際必集廣陵於是大列旗幟營幕被野|{
	幟昌志翻被皮義翻}
陳人以為隋兵大至急發兵為備既知防人交代其衆復散後以為常不復設備|{
	復扶又翻}
又使兵緣江時獵人馬喧譟故弼之濟江陳人不覺韓擒虎將五百人自横江宵濟采石|{
	横江浦在和州界采石磯在今太平州北三十里對岸津渡處將即亮翻下將兵同}
守者皆醉遂克之|{
	德祐甲戌十有二月沙武口之事亦猶此}
晉王廣帥大軍屯六合鎭桃葉山|{
	隋志江都郡六合縣舊曰尉氏置秦郡後齊置秦州後周改州曰方州改郡曰六合開皇初郡廢四年改尉氏曰六合張舜民曰桃葉山即今瓜步鎭之地帥讀曰率}
丙寅采石戌主徐子建馳啟告變丁卯召公卿入議軍旅戊辰陳主下詔曰犬羊陵縱侵竊郊畿蜂蠆有毒宜時掃定|{
	蠆丑邁翻}
朕當親御六師廓清八表内外並可戒嚴以驃騎將軍蕭摩訶護軍將軍樊毅中領軍魯廣達並為都督|{
	驃匹妙翻騎奇寄翻}
司空司馬消難湘州刺史施文慶並為大監軍|{
	去年冬陳主擢施文慶督湘州未及之鎭而隋兵渡江難乃旦翻監工銜翻}
遣南豫州刺史樊猛帥舟師出白下|{
	陳南豫州治宣城時徙鎮姑孰白下城合白石疊唐武德移江寧縣於此名白下縣帥讀曰率}
散騎常侍臯文奏將兵鎮南豫州重立賞格僧尼道士盡令執役|{
	尼女夷翻}
庚午賀若弼攻拔京口執南徐州刺史黄恪|{
	南徐州治京口}
弼軍令嚴肅秋毫不犯有軍士於民間酤酒者弼立斬之所俘獲六千餘人弼皆釋之給糧勞遣|{
	勞力到翻}
付以敕書令分道宣諭|{
	令力丁翻}
于是所至風靡樊猛在建康其子巡攝行南豫州事辛未韓擒虎進攻姑孰半日拔之執巡及其家口|{
	今太平州當塗縣南二里有姑孰溪西入大江蓋因舊鎮而得名}
臯文奏敗還江南父老素聞擒虎威信來謁軍門者晝夜不絶魯廣達之子世貞在新蔡與其弟世雄及所部降於擒虎|{
	侯景之亂魯悉達糾合鄉人以保新蔡魯氏遂世襲以事陳新蔡注見一百五十四卷梁世祖承聖元年降戶江翻}
遣使致書招廣達|{
	使疏吏翻}
廣達時屯建康自劾詣廷尉請罪|{
	劾戶槩翻又戶得翻}
陳主慰勞之加賜黄金遣還營|{
	勞力到翻}
樊猛與左衛將軍蔣元遜將青龍八十艘於白下遊奕以禦六合兵陳主以猛妻子在隋軍懼有異志欲使鎭東大將軍任忠代之|{
	任音壬}
令蕭摩訶徐諭猛猛不悦陳主重傷其意而止|{
	重如字難也}
于是賀若弼自北道韓擒虎自南道並進|{
	京口於建康為北姑孰於建康為南}
緣江諸戌望風盡走弼分兵斷曲阿之衝而入|{
	曲阿本雲陽秦時人言其地有天子氣始皇鑿北坑以敗其勢截直道使阿曲改曰曲阿其地在武進丹徒二縣之間弼分兵斷其衝恐三吳之兵入救建康掎其後也斷音短}
陳主命司徒豫章王叔英屯朝堂蕭摩訶屯樂遊苑|{
	朝直遥翻樂音洛}
樊毅屯耆闍寺|{
	開視遮翻}
魯廣達屯白土岡忠武將軍孔範屯寶田寺|{
	忠武將軍梁置班十九陳擬官品第四秩中二千石位次四平將軍}
己卯任忠自吳興入赴|{
	去年使任忠出守吳興}
仍屯朱雀門|{
	晉孝武帝建朱雀門上有兩銅雀前直大航謂之朱雀航}
辛未賀若弼進據鍾山|{
	鍾山在今上元縣東北十八里輿地志古曰金陵山縣名因此又名蔣山漢末秣陵尉蔣子文討賊死此山下孫氏都秣陵以其祖諱鍾因改名蔣山}
頓白土岡之東晉王廣遣摠管杜彦與韓擒虎合軍步騎二萬屯于新林|{
	新林浦去今建康城二十里西直白鷺洲}
蘄州摠管王世積以舟師出九江|{
	蘄音機又音其王世積闡熙新囶人按班志廬江郡尋陽縣禹貢九江皆在南東合為大江應劭曰江自廬江尋陽分為九漢之尋陽縣在今蘄州界王世積以舟師自蘄水出大江囶古國字}
破陣將紀瑱於蘄口|{
	蘄水入江之口將即亮翻下同瑱他殿翻又音鎭}
陳人大駭降者相繼|{
	降戶江翻}
晉王廣上狀|{
	上時掌翻}
帝大悦宴賜羣臣時建康甲士尚十餘萬人陳主素怯懦不達軍士|{
	懦乃卧翻又奴亂翻士讀曰事}
唯日夜啼泣臺内處分一以委施文慶|{
	處昌呂翻分扶問翻}
文慶既知諸將疾已恐其有功乃奏曰此輩怏怏|{
	怏於兩翻}
素不伏官迫此事機那可專信由是諸將凡有啟請率皆不行賀若弼之攻京口也蕭摩訶請將兵逆戰陳主不許及弼至鍾山摩訶又曰弼懸軍深入壘塹未堅|{
	塹七艶翻}
出兵掩襲可以必克又不許陳主召摩訶任忠於内殿議軍事忠曰兵法客貴速戰主貴持重今國家足兵足食宜固守臺城緣淮立栅北軍雖來勿與交戰分兵斷江路|{
	斷丁管翻下同}
無令彼信得通給臣精兵一萬金翅三百艘下江徑掩六合彼大軍必謂其度江將士已被俘獲自然挫氣|{
	被皮義翻}
淮南土人與臣舊相知悉今聞臣往必皆景從|{
	師古日景從言如景之從形也}
臣復揚聲欲往徐州斷彼歸路則諸軍不擊自去|{
	徐州彭汴之路也復扶又翻斷丁管翻}
待春水既漲上江周羅睺等衆軍必沿流赴援|{
	周羅睺時督水軍在郢漢}
此良策也陳主不能從明日歘然曰|{
	歘許勿翻}
兵久不決令人腹煩可呼蕭郎一出擊之任忠叩頭苦請勿戰孔範又奏請作一決當為官勒石燕然|{
	孔範以竇憲破匈奴事自詭姦諂之誤國亡家如此為于偽翻下同燕於賢翻}
陳主從之謂摩訶曰公可為我一決摩訶曰從來行陳|{
	行戶剛翻陳讀曰陣}
為國為身今日之事兼為妻子陳主多出金帛賦諸軍以充賞|{
	賦給與也分畀也}
甲申使魯廣達陳於白土岡|{
	陳讀曰陣下同}
居諸軍之南任忠次之樊毅孔範又次之蕭摩訶軍最在北諸軍南北亘二十里首尾進退不相知賀若弼將輕騎登山望見衆軍因馳下與所部七摠管楊牙員明等|{
	將即亮翻騎奇寄翻下同員音運姓也}
甲士凡八千勒陳以待之陳主通於蕭摩訶之妻故摩訶初無戰意唯魯廣達以其徒力戰與弼相當隋師退走者數四弼麾下死者二百七十三人弼縱烟以自隱窘而復振|{
	窘渠隕翻復扶又翻下同}
陳兵得人頭皆走獻陳主求賞弼知其驕惰更引兵趣孔範|{
	趣七喻翻又讀曰趨}
範兵暫交即走陳諸軍顧之騎卒亂潰不可復止死者五千人員明擒蕭摩訶送於弼弼命牽斬之摩訶顔色自若弼乃釋而禮之任忠馳入臺見陳主言敗狀曰官好住|{
	好宜也住止也今南人猶有是言}
臣無所用力矣陳主與之金兩縢|{
	縢徒登翻以繩約物曰縢}
使募人出戰忠曰陛下唯當具舟楫就上流衆軍|{
	謂往就周羅睺等}
臣以死奉衛陳主信之敕忠出部分|{
	分扶問翻}
令宫人裝束以待之怪其久不至時韓擒虎自新林進軍忠已帥數騎迎降於石子岡|{
	帥讀曰率降戶江翻下同}
領軍蔡徵守朱雀航聞擒虎將至衆懼而潰忠引擒虎軍直入朱雀門陳人欲戰忠揮之曰老夫尚降諸軍何事|{
	軍或作君}
衆皆散走于是城内文武百司皆遁唯尚書僕射袁憲在殿中尚書令江摠等數人居省中陳主謂袁憲曰我從來接遇卿不勝餘人今日但以追愧|{
	此猶劉禪之於郤正也}
非唯朕無德亦是江東衣冠道盡陳主遑遽將避匿憲正色曰北兵之入必無所犯大事如此陛下去欲安之臣願陛下正衣冠御正殿依梁武帝見侯景故事|{
	事見一百六十二卷梁武帝太清三年}
陳主不從下榻馳去曰鋒刃之下未可交當吾自有計從宫人十餘出後堂景陽殿將自投于井憲苦諫不從後閤舍人夏侯公韻以身蔽井|{
	夏戶雅翻}
陳主與爭久之乃得入既而軍人窺井呼之不應欲下石乃聞叫聲以繩引之驚其太重及出乃與張貴妃孔貴嬪同束而上|{
	祝穆曰景陽井在法寶寺或云白蓮閣下有小池面方丈餘或云在保寧寺覽輝亭側舊傳云欄有石脉以帛拭之作胭脂痕一名胭脂井又名辱井梁制有殿中舍人守舍人陳制殿中舍人為三品藴位守舍人為三品勲位在九品之外後閣舍人蓋殿中舍人之守後閣者}
沈后居處如常|{
	處昌呂翻}
太子深年十五閉閤而坐舍人孔伯魚侍側|{
	此太子舍人也梁制太子中舍人四人掌其坊之禁令舍人十六人掌文記中舍人八班舍人三班陳制中舍人六百石舍人亦如之}
軍士叩閤而入深安坐勞之曰|{
	勞力到翻}
戎旅在塗不至勞也軍士咸致敬焉時陳人宗室王侯在建康者百餘人陳主恐其為變皆召入令屯朝堂使豫章王叔英摠督之又隂為之備及臺城失守相帥出降|{
	朝直遥翻下同守式又翻帥讀曰率降戶江翻}
賀若弼乘勝至樂遊苑|{
	樂音洛}
魯廣達猶督餘兵苦戰不息所殺獲數百人會日暮乃解甲面臺再拜慟哭謂衆曰我身不能救國負辠深矣士卒皆流涕歔欷|{
	歔音虚欷音希又許既翻}
遂就擒諸門衛皆走弼夜燒北掖門入聞韓擒虎已得陳叔寶呼視之叔寶惶懼流汗股栗向弼再拜弼謂之曰小國之君當大國之卿拜乃禮也入朝不失作歸命侯|{
	朝直遥翻孫皓降晉封歸命侯}
無勞恐懼既而恥功在韓擒虎後與擒虎相訽挺刃而出|{
	訽苦侯翻罵也挺拔也}
欲令蔡徵為叔寶作降箋命乘騾車歸已|{
	騾盧戈翻}
事不果弼置叔寶於德教殿以兵衛守高熲先入建康熲子德弘為晉王廣記室|{
	熲居永翻隋制諸王記室參軍在録事功曹之下}
廣使德弘馳詣熲所令留張麗華熲曰昔太公蒙面以斬妲己|{
	妲己有蘇氏美女商紂嬖之武王勝殷殺紂并殺妲己妲當割翻已音紀}
今豈可留麗華乃斬之於青溪德弘還報廣變色曰昔人云無德不報|{
	詩大雅抑之詞}
我必有以報高公矣由是恨熲|{
	使高熲留麗華而廣納之文帝必怒安得成他日奪嫡之謀是誠宜德之也顧恨之邪史為廣殺熲張本}
丙戌晉王廣入建康以施文慶受委不忠曲為諂佞以蔽耳目沈客卿重賦厚斂以悦其上與太市令陽慧朗刑法監徐析尚書都令史暨慧皆為民害斬於石闕下以謝三吳|{
	斂力贍翻暨戟乙翻陽慧朗一作惠朗暨慧之下逸景字}
使高熲與元帥府記室裴矩|{
	帥所類翻}
收圖籍封府庫資財一無所取天下皆稱廣以為賢矩讓之之弟子也|{
	裴讓之見一百五十八卷梁武帝大同四年}
廣以賀若弼先期決戰違軍令收以屬吏|{
	先悉薦翻屬之欲翻}
上驛召之詔廣曰平定江表弼與韓擒虎之力也賜物萬段又賜弼與擒虎詔美其功開府儀同三司王頒僧辯之子夜發陳高祖陵焚骨取灰投水而飲之|{
	報讐也陳高祖殺僧辯事見一百六十六卷梁敬帝紹泰元年}
既而自縛歸辠於晉王廣廣以聞上命赦之詔陳高祖世祖高宗陵摠給五戶分守之上遣使以陳亡告許善心|{
	使疏吏翻}
善心衰服號哭於西階之下藉草東向坐三日|{
	去年陳遣善心來聘留於客館不遣還事見上卷西階賓階也衰服藉草喪禮也衰吐雷翻號戶刀翻藉慈夜翻}
敕書唁焉|{
	唁魚戰翻弔生曰唁}
明日有詔就館拜通直散騎常侍|{
	散悉亶翻騎奇寄翻}
賜衣一襲|{
	衣單複具曰襲}
善心哭盡哀入房改服|{
	改衰服服賜服}
復出北面立|{
	復扶又翻}
垂泣再拜受詔明日乃朝|{
	朝直遥翻}
伏泣於殿下悲不能興上顧左右曰我平陳國唯獲此人既能懷其舊君即我之誠臣也敕以本官直門下省|{
	通直散騎常侍屬門下省今敕善心以本官直門下省何也按唐六典晉始有門下省散騎常侍雖隸門下别為一省潘岳云寓直散騎之省是也此隋所以命許善心以通直散騎常侍直門下省歟}
陳水軍都督周羅睺與郢州刺史荀法尚守江夏|{
	江夏陳郢州治所夏戶雅翻}
秦王俊督三十摠管水陸十餘萬屯漢口不得進|{
	漢水入江之口即沔口也}
相持踰月陳荆州刺史陳慧紀遣南康内史呂忠肅屯岐亭|{
	按楊素傳忠肅屯岐亭正據江峽則岐亭在西陵峽口也 考異曰隋書作呂仲肅南史作呂肅今從陳書}
據巫峽|{
	按水經江水出巫峽過秭歸夷陵逕流頭狼尾灘而後東逕西陵峽去年冬楊素破戚昕其舟師已過狼尾而東呂忠肅所據者蓋西陵峽也當從楊素傳作江峽為通}
於北岸鑿巖綴鐵鎖三條 |{
	考異曰南史作五條今從隋書}
横截上流以遏隋船忠肅竭其私財以充軍用楊素劉仁恩奮兵擊之四十餘戰忠肅守險力爭隋兵死者五千餘人陳人盡取其鼻以求功賞既而隋師屢捷獲陳之士卒三縱之忠肅棄栅而遁素徐去其鎖|{
	去羌呂翻}
忠肅復據荆門之延洲素遣巴蜑千人|{
	蜑亦蠻也居巴中老曰巴蜑此水蜑之習於用舟者也蜑徒旱翻}
乘五牙四艘以拍竿碎其十餘艦|{
	艘蘇遭翻艦戶黯翻}
遂大破之俘甲士二千餘人忠肅僅以身免陳信州刺史顧覺屯安蜀城棄城走|{
	梁置信州於巴東西魏取之其地時屬隋故陳信州刺史屯於安蜀城}
陳慧紀屯公安|{
	公安陳荆州治所}
悉燒其儲蓄引兵東下於是巴陵以東無復城守者陳慧紀帥將士三萬人樓船千餘艘沿江而下|{
	復扶又翻帥讀日率下同將即亮翻下同}
欲入援建康為秦王俊所拒不得前是時陳晉熙王叔文罷湘州還至巴州慧紀推叔文為盟主|{
	巴州治巴陵}
而叔文已帥巴州刺史畢寶等致書請降於俊俊遣使迎勞之|{
	帥讀曰率降戶江翻使疏吏翻下同勞力到翻}
會建康平晉王廣命陳叔寶手書招上江諸將使樊毅詣周羅睺陳慧紀子正業詣慧紀諭指時諸城皆解甲羅睺乃與諸將大臨三日|{
	將即亮翻臨力浸翻}
放兵散然後詣俊降陳慧紀亦降上江皆平楊素下至漢口與俊會王世積在蘄口聞陳已亡告諭江南諸郡於是江州司馬黄偲棄城走|{
	蘄音機又音其偲音思}
豫章諸郡太守皆詣世積降|{
	守式又翻}
癸巳詔遣使者巡撫陳州郡二月乙未廢淮南行臺省|{
	晉王廣於時將凱還也}
蘇威奏請五百家置鄉正使治民間辭訟|{
	治直之翻下同}
李德林以為本廢鄉官判事為其里閭親識剖斷不平|{
	為其于偽翻斷丁亂翻}
今令鄉正專治五百家恐為害更甚且要荒小縣有不至五百家者|{
	令力丁翻要一遥翻}
豈可使兩縣共管一鄉帝不聽丙申制五百家為鄉置鄉正一人百家為里置里長一人|{
	長知兩翻}
陳吳州刺史蕭瓛能得物情陳亡吳人推瓛為主|{
	瓛戶官翻}
右衛大將軍武川宇文述帥行軍摠管元契張默言等討之落叢公燕榮以舟師自東海至|{
	隋書宇文述代郡武川人地理志馬邑郡善陽縣大業初置代郡順政郡鳴水縣西魏置落叢縣及落叢郡順政西魏之興州也東海郡海州燕榮舟師自海道入湖可至吳州陳置吳州於吳郡燕因肩翻}
陳永新侯陳君範自晉陵犇瓛|{
	沈約志永新縣吳立屬安成太守隋廢安成郡為安復縣晉陵與吳接壤}
并軍拒述述軍且至瓛立栅於晉陵城東留兵拒述遣其將王褒守吳州自義興入太湖欲掩述後述進破其栅迴兵擊瓛大破之|{
	栅測革翻將即亮翻}
又遣兵别道襲吳州王褒衣道士服棄城走|{
	衣於既翻}
瓛以餘衆保包山|{
	包山在太湖中其地西北距吳縣百二十里又名洞庭山四面皆水地占三鄉環四十里土宜橘柚}
燕榮擊破之瓛將左右數人匿民家為人所執述進至奉公埭|{
	燕因肩翻將音如字領也攜也埭徒蓋翻}
陳東揚州刺史蕭巖以會稽降與瓛皆送長安斬之|{
	以巖等驅江陵士女降陳也事見上卷陳長城公禎明元年}
楊素之下荆門也遣别將龎暉將兵略地南至湘州城中將士莫有固志|{
	將即亮翻下同}
刺史岳陽王叔愼年十八置酒會文武僚吏酒酣叔愼歎曰君臣之義盡於此乎|{
	按陳湘州刺史陳叔文既去鎮施文慶寔代之阻隋兵不及至湘州必有守之者但未知叔愼何時所命耳}
長史謝基伏而流涕湘州助防遂興侯正理在坐|{
	遂興縣侯也沈約志廬陵郡有遂興縣吳立曰新興晉武帝太康元年更名長知兩翻坐徂卧翻}
乃起曰主辱臣死諸君獨非陳國之臣乎今天下有難|{
	難乃旦翻}
寔致命之秋也縱其無成猶見臣節青門之外有死不能|{
	召平秦時東陵侯秦亡為民種瓜青門外正理自謂陳亡之後不能編於民伍以求活}
今日之機不可猶豫後應者斬衆咸許諾乃刑牲結盟仍遣人詐奉降書於龎暉|{
	降戶江翻}
暉信之尅期入城叔愼伏甲待之暉至執之以狥并其衆皆斬之叔慎坐于射堂招合士衆數日之中得五千人衡陽太守樊通武州刺史鄔居業皆請舉兵助之|{
	隋志長沙郡衡山縣舊置衡陽郡武陵郡舊置武州鄔姓其先仕晉為鄔大夫子孫因以為氏鄔烏古翻守式又翻}
隋所除湘州刺史薛胄將兵適至與行軍摠管劉仁恩共擊之叔愼遣其將陳正理與樊通拒戰|{
	將即亮翻}
兵敗胄乘勝入城擒叔愼仁恩破鄔居業於横橋亦擒之俱送秦王俊斬於漢口 嶺南未有所附數郡共奉高凉郡太夫人洗氏為主|{
	高凉縣置高凉郡洗音銑又音線}
號聖母保境拒守詔遣柱國韋洸等安撫嶺外陳豫章太守徐璒據南康拒之|{
	徐璒自豫章退保南康南康郡治贑縣洸古黄翻守式又翻璒都滕翻}
洸等不得進晉王廣遣陳叔寶遺夫人書|{
	遺于季翻}
諭以國亡使之歸隋夫人集首領數千人盡日慟哭遣其孫馮魂帥衆迎洸|{
	洗氏嫁馮融見一百六十三卷梁簡文帝大寶元年帥讀曰率}
洸擊斬徐璒入至廣州說諭嶺南諸州皆定|{
	說式芮翻考異曰隋帝紀十年八月壬申遣洸等巡撫嶺南百越皆服按陳以九年正月亡至來年八月并閏計之二十一月豈有洗氏猶不知者洗氏傳又云晉王遣陳主遺夫人書則事在九年三月前也帝紀所云蓋謂百越已服奏到朝廷之日也}
表馮魂為儀同三司册洗氏為宋康郡夫人|{
	宋文帝元嘉九年分高凉立宋康郡隋志高凉郡杜原縣舊有永寧宋康二郡}
洸夐之子也|{
	韋夐見一百六十七卷陳高祖永定三年夐休正翻}
衡州司馬任瓌勸都督王勇據嶺南|{
	隋志梁置衡州於廣州含洭縣任音壬瓌占囘翻}
求陳氏子孫立以為帝勇不能用以所部來降|{
	降戶江翻}
瓌棄官去瓌忠之弟子也|{
	任瓌志趣如此宜其能自表見 於唐元也蕭摩訶兒豚犬耳}
於是陳國皆平|{
	陳高祖受梁禪歲在丁丑至是而亡凡五主三十三年}
得州三十郡一百縣四百|{
	按隋志陳境當時有揚東揚南徐吳閩豐湘巴武江郢廣東衡衡高羅新隴建城桂東寧靜南定越南合崖安交愛凡三十州}
詔建康城邑宫室並平蕩耕墾更於石頭置蔣州|{
	以蔣山名州也}
晉王廣班師留王韶鎭石頭委以後事三月己巳陳叔寶與其王公百司發建康詣長安大小在路五百里纍纍不絶帝命權分長安士民宅以俟之内外修整遣使迎勞陳人至者如歸|{
	使疏吏翻勞力到翻下同}
夏四月辛亥帝幸驪山|{
	驪山在新豐縣}
親勞旋師乙巳諸軍凱入|{
	奏凱樂而入也}
獻俘于太廟陳叔寶及諸王侯將相并乘輿服御天文圖籍等以次行列|{
	將即亮翻相息亮翻乘繩證翻下同行戶剛翻}
仍以鐵騎圍之|{
	騎奇寄翻}
從晉王廣秦王俊入列于殿庭拜廣為太尉賜輅車乘馬衮冕之服玄圭白璧丙辰帝坐廣陽門觀|{
	廣陽門之觀闕也觀古玩翻}
引陳叔寶於前及太子諸王二十八人司空司馬消難以下至尚書郎凡二百餘人|{
	難乃旦翻}
帝使納言宣詔勞之|{
	勞力到翻}
次使内史令宣詔責以君臣不能相輔乃至滅亡叔寶及其羣臣並愧懼伏地屏息不能對|{
	屏必郢翻}
既而宥之初武元帝迎司馬消難|{
	見一百六十七卷陳永定二年皇考忠謚武元帝}
與消難結為兄弟情好甚篤|{
	好呼到翻}
帝每以叔父禮事之及平陳消難至特免死配為樂戶二旬而免猶以舊恩引見尋卒於家|{
	見賢遍翻卒子恤翻}
庚戌帝御廣陽門|{
	廣陽門大興宫城正南門也唐六典曰隋曰廣陽門開皇二年作仁夀元年改曰昭陽門唐武德元年改曰順天門神龍元年改承天門}
宴將士自門外夾道列布帛之積|{
	將即亮翻積子賜翻凡指所聚之物曰積則去聲取物而積疊之則入聲}
達于南郭班賜各有差凡用三百餘萬段故陳之境内給復十年|{
	復方目翻}
餘州免其租賦樂安公元諧進曰陛下威德遠被|{
	被皮義翻}
臣前請以突厥可汗為候正|{
	厥九勿翻可從刋入聲汗音寒}
陳叔寶為令史今可用臣言矣帝曰朕平陳國本以除逆非欲誇誕公之所奏殊非朕心突厥不知山川何能警候叔寶昏醉寧堪驅使諧默然而退辛酉進楊素爵為越公|{
	按隋書楊素自清河郡公進封郢國公素言逆人王誼前封於郢不願與之同改封越公}
以其子玄感為儀同三司玄奬為清河郡公賜物萬段粟萬石命賀若弼登御坐|{
	坐徂卧翻}
賜物八千段加位上柱國進爵宋公仍各加賜金寶及陳叔寶妹為妾賀若弼韓擒虎爭功於帝前弼曰臣在蔣山死戰破其鋭卒擒其驍將|{
	驍堅堯翻將即亮翻下同}
震揚威武遂平陳國韓擒虎畧不交陳|{
	陳讀曰陣}
豈臣之比擒虎曰本奉明旨令臣與弼同時合勢以取偽都弼乃敢先期|{
	先悉薦翻}
逢賊遂戰致令將士傷死甚多臣以輕騎五百兵不血刃直取金陵降任蠻奴|{
	騎奇寄翻降戶江翻任音壬}
執陳叔寶據其府庫傾其巢穴弼至夕方扣北掖門臣啟關而納之斯乃救罪不暇安得與臣相比帝曰二將俱為上勲於是進擒虎位上柱國賜物八千段有司劾擒虎放縱士卒淫汙陳宫|{
	劾戶槩翻又戶得翻汙烏路翻}
坐此不加爵邑加高熲上柱國進爵齊公|{
	熲居迥翻熲自勃海郡公進爵齊國公}
賜物九千段帝勞之曰公伐陳後人言公反朕已斬之君臣道合非青蠅所能間也|{
	勞力到翻青蠅以諭讒言間古莧翻}
帝從容命熲與賀若弼論平陳事|{
	從千容翻}
熲曰賀若弼先獻十策後於蔣山苦戰破賊臣文吏耳焉敢與大將論功|{
	焉於䖍翻將即亮翻}
帝大笑嘉其有讓帝之伐陳也使高熲問方略於上儀同三司李德林以授晉王廣至是帝賞其功授柱國封郡公賞物三千段已宣勑訖或說高熲曰今歸功於李德林諸將必當憤惋|{
	說輸芮翻惋烏貫翻}
且後世觀公有若虚行熲入言之乃止以秦王俊為揚州摠管四十四州諸軍事鎭廣陵晉王廣還并州晉王廣之戮陳五佞也|{
	五佞謂施文慶沈客卿陽慧朗徐析暨慧景}
未知都官尚書孔範散騎常侍王瑳王儀御史中丞沈瓘之罪故得免及至長安事並露乙未帝暴其過惡投之邊裔以謝吳越之人瑳刻薄貪鄙忌害才能儀傾巧側媚獻二女以求親昵|{
	散悉亶翻騎奇寄翻瑳倉何翻昵尼質翻}
瓘險慘苛酷發言邪諂故同辠焉帝給賜陳叔寶甚厚數得引見班同三品每預宴恐致傷心為不奏吳音|{
	數所角翻見賢遍翻為于偽翻}
後監守者奏言叔寶云既無秩位每預朝集願得一官號帝曰叔寶全無心肝監者又言叔寶常醉|{
	監古銜翻朝直遥翻}
罕有醒時帝問飲酒幾何對曰與其子弟日飲一石帝大驚使節其酒既而曰任其性不爾何以過日|{
	嗚呼此陳叔寶所以得死於枕席也}
帝以陳氏子弟既多恐其在京城為非乃分置邊州給田業使為生歲時賜衣服以安全之詔以陳尚書令江摠為上開府儀同三司僕射袁憲驃騎蕭摩訶領軍任忠皆為開府儀同三司|{
	射寅謝翻驃匹妙翻騎奇寄翻任音壬}
吏部尚書吳興姚察為祕書丞上嘉袁憲雅操下詔以為江表稱首授昌州刺史|{
	隋志舂陵郡後魏置南荆州西魏改曰昌州}
聞陳散騎常侍袁元友數直言於陳叔寶擢拜主爵侍郎|{
	散悉亶翻騎奇寄翻隋志主爵侍郎屬吏部尚書}
謂羣臣曰平陳之初我悔不殺任蠻奴受人榮禄兼當重寄不能横尸狥國乃云無所用力與弘演納肝何其遠也|{
	衛懿公與狄人戰于熒澤為狄人所殺弘演納肝以狥之}
帝見周羅睺慰諭之許以富貴羅睺垂泣對曰臣荷陳氏厚遇|{
	荷下可翻}
本朝淪亡無節可紀|{
	朝直遥翻下同}
得免於死陛下之賜也何富貴之敢望賀若弼謂羅睺曰聞公郢漢捉兵|{
	若人者翻捉把也}
即知揚州可得王師利涉果如所量|{
	量音良}
羅睺曰若得與公周旋勝負未可知|{
	周羅睺何以得此於賀若弼哉史家溢美耳}
頃之拜上儀同三司先是陳將羊翔來降|{
	先悉薦翻降戶江翻}
伐陳之役使為鄉導|{
	鄉讀曰嚮}
位至上開府儀同三司班在羅睺上韓擒虎於朝堂戲之曰不知機變乃立在羊翔之下能無愧乎|{
	朝直遥翻}
羅睺曰昔在江南久承令問|{
	令力定翻美也}
謂公天下節士今日所言殊非所望擒虎有愧色帝之責陳君臣也陳叔文獨欣然有得色|{
	得色自得其意而形於色}
既而復上表自陳|{
	復扶又翻}
昔在巴州已先送欵乞知此情望異常例帝雖嫌其不忠而欲懷柔江表乃授叔文開府儀同三司拜宜州刺史|{
	宜州置於京兆華原縣}
初陳散騎常侍韋鼎聘于周|{
	韋鼎傳陳太建中聘周散悉亶翻騎奇寄翻}
遇帝而異之謂帝曰公當大貴貴則天下一家歲一周天|{
	歲星木星也十二年一周天}
老夫當委質於公|{
	質如字}
及至德之初|{
	陳長城公即位改元至德}
鼎為太府卿盡賣田宅大匠卿毛彪問其故鼎曰江東王氣盡於此矣|{
	王于魂翻又音如字}
吾與爾當葬長安及陳平上召鼎為上儀同三司鼎叡之孫也|{
	韋叡著功名於梁武帝之時}
壬戌詔曰今率土大同含生遂性太平之法方可流行凡我臣民澡身浴德家家自修人人克念|{
	書曰惟狂克念作聖}
兵可立威不可不戢刑可助化不可專行禁衛九重之餘|{
	重直龍翻}
鎮守四方之外戎旅軍器皆宜停罷世路既夷羣方無事武力之子俱可學經民間甲伏悉皆除毁頒告天下咸悉此意賀若弼撰其所畫策上之|{
	若人者翻撰士免翻述也上時掌翻}
謂為御授平陳七策帝弗省|{
	省悉井翻視也}
曰公欲發揚我名我不求名公宜自載家傳|{
	傳宜戀翻}
弼位望隆重兄弟並封郡公為刺史列將家之珍玩不可勝計|{
	將即亮翻勝音升}
婢妾曳羅綺者數百|{
	羅交眼綺細綾}
時人榮之其後突厥來朝|{
	厥九勿翻朝直遥翻}
上謂之曰汝聞江南有陳國天子乎對曰聞之上命左右引突厥詣韓擒虎前曰此是執得陳國天子者擒虎厲色顧之突厥惶恐不敢仰視左衛將軍龎晃等短高熲於上上怒皆黜之|{
	龎晃自結納於潜躍之辰與上情契甚密而與高熲有隙與廣平王雄挾舊屢言熲之短故皆被黜熲居永翻}
親禮逾密因謂熲曰獨孤公猶鏡也每被磨瑩皎然益明初熲父賓為獨孤信僚佐賜姓獨孤氏故上常呼獨孤而不名|{
	按獨孤信之誅妻子徙蜀獨孤后以賓父之故吏每往來其家熲之遭遇豈專以才略哉外得君而内蒙君母親禮也及夫外則見忌於君内則失愛於君母隨見踈弃君臣之際可無謹乎}
樂安公元諧性豪俠有氣調|{
	調徒釣翻}
少與上同學甚相愛及即位累歷顯仕諧好排詆不能取媚左右|{
	少詩照翻好呼到翻}
與上柱國王誼善誼誅|{
	王誼誅見一百七十六卷陳長城公至德三年}
上稍疎忌之或告諧與從父弟上開府儀同三司滂臨澤侯田鸞|{
	隋志毗陵郡義興縣舊有臨澤縣從才用翻}
上儀同三司祈緒等謀反|{
	祈姓出於黄帝黄帝子得姓者十四人祈其一也又曰晉獻侯四世孫曰奚食邑於祈子孫以為氏}
下有司案驗奏諧謀令祈緒勒党項兵斷巴蜀|{
	下遐嫁翻令力丁翻斷音短}
又諧嘗與滂同謁上諧私謂滂曰我是主人殿上者賊也因令滂望氣滂曰彼雲似蹲狗走鹿|{
	蹲慈尊翻}
不如我輩有福德雲上大怒諧滂鸞緒並伏誅 |{
	考異曰德林傳云德林以梁士彦元諧頻有逆意江南抗衡上國乃著天命論上之諧傳云平陳後數歲人告諧謀反按諧請以叔寶為内史則陳亡時猶在楊雄方用事諧欲譛去之則雄未為司空故附於此按内史當依正文作令史按通鑑正文亦書元諧言請以陳叔寶為令史按内史隋之要官元諧安敢請以陳叔寶為是官邪}
閏月己卯以吏部尚書蘇威為右僕射|{
	射寅謝翻}
六月乙丑以荆州摠管楊素為納言 朝野皆稱封禪|{
	朝直遥翻下同稱當作請又切請稱舉也言朝野舉封禪事為言也}
秋七月丙午詔曰豈可命一將軍除一小國遐邇注意便謂太平以薄德而封名山用虚言而干上帝非朕攸聞而今而後言及封禪宜即禁絶 左衛大將軍廣平王雄貴寵特盛與高熲虞慶則蘇威稱為四貴雄寛容下士|{
	下遐嫁翻}
朝野傾屬|{
	屬之欲翻}
上惡其得衆|{
	惡烏路翻}
隂忌之不欲其典兵馬八月壬戌以雄為司空實奪之權雄既無職務乃杜門不通賓客|{
	雄以是能保其身於猜忌之朝}
帝踐阼之初柱國沛公鄭譯請修正雅樂詔太常卿牛弘國子祭酒辛彦之博士何妥等議之積年不决|{
	妥他果翻}
譯言古樂十二律旋相為宫各用七聲世莫能通譯因龜兹人蘇祇婆善琵琶始得其法推演為十二均八十四調以校太樂所奏例皆乖越譯又於七音之外更立一聲謂之應聲作書宣示朝廷|{
	隋志譯云考尋樂府鍾石律呂皆有宫商角徵羽變宫變徵之名七聲之内三聲乖應每恒求訪終莫能通先是周武帝時有龜兹人蘇祇婆從突厥皇后入國善胡琵琶聽其所奏一均之中間有七聲因而問之調有七種以其七調勘校七聲冥若合符一曰婆陁力華言平聲即宫聲也二曰雞識華言長聲即南呂聲也三曰沙識華言質直聲即角聲也四曰沙侯加儖華言應聲即變徵聲也五曰沙臘華言應和聲即徵聲也六曰般瞻華言五聲即羽聲也七曰俟利華言斛牛聲即變宫聲也譯因習而彈之始得七聲之正然其就此七調又有五旦之名旦作七調以華言譯之旦者則謂均也其聲亦應黄鍾太簇林鍾姑洗五均已外七律更無調聲譯遂因其所捻琵琶絃柱相飲為均推演其聲更立七均合成十二以應十二律律有七音音立一調故成七調十二律合八十四調旋轉相交盡皆和合仍以其聲考校太樂所奏林鍾之宫應用林鍾為宫乃用黄鍾為宫應用南呂為商乃用太簇為商應用應鍾為角乃取姑洗為角故林鍾一宫七聲三聲並戾其十一宫七十七音例皆乖越莫有通者又以編懸有八因作八音之樂七音之外更立一聲謂之應聲作書二十餘篇以明其指龜兹音丘兹賢曰今音丘勿翻兹音沮帷翻蓋急言耳}
與邳公世子蘇夔議累黍定律時人以音律久無通者非譯夔一朝可定帝素不悦學而牛弘不精音律何妥自恥宿儒反不逮譯等常欲沮壞其事|{
	沮在呂翻壞音怪}
乃立議非十二律旋相為宫及七調|{
	調徒釣翻下同}
競為異議各立朋黨或欲令各造樂待成擇其善者而從之妥恐樂成善惡易見|{
	易弋豉翻}
乃請帝張樂試之先白帝云黄鍾象人君之德及奏黄鍾之調帝曰滔滔和雅甚與我心會妥因奏止用黄鍾一宫不假餘律帝悦從之時又有樂工萬寶常|{
	萬姓也孟子門人有萬章}
妙達鍾律譯等為黄鍾調成奏之帝召問寶常寶常曰此亡國之音也帝不悦寶常請以水尺為律以調樂器上從之|{
	以調如字}
寶常造諸樂器其聲率下鄭譯調二律損益樂器不可勝紀|{
	勝音升}
其聲雅淡不為時人所好|{
	好呼到翻}
太常善聲者多排毁之蘇夔尤忌寶常夔父威方用事凡言樂者皆附之而短寶常寶常樂竟為威所抑寢不行及平陳獲宋齊舊樂器并江左樂工帝令廷奏之歎曰此華夏正聲也乃調五音為五夏二舞登歌房内十四調賓祭用之|{
	五夏昭夏皇夏諴夏需夏肆夏二舞文武二舞登歌升堂上而歌匏竹在下貴人聲也帝龍潜時倚琵琶作歌二首名曰地厚天高託言夫妻之義因即取之為房内曲十四調後周故事懸鍾磬法七正七倍合為十四蓋準變宫變徵凡為七聲有正有倍為十四也夏戶雅翻}
仍詔太常置清商署以掌之時天下既壹異代器物皆集樂府牛弘奏中國舊音多在江左|{
	典午南渡未能備樂石氏之亡樂人頗有自鄴而南者苻堅淮淝之敗晉始獲樂工備金石慕容垂破西燕盡獲苻氏舊樂子寶喪敗其鍾律令李佛等將太樂細伎奔慕容德德子超獻之姚秦以贖其母宋武平姚泓收歸建康故云多在江左}
前克荆州得梁樂|{
	克荆州見一百六十五卷梁元帝承聖三年}
今平蔣州又得陳樂史傳相承|{
	傳直戀翻}
以為合古請加修緝以備雅樂其後魏之樂及後周所用雜有邊裔之聲皆不可用請悉停之冬十二月詔弘與許善心姚察及通直郎虞世基參定雅樂|{
	按煬帝始置通直郎從六品屬謁者臺虞世基傳云以通直郎直内史省其通直散騎侍郎歟品從五}
世基荔之子也|{
	虞荔見一百六十八卷陳世祖天嘉二年荔力制翻}
己巳以黄州摠管周法尚為永州摠管|{
	隋志永安郡後齊置衡州開皇五年改曰黄州零陵郡平陳初置永州摠管府}
安集嶺南給黄州兵三千五百人為帳内陳桂州刺史錢季卿等皆詣法尚降|{
	始安郡梁置桂州降戶江翻}
定州刺史呂子廊|{
	鬱林郡梁置定州}
據山洞不受命法尚擊斬之 以駕部侍郎狄道辛公義為岷州刺史|{
	隋志駕部侍郎屬兵部尚書狄道縣屬金城郡臨洮郡溢樂縣西魏置岷州}
岷州俗畏疫一人病疫闔家避之病者多死公義命皆輿置已之廳事|{
	輿羊茹翻}
暑月病人或至數百聽廊皆滿|{
	聽與廳同}
公義設榻晝夜處其間|{
	處昌呂翻}
以秩禄具醫藥身自省問|{
	省悉景翻}
病者既愈乃召其親戚諭之曰死生有命豈能相染若相染者吾死久矣皆慙謝而去其後人有病者爭就使君|{
	使疏吏翻}
其家親戚固留養之始相慈愛風俗遂變後遷并州刺史下車先至獄中露坐親自驗問十餘日間决遣咸盡方還聽事受領新訟事皆立决若有未盡必須禁者公義即宿聽事終不還閤|{
	還音如字又從宣翻}
或諫曰公事有程使君何自苦公義曰刺史無德不能使民無訟豈可禁人在獄而安寢於家乎罪人聞之咸自欵服|{
	欵誠也欵服猶言誠服也}
後有訟者鄉閭父老遽曉之曰此小事何忍勤勞使君訟者多兩讓而止|{
	使疏吏翻}
十年春正月乙未以皇孫昭為河南王楷為華陽王|{
	華戶化翻}
昭廣之子也 二月上幸晉陽命高熲居守|{
	熲居永翻守手又翻}
夏四月辛酉至自晉陽 成安文子李德林|{
	成安縣名文謚也子爵也成安縣屬魏郡}
恃其才望論議好勝|{
	好呼到翻}
同列多疾之由是以佐命元功十年不徙級德林數與蘇威異議高熲常助威奏德林狠戾|{
	數所免翻狠戶墾翻}
上多從威議上賜德林莊店使自擇之德林請逆人高阿那肱衛國縣市店|{
	高阿那肱與王謙舉兵誅衛國縣本漢觀縣屬東郡光武改曰衛國魏收地形志屬頓丘郡隋開皇六年改曰觀城屬武陽郡}
上許之及幸晉陽店人訴稱高氏強奪民田於内造店賃之|{
	賃乃禁翻}
蘇威因奏德林誣罔妄奏自入司農卿李圓通等復助之曰此店收利如食千戶請計日追上自是益惡之|{
	復扶又翻下同惡烏路翻}
虞慶則等奉使關東巡省還|{
	使疏吏翻省悉景翻還從宣翻又如字}
皆奏稱鄉正專理辭訟黨與愛憎公行貨賄不便於民上令廢之德林曰兹事臣本以為不可然置來始爾|{
	置鄉正見上年}
復即停廢政令不一朝成暮毁深非帝王設法之義臣望陛下自今羣臣於律令輒欲改張即以軍法從事不然者紛紜未巳上遂發怒大詬云爾欲以我為王莽邪|{
	上以權數得國猜疑羣下以王莽簒漢變更法令而亡疑德林以况巳故怒詬苦候翻邪音耶}
先是德林稱父為太尉諮議以取贈官給事黄門侍郎猗氏陳茂等密奏德林父終於校書妄稱諮議上甚銜之|{
	德林之父蓋仕於魏齊之間後齊之制公府諮議參軍從第四品校書郎第九品猗氏縣屬河東郡}
至是上因數之|{
	數所具翻又所主翻}
曰公為内史典朕機密比不可豫計議者|{
	比毗至翻}
以公不弘耳寧自知乎又罔冒取店妄加父官朕實忿之而未能發今當以一州相遣耳因出為湖州刺史|{
	烏程縣舊置吳興郡隋置湖州宋白曰湖州古防風氏之國漢烏程縣之地隋置湖州因太湖而名長安東南三千四百四十一里}
德林拜謝曰臣不敢復望内史令請但預散參|{
	復扶又翻又音如字散參謂散官無職務而預朝參者散悉亶翻}
上不許遷懷州刺史|{
	河内郡舊置懷州}
而卒|{
	卒子恤翻}
李圓通本上微時家奴有器幹及為隋公以圓通及陳茂為參佐由是信任之梁國之廢也|{
	梁國廢見上卷陳長城公禎明元年}
上以梁太府卿柳莊為給事黄門侍郎莊有識度博學善辭令明習典故雅達政事上及高熲皆重之與陳茂同僚不能降意茂譛之於上上稍疎之出為饒州刺史|{
	隋志鄱陽郡梁置吳州陳廢隋平陳置饒州}
上性猜忌不悦學既任智以獲大位因以文法自矜明察臨下恒令左右覘視内外有過失則加以重辠|{
	恒戶登翻覘丑亷翻又丑艷翻}
又患令史汙私使人以錢帛遺之|{
	遺于季翻}
得犯立斬每於殿庭捶人一日之中或至數四嘗怒問事揮楚不甚|{
	問事者行杖之人也楚荆也以之箠人箠止橤翻}
即命斬之尚書左僕射高熲治書侍御史柳彧等諫|{
	治直之翻彧於六翻}
以為朝堂非殺人之所殿廷非决罰之地上不納熲等乃盡詣朝堂請罪|{
	朝直遥翻熲居永翻}
上顧謂領左右都督田元曰|{
	後齊之制有領左右府將軍之下置正副都督隋蓋因之煬帝改領左右府為備身府}
吾杖重乎元曰重帝問其狀元舉手曰陛下杖大如指捶人三十者比常杖數百故多死|{
	捶止橤翻下同}
上不懌乃令殿内去杖|{
	去羌呂翻}
欲有决罰各付所由|{
	所由猶言所主也}
後楚州行參軍李君才|{
	隋志江都郡山陽縣舊置山陽郡開皇十二年置楚州隋制州置刺史長史司馬參軍事行參軍}
上言上寵高熲過甚|{
	上言之上時掌翻}
上大怒命杖之而殿内無杖遂以馬鞭捶殺之|{
	捶止橤翻}
自是殿内復置杖未幾怒甚又於殿廷殺人|{
	復扶又翻幾居豈翻}
兵部侍郎馮基固諫|{
	兵部尚書統兵部職方駕部庫部四曹各置侍郎}
上不從竟於殿廷殺之上亦尋悔宣慰馮基而怒羣臣之不諫者 五月乙未詔曰魏末喪亂軍人權置坊府|{
	元魏之季兵制有六坊後齊因之亦曰六府喪息浪翻}
南征北伐居處無定|{
	處昌呂翻}
家無完堵地罕包桑|{
	包桑多根植桑至於根多民安其居之驗}
朕甚愍之凡是軍人可悉屬州縣墾田籍帳一與民同軍府統領宜依舊式罷山東河南及北方緣邊之地新置軍府 六月辛酉制民年五十免役收庸 秋七月癸卯以納言楊素為内史令冬十一月辛丑上祀南郊|{
	隋南郊為壇於國之南太陽門外道西一里去宫十}


|{
	里壇高七尺廣四丈孟春上辛祠所感帝赤慓怒於其上以太祖武元皇帝配}
江表自東晉已來刑法疏緩世族陵駕寒門平陳之後牧民者盡更變之|{
	更工衡翻}
蘇威復作五教使民無長幼悉誦之士民嗟怨民間復訛言隋欲徙之入關|{
	復扶又翻}
遠近驚駭於是婺州汪文進越州高智慧蘇州沈玄懀皆舉兵反|{
	隋志東陽郡平陳置婺州會稽郡梁置東陽州陳改吳州平陳改吳州後改越州吳郡陳置吳州平陳改蘇州懀烏外翻}
自稱天子署置百官樂安蔡道人蔣山李㥄饒州吳世華温州沈孝徹泉州王國慶杭州楊寶英交州李春等皆自稱大都督|{
	攷隋志無樂安下曰陳之故境則當於陳境求之沈約志鄱陽太守有樂安縣吳立新唐志台州有樂安縣唐初析臨海置以下文汪文進令蔡道人守樂安觀之蓋台州之樂安蔣山在蔣州江寧縣永嘉郡開皇九年置處州十二年改栝州唐高宗上元元年始析栝州之永嘉安固置温州安陸郡京山縣有温州非陳境當是永嘉之温州史追書耳建安郡陳置閩州平陳改曰泉州餘杭郡平陳置杭州交趾郡舊曰交州㥄力膺翻}
攻陷州縣陳之故境大抵皆反大者有衆數萬小者數千共相影響執縣令或抽其腸或臠其肉食之曰更能使儂誦五教邪|{
	邪音耶}
詔以楊素為行軍摠管以討之素將濟江使始興麥鐵杖戴束藳夜浮渡江|{
	隋志南海郡始興縣南齊置東衡州平陳權置廣州摠管府姓苑高要始興有麥姓}
覘賊還而復往為賊所擒|{
	覘丑亷翻又丑艶翻復扶又翻還從宣翻}
遣兵仗三十人防之鐵杖取賊刀亂斬防者殺之皆盡割其鼻懷之以歸素大奇之奏授儀同三司素帥舟師自揚子津入|{
	楊子津在今眞州楊子縣南帥讀曰率}
擊賊帥朱莫問於京口破之|{
	京口今鎮江府帥所類翻下同}
進擊晉陵賊帥顧世興無錫賊帥葉略皆平之|{
	隋志晉陵無錫二縣皆屬常州 考異曰北史楊素傳作葉皓今從隋書}
沈玄懀敗走素追擒之高智慧據浙江東岸為營周亘百餘里船艦被江|{
	艦戶黯翻被皮義翻}
素擊之子摠管南陽來護兒|{
	子摠管禆將也領兵屬摠管南陽郡舊置荆州開皇初改為鄧州姓苑郲子姓商之支孫食采於郲因以為氏後避難去邑漢功臣表有軑侯來倉}
言於素曰吳人輕鋭利在舟楫必死之賊難與爭鋒公宜嚴陳以待之|{
	陳讀曰陣}
勿與接刃請假奇兵數千潜度江掩破其壁使退無所歸進不得戰此韓信破趙之策也|{
	韓信破趙見十卷漢高帝三年}
素從之護兒以輕舸數百|{
	舸古我翻}
直登江岸襲破其營因縱火烟焰漲天賊顧火而懼素因縱兵奮擊大破之賊遂潰智慧逃入海素躡之至海曲召行軍記室封德彞計事|{
	姓苑封姓夏時封父之後}
德彛墜水人救獲免易衣見素竟不自言素後知之問其故曰私事也所以不白素嗟異之德彛名倫以字行隆之之孫也|{
	封隆之高齊佐命臣}
汪文進以蔡道人為司空守樂安素進討悉平之素遣摠管史萬歲帥衆二千自婺州别道踰嶺越海攻破溪洞不可勝數|{
	帥讀曰率勝音升}
前後七百餘戰轉鬭千餘里寂無聲問者十旬遠近皆以萬歲為沒萬歲置書竹筒中浮之於水汲者得之言於素素上其事|{
	上時掌翻}
上嗟歎賜萬歲家錢十萬素又破沈孝徹於温州步道向天台指臨海|{
	按新唐志天台山在台州唐興縣唐興本晉始豐縣始豐本吳之始平縣唐志云武德四年析臨海置始豐高宗上元二年更名唐興則吳之始豐隋已併入臨海天台山此時固屬臨海界}
逐捕遺逸前後百餘戰高智慧走保閩越上以素久勞於外令馳傳入朝|{
	傳株戀翻下同朝直遥翻}
素以餘賊未殄恐為後患復請行遂乘傳至會稽|{
	素既入朝復自長安乘傳至會稽復扶又翻傳株戀翻會工外翻}
王國慶自以海路艱阻非北人所習不設備素泛海奄至國慶惶遽弃州走餘黨散入海島或守溪洞素分遣諸將水陸追捕密令人說國慶使斬送智慧以自贖國慶乃執送智慧斬於泉州餘黨悉降|{
	將即亮翻說輸芮翻降戶江翻}
江南大定素班師上遣左領軍將軍獨孤陀至浚儀迎勞比到京師問者日至|{
	陀徒何翻勞力到翻比必寐翻}
拜素子玄奬為儀同三司賞賜甚厚陀信之子也|{
	獨孤信皇后之父後周功臣}
楊素用兵多權略馭衆嚴整每將臨敵輒求人過失而斬之多者百餘人少不下十數流血盈前言笑自若及其對陳先令一二百人赴敵陷陣則已|{
	少詩沼翻陳讀曰陣令力丁翻}
如不能陷而還者無問多少悉斬之又令二三百人復進|{
	還從宣翻復扶又翻}
還如向法將士股慄有必死之心由是戰無不勝稱為名將素時貴幸言無不從其從素行者微功必録至佗將|{
	將即亮翻}
雖有大功多為文吏所譴却故素雖殘忍士亦以此願從焉 以并州摠管晉王廣為揚州摠管鎮江都復以秦王俊為并州摠管|{
	復扶又翻}
番禺夷王仲宣反|{
	廣州舊治番禺隋為南海縣又分置番禺縣時廣州治始興仲宣所圍者南海也番禺音潘愚}
嶺南首領多應之引兵圍廣州韋洸中流矢卒|{
	洸古黄翻}
詔以其副慕容三藏檢校廣州道行軍事又詔給事郎裴矩巡撫嶺南|{
	唐六典云隋開皇六年始置六品以下散官並以郎為正階尉為從階正六品上為朝議郎下為武騎尉從六品上為通議郎下為屯騎尉正七品上為朝請郎下為驍騎尉從七品上為朝散郎下為游騎尉正八品上為給事郎下為飛騎尉從八品上為承奉郎下為旅騎尉正九品上為儒林郎下為雲騎尉從九品上為文林郎下為羽騎尉隋志煬帝減給事黄門侍郎員去給事之名移吏部給事郎名為門下之職位次黄門下此時裴矩蓋為吏部給事郎}
矩至南康得兵數千人仲宣遣别將周師舉圍東衡州|{
	東衡州亦治始興將即亮翻}
矩與大將軍鹿愿擊斬之|{
	鹿姓也風俗通後漢有巴郡太守鹿旗}
進至南海高凉洗夫人遣其孫馮暄將兵救廣州暄與賊將陳佛智素善逗留不進夫人知之大怒遣使執暄繫州獄|{
	洗悉典翻又先薦翻將即亮翻使疏吏翻}
更遣孫盎出討佛智斬之進會鹿愿於南海與慕容三藏合擊仲宣|{
	藏徂浪翻}
仲宣衆潰廣州獲全洗氏親被甲乘介馬張錦繖|{
	繖線旦翻}
引彀騎衛從裴矩巡撫二十餘州|{
	彀古侯翻騎奇寄翻從才用翻}
蒼梧首領陳坦等皆來謁見|{
	隋志蒼梧郡梁置成州隋置封州}
矩承制署為刺史縣令使還統其部落嶺表遂定矩復命上謂高熲楊素曰韋洸將二萬兵不能早度嶺朕每患其兵少|{
	將即亮翻少詩沼翻}
裴矩以三千敝卒徑至南海有臣若此朕亦何憂以矩為民部侍郎|{
	民部侍郎屬戶部尚書}
拜馮盎高州刺史|{
	高凉郡舊置高州}
追贈馮寶廣州摠管譙國公册洗氏為譙國夫人開譙國夫人幕府置長史以下官屬官給印章聽發部落六州兵馬若有機急便宜行事仍勅以夫人誠效之故特赦暄逗留之罪拜羅州刺史|{
	宋白曰羅州本招義郡秦屬象郡二漢屬合浦郡元嘉三年檀道濟于綾羅江口築石城因置羅州}
皇后賜夫人首飾及宴服一襲夫人並盛於金篋|{
	盛時征翻}
并梁陳賜物各藏一庫每歲時大會陳之於庭以示子孫曰我事三代主惟用一忠順之心今賜物具存此其報也汝曹皆念之盡赤心於天子番州摠管趙訥貪虐|{
	按隋志廣州治南海仁夀元年置番州趙訥貪虐必非是年事史因書之}
諸俚獠多亡叛|{
	俚音里獠魯皓翻}
夫人遣長史張融上封事論安撫之宜并言訥罪不可以招懷遠人上遣推訥得其賄竟致於法委夫人招慰亡叛夫人親載詔書自稱使者歷十餘州宣述上意諭諸俚獠所至皆降|{
	使疏吏翻降戶江翻}
上嘉之賜夫人臨振縣為湯沐邑|{
	臨振縣漢朱崖地隋煬帝置臨振郡今吉陽軍}
贈馮僕崖州摠管|{
	隋志朱崖郡梁置崖州}
平原公|{
	平原郡公也}


十一年春正月皇太子妃元氏薨|{
	為帝與皇后怒太子而廢之張本}
二月戊午吐谷渾遣使入貢吐谷渾可汗夸呂聞陳亡大懼|{
	吐從暾入聲谷音浴使疏吏翻可從刊入聲汗音寒}
遁逃保險不敢為寇夸呂卒子世伏立使其兄子無素奉表稱藩并獻方物請以女備後庭上謂無素曰若依來請佗國聞之必當相傚何以拒之朕情存安養各令遂性豈可聚斂子女以實後宫乎竟不許 平鄉令劉曠有異政|{
	平鄉縣屬襄國郡}
以義理曉諭訟者皆引咎而去獄中草滿庭可張羅遷臨潁令|{
	臨潁縣屬潁川郡潁川郡時為許州}
高熲薦曠清名善政為天下第一上召見勉之|{
	見賢遍翻}
謂侍臣曰若不殊奬何以為勸丙子優詔擢為莒州刺史|{
	隋志琅邪郡沂水縣舊置南青州後周改為莒州}
辛巳晦日有食之 初帝微時與滕穆王瓚不協帝為周相以瓚為大宗伯瓚恐為家禍隂欲圖帝帝隱之|{
	隋書瓚傳瓚美姿儀好書愛士有令名於當世周宣帝崩帝入禁中將總朝政令世子勇召之欲有計議瓚素與帝不協聞召不從曰作隨國公恐不能保何乃更為族滅事耶帝相周瓚拜大宗伯瓚見羣情未一恐為家禍隂有圖帝之計以是言之固周之忠臣也瓚藏旱翻}
瓚妃周高祖妹順陽公主也與獨孤后素不平隂為呪詛|{
	呪職救翻詛莊助翻}
帝命出之瓚不可秋八月瓚從帝幸栗園|{
	栗園在長安南}
暴薨時人疑其遇鴆乙亥帝至自栗園 沛達公鄭譯卒|{
	卒子恤翻}


資治通鑑卷一百七十七
