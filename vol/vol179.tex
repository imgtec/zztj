










 


 
 


 

  
  
  
  
  





  
  
  
  
  
 
  

  

  
  
  



  

 
 

  
   




  

  
  


    資治通鑑卷一百七十九 宋 司馬光 撰

  胡三省 音注

  隋紀三【起上章涒灘盡昭陽大淵獻凡四年}


  高祖文皇帝中

  開皇二十年春二月熙州人李英林反【隋志同安郡梁置豫州後改晉州後齊改江州陳復曰晉州開皇初曰熙州因晉熙郡名州也}
三月辛卯以揚州緫管司馬河内張衡為行軍緫管【隋制緫管府置長史司馬河内郡置懷州}
帥步騎五萬討平之【帥讀曰率騎奇寄翻}
 賀若弼復坐事下獄【若人者翻復扶又翻下遐嫁翻}
上數之曰【數所具翻又所主翻}
公有三太猛嫉妬心太猛自是非人心太猛無上心太猛既而釋之它日上謂侍臣曰弼將伐陳謂高熲曰陳叔寶可平也不作高鳥盡良弓藏邪【熲居永翻下同范蠡告大夫種嘗有是言邪音耶下同}
熲云必不然及平陳遽索内史又索僕射【索山客翻射寅謝翻}
我語熲曰【語牛倨翻下同}
功臣正宜授勲官【隋置上柱國至帥都督凡十一等為勲官}
不可預朝政【朝直遙翻}
弼後語熲皇太子於已出口入耳無所不盡公終久何必不得弼力何脉脉邪【脉脉有言不得吐之意}
意圖廣陵又圖荆州皆作亂之地意終不改也 夏四月壬戌突厥逹頭可汗犯塞【厥九勿翻可從刋入聲汗音寒}
詔命晉王廣楊素出靈武道【即靈州道}
漢王諒史萬歲出馬邑道【即朔州道}
以擊之長孫晟帥降人為秦州行軍緫管【天水郡置秦州長知兩翻晟承正翻帥讀曰率降戶江翻}
受晉王節度晟以突厥飲泉易可行毒【易以䜴翻}
因取諸藥毒水上流突厥人畜飲之多死于是大驚曰天雨惡水【雨于具翻}
其亡我乎因夜遁晨追之斬首千餘級 【考異曰煬帝紀曰出靈武無虜而還突厥傳曰晉王出靈州逹頭逃遁而去晟傳曰逹頭與王相抗蓋逹頭聞王來而遁晟將兵從别道與逹頭相遇耳}
史萬歲出塞至大斤山與虜相遇逹頭遣使問隋將為誰候騎報史萬歲也突厥復問得非敦煌戍卒乎【史萬歲戍敦煌事見一百七十五卷陳長城公至德元年使疏吏翻下同將即亮反騎奇寄翻復扶又翻下同敦徒門翻}
候騎曰是也逹頭懼而引去萬歲馳追百餘里縱擊大破之斬數千級逐北入磧數百里虜遠遁而還【磧七迹翻下同還從宣翻又如字 考異曰帝紀十九年六月史萬歲破賊據本傳在今年紀誤也按破賊當作破逹頭}
詔遣長孫晟復還大利城安撫新附逹頭復遣其弟子俟利伐從磧東攻唘民上又發兵助唘民守要路俟利伐走入磧唘民上表陳謝曰大隋聖人可汗憐養百姓如天無不覆【上時掌翻覆敷又翻}
地無不載染干如枯木更葉枯骨更肉千世萬世常為大隋典羊馬也【爲于偽翻}
帝又遣趙仲卿為唘民築金河定襄二城【隋志榆林郡金河縣隋初置榆關緫管定襄即雲内縣之恒安鎮}
 秦孝王俊久疾未能起遣使奉表陳謝上謂其使者曰我戮力創茲大業作訓垂範庶臣下守之汝為吾子而欲敗之【使疏吏翻敗補邁翻}
不知何以責汝俊慙怖【怖普布翻}
疾遂篤乃復拜俊上柱國六月丁丑俊薨【帝五子獨俊病死耳}
上哭之數聲而止俊所為侈麗之物悉命焚之王府僚佐請立碑【隋親王置師友文學長吏司馬諮議參軍掾屬主簿録事功曹記室戶倉兵等曹騎兵城局等參軍東西閣祭酒參軍事法田水鎧士等曹行參軍行參軍長兼行參軍典籖等釋名碑者葬時所設臣子追述君父之功以書其上初學記碑悲也所以上往事也}
上曰欲求名一卷史書足矣何用碑為若子孫不能保家徒與人作鎮石耳【鎮之人翻壓也}
俊子浩崔妃所生也庶子曰湛羣臣希旨奏漢之栗姬子榮郭后子彊皆隨母廢【栗姬子榮事見十六卷漢景帝六年七年郭后子彊事見四十三卷漢光武建武十七年十九年斯二事者二帝之失也可以為法乎}
今秦王二子母皆有罪不合承嗣上從之以秦國官為喪主 初上使太子勇參决軍國政事時有損益上皆納之勇性寛厚率意任情無矯飾之行【行下孟翻}
上性節儉勇嘗文飾蜀鎧【蜀鎧蜀人所作也蜀人工巧所作鎧甲已精麗而勇又文飾之}
上見而不悦戒之曰自古帝王未有好奢侈而能久長者汝為儲后【后君也儲后猶言儲君也好呼到翻}
當以儉約為先乃能奉承宗廟吾昔日衣服各留一物時復觀之以自警戒恐汝以今日皇太子之心忘昔時之事故賜汝以我舊所帶刀一枚并葅醬一合【淹菜為葅醬醢也肉醬跂醬皆謂之醢又菜葅謂之醬内則芥醬}
汝㫺作上士時常所食也【謂勇仕周時}
若存記前事應知我心後遇冬至百官皆詣勇勇張樂受賀上知之問朝臣曰近聞至日内外百官相帥朝東宫此何禮也太常少卿辛亶對曰於東宫乃賀也不得言朝【朝直遙翻帥讀曰率少始照翻}
上曰賀者正可三數十人隨情各去何乃有司徵召一時普集太子法服設樂以待之可乎【隋制太子衮冕垂白珠九旒青纊充耳犀笄玄衣纁裳衣山龍華蟲大宗彞五章裳藻粉米黼黻四章織成為之白紗内單黼領青褾襈裙革帶金鈎鍱大帶素帶不朱裏亦紕以朱綠隨裳色火山二章玉具劒火珠鏢首瑜玉雙珮朱組雙大綬四采赤白縹紺純朱質長一丈八尺三百二十首廣九寸小雙綬長二尺六寸色同大綬而首半之間施二玉環朱韈赤舄以金飾褾彼小翻襈雛冕翻鍱丑例翻又彼列翻紕頻彌翻鏢紕招翻綬音受縹匹沼翻純之尹翻長直亮翻廣苦曠翻}
因下詔曰禮有等差君臣不雜皇太子雖居上嗣義兼臣子而諸方岳正冬朝賀任土作貢别上東宫【别上時掌翻}
事非典則宜悉停斷【斷丁管翻}
自是恩寵始衰漸生猜阻勇多内寵昭訓雲氏尤幸【姓苑雲姓縉雲氏之後又魏書官氏志逹宥氏後改雲氏此其後也}
其妃元氏無寵遇心疾二日而薨【元妃薨見一百七十七卷十一年}
獨孤后意有佗故甚責望勇自是雲昭訓專内政生長寜王儼平原王裕安成王筠高良娣生安平王嶷襄成王恪王良媛生高陽王該建安王韶成姬生潁川王煚【娣音弟嶷魚力翻媛于眷翻煚居永翻}
後宫生孝實孝範后彌不平頗遣人伺察求勇過惡晉王廣彌自矯飾唯與蕭妃居處【伺相吏翻處昌呂翻}
後庭有子皆不育后由是數稱廣賢【數所角翻}
大臣用事者廣皆傾心與交【謂楊素等}
上及后每遣左右至廣所無貴賤廣必與蕭妃迎門接引為設美饌【為于偽翻饌士戀翻說文具食也}
申以厚禮【申重也}
婢僕往來者無不稱其仁孝上與后嘗幸其第廣悉屏匿美姬於别室【屏必郢翻}
唯留老醜者衣以縵綵【衣於既翻縵莫半翻縵無文者也}
給事左右屏帳改用縑素故絶樂器之絃不令拂去塵埃【去羌呂翻}
上見之以為不好聲色還宫以語侍臣【好呼到翻語牛倨翻}
意甚喜侍臣皆稱慶由是愛之特異諸子上密令善相者來和【相息亮翻}
徧視諸子對曰晉王眉上雙骨隆起貴不可言上又問上儀同三司韋鼎我諸兒誰得嗣位對曰至尊皇后所最愛者當與之非臣敢預知也【來和韋鼎皆識帝於潛躍故尤信之}
上笑曰卿不肯顯言邪【邪音耶}
晉王廣美姿儀性敏慧沈深嚴重【沈持林翻}
好學善屬文【好呼到翻屬之欲翻}
敬接朝士禮極卑屈由是聲名籍甚【言聲名狼籍甚盛朝直遙翻下同}
冠於諸王【冠古玩翻}
廣為揚州緫管入朝將還鎮入宫辭后伏地流涕后亦泫然泣下【泫戶畎翻}
廣曰臣性識愚下常守平生昆弟之意不知何罪失愛東宫恒蓄盛怒欲加屠陷【恒戶登翻}
每恐讒譛生於投杼【用曾參母子事}
鴆毒遇於杯勺【杯勺皆飲器周禮梓人為飲器勺一升勺市若翻}
是以勤憂積念懼履危亡后忿然曰睍地伐漸不可耐【勇小字睍地伐}
我為之娶元氏女【為于偽翻}
竟不以夫婦禮待之專寵阿雲【阿烏葛翻}
使有如許豚犬【曹操曰如袁本初劉景升兒豚犬耳後遂以詆其子}
前新婦遇毒而夭【夭於紹翻}
我亦不能窮治【治直之翻}
何故復於汝發如此意【復扶又翻}
我在尚爾我死後當魚肉汝乎每思東宫竟無正嫡至尊千秋萬歲之後遣汝等兄弟向阿雲兒前再拜問訊此是幾許苦痛邪【幾居豈翻邪音耶}
廣又拜嗚咽不能止后亦悲不自勝【勝音升}
自是后決意欲廢勇立廣矣廣與安州緫管宇文述素善【安陸郡置安州}
欲述近已奏為夀州刺史【淮南郡舊屬南則為豫州屬北則為揚州開皇九年改曰夀州近其靳翻}
廣尤親任緫管司馬張衡衡為廣畫奪宗之策廣問計於述述曰皇太子失愛已久令德不聞於天下大王仁孝著稱才能蓋世數經將領頻有大功【謂南平陳北伐突厥也數所角翻將即亮翻}
主上之與内宫咸所鍾愛【内宫即中宫避國諱故云然}
四海之望實歸大王然廢立者國家大事處人父子骨肉之間誠未易謀也【處昌呂翻易以䜴翻}
然能移主上意者惟楊素耳素所與謀者唯其弟約述雅知約請朝京師與約相見共圖之【朝直遙翻}
廣大悦多齎金寶資述入關約時為大理少卿【少始照翻}
素凡有所為皆先籌於約而後行之述請約盛陳器玩與之酣暢因而共博每陽不勝所齎金寶盡輸之約約所得既多稍以謝述述因曰此晉王之賜令述與公為歡樂耳【此呂不韋之故智耳}
約大驚曰何為爾述因通廣意說之曰【說輸芮翻}
夫守正履道固人臣之常致反經合義亦逹者之令圖【夫音扶令力正翻}
自古賢人君子莫不與時消息以避禍患公之兄弟功名蓋世當塗用事有年矣朝臣為足下家所屈辱者可勝數哉【朝直遙翻勝音升數所具翻}
又儲后以所欲不行每切齒於執政公雖自結於人主而欲危公者固亦多矣主上一旦弃羣臣公亦何以取庇今皇太子失愛於皇后主上素有廢黜之心此公所知也今若請立晉王在賢兄之口耳誠能因此時建大功王必永銘骨髓斯則去累卵之危成太山之安也約然之因以白素素聞之大喜撫掌曰吾之智思【思相吏翻}
殊不及此賴汝唘予約知其計行復謂素曰【復扶又翻}
今皇后之言上無不用宜因機會早自結託則長保榮祿傳祚子孫兄若遲疑一旦有變令太子用事恐禍至無日矣【令力丁翻}
素從之後數日素入侍宴微稱晉王孝悌恭儉有類至尊用此揣后意【揣初委翻}
后泣曰公言是也吾兒大孝愛每聞至尊及我遣内使到【内使猶言中使使疏吏翻}
必迎於境首言及違離未嘗不泣又其新婦亦大可憐我使婢去常與之同寢共食豈若睍地伐與阿雲對坐終日酣宴昵近小人【昵尼質翻}
疑阻骨肉我所以益憐阿?者【廣小字阿??眉波翻}
常恐其潛殺之素既知后意因盛言太子不才后遂遺素金【遺于季翻}
使贊上廢立勇頗知其謀憂懼計無所出使新豐人王輔賢造諸厭勝【新豐縣屬京兆厭於協翻}
又於後園作庶人村室屋卑陋勇時於中寢息布衣草褥冀以當之上知勇不自安在仁夀宫使楊素觀勇所為素至東宫偃息未入勇束帶待之素故久不進以激怒勇勇衘之形於言色素還言勇怨望恐有佗變願深防察上聞素譖毁甚疑之后又遣人伺覘東宫【伺相吏翻下同覘丑廉翻又丑艷翻}
纎介事皆聞奏因加誣飾以成其辠上遂疏忌勇乃於玄武門逹至德門【玄武門隋大興宫城正北門至德門在宫城東北隅}
量置候人以伺動静皆隨事奏聞【量音良}
又東宫宿衛之人侍官以上【侍官謂直閣直寢直齋直後備身直長等蓋東宫率府所統略同十二衛府}
名籍悉令屬諸衛府有勇健者咸屏去之【屏必郢翻去羌呂翻}
出左衛率蘇孝慈為淅州刺史【蘇孝慈有器幹故出之隋志淅陽郡西魏置淅州}
勇愈不悦太史令袁充言於上曰臣觀天文皇太子當廢上曰玄象久見【見賢遍翻}
羣臣不敢言耳充君正之子也【袁君正見一百六十二卷梁武帝太清三年}
晉王廣又令督王府軍事姑臧段逹【姑臧縣凉州武威郡治所}
私賂東宫幸臣姬威令伺太子動静密告楊素於是内外諠謗過失日聞段逹因脅姬威曰東宫過失主上皆知之矣已奉密詔定當廢立君能告之則大富貴威許諾即上書告之【上時掌翻}
秋九月壬子上至自仁夀宫【考異曰帝紀丁未至自仁夀宫今從太子勇傳}
翌日御大興殿【開皇三年上入新都名其城曰大興城正殿曰大興殿宫曰大興宫宫北苑曰大興苑或曰帝由大興郡襲封隋公以登大位故以名新都宫殿城苑}
謂侍臣曰我新還京師應開懷歡樂【樂音洛}
不知何意翻邑然愁苦吏部尚書牛弘對曰臣等不稱軄故至尊憂勞【稱尺證翻}
上既數聞譖毁疑朝臣悉知之【數所角翻}
故於衆中發問冀聞太子之過弘對既失旨上因作色謂東宫官屬曰仁夀宫此去不遠而令我每還京師嚴備仗衛如入敵國我為下利【令力丁翻還從宣翻又如字泄利也為于偽翻}
不解衣臥昨夜欲近厠【厠□也近其靳翻}
故在後房恐有警急還移就前殿豈非爾輩欲壞我家國邪【壞音怪邪音耶}
于是執太子左庶子唐令則等數人付所司訊鞫命楊素陳東宫事狀以告近臣素乃顯言之曰臣奉勅向京令皇太子檢校劉居士餘黨【言自仁夀宫奉勅向長安劉居士事見上卷十七年}
太子奉詔作色奮厲骨肉飛騰語臣云【語牛倨翻}
居士黨盡伏法遣我何處窮討爾作右僕射委寄不輕【射音夜}
自檢校之何關我事又云㫺大事不遂我先被誅【謂禪代時事被皮義翻}
今作天子竟乃令我不如諸弟一事以上不得自遂【上時掌翻}
因長歎回視云我大覺身妨去音上曰此兒不堪承嗣久矣皇后恒勸我廢之【恒戶登翻}
我以布衣時所生地復居長【復扶又翻長知兩翻}
望其漸改隱忍至今勇嘗指皇后侍兒謂人曰是皆我物此言幾許異事【幾居豈翻}
其婦初亡【謂元妃薨時也}
我深疑其遇毒嘗責之勇即懟曰會殺元孝矩【孝矩元妃之父懟直類翻}
此欲害我而遷怒耳長寜初生【勇長子儼封長寜王}
朕與皇后共抱養之自懷彼此連遣來索【索山客翻}
且雲定興女在外私合而生想此由來何必是其體胤㫺晉太子取屠家女其兒即好屠割【事見八十三卷晉惠帝元康九年好呼到翻}
今儻非類便亂宗祏【祏音石}
我雖德慙堯舜終不以萬姓付不肖子【堯舜知朱均不肖不付以天下}
我恒畏其加害如防大敵今欲廢之以安天下【恒戶登翻}
左衛大將軍五原公元旻諫曰【元旻封九原郡公五原郡豐州}
廢立大事詔旨若行後悔無及讒言罔極惟陛下察之上不應命姬威悉陳太子辠惡威對曰太子由來與臣語唯意在驕奢且云若有諫者正當斬之不殺百許人自然永息【以文理觀之不字必誤}
營起臺殿四時不輟前蘇孝慈解左衛率【率如字}
太子奮髯揚肘曰大丈夫會當有一日終不忘之決當快意又宫内所須【須求也索也}
尚書多執法不與輒怒曰僕射以下吾會戮一二人使知慢我之禍每云至尊惡我多側庶【惡烏路翻}
高緯陳叔寶豈孽子乎【言二君皆嫡出而亡國孽魚列翻說文庶子曰孽}
嘗令師姥卜吉凶【師姥巫媪也姥女老稱姥莫補翻}
語臣云【語牛倨翻}
至尊忌在十八年此期促矣上泫然曰誰非父母生乃至於此朕近覽齊書【是時李百藥所撰齊書未出帝所覽者蓋崔子發齊紀也泫戶畎翻}
見高歡縱其兒子不勝忿憤安可效尤邪於是禁勇及諸子部分收其黨與【勝音升邪音耶分扶問翻}
楊素舞文巧詆鍜鍊以成其獄居數日有司承素意奏元旻常曲事於勇情存附託在仁夀宫勇使所親裴弘以書與旻題云勿令人見上曰朕在仁夀宫有纖介事東宫必知疾於驛馬怪之甚久豈非此徒邪遣武士執旻於仗【左衛仗也}
右衛大將軍元胄時當下直不去因奏曰臣向不下直者為防元旻耳【為于偽翻}
上以旻及裴弘付獄先是勇見老枯槐問此堪何用或對曰古槐尤宜取火【先悉薦翻}
時衛士皆佩火燧【燧取火之木也}
勇命工造數千枚欲以分賜左右至是獲於庫又藥藏局貯艾數斛【隋志東宫門下坊統司經宫門内直典膳藥藏齋帥六局藏徂浪翻貯直呂翻}
索得之【索山客翻}
大以為怪以問姬威威曰太子此意别有所在至尊在仁夀宫太子常飼馬千匹【飼祥吏翻}
云徑往守城門自然餓死素以威言詰勇【詰去吉翻}
勇不服曰竊聞公家馬數萬匹勇忝備太子馬千匹乃是反乎素又發東宫服玩似加琱飾者悉陳之於庭以示文武羣臣為太子之辠上及皇后迭遣使責問勇勇不服【使疏吏翻下同}
冬十月乙丑上使人召勇勇見使者驚曰得無殺我邪【邪音耶}
上戎服陳兵御武德殿【武德殿在延恩殿西}
集百官立于東面諸親立於西面【諸親謂屬籍宗親也}
引勇及諸子列於殿庭命内史侍郎薛道衡宣詔廢勇及其男女為王公主者勇再拜言曰臣當伏尸都市為將來鑒戒幸蒙哀憐得全性命言畢泣下流襟既而舞蹈而去左右莫不閔默【哀之而不敢言}
長寜王儼上表乞宿衛辭情哀切【上時掌翻}
上覽之閔然楊素進曰伏望聖心同於螫手【蝮蛇螫手壯士斷腕楊素以讒慝滅人天性之親以此為喻亦太甚矣螫施隻翻}
不宜復留意【復扶又翻}
己巳詔元旻唐令則及太子家令鄒文騰【隋志太子家令掌刑法食膳倉庫什物奴婢等事}
左衛率司馬夏侯福【隋左右衛率各置長史司馬夏戶雅翻}
典膳監元淹【隋志典膳局置監丞各二人屬門下坊}
前吏部侍郎蕭子寶前主璽下士何竦【主璽下士後周官也璽斯氏翻}
並處斬妻妾子孫皆沒官【處昌呂翻}
車騎將軍榆林閻毗【閻毗榆林盛樂人以車騎將軍宿衛東宫閻姓也左傳晉有閻嘉騎奇寄翻}
東郡公崔君綽游騎尉沈福寶【開皇六年置武騎屯騎驍騎游騎飛騎旅騎雲騎羽騎八尉其品則正六品以下從九品以上綽昌約翻}
瀛州處士章仇太翼【瀛州河問郡後煬帝謂太翼曰于卿姓章仇四岳之胄與盧同源是賜姓為盧氏孫愐曰漢有章弇因避仇加仇字為章仇氏}
特免死各杖一百身及妻子資財田宅皆沒官副將作大匠高龍乂率更令晉文建【隋率更令掌東宫伎樂漏刻更工衡翻}
通直散騎侍郎元衡【隋制東宫亦有通直散騎侍郎散悉亶翻騎奇寄翻}
皆處盡【處其罪使自盡處昌呂翻}
於是集羣官於廣陽門外宣詔戮之乃移勇於内史省給五品料食賜楊素物三千段元胄楊約並千段賞鞫勇之功也文林郎楊孝政上書諫曰皇太子為小人所誤宜加訓誨不宜廢黜【上時掌翻}
上怒撻其胸初雲昭訓父定興出入東宫無節數進奇服異器以求悦媚【數所角翻}
左庶子裴政屢諫【隋制左庶子領門下坊}
勇不聽政謂定興曰公所為不合法度又元妃暴薨道路籍籍此於太子非令名也公宜自引退不然將及禍定興以告勇勇益疎政由是出為襄州總管【襄陽郡置襄州}
唐令則為勇所昵狎【昵尼質翻}
每令以絃歌敎内人右庶子劉行本責之【隋制右庶子領典書坊}
曰庶子當輔太子以正道何有取媚於房帷之間哉令則甚慙而不能改時沛國劉臻【隋志無沛國劉臻先世仕於江南江南僑置中原郡縣猶以沛國為貫}
平原明克讓【克讓以平原為貫猶劉臻也}
魏郡陸爽【魏郡置相州}
並以文學為勇所親行本怒其不能調護每謂三人曰卿等止解讀書耳【言但能讀書而不能行其所學解尸買翻}
夏侯福嘗于閣内與勇戲福大笑聲聞於外【聞音問夏戶雅翻}
行本聞之待其出數之曰【數所角翻}
殿下寛容賜汝顔色汝何物小人敢為䙝慢因付執法者治之【治直之翻下同}
數日勇為福致請乃釋之【為于偽翻}
勇嘗得良馬欲令行本乘而觀之行本正色曰至尊置臣於庶子欲令輔導殿下非為殿下作弄臣也勇慙而止及勇敗二人已卒上歎曰向使裴政劉行本在勇不至此勇嘗宴宫臣唐令則自彈琵琶歌娬媚娘【娬音武}
洗馬李綱【隋制門下坊司經局置洗馬四人洗悉薦翻}
起白勇曰令則身為公卿職當調護【左右庶子謂之宫卿漢高帝謂四皓曰煩公卒調護太子故言東宫官職當調護}
乃於廣坐自比倡優【坐徂卧翻倡音昌}
進淫聲穢視聽事若上聞令則罪在不測豈不為殿下之累邪【累力瑞翻邪音耶}
臣請速治其罪勇曰我欲為樂耳【治直之翻樂音洛}
君勿多事綱遂趨出及勇廢上召東宫官屬切責之皆惶懼無敢對者綱獨曰廢立大事今文武大臣皆知其不可而莫肯發言臣何敢畏死不一為陛下别白言之乎【為于偽翻下皆為謂為同别彼列翻}
太子性本中人可與為善可與為惡曩使陛下擇正人輔之足以嗣守鴻基今乃以唐令則為左庶子鄒文騰為家令二人唯知以絃歌鷹犬娛悦太子安得不至于是邪【邪音耶}
此乃陛下之過非太子之罪也因伏地流涕嗚咽上慘然良久曰李綱責我非為無理然徒知其一未知其二我擇汝為宫臣而勇不親任雖更得正人何益哉對曰臣所以不被親任者良由姦人在側故也【被皮義翻}
陛下但斬令則文騰更選賢才以輔太子安知臣之終見疎弃也【更工衡翻}
自古廢立冢嫡鮮不傾危【鮮息淺翻}
願陛下深留聖思無貽後悔上不悦罷朝【朝直遙翻}
左右皆為之股栗【為于偽翻}
會尚書右丞缺有司請人上指綱曰此佳右丞也即用之太平公史萬歲還自大斤山楊素害其功言於上曰突厥本降【厥九勿翻降戶江翻}
初不為寇來塞上畜牧耳遂寢之萬歲數抗表陳狀【陳其功狀也數所角翻}
上未之悟上廢太子方窮東宫黨與上問萬歲所在萬歲實在朝堂【朝直遙翻}
楊素曰萬歲謁東宫矣以激怒上上謂為信然令召萬歲時所將將士在朝堂稱寃者數百人【令刀丁翻將即亮翻}
萬歲謂之曰吾今日為汝極言於上事當決矣【為于偽翻}
既見上言將士有功為朝廷所抑詞氣憤厲上大怒令左右㩧殺之【㩧弼角翻又匹角翻擊也}
既而追之不及因下詔陳其罪狀天下共寃惜之十一月戊子立晉王廣為皇太子天下地震【廣始正位儲宫而天下地震其示戒亦昭昭矣}
太子請降章服宫官不稱臣十二月戊午詔從之以宇文述為左衛率始太子之謀奪宗也洪州緫管郭衍預焉【隋志豫章郡平陳置洪州緫管府}
由是徵衍為左監門率【隋志東宫置左右監門率掌詰門禁監工銜翻下同率所律翻}
帝囚故太子勇於東宫付太子廣掌之勇自以廢非其罪頻請見上申寃【見賢遍翻下同申伸也明也}
而廣遏之不得聞勇於是升樹大叫聲聞帝所冀得引見楊素因言勇情志昏亂為癲鬼所著不可復收帝以為然卒不得見【狂病而死者為顛鬼著直略翻復扶又翻卒子恤翻}
初帝之克陳也【開皇九年克陳}
天下皆以為將太平監察御史房彦謙【後齊御史臺置檢校御史十二人隋置監察御史十二人}
私謂所親曰主上忌刻而苛酷太子卑弱諸王擅權【言秦晉蜀三王分據方而也}
天下雖安方憂危亂其子玄齡亦密言於彦謙曰主上本無功德以詐取天下諸子皆驕奢不仁必自相誅夷今雖承平其亡可翹足待彦謙法夀之玄孫也【房法夀見一百三十二卷宋太宗泰始三年}
玄齡與杜果之兄孫如晦【杜果有名周隋聞}
皆預選【選者吏部選宣絹翻}
吏部侍郎高孝基名知人【有知人之名}
見玄齡歎曰僕閱人多矣未見如此郎者異日必為偉器恨不見其大成耳見如晦謂曰君有應變之才必任棟梁之重俱以子孫託之 帝晩年深信佛道鬼神辛巳始詔有毁佛及天尊嶽鎮海瀆神像者以不道論【隋志佛者西域天竺之迦維衛國浄飯王之太子釋迦牟尼捨太子位出家學道勤行精進覺悟一切種智而謂之佛道經云有元始天尊者生於太元之先稟自然之氣冲虛凝遠莫知其極天地淪壞劫數終盡而天尊之體常存不滅嶽者五嶽東嶽太山西嶽華山南嶽衡山北嶽恒山中嶽嵩山隋五嶽各置今又有吳山令蓋吳山亦謂之吳嶽也鎮即周官職方氏揚州其山鎮曰會稽荆州其山鎮曰衡山豫州其山鎮曰華山青州其山鎮曰沂山兖州其山鎮曰岱山雍州其山鎮曰嶽山幽州其山鎮曰醫無閭并州其山鎮曰恒山冀州其山鎮曰霍山隋開皇十四年詔東鎮沂山南鎮會稽山北鎮醫無閭山冀州鎮霍山並就山立祠東海於會稽縣界南海於南海鎮南並近海立祠及四瀆吳山並取側近巫一人主知洒埽十六年又詔北鎮於營州龍山立祠岱嶽華嶽衡嶽恒嶽嵩嶽皆先有廟四瀆江河淮濟}
沙門毁佛像道士毁天尊像者以惡逆論 是歲徵同州刺史蔡王智積入朝【隋志馮翊郡後魏置華州西魏改曰同州朝直遙翻}
智積帝之弟子也【智積帝弟整之子}
性修謹門無私謁自奉簡素帝甚憐之智積有五男止敎讀論語不令交通賓客或問其故智積曰卿非知我者其意蓋恐諸子有才能以致禍也 齊州行參軍章武王伽【齊郡齊州行參軍在諸曹行參軍之下典籖之上杜佑曰隋開皇三年詔佐官以曹為名者並以為司十二年諸州司以從事為名者並改為參軍煬帝置諸司書佐改行參軍為行書佐隋志河間郡平舒縣舊置章武郡伽求迦翻}
送流囚李參等七十餘人詣京師行至滎陽【滎陽縣屬鄭州}
哀其辛苦悉呼謂曰卿輩自犯國刑身嬰縲紲【縲黑索紲攣也所以拘罪人縲力追翻紲息列翻}
固其職也重勞援卒【援送之卒}
豈不愧心哉參等辭謝伽乃悉脱其枷鎖停援卒與約曰某日當至京師如致前却【謂或前或却不能如期}
吾當為汝受死【為于偽翻}
遂捨之而去流人感悦如期而至一無離叛上聞而驚異召見與語稱善久之於是悉召流人令攜負妻子俱入賜宴於殿庭而赦之因下詔曰凡在有生含靈稟性咸知善惡並識是非若臨以至誠明加勸導則俗必從化人皆遷善往以海内亂離德敎廢絶吏無慈愛之心民懷奸詐之意朕思遵聖敎以德化民而伽深識朕意誠心宣導參等感悟自赴憲司明是率土之人非為難敎若使官盡王伽之儔民皆李參之輩刑厝不用【厝七故翻}
其何遠哉乃擢伽為雍令【雍縣岐州治所雍於用翻}
 太史令袁充表稱隋興已後晝日漸長開皇元年冬至之景長一丈二尺七寸二分【長一亘亮翻}
自爾漸短至十七年短於舊三寸七分日去極近則景短而日長去極遠則景長而日短行内道則去極近行外道則去極遠【極北極也}
謹按元命包曰日月出内道璇璣得其常【六緯之書有春秋元命包孔安國曰璇美玉璣者正天文之器璇似宣翻}
京房别對曰太平日行上道升平行次道霸代行下道伏惟大隋唘運上感乾元景短日長振古希有【詩振古如茲毛傳曰振自也}
上臨朝【朝直遙翻}
謂百官曰景長之慶天之祐也今太子新立當須改元宜取日長之意以為年號是後百工作役並加程課以日長故也丁匠苦之【史言袁充誣天以病民}
仁夀元年春正月乙酉朔赦天下改元 以尚書右僕射楊素為左僕射納言蘇威為右僕射 丁酉徙河南王昭為晉王 突厥步迦可汗犯塞敗代州總管韓弘於恒安【鴈門郡隋代州厥九勿翻迦古牙翻可從刋入聲汗音寒敗補邁翻恒戶登翻}
 以晉王昭為内史令 二月乙卯朔日有食之 夏五月己丑突厥男女九萬口來降【降戶江翻}
 六月乙卯遣十六使巡省風俗【使疏吏翻省昔景翻}
 乙丑詔以天下學校生徒多而不精【校戶敎翻}
唯簡留國子學生七十人太學四門及州縣學並廢【漢置太學晉武帝立國子學後國子太學各置博士以授生徒後魏太和二十年於四門置學立四門博士自漢以來郡有文學隋郡縣皆置博士}
殿内將軍河間劉炫【殿内將軍即殿中將軍隋避諱改之屬左右衛河間郡瀛州炫熒絹翻}
上表切諫不聽【上時掌翻}
秋七月改國子學為太學 初帝受周禪恐民心未服故多稱符瑞以耀之其偽造而獻者不可勝計【勝音升}
冬十一月己丑有事於南郊如封禪禮板文備述前後符瑞以報謝云 山獠作亂【獠盧皓翻蜀有獠}
以衛尉少卿洛陽衛文昇為資州刺史【隋志洛陽縣屬河南郡洛州資陽郡西魏置資州治盤石少始照翻}
鎮撫之文昇名玄以字行初到官獠方攻大牢鎮【開皇十三年置大牢縣宋白曰榮州應靈縣本漢南安縣隋置大牢鎮九域志在州西一百五十里}
文昇單騎造其營【騎奇寄翻造七到翻}
謂曰我是刺史衘天子詔安養汝等勿驚懼也羣獠莫敢動于是說以利害渠帥感悦解兵而去【說輸芮翻帥所類翻下同}
前後歸附者十餘萬口帝大悦賜縑二千匹壬辰以文昇為遂州總管【隋志遂寜郡後周置遂州}
 潮成等五州獠反高州酋長馮盎馳詣京師請討之【隋志義安郡梁置東揚州後改曰瀛州平陳置潮州蒼梧郡梁置成州隋後改封州高涼郡置高州酋才由翻長知兩翻}
帝勅楊素與盎論賊形勢素歎曰不意蠻夷中有如是人即遣盎發江嶺兵擊之【江嶺謂江南嶺南也}
事平除盎漢陽太守【隋志漢陽郡後魏曰南秦州西魏曰成州守手又翻}
 詔以楊素為雲州道行軍元帥【隋志定襄郡開皇五年置雲州緫管府治大利}
長孫晟為受降使者【長知兩翻晟承正翻降戶江翻使疏吏翻}
挾唘民可汗北擊步迦【挾戶頰翻可從刋入聲汗音寒迦古牙翻}
二年春三月己亥上幸仁夀宫 突厥思力俟斤等【厥九勿翻俟渠之翻}
南度河掠唘民男女六千口雜畜二十餘萬而去【畜許又翻}
楊素帥諸軍追擊轉戰六十餘里大破之【帥讀曰率}
突厥北走素復進追夜及之【復扶又翻}
恐其越逸令其騎稍後親引兩騎并降突厥二人與虜並行虜不之覺候其頓舍未定趣後騎掩擊【騎奇寄翻趣讀曰促}
大破之悉得人畜以歸唘民自是突厥遠遁磧南無復寇抄【磧七迹翻抄楚交翻}
素以功進子玄感爵柱國賜玄縱爵淮南公【淮南郡公}
 兵部尚書柳述慶之孫也【柳慶見一百六十一卷梁武帝太清二年}
尚蘭陵公主怙寵使氣自楊素之屬皆下之【下遐嫁翻}
帝問符璽直長萬年韋雲起【符璽局屬門下省直長四人萬年屬京兆璽斯氏翻長知兩翻}
外間有不便事可言之述時侍側雲起奏曰柳述驕豪未嘗經事兵機要重非其所堪徒以主壻遂居要職臣恐物議以為陛下官不擇賢專私所愛斯亦不便之大者帝甚然其言顧謂述曰雲起之言汝藥石也可師友之秋七月丙戌詔内外官各舉所知柳述舉雲起除通事舍人【曹魏中書置通事一人掌呈奏案章正始中改為通事舍人屬中書省隋改中書省為内史省}
 益州總管蜀王秀容貌瓌偉【瓖古回翻}
有膽氣好武藝【好呼到翻}
帝每謂獨孤后曰秀必以惡終我在當無慮至兄弟必反矣大將軍劉噲之討西㸑也帝令上開府儀同三司楊武通將兵繼進【此必㸑翫再反時將即亮翻}
秀以嬖人萬智光為武通行軍司馬【嬖卑義翻又博義翻}
帝以秀任非其人譴責之因謂羣臣曰壞我法者子孫也【壞音怪}
譬如猛虎物不能害反為毛間蟲所損食耳遂分秀所統自長史元巖卒後秀漸奢僭【按隋書元巖傳開皇十三年巖卒是後仁夀四年帝卧疾仁夀宫又有黃門侍郎元巖與楊素柳述同侍疾參考廢太子勇傳柳述傳皆然如此則有兩元巖長知兩翻}
造渾天儀多捕山獠充宦者【獠魯皓翻}
車馬被服擬於乘輿【被皮義翻乘䋲證翻}
及太子勇以讒廢晉王廣為太子秀意甚不平太子恐秀終為後患隂令楊素求其罪而譖之【令力丁翻}
上遂徵秀秀猶豫欲謝病不行緫管司馬源師諫【源師即北齊源文宗之子蓋是時亦老矣}
秀作色曰此自我家事何預卿也師垂涕對曰師忝參府幕敢不盡忠聖上有勅追王以淹時月【以當從隋書源師傳作已蜀本作已}
今乃遷延未去百姓不識王心但生異議内外疑駭發雷霆之詔降一介之使王何以自明願王熟計之朝廷恐秀生變戊子以原州緫管獨孤楷為益州揔管【平涼郡置原州}
馳傳代之【傳株戀翻}
楷至秀猶未肯行楷諷諭久之乃就路楷察秀有悔色因勒兵為備秀行四十餘里將還襲楷覘知有備乃止【覘丑廉翻又丑艷翻}
 八月甲子皇后獨孤氏崩太子對上及宫人哀慟絶氣若不勝喪者【勝音升}
其處私室【處昌呂翻}
飲食言笑如平常又每朝令進二溢米而私令取肥肉脯鮓【乾肉為脯釀魚肉為鮓}
置竹筒中以蠟閉口衣襆裹而納之【襆防玉翻帊也}
著作郎王劭上言佛說人應生天上及生無量夀國之時【上時掌翻}
天佛放大光明以香花妓樂來迎【妓渠綺翻}
伏惟大行皇后福善禎符備諸祕記皆云是妙善菩薩【釋典菩普也薩濟也菩薩言能普濟衆生菩薄乎翻薩桑葛翻}
臣謹按八月二十二日仁夀宫内再雨金銀花【雨于具翻}
二十三日大寶殿後夜有神光【大寶殿在仁夀宫中寢殿也}
二十四日卯時永安宫北有自然種種音樂【種章勇翻}
震滿虚室至夜五更【更工衡翻}
奄然如寐遂即升遐與經文所說事皆符驗上覽之悲喜九月丙戌上至自仁夀宫 冬十月癸丑以工部尚書楊逹為納言逹雄之弟也【雄自廣平王改封清漳時又改封安德}
 閏月甲申詔楊素蘇威與吏部尚書牛弘等修定五禮【五禮吉凶軍賓嘉}
上令上儀同三司蕭吉為皇后擇葬地【為于偽翻}
得吉處云卜年二千卜世二百上曰吉凶由人不在於地高緯葬父豈不卜乎俄而國亡正如我家墓田若云不吉朕不當為天子若云不凶吾弟不當戰沒【上弟整從周武帝伐齊至并州力戰而死}
然竟從吉言吉告族人蕭平仲曰皇太子遣宇文左率深謝余云【宇文述時為左衛率率所律翻}
公前稱我當為太子竟有其驗終不忘也今卜山陵務令我早立我立之後當以富貴相報吾語之云後四載太子御天下【語牛倨翻載作亥翻}
若太子得政隋其亡乎吾前紿云【紿徒亥翻}
卜年二千者三十字也卜世二百者取世二傳也汝其識之【識職吏翻記也}
壬寅葬文獻皇后於太陵詔以楊素經營葬事勤求吉地論素此心事極誠孝豈與夫平戎定寇比【夫音扶}
其功業可别封一子義康公邑萬戶【義康郡公隋志高涼郡杜原縣舊有宋康郡平陳改曰義康郡}
并賜田三十頃絹萬段米萬石金珠綾錦稱是【稱尺證翻}
 蜀王秀至長安上見之不與語明日使使切讓之【使使下疏吏翻}
秀謝罪太子諸王流涕庭謝上曰頃者秦王糜費財物我以父道訓之今秀蠧害生民當以君道繩之於是付執法者開府儀同三司慶整諫曰【慶姓出齊大夫慶氏}
庶人勇既廢秦王已薨陛下見子無多【見賢遍翻}
何至如是蜀王性甚耿介今被重責【被皮義翻}
恐不自全上大怒欲斷其舌【斷丁管翻}
因謂羣臣曰當斬秀於市以謝百姓乃令楊素等推治之【治直之翻}
太子隂作偶人縛手釘心枷鎖杻械【釘丁定翻杻敝九翻}
書上及漢王姓名仍云請西岳慈父聖母收楊堅楊諒神䰟如此形狀勿令散蕩密埋之華山下【華戶化翻}
楊素發之又云秀妄述圖䜟稱京師妖異造蜀地徵神【妖於驕翻徵與禎同}
并作檄文云指期問罪置秀集中【集文集也隋志曰别集者蓋漢東京之所創也自靈均已降屬文之士衆矣然其志尚不同風流殊别後之君子欲觀其體勢而見其心靈故别聚焉名之為集辭人景慕並自記載以成書部}
俱以聞奏上曰天下寜有是邪【邪音耶}
十二月癸巳廢秀為庶人幽之内侍省不聽與妻子相見唯獠婢二人驅使【獠魯皓翻}
連坐者百餘人秀上表摧謝曰【上時掌翻}
伏願慈恩賜垂矜愍殘息未盡之間希與瓜子相見請賜一穴令骸骨有所瓜子其愛子也上因下詔數其十罪【數所具翻}
且曰我不知楊堅楊諒是汝何親後乃聽與其子同處【處昌呂翻}
初楊素嘗以少譴勅送南臺【南臺者御史臺也立國面朝後市臺省皆在南故尚書省曰南省御史臺曰南臺少詩沼翻}
命治書侍御史柳彧治之【治直之翻}
素恃貴坐彧牀彧從外來於階下端笏整容謂素曰奉勅治公之罪素遽下彧據案而坐立素於庭辨詰事狀素由是銜之【詰去吉翻}
蜀王秀嘗從彧求李文博所撰治道集【李文博博陵人仕隋不調性貞介鯁直好學不倦至子敎義名理特所留心讀書至治亂得失忠臣烈士未嘗不反覆吟翫長於議論亦善屬文著治道集十卷大行於世夫其文大行而仕不遇何也治直吏翻}
彧與之秀遺彧奴婢十口【遺于季翻}
及秀得罪素奏彧以内臣交通諸侯除名為民配戍懷遠鎮【新唐志營州有懷遠城}
帝使司農卿趙仲卿往益州窮按秀事秀之賓客經過之處仲卿必深文致法州縣長吏坐者大半【過音戈長知兩翻}
上以為能賞賜甚厚久之貝州長史裴肅【隋志清河郡後周置貝州}
遣使上書稱高熲以天挺良才元勛佐命為衆所疾以至廢弃【熲廢見上卷開皇十九年使疏吏翻上時掌翻}
願陛下錄其大功忘其小過又二庶人得辠已久【二庶人謂勇秀}
寜無革心願陛下弘君父之恩顧天性之義【經曰父子之道天性也}
各封小國觀其所為若能遷善漸更增益如或不悛【悛丑緣翻}
貶削非晩今者自新之路永絶愧悔之心莫見豈不哀哉書奏上謂楊素曰裴肅憂我家事此亦至誠也於是徵肅入朝【朝直遙翻}
太子聞之謂左庶子張衡曰使勇自新欲何為也衡曰觀肅之意欲令如吳太伯漢東海王耳【吳太伯注已見前漢東海王疆事見光武紀此張衡為裴肅解也令力丁翻}
肅至上面諭以勇不可復收之意而罷遣之肅俠之子也【裴俠見一百五十六卷梁武帝中大通五年復扶又翻}
楊素弟約及從父文思文紀【從才用翻}
族父忌並為尚書列卿諸子無汗馬之勞位至柱國刺史廣營資產自京師及諸方都會處邸店碾磑【碾尼展翻丁度集韻碾女箭翻所以離物器也磑五對翻並磨也}
便利田宅不可勝數【勝音升}
家僮千數後庭妓妾曳綺羅者以千數【妓渠綺翻}
第宅華侈制擬宫禁親故吏布列清顯【隋書素傳作親戚故吏此逸戚字}
既廢一太子及一王威權日盛朝臣有違忤者或至誅夷【忤五故翻}
有附會及親戚雖無才用必加進擢朝廷靡然莫不畏附敢與素抗而不撓者【撓奴敎翻屈也}
獨柳彧及尚書右丞李綱大理卿梁毗而已始毗為西寜州刺史【隋志越雋郡後周置嚴州開皇六年改曰西寜州十八年又改曰嶲州毗刺西寜蓋十八年以前也}
凡十一年蠻夷酋長皆以金多者為豪雋遞相攻奪略無寜歲毗患之後因諸酋長相帥以金遺毗【酋才由翻帥讀曰率長知兩翻遺于季翻}
毗置金坐側【坐徂卧翻}
對之慟哭而謂之曰此物飢不可食寒不可衣【衣於既翻}
汝等以此相滅不可勝數【勝音升}
今將此來欲殺我邪【邪音耶}
一無所納於是蠻夷感悟遂不相攻擊上聞而善之徵為大理卿處法平允【處昌呂翻允信也當也}
毗見楊素專權恐為國患乃上封事曰臣聞臣無有作威作福其害于而家凶於而國【書洪範之言上時掌翻}
竊見左僕射越國公素幸遇愈重權勢日隆搢紳之徒屬其視聽【言注耳目也屬之欲翻}
忤旨者嚴霜夏零阿旨者甘雨冬澍【忤五故翻澍之戍翻又殊遇翻}
榮枯由其脣吻廢興候其指麾所私皆非忠讜【讜音黨}
所進咸是親戚子弟布列兼州連縣天下無事容息異圖四海有虞必為禍始【黎陽之變卒如毗言}
夫姦臣擅命有漸而來【夫音扶}
王莽資之於積年桓玄基之於易世而卒殄漢祀終傾晉祚【二事具漢晉紀卒子恤翻}
陛下若以素為阿衡臣恐其心未必伊尹也伏願揆鑒古今量為處置【量音良處昌呂翻}
俾洪基永固率土幸甚書奏上大怒收毗繫獄親詰之【詰去吉翻}
毗極言素擅寵弄權將領之處殺戮無道【將即亮翻}
又太子蜀王罪廢之日百僚無不震竦唯素揚眉奮肘喜見容色【見賢遍翻}
利國家有事以為身幸上無以屈乃釋之其後上亦寖疎忌素乃下勅曰僕射國之宰輔不可躬親細務但三五日一向省評論大事外示優崇實奪之權也素由是終仁夀之末不復通判省事出楊約為伊州刺史【隋志河南郡陸渾縣東魏置伊川郡及北荆州後周改曰和州開皇初又改曰伊州}
素既被疎【被皮義翻}
吏部尚書柳述益用事攝兵部尚書參掌機密【按述傳仁夀中判兵部尚書事尋拜兵部尚書修掌機密抗表陳讓乃令攝兵部尚書事}
素由是惡之【惡烏路翻}
太子問於賀若弼曰楊素韓擒虎史萬歲皆稱良將其優劣何如弼曰楊素猛將非謀將韓擒虎鬬將非領將史萬歲驍將非大將太子曰然則大將誰也弼拜曰唯殿下所擇弼意自許也【若人者翻將即亮翻騎奇寄翻}
 交州俚帥李佛子作亂【交趾郡交州俚音里帥所類翻}
據越王故城【此城蓋秦漢間駱越之王所築也}
遣其兄子大權據龍編城【交州舊治龍編縣隋志治宋平而龍編以縣屬州}
其别帥李普鼎據烏延城【帥所類翻}
楊素薦瓜州刺史長安劉方【敦煌郡置瓜州}
有將帥之畧詔以方為交州道行軍總管統二十七營而進方軍令嚴肅有犯必斬然仁愛士卒有疾病者親臨撫養士卒亦以此懷之至都隆嶺遇賊擊破之進軍臨佛子營先諭以禍福佛子懼請降【降戶江翻}
送之長安

  三年秋八月壬申賜幽州摠管燕榮死【燕因肩翻}
榮性嚴酷鞭撻左右動至千數嘗見道次叢荆以為堪作杖命取之輒以試人人或自陳無罪榮曰後有罪當免汝既而有犯將杖之人曰前日被杖使君許以有罪宥之榮曰無罪尚爾况有罪邪【被皮義翻使疏吏翻邪音耶}
杖之自若觀州長史元弘嗣【隋志平原郡東光縣舊置勃海郡隋廢郡置觀州杜佑曰開皇三年改别駕治中為長史司馬觀古玩翻長知兩翻}
遷幽州長史懼為榮所辱固辭上勅榮曰弘嗣杖十已上罪皆須奏聞榮忿曰豎子何敢玩我於是遣弘嗣監納倉粟颺得一糠一粃皆罸之【監古衘翻颺與章翻又餘亮翻粃音比}
每笞雖不滿十然一日之中或至三數如是歷年怨隙日搆榮遂收弘嗣付獄禁絶其糧弘嗣抽絮雜水咽之【咽於甸翻}
其妻詣闕稱寃上遣使案驗【使疏吏翻}
奏榮暴虐贓穢狼籍徵還賜死元弘嗣代榮為政酷又甚之九月壬戌置常平官【開皇初置義倉今置常平官掌之}
 是歲龍門

  王通詣闕獻太平十二策【隋志龍門縣屬河東郡}
上不能用罷歸通遂敎授於河汾之間弟子自遠至者甚衆累徵不起楊素甚重之勸之仕通曰通有先人之敝廬足以蔽風雨薄田足以具粥【諸延翻厚粥}
讀書談道足以自樂【樂音洛}
願明公正身以治天下【治直之翻}
時和歲豐通也受賜多矣不願仕也或譖通於素曰彼實慢公公何敬焉素以問通通曰使公可慢則僕得矣不可慢則僕失矣得失在僕公何預焉素待之如初弟子賈瓊問息謗通曰無辯問止怨曰不爭通嘗稱無赦之國其刑必平重歛之國其刑必削【歛力贍翻}
又曰聞謗而怒者讒之囮也見譽而喜者佞之媒也絶囮去媒讒佞遠矣【囮余周翻又五戈翻烏媒也爾雅翼曰說文囮譯也率鳥者繫生烏以來之名曰囮讀若譌譽音余去羌呂翻}
大業末卒于家【卒子恤翻}
門人諡曰文中子【通卒門人議曰禮男子生有字所以昭德死有諡所以易名仲尼既沒文不在茲乎易曰黄裳元吉文在中也請諡曰文中子}
 突厥步迦可汗所部大亂鐵勒僕骨等十餘部皆叛步迦降於唘民【隋書鐵勒之先匈奴之苖裔也種類最多自西海之東依據山谷往往不絶獨洛河北有僕骨同羅韋紇拔也古覆羅並號俟斤蒙陳吐如紇斯結渾斛薛等諸姓勝兵可二萬伊吾以西焉耆之北傍白山則有契弊薄洛職乙咥蘇娑郍曷鳥讙紇骨也咥於尼讙等勝兵可二萬金山西有薛延陁咥勒兒十槃逹契等一萬餘兵康國北傍阿得水則有訶咥曷嶻撥忽比干具海曷比悉何嵳蘇拔也末渴逹等有三萬餘兵得嶷海東有蘇路羯三索咽蔑促隆忽等諸姓八千餘拂菻東則有恩屈阿蘭比褥九離伏嗢昏等近二萬人北海南則都波等雖姓氏各别總謂為鐵勒並無君長分屬東西兩突厥人性凶忍善於騎射貪婪尤甚以寇抄為生自突厥有國東西征討皆資其用以制北荒迦古牙翻可從刋入聲汗音寒}
步迦衆潰西奔吐谷渾長孫晟送唘民置磧口唘民於是盡有步迦之衆【磧七降翻}


  資治通鑑卷一百七十九


    


 


 



 

 
  







 


  
  
 
 
 


  

 















	
	









































 
  



















 





 












  
  
  

 





