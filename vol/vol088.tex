<!DOCTYPE html PUBLIC "-//W3C//DTD XHTML 1.0 Transitional//EN" "http://www.w3.org/TR/xhtml1/DTD/xhtml1-transitional.dtd">
<html xmlns="http://www.w3.org/1999/xhtml">
<head>
<meta http-equiv="Content-Type" content="text/html; charset=utf-8" />
<meta http-equiv="X-UA-Compatible" content="IE=Edge,chrome=1">
<title>資治通鑒_89-資治通鑑卷八十八_89-資治通鑑卷八十八</title>
<meta name="Keywords" content="資治通鑒_89-資治通鑑卷八十八_89-資治通鑑卷八十八">
<meta name="Description" content="資治通鑒_89-資治通鑑卷八十八_89-資治通鑑卷八十八">
<meta http-equiv="Cache-Control" content="no-transform" />
<meta http-equiv="Cache-Control" content="no-siteapp" />
<link href="/img/style.css" rel="stylesheet" type="text/css" />
<script src="/img/m.js?2020"></script> 
</head>
<body>
 <div class="ClassNavi">
<a  href="/24shi/">二十四史</a> | <a href="/SiKuQuanShu/">四库全书</a> | <a href="http://www.guoxuedashi.com/gjtsjc/"><font  color="#FF0000">古今图书集成</font></a> | <a href="/renwu/">历史人物</a> | <a href="/ShuoWenJieZi/"><font  color="#FF0000">说文解字</a></font> | <a href="/chengyu/">成语词典</a> | <a  target="_blank"  href="http://www.guoxuedashi.com/jgwhj/"><font  color="#FF0000">甲骨文合集</font></a> | <a href="/yzjwjc/"><font  color="#FF0000">殷周金文集成</font></a> | <a href="/xiangxingzi/"><font color="#0000FF">象形字典</font></a> | <a href="/13jing/"><font  color="#FF0000">十三经索引</font></a> | <a href="/zixing/"><font  color="#FF0000">字体转换器</font></a> | <a href="/zidian/xz/"><font color="#0000FF">篆书识别</font></a> | <a href="/jinfanyi/">近义反义词</a> | <a href="/duilian/">对联大全</a> | <a href="/jiapu/"><font  color="#0000FF">家谱族谱查询</font></a> | <a href="http://www.guoxuemi.com/hafo/" target="_blank" ><font color="#FF0000">哈佛古籍</font></a> 
</div>

 <!-- 头部导航开始 -->
<div class="w1180 head clearfix">
  <div class="head_logo l"><a title="国学大师官网" href="http://www.guoxuedashi.com" target="_blank"></a></div>
  <div class="head_sr l">
  <div id="head1">
  
  <a href="http://www.guoxuedashi.com/zidian/bujian/" target="_blank" ><img src="http://www.guoxuedashi.com/img/top1.gif" width="88" height="60" border="0" title="部件查字,支持20万汉字"></a>


<a href="http://www.guoxuedashi.com/help/yingpan.php" target="_blank"><img src="http://www.guoxuedashi.com/img/top230.gif" width="600" height="62" border="0" ></a>


  </div>
  <div id="head3"><a href="javascript:" onClick="javascript:window.external.AddFavorite(window.location.href,document.title);">添加收藏</a>
  <br><a href="/help/setie.php">搜索引擎</a>
  <br><a href="/help/zanzhu.php">赞助本站</a></div>
  <div id="head2">
 <a href="http://www.guoxuemi.com/" target="_blank"><img src="http://www.guoxuedashi.com/img/guoxuemi.gif" width="95" height="62" border="0" style="margin-left:2px;" title="国学迷"></a>
  

  </div>
</div>
  <div class="clear"></div>
  <div class="head_nav">
  <p><a href="/">首页</a> | <a href="/ShuKu/">国学书库</a> | <a href="/guji/">影印古籍</a> | <a href="/shici/">诗词宝典</a> | <a   href="/SiKuQuanShu/gxjx.php">精选</a> <b>|</b> <a href="/zidian/">汉语字典</a> | <a href="/hydcd/">汉语词典</a> | <a href="http://www.guoxuedashi.com/zidian/bujian/"><font  color="#CC0066">部件查字</font></a> | <a href="http://www.sfds.cn/"><font  color="#CC0066">书法大师</font></a> | <a href="/jgwhj/">甲骨文</a> <b>|</b> <a href="/b/4/"><font  color="#CC0066">解密</font></a> | <a href="/renwu/">历史人物</a> | <a href="/diangu/">历史典故</a> | <a href="/xingshi/">姓氏</a> | <a href="/minzu/">民族</a> <b>|</b> <a href="/mz/"><font  color="#CC0066">世界名著</font></a> | <a href="/download/">软件下载</a>
</p>
<p><a href="/b/"><font  color="#CC0066">历史</font></a> | <a href="http://skqs.guoxuedashi.com/" target="_blank">四库全书</a> |  <a href="http://www.guoxuedashi.com/search/" target="_blank"><font  color="#CC0066">全文检索</font></a> | <a href="http://www.guoxuedashi.com/shumu/">古籍书目</a> | <a   href="/24shi/">正史</a> <b>|</b> <a href="/chengyu/">成语词典</a> | <a href="/kangxi/" title="康熙字典">康熙字典</a> | <a href="/ShuoWenJieZi/">说文解字</a> | <a href="/zixing/yanbian/">字形演变</a> | <a href="/yzjwjc/">金 文</a> <b>|</b>  <a href="/shijian/nian-hao/">年号</a> | <a href="/diming/">历史地名</a> | <a href="/shijian/">历史事件</a> | <a href="/guanzhi/">官职</a> | <a href="/lishi/">知识</a> <b>|</b> <a href="/zhongyi/">中医中药</a> | <a href="http://www.guoxuedashi.com/forum/">留言反馈</a>
</p>
  </div>
</div>
<!-- 头部导航END --> 
<!-- 内容区开始 --> 
<div class="w1180 clearfix">
  <div class="info l">
   
<div class="clearfix" style="background:#f5faff;">
<script src='http://www.guoxuedashi.com/img/headersou.js'></script>

</div>
  <div class="info_tree"><a href="http://www.guoxuedashi.com">首页</a> > <a href="/SiKuQuanShu/fanti/">四库全书</a>
 > <h1>资治通鉴</h1> <!--         下载:【右键另存为】即可 --></div>
  <div class="info_content zj clearfix">
  
<div class="info_txt clearfix" id="show">
<center style="font-size:24px;">89-資治通鑑卷八十八</center>
    資治通鑑卷八十八   宋 司馬光 撰<br />
<br />
  胡三省 音註<br />
<br />
  晉紀十【起玄黓涒灘盡昭陽作噩凡二年】<br />
<br />
  孝懷皇帝下<br />
<br />
  永嘉六年春正月漢呼延后卒諡曰武元 漢鎮北將軍靳冲平北將軍卜珝寇并州【靳居焮翻姓也珝况羽翻】辛未圍晉陽 甲戌漢主聰以司空王育尚書令任顗女為左右昭儀【任音壬顗魚豈翻】中軍大將軍王彰中書監范隆左僕射馬景女皆為夫人右僕射朱紀女為貴妃皆金印紫綬【綬音受】聰將納太保劉殷女太弟义固諫聰以問太宰延年太傅景皆曰太保自云劉康公之後與陛下殊源【劉康公周之卿士食采於劉其後因以為氏劉聰匈奴之後以漢之甥冒姓劉氏故云殊源】納之何害聰悦拜殷二女英娥為左右貴嬪位在昭儀上【嬪毗賓翻】又納殷女孫四人皆為貴人位次貴妃於是六劉之寵傾後宫聰希復出外【復扶又翻】事皆中黄門奏决 故新野王歆牙門將胡亢聚衆於竟陵【亢音剛】自號楚公寇掠荆土以歆南蠻司馬新野杜曾為竟陵太守曾勇冠三軍能被甲游於水中【為曾亂荆州張本冠古玩翻被皮義翻】 二月壬子朔日有食之 石勒築壘於葛陂【皇覽曰汝南郡鮦陽縣有葛陂賢曰葛陂在今豫州新蔡縣西北】課農造舟將攻建業琅邪王睿大集江南之衆於壽春以鎮東長史紀瞻為揚威將軍都督諸軍以討之【睿為鎮東大將軍署瞻長史】會大雨三月不止勒軍中饑疫死者大半聞晉軍將至集將佐議之右長史刁膺請先送欵於睿求掃平河朔以自贖俟其軍退徐更圖之勒愀然長嘯【愀子小翻】中堅將軍夔安請就高避水【中堅將軍蓋石勒所置姓譜夔往春秋夔子之後】勒曰將軍何怯邪孔萇等三十餘將請各將兵分道夜攻壽春斬吳將頭據其城食其粟要以今年破丹陽定江南勒笑曰是勇將之計也【言其不逆計勝敗但勇於赴敵耳將即亮翻】各賜鎧馬一疋【鎧可亥翻疋僻吉翻】顧謂張賓曰於君意何如賓曰將軍攻陷京師囚執天子殺害王公妻略妃主擢將軍之髪不足以數將軍之罪【擢抜也抜其髪以數其罪猶不足言其罪多也數所具翻】奈何復相臣奉乎【復扶又翻】去年既殺王彌不當來此今天降霖雨於數百里中示將軍不應留此也鄴有三臺之固【水經註鄴城西北有三臺皆因城為之基漢建安十五年魏武所起中曰銅臺高十丈其後石虎更增二丈南則金虎臺高八丈北則氷井臺亦高八丈】西接平陽【謂近漢都可以壯聲援】山河四塞宜北徙據之以經營河北河北既定天下無處將軍之右者矣【處昌呂翻】晉之保壽春畏將軍往攻之耳彼聞吾去喜於自全何暇追襲吾後為吾不利邪【自古國於東南率多為自保之計亦自量其力之不足以進也賓料之審矣】將軍宜使輜重從北道先發將軍引大兵向壽春輜重既遠【重直用翻】大兵徐還何憂進退無地乎勒攘袂鼓髯曰張君計是也責刁膺曰君既相輔佐當共成大功奈何遽勸孤降【降戶江翻】此策應斬然素知君怯特相宥耳於是黜膺為將軍擢賓為右長史號曰右侯勒引兵發葛陂遣石虎帥騎二千向壽春【帥讀曰率騎奇寄翻】遇晉運船虎將士爭取之為紀瞻所敗【敗補邁翻】瞻追犇百里前及勒軍勒結陳待之【陳讀曰陣】瞻不敢擊退還壽春 漢主聰封帝為會稽郡公【會工外翻】加儀同三司聰從容謂帝曰卿昔為豫章王朕與王武子造卿武子稱朕於卿【從于容翻王濟字武子造七到翻】卿言聞其名久矣贈朕柘弓銀研【研與硯同】卿頗記否帝曰臣安敢忘之但恨爾日不早識龍顔聰曰卿家骨肉何相殘如此帝曰大漢將應天受命故為陛下自相驅除【為于偽翻】此殆天意非人事也且臣家若能奉武皇帝之業九族敦睦陛下何由得之聰喜以小劉貴人妻帝【妻七細翻】曰此名公之孫也卿善遇之 代公猗盧遣兵救晉陽三月乙未漢兵敗走卜珝之卒先奔靳冲擅收珝斬之聰大怒遣使持節斬冲【使疏吏翻】 聰納其舅子輔漢將軍張寔二女徽光麗光為貴人【此别一張寔非河西張軌之子】太后張氏之意也【張氏淵之側室生聰尊為太后】 涼州主簿馬魴說張軌【魴符方翻說輸芮翻】宜命將出師翼戴帝室軌從之馳檄關中共尊輔秦王且言今遣前鋒督護宋配帥步騎二萬徑趨長安【帥讀曰率趨七喻翻】西中郎將寔帥中軍三萬武威太守張琠帥胡騎二萬絡繹繼發【琠它典翻絡繹相繼不絶之意】 夏四月丙寅征南將軍山簡卒漢王聰封其子敷為渤海王驥為濟南王鸞為燕王鴻為楚王勱為齊王【勱音邁】權為秦王操為魏王持為趙王聰以魚蟹不供斬左都水使者襄陵王攄【襄陵縣漢屬河東郡】<br />
<br />
  【晉屬平陽郡觀後所謂亟斬王公則攄亦劉氏也攄抽居翻】作温明徽光二殿未成斬將作大匠望都公靳陵觀漁於汾水昏夜不歸中軍大將軍王彰諫曰比觀陛下所為臣實痛心疾首【比毗至翻】今愚民歸漢之志未專思晉之心猶盛劉琨咫尺刺客縱横【謂平陽去晉陽不遠也縱子容翻】帝王輕出一夫敵耳願陛下改往修來則億兆幸甚聰大怒命斬之王夫人叩頭乞哀乃囚之【王夫人彰女也】太后張氏以聰刑罰過差三日不食太弟义單于粲輿櫬切諫聰怒曰吾豈桀紂而汝輩生來哭人太宰延年太保殷等公卿列侯百餘人皆免冠涕泣曰陛下功高德厚曠世少比【少詩沼翻】往也唐虞今則陛下而頃來以小小不供亟斬王公直言忤旨遽囚大將【王公謂劉攄靳陵大將謂王彰亟欺冀翻忤五故翻】此臣等竊所未解【解戶買翻曉也】故相與憂之忘寢與食聰慨然曰朕昨大醉非其本心微公等言之朕不聞過各賜帛百匹使侍中持節赦彰曰先帝賴君如左右手君著勲再世朕敢忘之此段之過希君蕩然君能盡懷憂國朕所望也今進君驃騎將軍定襄郡公【驃匹妙翻】後有不逮幸數匡之【數所角翻】 王彌既死【事見上卷上年】漢安北將軍趙固平北將軍王桑恐為石勒所并欲引兵歸平陽軍中乏糧士卒相食乃自䂭磽津西渡【䂭丘交翻磽牛交翻】劉琨以兄子演為魏郡太守鎮鄴桑恐演邀之遣長史臨深為質於琨【姓譜臨姓大臨之後質音致】琨以固為雍州刺史【雍於用翻下同】桑為豫州刺史 賈疋等圍長安數月漢中山王曜連戰皆敗驅掠士女八萬餘口犇于平陽秦王業自雍入于長安【雍於用翻】五月漢主聰貶曜為龍驤大將軍行大司馬【驤思將翻】聰使河内王粲攻傅祇於三渚【據祇傳祇屯盟津小城盟冿河平侯祠有二渚又有淘渚故亦曰三渚】右將軍劉參攻郭默於懷會祇病薨城陷粲遷祇子孫并其士民二萬餘戶于平陽 六月漢主聰欲立責嬪劉英為皇后張太后欲立貴人張徽光聰不得已許之英尋卒 漢大昌文獻公劉殷卒【宋白曰隰州隰川縣漢蒲子縣劉淵僭亂置大昌郡】殷為相不犯顔忤旨【忤五故翻】然因事進規補益甚多漢主聰每與羣臣議政事殷無所是非羣臣出殷獨留為聰敷暢條理商榷事宜【商度也榷者舉其略也為于偽翻榷古岳翻】聰未嘗不從之殷常戒子孫曰事君當務幾諫凡人尚不可面斥其過况萬乘乎夫幾諫之功無異犯顔但不彰君之過所以為優耳【幾諫者見微而諫也侯希聖曰事君有顯諫者有幾諫者然而温柔忠厚者其說多行訐直強勁者其說多忤夫是以貴幾諫也幾居希翻乘繩證翻】官至侍中太保録尚書賜劒履上殿入朝不趨乘輿入殿然殷在公卿間常恂恂有卑讓之色故能處驕暴之國保其富貴不失令名以壽考自終【處昌呂翻】 漢主聰以河間王易為車騎將軍彭城王翼為衛將軍並典兵宿衛高平王悝為征南將軍鎮離石【悝苦回翻】濟南王驥為征西將軍築西平城以居之【西平城當築於平陽西濟子禮翻】魏王操為征東將軍鎮蒲子 趙固王桑自懷求迎於漢漢主聰遣鎮遠將軍梁伏疵將兵迎之未至長史臨深將軍牟穆帥衆一萬叛歸劉演固隨疵而西【疵疾移翻帥讀曰率下同】桑引其衆東奔青州固遣兵追殺之於曲梁【曲梁縣屬廣平郡曹魏置廣平郡始曲梁城劉昫曰唐洺州永年縣漢曲梁縣地也】桑將張鳳帥其餘衆歸演聰以固為荆州刺史領河南太守鎮洛陽 石勒自葛陂北行所過皆堅壁清野虜掠無所獲軍中饑甚士卒相食至東燕【據水經東燕城在酸棗縣東河水自酸棗東北過延津又逕東燕縣故城北余攷兩漢志東郡有燕縣無東燕縣其即是歟魏收地形志東燕縣晉屬濮陽國賢曰東燕故城今滑州胙城縣燕於賢翻】聞汲郡向冰聚衆數千壁枋頭【水經淇水至黎陽入河在遮害亭西十八里漢建安九年魏武於水口下大枋木以成堰遏淇水東入白溝以通漕運故時人號其處曰枋頭杜佑曰枋頭在今汲郡衛縣界宋白曰枋頭城在今衛縣南去河八里向式亮翻姓也枋音方】勒將濟河恐氷邀之張賓曰聞冰船盡在瀆中未上【未上者未上岸船不用則推之登陸使遠水而燥他日輕便於駕用上時掌翻】宜遣輕兵間道襲取以濟大軍【間古莧翻】大軍既濟氷必可擒也秋七月勒使支雄孔萇自文石津縛筏潛渡取其船勒引兵自棘津濟河【水經河水逕東燕縣故城北河水於是有棘津之名】擊冰大破之盡得其資儲軍埶復振遂長驅至鄴劉演保三臺以自固臨深牟穆等復帥其衆降於勒【復扶又翻帥讀曰率並下同降戶江翻下同】諸將欲攻三臺張賓曰演雖弱衆猶數千三臺險固攻之未易猝抜【易以豉翻】捨而去之彼將自潰方今王彭祖劉越石公之大敵也【王浚字彭祖劉琨字越石】宜先取之演不足顧也且天下饑亂明公雖擁大兵遊行羈旅人無定志非所以保萬全制四方也不若擇便地而據之廣聚糧儲西稟平陽以圖幽并【幽王浚并劉琨】此霸王之業也邯鄲襄國形勝之地【邯鄲縣漢屬趙國晉屬廣平襄國縣秦為信都項羽改曰襄國漢屬趙國晉屬廣平而信都别為縣前漢屬信都國後漢屬安平國邯鄲音寒丹宋白曰隋改襄國為龍岡縣唐為邢州所治也】請擇一而都之勒曰右侯之計是也遂進據襄國賓復言於勒曰今吾居此彭祖越石所深忌也恐城塹未固資儲未廣二寇交至【塹七艷翻】宜亟收野穀且遣使至平陽具陳鎮此之意【使疏吏翻下同】勒從之分命諸將攻冀州郡縣壁壘多降運其穀以輸襄國且表於漢主聰聰以勒為都督冀幽并營四州諸軍事【營州不在晉太康地志十九州之數晉地理志咸寧二年分昌黎遼東玄菟帶方樂浪等郡國五置平州至慕容熙據和龍始於宿軍置營州以刺史鎮之拓抜魏置營州於和龍勒時未有營州也郡國志營州地當營室分故曰營州】冀州牧進封上黨公劉琨移檄州郡期以十月會平陽擊漢琨素奢豪喜聲色【喜許記翻】河南徐潤以音律得幸於琨琨以為晉陽令潤驕恣干預政事護軍令狐盛數以為言【令狐之令力丁翻數所角翻】且勸琨殺之琨不從潤譖盛於琨琨收盛殺之琨母曰汝不能駕御豪傑以恢遠略而專除勝已禍必及我盛子泥奔漢具言虛實漢主聰大喜遣河内王粲中山王曜將兵寇并州以令狐泥為鄉導【鄉讀曰嚮】琨聞之東出收兵於常山及中山使其將郝詵張喬將兵拒粲【郝呼各翻】且遣使求救於代公猗盧詵喬俱敗死粲曜乘虛襲晉陽太原太守高喬并州别駕郝聿以晉陽降漢 【考異曰劉琨傳曰屬龎醇降于聰鴈門烏丸復反琨親出禦之粲乘虚襲取晉陽按琨上太子牋曰聰以七月十六日復决計送死臣即自東下率中山常山之卒並合樂平上黨諸軍未旋之間而晉陽傾潰十六國春秋亦云琨收兵常山本傳誤也】八月庚戌琨還救晉陽不及帥左右數十騎奔常山辛亥粲曜入晉陽壬子令狐泥殺琨父母粲曜送尚書盧志侍中許遐太子右衛率崔瑋于平陽聰復以曜為車騎大將軍以前將軍劉豐為并州刺史鎮晉陽九月聰以盧志為太弟太師崔瑋為太傅謝遐為太保高喬令狐泥皆為武衛將軍 己卯漢衛尉梁芬奔長安辛巳賈疋等奉秦王業為皇太子 【考異曰懷帝紀云賈疋討劉粲於】<br />
<br />
  【三輔走之關中小定奉秦王為太子按賈疋等以永嘉五年攻劉粲于新豐粲敗還平陽奉秦王入雍城六年三月劉曜弃長安走秦王入長安漢兵皆已退矣秦王為太子時劉粲方在晉陽懷紀誤】建行臺於長安祭壇告類【告類或攝或即位祭天之禮舜之攝也肆類于上帝孔安國注曰類謂攝位事類遂以攝告天及五帝湯黜夏命昭告于上天神后皆其事也】建宗廟社稷大赦以閻鼎為太子詹事總攝百揆【太子詹事統攝宫僚時太子建行臺故以詹事緫百揆時位號未正其實丞相之職也】加賈疋征西大將軍以秦州刺史南陽王保為大司馬命司空荀藩督攝遠近光禄大夫荀組領司隸校尉行豫州刺史與藩共保開封【開封縣漢屬河南郡晉屬滎陽郡】 秦州刺史裴苞據險以拒涼州兵張寔宋配等擊破之苞奔柔凶塢 冬十月漢主聰封其子恒為代王【恒戶登翻】逞為吳王朗為潁川王臯為零陵王旭為丹陽王京為蜀王坦為九江王晃為臨川王以王育為太保王彰為太尉任顗為司徒【顗魚豈翻】馬景為司空朱紀為尚書令范隆為左僕射呼延晏為右僕射 代公猗盧遣其子六修及兄子普根將軍衛雄范班箕澹帥衆數萬為前鋒以攻晉陽【澹徒覽翻又徒濫翻 考異曰十六國春秋云遣其子利孫宥六須載記云賓六須劉琨集云左右賢王又云右賢王撲速根本從後魏書 考異又曰箕澹十六國春秋後魏書作姬澹今從劉琨傳】猗盧自帥衆二十萬繼之劉琨收散卒數千為之鄉導【鄉讀曰嚮】六修與漢中山王曜戰於汾東曜兵敗墜馬中七創【中竹仲翻創初良翻下同】討虜將軍傅虎以馬授曜曜不受曰卿當乘以自免吾創已重自分死此【分扶問翻】虎泣曰虎蒙大王識抜至此常思效命今其時矣且漢室初基天下可無虎不可無大王也乃扶曜上馬驅令渡汾自還戰死曜入晉陽夜與大將軍粲鎮北大將軍豐掠晉陽之民踰蒙山而歸【五代志太原郡石艾縣有蒙山魏收曰石艾縣即漢晉之上艾縣也晉志上艾縣屬樂平郡又據五代志晉陽縣有蒙山此蓋蒙山跨晉陽石艾二縣界也】十一月猗盧追之戰於藍谷【藍谷在蒙山西南】漢兵大敗擒劉豐斬邢延等三千餘級【邢延叛琨見上卷五年】伏尸數百里猗盧因大獵壽陽山【壽陽山在樂平壽陽縣魏收地形志作壽陽縣此縣蓋晉置也宋白曰壽陽縣本漢榆次縣地晉置壽陽縣】陳閱皮肉山為之赤劉琨自營門步入拜謝固請進軍猗盧曰吾不早來致卿父母見害誠以相愧今卿已復州境吾遠來士馬疲弊且待後舉劉聰未可滅也遺琨馬牛羊各千餘疋車百乘而還【遺于季翻乘繩證翻】留其將箕澹段繁等戌晉陽琨徙居陽曲【陽曲縣屬太原郡在晉陽北】招集亡散盧諶為劉粲參軍亡歸琨【諶時壬翻】漢人殺其父志【考異曰劉聰載記志勸太弟义作亂被誅按志勸成都王穎起義兵諫穎攻長沙王义忠義敦篤始終不虧非勸义作亂者也今從盧諶傳】及弟謐詵贈傅虎幽州刺史 十二月漢主聰立皇后張氏以其父寔為左光禄大夫 彭仲蕩之子天護帥羣胡攻賈疋天護陽不勝而走疋追之夜墜澗中天護執而殺之【疋音雅疋殺彭仲蕩事見上卷五年考異曰帝紀曰疋討賊張連遇害疋傳天護攻之疋敗走墜澗死今從十六國春秋】漢以天護為涼州刺史衆推始平太守麴允領雍州刺史閻鼎與京兆太守梁綜爭權鼎遂殺綜麴允與撫夷護軍索綝馮翊太守梁肅合兵攻鼎鼎出犇雍為氐竇首所殺【胡羯方強賈閻麴索降心相從協力以輔晉室猶懼不能全况自相屠乎長安之敗徵見於此矣雍於用翻】 廣平游綸張豺擁衆數萬據苑鄉【姓譜游廣平望姓鄭公子偃字子游其後以為氏魏收志廣平郡任縣有范鄉城宋白曰任縣後漢南縣地後趙石氏於此置苑鄉縣唐為任縣屬邢州】受王浚假署【假署者承制權宜而補署假以職名】石勒遣夔安支雄等七將攻之破其外壘浚遣督護王昌帥諸軍及遼西公段疾陸眷【考異曰石勒載記及後魏書作就陸眷今從王浚傳】疾陸眷弟匹磾文鴦從弟末柸【磾丁奚翻從才用翻 考異曰後魏書作未破今從王浚傳】部衆五萬攻勒於襄國疾陸眷屯于渚陽【班固地理志禹貢絳水在信都入海水經註絳瀆北逕信都城東散入澤渚西至信都城東連于廣川縣張甲故瀆同歸于海疾陸眷蓋屯是渚之陽也】勒遣諸將出戰皆為疾陸眷所敗【敗補邁翻】陸眷大造攻具將攻城勒衆甚懼勒召將佐謀之曰今城塹未固糧儲不多彼衆我寡外無救援吾欲悉衆與之决戰何如諸將皆曰不如堅守以疲敵待其退而擊之張賓孔萇曰鮮卑之種【種章勇翻】段氏最為勇悍【悍下罕翻又侯肝翻】而末柸尤甚其銳卒皆在末柸所今聞疾陸眷刻日攻北城其大衆遠來戰鬬連日謂我孤弱不敢出戰意必懈惰【懈古隘翻】宜且勿出示之以怯鑿北城為突門二十餘道【墨子備突篇曰城百步一突門突門用車兩輪以木束之塗其上維置突門内度門廣狹之令人入門四尺中置窐突門旁為槖充竈狀又置艾寇即入下輪而塞之鼓槖薰之也杜佑曰突門鑿城内為闇門多少臨事令五六寸勿穿或於中夜於敵初來營列未定精騎從突門躍出擊其無備襲其不意】俟其來至列守未定出其不意直衝末柸帳彼必震駭不暇為計破之必矣末柸敗則其餘不攻而潰矣勒從之密為突門既而疾陸眷攻北城勒登城望之見其將士或釋仗而寢乃命孔萇督銳卒自突門出擊之【見其釋仗而寢知其懈也乃命萇出戰所謂見兵執者也】城上鼓譟以助其勢萇攻末柸帳不能克而退末柸逐之入其壘門為勒衆所獲疾陸眷等軍皆退走萇乘勝追擊枕尸三十餘里【枕職任翻】獲鎧馬五千匹【鎧可亥翻】疾陸眷收其餘衆還屯渚陽勒質末柸【質音致下同】遣使求和於疾陸眷【使疏吏翻】疾陸眷許之文鴦諫曰今以末柸一人之故而縱垂亡之虜得無為王彭祖所怨招後患乎疾陸眷不從復以鎧馬金銀賂勒【復扶又翻下同】且以末柸三弟為質而請末柸諸將皆勸勒殺末柸勒曰遼西鮮卑健國也與我素無仇讐為王浚所使耳今殺一人而結一國之怨非計也歸之必深德我不復為浚用矣乃厚以金帛報之遣石虎與疾陸眷盟于渚陽結為兄弟疾陸眷引歸王昌不能獨留亦引兵還薊勒召末柸與之燕飲誓為父子遣還遼西末柸在塗日南嚮而拜者三由是段氏專心附勒王浚之勢遂衰【孫武所謂親而離之此其近之矣然段氏專心附勒者末柸也若匹磾文鴦則終身與勒抗】游綸張豺請降於勒【降戶江翻】勒攻信都殺冀州刺史王象浚復以邵舉行冀州刺史保信都 是歲大疫 王澄少與兄衍名冠海内【少詩照翻冠古玩翻】劉琨謂澄曰卿形雖散朗而内實動俠【言其心輕易動又豪俠自喜也】以此處世【處昌呂翻】難得其死及在荆州悅成都内史王機謂為已亞使之内綜心膂【綜機縷也所以持經而施緯使不失其條理者也故謂能統理衆事者為綜理】外為爪牙澄屢為杜弢所敗【弢土刀翻敗補邁翻】望實俱損猶傲然自得無憂懼之意但與機日夜縱酒博奕由是上下離心南平太守應詹屢諫不聽澄自出軍擊杜弢軍于作塘【作塘縣後漢屬武陵郡晉屬南平郡五代志澧陽郡孱陵縣舊曰作塘】故山簡參軍王冲擁衆迎應詹為刺史詹以冲無賴棄之還南平【南平郡治江安】冲乃自稱刺史澄懼使其將杜蕤守江陵徙治孱陵【孱陵縣漢屬武陵郡晉屬南平郡應劭曰孱音踐師古音士連翻劉昫曰澧州安鄉縣漢孱陵地】尋又犇沓中【此沓中非姜維種麥之沓中蓋在孱陵之東】别駕郭舒諫曰使君臨州雖無異政然一州人心所繫今西收華容之兵足以擒此小醜【華容縣屬南郡】奈何自棄遽為奔亡乎澄不從欲將舒東下舒曰舒為萬里紀綱【舒為州别駕故自謂萬里紀綱】不能匡正令使君犇亡誠不忍渡江乃留屯沌口【水經註沌水南通沔陽縣之太白湖湖水東南通江謂之沌口沌持兖翻】琅邪王睿聞之召澄為軍諮祭酒以軍諮祭酒周顗代之澄乃赴召顗始至州【顗魚里翻】建平流民傅密等叛迎杜弢弢别將王真襲沔陽【沔陽梁武帝時方置郡據沈約志陶侃為荆州刺史初治沔陽則是時已有沔陽城矣當屬竟陵郡界宋白曰復州沔陽縣漢縣也郡國志曰沔陽縣即楚王城】顗狼狽失據征討都督王敦遣武昌太守陶侃尋陽太守周訪歷陽内史甘卓【懷帝永嘉五年睿加敦都督征討諸軍事惠帝永興元年分淮南之烏江歷陽二縣置歷陽郡】共擊弢郭進屯豫章為諸軍繼援 【考異曰王澄傳曰時王敦為江州鎮豫章按敦時為揚州刺史都督征討諸軍非為江州也】王澄過詣敦自以名聲素出敦右猶以舊意侮敦敦怒誣其與杜弢通信遣壯士搤殺之【搤乙革翻】王機聞澄死懼禍以其父毅兄矩皆嘗為廣州刺史就敦求廣州敦不許會廣州將温邵等叛刺史郭訥迎機為刺史機遂將奴客門生千餘人入廣州 【考異曰王澄死周顗敗王敦鎮豫章機入廣州紀傳皆無年月按衛玠傳玠依敦於豫章以永嘉六年卒故附於此】訥遣兵拒之將士皆機父兄時部曲【王機父毅為廣州刺史甚得南越之情】不戰迎降【降戶江翻下同】訥乃避位以州授之 王如軍中饑乏官軍討之其黨多降如計窮遂降於王敦【考考異曰如降亦無年月明年有如餘黨入漢中故附此】 鎮東軍司顧榮前太子洗馬衛玠皆卒【洗悉薦翻】玠瓘之孫也美風神善清談常以為人有不及可以情恕非意相干可以理遣故終身不見喜愠之色【見賢遍翻】 江陽太守張啟【江陽縣漢屬犍為郡劉蜀分置江陽郡隋併入陵州隆山縣唐為眉州彭山縣】殺益州刺史王異而代之【異行三府事見上卷五年】啟翼之孫也尋病卒三府文武共表涪陵太守向沈行西夷校尉南保涪陵【沈持林翻涪音浮】南安赤亭羌姚弋仲東徙榆眉【水經註漢靈帝分獂道為南安郡赤亭水】<br />
<br />
  【出郡之東山赤谷西流逕城北南入渭水謂之赤亭川榆眉即漢扶風之隃麋縣晉省宋白曰隴州汧源縣東有隃麋澤有古城吳山縣亦漢榆麋縣地】戎夏襁負隨之者數萬自稱護羌校尉雍州刺史扶風公【夏戶雅翻雍於用翻】<br />
<br />
  孝愍皇帝上【諱鄴字彦旗武帝孫吳孝王晏之子也出繼伯父秦王東後襲封秦王諡灋禍亂方作曰愍在國遭憂曰愍】<br />
<br />
  建興元年【是年夏四月方改元建興】春正月丁丑朔漢主聰宴羣臣於光極殿使懷帝著青衣行酒【著陟略翻】庾珉王雋等不勝悲憤因號哭聰惡之【勝音升號戶刀翻惡烏路翻】有告珉等謀以平陽應劉琨者二月丁未聰殺珉雋等故晉臣十餘人【永嘉三年珉雋與帝俱没于虜】懷帝亦遇害【年三十】大赦復以會稽劉夫人為貴人【永嘉六年聰以夫人妻帝聰封帝為會稽公故曰會稽劉夫人會工外翻】<br />
<br />
  荀崧曰懷帝天資清劭【劭高也】少著英猷【少詩照翻】若遇承平足為守文佳主而繼惠帝擾亂之後東海專政故無幽厲之釁而有流亡之禍矣<br />
<br />
  乙亥漢太后張氏卒諡曰光獻張后不勝哀丁丑亦卒諡曰武孝【張后張太后之姪女勝音升】 己卯漢定襄忠穆公王彰卒 三月漢主聰立貴嬪劉娥為皇后為之起䳨儀殿【雄曰鳳雌曰䳨書曰鳳凰來儀為于偽翻下更為下為同】廷尉陳元達切諫以為天生民而樹之君使司牧之非以兆民之命窮一人之欲也晉氏失德大漢受之蒼生引領庶幾息肩是以光文皇帝【劉淵諡光文幾居希翻】身衣大布居無重茵【衣於既翻下同重直龍翻】后妃不衣錦綺乘輿馬不食粟愛民故也【乘繩證翻】陛下踐阼以來已作殿觀四十餘所加之軍旅數興【觀古玩翻數所角翻】餽運不息饑饉疾疫死亡相繼而益思營繕豈為民父母之意乎今有晉遺類西據關中南擅江表李雄奄有巴蜀王浚劉琨窺窬肘腋石勒曹嶷貢禀漸疎【嶷魚力翻貢謂貢獻稟謂稟承詔命】陛下釋此不憂乃更為中宫作殿豈目前之所急乎【為于偽翻】昔太宗居治安之世粟帛流衍猶愛百金之費免露臺之役【事見十五卷漢文帝後七年治直吏翻】陛下承荒亂之餘所有之地不過太宗之二郡【時聰所有之地漢河東西河二郡耳】戰守之備非特匈奴南越而已【漢文帝時惟備匈奴南越】而宫室之侈乃至於此臣所以不敢不冒死而言也聰大怒曰朕為天子營一殿何關汝鼠子乎乃敢妄言沮衆【沮在呂翻】不殺此鼠子朕殿不成命左右曳出斬之并其妻子同梟首東市【梟堅堯翻】使羣鼠共宂時聰在逍遥園李中堂元達先鎖腰而入即以鎖鎖堂下樹呼曰臣所言者社稷之計而陛下殺臣朱雲有言臣得與龍逢比干遊足矣【朱雲事見三十二卷漢成帝元延元年呼火故翻逢蒲江翻】左右曳之不能動大司徒任顗【任音壬顗魚豈翻】光禄大夫朱紀范隆驃騎大將軍河間王易等叩頭出血曰元達為先帝所知受命之初即引置門下【見八十五卷惠帝永興元年】盡忠竭慮知無不言臣等竊禄偷安每見之未嘗不發愧今所言雖狂直願陛下容之因諫諍而斬列卿其如後世何聰默然劉后聞之密敕左右停刑手疏上言今宫室已備無煩更營四海未壹宜愛民力廷尉之言社稷之福也陛下宜加封賞而更誅之四海謂陛下何如哉夫忠臣進諫者固不顧其身也而人主拒諫者亦不顧其身也陛下為妾營殿而殺諫臣使忠良結舌者由妾遠近怨怒者由妾公私困弊者由妾社稷阽危者由妾【為于偽翻阽服䖍音反玷之坫孟康音屋檐之檐如淳曰阽近邊知墮意】天下之罪皆萃於妾妾何以當之妾觀自古敗國喪家【敗補邁翻喪息浪翻】未始不由婦人心常疾之不意今日身自為之使後世視妾由妾之視昔人也【由與猶通洪氏隸釋曰古字多以由通為猶字樊毅修華嶽碑由復夕愓余謂樊碑之由其義尚也此由如也】妾誠無面目復奉巾櫛【櫛側瑟翻梳枇總名復扶又翻下同】願賜死此堂以塞陛下之過【塞悉則翻】聰覽之變色任顗等叩頭流涕不已聰徐曰朕比年已來微得風疾【比毗至翻】喜怒過差不復自制元達忠臣也朕未之察諸公乃能破首明之誠得輔弼之義也朕愧戢于心【戢側立翻藏也】何敢忘之命顗等冠履就坐【坐徂卧翻】引元達上【上時掌翻升堂也】以劉氏表示之曰外輔如公内輔如后朕復何憂賜顗等穀帛各有差更命逍遥園曰納賢園李中堂曰愧賢堂【更工衡翻】聰謂元達曰卿當畏朕而反使朕畏卿耶 西夷校尉向沈卒衆推汶山太守蘭維為西夷校尉【向式亮翻姓也沈持林翻汶音岷姓譜鄭穆公名蘭支庶以為氏漢有武陵太守蘭廣又匈奴傳亦有蘭氏非此蘭也】維率吏民北出欲向巴東【欲歸晉也】成將李恭費黑邀擊獲之【將即亮翻費扶沸翻】 夏四月丙午懷帝凶問至長安皇太子舉哀因加元服【鄭樵通志略曰魏氏天子冠一加其說曰古之士禮冠必三加彌尊所以喻其志至於天子諸侯無加數之文者將以踐阼臨人尊極德成不復與士以加喻勉為義禮冠於廟自魏不復在廟矣冠太子再加是時蓋仍魏禮】壬申即皇帝位大赦改元【始改元建興】以衛將軍梁芬為司徒雍州刺史麴允為尚書左僕射録尚書事【雍於用翻】京兆太守索綝為尚書右僕射領吏部京兆尹【索昔各翻綝丑林翻】是時長安城中戶不盈百蒿棘成林公私有車四乘【乘繩證翻】百官無章服印綬唯桑版署號而已尋以索綝為衛將軍領太尉軍國之事悉以委之 漢中山王曜司隸校尉喬智明寇長安平西將軍趙染帥衆赴之【帥讀曰率下同】詔麴允屯黄白城以拒之石勒使石虎攻鄴鄴潰劉演犇廩丘【廩丘縣前漢屬東郡後漢屬濟】<br />
<br />
  【隂郡晉屬濮陽國賢曰廩丘故城在今濮州雷澤縣北】三臺流民皆降於勒【降戶江翻】勒以桃豹為魏郡太守以撫之久之以石虎代豹鎮鄴初劉琨用陳留太守焦求為兖州刺史荀藩又用李述為兖州刺史述欲攻求琨召求還及鄴城失守琨復以劉演為兖州刺史鎮廩丘前中書侍郎郗鑒少以清節著名帥高平千餘家避亂保嶧山【水經註嶧山在鄒縣北繹邑之所依以為名也山東西二十里高秀獨出積石相臨殆無土壤石間多孔穴洞達相通往往有如數間屋處其俗謂之嶧孔遭亂輒將家入嶧外寇雖衆無所施害晉永嘉中郗鑒保此山今山南有大嶧名曰鄒公嶧詩所謂保有鳬繹復扶又翻少詩照翻帥讀曰率嶧音亦】琅邪王睿就用鑒為兖州刺史鎮鄒山【鄒山在魯郡鄒縣 考異曰劉琨集建興二年十一月壬寅朔與丞相牋曰焦求雖出寒鄉有文武膽幹苟晞用為陳留太守獨在河南距當石勒撫綏有方琨以求行領兖州刺史後聞荀公以李述為兖州以素論門望不可與求同日而論至於膽幹可以處危權一時之用李述亦不能及求而王玄年少便欲共討求琨以求已與玄構隙便召還而州界民物甚不安服連二千石及文武大姓連遣信使求刺史是以遣兄子演代求領兖州事往年春正月遣詣鄴至是斬王桑走趙固云云今勒據襄國逼近鄴城故令演轉南演今治在廩丘而李述郗鑒並欲爭兖州或云為荀公所用或云為明公所用大寇未殄而自共尋干戈此以大潰也輒敕演謹自守而已按王桑趙固之敗及石勒攻鄴皆在永嘉六年琨牋又云傳長安消息主上是秦王又建興二年十一月丙申朔元年十一月壬申朔十二月壬寅朔然則琨發牋之日建興元年十二月壬寅朔也傳寫誤耳】三人各屯一郡兖州吏民莫知所從 琅邪王睿以前廬江内史華譚為軍諮祭酒【華戶化翻】譚嘗在壽春依周馥睿謂譚曰周祖宣何故反【周馥字祖宣】譚曰周馥雖死天下尚有直言之士馥見寇賊滋蔓欲移都以紓國難【難乃旦翻】執政不悅興兵討之馥死未踰時而洛都淪没若謂之反不亦誣乎【事見上卷永嘉四年五年】睿曰馥位為征鎮握彊兵召之不入危而不持亦天下之罪人也譚曰然危而不持當與天下共受其責非但馥也睿參佐多避事自逸録事參軍陳頵【録事參軍掌總録衆曹管其文案自上佐以下違失者彈正以法掌凡諸司察之事白氏六帖曰州主簿郡督郵並今録事參軍之職余據睿以頵為録事參軍自别有主簿詳見辨誤】言於睿曰洛中承平之時朝士以小心恭恪為凡俗以偃蹇倨肆為優雅流風相染以至敗國【朝直遥翻敗補邁翻】今僚屬皆承西臺餘弊【江東謂洛都為西臺】養望自高是前車已覆而後車又將尋之也請自今臨使稱疾者皆免官睿不從三王之誅趙王倫也【見八十四卷惠帝永寧元年】制己亥格以賞功自是循而用之頵上言昔趙王簒逆惠皇失位三王起兵討之故厚賞以懷嚮義之心今功無大小皆以格斷【斷丁亂翻决也言功之輕重差次皆以己亥格决之】乃至金紫佩士卒之身符策委僕隸之門非所以重名器正紀綱也請一切停之頵出於寒微數為正論府中多惡之【數所角翻惡烏路翻】出頵為譙郡太守 吳興太守周玘宗族彊盛【玘墟里翻】琅邪王睿頗疑憚之睿左右用事者多中州亡官失守之士駕御吳人吳人頗怨玘自以失職又為刁協所輕恥恚愈甚【恚於避翻】乃隂與其黨謀誅執政以諸南士代之事泄玘憂憤而卒將死謂其子勰曰殺我者諸傖子也【勰音協傖助庚翻吳人謂中州人為傖】能復之乃吾子也 石勒攻李惲於上白斬之【惲於粉翻】王浚復以薄盛為青州刺史【上白城在安平廣宗縣李惲薄盛皆乞活帥復扶又翻】 王浚使棗嵩督諸軍屯易水召段疾陸眷欲與之共擊石勒疾陸眷不至【以釋其弟未柸德石勒故不肯會浚兵】浚怒以重弊賂拓抜猗盧并檄慕容廆等共討疾陸眷【廆戶罪翻】猗盧遣右賢王六修將兵會之為疾陸眷所敗【敗補邁翻】廆遣慕容翰攻段氏取徒河新城至陽樂【陽樂縣屬遼西郡賢曰陽樂在今平州東】聞六修敗而還翰因留鎮徒河壁青山初中國士民避亂者多北依王浚浚不能存撫又政法不立士民往往復去之【復扶又翻】段氏兄弟專尚武勇不禮士大夫唯慕容廆政事修明愛重人物故士民多歸之廆舉其英俊隨才授任以河東裴嶷【嶷魚力翻】北平陽躭廬江黄泓代郡魯昌為謀主廣平游邃北海逢羨【逢皮江翻】北平西方䖍【何氏姓苑少昊金天氏位主西方子孫以為氏】西河宋奭及封抽裴開為股肱平原宋該安定皇甫岌岌弟真蘭陵繆愷【繆靡幼翻】昌黎劉斌及封奕封裕典機要裕抽之子也裴嶷清方有幹略為昌黎太守兄武為玄菟太守武卒嶷與武子開以其喪歸過廆【自玄菟西歸道過棘城菟同都翻】廆敬禮之及去厚加資送行及遼西道不通嶷欲還就庾開曰鄉里在南奈何北行且等為流寓段氏彊慕容氏弱何必去此而就彼也嶷曰中國喪亂【喪息浪翻】今往就之是相帥而入虎口也【帥讀曰率】且道遠何由可達【言昌黎去河東既遠又路梗無由得達】若俟其清通又非歲月可冀【言天下方亂道路未有清通之時】今欲求託足之地豈可不慎擇其人汝觀諸段豈有遠略且能待國士乎慕容公修行仁義有霸王之志加以國豐民安今往從之高可以立功名下可以庇宗族汝何疑焉開乃從之既至廆大喜陽躭清直沈敏為遼西太守【沈持林翻】慕容翰破段氏於陽樂獲之廆禮而用之游邃逢羨宋奭皆嘗為昌黎太守【逢皮江翻】與黄泓俱避地於薊後歸廆王浚屢以手書召邃兄暢暢欲赴之邃曰彭祖刑政不修華戎離叛以邃度之必不能久兄且盤桓以俟之【易屯卦初九爻辭曰盤桓利居貞王弼曰不可以進故盤桓也馬曰盤桓旋也度徒洛翻】暢曰彭祖忍而多疑頃者流民北來命所在追殺之今手書殷勤我稽留不往將累及卿【累力瑞翻】且亂世宗族宜分以冀遺種遂從之卒與浚俱没【種章勇翻卒子恤翻】宋該與平原杜羣劉翔先依王浚又依段氏皆以為不足託帥諸流寓同歸於廆東夷校尉崔毖請皇甫岌為長史卑辭說諭終莫能致廆招之岌與弟真即時俱至【古語有之鳥則擇木木豈能擇鳥帥讀曰率說輸芮翻】遼東張統據樂浪帶方二郡與高句麗王乙弗利相攻連年不解樂浪王遵說統帥其民千餘家歸廆廆為之置樂浪郡【為于偽翻樂浪音洛琅句如字又音句驪力知翻】以統為太守遵參軍事王如餘黨涪陵李運巴西王建等自襄陽將三千餘家入漢中【涪音浮】梁州刺史張光遣參軍晉邈將兵拒之邈受運建賂勸光納其降【降戶江翻下同】光從之使居成固既而邈見運建及其徒多珍寶欲盡取之復說光曰【復扶又翻說輸芮翻】運建之徒不修農事專治器仗其意難測不如悉掩殺之不然必為亂光又從之【將貪於下帥闇於上梁州之禍復自此始治直之翻】五月邈將兵攻運建殺之建婿楊虎收餘衆擊光屯于厄水光遣其子孟萇討之不能克 壬辰以琅邪王睿為左丞相大都督督陜東諸軍事南陽王保為右丞相大都督督陜西諸軍事【所謂分陜也陜失冉翻】詔曰今當掃除鯨鯢【鯨鯢大魚鈎網所不能制以比敵人之魁桀者鯨巨京翻鯢五兮翻】奉迎梓宫【謂懷帝遇害於平陽梓宫未返也】令幽并兩州勒卒三十萬直造平陽【造七到翻下同】右丞相宜帥秦涼梁雍之師三十萬徑詣長安左丞相帥所領精兵二十萬徑造洛陽【帥讀曰率雍於用翻】同赴大期克成元勲 漢中山王曜屯蒲坂 石勒使孔萇擊定陵殺田徽【定陵縣漢屬潁川郡晉屬襄城郡田徽王浚用為兖州刺史】薄盛率所部降勒山東郡縣相繼為勒所取漢主聰以勒為侍中征東大將軍烏桓亦叛王浚潛附於勒【史言王浚之勢浸以孤弱】 六月劉琨與代公猗盧會于陘北謀擊漢【陘音刑】秋七月琨進據藍谷猗盧遣拓跋普根屯于北屈【北屈縣漢屬河東郡晉屬平陽郡春秋晉公子夷吾所居邑也宋白曰慈州夾城縣本漢北屈縣地師古曰屈居勿翻】琨遣監軍韓據自西河而南將攻西平【西平城在平陽西漢主聰築以居其子濟南王驥】漢主聰遣大將軍粲等拒琨驃騎將軍易等拒普根蕩晉將軍蘭陽等助守西平琨等聞之引兵還【還從宣翻又如字】聰使諸軍仍屯所在為進取之計 帝遣殿中都尉劉蜀詔左丞相睿以時進軍【殿中都尉屬二衛】與乘輿會除中原【乘繩證翻】八月癸亥蜀至建康睿辭以方平定江東未暇北伐以鎮東長史刁協為丞相左長史從事中郎彭城劉隗為司直【隗五罪翻】邵陵内史廣陵戴邈為軍諮祭酒【吳孫皓寶鼎元年分零陵北部都尉立邵陵郡宋白曰邵陵漢之昭陵縣吳立邵陵郡唐為邵州】參軍丹陽張闓為從事中郎【闓音開又可亥翻】尚書郎潁川鍾雅為記室參軍譙國桓宣為舍人豫章熊遠為主簿會稽孔愉為掾【會工外翻掾以絹翻】劉隗雅習文史善伺候睿意故睿特親愛之【為刁劉間王敦兄弟張本伺相吏翻】熊遠上書以為軍興以來處事不用律令【處昌呂翻】競作新意臨事立制朝作夕改至於主者不敢任灋每輒關諮【關白也】非為政之體也愚謂凡為駁議者皆當引律令經傳【駁北角翻傳直戀翻】不得直以情言無所依凖以虧舊典若開塞隨宜【塞悉則翻】權道制物此是人君之所得行非臣子所宜專用也睿以時方多事不能從初范陽祖逖少有大志【漢涿郡魏文帝更名曰范陽郡少詩照翻】與劉琨俱為司州主簿同寢中夜聞鷄鳴蹵琨覺曰【蹵子六翻蹋也覺居效翻寤也】此非惡聲也因起舞及渡江左丞相睿以為軍諮祭酒逖居京口【吳孫權自吳徙丹徒謂之京城有京峴山在其東其城因山為壘俯臨江津故曰京口】糾合驍健【繩三合為糾糾言合三為一也驍堅堯翻下同】言於睿曰晉室之亂非上無道而下怨叛也由宗室爭權自相魚肉遂使戎狄乘隙毒流中土今遺民既遭殘賊人思自奮大王誠能命將出師使如逖者統之以復中原郡國豪傑必有望風響應者矣睿素無北伐之志以逖為奮威將軍豫州刺史給千人廩布三千疋【給千人糧廩及布三千疋而已】不給鎧仗使自召募逖將其部曲百餘家渡江中流擊楫而誓曰祖逖不能清中原而復濟者有如大江【復扶又翻】遂屯淮隂【淮隂縣前漢屬臨淮郡後漢屬下邳國晉屬廣陵郡】起冶鑄兵募得二千餘人而後進 胡亢性猜忌殺其驍將數人杜曾懼潛引王冲之兵使攻亢【王亢荆州賊也】亢悉精兵出拒之城中空虛曾因殺亢而并其衆 周顗屯潯水城【廬山記曰尋陽縣在大江之北尋水之陽潯水城無乃古之尋陽城乎】為杜弢所困陶侃使明威將軍朱伺救之弢退保泠口【沈約志魏置將軍四十號明威第三弢吐刀翻水經註泠水南出九疑山北流逕冷道縣西南又北流注于都溪水又西北入于營水所謂泠口也楊正衡曰泠郎丁翻余攷此泠口去武昌甚遠又水經註江水自蘄春故城南又東得銅零口此無乃是乎】侃曰弢必步向武昌乃自徑道還郡以待之【徑道捷出之路】弢果來攻侃使朱伺逆擊大破之弢遁歸長沙周顗出潯水投王敦於豫章敦留之陶侃使參軍王貢告捷於敦敦曰若無陶侯便失荆州矣乃表侃為荆州刺史屯沔江【水經註沌水上承沔陽之白湖東南流逕沔陽縣南注于江謂之沌口陸游曰江陵之建寧鎮蓋沌口也王敦以陶侃為荆州鎮此明年徙林鄣侃傳云初鎮沌口移入沔江水經註林障故城在沔南沔江謂林障也】左丞相睿召周顗復以為軍諮祭酒 初氐王楊茂搜之子難敵遣養子販易於梁州私賣良人子一人張光鞭殺之難敵怨曰使君初來大荒之後兵民之命仰我氐活氐有小罪不能貰也【貰始制翻貸也恕也又神夜翻】及光與楊虎相攻各求救於茂搜茂搜遣難敵救光難敵求貨於光光不與楊虎厚賂難敵且曰流民珍貨悉在光所【謂晉邈所殺奪者】今伐我不如伐光難敵大喜光與虎戰使張孟萇居前難敵繼後難敵與虎夾擊孟萇大破之孟萇及其弟援皆死光嬰城自守九月光憤激成疾僚屬勸光退據魏興光按劒曰吾受國重任不能討賊今得死如登僊何謂退也聲絶而卒【張光雖以信用晉邈致寇其氣烈亦可尚也】州人推其少子邁領州事【少詩照翻】又與氐戰没衆推始平太守胡子序領梁州 荀藩薨于開封【荀藩傳祇相繼而没陜東二行臺惟荀組在耳考異曰帝紀曰薨于滎陽今從藩傳】漢中山王曜趙染攻麴允于黄白城允累戰皆敗詔以索綝為征東大將軍將兵助允 王貢自王敦所還至竟陵矯陶侃之命以杜曾為前鋒大都督擊王冲斬之悉降其衆【降戶江翻】侃召曾曾不至貢恐以矯命獲罪遂與曾反擊侃冬十月侃兵大敗僅以身免敦表侃以白衣領職侃復帥周訪等進擊杜弢大破之【復扶又翻】敦乃奏復侃官 漢趙染謂中山王曜曰麴允率大衆在外長安空虛可襲也曜使染帥精騎五千襲長安【帥讀曰率下同】庚寅夜入外城帝犇射鴈樓染焚龍尾及諸營【龍尾者依城築道陂陁而漸高登陴所由之路也又水經曰秦時有黑龍從南山出飲渭水其行道因山成跡長六十餘里頭臨渭水尾達樊川漢蕭何起未央宫斬龍首山而營之頭高二十丈尾漸下高五六丈所謂龍尾者此山之尾也】殺掠千餘人辛卯旦退屯逍遥園【水經註沈水上承皇子陂於樊川北逕長安城西與昆明池水合沈水又東北流逕鄧艾祠南又東分為二水一水東入逍遥園】壬辰將軍麴鑒自阿城帥衆五千救長安【阿城即秦阿房宫城也】癸巳染引還鑒追之與曜遇於零武鑒兵大敗【前漢北地郡有靈武縣後漢晉省至後魏置咸陽郡池陽靈武二縣並屬焉黄白城在池陽則此零武為前漢北地郡魏咸陽郡之靈武明矣】 楊虎楊難敵急攻梁州胡子序棄城走難敵自稱刺史漢中山王曜恃勝而不設備十一月麴允引兵襲之漢兵大敗殺其冠軍將軍喬智明【冠古玩翻】曜引歸平陽 王浚以其父字處道自謂應當塗高之䜟謀稱尊號【王浚又一袁術也】前勃海太守劉亮北海太守王摶【摶徒官翻】司空掾高柔切諫【掾于絹翻此又一高柔非魏之高柔】浚皆殺之燕國霍原志節清高屢辭徵辟浚以尊號事問之原不答浚誣原與羣盜通殺而梟其首【梟堅堯翻】於是士民駭怨而浚矜豪日甚不親政事所任皆苛刻小人棗嵩朱碩貪横尤甚【横戶孟翻】北州謡曰府中赫赫朱丘伯十囊五囊入棗郎【朱丘伯朱碩字也嵩浚之婿故曰棗郎】調發殷煩下不堪命多叛入鮮卑【調徒弔翻】從事韓咸監護柳城【柳城縣前漢屬遼西郡後漢晉省監工銜翻】盛稱慕容廆能接納士民欲以諷浚浚怒殺之浚始者唯恃鮮卑烏桓以為彊既而皆叛之加以蝗旱連年兵勢益弱石勒欲襲之未知虚實將遣使覘之【使疏吏翻覘癡亷翻又曰艶翻】參佐請用羊祜陸抗故事致書於浚【欲用敵國交鄰之禮】勒以問張賓賓曰浚名為晉臣實欲廢晉自立但患四海英雄莫之從耳其欲得將軍猶項羽之欲得韓信也將軍威振天下今卑辭厚禮折節事之猶懼不信况為羊陸之亢敵乎【折而設翻亢口浪翻】夫謀人而使人覺其情難以得志矣勒曰善十二月勒遣舍人王子春董肇多齎珍寶奉表於浚曰勒本小胡遭世饑亂流離屯厄【屯株倫翻難也】竄命冀州竊相保聚以救性命今晉祚淪夷中原無主殿下州鄉貴望【勒上黨武鄉人而浚太原人故云州鄉】四海所宗為帝王者非公復誰【復扶又翻】勒所以捐軀起兵誅討暴亂者正為殿下驅除爾伏願殿下應天順人【為于偽翻】早登皇祚勒奉戴殿下如天地父母殿下察勒微心亦當視之如子也又遺棗嵩書厚賂之【遺于季翻】浚以段疾陸眷新叛士民多棄已去聞勒欲附之甚喜謂子春曰石公一時豪傑據有趙魏乃欲稱藩於孤其可信乎子春曰石將軍才力彊盛誠如聖旨但以殿下中州貴望威行夷夏【夏戶雅翻】自古胡人為輔佐名臣則有矣未有為帝王者也石將軍非惡帝王不為而讓於殿下【惡烏路翻】顧以帝王自有歷數非智力之所取雖彊取之【彊其兩翻】必不為天人之所與故也項羽雖彊終為漢有石將軍之比殿下猶隂精之與太陽是以遠鑒前事歸身殿下此乃石將軍之明識所以遠過於人也殿下又何怪乎浚大悅封子春肇皆為列侯遣使報聘以厚幣酬之游綸兄統為浚司馬鎮范陽遣使私附於勒【游綸保據苑鄉偽降於勒勒已襲禽之】勒斬其使以送浚浚雖不罪統益信勒為忠誠無復疑矣 是歲左丞相睿遣世子紹鎮廣陵以丞相掾蔡謨為參軍謨克之子也 漢中山王曜圍河南尹魏浚於石梁【石梁塢在洛水北】兖州刺史劉演河内太守郭默遣兵救之曜分兵逆戰於河北敗之【河平大河之北即富平津之北也敗補邁翻】浚夜走獲而殺之代公猗盧城盛樂以為北都治故平城為南都又作新平城於灅水之陽使右賢王六修鎮之統領南部【盛樂縣前漢屬定襄郡後漢屬雲中郡平城漢屬鴈門郡括地志曰朔州定襄縣本漢平城縣拓抜魏之盛也置朔州於盛樂置桓州於平城平城謂之代都自高祖遷洛其後被六韓抜陵作亂故都為墟恒州寄治肆州秀容郡城雲州寄治并州界魏收地形志自陘嶺以北所記略矣隋之盛也北逐突厥復漢故塞省併後魏所置郡縣盛樂蓋在定襄郡大利縣界平城在馬邑郡雲内縣界唐破突厥北盡魏隋之路朔州善陽縣則漢定襄魏桑乾之地單于都護府金河縣則後魏道武所都也雲州雲中縣則後魏所都平城也然自單于都護府東北至朔州三百五十七里則盛樂距平城其道里可知矣杜佑曰雲中今馬邑郡北平城即今郡隋雲内縣恒安鎮魏書帝紀猗盧修故平城以為南都更南百里於灅水之陽黄瓜堆築新平城晉人謂之小平城杜佑又曰朔州馬邑郡魏都平城於郡北置懷朔鎮及遷洛後置朔州後魏初雲中在今郡北三百餘里定襄故城北北齊置朔州在故都西南新城一名平城也後移於馬邑即今郡城也郡治善陽縣亦漢定襄縣地有秦馬邑城武周塞紫河發源於此宋白曰唐振武軍舊單于都護府即漢定襄郡之盛樂縣也在隂山之陽黄河之北後魏所都盛樂是也唐平突厥於此置雲中都督府後改單于府班固地理志右北平浚靡縣灅水南至無終東入庚師古曰灅方水翻又音郎賄翻酈道元水經註庚水與鮑丘水合浚靡在東與平城相去甚遠新平城不在此灅水之陽也據魏書道武帝西如馬邑觀灅源則灅水蓋出於馬邑而東北流逕平城之南也酈道元魏人也其註水經叙代都之事宜詳初不言平城有灅水但言濕水逕平城南耳註曰濕水出鴈門隂館縣濕頭山一曰治水東北流出山逕隂館縣故城西故樓煩鄉也又有馬邑川水會桑乾水而注于濕水濕水東流過平城南又東流逕廣甯下洛縣南東至漁陽入笥溝又考班固地理志鴈門隂館縣樓煩鄉累頭山治水所出東至泉州入海師古曰累音力追翻治音弋之翻竊謂水出累頭山疑當時亦有累水之名師古音從平聲音相近也意道元所謂濕水即灅水也又考丁度集韻漯灅㶟三字同註云水出鴈門則亦有見於此矣灅類篇音魯水翻】<br />
<br />
  資治通鑑卷八十八  <br>
   </div> 

<script src="/search/ajaxskft.js"> </script>
 <div class="clear"></div>
<br>
<br>
 <!-- a.d-->

 <!--
<div class="info_share">
</div> 
-->
 <!--info_share--></div>   <!-- end info_content-->
  </div> <!-- end l-->

<div class="r">   <!--r-->



<div class="sidebar"  style="margin-bottom:2px;">

 
<div class="sidebar_title">工具类大全</div>
<div class="sidebar_info">
<strong><a href="http://www.guoxuedashi.com/lsditu/" target="_blank">历史地图</a></strong>  
<a href="http://www.880114.com/" target="_blank">英语宝典</a>  
<a href="http://www.guoxuedashi.com/13jing/" target="_blank">十三经检索</a> 
<br><strong><a href="http://www.guoxuedashi.com/gjtsjc/" target="_blank">古今图书集成</a></strong> 
<a href="http://www.guoxuedashi.com/duilian/" target="_blank">对联大全</a> <strong><a href="http://www.guoxuedashi.com/xiangxingzi/" target="_blank">象形文字典</a></strong> 

<br><a href="http://www.guoxuedashi.com/zixing/yanbian/">字形演变</a>  <strong><a href="http://www.guoxuemi.com/hafo/" target="_blank">哈佛燕京中文善本特藏</a></strong>
<br><strong><a href="http://www.guoxuedashi.com/csfz/" target="_blank">丛书&方志检索器</a></strong> <a href="http://www.guoxuedashi.com/yqjyy/" target="_blank">一切经音义</a>  

<br><strong><a href="http://www.guoxuedashi.com/jiapu/" target="_blank">家谱族谱查询</a></strong>  <strong><a href="http://shufa.guoxuedashi.com/sfzitie/" target="_blank">书法字帖欣赏</a></strong> 
<br>

</div>
</div>


<div class="sidebar" style="margin-bottom:0px;">

<font style="font-size:22px;line-height:32px">QQ交流群9:489193090</font>


<div class="sidebar_title">手机APP 扫描或点击</div>
<div class="sidebar_info">
<table>
<tr>
	<td width=160><a href="http://m.guoxuedashi.com/app/" target="_blank"><img src="/img/gxds-sj.png" width="140"  border="0" alt="国学大师手机版"></a></td>
	<td>
<a href="http://www.guoxuedashi.com/download/" target="_blank">app软件下载专区</a><br>
<a href="http://www.guoxuedashi.com/download/gxds.php" target="_blank">《国学大师》下载</a><br>
<a href="http://www.guoxuedashi.com/download/kxzd.php" target="_blank">《汉字宝典》下载</a><br>
<a href="http://www.guoxuedashi.com/download/scqbd.php" target="_blank">《诗词曲宝典》下载</a><br>
<a href="http://www.guoxuedashi.com/SiKuQuanShu/skqs.php" target="_blank">《四库全书》下载</a><br>
</td>
</tr>
</table>

</div>
</div>


<div class="sidebar2">
<center>


</center>
</div>

<div class="sidebar"  style="margin-bottom:2px;">
<div class="sidebar_title">网站使用教程</div>
<div class="sidebar_info">
<a href="http://www.guoxuedashi.com/help/gjsearch.php" target="_blank">如何在国学大师网下载古籍?</a><br>
<a href="http://www.guoxuedashi.com/zidian/bujian/bjjc.php" target="_blank">如何使用部件查字法快速查字?</a><br>
<a href="http://www.guoxuedashi.com/search/sjc.php" target="_blank">如何在指定的书籍中全文检索?</a><br>
<a href="http://www.guoxuedashi.com/search/skjc.php" target="_blank">如何找到一句话在《四库全书》哪一页?</a><br>
</div>
</div>


<div class="sidebar">
<div class="sidebar_title">热门书籍</div>
<div class="sidebar_info">
<a href="/so.php?sokey=%E8%B5%84%E6%B2%BB%E9%80%9A%E9%89%B4&kt=1">资治通鉴</a> <a href="/24shi/"><strong>二十四史</strong></a>&nbsp; <a href="/a2694/">野史</a>&nbsp; <a href="/SiKuQuanShu/"><strong>四库全书</strong></a>&nbsp;<a href="http://www.guoxuedashi.com/SiKuQuanShu/fanti/">繁体</a>
<br><a href="/so.php?sokey=%E7%BA%A2%E6%A5%BC%E6%A2%A6&kt=1">红楼梦</a> <a href="/a/1858x/">三国演义</a> <a href="/a/1038k/">水浒传</a> <a href="/a/1046t/">西游记</a> <a href="/a/1914o/">封神演义</a>
<br>
<a href="http://www.guoxuedashi.com/so.php?sokeygx=%E4%B8%87%E6%9C%89%E6%96%87%E5%BA%93&submit=&kt=1">万有文库</a> <a href="/a/780t/">古文观止</a> <a href="/a/1024l/">文心雕龙</a> <a href="/a/1704n/">全唐诗</a> <a href="/a/1705h/">全宋词</a>
<br><a href="http://www.guoxuedashi.com/so.php?sokeygx=%E7%99%BE%E8%A1%B2%E6%9C%AC%E4%BA%8C%E5%8D%81%E5%9B%9B%E5%8F%B2&submit=&kt=1"><strong>百衲本二十四史</strong></a>  <a href="http://www.guoxuedashi.com/so.php?sokeygx=%E5%8F%A4%E4%BB%8A%E5%9B%BE%E4%B9%A6%E9%9B%86%E6%88%90&submit=&kt=1"><strong>古今图书集成</strong></a>
<br>

<a href="http://www.guoxuedashi.com/so.php?sokeygx=%E4%B8%9B%E4%B9%A6%E9%9B%86%E6%88%90&submit=&kt=1">丛书集成</a> 
<a href="http://www.guoxuedashi.com/so.php?sokeygx=%E5%9B%9B%E9%83%A8%E4%B8%9B%E5%88%8A&submit=&kt=1"><strong>四部丛刊</strong></a>  
<a href="http://www.guoxuedashi.com/so.php?sokeygx=%E8%AF%B4%E6%96%87%E8%A7%A3%E5%AD%97&submit=&kt=1">說文解字</a> <a href="http://www.guoxuedashi.com/so.php?sokeygx=%E5%85%A8%E4%B8%8A%E5%8F%A4&submit=&kt=1">三国六朝文</a>
<br><a href="http://www.guoxuedashi.com/so.php?sokeytm=%E6%97%A5%E6%9C%AC%E5%86%85%E9%98%81%E6%96%87%E5%BA%93&submit=&kt=1"><strong>日本内阁文库</strong></a> <a href="http://www.guoxuedashi.com/so.php?sokeytm=%E5%9B%BD%E5%9B%BE%E6%96%B9%E5%BF%97%E5%90%88%E9%9B%86&ka=100&submit=">国图方志合集</a> <a href="http://www.guoxuedashi.com/so.php?sokeytm=%E5%90%84%E5%9C%B0%E6%96%B9%E5%BF%97&submit=&kt=1"><strong>各地方志</strong></a>

</div>
</div>


<div class="sidebar2">
<center>

</center>
</div>
<div class="sidebar greenbar">
<div class="sidebar_title green">四库全书</div>
<div class="sidebar_info">

《四库全书》是中国古代最大的丛书,编撰于乾隆年间,由纪昀等360多位高官、学者编撰,3800多人抄写,费时十三年编成。丛书分经、史、子、集四部,故名四库。共有3500多种书,7.9万卷,3.6万册,约8亿字,基本上囊括了古代所有图书,故称“全书”。<a href="http://www.guoxuedashi.com/SiKuQuanShu/">详细>>
</a>

</div> 
</div>

</div>  <!--end r-->

</div>
<!-- 内容区END --> 

<!-- 页脚开始 -->
<div class="shh">

</div>

<div class="w1180" style="margin-top:8px;">
<center><script src="http://www.guoxuedashi.com/img/plus.php?id=3"></script></center>
</div>
<div class="w1180 foot">
<a href="/b/thanks.php">特别致谢</a> | <a href="javascript:window.external.AddFavorite(document.location.href,document.title);">收藏本站</a> | <a href="#">欢迎投稿</a> | <a href="http://www.guoxuedashi.com/forum/">意见建议</a> | <a href="http://www.guoxuemi.com/">国学迷</a> | <a href="http://www.shuowen.net/">说文网</a><script language="javascript" type="text/javascript" src="https://js.users.51.la/17753172.js"></script><br />
  Copyright &copy; 国学大师 古典图书集成 All Rights Reserved.<br>
  
  <span style="font-size:14px">免责声明:本站非营利性站点,以方便网友为主,仅供学习研究。<br>内容由热心网友提供和网上收集,不保留版权。若侵犯了您的权益,来信即刪。scp168@qq.com</span>
  <br />
ICP证:<a href="http://www.beian.miit.gov.cn/" target="_blank">鲁ICP备19060063号</a></div>
<!-- 页脚END --> 
<script src="http://www.guoxuedashi.com/img/plus.php?id=22"></script>
<script src="http://www.guoxuedashi.com/img/tongji.js"></script>

</body>
</html>
