資治通鑑卷一百五十  宋 司馬光 撰

胡三省 音註

梁紀十四|{
	起著雍敦䍧盡閉逢困敦凡七年}


高祖武皇帝十四

大同四年春正月辛酉朔日有食之 東魏碭郡獲巨象送鄴|{
	魏收志孝昌二年置碭郡治下邑城屬徐州碭徒朗翻}
丁卯大赦改元元象 二月己亥上耕籍田 東魏大都督善無賀拔仁攻魏南汾州刺史韋子粲降之|{
	降戶江翻下同}
丞相泰滅子粲之族東魏大行臺侯景等治兵於虎牢|{
	冶直之翻}
將復河南諸州魏梁迥韋孝寛趙繼宗皆棄城西歸|{
	梁迥在潁川韋孝寛在汝南未知趙繼宗所棄何城也}
侯景攻廣州未拔|{
	廣州治襄城}
聞魏救兵將至集諸將議之行洛州事盧勇請進觀形勢|{
	東魏洛州治洛陽諸將即亮翻下同}
乃帥百騎至大隗山|{
	班志河南郡密縣有大隗山魏收志密縣屬滎陽郡五代志滎陽郡新城縣有大騩山帥讀曰率騎奇寄翻騩隗同五賄翻又音歸}
遇魏師日已暮勇乃置幡旗於樹顛夜分騎為十隊鳴角直前擒魏儀

同三司|{
	闕}
程華斬儀同三司王征蠻而還|{
	還從宣翻又如字}
廣州守將駱超遂以城降東魏丞相歡以勇行廣州事勇辯之從弟也|{
	盧辯仕於西魏而勇仕於東魏從才用翻}
於是南汾潁豫廣四州復入東魏 |{
	考異曰典畧侯景克廣州在十一月按北史魏文帝紀二月東魏陷南汾潁豫廣四州今從之}
初柔然頭兵可汗始得返國|{
	事見一百四十九卷普通二年可從刋入聲汗音寒}
事魏盡禮及永安以後雄據北方禮漸驕倨雖信使不絶不復稱臣|{
	使疏吏翻下同復扶又翻}
頭兵嘗至洛陽心慕中國乃置侍中黄門等官後得魏汝陽王典籖淳于覃親寵任事以為祕書監使典文翰及兩魏分裂頭兵轉不遜數為邉患|{
	數所角翻}
魏丞相泰以新都關中方有事山東欲結婚以撫之以舍人元翌女為化政公主妻頭兵弟塔寒|{
	妻七細翻}
又言於魏主請廢乙弗后納頭兵之女甲辰以乙弗后為尼|{
	魏立乙弗后見上卷大同元年}
使扶風王孚迎頭兵女為后頭兵遂留東魏使者元整不報其使 三月辛酉東魏丞相歡以沙苑之敗|{
	敗見上卷上年}
請解大丞相詔許之頃之復故 |{
	考異曰北齊帝紀止有高祖解丞相年月而無復故之文按興和元年議歷有丞相田曹參軍信都芳盖因邙山之捷而復也}
柔然送悼后於魏|{
	郁久閭后諡曰悼}
車七百乘馬萬匹駝二千頭至黑鹽池|{
	唐志鹽州五原縣有烏池白池烏池盖即黑鹽池也乘䋲證翻}
遇魏所遣鹵簿儀衛柔然營幕戶席皆東向扶風王孚請正南面后曰我未見魏主固柔然女也魏仗南面我自東向丙子立皇后郁久閭氏丁丑大赦以王盟為司徒丞相泰朝於長安還屯華州|{
	朝直遥翻下同華戶化翻}
夏四月庚寅東魏高歡朝於鄴|{
	既解丞相遂不書官而書姓通鑑紀實非如春秋之有所褒貶也}
壬辰還晉陽 五月甲戍東魏遣兼散騎常侍鄭伯猷來聘 |{
	考異曰魏帝紀在二月丙辰盖始受命時也今從梁帝紀}
秋七月東魏荆州刺史王則寇淮南|{
	此淮南謂光武弋陽之地在淮水上流之南非指古淮南郡治夀春之淮南}
癸亥詔以東冶徒李胤之得如來舍利大赦|{
	如來猶言佛也}
東魏侯景高敖曹等圍獨孤信于金墉太師歡帥大軍繼之|{
	帥讀曰率下同}
景悉燒洛陽内外官寺民居存者什二三魏主將如洛陽拜園陵|{
	魏自孝文帝遷洛以後孝武帝西遷以前園陵皆在洛州}
會信等告急遂與丞相泰俱東命尚書左僕射周惠達輔太子欽守長安開府儀同三司李弼車騎大將軍達奚武帥千騎為前驅八月庚寅丞相泰至糓城|{
	漢志河南郡有穀成縣師古注曰即今新安水經注曰穀城縣城西臨穀水騎奇寄翻下同}
侯景等欲整陳以待其至|{
	陳讀曰陣}
儀同三司太安莫多婁貸文請帥所部擊其前鋒|{
	莫多婁虜三字姓}
景等固止之貸文勇而專不受命與可朱渾道元以千騎前進夜遇李弼達奚武於孝水|{
	五代志新安縣有孝水水經注孝水出廆山之隂北流注于穀在河南城西四十里}
弼命軍士皷譟曳柴揚塵貸文走弼追斬之道元單騎獲免悉俘其衆送恒農|{
	恒戶登翻}
泰進軍瀍東|{
	瀍水出河南穀城縣北山東與千金渠合又東過洛陽縣南又東過偃師縣又東入于洛}
侯景等夜解圍去辛卯泰帥輕騎追景至河上景為陳北據河橋南屬邙山|{
	陳讀曰陣下陵陳置陳陷陳同屬之欲翻景置陳北據河橋者慮兵有利鈍先保固其北歸之路也南屬邙山可以見其兵多矣景軍參用馬步其置陣堅固宇文泰以輕騎來見其陣勢如此斂兵不進可也遽前合戰亦屢勝而驕耳}
與泰合戰泰馬中流矢驚逸遂失所之泰墜地東魏兵追及之左右皆散都督李穆下馬以策抶泰背罵曰籠東軍士|{
	抶丑栗翻打也荀子曰仁人之兵當之者潰觸之者角摧案角鹿埵隴種東籠而退耳楊倞注曰其義未詳盖皆摧敗披靡之貌陸德明曰東籠沾温貌也如衣服之沾濕然}
爾曹主何在而獨留此追者不疑其貴人捨之而過穆以馬授泰與之俱逸魏兵復振擊東魏兵大破之|{
	宇文泰先以輕騎合戰而不利大兵繼至軍勢復振故大破東魏}
東魏兵北走京兆忠武公高敖曹意輕泰建旗盖以陵陳魏人盡鋭攻之一軍皆沒敖曹單騎走投河陽南城|{
	河陽南城在河橋南岸北岸即北中城}
守將北豫州刺史高永樂歡之從祖兄子也|{
	樂音洛從才用翻下其從同}
與敖曹有怨閉門不受敖曹仰呼求繩不得拔刀穿闔未徹而追兵至|{
	杜預曰闔門扇也未徹未透也徹敕列翻}
敖曹伏橋下追者見其從奴持金帶|{
	從才用翻}
問敖曹所在奴指示之敖曹知不免奮頭曰來與汝開國公|{
	言得其頭西魏將以開國賞之也}
追者斬其首去高歡聞之如喪肝膽|{
	喪息浪翻}
杖高永樂二百贈敖曹太師大司馬太尉泰賞殺敖曹者布絹萬段歲歲稍與之比及周亡猶未能足|{
	比必利翻}
魏又殺東魏西兖州刺史宋顯等虜甲士萬五千人赴河死者以萬數初歡以万俟普尊老|{
	爵尊而齒老也万莫北翻俟渠之翻}
特禮之嘗親扶上馬其子洛免冠稽首曰|{
	上時掌翻稽音啟}
願出死力以報深恩及邙山之戰諸軍北度橋|{
	北度河橋也}
洛獨勒兵不動謂魏人曰万俟受洛干在此能來可來也魏人畏之而去歡名其所營地為囘洛|{
	唐志河陽關南有囘洛故城}
是日東西魏置陳既大首尾懸遠從旦至未戰數十合|{
	史言兩軍確鬭}
氛霧四塞|{
	塞悉則翻}
莫能相知魏獨孤信李遠居右趙貴怡峯居左戰並不利又未知魏主及丞相泰所在皆棄其卒先歸開府儀同三司李虎念賢等為後軍見信等退即與俱去泰由是燒營而歸留儀同三司長孫子彦守金墉王思政下馬舉長矟左右横擊一舉輒踣數人|{
	矟色角翻踣蒲北翻}
陷陳既深從者盡死思政被重創悶絶會日暮敵亦收兵思政每戰常著破衣弊甲|{
	從才用翻被皮義翻創初良翻下裹創同著陟畧翻}
敵不知其將帥故得免|{
	將即亮翻帥所類翻}
帳下督雷五安於戰處哭求思政會其已蘇割衣裹創扶思政上馬夜久始得還營平東將軍蔡祐下馬步鬪左右勸乘馬以備倉猝|{
	謂兵有邂逅乘馬則倉猝可以奔馳}
祐怒曰丞相愛我如子今日豈惜生乎帥左右十餘人合聲大呼擊東魏兵殺傷甚衆東魏圍之十餘重|{
	帥讀曰率呼火故反重直龍翻}
祐彎弓持滿四面拒之東魏人募厚甲長刀者直進取之去祐可三十步左右勸射之祐曰吾曹之命在此一矢豈可虚發將至十步祐乃射之應弦而倒|{
	射而亦翻}
東魏兵稍却祐徐引還魏主至恒農守將已棄城走所虜降卒在恒農者相與閉門拒守|{
	恒戶登翻將即亮翻降戶江翻}
丞相泰攻拔之誅其魁首數百人蔡祐追及泰於恒農夜見泰泰曰承先|{
	蔡祐字承先}
爾來吾無憂矣泰驚不得寢枕祐股然後安|{
	史言泰氣衰膽失枕之任翻}
祐每從泰戰常為士卒先戰還諸將皆争功祐終無所言泰每嘆曰承先口不言勲我當代其論叙|{
	史言蔡祐沉勇不伐}
泰留王思政鎮恒農除侍中東道行臺魏之東伐關中留守兵少|{
	少詩沼翻}
前後所虜東魏士卒散在民間聞魏兵敗謀作亂李虎等至長安計無所出與太尉王盟僕射周惠達等奉太子欽出屯渭北百姓互相剽掠關中大擾|{
	剽匹妙反}
於是沙苑所虜東魏都督趙青雀雍州民于伏德等遂反據長安子城伏德保咸陽|{
	雍於用翻後魏置咸陽郡於石安縣石安漢渭城縣秦之咸陽也石勒改曰石安五代志京兆渭陽縣舊置咸陽郡}
與咸陽太守慕容思慶各收降卒以拒還兵|{
	降卒東魏之卒降於西魏散在民間者也降戶江翻}
長安大城民相帥以拒青雀日與之戰|{
	帥讀曰率}
大都督侯莫陳順擊賊屢破之賊不敢出順崇之兄也扶風公王羆鎮河東大開城門悉召軍士謂曰今聞大軍失利青雀作亂諸人莫有固志王羆受委於此以死報恩有能同心者可共固守必恐城陷任自出城衆感其言皆無異志魏主留閿鄉|{
	閿音旻}
丞相泰以士馬疲弊不可速進且謂青雀等烏合不能為患曰我至長安以輕騎臨之必當面縛通直散騎常侍吳郡陸通諫曰|{
	陸通本吳人曾祖載從宋武帝入關及劉義真之敗沒於赫連遂居關中}
賊逆謀久定必無遷善之心蜂蠆有毒|{
	左傳臧文仲之言蠆丑邁翻}
安可輕也且賊詐言東寇將至|{
	東寇謂東魏之兵}
今若以輕騎臨之百姓謂為信然|{
	以賊言為信也}
益當驚擾今軍雖疲弊精鋭尚多以明公之威總大軍以臨之何憂不克泰從之引兵西入父老見泰至莫不悲喜士女相賀華州刺史宇文導引兵入咸陽斬思慶禽伏德南度渭與泰會攻青雀破之|{
	華戶化翻}
太保梁景睿以疾留長安與青雀通謀泰殺之東魏太師歡自晉陽將七千騎至孟津未濟聞魏師已遁遂濟河遣别將追魏師至崤|{
	魏收志太和十一年置崤縣屬恒農郡因三崤山以名縣隋并崤縣入河南熊耳縣程大昌曰三崤山一名嶔崟山元和志曰自東崤至西崤三十五里東崤長阪數里峻阜絶澗車不得方軌西崤全是石阪十二里險不異東崤此二崤皆在秦關之東漢關之西}
不及而還|{
	還從宣翻又如字下同}
歡攻金墉長孫子彦棄城走焚城中室屋俱盡歡毁金墉而還東魏之遷鄴也|{
	見一百五十六卷中大通六年}
主客郎中裴讓之留洛陽獨孤信之敗也|{
	謂邙山之敗信先棄軍西還}
讓之弟諏之随丞相泰入關為大行臺倉曹郎中歡囚讓之兄弟五人讓之曰昔諸葛亮兄弟事吳蜀各盡其心|{
	謂亮事蜀瑾事吴也}
况讓之老母在此不忠不孝必不為也明公推誠待物物亦歸心若用猜忌去霸業遠矣歡皆釋之九月魏主入長安丞相泰還屯華州|{
	華戶化翻}
東魏大都督賀拔仁擊邢磨納盧仲禮等平之|{
	磨納等起兵見上卷上年}
盧景裕本儒生太師歡釋之召舘於家使教諸子景裕講論精微難者或相詆訶大聲厲色言至不遜而景裕神采儼然風調如一從容往復|{
	難乃旦翻詆丁禮翻訶虎何翻調徒釣翻從千容翻}
無際可尋性清静歷官屢有進退無得失之色弊衣麤食恬然自安終日端嚴如對賓客 冬十月魏歸高敖曹竇泰莫多婁貸文之首于東魏 散騎常侍劉孝儀等聘於東魏|{
	散悉亶翻騎奇寄翻}
十二月魏是云寶襲洛陽東魏洛州刺史王元軌棄城走都督趙剛襲廣州拔之於是自襄廣以西城鎮復為魏|{
	魏收志孝昌中置襄州領襄城舞隂南安期城北南陽建城郡五代志潁川郡葉縣後齊置襄州復扶又翻下系復同}
魏自正光以後四方多事民避賦役多為僧尼至二百萬人寺有三萬餘區至是東魏始詔牧守令長擅立寺者計其功庸|{
	守式又翻長知兩翻庸用也勞也雇也}
以枉灋論 初魏伊川土豪李長夀為防蠻都督|{
	五代志河南郡陸渾縣齊置伊川郡領南陸渾縣春秋時秦晉遷陸渾之戎於伊川故郡以為名伊闕以南大山長谷蠻多居之魏置都督以防焉}
積功至北華州刺史孝武帝西遷長夀帥其徒拒東魏|{
	華戶化翻帥讀曰率}
魏以長夀為廣州刺史侯景攻拔其壁殺之其子延孫復收集父兵以拒東魏魏之貴臣廣陵王欣錄尚書長孫稚等皆㩗家往依之|{
	長知兩翻}
延孫資遣衛送使達關中東魏高歡患之數遣兵攻延孫不能克|{
	數所角反}
魏以延孫為京南行臺節度河南諸軍事廣州刺史|{
	京南謂洛京以南也}
延孫以澄清伊洛為己任魏以延孫兵少|{
	少詩沼翻}
更以長夀之壻京兆韋灋保為東洛州刺史|{
	西魏洛州治上洛以洛陽之地為東洛州}
配兵數百以助之灋保名祐以字行既至與延孫連兵置柵於伏流|{
	伏流城伊川郡治所隋改南陸渾縣曰伏流水經注曰陸渾故城東南八十許里有三塗山伊水逕其下又東北逕伏流嶺東劉澄之永初記稱伏流縣西有伏流阪者今山在縣南崖口北三千里許西則非也}
獨孤信之入洛陽也欲繕修宫室使外兵郎中天水權景宣|{
	曹魏置二十三郎有中兵外兵都兵别兵元魏以後中兵外兵又分左右}
帥徒兵三千出採運|{
	採運者使之採木運入洛陽城也}
會東魏兵至河南皆叛景宣間道西走|{
	間古莧翻}
與李延孫相會攻孔城拔之|{
	魏收志天平中置新城郡治孔城屬北荆州五代志河南郡伊闕縣舊曰新城置新城郡杜佑曰孔城防今伊闕縣東南故城是又曰高齊置孔城衍於夀安縣}
洛陽以南尋亦西附丞相泰即留景宣守張白塢|{
	塢在宜陽西北水經注河内軹縣有張白騎塢在溴水北原上據二溪之會北帶深隍三面阻險唯西面版築而已}
節度東南諸軍應關西者是歲延孫為其長史楊伯蘭所殺韋灋保即引兵據延孫之柵東魏將段琛等據宜陽|{
	將即亮翻琛五林翻}
遣陽州刺史牛道恒誘魏邉民|{
	魏收志天平初置陽州領宜陽金門郡治宜陽恒戶登翻誘音酉}
魏南兖州刺史韋孝寛患之|{
	按韋孝寛傳時西魏令孝寛領宜陽郡事遷南兖州刺史然南兖州治譙城在東魏境内孝寛未能取其地也}
乃詐為道恒與孝寛書論歸欵之意使諜人遺之於琛營|{
	諜徒協翻遺如字墜失也}
琛果疑道恒孝寛乘其猜阻出兵襲之擒道恒及琛崤澠遂清|{
	崤澠崤山及澠池也澠彌兖翻}
東道行臺王思政以玉壁險要|{
	五代志絳郡稷山舊置勲州勲州即玉壁也杜佑曰稷山縣南十二里即玉壁城}
請築城自恒農徙鎮之詔加都督汾晉并州諸軍事并州刺史行臺如故|{
	東西魏盖於汾州據險為界晉并皆入於東魏}
東魏以高澄攝吏部尚書始改崔亮年勞之制|{
	崔亮制停年格見一百四十九卷天監十八年}
銓擢賢能又沙汰尚書郎妙選人地以充之凡才名之士雖未薦擢皆引致門下與之遊宴講論賦詩士大夫以是稱之|{
	史言高澄收拾人物以傾元氏}


五年春正月乙卯以尚書左僕射蕭淵藻為中衛將軍丹陽尹何敬容為尚書令吏部尚書張纘為僕射纘弘策之子也|{
	張弘策帝舅也佐帝創業}
自晉宋以來宰相皆以文義自逸敬容獨勤簿領日旰不休為時俗所嗤鄙|{
	旰古按翻嗤丑之翻}
自徐勉周捨既卒|{
	普通五年周捨卒大同元年徐勉卒卒子恤翻}
當權要者外朝則何敬容内省則朱异|{
	三公卿監尚書為外朝官門下省為内省朝直遥翻异羊至翻}
敬容質慤無文以綱維為己任异善伺候人主意為阿諛用事三十年廣納貨賂欺罔視聽遠近莫不忿疾園宅玩好飲膳聲色窮一時之盛每休下|{
	休沐之日自省中出還私第為休下伺相吏翻好呼到翻下遐嫁翻}
車馬填門唯王承王稚及褚翔不往承稚之子|{
	王暕儉之子帝用之官至侍中尚書僕射}
翔淵之曾孫也|{
	禇淵蕭齊佐命}
丁巳御史中丞參禮儀事賀琛奏南北二郊及籍田往還並宜御輦不復乘輅詔從之祀宗廟仍乘玉輦|{
	駕馬為輅駕人為輦}
琛瑒之弟子也|{
	賀瑒以儒學進瑒雉杏翻又音暢}
辛酉東魏以尚書令孫騰為司徒 辛未上祀南郊 魏丞相泰於行臺置學取丞郎府佐德行明敏者充學生|{
	行下孟翻}
悉令旦治公務晚就講習|{
	時就學者苟能以其所治之事質之於經傳有所感發此其所學之進又豈螢窻雪案搜經摘傳者所能及邪治直之翻}
東魏丞相歡|{
	歡以沙苑之敗求自貶既復其官史復以丞相書之}
以徐州刺史房謨廣平太守羊敦廣宗太守竇瑗|{
	廣宗縣漢屬鉅鹿郡後屬安平國後魏太和二十一年立廣宗郡東魏屬司州瑗于眷反}
平原太守許惇有政績清能與諸刺史書褒稱謨等以勸之 夏五月甲戍東魏立丞相歡女為皇后乙亥大赦 魏以開府儀同三司李弼為司空 秋七月以扶風王孚為太尉 九月甲子東魏發畿内十萬人城鄴四十日罷冬十月癸亥以新宫成大赦改元興和魏置紙筆於陽武門外以求得失 十一月乙亥東

魏使散騎常侍王元景魏收來聘 東魏人以正光歷浸差|{
	魏行正光歷見一百四十九卷普通三年}
命校書郎李業興更加脩正|{
	杜佑曰漢之蘭臺及後漢東觀皆藏書之室當時文學之士使讐校于其中故有校書之職盖有校書之任而未為官也故以郎居其任則謂之校書郎以郎中居其任則謂之校書郎中至後魏始置祕書校書郎}
以甲子為元號曰興光歷既成行之 散騎常侍朱异奏頃來置州稍廣而小大不倫請分為五品其位秩高卑參僚多少皆以是為差|{
	參僚即參佐}
詔從之於是上品二十州次品十州次品八州次品二十三州下品二十一州時上方事征伐恢拓境宇北踰淮汝東距彭城西開牂牁|{
	牂牁音臧哥}
南平俚洞|{
	交廣界表俚人依阻深險各自為洞俚音里}
紛綸甚衆故异請分之其下品皆異國之人徒有州名而無土地或因荒徼之民所居村落置州及郡縣刺史守令皆用彼人為之|{
	就彼土以土人為之徼吉弔翻}
尚書不能悉領山川險遠職貢罕通五品之外又有二十餘州不知處所凡一百七州又以邊境鎮戌雖領民不多欲重其將帥皆建為郡或一人領二三郡太守州郡雖多而戶口日耗矣|{
	務廣地者荒貪人有者殘信哉將即亮翻帥所類翻}
魏自西遷以來禮樂散逸丞相泰命左僕射周惠達吏部郎中北海唐瑾損益舊章至是稍備|{
	瑾渠吝翻}


六年春正月壬申東魏以廣平公庫狄干為太保 丁丑東魏主入新宫|{
	東魏作新宫於鄴見上卷大同元年}
大赦 魏扶風王孚卒 二月己亥上耕籍田 魏鑄五銖錢 東魏大行臺侯景出三鵶|{
	杜佑曰三鵶在今汝州魯陽縣西南十九里有平高城周以禦齊高齊於縣東北十七里置魯城以禦周}
將復荆州|{
	三年西魏乘沙苑之敗取荆州}
魏丞相泰遣李弼獨孤信各將五千騎出武關景乃還|{
	將即亮翻騎奇寄翻}
魏文后既為尼居别宫|{
	大通四年魏廢文后為尼尼女夷翻}
悼后猶忌之乃以其子武都王戊為秦州刺史使文后随之官魏主雖限以大計|{
	謂以國事廢乙弗而立柔然女也}
而恩好不忘|{
	好呼到反}
密令養髮有追還之意|{
	養羊尚翻}
會柔然舉國渡河南侵|{
	渡河南侵靈夏}
時頗有言柔然以悼后故興師者帝曰豈有興百萬之衆為一女子邪|{
	為于偽翻}
雖然致人此言朕亦何顔以見將師|{
	將即亮翻帥所類翻}
乃遣中常侍曹寵齎手敕賜文后自盡文后泣謂寵曰願至尊千萬歲天下康寧死無恨也遂自殺鑿麥積崖而葬之號曰寂陵夏丞相泰召諸軍屯沙苑以備柔然右僕射周惠達發士馬守京城塹諸街巷召雍州刺史王羆議之|{
	塹七艶翻雍於用翻}
羆不應召謂使者曰若蠕蠕至渭北者王羆自帥鄉里破之|{
	王羆京兆人使所吏翻蠕人兖翻帥讀曰率}
不煩國家兵馬何為天子城中作如此驚擾由周家小兒恇怯致此|{
	恇去王翻}
柔然至夏州而退未幾悼后遇疾殂|{
	幾居豈翻}
五月乙酉魏行臺宫延和陕州刺史宫延慶降于東魏|{
	陕式冉翻降戶江翻}
東魏以河北馬場為義州以處之|{
	按杜佑通典衛州汲郡古牧野之地魏孝文帝太和十七年徙代畜於石濟之西故有河北馬場魏收志是時置義州治汲郡陳城領五城義寧新安澠池恒農宜陽金門郡五代志汲郡汲縣東魏置義州僑置七郡二十八縣則七郡皆僑置於汲縣界又據朱元旭傳時分河内汲郡二郡界扶風之地立義州以置關西歸正之民後周武帝滅齊改義州為衛州治汲處昌呂翻}
東魏陽州武公高永樂卒|{
	據北齊書高永樂封陽州縣公}
閠月丁丑朔日有食之 己丑東魏封皇兄景植為宜陽王皇弟威為清河王謙為潁川王 六月壬子東魏華山王鷙卒|{
	二年東魏以鷙為大司馬華戶化翻}
秋七月丁亥東魏使兼散騎常侍李象等來聘 八月戊午大赦 戊戌司空表昂卒遺疏不受贈諡敕諸子勿上行狀及立銘誌|{
	行狀狀其平生之行實上之于朝以請諡銘誌立碑於墓以傳後洪适曰東漢自路都尉始建墓闕盖表阡碑銘之濫觴也有文而傳於今則自謁者景君墓表始君以安帝元初元年卒齊葬穆妃議立石誌王儉以為非禮經所出元嘉中顔延之輩為之遂相祖述爾任昉作文章緣起又云墓碑自晉始予考酈氏水經所載漢刻已不少後魏與齊梁時相先後也豈碑碣多在北方南人未之見乎然郭林宗傳云林宗既葬同志者立碑蔡邕為其文謂盧植曰吾為碑銘多矣唯郭有道無愧色史稱王儉晉宋以來故事該憶無遺范書所載豈不知之今漢人舊刻猶存數十百碑云始於晉宋非也行下孟翻上時掌翻}
上不許贈本官諡穆正公 冬十一月魏太師念賢卒 吐谷渾自莫折念生之亂不通於魏伏連籌卒子夸呂立始稱可汗居伏俟城|{
	五代志隋破吐谷渾以伏俟城置西海郡其地有西王母石窟青海鹽池即在漢金城郡臨羌縣西北塞外王莽受卑和羌所獻地置西海郡者也北史伏俟城在青海西十五里吐從暾入聲谷音浴可從刋入聲汗音寒}
其地東西三千里南北千餘里官有王公僕射尚書郎中將軍之號是歲始遣使假道柔然聘於東魏|{
	使疏吏反}


七年春正月辛巳上祀南郊大赦辛丑祀明堂 宕昌王梁仚定為其下所殺弟彌定立|{
	宕徒浪翻仚許延翻考異曰梁帝紀作彌泰今從典略}
二月乙巳以彌定為河梁二州刺史宕昌王辛亥上耕籍田|{
	籍秦昔翻}
魏幽州刺史順陽王仲景坐事賜死|{
	西魏無幽州意豳州也}
三月魏夏州刺史劉平伏據上郡反|{
	魏收志上郡屬東夏州領石門因城縣隋志延安郡有因城縣夏戶雅翻}
大都督于謹討禽之 夏五月遣兼散騎常侍明少遐等聘於東魏秋七月己卯東魏宜陽王景植卒 魏以侍中宇文測為大都督行汾州事|{
	五代志龍泉郡後周置汾州隋改隰州治隰川縣}
測深之兄也為政簡惠得士民心地接東魏|{
	隰川東接東魏晉州界}
東魏人數來寇抄測禽獲之命解縛引與相見為設酒殽待以客禮|{
	數所角翻抄楚交翻為于偽翻}
并給糧餼|{
	餼許氣翻}
衛送出境東魏人大慙不復為寇|{
	復扶又翻}
汾晉之間遂通慶弔時論稱之或告測交通境外者丞相泰怒曰測為我安邊我知其志何得間我骨肉|{
	間古莧翻}
命斬之 魏丞相泰欲革易時政為彊國富民之灋大行臺度支尚書兼司農卿蘇綽盡其智能贊成其事减官員置二長|{
	度徒洛翻長知兩翻下令長同}
并置屯田以資軍國又為六條詔書九月始奏行之一曰清心二曰敦教化三曰盡地利四曰擢賢良五曰恤獄訟六曰均賦役泰甚重之嘗置諸坐右|{
	坐徂卧翻}
又令百司習誦之其牧守令長非通六條及計帳不得居官|{
	守式又翻計帳見上卷大同元年}
東魏詔羣官於麟趾閣議定灋制謂之麟趾格冬十月甲寅頒行之 乙巳東魏發夫五萬築漳濱堰三十五日罷 十一月丙戍東魏以彭城王韶為大尉度支尚書胡僧敬為司空僧敬名䖍以字行國珍之兄孫東魏主之舅也|{
	胡國珍靈后之父}
十二月東魏遣兼散騎常侍李騫來聘交趾李賁世為豪右仕不得志有并韶者富於詞藻詣選求官吏部尚書蔡撙以并姓無前賢除廣陽門郎|{
	廣陽門建康城西面南頭第一門姓譜并府盈翻撙祖本翻}
韶恥之賁與韶還鄉里會交州刺史武林侯諮以刻暴失衆心|{
	沈約志永平郡有武林縣宋文帝立永平晉穆帝升平五年分蒼梧立}
時賁監德州|{
	五代志日南郡梁置德州監工衘翻}
因連結數州豪傑俱反諮輸賄于賁奔還廣州上遣諮與高州刺史孫冏新州刺史盧子雄將兵擊之|{
	五代志梁大通中割番州合浦縣立高州在隋海康縣界端州新興縣梁立新州將即亮翻}
諮恢之子也|{
	鄱陽王恢上之弟也}
是歲魏又益新制十二條|{
	宇文泰前已行二十四條今又益十二條故曰新制}
東魏丞相歡以諸州調絹不依舊式|{
	調徒釣翻謂尺度不依舊式也}
民甚苦之奏令悉以四十尺為匹魏自喪亂以來|{
	謂孝昌以來也喪息浪翻}
農商失業六鎮之民相帥内徙就食齊晉歡因之以成霸業|{
	事見一百五十五卷中大通三年四年齊晉直謂春秋列國大界帥讀曰率}
東西分裂連年戰争河南州郡鞠為茂草|{
	小弁之詩曰踧踧周道鞠為茂革注云鞠窮也}
公私困竭民多餓死歡命諸州濱河及津梁|{
	凡江河濟渡之處皆曰津横絶水為橋以通往來曰梁}
皆置倉積穀以相轉漕供軍旅備飢饉又於幽瀛滄青四州傍海煮鹽軍國之費|{
	傍步浪翻}
粗得周贍|{
	粗坐五翻贍而艶翻}
至是東方連歲大稔穀斛至九錢山東之民稍復蘇息矣|{
	史言高歡於兵荒之餘能紓民力復扶又翻又如字}
東魏尚書令高澄尚静帝妹馮翊長公主生子孝琬朝貴賀之|{
	長知兩翻朝直遥翻}
澄曰此至尊之甥先賀至尊三日帝幸其第賜錦綵布絹萬匹於是諸貴競致禮遺|{
	遺于季反}
貨滿十室 東魏臨淮王孝友表曰令制百家為族二十五家為閭五家為比百家之内有帥二十五徵發皆免苦樂不均羊少狼多復有蠶食|{
	使狼將羊羊雖衆終為狼所噬况羊少而狼多乎喻族帥並緣侵漁閭帥閭帥又侵漁比帥比帥又浸漁其所領四家也比毘至翻帥所類翻樂音洛少詩沼翻復扶又翻}
此之為弊久矣京邑諸坊或七八百家唯一里正二史庶事無闕而况外州乎請依舊置三正之名不改|{
	三正即李冲建議所置三長}
而每閭止為二比計族省十一丁貲絹番兵所益甚多事下尚書寢不行|{
	貲絹謂討貲輸絹番兵謂番代為兵下遐嫁翻}
安成望族劉敬躬以妖術惑衆人多信之|{
	妖於驕翻 考異曰南史作敬官今從梁書}


八年春正月敬躬據郡反改元永漢署官屬進攻廬陵逼豫章南方久不習兵人情擾駭豫章内史張綰募兵以拒之綰纘之弟也二月戊戌江州刺史湘東王繹遣司馬王僧辯中兵曹子郢討敬躬受綰節度二月戊辰擒敬躬送建康斬之僧辯神念之子也|{
	天監七年王神念自魏來奔}
該博辯捷器宇肅然雖射不穿札|{
	左傳潘黨養由基罇甲而射之徹七札焉編甲如櫛齒相比曰札}
而志氣高遠 魏初置六軍 夏四月丙寅東魏使兼散騎常侍李繪來聘繪元忠之從子也|{
	李元忠勸成高歡討爾朱之謀從才用翻}
東魏丞相歡朝於鄴|{
	朝直遥翻}
司徒孫騰坐事免乙酉以彭城王韶録尚書事侍中廣陽王湛為太尉尚書右僕射高隆之為司徒初太尉尉景與丞相歡同歸爾朱榮|{
	見一百五十二卷大通二年}
其妻歡之姊也自恃勲戚貪縱不灋為有司所劾繫獄歡三詣闕泣請乃得免死|{
	史言高歡為是使勲貴知有王法劾戶槩翻又戶得翻}
丁亥降為驃騎大將軍開府儀同三司|{
	驃匹妙翻騎奇寄翻}
歡往造之景卧不起大叫曰殺我時趣邪|{
	造七到翻趣讀曰促後趣之同}
歡撫而拜謝之辛卯以庫狄干為太傅以領軍將軍婁昭為大司馬封祖裔為尚書右僕射六月甲辰歡還晉陽 八月庚戍東魏以開府儀同三司吏部尚書侯景為兼尚書僕射河南道大行臺随機防討|{
	既委景以備梁魏又使討叛貳随機則便宜從事其任重矣為侯景叛東魏張本}
魏以王盟為太保 東魏丞相歡擊魏入自汾絳連營四十里丞相泰使王思政守玉壁以斷其道|{
	斷音短後於玉壁置勲州杜佑曰玉壁城在絳州稷山縣西南十二里}
歡以書招思政曰若降當授以并州|{
	高歡以晉陽為根本并州之任要重於諸州降戶江翻}
思政復書曰可朱渾道元降|{
	道元降見上卷元年}
何以不得冬十月己亥歡圍玉壁凡九日遇大雪士卒飢凍多死者遂解圍去魏遣太子欽鎮蒲坂丞相泰出軍蒲坂至皂荚聞歡退度汾追之不及十一月東魏以可朱渾道元為并州刺史|{
	激于王思政之書也}
十二月魏主狩於華隂大享將士丞相泰帥諸將朝之|{
	華戶化翻將即亮翻帥讀曰率朝直遥翻}
起萬夀殿於沙苑北辛亥東魏遣兼散騎常侍楊斐來聘 |{
	考異曰典畧作陽斐今從魏}


|{
	書紀}
孫冏盧子雄討李賁以春瘴方起請待至秋廣州刺史新渝侯映不許|{
	吴立新喻縣屬安成郡渝當作喻}
武林侯諮又趣之冏等至合浦死者什六七|{
	趣讀曰促瘴之亮翻熱病也南方瘴熱春氣深則瘴起染之者必死軍行尤忌之}
衆潰而歸映憺之子也|{
	始興王憺上弟也憺徒敢翻又徒濫翻}
武林侯諮奏冏及子雄與賊交通逗留不進敇於廣州賜死子雄弟子略子烈主帥廣陵杜天合及弟僧明新安周文育等帥子雄之衆攻廣州欲殺映諮為子雄復寃|{
	主帥所類翻等帥讀曰率下同為于偽翻}
西江督護高要太守吳興陳霸先帥精甲三千救之|{
	高要縣漢屬蒼梧梁置高要郡隋為高要縣端州治所元和郡縣志曰端州當西江口入廣西要道祝穆曰西江源出邕州經潯融象柳等州入封州界合桂江漢武帝自巴蜀發夜郎兵下牂柯江會番禺即此水蕭子顯曰廣州統内西南二江川源深遠别置督護專征討之任}
大破子略等殺天合擒僧明文育霸先以僧明文育驍勇過人釋之以為主帥|{
	驍堅堯翻}
詔以霸先為直閤將軍|{
	陳霸先事始此}
魏丞相泰妻馮翊公主生子覺 東魏以光州刺史李元忠為侍中元忠雖處要任|{
	處昌呂翻}
不以物務干懷唯飲酒自娱丞相歡欲用為僕射世子澄言其放達常醉不可委以臺閣其子搔聞之請節酒|{
	搔蘇遭翻}
元忠曰我言作僕射不勝飲酒樂|{
	樂音洛}
爾愛僕射宜勿飲酒九年春正月壬戌東魏大赦改元武定 東魏御史中尉高仲密取吏部郎崔暹之妹既而棄之由是與暹有隙仲密選用御史多其親戚鄉黨高澄奏令改選暹方為澄所寵任仲密疑其搆已愈恨之仲密後妻李氏艶而慧|{
	崔暹之妹既去李氏繼室故曰後妻}
澄見而悦之李氏不從衣服皆裂以告仲密仲密益怨尋出為北豫州刺史|{
	北豫州治虎牢}
隂謀外叛丞相歡疑之遣鎮城奚夀興典軍事|{
	鎮城之職猶防城都督}
仲密但知民務仲密置酒延夀興伏壮士執之二月壬申以虎牢叛降魏|{
	降戶江翻}
魏以仲密為侍中司徒歡以仲密之叛由崔暹將殺之高澄匿暹為之固請|{
	為于偽翻下為之同}
歎曰我匄其命|{
	匄居大翻又居曷翻}
須與苦手|{
	言必痛杖之也}
澄乃出暹而謂大行臺都官郎陳元康曰卿使崔暹得杖勿復相見|{
	復扶又翻}
元康為之言於歡曰|{
	為于偽翻}
大王方以天下付大將軍大將軍有一崔暹不能免其杖父子尚爾况於他人歡乃釋之高季式在永安戌|{
	永安縣古彘邑也漢屬河東郡後漢順帝改曰永安縣魏收志曰建義元年置永安郡治永安城屬晉州時季式罷晉州戍之隋廢永安郡改為霍邑縣}
仲密遣信報之季式走告歡歡待之如舊魏丞相泰帥諸軍以應仲密|{
	帥讀曰率}
以太子少傅李遠為前驅至洛陽遣開府儀同三司于謹攻柏谷拔之三月壬申圍河橋南城東魏丞相歡將兵十萬至河北|{
	將即亮翻}
泰退軍瀍上縱火船於上流以燒河橋斛律金使行臺郎中張亮以小艇百餘載長鎻伺火船將至以釘釘之|{
	艇徒鼎翻小舡也伺相吏翻上釘音丁下釘丁定翻}
引鎻向岸橋遂獲全歡度河據邙山為陳不進者數日泰留輜重於瀍曲|{
	陳讀曰陣重直用翻瀍曲或作瀍西}
夜登邙山以襲歡候騎白歡曰賊距此四十餘里蓐食乾飯而來歡曰自當渴死乃正陣以待之|{
	歡欲堅陣以持之待其疲渇而後戰故云然乾音干}
戊申黎明泰軍與歡軍遇東魏彭樂以數千騎為右甄|{
	甄稽延翻}
衝魏軍之北垂所向奔潰遂馳入魏營人告彭樂叛歡甚怒俄而西北塵起樂使來告捷|{
	使所吏翻}
虜魏侍中開府儀同三司大都督臨洮王東蜀郡王榮宗江夏王昇鉅鹿王闡譙郡王亮詹事趙善及督將僚佐四十八人|{
	夏戶雅翻}
諸將乘勝擊魏大破之斬首三萬餘級|{
	洮土刀翻將即亮翻 考異曰北齊書云俘斬六萬級今從北史彭樂傳}
歡使彭樂追泰泰窘|{
	窘巨隕翻}
謂樂曰汝非彭樂邪癡男子今日無我明日豈有汝邪何不急還營收汝金寶樂從其言獲泰金帶一囊以歸言於歡曰黑獺漏刃破膽矣歡雖喜其勝而怒其失泰令伏諸地親捽其頭連頓之|{
	捽昨沒翻捽持其髻也}
并數以沙苑之敗|{
	事見上卷三年數所具翻}
舉刃將下者三噤齘良久|{
	噤齘切齒怒也噤直禁翻亦作䫴齘胡介翻}
樂曰乞五千騎復為王取之|{
	復扶又翻為于偽翻}
歡曰汝縱之何意而言復取邪|{
	復扶又翻下日復同}
命取絹三千匹壓樂背因以賜之|{
	為歡屬其子澄令防彭樂張本}
明日復戰泰為中軍中山公趙貴為左軍領軍若于惠等為右軍|{
	令狐德棻曰若于之先與魏俱起以國為姓魏書官氏志内入諸姓有若乎氏乎作于孫愐曰若人者翻}
中軍右軍合擊東魏大破之悉俘其步卒歡失馬赫連陽順下馬以授歡歡上馬走從者步騎七人|{
	上時掌翻從才用翻}
追兵至親信都督尉興慶曰王速去興慶腰有百箭足殺百人歡曰事濟以爾為懷州刺史|{
	魏收志天安二年置懷州於河内太和八年罷天平初復置領河内武德二郡}
若死用爾子興慶曰兒少願用兄歡許之|{
	少詩沼翻}
興慶拒戰矢盡而死 |{
	考異曰典畧作尉興敬今從北齊書北史}
東魏軍士有逃奔魏者告以歡所在 |{
	考異曰周賀抜勝傳云太祖見齊神武旗鼓識之今從典畧}
泰募勇敢三千人皆執短兵配大都督賀拔勝以攻之勝識歡於行間|{
	行戶剛翻}
執槊與十三騎逐之馳數里槊刃垂及|{
	槊色角翻}
因字之曰賀六渾賀拔破胡必殺汝歡氣殆絶河州刺史劉洪徽從旁射勝中其二騎|{
	河州時屬西魏境東魏使劉洪徽遥領刺史耳射而亦翻下同中竹仲翻}
武衛將軍段韶射勝馬斃之比副馬至|{
	比必利翻}
歡已逸去勝歎曰今日不執弓矢天也魏南郢州刺史耿令貴大呼獨入敵中|{
	魏收志梁置南郢州治赤石關領定城光城邊城郡五代志光州定城縣後齊置南郢州非西魏境也耿令貴亦遥領刺史耳呼火故翻}
鋒刃亂下人皆謂已死俄奮刀而還|{
	還從宣翻又如字}
如是數四當令貴前者死傷相繼乃謂左右曰吾豈樂殺人|{
	樂音洛}
壮士除賊不得不爾若不能殺賊又不為賊所傷何異逐坐人也|{
	逐坐人指當時持文墨議論者但能相随逐坐談而坐食也}
左軍趙貴等五將戰不利東魏兵復振|{
	復扶又翻}
泰與戰又不利會日暮魏兵遂遁東魏兵追之獨孤信于謹收散卒自後擊之追兵驚擾魏諸軍由是得全若于惠夜引去東魏兵追之惠徐下馬顧命厨人營食食畢謂左右曰長安死此中死有以異乎乃建旗鳴角收散卒徐還追騎疑有伏兵不敢逼泰遂入關屯渭上歡進至陕|{
	陕失冉翻}
泰遣開府儀同三司達奚武等拒之行臺郎中封子繪言於歡曰混壹東西正在今日昔魏太祖平漢中不乘勝取巴蜀|{
	事見六十七卷漢獻帝建安二十年}
失在遲疑後悔無及願大王不以為疑歡深然之集諸將議進止咸以為野無青草人馬疲瘦不可遠追陳元康曰兩雄交争歲月已久今幸而大捷天授我也時不可失當乘勝追之歡曰若遇伏兵孤何以濟元康曰王前沙苑失利彼尚無伏今奔敗若此何能遠謀若捨而不追必成後患歡不從|{
	為歡悔不用陳元康之言張本余謂邙山之戰盖俱傷而兩敗宇文泰雖力屈而遁高歡之氣亦衰矣安敢復深入乎}
使劉豐生將數千騎追泰遂東歸泰召王思政於玉壁將使鎮虎牢未至而泰敗乃使守恒農|{
	恒戶登翻}
思政入城令開門解衣而卧慰勉將士示不足畏後數日劉豐生至城下憚之不敢進引軍還思政乃脩城郭起樓櫓營農田積芻粟由是恒農始有守禦之備丞相泰求自貶魏主不許是役也魏諸將皆無功唯耿令貴與太子武衛率王胡仁|{
	魏盖改東宫武衛將軍為武衛率}
都督王文達力戰功多泰欲以雍岐北雍三州授之以州有優劣使探籌取之|{
	雍於用翻探吐南翻}
仍賜胡仁名勇令貴名豪文達名傑用彰其功於是廣募關隴豪右以增軍旅高仲密之將叛也隂遣人扇動冀州豪傑使為内應|{
	高乾兄弟本起兵於信都仲密故扇動其豪傑使為應於河北}
東魏遣高隆之馳驛慰撫由是得安高澄密書與隆之曰仲密枝黨與之俱西者宜悉收其家屬以懲將來隆之以為恩旨既行理無追改若復收治|{
	復扶又翻治直之翻}
示民不信脱使驚擾所虧不細乃啟丞相歡而罷之以太子詹事謝舉為尚書僕射 夏四月林邑王攻李賁賁將范修破林邑於九德|{
	吴分九真立九德郡五代志曰日南郡九德縣梁置德州將即亮翻}
清水氐酋李鼠仁|{
	清水縣漢屬天水郡晉屬畧陽郡五代志曰後魏置清水郡隋廢郡為縣屬秦州酋慈秋翻}
乘魏之敗據險作亂隴右大都督獨孤信屢遣軍擊之不克丞相泰遣典籖天水趙昶往諭之諸酋長聚議|{
	長知兩翻}
或從或否其不從者欲加刃於昶昶神色自若辭氣逾厲鼠仁感悟遂相帥降|{
	帥讀曰率降戶江翻}
氐酋梁道顯叛泰復遣昶諭降之|{
	酋慈秋翻復扶又翻下州復同}
徙其豪帥四千餘人并部落於華州|{
	帥所類翻華戶化翻}
泰即以昶為都督使領之 泰使諜潜入虎牢令守將魏光固守|{
	諜徒協翻將即亮翻}
侯景獲之改其書云宜速去縱諜入城光宵遁景獲高仲密妻子送鄴北豫洛二州復入於東魏五月壬辰東魏以克復虎牢降死罪以下囚唯不赦高仲密家丞相歡以高乾有義勲|{
	謂起兵於信都以奉歡也}
高昂死王事|{
	謂戰死于河陽也}
季式先自告|{
	謂先自永安戍奔告歡也}
皆為之請免其從坐|{
	為于偽翻}
仲密妻李氏當死高澄盛服見之曰今日何如李氏默然遂納之|{
	高澄以漁色使宗勲外叛其父幾死于兵長惡不悛衒服以誘納之他日楊燕之禍叔侄相屠釁由李氏豈天也邪}
乙未以侯景為司空|{
	賞平虎牢之功也}
秋七月魏大赦以王盟為太傅廣平王贊為司空 八月乙丑東魏以汾州刺史斛律金為大司馬東魏遣兼散騎常侍李渾等來聘 冬十一月甲午東魏主狩于西山|{
	鄴西無山盖邯鄲之西山也}
乙巳還宫高澄啟解侍中東魏主以其弟并州刺史太原公洋代之丞相歡築長城於肆州北山西自馬陵東至土墱|{
	馬陵盖東魏置戍之地九域志代州崞縣有土墱寨墱北史作隥音士鄧翻}
四十日罷 魏諸牧守共謁丞相泰泰命河北太守裴侠别立|{
	魏收志河北郡屬陕州本漢晉河東郡之河北縣地也隋廢郡復為縣屬河東郡守式又翻侠戶頰翻}
謂諸牧守曰裴俠清慎奉公為天下最有如侠者可與俱立衆默然無敢應者泰乃厚賜侠朝野歎服號為獨立君|{
	朝直遥翻下同}


十年春正月李賁自稱越帝置百官改元大德 三月癸巳東魏丞相歡巡行冀定二州|{
	行下孟翻}
校河北戶口損益因朝于鄴 甲午上幸蘭陵謁建寧陵使太子入守京城辛丑謁修陵|{
	晉置南東海郡於京口建蘭陵郡於延陵建寧陵梁紀曰建陵皇妣張皇后陵也修陵皇后郗氏陵也}
丙午東魏以開府儀同三司孫騰為太保 己酉上幸京口城北固樓更名北顧|{
	鎮江府圖經曰京口城因山為壘緣江為境爾雅丘絶高曰京故曰京口又府治東五里有京峴山京口得名以此北固山在府北一里迴嶺下臨長江即府治所據及甘泉寺基蕭正義傳曰京城之西有别嶺入江高數十丈三面臨水號曰北固蔡謨起樓其上以置軍實帝登望久之曰此嶺下足須固守然於京口實乃壮觀於是改曰北顧}
庚戌幸囘賓亭宴鄉里故老及所經近縣迎候者少長數千人|{
	少詩照翻長知兩翻}
各賚錢二千 壬子東魏以高澄為大將軍領中書監元弼為録尚書事左僕射司馬子如為尚書令侍中高洋為左僕射丞相歡多在晉陽孫騰司馬子如高岳高隆之皆歡之親黨也委以朝政|{
	朝直遥翻}
鄴中謂之四貴其權勢熏灼中外率多專恣驕貪歡欲損奪其權故以澄為大將軍領中書監移門下機事總歸中書|{
	門下省衆事侍中給事中等掌之今高歡移而總歸中書所以重澄之權}
文武賞罰皆禀於澄孫騰見澄不肯盡敬澄叱左右牽下於牀築以刀環立之門外太原公洋於澄前拜高隆之呼為叔父澄怒罵之|{
	隆之本洛隆人歡命為弟故洋以叔父呼之}
歡謂羣公曰兒子浸長|{
	長知兩翻}
公宜避之於是公卿以下見澄無不聳懼庫狄干澄姑之婿也|{
	干娶歡妹}
自定州來謁立於門外三日乃得見澄欲置腹心於東魏主左右擢中兵參軍崔季舒為中書侍郎澄每進書於帝有所諫請或文辭繁雜季舒輒脩飾通之帝報澄父子之語常與季舒論之曰崔中書我乳母也季舒挺之從子也|{
	從才用翻}
夏四月乙卯上還自蘭陵 五月甲申朔魏丞相泰朝於長安|{
	朝直遥翻}
甲午東魏遣散騎常侍魏季景來聘季景收之族叔也尚書令何敬容妾弟盗官米以書屬領軍河東王譽

|{
	屬之欲翻}
丁酉敬容坐免官 東魏廣陽王湛卒 魏琅琊貞獻公賀拔勝諸子在東者丞相歡盡殺之|{
	修邙山之怨也}
勝憤恨發疾而卒丞相泰常謂人曰諸將對敵神色皆動唯賀拔公臨陳如平時|{
	陳讀曰陣}
真大勇也 秋七月魏更權衡度量|{
	更工衡翻}
命尚書蘇綽損益三十六條之制總為五卷頒行之|{
	二十四條并新制十二條總為三十六條二十四條見上卷大同元年}
搜簡賢才為牧守令長皆依新制而遣焉|{
	守手又翻長知兩翻}
數年之間百姓便之 魏自正光以後政刑弛縱在位多貪汙丞相歡啟以司州從事宋遊道為御史中尉|{
	五代志後齊司州置牧屬官有别駕從事史治中從事史}
澄固請以吏部郎崔暹為之以遊道為尚書左丞澄謂暹遊道曰卿一人處南臺一人處北省|{
	處曷呂翻御史臺謂之南臺尚書省謂之北省杜佑曰御史臺在宫闕西南故名南臺尚書省在北故曰北省}
當使天下肅然暹選畢義雲等為御史時稱得人義雲衆敬之曾孫也|{
	宋明帝初畢衆敬降魏}
澄欲假暹威勢諸公在坐|{
	坐徂卧翻}
令暹後至通名高視徐步兩人挈裙而入澄分庭對揖暹不讓而坐觴再行即辭去澄留之食暹曰適受敇在臺檢校遂不待食而去澄降階送之他日澄與諸公出之東山|{
	時於鄴都治東山為遊晏之地}
遇暹于道前驅為赤棒所擊澄回馬避之尚書令司馬子如以丞相歡故人當重任意氣自高與太師咸陽王坦黷貨無厭|{
	厭於塩翻}
暹前後彈子如坦及并州刺史可朱渾道元等罪狀無不極筆宋遊道亦劾子如坦及太保孫騰司徒高隆之司空侯景尚書元羨等|{
	劾戶盖翻又戶得翻}
澄收子如繫獄一宿髮盡白辭曰司馬子如從夏州策杖投相王|{
	中大通四年歡破爾朱氏召子如於南岐州盖雍華路阻取道夏州東歸也夏戶雅翻}
王給露車一乘|{
	乘繩證翻}
觠牸牛犢|{
	觠巨員翻曲角也牸音字}
犢在道死唯觠角存此外皆取之於人丞相歡以書敇澄曰司馬令吾之故舊汝宜寛之澄駐馬行街出子如脱其鎻子如懼曰非作事邪|{
	懼澄殺之也}
八月癸酉削子如官爵九月甲申以濟隂王暉業為太尉|{
	濟子禮翻}
太師咸陽王坦以王還第|{
	罷太師也}
元羨等皆免官其餘死黜者甚衆久之歡見子如哀其憔悴|{
	憔慈消翻悴秦醉翻}
以膝承其首親為擇蝨|{
	為于偽翻蝨色節翻}
賜酒百缾羊五百口米五百石|{
	澄繩之以公法歡接之以舊恩此其父子駕御勲貴之術也}
高澄對諸貴極言褒美崔暹且戒屬之|{
	屬之欲翻}
丞相歡書與鄴下諸貴曰崔暹居憲臺咸陽王司馬令皆吾布衣之舊尊貴親暱無過二人|{
	暱尼質翻過工未翻}
同時獲罪吾不能救諸君其慎之宋遊道奏駁尚書違失數百條省中豪吏王儒之徒並鞭斥之令僕已下皆側目|{
	駁北角翻}
高隆之誣遊道有不臣之言罪當死給事黄門侍郎楊愔曰畜狗求吠|{
	愔於今翻畜許竹翻養也}
今以數吠殺之恐將來無復吠狗|{
	復扶又翻}
遊道竟坐除名澄謂遊道曰卿早從我向并州不爾彼經畧殺卿遊道從澄至晉陽以為大行臺吏部|{
	部下當有郎字}
己丑大赦 東魏以喪亂之後戶口失實徭賦不均|{
	喪息浪翻}
冬十月丁巳以太保孫騰大司徒高隆之|{
	魏齊官制司徒未嘗加大大字衍}
為括戶大使分行諸州|{
	使疏吏翻行下孟翻}
得無籍之戶六十餘萬僑居者皆勒還本屬十一月甲申以高隆之録尚書事以前大司馬婁昭為司徒 庚子東魏主祀圜丘東魏丞相歡襲擊山胡破之|{
	山胡即汾州山中稽胡也}
俘萬餘戶分配諸州 是歲東魏以散騎常侍魏收兼中書侍郎修國史自梁魏通好|{
	好呼到翻}
魏書每云想彼境内寧静此率土安和上復書去彼字而已收始定書云想境内清晏今萬里安和上亦效之

資治通鑑卷一百五十八
