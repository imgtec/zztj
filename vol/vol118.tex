<!DOCTYPE html PUBLIC "-//W3C//DTD XHTML 1.0 Transitional//EN" "http://www.w3.org/TR/xhtml1/DTD/xhtml1-transitional.dtd">
<html xmlns="http://www.w3.org/1999/xhtml">
<head>
<meta http-equiv="Content-Type" content="text/html; charset=utf-8" />
<meta http-equiv="X-UA-Compatible" content="IE=Edge,chrome=1">
<title>資治通鑒_119-資治通鑑卷一百十八_119-資治通鑑卷一百十八</title>
<meta name="Keywords" content="資治通鑒_119-資治通鑑卷一百十八_119-資治通鑑卷一百十八">
<meta name="Description" content="資治通鑒_119-資治通鑑卷一百十八_119-資治通鑑卷一百十八">
<meta http-equiv="Cache-Control" content="no-transform" />
<meta http-equiv="Cache-Control" content="no-siteapp" />
<link href="/img/style.css" rel="stylesheet" type="text/css" />
<script src="/img/m.js?2020"></script> 
</head>
<body>
 <div class="ClassNavi">
<a  href="/24shi/">二十四史</a> | <a href="/SiKuQuanShu/">四库全书</a> | <a href="http://www.guoxuedashi.com/gjtsjc/"><font  color="#FF0000">古今图书集成</font></a> | <a href="/renwu/">历史人物</a> | <a href="/ShuoWenJieZi/"><font  color="#FF0000">说文解字</a></font> | <a href="/chengyu/">成语词典</a> | <a  target="_blank"  href="http://www.guoxuedashi.com/jgwhj/"><font  color="#FF0000">甲骨文合集</font></a> | <a href="/yzjwjc/"><font  color="#FF0000">殷周金文集成</font></a> | <a href="/xiangxingzi/"><font color="#0000FF">象形字典</font></a> | <a href="/13jing/"><font  color="#FF0000">十三经索引</font></a> | <a href="/zixing/"><font  color="#FF0000">字体转换器</font></a> | <a href="/zidian/xz/"><font color="#0000FF">篆书识别</font></a> | <a href="/jinfanyi/">近义反义词</a> | <a href="/duilian/">对联大全</a> | <a href="/jiapu/"><font  color="#0000FF">家谱族谱查询</font></a> | <a href="http://www.guoxuemi.com/hafo/" target="_blank" ><font color="#FF0000">哈佛古籍</font></a> 
</div>

 <!-- 头部导航开始 -->
<div class="w1180 head clearfix">
  <div class="head_logo l"><a title="国学大师官网" href="http://www.guoxuedashi.com" target="_blank"></a></div>
  <div class="head_sr l">
  <div id="head1">
  
  <a href="http://www.guoxuedashi.com/zidian/bujian/" target="_blank" ><img src="http://www.guoxuedashi.com/img/top1.gif" width="88" height="60" border="0" title="部件查字,支持20万汉字"></a>


<a href="http://www.guoxuedashi.com/help/yingpan.php" target="_blank"><img src="http://www.guoxuedashi.com/img/top230.gif" width="600" height="62" border="0" ></a>


  </div>
  <div id="head3"><a href="javascript:" onClick="javascript:window.external.AddFavorite(window.location.href,document.title);">添加收藏</a>
  <br><a href="/help/setie.php">搜索引擎</a>
  <br><a href="/help/zanzhu.php">赞助本站</a></div>
  <div id="head2">
 <a href="http://www.guoxuemi.com/" target="_blank"><img src="http://www.guoxuedashi.com/img/guoxuemi.gif" width="95" height="62" border="0" style="margin-left:2px;" title="国学迷"></a>
  

  </div>
</div>
  <div class="clear"></div>
  <div class="head_nav">
  <p><a href="/">首页</a> | <a href="/ShuKu/">国学书库</a> | <a href="/guji/">影印古籍</a> | <a href="/shici/">诗词宝典</a> | <a   href="/SiKuQuanShu/gxjx.php">精选</a> <b>|</b> <a href="/zidian/">汉语字典</a> | <a href="/hydcd/">汉语词典</a> | <a href="http://www.guoxuedashi.com/zidian/bujian/"><font  color="#CC0066">部件查字</font></a> | <a href="http://www.sfds.cn/"><font  color="#CC0066">书法大师</font></a> | <a href="/jgwhj/">甲骨文</a> <b>|</b> <a href="/b/4/"><font  color="#CC0066">解密</font></a> | <a href="/renwu/">历史人物</a> | <a href="/diangu/">历史典故</a> | <a href="/xingshi/">姓氏</a> | <a href="/minzu/">民族</a> <b>|</b> <a href="/mz/"><font  color="#CC0066">世界名著</font></a> | <a href="/download/">软件下载</a>
</p>
<p><a href="/b/"><font  color="#CC0066">历史</font></a> | <a href="http://skqs.guoxuedashi.com/" target="_blank">四库全书</a> |  <a href="http://www.guoxuedashi.com/search/" target="_blank"><font  color="#CC0066">全文检索</font></a> | <a href="http://www.guoxuedashi.com/shumu/">古籍书目</a> | <a   href="/24shi/">正史</a> <b>|</b> <a href="/chengyu/">成语词典</a> | <a href="/kangxi/" title="康熙字典">康熙字典</a> | <a href="/ShuoWenJieZi/">说文解字</a> | <a href="/zixing/yanbian/">字形演变</a> | <a href="/yzjwjc/">金 文</a> <b>|</b>  <a href="/shijian/nian-hao/">年号</a> | <a href="/diming/">历史地名</a> | <a href="/shijian/">历史事件</a> | <a href="/guanzhi/">官职</a> | <a href="/lishi/">知识</a> <b>|</b> <a href="/zhongyi/">中医中药</a> | <a href="http://www.guoxuedashi.com/forum/">留言反馈</a>
</p>
  </div>
</div>
<!-- 头部导航END --> 
<!-- 内容区开始 --> 
<div class="w1180 clearfix">
  <div class="info l">
   
<div class="clearfix" style="background:#f5faff;">
<script src='http://www.guoxuedashi.com/img/headersou.js'></script>

</div>
  <div class="info_tree"><a href="http://www.guoxuedashi.com">首页</a> > <a href="/SiKuQuanShu/fanti/">四库全书</a>
 > <h1>资治通鉴</h1> <!--         下载:【右键另存为】即可 --></div>
  <div class="info_content zj clearfix">
  
<div class="info_txt clearfix" id="show">
<center style="font-size:24px;">119-資治通鑑卷一百十八</center>
    資治通鑑卷一百十八<br />
<br />
  宋 司馬光 撰<br />
<br />
  胡三省 音注<br />
<br />
  晉紀四十【起強圉大荒落盡屠維協洽凡三年】<br />
<br />
  安皇帝癸<br />
<br />
  義熙十三年春正月甲戌朔日有食之 秦主泓朝會百官於前殿【朝直遙翻】以内外危迫君臣相泣【以内則兄弟搆難外為晉夏所迫也】征北將軍齊公恢帥安定鎮戶三萬八千焚廬舍自北雍州趨長安【秦分嶺北五郡為北雍州鎮安定泓不用東平公紹懿横之言以召亂帥讀曰率雍於用翻趨七喻翻】自稱大都督建義大將軍移檄州郡欲除君側之惡揚威將軍姜紀帥衆歸之建節將軍彭完都棄隂密犇還長安恢至新支姜紀說恢曰國家重將大兵皆在東方京師空虚公亟引輕兵襲之必克恢不從南攻郿城鎮西將軍姚諶為恢所敗【說輸芮翻敗補邁翻將即亮翻郿音眉又音媚諶氏壬翻姚諶去年棄雍東奔遂屯于郿】長安大震泓馳使徵東平公紹遣姚裕及輔國將軍胡翼度屯灃西【關中無灃水澧當作灃灃水出鄠南灃谷北過上林苑入渭使疏吏翻】扶風太守姚儁等皆降於恢【降戶江翻】東平公紹引諸軍西還與恢相持於靈臺【水經注漢靈臺在秦阿房宫南鎬水逕其北】姚讚留寧朔將軍尹雅為弘農太守守潼關【太守式又翻】亦引兵還恢衆見諸軍四集皆有懼心其將齊黄等詣大軍降恢進兵逼紹讚自後擊之恢兵大敗殺恢及其三弟泓哭之慟葬以公禮 太尉裕引水軍發鼔城留其子彭城公義隆鎮彭城詔以義隆為監徐兖青冀四州諸軍事徐州刺史【監工衘翻】 涼公暠寢疾【暠古老翻】遺命長史宋繇曰吾死之後世子猶卿子也善訓導之二月暠卒【卒子恤翻下將卒同】官屬奉世子歆為大都督大將軍涼公領涼州牧大赦改元嘉興尊歆母天水尹氏為太后以宋繇錄三府事【三府大都督大將軍府凉公府州牧府也】諡暠曰武昭王廟號太祖 西秦安東將軍木弈于擊吐谷渾樹洛干破其弟阿柴於堯杆川【堯杆川在塞外杆居寒翻又居案翻】俘五千餘口而還【還而宣翻又如字】樹洛干走保白蘭山慙憤發疾將卒謂阿柴曰吾子拾䖍幼弱今以大事付汝樹洛干卒【卒子恤翻】阿柴立自稱驃騎將軍沙州刺史【驃匹妙翻騎奇寄翻】諡樹洛干曰武王阿柴稍用兵侵併其傍小種【種章勇翻】地方數千里遂為疆國 河西王蒙遜遣其將襲烏啼部大破之【烏啼虜居張掖刪丹縣金山之西將即亮翻】又撃卑和部降之【卑和羌居西海降戶江翻】 王鎭惡進軍澠池【澠彌兖翻】遣毛德祖襲尹雅於蠡吾城禽之【秦以雅為弘農太守屯蠡吾城據載記蠡吾城當在宜陽之西宋白曰蠡吾城後魏初猶屬弘農唐以來為澠池縣理所余按蠡吾自是漢清河國界城名此乃蠡城非蠡吾城也通鑑蓋承晉書之誤】雅殺守者而逃鎮惡引兵徑前扺潼關檀道濟沈林子自陜北渡河拔襄邑堡【陜式冉翻】秦河北太守薛帛犇河東【襄邑堡在河北郡河北縣漢晉屬河東郡秦分立河北郡】又攻秦并州刺史尹昭於蒲阪不克【阪音反】别將攻匈奴堡為姚成都所敗【將即亮翻下同敗蒲邁翻】辛酉滎陽守將傅洪以虎牢降魏秦主泓以東平公紹為太宰大將軍都督中外諸軍事假黄鉞改封魯公使督武衛將軍姚鸞等步騎五萬守潼關又遣别將姚驢救蒲阪沈林子謂檀道濟曰蒲阪城堅兵多不可猝拔攻之傷衆守之引日王鎮惡在潼關勢孤力弱不如與鎮惡合勢并力以争潼關若得之尹昭不攻自潰矣道濟從之三月道濟林子至潼關秦魯公紹引兵出戰道濟林子奮擊大破之斬獲以千數紹退屯定城【郭緣生述征記曰定城去潼關三十里夾道各一城渭水逕其北】據險拒守謂諸將曰道濟等兵力不多懸軍深入不過堅壁以待繼援吾分軍絶其糧道可坐禽也乃遣姚鸞屯大路以絶道濟糧道【自澠池西入關有兩路南路由囘谿阪自漢以前皆由之曹公惡南路之險更開北路遂以北路為大路載記曰紹留鸞守險以絶道濟糧道蓋鸞雖屯大路亦據險而邀絶糧道也紹初遣胡翼度據東原蓋與大路相為唇齒所謂據險也及沈林子襲鸞營翼度不能救何也人心危駭面面受敵故也】鸞遣尹雅將兵與晉戰於關南【關南潼關之南也】為晉兵所獲將殺之雅曰雅前日已當死幸得脱至今死固甘心然夷夏雖殊【夏戶雅翻】君臣之義一也晉以大義行師獨不使秦有守節之臣乎乃免之丙子夜沈林子將鋭卒襲鸞營斬鸞殺其士卒數千人紹又遣東平公讚屯河上以斷水道【斷丁管翻下兵斷同】沈林子擊之讚敗走還定城薛帛據河曲來降【河水自蒲阪南至潼關激而東流蒲阪河北之間謂之河曲】太尉裕將水軍自淮泗入清河將泝河西上【上時掌翻下必上北上同】先遣使假道於魏【使疏吏翻】秦主泓亦遣使請救於魏魏主嗣使羣臣議之皆曰潼關天險劉裕以水軍攻之甚難若登岸北侵其勢便易【易以䜴翻】裕聲言伐秦其志難測且秦婚姻之國不可不救也【秦女歸魏見上卷十一年】宜發兵斷河上流勿使得西博士祭酒崔浩曰裕圖秦久矣今姚興死子泓懦劣國多内難【難乃旦翻】裕乘其危而伐之其志必取若遏其上流裕心忿戾必上岸北侵是我代秦受敵也今柔然宼邊民食又乏若復與裕為敵發兵南赴則北宼愈深救北則南州復危【南州謂魏之南境相州瀕河諸郡復扶又翻】非良計也不若假之水道聽裕西上然後屯兵以塞其東【塞悉則翻】使裕克捷必德我之假道不捷吾不失救秦之名此策之得者也且南北異俗借使國家棄恒山以南【恒戶登翻】裕必不能以吳越之兵與吾争守河北之地安能為吾患乎夫為國計者惟社稷是利豈顧一女子乎議者猶曰裕西入關則恐吾斷其後腹背受敵北上則姚氏必不出關助我其勢必聲西而實北也嗣乃以司徒長孫暠督山東諸軍事【長知兩翻】又遣振威將軍娥清【孫愐泓娥姓也】冀州刺史阿薄干【魏書官氏志内入諸姓阿伏干氏後為阿氏】將步騎十萬屯河北岸庚辰裕引軍入河以左將軍向彌為北青州刺史留戍碻磝【晉氏南渡僑置青州於江北裕平廣固置北青州於東陽而江北之青州如故今向彌以北青州刺史戍碻磝東陽之青州亦如故向式亮翻】初裕命王鎮惡等若克洛陽須大軍到俱進鎮惡等乘利徑趨潼關【趨七喻翻】為秦兵所拒不得前久之乏食衆心疑懼或欲棄輜重還赴大軍【重直用翻】沈林子按劒怒曰相公志清六合今許洛已定關右將平事之濟否繫於前鋒奈何沮乘勝之氣棄垂成之功乎【沮在呂翻】且大軍尚遠賊衆方盛雖欲求還豈可得乎下官授命不顧【論語子張曰士見危授命】今日之事當自為將軍辦之【為于偽翻】未知二三君子將何面以見相公之旗鼔邪【相公謂裕也】鎮惡等遣使馳告裕求遣糧援裕呼使者開舫北戶【使疏吏翻舫甫妄翻方舟也大舟也】指河上魏軍以示之曰我語令勿進今輕佻深入【語牛倨翻佻他雕翻】岸上如此何由得遣軍鎮惡乃親至弘農說諭百姓【說輸芮翻】百姓競送義租軍食復振【復扶又翻下則復子復同】魏人以數千騎緣河隨裕軍西行軍人於南岸牽百丈【百丈者所以挽船今南人用麻䋲北人以竹為之陸游曰蜀人百丈以巨竹四破為之大如人臂】風水迅急有漂渡北岸者【漂匹招翻】輒為魏人所殺略裕遣軍擊之裁登岸則走退則復來夏四月裕遣白直隊主丁旿【裕選白丁之壯勇者入直左古使旿領之杜佑曰白直無月紿之數旿阮古翻】帥仗士七百人車百乘【帥讀曰率乘繩證翻】渡北岸去水百餘步為却月陣兩端抱河車置七仗士事畢使竪一白毦【竪上主翻說文曰竪立也毦仍吏翻績羽為之】魏人不解其意【解戶買翻曉也】皆未動裕先命寧朔將軍朱超石戒嚴白毦既舉超石帥二千人馳往赴之齎大弩百張一車益二十人設彭排於轅上魏人見營陣既立乃進圍之長孫嵩帥三萬騎助之四面肉薄攻營【薄普各翻肉薄者以身薄營血戰】弩不能制時超石别齎大鎚及矟千餘張【鎚傳為翻】乃斷矟長三四尺以鎚鎚之一矟輒洞貫三四人魏兵不能當一時犇潰死者相積臨陳斬阿薄干魏人退還畔城【斷丁管翻長直亮翻陳與陣同魏收地形志平原郡聊城縣有畔城】超石帥寧朔將軍胡藩寧遠將軍劉榮祖追擊又破之殺獲千計魏主嗣聞之乃恨不用崔浩之言秦魯公紹遣長史姚洽寧朔將軍安鸞護軍姚墨蠡【蠡魯戈翻】河東太守唐小方帥衆三千屯河北之九原阻河為固欲以絶檀道濟糧援【載記曰紹欲以絶弘農諸縣糧援】沈林子邀擊破之斬洽墨蠡小方殺獲殆盡林子因啓太尉裕曰紹氣蓋關中今兵屈於外國危於内恐其凶命先盡不得以膏齊斧耳【齊讀曰資應劭曰齊利也張晏曰齊如字征伐斧也以整齊天下也一說齊作齋凡師出入齊戒入廟而受斧鉞也】紹聞洽等敗死憤恚發病嘔血以兵屬東平公讚而卒【恚於避翻屬之欲翻卒子恤翻】讚既代紹衆力猶盛引兵襲林子林子復擊破之【復扶又翻】太尉裕至洛陽行視城塹【行下孟翻】嘉毛脩之完葺之功賜衣服玩好直二千萬【好呼到翻】 丁巳魏主嗣如高柳壬戌還平城河西王蒙遜大赦遣張掖太守沮渠廣宗詐降以誘<br />
<br />
  涼公歆【沮子余翻降戶江翻下同】歆發兵應之蒙遜將兵三萬伏於蓼泉【新唐書地理志甘州張掖郡西北百九十里有祁連山山北有建康軍軍西百二十里有蓼泉守捉城】歆覺之引兵還蒙遜追之歆與戰於解支澗【解支澗晉書作鮮支澗當從之】大破之斬首七千餘級蒙遜城建康置戍而還五月乙未齊郡太守王懿降於魏上書言劉裕在洛<br />
<br />
  宜發兵絶其歸路可不戰而克魏主嗣善之崔浩侍講在前嗣問之曰劉裕伐姚泓果能克乎對曰克之嗣曰何故對曰昔姚興好事虛名而少實用子泓懦而多病兄弟乖争【好呼到翻少詩紹翻謂弼懿恢皆與泓争國】裕乘其危兵精將勇何故不克【將即亮翻】嗣曰裕才何如慕容垂對曰勝之垂藉父兄之資修復舊業國人歸之若夜蟲之就火少加倚仗易以立功【少詩沼翻易以豉翻】劉裕奮起寒微不階尺土討滅桓玄興復晉室【事見一百一十三卷元興三年】北禽慕容超【事見一百一十五卷五年六年】南梟盧循【事見六年七年梟堅堯翻】所向無前非其才之過人安能如是乎嗣曰裕既入關不能進退我以精騎直擣彭城夀春【騎奇寄翻】裕將若之何對曰今西有屈丏【北史曰明元改赫連勃勃名曰屈丏北方言屈丏者卑下也】北有柔然窺伺國隙【伺相吏翻】陛下既不可親御六師雖有精兵未睹良將【將即亮翻】長孫嵩長於治國短於用兵非劉裕敵也【治直之翻下同】興兵遠攻未見其利不如且安静以待之【凡兵之動知敵之主知敵之將此之謂也】裕克秦而歸必簒其主關中華戎雜錯風俗勁悍【悍侯旰翻又下罕翻】裕欲以荆揚之化施之函秦此無異解衣包火張羅捕虎雖留兵守之人情未洽趨尚不同適足為寇敵之資耳【赫連之得關中崔浩固料之矣】願陛下按兵息民以觀其變秦地終為國家之有可坐而守也嗣笑曰卿料之審矣浩曰臣嘗私論近世將相之臣若王猛之治國苻堅之管仲也【治直之翻】慕容恪之輔幼主慕容暐之霍光也劉裕之平禍亂司馬德宗之曹操也嗣曰屈丏何如浩曰屈丏國破家覆孤孑一身【孑居列翻單也】寄食姚氏受其封殖不思醻恩報義而乘時徼利盜有一方【事見一百一十四卷三年徼一遙翻】結怨四鄰【謂與魏秦涼搆怨也】撅竪小人【撅與掘同其月翻撅竪言撅起自竪立也】雖能縱暴一時終當為人所吞食耳嗣大悦語至夜半賜浩御縹醪十觚【縹匹紹翻青白色曰縹醅酒曰醪觚飲器受三升此魏主所自御者故曰御縹醪】水精鹽一兩【鹽透明如水精故謂之水精鹽】曰朕味卿言如此鹽酒故欲與卿共饗其美然猶命長孫嵩叔孫建各簡精兵伺裕西過自成臯濟河南侵彭沛若不時過則引兵隨之【彭沛謂彭城沛郡也】 魏主嗣西巡至雲中遂濟河畋于大漠 魏置天地四方六部大人以諸公為之【諸公謂時居公位及位從公者】 秋七月太尉裕至陜【陜式冉翻】沈田子傅弘之入武關秦戍將皆委城走【將即亮翻】田子等進屯青泥秦主泓使給事黄門侍郎姚和都屯嶢柳以拒之【嶢音堯】 西秦相國翟勍卒【勍渠京翻】八月以尚書令曇達為左丞相左僕射元基為右丞相御史大夫麴景為尚書令侍中翟紹為左僕射【翟勍既卒曇達皆序遷通鑑即西秦舊史書之曇徒含翻】 太尉裕至閺鄉【閺音旻】沈田子等將攻嶢柳秦主泓欲自將以禦裕軍【將即亮翻】恐田子等襲其後欲先擊滅田子等然後傾國東出乃帥步騎數萬奄至青泥【帥讀曰率騎奇寄翻】田子本為疑兵所領裁千餘人聞泓至欲擊之傅弘之以衆寡不敵止之田子曰兵貴用奇不必在衆且今衆寡相懸勢不兩立若彼結圍既固則我無所逃矣不如乘其始至營陳未立先薄之可以有功遂帥所領先進弘之繼之秦兵合圍數重【陳讀曰陣重直龍翻】田子撫慰士卒曰諸君冒險遠來正求今日之戰死生一决封侯之業於此在矣士卒皆踴躍鼔譟執短兵奮擊秦兵大敗【沈田子以千餘人敗姚泓數萬之衆者置兵死地人自為戰也】斬馘萬餘級得其乘輿服御物【乘繩證翻】秦主泓犇還灞上初裕以田子等衆少【少詩沼翻】遣沈林子將兵自秦嶺往助之【秦嶺在長安南班固西都賦所謂前乘秦嶺自此出藍田關裕蓋遣林子自陽華循山西南至秦嶺】至則秦兵已敗乃相與追之關中郡縣多潛送欵於田子辛丑太尉裕至潼關以朱超石為河東太守使與振武將軍徐猗之會薛帛於河北共攻蒲阪秦平原公璞與姚和都共擊之【姚和都蓋青泥既敗而犇蒲阪也或曰和都當作成都】猗之敗死超石犇還潼關東平公讚遣司馬國璠引魏兵以躡裕後【璠孚袁翻躡尼輒翻】王鎮惡請帥水軍自河入渭以趨長安【帥讀曰率水經河水歷船司空與渭水會春秋之渭汭即其地也趨七喻翻】裕許之秦恢武將軍姚難自香城引兵而西【香城在渭水之北蒲津之口恢武將軍蓋姚秦創置】鎮惡追之秦主泓自灞上引兵還屯石橋以為之援【石橋在長安城洛門東北有石橋水經註曰石橋水南出馬嶺山積石據其東驪山距其西其水北逕鄭城西水上有橋東去鄭城十里故世以橋名水三輔黄圖曰洛門長安城北出東頭第一門】鎮北將軍姚彊與難合兵屯涇上以拒鎮惡【涇水出安定涇陽縣开頭山東南至陽陵入渭此涇上在漢京兆陽陵界】鎮惡使毛德祖進擊破之彊死難犇長安東平公讚退屯鄭城太尉裕進軍逼之泓使姚丕守渭橋胡翼度屯石積東平公讚屯灞東泓屯逍遙園【水經註沈水上承樊川皇子陂北逕長安城西與昆明池水合其枝津東北流逕鄧艾祠南又東分為二水一水入逍遙園】鎮惡沂渭而上【上時掌翻】乘蒙衝小艦行船者皆在艦内秦人見艦進而無行船者皆驚以為神【艦戶黯翻】壬戌旦鎮惡至渭橋令軍士食畢皆持仗登岸後登者斬衆既登渭水迅急艦皆隨流倏忽不知所在時泓所將尚數萬人【將即亮翻】鎮惡諭士卒曰吾屬並家在江南此為長安北門去家萬里舟楫衣糧皆已隨流今進戰而勝則功名俱顯不勝則骸骨不返無他岐矣【岐旁出之道】卿等勉之乃身先士卒【先悉薦翻】衆騰踴爭進大破姚丕於渭橋泓引兵救之為丕敗卒所蹂踐不戰而潰諶等皆死【蹂人九翻踐慈演翻諶氏壬翻】泓單馬還宫鎮惡入自平朔門【漢無平朔門蓋長安城北門也後人改其名耳】泓與姚裕等數百騎逃犇石橋東平公讚聞泓敗引兵赴之衆皆潰去胡翼度降于太尉裕泓將出降其子佛念年十一言於泓曰晉人將逞其欲雖降必不免不如引決【降戶江翻引決謂自裁也】泓憮然不應【憮音武悵也失意貌】佛念登宫牆自投而死【姚佛念雖不及劉諶然以童稚之年氣烈如此亦可尚也】癸亥泓將妻子羣臣詣鎮惡壘門請降鎮惡以屬吏【屬之欲翻】城中夷晉六萬餘戶鎮惡以國恩撫慰號令嚴肅百姓安堵九月太尉裕至長安鎮惡迎於灞上裕勞之曰成吾霸業者卿也鎮惡再拜謝曰明公之威諸將之力鎮惡何功之有裕笑曰卿欲學馮異耶【謂馮異謙退不伐而能定關中】鎮惡性貪秦府庫盈積鎮惡盜取不可勝紀【勝音升】裕以其功大不問或譛諸裕曰鎮惡藏姚泓偽輦將有異志裕使人覘之【覘丑廉翻又丑艶翻】鎮惡剔取其金銀棄輦於垣側裕意乃安裕收秦彞器渾儀土圭記里鼓指南車送詣建康【左傳祝佗曰成王分魯公以官司彞器杜預註彞器常用之器漢武帝時洛下閡鮮于妄人耿壽昌造員儀以考歷度和帝時賈逵又加黄道順帝時張衡又制渾象具内外規黄赤道南北極列二十四氣二十八宿中外星官及日月五緯以漏水轉之於殿上室内星中出没與天相應其後吳陵績造渾象王蕃制渾儀舊渾象以二分為一度凡周七尺三寸半分張衡更制以四分為一度凡周一丈四尺六寸王蕃以古制局小星辰稠穊衡器傷大難可轉移更制渾象以三分為一度凡周天一丈九寸五分分之三周禮大司徒以土圭之法測土深正日景以求地中日南則景短多暑日北則景長多寒日東則景夕多風日西則景朝多隂日至之景尺有五寸謂之地中註云土圭所以致四時日月之景也鄭司農云測土深謂南北東西之深也日南立表處太南近日也日北謂立表處太北遠日也景夕謂日昳景乃中立表處太東近日也景朝謂日未中而景中立表處太西遠日也玄謂晝漏半而置土圭表隂陽審其南北景短於土圭謂之日南是地於日為近南也景長於土圭謂之日北是地於日為近北也東於土圭謂之日東是地於日為近東也西於土圭謂之日西是地於日為近西也如是則寒暑隂風偏而不和是未得其所求凡日景於地千里而差一寸鄭司農又云土圭之長尺有五寸以夏至之日立八尺之表其景適與土圭等謂之地中今潁川陽城地為然晉輿服志記里鼔車駕四馬制如司南車崔豹古今註曰大章車所以識道里也起於西京亦曰記里車車上有二層皆有木人行一里下層撃鼓行十里上層擊鐲黄帝作指南車晉輿服志司南車一名指南車駕四馬其下制如樓三級四角金龍銜羽葆刻木為仙人衣羽衣立車上車雖囘轉手常南指大駕出行為先啓之乘蕭子顯曰指南車四周廂上施屋指南人衣裙襦天衣在廂中上四角皆施龍孑干緣帷色青孔雀毦烏布皁複幔漆畫輪駕牛皆銅校飾記里鼔車制如指南上施華蓋孑繖衣漆畫鼔機皆在内渾戶本翻】其餘金玉繒帛珍寶皆以頒賜將士【繒慈陵翻】秦平原公璞并州刺史尹昭以蒲阪降東平公讚帥宗族百餘人詣裕降【降戶江翻帥讀曰率】裕皆殺之送姚泓至建康斬於市【孝武大元九年姚萇建國改元白雀歲在甲申傳三主三十四年而亡】裕以薛辯為平陽太守使鎮捍北道裕議遷都洛陽諮議參軍王仲德曰非常之事固非常人所及必致駭動今暴師日久士卒思歸遷都之計未可議也裕乃止羌衆十餘萬口西犇隴上沈林子追擊至槐里俘虜萬計【姚氏羌也姚氏既滅故羌衆西奔】河西王蒙遜聞太尉裕滅秦怒甚門下校郎劉祥入言事【自曹操孫權置校事司察羣臣謂之校郎後遂因之蒙遜置諸曹校郎如門下校郎中兵校郎是也】蒙遜曰汝聞劉裕入關敢研研然也遂斬之【楊正衡曰研五見翻然有其音而無其義河西士民乃心晉室蒙遜胡人竊據其上聞裕入關慮其響應故斬祥以威衆以鎮服其心也姦雄之喜怒豈苟然哉魏書沮渠傳作妍妍華人服飾妍靡自喜故蒙遜云然妍讀如字音義皆通當從魏書】初夏王勃勃聞太尉裕伐秦謂羣臣曰姚泓非裕敵也且其兄弟内叛安能拒人裕取關中必矣然裕不能久留必將南歸留子弟及諸將守之【將即亮翻下鎮將同】吾取之如拾芥耳乃秣馬礪兵訓養士卒進據安定秦嶺北郡縣鎭戍皆降之裕遣使遺勃勃書【降戶江翻使疏吏翻遺于季翻】約為兄弟勃勃使中書侍郎皇甫徽為報書而隂誦之對裕使者口授舍人使書之裕讀其文歎曰吾不如也【史言夷豪多權數】 廣州刺史謝欣卒東海人徐道期聚衆攻陷州城進攻始興始興相彭城劉謙之討誅之詔以謙之為廣州刺史 癸酉司馬休之司馬父思司馬國璠司馬道賜魯軌韓延之刁雍王慧龍及桓温之孫道度道子族人桓謐桓璲陳郡袁式等皆詣魏長孫嵩降【姚秦既滅司馬休之等懼為裕所誅故皆降魏璠孚袁翻雍於容翻璲音遂降戶江翻】秦匈奴鎮將姚成都及弟和都舉鎮降魏魏主嗣詔民間得姚氏子弟送平城者賞之冬十月己酉嗣召長孫嵩等還司馬休之尋卒於魏【卒子恤翻】魏賜國璠爵淮南公道賜爵池陽子魯軌爵襄陽公刁雍表求南鄙自效嗣以雍為建義將軍【建義將軍魏以是號寵刁雍言使之建義以復父兄之仇】雍聚衆於河濟之間【濟子禮翻】擾動徐兖太尉裕遣兵討之不克雍進屯固山衆至二萬 詔進宋公爵為王增封十郡辭不受西秦王熾磐遣左丞相曇達等擊秦故將姚艾【艾秦上邽】<br />
<br />
  【之鎮將將即亮翻】艾遣使稱藩【使疏吏翻】熾磐以艾為征東大將軍秦州牧徵王松夀為尚書左僕射【十二年熾磐遣松夀屯馬頭以逼秦之上邽上邽降故徵還】 十一月魏叔孫建等討西山丁零翟蜀洛支等平之【西山魏安州之西山】 辛未劉穆之卒太尉裕聞之驚慟哀惋者累日【卒子恤翻惋烏貫翻】始裕欲留長安經畧西北而諸將佐皆久役思歸多不欲留【將即亮翻】會穆之卒裕以根本無託遂決意東還【還從宣翻又如字】穆之之卒也朝廷恇懼【恇音匡怯也】欲發詔以太尉左司馬徐羨之代之中軍諮議參軍張邵曰今誠急病任終在徐然世子無專命宜須諮之裕欲以王弘代穆之從事中郎謝晦曰休元輕易不若羨之【王弘字休元易以䜴翻】乃以羨之為吏部尚書建威將軍丹陽尹代管留任於是朝廷大事常決於穆之者並悉北諮裕以次子桂陽公義真為都督雍梁秦三州諸軍事安西將軍領雍東秦二州刺史【雍於用翻】義真時年十二以太尉諮議參軍京兆王脩為長史王鎮惡為司馬領馮翊太守沈田子毛德祖皆為中兵參軍仍以田子領始平太守德祖領秦州刺史天水太守傅弘之為雍州治中從事史先是隴上流戶寓關中者望因兵威得復本土及置東秦州【時裕未得天水東秦州即毛德祖所領或曰裕置東秦州使義真兼領先悉薦翻】知裕無復西略之意【復扶又翻下同】皆嘆息失望關中人素重王猛裕之克長安王鎮惡功為多由是南人皆忌之沈田子自以嶢柳之捷與鎮惡争功不平裕將還田子及傅弘之屢言於裕曰鎮惡家在關中不可保信裕曰今留卿文武將士精兵萬人彼若欲為不善正足自滅耳勿復多言裕私謂田子曰鍾會不得遂其亂者以有衛瓘故也【會瓘事見七十八卷魏元帝咸熙元年】語曰猛獸不如羣狐卿等十餘人何懼王鎮惡【為沈田子殺王鎮惡張本】<br />
<br />
  臣光曰古人有言疑則勿任任則勿疑裕既委鎮惡以關中而復與田子有後言是鬬之使為亂也惜乎百年之寇千里之土得之艱難失之造次【造七到翻】使豐鄗之都復輸寇手【鄗音浩】荀子曰兼并易能也堅凝之難信哉<br />
<br />
  三秦父老聞裕將還詣門流涕訴曰殘民不霑王化於今百年始覩衣冠人人相賀長安十陵是公家墳墓咸陽宫殿是公家室宅【漢高帝長陵惠帝安陵文帝覇陵景帝陽陵武帝茂陵昭帝平陵宣帝杜陵元帝渭陵成帝延陵哀帝義陵平帝康陵皆在關中凡十一陵言十者舉大數也長安咸陽宫殿皆漢故跡裕劉氏子孫故父老以是為言而留之】捨此欲何之乎裕為之愍然【為于偽翻】慰諭之曰受命朝廷不得擅留誠多諸君懷本之志今以次息【次息猶言次子也】與文武賢才共鎮此境勉與之居十二月庚子裕發長安自洛入河開汴渠而歸【汴音卞】 氐豪徐駭奴齊元子等擁部落三萬在雍遣使請降於魏【雍於用翻使疏吏翻降戶江翻】魏主嗣遣將軍王洛生河内太守楊聲等西行以應之 閏月壬申魏主嗣如大甯長川 秦雍人千餘家推襄邑令上谷寇讚為主以降於魏【讃秦之襄邑令也】魏主嗣拜讚魏郡太守久之秦雍人流入魏之河南滎陽河内者戶以萬數嗣乃置南雍州以讚為刺史封河南公治洛陽【治直之翻】立雍州郡縣以撫之讚善於招懷流民歸之者三倍其初 夏王勃勃聞太尉裕東還大喜【善用兵者觀釁而動】問於王買德曰朕欲取關中卿試言其方略買德曰關中形勝之地而裕以幼子守之狼狽而歸正欲急成簒事耳不暇復以中原為意【劉裕之心事崔浩王買德皆知之復扶又翻】此天以關中賜我不可失也青泥上洛南北之險要宜先遣遊軍斷之東塞潼關【斷丁管翻塞悉則翻】絶其水陸之路然後傳檄三輔施以威德則義真在網罟之中不足取也【勃勃敗義真取關中卒如買德之計罟音古】勃勃乃以其子撫軍大將軍璝都督前鋒諸軍事帥騎二萬向長安【璝古回翻帥讀曰率騎奇寄翻】前將軍昌屯潼關以買德為撫軍右長史屯青泥【劉裕得洛陽而不能禁寇讚窺伺於其側使義真守關中而不能禁夏兵之斷潼關青泥南歸彭城席未煖而義真敗既棄天下肉未寒而四鎮失宜也】勃勃將大軍為後繼【將即亮翻下同】是歲魏都坐大官章安侯封懿卒【坐徂臥翻】<br />
<br />
  十四年春正月丁酉朔魏主嗣至平城命護高車中郎將薛繁帥高車丁零北略至弱水而還【魏倣漢置匈奴中郎將之官置護高車中郎將帥讀曰率】 卒已大赦 夏赫連璝至渭陽關中民降之者屬路【降戶江翻屬之欲翻】龍驤將軍沈田子將兵拒之【驤思將翻】畏其衆盛退屯劉迴堡遣使還報王鎮惡鎮惡謂王脩曰公以十歲兒付吾屬當共思竭力而擁兵不進虜何由得平使者還以告田子田子與鎮惡素有相圖之志由是益忿懼未幾鎮惡與田子俱出北地以拒夏兵【赫連璝已至渭陽王沈烏能出北地乎此言北地者謂長安以北之地耳幾居豈翻】軍中訛言鎮惡欲盡殺南人以數十人送義眞南還因據關中反辛亥田子請鎮惡至傅弘之營計事田子求屏人語【屏必郢翻】使其宗人沈敬仁斬之幕下矯稱受太尉令誅之弘之犇告劉義真義真與王脩被甲登横門以察其變【横門長安城北出東頭第一門被皮義翻横音光】俄而田子帥數十人來言鎮惡反脩執田子數以專戮斬之【數所具翻】以冠軍將軍毛脩之代鎮惡為安西司馬【冠古玩翻】傅弘之大破赫連璝於池陽又破之於寡婦渡【按宋白續通典今慶州北十五里有寡婦山盖水發源是山其下流為寡婦渡】斬獲甚衆夏兵乃退壬戌太尉裕至彭城解嚴琅邪王德文先歸建康裕聞王鎮惡死表言沈田子忽發狂易奄害忠勲【狂易謂病狂而變易其常心易如字】追贈鎮惡左將軍青州刺史以彭城内史劉遵考為并州刺史領河東太守鎮蒲阪徵荆州刺史劉道憐為徐兖二州刺史裕欲以世子義符鎮荆州以徐州刺史劉義隆為司州刺史鎮洛陽中軍諮議張邵諫曰【諮議諮議參軍也】儲貳之重四海所繫不宜處外【處昌呂翻】乃更以義隆為都督荆益寧雍梁秦六州諸軍事西中郎將荆州刺史【更上衡翻雍於用翻】以南郡太守到彦之為南蠻校尉張邵為司馬領南郡相冠軍功曹王曇首為長史【冠古玩翻曇徒含翻】北徐州從事王華為西中郎主簿【晉置南徐州於京口北徐州仍治彭城到彦之王曇首王華輔義隆入立遂居將相之任】沈林子為西中郎參軍義隆尚幼府事皆決於邵曇首弘之弟也裕謂義隆曰王曇首沈毅有器度宰相才也汝每事諮之【沈持林翻】以南郡公劉義慶為豫州刺史義慶道憐之子也裕解司州領徐冀二州刺史 秦王熾磐以乞伏木弈干為沙州刺史鎮樂都【樂音洛】二月乙弗烏地延帥戶二萬降秦 三月遣使聘魏【使疏吏翻】 夏四月己巳魏徙冀定幽三州徒河於代都【魏主珪皇始二年克中山置安州又立行臺以鎮撫其民天興三年改曰定州領中山常山鉅鹿博陵北平河間高陽趙郡宋白曰初置安州尋改定州以安定天下為名徒河蓋徒河之民從慕容入中國留居三州者魏人因謂之徒河】 初和龍有赤氣四塞蔽日【塞悉則翻】自寅至申燕太史令張穆言於燕王跋曰此兵氣也今魏方彊盛而執其使者【謂留于什門也事見一百十六卷義熙十年】好命不通臣竊懼焉【好呼到翻】跋曰吾方思之五月魏主嗣東巡至濡源及甘松【濡乃官翻】遣征東將軍長孫道生安東將軍李先給事黄門侍郎奚觀帥精騎二萬襲燕【觀古玩翻帥讀曰率騎奇寄翻】又命驍騎將軍延普幽州刺史尉諾自幽州引兵趨遼西為之聲勢【魏書官氏志内入諸姓可地延氏為延氏驍堅堯翻西方尉遲氏後改為尉氏尉音鬱】嗣屯突門嶺以待之道生等拔乙連城進攻和龍與燕單于右輔古泥戰破之【義熙七年跋置單于四輔單音蟬】殺其將皇甫規【將即亮翻】燕王跋嬰城自守魏人攻之不克掠其民萬餘家而還【還從宣翻又如字】 六月太尉裕始受相國宋公九錫之命【十二年命下至是乃受】赦國中殊死以下崇繼母蘭陵蕭氏為太妃以太尉軍諮祭酒孔靖為宋國尚書令左長史王弘為僕射領選【選須絹翻】從事中郎傅亮蔡廓皆為侍中謝晦為右衛將軍右長史鄭鮮之為奉常行參軍殷景仁為祕書郎其餘百官悉依天朝之制【朝直遙翻下同】靖辭不受亮咸之孫【傳咸仕于武帝之間以直顯】廓謨之曾孫【蔡謨歷事成康穆三朝出藩入輔皆有聲績】鮮之渾之玄孫【鄭渾見六十六卷漢獻帝建安十七年】景仁融之曾孫也【殷融見九十四卷成帝咸和三年】景仁學不為文敏有思致【思相吏翻】口不談義深達理體至於國典朝儀舊章記注莫不撰録【撰士免翻】識者知其有當世之志魏天部大人白馬文貞公崔宏疾篤【去年魏置天地四方六部大人】<br />
<br />
  魏主遣侍臣問病一夜數返及卒詔羣臣及附國渠帥皆會葬【渠大也卒子恤翻帥所類翻】 秋七月戊午魏主嗣至平城九月甲寅魏人命諸州調民租戶五十石積於定相<br />
<br />
  冀三州【魏主珪天興四年置相州于鄴領魏陽平廣平汲郡東郡頓丘濮陽清河等郡冀州所領止長樂渤海武邑章武樂陵而已調徒弔翻相息亮翻】河西王蒙遜復引兵伐涼【復扶又翻下復歸同】涼公歆將拒之左長史張體順固諫乃止蒙遜芟其秋稼而還【芟所銜翻】歆遣使來告襲位冬十月以歆為都督七郡諸軍事鎮西大將軍酒泉公【都督燉煌酒泉晉興建康涼興及歆父暠所置會稽廣夏凡七縣】 姚艾叛秦降河西王蒙遜【姚艾稱藩於乞伏事見上年降戶江翻】蒙遜引兵迎之艾叔父雋言於衆曰秦王寛仁有雅度自可安居事之何為從河西王西遷衆咸以為然乃相與逐艾推雋為主復歸於秦秦王熾磐徵雋為侍中中書監賜爵隴西公以左丞相曇達為都督洮罕以東諸軍事征東大將軍秦州牧鎮南安【洮罕謂臨洮抱罕也曇徒含翻洮土刀翻】 劉義真年少【少詩照翻】賜與左右無節王脩每裁抑之左右皆怨譛脩於義真曰王鎮惡欲反故沈田子殺之脩殺田子是亦欲反也義真信之使左右劉乞等殺脩脩既死人情離駭莫相統壹義真悉召外軍入長安【外軍謂屯蒲阪以捍魏屯渭北以捍夏之軍也】閉門拒守關中郡縣悉降於夏赫連璝夜襲長安不克【降戶江翻璝古回翻】夏王勃勃進據咸陽長安樵采路絶宋公裕聞之使輔國將軍蒯恩如長安召義真東歸【蒯苦怪翻】以相國右司馬朱齡石為都督關中諸軍事右將軍雍州刺史代鎮長安【晉先置雍州於襄陽此為北雍州雍於用翻】裕謂齡石曰卿至可勅義真輕裝速發既出關然可徐行【然下當有後字】若關右必不可守可與義真俱歸又命中書侍郎朱超石慰勞河洛【勞力到翻】十一月齡石至長安義真將士貪縱大掠而東【將即亮翻】多載寶貨子女方軌徐行雍州别駕韋華犇夏【韋華本姚氏臣也裕用為雍州别駕】赫連璝帥衆三萬追義真建威將軍傅弘之曰公處分亟進今多將輜重【帥讀曰率處昌呂翻分扶問翻重直用翻】一日行不過十里虜追騎且至【騎奇寄翻下同】何以待之宜棄車輕行乃可以免義真不從俄而夏兵大至傅弘之蒯恩斷後【斷丁管翻】力戰連日至青泥晉兵大敗弘之恩皆為王買德所禽【買德先屯青泥故二將為所邀而見禽】司馬毛脩之與義真相失亦為夏兵所禽義真行在前會日暮夏兵不窮追故得免左右盡散獨逃草中中兵參軍段宏單騎追尋緣道呼之義真識其聲出就之曰君非段中兵邪身在此行矣【晉人多自稱為身】必不兩全可刎身頭以南使家公望絶【魏晉之間凡人子者稱其父曰家公人稱之曰尊公刎扶粉翻】宏泣曰死生共之下官不忍乃束義真於背單馬而歸義真謂宏曰今日之事誠無算畧然丈夫不經此何以知艱難夏王勃勃欲降傅弘之【降戶江翻】弘之不屈勃勃裸之弘之叫罵而死【裸郎果翻】勃勃積人頭為京觀號曰髑髏臺【觀古玩翻髑徒谷翻髏洛侯翻】長安百姓逐朱齡石齡石焚其宫殿犇潼關【義真既大掠長安而歸長安之人固仇視晉人也齡石奉宋公之命與義真俱歸可矣癡坐長安以待逐何歟】勃勃入長安大饗將士舉觴謂王買德曰卿往日之言一朞而驗可謂算無遺策此觴所集非卿而誰以買德為都官尚書封河陽侯龍驤將軍王敬先戍曹公壘【曹公壘在潼關曹操伐韓馬所築也驤思將翻】齡石往從之朱超石至蒲阪聞齡石所在亦往從之赫連昌攻敬先壘斷其水道【斷丁管翻】衆渴不能戰城且陷齡石謂超石曰弟兄俱死異域使老親何以為心爾求間道亡歸【間古莧翻】我死此無恨矣超石持兄泣曰人誰不死寧忍今日辭兄去乎遂與敬先及右軍參軍劉欽之皆被執送長安勃勃殺之【被皮義翻】欽之弟秀之悲泣不歡燕者十年欽之穆之之從兄子也【從才用翻】宋公裕聞青泥敗未知義真存亡刻日北伐【使裕能復北伐則聞青泥之敗當投袂而起矣何待刻日乎英雄所為固非常人所測識也】侍中謝晦諫以士卒疲弊請俟他年不從【晦請俟他年亦裕所謂識機變者也鄭鮮之之言則異於是】鄭鮮之上表以為虜聞殿下親征必併力守潼關徑往攻之恐未易可克【易以䜴翻】若輿駕頓洛則不足上勞聖躬且虜雖得志不敢乘勝過陜者猶攝服大威【陜失冉翻攝讀曰懾】為將來之慮故也若造洛而反虜必更有揣量之心或益生邊患【造七到翻揣初委翻量音良】况大軍遠出後患甚多昔歲西征劉鍾狼狽【謂十一年盜襲冶亭時也】去年北討廣州傾覆【謂徐道期陷廣州也】既往之效後來之鑒也今諸州大水民食寡乏三吳羣盜攻没諸縣皆由困於征役故也江南士庶引領顒顒【顒魚容翻】以望殿下之返斾聞更北出不測淺深之謀往還之期臣恐返顧之憂更在腹心也若慮西虜更為河洛之患者宜結好北虜北虜親則河南安【北虜魏也好呼到翻】河南安則濟泗静矣【濟子禮翻】會得段宏啓知義真得免裕乃止但登城北望慨然流涕而已降義真為建威將軍司州刺史以段宏為宋臺黄門郎領太子右衛率【率所律翻】裕以天水太守毛德祖為河東太守代劉遵考守蒲阪【裕雖知德祖善守而用之然人心已搖宜其不能固也為下德祖棄蒲阪張本】 夏王勃勃築壇於灞上即皇帝位改元武昌西秦王熾磐東巡十二月徙上邽民五千餘戶于枹<br />
<br />
  罕【枹音膚】 彗星出天津入太微經北斗絡紫微【晉書天文志曰箕四星一曰天津又曰天漢經尾箕之間謂之漢津太微天子庭也在北斗南紫微十五星在北斗北彗祥歲翻】八十餘日而滅魏主嗣復召諸儒【復扶又翻】術士問之曰今四海分裂災咎之應果在何國朕甚畏之卿輩盡言勿有所隱衆推崔浩使對浩曰夫災異之興皆象人事人苟無舋又何畏焉【舋許覲翻】昔王莽將簒漢彗星出入正與今同【漢書天文志曰哀帝建平二年彗星出牽牛七十餘日傳曰彗者所以除舊布新牽牛日月五星所從起三正之始彗而出之改更之象也其後卒有王莽簒國之禍】國家主尊臣卑民無異望晉室陵夷危亡不遠彗之為異其劉裕將簒之應乎衆無以易其言 宋公裕以䜟云昌明之後尚有二帝【晉書帝紀曰初簡文帝見䜟云晉祚盡昌明及孝武帝之在孕也李太后夢神人謂之曰汝生男以昌明為字及產東方始明因以為名簡文後悟乃流涕又曰䜟云昌明之後有二帝裕乃使縊帝而立恭帝以應二帝云䜟楚讚翻】乃使中書侍郎王韶之與帝左右密謀酖帝而立琅邪王德文德文常在帝左右飲食寢處未嘗暫離【處昌呂翻離力智翻】韶之伺之經時不得間會德文有疾出居於外戊寅韶之以散衣縊帝於東堂【年三十七伺相吏翻間古莧翻散悉亶翻縊於賜翻又於計翻】韶之廙之曾孫也【廙王敦之從弟廙羊至翻又逸職翻】裕因稱遺詔奉德文即皇帝位大赦 是歲河西王蒙遜奉表稱藩拜涼州刺史 尚書右僕射袁湛卒<br />
<br />
  恭皇帝【諱德文字德文安帝母弟也諡法尊賢貴義敬事供上尊賢敬讓愛民長弟執禮御賓芘親之闕皆曰恭長弟謂順長接弟御賓迎待賓也】<br />
<br />
  元熙元年春正月壬戌朔改元 立琅邪王妃褚氏為皇后后裒之曾孫也【褚裒崇德太后之父裒蒲侯翻】 魏主嗣畋于犢渚【據北史犢渚在柞山西臨河】 甲午徵宋公裕入朝【朝直遙翻】進爵為王裕辭 癸卯魏主嗣還平城 庚申葬安皇帝于休平陵 刺劉道憐司空出鎮京口【刺者敕字之誤也司空之上又當逸以字】夏將叱奴侯提帥步騎二萬攻毛德祖於蒲阪【將即亮翻】<br />
<br />
  【帥讀曰率騎奇寄翻阪音反】德祖不能禦全軍歸彭城二月宋公裕以德祖為滎陽太守戍虎牢【宋白曰虎牢古東虢國春秋為鄭之制邑漢為成臯縣穆天子傳天子獵于鄭有虎在葭中匕萃之士禽之以獻命畜之東虢號曰虎牢後為成臯縣北臨黄河後漢為成臯關後魏為東中郎將府唐為汜水縣】 夏主勃勃徵隱士京兆韋祖思祖思既至恭懼過甚勃勃怒曰我以國士徵汝汝乃以非類遇我汝昔不拜姚興今何獨拜我我在汝猶不以我為帝王我死汝曹弄筆當置我於何地邪遂殺之【勃勃之殺祖思虐矣然祖思之恭懼過甚勃勃以為薄已而殺之則勃勃為有見而祖思為無所守也】羣臣請都長安勃勃曰朕豈不知長安歷世帝王之都沃饒險固然晉人僻遠終不能為吾患魏與我風俗略同土壤鄰接自統萬距魏境裁百餘里朕在長安統萬必危若在統萬魏必不敢濟河而西諸卿適未見此耳皆曰非所及也【使勃勃常在猶云可也勃勃死則統萬為魏有古人所以貽厥子孫者固有道也】乃於長安置南臺以赫連璝領大將軍雍州牧録南臺尚書事【璝工回翻雍於用翻】勃勃還統萬大赦改元真興勃勃性驕虐視民如草芥常居城上置弓劒於側有所嫌忿手自殺之羣臣迕視者鑿其目【索隱曰迕者逆也迕五故翻】笑者決其脣諫者先截其舌而後斬之 初司馬楚之奉其父榮期之喪歸建康【榮期死見一百一十四卷安帝義熙二年】會宋公裕誅翦宗室之有才望者楚之叔父宣期兄貞之皆死楚之亡匿竟陵蠻中及從祖休之自江陵犇秦【休之宣帝弟魏中郎進之六世孫楚之宣帝弟太常馗之八世孫故休之於楚之為從祖休之犇秦見上卷義熙十一年從才用翻】楚之亡之汝潁間聚衆以謀復讎楚之少有英氣能折節下士【少詩照翻折而設翻下遐稼翻】有衆萬餘屯據長社裕使刺客沐謙往刺之【沐莫卜翻姓也風俗通漢有東平太守沐寵蜀本作沐音述非也】楚之待謙甚厚謙欲發未得間乃夜稱疾知楚之必往問疾因欲刺之【間古莧翻刺七亦翻】楚之果自齎湯藥往視疾情意勤篤謙不忍發乃出匕首於席下以狀告之曰將軍深為劉裕所忌願勿輕率以自保全遂委身事之為之防衛王鎮惡之死也沈田子殺其兄弟七人唯弟康得免逃就宋公裕于彭城裕以為相國行參軍【晉制諸公府置諸曹參軍又有正参軍行参軍長兼行参軍等員】康求還洛陽視母會長安不守康糾合關中徙民得百許人驅帥僑戶七百餘家共保金墉城【帥讀曰率下同】時宗室多逃亡在河南有司馬文榮者帥乞活千餘戶屯金墉城南【惠帝時并州饑荒其吏民隨東燕王騰東下號曰乞活是後流徙逐糧者亦曰乞活】又有司馬道恭自東垣帥三千人屯城西【按魏收地形志洛州新安郡有東垣縣注云二漢晉屬河東後屬參攷漢晉志河東郡有垣縣無東垣孝武太元十一年馮該撃斬苻丕於東垣此時已有東垣之名宋白曰宋武入洛更置東垣西垣二縣新唐書地理志河南府新安縣高祖武德初析置東垣縣則知東垣在新安界】司馬順明帥五千人屯陵雲臺司馬楚之屯柏谷塢魏河内鎮將于栗磾遊騎在芒山上【將即亮翻磾丁奚翻騎奇寄翻】攻逼交至康堅守六旬裕以康為河東太守遣兵救之平等皆散走【詳考上文未知平等為何人】康勸課農桑百姓甚親賴之司馬順明司馬道恭及平陽太守薛辯皆降於魏【降戶江翻下同】魏以辯為河東太守以拒夏人 夏四月秦征西將軍孔子帥騎五千討吐谷渾覔地於弱水南【孔子亦乞伏氏也禹貢導弱水至于合黎餘波入于流涉地志云弱水出刪丹縣亦謂之張掖河合黎在酒泉會水縣東北流沙張掖居延縣東北之居延澤是也曾氏曰弱水出窮谷】大破之覔地帥其衆六千降於秦拜弱水護軍 庚辰魏主嗣有事於東廟【古制左祖右社魏建宗廟於平城宫之東因曰東廟杜佑曰明元永興四年立太祖道武廟於白登山歲一祭無常月又於白登西大祖舊遊之處立昭成獻明太祖廟常以九月十月之交親祀焉則東廟者白登山廟也以山西又有廟故以此為東廟】助祭者數百國辛巳南巡至雁門 五月庚寅朔魏主嗣觀漁于灅水己亥還平城【灅力水翻】 涼公歆用刑過嚴又好治宫室【好呼到翻治直之翻】從事中郎張顯上疏以為涼土三分【謂李氏沮渠乞伏也】勢不支久兼并之本在於務農懷遠之略莫如寛簡今入歲已來隂陽失序風雨乖和是宜減膳徹懸【古者天子膳用六牲具馬牛羊犬豕雞諸侯膳用三牲懸樂懸也天子宫懸諸侯軒懸大荒大札天地有烖國有大故則減膳徹樂穀梁傳曰五穀不升為天饑一穀不升謂之嗛二穀不升謂之饑三穀不升謂之饉四穀不升謂之康五穀不升謂之大侵大侵之禮君食不兼味臺榭不塗弛侯廷道不除百官布而不祭鬼神禱而不祀白虎通曰一穀不升徹鶉鷃二穀不升徹鳬雁三穀不升徹雉兎四穀不升損囿獸五穀不升不備三牲】側身修道而更繁刑峻法繕築不止殆非所以致興隆也昔文王以百里而興二世以四海而滅【周文王興於岐周地方百里秦二世承始皇之後奄有四海卒以滅亡】前車之軌得失昭然太祖以神聖之姿為西夏所推左取酒泉右開西域【李暠廟號大祖為西夏所推事見一百一十二卷安帝隆安四年取酒泉見五年開西域亦見四年夏戶雅翻】殿下不能奉承遺志混壹涼土侔蹤張后【張后謂張軌及其子若孫也】將何以下見先王乎沮渠蒙遜西土之傑内修政事外禮英賢攻戰之際身均士卒百姓懷之樂為之用【沮子余翻樂音洛】臣謂殿下非但不能平殄蒙遜亦懼蒙遜方為社稷之憂歆覽之不悦主簿汜稱上疏諫曰【汜音凡】天之子愛人主殷勤至矣故政之不修下災異以戒告之改者雖危必昌不改者雖安必亡元年三月癸卯敦煌謙德堂陷【張駿據河西起謙光殿于姑臧自謂專制一方而事晉不改臣節雖謙而光也李暠得敦煌亦稱藩於晉起謙德堂其志猶張氏也敦煌徒門翻】八月效穀地裂二年元日昏霧四塞四月日赤無光二旬乃復十一月狐上南門今茲春夏地頻五震六月隕星于建康臣雖學不稽古行年五十有九請為殿下畧言耳目之所聞見不復能遠論書傳之事也【塞悉則翻上時掌翻為于偽翻不復扶又翻傳直戀翻】乃者咸安之初西平地裂狐入謙光殿前俄而秦師奄至城都不守【咸安簡文帝年號涼土以姑臧為都城孝武太元元年秦入姑臧蓋地裂狐入在咸安之初而其應在太元之初也】梁熙既為涼州不撫百姓專為聚斂【斂力贍翻】建元十九年姑臧南門崩隕石於閑豫堂明年為呂光所殺【大元元年秦主堅建元之十二年也堅以梁熙鎮涼州建元十九年堅敗於淮南晉太元之八年也明年呂光殺梁熙】段業稱制此方三年之中地震五十餘所既而先王龍興於瓜州【瓜州敦煌郡也攷之晉志張氏置沙州于敦煌未嘗置瓜州又攷之唐志沙州敦煌郡本瓜州武德五年曰西沙州貞觀七年曰沙州瓜州晉昌郡武德五年析沙州之常樂置盖李暠興于敦煌自稱秦涼二州牧其後遷于酒泉以敦煌為瓜州至唐復以敦煌為沙州以晉昌為瓜州而瓜州分為二州矣】蒙遜簒弑於張掖此皆目前之成事殿下所明知也效穀先王鴻漸之地【暠自效穀令得敦煌遂有七郡故云然易所謂鴻漸者鴻水鳥也自水而漸于干又漸于磐又漸于陸又漸于木自下而進漸升而上也】謙德即尊之室基陷地裂大凶之徵也日者太陽之精中國之象赤而無光中國將衰諺曰【諺音彦】野獸入家主人將去狐上南門亦變異之大者也今變夷益盛中國益微願殿下亟罷宫室之役止遊畋之娛延禮英俊愛養百姓以應天變防未然歆不從 秋七月宋公裕始受進爵之命八月移鎮夀陽以度支尚書劉懷慎為督淮北諸軍事徐州刺史鎮彭城【曹魏文帝置度支尚書掌軍國支計晉因之度徒洛翻】辛未魏主嗣東巡甲申還平城 九月宋王裕自解<br />
<br />
  揚州牧 秦左衛將軍匹達等將兵討彭利和于漒川大破之利和單騎奔仇池獲其妻子徙羌豪三千戶于枹罕漒川羌三萬餘尸皆安堵如故冬十月以尚書右僕射王松夀為益州刺史鎮漒川【漒其良翻騎奇寄翻枹音膚】 宋王裕以河南蕭條乙酉徙司州刺史義真為揚州刺史鎮石頭蕭太妃謂裕曰道憐汝布衣兄弟宜用為揚州裕曰寄奴於道憐豈有所惜【裕小字寄奴道憐蕭太妃所生也】揚州根本所寄事務至多非道憐所了太妃曰道憐年出五十豈不如汝十歲兒邪裕曰義真雖為刺史事無大小悉由寄奴道憐年長不親其事於聽望不足【聽望猶言觀聽也長知兩翻】太妃乃無言道憐性愚鄙而貪縱故裕不肯用 十一月丁亥朔日有食之 十二月癸亥魏主嗣西巡至雲中從君子津西渡河大獵於薛林山【按魏書帝紀薛林山在屋竇城西】 辛卯宋王裕加殊禮進王太妃為太后世子為太子<br />
<br />
  資治通鑑卷一百十八  <br>
   </div> 

<script src="/search/ajaxskft.js"> </script>
 <div class="clear"></div>
<br>
<br>
 <!-- a.d-->

 <!--
<div class="info_share">
</div> 
-->
 <!--info_share--></div>   <!-- end info_content-->
  </div> <!-- end l-->

<div class="r">   <!--r-->



<div class="sidebar"  style="margin-bottom:2px;">

 
<div class="sidebar_title">工具类大全</div>
<div class="sidebar_info">
<strong><a href="http://www.guoxuedashi.com/lsditu/" target="_blank">历史地图</a></strong>  
<a href="http://www.880114.com/" target="_blank">英语宝典</a>  
<a href="http://www.guoxuedashi.com/13jing/" target="_blank">十三经检索</a> 
<br><strong><a href="http://www.guoxuedashi.com/gjtsjc/" target="_blank">古今图书集成</a></strong> 
<a href="http://www.guoxuedashi.com/duilian/" target="_blank">对联大全</a> <strong><a href="http://www.guoxuedashi.com/xiangxingzi/" target="_blank">象形文字典</a></strong> 

<br><a href="http://www.guoxuedashi.com/zixing/yanbian/">字形演变</a>  <strong><a href="http://www.guoxuemi.com/hafo/" target="_blank">哈佛燕京中文善本特藏</a></strong>
<br><strong><a href="http://www.guoxuedashi.com/csfz/" target="_blank">丛书&方志检索器</a></strong> <a href="http://www.guoxuedashi.com/yqjyy/" target="_blank">一切经音义</a>  

<br><strong><a href="http://www.guoxuedashi.com/jiapu/" target="_blank">家谱族谱查询</a></strong>  <strong><a href="http://shufa.guoxuedashi.com/sfzitie/" target="_blank">书法字帖欣赏</a></strong> 
<br>

</div>
</div>


<div class="sidebar" style="margin-bottom:0px;">

<font style="font-size:22px;line-height:32px">QQ交流群9:489193090</font>


<div class="sidebar_title">手机APP 扫描或点击</div>
<div class="sidebar_info">
<table>
<tr>
	<td width=160><a href="http://m.guoxuedashi.com/app/" target="_blank"><img src="/img/gxds-sj.png" width="140"  border="0" alt="国学大师手机版"></a></td>
	<td>
<a href="http://www.guoxuedashi.com/download/" target="_blank">app软件下载专区</a><br>
<a href="http://www.guoxuedashi.com/download/gxds.php" target="_blank">《国学大师》下载</a><br>
<a href="http://www.guoxuedashi.com/download/kxzd.php" target="_blank">《汉字宝典》下载</a><br>
<a href="http://www.guoxuedashi.com/download/scqbd.php" target="_blank">《诗词曲宝典》下载</a><br>
<a href="http://www.guoxuedashi.com/SiKuQuanShu/skqs.php" target="_blank">《四库全书》下载</a><br>
</td>
</tr>
</table>

</div>
</div>


<div class="sidebar2">
<center>


</center>
</div>

<div class="sidebar"  style="margin-bottom:2px;">
<div class="sidebar_title">网站使用教程</div>
<div class="sidebar_info">
<a href="http://www.guoxuedashi.com/help/gjsearch.php" target="_blank">如何在国学大师网下载古籍?</a><br>
<a href="http://www.guoxuedashi.com/zidian/bujian/bjjc.php" target="_blank">如何使用部件查字法快速查字?</a><br>
<a href="http://www.guoxuedashi.com/search/sjc.php" target="_blank">如何在指定的书籍中全文检索?</a><br>
<a href="http://www.guoxuedashi.com/search/skjc.php" target="_blank">如何找到一句话在《四库全书》哪一页?</a><br>
</div>
</div>


<div class="sidebar">
<div class="sidebar_title">热门书籍</div>
<div class="sidebar_info">
<a href="/so.php?sokey=%E8%B5%84%E6%B2%BB%E9%80%9A%E9%89%B4&kt=1">资治通鉴</a> <a href="/24shi/"><strong>二十四史</strong></a>&nbsp; <a href="/a2694/">野史</a>&nbsp; <a href="/SiKuQuanShu/"><strong>四库全书</strong></a>&nbsp;<a href="http://www.guoxuedashi.com/SiKuQuanShu/fanti/">繁体</a>
<br><a href="/so.php?sokey=%E7%BA%A2%E6%A5%BC%E6%A2%A6&kt=1">红楼梦</a> <a href="/a/1858x/">三国演义</a> <a href="/a/1038k/">水浒传</a> <a href="/a/1046t/">西游记</a> <a href="/a/1914o/">封神演义</a>
<br>
<a href="http://www.guoxuedashi.com/so.php?sokeygx=%E4%B8%87%E6%9C%89%E6%96%87%E5%BA%93&submit=&kt=1">万有文库</a> <a href="/a/780t/">古文观止</a> <a href="/a/1024l/">文心雕龙</a> <a href="/a/1704n/">全唐诗</a> <a href="/a/1705h/">全宋词</a>
<br><a href="http://www.guoxuedashi.com/so.php?sokeygx=%E7%99%BE%E8%A1%B2%E6%9C%AC%E4%BA%8C%E5%8D%81%E5%9B%9B%E5%8F%B2&submit=&kt=1"><strong>百衲本二十四史</strong></a>  <a href="http://www.guoxuedashi.com/so.php?sokeygx=%E5%8F%A4%E4%BB%8A%E5%9B%BE%E4%B9%A6%E9%9B%86%E6%88%90&submit=&kt=1"><strong>古今图书集成</strong></a>
<br>

<a href="http://www.guoxuedashi.com/so.php?sokeygx=%E4%B8%9B%E4%B9%A6%E9%9B%86%E6%88%90&submit=&kt=1">丛书集成</a> 
<a href="http://www.guoxuedashi.com/so.php?sokeygx=%E5%9B%9B%E9%83%A8%E4%B8%9B%E5%88%8A&submit=&kt=1"><strong>四部丛刊</strong></a>  
<a href="http://www.guoxuedashi.com/so.php?sokeygx=%E8%AF%B4%E6%96%87%E8%A7%A3%E5%AD%97&submit=&kt=1">說文解字</a> <a href="http://www.guoxuedashi.com/so.php?sokeygx=%E5%85%A8%E4%B8%8A%E5%8F%A4&submit=&kt=1">三国六朝文</a>
<br><a href="http://www.guoxuedashi.com/so.php?sokeytm=%E6%97%A5%E6%9C%AC%E5%86%85%E9%98%81%E6%96%87%E5%BA%93&submit=&kt=1"><strong>日本内阁文库</strong></a> <a href="http://www.guoxuedashi.com/so.php?sokeytm=%E5%9B%BD%E5%9B%BE%E6%96%B9%E5%BF%97%E5%90%88%E9%9B%86&ka=100&submit=">国图方志合集</a> <a href="http://www.guoxuedashi.com/so.php?sokeytm=%E5%90%84%E5%9C%B0%E6%96%B9%E5%BF%97&submit=&kt=1"><strong>各地方志</strong></a>

</div>
</div>


<div class="sidebar2">
<center>

</center>
</div>
<div class="sidebar greenbar">
<div class="sidebar_title green">四库全书</div>
<div class="sidebar_info">

《四库全书》是中国古代最大的丛书,编撰于乾隆年间,由纪昀等360多位高官、学者编撰,3800多人抄写,费时十三年编成。丛书分经、史、子、集四部,故名四库。共有3500多种书,7.9万卷,3.6万册,约8亿字,基本上囊括了古代所有图书,故称“全书”。<a href="http://www.guoxuedashi.com/SiKuQuanShu/">详细>>
</a>

</div> 
</div>

</div>  <!--end r-->

</div>
<!-- 内容区END --> 

<!-- 页脚开始 -->
<div class="shh">

</div>

<div class="w1180" style="margin-top:8px;">
<center><script src="http://www.guoxuedashi.com/img/plus.php?id=3"></script></center>
</div>
<div class="w1180 foot">
<a href="/b/thanks.php">特别致谢</a> | <a href="javascript:window.external.AddFavorite(document.location.href,document.title);">收藏本站</a> | <a href="#">欢迎投稿</a> | <a href="http://www.guoxuedashi.com/forum/">意见建议</a> | <a href="http://www.guoxuemi.com/">国学迷</a> | <a href="http://www.shuowen.net/">说文网</a><script language="javascript" type="text/javascript" src="https://js.users.51.la/17753172.js"></script><br />
  Copyright &copy; 国学大师 古典图书集成 All Rights Reserved.<br>
  
  <span style="font-size:14px">免责声明:本站非营利性站点,以方便网友为主,仅供学习研究。<br>内容由热心网友提供和网上收集,不保留版权。若侵犯了您的权益,来信即刪。scp168@qq.com</span>
  <br />
ICP证:<a href="http://www.beian.miit.gov.cn/" target="_blank">鲁ICP备19060063号</a></div>
<!-- 页脚END --> 
<script src="http://www.guoxuedashi.com/img/plus.php?id=22"></script>
<script src="http://www.guoxuedashi.com/img/tongji.js"></script>

</body>
</html>
