






























































資治通鑑卷七     宋 司馬光 撰

胡三省 音註

秦紀二【起閼逢閹茂盡玄黓執徐几十九年}


始皇帝下

二十年荆軻至咸陽因王寵臣蒙嘉卑辭以求見王大喜朝服設九賓而見之【韋昭曰九賓周禮九儀也謂公侯伯子男孤卿大夫士也史記正義曰劉云設文物大備即謂九賓不得以周禮九賓義為釋劉原父曰賓謂傳擯之擯九賓擯者九人}
荆軻奉圖而進於王圖窮而匕首見【見賢遍翻}
因把王袖而揕之未至身王驚起袖絶荆軻逐王王環柱而走【環音宦}
羣臣皆愕卒起不意【愕五各翻卒讀曰猝後倉卒之卒皆同音}
盡失其度而秦法羣臣侍殿上者不得操尺寸之兵【操七刀翻}
左右以手共搏之且曰王負劒負劒王遂拔以擊荆軻斷其左股【斷丁管翻}
荆軻廢乃引匕首擿王中銅柱【索隱曰擿與擲同古字耳音持益翻中竹仲翻}
自知事不就罵曰事所以不成者以欲生刼之必得約契以報太子也遂體解荆軻以徇【體解支解也解佳買翻}
王於是大怒益發兵詣趙就王翦以伐燕與燕師代師戰於易水之西大破之

二十一年冬十月王翦拔薊【薊音計}
燕王及太子率其精兵東保遼東李信急追之代王嘉遺燕王書【遺于季翻}
令殺太子丹以獻丹匿衍水中【索隱曰衍水在遼東}
燕王使使斬丹欲以獻王王復進兵攻之【復扶又翻}
王賁伐楚【賁音奔翦之子也}
取十餘城王問於將軍李信曰吾欲取荆【王父莊襄王諱楚故謂楚為荆}
於將軍度用幾何人而足【度徒洛翻}
李信曰不過用二十萬王以問王翦王翦曰非六十萬人不可王曰王將軍老矣何怯也遂使李信蒙恬將二十萬人伐楚王翦因謝病歸頻陽【王翦頻陽人也班志頻陽縣屬京兆秦厲公所置應劭注曰縣在頻水之陽杜佑曰美陽本漢頻陽縣故城在縣西南三十里宋白曰因界内頻陽山而名}


二十二年王賁伐魏引河溝以灌大梁【班志陳留郡浚儀縣故大梁狼湯水所經也水經渠水出滎陽北河東南流至浚儀縣注云始皇使王賁攻魏斷故渠引水東南出以灌大梁因謂之梁溝}
三月城壞魏王假降殺之遂滅魏【降戶江翻}
王使人謂安陵君曰寡人欲以五百里地易安陵安陵君曰大王加惠以大易小甚幸雖然臣受地於魏之先王願終守之弗敢易王義而許之 李信攻平輿【班志汝南郡有平輿縣春秋沈子之國輿音預史記正義讀如字}
蒙恬攻寑【班志汝南郡冇寑縣應劭曰孫叔敖子所邑之寢丘是也世祖更名固始徐廣曰寑今固始寢丘師古曰寑子袵翻劉仲馮曰據後淮陽國已有固始此寑疑自别地余謂郡縣離合無常蓋後來併寑入固始也杜佑曰潁州治汝隂縣有寢丘秦蒙恬攻寢即此}
大破楚軍信又攻鄢郢破之【此鄢郢非楚故都之鄢郢也楚故都為白起所取秦已置南郡據楚都夀春以夀春為郢則其前自郢徙陳亦必以陳為郢矣然則此郢乃陳也鄢即潁川之鄢陵與平輿城父地皆相近或曰鄢郢當作鄢陵}
於是引兵而西與蒙恬會城父【班志沛郡有城父縣索隱曰在汝南即良鄉史記正義曰言引兵而會城父則是汝州郟城縣東父城者也括地志汝州郟城縣東四十里有父城故城即服䖍云城父楚北境者也又許州葉縣東北四十五里亦冇父城故城即杜預云襄城城父縣者也此二城父城之名耳服䖍城父是誤也左傳及水經注云楚大城城父使太子建居之十三州志云太子建所居城父謂今亳州城父是也此三家之說是城父之名班志云潁川父城縣沛郡城父縣據縣屬郡其名自分}
楚人因隨之三日三夜不頓舍大敗李信入兩壁殺七都尉【此郡都尉將兵從伐楚者也秦列郡有守有尉有監然秦漢之制行軍亦自冇都尉敗補邁翻}
李信犇還王聞之大怒自至頻陽謝王翦曰寡人不用將軍謀李信果辱秦軍將軍雖病獨忍棄寡人乎王翦謝病不能將王曰已矣勿復言【將即亮翻復扶又翻}
王翦曰必不得已用臣非六十萬人不可王曰為聽將軍計耳於是王翦將六十萬人伐楚王送至霸上【應劭曰霸上地名在霸水上在長安東三十里霸水古之滋水秦穆公更名}
王翦請美田宅甚衆王曰將軍行矣何憂貧乎王翦曰為大王將有功終不得封侯故及大王之嚮臣以請田宅為子孫業耳王大笑王翦既行至關【此當是出武關也}
使使還請善田者五輩或曰將軍之乞貸亦已甚矣【貸與貣同吐得翻從兵求物也}
王翦曰不然王怛中而不信人【史記注怛音麤徐廣曰一作粗}
今空國中之甲士而專委於我我不多請田宅為子孫業以自堅顧令王坐而疑我矣

二十三年王翦取陳以南至平輿楚人聞王翦益軍而來乃悉國中兵以禦之王翦堅壁不與戰楚人數挑戰【數所角翻挑戰者擿撓敵以求戰也挑徒了翻}
終不出王翦日休士洗沐而善飲食撫循之【飲於禁翻食祥吏翻後以義推}
親與士卒同食久之王翦使人問軍中戲乎對曰方投石超距【徐廣曰超一作拔裴駰曰據漢書云甘延夀投石拔距絶於等倫張晏曰范蠡兵法飛石重十二斤為機發行三百步延夀有力能手投之拔距猶超距也索隱曰超距猶跳躍也余謂投石以石投人也齊高固桀石以投人是也超距距躍也晉魏犨距躍三百是也}
王翦曰可用矣楚既不得戰乃引而東王翦追之令壯士擊大破楚師至蘄南【班志沛郡有蘄縣史記正義曰徐州縣也康以為江夏之蘄春其誤甚矣蘄渠之翻又音機}
殺其將軍項燕【項燕項梁之父也燕烏賢反}
楚師遂敗走王翦因乘勝略定城邑

二十四年王翦蒙武虜楚王負芻以其地置楚郡【楚至是亡矣按秦三十六郡無楚郡此蓋滅楚之時暫置耳後分為九江鄣會稽三郡}


二十五年大興兵使王賁攻遼東虜燕王喜【燕至是亡}
臣光曰燕丹不勝一朝之忿以犯虎狼之秦【勝音升}
輕慮淺謀挑怨速禍使召公之廟不祀忽諸【忽諸言忽然而亡也}
罪孰大焉而論者或謂之賢豈不過哉夫為國家者任官以才立政以禮懷民以仁交鄰以信是以官得其人政得其節百姓懷其德四鄰親其義夫如是則國家安如磐石熾如焱火【熾尺志翻焱弋瞻翻}
觸之者碎犯之者焦雖有彊暴之國尚何足畏哉丹釋此不為顧以萬乘之國決匹夫之怒逞盜賊之謀功隳身戮社稷為墟不亦悲哉夫其膝行蒲伏非恭也【蒲蓬逋翻手行也伏鼻墨翻伏地也}
復言重諾非信也【復言謂言必信而可復也重諾重然諾也}
糜金散玉非惠也刎首決腹非勇也要之謀不遠而動不義其楚白公勝之流乎【白公勝欲報其父讐不勝其忿以及其叔父事見左傳}
荆軻懷其豢養之私不顧七族【漢鄒陽曰荆軻湛七族應劭曰荆軻為燕刺秦王其族坐之湛沒}
欲以尺八匕首彊燕而弱秦不亦愚乎故揚子論之以要離為蛛蝥之靡聶政為壯士之靡【要離吳人為吳王闔廬刺慶忌言其力不足譬如蜘蛛之螫毒於人而靡死耳靡披靡而死也爾雅疏鼅鼄即鼄蝥方言自關以西秦晉之間謂之鼄蝥趙魏之間謂之鼅鼄蛛音朱蝥音矛靡温公揚子注音如字康美為切謂糜爛也余謂康音義俱非聶政事見一卷安王五年}
荆軻為刺客之靡皆不可謂之義又曰荆軻君子盜諸【吳祕曰荆軻以君子之道類之則盜爾}
善哉

王賁攻代虜代王嘉【嘉奔代見上卷十九年趙既不祀}
王翦悉定荆江南地降百越之君置會稽郡【秦會稽郡治吳縣兼有今閩越兩浙之地後漢分會稽置吳郡而會稽郡徙治山隂縣劉恕曰禹會諸侯江南而有功因名其山曰會稽猶言會計也會古外翻}
五月天下大酺 初齊君王后賢【君王后太史敫女襄王后}
事秦謹與諸侯信齊亦東邊海上【言齊東取島夷以海上為邊也或曰齊東邊海不與秦接故不受兵}
秦日夜攻三晉燕楚五國各自救以故齊王建立四十餘年不受兵及君王后且死戒王建曰羣臣之可用者某王曰請書之君王后曰善王取筆牘受言君王后曰老婦已忘矣【忘巫放翻}
君王后死后勝相齊【姓譜后本郈氏其後去邑史記正義曰勝音升}
多受秦間金【間古莧翻下同}
賓客入秦秦又多與金客皆為反間勸王朝秦不修攻戰之備不助五國攻秦秦以故得滅五國齊王將入朝雍門司馬前曰【左傳晉圍齊伐雍門之荻杜預曰雍門齊城門也經典釋文雍於用翻康於龍切非也}
所為立王者為社稷邪為王邪王曰為社稷司馬曰為社稷立王王何以去社稷而入秦【孟子曰民為大社稷次之君為輕}
齊王還車而反即墨大夫聞之見齊王曰齊地方數千里帶甲數百萬夫三晉大夫皆不便秦而在阿甄之間者百數【甄當作鄄音工掾翻}
王收而與之百萬人之衆使收三晉之故地即臨晉之關可以入矣【收三晉兵自河東攻秦則入臨晉關}
鄢郢大夫不欲為秦而在城南下者百數【城南下即南城之下也南城齊威王使檀子所守者}
王收而與之百萬之師使收楚故地即武關可以入矣【楚攻秦自南陽入武關}
如此則齊威可立秦國可亡豈特保其國家而已哉齊王不聽

二十六年王賁自燕南攻齊猝入臨淄民莫敢格者【格如字止也鬬也}
秦使人誘齊王約封以五百里之地【誘音酉}
齊王遂降秦遷之共【班志河内郡冇共縣史記正義曰今衛州有共城縣共音恭下同}
處之松柏之間餓而死【處昌呂翻}
齊人怨王建不早與諸侯合從聽姦人賓客以亡其國歌之曰松耶柏耶住建共者客耶疾建用客之不詳也【索隱曰謂不詳審用賓客不知其善否也齊田氏亡}
臣光曰從衡之說雖反覆百端然大要合從者六國之利也昔先王建萬國親諸侯使之朝聘以相交饗宴以相樂【樂音洛}
會盟以相結者無它欲其同心戮力以保家國也曏使六國能以信義相親則秦雖彊暴安得而亡之哉夫三晉者齊楚之藩蔽齊楚者三晉之根柢【柢都禮翻乂丁計翻}
形勢相資表裏相依故以三晉而攻齊楚自絶其根柢也以齊楚而攻三晉自撤其藩蔽也安有撤其藩蔽以媚盜曰盜將愛我而不攻豈不悖哉

王初并天下自以為德兼三皇功過五帝【伏羲神農黄帝為三皇少昊顓頊高辛唐堯虞舜為五帝宋均注援神契引甄耀度曰伏羲神農燧人為三皇黄帝顓頊帝嚳唐堯虞舜為五帝 孔穎達曰鄭玄注中候勑省圖引運斗樞伏羲女媧神農為三皇五帝者德合五帝座星者稱帝則黄帝金天氏高陽氏高辛氏陶唐氏有虞氏是也實六人而稱五者以其俱合五帝座星也白虎通取伏羲神農祝融為三皇帝者天之一名所以名帝帝者諦也言天蕩然無心忘於物我公平通遠舉事審諦故謂之帝也帝號同天名所莫加而稱皇者以皇是美大之名言大於帝也}
乃更號曰皇帝命為制令為詔【師古曰天子之言一曰制書二曰詔書制書謂其制度之命也如淳曰詔告也自秦漢以上唯天子得稱之更工衡翻}
自稱曰朕【古者君臣之間通稱曰朕自秦定制惟天子獨稱之}
追尊莊襄王為太上皇【太上者極尊之稱也始皇自號曰始皇帝故追尊莊襄王為太上皇自漢高帝以尊太公此後不復為追號}
制曰死而以行為諡則是子議父臣議君也甚無謂自今以來除諡法【周公作諡法緣行之美惡以立諡如幽厲之君雖孝子慈孫百世不能改也今秦除之畏後人加已以惡諡也諡神志翻}
朕為始皇帝後世以計數【史記正義數色主翻}
二世三世至于萬世傳之無窮初齊威宣之時鄒衍論著終始五德之運【所謂終始五德之運}


【者伏羲以木德王木生火故神農以火德王火生土故黄帝以土德王土生金故少昊以金德王金生水故顓頊以水德王水生木故帝嚳又以木德王木又生火故帝堯以火德王火又生土故帝舜以土德王土又生金故夏以金德王金又生水故商以水德王水又生木故周以木德王此五德之終而復始也鄒衍以為周得火德蓋以火流王屋為周受命之符且服色尚赤故也就衍之說以為終始秦當以土為行今始皇以水勝火自以水為行所謂推五勝也漢初以土為行蓋亦祖行之說也}
及始皇并天下齊人奏之始皇采用其說以為周得火德秦代周從所不勝為水德始改年朝賀皆自十月朔衣服旌旄節旗皆尚黑數以六為紀【改年句斷夏以建寅之月為歲首殷以建丑之月為歲首周以建子之月為歲首今始皇以建亥之月為歲首是改年也自此紀年皆以十月為歲首朝賀以十月朔以水為行故色尚黑水成數六故以六為紀}
丞相綰言燕齊荆地遠【避莊襄王諱故以楚為荆索隱曰丞相綰姓王}
不為置王無以鎮之請立諸子始皇下其議【下遐嫁翻凡自上而下之下皆同音}
廷尉斯曰【班書百官表廷尉秦官應劭曰聽獄必質諸朝廷與衆共之兵獄同制故稱廷尉師古曰廷平也冶獄貴平故以為號}
周文武所封子弟同姓甚衆然後屬疏遠相攻擊如仇讐周天子弗能禁止今海内賴陛下神靈一統皆為郡縣諸子功臣以公賦税重賞賜之甚足易制【易以豉翻史記正義音以職翻非也}
天下無異意則安寧之術也置諸侯不便始皇曰天下共苦戰鬬不休以有侯王賴宗廟天下初定又復立國是樹兵也而求其寧息豈不難哉廷尉議是分天下為三十六郡郡置守尉監【裴駰曰三川河東南陽南郡九江鄣郡會稽潁川碭郡泗水薛郡東郡琅邪齊郡上谷漁陽右北平遼西遼東代郡鉅鹿邯鄲上黨太原雲中九原雁門上郡隴西北地漢中巴郡蜀郡黔中長沙凡三十五郡與内史為三十六郡班書百官表郡守掌治其郡郡尉掌佐守典武職甲卒監御史掌監郡守始究翻監去聲康又居銜切余謂守尉監官名也當從去聲若監郡之監則從平聲記王制天子使其大夫為三監監于方伯之國陸德明釋文監古暫翻監於古銜翻可以知矣}
收天下兵聚咸陽銷以為鐘鐻【鐻與虡同音巨虡者所以懸鐘横曰筍植曰虞}
金人十二重各千石置宫庭中【史記正義曰漢書五行志時大人見臨洮長五丈足履六尺皆夷狄服凡十二人故銷兵器鑄而象之所謂金狄也}
一法度衡石丈尺徙天下豪桀於咸陽十二萬戶諸廟及章臺上林皆在渭南【上林在漢長安縣西南秦始起上林苑至漢武帝又增而廣之}
每破諸侯寫放其宫室作之咸陽北阪上【程大昌雍錄曰咸陽北阪漢武帝别名渭城阪即九嵕諸山麓也}
南臨渭自雍門以東至涇渭殿屋複道周閣相屬【徐廣曰雍門在高陵縣史記正義曰在今岐州雍縣東余按班志高陵縣屬左馮翊左輔都尉治焉雍縣屬右扶風二說相去何逹也三輔黄圖曰長安城西出北頭第一門曰雍門木名西城門但長安本秦離宫秦之咸陽則漢扶風之渭城也渭城與長安相去雖不遠然秦時長安未有十二門也豈作史者因漢之雍門而書之與涇渭言涇渭之交也複與復同音方目翻複道閣道也上下有道故謂之複}
所得諸侯美人鐘鼓以充入之

二十七年始皇巡隴西北地至雞頭山過回中焉【范史隗囂傳王孟塞雞頭道賢注曰在原州高平縣西括地志成州上禄縣東北二十里有雞頭山應劭曰回中在安定高平孟康曰回中在北地賢曰回中在汧括地志回中宫在雍西四十里史記正義曰言始皇西巡出隴右向西北出寧州西南行至成州出雞頭山東還過岐州之回中宫也余謂上書巡隴西北地則先至原州之雞頭山而還過回中道里為順若出成州之雞頭則須先過回中而後至雞頭以書法之前後觀之居然可見}
作信宫渭南已更命曰極廟【作宫已成而更名也索隱曰言為宫廟象天極故曰極廟天官書中宫曰天極是也}
自極廟道通驪山作甘泉前殿築甬道自咸陽屬之治馳道於天下【三輔黄圖曰甘泉宫一名雲陽宫關輔記曰林光宫一曰甘泉宫始皇造在今池陽縣西故甘泉山宫周匝十餘里漢武帝廣之周十九里又黄圖曰咸陽北至九嵕甘泉南至鄠杜東至河西至汧渭之交東西八百里南北四百里離宫别館聨望相屬甬道唐夾城之類也應劭曰築垣牆如街巷甬余隴翻賈山曰秦為馳道於天下東窮燕齊南極吳楚江湖之上瀕海之觀畢至道廣五十步三丈而樹厚築其外隱以金椎樹以青松應劭曰馳道天子所行道也若今之中道孔穎達曰馳道如今御路也是君馳走車馬之處故曰馳道屬之欲翻}


二十八年始皇東行郡縣上鄒嶧山立石頌功業【班志魯國鄒縣嶧山在北應劭曰邾文公遷于繹即此括地志鄒嶧山在兖州鄒縣南二十二里嶧音亦}
於是召集魯儒生七十人【孔穎達曰儒之言優也柔也能安人能服人又儒者濡也以先王之道能濡其身}
至泰山下議封禪諸儒或曰古者封禪為蒲車惡傷山之土石草木掃地而祭席用葅稭議各乖異始皇以其難施用由此絀儒生【括地志泰山在兖州博城縣西北三十里一曰岱宗服䖍曰封者增山之高禪廣地也張晏曰天高不可及於泰山上立封禪以祭之冀近神靈也項威曰封泰山告太平升中和之氣於天祭土為封謂負土於泰山為壇而祭也除地為墠後改墠為禪晉太康地記曰為壇於泰山以祭天示增高也為墠於梁父以祭地示廣也白虎通曰王者易姓而起必升封於泰山之上者何因高告高順其類故升封者增高也下禪梁父之基廣厚也刻石紀號者著已之功迹以自勸也增太山之高以報天附梁父之基以報地惡烏路翻師古曰蒲車以蒲裹輪葅稭班志作苴稭如淳曰苴讀如租稭讀曰戛晉灼曰苴藉也師古曰茅藉也苴本作葅假借用應劭曰稭藁本去皮以為席絀與黜同黜退也}
而遂除車道上自太山陽至顛立石頌德從隂道下禪於梁父【師古曰山南曰陽山北曰隂班志泰山郡有梁父縣師古注曰以山名縣括地志梁父山在兖州泗水縣北八十里父音甫}
其禮頗采太祝之祀雍上帝所用【班表奉常之屬有雍太祝令丞蓋漢仍秦制也秦作畤於雍以祀上帝今采其禮以為封禪禮}
而封藏皆祕之世不得而記也於是始皇遂東游海上行禮祠名山大川及八神【封禪書八神一曰天主祠天齊淵水二曰地主祠太山梁父三曰兵主祠蚩尤四曰隂主祠三山五曰陽主祠之罘山六曰月主祠之萊山七曰日主祠成山八曰四時主祠琅邪或曰八神齊自太公以來祠之}
始皇南登琅邪大樂之留三月作琅邪臺立石頌德明得意【班志琅邪郡有琅邪縣山海經琅邪臺在渤海間琅邪之東郭璞曰琅邪臨海邊有山曰琅邪臺越王句踐徙琅邪作觀臺以望東海史記曰始皇徙三萬家於臺下是其所作因越之舊也括地志琅邪山在密州諸城縣東南百四十里始皇立層臺于山上謂之琅邪臺邪音耶大樂之樂琅邪之風景也樂音洛}
初燕人宋毋忌羨門子高之徒稱有僊道形解【解佳買翻}
銷化之術燕齊迂怪之士皆爭傳習之【道經月中仙人宋毋忌白凙圖云火之精曰宋毋忌蓋其人火仙也張曰羨門子高仙人居碣石山上服䖍曰形解尸解也張晏曰人老而解去故骨如變化也今山中有龍骨世謂之龍解骨化去迂羽俱翻又憂俱翻}
自齊威王宣王燕昭王皆信其言使人入海求蓬萊方丈瀛洲云此三神山在渤海中去人不遠患且至則風引船去嘗有至者諸僊人及不死之藥皆在焉及始皇至海上諸方士齊人徐市等爭上書言之【太史公曰嬴姓分封者為徐氏姓譜曰皋陶子伯益佐禹有功封其子若木於徐}
請得齊戒與童男女求之【齊戒之齊讀曰齊}
於是遣徐市發童男女數千人入海求之船交海中皆以風為解【師古曰自解說云為風不得而至自解猶今言分疏}
曰未能至望見之焉始皇還過彭城【班志楚國冇彭城縣古彭祖國}
齊戒禱祠欲出周鼎泗水【水經泗水出魯國卞縣北山東南過彭城縣又東過下邳縣入淮時人相傳以為宋太丘社亡而周鼎没于泗水中故祠泗水欲出周鼎}
使千人沒水求之弗得乃西南渡淮水【水經淮水出南陽郡平氏縣桐柏山東南至淮陵入海行三千餘里}
之衡山南郡【班志衡山在長沙國湘南縣之東南括地志衡山一名岣嶁山在衡州湘潭縣西四十一里漢衡山國在江北秦拔楚郢置南郡唐為荆州江陵府之往也}
浮江至湘山祠逢大風幾不能渡【幾居依翻}
上問博士曰湘君何神對曰聞之堯女舜之妻葬此【班志湘水出零陵郡零陵縣陽朔山北至入江括地志黄陵廟在岳州湘隂縣北五十七里舜二妃之神二妃冢在湘隂縣一百六十里青草山上盛弘之荆州記青草湖南有青草山湖因山而名舜陟方死于蒼梧二妃死于江湘之間因葬焉博士以儒學為官漢成帝詔曰儒林之官四海淵源宜皆明於古今温故知新通達國體故謂之博士}
始皇大怒使刑徒三千人皆伐湘山樹赭其山【赭音者赤也}
遂自南郡由武關歸 初韓人張良其父祖以上五世相韓【張良大父開地相韓昭侯宣惠王襄哀王父平相釐王悼惠王凡五世}
及韓亡良散千金之產欲為韓報仇【為于偽翻}


二十九年始皇東游至陽武博浪沙中【班志陽武縣屬河南郡有博浪沙索隱曰今浚儀西北四十里有博浪城史記正義曰鄭州陽武縣有博浪沙當官道師古曰狼音浪史記作浪正義音狼}
張良令力士操鐵椎狙擊始皇誤中副車【狙玃屬狙之伺物必伏乘其便而擊之狙擊者謂伏其旁而狙伺以擊之也狙千恕翻又千余翻索隱曰漢官儀天子有屬車即副車奉車郎御而從後余謂副貳也漢有五時副車又在屬車之外}
始皇驚求弗得令天下大索十日【索山客翻}
始皇遂登之罘【班志之罘山在東萊腄縣括地志之罘山在萊州文登縣東北一百八十里罘音浮}
刻石旋之琅邪道上黨入【旋即還字之往也}


三十一年使黔首自實田【二十六年更名民曰黔首孔穎達曰黔黑也凡民以黑巾覆頭故謂之黔首}


三十二年始皇之碣石【班志大碣石山在右北平郡驪成縣西南文穎曰碣石在遼西郡絫縣酈道元曰濡水至絫縣碣石山今於此枕海有石如埇道數十里當山頂有大石如柱形往往而見立于鉅海之中名天橋柱碣音桀}
使燕人盧生求羨門【姓譜姜姓之後封于盧以國為氏}
刻碣石門壞城郭決通隄坊【壞音怪坊讀曰防}
始皇巡北邊從上郡入盧生使入海還因奏錄圖書曰亡秦者胡也【錄圖書如後世識緯之書鄭玄曰胡胡亥秦二世名也秦見圖書而不知此為人名反備此胡}
始皇乃遣將軍蒙恬發兵三十萬人北伐匈奴

三十三年發諸嘗逋亡人贅壻賈人為兵【賈誼曰秦人家貧子壯則出贅師古曰謂之贅壻言其不當出在妻家猶人身之有肬贅也轉貨販易者為商坐市販賣者為賈贅之鋭翻}
略取南越陸梁地【索隱曰謂南方之人其性陸梁故曰陸梁地班表漢高帝功臣有陸量侯須無詔以為列諸侯自置吏今長受令長沙主如淳曰陸量秦始皇本紀所謂陸梁地也}
置桂林南海象郡【桂林因產桂而名合浦以南山間無雜木冬夏長青葉長尺餘文潁曰桂林今鬰林師古曰桂林今桂州界是其地非鬰林也南海郡今廣州茂陵書曰象郡治臨塵去長安萬七千五百里韋昭曰今日南是也}
以讁徙民五十萬人戍五嶺與越雜處【所謂讁戍也晉志曰自北徂南入越之道必由嶺嶠時冇五處故曰五嶺師古曰嶺者西自衡山之南東窮于海一山之限耳而别標名則有五焉裴氏廣州記曰大庾始安臨賀桂陽掲陽為五嶺鄧德明南康記曰大庾嶺一也桂陽騎田嶺二也九真都龐嶺三也臨賀萌渚嶺四也始安越城嶺五也師古以裴說為是蜀注曰大庾嶺在䖍州永明嶺白芒嶺在道州臘嶺在郴州臨源嶺在桂州讁則革翻處昌呂翻}
蒙恬斥逐匈奴收河南地為四十四縣築長城因地形用制險塞起臨洮至遼東延袤萬餘里於是渡河據陽山逶迤而北【師古曰河南地當北地之北黄河之南余按河出積石過金城隴西安定北地郡界皆東北流北過朔方窳渾間方屈而東南流逕高闕南又自臨河縣東逕陽山南徐廣所謂陽山在河北隂山在河南者劉昭曰二山皆屬五原郡西安陽縣班志臨洮縣屬隴西郡洮水出西羌中北至枹罕東入河縣臨洮水因以為名洮土刀翻延長行也南北曰袤袤音茂逶於為翻迤以支翻}
㬥師於外十餘年蒙恬常居上郡統治之威振匈奴【㬥讀如字劉伯莊音僕括地志上郡故城在綏州上縣東南五十里}
三十四年謫治獄吏不直及覆獄故失者【覆獄者奏當已成而覆按之也故者知其當罪與不當罪而故出入之失者誤出入也}
築長城及處南越地【處昌呂翻}
丞相李斯上書曰異時諸侯並爭厚招游學今天下已定法令出一百姓當家則力農工士則學習法令今諸生不師今而學古以非當世惑亂黔首相與非法教人聞令下則各以其學議之入則心非出則巷議誇主以為名異趣以為高率羣下以造謗如此弗禁則主勢降乎上黨與成乎下禁之便臣請史官非秦記皆燒之【此燒列國史記也}
非博士官所職天下有藏詩書百家語者皆詣守尉雜燒之【秦之焚書焚天下之人所藏之書耳其博士官所藏則故在項羽燒秦宫室始併博士所藏者焚之此所以後之學者咎蕭何不能於收秦圖書之日併收之也}
有敢偶語詩書棄市以古非今者族吏見知不舉與同罪令下三十日不燒黥為城旦【應劭曰城旦旦起行治城四歲刑也}
所不去者醫藥卜筮種樹之書若有欲學法令者以吏為師制曰可魏人陳餘謂孔鮒曰秦將滅先王之籍而子為書籍之主其危哉子魚曰吾為無用之學知吾者惟友秦非吾友吾何危哉吾將藏之以待其求求至無患矣【孔鮒孔子八世孫字子魚鮒音附}


三十五年使蒙恬除直道道九原抵雲陽【班志雲陽縣屬馮翊}
塹山堙谷【塹七豔翻堙音因}
千八百里數年不就 始皇以為咸陽人多先王之宫庭小乃營作朝宫渭南上林苑中先作前殿阿房【師古曰阿近也以其去咸陽近且號阿房索隱曰此以形名宫也言其宫四阿房廣也三輔黄圖曰作宫阿基旁天下謂之阿房括地志秦阿房宫亦曰阿城在雍州長安縣西一十四里史記正義曰按宫在上林苑中雍州郭城西南面即阿房宮城東南面也房白郎翻}
東西五百步南北五十丈上可以坐萬人下可以建五丈旗周馳為閣道自殿下直抵南山【關中有南山北山自甘泉連延至嶻嶭九嵕為北山自終南太白連延至商嶺為南山}
表南山之顛以為闕為複道自阿房度渭屬之咸陽以象天極閣道絶漢抵營室也【天官書曰天極後十七星絶漢抵營室曰閣道北辰為天極營室二星天子之宫也}
隱宫徒刑者七十萬人【史記正義曰餘刑見於市朝宫刑一百日隱於䕃室養之乃可故曰隱宫下蠶室是也徒刑者有罪既加刑復罰作之也}
乃分作阿房宫或作驪山發北山石椁寫蜀荆地材【康曰寫四夜切舍車解馬為寫或作卸余謂此非舍車解馬之卸即前寫放宫室之寫讀如字}
皆至關中計宫三百【或曰皆至當屬上句關中記云東自函關弘農郡靈寶縣界西至隴關汧陽郡汧源縣界二關之間謂之關中東西千餘里}
關外四百餘於是立石東海上朐界中以為秦東門【班志東海郡朐縣始皇立石海上以為東門闕朐音劬}
因徙三萬家驪邑五萬家雲陽皆復不事十歲【復扶目翻除也不事者不供征役之事}
盧生說始皇曰方中人主時為微行以辟惡鬼【惡鬼謂羣}


【邪也}
惡鬼辟【辟讀曰避}
真人至願上所居宫毋令人知然後不死之藥殆可得也始皇曰吾慕真人自謂真人不稱朕【康曰稱去聲不稱不惬意也余謂康說非也始皇初并天下自稱曰朕至此不稱朕耳}
乃令咸陽之旁二百里内宫觀二百七十複道甬道相連帷帳鐘鼓美人充之各案署不移徙所行幸有言其處者罪死始皇幸梁山宫【班志梁山宫在扶風好畤縣括地志俗名望宫山在雍州好畤縣西十二里北去梁山九里雍錄曰唐奉天縣有梁山秦之梁山宫正在其地}
從山上見丞相車騎衆弗善也中人或告丞相丞相後損車騎始皇怒曰此中人泄吾語案問莫服捕時在旁者盡殺之自是後莫知行之所在羣臣受決事者悉於咸陽宫侯生盧生相與譏議始皇因亡去始皇聞之大怒曰盧生等吾尊賜之甚厚今乃誹謗我【誹敷尾翻}
諸生在咸陽者吾使人亷問或為妖言以亂黔首【亷察也秦有誹謗妖言之罪漢除之妖於遥翻}
於是使御史悉案問諸生【秦置御史掌討姦猾治大獄御史大夫統之}
諸生傳相告引【傳相告引者謂甲引乙乙復引丙也傳株戀翻相如字}
乃自除犯禁者四百六十餘人皆阬之咸陽使天下知之以懲後益發謫徙邊始皇長子扶蘇諫曰諸生皆誦法孔子【誦孔子之言以為法也}
今上皆重法繩之臣恐天下不安始皇怒使扶蘇北監蒙恬軍於上郡【為胡亥奪嫡殺扶蘇張本}


三十六年有隕石于東郡【東郡本衛地秦徙衛于野王以其地置東郡}
或刻其石曰始皇死而地分始皇使御史逐問莫服盡取石旁居人誅之燔其石【燔音煩爇也}
遷河北榆中三萬家【河北北河之北也}
賜爵一級

三十七年冬十月癸丑始皇出游左丞相斯從右丞相去疾守【去疾姓馮從才用翻守手又翻}
始皇二十餘子少子胡亥最愛請從上許之十一月行至雲夢望祀虞舜於九疑山【古者天子巡狩所至山川之神各以秩次望祭之酈道元曰營水出營陽郡冷道縣南留山西流逕九疑山其山磐碁蒼梧之野峰秀數郡之間羅巖九舉各導一溪岫壑負阻異嶺同勢遊者疑焉故曰九疑括地志九疑山在永州唐興縣東南百里其山九峰相似故名元次山曰九疑山在永州方四千里四州各近一隅九域志曰九疑在道州舜陵在女英峯下九疑之第六峰也太史公曰舜南狩崩于蒼梧之野歸葬于江南九疑山山海經曰舜之所葬在今道州零陵界則蒼梧九疑兩處也合而言之者誤也}
浮江下觀籍柯渡海渚過丹陽至錢唐【史記正義曰括地志海渚云在舒州同安縣東按舒州在江之中流疑海字誤籍秦昔翻柯音歌班志丹陽縣秦屬鄣郡括地志丹陽故城在潤州江寧縣東南五里班志錢唐縣屬會稽郡漢西部都尉所治唐為杭州治所}
臨浙江水波惡乃西百二十里從陿中渡【所謂水波惡處則今之由錢唐渡西陵者是也西陿中渡則今富陽分水之間徐廣曰蓋在餘杭也顧夷曰餘杭秦始皇至會稽經此立為縣}
上會稽【班志會稽山在會稽郡山隂縣南有禹冢禹井}
祭大禹望于南海立石頌德還過吳從江乘渡【江乘縣秦屬鄣郡漢屬丹陽郡括地志江乘故縣城在今潤州句容縣北六十里}
並海上北至琅邪之罘【並步浪翻罘音浮}
見巨魚射殺之遂並海西至平原津而病【平原縣秦屬齊郡漢分置平原郡史記正義曰今德州平原縣南六十里有張公故城城東有津後名張公渡恐此平原郡古津也漢書平津侯公孫弘所封亦近此蓋平津即此津余按公孫弘傳封渤海高城縣之平津鄉則平津非平原津也班志篤馬河至平原東北入海此蓋津渡處射而亦翻並步浪翻}
始皇惡言死【惡烏路翻}
羣臣莫敢言死事病益甚乃令中車府令行符璽事趙高為書賜扶蘇曰與喪會咸陽而葬書已封在趙高所未付使者【班書百官表太僕秦官其屬有車府令}
秋七月丙寅始皇崩於沙丘平臺【史記正義曰始皇崩在沙丘宫平臺之中}
丞相斯為上崩在外【為于偽翻}
恐諸公子及天下有變乃祕之不發喪棺載輼凉車中【文穎曰輼輬車如今喪轜車也孟康曰如衣車有窻牖閉之則温開之則凉故名如淳曰輼輬車其車廣大有羽飾沈約宋書禮志曰漢制大行載輼輬車四輪其飾如金根加施組連璧交路四角金龍飾銜璧垂五采飾羽流蘇前後雲畫帷裳文畫曲轓長與車等太僕御駕六白駱馬以黑藥灼其身為虎文史記正義曰棺音館又古玩翻輼音温凉一作輬音同}
故幸宦者驂乘所至上食百官奏事如故宦者輒從車中可其奏事獨胡亥趙高及幸宦者五六人知之初始皇尊寵蒙氏信任之蒙恬任外將蒙毅常居中參謀議名為忠信故雖諸將相莫敢與之争【將即亮翻}
趙高者生而隱宫【康曰餘刑顯於市朝宫刑在於隱室故曰隱宫}
始皇聞其彊力通於獄法舉以為中車府令使教胡亥決獄胡亥幸之趙高有罪始皇使蒙毅治之毅當高法應死始皇以高敏於事赦之復其官趙高既雅得幸於胡亥【雅素也}
又怨蒙氏乃說胡亥請詐以始皇命誅扶蘇而立胡亥為太子胡亥然其計趙高曰不與丞相謀恐事不能成乃見丞相斯曰上賜長子書及符璽皆在胡亥所【長子謂扶蘇}
定太子在君侯與高之口耳事將何如斯曰安得亡國之言此非人臣所當議也高曰君侯材能謀慮功高無怨長子信之此五者皆孰與蒙恬斯曰不及也高曰然則長子即位必用蒙恬為丞相君侯終不懷通侯之印歸鄉里明矣【通侯漢曰徹侯亦曰列侯應劭曰通亦徹也通者言功德通於王室也張晏曰列侯者見序列也}
胡亥慈仁篤厚可以為嗣願君審計而定之丞相斯以為然乃相與謀詐為受始皇詔立胡亥為太子更為書賜扶蘇【更工衡翻改也}
數以不能闢地立功士卒多耗【數所阻翻}
數上書直言誹謗日夜怨望不得罷歸為太子將軍恬不矯正知其謀皆賜死以兵屬禆將王離【數所角翻下同屬之欲翻付也康音蜀非下以屬同}
扶蘇發書泣入内舍欲自殺蒙恬曰陛下居外未立太子使臣將三十萬衆守邊公子為監此天下重任也今一使者來即自殺安知其非詐復請而後死未暮也使者數趣之扶蘇謂蒙恬曰父賜子死尚安復請即自殺【趣讀曰促復扶又翻}
蒙恬不肯死使者以屬吏繫諸陽周【班志陽周縣屬上郡史記正義曰陽周寧州羅川縣之邑屬之欲翻今按天寶元年改羅川縣為真寧縣}
更置李斯舍人為護軍【班百官表護軍都尉秦官又漢王以陳平為護軍中尉盡護諸將當是時恬已屬吏恐其軍有變故以李斯舍人為護軍使之護諸將也}
還報胡亥巳聞扶蘇死即欲釋蒙恬會蒙毅為始皇出禱山川還至趙高言於胡亥曰先帝欲舉賢立太子久矣而毅諫以為不可不若誅之乃繫諸代【據地理代距沙丘甚遠蓋毅還至代即就繫之}
遂從井陘抵九原【班志井陘在常山石邑縣西史記正義曰井陘故關在并州石艾縣東十八里即井陘口}
會暑輼車臭乃詔從官令車載一石鮑魚以亂之【孟康曰百二十斤曰石班書貨殖傳鯫鮑千釣師古注曰鯫膊魚也即今之不著鹽而乾者也鮑今之䱒魚也而說者乃讀鮑為鮠魚之鮠失義遠矣鄭康成以䱒於煏室乾之亦非也煏室乾之即鮑耳蓋今巴荆人所呼鰎魚者是也秦皇載鮑亂臭者則是䱒魚耳而煏室乾者本不臭也鮑白卯翻鯫音接䱒於業翻鮠五回翻煏蒲北翻鰎居偃翻}
從直道至咸陽發喪【直道即三十五年蒙恬所除者}
太子胡亥襲位九月葬始皇於驪山下錮三泉【師古曰三重之泉言至水也余謂錮者冶銅錮塞之也三泉者取九泉之數言之}
奇器珍怪徙藏滿之【謂徙府庫之物以實陵便房中藏才浪翻}
令匠作機弩有穿近者輒射之以水銀為百川江河大海機相灌輸【康注引劉伯莊云機相灌輸以防穿近者余按文勢自機弩至輒射之丈意已足機相灌輸是承水銀為百川江河大海之意作如是觀文意甚順射而亦翻史記正義灌音館輸音戌}
上具天文下具地理後宫無子者皆令從死【從才用翻}
葬既已下或言工匠為機藏皆知之藏重即泄大事盡閉之墓中【藏重即泄謂工匠若更為第二重機藏與外人近即泄其所以為機藏之事故大事盡則皆閉之墓中大事盡句絶謂既下窆則迀終之大事盡也重直龍翻}
二世欲誅蒙恬兄弟二世兄子子嬰諫曰趙王遷殺

李牧而用顔聚齊王建殺其故世忠臣而用后勝卒皆亡國【二事並見前卒子恤翻}
蒙氏秦之大臣謀士也而陛下欲一旦棄去之誅殺忠臣而立無節行之人【行下孟翻}
是内使羣臣不相信而外使鬬士之意離也二世弗聽遂殺蒙毅及内史恬恬曰自吾先人及至子孫積功信於秦三世矣【恬祖鷔父武及恬三世皆事秦有功}
今臣將兵三十餘萬身雖囚繫其勢足以倍畔【倍蒲妹翻}
然自知必死而守義者不敢辱先人之教以不忘先帝也乃吞藥自殺

揚子法言曰或問蒙恬忠而被誅忠奚可為也曰塹山堙谷起臨洮擊遼水力不足而屍有餘忠不足相也【相息亮翻}


臣光曰始皇方毒天下而蒙恬為之使恬不仁可知矣然恬明於為人臣之義雖無罪見誅能守死不貳斯亦足稱也【使如字}


二世皇帝上【諱胡亥始皇少子也}


元年冬十月戊寅大赦 春二世東行郡縣李斯從到碣石並海南至會稽而盡刻始皇所立刻石旁著大臣從者名【行下孟翻從才用翻並步浪翻著如字史記正義音丁略翻}
以章先帝成功盛德而還夏四月二世至咸陽謂趙高曰夫人生居世間也譬猶騁六驥過決隙也【康曰上音缺下丘逆翻余謂決如字決裂也裂開之隙其間不能以寸喻狹小也}
吾旣已臨天下矣欲悉耳目之所好窮心志之所樂【好呼到翻樂音洛}
以終吾年壽可乎高曰此賢主之所能行而昏亂主之所禁也雖然有所未可臣請言之夫沙丘之謀諸公子及大臣皆疑焉而諸公子盡帝兄大臣又先帝之所置也今陛下初立此其屬意怏怏皆不服恐為變臣戰戰栗栗惟恐不終陛下安得為此樂乎二世曰為之奈何趙高曰陛下嚴法而刻刑令有罪者相坐誅滅大臣及宗室然後收舉遺民貧者富之賤者貴之盡除先帝之故臣更置陛下之所親信者此則隂德歸陛下害除而姦謀塞羣臣莫不被潤澤蒙厚德陛下則高枕肆志寵樂矣【更工衡翻塞悉則翻枕之鴆翻}
計莫出於此二世然之乃更為法律務益刻深大臣諸公子有罪輒下高鞠治之於是公子十二人僇死咸陽市十公主矺死於杜【索隱曰矺貯格翻史記正義音宅與磔同謂磔裂支體而殺之温公類篇音竹格翻磓也杜故周之杜伯國班志杜縣屬京兆宣帝改曰杜陵}
財物入於縣官【漢謂天子為縣官此縣官猶言公家也}
相連逮者不可勝數【言事相連及皆逮之貢父曰其人存直追取之曰逮其人亡則討而捕之逮易辭捕加力也}
公子將閭昆弟三人囚於内宫議其罪獨後二世使使令將閭曰公子不臣罪當死吏致法焉將閭曰闕廷之禮吾未嘗敢不從賓贊也廊廟之位吾未嘗敢失節也受命應對吾未嘗敢失辭也何謂不臣【言不敢挾親親之恩廢為臣之節何得以此罪加之}
願聞罪而死使者曰臣不得與謀【與讀曰預}
奉書從事將閭乃仰天大呼天者三曰吾無罪昆弟三人皆流涕拔劍自殺宗室振恐公子高欲犇恐收族乃上書曰先帝無恙時臣入則賜食出則乘輿御府之衣臣得賜之中廐之寶馬臣得賜之臣當從死而不能為人子不孝為人臣不忠不孝不忠者無名以立於世臣請從死願葬驪山之足惟上幸哀憐之書上二世大說【說讀曰悦}
召趙高而示之曰此可謂急乎趙高曰人臣當憂死不暇何變之得謀二世可其書賜錢十萬以葬復作阿房宫盡徵材士五萬人為屯衛咸陽令教射狗馬禽獸當食者多【食讀曰飤又音祥吏翻}
度不足下調郡縣【史記正義曰下遐嫁翻調徒釣翻謂下郡縣而調發之也余謂下讀如字亦通}
轉輸菽粟芻槀皆令自齎糧食咸陽三百里内不得食其穀 秋七月陽城人陳勝陽夏人吳廣起兵於蘄【史記正義曰即河南陽城縣班志屬潁川郡陽夏縣屬淮陽國括地志陳州太康縣本漢陽夏縣地盤洲洪氏曰陽夏鄉去太康縣三十里夏音賈班志蘄縣屬沛郡有大澤鄉蘄音渠依翻}
是時閭左戍漁陽【鼂錯曰秦以謫發戌先發吏有謫及贅壻賈人後以嘗有市籍者又後以大父母嘗有市籍者後入閭取其左索隱曰閭左謂居閭里之左也秦時復除者居閭左今力役凡在閭左者盡發之也又云几居以富彊為右貧弱為左秦役戍多富者役盡兼取貧弱而發之也班志漁陽縣屬漁陽郡括地志漁陽故城在檀州密雲縣南十八里在漁水之陽}
九百人屯大澤鄉陳勝吳廣皆為屯長【師古曰人所聚為屯長帥也}
會天大雨道不通度已失期【度徒洛翻}
失期法皆斬陳勝吳廣因天下之愁怨乃殺將尉【師古曰其官本尉耳時領戍人故為將尉索隱曰尉官也漢舊儀大縣三人其尉將屯九百人故云將尉}
召令徒屬曰公等皆失期當斬假令毋斬而戍死者固什六七且壯士不死則已死則舉大名耳王侯將相寧有種乎【種章勇翻}
衆皆從之乃詐稱公子扶蘇項燕【以百姓賢扶蘇而楚人憐項燕也}
為壇而盟稱大楚陳勝自立為將軍吳廣為都尉攻大澤鄉拔之收而攻蘄蘄下【收大澤鄉之兵以攻蘄也}
乃令符離人葛嬰將兵徇蘄以東攻銍酇苦柘譙皆下之【班志符離銍酇譙屬沛郡姓譜葛國旣滅其後以國為氏柘苦二縣屬淮陽國宋白曰柘縣古襄氏之邑春秋時陳之株野漢為柘縣以邑有柘溝而名唐為宋州柘城縣亳州真源縣古苦縣地徇辭峻翻略地也銍竹乙翻酇本作䣜才多翻師古曰此縣本借酇字為之音嵯王莽改縣為贊治則此縣亦冇贊音苦音怙}
行收兵比至陳【比必寐翻}
車六七百乘騎千餘卒數萬人攻陳陳守尉皆不在獨守丞與戰譙門中不勝守丞死陳勝乃入據陳【班志陳縣屬淮陽國史記正義曰今陳州城本楚襄王所築陳國城也師古曰守丞謂郡丞之居守者一曰郡守之丞故曰守丞原父曰秦不以陳為郡何庸有守守謂非正官權守者耳余按秦分天下為郡縣郡置守尉監縣置令丞尉原父以此守為權守之守良是遷固二史作守令皆不在此作守尉皆不在蓋二史令下缺尉字而通鑑尉上缺令字也師古曰譙門謂門上為高樓以候望者耳樓一名譙故謂美麗之樓為麗譙亦呼為巢所巢車者亦於兵車之上為樓以望敵者也譙巢聲相近耳}
初大梁人張耳陳餘相與為刎頸交秦滅魏聞二人魏之名士重賞購求之張耳陳餘乃變名姓俱之陳為里監門以自食【師古曰監門卒之賤者耳餘以卑賤自隱張晏曰監門里正衛也監古銜翻}
里吏嘗以過笞陳餘陳餘欲起張耳躡之使受笞【欲起者不能受辱欲起與吏亢也躡尼輒翻躡其足也笞丑之翻}
吏去張耳乃引陳餘之桑下數之曰【數所具翻又所主翻}
始吾與公言何如今見小辱而欲死一吏乎陳餘謝之陳涉旣入陳張耳陳餘詣門上謁【陳勝字涉}
陳涉素聞其賢大喜陳中豪桀父老請立涉為楚王涉以問張耳陳餘耳餘對曰秦為無道滅人社稷暴虐百姓將軍出萬死之計為天下除殘也今始至陳而王之示天下私願將軍毋王急引兵而西遣人立六國後自為樹黨為秦益敵敵多則力分與衆則兵彊如此則野無交兵【六國皆為與國則兵不交鋒於野矣}
縣無守城【諸縣皆畔秦復為六國無復為秦守城者}
誅暴秦據咸陽以令諸侯諸侯亡而得立以德服之則帝業成矣今獨王陳恐天下懈也陳涉不聽遂自立為王號張楚【劉德曰若云張大楚國也張晏曰先是楚已為秦所滅今立楚為張也}
當是時諸郡縣苦秦法爭殺長吏以應涉謁者從東方來以反者聞二世怒下之吏【下遐嫁翻}
後使者至上問之對曰羣盜鼠竊狗偷郡守尉方逐捕今盡得不足憂也上悦陳王以吳叔為假王監諸將以西擊滎陽【吳廣字叔滎陽縣屬三川郡}
張耳陳餘復說陳王請奇兵北畧趙地【復扶又翻}
於是陳王以故所善陳人武臣為將軍邵騷為護軍【姓譜曰武姓宋武公之後余謂自有諡法以武為諡者多矣而必以武姓為宋武公之後何拘也唐志氏族以為武氏出自姬姓周平王少子生而有文在手曰武遂以為氏此由武后而傅會為之說也趙明誠金石錄有漢敦煌長史武班碑云昔殷王武丁克伐鬼方官族析分因以為氏邵姓周文王子邵公奭之後或言第十一子聃季載之後}
以張耳陳餘為左右校尉予卒三千人徇趙【予讀曰與}
陳王又令汝隂人鄧宗徇九江郡【殷王武丁封叔父于河北是為鄧侯後因氏焉班志云汝隂縣屬汝南郡故胡國九江本楚地秦滅楚分置九江郡項羽滅秦以封黥布者漢高祖更名淮南國後武帝復曰九江郡}
當此時楚兵數千人為聚者不可勝數【師古曰聚才喻翻}
葛嬰至東城【班志東城縣屬九江郡括地志東城故城在濠州定遠縣東南五十里}
立襄彊為楚王【姓譜襄魯莊公子襄仲之後}
聞陳王已立因殺襄彊還報陳王誅殺葛嬰陳王令周市北徇魏地以上蔡人房君蔡賜為上柱國【索隱曰房邑名也爵之於房號曰房君上柱國楚爵之尊者蔡以國為氏也}
陳王聞周文陳之賢人也習兵乃與之將軍印使西擊秦武臣等從白馬渡河【師古曰白馬津在今滑州白馬縣界括地志白馬故城在滑州衛南縣西南三十四里戴延之西征記曰白馬故城即衛之漕邑}
至諸縣說其豪桀豪桀皆應之乃行收兵得數萬人號武臣為武信君下趙十餘城餘皆城守乃引兵東北擊范陽【班志曰范陽縣屬涿郡應劭曰在范水之陽}
范陽蒯徹說武信君曰【蒯徹即蒯過班書避武帝諱改徹為通蒯丘怪翻姓也左傳晉有大夫蒯得}
足下必將戰勝而後略地攻得然後下城臣竊以為過矣誠聽臣之計可不攻而降城不戰而略地傳檄而千里定可乎【師古曰檄者以木簡為書長尺二寸用徵召也有急則加以鳥羽挿之所以示急疾也檄戶歷翻}
武信君曰何謂也徹曰范陽令徐公畏死而貪欲先天下降【先音悉薦翻}
君若以為秦所置吏誅殺如前十城則邊地之城皆為金城湯池不可攻也君若齎臣侯印以授范陽令使乘朱輪華轂驅馳燕趙之郊即燕趙城可無戰而降矣武信君曰善以車百乘騎二百侯印迎徐公燕趙聞之不戰以城下者三十餘城陳王旣遣周章以秦政之亂有輕秦之意不復設備【復音扶又翻}
博士孔鮒諫曰【鮒魏相子順之子孔子八世孫即前藏書者也}
臣聞兵法不恃敵之不我攻恃吾不可攻今王恃敵而不自恃若跌而不振悔之無及也【跌徒結翻踢而踣也}
陳王曰寡人之軍先生無累焉【累音良瑞翻}
周文行收兵至關車千乘卒數十萬至戲軍焉【師古曰戲水名在京兆新豐東今有戲水驛其水本出藍田北界至此而北流入渭蘇林曰戲在新豐東南三十里戲許宜翻}
二世乃大驚與羣臣謀曰奈何少府章邯曰【班表曰少府秦官掌山林池澤之賦以給共養姓譜齊人降子孫去邑為章氏少詩照翻邯下甘翻}
盜已至衆彊今發近縣不及矣驪山徒多【秦之刑徒已論者輸作驪山}
請赦之授兵以擊之二世乃大赦天下使章邯免驪山徒人奴產子悉發以擊楚軍大敗之【服䖍曰人奴產子家人之產奴師古曰奴產子猶今人云家生奴仲馮曰人奴一物產子又一物臣瓚曰人奴之產子今田客家兒也}
周文走張耳陳餘至邯鄲聞周章却又聞諸將為陳王徇地還者多以讒毁得罪誅乃說武信君令自王八月武信君自立為趙王以陳餘為大將軍【班表曰前後左右將軍周末官秦因之位上卿漢大將軍比三公}
張耳為右丞相邵騷為左丞相使人報陳王陳王大怒欲盡族武信君等家而發兵擊趙柱國房君諫曰秦未亡而誅武信君等家此生一秦也不如因而賀之使急引兵西擊秦陳王然之從其計徙繫武信君等家宫中封張耳子敖為成都君使使者賀趙令趣發兵西入關張耳陳餘說趙王曰王王趙非楚意【趣讀曰促上王如字下王于况翻}
特以計賀王楚已滅秦必加兵於趙願王毋西兵北徇燕代南收河内以自廣【燕涿郡以北之地代常山以北之地河内本魏地於時屬河東郡}
趙南據大河北有燕代楚雖勝秦必不敢制趙不勝秦必重趙趙乘秦楚之敝可以得志於天下趙王以為然因不西兵而使韓廣略燕李良略常山張黶略上黨【黶音烏點翻又於琰翻}
九月沛人劉邦起兵於沛【陶唐氏既衰其後有劉累以櫌龍事孔甲為豢龍氏及晉士會自秦歸晉其處者為劉氏師古曰沛本秦泗水郡之屬縣李斐曰沛小沛也索隱曰漢改泗水郡為沛郡治相城故以沛縣為小沛也沛博蓋翻漢高帝事始此}
下相人項梁起兵於吳【班志曰下相縣屬臨淮郡索隱曰案相水名出沛國沛有相縣於相水下流置縣故曰下相也括地志曰下相故城在泗州宿豫縣西北七十里項燕為楚將封於項子孫以邑為氏焉吳縣會稽郡冶所故吳都也}
狄人田儋起兵於齊【服䖍曰儋音負擔之擔師古曰儋音丁甘翻}
劉邦字季為人隆準龍顔左股有七十二黑子【服䖍曰準音拙應劭曰隆高也準頰權凖也顔額顙也李斐曰凖鼻也文穎曰音凖的之凖晉灼曰戰國策云眉目凖頞權衡史記秦始皇蜂目長凖李說文音是也師古曰頰權䪼字豈當借準為之服音應說皆失之黑子今中國通呼為黶子吳楚俗謂之誌誌者記也}
愛人喜施【喜許旣翻施式豉翻}
意豁如也常有大度不事家人生產作業初為泗上亭長【秦法十里一亭亭長主亭之吏亭謂停留客旅宿食之館史記正義曰國語有寓室即今之亭也亭長蓋今之里長民有訟諍吏留平辨得成其政泗上史記作泗水括地志泗水亭在徐州沛縣東一百步冇高祖廟}
單父人呂公好相人見季狀貌奇之以女妻之【班志單父縣屬山陽郡單音善父音甫妻七細翻呂公女是為呂后}
旣而季以亭長為縣送徒驪山徒多道亡自度比至皆亡之【度徒洛翻比必寐翻}
到豐西澤中亭止飲【應劭曰沛縣也豐其鄉也孟康曰後沛為郡而豐為縣師古曰豐本沛之聚邑耳}
夜乃解縱所送徒曰公等皆去吾亦從此逝矣徒中壯士願從者十餘人劉季被酒【師古曰被加也被酒為酒所加也被皮義翻}
夜徑澤中有大蛇當徑季拔劍斬蛇有老嫗哭曰吾子白帝子也化為蛇當道今赤帝子殺之因忽不見【嫗威遇翻老母也應劭曰秦襄公自以居西主少昊之神作西畤祠白帝至獻公時櫟陽兩金又作畦畤祠白帝少昊金德也赤帝堯後謂漢也殺之明漢當代秦}
劉季亡匿於芒碭山澤之間【班志芒縣屬沛郡碭縣屬梁國應劭曰二縣之間有山澤之固故隱其間宋白曰亳州永城縣漢芒縣地括地志宋州碭山縣在州東一百五十里本漢碭縣碭山在縣東芒音忙碭音唐師古又音宕是也}
數有奇怪沛中子弟聞之多欲附者及陳涉起沛令欲以沛應之掾主吏蕭何曹參曰【據曹參傳曰參為掾何為主吏孟康曰主吏功曹也姓譜宋支子食采于蕭後因為氏數所角翻掾于絹翻}
君為秦吏今欲背之【背音蒲妹翻}
率沛子弟恐不聽願君召諸亡在外者可得數百人因劫衆衆不敢不聽乃令樊噲召劉季劉季之衆已數十百人矣沛令後悔恐其有變乃閉城城守【師古曰城守者守其城也音狩後皆類此}
欲誅蕭曹蕭曹恐踰城保劉季【言投劉季以自保也}
劉季乃書帛射城上遺沛父老為陳利害【射而亦翻遺于季翻為于偽翻}
父老乃率子弟共殺沛令開門迎劉季立以為沛公【春秋之時楚僭王號其大夫多封縣公如申公葉公魯陽公之類是也今立季為沛公用楚制也}
蕭曹等為收沛子弟得三千人以應諸侯項梁者楚將項燕子也嘗殺人與兄子籍避仇吳中吳中賢士大夫皆出其下籍少時學書不成去學劒又不成項梁怒之籍曰書足以記名姓而已劒一人敵不足學學萬人敵於是項梁乃教籍兵法籍大喜略知其意又不肯竟學籍長八尺餘力能扛鼎【韋昭曰扛舉也索隱曰說文云扛横關對舉也長真亮翻扛音江}
才器過人會稽守殷通【徐廣曰爾時未言太守余謂戰國之時郡守只稱守景帝中二年七月始曰太守姓譜武王克商子孫分散以殷為氏守式又翻下同}
聞陳涉起欲發兵以應涉使項梁及桓楚將【將音即亮翻}
是時桓楚亡在澤中梁曰桓楚亡人莫知其處獨籍知之耳梁乃誡籍持劔居外梁復入與守坐曰請召籍使受命召桓楚守曰諾梁召籍入須臾梁眴籍曰可行矣【眴音舜動目而使之也}
於是籍遂拔劒斬守頭項梁持守頭佩其印綬【釋名印信也所以封物以為驗也亦言因也封物相因付也綬受也繫印之組也以相授受也應劭漢官儀曰綬長丈二尺法十二月廣三尺法天地人}
門下大驚擾亂籍所擊殺數十百人【言所殺自數十至百人也}
一府中皆慴伏莫敢起【說文曰慴失氣也音之涉翻}
梁乃召故所知豪吏諭以所為起大事遂舉吳中兵使人收下縣【下縣會稽管下諸縣也師古曰非郡所都故謂之下也}
得精兵八千人梁為會稽守籍為禆將徇下縣籍是時年二十四【項籍始于此}
田儋故齊王族也儋從弟榮榮弟横皆豪健宗彊能得人【從才用翻}
周市徇地至狄【周市魏人}
狄城守田儋詳為縛其奴從少年之廷欲謁殺奴【詳讀曰佯詐也應劭曰古殺奴婢皆當告官儋欲殺令故詐縳奴以謁也廷縣廷也師古曰廷音定}
見狄令因擊殺令而召豪吏子弟曰諸侯皆反秦自立齊古之建國也儋田氏當王遂自立為齊王發兵以擊周市周市軍還去田儋率兵東略定齊地韓廣將兵北徇燕燕地豪桀欲共立廣為燕王廣曰廣母在趙不可燕人曰趙方西憂秦南憂楚其力不能禁我且以楚之彊不敢害趙王將相之家趙獨安敢害將軍家乎韓廣乃自立為燕王居數月趙奉燕王母家屬歸之趙王與張耳陳餘北畧燕界趙王間出【師古曰謂投間隙而微出也}
為燕軍所得燕囚之欲求割地使者往請燕輒殺之有厮養卒走燕壁【如淳曰厮賤者也公羊傳曰厮役扈養韋昭曰析薪為厮炊烹為養厮音斯養羊尚翻}
見燕將曰君知張耳陳餘何欲曰欲得其王耳趙養卒笑曰君未知此兩人所欲也夫武臣張耳陳餘杖馬箠【杖直亮翻箠止蘂翻馬檛也}
下趙數十城此亦各欲南面而王豈欲為將相終已耶顧其勢初定未敢參分而王【參猶三也}
且以少長先立武臣為王以持趙心【少詩照翻長知兩翻}
今趙地已服此兩人亦欲分趙而王時未可耳今君乃囚趙王此兩人名為求趙王實欲燕殺之此兩人分趙自立夫以一趙尚易燕【易弋䜴翻}
況以兩賢王左提右挈而責殺王之罪滅燕易矣燕將乃歸趙王養卒為御而歸 周市自狄還至魏地欲立故魏公子甯陵君咎為王【甯陵即漢之寜陵縣屬陳留郡括地志曰宋川寧陵城古甯陵也}
咎在陳不得之魏魏地已定諸侯皆欲立周市為魏王市曰天下昏亂忠臣乃見【見賢遍翻}
今天下共畔秦其義必立魏王後乃可諸侯固請立市市終辭不受迎魏咎於陳五反陳王乃遣之立咎為魏王市為魏相 是歲二世廢衛君角為庶人衛絶祀【周之列國衛最後亡}


資治通鑑卷七  














































































































































