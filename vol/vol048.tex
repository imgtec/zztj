資治通鑑卷四十八   宋 司馬光 撰

胡三省 音註

漢紀四十|{
	起玄黓執徐盡旃蒙大荒落凡十四年}


孝和皇帝下

永元四年春正月遣大將軍左校尉耿夔授於除鞬印綬|{
	校戶教翻鞬九言翻}
使中郎將任尚持節衛護屯伊吾如南單于故事|{
	任音壬}
初廬江周榮辟袁安府安舉奏竇景|{
	事見上卷元年}
及争立北單于事|{
	見上卷上年}
皆榮所具草竇氏客太尉掾徐齮深惡之|{
	掾俞絹翻齮魚倚翻惡烏路翻}
脅榮曰子為袁公腹心之謀排奏竇氏竇氏悍士刺客滿城中謹備之矣|{
	悍下罕翻又侯旰翻}
榮曰榮江淮孤生得備宰士|{
	賢曰榮辟司徒府故稱宰士}
縱為竇氏所害誠所甘心因敇妻子|{
	敕戒也}
若卒遇飛禍|{
	卒讀曰猝賢曰飛禍言倉卒而死也余謂飛禍者言刺客竊發不可得而備若鳥之飛集也}
無得殯斂|{
	斂力贍翻}
冀以區區腐身覺悟朝廷 三月癸丑司徒袁安薨閏月丁丑以太常丁鴻為司徒 夏四月丙辰竇憲

還至京師|{
	還從宣翻又如字}
六月戊戍朔日有食之丁鴻上疏曰昔諸呂擅權統嗣幾移|{
	事見高后紀幾居希翻}
哀平之末廟不血食|{
	事見王莽紀鴻引此事以指言外戚之禍}
故雖有周公之親而無其德不得行其埶也|{
	賢曰言親賢兼重方可執政}
今大將軍雖欲敕身自約不敢僭差然而天下遠近皆惶怖承旨|{
	怖普布翻}
刺史二千石初除謁辭求通待報|{
	初除而謁之官則辭求通者求通名也待報者得謁與不得謁得辭與不得辭皆待報也}
雖奉符璽受臺敕|{
	符璽所以為信初除者詣尚書臺受敕璽斯氏翻}
不敢便去久者至數十日背王室|{
	背蒲妹翻}
向私門此乃上威損下權盛也人道悖於下效驗見於天雖有隱謀神照其情垂象見戒以告人君|{
	悖蒲内翻見賢遍翻}
禁微則易|{
	易以䜴翻}
救末則難人莫不忽於微細以致其大恩不忍誨義不忍割去事之徵未然之明鏡也|{
	言禍伏於隱微人多忽之及發見之後昭昭而不可掩是為未然之明鏡}
夫天不可以不剛不剛則三光不明王不可以不彊不彊則宰牧從横|{
	從子用翻又子容翻横戶孟翻又如字}
宜因大變改政匡失以塞天意|{
	塞悉則翻}
丙辰郡國十三地震 旱蝗 竇氏父子兄弟並為卿校|{
	卿九卿校諸校尉校戶教翻}
充滿朝廷穰侯鄧疊疊弟步兵校尉磊及母元憲女壻射聲校尉郭舉舉父長樂少府璜共相交結|{
	賢曰太后居長樂宫故有少府秩二千石樂音洛}
元舉並出入禁中舉得幸太后遂共圖為殺害|{
	謀弑逆也}
帝隂知其謀是時憲兄弟專權帝與内外臣僚莫由親接所與居者閹宦而已|{
	閹宦周禮謂之奄鄭玄注曰奄精氣蔽藏者今謂之宦人閹衣廉翻又衣檢翻}
帝以朝臣上下莫不附憲獨中常侍鉤盾令鄭衆謹敏有心幾|{
	百官志鉤盾令秩六百石宦者為之典諸近池苑囿遊觀之處屬少府幾事也心幾謂心事也今人謂人胷中有城府者為有心事朝直遥翻盾食尹翻幾居希翻}
不事豪黨遂與衆定議誅憲以憲在外|{
	謂出屯凉州時也}
慮其為亂忍而未發會憲與鄧疊皆還京師|{
	還從宣翻又如字}
時清河王慶恩遇尤渥|{
	渥厚漬也}
常入省宿止|{
	省禁中也}
帝將發其謀欲得外戚傳|{
	賢曰前書外戚傳傳直戀翻}
懼左右不敢使令慶私從千乘王求|{
	千乘王伉帝長兄也乘繩證翻}
夜獨内之又令慶傳語鄭衆求索故事|{
	賢曰謂文帝誅薄昭武帝誅竇嬰故事索山客翻}
庚申帝幸北宫詔執金吾五校尉勒兵屯衛南北宫|{
	執金吾掌宫外戒司非常北軍五校尉主五營士故令勒兵屯衛}
閉城門收捕郭璜郭舉鄧疊鄧磊皆下獄死|{
	下遐稼翻}
遣謁者僕射收憲大將軍印綬更封為冠軍侯|{
	憲先已封冠軍侯不受今復封以侯就國更居孟翻}
與篤景皆就國|{
	古回翻}
帝以太后故不欲名誅憲|{
	言不欲正名誅之}
為選嚴能相督察之|{
	為于偽翻}
憲篤景到國皆廹令自殺初河南尹張酺數以正法繩治竇景|{
	酺薄乎翻酺先為魏郡太守郡人鄭據奏竇景罪景遣掾夏猛私謝酺使罪據子酺收猛繫獄及入為河南尹景家人擊傷市卒吏捕得之景怒遣緹騎侯海毆傷市丞酺部吏楊章窮究正海罪徙朔方數所角翻治直之翻}
及竇氏敗酺上疏曰方憲等寵貴羣臣阿附唯恐不及皆言憲受顧命之託懷伊呂之忠至乃復比鄧夫人於文母|{
	賢曰案鄧夫人即穰侯鄧疊母元張酺論憲兼及其黨稱鄧夫人猶如前書霍光妻稱霍顯祁大伯母號祁夫人之類復扶又翻}
今嚴威既行皆言當死不顧其前後考折厥衷|{
	折之舌翻衷竹仲翻}
臣伏見夏陽侯瓌每存忠善前與臣言常有盡節之心檢敇賓客未嘗犯法臣聞王政骨肉之刑有三宥之義|{
	禮記公族有罪獄成有司讞于公曰某之罪在大辟公曰宥之有司又曰在大辟公又曰宥之有司又曰在大辟公又曰宥之及三宥不對走出致刑于甸人公又使人追之曰必宥之有司對曰無及也反命於公公素服如其倫之喪}
過厚不過薄今議者欲為選嚴能相|{
	為于偽翻相息亮翻侯國相也}
恐其廹切必不完免宜裁加貸宥以崇厚德帝感其言由是獨得全竇氏宗族賓客以憲為官者皆免歸故郡初班固奴嘗醉罵洛陽令种兢|{
	姓譜种本仲氏避難改焉}
兢因逮考竇氏賓客收捕固死獄中固嘗著漢書尚未就詔固女弟曹壽妻昭踵而成之|{
	昭即曹大家也}


華嶠論曰固之序事不激詭不抑抗|{
	賢曰激揚也詭毁也抑退也抗進也余謂激詭抑抗皆指史家作意以為文之病華戶化翻}
贍而不穢詳而有體使讀之者亹亹而不厭|{
	爾雅曰亹亹猶勉勉也音無匪翻}
信哉其能成名也固譏司馬遷是非頗謬於聖人|{
	賢曰言遷所是非與聖人乖謬即崇黄老而薄六經輕仁義而賤守節是也}
然其論議常排死節|{
	謂言龔勝竟夭天年之類}
否正直|{
	謂言王陵汲黯之戅之類}
而不叙殺身成仁之為美|{
	謂不立忠義傳}
則輕仁義賤守節甚矣

初竇憲納妻天下郡國皆有禮慶漢中郡亦當遣吏|{
	漢中郡在洛陽西千九百九十里}
戶曹李郃|{
	郡有戶曹主民戶祠祀農桑郃曷閤翻}
諫曰竇將軍椒房之親不修德禮而專權驕恣危亡之禍可翹足而待|{
	翹舉也}
願明府一心王室勿與交通太守固遣之郃不能止請求自行許之郃遂所在遲留以觀其變行至扶風|{
	潘岳關中記曰三輔舊治長安城中長吏各居其縣治民東都之後扶風出治槐里馮翊出治高陵}
而憲就國凡交通者皆坐免官漢中太守獨不與焉|{
	與讀曰預}
帝賜清河王慶奴婢輿馬錢帛珍寶充牣其第慶或時不安帝朝夕問訊進膳藥所以垂意甚備慶亦小心恭孝自以廢黜尤畏事慎法故能保其寵祿焉 帝除袁安子賞為郎任隗子屯為步兵校尉|{
	以安隗守正不附竇氏也任音壬隗五罪翻}
鄭衆遷大長秋|{
	百官志大長秋秩二千石承秦將行景帝更為大長秋或用士人中興常用宦者職掌奉宣中宫命凡給賜宗親及宗親當謁見者關通之中宫出則從張晏曰皇后卿師古曰秋者收成之時長者恒久之義故以為皇后官名}
帝策勲班賞衆每辭多受少帝由是賢之常與之議論政事宦官用權自此始矣秋七月己丑太尉宋由以竇氏黨策免自殺 八月

辛亥司空任隗薨 癸丑以大司農尹睦為太尉太傅鄧彪以老病上還樞機職|{
	上時掌翻尚書樞機之職鄧彪錄尚書事}
詔許焉以睦代彪錄尚書事 冬十月以宗正劉方為司空武陵零陵灃中蠻叛 護羌校尉鄧訓卒吏民羌胡旦夕臨者日數千人|{
	臨力鴆翻哭也}
羌胡或以刀自割又刺殺其犬馬牛羊|{
	刺七逆翻又七四翻}
曰鄧使君已死我曹亦俱死耳前烏桓吏士皆犇走道路|{
	賢曰訓前任烏桓校尉時吏士也}
至空城郭吏執不聽以狀白校尉徐傿|{
	傿蓋為烏桓校尉傿於建翻}
傿歎息曰此為義也乃釋之遂家家為訓立祠|{
	為于偽翻下同}
每有疾病輒請禱求福蜀郡太守聶尚代訓為護羌校尉欲以恩懷諸羌乃遣譯使招呼迷唐使還居大小榆谷|{
	迷唐去大小榆谷事見上卷章和二年鄧訓驅逐迷唐而聶尚招呼之欲以反鄧訓之政也聶昵輒翻使疏吏翻}
迷唐既還遣祖母卑鈌詣尚|{
	卑鈌蓋迷吾之母}
尚自送至塞下為設祖道令譯田汜等五人護送至廬落迷唐遂反與諸種共生屠裂汜等以血盟詛|{
	種章勇翻汜詳里翻詛莊助翻}
復寇金城塞|{
	復扶又翻}
尚坐免

五年春正月乙亥宗祀明堂登靈臺赦天下 戊子千乘貞王伉薨|{
	諡法臣諡直道不撓曰貞事君無猜曰貞清白守節曰貞固節幹事曰貞伉音抗}
辛卯封皇弟萬歲為廣宗王|{
	廣宗縣屬鉅鹿郡賢曰今貝州宗城縣隋煬帝}


|{
	諱廣故改為宗城}
甲寅太傅鄧彪薨 戊午隴西地震 夏四月壬子紹封阜陵殤王兄魴為阜陵王|{
	諡法未家短折曰殤阜陵殤王冲質王延之子元年嗣封三年薨無嗣今以魴紹封魴符方翻}
九月辛酉廣宗殤王萬歲薨無子國除 初竇憲既立於除鞬為北單于欲輔歸北庭|{
	事見上卷三年鞬居言翻}
會憲誅而止於除鞬自畔還北詔遣將兵長史王輔以千餘騎與任尚共追討斬之破滅其衆 耿夔之破北匈奴也|{
	事見上卷三年}
鮮卑因此轉徙據其地|{
	拓跋氏自北荒南徙蓋此時也}
匈奴餘種留者尚有十餘萬落|{
	種章勇翻}
皆自號鮮卑鮮卑由此漸盛 冬十月辛未太尉尹睦薨 十一月乙丑太僕張酺為太尉酺與尚書張敏等奏射聲校尉曹褒擅制漢禮破亂聖術宜加刑誅書凡五奏帝知酺守學不通|{
	言守其家學也}
雖寑其奏而漢禮遂不行|{
	褒制禮事見上卷章帝章和元年}
是歲武陵郡兵破叛蠻降之|{
	降戶江翻}
梁王暢與從官卞忌祠祭求福|{
	姓譜卞本自有周曹叔振鐸之後曹之支子封於卞遂以建族余按魯有下莊子楚有卞和}
忌等諂媚云神言王當為天子暢與相應答為有司所奏請徵詣詔獄帝不許但削成武單父二縣|{
	成武單父二縣本屬山陽後屬濟隂章帝以益梁國賢曰成武今曹州縣單父今宋州縣單音善}
暢慙懼上疏深自刻責曰臣天性狂愚不知防禁自陷死罪分伏顯誅|{
	分扶問翻}
陛下聖德枉法曲平|{
	賢曰曲平曲法申恩平處其罪}
横赦貸臣為臣受汙|{
	横胡孟翻汙惡也天下以赦暢為納汙是為暢受汙為于偽翻}
臣知大貸不可再得自誓束身約妻子不敢復出入失繩墨不敢復有所横費|{
	横戶孟翻復扶又翻}
租入有餘乞裁食睢陽穀熟虞蒙寧陵五縣還餘所食四縣|{
	四縣下邑尉氏薄郾也睢音雖}
臣暢小妻三十七人|{
	凡非正室者皆小妻也}
其無子者願還本家自選擇謹敕奴婢二百人其餘所受虎賁官騎及諸工技鼓吹倉頭奴婢兵弩廐馬皆上還本署|{
	虎賁士屬虎賁中郎將官騎騶騎也漢官儀曰騶騎王家名官騎與廐馬皆屬太僕工技屬尚方鼓吹屬黄門倉頭奴婢屬永巷御府奚官等令兵弩屬考工令各有本署也賁音奔技渠綺翻吹昌瑞翻上時掌翻}
臣暢以骨肉近親亂聖化汙清流|{
	汙烏故翻}
既得生活誠無心面目以凶惡復居大宫食大國張官屬藏什物|{
	賢曰古者師行二五為什食器之類必共之故曰什物食具今人通謂生生之具為什物復扶又翻}
願陛下加恩開許上優詔不聽 護羌校尉貫友|{
	貫姓也漢初有趙相貫高}
遣譯使構離諸羌誘以財貨由是解散|{
	使疏吏翻誘音酉}
乃遣兵出塞攻迷唐於大小榆谷獲首虜八百餘人收麥數萬斛遂夹逢留大河築城塢|{
	此大河即黄河河水至此有逢留之名在二榆谷北}
作大航造河橋欲度兵擊迷唐|{
	酈道元水經注曰於河狭作橋航戶剛翻}
迷唐率部落遠徙依賜支河曲|{
	西羌傳賜支者禹貢所謂析支者也羌居河關之西南濱於賜支至於河首緜地千里司馬彪曰西羌自析支以西濱河首在右居也河水屈而東北流逕於析支之地是為河曲矣應劭曰禹貢析支屬雍州在河關之西東去河關千餘里羌人所居謂之河曲羌}
單于屯屠何死單于宣弟安國立安國初為左賢王無稱譽及為單于單于適之子右谷蠡王師子以次轉為左賢王|{
	谷音鹿蠡盧奚翻}
師子素勇黠多知|{
	黠下入翻知古智字通}
前單于宣及屯屠何皆愛其氣決數遣將兵出塞|{
	數所角翻下同}
掩擊北庭還受賞賜天子亦加殊異由是國中盡敬師子而不附安國安國欲殺之諸新降胡初在塞外數為師子所驅掠|{
	在塞外謂先屬北部時降戶江翻}
多怨之安國委計降者與同謀議師子覺其謀乃别居五原界每龍庭會議|{
	匈奴龍庭本在塞外是時南單于居塞内亦謂所居為龍庭}
師子輒稱病不往度遼將軍皇甫稜知之亦擁護不遣單于懷憤益甚

六年春正月皇甫稜免以執金吾朱徽行度遼將軍時單于與中郎將杜崇不相平乃上書告崇崇諷西河太守令斷單于章|{
	中郎將使匈奴中郎將也斷音短單于居西河美稷故諷令太守斷其章使不上聞}
單于無由自聞崇因與朱徽上言南單于安國疎遠故胡親近新降|{
	遠于願翻近其靳翻降戶江翻}
欲殺左賢王師子及左臺且渠劉利等又右部降者謀共廹脅安國起兵背畔|{
	且子余翻背蒲妹翻}
請西河上郡安定為之儆備帝下公卿議|{
	下遐稼翻}
皆以為蠻夷反覆雖難測知然大兵聚會必未敢動揺今宜遣有方略使者之單于庭|{
	之往也使疏吏翻}
與杜崇朱徽及西河太守并力觀其動静如無他變可令崇等就安國會其左右大臣責其部衆横暴為邊害者共平罪誅|{
	相與平處其罪當誅者則誅之横戶孟翻}
若不從命令為權時方略事畢之後裁行賞賜|{
	賢曰裁量賜物不多與也}
亦足以威示百蠻於是徽崇遂發兵造其庭|{
	造七到翻}
安國夜聞漢軍至大驚棄帳而去|{
	帳單于所居即謂之穹廬又謂之廬帳}
因舉兵欲誅師子師子先知乃悉將廬落入曼柏城|{
	曼柏縣屬五原郡}
安國追到城下門閉不得入朱徽遣吏譬和之安國不聽城既不下乃引兵屯五原崇徽因發諸郡騎追赴之急衆皆大恐安國舅骨都侯喜為等慮并被誅乃格殺安國|{
	被皮義翻 考異曰帝紀在去年誤今從南匈奴傳}
立師子為亭獨尸逐侯鞮單于|{
	鞮丁奚翻}
己卯司徒丁鴻薨 二月丁未以司空劉方為司徒太常張奮為司空 夏五月城陽懷王淑薨無子國除 秋七月京師旱 西域都護班超發龜兹鄯善等八國兵合七萬餘人|{
	龜兹音丘慈鄯上扇翻}
討焉耆到其城下誘焉耆王廣尉犂王汎等於陳睦故城斬之傳首京師|{
	誘音酉 考異曰袁紀汎作沉今從超傳}
因縱兵鈔掠|{
	鈔楚交翻}
斬首五千餘級獲生口萬五千人更立焉耆左侯元孟為焉耆王|{
	焉耆國有左右將左右侯更工衡翻}
超留焉耆半歲慰撫之於是西域五十餘國悉納質内屬至於海濱|{
	西海之濱也所謂條支大秦蒙奇兜勒諸國也質音致}
四萬里外皆重譯貢獻|{
	重直龍翻班超所以成西域之功者以匈奴衰困力不能及西域也}
南單于師子立降胡五六百人夜襲師子|{
	降戶江翻}
安集掾王恬將衛護士與戰破之|{
	使匈奴中郎將置掾隨事為員安集掾以安集匈奴為稱也光武在河北亦置安集掾以天下未定使之安集斯民也建武二十六年使匈奴中郎將置安集掾史將弛刑五十人持兵弩隨單于所處參辭訟察動静掾俞絹翻}
於是降胡遂相驚動十五部二十餘萬人皆反脅立前單于屯屠何子薁鞮日逐王逢侯為單于|{
	鞮當作鞬賢曰前鞮鞬兩字通今不改亦可薁於六翻鞬九言翻}
遂殺略吏民燔燒郵亭廬帳將車重向朔方欲度幕北|{
	郵音尤重直用翻}
九月癸丑以光祿勲鄧鴻行車騎將軍事與越騎校尉馮柱行度遼將軍朱徽將左右羽林北軍五校士及郡國迹射緣邊兵|{
	賢曰漢有迹射士言尋迹而射也}
烏桓校尉任尚將烏桓鮮卑合四萬人討之時南單于及中郎將杜崇屯牧師城|{
	漢邊郡有牧師菀以養馬此牧師菀城也當在西河郡美稷縣界}
逢侯將萬餘騎攻圍之冬十一月鄧鴻等至美稷逢侯乃解圍去向滿夷谷南單于遣子將萬騎及杜崇所領四千騎與鄧鴻等追擊逢侯於大城塞|{
	大城縣故屬西河郡郡國志屬朔方郡}
斬首四千餘級任尚率鮮卑烏桓要擊逢侯於滿夷谷|{
	要一遥翻}
復大破之|{
	復扶又翻下同}
前後凡斬萬七千餘級逢侯遂率衆出塞漢兵不能追而還|{
	還從宣翻又如字}
以大司農陳寵為廷尉寵性仁矜數議疑獄|{
	數所角翻}
每附經典務從寛恕刻敝之風於此少衰|{
	少詩沼翻}
帝以尚書令江夏黄香為東郡太守香辭以典郡從政才非所宜乞留備宂官|{
	宂而隴翻散也}
賜以督責小職任之宫臺煩事|{
	宫謂宫中臺謂尚書臺也尚書出納王命故云宫臺煩事}
帝乃復留香為尚書令增秩二千石|{
	按百官志尚書令秩千石今特增秩二千石以香在尚書日久又辭不拜郡故復留為尚書令而祿以郡守祿}
甚見親重香亦祗勤物務憂公如家

七年春正月鄧鴻等軍還馮柱將虎牙營留屯五原鴻坐逗留失利下獄死後帝知朱徽杜崇失胡和又禁其上書以致胡反皆徵下獄死|{
	下遐稼翻}
夏四月辛亥朔日有食之 秋七月乙巳易陽地裂|{
	余按地理志及郡國志易陽縣屬趙國應劭曰易水出涿郡故安師古及賢皆曰縣在易水之陽此皆承應劭之誤也易水在燕南界漢屬河間郡界此時趙國僅有唐邢洺二州之地安得有屬縣遠在易水之陽邪五代史志洺州臨洺縣舊曰易陽後齊廢入襄國縣後周改為易陽縣别置襄國縣隋間皇六年改易陽縣為邯鄲縣十年改邯鄲縣為臨洺而别置邯鄲縣由是觀之漢易陽縣當在邯鄲襄國二縣之間}
九月癸卯京師地震 樂成王黨坐賊殺人削東光鄡二縣|{
	東光縣本屬勃海郡鄡縣本屬鉅鹿郡章帝以益樂成國鄡音羌堯翻舊禁宫人出嫁不得適諸國有故掖庭技人哀置嫁男子章初黨召入宫與通初欲上書告之黨賂哀置姊昭殺初}


八年春二月立貴人隂氏為皇后后識之曾孫也 夏四月樂成靖王黨薨子哀王崇立尋薨無子國除 五月河内陳留蝗 南匈奴右温禺犢王烏居戰畔出塞|{
	賢曰温禺犢王名烏居戰}
秋七月度遼將軍龎奮越騎校尉馮柱追擊破之徙其餘衆及諸降胡二萬餘人於安定北地|{
	安定郡在雒陽西千七百里北地郡在雒陽西千一百里}
車師後部王涿鞮反擊前王尉畢大獲其妻子|{
	尉畢大西域傳作尉卑大時戊巳校尉索頵欲廢後部王涿鞮涿鞮忿前王尉卑大賣已因反撃尉卑大鞮丁奚翻}
九月京師蝗 冬十月乙丑北海王威以非敬王子又坐誹謗自殺 十二月辛亥陳敬王羨薨 丁巳南宫宣室殿火 護羌校尉貫友卒以漢陽太守史充代之充至遂發湟中羌胡出塞擊迷唐迷唐迎敗充兵|{
	敗補邁翻}
殺數百人充坐徵以代郡太守吳祉代之

九年春三月庚辰隴西地震 癸巳濟南安王康薨|{
	諡法好和不争曰安寛裕和平曰安濟子禮翻}
西域長史玉林擊車師後王斬之|{
	後王涿鞮}
夏四月丁卯封樂成王黨子廵為樂成王五月封皇后父屯騎校尉隂綱為吳房侯|{
	郡國志吳房縣屬汝南郡有棠谿亭左傳房國楚靈王所滅又楚昭王封吳王夫概于棠谿地道記有吳城吳房蓋合吳城房國以名縣也}
以特進就第 六月旱蝗 秋八月鮮卑寇肥如遼東太守祭參坐沮敗下獄死|{
	賢曰肥如縣屬遼西郡前書音義曰肥子奔燕封於此今平州也按祭彤傳參守遼東鮮卑入郡界參坐沮敗下獄死蓋寇遼西之肥如遂入遼東郡界也沮在呂翻}
閏月辛巳皇太后竇氏崩初梁貴人既死|{
	事見四十六卷章帝建初八年}
宫省事秘莫有知帝為梁氏出者舞隂公主子梁扈遣從兄襢奏記三府|{
	扈梁松子也帝母梁貴人少失母為伯母舞隂公主所養從才用翻賢曰襢古禪字}
以為漢家舊典崇貴母氏而梁貴人親育聖躬不蒙尊號求得申議|{
	賢曰求申理而議之也}
太尉張酺言狀帝感慟良久曰|{
	毛晃曰良頗也良久頗久也或曰良久少久也一曰良畧也聲輕故轉畧為良慟徒弄翻大哭也哀過也}
於君意若何酺請追上尊號存錄諸舅|{
	錄采也收拾也}
帝從之會貴人姊南陽樊調妻嫕|{
	嫕音於計翻考異曰袁紀嫕皆作憑今從皇后紀梁竦傳}
上書自訟曰妾父竦寃死牢獄骸骨不掩母氏年踰七十及弟棠等遠在絶域不知死生願乞收竦朽骨使母弟得歸本郡帝引見嫕乃知貴人枉殁之狀三公上奏請依光武黜呂太后故事|{
	事見四十五卷光武中元元年按此事乃光武之失而可引之為故典乎}
貶竇太后尊號不宜合葬先帝百官亦多上言者帝手詔曰竇氏雖不遵法度而太后常自減損朕奉事十年|{
	自嗣位至是十年}
深惟大義|{
	惟思也}
禮臣子無貶尊上之文恩不忍離義不忍虧案前世上官太后亦無降黜|{
	謂上官桀父子誅不累及上官后也事見二十二卷昭帝元鳳元年}
其勿復議|{
	復扶又翻}
丙申葬章德皇后 燒唐羌迷唐率衆八千人寇隴西脅塞内諸種羌合步騎三萬人擊破隴西兵殺大夏長|{
	大夏縣屬隴西郡宋白曰今大夏縣屬河州夏戶雅翻種章勇翻長知兩翻}
詔遣行征西將軍劉尚越騎校尉趙世副之|{
	以趙世副劉尚也考異曰西羌傳作趙代今從帝紀 余謂唐太宗諱世民賢注范史偶檢點及此遂改世為代耳}
將漢兵羌胡共三萬人討之尚屯狄道世屯枹罕|{
	狄道枹罕二縣皆屬隴西郡宋白曰狄道縣屬蘭州枹罕縣河州治所枹音膚}
尚遣司馬寇盱監諸郡兵四面並會|{
	監古銜翻}
迷唐懼棄老弱犇入臨洮南|{
	犇入臨洮南山也}
尚等追至高山大破之斬虜千餘人迷唐引去漢兵死傷亦多不能復追|{
	復扶又翻}
乃還 九月庚申司徒劉方策免自殺甲子追尊梁貴人為皇太后諡曰恭懷追復喪制冬

十月乙酉改葬梁太后及其姊大貴人於西陵|{
	西陵蓋以其地在敬陵之西故稱西陵猶薄太后陵在霸陵南因謂之南陵也賢曰初后葬有闕故改葬}
擢樊調為羽林左監追封諡皇太后父竦為褒親愍侯|{
	諡法在國逢難曰愍}
遣使迎其喪葬於恭懷皇后陵旁徵還竦妻子封子棠為樂平侯|{
	樂平侯國屬東郡故清縣也章帝更名}
棠弟雍為乘氏侯|{
	乘氏侯國屬濟隂郡春秋之乘丘也乘繩證翻}
雍弟翟為單父侯|{
	單父音善甫}
位皆特進賞賜以巨萬計寵遇光於當世梁氏自此盛矣清河王慶始敢求上母宋貴人冢|{
	宋貴人冢在雒陽城北樊濯聚上時掌翻}
帝許之詔太官四時給祭具慶垂涕曰生雖不獲供養|{
	供俱用翻養羊尚翻}
終當奉祭祀私願足矣欲求作祠堂恐有自同恭懷梁后之嫌遂不敢言常泣向左右以為没齒之恨|{
	齒年也}
後上言外祖母王年老乞詣雒陽療疾於是詔宋氏悉歸京師|{
	宋氏歸故郡事見四十六卷章帝建初七年}
除慶舅衍俊蓋暹等皆為郎 十一月癸卯以光祿勲河南呂蓋為司徒 十二月丙寅司空張奮罷壬申以太僕韓稜為司空 西域都護定遠侯班超遣掾甘英使大秦條支|{
	東觀記曰以漢中郡南鄭縣之西鄉千戶封超為定遠侯賢曰定遠故城在今洋州西鄉縣南西域傳曰自皮山西南經烏秅涉懸度歷罽賓六十餘日行至烏弋山離國復西南馬行百餘日至條支條支臨西海海水曲環其南及東北三面路絶唯西北隅通陸道大秦國西漢之犂靬也在西海西其人民長大平正有類中國故謂之大秦今拂菻國是也掾俞絹翻使疏吏翻}
窮西海皆前世所不至莫不備其風土傳其珍怪焉及安息西界|{
	自條支轉北而東馬行六十餘日至安息}
臨大海欲度船人謂英曰海水廣大往來者逢善風|{
	善風謂順風也}
三月乃得度若遇遲風亦有二歲者故入海人皆齎三歲糧海中善使人思土戀慕數有死亡者|{
	數所角翻}
英乃止十年夏五月京師大水 秋七月己巳司空韓稜薨八月丙子以太常太山巢堪為司空 冬十月五州雨水行征西將軍劉尚越騎校尉趙世坐畏懦徵下獄免

謁者王信領尚營屯枹罕謁者耿譚領世營屯白石|{
	白石縣本屬金城郡時屬隴西郡水經注白石川水南逕白石城西而注灕水水又逕白石縣故城南闞駰曰白石縣在狄道縣西北二百八十五里賢曰白石山在今蘭州或曰河州鳳林縣本漢白石縣張駿改為永固唐為烏州後廢州置安昌縣後又更名鳳林杜佑曰直道縣有白石山}
譚乃設購賞諸種頗來内附迷唐恐乃請降信譚遂受降罷兵十二月迷唐等帥種人詣闕貢獻|{
	帥讀曰率種章勇翻}
戊寅梁節王暢薨初居巢侯劉般薨|{
	居巢縣屬廬江郡般建初三年薨}
子愷當嗣稱父遺意讓其弟憲遁逃久之有司奏絶愷國肅宗美其義特優假之愷猶不出積十餘歲有司復奏之|{
	復扶又翻}
侍中賈逵上書曰孔子稱能以禮讓為國乎何有|{
	見論語}
有司不原樂善之心|{
	樂音洛}
而繩以循常之法懼非長克讓之風成含弘之化也|{
	長知兩翻}
帝納之下詔曰王法崇善成人之美其聽憲嗣爵遭事之宜後不得以為比乃徵愷拜為郎 南單于師子死單于長之子檀立為萬氏尸逐鞮單于

十一年夏四月丙寅赦天下 帝因朝會召見諸儒|{
	朝直遥翻見賢遍翻}
使中大夫魯丕與侍中賈逵尚書令黄香等相難數事|{
	難乃旦翻以經疑相難也下同}
帝善丕說罷朝特賜衣冠丕因上疏曰臣聞說經者傳先師之言非從已出不得相讓相讓則道不明若規矩權衡之不可枉也|{
	賢曰規圓也矩方也權秤錘衡秤衡也}
難者必明其據說者務立其義|{
	漢儒專門名家各守師說故發難者必明其師之說以為據答難者亦必務立大義以申其師之說}
浮華無用之言不陳於前故精思不勞而道術愈章|{
	章明也思相吏翻}
法異者各令自說師法博觀其義無令芻蕘以言得罪|{
	自比於芻蕘謙也蕘如招翻}
幽遠獨有遺失也

十二年夏四月戊辰秭歸山崩|{
	賢曰秭歸縣屬南郡古之夔國今歸州也袁山松曰屈原此縣人既被流放忽然蹔歸其姊亦來因名其地為秭歸秭亦姊也音蔣兕翻}
秋七月辛亥朔日有食之 九月戊午太尉張酺免 丙寅以大司農張禹為太尉 燒當羌豪迷唐既入朝其餘種人不滿二千飢窘不立|{
	不能自立也種章勇翻下同}
入居金城帝令迷唐將其種人還大小榆谷迷唐以漢作河橋|{
	即五年貫友所作之橋}
兵來無常故地不可復居|{
	復扶又翻下同}
辭以種人飢餓不肯遠出護羌校尉吳祉等多賜迷唐金帛令糴穀市畜|{
	畜許又翻}
促使出塞種人更懷猜驚是歲迷唐復叛脅將湟中諸胡寇鈔而去|{
	鈔楚交翻}
王信耿譚吳祉皆坐徵十三年秋八月己亥北宫盛饌門閣火|{
	盛饌門閣御㕑門閣也晉書天文志曰紫宫垣西南角外二星内二星曰内㕑主六宫之内飲食后妃夫人與太子宴飲東北維外六星曰天㕑主盛饌皇居則象於天極故北宫有盛饌門閣}
迷唐復還賜支河曲將兵向塞護羌校尉周鮪與金城太守侯霸|{
	金城郡在洛陽西二千八百里鮪于軌翻}
及諸郡兵屬國羌胡合三萬人至允川|{
	水經注曰允川去賜支河曲數十里在大小榆谷之西}
侯霸擊破迷唐種人瓦解降者六千餘口|{
	種章勇翻降戶江翻}
分徙漢陽安定隴西迷唐遂弱遠踰賜支河首依發羌居|{
	發羌羌之别種或曰唐之吐蕃即其後也}
久之病死其子來降戶不滿數十 荆州雨水 冬十一月丙辰詔曰幽并凉州戶口率少|{
	幽州部涿郡廣陽代郡上谷漁陽右北平遼西遼東玄菟樂浪等郡并州部上黨太原上郡西河五原雲中定襄鴈門朔方等郡幽州大郡戶猶十萬餘唯玄菟戶一千五百二十四并州大郡三萬餘小郡不滿二千凉州大郡不滿三萬燉煌七百四十八而已少詩沼翻}
邊役衆劇束脩良吏進仕路狭|{
	束脩謂束髪自脩者也}
撫接夷狄以人為本其令緣邊郡口十萬以上歲舉孝廉一人不滿十萬二歲舉一人五萬以下三歲舉一人 鮮卑寇右北平|{
	右北平郡在雒陽北二千三百里}
遂入漁陽漁陽太守擊破之 戊辰司徒呂蓋以老病致仕 巫蠻許聖以郡收税不均怨恨遂反|{
	賢曰巫縣屬南郡故城在今夔州巫山縣}
辛卯寇南郡

十四年春安定降羌燒何種反|{
	燒當與燒何各是一種種章勇翻下同}
郡兵擊滅之時西海及大小榆谷左右無復羌寇|{
	水經河水自東河曲逕西海郡南又東逕允川而歷大小榆谷北復扶又翻}
隃麋相曹鳳上言|{
	隃麋侯國屬右扶風隃音踰麋音眉賢曰隃麋故城在今隴州汧陽縣東南}
自建武以來西羌犯法者常從燒當種起所以然者以其居大小榆谷土地肥美有西海魚鹽之利|{
	西海有允谷鹽池}
阻大河以為固又近塞諸種易以為非|{
	易以䜴翻}
難以攻伐故能彊大常雄諸種恃其拳勇|{
	詩云無拳無勇毛萇注云拳力也}
招誘羌胡今者衰困黨援壞沮|{
	誘音酉沮在呂翻}
亡逃棲竄遠依發羌臣愚以為宜及此時建復西海郡縣|{
	建立也立策復置郡縣也置西海郡見三十六卷平帝元始四年}
規固二榆|{
	規圖也謀也}
廣設屯田隔塞羌胡交關之路|{
	塞悉則翻}
遏絶狂狡窺欲之源又殖穀富邊省委輸之役|{
	委於偽翻輸春遇翻}
國家可以無西方之憂上從之繕修故西海郡徙金城西部都尉以戍之|{
	孟康曰金城西部都尉府在金城縣}
拜鳳為金城西部都尉屯龍耆|{
	賢曰龍耆即龍支也今鄯州縣宋白曰鄯州龍支縣本漢允吾縣也取縣西龍支堆為名}
後增廣屯田列屯夹河合三十四部其功垂立會永初中諸羌叛乃罷 三月戊辰臨辟雍饗射赦天下夏四月遣使者督荆州兵萬餘人分道討巫蠻許聖等大破之聖等乞降悉徙置江夏|{
	晉宋之荆州蠻分居沔中西陽者即巫蠻之餘種也降戶江翻}
隂皇后多妬忌寵遇浸衰數懷恚恨|{
	數所角翻恚衣避翻}
后外祖母鄧朱出入宫掖有言后與朱共挟巫蠱道者|{
	賢曰巫師為蠱故曰巫蠱左傳注曰蠱惑也}
帝使中常侍張慎與尚書陳褒案之劾以大逆無道|{
	劾戶槩翻又戶得翻}
朱二子奉毅后弟輔皆考死獄中六月辛卯后坐廢遷于桐宫以憂死父特進綱自殺后弟軼敞及朱家屬徙日南比景|{
	日南郡秦象郡也武帝更名在雒陽南萬三千四百里比景縣屬焉如淳曰日中於頭上景在已下故名之師古曰日南言其在日之南所謂開北戶以向日者軼音逸}
秋七月壬子常山殤王側薨無子立其兄防子侯章為常山王|{
	防子縣屬常山國}
三州大水班超久在絶域|{
	超始出西域見四十五卷明帝永平十六年}
年老思土上書乞歸曰臣不敢望到酒泉郡但願生入玉門關|{
	酒泉郡在雒陽西四千七百里賢曰玉門關屬燉煌郡今沙州也去長安三千六百里關在燉煌縣西北酒泉郡今肅州也去長安二千八百五十里}
謹遣子勇隨安息獻物入塞及臣生在令勇日見中土朝廷久之未報超妹曹大家|{
	超妹昭嫁扶風曹夀博學高才有節行法度帝數召入宫令皇后諸貴人師事焉號曰大家家今人相傳讀曰姑又據皇后紀冲帝母虞貴人梁冀秉政抑而不加爵號但稱大家而已則大家者宫中相尊之稱也}
上書曰蠻夷之性悖逆侮老|{
	悖蒲内翻又蒲沒翻}
而超旦暮入地久不見代恐開姦宄之原生逆亂之心而卿大夫咸懷一切莫肯遠慮如有卒暴|{
	卒讀曰猝下同}
超之氣力不能從心便為上損國家累世之功下棄忠臣竭力之用誠可痛也故超萬里歸誠自陳苦急延頸踰望|{
	賢曰踰遥也高祖踰謂黥布曰何苦而反余按前書當作隃讀曰遥傳寫誤作踰}
三年於今未蒙省錄|{
	省悉景翻}
妾竊聞古者十五受兵六十還之|{
	賢曰周禮鄉大夫職曰國中七尺以及六十有五皆征之征謂賦税從征役也韓詩外傳曰二十行役六十免役與周禮國中同即知一與周禮七尺同禮國中六十免役野即六十有五晚於國中五年國中七尺從役野六尺即是野又早於國中五年七尺謂二十六尺即十五也此言十五受兵據野外為言六十還之據國中為說也}
亦有休息不任職也|{
	任音壬}
故妾敢觸死為超求哀匄超餘年|{
	為于偽翻賢曰匄乞也}
一得生還復見闕庭使國家無勞遠之慮西域無倉卒之憂超得長蒙文王葬骨之恩|{
	新序曰周文王作靈臺掘地得死人之骨王曰更葬之吏曰此無主矣文王曰有天下者天下之主也有一國者一國之主也寡人固其主又安求之主遂更葬之天下皆曰文王賢矣澤及朽骨而况於人乎}
子方哀老之惠|{
	賢曰田子方魏文侯之師也見君之老馬棄之曰少盡其力老而棄之非仁也於是收而養之}
帝感其言乃徵超還八月超至雒陽拜為射聲校尉九月卒 |{
	考異曰本傳稱超十二年上疏十四年至雒陽而妹昭上書曰延頸踰望三年於今注引東觀記曰安息遣使獻大雀師子超遣子勇隨入塞按帝紀十三年安息國入貢袁紀載超書亦在十三年今并置其書於此袁紀又云超到數月薨今從本傳}
超之被徵|{
	被皮義翻}
以戊巳校尉任尚代為都護尚謂超曰君侯在外國三十餘年而小人猥承君後任重慮淺宜有以誨之超曰年老失智君數當大位|{
	數所角翻}
豈班超所能及哉必不得已願進愚言塞外吏士本非孝子順孫皆以罪過徙補邊屯而蠻夷懷鳥獸之心難養易敗|{
	易以䜴翻}
今君性嚴急水清無大魚察政不得下和|{
	家語孔子曰水至清則無魚人至察則無徒}
宜蕩佚簡易|{
	易以豉翻}
寛小過總大綱而已超去尚私謂所親曰我以班君當有奇策今所言平平耳尚後竟失邊和如超所言|{
	為任尚徵還漢失西域張本}
初太傅鄧禹嘗謂人曰吾將百萬之衆未嘗妄殺一人後世必有興者其子護羌校尉訓有女曰綏性孝友好書傳|{
	好呼到翻傳柱戀翻}
常晝脩婦業暮誦經典家人號曰諸生叔父陔曰嘗聞活千人者子孫有封兄訓為謁者使修石臼河歲活數千人|{
	陔柯開翻石臼河事見四十六卷章帝建初三年}
天道可信家必蒙福綏後選入宫為貴人恭肅小心動有法度承事隂后接撫同列常克己以下之|{
	謂克去有己之私不欲上人也下遐稼翻}
雖宫人隸役皆加恩借|{
	既有以恩之又假借以辭色}
帝深嘉焉嘗有疾帝特令其母兄弟入親醫藥不限以日數貴人辭曰宫禁至重而使外舍久在内省|{
	外舍猶言外家内省猶言内禁也}
上令陛下有私幸之譏|{
	私幸謂私於所幸者}
下使賤妾獲不知足之謗上下交損|{
	謂交有所損}
誠不願也帝曰人皆以數入為榮|{
	數所角翻下同}
貴人反以為憂邪每有讌會諸姬競自脩飾貴人獨尚質素其衣有與隂后同色者即時解易若並時進見|{
	見賢遍翻下同}
則不敢正坐離立|{
	賢曰離並也禮記曰離坐離立}
行則僂身自卑|{
	僂力主翻俯也}
帝每有所問常逡廵後對不敢先后言|{
	先悉薦翻}
隂后短小舉指時失儀左右掩口而笑貴人獨愴然不樂為之隱諱若己之失|{
	樂音洛為于偽翻}
帝知貴人勞心曲體歎曰脩德之勞乃如是乎後隂后寵衰貴人每當御見|{
	御進也見賢遍翻}
輒辭以疾時帝數失皇子貴人憂繼嗣不廣數選進才人以博帝意|{
	西漢宫中爵號有美人良人若才人蓋東都所置也博廣也}
隂后見貴人德稱日盛|{
	稱尺證翻}
深疾之|{
	疾與嫉同妬也}
帝嘗寢病危甚隂后密言我得意不令鄧氏復有遺類|{
	復扶又翻}
貴人聞之流涕言曰我竭誠盡心以事皇后竟不為所祐今我當從死|{
	從才用翻}
上以報帝之恩中以解宗族之禍下不令隂氏有人豕之譏|{
	人豕即人彘事見十二卷惠帝元年}
即欲飲藥宫人趙玉者固禁止之因詐言屬有使來|{
	屬之欲翻會也使疏吏翻}
上疾已愈貴人乃止明日上果瘳|{
	瘳丑留翻}
及隂后之廢貴人請救不能得帝欲以貴人為皇后貴人愈稱疾篤深自閉絶冬十月辛卯詔立貴人鄧氏為皇后后辭讓不得已然後即位郡國貢獻悉令禁絶|{
	漢郡國貢獻進御之外别上皇后宫}
歲時但供紙墨而已|{
	毛晃曰楮籍不知所始後漢蔡倫以魚網木皮為紙俗以為紙始於倫非也案前書外戚傳已有赫蹏紙矣墨膠煤以為之}
帝每欲官爵鄧氏后輒哀請謙讓故兄騭終帝世不過虎賁中郎將|{
	騭職日翻賢曰東觀記騭音陟}
丁酉司空巢堪罷 十一月癸卯以大司農沛國徐防為司空防上疏以為漢立博士十有四家|{
	漢官儀曰光武中興恢弘稽古易有施孟梁丘賀京房書有歐陽和伯夏侯勝建詩有申公轅固韓嬰春秋有嚴彭祖顔安樂禮有戴德戴聖凡十四博士}
設甲乙之科|{
	前書博士弟子歲課甲科四十人為郎中乙科二十人為太子舍人丙科四十人為文學掌故}
以勉勸學者伏見太學試博士弟子皆以意說不修家法|{
	賢曰諸經為業各自名家}
私相容隱開生姦路每有策試|{
	策編簡也策試即射策也漢書音義曰作簡策難問列置案上在試者意投射取而答之謂之射策}
輒興諍訟|{
	諍讀與爭同}
論議紛錯互相是非孔子稱述而不作|{
	見論語賢曰祖述先聖之言不自制作}
又曰吾猶及史之闕文|{
	亦見論語賢曰古者史官於書所有不知則闕以待能者孔子言吾少時猶及見古史官之闕文今則無之疾時多穿鑿也}
今不依章句妄生穿鑿以遵師為非義意說為得理|{
	意說者創意而為之說}
輕侮道術浸以成俗誠非詔書實選本意改薄從忠三代常道|{
	賢曰太史公曰夏之政忠忠之敝小人以野故殷人承之以敬敬之敝小人以鬼故周人承之以文文之敝小人以僿故救僿莫若以忠三王之道若循環周而復始僿音西志翻史記僿作薄}
專精務本儒學所先|{
	先悉薦翻}
臣以為博士及甲乙策試宜從其家章句開五十難以試之|{
	難乃旦翻}
解釋多者為上第引文明者為高說若不依先師義有相伐|{
	賢曰伐謂相攻伐也}
皆正以為非上從之 是歲初封大長秋鄭衆為鄛鄉侯|{
	賞誅竇憲功也宦官封侯自此始賢曰說文曰南陽郡棘陽縣有鄛鄉鄛音上交翻}


十五年夏四月甲子晦日有食之時帝遵肅宗故事兄弟皆留京師有司以日食隂盛奏遣諸王就國詔曰甲子之異責由一人諸王幼稺早離顧復|{
	詩小雅蓼莪之篇曰父兮生我母兮鞠我顧我復我出入腹我鄭氏箋曰顧旋視也復反覆也離力智翻}
弱冠相育|{
	冠古玩翻}
常有蓼莪凱風之哀|{
	詩小雅曰蓼蓼者莪匪莪伊蒿哀哀父母生我劬勞又國風曰凱風自南吹彼棘心棘心夭夭母氏劬勞蓼力竹翻}
選懦之恩知非國典且復宿留|{
	賢曰選懦慈戀不決之意也懦音人兖翻復扶又翻宿音秀留音澑}
秋九月壬午車駕南廵清河濟北河間三王並從|{
	濟子禮翻從才用翻}
四州雨水冬十月戊申帝幸章陵戊午進幸雲夢|{
	賢曰雲夢今安州縣也即在雲夢澤中}
時太尉張禹留守|{
	守手又翻}
聞車駕當幸江陵以為不宜冒險遠遊驛馬上諫|{
	上時掌翻}
詔報曰祠謁既訖|{
	謂幸章陵祠謁四親陵廟}
當南禮大江會得君奏臨漢回輿而旋十一月甲申還宫 嶺南舊貢生龍眼荔枝十里一置五里一|{
	賢曰交州記曰龍眼樹高五六丈似荔支而小廣州記曰子似荔支而圓七月熟荔支樹高五六丈大如桂樹實如雞子甘而多汁似安石榴有甜醋者至日禺中翕然俱赤即可食置謂驛也即也立之道旁荔立計翻}
晝夜傳送|{
	傳直戀翻}
臨武長汝南唐羌|{
	賢曰臨武縣屬桂陽郡今郴州縣嶺南入獻道經臨武長知兩翻}
上書曰臣聞上不以滋味為德下不以貢膳為功伏見交阯七郡|{
	交阯州部南海蒼梧鬱林合浦交阯九真日南七郡}
獻生龍眼等鳥驚風發|{
	言其疾也}
南州土地炎熱惡蟲猛獸不絶於路至於觸犯死亡之害死者不可復生來者猶可救也|{
	復扶又翻下同}
此二物升殿未必延年益壽帝下詔曰遠國珍羞本以薦奉宗廟|{
	宗廟之薦各以其土之所有而致之貴遠物也}
苟有傷害豈愛民之本其敇太官勿復受獻 是歲初令郡國以日北至按薄刑|{
	時有司奏以為夏至則微隂起靡草死可以決小事遂令以日北至按薄刑賢曰禮記月令曰孟夏之月靡草死麥秋至斷薄刑決小罪按五月一隂爻生可以言微隂今月令云孟夏乃是純陽之月此言夏至者與月令不同余按安帝永初元年魯恭言自永元十五年按薄刑改用孟夏則夏至乃謂夏之初至范史以日北至書之其誤後人甚矣}


十六年秋七月旱 辛酉司徒魯恭免 庚午以光祿勲張酺為司徒八月己酉酺薨 冬十月辛卯以司空徐防為司徒大鴻臚陳寵為司空|{
	臚陵如翻}
十一月己丑帝行幸氏登百岯山|{
	氏縣屬河南尹賢曰即柏岯山也在洛州氏縣南爾雅云山一成曰岯工侯翻岯平眉翻}
北匈奴遣使稱臣貢獻願和親修呼韓邪故約帝以其舊禮不備未許而厚加賞賜不答其使

元興元年春高句驪王宫入遼東塞寇畧六縣|{
	句驪至宫浸強數犯邊句如字又音駒驪力知翻}
夏四月庚午赦天下改元 秋九月遼東太守耿夔擊高句驪破之 冬十二月辛未帝崩于章德前殿|{
	年二十七}
初帝失皇子前後十數後生者輒隱袐養於民間羣臣無知者及帝崩鄧皇后乃收皇子於民間長子勝有痼疾|{
	痼音固痼疾堅久之疾也長知兩翻}
少子隆生始百餘日|{
	少詩照翻}
迎立以為皇太子是夜即皇帝位|{
	廢長立幼卒以不終為羣臣疑勝疾非痼周章有異謀張本}
尊皇后曰皇太后太后臨朝|{
	朝直遥翻}
是時新遭大憂法禁未設宫中亡大珠一箧|{
	箧詰恊翻竹笥也}
太后念欲考問必有不辜|{
	考問則下之獄辭所連及必有無辜而被逮者}
乃親閲宫人觀察顔色即時首服|{
	首式救翻}
又和帝幸人吉成御者共枉吉成以巫蠱事下掖庭考訊辭證明白|{
	幸人常見幸於和帝者也御者即侍者辭謂告者之辭證謂證佐也下遐稼翻}
太后以吉成先帝左右待之有恩平日尚無惡言今反若此不合人情|{
	謂婦人之情有寵則上僭而生譛愬吉成在先帝之時后待之以恩尚未嘗挟寵而有惡言加於后今帝已晏駕太后臨朝不應反為巫蠱}
更自呼見實覈果御者所為|{
	實覈者審考其實也}
莫不歎服以為聖明 北匈奴重遣使詣敦煌貢獻|{
	重直用翻敦音屯}
辭以國貧未能備禮願請大使當遣子入侍|{
	賢曰天子降大使至其國即遣子隨大使入侍}
太后亦不答其使加賜而已 雒陽令廣漢王渙居身平正能以明察發擿姦伏|{
	擿他狄翻}
外行猛政内懷慈仁凡所平斷|{
	斷下亂翻}
人莫不悦服京師以為有神是歲卒官|{
	卒于官也卒子恤翻}
百姓市道莫不咨嗟流涕渙喪西歸道經弘農民庶皆設槃案於路|{
	以祭渙也槃以盛祭物案以陳槃今野人之祭猶然}
吏問其故咸言平常持米到雒為吏卒所鈔|{
	賢曰鈔掠也余謂此言鈔者非至如盜賊之鈔掠特不以道而侵取之故曰鈔音楚交翻}
恒亡其半|{
	恒戶登翻}
自王君在事|{
	在事謂在官當事也}
不見侵枉故來報恩雒陽民為立祠作詩每祭輒弦歌而薦之|{
	以所作詩被之弦歌也為于偽翻}
太后詔曰夫忠良之吏國家之所以為治也|{
	治直之翻}
求之甚勤得之至寡今以渙子石為郎中以勸勞勤

資治通鑑卷四十八
