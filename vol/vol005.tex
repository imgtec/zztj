\chapter{資治通鑑卷五}



\hfill {}  |{朝散大夫 右諫議大夫 權御史 中丞充理檢使上護軍賜紫金魚袋 臣} 司馬光 奉敕編集

\hfill 後學天台 胡三省 音註

周紀五|{
	起屠維赤奮若
	盡旃蒙大荒落
	凡十七年
	始己丑
	終乙巳也
	}


赧王下


四十三年


楚以左徒黃歇侍太子完爲質於秦|{
	左徒楚官名
	史記正義曰蓋今在左右拾遺補闕之類
	質音致
	按去年秦欲與韓魏伐楚黃歇上書止之歸而報楚楚遂使歇侍太子爲質於秦爲楚王疾病歇使太子亡歸楚張本
	歇許竭翻
	}
秦置南陽郡|{
	凡山南水北皆謂之南陽
	晉南陽在脩武以在太行之南大河之北也
	秦置南陽郡以在南山之南漢水之北也}
秦魏共伐燕|{
	燕因肩翻}
燕惠王薨子武成王立

四十四年

趙藺相如伐齊至平邑|{
	括地志平邑故城在魏州昌樂縣東北四十里
	藺力刃翻
	樂音洛
	}
趙田部吏趙奢收租税|{
	田部吏部收田之租税者也}
平原君家不肯出趙奢以灋治之殺平原君用事者九人|{
	治直之翻平原君之家臣用事而不肯出租税者也}
平原君怒將殺之
趙奢曰君於趙爲貴公子今縱君家而不奉公則灋削灋削則國弱國弱則諸侯加兵是無趙也|{
	削侵也奪也弱劣也懦也}
君安得有此富乎以君之貴奉公如法則上下平上下平則國強國強則趙固而君爲貴戚豈輕於天下邪|{
	邪音耶
	戚親也
	言平原君於趙則王族親戚之貴者也}
平原君以爲賢|{
	賢善也能也}
言之於王
王使治國賦國賦大平民富而府庫實|{
	觀此則趙奢豈特善兵哉可使治國也
	治直之翻
	}


四十五年

秦伐趙圍閼與|{
	司馬彪志上黨郡涅縣有閼與聚
	水經註上黨沾縣有梁榆城即閼與故
	城盧諶征艱賦曰訪梁榆之虛郭乃閼與之舊平
	史記正義曰閼與在潞州銅鞮縣西北二十里
	又儀州和順縣亦有閼與城
	儀潞相近二所未詳
	又閼與山在潞州武安縣西南五十里趙奢拒秦軍於閼與即山北也
	河東圖遼州和順縣晉大夫梁餘子養邑秦伐閼與趙奢救之
	是此遼州即唐之儀州
	閼阿葛翻又於達翻
	康音曷又音嫣
	與音預又音余
	史記正義曰閼於連翻
	漢書音義涅乃結翻
	聚材喻翻
	沾他兼翻
	諶時壬翻
	鞮丁兮翻
	潞魯故翻
	}
趙王召廉頗樂乘而問之|{
	索隱曰樂乘樂毅之宗人也
	頗普河翻
	}
曰可救否皆曰道遠險陿難救|{
	陿與狹同隘也
	}
問趙奢趙奢對曰道遠險陿譬猶兩鼠鬬於穴中將勇者勝|{
	言將是勇者勝也將平聲
	或曰帥勇者則勝將去聲
	}
王乃令趙奢將兵救之
去邯鄲三十里而止|{
	令盧經翻
	邯鄲音寒丹}
令軍中曰有以軍事諫者死|{
	令力正翻
	趙奢此令非以禁約所部以愚秦軍也}
秦師軍武安西|{
	班志武安縣屬魏郡
	宋白曰洛州治永年縣隋改廣平爲永年屬武安郡
	秦軍勒兵武安西即此地
	劉昫曰磁州治陽縣漢武安縣地隋又置武安縣亦屬磁州
	磁祥之翻
	昫吁勿翻
	}
鼓譟勒兵武安屋瓦盡振
趙軍中候有一人言急救武安趙奢立斬之|{
	此軍之中候也
	漢北軍中候之官本此
	或曰軍中之候軍吏也
	}
堅壁二
十八日不行復益增壘|{
	復扶又翻又音如字
	壘力水翻}
秦間入趙軍趙奢善食遣之
閒以報秦將|{
	閒古莧翻
	此孫子所謂反間也
	食祥吏翻
	將即亮翻
	}
秦將大喜曰夫去國三十里而軍不行乃增壘閼與非趙地也趙奢既已遣閒卷甲而趨|{
	卷讀曰捲
	凡捲舒之卷皆同音
	}
一日一夜而至去閼與五十里而軍軍壘成秦師聞之悉甲而往
趙軍士許歷請以軍事諫趙奢進之|{
	姓譜許姓本自姜姓炎帝之後太嶽之胤其後以國爲氏}
許歷曰秦人不意趙至此其來氣盛將軍必厚集其陳以待之|{
	陳讀曰陣}
不然必敗趙奢曰請受教許歷請刑趙奢曰胥後令邯鄲|{
	索隱曰案胥須古人通用今者須後令謂胥爲須須者待也待後令謂許歷之言更不擬誅之故更待後令也
	邯鄲二字當爲欲戰謂臨戰之時許歷復諫也
	余謂胥語絶許歷請刑趙奢令其且待也
	蓋謂敢諫者死邯鄲之令耳今既自邯鄲進軍近閼與矣許歷之諫固在邯鄲之後不當用邯鄲之令以殺之故曰後令邯鄲
	令力正翻
	邯鄲音寒丹
	令力丁翻
	}
許歷復請諫曰先據北山上者勝後至者敗趙奢許諾即發萬人趨之
秦師後至争山不得上|{
	趨七喻翻又音如字
	得上時掌翻
	}
趙奢縱兵擊秦師秦師大敗解閼與而還|{
	還從宣翻又音如字}
趙王封奢爲馬服君|{
	服虔曰馬服猶言服馬也
	括地志邯鄲縣西北有馬服山}
與廉藺同位以許歷爲國尉 穰侯言客卿竈於秦王|{
	穰人羊翻}
使伐齊取剛壽以廣其陶邑|{
	括地志故剛城在兖州龔丘縣
	壽鄆州之縣也
	余據唐志鄆州壽張縣武德初置壽州
	通鑑書此以發范睢間穰侯之事間古莧翻}
初魏人范睢|{
	姓譜范本陶唐氏之後墮會爲晉大夫食采於范後有氏焉睢音雖}
從中大夫須賈使於齊|{
	戰國之時仍周之制置上中下三大夫
	漢百官表中大夫掌論議
	須姓密須氏之後
	風俗通須姓太昊之後
	蓋本之須句
	使疏吏翻
	句音朐
	}
齊襄王聞其辯口私賜之金及牛酒
須賈以爲睢以國陰事告齊也歸而告其相魏齊
魏齊怒笞擊范睢折脅摺齒
睢佯死卷以簀置厠中使客醉者更溺之|{
	索隱曰折脅摺齒謂擊折其脅又拉折其齒也
	簀謂葦荻之薄用之以卷其屍也
	余謂簀字從竹蓋竹爲之非葦荻之薄也
	又謂竹東南之產北人貴之自江以北饒葦荻人率織之以爲薄寢或以爲薦籍
	索隱以葦薄爲簀習於所見而從俗所呼者耳
	相息亮翻
	笞丑之翻
	摺力答翻
	卷讀曰捲
	簀竹革翻
	更工衡翻
	溺奴吊翻
	}
以懲後令無妄言者|{
	令力丁翻
	}
范睢謂守者曰能出我我必有厚謝守者乃請棄簀中死人
魏齊醉曰可矣范睢得出
魏齊悔復召求之|{
	令盧經翻復扶又翻又音如字
	}
魏人鄭安平遂操范睢亡匿更姓名曰張禄
秦謁者王稽使於魏|{
	謁者秦官漢因之
	志云主殿上時節威儀
	謁者僕射一人爲謁者臺率其下有給事謁者有灌謁者
	操七刀翻
	使疏吏翻
	率讀曰帥
	}
范睢夜見王稽稽濳載與俱歸薦之於王王見之於離宮|{
	離宮别宮也}
范睢佯爲不知永巷而入其中|{
	佯音羊古字多作陽詐也
	如淳曰周宣王姜后脱簪珥待罪永巷後改爲掖庭
	師古曰永長也
	本謂宮中之長巷也或曰宮中獄也}
王來而宦者怒逐之曰王至范睢謬曰秦安得王秦獨有太后穰侯耳|{
	謬靡幼翻誤也詐也穰人羊翻}
王微聞其言乃屏左右跽而請曰先生何以幸教寡人對曰唯唯如是者三|{
	謬靡幼翻穰人羊翻屏卑郢翻又卑正翻後凡屛退之屛皆同音跽忌已翻跪也唯于癸翻蓋應聲也凡唯諾之唯皆同音}
王曰先生卒不幸教寡人邪|{
	卒子恤翻終也邪音耶}
范睢曰非敢然也|{
	睢音雖然猶言如是也}
臣覊旅之臣也交疎於王而所願陳者皆匡君之事處人骨肉之間|{
	處昌呂翻
	}
願效愚忠而未知王之心也此所以王三問而不敢對者也
臣知今日言之於前明日伏誅於後然臣不敢避也
且死者人之所必不免也苟可以少有補於秦而死此臣之所大願也|{
	少始紹翻}
獨恐臣死之後天下杜口裹足莫肯鄉秦耳|{
	謂天下之士懲睢之死不敢復言鄉讀曰嚮}
王跽曰先生是何言也今者寡人得見先生是天以寡人溷先生而存先王之宗廟也|{
	溷謂溷瀆之也
	漢陸賈曰毋久溷公即此義音戶困翻
	毛晃曰溷濁也又汚辱也
	}
事無大小上及太后下至大臣願先生悉以教寡人
無疑寡人也范睢拜王亦拜
范睢曰以秦國之大士卒之勇以治諸侯譬若走韓盧而搏蹇兎也|{
	韓盧天下之駿犬
	蹇兎病足之兎
	韓盧搏兎無不獲者况蹇兎乎
	治直之翻
	}
而閉關十五年不敢窺兵於山東者是穰侯爲秦謀不忠而大王之計亦有所失也|{
	穰人羊翻
	爲于偽翻
	}
王跽曰寡人願聞失計然左右多竊聼者范睢未敢言内先言外事以觀王之俯仰
因進曰夫穰侯越韓魏而攻齊剛壽非計也|{
	夫音扶
	}
齊湣王南攻楚破軍殺將|{
	謂殺唐昧也見上卷十四年
	湣讀曰閔
	將即亮翻
	}
再辟地千里|{
	辟讀曰闢昧莫葛翻
	}
而齊尺寸之地無得焉者豈不欲得地哉形勢不能有也
諸侯見齊之罷敝起兵而伐齊大破之齊幾於亡|{
	事見上卷三十一年
	罷讀曰疲
	幾居依翻
	}
以其伐楚而肥韓魏也
今王不如遠交而近攻得寸則王之寸也得尺亦王之尺也
今夫韓魏中國之處|{
	夫音扶康曰處敞呂翻余謂處昌據翻於世俗常言音義爲長}
而天下之樞也|{
	以門戶爲喻門戶之闔闢皆由於樞}
王若用霸必親中國以爲天下樞以威楚趙|{
	用霸者請用霸天下之術}
楚強則附趙趙強則附楚|{
	強者未易柔服故先親附弱者易以豉翻}
楚趙皆附齊必懼矣齊附則韓魏因可虜也王曰善乃以范睢爲客卿與謀兵事|{
	范睢謀兵事則三晉受兵禍而穰侯兄弟皆爲秦所逐矣}


四十六年


秦中更胡傷攻趙閼與不拔|{
	更工衡翻胡傷意謂即上卷客卿之胡陽
	閼於葛翻又於連翻
	與音預
	}


四十七年


秦王用范睢之謀使五大夫綰伐魏拔懷|{
	班志懷縣屬河内郡
	括地志曰懷縣在懷州武陟縣西十一里
	睢息隨翻
	}


四十八年


秦悼太子質於魏而卒|{
	質音致卒子恤翻}


四十九年

秦拔魏邢丘
范睢日益親用事因承間說王曰|{
	睢息隨翻
	閒古莧翻
	說式芮翻
	}
臣居山東時聞齊之有孟嘗君不聞有王
聞秦有太后穰侯不聞有王
夫擅國之謂王能利害之謂王制殺生之謂王
今太后擅行不顧穰侯出使不報華陽涇陽擊斷無諱|{
	夫音扶
	使疏吏翻
	華戶化翻
	斷丁亂翻凡斷决之斷皆同音
	}
高陵進不請四貴備而國不危者未之有也
爲此四貴者下乃所謂無王也
穰侯使者操王之重決制於諸侯剖符於天下|{
	操七刀翻
	謂剖符而出使也}
征敵伐國莫敢不聼戰勝攻取則利歸於陶|{
	陶穰侯封邑}
戰敗則怨結於百姓而禍歸於社稷
臣又聞之木實繁者披其枝披其枝者傷其心大其都者危其國|{
	左傳祭仲曰都城過百雉國之害也
	辛伯曰大都耦國亂之本也
	申無宇曰鄭京櫟實殺曼伯宋蕭亳實殺子游衛蒲戚實出獻公齊渠丘實殺無知而陳蔡不羮亦殺楚靈王
	此皆大都危國也
	傳直戀翻
	祭則介翻
	陸德明櫟音立曼音萬羮音郎}
尊其臣者卑其主|{
	如下事之類}
淖齒管齊射王股擢王䈥懸之於廟梁宿昔而死|{
	管掌也
	擢拔也
	宿昔一夕之間也
	淖齒弑齊湣王事見上卷三十一年
	淖奴教翻
	射而亦翻
	}
李兑管趙囚主父於沙丘百日而餓死|{
	事見上卷二十年}
今臣觀四貴之用事此亦淖齒李兑之類也
夫三代之所以亡國者
君專授政於臣
縱酒弋獵
其所授者妬賢疾能
御下蔽上以成其私
不爲主計
而主不覺悟
故失其國|{
	夫音扶}
今自有秩以上至諸大吏|{
	漢承秦制鄉置有秩
	漢官曰鄉戶五千則置有秩掌一鄉之入
	風俗通曰有秩則田間大夫言其官裁有秩耳
	大吏謂左右中更以上爲吏者也
	秩直乙翻
	}
下及王左右無非相國之人者見王獨立於朝臣竊爲王恐萬世之後有秦國者非王子孫也|{
	相息亮翻
	朝直遥翻
	爲干偽翻
	}
王以爲然於是廢太后逐穰侯高陵華陽涇陽君於關外以范睢爲丞相封爲應侯|{
	應於陵翻國名周武王之子封於應其地在唐安州界}
魏王使須賈聘於秦應侯敝衣間步而往見之|{
	間步投間隙徒步而行也間古莧翻}
須賈驚曰范叔固無恙乎|{
	范睢字叔恙憂也病也又噬蟲善食人心者也古人相問率曰無恙朱熹曰古者草居多被噬蟲之毒故相問曰無恙乎恙餘亮翻噬時制翻}
留坐飲食取一綈袍贈之|{
	綈田黎翻厚繒也袍步刀翻長襦也記玉藻曰纊爲繭緼爲孔頴達曰純著新綿者爲襺雜用舊絮者爲}
遂爲須賈御而至相府曰我爲君先入通於相君須賈怪其久不出問於門下門下曰無范叔鄉者吾相張君也|{
	相息亮翻爲于偽翻睢更姓名曰張禄故云然鄉讀曰嚮}
須賈知見欺乃膝行入謝罪|{
	膝行屈膝就地而行以示跪伏}
應侯坐責讓之且曰爾所以得不死者以綈戀戀尚有故人之意耳乃大供具請諸侯賓客坐須賈於堂下置莝豆於前而馬食之|{
	莝寸斬之藁雜豆以飼馬莝豆兩物也莝寸卧翻食祥吏翻}
使歸告魏王曰速斬魏齊頭來不然且屠大梁|{
	屠殺也自古以來以攻下城而盡殺城中人爲屠城亦曰洗城}
須賈還以告魏齊魏齊犇趙匿於平原君家|{
	還從宣翻又音如字平原君趙勝趙王之貴介弟也貴盛於趙以好士聞於諸侯故魏齊奔歸之而就匿焉}
趙惠文王薨子孝成王丹立以平原君爲相|{
	相息亮翻}


五十年

秦宣太后薨九月穰侯出之陶|{
	薨呼肱翻穰人羊翻}
臣光曰穰侯援立昭王除其災害|{
	事見三卷十年援于元翻手引也}
薦白起爲將|{
	見上卷二十二年將即亮翻}
南取鄢郢東屬地於齊|{
	言拓地東連於齊也事並見上卷鄢於晚翻郢以井翻屬之欲翻}
使天下諸侯稽首而事秦|{
	稽音啟}
秦益強大者穰侯之功也雖其專恣驕貪足以賈禍|{
	賈音古言其致禍如商賈之賈物也凡商賈之賈皆同音}
亦未至盡如范睢之言若睢者亦非能爲秦忠謀直欲得穰侯之處故搤其吭而奪之耳|{
	睢息隨翻爲于偽翻搤音厄說文曰捉也吭音剛咽也}
遂使秦王絶母子之義失舅甥之恩要之睢眞傾危之士哉

秦王以子安國君爲太子|{
	爲安國君立子異人爲嗣張本嗣祥吏翻}
秦伐趙取三城趙王新立太后用事求救於齊齊人曰必以長安君爲質|{
	索隱曰趙亦有長安今其地闕孔衍曰長安君惠文王之少子也史記正義曰長安君以長安善故名也質音致索山客翻少詩照翻}
太后不可
齊師不出大臣強諫|{
	強諫猶力諫也}
太后明謂左右曰復言長安君爲質者老婦必唾其面|{
	復扶又翻唾吐卧翻口液也明謂左右者顯言之也}
左師觸龍願見太后太后盛氣而胥之入|{
	胥待也言盛氣以待其入也}
左師公徐趨而坐自謝曰老臣病足不得見久矣竊自恕而恐太后體之有所苦也故願望見太后太后曰老婦恃輦而行曰食得毋衰乎曰恃粥耳太后不和之色稍解左師公曰老臣賤息舒祺|{
	春秋時宋國之官有左右師上卿也趙以觸龍爲左師蓋冗散之官以優老臣者也息子也祺音其冗而龍翻散悉亶翻}
最少不肖而臣衰竊憐愛之願得補黑衣之缺以衛王宮昧死以聞|{
	黑衣衛士之服也觸龍先爲其少子言以發太后之問也昧死言忘其死也少詩照翻又音小昧莫佩翻}
太后曰諾年幾何矣對曰十五歲矣雖少願及未填溝壑而託之|{
	幾居豈翻謙言死必填溝壑願及未死而託少子也}
太后曰丈夫亦愛少子乎對曰甚於婦人太后笑曰婦人異甚對曰老臣竊以爲媪之愛燕后賢於長安君|{
	媪烏浩翻婦之老者之稱趙太后之女嫁於燕故稱之曰燕后燕因肩翻}
太后曰君過矣不若長安君之甚左師公曰父母愛其子則爲之計深遠媪之送燕后也持其踵而泣念其遠也亦哀之矣已行非不思也祭祀則祝之曰必勿使反豈非爲之計長久爲子孫相繼爲王也哉太后曰然左師公曰今三世以前至於趙王之子孫爲侯者其繼有在者乎曰無有曰此其近者禍及身遠者及其子孫豈人主之子侯則不善哉位尊而無功奉厚而無勞|{
	奉讀曰俸凡奉禄之奉皆同音}
而挾重器多也今媪尊長安君之位而封之以膏腴之地多與之重器而不及今令有功於國|{
	令力丁翻使也}
一旦山陵崩長安君何以自託於趙哉太后曰諾恣君之所使之於是爲長安君約車百乘質於齊|{
	爲于偽翻乘䋲證翻質音致}
齊師乃出秦師退 齊安平君田單將趙師以伐燕取中陽|{
	徐廣曰陽一作人史記正義曰燕無中陽括地志中山故城一名中人亭在定州唐縣北四十一里是時蓋屬燕將即亮翻燕因肩翻}
又伐韓取注人|{
	括地志注城在汝州梁縣西四十五里}
齊襄王薨子建立建年少國事皆决於君王后|{
	少詩照翻}


五十一年


秦武安君伐韓取南陽攻太行道絶之|{
	秦封白起爲武安君韓之南陽即河内野王之地班志太行山在野王西北括地志在懷州河内縣北四十五里行戶剛翻}
楚頃襄王疾病|{
	疾至於甚曰病}
黃歇言於應侯曰今楚王疾恐不起秦不如歸其太子太子得立其事秦必重而德相國無窮是親與國而得儲萬乘也不歸則咸陽布衣耳|{
	四十三年黃歇與楚太子爲質於秦應於陵翻相息亮翻乘䋲證翻歇許竭翻}
楚更立君必不事秦|{
	更工衡翻}
是失與國而絶萬乘之和非計也應侯以告王王曰令太子之傅先往問疾反而後圖之黃歇與太子謀曰秦之留太子欲以求利也今太子力未能有以利秦也|{
	令力丁翻歇許竭翻}
而陽文君子二人在中王若卒大命|{
	謂死也卒終也音子恤翻}
太子不在陽文君子必立爲後太子不得奉宗廟矣不如亡秦與使者俱出|{
	逃去爲亡使疏吏翻}
臣請止以死當之太子因變服爲楚使者御而出關而黃歇守舍常爲太子謝病度太子已遠|{
	歇許竭翻守舍者守楚太子所寓館舍常爲于偽翻度徒洛翻}
乃自言於王曰楚太子已歸出遠矣歇願賜死王怒欲聼之應侯曰歇爲人臣出身以狥其主太子立必用歇不如無罪而歸之|{
	言無以罪加歇而歸之於楚以結其和親也應於陵翻}
以親楚王從之黃歇至楚三月秋頃襄王薨考烈王即位|{
	頃音傾秋即是年秋考烈王即太子完}
以黃歇爲相封以淮北地號曰春申君|{
	史記歇初封春申君賜淮北十四縣後徙封江東因城吳故墟以爲都邑今蘇州是也相息亮翻}


五十三年

楚人納州于秦以平|{
	司馬彪志南郡州陵縣註云楚考烈王納州于秦即其地}
武安君伐韓拔野王上黨路絶|{
	武安君上逸秦字史記正義曰從太行西北澤潞等州皆上黨郡地釋名云上黨所治在山上其所最高故曰上黨}
上黨守馮亭|{
	姓譜畢公高之子食采於馮城因以命氏鄭有大夫馮簡子守式又翻}
與其民謀曰鄭道已絶|{
	韓都新鄭自上黨趣鄭由野王度河今秦拔野王故鄭道絶}
秦兵日進韓不能應不如以上黨歸趙趙受我秦必攻之趙被秦兵必親韓|{
	應於證翻被皮義翻}
韓趙爲一則可以當秦矣乃遣使者告於趙曰韓不能守上黨入之秦|{
	謂韓獻上黨於秦使疏吏翻}
其吏民皆安於趙不樂爲秦|{
	爲于偽翻樂音洛下同}
有城市邑十七|{
	城市邑言邑之有城市者指言大邑也}
願再拜獻之大王趙王以告平陽君豹對曰聖人甚禍無故之利|{
	甚禍者言甚以爲禍也}
王曰人樂吾德何謂無故對曰秦蠶食韓地中絶不令相通固自以爲坐而受上黨也韓氏所以不入於秦者欲嫁其禍於趙也|{
	毛晃曰推惡與人曰嫁怨嫁禍推吐雷翻}
秦服其勞而趙受其利雖強大不能得之於弱小弱小固能得之於強大乎豈得謂之非無故哉不如勿受王以告平原君平原君請受之王乃使平原君往受地|{
	秦有吞天下之心使趙不受上黨而秦得之亦必據上黨而攻趙故趙之禍不在於受上黨而在於用趙括}
以萬戶都三封其太守爲華陽君|{
	守式又翻}
以千戶都三封其縣令爲侯吏民皆益爵三級馮亭垂涕不見使者曰吾不忍賣主地而食之也|{
	使疏吏翻}


五十五年

秦左庶長王齕攻上黨拔之|{
	長知丈翻齕音紇杜佑恨勿翻康胡骨翻}
上黨民走趙趙廉頗軍於長平|{
	司馬彪志上黨泫氏縣有長平亭括地志長平故城在上黨縣西四十一里杜佑曰白起阬趙卒於長平有頭顱山築臺於壘中因山爲臺宋白曰秦坑趙卒於長平今澤州之北高平縣西北二十一里長平故城是也頗普何翻泫工玄翻顱音盧壘魯水翻}
以按據上黨民|{
	毛晃曰按於旰翻抑也止也據也余謂此按據二字按字當以抑止爲義據依據也引援也拒守也言廉頗依據上黨地險引援上黨之民而拒守也康曰按音遏此義亦通但按字無遏音}
王齕因伐趙趙軍數戰不勝|{
	數所角翻}
止一禆將四尉|{
	禆將軍之副將也尉軍中諸部都尉也禆頻彌翻將即亮翻}
趙王與樓昌虞卿謀|{
	風俗通曰凡氏之興九事氏於號者唐虞夏殷是也氏於國者齊魯宋衛是也氏於事者巫卜陶匠是也氏於字者伯仲叔季是也氏於謚者戴武宣穆是也}
樓昌請發重使爲媾|{
	媾音構和也}
虞卿曰今制媾者在秦秦必欲破王之軍矣雖往請媾秦將不聼不如發使以重寶附楚魏楚魏受之則秦疑天下之合從|{
	從子容翻}
媾乃可成也王不聼使鄭朱媾於秦秦受之王謂虞卿曰秦内鄭朱矣|{
	虞卿時爲趙之相}
對曰王必不得媾而軍破矣何則天下之賀戰勝者皆在秦矣夫鄭朱貴人也秦王應侯必顯重之以示天下|{
	夫音扶應於陵翻}
天下見王之媾於秦必不救王秦知天下之不救王則媾不可得成矣既而秦果顯鄭朱而不與趙媾|{
	史言趙之喪師蹙國不特以趙括代廉頗之故亦由不用虞卿之計也}
秦數敗趙兵廉頗堅壁不出趙王以頗失亡多而更怯不戰怒數讓之|{
	數所角翻屢也敗補邁翻}
應侯又使人行千金於趙爲反間曰秦之所畏獨畏馬服君之子趙括爲將耳|{
	間古莧翻將即亮翻}
廉頗易與且降矣|{
	易弋䜴翻降戶江翻}
趙王遂以趙括代頗將藺相如曰王以名使括若膠柱鼓瑟耳|{
	鼔瑟者絃有緩急調絃之緩急在柱之運轉若膠其柱則絃不可得而調緩者一於緩急者一於急無活法矣}
括徒能讀其父書傳不知合變也|{
	兵以正合以奇變傳直戀翻}
王不聼初趙括自少時學兵灋以天下莫能當嘗與其父奢言兵事奢不能難|{
	少詩照翻難乃旦翻辯折之也}
然不謂善括母問其故奢曰兵死地也而括易言之|{
	易以䜴翻輕也}
使趙不將括則己若必將之|{
	將即亮翻下同}
破趙軍者必括也及括將行其母上書言括不可使王曰何以|{
	上時掌翻言以何事知其不可使也}
對曰始妾事其父時爲將身所奉飯而進食者以十數|{
	奉讀曰捧}
所友者以百數王及宗室所賞賜者盡以與軍吏士大夫受命之日不問家事今括一旦爲將東鄉而朝軍吏無敢仰視之者|{
	將即亮翻朝直遥翻}
王所賜金帛歸藏于家而日視便利田宅可買者買之王以爲如其父父子異心願王勿遣王曰母置之吾已决矣|{
	置止也廢也置之言廢置此事止勿言也}
母因曰即如有不稱妾請無隨坐|{
	稱尺證翻不稱言不勝任也隨坐相隨而坐罪也觀此則知古者敗軍之將罪併及其家}
趙王許之秦王聞括己爲趙將乃陰使武安君爲上將軍而王齕爲禆將令軍中有敢泄武安君將者斬趙括至軍悉更約束|{
	齕恨勿翻更工衡翻}
易置軍吏出兵擊秦師武安君佯敗而走|{
	佯音羊詐也}
張二奇兵以劫之|{
	劫勢脅也說文人欲去以力脅止曰劫}
趙括乘勝追造秦壁|{
	造七到翻詣也}
壁堅拒不得入奇兵二萬五千人絶趙軍之後又五千騎絶趙壁間|{
	騎奇寄翻}
趙軍分而爲二糧道絶武安君出輕兵擊之趙戰不利因築壁堅守以待救至秦王聞趙食道絶自如河内發民年十五以上悉詣長平遮絶趙救兵及糧食|{
	如往也上時掌翻遮者遮斷其路}
齊人楚人救趙趙人乏食請粟于齊王弗許周子曰夫趙之於齊楚扞蔽也|{
	夫音扶}
猶齒之有脣也脣亡則齒寒今日亡趙明日患及齊楚矣救趙之務宜若奉漏甕沃焦釡然|{
	奉讀曰捧言惟恐不及也}
且救趙高義也却秦師顯名也義救亡國威却強秦不務爲此而愛粟爲國計者過矣齊王弗聼九月趙軍食絶四十六日皆内陰相殺食急來攻壘|{
	史言急來攻壘趙括爲計如此耳下言欲出而不能出趙括自出而死其勢可見}
欲出爲四隊四五復之不能出|{
	言括欲分其卒爲四隊更攻秦壘自一隊至四隊至五則復之而不能出也}
趙括自出鋭卒搏戰秦人射殺之|{
	射而亦翻}
趙師大敗卒四十萬人皆降|{
	降戶江翻}
武安君曰秦已拔上黨上黨民不樂爲秦而歸趙趙卒反覆非盡殺之恐爲亂乃挾詐而盡坑殺之遺其小者二百四十人歸趙|{
	樂爲上音洛下于偽翻又音如字四十餘萬人皆死而獨遺小者二百四十人得歸趙此非得脱也白起之譎也強壯盡死則小弱得歸者必言秦之兵威所以破趙人之膽將以乘勝取邯鄲也爲應侯所沮故白起之計不得行耳譎古宂翻邯鄲音寒丹應於陵翻沮在呂翻卒子恤翻}
前後斬首虜四十五萬人趙人大震|{
	此言秦兵自挫廉頗至大破趙括前後所斬首虜之數耳兵非大敗四十萬人安肯束身而死邪}


五十六年

十月武安君分軍爲三王齕攻趙武安皮牢拔之|{
	史記正義曰皮牢故城在絳州龍門縣西一里余謂秦兵已至上黨不應復回攻絳州之皮牢宋白曰蒲州龍門縣秦爲皮氏縣今縣西一里八十步古皮氏城是也恐不可以皮氏爲皮牢}
司馬梗北定太原|{
	太原即漢太原郡地在上黨西北}
盡有上黨地韓魏使蘇代厚幣說應侯曰武安君即圍邯鄲乎|{
	說式芮翻邯鄲音寒丹}
曰然蘇代曰趙亡則秦王王矣|{
	秦之稱王自王其國耳今破趙國則將王天下也}
武安君爲三公君能爲之下乎雖欲無爲之下固不得已矣秦嘗攻韓圍邢丘困上黨|{
	四十九年通鑑書秦拔魏邢丘豈其時邢丘之地固屬韓邪}
上黨之民皆反爲趙天下不樂爲秦民之日久矣|{
	樂音洛}
今亡趙北地入燕東地入齊南地入韓魏則君之所得民無幾何人矣不如因而割之無以爲武安君功也應侯言於秦王曰秦兵勞請許韓趙之割地以和且休士卒王聼之割韓垣雍趙六城以和|{
	應於陵翻司馬彪志河南卷縣有垣雍城或曰古衡雍注曰今縣所治城是也史記正義曰垣雍城在今鄭州原武縣西北七里雍於用翻}
正月皆罷兵|{
	觀此則亦用十月爲歲首盖因秦記而書之也}
武安君由是與應侯有隙|{
	爲秦殺白起張本}
趙王將使趙郝約事於秦割六縣|{
	約事約結和之事也郝音釋徐廣曰一作赦}
虞卿謂趙王曰秦之攻王也倦而歸乎王以其力尚能進愛王而弗攻乎王曰秦不遺餘力矣必以倦而歸也|{
	遺失也}
虞卿曰秦以其力攻其所不能取倦而歸王又以其力之所不能取以送之是助秦自攻也來年秦攻王王無救矣|{
	言無救於講和之失計也}
趙王計未定樓緩至趙趙王與之計之樓緩曰虞卿得其一不得其二秦趙搆難而天下皆說|{
	難乃旦翻說讀曰悦}
何也曰吾且因強而乘弱矣今趙不如亟割地爲和以疑天下|{
	緩謂趙與秦和則天下疑趙有秦之援將不敢乘弱而圖之}
慰秦之心不然天下將因秦之怒乘趙之敝瓜分之趙且亡何秦之圖乎虞卿聞之復見曰危哉樓子之計是愈疑天下|{
	復扶又翻又因如字卿謂趙與秦和則天下愈疑而不肯親趙也}
而何慰秦之心哉獨不言其示天下弱乎且臣言勿與者非固勿與而已也秦索六城於王而王以六城賂齊齊秦之深讐也|{
	索山客翻齊自宣湣以來親楚而讐秦孟嘗君嘗率諸侯伐秦至函谷湣讀曰閔}
其聼王不待辭之畢也則是王失之於齊而取償於秦而示天下有能爲也|{
	言趙失地於賂齊而能攻秦取其地以償所失}
王以此發聲兵未窺於境臣見秦之重賂至趙而反媾於王也從秦爲媾韓魏聞之必盡重王|{
	媾居候翻說文媾重婚也引易匪寇婚媾夫已婚而夫妻反目而不和既而復和者爲媾此言秦趙爲寇讐而交兵至今而復和故以媾爲言也重直龍翻}
是王一舉而結三國之親而與秦易道也|{
	秦脅韓魏使事秦趙結韓魏使親趙是與秦易道易音如字}
趙王曰善使虞卿東見齊王與之謀秦虞卿未返秦使者已在趙矣|{
	求和於趙也使疏吏翻}
樓緩聞之亡去趙王封虞卿以一城秦之始伐趙也魏王問於大夫皆以爲秦伐趙於魏便孔斌曰|{
	斌悲巾翻}
何謂也曰勝趙則吾因而服焉不勝趙則吾承敝而擊之子順曰不然秦自孝公以來戰未嘗屈今又屬其良將|{
	良將謂白起也屬之欲翻將即亮翻}
何敝之承大夫曰縱其勝趙於我何損鄰之羞國之福也子順曰秦貪暴之國也勝趙必復他求|{
	復扶又翻}
吾恐於時魏受其師也|{
	於時猶言於此時也}
先人有言鷰雀處屋|{
	處昌呂翻}
子母相哺呴呴焉相樂也|{
	哺音步呴或作姁音况羽翻康吁句切樂音洛}
自以爲安矣竈突炎上|{
	突徒忽翻竈䆫謂之突陸德明曰上時掌翻又如字}
棟宇將焚燕雀顔不變不知禍之將及己也今子不悟趙破患將及己可以人而同於燕雀乎子順者孔子六世孫也|{
	孔子生伯魚伯魚生子思子思生子上子上生子家子家生子京子京生子高子高生子順}
初魏王聞子順賢遣使者奉黃金束帛聘以爲相|{
	使疏吏翻相息亮翻}
子順曰若王能信用吾道吾道固爲治世也|{
	爲于偽翻治直之翻}
雖蔬食飲水吾猶爲之若徒欲制服吾身委以重禄吾猶一夫耳魏王奚少於一夫|{
	食祥吏翻少始紹翻}
使者固請子順乃之魏|{
	之如也往也}
魏王郊迎以爲相子順改嬖寵之官以事賢才奪無任之禄以賜有功|{
	嬖卑義翻又博計翻無任之禄謂不任事而食禄者}
諸喪職者咸不悦乃造謗言|{
	喪息浪翻}
文咨以告子順|{
	文姓也越有大夫文種}
子順曰民之不可與慮始久矣古之善爲政者其初不能無謗子產相鄭三年而後謗止|{
	左傳子產相鄭一年輿人誦之曰取我衣冠而褚之取我田疇而伍之孰殺子產吾其與之及三年又誦之曰我有子弟子產誨之我有田疇子產殖之子產而死誰其嗣之禇丁呂翻禇所以貯藏衣物左傳鄭賈人欲脱智罃將寘諸禇中而出}
吾先君之相魯三月而後謗止今吾爲政日新雖不能及賢庸知謗乎文咨曰未識先君之謗何也子順曰先君相魯人誦之曰麛裘而芾投之無戾芾而麛裘投之無郵及三月政化既成民又誦曰衮衣章甫實獲我所章甫衮衣惠我無私|{
	麛鹿子也以其皮爲裘記曰一命緼芾黝珩再命赤芾黝珩三命赤芾葱珩大夫以上赤芾乘軒戾罪也郵與尤同過也章甫殷冠孔子曰丘長居宋冠章甫之冠古者大夫羔裘以居狐裘以朝麛裘而芾謂芾與麛裘相稱刺孔子也衮衣而章甫言孔子相魯能行古之道也麛莫兮翻康綿披切芾分勿翻協韻方蓋翻戾郎計翻康曰力結切曲也音義非}
文咨喜曰乃今知先生不異乎聖賢矣子順相魏凡九月陳大計輒不用乃喟然曰|{
	喟去貴翻喟然發嘆之聲}
言不見用是吾言之不當也言不當於主|{
	當丁浪翻}
居人之官食人之禄是尸利素餐吾罪深矣|{
	尸主也素空也尸利言仕不能行道而主於利也素餐言空食君之禄而不能有所爲也}
退而以病致仕|{
	致仕言致其仕事}
人謂子順曰王不用子子其行乎答曰行將何之山東之國將并於秦秦爲不義義所不入遂寢於家新垣固請子順曰|{
	新垣姓也陳留風俗傳周畢公之後居於梁爲新垣氏梁有新垣衍漢有新垣平是也}
賢者所在必興化致治|{
	治直吏翻}
今子相魏未聞異政而即自退意者志不得乎何去之速也子順曰以無異政所以自退也且死病無良醫|{
	病不可爲則良醫束手故無良醫}
今秦有吞食天下之心以義事之固不獲安救亡不暇何化之興昔伊摯在夏|{
	伊摯即伊尹伊尹五就桀五就湯摯音至}
呂望在商|{
	史記曰太公博聞嘗事紂紂無道去之游說諸侯無所遇而卒西歸周西伯}
而二國不治豈伊呂之不欲哉勢不可也|{
	治直吏翻}
當今山東之國敝而不振三晉割地以求安二周折而入秦燕齊楚已屈服矣|{
	燕因肩翻}
以此觀之不出二十年天下其盡爲秦乎|{
	自此至秦始皇二十五年并天下凡三十八年}
秦王欲爲應侯必報其仇|{
	爲于偽翻應於陵翻}
聞魏齊在平原君所|{
	四十九年魏齊奔趙匿於平原君家}
乃爲好言誘平原君至秦而執之|{
	誘音酉}
遣使謂趙王曰不得齊首吾不出王弟於關魏齊窮抵虞卿虞卿棄相印與魏齊偕亡|{
	使疏吏翻相息亮翻}
至魏欲因信陵君以走楚信陵君意難見之魏齊怒自殺趙王卒取其首以與秦|{
	卒子恤翻}
秦乃歸平原君九月五大夫王陵復將兵伐趙|{
	復扶又翻}
武安君病不任行|{
	任如林翻不任謂不堪也}


五十七年

正月王陵攻邯鄲少利|{
	邯鄲音寒丹少利謂兵頗失利也少始紹翻}
益發卒佐陵陵亡五校|{
	校戶教翻校猶部隊也立軍之法一人曰獨二人曰比三人曰參比參曰伍五人爲列列有頭二列爲火十人有長立火子五火爲隊隊五十人有頭二隊爲官官百人立長二官爲曲曲二百人立侯二曲爲部部四百人立司馬二部爲校校八百人立尉二校爲禆將千六百人立將軍二禆將軍三千二百人有將軍副將軍}
武安君病愈王欲使代之武安君曰邯鄲實未易攻也|{
	易以豉翻}
且諸侯之救日至彼諸侯怨秦之日久矣秦雖勝於長平士卒死者過半國内空遠絶河山而争人國都|{
	自秦而攻邯鄲有大河及王屋大行諸山之阻横度曰絶}
趙應其内諸侯攻其外破秦軍必矣王自命不行|{
	秦王親命之行而不肯行也}
乃使應侯請之武安君終辭疾不肯行乃以王齕代王陵|{
	應於陵翻齕恨勿翻}
趙王使平原君求救於楚平原君約其門下食客文武備具者二十人與之俱得十九人餘無可取者毛遂自薦於平原君|{
	姓譜毛本自周武王母弟毛公}
平原君曰夫賢士之處世也譬若錐之處囊中其末立見|{
	夫音扶毛晃曰錐鋭也又器如鑽囊袋也有底曰囊處昌呂翻見賢遍翻}
今先生處勝之門下三年於此矣左右未有所稱誦勝未有所聞是先生無所有也先生不能先生留|{
	以毛遂爲不能而使之留也}
毛遂曰臣乃今日請處囊中耳使遂蚤得處囊中乃脱頴而出非特其末見而已|{
	處昌呂翻見賢遍翻毛晃曰錐鋩曰穎}
平原君乃與之俱十九人相與目笑之|{
	索隱曰謂目視而侮笑之}
平原君至楚與楚王言合從之利害日出而言之日中不决毛遂按劍歷階而上謂平原君曰從之利害兩言而决耳|{
	兩言謂利與害也從子容翻上時掌翻}
今日出而言日中不决何也楚王怒叱曰胡不下|{
	胡何也}
吾乃與而君言|{
	而猶汝也}
汝何爲者也毛遂按劍而前曰王之所以叱遂者以楚國之衆也今十步之内王不得恃楚國之衆也王之命懸於遂手吾君在前叱者何也且遂聞湯以七十里之地王天下|{
	王于况翻}
文王以百里之壤而臣諸侯豈其士卒衆多哉誠能據其勢而奮其威也今楚地方五千里持戟百萬此霸王之資也以楚之強天下弗能當白起小豎子耳率數萬之衆興師以與楚戰一戰而舉鄢郢再戰而燒夷陵|{
	事見上卷三十七年史記正義曰鄢鄉故城在襄州率道縣西南九里安郢城在荆州江陵縣東北七里鄢於幰翻郢以井翻}
三戰而辱王之先人|{
	謂焚夷楚之陵廟也}
此百世之怨而趙之所羞而王弗之惡焉合從者爲楚非爲趙也吾君在前叱者何也楚王曰唯唯|{
	惡烏路翻爲于偽翻唯于癸翻}
誠若先生之言謹奉社稷以從|{
	從如字}
毛遂曰從定乎楚王曰定矣毛遂謂楚王之左右曰取雞狗馬之血來|{
	索隱曰盟之所用牲貴賤不同天子用牛及馬諸侯用犬及豭大夫以下用雞今此總言盟之用血故云取雞狗馬之血來耳索山客翻豭居牙翻}
毛遂奉銅盤而跪進之楚王曰王當歃血以定從|{
	索隱曰歃血若周禮則用珠盤奉讀曰捧歃色洽翻又所甲翻從子容翻}
次者吾君次者遂遂定從於殿上毛遂左手持盤血而右手招十九人曰公等相與歃此血於堂下公等録録所謂因人成事者也|{
	說文録録隨從之貌音禄索隱曰音六王劭曰録借字耳}
平原君已定從而歸至於趙曰勝不敢相天下士矣|{
	相息亮翻}
遂以毛遂爲上客於是楚王使春申君將兵救趙魏王亦使將軍晉鄙將兵十萬救趙|{
	晉以國爲氏}
秦王使謂魏王曰吾攻趙旦暮且下諸侯敢救之者吾已拔趙必移兵先撃之魏王恐遣人止晉鄙留兵壁鄴|{
	班志鄴縣屬魏郡}
名爲救趙實挾兩端|{
	兩端名爲救趙實貳於秦}
又使將軍新垣衍間入邯鄲|{
	間入由間道而入也間古莧翻邯鄲音寒丹}
因平原君說趙王欲共尊秦爲帝以却其兵齊人魯仲連在邯鄲聞之往見新垣衍曰彼秦者棄禮義而上首功之國也|{
	秦以戰而能斬首有功者爲上故曰上首功上尚也索隱曰秦法斬首多爲上功斬一人首則賜爵一級故謂秦爲上首功之國}
彼即肆然而爲帝於天下則連有蹈東海而死耳不願爲之民也且梁未睹秦稱帝之害故耳吾將使秦王烹醢梁王新垣衍怏然不悦|{
	怏於兩翻}
曰先生惡能使秦王烹醢梁王|{
	惡音烏}
魯仲連曰固也吾將言之昔者九侯鄂侯文王紂之三公也九侯有子而好獻之於紂紂以爲惡醢九侯|{
	醢呼改翻肉醬也}
鄂侯争之強辯之疾故脯鄂侯文王聞之喟然而嘆故拘之牖里之庫百日欲令之死|{
	司馬彪志河内郡蕩隂縣有牖里城紂囚文王於此史記正義曰其地在蕩陰縣北九里喟于貴翻牖音酉令力丁翻}
今秦萬乘之國也梁亦萬乘之國也俱據萬乘之國各有稱王之名奈何睹其一戰而勝欲從而帝之卒就脯醢之地乎|{
	乘䋲證翻卒子恤翻}
且秦無己而帝則將行其天子之禮以號令於天下則且變易諸侯之大臣彼將奪其所不肖而與其所賢奪其所憎而與其所愛彼又將使其子女讒妾爲諸侯妃姬處梁之宮|{
	處昌呂翻}
梁王安得晏然而已乎而將軍又何以得故寵乎新垣衍起再拜曰吾乃今知先生天下之士也吾請出不敢復言帝秦矣|{
	復扶又翻考異曰史記魯仲連傳云新垣衍謝請出不敢復言帝秦秦將聞之爲却軍五十里按仲連所言不過論帝秦之利害耳使新垣衍慙怍而去則有之秦將何預而退軍五十里乎此亦游談者之誇大也不取垣于元翻怍才各翻將即亮翻爲于偽翻}
燕武成王薨子孝王立初魏公子無忌仁而下士|{
	下遐稼翻}
致食客三千人魏有

隱士曰侯嬴|{
	洪氏隸釋有漢金鄉守長侯君之碑云其先出自岐周文王之後封於鄭鄭共仲賜氏曰侯厥胤宣多以功佐國審如是則侯姓出於侯宣多贏音盈曹植音羸瘦之羸}
年七十家貧爲大梁夷門監者|{
	大梁魏都夷門蓋大梁城北門監古衘翻}
公子置酒大會賓客坐定公子從車騎虚左自迎侯生|{
	古者乘車尊者在左虛左以迎尊侯生而禮之也騎奇寄翻}
侯生攝敝衣冠直上載公子上坐不讓|{
	直上時掌翻上坐之坐才卧翻}
公子執轡愈恭侯生又謂公子曰臣有客在市屠中願枉車騎過之公子引車入市侯生下見其客朱亥|{
	姓譜朱本高陽周封其後於邾後爲楚所滅子孫乃去邑氏朱}
睥睨故久立與其客語|{
	睥睨不正視也睥正詣翻睨研計翻}
微察公子公子色愈和乃謝客就車至公子家公子引侯生坐上坐徧贊賓客賓客皆驚|{
	索隱曰贊告也謂以侯生徧告賓客徧與遍同}
及秦圍趙趙平原君之夫人公子無忌之姊也平原君使者冠蓋相屬於魏|{
	屬之欲翻下乃屬同}
讓公子曰勝所以自附於婚姻者以公子之高義能急人之困也今邯鄲旦暮降秦而魏救不至縱公子輕勝棄之獨不憐公子姊邪|{
	邯鄲音寒丹降戶江翻邪音耶}
公子患之數請魏王勑晉鄙令救趙|{
	數所角翻令力丁翻}
及賓客辯士游說萬端王終不聼公子乃屬賓客|{
	說式芮翻屬之欲翻}
約車騎百餘乘欲赴鬬以死於趙過夷門見侯生侯生曰公子勉之矣老臣不能從|{
	乘䋲證翻從才用翻}
公子去行數里心不快|{
	以侯生既不從行乂不爲之畫計謀也}
復還見侯生侯生笑曰臣固知公子之還也|{
	復扶又翻又音如字還從宣翻又音如字}
今公子無佗端而欲赴秦軍|{
	無佗端言無佗奇策以發端也}
譬如以肉投餒虎何功之有公子再拜問計侯嬴屛人曰吾聞晉鄙兵符在王卧内而如姬最幸力能竊之嘗聞公子爲如姬報其父仇|{
	屛必郢翻史記曰如姬之父爲人所殺公子使客斬其仇頭以進如姬屛卑郢翻爲于偽翻下同}
如姬欲爲公子死無所辭公子誠一開口則得虎符|{
	虎威猛之獸故以爲兵符漢有銅虎符}
奪晉鄙之兵北救趙西却秦此五伯之功也|{
	伯讀曰霸}
公子如其言果得兵符公子行侯生曰將在外君令有所不受|{
	孫武子之言將即亮翻令力定翻}
有如晉鄙合符而不授兵復請之則事危矣|{
	復扶又翻}
臣客朱亥其人力士可與俱晉鄙若聼大善不聼可使擊之於是公子請朱亥與俱至鄴晉鄙合符疑之舉手視公子曰吾擁十萬之衆屯於境上今單車來代之何如哉朱亥䄂四十斤鐵椎椎殺晉鄙|{
	椎直追翻齊人謂之終葵鐵椎以鐵爲之椎殺擊殺也與槌同}
公子遂勒兵下令軍中曰父子俱在軍中者父歸兄弟俱在軍中者兄歸獨子無兄弟者歸養|{
	養羊尚翻後養上爲養同}
得選兵八萬人將之而進王齕久圍邯鄲不拔諸侯來救戰數不利|{
	齕恨勿翻邯鄲音寒丹數所角翻}
武安君聞之曰王不聼吾計今何如矣|{
	白起以爲邯鄲未易攻而王齕軍果不利故以爲言}
王聞之怒強起武安君|{
	強其兩翻}
武安君稱病篤不肯起


五十八年

十月免武安君爲士伍遷之陰密|{
	如淳曰律有罪失官爵稱士伍師古曰謂奪其爵令爲士伍言使從士卒之伍也班志陰密縣屬安定郡古密國詩所謂密人不恭者也括地志陰密故城在涇州鶉觚縣西古密須氏之國}
十二月益發卒軍汾城旁|{
	汾城即漢河東臨汾縣城也去邯鄲尚遠秦蓋屯兵于此爲王齕聲援括地志臨汾故城在絳州正平縣東北三十五里}
武安君病未行諸侯攻王齕齕數却|{
	齕恨勿翻數所角翻}
使者日至|{
	使疏吏翻}
王乃使人遣武安君不得留咸陽中武安君出咸陽西門十里至杜郵|{
	水經注渭水故渠逕安陵南渠側有杜郵亭又逕渭城北秦咸陽漢之渭城也史記正義曰今咸陽縣城本秦時杜郵也在雍州西北三十五里郵音尤雍于用翻}
王與應侯羣臣謀曰白起之遷意尚怏怏有餘言|{
	怏于兩翻}
王乃使使者賜之劍武安君遂自殺秦人憐之鄉邑皆祭祀焉魏公子無忌大破秦師於邯鄲下王齕解邯鄲圍走鄭安平爲趙所困將二萬人降趙|{
	邯鄲音寒丹降戶江翻}
應侯由是得罪|{
	鄭安平匿范睢以見王稽因此入秦爲相故睢保任安平而用之今安平降趙故睢由此得罪秦法保任其人而不稱者與同罪應于陵翻}
公子無忌既存趙遂不敢歸魏與賓客留居趙使將將其軍還魏|{
	將即亮翻還從宣翻又音如字}
趙王與平原君計以五城封公子趙王掃除自迎執主人之禮引公子就西階公子側行辭讓從東階上|{
	記曲禮主人就東階客就西階客若降等則就主人之階上時掌翻}
自言辠過|{
	辠古罪字秦始皇以辠字近皇字改爲罪}
以負於魏無功於趙趙王與公子飲至暮口不忍獻五城以公子退讓也趙王以鄗爲公子湯沐邑|{
	師古曰凡言湯沐邑謂以其賦税供湯沐之具也鄗呼各翻}
魏亦復以信陵奉公子|{
	杜佑曰信陵君邑于今宋州寧陵縣}
公子聞趙有處士毛公隱於博徒薛公隱於賣漿家|{
	處昌呂翻姓譜薛本自黃帝任姓之後裔孫奚仲居薛歷夏殷周六十四代爲諸侯後因氏焉}
欲見之兩人不肯見公子乃間步從之游平原君聞而非之公子曰吾聞平原君之賢故背魏而救趙今平原君所與遊徒豪舉耳|{
	間古莧翻背蒲妹翻索隱曰謂豪者舉之}
不求士也以無忌從此兩人遊尚恐其不我欲也平原君乃以爲羞乎爲裝欲去|{
	爲裝者爲行裝也}
平原君免冠謝乃止平原君欲封魯連|{
	以其折新垣衍言帝秦也}
使者三返終不肯受|{
	使疏吏翻}
又以千金爲魯連壽魯連笑曰所貴於天下士爲人排患釋難解紛亂而無取也|{
	爲人之爲于偽翻難乃旦翻}
即有取是商賈之事也|{
	賈音古下同}
遂辭平原君而去終身不復見|{
	復扶又翻}
秦太子之妃曰華陽夫人|{
	蓋食湯沐邑于華陽因以爲號華戶化翻}
無子夏姫生子異人異人質於趙秦數伐趙趙人不禮之|{
	夏戶雅翻質音致數所角翻}
異人以庶孽孫質於諸侯車乘進用不饒|{
	張晏曰孺子曰孽子何休曰孽子賤子也非嫡正之子曰孽師古曰孽庶子也唐韻曰猶木之有孽生也異人于秦太子爲庶子于秦王爲庶孽孫孽魚列翻索隱曰進者財也宜依小顔讀爲賮古字多假借用之進音才刃翻}
居處困不得意陽翟大賈呂不韋適邯鄲見之曰此奇貨可居|{
	賈音古邯鄲音寒丹賈人居積滯貨伺時以牟利以異人方財貨也}
乃往見異人說曰吾能大子之門|{
	說式芮翻}
異人笑曰且自大君之門不韋曰子不知也吾門待子門而大異人心知所謂乃引與坐深語不韋曰秦王老矣太子愛華陽夫人夫人無子子之兄弟二十餘人子傒有秦國之業|{
	華戶化翻下同子傒蓋秦太子之子愛而居長者康曰傒胡啟切余謂傒字即左傳高傒之傒陸德明曰傒音兮}
士倉又輔之|{
	姓譜士姓晉士蒍之後}
子居中不甚見幸久質諸侯太子即位子不得争爲嗣矣|{
	質音致嗣祥吏翻}
異人曰然則奈何不韋曰能立適嗣者獨華陽夫人耳|{
	適讀曰嫡下爲適同}
不韋雖貧請以千金爲子西遊立子爲嗣異人曰必如君策請得分秦國與君共之不韋乃以五百金與異人令結賓客復以五百金買奇物玩好|{
	復扶又翻好呼到翻}
自奉而西見華陽夫人之姊而以奇物獻於夫人因譽子異人之賢|{
	譽音余}
賓客徧天下常日夜泣思太子及夫人曰異人也以夫人爲天夫人大喜不韋因使其姊說夫人曰|{
	說式芮翻}
夫以色事人者色衰則愛弛|{
	夫音扶}
今夫人愛而無子不以繁華時蚤自結於諸子中賢孝者舉以爲適|{
	適讀曰嫡}
即色衰愛弛雖欲開一言尚可得乎今子異人賢而自知中子|{
	中讀曰仲}
不得爲適夫人誠以此時拔之是子異人無國而有國夫人無子而有子也則終身有寵於秦矣夫人以爲然承間言於太子曰|{
	間古莧翻}
子異人絶賢|{
	毛晃曰絶奇冠也相去遼遠也}
來往者皆稱譽之因泣曰妾不幸無子願得子異人立以爲子以託妾身太子許之與夫人刻玉符約以爲嗣因厚餽遺異人|{
	嗣祥吏翻遺于季翻}
而請呂不韋傅之異人名譽盛於諸侯呂不韋娶邯鄲諸姬絶美者與居|{
	娶字當從史記作取邯鄲音寒丹}
知其有娠|{
	應劭曰娠震動懷任之意左傳曰邑姜方娠孟康曰娠音身漢史娠多作身古今字也師古曰孟說是也漢書皆以娠爲任身字邑姜方震自震動之震不作娠}
異人從不韋飲見而請之不韋佯怒既而獻之孕期年而生子政|{
	佯音羊期讀曰朞盖任身十二月而生也子政是爲始皇爲呂不韋以此賈禍張本}
異人遂以爲夫人邯鄲之圍趙人欲殺之異人與不韋行金六百斤予守者|{
	予讀兑曰與月}
亡赴秦軍遂得歸異人楚服而見華陽夫人|{
	楚服爲楚人之服或曰楚楚盛服也}
夫人曰吾楚人也當自子之因更其名曰楚|{
	更工衡翻}


五十九年

秦將軍摎伐韓|{
	摎史記正義紀虬翻康曰居由切}
取陽城負黍斬首四萬伐趙取二十餘縣斬首虜九萬赧王恐背秦與諸侯約從將天下鋭師出伊闕攻秦令無得通陽城|{
	從子容翻將即亮翻令力丁翻}
秦王使將軍摎攻西周赧王入秦頓首受罪盡獻其邑三十六口三萬秦受其獻歸赧王於周是歲赧王崩|{
	皇甫諡曰周凡三十七王八百六十七年}


資治通鑑卷五
