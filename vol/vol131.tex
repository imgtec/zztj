










 


 
 


 

  
  
  
  
  





  
  
  
  
  
 
  

  

  
  
  



  

 
 

  
   




  

  
  


    資治通鑑卷一百三十一 宋 司馬光 撰

  胡三省 音註

  宋紀十三【柔兆敦牂二年】

  太宗明皇帝上之下

  泰始二年春正月己丑朔魏大赦改元天安 癸巳徵會稽太守尋陽王子房為撫軍將軍以巴陵王休若代之【去年子房已舉兵應尋陽欲以休若代之會工外翻】甲午中外戒嚴以司徒建安王休仁都督征討諸軍事車騎將軍江州刺史王玄謨副之【使王玄謨拒尋陽之兵因以為江州不復用休祐】休仁軍於南州以沈攸之為尋陽太守將兵屯虎檻【虎檻州名在赭圻東北江中蕪湖之西南也守式又翻將即亮翻下同】時玄謨未前鋒凡十軍絡繹繼至每夜各立姓號不相禀受攸之謂諸將曰今衆軍姓號不同若有耕夫漁父夜相呵叱【呵虎何翻叱昌栗翻】便致駭亂取敗之道也請就一軍取號衆咸從之【史言沈攸之有將帥之畧所以能立功】鄧琬稱說符瑞詐稱受路太后璽書帥將佐上尊號於晉安王子勛【璽斯氏翻帥讀曰率上時掌翻勛古勲字】乙未子勛即皇帝位於尋陽改元義嘉以安陸王子綏為司徒揚州刺史尋陽王子房臨海王子頊並加開府儀同三司以鄧琬為尚書右僕射張悦為吏部尚書袁顗加尚書左僕射【顗魚豈翻】其餘將佐及諸州郡除官進爵號各有差 丙申以征虜司馬申令孫為徐州刺史令孫坦之子也【元嘉孝建之間申坦為將帥】置司州於義陽【文帝元嘉末置司州於汝南孝武大明中省廢今復置之領義陽隨陽安陸南汝南四郡水行至建康二千七百里陸行一千七百里】以義陽内史龎孟蚪為司州刺史【龎皮江翻蚪渠幽翻】徐州刺史薛安都冀州刺史清河崔道固皆舉兵應尋陽上徵兵於青州刺史沈文秀文秀遣其將劉彌之等將兵赴建安會薛安都遣使邀文秀文秀更令彌之等應安都濟隂太守申闡據睢陵應建康【將即亮翻使疏吏翻濟子禮翻睢音雖睢陵縣前漢屬臨淮郡後漢屬下邳郡孝武大明元年度屬濟隂郡沈約曰濟隂本屬兖州其民流寓徐土因割地為郡境隋并睢陵入夏丘縣唐以夏丘為虹縣屬泗州復漢舊縣名也虹漢書音貢今音絳杜佑曰睢陵縣故城在泗州下邳縣東南】安都遣其從子直閣將軍索兒太原太守清河傅靈越等攻之【文帝元嘉十年割濟南太山立太原郡唐齊州長清縣宋太原郡地也從才用翻守手又翻】闡令孫之弟也安都壻裴祖隆守下邳劉彌之至下邳更以所領應建康襲擊祖隆祖隆兵敗與征北參軍垣崇祖奔彭城崇祖護之之從子也【垣護之為將著功名於元嘉孝建之間】彌之族人北海太守懷恭從子善明皆舉兵以應彌之薛索兒聞之釋睢陵引兵擊彌之彌之戰敗走保北海申令孫進據淮陽【晉安帝義熙中土斷立淮陽郡於下邳角城唐泗州治宿豫縣古角城也】請降於索兒【降戶江翻】龎孟蚪亦不受命舉兵應尋陽帝召尋陽王長史行會稽郡事孔覬為太子詹事以平西司馬庾業代之又遣都水使者孔璪入東慰勞【漢官有都水長屬少府晉屬大司農後遂置都水使者掌河津漕渠凡水利事并督治船艦會工外翻覬音冀使疏吏翻璪子皓翻勞力到翻】璪說覬以建康虚弱不如擁五郡以應袁鄧【東揚州所統五郡說輸芮翻】覬遂兵馳檄奉尋陽吳郡太守顧琛吳興太守王曇生【琛五林翻曇徒含翻】義興太守劉延熙晉陵太守袁標皆據郡應之上又以庾業代延熙爲義興業至長塘湖即與延熙合益州刺史蕭惠開聞晉安王子勛舉兵集將佐謂之曰湘東太祖之昭晉安世祖之穆其於當璧並無不可【左傳楚共王無冢適有寵子五人無適立焉乃大有事於羣望而祈曰請神擇於五人使主社稷乃徧以璧見於羣望曰當璧而拜者神所立也既乃密埋璧於太室之庭使五人齋而入拜長康王跨之靈王肘加焉子干子晢皆遠之平王弱抱而入再拜皆厭紐其後卒有楚國昭市招翻】但景和雖昏本是世祖之嗣【廢帝改元景和】不任社稷其次猶多吾荷世祖之眷當推奉九江【任音壬荷下可翻自宋以來率謂江州為九江晁氏志曰太湖一湖而曰五湖昭餘祁一澤而曰九澤九江一水而曰九江余按書禹貢曰荆及衡陽惟荆州江漢朝宗于海九江孔殷孔安國注曰江於此州界分為九道甚得地勢之中漢書地理志廬江郡尋陽縣禹貢九江在南皆東合於大江應劭曰江自廬江尋陽分為九尋陽地記曰九江一曰烏江二曰蜯江三曰烏白江四曰嘉靡江五曰畎江六曰源江七曰廩江八曰提江九曰箘江張須元九江圖云一曰三里江二曰五州江三曰嘉靡江四曰烏土江五曰白蚌江六曰白烏江七曰箘江八曰沙提江九曰廪江參差隨水長短或百里或五十里始於鄂陵終於江口會於桑落洲太康地記曰九江劉歆以為湖漢九水入彭蠡澤也夏撰曰據此數說皆謂江水至是分為九道獨曾氏謂為不然曾氏謂下文導江過九江至于東陵東迤北會為匯說者謂東陵巴陵也盖今巴陵與夷陵相為東西夷陵一曰西陵則巴陵為東陵可知許慎曰迤邪行也今江水過洞庭至巴陵而後東北邪行合於彭蠡即經所謂過九江至于東陵東迤北會為匯也由是觀之九江不在潯陽明矣所謂九江者盖今洞庭也考之前志沅水漸水潕水辰水叙水酉水醴水湘水資水皆合洞庭中東入于江所謂九江者豈非此乎宋白曰江州潯陽郡禹貢九江孔殷彭蠡既瀦彭蠡在州南東五十三里九江在州西北二十五里是也然則彭蠡以東為揚州之域九江以西即荆州之域周景式廬山記云柴桑彭蠡之郊古三苗國舊属廬江地又案尋陽記云春秋時為吳之西境楚之東境本在大江之北今蘄州界古蘭城是也秦并天下以此属廬江郡漢属淮南國後漢為豫章廬江二郡之境三國之時此地雖為督護要津而未立郡吳但分尋陽隸武昌晉初尋陽猶理江北温嶠移於此始置尋陽郡隋為九江郡余按秦并天下置九江郡項羽封黥布為九江王都六漢地理志所謂九江在潯陽縣南沈約宋志尋陽本縣名因水名縣水南注江二漢属廬江吳立蘄春郡尋陽縣属焉此時尋陽之地在江北晉亂立尋陽郡後郡治於柴桑而尋陽之名遂移於江南晉惠帝置江州治豫章成帝移江州治尋陽時人盖因漢志所謂九江在尋陽縣南而尋陽又為江州治所遂謂尋陽為九江若禹貢之九江其地實難考見若必以夷陵為西陵遂以巴陵為禹貢之東陵摭取會洞庭之水為九江考之前志會洞庭者不止九水而酈道元水經注謂廬江郡有東陵鄉江夏有西陵縣故是言東尚書云江水過九江至于東陵者也西南流水積為湖湖西有青林山又考水經注自沔口以下有湖口水加湖江水武口水烏石水舉水巴水希水蘄水利水皆南流注于江而後至青林水口亦可傳合九水之說但未敢以為是九河之迹至漢已不可悉考而欲彊為九江之說難矣】乃遣巴郡太守費欣壽將五千人東下於是湘州行事何慧文廣州刺史袁曇遠梁州刺史柳元怙山陽太守程天祚皆附於子勛元怙元景之從兄也【費扶沸翻將即亮翻曇徒含翻從才用翻】是歲四方貢計皆歸尋陽【貢謂貢方物計謂上計帳】朝廷所保唯丹陽淮南等數郡其間諸縣或應子勛東兵已至永世【吳分溧陽為永平縣晉武帝太康元年更名永世縣屬丹陽郡其地盖在今安吉州建康府廣德軍三郡界下云永世令叛義興兵垂至延陵則其地又犬牙入今常州界東兵欲自此進取曲阿】宫省危懼上集羣臣以謀成敗蔡興宗曰今普天同叛宜鎮之以靜至信待人叛者親戚布在宫省若繩之以灋則土崩立至宜明罪不相及之義【父子兄弟罪不相及古義也】物情既定人有戰心六軍精勇器甲犀利以待不習之兵其勢相萬耳願陛下勿憂上善之【蔡興宗豈特以方嚴自將盖識時審勢者也】 建武司馬劉順說豫州刺史殷琰使應尋陽【說輸芮翻】琰以家在建康未許右衛將軍柳光世自省内出奔彭城過壽陽言建康必不能守琰信之且素無部曲為土豪前右軍參軍杜叔寶等所制不得已而從之琰以叔寶為長史内外軍事皆叔寶專之上謂蔡興宗曰諸處未平殷琰已復同逆【復扶又翻】頃日人情云何事當濟不【不讀曰否】興宗曰逆之與順臣無以辨今商旅斷絶米甚豐賤四方雲合而人情更安【湘東簒位非其本心尋陽起兵名正言順故曰逆之與順臣無以辨商旅斷絶米甚豐賤者前朝之積也四方雲合人情更安者積苦於狂暴而驟樂寛政也天下嗸嗸新主之資斯言豈不信哉】以此卜之清蕩可必但臣之所憂更在事後猶羊公言既平之後方當勞聖慮耳【羊祐之言見八十卷晉武帝咸寧四年】上曰誠如卿言上知琰附尋陽非本意乃厚撫其家以招之 汝南新蔡二郡太守周矜起兵於懸瓠以應建康【汝南郡時治懸瓠宋以新蔡郡帖治汝南故周矜領二郡太守自是二郡太守多矣】袁覬誘矜司馬汝南常珍奇執矜斬之【誘音酉】以珍奇代為太守 上使宂從僕射垣榮祖還徐州說薛安都【諸垣自界陽歸南世在青徐立効為士人所信重故使還說薛安都宂而隴翻從才用翻說輸芮翻下說孝同】安都曰今京都無百里地【京都謂建康四方皆奉尋陽故言無百里地】不論攻圍取勝自可拍手笑殺且我不欲負孝武榮祖曰孝武之行足致餘殃【不善之積必有餘殃考武貪淫濟以奢虐人倫道盡故榮祖云然行下孟翻】今雖天下雷同【雷之聲物無不隋時而應者故以響應為雷同】正是速死無能為也安都不從因留榮祖使為將【將即亮翻】榮祖崇祖之從父兄也【從才用翻】 兖州刺史殷孝祖之甥司法參軍葛僧韶請徵孝祖入朝【據南史司法參軍當作司徒參軍請下當有徵字朝直遥翻】上遣之時薛索兒屯據津逕僧韶間行得至【間古莧翻】說孝祖曰景和凶狂開闢未有朝野危極假命漏刻【朝直遥翻下同】主上夷兇翦暴更造天地國亂朝危宜立長君【更工衡翻長知兩翻】而羣迷相煽搆造無端貪利幼弱競懷希望使天道助逆羣凶事申則主幼時艱權柄不一兵難互起豈有自容之地舅少有立功之志【難乃旦翻少詩照翻】若能控濟義勇還奉朝廷【控引也濟謂濟河音子禮翻南史作控濟河義勇文意尤為明暢】非唯匡主静亂乃可以垂名竹帛孝祖具問朝廷消息僧韶隨方訓譬并陳兵甲精彊主上欲委以前驅之任孝祖即日委妻子於瑕丘【瑕丘縣故魯瑕邑漢属山陽郡魏晋省宋為兖州治所】帥文武二千人隨僧韶還建康【帥讀曰率】時四方皆附尋陽朝廷唯保丹陽一郡而永世令孔景宣復叛【復扶又翻】義興兵垂至延陵【晉武帝太康二年分曲阿之延陵鄉立延陵縣屬晉陵郡】内外憂危咸欲奔散孝祖忽至衆力不少並傖楚壯士【江南謂中原人為傖荆州人為楚少詩照翻傖助庚翻】人情大安甲辰進孝祖號撫軍將軍假節都督前鋒諸軍事遣向虎檻寵賚甚厚【賚來代翻】初上遣東平畢衆敬詣兖州募人至彭城薛安都以利害說之【說輸芮翻】矯上命以衆敬行兖州事衆敬從之殷孝祖使司馬劉文石守瑕丘衆敬引兵擊殺之安都素與孝祖有隙使衆敬盡殺孝祖諸子州境皆附之【為畢衆敬以兖州降魏張本】唯東平太守申纂據無鹽不從【為申纂以城拒魏而死張本無鹽縣自漢晉以來属東平隋廢省其地當在唐鄆州界水經注濟水逕壽張縣須朐城西濟水西有安民亭亭北對安民山東臨濟水水東即無鹽縣界也杜佑曰鄆州治須昌縣也無鹽故城在今縣東東平國故城亦在縣東】纂鍾之曾孫也【申鍾見九十五卷晉成帝咸和九年】 丙午上親總兵出頓中堂辛亥以山陽王休祐為豫州刺史督輔國將軍彭城劉勔【勔彌兖翻】寧朔將軍廣陵呂安國等諸軍西討殷琰 【考異曰宋畧二月庚申以休祐都督西討今從宋書】巴陵王休若督建威將軍吳興沈懷明尚書張永輔國將軍蕭道成等諸軍東討孔覬時將士多東方人父兄子弟皆已附覬【顗音冀】上因送軍普加宣示曰朕方務德簡刑使父子兄弟罪不相及助順向逆者一以所從為斷【斷丁亂翻】卿等當深逹此懷勿以親戚為慮也衆於是大悦凡叛者親黨在建康者皆使居職如故壬子路太后殂 【考異曰宋畧南史皆曰義嘉難作太后心幸之延上飲酒置毒以進侍】

  【者引上衣上寤起以其巵上壽是日太后崩喪事如禮宋書無之今不取】 孔覬遣其將孫曇瓘等軍於晉陵九里【其地在晉陵西北九里因以為名將即亮翻曇徒含翻】部陳甚盛【陳讀曰陣下戰陳同】沈懷明至奔牛所領寡弱乃築壘自固張永至曲阿未知懷明安否百姓驚擾永退還延陵就巴陵王休若諸將帥咸勸休若退保破岡【帥所類翻】其日大寒風雪甚猛塘埭决壞【壞徒耐翻以土遏水曰埭】衆無固心休若宣令敢有言退者斬衆小定乃築壘息甲尋得懷明書賊定未進軍主劉亮又至兵力轉盛人情乃安亮懷慎之從孫也【從才用翻】殿中御史吳喜以主書事世祖稍遷河東太守【晉成帝咸康三年庾亮鎮荆州以司州僑戶立河東郡隋唐之松滋縣即其地也】至是請得精兵三百致死於東上假喜建武將軍簡羽林勇士配之議者以喜刀筆主者未嘗為將不可遣【將即亮翻下同】中書舍人巢尚之曰喜昔隨沈慶之屢經軍旅性既勇决又習戰陳若能任之必有成績諸人紛紜皆是不别才耳【别彼列翻】乃遣之喜先時數奉使東吳【先悉薦翻數所角翻使疏吏翻】性寛厚所至人並懷之百姓聞吳河東來皆望風降散【降戶江翻】故喜所至克捷永世人徐崇之攻孔景宣斬之喜板崇之領縣事喜至國山【國山在陽羨縣界晉立義興郡分陽羨置國山縣屬焉隋廢國山入義興縣】遇東軍進擊大破之自國山進屯吳城【吳城當在義興西南九域志所謂泰伯城是也】劉延熙遣其將楊玄等拒戰喜兵力甚弱玄等衆盛喜奮擊斬之進逼義興延熙栅斷長橋保郡自守【義興今常州之宜興也我朝自太平興國元年避太宗御名改為宜興此長橋盖在荆溪之上今宜興縣南二十步有荆溪上承百瀆兼受數郡之水劉延熙盖柵斷荆溪之橋以自保輿地志曰今常州宜興縣南三十步有長橋即周處斬蛟之所】喜築壘與之相持庾業於長塘湖口夾岸築城有衆七千人與延熙遥相應接【庾業叛建康與延熙合見上】沈懷明張永與晉陵軍相持久不决外監朱幼舉司徒參軍督護任農夫驍勇有膽力【任音壬驍堅堯翻】上以四百人配之使助東討農夫自延陵出長塘庾業築城猶未合農夫馳往攻之力戰大破之庾業弃城走義興【走音奏】農夫收其船仗進向義興助吳喜二月己未朔喜渡水攻郡城【渡荆溪之水也】分兵擊諸壘登高指麾若令四面俱進者義興人大懼諸壘皆潰延熙赴水死遂克義興魏丞相太原王乙渾專制朝權【朝直遥翻下同】多所誅殺安

  遠將軍賈秀掌吏曹事渾屢言於秀為其妻求稱公主秀曰公主豈庶姓所宜稱【魏制掌吏曹事即掌選曹事吏部尚書之職也凡非國之同姓皆謂之庶姓為于偽翻】秀寧取死今日不可取笑後世渾怒罵曰老奴官慳會侍中拓跋丕告渾謀反庚申馮太后收渾誅之秀彛之子【賈彞見一百八卷晉孝武太元二十年】丕烈帝之玄孫也【拓跋翳槐追謚烈皇帝】太后臨朝稱制引中書令高久中書侍郎高閭及賈秀共參大政 沈懷明張永蕭道成等軍於九里西與東軍相持東軍聞義興敗皆震恐上遣積射將軍濟陽江方興御史王道隆至晉陵視東軍形勢【濟子禮翻】孔顗將孫曇瓘程扞宗列五城互相連帶扞宗城猶未固王道隆與諸將謀曰扞宗城猶未立可以藉手上副聖旨下成衆氣辛酉道隆帥所領急攻拔之【帥讀曰率】斬扞宗首永等因乘勝進擊曇瓘等壬戌曇瓘等兵敗與袁標俱棄城走遂克晉陵吳喜軍至義鄉【晉惠帝永興元年分吳興之長城立義鄉縣属義興郡今湖州古吳興也長興縣古長城也在州西北七十里】孔璪屯吳興南亭太守王曇生詣璪計事聞臺軍已近璪大懼墮牀曰懸賞所購唯我而已今不遽走將為人擒遂與曇生奔錢唐【孔璪將命于東乃勸孔覬舉兵故懼而走璪子皓翻】喜入吳興任農夫引兵向吳郡顧琛棄郡奔會稽【琛丑林翻會工外翻】上以四郡既平【四郡晉陵義興吳興吳郡也】乃留吳喜使統沈懷明等諸將東擊會稽召張永等北擊彭城江方興等南擊尋陽 以吏部尚書蔡興宗為左僕射侍中禇淵為吏部尚書 丁卯吳喜軍至錢唐孔璪王曇生奔浙東喜遣彊弩將軍任農夫等引兵向黄山浦【黄山浦今漁浦是也漁浦東南即後黄山諸暨志長寧鄉在縣東四十五里管五里一曰黄山里在今越州西北四十五里】東軍據岸結寨農夫等擊破之喜自柳浦渡取西陵【柳浦即今浙江亭東跨浦橋之浦也劉昫唐書曰隋於餘杭縣置杭州又自餘杭徙治錢唐又移州於柳浦今州城是】擊斬庾業會稽人大懼將士多奔亡孔覬不能制【將即亮翻覬音冀】戊寅上虞令王晏起兵攻郡覬逃奔嵴山【嵴資昔翻據南史覬門生載覬以小船竄于嵴山村】車騎從事中郎張綏封府庫以待吳喜【騎奇計翻】己卯王晏入城殺綏執尋陽王子房於别署【張綏盖遷子房於别署故王晏就執之】縱兵大掠府庫皆空獲孔璪殺之庚辰嵴山民縳孔覬送晏晏謂之曰此事孔璪所為無預卿事可作首辭【首式又翻首辭所以首罪】當相為申上【為于偽翻上時掌翻】覬曰江東處分莫不由身【處昌呂翻分扶問翻】委罪求活便是君輩行意耳晏乃斬之【史言孔覬臨死不改節】顧琛王曇生袁摽等詣吳喜歸罪【謂歸而首罪也】喜皆宥之東軍主凡七十六人【一軍之帥謂之軍主】臨陳斬十七人其餘皆原宥【為吳喜得罪張本陳讀曰陣】 薛索兒攻申闡久不下使申令孫入睢陵說闡闡出降索兒并令孫殺之【古人有言禍莫大於殺已降為申令孫之子殺薛索兒張本說輸芮翻降戶江翻下同】 山陽王休祐在歷陽輔國將軍劉勔進軍小峴【勔彌兖翻峴戶典翻】殷琰所署南汝隂太守裴季之以合肥來降【沈約曰江左置南汝隂郡所治即合肥縣降戶江翻】 鄧琬性鄙闇貪吝既執大權父子賣官鬻爵使婢僕出市道販賣酣歌博弈日夜不休【酣戶甘翻】大自矜遇賓客到門者歷旬不得前内事悉委禇靈嗣等三人羣小横恣競為威福於是士民忿怨内外離心【史言尋陽敗亡之由横戶孟翻】琬遣孫冲之帥龍驤將軍薛常寶陳紹宗焦度等兵一萬為前鋒據赭坼【劉昫曰池州南陵縣漢春穀縣地梁置南陵縣治赭坼城唐長安四年移治青陽城帥讀曰率赭音者坼音畿】冲之於道與晉安王子勛書曰舟檝已辦糧仗亦整【檝與楫同】三軍踴躍人爭効命便欲沿流挂帆直取白下【白下在江寧縣界臨江津】願速遣陶亮衆軍兼行相接分據新亭南洲則一麾定矣子勛加冲之左衛將軍以陶亮為右衛將軍統郢荆湘梁雍五州兵【雍於用翻】合二萬人一時俱下陶亮本無幹畧聞建安王休仁自上【上時掌翻】殷孝祖又至不敢進屯軍鵲洲【鵲洲在宣城郡南陵縣左傳之鵲岸也杜預曰鵲岸謂廬江舒縣鵲尾渚審是則鵲頭在宣城界鵲尾在廬江界鵲洲則江中之洲也】殷孝祖負其誠節【誠節謂委鎮勤王不顧妻子也】陵轢諸將【陵侵也侮也轢車踐也音狼狄翻】臺軍有父子兄弟在南者孝祖悉欲推治【南南軍也南謂尋陽在南臺軍泝江南上而攻之治直之翻】由是人情乖離莫樂為用【樂音洛】寧朔將軍沈攸之内撫將士外諧羣帥衆並賴之孝祖每戰常以鼔蓋自隨軍中人相謂殷統軍可謂死將矣【撫將即亮翻下同帥所類翻】今與賊交鋒而以羽儀自摽顯若善射者十人共射之【共射而亦翻】欲不斃得乎三月庚寅衆軍水陸並進攻赭圻陶亮等引兵救之孝祖於陳為流矢所中死【陳讀曰陣中竹仲翻】軍主范濳帥五百人降於亮【帥讀曰率下同降戶江翻】人情震駭並謂沈攸之宜代孝祖為統時建安王休仁屯虎檻遣寧朔將軍江方興龍驤將軍襄陽劉靈遺各將三千人赴赭圻【驤思將翻各將即亮翻】攸之以為孝祖既死亮等有乘勝之心明日若不更攻則示之以弱方興名位相亞必不為已下【攸之方興皆寧朔將軍故言名位相亞亞次也】軍政不壹致敗之由也乃帥諸軍主詣方興曰今四方並反國家所保無復百里之地唯有殷孝祖為朝廷所委賴鋒鏑裁交輿尸而反文武喪氣【喪息浪翻】朝野危心事之濟否唯在明旦一戰戰若不捷大事去矣詰朝之事【詰起吉翻杜預曰詰朝明旦】諸人或謂吾應統之自卜懦薄幹畧不如卿今輒相推為統但當相與勠力耳方興甚悦許諾攸之既出諸軍主並尤之攸之曰吾本濟國活家豈計此之升降且我能下彼彼必不能下我豈可自措同異也【沈攸之成尋陽之功懾也郢城之敗驕也下戶嫁翻】孫冲之謂陶亮曰孝祖梟將一戰便死天下事定矣不須復戰便當直取京都亮不從【孫冲之狃殷孝祖之死便欲順流長驅輕敵如此使陶亮從其計必與沈攸之等遇亦將以輕敵取敗矣梟堅堯翻將即亮翻復扶又翻】辛卯方興帥諸將進戰【帥讀曰率下同】建安王休仁又遣軍主郭季之步兵校尉杜幼文屯騎校尉垣恭祖龍驤將軍濟地頓生京兆段佛榮【濟地頓生四字必有誤】等三萬人往會戰自寅及午大破之追北至姥山而還【今太平州當塗縣西北四十五里有慈姥山又巢湖中有姥山】幼文驥之子也【元嘉中杜驥任當方面】孫冲之於湖白口【巢湖口及白水口也】築二城軍主竟陵張興世攻拔之壬辰詔以沈攸之為輔國將軍假節代殷孝祖督前鋒諸軍事陶亮聞湖白二城不守大懼急召孫冲之還鵲尾留薛常寶等守赭圻先於姥山及諸岡分立營寨亦各散還共保濃湖【濃湖在鵲尾下先悉薦翻】時軍旅大起國用不足募民上錢穀者賜以荒縣荒郡或五品至三品散官有差【荒郡荒縣極邉郡縣被兵荒殘者也賜之者以郡守縣令及參佐等職名賜之上時掌翻散悉亶翻】軍中食少【少詩沼翻】建安王休仁撫循將士均其豐儉弔死問傷身自隱卹【隱度也痛也恤憂也愍也】故十萬之衆莫有離心鄧琬遣其豫州刺史劉胡帥衆三萬鐵騎二千東屯鵲尾并舊兵凡十餘萬【舊兵謂尋陽先所遣陶亮孫冲之等之兵騎奇計翻下同】胡宿將勇健多權畧屢有戰功將士畏之【將即亮翻下同】司徒中兵參軍冠軍蔡那【蔡那南陽冠軍人冠古玩翻】子弟在襄陽胡每戰懸之城外【劉胡自襄陽東下拘蔡那子弟以隨軍為蔡道淵執子勛張本】那進戰不顧吳喜既定三吳帥所領五千人并運資實至于赭圻【史言建康兵勢益盛】 薛索兒將馬步萬餘人自睢陵渡淮進逼青冀二州刺史張永營丙申詔南徐州刺史桂陽王休範統北討諸軍事進據廣陵又詔蕭道成將兵救永 戊戌尋陽王子房至建康上宥之貶爵為松滋侯 庚子魏以隴西王源賀為太尉 上遣寧朔將軍劉懷珍帥龍驤將軍王敬則等步騎五千助劉勔討壽陽斬廬江太守劉道蔚懷珍善明之從子也【劉善明彌之之從子蔚紆勿翻從才用翻】 中書舍人戴明寶啓上遣軍主竟陵黄回募兵擊斬尋陽所署馬頭太守王廣元【杜佑曰馬頭城在壽州盛康縣北】 前奉朝請壽陽鄭黑起兵於淮上以應建康東扞殷琰西拒常珍奇乙巳以黑為司州刺史【以鄭黑之東扞西拒觀之則起兵淮上盖在東西正陽之間朝直遥翻 考異曰宋殷琰傳作鄭墨今從宋本紀宋畧】 殷琰將劉順柳倫皇甫道烈龎天生等馬步八千人東據宛唐【宛唐按水經注作死雩云肥水過九江成德縣西北入芍陂又北右合閻潤水水積為陽湖陽湖水自塘西北逕死雩亭宋泰始初劉順據之以拒劉勔杜佑通典作死虎曰死虎地名在壽州壽春縣東四十餘里龎皮江翻】劉勔率衆軍並進去順數里立營【勔彌兖翻帥讀曰率下同】時琰所遣諸軍並受順節度而以皇甫道烈土豪柳倫臺之所遣順本卑微唯不使統督二軍【土豪既不可令臺之所遣者又不可令則置帥果何為也其敗宜矣】勔始至塹壘未立【塹七艷翻】順欲擊之道烈倫不同順不能獨進乃止勔營既立不可復攻因相持守【復扶又翻】 壬子斷新錢【并元嘉四銖孝建四銖皆斷不用也斷讀如短】專用古錢 沈攸之帥諸軍圍赭圻薛常寶等糧盡告劉胡求救胡以囊盛米繫流查及船腹【盛時征翻查助加翻水中浮木也船腹船中心也】陽覆船順風流下以餉之沈攸之疑其有異遣人取船及流查大得囊米丙辰劉胡帥步卒一萬夜斫山開道以布囊運米餉赭圻平旦至城下猶隔小塹未能入沈攸之帥諸軍邀之殊死戰胡衆大敗捨糧棄甲緣山走斬獲甚衆胡被創僅得還營【被皮義翻】常寶等惶懼夏四月辛酉開城突圍走還胡軍攸之拔赭圻城斬其寧朔將軍沈懷寶等納降數千人陳紹宗單舸奔鵲尾【降戶江翻舸古我翻】建安王休仁自虎檻進屯赭圻劉胡等兵猶盛上欲綏慰人情遣吏部尚書禇淵至虎檻選用將士時以軍功除官者衆板不能供【程大昌曰魏晉至梁陳授官有板長一尺二寸厚一寸濶七寸授官之辭在于板上為鵠頭書】始用黄紙鄧琬以晉安王子勛之命徵袁顗下尋陽顗悉雍州之衆馳下琬以黄門侍郎劉道憲行荆州事侍中孔道存行雍州事【雍於用翻】上庸太守柳世隆乘虚襲襄陽不克世隆元景之弟子也 散騎侍郎明僧暠起兵攻沈文秀以應建康【明氏自云吳太伯之裔百里奚之子孟明視以明為姓散悉亶翻騎奇計翻暠工老翻】壬午以僧暠為青州刺史平原樂安二郡太守王玄默據琅邪【武帝平齊置平原郡於梁鄒樂安郡於千乘玄默據琅邪起兵非就郡起兵也劉昫曰平原隋改曰龔丘屬兖州】清河廣川二郡太守王玄邈據盤陽城【武帝置清河郡於盤陽廣川郡於武彊五代志齊郡長山縣舊曰武彊置廣川後併東清河平原二郡入焉改曰東平原郡隋廢郡改武彊曰長山則是平原清河廣川三郡皆置於隋長山縣界盤陽漢般陽縣也屬濟南郡應劭曰在般水之陽按水經注般陽縣西南即梁鄒縣劉昫曰唐淄川淄川縣漢般陽縣地也】高陽勃海二郡太守劉乘民據臨濟城【文帝置高陽郡於樂安地孝武置勃海郡於臨淄地臨濟縣屬樂安郡按水經注臨濟縣在梁鄒東北臨濟子禮翻】並起兵以應建康玄邈玄謨之從弟乘民彌之之從子也【從才用翻】沈文秀遣軍主解彦士攻北海拔之殺劉彌之乘民從弟伯宗合帥鄉黨復取北海【解戶買翻帥讀曰率復扶又翻下而復更復可復假復國復復嬰同】因引兵向青州所治東陽城【杜佑曰東陽城青州所治益都縣東城是也治直之翻】文秀拒之伯宗戰死僧暠玄默玄邈乘民合兵攻東陽城每戰輒為文秀所破離而復合如此者十餘卒不能克【言不能克東陽城卒子恤翻】 杜叔寶謂臺軍住歷陽不能遽進及劉勔等至上下震恐劉順等始行唯齎一月糧既與勔相持糧盡叔寶發軍千五百乘載米餉順自將五千精兵送之【乘繩證翻將即亮翻】呂安國聞之言於劉勔曰劉順精甲八千我衆不能居半相持既久彊弱勢殊更復推遷則無以自立所賴者彼糧行竭我食有餘耳若使叔寶米至非唯難可復圖我亦不能持久今唯有間道襲其米車【間古莧翻】出彼不意若能制之當不戰走矣勔以為然以疲弱守營簡精兵千人配安國及龍驤將軍黄囘使從間道出順後於横塘抄之【水經注閻潤水上承施水於合肥縣北復逕縣西積為陽湖陽湖水自塘西北逕死雩亭南夾横塘西注宋泰始初劉順據之以拒劉勔杜叔寶送糧死雩劉勔破之此塘驤思將翻抄楚交翻】安國始行齎二日熟食食盡叔寶不至將士欲還安國曰卿等旦已一食今晚米車不容不至若其不至夜去不晩叔寶果至以米車為函箱陳【陳讀曰陣】叔寶於外為遊軍幢主楊仲懷將五百人居前【幢傳江翻】安國囘等擊斬之及其士卒皆盡叔寶至囘欲乘勝擊之安國曰彼將自走不假復擊退三十里止宿夜遣騎參侯【騎奇寄翻】叔寶果棄米車走安國復夜往燒米車驅牛二千餘頭而還【還從宣翻又如字】五月丁亥朔夜劉順衆潰走淮西就常珍奇【常珍奇據懸瓠在淮水之西走音】於是劉勔鼔行進向壽陽叔寶斂居民及散卒嬰城自守勔與諸軍分營城外山陽王休祐與殷琰書為陳利害【為于偽翻】上又遣御史王道隆齎詔宥琰罪勔與琰書并以琰兄瑗子邈書與之琰與叔寶等皆有降意【降戶江翻下同】而衆心不壹復嬰城固守弋陽西山蠻田益之起兵應建康詔以益之為輔國將軍督弋陽蠻事壬辰以輔國將軍沈攸之為雍州刺史【雍於用翻】丁未以尚書左僕射王景文為中軍將軍庚戌以寧朔將軍劉乘民為冀州刺史【沈攸之行南征軍前欲以代袁顗劉乘民在臨濟就以冀州授之】 甲寅葬昭太后於修寧陵【路太后諡曰昭修寧陵在孝武陵東南】 張永蕭道成等與薛索兒戰大破之索兒退保石梁【今揚州六合縣有石梁河江左置石梁郡隋唐之間置石梁縣】食盡而潰走向樂平【樂平縣前漢曰清屬東郡後漢章帝更名樂平江左以樂平縣民流寓者僑立樂平縣於鍾離郡界】為申令孫子孝叔所斬【申孝叔報父之仇也】薛安都子道智走向合肥詣裴季之降傳靈越走至淮西武衛將軍沛郡王廣之生獲之送詣劉勔勔詰其叛逆【詰去吉翻】靈越曰九州倡義豈獨在我薛公不能專任智勇委付子姪此其所以敗也人生歸於一死實無面求活勔送詣建康上欲赦之靈越辭終不改乃殺之 鄧琬以劉胡與沈攸之等相持久不决乃加袁顗督征討諸軍事六月甲戌顗帥樓船千艘戰士二萬來入鵲尾顗本無將畧性又怯橈【顗魚豈翻帥讀曰率艘蘇遭翻將即亮翻橈奴教翻】在軍中未嘗戎服語不及戰陳唯賦詩談義而已不復撫接諸將劉胡每論事酬對甚簡【陳讀曰陣復扶又翻酬荅也】由此大失人情胡常切齒恚恨【恚於避翻】胡以南運米未至軍士匱乏就顗借襄陽之資顗不許曰都下兩宅未成方應經理【兩敵相向勝負之决存亡係焉袁顗乃欲留襄陽之資以經理私宅子勛既敗都下兩宅豈顗有哉】又信往來之言云建康米貴斗至數百以為將不攻自潰擁甲以待之【擁甲猶言擁兵也】 田益之帥蠻衆萬餘人圍義陽【宋白曰義陽本漢平氏縣義陽鄉之地魏黄初中分平氏居義陽郡及義陽縣】鄧琬使司州刺史龎孟蚪帥精兵五千救之益之不戰潰去 安成太守劉襲始安内史王識之 【考異曰宋書作王職之今從宋畧】建安内史趙道生並舉郡來降【降戶江翻】襲道憐之孫也【道憐武帝之弟】蕭道成世子賾為南康贑令【賾士革翻蕭道成為齊公賾始為世子此世字衍贑縣漢屬豫章郡吳屬廬陵郡晋分屬南康郡章貢二水合而為贑音古暗翻】鄧琬遣使收繫之【使疏吏翻】門客蘭陵桓康擔賾妻裴氏及其子長懋子良逃於山中與賾族人蕭欣祖等結客得百餘人攻郡破獄出賾南康相沈肅之帥將吏追賾賾與戰擒之賾自號寧朔將軍據郡起兵【據南康郡也帥讀曰率將即亮翻 考異曰宋鄧琬傳云世子與南康相沈用之等據郡起義宋畧亦云沈肅之以郡招起義按賾始自獄中劫出琬所署南康相不容便與之同今從蕭子顯南齊書記】與劉襲等相應琬以中護軍殷孚為豫章太守督上流五郡【豫章廬陵臨川安成南康五郡皆在南江上流】以防襲等 衡陽内史王應之起兵應建康襲擊湘州行事何慧文於長沙應之與慧文捨軍身戰斫慧文八創【創初良翻】慧文斫應之斷足殺之 始興人劉嗣祖等據郡起兵應建康廣州刺史袁曇遠遣其將李萬周等討之嗣祖誑萬周云尋陽已平萬周還襲番禺擒曇遠斬之【曇徒含翻誑居况翻番禺音潘愚】上以萬周行廣州事 初武都王楊元和治白水【據北史此武都之白水也按五代志武昌建威縣舊立白水郡建威唐省入階州將利縣】微弱不能自立棄國奔魏元和從弟僧嗣復自立屯葭蘆【從才用翻復扶又翻下開復同】費欣壽至巴東【費扶沸翻】巴東人任叔兒據白帝自號輔國將軍擊欣壽斬之【蕭惠開遣欣壽東下見上正月】叔兒遂阻守三峽【江水自巴東至夷陵其間有廣溪峽巫峽西陵峽謂之三峽一曰三峽西峽歸峽巫峽七百里中兩岸連山畧無缺處隱天蔽日非日中夜分不見日月】蕭惠開復遣治中程法度將兵三千出梁州楊僧嗣帥羣氐斷其道間使以聞【帥讀曰率斷丁管翻間古莧翻使疏吏翻】秋七月丁酉以僧嗣為北秦州刺史武都王 諸軍與袁顗相拒於濃湖久未决龍驤將軍張興世建議曰賊據上流兵彊地勝我雖持之有餘而制之不足若以奇兵數千潛出其上因險而壁見利而動使其首尾周遑進退疑阻中流既梗糧運自艱此制賊之奇也錢溪江岸最狹【新唐書地理志宣州南陵縣有梅根監錢官下云陳慶至錢溪軍於梅根盖今之梅根港是也以有鑄錢監故謂之錢溪】去大軍不遠下臨洄洑【旋流曰洄伏流曰洑】船下必來泊岸又有橫浦可以藏船千人守險萬夫不能過衝要之地莫出於此沈攸之吳喜並贊其策會龎孟蚪引兵來助殷琰【龎孟蚪自義陽來援壽陽】劉勔遣使求援甚急【勔彌兖翻使疏吏翻】建安王休仁欲遣興世救之沈攸之曰孟蚪蟻聚必無能為遣别將馬步數千足以相制【將即亮翻下同】興世之行是安危大機必不可輟乃遣段佛榮將兵救勔而選戰士七千輕舸二百配興世興世帥其衆沂流稍上尋復退歸【舸古我翻帥讀曰率上時掌翻復扶又翻】如是者累日劉胡聞之笑曰我尚不敢越彼下取揚州【揚州謂建康】張興世何物人欲輕據我上不為之備一夕四更【更工衡翻】值便風興世舉帆直前渡湖白過鵲尾胡既覺乃遣其將胡靈秀將兵於東岸翼之而進戊戌夕興世宿景洪浦靈秀亦留興世遣其將黄道標帥七十舸徑趣錢溪立營寨【趣七喻翻】己亥興世引兵進據之靈秀不能禁庚子劉胡自將水步二十六軍來攻錢溪 【考異曰宋畧曰胡進軍鵲頭遣其將陳慶以三百舸逼錢溪今從宋書】將士欲迎擊之興世禁之曰賊來尚遠氣盛而矢驟驟既易盡【言矢易盡易以鼔翻下同】盛亦易衰不如待之令將士治城如故俄而胡來轉近船入洄洑興世命壽寂之任農夫率壯士數百擊之衆軍相繼並進胡敗走斬首數百胡收兵而下時興世城寨未固建安王休仁慮袁顗并力更攻錢溪欲分其勢辛丑命沈攸之吳喜等以皮艦進攻濃湖【以牛皮冒艦以禦矢石因謂之皮艦艦戶闇翻】斬獲千數是日劉胡帥步卒二萬鐵馬一千欲更攻興世未至錢溪數十里袁顗以濃湖之急遽追之錢溪城由此得立胡遣人傳唱錢溪已平衆並懼沈攸之曰不然若錢溪實敗萬人中應有一人逃亡得還者必是彼戰失利唱空聲以惑衆耳勒軍中不得妄動【勒約勒也】錢溪捷報尋至攸之以錢溪所送胡軍耳鼻示濃湖袁顗駭懼攸之日暮引歸龍驤將軍劉道符攻山陽程天祚請降【程天祚附子勛見上正月】龎孟蚪進至弋陽劉勔遣呂安國等迎擊於蓼潭【漢志六安國有蓼縣晉屬安豐郡水經注决水逕蓼縣故城東灌水會焉所謂蓼潭當在此處】大破之孟蚪走向義陽王玄謨之子曇善起兵據義陽以應建康【曇徒含翻】孟蚪走死蠻中 劉胡遣輔國將軍薛道標襲合肥殺汝隂太守裴季之【裴季之以合肥降劉勔見上二月】劉勔遣輔國將軍垣閎擊之閎閬之弟【垣閬見一百二十九卷孝武帝大明三年】道標安都之子也 淮西人鄭叔舉起兵擊常珍奇以應鄭黑辛亥以叔舉為北豫州刺史 崔道固為土人所攻閉門自守【崔道固以歷城應尋陽見上正月】上遣使宣慰道固請降【降戶江翻下同】甲寅復以道固為徐州刺史【道固本刺冀州復扶又翻下不復足復同】 八月皇甫道烈等聞龎孟蚪敗並開門出降【死虎師潰皇甫道烈盖奔還壽陽】 張興世既據錢溪濃湖軍乏食鄧琬大送資糧畏興世不敢進劉胡率輕舸四百由鵲頭内路欲攻錢溪【鵲洲在江中江水分流故有内路外路舸古我翻】既而謂長史王念叔曰吾少習步戰未閑水鬭【少詩照翻閑亦習也】若步戰恒在數萬人中水戰在一舸之上舸舸各進不復相關【恒戶登翻復扶又翻】正在三十人中此非萬全之計吾不為也乃託瘧疾住鵲頭不進【瘧逆約翻疾而寒熱送作為瘧】遣龍驤將軍陳慶將三百舸向錢溪戒慶不須戰張興世吾之所悉自當走耳陳慶至錢溪軍於梅根胡遣别將王起將百舸攻興世興世擊起大破之胡率其餘舸馳還謂顗曰興世營寨已立不可猝攻昨日小戰未足為損陳慶已與南陵大雷諸軍共遏其上大軍在此鵲頭諸將又斷其下流【斷丁管翻】已墮圍中不足復慮顗怒胡不戰謂曰糧運鯁塞當如此何【鯁塞言若魚骨之鯁塞咽㗋然塞悉則翻】胡曰彼尚得泝流越我而上【上時掌翻】此運何以不得沿流越彼而下邪乃遣安北府司馬沈仲玉將千人步趣南陵迎糧【趣七喻翻下同】仲玉至南陵載米三十萬斛錢布數十舫竪榜為城【舫南妄翻竪臣庾翻立也榜補曩翻木片也】規欲突過行至貴口不敢進【水經注江水自石城東入為貴口今池州貴池縣之池口即貴口也張舜民曰自銅陵舟六十許里至梅根港又五十許里至貴池口】遣間信報胡【間古莧翻】令遣重軍援接張興世遣壽寂之任農夫等將三千人至貴口擊之仲玉走還顗營悉虜其資實胡衆駭懼胡將張喜來降【將即亮翻降戶江翻下同】鎮東中兵參軍劉亮進兵逼胡營胡不能制袁顗懼曰賊入人肝肺裏何由得活胡隂謀遁去己卯誑顗云欲更帥步騎二萬【誑居况翻帥讀曰率騎奇寄翻】上取錢溪兼下大雷餘運令顗悉選馬配之其日胡委顗去徑趣梅根先令薛常寶辦船悉南陵諸軍燒大雷諸城而走至夜顗方知之大怒罵曰今乃為小子所誤呼取常所乘善馬飛鷰謂其衆曰我當自追之因亦走庚辰建安王休仁勒兵入顗營納降卒十萬遣沈攸之等追顗顗走至鵲頭與戍主薛伯珍并所領數千人偕去欲向尋陽夜止山間殺馬以勞將士【勞力到翻將即亮翻】顧謂伯珍曰我非不能死且欲一至尋陽謝罪主上然後自刎耳因慷慨叱左右索節無復應者及旦伯珍請屏人言事【刎扶粉翻索山客翻復扶又翻屏必郢翻】遂斬顗首【南史顗走至青林山見殺】詣錢溪軍主襄陽俞湛之湛之因斬伯珍并送首以為已功劉胡帥二萬人向尋陽詐晉安王子勛云袁顗已降軍皆散唯已帥所領獨返宜速處分為一戰之資【帥讀曰率處昌呂翻分扶問翻】當停據湓城誓死不貳乃於江外夜趣沔口【江中洲嶼節節有之舟行附南岸者謂之内路附北岸者謂之外路】鄧琬聞胡去憂惶無計呼中書舍人禇靈嗣等謀之並不知所出張悦詐稱疾呼琬計事令左右伏甲帳後戒之若聞索酒便出【索山客翻】琬既至悦曰卿首倡此謀今事已急計將安出琬曰正當斬晉安王封府庫以謝罪耳悦曰今日寧可賣殿下求活邪因呼酒子洵提刀出斬琬【洵悦子也】中書舍人潘欣之聞琬死勒兵而至悦使人語之曰【語牛倨翻】鄧琬謀反今已梟戮【梟堅堯翻】欣之乃還取琬子並殺之悦因單舸齎琬首馳下詣建安王休仁降【舸古我翻降戶江翻】尋陽亂【無主故亂】蔡那之子道淵在尋陽被繫作部【作部主作器仗在尋陽城外】脱鎻入城執子勛囚之沈攸之諸軍至尋陽斬晉安王子勛傳首建康時年十一【晉安舉兵實義舉也鄧琬不足道若袁顗孔覬豈可謂不得其死哉世毋以成敗論之】初鄧琬遣臨川内史張淹自鄱陽嶠道入三吳軍於上饒【晉太康地志鄱陽郡有上饒縣而晉書無之當是吳立今為信州有路通鄱陽宋白曰信州上饒縣本秦番縣界漢為番陽縣今州古城遺蹟開皇中所廢古上饒也所謂上饒者以其旁下饒州之故也九域志番陽東南至上饒五百四十里】聞劉胡敗軍副鄱陽太守費曄斬淹以降淹暢之子也【張暢元嘉之季從世祖為徐州長史費父沸翻】廢帝之世衣冠懼禍咸欲遠出至是流離外難【難乃旦翻】百不一存衆乃服蔡興宗之先見【先見事見上卷元年】九月壬辰以山陽王休祐為荆州刺史癸巳解嚴大赦庚子司徒休仁至尋陽遣吳喜張興世向荆州沈懷明向郢州劉亮及寧朔將軍南陽張敬兒向雍州【雍於用翻】孫超之向湘州沈思仁任農夫向豫章平定餘寇劉胡逃至石城【此竟陵之石城今郢州長壽縣是也】捕得斬之郢州行事張沈變形為沙門走追獲殺之【沈持林翻】荆州行事劉道憲聞濃湖平散兵遣使歸罪【使疏吏翻】荆州治中宗景等勒兵入城殺道憲執臨海王子頊以降孔道存知尋陽已平遣使請降尋聞柳世隆劉亮當至道存及三子皆自殺【孔道存時為雍州行事】上以何慧文才兼將吏【有將才又有吏才也將即亮翻】使吳喜宣旨赦之慧文曰既䧟逆節手害忠義【謂殺王應之也】何面見天下之士遂自殺【史言何慧文不肯苟活】安陸王子綏臨海王子頊邵陵王子元並賜死劉順及餘黨在荆州者皆伏誅【劉順自死虎奔淮西又自淮西奔荆州】詔追贈諸死節之臣及封賞有功者各有差 己酉魏初立郡學置博士助敎生員從中書令高允相州刺史李訢之請也【古者家有塾黨有庠術有序國有學秦雖焚書坑儒齊魯學者未嘗廢業漢文翁守蜀起立學官學者比齊魯武帝令天下郡國皆立學校官則學官之立尚矣此書魏初立郡學置官及生員者盖悲五胡兵爭不暇立學魏起北荒數世之後始及此既悲之猶幸斯文之墜地而復振也相息亮翻訢許斤翻】訢崇之子也【此别一李崇非頓丘之李崇也】 上既誅晉安王子勛等待世祖諸子猶如平日司徒休仁還自尋陽言於上曰松滋侯兄弟尚在將來非社稷計宜早為之所冬十月乙卯松滋侯子房永嘉王子仁始安王子真淮南王子孟南平王子產廬陵王子輿子趨子期東平王子嗣子悦並賜死及鎮北諮議參軍路休之司徒從事中郎路茂之【二路皆昭太后子姪】兖州刺史劉祗中書舍人嚴龍皆坐誅世祖二十八子於此盡矣【休仁尚書下省之禍自取之也導上使去其兄子上手滑矣其視諸弟何有哉蕭齊易姓劉氏殱焉骨肉相殘禍至此極有國有家者其鑒于兹】祗義欣之子也 劉勔圍壽陽垣閎攻合肥俱未下勔患之召諸將會議馬隊主王廣之曰得將軍所乘馬判能平合肥【判斷也决也】幢主皇甫肅怒曰廣之敢奪節下馬可斬勔笑曰觀其意必能立功即推鞌下馬與之【幢傳江翻推吐雷翻】廣之往攻合肥三日克之薛道標突圍奔淮西歸常珍奇勔擢廣之為軍主廣之謂肅曰節下若從卿言何以平賊卿不賞才乃至於此肅有學術及勔卒【卒子恤翻】更依廣之廣之薦於齊世祖為東海太守【史竟其事以言王廣之能以恩易怨】 沈靈寶自廬江引兵攻晉熙【晉安帝分廬江立晉熙郡今舒州即其地也晉熙先附尋陽故攻之】晉熙太守閻湛之棄城走 徐州刺史薛安都益州刺史蕭惠開梁州刺史柳元怙 【考異曰宋畧作元哲今從宋書】兖州刺史畢衆敬 【考異曰宋畧作畢榛後魏書云小名㮈今從本傳】豫章太守殷孚汝南太守常珍奇並遣使乞降【尋陽已平故並乞降使疏吏翻降戶江翻】上以南方已平欲示威淮北乙亥命鎮軍將軍張永中領軍沈攸之將甲士五萬迎薛安都【將甲即亮翻 考異曰後魏紀安都與常珍奇降皆在九月而宋本紀宋畧遣張永等北出皆在十月今從之】蔡興宗曰安都歸順此誠非虛正須單使尺書【使疏吏翻】今以重兵迎之勢必疑懼或能招引北虜為患方深若以叛臣罪重不可不誅則曏之所宥亦已多矣况安都外據大鎮密邇邊陲地險兵彊攻圍難克考之國計尤宜馴養【馴松倫翻】如其外叛將為朝廷旰食之憂上不從【為上愧蔡興宗張本旰苦汗翻】謂征北司馬行南徐州事蕭道成曰吾今因此北討卿意以為何如對曰安都狡猾有餘今以兵逼之恐非國之利上曰諸軍猛銳何往不克卿勿多言【蔡興宗蕭道成人地雖殊所見不異盖識時逹變唯智者者能之文武無二道也】都聞大兵北上【地勢西北高東南下濟泗沂之水皆南流逕彭城而注于淮故謂南兵北向為北上上時掌翻】懼遣使乞降於魏常珍奇亦以懸瓠降魏 【考異曰宋畧十二月甲寅珍奇復以郡叛盖於時宋朝始聞之耳】皆請兵自救 戊寅立王子昱為太子 薛安都以其子為質於魏【質音致】魏遣鎮東大將軍代人尉元鎮東將軍魏郡孔伯恭等帥騎一萬出東道救彭城鎮西大將軍西河公石都督荆豫南雍州諸軍事張窮奇出西道救懸弧【尉紆勿翻帥讀曰率騎奇寄翻雍於用翻魏無南雍州下又書安都都督徐雍五州諸軍事盖一時創置尋省併也西河公石魏之宗室】以安都為都督徐雍等五州諸軍事鎮南大將軍徐州刺史河東公常珍奇為平南將軍豫州刺史河内公兖州刺史申纂詐降於魏尉元受之而隂為之備魏師至無鹽纂閉門拒守【為魏攻殺申纂張本】薛安都之召魏兵也畢衆敬不與之同遣使來請降【是年春畢衆敬叛建康從薛安都及安都降魏乃不與之同耳】上以衆敬為兖州刺史衆敬子元賓在建康先坐它罪誅 【考異曰後魏書衆敬傳云元賓有它罪彧獨不捨之宋畧云榛與安都已誅矣今從之】衆敬聞之怒拔刀斫柱曰吾皓首唯一子不能全安用獨生十一月壬子魏師至瑖丘衆敬請降於魏尉元遣部將先據其城【將即亮翻】衆敬悔恨數日不食元長驅而進十二月己未軍于秺【秺縣漢屬濟隂郡後漢省其地當在唐曹州界孟康曰秺音妬】西河公石至上蔡常珍奇帥文武出迎石欲頓軍汝北未即入城【懸弧城在汝水南】中書博士鄭羲曰今珍奇雖來意未可量【量音良】不如直入其城奪其管籥【管鍵也籥關牡也】據有府庫制其腹心策之全者也石遂策馬入城因置酒嬉戲羲曰觀珍奇之色甚不平不可不為之備乃嚴兵設備其夕珍奇使人燒府屋欲為變以石有備而止羲谿之曾孫也淮西七郡民多不願屬魏連營南奔【淮西七郡汝南新蔡汝陽汝隂陳郡南頓潁川】魏遣建安王陸馛宣慰新附民有䧟軍為奴婢者馛悉免之【馛蒲撥翻】新民乃悦 乙丑詔坐依附尋陽削官爵禁錮者皆從原籍隨才銓用劉勔圍壽陽自首春至于末冬内攻外禦戰無不捷

  以寛厚得將士心【將即亮翻】尋陽既平上使中書為詔諭殷琰蔡興宗曰天下既定是琰思過之日陛下宜賜手詔數行以相慰引【慰者安其心引者引使歸順行戶剛翻】今直中書為詔彼必疑謂非真非所以速清方難也【難乃旦翻方難謂一方之難】不從琰得詔謂劉勔詐為之不敢降【降戶江翻下同】杜叔寶閉絶尋陽敗問有傳者即殺之守備益固凡有降者上輒送壽陽城下使與城中人語由是衆情離沮【沮在呂翻】琰欲請降於魏主簿譙郡夏侯詳說琰曰【說輸芮翻下同】今日之舉本効忠節若社稷有奉便當歸身朝廷何可北面左衽乎且今魏軍近在淮次【謂西河公石在汝南之軍也】官軍未測吾之去就若建使歸欵【建使當作遣使欵誠也使疏吏翻】必厚相慰納豈止免罪而已琰乃使詳出見劉勔詳說勔曰今城中士民知困而猶固守者畏將軍之誅皆欲自歸於魏願將軍緩而赦之則莫不相帥而至矣【帥讀曰率下同】勔許諾使詳至城下呼城中人諭以勔意丙寅琰帥將佐面縛出降勔悉加慰撫不戮一人入城約勒將士【將即亮翻】士民貲財秋毫無所失壽陽人大悦魏兵至師水【水經注師水源出大潰山又北逕賢首山西又東逕義陽故城北南北對境圖濟河在今信陽軍羅山縣西北界】將救壽陽聞琰已降乃掠義陽數千人而去久之琰復仕至少府而卒【復扶又翻卒子恤翻】 蕭惠開在益州多任刑誅蜀人猜怨聞費欣壽敗没【費扶沸翻】程法度不得前【事並見上六月】於是晉原一郡反【沈約曰李雄分蜀郡之江原臨卭為漢原郡晉穆帝更名晉原唐為蜀州】諸郡皆應之合兵圍成都城中東兵不滿二千惠開悉遣蜀人出獨與東兵拒守【東兵惠開隨行部曲也】蜀人聞尋陽已平爭欲屠城衆至十餘萬人惠開每遣兵出戰未嘗不捷上遣其弟惠基自陸道使成都赦惠開罪惠基至涪蜀人遏留惠基不聽進惠基帥部曲擊之斬其渠帥【使疏吏翻涪音浮基帥讀曰率渠帥所類翻】然後得前惠開奉旨歸降城圍得解上遣惠開宗人寶首自水道慰勞益州寶首欲以平蜀為已功更奬說蜀人使攻惠開【勞力到翻奬勸也說式芮翻誘也】於是處處蜂起凡諸離散者一時還合與寶首進逼成都衆號二十萬惠開欲擊之將佐皆曰今慰勞使至而拒之【使疏吏翻下同】何以自明惠開曰今表啓路絶不戰則何以得通使京師乃遣宋寧太守蕭惠訓等將萬兵與戰大破之生擒寶首囚于成都【沈約曰文帝元嘉十年免吳營僑立宋寧郡寄治成都將即亮翻下同】遣使言狀上使執送寶首召惠開還建康上問以舉兵狀惠開曰臣唯知逆順不識天命且非臣不亂非臣不平上釋之是歲僑立兖州治淮隂徐州治鍾離青冀二州共一刺史治鬱州【兖徐青冀皆降於魏故立僑州】鬱州在海中周數百里累石為城高八九尺【高居傲翻】虚置郡縣荒民無幾【幾居豈翻】 張永沈攸之進兵逼彭城軍于下【丘盖翻】分遣羽林監王穆之將卒五千守輜重於武原【水經注武原縣在下邳縣西北按武原縣自漢以來屬彭城郡宋志南彭城郡有武原縣而徐州之彭城無之盖自晉氏永嘉之亂其民南徙而故縣丘墟也杜佑曰泗州下邳縣北有漢武原縣故城重直用翻】魏尉元至彭城薛安都出迎元遣李璨與安都先入城收其管籥别遣孔伯恭以精甲二千安撫内外然後入【受降如受敵尉元得之】其夜張永攻南門不克而退元不禮於薛安都安都悔降復謀叛魏【降戶江翻復扶又翻】元和之不果發【和之者諧輯之也或曰和之當作知之】安都重賂元等委罪於女婿裴祖隆而殺之元使李珠與安都守彭城自將兵擊張永絶其糧道又破王穆之於武原穆之帥餘衆就永元進攻之【帥讀曰率】

  資治通鑑卷一百三十一  
    


 


 



 

 
  







 


  
  
 
 
 


  

 















	
	









































 
  



















 





 












  
  
  

 





