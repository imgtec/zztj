<!DOCTYPE html PUBLIC "-//W3C//DTD XHTML 1.0 Transitional//EN" "http://www.w3.org/TR/xhtml1/DTD/xhtml1-transitional.dtd">
<html xmlns="http://www.w3.org/1999/xhtml">
<head>
<meta http-equiv="Content-Type" content="text/html; charset=utf-8" />
<meta http-equiv="X-UA-Compatible" content="IE=Edge,chrome=1">
<title>資治通鑒_206-資治通鑑卷二百五_206-資治通鑑卷二百五</title>
<meta name="Keywords" content="資治通鑒_206-資治通鑑卷二百五_206-資治通鑑卷二百五">
<meta name="Description" content="資治通鑒_206-資治通鑑卷二百五_206-資治通鑑卷二百五">
<meta http-equiv="Cache-Control" content="no-transform" />
<meta http-equiv="Cache-Control" content="no-siteapp" />
<link href="/img/style.css" rel="stylesheet" type="text/css" />
<script src="/img/m.js?2020"></script> 
</head>
<body>
 <div class="ClassNavi">
<a  href="/24shi/">二十四史</a> | <a href="/SiKuQuanShu/">四库全书</a> | <a href="http://www.guoxuedashi.com/gjtsjc/"><font  color="#FF0000">古今图书集成</font></a> | <a href="/renwu/">历史人物</a> | <a href="/ShuoWenJieZi/"><font  color="#FF0000">说文解字</a></font> | <a href="/chengyu/">成语词典</a> | <a  target="_blank"  href="http://www.guoxuedashi.com/jgwhj/"><font  color="#FF0000">甲骨文合集</font></a> | <a href="/yzjwjc/"><font  color="#FF0000">殷周金文集成</font></a> | <a href="/xiangxingzi/"><font color="#0000FF">象形字典</font></a> | <a href="/13jing/"><font  color="#FF0000">十三经索引</font></a> | <a href="/zixing/"><font  color="#FF0000">字体转换器</font></a> | <a href="/zidian/xz/"><font color="#0000FF">篆书识别</font></a> | <a href="/jinfanyi/">近义反义词</a> | <a href="/duilian/">对联大全</a> | <a href="/jiapu/"><font  color="#0000FF">家谱族谱查询</font></a> | <a href="http://www.guoxuemi.com/hafo/" target="_blank" ><font color="#FF0000">哈佛古籍</font></a> 
</div>

 <!-- 头部导航开始 -->
<div class="w1180 head clearfix">
  <div class="head_logo l"><a title="国学大师官网" href="http://www.guoxuedashi.com" target="_blank"></a></div>
  <div class="head_sr l">
  <div id="head1">
  
  <a href="http://www.guoxuedashi.com/zidian/bujian/" target="_blank" ><img src="http://www.guoxuedashi.com/img/top1.gif" width="88" height="60" border="0" title="部件查字,支持20万汉字"></a>


<a href="http://www.guoxuedashi.com/help/yingpan.php" target="_blank"><img src="http://www.guoxuedashi.com/img/top230.gif" width="600" height="62" border="0" ></a>


  </div>
  <div id="head3"><a href="javascript:" onClick="javascript:window.external.AddFavorite(window.location.href,document.title);">添加收藏</a>
  <br><a href="/help/setie.php">搜索引擎</a>
  <br><a href="/help/zanzhu.php">赞助本站</a></div>
  <div id="head2">
 <a href="http://www.guoxuemi.com/" target="_blank"><img src="http://www.guoxuedashi.com/img/guoxuemi.gif" width="95" height="62" border="0" style="margin-left:2px;" title="国学迷"></a>
  

  </div>
</div>
  <div class="clear"></div>
  <div class="head_nav">
  <p><a href="/">首页</a> | <a href="/ShuKu/">国学书库</a> | <a href="/guji/">影印古籍</a> | <a href="/shici/">诗词宝典</a> | <a   href="/SiKuQuanShu/gxjx.php">精选</a> <b>|</b> <a href="/zidian/">汉语字典</a> | <a href="/hydcd/">汉语词典</a> | <a href="http://www.guoxuedashi.com/zidian/bujian/"><font  color="#CC0066">部件查字</font></a> | <a href="http://www.sfds.cn/"><font  color="#CC0066">书法大师</font></a> | <a href="/jgwhj/">甲骨文</a> <b>|</b> <a href="/b/4/"><font  color="#CC0066">解密</font></a> | <a href="/renwu/">历史人物</a> | <a href="/diangu/">历史典故</a> | <a href="/xingshi/">姓氏</a> | <a href="/minzu/">民族</a> <b>|</b> <a href="/mz/"><font  color="#CC0066">世界名著</font></a> | <a href="/download/">软件下载</a>
</p>
<p><a href="/b/"><font  color="#CC0066">历史</font></a> | <a href="http://skqs.guoxuedashi.com/" target="_blank">四库全书</a> |  <a href="http://www.guoxuedashi.com/search/" target="_blank"><font  color="#CC0066">全文检索</font></a> | <a href="http://www.guoxuedashi.com/shumu/">古籍书目</a> | <a   href="/24shi/">正史</a> <b>|</b> <a href="/chengyu/">成语词典</a> | <a href="/kangxi/" title="康熙字典">康熙字典</a> | <a href="/ShuoWenJieZi/">说文解字</a> | <a href="/zixing/yanbian/">字形演变</a> | <a href="/yzjwjc/">金 文</a> <b>|</b>  <a href="/shijian/nian-hao/">年号</a> | <a href="/diming/">历史地名</a> | <a href="/shijian/">历史事件</a> | <a href="/guanzhi/">官职</a> | <a href="/lishi/">知识</a> <b>|</b> <a href="/zhongyi/">中医中药</a> | <a href="http://www.guoxuedashi.com/forum/">留言反馈</a>
</p>
  </div>
</div>
<!-- 头部导航END --> 
<!-- 内容区开始 --> 
<div class="w1180 clearfix">
  <div class="info l">
   
<div class="clearfix" style="background:#f5faff;">
<script src='http://www.guoxuedashi.com/img/headersou.js'></script>

</div>
  <div class="info_tree"><a href="http://www.guoxuedashi.com">首页</a> > <a href="/SiKuQuanShu/fanti/">四库全书</a>
 > <h1>资治通鉴</h1> <!--         下载:【右键另存为】即可 --></div>
  <div class="info_content zj clearfix">
  
<div class="info_txt clearfix" id="show">
<center style="font-size:24px;">206-資治通鑑卷二百五</center>
    資治通鑑卷二百五   宋 司馬光 撰<br />
<br />
  胡三省 音注<br />
<br />
  唐紀二十一【起玄黓執徐盡柔兆涒灘凡五年】<br />
<br />
  則天順聖皇后中之上<br />
<br />
  長夀元年【是年四月改元如意九月改元長夀自四月以前猶是天授三年】正月戊辰朔太后享萬象神宫 臘月立故于闐王尉遲伏闍雄之子瑕為于闐王【闐徒賢翻尉紆勿翻闍視遮翻】 春一月丁卯太后引見存撫使所舉人【遣存撫使見上卷天授元年見賢遍翻使疏吏翻】無問賢愚悉加擢用高者試鳳閣舍人給事中次試員外郎侍御史補闕拾遺校書郎【唐校書郎正九品上 按考異曰統紀天授二年二月十道舉人石艾縣令王山齡等六十人擢為拾遺補闕懷州録事参軍霍獻可等二十四人為御史并州録事參軍徐昕等二十四人為著作佐郎及評事内黄尉崔宣道等二十三人為衛佐疑與此只是一事】試官自此始時人為之語曰補闕連車載拾遺平斗量【容齋隨筆以為此語出於張鷟】欋推侍御史【欋其俱翻爾雅釋名曰齊魯謂四齒杷為欋推吐雷翻】盌脱校書郎【盌烏管翻坡詩但信櫝藏終自售豈知盌脱本無模】有舉人沈全交續之曰心存撫使眯目聖神皇【戶吴翻麫粘也眯莫禮翻物入目中也老子曰播糠迷目】為御史紀先知所禽劾其誹謗朝政請杖之朝堂然後付法【劾戶槩翻又戶得翻誹敷尾翻朝直遥翻】太后笑曰但使卿輩不濫何恤人言宜釋其罪先知大慙太后雖濫以禄位收天下人心然不稱職者尋亦黜之或加刑誅挾刑賞之柄以駕御天下政由己出明察善斷故當時英賢亦競為之用【稱尺證翻斷丁亂翻】寧陵丞廬江郭覇以諂諛干太后【寧陵縣属宋州本戰國時魏之甯城漢高祖改為寜陵縣廬江漢龍舒縣地屬廬江郡梁置湖州隋廢州為廬江縣屬廬州 考異曰新傳名弘霸舊傳御史臺記皆單名霸唯統紀延載元年云弘霸僉載云應革命舉盖正謂此時也今從臺記】拜監察御史【監古衘翻】中丞魏元忠病霸往問之因嘗其糞喜曰大夫糞甘則可憂【中丞而呼為大夫過呼之也】今苦無傷也元忠大惡之【惡烏路翻】遇人輒告之 戊辰以夏官尚書楊執柔同平章事執柔恭仁弟之孫也太后以外族用之【太后母楊氏尚辰羊翻】 初隋煬帝作東都【見一百八十卷大業元年焬羊亮翻】無外城僅有短垣而已至是鳳閣侍郎李昭德始築之 左臺中丞來俊臣羅告同平章事任知古狄仁傑裴行本司禮卿崔宣禮前文昌左丞盧獻御史中丞魏元忠潞州刺史李嗣真謀反【任音壬嗣祥吏翻 考異曰舊來俊臣傳云地官尚書狄仁傑益州長史任令暉冬官尚書李遊道秋官尚書袁智弘司賓卿崔基文昌左史盧獻等六人並為羅告李嶠傳云太后使給事中李嶠與大理少卿張德裕侍御史劉憲覆其獄德裕等雖知其枉懼罪並從俊臣所奏嶠曰豈有知其枉濫而不為申明哉孔子曰見義不為無勇也乃與德裕等列其枉狀由是忤旨出為潤州司馬按嶠平生行事恐不能如此今不取】先是來俊臣奏請降敕一問即承反者得減死【先悉薦翻】及知古等下獄【下遐嫁翻】俊臣以此誘之【誘音酉】仁傑對曰大周革命萬物惟新唐室舊臣甘從誅戮反是實俊臣乃少寛之【少詩沼翻下同】判官王德壽謂仁傑曰【判官俊臣之屬官也】尚書定減死矣德壽業受驅策欲求少階級煩尚書引楊執柔可乎仁傑曰皇天后土遣狄仁傑為如此事以頭觸柱血流被面德壽愳而謝之【被皮義翻】侯思止鞫魏元忠元忠辭氣不屈思止怒命倒曳之元忠曰我薄命譬如墜驢足絓於鐙為所曳耳【絓戶掛翻鐙都鄧翻】思止愈怒更曳之元忠曰侯思止汝若須魏元忠頭則截取何必使承反也狄仁傑既承反有司待報行刑不復嚴備仁傑裂衾帛書寃狀置棉衣中謂王德壽曰天時方熱請授家人去其綿德壽許之仁傑子光遠得書持之告變得召見【復扶又翻去羌呂翻見賢遍翻】則天覽之以問俊臣對曰仁傑等下獄臣未嘗褫其巾帶【禠地爾翻】寢處甚安【處昌呂翻】苟無事實安肯承反太后使通事舍人周綝往視之俊臣暫假仁傑等巾帶羅立於西使綝視之綝不敢視惟東顧唯諾而已【綝丑林翻唯于癸翻】俊臣又詐為仁傑等謝死表使綝奏之樂思晦男未十歲沒入司農【思晦死見上卷上年】上變得召見【上時掌翻見賢遍翻】太后問狀對曰臣父已死臣家已破但惜陛下法為俊臣等所弄陛下不信臣言乞擇朝臣之忠清陛下素所信任者【朝直遥翻】為反狀以付俊臣無不承反矣太后意稍寤召見仁傑等問曰卿承反何也對曰不承則已死於拷掠矣【陸德明經典釋文掠音亮】太后曰何為作謝死表對曰無之出表示之乃知其詐於是出此七族庚午貶知古江夏令仁傑彭澤令宣禮夷陵令元忠涪陵令獻西鄉令【江夏木漢沙羨縣地屬江夏郡晉改沙羨為沙陽江漢二水會于縣西春秋謂之夏汭晋宋謂之夏口宋置江夏郡治於此隋因郡名置江夏縣唐屬鄂州彭澤漢縣屬豫章隋更名龍城唐復曰彭澤屬江州涪陵縣漢屬巴郡劉蜀置涪陵郡隋涪陵縣屬渝州唐武德元年分置涪州為州治所西鄉即漢成固縣地蜀置西鄉縣後魏為洋州治所夏戶雅翻涪音浮】流行本嗣真于嶺南俊臣與武承嗣等固請誅之太后不許俊臣乃獨稱行本罪尤重請誅之秋官郎中徐有功駮之【駮比角翻】以為明主有更生之恩【更工衡翻】俊臣不能將順虧損恩信殿中侍御史貴鄉霍獻可【後魏分舘陶西界置貴鄉縣於趙城周建德七年自趙城東南移三十里以孔思集寺為縣治所大象二年於縣置魏州】宣禮之甥也言於太后曰陛下不殺崔宣禮臣請隕命於前以頭觸殿階血流霑地以示為人臣者不私其親太后皆不聼獻可常以緑帛裹其傷微露之於幞頭下【續事始曰三代黔首以帛絹裹髪周武帝裁為四脚各以幞頭馬周請重繫前脚】冀太后見之以為忠 甲戌補闕薛謙光上疏【上時掌翻】以為選舉之法宜得實才取舍之間風化所繫今之選人咸稱覓舉奔競相尚諠訴無慙【選宣戀翻】至於才應經邦惟令試策武能制敵止驗彎弧昔漢武帝見司馬相如賦恨不同時及置之朝廷終文園令【漢司馬相如為子虚賦武帝讀而善之曰朕獨不得與此人同時楊得意曰臣邑人司馬相如自言為此賦上召以為郎後為孝文園令病免而卒】知其不堪公卿之任故也吴起將戰左右進劍起曰將者提鼓揮桴臨敵决疑一劍之任非將事也【將者非將即亮翻桴方無翻】然則虚文豈足以佐時善射豈足以克敵要在文吏察其行能武吏觀其勇畧考居官之臧否【行下孟翻否音鄙】行舉者賞罸而已 來俊臣求金於左衛大將軍泉獻誠不得誣以謀反下獄乙亥縊殺之【下遐嫁翻縊於計翻】 庚辰司刑卿檢校陜州刺史李游道為冬官尚書同平章事【陜失冉翻】 二月己亥吐蕃党項部落萬餘人内附【吐從暾入聲党底浪翻】分置十州 戊午以秋官尚書袁智弘同平章事【秋官刑部】 夏四月丙申赦天下改元如意【如意元年起此】 五月丙寅禁天下屠殺及捕魚蝦江淮旱飢民不得采魚蝦餓死者甚衆【后禁屠捕而殺人如刈草菅可以人而不如物乎蝦戶加翻】右拾遺張德生男三日私殺羊會同僚補闕杜肅懷一餤【餤徒濫翻又弋廉翻徒甘翻】上表告之【上時掌翻】明日太后對仗謂德曰聞卿生男甚喜德拜謝太后曰何從得肉德叩頭服辠太后曰朕禁屠宰吉凶不預然卿自今召客亦須擇人出肅表示之肅大慙舉朝欲唾其面【朝直遥翻唾吐臥翻】吐蕃酋長曷蘇帥部落請内附以右玉鈐衛將軍張玄遇為安撫使將精卒二萬迎之六月軍至大渡水西曷蘇事洩為國人所禽别部酋長昝捶帥羌蠻八千餘人内附玄遇以其部落置萊川州而還【酋慈由翻長知兩翻帥讀曰率鈐其廉翻使疏吏翻將即亮翻又音如字昝子感翻捶止橤翻新書作挿黎州都督府所管覊縻州有米川州新書作葉州還從宣翻又音如字 考異曰唐紀作沓揺今從實録】 辛亥萬年主簿徐堅上疏以為書有五聼之道【上時掌翻疏所據翻周禮小司寇以五聼聼獄訟求民情一曰辭聼觀其所出言不直則煩二曰色聼觀其顔色不直則赧然三曰氣聼不直則喘四曰耳聼觀其聼聆不直則惑五曰目聼觀其眸子不直則眊然】令著三覆之奏【見一百九十三卷太宗貞觀五年】竊見比有敕推按反者【比毗至翻】令使者得實即行斬决【令力丁翻使疏吏翻】人命至重死不再生萬一懷枉吞聲赤族豈不痛哉此不足肅姦逆而明典刑適所以長威福而生疑懼臣望絶此處分【長知兩翻處昌呂翻下處事同分扶問翻】依法覆奏又法官之任宜加簡擇有用法寛平為百姓所稱者願親而任之有處事深酷不允人望願踈而退之堅齊聃之子也【處昌呂翻徐齊聃見二百一卷高宗咸亨元年聃它甘翻】 夏官侍郎李昭德密言於太后曰魏王承嗣權太重【夏官兵部嗣祥吏翻】太后曰吾姪也故委以腹心昭德曰姪之於姑其親何如子之於父子猶有簒弑其父者况姪乎今承嗣既陛下之姪為親王又為宰相【相息亮翻】權侔人主臣恐陛下不得久安天位也太后矍然曰朕未之思【矍九縳翻】秋七月戊寅以文昌左相同鳳閣鸞臺三品武承嗣為特進納言武攸寧為冬官尚書【嗣祥吏翻冬官工部尚辰羊翻】夏官尚書同平章事楊執柔為地官尚書並罷政事以秋官侍郎新鄭崔元綜為鸞臺侍郎【秋官刑部新鄭春秋鄭國都鄭武公隨周平王東遷邑於虢鄶之間莊公所謂吾先君新邑于此是也漢為新鄭縣屬河南郡晉魏省隋開皇十六復置屬鄭州】夏官侍郎李昭德為鳳閣侍郎檢校天官侍郎姚璹為文昌左丞【夏官兵部鳳閣中書天官吏部改尚書為文昌璹殊六翻】檢校地官侍郎李元素為文昌右丞與司賓卿崔神基【地官戶部司賓卿即鴻臚卿】並同平章事【考異曰舊昭德傳舉明德累遷至鳳閣侍郎長壽二年增置夏官侍郎以昭德為之是歲遷鳳閣鸞臺平章事新紀表傳皆云昭德自夏官侍郎遷鳳閣侍郎同平章事蓋昭德自鳳閣為夏官自夏官復為鳳閣也婁師德傳長壽元年增置夏官侍郎今從之崔神基實録作崔基今從新紀表】璹思廉之孫【姚思廉事隋及唐】元素敬玄之弟也【李敬玄相高宗】辛巳以營繕大匠王璿為夏官尚書同平章事【光宅改將作監為營繕監璿似宣翻】承嗣亦毁昭德於太后太后曰吾任昭德始得安眠此代吾勞汝勿言也是時酷吏恣横【横下孟翻】百官畏之側足昭德獨廷奏其姧太后好祥瑞【好呼到翻】有獻白石赤文者執政詰其異【詰去吉翻】對曰以其赤心昭德怒曰此石赤心它石盡反邪【邪音耶】左右皆笑襄州人胡慶以丹漆書龜腹曰天子萬萬年詣闕獻之昭德以刀刮盡奏請付法太后曰此心亦無惡命釋之太后習猫使與鸚鵡共處【處昌呂翻】出示百官傳觀未遍猫饑鸚鵡食之太后甚慙 太后自垂拱以來任用酷吏先誅唐宗室貴戚數百人次及大臣數百家其刺史郎將以下不可勝數【將即亮翻勝音升】每除一官戶婢竊相謂曰【戶婢官婢之直宫中門戶者】鬼朴又來矣不旬月輒遭掩捕族誅監察御史朝邑嚴善思【後魏分馮翊置澄城郡仍置南五泉縣西魏改為朝邑縣隋唐屬司州監古衘翻朝直遥翻】公直敢言時告密者不可勝數【勝音升】太后亦厭其煩命善思按問引虚伏辠者八百五十餘人羅織之黨為之不振【為于偽翻】乃相與搆陷善思坐流驩州【舊志驩州至京師陸路一萬二千四百五十二里水路一萬七千里至東都一萬一千五百九十五里水路一萬六千二百二十里宋白曰驩州日南郡堯放驩兜于崇山即此】太后知其枉尋復召為渾儀監丞【后改司天監為渾儀監丞從七品下復扶又翻渾戶本翻】善思名譔以字行【譔七免翻】右補闕新鄭朱敬則以太后本任威刑以禁異議今既革命衆心已定宜省刑尚寛乃上疏以為李斯相秦用刻薄變詐以屠諸侯不知易之以寛和卒至土崩此不知變之祸也【事見秦紀上時掌翻相息亮翻卒子恤翻】漢高祖定天下陸賈叔孫通說之以禮義傳世十二此知變之善也【說輪芮翻事見漢紀】自文明草昧天地屯象【草造也昧蒙也造物之始始於冥昧言后稱制之初改元文明造始之時也屯者物之始蒙者物之稺言后稱制之初猶天地生物之始屯涉倫翻】三叔流言四凶搆難【三叔指韓霍諸王四凶指徐敬業等難乃旦翻】不設鉤距無以應天順人不切刑名不可摧奸息暴故置神器開告端【謂鑄匭以開告密之門也】曲直之影必呈包藏之心盡露神道助直無罪不除蒼生晏然紫宸易主然而急趨無善迹【以步趨為諭也】促柱少和聲【以琴瑟為諭也少詩沼翻】向時之妙策乃當今之芻狗也【芻狗祭祀所用既祭則弃之矣】伏願覽秦漢之得失考時事之合宜審糟粕之可遺【以酒為喻也取其醇汁而去其糟粕】覺蘧廬之須毁【莊子曰蘧廬可以一宿而不可以久處郭象注云蘧廬傳舍也】去萋斐之牙角【詩云萋兮斐兮成是貝錦彼譛人者亦已大甚去羌呂翻】頓奸險之鋒芒窒羅織之源掃朋黨之迹使天下蒼生坦然大悦豈不樂哉【樂音洛】太后善之賜帛三百段侍御史周矩上疏曰推劾之吏皆相矜以虐泥耳籠頭枷研楔轂【枷研以重枷研其頸楔轂以鐵圈轂其首而加楔楔先結翻轂呼角翻】摺膺籖爪【摺與拉同力答翻摧也折也膺胸也籖爪以竹籖其爪甲今鞫獄者十指下籖即其遺虐】懸髪薰耳號曰獄持或累日節食連宵緩問晝夜揺撼使不得眠號曰宿囚此等既非木石且救目前苟求賖死【賖遠也言伏法而死較死於獄中為稍賖也】臣竊聽輿議皆稱天下太平何苦須反豈被告者盡是英䧺欲求帝王邪但不勝楚毒自誣耳【被皮義翻勝音升】願陛下察之今滿朝側息不安【朝直遥翻】皆以為陛下朝與之密夕與之讐不可保也周用仁而昌秦用刑而亡願陛下緩刑用仁天下幸甚太后頗采其言制獄稍衰 【考異曰御史臺記云書奏遂授洛州司功舊薛懷義傳云矩劾奏懷義遷矩天官員外郎竟為懷義所搆下獄免官御史臺記又云時天官選曹無緒敕矩監之侍郎李景謀為矩所制乃引為員外不閑於吏道自此左出矣據舊傳矩劾奏薛懷義在後若此年出為洛州司功則不當復劾懷義但舊傳矩疏在載初元年二月是時制獄未息今因朱敬則疏終言之】 太后春秋雖高善自塗澤雖左右不覺其衰丙戌敕以齒落更生九月庚子御則天門赦天下改元【至是方改元長夀自此以後方是長壽元年】更以九月為社【更工衡翻】 制於并州置北都 癸丑同平章事李遊道王璿袁智弘崔神基李元素春官侍郎孔思元益州長史任令輝皆為王弘義所陷流嶺南【璿似宣翻長知兩翻任音壬】 左羽林中郎將來子珣坐事流愛州尋卒【愛州至京師八千八百里東都八千一百里將即亮翻卒子恤翻】初新豐王孝傑從劉審禮撃吐蕃為副摠管與審禮皆没於吐蕃【新豐縣屬雍州後改昭應劉審禮没見二百二卷高宗儀鳳三年吐從瞰入聲】贊普見孝傑泣曰貌類吾父厚禮之後竟得歸累遷右鷹揚衛將軍【光宅改左右武衛為左右鷹揚衛】孝傑久在吐蕃知其虚實會西州都督唐休璟請復取龜兹于闐踈勒碎葉四鎮【復扶又翻又音如字龜兹音丘慈又音屈佳闐徒賢翻又徒見翻弃四鎮見二百一卷高宗咸亨元年】敕以孝傑為武威軍摠管與武衛大將軍阿史那忠蓈將兵撃吐蕃【此時既改武衛為鷹揚衛不應復以舊官名命忠蓈豈史家仍襲舊官名而書之邪將又音如字】冬十月丙戍大破吐蕃復取四鎮置安西都護府於龜兹發兵戍之<br />
<br />
  二年正月壬辰朔太后享萬象神宫以魏王承嗣為亞獻梁王三思為終獻太后自制神宫樂用舞者九百人戶婢團兒為太后所寵信有憾於皇嗣乃譛皇嗣妃<br />
<br />
  劉氏德妃竇氏為厭呪【厭於恊翻又於琰翻】癸巳妃與德妃朝太后於嘉豫殿【朝直遥翻】既退同時殺之 【考異曰新本紀云臘月癸亥殺皇嗣妃竇氏舊傳云正月二日今從之今按德妃竇氏即玄宗母也】瘞於宫中莫知所在【瘞於計翻】德妃抗之曾孫也【竇抗太穆皇后之從兄】皇嗣畏忤旨不敢言【忤五故翻】居太后前容止自如團兒復欲害皇嗣有言其情於太后者太后乃殺團兒【復扶又翻 考異曰劉子玄太上實錄云韋國兒諂佞多端天后尤所信任欲私於上而拒焉怨望遂作桐人潜埋於二妃院内譛殺之又矯制按問上今從則天實錄】是時告密者皆誘人奴婢告其主以求功賞德妃父孝諶為潤州刺史有奴妄為妖異以恐德妃母龎氏【誘音酉諶氏壬翻妖於喬翻寵皮江翻】龎氏懼奴請夜祠禱解因發其事下監察御史龍門薛季昶按之【監古衘翻下遐嫁翻】季昶誣奏以為與德妃同祝詛先涕泣不自勝【祝職救翻勝音升】乃言曰龎氏所為臣子所不忍道太后擢季昶為給事中龎氏當斬其子希瑊【瑊古咸翻】詣侍御史徐有功訟寃有功牒所司停刑上奏論之以為無罪季昶奏有功阿黨惡逆請付法法司處有功罪當絞令史以白有功【侍御史之屬有令史十七人上時掌翻處昌呂翻】有功歎曰豈我獨死諸人永不死邪既食掩扇而寢人以為有功苟自強必内憂懼密伺之方熟寢【伺相吏翻】太后召有功迎謂曰卿比按獄失出何多對曰失出人臣之小過好生聖人之大德【誤出人罪謂之失出比毗至翻好呼到翻】太后默然由是龎氏得減死與其三子皆流嶺南孝諶貶羅州司馬有功亦除名 【考異曰舊有功傳有功為御史坐龎氏除名尋起為左司郎中竇孝諶傳長夀二年龎氏為酷吏所䧟御史臺記有功自秋官員外郎坐龎氏除名為流人月餘授御史按實録有功天授初累補司刑丞秋官員外郎稍遷郎中後以公事免萬歲通天元年擢拜殿中侍御史今從之】 戊申姚璹奏請令宰相撰時政記【會要璹以為帝王謨訓不可闕於紀述史官踈遠無因得書請自今以後所論軍國政要宰臣一人撰録號為時政記】月送史舘從之時政記自此始 臘月丁卯降皇孫成器為壽春王恒王成義為衡陽王【恒戶登翻】楚王隆基為臨淄王衛王隆範為巴陵王趙王隆業為彭城王皆睿宗之子也 春一月庚子以夏官侍郎婁師德同平章事師德寛厚清慎犯而不校與李昭德俱入朝【朝直遥翻】師德體肥行緩昭德屢待之不至怒罵曰田舍夫師德徐笑曰師德不為田舍夫誰當為之其弟除代州刺史將行師德謂曰吾備位宰相汝復為州牧【復扶又翻】榮寵過盛人所疾也將何以自免弟長跪曰自今雖有人唾某面某拭之而已庶不為兄憂師德愀然曰【愀七小翻】此所以為吾憂也人唾汝面怒汝也汝拭之乃逆其意所以重其怒夫唾不拭自乾【乾音干】當笑而受之 甲寅前尚方監裴匪躬内常侍范雲仙坐私謁皇嗣腰斬於市【光宅改少府監為尚方監内侍省有内常侍六人正五品下漢中常侍之職也 考異曰舊來俊臣傳云按張䖍朂范雲仙於洛陽牧院䖍朂等不堪其苦自訟於徐有功俊臣命衛士以亂刀斫殺之雲仙亦言歷事先朝稱所司寃苦俊臣命截去其舌士庶膽破無敢言者按張䖍朂天授二年被殺雲仙此年坐謁皇嗣斬今從實録】自是公卿以下皆不得見又有告皇嗣潜有異謀者太后命來俊臣鞫其左右左右不勝楚毒皆欲自誣【勝音升】太常工人京兆安金藏【時公卿不得見皇嗣唯金藏等工人得在左右】大呼謂俊臣曰公既不信金藏之言請剖心以明皇嗣不反即引佩刀自剖其胸五藏皆出流血被地太后聞之令轝入宫中【呼火故翻藏徂浪翻被皮義翻轝羊茹翻】使醫内五藏以桑皮線縫之傅以藥經宿始蘇太后親臨視之歎曰吾有子不能自明使汝至此即命俊臣停推【停其獄不復推鞫也】睿宗由是得免 罷舉人習老子更習太后所造臣軌【更工衡翻習老子見二百二卷高宗上元元年】 二月丙子新羅王政明卒遣使立其子理洪為王【卒子恤翻使疏吏翻】 乙亥禁人間錦侍御史侯思止私畜錦李昭德按之杖殺於朝堂【朝直遥翻】 或告嶺南流人謀反太后遣司刑評事萬國俊攝監察御史就按之【監古衘翻】國俊至廣州悉召流人矯制賜自盡流人號呼不服【號戶高翻】國俊驅就水曲盡斬之一朝殺三百餘人然後詐為反狀還奏因言諸道流人亦必有怨望謀反者不可不早誅太后喜擢國俊為朝散大夫行侍御史【朝直遥翻散悉亶翻】更遣右翊衛兵曹參軍劉光業【按武德四年已改左右翊衛為左右衛疑翊字衍兵曹參軍掌王府武官宿衛番第受其名數而大將軍配焉】司刑評事王德夀苑南面監丞鮑思恭【唐京都苑各有四面監監各一人從六品下副監一人從七品下丞一人正八品下各掌所管面苑内宫舘園池與其種植修葺之事丞則掌判監事】尚輦直長王大貞【長知兩翻】右武威衛兵曹參軍屈貞筠皆攝監察御史詣諸道按流人光業等以國俊多殺蒙賞爭効之光業殺七百人德夀殺五百人自餘少者不減百人其遠年雜犯流人亦與之俱斃太后頗知其濫制六道流人未死者并家屬皆聽還鄉里國俊等亦相繼死或得罪流竄 【考異曰實録曰光業等亦受鸞臺侍郎傳遊藝之旨按天授二年遊藝已死舊遊藝傳曰遊藝請則天發六道使雖身死之後竟從其謀武后本遣萬國俊一使國俊還言諸道流人亦反故更遣五使耳遊藝豈豫知遣六道使此所謂天下之惡皆歸焉者也潘遠紀閒曰補闕李秦授寓直中書進封事曰陛下自登極誅斥李氏及諸大臣其家人親族流放在外以臣所料且數萬人如一旦同心招集為逆出陛下不意臣恐社稷必危䜟曰代武者劉夫劉者流也陛下不殺此輩臣恐為祸深焉天后納之夜中召入謂曰卿名秦授天以卿授朕也何啟予心即拜考功員外郎仍知制誥賜朱紱女妓十人金帛稱是與謀發敕使十人於十道安慰流者其實賜墨敕與牧守有流放者殺之天后度流人已死又使使者安撫流人曰吾前發十道使安慰流人何使者不曉吾意擅加殺害深為酷暴其輒殺流人使並所在鎻項將至害流人處斬之以快亡魂諸流人未死或它事繫者兼家口放還按當時止誅嶺南一道因萬國俊言更發五道使非併發十道使也十道在近地者何嘗冇流人也國俊既以多殺受賞餘使或病死或自以它罪流竄必無并斬之理今並從實録及舊傳】 來俊臣誣冬官尚書蘇幹云在魏州與琅邪王冲通謀【冲舉兵見上卷垂拱四年】夏四月乙未殺之 五月癸丑棣州河溢【棣州後漢樂安郡中廢唐武德四年分滄州之厭次陽信滴河樂陵置棣州】 秋九月丁亥朔日有食之 魏王承嗣等五千人表請加尊號曰金輪聖神皇帝 乙未太后御萬象神宫受尊號赦天下作金輪等七寶【七寶一曰金輪寶曰白象寶曰女寶曰馬寶曰珠寶曰主兵臣寶曰主藏臣寶】每朝會陳之殿庭【朝直遥翻】 庚子追尊昭安皇帝曰渾元昭安皇帝【渾戶本翻】文穆皇帝曰立極文穆皇帝孝明高皇帝皇后從帝號【后又追尊其三世】辛丑以文昌左丞同平章事姚璹為司賓卿罷政事以司賓卿萬年豆盧欽望為内史【新書宰相世系表豆盧氏本姓慕容氏北地王精降後魏北人謂歸義為豆盧因以為氏】文昌左丞韋巨源同平章事秋官侍郎吴人陸元方為鸞臺侍郎同平章事巨源孝寛之玄孫也【韋孝寛事宇文氏為名將】<br />
<br />
  延載元年【是年五月改元】正月丙戌太后享萬象神宫 突厥可汗骨篤禄卒其子幼弟默啜自立為可汗臘月甲戌默啜宼靈州 室韋反【北史曰室韋盖契丹之類其南者為契丹在北者為室韋新書室韋契丹别種東胡之北邉盖丁零苖裔也地據黄龍北傍峱越河直京師東北七千里東黑水靺鞨西突厥南契丹北渤海其國無君長惟大酋皆號莫賀咄筦攝其部而附於突厥】遣右鷹揚衛大將軍李多祚擊破之 春一月以婁師德為河源等軍檢校營田大使【使疏吏翻下同】 二月武威道摠管王孝傑破吐蕃㪍論贊刃突厥可汗俀子等於冷泉及大嶺【俀子西突厥部所立也俀吐猥翻弱也大嶺谷名】各三萬餘人碎葉鎮守使韓思忠破泥熟俟斤等萬餘人【俟渠之翻考異曰此事諸書皆無唯統紀有之統紀又破吐蕃萬泥勲沒馱城此語不可暁今刪去】 庚午以僧懷義為代北道行軍大摠管【考異曰實録新紀皆云伐逆道今從舊懷義傳】以討默啜 三月甲申以鳳閣舍人蘇味道為鳳閣侍郎同平章事李昭德檢校内史更以僧懷義為朔方道行軍大摠管以李昭德為長史蘇味道為司馬帥契苾明曹仁師沙吒忠義等十八將軍以討默啜【帥讀曰率下同契欺訖翻苾毗必翻吒陟加翻】未行虜退而止昭德嘗與懷義議事失其旨懷義撻之昭德惶懼請罪夏四月壬戍以夏官尚書武威道大摠管王孝傑同鳳閣鸞臺三品 五月魏王承嗣等二萬六千餘人上尊號曰越古金輪聖神皇帝【上時掌翻】甲午御則天門樓受尊號赦天下改元 天授中遣監察御史夀春裴懷古安集西南蠻六月癸丑永昌蠻酋薰期帥部落二十餘萬戶内附【姚州境冇永昌蠻居永昌郡地薰期新書作董期監古衘翻酋慈由翻】 河内有老尼居神都麟趾寺與嵩山人韋什方等以妖妄惑衆【尼女夷翻妖於喬翻】尼自號净光如來云能知未然什方自云吴赤烏年生又有老胡亦自言五百歲云見薛師已二百年矣【僧懷義本馮小實也太后使與薛紹通昭穆故考胡謂之薛師】容貌愈少【少詩照翻】太后甚信重之賜什方姓武氏秋七月癸未以什方為正諫大夫同平章事制云邁軒代之廣成【莊子曰廣成子居崆峒之上黄帝立於下風而問道廣成子曰吾修身千二百歲矣吾形未嘗衰黄帝名軒轅因曰軒氏】逾漢朝之河上【葛洪曰河上公者莫知其姓名也漢文帝時結草庵于河之濱文帝從之問老子河上公曰余注是經以來千七百餘年朝直遥翻】八月什方乞還山制罷遣之戊辰以王孝傑為瀚海道行軍摠管仍受朔方道行軍大摠管薛懷義節度 己巳以司賓少卿姚璹為納言左肅政中丞原武楊再思為鸞臺侍郎洛州司馬杜景儉為鳳閣侍郎並同平章事豆盧欽望請京官九品以上輸兩月俸以贍軍【唐制一品月俸八千食料一千八百雜用一千二百二品月俸六千五百食料一千五百雜用一千三品月俸五千一百食料一千一百雜用九百四品月俸三千五百食料七百雜用七百五品月俸三千食料雜用六百六品月俸二千食料雜用四百七品月俸一千七百五十食料雜用三百五十八品月俸一千三百食料三百雜用二百五十九品月俸一千五十食料二百五十雜用二百行署月俸一百四十食料三十俸扶用翻贍昌艶翻】轉帖百官令拜表【轉帖者止書一帖使吏以轉示百官】百官但赴拜不知何事拾遺王求禮謂欽望曰明公禄厚輸之無傷卑官貧廹柰何不使其知而欺奪之乎欽望正色拒之既上表【上時掌翻】求禮進言曰陛下富有四海軍國有儲何藉貧官九品之俸而欺奪之姚璹曰求禮不識大體求禮曰如姚璹為識大體者邪事遂寢 戊寅鸞臺侍郎同平章事崔元綜坐事流振州 武三思帥四夷酋長請鑄銅銕為天樞立於端門之外【端門洛陽皇城正南門】銘紀功德黜唐頌周以姚璹為督作使【使疏吏翻】諸胡聚錢百萬億買銅銕不能足賦民間農器以足之 九月壬午朔日有食之 殿中丞來俊臣坐贓貶同州參軍王弘義流瓊州【曹魏初置殿中監隋煬帝置少監及丞舊志瓊州至兩京與崖州道里相類 考異曰統紀云萬歲通天元年五月監察御史紀履忠劾奏御史中丞來俊臣犯狀有五請下獄理罪御史臺紀履忠與來俊臣不恊具衣冠而弹之不果黜授顔城尉俊臣誅授左領軍衛胄曹新傳云俊臣納賈人金為御史紀履忠所劾下獄當死后忠其上變得不誅免為民按舊傳云俊臣為履忠所告下獄長夀二年除殿中丞又坐贓出為同州參軍萬歲通天元年召為合宫尉統紀云萬歲通天元年紀履忠劾奏誤也王弘義傳云延載元年俊臣貶弘義亦流瓊州是俊臣長夀二年已前坐贓下獄此年又坐贓貶今從舊傳】詐稱敕追還至漢北侍御史胡元禮遇之按驗得其姦狀杖殺之内史李昭德恃太后委遇頗專權使氣人多疾之前魯王府功曹參軍丘愔上疏攻之【唐諸王府功曹參軍事正七品上掌文官簿書考課陳設愔于今翻上時掌翻下長上同疏所據翻】其畧曰陛下天授以前萬機獨斷【斷丁亂翻】自長夀以來委任昭德參奉機密獻可替否事有便利不預諮謀要待畫日將行【凡制敕皆進畫日而後行】方乃别生駁異【駁比角翻】揚露專擅顯示於人歸美引愆義不如此【善則稱君過則稱已人臣之義也】又曰臣觀其膽乃大於身鼻息所衝上拂雲漢又曰蠓宂壞隄針芒寫氣權重一去收之極難長上果毅鄧注【唐六典長上折衝果毅應宿衛者並一日上兩日下】又著石論數千言述昭德專權之狀鳳閣舍人逢弘敏取奏之【逢皮江翻】太后由是惡昭德壬寅貶昭德為南賓尉【惡烏路翻南賓縣屬欽州本漢合浦縣地隋開皇十八年置南賓縣】尋又免死流竄 太后出黎花一枝以示宰相宰相皆以為瑞杜景儉獨曰今草木黄落而此更發榮隂陽不時咎在臣等因拜謝太后曰卿真宰相也【相悉亮翻】 冬十月壬申以文昌右丞李元素為鳳閣侍郎左肅政中丞周允元檢校鳳閣侍郎並同平章事【校古効翻】允元豫州人也 嶺南獠反以容州都督張玄遇為桂永等州經畧大使以討之【容州漢合浦縣地隋為合浦郡之北流縣唐武德四年分置銅州貞觀元年改容州因容山為名獠魯皓翻使疏吏翻】<br />
<br />
  天冊萬歲元年【是年九月改元天冊萬歲】正月辛巳朔太后加號慈氏越古金輪聖神皇帝赦天下改元證聖 周允元與司刑少卿皇甫文備奏内史豆盧欽望同平章事韋巨源杜景儉蘇味道陸元方附會李昭德不能匡正欽望貶趙州【舊志趙州至京師東北一千八百四十二里東都一千三十三里】巨源貶麟州【考異曰舊紀傳新紀表傳皆作鄜州統紀作瀛州實録唐歷作麟州今從之】景儉貶溱州【貞觀】<br />
<br />
  【五年置麟州以處生羌屬松州都督府十六年開山洞置溱州屬黔州都督府舊志溱州至京師三千四百八十里東都四千二百里溱側詵翻】味道貶集州元方貶綏州刺史【舊志集州京師西南一千四百二十五里至東都二千六百里綏州京師東北一千里至東都一千八百一十九里】初明堂既成太后命僧懷義作夾紵大像【紵直呂翻檾屬今人謂之紵麻夾紵者以紵布夾縫為大像後所謂麻主是也】其小指中猶容數十人於明堂北構天堂以貯之【貯丁呂翻】堂始構為風所摧更構之日役萬人采木江嶺數年之間所費以萬億計府藏為之耗竭【藏徂浪翻為于偽翻】懷義用財如糞土太后一聽之無所問每作無遮會用錢萬緡士女雲集又散錢十車使之爭拾相蹈踐有死者【踐息淺翻】所在公私田宅多為僧有懷義頗厭入宫多居白馬寺所度力士為僧者滿千人侍御史周矩疑有姦謀固請按之太后曰卿姑退朕即令往矩至臺懷義亦至乘馬就階而下坦腹於床矩召吏將按之遽躍馬而去矩具奏其狀太后曰此道人病風不足詰所度僧惟卿所處【詰去吉翻處昌呂翻】悉流遠州遷矩天官員外郎乙未作無遮會於明堂鑿地為阬深五丈【深式浸翻】結綵為宫殿佛像皆於阬中引出之云自地涌出又殺牛取血畫大像首高二百尺云懷義刺膝血為之【高居傲翻刺七亦翻】丙申張像於天津橋南設齋時御醫沈南璆【唐六典尚藥局屬殿中省冇侍御醫四人從六品上璆音求】亦得幸於太后懷義心愠【温於問翻】是夕密燒天堂延及明堂火照城中如晝比明皆盡【比必利翻】暴風裂血像為數百段太后耻而諱之但云内作工徒誤燒麻主遂涉明堂時方酺宴左拾遺劉承慶請輟朝停酺以荅天譴【酺音蒲朝直遥翻】太后將從之姚璹曰㫺成周宣榭卜代愈隆漢武建章盛德彌永【左傳宣十五年夏成周宣榭火班書曰榭所以藏樂器宣其名也漢武時柏梁臺災乃大營建章姚璹引二事傅以已說以逢君之惡】今明堂布政之所非宗廟也不應自貶損太后乃御端門觀酺如平日命更造明堂天堂仍以懷義充使【使疏吏翻】又鑄銅為九州鼎【神都鼎曰豫州高一丈八尺受千八百石冀州鼎曰武興雍州鼎曰長安兖州鼎曰日觀青州鼎曰少陽徐州鼎曰車源揚州鼎曰江都荆州鼎曰江陵梁州鼎曰咸都八州鼎高一丈四尺各受千二百石 考異曰舊傳云懷義帥人作號頭安置之按天冊萬歲元年二月懷義死神功元年九鼎始成舊傳誤也或懷義死時方鑄耳】及十二神【十二神子屬鼠丑屬牛寅屬虎卯屬兎辰屬龍巳屬蛇午屬馬未屬羊申屬猴酉屬鷄戌屬狗亥屬豬】 皆高一丈【高古犒翻】各置其方先是河内老尼晝食一麻一米夜則烹宰宴樂畜弟子百餘人淫穢靡所不為武什方自言能合長年藥【先悉薦翻樂音洛畜吁玉翻合音閣】太后遣乘驛於嶺南采藥及明堂火尼入唁太后【唁魚變翻】太后怒叱之曰汝常言能前知何以不言明堂火因斥還河内弟子及老胡等皆逃散又有發其姦者太后乃復召尼還麟趾寺弟子畢集敕給使掩捕盡獲之【復扶又翻唐六典北齊内職有散給使五十人唐因之置内給使無常員屬宫闈局凡宦人無官品者稱内給使又冇小給使學生五十人】皆沒為官婢什方還至偃師【偃師縣屬河南府在洛城東六十里】聞事露自絞死庚子以明堂火告廟下制求直言劉承慶上疏以為火發既從麻主後及總章所營佛舍恐勞無益請罷之又明堂所以統和天人【統他綜翻】一旦焚毁臣下何心猶為酺宴憂喜相爭傷於情性又陛下垂制博訪許陳至理而左史張鼎以為今旣火流王屋彌顯大周之祥【武王代紂既渡河有火至於王屋流為烏馬融曰王屋王所居屋】通事舍人逢敏奏稱彌勒成道時有天魔燒宫七寶臺須臾散壞【魔莫婆翻考異曰僉載以七寶臺散壞為姚璹之語今從實録】斯實謟妄之邪言非君臣之正論伏願陛下乾乾翼翼【易曰君子終日乾乾夕惕若詩曰小心翼翼】無戾天人之心而興不急之役則兆人蒙賴福禄無窮獲嘉主簿彭城劉知幾【獲嘉縣本汲縣之新史鄉漢武帝行幸過此聞獲呂嘉因置獲嘉縣屬河内郡後周置修武郡隋置殷州尋廢州為獲嘉縣唐屬懷州彭城縣帶徐州幾居希翻】表陳四事其一以為皇業權輿【爾雅權輿始也】天地開闢嗣君即位黎元更始【更工衡翻】時則藉非常之慶以申再造之恩今六合清晏而赦令不息近則一年再降遠則每歲無遺至於違法悖禮之徒【悖蒲内翻下同】無賴不仁之輩編戶則寇攘為業當官則贓賄是求而元日之朝指期天澤重陽之節佇降皇恩【重直龍翻】如其忖度咸果釋免或有名垂結正罪將斷决【度徒洛翻斷丁亂翻】竊行貨賄方便規求故致稽延畢霑寛宥用使俗多頑悖時罕廉隅為善者不預恩光作惡者獨承徼幸【徼古堯翻】古語曰小人之幸君子之不幸【太宗亦嘗引是言】斯之謂也望陛下而今而後頗節於赦使黎氓知禁姦宄肅清其二以為海内具僚九品以上每歲逢赦必賜階勛【唐制文散階二十九武散階亦二十九勲級十有二轉】至於朝野宴集公私聚會緋服衆於青衣【上元敕四品服深緋五品服淺緋九品服深青朝直遥翻下同】象板多於木笏【唐制五品以上笏用象九品以上用木】皆榮非德舉位罕才升不知何者為妍蚩何者為美惡臣望自今以後稍息私恩使有善者逾效忠勤無才者咸知勉勵其三以為陛下臨朝踐極取士大廣六品以下軄事清官遂乃方之土芥比之沙礫【礫音歷】若遂不加沙汰臣恐有穢皇風其四以為今之牧伯遷代太速倏來忽往蓬轉萍流既懷苟且之謀何暇循良之政望自今刺史非三歲以上不可遷官仍明察功過尤甄賞罰疏奏太后頗嘉之【甄稽延翻别也疏所去翻】是時官爵易得而法網嚴峻【易以䜴翻】故人競為趨進而多䧟刑戮知幾乃著思慎賦以刺時見志焉 丙午以王孝傑為朔方道行軍摠管撃突厥 春二月己酉朔日有食之 僧懷義益驕恣太后惡之【惡烏路翻】既焚明堂心不自安言多不順太后密選宫人有力者百餘人以防之壬子執之於瑶光殿前樹下使建昌王武攸寜帥壮士敺殺之【帥讀曰率敺烏口翻 考異曰舊傳云又有發其隂謀者太平公主乳母張夫人令壮士縳而縊殺之送尸白馬寺其侍者僧徒皆流竄遠惡處李啇隱宜都内人傳云武后簒既久頗放縱耽内習不敬宗廟四方日有叛逆防豫不暇時宜都内人以唾壺進思有以諫者后坐帷下倚檀几與語問四方事宜内人曰大家知女卑於男邪后曰知内人曰古有女媧亦不正是天子佐伏羲理九州耳後世孃姥冇越出房閤斷天下事皆不得其正多是輔昏主不然抱小兒獨大家改夫姓改去釵釧襲服冠冕符瑞日至大臣不敢動真天子也然今内之弄臣狎人朝夕進御者久未屏去妾疑此未當天意后曰何内人曰女隂也男陽也陽尊而隂卑雖大家以隂事主天然宜體取剛亢明烈以銷群陽陽銷然後隂得志也今狎弄日至處大家夫宫尊位其勢隂求陽也陽勝而隂亦微不可久也大家始今日能屛去男妾獨立天下則陽之剛亢明烈可有矣如是過萬萬世男子益削女子益專妾之願在此后雖不能盡用然即日下令誅作明堂者此盖文士寓言今從實録】送尸白馬寺焚之以造塔 甲子太后去慈氏越古之號【去羌呂翻】 三月丙辰鳳閣侍郎同平章事周允元薨 夏四月天樞成【天樞其制若柱】高一百五尺【高占犒翻】徑十二尺八面各徑五尺下為鐵山周百七十尺以銅為蟠龍麒麟縈繞之上為騰雲承露盤徑三丈四龍人立捧火珠高一丈工人毛婆羅造模武三思為文刻百官及四夷酋長名【高古犒翻酋慈由翻長知兩翻】太后自書其榜曰大周萬國頌德天樞 秋七月辛酉吐蕃寇臨洮【臨洮洮州洮土刀翻】以王孝傑為肅邉道行軍大摠管以討之 九月甲寅太后合祭天地於南郊加號天冊金輪大聖皇帝赦天下改元【改元天冊萬歲】 冬十月突厥默啜遣使請降【使疏吏翻降戶江翻】太后喜冊授左衛大將軍歸國公<br />
<br />
  萬歲通天元年【是年九月始改元】臘月甲戌太后發神都甲申封神嶽【后以嵩山為神嶽 考異曰統紀作壬午實録作甲申按去歲下制云臘月十六日有事於神嶽長歷是月甲戌朔壬午九日甲申十一日皆非十六日今從實録】赦天下改元萬歲登封天下百姓無出今年租税大酺九日【酺音蒲】丁亥禪於少室【戴延之曰嵩山三十六峯東曰太室西曰少室相去十七里嵩其總名也謂之室以其下各有石室焉少室高八百六十丈方十里與太室相埒但小耳】己丑御朝覲壇受賀【朝直遥翻】癸巳還宫甲午謁太廟 右千牛衛將軍安平王武攸緒少有志行恬澹寡欲扈從封中嶽還【少詩照翻行下孟翻從才用翻】即求弃官隱於嵩山之陽太后疑其詐許之以觀其所為攸緒遂優游巖壑冬居茅椒【茅椒編之為室性煖可以禦寒】夏居石室一如山林之士太后所賜及王公所遺野服器玩【遺才季翻】攸緒一皆置之不用塵埃凝積買田使奴耕種與民無異 【考異曰舊傳云聖歷中弃官隱嵩山今從實録】 春一月甲寅以婁師德為肅邉道行軍副總管撃吐蕃己巳以師德為左肅政大夫知政事如故 【考異曰實録云己巳秋官尚書婁師德為肅政御史大夫知政事如故舊傳曰萬歲登封元年轉左肅政御史大夫仍依舊知政事證聖元年吐蕃寇洮州令師德與夏官尚書王孝傑討之按證聖年號在登封前此傳尤為謬誤新傳云師德為河源積石懷遠軍及河蘭鄯廓州檢校營田大使入遷秋官尚書改左肅政御史大夫並知政事證聖中與王孝傑拒吐蕃於洮州今據寔録延載元年一月自宰相出為營田大使新書宰相表長夀二年師德平章事延載元年出軍營田大使萬歲通天元年一月甲寅師德為左肅政御史大夫肅邊道行軍總管統紀云秋官尚書知政事婁師德充副總管討吐蕃盖師德之出為營田大使不解宰相之職也今從實録新本紀】改長安崇尊廟為太廟【崇尊廟見上卷天授元年】 二月辛巳尊神嶽天中王為神嶽天中黄帝靈妃為天中黄后啟為齊聖皇帝封啟母神為玉京太后【夏后啟母石在嵩山】 三月壬寅王孝傑婁師德與吐蕃將論欽陵贊婆戰於素羅汗山【據婁師德傳素羅汗山在洮州界將即亮翻】 唐兵大敗孝傑坐免為庶人師德貶原州員外司馬 【考異曰新紀四月庚子貶師德而無免孝傑日新表三月壬寅孝傑免按實録三月壬寅撫州火下言孝傑敗盖皆據奏到之日耳二人同罪貶必同時不容隔月不知果在何日今但依實録因其軍敗終言貶官之事而已】師德因署移牒驚曰官爵盡無邪既而曰亦善亦善不復介意【復扶又翻】 丁巳新明堂成高二百九十四尺方三百尺規模率小於舊上施金塗鐵鳳高二丈【高古犒翻】後為大風所損更為銅火珠羣龍捧之【更工衡翻】號曰通天宫赦天下改元萬歲通天 大食請獻師子姚璹上疏以為師子專食肉遠道傳致【傳知戀翻】肉既難得極為勞費陛下鷹犬不畜漁獵悉停豈容菲薄於身而厚給於獸乃却之以檢校夏官侍郎孫元亨同平章事 夏五月壬子營州契丹松漠都督李盡忠歸誠州刺史孫萬榮舉兵反攻䧟營州【開元十道志曰舜築柳城即虞舜已前已有柳城之地因有營州之稱郡國志云當營室分故曰營州後漢末遼西烏丸蹋頓所居後魏於平州界置遼西郡周平齊猶為高寶寧所據隋討平寶寧始置營州松漠都督府及歸城州太宗以内屬契丹部落置】殺都督趙文翽【契欺訖翻又音喫翻呼會翻】盡忠萬榮之妹夫也皆居於營州城側文翽剛愎契丹饑不加賑給視酋長如奴僕故二人怨而反【愎弼力翻賑津忍翻酋慈由翻長知兩翻】乙丑遣左鷹揚衛將軍曹仁師右金吾衛大將軍張玄遇左威衛大將軍李多祚司農少卿麻仁節等二十八將討之【八將即亮翻】秋七月辛亥以春官尚書梁王武三思為榆關道安撫大使【榆關在勝州界與突厥接非所以備契丹也營州城西四百八十里有榆關守捉城所謂臨渝之險也榆當作渝史於此以後多以渝作榆讀者宜詳考使疏吏翻】姚璹副之以備契丹改李盡忠為李盡滅孫萬榮為孫萬斬【武后改突厥骨咄禄為不卒禄又改李盡忠為李盡滅孫萬榮為孫萬斬此事何異王莾所為顧有成敗之異耳】盡忠尋自稱無上可汗據營州【可從刋入聲汗音寒】以萬榮為前鋒略地所向皆下旬日兵至數萬進圍檀州【檀州本漢漁陽郡傂奚縣地舊置安州後周改為玄州隋開皇十六年置檀州】清邊前軍副總管張九節擊却之八月丁酉曹仁師張玄遇麻仁節與契丹戰於硤石谷【平州有西硤石東硤石二戍】唐兵大敗先是契丹破營州【先悉薦翻】獲唐俘數百囚之地牢聞唐兵將至使守牢霫紿之曰【使霫守唐俘於地牢故曰守牢霫霫而立翻紿蕩亥翻】吾輩家屬饑寒不能自存唯俟官軍至即降耳【降戶江翻下同】既而契丹引出其俘飼以糠粥慰勞之曰【飼祥吏翻勞力到翻】吾養汝則無食殺汝又不忍今縱汝去遂釋之俘至幽州具言其狀諸軍聞之爭欲先入至黄麞谷【據舊書黄麞谷在西硤石】虜又遣老弱迎降故遺老牛瘦馬於道側仁師等三軍弃步卒將騎兵先進契丹設伏横撃之飛索以䌈玄遇仁節生獲之【將即亮翻騎奇寄翻索昔各翻字書無䌈字今讀與榻同德盍翻或曰吐合翻】將卒死者填山谷鮮有脱者【鮮息淺翻】契丹得軍印詐為牒令玄遇等署之牒總管燕匪石宗懷昌等云官軍已破賊若至營州軍將皆斬兵不叙勲【燕因肩翻將即亮翻】匪石等得牒晝夜兼行不遑寢食以赴之士馬疲弊契丹伏兵於中道邀之全軍皆沒九月制天下繋囚及庶士家奴驍勇者官償其直發以撃契丹【驍堅堯翻】初令山東近邊諸州置武騎團兵以同州刺史建安王武攸宜為右武威衛大將軍充清邊道行軍大總管以討契丹右拾遺陳子昂為攸宜府參謀【以本官參謀軍事不列為品秩】上疏曰恩制免天下罪人及募諸色奴充兵討撃契丹此乃捷急之計非天子之兵且比來刑獄久清罪人全少【比毗至翻少詩沼翻】奴多怯弱不慣征行【慣古患翻】縱其募集未足可用况今天下忠臣義士萬分未用其一契丹小孽【孽魚列翻】假命待誅何勞免罪贖奴損國大體臣恐此策不可威示天下 丁巳突厥寇凉州執都督許欽明 【考異曰實録云吐蕃寇凉州都督許欽明為賊所殺按明年正月默啜寇霛州以欽明自随又默啜將襲孫萬榮殺欽明以祭天實録云吐蕃誤也】欽明紹之曾孫也【許紹預凌烟閣二十四功臣之列】時出按部突厥數萬奄至城下欽明拒戰為所虜欽明兄欽寂時為龍山軍討擊副使與契丹戰於崇州【龍山即慕容氏和龍之山也崇州奚州也武德五年分饒樂都督府之可汗部置貞觀三年徙治營州之廢陽師鎮】軍敗被擒虜將圍安東令欽寂說其屬城未下者【說輸芮翻】安東都護裴玄珪在城中【高宗總章元年置安東都護府於平壤城上元元年徙遼東郡故城儀鳳二年又徙新城開元二年徙平州天寶二年徙遼西故郡城疑此時已徙平州宋白曰營州東南二百七十里有保定軍舊安東都護府】欽寂謂曰狂賊天殃滅在朝夕公但勵兵謹守以全忠節虜殺之 吐蕃復遣使請和親【復扶又翻】太后遣右武衛胄曹參軍貴鄉郭元振往察其宜【胄曹參軍掌兵械公廨興善罸讁大朝會行從則受黄質甲鎧弓矢於衛尉】吐蕃將論欽陵請罷安西四鎮戍兵并求分十姓突厥之地【長夀元年置四鎮戍兵十姓突厥五咄陸五弩失畢也】元振曰四鎮十姓與吐蕃種類本殊【種章勇翻】今請罷唐兵豈非有兼并之志乎欽陵曰吐蕃苟貪土地欲為邊患則東侵甘凉豈肯規利於萬里之外邪乃遣使者隨元振入請之朝廷疑未决元振上疏【使疏吏翻朝直遥翻上時掌翻】以為欽陵求罷兵割地此乃利害之機誠不可輕舉措也今若直拒其善意則為邊患必深四鎮之利遠甘凉之害近不可不深圖也宜以計緩之使其和望未絶則善矣彼四鎮十姓吐蕃之所甚欲也而青海吐谷渾亦國家之要地也【吐從瞰入聲谷音浴】今報之宜曰四鎮十姓之地本無用於中國所以遣兵戍之欲以鎮撫西域分吐蕃之勢使不得併力東侵也今若果無東侵之志當歸我吐谷渾諸部及青海故地【吐谷渾地沒吐蕃見二百二卷高宗咸亨三年薛仁貴敗於大非州青海亦沒】則五俟斤部亦當以歸吐蕃【西突厥五弩失畢部各有酋長曰五俟斤俟渠之翻】如此則足以塞欽陵之口【塞悉則翻】而亦未與之絶也若欽陵小有乖違則曲在彼矣且四鎮十姓欵附日久今未察其情之向背事之利害【背蒲妹翻】遥割而弃之恐傷諸國之心非所以御四夷也太后從之 【考異曰御史臺記論欽陵必欲得四鎮及益州通市乃和親朝廷不許制書至河源納言婁師德患之曰制書到彼必入寇柰何監察御史南陽張彦先時按河源積石諸軍謂師德曰但稽制書虜必狐疑吾乃先為之備虜至必不捷矣師德從之欽陵入寇果無功由是得罪於其國按師德延載元年一月日同平章事充河源積石懷遠等軍營田大使萬歲通天元年一月為肅邊道行軍總管與王孝傑同撃吐蕃敗於素羅汗山尋貶原州司馬是歲吐蕃復求和欽陵請割四鎮之地神功元年正月師德復同平章事九月乃守納言御史臺記誤也】元振又上言吐蕃百姓疲於徭戍早願和親欽陵利於統兵專制獨不欲歸欵若國家歲發和親使【使疏吏翻】而欽陵常不從命則彼國之人怨欽陵日深望國恩日甚設欲大舉其徒固亦難矣斯亦離間之漸【間古莧翻】可使其上下猜阻禍亂内興矣太后深然之元振名震以字行 庚申以并州長史王方慶為鸞臺侍郎與殿中監萬年李道廣並同平章事 突厥默啜請為太后子并為其女求昏悉歸河西降戶帥其部衆為國討契丹【并為衆為並于偽翻帥讀曰率】太后遣豹韜衛大將軍閻知微【龍朔改左右屯衛為左右武威衛光宅又改為左右豹韜衛】左衛郎將攝司賓卿田歸道冊授默啜左衛大將軍遷善可汗知微立德之孫歸道仁會之子也【閻立德以巧思稱田仁會良吏也】冬十月辛卯契丹李盡忠卒孫萬榮代領其衆突厥默啜乘間襲松漠虜盡忠萬榮妻子而去【卒子恤翻間古莧翻】太后進拜默啜為頡跌利施大單干立功報國可汗【頡戶結翻跌徒結翻單音蟬】孫萬榮收合餘衆軍埶復振【復扶又翻】遣别帥駱務整何阿小為前鋒【帥所類翻阿烏葛翻】攻䧟冀州殺刺史陸寶積屠吏民數千人又攻瀛州河北震動制起彭澤令狄仁傑【長夀元年仁傑貶】為魏州刺史前刺史獨孤思莊畏契丹猝至悉驅百姓入城繕修守備仁傑至悉遣還農曰賊猶在遠何煩如是萬一賊來吾自當之百姓大悦時契丹入寇軍書填委夏官郎中硤石姚元崇剖析如流皆有條理【後魏太和十一年於崤陵置崤縣屬恒農郡隋并入熊耳縣屬河南郡唐武德元年復置貞觀十四年移治硤石塢因更名硤石】太后奇之擢為夏官侍郎 太后思徐有功用法平【長夀二年有功除名】擢拜左臺殿中侍御史聞者無不相賀鹿城主簿宗城潘好禮【鹿城漢安定侯國時縣西七里故城是也周齊為安定縣隋改為鹿城縣唐屬冀州唐制上縣主簿正九品下中下縣從九品上好呼到翻】著論稱有功蹈道依仁固守誠節不以貴賤死生易其操履設客問曰徐公於今誰與為比主人曰四海至廣人物至多或匿迹韜光僕不敢誣若所聞見則一人而已當於古人中求之客曰何如張釋之主人曰釋之所行者甚易徐公所行者甚難難易之間優劣見矣【易以䜴翻下不易同見賢遍翻】張公逢漢文之時天下無事至如盗高廟玉環及渭橋驚馬守法而已【事見十四卷漢文帝三年】豈不易哉徐公逢革命之秋屬惟新之運【屬之欲翻】唐朝遺老或包藏祸心使人主有疑【朝直遥翻】如周興來俊臣乃堯年之四凶也崇飾惡言以誣盛德而徐公守死善道深相明白幾䧟囹圄數挂網羅【幾居希翻囹盧丁翻圄魚巨翻數所角翻】此吾子所聞豈不難哉 【考異曰朝野僉載云時來俊臣羅織人罪皆先進狀敕依即奏籍沒徐冇功出死囚亦先進狀某人罪合免敕依然後斷雪冇功好出罪皆先奉進止非是自專盖時人見俊臣所誅有功所雪往往得其所欲疑以為先進狀耳若有功一一先奉進止何至三䧟死刑乎今不取】客曰使為司刑卿乃得展其才矣主人曰吾子徒見徐公用法平允謂可置司刑僕覩其人方寸之地何所不容若其用之何事不可豈直司刑而已哉<br />
<br />
  資治通鑑卷二百五  <br>
   </div> 

<script src="/search/ajaxskft.js"> </script>
 <div class="clear"></div>
<br>
<br>
 <!-- a.d-->

 <!--
<div class="info_share">
</div> 
-->
 <!--info_share--></div>   <!-- end info_content-->
  </div> <!-- end l-->

<div class="r">   <!--r-->



<div class="sidebar"  style="margin-bottom:2px;">

 
<div class="sidebar_title">工具类大全</div>
<div class="sidebar_info">
<strong><a href="http://www.guoxuedashi.com/lsditu/" target="_blank">历史地图</a></strong>  
<a href="http://www.880114.com/" target="_blank">英语宝典</a>  
<a href="http://www.guoxuedashi.com/13jing/" target="_blank">十三经检索</a> 
<br><strong><a href="http://www.guoxuedashi.com/gjtsjc/" target="_blank">古今图书集成</a></strong> 
<a href="http://www.guoxuedashi.com/duilian/" target="_blank">对联大全</a> <strong><a href="http://www.guoxuedashi.com/xiangxingzi/" target="_blank">象形文字典</a></strong> 

<br><a href="http://www.guoxuedashi.com/zixing/yanbian/">字形演变</a>  <strong><a href="http://www.guoxuemi.com/hafo/" target="_blank">哈佛燕京中文善本特藏</a></strong>
<br><strong><a href="http://www.guoxuedashi.com/csfz/" target="_blank">丛书&方志检索器</a></strong> <a href="http://www.guoxuedashi.com/yqjyy/" target="_blank">一切经音义</a>  

<br><strong><a href="http://www.guoxuedashi.com/jiapu/" target="_blank">家谱族谱查询</a></strong>  <strong><a href="http://shufa.guoxuedashi.com/sfzitie/" target="_blank">书法字帖欣赏</a></strong> 
<br>

</div>
</div>


<div class="sidebar" style="margin-bottom:0px;">

<font style="font-size:22px;line-height:32px">QQ交流群9:489193090</font>


<div class="sidebar_title">手机APP 扫描或点击</div>
<div class="sidebar_info">
<table>
<tr>
	<td width=160><a href="http://m.guoxuedashi.com/app/" target="_blank"><img src="/img/gxds-sj.png" width="140"  border="0" alt="国学大师手机版"></a></td>
	<td>
<a href="http://www.guoxuedashi.com/download/" target="_blank">app软件下载专区</a><br>
<a href="http://www.guoxuedashi.com/download/gxds.php" target="_blank">《国学大师》下载</a><br>
<a href="http://www.guoxuedashi.com/download/kxzd.php" target="_blank">《汉字宝典》下载</a><br>
<a href="http://www.guoxuedashi.com/download/scqbd.php" target="_blank">《诗词曲宝典》下载</a><br>
<a href="http://www.guoxuedashi.com/SiKuQuanShu/skqs.php" target="_blank">《四库全书》下载</a><br>
</td>
</tr>
</table>

</div>
</div>


<div class="sidebar2">
<center>


</center>
</div>

<div class="sidebar"  style="margin-bottom:2px;">
<div class="sidebar_title">网站使用教程</div>
<div class="sidebar_info">
<a href="http://www.guoxuedashi.com/help/gjsearch.php" target="_blank">如何在国学大师网下载古籍?</a><br>
<a href="http://www.guoxuedashi.com/zidian/bujian/bjjc.php" target="_blank">如何使用部件查字法快速查字?</a><br>
<a href="http://www.guoxuedashi.com/search/sjc.php" target="_blank">如何在指定的书籍中全文检索?</a><br>
<a href="http://www.guoxuedashi.com/search/skjc.php" target="_blank">如何找到一句话在《四库全书》哪一页?</a><br>
</div>
</div>


<div class="sidebar">
<div class="sidebar_title">热门书籍</div>
<div class="sidebar_info">
<a href="/so.php?sokey=%E8%B5%84%E6%B2%BB%E9%80%9A%E9%89%B4&kt=1">资治通鉴</a> <a href="/24shi/"><strong>二十四史</strong></a>&nbsp; <a href="/a2694/">野史</a>&nbsp; <a href="/SiKuQuanShu/"><strong>四库全书</strong></a>&nbsp;<a href="http://www.guoxuedashi.com/SiKuQuanShu/fanti/">繁体</a>
<br><a href="/so.php?sokey=%E7%BA%A2%E6%A5%BC%E6%A2%A6&kt=1">红楼梦</a> <a href="/a/1858x/">三国演义</a> <a href="/a/1038k/">水浒传</a> <a href="/a/1046t/">西游记</a> <a href="/a/1914o/">封神演义</a>
<br>
<a href="http://www.guoxuedashi.com/so.php?sokeygx=%E4%B8%87%E6%9C%89%E6%96%87%E5%BA%93&submit=&kt=1">万有文库</a> <a href="/a/780t/">古文观止</a> <a href="/a/1024l/">文心雕龙</a> <a href="/a/1704n/">全唐诗</a> <a href="/a/1705h/">全宋词</a>
<br><a href="http://www.guoxuedashi.com/so.php?sokeygx=%E7%99%BE%E8%A1%B2%E6%9C%AC%E4%BA%8C%E5%8D%81%E5%9B%9B%E5%8F%B2&submit=&kt=1"><strong>百衲本二十四史</strong></a>  <a href="http://www.guoxuedashi.com/so.php?sokeygx=%E5%8F%A4%E4%BB%8A%E5%9B%BE%E4%B9%A6%E9%9B%86%E6%88%90&submit=&kt=1"><strong>古今图书集成</strong></a>
<br>

<a href="http://www.guoxuedashi.com/so.php?sokeygx=%E4%B8%9B%E4%B9%A6%E9%9B%86%E6%88%90&submit=&kt=1">丛书集成</a> 
<a href="http://www.guoxuedashi.com/so.php?sokeygx=%E5%9B%9B%E9%83%A8%E4%B8%9B%E5%88%8A&submit=&kt=1"><strong>四部丛刊</strong></a>  
<a href="http://www.guoxuedashi.com/so.php?sokeygx=%E8%AF%B4%E6%96%87%E8%A7%A3%E5%AD%97&submit=&kt=1">說文解字</a> <a href="http://www.guoxuedashi.com/so.php?sokeygx=%E5%85%A8%E4%B8%8A%E5%8F%A4&submit=&kt=1">三国六朝文</a>
<br><a href="http://www.guoxuedashi.com/so.php?sokeytm=%E6%97%A5%E6%9C%AC%E5%86%85%E9%98%81%E6%96%87%E5%BA%93&submit=&kt=1"><strong>日本内阁文库</strong></a> <a href="http://www.guoxuedashi.com/so.php?sokeytm=%E5%9B%BD%E5%9B%BE%E6%96%B9%E5%BF%97%E5%90%88%E9%9B%86&ka=100&submit=">国图方志合集</a> <a href="http://www.guoxuedashi.com/so.php?sokeytm=%E5%90%84%E5%9C%B0%E6%96%B9%E5%BF%97&submit=&kt=1"><strong>各地方志</strong></a>

</div>
</div>


<div class="sidebar2">
<center>

</center>
</div>
<div class="sidebar greenbar">
<div class="sidebar_title green">四库全书</div>
<div class="sidebar_info">

《四库全书》是中国古代最大的丛书,编撰于乾隆年间,由纪昀等360多位高官、学者编撰,3800多人抄写,费时十三年编成。丛书分经、史、子、集四部,故名四库。共有3500多种书,7.9万卷,3.6万册,约8亿字,基本上囊括了古代所有图书,故称“全书”。<a href="http://www.guoxuedashi.com/SiKuQuanShu/">详细>>
</a>

</div> 
</div>

</div>  <!--end r-->

</div>
<!-- 内容区END --> 

<!-- 页脚开始 -->
<div class="shh">

</div>

<div class="w1180" style="margin-top:8px;">
<center><script src="http://www.guoxuedashi.com/img/plus.php?id=3"></script></center>
</div>
<div class="w1180 foot">
<a href="/b/thanks.php">特别致谢</a> | <a href="javascript:window.external.AddFavorite(document.location.href,document.title);">收藏本站</a> | <a href="#">欢迎投稿</a> | <a href="http://www.guoxuedashi.com/forum/">意见建议</a> | <a href="http://www.guoxuemi.com/">国学迷</a> | <a href="http://www.shuowen.net/">说文网</a><script language="javascript" type="text/javascript" src="https://js.users.51.la/17753172.js"></script><br />
  Copyright &copy; 国学大师 古典图书集成 All Rights Reserved.<br>
  
  <span style="font-size:14px">免责声明:本站非营利性站点,以方便网友为主,仅供学习研究。<br>内容由热心网友提供和网上收集,不保留版权。若侵犯了您的权益,来信即刪。scp168@qq.com</span>
  <br />
ICP证:<a href="http://www.beian.miit.gov.cn/" target="_blank">鲁ICP备19060063号</a></div>
<!-- 页脚END --> 
<script src="http://www.guoxuedashi.com/img/plus.php?id=22"></script>
<script src="http://www.guoxuedashi.com/img/tongji.js"></script>

</body>
</html>
