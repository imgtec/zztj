






























































資治通鑑卷二百八十  宋 司馬光 撰

胡三省 音註

後晉紀一|{
	柔兆君難一年 石氏自代北從晉王起太原既又以太原起事而得中原太原治晉陽契丹遂以晉命之故國號為晉}


高祖聖文章武明德孝皇帝上之上

|{
	諱敬瑭姓石氏其父臬捩雞本出於夷自朱邪歸唐從朱邪入居陰山其姓石不知其得姓之始五代會要曰晉既得天下祖衛大夫石碏}


天福元年|{
	是年十一月方改元即位}
春正月吳徐知誥始建大元帥府|{
	吳命徐知誥為大元帥見上卷上年冬十月}
以幕職分判吏戶禮兵刑工部及鹽鐵 丁未唐主立子重美為雍王|{
	雍於用翻}
癸丑唐主以千春節置酒|{
	唐主以生日為千春節五代會要曰帝以唐光啓元年正月十三日生既以晉元紀年故書潞王為唐主}
晉國長公主上壽畢辭歸晉陽|{
	上時掌翻}
帝醉曰何不且留遽歸欲與石郎反邪石敬瑭聞之益懼 三月丙午以翰林學士禮部侍郎馬胤孫為中書侍郎同平章事胤孫性謹懦中書事多凝滯又罕接賓客時人目為三不開謂口印門也 石敬瑭盡收其貨之在洛陽及諸道者歸晉陽託言以助軍費人皆知其有異志唐主夜與近臣從容語曰|{
	唐主好與近臣夜語見上卷上年從千容翻}
石郎於朕至親無可疑者但流言不釋萬一失歡何以解之皆不對端明殿學士給事中李崧退謂同僚呂琦曰|{
	李崧時與呂琦同入直}
吾輩受恩深厚豈得自同衆人一槩觀望邪計將安出琦曰河東若有異謀必結契丹為援契丹母以□華在中國屢求和親但求策稜等未獲故和未成耳|{
	□華契丹主安巴堅長子也來降見二百七十七卷明宗長興元年求策稜見三年契丹母謂舒嚕后也}
今誠歸策稜等與之和歲以禮幣約直十餘萬緡遺之|{
	遺唯季翻}
彼必驩然承命如此則河東雖欲陸梁無能為矣崧曰此吾志也然錢糓皆出三司宜更與張相謀之|{
	相息亮翻}
遂告張延朗延朗曰如學士計不惟可以制河東亦省邊費之什九|{
	言什省其九}
計無便於此者若主上聽從但責辦於老夫請於庫財之外捃拾以供之|{
	捃居運翻}
它夕二人密言於帝帝大喜稱其忠二人私草遺契丹書以俟命久之帝以其謀告樞密直學士薛文遇文遇對曰以天子之尊屈身奉夷狄不亦辱乎又虜若循故事求尚公主何以拒之|{
	唐自太宗以宗室女為公主下嫁諸蕃謂之和蕃公主其後囘紇有功於中國至屈帝女以女之}
因誦戎昱昭君詩曰安危託婦人帝意遂變|{
	戎昱唐人也能詩漢元帝以王昭君嫁匈奴後人憐之競為歌詩以言其事}
一日急召崧琦至後樓盛怒責之曰卿輩皆知古今欲佐人主致太平今乃為謀如是朕一女尚乳臭卿欲弃之沙漠邪且欲以養士之財輸之虜庭|{
	養士為養兵也言其欲割養兵之財以和蕃}
其意安在二人懼汗流浹背|{
	浹即恊翻}
曰臣等志在竭愚以報國非為虜計也|{
	為于偽翻}
願陛下察之拜謝無數帝詬責不已|{
	詬古候翻又許候翻}
呂琦氣竭拜少止帝曰呂琦強項骨視朕為人主耶琦曰臣等為謀不臧願陛下治其辠多拜何為|{
	治直之翻}
帝怒稍解止其拜各賜巵酒罷之|{
	罷使出就所舍}
自是羣臣不敢復言和親之策|{
	復扶又翻}
丁巳以琦為御史中丞蓋疎之也|{
	呂琦為唐主所親事始二百七十七卷明宗長興元年御史中丞居外朝不得入直禁中故曰疎}
吳徐知誥以其子副都統景通為太尉副元帥都統判官宋齊丘行軍司馬徐玠為元帥府左右司馬 閩主昶改元通文立賢妃李氏為皇后|{
	即李春鷰也}
尊皇太后曰太皇太后 靜江節度使同平章事馬希杲有善政監軍裴仁煦譛之於楚王希範|{
	煦吁句翻}
言其收衆心希範疑之夏四月漢將孫德威侵蒙桂二州|{
	蒙州本漢蒼梧郡之荔浦縣隋分荔浦置隨化縣唐武德四年改為立山於縣置荔州尋改為㳟州貞觀八年改為蒙州州東蒙山山下有蒙水人多性蒙故也宋熙寧五年廢蒙州以立山縣屬昭州}
希範命其弟武安節度副使希廣權知軍府事自將步騎五千如桂州希杲懼其母華夫人|{
	華戶化翻}
逆希範於全義嶺|{
	全義嶺在桂州全義縣即始安嶺也}
謝曰希杲為治無狀致寇戎入境|{
	治直吏翻}
煩殿下親涉險阻皆妾之罪也願削封邑洒掃掖庭以贖希杲辠|{
	灑所買翻又所賣翻掃素早翻又素報翻}
希範曰吾久不見希杲聞其治行尤異故來省之無它也|{
	治直吏翻行下孟翻省悉景翻無它言無它故也}
漢兵自蒙州引去徙希杲知朗州|{
	為希範殺希杲張本}
高從誨遣使奉牋於徐知誥勸即帝位|{
	高從誨以區區三州介居唐吳蜀之間利其賞賜所向稱臣諸國謂之高賴子其有以也夫}
初石敬瑭欲嘗唐主之意累表自陳羸疾|{
	羸倫為翻}
乞解兵柄移它鎮|{
	兵柄謂北面馬步軍都揔管之任}
帝與執政議從其請移鎮鄆州房暠李崧呂琦等皆力諫以為不可帝猶豫久之五月庚寅夜李崧請急在外|{
	請急請告也}
薛文遇獨直帝與之議河東事文遇曰諺有之當道築室三年不成兹事斷自聖志|{
	諺魚變翻斷丁亂翻}
羣臣各為身謀安肯盡言以臣觀之河東移亦反不移亦反在旦暮耳不若先事圖之|{
	先悉薦翻河東事情凡在清泰朝野之人誰不知者其所以重於言重於發懼言之則發大難之端在已而無以善其後耳清泰主鬱鬱于此久矣薛文遇一言當心遂決然而不顧}
先是術者言國家今年應得賢佐出奇謀定天下|{
	先悉薦翻}
帝意文遇當之聞其言大喜曰卿言殊豁吾意成敗吾決行之即為除目付學士院使草制|{
	御筆親除付外行者謂之除目其經宰相奏擬而行者亦謂之除目}
辛卯以敬瑭為天平節度使以馬軍都指揮使河陽節度使宋審䖍為河東節度使|{
	宋審䖍從唐主起于鳳翔故欲以之代敬瑭}
制出兩班聞呼敬瑭名相顧失色|{
	兩班謂文武官班}
甲午以建雄節度使張敬逹為西北蕃漢馬步都部署趣敬瑭之鄆州|{
	趣讀曰促天平節度治鄆州鄆音運}
敬瑭疑懼謀於將佐曰吾之再來河東也主上面許終身不代除|{
	唐主此言當在即位之初敬瑭入朝遣還鎮時也}
今忽有是命得非如今年千春節與公主所言乎我不興亂朝廷發之安能束手死於道路乎今且發表稱疾以觀其意若其寛我我當事之若加兵於我我則改圖耳|{
	觀敬瑭此言則求援於契丹者木心先定之計也桑維翰之言正會其意耳}
幕僚段希堯極言拒之敬瑭以其朴直不責也節度判官華陰趙瑩勸敬瑭赴鄆州觀察判官平遥薛融曰融書生不習軍旅都押牙劉知遠曰明公久將兵得士卒心今據形勝之地士馬精彊若稱兵傳檄|{
	稱舉也}
帝業可成奈何以一紙制書自投虎口乎掌書記洛陽桑維翰曰主上初即位明公入朝主上豈不知蛟龍不可縱之深淵邪|{
	古語有之魚不可脱于淵神龍失勢與蚯蚓同}
然卒以河東復授公|{
	卒子恤翻復扶又翻}
此乃天意假公以利器明宗遺愛在人主上以庶孽代之羣情不附公明宗之愛壻今主上以反逆見待此非首謝可免|{
	首式又翻}
但力為自全之計契丹素與明宗約為兄弟今部落近在雲應|{
	契丹牙帳自明宗長興三年屯納喇泊}
公誠能推心屈節事之萬一有急朝呼夕至何患無成敬瑭意遂決先是朝廷疑敬瑭|{
	先悉薦翻}
以羽林將軍寶鼎楊彦詢為北京副留守|{
	寶鼎縣屬河中府漢之汾陰縣也唐玄宗開元二十一年祀汾陰獲寶鼎由是更名九域志宋大中祥符四年改寶鼎為滎河縣在河中府北一百里}
敬瑭將舉事亦以情告之彦詢曰不知河東兵糧幾何能敵朝廷乎左右請殺彦詢敬瑭曰惟副使一人我自保之汝輩勿言也|{
	按薛史稱楊彦詢為人沈厚當以此得全}
戊戌昭儀節度使皇甫立奏敬瑭反|{
	并潞二鎮接境故知其事而先奏之}
敬瑭表帝養子不應承祀請傳位許王|{
	許王從益明宗之子也}
帝手裂其表抵地以詔荅之曰卿於卾王固非疎遠衛州之事天下皆知|{
	謂敬瑭盡殺閔帝從騎獨置帝干衛州也事見上卷清泰元年卾王即謂閔帝潞王入立以太后令降閔帝為卾王}
許王之言何人肯信壬寅制削奪敬瑭官爵乙巳以張敬逹兼太原四面排陳使|{
	陳讀曰陣下同}
河陽節度使張彦琦為馬步軍都指揮使以安國節度使安審琦為馬軍都指揮使以保義節度使相里金為步軍都指揮使以右監門上將軍武廷翰為壕寨使|{
	相息亮翻監古銜翻}
丙午以張敬逹為太原四面兵馬都部署以義武節度使楊光遠為副部署|{
	為楊光遠殺張敬逹降晉張本}
丁未又以張敬逹知太原行府事以前彰武節度使高行周為太原四面招撫排陳等使光遠既行定州軍亂牙將千乘方太討平之|{
	漢置千乘國後改樂安郡至隋廢樂安郡置千乘縣唐屬青州九域志千乘縣在青州北八十里乘繩證翻}
張敬逹將兵三萬營于晉安鄉|{
	晉安鄉在晉陽城南薛史晉安寨在晉祠南}
戊申敬逹奏西北先鋒馬軍都指揮使安審信叛奔晉陽審信金全之弟子也敬瑭與之有舊|{
	安氏羣從與石敬瑭本皆代北人}
先是雄義都指揮使馬邑安元信|{
	先悉薦翻馬邑縣屬朔州}
將所部六百餘人戍代州代州刺史張朗善遇之元信密說朗曰吾觀石令公長者|{
	說式芮翻石敬瑭加中書令故稱為令公長知兩翻}
舉事必成公何不潜遣人通意可以自全朗不從由是互相猜忌元信謀殺朗不克帥其衆奔審信審信遂帥麾下數百騎與元信掠百井奔晉陽|{
	帥讀曰率}
敬瑭謂元信曰汝見何利害捨彊而歸弱對曰元信非知星識氣顧以人事決之耳夫帝王所以御天下莫重於信今主上失大信於令公親而貴者且不自保|{
	石敬瑭身為帝壻可謂親矣官為中書令建節縂兵專制北面可謂貴矣}
况疎賤乎其亡可翹足而待何彊之有敬瑭悦委以軍事振武西北巡檢使安重榮戍代北|{
	歐史安重榮為振武巡邊指揮使}
帥步騎五百奔晉陽|{
	帥讀曰率下同}
重榮朔州人也以宋審䖍為寧國節度充侍衛馬軍都指揮使|{
	石敬瑭既不受代故使宋審䖍領節掌宿衛審䖍唐主振鳳翔時牙將}
天雄節度使劉延皓恃后族之勢驕縱|{
	劉延皓唐主后弟}
奪人財產減將士給賜宴飲無度捧聖都虞候張令昭因衆心怨怒謀以魏博應河東癸丑未明帥衆攻牙城克之延皓脱身走亂兵大掠令昭奏延皓失於撫御以致軍亂臣以撫安士卒權領軍府|{
	臣以之以當作已}
乞賜旌節延皓至洛陽唐主怒命遠貶皇后為之請|{
	為于偽翻 考異曰廢帝實録延皓皇后之姪按薛史唐餘録歐陽史皆云延皓后之弟應州人也延朗宋州虞城人也獨廢帝實録云后姪今不取}
六月庚申止削延皓官爵歸私第 辛酉吳太保同平章事徐景遷以疾罷以其弟景遂代為門下侍郎參政事 癸亥唐主以張令昭為右千牛衛將軍權知天雄軍府事令昭以調發未集|{
	調徒釣翻}
且受新命尋有詔徙齊州防禦使令昭託以十卒所留實俟河東之成敗唐主遣使諭之令昭殺使者甲戌以宣武節度使兼中書令范延光為天雄四面行營招討使知魏博行府事|{
	魏博恐當作魏州}
以張敬逹充太原四面招討使以楊光遠為副使丙子以西京留守李周為天雄軍四面行營副招討使 石敬瑭之子右衛上將軍重殷皇城副使重裔聞敬瑭舉兵匿於民間井中弟沂州都指揮使敬德殺其妻女而逃尋捕得死獄中從弟彰聖都指揮使敬威自殺秋七月戊子獲重殷重裔誅之|{
	重直龍翻從才用翻 考異曰薛史七月己丑誅右衛上將軍石重英皇城副使石重裔皆敬瑭之子也廢帝實録云石諱妷男尚食使重又供奉官重英與薛史不同按重乂敬瑭子即位後為張從賓所殺實録誤也廣本英作殷今從之}
并族所匿之家 庚寅楚王希範自桂州北還|{
	四月至桂州七月方還還從宣翻又如字}
雲州步軍指揮使桑遷奏應州節度使尹暉逐雲州節度使沙彦珣收其兵應河東丁酉彦珣表遷謀叛應河東引兵圍子城彦珣犯圍走出西山據雷公口明日收兵入城擊亂兵遷敗走軍城復安是日尹暉執遷送洛陽斬之 丁未范延光拔魏州斬張令昭詔悉誅其黨七指揮 張敬逹發懷州彰聖軍戍虎北口|{
	虎北口在汾水北彰聖軍本洛城屯衛兵也先是分屯懷州又自懷州發赴張敬逹軍前敬逹又發之戍虎北口}
其指揮使張萬迪將五百騎奔河東丙辰詔盡誅其家 石敬瑭遣間使求救於契丹|{
	間古莧翻使疏吏翻時張敬逹在代州雲應兩鎮亦不從敬瑭故遣使從間道趨契丹帳}
令桑維翰草表稱臣於契丹主且請以父禮事之約事捷之日割盧龍一道及鴈門關以北諸州與之劉知遠諫曰稱臣可矣以父事之太過厚以金帛賂之自足致其兵不必許以土田恐異日大為中國之患悔之無及敬瑭不從|{
	他日卒如劉知遠之言為契丹入中國張本}
表至契丹契丹主大喜|{
	喜中國有釁之可乘也}
白其母曰兒此夢石郎遣使來|{
	其母即舒嚕太后比毗至翻近也}
今果然此天意也|{
	自是之後遼滅晉金破宋闕}


|{
	今之疆理西越益寧南盡交廣至于海外皆石敬瑭捐割關隘以啓之也其果天意乎}
乃為

復書許俟仲秋傾國赴援|{
	俟秋高馬肥而後進}
八月己未以范延光為天雄節度使李周為宣武節度使同平章事癸亥應州言契丹三千騎攻城 張敬逹築長圍以攻晉陽石敬瑭以劉知遠為馬步都指揮使安重榮張萬迪降兵皆隸焉知遠用法無私撫之如一由是人無貳心敬瑭親乘城坐卧矢石下知遠曰觀敬逹輩高壘深塹欲為持久之計無它奇策不足慮也願明公四出間使|{
	間古莧翻使疏吏翻}
經畧外事守城至易知遠獨能辦之|{
	易以豉翻用兵之計攻城最下以敬瑭知遠之守又有契丹之援而敬逹欲以持久制之宜其敗也}
敬瑭執知遠手撫其背而賞之 戊寅以成德節度使董温琪為東北面副招討使以佐盧龍節度使趙德鈞 唐主使端明殿學士呂琦至河東行營犒軍|{
	犒苦到翻}
楊光遠謂琦曰願附奏陛下幸寛宵旰|{
	旰古按翻}
賊若無援旦夕當平若引契丹當縱之令入可一戰破也|{
	楊光遠之計狃王晏球定州之勝欲縱之令入而與之戰殊不知戰無常勝而關隘不可不扼也尋而契丹徑入唐兵一戰而敗遂為所困矣}
帝甚悦帝聞契丹許石敬瑭以仲秋赴援屢督張敬逹急攻晉陽不能下每有營構多值風雨長圍復為水潦所壞竟不能合|{
	復扶又翻壞音怪史言天方相晉張敬逹無所施其力}
晉陽城中日窘粮儲浸乏|{
	若契丹之援不至晉不能支矣}
九月契丹主將五萬騎號三十萬自揚武谷而南|{
	楊武谷在代州崞縣薛史陽武谷在朔州南 考異曰代州今有楊武寨其北有長城嶺聖佛谷今從漢高祖實録作楊武}
旌旗不絶五十餘里代州刺史張朗忻州刺史丁審琦嬰城自守|{
	九域志代州南至忻州一百六十里忻州南至太原一百四十里}
虜騎過城下亦不誘脅|{
	誘音酉}
審琦洺州人也辛丑契丹主至晉陽陳於汾北之虎北口|{
	陳讀曰陣下同 考異曰按幽州北山口名虎北口亦名古北口此在太原而云陳於虎北口又云歸虎北口蓋太原城側别有地名虎北口也}
先遣人謂敬瑭曰吾欲今日即破賊可乎敬瑭遣人馳告曰南軍甚厚不可輕|{
	唐兵自南來攻晉陽故謂之南軍}
請俟明日議戰未晩也使者未至契丹已與唐騎將高行周苻彦卿合戰敬瑭乃遣劉知遠出兵助之張敬逹楊光遠安審琦以步兵陳於城西北山下契丹遣輕騎三千不被甲直犯其陳唐兵見其羸爭逐之至汾曲|{
	被皮義翻羸偷為翻汾曲汾水之曲也}
契丹涉水而去唐兵循㟁而進契丹伏兵自東北起衝唐兵斷而為二步兵在北者多為契丹所殺騎兵在南者引歸晉安寨契丹縱兵乘之唐兵大敗步兵死者近萬人|{
	近其靳翻}
騎兵獨全敬逹等收餘衆保晉安契丹亦引兵歸虎北口敬瑭得唐降兵千餘人劉知遠勸敬瑭盡殺之|{
	唐兵雖敗其衆尚彊劉知遠懼降兵復叛歸故勸殺之}
是夕敬瑭出北門|{
	出晉陽城北門也}
見契丹主契丹主執敬瑭手恨相見之晩|{
	以前此未識面故然亦必石敬瑭之氣貌有以聳其瞻視也}
敬瑭問曰皇帝遠來士馬疲倦遽與唐戰而大勝何也契丹主曰始吾自北來謂唐必斷雁門諸路|{
	斷音短鴈門有東陘西陘之險嵉縣有陽武石門之隘}
伏兵險要則吾不可得進矣|{
	使張敬逹等果知出此豈有晉安之困哉}
使人偵視皆無之|{
	偵丑鄭翻}
吾是以長驅深入知大事必濟也兵既相接我氣方鋭彼氣方沮若不乘此急擊之|{
	言當乘初至之鋭而用其鋒也}
曠日持久則勝負未可知矣此吾所以亟戰而勝不可以勞逸常理論也敬瑭甚歎伏壬寅敬瑭引兵會契丹圍晉安寨置營於晉安之南長百餘里厚五十里多設鈴索吠犬人跬步不能過|{
	長直亮翻厚戶茂翻索昔名翻吠房廢翻跬犬橤翻半步也又司馬法曰一舉足曰跬跬三尺也}
敬逹等士卒猶五萬人馬萬匹四顧無所之|{
	兵法置之死地而後生若張敬逹等能于圍落未合之時勉諭將士竭力致死决戰勝負未可知也}
甲辰敬逹遣使告敗於唐自是聲問不復通|{
	復扶又翻}
唐主大愳遣彰聖都指揮使苻彦饒將洛陽步騎兵屯河陽詔天雄節度使兼中書令范延光將魏州兵二萬由青山趣榆次|{
	青山即邢州青山口也趣七喻翻}
盧龍節度使東北面招討使兼中書令北平王趙德鈞將幽州兵出契丹軍後|{
	欲使趙德鈞自飛狐道出代州以斷契丹之後}
耀州防禦使潘環糺合西路戍兵|{
	糺與糾同說文繩三合為糺故凡合集兵衆者謂之糺合糺集西路戍兵謂蒲潼以西諸道戍兵也}
由晉絳兩乳嶺出慈隰共救晉安寨契丹主移帳於柳林|{
	柳林當在晉安寨南}
遊騎過石會關不見唐兵丁未唐主下詔親征雍王重美曰|{
	雍於用翻}
陛下目疾未平未可遠涉風沙臣雖童稚願代陛下北行帝意本不欲行聞之頗悦張延朗劉延皓及宣徽南院使劉延朗皆勸帝行帝不得已戊申發洛陽謂盧文紀曰朕雅聞卿有相業故排衆議首用卿|{
	相息亮翻文紀唐主清泰元年四月即位七月相盧文紀}
今禍難如此|{
	難乃旦翻}
卿嘉謀皆安在乎文紀但拜謝不能對己酉遣劉延朗監侍衛步軍都指揮使苻彦饒軍赴潞州為大軍後援|{
	大軍謂晉安寨之軍監古銜翻}
諸軍自鳳翔推戴以來|{
	推戴事見上卷清泰元年}
驕悍不為用彦饒恐其為亂不敢束之以法|{
	悍下罕翻又侯旰翻兵驕而不為用與無兵同潞王以驕兵推戴而得天下亦以驕兵不為用而失天下固其宜也}
帝至河陽心憚北行召宰相樞密使議進取方略盧文紀希帝旨言國家根本大半在河南胡兵倏來忽往不能久留晉安大寨甚固况己發三道兵救之|{
	謂范延光趙德鈞潘環三帥之兵}
河陽天下津要|{
	北兵犯洛須自河陽渡河故云然}
車駕宜留此鎮撫南北且遣近臣往督戰苟不能解圍進亦未晩張延朗欲因事令趙延壽得解樞務|{
	趙延壽時為樞密使欲求解而未能}
因曰文紀言是也帝訪於餘人無敢異言者澤州刺史劉遂凝鄩之子也潜自通於石敬瑭|{
	應順初劉遂雍以長安拒王思同而迎潞王者亦劉鄩之子也是其兄弟隨時反復以求禄利白晝攫金見金而不見人者也}
表稱車駕不可踰太行|{
	行戶剛翻澤州當太行之道}
帝議近臣可使北行者張延朗與翰林學士須昌和凝等|{
	須昌即九域志鄆州所治之須城縣蓋後唐避李國昌諱改須昌為須城而歐史與通鑑則仍舊縣名而不改也}
皆曰趙延壽父德鈞以盧龍兵來赴難|{
	難乃旦翻}
宜遣延壽會之庚戌遣樞密使忠武節度使隨駕諸軍都部署兼侍中趙延壽將兵二萬如潞州辛亥帝如懷州以右神武統軍康思立為北面行營馬軍都指揮使帥扈從騎兵赴團栢谷|{
	帥讀曰率從才用翻九域志太原府祁縣有團栢谷}
思立晉陽胡人也帝以晉安為憂問策於羣臣吏部侍郎永清龍敏請立李贊華為契丹主|{
	唐如意元年分安次縣置武隆縣景雲元年曰會昌天寶元年改曰永清屬幽州匈奴須知永清縣在幽州東南一百七十里舜以龍為納言子孫以名為氏又或以為豢龍氏之後項羽將有龍且漢有龍伯高李贊華契丹主之兄也明宗長興元年來降賜姓名時在洛陽}
令天雄盧龍二鎮分兵送之|{
	欲今范延光趙德鈞分兵送之}
自幽州趣西樓朝廷露檄言之契丹主必有内顧之憂|{
	露檄者欲使契丹知之觀他日契丹舒嚕太后責趙德鈞之言則龍敏之策為可行唐主悉不用耳}
然後選募軍中精鋭以擊之此亦解圍之一策也帝深以為然而執政恐其無成議竟不決帝憂沮形於神色但日夕酣飲悲歌羣臣或勸其北行則曰卿勿言石郎使我心膽墮地|{
	李嗣源舉兵向洛則莊宗為之神色沮喪石敬瑭阻兵拒命則潞王自謂使之心膽墮地何平時之臨敵甚勇一旦乃惴怯如此也蓋莊宗之與明宗潞王之與晉祖皆同出入兵間内揆其智力無以大相過而乘時用勢偶有不相及者則其氣先餒故也}
冬十月壬戌詔大括天下將吏及民間馬|{
	將即亮翻}
又發民為兵每七戶出征夫一人|{
	考異曰薛史云十戶今從廢帝實録}
自備鎧仗謂之義軍期以十一月俱集命陳州刺史郎萬金教以戰陳|{
	郎萬金當時勇將也}
用張延朗之謀也凡得馬二千餘匹征夫五千人實無益於用而民間大擾 初趙德鈞陰蓄異志欲因亂取中原|{
	趙德鈞之志圖非望亦見潞王得之之易也}
自請救晉安寨唐主命自飛狐踵契丹後鈔其部落|{
	鈔楚交翻}
德鈞請將銀鞍契丹直三千騎|{
	趙德鈞在幽州以契丹來降之驍勇者置銀鞍契丹直}
由土門路西入帝許之趙州刺史北面行營都指揮使劉在明先將兵戍易州德鈞過易州命在明以其衆自隨在明幽州人也德鈞至鎮州以董温琪領招討副使邀與偕行|{
	董温琪時鎮鎮州}
又表稱兵少須合澤潞兵乃自吳兒谷趣潞州|{
	吳兒谷在潞州黎城東北涉縣西南}
癸酉至亂柳時范延光受詔將部兵二萬屯遼州德鈞又請與魏博軍合延光知德鈞合諸軍志趣難測表稱魏博兵已入賊境無容南行數百里與德鈞合乃止 漢主以宗正卿兼工部侍郎劉濬為中書侍郎同平章事濬崇望之子也|{
	劉崇望相昭宗}
十一月以趙德鈞為諸道行營都統依前東北面行

營招討使以趙延壽為河東道南面行營招討使以翰林學士張礪為判官庚寅以范延光為河東道東南面行營招討使以宣武節度使同平章事李周副之辛卯以劉延朗為河東道南面行營招討副使趙延壽遇趙德鈞於西湯|{
	歐史西湯作西唐薛史作西唐店}
悉以兵屬德鈞唐主遣呂琦賜德鈞敕告且犒軍|{
	賜以諸道行營都統勅告也犒苦到翻}
德鈞志在併范延光軍逗留不進詔書屢趣之|{
	趣讀曰促}
德鈞乃引兵北屯團柏谷口 癸巳吳主詔齊王知誥置百官以金陵府為西都 前坊州刺史劉景巖延州人也多財而喜俠|{
	喜許記翻}
交結豪傑家有丁夫兵仗人服其彊埶傾州縣彰武節度使楊漢章無政失夷夏心會括馬及義軍漢章帥步騎數千人將赴軍期|{
	夏戶雅翻帥讀曰率}
閲之於野景巖潜使人撓之曰契丹彊盛汝曹有去無歸衆懼殺漢章奉景巖為留後唐主不獲已丁酉以景巖為彰武留後|{
	撓呼高翻撓亂之也史言徵發過甚強人以其所不堪適足為州里姦豪之資}
契丹主謂石敬瑭曰吾三千里赴難|{
	難乃旦翻}
必有成功觀汝器貌識量真中原之主也|{
	契丹主初來赴難石敬瑭出見之於晉陽北門此時固得之眉睫間矣及圍晉安軍中旦暮見審之既熟然後發此言然味其言不徒取其氣貌又取其識量則其所謂觀者必有異乎常人之觀矣}
吾欲立汝為天子敬瑭辭讓者數四將吏復勸進乃許之|{
	復扶又翻}
契丹主作冊書命敬瑭為大晉皇帝自解衣冠授之|{
	石敬瑭蓋以北服即位}
築壇於柳林是日即皇帝位 |{
	考異曰廢帝實録閏月丁卯胡立石諱為天子於柳林誤也今從晉高祖實録薛史契丹冊文}
割幽薊瀛莫涿檀順新媯儒武雲應寰朔蔚十六州以與契丹|{
	儒志領晉山一縣武州領文德一縣武州唐志有之儒州蓋晉王鎮河東所表置後唐明宗天成元年以興唐軍置寰州領寰清一縣隸應州彰國節度人皆以石晉割十六州為北方自撤藩籬之始余謂鴈門以北諸州弃之猶有關隘可守漢建安喪亂弃陘北之地不害為魏晉之彊是也若割燕薊順等州則為失地險然盧龍之險在營平二州界自劉守光僭竊周德威攻取契丹乘間遂據營平自同光以來契丹南牧直抵涿易其失險也久矣薊音計媯居為翻蔚紆勿翻}
仍許歲輸帛三十萬匹己亥制改長興七年為天福元年|{
	此清泰元年也而以為唐明宗長興七年以潞王為簒也}
大赦敕命法制皆遵明宗之舊以節度判官趙瑩為翰林學士承旨戶部侍郎知河東軍府事掌書記桑維翰為翰林學士禮部侍郎權知樞密使事觀察判官薛融為侍御史知雜事節度推官白水竇貞固為翰林學士|{
	白水縣屬同州宋白曰白水縣漢栗邑又為漢衙縣春秋彭衙地後魏和平三年分澄城置白水縣南臨白水因名九域志在州西北一百二十里}
軍城都巡檢使劉知遠為侍衛馬軍都指揮使|{
	軍城謂河東軍城晉陽受圍之時劉知遠為都巡檢使}
客將景延廣為步軍都指揮使延廣陜州人也|{
	陜失冉翻}
立晉國長公主為皇后契丹主雖軍柳林其輜重老弱皆在虎北口每日暝輒結束以備倉猝遁逃|{
	重直用翻暝莫定翻觀契丹在虎北口其所以自為備者與夫詐趙德鈞之事其畏中國之心為何如哉}
而趙德鈞欲倚契丹取中國至團栢踰月按兵不戰去晉安纔百里聲問不能相通德鈞累表為延壽求成德節度使|{
	為于偽翻}
曰臣今遠征幽州埶孤欲使延壽在鎮州左右便於應接|{
	言延壽在常山則左可以應接薊門右可以應接團栢}
唐主曰延壽方擊賊何暇往鎮州俟賊平當如所請德鈞求之不已唐主怒曰趙氏父子堅欲得鎮州何意也苟能却胡寇雖欲代吾位吾亦甘心若玩寇邀君但恐大兎俱斃耳|{
	戰國策曰韓子盧者天下之駿犬也東郭㕙者天下之狡兎也盧逐㕙環山者三騰山者五兎死于前犬廢于後田父見而并獲之}
德鈞聞之不悦閏月趙延壽獻契丹主所賜詔及甲馬弓劒詐云德鈞遣使致書於契丹主為唐結好說令引兵歸國|{
	使疏吏翻為于偽翻好呼到翻說式芮翻}
其實别為密書厚以金帛賂契丹主云若立已為帝請即以見兵南平洛陽|{
	見兵謂其父子見統之兵也見賢遍翻}
與契丹為兄弟之國仍許石氏常鎮河東契丹自以深入敵境晉安未下德鈞兵尚彊范延光在其東又恐山北諸州邀其歸路|{
	山北諸州謂雲應寰朔等州}
欲許德鈞之請帝聞之大懼亟使桑維翰見契丹主說之曰大國舉義兵以救孤危一戰而唐兵瓦解退守一柵食盡力窮趙北平父子不忠不信|{
	趙德鈞封北平王故稱之言其不忠於唐不信於契丹也}
畏大國之彊且素蓄異志按兵觀變非以死徇國之人何足可畏而信其誕妄之辭貪豪末之利|{
	秋豪之末言其細也}
弃垂成之功乎且使晉得天下將竭中國之財以奉大國豈此小利之比乎契丹主曰爾見捕鼠者乎不備之猶或齧傷其手况大敵乎|{
	齧魚結翻}
對曰今大國已扼其喉安能齧人乎契丹主曰吾非有渝前約也|{
	渝變也前約謂使晉帝中國}
但兵家權謀不得不爾對曰皇帝以信義救人之急四海之人俱屬耳目|{
	屬之欲翻}
奈何二三其命|{
	左傳晉侯使韓穿來言汶陽之田歸之于齊季文子曰一年之間或予或奪二三孰甚焉}
使大義不終臣竊為皇帝不取也|{
	為于偽翻}
跪於帳前自旦至暮涕泣争之契丹主乃從之指帳前石謂德鈞使者曰我已許石郎此石爛可改矣 龍敏謂前鄭州防禦使李懿曰君國之近親今社稷之危翹足可待君獨無憂乎懿為言趙德鈞必能破敵之狀|{
	為于偽翻}
敏曰我燕人也|{
	龍敏幽州永清縣人}
知德鈞之為人怯而無謀但於守城差長耳况今内蓄姦謀豈可恃乎僕有狂策但恐朝廷不肯為耳今從駕兵尚萬餘人馬近五千匹|{
	近其靳翻}
若選精騎一千使僕與郎萬金將之自介休山路夜冒虜騎入晉安寨|{
	郎萬金當時勇將自介休山路逹平遥則可得而至晉安寨將即亮翻冒莫北翻}
但使其半得入則事濟矣張敬逹等陷於重圍|{
	重直龍翻}
不知朝廷聲問若知大軍近在團栢雖有鐵障可衝陷况虜騎乎懿以白唐主唐主曰龍敏之志極壮用之晩矣|{
	龍敏之策非不可行也其如兵驕而不可用何唐主老於行間蓋亦有見於此}
丹州義軍作亂逐刺史康承詢承詢奔鄜州|{
	九域志丹州西至鄜州一百七十五里鄜芳蕪翻}
晉安寨被圍數月|{
	是年九月晉安寨被圍被皮義翻}
高行周苻彦鄉數引騎兵出戰|{
	數所角翻}
衆寡不敵皆無功芻粮俱竭削柹淘糞以飼馬馬相啗尾鬛皆秃|{
	柹方肺翻斫木札也木札已薄更削之使薄使馬可啗淘糞者淘馬糞中草䈥復以飼馬飼祥吏翻啗徒濫翻秃他谷翻}
死則將士分食之援兵竟不至張敬逹性剛時謂之張生鐵|{
	歐史曰張敬逹小字生鐵}
楊光遠安審琦勸敬逹降於契丹敬逹曰吾受明宗及今上厚恩|{
	歐史張敬逹明宗時為河東馬步軍都指揮使領欽州刺史屢遷彰國大同節度使徙鎮武信晉昌故敬逹自謂受厚恩也然明宗置武信軍於遂州尋為孟知祥所陷張敬逹未嘗往鎮晉得中國始改長安為晉昌軍歐亦考之未詳也通鑑前書敬逹自建雄節度代敬瑭建雄軍晉州也歐史誤以為晉昌耳又不知武信緣何而誤降戶江翻}
為元帥而敗軍其罪已大况降敵乎今援兵旦暮至且當俟之必若力盡埶窮則諸軍斬我首|{
	軍當作君}
攜之出降自求多福未為晩也|{
	史言張敬逹之志節}
光遠目審琦欲殺敬逹審琦未忍高行周知光遠欲圖敬逹常引壮騎尾而衛之敬逹不知其故謂人曰行周每踵余後何意也行周乃不敢隨之諸將每旦集於招討使營甲子高行周苻彦卿未至光遠乘其無備斬敬逹首帥諸將上表降於契丹|{
	帥讀曰率}
契丹主素聞諸將名皆慰勞|{
	勞力到翻下詔勞同}
賜以裘帽因戲之曰汝輩亦大惡漢|{
	北人謂南人為漢大惡猶今人謂桀烈者為得人憎也王昭遠所謂惡小兒亦此意}
不用鹽酪啗戰馬萬匹光遠等大慙契丹主嘉張敬逹之忠命收葬而祭之謂其下及晉諸將曰汝曹為人臣當效敬逹也時晉安寨馬猶近五千|{
	近其靳翻}
鎧仗五萬契丹悉取以歸其國悉以唐之將卒授帝語之曰勉事而主|{
	語牛倨翻而汝也}
馬軍都指揮使康思立憤惋而死|{
	惋烏貫翻}
帝以晉安已降遣使諭諸州代州刺史張朗斬其使呂琦奉唐主詔勞北軍|{
	北軍謂鴈門以北諸州固守之軍}
至忻州遇晉使亦斬之謂刺史丁審琦曰虜過城下而不顧其心可見還日必無全理不若早帥兵民自五臺犇鎮州|{
	自五臺縣東南至鎮州三百六十里即取飛狐路也帥讀曰率下同}
將行審琦悔之閉牙城不從州兵欲攻之琦曰家國如此何為復相屠滅|{
	復扶又翻}
乃帥州兵趣鎮州|{
	州兵忻州兵也趣七喻翻}
審琦遂降契丹 契丹主謂帝曰桑維翰盡忠於汝宜以為相丙寅以趙瑩為門下侍郎桑維翰為中書侍郎並同平章事維翰仍權知樞密使事以楊光遠為侍衛馬步軍都指揮使|{
	以楊光遠殺張敬逹以晉安寨降故擢用之}
以劉知遠為保義節度使侍衛馬步軍都虞候 帝與契丹主將引兵而南欲留一子守河東咨於契丹主|{
	謀事為咨今北人以咨為重自行臺行省移文書於内臺内省率謂之咨}
契丹主令帝盡出諸子自擇之帝兄子重貴父敬儒早卒帝養以為子貌類帝而短小契丹主指之曰此大目者可也乃以重貴為北京留守|{
	契丹主知重貴之可異日景延廣果立之然所謂可者言於帝諸子中為可耳契丹主固窺之矣}
太原尹河東節度使|{
	以留守為尹為帥循唐之舊制也}
契丹以其將高謨翰為前鋒與降卒皆進|{
	降卒唐晉安寨之兵也}
丁卯至團栢與唐兵戰趙德鈞趙延壽先遁苻彦饒張彦琦劉延朗劉在明繼之士卒大潰相騰踐死者萬計己巳延朗在明至懷州唐主始知帝即位楊光遠降衆議以天雄軍府尚完契丹必憚山東未敢南下|{
	天雄軍在太行山之東}
車駕宜幸魏州唐主以李崧素與范延光善|{
	時范延光鎮魏州}
召崧謀之薛文遇不知而繼至|{
	李崧薛文遇同在直文遇不知獨召崧以為並召也故繼崧而至}
唐主怒變色崧躡文遇足|{
	躡尼輒翻}
文遇乃去唐主曰我見此物肉顫|{
	顫之賤翻肉寒動為顫}
適幾欲抽佩刀刺之|{
	幾居希翻刺七亦翻}
崧曰文遇小人淺謀誤國刺之益醜|{
	唐主得薛文遇於起事之初及即位使之豫謀議沮李崧等和契丹之計及贊唐主移鎮天平皆文遇為之也今事敗而歸咎焉}
崧因勸唐主南還|{
	還從宣翻又如字}
唐主從之洛陽聞北軍敗|{
	北軍謂趙德鈞苻彦饒等屯團栢之兵}
衆心大震居人四出逃竄山谷門者請禁之|{
	門者洛城守關者也}
河南尹雍王重美曰國家多難|{
	難乃旦翻}
未能為百姓主又禁其求生徒增惡名耳不若聽其自便事寧自還乃出令任從所適衆心差安|{
	還從宣翻重美之識度蓋亦異乎庸常卒之父子俱死自古以來負才識而不得展以死於多難者多矣}
壬申唐主還至河陽命諸將分守南北城|{
	河陽有南北中潬三城守南北城所以衛河橋}
張延朗請幸滑州庶與魏博聲勢相接唐主不能決趙德鈞趙延壽南奔潞州唐敗兵稍稍從之其將時賽帥盧龍輕騎東還漁陽|{
	賽先代翻帥讀曰率漁陽即謂幽州唐人多言之安禄山反於幽州南向京輔白居易歌之以為漁陽鼙鼓動地來是也}
帝先遣昭義節度使高行周還具食|{
	使還潞州先供頓以待軍}
至城下見德鈞父子在城上行周曰僕與大王鄉曲|{
	趙德鈞封北平王故高行周稱之為大王德鈞幽州人行周媯州人皆燕人也故云鄉曲}
敢不忠告城中無斗粟可守不若速迎車駕甲戌帝與契丹主至潞州德鈞父子迎謁於高河契丹主慰諭之父子拜帝於馬首進曰别後安否帝不顧亦不與之言|{
	以其欲爭為帝恨之也}
契丹主問德鈞曰汝在幽州所置銀鞍契丹直何在德鈞指示之契丹主命盡殺之於西郊|{
	潞州西郊也}
凡三千人遂瑣德鈞延壽送歸其國|{
	瑣與鎖同}
德鈞見舒嚕太后悉以所齎寶貨并籍其田宅獻之太后問曰汝近者何為往太原德鈞曰奉唐主之命太后指天曰汝從吾兒求為天子何妄語邪|{
	言德鈞舉兵往太原欲從契丹主求為帝耳何乃妄言奉唐主之命邪}
又自指其心曰此不可欺也又曰吾兒將行吾戒之云趙大王若引兵北向渝關亟須引歸太原不可救也汝欲為天子何不先擊退吾兒徐圖亦未晩|{
	徐圖謂徐圖為天子也}
汝為人臣既負其主不能擊敵又欲乘亂邀利所為如此何面目復求生乎德鈞俛首不能對|{
	復扶又翻俛音免以正義責之故不能對}
又問器玩在此|{
	謂德鈞所齎以獻者也}
田宅何在德鈞曰在幽州太后曰幽州今屬誰曰屬太后太后曰然則又何獻焉|{
	此即魏王繼岌留王宗弼所獻謂此皆我家物之意}
德鈞益慙自是鬱鬱不多食踰年而卒張礪與延壽俱入契丹契丹主復以為翰林學士|{
	張礪唐明宗時為翰林學士唐主遣礪督趙延壽進軍於圑柏由是與延壽俱入契丹卒以病中國}
帝將發上黨契丹主舉酒屬帝曰|{
	屬之欲翻}
余遠來徇義今大事已成我若南向河南之人必大驚駭汝宜自引漢兵南下人必不甚愳我令太相温將五千騎衛送汝至河梁|{
	按吐蕃契丹皆大太相河梁即河陽橋 考異曰廢帝實録作高謨翰范實陷蕃記作高謨翰歐陽史作高牟翰蓋蕃名太相温漢名高謨翰今從晉高祖實録}
欲與之渡河者多少隨意余且留此俟汝音聞|{
	聞音問}
有急則下山救汝|{
	下山下太行也}
若洛陽既定吾即北返矣與帝執手相泣久之不能别解白貂裘以衣帝|{
	貂出於北方黑貂之裘南方猶可致白貂之裘南方鮮有之陸佃埤雅曰貂亦鼠類縟毛者也其皮煖於狐貉衣於既翻}
贈良馬二十匹戰馬千二百匹曰世世子孫勿相忘又曰劉知遠趙瑩桑維翰皆創業功臣無大故勿弃也初張敬逹既出師唐主遣左金吾大將軍歷山高漢筠守晉州|{
	河中府河東縣有歷山薛史高漢筠齊州歷山人當從之張敬逹以晉州帥出專征太原故使高漢筠守晉州}
敬逹死建雄節度副使田承肇帥衆攻漢筠於府署|{
	帥讀曰率}
漢筠開門延承肇入從容謂曰僕與公俱受朝寄|{
	從千容翻朝直遥翻下同}
何相迫如此承肇曰欲奉公為節度使漢筠曰僕老矣義不為亂首死生惟公所處|{
	處昌呂翻}
承肇目左右欲殺之軍士投刃於地曰高金吾累朝宿德奈何害之承肇乃謝曰與公戲耳聽漢筠歸洛陽帝遇諸塗|{
	高漢筠蓋自晉州出含口至河陽而帝自太行南下故遇諸塗}
曰朕憂卿為亂兵所傷今見卿甚喜 符彦饒張彦琪至河陽密言於唐主曰今胡兵大下河水復淺|{
	復扶又翻}
人心已離此不可守己丑唐主命河陽節度使萇從簡與趙州刺史劉在明守河陽南城遂斷浮梁|{
	斷音短}
歸洛陽遣宦者秦繼旻皇城使李彦紳殺昭信節度使李□華於其第|{
	李□華契丹主之兄故殺之}
己卯帝至河陽萇從簡迎降舟楫已具|{
	唐主雖斷河梁而萇從簡具舟楫以濟晉兵降戶江翻}
彰聖軍執劉在明以降|{
	彰聖軍蓋留戍河陽者}
帝釋之使復其所 唐主命馬軍都指揮使宋審䖍步軍都指揮使符彦饒河陽節度使張彦琪宣徽南院使劉延朗將千餘騎至白馬阪行戰地|{
	白司馬阪也在洛陽北史逸司字行下孟翻}
有五十餘騎奔于北軍|{
	此北軍謂晉兵從太原至河陽者也}
諸將謂審䖍曰何地不可戰誰敢立於此|{
	言人心已離也}
乃還|{
	還從宣翻又如字}
庚辰唐主又與四將議復向河陽|{
	四將即謂宋審䖍等四人復扶又翻}
而將校皆已飛狀迎帝帝慮唐主西奔遣契丹千騎扼澠池|{
	澠彌兖翻}
辛巳唐主與曹太后劉皇后雍王重美及宋審䖍等攜傳國寶登玄武樓自焚|{
	年五十一宋審䖍與唐主起事於鳳翔親將也故與之俱死雍於用翻}
皇后積薪欲燒宫室|{
	此皇后謂唐主劉皇后}
重美諫曰新天子至必不露居它日重勞民力|{
	重勞直用翻}
死而遺怨將安用之乃止王淑妃謂太后曰事急矣宜且避匿以俟姑夫|{
	太后曹太后也姑夫謂帝也皇后曹太后之女故王淑妃使之避匿以俟帝來}
太后曰吾子孫婦女一朝至此|{
	子謂唐主孫謂重美婦謂劉后女謂唐主之女}
何忍獨生妹自勉之淑妃乃與許王從益匿於毬場獲免是日晩帝入洛陽止於舊第唐兵皆解甲待罪帝慰而釋之帝命劉知遠部署京城知遠分漢軍使還營館契丹於天宫寺|{
	館古玩翻}
城中肅然無敢犯令士民避亂竄匿者數日皆還復業|{
	史言劉知遠之才畧}
初帝在河東為唐朝所忌中書侍郎同平章事判三司張延朗不欲河東多蓄積凡財賦應留使之外盡收取之|{
	唐制諸州財賦為三一上供輸之京師以供上用也二送使輸送於節度觀察使府三留州留為州家用度其後天下悉裂為籓鎮支郡則仍謂之留州會府則謂之留使朝直遥翻使疏吏翻}
帝以是恨之壬午百官入見|{
	見賢遍翻}
獨收延朗付御史臺餘皆謝恩|{
	漢馮衍有言在人惡其罵我在我欲其罵人晉祖初入洛而先收張延朗不惟示天下以褊亦非所以勸居官奉職者也既誅又悔之則無及矣}
甲申車駕入宫大赦應中外官吏一切不問惟賊臣張延朗劉延皓劉延朗姦邪貪猥罪難容貸中書侍郎平章事馬胤孫樞密使房暠宣徽使李專美河中節度使韓昭胤等雖居重位不務詭隨並釋罪除名中外臣僚先歸順者委中書門下别加任使劉延皓匿於龍門|{
	九域志河南府河南縣有龍門鎮}
數日自經死劉延朗將犇南山|{
	洛城之南山即伊陽諸山}
捕得殺之斬張延朗既而選三司使難其人帝甚悔之 閩人聞唐主之亡歎曰潞王之罪天下未之聞也將如吾君何|{
	史言閩人怨毒其君}
十二月辛酉朔帝如河陽餞太相温及契丹兵歸國 追廢唐主為庶人 丁亥以馮道兼門下侍郎同平章事 曹州刺史鄭阮貪暴指揮使石重立因亂殺之|{
	因亂者因中原之亂也史言貪暴之人不惟難免於治世亦難免於亂世}
族其家 辛卯以唐中書侍郎姚顗為刑部尚書 初朔方節度使張希崇為政有威信民夷愛之興屯田以省漕運在鎮五年求内徙唐潞王以為靜難節度使|{
	難乃旦翻}
帝與契丹修好恐其復取靈武|{
	契丹既得燕雲恐其乘勢又取靈武好呼到翻復扶又翻}
癸巳復以希崇為朔方節度使 初成德節度使董温琪貪暴積貨巨萬以牙内都虞候平山祕瓊為腹心|{
	平山縣屬鎮州本隋所置房山縣唐天寶末安禄山反玄宗改鹿泉縣為獲鹿房山縣為平山九域志平山在州西六十五里}
温琪與趙德鈞俱没於契丹|{
	趙德鈞邀董温琪同救晉安與之俱没}
瓊盡殺温琪家人瘞於一坎而取其貨|{
	象有齒而焚其身賄也為祕瓊為范延光所殺張本瘞於計翻}
自稱留後表稱軍亂 同州小校門鐸殺節度使楊漢賓焚掠州城|{
	河南官氏志後魏改叱門氏為門氏又有吐門氏改為門氏又有庫門氏改為門氏校戶教翻}
詔贈李□華燕王|{
	燕於贒翻}
遣使送其喪歸國 張朗將其衆入朝|{
	帝初起事張朗守代州不從將即亮翻}
庚子以唐中書侍郎盧文紀為吏部尚書以皇城使

晉陽周瓌為大將軍充三司使瓌辭曰臣自知才不稱職|{
	稱尺證翻}
寧以避事見弃猶勝冒寵獲辜帝許之 帝聞平盧節度使房知温卒遣天平節度使王建立將兵廵撫青州|{
	以虞變也將即亮翻下同}
改興唐府曰廣晉府|{
	後唐改魏州為興唐府晉興又改為廣晉府以易世而易府名也}
安遠節度使盧文進聞帝為契丹所立自以本契丹叛將|{
	盧文進自契丹來奔見二百七十五卷明宗天成元年}
辛丑弃鎮奔吳|{
	九域志安州東至黄州四百里東南至卾州三百六十里黄卾皆吳土也}
所過鎮戍召其主將告之故皆拜辭而退 徐知誥以鎮南節度使太尉兼中書令李德誠德勝節度使兼中書令周本位望隆重欲使之帥衆推戴本曰我受先王大恩|{
	周本所言先王謂楊行密也帥讀曰率}
自徐温父子用事恨不能救楊氏之危又使我為此可乎其子弘祚強之|{
	強其兩翻}
不得已與德誠帥諸將詣江都表吳主陳知誥功德請行册命又詣金陵勸進宋齊丘謂德誠之子建勲曰尊公太祖元勲|{
	吳楊行密廟號太祖}
今日掃地矣於是吳宫多妖|{
	吳宫謂江都宫妖一遥翻}
吳主曰吳祚其終乎左右曰此乃天意非人事也高麗王建用兵擊破新羅百濟於是東夷諸國皆附

之有二京六府九節度百二十郡|{
	王建得高麗見二百七十一卷梁均王龍德三年}


資治通鑑卷二百八十














































































































































