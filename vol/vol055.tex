資治通鑑卷五十五
宋 司馬光 撰

胡三省 音註

漢紀四十七|{
	起閼逢執徐盡柔兆敦牂凡三年}


孝桓皇帝中

延熹七年春二月丙戌邟鄉忠侯黄瓊薨|{
	賢曰說文云邟潁川縣也漢潁川有周承休侯國元始三年更名曰邟音亢 考異曰范書四年瓊免司空至七年卒袁紀七年瓊以太尉薨范書楊秉五年代劉矩為太尉袁紀此年瓊卒秉乃為太尉今從范書}
將葬四方遠近名士會者六七千人初瓊之教授於家徐穉從之諮訪大義及瓊貴穉絶不復交至是穉往弔之進酹哀哭而去|{
	穉直利翻復扶又翻酹盧對翻醊祭以酒沃地曰酹}
人莫知者諸名士推問喪宰|{
	喪宰典喪者也}
宰曰先時有一書生來衣麤薄而哭之哀不記姓字衆曰必徐孺子也|{
	徐穉字孺子先悉薦翻衣於既翻}
於是選能言者陳留茅容輕騎追之及於塗容為沽酒市肉穉為飲食|{
	為于偽翻下同}
容問國家之事穉不答更問稼穡之事穉乃答之容還以語諸人|{
	語牛倨翻}
或曰孔子云可與言而不與言失人|{
	論語載孔子之言}
然則孺子其失人乎太原郭泰曰不然孺子之為人清潔高亷飢不可得食寒不可得衣|{
	食讀曰飤衣於既翻}
而為季偉飲酒食肉此為已知季偉之賢故也|{
	茅容字季偉此為如字}
所以不答國事者是其智可及其愚不可及也|{
	亦以孔子之言語諸人蓋以甯武子况徐孺子}
泰博學善談論初游雒陽時人莫識陳留符融|{
	符姓也此符從竹從付非草付之苻}
一見嗟異因以介於河南尹李膺|{
	古者主冇儐客有介孔叢子曰士無介不見介因也}
膺與相見曰吾見士多矣未有如郭林宗者也|{
	郭泰字林宗}
其聰識通朗高雅密博今之華夏鮮見其儔|{
	夏戶雅翻鮮息淺翻}
遂與為友於是名震京師後歸鄉里衣冠諸儒送至河上車數千兩|{
	兩音亮}
膺唯與泰同舟而濟衆賓望之以為神仙焉|{
	自雒陽至太原渡河而西北}
泰性明知人好奬訓士類|{
	好呼到翻}
周遊郡國茅容年四十餘耕於野與等輩避雨樹下衆皆夷踞相對|{
	賢曰夷平也說文曰踞蹲也論語曰原壤夷俟言平坐踞傲也}
容獨危坐愈恭|{
	危坐正襟盡前而坐}
泰見而異之因請寓宿旦日容殺雞為饌|{
	饌雛皖翻又雛戀翻}
泰謂為已設容分半食母餘半庋置|{
	食讀曰飤毛晃曰板為閣以藏物曰庋舉綺翻}
自以草蔬與客同飯|{
	賢曰草麄也飯父遠翻}
泰曰卿賢哉遠矣|{
	既言賢哉乂言遠矣言其賢去常人甚遠}
郭林宗猶減三牲之具以供賓旅|{
	三牲之具謂養親之具也孝經曰日用三牲之養賓旅猶言賓客也}
而卿如此乃我友也起對之揖勸令從學卒為盛德|{
	卒子恤翻}
鉅鹿孟敏客居太原荷甑墮地不顧而去|{
	荷下可翻甑子孕翻譙周古史考曰黄帝始作甑周官考工記甑實二鬴註云六斗四升曰鬴古者陶而為甑釋器云䰝謂鬵鬵鉹也孫炎曰關東人謂甑為鬵涼州人謂甑為鉹䰝即甑字}
泰見而問其意對曰甑已破矣視之何益泰以為有分决與之言知其德性因勸令游學遂知名當世陳留申屠蟠家貧傭為漆工鄢陵庾乘少給事縣廷為門士|{
	鄢陵縣屬潁川郡師古曰鄢音偃陸德明曰鄢謁晩翻又於建翻賢曰門士即門卒少詩照翻}
泰見而奇之其後皆為名士自餘或出於屠沽卒伍因泰奬進成名者甚衆陳國童子魏昭請於泰曰經師易遇人師難遭|{
	經師謂專門名家教授有師法者人師謂謹身修行足以範俗者易以䜴翻}
願在左右供給灑掃|{
	灑所賣翻乂山寄翻掃悉報翻}
泰許之泰嘗不佳|{
	謂體中有不節適也語曰不佳微有疾也}
命昭作粥粥成進泰泰呵之曰|{
	呵責怒也音虎何翻}
為長者作粥不加意敬使不可食以杯擲地昭更為粥重進泰復呵之|{
	為于偽翻重直龍翻復扶又翻}
如此者三昭姿容無變泰乃曰吾始見子之面而今而後知卿心耳遂友而善之陳留左原為郡學生犯灋見斥泰遇諸路為設酒肴以慰之謂曰昔顔涿聚梁甫之巨盜段干木晉國之大駔卒為齊之忠臣魏之名賢|{
	呂氏春秋曰顔涿聚梁父大盜也學於孔子左傳晉伐齊戰于犂丘齊師敗績知伯親禽顔庚杜預註曰犂丘隰也顔庚齊大夫顔涿聚也又曰晉荀瑶伐鄭鄭請救于齊齊師將興陳成子設乘車兩馬繫五邑焉召顔涿聚之子晉曰隰之役而父死焉今君命汝是邑服車而朝毋廢前勞呂氏春秋曰段干木晉國之駔說文曰駔會也謂合兩家之買賣如今之度市也新序曰魏文侯過段干木之廬而軾之國人誦之曰吾君好正段干木之敬吾君好忠段干木之隆秦欲攻魏司馬唐諫曰段干木賢者也而魏禮之天下莫不聞毋乃不可加兵乎駔子朗翻卒子恤翻}
蘧瑗顔囘尚不能無過|{
	論語曰蘧伯玉使人於孔子子問之曰夫子何為對曰夫子欲寡其過而未能也又語曰顔回好學不貳過蘧求於翻瑗于眷翻}
况其餘乎慎勿恚恨責躬而已|{
	恚於避翻}
原納其言而去或有譏泰不絶惡人者泰曰人而不仁疾之已甚亂也|{
	賢曰論語孔子之言也鄭玄注云不仁之人當以風化之若疾之甚是益使為亂也}
原後忽更懷忿結客欲報諸生其日泰在學原愧負前言因遂罷去後事露衆人咸謝服焉或問范滂曰郭林宗何如人滂曰隱不違親|{
	賢曰介推之類}
貞不絶俗|{
	賢曰柳下惠之類}
天子不得臣諸侯不得友吾不知其它泰嘗舉有道不就|{
	舉有道事始五十卷安帝建光元年}
同郡宋冲素服其德以為自漢元以來未見其匹嘗勸之仕|{
	漢元謂漢初也匹儔也等也偶也}
泰曰吾夜觀乾象晝察人事天之所廢不可支也吾將優游卒歲而已|{
	卒子恤翻}
然猶周旋京師誨誘不息|{
	誘音酉}
徐穉以書戒之曰大木將顛非一繩所維何為栖栖不遑寧處|{
	賢曰顛仆也維繫也喻時將衰季非一人所能救也尹焞曰栖栖猶皇皇也處昌呂翻}
泰感寤曰謹拜斯言以為師表濟隂黄允以雋才知名|{
	濟子禮翻}
泰見而謂曰卿高才絶人足成偉器年過四十聲名著矣然至於此際當深自匡持不然將失之矣後司徒袁隗欲為從女求姻|{
	為干偽翻從才用翻}
見允歎曰得婿如是足矣允聞而黜遣其妻|{
	允妻夏侯氏允黜其妻欲婿于袁也}
妻請大會宗親為别因於衆中攘袂數允隱慝十五事而去允以此廢於時|{
	當時清議為何如哉數所矩翻慝吐得翻}
初允與漢中晉文經並恃其才智曜名遠近徵辟不就託言療病京師不通賓客公卿大夫遣門生旦暮問疾郎吏雜坐其門猶不得見三公所辟召者輒以詢訪之隨所臧否|{
	否音鄙}
以為與奪符融謂李膺曰二子行業無聞|{
	行下孟翻下同}
以豪傑自置遂使公卿問疾王臣坐門融恐其小道破義空譽違實特宜察焉膺然之二人自是名論漸衰賓徒稍省旬日之間慙歎逃去後並以罪廢棄陳留仇香至行純嘿|{
	姓譜仇姓宋大夫仇牧之後行下孟翻下同}
鄉黨無知者年四十為蒲亭長|{
	蒲亭屬陳留郡考城縣}
民有陳元獨與母居母詣香告元不孝香驚曰吾近日過元舍廬落整頓|{
	賢曰落居也今人謂院為落}
耕耘以時此非惡人當是教化未至耳母守寡養孤苦身投老柰何以一旦之忿棄歷年之勤乎且母養人遺孤不能成濟若死者有知百歲之後當何以見亡者母涕泣而起香乃親到元家為陳人倫孝行譬以禍福之言元感悟卒為孝子|{
	為于偽翻卒子恤翻}
考城令河内王奐署香主簿|{
	考城縣屬陳留郡故菑縣章帝惡其名改曰考城}
謂之曰聞在蒲亭陳元不罸而化之得無少鷹鸇之志邪|{
	鷹鸇以鷙擊為事左傳見無禮者誅之如鷹鸇之逐鳥雀也少詩沼翻}
香曰以為鷹鸇不若鸞鳳故不為也奐曰枳棘之林非鸞鳳所集百里非大賢之路|{
	賢曰時奐為縣令故自稱百里也}
乃以一月奉資香|{
	奉讀曰俸}
使入太學郭泰符融齎刺謁之|{
	書姓名以自通求見曰刺秦漢之間謂之謁}
因留宿明旦泰起下牀拜之曰君泰之師非泰之友也香學畢歸鄉里雖在宴居|{
	賢曰宴安也朱子曰宴居閒暇無事之時}
必正衣服妻子事之若嚴君妻子有過免冠自責妻子庭謝思過香冠妻子乃敢升堂終不見其喜怒聲色之異不應徵辟卒於家 三月癸亥隕石于鄠|{
	鄠縣屬扶風鄠音戶}
夏五月己丑京師雨雹 荆州刺史度尚募諸蠻夷擊艾縣賊大破之降者數萬人桂陽宿賊卜陽潘鴻等逃入深山|{
	宿賊言積久為賊者}
尚窮追數百里破其三屯多獲珍寶陽鴻黨衆猶盛尚欲擊之而士卒驕富莫有鬭志尚計緩之則不戰逼之必逃亡乃宣言卜陽潘鴻作賊十年習於攻守今兵寡少未易可進|{
	易以䜴翻}
當須諸郡所發悉至乃并力攻之申令軍中恣聽射獵|{
	申令者既下令而申言之申重也}
兵士喜悦大小皆出尚乃密使所親客潛焚其營珍積皆盡獵者來還莫不泣涕尚人人慰勞深自咎責|{
	以失火自咎責也勞力到翻}
因曰卜陽等財寶足富數世諸卿但不并力耳所亡少少|{
	少詩沼翻}
何足介意衆咸憤踴尚敇令秣馬蓐食明旦徑赴賊屯陽鴻等自以深固不復設備|{
	復扶又翻}
吏士乘鋭遂破平之尚出兵三年|{
	延熹五年尚刺荆州至是三年矣}
羣寇悉定封右鄉侯 冬十月壬寅帝南巡庚申幸章陵戊辰幸雲夢臨漢水還幸新野時公卿貴戚車騎萬計徵求費役不可勝極|{
	勝音升}
護駕從事桂陽胡騰上言|{
	護駕從事蓋荆州刺史所遣護車駕者也}
天子無外|{
	春秋公羊傳曰王者無外}
乘輿所幸即為京師臣請以荆州刺史比司隸校尉臣自同都官從事帝從之自是肅然莫敢妄干擾郡縣|{
	荆州刺史得察舉所部郡縣而不可得察舉扈從之臣若比司隸校尉則得察舉其姦故肅然也}
帝在南陽左右並通姦利詔書多除人為郎太尉楊秉上疏曰太微積星名為郎位|{
	賢曰史記天官書曰太微宮五帝坐後聚二十五星蔚然曰郎位積聚也}
入奉宿衛出牧百姓宜割不忍之恩以斷求欲之路|{
	斷丁管翻}
於是詔除乃止 護羌校尉段熲擊當煎羌破之 十二月辛丑車駕還宮 中常侍汝陽侯唐衡武原侯徐璜皆卒|{
	汝陽縣屬汝南郡武原縣屬彭城國}
初侍中寇榮恂之曾孫也性矜潔少所與|{
	少詩沼翻}
以此為權寵所疾榮從兄子尚帝妺益陽長公主帝又納其從孫女於後宮|{
	從才用翻長知兩翻}
左右益忌之遂共陷以罪與宗族免歸故郡|{
	寇氏本上谷昌平人}
吏承望風旨持之浸急榮恐不免詣闕自訟未至刺史張敬追劾榮以擅去邊|{
	刺史蓋幽州刺史也劾戶槩翻又戶得翻}
有詔捕之榮逃竄數年會赦不得除積窮困乃自亡命中上書曰陛下統天理物作民父母自生齒以上咸蒙德澤|{
	大戴禮曰男子八月生齒女子七月生齒}
而臣兄弟獨以無辜為專權之臣所見批扺|{
	賢曰說文曰扺側擊也批音片支翻余按前書音義批音蒲結翻扺諸氏翻}
青蠅之人所共構會|{
	詩曰營營青蠅止于樊豈弟君子無信讒言青蠅能汚白使黑汚黑使白喻佞人變亂善惡也}
令陛下忽慈母之仁發投杼之怒|{
	事見三卷周赧王七年}
殘諂之吏張設機網並驅爭先若赴仇敵罸及死沒髠剔墳墓|{
	謂剪伐松栢如人之髠剔也}
欲使嚴朝必加濫罰|{
	朝直遥翻}
是以不敢觸突天威而自竄山林以俟陛下發神聖之聽啟獨覩之明救可濟之人援没溺之命不意滯怒不為春夏息|{
	賢曰春夏生長萬物故不宜怒為于偽翻下同}
淹恚不為歲時怠|{
	滯怒淹恚言怒恚積蓄久而不化也恚於避翻}
遂馳使郵驛布告遠近嚴文尅剝痛於霜雪逐臣者窮人途追臣者極車軌雖楚購伍員|{
	史記楚人伍奢為平王太子建太傅費無極譖殺奢奢子員字子胥奔吳楚購之得伍員者賜粟五萬石爵執珪員音云}
漢求季布|{
	事見十卷高祖五年}
無以過也臣遇罰以來三赦再贖無驗之罪足以蠲除|{
	賢曰無驗謂無罪狀可案驗也}
而陛下疾臣愈深有司咎臣甫力|{
	賢曰甫始也力甚也}
止則見埽滅行則為亡虜苟生則為窮人極死則為寃鬼天廣而無以自覆|{
	覆敷救翻}
地厚而無以自載蹈陸土而有沉淪之憂遠巖墻而有鎮壓之患|{
	遠于願翻}
如臣犯元惡大憝|{
	賢曰憝惡言元惡之人大為人之所惡也憝徒對翻}
足以陳原野備刀鋸|{
	賢曰鋸刖刑也國語曰刑有五大者陳諸原野}
陛下當班布臣之所坐以解衆論之疑臣思入國門坐於肺石之上使三槐九棘平臣之罪|{
	周禮秋官曰左九棘孤卿大夫位焉右九棘公侯伯子男位焉面三槐三公位焉左嘉石平罷民右肺石達窮民註肺石赤石也槐取其懷來棘取其赤心外刺}
而閶闔九重|{
	賢曰閶闔天門也重直龍翻}
陷穽步設舉趾觸罘罝|{
	賢曰穽阬穽也說文罘兔網也罝亦兔網也音浮嗟}
動行絓羅網|{
	絓古賣翻罥也}
無緣至萬乘之前|{
	乘繩證翻}
永無見信之期悲夫久生亦復何聊|{
	復扶又翻}
蓋忠臣殺身以解君怒孝子殞命以寧親怨故大舜不避塗廩浚井之難|{
	史記舜父瞽叟常欲殺舜使舜塗廩從下焚廩舜乃以兩笠自扞而下又使穿井舜為匿空旁出舜既入深父乃下土實之舜從旁空出去難乃旦翻}
申生不辭姬氏讒邪之謗|{
	左傳驪姬嬖於晉獻公欲殺太子申生謂申生曰君夢齊姜必速祭之太子祭于曲沃歸胙于公公田姬置諸宫六日公至毒而獻之公祭之地地墳與犬犬斃與小臣小臣斃姬泣曰賊由太子太子奔新城或謂太子子辭君必辨焉太子曰我辭姬必有罪遂縊而死}
臣敢忘斯義不自斃以解明朝之忿哉乞以身塞責|{
	朝直遥翻塞悉則翻}
願陛下匄兄弟死命|{
	賢曰匄乞也音蓋}
使臣一門頗有遺類以崇陛下寛饒之惠先死陳情臨章泣血帝省章愈怒|{
	先悉薦翻省悉井翻}
遂誅榮寇氏由是衰廢 |{
	考異曰袁紀置此事於延熹元年按范書榮傳云延熹中被罪榮書又云遇罰以來三赦再贖不知榮死果在何年按襄楷竇武上書皆言梁孫寇鄧之誅今置於此}


八年春正月帝遣中常侍左悺之苦縣祠老子|{
	賢曰史記曰老子者楚苦縣厲鄉曲仁里人也名耳字耼姓李為周守藏史有神廟故就祠之苦縣屬陳國故城在今亳州谷陽縣苦音戶又如字}
勃海王悝素行險僻|{
	悝苦回翻行下孟翻}
多僭傲不法北軍中候陳留史弼上封事曰臣聞帝王之於親戚愛雖隆必示之以威體雖貴必禁之以度如是和睦之道興骨肉之恩遂矣竊聞勃海王悝外聚剽輕不逞之徒|{
	賢曰剽悍也逞快也謂被侵枉不快之人也左傳曰率羣不逞之人余謂不逞謂包藏禍心而不得逞者剽匹妙翻}
内荒酒樂出入無常所與羣居皆家之棄子朝之斥臣|{
	朝直遥翻下同}
必有羊勝伍被之變|{
	羊勝事見十六卷景帝中二年伍被事見十九卷武帝元狩元年}
州司不敢彈糾|{
	州司謂州刺史之屬}
傅相不能匡輔陛下隆於友于|{
	書曰惟孝友于兄弟}
不忍遏絶恐遂滋蔓|{
	滋長也蔓延也左傳曰無使滋蔓蔓難圖也}
為害彌大乞露臣奏宣示百僚平處其法|{
	處昌呂翻}
法决罪定乃下不忍之詔臣下固執然後少有所許|{
	少詩沼翻}
如是則聖朝無傷親之譏勃海有享國之慶不然懼大獄將興矣上不聽悝果謀為不道|{
	帝紀曰悝謀反}
有司請廢之詔貶為癭陶王食一縣|{
	賢曰癭陶縣屬鉅鹿郡故城在今趙州癭陶縣西南癭於郢翻}
丙申晦日有食之詔公卿校尉舉賢良方正|{
	校戶教翻}
千秋萬歲殿火 中常侍侯覽兄參為益州刺史殘暴貪婪|{
	婪盧含翻}
累臧億計太尉楊秉奏檻車徵參參於道自殺閱其車重三百餘兩皆金銀錦帛|{
	重直用翻}
秉因奏曰臣案舊典宦者本在給使省闥司昏守夜而今猥受過寵執政操權|{
	操七刀翻}
附會者因公襃舉違忤者求事中傷|{
	忤五故翻中竹仲翻}
居法王公富擬國家飲食極肴膳僕妾盈紈素中常侍侯覽弟參貪殘元惡自取禍滅覽顧知釁重必有自疑之意臣愚以為不宜復見親近|{
	復扶又翻近其靳翻}
昔懿公刑邴之父奪閻職之妻而使二人參乘卒有竹中之難|{
	左氏傳齊懿公之為公子也與邴之父爭田弗勝及即位乃掘而刖之而使僕納閻職之妻而使職參乘公游于申池二人浴于池以鞭抶職職怒曰人奪汝妻而不怒一抶汝庸何傷職曰與刖其父而不能病者何如乃謀弑公納諸竹中邴音丙又彼病翻左傳作歜昌欲翻卒子恤翻難乃旦翻}
覽宜急屏斥投畀有虎|{
	畀與也詩曰取彼讒人投畀豺虎屛必郢翻}
若斯之人非恩所宥請免官送歸本郡書奏尚書召對秉掾屬詰之曰|{
	賢曰召秉掾屬問之詰去吉翻}
設官分職各有司存三公統外御史察内今越奏近官經典漢制何所依據其開公具對秉使對曰春秋傳曰除君之惡唯力是視|{
	左傳載寺人披之言此經典也}
鄧通懈慢申屠嘉召通詰責文帝從而請之|{
	事見十五卷文帝後二年此漢制也}
漢世故事三公之職無所不統尚書不能詰帝不得已竟免覽官司隸校尉韓縯因奏左悺罪惡及其兄太僕南鄉侯稱請託州郡聚斂為姦|{
	斂力贍翻}
賓客放縱侵犯吏民悺稱皆自殺縯又奏中常侍具瑗兄沛相恭臧罪徵詣廷尉瑗詣獄謝上還東武侯印綬|{
	東武城屬清河郡據宦者傳瑗封東武陽侯東武陽屬東郡上時掌翻}
詔貶為都鄉侯超及璜衡襲封者並降為鄉侯 |{
	考異曰楊秉傳南巡之明年秉劾侯覽則是在此年矣宦者傳韓縯奏具瑗瑗坐奪國為鄉侯與秉傳所云削瑗國共是一時事明矣而袁紀載在去年春與范不同今從范書}
子弟分封者悉奪爵土劉普等貶為關内侯尹勲等亦皆奪爵 帝多内寵宫女至五六千人及驅役從使復兼倍於此|{
	驅役者嬖倖挾勢驅掠良人以供掖庭私役者也從使者趨勢附力樂從而為之使者也復扶又翻}
而鄧后恃尊驕忌與帝所幸郭貴人更相譖訴|{
	更工衡翻}
癸亥廢皇后鄧氏送暴室以憂死|{
	漢官儀曰暴室在掖庭内丞一人主宮中婦人疾病者其皇后貴人有罪者亦就此室}
河南尹鄧萬世虎賁中郎將鄧會皆下獄誅|{
	下遐稼翻}
護羌校尉段熲擊罕姐羌破之|{
	姐且也翻又音紫}
三月辛巳赦天下 宛陵大姓羊元羣罷北海郡|{
	宛陵縣屬河南尹}
臧汚狼籍郡舍溷軒有奇巧|{
	賢曰溷軒厠屋}
亦載之以歸河南尹李膺表按其罪元羣行賂宦官膺竟反坐|{
	反坐按其罪而不得行反自坐罪}
單超弟遷為山陽太守以罪繫獄廷尉馮緄考致其死|{
	考鞠而致其死罪也緄古本翻}
中官相黨共飛章誣緄以罪中常侍蘇康管霸固天下良田美業|{
	固障固也}
州郡不敢詰大司農劉祐移書所在依科品沒入之帝大怒與膺緄俱輸作左校 夏四月甲寅安陵園寑火|{
	安陵惠帝陵也}
丁巳詔壞郡國諸淫祀|{
	壞音怪}
特留雒陽王渙密縣卓茂二祠 五月丙戌太尉楊秉薨秉為人清白寡欲嘗稱我有三不惑酒色財也秉既没所舉賢良廣陵劉瑜乃至京師上書言中官不當比肩裂土競立胤嗣繼體傳爵|{
	順帝陽嘉四年著令聽中官以養子襲爵}
又嬖女充積冗食空宫|{
	無事而食謂之冗食冗而隴翻}
傷生費國又第舍增多窮極奇巧掘山攻石促以嚴刑州郡官府各自考事姦情賕賂皆為吏餌民愁鬱結起入賊黨官輒興兵誅討其罪貧困之民或有賣其首級以要酬賞|{
	要一遙翻}
父兄相代殘身妻孥相視分裂又陛下好微行近習之家|{
	好呼到翻}
私幸宦者之舍賓客市買熏灼道路因此暴縱無所不容惟陛下開廣諫道|{
	諫道謂言路也}
博觀前古遠佞邪之人|{
	遠于願翻}
放鄭衛之聲則政致和平德感祥風矣|{
	孝經援神契曰德至八方則祥風至}
詔特召瑜問災咎之徵執政者欲令瑜依違其辭乃更策以它事瑜復悉心對八千餘言有切於前|{
	復扶又翻下同}
拜為議郎 荆州兵朱蓋等叛與桂陽賊胡蘭等復攻桂陽太守任胤棄城走|{
	任音壬}
賊衆遂至數萬轉攻零陵太守下邳陳球固守拒之零陵下溼編木為城|{
	零陵郡武帝置宋白曰郡古理在今全州清湘縣南七十八里古城存焉}
郡中惶恐掾史白球遣家避難|{
	難乃旦翻}
球怒曰太守分國虎符受任一邦豈顧妻孥而沮國威乎|{
	孥音奴沮在呂翻}
復言者斬乃弦大木為弓羽矛為矢引機發之多所殺傷|{
	此則今划車弩之類}
賊激流灌城球輒於内因地埶反决水淹賊相拒十餘日不能下時度尚徵還京師詔以尚為中郎將率步騎三萬餘人救球發諸郡兵并埶討擊大破之斬蘭等首三千餘級復以尚為荆州刺史蒼梧太守張叙為賊所執及任胤皆徵棄市胡蘭餘黨南走蒼梧交趾刺史張磐擊破之賊復還入荆州界度尚懼為已負|{
	負罪負也懼以不能盡滅羣賊為罪}
乃偽上言蒼梧賊入荆州界於是徵磐下廷尉|{
	上時掌翻下遐稼翻}
辭狀未正會赦見原磐不肯出獄方更牢持械節|{
	竹約為節械節亦械之刻約處也 考異曰按張磐會赦得原檢帝紀此後未有赦不知會何赦也六年三月赦前此二年永康元年六月赦後此二年今從帝紀}
獄吏謂磐曰天恩曠然而君不出可乎磐曰磐備位方伯|{
	古者八州八伯漢州刺史古方伯之任也}
為尚所枉受罪牢獄夫事有虛實法有是非磐實不辜赦無所除如忍以苟免永受侵辱之恥生為惡吏死為敝鬼乞傳尚詣廷尉|{
	以傳車召致廷尉也傳株戀翻又直戀翻}
面對曲直足明真偽尚不徵者磐埋骨牢檻終不虛出望塵受枉廷尉以其狀上|{
	上時掌翻}
詔書徵尚到廷尉辭窮受罪以先有功得原 閏月甲午南宫朔平署火|{
	此朔平司馬署也百官志朔平司馬主北宫北門}
段熲擊破西羌進兵窮追展轉山谷間自春及秋無日不戰虜遂敗散凡斬首二萬三千級獲生口數萬人降者萬餘落|{
	降戶江翻}
封熲都鄉侯 秋七月以太中大夫陳蕃為太尉蕃讓於太常胡廣議郎王暢㢮刑徒李膺帝不許暢龔之子也|{
	王龔事安帝為公}
嘗為南陽太守疾其多貴戚豪族下車奮厲威猛大姓有犯或使吏發屋伐樹堙井夷竈|{
	破其家業也}
功曹張敞奏記諫曰文翁召父卓茂之徒|{
	召讀曰邵}
皆以温厚為政流聞後世發屋伐樹將為嚴烈雖欲懲惡難以聞遠|{
	聞音問}
郡為舊都侯甸之國|{
	古者天子之制規方千里以為甸服又其外五百里為侯服光武起於南陽其後謂之南都又於雒陽在侯甸之内故云然}
園廟出於章陵三后生自新野|{
	賢曰南頓君以上四廟在章陵光烈皇后和帝隂后鄧后並新野人}
自中興以來功臣將相繼世而隆愚以為懇懇用刑不如行恩孳孳求姦|{
	孳孳猶汲汲也}
未若禮賢舜舉臯陶不仁者遠|{
	論語載子夏之言陶音遥}
化人在德不在用刑暢深納其言更崇寛政教化大行 八月戊辰初令郡國有田者畝歛税錢|{
	賢曰畝十錢也余據宦者傳張讓等說靈帝歛天下田畝税十錢非此時事也蓋漢田租三十税一而計畝斂錢則自此始}
九月丁未京師地震 冬十月司空周景免以太常劉茂為司空茂愷之子也|{
	劉愷以讓國重於時位至公}
郎中竇武融之玄孫也有女為貴人采女田聖有寵於帝帝將立之為后司隸校尉應奉上書曰母后之重興廢所因漢立飛燕胤祀泯絶|{
	事見三十三卷哀帝建平元年}
宜思關雎之所求|{
	關雎樂得淑女以配君子}
遠五禁之所忌|{
	韓詩外傳曰婦人有五不娶喪婦之長女不娶為其不受命也世有惡疾不娶棄於天也世有刑人不娶棄于人也亂家女不娶類不正也逆家女不娶廢人倫也遠于願翻}
太尉陳蕃亦以田氏卑微竇族良家爭之甚固帝不得已辛巳立竇貴人為皇后拜武為特進城門校尉封槐里侯 十一月壬子黄門北寺火 陳蕃數言李膺馮緄劉祐之枉|{
	數所角翻下同}
請加原宥升之爵任言及反覆誠辭懇切以至流涕帝不聽應奉上疏曰夫忠賢武將國之心膂|{
	將即亮翻}
竊見左校弛刑徒馮緄劉祐李膺等誅舉邪臣肆之以法|{
	賢曰肆陳也}
陛下既不聽察而猥受譖訴遂令忠臣同愆元惡自春迄冬不蒙降恕遐邇觀聽為之歎息|{
	為于偽翻}
夫立政之要記功忘失是以武帝捨安國於徒中|{
	賢曰景帝時韓安國為梁大夫坐法扺罪後梁内史缺起徒中為二千石此言武帝誤也}
宣帝徵張敞於亡命|{
	事見二十七卷宣帝甘露元年}
緄前討蠻荆均吉甫之功|{
	詩曰顯允方叔征伐玁狁蠻荆來威鄭玄注云方叔先與吉甫征伐玁狁今特征伐蠻荆皆使來服宣王之威緄以順帝時討武陵長沙蠻夷有功故以吉甫比之}
祐數臨督司有不吐茹之節|{
	賢曰謂祐奏梁冀弟旻又為司隷校尉權豪畏之也詩曰柔亦不茹剛亦不吐數所角翻}
膺著威幽并遺愛度遼|{
	膺為漁陽太守烏桓校尉皆幽部也度遼將軍則屯并部是其著威遺愛之地}
今三垂蠢動王旅未振乞原膺等以備不虞書奏乃悉免其刑久之李膺復拜司隸校尉|{
	復扶又翻下同}
時小黄門張讓弟朔為野王令貪殘無道畏膺威嚴逃還京師|{
	野王縣屬河内郡而河内郡屬司部畏膺察舉其罪故逃還京師也}
匿於兄家合柱中|{
	合木為柱安足以容人合柱謂兩柱相直兩屋相合處也}
膺知其狀率吏卒破柱取朔付雒陽獄受辭畢即殺之讓訴寃於帝帝召膺詰以不先請便加誅之意對曰昔仲尼為魯司寇七日而誅少正卯今臣到官已積一旬私懼以稽留為愆不意獲速疾之罪誠自知釁責死不旋踵特乞留五日尅殄元惡退就鼎鑊始生之願也帝無復言顧謂讓曰此汝弟之罪司隸何愆乃遣出自此諸黄門常侍皆鞠躬屛氣|{
	屛必郢翻}
休沐不敢出宫省帝怪問其故並叩頭泣曰畏李校尉時朝廷日亂綱紀頹阤|{
	阤丈爾翻壞也}
而膺獨持風裁|{
	賢曰裁音才代翻}
以聲名自高士有被其容接者名為登龍門云|{
	賢曰以魚為喻也龍門河水所下之口在今絳州龍門縣辛氏三秦記曰河津一名龍門水險不通魚鼈之屬莫能上江海大魚數千薄集龍門下不得上上則為龍被皮義翻}
徵東海相劉寛為尚書令寛崎之子也|{
	劉崎事順帝為司徒崎丘宜翻}
歷典三郡|{
	賢曰東海王彊曾孫臻之相也按寛傳云是年自東海相徵為尚書令遷南陽太守典歷三郡}
温仁多恕雖在倉卒|{
	卒讀曰猝}
未嘗疾言遽色吏民有過但用蒲鞭罰之|{
	古者鞭用生皮為之}
示辱而已終不加苦每見父老慰以農里之言少年勉以孝悌之訓人皆悦而化之

九年春正月辛卯朔日有食之詔公卿郡國舉至孝太常趙典所舉荀爽對策曰昔者聖人建天地之中而謂之禮衆禮之中昏禮為首陽性純而能施隂體順而能化以禮濟樂節宣其氣|{
	爽言正指帝多内寵也左傳晉侯有疾醫和視之曰疾不可為也是謂疾如蠱非鬼非食惑以喪志公曰女不可近乎對曰節之先王之樂所以節百事也天有六氣過則為災於是乎節宣其氣也施式智翻}
故能豐子孫之祥致老夀之福及三代之季淫而無節陽竭於上隂隔於下故周公之戒曰時亦罔或克壽|{
	尚書無逸之辭}
傳曰趾適屨孰云其愚何與斯人追欲喪軀誠可痛也|{
	賢曰適猶從也言喪身之愚甚於截趾也喪息浪翻}
臣竊聞後宫采女五六千人從官侍使復在其外|{
	從才用翻從官謂後宫有爵秩而常從者侍使則侍后妃貴人左右而給使令未有爵秩者也復扶又翻下同}
空賦不辜之民以供無用之女百姓窮困於外隂陽隔塞於内|{
	塞悉則翻}
故感動和氣災異屢臻臣愚以為諸未幸御者一皆遣出使成妃合|{
	妃讀曰配}
此誠國家之大福也詔拜郎中司隸豫州饑死者什四五至有滅戶者|{
	戶謂著戶籍於官者也滅}


|{
	戶則無老無弱皆死於飢無復遺種也}
詔徵張奐為大司農復以皇甫規代為度遼將軍規自以為連在大位欲求退避數上病不見聽|{
	數所角翻上時掌翻}
會友人喪至規越界迎之因令客密告并州刺史胡芳言規擅遠軍營|{
	遠于願翻}
當急舉奏芳曰威明欲避第仕塗|{
	度遼將軍屯西河界并州刺史所部也皇甫規字威明賢曰言欲歸第避仕宦之塗也}
故激發我耳吾當為朝廷愛才|{
	為于偽翻}
何能申此子計邪遂無所問 夏四月濟隂東郡濟北平原河水清|{
	濟子禮翻}
司徒許栩免五月以太常胡廣為司徒庚午上親祠老子於濯龍宫以文罽為壇飾|{
	罽居例翻西夷織毛為布曰罽}
淳金釦器|{
	釦去厚翻說文金飾器口}
設華蓋之坐用郊天樂|{
	史言其非禮坐徂卧翻}
鮮卑聞張奐去招結南匈奴及烏桓同叛六月南匈奴烏桓鮮卑數道入塞寇掠緣邊九郡秋七月鮮卑復入塞誘引東羌與共盟詛|{
	詛莊助翻}
於是上郡沈氐安定先零諸種|{
	種章勇翻}
共寇武威張掖緣邊大被其毒|{
	被皮義翻}
詔復以張奐為護匈奴中郎將以九卿秩|{
	護匈奴中郎將秩比二千石九卿秩中二千石}
督幽并涼三州及度遼烏桓二營|{
	度遼將軍及護烏桓校尉營也}
兼察刺史二千石能否 初帝為蠡吾侯受學於甘陵周福及即位擢福為尚書時同郡河南尹房植有名當朝|{
	朝直遥翻}
鄉人為之謡曰天下規矩房伯武因師獲印周仲進|{
	房植字伯武周福字仲進}
二家賓客互相譏揣|{
	揣初委翻揣度也量也度量其輕重長短而為譏議也}
遂各樹朋徒漸成尤隙由是甘陵有南北部黨人之議自此始矣汝南太守宗資以范滂為功曹南陽太守成瑨以岑晊為功曹|{
	瑨即刃翻晊音質}
皆委心聽任使之褒善糾違肅清朝府|{
	朝郡朝也公卿牧守所居皆曰府朝直遥翻}
滂尤剛勁疾惡如讐滂甥李頌素無行中常侍唐衡以屬資|{
	行下孟翻屬之欲翻}
資用為吏滂寢而不召資遷怒捶書佐朱零|{
	百官志郡閣下及諸曹各有書佐幹主文書}
零仰曰范滂清裁|{
	賢曰裁音才代翻裁制也言其清而有制也}
今日寧受笞而死滂不可違資乃止郡中中人以下莫不怨之於是二郡為謡曰汝南太守范孟博南陽宗資主畫諾|{
	孟博范滂字也諾者隨言而應無所違也畫諾猶畫可也}
南陽太守岑公孝弘農成瑨但坐嘯|{
	公孝岑晊字也嘯吟也言但坐而吟嘯於郡事無所豫也}
太學諸生三萬餘人郭泰及潁川賈彪為其冠|{
	冠古玩翻}
與李膺陳蕃王暢更相褒重|{
	更工衡翻}
學中語曰天下模楷李元禮不畏強禦陳仲舉天下俊秀王叔茂|{
	李膺字元禮陳蕃字仲舉王暢字叔茂}
於是中外承風競以臧否相尚|{
	否音鄙}
自公卿以下莫不畏其貶議屣履到門|{
	屣履者履不躡跟也}
宛有富賈張汎者|{
	宛於元翻賈音古 考異曰陳蕃傳作張汜謝承書作張子禁今從岑晊傳}
與後宫有親又善雕鏤玩好之物頗以賂遺中官以此得顯位用埶縱横|{
	鏤郎豆翻好呼到翻遺于季翻横戶孟翻}
岺晊與賊曹史張牧|{
	賊曹主盜賊事}
勸成瑨收捕汎等既而遇赦瑨竟誅之并收其宗族賓客殺二百餘人後乃奏聞小黄門晉陽趙津貪暴放恣為一縣巨患太原太守平原劉瓆|{
	丁度集韻瓆職日翻}
使郡吏王允討捕亦於赦後殺之于是中常侍侯覽使張汎妻上書訟寃宦者因緣譖訴瑨瓆帝大怒徵瑨瓆皆下獄|{
	下遐稼翻}
有司承旨奏瑨瓆罪當棄市山陽太守翟超|{
	翟萇伯翻}
以郡人張儉為東都督郵侯覽家在防東|{
	百官志郡有五部督郵監屬縣郡國志防東縣屬山陽郡賢曰故城在今兖州金鄉縣南}
殘暴百姓覽喪母還家|{
	喪息浪翻}
大起塋冢|{
	塋音營}
儉舉奏覽罪而覽伺候遮|{
	昨結翻後乃作截}
章竟不上|{
	上時掌翻}
儉遂破覽冢宅籍没資財具奏其狀復不得御|{
	復扶又翻御進也謂其奏不得進也 考異曰袁紀儉行部至平陵逢覽母儉按劔怒曰何等女子干督郵此非賊邪使吏卒收覽母殺之追擒覽家屬賓客死者百餘人皆僵尸道路伐其園宅井堙木刋雞犬器物悉無遺類苑康傳亦云張儉殺侯覽母按其宗黨或有迸匿太山界者康窮相收掩無得遺脱覽大怨之徵詣廷尉坐徙日南案侯覽傳云覽喪母還家陳蕃傳云翟超沒入侯覽財產坐髠鉗皆不云儉殺其母若果殺之則苑康不止徙日南也侯覽傳又云建寧二年喪母蓋以誅黨人在其年致此誤耳}
徐璜兄子宣為下邳令暴虐尤甚嘗求故汝南太守李暠女不能得|{
	暠古老翻}
遂將吏卒至暠家載其女歸戲射殺之|{
	將即亮翻射而亦翻}
東海相汝南黄浮聞之收宣家屬無少長悉考之|{
	少詩照翻長知兩翻}
掾史以下固爭浮曰徐宣國賊今日殺之明日坐死足以瞑目矣即案宣罪棄市暴其尸|{
	暴步木翻}
於是宦官訴寃於帝帝大怒超浮並坐髠鉗輸作右校|{
	校戶教翻}
太尉陳蕃司空劉茂共諫請瑨瓆超浮等罪 |{
	考異曰陳蕃傳又有司徒劉矩按時胡廣為司徒非矩也}
帝不悦有司劾奏之茂不敢復言|{
	劾戶槩翻又戶得翻復扶又翻下同}
蕃乃獨上疏曰今寇賊在外四支之疾内政不理心腹之患臣寢不能寐食不能飽實憂左右日親忠言日疎内患漸積外難方深|{
	難乃旦翻}
陛下超從列侯繼承天位|{
	賢曰言帝以蠡吾侯即位}
小家畜產百萬之資子孫尚恥愧失其先業况乃產兼天下受之先帝而欲懈怠以自輕忽乎誠不愛已不當念先帝得之勤苦邪前梁氏五侯毒徧海内天啟聖意收而戮之|{
	賢曰五侯謂胤讓淑忠戟與冀同時誅事見冀傳}
天下之議冀當小平明鑑未遠覆車如昨而近習之權復相扇結小黄門趙津大猾張汎等肆行貪虐姦媚左右前太原太守劉瓆南陽太守成瑨糾而戮之雖言赦後不當誅殺原其誠心在乎去惡|{
	去羌呂翻}
至於陛下有何悁悁|{
	說文曰悁悁恚忿也悁縈年翻}
而小人道長熒惑聖聽遂使天威為之發怒|{
	長知兩翻為於偽翻}
必加刑譴已為過甚况乃重罸令伏歐刀乎又前山陽太守翟超東海相黄浮奉公不橈疾惡如讎超没侯覽財物浮誅徐宣之罪並蒙刑坐不逢赦恕覽之從横|{
	從才用翻横戶孟翻}
没財已幸宣犯釁過死有餘辜昔丞相申屠嘉召責鄧通雒陽令董宣折辱公主而文帝從而請之光武加以重賞|{
	申屠嘉事見十四卷文帝後二年董宣事見四十三卷光武建武十九年}
未聞二臣有專命之誅而今左右羣豎惡傷黨類|{
	惡烏路翻}
妄相交構致此刑譴聞臣是言當復嗁訴陛下深宜割塞近習與政之源|{
	嗁與啼同塞悉則翻與讀曰豫}
引納尚書朝省之士|{
	朝直遥翻}
簡練清高斥黜佞邪如是天和於上地洽於下休禎符瑞豈遠乎哉帝不納宦官由此疾蕃彌甚選舉奏議輒以中詔譴郤長史以下多至抵罪猶以蕃名臣不敢加害平原襄楷詣闕上疏曰臣聞皇天不言以文象設教臣竊見太微天廷五帝之坐而金火罸星揚光其中|{
	天文志太微天子庭也五帝之坐也賢曰太白金也熒惑火也天文志曰逆夏令傷火氣罸見熒惑逆秋令傷金氣罸見太白故金火並為罰星也坐徂卧翻}
於占天子凶又俱入房心|{
	天文志房四星為明堂天子布政之宫也心三星天王正位也中星曰明堂天子位焉前星為太子後星為庶子}
法無繼嗣前年冬大寒殺鳥獸害魚鼈城傍竹栢之葉有傷枯者|{
	續漢志曰延熹七年雒陽城傍竹栢葉有傷枯者 考異曰帝紀此年十二月書雒城傍竹栢枯傷誤也}
臣聞於師曰栢傷竹枯不出二年天子當之今自春夏以來連有霜雹及大雨雷電臣作威作福刑罰急刻之所感也太原太守劉瓆南陽太守成瑨志除姦邪其所誅翦皆合人望而陛下受閹豎之譖乃遠加考逮三公上書乞哀瓆等不見採察而嚴被譴讓憂國之臣將遂杜口矣臣聞殺無罪誅賢者禍及三世|{
	黄石公三畧曰傷賢者殃及三世蔽賢者身當其害達賢者福流子孫疾賢者名不全}
自陛下即位以來頻行誅罸梁寇孫鄧並見族滅|{
	賢曰梁冀寇榮孫夀鄧萬世等也}
其從坐者又非其數李雲上書明主所不當諱杜衆乞死諒以感悟聖朝曾無赦宥而并被殘戮天下之人咸知其寃|{
	事見上卷二年被皮義翻}
漢興以來未有拒諫誅賢用刑太深如今者也昔文王一妻誕至十子|{
	史記大姒文王正妃也其長子伯邑考次武王發次管叔鮮次周公旦次蔡叔度次曹叔振鐸次成叔武次霍叔處次康叔封次季載同母兄弟十人}
今宫女數千未聞慶育宜修德省刑以廣螽斯之祚|{
	螽斯言后妃不妬忌子孫衆多也}
案春秋以來及古帝王未有河清臣以為河者諸侯位也|{
	孝經援神契曰五嶽視三公四瀆視諸侯}
清者屬陽濁者屬隂河當濁而反清者隂欲為陽諸侯欲為帝也京房易傳曰河水清天下平今天垂異地吐妖人癘疫三者並時而有河清猶春秋麟不當見而見孔子書之以為異也|{
	公羊傳西狩獲麟有以告者孔子曰孰為來哉孰為來哉蓋以為異也見賢遍翻}
願賜清間極盡所言書奏不省|{
	間讀曰閑省悉井翻}
十餘日復上書曰臣聞殷紂好色妲已是出|{
	好呼到翻下同殷紂冒色有蘇氏以妲已女之妲當割翻}
葉公好龍真龍游廷|{
	葉公子高好龍天龍聞而降之窺頭於牖}
今黄門常侍天刑之人|{
	謂已受熏腐之刑得罪于天者也}
陛下愛待兼倍常寵係嗣未兆豈不為此|{
	為于偽翻}
又聞宫中立黄老浮屠之祠|{
	賢曰浮屠即佛陁聲之轉耳謂佛也}
此道清虛貴尚無為好生惡殺省慾去奢|{
	惡烏路翻去羌呂翻}
今陛下耆欲不去|{
	耆讀曰嗜}
殺罸過理既乖其道豈獲其祚哉浮屠不三宿桑下不欲久生恩愛精之至也|{
	賢曰言浮屠之人寄桑下者不經三宿便即移去示無愛戀之心也}
其守一如此乃能成道今陛下淫女艶婦極天下之麗甘肥飲美單天下之味|{
	單與殫同}
奈何欲如黄老乎書上即召入詔尚書問狀楷言古者本無宦臣武帝末數游後宫始置之耳|{
	數所角翻}
尚書承旨|{
	承旨謂承宦官風指也}
奏楷不正辭理而違背經蓻假借星宿|{
	背蒲妹翻宿音秀}
造合私意|{
	合音閤牽合也}
誣上罔事請下司隸正楷罪灋|{
	下遐嫁翻}
收送雒陽獄帝以楷言雖激切然皆天文恒象之數故不誅猶司寇論刑|{
	司宼二歲刑也}
自永平以來臣民雖有習浮屠術者而天子未之好至帝始篤好之|{
	好呼到翻}
常躬自禱祠由是其法浸盛故楷言及之符節令汝南蔡衍|{
	百官志符節令秩六百石為符節臺率主符節事屬少府}
議郎劉瑜表救成瑨劉瓆言甚切厲亦坐免官瑨瓆竟死獄中瑨瓆素剛直有經術知名當時故天下惜之岑晊張牧逃竄獲免晊之亡也親友競匿之賈彪獨閉門不納時人望之|{
	賢曰望怨也余謂望責望也}
彪曰傳言相時而動無累後人|{
	左傳之文相息亮翻累力瑞翻}
公孝以要君致釁|{
	要一遥翻}
自遺其咎|{
	遺于季翻}
吾已不能奮戈相待反可容隱之乎於是咸服其裁正彪嘗為新息長|{
	新息縣屬汝南郡賢曰今豫州縣長知兩翻}
小民困貧多不養子彪嚴為其制與殺人同罪城南有盜刼害人者北有婦人殺子者彪出案驗掾吏欲引南|{
	引南者引車南行者}
彪怒曰賊寇害人此則常理母子相殘逆天違道遂驅車北行案致其罪城南賊聞之亦面縛自首|{
	首式救翻}
數年間人養子者以千數曰此賈父所生也皆名之為賈 河南張成善風角|{
	賢曰風角謂候四方四隅之風以占吉凶也}
推占當赦教子殺人司隸李膺督促收捕既而逢宥獲免膺愈懷憤疾竟案殺之 |{
	考異曰黨錮傳云膺為河南尹按膺此事非作尹時也}
成素以方伎交通宦官|{
	伎渠綺翻}
帝亦頗訊其占|{
	訊問也}
宦官教成弟子牢修上書告膺等養太學游士交結諸郡生徒更相驅馳共為部黨誹訕朝廷|{
	更工衡翻說文曰誹謗也蒼頡篇誹非也}
疑亂風俗 |{
	考異曰袁紀作牢順今從范書}
於是天子震怒班下郡國|{
	下遐稼翻下同}
逮捕黨人布告天下使同忿疾案經三府|{
	案文案也以考驗為義}
太尉陳蕃卻之曰今所案者皆海内人譽憂國忠公之臣此等猶將十世宥也|{
	左傳晉范宣子囚叔向祁奚見宣子曰謀而鮮過惠訓不倦者叔向有焉猶將十世宥之以勸能者}
豈有罪名不章而致收掠者乎|{
	掠音亮}
不肯平署|{
	賢曰平署猶連署也}
帝愈怒遂下膺等於黄門北寺獄|{
	時宦官專權置黄門北寺獄自武帝以來中都官詔獄所未有也下遐稼翻}
其辭所連及太僕潁川杜密御史中丞陳翔及陳寔范滂之徒二百餘人或逃遁不獲皆懸金購募使者四出相望陳寔曰吾不就獄衆無所恃乃自往請囚范滂至獄獄吏謂曰凡坐繫者皆祭臯陶滂曰臯陶古之直臣知滂無罪將理之於帝|{
	賢曰帝謂天也陶音遥}
如其有罪祭之何益衆人由此亦止陳蕃復上書極諫|{
	復扶又翻}
帝諱其言切託以蕃辟召非其人策免之 |{
	考異曰袁紀李膺下獄在九月范書蕃免在七月蕃傳上書極諫曰膺等或禁錮閉隔或死徙非所云云按膺等赦出在明年六月再下獄死徙在建寧二年十月蕃既以此年七月免則蕃傳所云疑非蕃書也又袁紀無陳蕃免事靈帝即位以太尉陳蕃為太傅按蕃免後有太尉周景蓋袁紀誤也}
時黨人獄所染逮者皆天下名賢|{
	染謂獄辭所汙染也逮謂連及也}
度遼將軍皇甫規自以西州豪桀恥不得與|{
	與讀曰預}
乃自上言臣前薦故大司農張奐是附黨也又臣昔論輸左校時大學生張鳳等上書訟臣是為黨人所附也|{
	薦張奐事見上卷六年張鳳上書事見五年}
臣宜坐之朝廷知而不問杜密素與李膺名行相次|{
	行下孟翻下同}
時人謂之李杜故同時被繫密嘗為北海相行春到高密|{
	百官志凡郡國守相嘗以春行所主縣勸民農桑振救乏絶高密縣屬北海國}
見鄭玄為鄉嗇夫知其異器即召署郡職遂遣就學卒成大儒|{
	卒子恤翻}
後密去官還家每謁守令多所陳託同郡劉勝亦自蜀郡告歸鄉里閉門掃軌|{
	賢曰軌車迹也言絶人事}
無所干及太守王昱謂密曰劉季陵清高士|{
	劉勝字季陵}
公卿多舉之者密知昱以激已對曰劉勝位為大夫見禮上賓|{
	位為大夫謂在朝列也見禮上賓謂郡守接遇之也}
而知善不薦聞惡無言隱情惜已自同寒蟬|{
	賢曰寒蟬謂寂默也楚辭曰悲哉秋之為氣也蟬寂寞而無聲}
此罪人也今志義力行之賢而密達之違道失節之士而密糾之使明府賞刑得中令問休揚不亦萬分之一乎昱慙服待之彌厚 九月以光祿勲周景為太尉 司空劉茂免 冬十二月以光祿勲汝南宣酆為司空|{
	姓譜宣以諡為氏}
以越騎校尉竇武為城門校尉武在位多辟名士清身疾惡禮賂不通妻子衣食裁充足而已得兩宫賞賜|{
	兩宫謂天子及皇后}
悉散與太學諸生及匄施貧民|{
	匄居太翻與也施式豉翻}
由是衆譽歸之 匈奴烏桓聞張奐至皆相率還降|{
	降戶江翻}
凡二十萬口奐但誅其首惡餘皆慰納之唯鮮卑出塞去朝廷患檀石槐不能制遣使持印綬封為王欲與和親檀石槐不肯受而寇抄滋甚|{
	抄楚交翻}
自分其地為三部從右北平以東至遼東接夫餘濊貊二十餘邑為東部|{
	夫音扶濊音穢貊莫百翻}
從右北平以西至上谷十餘邑為中部從上谷以西至敦煌烏孫二十餘邑為西部各置大人領之|{
	觀此則夷狄亦有邑居矣檀石槐蓋盡有匈奴故地敦徒門翻}


資治通鑑卷五十五
