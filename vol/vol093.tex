\chapter{資治通鑑卷九十三}
宋 司馬光 撰

胡三省 音註

晉紀十五|{
	起閼逢涒灘盡強圉大淵獻凡四年}


肅宗明皇帝下

太寜二年春正月王敦誣周嵩周筵與李脫謀為不軌收嵩筵於軍中殺之遣參軍賀鸞就沈充於吳盡殺周札諸兄子進兵襲會稽|{
	會工外翻}
札拒戰而死 後趙將兵都尉石瞻寇下邳彭城取東莞東海|{
	東莞縣漢屬琅邪郡莞音官武帝泰始元年分琅邪立東莞郡將即亮翻}
劉遐退保泗口|{
	水經註泗水自淮陽城東流逕角城北而東南流注于淮謂之泗口杜佑曰泗口在今臨淮郡宿遷縣界}
司州刺史石生擊趙河南太守尹平於新安斬之|{
	新安縣漢屬弘農郡晉屬河南郡守式又翻下同}
掠五千餘戶而歸自是二趙搆隙日相攻掠河東弘農之間民不聊生矣|{
	河東弘農二趙之界上也}
石生寇許潁|{
	許昌潁川同是一郡地}
俘獲萬計攻郭誦于陽翟誦與戰大破之生退守康城|{
	魏收地形志陽翟縣有康城}
後趙汲郡内史石聰聞生敗馳救之進攻司州刺史李矩潁川太守郭默皆破之 成主雄后任氏無子|{
	任音壬}
有妾子十餘人雄立其兄蕩之子班為太子使任后母之羣臣請立諸子雄曰吾兄先帝之嫡統有奇才大功事垂克而早世朕常悼之|{
	蕩死見八十五卷惠帝太安二年}
且班仁孝好學必能負荷先烈太傅驤司徒王達諫曰先王立嗣必子者所以明定分而防簒奪也|{
	好呼到翻荷下可翻又如字驤思將翻分扶問翻}
宋宣公吳餘祭足以觀矣|{
	漢書曰立嗣必子所從來遠矣公羊傳曰宋宣公謂繆公曰以吾愛與夷則不若愛汝以為社稷宗廟主則與夷不若汝盍終為君矣宣公死繆公立逐其二子莊公馮與左師勃而致國乎與夷故君子大居正宋之禍宣公為之也吳子謁餘祭夷昧與季子同母季子弱而才兄弟皆愛之同欲立之以為君謁曰今若退而與季子國季子猶不受也請無與子而與弟弟兄迭為君而致國乎季子夷昧死則國宜之季子季子使而亡焉僚者長庶也即之季子使反而君之闔閭曰先君所以不與子國而與弟者凡為季子故也將從先君之命歟則國宜之季子如不從先君之命則我宜立者也僚惡得為君乎於是使專諸刺僚張守節曰祭側界翻昧莫葛翻}
雄不聽驤退而流涕曰亂自此始矣|{
	為下雄諸子殺班張本}
班為人謙㳟下士|{
	下遐稼翻}
動遵禮灋雄每有大議輒令豫之 夏五月甲申張茂疾病執世子駿手泣曰吾家世以孝友忠順著稱今雖天下大亂汝奉承之不可失也且下令曰吾官非王命苟以集事豈敢榮之死之日當以白帢入棺勿以朝服斂是日薨|{
	帢若洽翻朝直遥翻斂力贍翻}
愍帝使者史淑在姑臧|{
	長安覆沒淑無所歸故留姑臧使疏吏翻}
左長史汜禕右長史馬謨等|{
	汜音凡禕吁韋翻}
使淑拜駿大將軍凉州牧西平公赦其境内前趙主曜遣使贈茂太宰諡曰成烈王拜駿上大將軍凉州牧凉王 王敦疾甚矯詔拜王應為武衛將軍以自副以王含為驃騎大將軍開府儀同三司|{
	驃匹妙翻}
錢鳳謂敦曰脫有不諱便當以後事付應邪敦曰非常之事非常人所能為|{
	司馬相如難蜀父老曰蓋世必有非常之人然後有非常之事}
且應年少|{
	少詩沼翻}
豈堪大事我死之後莫若釋兵散衆歸身朝廷保全門戶上計也退還武昌收兵自守貢獻不廢中計也及吾尚存悉衆而下萬一僥倖下計也鳳謂其黨曰公之下計乃上策也遂與沈充定謀俟敦死即作亂|{
	以王敦之狠戾而濟之以沈充錢鳳所謂凶德參會}
又以宿衛尚多奏令三番休二初帝親任中書令温嶠敦惡之|{
	惡烏路翻}
請嶠為左司馬嶠乃繆為勤敬|{
	繆靡幼翻詐也}
綜其府事時進密謀以附其欲深結錢鳳為之聲譽每曰錢世儀精神滿腹嶠素有藻鑑之名|{
	錢鳳字世儀藻鑑謂善於人倫藻鑑也人有美質而加之褒飾謂之黼藻如衣裳之加藻火黼黻也鑑所以别妍醜故明於知人而能褒奬後進者有藻鑑之名}
鳳甚悦深與嶠結好|{
	好呼到翻}
會丹楊尹缺|{
	晉都建康以丹陽太守為尹宋齊梁皆因之洪适曰西漢丹陽郡則治宛陵丹陽縣則今之建康也東漢史皆作丹陽西晉移郡於建業元帝改太守為丹陽尹地理志曰山多赤柳故名它書載漢晉此郡少有從木者至唐天寶年始以京口為丹陽郡改曲阿為丹陽縣皆非漢舊壤也}
嶠言於敦曰京尹咽喉之地|{
	咽音烟}
公宜自選其才恐朝廷用人或不盡理敦然之問嶠誰可者嶠曰愚謂無如錢鳳鳳亦推嶠嶠偽辭之敦不聽六月表嶠為丹楊尹且使覘伺朝廷|{
	覘丑廉翻又丑艶翻}
嶠恐既去而錢鳳於後間止之|{
	間古莧翻}
因敦餞别嶠起行酒至鳳鳳未及飲嶠偽醉以手版擊鳳幘墜作色曰錢鳳何人温太真行酒而敢不飲|{
	温嶠字太真}
敦以為醉兩釋之嶠臨去與敦别涕泗横流出閣復入者再三|{
	復扶又翻}
行後鳳謂敦曰嶠於朝廷甚密而與庾亮深交未可信也敦曰太真昨醉小加聲色何得便爾相讒嶠至建康盡以敦逆謀告帝請先為之備又與庾亮共畫討敦之謀敦聞之大怒曰吾乃為小物所欺與司徒導書曰太真别來幾日作如此事當募人生致之自拔其舌|{
	王敦遥制朝權其所甚害者如郗鑒温嶠終不得以肆其毒以此知建康綱紀尚能自立也}
帝將討敦以問光禄勲應詹詹勸成之帝意遂决丁卯加司徒導大都督領揚州刺史以温嶠都督東安北部諸軍事|{
	以下文應詹都督橋南諸軍觀之則東安北部謂秦淮水北諸軍也}
與右將軍卞敦守石頭 |{
	考異曰敦傳云王敦表為征虜將軍都督石頭軍事明帝討敦以為鎮南將軍假節今從明帝紀}
應詹為護軍將軍都督前鋒及朱雀橋南諸軍事郗鑒行衛將軍都督從駕諸軍事|{
	郗丑之翻從才用翻}
庾亮領左衛將軍以吏部尚書卞壼行中軍將軍|{
	壼苦本翻}
郗鑒以為軍號無益事實固辭不受請召臨淮太守蘇峻兖州刺史劉遐同討敦|{
	夫理順者難恃勢弱則不支以敦鳳同惡相濟率大衆以犯闕雖諸公忠赤若只以臺中見兵拒之是復周戴石頭之事微郗鑒建請而召劉遐蘇峻殆矣}
詔徵峻遐及徐州刺史王邃豫州刺史祖約廣陵太守陶瞻等入衛京師帝屯于中堂|{
	按蕭子顯齊書高帝紀桂陽王休範之反諸貴會議帝曰中堂舊是置兵地領軍宜屯宣陽門為諸軍節度則中堂當在宣陽門外}
司徒導聞敦疾篤帥子弟為敦發哀|{
	帥讀曰率為于偽翻}
衆以為敦信死咸有奮志於是尚書騰詔下敦府|{
	下遐嫁翻}
列敦罪惡曰敦輒立兄息以自承代未有宰相繼體而不由王命者也|{
	息子也謂以兄含子應為嗣也}
頑凶相奬無所顧忌志騁凶醜以窺神器天不長姦|{
	聘丑郢翻長丁丈翻}
敦以隕斃鳳承凶宄彌復煽逆|{
	復扶又翻}
今遣司徒導等虎旅三萬十道並進平西將軍邃等精鋭三萬水陸齊勢朕親統諸軍討鳳之罪有能殺鳳送首封五千戶侯 |{
	考異曰晉春秋此詔在王導為敦發喪前故云有能斬送敦首封萬戶侯賞布萬匹按此詔云敦以隕斃是稱敦已死也不應復購敦首今從敦傳}
諸文武為敦所授用者一無所問無或猜嫌以取誅滅敦之將士從敦彌年違離家室|{
	將即亮翻下自將親將同離力智翻}
朕甚愍之其單丁在軍皆遣歸家終身不調|{
	單丁謂家止有男丁一人無兼次者調徒釣翻}
其餘皆與假三年|{
	假居訝翻}
休訖還臺當與宿衛同例三番|{
	謂三番休二也}
敦見詔甚怒而病轉篤不能自將將舉兵伐京師使記室郭璞筮之璞曰無成敦素疑璞助温嶠庾亮及聞卦凶乃問璞曰卿更筮吾夀幾何璞曰思向卦明公起事必禍不久若住武昌夀不可測敦大怒曰卿夀幾何曰命盡今日日中敦乃收璞斬之敦使錢鳳及冠軍將軍鄧岳前將軍周撫等帥衆向京師|{
	冠古玩翻帥讀曰率}
王含謂敦曰此乃家事吾當自行於是以含為元帥|{
	帥所類翻}
鳳等問曰事克之日天子云何敦曰尚未南郊何得稱天子便盡卿兵勢保護東海王及裴妃而已|{
	元帝以第三子冲奉東海王越後裴妃越妃也}
乃上疏以誅姦臣温嶠等為名秋七月壬申朔王含等水陸五萬奄至江寜南岸|{
	武帝太康二年分秣陵立臨江縣二年更名江寜南岸即秦淮南岸也考異曰敦傳及晉春秋皆云三萬今從明帝紀}
人情恟懼温嶠移屯水北燒朱雀桁以挫其鋒|{
	恟許拱翻桁與航同}
含等不得渡帝欲親將兵擊之聞橋已絶大怒嶠曰今宿衛寡弱徵兵未至若賊豕突危及社稷宗廟且恐不保何愛一橋乎司徒導遺含書曰近承大將軍困篤|{
	參問起居謂之參承詗候安否謂之詗承遺子季翻}
或云已有不諱尋知錢鳳大嚴欲肆姦逆謂兄當抑制不逞|{
	言當抑制鳳等使不得逞其凶逆也}
還藩武昌今乃與犬羊俱下兄之此舉謂可得如大將軍昔年之事乎|{
	謂如元帝永昌元年敦克石頭時也}
昔者佞臣亂朝|{
	謂刁協劉隗也朝直遥翻}
人懷不寜如導之徒心思外濟|{
	言思投外以自濟也}
今則不然大將軍來屯于湖漸失人心君子危怖|{
	怖普布翻}
百姓勞弊臨終之日委重安期安期斷乳幾日又於時望便可襲宰相之迹耶|{
	王應字安期斷讀曰短}
自開闢以來頗有宰相以孺子為之者乎諸有耳者皆知將為禪代非人臣之事也|{
	謂此事深駭衆聽皆知敦應謀簒}
先帝中興遺愛在民聖主聰明德洽朝野兄乃欲妄萌逆節凡在人臣誰不憤歎導門小大受國厚恩今日之事明目張膽為六軍之首寜為忠臣而死不為無賴而生矣含不答或以為王含錢鳳衆力百倍苑城小而不固|{
	苑城蓋孫氏都秣陵所築晉置建康於秣陵水北南渡建都依苑城以為守}
宜及軍勢未成大駕自出拒戰郗鑒曰羣逆縱逸勢不可當可以謀屈難以力競且含等號令不一抄盜相尋|{
	抄楚交翻}
吏民懲往年暴掠皆人自為守乘逆順之勢何憂不克且賊無經畧遠圖惟恃豕突一戰曠日持久必啓義士之心令智力得展今以此弱力敵彼彊寇决勝負於一朝定成敗於呼吸萬一蹉跌|{
	蹉七何翻跌徒結翻}
雖有申胥之徒義存投袂|{
	左傳吳人入郢楚大夫申包胥赴秦求救卒以存楚投袂言匆遽也傳曰楚子聞之投袂而起}
何補於既往哉帝乃止帝帥諸軍出屯南皇堂癸酉夜募壯士遣將軍段秀中軍司馬曹渾等帥甲卒千人渡水掩其未備平旦戰於越城|{
	越城在秦淮南帥讀曰率下同}
大破之斬其前鋒將何康秀匹磾之弟也|{
	磾丁奚翻}
敦聞含敗大怒曰我兄老婢耳門戶衰世事去矣顧謂參軍呂寶曰我當力行因作勢而起困乏復卧|{
	氣不能充體為困力不能舉身為乏}
乃謂其舅少府羊鑒及王應曰|{
	少詩照翻}
我死應便即位先立朝廷百官然後營葬事敦尋卒應祕不發喪裹尸以席蠟塗其外埋於廳事中與諸葛瑶等日夜縱酒淫樂|{
	樂音洛}
帝使吳興沈楨說沈充|{
	說輸芮翻}
許以為司空充曰三司具瞻之重豈吾所任|{
	任音壬}
幣厚言甘古人所畏也|{
	詩節南山曰赫赫師尹民具爾瞻左傳晉郤芮曰幣重而言甘誘我也}
且丈夫共事終始當同豈可中道改易人誰容我乎遂舉兵趣建康|{
	趣七喻翻}
宗正卿虞潭以疾歸會稽|{
	按漢晉以來宗正列於九卿然未以卿字繫官梁置十一寺始繫卿字此卿字衍會工外翻}
聞之起兵餘姚以討充|{
	餘姚縣屬會稽郡}
帝以潭領會稽内史前安東將軍劉超宣城内史鍾雅皆起兵以討充義興人周蹇殺王敦所署太守劉芳|{
	晉惠帝永興元年分吳興之陽羨丹楊之永世立義興都}
平西將軍祖約逐敦所署淮南太守任台|{
	約屯夀春故得逐台任音壬}
沈充帥衆萬餘人與王含軍合司馬顧颺說充曰今舉大事而天子已扼其咽喉|{
	咽音烟}
鋒摧氣沮相持日久必致禍敗|{
	沮在呂翻}
今若決破柵塘因湖水以灌京邑|{
	此即玄武湖水也在建康城北今在上元縣北十里}
乘水勢縱舟師以攻之此上策也藉初至之銳并東西軍之力|{
	東軍謂沈充軍西軍謂王含錢鳳等軍也}
十道俱進衆寡過倍理必摧䧟中策也轉禍為福召錢鳳計事因斬之以降|{
	降戶江翻}
下策也充皆不能用颺逃歸于吳丁亥劉遐蘇峻等帥精卒萬人至帝夜見勞之|{
	勞力到翻}
賜將士各有差沈充錢鳳欲因北軍初到疲困擊之乙未夜充鳳從竹格渚渡淮|{
	秦淮在今建康上元縣南三里秦始皇時望氣者言金陵有天子氣使鑿山為瀆以斷地脉故曰秦淮或云淮水發源屈曲不類人工}
護軍將軍應詹建威將軍趙胤等拒戰不利充鳳至宣陽門|{
	晉都建康外城環之以籬諸門皆用洛城門名宣陽門在城南面}
拔柵將戰劉遐蘇峻自南塘横擊大破之|{
	晉都建康自江口沿淮築堤南塘秦淮之南塘岸也}
赴水死者三千人遐又破沈充於青溪|{
	青溪水發源於鍾山接於秦淮吳孫權鑿城北塹以洩玄武湖水}
尋陽太守周光聞敦舉兵帥千餘人來赴|{
	沈約曰尋陽本縣名因水名縣水南注江漢屬廬江郡惠帝永興元年分廬江武昌立尋陽郡治柴桑縣}
既至求見敦王應辭以疾光退曰今我遠來而不得見公其死乎遽見其兄撫曰王公已死兄何為與錢鳳作賊衆皆愕然丙申王含等燒營夜遁丁酉帝還宫大赦惟敦黨不原命庾亮督蘇峻等追沈充於吳興温嶠督劉遐等追王含錢鳳於江寜分命諸將追其黨與劉遐軍人頗縱虜掠嶠責之曰天道助順故王含勦絶|{
	勦子少翻}
豈可因亂為亂也遐惶恐拜謝王含欲犇荆州王應曰不如江州|{
	荆州王舒江州王彬}
含曰大將軍平素與江州云何而欲歸之應曰此乃所以宜歸也江州當人彊盛時能立同異此非常人所及今覩困厄必有愍惻之心荆州守文豈能意外行事邪|{
	王應之見猶能出乎尋常此敦所以以之為後歟能立同異謂哭周顗數敦罪及諫敦為逆也}
含不從遂奔荆州王舒遣軍迎之沈含父子於江|{
	沈持林翻}
王彬聞應當來密具舟以待之不至深以為恨錢鳳走至闔廬洲|{
	闔廬洲在江中賀循曰江中劇地惟有闔廬一處地勢險奥亡逃所聚}
周光斬之詣闕自贖 |{
	考異曰晉春秋云戴淵弟良斬鳳今從敦傳}
沈充走失道誤入故將吳儒家儒誘充内重壁中|{
	重壁複壁也重直龍翻}
因笑謂充曰三千戶侯矣|{
	時臺格募斬錢鳳者封五千戶侯斬沈充者封三千戶侯}
充曰爾以義存我我家必厚報汝若以利殺我我死汝族滅矣儒遂殺之傳首建康敦黨悉平充子勁當坐誅鄉人錢舉匿之得免其後勁竟滅吳氏有司發王敦瘞|{
	瘞於計翻}
出尸焚其衣冠跽而斬之|{
	跽巨几翻跪也}
與沈充首同懸於南桁|{
	南桁即朱雀桁}
郗鑒言於帝曰前朝誅楊駿等|{
	朝直遙翻}
皆先極官刑後聽私殯臣以為王誅加於上私義行於下宜聽敦家收葬於義為弘帝許之司徒導等皆以討敦功受封賞周撫與鄧岳俱亡周光欲資給其兄而取岳撫怒曰我與伯山同亡|{
	鄧岳字伯山}
何不先斬我會岳至撫出門遥謂之曰何不速去今骨肉尚欲相危況他人乎岳迴舟而走與撫共入西陽蠻中明年詔原敦黨撫岳出首|{
	首式救翻}
得免死禁錮故吳内史張茂妻陸氏傾家產帥茂部曲為先登以討沈充報其夫仇|{
	沈充殺張茂見上卷元帝永昌元年帥讀曰率}
充敗陸氏詣闕上書為茂謝不克之責詔贈茂太僕|{
	克能也謝茂守郡不能式遏寇虐為充所殺也為于偽翻}
有司奏王彬等敦之親族皆當除名詔曰司徒導以大義滅親猶將百世宥之況彬等皆公之近親乎悉無所問有詔王敦綱紀除名參佐禁錮|{
	綱紀綜理府事者也參佐諸僚屬也}
温嶠上疏曰王敦剛愎不仁忍行殺戮朝廷所不能制骨肉所不能諫處其朝者恒懼危亡|{
	朝府朝也愎蒲逼翻朝直遥翻處昌呂翻下晏處同恒戶登翻}
故人士結舌道路以目|{
	但以目相視不敢發言}
誠賢人君子道窮數盡遵養時晦之辰也|{
	周頌酌之詩曰遵養時晦毛氏註云遵率養取晦昧也鄭氏箋云養是闇昧之君以老其惡}
原其私心豈遑晏處|{
	晏處猶言安處}
如陸玩劉胤郭璞之徒常與臣言備知之矣必其贊導凶悖|{
	悖蒲内翻又蒲沒翻}
自當正以典刑如其枉䧟姦黨謂宜施之寛貸臣以玩等之誠聞於聖聽當受同賊之責苟默而不言實負其心惟陛下仁聖裁之郗鑒以為先王立君臣之教貴於仗節死義王敦佐吏雖多逼廹然進不能止其逆謀退不能脫身遠遁凖之前訓宜加義責|{
	謂以大義責之}
帝卒從嶠議|{
	卒子恤翻}
冬十月以司徒導為太保領司徒加殊禮西陽王羕領太尉|{
	羕余亮翻}
應詹為江州刺史劉遐為徐州刺史代王邃鎮淮陰蘇峻為歷陽内史|{
	為蘇峻以歷陽稱兵張本}
加庾亮護軍將軍温嶠前將軍導固辭不受應詹至江州吏民未安詹撫而懷之莫不悦服 十二月凉州將辛晏據枹罕不服|{
	將即亮翻枹罕縣前漢屬金城郡後漢屬隴西郡晉自張軌鎮河西表分西平界置晉興郡枹罕縣屬焉枹音膚}
張駿將討之從事劉慶諫曰霸王之師必須天時人事相得然後乃起辛晏凶狂安忍其亡可必|{
	殺人而心不矜惻顔不顰蹙者為忍忍而安之則其亡必矣}
柰何以饑年大舉盛寒攻城乎駿乃止駿遣參軍王騭聘於趙|{
	騭之日翻}
趙主曜謂之曰貴州欵誠和好卿能保之乎騭曰不能侍中徐邈曰君來結好而云不能保何也|{
	好呼到翻}
騭曰齊桓貫澤之盟憂心兢兢諸侯不召自至葵丘之會振而矜之叛者九國|{
	公羊傳僖三年齊侯宋公江人黄人盟于貫澤江人黄人者何遠國之辭也遠國至矣則中國曷為獨言齊宋至爾大國言齊宋小國言江黄則其餘為莫敢不至也九年九月戊辰諸侯盟于葵丘桓之盟不日此何以日危之也何危爾貫澤之會桓公有憂中國之心不召而至者江人黄人也葵丘之會桓公震而矜之叛者九國震之者何猶曰振振然矜之者何猶曰莫若我也}
趙國之化常如今日可也若政教陵遲尚未能察邇者之變況鄙州乎曜曰此凉州之君子也擇使可謂得人矣|{
	使疏吏翻}
厚禮而遣之 是歲代王賀傉始親國政|{
	元帝大興四年賀傉立至是始能親政傉奴沃翻}
以諸部多未服乃築城於東木根山|{
	河西有木根山在五原郡東北此木根山在河東故曰東木根山}
徙居之三年春二月張駿承元帝凶問大臨三日|{
	臨力鴆翻}
會黄龍見嘉泉|{
	據駿傳嘉泉在武威揖次縣揖次前漢作揟次孟康曰揟子如翻次音咨}
汜禕等請改年以章休祥駿不許|{
	汜音凡褘吁韋翻}
辛晏以枹罕降駿復收河南之地|{
	凉州諸郡獨金城在河南}
贈故譙王承甘卓戴淵周顗虞望郭璞王澄等官|{
	承當作氶王敦之難諸人死之故贈以官}
周札故吏為札訟寃|{
	為于偽翻}
尚書卞壼議以為札守石頭開門延寇|{
	事見上卷元帝永昌元年壺苦本翻}
不當贈諡司徒導以為往年之事敦姦逆未彰自臣等有識以上皆所未悟與札無異既悟其姦札便以身許國尋取梟夷|{
	事見上太寜二年梟堅堯翻}
臣謂宜與周戴同例郗鑒以為周戴死節周札延寇事異賞均何以勸沮|{
	沮在呂翻}
如司徒議謂往年有識以上皆與札無異則譙王周戴皆應受責何贈諡之有今三臣既褒則札宜受貶明矣導曰札與譙王周戴雖所見有異同皆人臣之節也鑒曰敦之逆謀履霜日久|{
	易曰履霜堅冰至}
緣札開門令王師不振若敦前者之舉義同桓文則先帝可為幽厲邪然卒用導議|{
	卒子恤翻}
贈札衛尉 後趙王勒加宇文乞得歸官爵使之擊慕容廆|{
	以元年廆執其使送建康也廆戶罪翻}
廆遣世子皝索頭段國共擊之|{
	皝呼廣翻索頭即拓跋氏索悉各翻}
以遼東相裴嶷為右翼慕容仁為左翼乞得歸據澆水以拒皝|{
	澆水即澆洛水也嶷魚力翻澆古堯翻}
遣兄子悉拔雄拒仁 |{
	考異曰燕書征虜仁傳作悉拔堆後魏書字文莫槐傳作乞得龜悉拔堆載記亦作龜燕書武宣紀作乞得歸悉拔雄今從之}
仁擊悉拔雄斬之乘勝與皝攻乞得歸大破之乞得歸棄軍走皝仁進入其國城使輕兵追乞得歸過其國三百餘里而還盡獲其國重器畜產以百萬計民之降附者數萬|{
	降戶江翻}
三月段末柸卒弟牙立 戊辰立皇子衍為太子大赦 趙主曜立皇后劉氏 北羌王盆句除附於趙|{
	句古侯翻又權俱翻又音駒}
後趙將石佗自鴈門出上郡襲之|{
	將即亮翻}
俘三千餘落獲牛馬羊百餘萬而歸趙主曜遣中山王岳追之曜屯于富平為岳聲援岳與石佗戰於河濱斬之|{
	富平縣屬北地郡河濱大河之濱也水經河水過富平縣西佗徒河翻唐勝州河濱縣隋榆林縣地杜佑曰富平本漢舊縣後漢移富平縣於今彭原郡界富平故城是也案靈州乃漢富平縣地今京兆富平縣西南有漢懷德故城此富平蓋漢懷德縣地}
後趙兵死者六千餘人岳悉收所虜而歸 楊難敵襲仇池克之執田崧立之於前左右令崧拜崧瞋目叱之曰氐狗安有天子牧伯而向賊拜乎難敵字謂之曰子岱|{
	田崧字子岱趙使崧鎮仇池見上卷太寜元年瞋七人翻}
吾當與子共定大業子忠於劉氏豈不能忠於我乎崧厲色大言曰賊氐汝本奴才何謂大業我寜為趙鬼不為汝臣顧排一人奪其劒前刺難敵不中|{
	刺七亦翻中竹仲翻}
難敵殺之 都尉魯潛以許昌叛降於後趙|{
	降戶江翻下同}
夏四月後趙將石瞻攻兖州刺史檀斌於鄒山|{
	晉本紀斌作贇載記作斌將即亮翻斌音彬考異曰帝紀作石良今從石勒載記}
殺之 後趙西夷中郎將王騰襲殺并州刺史崔琨上黨内史王眘據并州降趙|{
	劉琨鎮并州愍帝建興四年為石勒所破置并州刺史治上黨王眘章武人初起兵擾勒渤海河間諸郡後歸于勒使守上黨眘古慎字}
五月以陶侃為征西大將軍都督荆湘雍梁四州諸軍事荆州刺史|{
	雍於用翻}
荆州士女相慶侃性聰敏㳟勤終日斂膝危坐軍府衆事檢攝無遺|{
	攝録也整也}
未嘗少閒|{
	少詩沼翻}
常語人曰大禹聖人乃惜寸隂|{
	禹不貴寸璧而重寸隂語牛倨翻}
至於衆人當惜分隂豈可但逸遊荒醉生無益於時死無聞於後是自棄也諸參佐或以談戲廢事者命取其酒器蒱博之具悉投之於江將吏則加鞭扑曰樗蒱者牧猪奴戲耳|{
	晉人多好樗蒱以五木擲之其采有黑犢有雉有盧得盧者勝扑蒱卜翻}
老莊浮華非先王之法言不益實用君子當正其威儀何有蓬頭跣足自謂宏達耶有奉饋者必問其所由若力作所致雖微必喜慰賜參倍|{
	參猶三也}
若非理得之則切厲訶辱|{
	切峻切厲嚴厲也}
還其所饋嘗出遊見人持一把未熟稻侃問用此何為人云行道所見聊取之耳侃大怒曰汝既不佃而戲賊人稻|{
	佃停年翻治田也}
執而鞭之是以百姓勤於農作家給人足嘗造船其木屑竹頭侃皆令籍而掌之|{
	皆令籍記而典掌之}
人咸不解所以|{
	解胡買翻曉也以猶用也}
後正會積雪始晴聽事前餘雪猶濕|{
	聽他經翻}
乃以木屑布地及桓温伐蜀又以侃所貯竹頭作丁裝船|{
	貯丁呂翻}
其綜理微密皆此類也 後趙將石生屯洛陽寇掠河南司州刺史李矩潁川太守郭默軍數敗又乏食乃遣使附於趙趙主曜使中山王岳將兵萬五千人趣孟津|{
	數所角翻趣七喻翻}
鎮東將軍呼延謨帥荆司之衆自崤澠而東|{
	時荆州仍屬晉司州之地多入後趙劉曜得其民處之關中者使謨帥而東耳或曰劉聰以洛陽為荆州此所謂荆司皆晉司州之衆也帥讀曰率下同}
欲會矩默共攻石生岳克孟津石梁二戍|{
	此孟津戍蓋置於河隂石梁戍在洛北}
斬獲五千餘級進圍石生於金墉後趙中山公虎帥步騎四萬入自成臯關與岳戰於洛西岳兵敗中流矢|{
	中竹仲翻}
退保石梁虎作塹柵環之|{
	環音宦}
遏絶内外岳衆飢甚殺馬食之虎又擊呼延謨斬之曜自將兵救岳虎帥騎三萬逆戰趙前軍將軍劉黑擊虎將石聰於八特阪|{
	水經註澗水出河南新安縣東南東北流逕函谷東阪東謂之八特阪}
大破之曜屯於金谷|{
	水經註金谷水出太白原東南流歷金谷又東南流逕晉石崇故居在河南界}
夜軍中無故大驚士卒奔潰乃退屯澠池|{
	澠彌兖翻}
夜又驚潰遂歸長安六月虎拔石梁禽岳及其將佐八十餘人氐羌三千餘人皆送襄國阬其士卒九千人遂攻王騰於并州執騰殺之阬其士卒七千餘人曜還長安素服郊次哭七日乃入城因憤恚成疾|{
	恚於避翻}
郭默復為石聰所敗棄妻子南奔建康李矩將士隂謀叛降後趙矩不能討亦帥衆南歸|{
	復扶又翻敗補邁翻帥讀曰率}
衆皆道亡惟郭誦等百餘人隨之卒於魯陽|{
	魯陽縣屬南陽郡}
矩長史崔宣帥其餘衆三千降于後趙於是司豫徐兖之地率皆入於後趙以淮為境矣 趙主曜以永安王胤為大司馬大單于徙封南陽王置單于臺于渭城|{
	單音蟬}
其左右賢王以下皆以胡羯鮮卑氐羌豪桀為之|{
	羯居謁翻}
秋七月辛未以尚書令郗鑒為車騎將軍都督徐兖青三州諸軍事兖州刺史鎮廣陵 閏月以尚書左僕射荀崧為光禄大夫録尚書事尚書鄧攸為左僕射 右衛將軍虞胤元敬皇后之弟也|{
	元帝為琅邪王虞為妃即位追諡曰敬皇后祔廟從元帝諡曰元敬}
與左衛將軍南頓王宗|{
	宗汝南王亮之子也}
俱為帝所親任典禁兵直殿内多聚勇士以為羽翼王導庾亮皆忌之頗以為言帝待之愈厚宫門管鑰皆以委之|{
	管鍵也鑰關牡也今謂之鎖匙}
帝寢疾亮夜有所表從宗求鑰宗不與叱亮使曰此汝家門戶邪亮益忿之|{
	為下亮殺宗張本使疏吏翻}
及帝疾篤不欲見人羣臣無得進者亮疑宗胤及宗兄西陽王羕有異謀排闥入升御床見帝流涕言羕與宗等謀廢大臣自求輔政請黜之帝不納|{
	羕余亮翻}
壬午帝引太宰羕司徒導尚書令卞壼車騎將軍郗鑒護軍將軍庾亮領軍將軍陸曄|{
	按晉制領軍將軍在護軍將軍之上今先書庾亮而後陸曄亮以外戚受遺專權故也}
丹楊尹温嶠並受遺詔輔太子更入殿將兵直宿|{
	更工衡翻迭也}
復拜壼右將軍亮中書令曄録尚書事|{
	復扶又翻}
丁亥降遺詔戊子帝崩|{
	年二十七}
帝明敏有機斷|{
	斷丁亂翻}
故能以弱制彊誅翦逆臣克復大業己丑太子即皇帝位生五年矣羣臣進璽|{
	進璽於嗣君也璽斯氏翻}
司徒導以疾不至卞壼正色於朝曰|{
	朝直遥翻下同}
王公豈社稷之臣邪大行在殯嗣皇未立寜是人臣辭疾之時也導聞之輿疾而至大赦增文武位二等尊庾后為皇太后羣臣以帝幼冲奏請太后依漢和熹皇后故事|{
	言臨朝稱制也}
太后辭讓數四乃從之秋九月癸卯太后臨朝稱制以司徒導録尚書事與中書令庾亮尚書令卞壼參輔朝政然事之大要皆决於亮加郗鑒車騎大將軍陸曄左光禄大夫皆開府儀同三司以南頓王宗為驃騎將軍|{
	驃匹妙翻}
虞胤為大宗正尚書召樂廣之子謨為郡中正|{
	樂廣南陽人蓋召謨為本郡中正}
庾珉族人怡為廷尉評|{
	漢置廷尉平晉曰廷尉評}
謨怡各稱父命不就卞壼奏曰人非無父而生職非無事而立有父必有命居職必有悔|{
	易繫辭曰悔吝者憂虞之象也}
有家各私其子則為王者無民君臣之道廢矣樂廣庾珉受寵聖世身非已有況及後嗣而可專哉所居之職若順夫羣心則戰戍者之父母皆當命子以不處也|{
	言人莫不惡死若各順其心則有戰戍之事為父母者皆不欲使其子就死地也處昌呂翻}
謨怡不得已各就職 辛丑葬明帝於武平陵冬十一月癸巳朔日有食之 慕容廆與段氏方睦

為段牙謀使之徙都牙從之即去令支國人不樂|{
	為于偽翻樂音洛令音鈴師古郎定翻支音祗}
段疾陸眷之孫遼欲奪其位以徙都為牙罪十二月帥國人攻牙殺之|{
	帥讀曰率}
自立|{
	句斷}
段氏自務勿塵以來日益彊盛其地西接漁陽東界遼水所統胡晉三萬餘戶控弦四五萬騎 荆州刺史陶侃以寜州刺史王堅不能禦寇是歲表零陵太守南陽尹奉為寜州刺史以代之先是王遜在寜州|{
	先悉薦翻}
蠻酋梁水太守㸑量益州太守李逷|{
	沈約曰梁水太守晉成帝分興古郡立蓋先以授蠻酋殺㸑量之後始用王官也益州郡後漢置蜀更名建寜郡惠帝太安二年分建寜以西七縣别立益州郡懷帝永嘉二年更名晉寜郡此復有益州太守蓋亦以為位號授蠻酋也逷他歷翻}
皆叛附於成遜討之不能克奉至州重募徼外夷刺㸑量殺之諭降李逷|{
	徼吉弔翻刺七亦翻降戶江翻}
州境遂安 代王賀傉卒|{
	傉奴沃翻}
弟紇那立

顯宗成皇帝上之上|{
	諱衍字世根明帝長子也諡法安民立政曰成}


咸和元年春二月大赦改元 趙以汝南王咸為太尉録尚書事光禄大夫劉綏為大司徒卜泰為大司空劉后疾病趙主曜問所欲言劉氏泣曰妾幼鞠於叔父昶|{
	鞠養也昶丑兩翻}
願陛下貴之叔父皚之女芳有德色|{
	皚魚開翻}
願以備後宫言終而卒曜以昶為侍中大司徒録尚書事立芳為皇后尋又以昶為太保 三月後趙主勒夜微行檢察諸營衛齎金帛以賂門者求出永昌門候王假欲收捕之從者至乃止|{
	從才用翻}
旦召假以為振忠都尉爵關内侯|{
	振忠都尉後趙所置也}
勒召記室參軍徐光光醉不至黜為牙門光侍直有愠色|{
	愠於問翻愠色者含怒而見於色也}
勒怒并其妻子囚之 夏四月後趙將石生寇汝南執内史祖濟六月癸亥泉陵公劉遐卒|{
	泉陵縣屬零陵郡}
癸酉以車騎大將軍郗鑒領徐州刺史征虜將軍郭默為北中郎將監淮北諸軍事領遐部曲遐子肇尚幼遐妹夫田防及故將史迭等不樂他屬|{
	樂音洛}
共以肇襲遐故位而叛臨淮太守劉矯掩襲遐營|{
	劉遐屯泗口在臨淮下邳之間故矯得以掩襲其營}
斬防等遐妻邵續女也驍果有父風|{
	驍古老翻}
遐嘗為後趙所圍妻單將數騎拔遐出於萬衆之中及田防等欲作亂遐妻止之不從乃密起火燒甲仗都盡故防等卒敗|{
	卒子恤翻}
詔以肇襲遐爵|{
	襲爵泉陵公}
司徒導稱疾不朝|{
	朝直遥翻下同}
而私送郗鑒卞壼奏導虧法從私無大臣之節請免官雖事寢不行舉朝憚之壼儉素廉潔裁斷切直當官幹實性不弘裕不肯苟同時好|{
	斷丁亂翻好呼到翻}
故為諸名士所少|{
	重之曰多輕之曰少少妙紹翻}
阮孚謂之曰卿常無閒泰如含瓦石不亦勞乎壼曰諸君子以道德恢弘風流相尚執鄙吝者非壼而誰時貴游子弟多慕王澄謝鯤為放達壼厲色於朝曰悖禮傷教罪莫大焉中朝傾覆實由於此欲奏推之|{
	中朝謂西晉奏推奏之于上推按其罪也}
王導庾亮不聽乃止 成人討越嶲斯叟破之|{
	討斯叟事始上卷明帝太興元年嶲音髓}
秋七月癸丑觀陽烈侯應詹卒|{
	觀陽縣屬零陵郡吳立}
初王導輔政以寛和得衆及庾亮用事任法裁物頗失人心豫州刺史祖約自以名輩不後郗卞|{
	名為一時所稱輩以年齒為等}
而不豫顧命又望開府復不得|{
	晉世四征四鎮大將軍乃得開府約平西將軍耳烏得望開府邪復扶又翻}
及諸表請多不見許遂懷怨望及遺詔褒進大臣又不及約與陶侃二人皆疑庾亮刪之|{
	刪削除也}
歷陽内史蘇峻有功於國|{
	謂破沈充錢鳳也}
威望漸著有銳卒萬人器械甚精朝廷以江外寄之而峻頗懷驕溢有輕朝廷之志招納亡命衆力日多皆仰食縣官運漕相屬|{
	仰牛向翻屬之欲翻}
稍不如意輒肆忿言亮既疑峻約又畏侃之得衆八月以丹楊尹温嶠為都督江州諸軍事江州刺史鎮武昌尚書僕射王舒為會稽内史|{
	會工外翻}
以廣聲援又修石頭以備之|{
	亮修石頭適以資蘇峻拒義師耳}
丹楊尹阮孚以太后臨朝政出舅族謂所親曰今江東創業尚淺主幼時艱庾亮年少|{
	少詩照翻}
德信未孚以吾觀之亂將作矣遂求出為廣州刺史爭咸之子也 冬十月立帝母弟岳為吳王 南頓王宗自以失職怨望|{
	宗解兵衛故自以為失職}
又素與蘇峻善庾亮欲誅之宗亦欲廢執政御史中丞鍾雅劾宗謀反|{
	劾戶槩翻又戶得翻}
亮使右衛將軍趙胤收之宗以兵拒戰為胤所殺貶其族為馬氏三子綽超演皆廢為庶人免太宰西陽王羕降封弋陽縣王大宗正虞胤左遷桂陽太守宗宗室近屬羕先帝保傅|{
	羕宗兄弟也宗言近屬羕言保傅宗叙族羕叙官也}
亮一旦翦黜由是愈失遠近之心宗黨卞闡亡奔蘇峻亮符峻送闡峻保匿不與宗之死也帝不之知久之帝問亮曰常日白頭公何在亮對以謀反伏誅帝泣曰舅言人作賊便殺之人言舅作賊當如何亮懼變色 趙將黄秀等寇鄼|{
	鄼縣漢屬南陽郡及晉分為順陽郡治所鄼音贊}
順陽太守魏該帥衆奔襄陽|{
	帥讀曰率}
後趙王勒用程遐之謀營鄴宫使世子弘鎮鄴配禁

兵萬人車騎所統五十四營悉配之以驍騎將軍領門臣祭酒王陽專統六夷以輔之|{
	驍堅堯翻}
中山公虎自以功多無去鄴之意及修二臺遷其家室虎由是怨程遐|{
	為後虎殺遐及弘張本}
十一月後趙石聰攻夀春祖約屢表請救朝廷不為出兵|{
	為于偽翻}
聰遂寇逡遒阜陵|{
	二縣皆屬淮南郡師古曰逡音峻遒音才由翻春秋公會吳于槖臯杜預云淮南逡遒縣劉昫曰唐廬州慎縣漢逡遒縣地}
殺掠五千餘人建康大震詔加司徒導大司馬假黄鉞都督中外諸軍事以禦之軍于江寜蘇峻遣其將韓晃擊石聰走之導解大司馬朝議又欲作涂塘以遏胡寇祖約曰是棄我也益懷憤恚|{
	作涂塘則夀春在涂塘之外朝直遥翻涂讀曰滁恚於避翻}
十二月濟岷太守劉闓等|{
	晉志曰或云魏平蜀徙其豪將家於濟河北為濟岷郡太康地志無此郡未詳濟子禮翻}
殺下邳内史夏侯嘉以下邳叛降于後趙|{
	夏戶雅翻降戶江翻}
石瞻攻河南太守王瞻于邾抜之|{
	劉薈鄒山記曰邾城在魯國鄒縣鄒山之南去山二里左傳文十三年邾遷于繹即此城也}
彭城内史劉續復據蘭陵石城|{
	魏收地形志蘭陵縣有石城山復扶又翻}
石瞻攻抜之 後趙王勒以牙門將王波為記室參軍典定九流始立秀孝試經之制|{
	秀孝試經晉制也後趙至此始行之}
張駿畏趙人之逼是歲徙隴西南安民二千餘家於姑臧又遣使修好於成以書勸成主雄去尊號稱藩於晉|{
	使疏吏翻下同好呼到翻去羌呂翻下乃去同}
雄復書曰吾過為士大夫所推然本無心於帝王思為晉室元功之臣掃除氛埃而晉室陵遲德聲不振引領東望有年月矣會獲來貺情在闇至|{
	言引領望晉此情常在而駿書適至闇與之合也}
有何己巳自是聘使相繼

二年春正月朱提太守楊術與成將羅恒戰於臺登兵敗術死|{
	朱提音銖時}
夏五月甲申朔日有食之 趙武衛將軍劉朗帥騎三萬襲楊難敵於仇池弗克|{
	帥讀曰率下同}
掠三千餘戶而歸 張駿聞趙兵為後趙所敗|{
	敗補邁翻下同}
乃去趙官爵|{
	去羌呂翻}
復稱晉大將軍凉州牧遣武威太守竇濤金城太守張閬武興太守辛巖|{
	惠帝永寜中張軌表請合秦雍流移人於姑臧西北置武興郡閬音浪}
揚烈將軍宋輯等帥衆數萬會韓璞攻掠趙秦州諸郡|{
	韓璞時在冀帥讀曰率}
趙南陽王胤將兵擊之|{
	將即亮翻}
屯狄道枹罕護軍辛晏告急|{
	枹音膚}
秋駿使韓璞辛巖救之璞進度沃干嶺|{
	沃干嶺在晉興郡大夏縣東南洮水西北}
巖欲速戰璞曰夏末以來日星數有變|{
	數所角翻}
不可輕動且曜與石勒相攻胤必不能久與我相守也與胤夾洮相持七十餘日|{
	水經註洮水過狄道城西洮土刀翻}
冬十月璞遣辛巖督運於金城胤聞之曰韓璞之衆十倍於吾吾糧不多難以持久今虜分兵運糧天授我也若敗辛巖璞等自潰乃帥騎三千襲巖於沃干嶺敗之|{
	敗補邁翻}
遂前逼璞營璞衆大潰胤乘勝追奔濟河攻抜令居|{
	令居縣漢屬金城郡張寔置廣武郡令居分屬焉}
斬首二萬級進據振武|{
	振武在姑臧東南廣武西北}
河西大駭張閬辛晏帥其衆數萬降趙駿遂失河南之地 庾亮以蘇峻在歷陽終為禍亂欲下詔徵之訪於司徒導導曰峻猜險必不奉詔不若且苞容之亮言於朝曰峻狼子野心終必為亂|{
	左傳楚令尹子文曰諺曰狼子野心是乃狼也其可畜乎朝直遥翻}
今日徵之縱不順命為禍猶淺若復經年不可復制猶七國之於漢也|{
	漢鼂錯議削吳楚曰今削之亦反不削亦反削之反疾禍小不削反遲禍大亮以為比復扶又翻}
朝臣無敢難者獨光禄大夫卞壼爭之曰峻擁彊兵逼近京邑路不終朝|{
	歷陽之與建康一江之隔耳難乃旦翻近其靳翻}
一旦有變易為蹉跌|{
	易以豉翻蹉七何翻跌徒結翻}
宜深思之亮不從壼知必敗與温嶠書曰元規召峻意定|{
	庾亮字元規}
此國之大事峻已出狂意而召之是更速其禍也必縱毒蠚以向朝廷朝廷威力雖盛不知果可擒不|{
	蠚呼各翻螫也不讀曰否}
王公亦同此情吾與之爭甚懇切不能如之何本出足下以為外援|{
	謂以嶠鎮尋陽也}
而今更恨足下在外不得相與共諫止之或當相從耳嶠亦累書止亮舉朝以為不可|{
	朝直遥翻}
亮皆不聽峻聞之遣司馬何仍詣亮曰討賊外任遠近惟命至於内輔實非所堪亮不許召北中郎將郭默為後將軍領屯騎校尉|{
	郭默時監淮北軍騎奇寄翻}
司徒右長史庾氷為吳國内史皆將兵以備峻氷亮之弟也於是下優詔徵峻為大司農加散騎常侍位特進以弟逸代領部曲峻上表曰昔明皇帝親執臣手使臣北討胡寇今中原未靖臣何敢即安乞補青州界一荒郡以展鷹犬之用復不許|{
	復扶又翻}
峻嚴裝將赴召猶豫未决參軍任讓謂峻曰將軍求處荒郡而不見許|{
	任音壬處昌呂翻}
事勢如此恐無生路不如勒兵自守阜陵令匡術亦勸峻反|{
	阜陵縣屬淮南郡晉志曰阜陵漢明帝時淪為麻湖麻湖在今和州歷陽縣西三十里}
峻遂不應命温嶠聞之即欲帥衆下衛建康|{
	帥讀曰率}
三吳亦欲起義兵亮並不聽而報嶠書曰吾憂西陲過於歷陽|{
	西陲謂陶侃也}
足下無過雷池一步也|{
	雷池即在大雷之東今池州界水經註青林水西南歷尋陽分為二}
闕

朝廷遣使諭峻|{
	使疏吏翻}
峻曰臺下云我欲

反豈得活耶我寜山頭望廷尉不能廷尉望山頭往者國家危如累卵非我不濟狡兔既死獵犬宜烹|{
	越范蠡遺大夫種曰狡兔死走狗烹}
但當死報造謀者耳|{
	言欲報庾亮也}
峻知祖約怨朝廷乃遣參軍徐會推崇約請共討庾亮約大喜其從子智衍並勸成之|{
	從才用翻}
譙國内史桓宣謂智曰本以彊胡未滅將戮力討之使君若欲為雄霸何不助國討峻則威名自舉今乃與峻俱反此安得久乎智不從宣詣約請見約知其欲諫拒而不内|{
	内讀曰納}
宣遂絶約不與之同|{
	約于是赴歷陽宣將其衆營于馬頭山}
十一月約遣兄子沛内史渙女壻淮南太守許柳以兵會峻逖妻柳之姊也固諫不從詔復以卞壼為尚書令領右衛將軍以鄶稽内史王舒行揚州刺史事|{
	鄶稽即會稽音古外翻王舒傳曰時徵蘇峻王導欲出舒為外援授會稽内史舒以父名會辭朝議以字同音異于禮無嫌舒復陳音雖異而字同求改他郡于是改會字為鄶}
吳興太守虞潭督三吳等諸郡軍事 尚書左丞孔坦司徒司馬丹楊陶囘言於王導請及峻未至急斷阜陵守江西當利諸口|{
	阜陵有麻湖之阻守當利諸口則峻兵不能渡江}
彼少我衆一戰决矣|{
	少詩沼翻}
若峻未來可往逼其城今不先往峻必先至峻至則人心危駭難與戰矣此時不可失也導然之庾亮不從十二月辛亥蘇峻使其將韓晃張健等襲䧟姑孰取鹽米|{
	姑孰臨江渚舟船所湊晉積鹽米於此}
亮方悔之壬子彭城王雄章武王休叛犇峻雄釋之子也|{
	彭城王釋宣帝弟穆王權之子章武王休義陽王望之孫}
庚申京師戒嚴假庾亮節都督征討諸軍事以左衛將軍趙胤為歷陽太守使左將軍司馬流將兵據慈湖以拒峻|{
	慈湖在姑孰今在太平州當塗縣北六十五里泝江而上過三山十餘里至溧洲自溧洲過白土磯入慈湖夾}
以前射聲校尉劉超為左衛將軍侍中禇翜典征討軍事|{
	翜色洽翻}
亮使弟翼以白衣領數百人備石頭丙寅徙琅邪王昱為會稽王吳王岳為琅邪王 宣

城内史桓彛欲起兵以赴朝廷其長史禆惠|{
	姓譜禆姓鄭禆諶之後}
以郡兵寡弱山民易擾|{
	宣城之南山越居之自吳以來屢為寇亂易以豉翻}
謂宜且案甲以待之彛厲色曰見無禮於其君者若鷹鸇之逐鳥雀|{
	左傳魯大夫臧文仲之言}
今社稷危逼義無宴安辛未彛進屯蕪湖韓晃擊破之因進攻宣城|{
	宣城郡治宛陵縣宣城别為縣賢曰宣城故城在今宣州南陵縣東}
彛退保廣德|{
	何承天曰廣德漢舊縣沈約曰二漢志並無疑是吳所立屬宣城郡桐川志後漢置廣德縣晉并入宣城今廣德軍是也}
晃大掠諸縣而還|{
	還從宣翻又如字}
徐州刺史郗鑒欲帥所領赴難|{
	帥讀曰率難乃旦翻}
詔以北寇不許 是歲後趙中山公虎擊代王紇那戰于句注陘北|{
	張守節曰句注山在代州鴈門縣西北三十里據唐志鴈門縣有東陘關西陘關即其地也句音鉤}
紇那兵敗徙都大甯以避之|{
	據水經註大甯即廣甯也廣甯前漢曰廣寜屬上谷郡後漢曰廣甯晉武帝太康中分置廣甯郡}
代王鬱律之子翳槐居於其舅賀蘭部紇那遣使求之賀蘭大人藹頭擁護不遣紇那與宇文部共擊藹頭不克

資治通鑑卷九十三
