<!DOCTYPE html PUBLIC "-//W3C//DTD XHTML 1.0 Transitional//EN" "http://www.w3.org/TR/xhtml1/DTD/xhtml1-transitional.dtd">
<html xmlns="http://www.w3.org/1999/xhtml">
<head>
<meta http-equiv="Content-Type" content="text/html; charset=utf-8" />
<meta http-equiv="X-UA-Compatible" content="IE=Edge,chrome=1">
<title>資治通鑒_236-資治通鑑卷二百三十五_236-資治通鑑卷二百三十五</title>
<meta name="Keywords" content="資治通鑒_236-資治通鑑卷二百三十五_236-資治通鑑卷二百三十五">
<meta name="Description" content="資治通鑒_236-資治通鑑卷二百三十五_236-資治通鑑卷二百三十五">
<meta http-equiv="Cache-Control" content="no-transform" />
<meta http-equiv="Cache-Control" content="no-siteapp" />
<link href="/img/style.css" rel="stylesheet" type="text/css" />
<script src="/img/m.js?2020"></script> 
</head>
<body>
 <div class="ClassNavi">
<a  href="/24shi/">二十四史</a> | <a href="/SiKuQuanShu/">四库全书</a> | <a href="http://www.guoxuedashi.com/gjtsjc/"><font  color="#FF0000">古今图书集成</font></a> | <a href="/renwu/">历史人物</a> | <a href="/ShuoWenJieZi/"><font  color="#FF0000">说文解字</a></font> | <a href="/chengyu/">成语词典</a> | <a  target="_blank"  href="http://www.guoxuedashi.com/jgwhj/"><font  color="#FF0000">甲骨文合集</font></a> | <a href="/yzjwjc/"><font  color="#FF0000">殷周金文集成</font></a> | <a href="/xiangxingzi/"><font color="#0000FF">象形字典</font></a> | <a href="/13jing/"><font  color="#FF0000">十三经索引</font></a> | <a href="/zixing/"><font  color="#FF0000">字体转换器</font></a> | <a href="/zidian/xz/"><font color="#0000FF">篆书识别</font></a> | <a href="/jinfanyi/">近义反义词</a> | <a href="/duilian/">对联大全</a> | <a href="/jiapu/"><font  color="#0000FF">家谱族谱查询</font></a> | <a href="http://www.guoxuemi.com/hafo/" target="_blank" ><font color="#FF0000">哈佛古籍</font></a> 
</div>

 <!-- 头部导航开始 -->
<div class="w1180 head clearfix">
  <div class="head_logo l"><a title="国学大师官网" href="http://www.guoxuedashi.com" target="_blank"></a></div>
  <div class="head_sr l">
  <div id="head1">
  
  <a href="http://www.guoxuedashi.com/zidian/bujian/" target="_blank" ><img src="http://www.guoxuedashi.com/img/top1.gif" width="88" height="60" border="0" title="部件查字,支持20万汉字"></a>


<a href="http://www.guoxuedashi.com/help/yingpan.php" target="_blank"><img src="http://www.guoxuedashi.com/img/top230.gif" width="600" height="62" border="0" ></a>


  </div>
  <div id="head3"><a href="javascript:" onClick="javascript:window.external.AddFavorite(window.location.href,document.title);">添加收藏</a>
  <br><a href="/help/setie.php">搜索引擎</a>
  <br><a href="/help/zanzhu.php">赞助本站</a></div>
  <div id="head2">
 <a href="http://www.guoxuemi.com/" target="_blank"><img src="http://www.guoxuedashi.com/img/guoxuemi.gif" width="95" height="62" border="0" style="margin-left:2px;" title="国学迷"></a>
  

  </div>
</div>
  <div class="clear"></div>
  <div class="head_nav">
  <p><a href="/">首页</a> | <a href="/ShuKu/">国学书库</a> | <a href="/guji/">影印古籍</a> | <a href="/shici/">诗词宝典</a> | <a   href="/SiKuQuanShu/gxjx.php">精选</a> <b>|</b> <a href="/zidian/">汉语字典</a> | <a href="/hydcd/">汉语词典</a> | <a href="http://www.guoxuedashi.com/zidian/bujian/"><font  color="#CC0066">部件查字</font></a> | <a href="http://www.sfds.cn/"><font  color="#CC0066">书法大师</font></a> | <a href="/jgwhj/">甲骨文</a> <b>|</b> <a href="/b/4/"><font  color="#CC0066">解密</font></a> | <a href="/renwu/">历史人物</a> | <a href="/diangu/">历史典故</a> | <a href="/xingshi/">姓氏</a> | <a href="/minzu/">民族</a> <b>|</b> <a href="/mz/"><font  color="#CC0066">世界名著</font></a> | <a href="/download/">软件下载</a>
</p>
<p><a href="/b/"><font  color="#CC0066">历史</font></a> | <a href="http://skqs.guoxuedashi.com/" target="_blank">四库全书</a> |  <a href="http://www.guoxuedashi.com/search/" target="_blank"><font  color="#CC0066">全文检索</font></a> | <a href="http://www.guoxuedashi.com/shumu/">古籍书目</a> | <a   href="/24shi/">正史</a> <b>|</b> <a href="/chengyu/">成语词典</a> | <a href="/kangxi/" title="康熙字典">康熙字典</a> | <a href="/ShuoWenJieZi/">说文解字</a> | <a href="/zixing/yanbian/">字形演变</a> | <a href="/yzjwjc/">金 文</a> <b>|</b>  <a href="/shijian/nian-hao/">年号</a> | <a href="/diming/">历史地名</a> | <a href="/shijian/">历史事件</a> | <a href="/guanzhi/">官职</a> | <a href="/lishi/">知识</a> <b>|</b> <a href="/zhongyi/">中医中药</a> | <a href="http://www.guoxuedashi.com/forum/">留言反馈</a>
</p>
  </div>
</div>
<!-- 头部导航END --> 
<!-- 内容区开始 --> 
<div class="w1180 clearfix">
  <div class="info l">
   
<div class="clearfix" style="background:#f5faff;">
<script src='http://www.guoxuedashi.com/img/headersou.js'></script>

</div>
  <div class="info_tree"><a href="http://www.guoxuedashi.com">首页</a> > <a href="/SiKuQuanShu/fanti/">四库全书</a>
 > <h1>资治通鉴</h1> <!--         下载:【右键另存为】即可 --></div>
  <div class="info_content zj clearfix">
  
<div class="info_txt clearfix" id="show">
<center style="font-size:24px;">236-資治通鑑卷二百三十五</center>
    資治通鑑卷二百三十五  宋 司馬光 撰<br />
<br />
  胡三省 音註<br />
<br />
  唐紀五十一【起閼逢閹茂六月盡上章執徐凡六年有奇】<br />
<br />
  德宗神武聖文皇帝十<br />
<br />
  貞元十年六月壬寅朔昭義節使李抱真薨其子殿中侍御史緘與抱真從甥元仲經謀祕不喪詐為抱真表求以職事授緘又詐為其父書遣禆將陳榮詣王武俊假貨財武俊怒曰吾與乃公厚善欲同奬王室耳豈與汝同惡耶聞乃公已亡【乃猶汝也公猶翁也】乃敢不俟朝命而自立【朝直遥翻】又敢告我況有求也使榮歸寄聲質責緘【質正也以正義責之也】昭義步軍都虞候王延貴汝州梁人也【梁縣漢晉屬河南郡後魏置汝北郡隋分置承休縣而梁縣仍故唐以承休縣帶汝州故梁縣在其西南四十五里】素以義勇聞上知抱真已薨遣中使第五守進往觀變且以軍事委王延貴守進至上黨【昭義軍治上黨】緘稱抱真有疾不能見三日緘乃嚴兵詣守進守進謂之曰朝廷已知相公捐館【捐弃也言死者弃其館舍而逝也】令王延貴權知軍事侍御宜喪行服緘愕然出謂諸將曰朝廷不許緘掌事諸君意如何莫對緘愳乃歸喪以使印及管鑰授監軍【使印節度之印也監古銜翻】守進召延貴宣口詔令視事【口宣所受詔旨故曰口詔】趣緘赴東都【趣赴東都歸私第趣讀曰促】元仲經出走延貴悉歸罪於仲經捕斬之詔以延貴權知昭義軍事 雲南王異牟尋遣其弟湊羅棟【棟郎甸翻】獻地圖土貢及吐蕃所給金印請復號南詔【夷語以王為詔其先渠帥有六自號六詔曰蒙巂詔越析詔浪穹詔邆睒詔施浪詔蒙舍詔蒙舍詔在諸部南故稱南詔至蒙歸義玄宗封為雲南王因號雲南】癸丑以祠部郎中袁滋為冊南詔使 【考異曰舊南詔傳十年八月遣溱羅棟獻吐蕃印新傳曰異牟尋與崔佐時盟點蒼山敗突厥於神川明年六月冊異牟尋為南詔王按實錄乃今年六月新舊傳皆誤也韋臯奏狀皆稱雲南王而竇滂雲南别錄曰詔袁滋冊異牟尋為南詔蓋從其請南詔之名自此始也蠻語詔即王也新傳云南詔王亦誤 余按異牟尋破吐蕃於神川考異誤作突厥】賜銀窠金印文曰貞元冊南詔印滋至其國異牟尋北面跪受冊印稽首再拜因與使者宴出玄宗所賜銀平脱馬頭盤二以示滋又指老笛工歌女曰皇帝所賜龜兹樂【唐十部樂有龜兹樂有彈箏豎箜琵琶五絃横笛笙簫觱篥荅臘鼓毛員鼓都曇鼔侯提鼔雞婁鼔腰鼔擔鼔齊鼔貝皆一銅鈸二舞者四人設五方師子高丈餘飾以方色每師子有十二人畫衣執紅拂首加紅袜謂之師子郎龜兹音㐀慈】惟二人在耳滋曰南詔當深思祖考子子孫孫盡忠於唐異牟尋拜曰敢不謹承使者之命 賜義武節度使張昇雲名茂昭 【考異曰舊傳於其父孝忠卒時言改名年代記在此年九月今從實錄】 御史中丞穆贊按度支吏贓罪【度徒洛翻】裴延齡欲出之【庇吏欲出其罪】贊不從延齡譛之貶饒州别駕朝士畏延齡側目【畏之不敢正視】贊寧之子也【天寶末安祿山反穆寧起兵於河北以討之】韋臯奏破吐蕃於峨和城【武德元年以漢蠶陵縣地置翼州管内有峨和城】秋七月壬申朔以王延貴為昭義留後賜名䖍休昭義行軍司馬攝洺州刺史元誼聞䖍休為留後意不平表請以磁邢洺别為一鎮昭義精兵多在山東【昭義軍鎮潞州謂磁邢洺三州為山東】誼厚賚以悅之上屢遣中使諭之不從臨洺守將夏侯仲宣以城歸䖍休䖍休遣磁州刺史馬正卿督禆將石定蕃等將兵五千擊洺州定蕃帥其衆二千叛歸誼【帥讀曰率】正卿退還詔以誼為饒州刺史誼不行䖍休自將兵攻之引洺水以灌城 黄少卿陷欽横潯貴等州攻孫公器於邕州 九月王䖍休破元誼兵進拔鷄澤【鷄澤漢廣平縣地武德四年置鷄澤縣屬洺州九域志在州東北六十里】 裴延齡奏稱官吏太多自今缺員請且勿補收其俸以實府庫上欲修神龍寺須五十尺松不可得延齡曰臣近見同州一谷木數千株皆可八十尺上曰開元天寶間求美材於近畿猶不可得今安得有之對曰天生珍材固待聖君乃出開元天寶何從得之延齡奏左藏庫司多有失落近因檢閲使置薄書乃於糞土之中得銀十三萬兩其匹段雜貨百萬有餘【匹段雜貨使在糞土之中已應腐爛不可用雖甚愚之人亦知其妄誕也德宗不加之罪延齡復何所忌憚乎】此皆已弃之物即是羨餘【羨代線翻】悉應移入雜庫以供别勅支用太府少卿韋少華不伏抗表稱此皆每月申奏見在之物【見賢遍翻】請加推驗執政請令三司詳覆上不許亦不罪少華延齡每奏對恣為詭譎皆衆所不敢言亦未嘗聞者延齡處之不疑【處昌呂翻】上亦頗知其誕妄但以其好詆毁人【好呼到翻】冀聞外事故親厚之【德宗親厚裴延齡不特冀聞外事也亦以進奉逢其欲耳】羣臣畏延齡有寵莫敢言惟鹽鐵轉運使張滂京兆尹李充司農卿李銛【銛息廉翻】以職事相關時證其妄而陸䞇獨以身當之日陳其不可用十一月壬申䞇上書極陳延齡姦詐數其罪惡【數所角翻】其畧曰延齡以聚歛為長策【歛力贍翻】以詭妄為嘉謀以掊克歛怨為匪躬【掊蒲侯翻】以靖譛服讒為盡節【左傳少皥氏有不才子毁信廢忠崇飾惡言服讒蒐慝以誣盛德天下之民謂之窮奇】摠典籍之所惡以為智術冒聖哲之所戒以為行能【惡鳥路翻行下孟翻】可謂堯代之共工【書堯典帝曰疇咨若予采驩兜曰共工方鳩僝功帝曰吁靜言庸違象恭滔天共音恭】魯邦之少卯也【家語孔子為魯司寇攝行相事七日而誅少正卯戮之于兩觀之下子貢進曰夫少正卯魯之聞人也夫子為政而始誅之或者為失乎孔子曰天下有大惡者五而竊盜不豫焉一曰心逆而險二曰行僻而堅三曰言偽而辯四曰記醜而博五曰順非而澤此五者有一於人則不免君子之誅而少正卯皆兼有之其居處足以撮徒成黨其談說足以飭褒榮衆其彊禦足以反是獨立此乃人之姦雄有不可以不除】蹟其姦蠧日長月滋【長知丈翻】隂祕者固未盡彰敗露者尤難悉數又曰陛下若意其負謗則誠宜亟為辨明【為于偽翻】陛下若知其無良又安可曲加容掩又曰陛下姑欲保持曾無詰問延齡謂能蔽惑不復懼思移東就西便為課績取此適彼遂號羨餘愚弄朝廷有同兒戲又曰矯詭之能誣罔之辭遇事輒行應口便靡日不有靡時不為又難以備陳也又曰昔趙高指鹿為馬【事見八卷秦二世三年】臣謂鹿之與馬物理猶同豈若延齡掩有為無指無為有又曰延齡凶妄流布寰區上自公卿近臣下逮輿臺賤品【左傳芊尹無宇曰王臣公公臣大夫大夫臣士士臣皁皁臣輿輿臣隸隸臣僚僚臣僕僕臣臺】諠諠談議億萬為徒能以上言【上時掌翻】其人有幾臣以卑鄙任當台衡情激于衷雖欲罷而不能自默也書奏上不悦待延齡益厚 十二月王䖍休乘氷合度壕急攻洺州元誼出兵擊之䖍休不勝而返日暮氷解士卒死者太半 中書侍郎同平章事陸䞇以上知待之厚事有不可常力爭之所親或規其太鋭䞇曰吾上不負天子下不負所學它無所恤裴延齡日短䞇於上趙憬之入相也䞇實引之既而有憾於贄【事見上卷八年九年】密以䞇所譏彈延齡事告延齡故延齡益得以為計上由是信延齡而不直贄贄與憬約至上前極論延齡姦邪上怒形於色憬默而無言壬戌贄罷為太子賓客 【考異曰韓愈順宗實錄曰德宗在位稍久益自攬機柄親治細事失人君大體宰相益不得行其職而議者乃云由贄而然按凡為宰相者皆欲專權安肯自求失職不任宰相乃德宗之失而歸咎於贄豈人情也又贄論朝官缺員狀云頃之輔臣鮮克勝任過蒙容養苟備職員致勞睿思巨細經慮此乃諫德宗不任宰相親治細事之辭也】初渤海文王欽茂卒子宏臨早死族弟元義立元義猜虐國人殺之立臨宏之子華嶼是為成王改元中興華嶼卒復立欽茂少子嵩鄰【復扶又翻】是為康王改元正歷【渤海自大祚榮立國開元之間其子武藝立益以彊盛東北諸夷皆畏而臣之故元仁安更五代以至於宋耶律雖數加兵不能服也故通鑑歷叙其世為詳】<br />
<br />
  十一年春二月乙巳冊拜嵩鄰為忽汗州都督勃海王【考異曰實錄云乙巳冊大嶺嵩鄰為勃海郡王今從新傳】 陸贄既罷相裴延齡<br />
<br />
  因譛京兆尹李充衛尉卿張滂前司農卿李銛黨於䞇會旱延齡奏言贄等失埶怨望言於衆曰天下旱百姓且流亡度支多欠諸軍芻糧軍中人馬無所食其事奈何【言其事勢將奈之何】以動搖衆心其意非止欲中傷臣而已【中竹仲翻言不獨以此為延齡罪且欲危社稷】後數日上獵苑中適有神策軍士訴云度支不給馬芻上意延齡言為信遽還宫夏四月壬戌貶䞇為忠州别駕充為涪州長史滂為汀州長史【開元二十四年開撫福二州山洞置汀州舊志忠州京師南二千一百二十二里汀州京師東南六千一百七十二里譙周巴記曰後漢初平六年江臨江縣屬永寧郡今忠州城東臨江古城是也後魏廢帝二年改為臨州因臨江縣以名州也隋廢州以其地併入巴東郡貞觀四年置忠州以其地連巴徼心懷忠信為名浪州漢涪陵縣地隋置涪州京師南二千三百五十里】銛為邵州長史【邵州京師東南三千四百里宋白曰邵州漢為昭陵縣吴改邵陵分零陵比部為邵陵郡隋立建州尋廢州以邵陵縣屬潭州唐貞觀十一年置邵州】初陽城自處士徵為諫議大夫【見二百三十二卷二年處昌呂翻】拜官不辭未至京師人皆想望風采曰城必諫諍死職下及至諸諫官紛紛言事細碎天子益厭苦之而城方與二弟及客日夜痛飲人莫能窺其際皆以為虚得名耳前進士河南韓愈作爭臣論以譏之【爭讀曰諍】城亦不以屑意有欲造城而問者【屑潔也顧也造七到翻】城揣知其意輒強與酒【揣初委翻強其兩翻】客或時先醉仆席上城或時先醉卧客懷中不能聽客語及陸䞇等坐貶上怒未解中外惴恐【惴之睡翻】以為罪且不測無敢救者城聞而起曰不可令天子信用姦臣殺無罪人即帥拾遺王仲舒歸登右補闕熊執易崔邠等守延英門【延英門延英殿門也程大昌曰案六典宣政殿門西上閤門之西即為延英門門之左曰延英殿故陽城欲救陸䞇約王仲舒等守延英殿閤上書伏閤不去也帥讀曰】上疏論延齡姦佞贄等無罪上大怒欲加城等罪太<br />
<br />
  子為之營救【為于偽翻】上意乃解令宰相諭遣之於是金吾將軍張萬福聞諫官伏閣諫趨往至延英門大言賀曰朝廷有直臣天下必太平矣遂遍拜城與仲舒等已而連呼太平萬歲太平萬歲萬福武人年八十餘自此名重天下登崇敬之子也【崇敬明禮家學歷事玄肅代及帝四世】時朝夕相延齡陽城曰脱以延齡為相城當取白麻壞之【唐故事中書用黄白二麻為綸命輕重之辯其後翰林學士專掌内命中書用黄麻其白皆在翰林院拜授將相德音赦宥則用之宋白曰唐故事白麻皆内庭代言命輔臣除節將恤災患討不庭則用之宰臣於正衙受付若命相之書則通事舍人承旨皆宣讃訖始下有司翰林志凡赦書德音立后建儲行大誅討拜免三公宰相命將日並使白麻紙不使印雙日起草候閤門鑰入而後進呈至隻日百寮並班於宣政殿樞密使引按自東上閤門出若拜免宰相即便付通事舍人餘付中書門下並通事舍人宣示若機務急速亦雙日甚速者雖休假亦追班宣示按制按也册則有册按冊公主亦作閤門出按壞音怪】慟哭於庭有李繁者泌之子也城盡疏延齡過惡欲密論之以繁故人子【陽城之除諫議李泌之薦也】使之繕寫繁徑以告延齡延齡先詣上一一自解疏入上以為妄不之省【省悉景翻】 丙寅幽州奏破奚王啜利等六萬餘衆 回鶻奉誠可汗卒無子國人立其相骨咄祿為可汗骨咄祿本姓跌氏【奚結翻跌徒結翻跌與回紇同出鐵勒而異種】辨慧有勇略自天親時【回鶻天親可汗合骨咄祿也】典兵馬用事大臣諸酋長皆畏服之既為可汗冒姓藥葛羅氏【回紇可汗姓藥葛羅骨咄祿捨其本姓冒其姓以嗣其國酋慈由翻長知兩翻】遣使來告喪自天親可汗以上子孫幼穉者皆内之闕庭【唐之闕庭也】五月丁丑以宣武留後李萬榮昭義左司馬領留後王䖍休皆為節度使 甲申河東節度使李自良薨戊子監軍王定遠奏請以行軍司馬李說為留後說神通之五世孫也【淮安王神通高祖之從弟起兵關西首應義旗說讀為悦下同】 庚寅遣祕書監張薦冊拜回鶻可汗骨咄祿為騰里邏羽錄没密施合胡祿毗伽懷信可汗【咄當没翻邏郎佐翻】 癸巳以李說為河東留後知府事說深德王定遠請鑄監軍印監軍有印自定遠始 秋七月丙寅朔陽城改國子司業坐言裴延齡故也 王定遠自恃有功於李說專河東軍政易置諸將說不能盡從由是有隙定遠以私怒拉殺大將彭令茵【拉盧合翻】埋馬矢中將士皆憤怒說奏其狀定遠聞之直詣說拔刀刺之【刺七亦翻】說走免定遠召諸將以箱貯敕及告身二十餘通【箱竹笥也貯丁呂翻】示之曰有敕令說詣京師以行軍司馬李景略為留後【李景略為李說所忌蓋起於此】諸君皆遷官衆皆拜大將馬良輔竊視箱中皆定遠告身及所受敕也乃麾衆曰敕告皆偽不可受也定遠走登乾陽樓【乾陽樓蓋晉陽宫城南門樓】呼其麾下莫應踰城而墜為枯枿所傷而死【枿五葛翻木之伐去者其遺餘為枿 考異曰舊說傳曰定遠殺彭令茵說具以事聞德宗以定遠有奉天扈從功恕死停任制未至定遠怒說奏聞趨府謀殺說昇堂未坐抽刀刺說說走而獲免又曰定遠墜城下槎枿傷而不死尋有詔削奪長流崖州今從實錄】 八月辛亥司徒兼侍中北平莊武王馬燧薨 閏月戊辰元誼以洺州詐降王䖍休遣裨將將二千人入城誼皆殺之【降戶江翻將郎亮翻】九月丁巳加韋臯雲南安撫使【以安撫南詔為官名】 横海節<br />
<br />
  度使程懷直不恤士卒獵於野數日不歸懷直從父兄懷信為兵馬使因衆心之怨閉門拒之懷直奔歸京師冬十月丁丑以懷信為横海留後 南詔攻吐蕃昆明城取之【昆明城在西㸑西北有鹽池之利】又虜施順二蠻王【施順二蠻皆烏蠻種施蠻在鐵橋西北居大施睒歛尋睒順蠻在劒睒西北四百里睒失冉翻】<br />
<br />
  十二年春正月庚子元誼石定蕃等帥洺州兵五千人及其家人萬餘口奔魏州【帥讀曰率】上釋不問命田緒安撫之 乙丑以渾瑊王武俊並兼中書令己巳加嚴震田緒劉濟韋臯並同平章事天下節度觀察使悉加檢校官以悦其意 三月甲午韋臯奏降西南蠻高萬唐等二萬餘口【降戶江翻】 乙巳以閑廄宫苑使李齊運為禮部尚書【閑廄宫苑二使李齊運蓋兼為之】戶部侍郎裴延齡為戶部尚書使職如故齊運無才能學術專以柔佞得幸于上每宰相對罷則齊運次進決其議或病卧家上欲有所除授往往遣中使就問之 丙子韶王暹薨【暹皇弟也】 魏博節度使田緒尚嘉城公主有庶子三人季安最幼公主子之以為副大使夏四月庚午緒暴薨左右匿之使季安領軍事年十五乙亥喪推季安為留後 庚辰上生日故事命沙門道士講論於麟德殿至是始命以儒士參之四門博士韋渠牟嘲談□給【後魏劉芳表云太和二十年立四門博士於四門置學按禮記云天子設四學鄭注云周四郊之虞庠也今以其遼遠故置於四門請移與太學同處從之唐百官志四門館博士正七品上掌教七品以上侯伯子男子為生及庶人子為俊士生者】上悦之旬月遷右補闕始有寵 五月丙申邠寧節度使張獻甫暴薨監軍楊明義請都虞候楊朝晟權知留後丙辰以朝晟為邠寧節度使 六月乙丑以監句當左神策竇文場監句當右神策霍仙鳴皆為護軍中尉監左神威軍使張尚進監右神威軍使焦希望皆為中護軍【句古翻當丁浪翻左右神策中尉始於竇霍自此宦官之權日以益重不可復制矣下護軍中尉一等為中護軍此職事官之掌禁兵者非如唐初所置勲級所謂上護軍護軍也宋白曰德宗以梁洋扈從之功舉西漢謁者隨何下淮南功拜為中尉事故命神策監軍為中尉】初上置六統軍視六尚書以處節度使罷鎮者【興元元年置六統軍事見二百二十九卷處昌呂翻】相承用麻紙寫制至是文場諷宰相比統軍降麻翰林學士鄭絪【絪音因】奏言故事惟封王命相用白麻今以命中尉不識陛下特以寵文場邪遂為著令也【著令者定著為令】上乃謂文場曰武德貞觀時中人不過員外將軍同正耳衣緋者無幾自輔國以來墮壞制度【衣於既翻墮讀曰隳壞音怪】朕今用爾不謂無私若復以麻制宣告天下必謂爾脅我為之矣【復扶又翻】文場叩頭謝遂焚其麻命并統軍自今中書降敕明日上謂絪曰宰相不能違拒中人朕得卿言方悟耳是時竇霍埶傾中外藩鎮將帥多出神策軍臺省清要亦有出其門者矣 宣武節度使李萬榮病風昏不知事霍仙鳴薦宣武押牙劉沐可委軍政辛巳以沐為行軍司馬 宣歙觀察使劉贊卒初上以奉天窘乏故還宫以來尤專意聚歛【歛力贍翻下同】藩鎮多以進奉市恩皆云稅外方圓【折則成方轉則成圓言於常税之外别自轉折以致貨財也】亦云用度羨餘其實或割留常賦或增歛百姓或減刻利祿或販鬻蔬果往往私自入所進纔什一二李兼在江西有月進韋臯在四川有日進其後常州刺史濟源裴肅以進奉遷浙東觀察使刺史進奉自肅始【濟子禮翻】及劉贊卒判官嚴綬掌留務竭府庫以進奉徵為刑部員外郎幕僚進奉自綬始綬蜀人也【史不能審其郡縣故止云蜀人】 李萬榮疾病其子迺為兵馬使甲申迺集諸將責李湛伊婁說張丕以不憂軍事斥之外縣【說讀為悦】上遣中使第五守進至汴州宣慰始畢軍士十餘人呼曰【呼火故翻下同】兵馬使勤勞無賞劉沐何人為行軍司馬沐懼陽中風舁出【中竹仲翻舁余又羊茹翻】軍士又呼曰倉官劉叔何給納有姦殺而食之又欲斫守進迺止之迺又殺伊婁說張丕【說讀曰悦】都虞□城鄧惟恭與萬榮鄉里相善萬榮常委以腹心迺亦倚之至是惟恭與監軍俱文珍謀執迺送京師秋七月乙未以東都留守董晉同平章事兼宣武節度使以萬榮為太子少保貶迺䖍州司馬丙申萬榮薨鄧惟恭既執李迺遂權軍事自謂當代萬榮不遣人迎董晉晉既受詔即與傔從十餘人赴鎮【傔苦念翻從才用翻】不用兵衛至鄭州迎者不至【九域志鄭州東至汴州一百五十里】鄭州人為晉懼【為于偽翻】或勸晉且留觀變有自汴州出者言於晉曰不可入晉不對遂行惟恭以晉來之速不及謀晉去城十餘里惟恭乃帥諸將出迎【帥讀曰率】晉命惟恭勿下馬氣色甚和惟恭差自安既入仍委惟恭以軍政初劉玄佐增汴州兵至十萬遇之厚李萬榮鄧惟恭每加厚焉士卒驕不能禦【禦一作御】乃置腹心之士幕於公庭廡下挾弓執劒以備之時勞賜酒肉【廡音武勞力到翻】晉至之明日悉罷之【董晉之意以謂此士前帥之腹心吾新來為帥若亦恃為腹心不足為吾衛而適足以生變罷之則待諸軍如一且示無所猜間】 戊戌韓王迥薨【迥上弟也】 壬子詔以宣武將士鄧惟恭等有執送李迺功各遷官賜錢其為迺所脅邀逼制使者皆勿問【脅所謂脅從也言李迺以威力脅使其下以邀逼中使唐時謂中使為敕使亦謂之制使使疏吏翻】八月乙未朔日有食之 己巳以田季安為魏博節度使 丙子以汝州刺史陸長源為宣武行軍司馬朝議以董晉柔仁多可恐不能集事【朝議謂朝廷之議多可言凡人有請悉從不能裁以理法】故以長源佐之長源性剛刻多更張舊事【更工衡翻】晉初皆許之案成則命且罷由是軍中得安【為長源以剛刻致祸張本】 丙戌門下侍郎同平章事趙憬薨 初上不欲生代節度使常自擇行軍司馬以為儲帥【行軍司馬掌弼戎政居則習蒐狩有役則申戰守之法器械糧備軍籍賜予皆專焉帥所類翻】李景略為河東行軍司馬李說忌之回鶻梅錄入貢過太原說與之宴梅錄爭坐次說不能遏景略叱之梅錄識其聲趨前拜之曰非豐州李端公邪【李景略折梅錄見二百三十三卷三年唐人呼侍御為端公李肇國史補曰宰相相呼曰堂老兩省曰閣老尚書曰院長御史曰端公】又拜遂就下坐座中皆屬目於景略【屬之欲翻】說益不平乃厚賂中尉竇文場使去之【去羌呂翻】會有傳回鶻將入寇者上憂之以豐州當虜衝擇可守者文場因薦景略九月甲午以景略為豐州都防禦使窮邊氣寒土瘠民貧景略以勤儉帥衆【帥讀曰率】二歲之後儲備完實雄於北邊 盧邁得風疾庚子賈躭私忌【父母及祖父母曾祖父母死日為私忌】宰相絶班【言宰相班絶無一人】上遣中使召主書承旨【唐制尚書省主書從八品下中書省從七品上堂吏也】 丙午戶部尚書判度支裴延齡卒中外相賀上獨悼惜之 壬子吐蕃寇慶州 冬十月甲戌以諫議大夫崔損給事中趙宗儒並同平章事損玄暐之弟孫也【崔玄暐有誅二張復中宗之功】嘗為裴延齡所薦故用之 十一月乙未以右補闕韋渠牟為左諫議大夫上自陸贄貶官【去年四月陸䞇貶】尤不任宰相自御史刺史縣令以上皆自選用中書行文書而已然深居禁中所取信者裴延齡李齊運戶部郎中王紹司農卿李實翰林學士韋執誼及渠牟皆權傾宰相趨附盈門紹謹密無損益實狡險掊克執誼以文章與上唱和【掊蒲侯翻和胡卧翻】年二十餘自右拾遺召入翰林渠牟形神恌躁【恌他彫翻】尤為上所親狎上每對執政漏不過三刻渠牟奏事率至六刻語笑欵狎往往聞外【聞音問】所薦引咸不次遷擢率皆庸鄙之士 宣武都虞候鄧惟恭内不自安潜結將士二百餘人謀作亂事覺董晉悉捕斬其黨械惟恭送京師己未詔免死汀州安置【投竄於荒遠州郡謂之安置】<br />
<br />
  十三年春正月壬寅吐蕃遣使請和親上以吐蕃數負約【數所角翻】不許 上以方渠合道木波皆吐蕃要路欲城之【九域志環州治通遠縣唐方渠縣地有木波馬嶺石昌合道四鎮】使問邠寧節度使楊朝晟須幾何兵對曰邠寧兵足以城之不煩他道上復使問之曰曏城鹽州【城鹽州見上卷九年復扶又翻】用兵七萬僅能集事今三城尤逼虜境兵當倍之事更相反何也對曰城鹽州之衆虜皆知之今本鎮兵不旬日至塞下出其不意而城之虜謂吾衆亦不減七萬其衆未集不敢輕來犯我不過三旬吾城已畢留兵戍之虜雖至無能為也【此後周韋孝寛城汾石之故智也】城旁草盡不能久留虜退則運芻糧以實之此萬全之策也若大集諸道兵踰月始至虜亦集衆而來與我爭戰勝負未可知何暇築城哉上從之二月朝晟分軍為三各築一城軍吏曰方渠無井不可屯軍判官孟子周曰方渠承平之時居人成市無井何以聚人乎命浚眢井【眢井廢井也眢烏歡翻】果得甘泉【方渠縣鹹河從土橋歸德川同家谷三處源來鹹苦不可食甜河在城西從蕃部鼻家族北界來供人飲食】三月三城成 【考異曰實錄先是邠寧楊朝晟奏方渠合道木波皆賊路也請城其地以備之詔問須幾何人邠志曰十三年春詔問楊公曰方渠合道木波皆賊路也城之可乎若以為可更要幾兵二月十一日起復除本官十四日制書到軍十八日軍二十六日軍次石堂谷二十八日功就三城今從邠志而不取其日】夏四月庚申楊朝晟軍還至馬嶺【唐馬嶺縣屬慶州劉昫曰馬嶺隋縣治天家堡貞觀八年移理新城以縣西有馬嶺坂宋白曰鹽州治五原即漢馬嶺縣地今州南抵慶州馬嶺縣北界杜佑馬嶺縣漢舊牧地川形似馬嶺】吐蕃始出兵追之相拒數日而去朝晟遂城馬嶺而還開地三百里皆如其素【皆如其素所慮之期也建中間朔方兵破李納軍朝晟為之也蓋其智畧誠有足稱者還從宣翻又如字】 庚午義成節度使李復薨庚辰以陜虢觀察使姚南仲為義成節度使監軍薛盈珍方大會聞之言曰姚大夫書生豈將才也【將即亮翻】判官盧坦私謂人曰姚大夫外雖柔中甚剛監軍侵之必不受軍府之禍自此始矣吾恐為所留遂自它道潛去南仲果以牒請之不遇得免既而盈珍與南仲有隙幕府多以罪貶有死者【事見後十六年史言盧坦庶乎見幾】 吐蕃贊普乞立贊卒子足之煎立 六月壬午韋臯奏吐蕃入寇嶲州刺史曹高仕破之於臺登城下【臺登漢縣唐屬嶲州由清溪關西南至臺登五百五十里】 光祿少卿同正張茂宗【員外置同正員起於高宗之時】茂昭之弟也【茂昭時為義武節度使】許尚義章公主【義章公主上女也義章縣名屬彬州宋白曰漢彬縣地隋末蕭銑分彬縣立】未成昏茂宗母卒遺表請終嘉禮上許之秋八月癸酉起復茂宗左衛將軍同正左拾遺義興蔣乂上疏諫 【考異曰實錄作蔣武按舊傳乂本名武】以為兵革之急古有墨衰從事者【衰倉回翻左傳晉文公卒未葬秦穆公伐鄭晉襄公墨衰絰以敗秦師于殽】未聞駙馬起復尚主也上遣中使諭之不止乃特召對於延英【唐中世以後召對宰輔乃開延英今蔣乂特以拾遺召對】謂曰人間多借吉成昏者卿何執此之堅對曰昏姻喪紀人之大倫吉凶不可凟也委巷之家不知禮教【委巷曲巷也言其屈曲僻陋】其女孤貧無恃【言貧而喪其親也】或有借吉從人未聞男子借吉娶婦者也太常博士韋彤裴堪復上疏諫【復扶又翻】上不悦命趣下嫁之期【趣讀曰促】辛巳成婚 九月己丑中書侍郎同平章事盧邁以病罷為太子賓客 冬十月淮西節度使吳少誠擅開刀溝入汝【刀溝新舊書皆作司洧水】上遣中使諭止之不從命兵部郎中盧羣往詰之【詰去吉翻】少誠曰開此水大利於人羣曰君令臣行雖利人臣敢專乎公承天子之令而不從何以使下吏從公之令乎少誠遽為之罷役【為于偽翻史言杖大義者獷悍不能不為之革面】 十二月徐州節度使張建封入朝先是宫中市外間物令官吏主之隨給其直【先悉薦翻】比歲以宦者為使【比毗至翻近也】謂之宫市抑買人物稍不如本估【估者價也】其後不復行文書【復扶又翻下同】置白望數百人於兩市【白望者言使人於市中左右望白取其物不還本價也兩市長安城中東市西市也隋名東市曰都會西市曰利人】及要閙坊曲閲人所賣物但稱宫市則歛手付與真偽不復可□無敢問所從來及論價之高下者率用直百錢物買人直數千物多以紅紫染故衣敗繒【繒慈陵翻】尺寸裂而給之仍索進奉門戶及脚價錢【索山客翻進奉門戶者言進奉所經由門戶皆有費用如漢靈帝時所謂導行費也脚價謂僦人負荷進奉物入内有雇脚之費】人將物詣市【將齎持也】至有空手而歸者名為宫市其實奪之商賈有良貨皆深匿之【賈音古】每敕使出雖沽漿賣餅者皆撤業閉門嘗有農夫以驢負柴宦者稱宫市取之與絹數尺又就索門戶【索山客翻】仍邀驢送柴至内農夫啼泣以所得絹與之不肯受曰須得爾驢【須者意所欲也】農夫曰我有父母妻子待此然後食【言待此驢負物貿易然後可以給食】今以柴與汝不取直而歸汝尚不肯我有死而已遂敺宦者街吏擒以聞 【烏口翻街吏即金吾左右街使之屬吏】詔黜宦者賜農夫絹十匹然宫市亦不為之改諫官御史數諫不聽【為于偽翻數所角翻】建封入朝具奏之上頗嘉納以問戶部侍郎判度支蘇弁弁希宦者意對曰京師游手萬家無土著生業【著直畧翻】仰宫市取給【仰牛向翻】上信之故凡言宫市者皆不聽<br />
<br />
  十四年春二月乙亥名申光蔡軍曰彰義【吳少誠時據淮西有申光蔡三州】 夏閏五月庚申以神策行營節度使韓全義為夏綏銀宥節度使全義時屯長武城詔帥其衆赴鎮【帥讀曰率】士卒以夏州磧鹵【磧沙磧鹵醎鹵磧鹵之地五穀不生磧七迹翻】又盛夏不樂徙居【樂音洛】辛酉軍亂殺大將王栖巖全義踰城走【史言韓全義駑怯無御衆之略徒以憑結宦官致節鉞】都虞候高崇文誅首亂者衆然後定崇文幽州人也丙子以崇文為長武城都知兵馬使不降敕令中使口宣授之【口宣聖旨而授之官使掌兵史言德宗重宦臣而輕詔命】 秋七月壬申給事中同平章事趙宗儒罷為右庶子以工部侍郎鄭餘慶為中書侍郎同平章事 八月初置左右神策統軍【觀此則知神策在六軍之外】時禁軍戍邊稟賜優厚【稟給也】諸將多請遙隸神策軍稱行營皆統於中尉其軍遂至十五萬人 京兆尹吳湊屢言宫市之弊宦者言湊屢奏宫市皆右金吾都知趙洽田秀嵓之謀也丙午洽秀嵓坐流天德軍【都知金吾府吏右職也】 九月丙申以陜虢觀察使于頔為山南東道節度使【頔音迪】 丁卯王倕薨【倕肅宗子倕音垂】 彰武節度使吳少誠遣兵掠壽州霍山【彰武當作彰義霍山本漢廬江之灊城縣梁置霍州隋置霍山縣唐屬壽州開元二十七年改霍山曰盛唐天寶初析盛唐别置霍山縣其地屬今壽州六安縣界】殺鎮遏使謝詳【宋白曰貞元六年初置藍田渭橋等鎮遏使】侵地五十餘里置兵鎮守 太學生薛約師事司業陽城坐言事徙連州城送之郊外上以城黨罪人己巳左遷城道州刺史 【考異曰實錄新舊傳無年月柳宗元陽公遺愛碣曰四年五月皇帝以銀印赤紱即隱所起陽公為諫議大夫後七年廷諍懇至帝尤嘉異遷國子司業又四年九月己巳出拜道州刺史太學魯郡季償廬江何蕃等百六十人投業奔走稽首闕下叫閽顯天願乞復舊朝廷重更其事如己巳詔今從之】城治民如治家州之賦税不登觀察使數加誚讓【治直之翻數所角翻誚才笑翻】城自署其考曰撫字心勞徵科政拙考下下觀察使遣判官督其賦至州城先自囚於獄判官大驚馳入謁城於獄曰使君何罪某奉命來候安否耳留一二日未去城不復歸【復扶又翻】館門外有故門扇横地城書夜坐卧其上判官不自安辭去其後又遣他判官往按之他判官載妻子中道逸去【陽城之名德人知敬之彼不之知而使按之者果何人也】 冬十月丁酉通王諶薨【諶上子也音氏壬翻】 庚子夏州節度使韓全義奏破吐蕃於鹽州西北 明州鎮將栗鍠【姓譜栗姓栗陸氏之後漢長安有富室栗氏】殺刺史盧雲誘山越作亂攻䧟浙東州縣【明州山越今慈溪鄞縣南界奉化縣西北界山民也鍠戶盲翻又音皇誘音酉】<br />
<br />
  十五年春正月甲寅雅王逸薨【逸皇弟也】 二月丁丑宣武節度使董晉薨乙酉以其行軍司馬陸長源為節度使長源性刻急恃才傲物判官孟叔度輕佻淫縱【佻他彫翻】好慢侮將士軍中皆惡之【惡烏路翻】董晉薨長源知留後揚言曰將士弛慢日久當以法齊之耳衆皆懼或勸之發財以勞軍【勞力到翻】長源曰我豈河北賊以錢買健兒求節钺邪故事主帥薨【帥所類翻】給軍士布以制服長源命給其直叔度高鹽直下布直人不過得鹽三二斤軍中怨怒長源亦不為之備是日軍士作亂殺長源叔度臠食之立盡【史言陸長源之死唐朝用違其才耳若孟叔度則死有餘罪】監軍俱文珍以宋州刺史劉逸準久為宣武大將得衆心密書召之逸準引兵徑入汴州亂衆乃定 以常州刺史李錡為浙西觀察使諸道鹽鐵轉運使錡國貞之子也【錡魚豈翻又音奇肅宗末李國貞為絳州行營兵所殺】閑廄宫苑使李齊運受其賂數十萬薦之於上故用之錡刻剝以事進奉上由是悦之【為李錡以浙西叛張本】 庚辰浙東觀察使裴肅擒栗鍠於台州斬之 己丑以劉逸準為宣武節度使賜名全諒 三月甲寅吳少誠遣兵襲唐州殺監軍邵國朝鎮遏使張嘉瑜掠百姓千餘人而去 戊午昭義節度使王䖍休薨戊辰以河陽懷州節度使李元淳為昭義節度使 癸巳山南西道節度使嚴震薨 南詔異牟尋遣使與韋臯約共擊吐蕃臯以兵糧未集請俟它年【韋臯有智畧恐南詔貌與而未悉其心也故以兵糧未集辭此可與智者道】 山南西道都虞候嚴礪謟事嚴震震病使知留後遺表薦之秋七月乙巳以礪為山南西道節度使 八月陳許節度使曲環薨乙未吳少誠遣兵掠臨潁【臨潁漢古縣唐屬許州九域志在許州東南六十里宋白曰隋大業四年自故城移於臨穎臯其地實岡阜也】陳州刺史上官涚知陳許留後遣大將王令忠將兵三千救之皆為少誠所虜丙午以涚為陳許節度使【涚舒芮翻】少誠遂圍許州涚欲弃城走營田副使劉昌裔止之曰城中兵足以辦賊但閉城勿與戰不過數日賊氣自衰吾以全制其弊蔑不克矣【蔑無也】少誠晝夜急攻昌裔募勇士千人鑿城出擊少誠大破之城由是全昌裔兖州人也少誠又寇西華【西華漢縣唐屬陳州九域志在州西八十里】陳許大將孟元陽拒却之陳許都知兵馬使安國寧與上官涚不叶謀翻城應少誠劉昌裔以計斬之召其麾下人給二縑伏兵要巷見持縑者悉斬之無得脱者庚辰宣武節度使劉全諒薨軍中思劉玄佐之恩推其甥都知兵馬使匡城韓弘為留後弘將兵識其材鄙勇怯指顧必堪其事 丙辰詔削奪吳少誠官爵令諸道進兵討之 辛酉以韓弘為宣武節度使先是少誠與劉全諒約共攻陳許【先悉薦翻】以陳州歸宣武使者數輩猶在館弘悉驅出斬之選卒三千會諸軍擊少誠於許下少誠由是失埶【無同惡相濟故失勢】 冬十月乙丑邕王謜薨【謜徐園翻】太子之子也上愛而子之及薨諡曰文敬太子 山南東道節度使于頔安黄節度使伊慎知壽州事王宗與上官涗韓弘進擊吳少誠屢破之十一月壬子于頔奏拔吳房朗山【後魏置襄城郡於漢汝南西平之地仍置遂寧縣隋大業初改曰吳房吴房本漢縣名應劭曰本房子國楚以封吳夫概王故曰吳房朗山漢安昌縣地後魏置初安郡隋開皇十八年改安昌為朗山唐並屬蔡州宋朝避聖祖諱改朗山為確山九域志吳房在蔡州西北七十里朗山在蔡州西南七十五里】 十二月辛未中書令咸寧王渾瑊薨于河中【渾瑊封咸寧郡王】瑊性謙謹雖位窮將相無自矜大之色每貢物必躬自閲視受賜如在上前由是為上所親愛上還自興元雖一州一鎮有兵者皆務姑息瑊每奏事不過【唐制凡奏事得可者皆過門下省中書省不過者寢其奏不下也】輒私喜曰上不疑我故能以功名終 六州党項自永泰以來居于石州【代宗永泰之後改為大歷六州党項部落曰野利越詩野利龍兒野利厥律兒黄野海野窣等居慶州號東山部夏州號平夏部永泰之後稍徙石州】永安鎮將阿史那思暕侵漁不已【唐蓋置永安鎮將於石州以綏御党項暕古限翻】党項部落悉逃奔河西 諸軍討吳少誠者既無統帥【帥所類翻】每出兵人自規利【規圖也】進退不壹乙未諸軍自潰於小溵水【溵與㶏同音殷又音隱水經注潁水東南過臨潁縣小㶏水注之又東過西華縣北又南過汝陽縣北又東南過南頓縣北大㶏水從西來注之宋白曰蔡州汝陽縣隋開皇十七年改為溵水今界内水有大溵小溵之名其年又於上蔡縣東北别置汝陽縣】委弃器械資糧皆為少誠所有於是始議置招討使 吐蕃衆五萬分擊南詔及嶲州異牟尋與韋臯各兵禦之吐蕃無功而還【還音旋又如字】<br />
<br />
  十六年春正月恒冀易定陳許河陽四軍與吳少誠戰皆不利而退夏綏節度使韓全義本出神策軍中尉竇文場愛厚之薦於上使統諸軍討吳少誠二月乙酉以全義為蔡州四面行營招討使十七道兵皆受全義節度【為韓全義喪師張本】 宣武軍自劉玄佐薨凡五作亂【貞元八年玄佐薨汴卒拒吳湊而立其子士寧李萬榮既逐士寧十年韓惟清等亂十二年萬榮死其子迺以兵亂董晉既入汴鄧惟恭復謀亂十四年晉薨兵又亂殺留後凡五亂】士卒益驕縱輕其主帥【帥所類翻】韓弘視事數月皆知其主名有郎將劉鍔常為唱首三月弘陳兵牙門召鍔及其黨三百人數之以數預於亂【數之之數音所具翻數預之數所角翻】自以為功悉斬之血流丹道自是至弘入朝【憲宗元和十四年韓弘入朝】二十一年士卒無一人敢讙呼於城郭者【讙許元翻呼火故翻】 義成監軍薛盈珍為上所寵信欲奪節度使姚南仲軍政南仲不從由是有隙盈珍譛其幕僚馬摠貶泉州别駕福建觀察使柳冕謀害摠以媚盈珍遣幕僚寶鼎薛戎攝泉州事使按致摠罪戎為辨析其無辜【為于偽翻】冕怒召戎囚之使守卒恣為侵辱如此彌月徐誘之使誣摠戎終不從摠由是獲免冕芳之子也【柳芳有史學事玄宗肅宗】盈珍屢毁南仲於上上疑之盈珍乃遣小吏程務盈乘驛誣奏南仲罪牙將曹文洽亦奏事長安知之晨夜兼行追及務盈於長樂驛【長樂驛在長安城東滻坡】與之同宿中夜殺之沈盈珍表於厠中自作表雪南仲之寃且首專殺之罪【首式又翻】亦作狀白南仲遂自殺明旦門不啟驛吏排之入得表狀於文洽尸旁上聞而異之徵盈珍入朝南仲恐盈珍讒之益深亦請入朝夏四月丙子南仲至京師待罪於金吾【金吾左右仗凡内外官之待罪者詣焉】詔釋之召見【見賢遍翻】上問盈珍擾卿邪對曰盈珍不擾臣但亂陛下法耳且天下如盈珍輩何可勝數【勝音升數所具翻】雖使羊杜復生【羊杜謂羊祜杜預復扶又翻】亦不能行愷悌之政成攻取之功也上默然竟不罪盈珍乃使掌機密盈珍又言於上曰南仲惡政皆幕僚馬少微贊之也詔貶少微江南官遣中使送之推墜江中而死【推吐雷翻】 黔中觀察使韋士宗政令苛刻【黔渠今翻】丁亥牙將傳近等逐之出奔施州【九域志黔州東北至施州四百一十一里】 新羅王敬則卒庚寅冊命其嫡孫俊邕為新羅王 韓全義素無勇畧專以巧佞貨賂結宦官得為大帥【帥所類翻】每議軍事宦者為監軍者數十人坐帳中爭論紛然莫能決而罷天漸暑士卒久屯沮洳之地【沮將預翻洳人恕翻沮洳漸濕也】多病疫人有離心五月庚戌與吳少誠將吳秀吳少陽等戰于溵南廣利原【溵南溵水之南也】鋒鏑纔交諸軍大潰秀等乘之全義退保五樓【五樓在溵水縣西南】少陽滄州清池人也【宋白曰漢浮陽縣隋開皇十八年改曰清池因縣東南有清池為名】 山南東道節度使于頔因討吳少誠大募戰士繕甲厲兵聚歛貨財恣行誅殺有據漢南之志專以慢上陵下為事上方姑息藩鎮知其所為無如之何頔誣鄧州刺史元洪贓罪【至德元載升襄陽防禦使為山南東道節度使領襄鄧隨唐安均房金商九州貞元元年以鄧州隸東都畿以此觀之此時復領鄧州矣】朝廷不得已流洪端州遣中使護送至棗陽【棗陽漢舂陵之地隋置棗陽縣唐初屬唐州貞觀十一年廢屬隨州九域志在州西北一百六十里距襄州一百三十五里】頔遣兵劫取歸襄州中使奔歸頔表洪責太重上復以洪為吉州長史乃遣之【復扶又翻】又怒判官薛正倫奏貶峽州長史比敕下【比必利翻及也】頔怒已解復奏留為判官上一一從之徐泗濠節度使張建封鎮彭城十餘年【貞元四年張建封鎮彭城】軍府稱治【治直吏翻】病篤請除代人辛亥以蘇州刺史韋夏卿為徐泗濠行軍司馬敕下建封已薨夏卿執誼之從祖兄也徐州判官鄭通誠知留後恐軍士為變會浙西兵過彭城通誠欲引入城為援軍士怒壬子數千人斧庫門出甲兵擐執之【擐音患】圍牙城刼建封子前虢州參軍愔令知軍府事【愔挹淫翻】殺通誠及大將段伯熊等數人械繫監軍上聞之以吏部員外郎李鄘為徐州宣慰使鄘直抵其軍【鄘余封翻】召將士宣朝旨諭以禍福【朝直遙翻下同】脱監軍械使復其位凶黨不敢犯愔上表稱兵馬留後鄘以非朝命不受使削去然後受之以歸【去羌呂翻】 霛州破吐蕃於烏蘭橋【唐書地理志會州烏蘭縣有烏蘭關橋當在關外黄河上】 丙寅韋士宗復入黔中【是年四月韋士宗為牙將傅近所逐黔音禽又其廉翻】 湖南觀察使河中呂渭奏永州刺史陽履賄履表稱所歛物皆備進奉上召詣長安丁丑命三司使鞫之詰其物費用所歸履曰己市馬進之矣又詰馬主為誰馬齒幾何對曰馬主東西南北之人今不知所之按禮齒路馬有誅【曲禮之言】故不知其齒所對率如此上悅其進奉之言釋之但免官而已【德宗之猜忌如楊炎竇參位居宰輔皆以歸過於君不置之地上陽履以敗而表稱進奉謂非歸過於君可乎德宗悦其進奉之言而釋其罪夫好貨非美名也人雖有好貨者苟加以好貨之名則必怫然而不受德宗果何為而安受此名也余意陽履於賄既敗之後必有所進以求免於罪德宗不徒悦其言而已】 丙戌加淄青節度使李師古同平章事 徐州亂兵為張愔表求旄節【為于偽翻】朝廷不許加淮南節度使杜佑同平章事兼徐濠泗節度使使討之佑大具舟艦遣牙將孟準為前鋒濟淮而敗佑不敢進泗州刺史張伾出兵攻埇橋大敗而還朝廷不得已除愔徐州團練使以伾為泗州留後濠州刺史杜兼為濠州留後仍加佑兼濠泗觀察使【分濠泗隸淮南以弱徐州之權 考異曰實錄十二月癸卯泗州濠州宣令淮南觀察使收管今因此終言之】兼正倫五世孫也【杜正倫相太宗高宗】性狡險彊忍建封之疾亟也兼隂圖代之自濠州疾驅至府幕僚李藩與同列入問建封疾出見之泣曰僕射疾危如此【張建封加僕射故稱之】公宜在州防遏今弃州此來欲何為也宜速去不然當奏之兼錯愕出不意遂徑歸建封薨藩歸揚州兼誣奏藩於建封之薨搖動軍情上大怒密詔杜佑使殺之佑素重藩懷詔旬日不忍因引藩論佛經曰佛言果報有諸藩曰有之【佛經言人所造作善惡為果隨其所作而應之以禍福為報】佑曰審如此君宜遇事無恐因出詔示藩藩神色不變曰此真報也佑曰君慎勿出口吾已密論用百口保君矣【人謂其家之親屬為百口】上猶疑之召藩詣長安望見藩儀度安雅乃曰此豈為惡者邪即除祕書郎 新羅王俊邕卒國人立其子重熙【重直龍翻】 秋七月吳少誠進擊韓全義於五樓諸軍復大敗【復扶又翻下同】全義夜遁保溵水【溵水縣漢汝陽縣地隋置溵水縣廢汝陽入焉唐屬陳州九域志在州西南八十里】 盧龍節度使劉濟弟源為涿州刺史不受濟命濟引兵擊擒之九月癸卯義成節度使盧羣薨甲戌以尚書左丞李元素代之賈躭曰凡就軍中除節度使必有愛憎向背【背蒲妹翻】喜懼者相半故衆心多不安自今願陛下只自朝廷除人庶無它變上以為然 中書侍郎同平章事鄭餘慶與戶部侍郎判度支于䪹素善䪹所奏事餘慶多勸上從之上以為朋比【䪹薄諧翻又蒲回翻比毗至翻】庚戌貶餘慶郴州司馬䪹泉州司戶【郴丑林翻宋白曰泉州江左之晉安郡隋置泉州舊理閩縣後移於南安縣唐聖歷元年分泉州之南安莆田龍溪三縣置武榮州景雲二年改泉州舊志泉州京師東南七千三百里考異曰舊傳曰時歲旱人饑與宰相議將賑給禁衛十軍事未行為中書吏所洩餘慶貶郴州司馬按實錄】<br />
<br />
  【餘慶與于䪹同貶餘慶制辭云乃乖正直有涉比周弃法弄情公行黨庇䪹制辭云性本纎狡行惟黨附奏對每乖於事實傾邪有蠧于彜章今從之】䪹頔之兄也 癸丑吳少誠進逼溵水數里置營韓全義復帥諸軍退保陳州【帥讀曰率下同】宣武河陽兵私歸本道獨陳許將孟元陽神策將蘇光榮帥所部留軍溵水全義以詐誘昭義將夏侯仲宣義成將時昂河陽將權文變河中將郭湘等斬之欲以威衆全義至陳州刺史劉昌裔登城謂之曰天子命公討蔡州今乃來此昌裔不敢納請舍於城外既而昌裔齎牛酒入全義營犒師全義驚喜心服之己未孟元陽等與吳少誠戰殺二千餘人 庚申以太常卿齊抗為中書舍人同平章事【新書宰相表齊抗為中書侍郎同平章事】 癸亥以張愔為徐州留後 冬十月吳少誠引兵還蔡州【孟元陽折其鋒故退】先是韋臯聞諸軍討少誠無功【先悉薦翻】上言請以渾瑊賈躭為元帥統諸軍【渾瑊薨於去年十二月韋臯蓋上言於瑊未薨之前】若重煩元老【重難也】則臣請以精鋭萬人下巴峽出荆楚以翦凶逆【臯欲為元帥然亦以大言衒朝廷耳彼豈肯去西川邪】不然因其請罪而赦之罷兩河諸軍以休息公私亦策之次也若少誠一旦罪盈惡稔為麾下所殺則又當以其爵位授之是除一少誠生一少誠為患無窮矣賈躭言于上曰賊意蓋亦望恩貸恐須開其生路上從之會少誠致書幣於監官軍者求昭洗監軍奏之戊子詔赦少誠及彰義將士復其官爵【考異曰實錄九月壬寅宰相對於延英賈躭奏曰昨韓全義五樓退軍賊不敢追趂者應望國家恩貸恐須】<br />
<br />
  【開其生路上是之按全義自五樓退保溵水少誠逼溵水下營全義又退保陳州非不敢追趂也又云諸軍討蔡州未嘗整陣交鋒而王師累挫潰吳少誠知王師無能為致書幣以告監軍願求昭洗上既納賈躭之議又得監軍善奏遂復其官爵按少誠知王師無能為則愈當侵軼豈肯從監軍求昭洗蓋少誠起兵以來不能無疲弊故求休息耳今不取】 己丑河東節度使李說薨甲午以其行軍司馬鄭儋為節度使【儋都甘翻】上擇可以代儋者以刑部員外郎嚴綬嘗以幕僚進奉【嚴綬進奉事見上十二年】記其名【史言德宗好貨惟進奉者則牢記其姓名】即用為行軍司馬 吐蕃數為韋臯所敗【數所角翻下同敗補邁翻】是歲其曩貢臘城等九節度嬰籠官馬定德帥其部落來降定德有智略吐蕃諸將行兵皆稟其謀策常乘驛計事至是以兵數不利恐獲罪遂來奔【據舊書云吐蕃酋帥兼監統曩貢臘城等九節度□□籠宜馬定德與其大將八十七人舉部落來降定德有計畫□□知兵法及山川地形至是以邊功不立慁得罪而來如此則□□别是一人籠官馬定德又是一人考之字書亦無□字然通鑑所書全用舊書韋臯傳文蓋舊書韋臯傳與吐蕃傳自相牴牾帥讀曰率】<br />
<br />
  資治通鑑卷二百三十五  <br>
   </div> 

<script src="/search/ajaxskft.js"> </script>
 <div class="clear"></div>
<br>
<br>
 <!-- a.d-->

 <!--
<div class="info_share">
</div> 
-->
 <!--info_share--></div>   <!-- end info_content-->
  </div> <!-- end l-->

<div class="r">   <!--r-->



<div class="sidebar"  style="margin-bottom:2px;">

 
<div class="sidebar_title">工具类大全</div>
<div class="sidebar_info">
<strong><a href="http://www.guoxuedashi.com/lsditu/" target="_blank">历史地图</a></strong>  
<a href="http://www.880114.com/" target="_blank">英语宝典</a>  
<a href="http://www.guoxuedashi.com/13jing/" target="_blank">十三经检索</a> 
<br><strong><a href="http://www.guoxuedashi.com/gjtsjc/" target="_blank">古今图书集成</a></strong> 
<a href="http://www.guoxuedashi.com/duilian/" target="_blank">对联大全</a> <strong><a href="http://www.guoxuedashi.com/xiangxingzi/" target="_blank">象形文字典</a></strong> 

<br><a href="http://www.guoxuedashi.com/zixing/yanbian/">字形演变</a>  <strong><a href="http://www.guoxuemi.com/hafo/" target="_blank">哈佛燕京中文善本特藏</a></strong>
<br><strong><a href="http://www.guoxuedashi.com/csfz/" target="_blank">丛书&方志检索器</a></strong> <a href="http://www.guoxuedashi.com/yqjyy/" target="_blank">一切经音义</a>  

<br><strong><a href="http://www.guoxuedashi.com/jiapu/" target="_blank">家谱族谱查询</a></strong>  <strong><a href="http://shufa.guoxuedashi.com/sfzitie/" target="_blank">书法字帖欣赏</a></strong> 
<br>

</div>
</div>


<div class="sidebar" style="margin-bottom:0px;">

<font style="font-size:22px;line-height:32px">QQ交流群9:489193090</font>


<div class="sidebar_title">手机APP 扫描或点击</div>
<div class="sidebar_info">
<table>
<tr>
	<td width=160><a href="http://m.guoxuedashi.com/app/" target="_blank"><img src="/img/gxds-sj.png" width="140"  border="0" alt="国学大师手机版"></a></td>
	<td>
<a href="http://www.guoxuedashi.com/download/" target="_blank">app软件下载专区</a><br>
<a href="http://www.guoxuedashi.com/download/gxds.php" target="_blank">《国学大师》下载</a><br>
<a href="http://www.guoxuedashi.com/download/kxzd.php" target="_blank">《汉字宝典》下载</a><br>
<a href="http://www.guoxuedashi.com/download/scqbd.php" target="_blank">《诗词曲宝典》下载</a><br>
<a href="http://www.guoxuedashi.com/SiKuQuanShu/skqs.php" target="_blank">《四库全书》下载</a><br>
</td>
</tr>
</table>

</div>
</div>


<div class="sidebar2">
<center>


</center>
</div>

<div class="sidebar"  style="margin-bottom:2px;">
<div class="sidebar_title">网站使用教程</div>
<div class="sidebar_info">
<a href="http://www.guoxuedashi.com/help/gjsearch.php" target="_blank">如何在国学大师网下载古籍?</a><br>
<a href="http://www.guoxuedashi.com/zidian/bujian/bjjc.php" target="_blank">如何使用部件查字法快速查字?</a><br>
<a href="http://www.guoxuedashi.com/search/sjc.php" target="_blank">如何在指定的书籍中全文检索?</a><br>
<a href="http://www.guoxuedashi.com/search/skjc.php" target="_blank">如何找到一句话在《四库全书》哪一页?</a><br>
</div>
</div>


<div class="sidebar">
<div class="sidebar_title">热门书籍</div>
<div class="sidebar_info">
<a href="/so.php?sokey=%E8%B5%84%E6%B2%BB%E9%80%9A%E9%89%B4&kt=1">资治通鉴</a> <a href="/24shi/"><strong>二十四史</strong></a>&nbsp; <a href="/a2694/">野史</a>&nbsp; <a href="/SiKuQuanShu/"><strong>四库全书</strong></a>&nbsp;<a href="http://www.guoxuedashi.com/SiKuQuanShu/fanti/">繁体</a>
<br><a href="/so.php?sokey=%E7%BA%A2%E6%A5%BC%E6%A2%A6&kt=1">红楼梦</a> <a href="/a/1858x/">三国演义</a> <a href="/a/1038k/">水浒传</a> <a href="/a/1046t/">西游记</a> <a href="/a/1914o/">封神演义</a>
<br>
<a href="http://www.guoxuedashi.com/so.php?sokeygx=%E4%B8%87%E6%9C%89%E6%96%87%E5%BA%93&submit=&kt=1">万有文库</a> <a href="/a/780t/">古文观止</a> <a href="/a/1024l/">文心雕龙</a> <a href="/a/1704n/">全唐诗</a> <a href="/a/1705h/">全宋词</a>
<br><a href="http://www.guoxuedashi.com/so.php?sokeygx=%E7%99%BE%E8%A1%B2%E6%9C%AC%E4%BA%8C%E5%8D%81%E5%9B%9B%E5%8F%B2&submit=&kt=1"><strong>百衲本二十四史</strong></a>  <a href="http://www.guoxuedashi.com/so.php?sokeygx=%E5%8F%A4%E4%BB%8A%E5%9B%BE%E4%B9%A6%E9%9B%86%E6%88%90&submit=&kt=1"><strong>古今图书集成</strong></a>
<br>

<a href="http://www.guoxuedashi.com/so.php?sokeygx=%E4%B8%9B%E4%B9%A6%E9%9B%86%E6%88%90&submit=&kt=1">丛书集成</a> 
<a href="http://www.guoxuedashi.com/so.php?sokeygx=%E5%9B%9B%E9%83%A8%E4%B8%9B%E5%88%8A&submit=&kt=1"><strong>四部丛刊</strong></a>  
<a href="http://www.guoxuedashi.com/so.php?sokeygx=%E8%AF%B4%E6%96%87%E8%A7%A3%E5%AD%97&submit=&kt=1">說文解字</a> <a href="http://www.guoxuedashi.com/so.php?sokeygx=%E5%85%A8%E4%B8%8A%E5%8F%A4&submit=&kt=1">三国六朝文</a>
<br><a href="http://www.guoxuedashi.com/so.php?sokeytm=%E6%97%A5%E6%9C%AC%E5%86%85%E9%98%81%E6%96%87%E5%BA%93&submit=&kt=1"><strong>日本内阁文库</strong></a> <a href="http://www.guoxuedashi.com/so.php?sokeytm=%E5%9B%BD%E5%9B%BE%E6%96%B9%E5%BF%97%E5%90%88%E9%9B%86&ka=100&submit=">国图方志合集</a> <a href="http://www.guoxuedashi.com/so.php?sokeytm=%E5%90%84%E5%9C%B0%E6%96%B9%E5%BF%97&submit=&kt=1"><strong>各地方志</strong></a>

</div>
</div>


<div class="sidebar2">
<center>

</center>
</div>
<div class="sidebar greenbar">
<div class="sidebar_title green">四库全书</div>
<div class="sidebar_info">

《四库全书》是中国古代最大的丛书,编撰于乾隆年间,由纪昀等360多位高官、学者编撰,3800多人抄写,费时十三年编成。丛书分经、史、子、集四部,故名四库。共有3500多种书,7.9万卷,3.6万册,约8亿字,基本上囊括了古代所有图书,故称“全书”。<a href="http://www.guoxuedashi.com/SiKuQuanShu/">详细>>
</a>

</div> 
</div>

</div>  <!--end r-->

</div>
<!-- 内容区END --> 

<!-- 页脚开始 -->
<div class="shh">

</div>

<div class="w1180" style="margin-top:8px;">
<center><script src="http://www.guoxuedashi.com/img/plus.php?id=3"></script></center>
</div>
<div class="w1180 foot">
<a href="/b/thanks.php">特别致谢</a> | <a href="javascript:window.external.AddFavorite(document.location.href,document.title);">收藏本站</a> | <a href="#">欢迎投稿</a> | <a href="http://www.guoxuedashi.com/forum/">意见建议</a> | <a href="http://www.guoxuemi.com/">国学迷</a> | <a href="http://www.shuowen.net/">说文网</a><script language="javascript" type="text/javascript" src="https://js.users.51.la/17753172.js"></script><br />
  Copyright &copy; 国学大师 古典图书集成 All Rights Reserved.<br>
  
  <span style="font-size:14px">免责声明:本站非营利性站点,以方便网友为主,仅供学习研究。<br>内容由热心网友提供和网上收集,不保留版权。若侵犯了您的权益,来信即刪。scp168@qq.com</span>
  <br />
ICP证:<a href="http://www.beian.miit.gov.cn/" target="_blank">鲁ICP备19060063号</a></div>
<!-- 页脚END --> 
<script src="http://www.guoxuedashi.com/img/plus.php?id=22"></script>
<script src="http://www.guoxuedashi.com/img/tongji.js"></script>

</body>
</html>
