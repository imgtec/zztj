資治通鑑卷一百    宋 司馬光 撰

胡三省 音註

晉紀二十二|{
	起旃蒙單閼盡屠維恊洽凡五年}


孝宗穆皇帝中之下

永和十一年春正月故仇池公楊毅弟宋奴使其姑子梁式王刺殺楊初初子國誅式王及宋奴自立為仇池公桓溫表國為鎮北將軍秦州刺史 二月秦大蝗百草無遺牛馬相噉毛|{
	無草可食故相噉毛噉徒濫翻又徒覽翻}
夏四月燕主雋自和龍還薊|{
	燕主如龍城見上卷上年薊音計}
先是幽冀之人以雋為東遷|{
	和龍直薊之東先悉薦翻}
互相驚擾所在屯結羣臣請討之雋曰羣小以朕東巡故相惑為亂耳今朕既至尋當自定不足討也 蘭陵太守孫黑濟北太守高柱|{
	守式又翻濟子禮翻}
建興太守高瓫|{
	瓫蒲奔翻}
及秦河内太守王會黎陽太守韓高皆以郡降燕|{
	史言燕彊諸反側子皆附之降戶江翻}
秦淮南王生幼無一目性粗暴其祖父洪嘗戲之曰吾聞瞎兒一涙信乎|{
	瞎許轄翻一目盲也}
生怒引佩刀自刺出血曰此亦一涙也洪大驚鞭之生曰性耐刀槊不堪鞭棰|{
	槊色角翻棰止橤翻}
洪謂其父健曰此兒狂悖|{
	悖蒲内翻又蒲没翻}
宜早除之不然必破人家健將殺之健弟雄止之曰兒長自應改何可遽爾及長力舉千鈞手格猛獸|{
	格擊也長知兩翻}
走及奔馬擊刺騎射冠絶一時|{
	騎奇寄翻冠古玩翻下同}
獻哀太子卒|{
	秦太子萇諡曰獻哀}
彊后欲立少子晉王柳|{
	彊其兩翻少詩照翻}
秦主健以䜟文有三羊五眼乃立生為太子|{
	為苻生以凶暴不克紹張本}
以司空平昌王菁為太尉尚書令王墮為司空司隸校尉梁楞為尚書令|{
	楞盧登翻}
姚襄所部多勸襄北還襄從之五月襄攻冠軍將軍高季於外黄|{
	外黄縣自漢以來屬陳留郡賢曰外黄故城在今汴州雍丘縣東冠古玩翻}
會季卒襄進據許昌 六月丙子秦主健寢疾庚辰平昌公菁勒兵入東宫將殺太子生而自立時生侍疾西宫|{
	秦主所居為西宫}
菁以為健已卒|{
	卒子恤翻}
攻東掖門健聞變登端門陳兵自衛衆見健惶懼皆捨仗逃散健執菁數而殺之|{
	數所角翻}
餘無所問壬午以大司馬武都王安都督中外諸軍事|{
	苻雄死健以菁都督中外諸軍菁以逆誅以安代之}
甲申健引太師魚遵丞相雷弱兒太傅毛貴司空王墮尚書令梁楞左㒒射梁安右僕射段純吏部尚書辛牢等受遺詔輔政健謂太子生曰六夷酋帥及犬臣執權者若不從汝命宜漸除之|{
	為苻生虐殺大臣張本酋慈由翻帥所類翻}


臣光曰顧命大臣所以輔導嗣子為之羽翼也為之羽翼而敎使剪之能無斃乎知其不忠則勿任而已矣任以大柄又從而猜之鮮有不召亂者也

乙酉健卒|{
	年三十九}
諡曰景明皇帝廟號高祖丙戌太子生即位|{
	苻生字長生健第三子也}
大赦改元夀光羣臣奏曰未踰年而改元非禮也|{
	古禮君薨世子即位既踰年而後稱元年}
生怒窮推議主得右僕射段純殺之 秋七月以吏部尚書周閔為左僕射或告會稽王昱曰|{
	會工外翻}
武陵王第中大修器仗將謀非常|{
	武陵王晞也}
昱以告太常王彪之彪之曰武陵王之志盡於馳騁畋獵而已耳|{
	騁丑郢翻}
深願靜之以安異同之論勿復以為言昱善之|{
	為武陵終以此得禍彪之所不能救張本復扶又翻}
秦主生尊母彊氏曰皇太后立妃梁氏為皇后梁氏安之女也以其嬖臣太子門大夫南安趙韶為右僕射|{
	續漢志太子門大夫二人職比郎將嬖卑義翻又博計翻}
太子舍人趙誨為中護軍著作郎董榮為尚書 凉王祚淫虐無道上下怨憤祚惡河州刺史張瓘之彊|{
	張駿置河州治枹罕惡烏路翻}
遣張掖太守索孚代瓘守枹罕|{
	索昔各翻枹音膚}
使瓘討叛胡又遣其將易揣張玲帥步騎萬三千以襲瓘|{
	將即亮翻易讀如字姓也揣初委翻玲盧經翻帥讀曰率騎奇寄翻}
張掖人王鸞知術數言於祚曰此軍出必不還凉國將危并陳祚三不道祚大怒以鸞為訞言|{
	訞於驕翻}
斬以狥鸞臨刑曰我死軍敗於外王死於内必矣祚族㓕之瓘聞之斬孚起兵擊祚傳檄州郡廢祚以侯還第復立凉寧侯曜靈|{
	曜靈廢見上卷上年}
易揣張玲軍始濟河瓘擊破之揣等單騎奔還瓘軍躡之姑臧振恐驍騎將軍敦煌宋混兄修與祚有隙懼禍|{
	驍堅堯翻敦徒門翻}
八月混與弟澄西走合衆萬餘人以應瓘還向姑臧祚遣楊秋胡將曜靈於東苑拉其腰而殺之|{
	將如字拉盧合翻}
埋於沙阬諡曰哀公 秦主生封衛大將軍黄眉為廣平王前將軍飛為新興王皆素所善也徵大司馬武都王安領太尉|{
	健臨没以安督中外諸軍然尚在蒲阪今生乃召之}
以晉王柳為征東大將軍并州牧鎮蒲阪|{
	阪音反}
魏王庾為鎮東大將軍豫州牧鎮陜城|{
	庾踈鳩翻陜失冉翻}
中書監胡文中書令王魚言於生曰比有星孛于大角熒惑入東井大角帝坐東井秦分|{
	天文志大角在攝提間大角者天王坐也東井八星東井輿鬼秦雍州分比毗至翻孛蒲内翻坐狙臥翻分扶問翻}
於占不出三年國有大喪大臣戮死願陛下修德以禳之生曰皇后與朕對臨天下可以應大喪矣毛太傳梁車騎梁僕射受遺輔政可以應大臣矣九月生殺梁后及毛貴梁楞梁安貴后之舅也右僕射趙韶中護軍趙誨皆洛州刺史俱之從弟也|{
	趙俱鎮宜陽事見上卷上年從才用翻}
有寵於生乃以俱為尚書令俱固辭以疾謂韶誨曰汝等不復顧祖宗|{
	復扶又翻}
欲為㓕門之事毛梁何罪而誅之吾何功而代之汝等可自為吾其死矣遂以憂卒|{
	卒子恤翻}
凉宋混軍于武始大澤|{
	張駿分狄道縣立武始郡宋混西走起兵必不東向狄道水經都野澤在武威縣東北注云在姑臧城北三百里都野即禹貢之豬野其水上承姑臧武始澤澤在姑臧西}
為曜靈發哀|{
	為于偽翻}
閏月混軍至姑臧凉王祚收張瓘弟琚及子嵩將殺之琚嵩聞之募市人數百揚言張祚無道我兄大軍已至城東敢舉手者誅三族遂開西門納混兵領軍將軍趙長等懼罪|{
	趙長請立祚者也故懼罪}
入閣呼張重華母馬氏出殿立凉武侯玄靚為主|{
	靚疾郢翻又疾正翻}
易揣等引兵入殿收長等殺之祚案劔殿上大呼叱左右力戰|{
	呼火故翻}
祚素失衆心莫肯為之鬭者|{
	為于偽翻}
遂為兵人所殺混等梟其首|{
	梟堅堯翻}
宣示中外暴尸道左城内咸稱萬歲以庶人禮葬之并殺其二子混琚上玄靚為大將軍凉州牧西平公|{
	上時掌翻}
赦境内復稱建興四十三年|{
	張祚改建興年號見上卷上年}
時玄靚始七歲張瓘至姑臧推玄靚為凉王自為使持節都督中外諸軍事尚書令凉州牧張掖郡公|{
	使疏吏翻}
以宋混為尚書僕射隴西人李儼據郡不受瓘命用江東年號|{
	用永和年號也}
衆多歸之|{
	為李儼歸秦張本}
瓘遣其將牛霸討之|{
	將即亮翻}
未至西平人衛綝亦據郡叛|{
	綝丑林翻}
霸兵潰奔還瓘遣弟琚擊綝敗之|{
	敗補邁翻}
酒泉太守馬基起兵以應綝瓘遣司馬張姚王國擊斬之 冬十月以豫州刺史謝尚督并冀幽三州|{
	時江左僑立青冀并幽四州於江北}
鎮夀春|{
	南渡初祖逖以豫州刺史治譙城永昌元年祖約退屯夀春成帝咸和四年庾亮以豫州刺史治蕪湖咸康四年毛寶以豫州刺史治邾城六年庾翼以豫州刺史治蕪湖永和元年趙胤以豫州刺史治牛渚二年尚以豫州刺史治蕪湖今進夀春皆建康西藩也進取則屯夀春守江則多在歷陽蕪湖二處}
鎮北將軍段龕與燕主雋書抗中表之儀|{
	雋段氏出也故龕與之抗中表之儀龕苦舍翻下同}
非其稱帝雋怒十一月以太原王恪為大都督撫軍將軍陽騖副之以擊龕|{
	騖音務}
秦以辛牢守尚書令趙韶為左僕射尚書董榮為右

僕射中護軍趙誨為司隸校尉 十二月高句麗王釗遣使詣燕納質修貢以請其母|{
	句如字又音駒麗力知翻燕囚釗母見九十七卷成帝咸康八年質音致}
燕主雋許之遣殿中將軍刁龕送釗母周氏歸其國以釗為征東大將軍營州刺史封樂浪公|{
	樂浪音洛琅}
王如故|{
	使為高句麗王如故}
上黨人馮鴦逐燕太守段剛據安民城|{
	魏收地形志燕上黨太守治安民城安民城在襄垣縣蓋永嘉中劉琨遣張倚所築以安上黨之民因以為名}
自稱太守遣使來降|{
	使疏吏翻降戶江翻}
秦丞相雷弱兒性剛直以趙韶董榮亂政每公言於朝|{
	朝直遥翻}
見之常切齒韶榮譛之於秦主生生殺弱兒及其九子二十七孫於是諸羌皆有離心|{
	雷弱兒南安羌酋也以非罪而死故諸羌皆有離心}
生雖諒隂遊飲自若彎弓露刃以見朝臣錘鉗鋸鑿|{
	錘傳追翻鉗其廉翻鋸居御翻}
可以害人之具備置左右即位未幾|{
	幾居豈翻}
后妃公卿已下至於僕隸凡殺五百餘人截脛拉脅鋸項刳胎者比比有之|{
	脛形定翻膝下骨直而長者拉盧合翻比簿必翻}
燕主雋以段龕方彊謂太原王恪曰若龕遣軍拒河不得渡者可直取呂護而還|{
	呂護時據夏王}
恪分遣輕軍先至河上具舟楫以觀龕志趣龕弟羆驍勇有智謀|{
	驍堅堯翻}
言於龕曰慕容恪善用兵加之衆盛若聽其濟河進至城下恐雖乞降不可得也|{
	降戶江翻下同}
請兄固守羆帥精銳拒之於河幸而戰捷兄帥大衆繼之|{
	帥讀曰率下同}
必有大功若其不捷不若早降猶不失為千戶侯也龕不從羆固請不已龕怒殺之

十二年春正月燕太原王恪引兵濟河未至廣固百餘里段龕帥衆三萬逆戰丙申恪大破龕於淄水|{
	據載記恪破龕於濟水之南今言未至廣固百餘里蓋在淄水而會戰也水經濁水逕廣固城西東流至廣饒入巨淀又比合于淄水}
執其弟欽斬右長史袁範等齊王友辟閭蔚被創|{
	段龕自稱齊王故置王友之官蔚紆勿翻創初良翻}
恪聞其賢遣人求之蔚已死士卒降者數千人龕脫走還城固守恪進軍圍之 秦司空王墮性剛峻右僕射董榮侍中強國皆以佞幸進墮疾之如讐每朝見榮未嘗與之言|{
	每朝句絶朝直遙翻見如字}
或謂墮曰董君貴幸無比公宜小降意接之墮曰董龍是何雞狗|{
	龍董榮小字}
而令國士與之言乎會有天變榮與強國言於秦主生曰|{
	強其兩翻姓也}
今天譴甚重宜以貴臣應之生曰貴臣惟有大司馬及司空耳榮曰大司馬國之懿親不可殺也|{
	大司馬謂武都王安生叔父也}
乃殺王墮將刑榮謂之曰今日復敢比董龍於雞狗乎|{
	復扶又翻}
墮瞋目叱之|{
	瞋七人翻}
洛州刺史杜郁墮之甥也左僕射趙韶惡之|{
	惡烏路翻}
譛於生以為貳於晉而殺之壬戍生宴羣臣於太極殿以尚書令辛牢為酒監酒酣生怒曰何不彊人酒而猶有坐者|{
	監古暫翻強其兩翻}
引弓射牢殺之|{
	射而亦翻}
羣臣懼莫敢不醉偃仆失冠生乃悦匈奴大人劉務桓卒弟閼頭立將貳於代二月代王什翼犍引兵西巡臨河閼頭懼請降|{
	犍居言翻閼於葛翻降戶江翻下同}
燕太原王恪招撫段龕諸城|{
	恪圉廣固未下故先招撫其統内諸城}
己丑龕所署徐州刺史陽都公王騰舉衆降恪命騰以故職還屯陽都|{
	段龕置徐州於琅邪陽都縣杜佑曰漢陽都縣故城在沂州沂水縣南}
秦征東大將軍晉王柳遣參軍閻負梁殊使於凉以書說凉王玄靚|{
	使疏吏翻下同說輸芮翻}
負殊至姑臧張瓘見之曰我晉臣也臣無境外之交二君何以來辱負殊曰晉王與君隣藩雖山河阻絶風通道會|{
	秦使苻柳鎮蒲阪非與凉州鄰也故以風通道會為言}
故來修好|{
	好呼到翻下同}
君何怪焉瓘曰吾盡忠事晉於今六世矣|{
	軌寔茂駿重華曜靈祚為七世今言六世斥祚不以為世數}
若與苻征東通使是上違先君之志下隳士民之節其可乎負殊曰晉室衰微墜失天命固已久矣是以凉之二王北面二趙唯知機也|{
	張茂稱藩於前趙張駿稱藩於後趙}
今大秦威德方盛凉王若欲自帝河右則非秦之敵欲以小事大則曷若捨晋事秦長保福禄乎瓘曰中州好食言|{
	好呼到翻}
嚮者石氏使車適返而戎騎已至|{
	使疏吏翻永和二年張重華嗣位遣使奉章於石虎虎繼遣王擢來寇騎奇寄翻}
吾不敢信也負殊曰自古帝王居中州者政化各殊趙為奸詐秦敦信義豈得一槩待之乎|{
	槩所以平斗斛一槩待之言無所高下也}
張先楊初皆阻兵不服先帝討而擒之|{
	擒張先見九十八卷六年未嘗擒揚初也負殊姑為是言耳}
赦其罪戻寵以爵秩固非石氏之比也瓘曰必如君言秦之威德無敵何不先取江南則天下盡為秦有征東何辱命焉負殊曰江南文身之俗|{
	古者荆蠻之俗斷髪文身以避蛟龍之害負殊以此斥言之耳是時衣冠文物皆在江南且正朔所在也負殊吠堯刺由知各為其主而已}
道汚先叛化隆後服|{
	鄭玄曰汚猶殺也易曰高宗伐鬼方三年克之世之說者以為荆楚輕悍道汚先叛化隆後服故負殊亦以此斥言江南}
主上以為江南必須兵服河右可以義懷故遣行人先申大好|{
	好呼到翻}
若君不達天命則江南得延數年之命而河右恐非君之土也瓘曰我跨據三州|{
	三州謂凉河沙張茂及張駿所分置者也}
帶甲十萬西苞葱嶺東距大河伐人有餘况於自守何畏於秦負殊曰貴州山河之固孰若殽函民物之饒孰若秦雍|{
	雍於用翻}
杜洪張琚因趙氏成資兵彊財富有囊括關中席卷四海之志先帝戎旗西指氷消雲散旬月之間不覺易主|{
	事見九十八卷六年}
主上若以貴州不服赫然奮怒控弦百萬鼓行而西未知貴州將何以待之瓘笑曰兹事當决之於王非身所了|{
	了决也}
負殊曰凉王雖英睿夙成然年在幼冲君居伊霍之任國家安危繫君一舉耳瓘懼乃以玄靚之命遣使稱藩於秦秦因玄靚所稱官爵而授之 將軍劉度攻秦青州刺史王朗於盧氏|{
	盧氏縣漢屬弘農郡晋屬上洛郡唐屬虢州}
燕將軍慕輿長卿入軹關攻秦幽州刺史彊哲于裴氏堡|{
	永嘉之亂裴氏舉宗據險築堡以自守後人因而置屯戌故堡猶有裴氏之名蓋在河東界長知兩翻}
秦主生遣前將軍新興王飛拒度建節將軍鄧羌拒長卿飛未至而度退羌與長卿戰大破之獲長卿及甲首二千餘級 桓温請移都洛陽修復園陵章十餘上|{
	上時掌翻}
不許拜温征討大都督督司冀二州諸軍事以討姚襄 三月秦主生發三輔民治渭橋|{
	治直之翻}
金紫光禄大夫程肱諫以為妨農生殺之夏四月長安大風發屋拔木|{
	風捲屋瓦掀簷桷為發屋}
秦宫中驚擾或稱賊至宫門晝閉五日乃止秦主生推告賊者刳出其心左光禄大夫強平諫曰天降災異陛下當愛民事神緩刑崇德以應之乃可弭也|{
	弭止也}
生怒鑿其頂而殺之衛將軍廣平王黄眉前將軍新興王飛建節將軍鄧羌以平太后之弟叩頭固諫生弗聽出黄眉為左馮翊飛為右扶風羌行咸陽太守|{
	前漢扶風渭城縣秦之咸陽也後漢晉省魏收地形志咸陽郡治石安縣即漢渭城也石勒更名是郡蓋永嘉之後羣胡所置也}
猶惜其驍勇故皆弗殺|{
	驍堅堯翻}
五月太后彊氏以憂恨卒諡曰明德 姚襄自許昌攻周成于洛陽|{
	周城襲據洛陽見上卷十年}
六月秦主生下詔曰朕受皇天之命君臨萬邦嗣統以來有何不善而謗讟之音扇滿天下|{
	杜預曰讟誹也讟徒木翻}
殺不過千而謂之殘虐行者比肩未足為希|{
	希少也}
方當峻刑極罰復如朕何|{
	復扶又翻}
自去春以來潼關之西至於長安虎狼為暴晝則繼道|{
	言虎狼相繼於路也繼蜀本作斷}
夜則發屋不食六畜|{
	畜許又翻}
專務食人凡殺七百餘人民廢耕桑相聚邑居而為害不息秋七月秦羣臣奏請禳災|{
	禳如羊翻除殃祭也}
生曰野獸飢則食人飽當自止何禳之有且天豈不愛民哉正以犯罪者多故助朕殺之耳|{
	史言苻生之虐甚於桀紂}
丙子燕獻懷太子曄卒 姚襄攻洛陽踰月不克長史王亮諫曰明公英名蓋世兵彊民附今頓兵堅城之下力屈威挫或為它寇所乘此危亡之道也襄不從桓温自江陵北伐遣督護高武據魯陽輔國將軍戴施屯河上自帥大兵繼進|{
	帥讀曰率下同}
與寮屬登平乘樓|{
	平乘樓大船之樓}
望中原歎曰遂使神州陸沉百年丘墟王夷甫諸人不得不任其責|{
	以王衍等尚清談而不恤王事以致夷狄亂華也}
記室陳郡袁宏曰|{
	晋諸公諸從公府皆有記室掌表疏牋記書檄}
運有興廢豈必諸人之過温作色曰昔劉景升有千斤大牛噉芻豆十倍於常牛負重致遠曾不若一羸㹀|{
	温意以牛况宏徒能糜俸禄而無經世之用劉表字景升噉徒濫翻又徒覽翻羸倫為翻㹀疾置翻牝牛也}
魏武入荆州|{
	漢獻帝建安十三年曹操入荆州}
殺以享軍八月己亥温至伊水|{
	伊水在洛陽城南}
姚襄撤圍拒之匿精銳於水北林中遣使謂温曰承親帥王師以來襄今奉身歸命願敕三軍小却當拜伏道左温曰我自開復中原展敬山陵無豫君事欲來者便前相見在近無煩使人|{
	使疏吏翻}
襄拒水而戰温結陳而前|{
	陳讀曰陣}
親被甲督戰|{
	被皮義翻}
襄衆大敗死者數千人襄帥麾下數千騎奔于洛陽北山|{
	洛陽北山北芒山也騎奇寄翻}
其夜民弃妻子隨襄者五千餘人襄勇而愛人雖戰屢敗民知襄所在輒扶老携幼奔馳而赴之温軍中傳言襄病創已死|{
	創初良翻}
許洛士民為温所得者無不北望而泣|{
	史言姚襄得人心}
襄西走温追之不及弘農楊亮自襄所來奔温問襄之為人亮曰襄神明器宇孫策之儔而雄武過之|{
	儔等也類也}
周成率衆出降|{
	降戶江翻下同}
温屯故太極殿前既而徙屯金墉城己丑謁諸陵有毁壞者修復之各置陵令|{
	漢起陵邑邑各置令後遂因之諸陵各置陵令屬太常}
表鎮西將軍謝尚都督司州諸軍事鎮洛陽以尚未至留潁川太守毛穆之督護陳午河南太守戴施以二千人戍洛陽衛山陵徙降民三千餘家於江漢之間執周成以歸姚襄奔平陽秦并州刺史尹赤復以衆降襄|{
	尹赤叛襄見上卷八年}
襄遂據襄陵|{
	襄陵縣漢屬河東郡晋屬平陽郡後魏改襄陵為禽昌縣隋唐復曰襄陵}
秦大將軍張平擊之|{
	永和七年張平降秦已而二於燕通鑑以秦所授官繫之}
襄為平所敗|{
	敗補邁翻}
乃與平約為兄弟各罷兵 段龕遣其屬段蕰來求救|{
	蕰紆粉翻}
詔徐州刺史荀羨將兵隨蕰救之羨至琅邪|{
	此古琅邪也}
憚燕兵之強不敢進王騰寇鄄城|{
	鄄城縣漢屬東郡晋屬濮陽此非古鄄城縣蓋僑縣也}
羨進攻陽都會霖雨城壞獲騰斬之|{
	段龕署王騰為徐州刺史屯陽都時降于燕為燕來寇}
冬十月癸巳朔日有食之 秦主生夜食棗多旦而有疾召太醫令程延使診之|{
	診止尹翻候脉也}
延曰陛下無它疾食棗多耳生怒曰汝非聖人安知吾食棗遂斬之 燕大司馬恪圍段龕於廣固諸將請急攻之恪曰用兵之勢有宜緩者有宜急者不可不察若彼我勢敵外有強援恐有腹背之患則攻之不可不急若我強彼弱無援於外力足制之者當覊縻守之以待其斃兵法十圍五攻正謂此也|{
	孫子曰用兵之法十則圍之五則攻之}
龕兵尚衆未有離心濟南之戰|{
	即淄水之戰曰濟南者以濟水南北大界言之}
非不銳也但龕用之無術以取敗耳今憑阻堅城上下戮力我盡銳攻之計數日可拔然殺吾士卒必多矣自有事中原兵不蹔息|{
	蹔與暫同}
吾每念之夜而忘寐奈何輕用其死乎要在取之不必求功之速也諸將皆曰非所及也軍中聞之人人感悦於是為高牆深塹以守之|{
	塹七艶翻}
齊人争運糧以饋燕軍龕嬰城自守樵采路絶城中人相食龕悉衆出戰恪破之於圍裏|{
	時外築長圍故戰於圍裏}
先分騎屯諸門|{
	屯廣固城諸門也騎奇寄翻}
龕身自衝盪|{
	盪徒朗翻又他浪翻}
僅而得入餘兵皆沒於是城中氣沮莫有固志|{
	沮在呂翻}
十一月丙子龕面縛出降并執朱禿送薊|{
	降戶江翻薊音計}
恪撫安新民悉定齊地徙鮮卑胡羯三千餘戶于薊燕主雋具朱禿五刑|{
	朱禿殺慕容鉤而奔龕見上卷十年}
以段龕為伏順將軍恪留慕容塵鎮廣固以尚書左丞鞠殷為東莱太守章武太守鮮于亮為齊郡太守乃還殷彭之子也彭時為燕大長秋以書戒殷曰王彌曹嶷必有子孫|{
	嶷魚力翻}
汝善招撫勿尋舊怨以長亂源|{
	長知兩翻}
殷推求得彌從子立嶷孫巖於山中請與相見深結意分|{
	從才用翻下同分扶問翻}
彭復遣使遺以車馬衣服|{
	復扶又翻遺于季翻}
郡民由是大和|{
	鞠彭自東莱歸燕見九十一卷元帝大興二年}
荀羨聞龕已敗退還下邳留將軍諸葛攸高平太守劉莊將三千人守琅邪參軍譙國戴?等將二千人守泰山|{
	楊正衡曰?音遁}
燕將慕容蘭屯汴城|{
	汴城即浚儀城余謂汴當作卞魯國卞縣城也劉昫曰兖州泗水縣卞縣古城也}
羨擊斬之 詔遣兼司空散騎常侍車灌等持節如洛陽修五陵|{
	宣帝陵在河隂首陽山景帝陵曰峻平文帝陵曰崇陽武帝陵曰峻陽惠帝陵曰太陽散悉亶翻騎奇寄翻車尺奢翻}
十二月庚戌帝及羣臣皆服緦臨於太極殿三日|{
	緦十五升布抽去其半臨力鴆翻}
司州都督謝尚以疾不行以丹陽尹王胡之代之胡之廙之子也|{
	王廙王敦之從弟見八十九卷愍帝建興三年廙羊至翻又逸職翻}
是歲仇池公楊國從父俊殺國自立以俊為仇池公國子安奔秦|{
	其後秦用偒安以取仇池豈即國之子邪}


升平元年春正月壬戌朔帝加元服太后詔歸政大赦改元太后徙居崇德宫 燕主雋徵幽州刺史乙逸為左光禄大夫逸夫婦共載鹿車子璋從數十騎服飾甚麗奉迎於道逸大怒閉車不與言到城深責之|{
	到城謂到薊城也永和八年燕王都薊於龍城置留臺以乙逸領留務蓋以幽州刺史鎮龍城也騎奇寄翻}
璋猶不悛|{
	悛丑緣翻下同}
逸常憂其敗而璋更被擢任歷中書令御史中丞|{
	被皮義翻}
逸乃歎曰吾少自修立|{
	少詩照翻}
克已守道僅能免罪璋不治節儉專為奢縱|{
	治直之翻}
而更居清顯此豈惟璋之忝幸實時世之陵夷也 二月癸丑燕主雋立其子中山王暐為太子大赦改元光夀 太白入東井秦有司奏太白罰星東井秦分|{
	分扶問翻}
必有暴兵起京師秦主生曰太白入井自為渴耳|{
	為于偽翻}
何所怪乎 姚襄將圖關中夏四月自北屈進屯杏城|{
	北屈縣漢屬河東郡晋屬平陽郡師古曰屈吾勿翻晋公子夷吾所居班志禹貢壺口山在北屈縣東南水經注北屈西距河十里孟門山在河上襄蓋自北屈渡河而屯杏城五代志汾州昌寧縣有壺口山宋白曰慈州吉郷縣漢北屈縣今縣北二十一里古城即漢理魏收地形志澄城縣有杏城師古曰澄城漢馮翊之徵縣也徵音懲據載記杏城在馬蘭山北杜佑曰姚萇置杏城鎮在坊州西七里}
遣輔國將軍姚蘭略地敷城|{
	敷城唐坊州鄜城縣是也後魏置敷城縣隋改曰鄜城}
曜武將軍姚益生|{
	曜武將軍蓋趙石氏所署置}
左將軍王欽盧各將兵招納諸羌胡蘭襄之從兄|{
	從才用翻}
益生襄之兄也 胡及秦民歸之者五萬餘戶秦將苻飛龍擊蘭擒之襄引兵進據黄落秦主生遣衛大將軍廣平王黄眉平北將軍苻道龍驤將軍東海王堅|{
	驤思將翻}
建節將軍鄧羌|{
	漢魏之間置建節中郎將後以為將軍號}
將步騎萬五千以禦之襄堅壁不戰羌謂黄眉曰襄為桓温張平所敗銳氣喪矣|{
	敗補邁翻喪息浪翻}
然其為人強狠|{
	狠戶墾翻}
若鼓譟揚旗直壓其壘彼必忿恚而出|{
	恚於避翻}
可一戰擒也五月羌帥騎三千壓其壘門而陳|{
	帥讀曰率騎奇寄翻陳讀曰陣}
襄怒悉衆出戰羌陽不勝而走襄追之至于三原|{
	三原在漢馮翊池陽縣界宋白曰苻堅於嶻嶭北置三原護軍後周置三原縣}
羌迴騎擊之黄眉等以大衆繼至襄兵大敗襄所乘駿馬曰黧眉騧|{
	黧音黎又音良脂翻黑而黄色曰黧騧古瓜翻黄馬黑喙曰騧}
馬倒秦兵擒而斬之弟萇帥其衆降|{
	萇仲良翻降戶江翻}
襄載其父弋仲之柩在軍中|{
	柩巨救翻在牀曰尸在棺曰柩}
秦主生以王禮葬弋仲於孤磐|{
	孤磐在天水冀縣界}
亦以公禮葬襄黄眉等還長安生不之賞數衆辱黄眉|{
	數所角翻}
黄眉怒謀弑生發覺伏誅事連王公親戚死者甚衆戊寅燕主雋遣撫軍將軍垂中軍將軍䖍護軍將軍平熙帥步騎八萬攻敕勒於塞北|{
	新唐書曰敕勒其先匈奴也元魏時號高車部其後訛為鐵勒唐之鐵勒十五種是也載記作丁零勑勒}
大破之俘斬十餘萬獲馬十三萬匹牛羊億萬頭匈奴單于賀賴頭帥部落三萬五千口降燕|{
	自東漢以來匈奴入居塞内者凡十九種賀賴其一也單音蟬}
燕人處之代郡平舒城|{
	漢代郡有平舒縣勃海有東平舒縣東平舒後漢屬河間國晋屬章武國代郡之平舒未嘗改屬書代郡以别章武之平舒代郡之平舒當在唐蔚之北界處昌呂翻}
秦主生夢大魚食蒲|{
	苻氏本蒲家也故以夢魚食蒲為異}
又長安謡曰東海大魚化為龍男皆為王女為公生乃誅太師錄尚書事廣甯公魚遵并其七子十孫金紫光禄大夫牛夷懼禍求為荆州|{
	秦荆州治豐陽川}
生不許以為中軍將軍引見調之曰|{
	調徒彫翻調戲也}
牛性遲重善持轅軛|{
	轅輈也轅前曰軛加之牛項軛音厄}
雖無驥足動負百石夷曰雖服大車未經峻壁願試重載乃知勲績|{
	載才再翻}
生笑曰何其快也公嫌所載輕乎朕將以魚公爵位處公|{
	處昌呂翻}
夷懼歸而自殺生飲酒無晝夜或連月不出奏事不省往往寢落|{
	省悉景翻落當作格音閤留止不下曰格}
或醉中决事左右因以為奸賞罰無凖或至申酉乃出視朝|{
	朝直遥翻}
乘醉多所殺戮自以眇目諱言殘缺偏隻少無不具之類誤犯而死者不可勝數|{
	勝音升數所具翻}
好生剝牛羊驢馬燖雞豚鵞鴨|{
	好呼到翻燖徐亷翻湯瀹去其毛曰燖}
縱之殿前數十為羣或剝人面皮使之歌舞臨觀以為樂|{
	樂音洛}
嘗問左右曰自吾臨天下汝外間何所聞或對曰聖明宰世賞罰明當|{
	當丁浪翻}
天下唯歌太平怒曰汝媚我也引而斬之它日又問或對曰陛下刑罰微過又怒曰汝謗我也亦斬之勲舊親戚誅之殆盡羣臣得保一日如度十年東海王堅素有時譽|{
	時譽者為時人所稱美也}
與故姚襄參軍薛讃權翼善讃翼密說堅曰|{
	說輸芮翻}
主上猜忍暴虐中外離心方今宜主秦祀者非殿下而誰願早為計勿使他姓得之堅以問尚書呂婆樓婆樓曰僕刀鐶上人耳|{
	魏晋之間率以刀鐶築殺人言將為生所殺也或曰刀以鋒刃為用刀環以上無所用之婆樓以自喻鐶戶闔翻}
不足以辦大事僕里舍有王猛其人謀略不世出|{
	不世出者言世間不常生此人}
殿下宜請而咨之堅因婆樓以招猛一見如舊友語及時事堅大悦自謂如劉玄德之遇諸葛孔明也|{
	見六十五卷漢獻帝建安十二年}
六月太史令康權言於秦主生曰|{
	姓譜曰康衛康叔之後亦西胡姓}
昨夜三月並出孛星入太微連東井|{
	孛蒲内翻}
自去月上旬沈隂不雨以至於今將有下人謀上之禍|{
	此亦據洪範五行傳言之也沈持林翻}
生怒以為妖言撲殺之|{
	妖於驕翻撲弼角翻}
特進領御史中丞梁平老等謂堅曰主上失德上下嗷嗷|{
	嗷嗷衆口忍聲}
人懷異志燕晉二方伺隙而動|{
	伺相吏翻}
恐禍發之日家國俱亡此殿下之事也宜早圖之堅心然之畏生趫勇未敢發|{
	趫丘妖翻捷也}
生夜對侍婢言曰阿灋兄弟亦不可信|{
	阿傳讀從安入聲}
明當除之|{
	明謂明旦猶言明日也}
婢以告堅及堅兄清河王灋灋與梁平老及特進光禄大夫強汪帥壯士數百潜入雲龍門|{
	魏明帝起洛陽宫宫城正南門曰雲龍門苻氏據長安亦以宫城正南門為雲龍門帥讀曰率下同}
堅與呂婆樓帥麾下三百人鼓譟繼進宿衛將士皆舍仗歸堅|{
	舍讀曰捨}
生猶醉寐堅兵至生驚問左右曰此輩何人左右曰賊也生曰何不拜之堅兵皆笑生又大言何不速拜不拜者斬之堅兵引生置别室廢為越王尋殺之諡曰厲王|{
	年二十三}
堅以位讓灋灋曰汝嫡嗣且賢宜立|{
	堅母苟氏雄之元妃故謂堅為嫡嗣}
堅曰兄年長宜立|{
	長知兩翻}
堅母苟氏泣謂羣臣曰社稷事重小兒自知不能它日有悔失在諸君羣臣皆頓首請立堅堅乃去皇帝之號|{
	去羌呂翻}
稱大秦天王即位於太極殿|{
	苻堅字永固雄之子也}
誅生幸臣中書監董榮左僕射趙韶等二十餘人大赦改元永興追尊父雄為文桓皇帝母苟氏為皇太后妃苟氏為皇后世子宏為白皇太子以清河王灋為都督中外諸軍事丞相錄尚書事東海公諸王皆降爵為公以從祖右光禄大夫永安公侯為太尉晋公柳為車騎大將軍尚書令|{
	從才用翻騎奇寄翻}
封弟融為陽平公雙為河南公子丕為長樂公|{
	樂音洛}
暉為平原公熙為廣平公叡為鉅鹿公以漢陽李威為左僕射|{
	李威於堅母有辟陽之寵故擢用之}
梁平老為右僕射強汪為領軍將軍呂婆樓為司隸校尉王猛為中書侍郎融好文學|{
	好呼到翻}
明辯過人耳聞則誦過目不忘力敵百夫善騎射擊刺少有令譽|{
	少詩照翻}
堅愛重之常與共議國事融經綜内外刑政修明薦才揚滯補益弘多|{
	弘大也}
丕亦有文武才幹治民斷獄皆亞於融|{
	史言堅有弟有子如此而無救於敗亡明天之所弃非人之所能支也治直之翻斷丁亂翻}
威苟太后之姑子也素與魏王雄友善生屢欲殺堅賴威營救得免威得幸於苟太后堅事之如父威知王猛之賢常勸堅以國事任之堅謂猛曰李公知君猶鮑叔牙之知管仲也|{
	管仲少與鮑叔牙遊鮑叔知其賢善遇之管仲曰吾始困時與鮑叔賈分財多自與鮑叔不以我為貪知我貧也吾嘗為鮑叔謀事而更窮困鮑叔不以我為愚知時有利不利也吾嘗三仕三見逐鮑叔不以我為不肖知我不遭時也吾嘗三戰三北鮑叔不以我為怯知我有老母也公子糾敗召忽死之吾幽囚受辱鮑叔不以我為無耻知我不羞小節而耻功名不顯於天下也生我者父母知我者鮑子也}
猛以兄事之 燕主雋殺段龕阬其徒三千餘人|{
	龕苦含翻}
秋七月秦大將軍冀州牧張平遣使請降|{
	降戶江翻}
拜并州刺史 八月丁未立皇后何氏后故散騎侍郎廬江何凖之女也|{
	散亶翻騎奇寄翻}
禮如咸康而不賀|{
	成帝咸康二年立杜后}
秦王堅以權翼為給事黄門侍郎|{
	權翼仕秦久當事任而卒歸姚氏料其受苻堅信用雖不為莊舄之越吟固隱之於心也}
薛讃為中書侍郎與王猛並掌機密九月追復太師魚遵等官以禮改葬子孫存者皆隨才擢叙 張平據新興雁門西河太原上黨上郡之地壁壘三百餘夷夏十餘萬戶|{
	壁壘蓋時遭亂離豪望自相保聚所築者石氏用張平為并州故得有其地有其民夏戶雅翻}
拜置征鎮欲與燕秦為敵國|{
	石氏之敗平兩附燕秦今恃其強欲與燕秦為敵國}
冬十月平寇略秦境|{
	平蓋間秦之有内難也安知由是而敗亡乎}
秦王堅以晋公柳都督并冀州諸軍事加并州牧鎮蒲阪以禦之 十一月癸酉燕主雋自薊徙都鄴|{
	薊音計}
秦太后苟氏遊宣明臺見東海公灋之第門車馬輻凑恐終不利於秦王堅乃與李威謀賜法死堅與法訣於東堂慟哭嘔血諡曰獻哀公封其子陽為東海公敷為清河公|{
	為後陽謀復讐張本}
十二月乙巳燕主雋入鄴宫大赦復作銅雀臺|{
	魏武建國於鄴作銅雀臺石氏增修之兵亂圮毁慕容都鄴復作使如舊}
以太常王彪之為左僕射 秦王堅行至尚書以文案不治|{
	治直之翻}
免左丞程卓官以王猛代之堅舉異才修廢職課農桑恤困窮禮百神立學校旌節義繼絶世秦民大悦|{
	史言苻堅能用王猛以治秦校戶教翻}
二年春正月司徒昱稽首歸政|{
	稽音啓}
帝不許 初馮鴦既以上黨來降|{
	見上永和十一年}
又附於張平又自歸於燕既而復叛燕|{
	復扶又翻}
二月燕司徒上庸王評討之不克 秦王堅自將討張平|{
	將即亮翻}
以鄧羌為前鋒督護帥騎五千軍于汾上|{
	汾水之上也帥讀曰率騎奇寄翻}
平使養子蚝禦之|{
	蚝七吏翻}
蚝多力趫捷|{
	趫丘妖翻}
能曳牛却走城無高下皆可超越與羌相持旬餘莫能相勝三月堅至銅壁|{
	河汾之間有銅川其民遇亂築壘壁以自守因曰銅壁}
平盡衆出戰蚝單馬大呼出入秦陳者四五|{
	呼火故翻陳讀曰陣}
堅募人生致之鷹揚將軍呂光刺蚝中之|{
	刺七亦翻中竹仲翻}
鄧羌擒蚝以獻平衆大潰平懼請降|{
	降戶江翻下同}
堅拜平右將軍以蚝為虎賁中郎將|{
	賁音奔將即亮翻下同}
蚝本姓弓|{
	姓譜弓姓魯叔弓之後}
上黨人也堅寵待甚厚常置左右秦人稱鄧羌張蚝皆萬人敵光婆樓之子也堅徙張平部民三千餘戶于長安 甲戌燕主雋遣領軍將軍慕輿根將兵助司徒評攻馮鴦根欲急攻之評曰鴦壁堅不如緩之根曰不然公至城下經月未嘗交鋒賊謂國家力止於此遂相固結冀幸萬一|{
	言鴦心僥倖於萬一可以保城也}
今根兵初至形勢方振賊衆恐懼皆有離心計慮未定從而攻之無不克者遂急攻之鴦與其黨果相猜忌鴦奔野王依呂護其衆盡降 夏四月秦王堅如雍祠五畤六月如河東祠后土|{
	用漢禮也雍於用翻畤音止}
秋八月豫州刺史謝奕卒奕安之兄也司徒昱以建武將軍桓雲代之雲温之弟也訪於僕射王彪之彪之曰雲非不才然温居上流已割天下之半其弟復處西藩|{
	東晋豫州鎮江西建康在江東故以豫州為西藩復扶又翻處昌呂翻下同}
兵權萃於一門非深根固蔕之宜人才非可豫量|{
	量音良}
但當令不與殿下作異者耳昱頷之曰君言是也壬申以吳興太守謝萬為西中郎將監司豫冀并四州諸軍事豫州刺史|{
	司豫冀并所統皆僑郡也監工銜翻}
王羲之與桓温牋曰謝萬才流經通|{
	言其才具可以經世於時人流輩中為通逹也}
使之處廊廟固是後來之秀今以之俯順荒餘近是違才易務矣|{
	言邊郡兵民皆兵荒之餘彫瘵未蘇而獷悍難調當俯就而柔順之今萬非其才而用之則為違才務事也以萬之才可以處廊廟而使之處邊鄙則為易事處昌呂翻近其靳翻}
又遺萬書曰|{
	遺于季翻}
以君邁往不屑之韻而俯同羣碎誠難為意也|{
	言其矜高不屑軍中之細務也}
然所謂通識正當隨事行藏耳願君每與士卒之下者同甘苦則盡善矣萬不能用徐兖二州刺史荀羨有疾以御史中丞郗曇為軍司|{
	曇徒含翻為萬曇皆不勝其任張本 考異曰帝紀謝萬為豫州下云郗曇為北中郎將督五州軍事徐兖二州刺史曇傳云荀羨有疾以曇為軍司頃之羨徵還除曇北中郎將都督刺史按帝紀十二月北中郎將荀羨及慕容雋戰于山茌王師敗績燕書十二月荀羨寇泰山殺太守賈堅載記荀羨殺賈堅下云敗羨復陷山莊故知八月曇未為徐兖二州恐始為軍司耳}
曇鑒之子也 九月庚辰秦王堅還長安以太尉侯守尚書令|{
	永安公苻侯}
於是秦大旱堅减膳徹樂命后妃以下悉去羅紈|{
	師古曰紈素今之絹也去羌呂翻}
開山澤之利公私共之息兵養民旱不為災王猛曰親幸用事宗親勲舊多疾之特進姑臧侯樊世本氐豪佐秦主健定關中謂猛曰吾輩耕之君食之邪猛曰非徒使君耕之又將使君炊之世大怒曰要當懸汝頭於長安城門不然吾不處世|{
	處昌呂翻}
猛以白堅堅曰必殺此老氐然後百寮可肅會世入言事與猛爭論於堅前世欲起擊猛堅怒斬之於是羣臣見猛皆屛息|{
	屏息不敢息也氣一出入為息屏必郢翻}
趙之亡也其將張平李歷高昌皆遣使降燕已而降晋又降秦各受爵位欲中立以自固|{
	李歷高昌初降晋張平降秦永和七年也八年歷昌降秦是年又與張平俱降燕苻生死後張平又降晋各受爵位將即亮翻使疏吏翻降戶江翻}
燕主雋使司徒評討張平於并州司空陽騖討高昌於東燕樂安王臧討李歷於濮陽騖攻昌别將於黎陽不拔歷奔滎陽其衆皆降并州壁壘百餘降於燕雋以右僕射悦綰為并州刺史以撫之平所署征西將軍諸葛驤等帥壁壘百三十八降於燕|{
	驤思將翻帥讀曰率下同}
雋皆復其官爵平帥衆三千奔平陽復請降於燕|{
	復扶又翻}
冬十月泰山太守諸葛攸攻燕東郡入武陽|{
	後漢東郡治東武陽武帝咸康二年封子允以東不可為國名而東郡有濮陽縣改曰濮陽國允改封淮南還曰東郡趙王倫篡位廢太孫臧為濮陽王東郡遂名濮陽此蓋燕復名東郡晋志武陽縣分屬陽平郡劉昫曰魏州朝城縣隋武陽縣地天寶七年更名}
燕主雋遣大司馬恪統陽騖及樂安王臧之兵以擊之攸敗走還泰山恪遂渡河略地河南分置守宰 燕主雋欲經營秦晋十二月令州郡校實見丁|{
	校實檢校其實數也見賢遍反}
戶留一丁餘悉發為兵欲使步卒滿一百五十萬期來春大集洛陽武邑劉貴上書極陳百姓彫弊發兵非法|{
	法未有戶留一丁而悉發為兵者}
必致土崩之變雋善之乃更令三五發兵寛其期日以來冬集鄴時燕調發繁數|{
	調徒弔翻數所角翻}
官司各遣使者道路㫄午郡縣苦之太尉領中書監封奕請自今非軍期嚴急不得遣使|{
	使疏吏翻}
自餘賦發皆責成州郡其羣司所遣彈督在外者一切攝還|{
	攝收也追也}
雋從之 燕泰山太守賈堅屯山茌|{
	山茌即前漢之茌縣屬泰山郡後漢改曰山茌茌仕疑翻}
荀羨引兵擊之堅所將纔七百餘人|{
	將即亮翻}
羨兵十倍於堅堅將出戰諸將皆曰衆少不如固守|{
	少詩沼翻}
堅曰固守亦不能免不如戰也遂出戰身先士卒|{
	先悉薦翻}
殺羨兵千餘人復還入城|{
	復扶又翻下同}
羨進攻之堅歎曰吾自結髪志立功名而每值窮阨豈非命乎|{
	堅欲折其鋒使羨懼而退耳羨進攻之堅計窮矣}
與其屈辱而生不若守節而死乃謂將士曰今危困計無所設卿等可去吾將止死將士皆泣曰府君不出衆亦俱死耳乃扶堅上馬堅曰我如欲逃必不相遣今當為卿曹决鬭|{
	為于偽翻}
若勢不能支卿等可趣去|{
	趣讀曰促}
勿復顧我也乃開門直出羨兵四集堅立馬橋上左右射之|{
	射而亦翻}
皆應弦而倒羨兵衆多從塹下斫橋堅人馬俱陷生擒之遂拔山茌|{
	塹七艶翻}
羨謂堅曰君父祖世為晋臣奈何背本不降|{
	背蒲妹翻降戶江翻下同}
堅曰晋自弃中華非吾叛也|{
	堅發此言江東將相其愧多矣}
民既無主強則託命既已事人安可改節吾束脩自立|{
	謂從師就學便有志於自立朱子曰脩脯也十脡為束古者從師必以束脩為禮}
涉趙歷燕未嘗易志|{
	堅不降燕見九十八卷永和七年}
君何怱怱相謂降乎羨復責之|{
	復扶又翻}
堅怒曰豎子兒女御乃公|{
	自稱為乃公慢羨而孩視之也曰御者言若駕御兒女然}
羨怒執置雨中數日堅憤惋而卒|{
	惋烏貫翻}
燕青州刺史慕容塵遣司馬悦明救泰山羨兵大敗燕復取山茌燕主雋以賈堅子活為任城太守|{
	任音壬}
荀羨疾篤徵還以郗曇為北中郎將都督徐兖青冀幽五州諸軍事|{
	五州惟徐州有實土郗丑之翻曇徒含翻}
徐兖二州刺史鎮下邳 燕吳王垂娶段末柸女生子令寶段氏才高姓烈自以貴姓|{
	段與慕容本抗衡之國故自以為貴姓}
不尊事可足渾后可足渾氏銜之燕主雋素不快於垂|{
	事見上卷永和十年}
中常侍湼皓|{
	湼乃結翻姓也}
因希旨告段氏及吳國典書令遼東高弼為巫蠱欲以連汙垂|{
	晋制王國置典書典祠學官令慕容氏因之典書令天朝吏部尚書之職齊王攸傳國相上長吏缺典書令請求差選是也西晋典書令在常侍侍郎上及渡江則侍郎次常侍而典書令居三卿下汙烏故翻}
雋收段氏及弼下大長秋廷尉考驗|{
	下遐稼翻}
段氏及弼志氣確然終無撓辭掠治日急|{
	撓奴教翻掠音亮冶直之翻}
垂愍之私使人謂段氏曰人生會當一死何堪楚毒如此不若引服|{
	引服自引而誣服也}
段氏歎曰吾豈愛死者耶若自誣以惡逆上辱祖宗下累於王|{
	累力瑞翻}
固不為也辯答益明故垂得免禍而段氏竟死於獄中出垂為平州刺史鎮遼東垂以段氏女弟為繼室可足渾氏黜之以其妹長安君妻垂垂不悦由是益惡之|{
	為慕容垂出奔張本妻七細翻烏路翻}
匈奴劉閼頭部落多叛懼而東走乘氷渡河半度而氷解後衆悉歸劉悉勿祈閼頭奔代|{
	代在北河之東閼於焉翻又於葛翻}
悉勿祈務桓之子也|{
	務桓卒見本卷永和十二年}


三年春二月燕主雋立子泓為濟北王|{
	濟子禮翻}
冲為中山王 燕人殺段勤勤弟思來奔|{
	段勤降燕見上卷永和八年}
燕主雋宴羣臣于蒲池|{
	蒲池在鄴}
語及周太子晋|{
	周靈王之太子曰晋慧而早卒國語諫壅穀洛者即晋也晋既卒弟貴立是為景王景王崩而子朝子丐爭立周遂以亂}
潜然流涕曰|{
	澘所奸翻}
才子難得自景先之亡|{
	燕太子曄字景先}
吾髪中白|{
	毛晃曰中直衆翻半也}
卿等謂景先何如司徒左長史李績對曰獻懷太子之在東宫|{
	曄諡曰獻懷}
臣為中庶子|{
	晋志曰太子中庶子職如侍中}
太子志業敢不知之太子大德有八至孝一也聰敏二也沈毅三也疾諛喜直四也|{
	沈持林翻喜許記翻}
好學五也多藝六也謙恭七也好施八也|{
	好呼到翻施式䜴翻}
雋曰卿譽之雖過|{
	譽音余}
然此兒在吾死無憂矣景茂何如|{
	燕太子暐字景荗}
時太子暐侍側績曰皇太子天資岐嶷|{
	嶷魚力翻毛萇曰岐知意也嶷識也}
雖八德已聞而二闕未補好遊畋而樂絲竹|{
	樂五敎翻}
此其所以損也雋顧謂暐曰伯陽之言藥石之惠也|{
	李績字伯陽}
汝宜誡之暐甚不平|{
	為李績以憂卒張本}
雋夢趙王虎齧其臂|{
	齧魚結翻}
乃發虎墓求尸不獲購以百金鄴女子李菟知而告之|{
	莬同都翻}
得尸於東明觀下|{
	水經注洹水東北流逕鄴城南又更分為二水北逕東明觀下觀古玩翻}
僵而不腐雋蹋而罵之曰|{
	僵居良翻蹋與踏同}
死胡何敢怖生天子數其殘暴之罪而鞭之投於漳水尸倚橋柱不流|{
	水經注漳水逕紫陌西趙建武十一年造紫陌浮橋慕容雋投石虎尸處也怖晋布翻數所具翻}
及秦滅燕王猛為之誅李莬收而葬之|{
	史終言之為于偽翻}
秦平羌護軍高離據略陽叛永安威公侯討之未克而卒夏四月驍騎將軍鄧羌|{
	驍堅堯翻}
秦州刺史啖鐵討平之|{
	啖徒覧翻氐姓也}
匈奴劉悉勿祈卒弟衛辰殺其子而代之五月秦王堅如河東六月大赦改元甘露 凉州牧張瓘猜忌苛虐專以愛憎為賞罰郎中殷郇諫之|{
	郇須倫翻}
瓘曰虎生三日自能食肉不須人教也由是人情不附輔國將軍宋混性忠鯁瓘憚之欲殺混及弟澄因廢凉王玄靚而代之|{
	靚疾正翻又疾郢翻}
徵兵數萬集姑臧混知之與澄帥壯士楊和等四十餘騎奄入南城|{
	王隱晉書曰凉州城有龍形故曰臥龍城南北七里東西二里本匈奴所築後張氏世居之又增築四城箱各千步并舊城為五又據張駿傳駿於姑臧城南築作五殿四面各依方色四時逓居之則南城張氏所居也帥讀曰率騎奇寄翻}
宣告諸營曰張瓘謀逆被太后令誅之|{
	被皮義翻}
俄而衆至二千瓘帥衆出戰混擊破之瓘麾下玄臚刺混不能穿甲|{
	玄姓也風俗通古諸侯有玄都國臚陵如翻刺七亦翻下同}
混擒之瓘衆悉降|{
	降戶江翻}
瓘與弟琚皆自殺混夷其宗族玄靚以混為使持節都督中外諸軍事驃騎大將軍酒泉郡侯代瓘輔政|{
	驃匹妙翻}
混乃請玄靚去凉王之號|{
	張祚始稱凉王見九十九卷永和十年張瓘推玄靚為凉王見上十一年去羌呂翻}
復稱凉州牧混謂玄臚曰卿刺我幸而不傷今我輔政卿其懼乎臚曰臚受瓘恩惟恨刺節下不深耳竊無所懼混義之任為心膂 高昌不能拒燕秋七月自白馬奔滎陽 秦王堅自河東還以驍騎將軍鄧羌為御史中丞|{
	驍堅堯翻}
八月以咸陽内史王猛為侍中中書令領京兆尹特進光禄大夫強德太后之弟也|{
	強太后秦主健之后也}
酗酒豪横|{
	酗吁句翻孔安國曰以酒為凶曰酗賈公彦曰據字酒㫄為凶是因酒為凶者也横戶孟翻}
掠人財貨子女為百姓患猛下車收德奏未及報已陳尸於市堅馳使赦之不及與鄧羌同志疾糾案無所顧忌數旬之間權豪貴戚殺戮刑免者二十餘人朝廷震栗奸猾屏氣|{
	屏必郢翻}
路不拾遺堅歎曰吾始今知天下之有灋也 泰山太守諸葛攸將水陸二萬擊燕|{
	將即亮翻下同}
入自石門屯于河渚燕上庸王評長樂太守傅顔帥步騎五萬與攸戰于東阿攸兵大敗|{
	樂音洛帥讀曰率騎奇寄翻}
冬十月詔謝萬軍下蔡郗曇軍高平以擊燕萬矜豪傲物但以嘯詠自高未嘗撫衆兄安深憂之謂萬曰汝為元帥|{
	帥所類翻下同}
宜數接對諸將以悦其心|{
	數所角翻}
豈有傲誕如此而能濟事也萬乃召集諸將一無所言直以如意指四坐云諸將皆勁卒諸將益恨之|{
	如意鐵如意也坐徂臥翻凡奮身行伍者以兵與卒為諱既為將矣而稱之為卒所以益恨也}
安慮萬不免乃自隊帥以下無不親造厚相親託|{
	造七到翻晋史言安性遲緩而為其弟慮乃周密如此宜其能為晋室内消桓温之變外破苻秦之師也}
既而萬帥衆入渦潁以援洛陽|{
	渦水至山桑入淮潁水至下蔡入淮謝尚之兵自下蔡而入渦潁之間}
郗曇以病退屯彭城萬以為燕兵大盛故曇退即引兵還衆遂驚潰|{
	進師易退師難是以善將者欲退師必廣為方畧而後引退不唯防敵人之追截亦慮已衆之驚潰也}
萬狼狽單歸軍士欲因其敗而圖之以安故而止既至詔廢萬為庶人降曇號建武將軍於是許昌潁川譙沛諸城相次皆沒於燕 秦王堅以王猛為吏部尚書尋遷太子詹事十一月為左僕射餘官如故 十二月封武陵王晞子㻱為梁王|{
	㻱與璡同音津}
大旱 辛酉燕主雋寢疾謂大司馬太原王恪曰吾病必不濟今二方未平|{
	二方謂晋秦也}
景茂冲幼國家多難吾欲效宋宣公以社稷屬汝|{
	難乃旦翻宋宣公舍其子與夷而立其弟穆公屬之欲翻}
何如恪曰太子雖幼勝殘致治之主也|{
	勝音升治直之翻}
臣實何人敢干正統雋怒曰兄弟之間豈虛飾邪恪曰陛下若以臣能荷天下之任者豈不能輔少主乎|{
	荷下可翻少詩照翻}
雋喜曰汝能為周公吾復何憂|{
	復扶又翻}
李績清方忠亮汝善遇之|{
	闕}
召吳王垂還鄴|{
	自遼東召還也}
秦王堅以王猛為輔國將軍司隸校尉居中宿衛僕射詹事侍中中書令領選如故|{
	選須絹翻}
猛上疏辭讓因薦散騎常侍陽平公融光禄散騎西河任羣|{
	光禄散騎以光禄大夫為散騎常侍也散悉亶翻騎奇寄翻任音壬}
處士京兆朱彤自代|{
	處昌呂翻}
堅不許而以融為侍中中書監左僕射任羣為光禄大夫領太子家令|{
	晋志曰太子家令主刑獄穀貨飲食職比司農少府}
朱彤為尚書侍郎領太子庶子|{
	漢制尚書有侍郎三十六人尚書郎初從三署詣臺試守尚書郎中歲滿稱尚書郎三年稱侍郎晋志曰庶子職比散騎常侍中書監令朱彤當作朱肜}
猛時年三十六歲中五遷|{
	猛自尚書左丞遷咸陽内史又遷侍中中書令領京兆尹又遷吏部尚書尋遷太子詹事為左僕射及今凡五遷}
權傾内外人有毁之者堅輒罪之於是羣臣莫敢復言|{
	復扶又翻}
以左僕射李威領護軍右僕射梁平老為使持節都督北垂諸軍事鎮北大將軍戍朔方之西|{
	使疏吏翻}
丞相司馬賈雍為雲中護軍戍雲中之南燕所徵郡國兵悉集鄴城|{
	去年所徵今乃悉集}


資治通鑑卷一百














































































































































