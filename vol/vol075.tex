<!DOCTYPE html PUBLIC "-//W3C//DTD XHTML 1.0 Transitional//EN" "http://www.w3.org/TR/xhtml1/DTD/xhtml1-transitional.dtd">
<html xmlns="http://www.w3.org/1999/xhtml">
<head>
<meta http-equiv="Content-Type" content="text/html; charset=utf-8" />
<meta http-equiv="X-UA-Compatible" content="IE=Edge,chrome=1">
<title>資治通鑒_76-資治通鑑卷七十五_76-資治通鑑卷七十五</title>
<meta name="Keywords" content="資治通鑒_76-資治通鑑卷七十五_76-資治通鑑卷七十五">
<meta name="Description" content="資治通鑒_76-資治通鑑卷七十五_76-資治通鑑卷七十五">
<meta http-equiv="Cache-Control" content="no-transform" />
<meta http-equiv="Cache-Control" content="no-siteapp" />
<link href="/img/style.css" rel="stylesheet" type="text/css" />
<script src="/img/m.js?2020"></script> 
</head>
<body>
 <div class="ClassNavi">
<a  href="/24shi/">二十四史</a> | <a href="/SiKuQuanShu/">四库全书</a> | <a href="http://www.guoxuedashi.com/gjtsjc/"><font  color="#FF0000">古今图书集成</font></a> | <a href="/renwu/">历史人物</a> | <a href="/ShuoWenJieZi/"><font  color="#FF0000">说文解字</a></font> | <a href="/chengyu/">成语词典</a> | <a  target="_blank"  href="http://www.guoxuedashi.com/jgwhj/"><font  color="#FF0000">甲骨文合集</font></a> | <a href="/yzjwjc/"><font  color="#FF0000">殷周金文集成</font></a> | <a href="/xiangxingzi/"><font color="#0000FF">象形字典</font></a> | <a href="/13jing/"><font  color="#FF0000">十三经索引</font></a> | <a href="/zixing/"><font  color="#FF0000">字体转换器</font></a> | <a href="/zidian/xz/"><font color="#0000FF">篆书识别</font></a> | <a href="/jinfanyi/">近义反义词</a> | <a href="/duilian/">对联大全</a> | <a href="/jiapu/"><font  color="#0000FF">家谱族谱查询</font></a> | <a href="http://www.guoxuemi.com/hafo/" target="_blank" ><font color="#FF0000">哈佛古籍</font></a> 
</div>

 <!-- 头部导航开始 -->
<div class="w1180 head clearfix">
  <div class="head_logo l"><a title="国学大师官网" href="http://www.guoxuedashi.com" target="_blank"></a></div>
  <div class="head_sr l">
  <div id="head1">
  
  <a href="http://www.guoxuedashi.com/zidian/bujian/" target="_blank" ><img src="http://www.guoxuedashi.com/img/top1.gif" width="88" height="60" border="0" title="部件查字,支持20万汉字"></a>


<a href="http://www.guoxuedashi.com/help/yingpan.php" target="_blank"><img src="http://www.guoxuedashi.com/img/top230.gif" width="600" height="62" border="0" ></a>


  </div>
  <div id="head3"><a href="javascript:" onClick="javascript:window.external.AddFavorite(window.location.href,document.title);">添加收藏</a>
  <br><a href="/help/setie.php">搜索引擎</a>
  <br><a href="/help/zanzhu.php">赞助本站</a></div>
  <div id="head2">
 <a href="http://www.guoxuemi.com/" target="_blank"><img src="http://www.guoxuedashi.com/img/guoxuemi.gif" width="95" height="62" border="0" style="margin-left:2px;" title="国学迷"></a>
  

  </div>
</div>
  <div class="clear"></div>
  <div class="head_nav">
  <p><a href="/">首页</a> | <a href="/ShuKu/">国学书库</a> | <a href="/guji/">影印古籍</a> | <a href="/shici/">诗词宝典</a> | <a   href="/SiKuQuanShu/gxjx.php">精选</a> <b>|</b> <a href="/zidian/">汉语字典</a> | <a href="/hydcd/">汉语词典</a> | <a href="http://www.guoxuedashi.com/zidian/bujian/"><font  color="#CC0066">部件查字</font></a> | <a href="http://www.sfds.cn/"><font  color="#CC0066">书法大师</font></a> | <a href="/jgwhj/">甲骨文</a> <b>|</b> <a href="/b/4/"><font  color="#CC0066">解密</font></a> | <a href="/renwu/">历史人物</a> | <a href="/diangu/">历史典故</a> | <a href="/xingshi/">姓氏</a> | <a href="/minzu/">民族</a> <b>|</b> <a href="/mz/"><font  color="#CC0066">世界名著</font></a> | <a href="/download/">软件下载</a>
</p>
<p><a href="/b/"><font  color="#CC0066">历史</font></a> | <a href="http://skqs.guoxuedashi.com/" target="_blank">四库全书</a> |  <a href="http://www.guoxuedashi.com/search/" target="_blank"><font  color="#CC0066">全文检索</font></a> | <a href="http://www.guoxuedashi.com/shumu/">古籍书目</a> | <a   href="/24shi/">正史</a> <b>|</b> <a href="/chengyu/">成语词典</a> | <a href="/kangxi/" title="康熙字典">康熙字典</a> | <a href="/ShuoWenJieZi/">说文解字</a> | <a href="/zixing/yanbian/">字形演变</a> | <a href="/yzjwjc/">金 文</a> <b>|</b>  <a href="/shijian/nian-hao/">年号</a> | <a href="/diming/">历史地名</a> | <a href="/shijian/">历史事件</a> | <a href="/guanzhi/">官职</a> | <a href="/lishi/">知识</a> <b>|</b> <a href="/zhongyi/">中医中药</a> | <a href="http://www.guoxuedashi.com/forum/">留言反馈</a>
</p>
  </div>
</div>
<!-- 头部导航END --> 
<!-- 内容区开始 --> 
<div class="w1180 clearfix">
  <div class="info l">
   
<div class="clearfix" style="background:#f5faff;">
<script src='http://www.guoxuedashi.com/img/headersou.js'></script>

</div>
  <div class="info_tree"><a href="http://www.guoxuedashi.com">首页</a> > <a href="/SiKuQuanShu/fanti/">四库全书</a>
 > <h1>资治通鉴</h1> <!--         下载:【右键另存为】即可 --></div>
  <div class="info_content zj clearfix">
  
<div class="info_txt clearfix" id="show">
<center style="font-size:24px;">76-資治通鑑卷七十五</center>
    資治通鑑卷七十五   宋 司馬光 撰<br />
<br />
  胡三省 音註<br />
<br />
  魏紀七【起柔兆攝提格盡玄黓涒灘凡七年】<br />
<br />
  邵陵厲公中<br />
<br />
  正始七年春二月吳車騎將軍朱然寇柤中【柤讀如祖揚正衡測瓜翻】殺略數千人而去 幽州刺史毋丘儉以高句驪王位宫數為侵叛【句如字又音駒驪力知翻數所角翻下同】督諸軍討之位宫敗走儉遂屠丸都【高句驪都於丸都之下多大山深谷毋丘儉傳謂縣車束馬以上丸都可知矣唐志自鴨渌江口舟行百餘里乃小舫泝流東北行凡五百三十里而至丸都城】斬獲首虜以千數句驪之臣得來數諫位宫位宫不從得來歎曰立見此地將生蓬蒿遂不食而死儉令諸軍不壞其墓【壞音怪】不伐其樹得其妻子皆放遣之位宫單將妻子逃竄儉引軍還未幾復擊之【幾居豈翻復扶又翻】位宫遂奔買溝【後漢書東夷傳買溝婁北沃沮之地去南沃沮八百餘里句麗名城為溝婁杜佑曰北沃沮一名買溝婁又曰高句麗居紇升骨城漢為縣屬玄莬郡賜以衣幘朝服鼓吹常從郡受之後稍驕恣不復詣郡但於東界築小城以受之遂名此城為幘溝漊溝漊者高麗名城也建安中其王伊夷模更作新國都於丸都山下在沸流水西魏正始中毋丘儉屠丸都銘不耐城而還又曰東沃沮在蓋馬大山之東北沃沮一名買溝漊去南沃沮八百餘里與挹婁接】儉遣玄莬太守王頎追之過沃沮千有餘里【沃沮之地在蓋馬大山之東漢武帝滅朝鮮開置玄莬郡治沃沮城後玄莬内徙沃沮更屬樂浪光武廢省就以其渠帥為縣侯其國小迫於句驪遂臣屬焉莬同都翻頎渠希翻沮千余翻】至肅慎氏南界【魏東夷挹婁之國即古肅慎氏也】刻石紀功而還所誅納八千餘口【言誅殺者及納降者總八千餘口還從宣翻又如字】論功受賞侯者百餘人 秋九月吳主以驃騎將軍步隲為丞相【驃匹妙翻】車騎將軍朱然為左大司馬衛將軍全琮為右大司馬分荆州為二部以鎮南將軍呂岱為上大將軍督右部自武昌以西至蒲圻【水經注陸水出長沙下嶲縣西逕蒲圻縣北又逕蒲山北入大江謂之陸口江水又逕蒲山北對蒲洲洲頭即蒲圻縣治武昌志曰蒲山今在嘉魚縣境蓋蒲圻縣初置于此宋白曰蒲圻縣漢沙羨縣地吳黃武二年於沙羨縣置蒲圻縣在荆江口因湖以稱故曰蒲圻】以威北將軍諸葛恪為大將軍督左部代陸遜鎮武昌 漢大赦大司農河南孟光【光河南洛陽人漢末逃入蜀】於衆中貴費禕曰夫赦者偏枯之物【木之一邊碩茂一邊焦槁者謂之偏枯赦者赦有罪也有罪者赦則姦惡之人抵法而獲免於罪良善之人受抑而不獲伸故謂之偏枯之物】非明世所宜有也衰敝窮極必不得已然後乃可權而行之耳今主上仁賢百僚稱職何有旦夕之急而數施非常之恩【稱尺證翻數所角翻】以惠姦宄之惡乎禕但顧謝踧踖而已【踧子六翻踖資息翻】初丞相亮時有言公惜赦者亮答曰治世以大德不以小惠【治直之翻】故匡衡吳漢不願為赦【匡衡疏見三十八卷元帝永光二年吳漢言見四十三卷光武建武二十年】先帝亦言吾周旋陳元方鄭康成間【陳紀字元方鄭玄字康成】每見啟告治亂之道悉矣曾不語赦也【治直吏翻】若劉景升季玉父子【劉琮字季玉】歲歲赦宥何益於治【治直吏翻】由是蜀人稱亮之賢知禕不及焉【蜀人賢孔明而劣費禕固不特惜赦一事而已】<br />
<br />
  陳夀評曰諸葛亮為政軍旅數興而赦不妄下【數所角翻下同下遐稼翻】不亦卓乎<br />
<br />
  吳人不便大錢乃罷之【青龍四年吳鑄大錢一當五百景初二年吳又鑄大錢一當千】漢主以凉州刺史姜維為衛將軍與大將軍費禕並<br />
<br />
  録尚書事【費父沸翻】汶山平康夷反維討平之【漢武帝元封二年分蜀郡北部置汶山郡宣帝地節三年合蜀郡蜀又分為汶山郡又立平康縣屬馬杜佑曰汶山郡今蜀郡西北通化郡地冉駹所居也宋白曰茂州通化郡古汶山郡劉昫曰維州薛城縣蜀將姜維討汶山叛羌即此地也今州城即姜維故壘汶讀曰岷】漢主數出遊觀增廣聲樂太子家令巴西譙周上疏諫曰昔王莽之敗豪桀並起以争神器才智之士思望所歸未必以其勢之廣陿惟其德之厚薄也於時更始公孫述等多已廣大【更工衡翻】然莫不快情恣欲怠於為善世祖初入河北馮異等勸之曰當行人所不能為者遂務理寃獄崇節儉北州歌歎聲布四遠於是鄧禹自南陽追之【事見三十九卷漢更始元年】吳漢寇恂素未之識舉兵助之其餘望風慕德邳肜耿純劉植之徒至於輿病齎棺襁負而至不可勝數【事並見更始二年肜余中翻勝音升】故能以弱為強而成帝業及在洛陽嘗欲小出銚期進諫即時還車【銚期傳曰光武嘗與期門近出期頓首車前曰臣聞古今之誡變生不意誠不願陛下微行數出帝為之回輿而還銚音姚】及潁川盜起寇恂請世祖身往臨賊聞言即行【事見四十二卷建武八年】故非急務欲小出不敢至於急務欲自安不為帝者之欲善如此故傳曰百姓不徒附誠以德先之也【傳直戀翻先悉薦翻】今漢遭厄運天下三分雄哲之士思望之時也【言思望賢主混一】臣願陛下復行人所不能為者以副人望【復扶又翻】且承事宗廟所以率民尊上也今四時之祀或有不臨而池苑之觀或有仍出臣之愚滯私不自安夫憂責在身者不暇盡樂【樂音洛下同】先帝之志堂構未成【書大誥曰若考作室既底法厥子乃弗肯堂矧肯構】誠非盡樂之時願省減樂官後宫凡所增造但奉脩先帝所施【施式支翻設也】下為子孫節儉之教漢主不聽<br />
<br />
  八年春正月吳全琮卒 二月日有食之時尚書何晏等朋附曹爽好變改法度太尉蔣濟上疏曰昔大舜佐治戒在比周【舜之佐堯也驩兜共工自相稱引則流放之讒說殄行則堲之戒比周也好呼到翻治直吏翻下同比毗至翻】周公輔政慎於其朋【書洛誥周公戒成王曰孺子其朋孺子其朋其往孔安國注曰少子慎其朋黨少子慎其朋黨戒其自今已往】夫為國法度惟命世大才乃能張其綱維以垂於後豈中下之吏所宜改易哉終無益於治適足傷民宜使文武之臣各守其職率以清平則和氣祥瑞可感而致也 吳主詔徙武昌宫材瓦繕脩建業宫有司奏言武昌宫已二十八歲【吳以漢獻帝建安二十四年都武昌至是已二十八年】恐不堪用宜下所在【下遐稼翻】通更伐致【伐致謂伐材木而致之通者凡吳境内悉然也】吳主曰大禹以卑宫為美今軍事未巳所在賦歛【歛力贍翻】若更通伐妨損農桑徙武昌材瓦自可用也乃徙居南宫三月改作太初宫【晉太康地記曰吳有太初宫方三百丈】令諸將及州郡皆義作【以下奉上義當助作宫室】大將軍爽用何晏鄧颺丁謐之謀遷太后於永寧宫【據後魏起永寧寺於銅駞街西意即前魏永寧殿故處也又據陳夀志太后稱永寧宫非徙也意者晉諸臣欲增曹爽之惡以遷字加之耳晉書五行志曰爽遷太后於永寧宫太后與帝相泣而别蓋亦承晉諸臣所記也】專擅朝政【朝直遥翻】多樹親黨屢改制度太傅懿不能禁與爽有隙五月懿始稱疾不與政事【為司馬懿誅曹爽等張本與讀曰預】 吳丞相步隲卒 帝好䙝近羣小【近其靳翻】遊宴後園秋七月尚書何晏上言自今御幸式乾殿【參考魏晉所記式乾殿當在皇后宫坤為母乾為父言皇后為天下母以乾為式從夫之義也】及遊豫後園宜皆從大臣詢謀政事講論經義為萬世法冬十二月散騎常侍諫議大夫孔乂上言【秦置諫大夫掌論議後漢增為諫議大夫】今天下已平陛下可絶後園習騎乘馬【騎奇寄翻】出必御輦乘車天下之福臣子之願也帝皆不聽 吳主大發衆集建業揚聲欲入寇揚州刺史諸葛誕使安豐太守王基策之【安豐縣漢屬六安國後漢屬廬江郡魏分置安豐郡屬豫州策之者計之也】基曰今陸遜等已死孫權年老内無賢嗣中無謀主權自出則懼内釁卒起【卒讀曰猝】癰疽發潰遣將則舊將已盡新將未信【疽千余翻將即亮翻】此不過欲補䘺支黨【䘺丈澗翻縫也】還自保護耳已而吳果不出 是歲雍凉羌胡叛降漢【雍於用翻降戶江翻】漢姜維將兵出隴右以應之與雍州刺史郭淮討蜀護軍夏侯霸戰于洮西【水經注洮水與蜀白水俱出西傾山山南即白水源山東即洮水源洮水東流逕吐谷渾中又東逕臨洮安故狄道又北至抱罕入于河諸縣皆在洮東若洮西則羌虜所居也洮士刀翻】胡王白虎文治無戴等率部落降維維徙之入蜀【蜀志曰居于繁縣據姜維傳則白虎文與治無戴二人也又魏志曹真討破叛胡治元多蓋諸胡有治姓也】淮進擊羌胡餘黨皆平之<br />
<br />
  九年春二月中書令孫資癸巳中書監劉放三月甲午司馬衛臻各遜位以侯就第位特進【雞棲樹之言固中而三馬食一槽矣】夏四月以司空高柔為司徒光禄大夫徐邈為司空<br />
<br />
  邈嘆曰三公論道之官無其人則缺【書曰三公論道經邦燮理隂陽官不必備惟其人】豈可以老病忝之哉【忝辱也】遂固辭不受 五月漢費禕出屯漢中自蔣琬及禕雖身居於外慶賞刑威皆遥先諮斷【斷丁亂翻諮斷者諮之使斷决也】然後乃行禕雅性謙素當國功名略與琬比 秋九月以車騎將軍王凌為司空 涪陵夷反【涪陵縣漢屬巴郡蜀分置涪陵郡唐之涪州宋白曰涪州涪陵郡漢為涪陵縣地蜀先主以地控江源於此立涪陵郡領漢平漢葭二縣四夷縣道記云故城在蜀江之南涪江之西其涪江南自黔中來由城之西泝蜀江十五里有雞鳴峽上有枳城即漢枳縣也李雄據蜀後枳縣荒廢桓温定蜀别立枳縣於今郡東北十里周武帝保定四年涪陵首領田思鶴歸化於故枳城立涪陵鎮隋開皇三年移漢平縣於鎮城仍改漢平為涪陵縣因鎮為名唐為涪州元和三年以涪州疆理與黔中接近勑隸黔中按華陽國志云涪陵巴之南鄙從枳縣入泝涪水秦司馬錯由之取楚黔中地漢興常為都尉理山險水灘人多獽蜑唯出丹漆蜜枳縣即涪州所理漢建安中涪陵謝本以涪陵廣大白州牧劉璋分置丹興漢葭二縣以為郡璋乃分涪陵立永寧兼丹興漢葭合四縣置屬國都尉理涪陵蜀先主改為郡改永寧曰萬寧又增立漢復縣後主又立漢平縣晉太康地志省丹興縣郡移理漢復又言萬寧在郡南水道九百里其萬寧蓋今費州是蜀後主延熙中涪陵大姓徐巨反鄧艾討平之漢涪陵蓋在今涪州東南三百三十里黔州是其故理在江之東又言漢復縣北至涪陵九十里蓋今黔州所管洪杜縣是其故理又言漢葭在郡東百里澧源出界蓋今州東九十里故黔州城是其丹興縣蓋在今黔州東二百里黔江縣是又按漢平縣在今涪州東百二十里羅浮山之北岷江之南白水入江處側近又按十三州志枳在郡東北涪陵在郡東按今黔州亦與巴郡東南相抵據謝本所論晉志所云今夷費思播及黔府等五州悉是涪陵故地又隋圖經黔中是武陵郡酉陽縣地按漢酉陽在今溪州犬鄉縣界與黔州約相去千餘里今三亭縣西北九百餘里别有酉陽城乃劉蜀所置非漢之酉陽隋圖經及貞觀地志並言蜀所置酉陽為漢酉陽蓋誤認漢涪陵之地也自永嘉後没於夷獠元魏後圖記不傳至後周田思鶴歸化初於其地立奉州續改黔州大業中又改黔安郡因周隋州郡名遂與秦漢黔中郡交牙難辨其秦黔中郡理在今辰州西二十里黔中故城是漢改黔中為武陵郡移治義陵即今辰州叙浦縣是後漢移理臨沅即今朗州所理是今辰錦叙奨溪澧朗施八州是秦漢黔中郡地與今黔州及夷費思播隔越峻嶺東有沅江水及諸溪並合而東注洞庭湖嶺西有巴江水一名涪陵江自牂柯北歷播費思黔等州北注岷江以山川言之巴郡之涪陵與黔中故地炳然自分矣】漢車騎將軍鄧芝討平之大將軍爽驕奢無度飲食衣服擬於乘輿【乘繩證翻】尚方<br />
<br />
  珍玩充牣其家又私取先帝才人以為伎樂【伎渠綺翻】作窟室綺疏四周【窟室掘地為室也賢曰綺疏謂鏤為綺文】數與其黨何晏等縱酒其中弟羲深以為憂數涕泣諫止之爽不聽爽兄弟數俱出遊【數所角翻】司農沛國桓範謂曰總萬機典禁兵不宜並出若有閉城門誰復内人者【復扶又翻下同】爽曰誰敢爾邪初清河平原爭界八年不能决冀州刺史孫禮請天府所藏烈祖封平原時圖以决之【烈祖明帝也封平原王畫壤分國有地圖在天府周禮有天府鄭玄注云掌祖廟之寶藏賢能之書及功書皆藏于天府】 爽信清河之訴云圖不可用禮上疏自辨辭頗剛切爽大怒劾禮怨望結刑五歲【結刑五歲者但結以徒作五歲之罪而不使之輸作也劾戶槩翻又戶得翻】久而復為并州刺史往見太傅懿有忿色而無言懿曰卿得并州少邪恚理分界失分乎【魏并州統太原上黨西河雁門新興冀州大於諸州并州遠接荒外故意其觖望懿多權數以此言擿發禮耳少詩沼翻恚於避翻分扶問翻】禮曰何明公言之乖也禮雖不德豈以官位往事為意邪本謂明公齊蹤伊呂匡輔魏室上報明帝之託下建萬世之勲今社稷將危天下兇兇【兇許拱翻】此禮之所以不悦也因涕泣橫流懿曰且止忍不可忍【至此禮入懿數中矣】冬河南尹李勝出為荆州刺史過辭太傅懿懿令兩婢侍持衣衣落指口言渇婢進粥懿不持杯而飲粥皆流出霑胸勝曰衆情謂明公舊風發動【魏武之辟懿也懿辭以風庳故勝以為舊風發動】何意尊體乃爾懿使聲氣纔屬【詐為羸惙之狀也屬之欲翻】說年老枕疾死在旦夕【枕之鴆翻】君當屈并州并州近胡【近其靳翻】好為之備恐不復相見以子師昭兄弟為託勝曰當還忝本州【李勝南陽人故謂荆州為本州】非并州懿乃錯亂其辭曰君方到并州勝復曰當忝荆州懿曰年老意荒不解君言【解戶買翻曉也】今還為本州盛德壯烈好建功勲勝退告爽曰司馬公尸居餘氣形神已離不足慮矣【言其形神已離惟尸在而餘殘喘耳】他日又向爽等垂泣曰【無聲而出涕曰垂泣】太傅病不可復濟令人愴然故爽等不復設備何晏聞平原管輅明於術數請與相見十二月丙戌輅往詣晏晏與之論易時鄧颺在坐【坐徂卧翻】謂輅曰君自謂善易而語初不及易中辭義何也輅曰夫善易者不言易也晏含笑贊之曰可謂要言不煩也因謂輅曰試為作一卦【為于偽翻】知位當至三公不【不讀曰否】又問連夢見青蠅數十來集鼻上驅之不去何也輅曰昔元凱輔舜【左傳高陽氏有才子八人蒼舒隤敳禱戭大臨厖降庭堅仲容叔達齊聖廣淵明允篤誠天下之民謂之八愷高辛氏有才子八人伯奮仲堪叔獻季仲伯虎仲熊叔豹季貍忠肅共懿宣慈惠和天下之民謂之八元】周公佐周皆以和惠謙恭享有多福此非卜筮所能明也今君侯位尊勢重而懷德者鮮【鮮息淺翻】畏威者衆殆非小心求福之道也又鼻者天中之山【相書以鼻為天中自唇以上為人中裴松之曰相書謂鼻之所在為天中鼻有山象故曰天中之山】高而不危所以長守貴今青蠅臭惡而集之位峻者顛輕豪者亡不可不深思也願君侯褒多益寡【褒蒲侯翻與掊同取也爾雅褒鳩樓聚也徐云樓歛也此言晏據權勢揆分為多當思自減損也】非禮勿履然後三公可至青蠅可驅也颺曰此老生之常譚輅曰夫老生者見不生常譚者見不譚【言必見其死也譚與談同】輅還邑舍【邑舍平原邑邱也】具以語其舅【語牛倨翻】舅責輅言太切至輅曰與死人語何所畏邪舅大怒以輅為狂 吳交趾九真夷賊攻沒城邑交部騷動吳主以衡陽督軍都尉陸胤為交州刺史安南校尉胤入境喻以恩信降者五萬餘家州境復清 太傅懿隂與其子中護軍師散騎常侍昭謀誅曹爽【懿雖稱疾先已置二子於要地矣】<br />
<br />
  嘉平元年【是年四月方改元】春正月甲午帝謁高平陵【高平陵明帝陵也水經注大石山在洛陽南山阿有魏明帝高平陵縣盛曰高平陵去洛城九十里】大將軍爽與弟中領軍羲武衛將軍訓散騎常侍彦皆從【從才用翻】太傅懿以皇太后令閉諸城門勒兵據武庫授兵出屯洛水浮橋【水經注洛城南出西頭第二門曰宣陽門漢之小苑門也對閶闔南直洛水浮橋】召司徒高柔假節行大將軍事據爽營太僕王觀行中領軍事據羲營因奏爽罪惡於帝曰臣昔從遼東還先帝詔陛下秦王及臣升御牀把臣臂深以後事為念【事見上卷明帝景初三年】臣言太祖高祖亦屬臣以後事【屬之欲翻按晉紀懿自為文帝所信重太祖未嘗以後事屬之也若文帝則以明帝屬懿】此自陛下所見無所憂苦萬一有不如意臣當以死奉明詔今大將軍爽背棄顧命【背蒲妹翻陸德明曰顧音古】敗亂國典内則僭擬外則專權破壞諸營【敗補邁翻壞音怪】盡據禁兵羣官要職皆置所親殿中宿衛易以私人根據盤互縱恣日甚又以黃門張當為都監【監古衘翻】伺察至尊離間二宫【伺相吏翻間古莧翻】傷害骨肉天下洶洶人懷危懼陛下便為寄坐【寄坐謂雖處天子之位猶寄寓也】豈得久安此非先帝詔陛下及臣升御牀之本意也臣雖朽邁【朽邁謂年老裒朽日月已過也】敢忘往言太尉臣濟等皆以爽為有無君之心兄弟不宜典兵宿衛奏永寧宫皇太后令敕臣如奏施行臣輒敕主者及黃門令罷爽羲訓吏兵以侯就第不得逗留以稽車駕敢有稽留便以軍法從事臣輒力疾將兵屯洛水浮橋伺察非常【輒專也懿雖挾太后以臨爽而其奏自言輒者至再以天子在爽所也】爽得懿奏事不通迫窘不知所為留車駕宿伊水南【水經注來儒之水出于半石之山西南流逕大石山又西至高都城東西入伊水伊水又東北過伊關中又東北至洛陽縣南北入于洛】伐木為鹿角發屯田兵數千人以為衛【魏武創業令州郡例置田官故洛陽亦有屯田兵】懿使侍中高陽許允及尚書陳泰說爽宜早自歸罪【說輸芮翻】又使爽所信殿中校尉尹大目謂爽唯免官而已【魏晉之制有殿中將軍中郎校尉司馬尹大目說爽猶未疑司馬氏也至其追語文欽乃覺耳】以洛水為誓泰羣之子也初爽以桓範鄉里老宿【範沛國人譙沛鄉里也老耆也宿舊也】於九卿中特禮之然不甚親也及懿起兵以太后令召範欲使行中領軍範欲應命其子止之曰車駕在外不如南出範乃出至平昌城門【水經注平昌門故平門也洛城南出西頭第三門】城門已閉門候司蕃故範舉吏也【司姓也左傳鄭有司臣】範舉手中版示之矯曰有詔召我卿促開門蕃欲求見詔書【以此觀之此時猶用版詔至晉時則有青紙詔矣】範呵之曰卿非我故吏邪何以敢爾乃開之範出城顧謂蕃曰太傅圖逆卿從我去蕃徒行不能及遂避側【避於道旁也】懿謂蔣濟曰智囊往矣濟曰範則智矣然駑馬戀棧豆爽必不能用也【駑音奴言爽顧戀室家而慮不及遠必不能用範計棧士限翻】範至勸爽兄弟以天子詣許昌發四方兵以自輔爽疑未决範謂羲曰此事昭然卿用讀書何為邪於今日卿等門戶求貧賤復可得乎【復扶又翻】且匹夫質一人尚欲望活【此謂漢末劫質也質音致】卿與天子相隨令於天下誰敢不應也俱不言範又謂羲曰卿别營近在闕南【中領軍營懿已遣王觀據之惟别營在耳】洛陽典農治在城外【洛陽典農中郎將典農都尉所治也】呼召如意今詣許昌不過中宿【中宿次宿也左傳曰命汝三宿汝中宿至陸德明曰中丁仲翻】許昌别庫足相被假【許昌别庫貯兵甲洛陽有武軍故曰别庫被假謂授兵也被皮義翻】所憂當在穀食而大司農印章在我身羲兄弟默然不從自甲夜至五鼓【甲夜初夜也夜有五更一更為甲夜二更為乙夜三更為丙夜四更為丁夜五更為戊夜】爽乃投刀於地曰我亦不失作富家翁範哭曰曹子丹佳人生汝兄弟㹠犢耳【曹真字子丹㹠與豚同小豕曰㹠小牛曰犢】何圖今日坐汝等族滅也爽乃通懿奏事白帝下詔免已官奉帝還宫爽兄弟歸家懿發洛陽吏卒圍守之【洛陽令所主吏卒也】四角作高樓令人在樓上察視爽兄弟舉動爽挾彈到後園中【彈徒案翻】樓上便唱言故大將軍東南行爽愁悶不知為計戊戌有司奏黃門張當私以所擇才人與爽疑有姦收當付廷尉考實辭云爽與尚書何晏鄧颺丁謐司隸校尉畢軌荆州刺史李勝等隂謀反逆須三月中發於是收爽羲訓晏颺謐軌勝并桓範皆下獄劾以大逆不道【下遐稼翻劾戶槩翻又戶得翻】與張當俱夷三族【考異曰魏氏春秋曰宣王使晏典治爽等獄晏窮治黨與冀以獲宥宣王曰凡有八族晏疏丁鄧等七姓宣】<br />
<br />
  【王曰未也晏窮急乃曰豈謂晏乎宣王曰是也乃收晏按宣王方治爽黨安肯使晏典其獄就令有之晏豈不自知與爽最親而冀獨免乎此殆孫盛承說者之妄耳】初爽之出也司馬魯芝留在府聞有變將營騎斫津門出赴爽【營騎大將軍營騎士也津門洛城南出西頭第一門也亦曰建城門騎奇寄翻】及爽解印綬【綬音受】將出主簿楊綜止之曰公挾主握權捨此以至東市乎【言必將見誅於市也】有司奏收芝綜治罪【治直之翻】太傅懿曰彼各為其主也【為于偽翻】宥之頃之以芝為御史中丞綜為尚書郎魯芝將出呼參軍辛敞欲與俱去敞毗之子也其姊憲英為太常羊耽妻敞與之謀曰天子在外太傅閉城門人云將不利國家於事可得爾乎【爾猶云如此也】憲英曰以吾度之【度徒洛翻】太傅此舉不過以誅曹爽耳敞曰然則事就乎憲英曰得無殆就【殆近也】爽之才非太傅之偶也【偶匹也】敞曰然則敞可以無出乎憲英曰安可以不出職守人之大義也凡人在難【難乃旦翻】猶或卹之為人執鞭而棄其事不祥莫大焉且為人任為人死親昵之職也【昵尼質翻左傳晏子曰君為社稷死則死之若為已死非其私昵誰敢任之昵私愛也此言親者則可為質任愛昵者則可為之死】從衆而已敞遂出事定之後敞歎曰吾不謀於姊幾不獲於義【幾居希翻】先是爽辟王沈及太山羊祜沈勸祜應命祜曰委質事人復何容易【先悉薦翻沈持林翻下同質如字復扶又翻易以䜴翻】沈遂行及爽敗沈以故吏免乃謂祜曰吾不忘卿前語祜曰此非始慮所及也【言始慮亦不料爽至此不欲受知幾之名也】爽從弟文叔妻夏侯令女【夏侯氏之女名令女夏戶雅翻】早寡而無子其父文寧欲嫁之令女刀截兩耳以自誓居常依爽爽誅其家上書絶昏強迎以歸復將嫁之【強其兩翻復扶又翻下同】令女竊入寢室引刀自斷其鼻【斷丁管翻】其家驚惋【惋烏貫翻驚歎也】謂之曰人生世間如輕塵棲弱草耳何至自苦乃爾且夫家夷滅已盡守此欲誰為哉【為于偽翻】令女曰吾聞仁者不以盛衰改節義者不以存亡易心曹氏前盛之時尚欲保終况今衰亡何忍棄之此禽獸之行吾豈為乎司馬懿聞而賢之聽使乞子字養為曹氏後何晏等方用事自以為一時才傑人莫能及晏嘗為名士品目曰唯深也故能通天下之志夏侯泰初是也唯幾也故能成天下之務司馬子元是也唯神也不疾而速不行而至吾聞其語未見其人蓋欲以神况諸己也【夏侯玄字泰初司馬師字子元晏引易大傳之辭以為品目幾居希翻】選部郎劉陶曄之子也少有口辯【少詩沼翻】鄧颺之徒稱之以為伊呂陶嘗謂傳玄曰仲尼不聖何以知之智者於羣愚如弄一丸於掌中而不能得天下何以為聖玄不復難【難乃旦翻】但語之曰【語牛倨翻】天下之變無常也今見卿窮及曹爽敗陶退居里舍乃謝其言之過管輅之舅謂輅曰爾前何以知何鄧之敗輅曰鄧之行步筋不束骨脈不制肉起立傾倚若無手足此為鬼躁何之視候則魂不守宅血不華色精爽烟浮容若槁木此為鬼幽二者皆非遐福之象也【管輅之與何鄧言也其陳義近於古人至答其舅論何鄧之所以敗則相者之說耳何前後之相戾也】何晏性自喜【喜許記翻】粉白不去手【以自塗澤也】行步顧影尤好老莊之書【好呼到翻】與夏侯玄荀粲及山陽王弼之徒競為清談祖尚虚無謂六經為聖人糟粕【莊子曰桓公讀書於堂上輪扁斵輪於堂下釋椎鑿而上問桓公曰敢問公所讀者何言邪公曰聖人之言也曰聖人在乎公曰已死矣曰然則君之所讀者古人之糟粕已矣古之人與其不可傳者死矣糟酒滓也司馬云爛食曰粕又云糟爛為粕許慎曰粕已漉粗糟也音匹各翻又普白翻】由是天下士大夫爭慕效之遂成風流不可復制焉【凊談之禍始此】粲彧之子也丙午大赦 丁未以太傅懿為丞相加九錫懿固辭不受 初右將軍夏侯霸為曹爽所厚以其父淵死於蜀【事見六十八卷漢獻帝建安二十四年】常切齒有報仇之志為討蜀護軍屯於隴西統屬征西【屬征西將軍府所統】征西將軍夏侯玄霸之從子爽之外弟也【曹氏夏侯氏之出也玄父尚又娶於曹氏故玄於爽為外弟】爽既誅司馬懿召玄詣京師【為後司馬師殺玄張本】以雍州刺史郭淮代之霸素與淮不叶以為禍必相及大懼遂奔漢漢主謂曰卿父自遇害於行間耳【行戶剛翻】非我先人之手刃也遇之甚厚姜維問於霸曰司馬懿既得彼政當復有征伐之志不【復扶又翻不讀曰否】霸曰彼方營立家門未遑外事有鍾士季者其人雖少【少詩照翻】若管朝政吳蜀之憂也【朝直遥翻】士季者鍾繇之子尚書郎會也【爲司馬昭用會以伐蜀張本】 三月吳左大司馬朱然卒然長不盈七尺氣候分明内行脩潔【行下孟翻】終日欽欽若在戰場【毛萇曰欽欽言使人樂進也】臨急膽定過絶於人雖世無事每朝夕嚴鼓【嚴皷疾擊鼓也今人謂之擂鼓】兵在營者咸行裝就隊以此玩敵使不知所備故出輒有功【雖不出兵而常為行備敵人之覘者玩以為常則不知所以備豫矣】然寢疾增篤吳主晝為減膳夜為不寐【為于偽翻下同】中使醫藥口食之物相望於道然每遣使表疾病消息吳主輒召見口自問訊入賜酒食出賜布帛及卒吳主為之哀慟 夏四月乙丑改元【曹爽誅後方改元嘉平】 曹爽之在伊南也昌陵景侯蔣濟與之書【諡法由義而濟曰景耆意大慮曰景】言太傅之旨不過免官而已爽誅濟進封都鄉侯上疏固辭不許濟病其言之失【以失言於爽為已病也】遂病丙子卒 秋漢衛將軍姜維寇雍州依麴山築二城【麴山蓋在羌中魏雍州西南界據郭淮傳麴山在翅上翅鳥翅也鳥翅要地也魏屯兵守之】使牙門將句安李歆等守之【句音鉤又古候翻姓也姓譜句芒氏之後史記有句彊今蜀中猶有句姓】聚羌胡質任侵偪諸郡【質音致】征西將軍郭淮與雍州刺史陳泰禦之泰曰麴城雖固去蜀險遠當須運糧羌夷患維勞役必未肯附今圍而取之可不血刃而拔其城雖其有救山道阻險非行兵之地也淮乃使泰率討蜀護軍徐質南安太守鄧艾進兵圍麴城斷其運道及城外流水安等挑戰不許【斷丁管翻挑徒了翻】將士困窘分糧聚雪以引日月【窘巨隕翻】維引兵救之出自牛頭山【牛頭山蓋在洮水之南以形名山魏收地形志後魏真君四年置仇池郡治階陵縣縣有牛頭山五代志牛頭山在成州上禄縣界】與泰相對泰曰兵法貴在不戰而屈人【孫子曰百戰百勝非善之善者也不戰而屈人善之善者也】今絶牛頭維無反道則我之禽也敕諸軍各堅壘勿與戰遣使白淮使淮趣牛頭截其還路【趣七喻翻】淮從之進軍洮水【洮土刀翻】維懼遁走安等絶遂降淮因西擊諸羌鄧艾曰賊去未遠或能復還【復扶又翻】宜分諸軍以備不虞於是留艾屯白水北【水經注白水出隴西臨洮縣西南西傾山東南流逕鄧至城南即艾所屯地以鄧艾至此故以名城】三日維遣其將廖化自白水南向艾結營【廖力救翻今力弔翻】艾謂諸將維今卒還【卒讀曰猝】吾軍人少【少詩照翻】法當來渡而不作橋此維使化持吾令不得還維必自東襲取洮城洮城在水北去艾屯六十里艾即夜潛軍徑到維果來渡而艾先至據城得以不敗漢軍遂還 兖州刺史令狐愚【姓譜周文王之子高封於畢其後有畢萬萬子犨封於魏為魏氏犨子顆封於令狐為令狐氏令力呈翻】司空王凌之甥也屯於平阿【水經注淮水過當塗縣北又北沙水注之淮之西有平阿縣故城晉志平阿縣屬淮南郡有塗山】甥舅並典重兵專淮南之任凌與愚隂謀以帝闇弱制於強臣聞楚王彪有智勇欲共立之迎都許昌九月愚遣其將張式至白馬與彪相聞【楚王彪武帝子黃初三年徙王白馬白馬縣屬東郡】凌又遣舍人勞精詣洛陽【勞姓也精名也姓譜其先居東海勞山因氏焉後漢有琅邪勞丙】語其子廣【語牛倨翻】廣曰凡舉大事應本人情曹爽以驕奢失民何平叔虚華不治【何晏字平叔】丁畢桓鄧雖並有宿望皆專競於世加變易朝典【朝直遥翻下同】政令數改【數所角翻】所存雖高而事不下接【言雖存心於高曠而不切事情與下不接也】民習於舊衆莫之從故雖勢傾四海聲震天下同日斬戮名士減半而百姓安之莫之或哀失民故也今司馬懿情雖難量【量音良】事未有逆而擢用賢能廣樹勝已【謂蔣濟高柔孫禮陳泰郭淮鄧艾等】脩先朝之政令【朝直遥翻】副衆心之所求爽之所以為惡者彼莫不必改【必當作畢】夙夜匪懈以恤民為先【懈古隘翻】父子兄弟並握兵要未易亡也【易以䜴翻】凌不從 冬十一月令狐愚復遣張式詣楚王【復扶又翻】未還會愚病卒 十二月辛卯即拜王凌為太尉【即拜者就夀春拜為太尉】庚子以司隸校尉孫禮為司空 光禄大夫徐邈卒邈以清節著名盧欽嘗著書稱邈曰徐公志高行潔【行下孟翻】才博氣猛其施之也高而不狷【狷吉掾翻】潔而不介博而守約猛而能寛聖人以清為難而徐公之所易也【易以䜴翻】或問欽徐公當武帝之時人以為通自為凉州刺史【明帝太和初邈為凉州刺史】及還京師人以為介何也欽答曰往者毛孝先崔季珪用事貴清素之士于時皆變易車服以求名高【事見六十五卷漢獻帝建安十三年毛玠字孝先崔琰字季珪】而徐公不改其常故人以為通比來天下奢靡轉相倣傚【比毗寐翻近也比來猶言近來也】而徐公雅尚自若不與俗同故前日之通乃今日之介也是世人之無常而徐公之有常也欽毓之子也【毓余六翻】二年夏五月以征西將軍郭淮為車騎將軍 初會稽潘夫人有寵於吳主【會古外翻】生少子亮【少詩照翻】吳主愛之全公主既與太子和有隙【事見上卷正始六年】欲豫自結數稱亮美以其夫之兄子尚女妻之【為後孫綝殺尚廢亮遷全公主張本數所角翻妻七細翻】吳主以魯王霸結朋黨以害其兄心亦惡之【惡烏路翻】謂侍中孫峻曰子弟不睦臣下分部【分部謂各分部黨若漢甘陵南北部】將有袁氏之敗【事見六十四卷建安七年】為天下笑若使一人立者安得不亂乎遂有廢和立亮之意然猶沈吟者歷年【沈吟者欲决而未决之意今人猶有此語沈持林翻】峻靜之曾孫也【孫静堅之季弟見六十二卷建安元年】秋吳主遂幽太子和驃騎將軍朱據諫曰太子國之本根加以雅性仁孝天下歸心昔晉獻用驪姬而申生不存【注已見前】漢武信江充而戾太子寃死【事見二十二卷漢武帝征和二年】臣竊懼太子不堪其憂雖立思子之宫無所復及矣吳主不聽據與尚書僕射屈晃【屈居勿翻】率諸將吏泥頭自縳連日詣闕請和吳主登白爵觀見甚惡之【白爵觀在建業宫中觀古玩翻】敕據晃等無事怱怱【怱怱急遽不諦細也】無難督陳正五營督陳象各上書切諫【吳主置左右無難營兵又置五營營兵各置督領之】據晃亦固諫不已吳主大怒族誅正象牽據晃入殿據晃猶口諫叩頭流血辭氣不撓【撓奴教翻】吳主杖之各一百左遷據為新都郡丞晃斥歸田里羣司坐諫誅放者以十數遂廢太子和為庶人徙故鄣【故鄣縣屬丹陽郡賢曰秦鄣郡所治也在今湖州安吉縣界師古曰鄣音章】賜魯王霸死殺楊笁流其尸於江又誅全寄吳安孫奇皆以其黨霸譖和故也【黨霸譖和事見上卷正始六年】初楊笁少獲聲名【少詩照翻】而陸遜謂之終敗勸笁兄穆令與之别族【别彼列翻分也】及笁敗穆以數諫戒笁得免死【數所角翻】朱據未至官中書令孫弘以詔書追賜死 冬十月廬江太守文欽偽叛以誘吳偏將軍朱異【誘音酉】欲使異自將兵迎已異知其詐表吳主以為欽不可迎吳主曰方今北土未一欽欲歸命宜且迎之若嫌其有譎者【譎古穴翻】但當設計網以羅之盛重兵以防之耳乃遣偏將軍呂據督二萬人與異并力至北界【北界謂魏吳分界之地在魏廬江郡南於吳為北】欽果不降【降戶江翻】異桓之子據範之子也 十一月大利景侯孫禮卒【據孫禮傳禮封大利亭侯】 吳主立子亮為太子 吳主遣軍十萬作堂邑塗塘以淹北道【堂邑縣前漢屬臨淮郡後漢屬廣陵郡魏吳在兩界之間為棄地賢曰堂邑今揚州六合縣杜佑曰揚州六合縣春秋楚之棠邑漢為堂邑淹北道以絶魏兵之窺建業吳主老矣良將多死為自保之規摹而已塗當作涂讀曰滁】 十二月甲辰東海定王霖卒【諡法純行不爽曰定安民法古曰定】 征南將軍王昶上言孫權流放良臣【良臣謂朱據等昶丑兩翻】適庶分爭【適讀曰嫡】可乘釁擊吳朝廷從之遣新城太守南陽州泰襲巫秭歸【州姓也泰名也晉有州綽風俗通云其先食采於州因氏焉】荆州刺史王基向夷陵【魏荆州刺史與征南府並屯宛時已徙屯新野】昶向江陵引竹絙為橋渡水擊之【絙居登翻大索也吳引沮漳之水浸江陵以北之地以限魏兵故昶為橋以渡之】吳大將施績夜遁入江陵【績朱然之子也然本施氏朱治以為子魏人本其所自出之姓稱之】昶欲引致平地與戰乃先遣五軍案大道發還使吳望見而喜又以所獲鎧馬甲首環城以怒之【鎧可亥翻環音宦】設伏兵以待之績果來追昶與戰大破之斬其將鍾離茂許旻 漢姜維復寇西平不克【復扶又翻】<br />
<br />
  三年春正月王基州泰擊吳兵皆破之降者數千口二月以尚書令司馬孚為司空 夏四月甲申以王昶為征南大將軍【以破吳兵進位也】 壬辰大赦太尉王凌聞吳人塞涂水【即前所作堂邑塗塘也楊正衡曰涂音滁據今滁河自滁州至真州塞悉則翻】欲因此發兵大嚴諸軍表求討賊詔報不聽凌遣將軍楊弘以廢立事告兖州刺史黃華華弘連名以白司馬懿懿將中軍乘水道討凌先下赦赦凌罪又為書諭凌已而大軍掩至百尺【水經注沙水東南過陳縣又東南流注于潁謂之交口水次有大堰即古百尺堰司馬宣王討王凌大軍掩至百尺即此地杜佑曰百尺在陳州宛丘縣不意其至而至曰掩至掩者掩其不備也我朝析汝隂之百尺鎮置萬夀縣】凌自知埶窮乃乘船單出迎懿遣掾王彧謝罪送印綬節鉞【掾俞絹翻】懿軍到丘頭【水經潁水過南頓縣又東逕丘頭丘頭南枕水魏書郡國志曰王凌面縛於此故號武丘杜佑曰即今潁川沈丘縣】凌面縛水次懿承詔遣主簿解其縛凌既蒙赦加恃舊好不復自疑【好呼到翻復扶又翻】徑乘小船欲趨懿【趨逡遇翻】懿使人逆止之住船淮中【水經注潁水自丘頭東南至慎縣又東南入于淮懿蓋進軍已近淮】相去十餘丈凌知見外【凌與懿同為公初以為蒙赦而欲趨懿懿逆拒之乃知以罪而見外】乃遥謂懿曰卿直以折簡召我我當敢不至邪而乃引軍來乎懿曰以卿非肯逐折簡者故也【古者簡長二尺四寸短者半之漢制簡長二尺短者半之蓋單執一札謂之簡折簡者折半之簡言其禮輕也又按南史孔闓為孔珪草表珪以示謝朓朓嗟吟良久手自折簡寫之】凌曰卿負我懿曰我寧負卿不負國家遂遣步騎六百送凌西詣京師【自潁河泝流而西詣洛陽】凌試索棺釘以觀懿意懿命給之【給棺釘者示之以必死索山客翻釘音丁】五月甲寅凌行到項遂飲藥死懿進至夀春張式等皆自首【首式救翻】懿窮治其事【治直之翻】諸相連者悉夷三族發凌愚冢剖棺暴尸於所近市三日【近其靳翻】燒其印綬章服親土埋之【孟子曰比化者毋使土親膚親土者臝葬也綬音受】初令狐愚為白衣時常有高志衆人謂愚必興令狐氏【令力呈翻】族父弘農太守邵獨以為愚性倜儻【倜他歷翻倜儻卓異也】不脩德而願大必滅我宗愚聞之心甚不平及邵為虎賁中郎將而愚仕進已多所更歷【更工衡翻】所在有名稱【稱昌孕翻凡名號謂之稱孟子題辭曰子者男子之通稱】愚從容謂邵曰【從千容翻】先時聞大人謂愚為不繼【先悉薦翻】今竟云何邪邵熟視而不答私謂妻子曰公治性度猶如故也【令狐愚字公治】以吾觀之終當敗滅但不知我久當坐之不邪【不讀曰否】將逮汝曹耳邵沒後十餘年而愚族滅【此晉人作魏史所書云爾】愚在兖州辟山陽單固為别駕【單音善】與治中楊康並為愚腹心及愚卒康應司徒辟至洛陽露愚隂事愚由是敗懿至夀春見單固問曰令狐反乎曰無有楊康白事事與固連【康所白愚隂事事與固連也】遂收捕固及家屬皆繫廷尉考實數十固固云無有【上固其名下固固執也】懿録楊康【録收也】與固對相詰【詰去吉翻】固辭窮乃罵康曰老傭【傭雇也奴僕受雇者曰傭老傭猶言老奴也】既負使君乂滅我族【使君謂令狐愚也】顧汝當活邪康初自冀封侯後以辭頗參錯【言獄辭與單固參雜也】亦并斬之臨刑俱出獄固又罵康曰老奴汝死自分耳【分扶問翻】若令死者有知汝何面目以行地下乎詔以楊州刺史諸葛誕為鎮東將軍都督揚州諸軍事【王凌死而用諸葛誕誕亦終於為魏以司馬懿之明達豈不知誕之乃心魏氏哉大敵在境帥難其才也】 吳主立潘夫人為皇后大赦改元太元 六月賜楚王彪死盡録諸王公置鄴使有司察之不得與人交關【慮復如楚王彪為變也】 秋七月壬戌皇后甄氏殂【甄之人翻】 辛未以司馬孚為太尉 八月戊寅舞陽宣文侯司馬懿卒【史記懿死為王凌之祟信乎儻其果能然固忠勇之鬼也通鑑不語怪今著之以示為人臣者】詔以其子衛將軍師為撫軍大將軍録尚書事【魏晉之制驃騎車騎衛將軍伏波撫軍都護鎮軍中軍四征四鎮龍驤典軍上軍輔國等大將軍位皆從公至録尚書事則專制朝政矣】 初南匈奴自謂其先本漢室之甥因冒姓劉氏太祖留單于呼厨泉於鄴分其衆為五部居并州境内【事見六十七卷漢獻帝建安二十一年】左賢王豹單于於扶羅之子也為左部帥部族最強【帥所類翻】城陽太守鄧艾【前漢置城陽國後漢省入琅邪國魏武帝平青州復置城陽郡】上言單于在内羌夷失統合散無主今單于之尊日疏而外土之威日重【謂南單于留鄴雖有尊名日與部落疏而左賢王豹居外部族最強其威日重也】則胡虜不可不深備也聞劉豹部有叛胡可因叛割為二國以分其埶去卑功顯前朝【謂去卑侍衛漢獻帝東還也事見六十一卷興平元年朝直遥翻】而子不繼業宜加其子顯號使居雁門離國弱寇【離國者離匈奴劉豹之國為二也】追録舊勲此御邊長計也又陳羌胡與民同處者【處昌呂翻】宜以漸出之使居民表【表外也使居編民之外也】以崇亷恥之教塞姦宄之路【塞悉則翻】司馬師皆從之【鄧艾所陳先於徙戎論司馬師既從之矣然卒不能料其亂華之漸抑所謂漸出之者行之而不究邪豈天將啟胡羯氐羌非人之所能為也】 吳立節中郎將陸抗屯柴桑詣建業治病病差【差楚懈翻病瘳也】當還吳主涕泣與别謂曰吾前聽用讒言與汝父大義不篤以此負汝前後所問一焚滅之莫令人見也【一焚滅之言一切悉焚滅之也責問陸遜事見上卷正始六年】是時吳主頗寤太子和之無罪冬十一月吳主祀南郊還得風疾欲召和還全公主及侍中孫峻中書令孫弘固爭之【爭者恐和復立為已患也】乃止吳主以太子亮幼少議所付託孫峻薦大將軍諸葛恪可付大事【此時通吳國上下皆以恪為才而峻薦之峻本無殺恪之心也恪死於峻手其罪在恪峻既竊權授之弟綝以亂吳國其罪在峻讀史者其審諸】吳主嫌恪剛狠自用【狠戶懇翻】峻曰當今朝臣之才無及恪者乃召恪於武昌恪將行上大將軍呂岱戒之曰世方多難【難乃旦翻】子每事必十思恪曰昔季文子三思而後行夫子曰再思可矣【見論語季文子魯大夫季孫行父也】今君令恪十思明恪之劣也岱無以答時咸謂之失言<br />
<br />
  虞喜論曰夫託以天下至重也以人臣行主威至難也兼二至而管萬機能勝之者鮮矣【勝音升鮮息淺翻】呂侯國之元耆【元耆猶言元老也】志度經遠甫以十思戒之而便以示劣見拒此元遜之疏機神不俱者也【諸葛恪字元遜疏讀曰疎機者逢事會而發神者人之靈明逢事會而靈明無以應之則為不俱矣】若因十思之義廣諮當世之務聞善速於雷動從諫急於風移豈得隕身殿堂死於凶豎之刃【謂恪後為孫峻所殺也】世人奇其英辯造次可觀【造七到翻】而哂呂侯無對為陋【哂矢忍翻】不思安危終始之慮是樂春藻之繁華【樂音洛】忘秋實之甘口也昔魏人伐蜀蜀人禦之精嚴垂發而費禕方與來敏對棊意無厭倦敏以為必能辦賊【事見上卷正始五年】言其明略内定貌無憂色也况長寧以為君子臨事而懼好謀而成【臨事而懼好謀而成論語記孔子之言而所謂長寧者未知其為誰也】蜀為蕞爾之國【蕞祖外翻】而方向大敵所規所圖唯守與戰何可矜已有餘晏然無戚斯乃禕性之寛簡不防細微卒為降人郭循所害【循當作脩注見後卒子恤翻】豈非兆見於彼而禍成於此哉【見賢遍翻】往聞長寧之甄文偉【甄别也】今覩元遜之逆呂侯二事體同皆足以為世鑒也<br />
<br />
  恪至建業見吳主於卧内受詔牀下以大將軍領太子太傅孫弘領少傅詔有司諸事一統於恪惟殺生大事然後以聞為制羣官百司拜揖之儀各有品序【諸葛恪本盛氣者也吳主既任之又為制百司拜揖之儀品是其氣愈盛矣使無東關之捷合肥之敗恪亦不能濟吳之國事也為于偽翻】又以會稽太守北海滕胤為太常胤吳主壻也【為恪胤皆敗張本會古外翻】 十二月以光禄勲滎陽鄭冲為司空漢費禕還成都【費父沸翻】望氣者云都邑無宰相位乃復<br />
<br />
  北屯漢夀【以禕之才識乃復信望氣者之說邪葭萌縣漢屬廣漢郡蜀先主改曰漢夀縣屬梓潼郡】是歲漢尚書令呂乂卒以侍中陳祇守尚書令【祇為尚書】<br />
<br />
  【令黃皓自此愈用事矣】<br />
<br />
  四年春正月癸卯以司馬師為大將軍 吳主立故太子和為南陽王使居長沙仲姬子奮為齊王居武昌王夫人子休為琅邪王居虎林【虎林濱大江吳置督守之其後孫綝遣朱異自虎林襲夏口兵至武昌而夏口督孫壹奔魏則虎林又在武昌之下】 二月立皇后張氏大赦后故凉州刺史既之孫東莞太守緝之女也【東莞縣漢屬琅邪郡魏分為郡沈約曰晉武帝泰始元年分琅邪立東莞郡當是魏既分而復屬於琅邪晉又分也莞音官】召緝拜光禄大夫【為下司馬師殺緝張本】 吳改元神鳳大赦吳潘后性剛戾吳主疾病后使人問孫弘以呂后稱制故事左右不勝其虐【勝音升】伺其昏睡縊殺之託言中惡【縊於賜翻又於計翻中惡暴病而死也中竹仲翻】後事泄坐死者六七人【斯事也實吳用事之臣所為也潘后欲求稱制左右小人正當相與從臾為之安有不勝其虐而縊殺之之理吳史緣飾後人遂因而書之云爾孟子曰盡信書不如無書誠哉】吳主病困召諸葛恪孫弘滕胤及將軍呂據侍中孫峻入卧内屬以後事【屬之欲翻】夏四月吳主殂【年七十一】孫弘素與諸葛恪不平懼為恪所治【治直之翻】祕不發喪欲矯詔誅恪孫峻以告恪恪請弘咨事【謀事曰咨】於坐中殺之【坐徂卧翻】乃發喪諡吳主曰大皇帝【沈約曰諡大諡法所不載】太子亮即位【孫亮字子明權少子也即位時年十歲】大赦改元建興閏月以諸葛恪為太傅滕胤為衛將軍呂岱為大司馬恪乃命罷視聽息校官【吳主權置校官典校諸官府及州郡文書專任以為耳目今息校官即所謂罷視聽也】原逋責除關稅【古者關譏而不征後世始征之關之有稅非古也除之是也】崇恩澤衆莫不悦恪每出入百姓延頸思見其狀恪不欲諸王處濱江兵馬之地【處昌呂翻】乃徙齊王奮於豫章琅邪王休於丹陽【奮休皆吳主亮之兄也】奮不肯徙恪為牋以遺奮曰【遺于季翻】帝王之尊與天同位是以家天下臣父兄仇讐有善不得不舉親戚有惡不得不誅所以承天理物先國後家【先後皆去聲】蓋聖人立制百代不易之道也昔漢初興多王子弟至於大強輒為不軌上則幾危社稷【謂吳楚七國淮南濟北燕廣陵也王于况翻幾居希翻】下則骨肉相殘【謂如廣川王去之類】其後懲戒以為大諱自光武以來諸王有制惟得自娛於宫内不得臨民干與政事其與交通皆有重禁【光武設科禁藩王不得交通賓客干與讀曰預】遂以全安各保福祚此則前世得失之驗也大行皇帝覽古戒今防牙遏萌【牙與芽同】慮於千載【載于亥翻】是以寢疾之日分遣諸王各早就國詔策勤渠科禁嚴峻其所戒敕無所不至誠欲上安宗廟下全諸王各早就國承無凶國害家之悔也【書洪範曰凶于而國害于而家承當作永】大王宜上惟太伯順父之志【周太王三子長曰太伯次曰仲雍次曰季歷季歷之子曰昌有聖德太王欲傳國季歷以及昌太伯仲雍遂逃之荆蠻讓國季歷以成父之志惟思也】中念河間獻王東海王彊恭順之節【漢河間獻王德於武帝兄也東海王彊於明帝異母兄也二王之事二帝極為恭順事並見漢紀】下存前世驕恣荒亂之王以為警戒而聞頃至武昌以來多違詔敕不拘制度擅發諸將兵治護宫室【治直之翻】又左右常從有罪過者當以表聞公付有司而擅私殺事不明白【吳諸王有常從吏兵置常從督以領之明顯也白奏也謂不顯奏其罪而擅殺之也從才用翻】中書楊融親受詔敕所當恭肅乃云正自不聽禁【謂不聽禁約也】當如我何聞此之日小大驚怪莫不寒心里語曰明鑑所以照形古事所以知今大王宜深以魯王為戒【謂魯王霸也】改易其行【行下孟翻】戰戰兢兢盡禮朝廷如此則無求不得若棄忘先帝法教懷輕慢之心臣下寧負大王不敢負先帝遺詔寧為大王所怨疾豈敢忘尊主之威而令詔敕不行於藩臣邪向使魯王早納忠直之言懷驚懼之慮【驚當作兢】則享祚無窮豈有滅亡之禍哉夫良藥苦口唯病者能甘之忠言逆耳唯達者能受之今者恪等慺慺【慺盧侯翻慺慺恭謹貌】欲為大王除危殆於萌牙【為于偽翻】廣福慶之基原是以不自知言至【至極也切也】願蒙三思王得牋懼遂移南昌【南昌縣豫章郡治所】初吳大帝築東興隄以遏巢湖【吳主權黃龍二年築東興隄】其後入寇淮南敗以内船遂廢不復治【謂正始二年芍陂之敗也遏巢湖所以利舟師而反為湖内之船所敗故廢而不治復扶又翻治直之翻】冬十月太傅恪會衆於東興更作大隄左右結山俠築兩城【今柵江口冇兩山濡須山在和州界謂之東關七寶山在無為軍界謂之西關兩山對峙中為石梁鑿石通水唐志廬州巢縣東南四十里有故東關俠讀曰夾古者俠夾二字通漢靈帝光和二年華山亭碑其文有云吏卒俠路晉宋書諸正有俠轂隊皆以夾為俠】各留千人使將軍全端守西城都尉留略守東城【留姓也漢功臣表有彊園侯留肹姓譜曰衛大夫留封人之後漢末避地會稽遂居東陽為郡豪族】引軍而還鎮東將軍諸葛誕言於大將軍師曰今因吳内侵使文舒逼江陵仲恭向武昌【王昶字文舒毋丘儉字仲恭】以羈吳之上流然後簡精卒攻其兩城比救至可大獲也【比必寐翻】是時征南大將軍王昶征東將軍胡遵鎮南將軍毋丘儉等各獻征吳之計朝廷以三征計異【漢置四征將軍謂征東征西征南征北也其後又置四鎭將軍有功進號則自鎮為征毋丘儉方為鎮南而曰三征史槩言之】詔問尚書傅嘏嘏對曰議者或欲汎舟徑濟橫行江表或欲四道並進攻其城壘或欲大佃疆場【佃讀曰田】觀釁而動誠皆取賊之常計也然自治兵以來出入三載非掩襲之軍也【治直之翻】賊之為寇幾六十年矣【自漢建安十三年赤壁之戰吳魏始為寇敵至是年凡五十五年矣吳魏通者三年耳幾居希翻】君臣相保吉凶共患又喪其元帥【喪息浪翻】上下憂危設令列船津要堅城據險橫行之計其殆難捷今邊壤之守與賊相遠賊設羅落又特重密【謂設烽燧遠候望以羅落邊面也羅布也落與絡同聨絡也莊子曰牛馬四足是謂天落馬首穿牛鼻是謂人用此落字重直龍翻】間諜不行【間古莧翻下同諜逹協翻】耳目無聞夫軍無耳目校察未詳而舉大衆以臨巨險此為希幸徼功【徼一遥翻】先戰而後求勝非全軍之長策也唯有進軍大佃最差完牢可詔昶遵等擇地居險審所錯置【錯倉故翻】及令三方一時前守奪其肥壤使還塉土一也【塉秦昔翻】兵出民表寇鈔不犯二也【鈔楚交翻】招懷近路降附日至三也【降戶江翻】羅落遠設間構不來四也賊退其守羅落必淺佃作易立五也【易以䜴翻】坐食積穀士不運輸六也釁隙時聞討襲速决七也凡此七者軍事之急務也不據則賊擅便資據之則利歸於國不可不察也夫屯壘相偪形勢已交智勇得陳巧拙得用策之而知得失之計角之而知有餘不足虜之情偽將焉所逃【焉於䖍翻】夫以小敵大則役煩力竭以貧敵富則歛重財匱【歛力瞻翻】故曰敵逸能勞之飽能饑之【孫武子兵法之言也】此之謂也司馬師不從十一月詔王昶等三道擊吳十二月王昶攻南郡毋丘儉向武昌胡遵諸葛誕率衆十萬攻東興甲寅吳太傅恪將兵四萬晨夜兼行救東興胡遵等敕諸軍作浮橋以度陳於隄上【陳讀曰陣】分兵攻兩城城在高峻不可卒拔【卒讀曰猝】諸葛恪使冠軍將軍丁奉與呂據留贊唐咨為前部從山西上【上時掌翻下同】奉謂諸將曰今諸軍行緩若賊據便地則難以爭鋒我請趨之【趨七喻翻】乃辟諸軍使下道【辟讀曰闢辟諸軍使避路而已軍前進也】奉自率麾下三千人徑進時北風奉舉帆二日即至東關遂據徐塘【徐塘蓋近東關】時天雪寒胡遵等方置酒高會奉見其前部兵少謂其下曰取封侯爵賞正在今日乃使兵皆解鎧去矛戟【去羌呂翻】但兜鍪刀楯倮身緣堨【兜鍪首鎧鍪莫侯翻楯食尹翻倮魯果翻堨阿葛翻】魏人望見大笑之不即嚴兵吳兵得上便鼓譟斫破魏前屯呂據等繼至魏軍驚擾散走爭渡浮橋橋壞絶自投於水更相蹈藉【更工衡翻】前部督韓綜樂安太守桓嘉等皆沒死者數萬綜故吳叛將【綜叛吳事見七十卷明帝太和元年】數為吳害【數所角翻】吳大帝常切齒恨之諸葛恪命送其首以白大帝廟獲車乘牛馬騾驢各以千數【乘繩證翻騾盧戈翻】資器山積振旅而歸 初漢姜維寇西平【見上卷嘉平二年】獲中郎將郭偱【偱徧考字書無其字又考三國志三少帝紀作郭脩蜀志張嶷傳亦作郭脩裴松之注亦云脩字孝先費禕傳作郭循後主傳亦然今三國志舊本凡書循者多從偱余謂此偱即脩字之誤也後人以偱字無所出又改亻為彳遂為循字耳盤洲洪氏曰自東漢以來凡盾字皆作偱字又曰漢隸循脩頗相近隸法循脩只爭一畫】漢人以為左將軍偱欲刺漢主不得親近每因上夀且拜且前【刺七亦翻近其靳翻上時掌翻】為左右所遏事輒不果【為下偱殺費禕張本】<br />
<br />
  資治通鑑卷七十五<br />
<br />
<史部,編年類,資治通鑑>  <br>
   </div> 

<script src="/search/ajaxskft.js"> </script>
 <div class="clear"></div>
<br>
<br>
 <!-- a.d-->

 <!--
<div class="info_share">
</div> 
-->
 <!--info_share--></div>   <!-- end info_content-->
  </div> <!-- end l-->

<div class="r">   <!--r-->



<div class="sidebar"  style="margin-bottom:2px;">

 
<div class="sidebar_title">工具类大全</div>
<div class="sidebar_info">
<strong><a href="http://www.guoxuedashi.com/lsditu/" target="_blank">历史地图</a></strong>  
<a href="http://www.880114.com/" target="_blank">英语宝典</a>  
<a href="http://www.guoxuedashi.com/13jing/" target="_blank">十三经检索</a> 
<br><strong><a href="http://www.guoxuedashi.com/gjtsjc/" target="_blank">古今图书集成</a></strong> 
<a href="http://www.guoxuedashi.com/duilian/" target="_blank">对联大全</a> <strong><a href="http://www.guoxuedashi.com/xiangxingzi/" target="_blank">象形文字典</a></strong> 

<br><a href="http://www.guoxuedashi.com/zixing/yanbian/">字形演变</a>  <strong><a href="http://www.guoxuemi.com/hafo/" target="_blank">哈佛燕京中文善本特藏</a></strong>
<br><strong><a href="http://www.guoxuedashi.com/csfz/" target="_blank">丛书&方志检索器</a></strong> <a href="http://www.guoxuedashi.com/yqjyy/" target="_blank">一切经音义</a>  

<br><strong><a href="http://www.guoxuedashi.com/jiapu/" target="_blank">家谱族谱查询</a></strong>  <strong><a href="http://shufa.guoxuedashi.com/sfzitie/" target="_blank">书法字帖欣赏</a></strong> 
<br>

</div>
</div>


<div class="sidebar" style="margin-bottom:0px;">

<font style="font-size:22px;line-height:32px">QQ交流群9:489193090</font>


<div class="sidebar_title">手机APP 扫描或点击</div>
<div class="sidebar_info">
<table>
<tr>
	<td width=160><a href="http://m.guoxuedashi.com/app/" target="_blank"><img src="/img/gxds-sj.png" width="140"  border="0" alt="国学大师手机版"></a></td>
	<td>
<a href="http://www.guoxuedashi.com/download/" target="_blank">app软件下载专区</a><br>
<a href="http://www.guoxuedashi.com/download/gxds.php" target="_blank">《国学大师》下载</a><br>
<a href="http://www.guoxuedashi.com/download/kxzd.php" target="_blank">《汉字宝典》下载</a><br>
<a href="http://www.guoxuedashi.com/download/scqbd.php" target="_blank">《诗词曲宝典》下载</a><br>
<a href="http://www.guoxuedashi.com/SiKuQuanShu/skqs.php" target="_blank">《四库全书》下载</a><br>
</td>
</tr>
</table>

</div>
</div>


<div class="sidebar2">
<center>


</center>
</div>

<div class="sidebar"  style="margin-bottom:2px;">
<div class="sidebar_title">网站使用教程</div>
<div class="sidebar_info">
<a href="http://www.guoxuedashi.com/help/gjsearch.php" target="_blank">如何在国学大师网下载古籍?</a><br>
<a href="http://www.guoxuedashi.com/zidian/bujian/bjjc.php" target="_blank">如何使用部件查字法快速查字?</a><br>
<a href="http://www.guoxuedashi.com/search/sjc.php" target="_blank">如何在指定的书籍中全文检索?</a><br>
<a href="http://www.guoxuedashi.com/search/skjc.php" target="_blank">如何找到一句话在《四库全书》哪一页?</a><br>
</div>
</div>


<div class="sidebar">
<div class="sidebar_title">热门书籍</div>
<div class="sidebar_info">
<a href="/so.php?sokey=%E8%B5%84%E6%B2%BB%E9%80%9A%E9%89%B4&kt=1">资治通鉴</a> <a href="/24shi/"><strong>二十四史</strong></a>&nbsp; <a href="/a2694/">野史</a>&nbsp; <a href="/SiKuQuanShu/"><strong>四库全书</strong></a>&nbsp;<a href="http://www.guoxuedashi.com/SiKuQuanShu/fanti/">繁体</a>
<br><a href="/so.php?sokey=%E7%BA%A2%E6%A5%BC%E6%A2%A6&kt=1">红楼梦</a> <a href="/a/1858x/">三国演义</a> <a href="/a/1038k/">水浒传</a> <a href="/a/1046t/">西游记</a> <a href="/a/1914o/">封神演义</a>
<br>
<a href="http://www.guoxuedashi.com/so.php?sokeygx=%E4%B8%87%E6%9C%89%E6%96%87%E5%BA%93&submit=&kt=1">万有文库</a> <a href="/a/780t/">古文观止</a> <a href="/a/1024l/">文心雕龙</a> <a href="/a/1704n/">全唐诗</a> <a href="/a/1705h/">全宋词</a>
<br><a href="http://www.guoxuedashi.com/so.php?sokeygx=%E7%99%BE%E8%A1%B2%E6%9C%AC%E4%BA%8C%E5%8D%81%E5%9B%9B%E5%8F%B2&submit=&kt=1"><strong>百衲本二十四史</strong></a>  <a href="http://www.guoxuedashi.com/so.php?sokeygx=%E5%8F%A4%E4%BB%8A%E5%9B%BE%E4%B9%A6%E9%9B%86%E6%88%90&submit=&kt=1"><strong>古今图书集成</strong></a>
<br>

<a href="http://www.guoxuedashi.com/so.php?sokeygx=%E4%B8%9B%E4%B9%A6%E9%9B%86%E6%88%90&submit=&kt=1">丛书集成</a> 
<a href="http://www.guoxuedashi.com/so.php?sokeygx=%E5%9B%9B%E9%83%A8%E4%B8%9B%E5%88%8A&submit=&kt=1"><strong>四部丛刊</strong></a>  
<a href="http://www.guoxuedashi.com/so.php?sokeygx=%E8%AF%B4%E6%96%87%E8%A7%A3%E5%AD%97&submit=&kt=1">說文解字</a> <a href="http://www.guoxuedashi.com/so.php?sokeygx=%E5%85%A8%E4%B8%8A%E5%8F%A4&submit=&kt=1">三国六朝文</a>
<br><a href="http://www.guoxuedashi.com/so.php?sokeytm=%E6%97%A5%E6%9C%AC%E5%86%85%E9%98%81%E6%96%87%E5%BA%93&submit=&kt=1"><strong>日本内阁文库</strong></a> <a href="http://www.guoxuedashi.com/so.php?sokeytm=%E5%9B%BD%E5%9B%BE%E6%96%B9%E5%BF%97%E5%90%88%E9%9B%86&ka=100&submit=">国图方志合集</a> <a href="http://www.guoxuedashi.com/so.php?sokeytm=%E5%90%84%E5%9C%B0%E6%96%B9%E5%BF%97&submit=&kt=1"><strong>各地方志</strong></a>

</div>
</div>


<div class="sidebar2">
<center>

</center>
</div>
<div class="sidebar greenbar">
<div class="sidebar_title green">四库全书</div>
<div class="sidebar_info">

《四库全书》是中国古代最大的丛书,编撰于乾隆年间,由纪昀等360多位高官、学者编撰,3800多人抄写,费时十三年编成。丛书分经、史、子、集四部,故名四库。共有3500多种书,7.9万卷,3.6万册,约8亿字,基本上囊括了古代所有图书,故称“全书”。<a href="http://www.guoxuedashi.com/SiKuQuanShu/">详细>>
</a>

</div> 
</div>

</div>  <!--end r-->

</div>
<!-- 内容区END --> 

<!-- 页脚开始 -->
<div class="shh">

</div>

<div class="w1180" style="margin-top:8px;">
<center><script src="http://www.guoxuedashi.com/img/plus.php?id=3"></script></center>
</div>
<div class="w1180 foot">
<a href="/b/thanks.php">特别致谢</a> | <a href="javascript:window.external.AddFavorite(document.location.href,document.title);">收藏本站</a> | <a href="#">欢迎投稿</a> | <a href="http://www.guoxuedashi.com/forum/">意见建议</a> | <a href="http://www.guoxuemi.com/">国学迷</a> | <a href="http://www.shuowen.net/">说文网</a><script language="javascript" type="text/javascript" src="https://js.users.51.la/17753172.js"></script><br />
  Copyright &copy; 国学大师 古典图书集成 All Rights Reserved.<br>
  
  <span style="font-size:14px">免责声明:本站非营利性站点,以方便网友为主,仅供学习研究。<br>内容由热心网友提供和网上收集,不保留版权。若侵犯了您的权益,来信即刪。scp168@qq.com</span>
  <br />
ICP证:<a href="http://www.beian.miit.gov.cn/" target="_blank">鲁ICP备19060063号</a></div>
<!-- 页脚END --> 
<script src="http://www.guoxuedashi.com/img/plus.php?id=22"></script>
<script src="http://www.guoxuedashi.com/img/tongji.js"></script>

</body>
</html>
