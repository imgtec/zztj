\section{資治通鑑卷二百八十八}
宋 司馬光 撰

胡三省 音註

後漢紀三|{
	起著雍涒灘三月盡屠維作噩凡一年有奇}


高祖睿文聖武昭肅孝皇帝下

乾祐元年三月丙辰史弘肇起復加兼侍中 侯益家富於財厚賂執政及史弘肇等由是大臣爭譽之|{
	執政謂蘇逢吉楊邠等皆當時大臣也譽音余}
丙寅以益兼中書令行開封尹|{
	行者行尹事未正除也}
改廣晉為大名府|{
	左傳晉卜偃曰魏大名也取以名府}
晉昌軍為永興軍|{
	以革晉命故改廣晉與晉昌}
侯益盛毁王景崇於朝言其恣横|{
	侯益以王景崇欲殺已幸免而歸朝故毁之横戶孟翻}
景崇聞益尹開封知事已變内不自安且怨朝廷|{
	怨朝廷不能體先帝遺旨反聽侯益之讒也}
會詔遣供奉官王益如鳳翔徵趙匡贊牙兵詣闕趙思綰等甚懼|{
	趙思綰趙匡贊牙校也見上卷}
景崇因以言激之思綰途中謂其黨常彦卿曰小太尉已落其手|{
	趙思綰等本趙延夀部曲故呼匡贊為小太尉言匡贊入朝為已落漢人之手也}
吾屬至京師并死矣奈何彦卿曰臨幾制變子勿復言|{
	復扶又翻}
癸酉至長安永興節度副使安友規巡檢喬守温出迎王益置酒於客亭|{
	諸州鎮皆有客亭以為迎送宴餞之所}
思綰前白曰壕寨使已定舍館於城東|{
	壕寨使掌營造浚築及次舍下寨}
今將士家屬皆在城中|{
	趙思綰部兵先從趙匡贊鎮長安故家屬在城中}
欲各入城挈家詣城東宿友規等然之時思綰等皆無鎧仗既入西門有州校坐門側思綰遽奪其劒斬之其徒因大譟持白梃|{
	校戶教翻梃徒鼎翻}
殺守門者十餘人分遣其黨守諸門思綰入府開庫取鎧仗給之友規等皆逃去思綰遂據城集城中少年得四千餘人繕城隍葺樓堞旬日間戰守之具皆備|{
	少詩照翻葺七入翻堞達恊翻}
王景崇諷鳳翔吏民表景崇知軍府事朝廷患之甲戌徙靜難節度使王守恩為永興節度使|{
	欲以制趙思綰難乃旦翻}
徙保義節度使趙暉為鳳翔節度使|{
	欲以制王景崇}
並同平章事以景崇為邠州留後令便道之官虢州伶人靖邊庭殺團練使田令方驅掠州民奔趙思綰|{
	靖姓也優伶之名與姓通取一義所以為謔也何氏姓苑曰靖姓齊靖郭君之後風俗通曰靖姓單靖公之後也}
至潼關|{
	虢州西北至潼關百有餘里}
潼關守將出擊之其衆皆潰 初契丹主北歸至定州|{
	契丹主德光北歸死于殺胡林此謂卾約北歸至定州也}
以義武節度副使耶律忠為節度使徙故節度使孫方簡為大同節度使|{
	晉齊王開運三年契丹主德光以孫方簡為義武節度使 考異曰實録方簡作方諫按方簡避周諱改名方諫實録誤也}
方簡怨恚且懼入朝為契丹所留遷延不受命|{
	恚於避翻朝直遥翻}
帥其黨三千人保狼山故寨|{
	孫方簡兄弟保狼山見二百八十五卷晉開運三年帥讀曰率下同}
控守要害契丹攻之不克未幾遣使請降|{
	幾居豈翻}
帝復其舊官以扞契丹|{
	復以為義武節度使}
邪律忠聞鄴都既平|{
	去年十一月杜重威降鄴都平事見上卷}
常懼華人為變詔以成德留後劉在明為幽州道馬步都部署使出兵經略定州未行忠與滿達勒焚掠定州悉驅其人弃城北去|{
	定州東至鎮州止隔祁州耳契丹聞鎮州將出兵故弃城而去}
孫方簡自狼山帥其衆數百還據定州又奏以弟行友為易州刺史方遇為泰州刺史每契丹入寇兄弟奔命|{
	奔命者奔走以救急也}
契丹頗畏之於是晉末州縣䧟契丹者皆復為漢有矣丙子以劉在明為成德節度滿達勒至其國契丹主責失守滿達勒不服曰因朝廷徵漢官致亂耳|{
	謂徵馮道等也事見上卷上年}
契丹主鴆殺之 蘇逢吉等為相多遷補官吏楊邠以為虚費國用所奏多抑之逢吉等不悦中書侍郎兼戶部尚書同平章事李濤上疏言今關西紛擾外禦為急|{
	上時掌翻}
二樞密皆佐命功臣官雖貴而家未富宜授以要害大鎮|{
	李濤之疏承蘇逢吉之意也二樞密謂楊邠郭威}
樞機之務在陛下目前易以裁決|{
	易以䜴翻}
逢吉禹珪自先帝時任事皆可委也楊邠郭威聞之見太后泣訴稱臣等從先帝起艱難中今天子聽人言欲弃之於外况關西方有事|{
	謂岐雍舉兵反}
臣等何忍自取安逸不顧社稷若臣等必不任職乞留過山陵太后怒以讓帝曰國家勲舊之臣奈何聽人言而逐之帝曰此宰相所言也因詰責宰相|{
	詰去吉翻}
濤曰此疏臣獨為之它人無預丁丑罷濤政事勒歸私第|{
	為將相交惡張本}
是日邠涇同華四鎮|{
	邠帥王守恩涇帥史匡威同帥張彦威華帥扈從珂華戶化翻}
俱上言護國節度使兼中書令李守貞與永興鳳翔同反|{
	趙思綰據永興王景崇據鳳翔}
始守貞聞杜重威死而懼|{
	杜重威死見上卷是年正月}
隂有異志自以晉世嘗為上將有戰功|{
	李守貞破契丹于馬家口而克青州又破契丹於陽城其功不細}
素好施得士卒心|{
	好呼到翻施式豉翻}
漢室新造天子年少初立|{
	少詩照翻}
執政皆後進有輕朝廷之志乃招納亡命養死士治城塹繕甲兵晝夜不息遣人間道齎蠟丸結契丹屢為邉吏所獲|{
	治直之翻間古莧翻}
浚儀人趙修已素善術數|{
	舊唐書地理志浚儀故縣隋置在今縣北三十里唐武德四年移縣于州北羅城内貞觀元年移於州西一里後治郭下}
自守貞鎮滑州署司戶參軍累從移鎮|{
	晉開運初李守貞鎮義成後徙鎮泰寜天平歸德至是鎮護國為亂}
為守貞言時命不可勿妄動|{
	為于偽翻}
前後切諫非一守貞不聽乃稱疾歸鄉里僧摠倫以術媚守貞言其必為天子守貞信之又嘗會將佐置酒引弓指舐掌虎圖曰吾有非常之福當中其舌一發中之|{
	舐直氏翻中竹仲翻}
左右皆賀守貞益自負會趙思綰據長安奉表獻御衣於守貞守貞自謂天人恊契乃自稱秦王遣其驍將平陸王繼勲據潼關|{
	舊唐書地理志陜州平陸縣隋之河北縣也唐天寶三載陜郡太守李齊物開三門石下得戟大刃有平陸篆字因改為平陸縣九域志平陸縣在陜州北五里}
以思綰為晉昌節度使同州距河中最近|{
	河中府西至同州六十里耳}
匡國節度使張彦威 |{
	考異曰周太祖實録作彦成蓋避周祖諱薛史因之今從廣本}
常詗守貞所為|{
	詗古永翻又翾正翻}
奏請先為之備詔滑州馬軍都指揮使羅金山將部兵戍同州故守貞起兵同州不為所併金山雲州人也 定難節度使李彛殷發兵屯境上奏稱去三載前|{
	難乃旦翻去已往也}
羌族㖡母|{
	龍龕手鏡㖡音夜母讀如謨}
殺綏州刺史李仁裕叛去請討之慶州上言請益兵為備|{
	以備羌也}
詔以司天言今歲不利先舉兵諭止之 夏四月辛巳陜州都監王玉奏克復潼關|{
	監古衘翻}
帝與左右謀以太后怒李濤離間|{
	間古莧翻}
欲更進用二樞密以明非帝意左右亦疾二蘇之專欲奪其權共勸之|{
	二樞密楊邠郭威二蘇逢吉禹珪}
壬午制以樞密使楊邠為中書侍郎兼吏部尚書同平章事樞密使郭威為樞密使又加三司使王章同平章事凡中書除官諸司奏事帝皆委邠斟酌自是三相拱手|{
	三相竇貞固蘇逢吉蘇禹珪}
政事盡決於邠事有未更邠所可否者|{
	更工衡翻經也}
莫敢施行遂成凝滯三相每進擬用人苟不出邠意雖簿尉亦不之與邠素不喜書生|{
	喜許記翻}
常言國家府廪實甲兵彊乃為急務至於文章禮樂何足介意既恨二蘇排已|{
	以其使李濤上疏請出二樞密為外鎮也}
又以其除官太濫為衆所非欲矯其弊由是艱於除拜士大夫往往有自漢興至亡不霑一命者|{
	此所謂士大夫指言内外在官之人命言漢朝之命}
凡門䕃及百司入仕者悉罷之|{
	門䕃謂任子也百司入仕所謂流外也}
雖由邠之愚蔽時人亦咎二蘇之不公所致云 以鎮寧節度使郭從義充永興行營都部署將侍衛兵討趙思綰戊子以保義節度使白文珂為河中行營都部署内客省使王峻為都監辛卯削奪李守貞官爵命文珂等會兵討之乙未以寧江節度使侍衛步軍都指揮使尚洪遷為西面行營都虞侯|{
	寧江軍夔州時屬蜀境尚洪遷遥領也}
王景崇遷延不之邠州閲集鳳翔丁壯詐言討趙思綰仍牒邠州會兵|{
	王景崇欲并岐邠之兵以舉事}
契丹主如遼陽|{
	漢遼東郡有遼陽縣大梁水與遼水會處也契丹于此置遼陽府歐史自黄龍府西北行一千三百里至遼陽府按遼陽府契丹之東京舊勃海地距燕京二千五百一十里}
故晉主與太后皇后皆謁見|{
	見賢遍翻}
有齊納爾者契丹主之妻兄也聞晉主有女未嫁詣晉主求之晉主辭以幼後數日契丹主使人馳取其女而去與齊納爾 王景崇遺蜀鳳州刺史徐彦書求通互市|{
	遺唯季翻}
壬戌蜀主使彦復書招之契丹主留晉翰林學士徐台符於幽州|{
	徐台符從契丹主北去見上卷上年}
台符逃歸 五月乙亥滑州言河決魚池|{
	魚池地名河決之後謂之魚池口}
六月戊寅朔日有食之 辛巳以奉國左廂都虞侯劉詞充河中行營馬步都虞侯 乙酉王景崇遣使請降于蜀亦受李守貞官爵 高從誨既與漢絶|{
	見上卷天福十二年}
北方商旅不至境内貧乏乃遣使上表謝罪乞修職貢詔遣使慰撫之 西面行營都虞侯尚洪遷攻長安傷重而卒|{
	卒子恤翻}
秋七月以工部侍郎李穀充西南面行營都轉運使|{
	為李穀見親任于周朝張本}
庚申加樞密使郭威同平章事 蜀司空兼中書侍郎同平章事張業性豪侈強市人田宅|{
	以威力臨人人畏其威力不得已而就與為市是為強市}
藏匿亡命於私第置獄繫負債者或歷年至有瘐死者|{
	蘇林曰瘐病也囚徒病律名為瘐如淳曰律囚以飢寒死曰瘐音勇主翻}
其子檢校左僕射繼昭好擊劒|{
	好呼到翻}
嘗與僧歸信訪善劒者右匡聖都指揮使孫漢韶與業有隙密告業繼昭謀反翰林承旨李昊奉聖控鶴馬步都指揮使安思謙復從而譛之|{
	復扶又翻}
甲子業入朝蜀主命壯士就都堂擊殺之下詔暴其罪惡籍没其家樞密使保寧節度使兼侍中王處回亦專權貪縱|{
	王處回以節兼侍中不在閬州}
賣官鬻獄四方饋獻皆先輸處回次及内府|{
	此所謂四方止以蜀之境上言之}
家貲巨萬子德鈞亦驕横|{
	横戶孟翻}
張業既死蜀主不忍殺處回聽歸私第處回惶恐辭位以為武德節度使兼中書令|{
	王處回亦不得至梓州}
蜀主欲以普豐庫使高延昭茶酒庫使王昭遠為樞密使|{
	普豐茶酒二庫使皆蜀所置}
以其名位素輕乃授通奏使知樞密院事|{
	通奏使亦蜀所置}
昭遠成都人幼以僧童從其師入府蜀高祖愛其敏慧令給事蜀主左右至是委以機務府庫金帛恣其取與不復會計|{
	復扶又翻會古外翻至于宋興蜀主遂以用王昭遠亡國}
戊辰以郭從義為永興節度使白文珂兼知河中行府事|{
	時郭從義討長安就以永興節授之白文珂討河中因使之知行府事}
蜀主以翰林承旨尚書左丞李昊為門下侍郎兼戶部尚書翰林學士兵部侍郎徐光溥為中書侍郎兼禮部尚書並同平章事蜀安思謙謀盡去舊將|{
	去羌呂翻}
又譛衛聖都指揮使兼中書令趙廷隱謀反欲代其位夜發兵圍其第會山南西道節度使李廷珪入朝極言廷隱無罪乃得免廷隱因稱疾固請解軍職甲戍蜀主許之|{
	史言蜀主以新間舊}
鳳翔節度使趙暉至長安乙亥表王景崇反狀益明請進兵擊之 初高祖鎮河東皇弟崇為馬步都指揮使與蕃漢都孔目官郭威爭權有隙及威執政崇憂之節度判官鄭珙勸崇為自全計崇從之珙青州人也|{
	珙居竦翻}
八月庚辰崇表募兵四指揮自是選募勇士招納亡命繕甲兵實府庫罷上供財賦皆以備契丹為名朝廷詔令多不稟承|{
	於是之時劉崇則為跋扈然郭威既立天下為周河東非素有備殆不能守也}
自河中永興鳳翔三鎮拒命以來朝廷繼遣諸將討之昭義節度使常思屯潼關白文珂屯同州趙暉屯咸陽|{
	常思白文珂不敢逼河中趙暉不敢逼鳳翔}
惟郭從義王峻置柵近長安而二人相惡如水火|{
	近其靳翻惡如字又烏路翻}
自春徂秋皆相仗莫肯攻戰帝患之欲遣重臣臨督壬午以郭威為西面軍前招慰安撫使 |{
	考異曰薛史周大祖記七月十三日授同平章事即遣西征以安慰招撫為名八月六日發離京師按漢隱帝周太祖實録七月加平章事制詞無西征之言至八月壬午方受命出征蓋薛史之誤}
諸軍皆受威節度威將行問策於太師馮道道曰守貞自謂舊將為士卒所附願公勿愛官物以賜士卒則奪其所恃矣威從之|{
	郭威以卒伍之雄而問策於馮道之老腐者觀其所以荅與威所以從則人之材識不合乎道者則有之若其量勢應物未可妄議}
由是衆心始附於威|{
	為郭威得天下張本}
詔白文珂趣河中趙暉趣鳳翔|{
	趣七喻翻}
甲申蜀主以趙廷隱為太傅賜爵宋王國有大事就第問之 戊子蜀改鳳翔曰岐陽軍|{
	以鳳翔之地在岐山之陽也}
己丑以王景崇為岐陽節度使同平章事 乙未以錢弘俶為東南兵馬都元帥鎮海鎮東節度使兼中書令吳越國王 郭威與諸將議攻討諸將欲先取長安鳳翔鎮國節度使扈從珂曰今三叛連衡推守貞為主守貞亡則兩鎮自破矣若捨近而攻遠萬一王趙拒吾前守貞掎吾後|{
	掎居蟻翻}
此危道也威善之於是威自陜州白文珂及寧江節度使侍衛步軍都指揮使劉詞自同州常思自潼關三道攻河中|{
	九域志陜州北至河中二百三十七里同州東至河中六十里潼關度河至河中一百餘里陜失冉翻}
威撫養士卒與同苦樂小有功輒賞之微有傷常親視之士無賢不肖有所陳啟皆温辭色而受之違忤不怒小過不責|{
	樂音洛忤五故翻}
由是將卒咸歸心於威始李守貞以禁軍皆嘗在麾下受其恩施|{
	施式䜴翻下好施同}
又士卒素驕苦漢法之嚴謂其至則叩城奉迎可以坐而待之|{
	李守貞習見鳳翔太原之事以楊思權期漢兵耳}
既而士卒新受賜於郭威皆忘守貞舊恩己亥至城下揚旗伐鼓踊躍詬譟|{
	詬古侯翻又許侯翻}
守貞視之失色白文珂克西關城柵於河西|{
	河中西關城在河西所以護蒲津浮梁者也}
常思柵於城南威柵於城西|{
	常思郭威蓋近城立柵}
未幾威以常思無將領才先遣歸鎮|{
	幾居豈翻將即亮翻下同遣常思歸潞州史言郭威能審人之能否}
諸將欲急攻城威曰守貞前朝宿將健鬭好施|{
	施式䜴翻}
屢立戰功况城臨大河樓堞完固未易輕也|{
	易以䜴翻}
且彼馮城而鬭|{
	馮讀曰憑}
吾仰而攻之何異帥士卒投湯火乎|{
	帥讀曰率}
夫勇有盛衰攻有緩急時有可否事有後先不若且設長圍而守之使飛走路絶吾洗兵牧馬坐食轉輸|{
	輸舂遇翻}
温飽有餘俟城中無食公帑家財皆竭|{
	帑它郎翻}
然後進梯衝以逼之飛羽檄以招之彼之將士脱身逃死父子且不相保况烏合之衆乎思綰景崇但分兵縻之不足慮也|{
	縻忙皮翻繫也史言郭威方畧亦因周之史官潤色已成之文}
乃發諸州民夫二萬餘人使白文珂等帥之刳長壕築連城列隊伍而圍之威又謂諸將曰守貞曏畏高祖不敢鴟張|{
	曏謂昔時也鴟張言如鴟之張翼欲高舉遠飛也}
以我輩崛起太原事功未著有輕我心故敢反耳正宜靜以制之乃偃旗卧鼓但循河設火鋪連延數十里番步卒以守之遣水軍檥舟於岸寇有潜往來者無不擒之於是守貞如坐網中矣|{
	鋪普故翻檥魚倚翻番步卒者使步卒分番迭守張敬達之圍晉陽郭威之圍河中皆欲以持久制之然敬達以敗郭威以勝者晉陽有援而河中無援也司馬仲達急攻孟達而緩攻公孫淵亦以有援無援而為緩急耳}
蜀武德節度使兼中書令王處回請老辛丑以太子太傅致仕 南漢主|{
	中國既國號曰漢故嶺南之漢書南漢以别之}
遣知制誥宣化鍾允章|{
	宣化漢領方縣地晉置晉興郡隋廢郡置宣化縣及晉興縣唐以宣化為邕州治所晉興亦屬邕州}
求昏於楚楚王希廣不許南漢主怒問允章馬公復能經畧南土乎|{
	復扶又翻}
對曰馬氏兄弟方爭亡於不暇安能害我南漢主曰然希廣懦而吝嗇其士卒忘戰日久此乃吾進取之秋也|{
	為南漢舉兵攻楚張本}
武平節度使馬希萼請與楚王希廣各修職貢求朝廷别加官爵|{
	欲使潭朗如二國然}
希廣用天策府内都押牙歐弘練進奏官張仲荀謀厚賂執政使拒其請九月壬子賜希萼及楚王希廣詔書諭以兄弟宜相輯睦凡希萼所貢當附希廣以聞希萼不從 蜀兵援王景崇軍于散關趙暉遣都監李彦從襲擊破之 |{
	考異曰實録戊辰郭諱上言都監李彦從將兵掩襲州賊至大散關殺賊三千餘其餘弃甲而遁漢隱帝實録九月李彦從敗蜀兵于散關而蜀後主實録無之蜀實録十月安思謙敗漢兵於時家竹林遂焚蕩寶雞十二月又敗漢兵于玉女潭而漢實録無之蓋兩國各舉其勝而諱其敗耳然漢實録言官軍不滿萬人而蜀兵數倍是二三萬人非小役也豈得全不書殺三千人非小敗也豈十月遽能再舉蓋九月止是蜀邊將小出兵為漢所敗漢將因張大而奏之耳又蜀實録十月但云思謙退次鳳州不云歸興元十二月云思謙自興元進次鳳州蓋十月脱畧耳}
蜀兵遁去 蜀主以張業王處回執政事多壅蔽己未始置匭函|{
	匭居洧翻}
後改為獻納函 王景崇盡殺侯益家屬七十餘人|{
	怨侯益之毁已於朝也}
益子前天平行軍司馬仁矩先在外得免庚申以仁矩為隰州刺史仁矩子延廣尚在襁褓乳母劉氏以己子易之|{
	凡擇乳母必取新生子者許之攜子故得以易}
抱延廣而逃乞食至于大梁歸子益家 李守貞屢出兵欲突長圍皆敗而返遣人齎蠟丸求救于唐蜀契丹皆為邏者所獲|{
	邏郎佐翻}
城中食且盡殍死者日衆|{
	殍被表翻}
守貞憂形於色召摠倫詰之|{
	摠倫媚守貞見上三月詰去吉翻}
摠倫曰大王當為天子人不能奪但此分野有災|{
	分扶問翻}
待磨滅將盡只餘一人一騎乃大王鵲起之時也|{
	莊子曰鵲上高城乘危而巢於高枝之巔城壞巢折凌風而起故君子之居世也得時則義行失時則鵲起}
守貞猶以為然冬十月王景崇遣其子德讓趙思綰遣其子懷乂見蜀主于成都戊寅景崇遣兵出西門趙暉擊破之遂取西關城景崇退守大城塹而圍之數挑戰不出|{
	數所角翻挑徒了翻}
暉潜遣千餘人擐甲執兵|{
	擐音宦}
効蜀旗幟循南山而下|{
	幟昌志翻}
令諸軍聲言蜀兵至矣景崇果遣兵數千出迎之暉設伏掩擊盡殪之|{
	殪壹計翻}
自是景崇不復敢出|{
	復扶又翻}
蜀主遣山南西道節度使安思謙將兵救鳳翔左僕射兼門下侍郎同平章事母昭裔上疏諫曰臣竊見莊宗皇帝志貪西顧前蜀主意欲北行|{
	志貪西顧言後唐莊宗利蜀之富而伐之也前蜀主謂王衍意欲北行言其鋭意幸秦州也事並見莊宗紀上時掌翻}
凡在庭臣皆貢諫疏殊無聽納有何所成只此兩朝可為鑒誡不聽又遣雄武節度使韓保貞引兵出汧陽以分漢兵之勢|{
	汧陽縣屬隴州九域志在州東六十七里汧苦堅翻}
王景崇遣前義成節度使酸棗李彦舜等逆蜀兵|{
	酸棗古縣唐屬汴州九域志在州東北九十里}
丙申安思謙屯右界|{
	右界蓋寶雞西界漢蜀分彊處也}
漢兵屯寶雞思謙遣眉州刺史申貴將兵二千趣模壁|{
	趣七喻翻}
設伏於竹林丁酉旦貴以兵數百壓寶雞而陳|{
	陳讀曰陣}
漢兵逐之遇伏而敗蜀兵逐北破寶雞寨蜀兵去漢兵復入寶雞|{
	復扶又翻}
己亥思謙進屯渭水|{
	渭水過寶雞縣北}
漢益兵五千戍寶雞思謙畏之謂衆曰糧少敵強宜更為後圖辛丑退屯鳳州尋歸興元|{
	興元安思謙本鎮也}
貴潞州人也荆南節度使南平文獻王高從誨寢疾以其子節度副使保融判内外兵馬事癸卯從誨卒|{
	年五十八}
保融知留後|{
	保融從誨第三子史不言其得立之因}
彰武節度使高允權與定難節度使李彞殷有隙|{
	延州北至夏州三百八十里二鎮接境違言易生難乃旦翻}
李守貞密求援於彞殷發兵屯延丹境上聞官軍圍河中乃退甲辰允權以狀聞彞殷亦自訴朝廷和解之 初高祖入大梁太師馮道太子太傅李崧皆在真定|{
	事見上卷天福十二年}
高祖以道第賜蘇禹珪崧第賜蘇逢吉崧第中瘞藏之物及洛陽别業|{
	瘞於計翻别置田園於它所謂之别業亦謂之莊}
逢吉盡有之及崧歸朝自以形迹孤危|{
	石晉之時漢高祖夙有憾于李崧即位後崧始歸朝故内懼}
事漢權臣常惕惕謙謹多稱疾杜門而二弟嶼㠖|{
	嶼以與翻㠖宜崎翻}
與逢吉子弟俱為朝士時乘酒出怨言云奪我居第家貲逢吉由是惡之未幾|{
	惡烏路翻幾居豈翻}
崧以兩京宅劵獻於逢吉逢吉愈不悦翰林學士陶穀先為崧所引用復從而譛之|{
	復扶又翻}
漢法既嚴而侍衛都指揮使史弘肇尤殘忍寵任孔目官解暉|{
	解戶買翻姓也鄭樵姓氏畧曰自唐叔虞食邑于解今解縣也至春秋之時晉有解狐解揚}
凡入軍獄者使之隨意鍛鍊無不自誣及三叛連兵|{
	三叛謂李守貞王景崇趙思綰}
羣情震動民間或訛言相驚駭弘肇掌部禁兵廵邏京城|{
	部者部分之也邏郎佐翻}
得罪人不問輕重於法何如皆專殺不請或決口斮筋折脛無虚日雖姦盜屛跡而寃死者甚衆莫敢辯訴|{
	斷音短斮側畧翻拆而設翻脛戶定翻屏卑郢翻又卑正翻}
李嶼僕夫葛延遇為嶼販鬻多所欺匿嶼抶之|{
	抶丑栗翻}
督其負甚急延遇與蘇逢吉之僕李澄謀上變告嶼謀反|{
	孔子有言治家者不敢失於臣妾而况居昏暴之朝乎上時掌翻}
逢吉聞而誘致之|{
	誘音酉}
因召崧至第收送侍衛獄|{
	侍衛獄即侍衛司獄所謂軍獄也}
嶼自誣云與兄崧弟㠖甥王凝及家僮合二十人謀因山陵發引|{
	引羊晉翻}
縱火焚京城作亂又遣人以蠟書入河中城結李守貞又遣人召契丹兵及具獄上|{
	上時掌翻}
逢吉取筆改二十為五十字十一月甲寅下詔誅崧兄弟家屬及辭所連及者皆陳尸於市|{
	蘇逢吉取李崧之家貲又從而夷其家曾未朞年逢吉亦身死而家破天道不遠人猶冒貨而不顧可哀也哉}
仍厚賞葛延遇等時人無不寃之自是士民家皆畏憚僕隸往往為所脅制它日祕書郎真定李昉詣陶穀|{
	昉甫兩翻}
穀曰君於李侍中近遠昉曰族叔父穀曰李氏之禍穀有力焉昉聞之汗出穀邠州人也本姓唐避晉高祖諱改焉|{
	姓譜姓苑皆謂陶姓唐姓並出陶唐氏之後唐穀之改姓陶据此也}
史弘肇尤惡文士|{
	惡烏路翻}
常曰此屬輕人難耐每謂吾輩為卒|{
	此事亦誠有之但以此而例惡文士則過矣}
弘肇領歸德節度使委親吏楊乙收屬府公利乙依勢驕横|{
	史弘肇領宋州節而掌侍衛留京師使節度副使治府事副使其屬也故謂之屬府公利言公取所當得者横戶孟翻}
合境畏之如弘肇副使以下望風展敬乙皆下視之月率錢萬緡以輸弘肇士民不勝其苦|{
	史言史弘肇所謂公利其實皆虐民而取之輸舂遇翻勝音升}
初沈丘人舒元|{
	沈丘古寢丘也唐神龍二年改曰沈丘屬潁州九域志在州西一百一十里沈式荏翻}
嵩山道士楊訥俱以遊客干李守貞守貞為漢所攻遣元更姓朱訥更姓李名平間道奉表求救於唐|{
	朱元遂留為南唐用間古莧翻}
唐諫議大夫查文徽|{
	查鉏加翻}
兵部侍郎魏岑請出兵應之唐主命北面行營招討使李金全將兵救河中以清淮節度使劉彦貞副之|{
	唐置清淮軍于夀州}
文徽為監軍使岑為沿淮巡檢使軍于沂州之境金全與諸將方會食騎白有漢兵數百在澗北皆羸弱|{
	羸倫為翻}
請掩之金全令曰敢言過澗者斬|{
	過音戈}
及暮伏兵四起金鼓聞十餘里|{
	聞音問}
金全曰曏可與之戰乎時唐士卒厭兵莫有鬭志又河中道遠勢不相及丙寅唐兵退保海州|{
	是時沂州屬漢海州屬唐九域志沂州之界東南至海州一百里}
唐主遺帝書謝請復通商旅|{
	與中國絶和故商旅不通今遺書謝前過請復通商旅遺唯季翻復扶又翻}
且請赦守貞朝廷不報壬申葬睿文聖武昭肅孝皇帝于睿陵|{
	睿陵在河南府告成縣}


廟號高祖 十二月丁丑以高保融為荆南節度使同平章事 辛巳南漢主以内常侍吳懷恩為開府儀同三司西北面招討使將兵擊楚攻賀州楚王希廣遣決勝指揮使徐知新等將兵五千救之未至南漢人已拔賀州鑿大穽於城外覆以竹箔加上|{
	以竹箔覆穽而加土於竹箔之上穽才性翻覆敷又翻箔白各翻}
下施機軸自塹中穿宂通穽中知新等至引兵攻城南漢遣人自穴中發機楚兵悉陷南漢出兵從而擊之楚兵死者以千數知新等遁歸希廣斬之南漢兵復陷昭州|{
	復扶又翻九域志賀州西至昭州三百餘里}
王景崇累表告急于蜀蜀主命安思謙再出兵救之壬午思謙自興元引兵屯鳳州請先運糧四十萬斛乃可出境蜀主曰觀思謙之意安肯為朕進取|{
	為于偽翻}
然亦發興州興元米數萬斛以饋之戊子思謙進屯散關遣馬步使高彦儔眉州刺史申貴擊漢箭筈安都寨破之|{
	筈音括箭筈嶺名有箭筈關}
庚寅思謙敗漢兵於玉女潭|{
	敗補邁翻}
漢兵退屯寶雞思謙進屯模壁韓保貞出新關|{
	新關在隴州汧源縣西唐大中六年隴州防禦使薛逵徙築謂之安戎關汧隴之人謂大震為故關安戎為新關九域志隴州汧源縣有新關鎮}
壬辰軍于隴州神前漢兵不出保貞亦不敢進趙暉告急于郭威威自往赴之時李守貞遣副使周光遜禆將王繼勲聶知遇守城西|{
	聶尼輒翻姓也姓苑楚大夫食采於聶因以為氏}
威戒白文珂劉詞曰賊苟不能突圍終為我禽萬一得出則吾不得復留於此成敗之機於是乎在賊之驍鋭盡在城西我去必來突圍爾曹謹備之威至華州聞蜀兵食盡引去|{
	考異曰十國紀年蜀廣政十二年正月甲寅思謙以軍食匱竭自模壁退次鳳州上表待罪蓋去年冬末已}


|{
	退軍明年正月表始到成都耳今從周太祖實錄模壁一作模壁}
威乃還|{
	還從宣翻}
韓保貞聞安思謙去亦退保弓川寨|{
	九域志秦州東一百六十五里有弓門寨}
蜀中書侍郎兼禮部尚書同平章事徐光溥坐以豔辭挑前蜀安康長公主|{
	桃徒了翻長知兩翻}
丁酉罷守本官

隱皇帝上|{
	諱承祐高祖第二子也}


乾祐二年春正月乙巳朔大赦 郭威將至河中|{
	自華州還也}
白文珂出迎之戊申夜李守貞遣王繼勲等引精兵千餘人循河而南襲漢柵坎岸而登遂入之|{
	此漢兵柵於河西者也王繼勲知漢兵據河之西以臨河東守備必厚故循河而南坎岸而上以攻之}
縱火大譟軍中狼狽不知所為劉詞神色自若下令曰小盜不足驚也帥衆擊之|{
	帥讀曰率}
客省使閻晉卿曰|{
	梁有客省使副宋因之掌四方進奉及四夷朝貢牧伯朝覲賜酒饌甕餼宰相近臣禁軍將校節級諸州進奉使賜物回詔之事}
賊甲皆黄紙為火所照易辨耳|{
	易以䜴翻}
柰衆無鬬志何禆將李韜曰安有無事食君禄有急不死鬬者邪援矟先進|{
	援于元翻矟音槊}
衆從之河中兵退走死者七百人繼勲重傷僅以身免己酉郭威至劉詞迎馬首請罪威厚賞之曰吾所憂正在於此微兄健鬬|{
	微無也}
幾為虜嗤|{
	用漢光武語幾居依翻嗤丑之翻}
然虜伎殫於此矣|{
	伎渠綺翻}
晉卿忻州人也守貞之欲攻河西柵也先遣人出酤酒於村墅或貫與不責其直邏騎多醉|{
	酤音沽墅承與翻貫始制翻邏郎佐翻}
由是河中兵得潜行入寨幾至不守郭威乃下令將士非犒宴毋得私飲|{
	犒苦到翻}
愛將李審晨飲少酒|{
	少酒言所飲不多也少詩沼翻}
威怒曰汝為吾帳下首違軍令何以齊衆立斬以徇 甲寅蜀安思謙退屯鳳州上表待罪蜀主釋不問|{
	軍行逗撓者必誅釋而不問為失刑}
詔以靜州隸定難軍|{
	唐置静邉州都督於銀州界以處党項降者難乃旦翻}
二月辛未李彛殷上表謝彛殷以中原多故有輕傲之志每藩鎮有叛者常隂助之邀其重賂朝廷知其事亦以恩澤羈縻之|{
	史言拓拔據銀夏漸以驁桀遂成宋朝繼遷之叛}
淮北羣盜多請命於唐唐主遣神衛都虞侯皇甫暉等將兵萬人出海泗以招納之|{
	皇甫暉即與趙在禮作亂以成後唐莊宗之禍者也奔南唐見二百八十六卷高祖天福十二年海泗二州名}
蒙城鎮將咸師朗等降於暉|{
	蒙城隋之山桑縣唐天寶元年更名蒙城屬亳州九域志在州南一百六十里}
徐州將成德欽敗唐兵於峒峿鎮|{
	峒達貢翻又嵸董翻峿五乎翻}
俘斬六百級暉等引歸晉李太后詣契丹主請依漢人城寨之側給田以耕桑自贍契丹主許之并晉主遷於建州|{
	歐史曰自遼陽府東南行千二百里至建州今按建州在遼陽之西北其南則義州其北則土河土河之北則契丹之中京大定府大定府南至燕京一千一百五十里北至上京臨潢府七百里金人疆域圖建州南至燕京一千二百四十五里遼陽府治遼陽縣至燕京二千二百一十里薛史曰自遼陽行十數日過儀州霸州至建州陳元靚曰大元建州領建平永霸二縣屬大定府路}
未至安太妃卒於路遺令必焚我骨南向颺之|{
	颺余章翻}
庶幾魂魄歸達于漢|{
	白虎通曰魂者沄也沄沄行不休也魄者迫也迫迫然著於人也}
既至建州得田五十餘頃晉主令從者耕其中以給食|{
	從才用翻}
頃之舒嚕王遣騎取晉主寵姬趙氏聶氏而去舒嚕王者契丹主德光之子也 三月己未以歸德牙内指揮使史德珫領忠州刺史|{
	珫昌中翻忠州時屬蜀}
德珫弘肇之子也頗讀書常不樂父之所為|{
	樂音洛}
有舉人呼譟於貢院門蘇逢吉命執送侍衛司欲其痛箠而黥之|{
	呼火故翻箠止橤翻貢院門禮部貢院門也五季自梁以來雖皆右武之時而諸州取解禮部試進士未嘗廢唐明宗天成二年勑新及第進士有聞喜宴今後逐年賜錢四百貫其進士試詩賦文策帖經對義蓋朝廷猶重科舉之士故史德珫雖將家子亦愛護士流}
德珫言於父曰書生無禮自有臺府治之非軍務也此乃公卿欲彰大人之過耳|{
	謂蘇逢吉知史弘肇不喜書假手以逞若堕其術是自彰已過治直之翻}
弘肇大然之即破械遣之 楚將徐進敗蠻於風陽山斬首五千級|{
	敗補邁翻}
夏五月壬午太白晝見|{
	見賢遍翻}
民有仰視之者為邏卒所執|{
	邏郎佐翻}
史弘肇腰斬之 河中城中食且盡民餓死者什五六癸卯李守貞出兵五千餘人齎梯橋分五道以攻長圍之西北隅郭威遣都監吳䖍裕引兵横擊之河中兵敗走殺傷大半奪其攻具五月丙午守貞復出兵又敗之|{
	復扶又翻}
擒其將魏延朗鄭賓壬子周光遜王繼勲聶知遇帥其衆千餘人來降|{
	周光遜王繼勲李守貞之驍將也帥讀曰率}
守貞將士降者相繼威乘其離散庚申督諸軍百道攻之|{
	此司馬文王取諸葛誕之故智}
趙思綰好食人肝嘗面剖而膾之|{
	好呼到翻按禮記内則聶而細切之者為膾盜跖膾人肝而餔之莊子寓言耳豈知後世真有趙思綰者乎}
膾盡人猶未死又好以酒吞人膽謂人曰吞此千枚則膽無敵矣及長安城中食盡取婦女幼稚為軍糧|{
	椎直利翻}
日計數而給之每犒軍輒屠數百人如羊豕法思綰計窮不知所出郭從義使人誘之初思綰少時|{
	犒苦到翻誘音酉少詩照翻}
求為左驍衛上將軍致仕李肅僕肅不納曰是人目亂而語誕|{
	誕徒旱翻大言謂之誕}
它日必為叛臣肅妻張氏全義之女也|{
	張全義鎮洛著功名于梁唐之間}
曰君今拒之後且為患乃厚以金帛遺之|{
	遺唯季翻}
及思綰據長安肅閒居在城中思綰數就見之拜伏如故禮|{
	天福十二年趙在禮自長安朝契丹其禆將留長安者作亂李肅討誅之是其威望必重趙思綰又懷其疇昔之惠故雖竊據其見肅也猶如奴事主之禮數所角翻}
肅曰是子亟來且汙我欲自殺|{
	亟去吏翻汚烏故翻}
妻曰曷若勸之歸國|{
	史言李肅之妻有智}
會思綰問自全之計肅乃與判官程讓能說思綰曰|{
	說式芮翻}
公本與國家無嫌但懼罪耳今國家三道用兵俱未有功|{
	三道用兵謂郭威攻河中趙暉攻鳳翔郭從義攻思綰也}
若以此時翻然改圖朝廷必喜自可不失富貴孰與坐而待斃乎思綰從之遣使詣闕請降乙丑以思綰為華州留後|{
	以為鎮國軍留後}
都指揮使常彦卿為虢州刺史令便道之官|{
	不使入朝所以安其反側之心}
吳越内牙都指揮使鈄淊胡進思之黨也 |{
	考異曰吳越備史}


|{
	十國紀年鈄姓皆金旁斗按何氏姓苑元和姓纂皆無此姓今按字書鈄音他口徒口二切皆云姓也余又按廣韻云鈄姓出姓苑也}
或告其謀叛辭連丞相弘億吳越王弘俶不欲窮治貶滔於處州|{
	治直之翻}
六月癸酉朔日有食之秋七月甲辰趙思綰釋甲出城受詔郭從義以兵守

其南門復遣還城|{
	復扶又翻還從宣翻又如字}
思綰求其牙兵及鎧仗從義亦給之思綰遷延收斂財賄三改行期從義等疑之密白郭威請圖之威許之壬子從義與都監南院宣徽使王峻|{
	當作宣徽南院使}
按轡入城處于府舍|{
	處昌呂翻}
召思綰酌别因執之并常彦卿及其父兄部曲三百人皆斬於市 甲寅郭威攻河中克其外郭李守貞收餘衆退保子城諸將請急攻之威曰夫鳥窮則啄况一軍乎涸水取魚安用急為壬戌李守貞與妻及子崇勲等自焚威入城獲其子崇玉等及所署丞相靖孫愿樞密使劉苪國師總倫等送大梁磔於市|{
	靖姓也其名同都翻與嵞同說文禹會諸侯於塗山之塗作嵞磔音竹格翻}
徵趙修已為翰林天文|{
	以趙修己數諫李守貞也盛唐有天文博士天文生皆屬司天監其待詔於翰林院者曰翰林天文}
威閲守貞文書得朝廷權臣乃藩鎮與守貞交通書詞意悖逆欲奏之|{
	悖蒲妹翻又蒲沒翻}
祕書郎榆次王溥諫曰魑魅乘夜爭出見日自消|{
	魑丑知翻魅明祕翻魑魅野鬼山精之屬}
願一切焚之以安反側威從之|{
	王漙之進用於周由此言也郭威西征于外則得李穀王溥於内則得范質此豈一時倔彊武人之所能及哉}
三叛既平|{
	是時鳳翔猶未平也因帝驕縱而槩言之}
帝浸驕縱與左右狎暱|{
	暱尼質翻}
飛龍使瑕丘後匡贊|{
	後讀如字姓也鄭樵氏族畧云後姓望出東海開封有此姓}
茶酒使太原郭允明以謟媚得幸帝好與之為廋辭醜語|{
	廋辭隱語也好呼到翻廋所鳩翻}
太后屢戒之帝不以為意癸亥太常卿張昭上言宜親近儒臣講習經訓不聽|{
	近其靳翻}
昭即昭遠避高祖諱改之 戊辰加永興節度使郭從義同平章事徙鎮國節度使扈從珂為護國節度使以河中行營馬步都虞侯劉詞為鎮國節度使 唐主復進用魏岑|{
	魏岑以罪黜見二百八十六卷高祖天福十二年唐主之保大五年也}
吏部郎中會稽鍾謨尚書員外郎李德明始以辯慧得幸參預國政|{
	會古外翻}
二人皆恃恩輕躁雖不與岑為黨而國人皆惡之戶部員外郎范沖敏性狷介|{
	惡烏路翻狷吉掾翻}
乃教天威都虞侯王建封上書歷詆用事者請進用正人唐主謂建封武臣典兵不當干預國政大怒流建封於池州未至殺之沖敏弃市唐主聞河中破以朱元為駕部員外郎待詔文理院李平為尚書員外郎|{
	李守貞遣朱元李平至唐見去年十一月文理院南唐所置尚書員外郎無曹局蓋於二十四司郎員外置也}
吳越王弘俶以丞相弘億判明州|{
	以鈄淊事出弘億}
西京留守同平章事王守恩性貪鄙專事聚斂|{
	斂力贍翻}
喪車非輸錢不得出城下至抒厠行乞之人不免課率|{
	抒叙呂翻抒厠取人家虎子寫去穢惡渫水洗之者也}
或縱麾下令盜人財有富室娶婦守恩與俳優數人往為賓客得銀數鋌而返|{
	賓一作賀鋌徒鼎翻}
八月甲申郭威自河中還過洛陽守恩自恃位兼將相|{
	留守節度使同平章事所謂位兼將相也}
肩輿出迎威怒以為慢已辭以浴不見即以頭子命保義節度使同平章事白文珂代守恩為留守|{
	沈括曰後唐莊宗復樞密使郭崇韜安重誨相繼為之始分領政事不關由中書直行下者謂之宣如中書之勅小事則發頭子擬堂帖也}
文珂不敢違守恩猶坐客次|{
	客次猶今言客位也坐于客次以俟見}
吏白新留守已視事於府矣守恩大驚狼狽而歸見家屬數百已逐出府在通衢矣朝廷不之問以文珂兼侍中充西京留守

歐陽修論曰自古亂亡之國必先壞其法制|{
	壞音怪}
而後亂從之此勢之然也五代之際是已文珂守恩皆漢大臣而周太祖以一樞密使頭子而易置之如更戍卒|{
	更工衡翻}
是時太祖未有無君之志而所為如此者蓋習為常事故文珂不敢違守恩不敢拒太祖既處之不疑|{
	處昌呂翻}
而漢廷君臣亦置而不問豈非綱紀壞亂之極而至於此歟|{
	余按唐閔帝之初朱弘昭馮贇以樞密院宣易置諸鎮以致潞王之亂雖成敗不同而樞密權重則有自來矣}
是以善為天下慮者不敢忽於微而常杜其漸也可不戒哉

守恩至大梁恐獲罪廣為貢獻重賂權貴朝廷亦以守恩首舉潞州歸漢|{
	事見二百八十六卷天福十二年}
故宥之但誅其用事者數人而已 馬希萼悉調朗州丁壯為鄉兵|{
	調徒釣翻}
造號靜江軍|{
	造號言創立軍號也}
作戰艦七百艘|{
	艦戶黯翻艘蘇遭翻}
將攻潭州其妻苑氏諫曰|{
	姓苑商武丁子子文受封於苑因以為氏左傳齊有大夫苑何忌趙明城金石録有漢荆州從事苑鎮碑曰其先出苑柏何為晉樂正世掌朝禮之制又有苑子園寔能掌隂陽之理按姓氏志皆云苑氏出苑何忌之後今此碑所謂苑柏何與子園左傳國語皆無其人故録之以傳知者}
兄弟相攻勝負皆為人笑不聽引兵趣長沙|{
	趣七喻翻}
馬希廣聞之曰朗州吾兄也不可與爭當以國讓之而已劉彦瑫李弘臯固爭以為不可乃以岳州刺史王贇為都部署戰棹指揮使以彦瑫監其軍己丑大破希萼於僕射洲獲其戰艦三百艘贇追希萼將及之希廣遣使召之曰勿傷吾兄贇引兵還贇環之子也|{
	還從宣翻又如字王環馬氏之良將也}
希萼自赤沙湖乘輕舟遁歸|{
	赤沙湖在洞庭湖西與洞庭湖通水經注云澧水與赤沙湖水會湖水北通江而南注澧}
苑氏泣曰禍將至矣余不忍見也赴井而死 戊戌郭威至大梁入見帝勞之|{
	見賢遍翻勞力到翻}
賜金帛衣服玉帶鞍馬辭曰臣受命期年|{
	去年七月命郭威西征至是踰朞矣期讀曰朞}
僅克一城何功之有且臣將兵在外凡鎮安京師供億所須使兵食不乏皆諸大臣居中者之力也臣安敢獨膺此賜請徧賞之又議加方鎮辭曰楊邠位在臣上未有茅王|{
	時楊邠為樞密使位在郭威上未嘗領節鎮}
且帷幄之臣不可以弘肇為比|{
	郭威自言職居近密乃帷幄之臣史弘肇掌侍衛兵所以領節不可以為比}
九月壬寅徧賜宰相樞密宣徽三司侍衛使九人與威如一|{
	時宰相三人竇貞固蘇逢吉蘇禹珪樞密使楊邠宣徽使王峻吳䖍裕三司使王章侍衛使史弘肇凡八人餘一人則未之知也或者併郭威為九人歟}
帝欲特賞威辭曰運籌建畫出於廟堂發兵饋糧資於藩鎮暴露戰鬭在於將士而功獨歸臣臣何以堪之乙巳加威兼侍中史弘肇兼中書令辛亥加竇貞固司徒蘇逢吉司空蘇禹珪左僕射楊邠右僕射諸大臣議以朝廷執政溥加恩恐藩鎮觖望|{
	觖窺瑞翻又古宂翻怨望也}
乙卯加天雄節度使高行周守太師山南東道節度使安審琦守太傅泰寧節度使符彦卿守太保河東節度使劉崇兼中書令己未加忠武節度使劉信天平節度使慕容彦超平盧節度使劉銖並兼侍中辛酉加朔方節度使馮暉定難節度使李彞殷|{
	難乃旦翻}
兼中書令冬十月壬申加義武節度使孫方簡武寧節度使劉贇同平章事壬午加吳越王弘俶尚書令楚王希廣太尉丙戌加荆南節度使高保融兼侍中議者以為郭威不專有其功推以分人|{
	推吐雷翻又如字}
信為美矣而國家爵位以一人立功而覃及天下|{
	覃徒含翻布也廣也}
不亦濫乎 吳越王弘俶募民能墾荒田者勿收其税由是境内無弃田或請糾民遺丁以增賦|{
	遺丁謂民年已成丁而戶籍遺漏未嘗當賦役者}
仍自掌其事弘俶杖之國門國人皆悦楚靜江節度使馬希瞻以兄希萼希廣交爭屢遣使諫止不從知終覆族疽發于背丁亥卒 契丹寇河北所過殺掠節度使刺史各嬰城自守遊騎至貝州及鄴都之北境|{
	按九域志貝州之南三十里即鄴都北界}
帝憂之己丑遣樞密使郭威督諸將禦之以宣徽使王峻監其軍|{
	為王峻佐郭威舉兵向闕張本}
十一月契丹聞漢兵渡河乃引去辛亥郭威軍至鄴都令王峻分軍趣鎮定|{
	趣七喻翻}
戊午威至邢州 唐兵度淮攻正陽|{
	九域志潁州潁上縣有正陽鎮臨淮津}
十二月潁州將白福進擊敗之|{
	敗補邁翻}
楊邠為政苛細初邢州人周璨為諸衛將軍罷秩無依從王景崇西征景崇叛遂為之謀主邠奏諸前資官喜揺動藩臣|{
	前資官謂官資皆前朝所授者也喜許記翻}
宜悉遣詣京師既而四方雲集日遮宰相馬求官辛卯邠復奏前資官宜分居兩京以俟有闕而補之漂泊失所者甚衆|{
	復扶又翻}
邠又奏行道往來者皆給過所|{
	盛唐之制天下關二十六度關者從司門郎中給過所猶漢時度關用傳也宋白曰古書之帛為繻刻木為契二物通謂過所也}
既而官司填咽民情大擾乃止 趙暉急攻鳳翔周璨謂王景崇曰公曏與蒲雍相表裏|{
	蒲謂李守貞雍謂趙思綰雍於用翻}
今二鎮已平蜀兒不足恃|{
	王景崇求援於蜀而蜀兵不至故言不足恃}
不如降也景崇曰善吾更思之後數日外攻轉急景崇謂其黨曰事窮矣吾欲為急計乃謂其將公孫輦張思練曰趙暉精兵多在城北來日五鼓前爾二人燒城東門詐降勿令寇入吾與周璨以牙兵出北門突暉軍縱無成而死猶勝束手皆曰善癸巳未明輦思練燒東門請降府牙火亦發二將遣人詗之|{
	詗古永翻又翾正翻}
景崇已與家人自焚矣璨亦降 丁酉密州刺史王萬敢擊唐海州荻水鎮殘之|{
	金人疆域圖荻水鎮在海州贑榆縣}
是月南漢主如英州|{
	南漢以唐廣州湞陽縣之地置英州九域志廣州北至英州四百二十里}
是歲唐泉州刺史留從効兄南州副使從願酖刺史董思安而代之|{
	晉齊王開運二年唐改漳州為南州以董思安為刺史唐之保大三年也事見二百八十四卷}
唐主不能制置清源軍於泉州以從効為節度使

資治通鑑卷二百八十八
