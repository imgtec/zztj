資治通鑑卷二百十九  宋 司馬光 撰

胡三省 音註

唐紀三十五|{
	起柔兆涒灘十月盡彊圍作噩閏月不滿一年}


肅宗文明武德大聖大宣孝皇帝中之上

至德元載冬十月辛巳朔日有食之既 上發順化|{
	宋白曰慶州貞觀以來為弘化郡天寶後為安化郡至德為順化郡}
癸未至彭原 初李林甫為相諫官言事皆先白宰相退則又以所言白之御史言事須大夫同署至是敕盡革其弊開諫諍之塗又令宰相分直政事筆承旨旬日而更|{
	令宰相在政事堂分日當筆及承上旨更工衡翻}
懲林甫及楊國忠之專權故也 第五琦見上於彭原請以江淮租庸市輕貨泝江漢而上至洋川|{
	見賢遍翻上時掌翻洋川郡洋州本音羊今人多讀如祥}
令漢中王瑀陸運至扶風以助軍 |{
	考異曰鄴侯家傳云薦元載令于鄖鄉縣置院以督運按載傳是時在蘇州及洪州未嘗在鄖鄉今不取}
上從之尋加琦山南等五道度支使|{
	度支使始此宋白曰故事度支案郎中判入員外判出侍郎總統押案而已官衘不言專判度支開元已後時事多故遂有他官來判者乃曰度支使或曰判度支或曰知度支事或曰勾當度支使雖名稱不同其事一也度徒洛翻}
琦作榷鹽法用以饒|{
	琦變鹽法盡榷天下鹽就山海井竈置鹽院使吏出糶舊業鹽戶併游民願業者為亭戶免其雜徭盜煮私市者論以法百姓除租庸外無得横賦人不益税而上用以饒榷古岳翻}
房琯喜賓客|{
	喜許記翻}
好談論|{
	好呼到翻}
多引拔知名之士而輕鄙庸俗人多怨之北海太守賀蘭進明詣行在上命琯以為南海太守兼御史大夫充嶺南節度使|{
	南海郡廣州是時兵興方鎮重任必兼臺省長官以至外府僚佐亦帶朝衘迄于五季遂為永制其帶臺衘自監察御史至御史大夫為憲衘守手又翻}
琯以為攝御史大夫進明入謝上怪之進明因言與琯有隙且曰晉用王衍為三公祖尚浮虚致中原板蕩|{
	王衍事見晉紀板蕩之詩刺周室大壞天下無綱紀文章之詩也後人率引此二詩以諭天下大亂毛氏傳曰板板反也正義曰釋訓云板板僻也邪僻即反戾之義故為反也鄭曰蕩蕩法度廢壞之貌}
今房琯專為迃闊大言以立虚名所引用皆浮華之黨真王衍之比也陛下用為宰相恐非社稷之福且琯在南朝佐上皇使陛下與諸王分領諸道節制|{
	事見上卷上即位於靈武進駐彭原其地在關山之北上皇在成都其地在關山之南故謂之南朝}
仍置陛下於沙塞空虚之地又布私黨於諸道使統大權|{
	蓋指李峴李承式鄧景山等}
其意以為上皇一子得天下則己不失富貴此豈忠臣所為乎上由是疎之 房琯上疏請自將兵復兩京上許之 |{
	考異曰唐歷上以房琯有重名虚己以待之禮遇加等琯推誠謇諤亦以天下為己任知無不為其所引進皆一時名士其嫉惡太甚雅有宰相望其於彌綸天下非所長也後頗以直忤旨上以名高隱忍漸不能容矣琯遂請兵為元帥許之今從實錄據考異則上之疎琯非特因進明之言也}
加持節招討西京兼防禦蒲潼兩關兵馬節度等使琯請自選參佐以御史中丞鄧景山為副戶部侍郎李揖為行軍司馬給事中劉秩為參謀既行又令兵部尚書王思禮副之琯悉以戎務委李揖劉秩二人皆書生不閑軍旅|{
	閑習也}
琯謂人曰賊曳落河雖多安能敵我劉秩琯分為三軍使禆將楊希文將南軍自宜夀入|{
	天寶元年更盩厔縣曰宜夀屬鳳翔郡}
劉貴哲將中軍自武功入李光進將北軍自奉天入光進光弼之弟也 以賀蘭進明為河南節度使 頴王璬之至成都也|{
	見上卷璬公了翻}
崔圓迎謁拜於馬首璬不之止圓恨之璬視事兩月吏民安之圓奏罷璬使歸内宅|{
	京師有十宅以處諸王未出閤者此時在成都亦即行宫為内宅}
以武部侍郎李峘為劒南節度使代之|{
	峘胡登翻 考異曰肅宗實錄明年正月甲寅以峘為劒南節度使蓋峘已受上皇命而肅宗申命之也}
峘峴之兄也上皇尋命璬與陳王珪詣上宣慰至是見上於彭原延王玢從上皇入蜀追車駕不及上皇怒欲誅之漢中王瑀救之乃命玢亦詣上所|{
	玢音彬 考異曰明皇雜錄賀蘭進明之初守北海也城卑不完儲積于外寇又將至懼資其用進明遂焚之適有寺人至北海求貨于進明不獲歸以損軍用聞於上遂詔罷郡守屬延王玢從上不及遣中使訪之而加刑焉會進明赴蜀遇使訪于路曰王罪不宜及刑願少留於路使者感而受約既至蜀進明言于上曰延王陛下之愛子也無兵權以變其心無郡國以驕其志間道于豺狼乃責其不以時至陛下罪之人復何望臣恐漢武望思之築將見於聖朝矣因遽馳使赦之謂進明曰俾父子如初卿之力也遂遣進明往靈武道遇延王進明馳馬亦慰之王望之降車稽首而去肅宗謂之曰卿解平原之圍阻賊寇之軍而不以讒口介意復全我兄弟乃社稷之臣因授御史大夫今從舊傳}
甲申令狐潮王福德復將步騎萬餘攻雍丘|{
	復扶又翻}
張巡出擊大破之斬首數千級賊遁去 房琯以中軍北軍為前鋒庚子至便橋辛丑二軍遇賊將安守忠於咸陽之陳濤斜|{
	陳濤澤在咸陽縣東其路斜出故曰陳濤斜又宋敏求退朝錄引唐人文集曰唐宫人墓謂之宫人斜四仲遣使者祭之然則陳濤斜者豈亦因内人所葬地而名之邪}
琯效古法用車戰以牛車二千乘馬步夾之賊順風皷譟牛皆震駭賊縱火焚之人畜大亂|{
	乘䋲證翻畜許救翻}
官軍死傷者四萬餘人存者數千而己癸卯琯自以南軍戰又敗|{
	南軍宜夀之軍也}
楊希文劉貴哲皆降於賊上聞琯敗大怒李泌為之營救|{
	為于偽翻}
上乃宥之待琯如初以薛景仙為關内節度副使 敦煌王承寀至囘紇牙帳|{
	承寀使囘紇見上卷九月敦徒門翻}
囘紇可汗以女妻之|{
	妻七細翻}
遣其貴臣與承宷及僕固懷恩皆來見上於彭原|{
	見賢遍翻}
上厚禮其使者而歸之賜囘紇女號毗伽公主|{
	伽求迦翻}
尹子奇圍河間四十餘日不下史思明引兵會之顔真卿遣其將和琳將萬二千人救河間思明逆擊擒之遂陷河間執李奐送洛陽殺之又陷景城太守李暐赴湛水死|{
	新書作赴河死}
思明使兩騎齎尺書以招樂安樂安即時舉郡降|{
	樂安郡棣州景城既陷樂安孤絶即時降賊蓋人心危懼城主不能守也}
又使其將康没野波將先鋒攻平原兵未至顔真卿知力不敵壬寅棄郡渡河南走思明即以平原兵攻清河博平皆陷之|{
	清河郡貝州博平郡博州 考異曰河洛春秋云蔡希德引兵攻貝州貝州陷攻博州五日城陷今從肅宗實録}
思明引兵圍烏承恩於信都承恩降親導思明入城交兵馬倉庫馬三千匹兵五萬人|{
	信都郡冀州降戶江翻史言烏承恩兵力足以拒守}
思明送承恩詣洛陽祿山復其官爵饒陽禆將束鹿張興力舉千鈞性復明辨|{
	將即亮翻束鹿縣屬饒陽郡本鹿城縣天寶十五載更名劉昫曰束鹿漢安定侯國今縣西七里故城是也齊周為安定縣隋改曰鹿城明皇以安祿山反改常山之鹿泉曰獲鹿饒陽之鹿城曰束鹿以厭之復扶又翻}
賊攻饒陽彌年不能下|{
	饒陽受攻事始二百十七卷天寶十四載 考異曰此事出河洛春秋前云賊攻深州經月不下後云興戰守彌年而城池轉固蓋前云經月者今次攻城也後云彌年者并計前後之數也}
及諸郡皆陷思明并力圍之外救俱絶太守李系窘迫赴火死|{
	守式又翻窘渠隕翻}
城遂陷思明擒興立於馬前謂曰將軍真壯士能與我共富貴乎興曰興唐之忠臣固無降理今數刻之人耳|{
	張興志在必死自言命在晷刻}
願一言而死思明曰試言之興曰主上待祿山恩如父子羣臣莫及不知報德乃興兵指闕塗炭生人大丈夫不能翦除凶逆乃北面為之臣乎僕有短策足下能聽之乎足下所以從賊求富貴耳譬如鷰巢于幕|{
	引左傳吳季札之言}
豈能久安何如乘間取賊|{
	間古莧翻}
轉禍為福長享富貴不亦美乎思明怒命張於木上鋸殺之詈不絶口以至於死|{
	如史所云則河北二十四郡惟張興可以言義士耳}
賊每破一城城中衣服財賄婦人皆為所掠男子壯者使之負擔|{
	擔都濫翻}
羸病老幼皆以刀槊戲殺之禄山初以卒三千人授思明使定河北至是河北皆下之|{
	按史思明與郭李相持於常山博陵祿山蓋屢益其兵及郭李入井陘思明乃能下河北此蓋逆黨稱其才而史不削耳}
郡置防兵三千雜以胡兵鎮之思明還博陵尹子奇將五千騎度河略北海欲南取江淮會囘紇可汗遣其臣葛邏支將兵入援|{
	邏郎佐翻}
先以二千騎奄至范陽城下子奇聞之遽引兵歸 十二月戊午囘紇至帶汗谷|{
	新書作呼延谷蓋語轉耳汗音寒}
與郭子儀軍合辛酉與同羅及叛胡戰於榆林河北|{
	榆林郡勝州大河經其北}
大破之斬首三萬捕虜一萬河曲皆平子儀還軍洛交|{
	洛交郡本鄜州上郡天寶元年更郡名}
上命崔渙宣慰江南兼知選舉 令狐潮帥衆萬餘營雍丘城北|{
	帥讀曰率}
張巡邀擊大破之賊遂走 永王璘幼失母|{
	璘郭順儀之子也順儀早死}
為上所鞠養常抱之以眠從上皇入蜀上皇命諸子分總天下節制|{
	事見上卷七月}
諫議大夫高適諫以為不可上皇不聽璘領四道節度都使鎮江陵時江淮租賦山積於江陵璘召募勇士數萬人日費巨萬璘生長深宫不更人事子襄城王瑒有勇力好兵有薛鏐等為之謀主|{
	長知兩翻更工衡翻瑒徒杏翻又音暢好呼到翻鏐力求翻}
以為今天下大亂惟南方完富璘握四道兵封疆數千里宜據金陵|{
	康曰楚威王埋金以鎮王氣故曰金陵}
保有江表如東晉故事上聞之敕璘歸覲于蜀璘不從江陵長史李峴辭疾赴行在|{
	璘將稱兵峴不欲預其禍也}
上召高適與之謀適陳江東利害且言璘必敗之狀十二月置淮南節度使領廣陵等十二郡以適為之置淮南西道節度使領汝南等五郡以來鎮為之|{
	淮南節度使領揚州廣陵郡楚州山陽郡滁州全椒郡和州歷陽郡夀州淮南郡廬州合肥郡舒州同安郡光州弋陽郡鄿州鄿春郡安州安陸郡黄州齊安郡申州義陽郡沔州漢陽郡凡十二淮南西道節度使領蔡州汝南郡鄭州滎陽郡許州潁川郡光州弋陽郡申州義陽郡已上皆據新書方鎮表但義陽弋陽已屬淮南節度當考}
使與江東節度使韋陟共圖璘|{
	方鎮表至德二載置江東防禦使治杭州蓋謂浙江之東也韋陟所節度者蓋江南東道也其巡屬兼有浙東西及昇宣歙諸州}
安禄山遣兵攻頴川城中兵少無蓄積太守薛愿長史龐堅悉力拒守繞城百里廬舍林木皆盡朞年救兵不至祿山使阿史那承慶益兵攻之晝夜死鬭十五日城陷執愿堅送洛陽祿山縛於洛濱木上凍殺之 上問李泌曰今敵彊如此何時可定對曰臣觀賊所獲子女金帛皆輸之范陽|{
	輸舂遇翻}
此豈有雄據四海之志邪|{
	邪音耶}
今獨虜將或為之用中國之人惟高尚等數人自餘皆脅從耳以臣料之不過二年天下無寇矣上曰何故對曰賊之驍將不過史思明安守忠田乾真張忠志阿史那承慶等數人而已|{
	將即亮翻驍堅堯翻過古禾翻又古卧翻張忠志即安忠志此時已復舊養父之姓}
今若令李光弼自太原出井陘郭子儀自馮翊入河東則思明忠志不敢離范陽常山守忠乾真不敢離長安|{
	令力丁翻陘音刑離力智翻}
是以兩軍縶其四將也從禄山者獨承慶耳願敕子儀勿取華陰|{
	華戶化翻}
使兩京之道常通陛下以所徵之兵軍於扶風與子儀光弼互出擊之彼救首則擊其尾救尾則擊其首使賊往來數千里疲於奔命我常以逸待勞賊至則避其鋒去則乘其弊不攻城不遏路來春復命建寜為范陽節度大使並塞北出|{
	復扶又翻又音如字使疏吏翻並步浪翻}
與光弼南北掎角以取范陽|{
	泌欲使建寜自靈夏並豐勝雲朔之塞直擣媯檀攻范陽之北光弼自太原取恒定以攻范陽之南}
覆其巢穴賊退則無所歸留則不獲安然後大軍四合而攻之必成擒矣|{
	使肅宗用泌策史思明豈能再為關洛之患乎}
上悦時張良娣與李輔國相表裏皆惡泌建寜王倓謂泌曰先生舉倓於上得展臣子之效無以報德請為先生除害|{
	娣大計翻惡烏路翻為于偽翻倓徒甘翻}
泌曰何也倓以良娣為言泌曰此非人子所言願王姑置之勿以為先倓不從 甲辰永王璘擅引兵東巡沿江而下軍容甚盛然猶未露割據之謀吳郡太守兼江南東路采訪使李希言平牒璘詰其擅引兵東下之意|{
	璘離珍翻守式又翻詰去吉翻使疏吏翻方鎮位任等夷者平牒}
璘怒分兵遣其將渾惟明襲希言於吳郡|{
	將即亮翻吳郡蘇州}
季廣琛襲廣陵長史淮南采訪使李成式於廣陵|{
	琛丑林翻廣陵郡揚州長知兩翻}
璘進至當塗希言遣其將元景曜及丹徒太守閻敬之將兵拒之|{
	今之當塗本漢丹陽縣地晉分丹楊置于湖縣成帝以江北當塗縣流人寓居于湖乃改為當塗縣仍僑置淮南郡隋廢淮南郡以縣屬丹楊郡唐屬宣城郡丹徒縣帶潤州丹楊郡唐未嘗以丹徒名郡徒當作楊守式又翻}
李成式亦遣其將李承慶拒之璘擊斬敬之以徇景曜承慶皆降於璘江淮大震高適與來瑱韋陟會於安陸結盟誓衆以討之|{
	韋陟蓋赴鎮中道聞變遂會於安陸降戶江翻瑱它甸翻}
于闐王勝聞安禄山反命其弟曜攝國事自將兵五千入援|{
	闐徒賢翻又徒見翻}
上嘉之拜特進兼殿中監 令狐潮李庭望攻雍丘數月不下乃置杞州築城於雍丘之北|{
	令力丁翻雍丘唐初置州貞觀元年廢賊復置之築城以逼雍丘}
以絶其糧援賊常數萬人而張巡衆纔千餘每戰輒克河南節度使虢王巨屯彭城假巡先鋒使是月魯東平濟陰陷于賊|{
	彭城郡徐州魯郡兖州東平郡鄆州濟子禮翻}
賊將楊朝宗帥馬步二萬將襲寜陵斷巡後|{
	斷丁管翻}
巡遂拔雍丘東守寜陵以待之|{
	帥讀曰率范成大北使錄雍丘百二十里至寜陵}
始與睢陽太守許遠相見是日楊朝宗至寜陵城西北巡遠與戰晝夜數十合大破之斬首萬餘級流尸塞汴而下|{
	睢音雖守式又翻塞悉則翻}
賊收兵夜遁敕以巡為河南節度副使巡以將士有功遣使詣虢王巨請空名告身及賜物巨唯與折衝果毅告身三十通不與賜物巡移書責巨巨竟不應|{
	使疏吏翻將即亮翻折之舌翻}
是歲置北海節度使領北海等四郡|{
	領青州北海郡密州高密郡登州東牟郡萊州東萊郡}
上黨節度使領上黨等三郡|{
	領潞州上黨郡澤州長平郡沁州陽城郡}
興平節度使領上洛等四郡|{
	領商州上洛郡金州安康郡岐州鳳翔郡方鎮表止著三郡餘一郡當考鳳翔郡縣東原先有興平軍因置為節鎮}
吐蕃陷威戎神威定戎宣威制勝金天天成等軍石堡城百谷城雕窠城|{
	定戎軍在石堡城北隔澗七里廓州西南百四十里有洪濟橋金天軍其東南八十里有百谷城河州西八十里索恭川有天成軍西百餘里有雕窠城皆天寶十三載置}
初林邑王范真龍為其臣摩訶漫多伽獨所殺盡滅范氏|{
	據新書此事在貞觀十九年通鑑因其改國號環王書之以始事范氏己自晉以來世有林邑至是而滅}
國人立其王頭黎之女為王女不能治國更立頭黎之姑子諸葛地謂之環王妻以女王|{
	更工衡翻妻七細翻}
二載春正月上皇下誥以憲部尚書李麟同平章事總理百司命崔圓奉誥赴彭原麟懿祖之後也|{
	懿祖光皇帝諱天錫太祖之父也麟懿祖次子乞豆之後}
安禄山自起兵以來目漸昏至是不復睹物|{
	復扶又翻}
又病疽性益躁暴左右使令小不如意動加箠撻或時殺之既稱帝深居禁中大將希得見其面皆因嚴莊白事莊雖貴用事亦不免箠撻閹宦李猪兒被撻尤多|{
	舊書曰李猪兒出契丹部落十數歲事祿山甚黠慧祿山持刃盡去其勢血射數升欲死祿山以灰火傅之盡日而蘇因為閹人遂見信用}
左右人不自保祿山嬖妾段氏生子慶恩欲以代慶緒為後慶緒常懼死不知所出莊謂慶緒曰事有不得已者時不可失慶緒曰兄有所為敢不敬從又謂猪兒曰汝前後受撻寜有數乎不行大事死無日矣猪兒亦許諾莊與慶緒夜持兵立帳外猪兒執刀直入帳中斫祿山腹左右懼不敢動祿山捫枕旁刀不獲|{
	舊書曰祿山眼無所見牀頭常有一刀}
撼帳竿曰必家賊也腸已流出數斗遂死掘牀下深數尺|{
	深式浸翻}
以氈裹其尸埋之誡宫中不得泄乙卯旦莊宣言於外云祿山疾亟立晉王慶緒為太子尋即帝位尊禄山為太上皇然後喪慶緒性昏懦言辭無序莊恐衆不服不令見人慶緒日縱酒為樂|{
	懦奴過翻又奴亂翻令力丁翻樂音洛}
兄事莊以為御史大夫馮翊王事無大小皆取決焉厚加諸將官爵以悦其心|{
	將即亮翻}
上從容謂李泌曰廣平為元帥踰年今欲命建寜專征又恐勢分立廣平為太子何如對曰臣固嘗言之矣戎事交切須即區處|{
	從千容翻泌毗必翻帥所類翻處昌呂翻}
至於家事當俟上皇不然後代何以辨陛下靈武即位之意邪此必有人欲令臣與廣平有隙耳臣請以語廣平|{
	邪音耶語牛倨翻}
廣平亦必未敢當泌出以告廣平王俶俶曰此先生深知其心欲曲成其美也乃入固辭曰陛下猶未奉晨昏|{
	俶昌六翻謂人子晨省昏定之禮}
臣何心敢當儲副願俟上皇還宫臣之幸也上賞慰之|{
	還從宣翻又音如字}
李輔國本飛龍小兒|{
	凡廐牧五坊禁苑給使者皆謂之小兒李輔國以閹奴為閑廐小兒}
粗閑書計給事太子宫上委信之輔國外恭謹寡言而内狡險見張良娣有寵陰附會之與相表裏建寜王倓數於上前詆訐二人罪惡|{
	粗坐五翻娣大計翻倓徙甘翻數所角翻計居謁翻}
二人譖之於上曰倓恨不得為元帥|{
	不用倓為元帥見上卷上年九月}
謀害廣平王上怒賜倓死 |{
	考異曰鄴侯家傳曰肅宗自馬嵬北行至同官縣食于土豪李謙家張良娣稱腹痛不能乘馬併小女寄謙家而去上即位使人迎之迎者或有他說建寜聞而數以為言舊傳曰倓屢言良娣頗專恣與護國連結内外欲傾動皇嗣未知孰是實録新舊本紀皆無倓死年月列傳云倓死明年冬廣平王復兩京然則倓死在至德元載也按鄴侯家傳上從容言曰廣平為元帥經年今欲命建寜為元帥則是至德二載倓猶在也又云代宗使自彭原迎倓喪故置于此護國當作輔國}
於是廣平王俶及李泌皆内懼俶謀去輔國及良娣泌曰不可王不見建寜之禍乎俶曰竊為先生憂之|{
	去羌呂翻為于偽翻 考異曰鄴侯家傳曰先公在内院未起輔國體肥重因近牀語遂以身壓先公先公素服氣乃閉氣良久而去按泌方為上所厚恐輔國亦不敢擅殺今不取}
泌曰泌與主上有約矣|{
	謂上許泌以賊平任行高志見上卷上年九月}
俟平京師則去還山庶免於患俶曰先生去則俶愈危矣泌曰王但盡人子之孝良娣婦人王委曲順之亦何能為|{
	吾觀代宗所以卒免張后之禍者用李泌之言也}
上謂泌曰今郭子儀李光弼已為宰相若克兩京平四海則無官以賞之奈何對曰古者官以任能爵以酬功漢魏以來雖以郡縣治民|{
	治直之翻}
然有功則錫以茅土傳之子孫至于周隋皆然唐初未得關東故封爵皆設虚名其食實封者給繒布而已|{
	唐制食實封者凡一戶則以一丁之歲調給之}
貞觀中太宗欲復古制大臣議論不同而止|{
	見一百九十五卷貞觀十三年}
由是賞功者多以官夫以官賞功有二害非才則廢事權重則難制是以功臣居大官者皆不為子孫之遠圖務乘一時之權以邀利無所不為曏使祿山有百里之國則亦惜之以傳子孫不反矣為今之計俟天下既平莫若疏爵土以賞功臣則雖大國不過二三百里可比今之小郡豈難制哉於人臣乃萬世之利也上曰善|{
	夫音扶過古禾翻考異曰鄴侯家傳曰泌既與上論封爵之事因曰若臣者受賞與它人異上曰何故公曰臣絶粒無家禄位與茅土皆非所要為陛下帷幄運籌收京師後但枕天子膝睡一覺使有司奏客星犯帝座一動天文是矣上大笑及南幸扶風每頓皆令先公領元帥兵先清行宫收管鑰奏報然後上至至保定郡先公于本院寐上來入院不令人驚登牀捧先公首置于膝上久方覺上曰天子膝已枕睡了剋復効在何時還朕可也欲起謝恩持之不許對曰當如郡名必保定矣此近戲謔今不取}
上聞安西北庭及拔汗那大食諸國兵至凉鄯甲子幸保定|{
	保定郡本涇州安定郡去載更郡名鄯音善又時戰翻}
丙寅劒南兵賈秀等五千人謀反將軍席元慶臨卭

太守柳奕討誅之|{
	臨卭郡卭州卭渠容翻守式又翻}
河西兵馬使蓋庭倫|{
	蓋古盍翻}
與武威九姓商胡安門物等殺節度使周泌|{
	使疏吏翻泌毗必翻}
聚衆六萬武威大城之中小城有七|{
	武威郡凉州冶姑臧舊城匈奴所築南北七里東西三里張氏據河西又增築四城箱各千步并舊城為五餘二城未知誰所築也}
胡據其五二城堅守度支判官崔稱與中使劉日新以二城兵攻之旬有七日平之 史思明自博陵蔡希德自太行高秀巗自大同牛廷介自范陽引兵共十萬寇太原|{
	行戶剛翻博陵郡定州蔡希德自上黨下太行道也高秀巗為賊守大同自此趨太原牛廷介自幽州與史思明等合}
李光弼麾下精兵皆赴朔方餘團練烏合之衆不滿萬人思明以為太原指掌可取既得之當遂長驅取朔方河隴太原諸將皆懼議修城以待之光弼曰太原城周四十里|{
	太原都城左汾右晉濳丘在中長四千三百二十一步廣二千一百二十二步周萬五千一百五十三步宫城在都城西北周二千五百二十步汾東曰東城貞觀十一年長史李勣所築兩城之間曰中城武后築以合東城周四十里者止言都城耳}
賊垂至而興役是未見敵先自困也乃帥士卒及民於城外鑿壕以自固作墼數十萬|{
	帥讀曰率墼古歷翻範土為之}
衆莫知所用及賊攻城於外光弼用之增壘於内壞輒補之思明使人取攻具於山東以胡兵三千衛送之至廣陽|{
	廣陽漢上艾縣後漢改石艾縣天寶元年更名屬太原府井陘關在其東葦澤關在其東北皆通山東之道}
别將慕容溢張奉璋邀擊盡殺之思明圍太原月餘不下乃選驍鋭為遊兵戒之曰我攻其北則汝濳趣其南攻東則趣西有隙則乘之|{
	趣七喻翻}
而光弼軍令嚴整雖寇所不至警邏未嘗少懈賊不得入光弼購募軍中苟有小技皆取之隨能使之人盡其用得安邊軍錢工三善穿地道|{
	安邊軍在蔚州興唐縣蔚州有銅冶有錢官故有錢工時得其三人也}
賊於城下仰而侮詈光弼遣人從地道中曳其足而入臨城斬之自是賊行皆視地賊為梯衝土山以攻城光弼為地道以迎之近城輒䧟|{
	近其靳翻}
賊初逼城急光弼作大礮飛巨石一輒弊二十餘人賊死者什二三乃退營於數十步外|{
	退營於礮所不能及之地礮匹貌翻}
圍守益固光弼遣人詐與賊約刻日出降賊喜不為備光弼使穿地道周賊營中搘之以木|{
	搘章移翻拄也}
至期光弼勒兵在城上遣禆將將數千人出如降狀賊皆屬目|{
	屬之欲翻}
俄而營中地陷死者千餘人賊衆驚亂官軍鼓譟乘之俘斬萬計會安祿山死慶緒使思明歸守范陽留蔡希德等圍太原 慶緒以尹子奇為汴州刺史河南節度使甲戍子奇以歸檀及同羅奚兵十三萬趣睢陽|{
	歸當作媯媯州也唐人雜史多有作歸檀者蓋誤也趣七喻翻睢音雖}
許遠告急于張巡巡自寜陵引兵入睢陽|{
	自寜陵東至睢陽四十五里}
巡有兵三千人與遠兵合六千八百人賊悉衆逼城巡督勵將士晝夜苦戰或一日至二十合凡十六日擒賊將六十餘人殺士卒二萬餘衆氣自倍遠謂巡曰遠懦不習兵|{
	將即亮翻懦奴過翻乂奴亂翻}
公智勇兼濟遠請為公守公請為遠戰自是之後遠但調軍糧|{
	為于偽翻調徒釣翻}
修戰具居中應接而已戰鬭籌畫一出于巡賊遂夜遁 郭子儀以河東居兩京之間得河東則兩京可圖|{
	河東郡蒲州自河東進兵攻取潼關則兩京之路中斷然後可圖也}
時賊將崔乾祐守河東丁丑子儀濳遣人入河東與唐官陷賊者謀俟官軍至為内應 初平盧節度使劉正臣自范陽敗歸|{
	事見上卷上年}
安東都護王玄志鴆殺之祿山以其黨徐歸道為平盧節度使玄志復與平盧將侯希逸襲殺之|{
	復扶又翻}
又遣兵馬使董秦將兵以葦茷度海與大將田神功擊平原樂安下之防河招討使李銑承制以秦為平原太守|{
	茷音伐秦將即亮翻又音如字守式又翻}
二月戊子上至鳳翔 郭子儀自洛交引兵趣河東|{
	宋白曰鄜州洛交郡漢上郡雕隂之地後魏為東秦州又改為北華州廢帝改為鄜州取鄜畤為名隋自杏城移治五交城天寶改洛交郡治洛交縣取洛水之交也趣七喻翻}
分兵取馮翊|{
	馮翊郡同州兼取蒲同則跨據河東西以圖關陜可以制賊}
己丑夜河東司戶韓旻等飜河東城迎官軍|{
	新志戶曹司戶參軍事掌戶籍計帳道路過所蠲符雜徭逋負良賤芻槀逆旅婚姻田訟旌别孝悌}
殺賊近千人|{
	近其靳翻}
崔乾祐踰城得免城北兵攻城且拒官軍子儀擊破之乾祐走子儀追擊之斬首四千級捕虜五千人乾祐至安邑|{
	安邑縣時屬解州}
安邑人開門納之半入閉門擊之盡殪|{
	殪一計翻}
乾祐未入自白逕嶺亡去|{
	白逕嶺在解縣東}
遂平河東 上至鳳翔旬日隴右河西安西西域之兵皆會江淮庸調亦至洋川漢中|{
	江淮庸調泝漢而上梁洋調徒弔翻}
上自散關通表成都信使駱驛|{
	往來不絶曰駱驛使疏吏翻}
長安人聞車駕至從賊中自拔而來者日夜不絶西師憩息既定|{
	憇去例翻}
李泌請遣安西及西域之衆如前策並塞東北自歸檀南取范陽上曰今大衆已集庸調亦至當乘兵鋒擣其腹心而更引兵東北數千里先取范陽不亦迂乎對曰今以此衆直取兩京必得之然賊必再彊我必又困非久安之策上曰何也對曰今所恃者皆西北守塞及諸胡之兵性耐寒而畏暑若乘其新至之銳攻祿山已老之師其勢必克兩京春氣已深賊收其餘衆遁歸巢穴關東地熱官軍必困而思歸不可留也賊休兵秣馬伺官軍之去必復南來然則征戰之勢未有涯也|{
	伺相吏翻復扶又翻後果如泌所料}
不若先用之於寒鄉除其巢穴則賊無所歸根本永絶矣上曰朕切於晨昏之戀|{
	言急于復兩京迎上皇}
不能待此決矣|{
	言決不能從泌之策也}
關内節度使王思禮軍武功兵馬使郭英乂軍東原王難得軍西原|{
	此即武功之東原西原也蜀諸葛亮駐師之地使疏吏翻}
丁酉安守忠等寇武功郭英乂戰不利矢貫其頤而走王難得望之不救亦走思禮退軍扶風賊遊兵至大和關去鳳翔五十里鳳翔大駭戒嚴 李光弼將敢死士出擊蔡希德大破之斬首七萬餘級希德遁去|{
	將即亮翻又音如字}
安慶緒以史思明為范陽節度使兼領恒陽軍事封媯川王|{
	唐會要恒陽軍置於恒州郭下恒戶登翻媯居為翻}
以牛廷介領安陽軍事|{
	時慶緒分兵屯鄴郡安陽縣因所屯之地而曰安陽軍}
張忠志為常山太守兼團練使鎮井陘口餘各令歸舊任募兵以禦官軍|{
	守式又翻陘音刑令力丁翻}
先是安禄山得兩京珍貨悉輸范陽思明擁彊兵據富資益驕横|{
	先悉薦翻横戶孟翻}
浸不用慶緒之命慶緒不能制|{
	為思明殺慶緒張本}
戊戍永王璘敗死|{
	璘離珍翻 考異曰新舊紀傳實錄唐歷皆不見璘敗在何處若在當塗不應登城望見瓜步楊子李白永王東巡歌云龍盤虎踞帝王州帝子金陵訪古丘又云初從雲夢開朱邸更取金陵作小山如此似已據金陵但於諸書别無所見疑未敢質 余詳考下文璘所登以望瓜步楊子者蓋登丹楊郡城也璘自當塗進兵擊斬丹楊太守閻敬之遂據丹楊城然後可以望見楊子及瓜步江津之兵及其敗也自丹陽奔晉陵以趣鄱陽其道里節次可驗}
其黨薛鏐皆伏誅時李成式與河北招討判官李銑合兵討璘銑兵數千軍于楊子|{
	楊子本為鎮屬江都縣高宗廢鎮置楊子縣即今真州治所}
成式使判官裴茂|{
	新書作裴茙}
將兵三千軍于瓜步廣張旗幟列於江津璘與其子瑒登城望之始有懼色季廣琛召諸將謂曰吾屬從王至此天命未集人謀已隳不如及兵鋒未交早圖去就死於鋒鏑永為逆臣矣諸將皆然之於是廣琛以麾下奔廣陵渾惟明奔江寜|{
	是年以丹楊之江寜縣置昇州江寜縣}
馮季康奔白沙|{
	今真州治所唐之白沙鎮也時屬廣陵郡}
璘憂懼不知所出其夕江北之軍多列炬火光照水中一皆為兩璘軍又以火應之璘以為官軍已濟江遽挈家屬與麾下濳遁及明不見濟者乃復入城收兵具舟楫而去|{
	復扶又翻}
成式將趙侃等濟江至新豐|{
	新書曰新豐陵攷其地在晉陵界蓋南朝山陵之名}
璘使瑒及其將高仙琦將兵擊之侃等逆戰射瑒中肩|{
	射而亦翻中竹仲翻}
璘兵遂潰璘與仙琦收餘衆南奔鄱陽|{
	鄱陽郡饒州}
收庫物甲兵欲南奔嶺表江西采訪使皇甫侁|{
	江西江南西道也史從簡便曰江西侁所臻翻}
遣兵追討擒之濳殺之於傳舍|{
	傳張戀翻}
瑒亦死于亂兵侁使人送璘家屬還蜀上曰侁既生得吾弟何不送之於蜀而擅殺之邪遂廢侁不用 庚子郭子儀遣其子旴及兵馬使李韶光大將王祚濟河擊潼關破之 |{
	考異曰實錄三月朔方節度使郭子儀大破賊于潼關汾陽家傳云正月二十八日使宗子懷文濳募郭俊苟文俊入河東搆忠義與大軍約期以翻城公乃進軍出洛交分兵收馮翊二月十一日郭俊等伺大軍將至中夜舉火剋斬幽檀勁卒千人崔乾祐尋縋而免乾祐先置兵於城北廢府遂以三千兵攻城自領馬步五千伏於關城中公使旴及僕固懷恩等先擊之賊大破遽焚橋我軍蹈之而滅乾祐棄關城尋白涇嶺而逸遂收河東郡舊子儀傳曰二年三月子儀大破賊於潼關崔乾祐退保蒲津時永樂尉趙復河東司戶韓旻司士徐炅宗子李藏鋒等䧟賊在蒲州四人密謀伺王師至則為内應及子儀攻蒲州趙復等斬賊守陴者開門納子儀乾祐與麾下數千人北走安邑百姓偽降乾祐兵入將半下懸門擊之乾祐未入遂得脱身東走子儀遂收陜郡永豐倉自是潼陜之間無復寇鈔唐歷云子儀收蒲州又襲下潼關按潼關在河東馮翊之南若未破河東馮翊安能先取潼關又實錄云三月取河東而下復載二月戊戍以後事與舊傳皆誤也今從汾陽傳及唐歷}
斬首五百級安慶緒遣兵救潼關郭旴等大敗死者萬餘人李韶光王祚戰死僕固懷恩抱馬首浮度渭水退保河東 |{
	考異曰汾陽家傳云偽關西節度安守忠帥兵至二十九日公使僕固懷恩王仲昇陳于永豐倉南及暮百戰斬一萬級李韶光王祚決戰而死唐歷子儀襲下潼關及同州盛兵潼關以守之賊將李歸仁來救子儀戰大敗死者萬餘衆退守河東歸仁遂攻陷同州刺史蕭賁死之盡屠城中舊僕固懷恩傳云懷恩退至渭水無舟楫抱馬以度存者僅半奔歸河東按子儀不得馮翊則西路不通後奉詔赴鳳翔歷馮翊而去則馮翊不陷也潼關者兩京往來之路賊所必争也子儀若不敗則何以棄潼關而不守今參取衆書可信者存之}
三月辛酉以左相韋見素為左僕射中書侍郎同平章事裴冕為右僕射並罷政事初楊國忠惡憲部尚書苗晉卿|{
	惡烏路翻}
安祿山之反也請出晉卿為陜郡太守兼陜弘農防禦使|{
	兼二郡防禦}
晉卿固辭老病上皇不悦使之致仕及長安失守晉卿濳竄山谷上至鳳翔手敕徵之為左相軍國大務悉咨之 上皇思張九齡之先見|{
	謂識祿山有反相也事見二百十四卷開元二十二年}
為之流涕|{
	為于偽翻}
遣中使至曲江祭之|{
	張九齡韶州曲江人使疏吏翻宋白曰曲江縣以湞水屈曲為名}
厚恤其家 尹子奇復引大兵攻睢陽|{
	復扶又翻}
張廵謂將士曰吾受國恩所守正死耳但念諸君捐軀命膏草野|{
	膏居號翻}
而賞不酬勲|{
	以虢王巨靳告身不與賜物恐將士怨望而不力戰故先以此言慰撫之}
以此痛心耳將士皆激勵請奮巡遂椎牛大饗士卒盡軍出戰賊望見兵少笑之巡執旗帥諸將直衝賊陳|{
	少始紹翻帥讀曰率陳讀曰陣}
賊乃大潰斬將三十餘人殺士卒三千餘人逐之數十里明日賊又合軍至城下廵出戰晝夜數十合屢摧其鋒而賊攻圍不輟 辛未安守忠將騎二萬寇河東郭子儀擊走之斬首八千級捕虜五千人|{
	將即亮翻又音如字騎奇寄翻}
夏四月顏真卿自荆襄北詣鳳翔|{
	真卿棄平原渡河欲赴行在而陜洛為賊所梗故南奔荆襄然後自荆襄取上津路北詣鳳翔}
上以為憲部尚書|{
	憲部刑部尚辰羊翻}
上以郭子儀為司空天下兵馬副元帥|{
	帥所類翻 考異曰唐歷四月子儀為司空尋以廣平王為元帥子儀為副元帥按鄴侯家傳廣平在靈武已為元帥唐歷誤也}
使將兵赴鳳翔|{
	將即亮翻又音如字}
庚寅李歸仁以鐵騎五千邀之于三原北|{
	三原本漢池陽地後魏置三原縣}
子儀使其將僕固懷恩王仲昇渾釋之李若幽|{
	考異曰汾陽家傳作桑如珪今從舊傳}
伏兵擊之於白渠留運橋殺傷略盡歸仁游水而逸|{
	白渠漢白公所開因名}
若幽神通之玄孫也|{
	淮安王神通隋義寜初起兵應高祖}
子儀與王思禮軍合於西渭橋進屯潏西|{
	唐都長安跨渭為三橋東曰東渭橋中曰中渭橋西曰西渭橋程大昌曰秦漢唐架渭者凡三橋在咸陽西十里名便橋漢武帝造在咸陽東南二十二里者名中渭橋秦始皇造在萬年縣東南四十里者為東渭橋不知始于何世水經注潏水出杜陵之樊川過漢長安城西而北注于渭潏音決}
安守忠李歸仁軍於京城西清渠|{
	程大昌雍録有漢唐要地參出圖唐京城西有漕渠南出豐水逕延平金光二門至京城西北角屈而東流逕漢故長安城南至芳林園西又屈而北流入渭清渠在漕渠之東直秦之故杜南城稍東即香積寺北}
相守七日官軍不進五月癸丑守忠偽退子儀悉師逐之賊以驍騎九千為長蛇陳|{
	陳讀曰陣}
官軍擊之首尾為兩翼夾擊官軍官軍大潰判官韓液監軍孫知古皆為賊所擒軍資器械盡棄之子儀退保武功|{
	監古衘翻 考異曰汾陽家傳曰賊帥安守忠李歸仁領八萬兵屯于昆明池西五月三日陳於清渠之側公大破之追奔十餘里斬首二萬級六月救兵至又陣於清渠我師敗績以冒暑毒師人多病遂收兵赴鳳翔今從舊傳}
中外戒嚴是時府庫無蓄積朝廷專以官爵賞功諸將出征皆給空名告身自開府特進列卿大將軍下至中郎郎將聽臨事注名其後又聽以信牒授人官爵有至異姓王者|{
	信牒者未有告身先給牒以為信也}
諸軍但以職任相統攝不復計官爵高下及清渠之敗復以官爵收散卒|{
	恐其潰散畏罪而歸賊復以官爵收之復扶又翻}
由是官爵輕而貨重大將軍告身一通纔易一醉凡應募入軍者一切衣金紫至有朝士僮僕衣金紫稱大官而執賤役者|{
	衣於既翻}
名器之濫至是而極焉 房琯性高簡時國家多難|{
	難乃旦翻}
而琯多稱病不朝謁|{
	朝直遥翻}
不以職事為意日與庶子劉秩諫議大夫李揖高談釋老或聽門客董庭蘭鼓琴庭蘭以是大招權利御史奏庭蘭贓賄丁巳罷琯為太子少師|{
	房琯既敗師而不思補過罷之為散官猶輕典也}
以諫議大夫張鎬為中書侍郎同平章事上常使僧數百人為道場于内晨夜誦佛鎬諫曰帝王當修德以弭亂安人未聞飯僧可致太平也上然之|{
	飯扶晚翻}
庚申上皇追冊上母楊妃為元獻皇后|{
	妃隋納言士達之曾孫景雲初入東宫為良媛實生上}
山南東道節度使魯炅守南陽賊將武令珣田承嗣相繼攻之城中食盡一鼠直錢數百餓死者相枕籍|{
	枕職任翻}
上遣宦官將軍曹日昇往宣慰|{
	以宦官而為將軍故謂之宦官將軍}
圍急不得入日昇請單騎入致命襄陽太守魏仲犀不許會顔真卿自河北至|{
	是年夏四月顔真卿已自荆襄北詣靈武曹日昇之至襄陽蓋在四月之前}
曰曹將軍不顧萬死以致帝命何為沮之借使不達不過亡一使者達則一城之心固矣日昇與十騎偕往賊畏其鋭不敢逼城中自謂望絶及見日昇大喜日昇復為之至襄陽取糧|{
	復扶又翻}
以千人運糧而入賊不能遏炅在圍中凡周歲|{
	去年五月賊圍南陽至是周歲}
晝夜苦戰力竭不能支壬戍夜開城帥餘兵數千突圍而出奔襄陽承嗣追之轉戰二日不能克而還|{
	帥讀曰率還音旋又如字}
時賊欲南侵江漢賴炅扼其衝要南夏得全|{
	夏戶雅翻}
司空郭子儀詣闕請自貶|{
	以清渠之敗也}
甲子以子儀為左僕射 尹子奇益兵圍睢陽益急張巡於城中夜鳴鼓嚴隊若將出擊者賊聞之達旦儆備既明巡乃寢兵絶鼓賊以飛樓瞰城中無所見遂解甲休息巡與將軍南霽雲|{
	南姓也周有南仲魯有大夫南遺}
郎將雷萬春等十餘將各將五十騎開門突出直衝賊營至子奇麾下營中大亂斬賊將五十餘人殺士卒五千餘人巡欲射子奇而不識乃剡蒿為矢|{
	射而亦翻下雲射同剡以冉翻削也}
中者喜|{
	中竹仲翻}
謂巡矢盡走白子奇乃得其狀使霽雲射之喪其左目幾獲之|{
	喪息浪翻幾居依翻}
子奇乃收軍退還 六月田乾真圍安邑會陜郡賊將楊務欽密謀歸國河東太守馬承光以兵應之務欽殺城中諸將不同己者飜城來降乾真解安邑遁去 將軍王去榮以私怨殺本縣令當死|{
	王去榮富平人}
上以其善用礮壬辰敇免死以白衣於陜郡効力|{
	時陜郡新復介居兩京之間賊所必攻也上欲免去榮之死而收其力用而不計其隳國法也 考異曰實錄云於河東承天軍効力據賈至集陜郡也今從之}
中書舍人賈至不即行下|{
	下遐嫁翻下上下同}
上表以為去榮無狀殺本縣之君易曰臣弑其君子弑其父非一朝一夕之故其所由來者漸矣|{
	易坤卦文言之辭}
若縱去榮可謂生漸矣議者謂陜郡初復非其人不可守然則它無去榮者何以亦能堅守乎陛下若以礮石一能即免殊死今諸軍技藝絶倫者|{
	技渠綺翻}
其徒寔繁必恃其能所在犯上復何以止之|{
	復扶又翻}
若止捨去榮而誅其餘者則是法令不一而誘人觸罪也|{
	誘音酉}
今惜一去榮之材而不殺必殺十如去榮之材者不亦其傷益多乎夫去榮逆亂之人也焉有逆於此而順於彼亂於富平而治於陜郡悖於縣君而不悖於大君歟|{
	夫音扶去榮縣民也縣令則其君也大君謂天子治直吏翻悖蒲妹翻又蒲沒翻}
伏惟明主全其遠者大者則禍亂不日而定矣上下其事令百官議之|{
	下戶嫁翻}
太子太師韋見素等議以為法者天地大典帝王猶不敢擅殺是臣下之權過於人主也|{
	過古禾翻又古卧翻}
去榮既殺人不死則軍中凡有技能者亦自謂無憂所在暴横|{
	技渠綺翻横戶孟翻}
為郡縣者不亦難乎陛下為天下主愛無親疎得一去榮而失萬姓何利之有於律殺本縣令列於十惡|{
	唐初房玄齡依隋定律有十惡之條一曰謀反二曰謀大逆三曰謀叛四曰謀惡逆五曰不道六曰大不敬七曰不孝八曰不睦九曰不義十曰内亂犯十惡者不得依議請之例其不義之條注曰謂殺本屬府主刺史縣令見受業師吏卒殺本部五品已上官長及聞夫喪匿不舉哀若作樂釋服從吉及改嫁}
而陛下寛之王法不行人倫道屈臣等奉詔不知所從夫國以法理軍以法勝有恩無威慈母不能使其子陛下厚養戰士而每戰少利豈非無法邪|{
	夫音扶少始紹翻邪音耶}
今陜郡雖要不急於法也有法則海内無憂不克况陜郡乎無法則陜郡亦不可守得之何益而去榮末技陜郡不以之存亡王法有無國家乃為之輕重此臣等所以區區願陛下守貞觀之法上竟捨之|{
	陜失冉翻觀古玩翻}
至曾之子也|{
	賈曾見二百十卷先天元年}
南充土豪何滔作亂執本郡防禦使楊齊魯|{
	南充郡果州}
劒南節度使盧元裕兵討平之|{
	使疏吏翻}
秋七月河南節度使賀蘭進明克高密琅邪殺賊二萬餘人|{
	邪音耶}
戊申夜蜀郡兵郭千仞等反六軍兵馬使陳玄禮劒南節度使李峘討誅之|{
	峘胡登翻}
壬子尹子奇復徵兵數萬攻睢陽先是許遠於城中積糧至六萬石|{
	睢音雖復扶又翻先悉荐翻}
虢王巨以其半給濮陽濟陰二郡|{
	濮博木翻濟子禮翻}
遠固爭之不能得既而濟陰得糧遂以城叛而睢陽城至是食盡將士人廩米日一合|{
	廩當作禀音筆錦翻給也合音閤十龠為合}
雜以茶紙樹皮為食而賊糧運通兵敗復徵|{
	復扶又翻}
睢陽將士死不加益諸軍饋救不至士卒消耗至一千六百人皆饑病不堪鬭遂為賊所圍張巡乃修守具以拒之賊為雲梯勢如半虹|{
	杜佑曰以大木為床下置六輪上立雙牙牙有檢梯節長丈二尺有四桄桄相去四尺勢微回遞互相檢飛于雲間以窺城中有上城梯首冠雙轆轤枕城而上謂之飛雲梯}
置精卒二百於其上推之臨城|{
	推吐雷翻}
欲令騰入巡豫於城鑿三穴候梯將至於一穴中出大木末置鐵鈎鉤之使不得退一穴中出一木拄之使不得進一穴中出一木木末置鐵籠盛火焚之其梯中折|{
	盛時征翻折而設翻}
梯上卒盡燒死賊又以鈎車鈎城上棚閣|{
	棚閣者於城上架木為棚跳出城外四五尺許上有屋宇可蔽風雨戰士居之以臨禦外敵今人謂之敵樓}
鈎之所及莫不崩䧟巡以大木末置連鏁鏁末置大鐶搨其鈎頭|{
	鏁蘇果翻搨吐盍翻}
以革車拔之入城截其鈎頭而縱車令去賊又造木驢攻城巡鎔金汁灌之應投銷鑠賊又於城西北隅以土囊積柴為磴道|{
	磴都鄧翻}
欲登城巡不與爭利每夜潜以松明乾蒿投之於中|{
	松明者松枯而油存可燎之以為明乾音干}
積十餘日賊不之覺因出軍大戰使人順風持火焚之賊不能救經二十餘日火方滅巡之所為皆應機立辦賊服其智不敢復攻遂於城外穿三重壕立木柵以守巡|{
	復扶又翻重直龍翻}
巡亦於内作壕以拒之 丁巳賊將安武臣攻陜郡楊務欽戰死賊遂屠陜|{
	以孤城介居彊寇之間外無救援宜其受屠}
崔渙在江南選補冒濫者衆八月罷渙為餘杭太守|{
	杭州餘杭郡隋於餘杭縣置杭州後自餘杭移治錢唐後又移治柳浦今州城是也餘杭漢古縣也寰宇記曰禹捨舟登陸於此因名餘杭}
江東采訪防禦使 以張鎬兼河南節度采訪等使代賀蘭進明 靈昌太守許叔冀為賊所圍救兵不至拔衆奔彭城 |{
	考異曰實錄云拔其衆南投睢陽郡按張中丞傳云許叔冀在譙郡蓋叔冀欲投睢陽為賊所圍遂投彭城譙郡耳今從新記}
睢陽士卒死傷之餘纔六百人張巡許遠分城而守之巡守東北遠守西南與士卒同食茶紙不復下城賊士攻城者巡以順逆說之往往棄賊來降為巡死戰前後二百餘人|{
	復扶又翻說式芮翻為于偽翻}
是時許叔冀在譙郡尚衡在彭城賀蘭進明在臨淮|{
	漢武帝置臨淮郡後漢明帝更名下邳其疆域廣矣梁於漢徐縣地置高平郡隋開皇十八年廢郡為徐城縣屬泗州下邳郡時泗州治宿預也武后長安四年割徐城南界兩鄉於沙熟淮口置臨淮縣開元二十三年移泗州治臨淮天寶元年更為臨淮郡}
皆擁兵不救城中日蹙巡乃令南霽雲將三十騎犯圍而出告急於臨淮霽雲出城賊衆數萬遮之霽雲直衝其衆左右馳射賊衆披靡|{
	披普彼翻}
止亡兩騎既至臨淮見進明進明曰今日睢陽不知存亡兵去何益霽雲曰睢陽若䧟霽雲請以死謝大夫且睢陽既拔即及臨淮譬如皮毛相依安得不救進明愛霽雲勇壯不聽其語強留之|{
	強其兩翻}
具食與樂延霽雲坐霽雲慷慨泣且語曰霽雲來睢陽之人不食月餘矣霽雲雖欲獨食且不下咽|{
	咽烏前翻喉也}
大夫坐擁彊兵觀睢陽䧟没曾無分災救患之意豈忠臣義士之所為乎因齧落一指以示進明 |{
	考異曰韓愈書張中丞傳後云因拔所佩刀斷一指血淋漓以示賀蘭一座大驚皆感激為雲泣下按柳宗元霽雲碑云自噬其指曰噉此足矣今從舊傳}
曰霽雲既不能達主將之意請留一指以示信歸報座中往往為泣下|{
	為于偽翻}
霽雲察進明終無出師意遂去至寜陵與城使亷坦同將步騎三千人|{
	張巡自寜陵入睢陽蓋使亷坦守寜陵城城使巡所署置也將即亮翻使疏吏翻}
閏月戊申夜冒圍且戰且行至城下大戰壞賊營|{
	壞音怪}
死傷之外僅得千人入城城中將吏知無救皆慟哭賊知援絶圍之益急初房琯為相惡賀蘭進明|{
	事見去載十月惡烏路翻}
以為河南節度使以許叔冀為進明都知兵馬使俱兼御史大夫叔冀自恃麾下精鋭且官與進明等不受其節制故進明不敢分兵非惟疾巡遠功名亦懼為叔冀所襲也|{
	史言房琯以私憾進明用許叔冀以制其肘腋使不敢分兵救巡遠然以進明之才借使出兵亦未必能制勝}
戊辰上勞饗諸將|{
	勞力到翻}
遣攻長安謂郭子儀曰事之濟否在此行也對曰此行不捷臣必死之 |{
	考異曰汾陽家傳閏八月二十三日肅宗授代宗鉞俾誅元惡詔公為副元帥二十三日出鳳翔實錄九月丁亥元帥領兵十五萬辭出又云戊子回紇葉護至扶風蓋郭子儀以閏月二十三日先行屯扶風九月十三日廣平乃也}
辛未御史大夫崔光遠破賊於駱谷光遠行軍司馬王伯倫判官李椿將二千人攻中渭橋殺賊守橋者千人乘勝至苑門|{
	長安苑門也}
賊有先屯武功者聞之奔歸遇于苑北合戰殺伯倫擒椿送洛陽然自是賊不復屯武功矣|{
	復扶又翻下同}
賊屢攻上黨常為節度使程千里所敗|{
	敗補邁翻}
蔡希德復引兵圍上黨|{
	上黨郡潞州為程千里被擒張本}


資治通鑑卷二百十九
















































































































































