<!DOCTYPE html PUBLIC "-//W3C//DTD XHTML 1.0 Transitional//EN" "http://www.w3.org/TR/xhtml1/DTD/xhtml1-transitional.dtd">
<html xmlns="http://www.w3.org/1999/xhtml">
<head>
<meta http-equiv="Content-Type" content="text/html; charset=utf-8" />
<meta http-equiv="X-UA-Compatible" content="IE=Edge,chrome=1">
<title>資治通鑒_12-資治通鑑卷十一_12-資治通鑑卷十一</title>
<meta name="Keywords" content="資治通鑒_12-資治通鑑卷十一_12-資治通鑑卷十一">
<meta name="Description" content="資治通鑒_12-資治通鑑卷十一_12-資治通鑑卷十一">
<meta http-equiv="Cache-Control" content="no-transform" />
<meta http-equiv="Cache-Control" content="no-siteapp" />
<link href="/img/style.css" rel="stylesheet" type="text/css" />
<script src="/img/m.js?2020"></script> 
</head>
<body>
 <div class="ClassNavi">
<a  href="/24shi/">二十四史</a> | <a href="/SiKuQuanShu/">四库全书</a> | <a href="http://www.guoxuedashi.com/gjtsjc/"><font  color="#FF0000">古今图书集成</font></a> | <a href="/renwu/">历史人物</a> | <a href="/ShuoWenJieZi/"><font  color="#FF0000">说文解字</a></font> | <a href="/chengyu/">成语词典</a> | <a  target="_blank"  href="http://www.guoxuedashi.com/jgwhj/"><font  color="#FF0000">甲骨文合集</font></a> | <a href="/yzjwjc/"><font  color="#FF0000">殷周金文集成</font></a> | <a href="/xiangxingzi/"><font color="#0000FF">象形字典</font></a> | <a href="/13jing/"><font  color="#FF0000">十三经索引</font></a> | <a href="/zixing/"><font  color="#FF0000">字体转换器</font></a> | <a href="/zidian/xz/"><font color="#0000FF">篆书识别</font></a> | <a href="/jinfanyi/">近义反义词</a> | <a href="/duilian/">对联大全</a> | <a href="/jiapu/"><font  color="#0000FF">家谱族谱查询</font></a> | <a href="http://www.guoxuemi.com/hafo/" target="_blank" ><font color="#FF0000">哈佛古籍</font></a> 
</div>

 <!-- 头部导航开始 -->
<div class="w1180 head clearfix">
  <div class="head_logo l"><a title="国学大师官网" href="http://www.guoxuedashi.com" target="_blank"></a></div>
  <div class="head_sr l">
  <div id="head1">
  
  <a href="http://www.guoxuedashi.com/zidian/bujian/" target="_blank" ><img src="http://www.guoxuedashi.com/img/top1.gif" width="88" height="60" border="0" title="部件查字,支持20万汉字"></a>


<a href="http://www.guoxuedashi.com/help/yingpan.php" target="_blank"><img src="http://www.guoxuedashi.com/img/top230.gif" width="600" height="62" border="0" ></a>


  </div>
  <div id="head3"><a href="javascript:" onClick="javascript:window.external.AddFavorite(window.location.href,document.title);">添加收藏</a>
  <br><a href="/help/setie.php">搜索引擎</a>
  <br><a href="/help/zanzhu.php">赞助本站</a></div>
  <div id="head2">
 <a href="http://www.guoxuemi.com/" target="_blank"><img src="http://www.guoxuedashi.com/img/guoxuemi.gif" width="95" height="62" border="0" style="margin-left:2px;" title="国学迷"></a>
  

  </div>
</div>
  <div class="clear"></div>
  <div class="head_nav">
  <p><a href="/">首页</a> | <a href="/ShuKu/">国学书库</a> | <a href="/guji/">影印古籍</a> | <a href="/shici/">诗词宝典</a> | <a   href="/SiKuQuanShu/gxjx.php">精选</a> <b>|</b> <a href="/zidian/">汉语字典</a> | <a href="/hydcd/">汉语词典</a> | <a href="http://www.guoxuedashi.com/zidian/bujian/"><font  color="#CC0066">部件查字</font></a> | <a href="http://www.sfds.cn/"><font  color="#CC0066">书法大师</font></a> | <a href="/jgwhj/">甲骨文</a> <b>|</b> <a href="/b/4/"><font  color="#CC0066">解密</font></a> | <a href="/renwu/">历史人物</a> | <a href="/diangu/">历史典故</a> | <a href="/xingshi/">姓氏</a> | <a href="/minzu/">民族</a> <b>|</b> <a href="/mz/"><font  color="#CC0066">世界名著</font></a> | <a href="/download/">软件下载</a>
</p>
<p><a href="/b/"><font  color="#CC0066">历史</font></a> | <a href="http://skqs.guoxuedashi.com/" target="_blank">四库全书</a> |  <a href="http://www.guoxuedashi.com/search/" target="_blank"><font  color="#CC0066">全文检索</font></a> | <a href="http://www.guoxuedashi.com/shumu/">古籍书目</a> | <a   href="/24shi/">正史</a> <b>|</b> <a href="/chengyu/">成语词典</a> | <a href="/kangxi/" title="康熙字典">康熙字典</a> | <a href="/ShuoWenJieZi/">说文解字</a> | <a href="/zixing/yanbian/">字形演变</a> | <a href="/yzjwjc/">金 文</a> <b>|</b>  <a href="/shijian/nian-hao/">年号</a> | <a href="/diming/">历史地名</a> | <a href="/shijian/">历史事件</a> | <a href="/guanzhi/">官职</a> | <a href="/lishi/">知识</a> <b>|</b> <a href="/zhongyi/">中医中药</a> | <a href="http://www.guoxuedashi.com/forum/">留言反馈</a>
</p>
  </div>
</div>
<!-- 头部导航END --> 
<!-- 内容区开始 --> 
<div class="w1180 clearfix">
  <div class="info l">
   
<div class="clearfix" style="background:#f5faff;">
<script src='http://www.guoxuedashi.com/img/headersou.js'></script>

</div>
  <div class="info_tree"><a href="http://www.guoxuedashi.com">首页</a> > <a href="/SiKuQuanShu/fanti/">四库全书</a>
 > <h1>资治通鉴</h1> <!--         下载:【右键另存为】即可 --></div>
  <div class="info_content zj clearfix">
  
<div class="info_txt clearfix" id="show">
<center style="font-size:24px;">12-資治通鑑卷十一</center>
    資治通鑑卷十一     宋司馬光 撰<br />
<br />
  胡三省 音註<br />
<br />
  漢紀三【起屠維大淵獻盡重光赤奮若凡三年】<br />
<br />
  太祖高皇帝中<br />
<br />
  五年冬十月漢王追項羽至固陵【徐廣曰固陵在陽夏晉灼曰即固始縣余據班志固始與陽夏為兩縣皆屬淮陽國劉昭志陳國陽夏縣有固陵聚括地志固陵縣名在陳州宛丘縣西北四十二里】與齊王信魏相國越期會擊楚信越不至楚擊漢軍大破之漢王復堅壁自守謂張良曰諸侯不從奈何對曰楚兵且破二人未有分地【李奇曰言信越未有益地之分也韋昭曰信等雖名為王未為分畫疆界分扶問翻余謂韋說是】其不至固宜君王能與共天下可立致也齊王信之立非君王意【言信自請為假王乃立之耳非君王本意】信亦不自堅彭越本定梁地始君王以魏豹故拜越為相國【見上卷二年】今豹死越亦望王而君王不早定今能取睢陽以北至穀城皆以王彭越【班志睢陽縣屬梁國劉昭志穀城縣屬東郡春秋之小穀也括地志穀城故城在濟州東阿縣東二十六里睢陽宋州也自宋州以北至濟州穀城際黄河盡以封彭越】從陳以東傅海與韓王信【陳古陳國班志之淮陽國也唐為陳州自陳以東至于海并齊舊地盡以與齊王信】信家在楚其意欲復得故邑能出捐此地以許兩人使各自為戰則楚易破也【易以豉翻】漢王從之於是韓信彭越皆引兵來十一月劉賈南渡淮圍夀春遣人誘楚大司馬周殷殷畔楚以舒屠六【舒春秋之舒國也班志舒縣屬廬江郡括地志舒今廬江之故舒城是也】舉九江兵迎黥布【史記正義曰九江郡即夀州楚考烈王二十二年徙夀春號曰郢至王負芻為秦所㓕置九江郡至唐為廬夀滁濠等州之地】並行屠城父隨劉賈皆會十二月項王至垓下【李奇曰沛洨縣聚邑名洨下交翻張揖三蒼注垓堤名在沛郡史記正義曰按垓下是高岡絶巖今猶高三四大其聚邑及堤在垓之側因取名焉今在亳州真源縣東十里垓音該】兵少食盡與漢戰不勝入壁漢軍及諸侯兵圍之數重【重直龍翻】項王夜聞漢軍四面皆楚歌【應劭曰楚歌者雞鳴歌也漢已畧得楚地故楚歌者多雞鳴時歌也師古曰楚歌者為楚人之歌猶吳歈越吟耳若以雞鳴為歌曲之名於理則可不得云雞鳴時也高祖令戚夫人楚舞自為作楚歌豈有雞鳴時乎】乃大驚曰漢皆已得楚乎是何楚人之多也則夜起飲帳中悲歌忼慨泣數行下【忼苦廣翻行戶剛翻泣目中淚也】左右皆泣莫能仰視於是項王乘其駿馬名騅【騅朱惟翻蒼白雜毛曰騅孔頴達曰雜毛是體有二種之色相間雜】麾下壯士騎從者八百餘人直夜潰圍南出馳走平明漢軍乃覺之令騎將灌嬰以五千騎追之項王渡淮騎能屬者纔百餘人【屬之欲翻】至隂陵【班志隂陵縣屬九江郡括地志隂陵故城在濠州定遠縣西北六十里】迷失道問一田父田父紿曰左【紿蕩亥翻欺誑也】左乃䧟大澤中以故漢追及之項王乃復引兵而東至東城【班志東城縣屬九江郡括地志東城故城在定遠東南五十里】乃有二十八騎漢騎追者數千人項王自度不能脱【度徒洛翻】謂其騎曰吾起兵至今八歲矣身七十餘戰未嘗敗北遂霸有天下然今卒困於此【卒子恤翻】此天之亡我非戰之罪也今日固決死願為諸君決戰必潰圍斬將刈旗三勝之令諸君知天亡我非戰之罪也乃分其騎以為四隊四鄉【鄉讀曰嚮】漢軍圍之數重項王謂其騎曰吾為公取彼一將令四面騎馳下期山東為三處於是項王大呼馳下漢軍皆披靡【呼火故翻披普彼翻史記正義曰靡言精體低垂】遂斬漢一將是時郎中騎楊喜追項王【郎中騎即漢官所謂騎郎】項王瞋目而叱之喜人馬俱驚辟易數里【辟頻益翻易如字師古曰辟易謂開張而易其故處宋祈國語補音易以豉翻未知其何據】項王與其騎會為三處漢軍不知項王所在乃分軍為三復圍之項王乃馳復斬漢一都尉殺數十百人復聚其騎亡其兩騎耳乃謂其騎曰何如騎皆伏曰如大王言於是項王欲東渡烏江【臣瓚曰烏江在牛渚索隱曰按晉初屬臨淮括地志烏江亭即和州烏江縣是也晉初為縣水經曰江水人北得黄律口漢書所謂烏江亭長檥舡待項王即此地余據烏江浦在今和州烏江縣東五十里即亭長檥船待羽處】烏江亭長檥船待【徐廣曰檥音儀一音俄應劭曰檥正也孟康曰檥音蟻附也附船著岸也如淳曰南方謂整船向岸曰檥索隱曰檥字諸家各以意解耳鄒誕本作様船以尚翻劉氏亦有此音】謂項王曰江東雖小地方千里衆數十萬人亦足王也願大王急渡今獨臣有船漢軍至無以渡項王笑曰天之亡我我何渡為且籍與江東子弟八千人渡江而西今無一人還縱江東父兄憐而王我我何面目見之縱彼不言籍獨不愧於心乎乃以所乘騅馬賜亭長令騎皆下馬步行持短兵接戰獨籍所殺漢軍數百人身亦被十餘創【被皮義翻創初良翻】顧見漢騎司馬呂馬童曰若非吾故人乎馬童面之【張晏曰以故人難親斫之故背之也如淳曰面謂不正視也師古曰如說非面謂背之不正向也面縛亦反偝而縛之杜元凱以為但見其面非也貢父曰面之直向之耳】指示中郎騎王翳曰此項王也項王乃曰吾聞漢購我頭千金邑萬戶【史記正義曰漢以一斤金為千金當一萬錢也余謂一斤金與萬戶邑多少不稱正義之說未可為據也】吾為若德【班書德作得鄧展曰令公得我以為功也史記作德徐廣曰亦可是功德之德史記正義曰為于偽翻言呂馬童與項羽先是故人舊有恩德於已余謂羽盖謂我為汝自刎以德汝】乃自刎而死【刎武粉翻】王翳取其頭餘騎相蹂踐【蹂人九翻】爭項王相殺者數十人最其後楊喜呂馬童及郎中呂勝楊武各得其一體五人共會其體皆是故分其戶封五人皆為列侯【呂馬童封中水侯王翳封杜衍侯楊喜封赤泉侯楊武封吳防侯呂勝封湼陽侯】楚地悉定獨魯不下【秦魯縣屬薛郡項羽初封於此漢為魯國】漢王引天下兵欲屠之至其城下猶聞絃誦之聲為其守禮義之國為主死節乃持項王頭以示魯父兄魯乃降漢王以魯公禮葬項王於糓城【宋白曰宋州穀熟縣古穀城也漢於此置薄縣又改為穀陽縣】親為發哀哭之而去【為于偽翻】諸項氏枝屬皆不誅封項伯等四人皆為列侯賜姓劉氏諸民略在楚者皆歸之<br />
<br />
  太史公曰羽起隴畮之中【畮古畝字】三年遂將五諸侯滅秦【此時山東六國而齊趙韓魏燕並起從羽伐秦故云五諸侯】分裂天下而封王侯政由羽出位雖不終近古以來未嘗有也及羽背關懷楚【師古曰背關謂背約不王沛公於關中懷楚謂思東歸彭城也余謂背關懷楚文意一貫言羽棄背關中之形勝而懷鄉歸楚也不必分為兩節背蒲妹翻】放逐義帝而自立怨王侯叛已難矣自矜功伐奮其私智而不師古謂霸王之業欲以力征經營天下五年卒亡其國【卒子恤翻】身死東城尚不覺悟而不自責乃引天亡我非用兵之罪也豈不謬哉<br />
<br />
  揚子法言或問楚敗垓下方死曰天也諒乎曰漢屈羣策羣策屈羣力【諒信也屈盡也】楚憞羣策而自屈其力【憞徒對翻惡也】屈人者克自屈者負天曷故焉【温公曰何預天事】<br />
<br />
  漢王還至定陶【班志定陶縣屬濟隂郡古之陶邑宋為廣濟軍理所】馳入齊王信壁奪其軍 臨江王共尉不降【共敖項羽封為臨江王尉其子也】遣盧綰劉賈擊虜之 春正月更立齊王信為楚王王淮北都下邳【更工衡翻】封魏相國建城侯彭越為梁王王魏故地都定陶 令曰兵不得休八年萬民與苦甚【如淳師古皆曰與弋庶翻貢父曰與讀曰歟助辭】今天下事畢其赦天下殊死以下【如淳曰殊死死罪之明白也左傳曰斬其木而弗殊韋昭曰殊死斬刑也師古曰殊絶也異也言其身首離絶而異處貢父曰予按說文漢蠻夷殊然則殊自死刑之名】 諸侯王皆上疏請尊漢王為皇帝二月甲午王即皇帝位于汜水之陽【蔡邕曰上古天子稱皇其次稱帝其次稱王秦承三王之末自以德兼三皇五帝故并以為號漢高受命因而不改張晏曰汜水在濟隂界取其汜愛弘大而潤下也師古曰據叔孫通傳為皇帝于定陶則此水在濟隂是也括地志漢高祖即位壇在曹州濟隂縣界汜敷劍翻】更王后曰皇后太子曰皇太子追尊先媪曰昭靈夫人【高祖母曰劉媼文頴曰幽州及漢中皆謂老嫗為媪師古曰媪女老稱音烏老翻】詔曰【如淳曰詔告也自秦漢以下惟天子獨稱之漢制度帝之下書有四一曰策書二曰制書三曰詔書四曰誡敕策書者編簡也其制長二尺短者半之篆書起年月日稱皇帝以命諸侯王三公以罪免亦賜策而以隸書用尺一木兩行此為異也制書帝者制度之命其文曰制詔三公皆璽封尚書令印重封露布州郡也詔書詔告也其文曰告某官如故事誡敕謂敕刺史太守其文曰有詔敕某官他皆倣此】故衡山王吳芮從百粤之兵佐諸侯誅暴秦有大功諸侯立以為王項羽侵奪之地謂之番君其以芮為長沙王【吳芮封衡山王都邾今刲長沙王都臨湘番蒲何翻】又曰故粤王無諸世奉粤祀秦侵奪其地使其社稷不得血食諸侯伐秦無諸身率閩中兵以佐滅秦項羽廢而弗立今以為閩粤王王閩中地【粤王無諸句踐之後秦取其地置閩中郡今復以封之師古曰閩越今泉州建安是其地徐廣曰今建安候官地史記正義曰今閩州又改為福應劭曰閩音文 飾之文師古曰非也音緡閩人本蛇種故其字從虫】 帝西都洛陽 夏五月兵皆罷歸家 詔民前或相聚保山澤不書名數今天下已定令各歸其縣復故爵田宅【復扶目翻還也】吏以文法教訓辨告【師古曰辨告者分别義理以曉喻之】勿笞辱軍吏卒爵及七大夫以上皆令食邑【臣瓚曰秦制列侯乃得食邑今七大夫以上皆食邑所以寵之也師古曰七大夫公大夫也爵第七故謂之七大夫】非七大夫已下皆復其身及戶勿事【應劭曰不輸戶賦也如淳曰事謂役使也師古曰復其身及一戶之内皆不傜賦也復方目翻】 帝置酒洛陽南宮【括地志南宮在洛州洛陽縣東北二十六里洛陽故城中輿地志秦時洛陽已有南北宮】上曰【蔡邕曰上者尊位所在也但言上不敢言尊號耳】徹侯諸將毋敢隱朕皆言其情【徹通也應劭曰言其功德通於王室也後避武諱改曰通侯亦曰列侯】吾所以有天下者何項氏之所以失天下者何高起王陵對曰【張晏曰詔使高官者起故陵先對臣瓚曰漢帝年紀有信平侯臣陵都武侯臣起魏相邴吉奏高祖時奏事有將軍臣陵臣起師古曰張說非也若言高官者起則丞相蕭何太尉盧綰及張良陳平之屬皆在陵不得而先對也先譜齊太公之後食采於高因氏焉】陛下使人攻城畧地因以與之與天下同其利項羽不然有功者害之賢者疑之此其所以失天下也上曰公知其一未知其二夫運籌帷幄之中決勝千里之外吾不如子房填國家撫百姓給餉餽不絶糧道吾不如蕭何【填讀曰鎮餽與饋同】連百萬之衆戰必勝攻必取吾不如韓信三者皆人傑吾能用之此吾所以取天下者也項羽有一范增而不能用此所以為我禽也羣臣說服【說讀曰悦】韓信至楚召漂母賜千金召辱已少年令出跨下者以為中尉【事見九卷元年漂匹妙翻】告諸將相曰此壯士也方辱我時我寜不能殺之邪殺之無名故忍而就此 彭越既受漢封田横懼誅與其徒屬五百餘人入海居島中【海中山曰島史記正義曰海州東海縣有島山去岸八十里余按北史楊愔避讒東入田横島是島以横居之而得名島丁老翻】帝以田横兄弟本定齊地齊賢者多附焉今在海中不取後恐為亂乃使使赦横罪召之横謝曰臣烹陛下之使酈生【事見上卷四年】今聞其弟商為漢將臣恐懼不敢奉詔請為庶人守海島中使還報帝乃詔衛尉酈商曰【班表衛尉秦官掌宮門衛屯兵】齊王田横即至人馬從者敢動搖者致族夷【從才用翻言誅夷其族也】乃復使使持節具告以詔商狀【周禮司節掌守邦節辨其用以輔王命註云節者執以行為信邦節珍圭牙璋糓圭琬圭琰圭也守邦國用玉節以玉為之守都鄙用角節以角為之邦國之使節用金門關之節用符貨賄之節用璽道路之節用旌審此則古之所執以為信者皆謂之節自秦以來有璽符節則璽自璽符自符節自節分為三矣漢之節即古之旌節也鄭氏註以符節為漢宮中諸官詔符璽節為漢之印章旌節為漢使者所持節則知漢所謂節盖古之旌節也賢曰節者所以為信以竹為之柄長八尺以旄牛尾為之毦三重此漢制也】曰田横來大者王小者乃侯耳不來且舉兵加誅焉横乃與其客二人乘傳詣洛陽【如淳曰駟馬高足為置傳中足為馳傳下足為乘傳一馬二馬為軺傳急者乘一乘傳師古曰盖今之驛古者以車謂之傳車其後單置馬謂之驛騎漢律諸當乘傳及發駕置傳者皆持尺五寸木傳信封以御史大夫印章其乘傳參封之參三也有期會累封兩端端各兩封凡四封乘置馳傳五封之兩端各二中央一軺傳兩馬再封之一馬一封以馬駕軺車而乘傳曰一封軺傳史炤所謂依乘符傳而行者本此但擇焉而不精語焉而不詳耳終不若顏說簡而朙傳張戀翻】未至三十里至尸鄉廏置【應劭曰尸鄉在偃師城西臣瓚曰案廏置謂置馬以傳驛者】横謝使者曰人臣見天子當洗沐因止留謂其客曰横始與漢王俱南面稱孤【師古曰王者自稱曰孤盖為謙也老子道德經曰貴以賤為本高以下為基是以侯王自謂孤寡不糓】 今漢王為天子而横乃為亡虜北面事之其恥固已甚矣且吾烹人之兄與其弟併肩而事主【併步頂翻】縱彼畏天子之詔不敢動我獨不媿於心乎且陛下所以欲見我者不過欲一見吾面貌耳今斬吾頭馳三十里間形容尚未能敗猶可觀也遂自剄令客奉其頭從使者馳奏之帝曰嗟乎起自布衣兄弟三人更王豈不賢哉【更工衡翻】為之流涕而拜其二客為都尉發卒二千人以王者禮葬之【史記正義曰田横墓在偃師西十五里】既葬二客穿其冢傍孔皆自剄下從之帝聞之大驚以横客皆賢餘五百人尚在海中使使召之至則聞田横死亦皆自殺 初楚人季布為項籍將【季姓也周八士有季隨季騧魯有季氏】數窘辱帝【數所角翻窘巨隕翻困也】項籍滅帝購求布千金敢有舍匿罪三族【舍止也匿隱也】布乃髠鉗為奴自賣於魯朱家【髠枯昆翻鬄其髮也鉗其炎翻以鐵束項朱家魯之大俠】朱家心知其季布也買置田舍身之洛陽見滕公說曰季布何罪臣各為其主用職耳【師古曰職常也言此乃常道也一曰職主掌其事也為于偽翻】項氏臣豈可盡誅邪今上始得天下而以私怨求一人何示不廣也且以季布之賢漢求之急此不北走胡南走越耳夫忌壯士以資敵國此伍子胥所以鞭荆平之墓也【伍子胥楚大夫伍奢之子也楚平王信讒而殺伍奢子胥奔吳藉吳師以破楚入郢發平王墓而鞭其尸】君何不從容為上言之【從千容翻】滕公待間言於上如朱家指上乃赦布召拜郎中朱家遂不復見之【復扶又翻】布母弟丁公亦為項羽將逐窘帝彭城西短兵接帝急顧謂丁公曰兩賢豈相戹哉【孟康曰丁公及彭城賴齮追上故曰兩賢也師古曰孟說非也兩賢者高祖自謂併與固也言吾與固俱是賢豈相戹困哉故固感此言而止也雖與賴齮同追而高祖獨與固言也姓譜丁本自姜姓齊太公子諡丁公因以命氏】丁公引兵而還及項王滅丁公謁見【見賢遍翻】帝以丁公徇軍中【徇辭峻翻師古曰行示也】曰丁公為項王臣不忠使項王失天下者也遂斬之曰使後為人臣無傚丁公也<br />
<br />
  臣光曰高祖起豐沛以來罔羅豪桀招亡納叛亦已多矣及即帝位而丁公獨以不忠受戮何哉夫進取之與守成其勢不同當羣雄角逐之際民無定主來者受之固其宜也及貴為天子四海之内無不為臣苟不明禮義以示之使為臣者人懷貳心以徼大利則國家其能久安乎是故斷以大義【斷丁亂翻】使天下曉然皆知為臣不忠者無所自容而懷私結恩者雖至於活已猶以義不與也戮一人而千萬人懼其慮事豈不深且遠哉子孫享有天禄四百餘年宜矣<br />
<br />
  齊人婁敬戍隴西【姓譜婁邾婁國之後一曰離婁之後】過洛陽脱輓輅【蘇林曰輅音凍之一木横遮車前二人輓之三人推之師古曰輓音晚輅胡格翻音同】衣羊裘因齊人虞將軍求見上虞將軍欲與之鮮衣婁敬曰臣衣帛衣帛見衣褐衣褐見【衣著也帛繒也褐織毛布之衣也】終不敢易衣於是虞將軍入言上上召見問之婁敬曰陛下都洛陽豈欲與周室比隆哉上曰然婁敬曰陛下取天下與周異周之先自后稷封邰【班志邰縣屬右扶風師古曰即今武功故城是史記正義曰雍州武功縣西南二十三里故漦城是也說文曰邰炎帝之後姜姓所封國棄外家也毛萇云邰姜嫄國堯以天因邰而生后稷故因封之於邰音吐才翻】積德絫善【絫古累字】十有餘世至于太王王季文王武王而諸侯自歸之遂滅殷為天子及成王即位周公相焉乃營洛邑以為此天下之中也諸侯四方納貢職道里均矣有德則易以王無德則易以亡故周之盛時天下和洽諸侯四夷莫不賓服効其貢職及其衰也天下莫朝【朝直遥翻】周不能制也非唯其德薄也形勢弱也今陛下起豐沛卷蜀漢定三秦【卷讀曰捲】與項羽戰滎陽成臯之間大戰七十小戰四十使天下之民肝腦塗地父子暴骨中野不可勝數【勝音升】哭泣之聲未絶傷夷者未起【夷與痍同創也音延知翻】而欲比隆於成康之時臣竊以為不侔也且夫秦地被山帶河四塞以為固卒然有急【卒讀曰猝】百萬之衆可立具也因秦之故資甚美膏腴之地此所謂天府者也【府聚也萬物所聚謂之天府】陛下入關而都之山東雖亂秦之故地可全而有也夫與人鬭不搤其亢拊其背未能全其勝也【張晏曰搤與扼同捉持之也亢音岡又下郎翻喉嚨也】今陛下案秦之故地此亦搤天下之亢而拊其背也帝問羣臣羣臣皆山東人爭言周王數百年秦二世即亡洛陽東有成皐西有殽澠【師古曰殽謂殽山今陜州東二殽山是也澠即澠池】倍河鄉伊洛【河在洛陽城北故曰倍伊洛二水在洛陽城南故曰鄉倍蒲妹翻鄉讀曰嚮】其固亦足恃也上問張良良曰洛陽雖有此固其中小不過數百里田地薄四面受敵此非用武之國也關中左殽函右隴蜀沃野千里【師古曰沃者溉灌也言其土地皆有溉灌之利故曰沃野】南有巴蜀之饒北有胡苑之利【師古曰謂安定北地上郡之北與胡相接之地可以畜牧者也養禽獸謂之苑音於阮翻】阻三面而守獨以一面東制諸侯諸侯安定河渭漕輓天下西給京師【漢漕關東之粟自河入渭自渭而上輸之長安】諸侯有變順流而下足以委輸【康曰委於偽切即委積之委輸即轉輪之輸輸舂遇翻】此所謂金城千里天府之國也【府者物所聚也天物所聚不假人力故曰天府】婁敬說是也上即日車駕西都長安拜婁敬為郎中號曰奉春君賜姓劉氏【師古曰凡言車駕謂天子乘車而行不敢指斥也長安本秦之鄉名也高祖作都奉春君張晏曰春歲之始也今婁敬發事之始故曰奉春君也】張良素多病從上入關即道引不食穀【孟康曰道讀曰導服辟穀藥而静居行氣】杜門不出曰家世相韓及韓滅不愛萬金之資為韓報讐彊秦天下振動【事見七参秦始皇二十九年】今以三寸舌為帝者師封萬戶侯此布衣之極於良足矣願棄人間事欲從赤松子游耳【師古曰赤松子仙人號也神農時為雨師服水王教神農能入火自燒至昆山上常止西王母石室隨風雨上下炎帝少女追之亦得仙俱去】<br />
<br />
  臣光曰夫生之有死譬猶夜旦之必然自古及今固未有超然而獨存者也以子房之明辨達理足以知神僊之為虛詭矣然其欲從赤松子游者其智可知也夫功名之際人臣之所難處【處昌呂翻】如高帝所稱者三傑而已淮隂誅夷蕭何繫獄非以履盛滿而不止耶故子房託於神僊遺棄人間等功名於外物置榮利而不顧所謂明哲保身者【詩云既明且哲以保其身】子房有焉<br />
<br />
  六月壬辰大赦天下 秋七月燕王臧荼反上自將征之 趙景王耳長沙文王芮皆薨 九月虜臧荼壬子立太尉長安侯盧綰為燕王【班表太尉秦官掌武事漢制與丞相御史大夫為三公應劭曰自上安下曰尉據史記盧綰傳長安故咸陽也正義曰秦咸陽在渭北長安在渭南蕭何起未央宮之處】綰家與上同里閈【閈音汗閭也里門曰閈】綰生又與上同日上寵幸綰羣臣莫敢望故特王之【考異曰史記漢書高紀於此皆云使丞相噲將兵平代地按樊噲傳從平韓王信乃遷左丞相是時未為丞相又代地無反者噲傳亦無此事疑紀誤】項王故將利幾反【利幾以陳令降上侯之潁川上至洛陽召之利幾恐而反風俗通利姓也姓譜楚公子食采於利後以為氏】上自擊破之 後九月治長樂宮【程大昌雍録曰長樂宮本秦之興樂宮周迴二十里高祖改修而居之在長安城東隅樂音洛】 項王將鍾離昧素與楚王信善【昧莫曷翻下同】項王死後亡歸信漢王怨昧聞其在楚詔楚捕昧信初之國行縣邑陳兵出入【行丁孟翻】<br />
<br />
  六年冬十月人有上書告楚王信反者帝以問諸將皆曰亟發兵阬豎子耳帝默然又問陳平陳平曰人上書言信反信知之乎曰不知陳平曰陛下精兵孰與楚上曰不能過平曰陛下諸將用兵有能過韓信者乎上曰莫及也平曰今兵不如楚精而將不能及舉兵攻之是趣之戰也【趣讀曰促】竊為陛下危之上曰為之奈何平曰古者天子有巡狩會諸侯【白虎通曰天子所以巡狩者何巡者循也狩者收也謂循行天下收人道德太平恐遠近不同政化幽隱有不得其所者故必自行之謹敬重民之意也孟子曰天子適諸侯曰巡守巡守者巡所守也】陛下第出偽游雲夢【第但也】會諸侯於陳陳楚之西界信聞天子以好出游其埶必無事而郊迎謁謁而陛下因禽之此特一力士之事耳帝以為然乃發使告諸侯會陳吾將南游雲夢上因隨以行楚王信聞之自疑懼不知所為或說信曰斬鍾離昧以謁上上必喜無患信從之十二月上會諸侯於陳信持昧首謁上上令武士縛信載後車信曰果若人言狡兎死走狗烹高鳥盡良弓藏敵國破謀臣亡【師古曰黄石公三畧之言】天下已定我固當烹上曰人告公反遂械繫信以歸【械者加以杻械繫者加以徽索】因赦天下田肯賀上曰陛下得韓信又治秦中【如淳曰山東人謂關中為秦中師古曰謂關中秦地也】秦形勝之國也【張晏曰得形勢之勝便也】帶河阻山地埶便利其以下兵於諸侯譬猶居高屋之上建瓴水也【如淳曰瓴盛水瓶也居高屋之上而翻瓴水言其向下之勢順也建居偃翻瓴音鈴】夫齊東有琅邪即墨之饒【師古曰二縣近海財用之所出】南有泰山之固【泰山在齊之南境齊負以為固】西有濁河之限【晋灼曰齊西有平原河水東北過高唐高唐即平原也孟津號黄河故曰濁河也余謂孟津在河内去平原甚遠晉說失之拘盖河流渾濁故謂之濁河也】北有勃海之利【索隱曰崔浩云勃旁跌也旁跌出者横在濟北故齊都賦云海旁出為勃名曰勃海郡余據班志齊地北至勃海冇高樂高城陽信重合之地】地方二千里持戟百萬此東西秦也【言齊地形勝與秦亢衡也】非親子弟莫可使王齊者上曰善賜金五百斤上還至洛陽赦韓信封為淮隂侯信知漢王畏惡其能【惡烏路翻】多稱病不朝從【朝直遥翻朝見也從才用翻從遊也】居常鞅鞅羞與絳灌等列【鞅鞅志不滿也音於兩翻絳侯周勃灌將軍嬰】嘗過樊將軍噲噲跪拜送迎言稱臣曰大王乃肯臨臣信出門笑曰生乃與噲等為伍【為信怨望謀反張本】上嘗從容與信言諸將能將兵多少【從千容翻將即亮翻下同】上問曰如我能將幾何信曰陛下不過能將十萬上曰於君何如曰臣多多而益善耳上笑曰多多益善何為為我禽信曰陛下不能將兵而善將將此乃信之所以為陛下禽也且陛下所謂天授非人力也 甲申始剖符封諸功臣為徹侯【師古曰剖破也與其合符而分授之也剖普口翻】蕭何封酇侯【班志酇縣屬南陽郡孟康曰酇音讚】所食邑獨多【按班書功臣表蕭何封酇八千戶而曹參封平陽張良封留皆萬戶宜不得言何封邑獨多盖參以十二月甲申封何以正月丙午封功臣言何居上其意不能平者特同日受封樊酈絳灌諸人耳張良亦以丙午封諸人言何而不言良者盖高祖先使艮自擇齊三萬戶而良止受留萬戶故不敢言也】功臣皆曰臣等身被堅執鋭【被皮義翻】多者百餘戰少者數十合今蕭何未嘗有汗馬之勞徒持文墨議論顧反居臣等上何也帝曰諸君知獵乎夫獵追殺獸兎者狗也而發縱指示獸處者人也今諸君徒能得走獸耳功狗也至如蕭何發縱指示功人也【師古曰發縱謂解紲而放之也指示以手指示之今俗言放狗縱子用翻而讀者乃為蹤蹟之蹤非也書本皆不為蹤字自有逐蹤之狗不待人發也洪氏隸釋曰元祐中洺州治河堤得漢北海淳于長夏君碑其辭有曰紹縱先軌又北軍中侯郭仲奇碑云有山甫之縱又云徽縱顯又司隸校尉魯峻碑云比縱豹產又圉令趙君碑云羨其縱外黄令高彪碑云莫與比縱皆以縱為蹤蕭何傳發縱指示獸處顏師古注云書本皆不為蹤字讀者乃為蹤蹟之蹤非也據此數碑則漢人固多借用顔氏之注殆未然也】羣臣皆不敢言張良為謀臣亦無戰鬭功帝使自擇齊三萬戶良曰始臣起下邳與上會留【見八卷秦二世二年】此天以臣授陛下陛下用臣計幸而時中【中竹仲翻】臣願封留足矣不敢當三萬戶乃封張良為留侯封陳平為戶牖侯【戶牖鄉名屬陳留郡陽武縣徐廣曰陽武屬魏地戶牖今為東昏縣屬陳留索隱曰徐廣云陽武屬魏而地理志屬河南郡盖後陽武屬梁國耳徐又云戶牖今為東昏縣屬陳留與班書地理志同按是秦時戶牖鄉屬陽武至漢以戶牖鄉為東昏縣隸陳留郡也括地志東昏故城在汴州陳留縣東北九十里陳平亦十二月甲申封】平辭曰此非臣之功也上曰吾用先生謀戰勝克敵非功而何平曰非魏無知臣安得進上曰若子可謂不背本矣乃復賞魏無知【平因無知見上事見九卷二年背蒲妹翻復扶又翻】 帝以天下初定子幼昆弟少懲秦孤立而亡欲大封同姓以填撫天下【填讀曰鎮】春正月丙午分楚王信地為二國以淮東五十三縣立從兄將軍賈為荆王【索隱曰乃王吳地在淮東也余據班史時以故東陽郡鄣郡吳郡五十三縣王賈東陽漢下邳地鄣郡漢丹陽地吳郡即會稽地盖其地自淮東而南盡丹陽會稽也賈死後以其地王吳王濞故索隱云王吳地也如淳曰荆亦楚也賈逵曰秦莊襄王名楚故改曰荆遂行於世晋灼曰奮伐荆楚自秦之先固已稱荆索隱曰姚察按虞喜云總言荆者以山命國也今西南有荆山在陽羨界賈分封吳地而號荆王指取此義太康地志陽羨縣本名荆溪從才用翻】以薛郡東海彭城三十六縣立弟文信君交為楚王【薛郡漢之魯國東海秦之郯郡彭城後為楚國盖封交之時得三郡地景武之後楚國僅彭城數縣耳】壬子以雲中鴈門代郡五十三縣立兄宜信侯喜為代王以膠東膠西臨菑濟北博陽城陽郡七十三縣立微時外婦之子肥為齊王【據此則博陽於秦楚漢兵爭之時亦嘗置郡矣自淮東至此雜用古地名固不純用秦漢所置郡名也師古曰外婦謂與旁通者】諸民能齊言者皆以與齊【孟康曰此時民流移故使能齊言者還齊也史記正義曰按言齊國形勝次於秦中故以封子肥七十餘城近齊城邑能齊言者咸割屬齊親子故大其都也孟說恐非】 上以韓王信材武所王北近鞏洛南迫宛葉東有淮陽【韓之分晋其地南至宛葉西北包鞏洛接於新安宜陽東有潁川而淮陽之地則屬于楚及漢定天下韓王信剖符王潁川其地東兼有淮陽所謂北近南迫言其境相迫近耳不屬韓也宛於元翻葉式涉翻】皆天下勁兵處乃以太原郡三十一縣為韓國徙韓王信王太原以北備禦胡都晉陽信上書曰國被邊匈奴數入寇晉陽去塞遠請治馬邑【班志太原郡領二十一縣今以三十一縣為韓國盖定襄未置郡故太原之地北被邉兼有鴈門之馬邑也晋太康地記曰秦時建此城輒崩不成有馬周旋走反覆父老異之因依以築城遂名馬邑杜佑曰秦馬邑城在朔州善陽縣界李奇曰被音被馬之被師古曰被猶帶也皮義翻數所角翻】上許之 上已封大功臣二十餘人其餘日夜爭功不決未得行封上在洛陽南宮從複道望見諸將往往相與坐沙中語上曰此何語留侯曰陛下不知乎此謀反耳上曰天下屬安定【屬近也言近方安定也屬之欲翻】何故反乎留侯曰陛下起布衣以此屬取天下【屬殊玉翻】今陛下為天子而所封皆故人所親愛所誅皆生平所仇怨今軍吏計功以天下不足徧封此屬畏陛下不能盡封恐又見疑平生過失及誅故即相聚謀反耳上乃憂曰為之奈何留侯曰上平生所憎羣臣所共知誰最甚者上曰雍齒與我有故怨數嘗窘辱我【服䖍曰未起之時與我有故怨也師古曰每以勇力困辱高祖余觀帝初起令雍齒守豐齒雅不欲屬帝即以豐降魏可以見其有故怨矣雍於用翻數所角翻】我欲殺之為其功多故不忍【為于偽翻】留侯曰今急先封雍齒則羣臣人人自堅矣於是上乃置酒封雍齒為什方侯【蘇林曰什方漢中縣也師古曰地理志屬廣漢非漢中也今則屬益州什音十余按唐志什邡縣屬漢州盖垂拱又分益州置漢州也宋白曰什方縣舊治雍齒城今於城北四十步立縣】而急趨丞相御史定功行封【趨讀曰促漢之三公丞相職無不總御史大夫掌副丞相】羣臣罷酒皆喜曰雍齒尚為侯我屬無患矣<br />
<br />
  臣光曰張良為高帝謀臣委以心腹宜其知無不言安有聞諸將謀反必待高帝目見偶語然後乃言之邪盖以高帝初得天下數用愛憎行誅賞【數所角翻】或時害至公羣臣往往有觖望自危之心【觖古穴翻師古曰音决觖謂相觖也望怨望也韋昭曰觖猶冀也音冀索隱音企】故良因事納忠以變移帝意使上無阿私之失下無猜懼之謀國家無虞利及後世若良者可謂善諫矣<br />
<br />
  列侯畢已受封詔定元功十八人位次【師古曰謂蕭何曹參張敖周勃樊噲酈商奚涓夏侯嬰灌嬰傅寛靳歙王陵陳武王吸薛歐周昌丁復蟲達自第一至十八也余謂此但定蕭何等元功十八人位次耳至呂后時乃詔作高祖功臣位次凡一百四十餘人師古所謂自蕭何至蟲達十八人呂后所定位次也張敖於高祖九年始自趙王廢為宣平侯安得預元功十八人之數哉故師古註功臣位次云張耳及敖並為無大功盖以魯元之故呂后曲升之耳此說則得之】皆曰平陽侯曹參身被七十創攻城畧地功冣多宜第一【被皮義翻創初良翻】謁者關内侯鄂千秋進曰羣臣議皆誤【鄂本出姬姓晉鄂侯之後關内侯位次列侯爵第十九師古曰言有侯號而居京畿無國邑】夫曹參雖有野戰略地之功此特一時之事耳上與楚相距五歲失軍亡衆跳身遁者數矣【師古曰謂輕身走出也數所角翻下同】然蕭何常從關中遣軍補其處非上所詔令召而數萬衆會上之乏絶者數矣又軍無見糧【見賢遍翻】蕭何轉漕關中給食不乏陛下雖數亡山東蕭何常全關中以待陛下此萬世之功也今雖無曹參等百數何缺於漢漢得之不必待以全奈何欲以一旦之功而【口力】萬世之功哉蕭何第一曹參次之上曰善於是乃賜蕭何帶劔履上殿入朝不趨【古者君子必帶劔所以衛身且昭武備也秦法羣臣上殿不得持尺寸之兵草曰厞麻曰屨皮曰履屨履所以從軍軍容不入國故皆不許以上殿君前必趋崇敬也今賜何劍履上殿入朝不趋殊禮也】上曰吾聞進賢受上賞蕭何功雖高得鄂君乃益明於是因鄂千秋所食邑封為安平侯【索隱曰安平縣屬涿郡非甾川之東安平縣】是日悉封何父子兄弟十餘人皆有食邑益封何二千戶 上歸櫟陽夏五月丙午尊太公為太上皇【師古曰太上者極尊之稱也皇君也天子之父故號曰皇不預治國故不言帝】 初匈奴畏秦北徙十餘年及秦滅匈奴復稍南渡河【此北河也在朔方北】單于頭曼有太子曰冒頓【韋昭曰冒音瞞師古曰莫安翻索隱曰冒音墨又莫報翻】後有所愛閼氏【匈奴之閼氏猶中國之皇后閼於連翻氏音支下月氏同】生少子頭曼欲立之【少詩照翻】是時東胡彊而月氏盛【括地志凉肅甘沙庭州本月氏地】乃使冒頓質於月氏【質音致】既而頭曼急擊月氏月氏欲殺冒頓冒頓盜其善馬騎之亡歸頭曼以為壯令將萬騎冒頓乃作鳴鏑【應劭曰髐箭也韋昭曰矢鏑飛則鳴余見今軍中亦有鳴鏑於近笴之處開小竅矢飛急則凌風而鳴鏑音嫡髐呼交翻】習勒其騎射【勒其所部使習其令也】令曰鳴鏑所射而不悉射者斬之冒頓乃以鳴鏑自射其善馬既又射其愛妻左右或不敢射者皆斬之最後以鳴鏑射單于善馬左右皆射之於是冒頓知其可用從頭曼獵以鳴鏑射頭曼其左右亦皆隨鳴鏑而射【射而亦翻】遂殺頭曼盡誅其後母與弟及大臣不聽從者冒頓自立為單于東胡聞冒頓立乃使使謂冒頓欲得頭曼時千里馬冒頓問羣臣羣臣皆曰此匈奴寶馬也勿與冒頓曰奈何與人鄰國而愛一馬乎遂與之居頃之東胡又使使謂冒頓欲得單于一閼氏冒頓復問左右【復扶又翻】左右皆怒曰東胡無道乃求閼氏請擊之冒頓曰奈何與人鄰國愛一女子乎遂取所愛閼氏予東胡【予讀曰與下同】東胡王愈益驕東胡與匈奴中間有棄地莫居千餘里各居其邉為甌脱【服䖍曰甌脱作土室以伺也師古曰境上候望之處若今之伏宿處也甌一侯翻脱土活翻】東胡使使謂冒頓此棄地欲有之冒頓問羣臣羣臣或曰此棄地予之亦可勿與亦可於是冒頓大怒曰地者國之本也奈何予之諸言予之者皆斬之冒頓上馬令國中有後出者斬遂襲擊東胡東胡初輕冒頓不為備冒頓遂滅東胡既歸又西擊走月氏南并樓煩白羊河南王【師古曰樓煩白羊二王之居在河南】遂侵燕代悉復收蒙恬所奪匈奴故地【蒙恬奪匈奴地見七卷秦始皇三十一年】與漢關故河南塞至朝那膚施【班志朝那縣屬安定郡膚施縣屬上郡史記正義曰漢朝那故城在原州百泉縣西七十里膚施縣趙置秦因而不改今屬延州】是時漢兵方與項羽相距中國罷於兵革【罷讀曰疲】以故冒頓得自彊控弦之士三十餘萬【控弦引弓也控口弄翻】威服諸國秋匈奴圍韓王信於馬邑信數使使胡求和解漢發兵救之疑信數間使有二心【數所角翻間古莧翻使疏吏翻】使人責讓信信恐誅九月以馬邑降匈奴匈奴冒頓因引兵南踰句注【郡國志句注山險名在鴈門隂舘縣括地志句注山在代州鴈門縣西北三十里杜佑曰句注山即代州鴈門縣西陘嶺句音鉤又如字又音拘】攻太原至晉陽 帝悉去秦苛儀法為簡易【去羌呂翻除也後以義推易以豉翻下同】羣臣飲酒爭功醉或妄呼拔劔擊柱【呼火故翻】帝益厭之叔孫通說上曰【叔孫本出姬姓魯叔孫氏之後】夫儒者難與進取可與守成臣願徵魯諸生與臣弟子共起朝儀【朝直遥翻】帝曰得無難乎叔孫通曰五帝異樂三王不同禮禮者因時世人情為之節文者也臣願頗采古禮與秦儀雜就之上曰可試為之令易知度吾所能行者為之【易以豉翻度徒翻】於是叔孫通使徵魯諸生三十餘人【師古曰通為使者而徵魯諸生使疏吏翻】魯有兩生不肯行曰公所事者且十主皆面諛以得親貴【通事秦始皇二世陳涉項梁楚懷王項羽及帝凡七主且幾也言幾及十主也】今天下初定死者未葬傷者未起又欲起禮樂禮樂所由起積德百年而後可興也【師古曰言行德教百年然後可起禮樂】吾不忍為公所為公去矣無汙我【汙烏故翻】叔孫通笑曰若真鄙儒也不知時變【師古曰若汝也鄙言不通】遂與所徵三十人西【師古曰西入關】及上左右為學者【師古曰左右謂近臣也為學謂素有學術】與其弟子百餘人為綿蕞野外習之【應劭曰立竹及茅索營之習禮儀其中也如淳曰謂以茅剪樹也為纂位尊卑之次也春秋傳曰置茅蕝師古曰蕝與蕞同子悦翻如說是韋昭曰引繩為綿立表為蕞蕞兹會翻賈逵曰束茅以立表位為蕝纂文曰蕝今之纂字即悦翻又音纂】月餘言於上曰可試觀矣上使行禮曰吾能為此乃令羣臣習肄【肄弋二翻亦習也】<br />
<br />
  七年冬十月長樂宮成諸侯羣臣皆朝賀【時未起未央宫故帝御長樂宮受朝賀及蕭何既起未央前殿自惠帝以後皆御未央而長樂為太后所居謂之東朝樂音洛】先平明【師古曰未平明之前先悉薦翻】謁者治禮以次引入殿門陳東西鄉【治直之翻鄉讀曰嚮】衛官俠陛【衛官侍衛之官郎中及中郎執戟侍衛者是也俠與挾同挾殿陛之兩旁也或音夾】及羅立廷中皆執兵張旗幟【幟昌志翻】於是皇帝傳警【漢儀云帝輦動則左右侍帷幄者稱警是也漢書音義天子出稱警傳聲而唱以警外也】輦出房【沈約曰輦車周禮王后五路之卑者也后從容宮中所乘非王車也漢制乘輿御之或使人輓或駕果下馬不知何代去其輪】引諸侯王以下至吏六百石【漢吏六百石銅印墨綬奉月七十斛】以次奉賀莫不振恐肅敬至禮畢復置法酒【禮畢謂朝禮畢也師古曰法酒猶言禮爵謂不飲之至醉】諸侍坐殿上皆伏抑首【師古曰抑屈也謂依禮法不敢平坐而視】以尊卑次起上夀觴九行謁者言罷酒御史執法舉不如儀者輒引去【執法即御史也杜佑曰御史之名周官有之盖掌贊書而授法令非今任也戰國時亦有御史秦趙澠池之會各令書其事秦漢為糾察之任秦以御史監郡漢初定禮儀御史執法舉不如儀者輒引去是也】竟朝置酒無敢讙譁失禮者【竟朝言行朝禮至禮畢也朝直遥翻讙與喧同許元翻】於是帝曰吾乃今日知為皇帝之貴也乃拜叔孫通為太常【班表奉常秦官掌宗廟禮儀景帝中六年改曰太常此不書奉常而書太常者使人易知】賜金五百斤初秦有天下悉内六國禮儀采擇其尊君抑臣者存之及通制禮頗有所增損大抵皆襲秦故自天子稱號下至佐僚及宫室官名少所變改其書後與律令同錄藏於理官【師古曰理官即法官也】法家又復不傳民臣莫有言者焉<br />
<br />
  臣光曰禮之為物大矣用之於身則動静有法而百行備焉用之於家則内外有别而九族睦焉【行下孟翻别彼列翻】用之於鄉則長幼有倫而俗化美焉用之於國則君臣有叙而政治成焉【治直吏翻】用之於天下則諸侯順服而紀綱正焉豈直几席之上戶庭之間得之而不亂哉夫以高祖之明達聞陸賈之言而稱善【見下卷十一年】睹叔孫之儀而歎息然所以不能肩於三代之王者病於不學而已當是時得大儒而佐之與之以禮為天下其功烈豈若是而止哉惜夫叔孫生之器小也徒竊禮之糠粃以依世諧俗取寵而已【穀皮曰糠穀不成曰粃粃與秕同】遂使先王之禮淪没而不振以迄于今豈不痛甚矣哉是以揚子譏之曰昔者魯有大臣史失其名曰何如其大也曰叔孫通欲制君臣之儀召先生於魯所不能致者二人曰若是則仲尼之開迹諸侯也非邪曰仲尼開迹將以自用也【宋咸曰謂開布其迹於諸侯之國猶言歷聘也】如委已而從人雖有規矩準繩焉得而用之【焉於䖍翻】善乎揚子之言也夫大儒者惡肯毀其規矩準繩以趋一時之功哉【惡音烏趋七喻翻】<br />
<br />
  上自將擊韓王信破其軍於銅鞮【班志銅鞮縣屬上黨郡上黨記晋銅鞮伯華所邑去銅鞮故宮二十里唐屬潞州宋白曰縣有銅鞮水故名鞮丁奚翻】斬其將王喜信亡走匈奴白土人曼丘臣王黄等立趙苗裔趙利為王【班志白土縣屬上郡括地志白土故城在鹽州白池東北九十里又云近延州余據班志圜水出白土縣西東入河師古曰圜音銀今銀州銀水是則白土縣在唐銀州界按圜字乃圁字之誤通典圁水在銀州儒林縣東北今謂之無定河師古又曰曼丘毌丘本一姓也語有緩急耳曼音萬姓譜齊有曼丘不擇】復收信敗散兵與信及匈奴謀攻漢匈奴使左右賢王將萬餘騎與王黄等屯廣武以南至晉陽【班志匈奴置左右賢王左右谷蠡王最為大國班志廣武縣屬太原郡史記正義廣武故城在代州鴈門界句注山南杜佑曰代州鴈門郡治鴈門縣漢廣武縣故城在西南宋白曰隋改廣武縣為鴈門避太子諱也】漢兵擊之匈奴輒敗走已復屯聚漢兵乘勝追之會天大寒雨雪【大戴記曰盛隂之氣在雨水則凝滯而為雪雨于具翻自上而下曰雨後以義推】士卒墮指者什二三【師古曰什人之中二三墮指】上居晉陽聞冒頓居代谷【史記正義曰代谷今媯州余據唐媯州在幽州西北此代谷在句注之北後魏都平城建為代都盖因代谷而名也唐屬雲州界】欲擊之使人覘匈奴【覘丑廉翻又勑艷翻窺偵也】冒頓匿其壯士肥牛馬但見老弱及羸畜【羸倫為翻畜許救翻】使者十輩來皆言匈奴可擊上復使劉敬往使匈奴【復扶又翻】未還漢悉兵三十二萬北逐之踰句注劉敬還報曰兩國相擊此宜夸矜見所長【見賢遍翻示也下欲見同】今臣往徒見羸瘠老弱此必欲見短伏奇兵以爭利愚以為匈奴不可擊也是時漢兵已業行【凡事巳為而未成曰業】上怒罵劉敬曰齊虜以口舌得官今乃妄言沮吾軍【沮才汝翻止也】械繫敬廣武帝先至平城兵未盡到冒頓縱精兵四十萬騎圍帝於白登七日【班志平城縣屬鴈門郡服䖍曰白登臺名去平城七里師古曰白登在平城東南去平城十餘里括地志朔州定襄縣本漢平城縣東北三十里有白登山山上有臺名曰白登臺】漢兵中外不得相救餉帝用陳平袐計使使間厚遺閼氏【應劭曰陳平使畫工圖美女間遺閼氏曰漢有美女如此今皇帝用急欲獻之閼氏畏其奪已寵言於冒頓令解圍余謂袐計者以其失中國之體故袐而不傳間古莧翻遺于季翻】閼氏謂冒頓曰兩主不相困今得漢地而單于終非能居之也且漢主亦有神靈單于察之冒頓與王黄趙利期而黄利兵不來疑其與漢有謀乃解圍之一角會天大霧漢使人往來匈奴不覺陳平請令彊弩傅兩矢外鄉【師古曰傅讀曰附每一弩而加兩矢外嚮以禦敵也鄉讀曰嚮】從解角直出帝出圍欲驅太僕滕公固徐行至平城漢大軍亦到胡遂解去漢亦罷兵歸令樊噲止定代地上至廣武赦劉敬曰吾不用公言以困平城吾皆已斬前使十輩矣乃封敬二千戶為關内侯號為建信侯帝南過曲逆【班志曲逆縣屬中山國張晏曰濡水於城北曲而西流故曰曲逆後漢章帝醜其名改曰蒲隂杜佑曰中山郡北平縣秦曲逆縣後漢蒲隂縣曲逆讀皆如字文選高祖功臣贊注曰西區句翻逆音遇非也顔之推曰俗儒讀曲逆侯為去遇票姚校尉曰飄揺票姚諸儒有兩音最無謂者曲逆為去遇也】曰壯哉縣吾行天下獨見洛陽與是耳乃更封陳平為曲逆侯盡食之平從帝征伐凡六出奇計輒益封邑焉 十二月上還過趙趙王敖執子壻禮甚卑【敖尚帝女魯元公主故執子壻禮】上箕倨慢罵之【師古曰箕倨者謂伸兩脚其形如箕曲禮曰坐毋箕孔頴逹曰箕謂舒展兩足狀如箕舌也】趙相貫高趙午等皆怒【貫姓也原伯貫之後】曰吾王孱王也【孟康曰孱音潺湲之潺冀州謂懦弱者為孱師古音士連翻】乃說王曰天下豪傑並起能者先立今王事帝甚恭而帝無禮請為王殺之張敖齧其指出血曰【師古曰自齧其指出血以表至誠而為誓約不背漢也為于偽翻】君何言之誤先人亡國賴帝得復國【張耳亡國事見九卷元年復國事見十卷三年】德流子孫秋豪皆帝力也【豪至秋而纎鋭秋豪言其細微也】願君無復出口【復扶又翻】貫高趙午等皆相謂曰乃吾等非也吾王長者不倍德【長知兩翻倍蒲妹翻】且吾等義不辱今帝辱我王故欲殺之何洿王為【洿烏故翻染涴也】事成歸王事敗獨身坐耳【言獨以身坐弑帝之罪】 匈奴攻代代王喜棄國自歸【喜即帝兄仲也六年春正月以代地立喜為代王韓王信故國】赦為郃陽侯【班志郃陽縣屬左馮翊詩所謂在郃之陽者也括地志郃陽故城在同州河西縣南三十里郃音合】 辛卯立皇子如意為代王【如意戚夫人之子後徙王趙】 春二月上至長安蕭何治未央宮【未央宮在長安城西南隅周迴二十八里元和志曰東距長樂宮一里中隔武庫括地志未央宫在雍州長安縣西北十里長安故城中】上見其壯麗甚怒謂何曰天下匈匈勞苦數歲成敗未可知是何治宮室過度也何曰天下方未定故可因以就宫室且夫天子以四海為家非壯麗無以重威且無令後世有以加也上說【說讀曰悦】<br />
<br />
  臣光曰王者以仁義為麗道德為威未聞其以宮室填服天下也【填讀曰鎮】天下未定當克己節用以趨民之急【趨七喻翻】而顧以宫室為先豈可謂之知所務哉昔禹卑宫室而桀為傾宫【孔子曰禹卑宫室而盡力乎溝洫桀為傾宮瑶臺以殫百姓之財】創業垂統之君躬行節儉以示子孫其末流猶入於淫靡况示之以侈乎乃云無令後世有以加豈不謬哉至于孝武卒以宮室罷敝天下【卒子恤翻罷讀曰疲】未必不由酇侯啟之也<br />
<br />
  上自櫟陽徙都長安【先雖以婁敬張良之言西都關中然都邑未成則猶居櫟陽今未央宫成始自櫟陽徙都長安】 初置宗正官以序九族【班表宗正秦官掌親屬平帝元始元年更名宗伯】 夏四月帝行如洛陽<br />
<br />
  資治通鑑卷十一<br />
<br />
<史部,編年類,資治通鑑>  <br>
   </div> 

<script src="/search/ajaxskft.js"> </script>
 <div class="clear"></div>
<br>
<br>
 <!-- a.d-->

 <!--
<div class="info_share">
</div> 
-->
 <!--info_share--></div>   <!-- end info_content-->
  </div> <!-- end l-->

<div class="r">   <!--r-->



<div class="sidebar"  style="margin-bottom:2px;">

 
<div class="sidebar_title">工具类大全</div>
<div class="sidebar_info">
<strong><a href="http://www.guoxuedashi.com/lsditu/" target="_blank">历史地图</a></strong>  
<a href="http://www.880114.com/" target="_blank">英语宝典</a>  
<a href="http://www.guoxuedashi.com/13jing/" target="_blank">十三经检索</a> 
<br><strong><a href="http://www.guoxuedashi.com/gjtsjc/" target="_blank">古今图书集成</a></strong> 
<a href="http://www.guoxuedashi.com/duilian/" target="_blank">对联大全</a> <strong><a href="http://www.guoxuedashi.com/xiangxingzi/" target="_blank">象形文字典</a></strong> 

<br><a href="http://www.guoxuedashi.com/zixing/yanbian/">字形演变</a>  <strong><a href="http://www.guoxuemi.com/hafo/" target="_blank">哈佛燕京中文善本特藏</a></strong>
<br><strong><a href="http://www.guoxuedashi.com/csfz/" target="_blank">丛书&方志检索器</a></strong> <a href="http://www.guoxuedashi.com/yqjyy/" target="_blank">一切经音义</a>  

<br><strong><a href="http://www.guoxuedashi.com/jiapu/" target="_blank">家谱族谱查询</a></strong>  <strong><a href="http://shufa.guoxuedashi.com/sfzitie/" target="_blank">书法字帖欣赏</a></strong> 
<br>

</div>
</div>


<div class="sidebar" style="margin-bottom:0px;">

<font style="font-size:22px;line-height:32px">QQ交流群9:489193090</font>


<div class="sidebar_title">手机APP 扫描或点击</div>
<div class="sidebar_info">
<table>
<tr>
	<td width=160><a href="http://m.guoxuedashi.com/app/" target="_blank"><img src="/img/gxds-sj.png" width="140"  border="0" alt="国学大师手机版"></a></td>
	<td>
<a href="http://www.guoxuedashi.com/download/" target="_blank">app软件下载专区</a><br>
<a href="http://www.guoxuedashi.com/download/gxds.php" target="_blank">《国学大师》下载</a><br>
<a href="http://www.guoxuedashi.com/download/kxzd.php" target="_blank">《汉字宝典》下载</a><br>
<a href="http://www.guoxuedashi.com/download/scqbd.php" target="_blank">《诗词曲宝典》下载</a><br>
<a href="http://www.guoxuedashi.com/SiKuQuanShu/skqs.php" target="_blank">《四库全书》下载</a><br>
</td>
</tr>
</table>

</div>
</div>


<div class="sidebar2">
<center>


</center>
</div>

<div class="sidebar"  style="margin-bottom:2px;">
<div class="sidebar_title">网站使用教程</div>
<div class="sidebar_info">
<a href="http://www.guoxuedashi.com/help/gjsearch.php" target="_blank">如何在国学大师网下载古籍?</a><br>
<a href="http://www.guoxuedashi.com/zidian/bujian/bjjc.php" target="_blank">如何使用部件查字法快速查字?</a><br>
<a href="http://www.guoxuedashi.com/search/sjc.php" target="_blank">如何在指定的书籍中全文检索?</a><br>
<a href="http://www.guoxuedashi.com/search/skjc.php" target="_blank">如何找到一句话在《四库全书》哪一页?</a><br>
</div>
</div>


<div class="sidebar">
<div class="sidebar_title">热门书籍</div>
<div class="sidebar_info">
<a href="/so.php?sokey=%E8%B5%84%E6%B2%BB%E9%80%9A%E9%89%B4&kt=1">资治通鉴</a> <a href="/24shi/"><strong>二十四史</strong></a>&nbsp; <a href="/a2694/">野史</a>&nbsp; <a href="/SiKuQuanShu/"><strong>四库全书</strong></a>&nbsp;<a href="http://www.guoxuedashi.com/SiKuQuanShu/fanti/">繁体</a>
<br><a href="/so.php?sokey=%E7%BA%A2%E6%A5%BC%E6%A2%A6&kt=1">红楼梦</a> <a href="/a/1858x/">三国演义</a> <a href="/a/1038k/">水浒传</a> <a href="/a/1046t/">西游记</a> <a href="/a/1914o/">封神演义</a>
<br>
<a href="http://www.guoxuedashi.com/so.php?sokeygx=%E4%B8%87%E6%9C%89%E6%96%87%E5%BA%93&submit=&kt=1">万有文库</a> <a href="/a/780t/">古文观止</a> <a href="/a/1024l/">文心雕龙</a> <a href="/a/1704n/">全唐诗</a> <a href="/a/1705h/">全宋词</a>
<br><a href="http://www.guoxuedashi.com/so.php?sokeygx=%E7%99%BE%E8%A1%B2%E6%9C%AC%E4%BA%8C%E5%8D%81%E5%9B%9B%E5%8F%B2&submit=&kt=1"><strong>百衲本二十四史</strong></a>  <a href="http://www.guoxuedashi.com/so.php?sokeygx=%E5%8F%A4%E4%BB%8A%E5%9B%BE%E4%B9%A6%E9%9B%86%E6%88%90&submit=&kt=1"><strong>古今图书集成</strong></a>
<br>

<a href="http://www.guoxuedashi.com/so.php?sokeygx=%E4%B8%9B%E4%B9%A6%E9%9B%86%E6%88%90&submit=&kt=1">丛书集成</a> 
<a href="http://www.guoxuedashi.com/so.php?sokeygx=%E5%9B%9B%E9%83%A8%E4%B8%9B%E5%88%8A&submit=&kt=1"><strong>四部丛刊</strong></a>  
<a href="http://www.guoxuedashi.com/so.php?sokeygx=%E8%AF%B4%E6%96%87%E8%A7%A3%E5%AD%97&submit=&kt=1">說文解字</a> <a href="http://www.guoxuedashi.com/so.php?sokeygx=%E5%85%A8%E4%B8%8A%E5%8F%A4&submit=&kt=1">三国六朝文</a>
<br><a href="http://www.guoxuedashi.com/so.php?sokeytm=%E6%97%A5%E6%9C%AC%E5%86%85%E9%98%81%E6%96%87%E5%BA%93&submit=&kt=1"><strong>日本内阁文库</strong></a> <a href="http://www.guoxuedashi.com/so.php?sokeytm=%E5%9B%BD%E5%9B%BE%E6%96%B9%E5%BF%97%E5%90%88%E9%9B%86&ka=100&submit=">国图方志合集</a> <a href="http://www.guoxuedashi.com/so.php?sokeytm=%E5%90%84%E5%9C%B0%E6%96%B9%E5%BF%97&submit=&kt=1"><strong>各地方志</strong></a>

</div>
</div>


<div class="sidebar2">
<center>

</center>
</div>
<div class="sidebar greenbar">
<div class="sidebar_title green">四库全书</div>
<div class="sidebar_info">

《四库全书》是中国古代最大的丛书,编撰于乾隆年间,由纪昀等360多位高官、学者编撰,3800多人抄写,费时十三年编成。丛书分经、史、子、集四部,故名四库。共有3500多种书,7.9万卷,3.6万册,约8亿字,基本上囊括了古代所有图书,故称“全书”。<a href="http://www.guoxuedashi.com/SiKuQuanShu/">详细>>
</a>

</div> 
</div>

</div>  <!--end r-->

</div>
<!-- 内容区END --> 

<!-- 页脚开始 -->
<div class="shh">

</div>

<div class="w1180" style="margin-top:8px;">
<center><script src="http://www.guoxuedashi.com/img/plus.php?id=3"></script></center>
</div>
<div class="w1180 foot">
<a href="/b/thanks.php">特别致谢</a> | <a href="javascript:window.external.AddFavorite(document.location.href,document.title);">收藏本站</a> | <a href="#">欢迎投稿</a> | <a href="http://www.guoxuedashi.com/forum/">意见建议</a> | <a href="http://www.guoxuemi.com/">国学迷</a> | <a href="http://www.shuowen.net/">说文网</a><script language="javascript" type="text/javascript" src="https://js.users.51.la/17753172.js"></script><br />
  Copyright &copy; 国学大师 古典图书集成 All Rights Reserved.<br>
  
  <span style="font-size:14px">免责声明:本站非营利性站点,以方便网友为主,仅供学习研究。<br>内容由热心网友提供和网上收集,不保留版权。若侵犯了您的权益,来信即刪。scp168@qq.com</span>
  <br />
ICP证:<a href="http://www.beian.miit.gov.cn/" target="_blank">鲁ICP备19060063号</a></div>
<!-- 页脚END --> 
<script src="http://www.guoxuedashi.com/img/plus.php?id=22"></script>
<script src="http://www.guoxuedashi.com/img/tongji.js"></script>

</body>
</html>
