<!DOCTYPE html PUBLIC "-//W3C//DTD XHTML 1.0 Transitional//EN" "http://www.w3.org/TR/xhtml1/DTD/xhtml1-transitional.dtd">
<html xmlns="http://www.w3.org/1999/xhtml">
<head>
<meta http-equiv="Content-Type" content="text/html; charset=utf-8" />
<meta http-equiv="X-UA-Compatible" content="IE=Edge,chrome=1">
<title>資治通鑒_212-資治通鑑卷二百十一_212-資治通鑑卷二百十一</title>
<meta name="Keywords" content="資治通鑒_212-資治通鑑卷二百十一_212-資治通鑑卷二百十一">
<meta name="Description" content="資治通鑒_212-資治通鑑卷二百十一_212-資治通鑑卷二百十一">
<meta http-equiv="Cache-Control" content="no-transform" />
<meta http-equiv="Cache-Control" content="no-siteapp" />
<link href="/img/style.css" rel="stylesheet" type="text/css" />
<script src="/img/m.js?2020"></script> 
</head>
<body>
 <div class="ClassNavi">
<a  href="/24shi/">二十四史</a> | <a href="/SiKuQuanShu/">四库全书</a> | <a href="http://www.guoxuedashi.com/gjtsjc/"><font  color="#FF0000">古今图书集成</font></a> | <a href="/renwu/">历史人物</a> | <a href="/ShuoWenJieZi/"><font  color="#FF0000">说文解字</a></font> | <a href="/chengyu/">成语词典</a> | <a  target="_blank"  href="http://www.guoxuedashi.com/jgwhj/"><font  color="#FF0000">甲骨文合集</font></a> | <a href="/yzjwjc/"><font  color="#FF0000">殷周金文集成</font></a> | <a href="/xiangxingzi/"><font color="#0000FF">象形字典</font></a> | <a href="/13jing/"><font  color="#FF0000">十三经索引</font></a> | <a href="/zixing/"><font  color="#FF0000">字体转换器</font></a> | <a href="/zidian/xz/"><font color="#0000FF">篆书识别</font></a> | <a href="/jinfanyi/">近义反义词</a> | <a href="/duilian/">对联大全</a> | <a href="/jiapu/"><font  color="#0000FF">家谱族谱查询</font></a> | <a href="http://www.guoxuemi.com/hafo/" target="_blank" ><font color="#FF0000">哈佛古籍</font></a> 
</div>

 <!-- 头部导航开始 -->
<div class="w1180 head clearfix">
  <div class="head_logo l"><a title="国学大师官网" href="http://www.guoxuedashi.com" target="_blank"></a></div>
  <div class="head_sr l">
  <div id="head1">
  
  <a href="http://www.guoxuedashi.com/zidian/bujian/" target="_blank" ><img src="http://www.guoxuedashi.com/img/top1.gif" width="88" height="60" border="0" title="部件查字,支持20万汉字"></a>


<a href="http://www.guoxuedashi.com/help/yingpan.php" target="_blank"><img src="http://www.guoxuedashi.com/img/top230.gif" width="600" height="62" border="0" ></a>


  </div>
  <div id="head3"><a href="javascript:" onClick="javascript:window.external.AddFavorite(window.location.href,document.title);">添加收藏</a>
  <br><a href="/help/setie.php">搜索引擎</a>
  <br><a href="/help/zanzhu.php">赞助本站</a></div>
  <div id="head2">
 <a href="http://www.guoxuemi.com/" target="_blank"><img src="http://www.guoxuedashi.com/img/guoxuemi.gif" width="95" height="62" border="0" style="margin-left:2px;" title="国学迷"></a>
  

  </div>
</div>
  <div class="clear"></div>
  <div class="head_nav">
  <p><a href="/">首页</a> | <a href="/ShuKu/">国学书库</a> | <a href="/guji/">影印古籍</a> | <a href="/shici/">诗词宝典</a> | <a   href="/SiKuQuanShu/gxjx.php">精选</a> <b>|</b> <a href="/zidian/">汉语字典</a> | <a href="/hydcd/">汉语词典</a> | <a href="http://www.guoxuedashi.com/zidian/bujian/"><font  color="#CC0066">部件查字</font></a> | <a href="http://www.sfds.cn/"><font  color="#CC0066">书法大师</font></a> | <a href="/jgwhj/">甲骨文</a> <b>|</b> <a href="/b/4/"><font  color="#CC0066">解密</font></a> | <a href="/renwu/">历史人物</a> | <a href="/diangu/">历史典故</a> | <a href="/xingshi/">姓氏</a> | <a href="/minzu/">民族</a> <b>|</b> <a href="/mz/"><font  color="#CC0066">世界名著</font></a> | <a href="/download/">软件下载</a>
</p>
<p><a href="/b/"><font  color="#CC0066">历史</font></a> | <a href="http://skqs.guoxuedashi.com/" target="_blank">四库全书</a> |  <a href="http://www.guoxuedashi.com/search/" target="_blank"><font  color="#CC0066">全文检索</font></a> | <a href="http://www.guoxuedashi.com/shumu/">古籍书目</a> | <a   href="/24shi/">正史</a> <b>|</b> <a href="/chengyu/">成语词典</a> | <a href="/kangxi/" title="康熙字典">康熙字典</a> | <a href="/ShuoWenJieZi/">说文解字</a> | <a href="/zixing/yanbian/">字形演变</a> | <a href="/yzjwjc/">金 文</a> <b>|</b>  <a href="/shijian/nian-hao/">年号</a> | <a href="/diming/">历史地名</a> | <a href="/shijian/">历史事件</a> | <a href="/guanzhi/">官职</a> | <a href="/lishi/">知识</a> <b>|</b> <a href="/zhongyi/">中医中药</a> | <a href="http://www.guoxuedashi.com/forum/">留言反馈</a>
</p>
  </div>
</div>
<!-- 头部导航END --> 
<!-- 内容区开始 --> 
<div class="w1180 clearfix">
  <div class="info l">
   
<div class="clearfix" style="background:#f5faff;">
<script src='http://www.guoxuedashi.com/img/headersou.js'></script>

</div>
  <div class="info_tree"><a href="http://www.guoxuedashi.com">首页</a> > <a href="/SiKuQuanShu/fanti/">四库全书</a>
 > <h1>资治通鉴</h1> <!--         下载:【右键另存为】即可 --></div>
  <div class="info_content zj clearfix">
  
<div class="info_txt clearfix" id="show">
<center style="font-size:24px;">212-資治通鑑卷二百十一</center>
    資治通鑑卷二百十一  宋 司馬光 撰<br />
<br />
  胡三省 音注<br />
<br />
  唐紀二十七【起閼逢攝提格盡彊圉大荒落凡四年】<br />
<br />
  玄宗至道大聖大明孝皇帝上之中<br />
<br />
  開元二年春正月壬申制選京官有才識者除都督刺史都督刺史有政迹者除京官【京官即在朝官也】使出入常均永為恒式【恒戶登翻】 己卯以盧懷慎檢校黄門監【去年改門下省為黄門侍中為監檢校黄門監檢校侍中也】 舊制雅俗之樂皆隸太常上精曉音律以太常禮樂之司不應典倡優雜伎【倡音昌伎渠綺翻下同】乃更置左右教坊以教俗樂命右驍衛將軍范及為之使【更工衡翻使疏吏翻】又選樂工數百人自教法曲於棃園謂之皇帝棃園弟子【棃園在禁苑中注已見前】又教宫中使習之又選伎女置宜春院【宜春院當在西内宜春門内近射殿】給賜其家禮部侍郎張廷珪酸棗尉袁楚客皆上疏以為上春秋鼎盛宜崇經術邇端士尚樸素深以悦鄭聲好遊獵為戒【上時掌翻疏所去翻好呼到翻】上雖不能用咸嘉賞之 中宗以來貴戚爭營佛寺奏度人為僧兼以偽妄富戶彊丁多削髪以避徭役所在充滿姚崇上言佛圖澄不能存趙【石虎敬重佛圖澄澄死而趙亡】鳩摩羅什不能存秦【姚興師鳩摩羅什興死而秦亡】齊襄梁武未免禍殃但使蒼生安樂即是福身何用妄度姦人使壞正法【樂音洛壞音怪】上從之丙寅命有司沙汰天下僧尼【尼女夷翻】以偽妄還俗者萬二千餘人 初營州都督治柳城以鎮撫奚契丹則天之世都督趙文翽失政奚契丹攻陷之【見二百五卷武后萬歲通天元年契欺訖翻又音喫翽呼會翻】是後寄治幽州東漁陽城【據舊書漁陽城在幽州東二百里】或言靺鞨奚霫大欲降唐止以唐不建營州無所依投為默啜所侵擾故且附之【靺鞨音末曷霫而立翻降戶江翻啜陟劣翻】若唐復建營州則相帥歸化矣【復扶又翻帥讀曰率】并州長史和戎大武等軍州節度大使薛訥信之【大武軍在代州北後改曰大同軍使疏吏翻】奏請擊契丹復置營州上亦以冷陘之役欲討契丹【冷陘敗見上卷先天元年】羣臣姚崇等多諫甲申以訥同紫微黄門三品將兵擊契丹【將即亮翻】羣臣乃不敢言 薛王業之舅王仙童侵暴百姓御史彈奏業為之請【彈徒丹翻為于偽翻】敕紫微黄門覆按姚崇盧懷慎等奏仙童辠狀明白御史所言無所枉不可縱捨上從之由是貴戚束手 二月庚寅朔太史奏太陽應虧不虧姚崇表賀請書之史冊從之 乙未突厥可汗默啜遣其子同俄特勒及妹夫火拔頡利發【厥九勿翻可從刋入聲汗音寒頡戶結翻 考異曰舊郭䖍瓘傳云默啜壻今從舊突厥傳及唐歷舊䖍瓘傳作移江可汗突厥傳作移湼可汗今從唐紀】石附失畢將兵圍北庭都護府都護郭䖍瓘擊破之【阿烏葛翻將即亮翻敗補邁翻】同俄單騎逼城下䖍瓘伏壮士於道側突起斬之【騎奇寄翻】突厥請悉軍中資糧以贖同俄聞其已死慟哭而去 丁未敕自今所在毋得創建佛寺舊寺頹壞應葺者詣有司陳牒檢視然後聽之 閠月以鴻臚少卿朔方軍副大總管王晙兼安北大都護朔方道行軍大總管令豐安定遠三受降城及旁側諸軍皆受晙節度【靈州界有豐安定遠等軍在黄河外武德四年分豐州迴樂縣置豐安縣貞觀十三年省入迴樂杜佑曰豐安軍在靈武西黄河外百八十餘里定遠軍在靈武東北二百里黄河外臚陵如翻晙子峻翻降戶江翻】徙大都護府於中受降城【杜祐曰安北府東至榆林三百五十里南至朔方八百里西至九原三百五十里北至囘紇界七百里】置兵屯田 丁卯復置十道按察使【罷十道按察使見上卷上年復扶又翻使疏吏翻】以益州長史陸象先等為之【長和兩翻】 上思徐有功用法平直乙亥以其子大理司直惀為㳟陵令【惀力迍翻又力尹翻恭陵孝敬皇帝陵】竇孝諶之子光禄卿公希瑊等請以已官爵讓惀以報其德【竇孝諶事見二百五卷武后長夀二年諶氏壬翻瑊古咸翻】由是惀累遷申王府司馬【唐制大理司直從六品上親王府司馬從四品下】 丙子申王成義請以其府録事閻楚珪為其府參軍【唐親王府録事從九品上流外官也參軍正七品上】上許之姚崇盧懷慎上言先嘗得旨云王公駙馬有所奏請非墨敕皆勿行【引近旨以寢格其請】臣竊以量材授官當歸有司【量音良】若緣親故之恩得以官爵為惠踵習近事【近事謂中宗朝濫官之弊】實紊紀綱【紊音問】事遂寢由是請謁不行 突厥石阿失畢既失同俄不敢歸癸未與其妻來奔以為右衛大將軍封燕北郡王【燕因肩翻】命其妻曰金山公主 或告太子少保劉幽求太子詹事鍾紹京有怨望語下紫微省按問幽求等不服姚崇盧懷慎薛訥言於上曰幽求等皆功臣乍就閑職微有沮喪【下遐嫁翻沮在呂翻喪息浪翻】人情或然功業既大榮寵亦深一朝下獄恐驚遠聽戊子貶幽求為睦州刺史紹京為果州刺史【果州漢安漢縣地宋於安漢故城置南宕渠郡隋廢郡改安漢縣曰南充縣屬隆州武德四年置果州舊志睦州京師東南三千六百五十九里果州至京師二千五百五十八里 考異曰幽求傳曰姚崇素嫉忌之乃奏言幽求鬱怏於散職兼有怨言貶授睦州刺史紹京傳曰姚崇素惡紹京之為人因奏紹京發言怨望左遷綿州刺史今從實録】紫微侍郎王琚行邉軍未還【去年遣王琚按行北邉諸軍行下孟翻還從宣翻又如字】亦坐幽求黨貶澤州刺史【澤州京師東北一千三十里】 敕涪州刺史周利貞等十三人皆天后時酷吏【周利貞裴談張栖正張思敬王承本劉暉楊允康暐封珣行張知默衛遂忠公孫琰鍾思亷等凡十三人涪音浮】比周興等情狀差輕宜放歸草澤終身勿齒 西突厥十姓酋長都擔叛三月己亥磧西節度使阿史那獻克碎葉等鎮擒斬都擔降其部落二萬餘帳【厥九勿翻酋慈由翻長知兩翻擔都甘翻磧七迹翻降戶江翻 考異曰實録此月云獻擒賊帥都擔六月梟都擔首盖此月奏擒之六月傳首方至耳實録此月又云以西域二萬餘帳内附六月云擒其部落五萬餘帳新傳云三萬帳盖兵家好虚聲今從其少者】 御史中丞姜晦以宗楚客等改中宗遺詔【事見二百九卷睿宗景雲元年】青州刺史韋安石太子賓客韋嗣立刑部尚書趙彦昭特進致仕李嶠於時同為宰相不能匡正令監察御史郭震彈之【監古銜翻彈徒丹翻】且言彦昭拜巫趙氏為姑蒙婦人服與妻乘車詣其家甲辰貶安石為沔州别駕嗣立為岳州别駕彦昭為袁州别駕【舊志岳州京師東南二千二百二十七里袁州京師東南三千五百八十里沔彌兖翻 考異曰彦昭傳曰姚崇素惡彦昭之為人今從實録】嶠為滁州别駕【滁州漢全椒縣地江左為南北二譙州及新昌郡隋改南譙州曰滁州舊志滁州京師東南二千五百六十四里滁音除】安石至沔州晦又奏安石嘗檢校定陵【定陵中宗陵】盗隱官物下州徵贓【下遐嫁翻】安石歎曰此祗應須我死耳憤恚而卒【恚於避翻卒子恤翻】晦皎之弟也 毁天樞【造天樞見二百五卷武后延載元年】發匠鎔其鐵錢歷月不盡先是韋后亦於天街作石臺高數丈以頌功德【天街即京城朱雀街先悉薦翻高古號翻】至是并毁之夏四月辛巳突厥可汗默啜復遣使求昏【復扶又翻使疏吏翻】<br />
<br />
  自稱乾和永清太駙馬天上得果報天男突厥聖天骨咄禄可汗【天男猶云天子也咄當沒翻】 五月己丑以歲饑悉罷員外試檢校官【員外官一也試官二也檢校官三也罷之以其冗濫日靡俸廩也】自今非有戰功及别勑毋得注擬【此三項官今後非有戰功及别敕特行録用吏兵部毋得注擬】 己酉吐蕃相坌逹延【吐從暾入聲相息亮翻坌蒲頓翻】遺宰相書【遺于季翻】請先遣解琬至河源正二國封疆然後結盟琬嘗為朔方大總管故吐蕃請之前此琬以金紫光禄大夫致仕復召拜左散騎常侍而遣之【復扶又翻又如字】又命宰相復坌逹延書招懷之琬上言吐蕃必隂懷叛計請預屯兵十萬於秦渭等州以備之【史言解琬所言其識遠過崔漢衡上時掌翻】黄門監魏知古本起小吏因姚崇引薦以至同為相崇意輕之請知古攝吏部尚書知東都選事遣吏部尚書宋璟於門下過官【唐制凡文武職事官六品以下吏兵部進擬必過門下省量其階資校其才用以審定之若擬職不當隨其優屈退而量焉謂之過官選須絹翻】知古銜之崇二子分司東都恃其父有德於知古頗招權請託知古歸悉以聞它日上從容問崇【從千容翻】卿子才性何如今何官也崇揣知上意【揣初委翻】對曰臣有三子兩在東都為人多欲而不謹是必以事干魏知古臣未及問之耳上始以崇必為其子隱【為于偽翻】及聞崇奏喜問卿安從知之對曰知古微時臣卵而翼之【左傳楚子西謂白公勝曰勝如卵余翼而長之】臣子愚以為知古必德臣容其為非故敢干之耳上于是以崇為無私而薄知古負崇欲斥之崇固請曰臣子無狀撓陛下法【撓奴巧翻又奴教翻】陛下赦其辠已幸矣苟因臣逐知古天下必以陛下為私於臣累聖政矣【累力瑞翻】上久乃許之辛亥知古罷為工部尚書 【考異曰舊知古傳二年還京上屢有顧問恩意甚厚尋改紫微令姚崇深忌憚之隂加讒毀乃除工部尚書罷知政事新傳亦云由黄門監改紫微令今據實録知古自黄門監罷政事其所以罷從柳氏舊聞】 宋王成器申王成義於上兄也岐王範薛王業上之弟也王守禮上之從兄也【從才用翻】上素友愛近世帝王莫能及初即位為長枕大被與兄弟同寢諸王每旦朝於側門【朝直遥翻下同】退則相從宴飲鬬雞擊毬或獵於近郊遊賞别墅中使存問相望於道【墅承與翻使疏吏翻下同】上聽朝罷多從諸王遊在禁中拜跪如家人禮飲食起居相與同之於殿中設五幄與諸王更處其中【更工衡翻下更奏同處昌呂翻】或講論賦詩間以飲酒博奕遊獵【間古莧翻下讒間同】或自執絲竹成器善笛範善琵琶與上更奏之諸王或有疾上為之終日不食終夜不寢業嘗疾上方臨朝須臾之間使者十返上親為業煮藥【為于偽翻】囘飆吹火誤爇上須左右驚救之上曰但使王飲此藥而愈須何足惜【爇如悦翻須與鬚同】成器尤恭慎未嘗議及時政與人交結上愈信重之故讒間之言無自而入然專以聲色畜養娱樂之【畜吁玉翻樂音洛】不任以職事羣臣以成器等地逼請循故事出刺外州六月丁巳以宋王成器兼岐州刺史【舊制岐州京師西三百一十五里】申王成義兼州刺史 【考異曰實録舊傳作幽州今從唐歷舊紀】王守禮兼虢州刺史【虢州西至京師四百三十里】令到官但領大綱自餘州務皆委上佐主之【上佐長史司馬也】是後諸王為都護都督刺史者並凖此 丙寅吐蕃使其宰相尚飲藏來獻盟書【尚吐蕃之貴姓也】 上以風俗奢靡秋七月乙未制乘輿服御金銀器玩宜令有司銷毁以供軍國之用其珠玉錦繡焚于殿前【乘繩證翻】后妃以下皆毋得服珠玉錦繡戊戌敕百官所服帶及酒器馬衘鐙【鐙都鄧翻鞍鐙也】三品以上聽飾以玉四品以金五品以銀自餘皆禁之婦人服飾從其夫子【夫子者夫若子也】其舊成錦繡聽染為皂自今天下更毋得采珠玉織錦繡等物違者杖一百工人減一等【唐法杖一百决臀杖二十減一等則杖八十】罷兩京織錦坊臣光曰明皇之始欲為治【治直吏翻】能自刻厲節儉如此晩節猶以奢敗甚哉奢靡之易以溺人也詩云靡不有初鮮克有終【詩蕩之辭易以䜴翻鮮息淺翻】可不慎哉<br />
<br />
  薛訥與左監門衛將軍杜賓客定州刺史崔宣道等將兵六萬【監古衘翻將即亮翻 考異曰舊傳云兵二萬僉載云八萬人皆没今從唐紀】出檀州擊契丹賓客以為士卒盛夏負戈甲齎資糧深入寇境難以成功訥曰盛夏草肥羔犢孳息【小羊曰羔小牛曰犢孳津之翻生也】因糧於敵正得天時一舉滅虜不可失也行至灤水山峽中【薊州雄武軍東北行百二十里至鹽城守捉又東北渡灤河灤落官翻】契丹伏兵遮其前後從山上擊之唐兵大敗死者什八九訥與數十騎突圍得免虜中嗤之謂之薛婆【俗謂婦人之老曰婆言薛訥老怯如老婦人也】崔宣道將後軍聞訥敗亦走訥歸罪於宣道及胡將李思敬等八人【將即亮翻】制悉斬之於幽州庚子敕免訥死削除其官爵獨赦杜賓客之罪 壬寅以北庭都護郭䖍瓘為凉州刺史河西諸軍州節度使【使疏吏翻】 果州刺史鍾紹京心怨望數上疏妄陳休咎【數所角翻】乙巳貶溱州刺史【溱側詵翻】 丁未房州刺史襄王重茂薨輟朝三日追諡曰殤皇帝【以韋氏所立故仍諡曰皇帝重直龍翻朝直遥翻】 戊申禁百官家毋得與僧尼道士往還壬子禁人間鑄佛寫經 宋王成器等請獻興慶坊宅為離宫甲寅制許之始作興慶宫【興慶宫後謂之南内在皇城中南距京城之東直東内之南自東内逹南内有夾城複道經通化門逹南内人主往來兩宫外人莫知之】仍各賜成器等宅環於宫側【寧王岐王宅在安興坊薛王宅在勝業坊二坊相連皆在興慶宫西寧王即宋王也環音宦】又於宫西南置樓題其西曰花萼相輝之樓南曰勤政務本之樓上或登樓聞王奏樂則召升樓同宴或幸其所居盡歡賞賚優渥 乙卯以岐王範兼絳州刺史薛王業兼同州刺史 【考異曰實録云八月乙卯據長歷八月丙辰朔實録自此以下脱少今取唐歷舊本紀補之】仍敕宋王以下每季二人入朝【朝直遥翻】周而復始 民間訛言上采擇女子以充掖庭【掖音亦】上聞之八月乙丑令有司具車牛於崇明門【唐六典大明宫紫宸殿内朝正殿也殿之南面紫宸門紫宸門之左曰崇明門右曰光順門】自選後宫無用者載還其家敕曰燕寢之内尚令罷遣閭閻之間足可知悉 乙亥吐蕃將坌逹延乞力徐帥衆十萬寇臨洮軍蘭州至于渭源【果如解琬之言岷州溢樂縣古臨洮縣義寧二年更名渭源漢隴西之首陽縣也後魏分隴西置渭源郡又改首陽為渭源縣唐以縣屬渭州將即亮翻坌蒲頓翻帥讀曰率下同洮土刀翻】掠取牧馬命薛訥白衣攝左羽林將軍為隴右防禦使【薛訥以灤河之敗削除官爵故命以白衣攝官出隴右使疏吏翻下同】以右驍衛將軍常樂郭知運為副使【常樂漢廣至縣地曹魏分廣至置宜禾縣李暠於此置凉興郡隋置常樂鎮武德五年改鎮為縣屬瓜州驍堅堯翻樂音洛】與太僕少卿王晙帥兵擊之辛巳大募勇士詣河隴就訥教習初鄯州都督楊矩以九曲之地與吐蕃【事見上卷睿宗景雲元年鄯時戰翻又音善】其地肥饒吐蕃就之畜牧【畜吁玉翻】因以入寇矩悔懼自殺乙酉太子賓客薛謙光獻武后所製豫州鼎銘【武后鑄九州鼎自製銘】其末云上玄降鑒方建隆基【通典載豫州鼎銘曰犧農首出軒昊應期唐虞繼踵湯武乘時天下光宅域内雍熙上玄降鑒方建隆基】以為上受命之符姚崇表賀且請宣示史官頒告中外<br />
<br />
  臣光曰日食不驗太史之過也而君臣相賀是誣天也采偶然之文以為符命小臣之謟也而宰相因而實之是侮其君也上誣於天下侮其君以明皇之明姚崇之賢猶不免於是豈不惜哉<br />
<br />
  九月戊申上幸驪山温湯 敕以歲稔傷農令諸州脩常平倉法【太宗時置義倉及常平倉以備凶荒高宗以後稍假以給他費至神龍中略盡至是復置之】江嶺淮浙劔南地下濕不堪貯積不在此例【貯丁呂翻】突厥可汗默啜衰老昏虐愈甚壬子葛邏禄等部落詣凉州降【邏邸佐翻降戶江翻】 冬十月吐蕃復寇渭源【復抉又翻】丙辰上下詔欲親征發兵十餘萬人馬四萬匹 戊午上還宫 甲子薛訥與吐蕃戰于武街【水經注武街城在漢狄道縣東白石山西北唐為武街驛與大來谷皆屬臨洮渭源縣界劉昫曰武街驛在渭州西界】大破之時太僕少卿隴右羣牧使王晙帥所部二千人與訥會擊吐蕃坌逹延將吐蕃兵十萬屯大來谷晙選勇士七百衣胡服夜襲之【晙子峻翻帥讀曰率坌蒲頓翻將即亮翻衣於既翻】多置鼔角於其後五里前軍遇敵大呼後人鳴鼔角以應之虜以為大軍至驚懼自相殺傷死者萬計訥時在武街去大來谷二十里虜軍塞其中間【呼火故翻塞悉則翻】晙復夜出襲之虜大潰始得與訥軍合追奔至洮水復戰于長城堡【秦築長城起臨洮因以名堡復扶又翻】又敗之【敗補邁翻】前後殺獲數萬人豐安軍使王海賓戰死【使疏吏翻】戊辰姚崇盧懷慎等奏頃者吐蕃以河為境神龍中尚公主遂踰河築城置獨山九曲兩軍【即楊矩所與九曲之地也】去積石三百里又於河上造橋今吐蕃既叛宜毁橋拔城從之以王海賓之子忠嗣為朝散大夫尚輦奉御【朝直遥翻散悉亶翻】養之宫中 己巳突厥可汗默啜又遣使求昏上許以來歲迎公主 突厥十姓胡禄屋等諸部詣北庭請降【此西突厥也降戶江翻下同】命都護郭䖍瓘撫存之 乙酉命左驍衛郎將尉遲瓌使于吐蕃宣慰金城公主吐蕃遣其大臣宗俄因矛至洮水請和用敵國禮【驍堅堯翻將即亮翻尉紆勿翻使疏吏翻洮土刀翻】上不許自是連歲犯邊 【考異曰唐歷四年七月丁丑吐蕃以去年之敗遣其大臣宗俄因矛欵塞請和自恃兵彊求敵國之禮天子忿之按自此至四年非去年也既云以敗請和又何得云自恃兵彊既云天子忿之又當年八月已許其和今從舊傳】十一月辛卯葬殤皇帝 丙申遣左散騎常侍解琬<br />
<br />
  詣北庭宣慰突厥降者隨便宜區處【處昌呂翻】 十二月壬戌沙陀金山入朝 甲子置隴右節度大使須嗣鄯奉河渭蘭臨武洮岷郭疊宕十二州【須當作領嗣字衍奉當作秦郭當作廓臨州本漢隴西之狄道地晉置武始郡隋廢郡復為狄道縣屬蘭州天寶三載始分置臨州新舊志皆云然據此則已置臨州久矣武州古白馬之地漢武帝開置武都郡西魏改曰石門縣置武州宕州後魏宕昌羌之地後周置宕昌郡天和元年置宕州鄯時戰翻又音善宕徒浪翻】以隴右防禦副使郭知運為之 乙丑立皇子嗣真為鄫王 【考異曰實録於此作鄫王於後作郯王今從舊傳 余詳考新舊二史嗣真是年與嗣初嗣玄同封然嗣真實帝之第四子非長子也長子乃嗣直也次子則嗣謙也先天元年封嗣直郯王嗣謙郢王】嗣初為鄂王嗣主為鄄王【嗣主當作嗣玄】辛巳立郢王嗣謙為皇太子嗣真上之長子母曰劉華妃【劉華妃郯王嗣直之母若鄫王嗣真之母則錢妃也亦誤】嗣謙次子也母曰趙麗妃【帝置惠妃麗妃華妃以代三夫人】麗妃以倡進有寵于上故立之【以母寵而立其子母寵衰則子愛弛矣為後廢太子張本倡音昌】 是歲置幽州節度經略鎮守大使領幽易平檀媯燕六州【媯居為翻燕因肩翻】突騎施可汗守忠之弟遮弩恨所分部落少於其兄<br />
<br />
  遂叛入突厥請為鄉導以伐守忠【騎奇寄翻少詩照翻鄉讀曰嚮】默啜遣兵二萬擊守忠虜之而還【還從宣翻又如字 考異曰舊傳以為景龍三年事按實録娑葛既為十四姓可汗自後無娑葛名但屢云突騎施守忠入朝或者守忠即娑葛賜名邪景雲以後守忠猶在又開元二年六月阿史那獻奏有龍見于北庭為鎮將妻馮言之曰突厥施娑葛三年後破散默啜八年後自滅然則娑葛於時尚在也竟不知死于何年故附此】謂遮弩曰汝叛其兄何有於我遂幷殺之【書此以戒兄弟日尋干戈而假手於他人以逞其志者】<br />
<br />
  三年春正月癸卯以盧懷慎檢校吏部尚書兼黄門監懷慎清謹儉素不營資產雖貴為卿相所得俸賜隨散親舊【俸芳用翻】妻子不免饑寒所居不蔽風雨姚崇嘗有子喪【喪息浪翻】謁告十餘日政事委積懷慎不能决惶恐入謝於上上曰朕以天下事委姚崇以卿坐鎮雅俗耳崇既出須臾裁决俱盡頗有得色顧謂紫微舍人齊澣曰予為相可比何人澣未對崇曰何如管晏澣曰管晏之法雖不能施於後猶能沒身公所為法隨復更之似不及也【復扶又翻更工衡翻觀姚崇之所以問齊澣之所以對皆揣已以方人欲不失其實今之好議論者當大臣得權之時則譽之為伊傳周召為大臣者安受之而不愧失權之後則詆之為王莾董卓李林甫楊國忠為大臣者亦受之而不得以自明則今日之諂我者乃它日之毁我者也】崇曰然則竟如何澣曰公可謂救時之相耳崇喜投筆曰救時之相豈易得乎懷慎與崇同為相自以才不及崇每事推之【易以䜴翻推吐雷翻又如字】時人謂之伴食宰相<br />
<br />
  臣光曰㫺鮑叔之於管仲子皮之於子產皆位居其上能知其賢而下之授以國政孔子美之【管仲請囚於魯鮑叔受之以歸言于桓公曰管夷吾治于高傒使相可也桓公用之遂霸諸侯鄭子皮當國授子產政子產辭子皮曰虎帥以聽政孰敢不聼遂授以政鄭國大治下遐稼翻】曹參自謂不及蕭何一遵其法無所變更漢業以成【事見十二卷漢惠帝二年更工衡翻】夫不肖用事為其僚者愛身保禄而從之不顧國家之安危是誠罪人也賢智用事為其僚者愚惑以亂其治專固以分其權媢嫉以毁其功愎戾以竊其名是亦辠人也【治直吏翻媢音冒愎弼力翻】崇唐之賢相懷慎與之同心勠力以濟明皇太平之政夫何辠哉秦誓曰如有一介臣斷斷猗無它技【斷丁亂翻猗於綺翻乂於宜翻技渠綺翻下同】其心休休焉其如有容人之有技若己有之人之彦聖其心好之【好呼到翻】不啻如自其口出寔能容之以保我子孫黎民亦職有利哉懷慎之謂矣<br />
<br />
  御史大夫宋璟坐監朝堂杖人杖輕貶睦州刺史【監古衘翻朝直遥翻】 突厥十姓降者前後萬餘帳高麗莫離支文簡十姓之壻也二月與跌都督思泰等亦自突厥帥衆來降【麗力知翻奚結翻跌徒結翻帥讀曰率 考異曰實録二年九月壬子葛邏禄車鼻施失鉢羅俟斤等十二人詣凉州内屬乙卯胡禄屋闕及首領等一千三百十一人來降十月庚辰胡禄二萬帳詣北庭内屬明年正月突厥葛邏禄下首領裴羅逹干來降二月突厥十姓部落左廂五咄陸啜右廂五弩失畢俟斤等相繼内屬前後二千餘帳三月突厥支副忌等來朝詔曰胡禄屋大首領之匐忌四月三姓葛邏禄率衆歸國五月詔葛邏禄胡屋鼠尼施等又云宜令北庭都護湯嘉惠與葛邏禄胡屋等相應安西都護呂休璟與鼠尼施相應又云及新來十姓大首領計會掎角唐歷九月云胡禄屋闕啜十月云胡禄屋二萬帳新傳前云胡禄屋後云胡屋按十姓有胡禄居闕啜鼠尼施處半啜諸書名號雖各參差要之葛邏胡禄屋鼠尼施為三姓必矣然胡禄屋以二萬帳而云十姓内屬前後二千餘帳參差難據今從舊傳 余考新舊史時默啜既破突騎施不能撫安西突厥十姓故來降而高文簡則默啜之子壻也】制皆以河南地處之【處昌呂翻】 三月胡禄屋酋長支匐忌等入朝上以十姓降者浸多夏四月庚申以右羽林大將軍薛訥為凉州鎮大總管赤水等軍並受節度居凉州【凉州有赤水軍本赤烏鎮有赤青泉因名之幅員五千一百八十里軍之最大者也】左衛大將軍郭䖍瓘為朔州鎮大總管和戎等軍並受節度居并州【朔州蜀本作朔川新紀亦然】勒兵以備默啜默啜發兵擊葛邏禄胡禄屋鼠尼施等屢破之【葛邏禄本突厥諸族在北庭西北金山之西有三族一謀落二熾俟三踏實力當東西突厥間後稍南徙自號三姓葉護邏郎佐翻尼女夷翻】敕北庭都護湯嘉惠左散騎常侍解琬等發兵救之五月壬辰敕嘉惠等與葛邏禄胡禄屋鼠尼施及定邉道大總管阿史那獻互相應援 山東大蝗民或於田旁焚香膜拜設祭而不敢殺姚崇奏遣御史督州縣捕而瘞之【膜莫胡翻膜拜胡禮拜也瘞於計翻 考異曰舊傳開元四年山東蝗大起崇奏請捕瘞按本紀三年六月山東諸州大蝗姚崇奏差御史下諸道促官吏遣人驅撲焚瘞從之是歲田收有獲人不甚饑四年又云是夏山東河南河北蝗蟲大起遣使分捕而瘞之又實録今年十一月制以間者河南河北災蝗水潦明年正月辛未以右丞倪若水為汴州刺史五月敕曰今年蝗暴乃是孶生所由官司不早除遏任蟲成長看食田苗不恤人災自為身計向若信其拘忌不有指麾則山東之苗掃地俱盡然則三年有蝗崇令討捕不能盡明年又有蝗也今從本紀】議者以為蝗衆多除不可盡上亦疑之崇曰今蝗滿山東河南北之人流亡殆盡豈可坐視食苗曾不救乎借使除之不盡猶勝養以成災上乃從之盧懷慎以為殺蝗太多恐傷和氣崇曰㫺楚莊吞蛭而愈疾【賈誼書曰楚王食寒葅而得蛭因遂吞之腹有疾而不能食令尹入問疾曰吾食葅而得蛭不行其罪是法廢而威不立也譴而誅之恐監食者皆死遂吞之令尹曰天道無親唯德是輔王有仁德疾不為傷王疾果愈蛭之日翻】孫叔殺蛇而致福【說苑孫叔敖為兒時出遊見兩頭蛇殺而埋之還家而哭母問其故曰見兩頭蛇恐死母曰蛇安在曰聞見兩頭蛇者死恐人復見己殺而埋之矣母曰毋憂汝不死矣吾聞有隂德者天必報以福】奈何不忍於蝗而忍人之饑死乎若使殺蝗有禍崇請當之 秋七月庚辰朔日有食之 上謂宰相曰朕每讀書有所疑滯無從質問可選儒學之士日使入内侍讀盧懷慎薦太常卿馬懷素九月戊寅以懷素為左散騎常侍使與右散騎常侍禇無量更日侍讀【更工衡翻】每至閤門令乘肩輿以進或往别館道遠聽于宫中乘馬親送迎之待以師傅之禮以無量羸老特為之造腰輿【羸倫為翻腰輿令人舉之適與腰平為于偽翻】在内殿令内侍舁之【舁羊茹翻】 九姓思結都督磨散等來降己未悉除官遣還 西南蠻寇邊遣右驍衛將軍李玄道發戎瀘夔巴梁鳳等州兵三萬人【戎州本犍為郡梁置戎州瀘音盧】并舊屯兵討之 壬戍以凉州大總管薛訥為朔方道行軍大總管太僕卿呂延祚霛州刺史杜賓客副之以討突厥 甲子上幸鳳泉湯【唐六典岐州郿縣有鳳凰湯】十一月乙卯還京師 劉幽求自杭州刺史徙郴州刺史【郴州漢郴縣地爲桂陽郡治昕隋平陳置郴州郴丑林翻】憤恚【恚於避翻】甲申卒于道【卒子恤翻】 丁酉以左羽林大將軍郭䖍瓘兼安西大都護四鎮經略大使䖍瓘請自募關中兵萬人詣安西討擊皆給遞馱及熟食【遞馱者沿路遞發馬牛驢馱運兵器什物也唐六典曰驢載曰馱每馱一百斤其脚直一百里一百文山阪處一百二十文驢少處不得過一百五十文平易處不得下八十文其有人負處兩人分一馱又給熟食欲其速逹安西馱徒何翻】敕許之將作大匠韋湊上疏以為今西域服從雖或時有小盗竊舊鎮兵足以制之關中常宜充實以彊幹弱枝自頃西北二虜寇邊凡在丁壯征行略盡豈宜更募驍勇遠資荒服【驍堅堯翻】又一萬征人行六千餘里咸給遞馱熟食道次州縣將何以供秦隴之西戶口漸少凉州已往沙磧悠然【少詩沼翻下同磧七迹翻下同】遣彼居人如何取濟縱令必克其獲幾何儻稽天誅毋乃甚損請計所用所得校其多少則知利害昔唐堯之代兼愛夷夏中外乂安漢武窮兵遠征雖多克獲而中國疲耗【唐堯協和萬邦韋凑所謂兼愛夷夏也漢武事見漢紀夏戶雅翻】今論帝王之盛德者皆歸唐堯不歸漢武况邀功不成者復何足比議乎時姚崇亦以䖍瓘之策為不然既而䖍瓘卒無功【復扶又翻卒子恤翻】 初監察御史張孝嵩奉使廓州【監古銜翻使疏吏翻下同】還陳磧西利害請往察其形埶上許之聽以便宜從事拔汗那者古烏孫也内附歲久吐蕃與大食共立阿了逹為王發兵攻之拔汗那王兵敗奔安西求救孝嵩謂都護呂休璟曰不救則無以號令西域遂帥旁側戎落兵萬餘人出龜茲西數千里下數百城【璟俱永翻帥讀曰率龜茲曰丘慈下遐嫁翻】長驅而進是月攻阿了逹于連城孝嵩自擐甲督士卒急攻自巳至酉屠其三城俘斬千餘級阿了逹與數騎逃入山谷【擐音宦騎奇寄翻】孝嵩傳檄諸國威振西域大食康居大宛罽賓等八國皆遣使請降【罽音計】會有言其贓汚者坐繫凉州獄貶靈州兵曹參軍【兵曹參軍即司兵參軍是後復用孝嵩為都護著名西域】 京兆尹崔日知貪暴不法御史大夫李傑將糾之日知反構傑罪十二月侍御史楊瑒廷奏曰若糾彈之司使姧人得而恐愒【瑒雉杏翻又音暢愒呼葛翻】則御史臺可廢矣上遽命傑視事如故貶日知為歙縣丞【歙縣漢屬丹陽郡縣南有歙浦因以為名唐帶歙州歙書涉翻】 或上言按察使徒煩擾公私 【考異曰開元宰臣奏云李伯等不知伯何人也今去其名】請精簡刺史縣令停按察使上命召尚書省官議之姚崇以為今止擇十使猶患未盡得人况天下三百餘州縣多數倍安得刺史縣令皆稱其職乎【稱尺證翻下不稱同】乃止 尚書左丞韋玢【玢方貧翻】奏郎官多不舉軄請沙汰改授它官玢尋出為刺史宰相奏擬冀州敕改小州姚崇奏言臺郎寛怠及不稱職玢請沙汰乃是奉公臺郎甫爾改官玢即貶黜於外議者皆謂郎官謗傷臣恐後來左右丞指以為戒則省事何從而舉矣【省事謂尚書省之事也】伏望聖慈詳察使當官者無所疑愳乃除冀州刺史 突騎施守忠既死默啜兵還守忠部將蘇禄鳩集餘衆為之酋長【騎奇寄翻將即亮翻酋慈由翻長知兩翻】蘇禄頗善綏撫十姓部落稍稍歸之有衆二十萬遂據有西方尋遣使入見【使疏吏翻下同見賢遍翻】是歲以蘇禄為左羽林大將軍金方道經略大使【西方屬金故曰金方道】 皇后妹夫尚衣奉御長孫昕以細故與御史大夫李傑不協【殿中省有尚食尚藥尚衣尚舍尚乘尚輦六局各有奉御二人尚衣奉御掌供天子衣服詳其制度辨其名數而供其進御】<br />
<br />
  四年春正月昕與其妹夫楊仙玉於里巷伺傑而敺之【伺相吏翻敺烏口翻】傑上表自訴曰髪膚見毁雖則痛身冠冕被陵誠為辱國上大怒命於朝堂杖殺以謝百僚【上時掌翻被皮義翻朝直遥翻】仍以敕書慰傑曰昕等朕之密戚不能訓導使陵犯衣冠雖寘以極刑未足謝罪卿宜以剛腸疾惡勿以凶人介意 丁亥宋王成器更名憲申王成義更名撝【二王以成字犯昭成皇后諡號更名更工衡翻】 乙酉隴右節度使郭䖍瓘奏奴石良才等八人皆有戰功請除游擊將軍【唐制游擊將軍從五品下】敕下【下遐嫁翻】盧懷慎等奏曰郭䖍瓘恃其微效輒侮彛章為奴請五品【為于偽翻】實亂綱紀不可許上從之 丙午以鄫王嗣真為安北大都護安撫河東關内隴右諸蕃大使【據新舊書此亦郯王嗣直以為鄫王嗣真誤也而新舊書以安北為安西亦誤使疏吏翻】以安北大都護張知運為之副陜王嗣昇為安西大都護安撫河西四鎮諸蕃大使以安西都護郭䖍瓘為之副【陜失冉翻】二王皆不出閤諸王遥領節度自此始 二月丙辰上幸驪山温湯 吐蕃圍松州 丁卯上還宫 辛未以尚書右丞倪若水為汴州刺史兼河南采訪使【唐會要開元二十二年二月十九日初置十道採訪處置使據此則先置采訪使二十二年始置採訪處置使也舊志汴州京師東一千三百五十里】上雖欲重都督刺史選京官才望者為之然當時士大夫猶輕外任揚州采訪使班景倩入為大理少卿過大梁【唐汴州治浚儀縣古之大梁也】若水餞之行立望其行塵久之乃返謂官屬曰班生此行何異登仙 癸酉松州都督孫仁獻襲擊吐蕃於城下大破之 上嘗遣宦官詣江南取鵁鶄鸂䳵等【鵁居肴翻鶄咨盈翻鵁鶄似鳬而大脚高毛冠水鳥也爾雅曰鳽鵁鶄陸佃新義曰鵁鶄闘視不流其睛交據汧出不流所謂鵁鶄旋目者也爾雅翼鳽似鳬而脛高有毛冠江東人養以厭火災又謂之交精精目精也其目精交也陸龜蒙曰鵁鶄黑襟青脛丹爪噣色幾及項鸂苦奚翻䳵恥力翻鸂䳵亦水鳥也毛有五色陸佃埤雅曰鸂䳵五色尾有毛如船柂小于鴨性食短狐在山澤中無復毒氣故淮賦云鸂䳵尋邪而逐害此鳥盖溪中之勑邪逐害者故以名云陳昭裕建州圖經曰鸂䳵于水中宿老少若有勑令也亦有浮游雄者左雌者右羣伍皆有式度】欲置苑中使者所至煩擾道過汴州倪若水上言今農桑方急而羅捕禽鳥以供園池之翫遠自江嶺水陸傳送食以梁肉【傳張戀翻食祥吏翻】道路觀者豈不以陛下賤人而貴鳥乎陛下方當以鳳皇為凡鳥麒麟為凡獸况鵁鶄鸂䳵曷足貴也上手敕謝若水賜帛四十段縱散其鳥 山東蝗復大起【復扶又翻】姚崇又命捕之倪若水謂蝗乃天災非人力所及宜修德以禳之【禳如羊翻】劉聰時常捕埋之為害益甚拒御史不從其命崇牒若水曰劉聰偽主德不勝妖今日聖朝妖不勝德【妖於喬翻朝直遥翻】古之良守【守手又翻】蝗不入境若其修德可免彼豈無德致然若水乃不敢違夏五月甲辰勑委使者詳察州縣捕蝗勤惰者各以名聞由是連歲蝗災不至大饑 或言於上曰今歲選叙大濫【選須絹翻下典選同】縣令非才及入謝上悉召縣令於宣政殿庭【大明宫正殿曰含元殿其北曰宣政殿】試以理人策惟鄄城令韋濟詞理第一【鄄城古縣漢屬濟隂郡後漢為兖州治所晉屬濮陽郡唐帶濮州鄄吉掾翻】擢為醴泉令【自緊縣擢為次赤縣也】餘二百餘人不入第且令之官【百當作十】四十五人放歸學問吏部侍郎盧從愿左遷豫州刺史李朝隱左遷滑州刺史【舊志滑州去京師一千四百四十里朝直遥翻 考異曰韋濟傳云問安人策一道今從唐歷盧從愿傳曰上盡召新授縣令一時於殿庭策試考入下第者一切放歸學問唐歷試在四月從愿朝隱貶在五月朝隱傳云四年春以授縣非其人貶今從唐歷又韋濟傳曰時有人密奏上曰今歲吏部選叙大濫縣令非才全不簡擇及縣令謝官日引入殿庭問安人策試者一百餘人獨濟策第一或有不書紙者擢濟為醴泉令二十餘人還舊官四十五人放歸習讀今亦從唐歷】從愿典選六年與朝隱皆名稱職【史以從愿朝隱為稱職則或言為非矣稱尺證翻】初高宗之世馬載裴行儉在吏部最有名時人稱吏部前有馬裴後有盧李濟嗣立之子也【韋嗣立思謙之子長安中為相】 有胡人上言海南多珠翠奇寶【海南謂林邑扶南真臘諸國也上時掌翻】可往營致因言市舶之利【舶音白】又欲往師子國【師子國天竺旁國也居西南海中舊無人民止有鬼神及龍居之諸國商賈來共市易鬼神不見其形但出珍寶顯其所堪價商賈依價取之其地和適無冬夏之異諸國人聞其土樂因此競至或有停住者遂成大國能馴養師子因以名國】求靈藥及善醫之嫗寘之宫掖【嫗威遇翻掖音亦】上命監察御史楊範臣與胡人偕往求之範臣從容奏曰陛下前年焚珠玉錦繡示不復用今所求者何以異於所焚者乎彼市舶與商賈爭利殆非王者之體【從千容翻復扶又翻賈音古】胡藥之性中國多不能知况於胡嫗豈宜寘之宫掖夫御史天子耳目之官必有軍國大事臣雖觸冒炎瘴死不敢辭此特胡人眩惑求媚無益聖德竊恐非陛下之意願熟思之上遽自引咎慰諭而罷之 六月癸亥上皇崩于百福殿【年五十五 考異曰睿宗玄宗實録皆曰甲子按下云己巳睿宗一七齋度萬安公主為女道士今從舊本紀唐歷】己巳以上女萬安公主為女官欲以追福 癸酉拔曳固斬突厥可汗默啜首來獻時默啜北擊拔曳固大破之於獨樂水【樂音洛】恃勝輕歸不復設備遇拔曳固迸卒頡質略自柳林突出斬之【兵敗潰散士卒迸走故曰迸卒復扶又翻迸北孟翻】時大武軍子將郝靈荃奉使在突厥【子將小將也唐令制每軍大將一人别奏八人傔十六人副二人分掌軍務奏傔減大將半判官二人典四人總管四人二主左右虞侯二主左右押衙傔各五人子將八人資其分行陣辯金鼔及部署傔各二人】頡質畧以其首歸之【考異曰唐歷侔勃曳固今從實録唐歷又云靈荃引特勒囘紇部落斬默啜于毒樂河今從舊傳舊傳云入】<br />
<br />
  【蕃使郝靈儉今從唐歷又新舊紀皆云六月癸酉斬默啜唐歷亦在六月玄宗實録七月戊寅詔書與降附突厥云乘其衰弱早就翦除其能捉獲默啜者已立賞格盖未奏到耳】與偕詣闕懸其首於廣街拔曳固囘紇同羅霫僕固五部皆來降置於大武軍北默啜之子小可汗立骨咄禄之子闕特勒擊殺之【骨咄禄即骨篤禄默啜之兄也永淳二年反天授二年死默啜代立】及默啜諸子親信略盡立其兄右賢王默棘連是為毗伽可汗國人謂之小殺毗伽以國固讓闕特勒闕特勒不受乃以為左賢王專典兵馬 秋七月壬辰太常博士陳貞節蘇獻以太廟七室已滿請遷中宗神主於别廟奉睿宗神主袝太廟從之又奏遷昭成皇后祔睿宗室肅明皇后留祀於儀坤廟【肅明皇后睿宗之元妃也昭成后次妃也以生帝升祔睿宗而肅明后祀於别廟非禮也儀坤廟見上卷景雲二年】八月乙巳立中宗廟於太廟之西 辛未契丹李失活奚李大酺帥所部來降【武后萬歲通天時奚契丹叛帝即位之後孫佺薛訥相繼喪師兩蕃不敢乘勝憑陵中國乃相帥來降中國之勢安強有以服其心故也酺音蒲帥讀曰率降戶江翻下同】制以失活為松漠郡王行金吾大將軍兼松漠都督因其八部落酋長拜為刺史【貞觀末以契丹逹稽部為峭落州紇便部為彈汙州獨活部為無逢州芬問部為羽陵州突便部為日連州芮奚部為徒河州墜斤部為萬丹州伏部為匹黎赤山二州并松漠府凡八部十州今復以其酋長各為刺史】又以將軍薛泰督軍鎮撫之大酺為饒樂郡王行金吾大將軍兼饒樂都督失活盡忠之從父弟也【李盡忠即萬歲通天叛者樂音洛從才用翻】 吐蕃復請和【復扶又翻下禄復多復必復無復同】上許之 突厥默啜既死奚契丹拔曳固等諸部皆内附突騎施蘇禄復自立為可汗突厥部落多離散毗伽可汗患之乃召默啜時牙官暾欲谷以為謀主暾欲谷年七十餘【暾乃昆翻】多智略國人信服之突厥降戶處河曲者【北河之曲處昌呂翻】聞毗伽立多復叛歸之幷州長史王晙上言此屬徒以其國喪亂【喪息浪翻】故相帥來降若彼安寧必復叛去今置之河曲此屬桀黠實難制御往往不受軍州約束興兵剽掠【黠戶入翻剽匹妙翻】聞其逃者已多與虜聲問往來通傳委曲乃是畜養此屬使為間諜【畜吁玉翻間古莧翻】日月滋久奸詐逾深窺伺邊隙將成大患虜騎南牧必為内應【伺相吏翻騎奇寄翻】來逼軍州表裏受敵雖有韓彭不能取勝矣願以秋冬之交大集兵衆諭以利害給其資糧徙之内地二十年外漸變舊俗皆成勁兵雖一時暫勞然永久安靖比者守邉將吏及出境使人多為諛辭皆非事實【比毗至翻將即亮翻】或云北虜破滅或云降戶妥帖皆欲自衒其功非能盡忠狥國願察斯利口【孔子曰惡利口之覆邦家者】勿忘遠慮議者必曰國家曏時已嘗寘降戶於河曲皆獲安寧【謂貞觀時也】今何所疑此則事同時異不可不察曏者頡利既亡降者無復異心故得久安無變今北虜尚存【謂默啜雖死毗伽又立也】此屬或畏其威或懷其惠或其親屬豈樂南來較之彼時固不侔矣【彼時謂貞觀之時樂音洛】以臣愚慮徙之内地上也多屯士馬大為之備華夷相參人勞費廣次也正如今日下也願審茲三策擇利而行縱使因徙逃亡得者皆為唐有若留至河氷恐必有變疏奏未報降戶跌思泰阿悉爛等果叛冬十月甲辰命朔方大總管薛訥發兵追討之王晙引并州兵西濟河晝夜兼行追擊叛者破之斬獲三千級先是單于副都護張知運悉收降戶兵仗【先悉薦翻】令度河而南降戶怨怒御史中丞姜晦為廵邉使【使疏吏翻】降戶訴無弓矢不得射獵晦悉還之降戶得之遂叛張知運不設備與之戰于青剛嶺【青剛嶺在慶州方渠縣北靈州之南】為虜所擒欲送突厥至綏州境將軍郭知運以朔方兵邀擊之大破其衆於黑山呼延谷虜釋張知運而去上以張知運喪師斬之以徇毗伽可汗既得思泰等欲南入為寇暾欲谷曰唐主英武民和年豐未有間隙不可動也【喪息浪翻間古莧翻】我衆新集力尚疲羸且當息養數年始可觀變而舉毗伽又欲築城并立寺觀【羸倫為翻寺觀古玩翻】暾欲谷曰不可突厥人徒稀少【少詩沼翻】不及唐家百分之一所以能與為敵者正以逐水草居處無常【處昌呂翻】射獵為業人皆習武彊則進兵抄掠弱則竄伏山林唐兵雖多無所施用若築城而居變更舊俗【更工衡翻】一朝失利必為所滅釋老之法教人仁弱非用武爭勝之術不可崇也毗伽乃止庚午葬大聖皇帝于橋陵【橋陵在同州蒲城縣三十里是歲改蒲城縣為奉先】<br />
<br />
  【縣屬京兆尹】廟號睿宗御史大夫李傑護橋陵作判官王旭犯贓傑按之反為所構左遷衢州刺史【衢州漢新安太未之地晉改新安為信安改太未為龍丘屬東陽郡唐武德四年分置衢州衢州京師東南四千七百十二里】 十一月己卯黄門監盧懷慎疾亟上表薦宋璟李傑李朝隱盧從愿並明時重器所坐者小所棄者大望垂矜録上深納之乙未薨 【考異曰鄭處誨明皇雜録云懷慎為黄門監吏部尚書卧病既久宋璟盧從愿相與訪焉懷慎常器重二人持一人手謂曰公出入為藩輔主上求治甚切然享國歲久近者稍倦于勤必有人乘此而進矣君其志之按懷慎初為吏部時璟貶睦州及卒璟猶未歸從愿未嘗入相又四年未為享國歲久今不取】家無餘蓄惟一老蒼頭請自鬻以辦喪事【史言盧懷慎之奴異乎人奴】 丙申以尚書左丞源乹曜為黄門侍郎同平章事姚崇無居第寓居罔極寺以病痁謁告【唐會要神龍元年太平公主為天后立罔極寺於大寧坊開元二十年改為興唐寺痁失亷翻瘧疾也】上遣使問飲食起居狀日數十輩【使疏吏翻】源乾曜奏事或稱旨上輒曰此必姚崇之謀也或不稱旨輒曰何不與姚崇議之【稱尺證翻】乾曜常謝實然每有大事上常令乾曜就寺問崇癸卯乾曜請遷崇於四方館【四方館屬中書省】仍聽家人入侍疾上許之崇以四方館有簿書非病者所宜處固辭上曰設四方館為官吏也使卿居之為社稷也【處昌呂翻為于偽翻】恨不可使卿居禁中耳此何足辭崇子光禄少卿彛宗正少卿异廣通賓客頗受饋遺為時所譏【遺于季翻】主書趙誨為崇所親信【唐中書省有主書四人從七品上】受胡人賂事覺上親鞫問下獄當死崇復營救【下遐嫁翻復扶又翻】上由是不悦會曲赦京城敕特標誨名杖之一百流嶺南 【考異曰朝野僉載紫微舍人倪若水贓至八百貫因諸王内宴姚元崇諷之曰倪舍人正直百司嫉之欲成事何不為上言之諸王入衆共救之遂釋一無所問主書趙誨受蕃餉一刀子或直六七百錢元崇宣敕處死後有降崇乃曰别敕處死者决一百配流大理决趙誨一百不死夜遣給使縊殺之盖批字也今從舊傳】崇由是憂懼數請避相位【數所角翻】薦廣州都督宋璟自代十一月上將幸東都以璟為刑部尚書西京留守【璟俱永翻守式又翻】令馳驛詣闕遣内侍將軍楊思勗迎之【按舊書楊思勗傳時為内常侍右監門衛將軍内侍内侍省官之長内常侍則為之貳者也内侍從四品下内常侍正五品上】璟風度凝遠人莫測其際在塗竟不與思勗交言思勗素貴幸歸訴于上上嗟歎良久益重璟 丙辰上幸驪山温湯乙丑還宫閏月己亥姚崇罷為開府儀同三司源乾曜罷為京<br />
<br />
  兆尹西京留守【守手又翻】以刑部尚書宋璟守吏部尚書兼黄門監紫微侍郎蘇頲同平章事【頲他鼎翻】璟為相務在擇人隨材授任使百官各稱其職【稱尺證翻】刑賞無私敢犯顔直諫上甚敬憚之雖不合意亦曲從之突厥默啜自則天世為中國患朝廷旰食【朝直遥翻旰古按翻】傾天下之力不能克郝靈荃得其首自謂不世之功璟以天子好武功恐好事者競生心徼倖【好呼到翻徼工堯翻】痛抑其賞逾年始授郎將靈荃慟哭而死【郝靈荃因人以為功授以郎將非抑之也將即亮翻】璟與蘇頲相得甚厚頲遇事多讓於璟璟每論事則頲為之助璟嘗謂人曰吾與蘇氏父子皆同居相府僕射寛厚誠為國器【僕射謂蘇瓌也】然獻可替否吏事精敏則黄門過其父矣【按舊書蘇頲傳頲以紫微侍郎同紫微黄門平章事】姚宋相繼為相崇善應變成務璟善守法持正二人志操不同然協心輔佐使賦役寛平刑罰清省百姓富庶唐世賢相前稱房杜後稱姚宋它人莫得比焉二人每進見上輒為之起去則臨軒送之【見賢遍翻輒為于偽翻】及李林甫為相雖寵任過於姚宋然禮遇殊卑薄矣【史終言之】紫微舍人高仲舒博通典籍齊澣練習時務姚宋每坐二人以質所疑既而歎曰欲知古問高君欲知今問齊君可以無闕政矣 辛丑罷十道按察使【開元二年復置按察使】 舊制六品以下官皆委尚書省奏擬是歲始制員外郎御史起居遺補不擬【員外郎御史起居遺補皆臺省要官由人主親除不由尚書奏擬按唐制員外郎從六品侍御史起居郎亦從六品補闕七品拾遺及監察御史則八品耳】<br />
<br />
  五年春正月癸卯太廟四室壞上素服避正殿時上將幸東都【舊志東都至西京八百五十里】以問宋璟蘇頲對曰陛下三年之制未終【去年六月睿宗崩故云然】遽爾行幸恐未契天心災異為戒願且停車駕又問姚崇對曰太廟屋材皆苻堅時物歲久朽腐而壞適與行期相會何足異也【言不足以為災異】且王者以四海為家陛下以關中不稔幸東都百司供擬已備不可失信但應遷神主于太極殿更修太廟【更工衡翻】如期自行耳上大喜從之賜崇絹二百匹己酉上行享禮於太極殿命姚崇五日一朝仍入閤供奉【入閤供奉者應内殿朝參立于供奉班中姚崇舊相也盖立于供奉班首朝直遥翻】恩禮更厚有大政輒訪焉右散騎常侍禇無量上言隋文帝富有天下遷都之日豈取苻氏舊材以立太廟乎此特諛臣之言耳願陛下克謹天戒納忠諫遠諂諛【禇無量之言讜言也上時掌翻遠于願翻】上弗聽辛亥行幸東都過崤谷道隘不治【崤谷在陜州硤石縣】上欲免河南尹及知頓使官【車駕行幸有知頓使使疏吏翻】宋璟諫曰陛下方事廵幸今以此罪二臣臣恐將來民受其弊上遽命釋之璟曰陛下罪之以臣言而免之是臣代陛下受德也請令待罪朝堂而後赦之【朝直遥翻】上從之 【考異曰實録五月乙巳以李朝隱為河南尹宋璟傳云上次永寧之崤谷馳道隘狹車騎停擁河南尹李朝隐知頓使王怡失於部伍上令黜其官爵二傳相違盖當時河南尹不知何人非朝隱耳又明皇雜録曰上幸東都至繡嶺宫當時炎酷上以行宫狹隘謂左右曰此有佛寺乎吾將避暑於廣厦或云六軍填委於其中不可速行上謂高力士曰姚崇多計第往覘之力士囘奏云姚崇方袗絺綌乘小駟按轡于木隂下上悦曰吾得之矣遽命小駟而頓消暑溽乃歎曰小事尚如此觸類而長之天下固受其惠矣按正月東幸二月至東都未炎暑也今不取】 二月甲戌至東都赦天下 奚契丹既内附貝州刺史宋慶禮建議請復營州三月庚戌制復置營州都督於柳城【制復扶又翻又如字】兼平盧軍使管内州縣鎮戍皆如其舊【武后萬歲通天元年營州陷至是乃復】以太子詹事姜師度為營田支度使與慶禮等築之三旬而畢慶禮清勤嚴肅開屯田八十餘所招安流散數年之間倉廩充實市里浸繁 夏四月甲戍賜奚王李大酺妃辛氏號固安公主【酺音蒲】 己丑皇子嗣一卒追立為夏王諡曰悼【夏戶雅翻】嗣一母武惠妃攸止之女也【武攸止武后從子也】 突騎施酋長左羽林大將軍蘇禄部衆浸彊雖職貢不乏隂有窺邊之志五月十姓可汗阿史那獻欲發葛邏禄兵擊之上不許 初上微時與太常卿姜皎親善及誅竇懷貞等【誅懷貞等見上卷元年】皎預有功由是寵遇羣臣莫及常出入臥内與后妃連榻宴飲賞賜不可勝紀【勝音升】弟晦亦以皎故累遷吏部侍郎宋璟言皎兄弟權寵太盛非所以安之上亦以為然秋七月庚子以晦為宗正卿因下制曰西漢諸將以權貴不全【謂漢高帝時也將即亮翻】南陽故人以優閑自保【謂漢光武時也】皎宜放歸田園散官勲封皆如故【散悉亶翻】 壬寅隴右節度使郭知運大破吐蕃於九曲 安西副大都護湯嘉惠奏突騎施引大食吐蕃謀取四鎮圍鉢換及大石城【鉢換即撥換城大石城盖石國城也】已發三姓葛邏禄兵與阿史那獻擊之并州長史張嘉貞上言突厥九姓新降者散居太原<br />
<br />
  以北請宿重兵以鎮之辛酉置天兵軍於并州集兵八萬以嘉貞為天兵軍大使【天兵軍在并州城中】 太常少卿王仁惠奏則天立明堂不合古制又明堂尚質而窮極奢侈密邇宫掖人神雜擾甲子制復以明堂為乾元殿【毁乾元殿見二百四卷武后垂拱四年復扶又翻又如字】冬至元日受朝賀【朝直遥翻】季秋大享復就圓丘 九月中書門下省及侍中皆復舊名【改中書門下省及省官名見上卷元年】 貞觀之制中書門下及三品官入奏事必使諫官史官隨之有失則匡正美惡必記之諸司皆於正牙奏事御史彈百官服豸冠對仗讀彈文【獬豸冠法冠也一曰柱後惠文高五寸以纚為展筩鐵柱卷執法者服之觀王義方彈李義府事可見彈徒丹翻下同】故大臣不得專君而小臣不得為讒慝【慝吐得翻】及許敬宗李義府用事政多私僻奏事官多俟仗下於御坐前屏左右密奏【坐徂臥翻屏必郢翻】監奏御史【監奏御史意即殿中侍御史也監古銜翻下同】及待制官【永徽中命弘文館學士一人日待制于武德殿西門文明元年詔京官五品已上清官日一人待制于章善明福門先天末又以朝集使六品已上二人隨仗待制】遠立以俟其退諫官御史皆隨仗出仗下後事不復預聞【復扶又翻】武后以法制羣下諫官御史得以風聞言事自御史大夫至監察得互相彈奏率以險詖相傾覆【詖皮義翻】及宋璟為相欲復貞觀之政戊申制自今事非的須秘密者皆令對仗奏聞史官自依故事【唐制天子御正殿則左右俯陛而聽有命則退而書之若仗在紫宸内閤則夾香案分立殿下自永徽之後唯得對仗承旨仗下之後謀議皆不得預聞】 冬十月癸酉伊闕人孫平子上言春秋譏魯躋僖公【春秋文二年大事于太廟躋僖公傳曰逆祀也君子以為失禮禮無不順祀國之大事也而逆之可謂禮乎子雖齊聖不先父食久矣故禹不先鯀湯不先契文武不先不窋宋祖帝乙鄭祖厲王猶上祖也上音時掌翻】今遷中宗於别廟而祀睿宗正與魯同兄臣於弟猶不可躋【謂魯僖公嘗臣於閔公也】况弟臣於兄【謂睿宗之于中宗也】可躋之於兄上乎若以兄弟同昭【昭讀曰佋】則不應出兄置於别廟願下羣臣博議遷中宗入廟事下禮官【下遐嫁翻】太常博士陳貞節馮宗蘇獻議以為七代之廟不數兄弟殷代或兄弟四人相繼為君【殷時陽甲盤庚小辛小乙兄弟四人相繼為君】若數以為代則無祖禰之祭矣【禰乃禮翻】今睿宗之室當亞高宗故為中宗特立别廟中宗既升新廟睿宗乃祔高宗何嘗躋居中宗之上而平子引躋僖公為證誣罔聖朝漸不可長時論多是平子上亦以為然故議久不决蘇獻頲之從祖兄也【長知兩翻從才用翻 考異曰唐歷曰獻頲之再從叔今從舊志新表】故頲右之【左傳天子之所右者寡君右之右音又】卒從禮官議【卒子恤翻】平子論之不已謫為康州都城尉【都城漢端溪縣地晉立都城縣屬晉康郡隋省併入端溪屬信安郡唐分端溪置康州都城屬焉】新廟成【更作太廟成也】戊寅神主祔廟 上命宋璟蘇頲為諸皇子制名及國邑之號【為于偽翻】又令别制一佳名及佳號進之璟等上言七子均養著於國風【詩曹國風曰鳲鳩在桑其子七兮淑人君子其儀一兮注云鳲鳩之養其子朝從上下暮從下上平均如一上時掌翻】今臣等所制名號各三十餘輒混同以進以彰陛下覆燾無偏之德【覆敷又翻】上甚善之 十一月丙申契丹王李失活入朝【朝直遥翻 考異曰長歷十一月丁酉朔丙申十月晦也與實録差一日舊紀唐歷皆云十二月己亥契丹李失活來朝今從實録】十二月壬午以東平王外孫楊氏為永樂公主妻之【東平王續紀王慎之子也慎太宗子樂音洛妻子細翻】 祕書監馬懷素奏省中書散亂訛缺請選學術之士二十人整比校補從之【比毗至翻】于是搜訪逸書選吏繕寫命國子博士尹知章桑泉尉韋述等二十人同刋正【桑泉縣隋開皇十六年分猗氏縣置屬蒲州考異曰舊傳為櫟陽尉今從韋述集賢注記】以左散騎常侍禇無量為之使【使疏吏翻】於乾元殿前編校羣書<br />
<br />
  資治通鑑卷二百十一<br />
<br />
<史部,編年類,資治通鑑>  <br>
   </div> 

<script src="/search/ajaxskft.js"> </script>
 <div class="clear"></div>
<br>
<br>
 <!-- a.d-->

 <!--
<div class="info_share">
</div> 
-->
 <!--info_share--></div>   <!-- end info_content-->
  </div> <!-- end l-->

<div class="r">   <!--r-->



<div class="sidebar"  style="margin-bottom:2px;">

 
<div class="sidebar_title">工具类大全</div>
<div class="sidebar_info">
<strong><a href="http://www.guoxuedashi.com/lsditu/" target="_blank">历史地图</a></strong>  
<a href="http://www.880114.com/" target="_blank">英语宝典</a>  
<a href="http://www.guoxuedashi.com/13jing/" target="_blank">十三经检索</a> 
<br><strong><a href="http://www.guoxuedashi.com/gjtsjc/" target="_blank">古今图书集成</a></strong> 
<a href="http://www.guoxuedashi.com/duilian/" target="_blank">对联大全</a> <strong><a href="http://www.guoxuedashi.com/xiangxingzi/" target="_blank">象形文字典</a></strong> 

<br><a href="http://www.guoxuedashi.com/zixing/yanbian/">字形演变</a>  <strong><a href="http://www.guoxuemi.com/hafo/" target="_blank">哈佛燕京中文善本特藏</a></strong>
<br><strong><a href="http://www.guoxuedashi.com/csfz/" target="_blank">丛书&方志检索器</a></strong> <a href="http://www.guoxuedashi.com/yqjyy/" target="_blank">一切经音义</a>  

<br><strong><a href="http://www.guoxuedashi.com/jiapu/" target="_blank">家谱族谱查询</a></strong>  <strong><a href="http://shufa.guoxuedashi.com/sfzitie/" target="_blank">书法字帖欣赏</a></strong> 
<br>

</div>
</div>


<div class="sidebar" style="margin-bottom:0px;">

<font style="font-size:22px;line-height:32px">QQ交流群9:489193090</font>


<div class="sidebar_title">手机APP 扫描或点击</div>
<div class="sidebar_info">
<table>
<tr>
	<td width=160><a href="http://m.guoxuedashi.com/app/" target="_blank"><img src="/img/gxds-sj.png" width="140"  border="0" alt="国学大师手机版"></a></td>
	<td>
<a href="http://www.guoxuedashi.com/download/" target="_blank">app软件下载专区</a><br>
<a href="http://www.guoxuedashi.com/download/gxds.php" target="_blank">《国学大师》下载</a><br>
<a href="http://www.guoxuedashi.com/download/kxzd.php" target="_blank">《汉字宝典》下载</a><br>
<a href="http://www.guoxuedashi.com/download/scqbd.php" target="_blank">《诗词曲宝典》下载</a><br>
<a href="http://www.guoxuedashi.com/SiKuQuanShu/skqs.php" target="_blank">《四库全书》下载</a><br>
</td>
</tr>
</table>

</div>
</div>


<div class="sidebar2">
<center>


</center>
</div>

<div class="sidebar"  style="margin-bottom:2px;">
<div class="sidebar_title">网站使用教程</div>
<div class="sidebar_info">
<a href="http://www.guoxuedashi.com/help/gjsearch.php" target="_blank">如何在国学大师网下载古籍?</a><br>
<a href="http://www.guoxuedashi.com/zidian/bujian/bjjc.php" target="_blank">如何使用部件查字法快速查字?</a><br>
<a href="http://www.guoxuedashi.com/search/sjc.php" target="_blank">如何在指定的书籍中全文检索?</a><br>
<a href="http://www.guoxuedashi.com/search/skjc.php" target="_blank">如何找到一句话在《四库全书》哪一页?</a><br>
</div>
</div>


<div class="sidebar">
<div class="sidebar_title">热门书籍</div>
<div class="sidebar_info">
<a href="/so.php?sokey=%E8%B5%84%E6%B2%BB%E9%80%9A%E9%89%B4&kt=1">资治通鉴</a> <a href="/24shi/"><strong>二十四史</strong></a>&nbsp; <a href="/a2694/">野史</a>&nbsp; <a href="/SiKuQuanShu/"><strong>四库全书</strong></a>&nbsp;<a href="http://www.guoxuedashi.com/SiKuQuanShu/fanti/">繁体</a>
<br><a href="/so.php?sokey=%E7%BA%A2%E6%A5%BC%E6%A2%A6&kt=1">红楼梦</a> <a href="/a/1858x/">三国演义</a> <a href="/a/1038k/">水浒传</a> <a href="/a/1046t/">西游记</a> <a href="/a/1914o/">封神演义</a>
<br>
<a href="http://www.guoxuedashi.com/so.php?sokeygx=%E4%B8%87%E6%9C%89%E6%96%87%E5%BA%93&submit=&kt=1">万有文库</a> <a href="/a/780t/">古文观止</a> <a href="/a/1024l/">文心雕龙</a> <a href="/a/1704n/">全唐诗</a> <a href="/a/1705h/">全宋词</a>
<br><a href="http://www.guoxuedashi.com/so.php?sokeygx=%E7%99%BE%E8%A1%B2%E6%9C%AC%E4%BA%8C%E5%8D%81%E5%9B%9B%E5%8F%B2&submit=&kt=1"><strong>百衲本二十四史</strong></a>  <a href="http://www.guoxuedashi.com/so.php?sokeygx=%E5%8F%A4%E4%BB%8A%E5%9B%BE%E4%B9%A6%E9%9B%86%E6%88%90&submit=&kt=1"><strong>古今图书集成</strong></a>
<br>

<a href="http://www.guoxuedashi.com/so.php?sokeygx=%E4%B8%9B%E4%B9%A6%E9%9B%86%E6%88%90&submit=&kt=1">丛书集成</a> 
<a href="http://www.guoxuedashi.com/so.php?sokeygx=%E5%9B%9B%E9%83%A8%E4%B8%9B%E5%88%8A&submit=&kt=1"><strong>四部丛刊</strong></a>  
<a href="http://www.guoxuedashi.com/so.php?sokeygx=%E8%AF%B4%E6%96%87%E8%A7%A3%E5%AD%97&submit=&kt=1">說文解字</a> <a href="http://www.guoxuedashi.com/so.php?sokeygx=%E5%85%A8%E4%B8%8A%E5%8F%A4&submit=&kt=1">三国六朝文</a>
<br><a href="http://www.guoxuedashi.com/so.php?sokeytm=%E6%97%A5%E6%9C%AC%E5%86%85%E9%98%81%E6%96%87%E5%BA%93&submit=&kt=1"><strong>日本内阁文库</strong></a> <a href="http://www.guoxuedashi.com/so.php?sokeytm=%E5%9B%BD%E5%9B%BE%E6%96%B9%E5%BF%97%E5%90%88%E9%9B%86&ka=100&submit=">国图方志合集</a> <a href="http://www.guoxuedashi.com/so.php?sokeytm=%E5%90%84%E5%9C%B0%E6%96%B9%E5%BF%97&submit=&kt=1"><strong>各地方志</strong></a>

</div>
</div>


<div class="sidebar2">
<center>

</center>
</div>
<div class="sidebar greenbar">
<div class="sidebar_title green">四库全书</div>
<div class="sidebar_info">

《四库全书》是中国古代最大的丛书,编撰于乾隆年间,由纪昀等360多位高官、学者编撰,3800多人抄写,费时十三年编成。丛书分经、史、子、集四部,故名四库。共有3500多种书,7.9万卷,3.6万册,约8亿字,基本上囊括了古代所有图书,故称“全书”。<a href="http://www.guoxuedashi.com/SiKuQuanShu/">详细>>
</a>

</div> 
</div>

</div>  <!--end r-->

</div>
<!-- 内容区END --> 

<!-- 页脚开始 -->
<div class="shh">

</div>

<div class="w1180" style="margin-top:8px;">
<center><script src="http://www.guoxuedashi.com/img/plus.php?id=3"></script></center>
</div>
<div class="w1180 foot">
<a href="/b/thanks.php">特别致谢</a> | <a href="javascript:window.external.AddFavorite(document.location.href,document.title);">收藏本站</a> | <a href="#">欢迎投稿</a> | <a href="http://www.guoxuedashi.com/forum/">意见建议</a> | <a href="http://www.guoxuemi.com/">国学迷</a> | <a href="http://www.shuowen.net/">说文网</a><script language="javascript" type="text/javascript" src="https://js.users.51.la/17753172.js"></script><br />
  Copyright &copy; 国学大师 古典图书集成 All Rights Reserved.<br>
  
  <span style="font-size:14px">免责声明:本站非营利性站点,以方便网友为主,仅供学习研究。<br>内容由热心网友提供和网上收集,不保留版权。若侵犯了您的权益,来信即刪。scp168@qq.com</span>
  <br />
ICP证:<a href="http://www.beian.miit.gov.cn/" target="_blank">鲁ICP备19060063号</a></div>
<!-- 页脚END --> 
<script src="http://www.guoxuedashi.com/img/plus.php?id=22"></script>
<script src="http://www.guoxuedashi.com/img/tongji.js"></script>

</body>
</html>
