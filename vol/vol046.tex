<!DOCTYPE html PUBLIC "-//W3C//DTD XHTML 1.0 Transitional//EN" "http://www.w3.org/TR/xhtml1/DTD/xhtml1-transitional.dtd">
<html xmlns="http://www.w3.org/1999/xhtml">
<head>
<meta http-equiv="Content-Type" content="text/html; charset=utf-8" />
<meta http-equiv="X-UA-Compatible" content="IE=Edge,chrome=1">
<title>資治通鑒_47-資治通鑑卷四十六_47-資治通鑑卷四十六</title>
<meta name="Keywords" content="資治通鑒_47-資治通鑑卷四十六_47-資治通鑑卷四十六">
<meta name="Description" content="資治通鑒_47-資治通鑑卷四十六_47-資治通鑑卷四十六">
<meta http-equiv="Cache-Control" content="no-transform" />
<meta http-equiv="Cache-Control" content="no-siteapp" />
<link href="/img/style.css" rel="stylesheet" type="text/css" />
<script src="/img/m.js?2020"></script> 
</head>
<body>
 <div class="ClassNavi">
<a  href="/24shi/">二十四史</a> | <a href="/SiKuQuanShu/">四库全书</a> | <a href="http://www.guoxuedashi.com/gjtsjc/"><font  color="#FF0000">古今图书集成</font></a> | <a href="/renwu/">历史人物</a> | <a href="/ShuoWenJieZi/"><font  color="#FF0000">说文解字</a></font> | <a href="/chengyu/">成语词典</a> | <a  target="_blank"  href="http://www.guoxuedashi.com/jgwhj/"><font  color="#FF0000">甲骨文合集</font></a> | <a href="/yzjwjc/"><font  color="#FF0000">殷周金文集成</font></a> | <a href="/xiangxingzi/"><font color="#0000FF">象形字典</font></a> | <a href="/13jing/"><font  color="#FF0000">十三经索引</font></a> | <a href="/zixing/"><font  color="#FF0000">字体转换器</font></a> | <a href="/zidian/xz/"><font color="#0000FF">篆书识别</font></a> | <a href="/jinfanyi/">近义反义词</a> | <a href="/duilian/">对联大全</a> | <a href="/jiapu/"><font  color="#0000FF">家谱族谱查询</font></a> | <a href="http://www.guoxuemi.com/hafo/" target="_blank" ><font color="#FF0000">哈佛古籍</font></a> 
</div>

 <!-- 头部导航开始 -->
<div class="w1180 head clearfix">
  <div class="head_logo l"><a title="国学大师官网" href="http://www.guoxuedashi.com" target="_blank"></a></div>
  <div class="head_sr l">
  <div id="head1">
  
  <a href="http://www.guoxuedashi.com/zidian/bujian/" target="_blank" ><img src="http://www.guoxuedashi.com/img/top1.gif" width="88" height="60" border="0" title="部件查字,支持20万汉字"></a>


<a href="http://www.guoxuedashi.com/help/yingpan.php" target="_blank"><img src="http://www.guoxuedashi.com/img/top230.gif" width="600" height="62" border="0" ></a>


  </div>
  <div id="head3"><a href="javascript:" onClick="javascript:window.external.AddFavorite(window.location.href,document.title);">添加收藏</a>
  <br><a href="/help/setie.php">搜索引擎</a>
  <br><a href="/help/zanzhu.php">赞助本站</a></div>
  <div id="head2">
 <a href="http://www.guoxuemi.com/" target="_blank"><img src="http://www.guoxuedashi.com/img/guoxuemi.gif" width="95" height="62" border="0" style="margin-left:2px;" title="国学迷"></a>
  

  </div>
</div>
  <div class="clear"></div>
  <div class="head_nav">
  <p><a href="/">首页</a> | <a href="/ShuKu/">国学书库</a> | <a href="/guji/">影印古籍</a> | <a href="/shici/">诗词宝典</a> | <a   href="/SiKuQuanShu/gxjx.php">精选</a> <b>|</b> <a href="/zidian/">汉语字典</a> | <a href="/hydcd/">汉语词典</a> | <a href="http://www.guoxuedashi.com/zidian/bujian/"><font  color="#CC0066">部件查字</font></a> | <a href="http://www.sfds.cn/"><font  color="#CC0066">书法大师</font></a> | <a href="/jgwhj/">甲骨文</a> <b>|</b> <a href="/b/4/"><font  color="#CC0066">解密</font></a> | <a href="/renwu/">历史人物</a> | <a href="/diangu/">历史典故</a> | <a href="/xingshi/">姓氏</a> | <a href="/minzu/">民族</a> <b>|</b> <a href="/mz/"><font  color="#CC0066">世界名著</font></a> | <a href="/download/">软件下载</a>
</p>
<p><a href="/b/"><font  color="#CC0066">历史</font></a> | <a href="http://skqs.guoxuedashi.com/" target="_blank">四库全书</a> |  <a href="http://www.guoxuedashi.com/search/" target="_blank"><font  color="#CC0066">全文检索</font></a> | <a href="http://www.guoxuedashi.com/shumu/">古籍书目</a> | <a   href="/24shi/">正史</a> <b>|</b> <a href="/chengyu/">成语词典</a> | <a href="/kangxi/" title="康熙字典">康熙字典</a> | <a href="/ShuoWenJieZi/">说文解字</a> | <a href="/zixing/yanbian/">字形演变</a> | <a href="/yzjwjc/">金 文</a> <b>|</b>  <a href="/shijian/nian-hao/">年号</a> | <a href="/diming/">历史地名</a> | <a href="/shijian/">历史事件</a> | <a href="/guanzhi/">官职</a> | <a href="/lishi/">知识</a> <b>|</b> <a href="/zhongyi/">中医中药</a> | <a href="http://www.guoxuedashi.com/forum/">留言反馈</a>
</p>
  </div>
</div>
<!-- 头部导航END --> 
<!-- 内容区开始 --> 
<div class="w1180 clearfix">
  <div class="info l">
   
<div class="clearfix" style="background:#f5faff;">
<script src='http://www.guoxuedashi.com/img/headersou.js'></script>

</div>
  <div class="info_tree"><a href="http://www.guoxuedashi.com">首页</a> > <a href="/SiKuQuanShu/fanti/">四库全书</a>
 > <h1>资治通鉴</h1> <!--         下载:【右键另存为】即可 --></div>
  <div class="info_content zj clearfix">
  
<div class="info_txt clearfix" id="show">
<center style="font-size:24px;">47-資治通鑑卷四十六</center>
    資治通鑑卷四十六   宋 司馬光 撰<br />
<br />
  胡三省 音註<br />
<br />
  漢紀三十八【起柔兆困敦盡閼逢涒灘凡五年】<br />
<br />
  肅宗孝章皇帝上【諱炟顯宗第五子母賈貴人以馬后母養為嫡即位諡法温克令儀曰章伏侯古今註炟之字曰著】<br />
<br />
  建初元年春正月詔兖豫徐三州稟贍飢民上問司徒昱何以消復旱災【消復者消去災異而復其常】對曰陛下始踐天位雖有失得未能致異臣前為汝南太守典治楚事【賢曰永平十三年楚王英謀反連坐者在汝南昱時主劾之也治直之翻】繫者千餘人恐未能盡當其罪夫大獄一起寃者過半又諸徙者骨肉離分孤魂不祀宜一切還諸徙家蠲除禁錮使死生獲所則和氣可致帝納其言校書郎楊終上疏曰間者北征匈奴西開三十六國百姓頻年服役轉輸煩費愁苦之民足以感動天地陛下宜留念省察【漢蘭臺藏書之室也當時文學之士使讐校于其中故有校書之職劉向揚雄輩是也東都于蘭臺置令史典校秘書以郎居其任者謂之校書郎終徵詣蘭臺拜校書郎省悉景翻】帝下其章【下遐稼翻】第五倫亦同終議牟融昱皆以為孝子無改父之道【引論語孔子之言】征伐匈奴屯戍西域先帝所建不宜回異終復上疏曰秦築長城功役繁興胡亥不革卒亡四海【事見秦紀復扶又翻卒子恤翻】故孝元棄珠厓之郡【事見二十八卷元帝初元二年】光武絶西域之國【事見四十三卷光武建武二十二年】不以介鱗易我衣裳【賢曰介鱗喻遠夷言其人與魚鼈無異也衣裳謂中國也揚雄法言曰珠厓之絶捐之之力也否則介鱗易我衣裳】魯文公毁泉臺春秋譏之曰先祖為之而已毁之不如勿居而已以其無妨害於民也襄公作三軍昭公舍之君子大其復古以為不舍則有害於民也【舍讀曰捨】今伊吾之役樓蘭之屯兵【竇固等取伊吾見上卷永平十六年樓蘭即鄯善此兵蓋謂班超所將吏士也】久而未還非天意也帝從之 丙寅詔二千石勉勸農桑罪非殊死須秋案驗有司明慎選舉進柔良退貪猾順時令理寃獄是時承永平故事吏政尚嚴切尚書決事率近於重【近其靳翻】尚書沛國陳寵以帝新即位宜改前世苛俗乃上疏曰臣聞先王之政賞不僭刑不濫與其不得已寧僭無濫【左傳蔡大夫聲子之言】往者斷獄嚴明【斷丁亂翻下同】所以威懲姦慝姦慝既平必宜濟之以寛陛下即位率由此義數詔羣僚弘崇晏晏【賢曰晏晏温和也尚書考靈曜曰堯聰明文塞晏晏數所角翻】而有司未悉奉承猶尚深刻斷獄者急於篣格酷烈之痛【賢曰篣即榜也古字通用聲類曰笞也說文曰格擊也】執憲者煩於詆欺放濫之文或因公行私逞縱威福夫為政猶張琴瑟大絃急者小絃絶【賢曰新序臧孫魯大夫行猛政子貢非之曰夫政猶張琴瑟也大絃急則小絃絶矣故曰罰得則姦邪止賞得則下歡悦】陛下宜隆先王之道蕩滌煩苛之法輕薄箠楚以濟羣生【箠止蘂翻】全廣至德以奉天心帝深納寵言每事務於寛厚 酒泉太守段彭等兵會柳中擊車師攻交河城【賢曰前書車師前王居交河城河水分流繞城下故號交河去長安八千一百里故城在今西州交河縣】斬首三千八百級獲生口三千餘人北匈奴驚走車師復降【復扶又翻】會關寵已殁謁者王蒙等欲引兵還耿恭軍吏范羌時在軍中【先是恭遣羌至敦煌迎兵士寒服因隨王蒙軍出塞】固請迎恭諸將不敢前乃分兵二千人與羌從山北迎恭遇大雪丈餘軍僅能至城中夜聞兵馬聲以為虜來大驚羌遥呼曰【呼火故翻】我范羌也漢遣軍迎校尉耳【校戶教翻】城中皆稱萬歲開門共相持涕泣明日遂相隨俱歸虜兵追之且戰且行吏士素飢困發疏勒時尚有二十六人隨路死沒三月至玉門【賢曰玉門關名屬敦煌郡在今沙州臣賢案酒泉郡又有玉門縣據東觀記曰至敦煌明即玉門關也】唯餘十三人衣屨穿決形容枯槁中郎將鄭衆為恭以下洗沐易衣冠【衆先以軍司馬與馬廖擊車師至敦煌拜為中郎將為于偽翻】上疏奏恭以單兵守孤城當匈奴數萬之衆連月踰年心力困盡鑿山為井煮弩為糧前後殺傷醜虜數百千計卒全忠勇【卒子恤翻】不為大漢恥宜蒙顯爵以厲將帥【將即亮翻帥所類翻】恭至雒陽拜騎都尉詔悉罷戊巳校尉及都護官【二官明帝永平十七年置】徵還班超超將發還疏勒舉國憂恐其都尉黎弇曰漢使棄我【使疏吏翻下同】我必復為龜兹所滅耳誠不忍見漢使去因以刀自剄【前書疏勒國官有疏勒侯擊胡侯輔國侯都尉復扶又翻下同龜兹音邱慈】超還至于窴王侯以下皆號泣【窴徒賢翻號戶刀翻】曰依漢使如父母誠不可去【使疏吏翻】互抱超馬脚不得行超亦欲遂其本志乃更還疏勒疏勒兩城已降龜兹而與尉頭連兵【前書尉頭國居尉頭谷去長安八千六百五十里南與疏勒接】超捕斬反者擊破尉頭殺六百餘人疏勒復安 甲寅山陽東平地震 東平王蒼上便宜三事【上時掌翻】帝報書曰間吏民奏事亦有此言但明智淺短或謂儻是復慮為非不知所定得王深策恢然意解【恢然猶廓然也】思惟嘉謀以次奉行特賜王錢五百萬後帝欲為原陵顯節陵起縣邑【為于偽翻】蒼上疏諫曰竊見光武皇帝躬履儉約之行深覩始終之分【行下孟翻分扶問翻】勤勤懇懇以葬制為言【事見四十四卷光武建武二十六年】孝明皇帝大孝無違承奉遵行【事見上卷明帝永平十四年】謙德之美於斯為盛臣愚以園邑之興起自彊秦【秦始皇葬于驪山徙三萬家起驪邑西漢因之諸陵皆起陵邑至元帝乃止】古者邱隴且不欲其著明【賢曰禮記曰古者墓而不墳故言不欲其著明】豈况築郭邑建都郛哉【穀梁傳曰人之所聚曰都杜預註左傳曰郛郭也】上違先帝聖心下造無益之功虚費國用動揺百姓非所以致和氣祈豐年也陛下履有虞之至性【虞舜孝於親故以為言】追祖禰之深思臣蒼誠傷二帝純德之美不暢於無窮也帝乃止自是朝廷每有疑政輒驛使諮問【使疏吏翻】蒼悉心以對皆見納用 秋八月庚寅有星孛于天市【晉天文志參十星一曰天市又危三星亦為天市又天市垣二十二星在房心東北史記曰房為天駟東北十二星曰填中四星曰天市孛蒲内翻】 初益州西部都尉廣漢鄭純為政清潔化行夷貊君長感慕皆奉珍内附【貊莫百翻長知兩翻】明帝為之置永昌郡【明帝永平十年置益州西部都尉居嶲唐領不韋嶲唐比蘇楪榆邪龍雲南六縣十二年哀牢内屬置哀牢博南二縣合為永昌郡為于偽翻】以純為太守純在官十年而卒【守式又翻卒子恤翻】後人不能撫循夷人九月哀牢王類牢殺守令反攻博南 阜陵王延數懷怨望【數所角翻】有告延與子男魴造逆謀者【魴音房】上不忍誅冬十一月貶延為阜陵侯食一縣不得與吏民通【延徙王阜陵事見上卷明帝永平十六年】 北匈奴臯林温禺犢王將衆還居涿邪山南單于與邊郡及烏桓共擊破之【臯林温禺犢王本居涿邪山永平十六年祭彤等北伐將衆遁去今復還】是歲南部大饑詔贍給之二年春三月甲辰罷伊吾盧屯兵匈奴復遣兵守其地【伊吾盧置屯兵事見上卷永平十六年復扶又翻】 永昌越嶲益州三郡兵及昆明夷鹵承等擊哀牢王類牢於博南大破斬之【嶲音髓】夏四月戊子詔還坐楚淮陽事徙者四百餘家【楚獄見上】<br />
<br />
  【卷明帝永平十四年淮陽獄即阜陵王延徙封時也】 上欲封爵諸舅太后不聽會大旱言事者以為不封外戚之故有司請依舊典【賢曰漢制外戚以恩澤封侯故曰舊典】太后詔曰凡言事者皆欲媚朕以要福耳【要一遥翻】昔王氏五侯同日俱封黄霧四塞【事見三十卷成帝建始元年塞悉則翻】不聞澍雨之應【澍音注】夫外戚貴盛鮮不傾覆【鮮息淺翻】故先帝防慎舅氏不令在樞機之位又言我子不當與先帝子等【事見上卷永平十五年】今有司奈何欲以馬氏比隂氏乎且隂衛尉天下稱之省中御者至門出不及履此蘧伯玉之敬也【衛尉興也省中禁中也御者内人也蘧伯玉衛賢大夫蘧求於翻】新陽侯雖剛彊微失理然有方畧據地談論一朝無雙【新陽侯就也賢曰新陽縣屬汝南郡故城在今豫州真陽縣西南】原鹿貞侯勇猛誠信【原鹿侯識也原鹿縣屬汝南郡】此三人者天下選臣豈可及哉馬氏不及隂氏遠矣吾不才夙夜累息【息氣一出入之頃屛氣者累息乃一舒氣】常恐虧先后之法有毛髪之罪吾不釋言之不捨晝夜而親屬犯之不止治喪起墳又不時覺【治直之翻】是吾言之不立而耳目之塞也【塞悉則翻】吾為天下母而身服大練【賢曰大練大帛也杜預註左傳曰大帛厚繒也】食不求甘左右但著帠布無香薰之飾者欲身率下也【著側略翻】以為外親見之當傷心自敇但笑言太后素好儉【好呼到翻】前過濯龍【續漢志濯龍園名近北宫】門上見外家問起居者車如流水馬如游龍倉頭衣綠褠領袖正白【賢曰褠臂衣今之臂鞲以縛左右手於事便也余據字書臂鞲之鞲從革此褠從衣釋單衣也皆音古侯翻領袖正白言其新潔無垢汚也衣於既翻】顧視御者不及遠矣故不加譴怒但絶歲用而已冀以默愧其心猶懈怠無憂國忘家之慮【懈古隘翻】知臣莫若君况親屬乎吾豈可上負先帝之旨下虧先人之德重襲西京敗亡之禍哉【賢曰西京外戚呂祿呂產竇嬰上官桀安父子霍禹等皆被誅重直龍翻】固不許帝省詔悲歎復重請曰【省悉景翻復扶又翻重直用翻】漢興舅氏之封侯猶皇子之為王也太后誠存謙虚奈何令臣獨不加恩三舅乎且衛尉年尊兩校尉有大病【衛尉太后兄廖兩校尉兄防兄光也校戶教翻】如令不諱使臣長抱刻骨之恨宜及吉時不可稽留【漢封爵羣臣皆㳙吉】太后報曰吾反覆念之思令兩善【兩善謂國家無濫恩而外戚亦以安全也】豈徒欲獲謙讓之名而使帝受不外施之嫌哉【以恩澤封爵外家為外施也施式智翻】昔竇太后欲封王皇后之兄丞相條侯言高祖約無軍功不侯【事見十六卷景帝中三年】今馬氏無功於國豈得與隂郭中興之后等邪常觀富貴之家祿位重疊猶再實之木其根必傷【文子曰再實之木根必傷掘臧之家後必殃重直龍翻】且人所以願封侯者欲上奉祭祀下求温飽耳今祭祀則受太官之賜衣食則蒙御府餘資【自西都以來皇后家祀其父母太官供具御府令掌中衣服及補澣之屬飲食則太官主之此言衣食皆資於御府槩言之也】斯豈不可足而必當得一縣乎吾計之孰矣【古字孰熟通】勿有疑也夫至孝之行安親為上【楊子曰孝莫大於寧親寧親莫大於四表之驩心行下孟翻】今數遭變異【數所角翻】穀價數倍憂惶晝夜不安坐臥而欲先營外家之封違慈母之拳拳乎【賢曰拳拳猶勤勤也音權】吾素剛急有匈中氣不可不順也【匈中氣今所謂上氣之疾匈與胷同】子之未冠由於父母已冠成人則行子之志【冠古玩翻】念帝人君也吾以未踰三年之故自吾家族故得專之若隂陽調和邊境清静然後行子之志吾但當含飴弄孫【方言曰飴?也宋衛之間通語】不能復關政矣【關豫政也復扶又翻】上乃止太后嘗詔三輔諸馬昏親有屬託郡縣干亂吏治者以法聞【繩之以法而奏聞也屬之欲翻治直吏翻】太夫人葬起墳微高【太夫人太后母也漢列侯墳高四丈關内侯以下至庶人有差】太后以為言兄衛尉廖等即時減削其外親有謙素義行者【行下孟翻】輒假借温言賞以財位如有纎介則先見嚴恪之色【見賢遍翻】然後加譴其美車服不遵法度者便絶屬籍遣歸田里【絶外戚之屬籍也】廣平鉅鹿樂成王車騎朴素無金銀之飾【廣平王羨鉅鹿王恭樂成王黨皆明帝子】帝以白太后即賜錢各五百萬於是内外從化被服如一【被皮義翻】諸家惶恐倍於永平時置織室蠶於濯龍中【續漢志濯龍監屬鉤盾令本注曰濯龍亦園名近北宫】數往觀視以為娛樂【數所角翻樂音洛】常與帝旦夕言道政事及教授小王論語經書【小王諸王年尚幼未就國者】述叙平生雍和終日馬廖慮美業難終上疏勸成德政曰昔元帝罷服官【事見二十八卷初元五年】成帝御浣衣【言服浣濯之衣也】哀帝去樂府【事見三十三年綏和二年去羌呂翻】然而侈費不息至於衰亂者百姓從行不從言也【書曰違上所命從厥攸好行下孟翻】夫改政移風必有其本傳曰吳王好劍客百姓多創瘢【傳直戀翻創初良翻瘢蒲官翻㾗也好劍客蓋指吳王闔閭也】楚王好細腰宫中多餓死【墨子曰楚靈王好細腰而國多餓人】長安語曰【賢曰當時諺語】城中好高結四方高一尺【結讀曰髻】城中好廣眉四方且半額城中好大袖四方全匹帛斯言如戲有切事實前下制度未幾後稍不行【未幾言未幾時也幾居豈翻】雖或吏不奉法良由慢起京師今陛下素簡所安發自聖性【賢曰言儉素簡約后之所安】誠令斯事一竟【竟猶終也】則四海誦德聲薰天地【賢曰薰猶蒸也言芳聲薰天地也】神明可通况於行令乎太后深納之初安夷縣吏略妻卑湳種羌人婦【安夷縣屬金城郡杜預曰不以道取曰略湳乃感翻種章勇翻下同】吏為其夫所殺安夷長宗延追之出塞【長知兩翻】種人恐見誅遂共殺延而與勒姐吾良二種相結為宼【勒姐羌居勒姐溪因以為種名】於是燒當羌豪滇吾之子迷吾率諸種俱反【姐子也翻又音紫滇音顛】敗金城太守郝崇【敗補邁翻郝呼各翻姓譜殷帝乙冇子期封太原郝郷後因氏焉】詔以武威太守北地傅育為護羌校尉自安夷徙居臨羌【臨羌縣屬金城郡杜佑曰臨羌在今西平郡水經注湟水東合安夷川水又東逕安夷縣故城在漢西平亭東七十里湟水又東合勒姐溪水】迷吾又與封養種豪布橋等五萬餘人共寇隴西漢陽【本天水郡明帝永平十七年改名漢陽】秋八月遣行車騎將軍馬防長水校尉耿恭將北軍五校兵【武帝置北軍八校中壘屯騎越騎長水胡騎射聲步兵虎賁也中興省中壘胡騎虎賁惟越騎屯騎步兵長水射聲五校屯騎越騎步兵射聲各領士七百人長水領烏桓胡騎七百三十六人皆宿衛兵也】及諸郡射士三萬人擊之【馬防傳云積射士】第五倫上疏曰臣愚以為貴戚可封侯以富之不當任以職事何者繩以法則傷恩私以親則違憲伏聞馬防今當西征臣以太后恩仁陛下至孝恐卒有纎介難為意愛【賢曰恐卒然有小過愛而不罰則廢法也卒讀曰猝】帝不從馬防等軍到冀布橋等圍南部都尉於臨洮【前書隴西南部都尉治臨洮賢曰即今岷洮二州地】防進擊破之斬首虜四千餘人遂解臨洮圍其衆皆降唯布橋等二萬餘人屯望曲谷不下【酈道元註水經云望曲在臨洮西南去龍桑城二百里】十二月戊寅有星孛于紫宫【晉天文志中宫北極五星鉤陳六星皆在紫宫中紫宫垣十五星其西蕃七東蕃八孛蒲内翻】 帝納竇勲女為貴人有寵【為後諸竇竊權張本】貴人母即東海恭王女沘陽公主也【沘音比】 第五倫上疏曰光武承王莽之餘頗以嚴猛為政後代因之遂成風化郡國所舉類多辦職俗吏殊未有寛博之選以應上求者也陳留令劉豫冠軍令駟協【陳留縣屬陳留郡冠軍縣屬南陽郡冠古玩翻】並以刻薄之姿務為嚴苦吏民愁怨莫不疾之而今之議者反以為能違天心失經義非徒應坐豫協亦宜譴舉者務進仁賢以任時政不過數人則風俗自化矣臣嘗讀書記知秦以酷急亡國又目見王莽亦以苛法自滅故勤勤懇懇實在於此又聞諸王主貴戚驕奢踰制京師尚然何以示遠故曰其身不正雖令不行【論語孔子之言】以身教者從以言教者訟上善之倫雖天性峭直【賢曰峭峻也音七笑翻】然常疾俗吏苛刻論議每依寛厚云<br />
<br />
  三年春正月己酉宗祀明堂登靈臺赦天下 馬防擊布橋大破之 【考異曰帝紀防破羌在四月蓋春破而京師四月始聞也今從防傳】布橋將種人萬餘降詔徵防還留耿恭擊諸未服者斬首虜千餘人勒姐燒何等十三種數萬人皆詣恭降【姐音紫又子也翻種章勇翻】恭嘗以言事忤馬防【初恭出隴西上言薦竇固鎮撫凉部由是大忤於防忤五故翻】監營謁者承旨奏恭不憂軍事坐徵下獄免官【監古銜翻下遐稼翻】 三月癸巳立貴人竇氏為皇后 初顯宗之世治虖沱石臼河從都慮至羊腸倉【賢曰石臼河在今定州唐縣東北酈道元註水經云汾陽故城積粟所在謂之羊腸倉在晉陽西北石隥縈紆若羊腸焉故以為名今嵐州界羊腸坂是也唐嵐州宜芳縣本漢汾陽縣隋置嵐城縣唐更名宜芳杜佑曰宜芳縣有古秀容城漢羊腸倉余考水經註云案司馬彪郡國志常山南行唐縣有石臼谷蓋欲乘呼沱之水轉山東之漕自都慮至羊腸倉憑汾水以漕太原又考郡國志常山蒲吾縣註引古今注曰永平十年作常山呼沱河蒲吾渠通漕船又考班固地理志太原郡上艾縣註曰綿曼水東至蒲吾入呼沱水又蒲吾縣註曰大白渠水首受綿曼水東南至下曲陽入斯洨則知此漕自大白渠入綿曼水自綿曼水轉入汾水以達羊腸倉也慮音閭杜佑曰石臼河在定州唐昌縣唐昌漢苦陘縣也】欲令通漕太原吏民苦役連年無成死者不可勝筭【勝音升】帝以郎中鄧訓為謁者監領其事訓考量隱括【賢曰隱審量括之也孫卿子曰鉤木必待隱括蒸揉然後直也監古銜翻量音良】知其難成具以上言【上時掌翻】夏四月己巳詔罷其役更用驢輦【更工衡翻】歲省費億萬計全活徒士數千人訓禹之子也 閏月西域假司馬班超率疏勒康居于窴拘彌兵一萬人攻姑墨石城破之【前書姑墨國治南城去長安八千一百五十里】斬首七百級冬十二月丁酉以馬防為車騎將軍 武陵漊中蠻<br />
<br />
  【反賢曰漊水名源出今澧州崇義縣西北余據温公類篇漊郎侯翻】 是歲有司奏遣廣平王羨鉅鹿王恭樂成王黨俱就國上性篤愛不忍與諸王乖離遂皆留京師<br />
<br />
  四年春二月庚寅太尉牟融薨 夏四月戊子立皇子慶為太子 己丑徙鉅鹿王恭為江陵王汝南王暢為梁王常山王昞為淮陽王 辛卯封皇子伉為千乘王全為平春王【平春縣屬江夏郡伉音抗乘繩證翻】 有司連據舊典請封諸舅帝以天下豐稔方垂無事癸卯遂封衛尉廖為順陽侯【順陽侯國屬南陽郡賢曰故城在今鄧州穰縣西】車騎將軍防為潁陽侯【潁陽縣屬潁川郡】執金吾光為許侯【許縣屬潁川郡】太后聞之曰吾少壯時但慕竹帛志不顧命【賢曰言慕古人書名竹帛不顧命之長短少詩沼翻】今雖已老猶戒之在得【論語孔子曰及其老也戒之在得】故日夜惕厲【惕懼也厲危也】思自降損冀乘此道不負先帝所以化導兄弟共同斯志欲令瞑目之日無所復恨何意老志復不從哉【暝莫定翻復扶又翻】萬年之日長恨矣廖等並辭讓願就關内侯【考異曰皇后紀稱廖等並辭讓願就關内侯太后聞之云云廖等不得已受封爵按太后之辭皆不欲封廖】<br />
<br />
  【等之意而史家文勢反似太后欲令廖等受封今輒移廖等辭讓於太后語下使文勢有序讀者易解】帝不許廖等不得已受封爵而上書辭位帝許之五月丙辰防廖光皆以特進就第 甲戌以司徒鮑昱為太尉南陽太守桓虞為司徒 六月癸丑皇太后馬氏崩帝既為太后所養專以馬氏為外家故賈貴人不登極位賈氏親族無受寵榮者及太后崩但加貴人王赤綬【漢制貴人綠綬三采綠紫紺長二丈一尺二百四十首諸侯赤綬四采赤黄縹紺長二丈一尺三百首】安車一駟永巷宫人二百【賢曰永巷宫人宫婢也】御府雜帛二萬匹大司農黄金千斤錢二千萬而已 秋七月壬戌葬明德皇后【賢曰諡法中和純淑曰德】 校書郎楊終建言宣帝博徵羣儒論定五經於石渠閣【事見二十七卷甘露三年】方今天下少事【少詩沼翻】學者得成其業而章句之徒破壞大體【壞音怪】宜如石渠故事永為後世則帝從之冬十一月壬戌詔太常【句斷】將大夫博士郎官及諸儒會白虎觀議五經同異【將即亮翻將三署及虎賁羽林中郎將也大夫光祿大中中散諫議大夫也博士五經博士也郎官五署郎及尚書郎蘭臺東觀校書郎也白虎觀在北宫觀古玩翻】使五官中郎將魏應承制問侍中淳于恭奏帝親稱制臨決作白虎議奏名儒丁鴻樓望成封桓郁班固賈逵及廣平王羨皆與焉固超之兄也【與讀曰預】<br />
<br />
  五年春二月庚辰朔日有食之詔舉直言極諫 荆豫諸郡兵討漊中蠻破之【漊郎侯翻】 夏五月辛亥詔曰朕思遲直士側席異聞【賢曰遲猶希望也音持二翻側席謂不正坐所以待賢良也】其先至者各已發憤吐懣【懣莫困翻又莫旱翻】略聞子大夫之志矣皆欲置於左右顧問省納【句斷省悉景翻】建武詔書又曰堯試臣以職不直以言語筆札今外官多曠並可以補任 戊辰太傅趙熹薨 班超欲遂平西域上疏請兵曰臣竊見先帝欲開西域故北擊匈奴西使外國【使疏吏翻】鄯善于窴即時向化【鄯上扇翻】今拘彌莎車疏勒月氏烏孫康居復願歸附【復扶又翻】欲共并力破滅龜兹平通漢道若得龜兹則西域未服者百分之一耳前世議者皆曰取三十六國號為斷匈奴右臂【賢曰前書曰漢遣公主為烏孫夫人結為昆弟則是斷匈奴右臂也哀帝時劉歆上議曰武帝立五屬國起朔方伐朝鮮起玄菟樂浪以斷匈奴左臂也西伐大宛結烏孫裂匈奴之右臂也南面以西為右斷丁管翻】今西域諸國自日之所入莫不向化【西域傳曰自條支國乘水西行可百餘日近日所入也】大小欣欣貢奉不絶唯延耆龜兹獨未服從臣前與官屬三十六人奉使絶域備遭艱戹自孤守疏勒於今五載【使疏吏翻載子亥翻】胡夷情數臣頗識之問其城郭小大皆言倚漢與依天等【謂城郭之國若小若大其言皆然】以是效之【賢曰效猶驗也】則蔥領可通【古領嶺字通】龜兹可伐今宜拜龜兹侍子白霸為其國王以步騎數百送之與諸國連兵歲月之間龜兹可禽以夷狄攻夷狄計之善者也臣見莎車疏勒田地肥廣草牧饒衍不比敦煌鄯善間也【敦徒門翻】兵可不費中國而糧食自足且姑墨温宿二王特為龜兹所置【前書温宿國治温宿城去長安八千三百五十里】既非其種【種章勇翻】更相厭苦其埶必有降者若二國來降則龜兹自破【更工衡翻降戶江翻下同】願下臣章參考行事誠有萬分死復何恨【下遐稼翻復扶又翻】臣超區區特蒙神靈竊冀未便僵仆目見西域平定陛下舉萬年之觴【言西域平定廷臣畢賀天子為之舉觴也】薦勲祖廟布大喜於天下【賢曰薦進也勲功也左氏傳曰反行飲至舍爵策勲也余謂超蓋言平西域告成功於祖廟也】書奏帝知其功可成議欲給兵平陵徐幹上疏願奮身佐超帝以幹為假司馬將弛刑及義從千人就超【弛刑徒也義從自奮願從行者或曰義從胡也從才用翻】先是莎車以為漢兵不出【先悉薦翻】遂降於龜兹而疏勒都尉番辰亦叛【賢曰番音潘】會徐幹適至超遂與幹擊番辰大破之斬首千餘級欲進攻龜兹以烏孫兵彊宜因其力乃上言烏孫大國控弦十萬故武帝妻以公主【事見二十一卷元封六年妻七細翻】至孝宣帝卒得其用【事見二十四卷本始三年卒子恤翻】今可遣使招慰與共合力帝納之<br />
<br />
  六年春二月辛卯琅邪孝王京薨 夏六月丙辰太尉昱薨 辛未晦日有食之 秋七月癸巳以大司農鄧彪為太尉 武都太守廉范遷蜀郡太守成都民物豐盛邑宇逼側舊制禁民夜作以防火災而更相隱蔽燒者日屬【更工衡翻屬之欲翻聯也聯日有火也】范乃毁削先令但嚴使儲水而已百姓以為便歌之曰廉叔度來何暮【廉范字叔度】不禁火民安作【賢曰作協韻則護翻】昔無襦今五絝【襦汝朱翻短衣也絝五故翻脛衣也】 帝以沛王等將入朝遣謁者賜貂裘【說文曰貂鼠大而黄黑出胡丁零國】及太官食物珍果又使大鴻臚竇固持節郊迎【臚陵如翻】帝親自循行邸第【行下孟翻】豫設帷牀其錢帛器物無不充備<br />
<br />
  七年春正月沛王輔濟南王康東平王蒼中山王焉東海王政琅邪王宇來朝【政東海王彊子宇琅邪王京子濟子禮翻】詔沛濟南東平中山王贊拜不名【賢曰謂讃者不唱其名余謂四王帝諸父也故異其禮】升殿乃拜上親答之所以寵光榮顯加於前古每入宫輒以輦迎至省閣乃下【省閣入禁中閣門也】上為之興席改容【為于偽翻下同】皇后親拜於内皆鞠躬辭謝不自安【鞠曲也鞠躬曲身也】三月大鴻臚奏遣諸王歸國帝特留東平王蒼於京師 初明德太后為帝納扶風宋楊二女為貴人大貴人生太子慶梁松弟竦有二女亦為貴人小貴人生皇子肇竇皇后無子養肇為子宋貴人有寵於馬太后太后崩竇皇后寵盛與母沘陽公主謀陷宋氏【沘音比】外令兄弟求其纖過内使御者偵伺得失【賢曰偵候也音丑政翻廣雅曰偵問也伺相吏翻】宋貴人病思生兎【兎獸名口有缺凥有九孔舐毫而孕生子從口出霜前獵取而食之其味甚美】令家求之因誣言欲為厭勝之術【厭一葉翻又於琰翻】由是太子出居承祿觀【續漢志中藏府有承祿署】夏六月甲寅詔曰皇太子有失惑無常之性不可以奉宗廟大義滅親【春秋左氏傳之言】况降退乎今廢慶為清河王皇子肇保育皇后承訓懷衽【衽衣襟亦卧席也】今以肇為皇太子遂出宋貴人姊妹置丙舍【丙舍宫中之室以甲乙丙為次也續漢志南宫有丙署】使小黄門蔡倫案之二貴人皆飲藥自殺父議郎楊免歸本郡慶時雖幼亦知避嫌畏禍言不敢及宋氏帝更憐之敇皇后令衣服與太子齊等太子亦親愛慶入則共室出則同輿 己未徙廣平王羨為西平王【西平縣屬汝南郡賢曰西平故柏子國在今豫州吳房縣西北】秋八月飲酎畢【酎直又翻】有司復奏遣東平王蒼歸國【復扶】<br />
<br />
  【又翻】帝乃許之手詔賜蒼曰骨肉天性誠不以遠近為親疎然數見顔色【數所角翻】情重昔時念王久勞思得還休欲署大鴻臚奏不忍下筆顧授小黄門【賢曰大鴻臚奏王歸國小黄門受詔者臚陵如翻】中心戀戀惻然不能言於是車駕祖送【祖道供張以送之】流涕而訣復賜乘輿服御珍寶輿馬錢布以億萬計【復扶又翻乘繩證翻】 九月甲戍帝幸偃師【偃師縣屬河南郡】東涉卷津【卷縣屬河南郡其北即河津卷丘權翻】至河内下詔曰車駕行秋稼觀收穫【行下孟翻】因涉郡界皆精騎輕行無他輜重【重直用翻】不得輒修道橋遠離城郭【離力智翻】遣吏逢迎刺探起居【賢曰刺探謂伺也刺七亦翻探音湯勘翻】出入前後以為煩擾動務省約但患不能脱粟瓢飲耳【賢曰晏子相齊食脱粟之飯孔子曰顔回一瓢飲】己酉進幸鄴辛卯還宫 冬十月癸丑帝行幸長安封蕭何末孫熊為鄼侯進幸槐里岐山【槐里縣屬扶風杜佑曰槐里周曰犬丘秦曰廢丘漢改曰槐里岐山在扶風美陽縣】又幸長平御池陽宫東至高陵十二月丁亥還宫 東平獻王蒼疾病 【考異曰范書作憲今從袁紀】馳遣名醫小黄門侍疾使者冠蓋不絶於道又置驛馬千里傳問起居【傳直戀翻】<br />
<br />
  八年春正月壬辰王薨詔告中傅封上王自建武以來章奏並集覽焉遣大鴻臚持節監喪【上時掌翻監古銜翻】令四姓小侯諸國王主悉會葬 夏六月北匈奴三木樓訾大人稽留斯等率三萬餘人款五原塞降【稽留斯等部落蓋居三木樓山訾子斯翻】 冬十二月甲午上行幸陳留梁國淮陽頴陽戊申還宫 太子肇之立也梁氏私相慶諸竇聞而惡之【惡烏露翻】皇后欲專名外家忌梁貴人姊妹數譖之於帝【數所角翻】漸致疎嫌是歲竇氏作飛書陷梁竦以惡逆【賢曰飛書若今匿名書也】竦遂死獄中家屬徙九真貴人姊妹以憂死辭語連及梁松妻舞隂公主坐徙新城【新城縣屬河南郡賢曰今洛州伊闕縣】順陽侯馬廖謹篤自守而性寛緩不能教勒子弟皆<br />
<br />
  驕奢不謹校書郎楊終與廖書戒之曰君位地尊重海内所望黄門郎年幼血氣方盛【賢日廖弟防及光俱為黄門郎】既無長君退讓之風【孝文竇皇后兄長君退讓不敢以富貴驕人長知兩翻】而要結輕狡無行之客【要一遥翻行下孟翻】縱而莫誨視成任性覽念前往可為寒心廖不能從防光兄弟資產巨億大起第觀【觀古玩翻】彌亘街路食客常數百人防又多牧馬畜賦歛羌胡帝不喜之數加譴敇【歛力贍翻喜許記翻數所角翻】所以禁遏甚備由是權執稍損賓客亦衰廖子豫為步兵校尉投書怨誹於是有司并奏防光兄弟奢侈踰僭濁亂聖化悉免就國臨上路【上時掌翻】詔曰舅氏一門俱就國封四時陵廟無助祭先后者朕甚傷之其令許侯思諐田廬【許侯光也賢曰留之於京守田廬而思愆過也諐與愆同】有司勿復請【復扶又翻】以慰朕渭陽之情【秦康公送舅晉文公于渭陽念母之不見也其詩曰我見舅氏如母存焉】光比防稍爲謹密故帝特留之後復位特進豫隨廖歸國考擊物故【謂死於考掠也】後復有詔還廖京師【復扶又翻】諸馬既得罪竇氏益貴盛皇后兄憲為侍中虎賁中郎將弟篤為黄門侍郎並侍宫省賞賜累積喜交通賓客【喜許記翻】司空第五倫上疏曰臣伏見虎賁中郎將竇憲椒房之親典司禁兵出入省闥年盛志美卑讓樂善此誠其好士交結之方【樂音洛好呼到翻】然諸出入貴戚者類多瑕釁禁錮之人尤少守約安貧之節士大夫無志之徒更相販賣【少詩沼翻更工衡翻】雲集其門蓋驕佚所從生也三輔論議者至云以貴戚廢錮當復以貴戚浣濯之【復扶又翻】猶解酲當以酒也【病酒曰酲】詖險趣埶之徒誠不可親近【趣七喻翻近其靳翻】臣愚願陛下中宫嚴敇憲等閉門自守無妄交通士大夫防其未萌慮於無形令憲永保福祿君臣交歡無纖介之隙此臣之所至願也憲恃宫掖聲埶自王主及隂馬諸家莫不畏憚憲以賤直請奪沁水公主田園【沁水公主明帝女沁水縣屬河内郡師古曰沁音午浸翻】主逼畏不敢計【計猶今言計較也】後帝出過園【過工禾翻下同】指以問憲憲隂喝不得對【賢曰隂喝猶噎塞也隂音於禁翻喝音一介翻余謂喝訶也許葛翻隂密也潛也當帝問之時密訶左右不得對也觀帝以趙高指鹿為馬責憲則隂喝之義可知矣】後發覺帝大怒召憲切責曰深思前過奪主田園時何用愈趙高指鹿為馬【事見八卷秦二世三年賢曰愈差也】久念使人驚怖【怖普布翻】昔永平中常令隂黨隂博鄧疊三人更相糾察【賢曰以隂鄧皆外戚恐其踰侈故使更相糾察也博隂興之子更工衡翻】故諸豪戚莫敢犯法者今貴主尚見枉奪何况小民哉國家棄憲如孤雛腐鼠耳【賢曰鳥子生而啄曰雛】憲大懼皇后為毁服深謝良久乃得解【毁服猶降服也為于偽翻】使以田還主雖不繩其罪然亦不授以重任臣光曰人臣之罪莫大於欺罔是以明君疾之孝章謂竇憲何異指鹿為馬善矣然卒不能罪憲【卒子恤翻】則姦臣安所懲哉夫人主之於臣下患在不知其姦苟或知之而復赦之【復扶又翻】則不若不知之為愈也何以言之彼或為姦而上不之知猶有所畏既知而不能討彼知其不足畏也則放縱而無所顧矣是故知善而不能用知惡而不能去【去羌呂翻】人主之深戒也【温公此論用齊桓公管仲論郭公所以亡國之意為竇憲擅權張本】<br />
<br />
  下邳周䊸為雒陽令【䊸邕具翻】下車先問大姓主名吏數閭里豪強以對數【數所具翻】䊸厲聲怒曰本問貴戚若馬竇等輩豈能知此賣菜傭乎於是部吏望風旨争以激切為事貴戚跼蹐【跼音局蹐資昔翻毛氏曰跼曲也蹐累足也】京師肅清竇篤夜至止姦亭亭長霍延拔劔擬篤肆詈恣口篤以表聞詔召司隸校尉河南尹詣尚書譴問遣劒戟士收䊸送廷尉詔獄【劒戟士左右都掌之】數日貰出之【賢曰貰赦也市夜翻余謂以貰之為是則收之為非】 帝拜班超為將兵長史【大將軍置長史司馬其不置將軍而長史特將者為將兵長史】以徐幹為軍司馬别遣衛李邑護送烏孫使者邑到于窴值龜兹攻疏勒恐懼不敢前因上書陳西域之功不可成又盛毁超擁愛妻抱愛子安樂外國無内顧心【樂音洛】超聞之歎曰身非曾參而有三至之讒【事見三卷周赧王七年參疏簪翻】恐見疑於當時矣遂去其妻【去羌呂翻】帝知超忠乃切責邑曰縱超擁愛妻抱愛子思歸之士千餘人何能盡與超同心乎令邑詣超受節度詔若邑任在外者便留與從事【任音壬】超即遣邑將烏孫侍子還京師徐幹謂超曰邑前親毁君欲敗西域【敗補邁翻】今何不緣詔書留之更遣他吏送侍子乎超曰是何言之陋也以邑毁超故今遣之内省不疚何卹人言【賢曰疚病也卹憂也論語孔子曰内省不疚夫何憂何懼左氏傳曰詩云禮義不愆何卹人之言詩謂逸詩也省悉景翻】快意留之非忠臣也 帝以侍中會稽鄭弘為大司農【會工外翻】舊交趾七郡貢獻轉運皆從東冶汎海而至【交趾州部南海蒼梧鬱林合浦交趾九真日南七郡賢曰東冶縣屬會稽郡太康地理志云漢武帝名為東冶後改為東官今泉州閩縣是】風波艱阻沉溺相係【沉持林翻溺奴歷翻】弘奏開零陵桂陽嶠道自是夷通遂為常路【賢曰嶠嶺也夷平也余據武帝遣路博德伐南越出桂陽下湟水則舊有是路弘特開之使夷通】在職二年所省息以億萬計遭天下旱邊方有警民食不足而帑藏殷積【說文曰帑金帛所藏之府帑他朗翻藏徂浪翻】弘又奏宜省貢獻減徭費以利飢民帝從之<br />
<br />
  元和元年【是年八月方改元】春閏正月卒丑濟隂悼王長薨【濟子禮翻】 夏四月己卯分東平國封獻王子尚為任城王【任城國在雒陽東千一百里任音壬】 六月辛酉沛獻王輔薨 陳事者多言郡國貢舉率非功次故守職益懈【懈古隘翻】而吏事寖疏【疏與疎同】咎在州郡有詔下公卿朝臣議大鴻臚韋彪上議曰夫國以簡賢為務賢以孝行為首【行下孟翻下同】是以求忠臣必於孝子之門【賢曰孝經緯之文也】夫人才行少能相兼【少詩沼翻】是以孟公綽優於趙魏老不可以為滕薛大夫【論語孔子之言也公綽魯大夫趙魏晉卿之邑也家臣稱老公綽性寡欲趙魏老優閒無事滕薛小國大夫職繁故不可為也】忠孝之人持心近厚鍜鍊之吏持心近薄【蒼頡篇曰鍜椎也鍜鍊猶成熟言深文之吏入人之罪猶工冶陶鑄鍜鍊使之成熟也近其靳翻】士宜以才行為先不可純以閥閲【史記曰明其等曰閥積功曰閲行下孟翻】然其要歸在於選二千石二千石賢則貢舉皆得其人矣彪又上疏曰天下樞要在於尚書【賢曰百官志曰尚書主知公卿二千石吏官上書外國夷狄事故曰樞要】尚書之選豈可不重而間者多從郎官超升此位雖曉習文法長於應對然察察小慧類無大能宜鑒嗇夫捷急之對深思絳侯木訥之功也【嗇夫事見十四卷文帝三年】帝皆納之彪賢之玄孫也【韋賢相元帝】秋七月丁未詔曰律云掠者唯得榜笞立【蒼頡篇曰掠問也廣雅曰榜擊也音彭說文曰笞擊也立謂立而考訊之掠音亮榜音彭】又令丙箠長短有數【賢曰令丙為篇之次也前書音義曰令有先後有令甲令乙令丙又景帝定箠令箠長五尺本大一寸其竹也末薄半寸其平去節故云長短有數箠止蘂翻】自往者大獄以來掠者多酷鉆鑚之屬慘苦無極【大獄謂楚王英等獄也鉆其廉翻說文曰鉆鋷也國語曰中刑用鑽鑿皆謂慘酷其肌膚也】念其痛毒怵然動心【怵敕律翻悚懼也】宜及秋冬治獄明為其禁【治直之翻】 八月甲子太尉鄧彪罷以大司農鄭弘為太尉 癸酉詔改元【改元元和】丁酉車駕南廵詔所經道上州縣毋得設儲峙【賢曰儲積也峙具也言不得豫有蓄備峙丈里翻】命司空自將徒支拄橋梁【司空掌水土故使之拄竹柱翻】有遣使奉迎探知起居【探湯勘翻】二千石當坐 九月辛丑幸章陵十月己未進幸江陵還幸宛召前臨淮太守宛人朱暉拜尚書僕射【宛於元翻】暉在臨淮有善政民歌之曰彊直自遂南陽朱季吏畏其威民懷其惠時坐法免家居【東觀記曰坐考長史囚死獄中州奏免官】故上召而用之十一月己丑車駕還宫尚書張林上言縣官經用不足宜自煮鹽及復修武帝均輸之法【煮鹽均輸皆始於武帝賢曰武帝作均輸法謂州郡所出租賦并雇運之直官總取之市其土地所出之物官自轉輸於京謂之均輸】朱暉固執以為不可曰均輸之法與賈販無異【賈音古】鹽利歸官則下民窮怨誠非明主所宜行帝因發怒切責諸尚書暉等皆自繫獄三日詔敇出之曰國家樂聞駁義【樂音洛駁北角翻】黄髪無愆【黄髪老稱謂朱暉也】詔書過耳何故自繫暉因稱病篤不肯復署議【復扶又翻下同】尚書令以下惶怖謂暉曰今臨得譴讓【謂譴讓已臨乎其前也怖普布翻】奈何稱病其禍不細暉曰行年八十蒙恩得在機密當以死報若心知不可而順旨靁同負臣子之義今耳目無所聞見伏待死命遂閉口不復言諸尚書不知所為乃共劾奏暉【劾戶槩翻又戶得翻】帝意解寑其事後數日詔使直事郎問暉起居【賢曰直事郎謂署郎當次直者】太醫視疾太官賜食暉乃起謝【上既加禮乃起謝所謂彊直自遂也】復賜錢十萬布百匹衣十領 魯國孔僖涿郡崔駰【駰音因】同遊太學相與論孝武皇帝始為天子崇信聖道五六年間號勝文景及後恣已忘其前善鄰房生梁郁上書告駰僖誹謗先帝刺譏當世事下有司駰詣吏受訊【受訊謂受鞫問也下遐稼翻】僖以書自訟曰凡言誹謗者謂實無此事而虛加誣之也至如孝武皇帝政之美惡顯在漢史坦如日月是為直說書傳實事【傳柱戀翻】非虚謗也夫帝者為善為惡天下莫不知斯皆有以致之故不可以誅於人也【誅責也】且陛下即位以來政教未過【賢曰言政教未有過失也】而德澤有加天下所具也【謂天下之人所具知也】臣等獨何譏刺哉假使所非實是則固應悛改【悛丑緣翻】儻其不當【當丁浪翻】亦宜含容又何罪焉陛下不推原大數深自為計徒肆私忌以快其意臣等受戮死即死耳顧天下之人必回視易慮以此事闚陛下心自今以後苟見不可之事終莫復言者矣【復扶又翻下同】齊桓公親揚其先君之惡以唱管仲【國語曰魯莊公束縛管仲以與齊桓公公親迎於郊而與之坐問曰昔吾先君築臺以為高位田狩畢弋不聽國政卑聖侮士而唯女是崇九妃六嬪陳妾數百食必粱肉衣必文繡戎士凍餒是以國家不日引不月長恐宗廟不掃除社稷不血食敢問為此若何管子對以致霸之術】然後羣臣得盡其心今陛下乃欲為十世之武帝遠諱實事【此言十世不以赤劉之九為數直以武昭宣元成哀平光明及帝為數為于偽翻】豈不與桓公異哉臣恐有司卒然見構【卒讀曰猝】銜恨蒙枉不得自叙使後世論者擅以陛下有所比方寧可復使子孫追掩之乎謹詣闕伏待重誅書奏帝立詔勿問拜僖蘭臺令史【百官志蘭臺令史六百石掌奏及印工文書】 十二月壬子詔前以妖惡禁錮三屬者一皆蠲除之【賢曰三屬即三族也謂父族母族及妻族左傳曰以重幣錮之杜預曰禁錮勿令仕也妖於驕翻】但不得在宿衛而已 廬江毛義東平鄭均皆以行義稱於鄉里【行下孟翻】南陽張奉慕義名往之坐定而府檄適至以義守安陽令【賢曰檄召書也東觀記曰義為安陽尉府檄至令守令也安陽縣屬汝南郡賢曰安陽故城在今豫州新息縣西南】義捧檄而入喜動顔色奉心賤之辭去後義母死徵辟皆不至奉乃歎曰賢者固不可測往日之喜乃為親屈也【為于偽翻】均兄為縣吏頗受禮遺【遺于季翻】均諫不聽乃脱身為傭歲餘得錢帛歸以與兄曰物盡可復得【復扶又翻下同】為吏坐臧終身捐棄【臧與同】兄感其言遂為廉潔均仕為尚書免歸帝下詔褒寵義均賜穀各千斛常以八月長吏問起居加賜羊酒 【考異曰義傳云建初中今從均傳】 武威太守孟雲上言北匈奴復願與吏民合市詔許之北匈奴大且渠伊莫訾王等【且子閭翻訾子斯翻】驅牛馬萬餘頭來與漢交易南單于遣輕騎出上郡鈔之【鈔楚交翻】大獲而還帝復遣假司馬和恭等【姓譜和本自羲和之後一云卞和之後】將兵八<br />
<br />
  百人詣班超超因發疏勒于窴兵擊莎車莎車以賂誘疏勒王忠【莎素何翻】忠遂反從之西保烏即城超乃更立其府丞成大為疏勒王【更工衡翻】悉發其不反者以攻忠使人說康居王執忠以歸其國【超立忠為疏勒王見上卷明帝永平十七年說輸芮翻】烏即城遂降【降戶江翻】<br />
<br />
  資治通鑑卷四十六  <br>
   </div> 

<script src="/search/ajaxskft.js"> </script>
 <div class="clear"></div>
<br>
<br>
 <!-- a.d-->

 <!--
<div class="info_share">
</div> 
-->
 <!--info_share--></div>   <!-- end info_content-->
  </div> <!-- end l-->

<div class="r">   <!--r-->



<div class="sidebar"  style="margin-bottom:2px;">

 
<div class="sidebar_title">工具类大全</div>
<div class="sidebar_info">
<strong><a href="http://www.guoxuedashi.com/lsditu/" target="_blank">历史地图</a></strong>  
<a href="http://www.880114.com/" target="_blank">英语宝典</a>  
<a href="http://www.guoxuedashi.com/13jing/" target="_blank">十三经检索</a> 
<br><strong><a href="http://www.guoxuedashi.com/gjtsjc/" target="_blank">古今图书集成</a></strong> 
<a href="http://www.guoxuedashi.com/duilian/" target="_blank">对联大全</a> <strong><a href="http://www.guoxuedashi.com/xiangxingzi/" target="_blank">象形文字典</a></strong> 

<br><a href="http://www.guoxuedashi.com/zixing/yanbian/">字形演变</a>  <strong><a href="http://www.guoxuemi.com/hafo/" target="_blank">哈佛燕京中文善本特藏</a></strong>
<br><strong><a href="http://www.guoxuedashi.com/csfz/" target="_blank">丛书&方志检索器</a></strong> <a href="http://www.guoxuedashi.com/yqjyy/" target="_blank">一切经音义</a>  

<br><strong><a href="http://www.guoxuedashi.com/jiapu/" target="_blank">家谱族谱查询</a></strong>  <strong><a href="http://shufa.guoxuedashi.com/sfzitie/" target="_blank">书法字帖欣赏</a></strong> 
<br>

</div>
</div>


<div class="sidebar" style="margin-bottom:0px;">

<font style="font-size:22px;line-height:32px">QQ交流群9:489193090</font>


<div class="sidebar_title">手机APP 扫描或点击</div>
<div class="sidebar_info">
<table>
<tr>
	<td width=160><a href="http://m.guoxuedashi.com/app/" target="_blank"><img src="/img/gxds-sj.png" width="140"  border="0" alt="国学大师手机版"></a></td>
	<td>
<a href="http://www.guoxuedashi.com/download/" target="_blank">app软件下载专区</a><br>
<a href="http://www.guoxuedashi.com/download/gxds.php" target="_blank">《国学大师》下载</a><br>
<a href="http://www.guoxuedashi.com/download/kxzd.php" target="_blank">《汉字宝典》下载</a><br>
<a href="http://www.guoxuedashi.com/download/scqbd.php" target="_blank">《诗词曲宝典》下载</a><br>
<a href="http://www.guoxuedashi.com/SiKuQuanShu/skqs.php" target="_blank">《四库全书》下载</a><br>
</td>
</tr>
</table>

</div>
</div>


<div class="sidebar2">
<center>


</center>
</div>

<div class="sidebar"  style="margin-bottom:2px;">
<div class="sidebar_title">网站使用教程</div>
<div class="sidebar_info">
<a href="http://www.guoxuedashi.com/help/gjsearch.php" target="_blank">如何在国学大师网下载古籍?</a><br>
<a href="http://www.guoxuedashi.com/zidian/bujian/bjjc.php" target="_blank">如何使用部件查字法快速查字?</a><br>
<a href="http://www.guoxuedashi.com/search/sjc.php" target="_blank">如何在指定的书籍中全文检索?</a><br>
<a href="http://www.guoxuedashi.com/search/skjc.php" target="_blank">如何找到一句话在《四库全书》哪一页?</a><br>
</div>
</div>


<div class="sidebar">
<div class="sidebar_title">热门书籍</div>
<div class="sidebar_info">
<a href="/so.php?sokey=%E8%B5%84%E6%B2%BB%E9%80%9A%E9%89%B4&kt=1">资治通鉴</a> <a href="/24shi/"><strong>二十四史</strong></a>&nbsp; <a href="/a2694/">野史</a>&nbsp; <a href="/SiKuQuanShu/"><strong>四库全书</strong></a>&nbsp;<a href="http://www.guoxuedashi.com/SiKuQuanShu/fanti/">繁体</a>
<br><a href="/so.php?sokey=%E7%BA%A2%E6%A5%BC%E6%A2%A6&kt=1">红楼梦</a> <a href="/a/1858x/">三国演义</a> <a href="/a/1038k/">水浒传</a> <a href="/a/1046t/">西游记</a> <a href="/a/1914o/">封神演义</a>
<br>
<a href="http://www.guoxuedashi.com/so.php?sokeygx=%E4%B8%87%E6%9C%89%E6%96%87%E5%BA%93&submit=&kt=1">万有文库</a> <a href="/a/780t/">古文观止</a> <a href="/a/1024l/">文心雕龙</a> <a href="/a/1704n/">全唐诗</a> <a href="/a/1705h/">全宋词</a>
<br><a href="http://www.guoxuedashi.com/so.php?sokeygx=%E7%99%BE%E8%A1%B2%E6%9C%AC%E4%BA%8C%E5%8D%81%E5%9B%9B%E5%8F%B2&submit=&kt=1"><strong>百衲本二十四史</strong></a>  <a href="http://www.guoxuedashi.com/so.php?sokeygx=%E5%8F%A4%E4%BB%8A%E5%9B%BE%E4%B9%A6%E9%9B%86%E6%88%90&submit=&kt=1"><strong>古今图书集成</strong></a>
<br>

<a href="http://www.guoxuedashi.com/so.php?sokeygx=%E4%B8%9B%E4%B9%A6%E9%9B%86%E6%88%90&submit=&kt=1">丛书集成</a> 
<a href="http://www.guoxuedashi.com/so.php?sokeygx=%E5%9B%9B%E9%83%A8%E4%B8%9B%E5%88%8A&submit=&kt=1"><strong>四部丛刊</strong></a>  
<a href="http://www.guoxuedashi.com/so.php?sokeygx=%E8%AF%B4%E6%96%87%E8%A7%A3%E5%AD%97&submit=&kt=1">說文解字</a> <a href="http://www.guoxuedashi.com/so.php?sokeygx=%E5%85%A8%E4%B8%8A%E5%8F%A4&submit=&kt=1">三国六朝文</a>
<br><a href="http://www.guoxuedashi.com/so.php?sokeytm=%E6%97%A5%E6%9C%AC%E5%86%85%E9%98%81%E6%96%87%E5%BA%93&submit=&kt=1"><strong>日本内阁文库</strong></a> <a href="http://www.guoxuedashi.com/so.php?sokeytm=%E5%9B%BD%E5%9B%BE%E6%96%B9%E5%BF%97%E5%90%88%E9%9B%86&ka=100&submit=">国图方志合集</a> <a href="http://www.guoxuedashi.com/so.php?sokeytm=%E5%90%84%E5%9C%B0%E6%96%B9%E5%BF%97&submit=&kt=1"><strong>各地方志</strong></a>

</div>
</div>


<div class="sidebar2">
<center>

</center>
</div>
<div class="sidebar greenbar">
<div class="sidebar_title green">四库全书</div>
<div class="sidebar_info">

《四库全书》是中国古代最大的丛书,编撰于乾隆年间,由纪昀等360多位高官、学者编撰,3800多人抄写,费时十三年编成。丛书分经、史、子、集四部,故名四库。共有3500多种书,7.9万卷,3.6万册,约8亿字,基本上囊括了古代所有图书,故称“全书”。<a href="http://www.guoxuedashi.com/SiKuQuanShu/">详细>>
</a>

</div> 
</div>

</div>  <!--end r-->

</div>
<!-- 内容区END --> 

<!-- 页脚开始 -->
<div class="shh">

</div>

<div class="w1180" style="margin-top:8px;">
<center><script src="http://www.guoxuedashi.com/img/plus.php?id=3"></script></center>
</div>
<div class="w1180 foot">
<a href="/b/thanks.php">特别致谢</a> | <a href="javascript:window.external.AddFavorite(document.location.href,document.title);">收藏本站</a> | <a href="#">欢迎投稿</a> | <a href="http://www.guoxuedashi.com/forum/">意见建议</a> | <a href="http://www.guoxuemi.com/">国学迷</a> | <a href="http://www.shuowen.net/">说文网</a><script language="javascript" type="text/javascript" src="https://js.users.51.la/17753172.js"></script><br />
  Copyright &copy; 国学大师 古典图书集成 All Rights Reserved.<br>
  
  <span style="font-size:14px">免责声明:本站非营利性站点,以方便网友为主,仅供学习研究。<br>内容由热心网友提供和网上收集,不保留版权。若侵犯了您的权益,来信即刪。scp168@qq.com</span>
  <br />
ICP证:<a href="http://www.beian.miit.gov.cn/" target="_blank">鲁ICP备19060063号</a></div>
<!-- 页脚END --> 
<script src="http://www.guoxuedashi.com/img/plus.php?id=22"></script>
<script src="http://www.guoxuedashi.com/img/tongji.js"></script>

</body>
</html>
