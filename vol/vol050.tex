資治通鑑卷五十    宋 司馬光 撰

胡三省 音註

漢紀四十二|{
	起柔兆執徐盡閼逢困敦凡九年}


孝安皇帝中

元初三年春正月蒼梧鬰林合浦蠻夷反|{
	三郡皆屬交州}
二月遣侍御史任逴督州郡兵討之|{
	任音壬賢曰逴音丁角翻又音卓}
郡國十地震 三月辛亥日有食之 夏四月京師旱五月武陵蠻反州郡討破之 癸酉度遼將軍鄧遵率南單于擊零昌於靈州|{
	范書匈奴傳曰自置度遼將軍以來皆權行其事獨遵以皇太后從弟為真將軍此後更無行將軍者志云度遼將軍銀印青綬秩二千石靈州縣屬北地郡賢曰在今慶州馬嶺縣西北零音憐}
斬首八百餘級 越嶲徼外夷舉種内屬|{
	嶲音髓徼吉弔翻種章勇翻}
六月中郎將任尚|{
	任音壬}
遣兵擊破先零羌於丁奚城|{
	零音憐}
秋七月武陵蠻復反|{
	復扶又翻}
州郡討平之 九月築馮翊北界候塢五百所以備羌|{
	馮翊北界接安定北地}
冬十一月蒼梧鬰林合浦蠻夷降|{
	降戶江翻}
舊制公卿二千石刺史俱不行三年喪司徒劉愷以為非所以師表百姓宣美風俗丙戌初聼大臣行三年喪|{
	賢曰文帝遺詔以日易月於後大臣遂以為常至此復遵古制}
癸卯郡國九地震 十二月丁己任尚遣兵擊零昌於北地殺其妻子燒其廬舍斬首七百餘級|{
	羌勢自此衰矣}


四年春二月乙己朔日有食之 乙卯赦天下 壬戌武庫災 任尚遣當闐種羌榆鬼等刺殺杜季貢|{
	闐徒賢翻種章勇翻刺七亦翻下同}
封榆鬼為破羌侯 司空袁敞亷勁不阿權貴失鄧氏旨尚書郎張俊有私書與敞子俊怨家封上之|{
	怨於元翻上時掌翻}
夏四月戊申敞坐策免自殺俊等下獄當死|{
	下遐稼翻}
俊上書自訟臨刑太后詔以減死論 己巳遼西鮮卑連休等入寇 |{
	考異曰范書鮮卑傳上作連休下作休連今從上文按遼西郡在雒陽東北三千三百里賢曰遼西郡故城在今平州東陽樂城是}
郡兵與烏桓大人於秩居等共擊大破之斬首千三百級 六月戊辰三郡雨雹|{
	雨于具翻}
尹就坐不能定益州徵扺罪以益州刺史張喬領其軍屯招誘叛羌稍稍降散|{
	誘音酉降戶江翻}
秋七月京師及郡國十雨水 九月護羌校尉任尚復募効功種羌號封刺殺零昌|{
	復扶又翻}
封號封為羌王 冬十一月己卯彭城靖王恭薨 越嶲人以郡縣賦歛煩數|{
	嶲音髓歛力瞻翻數所角翻}
十二月大牛種封離等反殺遂久令|{
	遂久縣屬越嶲郡賢曰遂久故縣在今靡州界考異曰西南夷傳云五年叛今從帝紀}
甲子任尚與騎都尉馬賢共擊先零羌狼莫追至北地相持六十餘日戰於富平河上大破之|{
	范書帝紀作富平上河西羌傳作河上賢曰富平縣屬北地郡故城在今靈州回樂縣西南余按水經河水東北逕安定郡眴卷縣故城西注曰地理志河水别出為河溝東至富平北入河河水於此有上河之名前漢馮參為上河典農都尉則上河為是宋白曰唐靈州即漢富平縣之地杜佑曰漢富平今靈州迴樂縣應劭曰眗音旬日之旬卷音箘簬之箘}
斬首五千級狼莫逃去於是西河䖍人種羌萬人詣鄧遵降隴右平|{
	狼莫者零昌之謀主零昌既死而狼莫敗逃䖍人羌失援而降故隴右平降戶江翻}
是歲郡國十三地震

五年春三月京師及郡國五旱 夏六月高句驪與濊貊寇玄菟|{
	句如字又音駒驪力知翻濊音穢貊莫百翻菟同都翻}
永昌益州蜀郡夷皆叛應封離衆至十餘萬破壞二十餘縣|{
	壞音怪}
殺長吏焚掠百姓骸骨委積千里無人 秋八月丙申朔日有食之 代郡鮮卑入寇殺長吏 |{
	考異曰獨行傳云元初中鮮卑數百餘騎寇漁陽太守張顯率吏士追出塞遥望虜營煙火急趣之兵馬掾嚴授慮有伏兵苦諫止不聼顯䠞令進授不獲已前戰伏兵發授身被十創沒于陳顯拔刃追散兵不能制虜射中顯主簿衛福功曹徐咸遽趣之顯遂墮馬福以身擁蔽虜并殺之朝廷愍授等節詔書褒歎厚加賞賜按元初凡六年鮮卑不曾犯漁陽殺長吏惟是入代郡曾殺長吏今疑漁陽本是代郡史之誤也 余按張顯事通鑑已書于上卷殤帝延平元年從范書帝紀也}
發緣邉甲卒黎陽營兵屯上谷以備之冬十月鮮卑寇上谷攻居庸關|{
	郡國志居庸縣屬上谷郡新唐志幽州昌平縣西北三十五里有納欵關即居庸故關}
復發緣邉諸郡黎陽營兵積射士步騎二萬人|{
	復扶又翻}
屯列衝要 鄧遵募上郡全無種羌雕何刺殺狼莫|{
	刺七亦翻}
封雕何為羌侯自羌叛十餘年間|{
	永初元年羌叛至是年凡十二年}
軍旅之費凡用二百四十餘億府帑空竭邊民及内郡死者不可勝數|{
	帑它朗翻勝音升}
并凉二州遂至虚耗及零昌狼莫死|{
	零音憐}
諸羌瓦解三輔益州無復寇警詔封鄧遵為武陽侯邑三千戶|{
	東郡有東武陽泰山郡有南武陽鄧隲傳又作舞陽}
遵以太后從弟故爵封優大任尚與遵争功又坐詐增首級受賕枉法贓千萬已上十二月檻車徵尚棄市沒入財物鄧隲子侍中鳳嘗受尚馬隲髠妻及鳳以謝罪|{
	隲職日翻}
是歲郡國十四地震 太后弟悝閶皆卒封悝子廣宗為葉侯閶子忠為西華侯|{
	葉縣屬南陽郡西華縣屬汝南郡悝苦回翻閶音昌葉式涉翻}


六年春二月乙巳京師及郡國四十二地震 夏四月沛國勃海大風雨雹|{
	雨于具翻}
五月京師旱 六月丙戌平原哀王得薨無子 秋七月鮮卑寇馬城塞殺長吏|{
	馬城縣屬代郡賢曰搜神記昔秦人築城于武周塞以備胡將成而崩者數矣有馬馳走周旋反覆父老異之因依以築城城乃不崩遂以名焉其故城則今之朔州也余按續漢志搜神記所云乃鴈門郡之馬邑此乃代郡之馬城賢誤}
度遼將軍鄧遵及中郎將馬續率南單于追擊大破之 九月癸巳陳懷王竦薨無子國除|{
	竦陳王羨之孫}
冬十二月戊午朔日有食之既 郡國八地震 是歲太后徵和帝弟濟北王夀河間王開子男女年五歲以上四十餘人|{
	濟子禮翻}
及鄧氏近親子孫三十餘人並為開邸第|{
	賢曰蒼頡篇曰邸舍也為于偽翻}
教學經書躬自監試|{
	監工銜翻}
詔從兄河南尹豹越騎校尉康等曰末世貴戚食禄之家温衣羨飯乘堅驅良|{
	賢曰堅謂好車良謂善馬余按此語出史記范蠡傳從才用翻}
而面牆弗學不識臧否|{
	尚書曰不學牆面言正牆面而立無所見也否音鄙}
斯故禍敗之所從來也 豫章有芝草生太守劉祗欲上之|{
	上時掌翻}
以問郡人唐檀檀曰方今外戚豪盛君道微弱斯豈嘉瑞乎祇乃止 益州刺史張喬遣從事楊竦將兵至楪榆|{
	楪榆縣武帝開置屬益州郡有葉榆澤正縣東因名明帝分屬永昌郡楪與葉同}
擊封離等大破之斬首三萬餘級獲生口千五百人封離等惶怖斬其同謀渠帥詣竦乞降|{
	怖普布翻帥所類翻降戶江翻下同}
竦厚加慰納其餘三十六種皆來降附|{
	種章勇翻}
竦因奏長吏姦猾侵犯蠻夷者九十人皆減死論 初西域諸國既絶於漢|{
	事見上卷永初元年}
北匈奴復以兵威役屬之|{
	役使而臣屬之復扶又翻下同}
與共為邉寇敦煌太守曹宗患之|{
	敦徒門翻}
乃上遣行長史索班將千餘人屯伊吾以招撫之|{
	索姓出敦煌又左傳啇人七族有索氏上時掌翻索昔各翻上遣索班奏而遣之也行長史者行長史事未為真也}
於是車師前王及鄯善王復來降|{
	鄯正扇翻}
初疏勒王安國死無子國人立其舅子遺腹為王遺腹叔父臣磐在月氏月氏納而立之|{
	西域傳曰元初中安國以舅臣磐冇罪徙於月氏月氏王親愛之遺腹既立月氏遣兵送臣磐還疏勒國人素敬愛臣盤又畏憚月氏即共奪遺腹印綬迎臣磐立以為王氏音支}
後莎車畔于窴屬疏勒|{
	自明帝永平四年莎車屬于窴}
疏勒遂彊與龜兹于窴為敵國焉永寧元年|{
	是年夏四月改元}
春三月丁酉濟北惠王夀薨|{
	濟子禮翻}
北匈奴率車師後王軍就共殺後部司馬及敦煌長

史索班等|{
	賢曰司馬即屬戊巳校尉所統也和帝時置戊巳校尉鎮車師後部 考異曰班勇傳元初六年曹宗遣索班屯伊吾後數月北單于與車師後部共攻沒索班按本紀永寧元年車師後王叛殺部司馬車師傳亦曰永寧元年後王軍就及母沙麻反畔殺部司馬及敦煌行事蓋班以去年末屯伊吾今春見殺或今春奏事方到也}
遂擊走其前王畧有北道鄯善逼急求救於曹宗|{
	鄯上扇翻}
宗因此請出兵五千人擊匈奴以報索班之耻因復取西域公卿多以為宜閉玉門關絶西域太后聞軍司馬班勇有父風召詣朝堂問之|{
	班固西都賦曰左右廷中朝堂百僚之位朝堂蓋在殿庭左右朝直遥翻}
勇上議曰昔孝武皇帝患匈奴彊盛於是開通西域論者以為奪匈奴府藏|{
	上時掌翻藏徂浪翻}
斷其右臂|{
	斷丁亂翻}
光武中興未遑外事故匈奴負彊驅率諸國及至永平再攻敦煌|{
	敦徒門翻}
河西諸郡城門晝閉孝明皇帝深惟廟策|{
	惟思也賢曰古者謀事必就祖故言廟策也余謂古者遣將必於廟先定制勝之策故謂之廟策}
乃命虎臣出征西域|{
	虎臣謂其父超也}
故匈奴遠遁邉境得安及至永元莫不内屬會閒者羌亂西域復絶|{
	復扶又翻}
北虜遂遣責諸國備其逋租高其價直嚴以期會|{
	備償也西域屬漢之後不復以馬畜旃罽輸匈奴及與漢絶匈奴復遣使責其積年所逋逋欠也}
鄯善車師皆懷憤怨思樂事漢|{
	樂音洛}
其路無從前所以時有叛者皆由牧養失宜還為其害故也今曹宗徒耻于前負欲報雪匈奴|{
	負敗也報雪謂報伊吾之役雪索班之耻也}
而不尋出兵故事未度當時之宜也|{
	度徒洛翻}
夫要功荒外萬無一成|{
	要一遥翻荒外謂在荒服之外}
若兵連禍結悔無所及况今府藏未充|{
	藏徂浪翻}
師無後繼是示弱於遠夷暴短於海内臣愚以為不可許也舊敦煌郡有營兵三百人今宜復之復置護西域副校尉居於敦煌如永元故事|{
	校戶教翻}
又宜遣西域長史將五百人屯樓蘭西|{
	樓蘭即鄯善}
當焉耆龜兹徑路南彊鄯善于窴心膽北扞匈奴東近敦煌如此誠便|{
	龜兹音丘慈窴徒賢翻}
尚書復問勇利害云何|{
	勇既上議尚書復問使悉陳其利害復扶又翻}
勇對曰昔永平之末始通西域初遣中郎將居敦煌|{
	謂鄭衆也}
後置副校尉於車師|{
	謂耿恭關寵也}
既為胡虜節度又禁漢人不得有所侵擾故外夷歸心匈奴畏威今鄯善王尤還漢人外孫若匈奴得志則尤還必死|{
	賢曰尤還鄯善王名}
此等雖同鳥獸亦知避害若出屯樓蘭足以招附其心愚以為便|{
	此勇所謂利也}
長樂衛尉鐔顯|{
	類篇鐔如心翻姓也賢曰鐔音徒南翻唐韻又音尋}
廷尉綦母參|{
	綦母姓也左傳晉有綦母張}
司隸校尉崔據難曰|{
	難乃旦翻下同}
朝廷前所以棄西域者以其無益於中國而費難供也今車師已屬匈奴鄯善不可保信一旦反覆班將能保北虜不為邉害乎|{
	賢曰以勇為軍司馬故以將言之將音即亮翻}
勇對曰今中國置州牧者以禁郡縣姦猾盗賊也若州牧能保盗賊不起者臣亦願以要斬保匈奴之不為邉害也|{
	要讀曰腰}
令通西域則虜埶必弱虜埶弱則為患微矣孰與歸其府藏續其斷臂哉今置校尉以扞撫西域設長史以招懷諸國若棄而不立則西域望絶望絶之後屈就北虜緣邉之郡將受困害恐河西城門必將復有晝閉之儆矣|{
	明帝永平中北匈奴脅諸國共寇河西郡縣城門晝閉復扶又翻下同}
今不廓開朝廷之德而拘屯戍之費若此北虜遂熾豈安邉久長之策哉|{
	此勇所謂害也熾尺志翻}
太尉屬毛軫難曰今若置校尉則西域絡繹遣使求索無厭|{
	索山客翻厭於鹽翻百官志太尉掾屬二十四人東西曹掾比四百石餘掾比三百石屬比二百石}
與之則費難供不與則失其心一旦為匈奴所廹當復求救則為役大矣勇對曰今設以西域歸匈奴而使其恩德大漢不為鈔盗則可矣|{
	鈔楚交翻}
如其不然則因西域租入之饒兵馬之衆以擾動緣邉是為富仇讐之財增暴夷之埶也置校尉者宣威布德以擊諸國内向之心而疑匈奴覬覦之情|{
	覬音冀覦音俞}
而無費財耗國之慮也且西域之人無他求索其來入者不過稟食而已|{
	稟筆錦翻給也食讀曰飤}
今若拒絶埶歸北屬夷虜|{
	言其事埶所歸必至北屬匈奴}
并力以寇并凉則中國之費不止十億置之誠便於是從勇議復敦煌郡營兵三百人置西域副校尉居敦煌雖復覊縻西域然亦未能出屯|{
	謂未能如勇計出屯樓蘭西也然使盡行勇之計亦未必能羈制西域何者武帝通西域未能盡臣屬西域也及宣帝時日逐降呼韓邪内附始盡得西域明帝使班超通西域未能盡臣屬西域也及竇憲破北匈奴超始盡得西域今漢内困于諸羌而北匈奴游魂蒲類安能以五百人成功哉}
其後匈奴果數與車師共入寇鈔河西大被其害|{
	數所角翻鈔楚交翻被皮義翻}
沈氐羌寇張掖|{
	賢曰沈氏羌號也續漢書曰羌在上郡西河者號沈氐}
夏四月丙寅立皇子保為太子改元赦天下 己巳紹封陳敬王子崇為陳王濟北惠王子萇為樂成王|{
	濟子禮翻}
河間孝王子翼為平原王 六月護羌校尉馬賢將萬人討沈氐羌於張掖破之斬首千八百級獲生口千餘人餘虜悉降|{
	降戶江翻}
時當煎等大豪飢五等以賢兵在張掖乃乘虚寇金城賢還軍出塞斬首數千級而還燒當燒何種聞賢軍還復寇張掖殺長吏|{
	馬賢於時為健鬬然觀其往來奔命羌人輒議其後賢不思所以制之之術重以不恤軍事宜其冇射姑山之敗也還從宣翻又如字種章勇翻復扶又翻長知兩翻}
秋七月乙酉朔日有食之 冬十月己巳司空李郃免|{
	郃古合翻又閤翻}
癸酉以衛尉廬江陳褒為司空 京師及郡國三十三大水十二月永昌徼外撣國王雍曲調遣使者獻樂及幻人|{
	西南夷傳幻人能變化吐火自支解易牛馬頭自言我海西人海西即大秦也今按大秦即武帝時犁靬國今謂之拂菻撣音檀范書雍曲調作雍由調徼吉弔翻}
戊辰司徒劉愷請致仕許之以千石禄歸養 遼西鮮卑大人烏倫其至鞬各以其衆詣度遼將軍鄧遵降|{
	烏倫其至鞬二人也鞬居言翻}
癸酉以太常楊震為司徒 是歲郡國二十三地震 太后從弟越騎校尉康以太后久臨朝政宗門盛滿數上書太后|{
	從才用翻上元初六年書從兄康此書從弟徵諸范史當從兄字數所角翻上時掌翻}
以為宜崇公室自損私權言甚切至太后不從康謝病不朝太后使内侍者問之所使者乃康家先婢|{
	先本康家婢後入宫在大后左右}
自通中大人|{
	時宫中耆宿皆稱中大人}
康聞而詬之|{
	賢曰詬罵也音許遘翻又古候翻}
婢怨恚|{
	恚於避翻}
還白康詐疾而言不遜太后不怒免康官遣歸國|{
	康永初中紹封夷安侯}
絶屬籍 初當煎種飢五同種大豪盧怱忍良等千餘戶|{
	種章勇翻下同}
别留允街而首施兩端|{
	賢曰首施猶首鼠也允音鉛}


建光元年|{
	是年七月改元 考異曰陳禪傳曰北匈奴入遼東追拜禪遼東大守胡憚其威彊退還數百里禪不加兵但遣吏卒往曉慰之單于隨使還郡禪於學行禮為說道義以感化之單于懷服遺以胡中珍貨而去當在此年矣又按北單于漢朝所不能臣未嘗入朝天子安肯見遼東太守此事可疑今不取 余按和帝以來北匈奴益西徙自代郡以東至遼東塞外之地皆鮮卑烏桓居之北單于安能至遼東邪不取當也}
春護羌校尉馬賢召盧怱斬之因放兵擊其種人獲首虜二千餘忍良等皆亡出塞 幽州刺史巴郡馮煥玄菟太守姚光遼東太守蔡諷等將兵擊高句驪高句麗王宫遣子遂成詐降而襲玄菟遼東殺傷二千餘人|{
	菟同都翻句如字又音駒麗讀曰驪力知翻降戶江翻}
二月皇太后寢疾癸亥赦天下三月癸巳皇太后鄧氏崩未及大斂帝復申前命|{
	歛力贍翻復扶又翻下同}
封鄧隲為上蔡侯位特進|{
	封隲事見上卷永初元年}
丙午葬和熹皇后|{
	范曄曰漢世皇后無謚皆因帝謚以為稱雖呂氏專政上官臨制亦無殊號中興明帝始建光烈之稱其後並以德為配至于賢愚優劣混同一貫故馬竇二后俱稱德焉其餘皇帝之庶母及蕃王承統以追尊之重特為其號恭懷孝崇之比是也初平中蔡邕始追正和熹之諡其安思順烈以下皆依而加焉賢注云蔡邕集諡議曰漢世母后無諡明帝始建光烈之稱是後轉因帝號加之以德上下優劣混而為一違禮大行受大名小行受小名之制謚法冇功安民曰熹帝后一體禮亦宜同大行皇大后宜為和熹}
太后自臨朝以來水旱十載|{
	載子亥翻}
四夷外侵盗賊内起每聞民飢或逹旦不寐躬自減徹|{
	謂減膳徹樂之類}
以救災戹故天下復平歲還豐穰|{
	和熹臨朝之政可謂牝雞之晨唯家之索矣}
上始親政事尚書陳忠薦隱逸及直道之士潁川杜根平原成翊世之徒上皆納用之忠寵之子也初鄧太后臨朝根為郎中與同時郎上書言帝年長宜親政事|{
	長知兩翻}
太后大怒皆令盛以縑囊於殿上撲殺之|{
	盛時征翻撲普卜翻蜀本弼角翻}
既而載出城外根得蘇太后使人檢視根遂詐死三日目中生蛆|{
	蛆子余翻凡蠅所集其遺子之處生為蛆}
因得逃竄為宜城山中酒家保積十五年|{
	宜城縣屬南郡賢曰宜城故城在今襄州率道縣南其地出美酒廣雅云保使也言為人傭力保任而使也}
成翊世以郡吏亦坐諫太后不歸政抵罪帝皆徵詣公車拜根侍御史翊世尚書郎|{
	為諸鄧得罪張本}
或問根曰往者遇禍天下同義|{
	天下之士以根直諫遇禍同義之也}
知故不少|{
	少詩沼翻}
何至自苦如此根曰周旋民間非絶跡之處邂逅發露|{
	邂逅不期而會謂出於意料之外也}
禍及親知故不為也|{
	申屠蟠絶跡梁碭祖根之故智也}
戊申追尊清河孝王曰孝德皇皇妣左氏曰孝德后祖妣宋貴人曰敬隱后|{
	尊其所自出也謚法執義行善曰德綏柔士民曰德不顯尸國曰隱見美堅長曰隱}
初長樂太僕蔡倫受竇后諷旨誣䧟宋貴人|{
	事見四十六卷章帝建初七年樂音洛}
帝敇使自致廷尉倫飲藥死|{
	自致廷尉者使其自詣獄}
夏四月高句麗復與鮮卑入寇遼東蔡諷追擊於新昌戰歿|{
	新昌縣屬遼東郡}
功曹掾龍瑞兵馬掾公孫酺以身扞諷俱歿於陳|{
	范書東夷傳作功曹耿耗兵馬掾龍端酺音蒲陳讀曰陣}
丁巳尊帝嫡母耿姬為甘陵大貴人|{
	郡國志清河郡厝縣帝改名甘陵賢曰甘陵孝德皇之陵也因以為縣在今貝州清河縣東宋白曰貝州清河縣本周甘泉氏之地秦漢為信城縣後漢為厝縣桓帝改為甘陵故城在今縣西北清河王慶陵在今清河郡東南三十里故厝城}
甲子樂成王萇坐驕淫不法貶為蕪湖侯|{
	范書紀傳皆作臨湖侯賢曰臨湖縣屬廬江郡}
己巳令公卿下至郡國守相各舉有道之士一人尚書陳忠以詔書既開諫争|{
	争讀曰諍}
慮言事者必多激切或致不能容乃上疏豫通廣帝意曰臣聞仁君廣山藪之大|{
	賢曰左氏傳曰川澤納汙山藪藏疾瑾瑜匿瑕國君含垢天之道也}
納切直之謀忠臣盡謇諤之節不畏逆耳之害|{
	易曰王臣蹇蹇晉王豹傳作謇史記趙簡子曰衆人之唯唯不如周舍之諤諤家語孔子曰忠言逆耳而利於行也}
是以高祖舍周昌桀紂之譬|{
	周昌嘗燕入奏事高祖方擁戚姬昌還走帝逐得騎昌項問曰我何如主昌仰曰陛下即桀紂之主上笑自是心憚昌舍讀曰捨}
孝文喜袁盎人豕之譏|{
	事見十三卷文帝二年}
武帝納東方朔宣室之正|{
	事見十八卷武帝元光五年}
元帝容薛廣德自刎之切|{
	事見二十八卷元帝永光元年刎武粉翻}
今明詔崇高宗之德推宋景之誠引咎克躬咨訪羣吏|{
	時詔公卿百僚各上封事}
言事者見杜根成翊世等新蒙表録顯列二臺|{
	賢曰謂根為侍御史翊世為尚書郎也余按漢制尚書御史皆曰臺}
必承風響應争為切直若嘉謀異策宜輒納用如其管穴妄有譏刺|{
	賢曰管穴言小也史記扁鵲曰若以管窺天以隙視文隙即穴也}
雖苦口逆耳不得事實且優游寛容以示聖朝無諱之美若有道之士對問高者宜垂省覽|{
	省悉景翻}
特遷一等以廣直言之路書御|{
	御進也書御書進而經覽也}
有詔拜有道高第士沛國施延為侍中|{
	冇道高第舉有道對問為上第也姓譜魯大夫施伯出於魯惠公之子子尾字施父}
初汝南薛包少有至行父娶後妻而憎包分出之包日夜號泣不能去|{
	少詩照翻行下孟翻號戶刀翻}
至被敺扑|{
	以敲扑敺之也扑普卜翻}
不得已廬於舍外旦入洒掃|{
	洒所賣翻掃素報翻又並如字}
父怒又逐之乃廬於里門晨昏不廢|{
	不廢定省之禮也}
積歲餘父母慙而還之及父母亡弟子求分財異居包不能止乃巾分其財奴婢引其老者曰與我共事久若不能使也|{
	若汝也}
田廬取其荒頓者|{
	賢曰頓猶廢也}
曰吾少時所治意所戀也|{
	治直之翻}
器物取朽敗者曰我素所服食身口所安也弟子數破其產輒復賑給|{
	數所角翻復扶又翻}
帝聞其名令公車特徵|{
	特獨也獨徵之當時無與並者}
至拜侍中包以死自乞有詔賜告歸加禮如毛義|{
	毛義事見四十六卷章帝元和元年}
帝少號聰明故鄧太后立之及長多不德|{
	少詩沼翻長知兩翻}
稍不可太后意|{
	言意不以為可也}
帝乳母王聖知之太后徵濟北河間王子詣京師|{
	濟子禮翻}
河間王子翼美容儀太后奇之以為平原懷王後留京師王聖見太后久不歸政慮有廢置常與中黄門李閏江京候伺左右共毁短太后於帝帝每懷忿懼|{
	伺相吏翻}
及太后崩宫人先有受罰者懷怨恚|{
	恚於避翻}
因誣告太后兄弟悝宏閶先從尚書鄧訪取廢帝故事謀立平原王帝聞追怒令有司奏悝等大逆無道遂廢西平侯廣宗葉侯廣德西華侯忠陽安侯珍都鄉侯甫德皆為庶人|{
	忠閶之子珍悝兄京之子西華陽安二縣皆屬汝南郡悝苦回翻閶音昌葉式涉翻}
鄧隲以不與謀但免特進遣就國|{
	與讀曰豫}
宗族免官歸故郡|{
	鄧氏故南陽人}
沒入隲等貲財田宅徙鄧訪及家屬於遠郡郡縣廹逼廣宗及忠皆自殺又徙封隲為羅侯|{
	羅縣屬長沙郡}
五月庚辰隲與子鳳並不食而死隲從弟河南尹豹度遼將軍舞陽侯遵將作大匠暢皆自殺|{
	從才用翻}
唯廣德兄弟以母與閻后同產得留京師復以耿夔為度遼將軍徵樂安侯鄧康為太僕|{
	按范書鄧禹傳明帝分禹國為三封其三子季子珍為夷安侯康以珍之子紹封樂安當在夷安郡國志夷安高密二縣皆屬北海國賢曰夷安故城今高密縣外城}
丙申貶平原王翼為都鄉侯遣歸河間翼謝絶賓客閉門自守由是得免 初鄧后之立也|{
	見四十八卷和帝永元十四年}
太尉張禹司徒徐防欲與司空陳寵共奏追封后父訓寵以先世無奏請故事争之連日不能奪及訓追加封諡禹防復約寵俱遣子奉禮於虎賁中郎將隲寵不從故寵子忠不得志於鄧氏隲等敗忠為尚書數上疏䧟成其惡|{
	寵之所守是也忠之所為非也復扶又翻數所角翻}
大司農京兆朱寵痛隲無罪遇禍乃肉袒輿櫬|{
	櫬初覲翻賢曰襯親身棺}
上疏曰伏惟和熹皇后聖善之德為漢文母|{
	賢曰詩凱風曰母氏聖善文母文王之母太任也寵言太后有聖善之德比於文母也}
兄弟忠孝同心憂國社稷是賴|{
	賢曰殤帝崩大后與隲定立安帝故曰是賴}
功成身退讓國遜位歷世貴戚無與為比當享積善履謙之祐|{
	賢曰易曰積善之家必有餘慶又曰鬼神害盈而福謙}
而横為宫人單辭所䧟|{
	兩造不備又無徵左者為單辭横戶孟翻}
利口傾險反亂國家|{
	論語曰惡利口之覆邦家者}
罪無申證獄不訊鞠|{
	賢曰申明白也訊問也鞠窮也}
遂令隲等罹此酷䧟一門七人並不以命|{
	賢曰七人謂隲從弟豹遵暢隲子鳳鳳從弟廣宗忠也}
屍骸流離寃魂不反逆天感人率土喪氣|{
	喪息浪翻}
宜收還冢次寵樹遺孤奉承血祀以謝亡靈|{
	賢曰血祀謂祭廟殺牲取血以降神也}
寵知其言切自致廷尉陳忠復劾奏寵詔免官歸田里|{
	劾戶概翻又戶得翻}
衆庶多為隲稱枉者|{
	復扶又翻為于偽翻}
帝意頗悟乃譴讓州郡|{
	賢曰以逼迫廣宗等故也}
還葬隲等於北芒|{
	賢曰北芒山在雒陽城北}
諸從兄弟皆得歸京師|{
	從才用翻}
帝以耿貴人兄牟平侯寶監羽林左軍車騎|{
	羽林分左右監各主左右騎寶監古銜翻}
封宋楊四子皆為列侯宋氏為卿校侍中大夫謁者郎吏十餘人|{
	校戶教翻}
閻皇后兄弟顯景耀並為卿校典禁兵|{
	卿校九卿及諸校尉也}
於是内寵始盛帝以江京嘗迎帝於邸|{
	謂延平元年迎帝於清河邸也}
以為京功封都鄉侯封李閏為雍鄉侯閏京並遷中常侍京兼大長秋與中常侍樊豐黄門令劉安鈎盾令陳達|{
	百官志黄門令主省中諸宦者鉤盾令典諸近地苑囿游觀之處皆宦者為之盾食尹翻}
及王聖聖女伯榮扇動内外競為侈虐伯榮出入宫掖傳通姦賂司徒楊震上疏曰臣聞政以得賢為本治以去穢為務|{
	治直吏翻去羌呂翻}
是以唐虞俊乂在官四凶流放|{
	見尚書孔安國曰俊乂俊德能治之士馬融曰千人曰俊百人曰乂}
天下咸服以致雍熙|{
	雍和也熙亦和也}
方今九德未事|{
	賢曰尚書臯陶謨曰亦行有九德寛而栗柔而立愿而恭亂而敬擾而毅直而温簡而亷剛而塞彊而義又曰九德咸事俊乂在官孔安國曰使九德之人皆用事}
嬖倖充庭|{
	諡曰賤而得愛曰嬖}
阿母王聖出自賤微得遭千載|{
	載子亥翻}
奉養聖躬雖有推燥居溼之勤|{
	孝經援神契曰母之於子也鞠養殷勤推燥居溼絶少分甘也推吐雷翻}
前後賞惠過報勞苦而無厭之心不知紀極外交屬託|{
	厭於鹽翻屬之欲翻}
擾亂天下損辱清朝塵點日月|{
	朝直遥翻}
夫女子小人近之喜遠之怨實為難養|{
	論語曰唯女子與小人為難養也近之則不遜遠之則怨近其靳翻遠于願翻}
宜速出阿母令居外舍斷絶伯榮莫使往來|{
	斷丁管翻}
令恩德兩隆上下俱美奏御帝以示阿母等内倖皆懷忿恚|{
	恚於避翻}
而伯榮驕淫尤甚通於故朝陽侯劉護從兄瓌遂以為妻官至侍中得襲護爵|{
	賢曰護泗水王歙之從曾孫朝陽縣屬南郡故城在今鄧州穰縣南今謂之朝城從才用翻古回翻}
震上疏曰經制父死子繼兄亡弟及以防篡也|{
	賢曰公羊傳曰劉子單子以王猛入于王城者何西周也其言入何簒亂也冬十月王子猛卒此未踰年之君其稱王子猛卒何不予當也不予當者不予當父死子繼兄亡弟及也}
伏見詔書封故朝陽侯劉護再從兄瓌襲護爵為侯|{
	從才用翻}
護同產弟威今猶見在|{
	見賢遍翻}
臣聞天子專封封有功諸侯專爵爵有德今無它功行|{
	行下孟翻}
但以配阿母女一時之間既位侍中又制封侯不稽舊制不合經義行人喧譁百姓不安陛下宜鑒鏡既往順帝之則尚書廣陵翟酺|{
	范書列傳酺廣漢雒人陵當作漢廣漢郡屬益州翟直格翻酺音蒲}
上疏曰昔竇鄧之寵傾動四方兼官重紱|{
	重直龍翻}
盈金積貨至使議弄神器|{
	賢曰神器謂天位也老子曰天下神器不可為也余謂威福人主之神器此言弄威福耳}
改更社稷|{
	更工衡翻}
豈不以埶尊威廣以致斯患乎及其破壞頭顙墮地願為孤豚豈可得哉夫致貴無漸失必暴受爵非道殃必疾今外戚寵幸功鈞造化漢元以來未有等比|{
	漢元漢初也比頻寐翻}
陛下誠仁恩周洽以親九族然禄去公室政移私門覆車重尋寧無摧折|{
	重直龍翻折而設翻}
此最安危之極戒社稷之深計也昔文帝愛百金於露臺飾帷帳於皁囊|{
	文帝集上書皁囊以為殿帷}
或有譏其儉者上曰朕為天下守財耳|{
	為于偽翻}
豈得妄用之哉今自初政以來日月未久費用賞賜已不可算歛天下之財積無功之家帑藏單盡|{
	帑它朗翻藏徂浪翻單與殫同}
民物彫傷卒有不虞|{
	卒讀曰猝虞度也不虞謂事變出於虞度之外者也}
復當重賦|{
	復扶又翻}
百姓怨叛既生危亂可待也願陛下勉求忠貞之臣誅遠佞諂之黨割情欲之歡罷宴私之好|{
	遠于願翻好呼到翻}
心存亡國所以失之鑒觀興王所以得之庶災害可息豐年可招矣書奏皆不省|{
	省悉景翻}
秋七月己卯改元赦天下 壬寅太尉馬英薨 |{
	考異曰傳作策罷誤今從紀}
燒當羌忍良等以麻奴兄弟本燒當世嫡|{
	燒當豪帥東號和帝永元元年降其子麻奴永初元年叛出塞}
而校尉馬賢撫恤不至常有怨心遂相結共脅將諸種寇湟中攻金城諸縣八月賢將先零種擊之戰於牧苑不利|{
	漢邊郡皆冇牧苑以養馬此牧苑在金城界將即亮翻零音憐種章勇翻}
麻奴等又敗武威張掖郡兵於令居|{
	敗補邁翻令孟康音連師古音零}
因脅將先零沈氐諸種四千餘戶緣山西走寇武威賢追到鸞鳥|{
	鸞鳥縣屬武威郡鳥音雀賢曰鸞鳥故城在今涼州昌松縣北鸞音雚沽丸翻劉昫曰涼州神鳥縣漢鸞鳥縣地嘉麟縣則鸞鳥古城也}
招引之諸種降者數千|{
	降戶江翻}
麻奴南還湟中 甲子以前司徒劉愷為太尉初清河相叔孫光坐臧抵罪遂增禁錮二世|{
	臧古贓字通賢曰二世謂父子俱禁錮}
至是居延都尉范邠復犯臧罪朝廷欲依光比|{
	帝置居延屬國都尉别領居延一城屬涼州復扶又翻賢曰比類也以邠類光亦錮及其子也比音庇}
劉愷獨以春秋之義善善及子孫惡惡止其身所以進人於善也|{
	公羊傳曰曹公孫會自鄸出奔宋畔也曷為不言畔為公子喜時之後諱也春秋為賢者諱也何賢乎公子喜時讓國也君子之善善也長惡惡也短惡惡止其身善善及子孫賢者子孫故君子為其諱也}
如今使臧吏禁錮子孫以輕從重懼及善人非先王詳刑之意也|{
	左傳曰刑濫則懼及善人鄭玄曰詳審察也}
陳忠亦以為然有詔太尉議是 鮮卑其至護寇居庸關九月雲中太守成嚴擊之兵敗|{
	居庸關在上谷界蓋鮮卑先寇居庸關遂入雲中界也}
功曹楊穆以身扞嚴與之俱歿鮮卑於是圍烏桓校尉徐常於馬城度遼將軍耿夔與幽州刺史龐參發廣陽漁陽涿郡甲卒救之|{
	三郡皆屬幽州}
鮮卑解去 戊子帝幸衛尉馮石府留飲十餘日 |{
	考異曰袁紀曰十二月丙申乃還宫今從石傳}
賞賜甚厚拜其子世為黄門侍郎世弟二人皆為郎中石陽邑侯魴之孫也|{
	按范書馮魴封陽邑鄉侯魴音房}
父柱尚顯宗女獲嘉公主石襲公主爵為獲嘉侯|{
	獲嘉縣屬河内郡本汲之新中鄉也武帝行幸過此聞獲呂嘉因以名縣}
能取悦當世故為帝所寵 京師及郡國二十七雨水|{
	范書帝紀作二十九}
冬十一月己丑郡國三十五地震鮮卑寇玄菟|{
	菟同都翻}
尚書令祋諷等奏以為孝文定約禮之制光武皇帝絶告寧之典|{
	祋丁外翻又丁活翻姓也約禮謂以日易月也前書音義曰告寧休謁之名吉曰告凶曰寧}
貽則萬世誠不可改宜復斷大臣行三年喪|{
	斷丁管翻下同}
尚書陳忠上疏曰高祖受命蕭何創制大臣有寧告之科合於致憂之義|{
	論語曰人未有自致者也必也親喪乎}
建武之初新承大亂凡諸國政多趣簡易|{
	趣七喻翻易以䜴翻}
大臣既不得告寧而羣司營禄念私鮮循三年之喪|{
	鮮息淺翻}
以報顧復之恩者|{
	詩蓼莪云長我育我顧我復我欲報之德昊天罔極}
禮義之方實為彫損陛下聼大臣終喪聖功美業靡以尚兹孟子曰老吾老以及人之老幼吾幼以及人之幼天下可運於掌|{
	賢曰言敬吾老亦敬人之老愛吾幼亦愛人之幼有敬愛之心則天下歸順之也運掌言易也范氏曰老吾老以老者之禮養吾之老則謂事親也使天下之老者皆得其養故曰以及人之老幼吾幼以幼者之禮待吾之幼謂愛其子也使天下之幼者皆得其長故曰以及人之幼天下可運於掌言其易也}
臣願陛下登高北望以甘陵之思揆度臣子之心則海内咸得其所|{
	賢曰甘陵帝父母陵陵在清河故北望也度徒洛翻}
時宦官不便之竟寢忠奏庚子復斷二千石以上行三年喪|{
	元初三年聼大臣行三年喪今復斷之斷音短}


袁宏論曰古之帝王所以篤化美俗率民為善因其自然而不奪其情民猶有不及者而况毁禮止哀滅其天性乎

十二月高句驪王宫率馬韓濊貊數千騎圍玄菟|{
	韓有三種一曰馬韓二曰辰韓三曰弁辰馬韓在西有五十四國句如字又音駒驪力知翻濊音穢貊莫百翻}
夫餘王遣子尉仇台將二萬餘人與州郡并力討破之|{
	夫音扶}
是歲宫死子遂成立玄菟太守姚光上言欲因其喪發兵擊之議者皆以為可許陳忠曰宫前桀黠|{
	黠下八翻}
光不能討死而擊之非義也宜遣使弔問因責讓前罪赦不加誅取其後善帝從之

延光元年春三月丙午改元赦天下 護羌校尉馬賢追擊麻奴到湟中破之種衆散遁|{
	種章勇翻}
夏四月京師郡國四十一雨雹|{
	雨于具翻}
河西雹大者如斗 幽州刺史馮煥玄菟太守姚光數糾發姦惡|{
	數所角翻}
怨者詐作璽書譴責煥光賜以歐刀|{
	賢曰歐刀刑人之刀也歐音一口翻余謂古歐冶子善作劒故謂劒為歐刀當音烏侯翻}
又下遼東都尉龎奮使速行刑|{
	下遐稼翻}
奮即斬光收煥 |{
	考異曰帝紀建光元年四月甲戊龎奮承偽璽書殺姚光馮緄傳亦云建光元年按帝紀去年十二月高驪圍玄菟而高驪傳有姚光上言蓋光實以延光元年被殺紀傳誤以延為建又今年四月無甲戌}
煥欲自殺其子緄疑詔文有異|{
	緄古本翻}
止煥曰大人在州志欲去惡實無他故必是凶人妄詐規肆姦毒願以事自上|{
	去羌呂翻上時掌翻}
甘罪無晚煥從其言上書自訟果詐者所為徵奮抵罪 癸巳司空陳褒免五月庚戌宗正彭城劉授為司空 己巳封河間孝王子德為安平王嗣樂成靖王後|{
	自是樂成國改曰安平去年樂成王萇以罪廢今以德紹靖王後諡法柔德安衆曰靖恭已鮮言曰靖寛樂令終曰靖}
六月郡國蝗 秋七月癸卯京師及郡國十三地震 高句驪王遂成還漢生口詣玄菟降|{
	降戶江翻下同}
其後濊貊率服東垂少事|{
	濊音穢少詩沼翻}
䖍人羌與上郡胡反度遼將軍耿夔擊破之 八月楊陵園寢火|{
	景帝陵園寑也}
九月甲戌郡國二十七地震 鮮卑既累殺郡守膽氣轉盛控弦數萬騎冬十月復寇鴈門定襄|{
	復扶又翻}
十一月寇太原 燒當羌麻奴飢困將種衆詣漢陽太守耿种降|{
	種章勇翻种音冲}
是歲京師及郡國二十七雨水 帝數遣黄門常侍及中使伯榮往來甘陵|{
	數所角翻使疏吏翻}
尚書僕射陳忠上疏曰今天心未得隔并屢臻|{
	賢曰隔并謂水旱不節也尚書曰一極備凶一極無凶并音必姓翻}
青冀之域淫雨漏河|{
	賢曰漏溢也余謂雨久不止河隄為之决漏也}
徐岱之濱海水盆溢|{
	禹貢海岱及淮為徐州故曰徐岱盆讀與湓同音蒲悶翻}
兖豫蝗蝝滋生|{
	賢曰蝝螽子也音余專翻余按蝝蝗子也董仲舒云然左傳宣十五年冬蝝生劉歆曰蚍蜉子杜預曰螽子以冬生遇寒而死故不成螽爾雅曰蝝蝮蜪蝗也陸機草木疏云螽幽州人謂之舂箕蝗類也曰螽子者猶蝗子也}
荆楊稻收儉簿并凉二州羌戎叛戾加以百姓不足府帑虛匱|{
	帑它朗翻}
陛下以不得親奉孝德皇園廟比遣中使致敬甘陵|{
	比毗至翻}
朱軒駢馬相望道路|{
	賢曰朱軒車使者所乘也駢並也}
可謂孝至矣然臣竊聞使者所過威權翕赫震動郡縣王侯二千石至為伯榮獨拜車下|{
	為于偽翻下猥為誤為同}
發民修道繕理亭傳|{
	傳株戀翻}
多設儲偫|{
	偫大理翻具也}
徵役無度老弱相隨動有萬計賂遺僕從人數百匹|{
	謂縑帛也遺于季翻從才用翻}
頓踣呼嗟|{
	踣蒲墨翻僵也斃也}
莫不叩心河間託叔父之屬清河有陵廟之尊|{
	賢曰河間王開安帝叔也清河王延平也陵廟所在故曰尊}
及剖符大臣皆猥為伯榮屈節車下陛下不問必以為陛下欲其然也伯榮之威重於陛下陛下之柄在於臣妾水災之發必起於此昔韓嫣託副車之乘受馳視之使江都誤為一拜而嫣受歐刀之誅|{
	韓嫣有寵於武帝常與帝共卧起江都王入朝從上獵上林中天子車駕䟆通未行先使嫣乘副車從數十百騎馳視獸江都王望見以為天子辟從者伏謁道旁嫣驅不見既過江都王怒為太后泣請得歸國入宿衛比韓嫣太后由此銜嫣遂誅嫣嫣音偃}
臣願明主嚴天元之尊正乾剛之位|{
	賢曰天元猶乾元也}
不宜復令女使干錯萬機|{
	復扶又翻使如字}
重察左右得無石顯漏泄之姦|{
	石顯事見二十九卷元帝建昭二年重直用翻}
尚書納言得無趙昌譛崇之詐|{
	趙昌事見三十四卷哀帝建平四年}
公卿大臣得無朱博阿傅之援|{
	朱博事見三十四卷建平二年}
外屬近戚得無王鳳害商之謀|{
	王鳳事見三十卷成帝河平四年}
若國政一由帝命王事每决於已則下不得偪上臣不得干君常雨大水必當霽止四方衆異不能為害書奏不省|{
	省悉景翻}
時三府任輕機事專委尚書而災眚變咎輒切免三公|{
	賢曰切責也}
陳忠上疏曰漢興舊事丞相所請靡有不聼今之三公雖當其名而無其實選舉誅賞一由尚書尚書見任重於三公陵遲以來其漸久矣臣忠心常獨不安近以地震策免司空陳褒今者災異復欲切讓三公|{
	復扶又翻}
昔孝成皇帝以妖星守心移咎丞相|{
	事見三十三卷綏和二年}
卒不蒙上天之福|{
	卒子恤翻}
徒乖宋景之誠故知是非之分較然有歸矣|{
	分扶問翻}
又尚書决事多違故典罪法無例詆欺為先文慘言醜有乖章憲宜責求其意割而勿聼上順國典下防威福置方員於規矩審輕重於衡石|{
	此言决事當依典法也賢曰衡秤衡也三十斤為鈞四鈞為石}
誠國家之典萬世之法也 汝南太守山陽王龔政崇寛和好才愛士以袁閬為功曹|{
	好呼到翻閬音浪}
引進郡人黄憲陳蕃等憲雖不屈蕃遂就吏|{
	就辟而為吏也}
閬不修異操而致名當時蕃性氣高明龔皆禮之由是羣士莫不歸心憲世貧賤父為牛醫穎川荀淑至慎陽|{
	慎陽縣屬汝南郡憲縣人也賢曰在慎水之南因以名縣應劭曰慎水出東北入淮師古曰慎字本作滇音真後誤為慎耳今猶有真邱真陽縣知音不改也闞駰曰永平五年失印更刻遂誤以水為心余按水北為陽賢既云縣在水南而名慎陽何也}
遇憲於逆旅|{
	逆迎也設館舍以迎客故曰逆旅賢曰逆旅客舍}
時年十四淑竦然異之揖與語移日不能去|{
	移日言日移晷也}
謂憲曰子吾之師表也既而前至袁閬所未及勞問|{
	勞力到翻}
逆曰子國有顔子寧識之乎|{
	賢曰顔子顔回也閬汝南汝陽人}
閬曰見吾叔度耶|{
	黄憲字叔度}
是時同郡戴良才高倨傲而見憲未嘗不正容及歸罔然若有失也其母問曰汝復從牛醫兒來耶|{
	復扶又翻下同}
對曰良不見叔度自以為無不及既覩其人則瞻之在前忽然在後|{
	論語顔回慕孔子之言}
固難得而測矣陳蕃及同郡周舉嘗相謂曰時月之間不見黄生則鄙吝之萌復存乎心矣|{
	自朔至晦為一月三月為一時賢曰吝貪也余謂作事可卑賤者謂之鄙作事可羞恨者謂之吝}
太原郭泰少遊汝南|{
	少詩照翻}
先過袁閬不宿而退進往從憲累日方還或以問泰曰奉高之器譬諸氿濫雖清而易挹|{
	賢曰奉高閬字也爾雅側出氿泉正出濫泉氿音軌濫音檻易以䜴翻}
叔度汪汪若千頃陂澄之不清淆之不濁不可量也|{
	賢曰淆混也量音良}
憲初舉孝亷又辟公府友人勸其仕憲亦不拒之暫到京師即還竟無所就年四十八終

范曄論曰黄憲言論風旨無所傳聞然士君子見之者靡不服深遠去玼吝|{
	遠于願翻賢曰玼音此說文曰鮮色也據此文當為疵作玼者古字通也}
將以道周性全無德而稱乎|{
	賢曰道周備性全一無德而稱言其德大無能名焉}
余曾祖穆侯|{
	賢曰晉書曰范汪字玄平安北將軍汪生甯甯生泰泰生曄}
以為憲隤然其處順|{
	賢曰易繫辭曰夫坤隤然示人簡矣隤柔順貌音大回翻處昌呂翻}
淵乎其似道|{
	賢曰老子曰道冲而用之或不盈淵乎似萬物之宗言深而不可測也}
淺深莫臻其分|{
	分扶問翻}
清濁未議其方|{
	賢曰方所也}
若及門於孔氏其殆庶乎|{
	賢曰易繫辭曰顔氏之子其殆庶幾乎殆近也}


二年春正月旄牛夷反|{
	前漢旄牛縣屬蜀郡後漢屬華陽國志旄牛縣在卬萊山表}
益州刺史張喬擊破之 夏四月戊子爵乳母王聖為野王君 北匈奴連與車師入寇河西議者欲復閉玉門陽關以絶其患|{
	賢曰玉門陽關二關名也在敦煌西界皆在敦煌龍靭界復扶又翻下同}
敦煌太守張璫上書曰|{
	敦徒門翻}
臣在京師亦以為西域宜棄今親踐其土地乃知棄西域則河西不能自存謹陳西域三策北虜呼衍王常展轉蒲類秦海之間|{
	賢曰大秦國在西海西故曰秦海余按蒲類海在唐庭州界盖此時北匈奴雖微弱然東畏鮮卑不敢還故地但結連車師鄯善以擾河西故呼衍一部常為河西患若賢注以大秦海西之國為秦海則約言之耳西海廣遠甘英之不能越北匈奴兵威所未嘗役屬言展轉二海閒特當時上書者張言之耳}
專制西域共為寇鈔今以酒泉屬國吏士二千餘人集昆侖塞|{
	賢曰前書敦煌郡廣至縣有昆侖障宜禾都尉居也廣至故城在今瓜州常樂縣東注又見四十五卷明帝永平十七年與此稍異}
先擊呼衍王絶其根本因發鄯善兵五千人脅車師後部此上計也若不能出兵可置軍司馬將士五百人四郡供其犂牛穀食出據柳中此中計也|{
	四郡武威酒泉張掖敦煌賢曰柳中今西州縣余按西域傳柳中在後部金蒲城之北去交河城八十里杜佑曰唐平高昌以田地城為柳中縣鄯上扇翻}
如又不能則宜棄交河城收鄯善等悉使入塞此下計也朝廷下其議|{
	鄯上扇翻下遐稼翻}
陳忠上疏曰西域内附日久區區東望扣關者數矣此其不樂匈奴慕漢之效也|{
	數所角翻樂音洛}
今北虜已破車師埶必南攻鄯善弃而不救則諸國從矣|{
	言從北匈奴也}
若然則虜財賄益增膽埶益殖|{
	賢曰殖生也}
威臨南羌|{
	即湟中及南山諸羌}
與之交通如此河西四郡危矣河西既危不可不救則百倍之役興不訾之費發矣|{
	毛晃曰訾之為言量也不訾謂無量可比也訾子斯翻}
議者但念西域絶遠卹之煩費不見孝武苦心勤勞之意也方今敦煌孤危遠來告急復不輔助内無以慰勞吏民|{
	勞力到翻}
外無以威示百蠻蹙國減土非良計也臣以為敦煌宜置校尉按舊增四郡屯兵以西撫諸國帝納之如是復以班勇為西域長史|{
	賢曰西域都護之長史也余按班超未為都護亦為將兵長史敦徒門翻復扶又翻下同}
將兵五百人出屯柳中 秋七月丹陽山崩|{
	丹陽郡屬揚州}
九月郡國五雨水 冬十月辛未太尉劉愷罷甲戌以司徒楊震為太尉光禄勲東萊劉熹為司徒大鴻臚耿寶自候震|{
	候見也}
薦中常侍李閏兄於震曰李常侍國家所重欲令公辟其兄寶唯傳上意耳|{
	賢曰言非己本心傳在上之意}
震曰如朝廷欲令三府辟召故宜有尚書敇寶大恨而去執金吾閻顯亦薦所親於震震又不從司空劉授聞之即辟此二人由是震益見怨時詔遣使者大為王聖修第中常侍樊豐及侍中周廣謝惲等更相扇動傾揺朝廷|{
	為于偽翻惲於粉翻更工衡翻}
震上疏曰臣伏念方今災害滋甚百姓空虚三邊震擾|{
	三邊東西北也}
帑藏匱乏|{
	帑它朗翻臧狙浪翻}
殆非社稷安寧之時詔書為阿母興起第舍合兩為一連里竟街|{
	賢曰合兩坊而為一宅里即坊也}
雕修繕飾窮極巧伎|{
	伎渠綺翻}
攻山採石轉相迫促為費巨億周廣謝惲兄弟與國無肺府枝葉之屬|{
	府與腑同}
依倚近倖姦佞之人與之分威共權屬託州郡|{
	屬之欲翻}
傾動大臣宰司辟召承望旨意招來海内貪汙之人受其貨賂至有臧錮棄世之徒|{
	賢曰有贓賄禁錮之人余謂棄世者見棄於世也}
復得顯用白黑渾淆清濁同源天下讙譁為朝結譏|{
	渾戶本翻讙許元翻為于偽翻朝直遥翻}
臣聞師言上之所取|{
	師言衆言也}
財盡則怨力盡則叛怨叛之人不可復使惟陛下度之|{
	復扶又翻度徒洛翻}
上不聼 鮮卑其至鞬自將萬餘騎攻南匈奴於曼栢薁鞬日逐王戰死殺千餘人|{
	鞬居言翻薁於六翻}
十二月戊辰京師及郡國三地震 陳忠薦汝南周燮南陽馮良學行深純|{
	行下孟翻}
隱居不仕名重於世帝以玄纁羔幣聘之|{
	玄黑色纁淺絳色周官考工記曰三入為纁爾雅三染謂之纁孔穎逹曰束帛十端也端則二丈十端六玄四纁五兩三玄二纁纁是地色玄是天色賢曰禮卿執羔董仲舒春秋繁露曰凡贄卿用羔羔有角而不用執之不鳴殺之不嘷類死義者羔飲其母必跪類知禮者故以為贄纁許云翻}
燮宗族更勸之曰夫修德立行所以為國|{
	更工衡翻行下孟翻為于偽翻}
君獨何為守東岡之陂乎|{
	燮居汝南安城有先人草廬結于岡畔下有陂田常肆勤以自給}
燮曰夫修道者度其時而動動而不時焉得亨乎|{
	賢曰亨通也書曰慮善而動動惟厥時度徒洛翻焉於䖍翻}
與良皆自載至近縣稱病而還

三年春正月班勇至樓蘭以鄯善歸附特加三綬|{
	三綬疑當作王綬綬音受}
而龜兹王白英猶自疑未下|{
	龜兹音邱慈}
勇開以恩信白英乃率姑墨温宿自縛詣勇因發其兵步騎萬餘人到車師前王庭擊走匈奴伊蠡王於伊和谷|{
	蠡音黎}
收得前部五千餘人於是前部始復開通|{
	復扶又翻}
還屯田柳中 二月丙子車駕東廵辛卯幸泰山三月戊戌幸魯還幸東平至東郡歷魏郡河内而還|{
	還從宣翻又如字}
初樊豐周廣謝惲等見楊震連諫不從無所顧忌遂詐作詔書調發司農錢穀大匠見徒材木|{
	調徒弔翻見賢遍翻}
各起冢舍園池廬觀役費無數|{
	觀古玩翻}
震復上疏曰臣備台輔不能調和隂陽去年十二月四日京師地動其日戊辰三者皆土位在中官|{
	賢曰戊干辰支皆土也并地動故言三者 考異曰震傳作十一月四日按下文其日戊辰十一月丙申朔戊辰乃十二月四日也}
此中臣近官持權用事之象也臣伏惟陛下以邊境未寧躬自菲薄宫殿垣屋傾倚枝拄而已|{
	賢曰倚邪也拄音竹柱翻}
而親近倖臣未崇斷金|{
	賢曰易繫辭曰二人同心其利斷金言邪佞之人不與上同心近其靳翻斷丁亂翻王肅丁管翻}
驕溢踰法多請徒士盛修第舍賣弄威福道路讙譁地動之變殆為此發|{
	讙許元翻為于偽翻下同}
又冬無宿雪春節未雨百僚焦心而繕修不止誠致旱之徵也惟陛下奮乾剛之德棄驕奢之臣以承皇天之戒震前後所言轉切帝既不平之而樊豐等皆側目憤怨以其名儒未敢加害會河間男子趙騰上書指陳得失帝發怒遂收考詔獄結以罔上不道|{
	結者結定其罪}
震上疏救之曰臣聞殷周哲王小人怨詈則還自敬德|{
	賢曰尚書無逸之辭還反也敬德加謹以增修其德也}
今趙騰所坐激訐謗語為罪與手刃犯法有差乞為虧除全騰之命|{
	虧減也訐居楬翻}
以誘芻蕘輿人之言|{
	賢曰輿衆也詩曰詢于芻蕘左氏傳曰聼輿人之謀也誘音酉}
帝不聼騰竟伏尸都市及帝東廵樊豐等因乘輿在外競修第宅|{
	乘繩證翻}
太尉部掾高舒召大匠令史考校之|{
	漢公府諸曹掾各有分部賢曰史謂府史也余按漢諸官府各有令史}
得豐等所詐下詔書具奏須行還上之|{
	須待也待車駕行還上言其事下遐稼翻上時掌翻}
豐等惶怖|{
	怖普布翻}
會太史言星變逆行遂共譛震云自趙騰死後深用怨懟|{
	賢曰懟怨怒也音直類翻}
且鄧氏故吏有恚恨之心|{
	賢曰震初鄧隲辟之故曰故吏恚於避翻下同}
壬戌車駕還京師便時太學|{
	賢曰且於太學待吉時而後入也故曰便時前書便時上林延夀門杜佑曰便時取日時之便}
夜遣使者策收震太尉印綬震於是柴門絶賓客|{
	柴塞其門也}
豐等復惡之|{
	復扶又翻惡烏路翻下同}
令大鴻臚耿寶奏震大臣不服罪懷恚望有詔遣歸本郡|{
	震弘農華隂人}
震行至城西几陽亭|{
	雒陽城西也}
乃慷慨謂其諸子門人曰|{
	賢曰慷慨悲歎余謂慷慨不得意而見於辭色也}
死者士之常分|{
	分扶問翻}
吾蒙恩居上司疾姦臣狡猾而不能誅惡嬖女傾亂而不能禁|{
	嬖卑義翻又必計翻}
何面目復見日月|{
	復扶又翻下同}
身死之日以雜木為捾布單被裁足蓋形勿歸冢次勿設祭祀因飲酖而卒弘農太守移良|{
	風俗通曰齊公子雍食邑於移其後氏焉}
承樊豐等旨遣吏於陜縣留停震喪露棺道側讁震諸子代郵行書|{
	說文曰郵境上行書舍也廣雅曰郵驛也此言使震諸子代驛吏傳行文書也陜失冉翻}
道路皆為隕涕|{
	為于偽翻}
太僕征羌侯來歷曰|{
	征羌侯國屬汝南郡光武以歷曾祖歙有平羌隴之功改汝南當鄉縣為征羌國以封之賢曰征羌故城在今豫州郾城縣東南}
耿寶託元舅之親榮寵過厚不念報國恩而傾側姦臣傷害忠良其天禍亦將至矣歷歙之曾孫也|{
	歙許及翻}
夏四月乙丑車駕入宫 戊辰以光禄勲馮石為太尉 南單于檀死弟拔立為烏稽侯尸逐鞮單于|{
	鞮于奚翻}
時鮮卑數寇邊|{
	數所角翻下同}
度遼將軍耿夔與温禹犢王呼尤徽將新降者連年出塞擊之還使屯列衝要|{
	衝要者當敵之衝邊之要地也賢曰還令新降者屯列衝要降戶江翻}
耿夔徵發煩劇新降者皆怨恨大人阿族等遂反|{
	阿族者新降一部之大人也}
脅呼尤徽欲與俱去呼尤徽曰我老矣受漢家恩寧死不能相隨衆欲殺之有救者得免阿族等遂將其衆亡去中郎將馬翼與胡騎追擊破之斬獲殆盡|{
	賢曰殆近也欲死盡所餘無幾}
日南徼外蠻夷内屬|{
	徼吉弔翻}
六月鮮卑寇玄菟 庚午閬中山崩|{
	閬中縣屬巴郡賢曰臨閬中水因以為名今隆州縣宋白曰閬水紆曲經其三面縣居其中取以名之}
秋七月辛巳以大鴻臚耿寶為大將軍 王聖江京樊豐等譛太子乳母王男厨監邴吉等殺之|{
	厨監主飲食}
家屬徙比景太子思男吉數為歎息|{
	為于偽翻}
京豐懼有後害乃與閻后妄造虚無搆讒太子及東宫官屬帝怒召公卿以下議廢太子耿寶等承旨皆以為當廢太僕來歷與太常桓焉廷尉犍為張皓議曰|{
	犍居言翻}
經說年未滿十五過惡不在其身且男吉之謀太子容有不知宜選忠良保傅輔以禮義廢置事重此誠聖恩所宜宿留|{
	賢曰宿留猶停留也音秀溜}
帝不從焉郁之子也|{
	郁桓榮之子}
張皓退復上書曰昔賊臣江充造構讒逆傾覆戾園孝武久乃覺寤雖追前失悔之何及|{
	事見二十三卷武帝征和二年三年}
今皇太子方十歲未習保傅之教可遽責乎書奏不省|{
	省悉景翻}
九月丁酉廢皇太子保為濟隂王居於德陽殿西鍾下|{
	漢官儀曰崇玄門德陽殿也按帝紀德陽殿在北宫掖庭中蔡質漢儀曰正月旦天子幸德陽殿臨軒公卿將大夫百官各陪朝賀蠻貊胡羌朝貢畢見屬郡計吏皆覲宗室諸劉雜會又曰德陽殿周旋容萬人陛高二丈皆文石作壇激沼水于殿下天子正旦節會朝百僚於此殿濟子禮翻}
來歷乃要結光禄勲祋諷宗正劉瑋將作大匠薛皓侍中閭邱宏|{
	閭邱複姓左傳齊有閭邱嬰}
陳光趙代施延太中大夫九江朱倀等十餘人|{
	要一遥翻祋丁外翻又丁活翻倀丑羊翻}
俱詣鴻都門證太子無過帝與左右患之|{
	左石近習也}
乃使中常侍奉詔脅羣臣曰父子一體天性自然|{
	孟子曰父子之道天性也}
以義割恩為天下也|{
	為于偽翻}
歷諷等不識大典而與羣小共為讙譁外見忠直|{
	見賢遍翻}
而内希後福飾邪違義豈事君之禮朝廷廣開言路故且一切假貸若懷迷不反當顯明刑書諫者莫不失色薛皓先頓首曰固宜如明詔歷怫然|{
	賢曰字林怫欝也音扶勿翻余謂怫然憤欝之見於色者}
廷詰皓曰屬通諫何言而今復背之|{
	賢曰屬近也通猶共也近者共諫何乃相背也屬之欲翻復扶又翻背蒲妹翻}
大臣乘朝車處國事固得輾轉若此乎|{
	賢曰輾轉不定也詩曰輾轉反側處昌呂翻輾音展}
乃各稍自引起歷獨守闕連日不肯去帝大怒尚書令陳忠與諸尚書遂共劾奏歷等帝乃免歷兄弟官削國租|{
	削其征羌國租也劾戶槩翻又戶得翻}
黜歷母武安公主不得會見|{
	武安公主顯宗女也武安縣屬魏郡見賢遍翻}
隴西郡始還狄道|{
	永初五年隴西徙襄武}
燒當羌豪麻奴死弟犀苦立 庚申晦日有食之 冬十月上行幸長安十一月乙丑還雒陽 是歲京師及諸郡國二十三地震三十六大水雨雹|{
	雨于具翻}


資治通鑑卷五十
















































































































































