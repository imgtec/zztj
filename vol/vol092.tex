






























































資治通鑑卷九十二   宋 司馬光 撰

胡三省 音註

晉紀十四|{
	起玄黓敦牂盡昭陽協洽凡二年}


中宗元皇帝下

永昌元年春正月郭璞復上疏請因皇孫生下赦令|{
	璞去年己疏請肆赦皇孫去年十一月生復扶又翻}
帝從之乙卯大赦改元王敦以璞為記室參軍璞善卜筮知敦必為亂已預其禍甚憂之大將軍掾潁川陳述卒|{
	掾于絹翻}
璞哭之極哀曰嗣祖焉知非福也|{
	陳述字嗣祖亦敦府僚也焉於䖍翻}
敦既與朝廷乖離乃羇録朝士有時望者置已幕府|{
	朝直遥翻}
以羊曼及陳國謝鯤為長史曼祜之兄孫也曼鯤終日酣醉故敦不委以事|{
	敦收時望不過用西都諸王之故智耳酣戶甘翻}
敦將作亂謂鯤曰劉隗姦邪將危社稷吾欲除君側之惡何如鯤曰隗誠始禍然城狐社鼠|{
	後漢虞延曰城狐社鼠不畏熏燒謂有所憑托也又中山王勝曰社鼷不灌屋鼠不熏所託者然也爾雅翼曰管仲稱社束木而塗之鼠因往託焉燻之則恐燒其木灌之則恐敗其塗此鼠之所以不可得而殺者以社故也以喻君之左右}
敦怒曰君庸才豈達大體出為豫章太守|{
	守式又翻}
又留不遣戊辰敦舉兵於武昌上疏罪狀劉隗稱隗佞邪讒賊威福自由|{
	隗五罪翻}
妄興事役勞擾士民賦役煩重怨聲盈路臣備位宰輔不可坐視成敗輒進軍致討隗首朝懸諸軍夕退昔太甲顛覆厥度幸納伊尹之忠殷道復昌|{
	湯崩太甲顛覆湯之典刑伊尹放之於桐三年太甲悔過自怨自艾於桐伊尹以冕服奉太甲復歸于亳賴伊尹之訓以圖厥終古固有是事然非人臣所當為也}
願陛下深垂三思|{
	三息暫翻又如字}
則四海乂安社稷永固矣沈充亦起兵於吳興以應敦敦以充為大都督督護東吳諸軍事敦至蕪湖又上表罪狀刁協帝大怒乙亥詔曰王敦憑恃寵靈敢肆狂逆方朕太甲欲見幽囚是可忍也孰不可忍今親帥六軍以誅大逆|{
	帥讀曰率}
有殺敦者封五千戶侯敦兄光禄勲含乘輕舟逃歸于敦太子中庶子温嶠謂僕射周顗曰大將軍此舉似有所在當無濫邪顗曰不然|{
	顗魚豈翻}
人主自非堯舜何能無失人臣安可舉兵以脅之舉動如此豈得云非亂乎處仲狼抗無上其意寜有限耶|{
	王敦字處仲狼似犬銳頭白頰高前廣後貪而敢抗人故以為喻處昌呂翻}
敦初起兵遣使告梁州刺史甘卓約與之俱下卓許之及敦升舟而卓不赴使參軍孫雙詣武昌諫止敦敦驚曰甘侯前與吾語云何而更有異正當慮吾危朝廷耳吾今但除姦凶若事濟當以甘侯作公|{
	許卓作公啗之以利欲使同逆}
雙還報卓意狐疑或說卓且偽許敦殆敦至都而討之|{
	說輸芮翻}
卓曰昔陳敏之亂吾先從而後圖之|{
	事見八十六卷惠帝永興二年懷帝永嘉元年}
論者謂吾懼逼而思變心常愧之今若復爾何以自明|{
	復扶又翻下同}
卓使人以敦旨告順陽太守魏該|{
	守式又翻}
該曰我所以起兵拒胡賊者正欲忠於王室耳今王公舉兵向天子非吾所宜與也遂絶之|{
	史言甘卓不如魏該之忠果}
敦遣參軍桓熊說譙王氶請氶為軍司|{
	說輸芮翻氶音拯}
氶歎曰吾其死矣地荒民寡勢孤援絶將何以濟然得死忠義夫復何求|{
	夫音扶復扶又翻}
氶檄長沙虞悝為長史會悝遭母喪|{
	悝苦囘翻}
氶往弔之曰吾欲討王敦而兵少糧乏|{
	少詩沼翻}
且新到恩信未洽卿兄弟湘中之豪俊王室方危金革之事古人所不辭|{
	禮記子夏問曰三年之喪卒哭金革之事無避也者禮歟初有司歟孔子曰吾聞諸老聃昔者魯公伯禽有為為之也今以三年之喪從其利者吾弗知也春秋公羊傳曰古者臣有大喪則君三年不呼其門已練可以弁冕服金革之事君使之非也臣行之禮也閔子要絰而服事孔子蓋善之也}
將何以教之悝曰大王不以悝兄弟猥劣親屈臨之敢不致死然鄙州荒弊難以進討宜且收衆固守傳檄四方敦勢必分分而圖之庶幾可捷也|{
	幾居希翻}
氶乃囚桓羆以悝為長史以其弟望為司馬督護諸軍與零陵太守尹奉建昌太守長沙王循衡陽太守淮陵劉翼|{
	沈約曰晉惠帝元康九年分長沙東北下雋諸縣立建昌郡至宋為巴陵郡吳孫亮太平二年分長沙西部都尉立衡陽郡淮陵縣屬臨淮郡時亦分為郡}
舂陵令長沙易雄|{
	舂陵縣本前漢之舂陵侯國後徙國南陽省吳復立舂陵縣屬零陵郡姓譜易姓齊有大夫易牙}
同舉兵討敦雄移檄遠近列敦罪惡於是一州之内皆應氶惟湘東太守鄭澹不從|{
	吳孫亮太平二年分長沙東部都尉立湘東郡澹徒覽翻}
氶使虞望討斬之以徇四境澹敦姊夫也氶遣主簿鄧騫至襄陽|{
	晉梁州刺史鎮襄陽自周訪始宋白曰襄陽漢中廬縣地}
說甘卓曰劉大連雖驕蹇失衆心|{
	劉隗字大連說輸芮翻下同}
非有害於天下大將軍以其私憾稱兵向闕此忠臣義士竭節之時也公受任方伯奉辭伐罪乃桓文之功也卓曰桓文則非吾所能然志在徇國當共詳思之參軍李梁說卓曰昔隗囂跋扈竇融保河西以奉光武卒受其福|{
	事見四十一卷漢光武建武五年至四十三卷十二年卒子恤翻}
今將軍有重望於天下但當案兵坐以待之使大將軍事捷當委將軍以方面不捷朝廷必以將軍代之何憂不富貴而釋此廟勝|{
	孫子曰未戰而廟勝得筭多也未戰而廟不勝得筭少也}
决存亡於一戰耶騫謂梁曰光武當創業之初故隗竇可以文服從容顧望|{
	文服謂非心服特以虛文示相臣服而已從千容翻}
今將軍之於本朝非竇融之比也|{
	朝直遥翻}
襄陽之於太府|{
	襄陽以王敦府為太府}
非河西之固也使大將軍克劉隗還武昌增石城之戍|{
	賢曰石城故城在復州沔陽縣東南}
絶荆湘之粟將軍將安歸乎勢在人手而曰我處廟勝未之聞也且為人臣國家有難|{
	處昌呂翻難乃旦翻}
坐視不救於義安乎卓尚疑之騫曰今既不為義舉又不承大將軍檄此必至之禍愚智所見也且議者之所難以彼彊而我弱也今大將軍兵不過萬餘其留者不能五千而將軍見衆既倍之矣|{
	見賢遍翻}
以將軍之威名帥此府之精銳杖節鳴鼓以順討逆豈王含所能禦哉|{
	帥讀曰率}
遡流之衆勢不自救|{
	謂敦兵以東下若欲遡流西上以自救勢不相及也}
將軍之舉武昌若摧枯拉朽尚何顧慮耶|{
	拉盧合翻}
武昌既定據其軍實鎮撫二州|{
	二州謂荆江也}
以恩意招懷士卒使還者如歸此呂蒙所以克關羽也|{
	事見六十八卷漢獻帝建安二十四年}
今釋必勝之策安坐以待危亡不可以言智矣敦恐卓於後為變又遣參軍丹楊樂道融往邀之必欲與之俱東道融雖事敦而忿其悖逆|{
	悖蒲内翻又蒲没翻}
乃說卓曰主上親臨萬機自用譙王為湘州非專任劉隗也而王氏擅權日久卒見分政|{
	卒讀曰猝謂分任譙王氶等政不專歸于王氏也}
便謂失職背恩肆逆|{
	背蒲妹翻}
舉兵向闕國家遇君至厚今與之同豈不違負大義生為逆臣死為愚鬼永為宗黨之恥不亦惜乎為君之計莫若偽許應命而馳襲武昌大將軍士衆聞之必不戰自潰大勲可就矣卓雅不欲從敦聞道融之言遂决曰吾本意也乃與巴東監軍柳純南平太守夏侯承|{
	監工䘖翻夏戶雅翻}
宜都大守譚該等|{
	姓譜齊滅譚子孫以國為氏漢有河南尹譚閎又巴南大姓有譚氏盤瓠之後}
露檄數敦逆狀帥所統致討|{
	數所具翻}
遣參軍司馬讚孫雙奉表詣臺羅英至廣州約陶侃同進戴淵在江西|{
	戴淵出鎮合肥於建康為江西}
先得卓書表上之|{
	上時掌翻}
臺内皆稱萬歲陶侃得卓信即遣參軍高寶帥兵北下|{
	帥讀曰率}
武昌城中傳卓軍至人皆奔散敦遣從母弟南蠻校尉魏又|{
	敦從母魏氏又其弟也從才用翻}
將軍李恒|{
	恒戶登翻}
帥甲卒二萬攻長沙長沙城池不完資儲又闕人情震恐或說譙王氶南投陶侃或退據零桂氶曰吾之起兵志欲死於忠義豈可貪生苟免為奔敗之將乎|{
	將即亮翻}
事之不濟令百姓知吾心耳乃嬰城固守未幾|{
	幾魚豈翻}
虞望戰死甘卓欲留鄧騫為參軍騫不可乃遣參軍虞冲與騫偕至長沙遺譙王氶書勸之固守當以兵出沔口斷敦歸路|{
	遺于季翻斷丁管翻}
則湘圍自解氶復書稱江左中興草創始爾豈圖惡逆萌自寵臣吾以宗室受任志在隕命而至止尚淺凡百茫然足下能卷甲電赴猶有所及|{
	卷讀曰捲}
若其狐疑則求我於枯魚之肆矣|{
	莊子見車轍鮒鮒曰豈無斗升之水以活我乎莊子曰待我决西江之水而迎汝鮒曰如君言不如早索我於枯魚之肆}
卓不能從 二月甲午封皇子昱為琅邪王 後趙王勒立子弘為世子遣中山公虎將精卒四萬擊徐龕|{
	將即亮翻龕苦含翻}
龕堅守不戰虎築長圍守之趙王曜自將擊楊難敵難敵逆戰不勝退保仇池仇

池諸氐羌及故晉王保將楊韜隴西大守梁勛皆降於曜|{
	降戶江翻}
曜遷隴西萬餘戶於長安進攻仇池會軍中大疫曜亦得疾將引兵還恐難敵躡其後乃遣光國中郎將王獷說難敵|{
	光國中郎將趙所置也獷古猛翻說輸芮翻}
諭以禍福難敵遣使稱藩曜以難敵為假黄鉞都督益寜南秦凉梁巴六州隴上西域諸軍事上大將軍益寜南秦三州牧武都王|{
	吳孫氏始置上大將軍南秦州及巴州曜創其名其後北國率授楊氏南秦州刺史據有隂平武都二郡之地}
秦州刺史陳安求朝於曜曜辭以疾安怒以為曜已卒|{
	朝直遥翻卒子恤翻}
大掠而歸曜疾甚乘馬輿而還使其將呼延寔監輜重於後|{
	監工銜翻重直用翻}
安邀擊獲之謂寔曰劉曜已死子尚誰佐吾當與子共定大業寔叱之曰汝受人寵禄而叛之自視智能何如主上吾見汝不日梟首於上邽市|{
	梟堅堯翻}
何謂大業宜速殺我安怒殺之以寔長史魯憑為參軍安遣其弟集帥騎三萬追曜|{
	帥讀曰率騎奇寄翻}
衛將軍呼延瑜逆擊斬之安乃還上邽遣將襲汧城拔之|{
	汧縣屬扶風郡汧苦堅翻}
隴上氐羌皆附於安有衆十餘萬自稱大都督假黄鉞大將軍雍凉秦梁四州牧凉王|{
	雍於用翻}
以趙募為相國魯憑對安大哭曰吾不忍見陳安之死也安怒命斬之憑曰死自吾分|{
	分扶問翻}
懸吾頭於上邽市觀趙之斬陳安也遂殺之曜聞之慟哭曰賢人民之望也陳安於求賢之秋而多殺賢者吾知其無能為也休屠王石武以桑城降趙|{
	石武蓋亦匈奴種屠直於翻}
趙以武為秦州刺史封酒泉王 帝徵戴淵劉隗入衛建康隗至百官迎於道隗岸幘大言|{
	岸幘者幘微脫額也}
意氣自若及入見|{
	見賢遍翻}
與刁協勸帝盡誅王氏帝不許隗始有懼色司空導帥其從弟中領軍邃左衛將軍廙侍中侃彬及諸宗族二十餘人每旦詣臺待罪|{
	帥讀曰率從才用翻廙羊至翻又逸職翻}
周顗將入導呼之曰伯仁以百口累卿|{
	累力瑞翻周顗字伯仁欲使顗保護導以全其家也}
顗直入不顧既見帝言導忠誠申救甚至帝納其言顗喜飲酒|{
	喜許記翻}
至醉而出導猶在門又呼之顗不與言顧左右曰今年殺諸賊奴取金印如斗大繫肘後既出又上表明導無罪言甚切至導不之知甚恨之帝命還導朝服召見之導稽首曰|{
	朝直遥翻稽音啓}
逆臣賊子何代無之不意今者近出臣族帝跣而執其手曰茂弘方寄卿以百里之命|{
	王導字茂弘孔氏曰寄百里之命謂攝君之政令}
是何言邪三月以導為前鋒大都督加戴淵驃騎將軍|{
	驃匹妙翻}
詔曰導以大義滅親|{
	衛石碏之子厚與公子州吁弑衛桓公又與州吁如陳碏使告於陳而殺之君子曰石碏純臣也惡州吁而厚與焉大義滅親其是之謂乎}
可以吾為安東時節假之|{
	帝之初鎮揚州也領安東將軍}
以周顗為尚書左僕射王邃為右僕射帝遣王廙往諭止敦敦不從而留之廙更為敦用征虜將軍周札素矜險好利|{
	好呼到翻}
帝以為右將軍都督石頭諸軍事敦將至帝使劉隗軍金城|{
	金城在丹楊江乘蒲州上}
札守石頭帝親被甲徇師於郊外|{
	被皮義翻}
以甘卓為鎮南大將軍侍中都督荆梁二州諸軍事陶侃領江州刺史使各帥所統以躡敦後|{
	帥讀曰率下同}
敦至石頭欲攻劉隗杜弘言於敦曰劉隗死士衆多未易可克|{
	易以豉翻}
不如攻石頭周札少恩|{
	少詩沼翻}
兵不為用攻之必敗札敗則隗自走矣敦從之以弘為前鋒攻石頭札果開門納弘敦據石頭歎曰吾不復得為盛德事矣|{
	敦無君之心形於言也復扶又翻}
謝鯤曰何為其然也但使自今以往日忘日去耳|{
	言日復一日浸忘前事則君臣猜嫌之迹亦日去耳}
帝命刁協劉隗戴淵帥衆攻石頭王導周顗郭逸虞潭等三道出戰協等兵皆大敗太子紹聞之欲自帥將士决戰升車將出中庶子温嶠執鞚諫曰|{
	鞚苦貢翻}
殿下國之儲副柰何以身輕天下抽劒斬靷乃止敦擁兵不朝|{
	朝直遥翻下同}
放士卒劫掠宫省奔散惟安東將軍劉超案兵直衛及侍中二人侍帝側帝脫戎衣著朝服|{
	著陟畧翻}
顧而言曰欲得我處當早言何至害民如此又遣使謂敦曰|{
	使疏吏翻}
公若不忘本朝於此息兵則天下尚可共安如其不然朕當歸琅邪以避賢路刁協劉隗既敗俱入宫見帝於太極東除|{
	除殿陛也}
帝執協隗手流涕嗚咽勸令避禍協曰臣當死守不敢有貳帝曰今事逼矣安可不行乃令給協隗人馬使自為計協老不堪騎乘素無恩紀募從者皆委之行至江乘為人所殺送首於敦隗奔後趙官至太子太傅而卒|{
	成帝咸和八年劉隗從石虎戰死於潼關豈即此劉隗耶}
帝令公卿百官詣石頭見敦敦謂戴淵曰前日之戰有餘力乎淵曰豈敢有餘但力不足耳敦曰吾今此舉天下以為何如淵曰見形者謂之逆體誠者謂之忠敦笑曰卿可謂能言又謂周顗曰伯仁卿負我|{
	愍帝建興元年顗為杜弢所困投敦於豫章故敦以為德}
顗曰公戎車犯順下官親率六軍|{
	帥讀曰率}
不能其事使王旅奔敗以此負公辛未大赦以敦為丞相都督中外諸軍録尚書事江州牧封武昌郡公並讓不受初西都覆没四方皆勸進於帝|{
	見九十卷建武元年}
敦欲專國政忌帝年長難制|{
	長知兩翻}
欲更議所立王導不從及敦克建康謂導曰不用吾言幾至覆族|{
	幾居希翻}
敦以太子有勇略為朝野所嚮|{
	朝直遥翻}
欲誣以不孝而廢之大會百官問温嶠曰皇太子以何德稱聲色俱厲嶠曰鉤深致遠蓋非淺局所量|{
	量音良}
以禮觀之可謂孝矣|{
	言太子既有鉤深致遠之才而又盡事親之禮所以解敦不孝之誣也}
衆皆以為信然敦謀遂沮|{
	沮在呂翻}
帝召周顗於廣室|{
	廣室殿名}
謂之曰近日大事二宫無恙諸人平安大將軍固副所望邪|{
	恙余亮翻}
顗曰二宫自如明詔臣等尚未可知護軍長史郝嘏等勸顗避敦|{
	顗代戴淵為護軍將軍以郝嘏為長史}
顗曰吾備位大臣朝廷喪敗寜可復草間求活外投胡越邪|{
	喪息浪翻復扶又翻}
敦參軍呂猗嘗為臺郎|{
	晉謂尚書郎為臺郎}
性姦謟戴淵為尚書惡之|{
	惡烏路翻}
猗說敦曰周顗戴淵皆有高名足以惑衆近者之言曾無怍色|{
	謂二人答敦之言怍才各翻}
公不除之恐必有再舉之憂敦素忌二人之才心頗然之從容問王導曰周戴南北之望|{
	周顗汝南人戴淵廣陵人晉氏南渡二人名冠當時從千容翻}
當登三司無疑也導不答又曰若不三司止應令僕邪|{
	三司太尉司徒司空也令僕尚書令及左右僕射也}
又不答敦曰若不爾正當誅爾又不答丙子敦遣部將陳郡鄧岳收顗及淵|{
	將即亮翻}
先是敦謂謝鯤曰|{
	先悉薦翻}
吾當以周伯仁為尚書令戴若思為僕射|{
	戴淵字若思}
是日又問鯤近來人情何如鯤曰明公之舉雖欲大存社稷然悠悠之言實未逹高義|{
	言衆人議敦舉兵向闕非義舉也}
若果能舉用周戴則羣情帖然矣敦怒曰君麤疎邪二子不相當吾已收之矣鯤愕然自失參軍王嶠曰濟濟多士文王以寜|{
	詩大雅文王之詩濟子禮翻}
柰何戮諸名士敦大怒欲斬嶠衆莫敢言鯤曰明公舉大事不戮一人嶠以獻替忤旨便以釁鼓|{
	君所謂可而有否焉臣獻其否以成其可君所謂否而有可焉臣獻其可以替其否釁鼓殺人以血塗鼓也忤五故翻}
不亦過乎敦乃釋之黜為領軍長史|{
	大將軍府參軍黜為領軍長史足知敦府重於諸府矣}
嶠渾之族孫也顗被收路經太廟大言曰賊臣王敦傾覆社稷枉殺忠臣神祇有靈當速殺之|{
	被皮義翻祇堯移翻}
收人以戟傷其口血流至踵容止自若觀者皆為流涕|{
	為于偽翻下同}
并戴淵殺之於石頭南門之外帝使侍中王彬勞敦|{
	勞力到翻}
彬素與顗善先往哭顗然後見敦敦怪其容慘問之彬曰向哭伯仁情不能已敦怒曰伯仁自致刑戮且凡人遇汝|{
	以王彬之為人顗以凡人遇之亦可以見其風裁矣}
汝何哀而哭之彬曰伯仁長者兄之親友在朝雖無謇愕|{
	愕當作諤朝直遥翻}
亦非阿黨而赦後加之極刑所以傷惋也|{
	惋烏貫翻據元帝紀四月敦入石頭辛未大赦}
因勃然數敦曰|{
	數所具翻}
兄抗旌犯順殺戮忠良圖為不軌禍及門戶矣辭氣慷慨聲淚俱下敦大怒厲聲曰爾狂悖乃至此以吾為不能殺汝耶時王導在坐為之懼|{
	悖蒲内翻又蒲没翻坐徂卧翻為于偽翻}
勸彬起謝彬曰脚痛不能拜且此復何謝|{
	復扶又翻下同}
敦曰脚痛孰若頸痛彬殊無懼容竟不肯拜王導後料檢中書故事|{
	料音聊}
乃見顗救已之表執之流涕曰吾雖不殺伯仁伯仁由我而死|{
	自愧於敦三問不答之時也}
幽冥之中負此良友沈充抜吳國殺内史張茂初王敦聞甘卓起兵大懼卓兄子卬為敦參軍敦使卬歸說卓曰|{
	說輸芮翻下同}
君此自是臣節不相責也吾家計急不得不爾想便旋軍襄陽當更結好|{
	好呼到翻}
卓雖慕忠義性多疑少决|{
	少詩沼翻}
軍于豬口|{
	水經沔水東南逕江夏雲杜縣東夏水從西來注之註云即䐗口也䐗與豬同}
欲待諸方同出軍稽留累旬不前敦既得建康乃遣臺使以騶虞幡駐卓軍|{
	諸方謂待諸方鎮同出軍也騶虞仁獸故以騶虞幡駐軍使疏吏翻}
卓聞周顗戴淵死流涕謂卬曰吾之所憂正為今日|{
	為于偽翻下同}
且使聖上元吉太子無恙|{
	恙余亮翻}
吾臨敦上流亦未敢遽危社稷適吾徑據武昌敦勢逼必劫天子以絶四海之望不如還襄陽更思後圖即命旋軍都尉秦康與樂道融說卓曰今分兵斷彭澤|{
	彭澤縣屬豫章郡彭蠡湖自此入于大江分兵斷彭澤湖口可使敦上下不得相通斷丁管翻}
使敦上下不得相赴其衆自然離散可一戰擒也將軍起義兵而中止竊為將軍不取且將軍之下士卒各求其利欲求西還亦恐不可得也卓不從道融晝夜泣諫卓不聽道融憂憤而卒卓性本寛和忽更彊塞|{
	此彊謂彊暴也塞謂窒塞而不疏通塞悉則翻}
徑還襄陽意氣騷擾舉動失常識者知其將死矣王敦以西陽王羕為大宰|{
	羕余亮翻}
加王導尚書令王廙為荆州刺史改易百官及諸軍鎮轉徙黜免者以百數或朝行暮改惟意所欲敦將還武昌謝鯤言於敦曰公至都以來稱疾不朝是以雖建勲而人心實有未逹今若朝天子|{
	朝直遥翻下同}
使君臣釋然則物情皆悦服矣敦曰君能保無變乎對曰鯤近日入覲主上側席遲得見公|{
	王者側席待賢鯤用此語也遲直二翻待也}
宫省穆然必無虞也|{
	穆然和敬之意}
公若入朝鯤請侍從|{
	從才用翻}
敦勃然曰正復殺君等數百人亦復何損於時|{
	復扶又翻}
竟不朝而去夏四月敦還武昌初宜都内史天門周級|{
	吳孫皓永安六年分武陵立天門郡充縣有松梁山山有石石開處數十丈其高以弩仰射不至其上名天門因此名郡宋白曰澧州石門縣吳立天門郡隋罷郡為石門縣}
聞譙王氶起兵使其兄子該潛詣長沙申欵於氶|{
	申明也欵誠也}
魏乂等攻湘州急氶遣該及從事邵陵周崎間出求救|{
	此非潁川之邵陵吳孫皓寶鼎元年分零陵北部都尉立邵陵郡崎丘寄翻間古莧翻}
皆為邏者所得乂使崎語城中稱大將軍已克建康甘卓還襄陽外援理絕|{
	言以事理觀之外援已絕也邏郎佐翻語牛倨翻}
崎偽許之既至城下大呼曰援兵尋至努力堅守乂殺之乂拷該至死竟不言其故周級由是獲免乂等攻戰日逼敦又送所得臺中人書疏令乂射以示氶|{
	呼火故翻射而亦翻}
城中知朝廷不守莫不悵惋|{
	惋烏貫翻}
相持且百日劉翼戰死士卒死傷相枕|{
	枕職任翻}
癸巳乂拔長沙氶等皆被執乂將殺虞悝子弟對之號泣悝曰人生會當有死今闔門為忠義之鬼亦復何恨|{
	悝苦囘翻號戶刀翻復扶又翻}
乂以檻車載氶及易雄送武昌佐吏皆犇散惟主簿桓雄西曹書佐韓階|{
	府諸曹各有書佐}
從事武延毁服為僮從氶不離左右|{
	毁服者毁其常服為僮奴之服離力智翻}
乂見桓雄姿貌舉止非凡人憚而殺之韓階武延執志愈固荆州刺史王廙承敦旨殺氶於道中|{
	廙羊至翻又逸職翻}
階延送氶喪至都葬之而去易雄至武昌意氣忼慨曾無懼容|{
	忼口黨翻}
敦遣人以檄示雄而數之雄曰此實有之惜雄位微力弱不能救國難耳|{
	難乃旦翻}
今日之死固所願也敦憚其辭正釋之遣就舍衆人皆賀之雄笑曰吾安得生既而敦遣人潛殺之魏乂求鄧騫甚急鄉人皆為之懼|{
	為于偽翻}
騫笑曰此欲用我耳彼新得州多殺忠良故求我以厭人望也|{
	厭益涉翻}
乃往詣乂乂喜曰君古之解揚也|{
	左傳楚子圍宋晉使解揚如宋使無降楚鄭人囚而獻諸楚楚子厚賂之使反其言不許三而許之登諸樓車使呼宋人而告之遂致其君命楚子將殺之使與之言曰爾既許不穀而反之何故速即爾刑對曰受命而出有死無霣又可賂乎臣之許君以成命也死而成命臣之禄也楚子舍之以歸解戶買翻}
以為别駕詔以陶侃領湘州刺史王敦上侃復還廣州加散騎常侍|{
	上時掌翻}
甲午前趙羊后卒諡曰獻文 甘卓家人皆勸卓備王敦卓不從悉散兵佃作|{
	佃亭年翻}
聞諫輒怒襄陽太守周慮密承敦意詐言湖中多魚勸卓遣左右悉出捕魚五月乙亥慮引兵襲卓於寢室殺之傳首於敦并殺其諸子敦以從事中郎周撫督沔北諸軍事代卓鎮沔中|{
	自南鄭至襄陽沔水所由也故謂之沔中}
撫訪之子也敦既得志暴慢滋甚四方貢獻多入其府將相岳牧皆出其門|{
	舜有四岳十二牧故後之居方面者謂之岳牧}
以沈充錢鳳為謀主唯二人之言是從所譖無不死者以諸葛瑶鄧岳周撫李恒謝雍為爪牙|{
	恒戶登翻}
充等並凶險驕恣大起營府侵人田宅剽掠市道|{
	剽匹妙翻}
識者咸知其將敗焉 秋七月後趙中山公虎拔泰山執徐龕送襄國|{
	龕苦含翻}
後趙王勒盛之以囊於百丈樓上撲殺之|{
	盛時征翻揚正衡曰撲弼角翻}
命王伏都等妻子刳而食之|{
	龕殺王伏都見上卷大興三年}
阬其降卒三千人|{
	降戶江翻}
兖州刺史郄鑒在鄒山三年有衆數萬|{
	愍帝建興元年帝以鑒鎮鄒山今既數年矣所謂三年有衆數萬者言鑒既鎮鄒山之後三年之間民歸之者有此數也郄丑之翻}
戰争不息百姓飢饉掘野鼠蟄鷰而食之|{
	鷰經秋而蟄}
為後趙所逼退屯合肥尚書右僕射紀瞻以鑒雅望清德宜從容臺閣|{
	從千容翻}
上疏請徵之乃徵拜尚書徐兖間諸塢多降於後趙|{
	降戶江翻}
後趙置守宰以撫之 王敦自領寜益二州都督|{
	非君命故史以自領書之}
冬十月己丑荆州刺史武陵康侯王廙卒王敦以下邳内史王邃都督青徐幽平四州諸軍事鎮淮隂衛將軍王含都督沔南諸軍事領荆州刺史武昌太守丹楊王諒為交州刺史 |{
	考異曰諒傳永興三年敦以諒為交州按永興三年即惠帝光熙元年也諒傳誤}
使諒收交州刺史修湛新昌太守梁碩殺之|{
	吳孫皓建衡三年分交趾立新興郡武帝太康三年更名新昌郡}
諒誘湛斬之|{
	誘音酉}
碩舉兵圍諒於龍編|{
	龍編縣屬交阯郡州郡皆治焉}
祖逖既卒後趙屢寇河南拔襄城城父|{
	城父縣前漢屬沛郡後漢屬汝南郡魏晉屬譙國此河南槩言黄河之南非專指河南郡也父音甫}
圍譙豫州刺史祖約不能禦退屯夀春後趙遂取陳留梁鄭之間復騷然矣|{
	復扶又翻}
十一月以臨潁元公荀組為太尉辛酉薨罷司徒并丞相府王敦以司徒官屬為留府|{
	敦還武昌遥制朝政故有留府在建康}
帝憂憤成疾閏月己丑崩|{
	年四十七}
司空王導受遺詔輔政帝恭儉有餘而明斷不足|{
	斷丁亂翻}
故大業未復而禍亂内興庚寅太子即皇帝位大赦尊所生母荀氏為建安君 十二月趙王曜葬其父母於粟邑大赦陵下周二里上高百尺|{
	高倨傲翻}
計用六萬夫作之百日乃成役者夜作繼以脂燭民甚苦之游子遠諫不聽 後趙濮陽景侯張賓卒|{
	濮博木翻}
後趙王勒哭之慟曰天不欲成吾事耶何奪吾右侯之早也程遐代為右長史遐世子弘之舅也勒每與遐議有所不合輒嘆曰右侯捨我去乃令我與此輩共事豈非酷乎|{
	酷慘也虐也言天奪張賓之年何其虐我之慘也}
因流涕彌日 張茂使將軍韓璞帥衆取隴西南安之地置秦州|{
	南陽王保既死陳安不能有茂遂取之帥讀曰率}
慕容廆遣其世子皝襲段末柸入令支|{
	皝戶廣翻令支縣漢屬遼西郡晉省段氏據其地應劭曰令音鈴師古音郎定翻支裴松之音其兒翻}
掠其居民千餘家而還

肅宗明皇帝上|{
	諱紹字道畿元帝長子也諡法思慮果遠曰明}


大寜元年春正月成李驤任囘寇臺登|{
	臺登縣屬越嶲郡九州要記曰臺登縣有奴諾川鸚鵡山黑水之間若水出其下黄帝子昌意降居若水即此}
將軍司馬玫戰死越嶲太守李釗漢嘉太守王載|{
	漢嘉本前漢青衣縣屬蜀郡後漢順帝陽嘉二年更名漢嘉蜀分為漢嘉郡釗音昭}
皆以郡降于成|{
	降戶江翻}
二月庚戍葬元帝於建平陵 三月戊寅朔改元 饒安東光安陵三縣災|{
	三縣皆屬渤海郡惟東光漢舊縣饒安縣前漢之千童縣也後漢靈帝改曰饒安安陵縣晉置時皆為後趙之地}
燒七千餘家死者萬五千人 後趙寇彭城下邳徐州刺史卞敦與征北將軍王邃退保盱眙|{
	盱眙音吁怡}
敦壼之從父兄也|{
	壼苦本翻從才用翻}
王敦謀簒位諷朝廷徵已帝手詔徵之夏四月加敦黄鉞班劒|{
	劉良文選註曰班劒請執劒而從行者也呂向曰班列也言使勇士行列持劒以為儀仗也李周翰曰班劒木劒無刃假作劒形畫之以文故曰班也晉志文武官公給虎賁二十人持班劒}
奏事不名入朝不趨劒履上殿|{
	朝直遥翻上時掌翻}
敦移鎮姑孰屯于湖|{
	姑孰前漢丹陽春穀縣地今太平州當塗縣即姑孰之地縣南二里有姑孰溪西入大江于湖縣本吳督農校尉治武帝太康二年分丹陽縣立于湖縣杜佑曰宣州當塗縣城即晉姑孰城于湖故城在縣南張舜民曰今太平州跨姑孰溪陸游曰姑孰城在當塗北今州城正據姑孰溪溪東南數峯如黛蓋青山也自姑孰溪行夾中三十里至大信口出口泝江過大小褐山磯又過蟂磯蕪湖即于湖並大江有王敦城氣象宏敞並步浪翻考異曰晉春秋及後魏書僭晋傳云屯蕪湖晉書明帝紀云下屯于湖今從之}
以司空導為司徒敦自領揚州牧敦欲為逆王彬諫之甚苦敦變色目左右將收之彬正色曰君昔歲殺兄今又殺弟耶|{
	晉書王彬傳以為彬從兄稜爲敦所害故云然余據殺稜者王如雖出於敦之意猶假手于如也且稜於敦為從弟此言殺兄蓋以敦殺王澄也事見八十卷懷帝永嘉六年}
敦乃止以彬為豫章太守 後趙王勒遣使結好於慕容廆廆執送建康|{
	好呼到翻}
成李驤等進攻寜州刺史褒中壯公王遜使將軍姚嶽等拒之戰於螗蜋|{
	據水經註螗蜋即堂琅縣也前漢屬犍為郡後漢省郡國志犍為屬國朱提縣有堂狼山山多毒草盛夏之月飛鳥過之不能得去蜀置朱提郡堂狼縣屬焉}
成兵大敗嶽追至瀘水成兵爭濟溺死者千餘人嶽以道遠不敢濟而還|{
	溺奴狄翻還從宣翻又如字}
遜以嶽不窮追大怒鞭之怒甚冠裂而卒遜在州十四年|{
	懷帝永嘉四年遜至寜州至是適十四年}
威行殊俗州人立其子堅行州府事|{
	州寜州府南夷校尉府也}
詔除堅寜州刺史 廣州刺史陶侃遣兵救交州未至梁碩抜龍編奪刺史王諒節諒不與碩斷其右臂諒曰死且不避斷臂何為|{
	斷丁管翻}
踰旬而卒六月壬子立妃庾氏為皇后以后兄中領軍亮為中

書監 梁碩據交州凶暴失衆心陶侃遣參軍高寶攻碩斬之詔以侃領交州刺史進號征南大將軍開府儀同三司未幾吏部郎阮放求為交州刺史許之|{
	幾居豈翻}
放行至寜浦|{
	廣州記曰漢獻帝建安二十三年吳分鬱林郡立寜浦郡晉太康地志曰武帝太康七年改合浦屬國都尉立寜浦郡唐為横州寜浦縣浦滂五翻}
遇高寶為寶設饌|{
	為于偽翻饌雛晥翻又雛戀翻}
伏兵殺之寶兵擊放放走得免至州少時病卒|{
	少詩沼翻 考異曰放傳云成帝幼冲庾氏執政放求為交州下乃云逢高寶平梁碩還非成帝時也放傳誤}
放咸之族子也|{
	阮咸有名於魏晉之間}
陳安圍趙征西將軍劉貢于南安休屠王石武自桑城引兵趣上邽以救之|{
	屠直於翻趣士喻翻}
與貢合擊安大破之安收餘騎八千走保隴城|{
	騎奇寄翻}
秋七月趙主曜自將圍隴城别遣兵圍上邽安頻出戰輒敗右軍將軍劉幹攻平襄克之|{
	平襄縣漢屬天水郡晉屬略陽郡}
隴上諸縣悉降|{
	降戶江翻下同}
安留其將楊伯支姜冲兒守隴城自帥精騎突圍出奔陜中|{
	陜中在隴城南陜與陿同戶夾翻}
曜遣將軍平先等追之安左揮七尺大刀右運丈八虵矛近則刀矛俱發輒殪五六人|{
	殪壹計翻}
遠則左右馳射而走先亦勇捷如飛與安摶戰三交遂奪其虵矛|{
	三交戰三合也}
會日暮雨甚安棄馬與左右匿於山中趙兵索之不知所在|{
	索山客翻}
明日安遣其將石容覘趙兵|{
	將即亮翻覘丑廉翻又丑艷翻}
趙輔威將軍呼延青人獲之拷問安所在|{
	拷苦皓翻掠也擊也}
容卒不肯言|{
	卒子恤翻}
青人殺之雨霽青人尋其迹獲安於澗曲斬之安善撫將士與同甘苦及死隴上人思之為作壯士之歌|{
	歌曰隴上壯士有陳安軀幹雖小腹中寛愛養將士同心肝䯀騘交馬鐵鍛鞍七尺大刀奮如湍丈八虵矛左右盤十盪十决無當前戰始三交失虵矛棄我䯀騘竄岩幽為我外援而懸頭西流之水東流河一去不還柰子何為于偽翻}
楊伯支斬姜冲兒以隴城降别將宋亭斬趙募以上邽降曜徙秦州大姓楊姜諸族二千餘戶于長安氐羌皆送任請降|{
	任質任也}
以赤亭羌酋姚弋仲為平西將軍封平襄公|{
	酋慈由翻}
帝畏王敦之逼欲以郗鑒為外援|{
	郗丑之翻}
拜鑒兖州刺史都督揚州江西諸軍事鎮合肥王敦忌之表鑒為尚書令八月詔徵鑒還道經姑孰敦與之論西朝人士曰樂彦輔短才耳考其實豈勝滿武秋耶|{
	時江東謂洛都為西朝樂廣字彦輔滿奮字武秋朝直遥翻}
鑒曰彦輔道韻平淡愍懷之廢柔而能正武秋失節之士安得擬之|{
	事見八十三卷惠帝永康元年滿奮既收東宫官屬之辭太子者趙王倫之簒奮又奉璽綬故謂之失節}
敦曰當是時危機交急鑒曰丈夫當死生以之敦惡其言不復相見|{
	惡烏路翻復扶又翻}
久留不遣敦黨皆勸敦殺之敦不從鑒還臺遂與帝謀討敦 後趙中山公虎帥步騎四萬擊安東將軍曹嶷|{
	帥讀曰率嶷魚力翻}
青州郡縣多降之遂圍廣固|{
	水經注廣固城在漢齊郡廣縣西北四里四周絶澗阻水深隍曹嶷所築也九域志廣固城古樂安城今按青州益都縣西四十里有廣固城杜佑曰有大澗甚廣因曰廣固降戶江翻}
嶷出降送襄國殺之阬其衆三萬虎欲盡殺嶷衆青州刺史劉徵曰今留徵使牧民也無民焉牧|{
	焉於䖍翻}
徵將歸耳虎乃留男女七百口配徵使鎮廣固 趙主曜自隴上西擊凉州遣其將劉咸攻韓璞於冀城呼延宴攻寜羌護軍隂鑒於桑壁|{
	桑壁屬在南安界}
曜自將戎卒二十八萬軍于河上|{
	將即亮翻}
列營百餘里金鼓之聲動地河水為沸張茂臨河諸戍皆望風奔潰曜揚聲欲百道俱濟直抵姑臧凉州大震參軍馬岌勸茂親出拒戰長史汜褘怒請斬之岌曰汜公糟粕書生刺舉小才|{
	莊子曰桓公讀書於堂上輪扁斵輪於堂下問桓公曰敢問公所讀者何言也公曰聖人之書也曰聖人在乎曰已死矣曰然則君之所讀者古人之糟粕已矣古之人與其不可傳者死矣陸德明曰糟音遭李云酒滓也粕普各翻糟爛為粕刺者以直傷人舉者招人之過汜音凡}
不思家國大計明公父子欲為朝廷誅劉曜有年矣|{
	為于偽翻下為明同}
今曜自至遠近之情共觀明公此舉當立信勇之驗以副秦隴之望力雖不敵勢不可以不出茂曰善乃出屯石頭|{
	石頭在姑臧城東}
茂謂參軍陳珍曰劉曜舉三秦之衆乘勝席卷而來|{
	言新破陳安乘勝而來也卷讀曰捲}
將若之何珍曰曜兵雖多精卒至少|{
	少詩沼翻}
大抵皆氐羌烏合之衆恩信未洽且有山東之虞|{
	謂方與石勒相圖也}
安能捨其腹心之疾曠日持久與我爭河西之地邪若二旬不退珍請得弊卒數千為明公擒之|{
	為于偽翻}
茂喜使珍將兵救韓璞趙諸將爭欲濟河趙主曜曰吾軍勢雖盛然畏威而來者三分有二中軍疲困其實難用|{
	果如陳珍所料}
今但案甲勿動以吾威聲震之若出中旬張茂之表不至者吾為負卿矣茂尋遣使稱藩獻馬牛羊珍寶不可勝紀|{
	使疏吏翻勝音升}
曜拜茂侍中都督凉南北秦梁益巴漢隴右西域雜夷匈奴諸軍事太師凉州牧封凉王加九錫 楊難敵聞陳安死大懼與弟堅頭南奔漢中趙鎮西將軍劉厚追撃之大獲而還趙主曜以大鴻臚田崧為鎮南大將軍益州刺史鎮仇池難敵送任請降於成|{
	降戶江翻}
成安北將軍李稚受難敵賂不送難敵於成都趙兵退即遣歸武都難敵遂據險不服稚自悔失計亟請討之|{
	亟欺冀翻亟請數以為請也}
雄遣稚兄侍中中領軍琀與稚出白水征東將軍李夀及琀弟玝出隂平以撃難敵|{
	楊正衡曰琀胡紺翻玝音午}
羣臣諫不聽難敵遣兵拒之夀玝不得進而琀稚長驅至下辨|{
	辨步莧翻}
難敵遣兵斷其歸路四面攻之|{
	斷丁管翻}
琀稚深入無繼皆為難敵所殺死者數千人琀蕩之長子|{
	長知兩翻下同}
有才望雄欲以為嗣聞其死不食者數日 初趙主曜長子儉次子胤胤年十歲長七尺五寸|{
	長子之長知兩翻下同長七之長直亮翻}
漢主聰奇之謂曜曰此兒神氣非義真之比也|{
	儉字義真}
當以為嗣曜曰藩國之嗣能守祭祀足矣不敢亂長幼之序聰曰卿之勲德當世受專征之任|{
	言當世為方伯得專征伐也}
非它臣之比也吾當更以一國封義真乃封儉為臨海王立胤為世子既長多力善射驍捷如風|{
	驍堅堯翻}
靳凖之亂|{
	事見九十卷大興元年}
没於黑匿郁鞠部|{
	黑匿郁鞠既歸胤曜嘉其忠欵封為左賢王則亦匈奴之種也}
陳安既敗胤自言於郁鞠郁鞠大驚禮而歸之曜悲喜謂羣臣曰義光雖已為太子然冲幼儒謹恐不堪今之多難義孫故世子也|{
	曜大子熙字義光胤字義孫難乃旦翻}
材器過人且涉歷艱難吾欲法周文王漢光武以固社稷而安義光何如|{
	周文王舍伯邑考而立武王漢光武舍長子彊而立明帝}
太傅呼延宴等皆曰陛下為國家無窮之計豈惟臣等賴之實宗廟四海之慶左光禄大夫卜泰太子太保韓廣進曰陛下以廢立為是不應更問羣臣若以為疑固樂聞異同之言|{
	樂音洛}
臣竊以為廢太子非也昔文王定嗣於未立之前則可也光武以母失恩而廢其子豈足為聖朝之法|{
	朝直遥翻}
曏以東海為嗣未必不如明帝也胤文武才畧誠高絶於世然太子孝友仁慈亦足為承平賢主况東宫者民人所繫豈可輕動陛下誠欲如是臣等有死而已不敢奉詔曜默然胤進曰父之於子當愛之如一今黜熙而立臣臣何敢自安陛下苟以臣為頗堪驅策豈不能輔熙以承聖業乎必若以臣代熙臣請效死於此不敢聞命因歔欷流涕|{
	歔音虚欷許既翻又音希}
曜亦以熙羊后所生不忍廢也乃追諡前妃卜氏為元悼皇后泰即胤之舅也曜嘉其公忠以為上光禄大夫儀同三司領太子太傅封胤為永安王拜侍中衛大將軍都督二宫禁衛諸軍事開府儀同三司録尚書事|{
	二宫曜宫及熙宫也}
命熙於胤盡家人之禮|{
	不以儲嗣使熙廢兄弟之庸敬}
張茂大城姑臧修靈鈞臺|{
	元帝大興四年茂築靈鈞臺以閻曾諫而止今復修之}
别駕吳紹諫曰明公所以修城築臺者蓋懲既往之患耳|{
	謂懲劉曜來攻也}
愚以為苟恩未洽於人心雖處層臺|{
	處昌呂翻}
亦無所益適足以疑羣下忠信之志失士民繫託之望示怯弱之形啓鄰敵之謀將何以佐天子霸諸侯乎願亟罷兹役以息勞費茂曰亡兄一旦失身於物|{
	茂兄實為其下所殺事見上卷大興三年}
豈無忠臣義士欲盡節者哉顧禍生不意雖有智勇無所施耳王公設險勇夫重閉古之道也|{
	易曰王公設險以守其國左傳曰勇夫重閉而況國乎重直龍翻}
今國家未靖不可以太平之理責人於屯邅之世也|{
	屯株倫翻難也邅張連翻行不進貌}
卒為之|{
	卒子恤翻}
王敦從子允之方總角|{
	毛萇曰總角聚兩髦也從才用翻}
敦愛其聰警常以自隨敦常夜飲允之辭醉先卧敦與錢鳳謀為逆允之悉聞其言即於卧處大吐|{
	吐土故翻下同}
衣面並污|{
	汚烏故翻}
鳳出敦果照視見允之卧於吐中不復疑之會其父舒拜廷尉允之求歸省父|{
	復扶又翻省悉景翻}
悉以敦鳳之謀白舒舒與王導俱啓帝隂為之備敦欲彊其宗族陵弱帝室冬十一月徙王含為征東將軍都督揚州江西諸軍事王舒為荆州刺史監荆州沔南諸軍事|{
	監工銜翻}
王彬為江州刺史 後趙王勒以參軍樊坦為章武内史|{
	章武縣漢屬勃海郡武帝泰始元年分置章武國隋廢章武并入河間郡唐為瀛州}
勒見其衣冠敝壞問之坦率然對曰頃為羯賊所掠資財蕩盡勒笑曰羯賊乃爾無道邪|{
	羯居謁翻}
今當相償坦大懼叩頭泣謝勒賜車馬衣服裝錢三百萬而遣之 是歲越嶲斯叟攻成將任囘|{
	前漢西南夷傳云自嶲以東北君長以十數徙筰都最大師古曰徙及筰都二國也嶲音髓徙音斯此斯即漢之斯種也蜀謂之叟將即亮翻任音壬}
成主雄遣征南將軍費黑討之|{
	費扶沸翻}
會稽内史周札一門五侯|{
	札東遷縣侯兄靖子懋清流亭侯懋弟贊武康縣侯贊弟縉都鄉侯兄玘子勰烏程縣侯凡五侯會工外翻}
宗族彊盛吳士莫與為比王敦忌之敦有疾錢鳳勸敦早除周氏敦然之周嵩以兄顗之死|{
	事見元帝永昌元年顗魚豈翻}
心常憤憤敦無子養王含子應為嗣嵩嘗於衆中言應不宜統兵敦惡之|{
	惡烏路翻}
嵩與札兄子筵皆為敦從事中郎會道士李脫以妖術惑衆士民頗信事之|{
	妖於驕翻}


資治通鑑卷九十二














































































































































