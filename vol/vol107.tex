<!DOCTYPE html PUBLIC "-//W3C//DTD XHTML 1.0 Transitional//EN" "http://www.w3.org/TR/xhtml1/DTD/xhtml1-transitional.dtd">
<html xmlns="http://www.w3.org/1999/xhtml">
<head>
<meta http-equiv="Content-Type" content="text/html; charset=utf-8" />
<meta http-equiv="X-UA-Compatible" content="IE=Edge,chrome=1">
<title>資治通鑒_108-資治通鑑卷一百七_108-資治通鑑卷一百七</title>
<meta name="Keywords" content="資治通鑒_108-資治通鑑卷一百七_108-資治通鑑卷一百七">
<meta name="Description" content="資治通鑒_108-資治通鑑卷一百七_108-資治通鑑卷一百七">
<meta http-equiv="Cache-Control" content="no-transform" />
<meta http-equiv="Cache-Control" content="no-siteapp" />
<link href="/img/style.css" rel="stylesheet" type="text/css" />
<script src="/img/m.js?2020"></script> 
</head>
<body>
 <div class="ClassNavi">
<a  href="/24shi/">二十四史</a> | <a href="/SiKuQuanShu/">四库全书</a> | <a href="http://www.guoxuedashi.com/gjtsjc/"><font  color="#FF0000">古今图书集成</font></a> | <a href="/renwu/">历史人物</a> | <a href="/ShuoWenJieZi/"><font  color="#FF0000">说文解字</a></font> | <a href="/chengyu/">成语词典</a> | <a  target="_blank"  href="http://www.guoxuedashi.com/jgwhj/"><font  color="#FF0000">甲骨文合集</font></a> | <a href="/yzjwjc/"><font  color="#FF0000">殷周金文集成</font></a> | <a href="/xiangxingzi/"><font color="#0000FF">象形字典</font></a> | <a href="/13jing/"><font  color="#FF0000">十三经索引</font></a> | <a href="/zixing/"><font  color="#FF0000">字体转换器</font></a> | <a href="/zidian/xz/"><font color="#0000FF">篆书识别</font></a> | <a href="/jinfanyi/">近义反义词</a> | <a href="/duilian/">对联大全</a> | <a href="/jiapu/"><font  color="#0000FF">家谱族谱查询</font></a> | <a href="http://www.guoxuemi.com/hafo/" target="_blank" ><font color="#FF0000">哈佛古籍</font></a> 
</div>

 <!-- 头部导航开始 -->
<div class="w1180 head clearfix">
  <div class="head_logo l"><a title="国学大师官网" href="http://www.guoxuedashi.com" target="_blank"></a></div>
  <div class="head_sr l">
  <div id="head1">
  
  <a href="http://www.guoxuedashi.com/zidian/bujian/" target="_blank" ><img src="http://www.guoxuedashi.com/img/top1.gif" width="88" height="60" border="0" title="部件查字,支持20万汉字"></a>


<a href="http://www.guoxuedashi.com/help/yingpan.php" target="_blank"><img src="http://www.guoxuedashi.com/img/top230.gif" width="600" height="62" border="0" ></a>


  </div>
  <div id="head3"><a href="javascript:" onClick="javascript:window.external.AddFavorite(window.location.href,document.title);">添加收藏</a>
  <br><a href="/help/setie.php">搜索引擎</a>
  <br><a href="/help/zanzhu.php">赞助本站</a></div>
  <div id="head2">
 <a href="http://www.guoxuemi.com/" target="_blank"><img src="http://www.guoxuedashi.com/img/guoxuemi.gif" width="95" height="62" border="0" style="margin-left:2px;" title="国学迷"></a>
  

  </div>
</div>
  <div class="clear"></div>
  <div class="head_nav">
  <p><a href="/">首页</a> | <a href="/ShuKu/">国学书库</a> | <a href="/guji/">影印古籍</a> | <a href="/shici/">诗词宝典</a> | <a   href="/SiKuQuanShu/gxjx.php">精选</a> <b>|</b> <a href="/zidian/">汉语字典</a> | <a href="/hydcd/">汉语词典</a> | <a href="http://www.guoxuedashi.com/zidian/bujian/"><font  color="#CC0066">部件查字</font></a> | <a href="http://www.sfds.cn/"><font  color="#CC0066">书法大师</font></a> | <a href="/jgwhj/">甲骨文</a> <b>|</b> <a href="/b/4/"><font  color="#CC0066">解密</font></a> | <a href="/renwu/">历史人物</a> | <a href="/diangu/">历史典故</a> | <a href="/xingshi/">姓氏</a> | <a href="/minzu/">民族</a> <b>|</b> <a href="/mz/"><font  color="#CC0066">世界名著</font></a> | <a href="/download/">软件下载</a>
</p>
<p><a href="/b/"><font  color="#CC0066">历史</font></a> | <a href="http://skqs.guoxuedashi.com/" target="_blank">四库全书</a> |  <a href="http://www.guoxuedashi.com/search/" target="_blank"><font  color="#CC0066">全文检索</font></a> | <a href="http://www.guoxuedashi.com/shumu/">古籍书目</a> | <a   href="/24shi/">正史</a> <b>|</b> <a href="/chengyu/">成语词典</a> | <a href="/kangxi/" title="康熙字典">康熙字典</a> | <a href="/ShuoWenJieZi/">说文解字</a> | <a href="/zixing/yanbian/">字形演变</a> | <a href="/yzjwjc/">金 文</a> <b>|</b>  <a href="/shijian/nian-hao/">年号</a> | <a href="/diming/">历史地名</a> | <a href="/shijian/">历史事件</a> | <a href="/guanzhi/">官职</a> | <a href="/lishi/">知识</a> <b>|</b> <a href="/zhongyi/">中医中药</a> | <a href="http://www.guoxuedashi.com/forum/">留言反馈</a>
</p>
  </div>
</div>
<!-- 头部导航END --> 
<!-- 内容区开始 --> 
<div class="w1180 clearfix">
  <div class="info l">
   
<div class="clearfix" style="background:#f5faff;">
<script src='http://www.guoxuedashi.com/img/headersou.js'></script>

</div>
  <div class="info_tree"><a href="http://www.guoxuedashi.com">首页</a> > <a href="/SiKuQuanShu/fanti/">四库全书</a>
 > <h1>资治通鉴</h1> <!--         下载:【右键另存为】即可 --></div>
  <div class="info_content zj clearfix">
  
<div class="info_txt clearfix" id="show">
<center style="font-size:24px;">108-資治通鑑卷一百七</center>
    資治通鑑卷一百七   宋 司馬光 撰<br />
<br />
  胡三省 音註<br />
<br />
  晉紀二十九【起彊圉大淵獻盡重光單閼凡五年】<br />
<br />
  烈宗孝武皇帝中之下<br />
<br />
  太元十二年春正月乙巳以朱序為兖青二州刺史代謝玄鎮彭城序求鎮淮隂許之【序求鎮淮隂以燕方強必進取河南彭城去建康道遠聲援不接故也】以玄為會稽内史【優玄以内地也會工外翻】 丁未大赦 燕主垂觀兵河上【韋昭曰觀示也陳兵以示威武觀古玩翻】高陽王隆曰温詳之徒皆白面儒生烏合為羣徒恃長河以自固若大軍濟河必望旗震壞不待戰也垂從之戊午遣鎮北將軍蘭汗護軍將軍平幼於碻磝西四十里濟河隆以大衆陳於北岸【陳讀曰陣】温攀温楷果走趣城【盖趣東阿城也趣七喻翻】平幼追擊大破之詳夜將妻子犇彭城其衆三萬餘戶皆降於燕【降戶江翻】垂以太原王楷為兖州刺史鎮東阿初垂在長安秦王堅嘗與之交手語宂從僕射光祚言於堅曰【宂而隴翻從才用翻】陛下頗疑慕容垂乎垂非久為人下者也堅以告垂及秦主丕自鄴奔晉陽【事見上卷十年】祚與黄門侍郎封孚鉅鹿太守封勸皆來奔【祚從符丕在鄴見上卷九年】勸奕之子也【封奕仕燕燕興於昌黎奕有力焉】垂之再圍鄴也【見一百五卷九年】秦故臣西河朱肅等各以其衆來犇詔以祚等為河北諸郡太守皆營於濟北濮陽【濟北濮陽二郡濟子禮翻濮博木翻】羈屬温詳【師古曰言羈縻屬之而已】詳敗俱詣燕軍降【降戶江翻】垂赦之撫待如舊垂見光祚流涕沾衿【衿音今】曰秦王待我深吾事之亦盡但為二公猜忌【二公謂長樂公丕平原公暉也】吾懼死而負之【事見一百五卷九年】每一念之中宵不寐祚亦悲慟垂賜祚金帛祚固辭垂曰卿猶復疑邪【復扶又翻】祚曰臣昔者惟知忠於所事不意陛下至今懷之臣敢逃其死垂曰此乃卿之忠固吾所求也前言戲之耳【用孔子語】待之彌厚以為中常侍【光祚秦之宦者故處以此官】 翟遼遣其子釗寇陳頴朱序遣將軍秦膺擊走之 秦主登立妃毛氏為皇后勃海王懿為太弟后興之女也遣使拜東海王纂為使持節都督中外諸軍事太師領大司馬封魯王【使疏吏翻】纂弟師奴為撫軍大將軍并州牧封朔方公纂怒謂使者曰勃海王先帝之子南安王何以不立而自立乎長史王旅諫曰南安已立理無中改今寇虜未滅不可宗室之中自為仇敵也纂乃受命於是盧水胡彭沛穀屠各董成張龍世新平羌雷惡地等皆附於纂有衆十餘萬【以登纂連兵聲勢浸盛故相與歸之屠直於翻】 後秦主萇徙秦州豪傑三萬戶于安定【去年萇徙安定民以實長安今又徙秦州豪傑以實安定盖萇起兵以安定為根本而欲都長安故因道里遠近為次以漸徙之】 初安次人齊涉聚衆八千餘家據新柵降燕【安次縣前漢屬勃海後漢屬廣陽國晉屬燕國新柵盖在魏郡界降戶江翻】燕主垂拜涉魏郡太守既而復叛連張願願自帥萬餘人進屯祝阿之瓮口【祝阿縣漢屬平原郡晉屬濟南郡願自泰山進屯焉劉昫曰齊州禹城縣漢祝阿縣天寶元年更名宋白曰祝阿猶東阿也古祝國黄帝之後按古東阿齊為東阿漢為祝阿縣故城在今豐齊縣東北二里唐改禹城復扶又翻帥讀曰率瓮烏貢翻】招翟遼共應涉高陽王隆言於垂曰新柵堅固攻之未易猝拔【易以䜴翻】若久頓兵於其城下張願擁帥流民西引丁零【丁零謂翟遼帥讀曰率】為患方深願衆雖多然皆新附未能力鬬因其自至宜先擊之願父子恃其驍勇【驍堅堯翻】必不肯避去可一戰擒也願破則涉不能自存矣垂從之二月遣范陽王德陳留王紹龍驤將軍張崇【驤思將翻】帥步騎二萬會隆擊願軍至斗城去瓮口二十餘里解鞍頓息願引兵奄至燕人驚遽德兵退走隆勒兵不動願子龜出衝陳【陳讀曰陣】隆遣左右王末逆擊斬之隆徐進戰願兵乃退德行里餘復整兵還與隆合【復扶又翻】謂隆曰賊氣方銳宜且緩之隆曰願乘人不備宜得大捷而吾士卒皆以懸隔河津勢迫之故人思自戰【言兵為河津所隔前有強敵退則溺死故思之而各自為戰也】故能却之今賊不得利氣竭勢衰皆有進退之志不能齊奮宜亟擊之德曰吾唯卿所為耳遂進戰於瓮口大破之斬首七千八百級願脫身保三布口燕人進軍歷城【歷城縣自漢以來屬濟南郡】青兖徐州郡縣壁壘多降【降戶江翻】垂以陳留王紹為青州刺史鎮歷城德等還師新柵人冬鸞執涉送之【果如慕容隆所料唐韻冬姓也】垂誅涉父子餘悉原之 三月秦主登以竇衝為南秦州牧楊定為益州牧楊璧為司空梁州牧乞伏國仁為大將軍大單于苑川王【杜佑曰苑川在蘭州五泉縣近大小榆谷余謂杜佑以意言之單音蟬】 燕上谷人王敏殺太守封戢代郡人許謙逐太守賈閏各以郡附劉顯【為燕擊劉顯張本】 燕樂浪王温為尚書右僕射【燕下當有以字樂浪音洛琅】 夏四月戊辰尊帝母李氏為皇太妃儀服如太后 後秦征西將軍姚碩德為楊定所逼退守涇陽【涇陽縣前漢屬安定郡後漢晉省秦屬隴東郡杜佑曰漢涇陽縣在今平涼郡界涇陽故城是】定與秦魯王纂共攻之戰于涇陽碩德大敗後秦主萇自隂密救之纂退屯敷陸【隂密縣屬安定郡殷時密國也敷陸唐坊州鄜城縣即其地】 燕主垂自碻磝還中山慕容柔慕容盛慕容會來自長子【柔等去年自長子逃歸今始達中山】庚子垂為之大赦【喜子孫得全而東歸故為之肆赦為于偽翻】垂問盛長子人情如何為可取乎盛曰西軍擾擾人有東歸之志陛下唯當脩仁政以俟之耳若大軍一臨必投戈而來若孝子之歸慈父也垂悦癸未封柔為陽平王盛為長樂公【樂音洛】會為清河公高平人翟暢執太守徐含遠以郡降翟遼【降戶江翻】燕主<br />
<br />
  垂謂諸將曰遼以一城之衆反覆三國之間【三國謂晉及燕西燕】不可不討五月以章武王宙監中外諸軍事【監工銜翻】輔太子寶守中山垂自帥諸將南攻遼【帥讀曰率下同】以太原王楷為前鋒都督遼衆皆燕趙之人聞楷至皆曰太原王子吾之父母也【楷父恪相燕燕趙之人懷之故云然】相帥歸之遼懼遣使請降垂以遼為徐州牧封河南公前至黎陽受降而還【降戶江翻】井陘人賈鮑【井陘縣屬常山郡陘音刑】招引北山丁零翟遥等五千餘人夜襲中山䧟其郭章武王宙以奇兵出其外太子寶鼓譟於内合擊大破之盡俘其衆唯遥鮑單馬走免 劉顯地廣兵彊雄於北方會其兄弟乖争魏長史張衮言於魏王珪曰顯志在并吞今不乘其内潰而取之【奴真肺埿相繼來降故云然】必為後患然吾不能獨克請與燕共攻之珪從之復遣安同乞師於燕【去年魏遣安同乞師於燕以破窟咄故此言復復扶又翻】 詔徵會稽處士戴逵【會工外翻處昌呂翻】逵累辭不就郡縣敦逼不已逵逃匿于吳謝玄上疏曰逵自求其志【論語曰隱居以求其志】今王命未回將罹風霜之患陛下既已愛而器之亦宜使其身名並存請絶召命帝許之【玄為會稽内史故為逵上疏】逵?之兄也【戴?見一百四卷四年】 秦主登以其兄同成為司徒守尚書令封潁川王弟廣為中書監封安成王子崇為尚書左僕射封東平王燕主垂自黎陽還中山 吳深殺燕清河太守丁國章武人王祖殺太守白欽勃海人張申據高城以叛【高城縣屬勃海郡賢曰高城故城在今滄州鹽山縣南】燕主垂命樂浪王温討之 苑川王國仁帥騎三萬襲鮮卑大人密貴裕苟提倫三部于六泉【密貴為一部裕苟為一部提倫為一部六泉在高平帥讀曰率騎奇寄翻】秋七月與没奕千金熙戰于渇渾川【按載記國仁襲三部而没奕千金熙連兵襲國仁故遇戰于渇渾川其地當在天水勇士縣東北】沒奕千金熙大敗三部皆降【降戶江翻】秦主登軍于瓦亭後秦主萇攻彭沛穀堡拔之穀犇杏城【彭沛穀盧水胡也立堡於貳縣杜佑曰杏城在坊州西】萇還隂密以太子興鎮長安 燕趙王麟討王敏于上谷斬之 劉衛辰獻馬於燕劉顯掠之燕主垂怒遣太原王楷將兵助趙王麟擊顯大破之【將即亮翻】顯犇馬邑西山魏王珪引兵會麟擊顯於彌澤【按魏書帝紀彌澤在馬邑南】又破之顯犇西燕麟悉收其部衆獲馬牛羊以千萬數【劉顯滅而拓拔氏彊矣為慕容氏計者莫若兩利而俱存之可以無他日亡國之禍】呂光將彭晃徐炅攻張大豫于臨洮破之【張大豫奔臨洮見上卷上年洮土刀翻】大豫犇廣武王穆犇建康八月廣武人執大豫送姑臧斬之穆襲據酒泉自稱大將軍涼州牧 辛巳立皇子德宗為太子大赦 燕主垂立劉顯弟可泥為烏桓王以撫其衆徙八千餘落于中山 秦馮翊太守蘭櫝帥衆二萬自頻陽入和寧【頻陽縣秦厲公置自漢以來屬馮翊應劭曰在頻水之陽據載記和寧在嶺北杏城之東南帥讀曰率】與魯王纂謀攻長安纂弟師奴勸纂稱尊號纂不從師奴殺纂而代之櫝遂與師奴絶西燕主永攻櫝櫝請救於後秦後秦主萇欲自救之尚書令姚旻左僕射尹緯曰苻登近在瓦亭將乘虚襲吾後萇曰苻登衆盛非旦夕可制登遲重少决必不能輕軍深入比兩月間【比必寐翻及也】吾必破賊而返登雖至無能為也九月萇軍于泥源【漢書地理志北地郡有泥陽縣應劭注云泥水出郁郅北蠻中】 師奴逆戰大敗亡犇鮮卑後秦盡收其衆屠各董成等皆降【苻纂兄弟既敗苻登之勢孤矣屠直於翻】 秦主登進據胡空堡【秦屯騎校尉胡空所築堡也在新平界】戎夏歸之十餘萬【夏戶雅翻】 冬十月翟遼復叛燕【復扶又翻下同】遣兵與王祖張申寇抄清河平原【抄楚交翻】 後秦主萇進擊西燕王永於河西【西燕王當作西燕主此龍門至華隂河之西也】永走蘭櫝復列兵拒守萇攻之十二月禽櫝遂如杏城 後秦姚方成攻秦雍州刺史徐嵩壘拔之執嵩而數之【雍於用翻數所具翻】嵩罵曰汝姚萇罪當萬死苻黄眉欲斬之先帝止之【謂穆帝升平元年姚襄敗時也】授任内外榮寵極矣曾不如犬馬識所養之恩親為大逆【謂殺秦王堅於新平佛寺也】汝羌輩豈可以人理期也何不速殺我方成怒三斬嵩【三斬者斬其足斬其腰斬其頸也】悉阬其士卒以妻子賞軍後秦王萇掘秦主堅尸鞭撻無數剥衣倮形薦之以棘坎土而埋之【堅葬于徐嵩胡空二壘之間徐嵩之壘既陷故姚萇得掘墓鞭尸以逞其忿倮郎果翻薦藉也】凉州大饑米斗直錢五百人相食死者大半 呂光<br />
<br />
  西平太守康寧自稱匈奴王殺湟河太守強禧以叛【西平郡東漢之末分金城置唐之鄯州即其地也湟河郡河西張氏置盖亦在鄯州界内強其兩翻】張掖太守彭晃亦叛東結康寧西通王穆光欲自擊晃諸將皆曰今康寧在南伺釁而動【伺相吏翻】若晃穆未誅康寧復至【復扶又翻】進退狼狽勢必大危光曰實如卿言然我今不往是坐待其來也若三寇連兵【三寇謂康寧彭晃王穆】東西交至則城外皆非吾有大事去矣今晃初叛與穆寧情契未密出其倉猝取之差易耳【易以䜴翻】乃自帥騎三萬【帥讀曰率騎奇寄翻】倍道兼行既至攻之二旬拔其城誅晃初王穆起兵遣使招敦煌處士郭瑀【使疏吏翻敦徒門翻處昌呂翻】瑀歎曰今民將左袵吾忍不救之邪乃與同郡索嘏起兵應穆【索昔各翻】運粟三萬石以餉之穆以瑀為太府左長史軍師將軍嘏為敦煌太守既而穆聽讒言引兵攻嘏瑀諫不聽出城大哭舉手謝城曰吾不復見汝矣【復扶又翻】還而引被覆面【覆敷又翻】不與人言不食而卒【卒子恤翻】呂光聞之曰二虜相攻此成禽也不可以憚屢戰之勞而失永逸之機也【一勞永逸古語有之】遂帥步騎二萬攻酒泉克之進屯凉興【凉興郡河西張氏置在唐瓜州常樂縣界】穆引兵東還未至衆潰穆單騎走騂馬令郭文斬其首送之【騂馬縣屬酒泉郡盖魏晉間所置也騂思榮翻呂光新得河西黨叛於内敵攻於外雖數戰數勝而根本不固宜不足以貽子孫也】<br />
<br />
  十三年春正月康樂獻武公謝玄卒【樂音洛康樂縣屬豫章郡】 二月秦主登軍朝那【朝那縣自漢以來屬安定郡】後秦主萇軍武都【此武都亦當在安定界五代志朝那縣西魏置安武郡安武漢舊縣名武都之名當是因安武而名】 翟遼遣司馬眭瓊詣燕謝罪【眭姓也師古息隨翻類篇宜為翻】燕主垂以其數反覆斬瓊以絶之【數所角翻】遼乃自稱魏天王改元建光置百官 燕青州刺史陳留王紹為平原太守辟閭渾所逼退屯黄巾固【漢末黄巾保聚于其地因以為名齊人謂壘堡為固紹自歷城退屯焉其地在濟南郡章丘城北】燕主垂更以紹為徐州刺史渾蔚之子也【辟閭蔚見一百卷穆帝永和十二年蔚紆忽翻】因苻氏亂據齊地來降【後辟閭渾為慕容德所殺降戶江翻】 三月乙亥燕主垂以太子寶録尚書事授之以政自總大綱而已 燕趙王麟擊許謙破之【去年許謙叛燕附劉顯】謙犇西燕遂廢代郡悉徙其民於龍城 呂光之定涼州也杜進功居多光以為武威太守【事見上卷十年】貴寵用事羣僚莫及光甥石聰自關中來光問之曰中州人言我為政何如聰曰但聞有杜進耳不聞有舅光由是忌進而殺之光與羣寮宴語及政事參軍京兆段業曰明公用灋太峻光曰吳起無恩而楚彊商鞅嚴刑而秦興業曰起喪其身鞅亡其家皆殘酷之致也【吴起事見一卷周安王十五年商鞅事見二卷顯王三十一年喪息浪翻】明公方開建大業景行堯舜【詩曰高山仰止景行行止毛萇曰景大也鄭玄曰景明庶幾古人有高德者則慕仰之有明行者則而行之行下孟翻】猶懼不濟乃慕起鞅之為治【治直吏翻】豈此州士女所望哉光改容謝之【沮渠蒙遜兄弟舉兵所以推段業為重亦由此言為凉州人士所歸敬也】 夏四月戊午以朱序為都督司雍梁秦四州諸軍事雍州刺史戍洛陽【雍於用翻】以譙王恬代序為都督兖冀幽并諸軍事青兖二州刺史 苑川王國仁破鮮卑越質叱黎於平襄【平襄縣漢屬天水郡晉屬畧陽郡越質盖鮮卑部落之號後以為氏】獲其子詰歸 丁亥燕主垂立夫人段氏為皇后以太子寶領大單于【單音蟬】段氏右光禄大夫儀之女其妹適范陽王德儀寶之舅也【為後寶逼殺段后張本】追諡前妃段氏為成昭皇后【段氏死見一百卷穆帝升平二年】 五月秦太弟懿卒諡曰獻哀 翟遼徙屯滑臺【遼自黎陽徙屯滑臺既與燕絶欲阻河為固也滑臺城在白馬縣西春秋鄭廩延邑也唐為滑州】 六月苑川王乞伏國仁卒諡曰宣烈廟號烈祖其子公府尚幼羣下推國仁弟乾歸為大都督大將軍大單于河南王【時乞伏氏跨有涼州河南之地遂為國號為後公府殺乾歸張本單音蟬】大赦改元太初 魏王珪破庫莫奚於弱落水南【新唐書曰奚亦東胡種為匈奴所破保烏丸山漢曹操斬蹋頓盖其後也弱落水即饒樂水在奚中】秋七月庫莫奚復襲魏營【復扶又翻】珪又破之庫莫奚者本屬宇文部與契丹同類而異種其先皆為燕王皝所破徙居松漠之間【契丹國自西樓東去四十里至真珠塞又東行地勢漸高西望松林欎然數十里遂入平川契欺訖翻洪邁曰契丹之讀如喫惟新唐書有音種章勇翻】 秦後秦自春相持屢戰互有勝負至是各解歸關西豪傑以後秦久無成功多去而附秦 河南王乾歸立其妻邉氏為王后置百官倣漢制以南川侯出連乞都為丞相【出連亦以部落之號為氏】梁州刺史悌眷為御史大夫金城邉芮為左長史東秦州刺史祕宜為右長史【乞㐲氏置東秦州於安南】武始翟勍為左司馬【勍渠京翻】略陽王松夀為主簿從弟軻彈為梁州牧弟益州為秦州牧屈眷為河州牧【乞伏乾歸所置州牧不過分居河隴之間從才用翻】 八月秦主登立子崇為皇太子弁為南安王尚為北海王 燕護軍將軍平幼會章武王宙討吴深破之走保繹幕【繹幕縣自漢以來屬清河郡】 魏王珪隂有圖燕之志遣九原公儀奉使至中山燕主垂詰之曰【使疏吏翻詰去吉翻】魏王何以不自來儀曰先王與燕並事晉室世為兄弟【魏與燕皆鮮卑種也拓拔力微與慕容涉歸並事晉室】臣今奉使於理未失垂曰吾今威加四海豈得以昔日為比儀曰燕若不修德禮欲以兵威自彊此乃將帥之事【將即亮翻帥所類翻】非使臣所知也儀還言於珪曰燕主衰老太子闇弱范陽王自負材氣【是時慕容德在燕宗室中固自有與人不同者】非少主臣也【少詩照翻】燕主既沒内難必作【難乃旦翻】於時乃可圖也今則未可珪善之【為後魏攻燕張本】儀珪母弟翰之子也 九月河南王乾歸遷都金城 張申攻廣平王祖攻樂陵壬午燕高陽王隆將兵討之冬十月後秦主萇還安定秦主登就食新平帥衆萬餘圍萇營【帥讀曰率】四面大哭萇命營中哭以應之登乃退十二月庚子尚書令南康襄公謝石卒 燕太原王<br />
<br />
  楷趙王麟將兵會高陽王隆於合口【水經衡漳水過勃海建成縣又東左會呼沱别河故瀆又東北入清河謂之合口魏收地形志曰浮陽縣西接漳水衡水入焉今謂之合口】以擊張申王祖帥諸壘共救之【帥讀曰率】夜犯燕軍燕人逆擊走之隆欲追之楷麟曰王祖老賊或恐詐而設伏不如俟明隆曰此白地羣盜烏合而來徼幸一决【徼堅堯翻】非素有約束能壹其進退也今失利而去衆莫為用乘勢追之不過數里可盡擒也申之所恃唯在於祖祖破則申降矣【降戶江翻下同】乃留楷麟守申壘隆與平幼分道擊之比明【比必利翻及也】大獲而還懸所獲之首以示申甲寅申出降祖亦歸罪 秦以潁川王同成為太尉【同成秦主登之兄】十四年春正月燕以陽平王柔鎮襄國遼西王農在龍城五年庶務修舉乃上表曰臣頃因征即鎮【農誅餘巖擊高句麗因鎮龍城見上卷十年】所統將士安逸積年青徐荆雍遺寇尚繁【雍於用翻】願時代還展竭微效生無餘力没無遺恨臣之志也庚申燕主垂召農為侍中司隸校尉以高陽王隆為都督幽平二州諸軍事征北大將軍幽州牧建留臺於龍城以隆録留臺尚書事又以護軍將軍平幼為征北長史散騎常侍封孚為司馬【散悉亶翻騎奇寄翻】並兼留臺尚書隆因農舊規修而廣之遼碣遂安【遼碣謂遼水碣石碣其謁翻】後秦主萇以秦戰屢勝謂得秦王堅之神助亦於軍中立堅像而禱之曰臣兄襄敕臣復讎【穆帝升平元年姚襄為秦所殺】新平之禍【見上卷十年】臣行襄之命非臣罪也苻登陛下疎屬猶欲復讎况臣敢忘其兄乎且陛下命臣以龍驤建業【見一百五卷八年驤思將翻】臣敢違之今為陛下立像【為于偽翻】陛下勿追計臣過也秦主登升樓遥謂萇曰為臣弑君而立像求福庸有益乎因大呼曰【呼火故翻】弑君賊姚萇何不自出吾與汝決之萇不應久之以戰未有利軍中每夜數驚【數所角翻】乃斬像首以送秦 秦主登以河王乾歸為大將軍大單于金城王【單音蟬】 甲寅魏王珪襲高車破之二月呂光自稱三河王【呂光字世明光時有涼州河西之地未能兼有三河也】大赦改元麟嘉置百官光妻石氏子紹弟德世自仇池來至姑臧【長安之亂呂光之家奔仇池依楊氏】光立石氏為妃紹為世子癸巳魏王珪擊吐突隣部於女水【女水在弱落水西去平城三千餘里後魏顯祖改曰武川】大破之盡徙其部落而還 秦主登留輜重於大界【大界當在安定新平之間重直用翻】自將輕騎萬餘攻安定羌密造保克之【將即亮翻騎奇寄翻保當作堡】 夏四月翟遼寇滎陽執太守張卓 燕以長樂公盛鎮薊城脩繕舊宫【燕主雋初自龍城徙都薊有舊宫在焉樂音洛薊音計】五月清河民孔金斬吴深送首中山【吴深反事始上卷十一年】 金城王乾歸擊侯年部大破之於是秦涼鮮卑羌胡多附乾歸乾歸悉授以官爵 後秦主萇與秦主登戰數敗【數所角翻】乃遣中軍將軍姚崇襲大界登邀擊之於安丘【魏收地形志安定隂盤縣有安城】又敗之【敗補邁翻】 燕范陽王德趙王麟擊賀訥追犇至勿根山訥窮迫請降【降戶江翻】徙之上谷質其弟染干於中山【質音致】 秋七月以驃騎長史王忱為荆州刺史都督荆益寧三州諸軍【驃匹妙翻騎奇寄翻忱是壬翻】忱國寶之弟也 秦主登攻後秦右將軍吳忠等於平凉克之八月登據苟頭原以逼安定諸將勸後秦主萇决戰萇曰與窮寇競勝兵家之忌也吾將以計取之乃留尚書令姚旻守安定夜帥騎三萬【師讀曰率下同】襲秦輜重于大界克之【重戰輕防此苻登所以敗也重直用翻】殺毛后及南安王尚【尚秦主登之子也】擒名將數十人驅掠男女五萬餘口而還毛氏美而勇善騎射後秦兵入其營毛氏猶彎弓跨馬帥壮士數百人戰衆寡不敵為後秦所執萇將納之毛氏罵且哭曰姚萇汝先已殺天子【謂殺秦王堅也】今又欲辱皇后皇天后土寧汝容乎萇殺之諸將欲因秦軍駭亂擊之萇曰登衆雖亂怒氣猶盛未可輕也遂止【兵勝者驕兵怒者奮以奮乘驕則先敗而後勝者多矣姚萇見兵勢所以收衆而止】登收餘衆屯胡空堡萇使姚碩德鎮安定徙安定千餘家于隂密遣其弟征南將軍靖鎮之 九月庚午以左僕射陸納為尚書令 秦主登之東也後秦主萇使姚碩德置秦州守宰以從弟常戍隴城【從才用翻隴城縣漢屬天水郡晉省此時當屬畧陽郡】邢奴戍冀城姚詳戍畧陽楊定攻隴冀克之斬常執邢奴詳棄略陽犇隂密定自稱秦州牧隴西王秦因其所稱而授之 冬十月秦主登以竇衝為大司馬都督隴東諸軍事雍州牧【雍於用翻】楊定為左丞相都督中外諸軍事秦梁二州牧約共攻後秦又約監河西諸軍事并州刺史楊政都督河東諸軍事冀州刺史楊楷各帥其衆會長安【帥讀曰率下同】政楷皆河東人秦主丕既敗政楷收集流民數萬戶政據河西楷據湖陕之間遣使請命於秦登因而授之【大界既䧟苻登之兵勢衰矣故約竇衝等共攻後秦陕式冉翻使疏吏翻】燕樂浪悼王温為冀州刺史【燕冀州刺史治信都樂浪音洛琅】翟遼遣丁零故堤詐降於温帳【何承天姓苑有故姓帳謂帳下】乙酉刺温殺之【刺七亦翻】并其長史司馬驅帥守兵二百戶犇西燕燕遼西王農邀擊刺温者於襄國盡獲之惟堤走免 十一月枹罕羌彭奚念附於乞伏乾歸以奚念為北河州刺史【枹罕舊為河州治所乞伏氏先于境内置河州以屈眷為牧故以枹罕為北河州以奚念為刺史枹音膚】 初帝既親政事【太元元年崇德太后歸政帝始親政事】威權已出有人主之量已而溺於酒色委事於琅邪王道子道子亦嗜酒日夕與帝以酣歌為事又崇尚浮屠窮奢極費所親暱者皆姏姆僧尼【暱尼質翻姏姑三翻老女稱姆莫補翻女師也又音茂】左右近習爭弄權柄交通請託賄賂公行官賞濫雜刑獄謬亂尚書令陸納望宫闕歎曰好家居纎兒欲撞壞之邪【撞丈降翻纎者小之至言為纎兒謂不及小兒也】左衛領營將軍會稽許營【以左衛將軍領營兵是為左衛領營將軍許營一作榮】上疏曰今臺府局吏直衛武官及僕隸婢兒取母之姓者【官婢私合而生子不能審知其父取母之姓為姓】本無鄉邑品第皆得為郡守縣令或帶職在内及僧尼乳母競進親黨又受貨賂輒臨官領衆政教不均㬥濫無罪禁令不明刼盜公行昔年下書敇羣下盡規而衆議兼集無所採用臣聞佛者清遠玄虚之神今僧尼往往依傍灋服【傍步浪翻謂依傍佛法服僧尼之服而不遵其教也】五誡麄灋尚不能遵况精妙乎【佛有五戒不淫不盜不殺不妄語不遭酒敗】而流惑之徒競加敬事又侵漁百姓取財為惠亦未合布施之道也【施式吏翻】疏奏不省【省悉景翻】道子勢傾内外遠近犇湊帝漸不平然猶外加優崇侍中王國寶以讒佞有寵於道子扇動朝衆【朝直遥翻】諷八座啟道子宜進位丞相揚州牧假黄鉞加殊禮【晉氏渡江有吏部祠部五兵左民度支五尚書二僕射一令為八坐】護軍將軍南平車胤曰【沈約曰秦時有護軍都尉漢陳平為護軍中尉盡護諸將李廣為驍騎將軍屬護軍將軍盖護軍護諸將軍魏武以韓浩為護軍資重為護軍將軍資輕為中護軍車尺遮翻】此乃成王所以尊周公也今主上當陽【人主南面鄉明而立以治天下故曰當陽】非成王之比相王在位豈得為周公乎乃稱疾不署【不署名也】疏奏帝大怒而嘉胤有守中書侍郎范甯徐邈為帝所親信數進忠言【數所角翻】補正闕失指斥姦黨王國寶甯之甥也甯尤疾其阿諛勸帝黜之陳郡袁悦之有寵於道子國寶使悦之因尼妙音致書於太子母陳淑媛云國寶忠謹宜見親信帝知之發怒以他事斬悦之國寶大懼與道子共譛范甯出為豫章太守甯臨發上疏言今邉烽不舉而倉庫空匱古者使民歲不過三日【記王制古者用民之力歲不過三日任老者之事食北者之食】今之勞擾殆無三日之休至有生兒不復舉養【復扶又翻】鰥寡不敢嫁娶厝火積薪不足喻也【厝火積薪賈誼之言】甯又上言中原士民流寓江左歲月漸久人安其業凡天下之人原其先祖皆隨世遷移何至於今而獨不可謂宜正其封疆戶口皆以土斷【晉時中原士民南渡者皆於江左僑立郡縣以居之不以土著為斷斷丁亂翻】又人性無涯奢儉由勢今并兼之室亦多不贍非其財力不足盖由用之無節爭以靡麗相高無有限極故也禮十九為長殤以其未成人也【未成人而死曰殤其喪禮殺於成人長知兩翻】今以十六為全丁十三為半丁所任非復童幼之事【任音壬】豈不傷天理困百姓乎謂宜以二十為全丁十六為半丁則人無夭折生長繁滋矣【夭於兆翻長知兩翻】帝多納用之甯在豫章遣十五議曹下屬城【豫章領南昌海昏新塗建成望蔡永修建昌吳平豫章彭澤艾康樂豐城新昌宜豐鍾陵十六縣一縣負郭餘十五縣各遣一議曹下遐稼翻】採求風政并吏假還訊問官長得失【假居訝翻假還謂吏休假日滿而還府者】徐邈與甯書曰足下聽斷明允庶事無滯則吏慎其負【負謂罪也吏畏罪則每事加謹斷丁亂翻】而人聽不惑矣【人聽即民聽晉書史臣避唐太宗諱改民為人通鑑因之】豈須邑至里詣飾其游聲哉非徒不足致益寔乃蠶漁之所資【蠶漁謂所遣者蠶食漁取於民鄭玄曰趙魏之東寔實同聲寔是也詩實墉實壑實畝實籍實當作寔言韓侯之先祖微弱所受之國多滅絶今復舊職繼絶世故築治是城修是壑井牧是田畝收歛是賦稅使如故常孔穎達曰凡言實者已有其事而後實之今此方說所為不宜為實故轉為寔訓之為是也趙魏之東寔實同聲鄭以時事驗之也春秋桓六年州公寔來是由聲同故字有變異也余按徐邈所謂寔訓之為是於義亦通】豈有善人君子而干非其事多所告白者乎自古以來欲為左右耳目無非小人皆先因小忠而成其大不忠先藉小信而成其大不信遂使讒謟並進善惡倒置可不戒哉足下慎選綱紀【郡以僚佐為綱紀】必得國士以攝諸曹諸曹皆得良吏以掌文按【攝總也整也按據也文按謂諸曹文書留為按據者】又擇公方之人以為監司【監工銜翻】則清濁能否與事而明足下但平心處之【處昌呂翻】何取於耳目哉昔明德馬后未嘗顧左右與言【漢明帝后馬氏諡明德皇后】可謂遠識況大丈夫而不能免此乎 十二月後秦主萇使其東門將軍任瓫【東門將軍萇使守安定東門者也任音壬瓫與盆同音蒲奔翻】詐遣使招秦主登許開門納之【使疏吏翻】登將從之征東將軍雷惡地將兵在外【將即亮翻】聞之馳騎見登【騎奇寄翻】曰姚萇多詐不可信也登乃止萇聞惡地詣登謂諸將曰此羌見登事不成矣登以惡地勇略過人隂憚之惡地懼降於後秦【降戶江翻】萇以惡地為鎮軍將軍 秦以安成王廣為司徒<br />
<br />
  十五年春正月乙亥譙敬王恬薨 西燕主永引兵向洛陽朱序自河隂北濟河擊敗之【敗補邁翻】序追至白水【水經注白水出上黨高都縣故城西東流歷天井關序所至處去長子一百六十里】會翟遼謀向洛陽序乃引兵還擊走之留膺揚將軍朱黨戍石門使其子畧督護洛陽以參軍趙蕃佐之身還襄陽琅琊王道子恃寵驕恣侍宴酣醉或虧禮敬帝益不能平欲選時望為藩鎮以潜制道子問於太子左衛率王雅曰吾欲用王恭殷仲堪何如雅曰王恭風神簡貴志氣方嚴仲堪謹於細行【行下孟翻】以文義著稱然皆峻狭自是且幹略不長若委以方面天下無事足以守職若其有事必為亂階矣帝不從【為後王殷稱兵張本】恭藴之子【王藴后父也】仲堪融之孫也【殷融見九十六卷成帝咸康五年】二月辛巳以中書令王恭為都督青兖幽并冀五州諸軍事兖青二州刺史鎮京口三月戊辰大赦 後秦主萇攻秦扶風太守齊益男<br />
<br />
  於新羅堡克之益男走秦主登攻後秦天水太守張業生於隴東【隴東安定涇陽縣之地】萇救之登引去 夏四月秦鎮東將軍魏掲飛自稱衝天王【晉書載紀作魏褐飛太元元年秦遣庭中將軍魏曷飛擊氐羌意即此人也】帥氐胡攻後秦安北將軍姚當成於杏城【帥讀曰率】鎮軍將軍雷惡地叛應之攻鎮東將軍姚漢得於李潤【李潤地名在邢望李延夀曰馮翊東有李潤鎮按魏書宗室列傳安定王變除華州刺史表曰謹惟州居李潤堡雖是少梁舊地晉芮錫壤然胡夷内附遂為戎落請徙馮翊古城】後秦主萇欲自擊之羣臣皆曰陛下不憂六十里苻登【時登趣長安據新豐之千戶固】乃憂六百里魏掲飛何也萇曰登非可猝滅吾城亦非登所能猝拔惡地智畧非常若南引掲飛東結董成【董成屠各種也時據北地】得杏城李潤而據之長安東北非吾有也乃潜引精兵一千六百赴之掲飛惡地有衆數萬氐胡赴之者前後不絶萇每見一軍至輒喜羣臣怪而問之萇曰掲飛等扇誘同惡種類甚繁【種章勇翻】吾雖克其魁帥【帥所類翻】餘黨未易猝平【易以䜴翻】今烏集而至吾乘勝取之可一舉無餘也【此曹操取馬超韓遂故智耳】掲飛等見後秦兵少【少詩沼翻】悉衆攻之萇固壘不戰示之以弱潜遣其子中軍將軍崇帥騎數百出其後掲飛兵擾亂萇遣鎮遠將軍王超等縱兵擊之斬掲飛及其將士萬餘級惡地請降【降戶江翻】萇待之如初惡地謂人曰吾自謂智勇傑出一時而每遇姚翁輒困固其分也【分扶問翻史言姚萇能服雷惡地之心】萇命姚當成於所營之地每柵孔中輒樹一木以旌戰功【掘地作孔豎木以為柵故有柵孔】歲餘問之當成曰營地太小已廣之矣萇曰吾自結髪以來與人戰未嘗如此之快以千餘兵破三萬之衆營地惟小為奇豈以大為貴哉 吐谷渾視連遣使獻見於金城王乾歸乾歸拜視連沙州牧白蘭王【河西張茂以燉煌晉昌西域都護校尉玉門大護軍三郡三營為沙州吐谷渾未能有其地也李延夀曰此以吐谷渾部内有黄沙周迴數百里不生草木因號沙州使疏吏翻見賢遍翻】 丙寅魏王珪會燕趙王麟於意辛山【意辛山在牛川北賀蘭部所居也據北史踰隂山而北即賀蘭部】擊賀蘭紇突鄰紇奚三部破之紇突鄰紇奚皆降於魏【史言燕為魏驅除降戶江翻】 秋七月馮翊人郭質起兵於廣鄉以應秦【魏收地形志鄭縣有廣鄉原鄭縣時屬京兆】移檄三輔曰姚萇凶虐毒被神人【被皮義翻】吾屬世蒙先帝堯舜之仁【先帝謂秦主堅】非常伯納言之子【常伯侍中也納言尚書也】即卿校牧守之孫也【校戶教翻】與其含恥而存孰若蹈道而死於是三輔壁壘皆應之獨鄭縣人苟曜聚衆數千附於後秦秦以質為馮翊太守後秦以曜為豫州刺史【苟曜後持兩端為後秦所殺事見後】劉衛辰遣子直力鞮攻賀蘭部賀訥困急謂降於魏丙子魏王珪引兵救之直力鞮退【鞮田黎翻】珪徙訥部落處之東境【處昌呂翻】 八月劉牢之擊翟釗於鄄城釗走河北又敗翟遼於滑臺張願來降【翟遼張願叛見上卷十一年走音奏敗補邁翻鄄音絹】九月北平人吴柱聚衆千餘立沙門灋長為天子破北平郡轉寇廣都入白狼城【白狼縣前漢屬右北平郡後漢晉省魏收地形志後漢真君八年置建德郡治白狼城廣都縣屬焉燕時當屬北平郡】燕幽州牧高陽王隆方葬其夫人郡縣守宰皆會之衆聞柱反請隆還城遣大兵討之隆曰今閭閻安業民不思亂柱等以詐謀惑愚夫誘脅相聚無能為也【誘音酉】遂留葬訖遣廣平太守廣都令先歸【廣平當作北平】續遣安昌侯進將百餘騎趨白狼城【趨七喻翻】柱衆聞之皆潰窮捕斬之 以侍中王國寶為中書令俄兼中領軍【道子主之也】 丁未以吴郡太守王珣為尚書右僕射 吐谷渾視連卒子視羆立視羆以其父祖慈仁為四鄰所侵侮【吐谷渾辟奚視連慈仁見一百三卷簡文帝咸安元年】乃督厲將士欲建功業冬十月金城王乾歸遣使拜視羆沙州牧白蘭王視羆不受【為後乞伏乾歸伐吐谷渾張本】 十二月郭質及苟曜戰于鄭東質敗犇洛陽【鄭東鄭縣之東也】越質詰歸據平襄叛金城王乾歸【十三年越質詰歸附于乞伏氏】<br />
<br />
  十六年春正月燕置行臺於薊加長樂公盛録行臺尚書事【薊音計樂音洛】 金城王乾歸擊越質詰歸詰歸降【降戶江翻】乾歸以宗女妻之【妻七細翻】 賀染干謀殺其兄訥訥知之舉兵相攻魏王珪告于燕請為鄉導以討之【鄉讀曰嚮】二月甲戌燕主垂遣趙王麟將兵擊訥鎮北將軍蘭汗帥龍城之兵擊染干【賀染干部落盖居賀蘭部之東偏故燕以龍城之兵擊之將即亮翻帥讀曰率】三月秦主登自雍攻後秦安東將軍金榮于范氏堡<br />
<br />
  克之【雍於用翻】遂渡渭水攻京兆太守韋範于段氏堡不克進據曲牢【曲牢在杜縣東北】 夏四月燕蘭汗破賀染干於牛都【都聚也其地當在牛川夷人放牧於此聚會因名】苟曜有衆一萬密召秦主登許為内應登自曲牢向繁川【繁川盖即杜陵縣之樊川也】軍于馬頭原五月後秦主萇引兵逆戰登擊破之斬其右將軍吳忠萇收衆復戰【復扶又翻下同】姚碩德曰陛下慎於輕戰每欲以計取之今戰失利而更前逼賊何也萇曰登用兵遲緩不識虚實今輕兵直進遥據吾東【馬頭原之地盖在長安東】此必苟曜豎子與之有謀也【善用兵者觀敵之動而察知其情是以能制勝】緩之則其謀得成故及其交之未合急擊之以敗散其事耳【敗補邁翻】遂進戰大破之登退屯於郿【郿音媚今音眉】 秦兖州刺史強金槌據新平降後秦以其子逵為質【強其兩翻降戶江翻質音致】後秦主萇將數百騎入金槌營羣丁諫之萇曰金槌既去苻登又欲圖我將安所歸乎且彼初來欵附宜推心以結之奈何復以不信疑之乎【復扶又翻】既而羣氐欲取萇金槌不從【強金槌氐種秦之戚黨也姚萇推心待之以攜符登之黨】 六月甲辰燕趙王麟破賀訥於赤城禽之【水經河水自雲中楨陵縣南過赤城東又南過定襄桐過縣西又魏書帝紀登國三年幸東赤城明元泰常八年築長城於長川之南起自赤城西至五原延袤二千餘里】降其部落數萬【降戶江翻】燕主垂命麟歸訥部落徙染干於中山麟歸言於垂曰臣觀拓跋珪舉動終為國患不若攝之還朝使其弟監國事垂不從【攝録也收也慕容麟之姦詐知拓跋珪之終不可制而慕容垂不從其言天將啟珪以滅燕雖以垂之明畧不知覺也監工銜翻】西燕主永寇河南太守楊佺期擊破之 秋七月壬申燕主垂如范陽【范陽縣漢屬涿郡魏文帝改涿郡為范陽郡】 魏王珪遣其弟觚獻見於燕【見賢遍翻】燕主垂衰老子弟用事留觚以求良馬魏王珪弗與遂與燕絶【為燕魏搆難張本】使長史張衮求好於西燕【好呼到翻】觚逃歸燕太子寶追獲之垂待之如初秦主登攻新平後秦主萇救之登引去 秦驃騎將<br />
<br />
  軍沒奕干【驃匹妙翻騎奇寄翻】以其二子為質於金城王乾歸【質音致】請共擊鮮卑大兜乾歸與沒奕干攻大兜於鳴蟬堡克之【據載記大兜時據安陽城安陽城在唐秦州隴城縣界鳴蝉堡亦當在其地】兜微服走乾歸收其部衆而還【還從宣翻又如字】歸沒奕干二子沒奕干尋叛東合劉衛辰八月乾歸帥騎一萬討沒奕干【帥讀曰率】沒奕干犇他樓城【他樓城在高平唐太宗貞觀六年以突厥降戶置緣州治平高之他樓城高宗置他樓縣後入原州蕭關縣界】乾歸射之中目【射而亦翻中竹仲翻】 九月癸未以尚書右僕射王珣為左僕射太子詹事謝琰為右僕射太學博士范弘之論殷浩宜加贈諡因叙桓温不臣之迹是時桓氏猶盛王珣温之故吏也【王珣先為温府主簿】以為温廢昏立明有忠貞之節黜弘之為餘杭令【餘杭縣漢屬會稽郡顧來曰縣秦始皇立後漢分屬吳郡吳分屬吳興郡】弘之汪之孫也【范汪得罪於桓温見一百一卷哀帝升平五年】 冬十月壬辰燕主垂還中山【自范陽還也】 初柔然部人世服於代 【魏收曰神元之末掠騎得一奴髪始齊肩忘本姓名其主字之曰木骨閭木骨閭者首禿也木骨閭與郁久閭聲相近故後子孫因以為氏木骨閭既壮免奴為騎卒穆帝時坐後期當斬亡匿廣漠谿谷間收合逋逃得百餘人依紇突鄰木骨閭死子車鹿會雄健始有部衆自號柔然】其大人郁久閭地粟袁卒部落分為二長子匹候跋繼父居東邉次子緼紇提别居西邉【長知兩翻緼於粉翻紇戶骨翻】秦主堅滅代【滅代見一百四卷元年】柔然附於劉衛辰及魏王珪即位攻擊高車等諸部率皆服從獨柔然不事魏戊戌珪引兵擊之柔然舉部遁走珪追犇六百里諸將因張衮言於珪曰賊遠糧盡不如早還珪問諸將若殺副馬為三日食足乎【凡北人用騎兵各乘一馬又有一馬為副馬】皆曰足乃復倍道追之及於大磧南牀山下【是時魏盛跨有代北柔然西犇南牀山盖在大磧之西北史帝紀作南商山復扶又翻】大破之虜其半部匹跋及别部帥屋擊各收餘衆遁走【帥所類翻】珪遣長孫嵩長孫肥追之珪謂將佐曰卿曹知吾前問三日糧意乎曰不知也珪曰柔然驅畜產犇走數日至水必留我以輕騎追之計其道里不過三日及之矣皆曰非所及也嵩追斬屋擊於平望川肥追匹跋至涿邪山匹跋舉衆降【降戶江翻下同】獲緼紇提之子曷多汗【汗音寒】兄子社崘斛律等宗黨數百人【崘盧昆翻】緼紇提將奔劉衛辰珪追及之緼紇提亦降珪悉徙其部衆於雲中【為社崘復叛去而建國張本】 翟遼卒子釗代立改元定鼎攻燕鄴城燕遼西王農擊却之【為燕滅翟釗張本】三河王光遣兵乘虚伐金城王乾歸【乘其伐沒奕干之虚也】乾歸聞之引兵還光兵亦退 劉衛辰遣子直力鞮帥衆八九萬攻魏南部【鞮田黎翻又丁奚翻帥讀曰率】十一月己卯魏王珪引兵五六千人拒之壬午大破直力鞮於鐵岐山南直力鞮單騎走【騎奇寄翻下同】乘勝追之戊子自五原金津南濟河【金津當在五原郡宜梁九原二縣間】徑入衛辰國衛辰部落駭亂辛卯珪直抵其所居悦跋城【考之載記悦跋城即代來城也】衛辰父子出走壬辰分遣諸將輕騎追之將軍伊謂禽直力鞮於木根山【魏書官氏志拓跋鄰以次弟為伊婁氏後改為伊氏木根山在五原河西】衛辰為其部下所殺十二月珪軍于鹽池【漢地理志五原郡成宜縣有鹽官唐鹽州五原縣有烏白等池鹽宋白曰青白鹽池在鹽州北】誅衛辰宗黨五千餘人皆投尸于河【報元年衛辰藉兵於秦以滅代之怨也】自河以南諸部悉降【降戶江翻】獲馬三十餘萬匹牛羊四百餘萬頭國用由是遂饒衛辰少子勃勃亡犇薛于部【少詩照翻】珪使人求之薛于部帥太悉仗出勃勃以示使者曰【帥所類翻使疏吏翻】勃勃國破家亡以窮歸我我寧與之俱亡何忍執以與魏乃送勃勃於沒奕干沒奕干以女妻之【為勃勃殺沒奕干復建國張本妻七細翻】  戊申燕主垂如魯口 秦主登攻安定後秦主萇如隂密以拒之謂太子興曰苟曜聞吾北行【自長安如隂密為北行】必來見汝汝執誅之曜果見興於長安興使尹緯讓而誅之【善制敵者能因事而為功苟曜反覆於苻姚之間而長安去鄭二百里耳此姚氏腹脇之癰疽也使萇召之曜必不來萇在長安曜亦畏憚而不敢來萇外出以誘之曜亦疑而不敢來二秦交兵邉遽狎至萇之北行若不得已者苟曜無疑畏之心謂姚興居守為無能為者輕於一來卒以送死姚氏腹脇之疾去矣此非能因事而為功乎】萇敗登於安定城東【敗補邁翻】登退據路承堡【路承人姓名築堡自守時因以為名】萇置酒高會諸將皆曰若值魏武王【姚萇僭號追諡兄襄為魏武王】不令此賊至今陛下將牢太過耳【將牢謂先自固而不妄動也猶今人之言把穩】萇笑曰吾不如亡兄有四身長八尺五寸【長直亮翻】臂垂過膝人望而畏之一也將十萬之衆與天下爭衡望麾而進前無横陳二也【將即亮翻陳讀曰陣】温古知今講論道藝收羅英雋三也董帥大衆上下咸悦人盡死力四也【帥讀曰率】所以得建立功業驅策羣賢者正望筭略中有片長耳羣臣咸稱萬歲<br />
<br />
  資治通鑑卷一百七<br />
<br />
<史部,編年類,資治通鑑>  <br>
   </div> 

<script src="/search/ajaxskft.js"> </script>
 <div class="clear"></div>
<br>
<br>
 <!-- a.d-->

 <!--
<div class="info_share">
</div> 
-->
 <!--info_share--></div>   <!-- end info_content-->
  </div> <!-- end l-->

<div class="r">   <!--r-->



<div class="sidebar"  style="margin-bottom:2px;">

 
<div class="sidebar_title">工具类大全</div>
<div class="sidebar_info">
<strong><a href="http://www.guoxuedashi.com/lsditu/" target="_blank">历史地图</a></strong>  
<a href="http://www.880114.com/" target="_blank">英语宝典</a>  
<a href="http://www.guoxuedashi.com/13jing/" target="_blank">十三经检索</a> 
<br><strong><a href="http://www.guoxuedashi.com/gjtsjc/" target="_blank">古今图书集成</a></strong> 
<a href="http://www.guoxuedashi.com/duilian/" target="_blank">对联大全</a> <strong><a href="http://www.guoxuedashi.com/xiangxingzi/" target="_blank">象形文字典</a></strong> 

<br><a href="http://www.guoxuedashi.com/zixing/yanbian/">字形演变</a>  <strong><a href="http://www.guoxuemi.com/hafo/" target="_blank">哈佛燕京中文善本特藏</a></strong>
<br><strong><a href="http://www.guoxuedashi.com/csfz/" target="_blank">丛书&方志检索器</a></strong> <a href="http://www.guoxuedashi.com/yqjyy/" target="_blank">一切经音义</a>  

<br><strong><a href="http://www.guoxuedashi.com/jiapu/" target="_blank">家谱族谱查询</a></strong>  <strong><a href="http://shufa.guoxuedashi.com/sfzitie/" target="_blank">书法字帖欣赏</a></strong> 
<br>

</div>
</div>


<div class="sidebar" style="margin-bottom:0px;">

<font style="font-size:22px;line-height:32px">QQ交流群9:489193090</font>


<div class="sidebar_title">手机APP 扫描或点击</div>
<div class="sidebar_info">
<table>
<tr>
	<td width=160><a href="http://m.guoxuedashi.com/app/" target="_blank"><img src="/img/gxds-sj.png" width="140"  border="0" alt="国学大师手机版"></a></td>
	<td>
<a href="http://www.guoxuedashi.com/download/" target="_blank">app软件下载专区</a><br>
<a href="http://www.guoxuedashi.com/download/gxds.php" target="_blank">《国学大师》下载</a><br>
<a href="http://www.guoxuedashi.com/download/kxzd.php" target="_blank">《汉字宝典》下载</a><br>
<a href="http://www.guoxuedashi.com/download/scqbd.php" target="_blank">《诗词曲宝典》下载</a><br>
<a href="http://www.guoxuedashi.com/SiKuQuanShu/skqs.php" target="_blank">《四库全书》下载</a><br>
</td>
</tr>
</table>

</div>
</div>


<div class="sidebar2">
<center>


</center>
</div>

<div class="sidebar"  style="margin-bottom:2px;">
<div class="sidebar_title">网站使用教程</div>
<div class="sidebar_info">
<a href="http://www.guoxuedashi.com/help/gjsearch.php" target="_blank">如何在国学大师网下载古籍?</a><br>
<a href="http://www.guoxuedashi.com/zidian/bujian/bjjc.php" target="_blank">如何使用部件查字法快速查字?</a><br>
<a href="http://www.guoxuedashi.com/search/sjc.php" target="_blank">如何在指定的书籍中全文检索?</a><br>
<a href="http://www.guoxuedashi.com/search/skjc.php" target="_blank">如何找到一句话在《四库全书》哪一页?</a><br>
</div>
</div>


<div class="sidebar">
<div class="sidebar_title">热门书籍</div>
<div class="sidebar_info">
<a href="/so.php?sokey=%E8%B5%84%E6%B2%BB%E9%80%9A%E9%89%B4&kt=1">资治通鉴</a> <a href="/24shi/"><strong>二十四史</strong></a>&nbsp; <a href="/a2694/">野史</a>&nbsp; <a href="/SiKuQuanShu/"><strong>四库全书</strong></a>&nbsp;<a href="http://www.guoxuedashi.com/SiKuQuanShu/fanti/">繁体</a>
<br><a href="/so.php?sokey=%E7%BA%A2%E6%A5%BC%E6%A2%A6&kt=1">红楼梦</a> <a href="/a/1858x/">三国演义</a> <a href="/a/1038k/">水浒传</a> <a href="/a/1046t/">西游记</a> <a href="/a/1914o/">封神演义</a>
<br>
<a href="http://www.guoxuedashi.com/so.php?sokeygx=%E4%B8%87%E6%9C%89%E6%96%87%E5%BA%93&submit=&kt=1">万有文库</a> <a href="/a/780t/">古文观止</a> <a href="/a/1024l/">文心雕龙</a> <a href="/a/1704n/">全唐诗</a> <a href="/a/1705h/">全宋词</a>
<br><a href="http://www.guoxuedashi.com/so.php?sokeygx=%E7%99%BE%E8%A1%B2%E6%9C%AC%E4%BA%8C%E5%8D%81%E5%9B%9B%E5%8F%B2&submit=&kt=1"><strong>百衲本二十四史</strong></a>  <a href="http://www.guoxuedashi.com/so.php?sokeygx=%E5%8F%A4%E4%BB%8A%E5%9B%BE%E4%B9%A6%E9%9B%86%E6%88%90&submit=&kt=1"><strong>古今图书集成</strong></a>
<br>

<a href="http://www.guoxuedashi.com/so.php?sokeygx=%E4%B8%9B%E4%B9%A6%E9%9B%86%E6%88%90&submit=&kt=1">丛书集成</a> 
<a href="http://www.guoxuedashi.com/so.php?sokeygx=%E5%9B%9B%E9%83%A8%E4%B8%9B%E5%88%8A&submit=&kt=1"><strong>四部丛刊</strong></a>  
<a href="http://www.guoxuedashi.com/so.php?sokeygx=%E8%AF%B4%E6%96%87%E8%A7%A3%E5%AD%97&submit=&kt=1">說文解字</a> <a href="http://www.guoxuedashi.com/so.php?sokeygx=%E5%85%A8%E4%B8%8A%E5%8F%A4&submit=&kt=1">三国六朝文</a>
<br><a href="http://www.guoxuedashi.com/so.php?sokeytm=%E6%97%A5%E6%9C%AC%E5%86%85%E9%98%81%E6%96%87%E5%BA%93&submit=&kt=1"><strong>日本内阁文库</strong></a> <a href="http://www.guoxuedashi.com/so.php?sokeytm=%E5%9B%BD%E5%9B%BE%E6%96%B9%E5%BF%97%E5%90%88%E9%9B%86&ka=100&submit=">国图方志合集</a> <a href="http://www.guoxuedashi.com/so.php?sokeytm=%E5%90%84%E5%9C%B0%E6%96%B9%E5%BF%97&submit=&kt=1"><strong>各地方志</strong></a>

</div>
</div>


<div class="sidebar2">
<center>

</center>
</div>
<div class="sidebar greenbar">
<div class="sidebar_title green">四库全书</div>
<div class="sidebar_info">

《四库全书》是中国古代最大的丛书,编撰于乾隆年间,由纪昀等360多位高官、学者编撰,3800多人抄写,费时十三年编成。丛书分经、史、子、集四部,故名四库。共有3500多种书,7.9万卷,3.6万册,约8亿字,基本上囊括了古代所有图书,故称“全书”。<a href="http://www.guoxuedashi.com/SiKuQuanShu/">详细>>
</a>

</div> 
</div>

</div>  <!--end r-->

</div>
<!-- 内容区END --> 

<!-- 页脚开始 -->
<div class="shh">

</div>

<div class="w1180" style="margin-top:8px;">
<center><script src="http://www.guoxuedashi.com/img/plus.php?id=3"></script></center>
</div>
<div class="w1180 foot">
<a href="/b/thanks.php">特别致谢</a> | <a href="javascript:window.external.AddFavorite(document.location.href,document.title);">收藏本站</a> | <a href="#">欢迎投稿</a> | <a href="http://www.guoxuedashi.com/forum/">意见建议</a> | <a href="http://www.guoxuemi.com/">国学迷</a> | <a href="http://www.shuowen.net/">说文网</a><script language="javascript" type="text/javascript" src="https://js.users.51.la/17753172.js"></script><br />
  Copyright &copy; 国学大师 古典图书集成 All Rights Reserved.<br>
  
  <span style="font-size:14px">免责声明:本站非营利性站点,以方便网友为主,仅供学习研究。<br>内容由热心网友提供和网上收集,不保留版权。若侵犯了您的权益,来信即刪。scp168@qq.com</span>
  <br />
ICP证:<a href="http://www.beian.miit.gov.cn/" target="_blank">鲁ICP备19060063号</a></div>
<!-- 页脚END --> 
<script src="http://www.guoxuedashi.com/img/plus.php?id=22"></script>
<script src="http://www.guoxuedashi.com/img/tongji.js"></script>

</body>
</html>
