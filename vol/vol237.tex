資治通鑑卷二百三十七
宋 司馬光 撰

胡三省 音註

唐紀五十三|{
	起柔兆閹茂盡屠維赤奮若六月凡三年有奇}


憲宗昭文章武大聖至神孝皇帝上之上

|{
	諱淳改為純順宗長子通鑑書唐諸帝諡號自玄宗已下皆以葬陵諡冊為正帝本諡曰聖神章武孝皇帝大中三年平河湟始追崇諡號曰昭文章武大聖至神孝皇帝中睿之後唯順憲宣有尊崇諡號故因而書之}


元和元年春正月丙寅朔上帥羣臣詣興慶宫上上皇尊號|{
	從百官之請也帥讀曰率上時掌翻}
丁卯赦天下改元 辛未以鄂岳觀察使韓臯為奉義節度使|{
	德宗貞元十九年名安黄軍曰奉義}
癸酉以奉義留後伊宥為安州刺史兼安州留後宥慎之子也壬午加成德節度使王士真同平章事 甲申上皇崩於興慶宫|{
	年四十六}
劉闢既得旌節|{
	去年以闢知西川節度見上卷}
志益驕求兼領三川上不許闢遂發兵圍東川節度使李康於梓州|{
	東川節度使領梓劍綿普陵榮遂合渝瀘等州治梓州梓州漢郪縣地劉禪置東廣漢郡梁武陵王紀置新州隋為梓州舊志梓州至京師二千九十里宋白曰梓州取梓潼江為名}
欲以同幕盧文若為東川節度使推官莆田林藴力諫闢舉兵|{
	武德五年分南安置莆田縣時屬泉州風俗通曰林姓林放之後孫湎曰周平王次子林開之後魯有林放林雍齊有林元}
闢怒械繫於獄引出將斬之隂戒行刑者使不殺但數礪刃於其頸|{
	數所角翻}
欲使屈服而赦之藴叱之曰豎子當斬即斬我頸豈汝砥石邪闢顧左右曰真忠烈之士也乃黜為唐昌尉|{
	儀鳳元年分九隴導江郫置唐昌縣屬彭州九域志在州西二十八里}
上欲討闢而重於用兵|{
	謂以用兵為重事不敢輕試也}
公卿議者亦以為蜀險固難取杜黄裳獨曰闢狂戇書生|{
	戇竹巷翻}
取之如拾芥耳臣知神策軍使高崇文勇畧可用願陛下專以軍事委之勿置監軍闢必可擒上從之翰林學士李吉甫亦勸上討蜀上由是器之|{
	器所以適用器之者知其可用}
戊子命左神策行營節度使高崇文將步騎五千為前軍|{
	考異曰實録云為左軍按有左必有右而云李元奕為次軍則崇文必前軍也}
神策京西行

營兵馬使李元奕將步騎二千為次軍與山南西道節度使嚴礪同討闢時宿將名位素重者甚衆皆自謂當征蜀之選及詔用崇文皆大驚|{
	高崇文雖不足以望韓信而亦能動時人之驚者所居之地然也}
上與杜黄裳論及藩鎮黄裳曰德宗自經憂患務為姑息不生除節帥|{
	帥所類翻}
有物故者先遣中使察軍情所與則授之中使或私受大將賂歸而譽之|{
	譽音余}
即降旄鉞未嘗有出朝廷之意者陛下必欲振舉綱紀宜稍以法度裁制藩鎮則天下可得而理也上深以為然於是始用兵討蜀以至威行兩河皆黄裳啓之也|{
	史言杜黄裳開憲宗削平藩鎮之畧其功不在裴度下}
高崇文屯長武城練卒五千常如寇至卯時受詔辰時即行器械糗糧|{
	糗去久翻熬米麥為糗}
一無所闕甲午崇文出斜谷|{
	斜昌遮翻谷音浴又如字}
李元奕出駱谷同趣梓州崇文軍至興元軍士有食於逆旅折人筯者崇文斬之以徇|{
	折而設翻}
劉闢䧟梓州執李康二月嚴礪拔劍州斬其刺史文德昭|{
	嚴礪先拔劍州故高崇文因以鼓行入蜀礪之功為不可揜矣宋白曰劍州漢廣漢之梓潼縣華陽國志云諸葛亮相蜀鑿石架空為飛閣以通蜀漢晉以其地入梓潼郡梁為安州西魏伐蜀先下安州因克成都改安州為始州唐先天二年改為劍州舊志劍州至京師一千六百六十二里}
奚王誨落可入朝丁酉以誨落可為饒樂郡王遣歸|{
	樂音洛}
癸丑加魏博節度使田季安同平章事 戊午上與宰相論自古帝王或勤勞庶政或端拱無為互有得失何為而可杜黄裳對曰王者上承天地宗廟下撫百姓四夷夙夜憂勤固不可自暇自逸然上下有分|{
	分扶問翻}
紀綱有叙苟慎選天下賢材而委任之有功則賞有罪則刑選用以公賞刑以信則誰不盡力何求不獲哉明主勞於求人而逸於任人此虞舜所以能無為而治者也|{
	孔子曰無為而治者其舜也歟治直吏翻}
至於獄市煩細之事各有司存非人主所宜親也昔秦始皇以衡石程書|{
	史記盧生曰始皇天性剛戾天下之事無小大皆决於主上至以衡石程書日夜有程不中程者不得休息}
魏明帝自案行尚書事|{
	見七十二卷太和六年行下孟翻}
隋文帝衛士傳飱|{
	事見一百九十三卷太宗貞觀四年}
皆無補於當時取譏於後來其耳目形神非不勤且勞也所務非其道也夫人主患不推誠人臣患不竭忠苟上疑其下下欺其上將以求理|{
	理治也}
不亦難乎上深然其言 三月丙寅以神策行營京西節度使范希朝為右金吾大將軍 高崇文引兵自閬州趣梓州|{
	九域志閬州西南至梓州三百餘里趣七喻翻}
劉闢將邢泚引兵遁去崇文入屯梓州闢歸李康於崇文以求自雪崇文以康敗軍失守斬之 |{
	考異曰劉崇遠金華子雜編曰高駢在淮海周寶在浙西為節度使相與有隙駢忽遣使悔叙離絶願復和好請境會於金山寶謂其使者曰我非李康更要作家門功勲欺誑朝廷邪注云元和中李康鎮東川傳有異志駢祖崇文鎮西川乃偽設鄰好康不防備來會於境為崇文所斬補國史曰劉闢舉兵下東蜀連帥李康弃城奔走崇文下劍閣日長子日暉不當矢石欲戮之以勵衆師次綿州斬李康疏康擅離征鎮不為拒敵注云當時議論云康任懷州刺史曰杖殺武陟尉即崇文判官宋君平之父乘此為之復讎按金華子言固不知李康為劉闢所圍事而云崇文誘誅之補國史又不知被擒事而云弃城走此皆得於傳聞不可為據今從舊傳}
丙子嚴礪奏克梓州丁丑制削奪劉闢官爵 初韓全義入朝以其甥楊惠琳知夏綏留後|{
	朝直遥翻夏戶雅翻}
杜黄裳以全義出征無功驕蹇不遜直令致仕|{
	事見上卷永貞元年}
以右驍衛將軍李演為夏綏節度使惠琳勒兵拒之表稱將士逼臣為節度使河東節度使嚴綬表請討之詔河東天德軍合擊惠琳綬遣牙將阿跌光進及弟光顔將兵赴之|{
	阿烏葛翻跌徒結翻}
光進本出河曲步落稽兄弟在河東軍皆以勇敢聞 |{
	考異曰舊李光進傳曰肅宗自靈武觀兵光進從郭子儀破賊收兩京上元初郭子儀為朔方節度用光進為都知兵馬使尋遷渭北節度使大歷四年葬母於京城南原將相致祭者凡四十四幄此乃李光弼弟光進事也而劉昫置之此傳下乃云元和四年范希朝救易定表光進為馬步都虞侯其疎謬如此}
辛巳夏州兵馬使張承金斬惠琳傳首京師 東川節度使韋丹至漢中表言高崇文客軍遠鬬無所資若與梓州綴其士心必能有功夏四月丁酉以崇文為東川節度副使知節度事 |{
	考異曰實録於此云為東川節度使至十月除西川時則云東川節度副使知節度事蓋此時誤也}
潘孟陽所至專事遊宴從僕三百人多納賄賂|{
	從才用翻}
上聞之甲辰以孟陽為大理卿罷其度支鹽鐵轉運副使|{
	潘孟陽出使見上卷上年}
丙午策試制舉之士|{
	歐陽脩曰唐選舉之制天子自詔曰制舉所以待非常之才焉}
於是校書郎元稹|{
	稹止忍翻}
監察御史獨孤郁校書郎下邽白居易前進士蕭俛沈傳師出焉郁及之子|{
	獨孤及見二百二十三卷代宗永泰元年}
俛華之孫|{
	蕭華見二百二卷肅宗上元二年}
傳師既濟之子也|{
	沈既濟見二百二十六卷代宗大歷十四年}
杜佑請解財賦之職仍舉兵部侍郎度支使鹽鐵轉運副使李巽自代丁未加佑司徒罷其鹽鐵轉運使以巽為度支鹽鐵轉運使自劉晏之後居財賦之職者莫能繼之巽掌使一年|{
	掌使言掌使職也使疏吏翻}
征課所入類晏之多明年過之又一年加一百八十萬緍|{
	然則李巽勝劉晏乎曰不如也晏猶有遺利在民巽則盡取之也}
戊申加隴右經畧使秦州刺史劉澭保義軍節度使|{
	鳳翔普潤縣先置隴右軍今改名保義軍澭於容翻又於用翻}
辛酉以元稹為左拾遺白居易為盩厔尉集賢校理蕭俛為右拾遺|{
	集賢校理開元八年置俛音免}
沈傳師為校書郎稹上疏論諫職 |{
	考異曰稹自叙及新傳先上教本書論諫職在後今從舊傳}
以為昔太宗以王珪魏徵為諫官宴遊寢食未嘗不在左右又命三品以上入議大政必遣諫官一人隨之以參得失|{
	見一百九十二卷太宗貞觀元年}
故天下大理|{
	大理猶言大治也}
今之諫官大不得豫召見次不得參時政排行就列朝謁而已|{
	行戶剛翻}
近年以來正牙不奏事|{
	德宗貞元十八年罷正牙奏事事見上卷}
庶官罷廵對|{
	廵對猶今云轉對貞元十七年令常參官每日引見二人訪以政事謂之廵對至元和元年武元衡奏曰正衙已有待制官兩員貞元七年又有次對難議兩置詔今後每坐日兩人待制正衙退後於延英候對中書門下御史臺官依故事並不待制則是自正衙待制以外凡德宗所置次對皆罷矣宋白曰貞元七年令常參官日二人引見謂之廵對二十一年御史中丞李鄘奏凖貞元七年勅常參官並令依次對者伏以朝夕承命已有待制官兩員足備顧問今更置次對恐煩聖聽勅宜停}
諫官能舉職者獨誥命有不便則上封事耳君臣之際諷諭於未形籌畫於至密尚不能回至尊之盛意况於既行之誥令已命之除授而欲以咫尺之書收絲綸之詔誠亦難矣|{
	記曰王言如絲其出如綸王言如綸其出如綍}
願陛下時於延英召對使盡所懷豈可寘於其位而屏棄疎賤之哉|{
	屏必郢翻又卑正翻}
頃之復上疏|{
	復扶又翻}
以為理亂之始必有萌象開直言廣視聽理之萌也甘謟諛蔽近習亂之象也自古人君即位之初必有敢言之士人君苟受而賞之則君子樂行其道小人亦貪得其利不為回邪矣|{
	元稹此二語蓋自道出心事也樂音洛}
如是則上下之志通幽遠之情逹欲無理得乎|{
	理治也與亂對言}
苟拒而罪之則君子卷懷括囊以保其身|{
	孔子曰邦無道則可卷而懷之易坤之六四曰括囊無咎無譽文言曰天地閉賢人隱易曰括囊無咎無譽蓋言謹也括結也方言云閉也}
小人阿意迎合以竊其位矣如是則十步之事皆可欺也欲無亂得乎昔太宗初即政孫伏伽以小事諫太宗喜厚賞之|{
	見一百九十五卷貞觀十二年}
故當是時言事者惟患不深切未嘗以觸忌諱為憂也太宗豈好逆意而惡從欲哉|{
	好呼到翻惡烏路翻}
誠以順適之快小而危亡之禍大故也陛下踐阼今以周歲|{
	以當作己}
未聞有受伏伽之賞者臣等備位諫列曠日彌年不得召見每就列位屏氣鞠躬不敢仰視又安暇議得失獻可否哉供奉官尚爾况疎遠之臣乎|{
	兩省官自遺補以上皆供奉官也屏卑郢翻}
此蓋羣下因循之罪也因條奏請次對百官復正牙奏事禁非時貢獻等十事稹又以貞元中王伾王叔文以伎術得幸東宫永貞之際幾亂天下|{
	伎渠綺翻幾居希翻}
上書勸上早擇修正之士使輔導諸子以為太宗自為藩王與文學清修之士十八人居|{
	事見一百八十九卷高祖武德四年}
後代太子諸王雖有僚屬日益疎賤至於師傅之官非眊瞶廢疾不任事者|{
	眊莫報翻目昏也瞶五怪翻耳聾也任音壬}
則休戎罷帥不知書者為之|{
	帥所類翻}
其友諭贊議之徒尤為冗散之甚|{
	按唐制王府有諮議參軍有友有文學元稹所謂友諭贊議者蓋謂友以諭教諮議則讚議也冗散之官今謂之閒慢差遣冗而隴翻散蘇旱翻}
搢紳皆恥由之就使時得僻老儒生越月踰時僅獲一見又何暇傅之德義納之法度哉夫以匹士愛其子猶知求明哲之師而教之况萬乘之嗣繫四海之命乎|{
	乘繩證翻}
上頗嘉納其言時召見之 壬戍邵王約薨|{
	約上弟也}
五月丙子以横海留後程執㳟為節度使 庚辰尚書左丞同平章事鄭餘慶罷為太子賓客 辛卯尊太上皇后為皇太后 劉闢城鹿頭關連八柵屯兵萬餘人以拒高崇文六月丁酉崇文擊敗之|{
	敗補邁翻}
闢置柵於關東萬勝堆戊戌崇文遣驍將范陽高霞寓攻奪之下瞰關城|{
	瞰古濫翻}
凡八戰皆捷 加盧龍節度使劉濟兼侍中己亥加平盧節度使李師古兼侍中 庚子高崇文破劉闢於德陽|{
	武德三年分雒縣置德陽縣屬漢州九域志在州東北八十五里}
癸卯又破之於漢州嚴礪遣其將嚴秦破闢衆萬餘人於綿州石碑谷|{
	九域志漢州綿竹縣有石碑鎮意州字蓋竹字之誤也}
初李師古有異母弟曰師道常疎斥在外不免貧寠|{
	寠其矩翻}
師古私謂所親曰吾非不友於師道也吾年十五擁節旄自恨不知稼穡之艱難况師道復減吾數歲|{
	復扶又翻}
吾欲使之知衣食之所自來且以州縣之務付之計諸公必不察也及師古疾篤師道時知密州事好畫及觱篥|{
	好呼到翻畫戶卦翻觱壁吉翻篥力質翻胡人吹葭管謂之觱篥樂府雜録觱篥葭管也卷蘆為頭截竹為管出於胡地制法角音九孔漏聲五音唐編入鹵簿各為笳管用之與樂以為管六竅之制則為鳳管旋宫轉器以應律者也杜佑曰觱篥一名悲篥出於胡中其聲悲東夷有卷桃皮為之者亦出南蠻又樂府雜録云觱篥本龜兹樂}
師古謂判官高沐李公度曰迨吾之未亂也|{
	迨及也疾病則亂}
欲有問於子我死子欲奉誰為帥乎二人相顧未對師古曰豈非師道乎人情誰肯薄骨肉而厚他人顧置帥不善則非徒敗軍政也|{
	帥所類翻下同敗蒲邁翻}
且覆吾族師道為公侯子孫不務訓兵理人專習小人賤事以為已能果堪為帥乎幸諸公審圖之閏月壬戌朔師古薨沐公度祕不發喪濳逆師道于密州奉以為節度副使 秋七月癸丑高崇文破劉闢之衆萬人於玄武|{
	劉昫曰玄武漢氐道地晉改曰玄武五代史志玄武舊曰伍城後周置玄武郡隋開皇初廢郡改縣曰玄武唐屬梓州九域志在州西九十里}
甲午詔凡西川繼援之兵悉取崇文處分|{
	處昌呂翻分扶問翻}
壬寅葬至德大聖大安孝皇帝于豐陵|{
	豐陵在京兆富平縣東三十里甕金山}
廟號順宗 八月壬戌以妃郭氏為貴妃 丁卯立皇子寧為鄧王寛為澧王宥為遂王察為深王寰為洋王寮為絳王審為建王|{
	此皆以當時州名為封國之名}
李師道總軍務久之朝命未至師道謀於將佐或請出兵掠四境高沐固止之請輸兩税申官吏行鹽法|{
	以表謹事朝廷不襲朝廷所為也}
遣使相繼奉表詣京師杜黄裳請乘其未定而分之上以劉闢未平己巳以師道為平盧留後知鄆州事 堂後主書滑渙久在中書|{
	堂後主書即今之堂後官滑戶八翻姓也}
與知樞密劉光琦相結宰相議事有與光琦異者令渙逹意常得所欲杜佑鄭絪等皆低意善視之鄭餘慶與諸相議事渙從旁指陳是非餘慶怒叱之未幾罷相四方賂遺無虛日|{
	幾居豈翻遺唯季翻}
中書舍人李吉甫言其專恣請去之|{
	去羌呂翻}
上命宰相闔中書四門搜掩|{
	去羌呂翻闔轄臘翻}
盡得其姦狀九月辛丑貶渙雷州司戶|{
	宋白曰雷州漢合浦郡之徐聞縣地梁分置合州大同末以合肥為合州以此為南合州唐改雷州}
尋賜死籍沒家財凡數千萬 壬寅高崇文又敗劉闢之衆於鹿頭關|{
	敗補邁翻}
嚴秦敗劉闢之衆於神泉|{
	神泉漢涪城地晉置西園縣隋改為神泉縣以縣西有泉能愈疾也唐屬綿州九域志在州西北八十五里}
河東將阿跌光顔將兵會高崇文於行營愆期一日|{
	愆過也愆期過期也}
懼誅欲深入自贖軍于鹿頭之西斷其糧道|{
	斷音短}
城中憂懼於是闢綿江柵將李文悦|{
	綿水在綿州雒縣東三十里源出綿竹縣紫巖山}
鹿頭守將仇良輔皆以城降於崇文獲闢壻蘇彊士卒降者萬計崇文遂長驅直指成都所向崩潰軍不留行辛亥克成都劉闢盧文若帥數十騎西奔吐蕃|{
	帥讀曰率}
崇文使高霞寓等追之及於羊灌田|{
	彭州有羊灌田守捉}
闢赴江不死擒之文若先殺妻子乃繫石自沈|{
	沈持林翻}
崇文入成都屯於通衢休息士卒市肆不驚珍貨山積秋豪不犯檻劉闢送京師斬闢大將邢泚館驛廵官沈衍 |{
	考異曰林恩補國史曰衍與段文昌闢逼令判案禮同上介亦接諸公候謁崇文目段公曰公必為將相未敢奉薦揖起沈衍令梟首標於驛門二人誅賞之異未曉其意何如也}
餘無所問軍府事無巨細命一遵韋南康故事|{
	韋臯封南康郡王}
從容指撝一境皆平|{
	從于容翻撝許為翻}
初韋臯以西山運糧使崔從知卭州事劉闢反從以書諫闢闢發兵攻之從嬰城固守闢敗乃得免從融之曾孫也|{
	崔融事武后以文華著}
韋臯參佐房式韋乾度獨孤密符載郗士美段文昌等素服麻屨銜土請罪崇文皆釋而禮之草表薦式等厚贐而遣之|{
	以貨財送行曰贐}
目段文昌曰君必為將相未敢奉薦載廬山人|{
	盧山在江州尋陽未嘗置縣恐誤}
式琯之從子|{
	房琯相肅宗}
文昌志玄之玄孫也|{
	段志玄唐初開國功臣}
闢有二妾皆殊色監軍請獻之崇文曰天子命我討平凶豎當以撫百姓為先遽獻婦人以求媚豈天子之意邪崇文義不為此乃以配將吏之無妻者|{
	史言高崇文受命專征有可稱者}
杜黄裳建議征蜀及指受高崇文方畧|{
	受當作授}
皆懸合事宜崇文素憚劉澭|{
	時京西諸鎮諸將劉澭持軍號為嚴整故崇文憚之}
黄裳使謂之曰若無功當以劉澭相代故能得其死力及蜀平宰相入賀上目黄裳曰卿之功也 辛巳詔徵少室山人李渤為左拾遺|{
	少室山在河南登封縣少詩沼翻}
渤辭疾不至然朝政有得失渤輒附奏陳論|{
	朝直遥翻}
冬十月甲子易定節度使張茂昭入朝 制割資簡陵榮昌瀘六州隸東川|{
	資州漢資中縣地隋置資陽郡唐為資州乾元二年分資瀘普合四州之境置昌州}
房式等未至京師皆除省寺官|{
	史言憲宗急於收拾人才以安反側}
丙寅以高崇文為西川節度使戊辰以嚴礪為東川節度使庚午以將作監柳晟為山南西道節度使晟至漢中府兵討劉闢還未至城|{
	府兵漢中之兵也唐以漢中為興元府故謂之府兵非唐初所謂府兵也}
詔復遣戍梓州軍士怨怒脅監軍謀作亂晟聞之疾驅入城慰勞之|{
	復扶又翻下可復同勞力到翻}
既而問曰汝曹何以得成功對曰誅反者劉闢耳晟曰闢以不受詔命故汝曹得以立功豈可復使他人誅汝以為功邪衆皆拜謝請詣戍所如詔書軍府由是獲安 壬申以平盧留後李師道為節度使 戊子劉闢至長安并族黨誅之武寧節度使張愔有疾上表請代十一月戊申徵愔為工部尚書以東都留守王紹代之|{
	王紹本名純避上名改焉}
復以濠泗二州隸武寧軍|{
	分濠泗二州見一百三十五卷德宗貞元十六年}
徐人喜得二州故不為亂 丙辰以内常侍吐突承璀為左神策中尉|{
	璀七罪翻}
承璀事上於東宫以幹敏得幸|{
	為承璀喪師其甚幾於亂國張本}
是歲回鶻入貢始以摩尼偕來於中國置寺處之|{
	回鶻之摩尼猶中國之僧也其教與天竺又異按唐書會要十九卷回鶻可汗王令明教僧進法入唐大歷三年六月二十九日勅賜回鶻摩尼為之置寺賜額為大雲光明六年正月勅賜荆洪越等州各置大雲光明寺一所唐史補卷蕃人常與摩尼僧議政京城為之立寺其法日晚乃食飲水茹葷而不食乳酪其大摩尼數年一度來往本國小者年轉唐史回鶻列傳元和初再朝獻始以摩尼至日晏乃食可汗常與共國也處昌呂翻}
其法日晏乃食食葷而不食湩酪|{
	葷許云翻辛臭菜也湩多貢翻乳汁也}
回鶻信奉之可汗或與議國事

二年春正月辛卯上祀圜丘赦天下 上以杜佑高年重德禮重之常呼司徒而不名佑以老疾請致仕詔令佑每月入朝不過再三因至中書議大政它日聼歸樊川|{
	杜佑治亭觀于樊川與賓客置酒為樂}
門下侍郎同平章事杜黄裳有經濟大畧而不修小節故不得久在相位乙巳以黄裳同平章事充河中晉絳慈隰節度使己酉以戶部侍郎武元衡為門下侍郎翰林學士李吉甫為中書侍郎並同平章事吉甫聞之感泣謂中書舍人裴垍曰吉甫流落江淮踰十五年|{
	德宗貞元七年寶參貶陸贄相疑吉甫黨於參貶明州長史至是為相凡十六年垍其冀翻}
一旦蒙恩至此思所以報德惟在進賢而朝廷後進罕所接識君有精鑒願悉為我言之|{
	為于偽翻}
垍取筆疏三十餘人數月之間選用畧盡當時翕然稱吉甫為得人 二月癸酉邕州奏破黄賊獲其酋長黄承慶|{
	黄賊西原洞蠻也酋慈由翻長知兩翻}
夏四月甲子以右金吾大將軍范希朝為朔方靈鹽節度使以右神策鹽州定遠兵隸焉|{
	定遠軍本屬靈州靈鹽接境相距三百里定遠軍在黄河北岸蓋分戍鹽州也又按宋白續通典左神策京西北八鎮普潤鎮崇信城定平鎖闕  歸化城定遠城永安城郃陽縣也右神策五鎮奉天鎮麟游鎮良原鎮慶州鎮懷遠城也今曰右神策豈懷遠兵歟鹽州前此得專奏事朝廷今復屬朔方}
以革舊弊任邊將也|{
	范希朝自宿衛出師故言以革任邊將之弊}
秋八月劉濟王士真張茂昭爭私隙迭相表請加罪戊寅以給事中房式為幽州成德義武宣慰使和解之|{
	宋白曰乾元元年戶部尚書李峘除都統淮南江東江西節度觀察宣慰處置使宣慰之名始此}
九月乙酉密王綢薨|{
	綢上弟也}
夏蜀既平|{
	夏楊惠琳蜀劉闢}
藩鎮惕息|{
	言惕惕危懼苟延氣息也}
多求入朝鎮海節度使李錡亦不自安求入朝上許之遣中使至京口慰撫且勞其將士|{
	勞力到翻}
錡雖署判官王澹為留後實無行意屢遷行期澹與勅使數勸諭之|{
	數所角翻}
錡不悦上表稱疾請至歲暮入朝上以問宰相武元衡曰陛下初即政錡求朝得朝求止得止可否在錡將何以令四海上以為然下詔徵之錡詐窮遂謀反王澹既掌留務|{
	掌留務者掌留後事務}
於軍府頗有制置錡益不平密諭親兵使殺之會頒冬服|{
	唐養兵之制有春衣冬衣}
錡嚴兵坐幄中澹與勅使入謁有軍士數百譟於庭曰王澹何人擅主軍務曳下臠食之大將趙琦出慰止又臠食之注刃於勅使之頸詬詈將殺之|{
	詬許候翻又苦候翻}
錡陽驚救之冬十月己未詔徵錡為左僕射以御史大夫李元素為鎮海節度使庚申錡表言軍變殺留後大將先是錡選腹心五人為所部五州鎮將|{
	先悉薦翻}
姚志安處蘇州李深處常州趙惟忠處湖州丘自昌處杭州高肅處睦州|{
	處昌呂翻下處置同}
各有兵數千伺察刺史動靜|{
	伺相吏翻}
至是錡各使殺其刺史遣牙將庾伯良將兵三千治石頭|{
	治直之翻修治也}
常州刺史顔防用客李雲計矯制稱招討副使斬李深傳檄蘇杭湖睦請同進討湖州刺史辛祕潛募鄉閭子弟數百夜襲趙惟忠營斬之蘇州刺史李素為姚志安所敗|{
	敗補邁翻}
生致於錡具桎梏釘於船舷|{
	釘丁定翻舷胡田翻船邊曰舷}
未及京口會錡敗得免乙丑制削李錡官爵及屬籍|{
	李錡宗室也故著於屬籍}
以淮南節度使王鍔統諸道兵為招討處置使徵宣武義寧武昌兵|{
	此時無義寧軍新書作武寧當從之鍔五各翻}
并淮南宣歙兵俱出宣州|{
	淮南兵與宣歙兵會于宣州界乘上流之勢以臨京口是時宣州之地北盡當塗至江滸}
江西兵出信州浙東兵出杭州以討之 高崇文在蜀朞年一旦謂監軍曰崇文河朔一卒|{
	高崇文本幽州人}
幸有功致位至此西川乃宰相回翔之地崇文叨居日久豈敢自安屢上表稱蜀中安逸無所陳力願效死邊陲 |{
	考異曰舊崇文傳曰崇文不通文字厭大府按牘諮稟之煩且以優富之地無所陳力乞居塞上以扞邊戍懇疏屢上舊武元衡傳曰崇文理軍有法而不知州縣之政上難其代者今從補國史參以舊傳}
上擇可以代崇文者而難其人丁卯以門下侍郎同平章事武元衡同平章事充西川節度使 |{
	考異曰孫光憲北夢瑣言曰李德裕太尉未出學院盛有詞藻而不樂應舉吉甫相俾親表勉之掌武曰好驢馬不入行由是以品子叙官也吉甫相以武相元衡同列事多不叶每退公詞色不懌掌武啓白曰此出之何難乃請修狄梁公廟於是武相漸求出鎮其智計已聞于早成矣今從實録及舊傳}
李錡以宣州富饒欲先取之遣兵馬使張子良李奉仙田少卿將兵三千襲之三人知錡必敗與牙將裴行立同謀討之行立錡之甥也故悉知錡之密謀三將營於城外將發召士卒諭之曰僕射反逆官軍四集常湖二將繼死其勢已蹙今乃欲使吾輩遠取宣城|{
	宣州宣城郡}
吾輩何為隨之族滅豈若去逆效順轉禍為福乎衆悦許諾即夜還趨城|{
	趨七喻翻下兵趨趨山同}
行立舉火鼓譟應之於内引兵趨牙門錡聞子良等舉兵怒聞行立應之撫膺曰吾何望矣跣走匿樓下親將李鈞引挽彊三百趨山亭欲戰行立伏兵邀斬之錡舉家皆哭左右執錡裹之以幕縋於城下|{
	縋馳偽翻}
械送京師挽彊蕃落爭自殺尸相枕藉|{
	錡養挽彊蕃落事見上卷德宗貞元十七年枕職任翻藉慈夜翻}
癸酉本軍以聞|{
	本軍為浙西軍}
乙亥羣臣賀於紫宸殿|{
	紫宸殿在宣政殿北}
上愀然曰|{
	愀七小翻}
朕之不德致宇内數有干紀者|{
	數所角翻}
朕之愧也何賀之為宰相議誅錡大功以上親|{
	大功謂從父兄弟姊妹以上則朞親也}
兵部郎中蔣乂曰錡大功親皆淮安靖王之後也|{
	淮安王神通諡曰靖}
淮安有佐命之功陪陵享廟|{
	神通起兵以應義師以功陪葬獻陵配享高祖廟廷}
豈可以末孫為惡而累之乎|{
	累力瑞翻}
又欲誅其兄弟乂曰錡兄弟故都統國貞之子也國貞死王事|{
	事見二百二十二卷肅宗寶應元年}
豈可使之不祀乎宰相以為然辛巳錡從父弟宋州刺史銛等皆貶官流放|{
	銛思亷翻}
十一月甲申朔錡至長安上御興安門|{
	唐大明宫南面五門興安門西來第一門也}
面詰之對曰臣初不反張子良等教臣耳上曰卿為元帥子良等謀反何不斬之然後入朝錡無以對乃并其子師回腰斬之 |{
	考異曰誅錡後數日上遣中使齎黄衣二襲命有司收其尸并男以庶人禮葬焉國史補曰錡之擒也得侍婢一人隨之錡夜則裂襟自書筦榷之功言為張子良所賣教侍婢曰結之於帶吾若從容奏對必當為宰相楊益節度不得從容當受極刑矣我死汝必入内上必問汝當以此進之及錡伏法京城大霧三日不解或聞鬼哭憲宗又得帛書頗疑其寃内出黄衣二襲賜錡及子敕京兆收葬按李錡驕逆何寃之有今從實録}
有司請毁錡祖考塜廟中丞盧坦上言李錡父子受誅罪已塞矣|{
	塞悉則翻}
昔漢誅霍禹不罪霍光|{
	誅霍禹見二十五卷漢宣帝地節四年}
先朝誅房遺愛不及房玄齡|{
	誅房遺愛見一百九十九卷高宗永徽四年}
康誥曰父子兄弟罪不相及|{
	左傳晉胥臣引康誥之辭今尚書康誥無有此語}
况以錡為不善而罪及五代祖乎乃不毁有司籍錡家財輸京師翰林學士裴垍李絳上言以為李錡僭侈割剥六州之人以富其家或枉殺其身而取其財|{
	六州潤睦常蘇湖杭也}
陛下閔百姓無告故討而誅之今輦金帛以輸上京恐遠近失望願以逆人資財賜浙西百姓代今年租賦上嘉歎久之即從其言 昭義節度使盧從史内與王士真劉濟潛通而外獻策請圖山東|{
	時魏博恒冀在太行山之東}
擅引兵東出上召令還從史託言就食邢洺不時奉詔久之乃還 |{
	考異曰蔣階李司空論事云絳奏從史比來事就彰露頗多意不自安務欲生事所以曲陳利害頻獻計謀冀許用兵以求姑息令請親領士馬欲往邢洺假以就糧實為動衆去就之際情狀可知舊從史傳曰前年丁父憂朝旨未議起復屬王士真卒從史竊獻誅承宗計以希上意用是起授委其成功及詔下討賊兵出逗留不進隂與承宗通謀令軍士潛懷賊號按三年九月戊戌李吉甫罷相出鎮掦州四年二月丁卯鄭絪罷相三月乙酉王士真卒承宗始襲位四月壬辰從史起復若以從史山東就糧即請討承宗之時則於是吉甫絪皆已罷相何得有譖絪之事又貶從史制辭云况近年上請就食山東及遣旋師不時㳟命致動其衆覬生其心賴劉濟抗忠正之辭使邪豎絶遲迴之計加以偏毁鄰境密疏事情反覆百端高下在手若是討承宗時朝廷不違其請何嘗使之旋師蓋李鄭未罷之前從史嘗毁鄰道乞加征討因擅引兵出山東朝廷命旋師託以就食邢洺不時奉詔但不知事在何年月日所欲攻討者何人劉濟有何辭而從史肯旋今因李絳論李錡家財事并言之新史云從史與承宗連和有詔歸潞誤也}
它日上召李絳對於浴堂|{
	唐禁中有浴堂殿德宗以來常居之沈括曰浴堂殿在翰林院北翰林院别設北扉以便於應召按舊書裴延齡傳德宗謂延齡曰朕所居浴堂院殿一栿以年多之故似有損蠧欲換之未能以此知德宗常居浴堂殿也程大昌曰沈氏謂學士院北扉為在浴堂之南便於應召此誤也學士院在紫宸蓬萊殿之西浴堂殿自在紫宸之東不在學士院南也唐學士多對浴堂殿李絳之極論中官柳公權之濡紙繼燭皆其地也然自六典以及呂圖皆無此一殿石林葉氏曰學士院北扉者浴堂之南便於應召此恐未審也學士院之北為翰林院翰林院之北為少陽院設或浴堂在此亦為寢殿三殿之所間隔不容有北門可以與之相屬也館本唐圖則有浴堂殿而殿之位置乃在綾綺殿南也綾綺者長安志曰在蓬萊殿東也而有學士院者自在蓬萊正西也東西既已相絶中間多有别殿無由有門可以相為南北也長安志嘗記浴堂門浴堂殿浴堂院矣且曰文宗嘗於此殿召對鄭注而於浴堂殿對學士焉又别有浴堂院亦同一處可以知其必在大明矣而不著其正在何地故予意館圖所記在綾綺殿南者是也而元稹承旨廳記又有可證者其說曰乘輿奉郊廟則承旨得乘厩馬自浴殿由内朝以從若外賓進見於麟德則止直禁中以俟夫内朝也者紫宸殿也唐之郊廟皆在都城之南人主有事郊廟若非自丹鳳門出必由承天門出决不向後迂出西銀臺門也則浴堂之可趨内朝也内朝之必趨丹鳳門也其理固已可必矣又謂殿在蓬萊殿東即與紫宸殿相屬又可必矣然則館圖位置其與元稹所記殆相發揮大可信也至於外賓客見于麟德則麟德並學士院東則不待班從而可居院以待也合二語以想事宜則浴堂也必在紫宸殿東而不在其西也}
語之曰事有極異者朕比不欲言之|{
	語牛倨翻比毗至翻}
朕與鄭絪議勅從史歸上黨續徵入朝絪乃泄之於從史使稱上黨乏糧就食山東為人臣負朕乃爾將何以處之|{
	處昌呂翻}
對曰審如此滅族有餘矣然絪從史必不自言陛下誰從得之上曰吉甫密奏絳曰臣竊聞搢紳之論稱絪為佳士恐必不然或者同列欲專朝政|{
	朝直遥翻}
疾寵忌前願陛下更熟察之勿使人謂陛下信讒也上良久曰誠然絪必不至此非卿言朕幾誤處分|{
	幾居希翻處昌呂翻分扶問翻}
上又嘗從容問絳曰|{
	從千容翻}
諫官多謗訕朝政皆無事實朕欲謫其尤者一二人以儆其餘何如對曰此殆非陛下之意必有邪臣欲壅蔽陛下之聰明者人臣死生繫人主喜怒敢發口諫者有幾就有諫者皆晝度夜思朝刪暮減比得上逹什無二三|{
	度徒洛翻比必利翻及也}
故人主孜孜求諫猶懼不至况罪之乎如此杜天下之口非社稷之福也上善其言而止 羣臣請上尊號曰睿聖文武皇帝丙申許之 盩厔尉集賢校理白居易作樂府及詩百餘篇規諷時事流聞禁中上見而悦之召入翰林為學士 十二月丙辰上謂宰相曰太宗以神聖之資羣臣進諫者猶往復數四况朕寡昧自今事有違卿當十論無但一二而已 丙寅以高崇文同平章事充邠寧節度京西諸軍都統|{
	新志曰天寶末置天下兵馬元帥都統朔方河東河北平盧節度使都統之名始於此}
山南東道節度使于頔憚上英威為子季友求尚主|{
	為于偽翻}
上以皇女普寧公主妻之|{
	普寧郡公主容州普寧郡妻七細翻}
翰林學士李絳諫曰頔虜族|{
	頔于謹之裔孫謹之先于栗磾本姓勿忸于氏從拓拔氏起於代北故絳云然}
季友庶孽不足以辱帝女宜更擇高門美才|{
	更工衡翻}
上曰此非卿所知己卯公主適季友恩禮甚盛頔出望外大喜頃之上使人諷之入朝謝恩頔遂奉詔 |{
	考異曰實錄不見頔入朝月日今因尚主終言之}
是歲李吉甫撰元和國計簿上之|{
	上時掌翻}
總計天下方鎮四十八州府二百九十五縣千四百五十三其鳳翔鄜坊邠寧振武涇原銀夏靈鹽河東易定魏博鎮冀范陽滄景淮西淄青等十五道七十一州不申戶口外|{
	鳳翔鄜坊邠寧振武涇原銀夏靈鹽河東皆被邊易定魏博鎮冀范陽滄景淮西淄青皆藩鎮世襲故並不申戶口納賦税夏戶雅翻}
每歲賦税倚辦止於浙江東西宣歙淮南江西鄂岳福建湖南八道四十九州一百四十四萬戶比天寶税戶四分減三|{
	宋白曰國計簿比較數天寶州郡三百一十五元和見管總二百九十五比較天寶應供税州郡計少九十七天寶戶總八百三十八萬五千二百二十三元和見在戶總二百四十四萬二百五十四比較天寶數税戶通計少百九十四萬四千六百九十九天寶租税庸調每年計錢粟絹布絲綿約五千二百三十餘萬端匹屯貫石元和兩税榷酒斛㪷鹽利茶利總三千五百一十五萬一千二百二十八貫石比較天寶所入賦税計少一千七百一十四萬八千七百七十貫石歙書涉翻}
天下兵仰給縣官者八十三萬餘人|{
	仰牛向翻}
比天寶三分增一大率二戶資一兵其水旱所傷非時調發不在此數|{
	水旱所傷則量減賦税非時調發則出於常賦之外調徒釣翻}
三年春正月癸巳羣臣上尊號曰睿聖文武皇帝赦天下自今長吏詣闕無得進奉知樞密劉光琦|{
	代宗永泰中置内樞密使以宦者為之初不置司局但有屋三楹貯文書而已其職掌惟受表奏於内中進呈若人主有所處分則宣付中書門下施行後僖昭時楊復恭西門李玄欲奪宰相權乃於堂狀後帖黄指揮公事}
奏分遣諸使齎赦詣諸道意欲分其饋遺|{
	使疏吏翻下同遺唯季翻}
翰林學士裴垍李絳奏勅使所至煩擾不若但附急遞|{
	急遞古之傳遞馳驛兼程而行}
上從之光琦稱舊例上曰例是則從之苟為非是奈何不改 臨涇鎮將郝玼|{
	玼音此又且禮翻}
以臨涇地險要水草美吐蕃將入寇必屯其地言於涇原節度使段祐 |{
	考異曰舊傳作段佐新傳作佑今從實録}
奏而城之自是涇原獲安|{
	安史亂後原州沒于吐蕃是後遂以臨涇為理所}
二月戊寅咸安大長公主薨于回鶻|{
	蓬州咸安郡德宗貞元四年咸安公主下嫁回鶻見二百三十三卷長知亮翻}
三月回鶻騰里可汗卒 癸巳郇王總薨|{
	總上弟也}
辛亥御史中丞盧坦奏彈前山南西道節度使柳晟前浙東觀察使閻濟美違赦進奉|{
	彈唐千翻彈其違是年正月癸巳之赦也 考異曰舊晟傳云罷鎮入朝以違詔進奉為御史元稹所劾詔宥之今從實録舊濟美傳自福建觀察使徙浙西罷浙西也方在道見詔而貢獻無所還故帝為言之今據實録云離越州後方見赦文則是浙東新舊傳誤也}
上召坦褒慰之曰朕已釋其罪不可失信坦曰赦令宣布海内陛下之大信也晟等不畏陛下法奈何存小信弃大信乎上乃命歸所進於有司 夏四月上策試賢良方正直言極諫舉人伊闕尉牛僧孺陸渾尉皇甫湜|{
	陸渾縣春秋陸渾戎所居也東魏置伊川郡領南陸渾縣隋開皇初廢改縣曰伏流大業初改曰陸渾唐屬洺州}
前進士李宗閔皆指陳時政之失無所避|{
	李宗閔擢進士調華州参軍故曰前進士}
吏部侍郎楊於陵|{
	於音烏}
吏部員外郎韋貫之為考策官貫之署為上策上亦嘉之詔中書優與處分|{
	處昌呂翻分扶同翻}
李吉甫惡其言直|{
	惡烏路翻}
泣訴於上且言翰林學士裴垍王涯覆策|{
	審考為覆}
湜涯之甥也涯不先言垍無所異同上不得已罷垍涯學士垍為戶部侍郎涯為都官員外郎貫之為果州刺史後數日貫之再貶巴州刺史涯貶虢州司馬|{
	舊志果州至京師一千五百二十八里巴州二千三百六十里虢州四百三十里}
乙亥以楊於陵為嶺南節度使亦坐考策無異同也僧孺等久之不調|{
	調徒釣翻}
各從辟於藩府僧孺弘之七世孫|{
	牛弘相隋}
宗閔元懿之玄孫|{
	鄭王元懿高祖之子}
貫之福嗣之六世孫|{
	韋福嗣見一百八十二卷隋煬帝大業九年韋貫之本名淳避上名改焉}
湜睦州新安人也|{
	新安漢歙縣地江左置新安郡隋廢郡為縣大業初改為雉山唐文明元年復為新安開元二十年改為還淳永貞元年避上名改為清溪此云新安史依舊縣名}
丁丑罷五月朔宣政殿朝賀|{
	唐制元正冬至於正牙受朝賀至貞元七年勅每年五月一日御宣政殿與文武百寮相見京官九品以上外官因朝奏在京者並宜就列本以五月一日隂生臣子道長君父道衰非善月也因創是日相見之儀}
以荆南節度使裴均為右僕射均素附宦官得貴顯為僕射自矜大嘗入朝踰位而立中丞盧坦揖而退之均不從坦曰昔姚南仲為僕射位在此均曰南仲何人坦曰是守正不交權倖者坦尋改右庶子|{
	裴均惡之也}
五月翰林學士左拾遺白居易上疏以為牛僧孺等直言時事恩奬登科而更遭斥逐並出為關外官|{
	牛僧孺等從辟於藩府故以為關外官}
楊於陵等以考策敢收直言裴垍等以覆策不退直言皆坐譴謫盧坦以數舉職事黜庶子|{
	數所角翻}
此數人皆今之人望天下視其進退以卜時之否臧者也|{
	否音鄙}
一旦無罪悉疎弃之上下杜口衆心洶洶陛下亦知之乎且陛下既下詔徵之直言索之極諫|{
	索山客翻}
僧孺等所對如此縱未能推而行之又何忍罪而斥之乎昔德宗初即位亦徵直言極諫之士策問天旱穆質對云兩漢故事三公當免卜式著議弘羊可烹德宗深嘉之自畿尉擢為左補闕|{
	京兆府除兩赤縣外餘為畿縣唐制凡置都其郭下縣為赤縣餘縣亦為畿縣}
今僧孺等所言未過於穆質而遽斥之臣恐非嗣祖宗之道也質寧之子也|{
	穆寧與顔真卿同討安祿山}
丙午冊回鶻新可汗為愛登里囉汨密施合毗伽保義可汗 西原蠻酋長黄少卿請降六月癸亥以為歸順州刺吏|{
	黄少卿反見二百三十四卷德宗貞元十年}
沙陀勁勇冠諸胡吐蕃置之甘州|{
	沙陀降吐蕃見二百三十三卷貞元六年冠古玩翻}
每戰以為前鋒回鶻攻吐蕃取涼州吐蕃疑沙陀貳於回鶻欲遷之河外沙陀懼酋長朱邪盡忠與其子執宜謀復自歸於唐|{
	復扶又翻}
遂帥部落三萬循烏德鞬山而東|{
	帥讀曰率下同烏德鞬山在回鶻牙帳之西甘州東北史炤曰唐歷云即鬰督軍山虜語兩音也鞬居言翻}
行三日吐蕃追兵大至自洮水轉戰至石門|{
	水經註洮水至枹罕入河枹罕唐為河州石門水在高平縣西八十里唐於此置石門關在原州平高縣界}
凡數百合盡忠死士衆死者太半執宜帥其餘衆猶近萬人騎三千詣靈州降|{
	近其靳翻 考異曰趙鳳後唐懿祖紀年録懿祖諱執宜烈考諱盡忠自曾祖入覲復典兵於磧北德宗貞元五年回紇葛禄部及白眼突厥叛回紇忠貞可汗附于吐蕃因為鄉導驅吐蕃之衆三十萬寇我北庭烈考謂忠貞可汗曰吐蕃前年屠䧟靈鹽聞唐天子欲與吐蕃贊普和親可汗數世有功尚主恩若驕兒若贊普有寵於唐則可汗必無前日之寵矣忠貞曰若之何烈考曰唐將楊襲古固守北庭無路歸朝今吐蕃突厥併兵攻之儻無援助䧟亡必矣北庭既沒次及于吾可汗得無慮乎忠貞懼乃命其將頡千迦斯與烈考將兵援北庭貞元六年與吐蕃戰於磧口頡千迦斯不利而退烈考牙於城下以援襲古吐蕃攻圍經年諸部繼沒十二月北庭之衆劫烈考降於吐蕃由是舉族七千帳徙於甘州臣事贊普貞元十三年回紇奉誠可汗收復涼州大敗吐蕃之衆或有間烈考於贊普者云沙陀本回紇部人今聞回紇彊必為内應贊普將遷烈考之牙於河外時懿祖年已及冠白烈考曰吾家世為唐臣不幸䧟虜為它效命反見猜嫌不如乘其不意復歸本朝烈考然之貞元十七年自烏德鞬山率其部三萬東奔居三日吐蕃追兵大至自洮河轉戰至石門關委曲三千里凡數百戰烈考戰沒懿祖挾護靈輿收合餘衆至於靈州猶有馬三千騎勝兵一萬時范希朝為河西靈鹽節度使聞懿祖至自帥師蕃界應接而歸以事奏聞德宗遣中使賜詔慰勞賞錫數十萬因於鹽州置隂山府以懿祖為都督授特進驍衛將軍同正憲宗即位詔懿祖入朝元和元年七月乃自振武至長安授特進金吾衛將軍留宿衛時范希朝亦徵為金吾上將軍二年吐蕃誘我党項部寇犯河西天子復命希朝為靈鹽節度命懿祖將兵佐之賊平戍西受降城據德宗實録貞元十七年無沙陀歸國事范希朝傳德宗時為振武節度使元和二年乃為朔方靈鹽節度使誘致沙陀元和元年亦無沙陀朝見紀年録恐誤今從實録舊傳新書}
靈鹽節度使范希朝聞之自帥衆迎於塞上置之鹽州為市牛羊廣其畜牧善撫之|{
	為于偽翻}
詔置隂山府以執宜為兵馬使未幾盡忠弟葛勒阿波又帥衆七百詣希朝降|{
	幾居豈翻}
詔以為隂山府都督自是靈鹽每有征討用之所向皆捷靈鹽軍益彊|{
	為沙陀彊盛得中夏張本}
秋七月辛巳朔日有食之 以右庶子盧坦為宣歙觀察使蘇彊之誅也|{
	蘇彊劉闢之壻也元年以逆黨誅}
兄弘在晉州幕府自免歸人莫敢辟坦奏弘有才行不可以其弟故廢之請辟為判官上曰曏使蘇彊不死果有才行猶可用也|{
	行下孟翻}
况其兄乎坦到官值旱饑穀價日增或請抑其價坦曰宣歙土狹穀少所仰四方之來者若價賤則商船不復來益困矣既而米斗二百商旅輻湊|{
	後人用此策以救荒者盧坦發之也仰牛向翻復扶又翻}
九月庚寅以于頔為司空同平章事如故加右僕射裴均同平章事為山南東道節度使淮南節度使王鍔入朝鍔家巨富厚進奉及賂宦官求平章事翰林學士白居易以為宰相人臣極位非清望大功不應授昨除裴均外議已紛然|{
	裴均亦要結宦官者也}
今又除鍔則如鍔之輩皆生冀望若盡與之則典章大壞又不感恩不與則厚薄有殊或生怨望倖門一啓無可奈何且鍔在鎮五年|{
	德宗貞元十九年鍔為淮南帥}
百計誅求貨財既足自入進奉若除宰相四方藩鎮皆謂鍔以進奉得之競為刻剥則百姓何以堪之事遂寢 |{
	考異曰按舊李藩權德輿傳白居易集李絳論事集皆有諫加王鍔平章事事觀其辭意各是一時居易所論者淮南百姓日夜無憀又云鍔歸鎮與在朝望並不除宰相則是自淮南入朝未除河中時也權李同在中書受密旨云可兼宰相則初除河中時也李司空論事云至太原一二年間財力贍足則是除太原以後六年十一月李絳作相前也今附居易疏於初除太原之時又舊鍔傳云在淮南四年元和二年入朝按實録鍔以貞元十九年鎮淮南居易狀云五年誅求又云昨日裴均除平章事故置此}
壬辰加宣武節度使韓弘同平章事 丙申以戶部侍郎裴垍為中書侍郎同平章事上雖以李吉甫故罷垍學士|{
	是年四月罷垍學士}
然寵信彌厚故未幾復擢為相初德宗不任宰相天下細務皆自决之由是裴延齡輩得用事上在藩邸心固非之及即位選擢宰相推心委之嘗謂垍等曰以太宗玄宗之明猶藉輔佐以成其理|{
	謂藉房杜姚宋以成貞觀開元之治也理治也}
况如朕不及先聖萬倍者乎垍亦竭誠輔佐上嘗問垍為理之要何先對曰先正其心舊制民輸税有三一曰上供二曰送使三曰留州建中初定兩税貨重錢輕是後貨輕錢重民所出已倍其初其留州送使者所在又降省佑就實估以重歛於民|{
	省估者都省所立價也歛力贍翻}
及垍為相奏天下留州送使物請一切用省估其觀察使先税所理之州以自給不足然後許税於所屬之州由是江淮之民稍蘇息先是執政多惡諫官言時政得失|{
	先悉薦翻惡烏路翻}
垍獨賞之垍器局峻整人不敢干以私嘗有故人自遠詣之垍資給優厚從容欵狎其人乘間求京兆判司|{
	從千容翻間古莧翻凡州府諸曹參軍皆謂之判司}
垍曰公不稱此官|{
	稱尺證翻}
不敢以故人之私傷朝廷至公它日有盲宰相憐公者不妨得之垍則必不可 戊戌以中書侍郎同平章事李吉甫同平章事充淮南節度使 |{
	考異曰舊吉甫傳曰初裴均為僕射判度支交結權倖欲求宰相先是制試直言極諫料其中有譏刺時政忤犯權倖者因此均黨揚言皆執政教指冀以揺動吉甫賴諫官李約獨孤郁李正辭蕭俛密疏陳奏帝意乃解吉甫早歲知奬羊士諤擢為監察御史又司封員外郎呂温有詞藝吉甫亦眷接之竇羣初拜御史中丞奏請士諤為侍御史温為郎中知雜事吉甫怒其不先關白而所請又有超資者持之數日不行因而有隙羣遂伺得日者陳克明出入吉甫家密捕以聞憲宗詰之無姦狀吉甫以裴垍久在翰林憲宗親信必當大用遂密薦垍代已因自圖出鎮其年九月拜淮南節度使在揚州每有朝廷得失皆密疏論列按牛僧孺等指陳時政之失吉甫泣訴故貶考覆官裴垍等雖欲為讒若云執政自教舉人詆時政之失豈近人情邪吉甫自以誣搆鄭絪貶斥裴垍蓋憲宗察見其情而疎薄之故出鎮淮南及子德裕秉政掩先人之惡改定實録故有此說耳}
河中晉絳節度使邠宣公杜黄裳薨 冬十二月庚戌置行原州於臨涇|{
	唐原州本治平高縣廣德元年沒於吐蕃涇原節度使馬璘表置行原州於靈臺之百里城貞元十九年徙治平京至是徙治臨涇宋白曰臨涇本隋之湫谷縣}
以鎮將郝玼為刺史|{
	玼音此又且禮翻}
南詔王異牟尋卒子尋閤勸立

四年春正月戊子簡王遘薨|{
	遘代宗子}
渤海康王嵩璘卒子元瑜立改元永德 南方旱饑庚寅命左司郎中鄭敬德等為江淮二浙荆湖襄鄂等道宣慰使賑恤之將行上戒之曰朕宫中用帛一匹皆籍其數惟賙救百姓則不計費卿輩宜識此意勿效潘孟陽飲酒遊山而已|{
	事見元年}
給事中李藩在門下制敕有不可者即於黄紙後批之|{
	批匹迷翻}
吏請更連素紙藩曰如此乃狀也何名批敕裴垍薦藩有宰相器上以門下侍郎同平章事鄭絪循默取容二月丁卯罷絪為太子賓客擢藩為門下侍郎同平章事藩知無不言上甚重之 河東節度使嚴綬在鎮九年|{
	貞元十七年嚴綬鎮河東見上卷}
軍政補署一出監軍李輔光綬拱手而已裴垍具奏其狀請以李鄘代之三月乙酉以綬為左僕射以鳳翔節度使李鄘為河東節度使 成德節度使王士真薨其子副大使承宗自為留後|{
	為討王承宗張本}
河北三鎮相承各置副大使以嫡長為之父沒則代領軍務|{
	長知兩翻}
上以久旱欲降德音翰林學士李絳白居易上言 |{
	考異曰李司空論事及居易集皆有此奏語雖小異大指不殊蓋同上奏耳}
以為欲令實惠及人無如減其租税又言宫人驅使之餘其數猶廣事宜省費物貴徇情|{
	冗食宫中歲費給賜則非省費矣内多怨女則非徇情矣}
又請禁諸道横歛以充進奉又言嶺南黔中福建風俗多掠良人賣為奴婢乞嚴禁止閏月己酉制降天下繫囚蠲租税出宫人絶進奉禁掠賣皆如二人之請己未雨絳表賀曰乃知憂先於事故能無憂|{
	先悉薦翻}
事至而憂無救於事 初王叔文之黨既貶|{
	事始見上卷永貞元年}
有詔雖遇赦無得量移|{
	量音良}
吏部尚書鹽鐵轉運使李巽奏郴州司馬程异吏才明辨請以為揚子留後|{
	揚州揚子縣自大歷以來鹽鐵轉運使置廵院於此故置留後}
上許之巽精於督察吏人居千里之外戰栗如在巽前异句檢簿籍又精於巽卒獲其用|{
	句音鉤為程异以理財進用張本卒子恤翻}
魏徵玄孫稠貧甚以故第質錢於人平盧節度使李師道請以私財贖出之上命白居易草詔居易奏言事關激勸宜出朝廷師道何人敢掠斯美望敕有司以官錢贖還後嗣上從之出内庫錢二千緡贖賜魏稠仍禁質賣|{
	程大昌曰魏徵宅在丹鳳坊直出南面永興坊内會要曰元和四年上嘉魏徵諫諍詔訪其故居則質賣已更數姓析為九家矣上出内庫錢一百萬贖之以還其家禁其質賣}
王承宗叔父士則以承宗擅自立恐禍及宗與幕客劉栖楚俱自歸京師 |{
	考異曰舊傳栖楚為吏鎮州王承宗甚奇之今從實録}
詔以士則為神策大將軍 翰林學士李絳等奏曰陛下嗣膺大寶四年於兹而儲闈未立典冊不行是開窺覦之端乖重慎之義非所以承宗廟重社稷也伏望抑撝謙之小節行至公之大典丁卯制立長子鄧王寧為太子寧紀美人之子也 辛未靈鹽節度使范希朝奏以太原兵六百人衣糧給沙陀許之 夏四月山南東道節度使裴均恃有中人之助於德音後進銀器千五百餘兩|{
	是年正月赦天下禁無得進奉}
翰林學士李絳白居易等上言均欲以此嘗陛下願却之上遽命出銀器付度支|{
	度徒洛翻}
既而有旨諭進奏院自今諸道進奉無得申御史臺有訪問者輒以名聞白居易復以為言 |{
	考異曰居易集奏狀云伏見六七日來向外傳說皆云有進止令宣與諸道進奏院自今已後應有進奉並不用申報御史臺如有人勘問便録名奏來者内外相傳不無驚怪臣伏料此事多是虛傳且有此聞不敢不奏云云又曰若此果虚即望宣示中外令知聖旨使息虚聲按禁止進奉前後制敕非一不止於昨閏三月德音也去年三月柳晟閻濟美違赦進奉已為盧坦所彈憲宗云濟美離越州乃逢赦令釋其罪今裴均所進假使在德音前亦赦後矣又云赦書未到前已在道捨其過是則憲宗深惑于左右之言外示不受獻内實欲其來獻也然則居易所聞不為虚矣若其虛必辯明也實錄及李司空論事皆以此為憲宗之美今故直之復扶又翻}
上不聽 上欲革河北諸鎮世襲之弊乘王士真死欲自朝廷除人不從則興師討之裴垍曰李納跋扈不㳟|{
	李納之罪以興元赦令遂蒙含貸}
王武俊有功於國|{
	謂與李抱真破朱滔也}
陛下前許師道|{
	言許李師道承襲}
今奪承宗沮勸違理彼必不服由是議久不决上以問諸學士李絳等對曰河北不遵聲教誰不憤歎然今日取之或恐未能成德自武俊以來父子相承四十餘年|{
	自建中三年王武俊始有恒冀至是二十八年}
人情貫習不以為非|{
	貫讀曰慣慣習猶言慣熟}
况承宗已總軍務一旦易之恐未必奉詔又范陽魏博易定淄青以地相傳與成德同體彼聞成德除人必内不自安隂相黨助雖茂昭有請亦恐非誠|{
	張茂昭宿與王武俊有隙故請代承宗}
今國家除人代承宗彼鄰道勸成進退有利若所除之人得入彼則自以為功若詔令有所不行彼因潛相交結在於國體豈可遽休須興師四面攻討彼將帥則加官爵士卒則給衣糧按兵玩寇坐觀勝負而勞費之病盡歸國家矣|{
	自大歷貞元以來用兵之弊正如此}
今江淮大水公私困竭軍旅之事殆未可輕議也左軍中尉吐突承璀|{
	璀七罪翻}
欲希上意奪裴垍權自請將兵討之宗正少卿李拭奏稱承宗不可不討承璀親近信臣宜委以禁兵使統諸軍誰敢不服上以拭狀示諸學士曰此姦臣也知朕欲將承璀故上此奏卿曹記之自今勿令得進用|{
	示諸學士者蓋以此時凡入翰林者即日輔佐之選也故使知其姓名勿得擬用然帝知李拭之迎逢而卒將承璀何邪欲將即亮翻上時掌翻}
昭義節度使盧從史遭父喪朝廷久未起復從史懼因承璀說上|{
	說式芮翻}
請發本軍討承宗壬辰起復從史左金吾大將軍餘如故 初平凉之盟|{
	事見二百三十二卷唐德宗貞元三年}
副元帥判官路泌會盟判官鄭叔矩皆沒於吐蕃其後吐蕃請和泌子隨三詣闕號泣上表乞從其請|{
	路隨表請和蕃情切於其親也號戶刀翻}
德宗以吐蕃多詐不許至是吐蕃復請和|{
	復扶又翻}
隨又五上表詣執政泣請裴垍李藩亦言於上請許其和上從之五月命祠部郎中徐復使吐蕃 六月以靈鹽節度使范希朝為河東節度使朝議以沙陀在靈武廹近吐蕃|{
	朝直遥翻近其靳翻}
慮其反復又部落衆多恐長穀價|{
	長知兩翻}
乃命悉從希朝詣河東希朝選其驍騎千二百號沙陀軍置使以領之而處其餘衆于定襄州於是執宜始保神武川之黄花堆|{
	神武川在漢代郡桑乾縣界後魏置神武郡後周廢郡為神武縣屬朔州此時其地在馬邑善陽縣界處昌呂翻}
左軍中尉吐突承璀領功德使|{
	唐初置寺觀監天下僧尼道士女官皆屬鴻臚寺武后以僧尼屬祠部開元十四年以道士女官屬宗正寺天寶二載以道士屬司封崇玄館置大學士以宰相為之領兩京玄元宫及道院貞元四年崇玄館罷大學士置左右街大功德使東都功德使修功德使總僧尼之籍及功役元和二年以道士女官屬左右街功德使}
盛脩安國寺|{
	唐會要安國寺在長樂坊景雲元年勅捨龍潛舊宅為寺便以本封安國為名程大昌曰長樂坊在朱雀街東第四街}
奏立聖德碑高大一準華嶽碑|{
	玄宗立華嶽碑於華嶽祠前高五十餘尺華戶化翻}
先構碑樓請勅學士撰文且言臣已具錢萬緡欲酬之上命李絳為之絳上言堯舜禹湯未嘗立碑自言聖德惟秦始皇於廵遊所過刻石高自稱述未審陛下欲何所法且叙修寺之美不過壯麗觀遊豈所以光益聖德上覽奏承璀適在旁上命曳倒碑樓|{
	曳讀作拽音以列翻史炤音以制切非}
承璀言碑樓甚大不可曳請徐毁撤冀得延引乘間再論|{
	間古莧翻}
上厲聲曰多用牛曳之承璀乃不敢言凡用百牛曳之乃倒

資治通鑑卷二百三十七
