資治通鑑卷一百二十一
宋 司馬光 撰

胡三省 音註

宋紀三|{
	起著雍執徐盡上章敦牂凡三年}


太祖文皇帝上之中

元嘉五年春正月辛未魏京兆王黎卒 荆州刺史彭城王義康性聰察在州職事修治|{
	治直吏翻}
左光祿大夫范泰謂司徒王弘曰天下事重權要難居卿兄弟盛滿當深存降挹|{
	謂弘及弟曇首皆居權要}
彭城王帝之次弟宜徵還入朝共參朝政|{
	朝直遥翻}
弘納其言時大旱疾疫弘上表引咎遜位帝不許 秦商州刺史領澆河太守姚濬叛降河西|{
	晉時張祚以敦煌郡為商州時敦煌屬河西熾磐蓋以濬遥領商州而守澆河也澆堅堯翻降戶江翻}
秦王熾磐以尚書焦嵩代濬帥騎三千討之|{
	帥讀曰率騎奇寄翻}
二月嵩為吐谷渾元緒所執 魏改元神䴥|{
	䴥居牙翻牡鹿也以獲神鹿改元魏書靈徵志時定州獲白䴥白䴥鹿也又見于樂陵}
魏平北將軍尉眷攻夏主於上邽|{
	尉紆勿翻}
夏主退屯平涼奚斤進軍安定與丘堆娥清軍合斤馬多疫死士卒乏糧乃深壘自固遣丘堆督租于民間士卒暴掠不設儆備夏主襲之堆兵敗以數百騎還城夏主乘勝日來城下鈔掠不得芻牧諸將患之監軍侍御史安頡曰|{
	鈔楚交翻將即亮翻監工銜翻頡戶結翻}
受詔滅賊今更為賊所困退守窮城若不為賊殺當坐法誅進退皆無生理而諸王公晏然曾不為計乎|{
	奚斤封宜城王為司空}
斤曰今軍士無馬以步擊騎必無勝理當須京師救騎至合撃之頡曰今猛寇遊逸於外吾兵疲食盡不一決戰則死在旦夕救騎何可待乎等于就死死戰不亦可乎斤又以馬少為辭|{
	騎奇寄翻少詩沼翻}
頡曰今歛諸將所乘馬可得二百匹頡請募敢死之士出擊之就不能破敵亦可以折其鋭且赫連昌狷而無謀好勇而輕|{
	狷吉縣翻好呼到翻輕遣政翻}
每自出挑戰|{
	挑徒了翻}
衆皆識之若伏兵掩擊昌可擒也斤猶難之頡乃隂與尉眷等謀選騎待之既而夏主來攻城頡出應之夏主自出陳前戰|{
	陳讀曰陣}
軍士識其貌争赴之會天大風揚塵晝昏夏主敗走頡追之夏主馬蹶而墜遂擒之 |{
	考異曰十六國春秋鈔云承光三年五月戰于黑渠為魏所敗昌與數千騎奔還魏追騎亦至昌河内公費連烏提守高平徙諸城民七萬戶于安定以都之四年二月魏軍至安定三城潰昌奔秦州魏東平公娥清追擒之送于魏與後魏紀傳不同今從後魏書}
頡同之子也|{
	安同永興初八公之一也}
夏大將軍領司徒平原王定收其餘衆數萬犇還平涼即皇帝位|{
	定小字直僨夏王勃勃第五子}
大赦改元勝光三月辛巳赫連昌至平城魏主館之於西宫門内器用皆給乘輿之副又以妹始平公主妻之|{
	乘繩證翻妻七細翻}
假常忠將軍賜爵會稽公|{
	會公外翻}
以安頡為建節將軍賜爵西平公尉眷為寧北將軍進爵漁陽公魏主常使赫連昌侍從左右|{
	從才用翻}
與之單騎共逐鹿深入山澗昌素有勇名諸將咸以為不可魏主曰天命有在亦何所懼親遇如初奚斤自以為元帥而昌為偏裨所擒深恥之乃捨輜重|{
	重直龍翻下同}
齎三日糧追夏主於平涼娥清欲循水而往|{
	清蓋欲循涇水而進}
斤不從自北道邀其走路至馬髦嶺|{
	馬髦山之嶺也}
夏軍將遁會魏小將有罪亡歸於夏告以魏軍食少無水|{
	少詩沼翻}
夏主乃分兵邀斤前後夾撃之魏兵大潰斤及娥清劉拔皆為夏所擒|{
	去年魏遣劉拔與斤共撃夏}
士卒死者六七千人 |{
	考異曰宋索虜傳元嘉五年使大將吐伐斤西伐長安生擒赫連昌于安定封昌為公以妹妻之昌弟定在隴上吐伐斤乘勝以騎三萬討之定伏設於隴山彈筝谷破之斬吐伐斤盡坑其衆定率衆東還復克長安燾又自攻不克乃分軍戍大城而還今從後魏書}
丘堆守輜重在安定聞斤敗棄輜重奔長安與高涼王禮偕奔蒲阪|{
	阪音反}
夏人復取長安|{
	復扶又翻下復勸同}
魏主大怒命安頡斬丘堆代將其衆鎮蒲阪以拒之|{
	將即亮翻}
夏四月夏主遣使請和於魏魏主以詔諭之使降|{
	使疏吏翻降戶江翻}
壬子魏主西巡戊午畋於河西|{
	此君子津之西也}
大赦 五月秦文昭王熾磐卒太子暮末即位大赦改元永弘 平陸令河南成粲|{
	平陸縣自漢以來屬東平郡}
復勸王弘遜位|{
	復扶又翻}
弘從之累表陳請帝不得已六月庚戌以弘為衛將軍開府儀同三司 甲寅魏主如長川|{
	魏書帝紀泰常八年築長城於長川之南}
葬秦文昭王于武平陵廟號太祖秦王暮末以右丞相元基為侍中相國都督中外諸軍錄尚書事以鎮軍大將軍河州牧謙屯為驃騎大將軍|{
	河州治枹罕乞伏氏所都驃匹妙翻騎奇寄翻下同}
徵安北將軍涼州刺史段暉為輔國大將軍御史大夫|{
	段暉先鎮樂都}
叔父右禁將軍千年為鎮北將軍涼州牧鎮湟河以征北將軍木弈干為尚書令車騎大將軍以征南將軍吉毗為尚書僕射衛大將軍河西王蒙遜因秦喪伐秦西平西平太守麴承謂之曰殿下若先取樂都則西平必為殿下之有苟望風請服亦明主之所疾也蒙遜乃釋西平攻樂都相國元基帥騎三千救樂都|{
	元基自枹罕救樂都樂音洛}
甫入城而河西兵至攻其外城克之絶其水道城中饑渇死者大半東羌乞提從元基救樂都隂與河西通謀下繩引内其兵登城者百餘人鼓譟燒門元基帥左右奮撃河西兵乃退|{
	帥讀曰率}
初文昭王疾病謂暮末曰吾死之後汝能保境則善矣沮渠成都為蒙遜所親重汝宜歸之至是暮末遣使詣蒙遜許歸成都以求和|{
	成都為秦禽事見一百十九卷武帝永初三年沮子余翻使疏吏翻下同}
蒙遜引兵還遣使入秦弔祭暮末厚資送成都遣將軍王伐送之蒙遜猶疑之使恢武將軍沮渠奇珍伏兵於捫天嶺執伐并其騎士三百人以歸既而遣尚書郎王杼送伐還秦并遺暮末馬千匹及錦罽銀繒|{
	遺于季翻罽音計繒慈林翻}
秋七月暮末遣記室郎中馬艾如河西報聘 魏主還宫 八月復如廣甯觀溫泉|{
	水經註下洛縣故城魏燕州廣甯縣廣甯郡治魏土地記下洛城東南四十里有橋山山下有溫泉泉上有祭堂彫簷華宇被于浦上石池吐泉陽陽其下炎涼代序是水灼焉無改能治百疾赴者若流復扶又翻}
柔然紇升蓋可汗遣其子將萬餘騎寇魏邊|{
	紇戶骨翻可讀從刋入聲汗音寒將即亮翻騎奇寄翻}
魏主自廣甯還追之不及九月還宫 冬十月甲辰魏主北巡壬子畋于牛川 秦涼州牧乞伏千年嗜酒殘虐不恤政事秦王暮末遣使讓之千年懼犇河西|{
	奔河西王蒙遜也}
暮末以叔父光祿大夫沃陵為涼州牧鎮湟河 徐州刺史王仲德遣步騎二千伐魏濟陽陳留|{
	濟陽縣漢晉以來屬陳留郡此時陳留郡治浚儀杜佑曰濟陽縣故城在曹州寃句縣西南濟子禮翻 考異曰後魏紀云淮北鎮將按南史仲德時為安北將軍徐州刺史宋書仲德傳闕又宋書南史本紀北史本紀及宋魏諸臣列傳魏劉裕傳宋索虜傳皆無是年王仲德等伐魏事唯後魏本紀有之今從之}
魏主還宮 魏定州丁零鮮于臺陽等二千餘家叛入西山|{
	魏主珪皇始二年置安州于中山天興三年改曰定州西山即曲陽西山也}
州郡不能討閏月魏主遣鎮南將軍叔孫建討之 十一月乙未朔日有食之 魏主如西河校獵|{
	河水逕漢雲中楨陵縣西南平城在其東北故謂之西河}
十二月甲申還宮 河西王蒙遜伐秦至磐夷秦相國元基等將騎萬五千拒之蒙遜還攻西平征虜將軍出連輔政等將騎二千救之|{
	將即亮翻騎奇寄翻}
祕書監謝靈運自以名輩才能應參時政上唯接以

文義每侍宴談賞而已王曇首王華殷景仁名位素出靈運下並見任遇靈運意甚不平多稱疾不朝直|{
	不入朝不入直也曇徒含翻朝直遙翻}
或出郭遊行且二百里經旬不歸既無表聞又不請急上不欲傷大臣意諷令自解靈運乃上表陳疾上賜假令還會稽|{
	假居訝翻會工外翻}
而靈運遊飲自若為法司所糾坐免官 是歲師子王刹利摩訶及天竺迦毗黎王月愛皆遣使奉表入貢表辭皆如浮屠之言|{
	南史師子國天竺旁國也其地和適無冬夏之異五穀隨人種不須時節天竺有迦毗黎蘇摩黎斤陀利婆黎等國皆事佛道刹初轄翻迦古牙翻又居伽翻使疏吏翻}
魏鎮遠將軍平舒侯燕鳳卒|{
	燕鳳歷事拓跋氏四世}


六年春正月王弘上表乞解州錄以授彭城王義康|{
	州錄揚州及錄尚書事也}
帝優詔不許癸丑以義康為侍中都督揚南徐兖三州諸軍事司徒錄尚書事領南徐州刺史|{
	武帝永初二年加京口之徐州曰南徐淮北之徐州但曰徐南徐領南東海南琅邪晉陵義興南蘭陵南東莞臨淮淮陵南彭城南清河南高平南平昌南濟隂南濮陽南泰山濟陽南魯郡等郡}
弘與義康二府並置佐領兵共輔朝政|{
	佐參佐也所謂佐吏朝直遥翻}
弘既多疾且欲委遠大權每事推讓義康|{
	遠于願翻推吐雷翻}
由是義康專摠内外之務|{
	為義康專擅致禍張本}
又以撫軍將軍江夏王義恭為都督荆湘等八州諸軍事荆州刺史|{
	夏戶雅翻}
以侍中劉湛為南蠻校尉行府州事帝與義恭書誡之曰天下艱難家國事重雖曰守成實亦未易隆替安危在吾曹耳豈可不感尋王業大懼負荷|{
	感念致王業之艱難而尋繹為治之理也傳曰其父析薪其子不克負荷易以豉翻荷下可翻又音如字}
汝性褊急志之所滯其欲必行|{
	滯凝也積也褊方緬翻}
意所不存從物囬改此最弊事宜念裁抑衛青遇士大夫以禮與小人有恩西門安于矯性齊美|{
	西門豹性剛急常佩韋以自綏董安于性寛緩常佩弦以自警}
關羽張飛任偏同弊|{
	事見六十九卷魏文帝黄初二年}
行已舉事深宜鑒此若事異今日嗣子幼蒙|{
	陸德明曰蒙稚也}
司徒當周公之事汝不可不盡祇順之理爾時天下安危決汝二人耳汝一月自用錢不可過三十萬若能省此益美西楚府舍畧所諳究|{
	諳烏舍翻}
計當不須改作日求新異|{
	江左謂荆州為西楚}
凡訊獄多決當時難可逆慮此實為難至訊日虚懷博盡慎無以喜怒加人能擇善者而從之美自歸已不可專意自決以矜獨斷之明也|{
	斷丁亂翻}
名器深宜慎惜不可妄以假人昵近爵賜尤應裁量吾于左右雖為少恩|{
	昵尼質翻量音良少詩沼翻}
如聞外論不以為非也以貴凌物物不服以威加人人不厭此易達事耳|{
	易以豉翻}
聲樂嬉遊不宜令過蒲酒漁獵一切勿為|{
	蒲樗蒲也}
供用奉身皆有節度奇服異器不宜興長|{
	長丁丈翻今知兩翻}
又宜數引見佐史|{
	佐史當作佐吏晉宋之間藩府率謂參佐為佐吏數所角翻下同}
見不數則彼我不親不親無因得盡人情人情不盡復何由知衆事也|{
	詳觀宋文帝此書則江左之治稱元嘉良有以也復扶又翻}
夏酒泉公雋自平涼犇魏 丁零鮮于臺陽等請降於魏|{
	降戶江翻}
魏主赦之|{
	赦其去年叛入西山之罪}
秦出連輔政等未至西平河西王蒙遜拔西平執太守麴承|{
	蒙遜去年攻西平}
二月秦王暮末立妃梁氏為皇后子萬載為太子 三月丁巳立皇子劭為太子戊午大赦 辛酉以左衛將軍殷景仁為中領軍帝以章太后早亡|{
	章太后胡氏生帝五年被譴賜死帝即位諡曰章}
奉太后所生蘇氏甚謹蘇氏卒帝往臨哭欲追加封爵使羣臣議之景仁以為古典無之乃止 初秦尚書隴西辛進從文昭王遊陵霄觀彈飛鳥誤中秦王暮末之母傷其面|{
	觀古玩翻中竹仲翻}
及暮末即位問母面傷之由母以狀告暮末怒殺進并其五族二十七人|{
	史言暮末以虐亡國}
夏四月癸亥以尚書左僕射王敬弘為尚書令臨川王義慶為左僕射吏部尚書濟陽江夷為右僕射|{
	濟子禮翻}
初魏太祖命尚書鄧淵撰國記十餘卷未成而止世祖更命崔浩與中書侍郎鄧穎等續成之|{
	為浩以史事得禍張本}
為國書三十卷穎淵之子也 魏主將擊柔然治兵于南郊|{
	治直之翻}
先祭天然後部勒行陳|{
	行戶剛翻陳讀曰陣}
内外羣臣皆不欲行保太后固止之獨崔浩勸之尚書令劉絜等共推太史令張淵徐辯使言于魏主曰今兹己巳三隂之歲|{
	干以甲丙戊庚壬為陽乙丁己辛癸為隂支以子寅辰午申戌為陽丑卯己未酉亥為隂己巳皆隂而干支合于己巳是為三隂之歲}
歲星襲月太白在西方不可舉兵北伐必敗雖克不利于上羣臣因共贊之曰淵等少時嘗諫苻堅南伐堅不從而敗所言無不中不可違也|{
	少詩沼翻中竹仲翻}
魏主意不快詔浩與淵等論難于前|{
	難乃旦翻}
浩詰淵辯曰陽為德隂為刑故日食脩德月食脩刑夫王者用刑小則肆諸市朝大則陳諸原野|{
	陳諸原野用甲兵也此言本出漢書刑法志詰去吉翻朝直遙翻}
今出兵以討有罪乃所以修刑也臣竊觀天文比年以來月行掩昴至今猶然其占三年天子大破旄頭之國|{
	比毗至翻昴為旄頭胡星也}
蠕蠕高車旄頭之衆也|{
	蠕人兖翻}
願陛下勿疑淵辯復曰|{
	復扶又翻}
蠕蠕荒外無用之物得其地不可耕而食得其民不可臣而使輕疾無常難得而制有何汲汲而勞士馬以伐之浩曰淵辯言天道猶是其職至于人事形勢尤非其所知此乃漢世常談|{
	自韓安國主父偃至于嚴尤其論皆如此}
施之于今殊不合事宜何則蠕蠕本國家北邊之臣中間叛去|{
	見一百八卷晉孝武太元十九年}
今誅其元惡收其良民令復舊役非無用也世人皆謂淵辯通解數術明決成敗臣請試問之|{
	解戶買翻}
屬者統萬未亡之前|{
	屬之欲翻}
有無敗徵若其不知是無術也知而不言是不忠也時赫連昌在坐|{
	坐徂卧翻}
淵等自以未嘗有言慙不能對魏主大悦既罷公卿或尤浩曰今南寇方伺國隙|{
	伺相吏翻}
而捨之北伐若蠕蠕遠遁前無所獲後有彊寇將何以待之浩曰不然今不先破蠕蠕則無以待南寇南人聞國家克統萬以來内懷恐懼故揚聲動衆以衛淮北比吾破蠕蠕往還之間南寇必不動也|{
	比必寐翻}
且彼步我騎|{
	騎奇寄翻}
彼能北來我亦南往在彼甚困於我未勞況南北殊俗水陸異宜設使國家與之河南彼亦不能守也|{
	崔浩之料宋人審矣帝後屢出兵争河南卒以自弊吳呂蒙不肯取魏徐州正慮此耳}
何以言之以劉裕之雄傑吞併關中留其愛子輔以良將精兵數萬猶不能守全軍覆没|{
	事見一百十八卷晉安帝義熙十四年將即亮翻下同}
號哭之聲至今未已|{
	號戶高翻}
況義隆今日君臣非裕時之比主上英武士馬精彊彼若果來譬如以駒犢鬬虎狼也|{
	馬子曰駒牛子曰犢}
何懼之有蠕蠕恃其絶遠謂國家力不能制自寛日久故夏則散衆放畜秋肥乃聚背寒向溫|{
	背蒲妹翻}
南來寇鈔|{
	鈔楚交翻}
今掩其不備必望塵駭散牡馬護牝牝馬戀駒驅馳難制不得水草不過數日必聚而困弊可一舉而滅也蹔勞永逸時不可失|{
	蹔與暫同}
患在上無此意今上意已決奈何止之寇謙之謂浩曰蠕蠕果可克乎浩曰必克但恐諸將瑣瑣前後顧慮不能乘勝深入使不全舉耳|{
	瑣瑣細小也言志趣細小不能一舉而全取之也}
先是帝因魏使者還告魏主曰汝趣歸我河南地|{
	先悉薦翻使疏吏翻趣讀曰促}
不然將盡我將士之力魏主方議伐柔然聞之大笑謂公卿曰龜鼈小豎|{
	東南澤國也故詆之曰龜鼈小豎}
自救不暇夫何能為就使能來若不先滅蠕蠕乃是坐待寇至腹背受敵非良策也吾行決矣庚寅魏主發平城使北平王長孫嵩廣陵公樓伏連居守|{
	守手又翻魏書官氏志獻帝次弟為伊婁氏又有乙郍婁氏後並改為婁氏}
魏主自東道向黑山使平陽王長孫翰自西道向大娥山同會柔然之庭 五月壬辰朔日有食之 王敬弘固讓尚書令表求還東|{
	還從宣翻又如字下同}
癸巳更以敬弘為侍中特進左光祿大夫聽其東歸|{
	自建康歸會稽為東歸}
丁未魏主至漠南捨輜重帥輕騎兼馬襲撃柔然至栗水|{
	重直用翻帥讀曰率騎奇寄翻兼馬者每一騎兼有副馬也栗水在漠北近稽落山有漢將軍竇憲故壘在焉}
柔然紇升蓋可汗先不設備民畜滿野驚怖散去|{
	訖下没翻可從刋入聲汗音寒怖普布翻}
莫相收攝|{
	攝錄也飭整也}
紇升蓋燒廬舍絶迹西走莫知所之其弟匹黎先主東部聞有魏寇帥衆欲就其兄遇長孫翰翰邀擊大破之殺其大人數百 夏主欲復取統萬引兵東至侯尼城|{
	侯尼城在平涼東}
不敢進而還 河西王蒙遜伐秦秦王暮末留相國元基守枹罕遷保定連南安太守翟承伯等據罕开谷以應河西|{
	水經注隴西白石縣東有罕开溪又東則枹罕縣故城枹音膚开苦堅翻}
暮末擊破之進至治城|{
	魏收地形志涼州東陘郡有治城縣其地當在黄河南又涼州有建昌郡亦有治城縣}
西安太守莫者幼眷據汧川以叛|{
	此汧川非扶風之汧當亦在枹罕左右汧口堅翻}
暮末討之為幼眷所敗|{
	敗補邁翻}
還于定連蒙遜至枹罕遣世子興國進攻定連六月暮末逆撃興國于治城擒之追撃蒙遜至譚郊吐谷渾王慕璝遣其弟没利延將騎五千會蒙遜伐秦|{
	没利延即慕利延没慕聲相近也璝古回翻}
暮末遣輔國大將軍段暉等邀撃大破之 柔然紇升蓋可汗既走部落四散竄伏山谷雜畜布野|{
	畜許救翻}
無人收視魏主循栗水西行至菟園水|{
	菟園水在燕然山南去平城三千七百餘里菟同都翻又土故翻}
分軍搜討東西五千里南北三千里俘斬甚衆高車諸部乘魏兵勢鈔掠柔然柔然種類前後降魏者三十餘萬落|{
	鈔楚交翻種章勇翻降戶江翻下同}
獲戎馬百餘萬匹畜產車廬彌漫山澤亡慮數百萬|{
	亡無字通}
魏主循弱水西行至涿邪山|{
	邪讀曰耶}
諸將慮深入有伏兵勸魏主留止寇謙之以崔浩之言告魏主魏主不從秋七月引兵東還至黑山以所獲班賜將士有差既而得降人言可汗先被病|{
	被皮義翻}
聞魏兵至不知所為乃焚穹廬以車自載將數百人入南山民畜窘聚無人統領|{
	窘渠隕翻}
相去百八十里追兵不至乃徐西遁唯此得免後聞涼州賈胡言若復前行二日則盡滅之矣|{
	賈音古復扶又翻}
魏王深悔之紇升蓋可汗憤悒而卒子吳提立號敇連可汗|{
	魏收曰敇連魏言神聖也}
武都孝昭王楊玄疾病欲以國授其弟難當難當固辭請立玄子保宗而輔之玄許之玄卒保宗立難當妻姚氏勸難當自立難當乃廢保宗自稱都督雍涼秦三州諸軍事征西大將軍開府儀同三司秦州刺史武都王|{
	為後保宗難當争國張本雍於用翻}
河西王蒙遜遣使送穀三十萬斛以贖世子興國于秦|{
	使疏吏翻}
秦王暮末不許蒙遜乃立興國母弟菩提為世子|{
	蒙遜取佛書以名其子梵言菩提華言王道也菩薄乎翻}
暮末以興國為散騎常侍|{
	散悉亶翻騎奇寄翻下同}
以其妹平昌公主妻之|{
	妻七細翻}
八月魏主至漠南聞高車東部屯已尼陂|{
	北史烏洛侯國西北二十日行有于已尼大水所謂北海也烏洛侯直濡源西北已尼陂又當在其西北也}
人畜甚衆去魏軍千餘里遣左僕射安原等將萬騎擊之高車諸部迎降者數十萬落|{
	將即亮翻降戶江翻下同}
獲馬牛羊百餘萬冬十月魏主還平城徙柔然高車降附之民于漠南東至濡源|{
	濡乃官翻}
西暨五原隂山三千里中使之耕牧而收其貢賦命長孫翰劉絜安原及侍中代人古弼同鎮撫之自是魏之民間馬牛羊及氊皮為之價賤|{
	為于偽翻下必為同}
魏主加崔浩侍中特進撫軍大將軍以賞其謀畫之功浩善占天文常置銅鋌於酢器中|{
	酢與醋同倉故翻}
夜有所見即以鋌畫紙作字以記其異魏主每如浩家問以災異或倉猝不及束帶奉進疏食不暇精美|{
	疏麤也食祥吏翻}
魏主必為之舉筯或立嘗而還|{
	嘗口識其味也}
魏主嘗引浩出入卧内從容謂浩曰卿才智淵博事朕祖考著忠三世|{
	從干容翻道武明元及帝為三世}
故朕引卿以自近|{
	近其靳翻}
卿宜盡忠規諫勿有所隱朕雖或時忿恚不從卿言|{
	恚於避翻}
然終久深思卿言也嘗指浩以示新降高車渠帥曰汝曹視此人尫纎懦弱不能彎弓持矛|{
	尫弱也纎細也帥所類翻尫烏黄翻}
然其胷中所懷乃過于兵甲朕雖有征伐之志而不能自決前後有功皆此人所教也又敇尚書曰凡軍國大計汝曹所不能決者皆當咨浩然後施行 秦王暮末之弟軻殊羅烝於文昭王左夫人禿髪氏|{
	下淫上曰烝}
暮末知而禁之軻殊羅懼與叔父什寅謀殺暮末奉沮渠興國以奔河西|{
	沮子余翻}
使禿髪氏盜門鑰鑰誤門者以告暮末暮末悉收其黨殺之而赦軻殊羅執什寅鞭之什寅曰我負汝死不負汝鞭暮末怒刳其腹投尸于河 夏主少凶暴無賴不為世祖所知是月畋于隂槃|{
	少詩照翻隂槃縣漢屬安定晉屬京兆魏收地形志屬平原郡注又見前}
登苛藍山|{
	五代志平涼郡平涼縣有苛藍山漢涇陽縣故城在平涼縣南}
望統萬城泣曰先帝若以朕承大業者豈有今日之事乎 十一月己丑朔日有食之不盡如鈎星晝見至晡方没河北地闇|{
	見賢遍翻}
魏主西巡至柞山|{
	柞則洛翻}
十二月河西王蒙遜吐谷渾王慕璝皆遣使入貢|{
	璝古回翻使疏吏翻}
是歲魏内都大官中山文懿公李先青冀二州刺史安同皆卒先年九十五|{
	李先自燕降魏見一百八卷晉孝武太元二十一年}
秦地震野草皆自反七年春正月癸巳以吐谷渾王慕璝為征西將軍沙州刺史隴西公 庚子魏主還宫壬寅大赦癸卯復如廣甯臨溫泉|{
	復扶又翻}
二月丁卯魏陽平威王長孫翰卒戊辰魏主還宫 帝自踐位以來有恢復河南之志三月戊子詔簡甲卒五萬給右將軍到彦之統安北將軍王仲德兖州刺史竺靈秀舟師入河又使驍騎將軍段宏將精騎八千直指虎牢|{
	驍堅堯翻騎奇寄翻宏將即亮翻}
豫州刺史劉德武將兵一萬繼進後將軍長沙王義欣將兵三萬監征討諸軍事|{
	監工銜翻}
義欣道憐之子也|{
	道憐武帝之弟}
先遣殿中將軍田奇使於魏|{
	使疏吏疏}
告魏主曰河南舊是宋土中為彼所侵|{
	魏取河南見一百十九卷營陽王景平元年}
今當脩復舊境不關河北魏主大怒曰我生髪未燥已聞河南是我地此豈可得必若進軍今當權歛戍相避須冬寒地浄河冰堅合自更取之甲午以前南廣平太守尹冲為司州刺史|{
	江左僑置廣平郡于襄陽宋以朝陽縣境為實土屬雍州}
長沙王義欣出鎮彭城為衆軍聲援以游擊將軍胡藩戍廣陵行府州事 壬寅魏封赫連昌為秦王 魏有新徙敇勒千餘家苦于將吏侵漁|{
	將即亮翻}
出怨言期以草生馬肥亡歸漠北尚書令劉絜左僕射安原奏請及河冰未解徙之河西向春冰解使不得北遁魏主曰此曹習俗放散日久譬如囿中之鹿急則奔突緩之自定吾區處自有道不煩徙也|{
	處昌呂翻}
絜等固請不已乃聽分徙三萬餘落于河西西至白鹽池|{
	五原郡有白鹽池黑鹽池唐置鹽州以此得名}
敇勒皆驚駭曰圈我于河西欲殺我也|{
	圈其卷翻又其權翻}
謀西奔涼州劉絜屯五原河北|{
	水經注河水自朔方屈南過五原縣西}
安原屯悦拔城以備之癸卯敇勒數千騎叛北走絜追討之走者無食相枕而死|{
	枕之任翻}
魏南邊諸將|{
	將即亮翻下同}
表稱宋人大嚴將入寇請兵三萬先其未逆擊之|{
	先悉薦翻}
足以挫其鋭氣使不敢深入因請悉誅河北流民在境上者以絶其鄉導|{
	鄉讀曰嚮}
魏主使公卿議之皆以為當然|{
	當然猶言當如此也}
崔浩曰不可南方下濕|{
	天地之性西北高而東南下故東南之地卑濕沮洳}
入夏之後水潦方降草木蒙密地氣鬱蒸易生疾癘不可行師且彼既嚴備則城守必固|{
	易以豉翻守式又翻下戍守同}
留屯久攻則糧運不繼分軍四掠則衆力單寡無以應敵以今擊之未見其利彼若果能北來宜待其勞倦秋涼馬肥因敵取食徐往擊之此萬全之計也朝廷羣臣及西北守將從陛下征伐西平赫連|{
	事見上卷四年}
北破蠕蠕|{
	事見上年}
多獲美女珍寶牛馬成羣南邊諸將聞而慕之亦欲南鈔以取資財|{
	鈔楚交翻}
皆營私計為國生事不可從也魏主乃止諸將復表南寇已至|{
	為于偽翻復扶又翻乃復復叛同}
所部兵少|{
	少詩沼翻}
乞簡幽州以南勁兵助己戍守及就漳水造船嚴備以拒之|{
	欲就漳水造船分布河津以備宋也}
公卿皆以為宜如所請并署司馬楚之魯軌韓延之等為將帥使招誘南人|{
	將即亮翻帥所類翻誘音酉}
浩曰非長策也楚之等皆彼所畏忌今聞國家悉幽州以南精兵大造舟艦隨以輕騎|{
	艦戶黯翻騎奇寄翻}
謂國家欲存立司馬氏誅除劉宗必舉國震駭懼於滅亡當悉精鋭并心竭力以死争之則我南邊諸將無以禦之今公卿欲以威力却敵乃所以速之也張虚聲而召實害此之謂矣故楚之之徒往則彼來止則彼息其勢然也且楚之等皆纎利小才止能招合輕薄無賴而不能成大功徒使國家兵連禍結而已昔魯軌說姚興以取荆州至則敗散|{
	事見一百十七卷晉安帝義熙十二年說輸芮翻}
為蠻人掠賣為奴終于禍及姚泓此已然之效也魏主未以為然浩乃復陳天時|{
	復扶又翻}
以為南方舉兵必不利曰今兹害氣在揚州一也庚午自刑先者傷二也|{
	揚州於辰在丑而是歲在午丑為金庫午為火旺以火害金故害氣在揚州歲在庚午庚金也午火也以火尅金故為自刑}
日食晝晦宿值斗牛三也熒惑伏于翼軫主亂及喪四也太白未出進兵者敗五也|{
	去年十一月朔日食於星紀之分宿値斗牛熒惑罰星也所居之宿國受殃為死喪寇亂翼軫楚之分野屬荆州太白未出不利進兵太白兵象也宿音秀}
夫興國之君先脩人事次進地利後觀天時故萬舉萬全今劉義隆新造之國人事未洽災變屢見|{
	見賢遍翻}
天時不協舟行水涸地利不盡三者無一可而義隆行之必敗無疑魏主不能違衆言乃詔冀定相三州造船三千艘|{
	魏道武帝天興四年置相州於鄴相息亮翻}
簡幽州以南戍兵集河上以備之 秦乞伏什寅母弟前將軍白養鎮衛將軍去列以什寅之死有怨言秦王暮末皆殺之|{
	暮末淫刑以逞衆叛親離不亡得乎}
夏四月甲子魏主如雲中 勑勒萬餘落復叛走|{
	復扶又翻}
魏主使尚書封鐵追討滅之 六月己卯以氐王楊難當為冠軍將軍秦州刺史武都王|{
	冠古玩翻}
魏主使平南大將軍丹陽王大毗屯河上以司馬楚之為安南大將軍封琅邪王屯潁川以備宋 吐谷渾王慕璝將其衆萬八千襲秦定連|{
	璝古回翻將即亮翻}
秦輔國大將軍段暉等擊走之 到彦之自淮入泗水滲|{
	滲所禁翻說文曰水下漉為滲}
日行纔十里自四月至秋七月始至須昌|{
	須昌縣前漢屬東郡後漢晉屬東平郡杜佑曰鄆州古須句國漢為東平國地治須昌縣漢無鹽故城在今縣東東平國故城亦在縣東}
乃泝河西上|{
	上時掌翻}
魏主以河南四鎮兵少命諸軍悉收衆北渡|{
	四鎮金墉虎牢滑臺碻磝少詩沼翻}
戊子魏碻磝戍兵棄城去戊戌滑臺戍兵亦去庚子魏主以大鴻臚陽平公杜超為都督冀定相三州諸軍事太宰進爵陽平王鎮鄴為諸軍節度超密太后之兄也|{
	臚陵如翻冀州漢末所置治信都定州春秋鮮虞國戰國為中山國後燕慕容氏都中山後魏道武帝滅之於中山置安州天興三年改定州相州春秋晉東陽之地戰國時為魏之鄴邑晉時趙王石虎自襄國徙都之魏道武滅後燕至鄴欲立州訪於羣下對者曰昔河亶甲居相宜曰相州道武從之}
庚戌魏洛陽虎牢戍兵皆棄城去到彥之留朱脩之守滑臺尹冲守虎牢建武將軍杜驥守金墉驥預之玄孫也|{
	晉初杜預有平吳之功}
諸軍進屯靈昌津列守南岸至于潼關於是司兖既平諸軍皆喜王仲德獨有憂色曰諸賢不諳北土情偽|{
	諳烏南翻}
必墮其計胡虜雖仁義不足而凶狡有餘今歛戍北歸必并力完聚若河冰既合將復南來豈可不以為憂乎|{
	復扶又翻}
甲寅林邑王范陽邁遣使入貢|{
	使疏吏翻}
自陳與交州不睦乞蒙恕宥|{
	林邑自范奴文以來世與交州交兵}
八月魏主遣冠軍將軍安頡督護諸軍擊到彦之|{
	冠古玩翻頡尸結翻}
丙寅彦之遣裨將吳興姚聳夫渡河攻冶坂與頡戰|{
	將即亮翻}
聳夫兵敗死者甚衆戊寅魏主遣征西大將軍長孫道生會丹楊王大毗屯河上以禦彦之 燕太祖寢疾|{
	燕主跋也}
召中書監申秀侍中陽哲於内殿屬以後事|{
	屬之欲翻}
九月病甚輦而臨軒命太子翼攝國事勒兵聽政以備非常宋夫人欲立其子受居惡翼聽政謂翼曰上疾將瘳奈何遽欲代父臨天下乎翼性仁弱遂還東宫日三往省疾宋夫人矯詔絶内外遣閽寺傳問而已|{
	省悉景翻鄭康成曰閽人司晨昏以啟閉者寺之言侍也}
翼及諸子大臣並不得見|{
	見賢遍翻}
唯中給事胡福獨得出入專掌禁衛福慮宋夫人遂成其謀乃言于司徒錄尚書事中山公弘弘與壯士數十人被甲入禁中|{
	被皮義翻}
宿衛皆不戰而散宋夫人命閉東閤弘家僮庫斗頭勁捷有勇力踰閤而入至于皇堂射殺女御一人|{
	射而亦翻鄭康成曰女御所謂御妻御猶進也侍也}
太祖驚懼而殂弘遂即天王位|{
	弘字文通跋之少弟}
遣人巡城告曰天降凶禍大行崩背|{
	背蒲妹翻}
太子不侍疾羣公不奔喪疑有逆謀社稷將危吾備介弟之親|{
	介大也}
遂攝大位以寧國家百官扣門入者進陛二等|{
	陛階級也謂進階也}
太子翼帥東宫兵出戰而敗兵皆潰去弘遣使賜翼死|{
	帥讀曰率使疏吏翻}
太祖有子百餘人弘皆殺之諡太祖曰文成皇帝葬長谷陵 己丑夏主遣其弟謂以代伐魏鄜城|{
	鄜城在漢上郡界魏後置敷城郡隋改曰鄜城讀與敷同}
魏平西將軍始平公隗歸等擊之|{
	隗五罪翻}
殺萬餘人謂以代遁去夏主自將數萬人邀擊隗歸于鄜城東|{
	將即亮翻}
留其弟上谷公社干廣陽公度洛孤守平涼遣使來求和|{
	使疏吏翻}
約合兵滅魏遥分河北自恒山以東屬宋以西屬夏|{
	恒戶登翻}
魏主聞之治兵將伐夏|{
	治直之翻}
羣臣咸曰劉義隆兵猶在河中|{
	言在河之中流}
捨之西行前寇未必可克而義隆乘虚濟河則失山東矣|{
	此山東謂太行恒山以東即河北之地}
魏主以問崔浩對曰義隆與赫連定遥相招引以虚聲唱和|{
	和戶卧翻}
共窺大國義隆望定進定待義隆前皆莫敢先入譬如連雞不得俱飛無能為害也臣始謂義隆軍來當屯止河中兩道北上|{
	上時掌翻}
東道向冀州西道衝鄴如此則陛下當自討之不得徐行今則不然東西列兵徑二千里一處不得數千形分勢弱以此觀之儜兒情見|{
	儜尼耕翻困也弱也見賢遍翻}
此不過欲固河自守無北度意也赫連定殘根易摧|{
	易以豉翻}
擬之必仆克定之後東出潼關席卷而前|{
	卷讀曰捲}
則威震南極江淮以北無立草矣聖策獨發非愚近所及願陛下勿疑甲辰魏主如統萬遂襲平涼以衛兵將軍王斤鎮蒲坂|{
	坂音反}
斤建之子也|{
	王建佐魏主珪取中原}
秦自正月不雨至于九月民流叛者甚衆 冬十月以竟陵王義宣為南徐州刺史猶戍石頭|{
	義宣先戍石頭而南徐州鎮京口蓋帶刺史而猶戍石頭也}
戊午立錢署鑄四銖錢 到彦之王仲德沿河置守還保東平|{
	東平郡時治須昌}
乙亥魏安頡自委粟津濟河攻金墉金墉不治既久|{
	治直之翻}
又無糧食杜驥欲棄城走恐獲罪初高祖滅秦遷其鐘虡于江南|{
	虡音巨}
有大鐘沒于洛水帝使姚聳夫將千五百人往取之驥紿之曰金墉城已修完糧食亦足所乏者人耳今虜騎南渡當相與併力禦之|{
	將即亮翻紿蕩亥翻騎奇寄翻}
大功既立牽鍾未晩聳夫從之既至見城不可守乃引去驥遂南遁丙子安頡拔洛陽殺將士五千餘人杜驥歸言於帝曰本欲以死固守姚聳夫及城遽走人情沮敗不可復禁|{
	沮在呂翻復扶又翻}
上大怒誅聳夫于夀陽聳夫勇健諸偏裨莫及也魏河北諸軍會于七女津|{
	七女津當在東平西北岸}
到彦之恐其南渡遣禆將王蟠龍泝流奪其船杜超等擊斬之安頡與龍驤將軍陸侯進攻虎牢|{
	按北史陸侯當作陸俟驤思將翻}
辛巳拔之尹冲及滎陽太守清河崔模降魏|{
	降戶江翻 考異曰宋書云模抗節不降投塹死按後魏書模仕魏為武城男宋書悞也}
秦王暮末為河西所逼遣其臣王愷烏訥闐請迎于魏|{
	闐徒賢翻又徒見翻}
魏人許以平涼安定封之暮末乃焚城邑毁寶器帥戶萬五千東如上邽|{
	帥讀曰率考異日後魏乞伏國仁傳云為赫連定所逼遣烏訥等求迎宋氐胡傳云茂蔓聞赫連定敗將家戶及興國東征欲移居上邽今從十六國春秋}
至高田谷|{
	高田谷當在南安郡界未及至上邽也}
給事黄門侍郎郭恒謀劫沮渠興國以叛事覺暮末殺之|{
	恒戶登翻}
夏主聞暮末將至兵拒之暮末留保南安其故地皆入于吐谷渾|{
	自苑川至西平枹罕皆乞伏氏故地晉孝武帝太元八年歲在癸未乞伏國仁據隴西南安亦其地也}
十一月乙酉魏主至平涼夏上谷公社干等嬰城固守魏主使赫連昌招之不下乃使安西將軍古弼等將兵趣安定|{
	趣七喻翻}
夏主自鄜城還安定將步騎二萬北救平涼與弼遇弼偽退以誘之|{
	將即亮翻騎奇寄翻誘音酉}
夏主追之魏主使高車馳擊之夏兵大敗斬首數千級夏主還走登鶉觚原|{
	鶉觚縣前漢屬北地後漢晉屬安定有鶉觚原唐天寶元年改曰靈臺縣屬涇州鶉殊倫翻觚音孤}
為方陳以自固|{
	陳讀曰陣}
魏兵就圍之 壬辰加征南大將軍檀道濟都督征討諸軍事帥衆伐魏|{
	帥讀曰率下同}
甲午魏夀光侯叔孫建汝隂公長孫道生濟河而南到彦之聞洛陽虎牢不守諸軍相繼奔敗欲引兵還殿中將軍垣護之以書諫之以為宜使竺靈秀助朱脩之守滑臺自帥大軍進擬河北且曰昔人有連年攻戰失衆乏糧猶張膽争前莫肯輕退况今青州豐穰濟漕流通|{
	濟子禮翻下入濟同}
士馬飽逸威力無損若空棄滑臺坐喪成業|{
	喪息浪翻}
豈朝廷受任之旨邪|{
	受當作授}
彦之不從護之苖之子也|{
	垣苗邊將也武帝西征長安令苖鎮河濟之會俗謂之垣苖城祖子孫三世皆著功名於邊埀}
彥之欲焚舟步走王仲德曰洛陽既陷虎牢不守自然之勢也今虜去我猶千里滑臺尚有彊兵若遽捨舟南走士卒必散當引舟入濟至馬耳谷口更詳所宜|{
	馬耳谷口即馬耳關}
彦之先有目疾至是大動且將士疾疫乃引兵自清入濟|{
	水經濟水東北過夀張縣西界安民亭南汶水從東北來注之注云濟水又北汶水注之戴延之所謂清口也郭緣生述征記曰清河首受洪水北流濟或謂清即濟也禹貢濟東北會于汶今枯渠注巨澤巨澤北則清水與汶會也京相璠曰今濟北東阿東北有故清亭即春秋所謂清者也是濟水通得清之目焉亦水色清深用兼厥稱是故燕王曰吾聞齊有清濟濁河以為固即此水也}
南至歷城焚舟棄甲步趨彭城|{
	趨七喻翻}
竺靈秀棄須昌南奔湖陸青兖大擾長沙王義欣在彭城將佐恐魏兵大至|{
	將即亮翻}
勸義欣委鎮還都義欣不從魏兵攻濟南|{
	濟南郡治歷城}
濟南太守武進蕭承之帥數百人拒之|{
	晉武帝太康二年分丹徒曲阿立武進縣屬晉陵郡南渡後屬南東海郡今奔牛青城萬歲諸鎮皆其地}
魏衆大集承之使偃兵開城門衆曰賊衆我寡奈何輕敵之甚承之曰今懸守窮城事已危急若復示弱必為所屠唯當見彊以待之耳|{
	復扶又翻見賢遍翻}
魏人疑有伏兵遂引去|{
	承之蕭道成之父也}
魏軍圍夏主數日斷其水草|{
	斷丁管翻}
人馬饑渇丁酉夏主引衆下鶉觚原魏武衛將軍丘眷擊之夏衆大潰死者萬餘人夏主中重創單騎走|{
	中竹仲翻創初良翻騎奇寄翻}
收其餘衆驅民五萬西保上邽魏人獲夏主之弟丹楊公烏視拔武陵公秃骨及公侯以下百餘人是日魏兵乘勝進攻安定夏東平公乙斗棄城犇長安驅畧數千家西犇上邽 戊戌魏叔孫建攻竺靈秀于湖陸靈秀大敗死者五千餘人建還屯范城|{
	即東平郡之范縣城也杜佑曰濮州范縣晉大夫士會之邑}
己亥魏主如安定庚子還臨平涼掘塹圍之|{
	掘其月翻塹七艷翻}
安慰初附赦秦雍之民賜復七年|{
	雍于用翻復方目翻除其賦役也}
夏隴西守將降魏|{
	將即亮翻降戶江翻}
辛丑魏安頡督諸軍攻滑臺 河西王蒙遜遣尚書郎宗舒等入貢于魏魏主與之宴執崔浩之手以示舒等曰汝所聞崔公此則是也才畧之美於今無此朕動止咨之豫陳成敗若合符契未嘗失也魏以叔孫建都督冀青等四州諸軍事|{
	魏未得青州也使建督諸軍經畧之耳}
魏尚書庫結帥騎五千迎秦王暮末|{
	魏書官氏志北方諸姓庫傉官氏後改為庫氏帥讀曰率下同騎奇寄翻下同}
秦衛將軍吉毗以為不宜内徙暮末從之庫結引還内安諸羌萬餘人叛秦推安南將軍督八郡諸軍事廣甯太守焦遺為主|{
	魏收地形志廣寧郡治隴西鄣縣甯當作寧鄣縣後漢所置唐為渭州隴西縣地}
遺不從乃劫遺族子長城護軍亮為主|{
	五代志平涼郡百泉縣後魏置長城郡}
帥衆攻南安暮末請救于氐王楊難當難當遣將軍苻獻帥騎三千救之暮末與之合擊諸羌諸羌潰亮奔還廣甯暮末進軍攻之以手令與焦遺使取亮十二月遺斬亮首出降暮末進遺號鎮國將軍秦畧陽太守弘農楊顯以郡降夏|{
	晉武帝分天水置畧陽郡降戶江翻下同}
辛酉以長沙王義欣為豫州刺史鎮夀陽夀陽土荒民散城郭頹敗盜賊公行義欣隨宜經理境内安業道不拾遺城府完實遂為盛藩芍陂久廢義欣修治隄防|{
	治直之翻}
引河水入陂溉田萬餘頃無復旱災|{
	引肥河之水入芍陂也復扶又翻}
丁卯夏上谷公社干廣陽公度洛孤出降魏克平涼關中侯豆代田得奚斤娥清等獻于魏主魏主以夏主之后賜代田命斤膝行執酒以奉代田謂斤曰全汝生者代田也賜代田爵井陘侯|{
	曹魏置關中侯有爵未有邑猶秦漢之關内侯爵級在列侯下拓拔賞豆代田自關中侯進爵井陘侯則有邑矣而亦非君有實土也陘音刑}
加散騎常侍右衛將軍領内都幢將|{
	百人為幢幢有帥柔然之法也魏幢將主三郎衛士直宿禁中者自侍中已下中散已上皆統之將即亮翻下同}
夏長安臨晉武功守將皆走關中悉入于魏魏主留巴東公延普鎮安定以鎮西將軍王斤鎮長安壬申魏主東還以奚斤為宰士使負酒食以從|{
	宰士掌膳飲以斤敗軍失身辱之也時魏有宰官尚書宰士蓋其屬也從才用翻}
王斤驕矜不法信用左右調役百姓|{
	調徒弔翻}
民不堪命南奔漢川者數千家魏主案治得實斬斤以徇|{
	治直之翻}
右將軍到彦之安北將軍王仲德皆下獄免官|{
	下戶嫁翻}
兖州刺史竺靈秀坐棄軍伏誅上見垣護之書而善之以為北高平太守|{
	南高平郡僑郡也屬南兖州北高平郡古郡也屬兖州治湖陸}
彥之之北伐也甲兵資實甚盛及敗還委棄盪盡府藏武庫為之空虛|{
	為于偽翻}
它日上與羣臣宴有荒外降人在坐|{
	自南北分治各以其封畧之外為荒外降戶江翻坐徂卧翻}
上問尚書庫部郎顧琛庫中仗猶有幾許琛詭對有十萬人仗|{
	曹魏置尚書二十三郎庫部其一也掌戎器鹵簿儀仗琛丑林翻}
上既問而悔之得琛對甚喜琛和之曾孫也|{
	顧和見九十卷晉元帝大興元年}
彭城王義康與王弘並錄尚書義康意猶怏怏|{
	怏於兩翻}
欲得揚州形於辭旨以弘弟曇首居中為上所親委愈不悦弘以老病屢乞骸骨曇首自求吳郡上皆不許義康謂人曰王公久病不起神州詎宜卧治|{
	治直之翻}
曇首勸弘減府中文武之半以授義康上聽割二千人義康乃悦|{
	曇徒含翻}


資治通鑑卷一百二十一
