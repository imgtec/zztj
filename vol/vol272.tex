資治通鑑卷二百七十二 宋 司馬光 撰

胡三省 音註

後唐紀一|{
	昭陽協洽一年晉王李克用始封於晉存勗嗣封及即大位自以繼唐有天下國遂號曰唐通鑑曰後唐以别長安之唐}


莊宗光聖神閔孝皇帝上

|{
	諱存勗晉王克用長子也其先本號朱邪出於西突厥處月别部居沙陀磧自號沙陀而以朱邪為姓至執宜歸唐執宜子赤心有功於唐賜姓名李國昌編於屬籍克用赤心之子也五代會要曰執宜沙陀府都督拔野古之六代孫歐陽史曰拔野古朱邪同時人非其始祖}


同光元年|{
	是年四月始即位改元}
春二月晉王下教置百官于四鎮判官中選前朝士族欲以為相|{
	四鎮河東魏博易定鎮冀朝直遙翻相息亮翻下同}
河東節度判官盧質為之首質固辭|{
	盧質慢罵晉王諸弟又能辭相位于惟新之朝是必有見也}
請以義武節度判官豆盧革河東觀察判官盧程為之王即召革程拜行臺左右丞相 |{
	考異曰薛史作盧澄今從實錄莊宗列傳}
以質為禮部尚書 梁主遣兵部侍郎崔協等冊命吳越王鏐為吳越國王丁卯鏐始建國儀衛名稱多如天子之制|{
	稱尺證翻}
謂所居曰宫殿府署曰朝廷教令下統内曰制敕將吏皆稱臣|{
	將即亮翻}
惟不改元表疏稱吳越國而不言軍|{
	以建國不肯復稱鎮海鎮東軍節度}
以清海節度使兼侍中傳瓘為鎮海鎮東留後總軍府事置百官有丞相侍郎郎中員外郎客省等使|{
	使疏吏翻 考異曰十國紀年鏐功臣諸子領節制皆署而後請命居室服御窮極侈靡末年荒恣尤甚錢氏據兩浙逾八十年外厚貢獻内事奢僭地狹民衆賦斂苛暴雞魚卵菜纎悉收取斗升之逋罪至鞭背每笞一人則諸案吏各持其簿列于庭先唱一簿以所負多少為數笞已次吏復唱而笞之盡諸簿乃止少者猶笞數十多者至五百餘訖于國除人苦其政吳越備史稱鏐節儉衣衾用紬布常膳惟甆漆器寢帳壞文穆夫人欲易以青繒鏐不許嘗歲除夜會子孫鼓琴未數曲止之曰聞者以我為長夜之飲遂罷錢易家話稱鏐公宴不貳羮胾衣必三澣然後易劉恕以為錢元瓘子信撰吳越備史備史遺事忠懿王勲業志戊申英政錄弘倧子易撰家話俶子惟演撰錢氏慶系圖譜家王故事秦國主貢奉錄故吳越五王行事失實尤多虚美隱惡甚于它國按錢鏐起於貧賤知民疾苦必不至窮極侈靡其奢汰暴斂之事蓋其子孫所為也今從家話}
李繼韜雖受晉王命為安義留後|{
	事見上卷上年}
終不自安幕僚魏琢牙將申蒙復從而間之|{
	復扶又翻間古莧翻}
曰晉朝無人|{
	朝直遙翻}
終為梁所併耳會晉王置百官三月召監軍張居翰|{
	張居翰唐昭宗時為范陽監軍天復中大誅宦者節度使劉仁恭匿居翰于大安山之北谿以免其後梁兵攻仁恭遣居翰從晉王攻梁潞州以牽其兵晉遂取潞州因以居翰為昭義監軍}
節度判官任圜赴魏州|{
	任音壬}
琢蒙復說繼韜曰|{
	說式芮翻}
王急召二人情可知矣繼韜弟繼遠亦勸繼韜自託於梁繼韜乃使繼遠詣大梁請以澤潞為梁臣梁主大喜更命安義軍曰匡義|{
	更工衡翻}
以繼韜為節度使同平章事繼韜以二子為質|{
	質音致}
安義舊將裴約戍澤州泣諭其衆曰余事故使踰二紀|{
	故使謂繼韜父嗣昭也十二年為一紀使疏吏翻}
見其分財享士志滅仇讐不幸捐館|{
	死謂之捐館言弃捐館舍而逝也}
柩猶未葬而郎君遽背君親|{
	弃君事讐不惟背君亦背親之教命背蒲昧翻}
吾寧死不能從也遂據州自守梁主以其驍將董璋為澤州刺史將兵攻之繼韜散財募士堯山人郭威往應募威使氣殺人繫獄繼韜惜其才勇而逸之|{
	郭威事始此歐史云威嘗遊于市市有屠者以勇服其市人威醉呼屠者使進几割肉割不如法威叱之屠者披其腹示之曰爾勇者能殺我乎威即前取刀刺殺之一市皆驚而威自如為吏所執繼韜縱使亡去}
契丹寇幽州晉王問帥於郭崇韜|{
	帥所類翻}
崇韜薦横海節度使李存審時存審臥病己卯徙存審為盧龍節度使輿疾赴鎮以蕃漢馬步副總管李嗣源領横海節度使|{
	李嗣源時從晉王總兵使領横海節}
晉王築壇於魏州牙城之南夏四月己巳升壇祭告上帝遂即皇帝位|{
	曰遂者先有即位之心而今遂其事也}
國號大唐大赦改元|{
	因唐國號改天祐年號為同光}
尊母晉國太夫人曹氏為皇太后嫡母秦國夫人劉氏為皇太妃|{
	君子以是知帝之不終}
以豆盧革為門下侍郎盧程為中書侍郎並同平章事郭崇韜張居翰為樞密使|{
	徐無黨曰樞密使唐故事宦者為之其職甚微至此始參用士人而與宰相權任均矣余按唐末兩樞密與兩神策中尉號為四貴其職非甚微也特專用宦者為之耳項安世曰唐於政事堂後列五房有樞密房以主曹務則樞密之要宰相主之未始它付其後寵任宦人始以樞密歸之内侍}
盧質馮道為翰林學士張憲為工部侍郎租庸使|{
	宋白曰租庸使自天寶三年韋堅始}
又以義武掌書記李德休為御史中丞德休絳之孫也|{
	李絳相唐憲宗有直聲}
詔盧程詣晉陽冊太后太妃初太妃無子性賢不妬忌太后為武皇侍姬太妃嘗勸武皇善待之|{
	晉王克用諡武皇帝}
太后亦自謙退由是相得甚歡及受冊太妃詣太后宫賀有喜色太后忸怩不自安|{
	忸女六翻怩女夷翻}
太妃曰願吾兒享國久長吾輩獲沒於地園陵有主餘何足言因相向歔欷|{
	歔音虚欷音希又許旣翻}
豆盧革盧程皆輕淺無它能上以其衣冠之緒霸府元僚故用之|{
	按歐史豆盧為世名族革父瓚為唐舒州刺史唐末之亂革避地中山為王處直判官盧程不知其家世何人也唐昭宗時舉進士為鹽鐵出使巡官唐末避亂變服為道士遊燕趙間豆盧革為義武節度判官盧汝弼為河東節度副使二人皆故唐名族與程門地相等因共薦為河東節度推官帝議擇相而唐公卿故家遭亂喪亡且盡盧汝弼蘇循已死盧質又辭故用革程興王之君命相如此天下事可知矣}
初李紹宏為中門使郭崇韜副之至是自幽州召還|{
	梁貞明五年李紹宏出幽州事見上卷}
崇韜惡其舊人位在已上|{
	惡烏路翻}
乃薦張居翰為樞密使以紹宏為宣徽使紹宏由是恨之|{
	唐制宣微使在樞密使之下且權任不及遠甚}
居翰和謹畏事軍國機政皆崇韜掌之支度務使孔謙自謂才能勤効應為租庸使衆議以謙人微地寒不當遽總重任|{
	孔謙魏州孔目吏也晉王得魏州以為支度務使}
故崇韜薦張憲以謙副之謙亦不悦以魏州為興唐府建東京|{
	薛居正五代史晉王即位升魏州為東京興唐府改元城為興唐縣貴鄉為廣晉縣}
又於太原府建西京又以鎮州為真定府建北都以魏博節度判官王正言為禮部尚書行興唐尹太原馬步都虞候孟知祥為太原尹充西京副留守潞州觀察判官任圜為工部尚書兼真定尹充北京副留守|{
	京當作都}
皇子繼岌為北都留守興聖宫使判六軍諸衛事|{
	按後唐洛陽有西宫興聖宫此時未得洛陽當以魏州府舍為興聖宫宋白曰唐莊宗即位於魏州宰相豆盧革因進擬為興聖宫以皇子繼岌為興聖宫使}
時唐國所有凡十三節度五十州|{
	十三節度天雄成德義武横海盧龍大同振武鴈門河東護國晉絳安國昭義五十州魏博具澶相鄆洺磁鎮冀深趙易祁定滄景德瀛莫幽涿檀薊順營平蔚朔雲應新媯儒武忻代嵐石憲麟府并汾慈隰澤潞沁遼凡五十州而昭義領澤潞二州已附于梁止有十二節度四十八州耳}
閏月追尊皇曾祖執宜曰懿祖昭烈皇帝祖國昌曰獻祖文皇帝考晉王曰太祖武皇帝立宗廟於晉陽以高祖太宗懿宗昭宗洎懿祖以下為七室|{
	唐廟四親廟三}
甲午契丹寇幽州至易定而還|{
	還從宣翻又如字}
時契丹屢入寇鈔掠饋運|{
	鈔楚交翻}
幽州食不支半年衛州為梁所取潞州内叛人情岌岌以為梁未可取帝患之會鄆州將盧順密來奔先是梁天平節度使戴思遠屯楊村|{
	戴思遠屯楊村事始上卷梁貞明五年先悉薦翻}
留順密與巡檢使劉遂嚴都指揮使燕顒守鄆州|{
	燕音煙姓也顒魚容翻}
順密言於帝曰鄆州守兵不滿千人遂嚴顒皆失衆心可襲取也郭崇韜等皆以為懸軍遠襲萬一不利虛弃數千人順密不可從帝密召李嗣源於帳中謀之曰梁人志在吞澤潞不備東方若得東平則潰其心腹東平果可取乎|{
	鄆州本東平郡}
嗣源自胡柳有度河之慙|{
	事見二百七十卷梁貞明四年}
常欲立奇功以補過對曰今用兵歲久生民疲弊苟非出奇取勝大功何由可成臣願獨當此役必有以報帝悦壬寅遣嗣源將所部精兵五千自德勝趣鄆州比及楊劉|{
	趣七喻翻比必利翻按九域志鄆州東阿縣有楊劉鎮臨河津東阿東南至鄆州六十里以下文夜度河觀之則李嗣源之兵自德勝北城而東循河北岸而行至楊劉度口}
日已暮隂雨道黑將士皆不欲進高行周曰此天贊我也彼必無備夜度河至城下鄆人不知|{
	此是楊劉取徑道至鄆州城下不經東阿縣治所}
李從珂先登殺守卒啟關納外兵進攻牙城城中大擾癸卯旦嗣源兵盡入遂拔牙城劉遂嚴燕顒奔大梁嗣源禁焚掠撫吏民執知州事節度副使崔簹判官趙鳳送興唐|{
	簹都郎翻唐於魏州置興唐府}
帝大喜曰總管真奇才吾事集矣即以嗣源為天平節度使梁主聞鄆州失守大懼斬劉遂嚴燕顒於市罷戴思遠招討使降授宣化留後|{
	歐陽職方考梁置宣化軍於鄧州}
遣使詰讓北面諸將段凝王彦章等趣令進戰|{
	詰去吉翻趣讀曰促}
敬翔知梁室已危以繩内靴中入見梁主曰|{
	見賢遍翻}
先帝取天下不以臣為不肖所謀無不用今敵勢益彊而陛下棄忽臣言臣身無用不如死引繩將自經梁主止之問所欲言翔曰事急矣非用王彦章為大將不可救也|{
	敬翔以王彦章一時健鬬而取之耳觀其用兵無遠略烏足以救梁之亡乎}
梁主從之以彦章代思遠為北面招討使仍以段凝為副帝聞之自將親軍屯澶州命蕃漢馬步都虞候朱守殷守德勝戒之曰王鐵槍勇決乘憤激之氣必來唐突宜謹備之|{
	唐韻唐突作傏突又作盪突唐盪義同也史言晉王善於料王彦章不善於用人守德勝}
守殷王幼時所役蒼頭也|{
	歐史曰朱守殷少事帝為奴名曰會兒帝讀書會兒常侍左右}
又遣使遺吳王書|{
	遺唯季翻}
告以己克鄆州請同舉兵擊梁五月使者至吳徐温欲持兩端將舟師循海而北助其勝者嚴可求曰若梁人邀我登陸為援何以拒之温乃止梁主召問王彦章以破敵之期彦章對曰三日左右皆失笑|{
	自大梁出師拒晉三日不能至河上故笑其言}
彦章出兩日馳至滑州|{
	九域志大梁北至滑州二百一十里}
辛酉置酒大會隂遣人具舟於楊村夜命甲士六百皆持巨斧載冶者具鞲炭乘流而下|{
	楊村順流趣德勝水程十八里耳鞲蒲拜翻韋囊也鼓以吹火}
會飲尚未散彦章陽起更衣引精兵數千循河南岸趨德勝|{
	更工衡翻趣七喻翻}
天微雨朱守殷不為備舟中兵舉鏁燒斷之因以巨斧斬浮橋而彦章引兵急擊南城浮橋斷南城遂破時受命適三日矣守殷以小舟載甲士濟河救之不及彦章進攻潘張麻家口景店諸寨皆拔之|{
	潘張二姓同居一村因以為名店都念翻崔豹古今註曰店所以置貨鬻物也有姓景者先嘗設店於其地因以為名凡此皆河津之要晉人立寨守之}
聲勢大振帝遣宦者焦彦賓急趨楊劉|{
	趨七喻翻}
與鎮使李周固守命守殷弃德勝北城撤屋為栰|{
	栰音伐大曰栰小曰桴}
載兵械浮河東下助楊劉守備徙其芻糧薪炭于澶州所耗失殆半王彦章亦撤南城屋材浮河而下各行一岸每遇灣曲輒於中流交鬭飛矢雨集或全舟覆沒一日百戰互有勝負比及楊劉|{
	比必寐翻}
殆亡士卒之半|{
	此謂自德勝浮河東下之士卒也}
己巳王彦章段凝以十萬之衆攻楊劉百道俱進晝夜不息連巨艦九艘横亘河津以絶援兵|{
	艦戶黯翻艘蘇遭翻}
城垂陷者數四賴李周悉力拒之與士卒同甘苦彦章不能克退屯城南為連營以守之楊劉告急于帝請日行百里以赴之|{
	帝在澶州距楊劉幾二百里}
帝引兵救之曰李周在内何憂日行六十里不廢畋獵六月乙亥至楊劉梁兵塹壘重複嚴不可入|{
	重直龍翻}
帝患之問計于郭崇韜對曰今彦章據守津要意謂可以坐取東平苟大軍不南則東平不守矣臣請築壘於博州東岸以固河津旣得以應接東平又可以分賊兵勢但慮彦章詗知|{
	詗古永翻又翾正翻}
徑來薄我城不能就願陛下募敢死之士日令挑戰以綴之|{
	令力經翻挑徒了翻}
苟彦章旬日不東則城成矣時李嗣源守鄆州河北聲問不通人心漸離不保朝夕會梁右先鋒指揮使康延孝密請降於嗣源延孝者太原胡人|{
	歐史曰康延孝代北人為太原軍卒有罪亡命奔梁}
有罪亡奔梁時隸段凝麾下嗣源遣押牙臨漳范延光送延孝蠟書詣帝延光因言於帝曰楊劉控扼已固梁人必不能取請築壘馬家口以通鄆州之路帝從之遣崇韜將萬人夜發倍道趨博州至馬家口度河築城晝夜不息|{
	馬家口謂博州東岸也郭崇韜自楊劉夜發倍道而行恐衆人知之故也}
帝在楊劉與梁人書夜苦戰崇韜築新城凡六日王彦章聞之將兵數萬人馳至戊子急攻新城連巨艦十餘艘於中流以絶援路時板築僅畢城猶卑下沙土疏惡未有樓櫓及守備崇韜慰勞士卒以身先之|{
	先悉薦翻}
四面拒戰遣間使告急於帝帝自楊劉引大軍救之陳於新城西岸城中望之增氣大呼叱梁軍梁人斷紲斂艦帝艤舟將度|{
	間古莧翻使疏吏翻陳讀曰陣呼火故翻斷丁管翻紲息列翻索也艤魚倚翻亦作檥漢書引義整舟向岸曰檥}
彦章解圍退保鄒家口|{
	麻家口馬家口鄒家口皆沿河津渡之口亦因其土人所居之姓以為地名}
鄆州奏報始通李嗣源密表請正朱守殷覆軍之罪帝不從|{
	帝不誅朱守殷以成絳霄殿之禍}
秋七月丁未帝引兵循河而南彦章等弃鄒家口復趨楊劉甲寅遊奕將李紹興敗梁遊兵於清丘驛南|{
	敗補邁翻春秋晉宋曹衛同盟於清丘杜預註曰清丘今在濮陽縣東南此因古地名以名驛也}
段凝以為唐兵已自上流渡驚駭失色面數彦章尤其深入|{
	段凝聞清丘驛之敗以為唐兵已自上流度河逼汴而彦章等方與唐相持於下流責其深入鄆州之境無救于大梁之危也史言段凝内有所恃而陵主帥數所具翻}
乙卯蜀侍中魏王宗侃卒 戊午帝遣騎將李紹榮直抵梁營擒其斥候梁人益恐又以火栰焚其連艦|{
	連艦即列於河流以斷援兵者}
王彦章等聞帝引兵已至鄒家口己未解楊劉圍走保楊村唐兵追擊之復屯德勝梁兵前後急攻諸城士卒遭矢石溺水暍死者且萬人|{
	暍於歇翻傷署而死也}
委弃資糧鎧仗鍋幕動以千計|{
	鍋古禾翻釡也王彦章掩晉人之不備取勝于一時持久則敗矣使梁能終用之亦未必成功}
楊劉比至圍解|{
	比必利翻}
城中無食已三日矣 王彦章疾趙張亂政及為招討使謂所親曰待我成功還|{
	還從宣翻又如字}
當盡誅姦臣以謝天下趙張聞之私相謂曰我輩寧死於沙陀不可為彦章所殺相與恊力傾之段凝素疾彦章之能而諂附趙張在軍中與彦章動相違戾百方沮橈之|{
	沮在呂翻橈奴教翻}
惟恐其有功濳伺彦章過失以聞於梁主每捷奏至趙張悉歸功於凝由是彦章功竟無成及歸楊村梁主信讒猶恐彦章旦夕成功難制徵還大梁 |{
	考異曰歐陽史云末帝罷彦章以段凝為招討使彦章馳至京師入見以笏畫地陳勝敗之迹巖等諷有司劾彦章不恭勒還第今從實錄}
使將兵會董璋攻澤州甲子帝至楊劉勞李周曰微卿善守吾事敗矣|{
	勞力到翻}
中書侍郎同平章事盧程以私事干興唐府府吏不能應鞭吏背光祿卿兼興唐少尹任團圜之弟帝之從姊壻也|{
	從才用翻}
詣程訴之程罵曰公何等蟲豸欲倚婦力邪|{
	豸馳爾翻爾雅曰有足曰蟲無足曰豸}
團訴於帝帝怒曰朕誤相此癡物|{
	相息亮翻}
乃敢辱吾九卿欲賜自盡盧質力救之乃貶右庶子 裴約遣間使告急於帝帝曰吾兄不幸生此梟獍|{
	李嗣昭義兒也以齒於帝為兄獍讀如鏡}
裴約獨能知逆順顧謂北京内牙馬步軍都指揮使李紹斌曰澤州彈丸之地朕無所用|{
	彈丸之地言其小也自并路窺懷洛則澤州為要地帝志在自東平取大梁故云然彈徒旦翻}
卿為我取裴約以來|{
	為于偽翻}
八月壬申紹斌將甲士五千救之未至城已陷約死帝深惜之甲戌帝自楊劉還興唐 梁主命於滑州決河東注

曹濮及鄆以限唐兵|{
	濮博木翻}
初梁主遣段凝監大軍於河上敬翔李振屢請罷之|{
	監古銜翻 考異曰歐陽史以為太祖時事按晉人取魏博然後與梁以河為境故常以大兵守之太祖時未也就使當時曾屯軍河上亦未繫社稷之安危也况太祖時振言聽計從均王時始疎斤此必均王時事 也旣不知其的在何時故因凝任招討使而見之}
梁主曰凝未有過振曰俟其有過則社稷危矣至是凝厚賂趙張求為招討使翔振力爭以為不可趙張主之竟代王彦章為北面招討使於是宿將憤怒士卒亦不服天下兵馬副元帥張宗奭言於梁主曰臣為副元帥雖衰朽猶足為陛下扞禦北方段凝晚進功名未能服人衆議詾詾|{
	足為于偽翻訽許拱翻又音凶義與洶洶同}
恐貽國家深憂|{
	張宗奭此言必敬翔等欲借其重以覺寤梁主}
敬翔曰將帥繫國安危今國勢已爾|{
	言國勢之危已如此也}
陛下豈可尚不留意邪梁主皆不聽|{
	為段凝誤梁張本}
戊子凝將全軍五萬營于王村自高陵津濟河|{
	新唐書地理志澶州臨黄縣東南有盧津關一名高陵津王村亦因土人王氏聚居之地為名將即亮翻}
剽掠澶州諸縣至于頓丘|{
	剽匹妙翻澶時連翻}
梁主命王彦章將保鑾騎士及他兵合萬人屯兖鄆之境謀復鄆州以張漢傑監其軍庚寅帝引兵屯朝城|{
	宋白日朝城縣屬魏州本漢東武陽郡其後為縣唐武后改為武聖開元七年改為朝城九域志朝城縣在魏州東南八十里}
戊戌康延孝帥百餘騎來奔|{
	帥讀曰率騎奇寄翻}
帝解所御錦袍玉帶賜之以為南面招討都指揮使領博州刺史帝屏人問延孝以梁事|{
	屏必郢翻又卑正翻}
對曰梁朝地不為狹兵不為少|{
	朝直遥翻少詩沼翻下同}
然迹其行事終必敗亡何則主旣暗懦趙張兄弟擅權内結宫掖外納貨賂官之高下唯視賂之多少|{
	如温昭圖以納賂而得名藩段凝以納賂而得大將之類}
不擇才德不校勲勞段凝智勇俱無一旦居王彦章霍彦威之右自將兵以來專率斂行伍|{
	斂力贍翻又上聲行戶剛翻}
以奉權貴每出一軍不能專任將帥常以近臣監之|{
	如張漢傑監王彦章軍之類帥所類翻}
進止可否動為所制近又聞欲數道出兵令董璋引陜虢澤潞之兵自石會關趣太原|{
	陜失冉翻趣七喩翻}
霍彦威以汝洛之兵自相衛邢洺寇鎮定|{
	相息亮翻}
王彦章張漢傑以禁軍攻鄆州段凝杜晏球以大軍當陛下決以十月大舉臣竊觀梁兵聚則不少分則不多願陛下養勇蓄力以待其分兵帥精騎五千自鄆州直抵大梁擒其偽主旬月之間天下定矣|{
	康延孝之計與李嗣源郭崇韜所見略同帥讀曰率}
帝大悦 蜀主以文思殿大學士韓昭|{
	唐末之遷洛也改保寜殿為文思殿蜀蓋襲唐殿名}
内皇城使潘在迎 |{
	考異曰在迎先為内皇城使貶雅州蜀主北巡為馬步使今不知何官故且稱其舊官}
武勇軍使顧在珣為狎客陪侍遊宴與宫女雜坐或為豔歌相唱和或談嘲謔浪鄙俚䙝慢無所不至蜀主樂之|{
	史言蜀主有陳後主之風豔以贍翻和戶卧翻嘲陟交翻謔迄却翻俚音里䙝息列翻樂音洛}
在珣彦朗之子也|{
	顧彦朗唐昭宗時帥東川}
時樞密使宋光嗣等專斷國事|{
	斷丁亂翻}
恣為威虐務徇蜀主之欲以盜其權宰相王鍇庾傳素等|{
	鍇口駭翻}
各保寵祿無敢規正潘在迎每勸蜀主誅諫者無使謗國嘉州司馬劉贊獻陳後主三閣圖|{
	陳三閣見一百七十六卷長城公至德二年}
并作歌以諷賢良方正蒲禹卿對策語極切直蜀主雖不罪亦不能用也九月庚戌蜀主以重陽宴近臣於宣華苑|{
	重陽九月九日也九陽數也九月而又九日故曰重陽重直龍翻按路振九國志蜀主乾德元年改龍躍池為宣華苑}
酒酣嘉王宗壽乘間極言社稷將危流涕不已韓昭潘在迎曰嘉王好酒悲|{
	間古莧翻人有醉後而涕泣者俗謂之酒悲好呼到翻}
因諧笑而罷 帝在朝城梁段凝進至臨河之南|{
	魏州臨河縣南也隋志開皇六年置臨河縣新唐書地理志貞觀十七年省澶水縣入焉澶水即澶淵避高祖諱更淵為水臨河澶淵其地蓋相近也宋白曰臨河縣本東黎縣魏孝昌中分汲郡置黎陽郡領黎陽東黎頓丘三縣此即東黎也隋開皇五年置臨河縣九域志臨河縣在澶州西六十里}
澶西相南日有寇掠|{
	澶州之西相州之南也}
自德勝失利以來喪芻糧數百萬租庸副使孔謙暴斂以供軍民多流亡租税益少倉廩之積不支半歲|{
	喪息浪翻斂力贍翻積子賜翻又如字}
澤潞未下盧文進王郁引契丹屢過瀛涿之南|{
	此即言梁龍德二年契丹入鎮定境}
傳聞俟草枯冰合深入為寇又聞梁人欲大舉數道入寇|{
	即康延孝之言}
帝深以為憂召諸將會議宣徽使李紹宏等皆以為鄆州城門之外皆為寇境孤遠難守有之不如無之請以易衛州及黎陽於梁|{
	梁取衛州見上卷上年貞明二年晉盡取河北獨黎陽為梁守}
與之約和以河為境休兵息民俟財力稍集更圖後舉帝不悦曰如此吾無葬地矣乃罷諸將獨召郭崇韜問之對曰陛下不櫛沐不解甲十五餘年|{
	梁太祖開平二年帝嗣晉王位始戰於夾寨至是年凡在兵間十七年櫛側瑟翻}
其志欲以雪家國之讐恥也今已正尊號河北士庶日望升平始得鄆州尺寸之地不能守而弃之安能盡有中原乎臣恐將士解體將來食盡衆散雖畫河為境誰為陛下守之|{
	誰為于偽翻}
臣嘗細訪康延孝以河南之事度已料彼|{
	度徒洛翻}
日夜思之成敗之機決在今歲梁今悉以精兵授段凝據我南鄙又決河自固|{
	段凝自酸棗決河注鄆州以限唐兵號護駕水}
謂我猝不能渡恃此不復為備|{
	復扶又翻}
使王彦章侵逼鄆州其意冀有姦人動揺變生於内耳段凝本非將才不能臨機決策無足可畏降者皆言大梁無兵|{
	根本内虚為敵所窺所謂重戰輕防未有不敗亡者也降戶江翻下同}
陛下若留兵守魏固保楊劉自以精兵與鄆州合勢長驅入汴彼城中旣空虛必望風自潰苟偽主授首則諸將自降矣不然今秋穀不登軍糧將盡若非陛下決志大功何由可成諺曰當道築室三年不成帝王應運必有天命在陛下勿疑耳帝曰此正合朕志丈夫得則為王失則為虜吾行決矣司天奏今歲天道不利深入必無功帝不聽王彦章引兵踰汶水將攻鄆州|{
	汶水過鄆城南春秋以鄆讙龜隂為汶陽之田是也汶音問}
李嗣源遣李從珂將騎兵逆戰敗其前鋒於遞坊鎮|{
	敗補邁翻 考異曰薛史作遞公鎮今從實錄}
獲將士三百人斬首二百級彦章退保中都|{
	舊唐書地理志鄆州中都縣漢平陸縣舊治殷密城在今治西三十九里天寶元年改為中都縣移於今治九域志中都縣在鄆州東南六十里近世改中都為汶上縣殷密城宋白續通典作致密城}
戊辰捷奏至朝城帝大喜謂郭崇韜曰鄆州告捷足壯吾氣己巳命將士悉遣其家屬歸興唐|{
	自朝城行營遣歸魏州}
冬十月辛未朔日有食之 帝遣魏國夫人劉氏皇子繼岌歸興唐與之訣曰事之成敗在此一決若其不濟當聚吾家於魏宫而焚之|{
	史言帝此行非有廟勝之策}
仍命豆盧革李紹宏張憲王正言同守東京|{
	帝以魏州為東京興唐府}
壬申帝以大軍自楊劉濟河癸酉至鄆州中夜進軍踰汶以李嗣源為前鋒甲戌旦遇梁兵一戰敗之|{
	敗補邁翻}
追至中都圍其城城無守備少頃|{
	少頃謂少頃刻之間}
梁兵潰圍出追擊破之王彦章以數十騎走龍武大將軍李紹奇單騎追之識其聲曰王鐵槍也|{
	按薛史夏魯奇嘗事梁祖與彦章素善故識其語音騎奇寄翻}
拔稍刺之彦章重傷馬躓|{
	刺七亦翻重直隴翻躓陟利翻}
遂擒之并擒都監張漢傑|{
	監古銜翻}
曹州刺史李知節禆將趙廷隱劉嗣彬等二百餘人斬首數千級廷隱開封人嗣彬知俊之族子也|{
	劉知俊自徐降梁自梁降岐自岐降蜀為蜀所殺}
彦章嘗謂人曰李亞子鬬雞小兒何足畏至是帝謂彦章曰爾常謂我小兒今日服未又問爾名善將何不守兖州|{
	將即亮翻九域志中都東南至兖州九十里}
中都無壁壘何以自固彦章對曰天命己去無足言者帝惜彦章之材欲用之賜藥傅其創|{
	創初良翻}
屢遣人誘諭之彦章曰余本匹夫蒙梁恩位至上將與皇帝交戰十五年今兵敗力窮死自其分|{
	分扶問翻}
縱皇帝憐而生我我何面目見天下之人乎豈有朝為梁將暮為唐臣此我所不為也帝復遣李嗣源自往諭之|{
	復扶又翻}
彦章臥謂嗣源曰汝非邈佶烈乎|{
	佶其吉翻}
彦章素輕嗣源故以小名呼之於是諸將稱賀帝舉酒屬嗣源曰|{
	屬之欲翻}
今日之功公與崇韜之力也曏從紹宏輩語大事去矣帝又謂諸將曰曏所患惟王彦章今已就擒是天意滅梁也段凝猶在河上進退之計宜何向而可諸將以為傳者雖云大梁無備未知虛實今東方諸鎮兵皆在段凝麾下所餘空城耳以陛下天威臨之無不下者若先廣地東傅於海|{
	傅讀曰附}
然後觀釁而動可以萬全康延孝固請亟取大梁李嗣源曰兵貴神速今彦章就擒段凝必未之知就使有人走告疑信之間尚須三日設若知吾所向即發救兵直路則阻決河|{
	即謂段凝所決護駕水}
須自白馬南渡數萬之衆舟楫亦難猝辦此去大梁至近前無山險方陳横行|{
	陳讀曰陣}
書夜兼程信宿可至段凝未離河上|{
	離力智翻}
友貞已為吾擒矣延孝之言是也請陛下以大軍徐進臣願以千騎前驅帝從之令下諸軍皆踴躍願行是夕嗣源帥前軍倍道趣大梁|{
	帥讀曰率趣七喩翻}
乙亥帝發中都舁王彦章自隨|{
	舁音余又羊如翻}
遣中使問彦章曰吾此行克乎對曰段凝有精兵六萬雖主將非材亦未肯遽爾倒戈殆難克也帝知其終不為用遂斬之|{
	今汶上縣有王彦章墓及祠}
丁丑至曹州|{
	九域志曹州西南至大梁二百四十餘里}
梁守將降|{
	將即亮翻降戶江翻}
王彦章敗卒有先至大梁告梁主以彦章就擒唐軍長驅且至者梁主聚族哭曰運祚盡矣召羣臣問策皆莫能對梁主謂敬翔曰朕居常忽卿所言以至於此今事急矣卿勿以為懟|{
	懟直類翻怨也}
將若之何翔泣曰臣受先帝厚恩殆將三紀|{
	梁太祖鎮宣武敬翔即為幕屬以至為相汔于梁亡故自言受恩殆將三紀以此觀之則知二百六十六卷開平元年史言翔在幕府三十餘年誤也}
名為宰相其實朱氏老奴事陛下如郎君|{
	門生故吏下至僮奴呼主人之子皆曰郎君}
臣前後獻言莫匪盡忠陛下初用段凝臣極言不可|{
	事見上}
小人朋比|{
	指趙張也比毗至翻}
致有今日今唐兵且至段凝限於水北不能赴救|{
	言段凝之兵欲還救大梁為決河之水所限其道回遠}
臣欲請陛下出避狄陛下必不聽從請陛下出奇合戰陛下必不果決雖使良平更生誰能為陛下計者|{
	張良陳平以智輔漢高祖定天下後之言智者率稱之為于偽翻}
臣願先賜死不忍見宗廟之亡也因與梁主相向慟哭梁主遣張漢倫馳騎追段凝軍漢倫至滑州墜馬傷足|{
	九域志大梁北至滑州二百里此注與前注王彦章三日破賊事大梁至滑州有十里之差蓋九域志於大梁注及滑州注其道里遠近自有微差者今不敢輕改因兩存之中間若此類頗多}
復限水不能進|{
	復扶又翻}
時城中尚有控鶴軍數千朱珪請帥之出戰梁主不從|{
	帥讀曰率}
命開封尹王瓚驅市人乘城為備初梁陜州節度使邵王友誨全昱之子也性頴悟人心多向之|{
	陜式冉翻}
或言其誘致禁軍欲為亂|{
	誘音酉}
梁主召還與其兄友諒友能並幽于别第|{
	友能反見上卷梁龍德元年}
及唐師將至梁主疑諸兄弟乘危謀亂并皇弟賀王友雍建王友徽盡殺之 |{
	考異曰薛史云友諒友能友誨莊宗入汴同日遇害按中都旣敗均王親弟猶疑而殺之况其從弟嘗為亂者豈得獨存故附於此}
梁主登建國樓|{
	大梁宫城南門曰建國門其樓曰建國樓}
面擇親信厚賜之使衣野服|{
	衣於旣翻}
齎蠟詔促段凝軍|{
	蠟詔猶蠟書也命出於上故謂之蠟詔}
旣辭皆亡匿或請幸洛陽收集諸軍以拒唐唐雖得都城勢不能久留或請幸段凝軍控鶴都指揮使皇甫麟曰|{
	考異曰莊宗實錄麟作鏻今從莊宗列傳及薛史}
凝本非將才|{
	將即亮翻}
官由幸進|{
	段凝以其妹得進事見二百六十八卷梁太祖乾化元年}
今危窘之際|{
	窘渠隕翻}
望其臨機制勝轉敗為功難矣且凝聞彦章敗其膽已破安知能終為陛下盡節乎|{
	終為于偽翻下臣為同}
趙巖曰事勢如此一下此樓誰心可保梁主乃止復召宰相謀之鄭珏請自懷傳國寶詐降以紓國難|{
	復扶又翻珏古岳翻紓商居翻緩也難乃旦翻}
梁主曰今日固不敢愛寶但如卿此策竟可了否珏俛首久之|{
	俛音免}
曰但恐未了左右皆縮頸而笑梁主日夜涕泣不知所為置傳國寶於臥内忽失之已為左右竊之迎唐軍矣戊寅或告唐軍已過曹州塵埃漲天趙巖謂從者曰吾待温許州厚必不負我遂奔許州|{
	九域志大梁西南至許州一百七十五里從才用翻温韜由趙巖得許州見上卷梁龍德元年}
梁主謂皇甫麟曰李氏吾世讐理難降首|{
	降戶江翻首式又翻言以事理推之難于迎降而自首也一讀降首皆如字言難低頭為之下也}
不可俟彼刀鋸吾不能自裁卿可斷吾首|{
	斷音短}
麟泣曰臣為陛下揮劒死唐軍則可矣不敢奉此詔梁主曰卿欲賣我邪麟欲自剄|{
	剄古隕翻}
梁主持之曰與卿俱死麟遂弑梁主因自殺梁主為人温恭約|{
	約上當有儉字句斷}
無荒淫之失但寵信趙張使擅威福疎弃敬李舊臣|{
	敬翔李振皆佐梁太祖者}
不用其言以至於亡|{
	唐天祐三年梁受唐禪歲在丁卯三主十七年而亡}
己卯旦李嗣源軍至大梁攻封丘門|{
	大梁城北面二門封丘門在西酸棗門在東梁開平元年改封丘門為含曜門時人猶以舊門名稱之晉天福三年又改為宣陽門又汴京圖京城北四門從東曰陳橋門次曰封丘門}
王瓚開門出降嗣源入城撫安軍民是日帝入自梁門|{
	梁門大梁城西面北來第一門梁開平元年改為乾象門晉天福三年改為乾明門}
百官迎謁於馬首拜伏請罪帝慰勞之|{
	勞力到翻下勞賜同}
使各復其位李嗣源迎賀帝喜不自勝手引嗣源衣以頭觸之曰吾有天下卿父子之功也天下與爾共之|{
	帝於此際可謂喜而失節矣宜不能保有天下也勝音升}
帝命訪求梁主頃之或以其首獻 |{
	考異曰實錄帝慘然曰敵惠敵怨不在後嗣朕與梁主十年戰爭恨不生識其面按莊宗漆均王首藏之太社豈有欲全之之理此特虚言耳}
李振謂敬翔曰有詔洗滌吾輩相與朝新君乎|{
	朝直遥翻下同}
翔曰吾二人為梁宰相君昏不能諫國亡不能救新君若問將何辭以對是夕未曙|{
	曙常恕翻天明為曙}
或報翔曰崇政李太保已入朝矣|{
	梁以李振為崇政使故以稱之}
翔歎曰李振謬為丈夫朱氏與新君世為仇讐今國亡君死縱新君不誅何面目入建國門乎乃縊而死庚辰梁百官復待罪於朝堂|{
	復扶又翻}
帝宣敕赦之趙巖至許州温昭圖迎謁歸第斬首來獻盡没巖所齎之貨|{
	元徽趙巖可為怙權冒貨之戒}
昭圖復名韜|{
	梁賜温昭圖名見二百六十九卷均王貞明元年}
辛巳詔王瓚收朱友貞尸殯於佛寺漆其首函之藏於太社 |{
	考異曰薛史末帝紀云詔河南尹張全義收葬之今從實錄}
段凝自滑州濟河入援以諸軍排陳使杜晏球為前鋒至封丘遇李從珂晏球先降壬午凝將其衆五萬至封丘亦解甲請降凝帥諸大將先詣闕待罪帝勞賜之|{
	帥讀曰率勞力到翻}
慰諭士卒使各復其所凝出入公卿間揚揚自得無愧色梁之舊臣見者皆欲齕其面抉其心|{
	齕恨沒翻又下結翻齧也抉於決翻}
丙戌詔貶梁中書侍郎同平章事鄭珏為萊州司戶蕭頃為登州司戶翰林學士劉岳為均州司馬任贊為房州司馬姚顗為復州司馬封翹為唐州司馬李懌為懷州司馬竇夢徵為沂州司馬崇政學士劉光素為密州司戶陸崇為安州司戶御史中丞王權為隨州司戶以其世受唐恩而仕梁貴顯故也岳崇龜之從子|{
	劉崇龜見二百五十三卷唐僖宗廣明元年從才用翻}
顗萬年人|{
	萬年屬京兆府唐為赤縣時復以京兆為西京}
翹敖之孫|{
	封敖仕唐武宣朝入翰林位至尚書僕射}
懌京兆人權龜之孫也|{
	王龜式之兄也唐咸通間有名}
段凝杜晏球上言|{
	上時掌翻}
偽梁要人趙巖趙鵠張希逸張漢倫張漢傑張漢融朱珪等竊弄威福殘蠧羣生不可不誅詔敬翔李振首佐朱温共傾唐祚契丹博囉鄂博叛兄弃母負恩背國|{
	博囉鄂博奔梁見二百七十卷貞明四年背蒲妹翻}
宜與巖等並族誅於市自餘文武將吏一切不問又詔追廢朱温朱友貞為庶人毁其宗廟神主帝之與梁戰於河上也梁拱宸左廂都指揮使陸思鐸善射常於笴上自鏤姓名|{
	笴古我翻又公旱翻箭莖也鏤郎豆翻}
射帝中馬鞍|{
	射而亦翻中竹仲翻}
帝拔箭藏之至是思鐸從衆俱降帝出箭示之思鐸伏地待罪帝慰而釋之尋授龍武右廂都指揮使以豆盧革尚在魏命樞密使郭崇韜權行中書事梁諸藩鎮稍稍入朝或上表待罪帝皆慰釋之宋州節度使袁象先首來入朝陜州留後霍彦威次之象先輦珍貨數十萬徧賂劉夫人及權貴伶官宦者旬日中外爭譽之|{
	譽音余}
恩寵隆異己丑詔偽庭節度觀察防禦團練使刺史及諸將校並不議改更|{
	將即亮翻校戶教翻更工衡翻}
將校官吏先奔偽庭者一切不問庚寅豆盧革至自魏甲午加崇韜守侍中領成德節度使|{
	賞決策滅梁之功也}
崇韜權兼内外謀猷規益竭忠無隱頗亦薦引人物豆盧革受成而已無所裁正丙申賜滑州留後段凝姓名曰李紹欽耀州刺史杜

晏球曰李紹虔|{
	後各復本姓名}
乙酉梁西都留守河南尹張宗奭來朝復名全義|{
	梁改張全義名見二百六十六卷太祖開平元年}
獻幣馬千計帝命皇子繼岌皇弟存紀等兄事之|{
	繼岌皇嗣也豈可兄事梁之舊臣存紀皇弟也旣使其子以兄事全義又使其弟以兄事全義唐之家人長幼之序且不明矣是後中宫又從而父事之嘻甚矣晉陽之主好貨而已豈知有綱常哉}
帝欲發梁太祖墓斲棺焚其尸全義上言朱温雖國之深讐然其人已死刑無可加屠滅其家足以為報乞免焚斲以存聖恩帝從之但鏟其闕室削封樹而已|{
	張全義猶不忘梁祖河陽之恩鏟初限翻削其封樹者隳其墳赭其山也}
戊戌加天平節度使李嗣源兼中書令以北京留守繼岌為東京留守同平章事|{
	時以鎮州為北京魏州為東京}
帝遣使宣諭諭諸道梁所除節度使五十餘人皆上表入貢楚王殷遣其子牙内馬步都指揮使希範入見納洪鄂行營都統印|{
	梁命殷為洪鄂行營都統}
上本道將吏籍|{
	上時掌翻}
荆南節度使高季昌聞帝滅梁避唐廟諱更名季興|{
	以獻祖諱國昌也更工衡翻}
欲自入朝梁震曰唐有吞天下之志嚴兵守險猶恐不自保况數千里入朝乎且公朱氏舊將|{
	高季昌為梁將事始見二百六十三卷唐昭宗天復二年}
安知彼不以仇敵相遇乎季興不從 帝遣使以滅梁告吳蜀二國皆懼徐温尤嚴可求曰公前沮吾計|{
	謂自鄆州遣使會兵徐温欲以舟師浮海北進時也事見五月}
今將奈何可求笑曰聞唐主始得中原志氣驕滿御下無法不出數年將有内變吾卑辭厚禮保境安民以待之耳|{
	善哉覘也}
唐使稱詔吳人不受帝易其書用敵國之禮曰大唐皇帝致書于吳國主吳人復書稱大吳國主上大唐皇帝辭禮如牋表 吳人有告壽州團練使鍾泰章侵市官馬者徐知誥以吳王之命遣滁州刺史王稔巡霍丘因代為壽州團練使|{
	霍丘吳之邊邑徐知誥命王稔以巡邊為名因代泰章}
以泰章為饒州刺史徐温召至金陵使陳彦謙詰之者三|{
	詰去吉翻}
皆不對或問泰章何以不自辨泰章曰吾在揚州十萬軍中號稱壯士壽州去淮數里步騎不下五千苟有它志豈王稔單騎能代之乎我義不負國雖黜為縣令亦行况刺史乎何為自辨以彰朝廷之失徐知誥欲以法繩諸將請收泰章治罪|{
	治直之翻}
徐温曰吾非泰章已死於張顥之手|{
	事見二百六十六卷梁太祖開平二年}
今日富貴安可負之命知誥為子景通娶其女以解之|{
	為于偽翻}
彗星見輿鬼長丈餘|{
	輿鬼五星秦雍州分彗祥歲翻又徐醉翻見賢遍翻長直亮翻}
蜀司天監言國有大災蜀主詔於玉局化設道場|{
	玉局化在成都彭乘記曰後漢永壽元年李老君與張道陵至此有局脚玉床自地而出老君昇坐為道陵說南北斗經旣去而坐隱地中因成洞穴故以玉局名之道經以二十四化上應二十四氣玉局其一也流俗相傳而信奉之}
右補闕張雲上疏以為百姓怨氣上徹於天|{
	徹敕列翻}
故彗星見此乃亡國之徵非祈禳可弭蜀主怒流雲黎州卒於道 郭崇韜上言河南節度使刺史上表者但稱姓名未除新官恐負憂疑十一月始降制以新官命之滑州留後李紹欽因伶人景進納貨於宫掖除泰寧節度使帝幼善音律故伶人多有寵常侍左右帝或時自傅粉墨與優人共戲於庭以悦劉夫人優名謂之李天下嘗因為優自呼曰李天下李天下優人敬新磨遽前批其頰|{
	批蒲結翻又匹述翻反手擊也}
帝失色羣優亦駭愕新磨徐曰理天下者只有一人尚誰呼邪帝悦厚賜之帝嘗畋于中牟踐民稼|{
	九域志中牟縣在大梁西七十里踐慈演翻}
中牟令當馬前諫曰陛下為民父母奈何毁其所食使轉死溝壑乎帝怒叱去將殺之敬新磨追擒至馬前責之曰汝為縣令獨不知吾天子好獵邪|{
	好呼到翻下好采同}
奈何縱民耕種以妨吾天子之馳騁乎汝罪當死因請行刑帝笑而釋之諸伶出入宫掖侮弄縉紳羣臣憤嫉莫敢出氣|{
	書云狎侮君子罔以盡其心况使伶人侮弄之哉}
亦反有相附託以希恩澤者四方藩鎮爭以貨賂結之|{
	無材而干利祿者何可勝數哉}
其尤蠧政害人者景進為之首進好采閭閻鄙細事聞於上上亦欲知外間事遂委進以耳目進每奏事嘗屏左右問之|{
	屏必郢翻又卑正翻}
由是進得施其讒慝干預政事自將相大臣皆憚之孔巖常以兄事之|{
	孔巖當作孔謙}
壬寅岐王遣使致書賀帝滅梁以季父自居辭禮甚

倨|{
	岐王李茂貞自以與晉王克用在唐並列藩鎮又各以有功賜姓附唐屬籍義猶兄弟故于帝以季父自居}
癸卯河中節度使朱友謙入朝帝與之宴寵錫無筭 張全義請帝遷都洛陽從之 |{
	考異曰實錄甲辰議脩洛陽太廟按梁以汴州為東京洛京為西京莊宗以魏州為東京太原為西京真定為北都及滅梁廢東京為汴州以永平軍為西京而不云以洛陽為何京若以為東京則與魏州無以異諸書但謂之洛京亦未嘗有詔改梁西京為洛京至同光三年始詔依舊以洛京為東都或者以永平為西京時即改梁西京為洛京而史脱其文也今無可質正故但謂之洛陽}
乙巳賜朱友謙姓名曰李繼麟命繼岌兄事之 以康延孝為鄭州防禦使賜姓名曰李紹琛廢北都復為成德軍|{
	是年四月於鎮州建北都}
賜宣武節度使袁象先姓名曰李紹安匡國節度使温韜入朝賜姓名曰李紹冲紹冲多齎金帛賂劉夫人及權貴伶宦旬日復遣還鎮郭崇韜曰國家為唐雪恥|{
	為于偽翻}
温韜發唐山陵殆徧|{
	事見二百六十七卷梁太祖開平二年}
其罪與朱温相埓耳|{
	埒龍輟翻等也}
何得復居方鎮天下義士其謂我何上曰入汴之初已赦其罪竟遣之 戊申中書奏以國用未充請量留三省寺監官餘並停俟見任者滿二十五月以次代之|{
	見任謂見在官者見賢遍翻}
其西班上將軍以下令樞密院準此|{
	朝會之序武官班於西故曰西班}
從之人頗咨怨 初梁均王將祀南郊於洛陽聞楊劉陷而止|{
	事見二百七十卷貞明三年}
其儀物具在張全義請上亟幸洛陽謁廟畢|{
	唐東京亦有太廟末世東遷嘗嚴奉故張全義請上脩謁}
即祀南郊從之 丙辰復以梁東京開封府為宣武軍汴州梁以宋州為宣武軍詔更名歸德軍|{
	梁都汴徙宣武軍額于宋州更工衡翻}
詔文武官先詣洛陽 議者以郭崇韜勲臣為宰相不能知朝廷典故當用前朝名家以佐之|{
	朝直遥翻下同}
或薦禮部尚書薛廷珪太子少保李琪嘗為太祖冊禮使皆耆宿有文宜為相崇韜奏廷珪浮華無相業琪傾險無士風尚書左丞趙光胤亷潔方正自梁未亡北人皆稱其有宰相器|{
	三人者皆仕梁廷珪琪為太祖冊禮使必唐之時嘗奉朝命冊晉王者也}
豆盧革薦禮部侍郎韋說諳練朝章|{
	諳烏含翻}
丁巳以光胤為中書侍郎與說並同平章事光胤光逢之弟|{
	趙光逢見二百六十六卷梁太祖開平元年}
說岫之子廷珪逢之子也|{
	薛逢唐會昌間有文聲}
光胤性輕率喜自矜|{
	喜許記翻}
說謹重守常而已趙光逢自梁朝罷相|{
	梁均玉貞明元年趙光逢罷相}
杜門不交賓客光胤時往見之語及政事它日光逢署其戶曰請不言中書事 租庸副使孔謙畏張憲公正欲專使務|{
	言欲專租庸使一司事務也使疏吏翻}
言於郭崇韜曰東京重地須大臣鎮之非張公不可崇韜即奏以憲為東京副留守知留守事|{
	出張憲守魏州}
戊午以豆盧革判租庸兼諸道鹽鐵轉運使謙彌失望 己未加張全義守尚書令高季興守中書令時季興入朝上待之甚厚從容問曰|{
	從千容翻}
朕欲用兵於吳蜀二國何先季興以蜀道險難取乃對曰吳地薄民貧克之無益不如先伐蜀蜀土富饒又主荒民怨伐之必克克蜀之後順流而下取吳如反掌耳上曰善 辛酉復以永平軍大安府為西京京兆府|{
	梁改長安為永平軍見二百六十七卷太祖開平三年改京兆府為大安府見二百六十六卷開平元年}
甲子帝發大梁十二月庚午至洛陽 吳越王鏐以行軍司馬杜建徽為左丞相壬申詔以汴州宫苑為行宫 以耀州為順義軍延州為彰武軍鄧州為威勝軍晉州為建雄軍安州為安遠軍|{
	帝旣滅梁特改梁所置軍名耳凡諸藩帥未之易也梁改耀州曰崇州改義勝軍為靜勝軍乃岐所置延州唐保塞軍岐為忠義軍鄧州梁為宣化軍晉州梁始為定昌軍後改建寧軍安州梁為宣威軍}
自餘藩鎮皆復唐舊名 庚辰御史臺奏朱温簒逆刪改本朝律令格式|{
	梁改定律令格式事見二百六十七卷開平四年本朝謂前唐也}
悉收舊本焚之今臺司及刑部大理寺所用皆偽庭之法聞定州敕庫獨有本朝律令格式具在乞下本道錄進|{
	下戶嫁翻}
從之 李繼韜聞上滅梁憂懼不知所為欲北走契丹|{
	走音奏}
會有詔徵詣闕繼韜將行其弟繼遠曰兄以反為名何地自容往與不往等耳不若深溝高壘坐食積粟猶可延歲月入朝立死矣或謂繼韜曰先令公有大功於國|{
	先令公謂繼韜父嗣昭嗣昭官中書令故稱之}
主上於公季父也|{
	李嗣昭以晉王義兒於上為兄上於繼韜為季父}
往必無虞繼韜母楊氏善蓄財家貲百萬乃與楊氏偕行齎銀四十萬兩它貨稱是大布賂遺伶人宦官爭為之言曰|{
	稱尺證翻遺唯季翻為于偽翻下亦為同}
繼韜初無邪謀為姦人所惑耳嗣昭親賢不可無後楊氏復入宫見帝泣請其死|{
	復扶又翻下復賂子復同}
以其先人為言又求哀於劉夫人劉夫人亦為之言及繼韜入見待罪上釋之|{
	見賢遍翻}
留月餘屢從遊畋寵待如故皇弟義成節度使同平章事存渥深詆訶之|{
	繼韜兄弟欲殺存渥事見上卷梁均王龍德二年梁改滑州義成軍為宣義軍帝復唐舊}
繼韜心不自安復賂左右求還鎮上不許繼韜濳遣人遺繼遠書教軍士縱火冀天子復遣已撫安之事泄辛巳貶登州長史尋斬於天津橋南并其二子遣使斬李繼遠於上黨以李繼達充軍城巡檢召權知軍州事李繼儔詣闕繼儔據有繼韜之室料簡妓妾|{
	料音聊妓渠綺翻}
搜校貨財不時即路|{
	即就也}
繼達怒曰吾家兄弟父子同時誅死者四人|{
	繼韜及其二子并繼遠為四人}
大兄曾無骨肉之情|{
	繼韜兄弟七人繼儔居長故呼為大兄}
貪淫如此吾誠羞之無面視人生不如死甲申繼達衰服帥麾下百騎坐戟門呼曰|{
	史炤曰列棨戟於門故曰戟門帥讀曰率衰倉回翻呼火故翻}
誰與吾反者因攻牙宅|{
	牙宅即使宅也}
斬繼儔節度副使李繼珂聞亂募市人得千餘攻子城繼達知事不濟開東門歸私第|{
	東門潞州牙城東門也}
盡殺其妻子將奔契丹出城數里從騎皆散乃自剄|{
	從才用翻剄古頂翻}
甲申吳王復遣司農卿洛陽盧蘋來奉使嚴可求預料帝所問教蘋應對旣至皆如可求所料蘋還言唐主荒於遊畋嗇財拒諫内外皆怨 高季興在洛陽帝左右伶官求貨無厭|{
	伶官謂伶人及宦官也厭於鹽翻}
季興忿之帝欲留季興郭崇韜諫曰陛下新得天下諸侯不過遣子弟將佐入貢惟高季興身自入朝當褒賞以勸來者乃羈留不遣弃信虧義沮四海之心|{
	沮在呂翻}
非計也乃遣之季興倍道而去至許州|{
	九域志洛陽東至許州三百一十里}
謂左右曰此行有二失來朝一失縱我去一失|{
	言彼此俱失也}
過襄州節度使孔勍留宴中夜斬關而去|{
	勍渠京翻 考異曰五代史補季興已行浹旬莊宗且悔遽以急詔命襄州節度使劉訓伺便圖之無何季興至襄州就館而心動謂親吏曰梁先輩之言中矣與其住而生不若去而死遂弃輜重與部曲數百人南走至鳳林關已昏黑於是斬關而出是夜三更向之急詔果至劉訓度其去遠不可及而止王舉天下大定錄亦云莊宗遣使追之不及按季興自疑故斬關夜遁耳未必莊宗追之也今從薛史}
丁酉至江陵握梁震手曰不用君言幾不免虎口|{
	梁震所言見上幾居依翻}
又謂將佐曰新朝百戰方得河南|{
	以莊宗新得天下故曰新朝朝直遥翻}
乃對功臣舉手云吾於十指上得天下矜伐如此則他人皆無功矣其誰不解體又荒於禽色何能久長吾無憂矣乃繕城積粟招納梁舊兵為戰守之備|{
	史言帝荒淫驕矜為鄰敵及姦雄所窺}


資治通鑑卷二百七十二














































































































































