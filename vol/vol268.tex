\chapter{資治通鑑卷二百六十八}
宋 司馬光 撰

胡三省 音註

後梁紀三|{
	起重光協洽三月盡昭陽作噩十一月凡二年有奇}


太祖神武元聖孝皇帝下

乾化元年三月乙酉朔以天雄留後羅周翰為節度使清海靜海節度使兼中書令南平襄王劉隱病亟|{
	亟紀}


|{
	力翻}
表其弟節度副使巖權知留後丁亥卒|{
	隱年三十八}
巖襲位 岐王聚兵臨蜀東鄙蜀主謂羣臣曰自茂貞為朱溫所困吾常振其乏絶|{
	事並見前紀}
今乃負恩為寇誰為吾擊之|{
	誰為于偽翻}
兼中書令王宗侃請行蜀主以宗侃為北路行營都統司天少監趙温珪諫曰茂貞未犯邊諸將貪功深入糧道阻遠恐非國家之利蜀主不聽|{
	少詩照翻將即亮翻}
以兼侍中王宗祐太子少師王宗賀山南節度使唐道襲為三招討使|{
	三路進兵以代岐各路置一招討使王宗侃都統三招討之兵}
左金吾大將軍王宗紹為宗祐之副帥步騎十二萬代岐|{
	帥讀曰率下同}
壬辰宗侃等發成都旌旗數百里 岐王募華原賊帥温韜以為假子以華原為耀州美原為鼎州|{
	宋廢鼎州復為美原縣屬耀州宋白曰華原縣本漢祋栩縣地曹魏以來置北地郡元魏廢帝三年置通州郡泥陽縣隋開皇六年改泥陽為華原美原縣本秦漢頻陽縣苻秦置土門護軍後周置土門縣唐咸亨二年改為美原九域志耀州在長安北一百六十里}
置義勝軍以韜為節度使使帥邠岐兵寇長安詔感化節度使康懷貞忠武節度使牛存節以同華河中兵討之己酉懷貞等奏擊韜於車度走之|{
	車度地名在長安北同州界}
夏四月乙卯朔岐兵寇蜀興元唐道襲擊却之 上以久疾五月甲申朔大赦|{
	按歐史此下當有改元二字}
甲辰以清海留後劉巖為節度使 |{
	考異曰十國紀年甲辰太祖授陟清海節度使陟復名巖按薛史僭偽傳云前偽漢劉陟胡賓王劉氏興亡錄高祖巖皇考葬段氏得石版有篆文曰隱台巖因名其三子是先名巖後名陟也吳越備史乾化四年廣帥彭城巖遣陳用拙來使吳錄天祐十四年南漢王劉巖自立為漢唐烈祖實錄天祐十四年劉陟僭位改名巖太祖實錄乾化元年五月以清海節度副使劉陟為節度使二年四月以韋戩為潭廣和叶使云廣守淪謝其母弟巖為軍情所戴七月友珪加劉巖檢校太傅薛史梁末帝紀貞明五年九月削奪廣州節度使劉巖官爵吳越備史載制詞亦云彭城巖蓋嗣節度使後復名巖也惟莊宗實錄同光三年二月廣南劉陟遣何詞來使莊宗列傳自嗣立至建號皆云劉陟衆說不同未知孰是今以其首尾名巖故但稱劉巖云}
巖多延中國士人置於幕府出為刺史刺史無武人 蜀主如利州命太子監國|{
	監古銜翻}
六月癸丑朔至利州|{
	欲親總兵以繼伐岐之師}
燕王守光嘗衣赭袍|{
	衣於旣翻赭音者蓋唐世天子之服}
顧謂將吏曰今天下大亂英雄角逐吾兵彊地險亦欲自帝何如孫鶴曰今内難新平|{
	謂新平滄德斯言不當發于孫鶴難乃旦翻}
公私困竭太原窺吾西契丹伺吾北|{
	伺相利翻}
遽謀自帝未見其可大王但養士愛民訓兵積穀德政旣脩四方自服矣守光不悅又使人諷鎮定求尊已為尚父趙王鎔以告晉王晉王怒欲伐之諸將皆曰是為惡極矣行當族滅不若陽為推尊以稔之|{
	稔其惡也}
乃與鎔及義武王處直昭義李嗣昭振武周德威天德宋瑤六節度使|{
	五鎮并河東為六然自昭義以下皆屬河東}
共奉冊推守光為尚書令尚父守光不寤以為六鎮實畏已益驕乃具表其狀曰晉王等推臣臣荷陛下厚恩|{
	荷下可翻}
未之敢受竊思其宜不若陛下授臣河北都統則并鎮不足平矣|{
	并謂晉王鎮謂趙王鎔}
上亦知其狂愚乃以守光為河北道採訪使|{
	唐之盛時置十道採訪使河北其一也自安史亂後不復除授}
遣閤門使王瞳受旨史彥羣冊命之|{
	受旨蓋崇政院官屬猶樞密院承旨也梁避廟諱改承為受}
守光命僚屬草尚父采訪使受冊儀乙卯僚屬取唐冊太尉儀獻之守光視之問何得無郊天改元之事對曰尚父雖貴人臣也安有郊天改元者乎守光怒投之於地曰我地方二千里帶甲三十萬直作河北天子誰能禁我尚父何足為哉命趣具即帝位之儀|{
	趣讀曰促}
械繫瞳彥羣及諸道使者於獄旣而皆釋之 |{
	考異曰莊宗列傳劉守光傳云朱温命偽閤門使王瞳供奉官史彥章等使燕冊守光為河北道采訪使六月汴使至守光令所司定尚父採訪使儀注取二十四日受冊朱温傳亦云史彥章莊宗實錄作史彦璋編遺錄薛史皆作史彦羣今從之又莊宗實錄三月己丑鎮州遣押牙劉光業至言劉守光凶淫縱毒欲自尊大請稔其惡以咎之推為尚父乙未上至晉陽宫召張承業諸將等議討燕之謀諸將亦云宜稔其禍上令押衙戴漢超持墨制及六鎮書如幽州其辭曰天祐八年三月二十七日天德軍節度使宋瑤振武節度使周德威昭義節度使李嗣昭易定節度使王處直鎮州節度使王鎔河東節度使尚書令晉王謹奉冊進盧龍横海等軍節度檢校太師兼中書令燕王為尚書令尚父五月六鎮使至汴使亦集六月守光令有司定尚父採訪使儀則梁太祖實錄都不言守光事惟編遺錄云三月壬辰差閤門使王瞳受旨史彦羣齎國禮賜幽州劉守光甲午守光連上表章以鎮定旣與河東結懽兼同差使請當道却行天祐年號事守光尋捉王瞳史彦羣上下一行並囚禁數日後放出按莊宗實錄及南唐烈祖實錄皆云三月辛亥晉王遣戴漢超推守光為尚父辛亥三月二十七日也壬辰乃三月初八日王瞳等安得已在幽州甲午乃三月十日守光又安得上表云六鎮推臣為尚父編遺月日多差錯今不取}
帝命楊師厚將兵三萬屯邢州|{
	欲攻趙也}
蜀諸將擊岐

兵屢破之秋七月蜀主西還留御營使昌王宗鐬屯利州|{
	鐬火外翻}
辛丑帝避暑於張宗奭第|{
	開平元年張全義賜名宗奭見上卷按薛史張宗奭私第在洛陽會節坊}
亂其婦女殆徧宗奭子繼祚不勝憤耻|{
	勝音升}
欲弑之宗奭止之曰吾家頃在河陽為李罕之所圍|{
	見二百五十七卷唐僖宗文德元年}
啗木屑以度朝夕|{
	啗徒濫翻}
賴其救我得有今日此恩不可忘也乃止甲辰還宫 趙王鎔以楊師厚在邢州甚懼|{
	九域志邢州北至趙州一百四十四里耳兵臨其境故甚懼}
會晉王於承天軍晉王謂鎔父友也事之甚恭|{
	鎔先與晉王克用比肩事唐且通好}
鎔以梁寇為憂晉王曰朱温之惡極矣天將誅之雖有師厚輩不能救也脱有侵軼|{
	軼徒結翻}
僕自帥衆當之|{
	帥讀曰率}
叔父勿以為憂鎔捧巵為壽謂晉王為四十六舅|{
	晉王第四十六}
鎔幼子昭誨從行晉王斷衿為盟許妻以女|{
	斷都管翻衿音今妻七細翻}
由是晉趙之交遂固 八月庚申蜀主至成都|{
	自利州還}
燕王守光將稱帝將佐多竊議以為不可守光乃置斧質於庭|{
	質椹也}
曰敢諫者斬孫鶴曰滄州之破鶴分當死蒙王生全|{
	事見上卷開平四年分扶問翻劉守光囚父殺兄幽滄之人義不與共戴天可也孫鶴受劉守文委任不能以死殉之乃銜守光生全之恩忠諫而死是可以死而不能死可以無死而死也}
以至今日今日敢愛死而忘恩乎竊以為今日之帝未可也守光怒伏諸質上令軍士冎而噉之|{
	冎古瓦翻噉徒濫翻}
鶴呼曰不出百日大兵當至守光命以土窒其口寸斬之|{
	呼火故翻}
甲子守光即皇帝位國號大燕改元應天以梁使王瞳為左相盧龍判官齊涉為右相史彦羣為御史大夫 |{
	考異曰編遺錄云御史臺副使今從莊宗實錄}
受冊之日契丹陷平州燕人驚擾|{
	宋白曰平州舜十二州為營州之境周官職方在幽州之地春秋為山戎孤竹白狄肥子二國地漢為肥如石城之地唐武德初置平州於盧龍}
岐王使劉知俊李繼崇將兵擊蜀乙亥王宗侃王宗賀唐道襲王宗紹與之戰於青泥嶺|{
	青泥嶺在興州長舉縣西北五十里懸崖萬仭上多雲雨行者多逢泥淖}
蜀兵大敗馬步使王宗浩奔興州溺死於江|{
	嘉陵江也}
道襲奔興元先是步軍都指揮使王宗綰城西縣號安遠軍|{
	九域志西縣在興元府西一百里}
宗侃宗賀等收散兵走保之知俊繼崇追圍之衆議欲弃興元道襲曰無興元則無安遠利州遂為敵境矣|{
	九域志興元西至西縣百里西縣抵利州界四十五里自界首至利州二百六十四里}
吾必以死守之蜀主以昌王宗鐬為應援招討使定戎團練使王宗播為四招討馬步都指揮使|{
	蜀主先已遣三招討使伐岐今又以王宗鐬為應援招討使是為四招討}
將兵救安遠軍壁於廉讓之間|{
	廉水出大巴山北谷中讓水其源起於廉水漑田之餘東南流至古廉水城之側二水在南鄭縣東南杜佑曰綿州昌明縣有廉水讓水宋白續通典縣有清廉鄉讓水鄉}
與唐道襲合擊岐兵大破之於明珠曲明日又戰於鳬口斬其成州刺史李彦琛 九月帝疾稍愈聞晉趙謀入寇自將拒之戊戌以張宗奭為西都留守庚子帝發洛陽甲辰至衛州方食軍前奏晉軍已出井陘|{
	陘音刑}
帝遽命輦北趣邢洺晝夜倍道兼行丙午至相州|{
	九域志衛州北至相州一百二十五里自相州又北則趣邢洺趣七喻翻}
聞晉兵不出乃止相州刺史李思安不意帝猝至落然無具坐削官爵 湖州刺史錢鏢酗酒殺人|{
	鏢甫招翻酗呼匈翻}
恐吳越王鏐罪之冬十月辛亥朔殺都監潘長推官鍾安德奔于吳 晉王聞燕王守光稱帝大笑曰俟彼卜年吾當問其鼎矣|{
	以周成王卜年楚子問鼎之事戲笑守光}
張承業請遣使致賀以驕之|{
	使疏吏翻下通使之使同}
晉王遣太原少尹李承勲往承勲至幽州用鄰藩通使之禮燕之典客者曰吾王帝矣公當稱臣庭見|{
	見賢遍翻}
承勲曰吾受命於唐朝為太原少尹|{
	朝直遙翻}
燕王自可臣其境内豈可臣它國之使乎守光怒囚之數日出而問之曰臣我乎承勲曰燕王能臣我王則我請為臣不然有死而已守光竟不能屈 蜀主如利州|{
	聞王宗侃為岐所敗故復如利州以為繼援}
命太子監國決雲軍虞候王琮敗岐兵|{
	敗補邁翻}
執其將李彦太俘斬三千級王宗侃遣裨將林思諤自中巴間行至泥溪|{
	巴州在三巴之中謂之中巴興元之南有大行路逕孤雲兩角過米倉山則至巴州按後唐伐蜀還魏王繼岌與李紹琛軍行次舍泥溪當在劒州北利州界}
見蜀主告急蜀主命開道都指揮使王宗弼將兵救安遠及劉知俊戰于斜谷破之|{
	斜余遮翻谷余玉翻}
甲寅夜帝發相州乙卯至洹水是夜邊吏言晉趙兵南下帝即時進軍丙辰至魏縣|{
	洹水在魏州之西成安縣界九域志魏州成安縣有洹水鎮成安縣在州西三十五里魏縣在魏州西三十五里}
或告云沙陀至矣士卒恟懼多逃亡嚴刑不能禁旣而復告云無寇上下始定|{
	敗兵之氣沒世不復此之謂也而復扶又翻}
戊午貝州奏晉兵寇東武尋引去帝以夾寨柏鄉屢失利|{
	夾寨之敗見二百六十六卷開平二年柏鄉之敗見上卷本年}
故力疾北巡思一雪其恥意鬱鬱多躁忿功臣宿將往往以小過被誅衆心益懼|{
	薛史本紀帝至相州左龍驤都教練使鄧季筠魏博馬軍都指揮使何令稠右廂馬軍都指揮使陳令勲以部下馬瘦並腰斬于軍門次魏縣先鋒指揮使黄文靖伏誅}
旣而晉趙兵竟不出|{
	帝以忿兵輕行求雪再敗之恥使其果與晉趙遇亦必敗矣}
十一月壬午帝南還 燕主守光集將吏謀攻易定幽州參軍景城馮道以為未可|{
	景城縣屬瀛州漢舊縣名}
守光怒繫獄或救之得免道亡奔晉張承業薦於晉王以為掌書記|{
	馮道自此歷事唐晉漢周位極人臣不聞諫爭豈懲諫守光之禍邪}
丁亥王處直告難於晉|{
	難乃旦翻}
懷州刺史開封段明遠妹為美人戊子帝至獲嘉|{
	九域志獲嘉縣在懷州東北一百五十里}
明遠饋獻豐備帝悅|{
	段明遠後改名凝階此寵任位為上將梁遂以亡}
庚寅保塞節度使高萬興奏遣都指揮使高萬金將兵攻鹽州刺史高行存降|{
	按考異曰實錄開平三年六月丁未靈武韓遜奏收復鹽州擒偽刺史李繼直已下六十二人至此年降高行存下云鹽州與吐蕃党項犬牙相接為二境咽喉之地又烏池鹽醝之利戎羌意未嘗息唐建中初為吐蕃所陷砥其墉而去由是銀夏寧延洎于靈武歲以河南山東淮南青徐江浙等道兵士不啻四萬分護其地謂之防秋貞元九年朝政稍暇乃命副元帥渾瑊總兵三萬復取其地建百雉焉自是虜塵乃息邊患遂止唐代革命又復失之今纔動偏師遽牧襟要國之右臂瘡疣其息哉李茂貞養子多連繼字開平三年所收似屬鳳翔今又收復云唐革命失之前後必一誤或者開平旣得又失之也}
壬辰帝至洛陽疾復作|{
	復扶又翻}
蜀王宗弼敗岐兵於金牛|{
	敗補邁翻下同}
拔十六寨俘斬六千餘級擒其將郭存等丙申王宗鐬王宗播敗岐兵於黄牛川擒其將蘇厚等丁酉蜀主自利州如興元援軍旣集安遠軍望其旗|{
	旗謂蜀主之旗也}
王宗侃等鼓譟而出與援軍夾攻岐兵大破之拔二十一寨斬其將李廷志等己亥岐兵解圍遁去|{
	解安遠之圍而遁}
唐道襲先伏兵於斜谷邀擊又破之庚子蜀主西還|{
	岐兵旣敗走遂還}
岐王左右石簡顒讒劉知俊於岐王|{
	顒魚魚翻}
王奪其兵李繼崇言於王曰知俊壯士窮來歸我不宜以讒廢之王為之誅簡顒以安之|{
	為于偽翻}
繼崇召知俊舉族居於秦州|{
	李繼崇時鎮秦州繼崇尋不能守秦州劉知俊由此亦降於蜀}
戊申燕主守光將兵二萬寇易定攻容城|{
	容城漢縣名唐屬易州宋屬雄州}
王處直告急于晉十二月乙卯以朗州留後馬賨為永順節度使同平章事|{
	賨徂宗翻馬殷之弟也}
鎮南留後盧延昌遊獵無度百勝軍指揮使黎球殺之自立將殺譚全播全播稱疾請老乃免丙辰以球為虔州防禦使未幾球卒|{
	幾居豈翻}
牙將李彦圖代知州事全播愈稱疾篤劉巖聞全播病發兵攻韶州破之刺史廖爽奔楚|{
	唐天復二年虔人取韶州至是復為劉氏廖力救翻}
楚王殷表為永州刺史 丁巳蜀主至成都|{
	自興元還至成都}
戊午以靜海留後曲美為節度使 癸亥以靜江行軍司馬姚彥章為寧遠節度副使權知容州從楚王殷之請也劉巖遣兵攻容州殷遣都指揮使許德勲以桂州兵救之彦章不能守乃遷容州士民及其府藏奔長沙巖遂取容管及高州|{
	藏徂浪翻開平四年楚取容管及高州至是弃之}
甲子晉王遣蕃漢馬步摠管周德威將兵三萬攻燕以救易定是歲蜀主以内樞密使潘炕為武泰節度使|{
	唐置武泰軍於}


|{
	黔州}
炕從弟宣徽南院使峭為内樞密使|{
	從才用翻峭七肖翻}
二年春正月德威東出飛狐|{
	自代州出飛狐宋白曰飛狐縣漢代郡地曹魏封樂進于廣昌侯國後周於五龍城置廣昌縣隋改飛狐縣因縣北飛狐口為名}
與趙王將王德明義武將程巖會于易水|{
	趙王王鎔義武王處直}
丙戌三鎮兵進攻燕祁溝關下之|{
	三鎮并鎮定祁溝關在涿州南易州巨馬河之北自關而西至易州六十里巨馬河東至新城縣四十里}
戊子圍涿州|{
	宋白曰涿州古涿鹿地漢高帝置涿郡魏改范陽郡取漢涿縣在范水之陽為名唐大歷四年立涿州南至莫州一百六十里東北至幽州一百二十里}
刺史劉知温城守|{
	守手又翻}
劉守奇之客劉去非大呼於城下|{
	呼火故翻}
謂知温曰河東小劉郎來為父討賊|{
	為于偽翻}
何豫汝事而堅守邪守奇免胄勞之|{
	劉守奇奔晉見二百六十六卷開平元年勞力到翻}
知温拜於城上遂降周德威疾守奇之功譛諸晉王|{
	此周德威之褊也降戶江翻}
王召之守奇恐獲罪與去非及進士趙鳳來奔上以守奇為博州刺史去非鳳皆幽州人也先是燕主守光籍境内丁壯悉文面為兵雖士人不免鳳詐為僧奔晉守奇客之|{
	先悉薦翻}
丁酉德威至幽州城下守光來求救二月帝疾小愈議自將擊鎮定以救之 帝聞岐蜀相攻辛酉遣光祿卿盧玭等使于蜀遺蜀主書|{
	玭蒲眠翻遺唯季翻}
呼之為兄|{
	帝與蜀主偕起於細微者也蜀兵彊地險帝自度力不能制故用敵國禮呼之為兄}
甲子帝發洛陽從官以帝誅戮無常多憚行帝聞之益怒是日至白馬頓賜從官食多未至遣騎趣之于路|{
	從才用翻趣讀曰促}
左散騎常侍孫騭右諫議大夫張衍兵部郎中張儁最後至帝命撲殺之|{
	騭職日翻撲弼角翻考異曰梁祖實錄云賜自盡今從莊宗實錄}
衍宗奭之姪也丙寅帝至武陟|{
	九域志武陟縣在懷州東八十里}
段明遠供饋有加於前丁卯至獲嘉帝追思李思安去歲供饋有闕貶柳州司戶告辭稱明遠之能曰觀明遠之忠勤如此見思安之悖慢何如尋長流思安於崖州賜死|{
	時遠貶者悉賜死柳州遠踰嶺嶠崖州再涉鯨波思安寧得至邪}
明遠後更名凝|{
	更工衡翻}
乙亥帝至魏州命都招討使宣義節度使楊師厚副使前河陽節度使李周彞圍棗彊招討應接使平盧節度使賀德倫副使天平留後袁象先圍蓨縣|{
	九域志棗強縣在鎮州東南五十五里蓨縣在冀州東北一百五十里宋白曰蓨縣即漢條侯國隋開皇五年改條縣為蓨縣蓨音條}
德倫河西胡人象先下邑人也戊寅帝至貝州 辰州蠻酋宋鄴昌師益皆帥衆降于楚|{
	酋慈由翻帥讀曰率}
楚王殷以鄴為辰州刺史師益為溆州刺史|{
	溆音叙}
帝晝夜兼行三月辛巳至下博南登觀津冢|{
	漢觀津縣古城東南有青山即漢文帝竇后父少涓冢也涓是縣人遭秦之亂漁釣隱身墜淵而死景帝立后遣使者填以葬父起大墳於觀津城東南縣民謂之竇氏青山}
趙將苻習引數百騎巡邏不知是帝遽前逼之或告曰晉兵大至矣帝弃行幄亟引兵趣棗彊與楊師厚軍合|{
	邏郎佐翻自下博至棗強六十餘里趣七喻翻}
習趙州人也棗彊城小而堅趙人聚精兵數千人守之師厚急攻之數日不下城壞復脩死傷者以萬數|{
	此言攻城之卒死傷者也}
城中矢石將竭謀出降有一卒奮曰賊自柏鄉喪敗已來視我鎮人裂眥|{
	喪息浪翻眥疾智翻}
今往歸之如自投虎狼之口耳困窮如此何用身為我請獨往試之夜縋城出詣梁軍詐降|{
	縋馳偽翻}
李周彞召問城中之備對曰非半月未易下也|{
	易以䜴翻}
因謀曰|{
	謀當作請}
某旣歸命願得一劒効死先登取守城將首|{
	將即亮翻}
周彞不許使荷檐從軍卒得間舉檐擊周彞首踣地左右救至得免|{
	荷下可翻又如字檐都濫翻間古莧翻踣蒲北翻考異曰莊宗實錄頃之周彞晝寢左右未至其人抽檐擊周彞首踣于地求兵仗不獲周彝大呼左右救至獲免卒睨周彞曰吾比欲剚刃於朱温之腹非圖爾也誤矣編遺錄云時有一百姓來投軍中李周彞收于部伍間謂周彝曰請賜一劒願先登以破其牆未許間忽然抽茶檐子揮擊周彞頭上中檐幾仆于地左右擒之元是棗強邑中遣來詐降本意欲窺筭招討使楊師厚斯人不能辨乃誤中周彝按此卒從周彝請劒周彝不許而令負檐豈不知周彝非温也又帝王與將相居處侍衛不同豈容不識而誤中之若本欲殺楊師厚則似近之今旣可疑皆不取}
帝聞之愈怒命師厚晝夜急攻丙戌拔之無問老幼皆殺之流血盈城初帝引兵度河聲言五十萬晉忻州刺史李存審屯趙州患兵少裨將趙行實請入土門避之存審不可|{
	入土門則退歸晉陽矣}
及賀德倫攻蓨縣存審謂史建瑭李嗣肱曰吾王方有事幽薊無兵此來南方之事委吾輩數人今蓨縣方急吾輩安得坐而視之使賊得蓨縣必西侵深冀患益深矣當與公等以奇計破之存審乃引兵扼下博橋|{
	漳水逕下博縣蓋跨漳水為橋也}
使建瑭嗣肱分道擒生建瑭分其麾下為五隊隊各百人一之衡水一之南宫一之信都|{
	信都漢古縣唐帶冀州蓋其治所雖在郭下而所管地界則環冀州近郊皆是也}
一之阜城自將一隊深入與嗣肱遇梁軍之樵芻者皆執之獲數百人明日會于下博橋皆殺之留數人斷臂縱去曰為我語朱公晉王大軍至矣|{
	斷音短為于偽翻語牛倨翻}
時蓨縣未下帝引楊師厚兵五萬就賀德倫共攻之丁亥始至縣西未及置營建瑭嗣肱各將三百騎效梁軍旗幟服色與樵芻者雜行|{
	幟昌志翻}
日且暮至德倫營門殺門者縱火大譟弓矢亂發左右馳突旣暝各斬馘執俘而去營中大擾不知所為斷臂者復來曰晉軍大至矣|{
	復扶又翻}
帝大駭燒營夜遁|{
	以朱温之狡濟之以楊師厚使遇它敵猶在亂而能整今史建瑭等以奇兵撓之遂相與狼狽至于散遁不能復振者主將上下先有畏晉之心故也}
迷失道委曲行百五十里戊子旦乃至冀州蓨之耕者皆荷鉏奮梃逐之委弃軍資器械不可勝計|{
	梃徒鼎翻勝音升}
旣而復遣騎覘之|{
	覘丑廉翻又丑艶翻}
曰晉軍實未來此乃史先鋒遊騎耳|{
	晉王以史建瑭為先鋒指揮使故稱之}
帝不勝慙憤|{
	親御六軍見敵之遊兵而遁故慙師屢出而屢敗故憤不能自勝言其甚也}
由是病增劇不能乘肩輿留貝州旬餘諸軍始集|{
	潰散之甚久而後集}
義昌節度使劉繼威年少淫虐類其父|{
	劉繼威父守光}
淫於都指揮使張萬進家萬進怒殺之詰旦召大將周知裕告其故萬進自稱留後以知裕為左都押牙庚子遣使奉表請降亦遣使降于晉晉王命周德威安撫之知裕心不自安遂來奔帝為之置歸化軍|{
	為于偽翻}
以知裕為指揮使凡軍士自河朔來者皆隸之辛丑以萬進為義昌留後甲辰改義昌為順化軍以萬進為節度使|{
	為楊師厚劫徙張萬進張本}
乙巳帝發貝州丁未至魏州|{
	貝州南至魏州二百一十里}
戊申周德威遣禆將李存暉等攻瓦橋關|{
	九域志瓦橋關在涿州南一百二十里}
其將吏及莫州刺史李嚴皆降嚴幽州人也涉獵書傳|{
	傳直戀翻}
晉王使傅其子繼岌嚴固辭晉王怒將斬之教練使孟知祥徒跣入諫曰彊敵未滅大王豈宜以一怒戮嚮義之士乎|{
	言非所以招懷燕人}
乃免之知祥遷之弟子|{
	孟遷以邢州降晉又背晉以邢州降梁者也孟知祥始此}
李克讓之壻也|{
	李克讓晉王克用之弟}
吳鎮南節度使劉威歙州觀察使陶雅宣州觀察使李遇常州刺史李簡皆武忠王舊將有大功|{
	楊行密諡武忠王}
以徐温自牙將秉政|{
	徐温自右牙指揮使秉政見二百六十六卷開平元年}
内不能平李遇尤甚常言徐温何人吾未嘗識面一旦乃當國邪館驛使徐玠使於吳越道過宣州温使玠說遇入見新王|{
	說式芮翻見賢遍翻}
遇初許之玠曰公不爾|{
	不爾猶言不如此也}
人謂公反遇怒曰君言遇反殺侍中者非反邪侍中謂威王也|{
	楊渥諡威王李遇斥言徐温殺之}
温怒以淮南節度副使王檀為宣州制置使|{
	王檀恐當作王壇}
數遇不入朝之罪|{
	數所具翻朝直遙翻下同}
遣都指揮使柴再用帥昇潤池歙兵納檀于宣州|{
	帥讀曰率}
昇州副使徐知誥為之副遇不受代再用攻宣州踰月不克 夏四月癸丑以楚王殷為武安武昌靜江寧遠節度使洪鄂四面行營都統|{
	欲使攻楊氏之洪鄂也}
乙卯博王友文來朝|{
	來朝於魏州行宫}
請帝還東都丁巳發魏州己未至黎陽以疾淹留乙丑至滑州|{
	黎陽至滑州隔大河耳今滑州古城已淪於河}
維州羌胡董琢反蜀主遣保鑾軍使趙綽討平之 己巳帝至大梁帝聞嶺南與楚相攻甲戌以右散騎常侍韋戩等為潭廣和叶使|{
	戩子踐翻}
往解之 戊寅帝發大梁 周德威白晉王以兵少不足攻城|{
	言幽州城大而固非兵少所能攻}
晉王遣李存審將吐谷渾契苾騎兵會之|{
	契欺訖翻}
李嗣源攻瀛州刺史趙敬降 五月甲申帝至洛陽疾甚 司空門下侍郎同平章事薛貽矩卒 燕主守光遣其將單廷珪將精兵萬人出戰與周德威遇於龍頭岡|{
	龍頭岡在幽州城東南 考異曰莊宗實錄作羊頭岡今從莊宗列傳莊宗實錄四月己卯朔周德威擒單廷珪進軍大城莊宗薛史及莊宗列傳周德威傳云五月七日擒廷珪十二日次大城今從之}
廷珪曰今日必擒周楊五以獻楊五者德威小名也旣戰見德威於陳|{
	陳讀曰陣}
援槍單騎逐之|{
	援于元翻}
槍及德威背德威側身避之奮檛反擊廷珪墜馬|{
	單廷珪之馬方疾馳勢不得止周德威側身避其鋒馬差過前則德威已在槍裏奮檛擊廷珪廷珪安所避之此其所以墜馬也格鬭之勢刀不如棒謂此也}
生擒置於軍門燕兵退走德威引騎乘之燕兵大敗斬首三千級廷珪燕驍將也燕人失之奪氣 己丑蜀大赦 李遇少子為淮南牙將遇最愛之徐温執之至宣州城下示之其子啼號求生|{
	少詩照翻號戶高翻}
遇由是不忍戰|{
	舉大事者不顧家李遇旣與徐温為敵乃顧一子邪}
温使典客何蕘入城以吳王命說之|{
	蕘如招翻說式芮翻}
曰公本志果反請斬蕘以徇不然隨蕘納款遇乃開門請降温使柴再用斬之夷其族於是諸將始畏温莫敢違其命|{
	諸將謂劉威陶雅輩}
徐知誥以功遷昇州刺史知誥事温甚謹安於勞辱或通夕不解帶温以是特愛之每謂諸子曰汝輩事我能如知誥乎|{
	徐温以善事楊行密而竊吳國之權徐知誥以善事徐温而竊徐氏之權天邪人邪}
時諸州長吏多武夫專以軍旅為務不恤民事知誥在昇州獨選用廉吏脩明政教招延四方士大夫傾家貲無所愛洪州進士宋齊丘好縱横之術|{
	好呼到翻縱子容翻}
謁知誥知誥奇之辟為推官與判官王令謀參軍王翃專主謀議|{
	翃呼萌翻}
以牙吏馬仁裕周宗曹悰為腹心|{
	悰徂宗翻}
仁裕彭城人宗漣水人也|{
	為知誥簒楊氏張本}
閏月壬戌帝疾增甚謂近臣曰我經營天下三十年|{
	帝以唐僖宗中和三年鎮宣武創業之始也至是年三十一年}
不意太原餘孽更昌熾如此|{
	謂晉也孽魚列翻}
吾觀其志不小天復奪我年|{
	復扶又翻下同}
我死諸兒非彼敵也吾無葬地矣因哽咽|{
	哽古杏翻}
絶而復蘇|{
	氣絶而復息為蘇}
高季昌濳有據荆南之志乃奏築江陵外郭增廣之丙寅蜀門下侍郎同平章事王鍇罷為兵部尚書|{
	鍇口}


|{
	駭翻}
帝長子郴王友裕早卒|{
	郴丑林翻}
次假子博王友文|{
	友文本姓康名勤}
帝特愛之常留守東都兼建昌宫使|{
	帝以大梁舊第為建昌宫}
次郢王友珪其母亳州營倡也|{
	倡音昌薛史友珪小字遙喜母失其姓本亳州營妓也唐光啓中帝徇地亳州召而侍寢月餘將捨之而去以娠告是時元貞張后賢而有寵帝素憚之由是不果攜歸大梁因留亳州以别宅貯之及期妓以生男來告帝喜故字之曰遙喜後迎歸汴}
為左右控鶴都指揮使次均王友貞為東都馬步都指揮使初元貞張皇后嚴整多智帝敬憚之后殂|{
	張后殂于唐昭宗天祐元年}
帝縱意聲色諸子雖在外常徵其婦入侍帝往往亂之友文婦王氏色美帝尤寵之雖未以友文為太子帝意常屬之|{
	屬之欲翻}
友珪心不平友珪嘗有過帝撻之友珪益不自安帝疾甚命王氏召友文於東都欲與之訣且付以後事友珪婦張氏亦朝夕侍帝側知之密告友珪曰大家以傳國寶付王氏懷往東都吾屬死無日矣夫婦相泣左右或說之曰事急計生何不改圖時不可失|{
	古人有言曰淫而不父必有子禍豈不信哉說式芮翻}
六月丁丑朔帝命敬翔出友珪為萊州刺史即令之官已宣旨未行勑|{
	敬翔時為宣政使故使之行勑翔佐帝有年矣軍國大謀無不預隨事彌縫轉帝兇暴之氣以成功亦不為小寢疾彌留而出友珪於外使翔能為之謀則必有以處友珪而帝免剚刃之禍顛而不扶焉用彼相哉}
時左遷者多追賜死友珪益恐戊寅友珪易服微行入左龍虎軍見統軍韓勍以情告之勍亦見功臣宿將多以小過被誅懼不自保遂相與合謀|{
	臣子俱逆亦上之人有以致之也被皮義翻}
勍以牙兵五百人從友珪雜控鶴士入伏于禁中|{
	梁以侍衛親軍為控鶴軍}
中夜斬關入至寢殿侍疾者皆散走帝驚起問反者為誰友珪曰非它人也帝曰我固疑此賊恨不早殺之汝悖逆如此天地豈容汝乎|{
	悖蒲内翻又蒲沒翻}
友珪曰老賊萬段友珪僕夫馮廷諤刺帝腹刃出于背|{
	刺七亦翻}
友珪自以敗氈裹之瘞于寢殿|{
	年六十一瘞於計翻}
祕不發喪遣供奉官丁昭溥馳詣東都命均王友貞殺友文己卯矯詔稱博王友文謀逆遣兵突入殿中賴郢王友珪忠孝將兵誅之保全朕躬然疾因震驚彌致危殆宜令友珪權主軍國之務韓勍為友珪謀|{
	為于偽翻}
多出府庫金帛賜諸軍及百官以取悅辛巳丁昭溥還|{
	還從宣翻}
聞友文已死乃發喪宣遺制友珪即皇帝位時朝廷新有内難中外人情忷忷|{
	難乃旦翻忷許勇翻}
許州軍士更相告變匡國節度使韓建皆不之省亦不為備|{
	更工衡翻省悉景翻史言韓建死期將至}
丙申馬步都指揮使張厚作亂殺建 |{
	考異曰莊宗實錄九月建遇害今從薛史}
友珪不敢詰|{
	詰去吉翻}
甲辰以厚為陳州刺史 秋七月丁未大赦 天雄節度使羅周翰幼弱軍府事皆决於牙内都指揮使潘晏北面都招討使宣義節度使楊師厚軍於魏州久欲圖之憚太祖威嚴不敢發至是師厚館於銅臺驛|{
	因銅雀臺以名驛然銅雀臺在鄴不在魏州}
潘晏入謁執而殺之引兵入牙城據位視事壬子制以師厚為天雄節度使|{
	按考異曰梁功臣列傳楊師厚傳云太祖初弃天下郡府乘間為亂甚衆魏之衙内都指揮使潘晏與大將臧延範趙訓將謀反變有密告者師厚布兵擒捕斬之七月除魏博節度使薛史師厚傳略同今從莊宗列傳朱友珪傳及莊宗實錄}
徙周翰為宣義節度使|{
	唐僖宗文德元年羅弘信得魏博傳子至孫而亡}
以侍衛諸軍使韓勍領匡國節度使|{
	韓勍以同逆領節}
甲寅加吳越王鏐尚父 甲子以均王友貞為開封尹東都留守蜀太子元坦更名元膺|{
	宗懿更名元坦見上卷開平四年按歐史蜀主建時得銅}


|{
	牌子於什邡縣有文二十餘字建以為符讖因取之以名諸子故又更名元膺更工衡翻}
丙寅廢建昌宫使以河南尹張宗奭為國計使凡天下金穀舊隸建昌宫者悉主之|{
	梁祖受禪以博王友文領建昌宫使專領金穀友珪旣殺友文故廢之而置國計使}
八月龍驤軍三千人戍懷州者|{
	戍懷州所以備晉人自上黨下太行以窺洛陽}
潰亂東走所過剽掠|{
	剽匹妙翻 考異曰莊宗列傳友珪傳云重霸據懷州為亂壯健者團結於鞏村將為朱温雪恥明宗實錄杜晏球傳云龍驤軍作亂欲入京城已至河陽今按梁祖實錄戊子鄭州奏稱懷州屯駐龍驤騎軍潰散十一日夜至州南十五里鞏村安下及五鼔分隊逃逸安得據懷州及至河陽事也}
戊子遣東京馬步軍都指揮使霍彦威左耀武指揮使杜晏球討之庚寅擊破亂軍執其都將劉重遇於鄢陵甲午斬之|{
	為友貞以龍驤軍起義誅友珪張本}
郢王友珪旣簒立諸宿將多憤怒雖曲加恩禮終不悦告哀使至河中護國節度使冀王朱友謙泣曰先帝數十年開創基業前日變起宫掖聲聞甚惡|{
	聞音問}
吾備位藩鎮心竊恥之|{
	朱友謙本陜州牙將朱簡也唐末附朱温賜名友謙列於諸子故因此聲友珪弑逆之罪律以古法臣弑君子弑父凡在官者殺無赦則友珪之罪凡為梁之臣子者皆得而誅之也}
友珪加友謙侍中中書令以詔書自辯且徵之友謙謂使者曰所立者為誰先帝晏駕不以理吾且至洛陽問罪何以徵為戊戌以侍衛諸軍使韓勍為西面行營招討使督諸軍討之友謙以河中附於晉以求救九月丁未以感化節度使康懷貞為河中都招討使更以韓勍副之 友珪以兵部尚書知崇政院事敬翔太祖腹心恐其不利於己欲解其内職|{
	内職謂知崇政院事}
恐失人望庚午以翔為中書侍郎同平章事壬申以戶部尚書李振充崇政院使翔多稱疾不預事|{
	敬翔李振於此時皆先朝佐命功臣也李振代敬翔領崇政院使則振與友珪同惡敬翔雖稱疾不預事若律之以古人主在與在主亡與亡之法亦不免於死}
康懷貞等與忠武節度使牛存節合兵五萬屯河中城西攻之甚急晉王遣其將李存審李嗣肱李嗣恩將兵救之敗梁兵于胡壁|{
	敗補邁翻}
嗣恩本駱氏子也|{
	歐史義兒傳嗣恩本姓駱吐谷渾部人}
吳武忠王之疾病也周隱請召劉威|{
	事見二百六十五卷唐天祐二年}
威由是為帥府所忌|{
	帥府謂廣陵帥府帥所類翻}
或譛之於徐温温將討之威幕客黄訥說威曰|{
	說式芮翻}
公受謗雖深反本無狀若輕舟入覲則嫌疑皆亡矣威從之陶雅聞李遇敗亦懼與威偕詣廣陵温待之甚恭如事武忠王之禮優加官爵雅等悅服由是人皆重温訥蘇州人也温與威雅帥將吏請於李儼承制加嗣吳王隆演太師吳王|{
	隆演之嗣吳王李茂貞承制所加也楊行密因李儼來使尊之承制徐温等因其舊而請於儼帥讀曰率}
以温領鎮海節度使同平章事淮南行軍司馬如故温遣威雅還鎮|{
	劉威鎮洪州陶雅鎮歙州徐温事威雅如事楊行密貴而不敢忘舊者能矯情為之至於遣威雅歸鎮不特時人服之威雅亦心服矣自古以來英雄分量固自不同至於隨其分量以制一時之事則一也善觀史者毋忽諸}
辛巳蜀改劒南東川曰武德軍 朱友謙復告急于晉|{
	復扶又翻}
冬十月晉王自將自澤潞而西|{
	不自太原南出汾晉將即亮翻}
遇康懷貞於解縣|{
	宋白曰解縣漢舊縣後魏改為北解縣按此前解縣在臨晉縣界隋開皇十六年於此置解縣大業二年省九年自綏化故城移虞鄉縣於廢縣理唐武德元年改虞鄉縣為解縣仍於蒲州界别置虞鄉縣九域志解在蒲州東九十五里虞鄉在蒲州東六十里解戶買翻 考異曰莊宗同光四年實錄莊宗列傳薛史唐餘錄朱友謙傳皆云與汴軍遇於平陽大破之今從莊宗天祐九年實錄}
大破之斬首千級追至白徑嶺而還|{
	白徑嶺在河中安邑縣東}
梁兵解圍退保陜州|{
	九域志河中南至陜州二百三十八里陜失冉翻}
友謙身自至猗氏謝晉王|{
	九域志猗氏縣在河中府東北九十五里}
從者數十人撤武備詣晉王帳拜之為舅晉王夜置酒張樂友謙大醉晉王留宿帳中友謙安寢鼾息自如|{
	朱友謙以此示委心晉王無所猜間也鼾下旦翻}
明旦復置酒而罷 楊師厚旣得魏博之衆又兼都招討使宿衛勁兵多在麾下諸鎮兵皆得調發|{
	調徒釣翻}
威勢甚重心輕郢王友珪遇事往往專行不顧友珪患之發詔召之云有北邊軍機欲與卿面議師厚將行其腹心皆諫曰往必不測師厚曰吾知其為人雖往如我何乃帥精兵萬餘人度河趣洛陽|{
	帥讀曰率趣七喻翻}
友珪大懼丁亥至都門|{
	城外郭門曰都門}
留兵於外與十餘人入見|{
	見賢遍翻}
友珪喜甘言遜詞以悦之賜與巨萬癸巳遣還十一月趙將王德明將兵三萬掠武城|{
	武城漢之東武城縣唐屬}


|{
	貝州九域志在州東五十里}
至于臨清攻宗城下之癸丑楊師厚伏兵唐店邀擊大破之斬首五千餘級 甲寅葬神武元聖孝皇帝于宣陵|{
	宣陵在河南伊闕縣}
廟號太祖 吳淮南節度副使陳璋等將水軍襲楚岳州執刺史苑玫|{
	開平元年楚取岳州三年苑玫降楚至此為淮南所執玫自江西降楚楚使之守岳州也}
楚王殷遣水軍都指揮使楊定真救岳州璋等進攻荆南高季昌遣其將倪可福拒之吳恐楚人救荆南遣撫州刺史劉信將江撫袁吉信五州兵屯吉州為璋聲援|{
	屯吉州以張聲勢若將進兵攻潭衡者以牽制楚兵}
十二月戊寅蜀行營都指揮使王宗汾攻岐文州拔之守將李繼夔走|{
	文州古隂平之地將即亮翻下同}
是歲隰州都將劉訓殺刺史以州降晉晉王以為瀛州刺史訓永和人也|{
	永和縣屬隰州漢狐讘縣地隋為永和縣九域志在州西一百里}
虔州防禦使李彦圖卒州人奉譚全播知州事遣使内附詔以全播為百勝防禦使虔韶二州節度開通使|{
	虔州先有百勝指揮今因以為軍州之號開通使者言使之開通道路南達交廣也}
高季昌出兵聲言助梁伐晉進攻襄州山南東道節度使孔勍擊敗之自是朝貢路絶|{
	高季昌旣與孔勍交惡入梁之路遂絶不復朝貢敗補邁翻}
勍兖州人也均王上上|{
	諱友貞太祖第三子王溥會要曰太祖第四子母曰元貞皇后張氏即位改名瑱其後又改名鍠余按王溥云第四子者併假子博王友文數之也}


乾化三年春正月丁巳晉周德威拔燕順州|{
	唐貞觀四年平突厥以其部落置順祐化長四州六年以順州僑治營州南之五柳戍沈括曰幽州東北三十里有望京館東行少北十里餘出古長城又二十里至中頓又踰孫侯河行二十里至順州其北平斥土厚宜稼又東北行七十里至檀州金人疆域圖順州至燕京一百十五里匈奴須知順州南至燕京九十里其載道里遠近不同今並存之宋白曰幽州東北至順州八十里大元順州領懷柔密雲二縣屬大同府路}
癸亥郢王友珪朝享太廟|{
	朝直遙翻}
甲子祀圓丘大赦改元鳳歷 |{
	考異曰莊宗列傳云七日實錄云庚戌友珪祀圓丘改元今從薛史}
吳陳璋攻荆南不克而還荆南兵與楚兵會於江口以邀之|{
	江口荆江口也還音旋又如字}
璋知之舟二百艘駢為一列夜過二鎮兵遽出追之不能及|{
	艘蘇遭翻}
晉周德威拔燕安遠軍薊州將成行言等降于晉|{
	將即亮翻宋白曰薊州治漁陽本春秋無終子之國隋開皇初徙玄州於此煬帝廢州立漁陽郡唐初廢郡其地屬幽州開元十八年置薊州取古薊門關以名州西至幽州二百一十里}
二月壬午蜀大赦 郢王友珪旣得志遽為荒淫内外憤怒友珪雖啗以金繒終莫之附|{
	啗徒濫翻繒慈陵翻}
駙馬都尉趙巖犨之子|{
	趙犨守陳州拒黄巢有功見唐僖宗紀}
太祖之壻也|{
	巖尚太祖女長樂公主}
左龍虎統軍侍衛親軍都指揮使袁象先太祖之甥也|{
	袁象先父敬初尚太祖妹萬安大長公主}
巖奉使至大梁|{
	使疏吏翻}
均王友貞密與之謀誅友珪巖曰此事成敗在招討楊令公耳|{
	楊師厚官中書令為北面都招討使故稱之}
得其一言諭禁軍吾事立辦|{
	時梁重兵皆在楊師厚之手又勲名為衆所服故欲得其言諭禁軍}
均王乃遣腹心馬慎交之魏州說楊師厚曰郢王簒弑人望屬在大梁|{
	說式芮翻屬之欲翻}
公若因而成之此不世之功也且許事成之日賜犒軍錢五十萬緡|{
	犒苦到翻}
師厚與將佐謀之曰方郢王弑逆吾不能即討今君臣之分已定|{
	分扶問翻}
無故改圖可乎或曰郢王親弑君父賊也均王舉兵復讎義也奉義討賊何君臣之有彼若一朝破賊公將何以自處乎|{
	處昌呂翻}
師厚曰吾幾誤計|{
	幾居依翻}
乃遣其將王舜賢至洛陽隂與袁象先謀遣招討馬步都虞候譙人朱漢賓將兵屯滑州為外應|{
	譙漢縣唐帶亳州}
趙巖歸洛陽亦與象先密定計友珪治龍驤軍潰亂者|{
	去年懷州龍驤軍亂治直之翻}
搜捕其黨獲者族之經年不已時龍驤軍有戍大梁者友珪徵之均王因使人激怒其衆曰天子以懷州屯兵叛追汝輩欲盡阬之 |{
	考異曰莊宗列傳朱友貞傳及薛史歐陽史末帝紀云左右龍驤都戍汴友貞偽作友珪詔追還洛下莊宗實錄云友珪疑而召之按梁太祖實錄云丙戍東京言龍驤軍准詔追赴西京軍情不肯進發實友珪徵之非友貞偽作但激怒言坑之耳}
其衆皆懼莫知所為丙戌均王奏龍驤軍疑懼未肯前發戊子龍驤將校見均王泣請可生之路|{
	將即亮翻校戶教翻}
王曰先帝與汝輩三十餘年征戰經營王業今先帝尚為人所弑汝輩安所逃死乎因出太祖畫像示之而泣曰汝能自趣洛陽雪讎恥|{
	趣七喻翻}
則轉禍為福矣衆皆踊躍呼萬歲請兵仗王給之庚寅旦袁象先等帥禁兵數千人突入宫中|{
	帥讀曰率}
友珪聞變與妻張氏及馮廷諤趨北垣樓下將踰城自度不免|{
	趣七喻翻度徒洛翻}
令廷諤先殺妻後殺已廷諤亦自剄|{
	剄古頂翻斷首也}
諸軍十餘萬大掠都市|{
	汴兵未至洛陽禁衛諸軍已殺友珪矣}
百司逃散中書侍郎同平章事杜曉侍講學士李珽皆為亂兵所殺|{
	珽它鼎翻}
門下侍郎同平章事于兢宣政使李振被傷至晡乃定象先巖齎傳國寶詣大梁迎均王王曰大梁國家創業之地|{
	梁祖自宣武節度使并諸鎮}
何必洛陽乃即帝位於大梁復稱乾化三年追廢友珪為庶人復博王友文官爵 丙申晉李存暉攻燕檀州刺史陳確以城降|{
	匈奴須知檀州南至燕京一百六十里東南至薊州一百九十里宋白曰檀州古白檀之地}
蜀唐道襲自興元罷歸復為樞密使太子元膺廷疏道襲過惡|{
	疏分列也於朝會廷中條分列言其過惡故曰廷疏}
以為不應復典機要|{
	復扶又翻}
蜀主不悅庚子以道襲為太子太保 三月甲辰朔晉周德威拔燕盧臺軍 丁未帝更名鍠久之又名瑱|{
	更工衡翻鍠戶盲翻瑱它甸翻 考異曰薛史云貞明中更名瑱諸書皆無年月今因名鍠終言之}
庚戌加楊師厚兼中書令賜爵鄴王賜詔不名事無巨細必咨而後行 帝遣使招撫朱友謙友謙復稱藩奉梁年號|{
	去年朱友謙附晉今雖復稱藩實隂附于晉}
丙辰立皇弟友敬為康王 乙丑晉將劉光濬克古

北口|{
	檀州燕樂縣東有東軍北口二守捉北口長城口也沈括曰檀州東北五十里有金溝館自館少東北行乍原乍隰三十餘里至中頓過頓屈折北行峽中濟灤水通三十餘里鉤折投山隙以度所謂古北口也匈奴須知虎北口南至燕京三百里}
燕居庸關使胡令圭等奔晉|{
	幽州昌平縣北十五里有軍都陘西北三十五里有納款關即居庸故關}
戊辰以保義留後戴思遠為節度使鎮邢州|{
	唐昭義軍統潞澤邢洺磁五州唐末兵爭晉得潞州仍以為昭義軍自孟方立以至於梁以邢洺磁三州為昭義軍遂有兩昭義軍今梁改邢洺磁為保義軍而以陜州之保義軍為鎮國軍 考異曰薛史思遠傳云貞明中為邢州留後屬張萬進殺劉繼威命思遠鎮之按萬進殺繼威在前今從本紀}
燕主守光命大將元行欽將騎七千牧馬於山北募山北兵以應契丹|{
	劉守光求救于契丹故使元行欽募兵於山北以應之}
又以騎將高行珪為武州刺史以為外援晉李嗣源分兵徇山後八軍皆下之晉王以其弟存矩為新州刺史總之|{
	為存矩以驕惰致亂張本}
以燕納降軍使盧文進為裨將李嗣源進攻武州高行珪以城降元行欽聞之引兵攻行珪行珪使其弟行周質於晉軍以求救|{
	質音致}
李嗣源引兵救之行欽解圍去嗣源與行周追至廣邊軍|{
	媯州懷戎縣北有廣邊軍故白雲城也宋白曰廣邊軍在媯州北一百三十里高行周兄弟本貫廣邊軍鵰窠村}
凡八戰行欽力屈而降嗣源愛其驍勇養以為子 |{
	考異曰莊宗實錄行周作行温張昭周太祖實錄云燕城危䠞甲士亡散劉守光召元行欽行欽部下諸將以守光必敗赴召無益乃請行欽為燕帥稱留後行欽無如之何乃謂諸將曰我為帥亦須歸幽州衆然之行欽以行珪在武州慮為後患乃令人於懷戎掠得其子縶之自隨至武州行欽謂行珪曰將士立我為留後共汝父子同行先定軍府然後降太原若不從必殺汝子行珪曰大王委爾親兵遂圖叛逆吾死不能從也其子泣告行珪行珪謂曰元公謀逆何以順從與爾訣矣行珪守城月餘城中食盡士有飢色行珪乃召集居人謂之曰非不為父老惜家屬不幸軍士乏食可斬予首出降即坐見寧帖行珪為治有恩衆泣曰願出私糧濟軍以死共守乃夜縋其弟行周於晉軍乞兵救援周德威命李嗣本李嗣源安金全救武州比至行欽解圍矣嗣源與行珪追躡至廣邊軍行欽帥騎拒戰行珪呼謂行欽曰與公俱事劉家我為劉家守城爾則僭稱留後誰之過也今日之事何勞士衆與君抗衡以決勝負行欽驍猛騎射絶衆報曰可行周馬足微蹶將踣嗣源躍馬救之檛擊行欽幾墜行欽正身引弓射嗣源中髀貫鞍嗣源拔矢凡八戰控弦七發矢中行欽猶沫血酣戰不解是夜行欽窮蹙固守廣邊軍晉兵圍之嗣源遣人告之曰彼此戰將不假言諭事勢可量亟來相見必保功名翌日行欽面縳出降嗣源酌酒飲之撫其背曰吾子壯士也養為假子臨敵擒生必有所獲名聞軍中莊宗實錄薛史紀及行欽傳明宗實錄皆云行欽聞行珪降晉帥兵攻之惟周太祖實錄高行周傳云行欽稱留後行珪城守不從然恐行周卒時去燕亡已久行周名位尊顯門生故吏虛美其兄弟故與諸說特異今從衆書}
嗣源進攻儒州拔之|{
	唐末於媯州東置儒州領晉山一縣}
以行珪為代州刺史行周留事嗣源常與嗣源假子從珂分將牙兵以從|{
	將即亮翻從才用翻}
從珂母魏氏鎮州人先適王氏生從珂嗣源從晉王克用戰河北得魏氏以為妾故從珂為嗣源子及長以勇健知名嗣源愛之|{
	李從珂始此 考異曰張昭於國初脩唐廢帝實錄云廢帝諱從珂明宗皇帝之元子也母曰宣憲皇后魏氏鎮州平山人中和末明宗徇地山東留戍平山得魏后帝以光啟元年正月二十三日生於外舍屬趙人負盟用兵不息音問阻絶帝甫十歲方得歸宗時明宗為裨將性闊達不能治生曹后亦疏於畫略生計所資惟宣憲而已曹后未有胎胤幹家宜室帝與部曲王建立皇甫立代北往來供饋曹后憐之不異所生薛史末帝諱從珂本姓王氏鎮州人也母宣憲皇后魏氏以光啟元年生帝於平山景福中明宗為武皇騎將略地至平山遇魏氏虜之帝時年十餘歲明宗養為已子劉恕取廢帝錄以為明宗即位後不立從珂而欲立從榮從榮死傳位於從厚故人皆謂從珂為養子按張昭仕明宗為史官異代脩廢帝錄無所避諱而不言養子事似可信然李克用光啟以前未嘗徇地山東又從珂若果是明宗子明宗必不捨之而立從榮從珂亦當不服今從薛史}
吳行營招討使李濤帥衆二萬出千秋嶺攻吳越衣錦軍|{
	自杭州東南度千秋嶺則至杭州臨安縣薛史梁開平二年改臨安縣廣義鄉為衣錦鄉帥讀曰率衣於旣翻}
吳越王鏐以其子湖州刺史傳瓘為北面應援都指揮使以救之睦州刺史傳璙為招討收復都指揮使將水軍攻吳東洲以分其兵勢|{
	東洲即常州東洲也璙力弔翻又力小翻}
夏四月癸未以袁象先領鎮南節度使|{
	鎮南軍洪州時屬吳此所謂名號節度使也五代及十國皆有之}
同平章事 晉周德威進軍逼幽州南門壬辰燕主守光遣使致書於德威以請和語甚卑而哀德威曰大燕皇帝尚未郊天何雌伏如是邪|{
	漢趙温曰大丈夫當雄飛安能雌伏}
予受命討有罪者結盟繼好|{
	好呼到翻}
非所聞也不答書守光懼復遣人祈哀|{
	復扶又翻}
德威乃以聞於晉王 千秋嶺道險狹錢傳瓘使人伐木以斷吳軍之後而擊之|{
	斷音短}
吳軍大敗虜李濤及士卒三千餘人以歸 己亥晉劉光濬拔燕平州執刺史張在吉五月光濬攻營州刺史楊靖降|{
	宋白曰平州東北至營州六百九十里}
乙巳蜀主以兵部尚書王鍇為中書侍郎同平章事|{
	鍇口駭翻}
楊師厚與劉守奇將汴滑徐兗魏博邢洺之兵十萬大掠趙境|{
	楊師厚以燕晉交兵乘虚掠趙}
師厚自柏鄉入攻土門趣趙州守奇自貝州入趣冀州|{
	九域志柏鄉北至趙州七十餘里貝州北至冀州一百二十餘里趣七喻翻}
所過焚掠庚戌師厚至鎮州|{
	九域志趙州北至鎮州九十五里}
營於南門外燔其關城壬子師厚自九門退軍下博守奇引兵與師厚會攻下博拔之晉將李存審史建瑭戍趙州兵少趙王告急於周德威德威遣騎將李紹衡會趙將王德明同拒梁軍師厚守奇自弓高度御河而東|{
	隋煬帝大業四年穿永濟渠引沁水南達於河北通涿郡後人因謂之御河}
逼滄州張萬進懼請遷于河南師厚表徙萬進鎮青州以守奇為順化節度使|{
	去年改滄州義昌軍為順化軍}
吳遣宣州副指揮使花虔將兵會廣德鎮遏使渦信屯廣德|{
	渦古禾翻姓也}
將復寇衣錦軍|{
	復扶又翻}
吳越錢傳瓘就攻之 六月壬申朔晉王遣張承業詣幽州與周德威議軍事 丙子蜀主以道士杜光庭為金紫光祿大夫左諫議大夫封蔡國公進號廣成先生光庭博學善屬文|{
	屬之欲翻}
蜀主重之頗與議政事 吳越錢傳瓘拔廣德虜花虔渦信以歸戊子以張萬進為平盧節度使 辛卯燕主守光遣

使詣張承業請以城降承業以其無信不許 蜀太子元膺豭喙齙齒|{
	豭古牙翻牡豕也喙許穢翻齙步交翻露齒也}
目視不正而警敏知書善騎射性狷急猜忍|{
	狷吉掾翻}
蜀主命杜光庭選純靜有德者使侍東宫光庭薦儒者許寂徐簡夫太子未嘗與之交言日與樂工羣小嬉戲無度僚屬莫敢諫秋七月蜀主將以七夕出遊丙午太子召諸王大臣宴飲集王宗翰内樞密使潘峭翰林學士承旨高陽毛文錫不至太子怒曰集王不來必峭與文錫離間也|{
	峭七肖翻間古莧翻}
大昌軍使徐瑶常謙素為太子所親信酒行屢目少保唐道襲道襲懼而起丁未旦太子入白蜀主曰潘峭毛文錫離間兄弟蜀主怒命貶逐峭文錫以前武泰節度使兼侍中潘炕為内樞密使|{
	炕苦浪翻}
太子出道襲入蜀王以其事告之道襲曰太子謀作亂欲召諸將諸王以兵錮之|{
	曰錮者以禁錮為義}
然後舉事耳蜀主疑焉遂不出|{
	遂不以七夕出遊}
道襲請召屯營兵入宿衛許之内外戒嚴太子初不為備聞道襲召兵乃以天武甲士自衛捕潘峭毛文錫至檛之幾死|{
	檛則瓜翻}
囚諸東宫又捕成都尹潘嶠囚諸得賢門戊申徐瑤常謙與懷勝軍使嚴璘等各帥所部兵奉太子攻道襲|{
	帥讀曰率}
至清風樓道襲引屯營兵出拒戰道襲中流矢|{
	中竹仲翻}
逐至城西斬之 |{
	考異曰九國志建將七夕出遊先一日元膺召諸軍使及諸王宴飲邸第中且議七夕從行之禮而集王宗翰等不至又曰詰朝元膺入白建曰潘峭毛文錫離間兄弟將圖不軌又曰及聞唐襲徵兵乃遣伶官安悉香諭軍使全殊率天武甲士以自衛又曰明日徐瑤常謙與懷勝軍使嚴璘等協謀以所部兵挾元膺以逐唐襲元膺介馬率卒過其兄宗賀之門召與同進宗賀曰兵起無名不敢聞命又曰建急召宗侃宗賀及諸軍使令以兵討寇乃逐唐襲至城西斬之盡殺屯營兵又自大安門登陴以入攻瑤謙等歐陽史曰元膺與伶人安悉香軍將喻全殊率天武兵自衛召大將徐瑤常謙率兵出拒襲與襲戰神武門襲中流矢墜馬死十國紀年丁未元膺令軍使喻全殊帥天武兵自衛戊申徐瑤常謙及左大昌軍使王承燧等各帥所部兵奉元膺攻唐道襲道襲自私第被甲乘馬過王宗賀門邀之宗賀曰兵起無名且不奉詔公宜緩行元膺遣天武將唐據帥親兵逐道襲至城西斬之據九國志云徐瑤等挾元膺以逐唐襲似襲在宫中欲逐之也歐陽史云元膺召瑤等帥兵出拒襲攻東宫而元膺拒之紀年云瑤等奉元膺攻唐道襲道襲自私第被甲乘馬似道襲出在外第元膺就攻之也按道襲止以挾君自重旣勸蜀主發兵自衛豈肯更在外第必止於禁中也蓋瑤等引兵攻宫禁以求道襲道襲以屯營兵出拒戰兵敗走至城西為唐據所殺耳九國志又云元膺介馬率卒過其兄宗賀之門召與同進是元膺邀宗賀也紀年云道襲自私第被甲乘馬過宗賀門要之是道襲邀宗賀也按道襲私第安得有兵觀宗賀所答之辭似語太子非語道襲也若語道襲宜勸之速入宿衛豈得云公宜緩行也潘炕言太子非有他志陛下宜面諭大臣以安社稷蓋當時蜀主聞亂旣信遒襲之言又不忍討太子無決然號令故炕言太子無他志當召大臣討徐瑤等為亂者九國志云令宗侃等出兵討寇乃逐唐襲至城西斬之是官軍斬襲也若然何故明日亟加襲贈諡乎此必誤也}
殺屯營兵甚衆中外驚擾潘炕言於蜀主曰太子唐道襲爭權耳無它志也陛下宜面諭大臣以安社稷蜀主乃召兼中書令王宗侃王宗賀前利州團練使王宗魯使發兵討為亂者徐瑤常謙等宗侃等陳於西毬場門|{
	陳讀曰陣}
兼侍中王宗黯自大安門梯城而入與瑤謙戰於會同殿前殺數十人瑤死謙與太子奔龍躍池|{
	龍躍池即摩訶池}
匿於艦中|{
	艦戶黯翻}
己酉太子出就舟人匄食|{
	匄古太翻乞也}
舟人以告蜀主急遣集王宗翰往慰撫之比至|{
	比必利翻}
太子已為衛士所殺蜀主疑宗翰殺之大慟不已左右恐事變會張格呈慰諭軍民牓讀至不行斧鉞之誅將誤社稷之計蜀主收涕曰朕何敢以私害公於是下詔廢太子元膺為庶人宗翰奏誅手刃太子者元膺左右坐誅死者數十人貶竄者甚衆庚戌贈唐道襲太師諡忠壯復以潘峭為樞密使 甲子晉五院軍使拔莫州擒燕將畢元福八月乙亥李信拔瀛州 賜高季昌爵勃海王 晉王與趙王鎔會于天長|{
	即鎮州之天長鎮也}
楚寧遠節度使姚彥章將水軍侵吳鄂州吳以池州團練使呂師造為水陸行營應援使未至楚兵引去 九月甲辰以御史大夫姚洎為中書侍郎同平章事 燕主守光引兵夜出復取順州|{
	是年春正月晉周德威拔燕順州}
吳越王鏐遣其子傳瓘傳璙|{
	璙力彫翻又力弔力小三翻}
及大同節度使傳瑛攻吳常州營於潘葑|{
	今常州無錫縣有潘葑酒庫葑音封}
徐温曰浙人輕而怯|{
	輕墟正翻}
帥諸將倍道赴之|{
	帥讀曰率下同}
至無錫黑雲都將陳祐言於温曰彼謂吾遠來罷倦未能決戰|{
	罷讀曰疲}
請以所部乘其無備擊之乃自它道出敵後温以大軍當其前夾攻之吳越大敗斬獲甚衆 高季昌造戰艦五百艘治城塹繕器械為攻守之具|{
	治直之翻塹七艷翻}
招聚亡命交通吳蜀|{
	東通吳西通蜀}
朝廷浸不能制 冬十月己巳朔燕主守光帥衆五千夜出將入檀州庚午周德威自涿州引兵邀擊大破之守光以百餘騎逃歸幽州其將卒降者相繼 蜀潘炕屢請立太子蜀主以雅王宗輅類已信王宗傑才敏欲擇一人立之鄭王宗衍最幼其母徐賢妃有寵欲立其子使飛龍使唐文扆諷張格上表請立宗衍|{
	扆隱豈翻上時掌翻}
格夜以表示功臣王宗侃等詐云受密旨衆皆署名蜀主令相者視諸子亦希旨言鄭王相最貴|{
	相息亮翻}
蜀主以為衆人實欲立宗衍不得已許之曰宗衍幼懦能堪其任乎甲午立宗衍為太子|{
	為宗衍亡蜀張本}
受冊畢潘炕以朝廷無事稱疾請老蜀主不許涕泣固請乃許之國有大疑常遣使就第問之 嶺南節度使劉巖求昏於楚楚王許以女妻之|{
	妻七細翻}
盧龍巡屬皆入于晉燕主守光獨守幽州城求援於契丹契丹以其無信竟不救守光屢請降於晉晉人疑其詐終不許至是守光登城謂周德威曰俟晉王至吾則開門泥首聽命德威使白晉王十一月甲辰晉王以監軍張承業權知軍府事自詣幽州辛酉單騎抵城下謂守光曰朱温簒逆余本與公合河朔五鎮之兵興復唐祚|{
	五鎮潞鎮定幽滄本字下當有欲字}
公謀之不臧乃效彼狂僭鎮定二帥皆俛首事公|{
	鎮帥王鎔定帥王處直俛音免}
而公曾不之恤是以有今日之役|{
	守光攻易定晉王救之遂伐守光事見上年}
丈夫成敗須決所向公將何如守光曰今日俎上肉耳惟王所裁王憫之與折弓矢為誓|{
	折而設翻}
曰但出相見保無它也|{
	言不殺之}
守光辭以他日先是守光愛將李小喜多贊成守光之惡言聽計從權傾境内|{
	先悉薦翻}
至是守光將出降小喜止之是夕小喜踰城詣晉軍且言城中力竭壬戌晉王督諸軍四面攻城克之擒劉仁恭及其妻妾守光帥妻子亡去癸亥晉王入幽州|{
	唐昭宗乾寧二年劉仁恭據幽州至是父子俱敗亡帥讀曰率}
以寧國節度使王景仁為淮南西北行營招討應接使|{
	梁攻淮南攻其西北}
將兵萬餘侵廬壽|{
	廬壽二州名為王景仁為吳所敗張本}


資治通鑑卷二百六十八
