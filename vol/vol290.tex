<!DOCTYPE html PUBLIC "-//W3C//DTD XHTML 1.0 Transitional//EN" "http://www.w3.org/TR/xhtml1/DTD/xhtml1-transitional.dtd">
<html xmlns="http://www.w3.org/1999/xhtml">
<head>
<meta http-equiv="Content-Type" content="text/html; charset=utf-8" />
<meta http-equiv="X-UA-Compatible" content="IE=Edge,chrome=1">
<title>資治通鑒_291-資治通鑑卷二百九十_291-資治通鑑卷二百九十</title>
<meta name="Keywords" content="資治通鑒_291-資治通鑑卷二百九十_291-資治通鑑卷二百九十">
<meta name="Description" content="資治通鑒_291-資治通鑑卷二百九十_291-資治通鑑卷二百九十">
<meta http-equiv="Cache-Control" content="no-transform" />
<meta http-equiv="Cache-Control" content="no-siteapp" />
<link href="/img/style.css" rel="stylesheet" type="text/css" />
<script src="/img/m.js?2020"></script> 
</head>
<body>
 <div class="ClassNavi">
<a  href="/24shi/">二十四史</a> | <a href="/SiKuQuanShu/">四库全书</a> | <a href="http://www.guoxuedashi.com/gjtsjc/"><font  color="#FF0000">古今图书集成</font></a> | <a href="/renwu/">历史人物</a> | <a href="/ShuoWenJieZi/"><font  color="#FF0000">说文解字</a></font> | <a href="/chengyu/">成语词典</a> | <a  target="_blank"  href="http://www.guoxuedashi.com/jgwhj/"><font  color="#FF0000">甲骨文合集</font></a> | <a href="/yzjwjc/"><font  color="#FF0000">殷周金文集成</font></a> | <a href="/xiangxingzi/"><font color="#0000FF">象形字典</font></a> | <a href="/13jing/"><font  color="#FF0000">十三经索引</font></a> | <a href="/zixing/"><font  color="#FF0000">字体转换器</font></a> | <a href="/zidian/xz/"><font color="#0000FF">篆书识别</font></a> | <a href="/jinfanyi/">近义反义词</a> | <a href="/duilian/">对联大全</a> | <a href="/jiapu/"><font  color="#0000FF">家谱族谱查询</font></a> | <a href="http://www.guoxuemi.com/hafo/" target="_blank" ><font color="#FF0000">哈佛古籍</font></a> 
</div>

 <!-- 头部导航开始 -->
<div class="w1180 head clearfix">
  <div class="head_logo l"><a title="国学大师官网" href="http://www.guoxuedashi.com" target="_blank"></a></div>
  <div class="head_sr l">
  <div id="head1">
  
  <a href="http://www.guoxuedashi.com/zidian/bujian/" target="_blank" ><img src="http://www.guoxuedashi.com/img/top1.gif" width="88" height="60" border="0" title="部件查字,支持20万汉字"></a>


<a href="http://www.guoxuedashi.com/help/yingpan.php" target="_blank"><img src="http://www.guoxuedashi.com/img/top230.gif" width="600" height="62" border="0" ></a>


  </div>
  <div id="head3"><a href="javascript:" onClick="javascript:window.external.AddFavorite(window.location.href,document.title);">添加收藏</a>
  <br><a href="/help/setie.php">搜索引擎</a>
  <br><a href="/help/zanzhu.php">赞助本站</a></div>
  <div id="head2">
 <a href="http://www.guoxuemi.com/" target="_blank"><img src="http://www.guoxuedashi.com/img/guoxuemi.gif" width="95" height="62" border="0" style="margin-left:2px;" title="国学迷"></a>
  

  </div>
</div>
  <div class="clear"></div>
  <div class="head_nav">
  <p><a href="/">首页</a> | <a href="/ShuKu/">国学书库</a> | <a href="/guji/">影印古籍</a> | <a href="/shici/">诗词宝典</a> | <a   href="/SiKuQuanShu/gxjx.php">精选</a> <b>|</b> <a href="/zidian/">汉语字典</a> | <a href="/hydcd/">汉语词典</a> | <a href="http://www.guoxuedashi.com/zidian/bujian/"><font  color="#CC0066">部件查字</font></a> | <a href="http://www.sfds.cn/"><font  color="#CC0066">书法大师</font></a> | <a href="/jgwhj/">甲骨文</a> <b>|</b> <a href="/b/4/"><font  color="#CC0066">解密</font></a> | <a href="/renwu/">历史人物</a> | <a href="/diangu/">历史典故</a> | <a href="/xingshi/">姓氏</a> | <a href="/minzu/">民族</a> <b>|</b> <a href="/mz/"><font  color="#CC0066">世界名著</font></a> | <a href="/download/">软件下载</a>
</p>
<p><a href="/b/"><font  color="#CC0066">历史</font></a> | <a href="http://skqs.guoxuedashi.com/" target="_blank">四库全书</a> |  <a href="http://www.guoxuedashi.com/search/" target="_blank"><font  color="#CC0066">全文检索</font></a> | <a href="http://www.guoxuedashi.com/shumu/">古籍书目</a> | <a   href="/24shi/">正史</a> <b>|</b> <a href="/chengyu/">成语词典</a> | <a href="/kangxi/" title="康熙字典">康熙字典</a> | <a href="/ShuoWenJieZi/">说文解字</a> | <a href="/zixing/yanbian/">字形演变</a> | <a href="/yzjwjc/">金 文</a> <b>|</b>  <a href="/shijian/nian-hao/">年号</a> | <a href="/diming/">历史地名</a> | <a href="/shijian/">历史事件</a> | <a href="/guanzhi/">官职</a> | <a href="/lishi/">知识</a> <b>|</b> <a href="/zhongyi/">中医中药</a> | <a href="http://www.guoxuedashi.com/forum/">留言反馈</a>
</p>
  </div>
</div>
<!-- 头部导航END --> 
<!-- 内容区开始 --> 
<div class="w1180 clearfix">
  <div class="info l">
   
<div class="clearfix" style="background:#f5faff;">
<script src='http://www.guoxuedashi.com/img/headersou.js'></script>

</div>
  <div class="info_tree"><a href="http://www.guoxuedashi.com">首页</a> > <a href="/SiKuQuanShu/fanti/">四库全书</a>
 > <h1>资治通鉴</h1> <!--         下载:【右键另存为】即可 --></div>
  <div class="info_content zj clearfix">
  
<div class="info_txt clearfix" id="show">
<center style="font-size:24px;">291-資治通鑑卷二百九十</center>
    資治通鑑卷二百九十   宋 司馬光 撰<br />
<br />
  胡三省 音註<br />
<br />
  後周紀一【起重光大淵獻盡玄困敦八月凡一年有奇 周自以為周虢叔之後春秋戰國之世傳記謂虢叔之後有國者為虢公後謂之郭公虢郭音相近也虞大夫宫之奇曰虢仲虢叔王季之穆也郭之得姓本於周故建國號曰周通鑑因謂之後周】<br />
<br />
  太祖聖神恭肅文武孝皇帝上【姓郭氏諱威邢州堯山人父簡事晉為順州刺史】<br />
<br />
  廣順元年春正月丁卯漢太后下誥授監國符寶即皇帝位監國自臯門入宫【臯門大梁城外村名】即位於崇元殿制曰朕周室之裔虢叔之後國號宜曰周改元大赦楊邠史弘肇王章等皆贈官官為歛葬【楊邠等死見上卷上年為于偽翻歛力贍翻】仍訪其子孫叙用之凡倉場庫務掌納官吏無得收斗餘稱耗【斗餘槩量之外又取其餘也稱耗稱計斤釣石之外又多取之以備耗折今悉除之矯王章苛歛之弊也稱尺證翻】舊所進羨餘物悉罷之【羨弋戰翻羨餘唐之流弊也至五季而愈甚】犯竊盜及姦者並依晉天福元年以前刑名罪人非反逆無得誅及親族籍沒家貲【矯史弘肇虐刑之弊也】唐莊宗明宗晉高祖各置守陵十戶漢高祖陵軄員宫人時月薦享及守陵戶並如故初唐衰多盜不用律文更定峻法竊盜贓三匹者死晉天福中加至五匹姦有夫婦人無問強和男女並死【強謂男以威力加女女不得已而與之通姦者和謂男女相慕欲動情生而通姦者】漢法竊盜一錢以上皆死又罪非反逆往往族誅籍沒故帝即位首革其弊初楊邠以功臣國戚為方鎮者多不閑吏事【閑習也】乃以三司軍將補都押牙孔目官内知客其人自恃敇補多專横【横下孟翻】節度使不能制至是悉罷之帝命史弘肇親吏上黨李崇矩訪弘肇親族崇矩言弘肇弟弘福今存初弘肇使崇矩掌其家貲之籍由是盡得其產皆以授弘福帝賢之使隸皇子榮帳下 戊辰以前復州防禦使王彦超權武寧節度使【時劉贇將鞏廷美等守徐州】 漢李太后遷居西宫【按薛史漢太平宫蓋即西宫】己巳上尊號曰昭聖皇太后【上時掌翻】 開封尹兼中書令劉勲卒 癸酉加王峻同平章事 以衛尉卿劉皥主漢隱帝之喪【劉勲既卒它無親屬故也】 初河東節度使兼中書令劉崇聞隱帝遇害欲舉兵南向聞迎立湘隂公乃止曰吾兒為帝吾又何求太原少尹李驤隂說崇曰【說式芮翻】觀郭公之心終欲自取公不如疾引兵逾太行據孟津【行戶剛翻】俟徐州相公即位【湘隂公本鎮徐州故稱之】然後還鎮則郭公不敢動矣不然且為所賣崇怒曰腐儒欲離間吾父子命左右曳出斬之【間古莧翻曳讀曰拽音羊列翻】驤呼曰吾負經濟之才而為愚人謀事【呼火故翻為于偽翻】死固甘心家有老妻願與之同死崇并其妻殺之且奏於朝廷示無二心及贇廢崇乃遣使請贇歸晉陽詔報以湘隂公比在宋州【比毗至翻】今方取歸京師必令得所公勿以為憂公能同力相輔當加王爵永鎮河東鞏廷美楊温聞湘隂公失位奉贇妃董氏據徐州拒守以俟河東援兵【劉贇令鞏廷美等守徐州事始見上卷上年】帝使贇以書諭之廷美温欲降而懼死【降戶江翻】帝復遺贇書曰爰念斯人盡心於主【復扶又翻遺唯李翻主謂劉贇】足以賞其忠義何由責以悔尤俟新節度使入城【新節度使謂王彦超】當各除刺史公可更以委曲示之【唐宋主帥以手書諭示將佐率謂之委曲】 契丹之攻内丘也【事見上卷上年】死傷頗多又值月食軍中多妖異契丹主懼不敢深入引兵還【胡人用兵以月為候月食又多妖異故懼而不敢進妖一遥翻還從宣翻又如字】遣使請和於漢會漢亡安國節度使劉詞送其使者詣大梁帝遣左千牛衛將軍朱憲報聘且叙革命之由以金器玉帶贈之 帝以鄴都鎮撫河北控制契丹欲以腹心處之【處昌呂翻】乙亥以寧江節度使侍衛親軍都指揮使王殷為鄴都留守天雄節度使同平章事領軍如故【仍領侍衛親軍】仍以侍衛司從赴鎮 丙子帝帥百官詣西宫為漢隱帝舉哀成服皆如天子禮【去年遷隱帝梓宫於西宫事見上卷帥讀曰率為于偽翻】 慕容彦超遣使入貢帝慮其疑懼賜詔慰安之曰今兄事已至此言不欲繁望弟扶持同安億兆【漢祖慕容彦超之兄也今兄□史作令兄當從之】 戊寅殺湘隂公於宋州 是日劉崇即皇帝位於晉陽【劉崇漢祖母弟也】仍用乾祐年號所有者并汾忻代嵐憲隆蔚沁遼麟石十二州之地【宋白曰憲州故樓煩監牧唐昭宗龍紀元年李克用奏置憲州宋太宗之平太原折御卿自府州會兵攻劉繼元先克岢嵐軍次克隆州次克嵐州則隆州盖晉漢間所置其地在岢嵐嵐谷之間沁午鴆翻】以節度判官鄭珙為中書侍郎【珙居勇翻】觀察判官滎陽趙華為戶部侍郎並同平章事以次子承鈞為侍衛親軍都指揮使太原尹以節度副使李存瓌為代州防禦使禆將武安張元徽為馬步軍都指揮使【九域志武安縣属洛州在州西九十五里】陳光裕為宣徽使北漢主謂李存瓌張元徽曰【通鑑書嶺南之漢為南漢河東之漢為北漢】朕以高祖之業一朝墜地今日位號不得已而稱之顧我是何天子汝曹是何節度使邪由是不建宗廟祭祀如家人宰相月俸止百緡節度使止三十緡【按唐世百官俸錢自會昌以後不復增減三師二百萬三公百六十萬侍中百五十萬中書令兩省侍郎兩僕射東宫三師百四十萬尚書御史大夫東宫三少百萬節度使三十萬至梁開平五年宰臣俸二百千後唐同光四年定節度副使每月料錢四十千則節度使當又多今北漢主皆減其數俸扶用翻】自餘薄有資給而已故其國中少廉吏【少詩沼翻】客省使河南李光美嘗為直省官【三省有直省官凡百官詣宰相皆差直省官引接其職則外鎮客司通引之職也】頗諳故事【諳烏含翻】北漢朝廷制度皆出於光美北漢主聞湘隂公死哭曰吾不用忠臣之言以至於此為李驤立祠【為于偽翻】歲時祭之 己卯以太師馮道為中書令加竇貞固侍中蘇禹珪司空 王彦超奏遣使齎敕詣徐州鞏廷美等猶豫不肯啓關詔進兵攻之 帝謂王峻曰朕起於寒微備嘗艱苦遭時喪亂【喪息浪翻】一旦為帝王豈敢厚自奉養以病下民乎命峻疏四方貢獻珍美食物庚辰下詔悉罷之【按薛史本紀詔應天下州府舊貢滋味食饌之物所宜除減其兩浙進細酒海味薑瓜湖南子茶乳糖白沙糖橄欖子鎮州高公米水梨易定栗子河東白杜梨米粉菉豆粉玉屑籸子麫永興御田紅秔米新大麥麫興平蘇栗子華州麝香羚羊角熊膽獺肝朱柿熊白河中樹紅棗五味子輕餳同州石餅晉絳蒲萄黄消梨陜府鳳栖梨襄州紫薑新笋橘子安州折粳米糟味青州水梨河陽諸雜果子許州御李子鄭州新笋鵝梨懷州寒食杏仁申州蘘荷亳州萆薢沿淮州郡淮白魚今後不須進奉】其詔略曰所奉止於朕躬所損被於甿庶【被皮義翻甿謨耕翻】又曰積於有司之中甚為無用之物又詔曰朕生長軍旅【長知兩翻】不親學問未知治天下之道【治直之翻】文武官有益國利民之術各具封事以聞咸宜直書其事勿事辭藻帝以蘇逢吉之第賜王峻峻曰是逢吉所以族李崧也【事見二百八十八卷漢乾祐元年】辭而不處【使王峻處權勢之間皆以是心處之必不至有商州之禍矣處昌呂翻】 初契丹主北歸【見二百八十七卷漢高祖天福十二年】横海節度使潘實納棄鎮隨之契丹主以實納為西南路招討使及北漢主立契丹主使實納遺劉承鈞書北漢主使承鈞復書稱本朝淪亡紹襲帝位欲循晉室故事求援北朝【遺唯季翻晉室故事謂晉祖事契丹以求援故事也朝直遥翻下同】契丹主大喜北漢主發兵屯隂地黄澤團柏【屯隂地者欲窺晉隰屯黄澤者欲窺邢趙屯團柏者欲窺鎮定】丁亥以承鈞為招討使與副招討使白從暉都監李存瓌將步騎萬人寇晉州從暉吐谷渾人也郭崇威更名崇曹威更名英【皆避帝名也更工衡翻】 二月丁<br />
<br />
  酉以皇子天雄牙内都指揮使榮為鎮寧節度使選朝士為之僚佐以侍御史王敏為節度判官右補闕崔頌為觀察判官校書郎王朴為掌書記【為王朴見任於世宗張本】頌協之子【崔協相後唐明宗】朴東平人也 戊戌北漢兵五道攻晉州節度使王晏閉城不出劉承鈞以為怯蟻附登城晏伏兵奮擊北漢兵死傷者千餘人承鈞遣副兵馬使安元寶焚晉州西城元寶來降承鈞乃移軍攻隰州【九域志晉州西北至隰州二百五十里】癸卯隰州刺史許遷遣步軍都指揮使孫繼業迎擊北漢兵於長夀村【唐武德二年分隰州石樓置長夀縣貞觀元年省入石樓】執其將程筠等殺之未幾北漢兵攻州城【幾居豈翻】數日不克死傷甚衆乃引去遷鄆州人也 甲辰楚王希萼遣掌書記劉光輔入貢于唐 【考異曰湖湘故事光輔作光翰今從十國紀年】 帝悉出漢宫中寶玉器數十碎之於庭曰凡為帝王安用此物聞漢隱帝日與嬖寵於禁中嬉戲珍玩不離側【嬖卑義翻又必計翻離力智翻】茲事不遠宜以為鑑仍戒左右自今珍華悦目之物無得入宫 丁未契丹主遣其臣舒古濟與朱憲偕來【歐史作䮍朱憲使契丹見上正月】 賀即位 戊申敕前資官各聽自便居外州【漢隱帝乾祐二年冬楊邠奏前資官分居兩京事見二百八十八卷】 陳思讓未至湖南馬希萼已克長沙思讓留屯郢州敇召令還【去年十一月漢朝議發兵救潭州内難作而不果劉郭易姓之際必未暇遣將南畧未知陳思讓為誰朝所遣當考按薛史周太祖登極遣陳思讓帥偏師至安郢以圖進取長沙陷乃班師則帝所遣也】 丁巳遣尚書左丞田敏使契丹北漢主遣通事舍人李?使于契丹【?俗辯字從功從言宋景文手記曰北齊時里俗多作偽字始以巧言為辯至隋有柳?其字又以巩易巧矣】乞兵為援 詔加泰寧節度使慕容彦超中書令遣翰林學士魚崇諒詣兖州諭指崇諒即崇遠也【魚崇諒先因避漢祖諱改名】彦超上表謝三月壬戌朔詔報之曰向以前朝失德少主用讒【少詩照翻】倉猝之間召卿赴闕卿即奔馳應命信宿至京救國難而不顧身聞君召而不俟駕【難乃旦翻】以至天亡漢祚兵散梁郊降將敗軍相繼而至卿即便囘馬首徑反龜隂【兖州在龜山之隂慕容彦超赴大梁還兖州事並見上卷上年】為主為時【為于偽翻】有終有始所謂危亂見忠臣之節疾風知勁草之心若使為臣者皆能如茲則有國者誰不欲用所言朕潜龍河朔之際平難浚郊之時【難乃旦翻浚郊謂大梁之郊大梁有浚水詩云孑孑干旄在浚之郊韓愈從董晉于汴州賦曰非天子之洵美兮吾何為乎浚之都】緣不奉示喻之言亦不得差人至行闕且事主之道何必如斯若或二三於漢朝【朝直遥翻】又安肯忠信於周室以此為懼不亦過乎卿但悉力推心安民體國事朕之節如事故君不惟黎庶獲安抑亦社稷是賴但堅表率未議替移由衷之誠言盡於此【以慕容彦超不自安故以此詔撫諭之】唐以楚王希萼為天策上將軍武安武平靜江寧遠節度使兼中書令楚王以右僕射孫忌【孫忌即孫晟歐史曰晟一名忌又名鳳】客省使姚鳳為冊禮使 丙寅遣前淄州刺史陳思讓將兵戍磁州扼黄澤路【磁州西北當黄澤關路口磁墻之翻】楚王希萼既得志多思舊怨殺戮無度晝夜縱酒荒<br />
<br />
  淫悉以軍府事委馬希崇希崇復多私曲政刑紊亂【復扶又翻紊音問】府庫既盡於亂兵籍民財以賞賚士卒或封其門而取之士卒猶以不均怨望雖朗州舊將佐從希萼來者亦皆不悦有離心劉光輔之入貢於唐也【入貢見上月】唐主待之厚光輔密言湖南民疲主驕可取也唐主乃以營屯都虞候邊鎬為信州刺史將兵屯袁州潛謀進取小門使謝彦顒【顒魚容翻 考異曰湖州故事作謝彦叙周羽冲三楚新録作謝延澤今從十國紀年】本希萼家奴以首面有寵於希萼【首面龍陽之色也】至與妻妾雜坐恃恩專横【横戶孟翻】常肩随希崇或撫其背【記曲禮年長以倍則父事之十年以長則兄事之五年以長則肩隨之注云肩随者與之並行差退若拊背則狎之矣】希崇衘之故事府宴小門使執兵在門外希萼使彦顒預坐或居諸將之上諸將皆恥之希萼以府舍焚蕩命朗州靜江指揮使王逵副使周行逢帥所部兵千餘人治之【帥讀曰率下同治直之翻】執役甚勞又無犒賜【犒苦到翻】士卒皆怨竊言曰囚免死則役作之我輩從大王出萬死取湖南何罪而囚役之且大王終日酣歌豈知我輩之勞苦乎逵行逢聞之相謂曰衆怨深矣不早為計禍及吾曹壬申旦帥其衆各執長柯斧白梃逃歸朗州【柯斧柄也梃徒鼎翻】時希萼醉未醒左右不敢白癸酉始白之希萼遣湖南指揮使唐師翥將千餘人追之不及直抵朗州逵等乘其疲乏伏兵縱擊士卒死傷殆盡師翥脱歸【翥章恕翻】逵等黜留後馬光贊【去年馬希萼以子光贊鎮朗州】更以希萼兄子光惠知州事【更工衡翻】光惠希振之子也【希振馬殷之嫡長子也】尋奉光惠為節度使逵等與何敬真及諸軍指揮使張倣參决軍府事希萼具以狀言於唐唐主遣使以厚賞招諭之逵等納其賞縱其使不答其詔唐亦不敢詰也【使疏吏翻詰去吉翻為王逵等以朗州攻潭州張本】 王彦超奏克徐州殺鞏廷美等【鞏廷美等以無援敗死】 北漢李?至契丹契丹主使伊喇摩哩報之【拽羊列翻刺來逹翻】 丙子敕朝廷與唐本無仇怨緣淮軍鎮各守疆域無得縱兵民擅入唐境商旅往來無得禁止 己卯潞州送涉縣所獲北漢將卒二百六十餘人各賜衫袴巾履遣還【涉漢縣唐屬潞州九域志在州東北一百九十八里】 加吳越王弘俶諸道兵馬都元帥 夏四月壬辰朔濱淮州鎮上言淮南饑民過淮糴穀未敢禁止詔曰彼之生民與此何異宜令州縣津舖無得禁止【舖普故翻】 蜀通奏使高延昭固辭知樞密院丁未以前雲安榷鹽使太原伊審徵為通奏使知樞密院事【雲安漢巴郡之朐䏰縣地周武帝置雲安縣唐屬夔州以其產鹽置雲安監】審徵蜀高祖妹褒國公主之子也少與蜀主相親狎【少詩照翻】及知樞密政之大小悉以咨之審徵亦以經濟為已任而貪侈回邪與王昭遠相表裏蜀政由是浸衰吳越王弘俶徙廢王弘倧居東府【自衣錦軍徙居東府吳越以越州為東府】為築宫室治園圃娯悅之【為于偽翻治直之翻】歲時供饋甚厚契丹主遣使如北漢告以周使田敏來約歲輸錢十萬緡【輸舂遇翻】北漢主使鄭珙以厚賂謝契丹【珙居勇翻】自稱姪皇帝致書於叔天授皇帝請行冊禮 五月己巳遣左金吾將軍姚漢英等使于契丹契丹留之【契丹以北漢交之厚遂留周使】辛未北漢禮部侍郎同平章事鄭珙卒于契丹 【考異曰】<br />
<br />
  【晉陽見聞録鄭珙既逹虜庭虜君恩禮周厚虜俗以酒池肉林為名雖不飲酒如韋曜輩者亦加灌注縱成疾無復信之珙魁岸善飲罹無量之逼宴罷載歸一夕腐脇於穹廬之氈堵間輿尸而復命九國志契丹宴犒漢使必厚具酒肉以示夸大高祖鎮河東嘗命韋曜北使曜羸瘠不能飲酒虜人強之遂卒按韋曜孫皓時人韋昭也不能飲酒王保衡引以為文章而路振云高祖時人誤也】 甲戌義武節度使孫方簡避皇考諱更名方諫【更工衡翻】 定難節度李彛殷遣使奉表于北漢【節度之下當有使字難乃旦翻】 六月辛亥以樞密使同平章事王峻為左僕射兼門下侍郎樞密副使兵部侍郎范質判三司李穀為中書侍郎並同平章事穀仍判三司司徒兼侍中竇貞固司空兼中書侍郎同平章事蘇禹珪並罷守本官癸丑范質參知樞密院事丁巳以宣徽北院使翟光鄴兼樞密副使【翟萇伯翻又徒歷翻】初帝討河中已為人望所屬【帝討河中見二百八十八卷漢乾祐元年屬之欲翻】李穀時為轉運使帝數以微言動之穀俱以人臣盡節為對帝以是賢之即位首用為相時國家新造四方多故王峻夙夜盡心知無不為軍旅之謀多所禆益范質明敏強記謹守法度李穀沈毅有器略在帝前議論辭氣忼慨善譬諭以開主意【數所角翻沈持林翻忼苦廣翻此史言周朝新造輔相者能盡心營職以濟多艱】 武平節度使馬光惠愚懦嗜酒不能服諸將王逵周行逢何敬真謀以辰州刺史廬陵劉言驍勇得蠻夷心【劉言從彭玕奔楚因為楚將】欲迎以為副使言知逵等難制曰不往將攻我乃單騎赴之【九域志辰州東至朗州五百六十六里】既至衆廢光惠送于唐推言權武平留後【為王逵等殺劉言張本】表求旄節於唐唐人未許亦稱藩于周 吳越王弘俶以前内外馬步都統軍使仁俊無罪復其官爵【錢仁俊被幽見二百八十五卷晉齊王開運二年】 契丹遣燕王蘇葉等冊命北漢主為大漢神武皇帝妃為皇后北漢主更名旻【更工衡翻】 秋七月【按五代會要是月周追尊四廟】北漢主遣翰林學士博興衛融等詣契丹謝冊禮【博興即唐青州之博昌縣後唐避獻祖諱改曰博興九域志縣在州西北一百二十里】且請兵【請兵以攻周】 八月壬戌葬漢隱帝于頴陵【頴陵在許州陽翟縣】義武節度使孫方諫入朝壬子徙鎮國節度使以其<br />
<br />
  弟易州刺史行友為義武留後又徙建雄節度使王晏鎮徐州以武寧節度使王彦超代之【王晏與王彦超兩易所鎮】 戊午追立故夫人柴氏為皇后【柴氏先卒去年不死于劉銖之手】 九月北漢主遣招討使李存瓌將兵自團柏入寇契丹欲引兵會之【契丹之下當有主字】與酋長議於九十九泉【魏上地記曰沮陽城東八十里有牧牛山山下有九十九泉即滄河之上源也按魏收魏書天賜三年八月魏主登武要北原觀九十九泉武要縣漢屬定襄郡東部都尉治所宋白曰九十九泉在幽州西北一千餘里】諸部皆不欲南寇契丹主強之癸亥行至新州之火神淀【契丹雖破晉其力亦疲諸部瘡痍未瘳羸耗未復故不欲南寇宋白曰火神淀在新州西強其兩翻淀徒練翻淺水曰淀】燕王蘇頁及偉王之子太寧王烏遜作亂弑契丹主而立蘇頁契丹主德光之子舒嚕逃入南山諸部奉舒嚕以攻蘇頁舒嚕殺之并其族黨立舒嚕為帝改元應歷自火神淀入幽州遣使告于北漢北漢主遣樞密直學士上黨王得中如契丹賀即位復以叔父事之請兵以擊晉州契丹主年少好遊戲不親國事每夜酣飲逹旦乃寐日中方起國人謂之睡王後更名明【少詩照翻好呼到翻更工衡翻】 壬申蜀以吏部尚書御史中丞范仁恕為中書侍郎兼吏部尚書同平章事 楚王希萼既克長沙不賞許可瓊【許可瓊降希萼見上卷漢隱帝乾祐三年】疑可瓊怨望出為恭州刺史【唐武德五年析荔州之隋化縣置南恭州貞觀八年更名蒙州宋朝熙寜五年廢蒙州以立山縣隸昭州宋白曰恭州漢荔浦縣地唐置蒙州以州東面有蒙山山下有泉源流為蒙水山下人皆姓蒙故名】遣馬步都指揮使徐威左右軍馬步使陳敬遷水軍都指揮使魯公綰牙内侍衛指揮使陸孟俊帥部兵立寨于城西北隅以備朗兵【帥讀曰率下同】不存撫役者將卒皆怨怒謀作亂希崇知其謀戊寅希萼宴將吏徐威等不預希崇亦辭疾不至威等使人先驅踶齧馬十餘入府自帥其徒執斧斤白梃聲言縶馬奄至座上縱横擊人顛踣滿地【踶大計翻齧魚結翻縶陟立翻縱子容翻踣蒲北翻】希萼踰垣走威等執囚之【考異曰十國紀年作丁丑按湖湘故事在十九日今從之】執謝彦顒自頂及踵剉<br />
<br />
  之立希崇為武安留後縱兵大掠幽希萼於衡山縣【三國時吴分湘南縣置衡山縣唐屬衡州宋朝淳化四年分屬潭州九域志衡山縣在潭州西南三百二十里】劉言聞希崇立遣兵趣潭州【趣七喻翻】聲言討其簒奪之罪壬午軍于益陽之西希崇懼癸未發兵二千拒之又遣使如朗州求和請為鄰藩掌書記桂林李觀象說言曰【時人謂桂州為桂林說式芮翻】希萼舊將佐猶在長沙此必不欲與公為鄰不若先檄希崇取其首然後圖湖南可兼有也言從之希崇畏言即斷都軍判官楊仲敏掌書記劉光輔牙内指揮使魏師進都押牙黄勍等十餘人首【斷丁管翻勍渠京翻】遣前辰陽縣令李翊齎送朗州【辰陽地名馬氏置縣屬辰州宋白曰辰溪縣本漢辰陵縣後漢曰辰陽以縣在辰水之陽也隋改曰辰溪如此則馬氏用後漢縣名也】至則腐敗言與王逵等皆以為非仲敏等首怒責翊翊惶恐自殺希崇既襲位亦縱酒荒淫為政不公語多矯妄國人不附初馬希萼入長沙【事見上卷上年十二月】彭師暠雖免死猶杖背黜為民希崇以為師暠必怨之使送希萼于衡山實欲師暠殺之師暠曰欲使我為弑君之人乎奉事逾謹【暠古老翻】丙戌至衡山衡山指揮使廖偃匡圖之子也【晉天福四年廖匡圖與蠻戰死】與其季父節度廵官匡凝謀曰吾家世受馬氏恩今希萼長而被黜必不免禍【長知兩翻被彼義翻】盍相與輔之於是帥莊戶及鄉人悉為兵【佃豪家之田而納其租謂之莊戶帥讀曰率】與師暠共立希萼為衡山王以縣為行府斷江為柵【斷丁管翻江即謂湘江也】編竹為戰艦以師暠為武清節度使【武清節度使廖偃等自相署置耳】召募徒衆數日至萬餘人州縣多應之遣判官劉虛已求援于唐徐威等見希崇所為知必無成又畏朗州衡山之逼恐一朝喪敗俱及禍【喪息浪翻】欲殺希崇以自解希崇微覺之大懼密遣客將范守牧奉表請兵于唐唐主命邊鎬自袁州將兵萬人西趣長沙【將即亮翻趣七喻翻】 冬十月辛卯潞州廵檢陳思讓敗北漢兵於虒亭【敗補邁翻虒亭在潞州銅鞮縣九域志潞州襄垣縣有虒亭鎮虒音斯】 唐邊鎬引兵入醴陵【舊唐書地理志曰漢臨湘縣界有醴陵後漢立為縣隋廢唐武德四年分長沙縣置醴陵縣並屬潭州九域志醴陵縣在潭州東一百六十里范成大行程記袁州萍鄉縣至潭州醴陵縣兩日程耳】癸巳楚王希崇遣使犒軍壬寅遣天策府學士拓拔恒奉牋詣鎬請降恒歎曰吾久不死乃為小兒送降狀癸卯希崇帥弟姪迎鎬望塵而拜鎬下馬稱詔勞之【犒苦到翻帥讀曰率下同勞力到翻】甲辰希崇等從鎬入城鎬舍於瀏陽門樓【瀏音留】湖南將吏畢賀鎬皆厚賜之時湖南饑饉鎬大發馬氏倉粟賑之楚人大悦【馬殷據潭朗傳子希聲希範希廣希萼希崇至是而亡唐明宗天成三年楚歸吳敗將苖璘許德勲謂之曰待衆駒爭皁棧而後湖湘可圖今果如其言】 契丹遣彰國節度使蕭禹厥將奚契丹五萬會北漢兵入寇北漢主自將兵二萬自隂地關寇晉州丁未軍于城北三面置寨晝夜攻之遊兵至絳州時王晏已離鎮【離力智翻】王彦超未至巡檢使王萬敢權知晉州與龍捷都指揮使史彦超虎捷指揮使何徽共拒之【薛史本紀廣順元年改侍衛馬步軍額馬軍舊稱護聖改為龍捷步軍舊稱護國改為虎捷】史彦超雲州人也 癸丑唐武昌節度使劉仁贍帥戰艦二百取岳州撫納降附人忘其亡【劉仁贍善將故能為唐堅守夀州】仁贍金之子也【劉金為楊氏將】唐百官共賀湖南平起居郎高遠曰我乘楚亂取之甚易【易以豉翻】觀諸將之才但恐守之難耳遠幽州人也司徒致仕李建勲曰禍其始於此乎【唐之禍敗後果如二臣所料】唐主自即位以來未嘗親祠郊廟禮官以為請唐主曰俟天下一家然後告謝及一舉取楚謂諸國指麾可定魏岑侍宴言臣少遊元城樂其風土【少詩照翻樂音洛】俟陛下定中原乞魏博節度使唐主許之岑趨下拜謝其主驕臣佞如此馬希萼望唐人立己為潭帥而潭人惡希萼【惡烏路翻】共請邊鎬為帥【帥所類翻下同】唐主乃以鎬為武安節度使【為邊鎬為朗兵所逐張本】王峻有故人曰申師厚嘗為兖州牙將失軄饑寒望<br />
<br />
  峻馬拜謁於道會凉州留後折逋嘉施上表請帥於朝廷【折逋羌族也因以為姓】帝以絶域非人所欲募率府供奉官願行者【率府謂東宫十率府也】月餘無人應募峻薦師厚於帝丁巳以師厚為河西節度使 唐邊鎬趣馬希崇帥其族入朝【趣讀曰促帥讀曰率朝直遥翻】馬氏聚族相泣欲重賂鎬奏乞留居長沙鎬微哂曰國家與公家世為仇敵殆六十年【哂矢忍翻唐昭宗光啟三年馬殷從孫儒攻楊行密乾寜三年得湖南自此與江淮為敵國自光啟三年至是年適六十年】然未嘗敢有意窺公之國今公兄弟鬭䦧困窮自歸若復二三【復扶又翻】恐有不測之憂希崇無以應十一月辛酉與宗族及將佐千餘人號慟登舟【䦧馨激翻鬭也狠也戾也詩云兄弟䦧于墻復扶又翻號戶刀翻】送者皆哭響振川谷 帝以北漢契丹之兵猶在晉州甲子以王峻為行營都部署將兵救之詔諸軍皆受峻節度聼以便宜從事得自選擇將吏乙丑峻行帝自至城西錢之【大梁城西也】 楚靜江節度副使知桂州馬希隱武穆王殷之少子也楚王希廣希萼兄弟爭國南漢主以内侍吳懷恩為西北招討使將兵屯境上伺間密謀進取【少詩照翻伺相吏翻間古莧翻】希廣遣指揮使彭彦暉將兵屯龍峒以備之【桂州溪南有白龍洞在平地半山上】希萼自衡山遣使以彦暉為桂州都監在城外内巡檢使判軍府事希隱惡之【惡烏路翻】潜遣人告蒙州刺史許可瓊可瓊方畏南漢之逼即棄蒙州引兵趣桂州【蒙桂相去四百餘里趣七喻翻】與彦暉戰於城中彦暉敗奔衡山可瓊留屯桂州吳懷恩據蒙州進兵侵掠桂管大擾希隱可瓊不知所為但相與飲酒對泣南漢主遺希隱書【遺唯李翻】言武穆王奄有全楚富彊安靖五十餘年止由三十五舅三十舅兄弟尋戈自相魚肉【三十五舅謂希廣三十舅謂希萼漢主龑娶漢王殷女故呼希廣等為舅】舉先人基業北面仇讎【言舉國臣唐也】今聞唐兵已據長沙竊計桂林繼為所取當朝世為與國重以婚姻【朝直遥翻重直用翻】覩茲傾危忍不赴救已發大軍水陸俱進當令相公舅永擁節旄常居方面希隱得書與僚佐議降之支使潘玄珪以為不可丙寅吳懷恩引兵奄至城下希隱可瓊帥其衆夜斬關奔全州【九域志桂州北至全州一百六十三里晉高祖時馬氏改永州之湘源縣為清湘縣置全州本漢洮陽縣地地有洮水在清湘縣北帥讀曰率】桂州遂潰懷恩因以兵略定宜連梧嚴富昭柳龔象等州【唐乾封二年招致生獠以秦故桂林郡地置嚴州富州當是置於賀州富川縣】南漢始盡有嶺南之地 辛未唐邊鎬遣先鋒指揮使李承戩將兵如衡山趣馬希萼入朝【戩子踐翻趣讀曰促】庚辰希萼與將佐士卒萬餘人自潭州東下 王峻留陕州旬日帝以北漢攻晉州急憂其不守議自將由澤州路與峻會兵救之【帝欲自澤州而西王峻自陕度河而北取潞州而會于晉州陕失冉翻將即亮翻】且遣使諭峻十二月戊子朔下詔以三日西征使者至陕峻因使者言于帝曰晉州城堅未易可拔【易以豉翻】劉崇兵鋒方鋭不可力爭所以駐兵待其氣衰耳非臣怯也陛下新即位不宜輕動若車駕出汜水則慕容彦超引兵入汴大事去矣帝聞之自以手提耳曰幾敗吾事【王峻之言出於帝防虞之所不及而犁然有當于心故不覺自提其耳幾居依翻敗補邁翻】庚寅敇罷親征初泰寜節度使兼中書令慕容彦超聞徐州平【謂鞏廷美等死】疑懼愈甚乃招納亡命畜聚薪糧潜以書結北漢吏獲其書以聞又遣人詐為商人求援於唐帝遣通事舍人鄭好謙就申慰諭與之為誓彦超益不自安屢遣都押牙鄭麟詣闕偽輸誠欵實覘機事【覘丑亷翻又丑艶翻】又獻天平節度使高行周書其言皆謗毁朝廷與彦超相結之意帝笑曰此彦超之詐也以書示行周行周上表謝恩【漢初彦超與行周同攻魏因而結隙且兖鄆鄰藩彦超舉兵恐行周擬其後故偽為其書欲以間之帝反以其書示行周以結其心】既而彦超反跡益露丙申遣閤門使張凝將兵赴鄆州廵檢以備之【職官分紀閻門使副掌供奉乘輿朝會游幸大宴及贊引親王宰相百僚蕃客朝見辭糾彈失儀五代以來多以處武臣出將使命及總戎旅】 庚子王峻至絳州乙巳引兵趣晉州【趣七喻翻】晉州南有蒙阬最為險要峻憂北漢兵據之是日聞前鋒已度蒙阬喜曰吾事濟矣 慕容彦超奏請入朝帝知其詐即許之既而復稱境内多盗未敢離鎮【復扶又翻離力智翻】北漢主攻晉州久不克【是年十月庚子攻晉州至是五十餘日】會大雪<br />
<br />
  民相聚保山寨野無所掠軍乏食契丹思歸聞王峻至蒙阬燒營夜遁峻入晉州諸將請亟追之峻猶豫未决明日乃遣行營馬軍都指揮使仇弘超都排陳使藥元福左廂排陳使陳思讓康延沼將騎兵追之及於霍邑【九域志霍邑在晉州北一百三十五里】縱兵奮擊北漢兵墜崖谷死者甚衆霍邑道隘延沼畏懦不急追由是北漢兵得度藥元福曰劉崇悉發其衆挟胡騎而來志吞晉絳今氣衰力憊狼狽而遁不乘此翦撲必為後患【憊蒲拜翻撲普卜翻】諸將不欲進王峻復遣使止之【復扶又翻王峻自晉州遣使】遂還契丹比至晉陽士馬什喪三四【還從宣翻又如字比必利翻喪息浪翻】蕭禹厥恥無功釘大酋長一人於市旬餘而斬之【釘丁定翻酋慈秋翻長知兩翻】北漢主始息意於進取北漢土瘠民貧内供軍國外奉契丹賦繁役重民不聊生逃入周境者甚衆 唐主以鎮南節度使兼中書令宋齊丘為太傅以馬希萼為江南西道觀察使鎮洪州仍賜爵楚王以馬希崇為永泰節度使鎮舒州【唐盖置永泰軍於舒州】湖南將吏位高者拜刺史將軍卿監卑者以次拜官唐主嘉廖偃彭師暠之忠以偃為左殿直軍使萊州刺史【萊州屬周境廖偃遥領耳】師暠為殿直都虞候賜予甚厚【予讀曰與】湖南刺史皆入朝于唐永州刺史王贇獨後至唐主毒殺之 南漢主遣内侍省丞潘崇徹【唐内侍省有監有少監未嘗有丞此南漢創置也】將軍謝貫將兵攻郴州唐邊鎬發兵救之崇徹敗唐兵於義章【郴丑林翻宋白曰郴州漢郴縣隋置郴州敗補邁翻隋末蕭銑分郴置義章縣唐屬郴州九域志在州南八十五里宋朝避太宗潜藩舊名改曰宜章宋白曰縣北臨章水】遂取郴州邊鎬請除全道二州刺史以備南漢丙辰唐主以廖偃為道州刺史以黑雲指揮使張巒知全州【全道二州與南漢賀昭桂三州接界】 是歲唐主以安化節度使鄱陽王王延政為山南西道節度使【興元山南西道屬蜀唐使王延政遥領耳】更賜爵光山王【更工衡翻王延政之先本光山人故以爵之】初蒙城鎮將咸師朗將部兵降唐【見二百八十八卷漢乾祐二年將即亮翻】唐主以其兵為奉節都從邊鎬平湖南唐悉收湖南金帛珍玩倉粟乃至舟艦亭館花果之美者皆徙於金陵遣都官郎中楊繼勲等收湖南租賦以贍戍兵繼勲等務為苛刻湖南人失望行營糧料使王紹顔减士卒糧賜奉節指揮使孫朗曹進怒曰昔吾從咸公降唐唐待我豈如今日湖南將士之厚哉今有功不增禄賜又减之不如殺紹顔及鎬據湖南歸中原富貴可圖也二年春正月庚申夜孫朗曹進帥其徒作亂【帥讀曰率】束藁潜燒府門火不然邊鎬覺之出兵格鬬且命鳴鼓角朗進等以為將曉斬關奔朗州王逵問朗曰吾昔從武穆王與淮南戰屢捷【馬殷諡武穆王】淮南兵易與耳【易以䜴翻】今欲以朗州之衆復取湖南可乎朗曰朗在金陵數年備見其政事朝無賢臣軍無良將忠佞無别賞罰不當【朝直遥翻將即亮翻别彼列翻當丁浪翻】如此得國存幸矣何暇兼人朗請為公前驅取湖南如拾芥耳逵悦厚遇之【為于偽翻王逵等本有圖湖南之志於此遂决】 壬戌發開封府民夫五萬修大梁城旬日而罷 慕容彦超發鄉兵入城引泗水注壕中為戰守之備又多以旗幟授諸鎮將令募羣盜剽掠鄰境【幟昌志翻剽匹妙翻】所在奏其反狀甲子赦沂密二州不復隸泰寧軍【先收其巡屬以弱慕容彦超復扶又翻】以侍衛步軍都指揮使昭武節度使曹英為都部署討彦超【昭武軍利州屬蜀曹英遥領耳】齊州防禦使史延超為副部署皇城使河内向訓為都監【向姓也本自有殷宋文公支子向文旰旰孫戌以王父字為氏余按春秋左氏傳向戍宋桓公之後向式亮翻監古銜翻】陳州防禦使藥元福為行營馬部都虞候帝以元福宿將【藥元福歷事唐漢晉為將有功將即亮翻】命英訓無得以軍禮見之二人皆父事之唐主發兵五千軍于下邳以援彦超聞周兵將至退屯沭陽【下邳縣屬徐州東南至沭陽縣百里劉昫曰沭陽漢廩丘縣後魏改曰沭陽唐屬海州九域志在海州西南一百八十里杜佑曰海州沭陽縣漢原丘縣地梁置潼陽郡沭食聿翻】徐州巡檢使張令彬擊之大破唐兵殺溺死者千餘人獲其將燕敬權【燕於賢翻】初彦超以周室新造謂其易揺【易以䜴翻】故北召北漢及契丹南誘唐人使侵邊鄙冀朝廷奔命不暇然後乘間而動【誘音酉間古莧翻】及北漢契丹自晉州北走唐兵敗於沭陽彦超之勢遂沮【沮在呂翻】永興節度使李洪信自以漢室近親心不自安【洪信莫李太后之群從也】城中兵不滿千人王峻在陜以救晉州為名發其徒數百及北漢兵遁去遣禁兵千餘人戍長安洪信懼遂入朝 壬申王峻自晉州還入見【逐從宣翻又如字見賢遍翻】 曹英等至兖州設長圍慕容彦超屢出戰藥元福皆擊敗之【敗補邁翻】彦超不敢出十餘日長圍合遂進攻初彦超將反判官崔周度諫曰魯詩書之國自伯禽以來不能霸諸侯然以禮義守之可以長世公於國家非有私憾胡為自疑况主上開諭勤至苟撤備歸誠則坐享太山之安矣獨不見杜中令安襄陽李河中竟何所成乎【杜中令謂杜重威安襄陽謂安從進李河中謂李守貞皆以反而敗死事並見前紀】彦超怒及官軍圍城彦超括士民之財以贍軍坐匿財死者甚衆前陜州司馬閻弘魯寶之子也【閻寶背梁歸唐歷節鎮】畏彦超之暴傾家為獻彦超猶以為有所匿命周度索其家【索山客翻】周度謂弘魯曰君之死生繫財之豐約宜無所愛弘魯泣拜其妻妾曰悉出所有以救吾死皆曰竭矣周度以白彦超彦超不信收弘魯夫妻繫獄有乳母於泥中掊得金纒臂獻之冀以贖其主【掊蒲溝翻以手爬土也】彦超曰所匿必猶多榜掠弘魯夫妻肉潰而死以周度為阿庇斬於市 北漢遣兵寇府州防禦使折德扆敗之【敗補邁翻】殺二千餘人二月庚子德扆奏攻拔北漢岢嵐軍以兵戍之【舊唐書地理志曰嵐州嵐谷縣舊岢嵐軍也在嵐州宜芳縣北界長安二年分宜芳於岢嵐舊軍置嵐谷縣神龍二年廢縣置軍開元十二年復置縣此盖後唐復置軍也九域志岢嵐軍治嵐谷縣南至嵐州九十里岢枯我翻】 甲辰帝釋燕敬權等使歸唐謂唐主曰叛臣天下所共疾也不意唐主助之得無非計乎唐主大慙先所得中國人皆禮而歸之唐之言事者猶獻取中原之策中書舍人韓熙載曰郭氏有國雖淺為治已固【治直吏翻】我兵輕動必有害無益唐自烈祖以來【唐主昇廟號烈祖】常遣使泛海與契丹相結欲與之共制中國更相饋遺【更工衡翻邊唯季翻】約為兄弟然契丹利其貨徒以虚語往來實不為唐用也唐主好文學【好呼到翻】故熙載與馮延己延魯江文蔚潘佑徐鉉之徒皆至美官佑幽州人也【蔚紆勿翻】當時唐之文雅於諸國為盛然未嘗設科舉多因上書言事拜官至是始命翰林學士江文蔚知貢舉進士廬陵王克貞等三人及第【廬陵漢縣唐帝吉州蔚紆勿翻】唐主問文蔚卿取士何如前朝對曰前朝公舉私謁相半臣專任至公耳唐主悦中書舍人張緯前朝登第聞而衘之時執政皆不由科第相與沮毁竟罷貢舉【南唐罷貢舉時中國未嘗罷貢舉也】 三月戊辰以内客省使恩州團練使晉陽鄭仁誨為樞密副使【按是時中國無恩州此即南漢之恩州也鄭仁誨遥領團練使耳宋慶歷八年平王則改貝州為恩州始以嶺南之恩州為南恩州以别之】 甲戌改威勝軍曰武勝軍【舊以鄧州為威勝軍今避上名而改之】 唐主以太弟太保昭義節度使馮延己為左僕射前鎮海節度使徐景運為中書侍郎及右僕射孫晟皆同平章事既宣制戶部尚書常夢錫衆中大言曰白麻甚佳但不及江文蔚疏耳【江文蔚疏見二百八十七卷漢天福十二年】晟素輕延己謂人曰金盃玉盌乃貯狗矢乎【盌烏管翻貯丁呂翻】延己言於唐主曰陛下躬親庶務故宰相不得盡其才此治道所以未成也【治直吏翻】唐主乃悉以政事委之奏可而已既而延己不能勤事文書皆仰成胥史【仰牛向翻】軍旅則委之邊將頃之事益不治唐主乃復自覧之【復扶又翻】大理卿蕭儼惡延己為人數上疏攻之會儼坐失入人死罪【惡烏路翻數所角翻上時掌翻誤入人死罪謂之失入】鍾謨李德明輩必欲殺之延己曰儼誤殺一婦人諸君以為當死儼九卿也可誤殺乎獨上言儼素有直聲今所坐已會赦宜從寛宥儼由是得免人亦以此多之景運尋罷為太子少傅【按唐既置太弟官屬不應復有太子少傅當考】 夏四月丙戌朔日有食之帝以曹英等攻兖州久未克乙卯下詔親征以李穀權東京留守兼判開封府鄭仁誨權大内都巡檢又以侍衛馬軍都指揮使郭崇充在京都巡檢 唐主既克湖南遣其將李建期屯益陽以圖朗州以知全州張巒兼桂州招討使以圖桂州久之未有功唐主謂馮延己孫晟曰楚人求息肩於我【言湖南之人苦其主之虐政暴歛而求息肩於唐】我未有撫其瘡痍而虐用其力非所以副來蘇之望【書曰后來其蘇言楚人望唐之休息而唐又興兵役以疲之非所以副其望使唐主言而能行不揺於衆口烏有它日之敗乎】吾欲罷桂林之役歛益陽之戍以旌節授劉言何如晟以為宜然【宜然猶言宜如此也】延己曰吾出偏將舉湖南遠近震驚一旦三分喪二【得潭而失朗桂故謂之三分喪二喪息浪翻】人將輕我請委邊將察其形勢唐主乃遣統軍使侯訓將兵五千自吉州路趣全州【趣七喻翻】與張巒合兵攻桂州南漢伏兵於山谷巒等始至城下罷乏【罷讀曰疲】伏兵四起城中出兵夾擊之唐兵大敗訓死巒收散卒數百奔歸全州 五月庚申帝發大梁戊辰至兖州己巳帝使人招諭慕容彦超城上人語不遜庚午命諸軍進攻先是術者紿彦超云鎮星行至角亢角亢兖州之分【先悉薦翻鎮星土星也亢苦郎翻分扶問翻】其下有福彦超乃立祠而禱之令民間皆立黄幡【土色黄彦超令立幡以從其色人心悦則天意得人有離心厭勝何益】彦超性貪吝官軍攻城急猶瘞藏珍寶【瘞於計翻】由是人無鬬志將卒相繼有出降者乙亥官軍克城彦超方禱鎮星祠帥衆力戰【帥讀曰率】不勝乃焚鎮星祠與妻赴井死子繼勲出走追獲殺之官軍大掠城中死者近萬人【近其靳翻】初彦超將反募羣盜置帳下至者二千餘人皆山林獷悍【獷古猛翻悍侯旰翻又下罕翻】竟不為用帝欲悉誅兖州將吏翰林學士竇儀見馮道范質與之共白帝曰彼皆脅從耳乃赦之丁丑以端明殿學士顔衎權知兖州事【衎苦旱翻又苦旰翻】赦兖州管内彦超黨逃匿者期一月聽自首【首式又翻】前已伏誅者赦其親戚癸未降泰寧軍為防禦州【以慕容彦超據兖州拒命降節鎮為防禦州】 唐司徒致仕李建勲卒且死戒其家人曰時事如此吾得良死幸矣勿封土立碑聽人耕種於其上免為他日開發之標及江南之亡也【謂宋平金陵時】諸貴人高大之冢無不發者惟建勲冢莫知其處【李建勲知國事之日非而骸骨得保其藏可不謂智乎】六月乙酉朔帝如曲阜謁孔子祠【昔少皥氏自窮桑而徙曲阜魯侯伯禽所宅少皥氏之墟也應劭曰曲阜在魯城中委曲長七八里劉昫曰曲阜有闕里孔子所居後人立孔子祠自唐以來兖州治瑕丘而曲阜為屬縣九域志在州東四十里宋大中祥符五年改曲阜為仙源縣】既奠將拜左右曰孔子陪臣也不當以天子拜之帝曰孔子百世帝王之師敢不敬乎遂拜之又拜孔子墓命葺孔子祠禁孔林樵採【孔子廟在曲阜城西南隅闕里孔子墓在曲阜城北泗水上去城一里葬地蓋一頃墳南北十步東西十三步高一丈二尺前有瓴甋為祠壇方六尺與地平塋中異木以百數皆諸弟子自四方致之植於塋中魯人莫之識也】訪孔子顔淵之後以為曲阜令及主簿丙戌帝發兖州 乙未吳越順德太夫人吳氏卒丁酉蜀大水入成都【秦時蜀守季氷穿二江成都中皆可行舟郡縣志曰李氷鑿離澤又開二渠由永康過新繁入成都謂之外江又一渠由永康過郫入成都謂之内江高駢未築羅城内外江皆從城西入自駢築城遂從西北作糜棗堰外江遶城北而東注於合江内江循城南而與外水俱注江江自西來其地勢高所以有水患】漂沒千餘家溺死五千餘人壞太廟四室【壞音怪】戊戌蜀大赦賑水災之家 己亥帝至大梁【自兖州還至大梁】 朔方節度使兼中書令陳留王馮暉卒其子牙内都虞候繼業殺其兄繼勲自知軍府事 太子賓客李濤之弟澣在契丹為勤政殿學士與幽州節度使蕭海真善海真契丹主托雲之妻弟也澣說海真内附海真欣然許之澣因定州諜者田重霸齎絹表以聞【說式芮翻諜達協翻重直龍翻】且與濤書言契丹主童騃【騃五駭翻癡也】專事宴遊無遠志非前人之比【前人謂阿保機德光等】朝廷若能用兵必克不然與和必得二者皆利於速度其情勢他日終不能力助河東者也【度徒洛翻河東謂壮漢】壬寅重霸至大梁會中國多事不果從【北不得燕雲西不得河鄯靈夏宋人以為千古之恨觀温公書此事則元祐初棄米脂等四寨知中國之力不足也】 辛亥以馮繼業為朔方留後 樞密使王峻性輕躁多計數好權利喜人附已【躁則到翻好呼到翻喜許記翻】自以天下為己任每言事帝從之則喜或時未允輒愠懟【愠於運翻懟直類翻】往往發不遜語帝以其故舊且有佐命功【帝自鄴都入汴以至即位王峻之功為多】又素知其為人每優容之峻年長於帝帝即位猶以兄呼之或稱其字峻以是益驕副使鄭仁誨皇城使向訓恩州團練使李重進皆帝在藩鎮時腹心將佐也【重直龍翻將即亮翻】帝即位稍稍進用峻心嫉之累表稱疾求解機務以詗帝意【詗古永翻又翾正翻】帝屢遣左右敦諭峻對使者辭氣亢厲【亢苦浪翻】又遺諸道節度使書求保證【遺唯季翻】諸道各獻其書帝驚駭久之復遣左右慰勉令視事【復扶又翻】且曰卿儻不來朕且自往猶不至帝知樞密直學士陳觀與峻親善令往諭指觀曰陛下但聲言臨幸其第峻必不敢不來秋七月戊子峻入朝帝慰勞令視事【勞力到翻為貶王峻張本】重進滄州人其母即帝妹福慶長公主也【長知兩翻】 李穀足跌傷右臂【跌徒結翻】在告月餘帝以穀職業繁劇趣令入朝【在告在假也趣讀曰促朝直遥翻下同】辭以未任趨拜【任音壬】癸巳詔免朝參但令視事 蜀工部尚書判武德軍郭延鈞不禮於監押王承丕承丕謀作亂辛丑左奉聖都指揮使安次孫欽【安次縣屬幽州孫欽本燕人而仕於蜀】當以部兵戍邊往辭承丕承丕邀與俱見府公【府公謂郭延鈞也公者人之尊稱一府所尊故謂之府公】欽不知其謀從之承丕至則令左右擊殺延鈞屠其家稱奉詔處置軍府【處昌呂翻】即開府庫賞士卒出繫囚發屯戍將吏畢集欽謂承丕曰今延鈞已伏辜公宜出詔書以示衆承丕曰我能致公富貴勿問詔書欽始知承丕反因紿曰今内外未安我請以部兵為公巡察【為于偽翻】即躍馬而出承丕連呼之不止欽至營曉諭其衆帥以入府攻承丕【帥讀曰率】承丕左右欲拒戰欽叱之皆棄兵走遂執承丕斬之并其親黨傳首成都天平節度使守中書令高行周卒行周有勇而知義功高而不矜策馬臨敵叱咤風生平居與賓僚宴集侃侃和易个以是重之【咤涉駕翻易以䜴翻史言高行周所以能以功名終】 癸卯蜀主遣客省使趙季札如梓州慰撫吏民【以新經王承丕之亂也】漢法犯私鹽麴無問多少抵死鄭州民有以屋税受鹽於官過州城吏以為私鹽執而殺之其妻訟寃癸丑始詔犯鹽麴者以斤兩定刑有差【時勑諸色犯鹽麴所犯一斤已下至一兩杖八十配役五斤已下一斤已上徒三年五斤已上重杖一頓處死】<br />
<br />
  資治通鑑卷二百九十<br />
<br />
<史部,編年類,資治通鑑>  <br>
   </div> 

<script src="/search/ajaxskft.js"> </script>
 <div class="clear"></div>
<br>
<br>
 <!-- a.d-->

 <!--
<div class="info_share">
</div> 
-->
 <!--info_share--></div>   <!-- end info_content-->
  </div> <!-- end l-->

<div class="r">   <!--r-->



<div class="sidebar"  style="margin-bottom:2px;">

 
<div class="sidebar_title">工具类大全</div>
<div class="sidebar_info">
<strong><a href="http://www.guoxuedashi.com/lsditu/" target="_blank">历史地图</a></strong>  
<a href="http://www.880114.com/" target="_blank">英语宝典</a>  
<a href="http://www.guoxuedashi.com/13jing/" target="_blank">十三经检索</a> 
<br><strong><a href="http://www.guoxuedashi.com/gjtsjc/" target="_blank">古今图书集成</a></strong> 
<a href="http://www.guoxuedashi.com/duilian/" target="_blank">对联大全</a> <strong><a href="http://www.guoxuedashi.com/xiangxingzi/" target="_blank">象形文字典</a></strong> 

<br><a href="http://www.guoxuedashi.com/zixing/yanbian/">字形演变</a>  <strong><a href="http://www.guoxuemi.com/hafo/" target="_blank">哈佛燕京中文善本特藏</a></strong>
<br><strong><a href="http://www.guoxuedashi.com/csfz/" target="_blank">丛书&方志检索器</a></strong> <a href="http://www.guoxuedashi.com/yqjyy/" target="_blank">一切经音义</a>  

<br><strong><a href="http://www.guoxuedashi.com/jiapu/" target="_blank">家谱族谱查询</a></strong>  <strong><a href="http://shufa.guoxuedashi.com/sfzitie/" target="_blank">书法字帖欣赏</a></strong> 
<br>

</div>
</div>


<div class="sidebar" style="margin-bottom:0px;">

<font style="font-size:22px;line-height:32px">QQ交流群9:489193090</font>


<div class="sidebar_title">手机APP 扫描或点击</div>
<div class="sidebar_info">
<table>
<tr>
	<td width=160><a href="http://m.guoxuedashi.com/app/" target="_blank"><img src="/img/gxds-sj.png" width="140"  border="0" alt="国学大师手机版"></a></td>
	<td>
<a href="http://www.guoxuedashi.com/download/" target="_blank">app软件下载专区</a><br>
<a href="http://www.guoxuedashi.com/download/gxds.php" target="_blank">《国学大师》下载</a><br>
<a href="http://www.guoxuedashi.com/download/kxzd.php" target="_blank">《汉字宝典》下载</a><br>
<a href="http://www.guoxuedashi.com/download/scqbd.php" target="_blank">《诗词曲宝典》下载</a><br>
<a href="http://www.guoxuedashi.com/SiKuQuanShu/skqs.php" target="_blank">《四库全书》下载</a><br>
</td>
</tr>
</table>

</div>
</div>


<div class="sidebar2">
<center>


</center>
</div>

<div class="sidebar"  style="margin-bottom:2px;">
<div class="sidebar_title">网站使用教程</div>
<div class="sidebar_info">
<a href="http://www.guoxuedashi.com/help/gjsearch.php" target="_blank">如何在国学大师网下载古籍?</a><br>
<a href="http://www.guoxuedashi.com/zidian/bujian/bjjc.php" target="_blank">如何使用部件查字法快速查字?</a><br>
<a href="http://www.guoxuedashi.com/search/sjc.php" target="_blank">如何在指定的书籍中全文检索?</a><br>
<a href="http://www.guoxuedashi.com/search/skjc.php" target="_blank">如何找到一句话在《四库全书》哪一页?</a><br>
</div>
</div>


<div class="sidebar">
<div class="sidebar_title">热门书籍</div>
<div class="sidebar_info">
<a href="/so.php?sokey=%E8%B5%84%E6%B2%BB%E9%80%9A%E9%89%B4&kt=1">资治通鉴</a> <a href="/24shi/"><strong>二十四史</strong></a>&nbsp; <a href="/a2694/">野史</a>&nbsp; <a href="/SiKuQuanShu/"><strong>四库全书</strong></a>&nbsp;<a href="http://www.guoxuedashi.com/SiKuQuanShu/fanti/">繁体</a>
<br><a href="/so.php?sokey=%E7%BA%A2%E6%A5%BC%E6%A2%A6&kt=1">红楼梦</a> <a href="/a/1858x/">三国演义</a> <a href="/a/1038k/">水浒传</a> <a href="/a/1046t/">西游记</a> <a href="/a/1914o/">封神演义</a>
<br>
<a href="http://www.guoxuedashi.com/so.php?sokeygx=%E4%B8%87%E6%9C%89%E6%96%87%E5%BA%93&submit=&kt=1">万有文库</a> <a href="/a/780t/">古文观止</a> <a href="/a/1024l/">文心雕龙</a> <a href="/a/1704n/">全唐诗</a> <a href="/a/1705h/">全宋词</a>
<br><a href="http://www.guoxuedashi.com/so.php?sokeygx=%E7%99%BE%E8%A1%B2%E6%9C%AC%E4%BA%8C%E5%8D%81%E5%9B%9B%E5%8F%B2&submit=&kt=1"><strong>百衲本二十四史</strong></a>  <a href="http://www.guoxuedashi.com/so.php?sokeygx=%E5%8F%A4%E4%BB%8A%E5%9B%BE%E4%B9%A6%E9%9B%86%E6%88%90&submit=&kt=1"><strong>古今图书集成</strong></a>
<br>

<a href="http://www.guoxuedashi.com/so.php?sokeygx=%E4%B8%9B%E4%B9%A6%E9%9B%86%E6%88%90&submit=&kt=1">丛书集成</a> 
<a href="http://www.guoxuedashi.com/so.php?sokeygx=%E5%9B%9B%E9%83%A8%E4%B8%9B%E5%88%8A&submit=&kt=1"><strong>四部丛刊</strong></a>  
<a href="http://www.guoxuedashi.com/so.php?sokeygx=%E8%AF%B4%E6%96%87%E8%A7%A3%E5%AD%97&submit=&kt=1">說文解字</a> <a href="http://www.guoxuedashi.com/so.php?sokeygx=%E5%85%A8%E4%B8%8A%E5%8F%A4&submit=&kt=1">三国六朝文</a>
<br><a href="http://www.guoxuedashi.com/so.php?sokeytm=%E6%97%A5%E6%9C%AC%E5%86%85%E9%98%81%E6%96%87%E5%BA%93&submit=&kt=1"><strong>日本内阁文库</strong></a> <a href="http://www.guoxuedashi.com/so.php?sokeytm=%E5%9B%BD%E5%9B%BE%E6%96%B9%E5%BF%97%E5%90%88%E9%9B%86&ka=100&submit=">国图方志合集</a> <a href="http://www.guoxuedashi.com/so.php?sokeytm=%E5%90%84%E5%9C%B0%E6%96%B9%E5%BF%97&submit=&kt=1"><strong>各地方志</strong></a>

</div>
</div>


<div class="sidebar2">
<center>

</center>
</div>
<div class="sidebar greenbar">
<div class="sidebar_title green">四库全书</div>
<div class="sidebar_info">

《四库全书》是中国古代最大的丛书,编撰于乾隆年间,由纪昀等360多位高官、学者编撰,3800多人抄写,费时十三年编成。丛书分经、史、子、集四部,故名四库。共有3500多种书,7.9万卷,3.6万册,约8亿字,基本上囊括了古代所有图书,故称“全书”。<a href="http://www.guoxuedashi.com/SiKuQuanShu/">详细>>
</a>

</div> 
</div>

</div>  <!--end r-->

</div>
<!-- 内容区END --> 

<!-- 页脚开始 -->
<div class="shh">

</div>

<div class="w1180" style="margin-top:8px;">
<center><script src="http://www.guoxuedashi.com/img/plus.php?id=3"></script></center>
</div>
<div class="w1180 foot">
<a href="/b/thanks.php">特别致谢</a> | <a href="javascript:window.external.AddFavorite(document.location.href,document.title);">收藏本站</a> | <a href="#">欢迎投稿</a> | <a href="http://www.guoxuedashi.com/forum/">意见建议</a> | <a href="http://www.guoxuemi.com/">国学迷</a> | <a href="http://www.shuowen.net/">说文网</a><script language="javascript" type="text/javascript" src="https://js.users.51.la/17753172.js"></script><br />
  Copyright &copy; 国学大师 古典图书集成 All Rights Reserved.<br>
  
  <span style="font-size:14px">免责声明:本站非营利性站点,以方便网友为主,仅供学习研究。<br>内容由热心网友提供和网上收集,不保留版权。若侵犯了您的权益,来信即刪。scp168@qq.com</span>
  <br />
ICP证:<a href="http://www.beian.miit.gov.cn/" target="_blank">鲁ICP备19060063号</a></div>
<!-- 页脚END --> 
<script src="http://www.guoxuedashi.com/img/plus.php?id=22"></script>
<script src="http://www.guoxuedashi.com/img/tongji.js"></script>

</body>
</html>
