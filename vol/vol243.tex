






























































資治通鑑卷二百四十三 宋 司馬光 撰

胡三省 音注

唐紀五十九【起昭陽單閼盡著雍涒灘凡六年}


穆宗睿聖文惠孝皇帝下

長慶三年春正月癸未賜兩軍中尉以下錢二月辛卯賜統軍軍使等綿綵銀器各有差【綿當作錦}
戶部侍郎牛僧孺素為上所厚初韓弘之子右驍衛將軍公武為其父謀以財結中外【為其于偽翻}
及公武卒弘繼薨稺孫紹宗嗣主藏奴與吏訟於御史府【藏徂浪翻}
上憐之盡取弘財簿自閱視凡中外主權【主權謂中外官之有事權者}
多納弘貨獨朱句細字曰某年月日送戶部牛侍郎錢千萬不納【句古侯翻}
上大喜以示左右曰果然吾不繆知人【繆靡幼翻}
三月壬戌以僧孺為中書侍郎同平章事時僧孺與李德裕皆有入相之望德裕出為浙西觀察使八年不遷【至文宗太和三年用裴度薦始徵李德裕於浙西又為李宗閔所排出帥滑}
以為李逢吉排已引僧孺為相由是牛李之怨愈深 【考異曰舊德裕傳曰初李逢吉自襄陽入朝乃密賂纎人構成于方獄六月元稹裴度俱罷逢吉代裴度為相既得權位銳意報怨時德裕與僧孺俱有相望逢吉欲引僧孺懼紳與德裕禁中沮之九月出德裕浙西尋引僧孺同平章事繇是交怨愈深蓋德裕以此疑怨逢吉未必皆出逢吉之意也}
夏四月甲午安南奏陸州獠攻掠州縣【武德元年以寜越郡之安海玉山置玉山州貞觀元年州廢屬欽州高宗上元二年復置陸州東至廉州界三百里}
丙申賜宣徽院供奉官錢紫衣者百二十緡下至承

旨各有差【唐中世以後置宣徽院以宦者主之其大朝賀及聖節上夀則宣徽使宣答徐度却埽編曰宣徽使本唐宦者之官故其所掌皆瑣細之事本朝更用士人品秩亞二府有南北院南院比北院資望尤優然其職猶多因唐之舊賜羣臣新火及諸司使至崇班内侍供奉諸司工匠兵卒名藉及三班以下遷補假故鞠劾春秋及聖節大宴節度迎授恩命上元張燈四時祠祭契丹朝貢内庭學士赴上督其供帳内外進奉名物教坊伶人歲給衣帶郊御殿朝謁聖容賜酺國忌諸司使下别藉分產諸司工匠休假之類今觀穆宗所賜則宣徽院官員數多矣}
初翼城人鄭注眇小目下視而巧譎傾謟善揣人意【翼城縣屬絳州本漢絳縣地隋改翼城縣因縣古翼城為名揣初委翻}
以醫遊四方羇貧甚嘗以藥術干徐州牙將牙將悦之薦於節度使李愬愬餌其藥頗驗遂有寵署為牙推【牙推在節度推官之下}
浸預軍政妄作威福軍府患之監軍王守澄以衆情白愬請去之【去羌呂翻下同}
愬曰注雖如是然奇才也將軍試與之語【時中官多加諸衛將軍謂之内將軍}
苟無可取去之未晩乃使注往謁守澄守澄初有難色不得已見之坐語未久守澄大喜延之中堂促笑語恨相見之晩明日謂愬曰鄭生誠如公言自是又有寵於守澄權勢益張【張知亮翻}
愬署為廵官列於賓席注既用事恐牙將薦己者泄其本末密以他罪譛之於愬愬殺之及守澄入知樞密挈注以西為立居宅贍給之【為于偽翻}
遂薦於上上亦厚遇之自上有疾【去年冬十一月上有疾事見上卷}
守澄專制國事勢傾中外注日夜出入其家與之謀議語必通夕關通賂遺【遺唯季翻}
人莫能窺其迹始則有微賤巧宦之士或因以求進數年之後逹官車馬滿其門矣【為鄭注與李訓誅王守澄及甘露之禍張本}
工部尚書鄭權家多姬妾祿薄不能贍因注通於守澄以求節鎭己酉以權為嶺南節度使 五月壬申以尚書左丞柳公綽為山南東道節度使公綽過鄧縣【唐襄州之鄧城縣漢南陽之鄧縣也治古樊城隋改為安養縣天寶元年改為臨漢縣貞元二十一年移縣古鄧城乃改為鄧城縣九域志在州北二十里}
有二吏一犯一舞文衆謂公綽必殺犯者公綽判曰吏犯法法在姦吏亂法法亡竟誅舞文者 【考異曰柳氏叙訓曰公為襄陽節度使有名馬人争畫為圖圉人潔其蹄尾被蹴致斃命斬于鞠場賓吏請曰圉人備之不至良馬可惜公曰有良馬之貌含駑馬之性必殺之有齊衰者哭且獻狀曰遷三世十二喪于武昌為津吏所遏不得出公覽狀召軍侯擒之破其十二柩皆實以稻米時歲儉鄰境尤甚人以為神明之政按韓愈與公綽書曰殺所乘馬以祭踶死之士乃在鄂岳時事叙訓舊傳皆誤也察齊衰者乃是閉糶非美事今不取}
丙子以晉慈二州為保義軍以觀察使李寰為節度使 六月己丑以吏部侍郎韓愈為京兆尹六軍不敢犯法私相謂曰是尚欲燒佛骨【事見二百四十卷憲宗元和十四年}
何可犯也 秋七月癸亥嶺南奏黄洞蠻寇邕州破左江鎭【邕州宣化縣有左江右江二鎭左江出七源州界至合江鎭與右江水合為一水流入横州號鬱水右江源出峩利州界與雲南大槃水通左江道屬太平永平寨右江道屬横山寨各管羈縻州}
丙寅邕州奏黄洞蠻破欽州千金鎭刺史楊嶼奔石南砦【千金鎭當在欽州西南嶼徐與翻砦與寨同音豺夬翻}
南詔勸利卒國人請立其弟豐祐 【考異曰實錄九月辛酉南詔王立佺進其國信歲末又云南詔請立蒙勸利之弟豐祐云立佺者蓋誤也今從新傳}
豐祐勇敢善用其衆始慕中國不與父連名【南詔父子連名其先細奴邏生邏盛炎邏盛炎生炎閤炎閤死而立其弟盛邏皮盛邏皮生皮邏閤皮邏閤生閤邏鳳閤邏鳳生鳳迦異鳳迦異生異牟尋異牟尋生尋閤勸尋閤勸生勸龍晟勸利皆連名也為南詔彊盛寇邉張本}
八月癸巳邕管奏破黄洞蠻 丙申上自複道幸興慶宫至通化門樓【雍錄開元二十年築夾城通芙蓉園自大明宫夾東羅城複道由通化安興門次經春明門延喜門又可以逹曲江芙蓉園而外人不知也按複道自大明宫至通化門便可入興慶宫若經春明延興延喜門則至芙蓉園矣}
投絹二百匹施山僧【施式䜴翻}
上之濫賜皆此類不可悉紀 癸卯以左僕射裴度為司空山南西道節度使不兼平章事李逢吉惡度【惡烏路翻}
右補闕張又新等附逢吉競流謗毁傷度竟出之又新薦之子也【張薦事德宗屢使吐蕃囘鶻}
九月丙辰加昭義節度使劉悟同平章事 李逢吉為相内結知樞密王守澄勢傾朝野 【考異曰李讓夷敬宗實錄曰逢吉用族子仲言之謀因鄭注與守澄潛結上於東宫且言逢吉實立殿下上深德之又曰張又新李續之皆逢吉藩僚時又新為右補闕續之為度支員外郎劉昫承之為逢吉傳亦言逢吉令仲言賂注求結於守澄仲言辯譎多端守澄見之甚悦自是逢吉有助事無違者其李訓傳則云訓自流所還丁母憂居洛中時逢吉為留守思復為相乃使訓因鄭注結王守澄然則逢吉結守澄乃在文宗時非穆宗時也二傳自相違逢吉結守澄要為不誣然未必因鄭注李讓夷乃李德裕之黨惡逢吉欲重其罪使與李訓鄭注皆有連結之迹故云用訓謀因注以交守澄耳又張又新李續之為逢吉藩僚乃在逢吉再鎭襄陽後於此時未也今不取}
惟翰林學士李紳每承顧問常排抑之擬狀至内庭紳多所臧否【擬狀謂進狀所擬除目也翰林學士院在内庭蓋李逢吉所進擬者穆宗訪其可否於李紳故得言之否音鄙}
逢吉患之而上待遇方厚不能遠也【遠于願翻}
會御史中丞缺逢吉薦紳清直宜居風憲之地上以中丞亦次對官【程大昌曰德宗貞元七年詔每御延英令諸司長官二人奏本司事俄又令常參官必日引見二人訪以政事謂之廵對則是待制之外又别有廵對也蓋正謂待制者諸司長官也名為廵對者未為長官而在常參之數亦得更迭引對者也其曰次對官者即廵對官許亞次待制而俟對者也則次對不得正為待制矣今人作文凡言待制皆以次對名之則恐未審也然稱謂既熟雖唐人亦自不辯開成中勑今後遇入閣日次對官未要隨班出並於東階松木下立待宰臣奏事退令齊至香案前各奏本司公事左右史待次對官奏事訖同出案此所言嘗以諸司之長官待制者名為次對官矣若究其制實誤以待制為次對官也余按唐中世以後宰相對延英既退則待制官廵對官皆得引對摠可謂之次對官所謂次對官者謂次宰相之後而得對也非次待制官而入對也唐人本不誤程泰之自誤耳据宋白所紀貞元七年十一月勑則次對官者以常參官依次對為稱詳己見前註}
不疑而可之會紳與京兆尹御史大夫韓愈争臺參及他職事文移往來辭語不遜【故事京兆新除皆詣臺參逢吉欲激二人使争以愈兼御史大夫免臺參而紳愈果争不遜謂不相遜也}
逢吉奏二人不協冬十月丙戌以愈為兵部侍郎紳為江西觀察使 己丑以中書侍郎同平章事杜元頴同平章事充西川節度使【為杜元頴以刻削致寇張本}
辛卯安南奏黄洞蠻為寇 韓愈李紳入謝上各令自叙其事乃深寤壬辰復以愈為吏部侍郎紳為戶部侍郎【考異曰穆宗實錄曰紳性險果交結權倖自以望輕頗忌朝廷有名之士及居近署封植已類以樹黨援進}


【修之士懼為傷毒疾之常指鈞衡欲逞其私志時宰病之因以人情上論諫官歷獻疏方有江西之命行有日矣因延英對辭又泣請留侍故有是拜人情憂駭此蓋修穆宗實錄者惡紳故毁之如是今從敬宗實錄}
四年春正月辛亥朔上始御含元殿朝會【上即位四年矣是歲元正方御東内正牙人朝會}
初柳泌等既誅【見二百四十一卷元和十五年}
方士稍復因左右以進【復扶又翻}
上餌其金石之藥有處士張臯者上疏以為神慮澹則血氣和嗜欲勝則疾疹作【澹徒覽翻疹丑刃翻}
藥以攻疾無疾不可餌也昔孫思邈有言【孫思邈唐之名醫}
藥勢有所偏助令人藏氣不平【藏徂浪翻藏氣五藏之氣也}
借使有疾用藥猶須重愼庶人尚爾况於天子先帝信方士妄言餌藥致疾此陛下所詳知也豈得復循其覆轍乎【復扶又翻下同}
今朝野之人紛紜竊議但畏忤旨莫敢進言臣生長蓬艾【長知丈翻}
麋鹿與遊無所邀求但粗知忠義欲裨萬一耳上甚善其言使求之不獲 丁卯嶺南奏黄洞蠻寇欽州殺將吏【舊制欽州至京師五千二百五十一里}
庚午上疾復作壬申大漸命太子監國宦官欲請郭太后臨朝稱制太后曰昔武后稱制幾危社稷【事見武后紀幾居依翻}
我家世守忠義非武氏之比也太子雖少【少詩照翻}
但得賢宰相輔之卿輩勿預朝政何患國家不安自古豈有女子為天下主而能致唐虞之理乎取制書手裂之太后兄太常卿釗聞有是議密上牋曰若果徇其請臣請先帥諸子納官爵歸田里【帥讀曰率}
太后泣曰祖考之慶鍾於吾兄是夕上崩于寢殿【年三十}
癸酉以李逢吉攝冢宰丙子敬宗即位于太極東序初穆宗之立神策軍士人賜錢五十千【事見二百四十一卷元和十五年}
宰相議以太厚難繼乃下詔稱宿衛之勤誠宜厚賞屬頻年旱歉【屬之欲翻}
御府空虚邉兵尚未給衣霑恤期於均濟神策軍士人賜絹十匹錢十千畿内諸鎭又減五千仍出内庫綾二百萬匹付度支充邉軍春衣時人善之【李逢吉為相時人之所惡也一事之善則時人善之非是非之公歟度徒洛翻}
自戊寅至庚辰上賜宦官服色及錦綵金銀甚衆

或今日賜綠明日賜緋【史言上眤於近習賜予無度}
初穆宗既留李紳【事見上年}
李逢吉愈忌之紳族子虞頗以文學知名自言不樂仕進【樂音洛}
隱居華陽川【華陽川在虢州華陽山南華戶化翻}
及從父耆為左拾遺【從才用翻下從子同}
虞與耆書求薦誤逹於紳紳以書誚之且以語於衆人【誚才笑翻語牛倨翻}
虞深怨之乃詣逢吉悉以紳平日密論逢吉之語告之逢吉益怒使虞與補闕張又新及從子前河陽掌書記仲言等伺求紳短揚之於士大夫間【伺相吏翻}
且言紳潛察士大夫有羣居議論者輒指為朋黨白之於上由是士大夫多忌之及敬宗即位逢吉與其黨快紳失勢又恐上復用之【復扶又翻}
日夜謀議思所以害紳者楚州刺史蘇遇【楚州漢射陽縣地晉立山陽郡隋為楚州至京師二千五百一里}
謂逢吉之黨曰主工初聽政必開延英有次對官惟此可防【恐紳因次對言事而上復用之}
其黨以為然亟白逢吉曰事迫矣若俟聽政悔不可追逢吉乃令王守澄言於上曰陛下所以為儲貳臣備知之皆逢吉之力也如杜元頴李紳輩皆欲立深王【深王察後改名悰憲宗之子穆宗之弟也}
度支員外郎李續之等繼上章言之上時年十六疑未信會逢吉亦有奏言紳不利於上請加貶謫上猶再三覆問然後從之二月癸未貶紳為端州司馬【端州隋置取界内端溪為名煬帝初置信安郡武德又為端州天寶改為高安郡乾元復為州舊志至京師四千九百三十五里}
逢吉仍帥百官表賀【帥讀曰率}
既退百官復詣中書賀【復扶又翻下同}
逢吉方與張又新語門者弗内良久又新揮汗而出旅揖百官曰端溪之事又新不敢多讓【端州謂之端溪}
衆駭愕辟易憚之【辟音闢易音亦}
右拾遺内供奉吳思獨不賀逢吉怒以思為吐蕃告哀使丙戌貶翰林學士龎嚴為信州刺史蔣防為汀州刺史【唐上元元年割饒州之弋陽衢州之玉山建撫二州各三鄉置信州至京師東南三千八百里開元二十六年開福撫二州山洞置汀州至京師六千一百七十三里}
嚴夀州人與防皆紳所引也給事中于敖素與嚴善封還敕書人為之懼【為于偽翻}
曰于給事為龎蔣直寃犯宰相怒誠所難也及奏下乃言貶之太輕逢吉由是奬之張又新等猶忌紳日上書言貶紳太輕上許為殺之【為于偽翻}
朝臣莫敢言獨翰林侍讀學士韋處厚上疏【太宗選耆儒侍讀以質史籍疑義開元中集賢院置侍讀直學士時翰林有侍讀學士有侍書學士}
指述紳為逢吉之黨所讒人情歎駭紳蒙先朝奬用借使有罪猶宜容假以成三年無改之孝【論語孔子曰三年無改於父之道可謂孝矣}
况無罪乎於是上稍開寤 【考異曰處厚傳曰敬宗即位李逢吉用事素惡李紳乃搆成其罪禍將不測處厚乃上疏云云帝悟其事紳得減死貶端州司馬今從實錄處厚上疏在紳貶端州後}
會閱禁中文書有穆宗所封文書一篋之得裴度杜元頴李紳疏請立上為太子上乃嗟歎悉焚人所上譖紳書【所上時掌翻}
雖未即召還後有言者不復聽矣 己亥尊郭太后為太皇太后 乙巳尊上母王妃為皇太后太后越州人也丁未上幸中和殿擊毬自是數遊宴擊毬奏樂【數所角翻}


賞賜宦官樂人不可悉紀 三月壬子赦天下諸道常貢之外毋得進奉 甲寅上始對宰相於延英殿 初牛元翼在襄陽【牛元翼出深州鎭襄陽見上卷二年}
數賂王庭湊以請其家【數所角翻}
庭湊不與聞元翼薨甲子盡殺之 上視朝每晏戊辰日絶高尚未坐百官班於紫宸門外老病者幾至僵踣【僵居良翻踣蒲北翻}
諫議大夫李渤白宰相曰昨日疏論坐晩【論上坐朝之晩也}
今晨愈甚請出閤待罪於金吾仗【金吾左右仗在宣政殿前}
既坐班退左拾遺劉栖楚獨留進言曰憲宗及先帝皆長君四方猶多叛亂陛下富於春秋嗣位之初當宵衣求理而嗜寢樂色【宵衣未明而衣也理治也樂魚教翻}
日晏方起梓宫在殯鼓吹日喧【吹尺偽翻}
令聞未彰【聞音問下響聞同}
惡聲遐布臣恐福祚之不長請碎首玉階以謝諫職之曠遂以額叩龍墀見血不已響聞閤外 【考異曰實録曰莊周云為善無近名為惡無近刑意者既能為近名之善即必忍為近刑之惡栖楚本王承宗小吏果敢有聞逢吉擢而用之蓋取其鷹犬之効耳夫諫諍之道是豈能知之乎即如比干剖心當文王與紂之事也朱雲拆檻恐漢氏之為新室也時危事迫不得不然故忠臣有死諫之義至如上年少嗜寢坐朝稍晩蓋宰臣密勿諫臣封事而可止者也豈在暴揚面數激訐於羽儀之前致使上疑死諫為不難謂細事皆當碎首從此遂不覽章疏卒有克明之難實栖楚兆之况諫辭皆羣黨所作而使栖楚道之哉賣前直而資後詐殊可歎駭按李讓夷此論豈非惡栖楚而彊毁之邪今所不取}
李逢吉宣曰劉栖楚休叩頭俟進止【程大昌曰奏劄言取進止猶言此劄之或留或却合禀承可否也唐中葉遂以處分為進止而不曉文義者習而不察槩謂有旨為進止如玉堂宣底所載凡宣旨皆云有進止者相承之誤也}
栖楚捧首而起更論宦官事上連揮令出栖楚曰不用臣言請繼以死牛僧孺宣曰所奏知門外俟進止【此宰相宣上旨也言所奏知者謂所奏之事上已知之也}
栖楚乃出待罪於金吾仗於是宰相贊成其言上命中使就仗并李渤宣慰令歸尋擢栖楚為起居舍人仍賜緋栖楚辭疾不拜歸東都 庚午賜内教坊錢萬緡以備行幸【武德後置内教坊於禁中武后如意元年改曰雲韶府以中官為使開元二年又置内教坊于蓬萊宫側京都有左右教坊}
夏四月甲午淮南節度使王播罷鹽鐵轉運使【王播兼鹽鐵轉運見上卷二年}
乙未以布衣姜洽為補闕試大理評事陸洿布衣李虞劉堅為拾遺【六典注云隋置大理評事通典云唐置評事十人掌出使推覆後增為十二人新志評事八人從八品下陸洿特試官耳洿後五翻}
時李逢吉用事所親厚者張又新李仲言李續之李虞劉栖楚姜洽及拾遺張權輿程昔範又有從而附麗之者時人惡逢吉者目之為八關十六子【附依也麗著也自張又新至程昔範八人而附麗者又八人皆任要劇故目之為八關十六子關者要也 考異曰按宰相之門何嘗無特所親愛之士數蒙引接詢訪得失否臧人物其間忠邪渾殽固亦多矣其踈遠不得志者則從而怨疾之巧立品目以相譏誚此乃古今常態非獨逢吉之門有八關十六子也舊逢吉傳以為有求於逢吉者必先經此八人納賂無不如意亦恐未必然但逢吉之門險詖者為多耳此皆出于李讓夷敬宗實録按栖楚為吏敢與王承宗争事此乃正直之士何得為佞邪之黨哉蓋讓夷德裕之黨而栖楚為逢吉所善故深詆之耳}
卜者蘇玄明與染坊供人張韶善【染坊供人供役於染坊者也陸德明曰染如艷翻善而險翻}
玄明謂韶曰我為子卜【為于偽翻}
當升殿坐與我共食今主上晝夜毬獵多不在宫中大事可圖也韶以為然乃與玄明謀結染工無賴者百餘人丙申匿兵於紫草車載以入銀臺門【本草曰紫草出碭山山谷及楚地今處處有之人家園圃或種蒔其根所以染紫也爾雅謂之藐廣雅謂之茈䓞苗似蘭香節青二月有花紫白色秋實白三月採根隂乾以下文清思殿徵之所入者左銀臺門也在大明宫東面又北則玄化門}
伺夜作亂【伺相吏翻}
未逹所詣有疑其重載而詰之者【載才代翻}
韶急即殺詰者與其徒易服揮兵大呼趣禁庭上時在清思殿擊毬【自左銀臺門西入經太和殿至清思殿清思殿之南則宣徽殿北則珠鏡殿}
諸宦者見之驚駭急入閉門走白上盜尋斬關而入先是右神策中尉梁守謙有寵於上兩軍角伎藝【先悉薦翻伎渠綺翻}
上常佑右軍至是上狼狽欲幸右軍左右曰右軍遠恐遇盗不若幸左軍近【唐左神策軍右龍武軍左羽林軍皆列屯東内苑直左銀臺門東北角}
上從之左神策中尉河中馬存亮聞上至走出迎捧上足涕泣自負上入軍中遣大將康藝全將騎卒入宫討賊【將即亮翻}
上憂二太后隔絶【二太后太皇太后郭氏上母太后王氏也}
存亮復以五百騎迎二太后至軍【復扶又翻}
張韶升清思殿坐御榻與蘇玄明同食曰果如子言玄明驚曰事止此邪韶懼而走會康藝全與右軍兵馬使尚國忠引兵至合擊之殺韶玄明及其黨死者狼藉逮夜始定餘黨猶散匿禁苑中明日悉擒獲之時宫門皆閉上宿於左軍中外不知上所在人情恇駭【恇去王翻}
丁酉上還宫宰相帥百官詣延英門賀來者不過數十人【帥讀曰率}
盗所歷諸門監門宦者三十五人法當死己亥詔並杖之仍不改職任【兩中尉及諸宦者右之也}
壬寅厚賞兩軍立功將士 五月乙卯以吏部侍郎李程戶部侍郎判度支竇易直並同平章事上問相於李逢吉逢吉列上當時大臣有資望者程為之首【列上時掌翻}
故用之上好治宫室【好呼到翻治直之翻}
欲營别殿制度甚廣李程諫請以所具木石囘奉山陵上即從之 六月己卯朔以左神策大將軍康藝全為酈坊節度使【賞討張韶蘇玄明之功也}
上聞王庭湊屠牛元翼家歎宰輔非才使凶賊縱暴翰林學士韋處厚因上疏言裴度勲高中夏聲播外夷若置之巖廊委其參决河北山東必稟朝筭【夏戶雅翻朝直遥翻}
管仲曰人離而聽之則愚合而聽之則聖理亂之本非有他術順人則理違人則亂伏承陛下當食歎息恨無蕭曹今有裴度尚不能留此馮唐所以謂漢文得廉頗李牧不能用也【事見十五卷漢文帝十四年}
夫御宰相當委之信之親之禮之於事不効於國無勞則置之散寮【散蘇但翻冗散也}
黜之遠郡如此則在位者不敢不厲將進者不敢苟求臣與逢吉素無私嫌嘗為裴度無辜貶官【憲宗時韋處厚為考功郎韋貫之罷相處厚坐與之善出刺開州}
今之所陳上答聖明下逹羣議耳上見度奏狀無平章事以問處厚處厚具言李逢吉排沮之狀上曰何至是邪李程亦勸上加禮於度丙申加度同平章事張韶之亂馬存亮功為多存亮不自矜委權求出秋七月以存亮為淮南監軍使 夏綏節度使李祐入為左金吾大將軍【夏戶雅翻}
壬申進馬百五十匹上却之甲戌侍御史温造於閤内奏彈祐違敕進奉【因入閤而奏彈之也違敕者謂違三月壬子敕也}
請論如法詔釋之祐謂人曰吾夜半入蔡州城取吳元濟【事見二百四十卷憲宗元和十二年}
未嘗心動今日膽落於温御史矣 八月丁卯朔安南奏黄蠻入寇【黄蠻即黄洞蠻}
龍州刺史尉遲鋭上言牛心山素稱神異【尉紆勿翻牛心山在龍州江油縣西一里道教靈驗記李虎葬龍州之牛心山又牛心山靈異記梁武陵王紀理益州使李龍遷築城於牛心山龍遷既没即葬於山側鄉里為立祠武德中改為觀武氏革命鑿斷山脉明皇幸蜀有老人蘇坦奏曰牛心山國之祖墓今日蒙塵之禍乃則天掘鑿所致明皇即令修塡如舊明年誅祿山復宫闕以二記考之則李虎與龍遷即一人也然虎仕西魏未嘗仕梁}
有掘斷處請加補塞【塞悉則翻}
從之役數萬人於絶險之地東川為之疲弊【為于偽翻}
九月丁未波斯李蘇沙獻沉香亭子材左拾遺李漢上言此何異瑶臺瓊室上雖怒亦優容之【杜佑曰林邑出沉香土人破斷其木積以歲年朽爛而心節獨在置水中則沉故名沉香諸蕃志沉香所出非一形多異而名亦不一有如犀角者謂之犀角沉如燕口者謂之燕口沉如附子者謂之附子沉如梭者謂之梭沉紋堅而理緻者謂之横陽沉今其材可為亭子則條段又非諸沉比矣}
漢道明之六世孫也【道明淮陽王道玄之弟}
冬十月戊戌翰林學士韋處厚諫上宴遊曰先帝以酒色致疾損夀臣是時不死諫者以陛下年已十五故也今皇子纔一歲臣安敢畏死而不諫乎上感其言賜錦綵百匹銀器四 十一月戊午安南奏黄蠻與環王合兵攻䧟陸州殺刺史葛維 庚申葬睿聖文惠孝皇帝于光陵【光陵在同州奉先縣北十五里堯山}
廟號穆宗 王播以錢十萬緡賂王守澄求復領利權【是年四月王播罷鹽鐵轉運使}
十二月癸未諫議大夫獨孤朗張仲方起居郎柳公權起居舍人宋申錫拾遺李景讓薛廷老請開延英論其奸邪上問前廷争者不在中邪【争讀曰諍}
即日除劉栖楚諫議大夫景讓憕之曾孫【李憕天寶末守東都死於安禄山之難憕直陵翻}
廷老河中人也 十二月庚寅加天平節度使烏重胤同平章事 乙未徐泗觀察使王智興以上生日【按唐會要上以元和四年六月九日生今王智興於十二月請置戒壇預請之也}
請於泗州置戒壇度僧尼以資福【釋氏之法凡初度僧尼皆請戒壇受戒其未受戒者謂之沙彌無知及避征役者争趨之泗州有大聖塔人敬事之故王智興請於此置戒壇}
許之自元和以來敕禁此弊智興欲聚貨首請置之於是四方輻湊江淮尤甚智興家貲由此累鉅萬浙西觀察使李德裕上言若不鈐制【鈐其廉翻}
至降誕日方停計兩浙福建當失六十萬丁奏至即日罷之 是歲囘鶻崇德可汗卒弟曷薩特勒立

敬宗睿武昭愍孝皇帝【諱湛穆宗長子也諡法夙夜警惕曰愍}


寶歷元年春正月辛亥上祀南郊還御丹鳳樓赦天下改元先是鄠令崔聞外喧囂問之曰五坊人百姓【先悉薦翻鄠音戶囂虚驕翻烏口翻}
怒命擒以入曳之於庭時已昏黑良久詰之乃中使也上怒收繫御史臺是日與諸囚立金雞下【唐制凡國有赦宥刑部先集囚徒於闕下衛尉建金雞置鼓宫城門之右囚徒至則擊之宣制訖乃釋其囚}
忽有品官數十人【玄宗天寶十三年内侍省置高品一千六百九十六人品官白身二千九百三十二人皆羣閹也}
執梃亂捶破面折齒【梃徒鼎翻白木棓也捶止橤翻折面設翻}
絶氣乃去數刻而蘇復有繼來求擊之者【復扶又翻}
臺吏以席蔽之僅免上命復繫於臺獄【臺獄御史臺獄也}
而釋諸囚 中書侍郎同平章事牛僧孺以上荒淫嬖幸用事【嬖卑義翻又轉計翻}
又畏罪不敢言但累表求出乙卯升鄂岳為武昌軍以僧孺同平章事充武昌節度使 【考異曰皇甫松續牛羊日歷曰太牢既交惡黨潛豫姦謀太牢乃元和中青衫外郎耳穆宗世因承和薦不三二年位兼將相憲宗仙駕至灞上以從官召知制誥當時宰臣未盡兼職而獨綜集賢史館兩司出鎭未盡佩相印而太牢同平章事出夏口夏口去節十五年由太牢而加節焉太牢早孤母周氏冶蕩無檢鄉里云云兄弟羞赧乃令改醮既與前夫義絶矣及貴請以出母追贈禮云庶氏之母死何為哭於孔氏之廟乎又曰不為伋也妻者是不為白也母而李清心妻配牛幼簡是夏侯銘所謂魂而有知前夫不納於幽壤殁而可作後夫必訴於玄穹使其母為失行無適從之鬼上罔聖朝下欺先父得曰忠孝智識者乎作周秦行紀呼德宗為沈婆兒謂睿眞皇太后為沈婆此乃無君甚矣此朋黨之論今不取}
中旨復以王播兼鹽鐵轉運使【復扶又翻}
諫官屢争之上皆不納牛僧孺過襄陽山南東道節度使柳公綽服櫜鞬候於館舍【櫜姑勞翻鞬居言翻}
將佐諫曰襄陽地高於夏口【鄂州謂之夏口}
此禮太過公綽曰奇章公甫離台席【牛弘相隋封奇章公僧孺其裔孫也故唐人以稱之宰相之位取象三台故曰台席離力智翻}
方鎭重宰相所以尊朝廷也竟行之 上遊幸無常眤比羣小【眤尼質翻狎也近也比毗至翻黨也}
視朝月不再三【朝直遥翻下同}
大臣罕得進見【見賢遍翻}
二月壬午浙西觀察使李德裕獻丹扆六箴【扆於豈翻}
一曰宵衣以諷視朝希晩【朝直遥翻}
二曰正服以諷服御乖異三曰罷獻以諷徵求玩好【好呼到翻}
四曰納誨以諷侮棄讜言【讜音黨}
五曰辯邪以諷信任羣小六曰防微以諷輕出遊幸其納誨箴畧曰漢驁流湎舉白浮鍾【事見三十一卷漢成帝永始二年成帝諱驁音五到翻}
魏叡侈汰陵霄作宫【事見七十三卷魏明帝青龍三年明帝諱叡}
忠雖不忤善亦不從【忤五故翻}
以規為瑱是謂塞聰【左氏外傳楚靈王虐白公子張驟諫王曰不穀雖不能用吾憗寘之於耳對曰賴君之用也故言不然犀犛兕象其可盡乎其又以規為瑱也韋昭注曰瑱所以塞耳也言四獸之牙角可以為瑱難盡也而又以規諫為之乎瑱他甸翻塞悉則翻下同}
防微箴曰亂臣猖獗非可遽數玄服莫辯【漢宣帝時霍氏外孫任宣坐謀反誅宣子章亡在渭城界夜玄服入廟居廊間執戟立廟門待上至欲為逆覺伏誅}
觸瑟始仆【馬何羅事見二十二卷漢武帝征和四年}
柏谷微行犲豕塞路覩貌獻餐斯可戒懼【事見十七卷漢武帝建元三年}
上優詔答之 上既復繫崔於獄【復扶又翻}
給事中李渤上言縣令不應曳中人中人不應御囚【敕旨所囚繫者謂之御囚}
其罪一也然縣令所犯在赦前中人所犯在赦後中人横暴【横戶孟翻}
一至於此若不早正刑書臣恐四方藩鎭聞之則慢易之心生矣【易以豉翻}
諫議大夫張仲方上言畧曰鴻恩將布於天下而不行御前霈澤徧被於昆蟲而獨遺崔【被皮義翻}
自餘諫官論奏甚衆上皆不聽戊子李逢吉等從容言於上曰【從千容翻}
崔輒曳中人誠大不敬【律以對捍制使無人臣之禮為大不敬今崔曳中使故先以此罪坐之}
然其母故相韋貫之之姊也年垂八十自下獄積憂成疾【下遐稼翻}
陛下方以孝理天下此所宜矜念上乃愍然曰比諫官但言寃未嘗言其不敬亦不言有老母如卿所言朕何為不赦之【此以母子天性感之易所謂納約自牖者也但逢吉以權數耳比毗至翻}
即命中使釋其罪送歸家仍慰勞其母【勞力到翻}
母對中使杖四十 三月辛酉遣司門郎中于人文冊囘鶻曷薩特勒為愛登里囉汨没密於合毗伽昭禮可汗【囉魯何翻汨越筆翻}
夏四月癸巳羣臣上尊號曰文武大聖廣孝皇帝赦天下赦文但云左降官已經量移者宜與量移不言未量移者翰林學士韋處厚上言逢吉恐李紳量移故有此處置如此則應近年流貶官因李紳一人皆不得量移也【量音良處昌呂翻}
上即追赦文改之紳由是得移江州長史 秋七月甲辰鹽鐵使王播進羨餘絹百萬匹播領鹽鐵誅求嚴急正入不充而羨餘相繼【正入謂歲入有正額者羨弋線翻}
己未詔王播造競渡船二十艘【荆楚歲時記屈原以五月五日死於汨羅人傷其死並以舟楫拯之至今競渡是其遺俗自唐以來治競渡船務為輕駛前建龍頭後豎龍尾船之兩旁刻為龍鱗而綵繪之謂之龍舟植標於中流衆船鼔楫競進以争錦標有破舟折楫至於沉溺而不悔者}
運材於京師造之計用轉運半年之費諫議大夫張仲方等力諫乃減其半 諫官言京兆尹崔元畧以諸父事内常侍崔潭峻丁卯元畧遷戶部侍郎 昭義節度使劉悟之去鄆州也以鄆兵二千自隨為親兵八月庚戌悟暴疾薨子將作監主簿從諫匿其喪 【考異曰据李絳疏云悟八月十五日得病計是日便死故置此餘從杜牧書}
與大將劉武德及親兵謀以悟遺表求知留後司馬賈直言入責從諫曰爾父提十二州地歸朝廷【謂殺李師道以鄆青等州歸朝廷也事見二百四十一卷憲宗元和十四年}
其功非細祗以張汶之故【張汶事見上卷穆宗長慶二年汶音問}
自謂不潔淋頭【今人謂屎為不潔}
竟至羞死爾孺子何敢如此父死不哭何以為人從諫恐悚不能對乃喪 初陳留人武昭罷石州刺史為袁王府長史【石州漢離石縣地唐置石州京師東北一千二百九十一里袁王紳順宗子}
鬱鬱怨執政李逢吉與李程不相悦水部郎中李仍叔程之族人激怒之云程欲與昭官為逢吉所沮昭酒酣對左金吾兵曹茅彚言欲刺逢吉【刺七亦翻}
為人所告九月庚辰詔三司鞫之前河陽掌書記李仲言謂彚曰君言李程與昭謀則生不然必死彚曰寃死甘心誣人自全彚不為也獄成冬十月甲子武昭杖死李仍叔貶道州司馬李仲言流象州茅彚流崖州【彚于貴翻}
上欲幸驪山温湯左僕射李絳諫議大夫張仲方等屢諫不聽拾遺張權輿伏紫宸殿下叩頭諫曰昔周幽王幸驪山為犬戎所殺【史記周幽王愛褒姒褒姒不好笑王欲其笑萬方終不笑幽王為烽燧有寇至則舉烽火諸侯悉至而無寇褒姒乃大笑幽王悦之為數舉烽火其後不信諸侯益不至西夷犬戎攻幽王王舉烽徵兵兵莫至遂殺幽王驪山下}
秦始皇葬驪山國亡玄宗宫驪山而祿山亂先帝幸驪山享年不長【事並見前紀}
上曰驪山若此之凶邪我宜一往以驗彼言十一月庚寅幸温湯即日還宫謂左右曰彼叩頭者之言安足信哉【史言敬宗荒縱而愎諫}
丙申立皇子普為晉王 朝廷得劉悟遺表議者多言上黨内鎭與河朔異不可許左僕射李絳上疏以為兵機尚速威斷貴定【斷丁亂翻下裁斷同}
人情未一乃可伐謀劉悟死已數月朝廷尚未處分【處昌呂翻分扶問翻}
中外人意共惜事機今昭義兵衆必不盡與從諫同謀縱使其半叶同尚有其半効順從諫未嘗久典兵馬威惠未加於人又此道素貧【言昭義一道素來貧薄不比他道豐富}
非時必無優賞今朝廷但速除近澤潞一將充昭義節度使令兼程赴鎭從諫未及布置新使已至潞州所謂先人奪人之心也【先悉薦翻左傳趙宣子之言使疏吏翻下同}
新使既至軍心自有所繫從諫無位何名主張設使謀撓朝命【撓奴教翻又奴巧翻}
其將士必不肯從今朝廷久無處分【處昌呂翻分扶問翻}
彼軍不曉朝廷之意【彼軍謂昭義軍也}
欲効順則恐忽授從諫欲同惡則恐别更除人猶豫之間若有姦人為之畫策虚張賞設錢數軍士覬望尤難指揮【為于偽翻賞設猶言賞犒也覬凡利翻}
伏望速賜裁斷仍先下明敕【明敕猶言明詔斷丁亂翻下遐稼翻}
宣示軍衆奬其從來忠節【言澤潞一軍自李抱眞以來盡忠竭節於朝廷}
賜新使繒五十萬匹使之賞設【繒慈陵翻}
續除劉從諫一刺史從諫既粗有所得必且擇利而行萬無違拒設不從命臣亦以為不假攻討何則臣聞從諫已禁山東三州軍士不許自畜兵刀【昭義廵屬邢洛磁三州皆在山東}
足明羣心殊未得一帳下之事亦在不疑【言帳下必有圖從諫以為功者}
熟計利害决無即授從諫之理時李逢吉王守澄計議已定竟不用絳等謀 【考異曰實錄從諫以金幣賂當權者舊從諫傳曰李逢吉王守澄受其賂曲為奏請事有無難明今不取}
十二月辛丑以從諫為昭義留後劉悟煩苛從諫濟以寛厚衆頗附之李絳好直言李逢吉惡之故事僕射上日【好呼到翻惡烏路翻上時掌翻}
宰相送之百官立班中丞列位於廷尚書以下每月當牙【牙牙參也}
元和中伊愼為僕射太常博士韋謙上言舊儀太重削去之【去羌呂翻}
御史中丞王播恃逢吉之勢與絳相遇於塗不之避絳引故事上言僕射國初為正宰相【唐初太宗為尚書令羣臣不敢居其位自是不除授以左右僕射為尚書省長官其任為正宰相所謂參議朝政參知機務同平章事雖皆宰相之職然非正宰相也}
禮數至重儻人才黍位自宜别授賢良若朝命守官豈得有虧法制乞下百官詳定議者多從絳議【朝直遥翻下遐稼翻}
上聽行舊儀甲子以絳有足疾除太子少師分司 言事者多稱裴度賢不宜棄之藩鎭上數遣使至興元勞問度【數所角翻勞力到翻}
密示以還期度因求入朝逢吉之黨大懼二年春正月壬辰裴度自興元入朝李逢吉之黨百計毁之先是民間謡云【先悉薦翻}
緋衣小兒坦其腹天上有口被驅逐【緋衣裴字天上有口吳字謂度能擒吳元濟其才為可用也}
又長安城中有横亘六岡如乾象度宅偶居第五岡【六岡横亘如乾卦六畫之象裴度平樂里第偶居第五岡程大昌曰宇文愷之營隋都也曰朱雀街南北盡郭有六條高坡象乾卦六爻故於九二置宫殿以當帝王之居九三立百司以應君子之數九五貴位不欲常人居之故置玄都觀及興善寺以鎭其地劉禹錫賦看花詩即此也裴度宅在朱雀街東自北而南則為第四坊名永樂坊畧與玄都觀東西相對而其第之比觀基蓋退北兩坊不正相當也唐實錄裴度在興元自請入覲李逢吉之黨有張權輿者排之以為度名應圖䜟宅據乾岡不召而來其意可見蓋權輿之所謂宅據乾岡者即龍首第五坡之餘勢也然度之所居張說第在其西尤與玄都觀相近而張嘉貞之第正在坊北何獨指度為占據乾岡也小人挾私欺君皆此類}
張權輿上言度名應圖䜟宅占岡原不召而來其旨可見 【考異曰舊逢吉傳曰寶歷初度連上章請入覲逢吉之黨坐不安席如矢攅身乃相與為謀欲沮其來張權輿撰非衣小兒之謡傳於閭巷言度相有天分名應謡䜟而韋處厚於上前解析權輿所撰之言按權輿若撰謡言當更加以惡言不止云天上有口被驅逐觀此蓋民間先有此謡權輿因言度名應圖䜟非撰之也}
上雖年少【少詩照翻}
悉察其誣謗待度益厚度初至京師朝士塡門度留客飲京兆尹劉栖楚附度耳語侍御史崔咸舉觴罰度曰丞相不應許所由官呫囁耳語【京尹任煩劇故唐人謂府縣官為所由官項安世家說曰今坊市公人謂之所由呫叱涉翻囁而涉翻呫囁細語口動而聲不遠聞}
度笑而飲之栖楚不自安趨出二月辛未以度為司空同平章事度在中書左右忽白失印聞者失色度飲酒自如頃之左右白復於故處得印度不應或問其故度曰此必吏人盗之以印書劵耳急之則投諸水火緩之則復還故處【或問當左右白得印之時豈不可就詰其人以得印所自邪答曰晉公處此必有說請自詳度}
人服其識量上自即位以來欲幸東都宰相及朝臣諫者甚衆上

皆不聽决意必行己令度支員外郎盧貞按視修東都宫闕及道中行宫【自長安歷華陜至洛沿道皆有行宫如華隂之瓊岳宫金城宫鄭縣之神臺宫陜縣之繡嶺宫澠池之芳桂宫福昌之福昌宫永寜之崎岫宫蘭宫夀安之連昌宫興泰宫是也}
裴度從容言於上曰國家本設兩都以備廵幸自多難以來兹事遂廢【從千容翻難乃旦翻}
今宫闕營壘百司廨舍率已荒阤【阤施是翻廢也}
陛下儻欲行幸宜命有司歲月間徐加完葺然後可往上曰從來言事者皆云不當往如卿所言不往亦可會朱克融王庭湊皆請以兵匠助修東都三月丁亥勅以修東都煩擾罷之【史言修東都之役非以羣臣論諫而罷特畏幽鎮之稱兵而罷耳}
召盧貞還先是朝廷遣中使賜朱克融時服【先悉薦翻}
克融以為踈惡執留勅使又奏當道今歲將士春衣不足乞度支給三十萬端匹又奏欲將兵馬及丁匠五千助修宫闕上患之以問宰相欲遣重臣宣慰仍索敕使【索山客翻}
裴度對曰克融無禮已甚殆將斃矣譬如猛獸自於山林中咆哮跳踉【咆蒲交翻嘷也哮虚交翻闞也踉呂張翻又音郎}
久當自困必不敢輒離巢穴【離力智翻}
願陛下勿遣宣慰亦勿索敕使旬日之後徐賜詔書云聞中官至彼稍失去就俟還朕自有處分時服有司製造不謹朕甚欲知之已令區處【處昌呂翻分扶問翻}
其將士春衣從來非朝廷徵皆本道自備朕不愛數十萬匹物但素無此例不可獨與范陽所稱助修宮闕皆是虚語若欲直挫其姦宜云丁匠宜速遣來已令所在排比供擬【比毗至翻}
彼得此詔必蒼黄失圖若且示含容則云修宫闕事在有司不假丁匠遠來如是而已不足勞聖慮也上悦從之 立才人郭氏為貴妃妃晉王普之母也 横海節度使李全畧薨其子副大使同捷擅領留後重賂鄰道以求承繼【為文宗討李同捷張本}
夏四月戊申以昭義留後劉從諫為節度使 五月幽州軍亂殺朱克融及其子延齡【果如裴度之言}
軍中立其少子延嗣主軍務 六月甲子上御三殿令左右軍教坊内園為擊毬手搏雜戲戲酣有斷臂碎首者夜漏數刻乃罷 己卯上幸興福寺【唐會要興福寺在修德坊本王君廓宅貞觀八年太宗為太穆皇后追福立為弘福寺神龍元年改名元和十二年築夾城自雲韶門過芳林門西至修德里以通於興福佛寺}
觀沙門文溆俗講【釋比講說類談空有而俗講者又不能演空有之義徒以悦俗邀布施而已溆象呂翻}
癸未衡王絢薨【絢順宗子音翾縣翻}
壬辰宣索左藏見在銀十萬兩金七千兩悉貯内藏以便賜與【索山客翻貯丁呂翻藏徂浪翻見賢遍翻}
道士趙歸真說上以神仙僧惟貞齊賢正簡說上以禱祠求福【說式芮翻}
皆出入宫禁上信用其言山人杜景先請徧歷江嶺求訪異人有潤州人周息元自言夀數百歲上遣中使迎之八月乙巳息元至京師【潤州至京師二千八百二十一里}
上館之禁中山亭【館古玩翻}
朱延嗣既得幽州虐用其人都知兵馬使李載義與弟牙内兵馬使載寜共殺延嗣并屠其家三百餘人載義權知留後九月數延嗣之罪以聞【數所具翻}
載義承乾之後也【承乾太宗長子以罪廢}
庚申魏博節度使史憲誠妄奏李同捷為軍士所逐走歸本道請束身歸朝尋奏同捷復歸滄州【史言史憲誠玩侮朝廷公肆欺罔}
壬申以中書侍郎同平章事李程同平章事充河東節度使 冬十月乙亥以李載義為盧龍節度使 十一月甲申以門下侍郎同平章事李逢吉同平章事充山南東道節度使 上遊戲無度狎暱羣小【暱尼質翻}
善擊毬好手搏【好呼到翻下同}
禁軍及諸道争獻力士又以錢萬緡付内園令召募力士晝夜不離側【離力智翻}
又好深夜自捕狐狸性復褊急【復扶又翻}
力士或恃恩不遜輒配流籍没宦官小過動遭捶撻皆怨且懼十二月辛丑上夜獵還宫與宦官劉克明田務澄許文瑞及擊毬軍將蘇佐明王嘉憲石從寛閻惟直等二十八人飲酒上酒酣入室更衣【更工衡翻}
殿上燭怱滅蘇佐明等弑上於室内【年十八}
劉克明等矯稱上旨命翰林學士路隋草遺制以絳王悟權句當軍國事【絳王悟憲宗子句古侯翻當下浪翻}
壬寅宣遺制絳王見宰相百官於紫宸外廡克明等欲易置内侍之執權者於是樞密使王守澄楊承和中尉魏從簡梁守謙定議【唐末謂兩樞密兩中尉為四貴}
以衛兵迎江王涵入宫【自十六宅迎入宫也}
左右神策飛龍兵進討賊黨盡斬之克明赴井出而斬之絳王為亂兵所害時事起蒼猝守澄等以翰林學士韋處厚博通古今一夕處置皆與之共議【處昌呂翻}
守澄等欲號令中外而疑所以為辭處厚曰正名討罪於義何嫌安可依違有所諱避又問江王當如何踐阼處厚曰詰朝當以王教布告中外以己平内難【詰去吉翻難乃旦翻}
然後羣臣三表勸進以太皇太后令冊命即皇帝位當時皆從其言時不暇復問有司【復扶又翻}
凡百儀法皆出於處厚無不叶宜癸卯以裴度攝冢宰百官謁見江王於紫宸外廡【見賢遍翻}
王素服涕泣甲辰見諸軍使於少陽院【少陽院以地望準之當在宫城東北隅太子居之亦謂之東宫今按閣本大明宫圖少陽院在浴堂殿東其北又有温室宣徽清思太和珠鏡等殿不正在宫城東北隅也 考異曰魏謩文宗實錄見軍使事承見百官下不云别日今從敬宗實錄}
趙歸真等諸術士及敬宗時佞幸者皆流嶺南或邊地乙巳文宗即位更名昂【更工衡翻}
戊申尊母蕭氏為皇太后王太后為寶歷太后是時郭太后居興慶宫王太后居義安殿蕭太后居大内上性孝謹事三宫如一【自此以後凡言上者皆文宗也}
每得珍異之物先薦郊廟次奉三宫然後進御蕭太后閩人也 庚戌以翰林學士韋處厚為中書侍郎同平章事 上自為諸王深知兩朝之弊【謂穆敬兩朝也朝直遙翻下同}
及即位勵精求治去奢從儉【治直吏翻去羌呂翻}
詔宫女非有職掌者皆出之出三千餘人五坊鷹犬準元和故事量留校獵外悉放之【量音良}
有司供宫禁年支物並準貞元故事省教坊翰林摠監冗食千二百餘員【摠監苑摠監也}
停諸司新加衣糧【諸司内諸司也衣糧敬宗濫恩所加也}
御馬坊場及近歲别貯錢穀所占陂田【占古贍翻}
悉歸之有司先宣索組繡彫鏤之物悉罷之【鏤郎豆翻}
敬宗之世每月視朝不過一二上始復舊制每奇日未嘗不視朝【奇紀宜翻隻也唐制天子以隻日視朝}
對宰相羣臣延訪政事久之方罷待制官舊雖設之未嘗召對至是屢蒙延問其輟朝放朝皆用偶日中外翕然相賀以為太平可冀【欲治之主不世出人君初政儻有一二足以新民視聽天下之所望重矣然卒無以副天下之望者魏高貴鄉公晉懷帝唐德宗文宗是也}


文宗元聖昭獻孝皇帝上之上

【本名涵即位更名昂穆宗第三子}


太和元年春二月乙巳赦天下改元 李同捷擅據滄景朝廷經歲不問【去年三月李同捷擅領横海留後}
同捷冀易世之後或加恩貸三月壬戌朔遣掌書記崔從長奉表與其弟同志同巽俱入見【見賢遍翻}
請遵朝旨 上雖虚懷聽納而不能堅决與宰相議事已定尋復中變【復扶又翻}
夏四月丙辰韋處厚於延英極論之因請避位上再三慰勞之【勞力到翻}
忠武節度使王沛薨庚申以太僕卿高瑀為忠武節度使【瑀音禹}
自大歷以來節度使多出禁軍其禁軍大將資高者皆以倍稱之息貸錢於富室【倍者子錢倍於本錢稱者子本相侔也稱尺證翻}
以賂中尉動踰億萬然後得之未嘗由執政至鎮則重歛以償所負【歛力瞻翻}
及沛薨裴度韋處厚始奏以瑀代之中外相賀曰自今債帥鮮矣【帥所類翻鮮息淺翻少也}
五月丙子以天平節度使烏重胤為横海節度使以前横海節度副使李同捷為兖海節度使朝廷猶慮河南北節度使搆扇同捷使拒命乃加魏博史憲誠同平章事丁丑加盧龍李載義平盧康志睦成德王庭湊檢校官 鹽鐵使王播自淮南入朝力圖大用所獻銀器以千計綾絹以十萬計六月癸巳以播為左僕射同平章事 秋七月癸酉葬睿武昭愍孝皇帝于莊陵【莊陵在京兆三原縣西北五里}
廟號敬宗 李同捷託為將士所留不受詔乙酉武寜節度使王智興奏請將本軍三萬人自備五月糧以討同捷許之八月庚子削同捷官爵命烏重胤王智興康志睦史憲誠李載義與義成節度使李聽義武節度使張璠各帥本軍討之【璠扶元翻帥讀曰率}
同捷遣其子弟以珍玩女妓賂河北諸鎮【妓渠綺翻}
戊午李載義執其姪并所賂獻之史憲誠與李全畧為婚姻及同捷叛密以糧助之裴度不知其所為謂憲誠無貳心憲誠遣親吏至中書請事韋處厚謂曰晉公於上前以百口保爾使主【裴度封晉國公節度使為一道之主故對其屬吏稱之為使主使疏吏翻}
處厚則不然但仰俟所為自有朝典耳憲誠懼不敢復與同捷通【讀史者以為裴度於是時耄及之矣韋處厚較聰明不惟不知度亦不知處厚矣一推心以待之一明法以示之此正寛嚴相濟所以制御彊藩復扶又翻}
王庭湊為同捷求節鉞不獲【為于偽翻}
乃助之為亂出兵境上以撓魏師【撓奴教翻}
又遣使厚賂沙陀酋長朱邪執宜欲與之連兵執宜拒不受【酋慈由翻長知丈翻}
冬十月天平横海節度使烏重胤擊同捷屢破之十一月丙寅重胤薨庚辰以保義節度使李寰為横海節度使【穆宗長慶三年以晉慈二州為保義軍}
從王智興之請也 十二月庚戌加王智興同平章事

二年春三月己卯王智興攻棣州焚其三門 自元和之末宦官益横【横戶孟翻}
建置天子在其掌握【穆宗及上皆宦官所立}
威權出人主之右人莫敢言上親策制舉人賢良方正昌平劉蕡【蕡符分翻}
對策及言其禍其畧曰陛下宜先憂者宫闈將變社稷將危天下將傾海内將亂又曰陛下將杜篡弑之漸則居正位而近正人遠刀鋸之賤親骨鯁之直【近其靳翻遠于願翻}
輔相得以專其任【相息亮翻}
庶職得以守其官奈何以䙝近五六人摠天下大政禍稔蕭牆姦生帷幄臣恐曹節侯覽復生於今日【曹節侯覽見漢桓帝紀復扶又翻}
又曰忠賢無腹心之寄閽寺持廢立之權陷先君不得正其終致陛下不得正其始【謂宦官弑敬宗而立上也春秋穀梁傳曰定元年春王不言正月定無正也定之無正何也昭公之終非正終也定之始非正始也昭無正終故定無正始}
又曰威柄陵夷藩臣跋扈或有不逹人臣之節首亂者以安君為名不究春秋之微稱兵者以逐惡為義【微為春秋之微指也此二語蕡蓋慮夫強藩首亂稱兵以逐君側惡臣為名者}
則政刑不由乎天子征伐必自於諸侯【昭宗之世岐汴交兵以誅宦官為名卒如劉蕡之言}
又曰陛下何不塞隂邪之路屏䙝狎之臣【塞悉則翻屏必郢翻又卑正翻}
制侵陵迫脅之心復門戶掃除之役戒其所宜戒憂其所宜憂既不能治於前當治於後【治直之翻}
既不能正其始當正其終則可以䖍奉典謨克承丕構矣昔秦之亡也失於彊暴漢之亡也失於微弱彊暴則賊臣畏死而害上【謂趙高也高亦宦者也}
微弱則姦臣竊權而震主【謂外戚宦官蕡意專指宦官}
伏見敬宗皇帝不虞亡秦之禍不翦其萌伏惟陛下深軫亡漢之憂以杜其漸【蕡蓋謂敬宗以荒暴喪身又恐上以仁弱不能制宦官也}
則祖宗之鴻業可紹三五之遐軌可追矣【三五謂三皇五帝}
又曰臣聞昔漢元帝即位之初更制七十餘事【其畧見二十八卷漢元帝初元元年二年}
其心甚誠其稱甚美然而紀綱日紊【稱尺證翻紊音汶}
國祚日衰姦宄日彊黎元日困者以其不能擇賢明而任之失其操柄也【引漢元以為戒者蓋以帝之去奢從儉似漢元而優遊不斷亦類漢元也}
又日陛下誠能揭國權以歸相持兵柄以歸將則心無不達行無不孚矣【行下孟翻}
又曰法宜畫一官宜正名今分外官中官之員立南司北司之局【百官赴南牙朝會者謂之外官亦謂之南司宦官列局於玄武門内兩軍中尉護諸營於苑中謂之中官亦謂之北司}
或犯禁於南則亡命於北或正刑於外則破律於中法出多門人無所措實由兵農勢異而中外法殊也又曰今夏官不知兵籍止於奉朝請【朝直遥翻}
六軍不主兵事止於養勲階【兵部古夏官之職六軍工將軍大將軍將軍統軍皆以養勲階}
軍容合中官之政戎律附内臣之職【謂觀軍容使及諸監軍使也}
首一戴武弁疾文吏如仇讐足一蹈軍門視農夫如草芥謀不足以翦除兇逆而詐足以抑揚威福勇不足以鎮衛社稷而暴足以侵軼里閭羈絏藩臣【軼徒結翻又音逸也絏先列翻}
干陵宰輔隳裂王度汨亂朝經【朝直遥翻}
張武夫之威上以制君父假天子之命下以御英豪有藏姦觀釁之心無仗節死難之義【難乃旦翻}
豈先王經文緯武之旨邪又曰臣非不知言而禍應計行而身戮蓋痛社稷之危哀生人之困豈忍姑息時忌竊陛下一命之寵哉【周禮一命受職後世以授初品官為一命}
閏月丙戌朔史憲誠奏遣其子副大使唐都知兵馬使开志紹將兵二萬五千趣德州討李同捷【开苦堅翻按考異從亓當音渠之翻二音皆姓也趣七喻翻 考異曰實錄或作于志沼或作开志沼或作亓志紹舊紀作开志紹新紀傳作亓志沼今從之據考異紹當作沼}
時憲誠欲助同捷唐泣諫且請兵討之憲誠不能違 甲午賢良方正裴休李郃李甘杜牧馬植崔璵王式崔慎由等【郃曷閣翻璵音余}
二十二人中第皆除官【中竹仲翻}
考官左散騎常侍馮宿等見劉蕡策皆歎服而畏宦官不敢取詔下物論囂然稱屈【囂虚驕翻喧也}
諫官御史欲論奏執政抑之李郃曰劉蕡下第我輩登科能無厚顔乃上疏以為蕡所對策漢魏以來無與為比今有司以蕡指切左右不敢以聞恐忠良道窮綱紀遂絶况臣所對不及蕡遠甚乞回臣所授以旌蕡直不報蕡由是不得仕於朝終於使府御史【使府節度使幕府也御史幕僚所帶寄禄官亦謂之憲官}
牧佑之孫植勛之子【杜佑歷德順憲三朝位至公輔馬勛見二百三十卷德宗貞元元年 考異曰舊傳勛作曛誤也勛事見德宗實錄}
式起之孫慎由融之玄孫也【王起見二百四十一卷穆宗長慶元年崔融以文章顯於武后朝}
夏六月晉王普薨辛酉諡悼懷太子初蕭太后幼去鄉里有弟一人上即位命福建觀察

使求訪莫知所在有茶綱役人蕭洪【凡茶商販茶各以若干為一綱而輸税于官}
自言有姊流落商人趙縝引之見太后近親呂璋之妻【縝指忍翻}
亦不能辯與之俱見太后上以為得真舅甲子以為太子洗馬【為蕭洪詐覺流死張本洗悉薦翻}
峯州刺史王升朝叛庚辰安南都護武陵韓約討斬之【舊志峯州至京師一萬一千五百里宋白曰峯州治嘉寜縣漢麓泠縣地武陵漢臨沅縣之地隋置武陵縣唐帶朗州朝直遙翻}
王庭湊隂以兵及鹽粮助李同捷上欲討之秋七月甲辰詔中書集百官議其事宰相以下莫敢違衛尉卿殷侑獨以為廷湊雖附凶徒事未甚露宜且含容專討同捷己巳下詔罪狀廷湊命鄰道各嚴兵守備聽其自新九月丁亥王智興奏拔棣州 李寰自晉州引兵赴鎮不戢士卒所過殘暴至則擁兵不進但坐索供饋【索山客翻}
庚寅以寰為夏綏節度使【夏戶雅翻}
甲午詔削奪王廷湊官爵命諸軍四面進討 加王智興守司徒以前夏綏節度使傅良弼為横海節度使 岳王緄薨【緄順宗子音古本翻}
庚戌容管奏安南軍亂逐都護韓約 冬十月洋王

忻薨【忻憲宗子}
魏博敗横海兵於平原遂拔之【敗補邁翻}
十一月癸未朔易定節度使柳公濟奏攻李同捷堅固寨拔之【同捷築寨於滄州西以抗官兵以堅固為名}
又破其兵於寨東時河南北諸軍討同捷久未成功每有小勝則虚張首虜以邀厚賞朝廷竭力奉之江淮為之耗弊 傅良弼至陜而薨【陜失冉翻}
乙酉以左金吾大將軍李祐為横海節度使甲辰禁中昭德寺火【天火曰災人火曰火}
延及宫人所居燒死者數百人 十二月丁巳王智興奏兵馬使李君謀將兵濟河破無棣【無棣古齊國之北境周封太公賜履所至也漢為陽信縣界有無棣溝通海唐為無棣縣屬滄州九域志在州東南一百七里}
壬申中書侍郎同平章事韋處厚薨 李同捷軍勢日蹙王庭湊不能救乃遣人說魏博大將亓志紹【說式芮翻}
使殺史憲誠父子取魏博志紹遂作亂引所部兵二萬人還逼魏州丁丑命諫議大夫柏耆宣慰魏博且義成河陽兵以討志紹 戊寅以翰林學士路隋為中書侍郎同平章事 辛巳史憲誠奏亓志紹兵屯永濟【代宗大歷七年田承嗣分魏州之臨清置永濟縣屬貝州}
告急求援詔義成節度使李聽帥滄州行營諸軍以討志紹【帥讀曰率}


資治通鑑卷二百四十三  














































































































































