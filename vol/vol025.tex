<!DOCTYPE html PUBLIC "-//W3C//DTD XHTML 1.0 Transitional//EN" "http://www.w3.org/TR/xhtml1/DTD/xhtml1-transitional.dtd">
<html xmlns="http://www.w3.org/1999/xhtml">
<head>
<meta http-equiv="Content-Type" content="text/html; charset=utf-8" />
<meta http-equiv="X-UA-Compatible" content="IE=Edge,chrome=1">
<title>資治通鑒_26-資治通鑑卷二十五_26-資治通鑑卷二十五</title>
<meta name="Keywords" content="資治通鑒_26-資治通鑑卷二十五_26-資治通鑑卷二十五">
<meta name="Description" content="資治通鑒_26-資治通鑑卷二十五_26-資治通鑑卷二十五">
<meta http-equiv="Cache-Control" content="no-transform" />
<meta http-equiv="Cache-Control" content="no-siteapp" />
<link href="/img/style.css" rel="stylesheet" type="text/css" />
<script src="/img/m.js?2020"></script> 
</head>
<body>
 <div class="ClassNavi">
<a  href="/24shi/">二十四史</a> | <a href="/SiKuQuanShu/">四库全书</a> | <a href="http://www.guoxuedashi.com/gjtsjc/"><font  color="#FF0000">古今图书集成</font></a> | <a href="/renwu/">历史人物</a> | <a href="/ShuoWenJieZi/"><font  color="#FF0000">说文解字</a></font> | <a href="/chengyu/">成语词典</a> | <a  target="_blank"  href="http://www.guoxuedashi.com/jgwhj/"><font  color="#FF0000">甲骨文合集</font></a> | <a href="/yzjwjc/"><font  color="#FF0000">殷周金文集成</font></a> | <a href="/xiangxingzi/"><font color="#0000FF">象形字典</font></a> | <a href="/13jing/"><font  color="#FF0000">十三经索引</font></a> | <a href="/zixing/"><font  color="#FF0000">字体转换器</font></a> | <a href="/zidian/xz/"><font color="#0000FF">篆书识别</font></a> | <a href="/jinfanyi/">近义反义词</a> | <a href="/duilian/">对联大全</a> | <a href="/jiapu/"><font  color="#0000FF">家谱族谱查询</font></a> | <a href="http://www.guoxuemi.com/hafo/" target="_blank" ><font color="#FF0000">哈佛古籍</font></a> 
</div>

 <!-- 头部导航开始 -->
<div class="w1180 head clearfix">
  <div class="head_logo l"><a title="国学大师官网" href="http://www.guoxuedashi.com" target="_blank"></a></div>
  <div class="head_sr l">
  <div id="head1">
  
  <a href="http://www.guoxuedashi.com/zidian/bujian/" target="_blank" ><img src="http://www.guoxuedashi.com/img/top1.gif" width="88" height="60" border="0" title="部件查字,支持20万汉字"></a>


<a href="http://www.guoxuedashi.com/help/yingpan.php" target="_blank"><img src="http://www.guoxuedashi.com/img/top230.gif" width="600" height="62" border="0" ></a>


  </div>
  <div id="head3"><a href="javascript:" onClick="javascript:window.external.AddFavorite(window.location.href,document.title);">添加收藏</a>
  <br><a href="/help/setie.php">搜索引擎</a>
  <br><a href="/help/zanzhu.php">赞助本站</a></div>
  <div id="head2">
 <a href="http://www.guoxuemi.com/" target="_blank"><img src="http://www.guoxuedashi.com/img/guoxuemi.gif" width="95" height="62" border="0" style="margin-left:2px;" title="国学迷"></a>
  

  </div>
</div>
  <div class="clear"></div>
  <div class="head_nav">
  <p><a href="/">首页</a> | <a href="/ShuKu/">国学书库</a> | <a href="/guji/">影印古籍</a> | <a href="/shici/">诗词宝典</a> | <a   href="/SiKuQuanShu/gxjx.php">精选</a> <b>|</b> <a href="/zidian/">汉语字典</a> | <a href="/hydcd/">汉语词典</a> | <a href="http://www.guoxuedashi.com/zidian/bujian/"><font  color="#CC0066">部件查字</font></a> | <a href="http://www.sfds.cn/"><font  color="#CC0066">书法大师</font></a> | <a href="/jgwhj/">甲骨文</a> <b>|</b> <a href="/b/4/"><font  color="#CC0066">解密</font></a> | <a href="/renwu/">历史人物</a> | <a href="/diangu/">历史典故</a> | <a href="/xingshi/">姓氏</a> | <a href="/minzu/">民族</a> <b>|</b> <a href="/mz/"><font  color="#CC0066">世界名著</font></a> | <a href="/download/">软件下载</a>
</p>
<p><a href="/b/"><font  color="#CC0066">历史</font></a> | <a href="http://skqs.guoxuedashi.com/" target="_blank">四库全书</a> |  <a href="http://www.guoxuedashi.com/search/" target="_blank"><font  color="#CC0066">全文检索</font></a> | <a href="http://www.guoxuedashi.com/shumu/">古籍书目</a> | <a   href="/24shi/">正史</a> <b>|</b> <a href="/chengyu/">成语词典</a> | <a href="/kangxi/" title="康熙字典">康熙字典</a> | <a href="/ShuoWenJieZi/">说文解字</a> | <a href="/zixing/yanbian/">字形演变</a> | <a href="/yzjwjc/">金 文</a> <b>|</b>  <a href="/shijian/nian-hao/">年号</a> | <a href="/diming/">历史地名</a> | <a href="/shijian/">历史事件</a> | <a href="/guanzhi/">官职</a> | <a href="/lishi/">知识</a> <b>|</b> <a href="/zhongyi/">中医中药</a> | <a href="http://www.guoxuedashi.com/forum/">留言反馈</a>
</p>
  </div>
</div>
<!-- 头部导航END --> 
<!-- 内容区开始 --> 
<div class="w1180 clearfix">
  <div class="info l">
   
<div class="clearfix" style="background:#f5faff;">
<script src='http://www.guoxuedashi.com/img/headersou.js'></script>

</div>
  <div class="info_tree"><a href="http://www.guoxuedashi.com">首页</a> > <a href="/SiKuQuanShu/fanti/">四库全书</a>
 > <h1>资治通鉴</h1> <!--         下载:【右键另存为】即可 --></div>
  <div class="info_content zj clearfix">
  
<div class="info_txt clearfix" id="show">
<center style="font-size:24px;">26-資治通鑑卷二十五</center>
    資治通鑑卷二十五   宋 司馬光 撰<br />
<br />
  胡三省 音註<br />
<br />
  漢紀十七【起閼逢攝提格盡屠維協洽凡六年】<br />
<br />
  中宗孝宣皇帝上之下<br />
<br />
  地節三年春三月詔曰盖聞有功不賞有罪不誅雖唐虞不能化天下今膠東相王成勞來不怠【師古曰謂勸勉招懷百姓勞郎到翻來郎代翻】流民自占八萬餘口【師古曰隱度名數而來附業也占音之贍翻】治有異等之效【師古曰異于常等治直吏翻】其賜成爵關内侯秩中二千石未及徵用會病卒官【卒子恤翻】後詔使丞相御史問郡國上計長史守丞以政令得失【貢父曰郡使守丞國使長史皆一物也故揔言郡國上計長史守丞後漢百官志諸侯王相如太守長史如郡丞又邉郡有丞元有長史長史上計無疑矣上時掌翻】或對言前膠東相成偽自增加以蒙顯賞是後俗吏多為虛名云 夏四月戊申立子奭為皇太子以丙吉為太傅太中大夫疏廣為少傅【疏姓也 考異曰荀紀立皇太子在去年四月戊申漢書舊本亦然顔師古據疏廣及丙吉傳並云地節三年立皇太子知在此年者是也】封太子外祖父許廣漢為平恩侯【平恩侯國屬魏郡宋白曰魏為縣屬廣平郡唐屬洺州有平恩州】又封霍光兄孫中郎將雲為冠陽侯【恩澤侯表冠陽侯食邑於南陽郡】霍顯聞立太子怒恚不食歐血曰【恚於避翻歐烏口翻】此乃民間時子安得立即后有子反為王邪復教皇后令毒太子皇后數召太子賜食保阿輒先嘗之【保母阿母也復扶又翻數所角翻】后挾毒不得行 五月甲申丞相賢以老病乞骸骨賜黄金百斤安車駟馬罷就第丞相致仕自賢始六月壬辰以魏相為丞相辛丑丙吉為御史大夫疏廣為太子太傅廣兄子受為少傅太子外祖父平恩侯許伯以為太子少白使其弟中郎將舜監護太子家【許伯即許廣漢稱伯者盖尊之也少詩照翻監古銜翻】上以問廣廣對曰太子國儲副君師友必於天下英俊不宜獨親外家許氏且太子自有太傅少傅官屬已備今復使舜護太子家示陋【師古曰言獨親外家示天下以淺陋復扶又翻】非所以廣太子德于天下也上善其言以語魏相【語牛倨翻】相免冠謝曰此非臣等所能及廣由是見器重 亰師大雨雹大行丞東海蕭望之上疏言大臣任政一姓專權之所致【據望之傳為大行治禮丞】上素聞望之名拜為謁者時上博延賢俊民多上書言便宜輒下望之問狀【下遐稼翻】高者請丞相御史【師古曰望之以其人所言之狀請於丞相御史或以奏聞即見超擢】次者中二千石試事滿歲以狀聞【師古曰試令行其所言之事或以諸他職事試之劉仲馮曰觀其意共是一條不當中分却煩解說也顔說非也高者則令丞相御史試事歲滿各以狀聞誤斷其文爾余謂高者則請丞相御史試事次者中二千石試事文意固是一貫而分高次則非誤斷也】下者報聞罷【其言不可用故報聞而罷歸田里也】所白處奏皆可【師古曰當主上之意也處昌呂翻】 冬十月詔曰乃者九月壬申地震朕甚懼焉有能箴朕過失及賢良方正直言極諫之士以匡朕之不逮毋諱有司【師古曰箴戒也匡正也李奇曰諱避也雖有司在顯職皆言其過勿避之】朕既不德不能附遠是以邉境屯戍未息今復飭兵重屯久勞百姓【師古曰飭整也復扶又翻下同】非所以綏天下也其罷車騎將軍右將軍屯兵又詔池籞未御幸者假與貧民【蘇林曰折竹以䋲綿連禁禦使人不得往來律名為籞服䖍曰籞在池水中作室可用棲鳥鳥入中則捕之應劭曰池者陂池也籞者禁苑也臣瓚曰籞者所以養鳥也設為藩落周覆其上令鳥不得出猶苑之畜獸池之畜魚也師古曰蘇應二說是】郡國宫觀勿復修治【治直之翻下同】流民還歸者假公田貸種食【師古曰貸音吐戴翻 種五穀種也音章勇翻】且勿筭事【師古曰不出筭賦及給徭役】 霍氏驕侈縱横【横戶孟翻】太夫人顯廣治第室作乘輿輦加畫繡絪馮黄金塗韋絮薦輪【如淳曰絪亦茵馮謂所馮者也以黄金塗飾之師古曰茵褥也以繡為茵馮而黄金塗于輦也晉灼曰御輦以韋緣輪著之以絮師古曰取其行安不揺動也馮與憑同著音張呂翻】侍婢以五采絲輓顯游戲第中【師古曰輓謂牽引車輦也音晚】與監奴馮子都亂【師古曰監奴謂奴之監知家務者】而禹山亦並繕治第宅走馬馳逐平樂館雲當朝請數稱病私出【樂音洛朝直遥翻請才性翻數所角翻下同】多從賓客張圍獵黄山苑中使蒼頭奴上朝謁【文穎曰朝當用謁不自行而令奴上謁者也師古曰上謁若今參見尊貴而通人也孔穎逹曰漢家僕隸謂之蒼頭以蒼巾為飾異于民也上時掌翻】莫敢譴者顯及諸女晝夜出入長信宫殿中亡期度【師古曰長信宫上官太后所居亡古無字通】帝自在民閒聞知霍氏尊盛日久内不能善既躬親朝政御史大夫魏相給事中顯謂禹雲山女曹不務奉大將軍餘業【師古曰女音汝曹輩也】今大夫給事中他人壹間女能復自救邪【間古莧翻復扶又翻下同】後兩家奴争道【師古曰謂霍氏及御史家】霍氏奴入御史府欲躢大夫門御史為叩頭謝乃去【躢與蹋同為于偽翻】人以謂霍氏【師古曰告語也】顯等始知憂會魏大夫為丞相數燕見言事【見賢遍翻下同】平恩侯與侍中金安上等徑出入省中時霍山領尚書上令吏民得奏封事不關尚書羣臣進見獨往來【師古曰謂各各得盡言于上也】於是霍氏甚惡之【惡烏路翻】上頗聞霍氏毒殺許后而未察乃徙光女壻度遼將軍未央衛尉平陵侯范明友為光禄勲【功臣侯表平陵侯食邑於南陽郡之武當縣】出次壻諸吏中郎將羽林監任勝為安定太守【任音壬守式又翻下同】數月復出光姊壻給事中光禄大夫張朔為蜀郡太守羣孫壻中郎將王漢為武威太守頃之復徙光長女壻長樂衛尉鄧廣漢為少府戊戍更以張安世為衛將軍兩宫衛尉城門北軍兵屬焉【兩宫未央長樂也城門京城十二門屯兵也北軍北軍八校兵也更工衡翻】以霍禹為大司馬冠小冠【大司馬大將軍冠武弁大冠今貶禹故使冠小冠冠小之冠古玩翻】亡印綬【亡古無字通】罷其屯兵官屬特使禹官名與光俱大司馬者【蘇林曰特但也】又收范明友度遼將軍印綬但為光禄勲及光中女壻趙平為散騎騎都尉光禄大夫將屯兵又收平騎都尉印綬【散騎騎都尉以騎都尉而加散騎官也百官表云散騎中常侍皆加官中常侍得入禁中散騎騎並乘輿車如淳曰自列侯下至郎中皆得有散騎及中常侍加官是時散騎及中常侍各自一官無員也中讀曰仲】諸領胡越騎羽林及兩宫衛將屯兵悉易以所親信許史子弟代之 初孝武之世徵發煩數百姓貧耗窮民犯灋姦軌不勝【數所角翻勝音升又如字】於是使張湯趙禹之屬條定法令作見知故縱監臨部主之灋【師古曰見知人犯法不舉告為故縱而所監臨部主有罪併連坐之也監古衘翻】緩深故之罪【孟康曰孝武欲急刑吏深害及故入人罪者皆寛緩之也】急縱出之誅【師古曰吏釋罪人疑以為縱出則急誅之亦言尚酷】其後姦猾巧灋轉相比况禁罔寖密律令煩苛文書盈於几閣典者不能徧睹是以郡國承用者駮【師古曰不曉其指用意不同也】或罪同而論異姦吏因緣為市【師古曰弄法而受財若市買之交易】所欲活則傳生議【傳讀曰附】所欲䧟則予死比【師古曰比以例相比况也】議者咸寃傷之廷尉史鉅鹿路温舒上書曰臣聞齊有無知之禍而桓公以興【齊襄公為公子無知所殺雍廪復殺無知齊國大亂桓公自莒入立】晉有驪姬之難而文公用伯【晉獻公信驪姬之讒殺世子申生逐公子重耳夷吾乃立驪姬之子奚齊卓子皆為里克所殺夷吾入立復為秦所執既而歸之卒而子圉嗣秦納重耳子圉死迎文公遂霸諸侯難乃旦翻伯讀曰霸】近世趙王不終諸呂作亂而孝文為太宗【事見十三卷】繇是觀之禍亂之作將以開聖人也夫繼變亂之後必有異舊之恩此賢聖所以昭天命也往者昭帝即世無嗣昌邑淫亂乃皇天所以開至聖也臣聞春秋正即位大一統而慎始也【春秋之法繼弑君不言即位繼正即位正也】陛下初登至尊與天合符宜改前世之失正始受命之統滌煩文除民疾以應天意臣聞秦有十失其一尚存治獄之吏是也【治直之翻】夫獄者天下之大命也死者不可復生絶者不可復屬【復扶又翻師古曰屬連也音之欲翻】書曰與其殺不辜寧失不經【師古曰虞書大禹謨載咎繇之言辜罪也經常也言人命至重治獄宜慎寧失不常之過不濫殺無罪之人所以崇寛恕也】今治獄吏則不然上下相【敺與驅同】以刻為明深者獲公名平者多後患故治獄之吏皆欲人死非憎人也自安之道在人之死是以死人之血流離于市被刑之徒比肩而立【被皮義翻】大辟之計歲以萬數【辟毗亦翻】此仁聖之所以傷也太平之未洽凡以此也夫人情安則樂生【樂音洛】痛則思死棰楚之下何求而不得故囚人不勝痛則飾辭以示之【勝音升】吏治者利其然則指導以明之【治直吏翻】上奏畏卻則鍜鍊而周内之【上時掌翻晉灼曰精熟周悉致之法中也師古曰郤退也畏為上所郤退郤丘畧翻】盖奏當之成【師古曰當謂處其罪也】雖臯陶聽之猶以為死有餘辜【師古曰臯陶作士善聽獄訟故以為喻也陶音遥】何則成鍊者衆文致之罪明也故俗語曰畫地為獄議不入刻木為吏期不對【師古曰畫獄木吏尚不入對况真實乎期猶必也議必不入對】此皆疾吏之風悲痛之辭也唯陛下省灋制寛刑罰則太平之風可興於世上善其言 十二月詔曰間者吏用法巧文寖深是朕之不德也夫决獄不當使有罪興邪不辜蒙戮【晉灼曰當重而輕使有罪者起邪惡之心也師古曰有罪者更興邪惡無辜者反䧟重刑是决獄不平故也當丁浪翻】父子悲恨朕甚傷之今遣廷史與郡鞠獄任輕禄薄【如淳曰廷史廷尉史也以囚辭决獄事為鞠謂疑獄也李奇曰鞠窮也獄事窮竟也師古曰李說是也】其為置廷尉平秩六百石員四人其務平之以稱朕意於是每季秋後請讞時【為于偽翻稱尺證翻讞語蹇翻又魚戰翻又魚列翻議獄也】上常幸宣室齋居而决事【如淳曰宣室布政教之室也重用刑故齋戒以决事晉灼曰未央宫中有宣室殿師古曰晉說是也賈誼傳亦云受釐坐宣 室盖其殿在前殿之側也齋則居之】獄刑號為平矣涿郡太守鄭昌上疏言今明主躬垂明聽雖不置廷平獄將自正若開後嗣不若刪定律令【師古曰刪刋也有不便者則刋而除之】律令一定愚民知所避姦吏無所弄矣今不正其本而置廷平以理其末政衰聽怠則廷平將召權而為亂首矣【孟康曰召求也招致權著已也猶賣弄也師古曰孟說是也】昭帝時匈奴使四千騎田車師及五將軍擊匈奴【事見上卷本始三年】車師田者驚去車師復通于漢匈奴怒召其太子軍宿欲以為質軍宿焉耆外孫不欲質匈奴亡走焉耆車師王更立子烏貴為太子【復扶又翻下同質音致走音奏更工衡翻】及烏貴立為王與匈奴結婚姻教匈奴遮漢道通烏孫者是歲侍郎會稽鄭吉與校尉司馬憙【會古外翻憙許吏翻】將免刑罪人田渠犂積穀【罪人免其刑使屯田】發城郭諸國兵萬餘人【西域諸國有逐水草與匈奴同俗者謂之行國其城居者謂之城郭諸國也】與所將田士千五百人共擊車師破之車師王請降【降戶江翻】匈奴發兵攻車師吉憙引兵北逢之匈奴不敢前吉憙即留一侯與卒二十人留守王吉等引兵歸渠犂 【考異曰西域傳云地節二年以匈奴傳校之知在三年】車師王恐匈奴兵復至而見殺也乃輕騎奔烏孫吉即迎其妻子傳送長安【傳知戀翻】匈奴更以車師王昆弟兜莫為車師王收其餘民東徙不敢居故地而鄭吉始使吏卒三百人往田車師地以實之【為下元康二年匈奴争車師張本】 上自初即位數遣使者求外家【數所角翻】久遠多似類而非是是歲求得外祖母王媪【文頴曰幽州及漢中皆謂老嫗曰媪師古曰媪女老稱也音烏老翻】及媪男無故武【無故及武皆媪子也】上賜無故武爵關内侯旬日間賞賜以鉅萬計四年春二月賜外祖母號為博平君【據外戚傳以博平蠡吾二縣為湯沐邑而地理志博平縣屬東郡】封舅無故為平昌侯【平昌侯國屬平原郡】武為樂昌侯【樂昌侯國屬東郡恩澤侯表武封樂昌侯食邑于汝南】夏五月山陽濟隂雹如雞子深二尺五寸【深式浸翻】殺二十餘人飛鳥皆死 詔自今子有匿父母妻匿夫孫匿大父母皆勿治 立廣川惠王孫文為廣川王【本始四年廣川王去以罪自殺今復立文嗣封王】 霍顯及禹山雲自見日侵削數相對啼泣自怨【數所角翻】山曰今丞相用事縣官信之盡變易大將軍時灋令發揚大將軍過失又諸儒生多窶人子【師古曰窶貧而無禮孔頴逹曰貧無可為禮謂之窶音其羽翻】遠客飢寒喜妄說狂言【喜許吏翻】不避忌諱大將軍常讐之【師古曰言嫉之如仇讎也】今陛下好與諸儒生語【好呼到翻】人人自書對事多言我家者嘗有上書言我家昆弟驕恣其言疾痛山屛不奏後上書者益黠盡【屏必郢翻黠下入翻】奏封事輒使中書令出取之不關尚書益不信人又聞民間讙言霍氏毒殺許皇后【師古曰讙衆聲也音許爰翻毒許后事見上卷本始三年】寧有是邪顯恐急即具以實告禹山雲禹山雲驚曰如是何不早告禹等縣官離散斥逐諸壻用是故也此大事誅罰不小柰何於是始有邪謀矣雲舅李竟所善張赦見雲家卒卒【師古曰卒讀曰猝怱遽之貌也】謂竟曰今丞相與平恩侯用事可令太夫人言太后【太夫人謂霍顯上官太后霍氏外孫也】先誅此兩人移徙陛下在太后耳長安男子張章告之事下廷尉【下遐嫁翻】執金吾捕張赦等後有詔止勿捕山等愈恐相謂曰此縣官重太后故不竟也【師古曰重難也竟窮竟其事也】然惡端已見【見賢遍翻】久之猶發發即族矣不如先也【師古曰言先翻】遂令諸女各歸報其夫皆曰安所相避【師古曰言無處相避當受禍也】會李竟坐與諸侯王交通辭語及霍氏有詔雲山不宜宿衛免就第山陽太守張敞上封事曰臣聞公子季友有功于魯趙衰有功于晉田完有功于齊皆疇其庸延及子孫終後田氏簒齊趙氏分晉季氏顓魯【魯公子季友殺慶父立僖公以安魯國遂世為上卿專魯國之政晉公子重耳出亡趙衰從及其反國伯諸侯衰皆有功遂世為晉卿有軍行至趙鞅遂與智韓魏分晉國田完自陳奔齊桓公禮而用之桓公之伯完與有功其後陳成子得齊國之政至田和遂簒齊而有之】故仲尼作春秋迹盛衰譏世卿最甚乃者大將軍决大計安宗廟定天下功亦不細矣夫周公七年耳【周公輔成王七年而反政於成王】而大將軍二十歲【自武帝後元二年至地節二年適二十歲】海内之命斷於掌握【斷丁亂翻】方其隆盛時感動天地侵廹隂陽朝臣宜有明言曰陛下褒寵故大將軍以報功德足矣間者輔臣顓政貴戚太盛君臣之分不明【分扶問翻】請罷霍氏三侯皆就第及衛將軍張安世宜賜几杖歸休時存問召見【見賢遍翻】以列侯為天子師明詔以恩不聽羣臣以義固爭而後許之天下必以陛下為不忘功德而朝臣為知禮【朝直遥翻下同】霍氏世世無所患苦今朝廷不聞直聲【師古曰言朝臣不進直言以陳其事】而令明詔自親其文非策之得者也【師古曰言失計也】今兩侯已出人情不相遠以臣心度之【度徒洛翻】大司馬及其枝屬必有畏懼之心夫近臣自危非完計也臣敞願於廣朝白發其端直守遠郡【師古曰直讀曰值朝直遥翻】其路無由唯陛下省察【省悉井翻】上甚善其計然不召也禹山等家數有妖怪【數所角翻妖于驕翻】舉家憂愁山曰丞相擅減宗廟羔菟鼃可以此罪也【如淳曰高后時定令輒有擅議宗廟者棄市師古曰羔菟鼃所以供祭也菟吐故翻鼃古蛙字】謀令太后為博平君置酒【為于偽翻】召丞相平恩侯以下使范明友鄧廣漢承太后制引斬之因廢天子而立禹約定未發雲拜為玄菟太守【菟同都翻】太中大夫任宣為代郡太守會事發覺秋七月雲山明友自殺顯禹廣漢等捕得禹要斬【要古腰字通】顯及諸女昆弟皆棄市與霍氏相連坐誅滅者數十家太僕杜延年以霍氏舊人亦坐免官八月己酉皇后霍氏廢處昭臺宮【師古曰在上林苑中處昌呂翻】乙丑詔封告霍氏反謀者男子張章期門董忠左曹楊惲【百官表侍中左右曹皆加官晉灼曰漢儀注諸吏給事中日上朝謁平尚書奏事分為左右曹惲於粉翻】侍中金安上史高皆為列侯【章為博成侯忠高昌侯惲平通侯安上都成侯高為樂陵侯】惲丞相敞子安上車騎將軍日磾弟子高史良娣兄子也初霍氏奢侈茂陵徐生曰霍氏必亡夫奢則不遜不遜則侮上侮上者逆道也在人之右【師古曰右上也】衆必害之霍氏秉權日久害之者多矣天下害之而又行以逆道不亡何待乃上疏言霍氏泰盛陛下即愛厚之宜以時抑制無使至亡書三上輒報聞【漢制上書不行者輒報聞罷】其後霍氏誅滅而告霍氏者皆封人為徐生上書曰臣聞客有過主人者【為于偽翻過古禾翻】見其竈直突傍有積薪客謂主人更為曲突【突竈突囱也更□衡翻】遠徙其薪不者且有火患主人嘿然不應俄而家果失火鄰里共救之幸而得息於是殺牛置酒謝其鄰人灼爛者在于上行【師古曰灼謂被燒炙者也行戶剛翻】餘各以功次坐而不録言曲突者人謂主人曰郷使聽客之言不費牛酒終亡火患【郷讀曰嚮亡古無字通】今論功而請賓曲突徙薪無恩澤燋頭爛額為上客邪主人乃寤而請之今茂陵徐福數上書言霍氏且有變宜防絶之郷使福說得行【數所角翻郷讀曰嚮】則國無裂土出爵之費臣無逆亂誅滅之敗往事既已而福獨不蒙其功唯陛下察之貴徙薪曲突之策使居焦髪灼爛之右上乃賜福帛十匹後以為郎帝初立謁見高廟【見賢遍翻】大將軍光驂乘【漢制大駕大將軍驂乘乘䋲證翻下同】上内嚴憚之若有芒刺在背後車騎將軍張安世代光驂乘天子從容肆體甚安近焉【從千容翻師古曰肆放也展也近其靳翻】及光身死而宗族竟誅故俗傳霍氏之禍萌于驂乘【師古曰萌謂始生也】後十二歲霍后復徙雲林館【復扶又翻下同】乃自殺<br />
<br />
  班固贊曰霍光受襁褓之託任漢室之寄匡國家安社稷擁昭立宣雖周公阿衡何以加此【師古曰阿衡伊尹官號也阿倚也衡平也言天子所倚羣下取平也】然光不學亡術【亡古無字】闇於大理隂妻邪謀【晉灼曰不揚其過也】立女為后湛溺盈溢之欲【湛讀曰沈】以增顛覆之禍死財三年宗族誅夷哀哉<br />
<br />
  臣光曰霍光之輔漢室可謂忠矣然卒不能庇其宗何也【卒子恤翻】夫威福者人君之器也人臣執之久而不歸鮮不及矣【鮮息淺翻】昔孝昭之明十四而知上官桀之詐固可以親政矣况孝宣十九即位聰明剛毅知民疾苦而光久專大柄不知避去多置私黨充塞朝廷【塞則息翻】使人主蓄憤於上吏民積怨於下切齒側目待時而發其得免于身幸矣况子孫以驕侈趣之哉【趣讀曰促】雖然曏使孝宣專以禄秩賞賜富其子孫使之食大縣奉朝請亦足以報盛德矣乃復任之以政授之以兵及事叢釁積更加裁奪遂至怨懼以生邪謀豈徒霍氏之自禍哉亦孝宣醖釀以成之也昔鬬椒作亂于楚【楚若敖之支庶為閼氏】莊王滅其族而赦箴尹克黄以為子文無後何以勸善【事見左傳宣四年子文鬭穀於菟也箴尹楚官名克黄子文之孫箴之金翻】夫以顯禹雲山之罪雖應夷滅而光之忠勲不可不祀遂使家無噍類【噍才肖翻】孝宣亦少恩哉<br />
<br />
  九月詔減天下鹽賈【賈讀曰價】又令郡國歲上繫囚以掠笞若瘐死者【上時掌翻蘇林曰瘐病也囚徒病律名為瘐如淳曰律囚以飢寒而死曰瘐師古曰瘐病是也此言囚或以掠笞及飢寒及疾病而死如說非也瘐音庾或作瘉其音亦同或讀作瘐誤據本紀瘐死上有飢寒二字掠音亮】所坐縣名爵里【漢書本紀作名縣爵里師古曰名者其人名也縣所屬縣也爵其身之官爵也里所居邑里也】丞相御史課殿最以聞【師古曰凡言殿最者殿後也課居後也最凡要之首也課居先也殿音丁見翻】 十二月清河王年坐内亂廢遷房陵【武帝元光三年立清河王義以嗣代孝王後年義之孫也】 是歲北海太守廬江朱邑以治行第一入為大司農【行下孟翻】勃海太守龔遂入為水衡都尉先是勃海左右郡歲饑【師古曰左右謂側近相次者先悉薦翻】盗賊並起二千石不能禽制上選能治者【治直之翻下同】丞相御史舉故昌邑郎中令龔遂上拜為勃海太守召見【見賢遍翻】問何以治勃海息其盗賊對曰海瀕遐遠【師古曰瀕涯也音頻又音賓】不霑聖化其民困於飢寒而吏不恤故使陛下赤子盗弄陛下之兵於潢池中耳【師古曰赤子猶言初生幼小之意也嬰孩初生體赤故曰赤子積水曰潢音黄】今欲使臣勝之邪將安之也【師古曰勝謂以威力克而殺之安謂以德化撫而安之】上曰選用賢良固欲安之也遂曰臣聞治亂民猶治亂䋲不可急也唯緩之然後可治【治直之翻】臣願丞相御史且無拘臣以文灋得一切便宜從事上許焉加賜黄金贈遣乘傳至勃海界【傳知戀翻】郡聞新太守至發兵以迎遂皆遣還移書敕屬縣悉罷逐捕盗賊吏諸持鉏鉤田器者【師古曰鉤鎌也】皆為良民吏毋得問持兵者乃為賊遂單車獨行至府盗賊聞遂教令即時解散棄其兵弩而持鉤鉏於是悉平民安土樂業【樂音洛】遂乃開倉廪假貧民【師古曰假謂給與】選用良吏尉安牧養焉遂見齊俗奢侈好末技【好呼到翻技渠綺翻】不田作乃躬率以儉約勸民務農桑各以口率種樹畜養【遂令民口種一樹榆百本䪥五十本葱一畦韭家二母彘五雞畜許六翻】民有帶持刀劍者使賣劍買牛賣刀買犢曰何為帶牛佩犢勞來循行【勞力到翻來力代翻行下孟翻】郡中皆有畜積獄訟止息【通鑑書龔遂自勃海入為列卿因叙其政績】烏孫公主女為龜兹王絳賓夫人絳賓上書言得尚漢外孫願與公主女俱入朝【朝直遥翻】<br />
<br />
  元康元年春正月龜兹王及其夫人來朝皆賜印綬夫人號稱公主賞賜甚厚初作杜陵徙丞相將軍列侯吏二千石訾百萬者杜陵【時以京兆杜縣東原上為初陵更名杜縣曰杜陵訾讀曰貲】 三月詔以鳳皇集泰山陳留甘露降未央宫赦天下 有司復言悼園宜稱尊號曰皇考【本始元年諡親曰悼置園邑復扶又翻】夏五月立皇考廟冬置建章衛尉【未央長樂建章甘泉皆有衛尉各掌其宫門衛屯兵】 趙廣漢好用世吏子孫新進年少者【師古曰言舊吏家子孫而其人後出求進又年少也好呼到翻少詩照翻】專厲彊壯蠭氣【師古曰蠭與鋒同言鋒銳之氣】見事風生無所回避【師古曰風生言其速疾不可當也回曲也】率多果敢之計莫為持難終以此敗廣漢以私怨論殺男子榮畜【初廣漢客私酤酒長安市丞相吏逐去客疑男子蘇賢言之以語廣漢案賢賢父上書訟罪廣漢坐貶秩疑其邑子榮畜教令以它法論殺畜榮姓也周有榮公子孫以為氏】人上書言之事下丞相御史按驗【下遐嫁翻下同】廣漢疑丞相夫人殺侍婢欲以此脅丞相丞相按之愈急廣漢乃將吏卒入丞相府召其夫人跪庭下受辭【師古曰受其對辭也】收奴婢十餘人去丞相上書自陳事下廷尉治實丞相自以過譴笞傅婢出至外第乃死不如廣漢言帝惡之【惡烏路翻】下廣漢廷尉獄吏民守闕號泣者數萬人【號戶刀翻】臣生無益縣官願代趙京兆死使牧養小民【漢書本傳臣生之上有或言二字】廣漢竟坐要斬【要與腰同 考異曰本紀元康二年冬廣漢有罪要斬百官表本始三年廣漢為京兆尹六年要斬元康元年守京兆尹彭城太守遺按廣漢傳司直蕭望之劾奏廣漢摧辱大臣望之自司直為平原太守元康元年自平原太守為少府然則廣漢死當在元康元年本紀誤也廣漢傳又云地節三年七月丞相婢自絞死蓋婢死已數年而廣漢追發其事也】廣漢為京兆尹亷明威制豪彊小民得職【師古曰得職各得其常所也】百姓追思歌之 是歲少府宋疇坐議鳳皇下彭城未至京師不足美貶為泗水太傅傅王【泗水綜】上選博士諫大夫通政事者補郡國守相以蕭望之為平原太守望之上疏曰陛下哀愍百姓恐德之不究【師古曰究竟也謂周徧于天下】悉出諫官以補郡吏朝無争臣則不知過【朝直遥翻争讀曰諍】所謂憂其末而忘其本者也上廼徵望之入守少府 東海太守河東尹翁歸以治郡高第入為右扶風翁歸為人公亷明察郡中吏民賢不肖及姦邪罪名盡知之縣縣各有記籍自聽其政【師古曰言决斷諸縣姦邪之事不委令長也】有急名則少緩之吏民小解輒披籍【服䖍曰披有罪者籍也師古曰解讀曰懈】取人必於秋冬課吏大會中及出行縣【師古曰於大會之中及行縣時則取罪人以警衆行下孟翻下改行以行同】不以無事時其有所取也以一警百吏民皆服恐懼改行自新其為扶風選用亷平疾姦吏以為右職【職居諸吏之上為右職】接待有禮好惡與同之【好呼到翻惡烏路翻】其負翁歸罰亦必行然温良謙退不以行能驕人故尤得名譽於朝廷 初烏孫公主少子萬年有寵於莎車王【班書莎車國王治莎車城去長安九千九百五十里莎蘇禾翻】莎車王死而無子時萬年在漢莎車國人計欲自託於漢又欲得烏孫心上書請萬年為莎車王漢許之遣使者奚充國送萬年【姓譜奚姓夏車正奚仲之後】萬年初立暴惡國人不說【說讀曰悅下同】上令羣臣舉可使西域者前將軍韓增舉上黨馮奉世以衛侯使持節送大宛諸國客至伊循城【衛侯衛士侯也伊循城在鄯善國漢於其中置屯田吏士使疏吏翻】會故莎車王弟呼屠徵與旁國共殺其王萬年及漢使者奚充國自立為王時匈奴又發兵攻車師城不能下而去莎車遣使揚言北道諸國已屬匈奴矣【揚言謂宣揚其言於外也】於是攻劫南道與歃盟畔漢從鄯善以西皆絶不通【歃色甲翻鄯上肩翻】都護鄭吉校尉司馬憙皆在北道諸國間奉世與其副嚴昌計以為不亟擊之則莎車日彊其埶難制必危西域遂以節諭告諸國王因發其兵南北道合萬五千人進擊莎車攻拔其城莎車王自殺傳其首詣長安更立它昆弟子為莎車王【更工衡翻】諸國悉平威振西域奉世乃罷兵以聞帝召見韓增曰賀將軍所舉得其人奉世遂西至大宛大宛聞其斬莎車王敬之異於它使得其名馬象龍而還【師古曰言馬形似龍者仲馮曰此馬名曰象龍也宛于元翻還仲宣翻又如字】上甚說【說讀曰悅】議封奉世丞相將軍皆以為可獨少府蕭望之以為奉世奉使有指【師古曰本為送諸國客】而擅制違命發諸國兵雖有功效不可以為後灋即封奉世開後奉使者利以奉世為比【師古曰比必寐翻余謂當音毘寐翻】爭逐發兵要功萬里之外【師古曰逐競也要一遥翻】為國家生事于夷狄漸不可長【為于偽翻長知兩翻】奉世不宜受封上善望之議以奉世為光禄大夫<br />
<br />
  二年春正月赦天下 上欲立皇后時館陶主母華倢伃【館陶縣屬魏郡華戶化翻倢伃音接予下同】及淮陽憲王母張倢伃楚孝王母衛倢伃皆愛幸【淮陽憲王欽楚孝王嚻】上欲立張倢伃為后久之懲艾霍氏欲害皇太子【艾音乂】乃更選後宫無子而謹慎者二月乙丑立長陵王倢伃為皇后令母養太子封其父奉光為卭成侯【恩澤侯表卬成侯食邑于濟隂卭渠容翻】后無竈希得進見【見賢遍翻】 五月詔曰獄者萬民之命能使生者不怨死者不恨則可謂文吏矣今則不然用法或持巧心析律貳端深淺不平【師古曰析分也謂分破律條妄生端緒以出入人罪】奏不如實上亦亡由知【師古曰上者天子自謂也亡古無字通】四方黎民將何仰哉二千石各察官屬勿用此人吏或擅興徭役飾厨傳稱過使客【韋昭曰厨謂飲食傳謂傳舍言修飾意氣以稱過使而已師古曰使人及賓客來者稱其意而遣之令過去也傳知戀翻稱音尺證翻過者過度之過也】越職踰灋以取名譽譬如踐薄冰以待白日豈不殆哉【師古曰殆危也】今天下頗被疾疫之災朕甚愍之其令郡國被災甚者【被皮義翻】毋出今年租賦又曰聞古天子之名難知而易諱也其更諱詢【易以䜴翻】<br />
<br />
  【更工衡翻】 匈奴大臣皆以為車師地肥美近匈奴【近其靳翻】使漢得之多田積穀必害人國不可不争由是數遣兵擊車師田者【數所角翻】鄭吉將渠犁田卒七千餘人救之為匈奴所圍吉上言車師去渠犁千餘里漢兵在渠犁者少埶不能相救願益田卒上與後將軍趙充國等議欲因匈奴衰弱出兵擊其右地使不得復擾西域【復扶又翻】魏相上書諫曰臣聞之救亂誅暴謂之義兵兵義者王敵加于己不得已而起者謂之應兵兵應者勝争恨小故不忍憤怒者謂之忿兵兵忿者敗利人土地貨寶者謂之貪兵兵貪者破恃國家之大矜民人之衆欲見威於敵者【見賢遍翻】謂之驕兵兵驕者滅此五者非但人事乃天道也間者匈奴嘗有善意所得漢民輒奉歸之未有犯於邉境雖爭屯田車師不足致意中【謂不足介意也】今聞諸將軍欲興兵入其地【丞相不預中朝之議故言聞諸將軍大將軍車騎將軍前後左右將軍皆中朝官】臣愚不知此兵何名者也今邉郡困乏父子共犬羊裘食草莱之實常恐不能自存難以動兵【師古曰不可以兵事動之也】軍旅之後必有凶年【師古曰此引老子道經之語】言民以其愁苦之氣傷隂陽之和也出兵雖勝猶有後憂恐災害之變因此以生今郡國守相多不實選【師古曰言不得其人】風俗尤薄水旱不時按今年子弟殺父兄妻殺夫者凡二百二十二人臣愚以為此非小變也今左右不憂此【師古曰左右謂近臣在天子左右者】乃欲發兵報纎介之忿於遠夷殆孔子所謂吾恐季孫之憂不在顓臾而在蕭墻之内也【師古曰論語季氏將伐顓臾孔子謂冉有季路曰云云故相引之顓臾魯附庸國蕭牆者屏牆也】上從相言止遣長羅侯常惠將張掖酒泉騎往車師迎鄭吉及其吏士還渠犂召故車師太子軍宿在焉耆者立以為王盡徙車師國民令居渠犂遂以車師故地與匈奴以鄭吉為衛司馬使護鄯善以西南道 魏相好觀漢故事及便宜章奏【師古曰既觀國家故事又觀前人所奏便宜之章也好呼到翻】數條漢興已來國家便宜行事及賢臣賈誼鼂錯董仲舒等所言奏請施行之【數所角翻】相勑掾史按事郡國及休告從家還至府輒白四方異聞或有逆賊風雨災變郡不上相輒奏言之【上時掌翻】與御史大夫丙吉同心輔政上皆重之丙吉為人深厚不伐善自曾孫遭遇【師古曰遭遇謂升大位也】吉絶口不道前恩故朝廷莫能明其功也會掖庭宫婢則令民夫上書自陳嘗有阿保之功【師古曰謂未為宫婢時有舊夫見在俗間者】章下掖庭令考問【下遐嫁翻】則辭引使者丙吉知狀掖庭令將則詣御史府以視吉【師古曰視讀曰示】吉識謂則曰汝嘗坐養皇曾孫不謹督笞汝【師古曰督謂視察之】汝安得有功獨渭城胡組淮陽郭徵卿有恩耳分别奏組等共養勞苦狀【别彼列翻共居用翻養弋亮翻】詔吉求組徵卿已死有子孫皆受厚賞詔免則為庶人賜錢十萬上親見問然後知吉有舊恩而終不言上大賢之 帝以蕭望之經明持重議論有餘材任宰相【師古曰任堪也】欲詳試其政事復以為左馮翊【宋白曰馬輔也翊佐也義取輔佐京師復扶又翻下同】望之從少府出為左遷【少府正九卿三輔禄秩視九卿故為左遷】恐有不合意即移病【師古曰移病謂移書言病一曰以病而移居余謂前說是】上聞之使侍中成都侯金安上諭意曰所用皆更治民以考功【功臣表及霍光傳皆作都成侯此承望之本傳之誤師古曰更猶經歷也更工衡翻治直之翻】君前為平原太守日淺故復試之於三輔非有所聞也望之即起視事【師古曰所聞謂聞其短失】 初掖庭令張賀數為弟車騎將軍安世稱皇曾孫之材美及徵怪【師古曰徵證也數所角翻為于偽翻】安世輒絶止以為少主在上【少詩照翻】不宜稱述曾孫及帝即位而賀已死上謂安世曰掖庭令平生稱我將軍止之是也上追思賀恩欲封其冢為恩德侯【師古曰身死追封故曰封冢也】置守冢二百家賀有子蚤死子安世小男彭祖【師古曰子者言養以為子也】彭祖又小與上同席研書指欲封之先賜爵關内侯安世深辭賀封又求損守冢戶數稍減至三十戶上曰吾自為掖庭令非為將軍也【為于偽翻】安世乃止不敢復言 上心忌故昌邑王賀賜山陽太守張敞璽書令謹備盗賊察往來過客【昌邑王廢歸昌邑國除為山陽郡故令太守謹察之】毋下所賜書【師古曰密令警察不欲宣露也下遐嫁翻】敞於是條奏賀居處【處昌呂翻】著其廢亡之效【師古曰著明也】曰故昌邑王為人青黑色小目鼻末銳卑少須眉身體長大疾痿行步不便【少詩沼翻師古曰痿風痺疾也音人佳翻】臣敞嘗與之言欲動觀其意即以惡鳥感之曰昌邑多梟故王應曰然前賀西至長安殊無梟復來東至濟陽乃復聞梟聲【梟不孝鳥一名流離詩注少好而長醜爾雅作鶹草木疏曰梟也大則食其母劉子曰炎州冇鳥曰梟傴伏其子百日而長羽翼既成食母而飛盖稍長從母索食母無以應從是而死漢使東郡送梟五月五日作梟羮以賜百官音堅堯翻又于驕翻乃復扶又翻】察故王衣服言語跪起清狂不惠【蘇林曰凡狂者隂陽脉盡濁今此人不狂似狂者故言清狂也或曰色理清徐而心不慧曰清狂清狂  如今白癡者也韓子曰心不能審得失之地則謂之狂】臣敞前言哀王歌舞者張修等十人無子留守哀王園【賀父髆謚哀王】請罷歸故王聞之曰中人守園疾者當勿治【治直之翻】相殺傷者當勿灋欲令亟死太守柰何而欲罷之其天資喜由亂亡終不見仁義如此【師古曰喜好也由從也喜許吏翻】上乃知賀不足忌也<br />
<br />
  三年春三月詔封故昌邑王賀為海昬侯【海昬縣屬豫章郡後漢分立建昌縣宋白曰今建昌縣舊海昬縣也宋元嘉二年廢海昏縣移建昌居焉 考異曰王子侯表賀以四月壬子封宣紀賀封在丙吉之前按是歲四月癸亥朔無壬子表誤】 乙未詔曰朕微眇時御史大夫丙吉中郎將史曾史玄長樂衛尉許舜侍中光禄大夫許延夀皆與朕有舊恩及故掖庭令張賀輔導朕躬修文學經術恩惠卓異厥功茂焉詩不云乎無德不報【師古曰大雅抑之詩】封賀所子弟子侍中中郎將彭祖為陽都侯追賜賀諡曰陽都哀侯吉為博陽侯曾為將陵侯玄為平臺侯舜為博望侯延夀為樂成侯【地理志城陽國有陽都縣恩澤侯表博陽侯食邑於汝南郡之南頓縣平臺屬常山郡博望屬南陽郡樂成侯食邑於南陽之平氏】賀有孤孫霸年七歲拜為散騎中郎將賜爵關内侯【散悉亶翻騎奇寄翻】故人下至郡邸獄復作【師古曰復扶目翻】嘗有阿保之功者皆受官禄田宅財物各以恩深淺報之吉臨當封病上憂其不起將使人就加印紼而封之及其生存也【應劭曰吉時疾不能起欲如君視疾加朝服拖紳就封之也師古曰紼繫印之組也音弗】太子太傅夏侯勝曰此未死也臣聞有隂德者必饗其樂以及子孫【樂音洛】今吉未獲報而疾甚非其死疾也後病果愈張安世自以父子封侯在位太盛乃辭禄詔都内别藏張氏無名錢以百萬數【文穎曰都内主藏官也張晏曰安世以還官官不簿也百官表大司農屬官有都内令丞】安世謹慎周密每定大政已决輒移病出聞有詔令乃驚使吏之丞相府問焉自朝廷大臣莫知其與議也【與讀曰豫】嘗有所薦其人來謝安世大恨以為舉賢逹能豈有私謝邪絶弗復為通【師古曰有欲謝者皆不通也一曰告此人而絶之更不與相見也復扶又翻下同為于偽翻予謂絶弗為通者安世敕其閽人之辭也】有郎功高不調【師古曰調選也音徒釣翻】自言安世安世應曰君之功高明主所知人臣執事何長短而自言乎絶不許已而郎果遷【師古曰安世外陽距之而實令其遷】安世自見父子尊顯懷不自安為子延夀求出補吏【為于偽翻下同】上以為北地太守歲餘上閔安世年老復徵延夀為左曹太僕【以太僕而加左曹官也】 夏四月丙子立皇子欽為淮陽王皇太子年十二通論語孝經太傅疏廣謂少傅受曰吾聞知足不辱知止不殆【師古曰此老子之言而廣引之】今仕宦至二千石官成名立如此不去懼有後悔即日父子俱移病上疏乞骸骨上皆許之加賜黄金二十斤皇太子贈以五十斤公卿故人設祖道供張東都門外【供居共翻張竹亮翻】送者車數百兩【兩音亮】道路觀者皆曰賢哉二大夫或歎息為之下泣廣受歸郷里【廣受東海蘭陵人】日令其家賣金共具【師古曰日日設之也共讀曰供下同】請族人故舊賓客與相娛樂【樂音洛下同】或勸廣以其金為子孫頗立產業者【為于偽翻】廣曰吾豈老誖不念子孫哉【師古曰誖惑也音布内翻】顧自有舊田廬【師古曰顧思念也】令子孫勤力其中足以共衣食與凡人齊今復增益之以為贏餘但教子孫怠墮耳賢而多財則損其志愚而多財則益其過且夫富者衆之怨也吾既無以教化子孫不欲益其過而生怨又此金者聖主所以惠養老臣也故樂與郷黨宗族共饗其賜以盡吾餘日不亦可乎於是族人悅服潁川太守黄霸使郵亭郷官皆畜雞豚【師古曰郵亭書舍謂傳送文書所止處亦如今之驛館矣郷官者郷所治處也沈約曰漢制五家為伍伍長主之二五為什什長主之十什為里里魁主之十里為亭亭長主之十亭為郷有郷佐三老有秩嗇夫游徼各一人郷佐有秩主賦稅三老主教化嗇夫主争訟游徼主姦非畜吁玉翻下同】以贍鰥寡窮者然後為條教置父老師帥伍長【帥所類翻長知兩翻下同】班行之於民閒勸以為善防姦之意及務耕桑節用殖財種樹畜養去浮淫之費【去羌呂翻下同】其治米鹽靡密【師古曰米鹽言雜而且細】初若煩碎然霸精力能推行之吏民見者語次尋繹【師古曰繹謂抽引而出也】問它隂伏以相參考聰明識事吏民不知所出【師古曰不知其用何術也】咸稱神明豪釐不敢有所欺姦人去入它郡盗賊日少霸力行教化而後誅罰【師古曰力猶勤也言先以德教化於下若冇弗從然後用刑罰也後戶遘翻】務在成就全安長吏【師古曰不欲易代及損傷之也】許丞老病聾督郵白欲逐之【如淳曰許縣丞據地理志許縣屬潁川郡郡有部督郵分部屬縣】霸曰許丞亷吏雖老尚能拜起送迎止頗重聽何傷且善助之毋失賢者意或問其故霸曰數易長吏【數所角翻】送故迎新之費及姦吏因緣絶簿書盗財物【師古曰緣因也因交代之際而弃匿簿書以盗官物也】公私費耗甚多皆當出于民所易新吏又未必賢或不如其故徒相益為亂凡治道去其泰甚者耳霸以外寛内明得吏民心戶口歲增治為天下第一【治直之翻】徵守京兆尹頃之坐灋連貶秩有詔復歸潁川為太守以八百石居【太守秩二千石連貶故以八百石居】<br />
<br />
  四年春正月詔年八十以上非誣告殺傷人它皆勿坐【師古曰誣告人及殺傷人皆如舊法其餘則不論】 右扶風尹翁歸卒家無餘財秋八月詔曰翁歸亷平郷正【郷讀曰嚮】治民異等【治直之翻】其賜翁歸子黄金百斤以奉祭祀 上令有司求高祖功臣子孫失侯者得槐里公乘周廣漢等百三十六人皆賜黄金二十斤復其家令奉祭祀世世勿絶【公乘爵第八復方目翻 考異曰宣紀元康元年五月復高皇帝功臣絳侯周勃等百三十六人家子孫四年又賜功臣適後黄金人二十斤按功臣表詔復家者皆云元康四年其數非一不容盡誤盖紀誤耳】 丙寅富平敬侯張安世薨初扶陽節侯韋賢薨【恩澤侯表扶陽侯食邑於沛郡蕭縣諡法好亷自克曰節】長子弘有罪繫獄【弘為太常坐宗廟事繫獄】家人矯賢令以次子大河都尉玄成為後【服䖍曰今東平郡也本為濟東國國除為大河郡師古曰矯託也】玄成深知其非賢雅意即陽為病狂卧便利妄笑語昬亂【師古曰便利大小便音毗連翻】既葬當襲爵以狂不應召大鴻臚奏狀章下丞相御史案驗【下遐嫁翻】案事丞相史廼與玄成書【師古曰即案驗玄成事者】曰古之辭讓必有文義可觀故能垂榮於後今子獨壞容貌蒙恥辱為狂癡光曜晻而不宣【壞音怪晻讀與暗同】微哉子之所託名也【李奇曰名聲名也】僕素愚陋過為丞相執事【師古曰過猶謬也】願少聞風聲不然恐子傷高而僕為小人也玄成友人侍郎章【侍郎名章史逸其姓】亦上疏言聖王貴以禮讓為國宜優養玄成勿枉其志【師古曰枉屈也】使得自安衡門之下【師古曰衡門謂横一木於門上貧者之所居也】而丞相御史遂以玄成實不病劾奏之【劾戶槩翻下同】有詔勿劾引拜玄成不得已受爵帝高其節以玄成為河南太守車師王烏貴之走烏孫也烏孫留不遣漢遣使責烏孫烏孫送烏貴詣闕 初武帝開河西四郡隔絶羌與匈奴相通之路斥逐諸羌不使居湟中地【河西武威張掖酒泉敦煌四郡本匈奴昆邪休屠王地武帝開之置郡縣羌與匈奴隔遠不復得通湟中湟水左右地也其地肥美故斥逐諸羌不使居之水經注金城郡臨羌縣西北至塞外有西王母石室僊海鹽池北則湟水所出東流逕湟中城北故小月氏之地也又東逕臨羌破羌允街枝陽金城而合于大河】及帝即位光禄大夫義渠安國使行諸羌【戰國時西戎有義渠君為秦所滅子孫以國為姓】先零豪言願時度湟水北逐民所不田處畜牧【師古曰湟水出金城臨羌塞外東入河湟水之北是漢地仲馮口湟北非謂漢地也羌意欲稍北遷與匈奴合而為寇安國不知其情故受其詞詳下文可見余謂羌依南山度湟水而北固欲與匈奴合而湟北則漢地所以隔絶羌與匈奴通之路正在此零音憐】安國以聞後將軍趙充國劾安國奉使不敬【劾戶槩翻】是後羌人旁緣前言抵冒度湟水【師古曰旁依也抵冒犯突而前也旁音步浪翻冒音莫北翻】郡縣不能禁既而先零與諸羌種豪二百餘人解仇交質盟詛【師古曰羌人無大君長而諸種豪遞相殺伐故每有仇讎往來相報今解仇交質者自相親結欲入漢為寇也零音憐種章勇翻詛莊助翻】上聞之以問趙充國對曰羌人所以易制者【易以䜴翻】以其種自有豪數相攻擊埶不壹也【數所角翻下同】往三十餘歲西羌反時亦先解仇合約攻令居與漢相距五六年乃定【武帝元鼎五年西羌反攻故安枹罕次年即平至是五十一年師古曰合約共為要契也令音鈴】匈奴數誘羌人欲與之共擊張掖酒泉地使羌居之【數所角翻誘音酉】間者匈奴困於西方【謂本始三年為烏孫所破】疑其更遣使至羌中與相結臣恐羌變未止此且復結聨他種【復扶又翻】宜及未然為之備【師古曰未然者其計未成】後月餘羌侯狼何果遣使至匈奴藉兵【師古曰藉借也據充國傳狼何小月氏種】欲擊鄯善燉煌以絶漢道【鄯上扇翻燉音屯】充國以為狼何埶不能獨造此計疑匈奴使已至羌中先零䍐开乃解仇作約【蘇林曰䍐开在金城南師古曰䍐开羌之别種也此下言遣开豪雕庫宣天子至德䍐开之屬皆聞知明詔其下又云河南大开小开則䍐羌开羌姓族殊矣开音口堅翻而地理志天水有䍐开縣盖以此二種羌來降處之此地因以名縣也而今之羌姓有䍐开者總是䍐开之類合而言之因為姓耳變开為井字之訛也零音憐䍐即罕字】到秋馬肥變必起矣宜遣使者行邉兵【行下孟翻下同】豫為備勑視諸羌母令解仇【師古曰視讀曰示示語之也】以發覺其謀於是兩府復白遣義渠安國行視諸羌【兩府丞相御史府也此視觀也】分别善惡【别彼列翻】是時比年豐稔穀石五錢【比毗至翻】<br />
<br />
  資治通鑑卷二十五<br />
<br />
<史部,編年類,資治通鑑>  <br>
   </div> 

<script src="/search/ajaxskft.js"> </script>
 <div class="clear"></div>
<br>
<br>
 <!-- a.d-->

 <!--
<div class="info_share">
</div> 
-->
 <!--info_share--></div>   <!-- end info_content-->
  </div> <!-- end l-->

<div class="r">   <!--r-->



<div class="sidebar"  style="margin-bottom:2px;">

 
<div class="sidebar_title">工具类大全</div>
<div class="sidebar_info">
<strong><a href="http://www.guoxuedashi.com/lsditu/" target="_blank">历史地图</a></strong>  
<a href="http://www.880114.com/" target="_blank">英语宝典</a>  
<a href="http://www.guoxuedashi.com/13jing/" target="_blank">十三经检索</a> 
<br><strong><a href="http://www.guoxuedashi.com/gjtsjc/" target="_blank">古今图书集成</a></strong> 
<a href="http://www.guoxuedashi.com/duilian/" target="_blank">对联大全</a> <strong><a href="http://www.guoxuedashi.com/xiangxingzi/" target="_blank">象形文字典</a></strong> 

<br><a href="http://www.guoxuedashi.com/zixing/yanbian/">字形演变</a>  <strong><a href="http://www.guoxuemi.com/hafo/" target="_blank">哈佛燕京中文善本特藏</a></strong>
<br><strong><a href="http://www.guoxuedashi.com/csfz/" target="_blank">丛书&方志检索器</a></strong> <a href="http://www.guoxuedashi.com/yqjyy/" target="_blank">一切经音义</a>  

<br><strong><a href="http://www.guoxuedashi.com/jiapu/" target="_blank">家谱族谱查询</a></strong>  <strong><a href="http://shufa.guoxuedashi.com/sfzitie/" target="_blank">书法字帖欣赏</a></strong> 
<br>

</div>
</div>


<div class="sidebar" style="margin-bottom:0px;">

<font style="font-size:22px;line-height:32px">QQ交流群9:489193090</font>


<div class="sidebar_title">手机APP 扫描或点击</div>
<div class="sidebar_info">
<table>
<tr>
	<td width=160><a href="http://m.guoxuedashi.com/app/" target="_blank"><img src="/img/gxds-sj.png" width="140"  border="0" alt="国学大师手机版"></a></td>
	<td>
<a href="http://www.guoxuedashi.com/download/" target="_blank">app软件下载专区</a><br>
<a href="http://www.guoxuedashi.com/download/gxds.php" target="_blank">《国学大师》下载</a><br>
<a href="http://www.guoxuedashi.com/download/kxzd.php" target="_blank">《汉字宝典》下载</a><br>
<a href="http://www.guoxuedashi.com/download/scqbd.php" target="_blank">《诗词曲宝典》下载</a><br>
<a href="http://www.guoxuedashi.com/SiKuQuanShu/skqs.php" target="_blank">《四库全书》下载</a><br>
</td>
</tr>
</table>

</div>
</div>


<div class="sidebar2">
<center>


</center>
</div>

<div class="sidebar"  style="margin-bottom:2px;">
<div class="sidebar_title">网站使用教程</div>
<div class="sidebar_info">
<a href="http://www.guoxuedashi.com/help/gjsearch.php" target="_blank">如何在国学大师网下载古籍?</a><br>
<a href="http://www.guoxuedashi.com/zidian/bujian/bjjc.php" target="_blank">如何使用部件查字法快速查字?</a><br>
<a href="http://www.guoxuedashi.com/search/sjc.php" target="_blank">如何在指定的书籍中全文检索?</a><br>
<a href="http://www.guoxuedashi.com/search/skjc.php" target="_blank">如何找到一句话在《四库全书》哪一页?</a><br>
</div>
</div>


<div class="sidebar">
<div class="sidebar_title">热门书籍</div>
<div class="sidebar_info">
<a href="/so.php?sokey=%E8%B5%84%E6%B2%BB%E9%80%9A%E9%89%B4&kt=1">资治通鉴</a> <a href="/24shi/"><strong>二十四史</strong></a>&nbsp; <a href="/a2694/">野史</a>&nbsp; <a href="/SiKuQuanShu/"><strong>四库全书</strong></a>&nbsp;<a href="http://www.guoxuedashi.com/SiKuQuanShu/fanti/">繁体</a>
<br><a href="/so.php?sokey=%E7%BA%A2%E6%A5%BC%E6%A2%A6&kt=1">红楼梦</a> <a href="/a/1858x/">三国演义</a> <a href="/a/1038k/">水浒传</a> <a href="/a/1046t/">西游记</a> <a href="/a/1914o/">封神演义</a>
<br>
<a href="http://www.guoxuedashi.com/so.php?sokeygx=%E4%B8%87%E6%9C%89%E6%96%87%E5%BA%93&submit=&kt=1">万有文库</a> <a href="/a/780t/">古文观止</a> <a href="/a/1024l/">文心雕龙</a> <a href="/a/1704n/">全唐诗</a> <a href="/a/1705h/">全宋词</a>
<br><a href="http://www.guoxuedashi.com/so.php?sokeygx=%E7%99%BE%E8%A1%B2%E6%9C%AC%E4%BA%8C%E5%8D%81%E5%9B%9B%E5%8F%B2&submit=&kt=1"><strong>百衲本二十四史</strong></a>  <a href="http://www.guoxuedashi.com/so.php?sokeygx=%E5%8F%A4%E4%BB%8A%E5%9B%BE%E4%B9%A6%E9%9B%86%E6%88%90&submit=&kt=1"><strong>古今图书集成</strong></a>
<br>

<a href="http://www.guoxuedashi.com/so.php?sokeygx=%E4%B8%9B%E4%B9%A6%E9%9B%86%E6%88%90&submit=&kt=1">丛书集成</a> 
<a href="http://www.guoxuedashi.com/so.php?sokeygx=%E5%9B%9B%E9%83%A8%E4%B8%9B%E5%88%8A&submit=&kt=1"><strong>四部丛刊</strong></a>  
<a href="http://www.guoxuedashi.com/so.php?sokeygx=%E8%AF%B4%E6%96%87%E8%A7%A3%E5%AD%97&submit=&kt=1">說文解字</a> <a href="http://www.guoxuedashi.com/so.php?sokeygx=%E5%85%A8%E4%B8%8A%E5%8F%A4&submit=&kt=1">三国六朝文</a>
<br><a href="http://www.guoxuedashi.com/so.php?sokeytm=%E6%97%A5%E6%9C%AC%E5%86%85%E9%98%81%E6%96%87%E5%BA%93&submit=&kt=1"><strong>日本内阁文库</strong></a> <a href="http://www.guoxuedashi.com/so.php?sokeytm=%E5%9B%BD%E5%9B%BE%E6%96%B9%E5%BF%97%E5%90%88%E9%9B%86&ka=100&submit=">国图方志合集</a> <a href="http://www.guoxuedashi.com/so.php?sokeytm=%E5%90%84%E5%9C%B0%E6%96%B9%E5%BF%97&submit=&kt=1"><strong>各地方志</strong></a>

</div>
</div>


<div class="sidebar2">
<center>

</center>
</div>
<div class="sidebar greenbar">
<div class="sidebar_title green">四库全书</div>
<div class="sidebar_info">

《四库全书》是中国古代最大的丛书,编撰于乾隆年间,由纪昀等360多位高官、学者编撰,3800多人抄写,费时十三年编成。丛书分经、史、子、集四部,故名四库。共有3500多种书,7.9万卷,3.6万册,约8亿字,基本上囊括了古代所有图书,故称“全书”。<a href="http://www.guoxuedashi.com/SiKuQuanShu/">详细>>
</a>

</div> 
</div>

</div>  <!--end r-->

</div>
<!-- 内容区END --> 

<!-- 页脚开始 -->
<div class="shh">

</div>

<div class="w1180" style="margin-top:8px;">
<center><script src="http://www.guoxuedashi.com/img/plus.php?id=3"></script></center>
</div>
<div class="w1180 foot">
<a href="/b/thanks.php">特别致谢</a> | <a href="javascript:window.external.AddFavorite(document.location.href,document.title);">收藏本站</a> | <a href="#">欢迎投稿</a> | <a href="http://www.guoxuedashi.com/forum/">意见建议</a> | <a href="http://www.guoxuemi.com/">国学迷</a> | <a href="http://www.shuowen.net/">说文网</a><script language="javascript" type="text/javascript" src="https://js.users.51.la/17753172.js"></script><br />
  Copyright &copy; 国学大师 古典图书集成 All Rights Reserved.<br>
  
  <span style="font-size:14px">免责声明:本站非营利性站点,以方便网友为主,仅供学习研究。<br>内容由热心网友提供和网上收集,不保留版权。若侵犯了您的权益,来信即刪。scp168@qq.com</span>
  <br />
ICP证:<a href="http://www.beian.miit.gov.cn/" target="_blank">鲁ICP备19060063号</a></div>
<!-- 页脚END --> 
<script src="http://www.guoxuedashi.com/img/plus.php?id=22"></script>
<script src="http://www.guoxuedashi.com/img/tongji.js"></script>

</body>
</html>
