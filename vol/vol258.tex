<!DOCTYPE html PUBLIC "-//W3C//DTD XHTML 1.0 Transitional//EN" "http://www.w3.org/TR/xhtml1/DTD/xhtml1-transitional.dtd">
<html xmlns="http://www.w3.org/1999/xhtml">
<head>
<meta http-equiv="Content-Type" content="text/html; charset=utf-8" />
<meta http-equiv="X-UA-Compatible" content="IE=Edge,chrome=1">
<title>資治通鑒_259-資治通鑑卷二百五十八_259-資治通鑑卷二百五十八</title>
<meta name="Keywords" content="資治通鑒_259-資治通鑑卷二百五十八_259-資治通鑑卷二百五十八">
<meta name="Description" content="資治通鑒_259-資治通鑑卷二百五十八_259-資治通鑑卷二百五十八">
<meta http-equiv="Cache-Control" content="no-transform" />
<meta http-equiv="Cache-Control" content="no-siteapp" />
<link href="/img/style.css" rel="stylesheet" type="text/css" />
<script src="/img/m.js?2020"></script> 
</head>
<body>
 <div class="ClassNavi">
<a  href="/24shi/">二十四史</a> | <a href="/SiKuQuanShu/">四库全书</a> | <a href="http://www.guoxuedashi.com/gjtsjc/"><font  color="#FF0000">古今图书集成</font></a> | <a href="/renwu/">历史人物</a> | <a href="/ShuoWenJieZi/"><font  color="#FF0000">说文解字</a></font> | <a href="/chengyu/">成语词典</a> | <a  target="_blank"  href="http://www.guoxuedashi.com/jgwhj/"><font  color="#FF0000">甲骨文合集</font></a> | <a href="/yzjwjc/"><font  color="#FF0000">殷周金文集成</font></a> | <a href="/xiangxingzi/"><font color="#0000FF">象形字典</font></a> | <a href="/13jing/"><font  color="#FF0000">十三经索引</font></a> | <a href="/zixing/"><font  color="#FF0000">字体转换器</font></a> | <a href="/zidian/xz/"><font color="#0000FF">篆书识别</font></a> | <a href="/jinfanyi/">近义反义词</a> | <a href="/duilian/">对联大全</a> | <a href="/jiapu/"><font  color="#0000FF">家谱族谱查询</font></a> | <a href="http://www.guoxuemi.com/hafo/" target="_blank" ><font color="#FF0000">哈佛古籍</font></a> 
</div>

 <!-- 头部导航开始 -->
<div class="w1180 head clearfix">
  <div class="head_logo l"><a title="国学大师官网" href="http://www.guoxuedashi.com" target="_blank"></a></div>
  <div class="head_sr l">
  <div id="head1">
  
  <a href="http://www.guoxuedashi.com/zidian/bujian/" target="_blank" ><img src="http://www.guoxuedashi.com/img/top1.gif" width="88" height="60" border="0" title="部件查字,支持20万汉字"></a>


<a href="http://www.guoxuedashi.com/help/yingpan.php" target="_blank"><img src="http://www.guoxuedashi.com/img/top230.gif" width="600" height="62" border="0" ></a>


  </div>
  <div id="head3"><a href="javascript:" onClick="javascript:window.external.AddFavorite(window.location.href,document.title);">添加收藏</a>
  <br><a href="/help/setie.php">搜索引擎</a>
  <br><a href="/help/zanzhu.php">赞助本站</a></div>
  <div id="head2">
 <a href="http://www.guoxuemi.com/" target="_blank"><img src="http://www.guoxuedashi.com/img/guoxuemi.gif" width="95" height="62" border="0" style="margin-left:2px;" title="国学迷"></a>
  

  </div>
</div>
  <div class="clear"></div>
  <div class="head_nav">
  <p><a href="/">首页</a> | <a href="/ShuKu/">国学书库</a> | <a href="/guji/">影印古籍</a> | <a href="/shici/">诗词宝典</a> | <a   href="/SiKuQuanShu/gxjx.php">精选</a> <b>|</b> <a href="/zidian/">汉语字典</a> | <a href="/hydcd/">汉语词典</a> | <a href="http://www.guoxuedashi.com/zidian/bujian/"><font  color="#CC0066">部件查字</font></a> | <a href="http://www.sfds.cn/"><font  color="#CC0066">书法大师</font></a> | <a href="/jgwhj/">甲骨文</a> <b>|</b> <a href="/b/4/"><font  color="#CC0066">解密</font></a> | <a href="/renwu/">历史人物</a> | <a href="/diangu/">历史典故</a> | <a href="/xingshi/">姓氏</a> | <a href="/minzu/">民族</a> <b>|</b> <a href="/mz/"><font  color="#CC0066">世界名著</font></a> | <a href="/download/">软件下载</a>
</p>
<p><a href="/b/"><font  color="#CC0066">历史</font></a> | <a href="http://skqs.guoxuedashi.com/" target="_blank">四库全书</a> |  <a href="http://www.guoxuedashi.com/search/" target="_blank"><font  color="#CC0066">全文检索</font></a> | <a href="http://www.guoxuedashi.com/shumu/">古籍书目</a> | <a   href="/24shi/">正史</a> <b>|</b> <a href="/chengyu/">成语词典</a> | <a href="/kangxi/" title="康熙字典">康熙字典</a> | <a href="/ShuoWenJieZi/">说文解字</a> | <a href="/zixing/yanbian/">字形演变</a> | <a href="/yzjwjc/">金 文</a> <b>|</b>  <a href="/shijian/nian-hao/">年号</a> | <a href="/diming/">历史地名</a> | <a href="/shijian/">历史事件</a> | <a href="/guanzhi/">官职</a> | <a href="/lishi/">知识</a> <b>|</b> <a href="/zhongyi/">中医中药</a> | <a href="http://www.guoxuedashi.com/forum/">留言反馈</a>
</p>
  </div>
</div>
<!-- 头部导航END --> 
<!-- 内容区开始 --> 
<div class="w1180 clearfix">
  <div class="info l">
   
<div class="clearfix" style="background:#f5faff;">
<script src='http://www.guoxuedashi.com/img/headersou.js'></script>

</div>
  <div class="info_tree"><a href="http://www.guoxuedashi.com">首页</a> > <a href="/SiKuQuanShu/fanti/">四库全书</a>
 > <h1>资治通鉴</h1> <!--         下载:【右键另存为】即可 --></div>
  <div class="info_content zj clearfix">
  
<div class="info_txt clearfix" id="show">
<center style="font-size:24px;">259-資治通鑑卷二百五十八</center>
    資治通鑑卷二百五十八 寀 司馬光 撰<br />
<br />
  胡三省 音註<br />
<br />
  唐紀七十四【起屠維作噩盡重光大淵獻几三年】<br />
<br />
  昭宗聖穆景文孝皇帝上之上【諱傑懿宗第七子及即位改名敏又改名曄】<br />
<br />
  龍紀元年春正月癸巳朔赦天下改元 【考異曰唐年補録曰正月癸巳改文德二年為龍紀元年百寮上帝徽號曰聖文睿德光武弘孝皇帝新舊紀實録明年正月乃上尊號補録誤也舊紀又云以劒南西川節度兩川招撫制置使韋昭度為東都留守按昭度大順二年乃為留守舊紀誤也今皆從實録】 以翰林學士承旨兵部侍郎劉崇望同平章事 汴將龎師古拔宿遷軍於呂梁【九域志徐州彭城縣有呂梁洪鎮】時溥逆戰大敗還保彭城 壬子蔡將郭璠殺申叢送秦宗權於汴【璠孚袁翻 考異曰實録申叢裴涉欲復立宗權為帥汴將李璠知之斬叢涉以宗權送汴州薛居正五代史初申叢縳宗權折足而囚之雖納欵於太祖欲自獻於長安以邀旄鉞及姦謀不就乃欲復奉宗權以接取其柄為其將郭璠所殺縶宗權送於太祖即以璠為留後太祖遣都統判官韋震奏事且疏時溥之罪願委討伐仍請降滄兖二帥之命按全忠若自求兼領滄兖二鎮則明年朝廷命兼領滑州全忠猶辭不受今豈敢遽求滄兖邪若為滄兖二帥求之則兖帥朱瑾乃其仇讐也當時不知全忠欲以何人為滄帥諸書皆無其名薛史實録皆云申叢欲復立宗權按叢折宗權足而囚之豈有復奉為帥之理蓋郭璠欲奪其功誣之云爾新舊紀五代紀傳皆云郭璠殺申叢而實錄云李璠誤也李璠乃檻送宗權者】告朱全忠云叢謀復立宗權全忠以璠為淮西留後【朱全忠又并淮西以連襄鄧其勢愈盛矣】 戊申王建大破山行章於新繁殺獲近萬人行章僅以身免楊晟懼徙屯三交行章屯濛陽與建相持【儀鳳二年分九隴雒什邡三縣置濛陽縣屬彭州九域志在州東三十一里宋白曰縣在濛江之北故曰濛陽】 二月朱全忠送秦宗權至京師斬於獨柳【考異曰舊紀汴州行軍司馬李璠監送秦宗權并妻趙氏以獻斬於獨柳實録三月全忠獻宗權斬於獨柳】<br />
<br />
  【新紀二月戊辰朱全忠俘宗權以獻己丑宗權伏誅按宗權正月離汴不應三月始至長安戊辰獻俘不應至己丑始伏誅故但云二月】京兆尹孫揆監刑【監古銜翻】宗權於檻車中引首謂揆曰尚書察宗權豈反者邪但輸忠不效耳觀者皆笑揆逖之族孫也【孫逖仕至刑部侍郎揆五世從孫也】 三月加朱全忠兼中書令進爵東平郡王 【考異曰舊紀在四月封東平郡王薛居正五代史在三月亦云封東平今從實録止加中書令 據考異則進爵東平郡王六字合汰然按舊書帝紀光啟元年封全忠沛郡王此時雖未進爵東平固已封王矣】全忠既克蔡州軍勢益盛加奉國節度使趙德諲中書令【僖宗中和二年以蔡州為奉國軍命秦宗權為節度使文德元年以襄州為忠義軍命趙德諲為節度使宗權既亡未嘗以奉國節授人趙德諲亦未嘗兼奉國節當改奉國為忠義】加蔡州節度使趙犨同平章事充忠武節度使以陳州為理所【忠武本治許州趙犨陳人也又守陳有功因徒治所於陳犨昌牛翻】會犨有疾悉以軍府事授其弟昶表乞骸骨詔以昶代為忠武節度使未幾犨薨【幾居豈翻 考異曰薛居正五代史趙犨傳曰文德元年蔡州平朝廷議勲犨檢校司徒充泰寜軍節度使又改授浙西節度使不離宛丘兼領二鎮龍紀元年三月又以平巢蔡功就加平章事充忠武軍節度使仍以陳州為理所犨一日念弟昶共立軍功乃下令盡以軍州事付於昶遂上表乞骸後數月寢疾卒昶傳曰犨遙領泰寜軍節度使以昶為本州刺史俄而犨有疾遂以軍州盡付於昶詔授兵馬留後旋遷忠武軍節度使亦以陳州為理所時宗權未滅陳蔡封疆相接昶每選精銳深入蔡境蔡賊雖衆終不能抗以至宗權敗焉上云蔡州平以犨為忠武節度使下云昶為節度使時宗權未滅自相違今從犨傳】 丙申錢銶拔蘇州【去年冬錢銶攻蘇州事見上卷】徐約亡入海而死【光啟三年徐約據蘇州今走死】錢鏐以海昌都將沈粲權知蘇州 夏四月賜陝虢軍號保義【陜失冉翻】 五月甲辰潤州制置使阮結卒錢鏐以静江都將成及代之 李克用大發兵遣李罕之李存孝攻孟方立六月拔磁洺二州方立遣大將馬溉袁奉韜將兵數萬拒之戰於琉璃陂方立兵大敗二將皆為所擒克用乘勝進攻邢州方立性猜忌諸將多怨至是皆不為方立用方立慙懼飲藥死【中和二年孟方立據邢州】弟攝洺州刺史遷素得士心衆奉之為留後 【考異曰實録克用以弟克修守潞遣澤州刺史安金俊討方立方立因結諸鎮救援其將奚忠信攻遼州克用復遣李罕之等急攻方立將馬溉出戰為罕之所擒溉謂曰欲圖邢州當先取磁州及并師圍磁州方立與奚忠信帥兵大戰軍敗陷磁州而方立單騎還邢州忠信死焉方立愧之乃自圖死三軍立其弟遷求援汴州朱全忠遣王䖍裕赴之鎮州王鎔遺克用書而退唐年補録方立有謀將石元佐為金俊所獲金俊問之元佐請攻磁州破奚忠信金俊乃殺之方立果與忠信引兵入磁金俊與之戰大敗忠信死方立單騎入邢州愧見父老遂自裁薛居正五代史方立傳六月李存孝下洺磁兩郡方立遣馬溉袁奉韜盡率其衆逆戰於琉璃陂存孝擊之盡殪生獲馬溉奉韜初方立性苛急恩不逮下攻圍累旬夜自巡城慰諭守陴者皆倨方立知其不可乃飲酖而卒其從弟洺州刺史遷素得士心衆乃推為留後求援於汴時梁祖方攻時溥援兵不出按李罕之攻下磁州進攻洺州乃擒馬溉實録云溉為罕之謀取磁州蓋誤以石元佐為溉也又奚忠信去年已為李克脩所擒乃云與方立率兵大戰亦誤也舊紀六月邢洺節度使孟方立卒三軍推其弟洺州刺史遷為留後李克用出軍攻之新紀六月李克用寇邢州昭義軍節度使孟方立卒其弟遷自稱留後按唐年補録載王鎔奏得邢洺大將等狀以孟方立奄辭昭代二軍百姓同以親弟攝洺州刺史遷權知兵馬留後事及新舊紀實録薛史方立傳皆云立其弟遷惟太祖紀年録及薛史武皇紀云立其姪遷恐誤今從諸書】求援於朱全忠全忠假道於魏博羅弘信不許全忠乃遣大將王䖍裕將精甲數百間道入邢州共守【閒古莧翻為孟遷執王䖍裕降河東張本】 楊行密圍宣州城中食盡人相啗指揮使周進思據城逐趙鍠鍠將奔廣陵田頵追擒之未幾城中執進思以降【幾居豈翻降戶江翻】行密入宣州諸將争取金帛徐温獨據米囷為粥以食餓者温朐山人也【囷去倫翻倉圓曰囷食祥吏翻徐温之遠略已見於此矣】鍠將宿松周本勇冠軍中行密獲而釋之以為禆將【宿松漢皖縣地梁置高塘郡隋廢郡置宿松縣唐屬舒州九域志在州西南一百四十里宋白曰宿松縣漢元始中為松滋縣屬廬江郡晉武帝以荆州有松滋縣遂改為宿松冠古玩翻】鍠既敗左右皆散惟李德誠從鍠不去行密以宗女妻之【妻七細翻李德誠自此遂委質於楊氏海陵讓皇之世此心復能如從鍠之時乎】德誠西華人也行密表言於朝詔以行密為宣歙觀察使【朝直遙翻歙書涉翻】朱全忠與趙鍠有舊遣使求之行密謀於袁襲襲曰不若斬首以遺之【遺惟季翻】行密從之未幾襲卒行密哭之曰天不欲成吾大功邪何為折吾股肱也【折而設翻】吾好寛而襲每勸我以殺【好呼到翻】此其所以不夀與【與讀曰歟】孫儒遣兵攻廬州蔡儔以州降之【降戶江翻】朱珍拔蕭縣據之與時溥相拒朱全忠欲自往臨之珍命諸軍皆葺馬廏李唐賓部將嚴郊獨惰慢軍吏責之唐賓怒見珍訴之珍亦怒以唐賓為無禮拔劒斬之【珍唐賓交惡久矣乘怒殺之不復顧慮】遣白全忠云唐賓謀叛淮南左司馬敬翔恐全忠乘怒倉猝處置違宜【處昌呂翻】故留使者逮夜然後從容白之【從千容翻朱全忠兼領淮南節度以敬翔為左司馬逮夜而後言則全忠雖怒而未能發其暴】全忠果大驚翔因為畫策【為于偽翻下為之同】詐收唐賓妻子繫獄遣騎往慰撫全忠從之軍中始安秋七月全忠如蕭縣未至珍出迎命武士執之責以專殺而誅之【敬翔為全忠謀取朱珍猶用前計】諸將霍存等數十人叩頭為之請全忠怒以牀擲之乃退【使全忠不殺朱珍珍其肯為全忠用乎霍存等為之請弗思爾矣為于偽翻】丁未至蕭縣以龎師古代珍為都指揮使八月丙子全忠進攻時溥壁會大雨引兵還 冬十月平盧節度使王敬武薨子師範年十六軍中推為留後棣州刺史張蟾不從詔以太子少師崔安潜兼侍中充平盧節度使蟾迎安濳至州與之共討師範【為王師範殺張蟾張本】 以給事中杜孺休為蘇州刺史錢鏐不悅以知州事沈粲為制置指揮使【沈粲制其兵權杜孺休直寄坐耳】 楊行密遣馬步都虞侯田頵等攻常州【時錢鏐將杜稜守常州】 十一月上改名曄 上將祀圓丘故事中尉樞密皆䙆衫侍從僖宗之世已具襴笏【䙆暌桂翻衣裾分也襴音闌即今之袍也下施横幅因謂之襴新志曰唐初士人以棠苧襴衫為上服貴女功之始也一命以黄再命以黑三命以纁四命以緑五命以紫中書令馬周上議禮無服衫之文三代之制有深衣請加襴袖褾襈為士人上服開骻者為缺骻衫庶人服之長孫無忌又議服袍者下加襴緋紫皆視其品從才用翻】至是又令有司制法服【法服謂冕服劒佩也】孔緯及諫官禮官皆以為不可上出手札諭之曰卿等所論至當【當丁浪翻】事有從權勿以小瑕遂妨大禮於是宦官始服劒佩侍祠 【考異曰按田令孜楊復恭雖威權震主官不過金吾衛上將軍則其餘宦官必卑矣但諸書不見當時宦官所欲衣者何品秩之法服也】己酉祀圓丘赦天下上在藩邸素疾宦官及即位楊復㳟恃援立功【援立見上卷上年】所為多不法上意不平政事多謀于宰相孔緯張濬勸上舉大中故事抑宦者權復恭常乘肩輿至太極殿【太極殿西内前殿也】他日上與宰相言及四方反者孔緯曰陛下左右有將反者况四方乎上矍然問之緯指復恭曰復恭陛下家奴乃肩輿造前殿【矍居縳翻造七到翻】多養壯士為假子使典禁兵或為方鎮非反而何【楊復恭以假子守立為天威軍使守信為玉山軍使守貞為龍劒節度守忠為武定節度守厚為綿州刺史其餘假子為州刺史者甚衆號外宅郎君又養子六百人監諸道軍】復恭曰子壯士欲以收士心衛國家豈反邪上曰卿欲衛國家何不使姓李而姓楊乎復恭無以對復恭假子天威軍使楊守立本姓胡名弘立勇冠六軍【冠古玩翻】人皆畏之上欲討復恭恐守立作亂謂復恭朕欲得卿胡子在左右復恭見守立於上【見賢遍翻】上賜姓名李順節使掌六軍管鑰【北軍六軍皆分屯苑中屯營各有門晨夕啟閉】不期年擢至天武都頭領鎮海節度使俄加同平章事【天武亦神策五十四都之一期讀曰朞】及謝日臺吏申請班見百僚孔緯判不集【判臺申不使集百官】順節至中書色不悅他日語微及之緯曰宰相師長百僚【長知兩翻】故有班見相公職為都頭而於政事堂班見百僚於意安乎順節不敢復言【復扶又翻】朱全忠求領鹽鐵孔緯獨執以為不可謂進奏吏曰朱公須此職非興兵不可全忠乃止【史言孔緯相唐欲振紀綱惜制於時不得行其志耳】 田頵攻常州為地道入城中宵旌旗兵甲出於制置使杜稜之寢室遂虜之以兵三萬戍常州 朱全忠遣龎師古將兵自潁上趨淮南撃孫儒【宋僑置樓煩縣於汝隂郡界後魏以縣為下蔡郡治所後齊廢郡隋改為潁上縣唐屬潁州九域志在州東一百一十七里趨七喻翻】 十二月甲子王建敗山行章及西川騎將宋行能於廣都【敗補邁翻】行能奔還成都行章退守眉州壬申行章請降於建 戊寅孫儒自廣陵引兵度江壬午逐田頵取常州以劉建鋒守之儒還廣陵建鋒又逐成及取潤州【成及為錢鏐守潤州】 前山東南道節度使劉巨容之在襄陽也有申屠生教之燒藥為黄金田令孜之弟過襄陽巨容出金示之及寓居成都【中和四年巨容自襄陽奔成都】令孜求其方不與恨之是歲令孜殺巨容滅其族大順元年春正月戊子朔羣臣上尊號曰聖文睿德光武弘孝皇帝改元【上時掌翻】 李克用急攻邢州孟遷食竭力盡執王䖍裕及汴兵以降【僖宗中和二年孟方立據邢磁洺三州至是而亡考異曰唐末見聞録龍紀元年大軍守破邢州城孟遷投來拜李存孝邢州刺史十一月四日孟遷補充教練使太祖紀年録及薛居正五代史皆曰大順元年李存孝攻邢州急邢帥孟遷以邢洺磁三州歸於我執朱温之將王䖍裕等三百人以獻而無月太祖紀年録又曰太祖徙孟遷於太原以大將安金俊為邢洺團練使薛史孟遷傳曰大順元年二月遷執王䖍裕等乞降武皇令安金俊代之今從實録薛史䖍裕傳日時太祖大軍方討兗鄆未及救援邢人困而攜貳遷乃縶䖍裕送於太原尋為所殺按是時全忠方攻時溥未討兖鄆也王䖍裕傳誤】克用以安金俊為邢洺團練使 壬寅王建攻邛州【邛渠容翻】陳敬瑄一遣其大將彭城楊儒將兵三千助刺史毛湘守之湘出戰屢敗楊儒登城見建兵盛歎曰唐祚盡矣王公治衆嚴而不殘殆可以庇民乎【治直之翻】遂帥所部出降【帥讀曰率降戶江翻】建養以為子更其姓名曰王宗儒【更工衡翻】乙巳建留永平節度判官張琳為邛南招安使引兵還成都【復攻陳敬瑄也】琳許州人也陳敬瑄分兵布寨於犀浦郫導江等縣【垂拱二年分成都縣置犀浦縣郫漢古縣唐並屬成都府九域志郫縣在府西四十五里】發城中民戶一丁【不計其家丁數多少一戶則發一丁】晝則穿重壕採竹木運磚石【重直龍翻史炤曰古史考烏曹作塼】夜則登城擊柝巡警無休息韋昭度營於唐橋王建營於東閶門外建事昭度甚謹辛亥簡州將杜有遷執刺史員䖍嵩降於建【員音云又音運姓也】建以有遷知州事 汴將龎師古等衆號十萬度淮聲言救楊行密攻下天長壬子下高郵【下者降下】 二月己未資州將侯元綽執刺史楊戡降於王建建以元綽知州事 乙丑加朱全忠守中書令 龎師古引兵深入淮南己巳與孫儒戰於陵亭【九域志泰州興化縣有陵亭鎮】師古兵敗而還【遷從宣翻又如字】 楊行密遣其將馬敬言將兵五千乘虚襲據潤州李友將兵二萬屯青城將攻常州安仁義劉威田頵敗劉建鋒於武進【去年孫儒使劉建鋒據常州晉分曲阿縣置武進縣梁改為蘭陵隋廢唐垂拱二年又分晉陵置武進縣屬常州九域志縣有青城鎮】敬言仁義威屯潤州友合肥人威慎縣人也 李克用將兵攻雲州防禦使赫連鐸克其東城鐸求救於盧龍節度使李匡威匡威將兵三萬赴之丙子邢洺團練使安金俊中流矢死【中竹仲翻 考異曰實録四月丙辰朔李克用遣安金俊率師攻雲州赫連鐸求援於幽州李匡威匡威出師赴之戰於蔚州太原府軍大敗燕師執金俊獻於朝据太祖紀年録攻雲州在三月舊紀實録皆在四月恐是約奏到然紀年錄不言克用敗蓋諱之也今從唐末見聞録又紀年録唐末見聞録皆云金俊戰死實録云執獻之亦誤】河東萬勝軍使申信叛降於鐸會幽州軍至克用引還 時溥求救於河東李克用遣其將石君和將五百騎赴之 李克用巡潞州以供具不厚怒昭義節度使李克脩詬而笞之【詬古翻又許翻】克脩慙憤成疾三月薨 【考異曰太祖紀年録太祖遣李罕之李存孝攻邢州十月且命班命由上黨而歸克脩性吝嗇太祖左右徵賂於克脩旬日間費數十萬尚以為供張不豐掎其事笞克脩而歸太原俄而克脩憤恥寢疾薛史克脩傳曰龍紀元年武皇大舉以伐邢洺及班師因撫封於上黨按太祖紀但遣罕之存孝攻邢州不云親行蓋罕之存孝圍邢州克用但以大軍屯境上為之聲援去十月先還罕之存孝猶圍邢州故正月孟遷降也】克用表其弟決勝軍使克恭為昭義留後【為潞州叛克用張本】賜宣歙軍號寜國以楊行密為節度使 夏四月宿<br />
<br />
  州將張筠逐刺史張紹光附於時溥【去年朱全忠取宿州】朱全忠帥諸軍討之【帥讀曰率】溥出兵掠碭山【碭徒郎翻】全忠遣牙内都指揮使朱友裕撃之殺三千餘人擒石君和 【考異曰郗象梁太祖實録前云四月丙辰後云乙卯溥出兵按長歷乙卯三月晦日實錄誤也】友裕全忠之子也 乙丑陳敬瑄遣蜀州刺史任從海將兵二萬救邛州戰敗欲以蜀州降王建敬瑄殺之【任音壬】以徐公鉥代為蜀州刺史【鉥時橘翻】丙寅嘉州刺史朱實舉州降於建丙子僰道土豪文武堅執戎州刺史謝承恩降於建【僰道故僰侯國漢立縣為犍為郡治所梁置戒州僰蒲比翻】 赫連鐸李匡威表請討李克用【乘其不利也】朱全忠亦上言克用終為國患今因其敗臣請帥汴滑孟三軍【汴滑孟三鎮時皆屬全忠帥讀曰率】與河北三鎮共除之【河北三鎮謂盧龍李匡威成德王鎔魏博羅弘信】乞朝廷命大臣為統帥【帥所類翻】初張濬因楊復恭以進【事見二百五十四卷僖宗廣明元年】復恭中廢更附田令孜而薄復恭【更工衡翻改也附令孜事見中和元年】及復恭再用事深恨之【襄王煴之亂田令孜往依陳敬瑄自是之後復恭再用事】上知濬與復恭有隙特親倚之 【考異日舊傳再幸山南復恭代令孜為中尉罷濬知政事昭宗初在藩邸深疾宦官復恭有援立大勲恃恩任事上心不平之當時趨向者多言濬有方略能畫大計復用為宰相判度支据舊紀實録新紀表濬自光啟三年九月拜平章事至大順二年兵敗坐貶未嘗罷免舊傳誤也今從新傳】濬亦以功名為己任每自比謝安裴度克用之討黄巢屯河中也【見二百五十五卷僖宗中和二年三年】濬為都統判官【王鐸為都統奏濬為判官】克用薄其為人聞其作相私謂詔使曰張公好虚談而無實用【好呼到翻】傾覆之士也主上采其名而用之他日交亂天下必是人也濬聞而衘之上從容與濬論古今治亂【從千容翻】濬曰陛下英睿如此而中外制於彊臣【言中則制於宦官外則制於方鎮】此臣日夜所痛心疾首也上問以當今所急對曰莫若彊兵以服天下上於是廣募兵於京師至十萬人及全忠等請討克用上命三省御史臺四品以上議之【三省尚書省門下省中書省也四品以上尚書左右丞及六部侍郎門下中書省自左右諫議以上御史臺自中丞以上皆四品也】以為不可者什六七杜讓能劉崇望亦以為不可【杜讓能劉崇望二相也】濬欲倚外勢以擠楊復恭【宰相主兵則外廷之勢重擠子細翻又子西翻】乃曰先帝再幸山南沙陀所為也【謂光啟二年事見二百五十六卷】臣常慮其與河朔相表裏致朝廷不能制今兩河藩鎮共請討之【河南獨朱全忠河北獨李匡威請討克用耳餘皆不欲也】此千載一時【載子亥翻】但乞陛下付臣兵柄旬月可平失今不取後悔無及 【考異曰舊濬傳曰會朱全忠誅秦宗權安居受殺李克恭以潞州降全忠幽州李匡威雲州赫連鐸等奏請出軍討太原按時安居受未殺李克恭舊傳誤也太祖紀年録曰太祖中和破賊時濬為諫議大夫出軍判官常以虚誕誘太祖太祖薄其為人及聞濬入中書太祖常私於詔使曰張公傾覆之士先帝知其為人不至大任主上付之重位必亂天下濬知之隂衘太祖按濬自僖宗時為宰相紀誤】孔緯曰濬言是也復恭曰先朝播遷雖藩鎮跋扈亦由居中之臣措置未得其宜今宗廟甫安不宜更造兵端上曰克用有興復大功【謂破黄巢復京城也】今乘其危而攻之天下其謂我何緯曰陛下所言一時之體也張濬所言萬世之利也昨計用兵饋運犒賞之費一二年間未至匱乏在陛下斷志行之耳上以二相言協僶俛從之【斷丁亂反僶民尹反僶俛勉彊不得已之意】曰兹事今付卿二人無貽朕羞【觀帝此言亦知河東之不可伐矣】五月詔削奪克用官爵屬籍【克用賜姓故編之屬籍注已見前】以濬為河東行營都招討制置宣慰使京兆尹孫揆副之以鎮國節度使韓建為都虞侯兼供軍糧料使以朱全忠為南面招討使李匡威為北面招討使赫連鐸副之濬奏給事中牛徽為行營判官徽曰國家以喪亂之餘【喪息浪翻】欲為英武之舉横挑彊寇【挑徒了翻】離諸侯心吾見其顛沛也【沛音貝】遂以衰疾固辭徽僧孺之孫也【牛僧孺文宗大和中為相】 李克恭驕恣不曉軍事潞人素樂李克脩之簡儉【樂音洛】且死非其罪潞人憐之由是將士離心初潞人叛孟氏牙將安居受等召河東兵以取潞州【見二百五十五卷僖宗中和三年】及孟遷以邢洺磁州歸李克用克用寵任之以遷為軍城都虞侯羣從皆補右職【從才用翻其後孟知祥見任於莊宗亦遷之兄子也】居受等咸怨且懼昭義有精兵號後院將克用既得三州將圖河朔令李克恭選後院將尤驍勇者五百人送晉陽潞人惜之克恭遣牙將李元審及小校馮霸部送晉陽至銅鞮【銅鞮漢縣唐屬潞州九域志在州西北一百四十五里校戶教翻鞮丁奚翻】霸刼其衆以叛循山而南至於沁水【沁水漢縣名唐之沁水後魏泰寜郡地也北齊廢郡為永安縣隋開皇十八年改曰沁水唐屬澤州九域志在州西北二百里】衆已三千人李元審擊之為霸所傷歸於潞【考異曰元審與霸同部送後院將霸所以能獨叛而元審所以得不死者蓋後院將有叛有不叛者叛者從霸不叛者從元審故克用益元審兵使討霸也此段考異疑有闕文】庚子克恭就元審所館視之安居受帥其黨作亂【帥讀曰率】攻而焚之克恭元審皆死衆推居受為留後附於朱全忠居受使召馮霸不至居受懼出走為野人所殺霸引兵入潞自為留後 【考異曰編遺録八月甲寅馮霸殺李克恭來降上請河陽帥朱崇節領兵入潞兼充留後戊辰李克用圍之上遣葛從周率驍勇夜銜枚斫營突入上黨以壯潞人之心薛居正五代史梁太祖紀亦同按克用未嘗自圍潞也克恭傳李元審戰傷收軍於潞五月十五日克恭視元審於孔目吏劉崇之第是日州縣將安居受引兵攻克恭克恭元審並遇害州民推居受為留後居受遣人召馮霸於沁水霸不受命居受懼將奔歸朝廷至長子為野人所殺傳首馮霸軍霸乃引衆據潞州自稱留後求援於汴武皇令康君立討之汴將葛從周求援霸唐末見聞錄曰五月十七日昭義狀申軍變殺節使當日點汾州五縣土團將士赴昭義二十三日昭義僕射家累入府新紀五月壬寅安居受殺李克恭按壬寅十七日乃報到太原日也今從太祖紀年錄薛史克恭傳舊紀五月丙午潞州軍亂殺李克恭監軍使薛績本函克恭首獻之於朝濬方起兵朝廷稱賀此蓋克恭首到日也舊紀又曰七月全忠遣從周帥千騎入潞州唐太宗紀年録薛史唐紀五月葛從周入潞太早蓋因克恭死終言之編遺錄薛史梁紀八月克恭死太晚蓋因從周入潞推本之又從周入潞全忠始請孫揆付鎮當在揆被執前也今克恭死從紀年錄從周入潞從舊紀】時朝廷方討克用聞克恭死朝臣皆賀全忠遣河陽留後朱崇節將兵入潞州權知留後克用遣康君立李存孝將兵圍之壬子張濬帥諸軍五十二都及邠寜鄜夏雜虜合五萬人發京師【帥讀曰率下同鄜音夫夏戶雅翻】上御安喜樓餞之【安喜樓安喜門樓也按楊復恭之亂上御安喜門劉崇望謂禁軍曰天子親在街東督戰竊意安喜門即朱雀街東之安上門也】濬屏左右言於上曰【屏必郢翻又必正翻】俟臣先除外憂然後為陛下除内患【為于偽翻】楊復恭竊聽聞之兩軍中尉餞濬於長樂坂【長樂坂在長安城東即滻坡樂音洛坂音反】復恭屬濬酒【屬之欲翻】濬辭以醉復恭戲之曰相公杖鉞專征作態邪濬曰俟平賊還方見作態耳【未能成事而先為大言此張濬之疎也】復恭益忌之癸丑削奪李罕之官爵【以附李克用也】六月以孫揆為昭義節度使充招討副使 丁巳茂州刺史李繼昌帥衆救成都己未王建擊斬之辛酉資簡都制置應援使謝從本殺雅州刺史張承簡舉城降建【資簡相去二百十六里簡州北至成都百五十里雅州與邛州接壤相去二百七十里王建圖邛州以為根本兵威所及故謝從本以雅州降之】 孫儒求好於朱全忠全忠表為淮南節度使未幾全忠殺其使者遂復為仇敵【好呼到翻幾居豈翻復扶又翻】 光啟末德州刺史盧彦威逐義昌節度使楊全玫自稱留後【見二百五十六卷僖宗光啟元年玫莫杯翻】求旌節朝廷未許至是王鎔羅弘信因張濬用兵為之請【為于偽】乃以彦威為義昌節度使 張濬會宣武鎮國静難鳳翔保大定難諸軍於晉州【難乃旦翻】 更命義成軍曰宣義辛未以朱全忠為宣武宣義節度使【按方鎮表全忠以父名誠請改義成曰宣義更工衡翻】全忠以方有事徐楊徵兵遣戍殊為遼闊乃辭宣義請以胡真為節度使從之然兵賦出入皆制於全忠一如巡屬及胡真入為統軍竟以全忠為兩鎮節度使罷淮南不領焉【領淮南見上卷僖宗光啟三年】 秋七月官軍至隂地關【汾州靈石縣西南有隂地關 考異曰舊紀七月乙酉朔王師屯於隂地太原大將康君立以兵拒戰按君立時圍潞州何暇至隂地關又不言勝負今不取】朱全忠遣驍將葛從周將千騎潛自壺關夜抵潞州犯圍入城【九域志壼關西至潞州二十五里宋白曰壺關縣以山形似壺古於此置關故名 考異曰舊紀實録皆云從周權知留後又汴人圍澤州呼李罕之云葛司空已入潞府李存孝圍潞州呼城上人云葛僕射可歸大梁似從周實為留後也然薛居正五代史梁太祖紀云帝請以河陽節度使朱崇節為潞州留後實錄明年五月以前昭義節度使朱崇節為河陽節度使按河陽自解張全義圍以來常附屬於汴朱全忠以部將丁會張宗厚等為之留後非一人崇節蓋亦汴將為河陽留後全忠使權昭義留後既不能守復歸河陽耳諸書因謂之節度使蓋誤也後周但與崇節共守潞州以其名著故外人但稱從周不數崇節也】又遣别將李讜李重胤鄧季筠將兵攻李罕之於澤州又遣張全義朱友裕軍於澤州之北為從周應援 【考異曰編遺録八月遣從周入上黨九月壬寅上往河陽令李讜救應朱崇節又命朱友裕張全義簡精鋭過屯於澤州北應接取崇節從周以歸薛居正五代史梁太祖紀九月壬寅上至河陽遣李讜引軍趨澤為晉人所敗帝又遣朱友裕張全義率精兵至澤州北以為應援既而崇節從周棄潞來歸戊申帝斬李重裔遂班師按讜等初圍澤州時讜城上人云張相公圍太原葛司空已入潞府是當時南兵方盛非孫揆就擒從周棄潞州之後也故置於此】季筠下邑人也全忠奏臣已遣兵守潞州請孫揆赴鎮張濬亦恐昭義遂為汴人所據分兵二千使揆將之趣潞州【趣七喻翻】八月乙丑揆發晉州【九域志自晉州東至潞州三百八十五里】李存孝聞之以三百騎伏於長子西谷中【長子漢縣唐屬潞州九域志在州西南四十五里】揆建牙杖節褒衣大蓋擁衆而行【凡節度使其行前建牙旗杖所賜節褒衣大補博裾之衣大蓋即今之清涼繖】存孝突出擒揆及賜旌節中使韓歸範牙兵五百餘人追擊餘衆於刁黄嶺盡殺之存孝械揆及歸範以素練【紤克夜翻維縶之也】徇於潞州城下曰朝廷以孫尚書為潞帥命韓天使賜旌節【韓歸範銜天子之命故謂之天使帥所類翻使疏吏翻】葛僕射可速歸大梁令尚書視事遂縳以獻於克用克用囚之既而使人誘之欲以為河東副使【誘音酉】揆曰吾天子大臣兵敗而死分也【分扶問翻】豈能復事鎮使邪【節度使任居方鎮孫揆鄙薄之呼為鎮使復扶又翻】克用怒命以鋸鋸之鋸不能入揆罵曰死狗奴鋸人當用板夾汝豈知邪乃以板夾之至死罵不絶聲 丙寅孫儒攻潤州 蘇州刺史杜孺休到官【去年朝廷命杜孺休刺蘇州】錢鏐密使沈粲害之會楊行密將李友拔蘇州粲歸杭州鏐欲歸罪於粲而殺之粲奔孫儒王建退屯漢州【自成都退屯漢州】 陳敬瑄括富民財以供<br />
<br />
  軍置徵督院逼以桎梏箠楚使各自占凡有財者如匿虚占急徵【箠止橤翻占之瞻翻無其財而自占為有謂之虚占】咸不聊生李罕之告急於李克用【為汴兵所圍也】克用遣李存孝將五千騎救之 九月壬寅朱全忠軍於河陽汴軍之初圍澤州也呼李罕之曰相公每恃河東輕絶當道【當道猶云本道汴軍自謂也】今張相公圍太原葛僕射入潞府【張相公謂張濬葛僕射謂葛從周】旬月之間沙陀無穴自藏相公何路求生邪及李存孝至引精騎五百繞汴寨呼曰【呼火故翻】我沙陀之求穴者也欲得爾肉以飽士卒可令肥者出鬭汴將鄧季筠亦驍將也引兵出戰存孝生擒之是夕李讜李重胤收衆遁去存孝罕之隨而擊之至馬牢山大破之斬獲萬計追至懷州而還存孝復引兵攻潞州【復扶又翻】葛從周朱崇節棄潞州而歸戊申全忠庭責諸將橈敗之罪【橈奴教翻】斬李讜李重胤而還 【考異曰唐太祖紀年錄六月朱崇節葛從周據潞州李重胤鄧季筠張全義將兵七萬攻澤州李存孝將三千騎赴援初汴軍攻城門呼李罕之云云李存孝憤其言引鐵騎五百追擊入季筠營門生獲其都將十數是夜汴將李讜收軍而遁存孝罕之追擊至馬牢山斬首萬級追襲掩擊至於懷州而還存孝復引軍攻潞州九月二日葛從周帥衆棄城而遁唐末見聞録閏九月昭義軍前狀申昭義軍人拔滅逃遁收下城池擒獲到餘黨五十人巾縛送上至二十日行營都指揮使李存孝迴戈歸府薛居正五代史梁太祖紀九月壬寅帝至河陽遣李讜引軍趨澤潞行至馬牢川為晉人所敗帝又遣朱友裕張全義率精兵至澤州北以為應援既而崇節從周棄潞來歸戊申帝廷責諸將敗軍之罪斬李重胤以徇遂班師焉實録九月甲申朔康君立急攻潞州朱全忠駐河陽遣李讜引軍趨澤潞至馬牢山川與并師大戰不利鄧季筠被執復遣朱友裕張全義至澤州北應援葛從周朱崇節率衆棄潞州歸按六月李存孝若已破李讜追至懷州懷州去河陽止一程豈得九月方到河陽讜之敗必在九月戊申前一兩日也蓋紀年録因從周據潞州事終言之九月甲申朔十九日壬寅二十五日戊申若全忠至河陽始遣讜等趣澤潞既敗而從周等棄潞來歸七日之間豈容許事蓋薛史因讜敗追本前事耳若九月二日從周已棄潞州何得十九日後攻澤州者猶云葛司空入潞府乎蓋實録承紀年録而誤也今全忠往來月曰從薛史事則兼采諸書】李克用以康君立為昭義留後李存孝為汾州刺史存孝自謂擒孫揆功大當鎮昭義而君立得之憤恚不食者數日縱意刑殺始有叛克用之志李匡威攻蔚州虜其刺史邢善益赫連鐸引吐蕃黠戛斯衆數萬攻遮虜軍殺其軍使劉胡子克用遣其將李存信擊之不勝更命李嗣源為存信之副遂破之克用以大軍繼其後匡威鐸皆敗走【考異曰太祖紀年錄是月幽帥李匡威會赫連鐸引吐蕃黠戛斯之衆十萬寇我北鄙攻遮虜軍太祖御親軍出塞營於渾河川之田村李存孝引前鋒與賊戰於樂安鎮賊軍大敗遁走舊紀九月幽州雲州蕃漢兵三萬攻雁門太原府將李存信薛阿檀擊敗之實録閏月甲寅朔幽州李匡威下蔚州克用援兵至匡威大敗赫連鐸引吐蕃黠戛斯之衆攻遮虜軍克用營渾河川戰於樂安鎮破之鐸乃退軍此蓋約奏到日唐末見聞録十一月十五日發往向北打鹿有使報稱幽州李匡威收却蔚州十六日至十八日旋發諸州兵士至軍前二十九日大捷有榜曉告殺燕軍三萬餘人十九日知客押衙苖仲周賫榜到殺得退渾一千帳二十九日下復云十九日亦誤今但繫此月不書日】獲匡威之子武州刺史仁宗【新志河東道有武州領文德縣闕其建置之年】及鐸之壻俘斬萬計李嗣源性謹重亷儉諸將相會各自詫勇略【詫丑亞翻誇也】嗣源獨默然徐曰諸君喜以口擊賊【喜許記翻】嗣源但以手擊賊耳衆慙而止 楊行密以其將張行周為常州制置使閏月孫儒遣劉建鋒攻拔常州殺行周遂圍蘇州 【考異曰吳録十一月孫儒攻破望亭無錫諸屯遂至蘇州今從吳越備史在閏月】 卭州刺史毛湘本田令孜親吏王建攻之急食盡救兵不至壬戌湘謂都知兵馬使任可知曰吾不忍負田軍容吏民何罪爾可持吾頭歸王建乃沐浴以俟刃可知斬湘及二子降於建士民皆泣甲戌建持永平旌節入邛州以節度判官張琳知留後【時朝命以邛州建永平軍王建為節度使是年正月建攻邛州至是克之】繕完城隍撫安夷獠經營蜀雅【九域志邛州北至蜀州七十里西南至雅州一百六十里獠魯皓翻】冬十月癸未朔建引兵還成都蜀州將李行周逐徐公鉥舉城降建【鉥辛律翻】 乙酉朱全忠自河陽如滑州視事【朱全忠既領宣義節遂如滑州視事】遣使者請糧馬及假道於魏以伐河東羅弘信不許又請於鎮鎮人亦不許全忠乃自黎陽濟河撃魏加邠寜節度使王行瑜侍中佑國節度使張全義同<br />
<br />
  平章事 官軍出隂地關遊兵至於汾州李克用遣薛志勤李承嗣將騎三千營於洪洞【洪洞漢楊縣義寜元年改曰洪洞取縣北洪洞嶺為名屬晉州九域志在州北五十五里又北二百九十五里至汾州】李存孝將兵五千營於趙城【義寜元年分霍邑置趙城縣以春秋時晉獻公滅耿以賜趙夙因謂之趙城屬晉州九域志在州北八十五里宋白曰史記周穆王封造父於趙城注云在河東永安縣余按宋白既以趙城為造父所封之地此又引史記注何所折衷哉】鎮國節度使韓建以壯士三百夜襲存孝營存孝知之設伏以待之建兵不利静難鳳翔之兵不戰而走【難乃旦翻】河東兵乘勝逐北扺晉州西門張濬出戰又敗官軍死者近三千人【近其靳翻】静難鳳翔保大定難之軍先度河西歸濬獨有禁軍及宣武軍合萬人與韓建閉城拒守自是不敢復出【復扶又翻】存孝引兵攻絳州【九域志晉州南至絳州一百二十五里】十一月刺史張行恭棄城走存孝進攻晉州三日與其衆謀曰張濬宰相俘之無益天子禁兵不宜加害乃退五十里而軍【李存孝雖悍猶不敢攻執宰相犯獵禁兵分尚存故也】濬建自含口遁去【水經注洮水源出河東聞喜縣清襄山其水東逕大嶺下西流出謂之含口又西合於涑水即含山之口也】存孝取晉絳二州大掠慈隰之境先是克用遣韓歸範歸朝【韓歸範與孫揆俱擒李克用遣之歸朝】附表訟寃 【考異曰實錄十一月王師入隂地關至汾隰李克用遣將薛阿檀李承嗣拒之李存信以兵五千圍趙城韓建以華州兵戰存信設伏擊破之邠鳳之師未戰而走禁軍自潰由是大敗存信直壓晉州西門引軍攻絳州十二月壬午朔晉州刺史張行恭棄城而遁韓建以諸軍保晉州李存信追擊戰敗退保絳州張濬以汴卒禁軍屯晉州存信攻之三日濬建拔晉絳遁還存信收二州舊紀克用遣李存信薛阿檀拒王師於隂地三戰三捷由是河西鄜夏邠岐之軍渡河西歸韓建以諸軍保平陽存信追之建軍又敗建退保絳州張濬在晉州存信攻之三日相與謀云云遂退舍五十里十二月壬午朔濬建拔晉絳遁去存信收晉絳大掠河中四郡張濬傳曰十月濬軍至隂地邠岐華三鎮之師營平陽李存孝擊之一戰而敗進攻晉州薛居正五代史武皇紀曰十月張濬之師入晉州遊軍至邠濕武皇遣薛鐵山李承嗣將騎三千出隂地關營於洪洞遣李存孝將兵五千營於趙城華州韓建以壯士三百人冒犯存孝之營存孝追擊直壓晉州西門張濬之師出戰為存孝所敗自是閉壁不出存孝引軍攻絳州李存孝傳曰十月存孝引收潞之師圍張濬於平陽云云存孝引軍攻絳州十一月刺史張行恭棄城而去張濬韓建亦由含口而遁存孝收晉絳太祖紀年錄十月張濬之師入隂地關犯汾隰令薛鐵山李承嗣將騎三千出隂地繼發李存孝將兵五千進擊營於趙城敗韓建直壓晉州西門自是閉壁不出存孝攻絳州十二月晉州刺史張行恭棄城遁建濬由含山路逃遁遂收晉絳初濬部禁軍至晉州邠鳳之師望風遁歸蓋楊復恭隂沮之也唐末見聞録曰八月五日相公差晉州捉到天使閭大夫入京奏事兼貢表曰臣某乙言今月二十六日臣所部南界晉州長寜關使張承暉等投臣當道齎到宰臣張濬牓一道内稱招討處置使兼錄到詔白云陛下削臣屬籍奪臣本官仍欲會兵討問云云唐補紀曰朱全忠自攻破徐州頻貢章表克用與朱玫同立襄王以為大逆其朱玫以下並已誅鋤克用時最為魁首據其罪狀請舉天兵臣率師關東掎角相應朝廷遂以宰臣張濬為都統授崔胤為河中節度應援使大軍行到同州克用領蕃漢馬步稱三十萬入河北界其張濬使人探朱全忠兵馬並不來相應乃於昭義西與太原交戰不利而回朝廷知為全忠所賣便差使至克用所與賞給令回貶都統張濬於雲夢除崔胤於嶺外薛史李承嗣傳初大軍入隂地薛志勤與承嗣率騎三千抗之敗韓建之軍於蒙坑進收晉絳以功授忻州刺史時鳳翔軍營霍邑承嗣帥一軍收之岐人夜遁追擊至趙城合大軍攻平陽旬有三日而拔按李存信傳無攻晉絳事蓋舊紀十月存孝已背太原故此戰皆云存信實録因之而誤据五代紀傳太祖紀年録當是存孝又隰州隸河中節度所云入隂地關犯汾隰者蓋謂汾水之旁下濕曰隰耳又紀年録實録以張行恭為晉州刺史亦誤也今從薛史晉州刺史若已走則濬建安能保城實録誤也今從李存孝傳唐補紀云崔胤為河中節度尤為疏繆自餘諸書參取之】言臣父子三代受恩四朝破龎勛翦黄巢黜襄王存易定【執宜國昌克用三代歷武宣懿僖四朝破龎勛見二百五十一卷懿宗咸通十年翦黄巢見二百五十五卷僖宗中和三年四年黜襄王見二百五十六卷光啟二年存易定見光啟元年】致陛下今日冠通天之冠【日冠古玩翻】佩白玉之璽未必非臣之力也若以攻雲州為臣罪則拓跋思恭之取鄜延【拓跋思恭取鄜延以授其弟思孝】朱全忠之侵徐鄆【謂朱全忠攻時漙於徐攻朱瑄於鄆事並見上】何獨不討賞彼誅此臣豈無辭且朝廷當阽危之時則譽臣為韓彭伊呂【阽余亷翻又丁念翻臨危曰阽危譽音余稱譽】及既安之後則罵臣為戎羯胡夷今天下握兵立功之人獨不懼陛下他日之罵乎况臣果有大罪六師征之自有典刑何必幸臣之弱而後取之邪今張濬既出師則固難束手已集蕃漢兵五十萬欲直抵蒲潼與濬格鬭若其不勝甘從削奪不然方且輕騎叩閽頓首丹陛訴姦回於陛下之扆坐【扆隱豈翻記天子負扆南面而立扆畫斧屛風也設之戶牖間坐徂卧翻】納制敕於先帝之廟庭然後自拘司敗恭俟鈇質【司敗即司寇官】之表至濬己敗朝廷震恐濬與韓建踰王屋至河陽撤民屋為栰以濟河【河南王屋縣有王屋山王屋漢河東垣縣地後魏置長平縣後周置王屋郡隋廢郡為縣九域志縣在孟州西北一百三十里 考異曰實録明年二月云時張濬韓建久敗後為克用騎將李存信所追至是方自含山踰王屋出河清逹於河陽河溢無舟檝建壞民廬舍為木罌數百度河人多覆溺似太晩今因濬建走終言之】師徒失亡殆盡是役也朝廷倚朱全忠及河朔三鎮及濬至晉州全忠方連兵徐鄆雖遣將攻澤州而身不至行營乃求兵糧於鎮魏鎮魏倚河東為扞蔽皆不出兵惟華邠鳳翔鄄夏之兵會之【華戶化翻鄄當作鄜詳見辯誤】兵未交而孫揆被擒幽雲俱敗【幽李匡威雲赫連鐸】楊復恭復從中沮之故濬軍望風自潰【復從扶又翻史言張濬志節可憐】十二月孫儒拔蘇州殺李友 【考異曰莊宗列傳楊行密夀州夀春人初據】<br />
<br />
  【本州秦宗權遣孫儒及行密同攻陷楊州儒專據之龍紀元年儒出軍攻宣州行密襲據楊州稱留後北通時溥儒引軍攻之大順元年行密禦備力竭率衆夜遁出據宣州此說最為差誤國朝開寶中薛居正修五代史江南未平不見本國舊史据昭遠所記及唐年補録作行密傳但知行密非夀春人改為盧州又知行密非受宗權命與孫儒同陷楊州餘皆無次叙今按吳録太祖紀及高遠唐烈祖實錄行密傳云光啟三年十月秦彦畢師鐸出走行密入楊州十一月孫儒圍揚州文德元年四月儒陷楊州行密奔廬州八月自廬州帥兵攻宣州龍紀元年六月陷宣州殺趙鍠大順二年七月孫儒再渡江攻宣州景福元年六月執斬儒復歸楊州且龍紀元年孫儒方彊行密新得宣州安能襲據楊州踰年哉近修唐書行密傳全用吳録事迹乃云儒進攻行密行密復入楊州北通時溥扞儒朱全忠遣龎師古助行密敗於高郵行密懼退還宣州蓋承莊宗列傳五代史之誤而不考正也】安仁義等聞之焚潤州廬舍夜遁儒使沈粲守蘇州又遣其將歸傳道守潤州【楊行密遣安仁義破錢鏐之兵而取常蘇潤孫儒又從而奪之民之死於兵者不知其幾矣】辛丑汴將丁會葛從周撃魏度河取黎陽臨河【黎陽漢古縣唐屬衛州九域志在州東北一百二十里隋分黎陽縣置臨河縣唐屬相州】龎師古霍存下淇門衛縣【衛漢朝歌縣紂所都朝歌城在今縣西隋大業二年改曰衛縣唐屬衛州九域志衛州汲縣有淇門鎮】朱全忠自以大軍繼之是歲置昇州於上元縣以張雄為刺史【至德二載以潤州江寜縣】<br />
<br />
  【置昇州上元二年廢今復置 考異曰新地理志光啟三年以上元等四縣置昇州張雄傳大順初以上元為昇州授雄刺史吳録馮弘鐸傳大順元年復以上元為昇州命弘鐸為刺史按是時雄尚存今從雄傳】二年春正月羅弘信軍於内黄丙辰朱全忠擊之五戰皆捷至永定橋斬首萬餘級弘信懼遣使厚幣請和全忠命止焚掠歸其俘還軍河上魏博自是服於汴 庚申制以太保門下侍郎同平章事孔緯為荆南節度使中書侍郎同平章事張濬為鄂岳觀察使【二人罷相以晉絳喪師也】以翰林學士承旨兵部侍郎崔昭緯同平章事御史中丞徐彦若為戶部侍郎同平章事昭緯慎由從子【崔慎由相宣宗從才用翻】彦若商之子也【徐商見二百四十九卷宣宗大中十二年】楊復恭使人刼孔緯於長樂坡【長樂坡即長樂坂】斬其旌節資裝俱盡緯僅能自免李克用復遣使上表曰【復扶又翻】張濬以陛下萬代之業邀自已一時之功知臣與朱温深仇私相連結臣今身無官爵名是罪人不敢歸陛下藩方且欲於河中寄寓進退行止伏俟指麾【竊謂克用此表楊復恭密教之也】詔再貶孔緯均州刺史張濬連州刺史賜克用詔悉復其官爵使歸晉陽 【考異曰舊紀太原軍屯晉州克用遣中使韓歸範還朝因上表訴寃言賊臣張濬依倚全忠離間功臣朝廷欲令釋憾下羣臣議其可否左僕射韋昭度等議云云在十二月按是年昭度討陳敬瑄舊紀誤今從實録】 孫儒盡舉淮蔡之兵濟江癸酉自潤州轉戰而南田頵安仁義屢敗退楊行密城戍皆望風奔潰儒將李從立奄至宣州東溪【東溪在宣城東今謂之宛溪】行密守備尚未固衆心危懼夜使其將合肥臺濛將五百人屯溪西【溪西即宛溪之西】濛使士卒傳呼往返數四從立以為大衆繼至遽引去儒前軍至溧水【溧水漢溧陽縣隋分置溧水縣時屬昇州九域志在州東八十五里】行密使都指揮使李神福拒之神福陽退以示怯儒軍不設備神福夜帥精兵襲之俘斬千人【帥讀曰率下同】二月加李克用守中書令復李罕之官爵再貶張濬<br />
<br />
  繡州司戶 韋昭度將諸道兵十餘萬討陳敬瑄三年不能克【文德元年遣昭度討西川至是三年矣】饋運不繼朝議欲息兵三月乙亥制復敬瑄官爵 【考異曰新紀二月乙巳赦陳敬瑄己未詔王建罷兵不受命十國紀年亦曰二月乙巳復敬瑄官爵按二月辛巳朔無己未新紀誤也今從實録】令顧彦朗王建各帥衆歸鎮【使顧彦朗歸梓州王建歸申州】 王師範遣都指揮使盧弘擊棣州刺史張蟾弘引兵還攻師範師範使人以重賂迎之曰師範童騃【騃語駭翻癡愚也】不堪重任願得避位使保首領公之仁也弘以師範年少信之不設備【少詩照翻】師範密謂小校安邱劉鄩曰【安邱漢縣古根牟國唐屬密州九域志在州西北一百二十里校戶教翻鄩徐心翻】汝能殺弘吾以汝為大將弘入城師範伏甲而享之鄩殺弘於座及其黨數人師範慰諭士卒厚賞重誓自將以攻棣州執張蟾斬之崔安潛逃歸京師師範以鄩為馬步副都指揮使詔以師範為平盧節度使師範和謹好學【好呼到翻】每本縣令到官師範輒備儀衛往謁之令不敢當師範命客將挾持令坐於聽事【客將主唱導儐贊賓客漢晉鈴下威儀之職將即亮翻令力丁翻聽讀曰廳】自稱百姓王師範拜之於庭僚佐或諫師範曰吾敬桑梓所以教子孫不忘本也【詩維桑與梓必恭敬止注云父之所樹子不敢不恭敬】 張濬至藍田逃奔華州依韓建與孔緯密求救於朱全忠全忠上表為緯濬訟寃【為于偽翻】朝廷不得已並聽自便緯至商州而還亦寓居華州 邢洺節度使安知建潛通朱全忠【安金俊既死李克用以安知建代鎮邢洺】李克用表以李存孝代之知建懼奔青州朝廷以知建為神武統軍知建帥麾下三千人將詣京師過鄆州朱瑄與克用方睦伏兵河上斬之【薛史安知建奔青州自棣州沂河歸朝朱瑄邀斬之河上帥讀曰率】傳首晉陽 夏四月有彗星見於三台【斗魁下六星兩兩而比曰三台見賢遍翻】東行入太微長十丈餘【長直亮翻】甲申赦天下 成都城中乏食弃兒滿路【父子不能相贍至於弃之】民有潛入行營販米入城者邏者得之【邏郎佐翻】以白韋昭度昭度曰滿城饑甚忍不救之釋勿問亦有白陳敬瑄者敬瑄曰吾恨無術以救餓者彼能如是勿禁也由是販者浸多然所致不過斗升截筒徑寸半深五分量米而鬻之每筒百餘錢餓殍狼籍【深式禁翻量音良殍被表翻】軍民彊弱相陵將吏斬之不能禁乃更為酷法或斷腰或斜劈【斷丁管翻劈普壁翻】死者相繼而為者不止人耳目既熟不以為懼吏民日窘【窘巨隕翻】多謀出降敬瑄悉捕其族黨殺之慘毒備至内外都指揮使眉州刺史成都徐耕性仁恕所全活數千人田令孜曰公掌生殺而不刑一人有異志邪耕懼夜取俘囚戮於市王建見罷兵制書曰大功垂成奈何棄之謀於周庠庠勸建請韋公還朝獨攻成都克而有之【朝真遙翻】建表請陳敬瑄田令孜罪不可赦【請恐當作稱考異曰十國紀年朝議以建不奉詔而不能制更授西川行營招討制置使按此命蓋在昭度還朝之後也】願畢命以圖成功昭度無如之何由是未能東還建說昭度曰今關東藩鎮迭相吞噬此腹心之疾也相公宜早歸廟堂與天子謀之敬瑄疥㿅耳【㿅與癬同音息淺翻】當以日月制之責建可辦也昭度猶豫未決庚子建陰令東川將唐友通等擒昭度親吏駱保於行府門臠食之【韋昭度攻成都置行府以治事臠力兖翻】云其盜軍糧昭度大懼遽稱疾以印節授建牒建知三使留後【三使節度使招撫使制置使也】兼行營招討使即日東還建送至新都跪觴馬前泣拜而別【跪觴跪而奉觴也】昭度甫出劒門【劒門諸葛亮立關唐聖歷二年分普安永歸陰平置劒門縣屬劒州九域志在州東北五十五里】即以兵守之不復内東軍【復扶又翻】昭度至京師除東郡留守 【考異曰舊紀龍紀元年正月昭度為東都留守實録大順二年三月乙亥復陳敬瑄官爵丙子以昭度為東都留守按昭度已除留守不領西川節度及招討使則便應釋兵東歸不應更留在彼縱使彊留諸軍亦安肯稟服王建亦何必更說之云相公宜早歸廟堂與天子籌之舊傳建脅說昭度奏請還都建以重兵守劒門急攻成都昭度還以檢校司空充東都留守新傳亦同蓋今年三月既復敬瑄官爵但召昭度還朝王建不肯罷兵昭度為所牽率亦同執奏以為敬瑄不可赦既而為建所脅授兵東歸朝廷責其進退失據故左遷留守如新舊傳所云者是也今從之又昭度初圍成都楊守亮為招討副使顧彦朗為行軍司馬王建為都指揮使同在成都城下及昭度東歸時獨建在彼以兵授之不見二人者按三月乙亥詔書但云令彦朗各歸本鎮則是守亮先已歸也彦朗得此詔必亦歸獨昭度與建留在彼耳然建令東川將唐友通食駱保是彦朗身歸而留兵攻成都也】建急攻成都環城烽塹亘五十里【環音患】有狗屠王鷂【鷂亦肖翻】請詐得罪亡入城說之【說式芮翻】使上下離心建遣之鷂入見陳敬瑄田令孜則言建兵疲食盡將遁矣出則鬻茶於市陰為吏民稱建英武兵勢彊盛【為于偽翻】由是敬瑄等懈於守備【懈古隘翻】而衆心危懼建又遣其將京兆鄭渥詐降以覘之敬瑄以為將使乘城既而復以詐得歸【復扶又翻又如字】建由是悉知城中虚實以渥為親從都指揮使更姓名曰王宗渥【從才用翻更工衡翻】以武安節度使周岳為嶺南西道節度使【方鎮表中和三年升】<br />
<br />
  【湖南觀察為欽化軍節度光啟元年改武安軍】 李克用大舉擊赫連鐸敗其兵於河上【北河之上敗補邁翻】進圍雲州 楊行密遣其將劉威朱延夀將兵三萬撃孫儒於黄池【九域志宣州當塗縣有黄池鎮】威等大敗延夀舒城人也孫儒軍於黄池五月大水諸營皆没乃還楊州使其將康據和州【于放翻】安景思據滁州【和滁相去一百五十里】 丙午立皇子祐為德王 楊行密遣其將李神福攻和滁康降【降戶江翻】安景思走 秋七月李克用急攻雲州赫連鐸食盡奔吐谷渾部【赫連鐸本吐谷渾酋長聞成中其父帥種人三千帳自歸守雲州十五年至是而亡 考異曰舊紀實録皆云克用率兵出井陘屯常山大掠深趙盧龍李匡威自率步騎萬餘援王鎔按唐太祖紀年録是時克用方攻赫連鐸既平雲州乃封王鎔實録蓋因舊紀之誤又紀年録曰七月太祖進軍至於柳城會赫連鐸力屈食盡奔入吐渾云云實録云克用遣將急攻雲州蓋以前云克用親討王鎔故也按紀年録討王鎔在後實録誤】既而歸於幽州克用表大將石善友為大同防禦使 朱全忠遣使與楊行密約共攻孫儒儒恃其兵彊欲先滅行密後敵全忠移牒藩鎮數行密全忠之罪且曰俟平宣汴當引兵入朝除君側之惡【數所具翻】於是悉焚揚州廬舍盡驅丁壯及婦女度江殺老弱以充食行密將張訓李德誠潛入揚州滅餘火得穀數十萬斛以賑饑民【揚州之民仇孫儒而德楊行密使孫儒不死於宣州揚州之民亦必歸楊行密矣】泗州刺史張諫貸數萬斛以給軍訓以行密之命饋之諫由是德行密【為張諫降行密張本】邢洺節度使李存孝勸李克用攻鎮州克用從之八月克用南巡澤潞遂涉懷孟之境 朱全忠遣其將丁會攻宿州克其外城【元年夏四月宿州將張筠附於時漙】 乙未孫儒自蘇州出屯廣德【沈約曰廣德縣疑吳所立劉昫曰廣德縣漢故障也宋分宣州之廣德吳興之故障置綏安縣唐至德二年改為廣德以縣界廣德故城為名屬宣州九域志在州東一百二十里】楊行密引兵拒之儒圍其寨行密將上蔡李簡帥百餘人力戰破寨拔行密出之【帥讀曰率】 王建攻陳敬瑄益急敬瑄出戰輒敗巡内州縣率為建所取威戎節度使楊晟時饋之食建以兵據新都彭州道絶【田令孜以彭州為威戎軍】敬瑄出慰勉士卒皆不應辛丑田令孜登城謂建曰老夫曏於公甚厚何見困如是建曰父子之恩豈敢忘【言令孜養建為假子也】但朝廷命建討不受代者不得不然儻大師改圖建復何求【大師謂陳敬瑄復扶又翻】是夕令孜自攜西川印節詣建營授之【舊書龍紀元年五月壬辰朔漢州刺史王建陷成都遷陳敬瑄於雅州建自稱兵馬留後復用田令孜為監軍記事既有不同而紀年前後復有兩年之差】將士皆呼萬歲建泣謝請復為父子如初先是建常誘其將士曰成都城中繁盛如花錦一朝得之金帛子女恣汝曹所取節度使與汝曹迭日為之耳【先悉薦翻】壬寅敬瑄開門迎建【禧宗廣明元年陳敬瑄鎮西川至是而亡】建署其將張勍為馬步斬斫使使先入城乃謂將士曰吾與汝曹三年百戰今始得城汝曹不憂不富貴慎勿焚掠坊市吾己委張勍護之矣彼幸執而白我我猶得救之若先斬而後白我亦不能救也既而士卒有犯令者勍執百餘人皆捶其胷而殺之積尸於市衆莫敢犯故時人謂勍為張打胷【勍渠京翻捶止橤翻】癸卯建入城自稱西川留後小校韓武數於使廳上馬牙司止之【使廳節度使廳事也牙司吏也掌使牙之事數所角翻上時掌翻】武怒曰司徒許我迭日為節度使上馬何為建密遣人刺殺之【刺七亦翻】初陳敬瑄之拒朝命也田令孜欲盜其軍政謂敬瑄曰三兄尊重【敬瑄第三朝直遙翻】軍務煩勞不若盡以相付日具記事咨呈兄但高居自逸而已敬瑄素無智能忻然許之自是軍事皆不由己以至於亡建表敬瑄子陶為雅州刺史使隨陶之官明年罷歸寓居新津以一縣租賦贍之癸丑建分遣士卒就食諸州更文武堅姓名曰王宗阮謝從本曰王宗本【更其姓名以為假子更工衡翻】陳敬瑄將佐有器幹者建皆禮而用之【史言王建所以能有蜀】 六軍十二衛觀軍容使左神策軍中尉楊復恭總宿衛兵專制朝政諸假子皆為節度使刺史又養宦官子六百人皆為監軍假子龍劒節度使守貞武定節度使守忠不輸貢賦上表訕薄朝廷【龍劒節度領龍劒利閬四州武定節度領洋果階扶四州】上舅王瓌求節度使上訪於復恭復恭以為不可瓌怒詬之【詬古翻又許翻】瓌出入禁中頗用事復恭惡之奏以為黔南節度使【是時以黔中節度為永泰軍黔中以南則羈縻諸蠻州矣未知黔南節度置於何所豈楊復恭欲殺王瓌特創置此鎮以授之邪惡烏路翻黔渠今翻】至吉柏津【利州益昌縣有桔柏津益昌驛有古柏土人謂之桔柏因以名津據楊復恭傳王瓌取道興元至桔柏津】令山南西道節度使楊守亮覆諸江中宗族賓客皆死以舟敗聞上知復恭所為深恨之李順節既寵貴與復恭争權盡以復恭陰事告上上乃出復恭為鳳翔監軍復恭愠懟不肯行【愠於運翻怒也懟直類翻怨也】稱疾求致仕九月乙卯以復恭為上將軍致仕賜以几杖使者致詔命還復恭潛遣腹心張綰刺殺之【刺七亦翻】 加護國節度使王重盈兼中書令 東川節度使顧彦朗薨軍中推其弟彦暉知留後 冬十月壬午宿州刺史張筠降於丁會 癸未以永平節度使王建為西川節度使甲申廢永平軍【去年置水平節鎮於邛州以授王建建既得西川授以西川節而廢永平軍建志也】建既得西川留心政事容納直言好施樂士【好呼到翻施式豉翻】用人各盡其才謙恭儉素然多忌好殺諸將有功名者多因事誅之 楊復恭居第近玉山營【據舊史楊復恭居第在昭化里近其靳翻】假子守信為玉山軍使數往省之【數所角翻省悉景翻】或告復恭與守信謀反乙酉上御安喜樓陳兵自衛命天威都將李順節神策軍使李守節將兵攻其第張綰帥家衆拒戰【家衆楊復恭私所蓄養之人也帥讀曰率】守信引兵助之順節等不能克丙戌禁兵守含光門【含光門皇城南面西來第一門也】俟其開欲出掠兩市遇劉崇望立馬諭之曰天子親在街東督戰汝曹皆宿衛之士當於樓前殺賊立功勿貪小利自取惡名衆皆曰諾遂從崇望而東守信之衆望見兵來遂潰走守信與復恭挈其族自通化門出趣興元【通化門長安城東面北來第一門趣七喻翻】永安都頭權安追之擒張綰斬之【永安都亦神策五十四都之一】復恭至興元楊守亮楊守忠楊守貞及綿州刺史楊守厚同舉兵拒朝廷以討李順節為名守厚亦復恭假子也 李克用攻王鎔大破鎮兵於龍尾崗斬獲萬計遂拔臨城攻元氏柏鄉【薛居正曰龍尾崗在臨城西北臨城本房子天寶元年更名與元氏柏鄉皆屬趙州九域志臨城在趙州西南一百三里 考異曰唐太祖紀年録曰攻元氏斬首千級進拔雹水攻柏鄉按雹水屬易州克用方攻鎮州以救易定必不取其地恐誤也】李匡威引幽州兵救之克用大掠而還軍於邢州 十一月曹州都將郭銖殺刺史郭詞降於朱全忠【曹州天平節度使朱瑄巡屬也】 泰寜節度使朱瑾將萬餘人攻單州【唐末以宋州之單父碭山曹州之成武兖州之魚臺置單州九域志兖州西南至單州二百八十里單州時屬朱全忠單音善】 乙丑時溥將劉知俊帥衆二千降於朱全忠【帥讀曰率】知俊沛人徐之驍將也溥軍自是不振全忠以知俊為左右開道指揮使 辛未夀州將劉弘鄂惡孫儒殘暴【惡烏路翻下同】舉州降朱全忠 十二月乙酉汴將丁會張歸霸與朱瑾戰於金鄉大破之殺獲殆盡瑾單騎走免 天威都將李順節恃恩驕横【横戶孟翻】出入常以兵自隨兩軍中尉劉景宣西門君遂惡之白上恐其作亂戊子二人以詔召順節順即入至銀臺門二人邀順節於仗舍坐語供奉官似先知自後斬其首【似先知宦官也舊書帝紀作部將嗣光審斬順節頭 考異曰唐補紀景福二年四月十七日夜見掃星長十丈餘承旨陳匡用奏當有亂臣將入宫内昭宗乳母名曰芥子自即位加夫人衆呼白婆左神策軍天威都軍使胡弘立先是軍中馬騎官巧佞取容朝廷達官多重之楊復恭為軍主與改姓名為楊守節主上每出畋遊經天威軍内其楊守節以憸巧趨附乞與主上為兒既而允從頗生驕縱於是引聖人入堂室令妻妾對於庭簷或入内中經旬不出致主有撫楹之咎為臣懷通室之非承醉奏云王印金箱兒未曾識望阿郎畧將宣示以慰平生其白婆在側曰此寶非凡人得見不用發言於是奏曰除此老嫗方應太平從此白婆得罪不見蹤由兩神策軍以其事漸乖必為大禍與諸王商議須急去除於重陽節向樞密院中排宴喚入謝恩却出宣化門供奉官似先知袖劒揮之諸王相次剚刃以為葅醢按胡弘立即順節也新舊紀及諸書景福二年皆無此事蓋程匡柔傳聞之誤今日從實録事則參取諸書】從者大譟而出【從才用翻】於是天威捧日登封三都大掠永寜坊【三都皆神策五十四都之數】至暮乃定百官表賀 孫儒焚掠蘇常引兵逼宣州錢鏐復遣兵據蘇州【蘇州自此為錢氏所有復扶又翻】儒屢敗楊行密之兵旌旗輜重亘百餘里【重直用翻】行密求救於錢鏐鏐以兵食助之 以顧彦暉為東川節度使遣中使宋道弼賜旌節楊守亮使楊守厚囚道弼奪旌節發兵攻梓州癸卯彦暉求救於王建甲辰建遣其將華洪李簡王宗侃王宗弼救東川【華戶化翻姓也】建密謂諸將曰爾等破賊彦暉必犒師汝曹於行營報宴因而執之無煩再舉宗侃破守厚七砦【砦與寨同音士賣翻】守厚走歸綿州彦暉具犒禮諸將報宴宗弼以建謀告之彦暉乃以疾辭初李茂貞養子繼臻據金州均州刺史馮行襲攻下之【九域志均州西至金州七百里】詔以行襲為昭信防禦使治金州【方鎮表僖宗光啟元年置昭信防禦於金州 考異曰薛居正五代史行襲破楊守亮兵詔升金州節鎮以戎昭為軍額即以行襲為節度使按實録光化元年正月始以昭信防禦使馮行襲為昭信節度使新方鎮表光啟元年升金商都防禦使為節度使是年罷節度置昭信軍防禦使治金州光化元年始升昭信軍防禦使為節度使天佑二年賜號戎昭軍薛史誤也】楊守亮欲自金商襲京師行襲逆撃大破之 是歲賜涇原軍號曰彰義 【考異曰新表在乾寜元年今從實録】增領渭武二州 福建觀察使陳巖疾病遣使以書召泉州刺史王潮欲授以軍政未至而巖卒巖妻弟都將范暉諷將士推已為留後 【考異曰蔣文懌閩中實録云大順中巖薨十國紀年在大順二年昭宗實録在明年三月恐約奏到今從閩中録十國紀年又薛史閩中録閩書皆云范暉巖壻餘書皆云妻弟林仁志王氏啟運圖載監軍程克論表云妻弟此最得實今從之】<br />
<br />
  資治通鑑卷二百五十八  <br>
   </div> 

<script src="/search/ajaxskft.js"> </script>
 <div class="clear"></div>
<br>
<br>
 <!-- a.d-->

 <!--
<div class="info_share">
</div> 
-->
 <!--info_share--></div>   <!-- end info_content-->
  </div> <!-- end l-->

<div class="r">   <!--r-->



<div class="sidebar"  style="margin-bottom:2px;">

 
<div class="sidebar_title">工具类大全</div>
<div class="sidebar_info">
<strong><a href="http://www.guoxuedashi.com/lsditu/" target="_blank">历史地图</a></strong>  
<a href="http://www.880114.com/" target="_blank">英语宝典</a>  
<a href="http://www.guoxuedashi.com/13jing/" target="_blank">十三经检索</a> 
<br><strong><a href="http://www.guoxuedashi.com/gjtsjc/" target="_blank">古今图书集成</a></strong> 
<a href="http://www.guoxuedashi.com/duilian/" target="_blank">对联大全</a> <strong><a href="http://www.guoxuedashi.com/xiangxingzi/" target="_blank">象形文字典</a></strong> 

<br><a href="http://www.guoxuedashi.com/zixing/yanbian/">字形演变</a>  <strong><a href="http://www.guoxuemi.com/hafo/" target="_blank">哈佛燕京中文善本特藏</a></strong>
<br><strong><a href="http://www.guoxuedashi.com/csfz/" target="_blank">丛书&方志检索器</a></strong> <a href="http://www.guoxuedashi.com/yqjyy/" target="_blank">一切经音义</a>  

<br><strong><a href="http://www.guoxuedashi.com/jiapu/" target="_blank">家谱族谱查询</a></strong>  <strong><a href="http://shufa.guoxuedashi.com/sfzitie/" target="_blank">书法字帖欣赏</a></strong> 
<br>

</div>
</div>


<div class="sidebar" style="margin-bottom:0px;">

<font style="font-size:22px;line-height:32px">QQ交流群9:489193090</font>


<div class="sidebar_title">手机APP 扫描或点击</div>
<div class="sidebar_info">
<table>
<tr>
	<td width=160><a href="http://m.guoxuedashi.com/app/" target="_blank"><img src="/img/gxds-sj.png" width="140"  border="0" alt="国学大师手机版"></a></td>
	<td>
<a href="http://www.guoxuedashi.com/download/" target="_blank">app软件下载专区</a><br>
<a href="http://www.guoxuedashi.com/download/gxds.php" target="_blank">《国学大师》下载</a><br>
<a href="http://www.guoxuedashi.com/download/kxzd.php" target="_blank">《汉字宝典》下载</a><br>
<a href="http://www.guoxuedashi.com/download/scqbd.php" target="_blank">《诗词曲宝典》下载</a><br>
<a href="http://www.guoxuedashi.com/SiKuQuanShu/skqs.php" target="_blank">《四库全书》下载</a><br>
</td>
</tr>
</table>

</div>
</div>


<div class="sidebar2">
<center>


</center>
</div>

<div class="sidebar"  style="margin-bottom:2px;">
<div class="sidebar_title">网站使用教程</div>
<div class="sidebar_info">
<a href="http://www.guoxuedashi.com/help/gjsearch.php" target="_blank">如何在国学大师网下载古籍?</a><br>
<a href="http://www.guoxuedashi.com/zidian/bujian/bjjc.php" target="_blank">如何使用部件查字法快速查字?</a><br>
<a href="http://www.guoxuedashi.com/search/sjc.php" target="_blank">如何在指定的书籍中全文检索?</a><br>
<a href="http://www.guoxuedashi.com/search/skjc.php" target="_blank">如何找到一句话在《四库全书》哪一页?</a><br>
</div>
</div>


<div class="sidebar">
<div class="sidebar_title">热门书籍</div>
<div class="sidebar_info">
<a href="/so.php?sokey=%E8%B5%84%E6%B2%BB%E9%80%9A%E9%89%B4&kt=1">资治通鉴</a> <a href="/24shi/"><strong>二十四史</strong></a>&nbsp; <a href="/a2694/">野史</a>&nbsp; <a href="/SiKuQuanShu/"><strong>四库全书</strong></a>&nbsp;<a href="http://www.guoxuedashi.com/SiKuQuanShu/fanti/">繁体</a>
<br><a href="/so.php?sokey=%E7%BA%A2%E6%A5%BC%E6%A2%A6&kt=1">红楼梦</a> <a href="/a/1858x/">三国演义</a> <a href="/a/1038k/">水浒传</a> <a href="/a/1046t/">西游记</a> <a href="/a/1914o/">封神演义</a>
<br>
<a href="http://www.guoxuedashi.com/so.php?sokeygx=%E4%B8%87%E6%9C%89%E6%96%87%E5%BA%93&submit=&kt=1">万有文库</a> <a href="/a/780t/">古文观止</a> <a href="/a/1024l/">文心雕龙</a> <a href="/a/1704n/">全唐诗</a> <a href="/a/1705h/">全宋词</a>
<br><a href="http://www.guoxuedashi.com/so.php?sokeygx=%E7%99%BE%E8%A1%B2%E6%9C%AC%E4%BA%8C%E5%8D%81%E5%9B%9B%E5%8F%B2&submit=&kt=1"><strong>百衲本二十四史</strong></a>  <a href="http://www.guoxuedashi.com/so.php?sokeygx=%E5%8F%A4%E4%BB%8A%E5%9B%BE%E4%B9%A6%E9%9B%86%E6%88%90&submit=&kt=1"><strong>古今图书集成</strong></a>
<br>

<a href="http://www.guoxuedashi.com/so.php?sokeygx=%E4%B8%9B%E4%B9%A6%E9%9B%86%E6%88%90&submit=&kt=1">丛书集成</a> 
<a href="http://www.guoxuedashi.com/so.php?sokeygx=%E5%9B%9B%E9%83%A8%E4%B8%9B%E5%88%8A&submit=&kt=1"><strong>四部丛刊</strong></a>  
<a href="http://www.guoxuedashi.com/so.php?sokeygx=%E8%AF%B4%E6%96%87%E8%A7%A3%E5%AD%97&submit=&kt=1">說文解字</a> <a href="http://www.guoxuedashi.com/so.php?sokeygx=%E5%85%A8%E4%B8%8A%E5%8F%A4&submit=&kt=1">三国六朝文</a>
<br><a href="http://www.guoxuedashi.com/so.php?sokeytm=%E6%97%A5%E6%9C%AC%E5%86%85%E9%98%81%E6%96%87%E5%BA%93&submit=&kt=1"><strong>日本内阁文库</strong></a> <a href="http://www.guoxuedashi.com/so.php?sokeytm=%E5%9B%BD%E5%9B%BE%E6%96%B9%E5%BF%97%E5%90%88%E9%9B%86&ka=100&submit=">国图方志合集</a> <a href="http://www.guoxuedashi.com/so.php?sokeytm=%E5%90%84%E5%9C%B0%E6%96%B9%E5%BF%97&submit=&kt=1"><strong>各地方志</strong></a>

</div>
</div>


<div class="sidebar2">
<center>

</center>
</div>
<div class="sidebar greenbar">
<div class="sidebar_title green">四库全书</div>
<div class="sidebar_info">

《四库全书》是中国古代最大的丛书,编撰于乾隆年间,由纪昀等360多位高官、学者编撰,3800多人抄写,费时十三年编成。丛书分经、史、子、集四部,故名四库。共有3500多种书,7.9万卷,3.6万册,约8亿字,基本上囊括了古代所有图书,故称“全书”。<a href="http://www.guoxuedashi.com/SiKuQuanShu/">详细>>
</a>

</div> 
</div>

</div>  <!--end r-->

</div>
<!-- 内容区END --> 

<!-- 页脚开始 -->
<div class="shh">

</div>

<div class="w1180" style="margin-top:8px;">
<center><script src="http://www.guoxuedashi.com/img/plus.php?id=3"></script></center>
</div>
<div class="w1180 foot">
<a href="/b/thanks.php">特别致谢</a> | <a href="javascript:window.external.AddFavorite(document.location.href,document.title);">收藏本站</a> | <a href="#">欢迎投稿</a> | <a href="http://www.guoxuedashi.com/forum/">意见建议</a> | <a href="http://www.guoxuemi.com/">国学迷</a> | <a href="http://www.shuowen.net/">说文网</a><script language="javascript" type="text/javascript" src="https://js.users.51.la/17753172.js"></script><br />
  Copyright &copy; 国学大师 古典图书集成 All Rights Reserved.<br>
  
  <span style="font-size:14px">免责声明:本站非营利性站点,以方便网友为主,仅供学习研究。<br>内容由热心网友提供和网上收集,不保留版权。若侵犯了您的权益,来信即刪。scp168@qq.com</span>
  <br />
ICP证:<a href="http://www.beian.miit.gov.cn/" target="_blank">鲁ICP备19060063号</a></div>
<!-- 页脚END --> 
<script src="http://www.guoxuedashi.com/img/plus.php?id=22"></script>
<script src="http://www.guoxuedashi.com/img/tongji.js"></script>

</body>
</html>
