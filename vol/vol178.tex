<!DOCTYPE html PUBLIC "-//W3C//DTD XHTML 1.0 Transitional//EN" "http://www.w3.org/TR/xhtml1/DTD/xhtml1-transitional.dtd">
<html xmlns="http://www.w3.org/1999/xhtml">
<head>
<meta http-equiv="Content-Type" content="text/html; charset=utf-8" />
<meta http-equiv="X-UA-Compatible" content="IE=Edge,chrome=1">
<title>資治通鑒_179-資治通鑑卷一百七十八_179-資治通鑑卷一百七十八</title>
<meta name="Keywords" content="資治通鑒_179-資治通鑑卷一百七十八_179-資治通鑑卷一百七十八">
<meta name="Description" content="資治通鑒_179-資治通鑑卷一百七十八_179-資治通鑑卷一百七十八">
<meta http-equiv="Cache-Control" content="no-transform" />
<meta http-equiv="Cache-Control" content="no-siteapp" />
<link href="/img/style.css" rel="stylesheet" type="text/css" />
<script src="/img/m.js?2020"></script> 
</head>
<body>
 <div class="ClassNavi">
<a  href="/24shi/">二十四史</a> | <a href="/SiKuQuanShu/">四库全书</a> | <a href="http://www.guoxuedashi.com/gjtsjc/"><font  color="#FF0000">古今图书集成</font></a> | <a href="/renwu/">历史人物</a> | <a href="/ShuoWenJieZi/"><font  color="#FF0000">说文解字</a></font> | <a href="/chengyu/">成语词典</a> | <a  target="_blank"  href="http://www.guoxuedashi.com/jgwhj/"><font  color="#FF0000">甲骨文合集</font></a> | <a href="/yzjwjc/"><font  color="#FF0000">殷周金文集成</font></a> | <a href="/xiangxingzi/"><font color="#0000FF">象形字典</font></a> | <a href="/13jing/"><font  color="#FF0000">十三经索引</font></a> | <a href="/zixing/"><font  color="#FF0000">字体转换器</font></a> | <a href="/zidian/xz/"><font color="#0000FF">篆书识别</font></a> | <a href="/jinfanyi/">近义反义词</a> | <a href="/duilian/">对联大全</a> | <a href="/jiapu/"><font  color="#0000FF">家谱族谱查询</font></a> | <a href="http://www.guoxuemi.com/hafo/" target="_blank" ><font color="#FF0000">哈佛古籍</font></a> 
</div>

 <!-- 头部导航开始 -->
<div class="w1180 head clearfix">
  <div class="head_logo l"><a title="国学大师官网" href="http://www.guoxuedashi.com" target="_blank"></a></div>
  <div class="head_sr l">
  <div id="head1">
  
  <a href="http://www.guoxuedashi.com/zidian/bujian/" target="_blank" ><img src="http://www.guoxuedashi.com/img/top1.gif" width="88" height="60" border="0" title="部件查字,支持20万汉字"></a>


<a href="http://www.guoxuedashi.com/help/yingpan.php" target="_blank"><img src="http://www.guoxuedashi.com/img/top230.gif" width="600" height="62" border="0" ></a>


  </div>
  <div id="head3"><a href="javascript:" onClick="javascript:window.external.AddFavorite(window.location.href,document.title);">添加收藏</a>
  <br><a href="/help/setie.php">搜索引擎</a>
  <br><a href="/help/zanzhu.php">赞助本站</a></div>
  <div id="head2">
 <a href="http://www.guoxuemi.com/" target="_blank"><img src="http://www.guoxuedashi.com/img/guoxuemi.gif" width="95" height="62" border="0" style="margin-left:2px;" title="国学迷"></a>
  

  </div>
</div>
  <div class="clear"></div>
  <div class="head_nav">
  <p><a href="/">首页</a> | <a href="/ShuKu/">国学书库</a> | <a href="/guji/">影印古籍</a> | <a href="/shici/">诗词宝典</a> | <a   href="/SiKuQuanShu/gxjx.php">精选</a> <b>|</b> <a href="/zidian/">汉语字典</a> | <a href="/hydcd/">汉语词典</a> | <a href="http://www.guoxuedashi.com/zidian/bujian/"><font  color="#CC0066">部件查字</font></a> | <a href="http://www.sfds.cn/"><font  color="#CC0066">书法大师</font></a> | <a href="/jgwhj/">甲骨文</a> <b>|</b> <a href="/b/4/"><font  color="#CC0066">解密</font></a> | <a href="/renwu/">历史人物</a> | <a href="/diangu/">历史典故</a> | <a href="/xingshi/">姓氏</a> | <a href="/minzu/">民族</a> <b>|</b> <a href="/mz/"><font  color="#CC0066">世界名著</font></a> | <a href="/download/">软件下载</a>
</p>
<p><a href="/b/"><font  color="#CC0066">历史</font></a> | <a href="http://skqs.guoxuedashi.com/" target="_blank">四库全书</a> |  <a href="http://www.guoxuedashi.com/search/" target="_blank"><font  color="#CC0066">全文检索</font></a> | <a href="http://www.guoxuedashi.com/shumu/">古籍书目</a> | <a   href="/24shi/">正史</a> <b>|</b> <a href="/chengyu/">成语词典</a> | <a href="/kangxi/" title="康熙字典">康熙字典</a> | <a href="/ShuoWenJieZi/">说文解字</a> | <a href="/zixing/yanbian/">字形演变</a> | <a href="/yzjwjc/">金 文</a> <b>|</b>  <a href="/shijian/nian-hao/">年号</a> | <a href="/diming/">历史地名</a> | <a href="/shijian/">历史事件</a> | <a href="/guanzhi/">官职</a> | <a href="/lishi/">知识</a> <b>|</b> <a href="/zhongyi/">中医中药</a> | <a href="http://www.guoxuedashi.com/forum/">留言反馈</a>
</p>
  </div>
</div>
<!-- 头部导航END --> 
<!-- 内容区开始 --> 
<div class="w1180 clearfix">
  <div class="info l">
   
<div class="clearfix" style="background:#f5faff;">
<script src='http://www.guoxuedashi.com/img/headersou.js'></script>

</div>
  <div class="info_tree"><a href="http://www.guoxuedashi.com">首页</a> > <a href="/SiKuQuanShu/fanti/">四库全书</a>
 > <h1>资治通鉴</h1> <!--         下载:【右键另存为】即可 --></div>
  <div class="info_content zj clearfix">
  
<div class="info_txt clearfix" id="show">
<center style="font-size:24px;">179-資治通鑑卷一百七十八</center>
    資治通鑑卷一百七十八 宋 司馬光 撰<br />
<br />
  胡三省 音注<br />
<br />
  隋紀二【起玄黓困敦盡屠維協洽凡八年】<br />
<br />
  高祖文皇帝上之下<br />
<br />
  開皇十二年春二月己巳以蜀王秀為内史令兼右領軍大將軍國子博士何妥與尚書右僕射邳公蘇威爭議事積不相能威子夔為太子通事舍人【隋制太子通事舍人八人屬典書坊】少敏辯有盛名【少詩沼翻】士大夫多附之及議樂夔與妥各有所持詔百僚署其所同百僚以威故同夔者什八九妥恚曰吾席間函丈四十餘年【禮侍坐於先生席間函文何妥周武帝時已為太學博士故云然恚於避翻】反為昨暮兒之所屈邪【邪音耶】遂奏威與禮部尚書盧愷吏部侍郎薛道衡尚書右丞王弘考功侍郎李同和等共為朋黨【吏部侍郎考功侍郎皆屬吏部尚書尚書左右丞分司管轄隋制尚書二十四曹侍郎獨吏部侍郎班左右丞之上吏部侍郎正四品左右丞從四品】省中呼弘為世子同和為叔言二人如威之子弟也復言威以曲道任其從父弟徹肅罔冒為官等數事【復扶又翻從才用翻】上命蜀王秀上柱國虞慶則等雜案之事頗有狀上大怒秋七月乙巳威坐免官爵以開府儀同三司就第盧愷除名知名之士坐威得罪者百餘人初周室以來選無清濁【選宣絹翻】及愷攝吏部【按愷傳開皇九年拜禮部尚書攝吏部尚書】與薛道衡甄别士流【别被列翻】故涉朋黨之謗以至得罪未幾【幾居豈翻】上曰蘇威德行者【行下孟翻】但為人所誤耳命之通籍【通籍殿中則得預朝請】威好立條章【好呼到翻】每歲責民間五品不遜【孔安國曰五品謂五常遜順也】或答云管内無五品之家其不相應領類多如此又為餘糧簿欲使有無相贍民部侍郎朗茂以為煩迂不急皆奏罷之茂基之子也【郎基見一百六十五卷梁世祖承聖三年】嘗為衛國令有民張元預兄弟不睦丞尉請加嚴刑【隋志縣置令丞尉】茂曰元預兄弟本相憎疾又坐得罪彌益其忿非化民之意也乃徐諭之以義元預等各感悔頓首請罪遂相親睦稱為友悌 己巳上享太廟【隋立四親廟各以孟月享以太牢】 壬申晦日有食之 帝以天下用律者多踳駮【踳乖也駮錯也踳尺允翻駮北角翻】罪同論異八月甲戌制諸州死辠不得輒決悉移大理案覆事盡【盡竟也】然後上省奏裁【上時掌翻】 冬十月壬午上享太廟 十一月辛亥祀南郊 己未新義公韓擒虎卒【擒虎襲父雄爵新義郡公平陳之功以吏議不加封爵卒子恤翻】 十二月乙酉以内史令楊素為尚書右僕射與高熲專掌朝政素性疎辯高下在心朝臣之内【朝直遙翻】頗推高熲敬牛弘厚接薛道衡視蘇威蔑如也【蔑無也視之如無也又輕易也】自餘朝貴多被陵轢【陵乘也犯也侮也侵也陵轢踐也又車踐為轢轢郎擊翻】其才藝風調優於熲【調徒鈞翻】至于推誠體國處物平當【處昌呂翻當丁浪翻】有宰相識度不如熲遠矣右領軍大將軍賀若弼自謂功名出朝廷之右每以宰相自許既而楊素為僕射弼仍為將軍甚不平形於言色由是坐免官怨望愈甚久之上下弼獄【下戶嫁翻】謂之曰我以高熲楊素為宰相汝每昌言曰此二人惟堪啗飯耳【昌言明言于廣衆啗徒濫翻又徒覽翻】是何意也弼曰熲臣之故人素臣之舅子臣並知其為人誠有此語公卿奏弼怨望罪當死上曰臣下守法不移公可自求活理弼曰臣恃至尊威靈將八千兵度江【將即亮翻】擒陳叔寶竊以此望活上曰此巳格外重賞何用追論弼曰臣已蒙格外重賞今還格外望活既而上低回數日【低降意也回轉心也】惜其功特令除名歲餘復其爵位上亦忌之不復任使【復扶又翻】然每宴賜遇之甚厚 有司上言府藏皆滿【上時掌翻藏徂浪翻】無所容積于廊廡【廡罔甫翻】帝曰朕既薄賦於民又大經賜用【謂賞平陳將士】何得爾也【爾猶言如此】對曰入者常多於出略計每年賜用至數百萬段曾無減損於是更闢左藏院以受之【漢官有中藏令晉有中黃左右藏令隋初有右藏黃藏令至是始闢左藏院藏徂浪翻】詔曰寜積於人無藏府庫河北河東今年田租三分減一兵減半功調全免【田出租丁出調詳已見前兵受田計畝為功以其所出修器械備糗糧今已減其半調徒弔翻】時天下戶口歲增京輔及三河地少而人衆【京輔謂關内三河謂河東河南河北少與小同】衣食不給帝乃發使四出均天下之田其狹鄉每丁纔至二十畝老少又少焉【使疏吏翻老少詩照翻又少音詩沼翻】<br />
<br />
  十三年春正月壬子上祀感生帝【隋以火德王以赤帝赤熛怒為感生帝】壬戌行幸岐州【岐州扶風郡】 二月丙午詔營仁夀宫於岐州之北【仁夀宫在岐州普閏縣】使楊素監之【監古衘翻】素奏前萊州刺史宇文愷檢校將作大匠【隋志東萊郡舊置光州開皇五年更名萊州隋制未除授正官而領其務者為檢校官將作大匠掌工作宇文愷有巧思奏使之領作】記室封德彝為土木監【土木監掌土木之事以營宫暫置之非常設之官】於是夷山堙谷以立宫殿崇臺累榭宛轉相屬【屬之欲翻】役使嚴急丁夫多死疲頓顛仆推填坑坎覆以土石【推吐雷翻覆敷又翻】因而築為平地死者以萬數 丁亥上至自岐州 己卯立皇孫為豫章王廣之子也【古限翻】 丁酉制私家不得藏緯候圖讖【讖楚譖翻】 秋七月戊辰晦日有食之 是歲上命禮部尚書牛弘等議明堂制度宇文愷獻明堂木様上命有司規度安業里地將立之而諸儒異議久之不決乃罷之【隋志宇文愷依月令文造明堂木様重檐複廟五房四逹丈尺規矩皆有凖憑帝異之命有司於安業里為規兆蓋在長安南郭内也既以異議罷至大業中愷復奏明堂議及木様其議云尚書帝命驗曰帝者承天立五府以尊天重象赤曰文祖黃曰神斗白曰顯紀黑曰玄矩蒼曰靈府注曰唐虞之天府夏之世室殷之重屋周之明堂皆同矣尸子曰有虞氏曰總章周官考工記曰夏后氏世室堂修四七博四脩一注云脩南北之深也夏度以步今堂脩十四步其博益以四分修之一則明堂博十七步半也臣愷按三王之世夏最為近古從質尚文理應漸就寛大何因夏室乃大殷堂相形為論理恐不爾記云堂脩七博四脩若夏度以步則應脩七步注云今堂脩十四步乃是增益記文殷周二堂獨無加字便是其義類例不同山東禮本輒加二七之字何獨殷無加尋之文周闕增筵之義研覈其趣或是不然讐校古書並無二字此乃桑間俗儒信情加減黃圖議云夏后氏益其堂之大一百四十四尺周人明堂以為兩杼間馬宫之言止論堂之一面㨿此為凖則三代堂基並方得為上圓之制諸書所說並為下方鄭注周官獨為此義非直與古違異亦乃乖背禮文尋文求理深恐未愜尸子曰殷人陽館考工記曰殷人重屋堂脩七尋堂崇三尺四阿重屋注曰其脩七尋五丈六尺放夏周則其博九尋七丈二尺又曰周人明堂度九尺之筵東西九筵南北七筵堂崇一筵五室凡二筵禮記明堂位曰天子之廟複廟重檐鄭注云複廟重屋也注玉藻云天子廟及路寢皆如明堂制禮圖云於内室之上起通天之觀觀八十一尺得宫之數其聲濁君之象也大戴禮云明堂者古有之凡九室一室有四戶八牖以茅蓋上圓下方外水曰璧雝赤綴戶白綴牖堂高三尺東西九仞南北七筵其宫方三百步周書明堂曰堂方百一十二尺高四尺階博六尺三寸室居内方百尺室内方六十尺戶高八尺博四尺作洛曰明堂太廟路寤咸有四阿重亢重廊孔氏注云重亢累棟重廊累屋也禮圖曰秦明堂九室十二階各有所居呂氏春秋曰有十二堂與月令同並不論尺丈臣愷案十二階雖不與禮合一月一階非無理思黄圖曰堂方百四十四尺坤之策也方象地屋圓楣徑二百一十六尺法乾之策也圓象天室九宫法九州太室方六丈法坤之變數十二堂法十二月三十六戶法極隂之變數七十二牖法五行所行日數八逹象八風法八卦通天臺徑九尺象乾以九覆六高八十一尺法黃鍾九九之數二十八柱象二十八宿堂高三尺土階三等法三統堂四向五色法四時五行殿門去殿七十二步法五行所行門堂長四丈取太室三之二垣高無蔽目之照目恐當作日牖六尺其外倍之殿垣方在水内法地隂也水四周於外象四海圓法陽也水闊二十四丈象二十四氣水内徑三丈應覲禮經武帝立明堂汶上無室其外畧依此制太山通議今亡不可得而辨也元始四年起明堂辟雍長安城南門制度如儀一殿垣四面門八觀水外周堤壤高四尺禮圖曰建武三十年作明堂上圓下方上圓法天下方法地十二堂法日辰九室法九州室八牕八九七十二法一時之王室有二戶二九十八戶法土王十八日内堂正壇高三尺土階三等胡伯始注漢官云古清廟蓋以茅今蓋以瓦下藉茅以存古制自晉以前未有鵄尾其圓墻璧水一依本圖晉堂方構不合天文既缺重樓又無璧水空堂乖五室之義直殿違九階之文非古欺天一何過甚後魏於北臺城南造圓墻在璧水外門在水内迥立不與墻相連其堂上九室三三相重不依古制室間通巷違舛處多其室皆用墼累極成褊陋宋起居注曰孝武帝大明五年立明堂其墻宇規範擬則太廟唯十二間以應朞數梁武帝移宋時太極殿以為明堂無室十二間自古明堂圖惟有二本一是宗周劉熙阮諶劉昌宗等作三圖畧同一是後漢建武三十年作禮圖有本不詳撰人臣遠尋經傳傍求子史研究衆說摠撰今圖其様以木為之下為方堂堂有五室上為圓觀觀有四門會遼東之役不果行】上之滅陳也以陳叔寶屏風賜突厥大義公主【厥九勿翻】公主以其宗國之覆【謂周亡也】心常不平書屏風為詩叙陳亡以自寄上聞而惡之【惡烏路翻】禮賜漸薄彭公劉昶先尚周公主流人楊欽亡入突厥詐言昶欲與其妻作亂攻隋遣欽密告大義公主發兵擾邊都藍可汗信之乃不修職貢頗為邊患【可從刋入聲汗音寒】上遣車騎將軍長孫晟使於突厥【隋制車騎將軍階正五品非職事官騎奇寄翻使疏吏翻長知兩翻晟承正翻】微觀察之公主見晟言辭不遜又遣所私胡人安遂迦與楊欽計議【迦求伽翻】扇惑都藍晟至京師具以狀聞上遣晟往索欽【索山客翻】都藍不與曰檢校客内無此色人晟乃賂其逹官知欽所在夜掩獲之以示都藍因發公主私事國人大以為恥都藍執安遂迦等并以付晟上大喜加授開府儀同三司仍遣入突厥廢公主内史侍郎裴矩請說都藍使殺公主【說輸芮翻】時處羅侯之子染干號突利可汗【考異曰突厥傳云沙鉢畧子今從長孫晟傳】居北方遣使求婚【使疏吏翻】上使裴矩謂之曰當殺大義公主乃許婚突利復譖之於都藍【復扶又翻】都藍因發怒殺公主更表請婚朝議將許之【朝直遥翻】長孫晟曰臣觀雍虞閭反覆無信直以與玷厥有隙【雍虞閭都藍玷厥逹頭也】所以欲依倚國家雖與為婚終當叛去今若得尚公主承藉威靈玷厥染干必受其徵發彊而更反後恐難圖且染干者處羅侯之子素有誠款於今兩代前乞通婚不如許之招令南徙兵少力弱易可撫馴【少詩沼翻易以䜴翻馴松倫翻】使敵雍虞閭以為邊捍上曰善復遣晟慰諭染干許尚公主【為隋破都藍樹立染干張本復扶又翻】 牛弘使協律郎范陽祖孝孫等參定雅樂【隋制太常有協律郎二人隋志涿郡涿縣舊置范陽郡開皇初郡廢又上谷郡淶水縣舊曰遒開皇元年以范陽為遒縣更置范陽於此】從陳陽山太守毛爽受京房律法【從字之上更有孝孫二字文意乃明隋志南海郡含洭縣梁置陽山郡】布管飛灰順月皆驗又每律生五音十二律為六十音因而六之為三百六十音分直一歲之日以配七音而旋相為宫之法由是著名【名一作明】弘等乃奏請復用旋宫法上猶記何妥之言【妥言見上卷九年】注弘奏下不聽作旋宫但用黃鍾一宫于是弘等復為奏附順上意其前代金石並銷毁之以息異議弘等又作武舞以象隋之功德郊廟饗用一調迎氣用五調【郊廟用一調止用黄鍾一宫迎氣用五調春用角夏用徵中央用宫秋用商冬用羽調徒釣翻】舊工稍盡其餘聲律皆不復通【復扶又翻】十四年春三月樂成夏四月乙丑詔行新樂且曰民間音樂流僻日久棄其舊體競造繁聲宜加禁約務存其本萬寶常聽太常所奏樂泫然泣曰樂聲淫厲而哀天下不久將盡時四海全盛聞者皆謂不然大業之末其言卒驗【卒子恤翻】寶常貧而無子久之竟餓死且死悉取其書燒之【寶常撰樂譜六十四卷具論八音旋相為宫之法改絃移柱之變為八十四調一百四十四律終於千八百聲為之應手成曲】曰用此何為 先是臺省府寺及諸州皆置公廨錢【先悉薦翻廨古隘翻】收息取給工部尚書蘇孝慈【唐六典工部尚書周之冬官卿也漢五曹尚書其三曰民曹後漢以民曹兼主繕修工作鹽池園苑之事自晉宋齊梁陳營宗廟則權置起部尚書事畢省之後周依周官置大司空卿一人隋開皇三年始置工部尚書】以為官司出舉興生煩擾百姓敗損風俗【敗補邁翻】請皆禁止給地以營農上從之六月丁卯始詔公卿以下皆給職田【職分田起于後周頃畝以品為差下至隋唐代有增減】毋得治生與民爭利【治直之翻】 秋七月乙未以邳公蘇威為納言 初張賓歷既行【開皇四年行張賓歷見一百七十六卷陳長城公至德二年】廣平劉孝孫【隋志武安郡永平縣舊曰廣平置廣平郡仁夀元年改永平縣】冀州秀才劉焯【信都郡置冀州焯職略翻】並言其失賓方有寵於上劉暉附會之共短孝孫斥罷之後賓卒孝孫為掖縣丞【隋志萊州東萊郡治掖縣】委官入京上其事詔留直太史【以它官入太史曹為直太史上時掌翻】累年不調【調徒釣翻】乃抱其書使弟子輿櫬來詣闕下【櫬初覲翻】伏而慟哭執法拘而奏之帝異焉以問國子祭酒何妥妥言其善乃遣與賓歷比校短長直太史勃海張胄玄【隋志勃海郡開皇六年置棣州】與孝孫共短賓歷異論鋒起久之不定上令參問日食事楊素等奏太史凡奏日食二十有五率皆無驗胄玄所刻前後妙中【刻刻定也中竹仲翻】孝孫所刻驗亦過半于是上引孝孫胄玄等親自勞徠【勞力到翻】孝孫請先斬劉暉乃可定歷帝不懌又罷之孝孫尋卒【卒子恤翻】 關中大旱民飢上遣左右視民食得豆屑雜糠以獻上流涕以示羣臣深自咎責為之不御酒肉【為于偽翻】殆將一朞八月辛未上帥民就食於洛陽【帥讀曰率】勅斥候不得輒有驅逼男女參厠於仗衛之間遇扶老攜幼者輒引馬避之慰勉而去至艱險之處見負擔者【擔都濫翻】令左右扶助之 冬閏十月甲寅詔以齊梁陳宗祀廢絶命高仁英蕭宗陳叔寶以時修祭所須器物有司給之陳叔寶從帝登邙山【邙山在洛陽城北】侍飲賦詩曰日月光天德山河壯帝居太平無以報願上東封書【上時掌翻】并表請封禪帝優詔答之它日復侍宴【復扶又翻】及出帝目之曰此敗豈不由酒以作詩之功何如思安時事當賀若弼度京口彼人密唘告急【度京口事見上卷九年】叔寶飲酒遂不之省【省悉井翻】高熲至日猶見唘在牀下未開封此誠可笑蓋天亡之也昔苻氏征伐所得國皆榮貴其主【謂苻堅也事見晉紀】苟欲求名不知違天命與之官乃違天也 齊州刺史盧賁【隋志齊郡舊曰齊州治歷城】坐民飢閉民糶【糶它弔翻】除名帝後復欲授以一州賁對詔失旨又有怨言帝大怒遂不用皇太子為言此輩並有佐命功雖性行輕險【為于偽翻行下孟翻】誠不可弃帝曰我抑屈之全其命也微劉昉鄭譯盧賁柳裘皇甫績等則我不至此然此等皆反覆子也當周宣帝時以無賴得幸及帝大漸顔之儀等請以趙王輔政此輩行詐顧命於我【事見一百七十四卷陳宣帝太建十二年】我將為政又欲亂之故昉謀大逆譯為巫蠱【考異曰盧賁傳云昉為大逆於前譯為巫蠱於後案譯傳譯以開皇元年坐巫蠱廢昉以六年坐謀反誅賁】<br />
<br />
  【傳誤也】如賁之例皆不滿志任之則不遜置之則怨望自為難信非我棄之衆人見此謂我薄於功臣斯不然矣賁遂廢卒于家【卒子恤翻】 晉王廣帥百官抗表固請封禪【帥讀曰率】帝令牛弘創定儀注既成帝視之曰茲事體大朕何德以堪之但當東巡因致祭泰山耳十二月乙未車駕東巡 上好機祥小數【好呼到翻禨居希翻】上儀同三司蕭吉上書曰甲寅乙卯天地之合也【吉上時掌翻】今茲甲寅之年以辛酉朔旦冬至來年乙卯以甲子夏至冬至陽始郊天之日即至尊本命夏至隂始祀地之辰即皇后本命至尊德並乾之覆育【覆敷又翻】皇后仁同地之載養所以二儀元氣並會本辰上大悦賜物五百段吉懿之孫也【蕭懿梁武帝之兄追封長沙王】員外散騎侍郎王劭言上有龍顔戴干之表【劭云乾鑿度云泰表戴干鄭玄注云表者人形體之彰識也干盾也泰人之表戴干散悉亶翻騎奇寄翻】指示羣臣上悦拜著作郎【隋志祕書省領太史著作二曹著作曹置郎二人】劭前後上表【上時掌翻】言上受命符瑞甚衆又採民間歌謡引圖書讖緯捃摭佛經【讖楚譖翻捃居運翻摭之石翻】回易文字曲加誣飾撰皇隋靈感志三十卷奏之上令宣示天下劭集諸州朝集使盥手焚香而讀之曲折其聲有如歌詠經涉旬朔徧而後罷上益喜前後賞賜優洽【洽音狹朝直遥翻使疏吏翻】十五年春正月壬戌車駕頓齊州庚午為壇于泰山柴燎祀天以歲旱謝愆咎禮如南郊又親祀青帝壇赦天下 二月丙辰收天下兵器敢私造者坐之關中緣邊不在其例 三月己未至自東巡 仁夀宫成丁亥上幸仁夀宫時天暑役夫死者相次於道楊素悉焚除之上聞之不悦及至見制度壯麗大怒曰楊素殫民力為離宫為吾結怨天下【為吾于偽翻】素聞之惶恐慮獲譴以告封德彛曰公勿憂俟皇后至必有恩詔 【考異曰隋書北史皆曰宫成上令高熲前視奏稱頗傷綺麗太損人丁帝不悦素懼即於北門唘獨孤皇后曰帝王法有離宫别館今天下太平造一宫何足損費后以此理喻上上乃解今從唐書】明日上果召素入對獨孤后勞之曰【勞力到翻】公知吾夫婦老無以自娛盛飾此宫豈非忠孝賜錢百萬錦絹三千段素負貴恃才多所陵侮唯賞重德彞每引之與論宰相職務終日忘倦因撫其牀曰封郎必須據吾此坐【楊素賞重封德彝非但以其算略蓋心術亦相似】屢薦於帝帝擢為内史舍人 夏四月己丑朔赦天下六月戊子詔鑿底柱【底柱山在陜縣北大河中水經曰河水通砥柱間注云砥柱山名也昔禹治洪水山陵當水者鑿之故破山以通河河水分流包山而過山見水中若柱然故曰砥柱三穿既決水流疎分指狀表目亦謂之三門】 庚寅相州刺史豆盧通貢綾文布【燕慕容精歸魏北人謂歸義為豆盧子孫以為氏相息亮翻】命焚之於朝堂【朝直遥翻】 秋七月納言蘇威坐從祀泰山不敬免俄而復位上謂羣臣曰世人言蘇威詐清家累金玉此妄言也然其性狠戾【狠戶墾翻】不切世要求名太甚從已則悦違之必怒此其大病耳 戊寅上至自仁夀宫 冬十月戊子以吏部尚書韋世康為荆州摠管世康洸之弟也【韋洸安輯嶺南卒於官案隋書世康傳世康洸之兄洸古黄翻】和静謙恕在吏部十餘年時稱廉平【按世康傳自禮部尚書轉吏部尚書在開皇四年之前七年拜襄州刺史歷安州信州揔管十三年入朝復拜吏部尚書出入踐揚前後十餘年不專在吏部也】常有止足之志謂子弟曰禄豈須多防滿則年不待暮有疾便辭固懇乞骸骨帝不許使鎮荆州時天下惟有四總管并揚益荆以晉秦蜀三王及世康為之當時以為榮 十一月辛酉上幸温湯【驪山温湯也程大昌曰皇堂石井後周宇文護所造隋文帝又修屋宇并植松栢千餘株】 十二月戊子勅盜邊糧一升已上皆斬 【考異曰刑法志事在十六年今從帝紀】仍籍汲其家 己丑詔文武官以四考受代【唐虞以三年為一考後世以一年為一考】 汴州刺史令狐熙來朝【隋志榮陽郡浚儀縣東魏置梁州後周改曰汴州令狐出於魏氏春秋晉大夫魏顆封於令狐子孫以為氏汴皮變翻】考績為天下之最賜帛三百匹【雜物為段純物為匹】頒告天下熙整之子也【令狐整見一百五十九卷梁武帝中大同元年】<br />
<br />
  十六年春正月丁亥以皇孫裕為平原王筠為安成王嶷為安平王【嶷魚力翻】恪為襄城王該為高陽王韶為建安王煚為潁川王【煚居永翻】皆勇之子也 夏六月甲午初制工商不得仕進 秋八月丙戌詔決死罪者三奏然後行刑 【考異曰刑法志在十五年今從帝紀】 冬十月己丑上幸長春宫【隋志同州朝邑縣有長春宫】十一月壬子還長安 党項寇會州【隋志汶山郡後周置汶州開皇初改曰蜀州尋為會州党底朗翻】詔發隴西兵討降之帝以光化公主妻吐谷渾可汗世伏【妻七細翻吐從暾入聲谷音浴可從刋入聲汗音寒】世伏上表請稱公主為天后上不許【伏上時掌翻】十七年春二月癸未太平公史萬歲擊南寜羌平之【史萬歲襲爵太平縣公隋志太平縣屬絳郡南寜之地漢屬牂柯蜀漢屬南中晉屬寜州梁為南寜州其後為㸑氏所據自云本安邑人七世祖晉南寜太守中國亂遂王蠻中考之晉志未始有南寜郡西㸑蠻也非羌也通鑑因隋紀成文】初梁睿之克王謙也【見一百七十四卷陳宣帝太建十二年】西南夷獠莫不歸附唯南寜州酋帥㸑震恃遠不服【獠魯皓翻酋慈秋翻帥所類翻㸑七亂翻】睿上疏以為南寜州漢世牂柯之地【牂柯音臧柯上時掌翻】戶口殷衆金寶富饒梁南寜州刺史許文盛為湘東王徵赴荆州【徵兵以討侯景文盛赴之】屬東夏尚阻未遑遠畧【屬之欲翻夏戶雅翻】土民㸑瓚遂竊據一方【瓚從旱翻】國家遥授刺史其子震相承至今而震臣禮多虧貢賦不入乞因平蜀之衆略定南寜其後南寜夷㸑翫來降拜昆州刺史【就其地置昆州降戶江翻下同】既而復叛乃以左領軍將軍史萬歲為行軍總管帥衆擊之【復扶又翻帥讀曰率】入自蜻蛉川至於南中【蜻蛉川漢蜻蛉縣之地蜻倉經翻蛉郎丁翻】夷人前後屯據要害萬歲皆擊破之過諸葛亮紀功碑【唐史南詔王鳳迦異築柘東城有諸葛亮石刻故在】度西洱河【按唐史太宗擊西㸑開蜻蛉弄棟為縣弄棟西有大勃弄小勃弄二州蠻其西與黄瓜葉榆西洱河接西洱河即葉榆河也蘇軾曰南詔有西洱河即牂柯江也河形如月抱洱故名之為西洱河洱而止翻又而志翻】入渠濫川行千餘里破其三十餘部虜獲男女二萬餘口諸夷大懼遣使請降獻明珠徑寸于是勒石頌美隋德萬歲請將㸑翫入朝【使疏吏翻朝直遥翻請將音如字攜也領也】詔許之㸑翫隂有貳心不欲詣闕賂萬歲以金寶萬歲於是捨翫而還【為史萬歲得罪張本還從宣翻又如字】庚寅上幸仁夀宫 桂州俚帥李光仕作亂【始安郡梁置桂州俚音里帥所類翻下渠帥同】帝遣上柱國王世績與前桂州總管周法尚討之法尚發嶺南兵世績發嶺北兵俱會尹州【隋志鬱林郡梁置定州後改為南定州平陳改尹州】世績所部遇瘴不能進【瘴之亮翻熱病】頓于衡州【隋志衡山郡平陳置衡州】法尚獨討之光仕戰敗帥勁兵走保白石洞【白石洞在今尋州南六十里帥讀曰率下同】法尚大獲家口其黨有來降者輒以妻子還之居旬日降者數千人【降戶江翻】光仕衆潰而走追斬之帝又遣員外散騎侍郎何稠募兵討光仕稠諭降其黨莫崇等【姓苑何氏出自周成王母弟唐叔虞其後封于韓韓滅子孫分散江淮間音以韓為何字隨音變後遂為何氏莫姓楚莫敖之後散悉亶翻騎奇寄翻】承制署首領為州縣官稠妥之兄子也【妥他果翻】上以嶺南夷越數反【數所角翻】以汴州刺史令狐熙為桂州總管十七州諸軍事許以便宜從事刺史以下官得承制補授熙至部大弘恩信其溪洞渠帥更相謂曰【令力丁翻帥所類翻更工衡翻】前時總管皆以兵威相脅今者乃以手敎相諭我輩其可違乎於是相帥歸附先是州縣生梗【先悉薦翻】長吏多不得之官【長知兩翻】寄政於總管府熙悉遣之為建城邑【為于偽翻】開設學校華夷感化焉俚帥【帥所類翻】甯猛力在陳世已據南海隋因而撫之拜安州刺史猛力恃險驕倨未嘗參謁熙諭以恩信猛力感之詣府請謁不敢為非熙奏改安州為欽州【隋志寜越郡梁置安州今改欽州】帝以所在屬官不敬憚其上事難克舉三月壬辰詔<br />
<br />
  諸司論屬官罪有律輕情重者聽於律外斟酌決杖於是上下相驅迭相捶楚以殘暴為幹能以守法為懦弱【捶止橤翻懦乃卧翻又奴亂翻】帝以盜賊繁多命盜一錢以上皆弃市或三人共盜一瓜事發即死于是行旅皆晏起早宿【恐邂逅觸罪也 考異曰刑法志作晩宿必早字誤耳】天下懔懔有數人劫執事而謂之曰吾豈求財者邪【邪音耶】但為枉人來耳而謂我奏至尊自古以來體國立法未有盜一錢而死者也而不為我以聞吾更來而屬無類矣帝聞之為停此法【自古以來閭里奸豪持吏短長者則有之矣未聞持其上至此者宜隋季之多盜也天下之富一錢之積是以古之為政欲其平易近民為于偽翻而為而不而屬之而猶言汝也】帝嘗乘怒欲以六月杖殺人大理少卿河東趙綽固爭【隋制九寺各置卿少卿各一人河東縣蒲州河東郡治少始照翻】曰季夏之月天地成長庶類【長知兩翻下同】不可以此時誅殺帝報曰六月雖曰生長此時必有雷霆我則天而行有何不可遂殺之大理掌固來曠上言大理官司太寛【掌固蓋即漢之掌故唐省臺寺監皆有掌固因隋行也上時掌翻】帝以曠為忠直遣每旦於五品行中參見【遣猶使也行戶剛翻見賢遍翻】曠又告少卿趙綽濫免徒囚帝使信臣推驗初無阿曲帝怒命斬之綽固争以為曠不合死帝拂衣入閣綽矯言臣更不理曠自有佗事未及奏聞帝命引入閣綽再拜請曰臣有死辠三臣為大理少卿不能制御掌固使曠觸挂天刑一也囚不合死而臣不能死爭二也臣本無佗事而妄言求入三也帝解顔會獨孤后在坐【坐徂卧翻】命賜綽二金盃酒并盃賜之曠因免死徙廣州蕭摩訶子世略在江南作亂摩訶當從坐上曰世略年未二十亦何能為以其名將之子為人所逼耳【將即亮翻】因赦摩訶綽固諫不可上不能奪欲綽去而赦之因命綽退食綽曰臣奏獄未決不敢退上曰大理其為朕特赦摩訶也因命左右釋之【為于偽翻】刑部侍郎辛亶嘗衣緋褌【衣於既翻褌古渾翻褻衣也】俗云利官上以為厭蠱【厭一叶翻又於琰翻】將斬之綽曰法不當死臣不敢奉詔上怒甚曰卿惜辛亶而不自惜也命引綽斬之綽曰陛下寜殺臣不可殺辛亶至朝堂解衣當斬上使人謂綽曰竟何如對曰執法一心不敢惜死上拂衣而入良久乃釋之明日謝綽勞勉之賜物三百段【勞力到翻】時上禁行惡錢有二人在市以惡錢易好者武候執以聞【武候屬左右武候將軍掌晝夜巡察執捕奸非也】上令悉斬之綽進諫曰此人所坐當杖殺之非法上曰不關卿事綽曰陛下不以臣愚暗置在法司欲妄殺人豈得不關臣事上曰撼大木不動者當退對曰臣望感天心何論動木上復曰啜羮者熱則置之【復扶又翻下同啜昌悦翻】天子之威欲相挫邪【邪音耶】綽拜而益前訶之不肯【訶虎何翻】上遂入治書侍御史柳彧復上奏切諫上乃止【治直之翻復上時掌翻】上以綽有誠直之心每引入閣中或遇上與皇后同榻即呼綽坐評論得失前後賞賜萬計與大理卿薛胄同時俱名平恕然胄斷獄以情而綽守法俱為稱職【斷丁亂翻稱尺證翻】胄端之子也【薛端仕周為蔡州刺史無它異稱】帝晩節用法益峻御史於元日不劾武官衣劒之不齊者【劾戶蓋翻又戶得翻】帝曰爾為御史縱捨自由命殺之諫議大夫毛思祖諫又殺之【隋門下省置諫議大夫七人】將作寺丞以課麥遲晚【圭玄翻類篇曰麥莖也】武庫令以署庭荒蕪【武庫令屬衛尉寺】左右出使或授牧宰馬鞭鸚鵡【使疏吏翻授當作受】帝察知並親臨斬之帝既喜怒不恒不復依凖科律【恒戶登翻復扶又翻】信任楊素素復任情不平與鴻臚少卿陳延有隙【少始照翻】嘗經蕃客館庭中有馬屎又衆僕於氈上樗蒲以白帝帝大怒主客令及樗蒲者皆杖殺之捶陳延幾死【隋志鴻臚寺統典客令即主客也屎式爾翻糞也捶止橤翻幾居依翻】帝遣親衛大都督長安屈突通往隴西檢覆羣牧【隋志置左右親衛左右勛衛左右翊衛有大都督帥都督都督等官煬帝改大都督為校尉帥都督為旅帥都督為隊正屈突虜複姓其先昌黎徒河人徙家長安隴西郡渭州屈九勿翻】得隱匿馬二萬餘匹帝大怒將斬太僕卿慕容悉逹及諸監官千五百人【太僕卿掌牧畜之政故欲誅之】通諫曰人命至重陛下奈何以畜產之故殺千有餘人臣敢以死請帝瞋目叱之【瞋昌真翻】通又頓首曰臣一身分死【分扶問翻】就陛下匄千餘人命帝感寤曰朕之不明以至於此賴有卿忠言耳于是悉逹等皆減死論擢通為左武候將軍【隋志左右武候掌車駕出先驅後殿晝夜巡察執捕奸非烽候道路水屮所置巡狩師田則掌其營禁也】 上柱國劉昶與帝有舊帝甚親之其子居士任俠不遵法度數有罪【數所角翻】上以昶故每原之居士轉驕恣取公卿子弟雄健者輒將至家以車輪括其頸而棒之【棒蒲項翻】殆死能不屈者稱為壯士釋而與交黨與三百人擊路人【烏口翻】多所侵奪至於公卿妃主莫敢與校或告居士謀為不軌帝怒斬之公卿子弟坐居士除名者甚衆 楊素牛弘等復薦張胄玄歷術【去年帝勞徠胄玄既而罷之復扶又翻】上令楊素與術數人立議六十一事皆舊法久難通者令暉等與胄玄等辨析暉杜口一無所答胄玄通者五十四上乃拜胄玄員外散騎侍郎兼太史令賜物千段令參定新術【散悉亶翻騎奇寄翻】至是胄玄歷成夏四月戊寅詔頒新歷前造歷者劉暉四人並除名 秋七月桂州人李世賢反上議討之諸將數人請行【將即亮翻】上不許顧右武候大將軍虞慶則曰位居宰相爵乃上公【開皇初慶則嘗為尚書右僕射宰相之職也授上柱國封晉國公上公也】國家有賊遂無行意何也慶則拜謝恐懼乃以慶則為桂州道行軍總管討平之 秦王俊幼仁恕喜佛敎【喜許記翻】嘗請為沙門不許及為并州總管【俊為并州總管見上卷十年】漸好奢侈違越制度盛治宫室俊好内【治直之翻好呼報翻】其妻崔氏弘度之妹也性妬於瓜中進毒由是得疾徵還京師上以其奢縱丁亥免俊官以王就第崔妃以毒王廢絶賜死於家左武衛將軍劉昇【曹魏置武衛將軍自晉至於高齊並屬左右衛至隋始與左右衛並列於十二衛府】諫曰秦王非有佗過但費官物營廨舍而已【廨古隘翻】臣謂可容上曰法不可違楊素諫曰秦王之過不應至此願陛下詳之【復扶又翻】上曰我是五兒之父【上五子太子勇晉王廣秦王俊蜀王秀漢王諒】非兆民之父若如公意何不别制天子兒律以周公之為人尚誅管蔡我誠不及周公遠矣安能虧法乎卒不許【楊素逢君之惡者也它日贊決以廢勇立廣蓋有見於此卒子恤翻】 戊戌突厥突利可汗來逆女【厥九勿翻可從刋入聲汗音寒】上舍之太常敎習六禮【六禮納采問名納吉納徵請期親迎】妻以宗女安義公主上欲離間都藍故特厚其禮【妻七細翻間古莧翻】遣太常卿牛弘納言蘇威民部尚書斛律孝卿相繼為使【開皇三年改度支尚書為民部尚書使疏吏翻】突利本居北方既尚主長孫晟說其帥衆南徙居度斤舊鎮【度斤舊鎮蓋即都斤山突厥沙鉢畧舊所居也帥讀曰率】錫賚優厚【賚來代翻】都藍怒曰我大可汗也反不如染干於是朝貢遂絶亟來抄掠邊鄙【朝直遥翻下同亟去吏翻抄楚交翻】突利伺知動静輒遣奏聞由是邊鄙每先有備【伺相吏翻】 九月甲申上至自仁夀宫 何稠之自嶺南還也【是年二月何稠討嶺南】甯猛力請隨稠入朝稠見其疾篤遣還欽州與之約曰八九月間可詣京師相見使還奏狀上意不懌【使疏吏翻下同】冬十月猛力病卒上謂稠曰汝前不將猛力來今竟死矣稠曰猛力與臣約假令身死當遣子入侍【令力丁翻】越人性直其子必來猛力臨終果戒其子長真曰我與大使約不可失信【使疏吏翻】汝葬我畢宜即登路長真嗣為刺史如言入朝【嗣祥吏翻】上大悦曰何稠著信蠻夷乃至于此 魯公虞慶則之討李世賢也以婦弟趙什住為隨府長史【長知兩翻】什住通于慶則愛妾恐事泄乃宣言慶則不欲此行上聞之禮賜甚薄慶則還至潭州臨桂嶺【隋書虞慶則傳作潭州臨桂嶺宋白曰隋平陳改湘州為潭州杜佑曰取昭潭為名】觀眺山川形勢曰此誠險固加以足糧若守得其人攻不可拔使什住馳詣京師奏事觀上顔色什住因告慶則謀反下有司案驗【下遐稼翻】十二月壬子慶則坐死拜什住為柱國 高麗王湯聞陳亡大懼治兵積穀為拒守之策【麗力知翻治直之翻】是歲上賜湯璽書責以雖稱藩附誠節未盡且曰彼之一方雖地狹人少【璽斯氏翻少詩沼翻下同】今若黜王不可虚置終須更選官屬就彼安撫王若洒心易行【洒讀曰洗行下孟翻】率由憲章即是朕之良臣何勞别遣才彦王謂遼水之廣何如長江高麗之人多少陳國【少詩沼翻】朕若不存含育責王前愆命一將軍何待多力殷勤曉示許王自新耳湯得書惶恐將奉表陳謝會病卒【卒子恤翻】子元嗣立上使使拜元為上開府儀同三司襲爵遼東公【使使疏吏翻下同】元奉表謝恩因請封王上許之【自時隋終以高麗為意後遂以佳兵亡國】 吐谷渾大亂【吐從暾入聲谷音浴】國人殺世伏立其弟伏允為主遣使陳廢立之事并謝專命之罪且請依俗尚主上從之自是朝貢歲至【朝直遥翻】<br />
<br />
  十八年春二月甲辰上幸仁夀宫 高麗王元帥靺鞨之衆萬餘寇遼西【隋書靺鞨在高麗之北凡有七種其一號栗末部與高麗接其二曰伯咄部在栗末之北其三曰安車骨部在伯咄東北其四曰拂涅部在伯咄東其五曰號室部在拂湼東其六曰黑水部在安車骨西北其七曰白山部在栗末東南而黑水部尤為勁健即古之肅慎氏也遼西郡治柳城隋置營州總管府靺莫撥翻鞨戶葛翻】營州總管韋冲擊走之上聞而大怒乙巳以漢王諒王世積並為行軍元帥將水陸三十萬伐高麗【帥所類翻將即亮翻】以尚書左僕射高熲為漢王長史【長知兩翻】周羅為水軍緫管 延州刺史獨孤陀【隋志延安郡後魏置東夏州西魏改延州陀徒河翻】有婢曰徐阿尼【阿烏葛翻尼希夷翻下同】事猫鬼能使之殺人云每殺人則死家財物潛移於畜猫鬼家會獨孤后及楊素妻鄭氏俱有疾毉皆曰猫鬼疾也【隋書陀傳云徐阿尼事猫鬼每以子夜祀之言子者鼠也陀嘗從家中索酒其妻曰無錢可酤陀因謂阿尼曰可令猫鬼向越公家使我足錢也阿尼便呪之居數日猫鬼向素家十一年上初從并州還陀謂阿尼曰可令猫鬼向皇后所使多賜吾物阿尼復呪之遂入宫中大理丞楊遠乃於門下外省遣阿尼呼猫鬼阿尼于是夜中置香粥一盆以匙扣而呼之曰猫女可來無住宫中久之阿尼色正青若被牽曳者云猫鬼已至由是其事具得實畜吁玉翻下同】上以陀后之異母弟陀妻楊素異母妹由是意陀所為令高熲等雜治之具得其實【治直之翻】上怒令以犢車載陀夫妻將賜死獨孤后三日不食為之請命曰【為于偽翻下同】陀若蠧政害民者妾不敢言今坐為妾身敢請其命陀弟司勳侍郎整詣闕求哀【司勲侍郎屬吏部尚書】於是免陀死除名為民以其妻楊氏為尼先是有人訟其母為猫鬼所殺者上以為妖妄怒而遣之【妖於驕翻】至是詔誅被訟行猫鬼家【先悉薦翻被皮義翻】夏四月辛亥詔畜猫鬼蠱毒厭媚野道之家【隋書志江南諸郡往往畜蠱而宜春偏甚其法以五月五日聚百種虫大者至蛇小者至蝨合置器中令自相啖餘一種存者留之蛇則曰蛇蠱蝨則曰蝨蠱行以殺人因食入人腹内食其五藏死則其產移入蠱主之家三年不殺它人則畜者自鍾其弊累世子孫相傳不絶亦有隨女子嫁者厭於琰翻媚音魅】並投於四裔 六月丙寅下詔黜高麗王元官爵【麗力知翻】漢王諒軍出臨渝關【臨渝關在柳城西四百八十里所謂盧龍之險也渝漢書音喻】值水潦餽運不繼軍中乏食復遇疾疫【復扶又翻】周羅自東萊泛海趣平壤城【隋書平壤城東西六里隨山屈曲南臨浿水杜佑曰平壤城則故朝鮮國王險城也趣七喻翻】亦遭風船多飄沒秋九月己丑師還死者什八九高麗王元亦惶懼遣使謝罪【使疏吏翻下同】上表稱遼東糞土臣元上于是罷兵待之如初百濟王昌遣使奉表請為軍導帝下詔諭以高麗服罪朕已赦之不可致伐厚其使而遣之高麗頗知其事以兵侵掠其境 辛卯上至自仁夀宫 冬十一月癸未上祀南郊 十二月自京師至仁夀宫置行宫十有二所 南寜夷爨翫復反【復扶又翻】蜀王秀奏史萬歲受賂縱賊致生邊患上責萬歲萬歲詆讕【讕落干翻又力誕翻詆拒諱也讕逸辭也】上怒命斬之高熲及左衛大將軍元旻等固請曰萬歲雄略過人將士樂為致力雖古名將未能過也上意少解于是除名為民【將即亮翻樂音洛少詩沼翻】<br />
<br />
  十九年春正月癸酉赦天下 二月甲寅上幸仁夀宫【仁夀宫成於開皇十七年方其成也文帝怒欲罪楊素獨孤后喜而賞之繼此屢幸仁夀宫至仁夀之末卒死于仁夀宫仁者夀帝窮民力以作離宫可謂仁乎其不得死于是宫宜矣帝怒楊素而不加之罪其后喜則亦從而喜之豈非奢侈之能移人觸境而動至於流連而不知反卒貽萬世笑是知君德以節儉為貴也】突厥突利可汗因長孫晟奏言都藍可汗作攻具欲攻大同城【唐志自夏州北度烏水行五百三十餘里過橫水又行百一十九里至安樂戍戌在河西壖東壖有古大同城今大同城古永濟柵也厥九勿翻可從刋入聲汗音寒長知兩翻晟承正翻】詔以漢王諒為元帥【帥所類翻】尚書左僕射高熲出朔州道【隋志馬邑郡舊置朔州】右僕射楊素出靈州道【靈武郡後魏置靈州】上柱國燕榮出幽州道以擊都藍【燕因肩翻】皆取漢王節度然漢王竟不臨戎都藍聞之與逹頭可汗結盟合兵掩襲突利大戰長城下突利大敗都藍盡殺其兄弟子姪遂度河入蔚州【隋志雁門郡靈丘縣後周置蔚州蔚紆勿翻】突利部落散亡夜與長孫晟以五騎南走【騎奇寄翻下同】比旦行百餘里【比必寐翻及也】收得數百騎突利與其下謀曰今兵敗入朝一降人耳【朝直遥翻下同降戶剛翻】大隋天子豈禮我乎玷厥雖來本無寃隙【寃猶怨也】若往投之必相存濟晟知之密遣使者入伏遠鎮【使疏吏翻】令速舉烽突利見四烽俱發以問晟晟紿之曰城高地迥必遥見賊來我國家法若賊少舉二烽【紿蕩亥翻少詩沼翻】來多舉三烽大逼舉四烽彼見賊多而又近耳突利大懼謂其衆曰追兵已逼且可投城既入鎮晟留其逹官執室領其衆自將突利馳驛入朝【將如字】夏四月丁酉突利至長安帝大喜以晟為左勲衛驃騎將軍【隋制驃騎將軍正四品驃匹妙翻騎奇寄翻】持節護突厥上令突利與都藍使者因頭特勒相辨詰突利辭直上乃厚待之都藍弟郁速六弃其妻子與突利歸朝【使疏吏翻朝直遥翻】上嘉之使突利多遺之珍寶以慰其心【遺于季翻】高熲使上柱國趙仲卿將兵三千為前鋒至族蠡山【將即亮翻蠡音黎】與突厥遇交戰七日大破之追奔至乞伏泊復破之虜千餘口雜畜萬計突厥復大舉而至仲卿為方陳【厥九勿翻下同畜許又翻復扶又翻陳讀曰陣下同】四面拒戰凡五日會高熲大兵至合擊之突厥敗走追度白道踰秦山七百餘里而還【還從宣翻又如字】楊素軍與逹頭遇先是諸將與突厥戰慮其騎兵奔突皆以戎車步騎相參設鹿角為方陳騎在其内【此古法也雖衛青劉裕未之能易也所謂先為不可勝以待敵之可勝者也先悉薦翻騎奇寄翻下同】素曰此乃自固之道未足以取勝也于是悉除舊法令諸軍為騎陳逹頭聞之大喜曰天賜我也下馬仰天而拜帥騎兵十餘萬直前上儀同三司周羅曰賊陳未整請擊之帥精騎逆戰素以大兵繼之突厥大敗逹頭被重創而遁殺傷不可勝計其衆號哭而去【魏舒毁車崇卒以敗狄楊素除戎車為騎陳以破突厥皆鼔儳而勝耳帥讀曰率勝音升號戶高翻】六月丁酉以豫章王暕為内史令【暕古限翻】 宜陽公王<br />
<br />
  世積為涼州緫管其親信安定皇甫孝諧有罪【王世積封宜陽郡公隋志河南郡宜陽縣後魏置宜陽郡武威郡舊置涼州安定郡舊置涇州】吏捕之亡抵世積世積不納孝諧配防桂州【配防者配隸軍伍使之防守】因上變稱世積嘗令道人相其貴不【上時掌翻令力丁翻相息亮翻不讀曰否】道人答曰公當為國主又將之涼州【之往也】其所親謂世積曰河西天下精兵處可圖大事世積曰涼州土曠人希非用武之國世積坐誅拜孝諧上大將軍 獨孤后性妬忌後宫莫敢進御尉遲迥女孫有美色先没宫中【尉紆勿翻先悉薦翻下同】上於仁夀宫見而悦之因得幸后伺上聽朝【伺相吏翻朝直遥翻】隂殺之上由是大怒單騎從苑中出不由徑路入山谷間二十餘里高熲楊素等追及上扣馬苦諫上太息曰吾貴為天子不得自由高熲曰陛下豈以一婦人而輕天下上意少解【少詩沼翻】駐馬良久中夜方還宫后俟上於閣内及至后流涕拜謝熲素等和解之【還音旋又如字】因置酒極歡先是后以高熲父之家客甚見親禮【熲父賓為獨孤信參佐信被誅后以賓父之故吏數往來其家】至是聞熲謂已為一婦人遂衘之時太子勇失愛於上潛有廢立之志從容謂熲曰【從千容翻】有神告晉王妃言王必有天下若之何熲長跪曰長幼有序其可廢乎【長幼之長知兩翻】獨孤后知熲不可奪隂欲去之【為后譖熲張本去羌呂翻】會上令選東宫衛士以入上臺熲奏稱若盡取彊者恐東宫宿衛太劣上作色曰我有時出入宿衛須得勇毅太子毓德春宫左右何須壯士此極弊法如我意者恒於交番之日分向東宫上下團伍不别【三百人為團五人為伍恒戶登翻上時掌翻下遐嫁翻别彼列翻】豈非佳事我熟見前代公不須仍踵舊風熲子表仁娶太子女故上以此言防之熲夫人卒【卒子恤翻】獨孤后言於上曰高僕射老矣而喪夫人【喪息浪翻】陛下何能不為之娶【為于偽翻下同】上以后言告熲熲流涕謝曰臣今已老退朝唯齋居讀佛經而已【朝直遥翻】雖陛下垂哀之深至于納室非臣所願上乃止既而熲愛妾生男上聞之極喜后甚不悦上問其故后曰陛下尚復信高熲邪【復扶又翻邪音耶】始陛下欲為熲娶熲心存愛妾面欺陛下今其詐已見【見賢遍翻】安得信之上由是疎熲伐遼之役【去年伐遼】熲固諫不從及師無功后言於上曰熲初不欲行陛下強遣之【強其兩翻】妾固知其無功矣又上以漢王年少【少詩照翻】專委軍事於熲熲以任寄隆重每懷至公無自疑之意諒所言多不用諒甚衘之及還泣言於后曰兒幸免高熲所殺上聞之彌不平及擊突厥出白道進圖入磧【磧大磧也即所謂大漢磧七迹翻】遣使請兵【使疏吏翻】近臣緣此言熲欲反上未有所答熲已破突厥而還【此即謂前破突厥事還從宣翻又如字】及王世積誅推覈之際有宫禁中事云於熲處得之上大驚有司又奏熲及左右衛大將軍元旻元胄並與世積交通受其名馬之贈旻胄坐免官上柱國賀若弼吳州總管宇文㢸【若人者翻㢸古弼字】刑部尚書薛胄【開皇三年改都官尚書為刑部尚書】民部尚書斛律孝卿兵部尚書柳述等明熲無罪上愈怒皆以屬吏【屬之欲翻】自是朝臣無敢言者【朝直遥翻】秋八月癸卯熲坐免上柱國左僕射以齊公就第未幾上幸秦王俊第召熲侍宴熲歔欷悲不自勝【幾居豈翻歔音虚欷音希又許既翻勝音升】獨孤后亦對之泣上謂熲曰朕不負公公自負也因謂侍臣曰我於高熲勝於兒子雖或不見常似目前自其解落瞑然忘之【解落謂解官落職也瞑莫定翻】如本無高熲人臣不可以身要君【要一遥翻】自云第一也頃之熲國令上熲隂事【隋制王國公國皆有令有尉上時掌翻】稱其子表仁謂熲曰司馬仲逹初託疾不朝遂有天下【司馬懿字仲逹事見魏邵陵厲公紀朝直遥翻】公今遇此焉知非福【熲國公承望上指以此誣熲盖亦習見趙什住皇甫孝諧受賞而利之也焉於䖍翻】於是上大怒囚熲於内史省而鞫之憲司復奏沙門真覺嘗謂熲云明年國有大喪【憲司法司也復扶又翻喪息郎翻】尼令暉復云十七十八年皇帝有大厄十九年不可過上聞而益怒顧謂羣臣曰帝王豈可力求孔子以大聖之才猶不得天下熲與子言自比晉帝此何心乎有司請斬之上曰去年殺虞慶則今茲斬王世積如更誅熲天下其謂我何於是除名為民熲初為僕射【帝受禪熲即為僕射】其母戒之曰汝富貴已極但有一斫頭耳爾其慎之熲由是常恐禍變至是熲歡然無恨色先是國子祭酒元善言於上曰楊素麤疎蘇威怯懦元胄元旻正似鳬耳【楚辭曰寜汎汎若水中之鳬與波上下以全吾軀乎元善之意謂此先悉薦翻】可以付社稷者唯獨高熲上初然之及熲得罪上深責之善憂懼而卒【卒子恤翻】 九月以太常卿牛弘為吏部尚書弘選舉先德行而後文才務在審慎雖致停緩其所進用並多稱職【先悉薦翻行下孟翻後戶遘翻稱尺證翻】吏部侍郎高孝基鑒賞機悟清慎絶倫然爽俊有餘迹似輕薄時宰多以此疑之唯弘深識其真推心任委隋之選舉得人於斯為最時論彌服弘識度之遠 冬十月甲午以突厥突利可汗為意利珍豆唘民可汗華言意智健也【厥九勿翻可從刋入聲汗音寒】突厥歸唘民者男女萬餘口上命長孫晟將五萬人於朔州築大利城以處之【長知兩翻晟承正翻將即亮翻下同大利城在雲内縣東北隋志定襄郡治大利縣處昌呂翻下同】時安義公主已卒【十七年安義公主嫁突厥卒子恤翻】復使晟持節送宗女義成公主以妻之【復音扶又翻妻七細翻】晟奏染干部落歸者益衆雖在長城之内猶被雍虞閭抄掠不得寜居【抄楚文翻】請徙五原以河為固【鹽州五原之地】於夏勝兩州之間【隋志朔方郡後魏置夏州榆林郡開皇二十年置勝州杜佑曰勝州治榆林縣漢沙南縣地有雲中城拂雲堆金河紫河自馬邑郡善陽界流入縣西有漢五原城夏戶雅翻】東西至河南北四百里掘為横塹【塹七艷翻】令處其内【令力丁翻處昌呂翻】使得任情畜牧上從之又令上柱國趙仲卿屯兵二萬為唘民防逹頭【為于偽翻】代州總管韓洪等將步騎一萬鎮恒安【隋志雁門後周置肆州開皇五年改曰代州舊唐志恒安鎮在隋馬邑郡雲内縣界唐為雲州雲中縣即後魏所都平城之地恒戶登翻】逹頭騎十萬來寇韓洪軍大敗【騎奇寄翻】仲卿自樂寜鎮邀擊斬首千餘級 帝遣越公楊素出靈州行軍緫管韓僧夀出慶州【弘化郡開皇十六年置慶州】太平公史萬歲出燕州【涿郡懷戎縣後齊置北燕州後周去北字燕因肩翻】大將軍武威姚辨出河州【枹罕郡舊置河州】以擊都藍師未出塞十二月乙未都藍為部下所殺逹頭自立為步迦【迦音加】可汗其國大亂長孫晟言於上曰今官軍臨境戰數有功【數所角翻】虜内自攜離其主被殺乘此招撫可以盡降請遣染干部下分道招慰上從之降者甚衆【被皮義翻降戶江翻】<br />
<br />
  資治通鑑卷一百七十八  <br>
   </div> 

<script src="/search/ajaxskft.js"> </script>
 <div class="clear"></div>
<br>
<br>
 <!-- a.d-->

 <!--
<div class="info_share">
</div> 
-->
 <!--info_share--></div>   <!-- end info_content-->
  </div> <!-- end l-->

<div class="r">   <!--r-->



<div class="sidebar"  style="margin-bottom:2px;">

 
<div class="sidebar_title">工具类大全</div>
<div class="sidebar_info">
<strong><a href="http://www.guoxuedashi.com/lsditu/" target="_blank">历史地图</a></strong>  
<a href="http://www.880114.com/" target="_blank">英语宝典</a>  
<a href="http://www.guoxuedashi.com/13jing/" target="_blank">十三经检索</a> 
<br><strong><a href="http://www.guoxuedashi.com/gjtsjc/" target="_blank">古今图书集成</a></strong> 
<a href="http://www.guoxuedashi.com/duilian/" target="_blank">对联大全</a> <strong><a href="http://www.guoxuedashi.com/xiangxingzi/" target="_blank">象形文字典</a></strong> 

<br><a href="http://www.guoxuedashi.com/zixing/yanbian/">字形演变</a>  <strong><a href="http://www.guoxuemi.com/hafo/" target="_blank">哈佛燕京中文善本特藏</a></strong>
<br><strong><a href="http://www.guoxuedashi.com/csfz/" target="_blank">丛书&方志检索器</a></strong> <a href="http://www.guoxuedashi.com/yqjyy/" target="_blank">一切经音义</a>  

<br><strong><a href="http://www.guoxuedashi.com/jiapu/" target="_blank">家谱族谱查询</a></strong>  <strong><a href="http://shufa.guoxuedashi.com/sfzitie/" target="_blank">书法字帖欣赏</a></strong> 
<br>

</div>
</div>


<div class="sidebar" style="margin-bottom:0px;">

<font style="font-size:22px;line-height:32px">QQ交流群9:489193090</font>


<div class="sidebar_title">手机APP 扫描或点击</div>
<div class="sidebar_info">
<table>
<tr>
	<td width=160><a href="http://m.guoxuedashi.com/app/" target="_blank"><img src="/img/gxds-sj.png" width="140"  border="0" alt="国学大师手机版"></a></td>
	<td>
<a href="http://www.guoxuedashi.com/download/" target="_blank">app软件下载专区</a><br>
<a href="http://www.guoxuedashi.com/download/gxds.php" target="_blank">《国学大师》下载</a><br>
<a href="http://www.guoxuedashi.com/download/kxzd.php" target="_blank">《汉字宝典》下载</a><br>
<a href="http://www.guoxuedashi.com/download/scqbd.php" target="_blank">《诗词曲宝典》下载</a><br>
<a href="http://www.guoxuedashi.com/SiKuQuanShu/skqs.php" target="_blank">《四库全书》下载</a><br>
</td>
</tr>
</table>

</div>
</div>


<div class="sidebar2">
<center>


</center>
</div>

<div class="sidebar"  style="margin-bottom:2px;">
<div class="sidebar_title">网站使用教程</div>
<div class="sidebar_info">
<a href="http://www.guoxuedashi.com/help/gjsearch.php" target="_blank">如何在国学大师网下载古籍?</a><br>
<a href="http://www.guoxuedashi.com/zidian/bujian/bjjc.php" target="_blank">如何使用部件查字法快速查字?</a><br>
<a href="http://www.guoxuedashi.com/search/sjc.php" target="_blank">如何在指定的书籍中全文检索?</a><br>
<a href="http://www.guoxuedashi.com/search/skjc.php" target="_blank">如何找到一句话在《四库全书》哪一页?</a><br>
</div>
</div>


<div class="sidebar">
<div class="sidebar_title">热门书籍</div>
<div class="sidebar_info">
<a href="/so.php?sokey=%E8%B5%84%E6%B2%BB%E9%80%9A%E9%89%B4&kt=1">资治通鉴</a> <a href="/24shi/"><strong>二十四史</strong></a>&nbsp; <a href="/a2694/">野史</a>&nbsp; <a href="/SiKuQuanShu/"><strong>四库全书</strong></a>&nbsp;<a href="http://www.guoxuedashi.com/SiKuQuanShu/fanti/">繁体</a>
<br><a href="/so.php?sokey=%E7%BA%A2%E6%A5%BC%E6%A2%A6&kt=1">红楼梦</a> <a href="/a/1858x/">三国演义</a> <a href="/a/1038k/">水浒传</a> <a href="/a/1046t/">西游记</a> <a href="/a/1914o/">封神演义</a>
<br>
<a href="http://www.guoxuedashi.com/so.php?sokeygx=%E4%B8%87%E6%9C%89%E6%96%87%E5%BA%93&submit=&kt=1">万有文库</a> <a href="/a/780t/">古文观止</a> <a href="/a/1024l/">文心雕龙</a> <a href="/a/1704n/">全唐诗</a> <a href="/a/1705h/">全宋词</a>
<br><a href="http://www.guoxuedashi.com/so.php?sokeygx=%E7%99%BE%E8%A1%B2%E6%9C%AC%E4%BA%8C%E5%8D%81%E5%9B%9B%E5%8F%B2&submit=&kt=1"><strong>百衲本二十四史</strong></a>  <a href="http://www.guoxuedashi.com/so.php?sokeygx=%E5%8F%A4%E4%BB%8A%E5%9B%BE%E4%B9%A6%E9%9B%86%E6%88%90&submit=&kt=1"><strong>古今图书集成</strong></a>
<br>

<a href="http://www.guoxuedashi.com/so.php?sokeygx=%E4%B8%9B%E4%B9%A6%E9%9B%86%E6%88%90&submit=&kt=1">丛书集成</a> 
<a href="http://www.guoxuedashi.com/so.php?sokeygx=%E5%9B%9B%E9%83%A8%E4%B8%9B%E5%88%8A&submit=&kt=1"><strong>四部丛刊</strong></a>  
<a href="http://www.guoxuedashi.com/so.php?sokeygx=%E8%AF%B4%E6%96%87%E8%A7%A3%E5%AD%97&submit=&kt=1">說文解字</a> <a href="http://www.guoxuedashi.com/so.php?sokeygx=%E5%85%A8%E4%B8%8A%E5%8F%A4&submit=&kt=1">三国六朝文</a>
<br><a href="http://www.guoxuedashi.com/so.php?sokeytm=%E6%97%A5%E6%9C%AC%E5%86%85%E9%98%81%E6%96%87%E5%BA%93&submit=&kt=1"><strong>日本内阁文库</strong></a> <a href="http://www.guoxuedashi.com/so.php?sokeytm=%E5%9B%BD%E5%9B%BE%E6%96%B9%E5%BF%97%E5%90%88%E9%9B%86&ka=100&submit=">国图方志合集</a> <a href="http://www.guoxuedashi.com/so.php?sokeytm=%E5%90%84%E5%9C%B0%E6%96%B9%E5%BF%97&submit=&kt=1"><strong>各地方志</strong></a>

</div>
</div>


<div class="sidebar2">
<center>

</center>
</div>
<div class="sidebar greenbar">
<div class="sidebar_title green">四库全书</div>
<div class="sidebar_info">

《四库全书》是中国古代最大的丛书,编撰于乾隆年间,由纪昀等360多位高官、学者编撰,3800多人抄写,费时十三年编成。丛书分经、史、子、集四部,故名四库。共有3500多种书,7.9万卷,3.6万册,约8亿字,基本上囊括了古代所有图书,故称“全书”。<a href="http://www.guoxuedashi.com/SiKuQuanShu/">详细>>
</a>

</div> 
</div>

</div>  <!--end r-->

</div>
<!-- 内容区END --> 

<!-- 页脚开始 -->
<div class="shh">

</div>

<div class="w1180" style="margin-top:8px;">
<center><script src="http://www.guoxuedashi.com/img/plus.php?id=3"></script></center>
</div>
<div class="w1180 foot">
<a href="/b/thanks.php">特别致谢</a> | <a href="javascript:window.external.AddFavorite(document.location.href,document.title);">收藏本站</a> | <a href="#">欢迎投稿</a> | <a href="http://www.guoxuedashi.com/forum/">意见建议</a> | <a href="http://www.guoxuemi.com/">国学迷</a> | <a href="http://www.shuowen.net/">说文网</a><script language="javascript" type="text/javascript" src="https://js.users.51.la/17753172.js"></script><br />
  Copyright &copy; 国学大师 古典图书集成 All Rights Reserved.<br>
  
  <span style="font-size:14px">免责声明:本站非营利性站点,以方便网友为主,仅供学习研究。<br>内容由热心网友提供和网上收集,不保留版权。若侵犯了您的权益,来信即刪。scp168@qq.com</span>
  <br />
ICP证:<a href="http://www.beian.miit.gov.cn/" target="_blank">鲁ICP备19060063号</a></div>
<!-- 页脚END --> 
<script src="http://www.guoxuedashi.com/img/plus.php?id=22"></script>
<script src="http://www.guoxuedashi.com/img/tongji.js"></script>

</body>
</html>
