<!DOCTYPE html PUBLIC "-//W3C//DTD XHTML 1.0 Transitional//EN" "http://www.w3.org/TR/xhtml1/DTD/xhtml1-transitional.dtd">
<html xmlns="http://www.w3.org/1999/xhtml">
<head>
<meta http-equiv="Content-Type" content="text/html; charset=utf-8" />
<meta http-equiv="X-UA-Compatible" content="IE=Edge,chrome=1">
<title>資治通鑒_275-資治通鑑卷二百七十四_275-資治通鑑卷二百七十四</title>
<meta name="Keywords" content="資治通鑒_275-資治通鑑卷二百七十四_275-資治通鑑卷二百七十四">
<meta name="Description" content="資治通鑒_275-資治通鑑卷二百七十四_275-資治通鑑卷二百七十四">
<meta http-equiv="Cache-Control" content="no-transform" />
<meta http-equiv="Cache-Control" content="no-siteapp" />
<link href="/img/style.css" rel="stylesheet" type="text/css" />
<script src="/img/m.js?2020"></script> 
</head>
<body>
 <div class="ClassNavi">
<a  href="/24shi/">二十四史</a> | <a href="/SiKuQuanShu/">四库全书</a> | <a href="http://www.guoxuedashi.com/gjtsjc/"><font  color="#FF0000">古今图书集成</font></a> | <a href="/renwu/">历史人物</a> | <a href="/ShuoWenJieZi/"><font  color="#FF0000">说文解字</a></font> | <a href="/chengyu/">成语词典</a> | <a  target="_blank"  href="http://www.guoxuedashi.com/jgwhj/"><font  color="#FF0000">甲骨文合集</font></a> | <a href="/yzjwjc/"><font  color="#FF0000">殷周金文集成</font></a> | <a href="/xiangxingzi/"><font color="#0000FF">象形字典</font></a> | <a href="/13jing/"><font  color="#FF0000">十三经索引</font></a> | <a href="/zixing/"><font  color="#FF0000">字体转换器</font></a> | <a href="/zidian/xz/"><font color="#0000FF">篆书识别</font></a> | <a href="/jinfanyi/">近义反义词</a> | <a href="/duilian/">对联大全</a> | <a href="/jiapu/"><font  color="#0000FF">家谱族谱查询</font></a> | <a href="http://www.guoxuemi.com/hafo/" target="_blank" ><font color="#FF0000">哈佛古籍</font></a> 
</div>

 <!-- 头部导航开始 -->
<div class="w1180 head clearfix">
  <div class="head_logo l"><a title="国学大师官网" href="http://www.guoxuedashi.com" target="_blank"></a></div>
  <div class="head_sr l">
  <div id="head1">
  
  <a href="http://www.guoxuedashi.com/zidian/bujian/" target="_blank" ><img src="http://www.guoxuedashi.com/img/top1.gif" width="88" height="60" border="0" title="部件查字,支持20万汉字"></a>


<a href="http://www.guoxuedashi.com/help/yingpan.php" target="_blank"><img src="http://www.guoxuedashi.com/img/top230.gif" width="600" height="62" border="0" ></a>


  </div>
  <div id="head3"><a href="javascript:" onClick="javascript:window.external.AddFavorite(window.location.href,document.title);">添加收藏</a>
  <br><a href="/help/setie.php">搜索引擎</a>
  <br><a href="/help/zanzhu.php">赞助本站</a></div>
  <div id="head2">
 <a href="http://www.guoxuemi.com/" target="_blank"><img src="http://www.guoxuedashi.com/img/guoxuemi.gif" width="95" height="62" border="0" style="margin-left:2px;" title="国学迷"></a>
  

  </div>
</div>
  <div class="clear"></div>
  <div class="head_nav">
  <p><a href="/">首页</a> | <a href="/ShuKu/">国学书库</a> | <a href="/guji/">影印古籍</a> | <a href="/shici/">诗词宝典</a> | <a   href="/SiKuQuanShu/gxjx.php">精选</a> <b>|</b> <a href="/zidian/">汉语字典</a> | <a href="/hydcd/">汉语词典</a> | <a href="http://www.guoxuedashi.com/zidian/bujian/"><font  color="#CC0066">部件查字</font></a> | <a href="http://www.sfds.cn/"><font  color="#CC0066">书法大师</font></a> | <a href="/jgwhj/">甲骨文</a> <b>|</b> <a href="/b/4/"><font  color="#CC0066">解密</font></a> | <a href="/renwu/">历史人物</a> | <a href="/diangu/">历史典故</a> | <a href="/xingshi/">姓氏</a> | <a href="/minzu/">民族</a> <b>|</b> <a href="/mz/"><font  color="#CC0066">世界名著</font></a> | <a href="/download/">软件下载</a>
</p>
<p><a href="/b/"><font  color="#CC0066">历史</font></a> | <a href="http://skqs.guoxuedashi.com/" target="_blank">四库全书</a> |  <a href="http://www.guoxuedashi.com/search/" target="_blank"><font  color="#CC0066">全文检索</font></a> | <a href="http://www.guoxuedashi.com/shumu/">古籍书目</a> | <a   href="/24shi/">正史</a> <b>|</b> <a href="/chengyu/">成语词典</a> | <a href="/kangxi/" title="康熙字典">康熙字典</a> | <a href="/ShuoWenJieZi/">说文解字</a> | <a href="/zixing/yanbian/">字形演变</a> | <a href="/yzjwjc/">金 文</a> <b>|</b>  <a href="/shijian/nian-hao/">年号</a> | <a href="/diming/">历史地名</a> | <a href="/shijian/">历史事件</a> | <a href="/guanzhi/">官职</a> | <a href="/lishi/">知识</a> <b>|</b> <a href="/zhongyi/">中医中药</a> | <a href="http://www.guoxuedashi.com/forum/">留言反馈</a>
</p>
  </div>
</div>
<!-- 头部导航END --> 
<!-- 内容区开始 --> 
<div class="w1180 clearfix">
  <div class="info l">
   
<div class="clearfix" style="background:#f5faff;">
<script src='http://www.guoxuedashi.com/img/headersou.js'></script>

</div>
  <div class="info_tree"><a href="http://www.guoxuedashi.com">首页</a> > <a href="/SiKuQuanShu/fanti/">四库全书</a>
 > <h1>资治通鉴</h1> <!--         下载:【右键另存为】即可 --></div>
  <div class="info_content zj clearfix">
  
<div class="info_txt clearfix" id="show">
<center style="font-size:24px;">275-資治通鑑卷二百七十四</center>
    資治通鑑卷二百七十四 宋 司馬光 撰<br />
<br />
  胡三省 音註<br />
<br />
  後唐紀三【起旃蒙作噩十一月盡柔兆閹茂三月不滿一年】<br />
<br />
  莊宗光聖神閔孝皇帝下<br />
<br />
  同光三年十一月丙申蜀主至成都百官及後宫迎於七里亭【亭去成都城七里因以為名】蜀主入妃嬪中作回鶻隊入宫【効回鶻曳隊以入宫也】丁酉出見羣臣於文明殿【按五代會要梁開明元年改洛陽宫貞觀殿為文明殿貞觀殿洛陽宫前殿也唐昭宗遷洛後更名今蜀亦有文明殿蜀宫倣唐宫之制意文明唐末殿名也】泣下霑襟君臣相視竟無一言以救國患戊戌李紹琛至利州修桔柏浮梁【桔柏浮梁為蜀所斷故修之以濟】昭武節度使林思鍔先棄城奔閬州【蜀置昭武節度干利州九域志利州東南至閬州二百三十五里】遣使請降甲辰魏王繼岌至劍州【九域志劍州東北至利州一百九十里】蜀武信節度使兼中書令王宗壽以遂合渝瀘昌五州降【蜀置武信軍於遂州】王宗弼至成都登大玄門嚴兵自衛蜀主及太后自往勞之【勞力到翻】宗弼驕慢無復臣禮乙已劫遷蜀主及太后後宫諸王於西宫收其璽綬【璽斯氏翻綬音受】使親吏于義興門邀取内庫金帛悉歸其家其子承㳙仗劍入宫取蜀主寵姬數人以歸【㳙圭淵翻】丙午宗弼自稱權西川兵馬留後李紹琛進至綿州【九域志劍州西至綿州二百八十里】倉庫民居已為蜀兵所燔又斷綿江浮梁【斷丁管翻綿州謂之左綿以綿水逕其左故也】水深無舟楫可渡紹琛謂李嚴曰吾懸軍深入利在速戰乘蜀人破膽之時但得百騎過鹿頭關彼且迎降不暇【降戶江翻下同】若俟修繕橋梁必留數日或教王衍堅閉近關折吾兵勢【近關即謂鹿頭關折之舌翻】儻延旬浹則勝負未可知矣【言深入之兵利于飄忽震蕩難以持久】乃與嚴乘馬浮渡江從兵得濟者僅千人【從才用翻】溺死者亦千餘人遂入鹿頭關丁未進據漢州【九域志綿州西南至漢州一百八十九里】居三日後軍始至宗弼遣使以幣馬牛酒勞軍且以蜀主書遺李嚴【遺唯季翻】曰公來吾即降或謂嚴【或謂嚴者或以人語嚴也】公首建伐蜀之策【事見上卷上年】蜀人怨公深入骨髓不可往嚴不從欣然馳入成都【九域志漢州南至成都九十五里】撫諭吏民告以大軍繼至蜀君臣後宫皆慟哭蜀主引嚴見太后以母妻為託宗弼猶乘城為守備嚴悉命撤去樓櫓【乘登也去羌呂翻】己酉魏王繼岌至綿州蜀主命翰林學士李昊草降表又命中書侍郎同平章事王鍇草降書【降表以上皇帝降書以達軍前鍇口駭翻】遣兵部侍郎歐陽彬奉之以迎繼岌及郭崇韜王宗弼稱蜀君臣久欲歸命而内樞密使宋光嗣景潤澄宣徽使李周輅歐陽晃熒惑蜀主皆斬之函首送繼岌又責文思殿大學士禮部尚書成都尹韓昭佞諛梟于金馬坊門【金馬坊在成都城中以有金馬碧雞祠因而名坊又有碧雞坊】内外馬步都指揮使兼中書令徐延瓊果州團練使潘在迎嘉州刺史顧在珣及諸貴戚皆惶恐傾其家金帛妓妾以賂宗弼僅得免死【妓渠綺翻】凡素所不快者宗弼皆殺之辛亥繼岌至德陽【九域志德陽縣在漢州東北八十五里】宗弼遣使奉牋稱已遷蜀主於西第【已奉表降唐不敢稱西宫故稱西第】安撫軍城以俟王師又使其子承班以蜀主後宫及珍玩賂繼岌及郭崇韜求西川節度使繼岌曰此皆我家物奚以獻為留其物而遣之【宗弼之獻繼岌之留賢不肖之相去其間不能以寸】李紹琛留漢州八日以俟都統【都統繼岌也】甲寅繼岌至漢州王宗弼迎謁乙卯至成都丙辰李嚴引蜀主及百官儀衛出降於升遷橋【按薛史升遷橋在成都北五里】蜀主白衣銜璧牽羊草繩縈首百官衰絰徒跣輿櫬號哭俟命【衰倉回翻櫬初覲翻空棺為櫬號戶刀翻】繼岌受璧崇韜解縛焚櫬承制釋罪君臣東北向拜謝【唐昭宗大順二年王建取蜀至衍而亡】丁巳大軍入成都崇韜禁軍士侵掠市不改肆自出師至克蜀凡七十日 【考異曰實録自興師出洛至定蜀城計七十五日薛史因之按唐軍九月戊申離洛城十一月丁巳入成都止七十日耳實録薛史之誤也】得節度十【武德武信永平武泰鎮江山南武定天雄武興昭武凡十節度西川為蜀都不與也】州六十四【歐史職方考前蜀所有益漢彭蜀綿眉嘉劍梓遂果閬普陵資榮簡卬黎雅維茂文龍黔施夔忠萬歸峽興利開通涪渝瀘合昌巴蓬集壁渠戎梁洋金秦鳳階成五十三州而已】縣二百四十九兵三萬鎧仗錢糧金銀繒錦共以千萬計【繒慈陵翻】高季興聞蜀亡方食失匕箸【箸遲倨翻】曰是老夫之過也【高季興勸伐蜀見二百七十二卷元年】梁震曰不足憂也唐主得蜀益驕亡無日矣【梁震之料莊宗如燭照數計】安知其不為吾福乎【荆南之福則未聞也以三郡之地介乎強國之間惴惴僅能自全何福之有】楚王殷聞蜀亡上表稱臣已營衡麓之間為菟裘之地【衡麓衡山之麓也山足曰麓左傳魯隱公使營菟裘吾將老焉馬殷言將致事而歸老於衡麓聞蜀亡而懼也菟同都翻】願上印綬以保餘齡【齡年也記文王世子曰古者謂年齡齒亦齡也上時掌翻】上優詔慰諭之 平蜀之功李紹琛為多位在董璋上而璋素與郭崇韜善崇韜數召璋與議軍事【數所角翻】紹琛心不平謂璋曰吾有平蜀之功公等樸相從【樸蒲木翻蘇谷翻樸小木以喻董璋小材也】反呫囁於郭公之門【呫囁涉翻囁而涉翻呫囁細語也】謀相傾害吾為都將【帝命李紹琛為行營馬步軍都指揮使董璋為左廂虞侯故云然】獨不能以軍法斬公邪璋訢于崇韜十二月崇韜表璋為東川節度使 【考異曰莊宗實録十二月丙寅以靜難節度使董璋為東川節度副大使又康延孝傳云郭崇韜除董璋為東川節度使延孝與華州節度使毛璋見崇韜請以二部任尚書為東川帥崇韜怒曰紹琛反邪敢違吾節度不及二旬崇韜為繼岌所害按大軍以十一月二十八日丁巳入西川至十二月八日丙寅除董璋東川凡十日明年正月八日殺崇韜至此凡六十日而云不及二旬崇韜遇害日月殊不相合蓋十二月丙寅崇韜始表璋鎮東川之日耳非降制日也不及二旬亦恐誤】解其軍職【解董璋軍職則李紹琛不得以軍法令之此崇韜之所以保護董璋者也】紹琛愈怒曰吾冒白刃陵險阻定兩川璋乃坐有之邪【冒莫北翻】乃見崇韜言東川重地任尚書有文武才宜表為帥【任圜時以工部尚書參預軍機帥所類翻】崇韜怒曰紹琛反邪何敢違吾節度紹琛懼而退初帝遣宦者李從襲等從魏王繼岌伐蜀繼岌雖為都統軍中制置補署一出郭崇韜崇韜終日决事將吏賓客趨走盈庭而都統府惟大將晨謁外牙門索然【索蘇各翻索然言寂寞也】從襲等固恥之及破蜀蜀之貴臣大將爭以寶貨妓樂遺崇韜及其子廷誨【妓渠綺翻遺唯季翻】魏王所得不過匹馬束帛唾壺塵柄而已【塵之庾翻】從襲等益不平王宗弼之自為西川留後也賂崇韜求為節度使崇韜陽許之 【考異曰實録薛史皆云崇韜以蜀帥許之按崇韜有識畧豈可興大兵取西川反以與宗弼乎此庸人所不為也蓋于時宗弼尚據成都崇韜恐其悔而違拒故陽許之以安其意耳】既而久未得乃帥蜀人列狀見繼岌請留崇韜鎮蜀【帥讀曰率】從襲等因謂繼岌曰郭公父子專横【横戶孟翻】今又使蜀人請已為帥【帥所類翻】其志難測王不可不為之備繼岌謂崇韜曰主上倚侍中如山嶽不可離廟堂【郭崇韜官侍中故繼岌稱之離力智翻】豈肯棄元老於蠻夷之域乎且此非余之所敢知也請諸人詣闕自陳由是繼岌與崇韜互相疑【此段自平蜀之功以下為李紹琛反張本自初帝遣李從襲從繼岌以下為殺郭崇韜張本】會宋光葆自梓州來訢王宗弼誣殺宋光嗣等又崇韜徵犒軍錢數萬緡於宗弼宗弼靳之【犒苦到翻靳居焮翻】士卒怨怒夜縱火諠譟崇韜欲誅宗弼以自明己巳白繼岌收宗弼及王宗勲王宗渥皆數其不忠之罪【數所角翻】族誅之籍没其家蜀人爭食宗弼之肉 辛未閩忠懿王審知卒【年六十四】子延翰自稱威武留後【延翰字子逸審知長子也】汀州民陳本聚衆三萬圍汀州延翰遣右軍都監柳邕等將兵二萬討之【監才銜翻】 癸酉王承休王宗汭至成都【十月自秦州上道為始至成都】魏王繼岌詰之曰居大鎮擁彊兵何以不拒戰對曰畏大王神武曰然則何以不降對曰王師不入境曰所俱入羌者幾人對曰萬二千人曰今歸者幾人對曰二千人曰可以償萬人之死矣皆斬之并其子 丙子以知北都留守事孟知祥為西川節度使同平章事促召赴洛陽【召之至洛陽而後赴鎮為孟知祥據蜀張本】帝議選北都留守樞密承旨段徊等惡鄴都留守張憲不欲其在朝廷【段徊必宦人也】皆曰北都非張憲不可憲雖有宰相器【郭崇韜薦張憲為相帝欲用之故段徊等云然】今國家新得中原宰相在天子目前事有得失可以改更【更工衡翻】比之北都獨繫一方安危不為重也乃徙憲為太原尹知北都留守事【以尹知留守事非正為留守也】以戶部尚書王正言為興唐尹知鄴都留守事正言昏耄帝以武德使史彦瓊為鄴都監軍【後唐武德使本掌宫中事明宗時嘗旱已而雪暴坐庭中詔武德司宫中無掃雪是其證也】彦瓊本伶人也有寵於帝魏博等六州軍旅金穀之政皆决於彦瓊威福自恣陵忽將佐自正言以下皆諂事之【為王正言史彦瓊不能守鄴都張本】 初帝得魏州銀槍効節都近八千人以為親軍【見二百六十九卷梁均王貞明元年近其靳翻】皆勇悍無敵夾河之戰實賴其用屢立殊功常許以滅梁之日大加賞賚既而河南平【梁滅而河南平】雖賞賚非一而士卒恃功驕恣無厭【厭于鹽翻】更成怨望是歲大饑民多流亡租賦不充道路塗潦漕輦艱澀【漕水運輦陸運澀色立翻】東都倉廩空竭無以給軍士租庸使孔謙日於上東門外【洛城東面三門中曰建春左曰上東右曰永通九域志洛陽上東門建春門皆為鎮屬河南縣蓋喪亂丘墟非復盛唐之舊也】望諸州漕運至者隨以給之軍士乏食有雇妻鬻子者老弱採蔬於野百十為羣往往餒死流言怨嗟而帝遊畋不息己卯獵於白沙皇后皇子後宫畢從庚辰宿伊關辛巳宿潭泊壬午宿龕澗癸未還宫【自白沙至龕澗其地皆在洛陽東按薛史李愚避難居洛表白沙之别墅龕澗近伊闕從才用翻龕苦含翻】時大雪吏卒有僵仆於道路者伊汝間饑尤甚衛兵所過責其供餉不得則壞其什器【僵居良翻壞音怪】撤其室廬以為薪甚於寇盜縣吏皆竄匿山谷 有白龍見于漢宫漢主改元白龍更名曰龔【見賢遍翻更工衡翻】 長和驃信鄭旻遣其布燮鄭昭淳求昏於漢漢主以女增城公主妻之長和即唐之南詔也【唐末南詔改曰大禮至是又改曰長和五代會要曰郭崇韜平蜀之後得王衍所得蠻俘數十以天子命令使人入其部被止于界上惟國信蠻俘得往續有轉牒稱督爽大長和國宰相布燮等上大唐皇帝舅奏疏一封差人轉送黎州其紙厚硬如皮筆力遒健有詔體後有督爽陀酋忍爽王寶督彌勒忍爽董德義督爽長垣緯忍爽楊希燮等所署有采牋一軸轉韻詩一章章三韻共十聯有類擊筑詞頗有本朝姻親之意語亦不遜】 成德節度使李嗣源入朝 閏月己丑朔孟知祥至洛陽帝寵待甚厚 帝以軍儲不足謀於羣臣豆盧革以下皆莫知為計吏部尚書李琪上疏以為古者量入以為出【量音良】計農而發兵故雖有水旱之災而無匱乏之憂近代税農以養兵未有農富給而兵不足農捐瘠而兵豐飽者也今縱未能蠲省租税苟除折納紐配之法【折納謂抑民使折估而納其所無紐配謂紐數而科配之也】農亦可以小休矣帝即敕有司如琪所言然竟不能行 丁酉詔蜀朝所署官四品以上降授有差五品以下才地無取者悉縱歸田里其先降及有功者委崇韜隨事奬任又賜王衍詔畧曰固當裂土而封必不薄人于險三辰在上一言不欺【誓之以三辰而終殺之非信也】 庚子彰武保大節度使兼中書令高萬興卒【梁貞明四年高萬興兼鎮鄜延唐以延州置保塞軍岐改為忠義軍後唐改為彰武軍鄜保大軍】以其子保大留後允韜為彰武留後 帝以軍儲不充欲如汴州諫官上言不如節儉以足用自古無就食天子今楊氏未滅不宜示以虛實【謂吳近在淮南不宜使之知中國虛實上時掌翻】乃止 辛亥立皇弟存美為邕王存霸為永王存禮為薛王存渥為申王存乂為睦王存確為通王存紀為雅王 郭崇韜素疾宦官嘗密謂魏王繼岌曰大王它日得天下騬馬亦不可乘【騬食陵翻犗馬也以喻宦官史照曰犗音戒俗呼扇馬為改馬即犗馬也】况任宦官宜盡去之專用士人【去羌呂翻】呂知柔竊聽聞之【呂知柔時為都統牙通謁】由是宦官皆切齒時成都雖下而蜀中盜賊羣起布滿山林崇韜恐大軍既去更為後患命任圜張筠分道招討以是淹留未還帝遣宦者向延嗣促之崇韜不出郊迎及見禮節又倨【宦官固可疾然天子使之將命敬之者所以敬君也烏可倨見哉唐莊宗使刑臣將命于大臣非也郭崇韜倨見之亦非也嗚呼刑臣將命自唐開元以後皆然矣】延嗣怒李從襲謂延嗣曰魏王太子也主上萬福而郭公專權如是郭廷誨擁徒出入日與軍中驍將蜀土豪傑狎飲指天畫地近聞白其父請表已為蜀帥【帥所類翻下同】又言蜀地富饒大人宜善自為謀今諸軍將校皆郭氏之黨王寄身於虎狼之口一朝有變吾屬不知委骨何地矣因相向垂涕延嗣歸具以語劉后【語牛倨翻下語之同】后泣訢于帝請早救繼岌之死前此帝聞蜀人請崇韜為帥已不平至是聞延嗣之言不能無疑帝閲蜀府庫之籍曰人言蜀中珍貨無算何如是之微也延嗣曰臣聞蜀破其珍貨皆入於崇韜父子崇韜有金萬兩銀四十萬兩錢百萬緡名馬千匹它物稱是廷誨所取復在其外【稱尺證翻復扶又翻】故縣官所得不多耳帝遂怒形于色及孟知祥將行帝語之曰聞郭崇韜有異志卿到為朕誅之【為于偽翻】知祥曰崇韜國之勲舊不宜有此俟臣至蜀察之苟無它志則遣還【還從宣翻又如字】帝許之壬子知祥發洛陽帝尋復遣衣甲庫使馬彦珪【復扶又翻下后復同衣甲庫使盛唐無之蓋帝所置亦内諸司使之一也】馳詣成都觀崇韜去就如奉詔班師則已若有遷延跋扈之狀則與繼岌圖之【觀莊宗所以命孟知祥馬彦珪者如此就使李從襲等不以劉后教行之崇韜得東還亦必不能自全矣】彦珪見皇后說之曰【說式芮翻】臣見向延嗣言蜀中事勢憂在朝夕今上當斷不斷【言帝詔旨持兩端無决然使殺崇韜之命斷丁亂翻】夫成敗之機間不容髪安能緩急稟命于三千里外乎【成都至洛陽三千二百一十六里見舊唐書地理志】皇后復言于帝帝曰傳聞之言未知虛實豈可遽爾果决皇后不得請退自為教與繼岌令殺崇韜知祥行至石壕【石壕村在陜縣東新安縣西杜少陵詩所謂暮投石壕村者也九域志陜州陜縣有石壕鎮】彦珪夜叩門宣詔促知祥赴鎮知祥竊歎曰亂將作矣乃晝夜兼行【孟知祥倍道而行非能救郭崇韜之死也恐崇韜死而生它變耳】 初楚王殷既得湖南不征商旅由是四方商旅輻湊湖南地多鉛鐵殷用軍都判官高郁策【軍都判官諸軍都判官也高郁在馬殷府其位任在行軍司馬之上】鑄鉛鐵為錢商旅出境無所用之皆易它貨而去故能以境内所餘之物易天下百貨國以富饒湖南民不事桑蠶郁命民輸税者皆以帛代錢未幾民間機杼大盛【幾居豈翻高郁佐馬殷治湖南巧于使民而民勸趨于利蓋學管子之術者也】 吳越王鏐遣使者沈瑫致書以受玉冊封吳越國王告於吳【瑫土刀翻】吳人以其國名與已同【嫌其居越而兼吳國之名】不受書遣瑫還仍戒境上無得通吳越使者及商旅<br />
<br />
  明宗聖德和武欽孝皇帝上之上<br />
<br />
  【諱嗣源應州人世本夷狄無姓氏父電雁門都將帝少名邈佶烈太祖養以為子乃姓李名嗣源即位後改名亶】<br />
<br />
  天成元年【是年四月方改元見下卷】春正月庚申魏王繼岌遣李繼曮李嚴部送王衍及其宗族百官數千人詣洛陽河中節度使尚書令李繼麟自恃與帝故舊且有<br />
<br />
  功【梁之乾化二年朱友謙即以河中附晉故自恃故舊自附晉之後晉王與梁人戰于河上汾晉無後顧之虞以此為有功】帝待之厚【亦以此自恃】苦諸伶宦求匄無厭【厭于鹽翻】遂拒不與大軍之征蜀也繼麟閲兵遣其子令德將之以從景進與宦官譖之曰繼麟聞大軍起以為討已故驚懼閲兵自衛又曰崇韜所以敢倔彊於蜀者【從才用翻倔其勿翻彊其兩翻】與河中隂謀内外相應故也繼麟聞之懼欲身入朝以自明其所親止之繼麟曰郭侍中功高於我今事勢將危吾得見主上面陳至誠則讒人獲罪矣【郭侍中謂崇韜功高以其有滅梁蜀之功非已之所能及也讒人指伶宦也】癸亥繼麟入朝【為繼麟得禍張本】 魏王繼岌將發成都令任圜權知留事以俟孟知祥諸軍部署已定【部署行留已定也】是日馬彦珪至以皇后教示繼岌繼岌曰大軍垂發【垂發猶言臨發也】彼無亹端安可為此負心事公輩勿復言【復扶又翻】且主上無敕獨以皇后教殺招討使可乎李從襲等泣曰既有此迹萬一崇韜聞之中塗為變益不可救矣相與巧陳利害繼岌不得已從之甲子旦從襲以繼岌之命召崇韜計事繼岌登樓避之崇韜方升階繼岌從者李環撾碎其首并殺其子廷誨廷信【撾則瓜翻郭崇韜蓋與二子俱至繼岌所故同時見殺】外人猶未之知都統推官滏陽李崧謂繼岌曰今行軍三千里外初無敕旨擅殺大將大王奈何行此危事獨不能忍之至洛陽邪繼岌曰公言是也悔之無及崧乃召書吏數人登樓去梯【去羌呂翻】矯為敕書用蠟印宣之【以蠟摹刊為中書省印以印敕書而宣之也】軍中粗定崇韜左右皆竄匿獨掌書記滏陽張礪詣魏王府慟哭久之【張礪為崇韜府掌書記史言其事府主能始終】繼岌命任圜代崇韜總軍政 魏王通謁李廷安獻蜀樂工二百餘人有嚴旭者王衍用為蓬州刺史帝問曰汝何以得刺史對曰以歌帝使歌而善之許復故任【人皆謂帝克蜀而不察蜀之所以亡故不旋踵而敗不知此乃帝氣習也觀諸李存賢周匝之事可見】 戊辰孟知祥至成都時新殺郭崇韜人情未安知祥慰撫吏民犒賜將卒去留帖然【史言孟知祥之才所以能有蜀犒苦到翻】 閩人破陳本斬之【陳本圍汀州見上年十二月】 契丹主擊女真及勃海【女真始見於此其國本肅慎氏東漢謂之挹婁元魏謂之勿吉隋唐謂之靺鞨五代時始號女真女真有數種居混同江之南者為熟女真江之北者為生女真混同江即鴨渌水】恐唐乘虛襲之戊寅遣美楞嘉哩來修好【好呼到翻】馬彦珪還洛陽乃下詔暴郭崇韜之罪并殺其子廷說廷讓廷議【此郭崇韜諸子之在洛陽者也說讀曰悦】於是朝野駭惋【朝直遥翻惋烏貫翻】羣議紛然帝使宦者潛察之保大節度使睦王存乂崇韜之壻也宦者欲盡去崇韜之黨言存乂對諸將攘臂垂泣為崇韜稱寃【去羌呂翻為于偽翻】言辭怨望庚辰幽存乂於第尋殺之景進言河中人有告變言李繼麟與郭崇韜謀反崇韜死又與存乂連謀宦官因共勸帝速除之帝乃徙繼麟為義成節度使是夜遣蕃漢馬步使朱守殷以兵圍其第【歐史作圍其館蓋謂朱友謙無私第在洛陽也】驅繼麟出徽安門外殺之復其姓名曰朱友謙【唐昭宗之遷洛也車駕由徽安門入宫唐六典東都北面二門東曰延喜西曰徽安朱有謙賜姓名見二百七十二卷元年】友謙二子令德為武信節度使令錫為忠武節度使詔魏王繼岌誅令德於遂州鄭州刺史王思同誅令錫於許州【唐置忠武軍於許州匡國軍于同州至梁之時兩易軍號後唐滅梁皆復其故】河陽節度使李紹奇誅其家人於河中紹奇至其家友謙妻張氏帥家人二百餘口見紹奇【帥讀曰率】曰朱氏宗族當死願無濫及平人乃别其婢僕百人【别彼列翻】以其族百口就刑張氏又取鐵劵以示紹奇曰此皇帝去年所賜也我婦人不識書不知其何等語也紹奇亦為之慙【為于偽翻慙朝廷之失信】友謙舊將史武等七人時為刺史皆坐族誅時洛中諸軍饑窘【窘渠隕翻】妄為謡言伶官采之以聞於帝故朱友謙郭崇韜皆及於禍成德節度使兼中書令李嗣源亦為謡言所屬【屬之欲翻】帝遣朱守殷察之守殷私謂嗣源曰令公勲業振主宜自圖歸藩以遠禍【振當作震遠於願翻】嗣源曰吾心不負天地禍福之來無所可避皆委之於命耳【李嗣源答朱守殷之言安於死生禍福之際英雄識度自有不可及者】時伶宦用事勲舊人不自保嗣源危殆者數四賴宣徽使李紹宏左右營護以是得全 魏王繼岌留馬步都指揮使陳留李仁罕馬軍都指揮使東光潘仁嗣左廂都指揮使趙廷隱右廂都指揮使浚儀張業牙内指揮使文水武漳驍鋭指揮使平恩李延厚戍成都【為諸將在蜀卒為孟知祥効死張本】甲申繼岌發成都命李紹琛帥萬二千人為後軍行止常差中軍一舍【三十里為一舍差後於中軍三十里也帥讀曰率】 二月己丑朔以宣徽南院使李紹宏為樞密使【代郭崇韜也】魏博指揮使楊仁晸【晸知領翻】將所部兵戍瓦橋【將即亮翻】踰年代歸至貝州以鄴都空虛恐兵至為變敕留屯貝州時天下莫知郭崇韜之罪民間譌言云崇韜殺繼岌自王於蜀【王于况翻】故族其家朱友謙子建徽為澶州刺史帝密敕鄴都監軍史彦瓊殺之【澶州魏博巡屬也故密敕魏博監軍殺朱建徽澶時連翻】門者白留守王正言曰史武德夜半馳馬出城不言何往【史彦瓊以武德使出為監軍稱其内職】又譌言云皇后以繼岌之死歸咎於帝已弑帝矣故急召彦瓊計事人情愈駭【譌言方興而史彦瓊所為有可疑可駭者譌言所以益甚而亂隨之】楊仁晸部兵皇甫暉與其徒夜博不勝因人情不安遂作亂刼仁晸曰主上所以有天下吾魏軍力也【謂因魏博兵力以破梁】魏軍甲不去體馬不解鞍者十餘年今天下已定天子不念舊勞更加猜忌遠戍踰年方喜代歸去家咫尺不使相見【言使之留屯貝州不許還魏州也九域志貝州南至魏州二百二十五里】今聞皇后弑逆京師已亂將士願與公俱歸仍表聞朝廷若天子萬福興兵致討以吾魏博兵力足以拒之【皇甫暉銀槍效節卒也從莊宗戰河上習見莊宗之用兵與夫諸軍之勇怯故敢發此言】安知不更為富貴之資乎仁晸不從暉殺之又劫小校不從又殺之【校戶教翻】效節指揮使趙在禮聞亂衣不及帶踰垣而走暉追及曳其足而下之示以二首【示以楊仁晸及小校之首】在禮懼而從之亂兵遂奉以為帥【帥所類翻】焚掠貝州暉魏州人在禮涿州人也詰旦暉等擁在禮南趣臨清永濟館陶所過剽掠【趣七喻翻剽匹妙翻】壬辰晩有自貝州來告軍亂將犯鄴都者都巡檢使孫鐸等亟詣史彦瓊【亟紀力翻急也】請授甲乘城為備彦瓊疑鐸等有異志曰告者云今日賊至臨清計程須六日晩方至【九域志臨清縣南至魏州城一百五十里皇甫暉等以壬辰至臨清史彦瓊以為六日晩方至魏州者以師行日五十里故計其涉三日方至也壬辰二月四日六日謂二月六日也是日甲午】為備未晩孫鐸曰賊既作亂必乘吾未備晝夜倍道安肯計程而行請僕射帥衆乘城【史彦瓊蓋加僕射故孫鐸稱之帥讀曰率】鐸募勁兵千人伏於王莽河逆擊之賊既勢挫必當離散然後可撲討也【撲普木翻】必俟其至城下萬一有姦人為内應則事危矣彦瓊曰但嚴兵守城何必逆戰是夜賊前鋒攻北門弓弩亂發時彦瓊將部兵宿北門樓聞賊呼聲【呼火故翻】即時驚潰彦瓊單騎奔洛陽癸巳賊入鄴都孫鐸等拒戰不勝亡去趙在禮據宫城【帝即位于魏州以牙城為宫城】署皇甫暉及軍校趙進為馬步都指揮使縱兵大掠進定州人也王正言方據按召吏草奏無至者正言怒其家人曰賊已入城殺掠於市吏皆逃散公尚誰呼正言驚曰吾初不知也又索馬不能得【索山客翻】乃帥僚佐步出府門謁在禮【帥讀曰率】再拜請罪在禮亦拜曰士卒思歸耳尚書重德勿自卑屈慰諭遣之【王正言以戶部尚書出知留守故趙在禮稱之】衆推在禮為魏博留後具奏其狀北京留守張憲家在鄴都【去年張憲自鄴都留守遷北京故其家尚留鄴都】在禮厚撫之遣使以書誘憲憲不發封斬其使以聞【使疏吏翻誘音酉】 甲午以景進為銀青光禄大夫檢校右散騎常侍兼御史大夫上柱國 丙申史彦瓊至洛陽【自鄴都逃至洛陽】帝問可為大將者於樞密使李紹宏紹宏復請用李紹欽【伐蜀之役李紹宏已薦李紹欽而不用故言復】帝許之令條上方畧【上時掌翻】紹欽所請偏禆皆梁舊將已所善者帝疑之而止皇后曰此小事不足煩大將紹榮可辦也【紹榮元行欽】帝乃命歸德節度使李紹榮將騎三千詣鄴招撫【將即亮翻下同騎奇寄翻】亦徵諸道兵備其不服 郭崇韜之死也李紹琛謂董璋曰公復欲呫囁誰門乎【復扶又翻】璋懼謝罪魏王繼岌軍還至武連【還從宣翻武連漢梓潼縣界宋置武都郡及下辨縣又改下辨為武功縣後魏改為武連縣唐屬劍州九域志在州西八十五里】遇敕使諭以朱友謙已伏誅令董璋將兵之遂州誅朱令德時紹琛將後軍在魏城【西魏置魏城縣于巴西唐屬綿州九域志在州東六十五里宋白曰魏州本漢涪縣地西魏於涪縣立潼州析此立為魏城縣李膺記云肆溪東五十里有東西井井西為涪縣界井東為魏城界】聞之以帝不委已殺令德而委璋大驚俄而璋過紹琛軍不謁紹琛怒乘酒謂諸將曰國家南取大梁西定巴蜀皆郭公之謀而吾之戰功也至於去逆效順與國家犄角以破梁則朱公也【犄居蟻翻謂朱友謙以蒲同附晉相為犄角以破梁】今朱郭皆無罪族滅歸朝之後行及我矣【朝直遥翻】寃哉天乎奈何紹琛所將多河中兵【將即亮翻】河中將焦武等號哭於軍門曰西平王何罪闔門屠膾【號戶刀翻朱友謙再以河中附晉晉封為西平王闔門屠劊謂其家悉誅夷也】我若歸則與史武等同誅【言史武等既以河中將誅若東歸則亦與之同罪而誅死】决不復東矣【復扶又翻】是日魏王繼岌至泥溪紹琛至劍州遣人白繼岌云河中將士號哭不止欲為亂丁酉紹琛自劍州擁兵西還自稱西川節度三川制置等使移檄成都稱奉詔代孟知祥招諭蜀人三日間衆至五萬 戊戌李繼曮至鳳翔監軍使柴重厚不以符印與之促令詣闕【唐僖宗光啟三年李茂貞據鳳翔至是而代其後明宗復令李繼曮鎮鳳翔】 己亥魏王繼岌至利州李紹琛遣人斷桔柏津【斷丁管翻桔吉屑翻】繼岌聞之以任圜為副招討使將步騎七千與都指揮使梁漢顒【顒魚容翻】監軍李延安追討之 【考異曰莊宗實録己亥繼岌奏康延孝叛遣任圜追討按延孝丁酉叛于劍州豈得己亥奏報已至洛廣本己亥魏王至利州桔柏津使夜來告繼岌言李紹琛令斷浮梁繼岌署任圜為副招討使令率七千人騎與都指揮使梁漢顒監軍李延安討之今從之】 庚子邢州左右步直兵趙太等四百人【步直兵謂步兵長直者也】據城自稱安國留後詔東北面招討副使李紹真討之【李紹真即霍彦威】 辛丑任圜先令别將何建崇擊劍門關下之【恐李紹琛拒守劍門關故先擊下之紹琛將何所至哉】 李紹榮至鄴都攻其南門遣人以敕招諭之趙在禮以羊酒犒師拜於城上曰將士思家擅歸相公誠善為敷奏【犒苦到翻李紹榮以節度使同平章事故稱之為相公所謂使相也後之世凡建節者皆稱相公為于偽翻】得免於死敢不自新遂以敕徧諭軍士史彦瓊戟手大罵曰羣死賊城破萬段皇甫暉謂其衆曰觀史武德之言上不赦我矣因聚譟掠敕書手壞之【掠奪也壞音怪】守陴拒戰紹榮攻之不利以狀聞帝怒曰克城之日勿遺噍類【噍才笑翻】大發諸軍討之壬寅紹榮退屯澶州 甲辰夜從馬直軍士王温等五人殺軍使謀作亂擒斬之【從才用翻】從馬直指揮使郭從謙本優人也優名郭門高帝與梁相拒於得勝【得勝即德勝】募勇士挑戰從謙應募俘斬而還【挑徒了翻還從宣翻又如字】由是益有寵帝選諸軍驍勇者為親軍分置四指揮號從馬直從謙自軍使積功至指揮使郭崇韜方用事從謙以叔父事之睦王存乂以從謙為假子及崇韜存乂得罪從謙數以私財饗從馬直諸校【數所角翻校戶教翻】對之流涕言崇韜之寃及王温作亂帝戲之曰汝既負我附崇韜存乂又教王温反欲何為也從謙益懼既退隂謂諸校曰主人以王温之故俟鄴都平定盡阬若曹【若猶汝也】家之所有宜盡市酒肉勿為久計也由是親軍皆不自安【為張破敗作亂郭從謙弑逆張本郭崇韜勲舊也以無罪而族康延孝之亂皇甫暉之亂張破敗之亂卒以成郭從謙之弑皆由崇韜之死而將校之心不自安也】乙巳王衍至長安有詔止之【止不使至洛陽】 先是帝諸弟<br />
<br />
  雖領節度使皆留京師但食其俸【先悉薦翻】戊申始命護國節度使永王存霸至河中【既殺朱友謙故令存霸赴鎮以代之】 丁未李紹榮以諸道兵再攻鄴都庚戌禆將楊重霸帥衆數百登城【帥讀曰率】後無繼者重霸等皆死賊知不赦堅守無降意【降戶江翻】朝廷患之日發中使促魏王繼岌東還繼岌以中軍精兵皆從任圜討李紹琛留利州待之未得還【還從宣翻又如字】李紹榮討趙在禮久無功趙太據邢州未下滄州軍亂小校王景戡討定之因自為留後河朔州縣告亂者相繼帝欲自征鄴都宰相樞密使皆言京師根本車駕不可輕動帝曰諸將無可使者皆曰李嗣源最為勲舊帝心忌嗣源曰吾惜嗣源欲留宿衛皆曰它人無可者忠武節度使張全義亦言河朔多事久則患深宜令總管進討【時李嗣源雖留洛陽而蕃漢内外馬步軍都總管之官如故】若倚紹榮輩未見成功之期李紹宏亦屢言之帝以内外所薦【内則李紹宏外則張全義及在廷之臣】甲寅命嗣源將親軍討鄴都 延州言綏銀軍亂剽州城【綏銀時為夏州巡屬延州以鄰鎮奏言之耳趙珣聚米圖經宋康定慶歷間所進也其書云綏州故城見在延州東北無定河川西至夏州四百里南至延州界三百四十里北至銀州一百六十里夏州東至銀州二百里剽匹妙翻】 董璋將兵二萬屯綿州會任圜討李紹琛帝遣中使崔延琛至成都遇紹琛軍紿之曰吾奉詔召孟郎【孟知祥妻太祖弟克讓女也故呼為孟郎俗謂壻為郎也】公若緩兵自當得蜀既至成都勸孟知祥為戰守備知祥浚壕樹柵遣馬步都指揮使李仁罕將四萬人驍鋭指揮使李延厚將二千人討紹琛【既浚壕樹柵為守城之備又遣重兵出討以兵有邂逅戰苟不利則退守無倉卒失措之憂孟知祥初至西川其審慎如此然當時蜀之舊兵敗散已多北兵留戍計不過數千李仁罕所將未必及四萬之數更須博考】延厚集其衆詢之曰有少壯勇鋭欲立功求富貴者東【少詩照翻】衰疾畏懦厭行陳者西【行戶剛翻】得選兵七百人以行【兵不貴多而貴精也】是日任圜軍追及紹琛於漢州紹琛出兵逆戰招討掌書記張礪請伏精兵於後以羸兵誘之【郭崇韜之為招討使也以張礪為掌書記崇韜既死繼岌以任圜為招討副使以討李紹琛故礪以募屬從軍羸倫為翻誘音酉】圜從之使董璋以東川羸兵先戰而却紹琛輕圜書生又見其兵羸極力追之伏兵發大破之斬首數千級自是紹琛入漢州閉城不出 三月丁巳朔李紹真奏克邢州擒趙太等庚申紹真引兵至鄴都營於城西北以太等徇於鄴都城下而殺之【是不足以懼皇甫暉等適以堅其死守之心耳】 辛酉以威武節度副使王延翰為威武節度使【命王延翰嗣有閩土】 壬戌李嗣源至鄴都營於城西南甲子嗣源下令軍中詰旦攻城【詰去吉翻】是夜從馬直軍士張破敗作亂 【考異曰莊宗實録壬戌今上至鄴都癸亥夜張破敗作亂明日入鄴都明宗實錄三月六日帝至鄴都八日夜破敗作亂薛史莊宗紀壬子嗣源至鄴都甲寅夜破敗作亂明宗紀與實録同按長歷此月丁巳朔無壬子甲寅今從實録及明宗本紀】帥衆大譟【帥讀曰率下同】殺都將【將即亮翻】焚營舍詰旦亂兵逼中軍嗣源帥親軍拒戰不能敵亂兵益熾【從亂者愈衆也】嗣源叱而問之曰爾曹欲何為對曰將士從主上十年百戰以得天下今主上弃恩任威貝州戍卒思歸主上不赦云克城之後當盡阬魏博之軍【謂皇甫暉等也莊宗忿暉等不降嘗有克城之日勿遺噍類之語】近從馬直數卒諠競遽欲盡誅其衆【謂王温等亂也郭從謙因王温亂後矯言帝意以扇動張破敗等之亂心】我輩初無叛心但畏死耳今衆議欲與城中合勢撃退諸道之軍請主上帝河南令公帝河北為軍民之主【李嗣源官中書令故稱之為令公】嗣源泣諭之不從嗣源曰爾不用吾言任爾所為我自歸京師亂兵拔白刃環之【環音宦】曰此輩虎狼也不識尊卑令公去欲何之因擁嗣源及李紹真等入城城中不受外兵皇甫暉逆撃張破敗斬之外兵皆潰趙在禮帥諸校迎拜嗣源泣謝曰將士輩負令公【李嗣源以蕃漢馬步軍都總管統諸軍禦契丹凡河北諸鎮兵皆屬焉而魏兵作亂是負之也】敢不惟命是聽嗣源詭說在禮曰【說式芮翻】凡舉大事須藉兵力今外兵流散無所歸我為公出收之【藉慈夜翻為于偽翻外兵謂城外之兵嗣源紹真所領者也】在禮乃聽嗣源紹真俱出城宿魏縣散兵稍有至者 漢州無城塹樹木為柵乙丑任圜進攻其柵縱火焚之李紹琛引兵出戰於金雁橋【金雁橋在漢州雒縣東雁江之上俗傳曾有金雁故名】兵敗與十餘騎奔綿竹【九域志綿竹縣在漢州東北九十三里】追擒之孟知祥自至漢州犒軍與任圜董璋置酒高會引李紹琛檻車至座中知祥自酌大巵飲之【飲於禁翻】謂曰公已擁節旄又有平蜀之功何患不富貴而求入此檻車邪紹琛曰郭侍中佐命功第一兵不血刃取兩川一旦無罪族誅【郭侍中謂崇韜】如紹琛輩安保首領以此不敢歸朝耳【朝直遥翻】魏王繼岌既獲紹琛乃引兵倍道而東孟知祥獲陜虢都指揮使汝隂李肇河中都指揮使千乘侯弘實【陜失再翻乘繩證翻】以肇為牙内馬步都指揮使弘實副之【為李肇等為孟知祥用張本】蜀中羣盜猶未息知祥擇亷吏使治州縣蠲除横賦安集流散下寛大之令與民更始【孟知祥已有據蜀規摹治直之翻横戶孟翻更工衡翻】遣左廂都指揮使趙廷隱右廂都指揮使張業將兵分討羣盜悉誅之 李嗣源之為亂兵所逼也李紹榮有衆萬人營於城南嗣源遣牙將張䖍釗高行周等七人相繼召之欲與共誅亂者紹榮疑嗣源之詐留使者閉壁不應及嗣源入鄴都遂引兵去嗣源在魏縣衆不滿百又無兵仗李紹真所將鎮兵五千聞嗣源得出相帥歸之【鎮兵蓋鎮州兵也李嗣源本鎮鎮州故其兵相帥歸之帥讀曰率】由是嗣源兵稍振嗣源泣謂諸將曰吾明日當歸藩【欲歸鎮州也】上章待罪【上時掌翻章表也奏也】聽主上所裁李紹真及中門使安重誨曰此策非宜公為元帥不幸為凶人所劫李紹榮不戰而退歸朝必以公藉口【言李紹榮必奏天子稱已所以退師者以嗣源入魏與賊合也】公若歸藩則為據地邀君適足以實讒慝之言耳不若星行詣闕【星行者戴星而行也】面見天子庶可自明嗣源曰善丁卯自魏縣南趣相州【趣七喻翻】遇馬坊使康福【後唐起于太原馬牧多在并代莊宗在河上與梁戰置馬牧于相州以康福為小馬坊使以鎮之蓋以并代之廏牧為大馬坊也唐内諸司有小馬坊使宦官為之非此薛史唐莊宗曰康福體貌豐厚可令總轄馬牧由是署為馬坊使及明宗離魏縣會福牧小馬於相州乃驅而歸命】得馬數千匹始能成軍福蔚州人也【蔚紆勿翻】 平盧節度使苻習將本軍攻鄴都聞李嗣源軍潰引兵歸至淄州監軍使楊希望遣兵逆擊之【平盧節度治青州九域志青州西至淄州一百一十三里】習懼復引兵而西【復扶又翻】青州指揮使王公儼攻希望殺之因據其城時近侍為諸道監軍者【宦官常侍天子左右故曰近侍】皆恃恩與節度使爭權及鄴都軍變所在多殺之安義監軍楊繼源謀殺節度使孔勍勍先誘而殺之【勍渠京翻誘音酉】武寧監軍以李紹真從李嗣源謀殺其元從【元從謂舊從李紹真之將士所謂義故也紹真時從李嗣源監軍謀殺其元從之留彭城者】據城拒之權知留後淳于晏帥諸將先殺之【帥讀曰率下同】晏登州人也戊辰以軍食不足敕河南尹豫借夏秋税民不聊生忠武節度使尚書令齊王張全義聞李嗣源入鄴都憂懼不食辛未卒於洛陽【張全義之憂死自以薦李嗣原北討也】 租庸使以倉儲不足頗朘刻軍糧【朘息緣翻縮也減也】軍士流言益甚宰相懼帥百官上表言今租庸已竭内庫有餘諸軍室家不能相保儻不賑救【賑津忍翻】懼有離心俟過凶年其財復集【復扶又翻集聚也】上即欲從之劉后曰吾夫婦君臨萬國雖藉武功亦由天命命既在天人如我何【紂責命于天紂所以亡未聞妲已有是言也】宰相又於便殿論之后屬耳於屏風後【屬之欲翻】須臾出糚具及三銀盆皇幼子三人於外曰人言宫中蓄積多四方貢獻隨以給賜所餘止此耳請鬻以贍軍宰相惶懼而退【嗚呼皇后囊金寶繫馬鞍之時能盡將内庫所積而行乎】 李紹榮自鄴都退保衛州奏李嗣源已叛與賊合嗣源遣使上章自理一日數輩嗣源長子從審為金槍指揮使【莊宗得魏因魏銀槍軍置帳前銀槍都後又置金槍軍】帝謂從審曰吾深知爾父忠厚爾往諭朕意勿使自疑從審至衛州紹榮囚欲殺之從審曰公等既不亮吾父【亮信也】吾亦不能至父所【今人多謂不欲行為不能】請復還宿衛【復扶又翻還從宣翻又如字】乃釋之帝憐從審賜名繼璟待之如子是後嗣源所奏皆為紹榮所遏不得通嗣源由是疑懼石敬瑭曰夫事成於果决而敗於猶豫安有上將與叛卒入賊城而它日得保無恙乎【將即亮翻】大梁天下之要會也【大梁控引河汴南通淮泗北接滑魏舟車之所湊集且梁舊都也故云然】願假三百騎先往取之若幸而得之公宜引大軍亟進【亟紀力翻急也】如此始可自全【據大梁則逼洛陽嗣源可以自全莊宗將何以自全乎石敬塘惡察察言故云爾】突騎指揮使康義誠曰主上無道軍民怨怒公從衆則生守節則死【康義誠胡人獷直觀此言可見也為義誠由此為明宗所親任張本】嗣源乃令安重誨移檄會兵義誠代北胡人也時齊州防禦使李紹䖍【即王晏球】泰寧節度使李紹欽【即段凝】貝州刺史李紹英【即房知温】屯瓦橋【以備契丹】北京右廂馬軍都指揮使安審通屯奉化軍【五代會要後唐天成三年三月升奉化軍為泰州以清苑縣為理所新唐書地理志清苑縣屬莫州宋保州治清苑蓋又改泰州為保州也】嗣源皆遣使召之紹英瑕丘人本姓房名知温審通金全之姪也【安金全有却梁兵全晉陽之功】嗣源家在真定【嗣源鎮真定入朝于洛其家留真定】虞候將王建立先殺其監軍由是獲全【為嗣源以王建立鎮真定張本將即亮翻】建立遼州人也李從珂自横水將所部兵由盂縣趣鎮州【李從珂謫戍横水見上卷同光三年盂春秋晉之盂邑漢為縣中廢隋開皇十六年置原仇縣大業初改曰盂唐屬太原府九域志盂縣東北至鎮州一百里】與王建立軍合倍道從嗣源嗣源以李紹榮在衛州謀自白臯濟河分三百騎使石敬瑭將之前驅李從珂為殿【殿丁練翻】於是軍勢大盛嗣源從子從璋自鎮州引兵而南過邢州邢人奉為留後【河北蓋悉從嗣源矣從子之從才用翻】 癸酉詔懷遠指揮使白從暉將騎兵扼河陽橋【恐李嗣源自懷孟犯洛也】帝乃出金帛給賜諸軍樞密宣徽使及供奉内使景進等皆獻金帛以助給賜【事已至此帝及嬖倖始知財物之不可守】軍士負物而詬曰吾妻子已殍死得此何為【詬古翻又許翻殍被表翻】甲戌李紹榮自衛州至洛陽帝如鷂店勞之【薛史作耀店】紹榮曰鄴都亂兵已遣其黨翟建曰據博州欲濟河襲鄆汴【李紹榮所言指趙在禮所遣兵也殊不知李嗣源已定入汴之計矣勞力到翻翟萇伯翻】願陛下幸關東招撫之帝從之【關東謂汜水關以東】 景進等言於帝曰魏王未至康延孝初平西南猶未安王衍族黨不少聞軍駕東征恐其為變不若除之【少詩照翻】帝乃遣中使向延嗣【向式亮翻】齎敕往誅之敕曰王衍一行並從殺戮已印畫【印者用中書印畫者畫可敕又用御寶】樞密使張居翰覆視就殿柱揩去行字改為家字【揩口皆翻摩也去羌呂翻】由是蜀百官及衍僕役獲免者千餘人延嗣至長安盡殺衍宗族於秦川驛衍母徐氏且死呼曰【呼火故翻】吾兒以一國迎降不免族誅【降戶江翻】信義俱棄吾知汝行亦受禍矣 乙亥帝發洛陽丁丑次汜水戊寅遣李紹榮將騎兵循河而東【將即亮翻】李嗣源親黨從帝者多亡去或勸李繼璟宜早自脱繼璟終無行意帝屢遣繼璟詣嗣源繼璟固辭願死於帝前以明赤誠【赤誠猶言赤心誠者心之實言赤誠者謂赤心之實】帝聞嗣源在黎陽強遣繼璟渡河召之【強其兩翻下強出同此時召嗣源嗣源必不敢前】道遇李紹榮紹榮殺之【李繼璟以死事君以明父之心迹得其死矣】 吳越王鏐有疾如衣錦軍命鎮海鎮東節度使留後傳瓘監國【衣于既翻監古銜翻】吳徐温遣使來問疾左右勸鏐勿見鏐曰温隂狡此名問疾實使之覘我也【覘丑亷翻又丑艶翻】強出見之温果聚兵欲襲吳越聞鏐疾瘳而止【史言錢徐之智力足以相制而不足以相勝】鏐尋還錢塘【按九域志自臨安東還錢塘一百二十里】 吳以左僕射同平章事徐知誥為侍中右僕射嚴可求兼門下侍郎同平章事 庚辰帝發汜水【發汜水而東也】辛巳李嗣源至白臯遇山東上供絹數船取以賞軍【此蓋青兖上供沂河而上者也】安重誨從者爭舟行營馬步使陶玘斬以徇【從才用翻玘墟里翻】由是軍中肅然玘許州人也嗣源濟河至滑州遣人招苻習習與嗣源會於胙城【舊唐書地理志胙城漢南燕縣】安審通亦引兵來會知汴州孔循遣使奉表西迎帝亦遣使北輸密欵於嗣源曰先至者得之先是帝遣騎將滿城西方鄴守汴州【先是悉薦翻】石敬瑭使禆將李瓊以勁兵突入封丘門敬瑭踵其後自西門入遂據其城西方鄴請降敬瑭使趣嗣源壬午嗣源入大梁【趣讀曰促九域志胙城縣南至大梁一百二十里】是日帝至滎澤東【九域志滎澤縣西北距汜水四十五里】命龍驤指揮使姚彦温將三千騎為前軍曰汝曹汴人也【龍驤軍梁之舊兵本皆汴人】吾入汝境不欲使它軍前驅恐擾汝室家厚賜而遣之彦温即以其衆叛歸嗣源謂嗣源曰京師危迫主上為元行欽所惑事勢已離不可復事矣【元行欽賜姓名李紹榮復扶又翻】嗣源曰汝自不忠何言之悖也【悖蒲妹翻】即奪其兵指揮使潘環守王村寨有芻粟數萬帝遣騎視之環亦奔大梁帝至萬勝鎮【萬勝鎮在中牟縣東距大梁不過數十里耳】聞嗣源已據大梁諸軍離叛神色沮喪【沮在呂翻喪息浪翻】登高歎曰吾不濟矣即命旋師帝之出關也扈從兵二萬五千【從才用翻下從官同】及還已失萬餘人乃留秦州都指揮使張唐以步騎三千守關癸未帝還過甖子谷【劉昫曰甖子谷在成臯又云在汜水縣西汜水縣古之成臯縣】道狹每遇衛士執兵仗者輒以善言撫之曰適報魏王又進西川金銀五十萬【適報猶言近方得報也】到京當盡給爾曹對曰陛下賜已晩矣人亦不感聖恩帝流涕而已又索袍帶賜從官内庫使張容哥稱頒給已盡【索山客翻内庫使亦莊宗所置内諸司使之一】衛士叱容哥曰致吾君失社稷皆此閹豎輩也抽刀逐之或救之獲免容哥謂同類曰皇后吝財致此【吝財事見上】今乃歸咎於吾輩事若不測吾輩萬段吾不忍待也因赴河死【衛士言致禍之源出於宦官不特指張容哥一人容哥遂先赴河而死者蓋以身為内庫使内庫積而不發出納之吝諸軍以為罪禍必先及故遽引决耳】甲申帝至石橋西【石橋在洛城東】置酒悲涕謂李紹榮等諸將曰卿輩事吾以來急難富貴靡不同之【難乃旦翻】今致吾至此皆無一策以相救乎諸將百餘人皆截髪置地誓以死報因相與號泣【號戶刀翻】是日晩入洛城李嗣源命石敬瑭將前軍趣汜水收撫散兵嗣源繼之【李嗣源在河北時奏章為元行欽所壅遏猶可言也渡河據大梁莊宗嘗至萬勝鎮君臣相望數十里間耳既無一奏陳情又無一騎迎莊宗既還但以兵踵之而西此意何在哉】李紹䖍李紹英引兵來會【李紹䖍李紹英皆自瓦橋引兵踵嗣源之後而來會於大梁】丙戌宰相樞密使共奏魏王西軍將至車駕宜且控扼汜水收撫散兵以俟之帝從之自出上東門閲騎兵戒以詰旦東行<br />
<br />
  資治通鑑卷二百七十四  <br>
   </div> 

<script src="/search/ajaxskft.js"> </script>
 <div class="clear"></div>
<br>
<br>
 <!-- a.d-->

 <!--
<div class="info_share">
</div> 
-->
 <!--info_share--></div>   <!-- end info_content-->
  </div> <!-- end l-->

<div class="r">   <!--r-->



<div class="sidebar"  style="margin-bottom:2px;">

 
<div class="sidebar_title">工具类大全</div>
<div class="sidebar_info">
<strong><a href="http://www.guoxuedashi.com/lsditu/" target="_blank">历史地图</a></strong>  
<a href="http://www.880114.com/" target="_blank">英语宝典</a>  
<a href="http://www.guoxuedashi.com/13jing/" target="_blank">十三经检索</a> 
<br><strong><a href="http://www.guoxuedashi.com/gjtsjc/" target="_blank">古今图书集成</a></strong> 
<a href="http://www.guoxuedashi.com/duilian/" target="_blank">对联大全</a> <strong><a href="http://www.guoxuedashi.com/xiangxingzi/" target="_blank">象形文字典</a></strong> 

<br><a href="http://www.guoxuedashi.com/zixing/yanbian/">字形演变</a>  <strong><a href="http://www.guoxuemi.com/hafo/" target="_blank">哈佛燕京中文善本特藏</a></strong>
<br><strong><a href="http://www.guoxuedashi.com/csfz/" target="_blank">丛书&方志检索器</a></strong> <a href="http://www.guoxuedashi.com/yqjyy/" target="_blank">一切经音义</a>  

<br><strong><a href="http://www.guoxuedashi.com/jiapu/" target="_blank">家谱族谱查询</a></strong>  <strong><a href="http://shufa.guoxuedashi.com/sfzitie/" target="_blank">书法字帖欣赏</a></strong> 
<br>

</div>
</div>


<div class="sidebar" style="margin-bottom:0px;">

<font style="font-size:22px;line-height:32px">QQ交流群9:489193090</font>


<div class="sidebar_title">手机APP 扫描或点击</div>
<div class="sidebar_info">
<table>
<tr>
	<td width=160><a href="http://m.guoxuedashi.com/app/" target="_blank"><img src="/img/gxds-sj.png" width="140"  border="0" alt="国学大师手机版"></a></td>
	<td>
<a href="http://www.guoxuedashi.com/download/" target="_blank">app软件下载专区</a><br>
<a href="http://www.guoxuedashi.com/download/gxds.php" target="_blank">《国学大师》下载</a><br>
<a href="http://www.guoxuedashi.com/download/kxzd.php" target="_blank">《汉字宝典》下载</a><br>
<a href="http://www.guoxuedashi.com/download/scqbd.php" target="_blank">《诗词曲宝典》下载</a><br>
<a href="http://www.guoxuedashi.com/SiKuQuanShu/skqs.php" target="_blank">《四库全书》下载</a><br>
</td>
</tr>
</table>

</div>
</div>


<div class="sidebar2">
<center>


</center>
</div>

<div class="sidebar"  style="margin-bottom:2px;">
<div class="sidebar_title">网站使用教程</div>
<div class="sidebar_info">
<a href="http://www.guoxuedashi.com/help/gjsearch.php" target="_blank">如何在国学大师网下载古籍?</a><br>
<a href="http://www.guoxuedashi.com/zidian/bujian/bjjc.php" target="_blank">如何使用部件查字法快速查字?</a><br>
<a href="http://www.guoxuedashi.com/search/sjc.php" target="_blank">如何在指定的书籍中全文检索?</a><br>
<a href="http://www.guoxuedashi.com/search/skjc.php" target="_blank">如何找到一句话在《四库全书》哪一页?</a><br>
</div>
</div>


<div class="sidebar">
<div class="sidebar_title">热门书籍</div>
<div class="sidebar_info">
<a href="/so.php?sokey=%E8%B5%84%E6%B2%BB%E9%80%9A%E9%89%B4&kt=1">资治通鉴</a> <a href="/24shi/"><strong>二十四史</strong></a>&nbsp; <a href="/a2694/">野史</a>&nbsp; <a href="/SiKuQuanShu/"><strong>四库全书</strong></a>&nbsp;<a href="http://www.guoxuedashi.com/SiKuQuanShu/fanti/">繁体</a>
<br><a href="/so.php?sokey=%E7%BA%A2%E6%A5%BC%E6%A2%A6&kt=1">红楼梦</a> <a href="/a/1858x/">三国演义</a> <a href="/a/1038k/">水浒传</a> <a href="/a/1046t/">西游记</a> <a href="/a/1914o/">封神演义</a>
<br>
<a href="http://www.guoxuedashi.com/so.php?sokeygx=%E4%B8%87%E6%9C%89%E6%96%87%E5%BA%93&submit=&kt=1">万有文库</a> <a href="/a/780t/">古文观止</a> <a href="/a/1024l/">文心雕龙</a> <a href="/a/1704n/">全唐诗</a> <a href="/a/1705h/">全宋词</a>
<br><a href="http://www.guoxuedashi.com/so.php?sokeygx=%E7%99%BE%E8%A1%B2%E6%9C%AC%E4%BA%8C%E5%8D%81%E5%9B%9B%E5%8F%B2&submit=&kt=1"><strong>百衲本二十四史</strong></a>  <a href="http://www.guoxuedashi.com/so.php?sokeygx=%E5%8F%A4%E4%BB%8A%E5%9B%BE%E4%B9%A6%E9%9B%86%E6%88%90&submit=&kt=1"><strong>古今图书集成</strong></a>
<br>

<a href="http://www.guoxuedashi.com/so.php?sokeygx=%E4%B8%9B%E4%B9%A6%E9%9B%86%E6%88%90&submit=&kt=1">丛书集成</a> 
<a href="http://www.guoxuedashi.com/so.php?sokeygx=%E5%9B%9B%E9%83%A8%E4%B8%9B%E5%88%8A&submit=&kt=1"><strong>四部丛刊</strong></a>  
<a href="http://www.guoxuedashi.com/so.php?sokeygx=%E8%AF%B4%E6%96%87%E8%A7%A3%E5%AD%97&submit=&kt=1">說文解字</a> <a href="http://www.guoxuedashi.com/so.php?sokeygx=%E5%85%A8%E4%B8%8A%E5%8F%A4&submit=&kt=1">三国六朝文</a>
<br><a href="http://www.guoxuedashi.com/so.php?sokeytm=%E6%97%A5%E6%9C%AC%E5%86%85%E9%98%81%E6%96%87%E5%BA%93&submit=&kt=1"><strong>日本内阁文库</strong></a> <a href="http://www.guoxuedashi.com/so.php?sokeytm=%E5%9B%BD%E5%9B%BE%E6%96%B9%E5%BF%97%E5%90%88%E9%9B%86&ka=100&submit=">国图方志合集</a> <a href="http://www.guoxuedashi.com/so.php?sokeytm=%E5%90%84%E5%9C%B0%E6%96%B9%E5%BF%97&submit=&kt=1"><strong>各地方志</strong></a>

</div>
</div>


<div class="sidebar2">
<center>

</center>
</div>
<div class="sidebar greenbar">
<div class="sidebar_title green">四库全书</div>
<div class="sidebar_info">

《四库全书》是中国古代最大的丛书,编撰于乾隆年间,由纪昀等360多位高官、学者编撰,3800多人抄写,费时十三年编成。丛书分经、史、子、集四部,故名四库。共有3500多种书,7.9万卷,3.6万册,约8亿字,基本上囊括了古代所有图书,故称“全书”。<a href="http://www.guoxuedashi.com/SiKuQuanShu/">详细>>
</a>

</div> 
</div>

</div>  <!--end r-->

</div>
<!-- 内容区END --> 

<!-- 页脚开始 -->
<div class="shh">

</div>

<div class="w1180" style="margin-top:8px;">
<center><script src="http://www.guoxuedashi.com/img/plus.php?id=3"></script></center>
</div>
<div class="w1180 foot">
<a href="/b/thanks.php">特别致谢</a> | <a href="javascript:window.external.AddFavorite(document.location.href,document.title);">收藏本站</a> | <a href="#">欢迎投稿</a> | <a href="http://www.guoxuedashi.com/forum/">意见建议</a> | <a href="http://www.guoxuemi.com/">国学迷</a> | <a href="http://www.shuowen.net/">说文网</a><script language="javascript" type="text/javascript" src="https://js.users.51.la/17753172.js"></script><br />
  Copyright &copy; 国学大师 古典图书集成 All Rights Reserved.<br>
  
  <span style="font-size:14px">免责声明:本站非营利性站点,以方便网友为主,仅供学习研究。<br>内容由热心网友提供和网上收集,不保留版权。若侵犯了您的权益,来信即刪。scp168@qq.com</span>
  <br />
ICP证:<a href="http://www.beian.miit.gov.cn/" target="_blank">鲁ICP备19060063号</a></div>
<!-- 页脚END --> 
<script src="http://www.guoxuedashi.com/img/plus.php?id=22"></script>
<script src="http://www.guoxuedashi.com/img/tongji.js"></script>

</body>
</html>
