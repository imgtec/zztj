資治通鑑卷二百八   宋 司馬光 撰

胡三省 音注

唐紀二十四|{
	起旃蒙大荒落二月盡彊圉協洽凡二年有奇}


中宗大和大聖大昭孝皇帝中

神龍元年二月辛亥帝帥百官詣上陽宫問太后起居|{
	帥讀曰率 考異曰實録唐歷皆云乙亥誤也當是辛亥}
自是每十日一往 甲寅復國號曰唐|{
	天授元年武后更國號曰周今復舊}
郊廟社稷陵寢百官旗幟服色文字皆如永淳以前故事|{
	幟昌志翻}
復以神都為東都|{
	光宅元年改東都曰神都復扶又翻又如字}
北都為并州|{
	天授元年以并州為北都并卑經翻}
老君為玄元皇帝|{
	高宗乾封元年上老子尊號曰玄元皇帝武后革命改曰老君}
乙卯鳳閣侍郎同平章事韋承慶貶高要尉|{
	高要縣帶端州至京師五千七百五十里東都五千一百五十里}
正諫大夫同平章事房融除名流高州|{
	舊志高州京師南六千二百六十二里至東都五千五百二十里}
司禮卿崔神慶流欽州|{
	舊志欽州至京師五千二百五十一里}
楊再思為戶部尚書同中書門下三品西京留守|{
	尚辰羊翻守手又翻}
太后之遷上陽宫也|{
	見上卷是年正月}
太僕卿同中書門下三品姚元之獨嗚咽流涕桓彦範張柬之謂曰今日豈公涕泣時邪恐公禍由此始元之曰元之事則天皇帝久乍此辭違悲不能忍且元之前日從公誅姦逆人臣之義也今日别舊君亦人臣之義也雖獲罪實所甘心是日出為亳州刺史|{
	此姚元之所以為多智也舊志亳州至京師一千七百里至東都八百九十八里}
甲子立妃韋氏為皇后赦天下追贈后父玄貞為上洛王母崔氏為妃左拾遺賈虚已上疏以為異姓不王古今通制|{
	上時掌翻疏所去翻}
今中興之始萬姓喁喁|{
	喁魚容翻}
以觀陛下之政而先王后族|{
	王于况翻}
非所以廣德美於天下也且先朝贈后父太原王|{
	高宗贈武后父士彠太原郡王朝直遥翻}
殷鍳不遠須防其漸若以恩制已行宜令皇后固讓則益增謙冲之德矣不聽初韋后生邵王重潤長寧安樂二公主|{
	重直龍翻樂音洛}
上之遷房陵也|{
	遷房陵見二百三卷光宅元年垂拱元年}
安樂公主生於道中上特愛之上在房陵與后同幽閉備嘗艱危情愛甚篤上每聞敕使至輒惶恐欲自殺|{
	使疏吏翻}
后止之曰禍福無常寧失一死何遽如是上嘗與后私誓曰異時幸復見天日|{
	復扶又翻又如字}
當惟卿所欲不相禁制及再為皇后遂干預朝政如武后在高宗之世桓彦範上表以為易稱無攸遂在中饋貞吉|{
	易家人卦六二爻辭王弼注曰六二居内處中履得其位以隂應陽盡婦人之正義無所必遂職乎中饋巽順而已是以貞吉也朝直遥翻上時掌翻}
書稱牝鷄之辰惟家之索|{
	書牧誓之辭辰作晨孔安國曰索盡也喻婦人知外事雌代雄鳴則家盡婦奪夫政則國亡索西各翻}
伏見陛下每臨朝|{
	朝直遥翻}
皇后必施帷幔坐殿上|{
	幔莫半翻}
預聞政事臣竊觀自古帝王未有與婦人共政而不破國亡身者也且以隂乘陽違天也以婦陵夫違人也伏願陛下覽古今之戒以社稷蒼生為念令皇后專居中宫治隂敎|{
	記曰天子聽男教后聽女順天子理陽道后治隂德天子聽外治后聽内職教順成俗外内和順國家理治此之謂盛德治直之翻}
勿出外朝干國政|{
	朝直遥翻}
先是胡僧慧範以妖妄遊權貴之門與張易之兄弟善韋后亦重之及易之誅復稱慧範預其謀以功加銀青光禄大夫賜爵上庸縣公出入宫掖上數微行幸其舍彦範復表言慧範執左道以亂政請誅之|{
	先悉薦翻復扶又翻數所角翻下又數同記王制執左道以亂政者殺}
上皆不聽 初武后誅唐宗室有才德者先死惟吴王恪之子鬰林侯千里躁無才|{
	躁則到翻}
又數獻符瑞故獨得免上即位立為成王拜左金吾大將軍武后所誅唐諸王妃主駙馬等皆無人葬埋子孫或流竄嶺表或拘囚歷年或逃匿民間為人傭保至是制州縣求訪其柩以禮改葬|{
	柩音舊}
追復官爵召其子孫使之承襲無子孫者為擇後置之既而宗室子孫相繼而至皆召見|{
	為于偽翻見賢遍翻}
涕泣舞蹈各以親疎襲爵拜官有差二張之誅也洛州長史薛季昶謂張柬之敬暉曰二凶雖除產禄猶在|{
	產禄謂武三思等}
去草不去根終當復生|{
	去羌呂翻復扶又翻下可復同}
二人曰大事已定彼猶几上肉耳夫何能為所誅已多不可復益也季昶嘆曰吾不知死所矣朝邑尉武強劉幽求|{
	武強縣漢河間之武隊也晋更名屬武邑郡唐屬冀州朝直遥翻}
亦謂桓彦範敬暉曰武三思尚存公輩終無葬地若不早圖噬臍無及不從|{
	左傳鄧三甥勸鄧侯殺楚子曰若不早圖後君噬臍 考異曰御史臺記曰張東之勒兵於景運門將收諸武誅之彦範以事既竟不欲廣誅遽解其兵柬之固爭不果狄梁公傳曰袁謂張公曰昔有遺言使先收梁王武三思豈可捨諸張公曰但大事畢功此皆几上之物豈有逃乎按舊唐書薛季昶傳敬暉傳唐統紀唐歷狄梁公傳皆云張柬之敬暉不欲誅武三思惟御史臺記以為柬之固爭而彦範不從新唐書彦範傳亦云薛季昶勸誅三思會日暮事遽彦範不欲廣殺因曰三思几上肉耳留為天子藉手季昶嘆曰吾無死所矣按東之時為宰相首建此謀當是與桓敬等皆不可不應獨由彦範也}
上女安樂公主適三思子崇訓上官婉兒儀之女孫也儀死|{
	上官儀死見二百一卷高宗麟德元年}
沒入掖庭辯慧善屬文|{
	屬之欲翻}
明習吏事則天愛之自聖歷以後百司表奏多令參决及上即位又使專掌制命益委任之拜為媫妤|{
	媫妤音接予}
用事於中三思通焉故黨於武氏又薦三思於韋后引入禁中上遂與三思圖議政事張柬之等皆受制於三思矣 |{
	考異曰舊傳云誅易之明日三思因韋后之助潜入宫中内行相事反易國政居數日五王皆失柄受制於三思矣事似傷速今微加刪改}
上使韋后與三思雙陸|{
	雙陸者投瓊以行十二棊各行六棊故謂之雙陸}
而自居旁為之點籌三思遂與后通由是武氏之埶復振張柬之等數勸上誅諸武上不聽|{
	為于偽翻復扶又翻又如字數所角翻下同}
柬之等曰革命之際宗室諸李誅夷略盡今賴天地之靈陛下返正而武氏濫官僭爵安堵如故豈遠近所望邪願頗抑損其禄位以慰天下又不聽柬之等或撫牀嘆憤或彈指出血曰主上昔為英王時稱勇烈吾所以不誅諸武者欲使上自誅之以張天子之威耳|{
	張知兩翻}
今反如此事埶已去知復奈何|{
	復扶又翻}
上數微服幸武三思第監察御史清河崔皎密疏諫曰|{
	清河漢縣後漢和帝改曰甘陵晉復舊名唐帶貝州}
國命初復則天皇帝在西宫|{
	上陽宫在洛陽宫城之西故曰西宫}
人心猶有附會周之舊臣列居朝廷陛下柰何輕有外遊不察豫且之禍|{
	白龍魚㐲見困豫且且子余翻}
上洩之三思之黨切齒丙寅以太子賓客武三思為司空同中書門下三品 左散騎常侍譙王重福上之庶子也|{
	散悉亶翻騎奇寄翻重直龍翻下同}
其妃張易之之甥韋后惡之|{
	惡烏路翻}
譛於上曰重潤之死重福為之也|{
	重潤死見上卷長安元年}
由是貶濮州員外刺史又改均州刺史|{
	舊志濮州京師東北一千五百七十里至東都七百二十五里均州京師東南九百三十里至東都九百一十七里}
常令州司防守之 丁卯以右散騎常侍安定王武攸暨為司徒定王 辛未相王固讓太尉及知政事許之又立為皇太弟相王固辭而止|{
	相息亮翻}
甲戌以國子祭酒始平祝欽明同中書門下三品黄門侍郎知侍中事韋安石為刑部尚書罷知政事丁丑武三思武攸暨固辭新官爵及政事許之並加

開府儀同三司 立皇子義興王重俊為衛王北海王重茂為温王仍以重俊為洛州牧|{
	重直龍翻}
三月甲申制文明已來破家子孫皆復舊資廕唯徐敬業裴炎不在免限|{
	韋武之意也}
丁亥制酷吏周興來俊臣等已死者追奪官爵存者皆流嶺南惡地|{
	按舊書此時酷吏之存者唐奉一李秦授曹仁哲}
己丑以袁恕己為中書令 以安車徵安平王武攸

緒於嵩山|{
	武攸緒隱嵩山見二百五卷萬歲通天元年}
既至除太子賓客固請還山許之 制梟氏蟒氏皆復舊姓|{
	梟蟒氏見二百卷高宗永徽六年梟上堯翻}
術士鄭普思尚衣奉御葉静能|{
	葉舊音攝後音木葉之葉吴志孫皓傳有都尉葉雄}
皆以妖妄為上所信重|{
	妖於喬翻}
夏四月墨敕以普思為祕書監静能為國子祭酒|{
	墨敕出於禁中不由中書門下}
桓彦範崔玄暐固執不可上曰已用之無容遽改彦範曰陛下初即位下制云政令皆依貞觀故事貞觀中魏徵虞世南顔師古為祕書監孔頴逹為國子祭酒豈普思静能之比乎庚戌左拾遺李邕上疏以為詩三百一言以蔽之曰思無邪|{
	引論語孔子之言上時掌翻疏所去翻}
若有神仙能令人不死則秦始皇漢武帝得之矣佛能為人福利則梁武帝得之矣堯舜所以為帝王首者亦修人事而已尊寵此屬何補於國上皆不聽 上即位之日驛召魏元忠於高要|{
	魏元忠貶見上卷長安三年}
丁卯至都拜衛尉卿同平章事 甲戌以魏元忠為兵部尚書韋安石為吏部尚書李懷遠為右散騎常侍唐休璟為輔國大將軍|{
	璟俱永翻}
崔玄暐檢校益府長史楊再思檢校楊府長史祝欽明為刑部尚書並同中書門下三品元忠等皆以東宫舊僚褒之也|{
	史言中宗命相非以德授}
乙亥以張柬之為中書令 戊寅追贈故邵王重潤為懿德太子 五月壬午遷周廟七主於西京崇尊廟|{
	周立七廟見二百四卷武后天授元年崇尊廟見天授二年}
制武氏三代諱奏事者皆不得犯 乙酉立太廟社稷於東都 以張柬之等及武攸暨武三思鄭普思等十六人皆為立功之人賜以鐵劵自非反逆各恕十死 癸巳敬暉等帥百官上表以為五運迭興|{
	五運謂五德之運帥讀曰率}
事不兩大天授革命之際宗室誅竄殆盡豈得與諸武並封今天命惟新而諸武封建如舊並居京師開闢以來未有斯理願陛下為社稷計順遐邇心降其王爵以安内外上不許敬暉等畏武三思之讒以考功員外郎崔湜為耳目伺其動静|{
	湜常職翻伺相吏翻}
湜見上親三思而忌暉等乃悉以暉等謀告三思反為三思用三思引為中書舍人湜仁師之孫也|{
	崔仁師見一百九十二卷太宗貞觀元年}
先是殿中侍御史南皮鄭愔諂事二張|{
	南皮縣漢屬勃海郡唐武德初屬景州貞觀初屬滄州先悉薦翻愔於今翻}
二張敗貶宣州司士參軍坐贓亡入東都|{
	舊志宣州至東都二千五百一十里}
私謁武三思初見三思哭甚哀既而大笑三思素貴重甚恠之愔曰始見大王而哭哀大王將戮死而滅族也後乃大笑喜大王之得愔也大王雖得天子之意彼五人皆據將相之權|{
	五人謂張柬之敬暉桓彦範崔玄暐袁恕己也}
膽略過人廢太后如反掌大王自視埶位與太后孰重彼五人日夜切齒欲噬大王之肉非盡大王之族不足以快其志大王不去此五人危如朝露|{
	去羌呂翻朝露易晞}
而晏然尚自以為泰山之安此愔所以為大王寒心也|{
	為于偽翻下因為同}
三思大悦與之登樓問自安之策引為中書舍人與崔湜皆為三思謀主三思與韋后日夜譛暉等云恃功專權將不利於社稷上信之三思等因為上畫策不若封暉等為王罷其政事外不失尊寵功臣内實奪之權上以為然甲午以侍中齊公敬暉為平陽王桓彦範為扶陽王中書令漢陽公張柬之為漢陽王南陽公袁恕己為南陽王特進同中書門下三品博陵公崔玄暐為博陵王 |{
	考異曰統紀曰太后善自粉飾雖子孫在側不覺其衰老及在上陽宫不復櫛頮形容羸悴上入見大驚太后泣曰我自房陵迎汝來固以天下授汝矣而五賊貪功驚我至此上悲泣不自勝伏地拜謝死罪由是三思等得入其謀按中宗頑鄙不仁太后雖毀容涕泣未必能感動移其意其所以踈忌五王自用韋后三思之言耳今不取五王尊卑先後不定實録誅張易之時以張柬之為首賜鐵劵以崔玄暐為首封王及謫為司馬長流皆以敬暉為首舊傳及開元復官詔並以桓彦範為首按長安四年六月玄暐為鸞臺侍郎平章事十月張柬之自秋官侍郎同平章事十一月守鳳閣侍郎誅易之時唯此二人為相神龍元年正月袁恕己自司刑少卿為鳳閣侍郎同平章事庚戌張柬之為夏官尚書玄暐守内史敬暉桓彦範並為納言三月恕己守中書令四月東之為中書令敬暉為侍中五王遷轉先後如此疑實録但以誅易之時柬之首謀故以柬之為首暉與彦範同為侍中疑侍中在中書令上故削諸武表及罷政事皆以暉為首賜鐵券時玄暐已加特進暉等罷政方加特進而玄暐如舊疑特進雖散階而品秩最高故以玄暐為首彦範與暉同為侍中而彦範被禍最酷疑開元詔及史官特以為首未必以當時位次也天后中宗時侍中疑在中書令上}
罷知政事賜金帛鞍馬令朝朔望|{
	朝直遥翻}
仍賜彦範姓韋氏與皇后同籍尋又以玄暐檢校益州長史知都督事又改梁州刺史|{
	益州京師西南二千三百七十九里至東都三千一百一十六里梁州至京師一千二百二十三里東都二千七十八里}
三思令百官復修則天之政|{
	復扶又翻下温復翻}
不附武氏者斥之為五王所逐者復之大權盡歸三思矣五王之請削武氏諸王也求人為表衆莫肯為中書舍人岑羲為之語甚激切中書舍人偃師畢構次當讀表辭色明厲三思既得志羲改祕書少監出構為潤州刺史|{
	潤州京師東南二千八百二十一里至東都一千七百九十七里}
易州刺史趙履温桓彦範之妻兄也彦範之誅二張稱履温預其謀召為司農少卿履温以二婢遺彦範|{
	遺于季翻}
及彦範罷政事履温復奪其婢上嘉宋璟忠直屢遷黄門侍郎武三思嘗以事屬璟|{
	屬之欲翻}
璟正色拒之曰今太后既復子明辟王當以侯就第何得尚干朝政|{
	朝直遥翻}
獨不見產禄之事乎 以韋安石兼檢校中書令魏元忠兼檢校侍中又以李湛為右散騎常侍趙承恩為光禄卿楊元琰為衛尉卿先是元琰知三思浸用事請弃官為僧上不許敬暉聞之笑曰使我早知勸上許之髠去瓠頭豈不妙哉|{
	先悉薦翻去羌呂翻}
元琰多鬚類胡故暉戲之元琰曰功成名遂不退將危此乃由衷之請|{
	衷誠也由衷言出於誠心}
非徒然也暉知其意瞿然不悦|{
	瞿九遇翻瞿然驚視貌}
及暉等得罪元琰獨免上官婕妤勸韋后襲則天故事上表請天下士庶為

出母服喪三年|{
	上時掌翻為于偽翻所以感動帝心令其念武后也}
又請百姓年二十三為丁五十九免役|{
	唐制二十一為丁六十為老}
改易制度以收時望制皆許之 癸卯制降諸武梁王三思為德静王定王攸暨為樂夀王|{
	皆降封縣王也德静縣屬夏州樂夀縣屬深州}
河内王懿宗等十二王皆降為公以厭人心|{
	樂音洛厭於協翻}
甲辰以唐休璟為左僕射同中書門下三品如故|{
	璟俱永翻}
豆盧欽望為右僕射六月壬子以左驍衛大將軍裴思說充靈武軍大總管以備突厥|{
	驍堅堯翻說讀曰悦厥九勿翻}
癸亥命右僕射豆盧欽望有軍國重事中書門下可共平章先是僕射為二宰相|{
	先悉薦翻}
其後多兼中書門下之軄午前决朝政|{
	朝直遥翻}
午後决省事|{
	省事尚書省事也}
至是欽望專為僕射不敢預政事故有是命是後專拜僕射者不復為宰相矣|{
	復扶又翻}
又以韋安石為中書令魏元忠為侍中楊再思為檢校中書令 丁卯祔孝敬皇帝於太廟號義宗|{
	故太子弘諡孝敬皇帝帝兄也}
戊辰洛水溢流二千餘家 秋七月辛巳以太子賓客韋巨源同中書門下三品西京留守如故|{
	守式又翻}
特進漢陽王張柬之表請歸襄州養疾乙未以柬之為襄州刺史不知州事給全俸|{
	唐制特進正二品郡王從一品從品高給一品月俸八千食料一千八百雜用一千二百上州刺史從三品月俸五千一百雜用九百}
河南北十七州大水八月戊申以水災求直言右衛騎曹參軍西河宋務光上疏|{
	唐諸衛府有倉兵騎胄四曹参軍騎曹参軍掌西河雜畜簿帳牧養凡府馬承直以遠近分七番月一易之以敕出宫城者給馬西河縣屬汾州騎奇寄翻上時掌翻疏所去翻}
以為水隂類臣妾之象恐後庭有干外朝之政者|{
	朝直遥翻}
宜杜絶其萌今霖雨不止乃閉坊門以禳之至使里巷謂坊門為宰相言朝廷使之燮理隂陽也|{
	宋白曰唐制久雨則閉坊市北門以祈晴}
又太子國本宜早擇賢能而立之又外戚太盛如武三思等宜解其機要厚以禄賜又鄭普思葉静能以小技竊大位亦朝政之蠧也疏奏不省|{
	技渠綺翻朝直遥翻省悉景翻}
壬戌追立妃趙氏為恭皇后|{
	趙妃死見二百二卷高宗上元二年考異曰舊本紀云甲子今從實録}
孝敬皇帝妃裴氏為哀皇后 九月

壬午上祀昊天上帝皇地祗於明堂以高宗配 初上在房陵州司制約甚急刺史河東張知謇靈昌崔敬嗣|{
	河東舊蒲坂也屬河東郡隋廢郡改蒲坂為河東縣唐因之帶蒲州隋分酸棗縣置靈昌縣因河津以為名唐屬滑州謇九輦翻}
獨待遇以禮供給豐贍|{
	贍而艷翻}
上德之擢知謇自貝州刺史為左衛將軍賜爵范陽公敬嗣已卒求得其子汪嗜酒不堪釐職除五品散官|{
	唐六典隋煬帝置朝請大夫為正五品散官隋文帝置朝散大夫為正四品散官煬帝改從五品下}
改葬上洛王韋玄貞其儀皆如太原王故事|{
	武士彠封太原王}
癸巳太子賓客同中書門下三品韋巨源罷為禮部尚書以其從父安石為中書令故也|{
	從才用翻}
以左衛將軍上邽紀處納兼檢校太府卿處訥娶武三思之妻姊故也|{
	處昌呂翻}
冬十月命唐休璟留守京師|{
	守從又翻}
癸亥上幸龍門乙丑獵於新安而還|{
	還從宣翻又如字}
辛未以魏元忠為中書令楊再思為侍中 十一月戊寅羣臣上皇帝尊號曰應天皇帝皇后曰順天皇后|{
	上時掌翻}
壬午上與后謁謝太廟赦天下相王太平公主加實封皆滿萬戶|{
	相息亮翻}
己丑上御洛城南樓|{
	洛陽皇城之西南曰洛城門門内即洛城殿}
觀潑寒胡戲|{
	潑寒胡戲即乞寒胡戲本出於胡中西域康國十一月鼓舞乞寒以水交潑為樂武后末年始以季冬為之}
清源尉呂元泰上疏以為謀時寒若|{
	清源縣屬并州隋於古梗陽城置以水為名書洪範曰謀時寒若注云君能謀則時寒順之若順也上時掌翻疏所去翻}
何必祼身揮水鼓舞衢路以索之|{
	祼郎果翻索山客翻}
疏奏不納 壬寅則天崩於上陽宫年八十二遺制去帝號|{
	去羌呂翻}
稱則天大聖皇后王蕭二族及禇遂良韓瑗柳奭親屬皆赦之|{
	武后之立也王皇后蕭淑妃幽廢不得良死禇遂良韓瑗以諫死柳奭以王后親屬死其親屬皆流竄}
上居諒隂以魏元忠攝冢宰三日元忠素負忠直之望中外賴之武三思憚之矯太后遺制慰諭元忠賜實封百戶元忠捧制感咽涕泗見者曰事去矣|{
	知其不敢復論武氏事也}
十二月丁卯上始御同明殿見羣臣|{
	見賢遍翻六典東都皇宫南面三門中曰應天左曰與教右曰光政光政之北曰明福明福之西曰崇賢門其内曰集賢殿集賢之東曰億歲殿又東曰同明殿}
太后將合葬乾陵給事中嚴善思上疏以為乾陵玄宫以石為門鐵錮其縫|{
	縫扶用翻}
今啟其門必須鐫鑿神明之道體尚幽玄動衆加功恐多驚黷况合葬非古漢時諸陵皇后多不合葬魏晉已降始有合者望於乾陵之傍更擇吉地為陵若神道有知幽塗自當通會若其無知合之何益不從 是歲戶部奏天下戶六百一十五萬口三千七百一十四萬有畸

二年春正月戊戌以吏部尚書李嶠同中書門下三品中書侍郎于惟謙同平章事 閏月丙午制太平長寧安樂宜城新都定安金城公主並開府置官屬|{
	自長寧以下皆皇女也樂音洛}
武三思以敬暉桓彦範袁恕己尚在京師忌之乙卯出為滑洺豫三州刺史|{
	舊志滑州去京師一千四百四十里東都五百三十里洛州京師東北一千五百八十五里至東都八百五十七里豫州去京師一千五百四十里至東都六百七十里 考異曰實録新記新舊列傳皆不見崔玄暐及暉等出為刺史年月惟舊紀及統紀唐歷有此三人盖玄暐先已出矣但不知何時然暉等貶為司馬時乃刺朗亳郢均四州盖於後又經遷徙矣唐歷統紀以為在王同皎誅後今從之}
賜䦩鄉僧萬回號法雲公|{
	萬回姓張氏初毋祈於觀音像而妊回回生而愚八九歲乃能語雖父母亦以豚犬畜之其兄戍役於安西音問隔絶父母遣其問訊一日朝齎所備而往夕返其家父母異之弘農去安西萬里以其萬里而回因號萬回武后賜之錦袍金帶}
甲戌以突騎施酋長烏質勒為懷德郡王|{
	騎奇寄翻酋長上慈由翻下知兩翻}
二月乙未以刑部尚書韋巨源同中書門下三品仍與皇后叙宗族 丙申僧慧範等九人並加五品階賜爵郡縣公道士史崇恩等加五品階除國子祭酒同正|{
	唐會要曰永徽五年蔣孝璋除尚藥奉御員外特置仍同正員員外官自此始也}
葉静能加金紫光禄大夫 選左右臺及内外五品以上官二十人為十道廵察使|{
	使疏吏翻}
委之察吏撫人薦賢直獄二年一代考其功罪而進退之易州刺史魏人姜師度禮部員外郎馬懷素殿中侍御史臨漳源乾曜監察御史靈昌盧懷慎衛尉少卿滏陽李傑皆預焉|{
	魏縣漢屬魏郡時屬魏州晉愍帝諱鄴改鄴為臨漳時鄴城已淪覆矣後趙復為鄴縣東魏分鄴内黄斤丘肥鄉置臨漳縣屬魏郡周隋唐屬相州滏陽漢武安縣地後周置滏陽縣屬相州滏音釡}
三月甲辰中書令韋安石罷為戶部尚書戶部尚書蘇瓌為侍中西京留守|{
	守手又翻}
瓌頲之父也|{
	頲它鼎翻}
唐休璟致仕 初少府監丞弘農宋之問及弟兖州司倉之遜|{
	弘農縣帶虢州治弘農川唐制倉曹司倉參軍事掌租調公廨庖厨倉庫市肆}
皆坐附會張易之貶嶺南逃歸東都匿於友人光禄卿駙馬都尉王同皎家同皎疾武三思及韋后所為每與所親言之輒切齒之遜於簾下聞之密遣其子曇及甥校書郎李悛告三思欲以自贖三思使曇悛及撫州司倉冉祖雍|{
	撫州漢南昌南城縣地吴孫亮分置臨州郡隋平陳置撫州曇徒含翻悛丑緣翻}
上書告同皎與洛陽人張仲之祖延慶武當丞夀春周憬等|{
	夀春縣漢屬淮南郡晉避鄭太后諱改曰夀陽隋復曰夀春縣帶夀州璟俱永翻}
潜結壯士謀殺三思因勒兵詣闕廢皇后上命御史大夫李承嘉監察御史姚紹之按其事又命楊再思李嶠韋巨源參驗仲之言三思罪狀事連宫壼|{
	壼苦本翻}
再思巨源陽寐不聽嶠與紹之命反接送獄仲之還顧言不已紹之命撾之折其臂仲之大呼曰|{
	撾則瓜翻折而設翻呼火故翻}
吾已負汝死當訟汝於天庚戌同皎等皆坐斬 |{
	考異曰御史臺記曰同皎與張仲之等謀誅三思為宋談所發御史大夫李承嘉御史姚紹之按問事連椒宫内敕宰相問對諸宰佯假寐無所聞獨嶠與承嘉竊議同皎仲之等遇族又曰張仲之等謀誅武三思宋之遜子曇知其謀將發之未果會冉祖雍李恮於路白之雍恮以聞又曰張仲之宋之遜祖延慶謀於衣䄂中發銅弩射三思伺其便未果之遜子曇密發之敕李承嘉與紹之按於新開門内初紹之將直其事未定敕宰相對問諸相畏三思但僶俛佯不聞仲之延慶言諸相中有附會三思者屢與承嘉耳言復說誘紹之事乃變遂密置人力十餘命引仲之對問至則塞口反接送繫所紹之還謂仲之曰張三事不諧矣仲之固言三思反狀紹之命撾之而臂折仲之大呼天者六七謂紹之反賊我臂且折矣已輸你當訴爾於天曹乃自誣反而遇族朝野僉載曰初之遜諂附張易之兄弟出為兖州司倉遂亡歸王同皎匿之於小房皎慷慨之士也忿逆韋與武三思亂國與一二所親論之每至切齒之遜於簾下竊聽之遣姪曇上書告之以希逆韋之旨武三思等果大怒奏誅同皎之黨實録同皎與周憬等潜謀誅三思乃招集將士期以則天靈駕發引因刼殺三思李悛等知而告三思三思因言同皎等謀反竟坐斬唐歷統紀亦與實録略同而云仲之誤泄於友人宋之問之問偽應之祖雍之遜亦預其謀既而背之李悛之問甥也命以告三思因言同皎謀反舊傳云之問左遷瀧州參軍未幾逃還匿於張仲之家仲之與同皎等謀殺武三思之問令兄子發其事以自贖及同皎等獲罪起之問為鴻臚主簿按三思得幸於中宗韋后權傾天下同皎等若擅殺之豈得晏然無事苟無脅君之志豈得輕為此謀又云䄂中發銅弩此則殆同兒戲盖忿疾三思或與仲之憬等有欲殺之言而之遜等以告三思三思因教曇等誣告同皎云謀於靈駕發引日刼殺三思因廢皇后謀反耳今從僉載}
籍沒其家周憬亡入比干廟中大言曰比干古之忠臣知吾此心三思與皇后淫亂傾危國家行當梟首都市恨不及見耳遂自剄|{
	梟堅堯翻剄古頂翻}
之問之遜曇悛祖雍並除京官|{
	京官謂在京職官也亦謂之京司官}
加朝散大夫|{
	朝直遥翻散悉亶翻}
武三思與韋后日夜譛敬暉等不已復左遷暉為朗州刺史崔玄暐為均州刺史桓彦範為亳州刺史袁恕己為郢州刺史|{
	郢州漢安陵縣地江左置竟陵郡西魏置温州後周置郢州復扶又翻}
與暉等同立功者皆以為黨與坐貶 大置員外官自京司及諸州凡二千餘人宦官超遷七品以上員外官者又將千人魏元忠自端州還為相|{
	魏元忠先貶高要尉高要縣帶端州相息亮翻}
不復彊諫|{
	復扶又翻}
惟與時俯仰中外失望酸棗尉袁楚客|{
	酸棗縣漢晉屬陳留郡後齊廢隋開皇六年復置屬鄭州唐屬滑州}
致書元忠以為主上新服厥命惟新厥德|{
	引書咸有一德之文}
當進君子退小人以興大化豈可安其榮寵循默而已今不早建太子擇師傅而輔之一失也公主開府置僚屬二失也崇長緇衣使遊走權門借埶納賂三失也俳優小人盜竊品秩四失也有司選進賢才皆以貨取埶求五失也寵進宦者殆滿千人為長亂之階六失也|{
	長知兩翻}
王公貴戚賞賜無度競為侈靡七失也廣置員外官傷財害民八失也先朝宫女得自便居外出入無禁交通請謁九失也|{
	九失指言上宫婕妤賀婁尚宫之類朝直遥翻}
左道之人熒惑主聽盜竊禄位十失也|{
	十失指言葉静能鄭普思之類}
凡此十失君侯不正誰與正之哉元忠得書愧謝而已 夏四月改贈后父韋玄貞為酆王后四弟皆贈郡王|{
	四弟洵浩洞泚也}
己丑左散騎常侍同中書門下三品李懷遠致仕 處士韋月將上書告武三思潜通宫掖必為逆亂|{
	處昌呂翻}
上大怒命斬之黄門侍郎宋璟奏請推按|{
	璟俱永翻}
上益怒不及整巾屣履出側門|{
	側門非正出之門程大昌曰唐大明宫朝堂外左右金吾仗之側有曰側門者以其在端門旁側也以長安大明宫之側門推之則洛陽宫之側門從可知矣屣所徙翻履不躡跟也}
謂璟曰朕謂已斬乃猶未也命趨斬之|{
	趨與趣同尺玉翻}
璟曰人言中宫私於三思陛下不問而誅之臣恐天下必有竊議固請按之上不許璟曰必欲斬月將請先斬臣不然臣終不敢奉詔上怒少解|{
	少詩沼翻}
左御史大夫蘇珦|{
	珦式亮翻}
給事中徐堅大理卿長安尹思貞皆以為方夏行戮有違時令上乃命與杖流嶺南過秋分一日平曉廣州都督周仁軌斬之 |{
	考異曰朝野僉載曰周仁軌過秋分一日平暁斬之有敕捨之而不及統紀月將死附此年末唐紀在二月舊傳唐歷皆在五王死後按此年七月殺敬暉等若在後徐堅表不得云朱夏在辰思貞不得云發生之日也今約其書附於此月}
御史大夫李承嘉附武三思詆尹思貞於朝|{
	朝直遥翻}
思貞曰公附會奸臣將圖不軌先除忠臣邪承嘉怒劾奏思貞出為青州刺史|{
	舊志青州京師東北二千五百二十里至東都一千五百七里}
或謂思貞曰公平日訥於言及廷折承嘉何其敏邪|{
	折之舌翻}
思貞曰物不能鳴者激之則鳴承嘉恃威權相陵僕義不受屈亦不知言之從何而至也 武三思惡宋璟|{
	惡烏路翻}
出之檢校貝州刺史|{
	舊志貝州京師東北一千七百八十二里至東都九百九十三里}
五月庚申葬則天大聖皇后於乾陵 武三思使鄭愔告朗州刺史敬暉亳州刺史韋彦範|{
	桓彦範時賜姓韋因而稱之愔於今翻亳旁各翻}
襄州刺史張柬之郢州刺史袁恕己均州刺史崔玄暐與王同皎通謀六月戊寅貶暉崖州司馬彦範瀧州司馬柬之新州司馬恕己竇州司馬玄暐白州司馬|{
	瀧所江翻白州漢合浦縣地武德初置南州仍分合浦置慱白縣六年改曰白州 考異曰唐歷統紀皆於王同皎誅後即云三思令宣州司功鄭愔誣柬之等與王同皎謀反又貶玄暐等四人為僻遠州刺史按愔若於時已告云謀反則豈應猶得刺史又云告柬之等而柬之豈得獨不貶今從實録}
並員外置仍長任削其勲封復彦範姓桓氏 初韋玄貞流欽州而卒|{
	流欽州見二百三卷武后光宅元年卒子恤翻}
蠻酋甯承基兄弟逼取其女|{
	酋慈由翻}
妻崔氏不與承基等殺之及其四男洵浩洞泚|{
	洵音荀泚且禮翻}
上命廣州都督周仁軌使將兵二萬討之|{
	將即亮翻}
承基等亡入海仁軌追斬之以其首祭崔氏墓殺掠其部衆殆盡上喜加仁軌鎮國大將軍|{
	唐武散官無鎮國大將軍盖中宗創置以寵仁軌也}
充五府大使|{
	五府廣桂邕容瓊五都督府也使疏吏翻}
賜爵汝南郡公韋后隔簾拜仁軌以父事之及韋后敗仁軌以黨與誅 |{
	考異曰朝野僉載曰韋氏遭則天廢廬陵之後后父韋玄貞與妻女等並流嶺南被首領甯氏大族逼奪其女不伏遂殺貞夫妻七娘等並奪去及孝和即位皇后當途廣州都督周仁軌將兵誅甯氏走入南海軌追之殺掠並盡韋后隔簾拜以父事之用為并州長史後阿韋作逆軌以黨與誅今從實録參取諸書}
秋七月戊申立衛王重俊為太子|{
	重直龍翻}
太子性明果而官屬率貴遊子弟所為多不法左庶子姚珽屢諫不聽|{
	為太子不終張本珽作鼎翻}
珽璹之弟也|{
	姚璹相武后璹殊玉翻}
丙寅以李嶠為中書令 上將還西京辛未左散騎常侍李懷遠同中書門下三品充東都留守|{
	散悉亶翻騎奇寄翻守式又翻}
武三思隂令人疏皇后穢行牓於天津橋|{
	行下孟翻}
請加廢黜上大怒命御史大夫李承嘉窮覈其事承嘉奏言敬暉桓彦範張柬之袁恕己崔玄暐使人為之雖云廢后實謀大逆請族誅之三思又使安樂公主譛之於内|{
	安樂公主下嫁三思子崇訓故得使之譛五王樂音洛}
侍御史鄭愔言之於外上命法司結竟|{
	結竟者結其罪竟其獄也或曰竟盡也盡其命也愔於今翻}
大理丞三原李朝隱奏請暉等未經推鞫不可遽就誅夷|{
	朝直遥翻}
大理丞裴談奏請暉等宜據制書處斬籍沒不應更加推鞫|{
	處昌呂翻}
上以暉等嘗賜鐵劵許以不死乃長流暉於瓊州 |{
	考異曰實録初云嘉州舊傳作崖州今從統紀新傳}
彦範於瀼州柬之於瀧州|{
	武德四年平蕭銑分隋永熙郡之瀧水縣置瀧州瀧所江翻瀼州隋將劉方始開此路貞觀十二年尋劉方故道行逹交趾開拓夷獠置瀼州州在鬱林西南交趾之東北有瀼水以為州名}
恕己於環州|{
	貞觀十二年李弘節開拓生蠻置環州取環王國為名屬嶺南道}
玄暐於古州|{
	古州亦李弘節開夷獠置}
子弟年十六以上皆流嶺外擢承嘉為金紫光禄大夫進爵襄武郡公談為刑部尚書出李朝隱為聞喜令三思又諷太子上表請夷暉等三族上不許中書舍人崔湜說三思曰暉等異日北歸終為後患不如遣使矯制殺之三思問誰可使者湜薦大理正周利用利用先為五王所惡貶嘉州司馬乃以利用攝右臺侍御史奉使嶺外比至柬之玄暐已死遇彦範於貴州|{
	說輸芮翻使疏吏翻惡烏路翻比毗至翻貴州漢廣鬱縣地古西甌駱越所居後漢谷永為鬱林太守降烏滸人十餘萬開置七縣即此處也地在廣州西南安南府之北邕管所管郡縣是也隋分鬱林置鬱平縣屬南定州武德曰南尹州貞觀八年曰貴州}
令左右縛之曳於竹槎之上|{
	槎鉏加翻}
肉盡至骨然後杖殺得暉咼而殺之恕己素服黄金利用逼之使飲野葛汁|{
	本草鈎吻一名野葛陶弘景曰言其入口鈎人喉吻覈事而言乃是兩物未詳云何嶺表録異曰野葛毒草也俗呼為胡蔓草誤食之則用羊血解之陳藏器曰人食其葉飲冷水即死冷水發其毒也彼人以野葛飼人勿與冷水至肥大以冷水飲之至死懸尸於樹汁滴地生菌子收之名菌藥烈於野葛}
盡數升不死不勝毒憤掊地|{
	勝音升掊薄侯翻}
爪甲殆盡仍捶殺之|{
	捶止橤翻}
利用還擢拜御史中丞薛季昶累貶儋州司馬飲藥死|{
	儋都甘翻}
三思既殺五王權傾人主常言我不知代間何者謂之善人何者謂之惡人但於我善者則為善人於我惡者則為惡人耳時兵部尚書宗楚客將作大匠宗晉卿大府卿紀處訥鴻臚卿甘元柬皆為三思羽翼|{
	臚陵如翻}
御史中丞周利用侍御史冉祖雍太僕丞李俊光禄丞宋之遜監察御史姚紹之皆為三思耳目時人謂之五狗 九月戊午左散騎常侍同中書門下三品李懷遠薨 初李嶠為吏部侍郎欲樹私恩再求入相奏大置員外官廣引貴埶親識既而為相銓衡失序府庫減耗|{
	相息亮翻}
乃更表言濫官之弊且請遜位上慰諭不許 冬十月己卯車駕發東都以前檢校并州長史張仁愿檢校左屯衛大將軍兼洛州長史戊戌車駕至西京十一月乙巳赦天下 丙辰以蒲州刺史竇從一為雍州刺史|{
	雍於用翻}
從一德玄之子也|{
	竇德玄見二百一卷高宗麟德元年}
初名懷貞避皇后父諱更名從一|{
	更工衡翻}
多諂附權貴太平公主與僧寺爭碾磑|{
	碾魚蹇翻磑五對翻激水為之不勞人功而自運}
雍州司戶李元紘判歸僧寺|{
	唐制戶曹司戶參軍事掌戶籍計帳道路過所蠲符雜徭逋負良賤芻藁逆旅婚姻田訟旌别孝悌}
從一大懼亟命元紘改判元紘大署判後曰南山可移此判無動從一不能奪元紘道廣之子也|{
	李道廣見二百五卷武后萬歲通天元年}
初祕書監鄭普思納其女於後宫監察御史靈昌崔日用劾奏之上不聽|{
	監古衘翻劾戶槩翻又戶得翻}
普思聚黨於雍岐二州謀作亂事覺西京留守蘇瓌收繫窮治之普思妻第五氏以鬼道得幸於皇后上敕瓌勿治及車駕還西京瓌廷爭之|{
	守式又翻瓌古回翻治直之翻爭讀曰諍}
上抑瓌而佑普思侍御史范獻忠進曰請斬蘇瓌上曰何故對曰瓌為留守大臣不能先斬普思然後奏聞使之熒惑聖聽其罪大矣且普思反狀明白而陛下曲為申理臣聞王者不死殆謂是乎臣願先賜死不能北面事普思魏元忠曰蘇瓌長者用刑不枉普思法當死上不得已戊午流普思於儋州|{
	儋都甘翻}
餘黨皆伏誅 十二月己卯突厥默啜寇鳴沙|{
	靈州有鳴沙府武德二年以鳴沙縣置會州貞觀六年州廢更置環州以大河環曲為名九年州廢以縣還屬靈州是年為默啜所寇移治故豐安城宋白曰鳴沙本漢富平縣地後周於此置會州尋立鳴沙鎮隋立環州以大河環曲為名此地人馬行沙有聲異於餘沙故曰鳴沙}
靈武軍大總管沙吒忠義與戰軍敗死者六千餘人|{
	吒初交翻}
丁巳突厥進寇原會等州|{
	武德二年以平凉郡會寧鎮置西會州貞觀八年更名會州}
掠隴右牧馬萬餘匹而去免忠義官 安西大都護郭元振請突騎施烏質勒牙帳議軍事|{
	騎奇寄翻}
天大風雪元振立於帳前與烏質勒語久之雪深元振不移足烏質勒老不勝寒|{
	勝音升}
會罷而卒|{
	卒子恤翻}
其子娑葛勒兵將攻元振|{
	娑素何翻}
副使御史中丞解琬知之|{
	使疏吏翻解戶買翻姓也}
勸元振夜逃去元振曰吾以誠心待人何所疑愳且深在寇庭逃將安適安卧不動明旦入哭甚哀娑葛感其義待元振如初戊戌以娑葛襲嗢鹿州都督懷德王|{
	高宗顕慶元年以突騎施索葛莫賀部置嗢鹿州都督府嗢烏沒翻}
安樂公主恃寵驕恣賣官鬻獄埶傾朝野|{
	朝直遥翻}
或自為制敕掩其文令上署之上笑而從之竟不視也自請為皇太女上雖不從亦不譴責 |{
	考異曰統紀云安樂公主私請廢太子而立己為皇太女帝以問魏元忠元忠曰皇太子國之儲君生人之本今既無罪豈得輒有動搖欲以公主為皇太女駙馬復若為名號天下必甚怪愕恐非公主自安之道公主知之乃奏曰元忠山東木強田舍漢豈足與論國家權宜盛事儀注好惡阿母子尚自為天子况兒是公主作皇太女有何不可按中宗雖愚豈不知立皇太女為不可何必待元忠之言今從舊傳}
景龍元年|{
	是年九月方改元}
春正月庚戌制以突厥默啜寇邊命内外官各進平突厥之策右補闕盧俌上疏|{
	俌方矩翻}
以為郤縠說禮樂敦詩書為晉元帥|{
	左傳晉文公蒐於被廬作三軍謀元帥趙衰曰郤縠可臣亟聞其言矣說禮樂而敦詩書詩書義之府也禮樂德之則也德義利之本也君其試之乃使郤縠將中軍帥所類翻}
杜預射不穿扎建平吴之勲|{
	見八十一卷晉武帝太康元年}
是知中權制謀不取一夫之勇|{
	左傳曰中權後勁注曰中軍制謀}
如沙吒忠義驍將之材本不足以當大任又鳴沙之役王將先逃|{
	鳴沙之敗亦指言沙吒忠義驍堅堯翻將即亮翻}
宜正邦憲賞罰既明敵無不服又邊州刺史宜精擇其人使之蒐卒乘積資糧|{
	乘䋲證翻}
來則禦之去則備之去歲四方旱災未易興師|{
	易以豉翻}
當理内以及外綏近以來遠俟倉廪實士卒練然後大舉以討之上善之二月丙戌上遣武攸暨武三思詣乾陵祈雨既而雨降上喜制復武氏崇恩廟及昊陵順陵|{
	帝既復辟改武氏崇尊廟為崇恩廟太后崩廢崇恩廟昊陵順陵見二百四卷天授二年考異曰舊本紀正月己巳遣武攸暨武三思往乾陵祈雨於則天皇后新本紀甲午褒德榮先陵置令丞按長歷正月庚子朔無己巳二月庚午朔無甲午今從實録}
因名酆王廟曰褒德陵曰榮先|{
	去年追封后父韋玄貞為酆王}
又詔崇恩廟齋郎取五品子充太常博士楊孚曰太廟皆取七品以下子為齋郎今崇恩廟取五品子未知太廟當如何上命太廟亦凖崇恩廟孚曰以臣凖君猶為僭逆况以君凖臣乎上乃止庚寅敕改諸州中興寺觀為龍興|{
	唐會要神龍元年敕天下諸州各置大唐中興寺觀觀古玩翻}
自今奏事不得言中興|{
	示襲武氏後不改其政也}
右補闕權若訥上疏以為天地日月等字|{
	改制字見二百四卷武后天授元年}
皆則天能事賊臣敬暉等輕紊前規今削之無益於淳化存之有光於孝理又神龍元年制書一事以上並依貞觀故事豈可近捨母儀遠尊祖德疏奏手制褒美|{
	史言中宗無是非之心}
三月庚子吐蕃遣其大臣悉薰熱入貢|{
	吐從暾入聲}
夏四月辛巳以上所養雍王守禮女金城公主妻吐蕃贊普|{
	雍於用翻妻七細翻}
五月戊戌以左屯衛大將軍張仁愿為朔方道大總管以備突厥上以歲旱穀貴召太府卿紀處訥謀之明日武三思使知太史事迦葉志忠奏是夜攝提入太微宫|{
	姓譜迦葉天竺姓迦居伽翻晉天文志攝提六星直斗杓之南主建時節伺禨祥三思特使志忠傳會以獻諛耳}
至帝座|{
	太微宫中有太帝之座}
主大臣宴見納忠於天子上以為然|{
	史言帝愚暗為下所罔見賢遍翻}
敕稱處訥忠誠徹於玄象賜衣一襲帛六十段 六月丁卯朔日有食之 姚巂道討擊使監察御史晉昌唐九徵擊姚州叛蠻破之|{
	晉昌漢敦煌郡冥安縣地河西張氏置晉昌郡隋置瓜州改冥安為常樂縣武德四年復改常樂為晉昌縣巂音髓使疏吏翻監古衘翻}
斬獲三千餘人 皇后以太子重俊非其所生惡之|{
	重俊後宫所生史失其姓氏惡烏路翻}
特進德静王武三思尤忌太子上官婕妤以三思故每下制敕推尊武氏安樂公主與駙馬左衛將軍武崇訓常陵侮太子或呼為奴|{
	婕妤音接予樂音洛}
崇訓又教公主言於上請廢太子立己為皇太女太子積不能平秋七月辛丑太子與左羽林大將軍李多祚將軍李思冲李承况獨孤禕沙吒忠義等|{
	椲吁韋翻吒初切翻}
矯制發羽林千騎兵三百餘人|{
	騎奇寄翻}
殺三思崇訓於其第并親黨十餘人又使左金吾大將軍成王千里及其子天水王禧分兵守宫城諸門太子與多祚引兵自肅章門斬關而入叩閤索上官婕妤|{
	索山客翻下同 考異曰舊紀作庚子今從實録實録云斬關而入索韋氏所在舊重俊傳亦云求韋庶人及安樂公主所在今從舊后妃傳}
婕妤大言曰觀其意欲先索婉兒|{
	婉兒上官倢伃名也}
次索皇后次及大家上乃與韋后安樂公主上官倢伃登玄武門樓以避兵鋒使右羽林大將軍劉景仁帥飛騎百餘人屯於樓下以自衛楊再思蘇瓌李嶠與兵部尚書宗楚客左衛將軍紀訥擁兵二千餘人屯太極殿前閉門自守多祚先至玄武樓下欲升樓宿衛拒之多祚與太子狐疑按兵不戰冀上問之宫闈令石城楊思朂在上側|{
	唐制宫闈局令從七品下屬内侍省掌侍奉宫闈出入管籥石城縣屬羅州漢合浦縣地劉昫曰宋將檀道濟於綾羅江口築石城後因置羅州唐置石城縣歐陽修曰以石城水為名}
請擊之多祚壻羽林中郎將野呼利為前鋒總管|{
	將即亮翻下同}
思朂挺刃斬之|{
	挺拔也}
多祚軍奪氣上據檻俯謂多祚所將千騎曰汝輩皆朕宿衛之士何為從多祚反苟能斬反者勿患不富貴於是千騎斬多祚承况禕之忠義餘衆皆潰成王千里天水王禧攻右延明門|{
	閣本太極宫圖太極殿之左曰左延明門右曰右延明門}
將殺宗楚客紀處訥不克而死太子以百騎走終南山至鄠西能屬者纔數人|{
	走音奏鄠音戶屬之欲翻}
憩於林下為左右所殺上以其首獻太廟反祭三思崇訓之柩然後梟之朝堂|{
	柩音舊梟堅堯翻朝直遥翻}
更成王千里姓曰蝮氏同黨皆伏誅東宫僚屬無敢近太子尸者|{
	更工衡翻蝮芳福翻近其靳翻}
唯永和縣丞甯嘉朂解衣裹太子首號哭貶興平丞|{
	永和漢狐讘縣地後周置臨河縣又臨河郡隋廢郡改縣曰永和唐屬隰州讘之涉翻興平新書作平興平興漢高興縣地宋置平興縣帶宋熙郡隋廢郡以平興縣屬端州岐州有興平畿内也永和外縣嘉朂若自外縣丞得畿縣丞則非貶矣此必貶嶺外之興平也當從新書號戶高翻}
太子兵所經諸門守者皆坐流韋氏之黨奏請悉誅之上更命法司推斷大理卿宋城鄭惟忠曰大獄始决人心未安若復有改推則反仄者衆矣上乃止|{
	守手又翻更工衡翻斷丁亂翻復扶又翻}
以楊思朂為銀青光禄大夫行内常侍|{
	唐内常侍正五品下漢世之中常侍也六典内侍省内侍四人内常侍六人内侍之職掌在内侍奉出入宫掖宣傳詔令總掖庭宫闈奚官内僕内府五局之官屬内常侍為之貳}
癸卯赦天下贈武三思太尉梁宣王武崇訓開府儀同三思魯忠王安樂公主請用永泰公主故事以崇訓墓為陵給事中盧粲駮之以為永泰事出特恩|{
	永泰主死見上卷元年帝復辟以主禮改葬特恩號墓為陵亦非禮也駮北角翻}
今魯王主壻不可為比上手敕曰安樂與永泰無異同穴之義今古不殊粲又奏陛下以膝下之愛施及其夫|{
	施以豉翻}
豈可使上下無辨君臣一貫哉上乃從之公主怒出粲為陳州刺史|{
	舊志陳州京師一千五百二十里東都七百一十七里}
襄邑尉襄陽席豫|{
	襄邑縣漢晉屬陳留郡後魏屬陽夏郡後齊廢隋開皇十六年復置屬宋州襄陽縣帶襄州}
聞安樂公主求為太女嘆曰梅福譏切王氏|{
	梅福事見三十一卷漢成帝永始三年}
獨何人哉乃上書請立太子言甚深切太平公主欲表為諫官豫耻之逃去 八月戊寅皇后及王公已下表上尊號曰應天神龍皇帝改玄武門為神武門樓為制勝樓宗楚客又帥百官表請加皇后尊號曰順天翊聖皇后|{
	帥讀曰率}
上並許之 初右臺大夫蘇珦治太子重俊之黨囚有引相王者珦密為之申理|{
	珦許亮翻治直之翻重直龍翻相息亮翻為于偽翻}
上乃不問自是安樂公主及兵部尚書宗楚客日夜謀譛相王|{
	樂音洛尚辰羊翻}
使侍御史冉祖雍誣奏相王及太平公主云與重俊通謀請收付制獄上召吏部侍郎兼御史中丞蕭至忠使鞫之至忠泣曰陛下富有四海不能容一弟一妹而使人羅織害之乎相王昔為皇嗣固請於則天以天下讓陛下|{
	事見二百六卷武后聖歷元年嗣祥吏翻}
累日不食此海内所知奈何以祖雍一言而疑之上素友愛遂寢其事右補闕浚儀吴兢聞祖雍之謀|{
	浚儀古大梁也自漢以來屬陳留郡竹書紀年梁惠王為大溝以行圃田之水縣北有浚水像而儀之故曰浚儀}
上疏以為自文明以來國之祚胤不絶如綫|{
	上時掌翻疏所去翻綫私箭翻}
陛下龍興恩及九族求之瘴海升之闕庭|{
	事見上龍神元年}
况相王同氣至親六合無貳而賊臣日夜連謀乃欲䧟之極法禍亂之根將由此始夫任以權則雖踈必重奪其勢則雖親必輕|{
	夫音扶}
自古委信異姓猜忌骨肉以覆國亡家者幾何人矣况國家枝葉無幾陛下登極未久而一子以弄兵受誅一子以愆違遠竄|{
	受誅謂重俊遠竄謂重福}
惟餘一弟朝夕左右尺布斗粟之譏不可不慎|{
	尺布斗粟見十四卷漢文帝七年}
青蝇之詩良可畏也|{
	青蝇之詩周人刺幽王信讒也 考異曰實録載此事於今年八月而兢疏云陛下登極於今四稔則是明年所上也盖至忠所對在今年而實録因載兢疏耳}
相王寛厚恭謹安恬好讓故經武韋之世竟免於難|{
	好呼到翻難乃旦翻}
初右僕射中書令魏元忠以武三思擅權意常憤鬱及太子重俊起兵遇元忠子太僕少卿升於永安門|{
	唐六典太極宫城南面三門中曰承天東曰長樂西曰永安}
脅以自隨太子死升為亂兵所殺元忠揚言曰元惡已死雖鼎鑊何傷但惜太子隕沒耳上以其有功且為高宗武后所重故釋不問兵部尚書宗楚客太府卿紀處訥等共證元忠云與太子通謀請夷其三族制不許元忠懼表請解官爵以散秩還第|{
	散悉亶翻}
丙戌上手敕聽解僕射以特進齊公致仕 |{
	考異曰實録元忠致仕在九月今從舊本紀}
仍朝朔望|{
	朝直遥翻}
九月丁卯以吏部侍郎蕭至忠為黄門侍郎兵部尚書宗楚客為左衛將軍兼太府卿紀處訥為太府卿並同中書門下三品中書侍郎同中書門下三品于惟謙罷為國子祭酒 庚子赦天下改元|{
	改元景龍}
宗楚客等引右衛郎將姚廷筠為御史中丞使劾奏魏元忠|{
	劾戶槩翻又戶得翻}
以為侯君集社稷元勛及其謀反太宗就羣臣乞其命而不得竟流涕斬之|{
	見一百九十七卷太宗貞觀十七年}
其後房遺愛薛萬徹齊王祐等為逆雖復懿親皆從國法|{
	齊王祐見一百九十六卷貞觀十七年房薛見一百九十九卷高宗永徽四年復扶又翻下一復同}
元忠功不逮君集身又非國戚與李多祚等謀反男入逆徒是宜赤族汚宫但有朋黨飾辭營救以惑聖聽陛下仁恩欲掩其過臣所以犯龍鱗忤聖意者|{
	忤五故翻}
正以事關宗社耳上頗然之元忠坐繫大理貶渠州司馬|{
	渠州漢宕渠縣地後魏置流江縣及流江郡梁置渠州後周又為北宕渠郡隋復置渠州舊志渠州京師西南二千二百七十里至東都三千一百九十里}
宗楚客令給事中冉祖雍奏言元忠既犯大逆不應出佐渠州楊再思李嶠亦贊之上謂再思等曰元忠驅使日久朕特矜容制命已行豈容數改|{
	數所角翻}
輕重之權應自朕出卿等頻奏殊非朕意再思等惶懼拜謝監察御史袁守一復表彈元忠曰重俊乃陛下之子猶加昭憲元忠非勲非戚焉得獨漏嚴刑甲辰又貶元忠務川尉|{
	務川漢西陽縣地隋開皇末招慰蟄獠置務川縣屬巴東郡唐置思州監古銜翻下同渾徒丹翻重直龍翻焉於䖍翻}
頃之楚客又令袁守一奏言|{
	令力丁翻}
則天昔在三陽宫不豫狄仁傑奏請陛下監國元忠密奏以為不可此則元忠懷逆日久請加嚴誅上謂楊再思等曰以朕思之人臣事主必在一心豈有主上小疾遽請太子知事此乃仁傑欲樹私恩未見元忠有失守一欲借前事以陷元忠其可乎楚客乃止元忠行至涪陵而卒|{
	涪音浮卒子恤翻}
銀青光禄大夫上庸公聖善中天西明三寺主慧範於東都作聖善寺|{
	聖善寺盖為武后資福取母氏聖善之義唐會要聖善寺在長安城中章善坊神龍二年中宗為武后追福西明寺在延康坊本隋越國公楊素宅貞觀初賜濮長泰泰死乃立為寺}
長樂坡作大像|{
	長樂坡在長安城東亦謂之滻坡樂音洛}
府庫為之虛耗|{
	為于偽翻}
上及韋后皆重之勢傾内外無敢指目者戊申侍御史魏傳弓發其姦贓四十餘萬請寘極法上欲宥之傳弓曰刑賞國之大事陛下賞已妄加豈宜刑所不及上乃削黜慧範放於家宦官左監門大將軍薛簡等有寵於安樂公主|{
	監古銜翻樂音洛}
縱暴不法傳弓奏請誅之御史大夫竇從一懼固止之時宦官用事從一為雍州刺史及御史大夫|{
	雍於用翻}
誤見訟者無須必曲加承接|{
	意以為宦官而然須與鬚同}
以楊再思為中書令韋巨源紀處訥並為侍中 |{
	考異曰新表九月辛亥蘇瓌罷為行吏部尚書按二年壞請察正員官殿負者擇員外官代之三年面折祝欽明請皇后亞獻於時更為侍中表云今年罷誤也}
壬戍改左右羽林千騎為萬騎|{
	騎奇寄翻}
冬十月丁丑命左屯衛將軍張仁愿充朔方道大總管以擊突厥|{
	左屯衛之下逸大字}
比至虜已退|{
	比必利翻}
追擊大破之 習藝舘内教蘇安恒|{
	習藝舘本名内文學舘選官人有文學者一人為學士教習宫人武后改為習藝舘又改為翰林内教坊以地在禁中故也新書曰掌教習宫人書筭衆藝恒戶登翻}
矜高好奇|{
	好呼到翻}
太子重俊之誅武三思也安恒自言此我之謀太子敗或告之戊寅伏誅 十二月乙丑朔日有食之 是歲上遣使者分道詣江淮贖生|{
	帝以江淮之人采捕魚鼈為傷生分道遣使以錢物贖之使疏吏翻}
中書舍人房子李乂|{
	房子縣漢屬常山郡晋後魏屬趙郡隋唐屬趙州}
上疏諫曰江南鄉人|{
	鄉人猶言鄉民避太宗諱改民曰人上時掌翻疏所去翻}
采捕為業魚鼈之利黎元所資雖雲雨之私有霑於末利而生成之惠未洽於平人何則江湖之饒生育無限府庫之用支供易殚|{
	易以豉翻}
費之若少則所濟何成|{
	少詩沼翻}
用之儻多則常支有闕在其拯物豈若憂人且鬻生之徒惟利是視錢刀日至|{
	古有金刀錢布故曰錢刀}
網罟年滋施之一朝營之百倍|{
	施式豉翻}
未若迴救贖之錢物減貧無之徭賦活國愛人其福勝彼

資治通鑑卷二百八
















































































































































