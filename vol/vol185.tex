<!DOCTYPE html PUBLIC "-//W3C//DTD XHTML 1.0 Transitional//EN" "http://www.w3.org/TR/xhtml1/DTD/xhtml1-transitional.dtd">
<html xmlns="http://www.w3.org/1999/xhtml">
<head>
<meta http-equiv="Content-Type" content="text/html; charset=utf-8" />
<meta http-equiv="X-UA-Compatible" content="IE=Edge,chrome=1">
<title>資治通鑒_186-資治通鑑卷一百八十五_186-資治通鑑卷一百八十五</title>
<meta name="Keywords" content="資治通鑒_186-資治通鑑卷一百八十五_186-資治通鑑卷一百八十五">
<meta name="Description" content="資治通鑒_186-資治通鑑卷一百八十五_186-資治通鑑卷一百八十五">
<meta http-equiv="Cache-Control" content="no-transform" />
<meta http-equiv="Cache-Control" content="no-siteapp" />
<link href="/img/style.css" rel="stylesheet" type="text/css" />
<script src="/img/m.js?2020"></script> 
</head>
<body>
 <div class="ClassNavi">
<a  href="/24shi/">二十四史</a> | <a href="/SiKuQuanShu/">四库全书</a> | <a href="http://www.guoxuedashi.com/gjtsjc/"><font  color="#FF0000">古今图书集成</font></a> | <a href="/renwu/">历史人物</a> | <a href="/ShuoWenJieZi/"><font  color="#FF0000">说文解字</a></font> | <a href="/chengyu/">成语词典</a> | <a  target="_blank"  href="http://www.guoxuedashi.com/jgwhj/"><font  color="#FF0000">甲骨文合集</font></a> | <a href="/yzjwjc/"><font  color="#FF0000">殷周金文集成</font></a> | <a href="/xiangxingzi/"><font color="#0000FF">象形字典</font></a> | <a href="/13jing/"><font  color="#FF0000">十三经索引</font></a> | <a href="/zixing/"><font  color="#FF0000">字体转换器</font></a> | <a href="/zidian/xz/"><font color="#0000FF">篆书识别</font></a> | <a href="/jinfanyi/">近义反义词</a> | <a href="/duilian/">对联大全</a> | <a href="/jiapu/"><font  color="#0000FF">家谱族谱查询</font></a> | <a href="http://www.guoxuemi.com/hafo/" target="_blank" ><font color="#FF0000">哈佛古籍</font></a> 
</div>

 <!-- 头部导航开始 -->
<div class="w1180 head clearfix">
  <div class="head_logo l"><a title="国学大师官网" href="http://www.guoxuedashi.com" target="_blank"></a></div>
  <div class="head_sr l">
  <div id="head1">
  
  <a href="http://www.guoxuedashi.com/zidian/bujian/" target="_blank" ><img src="http://www.guoxuedashi.com/img/top1.gif" width="88" height="60" border="0" title="部件查字,支持20万汉字"></a>


<a href="http://www.guoxuedashi.com/help/yingpan.php" target="_blank"><img src="http://www.guoxuedashi.com/img/top230.gif" width="600" height="62" border="0" ></a>


  </div>
  <div id="head3"><a href="javascript:" onClick="javascript:window.external.AddFavorite(window.location.href,document.title);">添加收藏</a>
  <br><a href="/help/setie.php">搜索引擎</a>
  <br><a href="/help/zanzhu.php">赞助本站</a></div>
  <div id="head2">
 <a href="http://www.guoxuemi.com/" target="_blank"><img src="http://www.guoxuedashi.com/img/guoxuemi.gif" width="95" height="62" border="0" style="margin-left:2px;" title="国学迷"></a>
  

  </div>
</div>
  <div class="clear"></div>
  <div class="head_nav">
  <p><a href="/">首页</a> | <a href="/ShuKu/">国学书库</a> | <a href="/guji/">影印古籍</a> | <a href="/shici/">诗词宝典</a> | <a   href="/SiKuQuanShu/gxjx.php">精选</a> <b>|</b> <a href="/zidian/">汉语字典</a> | <a href="/hydcd/">汉语词典</a> | <a href="http://www.guoxuedashi.com/zidian/bujian/"><font  color="#CC0066">部件查字</font></a> | <a href="http://www.sfds.cn/"><font  color="#CC0066">书法大师</font></a> | <a href="/jgwhj/">甲骨文</a> <b>|</b> <a href="/b/4/"><font  color="#CC0066">解密</font></a> | <a href="/renwu/">历史人物</a> | <a href="/diangu/">历史典故</a> | <a href="/xingshi/">姓氏</a> | <a href="/minzu/">民族</a> <b>|</b> <a href="/mz/"><font  color="#CC0066">世界名著</font></a> | <a href="/download/">软件下载</a>
</p>
<p><a href="/b/"><font  color="#CC0066">历史</font></a> | <a href="http://skqs.guoxuedashi.com/" target="_blank">四库全书</a> |  <a href="http://www.guoxuedashi.com/search/" target="_blank"><font  color="#CC0066">全文检索</font></a> | <a href="http://www.guoxuedashi.com/shumu/">古籍书目</a> | <a   href="/24shi/">正史</a> <b>|</b> <a href="/chengyu/">成语词典</a> | <a href="/kangxi/" title="康熙字典">康熙字典</a> | <a href="/ShuoWenJieZi/">说文解字</a> | <a href="/zixing/yanbian/">字形演变</a> | <a href="/yzjwjc/">金 文</a> <b>|</b>  <a href="/shijian/nian-hao/">年号</a> | <a href="/diming/">历史地名</a> | <a href="/shijian/">历史事件</a> | <a href="/guanzhi/">官职</a> | <a href="/lishi/">知识</a> <b>|</b> <a href="/zhongyi/">中医中药</a> | <a href="http://www.guoxuedashi.com/forum/">留言反馈</a>
</p>
  </div>
</div>
<!-- 头部导航END --> 
<!-- 内容区开始 --> 
<div class="w1180 clearfix">
  <div class="info l">
   
<div class="clearfix" style="background:#f5faff;">
<script src='http://www.guoxuedashi.com/img/headersou.js'></script>

</div>
  <div class="info_tree"><a href="http://www.guoxuedashi.com">首页</a> > <a href="/SiKuQuanShu/fanti/">四库全书</a>
 > <h1>资治通鉴</h1> <!--         下载:【右键另存为】即可 --></div>
  <div class="info_content zj clearfix">
  
<div class="info_txt clearfix" id="show">
<center style="font-size:24px;">186-資治通鑑卷一百八十五</center>
    資治通鑑卷一百八十五<br />
<br />
  宋 司馬光 撰<br />
<br />
  胡三省 音註<br />
<br />
  唐紀一【起著雍攝提格正月盡七月不滿一年唐古國名陸德明曰周成王母弟叔虞封於唐其地帝堯夏禹所都之墟漢曰太原郡在古冀州太行恒山之西太原太岳之野李唐之先李虎與李弼等八人佐周伐魏有功皆為柱國家周閔帝受魏禪虎已卒乃追録其功封唐國公生子昞襲封昞生淵襲封起兵克長安進封唐王遂受隋禪國因號曰唐】<br />
<br />
  高祖神堯大聖光孝皇帝上之上【諱淵字叔德本隴西成紀人七世祖暠王西凉是為凉武昭王至曾孫熙家于武川熙孫虎從周文帝始家長安】<br />
<br />
  武德元年【是年五月受隋禪始改元】春正月丁未朔隋恭帝詔唐王劒履上殿贊拜不名【隋志按漢自天子至于百官無不佩刀蔡謨儀云大臣優禮者皆劒履上殿非侍臣解之蓋防刃也近代以木未詳所起東齊著令謂為象劍言象於劒周武帝時百官燕會並帶刀升座至開皇初因襲舊式朝服升殿亦不解焉十二年因蔡徵上事始制凡朝會應登殿坐者劒履俱脱其不坐者勑召奏事及須升殿亦就席解劒乃登納言黄門内史令侍郎舍人既夾侍之官則不脱其劒皆真刃非假又准晉咸康元年定今故事自天子以下皆衣冠帶劒今天子則玉具火珠鏢首惟侍臣帶劎上殿自王公以下非殊禮引升殿皆就席解而後升複下曰舄單下曰履諸非侍臣皆脱履升殿舄唯冕服及具服箸之履則諸服皆用凡朝會贊拜則曰某官某不名亦殊禮也上時掌翻鏢紕招翻】唐王既克長安以書諭諸郡縣於是東自商洛【隋志商洛縣屬上洛郡取商山洛水以名縣也】南盡巴蜀郡縣長吏及盜賊渠帥氐羌酋長爭遣子弟入見請降有司復書日以百數【長知兩翻帥所類翻酋才由翻見賢遍翻降戶江翻】王世充既得東都兵進擊李密於洛北敗之【敗補邁翻】遂屯鞏北【鞏縣之北】辛酉世充命諸軍各造浮橋度洛擊密橋先成者先進前後不一虎賁郎將王辯破密外柵【賁音奔將即亮翻】密營中驚擾將潰世充不知鳴角收衆密因帥敢死士乘之【帥讀曰率】世充大敗爭橋溺死者萬餘人【溺奴狄翻】王辯死世充僅自免洛北諸軍皆潰世充不敢入東都北趣河陽【趣七喻翻又逡須翻】是夜疾風寒雨軍士涉水沾濕道路凍死者又以萬數世充獨與數千人至河陽 【考異曰隋書北史李密傳曰世充復移營洛北南對鞏縣其後遂於洛水造浮橋悉衆以擊密密出擊之官軍稍却自相䧟溺者數萬人世充僅而獲免不敢還東都遂趣河陽其夜雨雪尺餘衆隨之者死亡殆盡王世充傳曰充敗績赴水溺死者萬餘人時天寒大雪兵士既度水衣皆霑濕在道凍死者又數萬人蒲山公傳曰世充移營就洛水之北與密隔洛水以相望密乃築長城掘深塹周迴七十里以自固十五日世充與密戰於石窟寺東密軍退敗世充度洛水以乘之逼倉城為營塹密縱兵疾戰世充兵馬棄仗奔亡沉溺死者不可勝數密又令露布上府曰世充以今月十一日平旦屯兵洛北偷入月城其月十五日世充及王辯才等又於倉城北偷渡水南敢逼城堞河洛記曰十六日充與密戰於石窟寺東又曰其夜遇風寒疾雨士卒凍死十不存一充脱身宵遁直向河陽餘如蒲山公傳略記曰辛酉王世充移兵洛北仍令諸軍臨岸布兵軍别造浮橋橋先成者輒渡既前後不一而李密伏發我師敗績爭橋赴水溺死者十五六雜記曰十二月越王遣太常少卿韋霽等率留守兵三萬並受世充節度又曰王辯縱等敗衆軍亦潰爭橋赴水死者大半王辯縱等皆沒唯世充敗免與數百騎奔大通城敗兵得還者於道遭大雨凍死者六七千人世充停留大通十餘日懼罪不還十四年正月越王遣世充兄世惲往大通慰諭赦世充喪師之罪按李道玄勸進於李密表云子時律始太簇未宜霡霂而澍雨忽降凍殕將盡今參取衆書日從蒲山公傳兩從河洛記】自繫獄請罪越王侗遣使赦之【侗它紅翻使疏吏翻】召還東都賜金帛美女以安其意世充收合亡散得萬餘人屯含嘉城【含嘉城盖在都城之北按舊書王世充傳含嘉倉城也】不敢復出【復扶又翻】密乘勝進據金墉城修其門堞廬舍而居之【堞達恊翻】鉦鼓之聲聞於東都【開音問】未幾擁兵三十萬陳於北邙【陳讀曰陣】南逼上春門乙丑金紫光禄大夫段達民部尚書韋津出兵拒之達望見密兵盛懼而先還密縱兵乘之軍遂潰韋津死 【考異曰隋書列傳不言戰日蒲山公傳此戰在四月九日略記亦云四月乙未李密率衆北據邙山南接上春門段達韋津出兵拒之兵未交而達懼先還入城軍遂潰亂乙未二十一日也今據河洛記正月十九日世充又與密戰於上春門外韋津沒焉又二月房彦藻與竇建德書亦云幕府以去月十九日親董貔虎西取洛邑其蒲山公傳四月已後月日與事多差互不合今日從河洛記事從略記及隋段達傳】於是偃師柏谷及河陽都尉獨孤武都檢校河内郡丞柳爕職方郎柳續等【隋制職方郎屬兵部尚書】各舉所部降於密竇建德朱粲孟海公徐圓朗等並遣使奉表勸進【降戶江翻使疏吏翻 考異曰河洛記云盧祖尚亦通表於密按祖尚本起兵為隋事恐不爾今不取】密官屬裴仁基等亦上表請正位號【上時掌翻】密曰東都未平不可議此 戊辰唐王以世子建成為左元帥秦公世民為右元帥【帥所類翻】督諸軍十餘萬人救東都東都乏食太府卿元文都等募守城不食公糧者進散官二品於是商賈執象而朝者不可勝數【象者象笏也西魏以來五品已上通用象牙賈音古朝直遥翻勝音升】二月己卯唐王遣太常卿鄭元璹將兵出商洛狥南陽【煬帝改鄧州為南陽郡璹殊玉翻將即亮翻】左領軍府司馬安陸馬元規狥安陸及荆襄【隋十二衛府各有長史司馬煬帝改安州為安陸郡荆州南郡襄州襄陽郡】李密遣房彦藻鄭頲等【頲他鼎翻】東出黎陽分道招慰州縣以梁郡太守楊汪為上柱國宋州總管【煬帝改宋州為梁郡守式又翻】又以手書與之曰昔在雍丘曾相追捕【事見一百八十三卷大業十二年】射鉤斬袂不敢庶幾【管仲射齊桓公中帶鉤桓公用之以相晉寺人披伐公子重耳斬其祛文公不怨今以祛爲袂射而亦翻幾居希翻】汪遣使往來通意密亦羈縻待之【使疏吏翻】彦藻以書招竇建德使來見密建德復書卑辭厚禮託以羅藝南侵請捍禦北垂彦藻還至衛州賊帥王德仁邀殺之【帥所類翻下同】德仁有衆數萬據林慮山【衛州隋為汲郡林慮山在魏郡林慮縣慮音廬】四出抄掠為數州之患【抄楚交翻】 三月己酉以齊公元吉為鎮北將軍【考異曰創業注改太原留守為鎮北府在去年十一月己巳盖因元吉進封齊公言之耳今從實錄】太<br />
<br />
  原道行軍元帥都督十五郡諸軍事聽以便宜從事隋煬帝至江都【大業十二年煬帝至江都】荒淫益甚宮中為百餘房各盛供張【張竹亮翻】實以美人日令一房為主人江都郡丞趙元楷掌供酒饌【饌雛戀翻又雛皖翻】帝與蕭后及幸姬歷就宴飲酒巵不離口從姬千餘人亦常醉【離力智翻從才用翻】然帝見天下危亂意亦擾擾不自安退朝則幅巾短衣策杖步遊徧歷臺館非夜不止汲汲顧景唯恐不足帝自曉占候卜相好為吳語【朝直遥翻相息亮翻好呼到翻】常夜置酒仰視天文謂蕭后曰外間大有人圖儂【吳人率自稱曰儂】然儂不失為長城公卿不失為沈后【長城公陳叔寶叔寶后沈氏】且共樂飲耳【樂音洛】因引滿沈醉【沈持林翻】又嘗引鏡自照顧謂蕭后曰好頭頸誰當斫之后驚問故帝笑曰貴賤苦樂更迭為之【樂音洛更工衡翻】亦復何傷【復扶又翻】帝見中原已亂無心北歸欲都丹楊【帝改蔣州為丹楊郡盖欲都建康也 考異曰大業記帝欲南巡會稽今從隋書】保據江東命羣臣廷議之内史侍郎虞世基等皆以為善右候衛大將軍李才極陳不可請車駕還長安與世基忿爭而出門下錄事衡水李桐客曰【隋制門下省置錄事通事令史各六人衡水縣屬信都郡開皇十六年分信都北界武邑西界下博南界置宋白曰衡水縣本漢桃縣】江東卑濕土地險狹内奉萬乘外給三軍民不堪命亦恐終散亂耳御史劾桐客謗毁朝政【乘䋲證翻劾戶槩翻又戶得翻朝直遥翻】於是公卿皆阿意言江東之民望幸已久陛下過江撫而臨之此大禹之事也【禹南巡狩會諸侯於會稽】乃命治丹楊宮將徙都之【治直之翻】時江都糧盡從駕驍果多關中人【從才用翻驍堅堯翻】久客思鄉里見帝無西意多謀叛歸郎將竇賢遂帥所部西走【將即亮翻下同帥讀曰率】帝遣騎追斬之【騎奇寄翻】而亡者猶不止帝患之虎賁郎將扶風司馬德戡素有寵於帝【賁音奔戡音堪】帝使領驍果屯於東城德戡與所善虎賁郎將元禮直閤裴䖍通謀曰【煬帝制左右監門府有直閤各六人正五品】今驍果人人欲亡我欲言之恐先事受誅【先悉薦翻】不言於後事發亦不免族滅奈何又聞關内淪沒李孝常以華陰叛【事見上卷上年華戶化翻】上囚其二弟欲殺之我輩家屬皆在西能無此慮乎二人皆懼曰然則計將安出德戡曰驍果若亡不若與之俱去二人皆曰善因轉相招引内史舍人元敏虎牙郎將趙行樞鷹揚郎將孟秉符璽郎牛方裕直長許弘仁薛世良城門郎唐奉義醫正張愷勲侍楊士覽等【隋初門下省統城門尚食尚藥符璽御府殿内等六局各有直長煬帝以城門尚食尚藥御府等五局隷殿内省改符璽監為郎城門置校尉後又改校尉為城門郎又置司醫醫佐等官意者醫正即司醫也勲侍三侍之一也璽斯氏翻長知兩翻】皆與之同謀日夜相結約於廣座明論叛計無所畏避有宫人白蕭后曰外間人人欲反后曰任汝奏之宮人言於帝帝大怒以為非所宜言斬之其後宮人復白后【復扶又翻】后曰天下事一朝至此無可救者何用言之徒令帝憂耳【令力丁翻】自是無復言者趙行樞與將作少監宇文智及素厚【少始照翻】楊士覽智及之甥也二人以謀告智及智及大喜德戡等期以三月望日結黨西遁智及曰主上雖無道威令尚行卿等亡去正如竇賢取死耳今天實喪隋【喪息浪翻】英雄並起同心叛者已數萬人因行大事此帝王之業也德戡等然之行樞薛世良請以智及兄右屯衛將軍許公化及為主結約既定乃告化及化及性駑怯聞之變色流汗既而從之 【考異曰蒲山公傳曰趙行樞楊士覽以司馬德戡謀告化及化及兄弟聞之大喜因引德戡等相見士及說德戡等曰足下等因百姓之心謀非常之事直欲走逃故非長策德戡曰為之奈何士及曰官家雖言無道臣下尚畏服之聞公叛亡必急相追捕竇賢之事殷鍳在近不如嚴勒士馬攻其宮闕因人之欲稱廢昏凶事必克成然後詳立明哲天下可安吾徒無患矣勲庸一集公等坐延榮禄縱事不成威聲大振足得官家膽懾不敢輕相追討遲疑之間自延數日比其議定公等行亦已遠如此則去住之計俱保萬全不亦可乎德戡等大悦曰明哲之望豈惟楊家衆心實在許公故是人天協契士及佯驚曰此非意所及但與公等思救命耳革命記曰帝知歷數將窮意欲南渡江水咸言不可帝知朝士不欲渡乃將毒藥醖酒二十石擬三月十六日為宴會而酖殺百官南陽公主恐其夫死乃陰告之而事泄為此始謀害帝以免禍並是兇逆之旅妄搆此詞于時上下離心人懷異志帝深猜忌情不與人醖若不虛藥須分付有處遣何人併醖二十石藥酒必其酒有酖毒一石堪殺千人審欲擬殺羣寮謀之者必有三五衆謀自然早泄豈得獨在南陽只是䖍通恥有殺害之謀推過惡於人主耳隋書化及傳云化及弑逆士及在公主第弗之知也智及遣家僮莊桃樹就第殺之桃樹不忍執詣智及久之乃見釋南陽公主傳責士及云但謀逆之日察君不預知耳舊唐書士及傳云化及謀逆以其主壻深忌之而不告按士及仕唐為宰相隋書亦唐初修或者史官為士及隱惡賈杜二書之言亦似可信但杜儒童自知醖藥酒為虛則南陽陰告之事亦非其實如賈潤甫之說則弑君之謀皆出士及而智及為良人矣今且從隋書而刪去莊桃樹事及南陽之語庶幾疑以傳疑】德戡使許弘仁張愷入備身府【帝改左右領左右府為左右備身府】告所識者云陛下聞驍果欲叛多醖毒酒欲因享會盡鴆殺之獨與南人留此驍果皆懼轉相告語【語牛倨翻】反謀益急乙卯德戡悉召驍果軍吏諭以所為皆曰唯將軍命是日風霾晝昏【霾亡皆翻雨土也】晡後德戡盜御廐馬濳厲兵刃是夕元禮裴䖍通直閤下專主殿内唐奉義主閉城門與䖍通相知諸門皆不下鍵【鍵戶偃翻陳楚謂戶鑰牡為鍵】至三更【更工衡翻】德戡於東城集兵得數萬人舉火與城外相應帝望見火且聞外諠囂問何事䖍通對曰草坊失火外人共救之耳時内外隔絶帝以為然智及與孟秉於城外集千餘人【此城外謂江都宮城之外】劫衛虎賁馮普樂布兵分守衢巷【左右候衛主晝夜巡察故劫之普樂盖虎賁郎將賁音奔樂音洛】燕王倓覺有變【倓元德太子昭之子代王侑之弟倓徒甘翻】夜穿芳林門側水竇而入至玄武門詭奏曰臣猝中風【中竹仲翻】命懸俄頃請得面辭裴䖍通等不以聞執囚之丙辰天未明德戡授䖍通兵以代諸門衛士䖍通自門將數百騎至成象殿【將即亮翻騎奇寄翻】宿衛者傳呼有賊䖍通乃還閉諸門獨開東門驅殿内宿衛者令出皆投仗而走右屯衛將軍獨孤盛謂䖍通曰何物兵勢太異䖍通曰事勢已然不預將軍事將軍慎毋動盛大罵曰老賊是何物語不及被甲與左右十餘人拒戰為亂兵所殺【被皮義翻 考異曰蒲山公傳䖍通於成象殿前遇將軍獨孤盛時内直宿陳兵廊下以拒之詬曰天子在此爾等何敢兇逆叱兵接戰兵皆倒戈䖍通謂盛曰公何暗於機會恐他人以公為勲耳盛叱之曰國家榮寵盛者正擬今日且宿衛天居唯當效之以死注弦不動俄為亂兵所擊斃於階下略記詰旦諸門已開而外傳叫有賊䖍通乃還閉諸門唯開正東一門而驅殿内執仗者出莫不投仗亂走屯衛大將軍獨孤盛揮刀叱之曰天子在此爾等走欲何之然亂兵交萃俄而斃於階下今從隋書亦采略記】盛楷之弟也【獨孤楷見一百七十九卷文帝仁夀二年】千牛獨孤開遠【煬帝制千牛十六人掌執千牛刀屬領左右府開遠獨孤后之兄子】帥殿内兵數百人詣玄覽門叩閤請曰兵仗尚全猶堪破賊陛下若出臨戰人情自定不然禍今至矣竟無應者軍士稍散賊執開遠義而釋之先是帝選驍健官奴數百人置玄武門【先悉薦翻】謂之給使以備非常待遇優厚至以宮人賜之司宮魏氏為帝所信【司宮盖即尚宮之職】化及等結之使為内應是日魏氏矯詔悉聽給使出外倉猝際制無一人在者德戡等引兵自玄武門入帝聞亂易服逃於西閤䖍通與元禮進兵排左閤魏氏啓之遂入永巷問陛下安在有美人出指之校尉令狐行達拔刀直進【校戶教翻令音鈴】帝映窻扉謂行達曰汝欲殺我邪【邪音耶】對曰臣不敢但欲奉陛下西還耳因扶帝下閤【還從宣翻又音如字】䖍通本帝為晉王時親信左右也帝見之謂曰卿非我故人乎何恨而反對曰臣不敢反但將士思歸欲奉陛下還京師耳【將即亮翻】帝曰朕方欲歸正為上江米船未至【夏口以上為上江為于偽翻】今與汝歸耳䖍通因勒兵守之至旦孟秉以甲騎迎化及【騎奇寄翻下同】化及戰栗不能言人有來謁之者但俛首據鞍稱罪過【罪過今世俗謙謝之辭俛音免】化及至城門【宮城門也】德戡迎謁引入朝堂號爲丞相【朝直遥翻下同號戶刀翻呼也】裴䖍通謂帝曰百官悉在朝堂陛下須親出慰勞【勞力到翻】進其從騎逼帝乘之【從才用翻】帝嫌其鞍勒弊更易新者乃乘之䖍通執轡挾刀出宮門賊徒喜譟動地化及揚言曰何用持此物出亟還與手【與手魏齊間人率有是言言與之毒手而殺之也宋孝建初薛安都助順有大功從弟道生亦以軍功為大司馬參軍犯罪為秣陵令薛淑之所鞭安都大怒乘馬執稍從數十人欲往殺淑之行至朱雀航逢柳元景問何之安都曰薛淑之鞭我從弟指往刺殺之元景曰小子無宜適卿往與手甚快安都既囘馬元景復呼使入車讓止之此與手之徵也亟紀力翻】帝問世基何在賊黨馬文舉曰已梟首矣於是引帝還至寢殿䖍通德戡等拔白刃侍立帝歎曰我何罪至此文舉曰陛下違棄宗廟巡遊不息外勤征討内極奢淫使丁壯盡於矢刃女弱填於溝壑四民喪業【喪息浪翻】盜賊蠭起專任佞諛飾非拒諫何謂無罪帝曰我實負百姓至於爾輩榮祿兼極何乃如是今日之事孰為首邪【邪音耶】德戡曰溥天同怨何止一人化及又使封德彝數帝罪【數所具翻又所主翻】帝曰卿乃士人何為亦爾德彞赧然而退帝愛子趙王杲年十二在帝側號慟不已䖍通斬之血濺御服【赧奴板翻慙而面赤也號戶刀翻濺子賤翻】賊欲弑帝帝曰天子死自有法何得加以鋒刃取鴆酒來文舉等不許使令狐行達頓帝令坐帝自解練巾授行達縊殺之【令力丁翻縊於賜翻又於計翻絞也 考異曰蒲山公傳河洛記皆云于洪達縊帝今從隋書及略記】初帝自知必及於難常以甖貯毒藥自隨【難乃旦翻甖於耕翻貯丁呂翻】謂所幸諸姬曰若賊至汝曹當先飲之然後我飲及亂顧索藥【索山客翻】左右皆逃散竟不能得蕭后與宮人撤漆牀板為小棺與趙王杲同殯於西院流珠堂帝每巡幸常以蜀王秀自隨囚於驍果營化及弑帝欲奉秀立之衆議不可乃殺秀及其七男又殺齊王及其二子并燕王倓隋氏宗室外戚無少長皆死【古限翻驍堅堯翻少詩照翻長知兩翻】唯秦王浩素與智及往來且以計全之【浩秦王俊之子】齊王素失愛於帝【失愛事始一百八十一卷大業四年】恒相猜忌帝聞亂顧蕭后曰得非阿孩邪【小字阿孩恒戶登翻】化及使人就第誅謂帝使收之曰詔使且緩兒【使疏吏翻】兒不負國家賊曳至街中斬之竟不知殺者為誰父子至死不相明又殺内史侍郎虞世基御史大夫裴藴左翊衛大將軍來護兒秘書監袁充右翊衛將軍宇文協千牛宇文皛【皛戶了翻】梁公蕭鉅等及其子鉅琮之弟子也【蕭琮故梁主琮藏宗翻】難將作江陽長張惠紹馳告裴藴【江陽縣帶江都郡舊廣陵也大業初更名長知兩翻】與惠紹謀矯詔發郭下兵收化及等扣門援帝【與上更有藴字文意乃明】議定遣報虞世基告反者不實抑而不許須臾難作【難乃旦翻】藴歎曰謀及播郎竟誤人事【播郎虞世基小字】虞世基宗人伋謂世基子符璽郎熙曰事勢已然吾將濟卿南度同死何益熙曰棄父背君【璽斯氏翻背蒲妹翻】求生何地感尊之懷【尊謂伋也】自此決矣世基弟世南抱世基號泣請代化及不許【號戶刀翻】黄門侍郎裴矩知必將有亂雖厮役皆厚遇之【厮音斯今人讀如瑟】又建策為驍果娶婦【事見上卷上年為于偽翻】及亂作賊皆曰非裴黄門之罪既而化及至矩迎拜馬首故得免化及以蘇威不預朝政亦免之【朝直遥翻】威名位素重往參化及化及集衆而見之曲加殊禮百官悉詣朝堂賀給事郎許善心獨不至許弘仁馳告之曰天子已崩宇文將軍攝政闔朝文武咸集【朝直遥翻】天道人事自有代終何預於叔而低回若此善心怒不肯行弘仁反走上馬泣而去化及遣人就家擒至朝堂既而釋之善心不舞蹈而出化及怒曰此人大負氣復命擒還殺之其母范氏年九十二撫柩不哭曰能死國難吾有子矣因卧不食十餘日而卒【上時掌翻復扶又翻柩音舊難乃旦翻卒子恤翻】唐王之入關也張季珣之弟仲琰為上洛令【張季珣死節見上卷上年上洛縣隋帶上洛郡】帥吏民拒守部下殺之以降【帥讀曰率降戶江翻】宇文化及之亂仲琰弟琮為千牛左右【隋制領左右府有千牛左右司射左右】化及殺之兄弟三人皆死國難時人愧之化及自稱大丞相總百揆以皇后令立秦王浩為帝居别宫令發詔畫敕書而已仍以兵監守之【令力丁翻監古銜翻】化及以弟智及為左僕射士及為内史令裴矩為右僕射 乙卯徙秦公世民為趙公 戊辰隋恭帝詔以十郡益唐國仍以唐王為相國總百揆唐國置丞相以下官又加九錫王謂僚屬曰此諂諛者所為耳孤秉大政而自加寵錫可乎必若循魏晉之迹彼皆繁文偽飾欺天罔人考其實不及五覇而求名欲過三王此孤常所非笑竊亦恥之或曰歷代所行亦何可廢王曰堯舜湯武各因其時取與異道皆推其至誠以應天順人未聞夏商之末必效唐虞之禪也若使少帝有知必不肯為【少詩照翻】若其無知孤自尊而飾讓平生素心所不為也但改丞相為相國府其九錫殊禮皆歸之有司宇文化及以左武衛將軍陳稜為江都太守綜領留事【守式又翻】壬申令内外戒嚴云欲還長安皇后六宫皆依舊式為御營營前别立帳化及視事其中仗衛部伍皆擬乘輿【乘䋲證翻】奪江都人舟檝取彭城水路西歸【煬帝改徐州為彭城郡檝與楫同】以折衝郎將沈光驍勇【煬帝置折衝郎將正四品掌領驍果屬領左右府將即亮翻下同驍堅堯翻】使將給使營於禁内【既立御營以御營之内為禁内】行至顯福宫虎賁郎將麥孟才虎牙郎錢傑【煬帝制十二衛府每衛置護軍四人掌副貳將軍尋改護軍為虎賁郎將而置虎牙郎將副焉虎牙郎下當有將字】與光謀曰吾儕受先帝厚恩今俛首事讎受其驅帥【儕士皆翻俛音免帥讀曰率不同】何面目視息世間哉吾必欲殺之死無所恨光泣曰是所望於將軍也孟才乃糾合恩舊【恩舊與之有舊恩者】帥所將數千人期以晨起將發時襲化及語洩【帥讀曰率洩息列翻】化及夜與腹心走出營外留人告司馬德戡等使討之光聞營内諠知事覺即襲化及營空無所獲值内史侍郎元敏數而斬之【數所具翻又所主翻】德戡引兵入圍之殺光其麾下數百人皆鬬死一無降者【降戶江翻】孟才亦死孟才鐵杖之子也【鐵杖死於渡遼之役】武康沈法興【隋志武康縣屬餘杭郡劉昫曰吳分烏程餘杭二縣立永安縣晉改為永康又改為武康唐分屬湖州】世為郡著姓宗族數千家法興為吳興太守【守式又翻】聞宇文化及弑逆舉兵以討化及為名 【考異曰太宗實録舊唐帝紀二月法興㨿丹楊起兵按法興起兵討化及當在弑逆後】比至烏程得精卒六萬遂攻餘杭毗陵丹陽皆下之【按烏程縣帶吳興郡沈法興既為吳興守而云舉兵至烏程者法興傳云大業末法興為吳興郡守東陽賊樓世幹略其郡煬帝詔法興與太僕丞元祐討之義寧二年江都亂法興執祐舉兵名討宇文化及三月發東陽行收兵趨江都下餘杭北至烏程衆六萬如此則自東陽至烏程也比必寐翻】據江表十餘郡自稱江南道大總管承制置百官 陳國公竇抗唐王之妃兄也煬帝使行長城於靈武【行循行也音下孟翻】聞唐王定關中癸酉帥靈武鹽川等數郡來降【煬帝改靈州為靈武郡鹽州為鹽川郡帥讀曰率】 夏四月稽胡寇富平【隋志富平縣屬京兆郡杜佑曰稽胡一名步落稽盖匈奴别種自離石以西安定以東方七八百里】將軍王師仁擊破之又五萬餘人寇宜春相國府諮議參軍竇軌將兵討之戰於黄欽山【宜春當作宜君隋志宜君縣屬京兆郡有清水水經注清水出雲陽縣之石門山東南流逕黄嶔山西將即亮翻下同】稽胡乘高縱火官軍小却軌斬其部將十四人拔隊中小校代之勒兵復戰【校戶教翻復扶又翻】軌自將數百騎居軍後令之曰聞鼓聲有不進者自後斬之既而鼓之將士争先赴敵稽胡射之不能止【騎奇寄翻射而亦翻】遂大破之虜男女二萬口 世子建成等至東都軍於芳華苑東都閉門不出遣人招諭不應李密出軍争之小戰各引去城中多欲為内應者趙公世民曰吾新定關中根本未固雖得東都不能守也遂不受戊寅引軍還【還從宣翻又音如字】世民曰城中見吾退必來追躡乃設三伏於三王陵以待之【水經注三王陵在河南縣西南柏亭東北三王或言周景王悼王定王也崔浩曰定當為敬子朝作亂西周政弱人荒悼敬二王與景王俱葬於此故世以三王名陵】段逹果將萬餘人追之遇伏而敗世民逐北抵其城下【東都城下】斬四千餘級遂置新安宜陽二郡【新安後周置中州及東垣縣隋廢州改縣名宜陽後魏置郡隋開皇初廢為縣與新安皆屬河南郡今並置郡】使行軍總管史萬寶盛彦師鎮宜陽【姓苑後漢西羌傳有北海大守盛苞其先姓奭避元帝諱改姓盛余按戰國策秦有盛橋則盛姓尚矣】呂紹宗任瓌將兵鎮新安而還【任音壬瓌古回翻置二郡内以蔽關輔外以圖東都還從宣翻又如字】 初五原通守櫟陽張長遜【煬帝改豐州為五原郡新唐書張五遜京兆櫟陽人隋志京兆無櫟陽縣櫟音藥守式又翻】以中原大亂舉郡附突厥突厥以為割利特勒郝瑗說薛舉【厥九勿翻郝呼各翻瑗子眷翻說輸芮翻下同】與梁師都及突厥連兵以取長安舉從之時啓民可汗之子咄苾【可從刋入聲汗音寒咄當没翻苾毗必翻】號莫賀咄設建牙直五原之北舉遣使與莫賀咄設謀入寇【咄當没翻使疏吏翻下同】莫賀咄設許之唐五使都水監宇文歆賂莫賀咄設【開皇初立都水臺置使者大業改為都水監改使者為監】且為陳利害止其出兵又說莫賀咄設遣張長遜入朝以五原之地歸之中國莫賀咄設並從之【為于偽翻咄當没翻朝直遥翻下同】己卯武都宕渠五原等郡皆降【武都漢郡西魏置武州煬帝復為郡宕渠漢縣梁置渠州煬帝改為宕渠郡此二郡與五原同日來降故連書之宕徒浪翻】王即以長遜為五原太守【守式又翻】長遜又詐為詔書與莫賀咄設示知其謀莫賀咄設乃拒舉師都等不納其使【使疏吏翻】 戊戌世子建成等還長安 【考異曰創業注在三月今從太宗實録】 東都號令不出四門人無固志朝議郎段世宏等謀應西師會西師已還【西師謂建成等之師還從宣翻又音如字】乃遣人招李密期以己亥夜納之事覺越王命王世充討誅之密聞城中已定乃還 宇文化及擁衆十餘萬據有六宫自奉養一如煬帝每於帳中南面坐人有白事者嘿然不對下牙方取啓狀與唐奉義牛方裕薛世良張愷等參决之【劉馮事始曰兵書曰牙旗者將軍之精凡始建牙必以制日制者其辰在五行以上剋下之日也又尚書曰門旗二口色紅八幅大將牙門之旗出引將軍前列又黄帝出軍决曰牙旗者將軍之精金枝者將軍之氣周禮司常職云軍旅會同置旌門夫以旌為門即旗門也後世軍中遂置牙門將又有牙兵典總此名者以押牙為名至於官府早晚軍吏兩謁亦名為衙呼謂既熟雖天子正殿受朝謁亦名正衙】以少主浩付尚書省令衛士十餘人守之【少始照翻】遣令史取其畫敕【隋門下省有録事通事令史各六人】百官不復朝參至彭城水路不通復奪民車牛得二千兩【復扶又翻朝直遥翻兩力讓翻】並載宫人珍寶其戈甲戎器悉令軍士負之道遠疲劇軍士始怨司馬德戡竊謂趙行樞曰君大謬誤我【行樞建言以化及為主】當今撥亂必籍英賢化及庸暗羣小在側事將必敗若之何行樞曰在我等耳廢之何難初化及既得政賜司馬德戡爵温國公加光禄大夫以其專統驍果心忌之後數日化及署諸將分部士卒【驍堅堯翻將即亮翻】以德戡為禮部尚書外示美遷實奪其兵柄德戡由是憤怨所獲賞賜皆以賂智及智及為之言乃使之將後軍萬餘人以從【及為于偽翻將即亮翻從才用翻】於是德戡行樞與諸將李本尹正卿宇文導師等謀以後軍襲殺化及更立德戡為主遣人詣孟海公結為外助【孟海公據曹州】遷延未發待海公報許宏仁張愷知之以告化及化及遣宇文士及陽為遊獵至後軍德戡不知事露出營迎謁因執之化及讓之曰與公戮力共定海内出於萬死今始事成方願共守富貴公又何反也德戡曰本殺昏主苦其淫虐推立足下而又甚之逼於物情不得已也化及縊殺之并殺其支黨十餘人孟海公畏化及之彊帥衆具牛酒迎之【帥讀曰率下同】李密據鞏洛以拒化及【洛水至鞏入河故曰鞏洛】化及不得西引兵向東郡東郡通守王軌以城降之【守式又翻降戶江翻下同】辛丑李密將井陘王君廊帥衆來降【隋志井陘縣屬恒山郡將即亮翻下同陘音刑帥讀曰率 考異曰太宗實録曰王君愕邯鄲人君廊寇畧邯鄲君愕往投之因謂君廊陳井陘之險勸先往據之君廊從其言屯井陘山歲餘會義師入定關中乃與君廊率所部萬餘人歸順拜大將軍與君廊事皆出太宗實録而不同如此今據高祖實録稱李密將王君廊降從君廊傳】君廊本羣盗有衆數千人與賊帥韋寶鄧豹合軍虞鄉【劉昫曰虞鄉縣漢解縣地後魏分置虞鄉縣隋志屬河東郡帥所類翻】唐王與李密俱遣使招之【使疏吏翻】寶豹欲從唐王君廊偽與之同乘其無備襲撃破之奪其輜重【重直用翻】奔李密密不禮之復來降【復扶又翻】拜上柱國假河内太守 蕭銑即皇帝位置百官凖梁室故事諡其從父琮為孝靖皇帝【謚神至翻從才用翻】祖巖為河間忠烈王父璿為文憲王【璿音旋】封董景珍等功臣七人皆為王遣宋王楊道生撃南郡下之徙都江陵【煬帝改荆州為南郡江陵帶郡】修復園廟引岑文本為中書侍郎使典文翰委以機密又使魯王張繡徇嶺南隋將張鎮周王仁壽等拒之既而聞煬帝遇弑皆降於銑欽州刺史甯長真亦以鬱林始安之地附於銑【煬帝改欽州為寜越郡長真刺史文帝所命也鬰林郡梁定州也後改為南定州平陳改為尹州大業初改為鬰州尋改為郡又改桂州為始安郡】漢陽太守馮盎以蒼梧高凉珠崖番禺之地附於林士宏【新唐書馮盎傳曰隋仁壽初盎平潮成叛獠拜漢陽太守隋亡奔還嶺表據有諸郡蒼梧郡梁置成州開皇初改為封州煬帝改為郡改高州為高凉郡崖州為珠崖郡番禺南海郡治番音潘禺音愚】銑士宏各遣人招交趾太守丘和【煬帝改交州為交趾郡】和不從銑遣甯長眞帥嶺南之兵自海道攻和【帥讀曰率下同】和欲出迎之司法書佐高士亷【煬帝改郡諸司參軍為書佐】說和曰長真兵數雖多懸軍遠至不能持久城中勝兵足以當之【說輸芮翻下並同勝音升】奈何望風受制於人和從之以士廉為軍司馬將水陸諸軍逆擊破之【將即亮翻】長真僅以身免盡俘其衆既而有驍果自江都至【驍堅堯翻】得煬帝凶問亦以郡附於銑士廉勱之子也【高勱北齊清河王岳之子勱音邁】始安郡丞李襲志遷哲之孫也【李遷哲見一百七十七卷太建二年】隋末散家財募士得三千人以保郡城蕭銑林士宏曹武徹迭來攻之皆不克聞煬帝遇弑帥吏民臨三日【臨力鴆翻】或說襲志曰公中州貴族【按李襲志之先隴西狄道人後為金州安康人此必出其家傳以門地自高耳】久臨鄙郡華夷悦服今隋室無主海内鼎沸以公威惠號令嶺表尉佗之業可坐致也【尉佗事見漢高帝紀佗徒何翻】襲志怒曰吾世繼忠貞今江都雖覆宗社尚存尉佗狂僭何足慕也欲斬說者衆乃不敢言堅守二年外無聲援城䧟為銑所虜銑以為工部尚書檢校桂州總管於是東自九江【煬帝改江州為九江郡】西抵三峽南盡交趾北距漢川【此漢川謂漢水以南之地非漢中之漢川郡】銑皆有之勝兵四十餘萬【勝音升】 煬帝凶問至長安唐王哭之慟曰吾北面事人失道不能救敢忘哀乎 五月山南撫慰使馬元規撃朱粲於冠軍破之【冠軍縣屬南陽郡使疏吏翻冠古玩翻】 王德仁既殺房彦藻【事見上二月】李密遣徐世勣討之德仁兵敗甲寅與武安通守袁子幹皆來降詔以德仁為鄴郡太守【煬帝改洺州為武安郡相州為魏郡此又改魏郡為鄴郡也守式又翻】 戊午隋㳟帝禪位於唐遜居代邸【隋開皇元年受禪歲在辛丑三主三十八年而亡 考異曰創業注此詔在四月今從實録】甲子唐王即皇帝位於太極殿【即隋大興殿也唐既受禪改為大極殿】遣刑部尚書蕭造告天於南郊大赦改元【改元武德】罷郡置州以太守為刺史【大業三年改州為郡】推五運為土德色尚黄 隋煬帝凶問至東都戊辰留守官奉越王即皇帝位【越王侗亦元德太子昭子】大赦改元皇泰是時於朝堂宣旨以時鐘金革【朝直遥翻說文曰鐘當也】公私皆即日大祥【大祥而禫】追諡大行曰明皇帝廟號世祖追尊元德太子曰成皇帝廟號世宗尊母劉良娣為皇太后以段達為納言陳國公王世充為納言鄭國公【隋制門下省納言二人】元文都為内史令魯國公皇甫無逸為兵部尚書杞國公又以盧楚為内史令【隋初内史省置監令各一人尋廢監置令二人】郭文懿為内史侍郎趙長文為黄門侍郎共掌朝政【長知兩翻朝直遥翻】時人號七貴皇泰主眉目如畫温厚仁愛風格儼然 辛未突厥始畢可汗遣骨咄禄特勒來【厥九勿翻可從刋入聲汗音寒突厥官子弟曰特勒咄當没翻】宴之於太極殿奏九部樂【杜佑曰武德初因隋舊制九部樂一宴樂二清商三西凉四扶南五高麗六龜兹七安國八踈勒九康國前一百八十一卷隋大業四年引杜佑註九部樂與此不同乂考宋祁新唐志唐有十部樂有十四國技以八國入十部而不明指八國為何國此亦異同而難考者也】時中國人避亂者多入突厥突厥彊盛東自契丹室韋西盡吐谷渾高昌諸國皆臣之【契欺訖翻又音喫吐從暾入聲谷音浴】控弦百餘萬帝以初起資其兵馬前後餉遺不可勝紀【遺于季翻勝音升】突厥恃功驕倨每遣使者至長安多暴横【使疏吏翻横戶孟翻】帝優容之 壬申命裴寂劉文静等修定律令置國子太學四門生合三百餘員【唐六典國子生文武官三品已上及國公子孫從二品已上曾孫太學生文武官五品已上及郡縣公子孫從三品曾孫四門生文武官七品已上及侯伯子男子若庶人子為後士生者後魏劉芳表云太和二十年立四門博士於四門置學按禮記云天子設四學鄭玄注周四郊之虞庠也今以其遼遠故置於四門請移與太學同處從之】郡縣學亦各置生員六月甲戌朔以趙公世民為尚書令黄臺公瑗為刑部侍郎【黄臺縣公東魏置黄臺縣於潁川大業初廢瑗子眷翻】相國府長史裴寂為右僕射知政事司馬劉文静為納言司録竇威為内史令李綱為禮部尚書參掌選事【長知兩翻選宣絹翻】掾殷開山為吏部侍郎【掾于絹翻】屬趙慈景為兵部侍郎韋義節為禮部侍郎主簿陳叔逹博陵崔民幹並為黄門侍郎【煬帝改定州為博陵郡】唐儉為内史侍郎録事參軍裴晞為尚書左丞以隋民部尚書蕭瑀為内史令禮部尚書竇璡為戶部尚書【按六典貞觀二十三年避大宗諱始改民部尚書為戶部尚書史家以後來官名書之也瑀音禹進則鄰翻】蔣公屈突通為兵部尚書【屈居勿翻】長安令獨孤懷恩為工部尚書瑗上之從子懷恩舅子也【從木用翻】上待裴寂特厚羣臣無與為比賞賜服玩不可勝紀【勝音升】命尚書奉御日以御膳賜寂【尚書當作尚食六典尚食奉御二人屬殿中省】視朝必引與同坐入閤則延之卧内言無不從稱為裴監而不名【寂仕隋為晉陽宮監親之以舊官稱之朝直遥翻】委蕭瑀以庶政事無大小無不關掌瑀亦孜孜盡力繩違舉過人皆憚之毁之者衆終不自理上嘗有敕而内史不時宣行【隋唐之制凡王言下内史省皆宣署申覆而施行之】上責其遲瑀對曰大業之世内史宣敕或前後相違有司不知所從其易在前其難在後臣在省日久【瑀在隋朝為内史侍郎故云然易以豉翻】備見其事今王業經始事繫安危遠方有疑恐失機會故臣每受一敕必勘審使與前敕不違始敢宣行稽緩之愆實由於此上曰卿用心如是吾復何憂【復扶又翻】 初帝遣馬元規慰撫山南南陽郡丞河東呂子臧獨據郡不從【是年二月遣馬元規】元規遣使數輩諭之【使疏吏翻】皆為子臧所殺及煬帝遇弑子臧發喪成禮然後請降拜鄧州刺史【南陽郡復為鄧州降戶江翻下同】封南郡公 廢大業律令頒新格 上每視事自稱名引貴臣同榻而坐劉文靜諫曰昔王導有言若太陽俯同萬物使羣生何以仰照【事見九十卷晉元帝太興元年】今貴賤失位非常久之道上曰昔漢光武與嚴子陵共寢子陵加足於帝腹【事見後漢書嚴光傳】今諸公皆名德舊齒平生親友宿昔之歡何可忘也公勿以為嫌 戊寅隋安陽令呂珉以相州來降【隋安陽縣帶相州相息亮翻】以為相州刺史 己卯祔四親廟主追尊皇高祖瀛州府君曰宣簡公皇曾祖司空曰懿王皇祖景王曰景皇帝廟號太祖祖妣曰景烈皇后皇考元王曰元皇帝廟號世祖妣獨孤氏曰元貞皇后追諡妃竇氏曰穆皇后【瀛州府君熙司空天賜景王亮元王昞】每歲祀昊天上帝皇地祗神州地祗以景帝配【神州地祗神州迎州冀州戎州拾州柱州營州咸州陽州九州之祇也祗翹移翻】感生帝明堂以元帝配【古者帝王之興必感五精之氣以生隋以火德王祀赤熛怒為感帝唐以土德王祀含樞紐為感帝】庚辰立世子建成為皇太子趙公世民為秦王齊公元吉為齊王宗室黄瓜公白駒為平原王【案白駒盖先封黄瓜縣公黄瓜縣盖拓拔魏所置在上邽界水經注黄瓜水發源黄瓜谷西東流逕黄瓜縣北又東北歸於藉水藉水既與黄瓜水合又東逕上邽城南】蜀公孝基為永安王柱國道玄為淮陽王長平公叔良為長平王鄭公神通為永康王安吉公神符為襄邑王柱國德良為新興王上柱國博义為隴西王上柱國奉慈為勃海王孝基叔良神符德良帝之從父弟博义奉慈弟子道玄從父兄子也【從才用翻】 癸未薛舉寇涇州【復以安定郡為涇州後魏置涇州治高平因涇水為名】以秦王世民為元帥將八總管兵以拒之【帥所類翻將即亮翻】 遣太僕卿宇文明逹招慰山東以永安王孝基為陜州總管【義寧初以河南陜縣為宏農郡今為陜州陜失冉翻】時天下未定凡邊要之州皆置總管府以統數州之兵 乙酉奉隋帝為酅國公【酅戶圭翻】詔曰近世以來時運遷革前代親族莫不誅夷興亡之效豈伊人力其隋蔡王智積等子孫並付所司量才選用【量音良】 東都聞宇文化及西來上下震懼有盖琮者【盖古盍翻姓也】上疏請說李密與之合勢拒化及【說輸芮翻】元文都謂盧楚等曰今讐恥未雪而兵力不足若赦密罪使撃化及兩賊自闘吾徐承其弊化及既破密兵亦疲又其將士利吾官賞易可離間【易以豉翻間占莧翻】并密亦可擒也楚等皆以為然即以琮為通直散騎常侍【散悉亶翻騎奇寄翻】齎敕書賜密 丙申隋信都郡丞東萊麴稜來降拜冀州刺史【隋信都郡入唐為冀州東萊郡為萊州宋白曰萊州古萊夷地春秋萊子之國齊滅萊以在國之東故曰東萊降戶江翻】 萬年縣法曹武城孫伏伽上表【周明帝二年分長安為萬年縣與長安並居京城隋改為大興縣唐受禪復為萬年與長安並為赤縣萬年縣治宣揚坊領朱雀街東五十四坊長安縣治長壽坊領街西五十四坊隋煬帝改縣尉為縣正尋改正為戶曹法曹分司以丞郡之六司唐復為縣尉而六司各置佐史孫㐲伽萬年法曹盖隋官也武城縣漢之東武城也唐志屬貝州伽求加翻】以為隋以惡聞其過亡天下陛下龍飛晉陽遠近響應未㫷年而登帝位徒知得之之易不知隋失之之不難也【惡烏路翻易以豉翻下同】臣謂宜易其覆轍務盡下情凡人君言動不可不慎竊見陛下今日即位而明日有獻鷂雛者此乃少年之事豈聖主所須哉【鷂戈召翻少詩詔翻】又百戲散樂亡國淫聲【百戲散樂齊周隋所以亡國散悉亶翻】近太常於民間借婦女裙襦五百餘襲【襦人朱翻】以充妓衣【妓渠綺翻】擬五月五日玄武門遊戲此亦非所以為子孫法也凡如此類悉宜廢罷善惡之習朝夕漸染【漸子廉翻】易以移人皇太子諸王參僚左右宜謹擇其人其有門風不能雍穆為人素無行義專好奢靡以聲色遊獵為事者皆不可使之親近也【行下孟翻好呼到翻近其靳翻】自古及今骨肉乖離以至敗國亡家未有不因左右離間而然也【間古莧翻】願陛下慎之上省表大悦【省悉景翻】下詔褒稱擢為治書侍御史【冶直之翻】賜帛三百匹仍頒示遠近 辛丑内史令延安靖公竇威薨以將作大匠竇抗兼納言黄門侍郎陳叔逹判納言【兼判皆非正官】 宇文化及留輜重於滑臺【滑臺滑州治所重直用翻】以王軌為刑部尚書使守之引兵北趣黎陽【趣七喻翻又逡須翻】李密將徐世勣據黎陽畏其軍鋒以兵西保倉城【將即亮翻】化及度河保黎陽分兵圍世勣密帥步騎二萬壁於清淇【汲郡之衛縣古朝歌也隋開皇十六年分置清淇縣大業初廢入衛縣李密盖壁于故縣也】與世勣以烽火相應深溝高壘不與化及戰化及每攻倉城密輒引兵以掎其後【掎居豈翻】密與化及隔水而語【隔淇水也】密數之曰卿本匈奴皂隸破野頭耳【隋書宇文述傳本姓破野頭役屬鮮卑侯豆歸從其主為宇文氏數所具翻】父兄子弟並受隋恩富貴累世舉朝莫二【朝置遥翻】主上失德不能死諫反行弑逆欲規簒奪不追諸葛瞻之忠誠【諸亮瞻亮之子蜀之亡也瞻死之】乃為霍禹之惡逆【霍禹光之子漢宣親政禹謀為大逆遂以滅族】天地所不容將欲何之若速來歸我尚可得全後嗣化及默然俯視良久瞋目大言曰與爾論相殺事何須作書語邪【嗣祥吏翻瞋昌眞翻邪音耶】密謂從者曰【從才用翻】化及庸愚如此忽欲圖為帝王吾當折杖驅之耳化及盛修攻具以逼倉城世勣於城外掘深溝以固守化及阻塹不得至城下世勣於塹中為地道出兵撃之【塹七艶翻】化及大敗焚其攻具時密與東都相持日久又東拒化及常畏東都議其後見盖琮至大喜遂上表乞降【盖古盍翻上時掌翻降戶江翻】請討滅化及以贖罪送所獲雄武郎將于洪建【煬帝募驍果置左右雄武府雄武郎將以領之也將即亮翻】遣元帥府記室參軍李儉上開府徐師譽等入見皇泰主命戮洪建於左掖門外如斛斯政之法【戮斛斯政見一百八十二卷大業十年帥所類翻下同見賢遍翻下同】元文都等以密降為誠實盛飾賓館於宣仁門東【六典東都東城在皇城之東東曰宣仁門】皇泰主引見儉等以儉為司農卿師譽為尚書右丞使具導從列鐃吹【從才用翻鐃女交翻似鈴無舌吹昌瑞翻】還館玉帛酒饌中使相望【饌雛戀翻又士免翻】冊拜密太尉尚書令東南道大行臺行軍元帥魏國公令先平化及然後入朝輔政【朝直遥翻】以徐世勣為右武侯大將軍仍下詔稱密忠欵且曰其用兵機略一禀魏公節度元文都喜於和解謂天下可定於上東門置酒作樂【六典東都城東面三門中曰建春南曰永通北曰上東】自段逹已下皆起舞王世充作色謂起居侍郎崔長文曰【六典起居郎因起居注以為名起居注者記録人君動止之事漢獻帝及西晉以後諸帝皆有起居注皆史官所録自隋置為職員列為侍臣專掌其事每季為卷送付史官按隋志煬帝減内史舍人員加置起居舍人員然未有侍郎起居侍郎始見於此長知兩翻】朝廷官爵乃以與賊其志欲何為邪【邪音耶】文都等亦疑世充欲以城應化及由是有隙然猶外相彌縫陽為親善秋七月皇泰主遣大理卿張權鴻臚卿崔善福賜李密書曰今日以前咸共刷蕩【臚如陵翻】使至以後【使疏吏翻】彼此通懷七政之重佇公匡弼【日月五星謂之七政佇待也】九伐之利委公指揮【周官大司馬以九伐之法正邦國】權等既至密北面拜受詔書既無西慮【密軍在鞏洛東都城在西】悉以精兵東撃化及密知化及軍糧且盡因偽與和化及大喜恣其兵食冀密饋之會密下有人獲罪亡扺化及具言其情化及大怒其食又盡乃度永濟渠與密戰于童山之下【隋志汲郡衛縣有同山】自辰逹酉密為流矢所中墮馬悶絶左右奔散追兵且至唯秦叔寶獨捍衛之密由是獲免叔寶復收兵與之力戰化及乃退【中竹仲翻復扶又翻】化及入汲郡求軍糧又遣使考掠東郡吏民以責米粟【使疏吏翻拷音考掠音亮】王軌等不堪其弊遣通事舍人許敬宗詣密請降【隋内書省有通事舍人十六人降戶江翻】以軌為滑州總管【改東郡為滑州滑州治白馬春秋衛之漕邑宋魏兵争以滑臺為重鎮隋開皇三年置滑州取滑臺為名也】以敬宗為元帥府記室與魏徵共掌文翰敬宗善心之子也【善心死於江都之難】房公蘇威在東郡隨衆降密密以其隋氏大臣虛心禮之威見密初不言帝室艱危唯再三舞蹈稱不圖今日復覩聖明時人鄙之化及聞王軌叛大懼自汲郡引兵欲取以北諸郡其將陳智畧帥嶺南驍果萬餘人樊文超帥江淮排䂎【將即亮翻帥讀曰率䂎作管翻】張童兒帥江東驍果數千人皆降於密文超子盖之子也【樊子盖事煬帝有守東都之功】化及猶有衆二萬北趣魏縣【隋志魏縣屬武陽郡時李密改武陽郡為魏州趣七喻翻又□須翻】密知其無能為西還鞏洛留徐世勣以備之 乙巳宣州刺吏周超撃朱粲敗之【宣州疑當作宜州敗補邁翻下同】 丁未梁師都寇靈州【復以靈武郡為靈州宋白曰靈州漢富平縣地後魏孝昌二年置靈州】驃騎將軍藺興粲撃破之【義師初起改隋鷹揚郎將曰軍頭尋改軍頭口驃騎將軍驃匹妙翻騎奇寄翻】 突厥闕可汗遣使内附【西突厥闕度設處於會寧隋亂自稱可汗使疏吏翻可從刋入聲汗音寒】初闕可汗附於李軌隋西戎使者曹瓊據甘州誘之【西戎使者盖隋煬帝所置宋白曰西魏廢帝二年以張掖為甘州隋大業以為張掖郡唐復以張掖郡為甘州誘音酉】乃更附瓊與之拒軌為軌所敗竄於逹斗拔谷與吐谷渾相表裏【敗補邁翻吐從暾入聲谷音浴】至是内附尋為李軌所滅 薛舉進逼高墌【新志寧州定平縣有高墌城墌章恕翻】遊兵至於岐【唐復以北地郡為豳州扶風郡為岐州詳見下卷】秦王世民深溝高壘不與戰會世民得瘧疾【瘧魚約翻】委軍事於長史納言劉文靜【劉文靜以納言為秦王行軍長史長知兩翻】司馬殷開山【殷開山以吏部侍郎為行軍司馬】且戒之曰薛舉懸軍深入食少兵疲若來挑戰【少始紹翻挑徒了翻】愼勿應也俟吾疾愈為君等破之【為于偽翻】開山退謂文靜曰王慮公不能辦故有此言耳且賊聞王有疾必輕我宜曜武以威之乃陳於高墌西南【陳讀曰陣】恃衆而不設備舉濳師掩其後壬子戰於淺水原【新志州宜禄縣有淺水原】八總管皆敗士卒死者什五六大將軍慕容羅□李安遠劉宏基皆沒世民引兵還長安舉遂拔高墌收唐兵死者為京觀【觀古玩翻】文靜等皆坐除名 乙卯榆林賊帥郭子和遣使來降【帥所類翻使疏吏翻下同降戶江翻】以為靈州總管 李密每戰勝必遣使告捷於皇泰主隋人皆喜王世充獨謂其麾下曰元文都輩刀筆吏耳吾觀其勢必為李密所擒且吾軍士屢與密戰没其父兄子弟前後已多一旦為之下吾屬無類矣欲以激怒其衆文都聞之大懼與盧楚等謀因世充入朝伏甲誅之【朝直遥翻】段逹性庸懦【懦奴過翻又奴亂翻】恐其事不就遣其壻張志以楚等謀告世充戊午夜三鼓世充勒兵襲含嘉門【含嘉門盖以通含嘉城而名 考異曰河洛記初元文都欲自為御史盧楚等為宣詔王世充固執以為不可乃止文都大恨盧楚私謂文都曰王世充是外軍一將非留守逹官比者領軍屢為奔徙吾方卹外姦且從捨過翻更宰制人事跋扈縱横此而不除恐為國患文都曰未可即殺且欲當朝上奏御前縳之鏁繋於獄楚曰善文都懷奏入殿臨欲施行趙季卿私告之世充遂奔含嘉以作亂是時宫中亦遣使傳報世充為皇姨故也初世充妻蕭氏早亡後有胡氏者復在江都皇泰主乃以皇姨嫁之至是爭權遂起兵馬文都等令趙方海前後追世充世充乃托疾不受召按世充正為與文都爭李密事相誅耳恐事不因此今不取】元文都聞變入奉皇泰主御乾陽殿【乾陽殿隋東都宫正殿】陳兵自衛命諸將閉門拒守【將即亮翻】將軍跋野綱將兵出遇世充下馬降之【跋野虜複姓】將軍費曜田闍戰於門外不利【費父沸翻闍視遮翻又音都】文都自將宿衛兵欲出玄武門以襲其後【玄武門宫城北門】長秋監段瑜【煬帝大業三年改内侍省為長秋監】稱求門鑰不獲稽留遂久天且曙文都復欲引兵出太陽門逆戰還至乾陽殿世充已攻太陽門得入【按舊書王世充傳太陽門宫城東門】皇甫無逸棄母及妻子斫右掖門西奔長安【六典皇城在都城西北隅南面三門中曰端門左曰左掖右曰右掖】盧楚匿於太官署【太官署在光禄寺百僚廨署皆在皇城之内】世充之黨擒之至興教門見世充【皇宫南面三門左曰興教】世充令亂斬殺之進攻紫微宫門皇泰主使人登紫微觀【觀門闕也音古玩翻】問稱兵欲何為世充下馬謝曰元文都盧楚等横見規圖【横戶孟翻】請殺文都甘從刑典段逹乃令將軍黄桃樹執送文都文都顧謂皇泰主曰臣今朝死陛下夕至矣皇泰主慟哭遣之出興教門亂斬如盧楚并殺盧元諸子段逹又以皇泰主命開門納世充世充悉遣人代宿衛者然後入見皇泰主於乾陽殿【見賢遍翻丅同】皇泰主謂世充曰擅相誅殺曾不聞奏豈為臣之道乎公欲肆其彊力敢及我邪【邪音耶】世充拜伏流涕謝曰臣蒙先皇采拔粉骨非報文都等苞藏禍心欲召李密以危社稷疾臣違異深積猜嫌臣廹於救死不暇聞奏若内懷不臧違負陛下天地日月實所照臨使臣闔門殄滅無復遺類【復扶又翻】詞淚俱發皇泰主以為誠引令升殿與語久之因與俱入見皇太后【皇泰主之母劉良娣】世充被髪為誓稱不敢有貳心【被皮義翻】乃以世充為左僕射總督内外諸軍事比及日中【比必寐翻】捕獲趙長文郭文懿殺之【二人盖盧元之黨長知兩翻】然後廵城告諭以誅元盧之意世充自含嘉城移居尚書省漸結黨援恣行威福用兄世惲為内史令【惲於粉翻】入居禁中子弟咸典兵馬分政事為十頭悉以其黨主之勢震内外莫不趨附皇泰主拱手而已李密將入朝【朝直遥翻】至温【隋志温縣屬河内郡】聞元文都等死乃<br />
<br />
  還金墉東都大饑私錢濫惡大半雜以錫鐶【隋開皇初見用之錢皆須和以錫鑞錫鑞既賤求利者多私鑄之錢不可禁約乃詔禁出錫鑞之處不得私採立榜置様錢不中様者不入於市大業之季王綱弛紊私鑄益多錢轉薄惡初焉每千猶重二斤漸輕至一斤或剪鐵鍱裁皮糊紙以為錢相雜用之錢鐶固宜多矣鐶戶關翻】其細如線米斛直錢八九萬初李密嘗受業於儒生徐文遠文遠為皇泰主國子祭酒自出樵采為密軍所執密令文遠南面坐備弟子禮北面拜之文遠曰老夫既荷厚禮【荷下可翻】敢不盡言未審將軍之志欲為伊霍以繼絶扶傾乎則老夫雖遲暮猶願盡力若為莾卓乘危邀利則無所用老夫矣密頓首曰昨奉朝命備位上公【朝直遥翻下同】冀竭庸虛匡濟國難【難乃旦翻】此密之本志也文遠曰將軍名臣之子【李密寛之子寛為周將以驍勇著名】失塗至此若能不遠而復猶不失為忠義之臣及王世充殺元文都等密復問計於文遠【復扶又翻】文遠曰世充亦門人也其為人殘忍隘既乘此勢必有異圖將軍前計為不諧矣非破世充不可入朝也【朝直遥翻】密曰始謂先生儒者不逹時事今乃坐决大計何其明也文遠孝嗣之玄孫也【徐孝嗣相蕭齊】 庚申詔隋氏離宫遊幸之所並廢之 戊辰遣黄臺公瑗安撫山南【瑗於眷翻】 己巳以隋右武衛將軍皇甫無逸為刑部尚書 隋河間郡丞王琮守郡城以拒羣盗【琮祖宗翻】竇建德攻之歲餘不下聞煬帝凶問帥吏士發喪乘城者皆哭建德遣使弔之琮因使者請降【帥讀曰率使疏吏翻降戶江翻下同】建德退舍具饌以待之琮言及隋亡俯伏流涕建德亦為之泣【饌雛戀翻又雛晼翻為於偽翻】諸將曰琮久拒我軍殺傷甚衆力盡乃降請烹之建德曰琮忠臣也吾方賞之以勸事君奈何殺之往在高雞泊為盗容可妄殺人今欲安百姓定天下豈得害忠良乎乃狥軍中曰先與王琮有怨敢妄動者夷三族以琮為瀛州刺史【復以河間郡為瀛州宋白曰瀛州漢為河間國後漢為樂成國後魏於樂成縣立瀛州取瀛海為名】於是河北郡縣聞之争附於建德【盗亦有道豈欺我哉】先是建德䧟景城執戶曹河東張玄素將殺之【景城縣隋志屬河間郡舊曰成平開皇十八年改名張玄素為縣戶曹也先悉薦翻】縣民千餘人號泣請代其死【號戶口翻】曰戶曹清愼無比大王殺之何以勸善建德乃釋之以為治書侍御史固辭及江都敗復以為黄門侍郎玄素乃起【史言隋之故官漸就仕於他姓治直之翻復扶又翻】饒陽令宋正本【隋志饒陽縣屬河間郡】博學有才氣說建德以定河北之策【說輸芮翻】建德引為謀主建德定都樂壽【隋志樂壽縣屬河間郡舊曰樂城開皇十八年改為廣城仁壽初改今名劉昫曰後魏移縣東北近古樂縣亭因改為樂壽焉按瀛州河間郡時治樂壽宋白曰太和十一年河間郡自樂城移理於今樂壽縣西一里樂壽亭城隋開皇廢郡置瀛州大業廢州為河間郡樂音洛】命所居曰金城宫備置百官<br />
<br />
  資治通鑑卷一百八十五<br />
<br />
<史部,編年類,資治通鑑>  <br>
   </div> 

<script src="/search/ajaxskft.js"> </script>
 <div class="clear"></div>
<br>
<br>
 <!-- a.d-->

 <!--
<div class="info_share">
</div> 
-->
 <!--info_share--></div>   <!-- end info_content-->
  </div> <!-- end l-->

<div class="r">   <!--r-->



<div class="sidebar"  style="margin-bottom:2px;">

 
<div class="sidebar_title">工具类大全</div>
<div class="sidebar_info">
<strong><a href="http://www.guoxuedashi.com/lsditu/" target="_blank">历史地图</a></strong>  
<a href="http://www.880114.com/" target="_blank">英语宝典</a>  
<a href="http://www.guoxuedashi.com/13jing/" target="_blank">十三经检索</a> 
<br><strong><a href="http://www.guoxuedashi.com/gjtsjc/" target="_blank">古今图书集成</a></strong> 
<a href="http://www.guoxuedashi.com/duilian/" target="_blank">对联大全</a> <strong><a href="http://www.guoxuedashi.com/xiangxingzi/" target="_blank">象形文字典</a></strong> 

<br><a href="http://www.guoxuedashi.com/zixing/yanbian/">字形演变</a>  <strong><a href="http://www.guoxuemi.com/hafo/" target="_blank">哈佛燕京中文善本特藏</a></strong>
<br><strong><a href="http://www.guoxuedashi.com/csfz/" target="_blank">丛书&方志检索器</a></strong> <a href="http://www.guoxuedashi.com/yqjyy/" target="_blank">一切经音义</a>  

<br><strong><a href="http://www.guoxuedashi.com/jiapu/" target="_blank">家谱族谱查询</a></strong>  <strong><a href="http://shufa.guoxuedashi.com/sfzitie/" target="_blank">书法字帖欣赏</a></strong> 
<br>

</div>
</div>


<div class="sidebar" style="margin-bottom:0px;">

<font style="font-size:22px;line-height:32px">QQ交流群9:489193090</font>


<div class="sidebar_title">手机APP 扫描或点击</div>
<div class="sidebar_info">
<table>
<tr>
	<td width=160><a href="http://m.guoxuedashi.com/app/" target="_blank"><img src="/img/gxds-sj.png" width="140"  border="0" alt="国学大师手机版"></a></td>
	<td>
<a href="http://www.guoxuedashi.com/download/" target="_blank">app软件下载专区</a><br>
<a href="http://www.guoxuedashi.com/download/gxds.php" target="_blank">《国学大师》下载</a><br>
<a href="http://www.guoxuedashi.com/download/kxzd.php" target="_blank">《汉字宝典》下载</a><br>
<a href="http://www.guoxuedashi.com/download/scqbd.php" target="_blank">《诗词曲宝典》下载</a><br>
<a href="http://www.guoxuedashi.com/SiKuQuanShu/skqs.php" target="_blank">《四库全书》下载</a><br>
</td>
</tr>
</table>

</div>
</div>


<div class="sidebar2">
<center>


</center>
</div>

<div class="sidebar"  style="margin-bottom:2px;">
<div class="sidebar_title">网站使用教程</div>
<div class="sidebar_info">
<a href="http://www.guoxuedashi.com/help/gjsearch.php" target="_blank">如何在国学大师网下载古籍?</a><br>
<a href="http://www.guoxuedashi.com/zidian/bujian/bjjc.php" target="_blank">如何使用部件查字法快速查字?</a><br>
<a href="http://www.guoxuedashi.com/search/sjc.php" target="_blank">如何在指定的书籍中全文检索?</a><br>
<a href="http://www.guoxuedashi.com/search/skjc.php" target="_blank">如何找到一句话在《四库全书》哪一页?</a><br>
</div>
</div>


<div class="sidebar">
<div class="sidebar_title">热门书籍</div>
<div class="sidebar_info">
<a href="/so.php?sokey=%E8%B5%84%E6%B2%BB%E9%80%9A%E9%89%B4&kt=1">资治通鉴</a> <a href="/24shi/"><strong>二十四史</strong></a>&nbsp; <a href="/a2694/">野史</a>&nbsp; <a href="/SiKuQuanShu/"><strong>四库全书</strong></a>&nbsp;<a href="http://www.guoxuedashi.com/SiKuQuanShu/fanti/">繁体</a>
<br><a href="/so.php?sokey=%E7%BA%A2%E6%A5%BC%E6%A2%A6&kt=1">红楼梦</a> <a href="/a/1858x/">三国演义</a> <a href="/a/1038k/">水浒传</a> <a href="/a/1046t/">西游记</a> <a href="/a/1914o/">封神演义</a>
<br>
<a href="http://www.guoxuedashi.com/so.php?sokeygx=%E4%B8%87%E6%9C%89%E6%96%87%E5%BA%93&submit=&kt=1">万有文库</a> <a href="/a/780t/">古文观止</a> <a href="/a/1024l/">文心雕龙</a> <a href="/a/1704n/">全唐诗</a> <a href="/a/1705h/">全宋词</a>
<br><a href="http://www.guoxuedashi.com/so.php?sokeygx=%E7%99%BE%E8%A1%B2%E6%9C%AC%E4%BA%8C%E5%8D%81%E5%9B%9B%E5%8F%B2&submit=&kt=1"><strong>百衲本二十四史</strong></a>  <a href="http://www.guoxuedashi.com/so.php?sokeygx=%E5%8F%A4%E4%BB%8A%E5%9B%BE%E4%B9%A6%E9%9B%86%E6%88%90&submit=&kt=1"><strong>古今图书集成</strong></a>
<br>

<a href="http://www.guoxuedashi.com/so.php?sokeygx=%E4%B8%9B%E4%B9%A6%E9%9B%86%E6%88%90&submit=&kt=1">丛书集成</a> 
<a href="http://www.guoxuedashi.com/so.php?sokeygx=%E5%9B%9B%E9%83%A8%E4%B8%9B%E5%88%8A&submit=&kt=1"><strong>四部丛刊</strong></a>  
<a href="http://www.guoxuedashi.com/so.php?sokeygx=%E8%AF%B4%E6%96%87%E8%A7%A3%E5%AD%97&submit=&kt=1">說文解字</a> <a href="http://www.guoxuedashi.com/so.php?sokeygx=%E5%85%A8%E4%B8%8A%E5%8F%A4&submit=&kt=1">三国六朝文</a>
<br><a href="http://www.guoxuedashi.com/so.php?sokeytm=%E6%97%A5%E6%9C%AC%E5%86%85%E9%98%81%E6%96%87%E5%BA%93&submit=&kt=1"><strong>日本内阁文库</strong></a> <a href="http://www.guoxuedashi.com/so.php?sokeytm=%E5%9B%BD%E5%9B%BE%E6%96%B9%E5%BF%97%E5%90%88%E9%9B%86&ka=100&submit=">国图方志合集</a> <a href="http://www.guoxuedashi.com/so.php?sokeytm=%E5%90%84%E5%9C%B0%E6%96%B9%E5%BF%97&submit=&kt=1"><strong>各地方志</strong></a>

</div>
</div>


<div class="sidebar2">
<center>

</center>
</div>
<div class="sidebar greenbar">
<div class="sidebar_title green">四库全书</div>
<div class="sidebar_info">

《四库全书》是中国古代最大的丛书,编撰于乾隆年间,由纪昀等360多位高官、学者编撰,3800多人抄写,费时十三年编成。丛书分经、史、子、集四部,故名四库。共有3500多种书,7.9万卷,3.6万册,约8亿字,基本上囊括了古代所有图书,故称“全书”。<a href="http://www.guoxuedashi.com/SiKuQuanShu/">详细>>
</a>

</div> 
</div>

</div>  <!--end r-->

</div>
<!-- 内容区END --> 

<!-- 页脚开始 -->
<div class="shh">

</div>

<div class="w1180" style="margin-top:8px;">
<center><script src="http://www.guoxuedashi.com/img/plus.php?id=3"></script></center>
</div>
<div class="w1180 foot">
<a href="/b/thanks.php">特别致谢</a> | <a href="javascript:window.external.AddFavorite(document.location.href,document.title);">收藏本站</a> | <a href="#">欢迎投稿</a> | <a href="http://www.guoxuedashi.com/forum/">意见建议</a> | <a href="http://www.guoxuemi.com/">国学迷</a> | <a href="http://www.shuowen.net/">说文网</a><script language="javascript" type="text/javascript" src="https://js.users.51.la/17753172.js"></script><br />
  Copyright &copy; 国学大师 古典图书集成 All Rights Reserved.<br>
  
  <span style="font-size:14px">免责声明:本站非营利性站点,以方便网友为主,仅供学习研究。<br>内容由热心网友提供和网上收集,不保留版权。若侵犯了您的权益,来信即刪。scp168@qq.com</span>
  <br />
ICP证:<a href="http://www.beian.miit.gov.cn/" target="_blank">鲁ICP备19060063号</a></div>
<!-- 页脚END --> 
<script src="http://www.guoxuedashi.com/img/plus.php?id=22"></script>
<script src="http://www.guoxuedashi.com/img/tongji.js"></script>

</body>
</html>
