






























































資治通鑑卷一百五十一 宋 司馬光 撰

胡三省 音註

梁紀七【起柔兆敦牂盡彊圉協洽凡二年}


高祖武皇帝七

普通七年春正月辛丑朔大赦 壬子魏以汝南王悦領太尉 魏安州石離宂城斛鹽三戍兵反應杜洛周衆合二萬洛周自松岍赴之【水經注大榆河出禦夷鎮北塞中南流逕密雲戊西又南流逕孔山西又歷密雲戍東右合孟廣水水西逕孔山南上有洞宂開明故謂之孔山大榆河又東南流白楊泉水注之水北白楊溪望離石大榆河又東南流出峽逕安州舊漁陽郡之滑鹽縣南世謂之斛鹽城西北去禦夷鎮二百里岍輕煙翻或曰岍字之誤也讀作陘唐志營州西北百里曰松陘嶺}
行臺常景使别將崔仲哲屯軍都關以邀之【將即亮翻}
仲哲戰沒元譚軍夜潰【元譚軍於居庸關見上卷上年}
魏以别將李琚代譚為都督仲哲秉之子也【崔秉時為燕州刺史}
初魏廣陽王深通於城陽王徽之妃徽為尚書令為胡太后所信任會恒州人請深為刺史徽言深心不可測及杜洛周反五原降戶在恒州者謀奉深為主深懼上書求還洛陽【時深軍于朔州恒戶登翻降戶江翻上時掌翻}
魏以左衛將軍楊津代深為北道大都督詔深為吏部尚書徽長夀之子也【長夀景穆子也}
五原降戶鮮于修禮等帥北鎮流民反於定州之左城【按楊津傳左城當在博陵界又水經注中山唐縣有左人城帥讀曰率}
改元魯興引兵向州城州兵禦之不利楊津至靈丘【靈丘縣漢屬代郡唐為蔚州}
聞定州危迫引兵救之入據州城修禮至津欲出擊之長史許被不聽津手劍擊 【手首又翻此以記檀弓子手弓釋之為據}
被走得免津開門出戰斬首數百賊退人心少安【少詩沼翻}
詔尋以津為定安州刺史兼北道行臺魏以揚州刺史長孫稚為大都督北討諸軍事與河間王琛共討修禮【長知兩翻琛丑林翻}
二月甲戌北伐衆軍解嚴 魏西部敕勒斛律洛陽反於桑乾西【乾音干}
與費也頭牧子相連結三月甲寅游擊將軍爾朱榮擊破洛陽於深井牧子於河西【北河之西}
夏四月乙酉臨川靖惠王宏卒 魏大赦 癸巳魏以侍中車騎大將軍城陽王徽為儀同三司【騎奇寄翻}
徽與給事黄門侍郎徐紇共毁侍中元順於太后出為護軍將軍太常卿順奉辭於西遊園紇侍側順指之謂太后曰此魏之宰噽【宰噽吳太宰噽也好讒吳王夫差信而任之以至亡國噽匹鄙翻}
魏國不亡此終不死紇脅肩而出【朱元晦曰脅肩竦體也小人側媚之態}
順抗聲叱之曰爾刀筆小才止堪供几案之用豈應汙辱門下斁我彞倫【汙烏故翻斁多路翻敗也}
因振衣而起太后默然 魏朔州城民鮮于阿胡等據城反【阿從安入聲}
杜洛周南出鈔掠薊城魏常景遣統軍梁仲禮擊破之丁未都督李琚與洛周戰于薊城之北敗沒常景帥衆拒之洛周引還上谷【鈔楚交翻薊音計帥讀曰率}
長孫稚行至鄴詔解大都督以河間王琛代之稚上言曏與琛同在淮南琛敗臣全【謂為裴邃所敗事見上卷上年}
遂成私隙今難以受其節度魏朝不聽【朝直遥翻}
前至呼沱稚未欲戰琛不從鮮于修禮邀擊稚於五鹿【杜預曰陽平元城縣東有五鹿即沙鹿也按呼沱不至元城界此别有五鹿非左氏所謂五鹿也沱徒河翻}
琛不赴救稚軍大敗稚琛並坐除名 五月丁未魏主下詔將北討内外戒嚴既而不行 衡州刺史元畧自至江南晨夕哭泣常如居喪及魏元乂死【乂死見上卷上年喪息亦翻}
胡太后欲召之知畧因刁雙獲免【事見一百四十九卷普通兀年}
徵雙為光禄大夫遣江革祖暅之南還以求略【江革祖暅之沒魏見上卷上年暅古鄧翻}
上備禮遣之寵贈甚厚畧始濟淮魏拜畧為侍中賜爵義陽王以司馬始賓為給事中栗法光為本縣令刁昌為東平太守刁雙為西兖州刺史凡畧所過一飱一宿皆賞之【畧來奔見一百四十九卷元年栗法光屯留人魏孝昌十年置西兖州於定陶領沛濟隂二郡是年則孝旻一年也}
魏以丞相高陽王雍為大司馬復以廣陽王深為大都督討鮮于修禮【復扶又翻}
章武王融為左都督裴衍為右都督並受深節度深以其子自随城陽王徽言於太后曰廣陽王攜其愛子握兵在外將有異志乃敕融衍潛為之備【疑則勿任任則勿疑既以深為大督而又使小督備之何以責其殄寇乎}
融衍以敕示深深懼事無大小不敢自决太后使問其故對曰徽銜臣次骨【李奇曰次骨者言深刻至骨}
臣疏遠在外徽之構臣無所不為自徽執政以來臣所表請多不從允徽非但害臣而已從臣將士有勲勞者皆見排抑不得比它軍仍深被憎嫉或因其有罪加以深文至於殊死以是從臣行者莫不悚懼有言臣善者視之如仇讐言臣惡者待之如親戚徽居中用事朝夕欲陷臣於不測之誅臣何以自安陛下若使徽出臨外州臣無内顧之憂庶可以畢命賊庭展其忠力太后不聽【為深死于盜手張本}
徽與中書舍人鄭儼等更相阿黨【更工衡翻}
外似柔謹内實忌克賞罰任情魏政由是愈亂 戊申魏燕州刺史崔秉帥衆弃城奔定州【燕州自去年八月為杜洛周所圍燕因肩翻}
乙丑魏以安西將軍宗正珍孫為都督【宗正複姓漢楚元王交之子郢客孫德世為宗正子孫因以為氏}
討汾州反胡【討劉蠡升也}
六月魏絳蜀陳雙熾聚衆反【蜀人徙居絳都者謂之絳蜀絳縣漢晉屬河東郡元魏分置絳郡魏收志郡屬東雍州}
自號始建王魏以假鎮西將軍長孫稚為討蜀都督 【考異曰費穆傳穆為都督平絳蜀不應有兩都督今從帝紀}
别將河東薛修義輕騎詣雙熾壘下曉以利害雙熾即降詔以修義為龍門鎮將【此龍門在河東北屈縣西魏世祖神䴥元年禽赫連昌改北屈為禽昌縣將即亮翻降戶江翻}
丙子魏徙義陽王略為東平王頃之遷大將軍尚書令為胡太后所委任與城陽王徽相埒【埒力輟翻}
然徐鄭用事略亦不敢違也【魏當時宗室畧其巨擘也史言其居淫昏之朝不能矯正}
杜洛周遣都督王曹紇真等將兵掠薊南【時杜洛周葛榮等作亂其軍中將領無不加以王爵曹紇真以都督加王號故曰都督王}
秋七月丙午行臺常景遣都督于榮等擊之於栗園【栗園當在范陽固安縣界固安之栗天下稱之}
大破之斬曹紇真及將卒三千餘級洛周帥衆南趣范陽【范陽縣前漢屬涿郡後漢章帝改涿郡為范陽郡帥讀曰率趣七喻翻}
景與榮等又破之 魏僕射元纂以行臺鎮恒州鮮于阿胡擁朔州流民寇恒州戊申陷平城纂奔冀州【恒州治平城平城魏之故都亦陷於賊其不競甚矣恒戶登翻}
上聞淮堰水盛夀陽城幾沒【觀此蓋淮堰復成也}
復遣郢州刺史元樹等自北道攻黎漿豫州刺史夏侯亶等自南道攻夀陽【復扶又翻下榮復同}
八月癸巳賊帥元洪業斬鮮于修禮請降于魏【帥所類翻降戶江翻}
賊黨葛榮復殺洪業自立 【考異曰北史廣陽王深傳云深以兵士頻經退散人無鬭情連營轉柵日行十里行達交津隔水而陳賊修禮常與葛榮謀後稍信朔州人毛普賢榮常銜之普賢昔為深統軍及在交津深使人諭之普賢乃有降意又使録事參軍元晏說賊程殺鬼果相猜貳榮遂殺普賢修禮而自立與魏帝紀全殊又其語雜亂難曉今從帝紀}
魏安北將軍都督恒朔討虜諸軍事爾朱榮過肆州肆州刺史尉慶賓忌之據城不出【尉紆勿翻}
榮怒舉兵襲肆州執慶賓還秀容署其從叔羽生為刺史魏朝不能制【此時爾朱榮已有無魏之心矣從才用翻朝直遥翻}
初賀拔允及弟勝岳從元纂在恒州平城之陷也允兄弟相失岳奔爾朱榮勝奔肆州榮克肆州得勝大喜曰得卿兄弟天下不足平也以為别將【將即亮翻}
軍中大事多與之謀 九月己酉鄱陽忠烈王恢卒葛榮既得杜洛周之衆【魏主武泰元年葛榮方并杜洛周此得鮮于修禮之衆也}
北趣瀛州【趣七喻翻}
魏廣陽忠武王深自交津引兵躡之【水經注漳水過武安縣東清水自涉縣東南來注之世謂决入之所為交漳口}
辛亥榮至白牛邏【據魏紀白牛邏在高陽博野縣邏郎佐翻}
輕騎掩擊章武莊武王融殺之榮自稱天子國號齊改元廣安深聞融敗停車不進侍中元晏密言於太后曰廣陽王盤桓不進坐圖非望有于謹者智略過人為其謀主風塵之際恐非陛下之純臣也太后深然之詔牓尚書省門募能獲謹者有重賞謹聞之謂深曰今女主臨朝【朝直遥翻}
信用讒佞苟不明白殿下素心恐禍至無日謹請束身詣闕歸罪有司遂徑詣牓下自稱于謹有司以聞太后引見大怒【見賢遍翻}
謹備論深忠欵兼陳停車之狀太后意解遂捨之深引軍還趣定州【趣七喻翻}
定州刺史楊津亦疑深有異志深聞之止於州南佛寺經二日深召都督毛諡等數人交臂為約危難之際期相拯恤【諡神至翻難乃旦翻}
諡愈疑之密告津云深謀不軌津遣諡討深深走出諡呼噪逐深深與左右閒行至博陵界【閒古莧翻漢桓帝置博陵郡元魏屬定州}
逢葛榮遊騎劫之詣榮【騎奇寄翻}
賊徒見深頗有喜者榮新立惡之【恐其徒有欲奉深為主者故惡之惡烏路翻}
遂殺深城陽王徽誣深降賊録其妻子【降戶江翻}
深府佐宋遊道為之訴理乃得釋遊道繇之玄孫也【宋繇事西梁李氏李滅事沮渠氏沮渠滅入魏為于偽翻}
甲申魏行臺常景破杜洛周斬其武川王賀拔文興等捕虜四百人 就德興陷魏平州殺刺史王買奴【魏平州治肥如即唐平州盧龍縣地}
天水民呂伯度本莫折念生之黨也後更據顯親以拒念生【漢光武置顯親侯國以封竇友以其兄融以河西歸附且以顯其孝文竇后之親也屬漢陽郡後魏屬天水郡至唐時秦州成紀縣治顯親川}
已而不勝亡歸胡琛【琛丑林翻}
琛以為大都督秦王資以士馬使擊念生伯度屢破念生軍復據顯親【復扶又翻下生復同}
乃叛琛東引魏軍念生窘迫乞降於蕭寶寅【窘渠隕翻}
寶寅使行臺左丞崔士和據秦州魏以伯度為涇州刺史封平秦郡公【魏大延二年置平秦郡於雍縣屬岐州}
大都督元修義停軍隴口久不進【隴口隴坻之口}
念生復反執士和送胡琛於道殺之久之伯度為万俟醜奴所殺【万莫北翻俟渠之翻}
賊勢益盛寶寅不能制胡琛與莫折念生交通事破六韓拔陵浸慢【琛應拔陵見上卷五年}
拔陵遣其臣費律至高平誘琛斬之【誘音酉}
醜奴盡并其衆 冬十一月庚辰大赦 丁貴嬪卒太子水漿不入口【太子統丁貴妃所生也卒子恤翻}
上使謂之曰毁不滅性【引孝經孔子之言}
况我在邪乃進粥數合【合古沓翻漢志曰十龠為合合者龠之量也}
太子體素肥壯腰帶十圍至是減削過半【過工禾翻}
夏侯亶等軍入魏境所向皆下辛巳魏揚州刺史李憲以夀陽降【降戶江翻下同}
宣猛將軍陳慶之入據其城凡降城五十二獲男女七萬五千口丁亥縱李憲還魏復以夀陽為豫州【自宋以來以夀陽為豫州裴叔業叛齊降魏魏以夀陽為揚州復漢魏之舊也今復以夀陽為豫州復宋齊之舊也復以扶又翻又如字}
改合肥為南豫州【天監五年徙豫州治合肥}
以夏侯亶為豫南豫二州刺史夀陽久罹兵革民多離散亶輕刑薄賦務農省役頃之民戶充復 杜洛周圍范陽戊戌民執魏幽州刺史王延年行臺常景送洛周開門納之【常景擊杜洛周數戰數勝而終于為虜者民樂於從亂而疾視其上也}
魏齊州平原民劉樹等反【宋武帝僑置平原郡于梁鄒屬冀州後入於魏改冀州為齊州平原為東平原郡}
攻陷郡縣頻敗州軍刺史元欣以平原房士達為將討平之【敗補邁翻將即亮翻}
曹義宗據穰城以逼新野【穰人羊翻}
魏遣都督魏承祖及尚書左丞南道行臺辛纂救之【考異曰梁書此年冬新野降魏書肅宗崩後新野猶在恐梁書誤蓋梁自前年攻新野此年魏使魏承祖救}


【之也又周于謹傳云孝昌二年與辛纂討義宗今以為據}
義宗戰不利不敢進纂雄之從父兄也 魏盜賊日滋征討不息國用耗竭豫徵六年租調猶不足【調徒弔翻}
乃罷百官所給酒肉又税入市者人一錢及邸店皆有税百姓嗟怨吏部郎中辛雄上疏以為華夷之民相聚為亂豈有餘憾哉正以守令不得其人【守式又翻}
百姓不堪其命故也宜及此時早加慰撫但郡縣選舉由來共輕貴遊儁才莫肯居此宜改其弊分郡縣為三等清官選補之法妙盡才望如不可並後地先才不得拘以停年【地門地也崔亮制停年格見一百四十九卷天監十八年後先並去聲}
三載黜陟【載子亥翻}
有稱職者補在京名官【稱尺證翻}
如不歷守令不得為内職則人思自勉枉屈可申彊暴自息矣不聽

大通元年【是年三月方改元大通}
春正月乙丑以尚書左僕射徐勉為僕射【時右僕射缺故左僕射止為僕射}
辛未上祀南郊 甲戌魏以司空皇甫度為司徒儀同三司蕭寶寅為司空魏分定相二州四郡置殷州【按魏收志殷州止領趙郡鉅鹿南鉅鹿三郡蓋初置時兼領相州之廣宗郡也殷州治廣阿劉昫曰北齊改為趙州隋改廣阿為大陸唐武德四年改為象城天寶二年改為昭慶以有建初啟運二陵也宋開寶五年改昭慶為隆平熙寧六年省隆平縣為隆平鎮入臨城縣相息亮翻}
以北道行臺博陵崔楷為刺史楷表稱州今新立尺刃斗糧皆所未有乞資以兵糧詔付外量聞【使量計合給兵量之數以聞也量音良}
竟無所給或勸楷留家單騎之官【騎奇寄翻}
楷曰吾聞食人之禄者憂人之憂若吾獨往則將士誰肯固志哉遂舉家之官葛榮逼州城或勸減弱小以避之楷遣幼子及一女夜出既而悔之曰人謂吾心不固虧忠而全愛也遂命追還賊至彊弱相懸又無守禦之具楷撫勉將士以拒之莫不争奮皆曰崔公尚不惜百口吾屬何愛一身連戰不息死者相枕【枕職任翻}
終無叛志辛未城陷楷執節不屈榮殺之【藩翰之任保境安民上也全城却敵次也死於城郭豈得己哉崔楷闔家并命其志節有可憐矣上之人實有罪焉}
遂圍冀州 蕭寶寅出兵累年將士疲弊秦賊擊之寶寅大敗於涇州收散兵萬餘人屯逍遥園東秦州刺史潘義淵以汧城降賊【秦州既為賊所據魏置東秦州于隴東郡治汧城即隋唐之汧源縣也將即亮翻汧口堅翻降戶江翻}
莫折念生進逼岐州城人執刺史魏蘭根應之豳州刺史畢祖暉戰沒行臺辛深弃城走【弃豳州城也}
北海王顥軍亦敗賊帥胡引祖據北華州【引祖恐當作弘祖魏孝文帝太和十五年置東秦州於杏城後改為北華州領中部敷城郡帥所類翻華戶化翻}
叱干麒麟據豳州以應天生【叱干虜複姓}
關中大擾雍州刺史楊椿募兵得七千餘人帥以拒守【雍於用翻帥讀曰率}
詔加椿侍中兼尚書右僕射為行臺節度關西諸將北地功曹毛鴻賓引賊抄掠渭北【抄楚交翻}
雍州録事參軍楊侃將兵三千掩擊之鴻賓懼請討賊自効遂擒送宿勤烏過仁烏過仁者明達之兄子也莫折天生乘勝寇雍州 【考異曰羊侃傳作莫遮今從魏書}
蕭寶寅部將羊侃隱身塹中射之應弦而斃【塹七艷翻射而亦翻}
其衆遂潰侃祉之子也 魏右民郎陽平路思令上疏【晉武帝置尚書右民郎}
以為師出有功在於將帥得其人則六合唾掌可清【人欲舉手有為先唾其掌六合天地東西南北也唾掌可清言其易也唾湯卧翻口液也}
失其人則三河方為戰地【此指漢三河之地為言魏都洛陽三河則畿甸也}
竊以比年將帥多寵貴子孫銜杯躍馬志逸氣浮軒眉扼腕以攻戰自許及臨大敵憂怖交懷雄圖鋭氣一朝頓盡乃令羸弱在前以當寇彊壯居後以衛身兼復器械不精進止無節以當負險之衆敵數戰之虜【腕烏貫翻怖普布翻羸倫為翻比毗至翻復扶又翻下復疑同數所角翻}
欲其不敗豈可得哉是以兵知必敗始集而先逃將帥畏敵遷延而不進國家謂官爵未滿屢加寵命復疑賞賚之輕日散金帛帑藏空竭民財殫盡【帑它朗翻藏徂浪翻}
遂使賊徒益甚生民彫弊凡以此也夫德可感義夫恩可勸死士今若黜陟幽明賞罰善惡簡練士卒繕修器械先遣辯士曉以禍福如其不悛以順討逆【悛丑緣翻}
如此則何異厲蕭斧而伐朝菌【戰國時雍門周有是言莊子曰朝菌不知晦朔音義云菌大芝也天隂生糞土見日則死梁簡文云菌欻生之芝也音其隕翻}
鼓洪爐而燎毛髮哉弗聽 戊子魏以皇甫度為太尉 己丑魏主以四方未平詔内外戒嚴將親出討竟亦不行 譙州刺史湛僧智圍魏東豫州【姓譜湛丈減翻姓也後漢有大司農湛重帝置譙州治新昌城領新昌高塘臨徐南梁郡五代志江都郡清流縣郡置新昌郡 考異曰魏帝紀及曹世表傳作湛僧今從梁夏侯夔傳}
將軍彭羣王辯圍琅邪魏敕青南青二州救琅邪【魏青州領齊北海樂安勃海高陽河間樂陵郡南青州當又置于其南}
司州刺史夏侯夔帥壯武將軍裴之禮等出義陽道攻魏平静穆陵隂山三關皆克之【水經注木陵關在黄武山東北晉西陽城西南帥讀曰率下同}
夔亶之弟之禮邃之子也 魏東清河郡山賊羣起詔以齊州長史房景伯為東清河太守【宋武帝僑置清河郡於盤陽屬冀州後入于魏為東清河郡屬齊州五代志齊州長山縣清河平原二郡併入焉}
郡民劉簡虎嘗無禮於景伯舉家亡去景伯窮捕禽之署其子為西曹掾令諭山賊賊以景伯不念舊惡皆相帥出降【掾俞絹翻降戶江翻}
景伯母崔氏通經有明識貝丘婦人列其子不孝【貝丘僑縣亦宋武帝置屬清河郡五代志齊州淄川縣舊曰貝丘置東清河郡按前注所謂清河郡置於盤陽者據魏收地形志宋郡也五代志長山之清河平原雙頭郡也房景伯所守者貝丘之東清河也}
景伯以白其母母曰吾聞聞名不如見面山民未知禮義何足深責乃召其母與之對榻共食使其子侍立堂下觀景伯供食未旬日悔過求還崔氏曰此雖面慙其心未也且置之凡二十餘日其子叩頭流血母涕泣乞還然後聽之卒以孝聞【卒子恤翻}
景伯法夀之族子也【房法夀見一百三十二卷宋明帝泰始三年}
二月秦賊據魏潼關【出蕭寶寅之後}
庚申魏東郡民趙顯

德反殺太守裴烟自號都督【魏東郡治滑臺城屬西兖州烟俗煙字}
將軍成景儁攻魏彭城魏以前荆州刺史崔孝芬為徐州行臺以禦之先是孝芬坐元乂黨與盧同等俱除名【盧同除名見上卷普通六年先悉薦翻}
及將赴徐州入辭太后太后謂孝芬曰我與卿姻戚【時太后為魏主納孝芬女為世婦故云然}
奈何内頭元乂車中稱此老嫗會須去之【嫗威遇翻去羌呂翻}
孝芬曰臣蒙國厚恩實無斯語假令有之誰能得聞若有聞者此於元乂親密過臣遠矣太后意解悵然有愧色景儁欲堰泗水以灌彭城孝芬與都督李叔仁等擊之景儁遁還 三月甲子魏主詔將西討中外戒嚴會秦賊西走復得潼關【復扶又翻}
戊辰詔回駕北討其實皆不行 葛榮久圍信都魏以金紫光禄大夫源子邕為北討大都督以救之初上作同泰寺又開大通門以對之取其反語相協【同泰反為大大通反為同是反語相協也反音翻}
上晨夕幸寺皆出入是門辛未上幸寺捨身甲戌還宫大赦改元【改是年為大通元年}
魏齊州廣川民劉鈞聚衆反【宋武帝僑置廣川郡屬冀州入魏屬齊州五代志齊州長山縣舊曰武彊置廣川郡}
自署大行臺清河民房項自署大都督屯據昌國城【魏收志東清河郡武城縣有昌國城}
夏四月魏將元斌之討東郡斬趙顯德【將即亮翻}
己酉柔然頭兵可汗遣使入貢於魏【可從刋入聲汗音寒使疏吏翻}
且請討羣賊魏人畏其反覆詔以盛暑且俟後敕魏蕭寶寅之敗也有司處以死刑詔免為庶人雍州刺史楊椿有疾求解復以寶寅為都督雍涇等四州諸軍事征西將軍雍州刺史開府儀同三司西討大都督自關以西皆受節度【處昌呂翻復扶又翻雍於用翻}
椿還鄉里【楊椿世居華隂}
其子昱將適洛陽椿謂之曰當今雍州刺史亦無踰於寶寅者但其上佐朝廷應遣心膂重臣何得任其牒用此乃聖朝百慮之一失也【朝直遥翻}
且寶寅不藉刺史為榮吾觀其得州喜悦特甚至於賞罰云為不依常憲恐有異心汝今赴京師當以吾此意啟二聖【二聖謂胡太后魏主}
并白宰輔更遣長史司馬防城都督欲安關中正須三人耳如其不遣必成深憂昱面啟魏主及太后皆不聽【是後寶寅以關中叛魏如楊椿所料}
五月丙寅成景儁攻魏臨潼竹邑拔之【魏置臨潼郡治臨潼城據水經城臨潼水故名竹邑即漢沛郡之竹縣也魏為南濟隂郡治所五代志下邳郡夏丘縣舊置臨潼郡彭城郡符離縣隋廢竹邑入焉宋白曰符離縣朝斛城西南七十里有竹邑城}
東宫直閤蘭欽攻魏蕭城厥固拔之【東宫亦有直閤將軍魏收志魏沛郡治蕭縣黄陽城又領内相縣有厥城領内猶言管内也}
欽斬魏將曹龍牙【將即亮翻}
六月魏都督李叔仁討劉鈞平之 秋七月魏陳郡

民劉獲鄭辯反於西華【西華縣漢屬汝南郡晉屬潁川郡元魏屬陳郡}
改元天授與湛僧智通謀【湛僧智時圍魏東豫州}
魏以行東豫州刺史譙國曹世表為東南道行臺以討之源子恭代世表為東豫州諸將以賊衆彊官軍弱且皆敗散之餘不敢戰欲保自固世表方病背腫轝出呼統軍是云寶【是云姓也魏書官氏志内入諸姓有是云氏}
謂曰湛僧智所以敢深入為寇者以獲辯皆州民之望為之内應也曏聞獲引兵欲迎僧智去此八十里今出其不意一戰可破獲破則僧智自走矣乃選士馬付寶暮出城比曉而至【比必利翻及也}
擊獲大破之窮討餘黨悉平僧智聞之遁還鄭辯與子恭親舊亡匿子恭所世表集將吏而責子恭收辯斬之 魏相州刺史樂安王鑒與北道都督裴衍共救信都【相息亮翻樂安當作安樂樂音洛}
鑒幸魏多故隂有異志遂據鄴叛降葛榮【降戶江翻下同}
己丑魏大赦初侍御史遼東高道穆奉使相州【使疏吏翻下同}
前刺史李世哲奢縱不法道穆案之世哲弟神軌用事道穆兄謙之家奴訴良【律禁壓良為賤謂本是良民壓為奴婢}
神軌收謙之繫廷尉赦將出神軌啟太后先賜謙之死朝士哀之【朝直遥翻}
彭羣王辯圍琅邪自夏及秋魏青州刺史彭城王劭遣司馬鹿悆南青州刺史胡平遣長史劉仁之將兵擊羣辯破之羣戰沒劭勰之子也【彭城王勰魏之賢王也死於高肇之譖悆羊茹翻將即亮翻勰音協}
八月魏遣都督源子邕李神軌裴衍攻鄴子邕行及湯隂【湯隂縣漢屬河内郡晉廢縣其地在汲郡界}
安樂王鑒遣弟斌之夜襲子邕營不克子邕乘勝進圍鄴城丁未拔之斬鑒傳首洛陽改姓拓拔氏魏因遣子邕裴衍討葛榮 九月秦州城民杜粲殺莫折念生闔門皆盡粲自行州事南秦州城民辛琛亦自行州事遣使詣蕭寶寅請降【琛丑林翻}
魏復以寶寅為尚書令還其舊封【寶寅涇州之敗免為庶人舊封者寶寅自丹陽郡公徙封梁郡公復扶又翻}
譙州刺史湛僧智圍魏東豫州刺史元慶和於廣陵【此廣陵城在新息縣界}
魏將軍元顯伯救之司州刺史夏侯夔自武陽引兵助僧智【武陽關義陽三關之一也}
冬十月夔至城下慶和舉城降夔以讓僧智僧智曰慶和欲降公不欲降僧智今往必乖其意且僧智所將應募烏合之人不可御以法公持軍素嚴必無侵暴受降納附深得其宜夔乃登城拔魏幟建梁幟【幟昌志翻}
慶和束兵而出吏民安堵獲男女四萬餘口

臣光曰湛僧智可謂君子矣忘其積時攻戰之勞【湛僧智自是年正月攻圍東豫州}
以授一朝新至之將知己之短不掩人之長功成不取以濟國事忠且無私可謂君子矣

元顯伯宵遁諸軍追之斬獲萬計詔以僧智領東豫州刺史鎮廣陵夔引軍屯安陽【魏收志東豫州汝南郡有安陽縣五代志汝南真陽縣隋廢魏安陽縣入焉}
遣别將屠楚城【魏收志梁置西楚州於楚城五代志汝南郡城陽縣梁置楚州}
由是義陽北道遂與魏絶 領軍曹仲宗東宫直閤陳慶之攻魏渦陽【渦古禾翻}
詔尋陽太守韋放將兵會之魏散騎常侍費穆引兵奄至【散悉亶翻騎奇寄翻}
放營壘未立麾下止有二百餘人放免胄下馬據胡牀處分【胡牀即今之交牀隋惡胡氏改曰交牀今之交倚是也處昌呂翻分扶問翻}
士皆殊死戰莫不一當百魏兵遂退放叡之子也【梁之將帥韋叡一人而已}
魏又遣將軍元昭等衆五萬救渦陽前軍至駝澗去渦陽四十里【今自肥河口泝江西上得駝澗灘其灘南對永夀館北至耶溪}
陳慶之欲逆戰韋放以魏之前鋒必皆輕鋭不如勿擊待其來至慶之曰魏兵遠來疲倦去我既遠必不見疑及其未集須挫其氣諸君若疑【君或作軍}
慶之請獨取之於是帥麾下二百騎進擊破之【帥讀曰率騎奇寄翻}
魏人驚駭慶之乃還與諸將連營而進背渦陽城與魏軍相持【背蒲妹翻}
自春至冬數十百戰將士疲弊聞魏人欲築壘於軍後曹仲宗等恐腹背受敵議引軍還慶之杖節軍門曰共來至此涉歷一載【去年慶之入夀陽至此涉歷一年}
糜費極多今諸君皆無鬬心唯謀退縮豈是欲立功名直聚為抄暴耳【抄楚交翻}
吾聞置兵死地乃可求生【兵法置之死地而後生}
須虜大合然後與戰審欲班師慶之别有密敕今日犯者當依敕行之仲宗等乃止魏人作十三城欲以控制梁軍慶之銜枚夜出陷其四城渦陽城主王緯乞降【緯于貴翻 考異曰魏帝紀九月辛卯東豫州刺史元慶和以城叛梁帝紀十月庚戌魏東豫州刺史元慶和以渦陽内屬夏侯夔傳湛僧智圍元慶和于廣陵慶和請降詔以僧智為東豫州鎮廣陵韋放傳普通八年曹仲宗攻渦陽放會之城主王偉降陳慶之傳大通元年隸曹仲宗伐渦陽城主王偉降詔以渦陽置西徐州然則廣陵渦陽兩處兩事梁紀慶和渦陽之間或更有脱字耳魏紀九月據聞慶和始叛之時梁紀十月據慶和降欵到日按陳慶之傳云自春至冬今從梁紀十月為定此别一廣陵非南兖州之廣陵也王偉當作王緯蓋草書之誤也}
韋放簡遣降者三十餘人分報魏諸營陳慶之陳其俘馘鼓譟随之九城皆潰追擊之俘斬略盡尸咽渦水所降城中男女三萬餘口 蕭寶寅之敗於涇州也或勸之歸罪洛陽或曰不若留關中立功自効行臺都令史河間馮景曰擁兵不還此罪將大【尚書有都令史故行臺亦置之}
寶寅不從自念出師累年糜費不貲一旦覆敗内不自安魏朝亦疑之【朝直遥翻}
中尉酈道元素名嚴猛司州牧汝南王悦【魏都洛陽置司州}
嬖人丘念弄權縱恣道元收念付獄悦請之於胡太后太后欲赦之道元殺之并以劾悦時寶寅反狀已露悦乃奏以道元為關右大使【嬖卑義翻又博計翻劾戶槩翻又戶得翻使疏吏翻}
寶寅聞之謂為取已甚懼長安輕薄子弟復勸使舉兵【復扶又翻下不復寅復同}
寶寅以問河東柳楷楷曰大王齊明帝子天下所屬【屬之欲翻}
今日之舉實允人望且謡言鸞生十子九子一子不關中亂【齊明帝諱鸞寶寅之父也徒玩翻卵壞也周秦以前以亂為治}
大王當治關中何所疑道元至隂盤驛【此隂盤縣驛也魏收地形志曰隂盤縣本屬安定晉屬京兆魏真君七年併新豐太和十一年復置隂盤縣鴻門戲水正屬縣界按漢安定郡與京兆相去遼遠中間為馮翊所隔自晉以後所置隂盤縣非漢安定之隂盤縣地也魏收不深考耳宋白曰京兆昭應縣東十三里有故城後漢靈帝末移安定郡隂盤縣寄理於此今亦謂之隂盤城後魏太和九年自此復移隂盤縣城於今昭應縣東三十二里零水西戲水東司馬村故城是也}
寶寅遣其將郭子恢攻殺之【將即亮翻}
收殯其尸表言白賊所害【秦人謂鮮卑為白虜自苻秦之亂鮮卑之種有因而留關中者是時亦相梃為盜因謂之白賊或曰白賊謂白地之寇也}
又上表自理稱為楊椿父子所譖寶寅行臺郎中武功蘇湛卧病在家寶寅令湛從母弟開府屬天水姜儉說湛【魏以寶寅為開府故有掾有屬從才用翻說式芮翻}
曰元略受蕭衍旨欲見勦除【略自梁還魏大見寵任故寶寅託以為言勦子小翻}
道元之來事不可測吾不能坐受死亡今須為身計不復作魏臣矣死生榮辱與卿共之湛聞之舉聲大哭儉遽止之曰何得便爾湛曰我百口今屠滅云何不哭哭數十聲徐謂儉曰為我白齊王【寶寅歸魏封為齊王故稱之為于偽翻下口為同}
王本以窮鳥投人賴朝廷假王羽翼榮寵至此屬國步多虞【屬之欲翻}
不能竭忠報德乃欲乘人間隙【間古莧翻}
信惑行路無識之語欲以羸敗之兵守關問鼎【守關謂寶寅欲守潼關之險割據關中問鼎謂欲窺天位成王定鼎于郟鄏三代之世寶也楚莊問鼎之大小輕重欲以兵威脅取之故以諭窺天位者羸倫為翻}
今魏德雖衰天命未改且王之恩義未洽於民但見其敗未見有成蘇湛不能以百口為王族滅寶寅復使謂曰我救死不得不爾所以不先相白者恐沮吾計耳【沮在呂翻}
湛曰凡謀大事當得天下奇才與之從事今但與長安博徒謀之此有成理不【不讀曰否}
湛恐荆棘必生於齋閤【此亦用伍胥伍被語意}
願賜骸骨歸鄉里庶得病死下見先人寶寅素重湛且知其不為己用聽還武功甲寅寶寅自稱齊帝改元隆緒赦其所部置百官都督長史毛遐【寶寅都督雍涇等四州又為西討大都督以遐為府長史}
鴻賓之兄也與鴻賓帥氐羌起兵於馬祇柵以拒寶寅【帥讀曰率}
寶寅遣大將軍盧祖遷擊之為遐所殺寶寅方祀南郊行即位禮未畢聞敗色變不暇整部伍狼狽而歸以姜儉為尚書左丞委以心腹文安周惠達為寶寅使在洛陽【文安縣前漢屬勃海後漢屬河間晉置章武郡文安屬焉使疏吏翻}
有司欲收之惠達逃歸長安寶寅以惠達為光禄勲丹陽王蕭贊聞寶寅反懼而出走趣白馬山【趣七喻翻}
至河橋為人所獲魏主知其不預謀釋而慰之行臺郎封偉伯等與關中豪傑謀舉兵誅寶寅事泄而死魏以尚書僕射長孫稚為行臺以討寶寅正平民薛鳳賢反【魏收志世祖置太平郡於河東聞喜縣孝文太和十八年改曰正平郡屬雍雍州領聞喜曲沃二縣}
宗人薛修義亦聚衆河東分據鹽池攻圍蒲坂東西連結以應寶寅詔都督宗正珍孫討之 十一月丁卯以護軍蕭淵藻為北討都督鎮渦陽戊辰以渦陽為西徐州【渦陽魏置譙州梁改為西徐州領南譙汴龍亢靳城潁川臨渙蒙郡渦音戈}
葛榮圍信都自春及冬冀州刺史元孚帥勵將士晝夜拒守【帥讀曰率下同}
糧儲既竭外無救援己丑城陷榮執孚逐出居民凍死者什六七孚兄祐為防城都督榮大集將士議其生死孚兄弟各自引咎争相為死【為于偽翻}
都督潘紹等數百人皆叩頭請就法以活使君榮曰此皆魏之忠臣義士於是同禁者五百人皆得免魏以源子邕為冀州刺史將兵討榮【將即亮翻}
裴衍表請同行詔許之子邕上言衍行臣請留臣行請留衍若逼使同行敗在旦夕不許十二月戊申行至陽平東北漳水曲榮帥軍十萬擊之子邕衍俱敗死相州吏民聞冀州已陷子邕等敗人不自保相州刺史恒農李神志氣自若【魏顯祖諱弘改弘農曰恒農相息亮翻恒戶登翻}
撫勉將士大小致力葛榮盡鋭攻之卒不能克【卒子恤翻}
秦州民駱超殺杜粲請降於魏【杜粲殺莫折念生駱超又殺杜粲羣盜互相屠滅以邀一時之利不足怪也降戶江翻}


資治通鑑卷一百五十一














































































































































